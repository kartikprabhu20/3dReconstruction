\chapter{\iftoggle{german}{Evaluierung}{Experiments and Evaluation}}\label{ch:evaluation}

%\todo{
%    \begin{enumerate}
%        \item baseline comparison, pix3d pix2vox, pix2vox++ classwise)
%        \item baseline different version of dataset with different threshold
%        \item ablation values per model
%        \item !!!!!finetuning with different datasize (pending)
%        \item !!!!!mixed training with different real datasize(pending)
%        \item training graphs
%    \end{enumerate}
%}

This chapter conducts experiments that will help us find solutions for the research questions discussed in \autoref{sec:goal}.
\autoref{sec:a-survey-on-photorealism} will contain the survey results conducted to check the photorealism of the proposed synthetic dataset.
This section will also compare the ratings given to other proclaimed photorealistic datasets and check whether the \gls{free} dataset compares to those datasets.
We will further evaluate the datasets using a \gls{tsne} as a qualitative measure and \gls{fid} as a quantitative measure, indicating the domain gaps concerning the real dataset.

As described in \autoref{sec:datasets}, we use different datasets specifically generated to evaluate our baseline models and check randomization parameters.
\autoref{sec:baseline} evaluates baseline models with real and synthetic versions of datasets.
\autoref{sec:fine-tuning} further evaluates models pre-trained on synthetic datasets by fine-tuning them using a real dataset.
\autoref{sec:mixed-training} deals with mixed training and its impact on performance with different mixing ratios of the real and synthetic datasets per mini-batch.
\autoref{sec:ablation-study-on-chairs}, will evaluate models on chair datasets with different randomization parameters and mixed training for these individual datasets.

\section{A Survey on Photorealism}\label{sec:a-survey-on-photorealism}

The participants received minimalistic information about the intention behind the survey.
The goal of the survey was to analyze if humans have the same perception of photographic and computer-generated images.
The survey was open to everyone with no time limit.
A total of 72 participants responded to the survey.
The survey was created using Google Forms, and the link was distributed.
The participants either used a mobile phone or a desktop to respond to the survey.
The survey constituted a total of 9 datasets.
\gls{front}~\cite{Fu20203DFRONT3F}, Hypersim~\cite{Roberts2020HypersimAP}, InteriorNet~\cite{InteriorNet18}, SceneNet~\cite{McCormac2017}, BlenderProc~\cite{dlr139317},
\gls{ai2thor}~\cite{ai2thor}, Openroom~\cite{Li_2021_CVPR}, Pix3D~\cite{Sun2018} and proposed \gls{free} dataset.
Only Pix3D was a real dataset, while all others are synthetic datasets proclaimed to be photorealistic.

The survey was composed of 3 sections.
\begin{enumerate}
    \item Section 1: Decide if the image is real or not real.

    In this section, there were 27 images, three each from the datasets as mentioned earlier.
    Each image had only two options to select: "Real" or "Not real"
    This approach eliminated any ambiguous perception towards the images.

    \item Section 2: Rate the image on a scale of 1 to 10 in terms of realism (1 -> least real, 10 -> most real).

    In this section, the participant used a Likert scale to rate the images based on photorealism.
    Similar to section 1, there were 27 questions of three images per dataset.

    \item Section 3: Rank the images from 1 to 9 (1 -> Most real, 9 -> Least real).

    In this section, the participant had only three questions, with each question having an image from each of the datasets arranged in a 3\x3 grid format.
    The users ranked them in the increasing order of the photorealism.
\end{enumerate}

%\subsection{Survey results}\label{subsec:survey-results}
%In this segment, we discuss the results of the survey collected from participants.

\subsection{Section 1: Real or Not}
In section 1, the participants had only two options to select from independently.
\autoref{fig:question1} shows that the real dataset Pix3D~\cite{Sun2018} was rightly recognized as real.
77\% of the real images that belonged to Pix3D were perceived to be real.
This shows that the participants were not convinced even with the real images, as 23\% were still recognized as not real.
Hyeperism~\cite{Roberts2020HypersimAP} got the best results of 59.7\% identified as real among the synthetic datasets,.
\gls{ai2thor} had the least amount of images recognized as real with just 5\% positive responses.
The proposed \gls{free} dataset had 8\% of images identified as real.
These two datasets were built in Unity, which shows that images generated using the automated Unity framework needs some improvement.
Suppose we have a threshold of 50\%, we see that the datasets for which the images selected are Not real below the threshold value belongs to datasets that are automated and not created by professionals manually.
As mentioned in \autoref{subsec:indoor-synthetic-datasets}, Openrooms, SceneNet, and Blenderproc are datasets obtained from automation.
We consider the \gls{free} dataset to be automated and belong to this category.
Openrooms have the most confidence votes among the automated images, with 37\% recognized as real images.
Even though \gls{free} was recognized as real the least number of times among the automated tools, it had better percentage than AI2THOR
which has a Unity-based framework to generate images and was manually configured by professionals by taking in reference of real-world images.

\begin{figure}
    \centering
    \resizebox{\textwidth}{!}{%% Creator: Matplotlib, PGF backend
%%
%% To include the figure in your LaTeX document, write
%%   \input{<filename>.pgf}
%%
%% Make sure the required packages are loaded in your preamble
%%   \usepackage{pgf}
%%
%% Figures using additional raster images can only be included by \input if
%% they are in the same directory as the main LaTeX file. For loading figures
%% from other directories you can use the `import` package
%%   \usepackage{import}
%%
%% and then include the figures with
%%   \import{<path to file>}{<filename>.pgf}
%%
%% Matplotlib used the following preamble
%%   \usepackage{fontspec}
%%   \setmainfont{DejaVuSerif.ttf}[Path=\detokenize{/Users/apple/opt/anaconda3/envs/kaolin/lib/python3.7/site-packages/matplotlib/mpl-data/fonts/ttf/}]
%%   \setsansfont{DejaVuSans.ttf}[Path=\detokenize{/Users/apple/opt/anaconda3/envs/kaolin/lib/python3.7/site-packages/matplotlib/mpl-data/fonts/ttf/}]
%%   \setmonofont{DejaVuSansMono.ttf}[Path=\detokenize{/Users/apple/opt/anaconda3/envs/kaolin/lib/python3.7/site-packages/matplotlib/mpl-data/fonts/ttf/}]
%%
\begingroup%
\makeatletter%
\begin{pgfpicture}%
\pgfpathrectangle{\pgfpointorigin}{\pgfqpoint{9.190000in}{5.000000in}}%
\pgfusepath{use as bounding box, clip}%
\begin{pgfscope}%
\pgfsetbuttcap%
\pgfsetmiterjoin%
\definecolor{currentfill}{rgb}{1.000000,1.000000,1.000000}%
\pgfsetfillcolor{currentfill}%
\pgfsetlinewidth{0.000000pt}%
\definecolor{currentstroke}{rgb}{1.000000,1.000000,1.000000}%
\pgfsetstrokecolor{currentstroke}%
\pgfsetdash{}{0pt}%
\pgfpathmoveto{\pgfqpoint{0.000000in}{0.000000in}}%
\pgfpathlineto{\pgfqpoint{9.190000in}{0.000000in}}%
\pgfpathlineto{\pgfqpoint{9.190000in}{5.000000in}}%
\pgfpathlineto{\pgfqpoint{0.000000in}{5.000000in}}%
\pgfpathclose%
\pgfusepath{fill}%
\end{pgfscope}%
\begin{pgfscope}%
\pgfsetbuttcap%
\pgfsetmiterjoin%
\definecolor{currentfill}{rgb}{1.000000,1.000000,1.000000}%
\pgfsetfillcolor{currentfill}%
\pgfsetlinewidth{0.000000pt}%
\definecolor{currentstroke}{rgb}{0.000000,0.000000,0.000000}%
\pgfsetstrokecolor{currentstroke}%
\pgfsetstrokeopacity{0.000000}%
\pgfsetdash{}{0pt}%
\pgfpathmoveto{\pgfqpoint{1.148750in}{0.550000in}}%
\pgfpathlineto{\pgfqpoint{8.271000in}{0.550000in}}%
\pgfpathlineto{\pgfqpoint{8.271000in}{4.400000in}}%
\pgfpathlineto{\pgfqpoint{1.148750in}{4.400000in}}%
\pgfpathclose%
\pgfusepath{fill}%
\end{pgfscope}%
\begin{pgfscope}%
\pgfpathrectangle{\pgfqpoint{1.148750in}{0.550000in}}{\pgfqpoint{7.122250in}{3.850000in}}%
\pgfusepath{clip}%
\pgfsetbuttcap%
\pgfsetmiterjoin%
\definecolor{currentfill}{rgb}{0.248058,0.667205,0.350250}%
\pgfsetfillcolor{currentfill}%
\pgfsetfillopacity{0.500000}%
\pgfsetlinewidth{0.000000pt}%
\definecolor{currentstroke}{rgb}{0.000000,0.000000,0.000000}%
\pgfsetstrokecolor{currentstroke}%
\pgfsetstrokeopacity{0.500000}%
\pgfsetdash{}{0pt}%
\pgfpathmoveto{\pgfqpoint{1.148750in}{4.225000in}}%
\pgfpathlineto{\pgfqpoint{6.655304in}{4.225000in}}%
\pgfpathlineto{\pgfqpoint{6.655304in}{4.019118in}}%
\pgfpathlineto{\pgfqpoint{1.148750in}{4.019118in}}%
\pgfpathclose%
\pgfusepath{fill}%
\end{pgfscope}%
\begin{pgfscope}%
\pgfpathrectangle{\pgfqpoint{1.148750in}{0.550000in}}{\pgfqpoint{7.122250in}{3.850000in}}%
\pgfusepath{clip}%
\pgfsetbuttcap%
\pgfsetmiterjoin%
\definecolor{currentfill}{rgb}{0.248058,0.667205,0.350250}%
\pgfsetfillcolor{currentfill}%
\pgfsetfillopacity{0.500000}%
\pgfsetlinewidth{0.000000pt}%
\definecolor{currentstroke}{rgb}{0.000000,0.000000,0.000000}%
\pgfsetstrokecolor{currentstroke}%
\pgfsetstrokeopacity{0.500000}%
\pgfsetdash{}{0pt}%
\pgfpathmoveto{\pgfqpoint{1.148750in}{3.813235in}}%
\pgfpathlineto{\pgfqpoint{5.402316in}{3.813235in}}%
\pgfpathlineto{\pgfqpoint{5.402316in}{3.607353in}}%
\pgfpathlineto{\pgfqpoint{1.148750in}{3.607353in}}%
\pgfpathclose%
\pgfusepath{fill}%
\end{pgfscope}%
\begin{pgfscope}%
\pgfpathrectangle{\pgfqpoint{1.148750in}{0.550000in}}{\pgfqpoint{7.122250in}{3.850000in}}%
\pgfusepath{clip}%
\pgfsetbuttcap%
\pgfsetmiterjoin%
\definecolor{currentfill}{rgb}{0.248058,0.667205,0.350250}%
\pgfsetfillcolor{currentfill}%
\pgfsetfillopacity{0.500000}%
\pgfsetlinewidth{0.000000pt}%
\definecolor{currentstroke}{rgb}{0.000000,0.000000,0.000000}%
\pgfsetstrokecolor{currentstroke}%
\pgfsetstrokeopacity{0.500000}%
\pgfsetdash{}{0pt}%
\pgfpathmoveto{\pgfqpoint{1.148750in}{3.401471in}}%
\pgfpathlineto{\pgfqpoint{4.775822in}{3.401471in}}%
\pgfpathlineto{\pgfqpoint{4.775822in}{3.195588in}}%
\pgfpathlineto{\pgfqpoint{1.148750in}{3.195588in}}%
\pgfpathclose%
\pgfusepath{fill}%
\end{pgfscope}%
\begin{pgfscope}%
\pgfpathrectangle{\pgfqpoint{1.148750in}{0.550000in}}{\pgfqpoint{7.122250in}{3.850000in}}%
\pgfusepath{clip}%
\pgfsetbuttcap%
\pgfsetmiterjoin%
\definecolor{currentfill}{rgb}{0.248058,0.667205,0.350250}%
\pgfsetfillcolor{currentfill}%
\pgfsetlinewidth{0.000000pt}%
\definecolor{currentstroke}{rgb}{0.000000,0.000000,0.000000}%
\pgfsetstrokecolor{currentstroke}%
\pgfsetstrokeopacity{0.000000}%
\pgfsetdash{}{0pt}%
\pgfpathmoveto{\pgfqpoint{1.148750in}{2.989706in}}%
\pgfpathlineto{\pgfqpoint{3.786620in}{2.989706in}}%
\pgfpathlineto{\pgfqpoint{3.786620in}{2.783824in}}%
\pgfpathlineto{\pgfqpoint{1.148750in}{2.783824in}}%
\pgfpathclose%
\pgfusepath{fill}%
\end{pgfscope}%
\begin{pgfscope}%
\pgfpathrectangle{\pgfqpoint{1.148750in}{0.550000in}}{\pgfqpoint{7.122250in}{3.850000in}}%
\pgfusepath{clip}%
\pgfsetbuttcap%
\pgfsetmiterjoin%
\definecolor{currentfill}{rgb}{0.248058,0.667205,0.350250}%
\pgfsetfillcolor{currentfill}%
\pgfsetlinewidth{0.000000pt}%
\definecolor{currentstroke}{rgb}{0.000000,0.000000,0.000000}%
\pgfsetstrokecolor{currentstroke}%
\pgfsetstrokeopacity{0.000000}%
\pgfsetdash{}{0pt}%
\pgfpathmoveto{\pgfqpoint{1.148750in}{2.577941in}}%
\pgfpathlineto{\pgfqpoint{3.160126in}{2.577941in}}%
\pgfpathlineto{\pgfqpoint{3.160126in}{2.372059in}}%
\pgfpathlineto{\pgfqpoint{1.148750in}{2.372059in}}%
\pgfpathclose%
\pgfusepath{fill}%
\end{pgfscope}%
\begin{pgfscope}%
\pgfpathrectangle{\pgfqpoint{1.148750in}{0.550000in}}{\pgfqpoint{7.122250in}{3.850000in}}%
\pgfusepath{clip}%
\pgfsetbuttcap%
\pgfsetmiterjoin%
\definecolor{currentfill}{rgb}{0.248058,0.667205,0.350250}%
\pgfsetfillcolor{currentfill}%
\pgfsetfillopacity{0.500000}%
\pgfsetlinewidth{0.000000pt}%
\definecolor{currentstroke}{rgb}{0.000000,0.000000,0.000000}%
\pgfsetstrokecolor{currentstroke}%
\pgfsetstrokeopacity{0.500000}%
\pgfsetdash{}{0pt}%
\pgfpathmoveto{\pgfqpoint{1.148750in}{2.166176in}}%
\pgfpathlineto{\pgfqpoint{3.028233in}{2.166176in}}%
\pgfpathlineto{\pgfqpoint{3.028233in}{1.960294in}}%
\pgfpathlineto{\pgfqpoint{1.148750in}{1.960294in}}%
\pgfpathclose%
\pgfusepath{fill}%
\end{pgfscope}%
\begin{pgfscope}%
\pgfpathrectangle{\pgfqpoint{1.148750in}{0.550000in}}{\pgfqpoint{7.122250in}{3.850000in}}%
\pgfusepath{clip}%
\pgfsetbuttcap%
\pgfsetmiterjoin%
\definecolor{currentfill}{rgb}{0.248058,0.667205,0.350250}%
\pgfsetfillcolor{currentfill}%
\pgfsetlinewidth{0.000000pt}%
\definecolor{currentstroke}{rgb}{0.000000,0.000000,0.000000}%
\pgfsetstrokecolor{currentstroke}%
\pgfsetstrokeopacity{0.000000}%
\pgfsetdash{}{0pt}%
\pgfpathmoveto{\pgfqpoint{1.148750in}{1.754412in}}%
\pgfpathlineto{\pgfqpoint{2.467685in}{1.754412in}}%
\pgfpathlineto{\pgfqpoint{2.467685in}{1.548529in}}%
\pgfpathlineto{\pgfqpoint{1.148750in}{1.548529in}}%
\pgfpathclose%
\pgfusepath{fill}%
\end{pgfscope}%
\begin{pgfscope}%
\pgfpathrectangle{\pgfqpoint{1.148750in}{0.550000in}}{\pgfqpoint{7.122250in}{3.850000in}}%
\pgfusepath{clip}%
\pgfsetbuttcap%
\pgfsetmiterjoin%
\definecolor{currentfill}{rgb}{0.248058,0.667205,0.350250}%
\pgfsetfillcolor{currentfill}%
\pgfsetlinewidth{0.000000pt}%
\definecolor{currentstroke}{rgb}{0.000000,0.000000,0.000000}%
\pgfsetstrokecolor{currentstroke}%
\pgfsetstrokeopacity{0.000000}%
\pgfsetdash{}{0pt}%
\pgfpathmoveto{\pgfqpoint{1.148750in}{1.342647in}}%
\pgfpathlineto{\pgfqpoint{1.709297in}{1.342647in}}%
\pgfpathlineto{\pgfqpoint{1.709297in}{1.136765in}}%
\pgfpathlineto{\pgfqpoint{1.148750in}{1.136765in}}%
\pgfpathclose%
\pgfusepath{fill}%
\end{pgfscope}%
\begin{pgfscope}%
\pgfpathrectangle{\pgfqpoint{1.148750in}{0.550000in}}{\pgfqpoint{7.122250in}{3.850000in}}%
\pgfusepath{clip}%
\pgfsetbuttcap%
\pgfsetmiterjoin%
\definecolor{currentfill}{rgb}{0.248058,0.667205,0.350250}%
\pgfsetfillcolor{currentfill}%
\pgfsetfillopacity{0.500000}%
\pgfsetlinewidth{0.000000pt}%
\definecolor{currentstroke}{rgb}{0.000000,0.000000,0.000000}%
\pgfsetstrokecolor{currentstroke}%
\pgfsetstrokeopacity{0.500000}%
\pgfsetdash{}{0pt}%
\pgfpathmoveto{\pgfqpoint{1.148750in}{0.930882in}}%
\pgfpathlineto{\pgfqpoint{1.511457in}{0.930882in}}%
\pgfpathlineto{\pgfqpoint{1.511457in}{0.725000in}}%
\pgfpathlineto{\pgfqpoint{1.148750in}{0.725000in}}%
\pgfpathclose%
\pgfusepath{fill}%
\end{pgfscope}%
\begin{pgfscope}%
\pgfpathrectangle{\pgfqpoint{1.148750in}{0.550000in}}{\pgfqpoint{7.122250in}{3.850000in}}%
\pgfusepath{clip}%
\pgfsetbuttcap%
\pgfsetmiterjoin%
\definecolor{currentfill}{rgb}{0.898885,0.305498,0.206767}%
\pgfsetfillcolor{currentfill}%
\pgfsetfillopacity{0.500000}%
\pgfsetlinewidth{0.000000pt}%
\definecolor{currentstroke}{rgb}{0.000000,0.000000,0.000000}%
\pgfsetstrokecolor{currentstroke}%
\pgfsetstrokeopacity{0.500000}%
\pgfsetdash{}{0pt}%
\pgfpathmoveto{\pgfqpoint{6.655304in}{4.225000in}}%
\pgfpathlineto{\pgfqpoint{8.271000in}{4.225000in}}%
\pgfpathlineto{\pgfqpoint{8.271000in}{4.019118in}}%
\pgfpathlineto{\pgfqpoint{6.655304in}{4.019118in}}%
\pgfpathclose%
\pgfusepath{fill}%
\end{pgfscope}%
\begin{pgfscope}%
\pgfpathrectangle{\pgfqpoint{1.148750in}{0.550000in}}{\pgfqpoint{7.122250in}{3.850000in}}%
\pgfusepath{clip}%
\pgfsetbuttcap%
\pgfsetmiterjoin%
\definecolor{currentfill}{rgb}{0.898885,0.305498,0.206767}%
\pgfsetfillcolor{currentfill}%
\pgfsetfillopacity{0.500000}%
\pgfsetlinewidth{0.000000pt}%
\definecolor{currentstroke}{rgb}{0.000000,0.000000,0.000000}%
\pgfsetstrokecolor{currentstroke}%
\pgfsetstrokeopacity{0.500000}%
\pgfsetdash{}{0pt}%
\pgfpathmoveto{\pgfqpoint{5.402316in}{3.813235in}}%
\pgfpathlineto{\pgfqpoint{8.271000in}{3.813235in}}%
\pgfpathlineto{\pgfqpoint{8.271000in}{3.607353in}}%
\pgfpathlineto{\pgfqpoint{5.402316in}{3.607353in}}%
\pgfpathclose%
\pgfusepath{fill}%
\end{pgfscope}%
\begin{pgfscope}%
\pgfpathrectangle{\pgfqpoint{1.148750in}{0.550000in}}{\pgfqpoint{7.122250in}{3.850000in}}%
\pgfusepath{clip}%
\pgfsetbuttcap%
\pgfsetmiterjoin%
\definecolor{currentfill}{rgb}{0.898885,0.305498,0.206767}%
\pgfsetfillcolor{currentfill}%
\pgfsetfillopacity{0.500000}%
\pgfsetlinewidth{0.000000pt}%
\definecolor{currentstroke}{rgb}{0.000000,0.000000,0.000000}%
\pgfsetstrokecolor{currentstroke}%
\pgfsetstrokeopacity{0.500000}%
\pgfsetdash{}{0pt}%
\pgfpathmoveto{\pgfqpoint{4.775822in}{3.401471in}}%
\pgfpathlineto{\pgfqpoint{8.271000in}{3.401471in}}%
\pgfpathlineto{\pgfqpoint{8.271000in}{3.195588in}}%
\pgfpathlineto{\pgfqpoint{4.775822in}{3.195588in}}%
\pgfpathclose%
\pgfusepath{fill}%
\end{pgfscope}%
\begin{pgfscope}%
\pgfpathrectangle{\pgfqpoint{1.148750in}{0.550000in}}{\pgfqpoint{7.122250in}{3.850000in}}%
\pgfusepath{clip}%
\pgfsetbuttcap%
\pgfsetmiterjoin%
\definecolor{currentfill}{rgb}{0.898885,0.305498,0.206767}%
\pgfsetfillcolor{currentfill}%
\pgfsetlinewidth{0.000000pt}%
\definecolor{currentstroke}{rgb}{0.000000,0.000000,0.000000}%
\pgfsetstrokecolor{currentstroke}%
\pgfsetstrokeopacity{0.000000}%
\pgfsetdash{}{0pt}%
\pgfpathmoveto{\pgfqpoint{3.786620in}{2.989706in}}%
\pgfpathlineto{\pgfqpoint{8.271000in}{2.989706in}}%
\pgfpathlineto{\pgfqpoint{8.271000in}{2.783824in}}%
\pgfpathlineto{\pgfqpoint{3.786620in}{2.783824in}}%
\pgfpathclose%
\pgfusepath{fill}%
\end{pgfscope}%
\begin{pgfscope}%
\pgfpathrectangle{\pgfqpoint{1.148750in}{0.550000in}}{\pgfqpoint{7.122250in}{3.850000in}}%
\pgfusepath{clip}%
\pgfsetbuttcap%
\pgfsetmiterjoin%
\definecolor{currentfill}{rgb}{0.898885,0.305498,0.206767}%
\pgfsetfillcolor{currentfill}%
\pgfsetlinewidth{0.000000pt}%
\definecolor{currentstroke}{rgb}{0.000000,0.000000,0.000000}%
\pgfsetstrokecolor{currentstroke}%
\pgfsetstrokeopacity{0.000000}%
\pgfsetdash{}{0pt}%
\pgfpathmoveto{\pgfqpoint{3.160126in}{2.577941in}}%
\pgfpathlineto{\pgfqpoint{8.271000in}{2.577941in}}%
\pgfpathlineto{\pgfqpoint{8.271000in}{2.372059in}}%
\pgfpathlineto{\pgfqpoint{3.160126in}{2.372059in}}%
\pgfpathclose%
\pgfusepath{fill}%
\end{pgfscope}%
\begin{pgfscope}%
\pgfpathrectangle{\pgfqpoint{1.148750in}{0.550000in}}{\pgfqpoint{7.122250in}{3.850000in}}%
\pgfusepath{clip}%
\pgfsetbuttcap%
\pgfsetmiterjoin%
\definecolor{currentfill}{rgb}{0.898885,0.305498,0.206767}%
\pgfsetfillcolor{currentfill}%
\pgfsetfillopacity{0.500000}%
\pgfsetlinewidth{0.000000pt}%
\definecolor{currentstroke}{rgb}{0.000000,0.000000,0.000000}%
\pgfsetstrokecolor{currentstroke}%
\pgfsetstrokeopacity{0.500000}%
\pgfsetdash{}{0pt}%
\pgfpathmoveto{\pgfqpoint{3.028233in}{2.166176in}}%
\pgfpathlineto{\pgfqpoint{8.271000in}{2.166176in}}%
\pgfpathlineto{\pgfqpoint{8.271000in}{1.960294in}}%
\pgfpathlineto{\pgfqpoint{3.028233in}{1.960294in}}%
\pgfpathclose%
\pgfusepath{fill}%
\end{pgfscope}%
\begin{pgfscope}%
\pgfpathrectangle{\pgfqpoint{1.148750in}{0.550000in}}{\pgfqpoint{7.122250in}{3.850000in}}%
\pgfusepath{clip}%
\pgfsetbuttcap%
\pgfsetmiterjoin%
\definecolor{currentfill}{rgb}{0.898885,0.305498,0.206767}%
\pgfsetfillcolor{currentfill}%
\pgfsetlinewidth{0.000000pt}%
\definecolor{currentstroke}{rgb}{0.000000,0.000000,0.000000}%
\pgfsetstrokecolor{currentstroke}%
\pgfsetstrokeopacity{0.000000}%
\pgfsetdash{}{0pt}%
\pgfpathmoveto{\pgfqpoint{2.467685in}{1.754412in}}%
\pgfpathlineto{\pgfqpoint{8.271000in}{1.754412in}}%
\pgfpathlineto{\pgfqpoint{8.271000in}{1.548529in}}%
\pgfpathlineto{\pgfqpoint{2.467685in}{1.548529in}}%
\pgfpathclose%
\pgfusepath{fill}%
\end{pgfscope}%
\begin{pgfscope}%
\pgfpathrectangle{\pgfqpoint{1.148750in}{0.550000in}}{\pgfqpoint{7.122250in}{3.850000in}}%
\pgfusepath{clip}%
\pgfsetbuttcap%
\pgfsetmiterjoin%
\definecolor{currentfill}{rgb}{0.898885,0.305498,0.206767}%
\pgfsetfillcolor{currentfill}%
\pgfsetlinewidth{0.000000pt}%
\definecolor{currentstroke}{rgb}{0.000000,0.000000,0.000000}%
\pgfsetstrokecolor{currentstroke}%
\pgfsetstrokeopacity{0.000000}%
\pgfsetdash{}{0pt}%
\pgfpathmoveto{\pgfqpoint{1.709297in}{1.342647in}}%
\pgfpathlineto{\pgfqpoint{8.271000in}{1.342647in}}%
\pgfpathlineto{\pgfqpoint{8.271000in}{1.136765in}}%
\pgfpathlineto{\pgfqpoint{1.709297in}{1.136765in}}%
\pgfpathclose%
\pgfusepath{fill}%
\end{pgfscope}%
\begin{pgfscope}%
\pgfpathrectangle{\pgfqpoint{1.148750in}{0.550000in}}{\pgfqpoint{7.122250in}{3.850000in}}%
\pgfusepath{clip}%
\pgfsetbuttcap%
\pgfsetmiterjoin%
\definecolor{currentfill}{rgb}{0.898885,0.305498,0.206767}%
\pgfsetfillcolor{currentfill}%
\pgfsetfillopacity{0.500000}%
\pgfsetlinewidth{0.000000pt}%
\definecolor{currentstroke}{rgb}{0.000000,0.000000,0.000000}%
\pgfsetstrokecolor{currentstroke}%
\pgfsetstrokeopacity{0.500000}%
\pgfsetdash{}{0pt}%
\pgfpathmoveto{\pgfqpoint{1.511457in}{0.930882in}}%
\pgfpathlineto{\pgfqpoint{8.271000in}{0.930882in}}%
\pgfpathlineto{\pgfqpoint{8.271000in}{0.725000in}}%
\pgfpathlineto{\pgfqpoint{1.511457in}{0.725000in}}%
\pgfpathclose%
\pgfusepath{fill}%
\end{pgfscope}%
\begin{pgfscope}%
\pgfsetbuttcap%
\pgfsetroundjoin%
\definecolor{currentfill}{rgb}{0.000000,0.000000,0.000000}%
\pgfsetfillcolor{currentfill}%
\pgfsetlinewidth{0.803000pt}%
\definecolor{currentstroke}{rgb}{0.000000,0.000000,0.000000}%
\pgfsetstrokecolor{currentstroke}%
\pgfsetdash{}{0pt}%
\pgfsys@defobject{currentmarker}{\pgfqpoint{-0.048611in}{0.000000in}}{\pgfqpoint{-0.000000in}{0.000000in}}{%
\pgfpathmoveto{\pgfqpoint{-0.000000in}{0.000000in}}%
\pgfpathlineto{\pgfqpoint{-0.048611in}{0.000000in}}%
\pgfusepath{stroke,fill}%
}%
\begin{pgfscope}%
\pgfsys@transformshift{1.148750in}{4.122059in}%
\pgfsys@useobject{currentmarker}{}%
\end{pgfscope}%
\end{pgfscope}%
\begin{pgfscope}%
\definecolor{textcolor}{rgb}{0.000000,0.000000,0.000000}%
\pgfsetstrokecolor{textcolor}%
\pgfsetfillcolor{textcolor}%
\pgftext[x=0.654731in, y=4.069297in, left, base]{\color{textcolor}\sffamily\fontsize{10.000000}{12.000000}\selectfont Pix3D}%
\end{pgfscope}%
\begin{pgfscope}%
\pgfsetbuttcap%
\pgfsetroundjoin%
\definecolor{currentfill}{rgb}{0.000000,0.000000,0.000000}%
\pgfsetfillcolor{currentfill}%
\pgfsetlinewidth{0.803000pt}%
\definecolor{currentstroke}{rgb}{0.000000,0.000000,0.000000}%
\pgfsetstrokecolor{currentstroke}%
\pgfsetdash{}{0pt}%
\pgfsys@defobject{currentmarker}{\pgfqpoint{-0.048611in}{0.000000in}}{\pgfqpoint{-0.000000in}{0.000000in}}{%
\pgfpathmoveto{\pgfqpoint{-0.000000in}{0.000000in}}%
\pgfpathlineto{\pgfqpoint{-0.048611in}{0.000000in}}%
\pgfusepath{stroke,fill}%
}%
\begin{pgfscope}%
\pgfsys@transformshift{1.148750in}{3.710294in}%
\pgfsys@useobject{currentmarker}{}%
\end{pgfscope}%
\end{pgfscope}%
\begin{pgfscope}%
\definecolor{textcolor}{rgb}{0.000000,0.000000,0.000000}%
\pgfsetstrokecolor{textcolor}%
\pgfsetfillcolor{textcolor}%
\pgftext[x=0.387940in, y=3.657533in, left, base]{\color{textcolor}\sffamily\fontsize{10.000000}{12.000000}\selectfont Hyperism}%
\end{pgfscope}%
\begin{pgfscope}%
\pgfsetbuttcap%
\pgfsetroundjoin%
\definecolor{currentfill}{rgb}{0.000000,0.000000,0.000000}%
\pgfsetfillcolor{currentfill}%
\pgfsetlinewidth{0.803000pt}%
\definecolor{currentstroke}{rgb}{0.000000,0.000000,0.000000}%
\pgfsetstrokecolor{currentstroke}%
\pgfsetdash{}{0pt}%
\pgfsys@defobject{currentmarker}{\pgfqpoint{-0.048611in}{0.000000in}}{\pgfqpoint{-0.000000in}{0.000000in}}{%
\pgfpathmoveto{\pgfqpoint{-0.000000in}{0.000000in}}%
\pgfpathlineto{\pgfqpoint{-0.048611in}{0.000000in}}%
\pgfusepath{stroke,fill}%
}%
\begin{pgfscope}%
\pgfsys@transformshift{1.148750in}{3.298529in}%
\pgfsys@useobject{currentmarker}{}%
\end{pgfscope}%
\end{pgfscope}%
\begin{pgfscope}%
\definecolor{textcolor}{rgb}{0.000000,0.000000,0.000000}%
\pgfsetstrokecolor{textcolor}%
\pgfsetfillcolor{textcolor}%
\pgftext[x=0.301067in, y=3.245768in, left, base]{\color{textcolor}\sffamily\fontsize{10.000000}{12.000000}\selectfont InteriorNet}%
\end{pgfscope}%
\begin{pgfscope}%
\pgfsetbuttcap%
\pgfsetroundjoin%
\definecolor{currentfill}{rgb}{0.000000,0.000000,0.000000}%
\pgfsetfillcolor{currentfill}%
\pgfsetlinewidth{0.803000pt}%
\definecolor{currentstroke}{rgb}{0.000000,0.000000,0.000000}%
\pgfsetstrokecolor{currentstroke}%
\pgfsetdash{}{0pt}%
\pgfsys@defobject{currentmarker}{\pgfqpoint{-0.048611in}{0.000000in}}{\pgfqpoint{-0.000000in}{0.000000in}}{%
\pgfpathmoveto{\pgfqpoint{-0.000000in}{0.000000in}}%
\pgfpathlineto{\pgfqpoint{-0.048611in}{0.000000in}}%
\pgfusepath{stroke,fill}%
}%
\begin{pgfscope}%
\pgfsys@transformshift{1.148750in}{2.886765in}%
\pgfsys@useobject{currentmarker}{}%
\end{pgfscope}%
\end{pgfscope}%
\begin{pgfscope}%
\definecolor{textcolor}{rgb}{0.000000,0.000000,0.000000}%
\pgfsetstrokecolor{textcolor}%
\pgfsetfillcolor{textcolor}%
\pgftext[x=0.212701in, y=2.834003in, left, base]{\color{textcolor}\sffamily\fontsize{10.000000}{12.000000}\selectfont OpenRooms}%
\end{pgfscope}%
\begin{pgfscope}%
\pgfsetbuttcap%
\pgfsetroundjoin%
\definecolor{currentfill}{rgb}{0.000000,0.000000,0.000000}%
\pgfsetfillcolor{currentfill}%
\pgfsetlinewidth{0.803000pt}%
\definecolor{currentstroke}{rgb}{0.000000,0.000000,0.000000}%
\pgfsetstrokecolor{currentstroke}%
\pgfsetdash{}{0pt}%
\pgfsys@defobject{currentmarker}{\pgfqpoint{-0.048611in}{0.000000in}}{\pgfqpoint{-0.000000in}{0.000000in}}{%
\pgfpathmoveto{\pgfqpoint{-0.000000in}{0.000000in}}%
\pgfpathlineto{\pgfqpoint{-0.048611in}{0.000000in}}%
\pgfusepath{stroke,fill}%
}%
\begin{pgfscope}%
\pgfsys@transformshift{1.148750in}{2.475000in}%
\pgfsys@useobject{currentmarker}{}%
\end{pgfscope}%
\end{pgfscope}%
\begin{pgfscope}%
\definecolor{textcolor}{rgb}{0.000000,0.000000,0.000000}%
\pgfsetstrokecolor{textcolor}%
\pgfsetfillcolor{textcolor}%
\pgftext[x=0.209921in, y=2.422238in, left, base]{\color{textcolor}\sffamily\fontsize{10.000000}{12.000000}\selectfont Blenderproc}%
\end{pgfscope}%
\begin{pgfscope}%
\pgfsetbuttcap%
\pgfsetroundjoin%
\definecolor{currentfill}{rgb}{0.000000,0.000000,0.000000}%
\pgfsetfillcolor{currentfill}%
\pgfsetlinewidth{0.803000pt}%
\definecolor{currentstroke}{rgb}{0.000000,0.000000,0.000000}%
\pgfsetstrokecolor{currentstroke}%
\pgfsetdash{}{0pt}%
\pgfsys@defobject{currentmarker}{\pgfqpoint{-0.048611in}{0.000000in}}{\pgfqpoint{-0.000000in}{0.000000in}}{%
\pgfpathmoveto{\pgfqpoint{-0.000000in}{0.000000in}}%
\pgfpathlineto{\pgfqpoint{-0.048611in}{0.000000in}}%
\pgfusepath{stroke,fill}%
}%
\begin{pgfscope}%
\pgfsys@transformshift{1.148750in}{2.063235in}%
\pgfsys@useobject{currentmarker}{}%
\end{pgfscope}%
\end{pgfscope}%
\begin{pgfscope}%
\definecolor{textcolor}{rgb}{0.000000,0.000000,0.000000}%
\pgfsetstrokecolor{textcolor}%
\pgfsetfillcolor{textcolor}%
\pgftext[x=0.381769in, y=2.010474in, left, base]{\color{textcolor}\sffamily\fontsize{10.000000}{12.000000}\selectfont 3DFRONT}%
\end{pgfscope}%
\begin{pgfscope}%
\pgfsetbuttcap%
\pgfsetroundjoin%
\definecolor{currentfill}{rgb}{0.000000,0.000000,0.000000}%
\pgfsetfillcolor{currentfill}%
\pgfsetlinewidth{0.803000pt}%
\definecolor{currentstroke}{rgb}{0.000000,0.000000,0.000000}%
\pgfsetstrokecolor{currentstroke}%
\pgfsetdash{}{0pt}%
\pgfsys@defobject{currentmarker}{\pgfqpoint{-0.048611in}{0.000000in}}{\pgfqpoint{-0.000000in}{0.000000in}}{%
\pgfpathmoveto{\pgfqpoint{-0.000000in}{0.000000in}}%
\pgfpathlineto{\pgfqpoint{-0.048611in}{0.000000in}}%
\pgfusepath{stroke,fill}%
}%
\begin{pgfscope}%
\pgfsys@transformshift{1.148750in}{1.651471in}%
\pgfsys@useobject{currentmarker}{}%
\end{pgfscope}%
\end{pgfscope}%
\begin{pgfscope}%
\definecolor{textcolor}{rgb}{0.000000,0.000000,0.000000}%
\pgfsetstrokecolor{textcolor}%
\pgfsetfillcolor{textcolor}%
\pgftext[x=0.384278in, y=1.598709in, left, base]{\color{textcolor}\sffamily\fontsize{10.000000}{12.000000}\selectfont SceneNet}%
\end{pgfscope}%
\begin{pgfscope}%
\pgfsetbuttcap%
\pgfsetroundjoin%
\definecolor{currentfill}{rgb}{0.000000,0.000000,0.000000}%
\pgfsetfillcolor{currentfill}%
\pgfsetlinewidth{0.803000pt}%
\definecolor{currentstroke}{rgb}{0.000000,0.000000,0.000000}%
\pgfsetstrokecolor{currentstroke}%
\pgfsetdash{}{0pt}%
\pgfsys@defobject{currentmarker}{\pgfqpoint{-0.048611in}{0.000000in}}{\pgfqpoint{-0.000000in}{0.000000in}}{%
\pgfpathmoveto{\pgfqpoint{-0.000000in}{0.000000in}}%
\pgfpathlineto{\pgfqpoint{-0.048611in}{0.000000in}}%
\pgfusepath{stroke,fill}%
}%
\begin{pgfscope}%
\pgfsys@transformshift{1.148750in}{1.239706in}%
\pgfsys@useobject{currentmarker}{}%
\end{pgfscope}%
\end{pgfscope}%
\begin{pgfscope}%
\definecolor{textcolor}{rgb}{0.000000,0.000000,0.000000}%
\pgfsetstrokecolor{textcolor}%
\pgfsetfillcolor{textcolor}%
\pgftext[x=0.188762in, y=1.186944in, left, base]{\color{textcolor}\sffamily\fontsize{10.000000}{12.000000}\selectfont S2R:3DFREE}%
\end{pgfscope}%
\begin{pgfscope}%
\pgfsetbuttcap%
\pgfsetroundjoin%
\definecolor{currentfill}{rgb}{0.000000,0.000000,0.000000}%
\pgfsetfillcolor{currentfill}%
\pgfsetlinewidth{0.803000pt}%
\definecolor{currentstroke}{rgb}{0.000000,0.000000,0.000000}%
\pgfsetstrokecolor{currentstroke}%
\pgfsetdash{}{0pt}%
\pgfsys@defobject{currentmarker}{\pgfqpoint{-0.048611in}{0.000000in}}{\pgfqpoint{-0.000000in}{0.000000in}}{%
\pgfpathmoveto{\pgfqpoint{-0.000000in}{0.000000in}}%
\pgfpathlineto{\pgfqpoint{-0.048611in}{0.000000in}}%
\pgfusepath{stroke,fill}%
}%
\begin{pgfscope}%
\pgfsys@transformshift{1.148750in}{0.827941in}%
\pgfsys@useobject{currentmarker}{}%
\end{pgfscope}%
\end{pgfscope}%
\begin{pgfscope}%
\definecolor{textcolor}{rgb}{0.000000,0.000000,0.000000}%
\pgfsetstrokecolor{textcolor}%
\pgfsetfillcolor{textcolor}%
\pgftext[x=0.432089in, y=0.775180in, left, base]{\color{textcolor}\sffamily\fontsize{10.000000}{12.000000}\selectfont AI2THOR}%
\end{pgfscope}%
\begin{pgfscope}%
\definecolor{textcolor}{rgb}{0.000000,0.000000,0.000000}%
\pgfsetstrokecolor{textcolor}%
\pgfsetfillcolor{textcolor}%
\pgftext[x=0.133206in,y=2.475000in,,bottom,rotate=90.000000]{\color{textcolor}\sffamily\fontsize{10.000000}{12.000000}\selectfont Datasets}%
\end{pgfscope}%
\begin{pgfscope}%
\pgfpathrectangle{\pgfqpoint{1.148750in}{0.550000in}}{\pgfqpoint{7.122250in}{3.850000in}}%
\pgfusepath{clip}%
\pgfsetbuttcap%
\pgfsetroundjoin%
\pgfsetlinewidth{1.505625pt}%
\definecolor{currentstroke}{rgb}{0.000000,0.000000,0.000000}%
\pgfsetstrokecolor{currentstroke}%
\pgfsetstrokeopacity{0.200000}%
\pgfsetdash{{5.550000pt}{2.400000pt}}{0.000000pt}%
\pgfpathmoveto{\pgfqpoint{4.709875in}{0.550000in}}%
\pgfpathlineto{\pgfqpoint{4.709875in}{4.400000in}}%
\pgfusepath{stroke}%
\end{pgfscope}%
\begin{pgfscope}%
\pgfsetrectcap%
\pgfsetmiterjoin%
\pgfsetlinewidth{0.803000pt}%
\definecolor{currentstroke}{rgb}{0.000000,0.000000,0.000000}%
\pgfsetstrokecolor{currentstroke}%
\pgfsetdash{}{0pt}%
\pgfpathmoveto{\pgfqpoint{1.148750in}{0.550000in}}%
\pgfpathlineto{\pgfqpoint{1.148750in}{4.400000in}}%
\pgfusepath{stroke}%
\end{pgfscope}%
\begin{pgfscope}%
\pgfsetrectcap%
\pgfsetmiterjoin%
\pgfsetlinewidth{0.803000pt}%
\definecolor{currentstroke}{rgb}{0.000000,0.000000,0.000000}%
\pgfsetstrokecolor{currentstroke}%
\pgfsetdash{}{0pt}%
\pgfpathmoveto{\pgfqpoint{8.271000in}{0.550000in}}%
\pgfpathlineto{\pgfqpoint{8.271000in}{4.400000in}}%
\pgfusepath{stroke}%
\end{pgfscope}%
\begin{pgfscope}%
\pgfsetrectcap%
\pgfsetmiterjoin%
\pgfsetlinewidth{0.803000pt}%
\definecolor{currentstroke}{rgb}{0.000000,0.000000,0.000000}%
\pgfsetstrokecolor{currentstroke}%
\pgfsetdash{}{0pt}%
\pgfpathmoveto{\pgfqpoint{1.148750in}{0.550000in}}%
\pgfpathlineto{\pgfqpoint{8.271000in}{0.550000in}}%
\pgfusepath{stroke}%
\end{pgfscope}%
\begin{pgfscope}%
\pgfsetrectcap%
\pgfsetmiterjoin%
\pgfsetlinewidth{0.803000pt}%
\definecolor{currentstroke}{rgb}{0.000000,0.000000,0.000000}%
\pgfsetstrokecolor{currentstroke}%
\pgfsetdash{}{0pt}%
\pgfpathmoveto{\pgfqpoint{1.148750in}{4.400000in}}%
\pgfpathlineto{\pgfqpoint{8.271000in}{4.400000in}}%
\pgfusepath{stroke}%
\end{pgfscope}%
\begin{pgfscope}%
\definecolor{textcolor}{rgb}{1.000000,1.000000,1.000000}%
\pgfsetstrokecolor{textcolor}%
\pgfsetfillcolor{textcolor}%
\pgftext[x=3.902027in,y=4.122059in,,]{\color{textcolor}\sffamily\fontsize{10.000000}{12.000000}\selectfont 167}%
\end{pgfscope}%
\begin{pgfscope}%
\definecolor{textcolor}{rgb}{1.000000,1.000000,1.000000}%
\pgfsetstrokecolor{textcolor}%
\pgfsetfillcolor{textcolor}%
\pgftext[x=3.275533in,y=3.710294in,,]{\color{textcolor}\sffamily\fontsize{10.000000}{12.000000}\selectfont 129}%
\end{pgfscope}%
\begin{pgfscope}%
\definecolor{textcolor}{rgb}{1.000000,1.000000,1.000000}%
\pgfsetstrokecolor{textcolor}%
\pgfsetfillcolor{textcolor}%
\pgftext[x=2.962286in,y=3.298529in,,]{\color{textcolor}\sffamily\fontsize{10.000000}{12.000000}\selectfont 110}%
\end{pgfscope}%
\begin{pgfscope}%
\definecolor{textcolor}{rgb}{1.000000,1.000000,1.000000}%
\pgfsetstrokecolor{textcolor}%
\pgfsetfillcolor{textcolor}%
\pgftext[x=2.467685in,y=2.886765in,,]{\color{textcolor}\sffamily\fontsize{10.000000}{12.000000}\selectfont 80}%
\end{pgfscope}%
\begin{pgfscope}%
\definecolor{textcolor}{rgb}{1.000000,1.000000,1.000000}%
\pgfsetstrokecolor{textcolor}%
\pgfsetfillcolor{textcolor}%
\pgftext[x=2.154438in,y=2.475000in,,]{\color{textcolor}\sffamily\fontsize{10.000000}{12.000000}\selectfont 61}%
\end{pgfscope}%
\begin{pgfscope}%
\definecolor{textcolor}{rgb}{1.000000,1.000000,1.000000}%
\pgfsetstrokecolor{textcolor}%
\pgfsetfillcolor{textcolor}%
\pgftext[x=2.088491in,y=2.063235in,,]{\color{textcolor}\sffamily\fontsize{10.000000}{12.000000}\selectfont 57}%
\end{pgfscope}%
\begin{pgfscope}%
\definecolor{textcolor}{rgb}{1.000000,1.000000,1.000000}%
\pgfsetstrokecolor{textcolor}%
\pgfsetfillcolor{textcolor}%
\pgftext[x=1.808218in,y=1.651471in,,]{\color{textcolor}\sffamily\fontsize{10.000000}{12.000000}\selectfont 40}%
\end{pgfscope}%
\begin{pgfscope}%
\definecolor{textcolor}{rgb}{1.000000,1.000000,1.000000}%
\pgfsetstrokecolor{textcolor}%
\pgfsetfillcolor{textcolor}%
\pgftext[x=1.429024in,y=1.239706in,,]{\color{textcolor}\sffamily\fontsize{10.000000}{12.000000}\selectfont 17}%
\end{pgfscope}%
\begin{pgfscope}%
\definecolor{textcolor}{rgb}{1.000000,1.000000,1.000000}%
\pgfsetstrokecolor{textcolor}%
\pgfsetfillcolor{textcolor}%
\pgftext[x=1.330104in,y=0.827941in,,]{\color{textcolor}\sffamily\fontsize{10.000000}{12.000000}\selectfont 11}%
\end{pgfscope}%
\begin{pgfscope}%
\definecolor{textcolor}{rgb}{1.000000,1.000000,1.000000}%
\pgfsetstrokecolor{textcolor}%
\pgfsetfillcolor{textcolor}%
\pgftext[x=7.463152in,y=4.122059in,,]{\color{textcolor}\sffamily\fontsize{10.000000}{12.000000}\selectfont 49}%
\end{pgfscope}%
\begin{pgfscope}%
\definecolor{textcolor}{rgb}{1.000000,1.000000,1.000000}%
\pgfsetstrokecolor{textcolor}%
\pgfsetfillcolor{textcolor}%
\pgftext[x=6.836658in,y=3.710294in,,]{\color{textcolor}\sffamily\fontsize{10.000000}{12.000000}\selectfont 87}%
\end{pgfscope}%
\begin{pgfscope}%
\definecolor{textcolor}{rgb}{1.000000,1.000000,1.000000}%
\pgfsetstrokecolor{textcolor}%
\pgfsetfillcolor{textcolor}%
\pgftext[x=6.523411in,y=3.298529in,,]{\color{textcolor}\sffamily\fontsize{10.000000}{12.000000}\selectfont 106}%
\end{pgfscope}%
\begin{pgfscope}%
\definecolor{textcolor}{rgb}{1.000000,1.000000,1.000000}%
\pgfsetstrokecolor{textcolor}%
\pgfsetfillcolor{textcolor}%
\pgftext[x=6.028810in,y=2.886765in,,]{\color{textcolor}\sffamily\fontsize{10.000000}{12.000000}\selectfont 136}%
\end{pgfscope}%
\begin{pgfscope}%
\definecolor{textcolor}{rgb}{1.000000,1.000000,1.000000}%
\pgfsetstrokecolor{textcolor}%
\pgfsetfillcolor{textcolor}%
\pgftext[x=5.715563in,y=2.475000in,,]{\color{textcolor}\sffamily\fontsize{10.000000}{12.000000}\selectfont 155}%
\end{pgfscope}%
\begin{pgfscope}%
\definecolor{textcolor}{rgb}{1.000000,1.000000,1.000000}%
\pgfsetstrokecolor{textcolor}%
\pgfsetfillcolor{textcolor}%
\pgftext[x=5.649616in,y=2.063235in,,]{\color{textcolor}\sffamily\fontsize{10.000000}{12.000000}\selectfont 159}%
\end{pgfscope}%
\begin{pgfscope}%
\definecolor{textcolor}{rgb}{1.000000,1.000000,1.000000}%
\pgfsetstrokecolor{textcolor}%
\pgfsetfillcolor{textcolor}%
\pgftext[x=5.369343in,y=1.651471in,,]{\color{textcolor}\sffamily\fontsize{10.000000}{12.000000}\selectfont 176}%
\end{pgfscope}%
\begin{pgfscope}%
\definecolor{textcolor}{rgb}{1.000000,1.000000,1.000000}%
\pgfsetstrokecolor{textcolor}%
\pgfsetfillcolor{textcolor}%
\pgftext[x=4.990149in,y=1.239706in,,]{\color{textcolor}\sffamily\fontsize{10.000000}{12.000000}\selectfont 199}%
\end{pgfscope}%
\begin{pgfscope}%
\definecolor{textcolor}{rgb}{1.000000,1.000000,1.000000}%
\pgfsetstrokecolor{textcolor}%
\pgfsetfillcolor{textcolor}%
\pgftext[x=4.891229in,y=0.827941in,,]{\color{textcolor}\sffamily\fontsize{10.000000}{12.000000}\selectfont 205}%
\end{pgfscope}%
\begin{pgfscope}%
\pgfsetbuttcap%
\pgfsetmiterjoin%
\definecolor{currentfill}{rgb}{1.000000,1.000000,1.000000}%
\pgfsetfillcolor{currentfill}%
\pgfsetfillopacity{0.800000}%
\pgfsetlinewidth{1.003750pt}%
\definecolor{currentstroke}{rgb}{0.800000,0.800000,0.800000}%
\pgfsetstrokecolor{currentstroke}%
\pgfsetstrokeopacity{0.800000}%
\pgfsetdash{}{0pt}%
\pgfpathmoveto{\pgfqpoint{1.229736in}{4.457847in}}%
\pgfpathlineto{\pgfqpoint{2.893579in}{4.457847in}}%
\pgfpathquadraticcurveto{\pgfqpoint{2.916717in}{4.457847in}}{\pgfqpoint{2.916717in}{4.480986in}}%
\pgfpathlineto{\pgfqpoint{2.916717in}{4.639230in}}%
\pgfpathquadraticcurveto{\pgfqpoint{2.916717in}{4.662369in}}{\pgfqpoint{2.893579in}{4.662369in}}%
\pgfpathlineto{\pgfqpoint{1.229736in}{4.662369in}}%
\pgfpathquadraticcurveto{\pgfqpoint{1.206597in}{4.662369in}}{\pgfqpoint{1.206597in}{4.639230in}}%
\pgfpathlineto{\pgfqpoint{1.206597in}{4.480986in}}%
\pgfpathquadraticcurveto{\pgfqpoint{1.206597in}{4.457847in}}{\pgfqpoint{1.229736in}{4.457847in}}%
\pgfpathclose%
\pgfusepath{stroke,fill}%
\end{pgfscope}%
\begin{pgfscope}%
\pgfsetbuttcap%
\pgfsetmiterjoin%
\definecolor{currentfill}{rgb}{0.248058,0.667205,0.350250}%
\pgfsetfillcolor{currentfill}%
\pgfsetfillopacity{0.500000}%
\pgfsetlinewidth{0.000000pt}%
\definecolor{currentstroke}{rgb}{0.000000,0.000000,0.000000}%
\pgfsetstrokecolor{currentstroke}%
\pgfsetstrokeopacity{0.500000}%
\pgfsetdash{}{0pt}%
\pgfpathmoveto{\pgfqpoint{1.252875in}{4.528190in}}%
\pgfpathlineto{\pgfqpoint{1.484264in}{4.528190in}}%
\pgfpathlineto{\pgfqpoint{1.484264in}{4.609176in}}%
\pgfpathlineto{\pgfqpoint{1.252875in}{4.609176in}}%
\pgfpathclose%
\pgfusepath{fill}%
\end{pgfscope}%
\begin{pgfscope}%
\definecolor{textcolor}{rgb}{0.000000,0.000000,0.000000}%
\pgfsetstrokecolor{textcolor}%
\pgfsetfillcolor{textcolor}%
\pgftext[x=1.576819in,y=4.528190in,left,base]{\color{textcolor}\sffamily\fontsize{8.330000}{9.996000}\selectfont Real}%
\end{pgfscope}%
\begin{pgfscope}%
\pgfsetbuttcap%
\pgfsetmiterjoin%
\definecolor{currentfill}{rgb}{0.898885,0.305498,0.206767}%
\pgfsetfillcolor{currentfill}%
\pgfsetfillopacity{0.500000}%
\pgfsetlinewidth{0.000000pt}%
\definecolor{currentstroke}{rgb}{0.000000,0.000000,0.000000}%
\pgfsetstrokecolor{currentstroke}%
\pgfsetstrokeopacity{0.500000}%
\pgfsetdash{}{0pt}%
\pgfpathmoveto{\pgfqpoint{2.057618in}{4.528190in}}%
\pgfpathlineto{\pgfqpoint{2.289007in}{4.528190in}}%
\pgfpathlineto{\pgfqpoint{2.289007in}{4.609176in}}%
\pgfpathlineto{\pgfqpoint{2.057618in}{4.609176in}}%
\pgfpathclose%
\pgfusepath{fill}%
\end{pgfscope}%
\begin{pgfscope}%
\definecolor{textcolor}{rgb}{0.000000,0.000000,0.000000}%
\pgfsetstrokecolor{textcolor}%
\pgfsetfillcolor{textcolor}%
\pgftext[x=2.381562in,y=4.528190in,left,base]{\color{textcolor}\sffamily\fontsize{8.330000}{9.996000}\selectfont Not Real}%
\end{pgfscope}%
\end{pgfpicture}%
\makeatother%
\endgroup%
}
    \caption[Distibution for Survey:Section 1]{The figure represents distribution for Section 1 of survey. The participants were asked to distinguish if the image was 'Real' or 'Nor Real'.
    The automated dataset is highlighted with bolder color. The dotted line is 50\% threshold. All the automated dataset have less than threshold votes for 'Real'.}
    \label{fig:question1}
\end{figure}

\subsection{Section 2: Likert Scale}
In section 2, the participants could select ratings from 1 to 10 (1 being the least photorealistic).
The distribution of values for each dataset can be seen in \autoref{fig:question2}, and the average ratings are as seen in \autoref{fig:question2_2}.
If we consider the scale of 1, which is the least rating given to the image, \gls{ai2thor} has the most votes.
Suppose we have a cut-off at scale of 2 and 3; Openrooms and \gls{free} are the least photorealistic, respectively.
Interestingly Interiornet has the least number of scale 1 instead of Pix3D, which is the real dataset.
However, Pix3D has the highest number of the perfect score(10) among all the datasets.
Coming to the averages, we again see the datasets created from the automated pipeline (Blenderproc, SceneNet, Openrooms, \gls{free}), have the least average ratings,
while the manually created datasets(Hyperism, \gls{front}, InteriorNet) have higher average ratings.
Pix3D has highest of the average ratings, closely followed by InteriorNet.
Even though \gls{free} has the least average, it is still comparable to other automated pipelines as highlighted in \autoref{fig:question2_2} and even the Unity-based \gls{ai2thor} dataset.

\begin{figure}
    \centering
    \resizebox{\textwidth}{!}{%% Creator: Matplotlib, PGF backend
%%
%% To include the figure in your LaTeX document, write
%%   \input{<filename>.pgf}
%%
%% Make sure the required packages are loaded in your preamble
%%   \usepackage{pgf}
%%
%% Figures using additional raster images can only be included by \input if
%% they are in the same directory as the main LaTeX file. For loading figures
%% from other directories you can use the `import` package
%%   \usepackage{import}
%%
%% and then include the figures with
%%   \import{<path to file>}{<filename>.pgf}
%%
%% Matplotlib used the following preamble
%%   \usepackage{fontspec}
%%   \setmainfont{DejaVuSerif.ttf}[Path=\detokenize{/Users/apple/opt/anaconda3/envs/kaolin/lib/python3.7/site-packages/matplotlib/mpl-data/fonts/ttf/}]
%%   \setsansfont{DejaVuSans.ttf}[Path=\detokenize{/Users/apple/opt/anaconda3/envs/kaolin/lib/python3.7/site-packages/matplotlib/mpl-data/fonts/ttf/}]
%%   \setmonofont{DejaVuSansMono.ttf}[Path=\detokenize{/Users/apple/opt/anaconda3/envs/kaolin/lib/python3.7/site-packages/matplotlib/mpl-data/fonts/ttf/}]
%%
\begingroup%
\makeatletter%
\begin{pgfpicture}%
\pgfpathrectangle{\pgfpointorigin}{\pgfqpoint{8.472206in}{4.360980in}}%
\pgfusepath{use as bounding box, clip}%
\begin{pgfscope}%
\pgfsetbuttcap%
\pgfsetmiterjoin%
\definecolor{currentfill}{rgb}{1.000000,1.000000,1.000000}%
\pgfsetfillcolor{currentfill}%
\pgfsetlinewidth{0.000000pt}%
\definecolor{currentstroke}{rgb}{1.000000,1.000000,1.000000}%
\pgfsetstrokecolor{currentstroke}%
\pgfsetdash{}{0pt}%
\pgfpathmoveto{\pgfqpoint{0.000000in}{0.000000in}}%
\pgfpathlineto{\pgfqpoint{8.472206in}{0.000000in}}%
\pgfpathlineto{\pgfqpoint{8.472206in}{4.360980in}}%
\pgfpathlineto{\pgfqpoint{0.000000in}{4.360980in}}%
\pgfpathclose%
\pgfusepath{fill}%
\end{pgfscope}%
\begin{pgfscope}%
\pgfsetbuttcap%
\pgfsetmiterjoin%
\definecolor{currentfill}{rgb}{1.000000,1.000000,1.000000}%
\pgfsetfillcolor{currentfill}%
\pgfsetlinewidth{0.000000pt}%
\definecolor{currentstroke}{rgb}{0.000000,0.000000,0.000000}%
\pgfsetstrokecolor{currentstroke}%
\pgfsetstrokeopacity{0.000000}%
\pgfsetdash{}{0pt}%
\pgfpathmoveto{\pgfqpoint{1.249956in}{0.148611in}}%
\pgfpathlineto{\pgfqpoint{8.372206in}{0.148611in}}%
\pgfpathlineto{\pgfqpoint{8.372206in}{3.998611in}}%
\pgfpathlineto{\pgfqpoint{1.249956in}{3.998611in}}%
\pgfpathclose%
\pgfusepath{fill}%
\end{pgfscope}%
\begin{pgfscope}%
\pgfpathrectangle{\pgfqpoint{1.249956in}{0.148611in}}{\pgfqpoint{7.122250in}{3.850000in}}%
\pgfusepath{clip}%
\pgfsetbuttcap%
\pgfsetmiterjoin%
\definecolor{currentfill}{rgb}{0.898885,0.305498,0.206767}%
\pgfsetfillcolor{currentfill}%
\pgfsetfillopacity{0.500000}%
\pgfsetlinewidth{0.000000pt}%
\definecolor{currentstroke}{rgb}{0.000000,0.000000,0.000000}%
\pgfsetstrokecolor{currentstroke}%
\pgfsetstrokeopacity{0.500000}%
\pgfsetdash{}{0pt}%
\pgfpathmoveto{\pgfqpoint{1.249956in}{3.823611in}}%
\pgfpathlineto{\pgfqpoint{2.502945in}{3.823611in}}%
\pgfpathlineto{\pgfqpoint{2.502945in}{3.617729in}}%
\pgfpathlineto{\pgfqpoint{1.249956in}{3.617729in}}%
\pgfpathclose%
\pgfusepath{fill}%
\end{pgfscope}%
\begin{pgfscope}%
\pgfpathrectangle{\pgfqpoint{1.249956in}{0.148611in}}{\pgfqpoint{7.122250in}{3.850000in}}%
\pgfusepath{clip}%
\pgfsetbuttcap%
\pgfsetmiterjoin%
\definecolor{currentfill}{rgb}{0.898885,0.305498,0.206767}%
\pgfsetfillcolor{currentfill}%
\pgfsetfillopacity{0.500000}%
\pgfsetlinewidth{0.000000pt}%
\definecolor{currentstroke}{rgb}{0.000000,0.000000,0.000000}%
\pgfsetstrokecolor{currentstroke}%
\pgfsetstrokeopacity{0.500000}%
\pgfsetdash{}{0pt}%
\pgfpathmoveto{\pgfqpoint{1.249956in}{3.411846in}}%
\pgfpathlineto{\pgfqpoint{3.294306in}{3.411846in}}%
\pgfpathlineto{\pgfqpoint{3.294306in}{3.205964in}}%
\pgfpathlineto{\pgfqpoint{1.249956in}{3.205964in}}%
\pgfpathclose%
\pgfusepath{fill}%
\end{pgfscope}%
\begin{pgfscope}%
\pgfpathrectangle{\pgfqpoint{1.249956in}{0.148611in}}{\pgfqpoint{7.122250in}{3.850000in}}%
\pgfusepath{clip}%
\pgfsetbuttcap%
\pgfsetmiterjoin%
\definecolor{currentfill}{rgb}{0.898885,0.305498,0.206767}%
\pgfsetfillcolor{currentfill}%
\pgfsetlinewidth{0.000000pt}%
\definecolor{currentstroke}{rgb}{0.000000,0.000000,0.000000}%
\pgfsetstrokecolor{currentstroke}%
\pgfsetstrokeopacity{0.000000}%
\pgfsetdash{}{0pt}%
\pgfpathmoveto{\pgfqpoint{1.249956in}{3.000082in}}%
\pgfpathlineto{\pgfqpoint{2.700785in}{3.000082in}}%
\pgfpathlineto{\pgfqpoint{2.700785in}{2.794199in}}%
\pgfpathlineto{\pgfqpoint{1.249956in}{2.794199in}}%
\pgfpathclose%
\pgfusepath{fill}%
\end{pgfscope}%
\begin{pgfscope}%
\pgfpathrectangle{\pgfqpoint{1.249956in}{0.148611in}}{\pgfqpoint{7.122250in}{3.850000in}}%
\pgfusepath{clip}%
\pgfsetbuttcap%
\pgfsetmiterjoin%
\definecolor{currentfill}{rgb}{0.898885,0.305498,0.206767}%
\pgfsetfillcolor{currentfill}%
\pgfsetfillopacity{0.500000}%
\pgfsetlinewidth{0.000000pt}%
\definecolor{currentstroke}{rgb}{0.000000,0.000000,0.000000}%
\pgfsetstrokecolor{currentstroke}%
\pgfsetstrokeopacity{0.500000}%
\pgfsetdash{}{0pt}%
\pgfpathmoveto{\pgfqpoint{1.249956in}{2.588317in}}%
\pgfpathlineto{\pgfqpoint{1.843477in}{2.588317in}}%
\pgfpathlineto{\pgfqpoint{1.843477in}{2.382435in}}%
\pgfpathlineto{\pgfqpoint{1.249956in}{2.382435in}}%
\pgfpathclose%
\pgfusepath{fill}%
\end{pgfscope}%
\begin{pgfscope}%
\pgfpathrectangle{\pgfqpoint{1.249956in}{0.148611in}}{\pgfqpoint{7.122250in}{3.850000in}}%
\pgfusepath{clip}%
\pgfsetbuttcap%
\pgfsetmiterjoin%
\definecolor{currentfill}{rgb}{0.898885,0.305498,0.206767}%
\pgfsetfillcolor{currentfill}%
\pgfsetfillopacity{0.500000}%
\pgfsetlinewidth{0.000000pt}%
\definecolor{currentstroke}{rgb}{0.000000,0.000000,0.000000}%
\pgfsetstrokecolor{currentstroke}%
\pgfsetstrokeopacity{0.500000}%
\pgfsetdash{}{0pt}%
\pgfpathmoveto{\pgfqpoint{1.249956in}{2.176552in}}%
\pgfpathlineto{\pgfqpoint{1.612664in}{2.176552in}}%
\pgfpathlineto{\pgfqpoint{1.612664in}{1.970670in}}%
\pgfpathlineto{\pgfqpoint{1.249956in}{1.970670in}}%
\pgfpathclose%
\pgfusepath{fill}%
\end{pgfscope}%
\begin{pgfscope}%
\pgfpathrectangle{\pgfqpoint{1.249956in}{0.148611in}}{\pgfqpoint{7.122250in}{3.850000in}}%
\pgfusepath{clip}%
\pgfsetbuttcap%
\pgfsetmiterjoin%
\definecolor{currentfill}{rgb}{0.898885,0.305498,0.206767}%
\pgfsetfillcolor{currentfill}%
\pgfsetlinewidth{0.000000pt}%
\definecolor{currentstroke}{rgb}{0.000000,0.000000,0.000000}%
\pgfsetstrokecolor{currentstroke}%
\pgfsetstrokeopacity{0.000000}%
\pgfsetdash{}{0pt}%
\pgfpathmoveto{\pgfqpoint{1.249956in}{1.764788in}}%
\pgfpathlineto{\pgfqpoint{3.096466in}{1.764788in}}%
\pgfpathlineto{\pgfqpoint{3.096466in}{1.558905in}}%
\pgfpathlineto{\pgfqpoint{1.249956in}{1.558905in}}%
\pgfpathclose%
\pgfusepath{fill}%
\end{pgfscope}%
\begin{pgfscope}%
\pgfpathrectangle{\pgfqpoint{1.249956in}{0.148611in}}{\pgfqpoint{7.122250in}{3.850000in}}%
\pgfusepath{clip}%
\pgfsetbuttcap%
\pgfsetmiterjoin%
\definecolor{currentfill}{rgb}{0.898885,0.305498,0.206767}%
\pgfsetfillcolor{currentfill}%
\pgfsetfillopacity{0.500000}%
\pgfsetlinewidth{0.000000pt}%
\definecolor{currentstroke}{rgb}{0.000000,0.000000,0.000000}%
\pgfsetstrokecolor{currentstroke}%
\pgfsetstrokeopacity{0.500000}%
\pgfsetdash{}{0pt}%
\pgfpathmoveto{\pgfqpoint{1.249956in}{1.353023in}}%
\pgfpathlineto{\pgfqpoint{1.678610in}{1.353023in}}%
\pgfpathlineto{\pgfqpoint{1.678610in}{1.147141in}}%
\pgfpathlineto{\pgfqpoint{1.249956in}{1.147141in}}%
\pgfpathclose%
\pgfusepath{fill}%
\end{pgfscope}%
\begin{pgfscope}%
\pgfpathrectangle{\pgfqpoint{1.249956in}{0.148611in}}{\pgfqpoint{7.122250in}{3.850000in}}%
\pgfusepath{clip}%
\pgfsetbuttcap%
\pgfsetmiterjoin%
\definecolor{currentfill}{rgb}{0.898885,0.305498,0.206767}%
\pgfsetfillcolor{currentfill}%
\pgfsetlinewidth{0.000000pt}%
\definecolor{currentstroke}{rgb}{0.000000,0.000000,0.000000}%
\pgfsetstrokecolor{currentstroke}%
\pgfsetstrokeopacity{0.000000}%
\pgfsetdash{}{0pt}%
\pgfpathmoveto{\pgfqpoint{1.249956in}{0.941258in}}%
\pgfpathlineto{\pgfqpoint{3.162412in}{0.941258in}}%
\pgfpathlineto{\pgfqpoint{3.162412in}{0.735376in}}%
\pgfpathlineto{\pgfqpoint{1.249956in}{0.735376in}}%
\pgfpathclose%
\pgfusepath{fill}%
\end{pgfscope}%
\begin{pgfscope}%
\pgfpathrectangle{\pgfqpoint{1.249956in}{0.148611in}}{\pgfqpoint{7.122250in}{3.850000in}}%
\pgfusepath{clip}%
\pgfsetbuttcap%
\pgfsetmiterjoin%
\definecolor{currentfill}{rgb}{0.898885,0.305498,0.206767}%
\pgfsetfillcolor{currentfill}%
\pgfsetlinewidth{0.000000pt}%
\definecolor{currentstroke}{rgb}{0.000000,0.000000,0.000000}%
\pgfsetstrokecolor{currentstroke}%
\pgfsetstrokeopacity{0.000000}%
\pgfsetdash{}{0pt}%
\pgfpathmoveto{\pgfqpoint{1.249956in}{0.529493in}}%
\pgfpathlineto{\pgfqpoint{2.700785in}{0.529493in}}%
\pgfpathlineto{\pgfqpoint{2.700785in}{0.323611in}}%
\pgfpathlineto{\pgfqpoint{1.249956in}{0.323611in}}%
\pgfpathclose%
\pgfusepath{fill}%
\end{pgfscope}%
\begin{pgfscope}%
\pgfpathrectangle{\pgfqpoint{1.249956in}{0.148611in}}{\pgfqpoint{7.122250in}{3.850000in}}%
\pgfusepath{clip}%
\pgfsetbuttcap%
\pgfsetmiterjoin%
\definecolor{currentfill}{rgb}{0.966551,0.497424,0.295040}%
\pgfsetfillcolor{currentfill}%
\pgfsetfillopacity{0.500000}%
\pgfsetlinewidth{0.000000pt}%
\definecolor{currentstroke}{rgb}{0.000000,0.000000,0.000000}%
\pgfsetstrokecolor{currentstroke}%
\pgfsetstrokeopacity{0.500000}%
\pgfsetdash{}{0pt}%
\pgfpathmoveto{\pgfqpoint{2.502945in}{3.823611in}}%
\pgfpathlineto{\pgfqpoint{3.195386in}{3.823611in}}%
\pgfpathlineto{\pgfqpoint{3.195386in}{3.617729in}}%
\pgfpathlineto{\pgfqpoint{2.502945in}{3.617729in}}%
\pgfpathclose%
\pgfusepath{fill}%
\end{pgfscope}%
\begin{pgfscope}%
\pgfpathrectangle{\pgfqpoint{1.249956in}{0.148611in}}{\pgfqpoint{7.122250in}{3.850000in}}%
\pgfusepath{clip}%
\pgfsetbuttcap%
\pgfsetmiterjoin%
\definecolor{currentfill}{rgb}{0.966551,0.497424,0.295040}%
\pgfsetfillcolor{currentfill}%
\pgfsetfillopacity{0.500000}%
\pgfsetlinewidth{0.000000pt}%
\definecolor{currentstroke}{rgb}{0.000000,0.000000,0.000000}%
\pgfsetstrokecolor{currentstroke}%
\pgfsetstrokeopacity{0.500000}%
\pgfsetdash{}{0pt}%
\pgfpathmoveto{\pgfqpoint{3.294306in}{3.411846in}}%
\pgfpathlineto{\pgfqpoint{4.217561in}{3.411846in}}%
\pgfpathlineto{\pgfqpoint{4.217561in}{3.205964in}}%
\pgfpathlineto{\pgfqpoint{3.294306in}{3.205964in}}%
\pgfpathclose%
\pgfusepath{fill}%
\end{pgfscope}%
\begin{pgfscope}%
\pgfpathrectangle{\pgfqpoint{1.249956in}{0.148611in}}{\pgfqpoint{7.122250in}{3.850000in}}%
\pgfusepath{clip}%
\pgfsetbuttcap%
\pgfsetmiterjoin%
\definecolor{currentfill}{rgb}{0.966551,0.497424,0.295040}%
\pgfsetfillcolor{currentfill}%
\pgfsetlinewidth{0.000000pt}%
\definecolor{currentstroke}{rgb}{0.000000,0.000000,0.000000}%
\pgfsetstrokecolor{currentstroke}%
\pgfsetstrokeopacity{0.000000}%
\pgfsetdash{}{0pt}%
\pgfpathmoveto{\pgfqpoint{2.700785in}{3.000082in}}%
\pgfpathlineto{\pgfqpoint{3.689986in}{3.000082in}}%
\pgfpathlineto{\pgfqpoint{3.689986in}{2.794199in}}%
\pgfpathlineto{\pgfqpoint{2.700785in}{2.794199in}}%
\pgfpathclose%
\pgfusepath{fill}%
\end{pgfscope}%
\begin{pgfscope}%
\pgfpathrectangle{\pgfqpoint{1.249956in}{0.148611in}}{\pgfqpoint{7.122250in}{3.850000in}}%
\pgfusepath{clip}%
\pgfsetbuttcap%
\pgfsetmiterjoin%
\definecolor{currentfill}{rgb}{0.966551,0.497424,0.295040}%
\pgfsetfillcolor{currentfill}%
\pgfsetfillopacity{0.500000}%
\pgfsetlinewidth{0.000000pt}%
\definecolor{currentstroke}{rgb}{0.000000,0.000000,0.000000}%
\pgfsetstrokecolor{currentstroke}%
\pgfsetstrokeopacity{0.500000}%
\pgfsetdash{}{0pt}%
\pgfpathmoveto{\pgfqpoint{1.843477in}{2.588317in}}%
\pgfpathlineto{\pgfqpoint{2.568892in}{2.588317in}}%
\pgfpathlineto{\pgfqpoint{2.568892in}{2.382435in}}%
\pgfpathlineto{\pgfqpoint{1.843477in}{2.382435in}}%
\pgfpathclose%
\pgfusepath{fill}%
\end{pgfscope}%
\begin{pgfscope}%
\pgfpathrectangle{\pgfqpoint{1.249956in}{0.148611in}}{\pgfqpoint{7.122250in}{3.850000in}}%
\pgfusepath{clip}%
\pgfsetbuttcap%
\pgfsetmiterjoin%
\definecolor{currentfill}{rgb}{0.966551,0.497424,0.295040}%
\pgfsetfillcolor{currentfill}%
\pgfsetfillopacity{0.500000}%
\pgfsetlinewidth{0.000000pt}%
\definecolor{currentstroke}{rgb}{0.000000,0.000000,0.000000}%
\pgfsetstrokecolor{currentstroke}%
\pgfsetstrokeopacity{0.500000}%
\pgfsetdash{}{0pt}%
\pgfpathmoveto{\pgfqpoint{1.612664in}{2.176552in}}%
\pgfpathlineto{\pgfqpoint{1.876451in}{2.176552in}}%
\pgfpathlineto{\pgfqpoint{1.876451in}{1.970670in}}%
\pgfpathlineto{\pgfqpoint{1.612664in}{1.970670in}}%
\pgfpathclose%
\pgfusepath{fill}%
\end{pgfscope}%
\begin{pgfscope}%
\pgfpathrectangle{\pgfqpoint{1.249956in}{0.148611in}}{\pgfqpoint{7.122250in}{3.850000in}}%
\pgfusepath{clip}%
\pgfsetbuttcap%
\pgfsetmiterjoin%
\definecolor{currentfill}{rgb}{0.966551,0.497424,0.295040}%
\pgfsetfillcolor{currentfill}%
\pgfsetlinewidth{0.000000pt}%
\definecolor{currentstroke}{rgb}{0.000000,0.000000,0.000000}%
\pgfsetstrokecolor{currentstroke}%
\pgfsetstrokeopacity{0.000000}%
\pgfsetdash{}{0pt}%
\pgfpathmoveto{\pgfqpoint{3.096466in}{1.764788in}}%
\pgfpathlineto{\pgfqpoint{4.481348in}{1.764788in}}%
\pgfpathlineto{\pgfqpoint{4.481348in}{1.558905in}}%
\pgfpathlineto{\pgfqpoint{3.096466in}{1.558905in}}%
\pgfpathclose%
\pgfusepath{fill}%
\end{pgfscope}%
\begin{pgfscope}%
\pgfpathrectangle{\pgfqpoint{1.249956in}{0.148611in}}{\pgfqpoint{7.122250in}{3.850000in}}%
\pgfusepath{clip}%
\pgfsetbuttcap%
\pgfsetmiterjoin%
\definecolor{currentfill}{rgb}{0.966551,0.497424,0.295040}%
\pgfsetfillcolor{currentfill}%
\pgfsetfillopacity{0.500000}%
\pgfsetlinewidth{0.000000pt}%
\definecolor{currentstroke}{rgb}{0.000000,0.000000,0.000000}%
\pgfsetstrokecolor{currentstroke}%
\pgfsetstrokeopacity{0.500000}%
\pgfsetdash{}{0pt}%
\pgfpathmoveto{\pgfqpoint{1.678610in}{1.353023in}}%
\pgfpathlineto{\pgfqpoint{1.909424in}{1.353023in}}%
\pgfpathlineto{\pgfqpoint{1.909424in}{1.147141in}}%
\pgfpathlineto{\pgfqpoint{1.678610in}{1.147141in}}%
\pgfpathclose%
\pgfusepath{fill}%
\end{pgfscope}%
\begin{pgfscope}%
\pgfpathrectangle{\pgfqpoint{1.249956in}{0.148611in}}{\pgfqpoint{7.122250in}{3.850000in}}%
\pgfusepath{clip}%
\pgfsetbuttcap%
\pgfsetmiterjoin%
\definecolor{currentfill}{rgb}{0.966551,0.497424,0.295040}%
\pgfsetfillcolor{currentfill}%
\pgfsetlinewidth{0.000000pt}%
\definecolor{currentstroke}{rgb}{0.000000,0.000000,0.000000}%
\pgfsetstrokecolor{currentstroke}%
\pgfsetstrokeopacity{0.000000}%
\pgfsetdash{}{0pt}%
\pgfpathmoveto{\pgfqpoint{3.162412in}{0.941258in}}%
\pgfpathlineto{\pgfqpoint{4.316481in}{0.941258in}}%
\pgfpathlineto{\pgfqpoint{4.316481in}{0.735376in}}%
\pgfpathlineto{\pgfqpoint{3.162412in}{0.735376in}}%
\pgfpathclose%
\pgfusepath{fill}%
\end{pgfscope}%
\begin{pgfscope}%
\pgfpathrectangle{\pgfqpoint{1.249956in}{0.148611in}}{\pgfqpoint{7.122250in}{3.850000in}}%
\pgfusepath{clip}%
\pgfsetbuttcap%
\pgfsetmiterjoin%
\definecolor{currentfill}{rgb}{0.966551,0.497424,0.295040}%
\pgfsetfillcolor{currentfill}%
\pgfsetlinewidth{0.000000pt}%
\definecolor{currentstroke}{rgb}{0.000000,0.000000,0.000000}%
\pgfsetstrokecolor{currentstroke}%
\pgfsetstrokeopacity{0.000000}%
\pgfsetdash{}{0pt}%
\pgfpathmoveto{\pgfqpoint{2.700785in}{0.529493in}}%
\pgfpathlineto{\pgfqpoint{3.393226in}{0.529493in}}%
\pgfpathlineto{\pgfqpoint{3.393226in}{0.323611in}}%
\pgfpathlineto{\pgfqpoint{2.700785in}{0.323611in}}%
\pgfpathclose%
\pgfusepath{fill}%
\end{pgfscope}%
\begin{pgfscope}%
\pgfpathrectangle{\pgfqpoint{1.249956in}{0.148611in}}{\pgfqpoint{7.122250in}{3.850000in}}%
\pgfusepath{clip}%
\pgfsetbuttcap%
\pgfsetmiterjoin%
\definecolor{currentfill}{rgb}{0.992388,0.693887,0.390081}%
\pgfsetfillcolor{currentfill}%
\pgfsetfillopacity{0.500000}%
\pgfsetlinewidth{0.000000pt}%
\definecolor{currentstroke}{rgb}{0.000000,0.000000,0.000000}%
\pgfsetstrokecolor{currentstroke}%
\pgfsetstrokeopacity{0.500000}%
\pgfsetdash{}{0pt}%
\pgfpathmoveto{\pgfqpoint{3.195386in}{3.823611in}}%
\pgfpathlineto{\pgfqpoint{4.052694in}{3.823611in}}%
\pgfpathlineto{\pgfqpoint{4.052694in}{3.617729in}}%
\pgfpathlineto{\pgfqpoint{3.195386in}{3.617729in}}%
\pgfpathclose%
\pgfusepath{fill}%
\end{pgfscope}%
\begin{pgfscope}%
\pgfpathrectangle{\pgfqpoint{1.249956in}{0.148611in}}{\pgfqpoint{7.122250in}{3.850000in}}%
\pgfusepath{clip}%
\pgfsetbuttcap%
\pgfsetmiterjoin%
\definecolor{currentfill}{rgb}{0.992388,0.693887,0.390081}%
\pgfsetfillcolor{currentfill}%
\pgfsetfillopacity{0.500000}%
\pgfsetlinewidth{0.000000pt}%
\definecolor{currentstroke}{rgb}{0.000000,0.000000,0.000000}%
\pgfsetstrokecolor{currentstroke}%
\pgfsetstrokeopacity{0.500000}%
\pgfsetdash{}{0pt}%
\pgfpathmoveto{\pgfqpoint{4.217561in}{3.411846in}}%
\pgfpathlineto{\pgfqpoint{5.239735in}{3.411846in}}%
\pgfpathlineto{\pgfqpoint{5.239735in}{3.205964in}}%
\pgfpathlineto{\pgfqpoint{4.217561in}{3.205964in}}%
\pgfpathclose%
\pgfusepath{fill}%
\end{pgfscope}%
\begin{pgfscope}%
\pgfpathrectangle{\pgfqpoint{1.249956in}{0.148611in}}{\pgfqpoint{7.122250in}{3.850000in}}%
\pgfusepath{clip}%
\pgfsetbuttcap%
\pgfsetmiterjoin%
\definecolor{currentfill}{rgb}{0.992388,0.693887,0.390081}%
\pgfsetfillcolor{currentfill}%
\pgfsetlinewidth{0.000000pt}%
\definecolor{currentstroke}{rgb}{0.000000,0.000000,0.000000}%
\pgfsetstrokecolor{currentstroke}%
\pgfsetstrokeopacity{0.000000}%
\pgfsetdash{}{0pt}%
\pgfpathmoveto{\pgfqpoint{3.689986in}{3.000082in}}%
\pgfpathlineto{\pgfqpoint{4.712161in}{3.000082in}}%
\pgfpathlineto{\pgfqpoint{4.712161in}{2.794199in}}%
\pgfpathlineto{\pgfqpoint{3.689986in}{2.794199in}}%
\pgfpathclose%
\pgfusepath{fill}%
\end{pgfscope}%
\begin{pgfscope}%
\pgfpathrectangle{\pgfqpoint{1.249956in}{0.148611in}}{\pgfqpoint{7.122250in}{3.850000in}}%
\pgfusepath{clip}%
\pgfsetbuttcap%
\pgfsetmiterjoin%
\definecolor{currentfill}{rgb}{0.992388,0.693887,0.390081}%
\pgfsetfillcolor{currentfill}%
\pgfsetfillopacity{0.500000}%
\pgfsetlinewidth{0.000000pt}%
\definecolor{currentstroke}{rgb}{0.000000,0.000000,0.000000}%
\pgfsetstrokecolor{currentstroke}%
\pgfsetstrokeopacity{0.500000}%
\pgfsetdash{}{0pt}%
\pgfpathmoveto{\pgfqpoint{2.568892in}{2.588317in}}%
\pgfpathlineto{\pgfqpoint{3.492146in}{2.588317in}}%
\pgfpathlineto{\pgfqpoint{3.492146in}{2.382435in}}%
\pgfpathlineto{\pgfqpoint{2.568892in}{2.382435in}}%
\pgfpathclose%
\pgfusepath{fill}%
\end{pgfscope}%
\begin{pgfscope}%
\pgfpathrectangle{\pgfqpoint{1.249956in}{0.148611in}}{\pgfqpoint{7.122250in}{3.850000in}}%
\pgfusepath{clip}%
\pgfsetbuttcap%
\pgfsetmiterjoin%
\definecolor{currentfill}{rgb}{0.992388,0.693887,0.390081}%
\pgfsetfillcolor{currentfill}%
\pgfsetfillopacity{0.500000}%
\pgfsetlinewidth{0.000000pt}%
\definecolor{currentstroke}{rgb}{0.000000,0.000000,0.000000}%
\pgfsetstrokecolor{currentstroke}%
\pgfsetstrokeopacity{0.500000}%
\pgfsetdash{}{0pt}%
\pgfpathmoveto{\pgfqpoint{1.876451in}{2.176552in}}%
\pgfpathlineto{\pgfqpoint{2.140238in}{2.176552in}}%
\pgfpathlineto{\pgfqpoint{2.140238in}{1.970670in}}%
\pgfpathlineto{\pgfqpoint{1.876451in}{1.970670in}}%
\pgfpathclose%
\pgfusepath{fill}%
\end{pgfscope}%
\begin{pgfscope}%
\pgfpathrectangle{\pgfqpoint{1.249956in}{0.148611in}}{\pgfqpoint{7.122250in}{3.850000in}}%
\pgfusepath{clip}%
\pgfsetbuttcap%
\pgfsetmiterjoin%
\definecolor{currentfill}{rgb}{0.992388,0.693887,0.390081}%
\pgfsetfillcolor{currentfill}%
\pgfsetlinewidth{0.000000pt}%
\definecolor{currentstroke}{rgb}{0.000000,0.000000,0.000000}%
\pgfsetstrokecolor{currentstroke}%
\pgfsetstrokeopacity{0.000000}%
\pgfsetdash{}{0pt}%
\pgfpathmoveto{\pgfqpoint{4.481348in}{1.764788in}}%
\pgfpathlineto{\pgfqpoint{5.404602in}{1.764788in}}%
\pgfpathlineto{\pgfqpoint{5.404602in}{1.558905in}}%
\pgfpathlineto{\pgfqpoint{4.481348in}{1.558905in}}%
\pgfpathclose%
\pgfusepath{fill}%
\end{pgfscope}%
\begin{pgfscope}%
\pgfpathrectangle{\pgfqpoint{1.249956in}{0.148611in}}{\pgfqpoint{7.122250in}{3.850000in}}%
\pgfusepath{clip}%
\pgfsetbuttcap%
\pgfsetmiterjoin%
\definecolor{currentfill}{rgb}{0.992388,0.693887,0.390081}%
\pgfsetfillcolor{currentfill}%
\pgfsetfillopacity{0.500000}%
\pgfsetlinewidth{0.000000pt}%
\definecolor{currentstroke}{rgb}{0.000000,0.000000,0.000000}%
\pgfsetstrokecolor{currentstroke}%
\pgfsetstrokeopacity{0.500000}%
\pgfsetdash{}{0pt}%
\pgfpathmoveto{\pgfqpoint{1.909424in}{1.353023in}}%
\pgfpathlineto{\pgfqpoint{2.140238in}{1.353023in}}%
\pgfpathlineto{\pgfqpoint{2.140238in}{1.147141in}}%
\pgfpathlineto{\pgfqpoint{1.909424in}{1.147141in}}%
\pgfpathclose%
\pgfusepath{fill}%
\end{pgfscope}%
\begin{pgfscope}%
\pgfpathrectangle{\pgfqpoint{1.249956in}{0.148611in}}{\pgfqpoint{7.122250in}{3.850000in}}%
\pgfusepath{clip}%
\pgfsetbuttcap%
\pgfsetmiterjoin%
\definecolor{currentfill}{rgb}{0.992388,0.693887,0.390081}%
\pgfsetfillcolor{currentfill}%
\pgfsetlinewidth{0.000000pt}%
\definecolor{currentstroke}{rgb}{0.000000,0.000000,0.000000}%
\pgfsetstrokecolor{currentstroke}%
\pgfsetstrokeopacity{0.000000}%
\pgfsetdash{}{0pt}%
\pgfpathmoveto{\pgfqpoint{4.316481in}{0.941258in}}%
\pgfpathlineto{\pgfqpoint{5.932176in}{0.941258in}}%
\pgfpathlineto{\pgfqpoint{5.932176in}{0.735376in}}%
\pgfpathlineto{\pgfqpoint{4.316481in}{0.735376in}}%
\pgfpathclose%
\pgfusepath{fill}%
\end{pgfscope}%
\begin{pgfscope}%
\pgfpathrectangle{\pgfqpoint{1.249956in}{0.148611in}}{\pgfqpoint{7.122250in}{3.850000in}}%
\pgfusepath{clip}%
\pgfsetbuttcap%
\pgfsetmiterjoin%
\definecolor{currentfill}{rgb}{0.992388,0.693887,0.390081}%
\pgfsetfillcolor{currentfill}%
\pgfsetlinewidth{0.000000pt}%
\definecolor{currentstroke}{rgb}{0.000000,0.000000,0.000000}%
\pgfsetstrokecolor{currentstroke}%
\pgfsetstrokeopacity{0.000000}%
\pgfsetdash{}{0pt}%
\pgfpathmoveto{\pgfqpoint{3.393226in}{0.529493in}}%
\pgfpathlineto{\pgfqpoint{4.052694in}{0.529493in}}%
\pgfpathlineto{\pgfqpoint{4.052694in}{0.323611in}}%
\pgfpathlineto{\pgfqpoint{3.393226in}{0.323611in}}%
\pgfpathclose%
\pgfusepath{fill}%
\end{pgfscope}%
\begin{pgfscope}%
\pgfpathrectangle{\pgfqpoint{1.249956in}{0.148611in}}{\pgfqpoint{7.122250in}{3.850000in}}%
\pgfusepath{clip}%
\pgfsetbuttcap%
\pgfsetmiterjoin%
\definecolor{currentfill}{rgb}{0.995463,0.847674,0.519262}%
\pgfsetfillcolor{currentfill}%
\pgfsetfillopacity{0.500000}%
\pgfsetlinewidth{0.000000pt}%
\definecolor{currentstroke}{rgb}{0.000000,0.000000,0.000000}%
\pgfsetstrokecolor{currentstroke}%
\pgfsetstrokeopacity{0.500000}%
\pgfsetdash{}{0pt}%
\pgfpathmoveto{\pgfqpoint{4.052694in}{3.823611in}}%
\pgfpathlineto{\pgfqpoint{4.942975in}{3.823611in}}%
\pgfpathlineto{\pgfqpoint{4.942975in}{3.617729in}}%
\pgfpathlineto{\pgfqpoint{4.052694in}{3.617729in}}%
\pgfpathclose%
\pgfusepath{fill}%
\end{pgfscope}%
\begin{pgfscope}%
\pgfpathrectangle{\pgfqpoint{1.249956in}{0.148611in}}{\pgfqpoint{7.122250in}{3.850000in}}%
\pgfusepath{clip}%
\pgfsetbuttcap%
\pgfsetmiterjoin%
\definecolor{currentfill}{rgb}{0.995463,0.847674,0.519262}%
\pgfsetfillcolor{currentfill}%
\pgfsetfillopacity{0.500000}%
\pgfsetlinewidth{0.000000pt}%
\definecolor{currentstroke}{rgb}{0.000000,0.000000,0.000000}%
\pgfsetstrokecolor{currentstroke}%
\pgfsetstrokeopacity{0.500000}%
\pgfsetdash{}{0pt}%
\pgfpathmoveto{\pgfqpoint{5.239735in}{3.411846in}}%
\pgfpathlineto{\pgfqpoint{6.360830in}{3.411846in}}%
\pgfpathlineto{\pgfqpoint{6.360830in}{3.205964in}}%
\pgfpathlineto{\pgfqpoint{5.239735in}{3.205964in}}%
\pgfpathclose%
\pgfusepath{fill}%
\end{pgfscope}%
\begin{pgfscope}%
\pgfpathrectangle{\pgfqpoint{1.249956in}{0.148611in}}{\pgfqpoint{7.122250in}{3.850000in}}%
\pgfusepath{clip}%
\pgfsetbuttcap%
\pgfsetmiterjoin%
\definecolor{currentfill}{rgb}{0.995463,0.847674,0.519262}%
\pgfsetfillcolor{currentfill}%
\pgfsetlinewidth{0.000000pt}%
\definecolor{currentstroke}{rgb}{0.000000,0.000000,0.000000}%
\pgfsetstrokecolor{currentstroke}%
\pgfsetstrokeopacity{0.000000}%
\pgfsetdash{}{0pt}%
\pgfpathmoveto{\pgfqpoint{4.712161in}{3.000082in}}%
\pgfpathlineto{\pgfqpoint{5.371629in}{3.000082in}}%
\pgfpathlineto{\pgfqpoint{5.371629in}{2.794199in}}%
\pgfpathlineto{\pgfqpoint{4.712161in}{2.794199in}}%
\pgfpathclose%
\pgfusepath{fill}%
\end{pgfscope}%
\begin{pgfscope}%
\pgfpathrectangle{\pgfqpoint{1.249956in}{0.148611in}}{\pgfqpoint{7.122250in}{3.850000in}}%
\pgfusepath{clip}%
\pgfsetbuttcap%
\pgfsetmiterjoin%
\definecolor{currentfill}{rgb}{0.995463,0.847674,0.519262}%
\pgfsetfillcolor{currentfill}%
\pgfsetfillopacity{0.500000}%
\pgfsetlinewidth{0.000000pt}%
\definecolor{currentstroke}{rgb}{0.000000,0.000000,0.000000}%
\pgfsetstrokecolor{currentstroke}%
\pgfsetstrokeopacity{0.500000}%
\pgfsetdash{}{0pt}%
\pgfpathmoveto{\pgfqpoint{3.492146in}{2.588317in}}%
\pgfpathlineto{\pgfqpoint{4.514321in}{2.588317in}}%
\pgfpathlineto{\pgfqpoint{4.514321in}{2.382435in}}%
\pgfpathlineto{\pgfqpoint{3.492146in}{2.382435in}}%
\pgfpathclose%
\pgfusepath{fill}%
\end{pgfscope}%
\begin{pgfscope}%
\pgfpathrectangle{\pgfqpoint{1.249956in}{0.148611in}}{\pgfqpoint{7.122250in}{3.850000in}}%
\pgfusepath{clip}%
\pgfsetbuttcap%
\pgfsetmiterjoin%
\definecolor{currentfill}{rgb}{0.995463,0.847674,0.519262}%
\pgfsetfillcolor{currentfill}%
\pgfsetfillopacity{0.500000}%
\pgfsetlinewidth{0.000000pt}%
\definecolor{currentstroke}{rgb}{0.000000,0.000000,0.000000}%
\pgfsetstrokecolor{currentstroke}%
\pgfsetstrokeopacity{0.500000}%
\pgfsetdash{}{0pt}%
\pgfpathmoveto{\pgfqpoint{2.140238in}{2.176552in}}%
\pgfpathlineto{\pgfqpoint{2.898625in}{2.176552in}}%
\pgfpathlineto{\pgfqpoint{2.898625in}{1.970670in}}%
\pgfpathlineto{\pgfqpoint{2.140238in}{1.970670in}}%
\pgfpathclose%
\pgfusepath{fill}%
\end{pgfscope}%
\begin{pgfscope}%
\pgfpathrectangle{\pgfqpoint{1.249956in}{0.148611in}}{\pgfqpoint{7.122250in}{3.850000in}}%
\pgfusepath{clip}%
\pgfsetbuttcap%
\pgfsetmiterjoin%
\definecolor{currentfill}{rgb}{0.995463,0.847674,0.519262}%
\pgfsetfillcolor{currentfill}%
\pgfsetlinewidth{0.000000pt}%
\definecolor{currentstroke}{rgb}{0.000000,0.000000,0.000000}%
\pgfsetstrokecolor{currentstroke}%
\pgfsetstrokeopacity{0.000000}%
\pgfsetdash{}{0pt}%
\pgfpathmoveto{\pgfqpoint{5.404602in}{1.764788in}}%
\pgfpathlineto{\pgfqpoint{6.492724in}{1.764788in}}%
\pgfpathlineto{\pgfqpoint{6.492724in}{1.558905in}}%
\pgfpathlineto{\pgfqpoint{5.404602in}{1.558905in}}%
\pgfpathclose%
\pgfusepath{fill}%
\end{pgfscope}%
\begin{pgfscope}%
\pgfpathrectangle{\pgfqpoint{1.249956in}{0.148611in}}{\pgfqpoint{7.122250in}{3.850000in}}%
\pgfusepath{clip}%
\pgfsetbuttcap%
\pgfsetmiterjoin%
\definecolor{currentfill}{rgb}{0.995463,0.847674,0.519262}%
\pgfsetfillcolor{currentfill}%
\pgfsetfillopacity{0.500000}%
\pgfsetlinewidth{0.000000pt}%
\definecolor{currentstroke}{rgb}{0.000000,0.000000,0.000000}%
\pgfsetstrokecolor{currentstroke}%
\pgfsetstrokeopacity{0.500000}%
\pgfsetdash{}{0pt}%
\pgfpathmoveto{\pgfqpoint{2.140238in}{1.353023in}}%
\pgfpathlineto{\pgfqpoint{2.568892in}{1.353023in}}%
\pgfpathlineto{\pgfqpoint{2.568892in}{1.147141in}}%
\pgfpathlineto{\pgfqpoint{2.140238in}{1.147141in}}%
\pgfpathclose%
\pgfusepath{fill}%
\end{pgfscope}%
\begin{pgfscope}%
\pgfpathrectangle{\pgfqpoint{1.249956in}{0.148611in}}{\pgfqpoint{7.122250in}{3.850000in}}%
\pgfusepath{clip}%
\pgfsetbuttcap%
\pgfsetmiterjoin%
\definecolor{currentfill}{rgb}{0.995463,0.847674,0.519262}%
\pgfsetfillcolor{currentfill}%
\pgfsetlinewidth{0.000000pt}%
\definecolor{currentstroke}{rgb}{0.000000,0.000000,0.000000}%
\pgfsetstrokecolor{currentstroke}%
\pgfsetstrokeopacity{0.000000}%
\pgfsetdash{}{0pt}%
\pgfpathmoveto{\pgfqpoint{5.932176in}{0.941258in}}%
\pgfpathlineto{\pgfqpoint{6.921378in}{0.941258in}}%
\pgfpathlineto{\pgfqpoint{6.921378in}{0.735376in}}%
\pgfpathlineto{\pgfqpoint{5.932176in}{0.735376in}}%
\pgfpathclose%
\pgfusepath{fill}%
\end{pgfscope}%
\begin{pgfscope}%
\pgfpathrectangle{\pgfqpoint{1.249956in}{0.148611in}}{\pgfqpoint{7.122250in}{3.850000in}}%
\pgfusepath{clip}%
\pgfsetbuttcap%
\pgfsetmiterjoin%
\definecolor{currentfill}{rgb}{0.995463,0.847674,0.519262}%
\pgfsetfillcolor{currentfill}%
\pgfsetlinewidth{0.000000pt}%
\definecolor{currentstroke}{rgb}{0.000000,0.000000,0.000000}%
\pgfsetstrokecolor{currentstroke}%
\pgfsetstrokeopacity{0.000000}%
\pgfsetdash{}{0pt}%
\pgfpathmoveto{\pgfqpoint{4.052694in}{0.529493in}}%
\pgfpathlineto{\pgfqpoint{5.008922in}{0.529493in}}%
\pgfpathlineto{\pgfqpoint{5.008922in}{0.323611in}}%
\pgfpathlineto{\pgfqpoint{4.052694in}{0.323611in}}%
\pgfpathclose%
\pgfusepath{fill}%
\end{pgfscope}%
\begin{pgfscope}%
\pgfpathrectangle{\pgfqpoint{1.249956in}{0.148611in}}{\pgfqpoint{7.122250in}{3.850000in}}%
\pgfusepath{clip}%
\pgfsetbuttcap%
\pgfsetmiterjoin%
\definecolor{currentfill}{rgb}{0.998539,0.954710,0.673049}%
\pgfsetfillcolor{currentfill}%
\pgfsetfillopacity{0.500000}%
\pgfsetlinewidth{0.000000pt}%
\definecolor{currentstroke}{rgb}{0.000000,0.000000,0.000000}%
\pgfsetstrokecolor{currentstroke}%
\pgfsetstrokeopacity{0.500000}%
\pgfsetdash{}{0pt}%
\pgfpathmoveto{\pgfqpoint{4.942975in}{3.823611in}}%
\pgfpathlineto{\pgfqpoint{5.371629in}{3.823611in}}%
\pgfpathlineto{\pgfqpoint{5.371629in}{3.617729in}}%
\pgfpathlineto{\pgfqpoint{4.942975in}{3.617729in}}%
\pgfpathclose%
\pgfusepath{fill}%
\end{pgfscope}%
\begin{pgfscope}%
\pgfpathrectangle{\pgfqpoint{1.249956in}{0.148611in}}{\pgfqpoint{7.122250in}{3.850000in}}%
\pgfusepath{clip}%
\pgfsetbuttcap%
\pgfsetmiterjoin%
\definecolor{currentfill}{rgb}{0.998539,0.954710,0.673049}%
\pgfsetfillcolor{currentfill}%
\pgfsetfillopacity{0.500000}%
\pgfsetlinewidth{0.000000pt}%
\definecolor{currentstroke}{rgb}{0.000000,0.000000,0.000000}%
\pgfsetstrokecolor{currentstroke}%
\pgfsetstrokeopacity{0.500000}%
\pgfsetdash{}{0pt}%
\pgfpathmoveto{\pgfqpoint{6.360830in}{3.411846in}}%
\pgfpathlineto{\pgfqpoint{6.855431in}{3.411846in}}%
\pgfpathlineto{\pgfqpoint{6.855431in}{3.205964in}}%
\pgfpathlineto{\pgfqpoint{6.360830in}{3.205964in}}%
\pgfpathclose%
\pgfusepath{fill}%
\end{pgfscope}%
\begin{pgfscope}%
\pgfpathrectangle{\pgfqpoint{1.249956in}{0.148611in}}{\pgfqpoint{7.122250in}{3.850000in}}%
\pgfusepath{clip}%
\pgfsetbuttcap%
\pgfsetmiterjoin%
\definecolor{currentfill}{rgb}{0.998539,0.954710,0.673049}%
\pgfsetfillcolor{currentfill}%
\pgfsetlinewidth{0.000000pt}%
\definecolor{currentstroke}{rgb}{0.000000,0.000000,0.000000}%
\pgfsetstrokecolor{currentstroke}%
\pgfsetstrokeopacity{0.000000}%
\pgfsetdash{}{0pt}%
\pgfpathmoveto{\pgfqpoint{5.371629in}{3.000082in}}%
\pgfpathlineto{\pgfqpoint{6.195963in}{3.000082in}}%
\pgfpathlineto{\pgfqpoint{6.195963in}{2.794199in}}%
\pgfpathlineto{\pgfqpoint{5.371629in}{2.794199in}}%
\pgfpathclose%
\pgfusepath{fill}%
\end{pgfscope}%
\begin{pgfscope}%
\pgfpathrectangle{\pgfqpoint{1.249956in}{0.148611in}}{\pgfqpoint{7.122250in}{3.850000in}}%
\pgfusepath{clip}%
\pgfsetbuttcap%
\pgfsetmiterjoin%
\definecolor{currentfill}{rgb}{0.998539,0.954710,0.673049}%
\pgfsetfillcolor{currentfill}%
\pgfsetfillopacity{0.500000}%
\pgfsetlinewidth{0.000000pt}%
\definecolor{currentstroke}{rgb}{0.000000,0.000000,0.000000}%
\pgfsetstrokecolor{currentstroke}%
\pgfsetstrokeopacity{0.500000}%
\pgfsetdash{}{0pt}%
\pgfpathmoveto{\pgfqpoint{4.514321in}{2.588317in}}%
\pgfpathlineto{\pgfqpoint{5.239735in}{2.588317in}}%
\pgfpathlineto{\pgfqpoint{5.239735in}{2.382435in}}%
\pgfpathlineto{\pgfqpoint{4.514321in}{2.382435in}}%
\pgfpathclose%
\pgfusepath{fill}%
\end{pgfscope}%
\begin{pgfscope}%
\pgfpathrectangle{\pgfqpoint{1.249956in}{0.148611in}}{\pgfqpoint{7.122250in}{3.850000in}}%
\pgfusepath{clip}%
\pgfsetbuttcap%
\pgfsetmiterjoin%
\definecolor{currentfill}{rgb}{0.998539,0.954710,0.673049}%
\pgfsetfillcolor{currentfill}%
\pgfsetfillopacity{0.500000}%
\pgfsetlinewidth{0.000000pt}%
\definecolor{currentstroke}{rgb}{0.000000,0.000000,0.000000}%
\pgfsetstrokecolor{currentstroke}%
\pgfsetstrokeopacity{0.500000}%
\pgfsetdash{}{0pt}%
\pgfpathmoveto{\pgfqpoint{2.898625in}{2.176552in}}%
\pgfpathlineto{\pgfqpoint{3.426199in}{2.176552in}}%
\pgfpathlineto{\pgfqpoint{3.426199in}{1.970670in}}%
\pgfpathlineto{\pgfqpoint{2.898625in}{1.970670in}}%
\pgfpathclose%
\pgfusepath{fill}%
\end{pgfscope}%
\begin{pgfscope}%
\pgfpathrectangle{\pgfqpoint{1.249956in}{0.148611in}}{\pgfqpoint{7.122250in}{3.850000in}}%
\pgfusepath{clip}%
\pgfsetbuttcap%
\pgfsetmiterjoin%
\definecolor{currentfill}{rgb}{0.998539,0.954710,0.673049}%
\pgfsetfillcolor{currentfill}%
\pgfsetlinewidth{0.000000pt}%
\definecolor{currentstroke}{rgb}{0.000000,0.000000,0.000000}%
\pgfsetstrokecolor{currentstroke}%
\pgfsetstrokeopacity{0.000000}%
\pgfsetdash{}{0pt}%
\pgfpathmoveto{\pgfqpoint{6.492724in}{1.764788in}}%
\pgfpathlineto{\pgfqpoint{7.119218in}{1.764788in}}%
\pgfpathlineto{\pgfqpoint{7.119218in}{1.558905in}}%
\pgfpathlineto{\pgfqpoint{6.492724in}{1.558905in}}%
\pgfpathclose%
\pgfusepath{fill}%
\end{pgfscope}%
\begin{pgfscope}%
\pgfpathrectangle{\pgfqpoint{1.249956in}{0.148611in}}{\pgfqpoint{7.122250in}{3.850000in}}%
\pgfusepath{clip}%
\pgfsetbuttcap%
\pgfsetmiterjoin%
\definecolor{currentfill}{rgb}{0.998539,0.954710,0.673049}%
\pgfsetfillcolor{currentfill}%
\pgfsetfillopacity{0.500000}%
\pgfsetlinewidth{0.000000pt}%
\definecolor{currentstroke}{rgb}{0.000000,0.000000,0.000000}%
\pgfsetstrokecolor{currentstroke}%
\pgfsetstrokeopacity{0.500000}%
\pgfsetdash{}{0pt}%
\pgfpathmoveto{\pgfqpoint{2.568892in}{1.353023in}}%
\pgfpathlineto{\pgfqpoint{3.162412in}{1.353023in}}%
\pgfpathlineto{\pgfqpoint{3.162412in}{1.147141in}}%
\pgfpathlineto{\pgfqpoint{2.568892in}{1.147141in}}%
\pgfpathclose%
\pgfusepath{fill}%
\end{pgfscope}%
\begin{pgfscope}%
\pgfpathrectangle{\pgfqpoint{1.249956in}{0.148611in}}{\pgfqpoint{7.122250in}{3.850000in}}%
\pgfusepath{clip}%
\pgfsetbuttcap%
\pgfsetmiterjoin%
\definecolor{currentfill}{rgb}{0.998539,0.954710,0.673049}%
\pgfsetfillcolor{currentfill}%
\pgfsetlinewidth{0.000000pt}%
\definecolor{currentstroke}{rgb}{0.000000,0.000000,0.000000}%
\pgfsetstrokecolor{currentstroke}%
\pgfsetstrokeopacity{0.000000}%
\pgfsetdash{}{0pt}%
\pgfpathmoveto{\pgfqpoint{6.921378in}{0.941258in}}%
\pgfpathlineto{\pgfqpoint{7.350032in}{0.941258in}}%
\pgfpathlineto{\pgfqpoint{7.350032in}{0.735376in}}%
\pgfpathlineto{\pgfqpoint{6.921378in}{0.735376in}}%
\pgfpathclose%
\pgfusepath{fill}%
\end{pgfscope}%
\begin{pgfscope}%
\pgfpathrectangle{\pgfqpoint{1.249956in}{0.148611in}}{\pgfqpoint{7.122250in}{3.850000in}}%
\pgfusepath{clip}%
\pgfsetbuttcap%
\pgfsetmiterjoin%
\definecolor{currentfill}{rgb}{0.998539,0.954710,0.673049}%
\pgfsetfillcolor{currentfill}%
\pgfsetlinewidth{0.000000pt}%
\definecolor{currentstroke}{rgb}{0.000000,0.000000,0.000000}%
\pgfsetstrokecolor{currentstroke}%
\pgfsetstrokeopacity{0.000000}%
\pgfsetdash{}{0pt}%
\pgfpathmoveto{\pgfqpoint{5.008922in}{0.529493in}}%
\pgfpathlineto{\pgfqpoint{5.536496in}{0.529493in}}%
\pgfpathlineto{\pgfqpoint{5.536496in}{0.323611in}}%
\pgfpathlineto{\pgfqpoint{5.008922in}{0.323611in}}%
\pgfpathclose%
\pgfusepath{fill}%
\end{pgfscope}%
\begin{pgfscope}%
\pgfpathrectangle{\pgfqpoint{1.249956in}{0.148611in}}{\pgfqpoint{7.122250in}{3.850000in}}%
\pgfusepath{clip}%
\pgfsetbuttcap%
\pgfsetmiterjoin%
\definecolor{currentfill}{rgb}{0.944483,0.976624,0.673049}%
\pgfsetfillcolor{currentfill}%
\pgfsetfillopacity{0.500000}%
\pgfsetlinewidth{0.000000pt}%
\definecolor{currentstroke}{rgb}{0.000000,0.000000,0.000000}%
\pgfsetstrokecolor{currentstroke}%
\pgfsetstrokeopacity{0.500000}%
\pgfsetdash{}{0pt}%
\pgfpathmoveto{\pgfqpoint{5.371629in}{3.823611in}}%
\pgfpathlineto{\pgfqpoint{6.393804in}{3.823611in}}%
\pgfpathlineto{\pgfqpoint{6.393804in}{3.617729in}}%
\pgfpathlineto{\pgfqpoint{5.371629in}{3.617729in}}%
\pgfpathclose%
\pgfusepath{fill}%
\end{pgfscope}%
\begin{pgfscope}%
\pgfpathrectangle{\pgfqpoint{1.249956in}{0.148611in}}{\pgfqpoint{7.122250in}{3.850000in}}%
\pgfusepath{clip}%
\pgfsetbuttcap%
\pgfsetmiterjoin%
\definecolor{currentfill}{rgb}{0.944483,0.976624,0.673049}%
\pgfsetfillcolor{currentfill}%
\pgfsetfillopacity{0.500000}%
\pgfsetlinewidth{0.000000pt}%
\definecolor{currentstroke}{rgb}{0.000000,0.000000,0.000000}%
\pgfsetstrokecolor{currentstroke}%
\pgfsetstrokeopacity{0.500000}%
\pgfsetdash{}{0pt}%
\pgfpathmoveto{\pgfqpoint{6.855431in}{3.411846in}}%
\pgfpathlineto{\pgfqpoint{7.251111in}{3.411846in}}%
\pgfpathlineto{\pgfqpoint{7.251111in}{3.205964in}}%
\pgfpathlineto{\pgfqpoint{6.855431in}{3.205964in}}%
\pgfpathclose%
\pgfusepath{fill}%
\end{pgfscope}%
\begin{pgfscope}%
\pgfpathrectangle{\pgfqpoint{1.249956in}{0.148611in}}{\pgfqpoint{7.122250in}{3.850000in}}%
\pgfusepath{clip}%
\pgfsetbuttcap%
\pgfsetmiterjoin%
\definecolor{currentfill}{rgb}{0.944483,0.976624,0.673049}%
\pgfsetfillcolor{currentfill}%
\pgfsetlinewidth{0.000000pt}%
\definecolor{currentstroke}{rgb}{0.000000,0.000000,0.000000}%
\pgfsetstrokecolor{currentstroke}%
\pgfsetstrokeopacity{0.000000}%
\pgfsetdash{}{0pt}%
\pgfpathmoveto{\pgfqpoint{6.195963in}{3.000082in}}%
\pgfpathlineto{\pgfqpoint{6.690564in}{3.000082in}}%
\pgfpathlineto{\pgfqpoint{6.690564in}{2.794199in}}%
\pgfpathlineto{\pgfqpoint{6.195963in}{2.794199in}}%
\pgfpathclose%
\pgfusepath{fill}%
\end{pgfscope}%
\begin{pgfscope}%
\pgfpathrectangle{\pgfqpoint{1.249956in}{0.148611in}}{\pgfqpoint{7.122250in}{3.850000in}}%
\pgfusepath{clip}%
\pgfsetbuttcap%
\pgfsetmiterjoin%
\definecolor{currentfill}{rgb}{0.944483,0.976624,0.673049}%
\pgfsetfillcolor{currentfill}%
\pgfsetfillopacity{0.500000}%
\pgfsetlinewidth{0.000000pt}%
\definecolor{currentstroke}{rgb}{0.000000,0.000000,0.000000}%
\pgfsetstrokecolor{currentstroke}%
\pgfsetstrokeopacity{0.500000}%
\pgfsetdash{}{0pt}%
\pgfpathmoveto{\pgfqpoint{5.239735in}{2.588317in}}%
\pgfpathlineto{\pgfqpoint{5.701363in}{2.588317in}}%
\pgfpathlineto{\pgfqpoint{5.701363in}{2.382435in}}%
\pgfpathlineto{\pgfqpoint{5.239735in}{2.382435in}}%
\pgfpathclose%
\pgfusepath{fill}%
\end{pgfscope}%
\begin{pgfscope}%
\pgfpathrectangle{\pgfqpoint{1.249956in}{0.148611in}}{\pgfqpoint{7.122250in}{3.850000in}}%
\pgfusepath{clip}%
\pgfsetbuttcap%
\pgfsetmiterjoin%
\definecolor{currentfill}{rgb}{0.944483,0.976624,0.673049}%
\pgfsetfillcolor{currentfill}%
\pgfsetfillopacity{0.500000}%
\pgfsetlinewidth{0.000000pt}%
\definecolor{currentstroke}{rgb}{0.000000,0.000000,0.000000}%
\pgfsetstrokecolor{currentstroke}%
\pgfsetstrokeopacity{0.500000}%
\pgfsetdash{}{0pt}%
\pgfpathmoveto{\pgfqpoint{3.426199in}{2.176552in}}%
\pgfpathlineto{\pgfqpoint{4.052694in}{2.176552in}}%
\pgfpathlineto{\pgfqpoint{4.052694in}{1.970670in}}%
\pgfpathlineto{\pgfqpoint{3.426199in}{1.970670in}}%
\pgfpathclose%
\pgfusepath{fill}%
\end{pgfscope}%
\begin{pgfscope}%
\pgfpathrectangle{\pgfqpoint{1.249956in}{0.148611in}}{\pgfqpoint{7.122250in}{3.850000in}}%
\pgfusepath{clip}%
\pgfsetbuttcap%
\pgfsetmiterjoin%
\definecolor{currentfill}{rgb}{0.944483,0.976624,0.673049}%
\pgfsetfillcolor{currentfill}%
\pgfsetlinewidth{0.000000pt}%
\definecolor{currentstroke}{rgb}{0.000000,0.000000,0.000000}%
\pgfsetstrokecolor{currentstroke}%
\pgfsetstrokeopacity{0.000000}%
\pgfsetdash{}{0pt}%
\pgfpathmoveto{\pgfqpoint{7.119218in}{1.764788in}}%
\pgfpathlineto{\pgfqpoint{7.712739in}{1.764788in}}%
\pgfpathlineto{\pgfqpoint{7.712739in}{1.558905in}}%
\pgfpathlineto{\pgfqpoint{7.119218in}{1.558905in}}%
\pgfpathclose%
\pgfusepath{fill}%
\end{pgfscope}%
\begin{pgfscope}%
\pgfpathrectangle{\pgfqpoint{1.249956in}{0.148611in}}{\pgfqpoint{7.122250in}{3.850000in}}%
\pgfusepath{clip}%
\pgfsetbuttcap%
\pgfsetmiterjoin%
\definecolor{currentfill}{rgb}{0.944483,0.976624,0.673049}%
\pgfsetfillcolor{currentfill}%
\pgfsetfillopacity{0.500000}%
\pgfsetlinewidth{0.000000pt}%
\definecolor{currentstroke}{rgb}{0.000000,0.000000,0.000000}%
\pgfsetstrokecolor{currentstroke}%
\pgfsetstrokeopacity{0.500000}%
\pgfsetdash{}{0pt}%
\pgfpathmoveto{\pgfqpoint{3.162412in}{1.353023in}}%
\pgfpathlineto{\pgfqpoint{3.788907in}{1.353023in}}%
\pgfpathlineto{\pgfqpoint{3.788907in}{1.147141in}}%
\pgfpathlineto{\pgfqpoint{3.162412in}{1.147141in}}%
\pgfpathclose%
\pgfusepath{fill}%
\end{pgfscope}%
\begin{pgfscope}%
\pgfpathrectangle{\pgfqpoint{1.249956in}{0.148611in}}{\pgfqpoint{7.122250in}{3.850000in}}%
\pgfusepath{clip}%
\pgfsetbuttcap%
\pgfsetmiterjoin%
\definecolor{currentfill}{rgb}{0.944483,0.976624,0.673049}%
\pgfsetfillcolor{currentfill}%
\pgfsetlinewidth{0.000000pt}%
\definecolor{currentstroke}{rgb}{0.000000,0.000000,0.000000}%
\pgfsetstrokecolor{currentstroke}%
\pgfsetstrokeopacity{0.000000}%
\pgfsetdash{}{0pt}%
\pgfpathmoveto{\pgfqpoint{7.350032in}{0.941258in}}%
\pgfpathlineto{\pgfqpoint{7.877606in}{0.941258in}}%
\pgfpathlineto{\pgfqpoint{7.877606in}{0.735376in}}%
\pgfpathlineto{\pgfqpoint{7.350032in}{0.735376in}}%
\pgfpathclose%
\pgfusepath{fill}%
\end{pgfscope}%
\begin{pgfscope}%
\pgfpathrectangle{\pgfqpoint{1.249956in}{0.148611in}}{\pgfqpoint{7.122250in}{3.850000in}}%
\pgfusepath{clip}%
\pgfsetbuttcap%
\pgfsetmiterjoin%
\definecolor{currentfill}{rgb}{0.944483,0.976624,0.673049}%
\pgfsetfillcolor{currentfill}%
\pgfsetlinewidth{0.000000pt}%
\definecolor{currentstroke}{rgb}{0.000000,0.000000,0.000000}%
\pgfsetstrokecolor{currentstroke}%
\pgfsetstrokeopacity{0.000000}%
\pgfsetdash{}{0pt}%
\pgfpathmoveto{\pgfqpoint{5.536496in}{0.529493in}}%
\pgfpathlineto{\pgfqpoint{6.162990in}{0.529493in}}%
\pgfpathlineto{\pgfqpoint{6.162990in}{0.323611in}}%
\pgfpathlineto{\pgfqpoint{5.536496in}{0.323611in}}%
\pgfpathclose%
\pgfusepath{fill}%
\end{pgfscope}%
\begin{pgfscope}%
\pgfpathrectangle{\pgfqpoint{1.249956in}{0.148611in}}{\pgfqpoint{7.122250in}{3.850000in}}%
\pgfusepath{clip}%
\pgfsetbuttcap%
\pgfsetmiterjoin%
\definecolor{currentfill}{rgb}{0.819608,0.923722,0.524798}%
\pgfsetfillcolor{currentfill}%
\pgfsetfillopacity{0.500000}%
\pgfsetlinewidth{0.000000pt}%
\definecolor{currentstroke}{rgb}{0.000000,0.000000,0.000000}%
\pgfsetstrokecolor{currentstroke}%
\pgfsetstrokeopacity{0.500000}%
\pgfsetdash{}{0pt}%
\pgfpathmoveto{\pgfqpoint{6.393804in}{3.823611in}}%
\pgfpathlineto{\pgfqpoint{7.119218in}{3.823611in}}%
\pgfpathlineto{\pgfqpoint{7.119218in}{3.617729in}}%
\pgfpathlineto{\pgfqpoint{6.393804in}{3.617729in}}%
\pgfpathclose%
\pgfusepath{fill}%
\end{pgfscope}%
\begin{pgfscope}%
\pgfpathrectangle{\pgfqpoint{1.249956in}{0.148611in}}{\pgfqpoint{7.122250in}{3.850000in}}%
\pgfusepath{clip}%
\pgfsetbuttcap%
\pgfsetmiterjoin%
\definecolor{currentfill}{rgb}{0.819608,0.923722,0.524798}%
\pgfsetfillcolor{currentfill}%
\pgfsetfillopacity{0.500000}%
\pgfsetlinewidth{0.000000pt}%
\definecolor{currentstroke}{rgb}{0.000000,0.000000,0.000000}%
\pgfsetstrokecolor{currentstroke}%
\pgfsetstrokeopacity{0.500000}%
\pgfsetdash{}{0pt}%
\pgfpathmoveto{\pgfqpoint{7.251111in}{3.411846in}}%
\pgfpathlineto{\pgfqpoint{7.613819in}{3.411846in}}%
\pgfpathlineto{\pgfqpoint{7.613819in}{3.205964in}}%
\pgfpathlineto{\pgfqpoint{7.251111in}{3.205964in}}%
\pgfpathclose%
\pgfusepath{fill}%
\end{pgfscope}%
\begin{pgfscope}%
\pgfpathrectangle{\pgfqpoint{1.249956in}{0.148611in}}{\pgfqpoint{7.122250in}{3.850000in}}%
\pgfusepath{clip}%
\pgfsetbuttcap%
\pgfsetmiterjoin%
\definecolor{currentfill}{rgb}{0.819608,0.923722,0.524798}%
\pgfsetfillcolor{currentfill}%
\pgfsetlinewidth{0.000000pt}%
\definecolor{currentstroke}{rgb}{0.000000,0.000000,0.000000}%
\pgfsetstrokecolor{currentstroke}%
\pgfsetstrokeopacity{0.000000}%
\pgfsetdash{}{0pt}%
\pgfpathmoveto{\pgfqpoint{6.690564in}{3.000082in}}%
\pgfpathlineto{\pgfqpoint{7.119218in}{3.000082in}}%
\pgfpathlineto{\pgfqpoint{7.119218in}{2.794199in}}%
\pgfpathlineto{\pgfqpoint{6.690564in}{2.794199in}}%
\pgfpathclose%
\pgfusepath{fill}%
\end{pgfscope}%
\begin{pgfscope}%
\pgfpathrectangle{\pgfqpoint{1.249956in}{0.148611in}}{\pgfqpoint{7.122250in}{3.850000in}}%
\pgfusepath{clip}%
\pgfsetbuttcap%
\pgfsetmiterjoin%
\definecolor{currentfill}{rgb}{0.819608,0.923722,0.524798}%
\pgfsetfillcolor{currentfill}%
\pgfsetfillopacity{0.500000}%
\pgfsetlinewidth{0.000000pt}%
\definecolor{currentstroke}{rgb}{0.000000,0.000000,0.000000}%
\pgfsetstrokecolor{currentstroke}%
\pgfsetstrokeopacity{0.500000}%
\pgfsetdash{}{0pt}%
\pgfpathmoveto{\pgfqpoint{5.701363in}{2.588317in}}%
\pgfpathlineto{\pgfqpoint{6.459750in}{2.588317in}}%
\pgfpathlineto{\pgfqpoint{6.459750in}{2.382435in}}%
\pgfpathlineto{\pgfqpoint{5.701363in}{2.382435in}}%
\pgfpathclose%
\pgfusepath{fill}%
\end{pgfscope}%
\begin{pgfscope}%
\pgfpathrectangle{\pgfqpoint{1.249956in}{0.148611in}}{\pgfqpoint{7.122250in}{3.850000in}}%
\pgfusepath{clip}%
\pgfsetbuttcap%
\pgfsetmiterjoin%
\definecolor{currentfill}{rgb}{0.819608,0.923722,0.524798}%
\pgfsetfillcolor{currentfill}%
\pgfsetfillopacity{0.500000}%
\pgfsetlinewidth{0.000000pt}%
\definecolor{currentstroke}{rgb}{0.000000,0.000000,0.000000}%
\pgfsetstrokecolor{currentstroke}%
\pgfsetstrokeopacity{0.500000}%
\pgfsetdash{}{0pt}%
\pgfpathmoveto{\pgfqpoint{4.052694in}{2.176552in}}%
\pgfpathlineto{\pgfqpoint{4.712161in}{2.176552in}}%
\pgfpathlineto{\pgfqpoint{4.712161in}{1.970670in}}%
\pgfpathlineto{\pgfqpoint{4.052694in}{1.970670in}}%
\pgfpathclose%
\pgfusepath{fill}%
\end{pgfscope}%
\begin{pgfscope}%
\pgfpathrectangle{\pgfqpoint{1.249956in}{0.148611in}}{\pgfqpoint{7.122250in}{3.850000in}}%
\pgfusepath{clip}%
\pgfsetbuttcap%
\pgfsetmiterjoin%
\definecolor{currentfill}{rgb}{0.819608,0.923722,0.524798}%
\pgfsetfillcolor{currentfill}%
\pgfsetlinewidth{0.000000pt}%
\definecolor{currentstroke}{rgb}{0.000000,0.000000,0.000000}%
\pgfsetstrokecolor{currentstroke}%
\pgfsetstrokeopacity{0.000000}%
\pgfsetdash{}{0pt}%
\pgfpathmoveto{\pgfqpoint{7.712739in}{1.764788in}}%
\pgfpathlineto{\pgfqpoint{7.976526in}{1.764788in}}%
\pgfpathlineto{\pgfqpoint{7.976526in}{1.558905in}}%
\pgfpathlineto{\pgfqpoint{7.712739in}{1.558905in}}%
\pgfpathclose%
\pgfusepath{fill}%
\end{pgfscope}%
\begin{pgfscope}%
\pgfpathrectangle{\pgfqpoint{1.249956in}{0.148611in}}{\pgfqpoint{7.122250in}{3.850000in}}%
\pgfusepath{clip}%
\pgfsetbuttcap%
\pgfsetmiterjoin%
\definecolor{currentfill}{rgb}{0.819608,0.923722,0.524798}%
\pgfsetfillcolor{currentfill}%
\pgfsetfillopacity{0.500000}%
\pgfsetlinewidth{0.000000pt}%
\definecolor{currentstroke}{rgb}{0.000000,0.000000,0.000000}%
\pgfsetstrokecolor{currentstroke}%
\pgfsetstrokeopacity{0.500000}%
\pgfsetdash{}{0pt}%
\pgfpathmoveto{\pgfqpoint{3.788907in}{1.353023in}}%
\pgfpathlineto{\pgfqpoint{4.349454in}{1.353023in}}%
\pgfpathlineto{\pgfqpoint{4.349454in}{1.147141in}}%
\pgfpathlineto{\pgfqpoint{3.788907in}{1.147141in}}%
\pgfpathclose%
\pgfusepath{fill}%
\end{pgfscope}%
\begin{pgfscope}%
\pgfpathrectangle{\pgfqpoint{1.249956in}{0.148611in}}{\pgfqpoint{7.122250in}{3.850000in}}%
\pgfusepath{clip}%
\pgfsetbuttcap%
\pgfsetmiterjoin%
\definecolor{currentfill}{rgb}{0.819608,0.923722,0.524798}%
\pgfsetfillcolor{currentfill}%
\pgfsetlinewidth{0.000000pt}%
\definecolor{currentstroke}{rgb}{0.000000,0.000000,0.000000}%
\pgfsetstrokecolor{currentstroke}%
\pgfsetstrokeopacity{0.000000}%
\pgfsetdash{}{0pt}%
\pgfpathmoveto{\pgfqpoint{7.877606in}{0.941258in}}%
\pgfpathlineto{\pgfqpoint{8.141393in}{0.941258in}}%
\pgfpathlineto{\pgfqpoint{8.141393in}{0.735376in}}%
\pgfpathlineto{\pgfqpoint{7.877606in}{0.735376in}}%
\pgfpathclose%
\pgfusepath{fill}%
\end{pgfscope}%
\begin{pgfscope}%
\pgfpathrectangle{\pgfqpoint{1.249956in}{0.148611in}}{\pgfqpoint{7.122250in}{3.850000in}}%
\pgfusepath{clip}%
\pgfsetbuttcap%
\pgfsetmiterjoin%
\definecolor{currentfill}{rgb}{0.819608,0.923722,0.524798}%
\pgfsetfillcolor{currentfill}%
\pgfsetlinewidth{0.000000pt}%
\definecolor{currentstroke}{rgb}{0.000000,0.000000,0.000000}%
\pgfsetstrokecolor{currentstroke}%
\pgfsetstrokeopacity{0.000000}%
\pgfsetdash{}{0pt}%
\pgfpathmoveto{\pgfqpoint{6.162990in}{0.529493in}}%
\pgfpathlineto{\pgfqpoint{6.921378in}{0.529493in}}%
\pgfpathlineto{\pgfqpoint{6.921378in}{0.323611in}}%
\pgfpathlineto{\pgfqpoint{6.162990in}{0.323611in}}%
\pgfpathclose%
\pgfusepath{fill}%
\end{pgfscope}%
\begin{pgfscope}%
\pgfpathrectangle{\pgfqpoint{1.249956in}{0.148611in}}{\pgfqpoint{7.122250in}{3.850000in}}%
\pgfusepath{clip}%
\pgfsetbuttcap%
\pgfsetmiterjoin%
\definecolor{currentfill}{rgb}{0.662745,0.856055,0.423299}%
\pgfsetfillcolor{currentfill}%
\pgfsetfillopacity{0.500000}%
\pgfsetlinewidth{0.000000pt}%
\definecolor{currentstroke}{rgb}{0.000000,0.000000,0.000000}%
\pgfsetstrokecolor{currentstroke}%
\pgfsetstrokeopacity{0.500000}%
\pgfsetdash{}{0pt}%
\pgfpathmoveto{\pgfqpoint{7.119218in}{3.823611in}}%
\pgfpathlineto{\pgfqpoint{7.646792in}{3.823611in}}%
\pgfpathlineto{\pgfqpoint{7.646792in}{3.617729in}}%
\pgfpathlineto{\pgfqpoint{7.119218in}{3.617729in}}%
\pgfpathclose%
\pgfusepath{fill}%
\end{pgfscope}%
\begin{pgfscope}%
\pgfpathrectangle{\pgfqpoint{1.249956in}{0.148611in}}{\pgfqpoint{7.122250in}{3.850000in}}%
\pgfusepath{clip}%
\pgfsetbuttcap%
\pgfsetmiterjoin%
\definecolor{currentfill}{rgb}{0.662745,0.856055,0.423299}%
\pgfsetfillcolor{currentfill}%
\pgfsetfillopacity{0.500000}%
\pgfsetlinewidth{0.000000pt}%
\definecolor{currentstroke}{rgb}{0.000000,0.000000,0.000000}%
\pgfsetstrokecolor{currentstroke}%
\pgfsetstrokeopacity{0.500000}%
\pgfsetdash{}{0pt}%
\pgfpathmoveto{\pgfqpoint{7.613819in}{3.411846in}}%
\pgfpathlineto{\pgfqpoint{7.943552in}{3.411846in}}%
\pgfpathlineto{\pgfqpoint{7.943552in}{3.205964in}}%
\pgfpathlineto{\pgfqpoint{7.613819in}{3.205964in}}%
\pgfpathclose%
\pgfusepath{fill}%
\end{pgfscope}%
\begin{pgfscope}%
\pgfpathrectangle{\pgfqpoint{1.249956in}{0.148611in}}{\pgfqpoint{7.122250in}{3.850000in}}%
\pgfusepath{clip}%
\pgfsetbuttcap%
\pgfsetmiterjoin%
\definecolor{currentfill}{rgb}{0.662745,0.856055,0.423299}%
\pgfsetfillcolor{currentfill}%
\pgfsetlinewidth{0.000000pt}%
\definecolor{currentstroke}{rgb}{0.000000,0.000000,0.000000}%
\pgfsetstrokecolor{currentstroke}%
\pgfsetstrokeopacity{0.000000}%
\pgfsetdash{}{0pt}%
\pgfpathmoveto{\pgfqpoint{7.119218in}{3.000082in}}%
\pgfpathlineto{\pgfqpoint{7.547872in}{3.000082in}}%
\pgfpathlineto{\pgfqpoint{7.547872in}{2.794199in}}%
\pgfpathlineto{\pgfqpoint{7.119218in}{2.794199in}}%
\pgfpathclose%
\pgfusepath{fill}%
\end{pgfscope}%
\begin{pgfscope}%
\pgfpathrectangle{\pgfqpoint{1.249956in}{0.148611in}}{\pgfqpoint{7.122250in}{3.850000in}}%
\pgfusepath{clip}%
\pgfsetbuttcap%
\pgfsetmiterjoin%
\definecolor{currentfill}{rgb}{0.662745,0.856055,0.423299}%
\pgfsetfillcolor{currentfill}%
\pgfsetfillopacity{0.500000}%
\pgfsetlinewidth{0.000000pt}%
\definecolor{currentstroke}{rgb}{0.000000,0.000000,0.000000}%
\pgfsetstrokecolor{currentstroke}%
\pgfsetstrokeopacity{0.500000}%
\pgfsetdash{}{0pt}%
\pgfpathmoveto{\pgfqpoint{6.459750in}{2.588317in}}%
\pgfpathlineto{\pgfqpoint{7.284085in}{2.588317in}}%
\pgfpathlineto{\pgfqpoint{7.284085in}{2.382435in}}%
\pgfpathlineto{\pgfqpoint{6.459750in}{2.382435in}}%
\pgfpathclose%
\pgfusepath{fill}%
\end{pgfscope}%
\begin{pgfscope}%
\pgfpathrectangle{\pgfqpoint{1.249956in}{0.148611in}}{\pgfqpoint{7.122250in}{3.850000in}}%
\pgfusepath{clip}%
\pgfsetbuttcap%
\pgfsetmiterjoin%
\definecolor{currentfill}{rgb}{0.662745,0.856055,0.423299}%
\pgfsetfillcolor{currentfill}%
\pgfsetfillopacity{0.500000}%
\pgfsetlinewidth{0.000000pt}%
\definecolor{currentstroke}{rgb}{0.000000,0.000000,0.000000}%
\pgfsetstrokecolor{currentstroke}%
\pgfsetstrokeopacity{0.500000}%
\pgfsetdash{}{0pt}%
\pgfpathmoveto{\pgfqpoint{4.712161in}{2.176552in}}%
\pgfpathlineto{\pgfqpoint{5.800283in}{2.176552in}}%
\pgfpathlineto{\pgfqpoint{5.800283in}{1.970670in}}%
\pgfpathlineto{\pgfqpoint{4.712161in}{1.970670in}}%
\pgfpathclose%
\pgfusepath{fill}%
\end{pgfscope}%
\begin{pgfscope}%
\pgfpathrectangle{\pgfqpoint{1.249956in}{0.148611in}}{\pgfqpoint{7.122250in}{3.850000in}}%
\pgfusepath{clip}%
\pgfsetbuttcap%
\pgfsetmiterjoin%
\definecolor{currentfill}{rgb}{0.662745,0.856055,0.423299}%
\pgfsetfillcolor{currentfill}%
\pgfsetlinewidth{0.000000pt}%
\definecolor{currentstroke}{rgb}{0.000000,0.000000,0.000000}%
\pgfsetstrokecolor{currentstroke}%
\pgfsetstrokeopacity{0.000000}%
\pgfsetdash{}{0pt}%
\pgfpathmoveto{\pgfqpoint{7.976526in}{1.764788in}}%
\pgfpathlineto{\pgfqpoint{8.108419in}{1.764788in}}%
\pgfpathlineto{\pgfqpoint{8.108419in}{1.558905in}}%
\pgfpathlineto{\pgfqpoint{7.976526in}{1.558905in}}%
\pgfpathclose%
\pgfusepath{fill}%
\end{pgfscope}%
\begin{pgfscope}%
\pgfpathrectangle{\pgfqpoint{1.249956in}{0.148611in}}{\pgfqpoint{7.122250in}{3.850000in}}%
\pgfusepath{clip}%
\pgfsetbuttcap%
\pgfsetmiterjoin%
\definecolor{currentfill}{rgb}{0.662745,0.856055,0.423299}%
\pgfsetfillcolor{currentfill}%
\pgfsetfillopacity{0.500000}%
\pgfsetlinewidth{0.000000pt}%
\definecolor{currentstroke}{rgb}{0.000000,0.000000,0.000000}%
\pgfsetstrokecolor{currentstroke}%
\pgfsetstrokeopacity{0.500000}%
\pgfsetdash{}{0pt}%
\pgfpathmoveto{\pgfqpoint{4.349454in}{1.353023in}}%
\pgfpathlineto{\pgfqpoint{6.130017in}{1.353023in}}%
\pgfpathlineto{\pgfqpoint{6.130017in}{1.147141in}}%
\pgfpathlineto{\pgfqpoint{4.349454in}{1.147141in}}%
\pgfpathclose%
\pgfusepath{fill}%
\end{pgfscope}%
\begin{pgfscope}%
\pgfpathrectangle{\pgfqpoint{1.249956in}{0.148611in}}{\pgfqpoint{7.122250in}{3.850000in}}%
\pgfusepath{clip}%
\pgfsetbuttcap%
\pgfsetmiterjoin%
\definecolor{currentfill}{rgb}{0.662745,0.856055,0.423299}%
\pgfsetfillcolor{currentfill}%
\pgfsetlinewidth{0.000000pt}%
\definecolor{currentstroke}{rgb}{0.000000,0.000000,0.000000}%
\pgfsetstrokecolor{currentstroke}%
\pgfsetstrokeopacity{0.000000}%
\pgfsetdash{}{0pt}%
\pgfpathmoveto{\pgfqpoint{8.141393in}{0.941258in}}%
\pgfpathlineto{\pgfqpoint{8.207339in}{0.941258in}}%
\pgfpathlineto{\pgfqpoint{8.207339in}{0.735376in}}%
\pgfpathlineto{\pgfqpoint{8.141393in}{0.735376in}}%
\pgfpathclose%
\pgfusepath{fill}%
\end{pgfscope}%
\begin{pgfscope}%
\pgfpathrectangle{\pgfqpoint{1.249956in}{0.148611in}}{\pgfqpoint{7.122250in}{3.850000in}}%
\pgfusepath{clip}%
\pgfsetbuttcap%
\pgfsetmiterjoin%
\definecolor{currentfill}{rgb}{0.662745,0.856055,0.423299}%
\pgfsetfillcolor{currentfill}%
\pgfsetlinewidth{0.000000pt}%
\definecolor{currentstroke}{rgb}{0.000000,0.000000,0.000000}%
\pgfsetstrokecolor{currentstroke}%
\pgfsetstrokeopacity{0.000000}%
\pgfsetdash{}{0pt}%
\pgfpathmoveto{\pgfqpoint{6.921378in}{0.529493in}}%
\pgfpathlineto{\pgfqpoint{7.646792in}{0.529493in}}%
\pgfpathlineto{\pgfqpoint{7.646792in}{0.323611in}}%
\pgfpathlineto{\pgfqpoint{6.921378in}{0.323611in}}%
\pgfpathclose%
\pgfusepath{fill}%
\end{pgfscope}%
\begin{pgfscope}%
\pgfpathrectangle{\pgfqpoint{1.249956in}{0.148611in}}{\pgfqpoint{7.122250in}{3.850000in}}%
\pgfusepath{clip}%
\pgfsetbuttcap%
\pgfsetmiterjoin%
\definecolor{currentfill}{rgb}{0.468897,0.771319,0.395771}%
\pgfsetfillcolor{currentfill}%
\pgfsetfillopacity{0.500000}%
\pgfsetlinewidth{0.000000pt}%
\definecolor{currentstroke}{rgb}{0.000000,0.000000,0.000000}%
\pgfsetstrokecolor{currentstroke}%
\pgfsetstrokeopacity{0.500000}%
\pgfsetdash{}{0pt}%
\pgfpathmoveto{\pgfqpoint{7.646792in}{3.823611in}}%
\pgfpathlineto{\pgfqpoint{8.174366in}{3.823611in}}%
\pgfpathlineto{\pgfqpoint{8.174366in}{3.617729in}}%
\pgfpathlineto{\pgfqpoint{7.646792in}{3.617729in}}%
\pgfpathclose%
\pgfusepath{fill}%
\end{pgfscope}%
\begin{pgfscope}%
\pgfpathrectangle{\pgfqpoint{1.249956in}{0.148611in}}{\pgfqpoint{7.122250in}{3.850000in}}%
\pgfusepath{clip}%
\pgfsetbuttcap%
\pgfsetmiterjoin%
\definecolor{currentfill}{rgb}{0.468897,0.771319,0.395771}%
\pgfsetfillcolor{currentfill}%
\pgfsetfillopacity{0.500000}%
\pgfsetlinewidth{0.000000pt}%
\definecolor{currentstroke}{rgb}{0.000000,0.000000,0.000000}%
\pgfsetstrokecolor{currentstroke}%
\pgfsetstrokeopacity{0.500000}%
\pgfsetdash{}{0pt}%
\pgfpathmoveto{\pgfqpoint{7.943552in}{3.411846in}}%
\pgfpathlineto{\pgfqpoint{8.141393in}{3.411846in}}%
\pgfpathlineto{\pgfqpoint{8.141393in}{3.205964in}}%
\pgfpathlineto{\pgfqpoint{7.943552in}{3.205964in}}%
\pgfpathclose%
\pgfusepath{fill}%
\end{pgfscope}%
\begin{pgfscope}%
\pgfpathrectangle{\pgfqpoint{1.249956in}{0.148611in}}{\pgfqpoint{7.122250in}{3.850000in}}%
\pgfusepath{clip}%
\pgfsetbuttcap%
\pgfsetmiterjoin%
\definecolor{currentfill}{rgb}{0.468897,0.771319,0.395771}%
\pgfsetfillcolor{currentfill}%
\pgfsetlinewidth{0.000000pt}%
\definecolor{currentstroke}{rgb}{0.000000,0.000000,0.000000}%
\pgfsetstrokecolor{currentstroke}%
\pgfsetstrokeopacity{0.000000}%
\pgfsetdash{}{0pt}%
\pgfpathmoveto{\pgfqpoint{7.547872in}{3.000082in}}%
\pgfpathlineto{\pgfqpoint{7.910579in}{3.000082in}}%
\pgfpathlineto{\pgfqpoint{7.910579in}{2.794199in}}%
\pgfpathlineto{\pgfqpoint{7.547872in}{2.794199in}}%
\pgfpathclose%
\pgfusepath{fill}%
\end{pgfscope}%
\begin{pgfscope}%
\pgfpathrectangle{\pgfqpoint{1.249956in}{0.148611in}}{\pgfqpoint{7.122250in}{3.850000in}}%
\pgfusepath{clip}%
\pgfsetbuttcap%
\pgfsetmiterjoin%
\definecolor{currentfill}{rgb}{0.468897,0.771319,0.395771}%
\pgfsetfillcolor{currentfill}%
\pgfsetfillopacity{0.500000}%
\pgfsetlinewidth{0.000000pt}%
\definecolor{currentstroke}{rgb}{0.000000,0.000000,0.000000}%
\pgfsetstrokecolor{currentstroke}%
\pgfsetstrokeopacity{0.500000}%
\pgfsetdash{}{0pt}%
\pgfpathmoveto{\pgfqpoint{7.284085in}{2.588317in}}%
\pgfpathlineto{\pgfqpoint{7.943552in}{2.588317in}}%
\pgfpathlineto{\pgfqpoint{7.943552in}{2.382435in}}%
\pgfpathlineto{\pgfqpoint{7.284085in}{2.382435in}}%
\pgfpathclose%
\pgfusepath{fill}%
\end{pgfscope}%
\begin{pgfscope}%
\pgfpathrectangle{\pgfqpoint{1.249956in}{0.148611in}}{\pgfqpoint{7.122250in}{3.850000in}}%
\pgfusepath{clip}%
\pgfsetbuttcap%
\pgfsetmiterjoin%
\definecolor{currentfill}{rgb}{0.468897,0.771319,0.395771}%
\pgfsetfillcolor{currentfill}%
\pgfsetfillopacity{0.500000}%
\pgfsetlinewidth{0.000000pt}%
\definecolor{currentstroke}{rgb}{0.000000,0.000000,0.000000}%
\pgfsetstrokecolor{currentstroke}%
\pgfsetstrokeopacity{0.500000}%
\pgfsetdash{}{0pt}%
\pgfpathmoveto{\pgfqpoint{5.800283in}{2.176552in}}%
\pgfpathlineto{\pgfqpoint{7.284085in}{2.176552in}}%
\pgfpathlineto{\pgfqpoint{7.284085in}{1.970670in}}%
\pgfpathlineto{\pgfqpoint{5.800283in}{1.970670in}}%
\pgfpathclose%
\pgfusepath{fill}%
\end{pgfscope}%
\begin{pgfscope}%
\pgfpathrectangle{\pgfqpoint{1.249956in}{0.148611in}}{\pgfqpoint{7.122250in}{3.850000in}}%
\pgfusepath{clip}%
\pgfsetbuttcap%
\pgfsetmiterjoin%
\definecolor{currentfill}{rgb}{0.468897,0.771319,0.395771}%
\pgfsetfillcolor{currentfill}%
\pgfsetlinewidth{0.000000pt}%
\definecolor{currentstroke}{rgb}{0.000000,0.000000,0.000000}%
\pgfsetstrokecolor{currentstroke}%
\pgfsetstrokeopacity{0.000000}%
\pgfsetdash{}{0pt}%
\pgfpathmoveto{\pgfqpoint{8.108419in}{1.764788in}}%
\pgfpathlineto{\pgfqpoint{8.207339in}{1.764788in}}%
\pgfpathlineto{\pgfqpoint{8.207339in}{1.558905in}}%
\pgfpathlineto{\pgfqpoint{8.108419in}{1.558905in}}%
\pgfpathclose%
\pgfusepath{fill}%
\end{pgfscope}%
\begin{pgfscope}%
\pgfpathrectangle{\pgfqpoint{1.249956in}{0.148611in}}{\pgfqpoint{7.122250in}{3.850000in}}%
\pgfusepath{clip}%
\pgfsetbuttcap%
\pgfsetmiterjoin%
\definecolor{currentfill}{rgb}{0.468897,0.771319,0.395771}%
\pgfsetfillcolor{currentfill}%
\pgfsetfillopacity{0.500000}%
\pgfsetlinewidth{0.000000pt}%
\definecolor{currentstroke}{rgb}{0.000000,0.000000,0.000000}%
\pgfsetstrokecolor{currentstroke}%
\pgfsetstrokeopacity{0.500000}%
\pgfsetdash{}{0pt}%
\pgfpathmoveto{\pgfqpoint{6.130017in}{1.353023in}}%
\pgfpathlineto{\pgfqpoint{7.218138in}{1.353023in}}%
\pgfpathlineto{\pgfqpoint{7.218138in}{1.147141in}}%
\pgfpathlineto{\pgfqpoint{6.130017in}{1.147141in}}%
\pgfpathclose%
\pgfusepath{fill}%
\end{pgfscope}%
\begin{pgfscope}%
\pgfpathrectangle{\pgfqpoint{1.249956in}{0.148611in}}{\pgfqpoint{7.122250in}{3.850000in}}%
\pgfusepath{clip}%
\pgfsetbuttcap%
\pgfsetmiterjoin%
\definecolor{currentfill}{rgb}{0.468897,0.771319,0.395771}%
\pgfsetfillcolor{currentfill}%
\pgfsetlinewidth{0.000000pt}%
\definecolor{currentstroke}{rgb}{0.000000,0.000000,0.000000}%
\pgfsetstrokecolor{currentstroke}%
\pgfsetstrokeopacity{0.000000}%
\pgfsetdash{}{0pt}%
\pgfpathmoveto{\pgfqpoint{8.207339in}{0.941258in}}%
\pgfpathlineto{\pgfqpoint{8.273286in}{0.941258in}}%
\pgfpathlineto{\pgfqpoint{8.273286in}{0.735376in}}%
\pgfpathlineto{\pgfqpoint{8.207339in}{0.735376in}}%
\pgfpathclose%
\pgfusepath{fill}%
\end{pgfscope}%
\begin{pgfscope}%
\pgfpathrectangle{\pgfqpoint{1.249956in}{0.148611in}}{\pgfqpoint{7.122250in}{3.850000in}}%
\pgfusepath{clip}%
\pgfsetbuttcap%
\pgfsetmiterjoin%
\definecolor{currentfill}{rgb}{0.468897,0.771319,0.395771}%
\pgfsetfillcolor{currentfill}%
\pgfsetlinewidth{0.000000pt}%
\definecolor{currentstroke}{rgb}{0.000000,0.000000,0.000000}%
\pgfsetstrokecolor{currentstroke}%
\pgfsetstrokeopacity{0.000000}%
\pgfsetdash{}{0pt}%
\pgfpathmoveto{\pgfqpoint{7.646792in}{0.529493in}}%
\pgfpathlineto{\pgfqpoint{8.009499in}{0.529493in}}%
\pgfpathlineto{\pgfqpoint{8.009499in}{0.323611in}}%
\pgfpathlineto{\pgfqpoint{7.646792in}{0.323611in}}%
\pgfpathclose%
\pgfusepath{fill}%
\end{pgfscope}%
\begin{pgfscope}%
\pgfpathrectangle{\pgfqpoint{1.249956in}{0.148611in}}{\pgfqpoint{7.122250in}{3.850000in}}%
\pgfusepath{clip}%
\pgfsetbuttcap%
\pgfsetmiterjoin%
\definecolor{currentfill}{rgb}{0.248058,0.667205,0.350250}%
\pgfsetfillcolor{currentfill}%
\pgfsetfillopacity{0.500000}%
\pgfsetlinewidth{0.000000pt}%
\definecolor{currentstroke}{rgb}{0.000000,0.000000,0.000000}%
\pgfsetstrokecolor{currentstroke}%
\pgfsetstrokeopacity{0.500000}%
\pgfsetdash{}{0pt}%
\pgfpathmoveto{\pgfqpoint{8.174366in}{3.823611in}}%
\pgfpathlineto{\pgfqpoint{8.372206in}{3.823611in}}%
\pgfpathlineto{\pgfqpoint{8.372206in}{3.617729in}}%
\pgfpathlineto{\pgfqpoint{8.174366in}{3.617729in}}%
\pgfpathclose%
\pgfusepath{fill}%
\end{pgfscope}%
\begin{pgfscope}%
\pgfpathrectangle{\pgfqpoint{1.249956in}{0.148611in}}{\pgfqpoint{7.122250in}{3.850000in}}%
\pgfusepath{clip}%
\pgfsetbuttcap%
\pgfsetmiterjoin%
\definecolor{currentfill}{rgb}{0.248058,0.667205,0.350250}%
\pgfsetfillcolor{currentfill}%
\pgfsetfillopacity{0.500000}%
\pgfsetlinewidth{0.000000pt}%
\definecolor{currentstroke}{rgb}{0.000000,0.000000,0.000000}%
\pgfsetstrokecolor{currentstroke}%
\pgfsetstrokeopacity{0.500000}%
\pgfsetdash{}{0pt}%
\pgfpathmoveto{\pgfqpoint{8.141393in}{3.411846in}}%
\pgfpathlineto{\pgfqpoint{8.372206in}{3.411846in}}%
\pgfpathlineto{\pgfqpoint{8.372206in}{3.205964in}}%
\pgfpathlineto{\pgfqpoint{8.141393in}{3.205964in}}%
\pgfpathclose%
\pgfusepath{fill}%
\end{pgfscope}%
\begin{pgfscope}%
\pgfpathrectangle{\pgfqpoint{1.249956in}{0.148611in}}{\pgfqpoint{7.122250in}{3.850000in}}%
\pgfusepath{clip}%
\pgfsetbuttcap%
\pgfsetmiterjoin%
\definecolor{currentfill}{rgb}{0.248058,0.667205,0.350250}%
\pgfsetfillcolor{currentfill}%
\pgfsetlinewidth{0.000000pt}%
\definecolor{currentstroke}{rgb}{0.000000,0.000000,0.000000}%
\pgfsetstrokecolor{currentstroke}%
\pgfsetstrokeopacity{0.000000}%
\pgfsetdash{}{0pt}%
\pgfpathmoveto{\pgfqpoint{7.910579in}{3.000082in}}%
\pgfpathlineto{\pgfqpoint{8.372206in}{3.000082in}}%
\pgfpathlineto{\pgfqpoint{8.372206in}{2.794199in}}%
\pgfpathlineto{\pgfqpoint{7.910579in}{2.794199in}}%
\pgfpathclose%
\pgfusepath{fill}%
\end{pgfscope}%
\begin{pgfscope}%
\pgfpathrectangle{\pgfqpoint{1.249956in}{0.148611in}}{\pgfqpoint{7.122250in}{3.850000in}}%
\pgfusepath{clip}%
\pgfsetbuttcap%
\pgfsetmiterjoin%
\definecolor{currentfill}{rgb}{0.248058,0.667205,0.350250}%
\pgfsetfillcolor{currentfill}%
\pgfsetfillopacity{0.500000}%
\pgfsetlinewidth{0.000000pt}%
\definecolor{currentstroke}{rgb}{0.000000,0.000000,0.000000}%
\pgfsetstrokecolor{currentstroke}%
\pgfsetstrokeopacity{0.500000}%
\pgfsetdash{}{0pt}%
\pgfpathmoveto{\pgfqpoint{7.943552in}{2.588317in}}%
\pgfpathlineto{\pgfqpoint{8.372206in}{2.588317in}}%
\pgfpathlineto{\pgfqpoint{8.372206in}{2.382435in}}%
\pgfpathlineto{\pgfqpoint{7.943552in}{2.382435in}}%
\pgfpathclose%
\pgfusepath{fill}%
\end{pgfscope}%
\begin{pgfscope}%
\pgfpathrectangle{\pgfqpoint{1.249956in}{0.148611in}}{\pgfqpoint{7.122250in}{3.850000in}}%
\pgfusepath{clip}%
\pgfsetbuttcap%
\pgfsetmiterjoin%
\definecolor{currentfill}{rgb}{0.248058,0.667205,0.350250}%
\pgfsetfillcolor{currentfill}%
\pgfsetfillopacity{0.500000}%
\pgfsetlinewidth{0.000000pt}%
\definecolor{currentstroke}{rgb}{0.000000,0.000000,0.000000}%
\pgfsetstrokecolor{currentstroke}%
\pgfsetstrokeopacity{0.500000}%
\pgfsetdash{}{0pt}%
\pgfpathmoveto{\pgfqpoint{7.284085in}{2.176552in}}%
\pgfpathlineto{\pgfqpoint{8.372206in}{2.176552in}}%
\pgfpathlineto{\pgfqpoint{8.372206in}{1.970670in}}%
\pgfpathlineto{\pgfqpoint{7.284085in}{1.970670in}}%
\pgfpathclose%
\pgfusepath{fill}%
\end{pgfscope}%
\begin{pgfscope}%
\pgfpathrectangle{\pgfqpoint{1.249956in}{0.148611in}}{\pgfqpoint{7.122250in}{3.850000in}}%
\pgfusepath{clip}%
\pgfsetbuttcap%
\pgfsetmiterjoin%
\definecolor{currentfill}{rgb}{0.248058,0.667205,0.350250}%
\pgfsetfillcolor{currentfill}%
\pgfsetlinewidth{0.000000pt}%
\definecolor{currentstroke}{rgb}{0.000000,0.000000,0.000000}%
\pgfsetstrokecolor{currentstroke}%
\pgfsetstrokeopacity{0.000000}%
\pgfsetdash{}{0pt}%
\pgfpathmoveto{\pgfqpoint{8.207339in}{1.764788in}}%
\pgfpathlineto{\pgfqpoint{8.372206in}{1.764788in}}%
\pgfpathlineto{\pgfqpoint{8.372206in}{1.558905in}}%
\pgfpathlineto{\pgfqpoint{8.207339in}{1.558905in}}%
\pgfpathclose%
\pgfusepath{fill}%
\end{pgfscope}%
\begin{pgfscope}%
\pgfpathrectangle{\pgfqpoint{1.249956in}{0.148611in}}{\pgfqpoint{7.122250in}{3.850000in}}%
\pgfusepath{clip}%
\pgfsetbuttcap%
\pgfsetmiterjoin%
\definecolor{currentfill}{rgb}{0.248058,0.667205,0.350250}%
\pgfsetfillcolor{currentfill}%
\pgfsetfillopacity{0.500000}%
\pgfsetlinewidth{0.000000pt}%
\definecolor{currentstroke}{rgb}{0.000000,0.000000,0.000000}%
\pgfsetstrokecolor{currentstroke}%
\pgfsetstrokeopacity{0.500000}%
\pgfsetdash{}{0pt}%
\pgfpathmoveto{\pgfqpoint{7.218138in}{1.353023in}}%
\pgfpathlineto{\pgfqpoint{8.372206in}{1.353023in}}%
\pgfpathlineto{\pgfqpoint{8.372206in}{1.147141in}}%
\pgfpathlineto{\pgfqpoint{7.218138in}{1.147141in}}%
\pgfpathclose%
\pgfusepath{fill}%
\end{pgfscope}%
\begin{pgfscope}%
\pgfpathrectangle{\pgfqpoint{1.249956in}{0.148611in}}{\pgfqpoint{7.122250in}{3.850000in}}%
\pgfusepath{clip}%
\pgfsetbuttcap%
\pgfsetmiterjoin%
\definecolor{currentfill}{rgb}{0.248058,0.667205,0.350250}%
\pgfsetfillcolor{currentfill}%
\pgfsetlinewidth{0.000000pt}%
\definecolor{currentstroke}{rgb}{0.000000,0.000000,0.000000}%
\pgfsetstrokecolor{currentstroke}%
\pgfsetstrokeopacity{0.000000}%
\pgfsetdash{}{0pt}%
\pgfpathmoveto{\pgfqpoint{8.273286in}{0.941258in}}%
\pgfpathlineto{\pgfqpoint{8.372206in}{0.941258in}}%
\pgfpathlineto{\pgfqpoint{8.372206in}{0.735376in}}%
\pgfpathlineto{\pgfqpoint{8.273286in}{0.735376in}}%
\pgfpathclose%
\pgfusepath{fill}%
\end{pgfscope}%
\begin{pgfscope}%
\pgfpathrectangle{\pgfqpoint{1.249956in}{0.148611in}}{\pgfqpoint{7.122250in}{3.850000in}}%
\pgfusepath{clip}%
\pgfsetbuttcap%
\pgfsetmiterjoin%
\definecolor{currentfill}{rgb}{0.248058,0.667205,0.350250}%
\pgfsetfillcolor{currentfill}%
\pgfsetlinewidth{0.000000pt}%
\definecolor{currentstroke}{rgb}{0.000000,0.000000,0.000000}%
\pgfsetstrokecolor{currentstroke}%
\pgfsetstrokeopacity{0.000000}%
\pgfsetdash{}{0pt}%
\pgfpathmoveto{\pgfqpoint{8.009499in}{0.529493in}}%
\pgfpathlineto{\pgfqpoint{8.372206in}{0.529493in}}%
\pgfpathlineto{\pgfqpoint{8.372206in}{0.323611in}}%
\pgfpathlineto{\pgfqpoint{8.009499in}{0.323611in}}%
\pgfpathclose%
\pgfusepath{fill}%
\end{pgfscope}%
\begin{pgfscope}%
\pgfsetbuttcap%
\pgfsetroundjoin%
\definecolor{currentfill}{rgb}{0.000000,0.000000,0.000000}%
\pgfsetfillcolor{currentfill}%
\pgfsetlinewidth{0.803000pt}%
\definecolor{currentstroke}{rgb}{0.000000,0.000000,0.000000}%
\pgfsetstrokecolor{currentstroke}%
\pgfsetdash{}{0pt}%
\pgfsys@defobject{currentmarker}{\pgfqpoint{-0.048611in}{0.000000in}}{\pgfqpoint{-0.000000in}{0.000000in}}{%
\pgfpathmoveto{\pgfqpoint{-0.000000in}{0.000000in}}%
\pgfpathlineto{\pgfqpoint{-0.048611in}{0.000000in}}%
\pgfusepath{stroke,fill}%
}%
\begin{pgfscope}%
\pgfsys@transformshift{1.249956in}{3.720670in}%
\pgfsys@useobject{currentmarker}{}%
\end{pgfscope}%
\end{pgfscope}%
\begin{pgfscope}%
\definecolor{textcolor}{rgb}{0.000000,0.000000,0.000000}%
\pgfsetstrokecolor{textcolor}%
\pgfsetfillcolor{textcolor}%
\pgftext[x=0.482975in, y=3.667908in, left, base]{\color{textcolor}\sffamily\fontsize{10.000000}{12.000000}\selectfont 3DFRONT}%
\end{pgfscope}%
\begin{pgfscope}%
\pgfsetbuttcap%
\pgfsetroundjoin%
\definecolor{currentfill}{rgb}{0.000000,0.000000,0.000000}%
\pgfsetfillcolor{currentfill}%
\pgfsetlinewidth{0.803000pt}%
\definecolor{currentstroke}{rgb}{0.000000,0.000000,0.000000}%
\pgfsetstrokecolor{currentstroke}%
\pgfsetdash{}{0pt}%
\pgfsys@defobject{currentmarker}{\pgfqpoint{-0.048611in}{0.000000in}}{\pgfqpoint{-0.000000in}{0.000000in}}{%
\pgfpathmoveto{\pgfqpoint{-0.000000in}{0.000000in}}%
\pgfpathlineto{\pgfqpoint{-0.048611in}{0.000000in}}%
\pgfusepath{stroke,fill}%
}%
\begin{pgfscope}%
\pgfsys@transformshift{1.249956in}{3.308905in}%
\pgfsys@useobject{currentmarker}{}%
\end{pgfscope}%
\end{pgfscope}%
\begin{pgfscope}%
\definecolor{textcolor}{rgb}{0.000000,0.000000,0.000000}%
\pgfsetstrokecolor{textcolor}%
\pgfsetfillcolor{textcolor}%
\pgftext[x=0.533295in, y=3.256144in, left, base]{\color{textcolor}\sffamily\fontsize{10.000000}{12.000000}\selectfont AI2THOR}%
\end{pgfscope}%
\begin{pgfscope}%
\pgfsetbuttcap%
\pgfsetroundjoin%
\definecolor{currentfill}{rgb}{0.000000,0.000000,0.000000}%
\pgfsetfillcolor{currentfill}%
\pgfsetlinewidth{0.803000pt}%
\definecolor{currentstroke}{rgb}{0.000000,0.000000,0.000000}%
\pgfsetstrokecolor{currentstroke}%
\pgfsetdash{}{0pt}%
\pgfsys@defobject{currentmarker}{\pgfqpoint{-0.048611in}{0.000000in}}{\pgfqpoint{-0.000000in}{0.000000in}}{%
\pgfpathmoveto{\pgfqpoint{-0.000000in}{0.000000in}}%
\pgfpathlineto{\pgfqpoint{-0.048611in}{0.000000in}}%
\pgfusepath{stroke,fill}%
}%
\begin{pgfscope}%
\pgfsys@transformshift{1.249956in}{2.897141in}%
\pgfsys@useobject{currentmarker}{}%
\end{pgfscope}%
\end{pgfscope}%
\begin{pgfscope}%
\definecolor{textcolor}{rgb}{0.000000,0.000000,0.000000}%
\pgfsetstrokecolor{textcolor}%
\pgfsetfillcolor{textcolor}%
\pgftext[x=0.311127in, y=2.844379in, left, base]{\color{textcolor}\sffamily\fontsize{10.000000}{12.000000}\selectfont Blenderproc}%
\end{pgfscope}%
\begin{pgfscope}%
\pgfsetbuttcap%
\pgfsetroundjoin%
\definecolor{currentfill}{rgb}{0.000000,0.000000,0.000000}%
\pgfsetfillcolor{currentfill}%
\pgfsetlinewidth{0.803000pt}%
\definecolor{currentstroke}{rgb}{0.000000,0.000000,0.000000}%
\pgfsetstrokecolor{currentstroke}%
\pgfsetdash{}{0pt}%
\pgfsys@defobject{currentmarker}{\pgfqpoint{-0.048611in}{0.000000in}}{\pgfqpoint{-0.000000in}{0.000000in}}{%
\pgfpathmoveto{\pgfqpoint{-0.000000in}{0.000000in}}%
\pgfpathlineto{\pgfqpoint{-0.048611in}{0.000000in}}%
\pgfusepath{stroke,fill}%
}%
\begin{pgfscope}%
\pgfsys@transformshift{1.249956in}{2.485376in}%
\pgfsys@useobject{currentmarker}{}%
\end{pgfscope}%
\end{pgfscope}%
\begin{pgfscope}%
\definecolor{textcolor}{rgb}{0.000000,0.000000,0.000000}%
\pgfsetstrokecolor{textcolor}%
\pgfsetfillcolor{textcolor}%
\pgftext[x=0.489146in, y=2.432614in, left, base]{\color{textcolor}\sffamily\fontsize{10.000000}{12.000000}\selectfont Hyperism}%
\end{pgfscope}%
\begin{pgfscope}%
\pgfsetbuttcap%
\pgfsetroundjoin%
\definecolor{currentfill}{rgb}{0.000000,0.000000,0.000000}%
\pgfsetfillcolor{currentfill}%
\pgfsetlinewidth{0.803000pt}%
\definecolor{currentstroke}{rgb}{0.000000,0.000000,0.000000}%
\pgfsetstrokecolor{currentstroke}%
\pgfsetdash{}{0pt}%
\pgfsys@defobject{currentmarker}{\pgfqpoint{-0.048611in}{0.000000in}}{\pgfqpoint{-0.000000in}{0.000000in}}{%
\pgfpathmoveto{\pgfqpoint{-0.000000in}{0.000000in}}%
\pgfpathlineto{\pgfqpoint{-0.048611in}{0.000000in}}%
\pgfusepath{stroke,fill}%
}%
\begin{pgfscope}%
\pgfsys@transformshift{1.249956in}{2.073611in}%
\pgfsys@useobject{currentmarker}{}%
\end{pgfscope}%
\end{pgfscope}%
\begin{pgfscope}%
\definecolor{textcolor}{rgb}{0.000000,0.000000,0.000000}%
\pgfsetstrokecolor{textcolor}%
\pgfsetfillcolor{textcolor}%
\pgftext[x=0.402273in, y=2.020850in, left, base]{\color{textcolor}\sffamily\fontsize{10.000000}{12.000000}\selectfont InteriorNet}%
\end{pgfscope}%
\begin{pgfscope}%
\pgfsetbuttcap%
\pgfsetroundjoin%
\definecolor{currentfill}{rgb}{0.000000,0.000000,0.000000}%
\pgfsetfillcolor{currentfill}%
\pgfsetlinewidth{0.803000pt}%
\definecolor{currentstroke}{rgb}{0.000000,0.000000,0.000000}%
\pgfsetstrokecolor{currentstroke}%
\pgfsetdash{}{0pt}%
\pgfsys@defobject{currentmarker}{\pgfqpoint{-0.048611in}{0.000000in}}{\pgfqpoint{-0.000000in}{0.000000in}}{%
\pgfpathmoveto{\pgfqpoint{-0.000000in}{0.000000in}}%
\pgfpathlineto{\pgfqpoint{-0.048611in}{0.000000in}}%
\pgfusepath{stroke,fill}%
}%
\begin{pgfscope}%
\pgfsys@transformshift{1.249956in}{1.661846in}%
\pgfsys@useobject{currentmarker}{}%
\end{pgfscope}%
\end{pgfscope}%
\begin{pgfscope}%
\definecolor{textcolor}{rgb}{0.000000,0.000000,0.000000}%
\pgfsetstrokecolor{textcolor}%
\pgfsetfillcolor{textcolor}%
\pgftext[x=0.313908in, y=1.609085in, left, base]{\color{textcolor}\sffamily\fontsize{10.000000}{12.000000}\selectfont OpenRooms}%
\end{pgfscope}%
\begin{pgfscope}%
\pgfsetbuttcap%
\pgfsetroundjoin%
\definecolor{currentfill}{rgb}{0.000000,0.000000,0.000000}%
\pgfsetfillcolor{currentfill}%
\pgfsetlinewidth{0.803000pt}%
\definecolor{currentstroke}{rgb}{0.000000,0.000000,0.000000}%
\pgfsetstrokecolor{currentstroke}%
\pgfsetdash{}{0pt}%
\pgfsys@defobject{currentmarker}{\pgfqpoint{-0.048611in}{0.000000in}}{\pgfqpoint{-0.000000in}{0.000000in}}{%
\pgfpathmoveto{\pgfqpoint{-0.000000in}{0.000000in}}%
\pgfpathlineto{\pgfqpoint{-0.048611in}{0.000000in}}%
\pgfusepath{stroke,fill}%
}%
\begin{pgfscope}%
\pgfsys@transformshift{1.249956in}{1.250082in}%
\pgfsys@useobject{currentmarker}{}%
\end{pgfscope}%
\end{pgfscope}%
\begin{pgfscope}%
\definecolor{textcolor}{rgb}{0.000000,0.000000,0.000000}%
\pgfsetstrokecolor{textcolor}%
\pgfsetfillcolor{textcolor}%
\pgftext[x=0.755938in, y=1.197320in, left, base]{\color{textcolor}\sffamily\fontsize{10.000000}{12.000000}\selectfont Pix3D}%
\end{pgfscope}%
\begin{pgfscope}%
\pgfsetbuttcap%
\pgfsetroundjoin%
\definecolor{currentfill}{rgb}{0.000000,0.000000,0.000000}%
\pgfsetfillcolor{currentfill}%
\pgfsetlinewidth{0.803000pt}%
\definecolor{currentstroke}{rgb}{0.000000,0.000000,0.000000}%
\pgfsetstrokecolor{currentstroke}%
\pgfsetdash{}{0pt}%
\pgfsys@defobject{currentmarker}{\pgfqpoint{-0.048611in}{0.000000in}}{\pgfqpoint{-0.000000in}{0.000000in}}{%
\pgfpathmoveto{\pgfqpoint{-0.000000in}{0.000000in}}%
\pgfpathlineto{\pgfqpoint{-0.048611in}{0.000000in}}%
\pgfusepath{stroke,fill}%
}%
\begin{pgfscope}%
\pgfsys@transformshift{1.249956in}{0.838317in}%
\pgfsys@useobject{currentmarker}{}%
\end{pgfscope}%
\end{pgfscope}%
\begin{pgfscope}%
\definecolor{textcolor}{rgb}{0.000000,0.000000,0.000000}%
\pgfsetstrokecolor{textcolor}%
\pgfsetfillcolor{textcolor}%
\pgftext[x=0.289968in, y=0.785555in, left, base]{\color{textcolor}\sffamily\fontsize{10.000000}{12.000000}\selectfont S2R:3DFREE}%
\end{pgfscope}%
\begin{pgfscope}%
\pgfsetbuttcap%
\pgfsetroundjoin%
\definecolor{currentfill}{rgb}{0.000000,0.000000,0.000000}%
\pgfsetfillcolor{currentfill}%
\pgfsetlinewidth{0.803000pt}%
\definecolor{currentstroke}{rgb}{0.000000,0.000000,0.000000}%
\pgfsetstrokecolor{currentstroke}%
\pgfsetdash{}{0pt}%
\pgfsys@defobject{currentmarker}{\pgfqpoint{-0.048611in}{0.000000in}}{\pgfqpoint{-0.000000in}{0.000000in}}{%
\pgfpathmoveto{\pgfqpoint{-0.000000in}{0.000000in}}%
\pgfpathlineto{\pgfqpoint{-0.048611in}{0.000000in}}%
\pgfusepath{stroke,fill}%
}%
\begin{pgfscope}%
\pgfsys@transformshift{1.249956in}{0.426552in}%
\pgfsys@useobject{currentmarker}{}%
\end{pgfscope}%
\end{pgfscope}%
\begin{pgfscope}%
\definecolor{textcolor}{rgb}{0.000000,0.000000,0.000000}%
\pgfsetstrokecolor{textcolor}%
\pgfsetfillcolor{textcolor}%
\pgftext[x=0.485484in, y=0.373791in, left, base]{\color{textcolor}\sffamily\fontsize{10.000000}{12.000000}\selectfont SceneNet}%
\end{pgfscope}%
\begin{pgfscope}%
\definecolor{textcolor}{rgb}{0.000000,0.000000,0.000000}%
\pgfsetstrokecolor{textcolor}%
\pgfsetfillcolor{textcolor}%
\pgftext[x=0.234413in,y=2.073611in,,bottom,rotate=90.000000]{\color{textcolor}\sffamily\fontsize{10.000000}{12.000000}\selectfont Datasets}%
\end{pgfscope}%
\begin{pgfscope}%
\pgfsetrectcap%
\pgfsetmiterjoin%
\pgfsetlinewidth{0.803000pt}%
\definecolor{currentstroke}{rgb}{0.000000,0.000000,0.000000}%
\pgfsetstrokecolor{currentstroke}%
\pgfsetdash{}{0pt}%
\pgfpathmoveto{\pgfqpoint{1.249956in}{0.148611in}}%
\pgfpathlineto{\pgfqpoint{1.249956in}{3.998611in}}%
\pgfusepath{stroke}%
\end{pgfscope}%
\begin{pgfscope}%
\pgfsetrectcap%
\pgfsetmiterjoin%
\pgfsetlinewidth{0.803000pt}%
\definecolor{currentstroke}{rgb}{0.000000,0.000000,0.000000}%
\pgfsetstrokecolor{currentstroke}%
\pgfsetdash{}{0pt}%
\pgfpathmoveto{\pgfqpoint{8.372206in}{0.148611in}}%
\pgfpathlineto{\pgfqpoint{8.372206in}{3.998611in}}%
\pgfusepath{stroke}%
\end{pgfscope}%
\begin{pgfscope}%
\pgfsetrectcap%
\pgfsetmiterjoin%
\pgfsetlinewidth{0.803000pt}%
\definecolor{currentstroke}{rgb}{0.000000,0.000000,0.000000}%
\pgfsetstrokecolor{currentstroke}%
\pgfsetdash{}{0pt}%
\pgfpathmoveto{\pgfqpoint{1.249956in}{0.148611in}}%
\pgfpathlineto{\pgfqpoint{8.372206in}{0.148611in}}%
\pgfusepath{stroke}%
\end{pgfscope}%
\begin{pgfscope}%
\pgfsetrectcap%
\pgfsetmiterjoin%
\pgfsetlinewidth{0.803000pt}%
\definecolor{currentstroke}{rgb}{0.000000,0.000000,0.000000}%
\pgfsetstrokecolor{currentstroke}%
\pgfsetdash{}{0pt}%
\pgfpathmoveto{\pgfqpoint{1.249956in}{3.998611in}}%
\pgfpathlineto{\pgfqpoint{8.372206in}{3.998611in}}%
\pgfusepath{stroke}%
\end{pgfscope}%
\begin{pgfscope}%
\definecolor{textcolor}{rgb}{1.000000,1.000000,1.000000}%
\pgfsetstrokecolor{textcolor}%
\pgfsetfillcolor{textcolor}%
\pgftext[x=1.876451in,y=3.720670in,,]{\color{textcolor}\sffamily\fontsize{10.000000}{12.000000}\selectfont 38}%
\end{pgfscope}%
\begin{pgfscope}%
\definecolor{textcolor}{rgb}{1.000000,1.000000,1.000000}%
\pgfsetstrokecolor{textcolor}%
\pgfsetfillcolor{textcolor}%
\pgftext[x=2.272131in,y=3.308905in,,]{\color{textcolor}\sffamily\fontsize{10.000000}{12.000000}\selectfont 62}%
\end{pgfscope}%
\begin{pgfscope}%
\definecolor{textcolor}{rgb}{1.000000,1.000000,1.000000}%
\pgfsetstrokecolor{textcolor}%
\pgfsetfillcolor{textcolor}%
\pgftext[x=1.975371in,y=2.897141in,,]{\color{textcolor}\sffamily\fontsize{10.000000}{12.000000}\selectfont 44}%
\end{pgfscope}%
\begin{pgfscope}%
\definecolor{textcolor}{rgb}{1.000000,1.000000,1.000000}%
\pgfsetstrokecolor{textcolor}%
\pgfsetfillcolor{textcolor}%
\pgftext[x=1.546717in,y=2.485376in,,]{\color{textcolor}\sffamily\fontsize{10.000000}{12.000000}\selectfont 18}%
\end{pgfscope}%
\begin{pgfscope}%
\definecolor{textcolor}{rgb}{1.000000,1.000000,1.000000}%
\pgfsetstrokecolor{textcolor}%
\pgfsetfillcolor{textcolor}%
\pgftext[x=1.431310in,y=2.073611in,,]{\color{textcolor}\sffamily\fontsize{10.000000}{12.000000}\selectfont 11}%
\end{pgfscope}%
\begin{pgfscope}%
\definecolor{textcolor}{rgb}{1.000000,1.000000,1.000000}%
\pgfsetstrokecolor{textcolor}%
\pgfsetfillcolor{textcolor}%
\pgftext[x=2.173211in,y=1.661846in,,]{\color{textcolor}\sffamily\fontsize{10.000000}{12.000000}\selectfont 56}%
\end{pgfscope}%
\begin{pgfscope}%
\definecolor{textcolor}{rgb}{1.000000,1.000000,1.000000}%
\pgfsetstrokecolor{textcolor}%
\pgfsetfillcolor{textcolor}%
\pgftext[x=1.464283in,y=1.250082in,,]{\color{textcolor}\sffamily\fontsize{10.000000}{12.000000}\selectfont 13}%
\end{pgfscope}%
\begin{pgfscope}%
\definecolor{textcolor}{rgb}{1.000000,1.000000,1.000000}%
\pgfsetstrokecolor{textcolor}%
\pgfsetfillcolor{textcolor}%
\pgftext[x=2.206184in,y=0.838317in,,]{\color{textcolor}\sffamily\fontsize{10.000000}{12.000000}\selectfont 58}%
\end{pgfscope}%
\begin{pgfscope}%
\definecolor{textcolor}{rgb}{1.000000,1.000000,1.000000}%
\pgfsetstrokecolor{textcolor}%
\pgfsetfillcolor{textcolor}%
\pgftext[x=1.975371in,y=0.426552in,,]{\color{textcolor}\sffamily\fontsize{10.000000}{12.000000}\selectfont 44}%
\end{pgfscope}%
\begin{pgfscope}%
\definecolor{textcolor}{rgb}{1.000000,1.000000,1.000000}%
\pgfsetstrokecolor{textcolor}%
\pgfsetfillcolor{textcolor}%
\pgftext[x=2.849165in,y=3.720670in,,]{\color{textcolor}\sffamily\fontsize{10.000000}{12.000000}\selectfont 21}%
\end{pgfscope}%
\begin{pgfscope}%
\definecolor{textcolor}{rgb}{1.000000,1.000000,1.000000}%
\pgfsetstrokecolor{textcolor}%
\pgfsetfillcolor{textcolor}%
\pgftext[x=3.755933in,y=3.308905in,,]{\color{textcolor}\sffamily\fontsize{10.000000}{12.000000}\selectfont 28}%
\end{pgfscope}%
\begin{pgfscope}%
\definecolor{textcolor}{rgb}{1.000000,1.000000,1.000000}%
\pgfsetstrokecolor{textcolor}%
\pgfsetfillcolor{textcolor}%
\pgftext[x=3.195386in,y=2.897141in,,]{\color{textcolor}\sffamily\fontsize{10.000000}{12.000000}\selectfont 30}%
\end{pgfscope}%
\begin{pgfscope}%
\definecolor{textcolor}{rgb}{1.000000,1.000000,1.000000}%
\pgfsetstrokecolor{textcolor}%
\pgfsetfillcolor{textcolor}%
\pgftext[x=2.206184in,y=2.485376in,,]{\color{textcolor}\sffamily\fontsize{10.000000}{12.000000}\selectfont 22}%
\end{pgfscope}%
\begin{pgfscope}%
\definecolor{textcolor}{rgb}{1.000000,1.000000,1.000000}%
\pgfsetstrokecolor{textcolor}%
\pgfsetfillcolor{textcolor}%
\pgftext[x=1.744557in,y=2.073611in,,]{\color{textcolor}\sffamily\fontsize{10.000000}{12.000000}\selectfont 8}%
\end{pgfscope}%
\begin{pgfscope}%
\definecolor{textcolor}{rgb}{1.000000,1.000000,1.000000}%
\pgfsetstrokecolor{textcolor}%
\pgfsetfillcolor{textcolor}%
\pgftext[x=3.788907in,y=1.661846in,,]{\color{textcolor}\sffamily\fontsize{10.000000}{12.000000}\selectfont 42}%
\end{pgfscope}%
\begin{pgfscope}%
\definecolor{textcolor}{rgb}{1.000000,1.000000,1.000000}%
\pgfsetstrokecolor{textcolor}%
\pgfsetfillcolor{textcolor}%
\pgftext[x=1.794017in,y=1.250082in,,]{\color{textcolor}\sffamily\fontsize{10.000000}{12.000000}\selectfont 7}%
\end{pgfscope}%
\begin{pgfscope}%
\definecolor{textcolor}{rgb}{1.000000,1.000000,1.000000}%
\pgfsetstrokecolor{textcolor}%
\pgfsetfillcolor{textcolor}%
\pgftext[x=3.739447in,y=0.838317in,,]{\color{textcolor}\sffamily\fontsize{10.000000}{12.000000}\selectfont 35}%
\end{pgfscope}%
\begin{pgfscope}%
\definecolor{textcolor}{rgb}{1.000000,1.000000,1.000000}%
\pgfsetstrokecolor{textcolor}%
\pgfsetfillcolor{textcolor}%
\pgftext[x=3.047006in,y=0.426552in,,]{\color{textcolor}\sffamily\fontsize{10.000000}{12.000000}\selectfont 21}%
\end{pgfscope}%
\begin{pgfscope}%
\definecolor{textcolor}{rgb}{1.000000,1.000000,1.000000}%
\pgfsetstrokecolor{textcolor}%
\pgfsetfillcolor{textcolor}%
\pgftext[x=3.624040in,y=3.720670in,,]{\color{textcolor}\sffamily\fontsize{10.000000}{12.000000}\selectfont 26}%
\end{pgfscope}%
\begin{pgfscope}%
\definecolor{textcolor}{rgb}{1.000000,1.000000,1.000000}%
\pgfsetstrokecolor{textcolor}%
\pgfsetfillcolor{textcolor}%
\pgftext[x=4.728648in,y=3.308905in,,]{\color{textcolor}\sffamily\fontsize{10.000000}{12.000000}\selectfont 31}%
\end{pgfscope}%
\begin{pgfscope}%
\definecolor{textcolor}{rgb}{1.000000,1.000000,1.000000}%
\pgfsetstrokecolor{textcolor}%
\pgfsetfillcolor{textcolor}%
\pgftext[x=4.201074in,y=2.897141in,,]{\color{textcolor}\sffamily\fontsize{10.000000}{12.000000}\selectfont 31}%
\end{pgfscope}%
\begin{pgfscope}%
\definecolor{textcolor}{rgb}{1.000000,1.000000,1.000000}%
\pgfsetstrokecolor{textcolor}%
\pgfsetfillcolor{textcolor}%
\pgftext[x=3.030519in,y=2.485376in,,]{\color{textcolor}\sffamily\fontsize{10.000000}{12.000000}\selectfont 28}%
\end{pgfscope}%
\begin{pgfscope}%
\definecolor{textcolor}{rgb}{1.000000,1.000000,1.000000}%
\pgfsetstrokecolor{textcolor}%
\pgfsetfillcolor{textcolor}%
\pgftext[x=2.008344in,y=2.073611in,,]{\color{textcolor}\sffamily\fontsize{10.000000}{12.000000}\selectfont 8}%
\end{pgfscope}%
\begin{pgfscope}%
\definecolor{textcolor}{rgb}{1.000000,1.000000,1.000000}%
\pgfsetstrokecolor{textcolor}%
\pgfsetfillcolor{textcolor}%
\pgftext[x=4.942975in,y=1.661846in,,]{\color{textcolor}\sffamily\fontsize{10.000000}{12.000000}\selectfont 28}%
\end{pgfscope}%
\begin{pgfscope}%
\definecolor{textcolor}{rgb}{1.000000,1.000000,1.000000}%
\pgfsetstrokecolor{textcolor}%
\pgfsetfillcolor{textcolor}%
\pgftext[x=2.024831in,y=1.250082in,,]{\color{textcolor}\sffamily\fontsize{10.000000}{12.000000}\selectfont 7}%
\end{pgfscope}%
\begin{pgfscope}%
\definecolor{textcolor}{rgb}{1.000000,1.000000,1.000000}%
\pgfsetstrokecolor{textcolor}%
\pgfsetfillcolor{textcolor}%
\pgftext[x=5.124328in,y=0.838317in,,]{\color{textcolor}\sffamily\fontsize{10.000000}{12.000000}\selectfont 49}%
\end{pgfscope}%
\begin{pgfscope}%
\definecolor{textcolor}{rgb}{1.000000,1.000000,1.000000}%
\pgfsetstrokecolor{textcolor}%
\pgfsetfillcolor{textcolor}%
\pgftext[x=3.722960in,y=0.426552in,,]{\color{textcolor}\sffamily\fontsize{10.000000}{12.000000}\selectfont 20}%
\end{pgfscope}%
\begin{pgfscope}%
\definecolor{textcolor}{rgb}{1.000000,1.000000,1.000000}%
\pgfsetstrokecolor{textcolor}%
\pgfsetfillcolor{textcolor}%
\pgftext[x=4.497834in,y=3.720670in,,]{\color{textcolor}\sffamily\fontsize{10.000000}{12.000000}\selectfont 27}%
\end{pgfscope}%
\begin{pgfscope}%
\definecolor{textcolor}{rgb}{1.000000,1.000000,1.000000}%
\pgfsetstrokecolor{textcolor}%
\pgfsetfillcolor{textcolor}%
\pgftext[x=5.800283in,y=3.308905in,,]{\color{textcolor}\sffamily\fontsize{10.000000}{12.000000}\selectfont 34}%
\end{pgfscope}%
\begin{pgfscope}%
\definecolor{textcolor}{rgb}{1.000000,1.000000,1.000000}%
\pgfsetstrokecolor{textcolor}%
\pgfsetfillcolor{textcolor}%
\pgftext[x=5.041895in,y=2.897141in,,]{\color{textcolor}\sffamily\fontsize{10.000000}{12.000000}\selectfont 20}%
\end{pgfscope}%
\begin{pgfscope}%
\definecolor{textcolor}{rgb}{1.000000,1.000000,1.000000}%
\pgfsetstrokecolor{textcolor}%
\pgfsetfillcolor{textcolor}%
\pgftext[x=4.003234in,y=2.485376in,,]{\color{textcolor}\sffamily\fontsize{10.000000}{12.000000}\selectfont 31}%
\end{pgfscope}%
\begin{pgfscope}%
\definecolor{textcolor}{rgb}{1.000000,1.000000,1.000000}%
\pgfsetstrokecolor{textcolor}%
\pgfsetfillcolor{textcolor}%
\pgftext[x=2.519432in,y=2.073611in,,]{\color{textcolor}\sffamily\fontsize{10.000000}{12.000000}\selectfont 23}%
\end{pgfscope}%
\begin{pgfscope}%
\definecolor{textcolor}{rgb}{1.000000,1.000000,1.000000}%
\pgfsetstrokecolor{textcolor}%
\pgfsetfillcolor{textcolor}%
\pgftext[x=5.948663in,y=1.661846in,,]{\color{textcolor}\sffamily\fontsize{10.000000}{12.000000}\selectfont 33}%
\end{pgfscope}%
\begin{pgfscope}%
\definecolor{textcolor}{rgb}{1.000000,1.000000,1.000000}%
\pgfsetstrokecolor{textcolor}%
\pgfsetfillcolor{textcolor}%
\pgftext[x=2.354565in,y=1.250082in,,]{\color{textcolor}\sffamily\fontsize{10.000000}{12.000000}\selectfont 13}%
\end{pgfscope}%
\begin{pgfscope}%
\definecolor{textcolor}{rgb}{1.000000,1.000000,1.000000}%
\pgfsetstrokecolor{textcolor}%
\pgfsetfillcolor{textcolor}%
\pgftext[x=6.426777in,y=0.838317in,,]{\color{textcolor}\sffamily\fontsize{10.000000}{12.000000}\selectfont 30}%
\end{pgfscope}%
\begin{pgfscope}%
\definecolor{textcolor}{rgb}{1.000000,1.000000,1.000000}%
\pgfsetstrokecolor{textcolor}%
\pgfsetfillcolor{textcolor}%
\pgftext[x=4.530808in,y=0.426552in,,]{\color{textcolor}\sffamily\fontsize{10.000000}{12.000000}\selectfont 29}%
\end{pgfscope}%
\begin{pgfscope}%
\definecolor{textcolor}{rgb}{0.662745,0.662745,0.662745}%
\pgfsetstrokecolor{textcolor}%
\pgfsetfillcolor{textcolor}%
\pgftext[x=5.157302in,y=3.720670in,,]{\color{textcolor}\sffamily\fontsize{10.000000}{12.000000}\selectfont 13}%
\end{pgfscope}%
\begin{pgfscope}%
\definecolor{textcolor}{rgb}{0.662745,0.662745,0.662745}%
\pgfsetstrokecolor{textcolor}%
\pgfsetfillcolor{textcolor}%
\pgftext[x=6.608131in,y=3.308905in,,]{\color{textcolor}\sffamily\fontsize{10.000000}{12.000000}\selectfont 15}%
\end{pgfscope}%
\begin{pgfscope}%
\definecolor{textcolor}{rgb}{0.662745,0.662745,0.662745}%
\pgfsetstrokecolor{textcolor}%
\pgfsetfillcolor{textcolor}%
\pgftext[x=5.783796in,y=2.897141in,,]{\color{textcolor}\sffamily\fontsize{10.000000}{12.000000}\selectfont 25}%
\end{pgfscope}%
\begin{pgfscope}%
\definecolor{textcolor}{rgb}{0.662745,0.662745,0.662745}%
\pgfsetstrokecolor{textcolor}%
\pgfsetfillcolor{textcolor}%
\pgftext[x=4.877028in,y=2.485376in,,]{\color{textcolor}\sffamily\fontsize{10.000000}{12.000000}\selectfont 22}%
\end{pgfscope}%
\begin{pgfscope}%
\definecolor{textcolor}{rgb}{0.662745,0.662745,0.662745}%
\pgfsetstrokecolor{textcolor}%
\pgfsetfillcolor{textcolor}%
\pgftext[x=3.162412in,y=2.073611in,,]{\color{textcolor}\sffamily\fontsize{10.000000}{12.000000}\selectfont 16}%
\end{pgfscope}%
\begin{pgfscope}%
\definecolor{textcolor}{rgb}{0.662745,0.662745,0.662745}%
\pgfsetstrokecolor{textcolor}%
\pgfsetfillcolor{textcolor}%
\pgftext[x=6.805971in,y=1.661846in,,]{\color{textcolor}\sffamily\fontsize{10.000000}{12.000000}\selectfont 19}%
\end{pgfscope}%
\begin{pgfscope}%
\definecolor{textcolor}{rgb}{0.662745,0.662745,0.662745}%
\pgfsetstrokecolor{textcolor}%
\pgfsetfillcolor{textcolor}%
\pgftext[x=2.865652in,y=1.250082in,,]{\color{textcolor}\sffamily\fontsize{10.000000}{12.000000}\selectfont 18}%
\end{pgfscope}%
\begin{pgfscope}%
\definecolor{textcolor}{rgb}{0.662745,0.662745,0.662745}%
\pgfsetstrokecolor{textcolor}%
\pgfsetfillcolor{textcolor}%
\pgftext[x=7.135705in,y=0.838317in,,]{\color{textcolor}\sffamily\fontsize{10.000000}{12.000000}\selectfont 13}%
\end{pgfscope}%
\begin{pgfscope}%
\definecolor{textcolor}{rgb}{0.662745,0.662745,0.662745}%
\pgfsetstrokecolor{textcolor}%
\pgfsetfillcolor{textcolor}%
\pgftext[x=5.272709in,y=0.426552in,,]{\color{textcolor}\sffamily\fontsize{10.000000}{12.000000}\selectfont 16}%
\end{pgfscope}%
\begin{pgfscope}%
\definecolor{textcolor}{rgb}{0.662745,0.662745,0.662745}%
\pgfsetstrokecolor{textcolor}%
\pgfsetfillcolor{textcolor}%
\pgftext[x=5.882716in,y=3.720670in,,]{\color{textcolor}\sffamily\fontsize{10.000000}{12.000000}\selectfont 31}%
\end{pgfscope}%
\begin{pgfscope}%
\definecolor{textcolor}{rgb}{0.662745,0.662745,0.662745}%
\pgfsetstrokecolor{textcolor}%
\pgfsetfillcolor{textcolor}%
\pgftext[x=7.053271in,y=3.308905in,,]{\color{textcolor}\sffamily\fontsize{10.000000}{12.000000}\selectfont 12}%
\end{pgfscope}%
\begin{pgfscope}%
\definecolor{textcolor}{rgb}{0.662745,0.662745,0.662745}%
\pgfsetstrokecolor{textcolor}%
\pgfsetfillcolor{textcolor}%
\pgftext[x=6.443264in,y=2.897141in,,]{\color{textcolor}\sffamily\fontsize{10.000000}{12.000000}\selectfont 15}%
\end{pgfscope}%
\begin{pgfscope}%
\definecolor{textcolor}{rgb}{0.662745,0.662745,0.662745}%
\pgfsetstrokecolor{textcolor}%
\pgfsetfillcolor{textcolor}%
\pgftext[x=5.470549in,y=2.485376in,,]{\color{textcolor}\sffamily\fontsize{10.000000}{12.000000}\selectfont 14}%
\end{pgfscope}%
\begin{pgfscope}%
\definecolor{textcolor}{rgb}{0.662745,0.662745,0.662745}%
\pgfsetstrokecolor{textcolor}%
\pgfsetfillcolor{textcolor}%
\pgftext[x=3.739447in,y=2.073611in,,]{\color{textcolor}\sffamily\fontsize{10.000000}{12.000000}\selectfont 19}%
\end{pgfscope}%
\begin{pgfscope}%
\definecolor{textcolor}{rgb}{0.662745,0.662745,0.662745}%
\pgfsetstrokecolor{textcolor}%
\pgfsetfillcolor{textcolor}%
\pgftext[x=7.415978in,y=1.661846in,,]{\color{textcolor}\sffamily\fontsize{10.000000}{12.000000}\selectfont 18}%
\end{pgfscope}%
\begin{pgfscope}%
\definecolor{textcolor}{rgb}{0.662745,0.662745,0.662745}%
\pgfsetstrokecolor{textcolor}%
\pgfsetfillcolor{textcolor}%
\pgftext[x=3.475660in,y=1.250082in,,]{\color{textcolor}\sffamily\fontsize{10.000000}{12.000000}\selectfont 19}%
\end{pgfscope}%
\begin{pgfscope}%
\definecolor{textcolor}{rgb}{0.662745,0.662745,0.662745}%
\pgfsetstrokecolor{textcolor}%
\pgfsetfillcolor{textcolor}%
\pgftext[x=7.613819in,y=0.838317in,,]{\color{textcolor}\sffamily\fontsize{10.000000}{12.000000}\selectfont 16}%
\end{pgfscope}%
\begin{pgfscope}%
\definecolor{textcolor}{rgb}{0.662745,0.662745,0.662745}%
\pgfsetstrokecolor{textcolor}%
\pgfsetfillcolor{textcolor}%
\pgftext[x=5.849743in,y=0.426552in,,]{\color{textcolor}\sffamily\fontsize{10.000000}{12.000000}\selectfont 19}%
\end{pgfscope}%
\begin{pgfscope}%
\definecolor{textcolor}{rgb}{1.000000,1.000000,1.000000}%
\pgfsetstrokecolor{textcolor}%
\pgfsetfillcolor{textcolor}%
\pgftext[x=6.756511in,y=3.720670in,,]{\color{textcolor}\sffamily\fontsize{10.000000}{12.000000}\selectfont 22}%
\end{pgfscope}%
\begin{pgfscope}%
\definecolor{textcolor}{rgb}{1.000000,1.000000,1.000000}%
\pgfsetstrokecolor{textcolor}%
\pgfsetfillcolor{textcolor}%
\pgftext[x=7.432465in,y=3.308905in,,]{\color{textcolor}\sffamily\fontsize{10.000000}{12.000000}\selectfont 11}%
\end{pgfscope}%
\begin{pgfscope}%
\definecolor{textcolor}{rgb}{1.000000,1.000000,1.000000}%
\pgfsetstrokecolor{textcolor}%
\pgfsetfillcolor{textcolor}%
\pgftext[x=6.904891in,y=2.897141in,,]{\color{textcolor}\sffamily\fontsize{10.000000}{12.000000}\selectfont 13}%
\end{pgfscope}%
\begin{pgfscope}%
\definecolor{textcolor}{rgb}{1.000000,1.000000,1.000000}%
\pgfsetstrokecolor{textcolor}%
\pgfsetfillcolor{textcolor}%
\pgftext[x=6.080557in,y=2.485376in,,]{\color{textcolor}\sffamily\fontsize{10.000000}{12.000000}\selectfont 23}%
\end{pgfscope}%
\begin{pgfscope}%
\definecolor{textcolor}{rgb}{1.000000,1.000000,1.000000}%
\pgfsetstrokecolor{textcolor}%
\pgfsetfillcolor{textcolor}%
\pgftext[x=4.382427in,y=2.073611in,,]{\color{textcolor}\sffamily\fontsize{10.000000}{12.000000}\selectfont 20}%
\end{pgfscope}%
\begin{pgfscope}%
\definecolor{textcolor}{rgb}{1.000000,1.000000,1.000000}%
\pgfsetstrokecolor{textcolor}%
\pgfsetfillcolor{textcolor}%
\pgftext[x=7.844632in,y=1.661846in,,]{\color{textcolor}\sffamily\fontsize{10.000000}{12.000000}\selectfont 8}%
\end{pgfscope}%
\begin{pgfscope}%
\definecolor{textcolor}{rgb}{1.000000,1.000000,1.000000}%
\pgfsetstrokecolor{textcolor}%
\pgfsetfillcolor{textcolor}%
\pgftext[x=4.069180in,y=1.250082in,,]{\color{textcolor}\sffamily\fontsize{10.000000}{12.000000}\selectfont 17}%
\end{pgfscope}%
\begin{pgfscope}%
\definecolor{textcolor}{rgb}{1.000000,1.000000,1.000000}%
\pgfsetstrokecolor{textcolor}%
\pgfsetfillcolor{textcolor}%
\pgftext[x=8.009499in,y=0.838317in,,]{\color{textcolor}\sffamily\fontsize{10.000000}{12.000000}\selectfont 8}%
\end{pgfscope}%
\begin{pgfscope}%
\definecolor{textcolor}{rgb}{1.000000,1.000000,1.000000}%
\pgfsetstrokecolor{textcolor}%
\pgfsetfillcolor{textcolor}%
\pgftext[x=6.542184in,y=0.426552in,,]{\color{textcolor}\sffamily\fontsize{10.000000}{12.000000}\selectfont 23}%
\end{pgfscope}%
\begin{pgfscope}%
\definecolor{textcolor}{rgb}{1.000000,1.000000,1.000000}%
\pgfsetstrokecolor{textcolor}%
\pgfsetfillcolor{textcolor}%
\pgftext[x=7.383005in,y=3.720670in,,]{\color{textcolor}\sffamily\fontsize{10.000000}{12.000000}\selectfont 16}%
\end{pgfscope}%
\begin{pgfscope}%
\definecolor{textcolor}{rgb}{1.000000,1.000000,1.000000}%
\pgfsetstrokecolor{textcolor}%
\pgfsetfillcolor{textcolor}%
\pgftext[x=7.778686in,y=3.308905in,,]{\color{textcolor}\sffamily\fontsize{10.000000}{12.000000}\selectfont 10}%
\end{pgfscope}%
\begin{pgfscope}%
\definecolor{textcolor}{rgb}{1.000000,1.000000,1.000000}%
\pgfsetstrokecolor{textcolor}%
\pgfsetfillcolor{textcolor}%
\pgftext[x=7.333545in,y=2.897141in,,]{\color{textcolor}\sffamily\fontsize{10.000000}{12.000000}\selectfont 13}%
\end{pgfscope}%
\begin{pgfscope}%
\definecolor{textcolor}{rgb}{1.000000,1.000000,1.000000}%
\pgfsetstrokecolor{textcolor}%
\pgfsetfillcolor{textcolor}%
\pgftext[x=6.871918in,y=2.485376in,,]{\color{textcolor}\sffamily\fontsize{10.000000}{12.000000}\selectfont 25}%
\end{pgfscope}%
\begin{pgfscope}%
\definecolor{textcolor}{rgb}{1.000000,1.000000,1.000000}%
\pgfsetstrokecolor{textcolor}%
\pgfsetfillcolor{textcolor}%
\pgftext[x=5.256222in,y=2.073611in,,]{\color{textcolor}\sffamily\fontsize{10.000000}{12.000000}\selectfont 33}%
\end{pgfscope}%
\begin{pgfscope}%
\definecolor{textcolor}{rgb}{1.000000,1.000000,1.000000}%
\pgfsetstrokecolor{textcolor}%
\pgfsetfillcolor{textcolor}%
\pgftext[x=8.042473in,y=1.661846in,,]{\color{textcolor}\sffamily\fontsize{10.000000}{12.000000}\selectfont 4}%
\end{pgfscope}%
\begin{pgfscope}%
\definecolor{textcolor}{rgb}{1.000000,1.000000,1.000000}%
\pgfsetstrokecolor{textcolor}%
\pgfsetfillcolor{textcolor}%
\pgftext[x=5.239735in,y=1.250082in,,]{\color{textcolor}\sffamily\fontsize{10.000000}{12.000000}\selectfont 54}%
\end{pgfscope}%
\begin{pgfscope}%
\definecolor{textcolor}{rgb}{1.000000,1.000000,1.000000}%
\pgfsetstrokecolor{textcolor}%
\pgfsetfillcolor{textcolor}%
\pgftext[x=8.174366in,y=0.838317in,,]{\color{textcolor}\sffamily\fontsize{10.000000}{12.000000}\selectfont 2}%
\end{pgfscope}%
\begin{pgfscope}%
\definecolor{textcolor}{rgb}{1.000000,1.000000,1.000000}%
\pgfsetstrokecolor{textcolor}%
\pgfsetfillcolor{textcolor}%
\pgftext[x=7.284085in,y=0.426552in,,]{\color{textcolor}\sffamily\fontsize{10.000000}{12.000000}\selectfont 22}%
\end{pgfscope}%
\begin{pgfscope}%
\definecolor{textcolor}{rgb}{1.000000,1.000000,1.000000}%
\pgfsetstrokecolor{textcolor}%
\pgfsetfillcolor{textcolor}%
\pgftext[x=7.910579in,y=3.720670in,,]{\color{textcolor}\sffamily\fontsize{10.000000}{12.000000}\selectfont 16}%
\end{pgfscope}%
\begin{pgfscope}%
\definecolor{textcolor}{rgb}{1.000000,1.000000,1.000000}%
\pgfsetstrokecolor{textcolor}%
\pgfsetfillcolor{textcolor}%
\pgftext[x=8.042473in,y=3.308905in,,]{\color{textcolor}\sffamily\fontsize{10.000000}{12.000000}\selectfont 6}%
\end{pgfscope}%
\begin{pgfscope}%
\definecolor{textcolor}{rgb}{1.000000,1.000000,1.000000}%
\pgfsetstrokecolor{textcolor}%
\pgfsetfillcolor{textcolor}%
\pgftext[x=7.729225in,y=2.897141in,,]{\color{textcolor}\sffamily\fontsize{10.000000}{12.000000}\selectfont 11}%
\end{pgfscope}%
\begin{pgfscope}%
\definecolor{textcolor}{rgb}{1.000000,1.000000,1.000000}%
\pgfsetstrokecolor{textcolor}%
\pgfsetfillcolor{textcolor}%
\pgftext[x=7.613819in,y=2.485376in,,]{\color{textcolor}\sffamily\fontsize{10.000000}{12.000000}\selectfont 20}%
\end{pgfscope}%
\begin{pgfscope}%
\definecolor{textcolor}{rgb}{1.000000,1.000000,1.000000}%
\pgfsetstrokecolor{textcolor}%
\pgfsetfillcolor{textcolor}%
\pgftext[x=6.542184in,y=2.073611in,,]{\color{textcolor}\sffamily\fontsize{10.000000}{12.000000}\selectfont 45}%
\end{pgfscope}%
\begin{pgfscope}%
\definecolor{textcolor}{rgb}{1.000000,1.000000,1.000000}%
\pgfsetstrokecolor{textcolor}%
\pgfsetfillcolor{textcolor}%
\pgftext[x=8.157879in,y=1.661846in,,]{\color{textcolor}\sffamily\fontsize{10.000000}{12.000000}\selectfont 3}%
\end{pgfscope}%
\begin{pgfscope}%
\definecolor{textcolor}{rgb}{1.000000,1.000000,1.000000}%
\pgfsetstrokecolor{textcolor}%
\pgfsetfillcolor{textcolor}%
\pgftext[x=6.674077in,y=1.250082in,,]{\color{textcolor}\sffamily\fontsize{10.000000}{12.000000}\selectfont 33}%
\end{pgfscope}%
\begin{pgfscope}%
\definecolor{textcolor}{rgb}{1.000000,1.000000,1.000000}%
\pgfsetstrokecolor{textcolor}%
\pgfsetfillcolor{textcolor}%
\pgftext[x=8.240313in,y=0.838317in,,]{\color{textcolor}\sffamily\fontsize{10.000000}{12.000000}\selectfont 2}%
\end{pgfscope}%
\begin{pgfscope}%
\definecolor{textcolor}{rgb}{1.000000,1.000000,1.000000}%
\pgfsetstrokecolor{textcolor}%
\pgfsetfillcolor{textcolor}%
\pgftext[x=7.828146in,y=0.426552in,,]{\color{textcolor}\sffamily\fontsize{10.000000}{12.000000}\selectfont 11}%
\end{pgfscope}%
\begin{pgfscope}%
\definecolor{textcolor}{rgb}{1.000000,1.000000,1.000000}%
\pgfsetstrokecolor{textcolor}%
\pgfsetfillcolor{textcolor}%
\pgftext[x=8.273286in,y=3.720670in,,]{\color{textcolor}\sffamily\fontsize{10.000000}{12.000000}\selectfont 6}%
\end{pgfscope}%
\begin{pgfscope}%
\definecolor{textcolor}{rgb}{1.000000,1.000000,1.000000}%
\pgfsetstrokecolor{textcolor}%
\pgfsetfillcolor{textcolor}%
\pgftext[x=8.256800in,y=3.308905in,,]{\color{textcolor}\sffamily\fontsize{10.000000}{12.000000}\selectfont 7}%
\end{pgfscope}%
\begin{pgfscope}%
\definecolor{textcolor}{rgb}{1.000000,1.000000,1.000000}%
\pgfsetstrokecolor{textcolor}%
\pgfsetfillcolor{textcolor}%
\pgftext[x=8.141393in,y=2.897141in,,]{\color{textcolor}\sffamily\fontsize{10.000000}{12.000000}\selectfont 14}%
\end{pgfscope}%
\begin{pgfscope}%
\definecolor{textcolor}{rgb}{1.000000,1.000000,1.000000}%
\pgfsetstrokecolor{textcolor}%
\pgfsetfillcolor{textcolor}%
\pgftext[x=8.157879in,y=2.485376in,,]{\color{textcolor}\sffamily\fontsize{10.000000}{12.000000}\selectfont 13}%
\end{pgfscope}%
\begin{pgfscope}%
\definecolor{textcolor}{rgb}{1.000000,1.000000,1.000000}%
\pgfsetstrokecolor{textcolor}%
\pgfsetfillcolor{textcolor}%
\pgftext[x=7.828146in,y=2.073611in,,]{\color{textcolor}\sffamily\fontsize{10.000000}{12.000000}\selectfont 33}%
\end{pgfscope}%
\begin{pgfscope}%
\definecolor{textcolor}{rgb}{1.000000,1.000000,1.000000}%
\pgfsetstrokecolor{textcolor}%
\pgfsetfillcolor{textcolor}%
\pgftext[x=8.289773in,y=1.661846in,,]{\color{textcolor}\sffamily\fontsize{10.000000}{12.000000}\selectfont 5}%
\end{pgfscope}%
\begin{pgfscope}%
\definecolor{textcolor}{rgb}{1.000000,1.000000,1.000000}%
\pgfsetstrokecolor{textcolor}%
\pgfsetfillcolor{textcolor}%
\pgftext[x=7.795172in,y=1.250082in,,]{\color{textcolor}\sffamily\fontsize{10.000000}{12.000000}\selectfont 35}%
\end{pgfscope}%
\begin{pgfscope}%
\definecolor{textcolor}{rgb}{1.000000,1.000000,1.000000}%
\pgfsetstrokecolor{textcolor}%
\pgfsetfillcolor{textcolor}%
\pgftext[x=8.322746in,y=0.838317in,,]{\color{textcolor}\sffamily\fontsize{10.000000}{12.000000}\selectfont 3}%
\end{pgfscope}%
\begin{pgfscope}%
\definecolor{textcolor}{rgb}{1.000000,1.000000,1.000000}%
\pgfsetstrokecolor{textcolor}%
\pgfsetfillcolor{textcolor}%
\pgftext[x=8.190853in,y=0.426552in,,]{\color{textcolor}\sffamily\fontsize{10.000000}{12.000000}\selectfont 11}%
\end{pgfscope}%
\begin{pgfscope}%
\pgfsetbuttcap%
\pgfsetmiterjoin%
\definecolor{currentfill}{rgb}{1.000000,1.000000,1.000000}%
\pgfsetfillcolor{currentfill}%
\pgfsetfillopacity{0.800000}%
\pgfsetlinewidth{1.003750pt}%
\definecolor{currentstroke}{rgb}{0.800000,0.800000,0.800000}%
\pgfsetstrokecolor{currentstroke}%
\pgfsetstrokeopacity{0.800000}%
\pgfsetdash{}{0pt}%
\pgfpathmoveto{\pgfqpoint{1.330943in}{4.056458in}}%
\pgfpathlineto{\pgfqpoint{7.508857in}{4.056458in}}%
\pgfpathquadraticcurveto{\pgfqpoint{7.531995in}{4.056458in}}{\pgfqpoint{7.531995in}{4.079597in}}%
\pgfpathlineto{\pgfqpoint{7.531995in}{4.237841in}}%
\pgfpathquadraticcurveto{\pgfqpoint{7.531995in}{4.260980in}}{\pgfqpoint{7.508857in}{4.260980in}}%
\pgfpathlineto{\pgfqpoint{1.330943in}{4.260980in}}%
\pgfpathquadraticcurveto{\pgfqpoint{1.307804in}{4.260980in}}{\pgfqpoint{1.307804in}{4.237841in}}%
\pgfpathlineto{\pgfqpoint{1.307804in}{4.079597in}}%
\pgfpathquadraticcurveto{\pgfqpoint{1.307804in}{4.056458in}}{\pgfqpoint{1.330943in}{4.056458in}}%
\pgfpathclose%
\pgfusepath{stroke,fill}%
\end{pgfscope}%
\begin{pgfscope}%
\pgfsetbuttcap%
\pgfsetmiterjoin%
\definecolor{currentfill}{rgb}{0.898885,0.305498,0.206767}%
\pgfsetfillcolor{currentfill}%
\pgfsetfillopacity{0.500000}%
\pgfsetlinewidth{0.000000pt}%
\definecolor{currentstroke}{rgb}{0.000000,0.000000,0.000000}%
\pgfsetstrokecolor{currentstroke}%
\pgfsetstrokeopacity{0.500000}%
\pgfsetdash{}{0pt}%
\pgfpathmoveto{\pgfqpoint{1.354081in}{4.126801in}}%
\pgfpathlineto{\pgfqpoint{1.585470in}{4.126801in}}%
\pgfpathlineto{\pgfqpoint{1.585470in}{4.207787in}}%
\pgfpathlineto{\pgfqpoint{1.354081in}{4.207787in}}%
\pgfpathclose%
\pgfusepath{fill}%
\end{pgfscope}%
\begin{pgfscope}%
\definecolor{textcolor}{rgb}{0.000000,0.000000,0.000000}%
\pgfsetstrokecolor{textcolor}%
\pgfsetfillcolor{textcolor}%
\pgftext[x=1.678026in,y=4.126801in,left,base]{\color{textcolor}\sffamily\fontsize{8.330000}{9.996000}\selectfont 1}%
\end{pgfscope}%
\begin{pgfscope}%
\pgfsetbuttcap%
\pgfsetmiterjoin%
\definecolor{currentfill}{rgb}{0.966551,0.497424,0.295040}%
\pgfsetfillcolor{currentfill}%
\pgfsetfillopacity{0.500000}%
\pgfsetlinewidth{0.000000pt}%
\definecolor{currentstroke}{rgb}{0.000000,0.000000,0.000000}%
\pgfsetstrokecolor{currentstroke}%
\pgfsetstrokeopacity{0.500000}%
\pgfsetdash{}{0pt}%
\pgfpathmoveto{\pgfqpoint{1.983023in}{4.126801in}}%
\pgfpathlineto{\pgfqpoint{2.214412in}{4.126801in}}%
\pgfpathlineto{\pgfqpoint{2.214412in}{4.207787in}}%
\pgfpathlineto{\pgfqpoint{1.983023in}{4.207787in}}%
\pgfpathclose%
\pgfusepath{fill}%
\end{pgfscope}%
\begin{pgfscope}%
\definecolor{textcolor}{rgb}{0.000000,0.000000,0.000000}%
\pgfsetstrokecolor{textcolor}%
\pgfsetfillcolor{textcolor}%
\pgftext[x=2.306968in,y=4.126801in,left,base]{\color{textcolor}\sffamily\fontsize{8.330000}{9.996000}\selectfont 2}%
\end{pgfscope}%
\begin{pgfscope}%
\pgfsetbuttcap%
\pgfsetmiterjoin%
\definecolor{currentfill}{rgb}{0.992388,0.693887,0.390081}%
\pgfsetfillcolor{currentfill}%
\pgfsetfillopacity{0.500000}%
\pgfsetlinewidth{0.000000pt}%
\definecolor{currentstroke}{rgb}{0.000000,0.000000,0.000000}%
\pgfsetstrokecolor{currentstroke}%
\pgfsetstrokeopacity{0.500000}%
\pgfsetdash{}{0pt}%
\pgfpathmoveto{\pgfqpoint{2.611965in}{4.126801in}}%
\pgfpathlineto{\pgfqpoint{2.843354in}{4.126801in}}%
\pgfpathlineto{\pgfqpoint{2.843354in}{4.207787in}}%
\pgfpathlineto{\pgfqpoint{2.611965in}{4.207787in}}%
\pgfpathclose%
\pgfusepath{fill}%
\end{pgfscope}%
\begin{pgfscope}%
\definecolor{textcolor}{rgb}{0.000000,0.000000,0.000000}%
\pgfsetstrokecolor{textcolor}%
\pgfsetfillcolor{textcolor}%
\pgftext[x=2.935909in,y=4.126801in,left,base]{\color{textcolor}\sffamily\fontsize{8.330000}{9.996000}\selectfont 3}%
\end{pgfscope}%
\begin{pgfscope}%
\pgfsetbuttcap%
\pgfsetmiterjoin%
\definecolor{currentfill}{rgb}{0.995463,0.847674,0.519262}%
\pgfsetfillcolor{currentfill}%
\pgfsetfillopacity{0.500000}%
\pgfsetlinewidth{0.000000pt}%
\definecolor{currentstroke}{rgb}{0.000000,0.000000,0.000000}%
\pgfsetstrokecolor{currentstroke}%
\pgfsetstrokeopacity{0.500000}%
\pgfsetdash{}{0pt}%
\pgfpathmoveto{\pgfqpoint{3.240906in}{4.126801in}}%
\pgfpathlineto{\pgfqpoint{3.472295in}{4.126801in}}%
\pgfpathlineto{\pgfqpoint{3.472295in}{4.207787in}}%
\pgfpathlineto{\pgfqpoint{3.240906in}{4.207787in}}%
\pgfpathclose%
\pgfusepath{fill}%
\end{pgfscope}%
\begin{pgfscope}%
\definecolor{textcolor}{rgb}{0.000000,0.000000,0.000000}%
\pgfsetstrokecolor{textcolor}%
\pgfsetfillcolor{textcolor}%
\pgftext[x=3.564851in,y=4.126801in,left,base]{\color{textcolor}\sffamily\fontsize{8.330000}{9.996000}\selectfont 4}%
\end{pgfscope}%
\begin{pgfscope}%
\pgfsetbuttcap%
\pgfsetmiterjoin%
\definecolor{currentfill}{rgb}{0.998539,0.954710,0.673049}%
\pgfsetfillcolor{currentfill}%
\pgfsetfillopacity{0.500000}%
\pgfsetlinewidth{0.000000pt}%
\definecolor{currentstroke}{rgb}{0.000000,0.000000,0.000000}%
\pgfsetstrokecolor{currentstroke}%
\pgfsetstrokeopacity{0.500000}%
\pgfsetdash{}{0pt}%
\pgfpathmoveto{\pgfqpoint{3.869848in}{4.126801in}}%
\pgfpathlineto{\pgfqpoint{4.101237in}{4.126801in}}%
\pgfpathlineto{\pgfqpoint{4.101237in}{4.207787in}}%
\pgfpathlineto{\pgfqpoint{3.869848in}{4.207787in}}%
\pgfpathclose%
\pgfusepath{fill}%
\end{pgfscope}%
\begin{pgfscope}%
\definecolor{textcolor}{rgb}{0.000000,0.000000,0.000000}%
\pgfsetstrokecolor{textcolor}%
\pgfsetfillcolor{textcolor}%
\pgftext[x=4.193792in,y=4.126801in,left,base]{\color{textcolor}\sffamily\fontsize{8.330000}{9.996000}\selectfont 5}%
\end{pgfscope}%
\begin{pgfscope}%
\pgfsetbuttcap%
\pgfsetmiterjoin%
\definecolor{currentfill}{rgb}{0.944483,0.976624,0.673049}%
\pgfsetfillcolor{currentfill}%
\pgfsetfillopacity{0.500000}%
\pgfsetlinewidth{0.000000pt}%
\definecolor{currentstroke}{rgb}{0.000000,0.000000,0.000000}%
\pgfsetstrokecolor{currentstroke}%
\pgfsetstrokeopacity{0.500000}%
\pgfsetdash{}{0pt}%
\pgfpathmoveto{\pgfqpoint{4.498790in}{4.126801in}}%
\pgfpathlineto{\pgfqpoint{4.730179in}{4.126801in}}%
\pgfpathlineto{\pgfqpoint{4.730179in}{4.207787in}}%
\pgfpathlineto{\pgfqpoint{4.498790in}{4.207787in}}%
\pgfpathclose%
\pgfusepath{fill}%
\end{pgfscope}%
\begin{pgfscope}%
\definecolor{textcolor}{rgb}{0.000000,0.000000,0.000000}%
\pgfsetstrokecolor{textcolor}%
\pgfsetfillcolor{textcolor}%
\pgftext[x=4.822734in,y=4.126801in,left,base]{\color{textcolor}\sffamily\fontsize{8.330000}{9.996000}\selectfont 6}%
\end{pgfscope}%
\begin{pgfscope}%
\pgfsetbuttcap%
\pgfsetmiterjoin%
\definecolor{currentfill}{rgb}{0.819608,0.923722,0.524798}%
\pgfsetfillcolor{currentfill}%
\pgfsetfillopacity{0.500000}%
\pgfsetlinewidth{0.000000pt}%
\definecolor{currentstroke}{rgb}{0.000000,0.000000,0.000000}%
\pgfsetstrokecolor{currentstroke}%
\pgfsetstrokeopacity{0.500000}%
\pgfsetdash{}{0pt}%
\pgfpathmoveto{\pgfqpoint{5.127731in}{4.126801in}}%
\pgfpathlineto{\pgfqpoint{5.359120in}{4.126801in}}%
\pgfpathlineto{\pgfqpoint{5.359120in}{4.207787in}}%
\pgfpathlineto{\pgfqpoint{5.127731in}{4.207787in}}%
\pgfpathclose%
\pgfusepath{fill}%
\end{pgfscope}%
\begin{pgfscope}%
\definecolor{textcolor}{rgb}{0.000000,0.000000,0.000000}%
\pgfsetstrokecolor{textcolor}%
\pgfsetfillcolor{textcolor}%
\pgftext[x=5.451676in,y=4.126801in,left,base]{\color{textcolor}\sffamily\fontsize{8.330000}{9.996000}\selectfont 7}%
\end{pgfscope}%
\begin{pgfscope}%
\pgfsetbuttcap%
\pgfsetmiterjoin%
\definecolor{currentfill}{rgb}{0.662745,0.856055,0.423299}%
\pgfsetfillcolor{currentfill}%
\pgfsetfillopacity{0.500000}%
\pgfsetlinewidth{0.000000pt}%
\definecolor{currentstroke}{rgb}{0.000000,0.000000,0.000000}%
\pgfsetstrokecolor{currentstroke}%
\pgfsetstrokeopacity{0.500000}%
\pgfsetdash{}{0pt}%
\pgfpathmoveto{\pgfqpoint{5.756673in}{4.126801in}}%
\pgfpathlineto{\pgfqpoint{5.988062in}{4.126801in}}%
\pgfpathlineto{\pgfqpoint{5.988062in}{4.207787in}}%
\pgfpathlineto{\pgfqpoint{5.756673in}{4.207787in}}%
\pgfpathclose%
\pgfusepath{fill}%
\end{pgfscope}%
\begin{pgfscope}%
\definecolor{textcolor}{rgb}{0.000000,0.000000,0.000000}%
\pgfsetstrokecolor{textcolor}%
\pgfsetfillcolor{textcolor}%
\pgftext[x=6.080617in,y=4.126801in,left,base]{\color{textcolor}\sffamily\fontsize{8.330000}{9.996000}\selectfont 8}%
\end{pgfscope}%
\begin{pgfscope}%
\pgfsetbuttcap%
\pgfsetmiterjoin%
\definecolor{currentfill}{rgb}{0.468897,0.771319,0.395771}%
\pgfsetfillcolor{currentfill}%
\pgfsetfillopacity{0.500000}%
\pgfsetlinewidth{0.000000pt}%
\definecolor{currentstroke}{rgb}{0.000000,0.000000,0.000000}%
\pgfsetstrokecolor{currentstroke}%
\pgfsetstrokeopacity{0.500000}%
\pgfsetdash{}{0pt}%
\pgfpathmoveto{\pgfqpoint{6.385615in}{4.126801in}}%
\pgfpathlineto{\pgfqpoint{6.617004in}{4.126801in}}%
\pgfpathlineto{\pgfqpoint{6.617004in}{4.207787in}}%
\pgfpathlineto{\pgfqpoint{6.385615in}{4.207787in}}%
\pgfpathclose%
\pgfusepath{fill}%
\end{pgfscope}%
\begin{pgfscope}%
\definecolor{textcolor}{rgb}{0.000000,0.000000,0.000000}%
\pgfsetstrokecolor{textcolor}%
\pgfsetfillcolor{textcolor}%
\pgftext[x=6.709559in,y=4.126801in,left,base]{\color{textcolor}\sffamily\fontsize{8.330000}{9.996000}\selectfont 9}%
\end{pgfscope}%
\begin{pgfscope}%
\pgfsetbuttcap%
\pgfsetmiterjoin%
\definecolor{currentfill}{rgb}{0.248058,0.667205,0.350250}%
\pgfsetfillcolor{currentfill}%
\pgfsetfillopacity{0.500000}%
\pgfsetlinewidth{0.000000pt}%
\definecolor{currentstroke}{rgb}{0.000000,0.000000,0.000000}%
\pgfsetstrokecolor{currentstroke}%
\pgfsetstrokeopacity{0.500000}%
\pgfsetdash{}{0pt}%
\pgfpathmoveto{\pgfqpoint{7.014556in}{4.126801in}}%
\pgfpathlineto{\pgfqpoint{7.245945in}{4.126801in}}%
\pgfpathlineto{\pgfqpoint{7.245945in}{4.207787in}}%
\pgfpathlineto{\pgfqpoint{7.014556in}{4.207787in}}%
\pgfpathclose%
\pgfusepath{fill}%
\end{pgfscope}%
\begin{pgfscope}%
\definecolor{textcolor}{rgb}{0.000000,0.000000,0.000000}%
\pgfsetstrokecolor{textcolor}%
\pgfsetfillcolor{textcolor}%
\pgftext[x=7.338501in,y=4.126801in,left,base]{\color{textcolor}\sffamily\fontsize{8.330000}{9.996000}\selectfont 10}%
\end{pgfscope}%
\end{pgfpicture}%
\makeatother%
\endgroup%
}
    \caption[Distibution for Survey:Section 2]{The figure represents distribution for Section 2 of survey. The participants were asked to rate the image based on photorealism(1 being the least photorealistic).
    \gls{free} has maximum number of least rating(1), but it is comparable to other automated datasets as highlighted. Pix3D has maximum number of highest rating(10).}
    \label{fig:question2}
\end{figure}

\begin{figure}
    \centering
    \resizebox{0.75\textwidth}{10cm}{%% Creator: Matplotlib, PGF backend
%%
%% To include the figure in your LaTeX document, write
%%   \input{<filename>.pgf}
%%
%% Make sure the required packages are loaded in your preamble
%%   \usepackage{pgf}
%%
%% Figures using additional raster images can only be included by \input if
%% they are in the same directory as the main LaTeX file. For loading figures
%% from other directories you can use the `import` package
%%   \usepackage{import}
%%
%% and then include the figures with
%%   \import{<path to file>}{<filename>.pgf}
%%
%% Matplotlib used the following preamble
%%   \usepackage{fontspec}
%%   \setmainfont{DejaVuSerif.ttf}[Path=\detokenize{/Users/apple/opt/anaconda3/envs/kaolin/lib/python3.7/site-packages/matplotlib/mpl-data/fonts/ttf/}]
%%   \setsansfont{DejaVuSans.ttf}[Path=\detokenize{/Users/apple/opt/anaconda3/envs/kaolin/lib/python3.7/site-packages/matplotlib/mpl-data/fonts/ttf/}]
%%   \setmonofont{DejaVuSansMono.ttf}[Path=\detokenize{/Users/apple/opt/anaconda3/envs/kaolin/lib/python3.7/site-packages/matplotlib/mpl-data/fonts/ttf/}]
%%
\begingroup%
\makeatletter%
\begin{pgfpicture}%
\pgfpathrectangle{\pgfpointorigin}{\pgfqpoint{5.535556in}{4.888302in}}%
\pgfusepath{use as bounding box, clip}%
\begin{pgfscope}%
\pgfsetbuttcap%
\pgfsetmiterjoin%
\definecolor{currentfill}{rgb}{1.000000,1.000000,1.000000}%
\pgfsetfillcolor{currentfill}%
\pgfsetlinewidth{0.000000pt}%
\definecolor{currentstroke}{rgb}{1.000000,1.000000,1.000000}%
\pgfsetstrokecolor{currentstroke}%
\pgfsetdash{}{0pt}%
\pgfpathmoveto{\pgfqpoint{0.000000in}{0.000000in}}%
\pgfpathlineto{\pgfqpoint{5.535556in}{0.000000in}}%
\pgfpathlineto{\pgfqpoint{5.535556in}{4.888302in}}%
\pgfpathlineto{\pgfqpoint{0.000000in}{4.888302in}}%
\pgfpathclose%
\pgfusepath{fill}%
\end{pgfscope}%
\begin{pgfscope}%
\pgfsetbuttcap%
\pgfsetmiterjoin%
\definecolor{currentfill}{rgb}{1.000000,1.000000,1.000000}%
\pgfsetfillcolor{currentfill}%
\pgfsetlinewidth{0.000000pt}%
\definecolor{currentstroke}{rgb}{0.000000,0.000000,0.000000}%
\pgfsetstrokecolor{currentstroke}%
\pgfsetstrokeopacity{0.000000}%
\pgfsetdash{}{0pt}%
\pgfpathmoveto{\pgfqpoint{0.475556in}{1.092302in}}%
\pgfpathlineto{\pgfqpoint{5.435556in}{1.092302in}}%
\pgfpathlineto{\pgfqpoint{5.435556in}{4.788302in}}%
\pgfpathlineto{\pgfqpoint{0.475556in}{4.788302in}}%
\pgfpathclose%
\pgfusepath{fill}%
\end{pgfscope}%
\begin{pgfscope}%
\pgfpathrectangle{\pgfqpoint{0.475556in}{1.092302in}}{\pgfqpoint{4.960000in}{3.696000in}}%
\pgfusepath{clip}%
\pgfsetbuttcap%
\pgfsetmiterjoin%
\definecolor{currentfill}{rgb}{0.121569,0.466667,0.705882}%
\pgfsetfillcolor{currentfill}%
\pgfsetfillopacity{0.500000}%
\pgfsetlinewidth{0.000000pt}%
\definecolor{currentstroke}{rgb}{0.000000,0.000000,0.000000}%
\pgfsetstrokecolor{currentstroke}%
\pgfsetstrokeopacity{0.500000}%
\pgfsetdash{}{0pt}%
\pgfpathmoveto{\pgfqpoint{0.701010in}{1.092302in}}%
\pgfpathlineto{\pgfqpoint{1.110928in}{1.092302in}}%
\pgfpathlineto{\pgfqpoint{1.110928in}{4.612302in}}%
\pgfpathlineto{\pgfqpoint{0.701010in}{4.612302in}}%
\pgfpathclose%
\pgfusepath{fill}%
\end{pgfscope}%
\begin{pgfscope}%
\pgfpathrectangle{\pgfqpoint{0.475556in}{1.092302in}}{\pgfqpoint{4.960000in}{3.696000in}}%
\pgfusepath{clip}%
\pgfsetbuttcap%
\pgfsetmiterjoin%
\definecolor{currentfill}{rgb}{0.121569,0.466667,0.705882}%
\pgfsetfillcolor{currentfill}%
\pgfsetfillopacity{0.500000}%
\pgfsetlinewidth{0.000000pt}%
\definecolor{currentstroke}{rgb}{0.000000,0.000000,0.000000}%
\pgfsetstrokecolor{currentstroke}%
\pgfsetstrokeopacity{0.500000}%
\pgfsetdash{}{0pt}%
\pgfpathmoveto{\pgfqpoint{1.213407in}{1.092302in}}%
\pgfpathlineto{\pgfqpoint{1.623324in}{1.092302in}}%
\pgfpathlineto{\pgfqpoint{1.623324in}{4.551370in}}%
\pgfpathlineto{\pgfqpoint{1.213407in}{4.551370in}}%
\pgfpathclose%
\pgfusepath{fill}%
\end{pgfscope}%
\begin{pgfscope}%
\pgfpathrectangle{\pgfqpoint{0.475556in}{1.092302in}}{\pgfqpoint{4.960000in}{3.696000in}}%
\pgfusepath{clip}%
\pgfsetbuttcap%
\pgfsetmiterjoin%
\definecolor{currentfill}{rgb}{0.121569,0.466667,0.705882}%
\pgfsetfillcolor{currentfill}%
\pgfsetfillopacity{0.500000}%
\pgfsetlinewidth{0.000000pt}%
\definecolor{currentstroke}{rgb}{0.000000,0.000000,0.000000}%
\pgfsetstrokecolor{currentstroke}%
\pgfsetstrokeopacity{0.500000}%
\pgfsetdash{}{0pt}%
\pgfpathmoveto{\pgfqpoint{1.725804in}{1.092302in}}%
\pgfpathlineto{\pgfqpoint{2.135721in}{1.092302in}}%
\pgfpathlineto{\pgfqpoint{2.135721in}{3.752222in}}%
\pgfpathlineto{\pgfqpoint{1.725804in}{3.752222in}}%
\pgfpathclose%
\pgfusepath{fill}%
\end{pgfscope}%
\begin{pgfscope}%
\pgfpathrectangle{\pgfqpoint{0.475556in}{1.092302in}}{\pgfqpoint{4.960000in}{3.696000in}}%
\pgfusepath{clip}%
\pgfsetbuttcap%
\pgfsetmiterjoin%
\definecolor{currentfill}{rgb}{0.121569,0.466667,0.705882}%
\pgfsetfillcolor{currentfill}%
\pgfsetfillopacity{0.500000}%
\pgfsetlinewidth{0.000000pt}%
\definecolor{currentstroke}{rgb}{0.000000,0.000000,0.000000}%
\pgfsetstrokecolor{currentstroke}%
\pgfsetstrokeopacity{0.500000}%
\pgfsetdash{}{0pt}%
\pgfpathmoveto{\pgfqpoint{2.238200in}{1.092302in}}%
\pgfpathlineto{\pgfqpoint{2.648118in}{1.092302in}}%
\pgfpathlineto{\pgfqpoint{2.648118in}{3.442875in}}%
\pgfpathlineto{\pgfqpoint{2.238200in}{3.442875in}}%
\pgfpathclose%
\pgfusepath{fill}%
\end{pgfscope}%
\begin{pgfscope}%
\pgfpathrectangle{\pgfqpoint{0.475556in}{1.092302in}}{\pgfqpoint{4.960000in}{3.696000in}}%
\pgfusepath{clip}%
\pgfsetbuttcap%
\pgfsetmiterjoin%
\definecolor{currentfill}{rgb}{0.121569,0.466667,0.705882}%
\pgfsetfillcolor{currentfill}%
\pgfsetlinewidth{0.000000pt}%
\definecolor{currentstroke}{rgb}{0.000000,0.000000,0.000000}%
\pgfsetstrokecolor{currentstroke}%
\pgfsetstrokeopacity{0.000000}%
\pgfsetdash{}{0pt}%
\pgfpathmoveto{\pgfqpoint{2.750597in}{1.092302in}}%
\pgfpathlineto{\pgfqpoint{3.160515in}{1.092302in}}%
\pgfpathlineto{\pgfqpoint{3.160515in}{3.440531in}}%
\pgfpathlineto{\pgfqpoint{2.750597in}{3.440531in}}%
\pgfpathclose%
\pgfusepath{fill}%
\end{pgfscope}%
\begin{pgfscope}%
\pgfpathrectangle{\pgfqpoint{0.475556in}{1.092302in}}{\pgfqpoint{4.960000in}{3.696000in}}%
\pgfusepath{clip}%
\pgfsetbuttcap%
\pgfsetmiterjoin%
\definecolor{currentfill}{rgb}{0.121569,0.466667,0.705882}%
\pgfsetfillcolor{currentfill}%
\pgfsetlinewidth{0.000000pt}%
\definecolor{currentstroke}{rgb}{0.000000,0.000000,0.000000}%
\pgfsetstrokecolor{currentstroke}%
\pgfsetstrokeopacity{0.000000}%
\pgfsetdash{}{0pt}%
\pgfpathmoveto{\pgfqpoint{3.262994in}{1.092302in}}%
\pgfpathlineto{\pgfqpoint{3.672911in}{1.092302in}}%
\pgfpathlineto{\pgfqpoint{3.672911in}{3.262422in}}%
\pgfpathlineto{\pgfqpoint{3.262994in}{3.262422in}}%
\pgfpathclose%
\pgfusepath{fill}%
\end{pgfscope}%
\begin{pgfscope}%
\pgfpathrectangle{\pgfqpoint{0.475556in}{1.092302in}}{\pgfqpoint{4.960000in}{3.696000in}}%
\pgfusepath{clip}%
\pgfsetbuttcap%
\pgfsetmiterjoin%
\definecolor{currentfill}{rgb}{0.121569,0.466667,0.705882}%
\pgfsetfillcolor{currentfill}%
\pgfsetfillopacity{0.500000}%
\pgfsetlinewidth{0.000000pt}%
\definecolor{currentstroke}{rgb}{0.000000,0.000000,0.000000}%
\pgfsetstrokecolor{currentstroke}%
\pgfsetstrokeopacity{0.500000}%
\pgfsetdash{}{0pt}%
\pgfpathmoveto{\pgfqpoint{3.775391in}{1.092302in}}%
\pgfpathlineto{\pgfqpoint{4.185308in}{1.092302in}}%
\pgfpathlineto{\pgfqpoint{4.185308in}{2.908547in}}%
\pgfpathlineto{\pgfqpoint{3.775391in}{2.908547in}}%
\pgfpathclose%
\pgfusepath{fill}%
\end{pgfscope}%
\begin{pgfscope}%
\pgfpathrectangle{\pgfqpoint{0.475556in}{1.092302in}}{\pgfqpoint{4.960000in}{3.696000in}}%
\pgfusepath{clip}%
\pgfsetbuttcap%
\pgfsetmiterjoin%
\definecolor{currentfill}{rgb}{0.121569,0.466667,0.705882}%
\pgfsetfillcolor{currentfill}%
\pgfsetlinewidth{0.000000pt}%
\definecolor{currentstroke}{rgb}{0.000000,0.000000,0.000000}%
\pgfsetstrokecolor{currentstroke}%
\pgfsetstrokeopacity{0.000000}%
\pgfsetdash{}{0pt}%
\pgfpathmoveto{\pgfqpoint{4.287787in}{1.092302in}}%
\pgfpathlineto{\pgfqpoint{4.697705in}{1.092302in}}%
\pgfpathlineto{\pgfqpoint{4.697705in}{2.789027in}}%
\pgfpathlineto{\pgfqpoint{4.287787in}{2.789027in}}%
\pgfpathclose%
\pgfusepath{fill}%
\end{pgfscope}%
\begin{pgfscope}%
\pgfpathrectangle{\pgfqpoint{0.475556in}{1.092302in}}{\pgfqpoint{4.960000in}{3.696000in}}%
\pgfusepath{clip}%
\pgfsetbuttcap%
\pgfsetmiterjoin%
\definecolor{currentfill}{rgb}{0.121569,0.466667,0.705882}%
\pgfsetfillcolor{currentfill}%
\pgfsetlinewidth{0.000000pt}%
\definecolor{currentstroke}{rgb}{0.000000,0.000000,0.000000}%
\pgfsetstrokecolor{currentstroke}%
\pgfsetstrokeopacity{0.000000}%
\pgfsetdash{}{0pt}%
\pgfpathmoveto{\pgfqpoint{4.800184in}{1.092302in}}%
\pgfpathlineto{\pgfqpoint{5.210101in}{1.092302in}}%
\pgfpathlineto{\pgfqpoint{5.210101in}{2.676537in}}%
\pgfpathlineto{\pgfqpoint{4.800184in}{2.676537in}}%
\pgfpathclose%
\pgfusepath{fill}%
\end{pgfscope}%
\begin{pgfscope}%
\pgfsetbuttcap%
\pgfsetroundjoin%
\definecolor{currentfill}{rgb}{0.000000,0.000000,0.000000}%
\pgfsetfillcolor{currentfill}%
\pgfsetlinewidth{0.803000pt}%
\definecolor{currentstroke}{rgb}{0.000000,0.000000,0.000000}%
\pgfsetstrokecolor{currentstroke}%
\pgfsetdash{}{0pt}%
\pgfsys@defobject{currentmarker}{\pgfqpoint{0.000000in}{-0.048611in}}{\pgfqpoint{0.000000in}{0.000000in}}{%
\pgfpathmoveto{\pgfqpoint{0.000000in}{0.000000in}}%
\pgfpathlineto{\pgfqpoint{0.000000in}{-0.048611in}}%
\pgfusepath{stroke,fill}%
}%
\begin{pgfscope}%
\pgfsys@transformshift{0.905969in}{1.092302in}%
\pgfsys@useobject{currentmarker}{}%
\end{pgfscope}%
\end{pgfscope}%
\begin{pgfscope}%
\definecolor{textcolor}{rgb}{0.000000,0.000000,0.000000}%
\pgfsetstrokecolor{textcolor}%
\pgfsetfillcolor{textcolor}%
\pgftext[x=0.792774in, y=0.639887in, left, base,rotate=45.000000]{\color{textcolor}\sffamily\fontsize{10.000000}{12.000000}\selectfont Pix3D}%
\end{pgfscope}%
\begin{pgfscope}%
\pgfsetbuttcap%
\pgfsetroundjoin%
\definecolor{currentfill}{rgb}{0.000000,0.000000,0.000000}%
\pgfsetfillcolor{currentfill}%
\pgfsetlinewidth{0.803000pt}%
\definecolor{currentstroke}{rgb}{0.000000,0.000000,0.000000}%
\pgfsetstrokecolor{currentstroke}%
\pgfsetdash{}{0pt}%
\pgfsys@defobject{currentmarker}{\pgfqpoint{0.000000in}{-0.048611in}}{\pgfqpoint{0.000000in}{0.000000in}}{%
\pgfpathmoveto{\pgfqpoint{0.000000in}{0.000000in}}%
\pgfpathlineto{\pgfqpoint{0.000000in}{-0.048611in}}%
\pgfusepath{stroke,fill}%
}%
\begin{pgfscope}%
\pgfsys@transformshift{1.418366in}{1.092302in}%
\pgfsys@useobject{currentmarker}{}%
\end{pgfscope}%
\end{pgfscope}%
\begin{pgfscope}%
\definecolor{textcolor}{rgb}{0.000000,0.000000,0.000000}%
\pgfsetstrokecolor{textcolor}%
\pgfsetfillcolor{textcolor}%
\pgftext[x=1.180132in, y=0.389808in, left, base,rotate=45.000000]{\color{textcolor}\sffamily\fontsize{10.000000}{12.000000}\selectfont InteriorNet}%
\end{pgfscope}%
\begin{pgfscope}%
\pgfsetbuttcap%
\pgfsetroundjoin%
\definecolor{currentfill}{rgb}{0.000000,0.000000,0.000000}%
\pgfsetfillcolor{currentfill}%
\pgfsetlinewidth{0.803000pt}%
\definecolor{currentstroke}{rgb}{0.000000,0.000000,0.000000}%
\pgfsetstrokecolor{currentstroke}%
\pgfsetdash{}{0pt}%
\pgfsys@defobject{currentmarker}{\pgfqpoint{0.000000in}{-0.048611in}}{\pgfqpoint{0.000000in}{0.000000in}}{%
\pgfpathmoveto{\pgfqpoint{0.000000in}{0.000000in}}%
\pgfpathlineto{\pgfqpoint{0.000000in}{-0.048611in}}%
\pgfusepath{stroke,fill}%
}%
\begin{pgfscope}%
\pgfsys@transformshift{1.930762in}{1.092302in}%
\pgfsys@useobject{currentmarker}{}%
\end{pgfscope}%
\end{pgfscope}%
\begin{pgfscope}%
\definecolor{textcolor}{rgb}{0.000000,0.000000,0.000000}%
\pgfsetstrokecolor{textcolor}%
\pgfsetfillcolor{textcolor}%
\pgftext[x=1.723243in, y=0.451237in, left, base,rotate=45.000000]{\color{textcolor}\sffamily\fontsize{10.000000}{12.000000}\selectfont Hyperism}%
\end{pgfscope}%
\begin{pgfscope}%
\pgfsetbuttcap%
\pgfsetroundjoin%
\definecolor{currentfill}{rgb}{0.000000,0.000000,0.000000}%
\pgfsetfillcolor{currentfill}%
\pgfsetlinewidth{0.803000pt}%
\definecolor{currentstroke}{rgb}{0.000000,0.000000,0.000000}%
\pgfsetstrokecolor{currentstroke}%
\pgfsetdash{}{0pt}%
\pgfsys@defobject{currentmarker}{\pgfqpoint{0.000000in}{-0.048611in}}{\pgfqpoint{0.000000in}{0.000000in}}{%
\pgfpathmoveto{\pgfqpoint{0.000000in}{0.000000in}}%
\pgfpathlineto{\pgfqpoint{0.000000in}{-0.048611in}}%
\pgfusepath{stroke,fill}%
}%
\begin{pgfscope}%
\pgfsys@transformshift{2.443159in}{1.092302in}%
\pgfsys@useobject{currentmarker}{}%
\end{pgfscope}%
\end{pgfscope}%
\begin{pgfscope}%
\definecolor{textcolor}{rgb}{0.000000,0.000000,0.000000}%
\pgfsetstrokecolor{textcolor}%
\pgfsetfillcolor{textcolor}%
\pgftext[x=2.233457in, y=0.446873in, left, base,rotate=45.000000]{\color{textcolor}\sffamily\fontsize{10.000000}{12.000000}\selectfont 3DFRONT}%
\end{pgfscope}%
\begin{pgfscope}%
\pgfsetbuttcap%
\pgfsetroundjoin%
\definecolor{currentfill}{rgb}{0.000000,0.000000,0.000000}%
\pgfsetfillcolor{currentfill}%
\pgfsetlinewidth{0.803000pt}%
\definecolor{currentstroke}{rgb}{0.000000,0.000000,0.000000}%
\pgfsetstrokecolor{currentstroke}%
\pgfsetdash{}{0pt}%
\pgfsys@defobject{currentmarker}{\pgfqpoint{0.000000in}{-0.048611in}}{\pgfqpoint{0.000000in}{0.000000in}}{%
\pgfpathmoveto{\pgfqpoint{0.000000in}{0.000000in}}%
\pgfpathlineto{\pgfqpoint{0.000000in}{-0.048611in}}%
\pgfusepath{stroke,fill}%
}%
\begin{pgfscope}%
\pgfsys@transformshift{2.955556in}{1.092302in}%
\pgfsys@useobject{currentmarker}{}%
\end{pgfscope}%
\end{pgfscope}%
\begin{pgfscope}%
\definecolor{textcolor}{rgb}{0.000000,0.000000,0.000000}%
\pgfsetstrokecolor{textcolor}%
\pgfsetfillcolor{textcolor}%
\pgftext[x=2.746741in, y=0.448647in, left, base,rotate=45.000000]{\color{textcolor}\sffamily\fontsize{10.000000}{12.000000}\selectfont SceneNet}%
\end{pgfscope}%
\begin{pgfscope}%
\pgfsetbuttcap%
\pgfsetroundjoin%
\definecolor{currentfill}{rgb}{0.000000,0.000000,0.000000}%
\pgfsetfillcolor{currentfill}%
\pgfsetlinewidth{0.803000pt}%
\definecolor{currentstroke}{rgb}{0.000000,0.000000,0.000000}%
\pgfsetstrokecolor{currentstroke}%
\pgfsetdash{}{0pt}%
\pgfsys@defobject{currentmarker}{\pgfqpoint{0.000000in}{-0.048611in}}{\pgfqpoint{0.000000in}{0.000000in}}{%
\pgfpathmoveto{\pgfqpoint{0.000000in}{0.000000in}}%
\pgfpathlineto{\pgfqpoint{0.000000in}{-0.048611in}}%
\pgfusepath{stroke,fill}%
}%
\begin{pgfscope}%
\pgfsys@transformshift{3.467953in}{1.092302in}%
\pgfsys@useobject{currentmarker}{}%
\end{pgfscope}%
\end{pgfscope}%
\begin{pgfscope}%
\definecolor{textcolor}{rgb}{0.000000,0.000000,0.000000}%
\pgfsetstrokecolor{textcolor}%
\pgfsetfillcolor{textcolor}%
\pgftext[x=3.197494in, y=0.325358in, left, base,rotate=45.000000]{\color{textcolor}\sffamily\fontsize{10.000000}{12.000000}\selectfont Blenderproc}%
\end{pgfscope}%
\begin{pgfscope}%
\pgfsetbuttcap%
\pgfsetroundjoin%
\definecolor{currentfill}{rgb}{0.000000,0.000000,0.000000}%
\pgfsetfillcolor{currentfill}%
\pgfsetlinewidth{0.803000pt}%
\definecolor{currentstroke}{rgb}{0.000000,0.000000,0.000000}%
\pgfsetstrokecolor{currentstroke}%
\pgfsetdash{}{0pt}%
\pgfsys@defobject{currentmarker}{\pgfqpoint{0.000000in}{-0.048611in}}{\pgfqpoint{0.000000in}{0.000000in}}{%
\pgfpathmoveto{\pgfqpoint{0.000000in}{0.000000in}}%
\pgfpathlineto{\pgfqpoint{0.000000in}{-0.048611in}}%
\pgfusepath{stroke,fill}%
}%
\begin{pgfscope}%
\pgfsys@transformshift{3.980349in}{1.092302in}%
\pgfsys@useobject{currentmarker}{}%
\end{pgfscope}%
\end{pgfscope}%
\begin{pgfscope}%
\definecolor{textcolor}{rgb}{0.000000,0.000000,0.000000}%
\pgfsetstrokecolor{textcolor}%
\pgfsetfillcolor{textcolor}%
\pgftext[x=3.788438in, y=0.482454in, left, base,rotate=45.000000]{\color{textcolor}\sffamily\fontsize{10.000000}{12.000000}\selectfont AI2THOR}%
\end{pgfscope}%
\begin{pgfscope}%
\pgfsetbuttcap%
\pgfsetroundjoin%
\definecolor{currentfill}{rgb}{0.000000,0.000000,0.000000}%
\pgfsetfillcolor{currentfill}%
\pgfsetlinewidth{0.803000pt}%
\definecolor{currentstroke}{rgb}{0.000000,0.000000,0.000000}%
\pgfsetstrokecolor{currentstroke}%
\pgfsetdash{}{0pt}%
\pgfsys@defobject{currentmarker}{\pgfqpoint{0.000000in}{-0.048611in}}{\pgfqpoint{0.000000in}{0.000000in}}{%
\pgfpathmoveto{\pgfqpoint{0.000000in}{0.000000in}}%
\pgfpathlineto{\pgfqpoint{0.000000in}{-0.048611in}}%
\pgfusepath{stroke,fill}%
}%
\begin{pgfscope}%
\pgfsys@transformshift{4.492746in}{1.092302in}%
\pgfsys@useobject{currentmarker}{}%
\end{pgfscope}%
\end{pgfscope}%
\begin{pgfscope}%
\definecolor{textcolor}{rgb}{0.000000,0.000000,0.000000}%
\pgfsetstrokecolor{textcolor}%
\pgfsetfillcolor{textcolor}%
\pgftext[x=4.223270in, y=0.327324in, left, base,rotate=45.000000]{\color{textcolor}\sffamily\fontsize{10.000000}{12.000000}\selectfont OpenRooms}%
\end{pgfscope}%
\begin{pgfscope}%
\pgfsetbuttcap%
\pgfsetroundjoin%
\definecolor{currentfill}{rgb}{0.000000,0.000000,0.000000}%
\pgfsetfillcolor{currentfill}%
\pgfsetlinewidth{0.803000pt}%
\definecolor{currentstroke}{rgb}{0.000000,0.000000,0.000000}%
\pgfsetstrokecolor{currentstroke}%
\pgfsetdash{}{0pt}%
\pgfsys@defobject{currentmarker}{\pgfqpoint{0.000000in}{-0.048611in}}{\pgfqpoint{0.000000in}{0.000000in}}{%
\pgfpathmoveto{\pgfqpoint{0.000000in}{0.000000in}}%
\pgfpathlineto{\pgfqpoint{0.000000in}{-0.048611in}}%
\pgfusepath{stroke,fill}%
}%
\begin{pgfscope}%
\pgfsys@transformshift{5.005143in}{1.092302in}%
\pgfsys@useobject{currentmarker}{}%
\end{pgfscope}%
\end{pgfscope}%
\begin{pgfscope}%
\definecolor{textcolor}{rgb}{0.000000,0.000000,0.000000}%
\pgfsetstrokecolor{textcolor}%
\pgfsetfillcolor{textcolor}%
\pgftext[x=4.727203in, y=0.310396in, left, base,rotate=45.000000]{\color{textcolor}\sffamily\fontsize{10.000000}{12.000000}\selectfont S2R:3DFREE}%
\end{pgfscope}%
\begin{pgfscope}%
\definecolor{textcolor}{rgb}{0.000000,0.000000,0.000000}%
\pgfsetstrokecolor{textcolor}%
\pgfsetfillcolor{textcolor}%
\pgftext[x=2.955556in,y=0.234413in,,top]{\color{textcolor}\sffamily\fontsize{10.000000}{12.000000}\selectfont Datasets}%
\end{pgfscope}%
\begin{pgfscope}%
\pgfsetbuttcap%
\pgfsetroundjoin%
\definecolor{currentfill}{rgb}{0.000000,0.000000,0.000000}%
\pgfsetfillcolor{currentfill}%
\pgfsetlinewidth{0.803000pt}%
\definecolor{currentstroke}{rgb}{0.000000,0.000000,0.000000}%
\pgfsetstrokecolor{currentstroke}%
\pgfsetdash{}{0pt}%
\pgfsys@defobject{currentmarker}{\pgfqpoint{-0.048611in}{0.000000in}}{\pgfqpoint{-0.000000in}{0.000000in}}{%
\pgfpathmoveto{\pgfqpoint{-0.000000in}{0.000000in}}%
\pgfpathlineto{\pgfqpoint{-0.048611in}{0.000000in}}%
\pgfusepath{stroke,fill}%
}%
\begin{pgfscope}%
\pgfsys@transformshift{0.475556in}{1.092302in}%
\pgfsys@useobject{currentmarker}{}%
\end{pgfscope}%
\end{pgfscope}%
\begin{pgfscope}%
\definecolor{textcolor}{rgb}{0.000000,0.000000,0.000000}%
\pgfsetstrokecolor{textcolor}%
\pgfsetfillcolor{textcolor}%
\pgftext[x=0.289968in, y=1.039541in, left, base]{\color{textcolor}\sffamily\fontsize{10.000000}{12.000000}\selectfont 0}%
\end{pgfscope}%
\begin{pgfscope}%
\pgfsetbuttcap%
\pgfsetroundjoin%
\definecolor{currentfill}{rgb}{0.000000,0.000000,0.000000}%
\pgfsetfillcolor{currentfill}%
\pgfsetlinewidth{0.803000pt}%
\definecolor{currentstroke}{rgb}{0.000000,0.000000,0.000000}%
\pgfsetstrokecolor{currentstroke}%
\pgfsetdash{}{0pt}%
\pgfsys@defobject{currentmarker}{\pgfqpoint{-0.048611in}{0.000000in}}{\pgfqpoint{-0.000000in}{0.000000in}}{%
\pgfpathmoveto{\pgfqpoint{-0.000000in}{0.000000in}}%
\pgfpathlineto{\pgfqpoint{-0.048611in}{0.000000in}}%
\pgfusepath{stroke,fill}%
}%
\begin{pgfscope}%
\pgfsys@transformshift{0.475556in}{1.598507in}%
\pgfsys@useobject{currentmarker}{}%
\end{pgfscope}%
\end{pgfscope}%
\begin{pgfscope}%
\definecolor{textcolor}{rgb}{0.000000,0.000000,0.000000}%
\pgfsetstrokecolor{textcolor}%
\pgfsetfillcolor{textcolor}%
\pgftext[x=0.289968in, y=1.545746in, left, base]{\color{textcolor}\sffamily\fontsize{10.000000}{12.000000}\selectfont 1}%
\end{pgfscope}%
\begin{pgfscope}%
\pgfsetbuttcap%
\pgfsetroundjoin%
\definecolor{currentfill}{rgb}{0.000000,0.000000,0.000000}%
\pgfsetfillcolor{currentfill}%
\pgfsetlinewidth{0.803000pt}%
\definecolor{currentstroke}{rgb}{0.000000,0.000000,0.000000}%
\pgfsetstrokecolor{currentstroke}%
\pgfsetdash{}{0pt}%
\pgfsys@defobject{currentmarker}{\pgfqpoint{-0.048611in}{0.000000in}}{\pgfqpoint{-0.000000in}{0.000000in}}{%
\pgfpathmoveto{\pgfqpoint{-0.000000in}{0.000000in}}%
\pgfpathlineto{\pgfqpoint{-0.048611in}{0.000000in}}%
\pgfusepath{stroke,fill}%
}%
\begin{pgfscope}%
\pgfsys@transformshift{0.475556in}{2.104712in}%
\pgfsys@useobject{currentmarker}{}%
\end{pgfscope}%
\end{pgfscope}%
\begin{pgfscope}%
\definecolor{textcolor}{rgb}{0.000000,0.000000,0.000000}%
\pgfsetstrokecolor{textcolor}%
\pgfsetfillcolor{textcolor}%
\pgftext[x=0.289968in, y=2.051951in, left, base]{\color{textcolor}\sffamily\fontsize{10.000000}{12.000000}\selectfont 2}%
\end{pgfscope}%
\begin{pgfscope}%
\pgfsetbuttcap%
\pgfsetroundjoin%
\definecolor{currentfill}{rgb}{0.000000,0.000000,0.000000}%
\pgfsetfillcolor{currentfill}%
\pgfsetlinewidth{0.803000pt}%
\definecolor{currentstroke}{rgb}{0.000000,0.000000,0.000000}%
\pgfsetstrokecolor{currentstroke}%
\pgfsetdash{}{0pt}%
\pgfsys@defobject{currentmarker}{\pgfqpoint{-0.048611in}{0.000000in}}{\pgfqpoint{-0.000000in}{0.000000in}}{%
\pgfpathmoveto{\pgfqpoint{-0.000000in}{0.000000in}}%
\pgfpathlineto{\pgfqpoint{-0.048611in}{0.000000in}}%
\pgfusepath{stroke,fill}%
}%
\begin{pgfscope}%
\pgfsys@transformshift{0.475556in}{2.610917in}%
\pgfsys@useobject{currentmarker}{}%
\end{pgfscope}%
\end{pgfscope}%
\begin{pgfscope}%
\definecolor{textcolor}{rgb}{0.000000,0.000000,0.000000}%
\pgfsetstrokecolor{textcolor}%
\pgfsetfillcolor{textcolor}%
\pgftext[x=0.289968in, y=2.558156in, left, base]{\color{textcolor}\sffamily\fontsize{10.000000}{12.000000}\selectfont 3}%
\end{pgfscope}%
\begin{pgfscope}%
\pgfsetbuttcap%
\pgfsetroundjoin%
\definecolor{currentfill}{rgb}{0.000000,0.000000,0.000000}%
\pgfsetfillcolor{currentfill}%
\pgfsetlinewidth{0.803000pt}%
\definecolor{currentstroke}{rgb}{0.000000,0.000000,0.000000}%
\pgfsetstrokecolor{currentstroke}%
\pgfsetdash{}{0pt}%
\pgfsys@defobject{currentmarker}{\pgfqpoint{-0.048611in}{0.000000in}}{\pgfqpoint{-0.000000in}{0.000000in}}{%
\pgfpathmoveto{\pgfqpoint{-0.000000in}{0.000000in}}%
\pgfpathlineto{\pgfqpoint{-0.048611in}{0.000000in}}%
\pgfusepath{stroke,fill}%
}%
\begin{pgfscope}%
\pgfsys@transformshift{0.475556in}{3.117123in}%
\pgfsys@useobject{currentmarker}{}%
\end{pgfscope}%
\end{pgfscope}%
\begin{pgfscope}%
\definecolor{textcolor}{rgb}{0.000000,0.000000,0.000000}%
\pgfsetstrokecolor{textcolor}%
\pgfsetfillcolor{textcolor}%
\pgftext[x=0.289968in, y=3.064361in, left, base]{\color{textcolor}\sffamily\fontsize{10.000000}{12.000000}\selectfont 4}%
\end{pgfscope}%
\begin{pgfscope}%
\pgfsetbuttcap%
\pgfsetroundjoin%
\definecolor{currentfill}{rgb}{0.000000,0.000000,0.000000}%
\pgfsetfillcolor{currentfill}%
\pgfsetlinewidth{0.803000pt}%
\definecolor{currentstroke}{rgb}{0.000000,0.000000,0.000000}%
\pgfsetstrokecolor{currentstroke}%
\pgfsetdash{}{0pt}%
\pgfsys@defobject{currentmarker}{\pgfqpoint{-0.048611in}{0.000000in}}{\pgfqpoint{-0.000000in}{0.000000in}}{%
\pgfpathmoveto{\pgfqpoint{-0.000000in}{0.000000in}}%
\pgfpathlineto{\pgfqpoint{-0.048611in}{0.000000in}}%
\pgfusepath{stroke,fill}%
}%
\begin{pgfscope}%
\pgfsys@transformshift{0.475556in}{3.623328in}%
\pgfsys@useobject{currentmarker}{}%
\end{pgfscope}%
\end{pgfscope}%
\begin{pgfscope}%
\definecolor{textcolor}{rgb}{0.000000,0.000000,0.000000}%
\pgfsetstrokecolor{textcolor}%
\pgfsetfillcolor{textcolor}%
\pgftext[x=0.289968in, y=3.570566in, left, base]{\color{textcolor}\sffamily\fontsize{10.000000}{12.000000}\selectfont 5}%
\end{pgfscope}%
\begin{pgfscope}%
\pgfsetbuttcap%
\pgfsetroundjoin%
\definecolor{currentfill}{rgb}{0.000000,0.000000,0.000000}%
\pgfsetfillcolor{currentfill}%
\pgfsetlinewidth{0.803000pt}%
\definecolor{currentstroke}{rgb}{0.000000,0.000000,0.000000}%
\pgfsetstrokecolor{currentstroke}%
\pgfsetdash{}{0pt}%
\pgfsys@defobject{currentmarker}{\pgfqpoint{-0.048611in}{0.000000in}}{\pgfqpoint{-0.000000in}{0.000000in}}{%
\pgfpathmoveto{\pgfqpoint{-0.000000in}{0.000000in}}%
\pgfpathlineto{\pgfqpoint{-0.048611in}{0.000000in}}%
\pgfusepath{stroke,fill}%
}%
\begin{pgfscope}%
\pgfsys@transformshift{0.475556in}{4.129533in}%
\pgfsys@useobject{currentmarker}{}%
\end{pgfscope}%
\end{pgfscope}%
\begin{pgfscope}%
\definecolor{textcolor}{rgb}{0.000000,0.000000,0.000000}%
\pgfsetstrokecolor{textcolor}%
\pgfsetfillcolor{textcolor}%
\pgftext[x=0.289968in, y=4.076771in, left, base]{\color{textcolor}\sffamily\fontsize{10.000000}{12.000000}\selectfont 6}%
\end{pgfscope}%
\begin{pgfscope}%
\pgfsetbuttcap%
\pgfsetroundjoin%
\definecolor{currentfill}{rgb}{0.000000,0.000000,0.000000}%
\pgfsetfillcolor{currentfill}%
\pgfsetlinewidth{0.803000pt}%
\definecolor{currentstroke}{rgb}{0.000000,0.000000,0.000000}%
\pgfsetstrokecolor{currentstroke}%
\pgfsetdash{}{0pt}%
\pgfsys@defobject{currentmarker}{\pgfqpoint{-0.048611in}{0.000000in}}{\pgfqpoint{-0.000000in}{0.000000in}}{%
\pgfpathmoveto{\pgfqpoint{-0.000000in}{0.000000in}}%
\pgfpathlineto{\pgfqpoint{-0.048611in}{0.000000in}}%
\pgfusepath{stroke,fill}%
}%
\begin{pgfscope}%
\pgfsys@transformshift{0.475556in}{4.635738in}%
\pgfsys@useobject{currentmarker}{}%
\end{pgfscope}%
\end{pgfscope}%
\begin{pgfscope}%
\definecolor{textcolor}{rgb}{0.000000,0.000000,0.000000}%
\pgfsetstrokecolor{textcolor}%
\pgfsetfillcolor{textcolor}%
\pgftext[x=0.289968in, y=4.582976in, left, base]{\color{textcolor}\sffamily\fontsize{10.000000}{12.000000}\selectfont 7}%
\end{pgfscope}%
\begin{pgfscope}%
\definecolor{textcolor}{rgb}{0.000000,0.000000,0.000000}%
\pgfsetstrokecolor{textcolor}%
\pgfsetfillcolor{textcolor}%
\pgftext[x=0.234413in,y=2.940302in,,bottom,rotate=90.000000]{\color{textcolor}\sffamily\fontsize{10.000000}{12.000000}\selectfont Average ratings}%
\end{pgfscope}%
\begin{pgfscope}%
\pgfsetrectcap%
\pgfsetmiterjoin%
\pgfsetlinewidth{0.803000pt}%
\definecolor{currentstroke}{rgb}{0.000000,0.000000,0.000000}%
\pgfsetstrokecolor{currentstroke}%
\pgfsetdash{}{0pt}%
\pgfpathmoveto{\pgfqpoint{0.475556in}{1.092302in}}%
\pgfpathlineto{\pgfqpoint{0.475556in}{4.788302in}}%
\pgfusepath{stroke}%
\end{pgfscope}%
\begin{pgfscope}%
\pgfsetrectcap%
\pgfsetmiterjoin%
\pgfsetlinewidth{0.803000pt}%
\definecolor{currentstroke}{rgb}{0.000000,0.000000,0.000000}%
\pgfsetstrokecolor{currentstroke}%
\pgfsetdash{}{0pt}%
\pgfpathmoveto{\pgfqpoint{5.435556in}{1.092302in}}%
\pgfpathlineto{\pgfqpoint{5.435556in}{4.788302in}}%
\pgfusepath{stroke}%
\end{pgfscope}%
\begin{pgfscope}%
\pgfsetrectcap%
\pgfsetmiterjoin%
\pgfsetlinewidth{0.803000pt}%
\definecolor{currentstroke}{rgb}{0.000000,0.000000,0.000000}%
\pgfsetstrokecolor{currentstroke}%
\pgfsetdash{}{0pt}%
\pgfpathmoveto{\pgfqpoint{0.475556in}{1.092302in}}%
\pgfpathlineto{\pgfqpoint{5.435556in}{1.092302in}}%
\pgfusepath{stroke}%
\end{pgfscope}%
\begin{pgfscope}%
\pgfsetrectcap%
\pgfsetmiterjoin%
\pgfsetlinewidth{0.803000pt}%
\definecolor{currentstroke}{rgb}{0.000000,0.000000,0.000000}%
\pgfsetstrokecolor{currentstroke}%
\pgfsetdash{}{0pt}%
\pgfpathmoveto{\pgfqpoint{0.475556in}{4.788302in}}%
\pgfpathlineto{\pgfqpoint{5.435556in}{4.788302in}}%
\pgfusepath{stroke}%
\end{pgfscope}%
\end{pgfpicture}%
\makeatother%
\endgroup%
}
    \caption[Average Ratings for Survey:Section 2]{The figure represents average rating given by the participant to each of the datasets in section 2 of the survey.
    The automated datasets are highlighted, all of them have lower average. \gls{ai2thor}(Unity based and manually designed) is also among the lower averages.}
    \label{fig:question2_2}
\end{figure}

\subsection{Section 3: Rank by Comparison}
In section 3 of the survey, the users compared all nine datasets and ranked them according to their photorealism(1 being the best rank).
\autoref{fig:question3} and \autoref{fig:question3_2} show distribution and average ranking for each of the datasets.
The real dataset Pix3D got the highest number of votes for rank 1, while \gls{free} got the least.
However, if we have a threshold of 5, meaning frequency of votes being in the top 5 ranks, then among the automated group, \gls{free} occurs most times, followed by SceneNet, Openrooms, and Blenderproc.
This investigation is significant because out of the nine datasets, four are automated, and these four have the least average rankings.
However, \gls{free} breaks the boundary to be in the top 5 most times.

\begin{figure}
    \centering
    \resizebox{\textwidth}{!}{%% Creator: Matplotlib, PGF backend
%%
%% To include the figure in your LaTeX document, write
%%   \input{<filename>.pgf}
%%
%% Make sure the required packages are loaded in your preamble
%%   \usepackage{pgf}
%%
%% Figures using additional raster images can only be included by \input if
%% they are in the same directory as the main LaTeX file. For loading figures
%% from other directories you can use the `import` package
%%   \usepackage{import}
%%
%% and then include the figures with
%%   \import{<path to file>}{<filename>.pgf}
%%
%% Matplotlib used the following preamble
%%   \usepackage{fontspec}
%%   \setmainfont{DejaVuSerif.ttf}[Path=\detokenize{/Users/apple/opt/anaconda3/envs/kaolin/lib/python3.7/site-packages/matplotlib/mpl-data/fonts/ttf/}]
%%   \setsansfont{DejaVuSans.ttf}[Path=\detokenize{/Users/apple/opt/anaconda3/envs/kaolin/lib/python3.7/site-packages/matplotlib/mpl-data/fonts/ttf/}]
%%   \setmonofont{DejaVuSansMono.ttf}[Path=\detokenize{/Users/apple/opt/anaconda3/envs/kaolin/lib/python3.7/site-packages/matplotlib/mpl-data/fonts/ttf/}]
%%
\begingroup%
\makeatletter%
\begin{pgfpicture}%
\pgfpathrectangle{\pgfpointorigin}{\pgfqpoint{8.499670in}{4.360980in}}%
\pgfusepath{use as bounding box, clip}%
\begin{pgfscope}%
\pgfsetbuttcap%
\pgfsetmiterjoin%
\definecolor{currentfill}{rgb}{1.000000,1.000000,1.000000}%
\pgfsetfillcolor{currentfill}%
\pgfsetlinewidth{0.000000pt}%
\definecolor{currentstroke}{rgb}{1.000000,1.000000,1.000000}%
\pgfsetstrokecolor{currentstroke}%
\pgfsetdash{}{0pt}%
\pgfpathmoveto{\pgfqpoint{0.000000in}{0.000000in}}%
\pgfpathlineto{\pgfqpoint{8.499670in}{0.000000in}}%
\pgfpathlineto{\pgfqpoint{8.499670in}{4.360980in}}%
\pgfpathlineto{\pgfqpoint{0.000000in}{4.360980in}}%
\pgfpathclose%
\pgfusepath{fill}%
\end{pgfscope}%
\begin{pgfscope}%
\pgfsetbuttcap%
\pgfsetmiterjoin%
\definecolor{currentfill}{rgb}{1.000000,1.000000,1.000000}%
\pgfsetfillcolor{currentfill}%
\pgfsetlinewidth{0.000000pt}%
\definecolor{currentstroke}{rgb}{0.000000,0.000000,0.000000}%
\pgfsetstrokecolor{currentstroke}%
\pgfsetstrokeopacity{0.000000}%
\pgfsetdash{}{0pt}%
\pgfpathmoveto{\pgfqpoint{1.249956in}{0.148611in}}%
\pgfpathlineto{\pgfqpoint{8.372206in}{0.148611in}}%
\pgfpathlineto{\pgfqpoint{8.372206in}{3.998611in}}%
\pgfpathlineto{\pgfqpoint{1.249956in}{3.998611in}}%
\pgfpathclose%
\pgfusepath{fill}%
\end{pgfscope}%
\begin{pgfscope}%
\pgfpathrectangle{\pgfqpoint{1.249956in}{0.148611in}}{\pgfqpoint{7.122250in}{3.850000in}}%
\pgfusepath{clip}%
\pgfsetbuttcap%
\pgfsetmiterjoin%
\definecolor{currentfill}{rgb}{0.248058,0.667205,0.350250}%
\pgfsetfillcolor{currentfill}%
\pgfsetfillopacity{0.500000}%
\pgfsetlinewidth{0.000000pt}%
\definecolor{currentstroke}{rgb}{0.000000,0.000000,0.000000}%
\pgfsetstrokecolor{currentstroke}%
\pgfsetstrokeopacity{0.500000}%
\pgfsetdash{}{0pt}%
\pgfpathmoveto{\pgfqpoint{1.249956in}{3.823611in}}%
\pgfpathlineto{\pgfqpoint{1.651210in}{3.823611in}}%
\pgfpathlineto{\pgfqpoint{1.651210in}{3.617729in}}%
\pgfpathlineto{\pgfqpoint{1.249956in}{3.617729in}}%
\pgfpathclose%
\pgfusepath{fill}%
\end{pgfscope}%
\begin{pgfscope}%
\pgfpathrectangle{\pgfqpoint{1.249956in}{0.148611in}}{\pgfqpoint{7.122250in}{3.850000in}}%
\pgfusepath{clip}%
\pgfsetbuttcap%
\pgfsetmiterjoin%
\definecolor{currentfill}{rgb}{0.248058,0.667205,0.350250}%
\pgfsetfillcolor{currentfill}%
\pgfsetfillopacity{0.500000}%
\pgfsetlinewidth{0.000000pt}%
\definecolor{currentstroke}{rgb}{0.000000,0.000000,0.000000}%
\pgfsetstrokecolor{currentstroke}%
\pgfsetstrokeopacity{0.500000}%
\pgfsetdash{}{0pt}%
\pgfpathmoveto{\pgfqpoint{1.249956in}{3.411846in}}%
\pgfpathlineto{\pgfqpoint{1.718085in}{3.411846in}}%
\pgfpathlineto{\pgfqpoint{1.718085in}{3.205964in}}%
\pgfpathlineto{\pgfqpoint{1.249956in}{3.205964in}}%
\pgfpathclose%
\pgfusepath{fill}%
\end{pgfscope}%
\begin{pgfscope}%
\pgfpathrectangle{\pgfqpoint{1.249956in}{0.148611in}}{\pgfqpoint{7.122250in}{3.850000in}}%
\pgfusepath{clip}%
\pgfsetbuttcap%
\pgfsetmiterjoin%
\definecolor{currentfill}{rgb}{0.248058,0.667205,0.350250}%
\pgfsetfillcolor{currentfill}%
\pgfsetlinewidth{0.000000pt}%
\definecolor{currentstroke}{rgb}{0.000000,0.000000,0.000000}%
\pgfsetstrokecolor{currentstroke}%
\pgfsetstrokeopacity{0.000000}%
\pgfsetdash{}{0pt}%
\pgfpathmoveto{\pgfqpoint{1.249956in}{3.000082in}}%
\pgfpathlineto{\pgfqpoint{1.517459in}{3.000082in}}%
\pgfpathlineto{\pgfqpoint{1.517459in}{2.794199in}}%
\pgfpathlineto{\pgfqpoint{1.249956in}{2.794199in}}%
\pgfpathclose%
\pgfusepath{fill}%
\end{pgfscope}%
\begin{pgfscope}%
\pgfpathrectangle{\pgfqpoint{1.249956in}{0.148611in}}{\pgfqpoint{7.122250in}{3.850000in}}%
\pgfusepath{clip}%
\pgfsetbuttcap%
\pgfsetmiterjoin%
\definecolor{currentfill}{rgb}{0.248058,0.667205,0.350250}%
\pgfsetfillcolor{currentfill}%
\pgfsetfillopacity{0.500000}%
\pgfsetlinewidth{0.000000pt}%
\definecolor{currentstroke}{rgb}{0.000000,0.000000,0.000000}%
\pgfsetstrokecolor{currentstroke}%
\pgfsetstrokeopacity{0.500000}%
\pgfsetdash{}{0pt}%
\pgfpathmoveto{\pgfqpoint{1.249956in}{2.588317in}}%
\pgfpathlineto{\pgfqpoint{1.718085in}{2.588317in}}%
\pgfpathlineto{\pgfqpoint{1.718085in}{2.382435in}}%
\pgfpathlineto{\pgfqpoint{1.249956in}{2.382435in}}%
\pgfpathclose%
\pgfusepath{fill}%
\end{pgfscope}%
\begin{pgfscope}%
\pgfpathrectangle{\pgfqpoint{1.249956in}{0.148611in}}{\pgfqpoint{7.122250in}{3.850000in}}%
\pgfusepath{clip}%
\pgfsetbuttcap%
\pgfsetmiterjoin%
\definecolor{currentfill}{rgb}{0.248058,0.667205,0.350250}%
\pgfsetfillcolor{currentfill}%
\pgfsetfillopacity{0.500000}%
\pgfsetlinewidth{0.000000pt}%
\definecolor{currentstroke}{rgb}{0.000000,0.000000,0.000000}%
\pgfsetstrokecolor{currentstroke}%
\pgfsetstrokeopacity{0.500000}%
\pgfsetdash{}{0pt}%
\pgfpathmoveto{\pgfqpoint{1.249956in}{2.176552in}}%
\pgfpathlineto{\pgfqpoint{2.854970in}{2.176552in}}%
\pgfpathlineto{\pgfqpoint{2.854970in}{1.970670in}}%
\pgfpathlineto{\pgfqpoint{1.249956in}{1.970670in}}%
\pgfpathclose%
\pgfusepath{fill}%
\end{pgfscope}%
\begin{pgfscope}%
\pgfpathrectangle{\pgfqpoint{1.249956in}{0.148611in}}{\pgfqpoint{7.122250in}{3.850000in}}%
\pgfusepath{clip}%
\pgfsetbuttcap%
\pgfsetmiterjoin%
\definecolor{currentfill}{rgb}{0.248058,0.667205,0.350250}%
\pgfsetfillcolor{currentfill}%
\pgfsetlinewidth{0.000000pt}%
\definecolor{currentstroke}{rgb}{0.000000,0.000000,0.000000}%
\pgfsetstrokecolor{currentstroke}%
\pgfsetstrokeopacity{0.000000}%
\pgfsetdash{}{0pt}%
\pgfpathmoveto{\pgfqpoint{1.249956in}{1.764788in}}%
\pgfpathlineto{\pgfqpoint{1.417145in}{1.764788in}}%
\pgfpathlineto{\pgfqpoint{1.417145in}{1.558905in}}%
\pgfpathlineto{\pgfqpoint{1.249956in}{1.558905in}}%
\pgfpathclose%
\pgfusepath{fill}%
\end{pgfscope}%
\begin{pgfscope}%
\pgfpathrectangle{\pgfqpoint{1.249956in}{0.148611in}}{\pgfqpoint{7.122250in}{3.850000in}}%
\pgfusepath{clip}%
\pgfsetbuttcap%
\pgfsetmiterjoin%
\definecolor{currentfill}{rgb}{0.248058,0.667205,0.350250}%
\pgfsetfillcolor{currentfill}%
\pgfsetfillopacity{0.500000}%
\pgfsetlinewidth{0.000000pt}%
\definecolor{currentstroke}{rgb}{0.000000,0.000000,0.000000}%
\pgfsetstrokecolor{currentstroke}%
\pgfsetstrokeopacity{0.500000}%
\pgfsetdash{}{0pt}%
\pgfpathmoveto{\pgfqpoint{1.249956in}{1.353023in}}%
\pgfpathlineto{\pgfqpoint{4.593736in}{1.353023in}}%
\pgfpathlineto{\pgfqpoint{4.593736in}{1.147141in}}%
\pgfpathlineto{\pgfqpoint{1.249956in}{1.147141in}}%
\pgfpathclose%
\pgfusepath{fill}%
\end{pgfscope}%
\begin{pgfscope}%
\pgfpathrectangle{\pgfqpoint{1.249956in}{0.148611in}}{\pgfqpoint{7.122250in}{3.850000in}}%
\pgfusepath{clip}%
\pgfsetbuttcap%
\pgfsetmiterjoin%
\definecolor{currentfill}{rgb}{0.248058,0.667205,0.350250}%
\pgfsetfillcolor{currentfill}%
\pgfsetlinewidth{0.000000pt}%
\definecolor{currentstroke}{rgb}{0.000000,0.000000,0.000000}%
\pgfsetstrokecolor{currentstroke}%
\pgfsetstrokeopacity{0.000000}%
\pgfsetdash{}{0pt}%
\pgfpathmoveto{\pgfqpoint{1.249956in}{0.941258in}}%
\pgfpathlineto{\pgfqpoint{1.316832in}{0.941258in}}%
\pgfpathlineto{\pgfqpoint{1.316832in}{0.735376in}}%
\pgfpathlineto{\pgfqpoint{1.249956in}{0.735376in}}%
\pgfpathclose%
\pgfusepath{fill}%
\end{pgfscope}%
\begin{pgfscope}%
\pgfpathrectangle{\pgfqpoint{1.249956in}{0.148611in}}{\pgfqpoint{7.122250in}{3.850000in}}%
\pgfusepath{clip}%
\pgfsetbuttcap%
\pgfsetmiterjoin%
\definecolor{currentfill}{rgb}{0.248058,0.667205,0.350250}%
\pgfsetfillcolor{currentfill}%
\pgfsetlinewidth{0.000000pt}%
\definecolor{currentstroke}{rgb}{0.000000,0.000000,0.000000}%
\pgfsetstrokecolor{currentstroke}%
\pgfsetstrokeopacity{0.000000}%
\pgfsetdash{}{0pt}%
\pgfpathmoveto{\pgfqpoint{1.249956in}{0.529493in}}%
\pgfpathlineto{\pgfqpoint{1.584334in}{0.529493in}}%
\pgfpathlineto{\pgfqpoint{1.584334in}{0.323611in}}%
\pgfpathlineto{\pgfqpoint{1.249956in}{0.323611in}}%
\pgfpathclose%
\pgfusepath{fill}%
\end{pgfscope}%
\begin{pgfscope}%
\pgfpathrectangle{\pgfqpoint{1.249956in}{0.148611in}}{\pgfqpoint{7.122250in}{3.850000in}}%
\pgfusepath{clip}%
\pgfsetbuttcap%
\pgfsetmiterjoin%
\definecolor{currentfill}{rgb}{0.488581,0.779931,0.397924}%
\pgfsetfillcolor{currentfill}%
\pgfsetfillopacity{0.500000}%
\pgfsetlinewidth{0.000000pt}%
\definecolor{currentstroke}{rgb}{0.000000,0.000000,0.000000}%
\pgfsetstrokecolor{currentstroke}%
\pgfsetstrokeopacity{0.500000}%
\pgfsetdash{}{0pt}%
\pgfpathmoveto{\pgfqpoint{1.651210in}{3.823611in}}%
\pgfpathlineto{\pgfqpoint{2.186215in}{3.823611in}}%
\pgfpathlineto{\pgfqpoint{2.186215in}{3.617729in}}%
\pgfpathlineto{\pgfqpoint{1.651210in}{3.617729in}}%
\pgfpathclose%
\pgfusepath{fill}%
\end{pgfscope}%
\begin{pgfscope}%
\pgfpathrectangle{\pgfqpoint{1.249956in}{0.148611in}}{\pgfqpoint{7.122250in}{3.850000in}}%
\pgfusepath{clip}%
\pgfsetbuttcap%
\pgfsetmiterjoin%
\definecolor{currentfill}{rgb}{0.488581,0.779931,0.397924}%
\pgfsetfillcolor{currentfill}%
\pgfsetfillopacity{0.500000}%
\pgfsetlinewidth{0.000000pt}%
\definecolor{currentstroke}{rgb}{0.000000,0.000000,0.000000}%
\pgfsetstrokecolor{currentstroke}%
\pgfsetstrokeopacity{0.500000}%
\pgfsetdash{}{0pt}%
\pgfpathmoveto{\pgfqpoint{1.718085in}{3.411846in}}%
\pgfpathlineto{\pgfqpoint{2.219652in}{3.411846in}}%
\pgfpathlineto{\pgfqpoint{2.219652in}{3.205964in}}%
\pgfpathlineto{\pgfqpoint{1.718085in}{3.205964in}}%
\pgfpathclose%
\pgfusepath{fill}%
\end{pgfscope}%
\begin{pgfscope}%
\pgfpathrectangle{\pgfqpoint{1.249956in}{0.148611in}}{\pgfqpoint{7.122250in}{3.850000in}}%
\pgfusepath{clip}%
\pgfsetbuttcap%
\pgfsetmiterjoin%
\definecolor{currentfill}{rgb}{0.488581,0.779931,0.397924}%
\pgfsetfillcolor{currentfill}%
\pgfsetlinewidth{0.000000pt}%
\definecolor{currentstroke}{rgb}{0.000000,0.000000,0.000000}%
\pgfsetstrokecolor{currentstroke}%
\pgfsetstrokeopacity{0.000000}%
\pgfsetdash{}{0pt}%
\pgfpathmoveto{\pgfqpoint{1.517459in}{3.000082in}}%
\pgfpathlineto{\pgfqpoint{1.885274in}{3.000082in}}%
\pgfpathlineto{\pgfqpoint{1.885274in}{2.794199in}}%
\pgfpathlineto{\pgfqpoint{1.517459in}{2.794199in}}%
\pgfpathclose%
\pgfusepath{fill}%
\end{pgfscope}%
\begin{pgfscope}%
\pgfpathrectangle{\pgfqpoint{1.249956in}{0.148611in}}{\pgfqpoint{7.122250in}{3.850000in}}%
\pgfusepath{clip}%
\pgfsetbuttcap%
\pgfsetmiterjoin%
\definecolor{currentfill}{rgb}{0.488581,0.779931,0.397924}%
\pgfsetfillcolor{currentfill}%
\pgfsetfillopacity{0.500000}%
\pgfsetlinewidth{0.000000pt}%
\definecolor{currentstroke}{rgb}{0.000000,0.000000,0.000000}%
\pgfsetstrokecolor{currentstroke}%
\pgfsetstrokeopacity{0.500000}%
\pgfsetdash{}{0pt}%
\pgfpathmoveto{\pgfqpoint{1.718085in}{2.588317in}}%
\pgfpathlineto{\pgfqpoint{3.155911in}{2.588317in}}%
\pgfpathlineto{\pgfqpoint{3.155911in}{2.382435in}}%
\pgfpathlineto{\pgfqpoint{1.718085in}{2.382435in}}%
\pgfpathclose%
\pgfusepath{fill}%
\end{pgfscope}%
\begin{pgfscope}%
\pgfpathrectangle{\pgfqpoint{1.249956in}{0.148611in}}{\pgfqpoint{7.122250in}{3.850000in}}%
\pgfusepath{clip}%
\pgfsetbuttcap%
\pgfsetmiterjoin%
\definecolor{currentfill}{rgb}{0.488581,0.779931,0.397924}%
\pgfsetfillcolor{currentfill}%
\pgfsetfillopacity{0.500000}%
\pgfsetlinewidth{0.000000pt}%
\definecolor{currentstroke}{rgb}{0.000000,0.000000,0.000000}%
\pgfsetstrokecolor{currentstroke}%
\pgfsetstrokeopacity{0.500000}%
\pgfsetdash{}{0pt}%
\pgfpathmoveto{\pgfqpoint{2.854970in}{2.176552in}}%
\pgfpathlineto{\pgfqpoint{5.061865in}{2.176552in}}%
\pgfpathlineto{\pgfqpoint{5.061865in}{1.970670in}}%
\pgfpathlineto{\pgfqpoint{2.854970in}{1.970670in}}%
\pgfpathclose%
\pgfusepath{fill}%
\end{pgfscope}%
\begin{pgfscope}%
\pgfpathrectangle{\pgfqpoint{1.249956in}{0.148611in}}{\pgfqpoint{7.122250in}{3.850000in}}%
\pgfusepath{clip}%
\pgfsetbuttcap%
\pgfsetmiterjoin%
\definecolor{currentfill}{rgb}{0.488581,0.779931,0.397924}%
\pgfsetfillcolor{currentfill}%
\pgfsetlinewidth{0.000000pt}%
\definecolor{currentstroke}{rgb}{0.000000,0.000000,0.000000}%
\pgfsetstrokecolor{currentstroke}%
\pgfsetstrokeopacity{0.000000}%
\pgfsetdash{}{0pt}%
\pgfpathmoveto{\pgfqpoint{1.417145in}{1.764788in}}%
\pgfpathlineto{\pgfqpoint{1.617772in}{1.764788in}}%
\pgfpathlineto{\pgfqpoint{1.617772in}{1.558905in}}%
\pgfpathlineto{\pgfqpoint{1.417145in}{1.558905in}}%
\pgfpathclose%
\pgfusepath{fill}%
\end{pgfscope}%
\begin{pgfscope}%
\pgfpathrectangle{\pgfqpoint{1.249956in}{0.148611in}}{\pgfqpoint{7.122250in}{3.850000in}}%
\pgfusepath{clip}%
\pgfsetbuttcap%
\pgfsetmiterjoin%
\definecolor{currentfill}{rgb}{0.488581,0.779931,0.397924}%
\pgfsetfillcolor{currentfill}%
\pgfsetfillopacity{0.500000}%
\pgfsetlinewidth{0.000000pt}%
\definecolor{currentstroke}{rgb}{0.000000,0.000000,0.000000}%
\pgfsetstrokecolor{currentstroke}%
\pgfsetstrokeopacity{0.500000}%
\pgfsetdash{}{0pt}%
\pgfpathmoveto{\pgfqpoint{4.593736in}{1.353023in}}%
\pgfpathlineto{\pgfqpoint{5.797496in}{1.353023in}}%
\pgfpathlineto{\pgfqpoint{5.797496in}{1.147141in}}%
\pgfpathlineto{\pgfqpoint{4.593736in}{1.147141in}}%
\pgfpathclose%
\pgfusepath{fill}%
\end{pgfscope}%
\begin{pgfscope}%
\pgfpathrectangle{\pgfqpoint{1.249956in}{0.148611in}}{\pgfqpoint{7.122250in}{3.850000in}}%
\pgfusepath{clip}%
\pgfsetbuttcap%
\pgfsetmiterjoin%
\definecolor{currentfill}{rgb}{0.488581,0.779931,0.397924}%
\pgfsetfillcolor{currentfill}%
\pgfsetlinewidth{0.000000pt}%
\definecolor{currentstroke}{rgb}{0.000000,0.000000,0.000000}%
\pgfsetstrokecolor{currentstroke}%
\pgfsetstrokeopacity{0.000000}%
\pgfsetdash{}{0pt}%
\pgfpathmoveto{\pgfqpoint{1.316832in}{0.941258in}}%
\pgfpathlineto{\pgfqpoint{1.617772in}{0.941258in}}%
\pgfpathlineto{\pgfqpoint{1.617772in}{0.735376in}}%
\pgfpathlineto{\pgfqpoint{1.316832in}{0.735376in}}%
\pgfpathclose%
\pgfusepath{fill}%
\end{pgfscope}%
\begin{pgfscope}%
\pgfpathrectangle{\pgfqpoint{1.249956in}{0.148611in}}{\pgfqpoint{7.122250in}{3.850000in}}%
\pgfusepath{clip}%
\pgfsetbuttcap%
\pgfsetmiterjoin%
\definecolor{currentfill}{rgb}{0.488581,0.779931,0.397924}%
\pgfsetfillcolor{currentfill}%
\pgfsetlinewidth{0.000000pt}%
\definecolor{currentstroke}{rgb}{0.000000,0.000000,0.000000}%
\pgfsetstrokecolor{currentstroke}%
\pgfsetstrokeopacity{0.000000}%
\pgfsetdash{}{0pt}%
\pgfpathmoveto{\pgfqpoint{1.584334in}{0.529493in}}%
\pgfpathlineto{\pgfqpoint{1.952150in}{0.529493in}}%
\pgfpathlineto{\pgfqpoint{1.952150in}{0.323611in}}%
\pgfpathlineto{\pgfqpoint{1.584334in}{0.323611in}}%
\pgfpathclose%
\pgfusepath{fill}%
\end{pgfscope}%
\begin{pgfscope}%
\pgfpathrectangle{\pgfqpoint{1.249956in}{0.148611in}}{\pgfqpoint{7.122250in}{3.850000in}}%
\pgfusepath{clip}%
\pgfsetbuttcap%
\pgfsetmiterjoin%
\definecolor{currentfill}{rgb}{0.701961,0.872972,0.448674}%
\pgfsetfillcolor{currentfill}%
\pgfsetfillopacity{0.500000}%
\pgfsetlinewidth{0.000000pt}%
\definecolor{currentstroke}{rgb}{0.000000,0.000000,0.000000}%
\pgfsetstrokecolor{currentstroke}%
\pgfsetstrokeopacity{0.500000}%
\pgfsetdash{}{0pt}%
\pgfpathmoveto{\pgfqpoint{2.186215in}{3.823611in}}%
\pgfpathlineto{\pgfqpoint{2.955284in}{3.823611in}}%
\pgfpathlineto{\pgfqpoint{2.955284in}{3.617729in}}%
\pgfpathlineto{\pgfqpoint{2.186215in}{3.617729in}}%
\pgfpathclose%
\pgfusepath{fill}%
\end{pgfscope}%
\begin{pgfscope}%
\pgfpathrectangle{\pgfqpoint{1.249956in}{0.148611in}}{\pgfqpoint{7.122250in}{3.850000in}}%
\pgfusepath{clip}%
\pgfsetbuttcap%
\pgfsetmiterjoin%
\definecolor{currentfill}{rgb}{0.701961,0.872972,0.448674}%
\pgfsetfillcolor{currentfill}%
\pgfsetfillopacity{0.500000}%
\pgfsetlinewidth{0.000000pt}%
\definecolor{currentstroke}{rgb}{0.000000,0.000000,0.000000}%
\pgfsetstrokecolor{currentstroke}%
\pgfsetstrokeopacity{0.500000}%
\pgfsetdash{}{0pt}%
\pgfpathmoveto{\pgfqpoint{2.219652in}{3.411846in}}%
\pgfpathlineto{\pgfqpoint{3.055597in}{3.411846in}}%
\pgfpathlineto{\pgfqpoint{3.055597in}{3.205964in}}%
\pgfpathlineto{\pgfqpoint{2.219652in}{3.205964in}}%
\pgfpathclose%
\pgfusepath{fill}%
\end{pgfscope}%
\begin{pgfscope}%
\pgfpathrectangle{\pgfqpoint{1.249956in}{0.148611in}}{\pgfqpoint{7.122250in}{3.850000in}}%
\pgfusepath{clip}%
\pgfsetbuttcap%
\pgfsetmiterjoin%
\definecolor{currentfill}{rgb}{0.701961,0.872972,0.448674}%
\pgfsetfillcolor{currentfill}%
\pgfsetlinewidth{0.000000pt}%
\definecolor{currentstroke}{rgb}{0.000000,0.000000,0.000000}%
\pgfsetstrokecolor{currentstroke}%
\pgfsetstrokeopacity{0.000000}%
\pgfsetdash{}{0pt}%
\pgfpathmoveto{\pgfqpoint{1.885274in}{3.000082in}}%
\pgfpathlineto{\pgfqpoint{2.353404in}{3.000082in}}%
\pgfpathlineto{\pgfqpoint{2.353404in}{2.794199in}}%
\pgfpathlineto{\pgfqpoint{1.885274in}{2.794199in}}%
\pgfpathclose%
\pgfusepath{fill}%
\end{pgfscope}%
\begin{pgfscope}%
\pgfpathrectangle{\pgfqpoint{1.249956in}{0.148611in}}{\pgfqpoint{7.122250in}{3.850000in}}%
\pgfusepath{clip}%
\pgfsetbuttcap%
\pgfsetmiterjoin%
\definecolor{currentfill}{rgb}{0.701961,0.872972,0.448674}%
\pgfsetfillcolor{currentfill}%
\pgfsetfillopacity{0.500000}%
\pgfsetlinewidth{0.000000pt}%
\definecolor{currentstroke}{rgb}{0.000000,0.000000,0.000000}%
\pgfsetstrokecolor{currentstroke}%
\pgfsetstrokeopacity{0.500000}%
\pgfsetdash{}{0pt}%
\pgfpathmoveto{\pgfqpoint{3.155911in}{2.588317in}}%
\pgfpathlineto{\pgfqpoint{4.760925in}{2.588317in}}%
\pgfpathlineto{\pgfqpoint{4.760925in}{2.382435in}}%
\pgfpathlineto{\pgfqpoint{3.155911in}{2.382435in}}%
\pgfpathclose%
\pgfusepath{fill}%
\end{pgfscope}%
\begin{pgfscope}%
\pgfpathrectangle{\pgfqpoint{1.249956in}{0.148611in}}{\pgfqpoint{7.122250in}{3.850000in}}%
\pgfusepath{clip}%
\pgfsetbuttcap%
\pgfsetmiterjoin%
\definecolor{currentfill}{rgb}{0.701961,0.872972,0.448674}%
\pgfsetfillcolor{currentfill}%
\pgfsetfillopacity{0.500000}%
\pgfsetlinewidth{0.000000pt}%
\definecolor{currentstroke}{rgb}{0.000000,0.000000,0.000000}%
\pgfsetstrokecolor{currentstroke}%
\pgfsetstrokeopacity{0.500000}%
\pgfsetdash{}{0pt}%
\pgfpathmoveto{\pgfqpoint{5.061865in}{2.176552in}}%
\pgfpathlineto{\pgfqpoint{5.931247in}{2.176552in}}%
\pgfpathlineto{\pgfqpoint{5.931247in}{1.970670in}}%
\pgfpathlineto{\pgfqpoint{5.061865in}{1.970670in}}%
\pgfpathclose%
\pgfusepath{fill}%
\end{pgfscope}%
\begin{pgfscope}%
\pgfpathrectangle{\pgfqpoint{1.249956in}{0.148611in}}{\pgfqpoint{7.122250in}{3.850000in}}%
\pgfusepath{clip}%
\pgfsetbuttcap%
\pgfsetmiterjoin%
\definecolor{currentfill}{rgb}{0.701961,0.872972,0.448674}%
\pgfsetfillcolor{currentfill}%
\pgfsetlinewidth{0.000000pt}%
\definecolor{currentstroke}{rgb}{0.000000,0.000000,0.000000}%
\pgfsetstrokecolor{currentstroke}%
\pgfsetstrokeopacity{0.000000}%
\pgfsetdash{}{0pt}%
\pgfpathmoveto{\pgfqpoint{1.617772in}{1.764788in}}%
\pgfpathlineto{\pgfqpoint{2.052463in}{1.764788in}}%
\pgfpathlineto{\pgfqpoint{2.052463in}{1.558905in}}%
\pgfpathlineto{\pgfqpoint{1.617772in}{1.558905in}}%
\pgfpathclose%
\pgfusepath{fill}%
\end{pgfscope}%
\begin{pgfscope}%
\pgfpathrectangle{\pgfqpoint{1.249956in}{0.148611in}}{\pgfqpoint{7.122250in}{3.850000in}}%
\pgfusepath{clip}%
\pgfsetbuttcap%
\pgfsetmiterjoin%
\definecolor{currentfill}{rgb}{0.701961,0.872972,0.448674}%
\pgfsetfillcolor{currentfill}%
\pgfsetfillopacity{0.500000}%
\pgfsetlinewidth{0.000000pt}%
\definecolor{currentstroke}{rgb}{0.000000,0.000000,0.000000}%
\pgfsetstrokecolor{currentstroke}%
\pgfsetstrokeopacity{0.500000}%
\pgfsetdash{}{0pt}%
\pgfpathmoveto{\pgfqpoint{5.797496in}{1.353023in}}%
\pgfpathlineto{\pgfqpoint{6.700317in}{1.353023in}}%
\pgfpathlineto{\pgfqpoint{6.700317in}{1.147141in}}%
\pgfpathlineto{\pgfqpoint{5.797496in}{1.147141in}}%
\pgfpathclose%
\pgfusepath{fill}%
\end{pgfscope}%
\begin{pgfscope}%
\pgfpathrectangle{\pgfqpoint{1.249956in}{0.148611in}}{\pgfqpoint{7.122250in}{3.850000in}}%
\pgfusepath{clip}%
\pgfsetbuttcap%
\pgfsetmiterjoin%
\definecolor{currentfill}{rgb}{0.701961,0.872972,0.448674}%
\pgfsetfillcolor{currentfill}%
\pgfsetlinewidth{0.000000pt}%
\definecolor{currentstroke}{rgb}{0.000000,0.000000,0.000000}%
\pgfsetstrokecolor{currentstroke}%
\pgfsetstrokeopacity{0.000000}%
\pgfsetdash{}{0pt}%
\pgfpathmoveto{\pgfqpoint{1.617772in}{0.941258in}}%
\pgfpathlineto{\pgfqpoint{2.186215in}{0.941258in}}%
\pgfpathlineto{\pgfqpoint{2.186215in}{0.735376in}}%
\pgfpathlineto{\pgfqpoint{1.617772in}{0.735376in}}%
\pgfpathclose%
\pgfusepath{fill}%
\end{pgfscope}%
\begin{pgfscope}%
\pgfpathrectangle{\pgfqpoint{1.249956in}{0.148611in}}{\pgfqpoint{7.122250in}{3.850000in}}%
\pgfusepath{clip}%
\pgfsetbuttcap%
\pgfsetmiterjoin%
\definecolor{currentfill}{rgb}{0.701961,0.872972,0.448674}%
\pgfsetfillcolor{currentfill}%
\pgfsetlinewidth{0.000000pt}%
\definecolor{currentstroke}{rgb}{0.000000,0.000000,0.000000}%
\pgfsetstrokecolor{currentstroke}%
\pgfsetstrokeopacity{0.000000}%
\pgfsetdash{}{0pt}%
\pgfpathmoveto{\pgfqpoint{1.952150in}{0.529493in}}%
\pgfpathlineto{\pgfqpoint{2.620906in}{0.529493in}}%
\pgfpathlineto{\pgfqpoint{2.620906in}{0.323611in}}%
\pgfpathlineto{\pgfqpoint{1.952150in}{0.323611in}}%
\pgfpathclose%
\pgfusepath{fill}%
\end{pgfscope}%
\begin{pgfscope}%
\pgfpathrectangle{\pgfqpoint{1.249956in}{0.148611in}}{\pgfqpoint{7.122250in}{3.850000in}}%
\pgfusepath{clip}%
\pgfsetbuttcap%
\pgfsetmiterjoin%
\definecolor{currentfill}{rgb}{0.868512,0.944637,0.569089}%
\pgfsetfillcolor{currentfill}%
\pgfsetfillopacity{0.500000}%
\pgfsetlinewidth{0.000000pt}%
\definecolor{currentstroke}{rgb}{0.000000,0.000000,0.000000}%
\pgfsetstrokecolor{currentstroke}%
\pgfsetstrokeopacity{0.500000}%
\pgfsetdash{}{0pt}%
\pgfpathmoveto{\pgfqpoint{2.955284in}{3.823611in}}%
\pgfpathlineto{\pgfqpoint{3.690915in}{3.823611in}}%
\pgfpathlineto{\pgfqpoint{3.690915in}{3.617729in}}%
\pgfpathlineto{\pgfqpoint{2.955284in}{3.617729in}}%
\pgfpathclose%
\pgfusepath{fill}%
\end{pgfscope}%
\begin{pgfscope}%
\pgfpathrectangle{\pgfqpoint{1.249956in}{0.148611in}}{\pgfqpoint{7.122250in}{3.850000in}}%
\pgfusepath{clip}%
\pgfsetbuttcap%
\pgfsetmiterjoin%
\definecolor{currentfill}{rgb}{0.868512,0.944637,0.569089}%
\pgfsetfillcolor{currentfill}%
\pgfsetfillopacity{0.500000}%
\pgfsetlinewidth{0.000000pt}%
\definecolor{currentstroke}{rgb}{0.000000,0.000000,0.000000}%
\pgfsetstrokecolor{currentstroke}%
\pgfsetstrokeopacity{0.500000}%
\pgfsetdash{}{0pt}%
\pgfpathmoveto{\pgfqpoint{3.055597in}{3.411846in}}%
\pgfpathlineto{\pgfqpoint{4.326233in}{3.411846in}}%
\pgfpathlineto{\pgfqpoint{4.326233in}{3.205964in}}%
\pgfpathlineto{\pgfqpoint{3.055597in}{3.205964in}}%
\pgfpathclose%
\pgfusepath{fill}%
\end{pgfscope}%
\begin{pgfscope}%
\pgfpathrectangle{\pgfqpoint{1.249956in}{0.148611in}}{\pgfqpoint{7.122250in}{3.850000in}}%
\pgfusepath{clip}%
\pgfsetbuttcap%
\pgfsetmiterjoin%
\definecolor{currentfill}{rgb}{0.868512,0.944637,0.569089}%
\pgfsetfillcolor{currentfill}%
\pgfsetlinewidth{0.000000pt}%
\definecolor{currentstroke}{rgb}{0.000000,0.000000,0.000000}%
\pgfsetstrokecolor{currentstroke}%
\pgfsetstrokeopacity{0.000000}%
\pgfsetdash{}{0pt}%
\pgfpathmoveto{\pgfqpoint{2.353404in}{3.000082in}}%
\pgfpathlineto{\pgfqpoint{2.754657in}{3.000082in}}%
\pgfpathlineto{\pgfqpoint{2.754657in}{2.794199in}}%
\pgfpathlineto{\pgfqpoint{2.353404in}{2.794199in}}%
\pgfpathclose%
\pgfusepath{fill}%
\end{pgfscope}%
\begin{pgfscope}%
\pgfpathrectangle{\pgfqpoint{1.249956in}{0.148611in}}{\pgfqpoint{7.122250in}{3.850000in}}%
\pgfusepath{clip}%
\pgfsetbuttcap%
\pgfsetmiterjoin%
\definecolor{currentfill}{rgb}{0.868512,0.944637,0.569089}%
\pgfsetfillcolor{currentfill}%
\pgfsetfillopacity{0.500000}%
\pgfsetlinewidth{0.000000pt}%
\definecolor{currentstroke}{rgb}{0.000000,0.000000,0.000000}%
\pgfsetstrokecolor{currentstroke}%
\pgfsetstrokeopacity{0.500000}%
\pgfsetdash{}{0pt}%
\pgfpathmoveto{\pgfqpoint{4.760925in}{2.588317in}}%
\pgfpathlineto{\pgfqpoint{6.064999in}{2.588317in}}%
\pgfpathlineto{\pgfqpoint{6.064999in}{2.382435in}}%
\pgfpathlineto{\pgfqpoint{4.760925in}{2.382435in}}%
\pgfpathclose%
\pgfusepath{fill}%
\end{pgfscope}%
\begin{pgfscope}%
\pgfpathrectangle{\pgfqpoint{1.249956in}{0.148611in}}{\pgfqpoint{7.122250in}{3.850000in}}%
\pgfusepath{clip}%
\pgfsetbuttcap%
\pgfsetmiterjoin%
\definecolor{currentfill}{rgb}{0.868512,0.944637,0.569089}%
\pgfsetfillcolor{currentfill}%
\pgfsetfillopacity{0.500000}%
\pgfsetlinewidth{0.000000pt}%
\definecolor{currentstroke}{rgb}{0.000000,0.000000,0.000000}%
\pgfsetstrokecolor{currentstroke}%
\pgfsetstrokeopacity{0.500000}%
\pgfsetdash{}{0pt}%
\pgfpathmoveto{\pgfqpoint{5.931247in}{2.176552in}}%
\pgfpathlineto{\pgfqpoint{6.800630in}{2.176552in}}%
\pgfpathlineto{\pgfqpoint{6.800630in}{1.970670in}}%
\pgfpathlineto{\pgfqpoint{5.931247in}{1.970670in}}%
\pgfpathclose%
\pgfusepath{fill}%
\end{pgfscope}%
\begin{pgfscope}%
\pgfpathrectangle{\pgfqpoint{1.249956in}{0.148611in}}{\pgfqpoint{7.122250in}{3.850000in}}%
\pgfusepath{clip}%
\pgfsetbuttcap%
\pgfsetmiterjoin%
\definecolor{currentfill}{rgb}{0.868512,0.944637,0.569089}%
\pgfsetfillcolor{currentfill}%
\pgfsetlinewidth{0.000000pt}%
\definecolor{currentstroke}{rgb}{0.000000,0.000000,0.000000}%
\pgfsetstrokecolor{currentstroke}%
\pgfsetstrokeopacity{0.000000}%
\pgfsetdash{}{0pt}%
\pgfpathmoveto{\pgfqpoint{2.052463in}{1.764788in}}%
\pgfpathlineto{\pgfqpoint{2.721219in}{1.764788in}}%
\pgfpathlineto{\pgfqpoint{2.721219in}{1.558905in}}%
\pgfpathlineto{\pgfqpoint{2.052463in}{1.558905in}}%
\pgfpathclose%
\pgfusepath{fill}%
\end{pgfscope}%
\begin{pgfscope}%
\pgfpathrectangle{\pgfqpoint{1.249956in}{0.148611in}}{\pgfqpoint{7.122250in}{3.850000in}}%
\pgfusepath{clip}%
\pgfsetbuttcap%
\pgfsetmiterjoin%
\definecolor{currentfill}{rgb}{0.868512,0.944637,0.569089}%
\pgfsetfillcolor{currentfill}%
\pgfsetfillopacity{0.500000}%
\pgfsetlinewidth{0.000000pt}%
\definecolor{currentstroke}{rgb}{0.000000,0.000000,0.000000}%
\pgfsetstrokecolor{currentstroke}%
\pgfsetstrokeopacity{0.500000}%
\pgfsetdash{}{0pt}%
\pgfpathmoveto{\pgfqpoint{6.700317in}{1.353023in}}%
\pgfpathlineto{\pgfqpoint{7.034695in}{1.353023in}}%
\pgfpathlineto{\pgfqpoint{7.034695in}{1.147141in}}%
\pgfpathlineto{\pgfqpoint{6.700317in}{1.147141in}}%
\pgfpathclose%
\pgfusepath{fill}%
\end{pgfscope}%
\begin{pgfscope}%
\pgfpathrectangle{\pgfqpoint{1.249956in}{0.148611in}}{\pgfqpoint{7.122250in}{3.850000in}}%
\pgfusepath{clip}%
\pgfsetbuttcap%
\pgfsetmiterjoin%
\definecolor{currentfill}{rgb}{0.868512,0.944637,0.569089}%
\pgfsetfillcolor{currentfill}%
\pgfsetlinewidth{0.000000pt}%
\definecolor{currentstroke}{rgb}{0.000000,0.000000,0.000000}%
\pgfsetstrokecolor{currentstroke}%
\pgfsetstrokeopacity{0.000000}%
\pgfsetdash{}{0pt}%
\pgfpathmoveto{\pgfqpoint{2.186215in}{0.941258in}}%
\pgfpathlineto{\pgfqpoint{3.155911in}{0.941258in}}%
\pgfpathlineto{\pgfqpoint{3.155911in}{0.735376in}}%
\pgfpathlineto{\pgfqpoint{2.186215in}{0.735376in}}%
\pgfpathclose%
\pgfusepath{fill}%
\end{pgfscope}%
\begin{pgfscope}%
\pgfpathrectangle{\pgfqpoint{1.249956in}{0.148611in}}{\pgfqpoint{7.122250in}{3.850000in}}%
\pgfusepath{clip}%
\pgfsetbuttcap%
\pgfsetmiterjoin%
\definecolor{currentfill}{rgb}{0.868512,0.944637,0.569089}%
\pgfsetfillcolor{currentfill}%
\pgfsetlinewidth{0.000000pt}%
\definecolor{currentstroke}{rgb}{0.000000,0.000000,0.000000}%
\pgfsetstrokecolor{currentstroke}%
\pgfsetstrokeopacity{0.000000}%
\pgfsetdash{}{0pt}%
\pgfpathmoveto{\pgfqpoint{2.620906in}{0.529493in}}%
\pgfpathlineto{\pgfqpoint{3.189348in}{0.529493in}}%
\pgfpathlineto{\pgfqpoint{3.189348in}{0.323611in}}%
\pgfpathlineto{\pgfqpoint{2.620906in}{0.323611in}}%
\pgfpathclose%
\pgfusepath{fill}%
\end{pgfscope}%
\begin{pgfscope}%
\pgfpathrectangle{\pgfqpoint{1.249956in}{0.148611in}}{\pgfqpoint{7.122250in}{3.850000in}}%
\pgfusepath{clip}%
\pgfsetbuttcap%
\pgfsetmiterjoin%
\definecolor{currentfill}{rgb}{0.997078,0.998770,0.745021}%
\pgfsetfillcolor{currentfill}%
\pgfsetfillopacity{0.500000}%
\pgfsetlinewidth{0.000000pt}%
\definecolor{currentstroke}{rgb}{0.000000,0.000000,0.000000}%
\pgfsetstrokecolor{currentstroke}%
\pgfsetstrokeopacity{0.500000}%
\pgfsetdash{}{0pt}%
\pgfpathmoveto{\pgfqpoint{3.690915in}{3.823611in}}%
\pgfpathlineto{\pgfqpoint{4.861238in}{3.823611in}}%
\pgfpathlineto{\pgfqpoint{4.861238in}{3.617729in}}%
\pgfpathlineto{\pgfqpoint{3.690915in}{3.617729in}}%
\pgfpathclose%
\pgfusepath{fill}%
\end{pgfscope}%
\begin{pgfscope}%
\pgfpathrectangle{\pgfqpoint{1.249956in}{0.148611in}}{\pgfqpoint{7.122250in}{3.850000in}}%
\pgfusepath{clip}%
\pgfsetbuttcap%
\pgfsetmiterjoin%
\definecolor{currentfill}{rgb}{0.997078,0.998770,0.745021}%
\pgfsetfillcolor{currentfill}%
\pgfsetfillopacity{0.500000}%
\pgfsetlinewidth{0.000000pt}%
\definecolor{currentstroke}{rgb}{0.000000,0.000000,0.000000}%
\pgfsetstrokecolor{currentstroke}%
\pgfsetstrokeopacity{0.500000}%
\pgfsetdash{}{0pt}%
\pgfpathmoveto{\pgfqpoint{4.326233in}{3.411846in}}%
\pgfpathlineto{\pgfqpoint{5.162178in}{3.411846in}}%
\pgfpathlineto{\pgfqpoint{5.162178in}{3.205964in}}%
\pgfpathlineto{\pgfqpoint{4.326233in}{3.205964in}}%
\pgfpathclose%
\pgfusepath{fill}%
\end{pgfscope}%
\begin{pgfscope}%
\pgfpathrectangle{\pgfqpoint{1.249956in}{0.148611in}}{\pgfqpoint{7.122250in}{3.850000in}}%
\pgfusepath{clip}%
\pgfsetbuttcap%
\pgfsetmiterjoin%
\definecolor{currentfill}{rgb}{0.997078,0.998770,0.745021}%
\pgfsetfillcolor{currentfill}%
\pgfsetlinewidth{0.000000pt}%
\definecolor{currentstroke}{rgb}{0.000000,0.000000,0.000000}%
\pgfsetstrokecolor{currentstroke}%
\pgfsetstrokeopacity{0.000000}%
\pgfsetdash{}{0pt}%
\pgfpathmoveto{\pgfqpoint{2.754657in}{3.000082in}}%
\pgfpathlineto{\pgfqpoint{3.389975in}{3.000082in}}%
\pgfpathlineto{\pgfqpoint{3.389975in}{2.794199in}}%
\pgfpathlineto{\pgfqpoint{2.754657in}{2.794199in}}%
\pgfpathclose%
\pgfusepath{fill}%
\end{pgfscope}%
\begin{pgfscope}%
\pgfpathrectangle{\pgfqpoint{1.249956in}{0.148611in}}{\pgfqpoint{7.122250in}{3.850000in}}%
\pgfusepath{clip}%
\pgfsetbuttcap%
\pgfsetmiterjoin%
\definecolor{currentfill}{rgb}{0.997078,0.998770,0.745021}%
\pgfsetfillcolor{currentfill}%
\pgfsetfillopacity{0.500000}%
\pgfsetlinewidth{0.000000pt}%
\definecolor{currentstroke}{rgb}{0.000000,0.000000,0.000000}%
\pgfsetstrokecolor{currentstroke}%
\pgfsetstrokeopacity{0.500000}%
\pgfsetdash{}{0pt}%
\pgfpathmoveto{\pgfqpoint{6.064999in}{2.588317in}}%
\pgfpathlineto{\pgfqpoint{6.733755in}{2.588317in}}%
\pgfpathlineto{\pgfqpoint{6.733755in}{2.382435in}}%
\pgfpathlineto{\pgfqpoint{6.064999in}{2.382435in}}%
\pgfpathclose%
\pgfusepath{fill}%
\end{pgfscope}%
\begin{pgfscope}%
\pgfpathrectangle{\pgfqpoint{1.249956in}{0.148611in}}{\pgfqpoint{7.122250in}{3.850000in}}%
\pgfusepath{clip}%
\pgfsetbuttcap%
\pgfsetmiterjoin%
\definecolor{currentfill}{rgb}{0.997078,0.998770,0.745021}%
\pgfsetfillcolor{currentfill}%
\pgfsetfillopacity{0.500000}%
\pgfsetlinewidth{0.000000pt}%
\definecolor{currentstroke}{rgb}{0.000000,0.000000,0.000000}%
\pgfsetstrokecolor{currentstroke}%
\pgfsetstrokeopacity{0.500000}%
\pgfsetdash{}{0pt}%
\pgfpathmoveto{\pgfqpoint{6.800630in}{2.176552in}}%
\pgfpathlineto{\pgfqpoint{7.335635in}{2.176552in}}%
\pgfpathlineto{\pgfqpoint{7.335635in}{1.970670in}}%
\pgfpathlineto{\pgfqpoint{6.800630in}{1.970670in}}%
\pgfpathclose%
\pgfusepath{fill}%
\end{pgfscope}%
\begin{pgfscope}%
\pgfpathrectangle{\pgfqpoint{1.249956in}{0.148611in}}{\pgfqpoint{7.122250in}{3.850000in}}%
\pgfusepath{clip}%
\pgfsetbuttcap%
\pgfsetmiterjoin%
\definecolor{currentfill}{rgb}{0.997078,0.998770,0.745021}%
\pgfsetfillcolor{currentfill}%
\pgfsetlinewidth{0.000000pt}%
\definecolor{currentstroke}{rgb}{0.000000,0.000000,0.000000}%
\pgfsetstrokecolor{currentstroke}%
\pgfsetstrokeopacity{0.000000}%
\pgfsetdash{}{0pt}%
\pgfpathmoveto{\pgfqpoint{2.721219in}{1.764788in}}%
\pgfpathlineto{\pgfqpoint{3.557164in}{1.764788in}}%
\pgfpathlineto{\pgfqpoint{3.557164in}{1.558905in}}%
\pgfpathlineto{\pgfqpoint{2.721219in}{1.558905in}}%
\pgfpathclose%
\pgfusepath{fill}%
\end{pgfscope}%
\begin{pgfscope}%
\pgfpathrectangle{\pgfqpoint{1.249956in}{0.148611in}}{\pgfqpoint{7.122250in}{3.850000in}}%
\pgfusepath{clip}%
\pgfsetbuttcap%
\pgfsetmiterjoin%
\definecolor{currentfill}{rgb}{0.997078,0.998770,0.745021}%
\pgfsetfillcolor{currentfill}%
\pgfsetfillopacity{0.500000}%
\pgfsetlinewidth{0.000000pt}%
\definecolor{currentstroke}{rgb}{0.000000,0.000000,0.000000}%
\pgfsetstrokecolor{currentstroke}%
\pgfsetstrokeopacity{0.500000}%
\pgfsetdash{}{0pt}%
\pgfpathmoveto{\pgfqpoint{7.034695in}{1.353023in}}%
\pgfpathlineto{\pgfqpoint{7.703451in}{1.353023in}}%
\pgfpathlineto{\pgfqpoint{7.703451in}{1.147141in}}%
\pgfpathlineto{\pgfqpoint{7.034695in}{1.147141in}}%
\pgfpathclose%
\pgfusepath{fill}%
\end{pgfscope}%
\begin{pgfscope}%
\pgfpathrectangle{\pgfqpoint{1.249956in}{0.148611in}}{\pgfqpoint{7.122250in}{3.850000in}}%
\pgfusepath{clip}%
\pgfsetbuttcap%
\pgfsetmiterjoin%
\definecolor{currentfill}{rgb}{0.997078,0.998770,0.745021}%
\pgfsetfillcolor{currentfill}%
\pgfsetlinewidth{0.000000pt}%
\definecolor{currentstroke}{rgb}{0.000000,0.000000,0.000000}%
\pgfsetstrokecolor{currentstroke}%
\pgfsetstrokeopacity{0.000000}%
\pgfsetdash{}{0pt}%
\pgfpathmoveto{\pgfqpoint{3.155911in}{0.941258in}}%
\pgfpathlineto{\pgfqpoint{4.259358in}{0.941258in}}%
\pgfpathlineto{\pgfqpoint{4.259358in}{0.735376in}}%
\pgfpathlineto{\pgfqpoint{3.155911in}{0.735376in}}%
\pgfpathclose%
\pgfusepath{fill}%
\end{pgfscope}%
\begin{pgfscope}%
\pgfpathrectangle{\pgfqpoint{1.249956in}{0.148611in}}{\pgfqpoint{7.122250in}{3.850000in}}%
\pgfusepath{clip}%
\pgfsetbuttcap%
\pgfsetmiterjoin%
\definecolor{currentfill}{rgb}{0.997078,0.998770,0.745021}%
\pgfsetfillcolor{currentfill}%
\pgfsetlinewidth{0.000000pt}%
\definecolor{currentstroke}{rgb}{0.000000,0.000000,0.000000}%
\pgfsetstrokecolor{currentstroke}%
\pgfsetstrokeopacity{0.000000}%
\pgfsetdash{}{0pt}%
\pgfpathmoveto{\pgfqpoint{3.189348in}{0.529493in}}%
\pgfpathlineto{\pgfqpoint{3.858104in}{0.529493in}}%
\pgfpathlineto{\pgfqpoint{3.858104in}{0.323611in}}%
\pgfpathlineto{\pgfqpoint{3.189348in}{0.323611in}}%
\pgfpathclose%
\pgfusepath{fill}%
\end{pgfscope}%
\begin{pgfscope}%
\pgfpathrectangle{\pgfqpoint{1.249956in}{0.148611in}}{\pgfqpoint{7.122250in}{3.850000in}}%
\pgfusepath{clip}%
\pgfsetbuttcap%
\pgfsetmiterjoin%
\definecolor{currentfill}{rgb}{0.996540,0.892734,0.569089}%
\pgfsetfillcolor{currentfill}%
\pgfsetfillopacity{0.500000}%
\pgfsetlinewidth{0.000000pt}%
\definecolor{currentstroke}{rgb}{0.000000,0.000000,0.000000}%
\pgfsetstrokecolor{currentstroke}%
\pgfsetstrokeopacity{0.500000}%
\pgfsetdash{}{0pt}%
\pgfpathmoveto{\pgfqpoint{4.861238in}{3.823611in}}%
\pgfpathlineto{\pgfqpoint{5.730621in}{3.823611in}}%
\pgfpathlineto{\pgfqpoint{5.730621in}{3.617729in}}%
\pgfpathlineto{\pgfqpoint{4.861238in}{3.617729in}}%
\pgfpathclose%
\pgfusepath{fill}%
\end{pgfscope}%
\begin{pgfscope}%
\pgfpathrectangle{\pgfqpoint{1.249956in}{0.148611in}}{\pgfqpoint{7.122250in}{3.850000in}}%
\pgfusepath{clip}%
\pgfsetbuttcap%
\pgfsetmiterjoin%
\definecolor{currentfill}{rgb}{0.996540,0.892734,0.569089}%
\pgfsetfillcolor{currentfill}%
\pgfsetfillopacity{0.500000}%
\pgfsetlinewidth{0.000000pt}%
\definecolor{currentstroke}{rgb}{0.000000,0.000000,0.000000}%
\pgfsetstrokecolor{currentstroke}%
\pgfsetstrokeopacity{0.500000}%
\pgfsetdash{}{0pt}%
\pgfpathmoveto{\pgfqpoint{5.162178in}{3.411846in}}%
\pgfpathlineto{\pgfqpoint{6.399377in}{3.411846in}}%
\pgfpathlineto{\pgfqpoint{6.399377in}{3.205964in}}%
\pgfpathlineto{\pgfqpoint{5.162178in}{3.205964in}}%
\pgfpathclose%
\pgfusepath{fill}%
\end{pgfscope}%
\begin{pgfscope}%
\pgfpathrectangle{\pgfqpoint{1.249956in}{0.148611in}}{\pgfqpoint{7.122250in}{3.850000in}}%
\pgfusepath{clip}%
\pgfsetbuttcap%
\pgfsetmiterjoin%
\definecolor{currentfill}{rgb}{0.996540,0.892734,0.569089}%
\pgfsetfillcolor{currentfill}%
\pgfsetlinewidth{0.000000pt}%
\definecolor{currentstroke}{rgb}{0.000000,0.000000,0.000000}%
\pgfsetstrokecolor{currentstroke}%
\pgfsetstrokeopacity{0.000000}%
\pgfsetdash{}{0pt}%
\pgfpathmoveto{\pgfqpoint{3.389975in}{3.000082in}}%
\pgfpathlineto{\pgfqpoint{3.891542in}{3.000082in}}%
\pgfpathlineto{\pgfqpoint{3.891542in}{2.794199in}}%
\pgfpathlineto{\pgfqpoint{3.389975in}{2.794199in}}%
\pgfpathclose%
\pgfusepath{fill}%
\end{pgfscope}%
\begin{pgfscope}%
\pgfpathrectangle{\pgfqpoint{1.249956in}{0.148611in}}{\pgfqpoint{7.122250in}{3.850000in}}%
\pgfusepath{clip}%
\pgfsetbuttcap%
\pgfsetmiterjoin%
\definecolor{currentfill}{rgb}{0.996540,0.892734,0.569089}%
\pgfsetfillcolor{currentfill}%
\pgfsetfillopacity{0.500000}%
\pgfsetlinewidth{0.000000pt}%
\definecolor{currentstroke}{rgb}{0.000000,0.000000,0.000000}%
\pgfsetstrokecolor{currentstroke}%
\pgfsetstrokeopacity{0.500000}%
\pgfsetdash{}{0pt}%
\pgfpathmoveto{\pgfqpoint{6.733755in}{2.588317in}}%
\pgfpathlineto{\pgfqpoint{7.502824in}{2.588317in}}%
\pgfpathlineto{\pgfqpoint{7.502824in}{2.382435in}}%
\pgfpathlineto{\pgfqpoint{6.733755in}{2.382435in}}%
\pgfpathclose%
\pgfusepath{fill}%
\end{pgfscope}%
\begin{pgfscope}%
\pgfpathrectangle{\pgfqpoint{1.249956in}{0.148611in}}{\pgfqpoint{7.122250in}{3.850000in}}%
\pgfusepath{clip}%
\pgfsetbuttcap%
\pgfsetmiterjoin%
\definecolor{currentfill}{rgb}{0.996540,0.892734,0.569089}%
\pgfsetfillcolor{currentfill}%
\pgfsetfillopacity{0.500000}%
\pgfsetlinewidth{0.000000pt}%
\definecolor{currentstroke}{rgb}{0.000000,0.000000,0.000000}%
\pgfsetstrokecolor{currentstroke}%
\pgfsetstrokeopacity{0.500000}%
\pgfsetdash{}{0pt}%
\pgfpathmoveto{\pgfqpoint{7.335635in}{2.176552in}}%
\pgfpathlineto{\pgfqpoint{7.736888in}{2.176552in}}%
\pgfpathlineto{\pgfqpoint{7.736888in}{1.970670in}}%
\pgfpathlineto{\pgfqpoint{7.335635in}{1.970670in}}%
\pgfpathclose%
\pgfusepath{fill}%
\end{pgfscope}%
\begin{pgfscope}%
\pgfpathrectangle{\pgfqpoint{1.249956in}{0.148611in}}{\pgfqpoint{7.122250in}{3.850000in}}%
\pgfusepath{clip}%
\pgfsetbuttcap%
\pgfsetmiterjoin%
\definecolor{currentfill}{rgb}{0.996540,0.892734,0.569089}%
\pgfsetfillcolor{currentfill}%
\pgfsetlinewidth{0.000000pt}%
\definecolor{currentstroke}{rgb}{0.000000,0.000000,0.000000}%
\pgfsetstrokecolor{currentstroke}%
\pgfsetstrokeopacity{0.000000}%
\pgfsetdash{}{0pt}%
\pgfpathmoveto{\pgfqpoint{3.557164in}{1.764788in}}%
\pgfpathlineto{\pgfqpoint{4.426547in}{1.764788in}}%
\pgfpathlineto{\pgfqpoint{4.426547in}{1.558905in}}%
\pgfpathlineto{\pgfqpoint{3.557164in}{1.558905in}}%
\pgfpathclose%
\pgfusepath{fill}%
\end{pgfscope}%
\begin{pgfscope}%
\pgfpathrectangle{\pgfqpoint{1.249956in}{0.148611in}}{\pgfqpoint{7.122250in}{3.850000in}}%
\pgfusepath{clip}%
\pgfsetbuttcap%
\pgfsetmiterjoin%
\definecolor{currentfill}{rgb}{0.996540,0.892734,0.569089}%
\pgfsetfillcolor{currentfill}%
\pgfsetfillopacity{0.500000}%
\pgfsetlinewidth{0.000000pt}%
\definecolor{currentstroke}{rgb}{0.000000,0.000000,0.000000}%
\pgfsetstrokecolor{currentstroke}%
\pgfsetstrokeopacity{0.500000}%
\pgfsetdash{}{0pt}%
\pgfpathmoveto{\pgfqpoint{7.703451in}{1.353023in}}%
\pgfpathlineto{\pgfqpoint{8.071266in}{1.353023in}}%
\pgfpathlineto{\pgfqpoint{8.071266in}{1.147141in}}%
\pgfpathlineto{\pgfqpoint{7.703451in}{1.147141in}}%
\pgfpathclose%
\pgfusepath{fill}%
\end{pgfscope}%
\begin{pgfscope}%
\pgfpathrectangle{\pgfqpoint{1.249956in}{0.148611in}}{\pgfqpoint{7.122250in}{3.850000in}}%
\pgfusepath{clip}%
\pgfsetbuttcap%
\pgfsetmiterjoin%
\definecolor{currentfill}{rgb}{0.996540,0.892734,0.569089}%
\pgfsetfillcolor{currentfill}%
\pgfsetlinewidth{0.000000pt}%
\definecolor{currentstroke}{rgb}{0.000000,0.000000,0.000000}%
\pgfsetstrokecolor{currentstroke}%
\pgfsetstrokeopacity{0.000000}%
\pgfsetdash{}{0pt}%
\pgfpathmoveto{\pgfqpoint{4.259358in}{0.941258in}}%
\pgfpathlineto{\pgfqpoint{5.295929in}{0.941258in}}%
\pgfpathlineto{\pgfqpoint{5.295929in}{0.735376in}}%
\pgfpathlineto{\pgfqpoint{4.259358in}{0.735376in}}%
\pgfpathclose%
\pgfusepath{fill}%
\end{pgfscope}%
\begin{pgfscope}%
\pgfpathrectangle{\pgfqpoint{1.249956in}{0.148611in}}{\pgfqpoint{7.122250in}{3.850000in}}%
\pgfusepath{clip}%
\pgfsetbuttcap%
\pgfsetmiterjoin%
\definecolor{currentfill}{rgb}{0.996540,0.892734,0.569089}%
\pgfsetfillcolor{currentfill}%
\pgfsetlinewidth{0.000000pt}%
\definecolor{currentstroke}{rgb}{0.000000,0.000000,0.000000}%
\pgfsetstrokecolor{currentstroke}%
\pgfsetstrokeopacity{0.000000}%
\pgfsetdash{}{0pt}%
\pgfpathmoveto{\pgfqpoint{3.858104in}{0.529493in}}%
\pgfpathlineto{\pgfqpoint{4.928114in}{0.529493in}}%
\pgfpathlineto{\pgfqpoint{4.928114in}{0.323611in}}%
\pgfpathlineto{\pgfqpoint{3.858104in}{0.323611in}}%
\pgfpathclose%
\pgfusepath{fill}%
\end{pgfscope}%
\begin{pgfscope}%
\pgfpathrectangle{\pgfqpoint{1.249956in}{0.148611in}}{\pgfqpoint{7.122250in}{3.850000in}}%
\pgfusepath{clip}%
\pgfsetbuttcap%
\pgfsetmiterjoin%
\definecolor{currentfill}{rgb}{0.993156,0.732334,0.422376}%
\pgfsetfillcolor{currentfill}%
\pgfsetfillopacity{0.500000}%
\pgfsetlinewidth{0.000000pt}%
\definecolor{currentstroke}{rgb}{0.000000,0.000000,0.000000}%
\pgfsetstrokecolor{currentstroke}%
\pgfsetstrokeopacity{0.500000}%
\pgfsetdash{}{0pt}%
\pgfpathmoveto{\pgfqpoint{5.730621in}{3.823611in}}%
\pgfpathlineto{\pgfqpoint{6.499690in}{3.823611in}}%
\pgfpathlineto{\pgfqpoint{6.499690in}{3.617729in}}%
\pgfpathlineto{\pgfqpoint{5.730621in}{3.617729in}}%
\pgfpathclose%
\pgfusepath{fill}%
\end{pgfscope}%
\begin{pgfscope}%
\pgfpathrectangle{\pgfqpoint{1.249956in}{0.148611in}}{\pgfqpoint{7.122250in}{3.850000in}}%
\pgfusepath{clip}%
\pgfsetbuttcap%
\pgfsetmiterjoin%
\definecolor{currentfill}{rgb}{0.993156,0.732334,0.422376}%
\pgfsetfillcolor{currentfill}%
\pgfsetfillopacity{0.500000}%
\pgfsetlinewidth{0.000000pt}%
\definecolor{currentstroke}{rgb}{0.000000,0.000000,0.000000}%
\pgfsetstrokecolor{currentstroke}%
\pgfsetstrokeopacity{0.500000}%
\pgfsetdash{}{0pt}%
\pgfpathmoveto{\pgfqpoint{6.399377in}{3.411846in}}%
\pgfpathlineto{\pgfqpoint{7.502824in}{3.411846in}}%
\pgfpathlineto{\pgfqpoint{7.502824in}{3.205964in}}%
\pgfpathlineto{\pgfqpoint{6.399377in}{3.205964in}}%
\pgfpathclose%
\pgfusepath{fill}%
\end{pgfscope}%
\begin{pgfscope}%
\pgfpathrectangle{\pgfqpoint{1.249956in}{0.148611in}}{\pgfqpoint{7.122250in}{3.850000in}}%
\pgfusepath{clip}%
\pgfsetbuttcap%
\pgfsetmiterjoin%
\definecolor{currentfill}{rgb}{0.993156,0.732334,0.422376}%
\pgfsetfillcolor{currentfill}%
\pgfsetlinewidth{0.000000pt}%
\definecolor{currentstroke}{rgb}{0.000000,0.000000,0.000000}%
\pgfsetstrokecolor{currentstroke}%
\pgfsetstrokeopacity{0.000000}%
\pgfsetdash{}{0pt}%
\pgfpathmoveto{\pgfqpoint{3.891542in}{3.000082in}}%
\pgfpathlineto{\pgfqpoint{5.295929in}{3.000082in}}%
\pgfpathlineto{\pgfqpoint{5.295929in}{2.794199in}}%
\pgfpathlineto{\pgfqpoint{3.891542in}{2.794199in}}%
\pgfpathclose%
\pgfusepath{fill}%
\end{pgfscope}%
\begin{pgfscope}%
\pgfpathrectangle{\pgfqpoint{1.249956in}{0.148611in}}{\pgfqpoint{7.122250in}{3.850000in}}%
\pgfusepath{clip}%
\pgfsetbuttcap%
\pgfsetmiterjoin%
\definecolor{currentfill}{rgb}{0.993156,0.732334,0.422376}%
\pgfsetfillcolor{currentfill}%
\pgfsetfillopacity{0.500000}%
\pgfsetlinewidth{0.000000pt}%
\definecolor{currentstroke}{rgb}{0.000000,0.000000,0.000000}%
\pgfsetstrokecolor{currentstroke}%
\pgfsetstrokeopacity{0.500000}%
\pgfsetdash{}{0pt}%
\pgfpathmoveto{\pgfqpoint{7.502824in}{2.588317in}}%
\pgfpathlineto{\pgfqpoint{8.037828in}{2.588317in}}%
\pgfpathlineto{\pgfqpoint{8.037828in}{2.382435in}}%
\pgfpathlineto{\pgfqpoint{7.502824in}{2.382435in}}%
\pgfpathclose%
\pgfusepath{fill}%
\end{pgfscope}%
\begin{pgfscope}%
\pgfpathrectangle{\pgfqpoint{1.249956in}{0.148611in}}{\pgfqpoint{7.122250in}{3.850000in}}%
\pgfusepath{clip}%
\pgfsetbuttcap%
\pgfsetmiterjoin%
\definecolor{currentfill}{rgb}{0.993156,0.732334,0.422376}%
\pgfsetfillcolor{currentfill}%
\pgfsetfillopacity{0.500000}%
\pgfsetlinewidth{0.000000pt}%
\definecolor{currentstroke}{rgb}{0.000000,0.000000,0.000000}%
\pgfsetstrokecolor{currentstroke}%
\pgfsetstrokeopacity{0.500000}%
\pgfsetdash{}{0pt}%
\pgfpathmoveto{\pgfqpoint{7.736888in}{2.176552in}}%
\pgfpathlineto{\pgfqpoint{8.037828in}{2.176552in}}%
\pgfpathlineto{\pgfqpoint{8.037828in}{1.970670in}}%
\pgfpathlineto{\pgfqpoint{7.736888in}{1.970670in}}%
\pgfpathclose%
\pgfusepath{fill}%
\end{pgfscope}%
\begin{pgfscope}%
\pgfpathrectangle{\pgfqpoint{1.249956in}{0.148611in}}{\pgfqpoint{7.122250in}{3.850000in}}%
\pgfusepath{clip}%
\pgfsetbuttcap%
\pgfsetmiterjoin%
\definecolor{currentfill}{rgb}{0.993156,0.732334,0.422376}%
\pgfsetfillcolor{currentfill}%
\pgfsetlinewidth{0.000000pt}%
\definecolor{currentstroke}{rgb}{0.000000,0.000000,0.000000}%
\pgfsetstrokecolor{currentstroke}%
\pgfsetstrokeopacity{0.000000}%
\pgfsetdash{}{0pt}%
\pgfpathmoveto{\pgfqpoint{4.426547in}{1.764788in}}%
\pgfpathlineto{\pgfqpoint{5.496556in}{1.764788in}}%
\pgfpathlineto{\pgfqpoint{5.496556in}{1.558905in}}%
\pgfpathlineto{\pgfqpoint{4.426547in}{1.558905in}}%
\pgfpathclose%
\pgfusepath{fill}%
\end{pgfscope}%
\begin{pgfscope}%
\pgfpathrectangle{\pgfqpoint{1.249956in}{0.148611in}}{\pgfqpoint{7.122250in}{3.850000in}}%
\pgfusepath{clip}%
\pgfsetbuttcap%
\pgfsetmiterjoin%
\definecolor{currentfill}{rgb}{0.993156,0.732334,0.422376}%
\pgfsetfillcolor{currentfill}%
\pgfsetfillopacity{0.500000}%
\pgfsetlinewidth{0.000000pt}%
\definecolor{currentstroke}{rgb}{0.000000,0.000000,0.000000}%
\pgfsetstrokecolor{currentstroke}%
\pgfsetstrokeopacity{0.500000}%
\pgfsetdash{}{0pt}%
\pgfpathmoveto{\pgfqpoint{8.071266in}{1.353023in}}%
\pgfpathlineto{\pgfqpoint{8.171580in}{1.353023in}}%
\pgfpathlineto{\pgfqpoint{8.171580in}{1.147141in}}%
\pgfpathlineto{\pgfqpoint{8.071266in}{1.147141in}}%
\pgfpathclose%
\pgfusepath{fill}%
\end{pgfscope}%
\begin{pgfscope}%
\pgfpathrectangle{\pgfqpoint{1.249956in}{0.148611in}}{\pgfqpoint{7.122250in}{3.850000in}}%
\pgfusepath{clip}%
\pgfsetbuttcap%
\pgfsetmiterjoin%
\definecolor{currentfill}{rgb}{0.993156,0.732334,0.422376}%
\pgfsetfillcolor{currentfill}%
\pgfsetlinewidth{0.000000pt}%
\definecolor{currentstroke}{rgb}{0.000000,0.000000,0.000000}%
\pgfsetstrokecolor{currentstroke}%
\pgfsetstrokeopacity{0.000000}%
\pgfsetdash{}{0pt}%
\pgfpathmoveto{\pgfqpoint{5.295929in}{0.941258in}}%
\pgfpathlineto{\pgfqpoint{6.299063in}{0.941258in}}%
\pgfpathlineto{\pgfqpoint{6.299063in}{0.735376in}}%
\pgfpathlineto{\pgfqpoint{5.295929in}{0.735376in}}%
\pgfpathclose%
\pgfusepath{fill}%
\end{pgfscope}%
\begin{pgfscope}%
\pgfpathrectangle{\pgfqpoint{1.249956in}{0.148611in}}{\pgfqpoint{7.122250in}{3.850000in}}%
\pgfusepath{clip}%
\pgfsetbuttcap%
\pgfsetmiterjoin%
\definecolor{currentfill}{rgb}{0.993156,0.732334,0.422376}%
\pgfsetfillcolor{currentfill}%
\pgfsetlinewidth{0.000000pt}%
\definecolor{currentstroke}{rgb}{0.000000,0.000000,0.000000}%
\pgfsetstrokecolor{currentstroke}%
\pgfsetstrokeopacity{0.000000}%
\pgfsetdash{}{0pt}%
\pgfpathmoveto{\pgfqpoint{4.928114in}{0.529493in}}%
\pgfpathlineto{\pgfqpoint{5.764059in}{0.529493in}}%
\pgfpathlineto{\pgfqpoint{5.764059in}{0.323611in}}%
\pgfpathlineto{\pgfqpoint{4.928114in}{0.323611in}}%
\pgfpathclose%
\pgfusepath{fill}%
\end{pgfscope}%
\begin{pgfscope}%
\pgfpathrectangle{\pgfqpoint{1.249956in}{0.148611in}}{\pgfqpoint{7.122250in}{3.850000in}}%
\pgfusepath{clip}%
\pgfsetbuttcap%
\pgfsetmiterjoin%
\definecolor{currentfill}{rgb}{0.969319,0.517416,0.304268}%
\pgfsetfillcolor{currentfill}%
\pgfsetfillopacity{0.500000}%
\pgfsetlinewidth{0.000000pt}%
\definecolor{currentstroke}{rgb}{0.000000,0.000000,0.000000}%
\pgfsetstrokecolor{currentstroke}%
\pgfsetstrokeopacity{0.500000}%
\pgfsetdash{}{0pt}%
\pgfpathmoveto{\pgfqpoint{6.499690in}{3.823611in}}%
\pgfpathlineto{\pgfqpoint{7.201884in}{3.823611in}}%
\pgfpathlineto{\pgfqpoint{7.201884in}{3.617729in}}%
\pgfpathlineto{\pgfqpoint{6.499690in}{3.617729in}}%
\pgfpathclose%
\pgfusepath{fill}%
\end{pgfscope}%
\begin{pgfscope}%
\pgfpathrectangle{\pgfqpoint{1.249956in}{0.148611in}}{\pgfqpoint{7.122250in}{3.850000in}}%
\pgfusepath{clip}%
\pgfsetbuttcap%
\pgfsetmiterjoin%
\definecolor{currentfill}{rgb}{0.969319,0.517416,0.304268}%
\pgfsetfillcolor{currentfill}%
\pgfsetfillopacity{0.500000}%
\pgfsetlinewidth{0.000000pt}%
\definecolor{currentstroke}{rgb}{0.000000,0.000000,0.000000}%
\pgfsetstrokecolor{currentstroke}%
\pgfsetstrokeopacity{0.500000}%
\pgfsetdash{}{0pt}%
\pgfpathmoveto{\pgfqpoint{7.502824in}{3.411846in}}%
\pgfpathlineto{\pgfqpoint{7.970953in}{3.411846in}}%
\pgfpathlineto{\pgfqpoint{7.970953in}{3.205964in}}%
\pgfpathlineto{\pgfqpoint{7.502824in}{3.205964in}}%
\pgfpathclose%
\pgfusepath{fill}%
\end{pgfscope}%
\begin{pgfscope}%
\pgfpathrectangle{\pgfqpoint{1.249956in}{0.148611in}}{\pgfqpoint{7.122250in}{3.850000in}}%
\pgfusepath{clip}%
\pgfsetbuttcap%
\pgfsetmiterjoin%
\definecolor{currentfill}{rgb}{0.969319,0.517416,0.304268}%
\pgfsetfillcolor{currentfill}%
\pgfsetlinewidth{0.000000pt}%
\definecolor{currentstroke}{rgb}{0.000000,0.000000,0.000000}%
\pgfsetstrokecolor{currentstroke}%
\pgfsetstrokeopacity{0.000000}%
\pgfsetdash{}{0pt}%
\pgfpathmoveto{\pgfqpoint{5.295929in}{3.000082in}}%
\pgfpathlineto{\pgfqpoint{6.867506in}{3.000082in}}%
\pgfpathlineto{\pgfqpoint{6.867506in}{2.794199in}}%
\pgfpathlineto{\pgfqpoint{5.295929in}{2.794199in}}%
\pgfpathclose%
\pgfusepath{fill}%
\end{pgfscope}%
\begin{pgfscope}%
\pgfpathrectangle{\pgfqpoint{1.249956in}{0.148611in}}{\pgfqpoint{7.122250in}{3.850000in}}%
\pgfusepath{clip}%
\pgfsetbuttcap%
\pgfsetmiterjoin%
\definecolor{currentfill}{rgb}{0.969319,0.517416,0.304268}%
\pgfsetfillcolor{currentfill}%
\pgfsetfillopacity{0.500000}%
\pgfsetlinewidth{0.000000pt}%
\definecolor{currentstroke}{rgb}{0.000000,0.000000,0.000000}%
\pgfsetstrokecolor{currentstroke}%
\pgfsetstrokeopacity{0.500000}%
\pgfsetdash{}{0pt}%
\pgfpathmoveto{\pgfqpoint{8.037828in}{2.588317in}}%
\pgfpathlineto{\pgfqpoint{8.338769in}{2.588317in}}%
\pgfpathlineto{\pgfqpoint{8.338769in}{2.382435in}}%
\pgfpathlineto{\pgfqpoint{8.037828in}{2.382435in}}%
\pgfpathclose%
\pgfusepath{fill}%
\end{pgfscope}%
\begin{pgfscope}%
\pgfpathrectangle{\pgfqpoint{1.249956in}{0.148611in}}{\pgfqpoint{7.122250in}{3.850000in}}%
\pgfusepath{clip}%
\pgfsetbuttcap%
\pgfsetmiterjoin%
\definecolor{currentfill}{rgb}{0.969319,0.517416,0.304268}%
\pgfsetfillcolor{currentfill}%
\pgfsetfillopacity{0.500000}%
\pgfsetlinewidth{0.000000pt}%
\definecolor{currentstroke}{rgb}{0.000000,0.000000,0.000000}%
\pgfsetstrokecolor{currentstroke}%
\pgfsetstrokeopacity{0.500000}%
\pgfsetdash{}{0pt}%
\pgfpathmoveto{\pgfqpoint{8.037828in}{2.176552in}}%
\pgfpathlineto{\pgfqpoint{8.104704in}{2.176552in}}%
\pgfpathlineto{\pgfqpoint{8.104704in}{1.970670in}}%
\pgfpathlineto{\pgfqpoint{8.037828in}{1.970670in}}%
\pgfpathclose%
\pgfusepath{fill}%
\end{pgfscope}%
\begin{pgfscope}%
\pgfpathrectangle{\pgfqpoint{1.249956in}{0.148611in}}{\pgfqpoint{7.122250in}{3.850000in}}%
\pgfusepath{clip}%
\pgfsetbuttcap%
\pgfsetmiterjoin%
\definecolor{currentfill}{rgb}{0.969319,0.517416,0.304268}%
\pgfsetfillcolor{currentfill}%
\pgfsetlinewidth{0.000000pt}%
\definecolor{currentstroke}{rgb}{0.000000,0.000000,0.000000}%
\pgfsetstrokecolor{currentstroke}%
\pgfsetstrokeopacity{0.000000}%
\pgfsetdash{}{0pt}%
\pgfpathmoveto{\pgfqpoint{5.496556in}{1.764788in}}%
\pgfpathlineto{\pgfqpoint{7.034695in}{1.764788in}}%
\pgfpathlineto{\pgfqpoint{7.034695in}{1.558905in}}%
\pgfpathlineto{\pgfqpoint{5.496556in}{1.558905in}}%
\pgfpathclose%
\pgfusepath{fill}%
\end{pgfscope}%
\begin{pgfscope}%
\pgfpathrectangle{\pgfqpoint{1.249956in}{0.148611in}}{\pgfqpoint{7.122250in}{3.850000in}}%
\pgfusepath{clip}%
\pgfsetbuttcap%
\pgfsetmiterjoin%
\definecolor{currentfill}{rgb}{0.969319,0.517416,0.304268}%
\pgfsetfillcolor{currentfill}%
\pgfsetfillopacity{0.500000}%
\pgfsetlinewidth{0.000000pt}%
\definecolor{currentstroke}{rgb}{0.000000,0.000000,0.000000}%
\pgfsetstrokecolor{currentstroke}%
\pgfsetstrokeopacity{0.500000}%
\pgfsetdash{}{0pt}%
\pgfpathmoveto{\pgfqpoint{8.171580in}{1.353023in}}%
\pgfpathlineto{\pgfqpoint{8.305331in}{1.353023in}}%
\pgfpathlineto{\pgfqpoint{8.305331in}{1.147141in}}%
\pgfpathlineto{\pgfqpoint{8.171580in}{1.147141in}}%
\pgfpathclose%
\pgfusepath{fill}%
\end{pgfscope}%
\begin{pgfscope}%
\pgfpathrectangle{\pgfqpoint{1.249956in}{0.148611in}}{\pgfqpoint{7.122250in}{3.850000in}}%
\pgfusepath{clip}%
\pgfsetbuttcap%
\pgfsetmiterjoin%
\definecolor{currentfill}{rgb}{0.969319,0.517416,0.304268}%
\pgfsetfillcolor{currentfill}%
\pgfsetlinewidth{0.000000pt}%
\definecolor{currentstroke}{rgb}{0.000000,0.000000,0.000000}%
\pgfsetstrokecolor{currentstroke}%
\pgfsetstrokeopacity{0.000000}%
\pgfsetdash{}{0pt}%
\pgfpathmoveto{\pgfqpoint{6.299063in}{0.941258in}}%
\pgfpathlineto{\pgfqpoint{7.569699in}{0.941258in}}%
\pgfpathlineto{\pgfqpoint{7.569699in}{0.735376in}}%
\pgfpathlineto{\pgfqpoint{6.299063in}{0.735376in}}%
\pgfpathclose%
\pgfusepath{fill}%
\end{pgfscope}%
\begin{pgfscope}%
\pgfpathrectangle{\pgfqpoint{1.249956in}{0.148611in}}{\pgfqpoint{7.122250in}{3.850000in}}%
\pgfusepath{clip}%
\pgfsetbuttcap%
\pgfsetmiterjoin%
\definecolor{currentfill}{rgb}{0.969319,0.517416,0.304268}%
\pgfsetfillcolor{currentfill}%
\pgfsetlinewidth{0.000000pt}%
\definecolor{currentstroke}{rgb}{0.000000,0.000000,0.000000}%
\pgfsetstrokecolor{currentstroke}%
\pgfsetstrokeopacity{0.000000}%
\pgfsetdash{}{0pt}%
\pgfpathmoveto{\pgfqpoint{5.764059in}{0.529493in}}%
\pgfpathlineto{\pgfqpoint{6.834068in}{0.529493in}}%
\pgfpathlineto{\pgfqpoint{6.834068in}{0.323611in}}%
\pgfpathlineto{\pgfqpoint{5.764059in}{0.323611in}}%
\pgfpathclose%
\pgfusepath{fill}%
\end{pgfscope}%
\begin{pgfscope}%
\pgfpathrectangle{\pgfqpoint{1.249956in}{0.148611in}}{\pgfqpoint{7.122250in}{3.850000in}}%
\pgfusepath{clip}%
\pgfsetbuttcap%
\pgfsetmiterjoin%
\definecolor{currentfill}{rgb}{0.898885,0.305498,0.206767}%
\pgfsetfillcolor{currentfill}%
\pgfsetfillopacity{0.500000}%
\pgfsetlinewidth{0.000000pt}%
\definecolor{currentstroke}{rgb}{0.000000,0.000000,0.000000}%
\pgfsetstrokecolor{currentstroke}%
\pgfsetstrokeopacity{0.500000}%
\pgfsetdash{}{0pt}%
\pgfpathmoveto{\pgfqpoint{7.201884in}{3.823611in}}%
\pgfpathlineto{\pgfqpoint{8.372206in}{3.823611in}}%
\pgfpathlineto{\pgfqpoint{8.372206in}{3.617729in}}%
\pgfpathlineto{\pgfqpoint{7.201884in}{3.617729in}}%
\pgfpathclose%
\pgfusepath{fill}%
\end{pgfscope}%
\begin{pgfscope}%
\pgfpathrectangle{\pgfqpoint{1.249956in}{0.148611in}}{\pgfqpoint{7.122250in}{3.850000in}}%
\pgfusepath{clip}%
\pgfsetbuttcap%
\pgfsetmiterjoin%
\definecolor{currentfill}{rgb}{0.898885,0.305498,0.206767}%
\pgfsetfillcolor{currentfill}%
\pgfsetfillopacity{0.500000}%
\pgfsetlinewidth{0.000000pt}%
\definecolor{currentstroke}{rgb}{0.000000,0.000000,0.000000}%
\pgfsetstrokecolor{currentstroke}%
\pgfsetstrokeopacity{0.500000}%
\pgfsetdash{}{0pt}%
\pgfpathmoveto{\pgfqpoint{7.970953in}{3.411846in}}%
\pgfpathlineto{\pgfqpoint{8.372206in}{3.411846in}}%
\pgfpathlineto{\pgfqpoint{8.372206in}{3.205964in}}%
\pgfpathlineto{\pgfqpoint{7.970953in}{3.205964in}}%
\pgfpathclose%
\pgfusepath{fill}%
\end{pgfscope}%
\begin{pgfscope}%
\pgfpathrectangle{\pgfqpoint{1.249956in}{0.148611in}}{\pgfqpoint{7.122250in}{3.850000in}}%
\pgfusepath{clip}%
\pgfsetbuttcap%
\pgfsetmiterjoin%
\definecolor{currentfill}{rgb}{0.898885,0.305498,0.206767}%
\pgfsetfillcolor{currentfill}%
\pgfsetlinewidth{0.000000pt}%
\definecolor{currentstroke}{rgb}{0.000000,0.000000,0.000000}%
\pgfsetstrokecolor{currentstroke}%
\pgfsetstrokeopacity{0.000000}%
\pgfsetdash{}{0pt}%
\pgfpathmoveto{\pgfqpoint{6.867506in}{3.000082in}}%
\pgfpathlineto{\pgfqpoint{8.372206in}{3.000082in}}%
\pgfpathlineto{\pgfqpoint{8.372206in}{2.794199in}}%
\pgfpathlineto{\pgfqpoint{6.867506in}{2.794199in}}%
\pgfpathclose%
\pgfusepath{fill}%
\end{pgfscope}%
\begin{pgfscope}%
\pgfpathrectangle{\pgfqpoint{1.249956in}{0.148611in}}{\pgfqpoint{7.122250in}{3.850000in}}%
\pgfusepath{clip}%
\pgfsetbuttcap%
\pgfsetmiterjoin%
\definecolor{currentfill}{rgb}{0.898885,0.305498,0.206767}%
\pgfsetfillcolor{currentfill}%
\pgfsetfillopacity{0.500000}%
\pgfsetlinewidth{0.000000pt}%
\definecolor{currentstroke}{rgb}{0.000000,0.000000,0.000000}%
\pgfsetstrokecolor{currentstroke}%
\pgfsetstrokeopacity{0.500000}%
\pgfsetdash{}{0pt}%
\pgfpathmoveto{\pgfqpoint{8.338769in}{2.588317in}}%
\pgfpathlineto{\pgfqpoint{8.372206in}{2.588317in}}%
\pgfpathlineto{\pgfqpoint{8.372206in}{2.382435in}}%
\pgfpathlineto{\pgfqpoint{8.338769in}{2.382435in}}%
\pgfpathclose%
\pgfusepath{fill}%
\end{pgfscope}%
\begin{pgfscope}%
\pgfpathrectangle{\pgfqpoint{1.249956in}{0.148611in}}{\pgfqpoint{7.122250in}{3.850000in}}%
\pgfusepath{clip}%
\pgfsetbuttcap%
\pgfsetmiterjoin%
\definecolor{currentfill}{rgb}{0.898885,0.305498,0.206767}%
\pgfsetfillcolor{currentfill}%
\pgfsetfillopacity{0.500000}%
\pgfsetlinewidth{0.000000pt}%
\definecolor{currentstroke}{rgb}{0.000000,0.000000,0.000000}%
\pgfsetstrokecolor{currentstroke}%
\pgfsetstrokeopacity{0.500000}%
\pgfsetdash{}{0pt}%
\pgfpathmoveto{\pgfqpoint{8.104704in}{2.176552in}}%
\pgfpathlineto{\pgfqpoint{8.372206in}{2.176552in}}%
\pgfpathlineto{\pgfqpoint{8.372206in}{1.970670in}}%
\pgfpathlineto{\pgfqpoint{8.104704in}{1.970670in}}%
\pgfpathclose%
\pgfusepath{fill}%
\end{pgfscope}%
\begin{pgfscope}%
\pgfpathrectangle{\pgfqpoint{1.249956in}{0.148611in}}{\pgfqpoint{7.122250in}{3.850000in}}%
\pgfusepath{clip}%
\pgfsetbuttcap%
\pgfsetmiterjoin%
\definecolor{currentfill}{rgb}{0.898885,0.305498,0.206767}%
\pgfsetfillcolor{currentfill}%
\pgfsetlinewidth{0.000000pt}%
\definecolor{currentstroke}{rgb}{0.000000,0.000000,0.000000}%
\pgfsetstrokecolor{currentstroke}%
\pgfsetstrokeopacity{0.000000}%
\pgfsetdash{}{0pt}%
\pgfpathmoveto{\pgfqpoint{7.034695in}{1.764788in}}%
\pgfpathlineto{\pgfqpoint{8.372206in}{1.764788in}}%
\pgfpathlineto{\pgfqpoint{8.372206in}{1.558905in}}%
\pgfpathlineto{\pgfqpoint{7.034695in}{1.558905in}}%
\pgfpathclose%
\pgfusepath{fill}%
\end{pgfscope}%
\begin{pgfscope}%
\pgfpathrectangle{\pgfqpoint{1.249956in}{0.148611in}}{\pgfqpoint{7.122250in}{3.850000in}}%
\pgfusepath{clip}%
\pgfsetbuttcap%
\pgfsetmiterjoin%
\definecolor{currentfill}{rgb}{0.898885,0.305498,0.206767}%
\pgfsetfillcolor{currentfill}%
\pgfsetfillopacity{0.500000}%
\pgfsetlinewidth{0.000000pt}%
\definecolor{currentstroke}{rgb}{0.000000,0.000000,0.000000}%
\pgfsetstrokecolor{currentstroke}%
\pgfsetstrokeopacity{0.500000}%
\pgfsetdash{}{0pt}%
\pgfpathmoveto{\pgfqpoint{8.305331in}{1.353023in}}%
\pgfpathlineto{\pgfqpoint{8.372206in}{1.353023in}}%
\pgfpathlineto{\pgfqpoint{8.372206in}{1.147141in}}%
\pgfpathlineto{\pgfqpoint{8.305331in}{1.147141in}}%
\pgfpathclose%
\pgfusepath{fill}%
\end{pgfscope}%
\begin{pgfscope}%
\pgfpathrectangle{\pgfqpoint{1.249956in}{0.148611in}}{\pgfqpoint{7.122250in}{3.850000in}}%
\pgfusepath{clip}%
\pgfsetbuttcap%
\pgfsetmiterjoin%
\definecolor{currentfill}{rgb}{0.898885,0.305498,0.206767}%
\pgfsetfillcolor{currentfill}%
\pgfsetlinewidth{0.000000pt}%
\definecolor{currentstroke}{rgb}{0.000000,0.000000,0.000000}%
\pgfsetstrokecolor{currentstroke}%
\pgfsetstrokeopacity{0.000000}%
\pgfsetdash{}{0pt}%
\pgfpathmoveto{\pgfqpoint{7.569699in}{0.941258in}}%
\pgfpathlineto{\pgfqpoint{8.372206in}{0.941258in}}%
\pgfpathlineto{\pgfqpoint{8.372206in}{0.735376in}}%
\pgfpathlineto{\pgfqpoint{7.569699in}{0.735376in}}%
\pgfpathclose%
\pgfusepath{fill}%
\end{pgfscope}%
\begin{pgfscope}%
\pgfpathrectangle{\pgfqpoint{1.249956in}{0.148611in}}{\pgfqpoint{7.122250in}{3.850000in}}%
\pgfusepath{clip}%
\pgfsetbuttcap%
\pgfsetmiterjoin%
\definecolor{currentfill}{rgb}{0.898885,0.305498,0.206767}%
\pgfsetfillcolor{currentfill}%
\pgfsetlinewidth{0.000000pt}%
\definecolor{currentstroke}{rgb}{0.000000,0.000000,0.000000}%
\pgfsetstrokecolor{currentstroke}%
\pgfsetstrokeopacity{0.000000}%
\pgfsetdash{}{0pt}%
\pgfpathmoveto{\pgfqpoint{6.834068in}{0.529493in}}%
\pgfpathlineto{\pgfqpoint{8.372206in}{0.529493in}}%
\pgfpathlineto{\pgfqpoint{8.372206in}{0.323611in}}%
\pgfpathlineto{\pgfqpoint{6.834068in}{0.323611in}}%
\pgfpathclose%
\pgfusepath{fill}%
\end{pgfscope}%
\begin{pgfscope}%
\pgfsetbuttcap%
\pgfsetroundjoin%
\definecolor{currentfill}{rgb}{0.000000,0.000000,0.000000}%
\pgfsetfillcolor{currentfill}%
\pgfsetlinewidth{0.803000pt}%
\definecolor{currentstroke}{rgb}{0.000000,0.000000,0.000000}%
\pgfsetstrokecolor{currentstroke}%
\pgfsetdash{}{0pt}%
\pgfsys@defobject{currentmarker}{\pgfqpoint{-0.048611in}{0.000000in}}{\pgfqpoint{-0.000000in}{0.000000in}}{%
\pgfpathmoveto{\pgfqpoint{-0.000000in}{0.000000in}}%
\pgfpathlineto{\pgfqpoint{-0.048611in}{0.000000in}}%
\pgfusepath{stroke,fill}%
}%
\begin{pgfscope}%
\pgfsys@transformshift{1.249956in}{3.720670in}%
\pgfsys@useobject{currentmarker}{}%
\end{pgfscope}%
\end{pgfscope}%
\begin{pgfscope}%
\definecolor{textcolor}{rgb}{0.000000,0.000000,0.000000}%
\pgfsetstrokecolor{textcolor}%
\pgfsetfillcolor{textcolor}%
\pgftext[x=0.482975in, y=3.667908in, left, base]{\color{textcolor}\sffamily\fontsize{10.000000}{12.000000}\selectfont 3DFRONT}%
\end{pgfscope}%
\begin{pgfscope}%
\pgfsetbuttcap%
\pgfsetroundjoin%
\definecolor{currentfill}{rgb}{0.000000,0.000000,0.000000}%
\pgfsetfillcolor{currentfill}%
\pgfsetlinewidth{0.803000pt}%
\definecolor{currentstroke}{rgb}{0.000000,0.000000,0.000000}%
\pgfsetstrokecolor{currentstroke}%
\pgfsetdash{}{0pt}%
\pgfsys@defobject{currentmarker}{\pgfqpoint{-0.048611in}{0.000000in}}{\pgfqpoint{-0.000000in}{0.000000in}}{%
\pgfpathmoveto{\pgfqpoint{-0.000000in}{0.000000in}}%
\pgfpathlineto{\pgfqpoint{-0.048611in}{0.000000in}}%
\pgfusepath{stroke,fill}%
}%
\begin{pgfscope}%
\pgfsys@transformshift{1.249956in}{3.308905in}%
\pgfsys@useobject{currentmarker}{}%
\end{pgfscope}%
\end{pgfscope}%
\begin{pgfscope}%
\definecolor{textcolor}{rgb}{0.000000,0.000000,0.000000}%
\pgfsetstrokecolor{textcolor}%
\pgfsetfillcolor{textcolor}%
\pgftext[x=0.533295in, y=3.256144in, left, base]{\color{textcolor}\sffamily\fontsize{10.000000}{12.000000}\selectfont AI2THOR}%
\end{pgfscope}%
\begin{pgfscope}%
\pgfsetbuttcap%
\pgfsetroundjoin%
\definecolor{currentfill}{rgb}{0.000000,0.000000,0.000000}%
\pgfsetfillcolor{currentfill}%
\pgfsetlinewidth{0.803000pt}%
\definecolor{currentstroke}{rgb}{0.000000,0.000000,0.000000}%
\pgfsetstrokecolor{currentstroke}%
\pgfsetdash{}{0pt}%
\pgfsys@defobject{currentmarker}{\pgfqpoint{-0.048611in}{0.000000in}}{\pgfqpoint{-0.000000in}{0.000000in}}{%
\pgfpathmoveto{\pgfqpoint{-0.000000in}{0.000000in}}%
\pgfpathlineto{\pgfqpoint{-0.048611in}{0.000000in}}%
\pgfusepath{stroke,fill}%
}%
\begin{pgfscope}%
\pgfsys@transformshift{1.249956in}{2.897141in}%
\pgfsys@useobject{currentmarker}{}%
\end{pgfscope}%
\end{pgfscope}%
\begin{pgfscope}%
\definecolor{textcolor}{rgb}{0.000000,0.000000,0.000000}%
\pgfsetstrokecolor{textcolor}%
\pgfsetfillcolor{textcolor}%
\pgftext[x=0.311127in, y=2.844379in, left, base]{\color{textcolor}\sffamily\fontsize{10.000000}{12.000000}\selectfont Blenderproc}%
\end{pgfscope}%
\begin{pgfscope}%
\pgfsetbuttcap%
\pgfsetroundjoin%
\definecolor{currentfill}{rgb}{0.000000,0.000000,0.000000}%
\pgfsetfillcolor{currentfill}%
\pgfsetlinewidth{0.803000pt}%
\definecolor{currentstroke}{rgb}{0.000000,0.000000,0.000000}%
\pgfsetstrokecolor{currentstroke}%
\pgfsetdash{}{0pt}%
\pgfsys@defobject{currentmarker}{\pgfqpoint{-0.048611in}{0.000000in}}{\pgfqpoint{-0.000000in}{0.000000in}}{%
\pgfpathmoveto{\pgfqpoint{-0.000000in}{0.000000in}}%
\pgfpathlineto{\pgfqpoint{-0.048611in}{0.000000in}}%
\pgfusepath{stroke,fill}%
}%
\begin{pgfscope}%
\pgfsys@transformshift{1.249956in}{2.485376in}%
\pgfsys@useobject{currentmarker}{}%
\end{pgfscope}%
\end{pgfscope}%
\begin{pgfscope}%
\definecolor{textcolor}{rgb}{0.000000,0.000000,0.000000}%
\pgfsetstrokecolor{textcolor}%
\pgfsetfillcolor{textcolor}%
\pgftext[x=0.489146in, y=2.432614in, left, base]{\color{textcolor}\sffamily\fontsize{10.000000}{12.000000}\selectfont Hyperism}%
\end{pgfscope}%
\begin{pgfscope}%
\pgfsetbuttcap%
\pgfsetroundjoin%
\definecolor{currentfill}{rgb}{0.000000,0.000000,0.000000}%
\pgfsetfillcolor{currentfill}%
\pgfsetlinewidth{0.803000pt}%
\definecolor{currentstroke}{rgb}{0.000000,0.000000,0.000000}%
\pgfsetstrokecolor{currentstroke}%
\pgfsetdash{}{0pt}%
\pgfsys@defobject{currentmarker}{\pgfqpoint{-0.048611in}{0.000000in}}{\pgfqpoint{-0.000000in}{0.000000in}}{%
\pgfpathmoveto{\pgfqpoint{-0.000000in}{0.000000in}}%
\pgfpathlineto{\pgfqpoint{-0.048611in}{0.000000in}}%
\pgfusepath{stroke,fill}%
}%
\begin{pgfscope}%
\pgfsys@transformshift{1.249956in}{2.073611in}%
\pgfsys@useobject{currentmarker}{}%
\end{pgfscope}%
\end{pgfscope}%
\begin{pgfscope}%
\definecolor{textcolor}{rgb}{0.000000,0.000000,0.000000}%
\pgfsetstrokecolor{textcolor}%
\pgfsetfillcolor{textcolor}%
\pgftext[x=0.402273in, y=2.020850in, left, base]{\color{textcolor}\sffamily\fontsize{10.000000}{12.000000}\selectfont InteriorNet}%
\end{pgfscope}%
\begin{pgfscope}%
\pgfsetbuttcap%
\pgfsetroundjoin%
\definecolor{currentfill}{rgb}{0.000000,0.000000,0.000000}%
\pgfsetfillcolor{currentfill}%
\pgfsetlinewidth{0.803000pt}%
\definecolor{currentstroke}{rgb}{0.000000,0.000000,0.000000}%
\pgfsetstrokecolor{currentstroke}%
\pgfsetdash{}{0pt}%
\pgfsys@defobject{currentmarker}{\pgfqpoint{-0.048611in}{0.000000in}}{\pgfqpoint{-0.000000in}{0.000000in}}{%
\pgfpathmoveto{\pgfqpoint{-0.000000in}{0.000000in}}%
\pgfpathlineto{\pgfqpoint{-0.048611in}{0.000000in}}%
\pgfusepath{stroke,fill}%
}%
\begin{pgfscope}%
\pgfsys@transformshift{1.249956in}{1.661846in}%
\pgfsys@useobject{currentmarker}{}%
\end{pgfscope}%
\end{pgfscope}%
\begin{pgfscope}%
\definecolor{textcolor}{rgb}{0.000000,0.000000,0.000000}%
\pgfsetstrokecolor{textcolor}%
\pgfsetfillcolor{textcolor}%
\pgftext[x=0.313908in, y=1.609085in, left, base]{\color{textcolor}\sffamily\fontsize{10.000000}{12.000000}\selectfont OpenRooms}%
\end{pgfscope}%
\begin{pgfscope}%
\pgfsetbuttcap%
\pgfsetroundjoin%
\definecolor{currentfill}{rgb}{0.000000,0.000000,0.000000}%
\pgfsetfillcolor{currentfill}%
\pgfsetlinewidth{0.803000pt}%
\definecolor{currentstroke}{rgb}{0.000000,0.000000,0.000000}%
\pgfsetstrokecolor{currentstroke}%
\pgfsetdash{}{0pt}%
\pgfsys@defobject{currentmarker}{\pgfqpoint{-0.048611in}{0.000000in}}{\pgfqpoint{-0.000000in}{0.000000in}}{%
\pgfpathmoveto{\pgfqpoint{-0.000000in}{0.000000in}}%
\pgfpathlineto{\pgfqpoint{-0.048611in}{0.000000in}}%
\pgfusepath{stroke,fill}%
}%
\begin{pgfscope}%
\pgfsys@transformshift{1.249956in}{1.250082in}%
\pgfsys@useobject{currentmarker}{}%
\end{pgfscope}%
\end{pgfscope}%
\begin{pgfscope}%
\definecolor{textcolor}{rgb}{0.000000,0.000000,0.000000}%
\pgfsetstrokecolor{textcolor}%
\pgfsetfillcolor{textcolor}%
\pgftext[x=0.755938in, y=1.197320in, left, base]{\color{textcolor}\sffamily\fontsize{10.000000}{12.000000}\selectfont Pix3D}%
\end{pgfscope}%
\begin{pgfscope}%
\pgfsetbuttcap%
\pgfsetroundjoin%
\definecolor{currentfill}{rgb}{0.000000,0.000000,0.000000}%
\pgfsetfillcolor{currentfill}%
\pgfsetlinewidth{0.803000pt}%
\definecolor{currentstroke}{rgb}{0.000000,0.000000,0.000000}%
\pgfsetstrokecolor{currentstroke}%
\pgfsetdash{}{0pt}%
\pgfsys@defobject{currentmarker}{\pgfqpoint{-0.048611in}{0.000000in}}{\pgfqpoint{-0.000000in}{0.000000in}}{%
\pgfpathmoveto{\pgfqpoint{-0.000000in}{0.000000in}}%
\pgfpathlineto{\pgfqpoint{-0.048611in}{0.000000in}}%
\pgfusepath{stroke,fill}%
}%
\begin{pgfscope}%
\pgfsys@transformshift{1.249956in}{0.838317in}%
\pgfsys@useobject{currentmarker}{}%
\end{pgfscope}%
\end{pgfscope}%
\begin{pgfscope}%
\definecolor{textcolor}{rgb}{0.000000,0.000000,0.000000}%
\pgfsetstrokecolor{textcolor}%
\pgfsetfillcolor{textcolor}%
\pgftext[x=0.289968in, y=0.785555in, left, base]{\color{textcolor}\sffamily\fontsize{10.000000}{12.000000}\selectfont S2R:3DFREE}%
\end{pgfscope}%
\begin{pgfscope}%
\pgfsetbuttcap%
\pgfsetroundjoin%
\definecolor{currentfill}{rgb}{0.000000,0.000000,0.000000}%
\pgfsetfillcolor{currentfill}%
\pgfsetlinewidth{0.803000pt}%
\definecolor{currentstroke}{rgb}{0.000000,0.000000,0.000000}%
\pgfsetstrokecolor{currentstroke}%
\pgfsetdash{}{0pt}%
\pgfsys@defobject{currentmarker}{\pgfqpoint{-0.048611in}{0.000000in}}{\pgfqpoint{-0.000000in}{0.000000in}}{%
\pgfpathmoveto{\pgfqpoint{-0.000000in}{0.000000in}}%
\pgfpathlineto{\pgfqpoint{-0.048611in}{0.000000in}}%
\pgfusepath{stroke,fill}%
}%
\begin{pgfscope}%
\pgfsys@transformshift{1.249956in}{0.426552in}%
\pgfsys@useobject{currentmarker}{}%
\end{pgfscope}%
\end{pgfscope}%
\begin{pgfscope}%
\definecolor{textcolor}{rgb}{0.000000,0.000000,0.000000}%
\pgfsetstrokecolor{textcolor}%
\pgfsetfillcolor{textcolor}%
\pgftext[x=0.485484in, y=0.373791in, left, base]{\color{textcolor}\sffamily\fontsize{10.000000}{12.000000}\selectfont SceneNet}%
\end{pgfscope}%
\begin{pgfscope}%
\definecolor{textcolor}{rgb}{0.000000,0.000000,0.000000}%
\pgfsetstrokecolor{textcolor}%
\pgfsetfillcolor{textcolor}%
\pgftext[x=0.234413in,y=2.073611in,,bottom,rotate=90.000000]{\color{textcolor}\sffamily\fontsize{10.000000}{12.000000}\selectfont Datasets}%
\end{pgfscope}%
\begin{pgfscope}%
\pgfpathrectangle{\pgfqpoint{1.249956in}{0.148611in}}{\pgfqpoint{7.122250in}{3.850000in}}%
\pgfusepath{clip}%
\pgfsetbuttcap%
\pgfsetroundjoin%
\pgfsetlinewidth{1.505625pt}%
\definecolor{currentstroke}{rgb}{0.000000,0.000000,0.000000}%
\pgfsetstrokecolor{currentstroke}%
\pgfsetstrokeopacity{0.200000}%
\pgfsetdash{{5.550000pt}{2.400000pt}}{0.000000pt}%
\pgfpathmoveto{\pgfqpoint{4.259358in}{0.148611in}}%
\pgfpathlineto{\pgfqpoint{4.259358in}{3.998611in}}%
\pgfusepath{stroke}%
\end{pgfscope}%
\begin{pgfscope}%
\pgfsetrectcap%
\pgfsetmiterjoin%
\pgfsetlinewidth{0.803000pt}%
\definecolor{currentstroke}{rgb}{0.000000,0.000000,0.000000}%
\pgfsetstrokecolor{currentstroke}%
\pgfsetdash{}{0pt}%
\pgfpathmoveto{\pgfqpoint{1.249956in}{0.148611in}}%
\pgfpathlineto{\pgfqpoint{1.249956in}{3.998611in}}%
\pgfusepath{stroke}%
\end{pgfscope}%
\begin{pgfscope}%
\pgfsetrectcap%
\pgfsetmiterjoin%
\pgfsetlinewidth{0.803000pt}%
\definecolor{currentstroke}{rgb}{0.000000,0.000000,0.000000}%
\pgfsetstrokecolor{currentstroke}%
\pgfsetdash{}{0pt}%
\pgfpathmoveto{\pgfqpoint{8.372206in}{0.148611in}}%
\pgfpathlineto{\pgfqpoint{8.372206in}{3.998611in}}%
\pgfusepath{stroke}%
\end{pgfscope}%
\begin{pgfscope}%
\pgfsetrectcap%
\pgfsetmiterjoin%
\pgfsetlinewidth{0.803000pt}%
\definecolor{currentstroke}{rgb}{0.000000,0.000000,0.000000}%
\pgfsetstrokecolor{currentstroke}%
\pgfsetdash{}{0pt}%
\pgfpathmoveto{\pgfqpoint{1.249956in}{0.148611in}}%
\pgfpathlineto{\pgfqpoint{8.372206in}{0.148611in}}%
\pgfusepath{stroke}%
\end{pgfscope}%
\begin{pgfscope}%
\pgfsetrectcap%
\pgfsetmiterjoin%
\pgfsetlinewidth{0.803000pt}%
\definecolor{currentstroke}{rgb}{0.000000,0.000000,0.000000}%
\pgfsetstrokecolor{currentstroke}%
\pgfsetdash{}{0pt}%
\pgfpathmoveto{\pgfqpoint{1.249956in}{3.998611in}}%
\pgfpathlineto{\pgfqpoint{8.372206in}{3.998611in}}%
\pgfusepath{stroke}%
\end{pgfscope}%
\begin{pgfscope}%
\definecolor{textcolor}{rgb}{1.000000,1.000000,1.000000}%
\pgfsetstrokecolor{textcolor}%
\pgfsetfillcolor{textcolor}%
\pgftext[x=1.450583in,y=3.720670in,,]{\color{textcolor}\sffamily\fontsize{10.000000}{12.000000}\selectfont 12}%
\end{pgfscope}%
\begin{pgfscope}%
\definecolor{textcolor}{rgb}{1.000000,1.000000,1.000000}%
\pgfsetstrokecolor{textcolor}%
\pgfsetfillcolor{textcolor}%
\pgftext[x=1.484021in,y=3.308905in,,]{\color{textcolor}\sffamily\fontsize{10.000000}{12.000000}\selectfont 14}%
\end{pgfscope}%
\begin{pgfscope}%
\definecolor{textcolor}{rgb}{1.000000,1.000000,1.000000}%
\pgfsetstrokecolor{textcolor}%
\pgfsetfillcolor{textcolor}%
\pgftext[x=1.383708in,y=2.897141in,,]{\color{textcolor}\sffamily\fontsize{10.000000}{12.000000}\selectfont 8}%
\end{pgfscope}%
\begin{pgfscope}%
\definecolor{textcolor}{rgb}{1.000000,1.000000,1.000000}%
\pgfsetstrokecolor{textcolor}%
\pgfsetfillcolor{textcolor}%
\pgftext[x=1.484021in,y=2.485376in,,]{\color{textcolor}\sffamily\fontsize{10.000000}{12.000000}\selectfont 14}%
\end{pgfscope}%
\begin{pgfscope}%
\definecolor{textcolor}{rgb}{1.000000,1.000000,1.000000}%
\pgfsetstrokecolor{textcolor}%
\pgfsetfillcolor{textcolor}%
\pgftext[x=2.052463in,y=2.073611in,,]{\color{textcolor}\sffamily\fontsize{10.000000}{12.000000}\selectfont 48}%
\end{pgfscope}%
\begin{pgfscope}%
\definecolor{textcolor}{rgb}{1.000000,1.000000,1.000000}%
\pgfsetstrokecolor{textcolor}%
\pgfsetfillcolor{textcolor}%
\pgftext[x=1.333551in,y=1.661846in,,]{\color{textcolor}\sffamily\fontsize{10.000000}{12.000000}\selectfont 5}%
\end{pgfscope}%
\begin{pgfscope}%
\definecolor{textcolor}{rgb}{1.000000,1.000000,1.000000}%
\pgfsetstrokecolor{textcolor}%
\pgfsetfillcolor{textcolor}%
\pgftext[x=2.921846in,y=1.250082in,,]{\color{textcolor}\sffamily\fontsize{10.000000}{12.000000}\selectfont 100}%
\end{pgfscope}%
\begin{pgfscope}%
\definecolor{textcolor}{rgb}{1.000000,1.000000,1.000000}%
\pgfsetstrokecolor{textcolor}%
\pgfsetfillcolor{textcolor}%
\pgftext[x=1.283394in,y=0.838317in,,]{\color{textcolor}\sffamily\fontsize{10.000000}{12.000000}\selectfont 2}%
\end{pgfscope}%
\begin{pgfscope}%
\definecolor{textcolor}{rgb}{1.000000,1.000000,1.000000}%
\pgfsetstrokecolor{textcolor}%
\pgfsetfillcolor{textcolor}%
\pgftext[x=1.417145in,y=0.426552in,,]{\color{textcolor}\sffamily\fontsize{10.000000}{12.000000}\selectfont 10}%
\end{pgfscope}%
\begin{pgfscope}%
\definecolor{textcolor}{rgb}{1.000000,1.000000,1.000000}%
\pgfsetstrokecolor{textcolor}%
\pgfsetfillcolor{textcolor}%
\pgftext[x=1.918712in,y=3.720670in,,]{\color{textcolor}\sffamily\fontsize{10.000000}{12.000000}\selectfont 16}%
\end{pgfscope}%
\begin{pgfscope}%
\definecolor{textcolor}{rgb}{1.000000,1.000000,1.000000}%
\pgfsetstrokecolor{textcolor}%
\pgfsetfillcolor{textcolor}%
\pgftext[x=1.968869in,y=3.308905in,,]{\color{textcolor}\sffamily\fontsize{10.000000}{12.000000}\selectfont 15}%
\end{pgfscope}%
\begin{pgfscope}%
\definecolor{textcolor}{rgb}{1.000000,1.000000,1.000000}%
\pgfsetstrokecolor{textcolor}%
\pgfsetfillcolor{textcolor}%
\pgftext[x=1.701367in,y=2.897141in,,]{\color{textcolor}\sffamily\fontsize{10.000000}{12.000000}\selectfont 11}%
\end{pgfscope}%
\begin{pgfscope}%
\definecolor{textcolor}{rgb}{1.000000,1.000000,1.000000}%
\pgfsetstrokecolor{textcolor}%
\pgfsetfillcolor{textcolor}%
\pgftext[x=2.436998in,y=2.485376in,,]{\color{textcolor}\sffamily\fontsize{10.000000}{12.000000}\selectfont 43}%
\end{pgfscope}%
\begin{pgfscope}%
\definecolor{textcolor}{rgb}{1.000000,1.000000,1.000000}%
\pgfsetstrokecolor{textcolor}%
\pgfsetfillcolor{textcolor}%
\pgftext[x=3.958418in,y=2.073611in,,]{\color{textcolor}\sffamily\fontsize{10.000000}{12.000000}\selectfont 66}%
\end{pgfscope}%
\begin{pgfscope}%
\definecolor{textcolor}{rgb}{1.000000,1.000000,1.000000}%
\pgfsetstrokecolor{textcolor}%
\pgfsetfillcolor{textcolor}%
\pgftext[x=1.517459in,y=1.661846in,,]{\color{textcolor}\sffamily\fontsize{10.000000}{12.000000}\selectfont 6}%
\end{pgfscope}%
\begin{pgfscope}%
\definecolor{textcolor}{rgb}{1.000000,1.000000,1.000000}%
\pgfsetstrokecolor{textcolor}%
\pgfsetfillcolor{textcolor}%
\pgftext[x=5.195616in,y=1.250082in,,]{\color{textcolor}\sffamily\fontsize{10.000000}{12.000000}\selectfont 36}%
\end{pgfscope}%
\begin{pgfscope}%
\definecolor{textcolor}{rgb}{1.000000,1.000000,1.000000}%
\pgfsetstrokecolor{textcolor}%
\pgfsetfillcolor{textcolor}%
\pgftext[x=1.467302in,y=0.838317in,,]{\color{textcolor}\sffamily\fontsize{10.000000}{12.000000}\selectfont 9}%
\end{pgfscope}%
\begin{pgfscope}%
\definecolor{textcolor}{rgb}{1.000000,1.000000,1.000000}%
\pgfsetstrokecolor{textcolor}%
\pgfsetfillcolor{textcolor}%
\pgftext[x=1.768242in,y=0.426552in,,]{\color{textcolor}\sffamily\fontsize{10.000000}{12.000000}\selectfont 11}%
\end{pgfscope}%
\begin{pgfscope}%
\definecolor{textcolor}{rgb}{1.000000,1.000000,1.000000}%
\pgfsetstrokecolor{textcolor}%
\pgfsetfillcolor{textcolor}%
\pgftext[x=2.570749in,y=3.720670in,,]{\color{textcolor}\sffamily\fontsize{10.000000}{12.000000}\selectfont 23}%
\end{pgfscope}%
\begin{pgfscope}%
\definecolor{textcolor}{rgb}{1.000000,1.000000,1.000000}%
\pgfsetstrokecolor{textcolor}%
\pgfsetfillcolor{textcolor}%
\pgftext[x=2.637625in,y=3.308905in,,]{\color{textcolor}\sffamily\fontsize{10.000000}{12.000000}\selectfont 25}%
\end{pgfscope}%
\begin{pgfscope}%
\definecolor{textcolor}{rgb}{1.000000,1.000000,1.000000}%
\pgfsetstrokecolor{textcolor}%
\pgfsetfillcolor{textcolor}%
\pgftext[x=2.119339in,y=2.897141in,,]{\color{textcolor}\sffamily\fontsize{10.000000}{12.000000}\selectfont 14}%
\end{pgfscope}%
\begin{pgfscope}%
\definecolor{textcolor}{rgb}{1.000000,1.000000,1.000000}%
\pgfsetstrokecolor{textcolor}%
\pgfsetfillcolor{textcolor}%
\pgftext[x=3.958418in,y=2.485376in,,]{\color{textcolor}\sffamily\fontsize{10.000000}{12.000000}\selectfont 48}%
\end{pgfscope}%
\begin{pgfscope}%
\definecolor{textcolor}{rgb}{1.000000,1.000000,1.000000}%
\pgfsetstrokecolor{textcolor}%
\pgfsetfillcolor{textcolor}%
\pgftext[x=5.496556in,y=2.073611in,,]{\color{textcolor}\sffamily\fontsize{10.000000}{12.000000}\selectfont 26}%
\end{pgfscope}%
\begin{pgfscope}%
\definecolor{textcolor}{rgb}{1.000000,1.000000,1.000000}%
\pgfsetstrokecolor{textcolor}%
\pgfsetfillcolor{textcolor}%
\pgftext[x=1.835118in,y=1.661846in,,]{\color{textcolor}\sffamily\fontsize{10.000000}{12.000000}\selectfont 13}%
\end{pgfscope}%
\begin{pgfscope}%
\definecolor{textcolor}{rgb}{1.000000,1.000000,1.000000}%
\pgfsetstrokecolor{textcolor}%
\pgfsetfillcolor{textcolor}%
\pgftext[x=6.248907in,y=1.250082in,,]{\color{textcolor}\sffamily\fontsize{10.000000}{12.000000}\selectfont 27}%
\end{pgfscope}%
\begin{pgfscope}%
\definecolor{textcolor}{rgb}{1.000000,1.000000,1.000000}%
\pgfsetstrokecolor{textcolor}%
\pgfsetfillcolor{textcolor}%
\pgftext[x=1.901993in,y=0.838317in,,]{\color{textcolor}\sffamily\fontsize{10.000000}{12.000000}\selectfont 17}%
\end{pgfscope}%
\begin{pgfscope}%
\definecolor{textcolor}{rgb}{1.000000,1.000000,1.000000}%
\pgfsetstrokecolor{textcolor}%
\pgfsetfillcolor{textcolor}%
\pgftext[x=2.286528in,y=0.426552in,,]{\color{textcolor}\sffamily\fontsize{10.000000}{12.000000}\selectfont 20}%
\end{pgfscope}%
\begin{pgfscope}%
\definecolor{textcolor}{rgb}{1.000000,1.000000,1.000000}%
\pgfsetstrokecolor{textcolor}%
\pgfsetfillcolor{textcolor}%
\pgftext[x=3.323100in,y=3.720670in,,]{\color{textcolor}\sffamily\fontsize{10.000000}{12.000000}\selectfont 22}%
\end{pgfscope}%
\begin{pgfscope}%
\definecolor{textcolor}{rgb}{1.000000,1.000000,1.000000}%
\pgfsetstrokecolor{textcolor}%
\pgfsetfillcolor{textcolor}%
\pgftext[x=3.690915in,y=3.308905in,,]{\color{textcolor}\sffamily\fontsize{10.000000}{12.000000}\selectfont 38}%
\end{pgfscope}%
\begin{pgfscope}%
\definecolor{textcolor}{rgb}{1.000000,1.000000,1.000000}%
\pgfsetstrokecolor{textcolor}%
\pgfsetfillcolor{textcolor}%
\pgftext[x=2.554030in,y=2.897141in,,]{\color{textcolor}\sffamily\fontsize{10.000000}{12.000000}\selectfont 12}%
\end{pgfscope}%
\begin{pgfscope}%
\definecolor{textcolor}{rgb}{1.000000,1.000000,1.000000}%
\pgfsetstrokecolor{textcolor}%
\pgfsetfillcolor{textcolor}%
\pgftext[x=5.412962in,y=2.485376in,,]{\color{textcolor}\sffamily\fontsize{10.000000}{12.000000}\selectfont 39}%
\end{pgfscope}%
\begin{pgfscope}%
\definecolor{textcolor}{rgb}{1.000000,1.000000,1.000000}%
\pgfsetstrokecolor{textcolor}%
\pgfsetfillcolor{textcolor}%
\pgftext[x=6.365939in,y=2.073611in,,]{\color{textcolor}\sffamily\fontsize{10.000000}{12.000000}\selectfont 26}%
\end{pgfscope}%
\begin{pgfscope}%
\definecolor{textcolor}{rgb}{1.000000,1.000000,1.000000}%
\pgfsetstrokecolor{textcolor}%
\pgfsetfillcolor{textcolor}%
\pgftext[x=2.386841in,y=1.661846in,,]{\color{textcolor}\sffamily\fontsize{10.000000}{12.000000}\selectfont 20}%
\end{pgfscope}%
\begin{pgfscope}%
\definecolor{textcolor}{rgb}{1.000000,1.000000,1.000000}%
\pgfsetstrokecolor{textcolor}%
\pgfsetfillcolor{textcolor}%
\pgftext[x=6.867506in,y=1.250082in,,]{\color{textcolor}\sffamily\fontsize{10.000000}{12.000000}\selectfont 10}%
\end{pgfscope}%
\begin{pgfscope}%
\definecolor{textcolor}{rgb}{1.000000,1.000000,1.000000}%
\pgfsetstrokecolor{textcolor}%
\pgfsetfillcolor{textcolor}%
\pgftext[x=2.671063in,y=0.838317in,,]{\color{textcolor}\sffamily\fontsize{10.000000}{12.000000}\selectfont 29}%
\end{pgfscope}%
\begin{pgfscope}%
\definecolor{textcolor}{rgb}{1.000000,1.000000,1.000000}%
\pgfsetstrokecolor{textcolor}%
\pgfsetfillcolor{textcolor}%
\pgftext[x=2.905127in,y=0.426552in,,]{\color{textcolor}\sffamily\fontsize{10.000000}{12.000000}\selectfont 17}%
\end{pgfscope}%
\begin{pgfscope}%
\definecolor{textcolor}{rgb}{0.662745,0.662745,0.662745}%
\pgfsetstrokecolor{textcolor}%
\pgfsetfillcolor{textcolor}%
\pgftext[x=4.276077in,y=3.720670in,,]{\color{textcolor}\sffamily\fontsize{10.000000}{12.000000}\selectfont 35}%
\end{pgfscope}%
\begin{pgfscope}%
\definecolor{textcolor}{rgb}{0.662745,0.662745,0.662745}%
\pgfsetstrokecolor{textcolor}%
\pgfsetfillcolor{textcolor}%
\pgftext[x=4.744206in,y=3.308905in,,]{\color{textcolor}\sffamily\fontsize{10.000000}{12.000000}\selectfont 25}%
\end{pgfscope}%
\begin{pgfscope}%
\definecolor{textcolor}{rgb}{0.662745,0.662745,0.662745}%
\pgfsetstrokecolor{textcolor}%
\pgfsetfillcolor{textcolor}%
\pgftext[x=3.072316in,y=2.897141in,,]{\color{textcolor}\sffamily\fontsize{10.000000}{12.000000}\selectfont 19}%
\end{pgfscope}%
\begin{pgfscope}%
\definecolor{textcolor}{rgb}{0.662745,0.662745,0.662745}%
\pgfsetstrokecolor{textcolor}%
\pgfsetfillcolor{textcolor}%
\pgftext[x=6.399377in,y=2.485376in,,]{\color{textcolor}\sffamily\fontsize{10.000000}{12.000000}\selectfont 20}%
\end{pgfscope}%
\begin{pgfscope}%
\definecolor{textcolor}{rgb}{0.662745,0.662745,0.662745}%
\pgfsetstrokecolor{textcolor}%
\pgfsetfillcolor{textcolor}%
\pgftext[x=7.068132in,y=2.073611in,,]{\color{textcolor}\sffamily\fontsize{10.000000}{12.000000}\selectfont 16}%
\end{pgfscope}%
\begin{pgfscope}%
\definecolor{textcolor}{rgb}{0.662745,0.662745,0.662745}%
\pgfsetstrokecolor{textcolor}%
\pgfsetfillcolor{textcolor}%
\pgftext[x=3.139192in,y=1.661846in,,]{\color{textcolor}\sffamily\fontsize{10.000000}{12.000000}\selectfont 25}%
\end{pgfscope}%
\begin{pgfscope}%
\definecolor{textcolor}{rgb}{0.662745,0.662745,0.662745}%
\pgfsetstrokecolor{textcolor}%
\pgfsetfillcolor{textcolor}%
\pgftext[x=7.369073in,y=1.250082in,,]{\color{textcolor}\sffamily\fontsize{10.000000}{12.000000}\selectfont 20}%
\end{pgfscope}%
\begin{pgfscope}%
\definecolor{textcolor}{rgb}{0.662745,0.662745,0.662745}%
\pgfsetstrokecolor{textcolor}%
\pgfsetfillcolor{textcolor}%
\pgftext[x=3.707634in,y=0.838317in,,]{\color{textcolor}\sffamily\fontsize{10.000000}{12.000000}\selectfont 33}%
\end{pgfscope}%
\begin{pgfscope}%
\definecolor{textcolor}{rgb}{0.662745,0.662745,0.662745}%
\pgfsetstrokecolor{textcolor}%
\pgfsetfillcolor{textcolor}%
\pgftext[x=3.523726in,y=0.426552in,,]{\color{textcolor}\sffamily\fontsize{10.000000}{12.000000}\selectfont 20}%
\end{pgfscope}%
\begin{pgfscope}%
\definecolor{textcolor}{rgb}{0.662745,0.662745,0.662745}%
\pgfsetstrokecolor{textcolor}%
\pgfsetfillcolor{textcolor}%
\pgftext[x=5.295929in,y=3.720670in,,]{\color{textcolor}\sffamily\fontsize{10.000000}{12.000000}\selectfont 26}%
\end{pgfscope}%
\begin{pgfscope}%
\definecolor{textcolor}{rgb}{0.662745,0.662745,0.662745}%
\pgfsetstrokecolor{textcolor}%
\pgfsetfillcolor{textcolor}%
\pgftext[x=5.780777in,y=3.308905in,,]{\color{textcolor}\sffamily\fontsize{10.000000}{12.000000}\selectfont 37}%
\end{pgfscope}%
\begin{pgfscope}%
\definecolor{textcolor}{rgb}{0.662745,0.662745,0.662745}%
\pgfsetstrokecolor{textcolor}%
\pgfsetfillcolor{textcolor}%
\pgftext[x=3.640759in,y=2.897141in,,]{\color{textcolor}\sffamily\fontsize{10.000000}{12.000000}\selectfont 15}%
\end{pgfscope}%
\begin{pgfscope}%
\definecolor{textcolor}{rgb}{0.662745,0.662745,0.662745}%
\pgfsetstrokecolor{textcolor}%
\pgfsetfillcolor{textcolor}%
\pgftext[x=7.118289in,y=2.485376in,,]{\color{textcolor}\sffamily\fontsize{10.000000}{12.000000}\selectfont 23}%
\end{pgfscope}%
\begin{pgfscope}%
\definecolor{textcolor}{rgb}{0.662745,0.662745,0.662745}%
\pgfsetstrokecolor{textcolor}%
\pgfsetfillcolor{textcolor}%
\pgftext[x=7.536262in,y=2.073611in,,]{\color{textcolor}\sffamily\fontsize{10.000000}{12.000000}\selectfont 12}%
\end{pgfscope}%
\begin{pgfscope}%
\definecolor{textcolor}{rgb}{0.662745,0.662745,0.662745}%
\pgfsetstrokecolor{textcolor}%
\pgfsetfillcolor{textcolor}%
\pgftext[x=3.991855in,y=1.661846in,,]{\color{textcolor}\sffamily\fontsize{10.000000}{12.000000}\selectfont 26}%
\end{pgfscope}%
\begin{pgfscope}%
\definecolor{textcolor}{rgb}{0.662745,0.662745,0.662745}%
\pgfsetstrokecolor{textcolor}%
\pgfsetfillcolor{textcolor}%
\pgftext[x=7.887358in,y=1.250082in,,]{\color{textcolor}\sffamily\fontsize{10.000000}{12.000000}\selectfont 11}%
\end{pgfscope}%
\begin{pgfscope}%
\definecolor{textcolor}{rgb}{0.662745,0.662745,0.662745}%
\pgfsetstrokecolor{textcolor}%
\pgfsetfillcolor{textcolor}%
\pgftext[x=4.777644in,y=0.838317in,,]{\color{textcolor}\sffamily\fontsize{10.000000}{12.000000}\selectfont 31}%
\end{pgfscope}%
\begin{pgfscope}%
\definecolor{textcolor}{rgb}{0.662745,0.662745,0.662745}%
\pgfsetstrokecolor{textcolor}%
\pgfsetfillcolor{textcolor}%
\pgftext[x=4.393109in,y=0.426552in,,]{\color{textcolor}\sffamily\fontsize{10.000000}{12.000000}\selectfont 32}%
\end{pgfscope}%
\begin{pgfscope}%
\definecolor{textcolor}{rgb}{1.000000,1.000000,1.000000}%
\pgfsetstrokecolor{textcolor}%
\pgfsetfillcolor{textcolor}%
\pgftext[x=6.115155in,y=3.720670in,,]{\color{textcolor}\sffamily\fontsize{10.000000}{12.000000}\selectfont 23}%
\end{pgfscope}%
\begin{pgfscope}%
\definecolor{textcolor}{rgb}{1.000000,1.000000,1.000000}%
\pgfsetstrokecolor{textcolor}%
\pgfsetfillcolor{textcolor}%
\pgftext[x=6.951100in,y=3.308905in,,]{\color{textcolor}\sffamily\fontsize{10.000000}{12.000000}\selectfont 33}%
\end{pgfscope}%
\begin{pgfscope}%
\definecolor{textcolor}{rgb}{1.000000,1.000000,1.000000}%
\pgfsetstrokecolor{textcolor}%
\pgfsetfillcolor{textcolor}%
\pgftext[x=4.593736in,y=2.897141in,,]{\color{textcolor}\sffamily\fontsize{10.000000}{12.000000}\selectfont 42}%
\end{pgfscope}%
\begin{pgfscope}%
\definecolor{textcolor}{rgb}{1.000000,1.000000,1.000000}%
\pgfsetstrokecolor{textcolor}%
\pgfsetfillcolor{textcolor}%
\pgftext[x=7.770326in,y=2.485376in,,]{\color{textcolor}\sffamily\fontsize{10.000000}{12.000000}\selectfont 16}%
\end{pgfscope}%
\begin{pgfscope}%
\definecolor{textcolor}{rgb}{1.000000,1.000000,1.000000}%
\pgfsetstrokecolor{textcolor}%
\pgfsetfillcolor{textcolor}%
\pgftext[x=7.887358in,y=2.073611in,,]{\color{textcolor}\sffamily\fontsize{10.000000}{12.000000}\selectfont 9}%
\end{pgfscope}%
\begin{pgfscope}%
\definecolor{textcolor}{rgb}{1.000000,1.000000,1.000000}%
\pgfsetstrokecolor{textcolor}%
\pgfsetfillcolor{textcolor}%
\pgftext[x=4.961551in,y=1.661846in,,]{\color{textcolor}\sffamily\fontsize{10.000000}{12.000000}\selectfont 32}%
\end{pgfscope}%
\begin{pgfscope}%
\definecolor{textcolor}{rgb}{1.000000,1.000000,1.000000}%
\pgfsetstrokecolor{textcolor}%
\pgfsetfillcolor{textcolor}%
\pgftext[x=8.121423in,y=1.250082in,,]{\color{textcolor}\sffamily\fontsize{10.000000}{12.000000}\selectfont 3}%
\end{pgfscope}%
\begin{pgfscope}%
\definecolor{textcolor}{rgb}{1.000000,1.000000,1.000000}%
\pgfsetstrokecolor{textcolor}%
\pgfsetfillcolor{textcolor}%
\pgftext[x=5.797496in,y=0.838317in,,]{\color{textcolor}\sffamily\fontsize{10.000000}{12.000000}\selectfont 30}%
\end{pgfscope}%
\begin{pgfscope}%
\definecolor{textcolor}{rgb}{1.000000,1.000000,1.000000}%
\pgfsetstrokecolor{textcolor}%
\pgfsetfillcolor{textcolor}%
\pgftext[x=5.346086in,y=0.426552in,,]{\color{textcolor}\sffamily\fontsize{10.000000}{12.000000}\selectfont 25}%
\end{pgfscope}%
\begin{pgfscope}%
\definecolor{textcolor}{rgb}{1.000000,1.000000,1.000000}%
\pgfsetstrokecolor{textcolor}%
\pgfsetfillcolor{textcolor}%
\pgftext[x=6.850787in,y=3.720670in,,]{\color{textcolor}\sffamily\fontsize{10.000000}{12.000000}\selectfont 21}%
\end{pgfscope}%
\begin{pgfscope}%
\definecolor{textcolor}{rgb}{1.000000,1.000000,1.000000}%
\pgfsetstrokecolor{textcolor}%
\pgfsetfillcolor{textcolor}%
\pgftext[x=7.736888in,y=3.308905in,,]{\color{textcolor}\sffamily\fontsize{10.000000}{12.000000}\selectfont 14}%
\end{pgfscope}%
\begin{pgfscope}%
\definecolor{textcolor}{rgb}{1.000000,1.000000,1.000000}%
\pgfsetstrokecolor{textcolor}%
\pgfsetfillcolor{textcolor}%
\pgftext[x=6.081718in,y=2.897141in,,]{\color{textcolor}\sffamily\fontsize{10.000000}{12.000000}\selectfont 47}%
\end{pgfscope}%
\begin{pgfscope}%
\definecolor{textcolor}{rgb}{1.000000,1.000000,1.000000}%
\pgfsetstrokecolor{textcolor}%
\pgfsetfillcolor{textcolor}%
\pgftext[x=8.188299in,y=2.485376in,,]{\color{textcolor}\sffamily\fontsize{10.000000}{12.000000}\selectfont 9}%
\end{pgfscope}%
\begin{pgfscope}%
\definecolor{textcolor}{rgb}{1.000000,1.000000,1.000000}%
\pgfsetstrokecolor{textcolor}%
\pgfsetfillcolor{textcolor}%
\pgftext[x=8.071266in,y=2.073611in,,]{\color{textcolor}\sffamily\fontsize{10.000000}{12.000000}\selectfont 2}%
\end{pgfscope}%
\begin{pgfscope}%
\definecolor{textcolor}{rgb}{1.000000,1.000000,1.000000}%
\pgfsetstrokecolor{textcolor}%
\pgfsetfillcolor{textcolor}%
\pgftext[x=6.265625in,y=1.661846in,,]{\color{textcolor}\sffamily\fontsize{10.000000}{12.000000}\selectfont 46}%
\end{pgfscope}%
\begin{pgfscope}%
\definecolor{textcolor}{rgb}{1.000000,1.000000,1.000000}%
\pgfsetstrokecolor{textcolor}%
\pgfsetfillcolor{textcolor}%
\pgftext[x=8.238455in,y=1.250082in,,]{\color{textcolor}\sffamily\fontsize{10.000000}{12.000000}\selectfont 4}%
\end{pgfscope}%
\begin{pgfscope}%
\definecolor{textcolor}{rgb}{1.000000,1.000000,1.000000}%
\pgfsetstrokecolor{textcolor}%
\pgfsetfillcolor{textcolor}%
\pgftext[x=6.934381in,y=0.838317in,,]{\color{textcolor}\sffamily\fontsize{10.000000}{12.000000}\selectfont 38}%
\end{pgfscope}%
\begin{pgfscope}%
\definecolor{textcolor}{rgb}{1.000000,1.000000,1.000000}%
\pgfsetstrokecolor{textcolor}%
\pgfsetfillcolor{textcolor}%
\pgftext[x=6.299063in,y=0.426552in,,]{\color{textcolor}\sffamily\fontsize{10.000000}{12.000000}\selectfont 32}%
\end{pgfscope}%
\begin{pgfscope}%
\definecolor{textcolor}{rgb}{1.000000,1.000000,1.000000}%
\pgfsetstrokecolor{textcolor}%
\pgfsetfillcolor{textcolor}%
\pgftext[x=7.787045in,y=3.720670in,,]{\color{textcolor}\sffamily\fontsize{10.000000}{12.000000}\selectfont 35}%
\end{pgfscope}%
\begin{pgfscope}%
\definecolor{textcolor}{rgb}{1.000000,1.000000,1.000000}%
\pgfsetstrokecolor{textcolor}%
\pgfsetfillcolor{textcolor}%
\pgftext[x=8.171580in,y=3.308905in,,]{\color{textcolor}\sffamily\fontsize{10.000000}{12.000000}\selectfont 12}%
\end{pgfscope}%
\begin{pgfscope}%
\definecolor{textcolor}{rgb}{1.000000,1.000000,1.000000}%
\pgfsetstrokecolor{textcolor}%
\pgfsetfillcolor{textcolor}%
\pgftext[x=7.619856in,y=2.897141in,,]{\color{textcolor}\sffamily\fontsize{10.000000}{12.000000}\selectfont 45}%
\end{pgfscope}%
\begin{pgfscope}%
\definecolor{textcolor}{rgb}{1.000000,1.000000,1.000000}%
\pgfsetstrokecolor{textcolor}%
\pgfsetfillcolor{textcolor}%
\pgftext[x=8.355487in,y=2.485376in,,]{\color{textcolor}\sffamily\fontsize{10.000000}{12.000000}\selectfont 1}%
\end{pgfscope}%
\begin{pgfscope}%
\definecolor{textcolor}{rgb}{1.000000,1.000000,1.000000}%
\pgfsetstrokecolor{textcolor}%
\pgfsetfillcolor{textcolor}%
\pgftext[x=8.238455in,y=2.073611in,,]{\color{textcolor}\sffamily\fontsize{10.000000}{12.000000}\selectfont 8}%
\end{pgfscope}%
\begin{pgfscope}%
\definecolor{textcolor}{rgb}{1.000000,1.000000,1.000000}%
\pgfsetstrokecolor{textcolor}%
\pgfsetfillcolor{textcolor}%
\pgftext[x=7.703451in,y=1.661846in,,]{\color{textcolor}\sffamily\fontsize{10.000000}{12.000000}\selectfont 40}%
\end{pgfscope}%
\begin{pgfscope}%
\definecolor{textcolor}{rgb}{1.000000,1.000000,1.000000}%
\pgfsetstrokecolor{textcolor}%
\pgfsetfillcolor{textcolor}%
\pgftext[x=8.338769in,y=1.250082in,,]{\color{textcolor}\sffamily\fontsize{10.000000}{12.000000}\selectfont 2}%
\end{pgfscope}%
\begin{pgfscope}%
\definecolor{textcolor}{rgb}{1.000000,1.000000,1.000000}%
\pgfsetstrokecolor{textcolor}%
\pgfsetfillcolor{textcolor}%
\pgftext[x=7.970953in,y=0.838317in,,]{\color{textcolor}\sffamily\fontsize{10.000000}{12.000000}\selectfont 24}%
\end{pgfscope}%
\begin{pgfscope}%
\definecolor{textcolor}{rgb}{1.000000,1.000000,1.000000}%
\pgfsetstrokecolor{textcolor}%
\pgfsetfillcolor{textcolor}%
\pgftext[x=7.603137in,y=0.426552in,,]{\color{textcolor}\sffamily\fontsize{10.000000}{12.000000}\selectfont 46}%
\end{pgfscope}%
\begin{pgfscope}%
\pgfsetbuttcap%
\pgfsetmiterjoin%
\definecolor{currentfill}{rgb}{1.000000,1.000000,1.000000}%
\pgfsetfillcolor{currentfill}%
\pgfsetfillopacity{0.800000}%
\pgfsetlinewidth{1.003750pt}%
\definecolor{currentstroke}{rgb}{0.800000,0.800000,0.800000}%
\pgfsetstrokecolor{currentstroke}%
\pgfsetstrokeopacity{0.800000}%
\pgfsetdash{}{0pt}%
\pgfpathmoveto{\pgfqpoint{1.330943in}{4.056458in}}%
\pgfpathlineto{\pgfqpoint{6.806306in}{4.056458in}}%
\pgfpathquadraticcurveto{\pgfqpoint{6.829445in}{4.056458in}}{\pgfqpoint{6.829445in}{4.079597in}}%
\pgfpathlineto{\pgfqpoint{6.829445in}{4.237841in}}%
\pgfpathquadraticcurveto{\pgfqpoint{6.829445in}{4.260980in}}{\pgfqpoint{6.806306in}{4.260980in}}%
\pgfpathlineto{\pgfqpoint{1.330943in}{4.260980in}}%
\pgfpathquadraticcurveto{\pgfqpoint{1.307804in}{4.260980in}}{\pgfqpoint{1.307804in}{4.237841in}}%
\pgfpathlineto{\pgfqpoint{1.307804in}{4.079597in}}%
\pgfpathquadraticcurveto{\pgfqpoint{1.307804in}{4.056458in}}{\pgfqpoint{1.330943in}{4.056458in}}%
\pgfpathclose%
\pgfusepath{stroke,fill}%
\end{pgfscope}%
\begin{pgfscope}%
\pgfsetbuttcap%
\pgfsetmiterjoin%
\definecolor{currentfill}{rgb}{0.248058,0.667205,0.350250}%
\pgfsetfillcolor{currentfill}%
\pgfsetfillopacity{0.500000}%
\pgfsetlinewidth{0.000000pt}%
\definecolor{currentstroke}{rgb}{0.000000,0.000000,0.000000}%
\pgfsetstrokecolor{currentstroke}%
\pgfsetstrokeopacity{0.500000}%
\pgfsetdash{}{0pt}%
\pgfpathmoveto{\pgfqpoint{1.354081in}{4.126801in}}%
\pgfpathlineto{\pgfqpoint{1.585470in}{4.126801in}}%
\pgfpathlineto{\pgfqpoint{1.585470in}{4.207787in}}%
\pgfpathlineto{\pgfqpoint{1.354081in}{4.207787in}}%
\pgfpathclose%
\pgfusepath{fill}%
\end{pgfscope}%
\begin{pgfscope}%
\definecolor{textcolor}{rgb}{0.000000,0.000000,0.000000}%
\pgfsetstrokecolor{textcolor}%
\pgfsetfillcolor{textcolor}%
\pgftext[x=1.678026in,y=4.126801in,left,base]{\color{textcolor}\sffamily\fontsize{8.330000}{9.996000}\selectfont 1}%
\end{pgfscope}%
\begin{pgfscope}%
\pgfsetbuttcap%
\pgfsetmiterjoin%
\definecolor{currentfill}{rgb}{0.488581,0.779931,0.397924}%
\pgfsetfillcolor{currentfill}%
\pgfsetfillopacity{0.500000}%
\pgfsetlinewidth{0.000000pt}%
\definecolor{currentstroke}{rgb}{0.000000,0.000000,0.000000}%
\pgfsetstrokecolor{currentstroke}%
\pgfsetstrokeopacity{0.500000}%
\pgfsetdash{}{0pt}%
\pgfpathmoveto{\pgfqpoint{1.983023in}{4.126801in}}%
\pgfpathlineto{\pgfqpoint{2.214412in}{4.126801in}}%
\pgfpathlineto{\pgfqpoint{2.214412in}{4.207787in}}%
\pgfpathlineto{\pgfqpoint{1.983023in}{4.207787in}}%
\pgfpathclose%
\pgfusepath{fill}%
\end{pgfscope}%
\begin{pgfscope}%
\definecolor{textcolor}{rgb}{0.000000,0.000000,0.000000}%
\pgfsetstrokecolor{textcolor}%
\pgfsetfillcolor{textcolor}%
\pgftext[x=2.306968in,y=4.126801in,left,base]{\color{textcolor}\sffamily\fontsize{8.330000}{9.996000}\selectfont 2}%
\end{pgfscope}%
\begin{pgfscope}%
\pgfsetbuttcap%
\pgfsetmiterjoin%
\definecolor{currentfill}{rgb}{0.701961,0.872972,0.448674}%
\pgfsetfillcolor{currentfill}%
\pgfsetfillopacity{0.500000}%
\pgfsetlinewidth{0.000000pt}%
\definecolor{currentstroke}{rgb}{0.000000,0.000000,0.000000}%
\pgfsetstrokecolor{currentstroke}%
\pgfsetstrokeopacity{0.500000}%
\pgfsetdash{}{0pt}%
\pgfpathmoveto{\pgfqpoint{2.611965in}{4.126801in}}%
\pgfpathlineto{\pgfqpoint{2.843354in}{4.126801in}}%
\pgfpathlineto{\pgfqpoint{2.843354in}{4.207787in}}%
\pgfpathlineto{\pgfqpoint{2.611965in}{4.207787in}}%
\pgfpathclose%
\pgfusepath{fill}%
\end{pgfscope}%
\begin{pgfscope}%
\definecolor{textcolor}{rgb}{0.000000,0.000000,0.000000}%
\pgfsetstrokecolor{textcolor}%
\pgfsetfillcolor{textcolor}%
\pgftext[x=2.935909in,y=4.126801in,left,base]{\color{textcolor}\sffamily\fontsize{8.330000}{9.996000}\selectfont 3}%
\end{pgfscope}%
\begin{pgfscope}%
\pgfsetbuttcap%
\pgfsetmiterjoin%
\definecolor{currentfill}{rgb}{0.868512,0.944637,0.569089}%
\pgfsetfillcolor{currentfill}%
\pgfsetfillopacity{0.500000}%
\pgfsetlinewidth{0.000000pt}%
\definecolor{currentstroke}{rgb}{0.000000,0.000000,0.000000}%
\pgfsetstrokecolor{currentstroke}%
\pgfsetstrokeopacity{0.500000}%
\pgfsetdash{}{0pt}%
\pgfpathmoveto{\pgfqpoint{3.240906in}{4.126801in}}%
\pgfpathlineto{\pgfqpoint{3.472295in}{4.126801in}}%
\pgfpathlineto{\pgfqpoint{3.472295in}{4.207787in}}%
\pgfpathlineto{\pgfqpoint{3.240906in}{4.207787in}}%
\pgfpathclose%
\pgfusepath{fill}%
\end{pgfscope}%
\begin{pgfscope}%
\definecolor{textcolor}{rgb}{0.000000,0.000000,0.000000}%
\pgfsetstrokecolor{textcolor}%
\pgfsetfillcolor{textcolor}%
\pgftext[x=3.564851in,y=4.126801in,left,base]{\color{textcolor}\sffamily\fontsize{8.330000}{9.996000}\selectfont 4}%
\end{pgfscope}%
\begin{pgfscope}%
\pgfsetbuttcap%
\pgfsetmiterjoin%
\definecolor{currentfill}{rgb}{0.997078,0.998770,0.745021}%
\pgfsetfillcolor{currentfill}%
\pgfsetfillopacity{0.500000}%
\pgfsetlinewidth{0.000000pt}%
\definecolor{currentstroke}{rgb}{0.000000,0.000000,0.000000}%
\pgfsetstrokecolor{currentstroke}%
\pgfsetstrokeopacity{0.500000}%
\pgfsetdash{}{0pt}%
\pgfpathmoveto{\pgfqpoint{3.869848in}{4.126801in}}%
\pgfpathlineto{\pgfqpoint{4.101237in}{4.126801in}}%
\pgfpathlineto{\pgfqpoint{4.101237in}{4.207787in}}%
\pgfpathlineto{\pgfqpoint{3.869848in}{4.207787in}}%
\pgfpathclose%
\pgfusepath{fill}%
\end{pgfscope}%
\begin{pgfscope}%
\definecolor{textcolor}{rgb}{0.000000,0.000000,0.000000}%
\pgfsetstrokecolor{textcolor}%
\pgfsetfillcolor{textcolor}%
\pgftext[x=4.193792in,y=4.126801in,left,base]{\color{textcolor}\sffamily\fontsize{8.330000}{9.996000}\selectfont 5}%
\end{pgfscope}%
\begin{pgfscope}%
\pgfsetbuttcap%
\pgfsetmiterjoin%
\definecolor{currentfill}{rgb}{0.996540,0.892734,0.569089}%
\pgfsetfillcolor{currentfill}%
\pgfsetfillopacity{0.500000}%
\pgfsetlinewidth{0.000000pt}%
\definecolor{currentstroke}{rgb}{0.000000,0.000000,0.000000}%
\pgfsetstrokecolor{currentstroke}%
\pgfsetstrokeopacity{0.500000}%
\pgfsetdash{}{0pt}%
\pgfpathmoveto{\pgfqpoint{4.498790in}{4.126801in}}%
\pgfpathlineto{\pgfqpoint{4.730179in}{4.126801in}}%
\pgfpathlineto{\pgfqpoint{4.730179in}{4.207787in}}%
\pgfpathlineto{\pgfqpoint{4.498790in}{4.207787in}}%
\pgfpathclose%
\pgfusepath{fill}%
\end{pgfscope}%
\begin{pgfscope}%
\definecolor{textcolor}{rgb}{0.000000,0.000000,0.000000}%
\pgfsetstrokecolor{textcolor}%
\pgfsetfillcolor{textcolor}%
\pgftext[x=4.822734in,y=4.126801in,left,base]{\color{textcolor}\sffamily\fontsize{8.330000}{9.996000}\selectfont 6}%
\end{pgfscope}%
\begin{pgfscope}%
\pgfsetbuttcap%
\pgfsetmiterjoin%
\definecolor{currentfill}{rgb}{0.993156,0.732334,0.422376}%
\pgfsetfillcolor{currentfill}%
\pgfsetfillopacity{0.500000}%
\pgfsetlinewidth{0.000000pt}%
\definecolor{currentstroke}{rgb}{0.000000,0.000000,0.000000}%
\pgfsetstrokecolor{currentstroke}%
\pgfsetstrokeopacity{0.500000}%
\pgfsetdash{}{0pt}%
\pgfpathmoveto{\pgfqpoint{5.127731in}{4.126801in}}%
\pgfpathlineto{\pgfqpoint{5.359120in}{4.126801in}}%
\pgfpathlineto{\pgfqpoint{5.359120in}{4.207787in}}%
\pgfpathlineto{\pgfqpoint{5.127731in}{4.207787in}}%
\pgfpathclose%
\pgfusepath{fill}%
\end{pgfscope}%
\begin{pgfscope}%
\definecolor{textcolor}{rgb}{0.000000,0.000000,0.000000}%
\pgfsetstrokecolor{textcolor}%
\pgfsetfillcolor{textcolor}%
\pgftext[x=5.451676in,y=4.126801in,left,base]{\color{textcolor}\sffamily\fontsize{8.330000}{9.996000}\selectfont 7}%
\end{pgfscope}%
\begin{pgfscope}%
\pgfsetbuttcap%
\pgfsetmiterjoin%
\definecolor{currentfill}{rgb}{0.969319,0.517416,0.304268}%
\pgfsetfillcolor{currentfill}%
\pgfsetfillopacity{0.500000}%
\pgfsetlinewidth{0.000000pt}%
\definecolor{currentstroke}{rgb}{0.000000,0.000000,0.000000}%
\pgfsetstrokecolor{currentstroke}%
\pgfsetstrokeopacity{0.500000}%
\pgfsetdash{}{0pt}%
\pgfpathmoveto{\pgfqpoint{5.756673in}{4.126801in}}%
\pgfpathlineto{\pgfqpoint{5.988062in}{4.126801in}}%
\pgfpathlineto{\pgfqpoint{5.988062in}{4.207787in}}%
\pgfpathlineto{\pgfqpoint{5.756673in}{4.207787in}}%
\pgfpathclose%
\pgfusepath{fill}%
\end{pgfscope}%
\begin{pgfscope}%
\definecolor{textcolor}{rgb}{0.000000,0.000000,0.000000}%
\pgfsetstrokecolor{textcolor}%
\pgfsetfillcolor{textcolor}%
\pgftext[x=6.080617in,y=4.126801in,left,base]{\color{textcolor}\sffamily\fontsize{8.330000}{9.996000}\selectfont 8}%
\end{pgfscope}%
\begin{pgfscope}%
\pgfsetbuttcap%
\pgfsetmiterjoin%
\definecolor{currentfill}{rgb}{0.898885,0.305498,0.206767}%
\pgfsetfillcolor{currentfill}%
\pgfsetfillopacity{0.500000}%
\pgfsetlinewidth{0.000000pt}%
\definecolor{currentstroke}{rgb}{0.000000,0.000000,0.000000}%
\pgfsetstrokecolor{currentstroke}%
\pgfsetstrokeopacity{0.500000}%
\pgfsetdash{}{0pt}%
\pgfpathmoveto{\pgfqpoint{6.385615in}{4.126801in}}%
\pgfpathlineto{\pgfqpoint{6.617004in}{4.126801in}}%
\pgfpathlineto{\pgfqpoint{6.617004in}{4.207787in}}%
\pgfpathlineto{\pgfqpoint{6.385615in}{4.207787in}}%
\pgfpathclose%
\pgfusepath{fill}%
\end{pgfscope}%
\begin{pgfscope}%
\definecolor{textcolor}{rgb}{0.000000,0.000000,0.000000}%
\pgfsetstrokecolor{textcolor}%
\pgfsetfillcolor{textcolor}%
\pgftext[x=6.709559in,y=4.126801in,left,base]{\color{textcolor}\sffamily\fontsize{8.330000}{9.996000}\selectfont 9}%
\end{pgfscope}%
\end{pgfpicture}%
\makeatother%
\endgroup%
}
    \caption[Distibution for Survey:Section 3]{The figure represents distribution for Section 3 of survey. The participants were asked to rank the images based on photorealism(1 being the best) by comparing images from all 9 datasets.
    The automated datasets are highlighted. \gls{free} dataset appears in top 5 maximum number of times, seen to the left of the dotted line.}
    \label{fig:question3}
\end{figure}

\begin{figure}
    \centering
    \resizebox{0.75\textwidth}{!}{%% Creator: Matplotlib, PGF backend
%%
%% To include the figure in your LaTeX document, write
%%   \input{<filename>.pgf}
%%
%% Make sure the required packages are loaded in your preamble
%%   \usepackage{pgf}
%%
%% Figures using additional raster images can only be included by \input if
%% they are in the same directory as the main LaTeX file. For loading figures
%% from other directories you can use the `import` package
%%   \usepackage{import}
%%
%% and then include the figures with
%%   \import{<path to file>}{<filename>.pgf}
%%
%% Matplotlib used the following preamble
%%   \usepackage{fontspec}
%%   \setmainfont{DejaVuSerif.ttf}[Path=\detokenize{/Users/apple/opt/anaconda3/envs/kaolin/lib/python3.7/site-packages/matplotlib/mpl-data/fonts/ttf/}]
%%   \setsansfont{DejaVuSans.ttf}[Path=\detokenize{/Users/apple/opt/anaconda3/envs/kaolin/lib/python3.7/site-packages/matplotlib/mpl-data/fonts/ttf/}]
%%   \setmonofont{DejaVuSansMono.ttf}[Path=\detokenize{/Users/apple/opt/anaconda3/envs/kaolin/lib/python3.7/site-packages/matplotlib/mpl-data/fonts/ttf/}]
%%
\begingroup%
\makeatletter%
\begin{pgfpicture}%
\pgfpathrectangle{\pgfpointorigin}{\pgfqpoint{5.543431in}{5.133701in}}%
\pgfusepath{use as bounding box, clip}%
\begin{pgfscope}%
\pgfsetbuttcap%
\pgfsetmiterjoin%
\definecolor{currentfill}{rgb}{1.000000,1.000000,1.000000}%
\pgfsetfillcolor{currentfill}%
\pgfsetlinewidth{0.000000pt}%
\definecolor{currentstroke}{rgb}{1.000000,1.000000,1.000000}%
\pgfsetstrokecolor{currentstroke}%
\pgfsetdash{}{0pt}%
\pgfpathmoveto{\pgfqpoint{0.000000in}{0.000000in}}%
\pgfpathlineto{\pgfqpoint{5.543431in}{0.000000in}}%
\pgfpathlineto{\pgfqpoint{5.543431in}{5.133701in}}%
\pgfpathlineto{\pgfqpoint{0.000000in}{5.133701in}}%
\pgfpathclose%
\pgfusepath{fill}%
\end{pgfscope}%
\begin{pgfscope}%
\pgfsetbuttcap%
\pgfsetmiterjoin%
\definecolor{currentfill}{rgb}{1.000000,1.000000,1.000000}%
\pgfsetfillcolor{currentfill}%
\pgfsetlinewidth{0.000000pt}%
\definecolor{currentstroke}{rgb}{0.000000,0.000000,0.000000}%
\pgfsetstrokecolor{currentstroke}%
\pgfsetstrokeopacity{0.000000}%
\pgfsetdash{}{0pt}%
\pgfpathmoveto{\pgfqpoint{0.475556in}{1.127740in}}%
\pgfpathlineto{\pgfqpoint{5.435556in}{1.127740in}}%
\pgfpathlineto{\pgfqpoint{5.435556in}{4.823740in}}%
\pgfpathlineto{\pgfqpoint{0.475556in}{4.823740in}}%
\pgfpathclose%
\pgfusepath{fill}%
\end{pgfscope}%
\begin{pgfscope}%
\pgfsetbuttcap%
\pgfsetroundjoin%
\definecolor{currentfill}{rgb}{0.000000,0.000000,0.000000}%
\pgfsetfillcolor{currentfill}%
\pgfsetlinewidth{0.803000pt}%
\definecolor{currentstroke}{rgb}{0.000000,0.000000,0.000000}%
\pgfsetstrokecolor{currentstroke}%
\pgfsetdash{}{0pt}%
\pgfsys@defobject{currentmarker}{\pgfqpoint{0.000000in}{-0.048611in}}{\pgfqpoint{0.000000in}{0.000000in}}{%
\pgfpathmoveto{\pgfqpoint{0.000000in}{0.000000in}}%
\pgfpathlineto{\pgfqpoint{0.000000in}{-0.048611in}}%
\pgfusepath{stroke,fill}%
}%
\begin{pgfscope}%
\pgfsys@transformshift{0.751111in}{1.127740in}%
\pgfsys@useobject{currentmarker}{}%
\end{pgfscope}%
\end{pgfscope}%
\begin{pgfscope}%
\definecolor{textcolor}{rgb}{0.000000,0.000000,0.000000}%
\pgfsetstrokecolor{textcolor}%
\pgfsetfillcolor{textcolor}%
\pgftext[x=0.541410in, y=0.482311in, left, base,rotate=45.000000]{\color{textcolor}\sffamily\fontsize{10.000000}{12.000000}\selectfont 3DFRONT}%
\end{pgfscope}%
\begin{pgfscope}%
\pgfsetbuttcap%
\pgfsetroundjoin%
\definecolor{currentfill}{rgb}{0.000000,0.000000,0.000000}%
\pgfsetfillcolor{currentfill}%
\pgfsetlinewidth{0.803000pt}%
\definecolor{currentstroke}{rgb}{0.000000,0.000000,0.000000}%
\pgfsetstrokecolor{currentstroke}%
\pgfsetdash{}{0pt}%
\pgfsys@defobject{currentmarker}{\pgfqpoint{0.000000in}{-0.048611in}}{\pgfqpoint{0.000000in}{0.000000in}}{%
\pgfpathmoveto{\pgfqpoint{0.000000in}{0.000000in}}%
\pgfpathlineto{\pgfqpoint{0.000000in}{-0.048611in}}%
\pgfusepath{stroke,fill}%
}%
\begin{pgfscope}%
\pgfsys@transformshift{1.302222in}{1.127740in}%
\pgfsys@useobject{currentmarker}{}%
\end{pgfscope}%
\end{pgfscope}%
\begin{pgfscope}%
\definecolor{textcolor}{rgb}{0.000000,0.000000,0.000000}%
\pgfsetstrokecolor{textcolor}%
\pgfsetfillcolor{textcolor}%
\pgftext[x=1.114652in, y=0.526572in, left, base,rotate=45.000000]{\color{textcolor}\sffamily\fontsize{10.000000}{12.000000}\selectfont ai2THOR}%
\end{pgfscope}%
\begin{pgfscope}%
\pgfsetbuttcap%
\pgfsetroundjoin%
\definecolor{currentfill}{rgb}{0.000000,0.000000,0.000000}%
\pgfsetfillcolor{currentfill}%
\pgfsetlinewidth{0.803000pt}%
\definecolor{currentstroke}{rgb}{0.000000,0.000000,0.000000}%
\pgfsetstrokecolor{currentstroke}%
\pgfsetdash{}{0pt}%
\pgfsys@defobject{currentmarker}{\pgfqpoint{0.000000in}{-0.048611in}}{\pgfqpoint{0.000000in}{0.000000in}}{%
\pgfpathmoveto{\pgfqpoint{0.000000in}{0.000000in}}%
\pgfpathlineto{\pgfqpoint{0.000000in}{-0.048611in}}%
\pgfusepath{stroke,fill}%
}%
\begin{pgfscope}%
\pgfsys@transformshift{1.853334in}{1.127740in}%
\pgfsys@useobject{currentmarker}{}%
\end{pgfscope}%
\end{pgfscope}%
\begin{pgfscope}%
\definecolor{textcolor}{rgb}{0.000000,0.000000,0.000000}%
\pgfsetstrokecolor{textcolor}%
\pgfsetfillcolor{textcolor}%
\pgftext[x=1.582875in, y=0.360796in, left, base,rotate=45.000000]{\color{textcolor}\sffamily\fontsize{10.000000}{12.000000}\selectfont Blenderproc}%
\end{pgfscope}%
\begin{pgfscope}%
\pgfsetbuttcap%
\pgfsetroundjoin%
\definecolor{currentfill}{rgb}{0.000000,0.000000,0.000000}%
\pgfsetfillcolor{currentfill}%
\pgfsetlinewidth{0.803000pt}%
\definecolor{currentstroke}{rgb}{0.000000,0.000000,0.000000}%
\pgfsetstrokecolor{currentstroke}%
\pgfsetdash{}{0pt}%
\pgfsys@defobject{currentmarker}{\pgfqpoint{0.000000in}{-0.048611in}}{\pgfqpoint{0.000000in}{0.000000in}}{%
\pgfpathmoveto{\pgfqpoint{0.000000in}{0.000000in}}%
\pgfpathlineto{\pgfqpoint{0.000000in}{-0.048611in}}%
\pgfusepath{stroke,fill}%
}%
\begin{pgfscope}%
\pgfsys@transformshift{2.404445in}{1.127740in}%
\pgfsys@useobject{currentmarker}{}%
\end{pgfscope}%
\end{pgfscope}%
\begin{pgfscope}%
\definecolor{textcolor}{rgb}{0.000000,0.000000,0.000000}%
\pgfsetstrokecolor{textcolor}%
\pgfsetfillcolor{textcolor}%
\pgftext[x=2.196925in, y=0.486674in, left, base,rotate=45.000000]{\color{textcolor}\sffamily\fontsize{10.000000}{12.000000}\selectfont Hyperism}%
\end{pgfscope}%
\begin{pgfscope}%
\pgfsetbuttcap%
\pgfsetroundjoin%
\definecolor{currentfill}{rgb}{0.000000,0.000000,0.000000}%
\pgfsetfillcolor{currentfill}%
\pgfsetlinewidth{0.803000pt}%
\definecolor{currentstroke}{rgb}{0.000000,0.000000,0.000000}%
\pgfsetstrokecolor{currentstroke}%
\pgfsetdash{}{0pt}%
\pgfsys@defobject{currentmarker}{\pgfqpoint{0.000000in}{-0.048611in}}{\pgfqpoint{0.000000in}{0.000000in}}{%
\pgfpathmoveto{\pgfqpoint{0.000000in}{0.000000in}}%
\pgfpathlineto{\pgfqpoint{0.000000in}{-0.048611in}}%
\pgfusepath{stroke,fill}%
}%
\begin{pgfscope}%
\pgfsys@transformshift{2.955556in}{1.127740in}%
\pgfsys@useobject{currentmarker}{}%
\end{pgfscope}%
\end{pgfscope}%
\begin{pgfscope}%
\definecolor{textcolor}{rgb}{0.000000,0.000000,0.000000}%
\pgfsetstrokecolor{textcolor}%
\pgfsetfillcolor{textcolor}%
\pgftext[x=2.717322in, y=0.425246in, left, base,rotate=45.000000]{\color{textcolor}\sffamily\fontsize{10.000000}{12.000000}\selectfont InteriorNet}%
\end{pgfscope}%
\begin{pgfscope}%
\pgfsetbuttcap%
\pgfsetroundjoin%
\definecolor{currentfill}{rgb}{0.000000,0.000000,0.000000}%
\pgfsetfillcolor{currentfill}%
\pgfsetlinewidth{0.803000pt}%
\definecolor{currentstroke}{rgb}{0.000000,0.000000,0.000000}%
\pgfsetstrokecolor{currentstroke}%
\pgfsetdash{}{0pt}%
\pgfsys@defobject{currentmarker}{\pgfqpoint{0.000000in}{-0.048611in}}{\pgfqpoint{0.000000in}{0.000000in}}{%
\pgfpathmoveto{\pgfqpoint{0.000000in}{0.000000in}}%
\pgfpathlineto{\pgfqpoint{0.000000in}{-0.048611in}}%
\pgfusepath{stroke,fill}%
}%
\begin{pgfscope}%
\pgfsys@transformshift{3.506667in}{1.127740in}%
\pgfsys@useobject{currentmarker}{}%
\end{pgfscope}%
\end{pgfscope}%
\begin{pgfscope}%
\definecolor{textcolor}{rgb}{0.000000,0.000000,0.000000}%
\pgfsetstrokecolor{textcolor}%
\pgfsetfillcolor{textcolor}%
\pgftext[x=3.237191in, y=0.362762in, left, base,rotate=45.000000]{\color{textcolor}\sffamily\fontsize{10.000000}{12.000000}\selectfont OpenRooms}%
\end{pgfscope}%
\begin{pgfscope}%
\pgfsetbuttcap%
\pgfsetroundjoin%
\definecolor{currentfill}{rgb}{0.000000,0.000000,0.000000}%
\pgfsetfillcolor{currentfill}%
\pgfsetlinewidth{0.803000pt}%
\definecolor{currentstroke}{rgb}{0.000000,0.000000,0.000000}%
\pgfsetstrokecolor{currentstroke}%
\pgfsetdash{}{0pt}%
\pgfsys@defobject{currentmarker}{\pgfqpoint{0.000000in}{-0.048611in}}{\pgfqpoint{0.000000in}{0.000000in}}{%
\pgfpathmoveto{\pgfqpoint{0.000000in}{0.000000in}}%
\pgfpathlineto{\pgfqpoint{0.000000in}{-0.048611in}}%
\pgfusepath{stroke,fill}%
}%
\begin{pgfscope}%
\pgfsys@transformshift{4.057778in}{1.127740in}%
\pgfsys@useobject{currentmarker}{}%
\end{pgfscope}%
\end{pgfscope}%
\begin{pgfscope}%
\definecolor{textcolor}{rgb}{0.000000,0.000000,0.000000}%
\pgfsetstrokecolor{textcolor}%
\pgfsetfillcolor{textcolor}%
\pgftext[x=3.944583in, y=0.675324in, left, base,rotate=45.000000]{\color{textcolor}\sffamily\fontsize{10.000000}{12.000000}\selectfont Pix3D}%
\end{pgfscope}%
\begin{pgfscope}%
\pgfsetbuttcap%
\pgfsetroundjoin%
\definecolor{currentfill}{rgb}{0.000000,0.000000,0.000000}%
\pgfsetfillcolor{currentfill}%
\pgfsetlinewidth{0.803000pt}%
\definecolor{currentstroke}{rgb}{0.000000,0.000000,0.000000}%
\pgfsetstrokecolor{currentstroke}%
\pgfsetdash{}{0pt}%
\pgfsys@defobject{currentmarker}{\pgfqpoint{0.000000in}{-0.048611in}}{\pgfqpoint{0.000000in}{0.000000in}}{%
\pgfpathmoveto{\pgfqpoint{0.000000in}{0.000000in}}%
\pgfpathlineto{\pgfqpoint{0.000000in}{-0.048611in}}%
\pgfusepath{stroke,fill}%
}%
\begin{pgfscope}%
\pgfsys@transformshift{4.608889in}{1.127740in}%
\pgfsys@useobject{currentmarker}{}%
\end{pgfscope}%
\end{pgfscope}%
\begin{pgfscope}%
\definecolor{textcolor}{rgb}{0.000000,0.000000,0.000000}%
\pgfsetstrokecolor{textcolor}%
\pgfsetfillcolor{textcolor}%
\pgftext[x=4.313230in, y=0.310396in, left, base,rotate=45.000000]{\color{textcolor}\sffamily\fontsize{10.000000}{12.000000}\selectfont S2R:3D-FREE}%
\end{pgfscope}%
\begin{pgfscope}%
\pgfsetbuttcap%
\pgfsetroundjoin%
\definecolor{currentfill}{rgb}{0.000000,0.000000,0.000000}%
\pgfsetfillcolor{currentfill}%
\pgfsetlinewidth{0.803000pt}%
\definecolor{currentstroke}{rgb}{0.000000,0.000000,0.000000}%
\pgfsetstrokecolor{currentstroke}%
\pgfsetdash{}{0pt}%
\pgfsys@defobject{currentmarker}{\pgfqpoint{0.000000in}{-0.048611in}}{\pgfqpoint{0.000000in}{0.000000in}}{%
\pgfpathmoveto{\pgfqpoint{0.000000in}{0.000000in}}%
\pgfpathlineto{\pgfqpoint{0.000000in}{-0.048611in}}%
\pgfusepath{stroke,fill}%
}%
\begin{pgfscope}%
\pgfsys@transformshift{5.160000in}{1.127740in}%
\pgfsys@useobject{currentmarker}{}%
\end{pgfscope}%
\end{pgfscope}%
\begin{pgfscope}%
\definecolor{textcolor}{rgb}{0.000000,0.000000,0.000000}%
\pgfsetstrokecolor{textcolor}%
\pgfsetfillcolor{textcolor}%
\pgftext[x=4.951186in, y=0.484085in, left, base,rotate=45.000000]{\color{textcolor}\sffamily\fontsize{10.000000}{12.000000}\selectfont SceneNet}%
\end{pgfscope}%
\begin{pgfscope}%
\definecolor{textcolor}{rgb}{0.000000,0.000000,0.000000}%
\pgfsetstrokecolor{textcolor}%
\pgfsetfillcolor{textcolor}%
\pgftext[x=2.955556in,y=0.234413in,,top]{\color{textcolor}\sffamily\fontsize{10.000000}{12.000000}\selectfont Datasets}%
\end{pgfscope}%
\begin{pgfscope}%
\pgfpathrectangle{\pgfqpoint{0.475556in}{1.127740in}}{\pgfqpoint{4.960000in}{3.696000in}}%
\pgfusepath{clip}%
\pgfsetrectcap%
\pgfsetroundjoin%
\pgfsetlinewidth{0.803000pt}%
\definecolor{currentstroke}{rgb}{0.690196,0.690196,0.690196}%
\pgfsetstrokecolor{currentstroke}%
\pgfsetdash{}{0pt}%
\pgfpathmoveto{\pgfqpoint{0.475556in}{1.295740in}}%
\pgfpathlineto{\pgfqpoint{5.435556in}{1.295740in}}%
\pgfusepath{stroke}%
\end{pgfscope}%
\begin{pgfscope}%
\pgfsetbuttcap%
\pgfsetroundjoin%
\definecolor{currentfill}{rgb}{0.000000,0.000000,0.000000}%
\pgfsetfillcolor{currentfill}%
\pgfsetlinewidth{0.803000pt}%
\definecolor{currentstroke}{rgb}{0.000000,0.000000,0.000000}%
\pgfsetstrokecolor{currentstroke}%
\pgfsetdash{}{0pt}%
\pgfsys@defobject{currentmarker}{\pgfqpoint{-0.048611in}{0.000000in}}{\pgfqpoint{-0.000000in}{0.000000in}}{%
\pgfpathmoveto{\pgfqpoint{-0.000000in}{0.000000in}}%
\pgfpathlineto{\pgfqpoint{-0.048611in}{0.000000in}}%
\pgfusepath{stroke,fill}%
}%
\begin{pgfscope}%
\pgfsys@transformshift{0.475556in}{1.295740in}%
\pgfsys@useobject{currentmarker}{}%
\end{pgfscope}%
\end{pgfscope}%
\begin{pgfscope}%
\definecolor{textcolor}{rgb}{0.000000,0.000000,0.000000}%
\pgfsetstrokecolor{textcolor}%
\pgfsetfillcolor{textcolor}%
\pgftext[x=0.289968in, y=1.242979in, left, base]{\color{textcolor}\sffamily\fontsize{10.000000}{12.000000}\selectfont 1}%
\end{pgfscope}%
\begin{pgfscope}%
\pgfpathrectangle{\pgfqpoint{0.475556in}{1.127740in}}{\pgfqpoint{4.960000in}{3.696000in}}%
\pgfusepath{clip}%
\pgfsetrectcap%
\pgfsetroundjoin%
\pgfsetlinewidth{0.803000pt}%
\definecolor{currentstroke}{rgb}{0.690196,0.690196,0.690196}%
\pgfsetstrokecolor{currentstroke}%
\pgfsetdash{}{0pt}%
\pgfpathmoveto{\pgfqpoint{0.475556in}{1.715740in}}%
\pgfpathlineto{\pgfqpoint{5.435556in}{1.715740in}}%
\pgfusepath{stroke}%
\end{pgfscope}%
\begin{pgfscope}%
\pgfsetbuttcap%
\pgfsetroundjoin%
\definecolor{currentfill}{rgb}{0.000000,0.000000,0.000000}%
\pgfsetfillcolor{currentfill}%
\pgfsetlinewidth{0.803000pt}%
\definecolor{currentstroke}{rgb}{0.000000,0.000000,0.000000}%
\pgfsetstrokecolor{currentstroke}%
\pgfsetdash{}{0pt}%
\pgfsys@defobject{currentmarker}{\pgfqpoint{-0.048611in}{0.000000in}}{\pgfqpoint{-0.000000in}{0.000000in}}{%
\pgfpathmoveto{\pgfqpoint{-0.000000in}{0.000000in}}%
\pgfpathlineto{\pgfqpoint{-0.048611in}{0.000000in}}%
\pgfusepath{stroke,fill}%
}%
\begin{pgfscope}%
\pgfsys@transformshift{0.475556in}{1.715740in}%
\pgfsys@useobject{currentmarker}{}%
\end{pgfscope}%
\end{pgfscope}%
\begin{pgfscope}%
\definecolor{textcolor}{rgb}{0.000000,0.000000,0.000000}%
\pgfsetstrokecolor{textcolor}%
\pgfsetfillcolor{textcolor}%
\pgftext[x=0.289968in, y=1.662979in, left, base]{\color{textcolor}\sffamily\fontsize{10.000000}{12.000000}\selectfont 2}%
\end{pgfscope}%
\begin{pgfscope}%
\pgfpathrectangle{\pgfqpoint{0.475556in}{1.127740in}}{\pgfqpoint{4.960000in}{3.696000in}}%
\pgfusepath{clip}%
\pgfsetrectcap%
\pgfsetroundjoin%
\pgfsetlinewidth{0.803000pt}%
\definecolor{currentstroke}{rgb}{0.690196,0.690196,0.690196}%
\pgfsetstrokecolor{currentstroke}%
\pgfsetdash{}{0pt}%
\pgfpathmoveto{\pgfqpoint{0.475556in}{2.135740in}}%
\pgfpathlineto{\pgfqpoint{5.435556in}{2.135740in}}%
\pgfusepath{stroke}%
\end{pgfscope}%
\begin{pgfscope}%
\pgfsetbuttcap%
\pgfsetroundjoin%
\definecolor{currentfill}{rgb}{0.000000,0.000000,0.000000}%
\pgfsetfillcolor{currentfill}%
\pgfsetlinewidth{0.803000pt}%
\definecolor{currentstroke}{rgb}{0.000000,0.000000,0.000000}%
\pgfsetstrokecolor{currentstroke}%
\pgfsetdash{}{0pt}%
\pgfsys@defobject{currentmarker}{\pgfqpoint{-0.048611in}{0.000000in}}{\pgfqpoint{-0.000000in}{0.000000in}}{%
\pgfpathmoveto{\pgfqpoint{-0.000000in}{0.000000in}}%
\pgfpathlineto{\pgfqpoint{-0.048611in}{0.000000in}}%
\pgfusepath{stroke,fill}%
}%
\begin{pgfscope}%
\pgfsys@transformshift{0.475556in}{2.135740in}%
\pgfsys@useobject{currentmarker}{}%
\end{pgfscope}%
\end{pgfscope}%
\begin{pgfscope}%
\definecolor{textcolor}{rgb}{0.000000,0.000000,0.000000}%
\pgfsetstrokecolor{textcolor}%
\pgfsetfillcolor{textcolor}%
\pgftext[x=0.289968in, y=2.082979in, left, base]{\color{textcolor}\sffamily\fontsize{10.000000}{12.000000}\selectfont 3}%
\end{pgfscope}%
\begin{pgfscope}%
\pgfpathrectangle{\pgfqpoint{0.475556in}{1.127740in}}{\pgfqpoint{4.960000in}{3.696000in}}%
\pgfusepath{clip}%
\pgfsetrectcap%
\pgfsetroundjoin%
\pgfsetlinewidth{0.803000pt}%
\definecolor{currentstroke}{rgb}{0.690196,0.690196,0.690196}%
\pgfsetstrokecolor{currentstroke}%
\pgfsetdash{}{0pt}%
\pgfpathmoveto{\pgfqpoint{0.475556in}{2.555740in}}%
\pgfpathlineto{\pgfqpoint{5.435556in}{2.555740in}}%
\pgfusepath{stroke}%
\end{pgfscope}%
\begin{pgfscope}%
\pgfsetbuttcap%
\pgfsetroundjoin%
\definecolor{currentfill}{rgb}{0.000000,0.000000,0.000000}%
\pgfsetfillcolor{currentfill}%
\pgfsetlinewidth{0.803000pt}%
\definecolor{currentstroke}{rgb}{0.000000,0.000000,0.000000}%
\pgfsetstrokecolor{currentstroke}%
\pgfsetdash{}{0pt}%
\pgfsys@defobject{currentmarker}{\pgfqpoint{-0.048611in}{0.000000in}}{\pgfqpoint{-0.000000in}{0.000000in}}{%
\pgfpathmoveto{\pgfqpoint{-0.000000in}{0.000000in}}%
\pgfpathlineto{\pgfqpoint{-0.048611in}{0.000000in}}%
\pgfusepath{stroke,fill}%
}%
\begin{pgfscope}%
\pgfsys@transformshift{0.475556in}{2.555740in}%
\pgfsys@useobject{currentmarker}{}%
\end{pgfscope}%
\end{pgfscope}%
\begin{pgfscope}%
\definecolor{textcolor}{rgb}{0.000000,0.000000,0.000000}%
\pgfsetstrokecolor{textcolor}%
\pgfsetfillcolor{textcolor}%
\pgftext[x=0.289968in, y=2.502979in, left, base]{\color{textcolor}\sffamily\fontsize{10.000000}{12.000000}\selectfont 4}%
\end{pgfscope}%
\begin{pgfscope}%
\pgfpathrectangle{\pgfqpoint{0.475556in}{1.127740in}}{\pgfqpoint{4.960000in}{3.696000in}}%
\pgfusepath{clip}%
\pgfsetrectcap%
\pgfsetroundjoin%
\pgfsetlinewidth{0.803000pt}%
\definecolor{currentstroke}{rgb}{0.690196,0.690196,0.690196}%
\pgfsetstrokecolor{currentstroke}%
\pgfsetdash{}{0pt}%
\pgfpathmoveto{\pgfqpoint{0.475556in}{2.975740in}}%
\pgfpathlineto{\pgfqpoint{5.435556in}{2.975740in}}%
\pgfusepath{stroke}%
\end{pgfscope}%
\begin{pgfscope}%
\pgfsetbuttcap%
\pgfsetroundjoin%
\definecolor{currentfill}{rgb}{0.000000,0.000000,0.000000}%
\pgfsetfillcolor{currentfill}%
\pgfsetlinewidth{0.803000pt}%
\definecolor{currentstroke}{rgb}{0.000000,0.000000,0.000000}%
\pgfsetstrokecolor{currentstroke}%
\pgfsetdash{}{0pt}%
\pgfsys@defobject{currentmarker}{\pgfqpoint{-0.048611in}{0.000000in}}{\pgfqpoint{-0.000000in}{0.000000in}}{%
\pgfpathmoveto{\pgfqpoint{-0.000000in}{0.000000in}}%
\pgfpathlineto{\pgfqpoint{-0.048611in}{0.000000in}}%
\pgfusepath{stroke,fill}%
}%
\begin{pgfscope}%
\pgfsys@transformshift{0.475556in}{2.975740in}%
\pgfsys@useobject{currentmarker}{}%
\end{pgfscope}%
\end{pgfscope}%
\begin{pgfscope}%
\definecolor{textcolor}{rgb}{0.000000,0.000000,0.000000}%
\pgfsetstrokecolor{textcolor}%
\pgfsetfillcolor{textcolor}%
\pgftext[x=0.289968in, y=2.922979in, left, base]{\color{textcolor}\sffamily\fontsize{10.000000}{12.000000}\selectfont 5}%
\end{pgfscope}%
\begin{pgfscope}%
\pgfpathrectangle{\pgfqpoint{0.475556in}{1.127740in}}{\pgfqpoint{4.960000in}{3.696000in}}%
\pgfusepath{clip}%
\pgfsetrectcap%
\pgfsetroundjoin%
\pgfsetlinewidth{0.803000pt}%
\definecolor{currentstroke}{rgb}{0.690196,0.690196,0.690196}%
\pgfsetstrokecolor{currentstroke}%
\pgfsetdash{}{0pt}%
\pgfpathmoveto{\pgfqpoint{0.475556in}{3.395740in}}%
\pgfpathlineto{\pgfqpoint{5.435556in}{3.395740in}}%
\pgfusepath{stroke}%
\end{pgfscope}%
\begin{pgfscope}%
\pgfsetbuttcap%
\pgfsetroundjoin%
\definecolor{currentfill}{rgb}{0.000000,0.000000,0.000000}%
\pgfsetfillcolor{currentfill}%
\pgfsetlinewidth{0.803000pt}%
\definecolor{currentstroke}{rgb}{0.000000,0.000000,0.000000}%
\pgfsetstrokecolor{currentstroke}%
\pgfsetdash{}{0pt}%
\pgfsys@defobject{currentmarker}{\pgfqpoint{-0.048611in}{0.000000in}}{\pgfqpoint{-0.000000in}{0.000000in}}{%
\pgfpathmoveto{\pgfqpoint{-0.000000in}{0.000000in}}%
\pgfpathlineto{\pgfqpoint{-0.048611in}{0.000000in}}%
\pgfusepath{stroke,fill}%
}%
\begin{pgfscope}%
\pgfsys@transformshift{0.475556in}{3.395740in}%
\pgfsys@useobject{currentmarker}{}%
\end{pgfscope}%
\end{pgfscope}%
\begin{pgfscope}%
\definecolor{textcolor}{rgb}{0.000000,0.000000,0.000000}%
\pgfsetstrokecolor{textcolor}%
\pgfsetfillcolor{textcolor}%
\pgftext[x=0.289968in, y=3.342979in, left, base]{\color{textcolor}\sffamily\fontsize{10.000000}{12.000000}\selectfont 6}%
\end{pgfscope}%
\begin{pgfscope}%
\pgfpathrectangle{\pgfqpoint{0.475556in}{1.127740in}}{\pgfqpoint{4.960000in}{3.696000in}}%
\pgfusepath{clip}%
\pgfsetrectcap%
\pgfsetroundjoin%
\pgfsetlinewidth{0.803000pt}%
\definecolor{currentstroke}{rgb}{0.690196,0.690196,0.690196}%
\pgfsetstrokecolor{currentstroke}%
\pgfsetdash{}{0pt}%
\pgfpathmoveto{\pgfqpoint{0.475556in}{3.815740in}}%
\pgfpathlineto{\pgfqpoint{5.435556in}{3.815740in}}%
\pgfusepath{stroke}%
\end{pgfscope}%
\begin{pgfscope}%
\pgfsetbuttcap%
\pgfsetroundjoin%
\definecolor{currentfill}{rgb}{0.000000,0.000000,0.000000}%
\pgfsetfillcolor{currentfill}%
\pgfsetlinewidth{0.803000pt}%
\definecolor{currentstroke}{rgb}{0.000000,0.000000,0.000000}%
\pgfsetstrokecolor{currentstroke}%
\pgfsetdash{}{0pt}%
\pgfsys@defobject{currentmarker}{\pgfqpoint{-0.048611in}{0.000000in}}{\pgfqpoint{-0.000000in}{0.000000in}}{%
\pgfpathmoveto{\pgfqpoint{-0.000000in}{0.000000in}}%
\pgfpathlineto{\pgfqpoint{-0.048611in}{0.000000in}}%
\pgfusepath{stroke,fill}%
}%
\begin{pgfscope}%
\pgfsys@transformshift{0.475556in}{3.815740in}%
\pgfsys@useobject{currentmarker}{}%
\end{pgfscope}%
\end{pgfscope}%
\begin{pgfscope}%
\definecolor{textcolor}{rgb}{0.000000,0.000000,0.000000}%
\pgfsetstrokecolor{textcolor}%
\pgfsetfillcolor{textcolor}%
\pgftext[x=0.289968in, y=3.762979in, left, base]{\color{textcolor}\sffamily\fontsize{10.000000}{12.000000}\selectfont 7}%
\end{pgfscope}%
\begin{pgfscope}%
\pgfpathrectangle{\pgfqpoint{0.475556in}{1.127740in}}{\pgfqpoint{4.960000in}{3.696000in}}%
\pgfusepath{clip}%
\pgfsetrectcap%
\pgfsetroundjoin%
\pgfsetlinewidth{0.803000pt}%
\definecolor{currentstroke}{rgb}{0.690196,0.690196,0.690196}%
\pgfsetstrokecolor{currentstroke}%
\pgfsetdash{}{0pt}%
\pgfpathmoveto{\pgfqpoint{0.475556in}{4.235740in}}%
\pgfpathlineto{\pgfqpoint{5.435556in}{4.235740in}}%
\pgfusepath{stroke}%
\end{pgfscope}%
\begin{pgfscope}%
\pgfsetbuttcap%
\pgfsetroundjoin%
\definecolor{currentfill}{rgb}{0.000000,0.000000,0.000000}%
\pgfsetfillcolor{currentfill}%
\pgfsetlinewidth{0.803000pt}%
\definecolor{currentstroke}{rgb}{0.000000,0.000000,0.000000}%
\pgfsetstrokecolor{currentstroke}%
\pgfsetdash{}{0pt}%
\pgfsys@defobject{currentmarker}{\pgfqpoint{-0.048611in}{0.000000in}}{\pgfqpoint{-0.000000in}{0.000000in}}{%
\pgfpathmoveto{\pgfqpoint{-0.000000in}{0.000000in}}%
\pgfpathlineto{\pgfqpoint{-0.048611in}{0.000000in}}%
\pgfusepath{stroke,fill}%
}%
\begin{pgfscope}%
\pgfsys@transformshift{0.475556in}{4.235740in}%
\pgfsys@useobject{currentmarker}{}%
\end{pgfscope}%
\end{pgfscope}%
\begin{pgfscope}%
\definecolor{textcolor}{rgb}{0.000000,0.000000,0.000000}%
\pgfsetstrokecolor{textcolor}%
\pgfsetfillcolor{textcolor}%
\pgftext[x=0.289968in, y=4.182979in, left, base]{\color{textcolor}\sffamily\fontsize{10.000000}{12.000000}\selectfont 8}%
\end{pgfscope}%
\begin{pgfscope}%
\pgfpathrectangle{\pgfqpoint{0.475556in}{1.127740in}}{\pgfqpoint{4.960000in}{3.696000in}}%
\pgfusepath{clip}%
\pgfsetrectcap%
\pgfsetroundjoin%
\pgfsetlinewidth{0.803000pt}%
\definecolor{currentstroke}{rgb}{0.690196,0.690196,0.690196}%
\pgfsetstrokecolor{currentstroke}%
\pgfsetdash{}{0pt}%
\pgfpathmoveto{\pgfqpoint{0.475556in}{4.655740in}}%
\pgfpathlineto{\pgfqpoint{5.435556in}{4.655740in}}%
\pgfusepath{stroke}%
\end{pgfscope}%
\begin{pgfscope}%
\pgfsetbuttcap%
\pgfsetroundjoin%
\definecolor{currentfill}{rgb}{0.000000,0.000000,0.000000}%
\pgfsetfillcolor{currentfill}%
\pgfsetlinewidth{0.803000pt}%
\definecolor{currentstroke}{rgb}{0.000000,0.000000,0.000000}%
\pgfsetstrokecolor{currentstroke}%
\pgfsetdash{}{0pt}%
\pgfsys@defobject{currentmarker}{\pgfqpoint{-0.048611in}{0.000000in}}{\pgfqpoint{-0.000000in}{0.000000in}}{%
\pgfpathmoveto{\pgfqpoint{-0.000000in}{0.000000in}}%
\pgfpathlineto{\pgfqpoint{-0.048611in}{0.000000in}}%
\pgfusepath{stroke,fill}%
}%
\begin{pgfscope}%
\pgfsys@transformshift{0.475556in}{4.655740in}%
\pgfsys@useobject{currentmarker}{}%
\end{pgfscope}%
\end{pgfscope}%
\begin{pgfscope}%
\definecolor{textcolor}{rgb}{0.000000,0.000000,0.000000}%
\pgfsetstrokecolor{textcolor}%
\pgfsetfillcolor{textcolor}%
\pgftext[x=0.289968in, y=4.602979in, left, base]{\color{textcolor}\sffamily\fontsize{10.000000}{12.000000}\selectfont 9}%
\end{pgfscope}%
\begin{pgfscope}%
\definecolor{textcolor}{rgb}{0.000000,0.000000,0.000000}%
\pgfsetstrokecolor{textcolor}%
\pgfsetfillcolor{textcolor}%
\pgftext[x=0.234413in,y=2.975740in,,bottom,rotate=90.000000]{\color{textcolor}\sffamily\fontsize{10.000000}{12.000000}\selectfont Ranks}%
\end{pgfscope}%
\begin{pgfscope}%
\pgfpathrectangle{\pgfqpoint{0.475556in}{1.127740in}}{\pgfqpoint{4.960000in}{3.696000in}}%
\pgfusepath{clip}%
\pgfsetrectcap%
\pgfsetroundjoin%
\pgfsetlinewidth{1.003750pt}%
\definecolor{currentstroke}{rgb}{0.000000,0.000000,0.000000}%
\pgfsetstrokecolor{currentstroke}%
\pgfsetdash{}{0pt}%
\pgfpathmoveto{\pgfqpoint{0.751111in}{2.555740in}}%
\pgfpathlineto{\pgfqpoint{0.751111in}{1.295740in}}%
\pgfusepath{stroke}%
\end{pgfscope}%
\begin{pgfscope}%
\pgfpathrectangle{\pgfqpoint{0.475556in}{1.127740in}}{\pgfqpoint{4.960000in}{3.696000in}}%
\pgfusepath{clip}%
\pgfsetrectcap%
\pgfsetroundjoin%
\pgfsetlinewidth{1.003750pt}%
\definecolor{currentstroke}{rgb}{0.000000,0.000000,0.000000}%
\pgfsetstrokecolor{currentstroke}%
\pgfsetdash{}{0pt}%
\pgfpathmoveto{\pgfqpoint{0.751111in}{4.235740in}}%
\pgfpathlineto{\pgfqpoint{0.751111in}{4.655740in}}%
\pgfusepath{stroke}%
\end{pgfscope}%
\begin{pgfscope}%
\pgfpathrectangle{\pgfqpoint{0.475556in}{1.127740in}}{\pgfqpoint{4.960000in}{3.696000in}}%
\pgfusepath{clip}%
\pgfsetrectcap%
\pgfsetroundjoin%
\pgfsetlinewidth{1.003750pt}%
\definecolor{currentstroke}{rgb}{0.000000,0.000000,0.000000}%
\pgfsetstrokecolor{currentstroke}%
\pgfsetdash{}{0pt}%
\pgfpathmoveto{\pgfqpoint{0.682223in}{1.295740in}}%
\pgfpathlineto{\pgfqpoint{0.820000in}{1.295740in}}%
\pgfusepath{stroke}%
\end{pgfscope}%
\begin{pgfscope}%
\pgfpathrectangle{\pgfqpoint{0.475556in}{1.127740in}}{\pgfqpoint{4.960000in}{3.696000in}}%
\pgfusepath{clip}%
\pgfsetrectcap%
\pgfsetroundjoin%
\pgfsetlinewidth{1.003750pt}%
\definecolor{currentstroke}{rgb}{0.000000,0.000000,0.000000}%
\pgfsetstrokecolor{currentstroke}%
\pgfsetdash{}{0pt}%
\pgfpathmoveto{\pgfqpoint{0.682223in}{4.655740in}}%
\pgfpathlineto{\pgfqpoint{0.820000in}{4.655740in}}%
\pgfusepath{stroke}%
\end{pgfscope}%
\begin{pgfscope}%
\pgfpathrectangle{\pgfqpoint{0.475556in}{1.127740in}}{\pgfqpoint{4.960000in}{3.696000in}}%
\pgfusepath{clip}%
\pgfsetrectcap%
\pgfsetroundjoin%
\pgfsetlinewidth{1.003750pt}%
\definecolor{currentstroke}{rgb}{0.000000,0.000000,0.000000}%
\pgfsetstrokecolor{currentstroke}%
\pgfsetdash{}{0pt}%
\pgfpathmoveto{\pgfqpoint{1.302222in}{2.135740in}}%
\pgfpathlineto{\pgfqpoint{1.302222in}{1.295740in}}%
\pgfusepath{stroke}%
\end{pgfscope}%
\begin{pgfscope}%
\pgfpathrectangle{\pgfqpoint{0.475556in}{1.127740in}}{\pgfqpoint{4.960000in}{3.696000in}}%
\pgfusepath{clip}%
\pgfsetrectcap%
\pgfsetroundjoin%
\pgfsetlinewidth{1.003750pt}%
\definecolor{currentstroke}{rgb}{0.000000,0.000000,0.000000}%
\pgfsetstrokecolor{currentstroke}%
\pgfsetdash{}{0pt}%
\pgfpathmoveto{\pgfqpoint{1.302222in}{3.815740in}}%
\pgfpathlineto{\pgfqpoint{1.302222in}{4.655740in}}%
\pgfusepath{stroke}%
\end{pgfscope}%
\begin{pgfscope}%
\pgfpathrectangle{\pgfqpoint{0.475556in}{1.127740in}}{\pgfqpoint{4.960000in}{3.696000in}}%
\pgfusepath{clip}%
\pgfsetrectcap%
\pgfsetroundjoin%
\pgfsetlinewidth{1.003750pt}%
\definecolor{currentstroke}{rgb}{0.000000,0.000000,0.000000}%
\pgfsetstrokecolor{currentstroke}%
\pgfsetdash{}{0pt}%
\pgfpathmoveto{\pgfqpoint{1.233334in}{1.295740in}}%
\pgfpathlineto{\pgfqpoint{1.371111in}{1.295740in}}%
\pgfusepath{stroke}%
\end{pgfscope}%
\begin{pgfscope}%
\pgfpathrectangle{\pgfqpoint{0.475556in}{1.127740in}}{\pgfqpoint{4.960000in}{3.696000in}}%
\pgfusepath{clip}%
\pgfsetrectcap%
\pgfsetroundjoin%
\pgfsetlinewidth{1.003750pt}%
\definecolor{currentstroke}{rgb}{0.000000,0.000000,0.000000}%
\pgfsetstrokecolor{currentstroke}%
\pgfsetdash{}{0pt}%
\pgfpathmoveto{\pgfqpoint{1.233334in}{4.655740in}}%
\pgfpathlineto{\pgfqpoint{1.371111in}{4.655740in}}%
\pgfusepath{stroke}%
\end{pgfscope}%
\begin{pgfscope}%
\pgfpathrectangle{\pgfqpoint{0.475556in}{1.127740in}}{\pgfqpoint{4.960000in}{3.696000in}}%
\pgfusepath{clip}%
\pgfsetrectcap%
\pgfsetroundjoin%
\pgfsetlinewidth{1.003750pt}%
\definecolor{currentstroke}{rgb}{0.000000,0.000000,0.000000}%
\pgfsetstrokecolor{currentstroke}%
\pgfsetdash{}{0pt}%
\pgfpathmoveto{\pgfqpoint{1.853334in}{2.975740in}}%
\pgfpathlineto{\pgfqpoint{1.853334in}{1.295740in}}%
\pgfusepath{stroke}%
\end{pgfscope}%
\begin{pgfscope}%
\pgfpathrectangle{\pgfqpoint{0.475556in}{1.127740in}}{\pgfqpoint{4.960000in}{3.696000in}}%
\pgfusepath{clip}%
\pgfsetrectcap%
\pgfsetroundjoin%
\pgfsetlinewidth{1.003750pt}%
\definecolor{currentstroke}{rgb}{0.000000,0.000000,0.000000}%
\pgfsetstrokecolor{currentstroke}%
\pgfsetdash{}{0pt}%
\pgfpathmoveto{\pgfqpoint{1.853334in}{4.235740in}}%
\pgfpathlineto{\pgfqpoint{1.853334in}{4.655740in}}%
\pgfusepath{stroke}%
\end{pgfscope}%
\begin{pgfscope}%
\pgfpathrectangle{\pgfqpoint{0.475556in}{1.127740in}}{\pgfqpoint{4.960000in}{3.696000in}}%
\pgfusepath{clip}%
\pgfsetrectcap%
\pgfsetroundjoin%
\pgfsetlinewidth{1.003750pt}%
\definecolor{currentstroke}{rgb}{0.000000,0.000000,0.000000}%
\pgfsetstrokecolor{currentstroke}%
\pgfsetdash{}{0pt}%
\pgfpathmoveto{\pgfqpoint{1.784445in}{1.295740in}}%
\pgfpathlineto{\pgfqpoint{1.922222in}{1.295740in}}%
\pgfusepath{stroke}%
\end{pgfscope}%
\begin{pgfscope}%
\pgfpathrectangle{\pgfqpoint{0.475556in}{1.127740in}}{\pgfqpoint{4.960000in}{3.696000in}}%
\pgfusepath{clip}%
\pgfsetrectcap%
\pgfsetroundjoin%
\pgfsetlinewidth{1.003750pt}%
\definecolor{currentstroke}{rgb}{0.000000,0.000000,0.000000}%
\pgfsetstrokecolor{currentstroke}%
\pgfsetdash{}{0pt}%
\pgfpathmoveto{\pgfqpoint{1.784445in}{4.655740in}}%
\pgfpathlineto{\pgfqpoint{1.922222in}{4.655740in}}%
\pgfusepath{stroke}%
\end{pgfscope}%
\begin{pgfscope}%
\pgfpathrectangle{\pgfqpoint{0.475556in}{1.127740in}}{\pgfqpoint{4.960000in}{3.696000in}}%
\pgfusepath{clip}%
\pgfsetrectcap%
\pgfsetroundjoin%
\pgfsetlinewidth{1.003750pt}%
\definecolor{currentstroke}{rgb}{0.000000,0.000000,0.000000}%
\pgfsetstrokecolor{currentstroke}%
\pgfsetdash{}{0pt}%
\pgfpathmoveto{\pgfqpoint{2.404445in}{1.715740in}}%
\pgfpathlineto{\pgfqpoint{2.404445in}{1.295740in}}%
\pgfusepath{stroke}%
\end{pgfscope}%
\begin{pgfscope}%
\pgfpathrectangle{\pgfqpoint{0.475556in}{1.127740in}}{\pgfqpoint{4.960000in}{3.696000in}}%
\pgfusepath{clip}%
\pgfsetrectcap%
\pgfsetroundjoin%
\pgfsetlinewidth{1.003750pt}%
\definecolor{currentstroke}{rgb}{0.000000,0.000000,0.000000}%
\pgfsetstrokecolor{currentstroke}%
\pgfsetdash{}{0pt}%
\pgfpathmoveto{\pgfqpoint{2.404445in}{2.975740in}}%
\pgfpathlineto{\pgfqpoint{2.404445in}{4.655740in}}%
\pgfusepath{stroke}%
\end{pgfscope}%
\begin{pgfscope}%
\pgfpathrectangle{\pgfqpoint{0.475556in}{1.127740in}}{\pgfqpoint{4.960000in}{3.696000in}}%
\pgfusepath{clip}%
\pgfsetrectcap%
\pgfsetroundjoin%
\pgfsetlinewidth{1.003750pt}%
\definecolor{currentstroke}{rgb}{0.000000,0.000000,0.000000}%
\pgfsetstrokecolor{currentstroke}%
\pgfsetdash{}{0pt}%
\pgfpathmoveto{\pgfqpoint{2.335556in}{1.295740in}}%
\pgfpathlineto{\pgfqpoint{2.473334in}{1.295740in}}%
\pgfusepath{stroke}%
\end{pgfscope}%
\begin{pgfscope}%
\pgfpathrectangle{\pgfqpoint{0.475556in}{1.127740in}}{\pgfqpoint{4.960000in}{3.696000in}}%
\pgfusepath{clip}%
\pgfsetrectcap%
\pgfsetroundjoin%
\pgfsetlinewidth{1.003750pt}%
\definecolor{currentstroke}{rgb}{0.000000,0.000000,0.000000}%
\pgfsetstrokecolor{currentstroke}%
\pgfsetdash{}{0pt}%
\pgfpathmoveto{\pgfqpoint{2.335556in}{4.655740in}}%
\pgfpathlineto{\pgfqpoint{2.473334in}{4.655740in}}%
\pgfusepath{stroke}%
\end{pgfscope}%
\begin{pgfscope}%
\pgfpathrectangle{\pgfqpoint{0.475556in}{1.127740in}}{\pgfqpoint{4.960000in}{3.696000in}}%
\pgfusepath{clip}%
\pgfsetrectcap%
\pgfsetroundjoin%
\pgfsetlinewidth{1.003750pt}%
\definecolor{currentstroke}{rgb}{0.000000,0.000000,0.000000}%
\pgfsetstrokecolor{currentstroke}%
\pgfsetdash{}{0pt}%
\pgfpathmoveto{\pgfqpoint{2.955556in}{1.715740in}}%
\pgfpathlineto{\pgfqpoint{2.955556in}{1.295740in}}%
\pgfusepath{stroke}%
\end{pgfscope}%
\begin{pgfscope}%
\pgfpathrectangle{\pgfqpoint{0.475556in}{1.127740in}}{\pgfqpoint{4.960000in}{3.696000in}}%
\pgfusepath{clip}%
\pgfsetrectcap%
\pgfsetroundjoin%
\pgfsetlinewidth{1.003750pt}%
\definecolor{currentstroke}{rgb}{0.000000,0.000000,0.000000}%
\pgfsetstrokecolor{currentstroke}%
\pgfsetdash{}{0pt}%
\pgfpathmoveto{\pgfqpoint{2.955556in}{2.555740in}}%
\pgfpathlineto{\pgfqpoint{2.955556in}{3.815740in}}%
\pgfusepath{stroke}%
\end{pgfscope}%
\begin{pgfscope}%
\pgfpathrectangle{\pgfqpoint{0.475556in}{1.127740in}}{\pgfqpoint{4.960000in}{3.696000in}}%
\pgfusepath{clip}%
\pgfsetrectcap%
\pgfsetroundjoin%
\pgfsetlinewidth{1.003750pt}%
\definecolor{currentstroke}{rgb}{0.000000,0.000000,0.000000}%
\pgfsetstrokecolor{currentstroke}%
\pgfsetdash{}{0pt}%
\pgfpathmoveto{\pgfqpoint{2.886667in}{1.295740in}}%
\pgfpathlineto{\pgfqpoint{3.024445in}{1.295740in}}%
\pgfusepath{stroke}%
\end{pgfscope}%
\begin{pgfscope}%
\pgfpathrectangle{\pgfqpoint{0.475556in}{1.127740in}}{\pgfqpoint{4.960000in}{3.696000in}}%
\pgfusepath{clip}%
\pgfsetrectcap%
\pgfsetroundjoin%
\pgfsetlinewidth{1.003750pt}%
\definecolor{currentstroke}{rgb}{0.000000,0.000000,0.000000}%
\pgfsetstrokecolor{currentstroke}%
\pgfsetdash{}{0pt}%
\pgfpathmoveto{\pgfqpoint{2.886667in}{3.815740in}}%
\pgfpathlineto{\pgfqpoint{3.024445in}{3.815740in}}%
\pgfusepath{stroke}%
\end{pgfscope}%
\begin{pgfscope}%
\pgfpathrectangle{\pgfqpoint{0.475556in}{1.127740in}}{\pgfqpoint{4.960000in}{3.696000in}}%
\pgfusepath{clip}%
\pgfsetbuttcap%
\pgfsetroundjoin%
\definecolor{currentfill}{rgb}{0.000000,0.000000,0.000000}%
\pgfsetfillcolor{currentfill}%
\pgfsetfillopacity{0.000000}%
\pgfsetlinewidth{1.003750pt}%
\definecolor{currentstroke}{rgb}{0.000000,0.000000,0.000000}%
\pgfsetstrokecolor{currentstroke}%
\pgfsetdash{}{0pt}%
\pgfsys@defobject{currentmarker}{\pgfqpoint{-0.041667in}{-0.041667in}}{\pgfqpoint{0.041667in}{0.041667in}}{%
\pgfpathmoveto{\pgfqpoint{0.000000in}{-0.041667in}}%
\pgfpathcurveto{\pgfqpoint{0.011050in}{-0.041667in}}{\pgfqpoint{0.021649in}{-0.037276in}}{\pgfqpoint{0.029463in}{-0.029463in}}%
\pgfpathcurveto{\pgfqpoint{0.037276in}{-0.021649in}}{\pgfqpoint{0.041667in}{-0.011050in}}{\pgfqpoint{0.041667in}{0.000000in}}%
\pgfpathcurveto{\pgfqpoint{0.041667in}{0.011050in}}{\pgfqpoint{0.037276in}{0.021649in}}{\pgfqpoint{0.029463in}{0.029463in}}%
\pgfpathcurveto{\pgfqpoint{0.021649in}{0.037276in}}{\pgfqpoint{0.011050in}{0.041667in}}{\pgfqpoint{0.000000in}{0.041667in}}%
\pgfpathcurveto{\pgfqpoint{-0.011050in}{0.041667in}}{\pgfqpoint{-0.021649in}{0.037276in}}{\pgfqpoint{-0.029463in}{0.029463in}}%
\pgfpathcurveto{\pgfqpoint{-0.037276in}{0.021649in}}{\pgfqpoint{-0.041667in}{0.011050in}}{\pgfqpoint{-0.041667in}{0.000000in}}%
\pgfpathcurveto{\pgfqpoint{-0.041667in}{-0.011050in}}{\pgfqpoint{-0.037276in}{-0.021649in}}{\pgfqpoint{-0.029463in}{-0.029463in}}%
\pgfpathcurveto{\pgfqpoint{-0.021649in}{-0.037276in}}{\pgfqpoint{-0.011050in}{-0.041667in}}{\pgfqpoint{0.000000in}{-0.041667in}}%
\pgfpathclose%
\pgfusepath{stroke,fill}%
}%
\begin{pgfscope}%
\pgfsys@transformshift{2.955556in}{4.235740in}%
\pgfsys@useobject{currentmarker}{}%
\end{pgfscope}%
\begin{pgfscope}%
\pgfsys@transformshift{2.955556in}{4.235740in}%
\pgfsys@useobject{currentmarker}{}%
\end{pgfscope}%
\begin{pgfscope}%
\pgfsys@transformshift{2.955556in}{4.655740in}%
\pgfsys@useobject{currentmarker}{}%
\end{pgfscope}%
\begin{pgfscope}%
\pgfsys@transformshift{2.955556in}{4.655740in}%
\pgfsys@useobject{currentmarker}{}%
\end{pgfscope}%
\begin{pgfscope}%
\pgfsys@transformshift{2.955556in}{4.655740in}%
\pgfsys@useobject{currentmarker}{}%
\end{pgfscope}%
\begin{pgfscope}%
\pgfsys@transformshift{2.955556in}{4.655740in}%
\pgfsys@useobject{currentmarker}{}%
\end{pgfscope}%
\begin{pgfscope}%
\pgfsys@transformshift{2.955556in}{4.655740in}%
\pgfsys@useobject{currentmarker}{}%
\end{pgfscope}%
\begin{pgfscope}%
\pgfsys@transformshift{2.955556in}{4.655740in}%
\pgfsys@useobject{currentmarker}{}%
\end{pgfscope}%
\begin{pgfscope}%
\pgfsys@transformshift{2.955556in}{4.655740in}%
\pgfsys@useobject{currentmarker}{}%
\end{pgfscope}%
\begin{pgfscope}%
\pgfsys@transformshift{2.955556in}{4.655740in}%
\pgfsys@useobject{currentmarker}{}%
\end{pgfscope}%
\end{pgfscope}%
\begin{pgfscope}%
\pgfpathrectangle{\pgfqpoint{0.475556in}{1.127740in}}{\pgfqpoint{4.960000in}{3.696000in}}%
\pgfusepath{clip}%
\pgfsetrectcap%
\pgfsetroundjoin%
\pgfsetlinewidth{1.003750pt}%
\definecolor{currentstroke}{rgb}{0.000000,0.000000,0.000000}%
\pgfsetstrokecolor{currentstroke}%
\pgfsetdash{}{0pt}%
\pgfpathmoveto{\pgfqpoint{3.506667in}{2.975740in}}%
\pgfpathlineto{\pgfqpoint{3.506667in}{1.295740in}}%
\pgfusepath{stroke}%
\end{pgfscope}%
\begin{pgfscope}%
\pgfpathrectangle{\pgfqpoint{0.475556in}{1.127740in}}{\pgfqpoint{4.960000in}{3.696000in}}%
\pgfusepath{clip}%
\pgfsetrectcap%
\pgfsetroundjoin%
\pgfsetlinewidth{1.003750pt}%
\definecolor{currentstroke}{rgb}{0.000000,0.000000,0.000000}%
\pgfsetstrokecolor{currentstroke}%
\pgfsetdash{}{0pt}%
\pgfpathmoveto{\pgfqpoint{3.506667in}{4.235740in}}%
\pgfpathlineto{\pgfqpoint{3.506667in}{4.655740in}}%
\pgfusepath{stroke}%
\end{pgfscope}%
\begin{pgfscope}%
\pgfpathrectangle{\pgfqpoint{0.475556in}{1.127740in}}{\pgfqpoint{4.960000in}{3.696000in}}%
\pgfusepath{clip}%
\pgfsetrectcap%
\pgfsetroundjoin%
\pgfsetlinewidth{1.003750pt}%
\definecolor{currentstroke}{rgb}{0.000000,0.000000,0.000000}%
\pgfsetstrokecolor{currentstroke}%
\pgfsetdash{}{0pt}%
\pgfpathmoveto{\pgfqpoint{3.437778in}{1.295740in}}%
\pgfpathlineto{\pgfqpoint{3.575556in}{1.295740in}}%
\pgfusepath{stroke}%
\end{pgfscope}%
\begin{pgfscope}%
\pgfpathrectangle{\pgfqpoint{0.475556in}{1.127740in}}{\pgfqpoint{4.960000in}{3.696000in}}%
\pgfusepath{clip}%
\pgfsetrectcap%
\pgfsetroundjoin%
\pgfsetlinewidth{1.003750pt}%
\definecolor{currentstroke}{rgb}{0.000000,0.000000,0.000000}%
\pgfsetstrokecolor{currentstroke}%
\pgfsetdash{}{0pt}%
\pgfpathmoveto{\pgfqpoint{3.437778in}{4.655740in}}%
\pgfpathlineto{\pgfqpoint{3.575556in}{4.655740in}}%
\pgfusepath{stroke}%
\end{pgfscope}%
\begin{pgfscope}%
\pgfpathrectangle{\pgfqpoint{0.475556in}{1.127740in}}{\pgfqpoint{4.960000in}{3.696000in}}%
\pgfusepath{clip}%
\pgfsetrectcap%
\pgfsetroundjoin%
\pgfsetlinewidth{1.003750pt}%
\definecolor{currentstroke}{rgb}{0.000000,0.000000,0.000000}%
\pgfsetstrokecolor{currentstroke}%
\pgfsetdash{}{0pt}%
\pgfpathmoveto{\pgfqpoint{4.057778in}{1.295740in}}%
\pgfpathlineto{\pgfqpoint{4.057778in}{1.295740in}}%
\pgfusepath{stroke}%
\end{pgfscope}%
\begin{pgfscope}%
\pgfpathrectangle{\pgfqpoint{0.475556in}{1.127740in}}{\pgfqpoint{4.960000in}{3.696000in}}%
\pgfusepath{clip}%
\pgfsetrectcap%
\pgfsetroundjoin%
\pgfsetlinewidth{1.003750pt}%
\definecolor{currentstroke}{rgb}{0.000000,0.000000,0.000000}%
\pgfsetstrokecolor{currentstroke}%
\pgfsetdash{}{0pt}%
\pgfpathmoveto{\pgfqpoint{4.057778in}{2.135740in}}%
\pgfpathlineto{\pgfqpoint{4.057778in}{3.395740in}}%
\pgfusepath{stroke}%
\end{pgfscope}%
\begin{pgfscope}%
\pgfpathrectangle{\pgfqpoint{0.475556in}{1.127740in}}{\pgfqpoint{4.960000in}{3.696000in}}%
\pgfusepath{clip}%
\pgfsetrectcap%
\pgfsetroundjoin%
\pgfsetlinewidth{1.003750pt}%
\definecolor{currentstroke}{rgb}{0.000000,0.000000,0.000000}%
\pgfsetstrokecolor{currentstroke}%
\pgfsetdash{}{0pt}%
\pgfpathmoveto{\pgfqpoint{3.988889in}{1.295740in}}%
\pgfpathlineto{\pgfqpoint{4.126667in}{1.295740in}}%
\pgfusepath{stroke}%
\end{pgfscope}%
\begin{pgfscope}%
\pgfpathrectangle{\pgfqpoint{0.475556in}{1.127740in}}{\pgfqpoint{4.960000in}{3.696000in}}%
\pgfusepath{clip}%
\pgfsetrectcap%
\pgfsetroundjoin%
\pgfsetlinewidth{1.003750pt}%
\definecolor{currentstroke}{rgb}{0.000000,0.000000,0.000000}%
\pgfsetstrokecolor{currentstroke}%
\pgfsetdash{}{0pt}%
\pgfpathmoveto{\pgfqpoint{3.988889in}{3.395740in}}%
\pgfpathlineto{\pgfqpoint{4.126667in}{3.395740in}}%
\pgfusepath{stroke}%
\end{pgfscope}%
\begin{pgfscope}%
\pgfpathrectangle{\pgfqpoint{0.475556in}{1.127740in}}{\pgfqpoint{4.960000in}{3.696000in}}%
\pgfusepath{clip}%
\pgfsetbuttcap%
\pgfsetroundjoin%
\definecolor{currentfill}{rgb}{0.000000,0.000000,0.000000}%
\pgfsetfillcolor{currentfill}%
\pgfsetfillopacity{0.000000}%
\pgfsetlinewidth{1.003750pt}%
\definecolor{currentstroke}{rgb}{0.000000,0.000000,0.000000}%
\pgfsetstrokecolor{currentstroke}%
\pgfsetdash{}{0pt}%
\pgfsys@defobject{currentmarker}{\pgfqpoint{-0.041667in}{-0.041667in}}{\pgfqpoint{0.041667in}{0.041667in}}{%
\pgfpathmoveto{\pgfqpoint{0.000000in}{-0.041667in}}%
\pgfpathcurveto{\pgfqpoint{0.011050in}{-0.041667in}}{\pgfqpoint{0.021649in}{-0.037276in}}{\pgfqpoint{0.029463in}{-0.029463in}}%
\pgfpathcurveto{\pgfqpoint{0.037276in}{-0.021649in}}{\pgfqpoint{0.041667in}{-0.011050in}}{\pgfqpoint{0.041667in}{0.000000in}}%
\pgfpathcurveto{\pgfqpoint{0.041667in}{0.011050in}}{\pgfqpoint{0.037276in}{0.021649in}}{\pgfqpoint{0.029463in}{0.029463in}}%
\pgfpathcurveto{\pgfqpoint{0.021649in}{0.037276in}}{\pgfqpoint{0.011050in}{0.041667in}}{\pgfqpoint{0.000000in}{0.041667in}}%
\pgfpathcurveto{\pgfqpoint{-0.011050in}{0.041667in}}{\pgfqpoint{-0.021649in}{0.037276in}}{\pgfqpoint{-0.029463in}{0.029463in}}%
\pgfpathcurveto{\pgfqpoint{-0.037276in}{0.021649in}}{\pgfqpoint{-0.041667in}{0.011050in}}{\pgfqpoint{-0.041667in}{0.000000in}}%
\pgfpathcurveto{\pgfqpoint{-0.041667in}{-0.011050in}}{\pgfqpoint{-0.037276in}{-0.021649in}}{\pgfqpoint{-0.029463in}{-0.029463in}}%
\pgfpathcurveto{\pgfqpoint{-0.021649in}{-0.037276in}}{\pgfqpoint{-0.011050in}{-0.041667in}}{\pgfqpoint{0.000000in}{-0.041667in}}%
\pgfpathclose%
\pgfusepath{stroke,fill}%
}%
\begin{pgfscope}%
\pgfsys@transformshift{4.057778in}{3.815740in}%
\pgfsys@useobject{currentmarker}{}%
\end{pgfscope}%
\begin{pgfscope}%
\pgfsys@transformshift{4.057778in}{3.815740in}%
\pgfsys@useobject{currentmarker}{}%
\end{pgfscope}%
\begin{pgfscope}%
\pgfsys@transformshift{4.057778in}{3.815740in}%
\pgfsys@useobject{currentmarker}{}%
\end{pgfscope}%
\begin{pgfscope}%
\pgfsys@transformshift{4.057778in}{4.235740in}%
\pgfsys@useobject{currentmarker}{}%
\end{pgfscope}%
\begin{pgfscope}%
\pgfsys@transformshift{4.057778in}{4.235740in}%
\pgfsys@useobject{currentmarker}{}%
\end{pgfscope}%
\begin{pgfscope}%
\pgfsys@transformshift{4.057778in}{4.235740in}%
\pgfsys@useobject{currentmarker}{}%
\end{pgfscope}%
\begin{pgfscope}%
\pgfsys@transformshift{4.057778in}{4.235740in}%
\pgfsys@useobject{currentmarker}{}%
\end{pgfscope}%
\begin{pgfscope}%
\pgfsys@transformshift{4.057778in}{4.655740in}%
\pgfsys@useobject{currentmarker}{}%
\end{pgfscope}%
\begin{pgfscope}%
\pgfsys@transformshift{4.057778in}{4.655740in}%
\pgfsys@useobject{currentmarker}{}%
\end{pgfscope}%
\end{pgfscope}%
\begin{pgfscope}%
\pgfpathrectangle{\pgfqpoint{0.475556in}{1.127740in}}{\pgfqpoint{4.960000in}{3.696000in}}%
\pgfusepath{clip}%
\pgfsetrectcap%
\pgfsetroundjoin%
\pgfsetlinewidth{1.003750pt}%
\definecolor{currentstroke}{rgb}{0.000000,0.000000,0.000000}%
\pgfsetstrokecolor{currentstroke}%
\pgfsetdash{}{0pt}%
\pgfpathmoveto{\pgfqpoint{4.608889in}{2.555740in}}%
\pgfpathlineto{\pgfqpoint{4.608889in}{1.295740in}}%
\pgfusepath{stroke}%
\end{pgfscope}%
\begin{pgfscope}%
\pgfpathrectangle{\pgfqpoint{0.475556in}{1.127740in}}{\pgfqpoint{4.960000in}{3.696000in}}%
\pgfusepath{clip}%
\pgfsetrectcap%
\pgfsetroundjoin%
\pgfsetlinewidth{1.003750pt}%
\definecolor{currentstroke}{rgb}{0.000000,0.000000,0.000000}%
\pgfsetstrokecolor{currentstroke}%
\pgfsetdash{}{0pt}%
\pgfpathmoveto{\pgfqpoint{4.608889in}{4.235740in}}%
\pgfpathlineto{\pgfqpoint{4.608889in}{4.655740in}}%
\pgfusepath{stroke}%
\end{pgfscope}%
\begin{pgfscope}%
\pgfpathrectangle{\pgfqpoint{0.475556in}{1.127740in}}{\pgfqpoint{4.960000in}{3.696000in}}%
\pgfusepath{clip}%
\pgfsetrectcap%
\pgfsetroundjoin%
\pgfsetlinewidth{1.003750pt}%
\definecolor{currentstroke}{rgb}{0.000000,0.000000,0.000000}%
\pgfsetstrokecolor{currentstroke}%
\pgfsetdash{}{0pt}%
\pgfpathmoveto{\pgfqpoint{4.540000in}{1.295740in}}%
\pgfpathlineto{\pgfqpoint{4.677778in}{1.295740in}}%
\pgfusepath{stroke}%
\end{pgfscope}%
\begin{pgfscope}%
\pgfpathrectangle{\pgfqpoint{0.475556in}{1.127740in}}{\pgfqpoint{4.960000in}{3.696000in}}%
\pgfusepath{clip}%
\pgfsetrectcap%
\pgfsetroundjoin%
\pgfsetlinewidth{1.003750pt}%
\definecolor{currentstroke}{rgb}{0.000000,0.000000,0.000000}%
\pgfsetstrokecolor{currentstroke}%
\pgfsetdash{}{0pt}%
\pgfpathmoveto{\pgfqpoint{4.540000in}{4.655740in}}%
\pgfpathlineto{\pgfqpoint{4.677778in}{4.655740in}}%
\pgfusepath{stroke}%
\end{pgfscope}%
\begin{pgfscope}%
\pgfpathrectangle{\pgfqpoint{0.475556in}{1.127740in}}{\pgfqpoint{4.960000in}{3.696000in}}%
\pgfusepath{clip}%
\pgfsetrectcap%
\pgfsetroundjoin%
\pgfsetlinewidth{1.003750pt}%
\definecolor{currentstroke}{rgb}{0.000000,0.000000,0.000000}%
\pgfsetstrokecolor{currentstroke}%
\pgfsetdash{}{0pt}%
\pgfpathmoveto{\pgfqpoint{5.160000in}{2.555740in}}%
\pgfpathlineto{\pgfqpoint{5.160000in}{1.295740in}}%
\pgfusepath{stroke}%
\end{pgfscope}%
\begin{pgfscope}%
\pgfpathrectangle{\pgfqpoint{0.475556in}{1.127740in}}{\pgfqpoint{4.960000in}{3.696000in}}%
\pgfusepath{clip}%
\pgfsetrectcap%
\pgfsetroundjoin%
\pgfsetlinewidth{1.003750pt}%
\definecolor{currentstroke}{rgb}{0.000000,0.000000,0.000000}%
\pgfsetstrokecolor{currentstroke}%
\pgfsetdash{}{0pt}%
\pgfpathmoveto{\pgfqpoint{5.160000in}{4.235740in}}%
\pgfpathlineto{\pgfqpoint{5.160000in}{4.655740in}}%
\pgfusepath{stroke}%
\end{pgfscope}%
\begin{pgfscope}%
\pgfpathrectangle{\pgfqpoint{0.475556in}{1.127740in}}{\pgfqpoint{4.960000in}{3.696000in}}%
\pgfusepath{clip}%
\pgfsetrectcap%
\pgfsetroundjoin%
\pgfsetlinewidth{1.003750pt}%
\definecolor{currentstroke}{rgb}{0.000000,0.000000,0.000000}%
\pgfsetstrokecolor{currentstroke}%
\pgfsetdash{}{0pt}%
\pgfpathmoveto{\pgfqpoint{5.091111in}{1.295740in}}%
\pgfpathlineto{\pgfqpoint{5.228889in}{1.295740in}}%
\pgfusepath{stroke}%
\end{pgfscope}%
\begin{pgfscope}%
\pgfpathrectangle{\pgfqpoint{0.475556in}{1.127740in}}{\pgfqpoint{4.960000in}{3.696000in}}%
\pgfusepath{clip}%
\pgfsetrectcap%
\pgfsetroundjoin%
\pgfsetlinewidth{1.003750pt}%
\definecolor{currentstroke}{rgb}{0.000000,0.000000,0.000000}%
\pgfsetstrokecolor{currentstroke}%
\pgfsetdash{}{0pt}%
\pgfpathmoveto{\pgfqpoint{5.091111in}{4.655740in}}%
\pgfpathlineto{\pgfqpoint{5.228889in}{4.655740in}}%
\pgfusepath{stroke}%
\end{pgfscope}%
\begin{pgfscope}%
\pgfpathrectangle{\pgfqpoint{0.475556in}{1.127740in}}{\pgfqpoint{4.960000in}{3.696000in}}%
\pgfusepath{clip}%
\pgfsetbuttcap%
\pgfsetmiterjoin%
\definecolor{currentfill}{rgb}{0.831065,0.238447,0.308804}%
\pgfsetfillcolor{currentfill}%
\pgfsetlinewidth{1.003750pt}%
\definecolor{currentstroke}{rgb}{0.000000,0.000000,0.000000}%
\pgfsetstrokecolor{currentstroke}%
\pgfsetdash{}{0pt}%
\pgfpathmoveto{\pgfqpoint{0.613334in}{2.555740in}}%
\pgfpathlineto{\pgfqpoint{0.888889in}{2.555740in}}%
\pgfpathlineto{\pgfqpoint{0.888889in}{4.235740in}}%
\pgfpathlineto{\pgfqpoint{0.613334in}{4.235740in}}%
\pgfpathlineto{\pgfqpoint{0.613334in}{2.555740in}}%
\pgfpathclose%
\pgfusepath{stroke,fill}%
\end{pgfscope}%
\begin{pgfscope}%
\pgfpathrectangle{\pgfqpoint{0.475556in}{1.127740in}}{\pgfqpoint{4.960000in}{3.696000in}}%
\pgfusepath{clip}%
\pgfsetbuttcap%
\pgfsetmiterjoin%
\definecolor{currentfill}{rgb}{0.956863,0.427451,0.262745}%
\pgfsetfillcolor{currentfill}%
\pgfsetlinewidth{1.003750pt}%
\definecolor{currentstroke}{rgb}{0.000000,0.000000,0.000000}%
\pgfsetstrokecolor{currentstroke}%
\pgfsetdash{}{0pt}%
\pgfpathmoveto{\pgfqpoint{1.164445in}{2.135740in}}%
\pgfpathlineto{\pgfqpoint{1.440000in}{2.135740in}}%
\pgfpathlineto{\pgfqpoint{1.440000in}{3.815740in}}%
\pgfpathlineto{\pgfqpoint{1.164445in}{3.815740in}}%
\pgfpathlineto{\pgfqpoint{1.164445in}{2.135740in}}%
\pgfpathclose%
\pgfusepath{stroke,fill}%
\end{pgfscope}%
\begin{pgfscope}%
\pgfpathrectangle{\pgfqpoint{0.475556in}{1.127740in}}{\pgfqpoint{4.960000in}{3.696000in}}%
\pgfusepath{clip}%
\pgfsetbuttcap%
\pgfsetmiterjoin%
\definecolor{currentfill}{rgb}{0.991465,0.677355,0.378085}%
\pgfsetfillcolor{currentfill}%
\pgfsetlinewidth{1.003750pt}%
\definecolor{currentstroke}{rgb}{0.000000,0.000000,0.000000}%
\pgfsetstrokecolor{currentstroke}%
\pgfsetdash{}{0pt}%
\pgfpathmoveto{\pgfqpoint{1.715556in}{2.975740in}}%
\pgfpathlineto{\pgfqpoint{1.991111in}{2.975740in}}%
\pgfpathlineto{\pgfqpoint{1.991111in}{4.235740in}}%
\pgfpathlineto{\pgfqpoint{1.715556in}{4.235740in}}%
\pgfpathlineto{\pgfqpoint{1.715556in}{2.975740in}}%
\pgfpathclose%
\pgfusepath{stroke,fill}%
\end{pgfscope}%
\begin{pgfscope}%
\pgfpathrectangle{\pgfqpoint{0.475556in}{1.127740in}}{\pgfqpoint{4.960000in}{3.696000in}}%
\pgfusepath{clip}%
\pgfsetbuttcap%
\pgfsetmiterjoin%
\definecolor{currentfill}{rgb}{0.996078,0.878431,0.545098}%
\pgfsetfillcolor{currentfill}%
\pgfsetlinewidth{1.003750pt}%
\definecolor{currentstroke}{rgb}{0.000000,0.000000,0.000000}%
\pgfsetstrokecolor{currentstroke}%
\pgfsetdash{}{0pt}%
\pgfpathmoveto{\pgfqpoint{2.266667in}{1.715740in}}%
\pgfpathlineto{\pgfqpoint{2.542222in}{1.715740in}}%
\pgfpathlineto{\pgfqpoint{2.542222in}{2.975740in}}%
\pgfpathlineto{\pgfqpoint{2.266667in}{2.975740in}}%
\pgfpathlineto{\pgfqpoint{2.266667in}{1.715740in}}%
\pgfpathclose%
\pgfusepath{stroke,fill}%
\end{pgfscope}%
\begin{pgfscope}%
\pgfpathrectangle{\pgfqpoint{0.475556in}{1.127740in}}{\pgfqpoint{4.960000in}{3.696000in}}%
\pgfusepath{clip}%
\pgfsetbuttcap%
\pgfsetmiterjoin%
\definecolor{currentfill}{rgb}{0.998078,0.999231,0.746021}%
\pgfsetfillcolor{currentfill}%
\pgfsetlinewidth{1.003750pt}%
\definecolor{currentstroke}{rgb}{0.000000,0.000000,0.000000}%
\pgfsetstrokecolor{currentstroke}%
\pgfsetdash{}{0pt}%
\pgfpathmoveto{\pgfqpoint{2.817778in}{1.715740in}}%
\pgfpathlineto{\pgfqpoint{3.093334in}{1.715740in}}%
\pgfpathlineto{\pgfqpoint{3.093334in}{2.555740in}}%
\pgfpathlineto{\pgfqpoint{2.817778in}{2.555740in}}%
\pgfpathlineto{\pgfqpoint{2.817778in}{1.715740in}}%
\pgfpathclose%
\pgfusepath{stroke,fill}%
\end{pgfscope}%
\begin{pgfscope}%
\pgfpathrectangle{\pgfqpoint{0.475556in}{1.127740in}}{\pgfqpoint{4.960000in}{3.696000in}}%
\pgfusepath{clip}%
\pgfsetbuttcap%
\pgfsetmiterjoin%
\definecolor{currentfill}{rgb}{0.901961,0.960784,0.596078}%
\pgfsetfillcolor{currentfill}%
\pgfsetlinewidth{1.003750pt}%
\definecolor{currentstroke}{rgb}{0.000000,0.000000,0.000000}%
\pgfsetstrokecolor{currentstroke}%
\pgfsetdash{}{0pt}%
\pgfpathmoveto{\pgfqpoint{3.368889in}{2.975740in}}%
\pgfpathlineto{\pgfqpoint{3.644445in}{2.975740in}}%
\pgfpathlineto{\pgfqpoint{3.644445in}{4.235740in}}%
\pgfpathlineto{\pgfqpoint{3.368889in}{4.235740in}}%
\pgfpathlineto{\pgfqpoint{3.368889in}{2.975740in}}%
\pgfpathclose%
\pgfusepath{stroke,fill}%
\end{pgfscope}%
\begin{pgfscope}%
\pgfpathrectangle{\pgfqpoint{0.475556in}{1.127740in}}{\pgfqpoint{4.960000in}{3.696000in}}%
\pgfusepath{clip}%
\pgfsetbuttcap%
\pgfsetmiterjoin%
\definecolor{currentfill}{rgb}{0.665283,0.864591,0.643214}%
\pgfsetfillcolor{currentfill}%
\pgfsetlinewidth{1.003750pt}%
\definecolor{currentstroke}{rgb}{0.000000,0.000000,0.000000}%
\pgfsetstrokecolor{currentstroke}%
\pgfsetdash{}{0pt}%
\pgfpathmoveto{\pgfqpoint{3.920000in}{1.295740in}}%
\pgfpathlineto{\pgfqpoint{4.195556in}{1.295740in}}%
\pgfpathlineto{\pgfqpoint{4.195556in}{2.135740in}}%
\pgfpathlineto{\pgfqpoint{3.920000in}{2.135740in}}%
\pgfpathlineto{\pgfqpoint{3.920000in}{1.295740in}}%
\pgfpathclose%
\pgfusepath{stroke,fill}%
\end{pgfscope}%
\begin{pgfscope}%
\pgfpathrectangle{\pgfqpoint{0.475556in}{1.127740in}}{\pgfqpoint{4.960000in}{3.696000in}}%
\pgfusepath{clip}%
\pgfsetbuttcap%
\pgfsetmiterjoin%
\definecolor{currentfill}{rgb}{0.400000,0.760784,0.647059}%
\pgfsetfillcolor{currentfill}%
\pgfsetlinewidth{1.003750pt}%
\definecolor{currentstroke}{rgb}{0.000000,0.000000,0.000000}%
\pgfsetstrokecolor{currentstroke}%
\pgfsetdash{}{0pt}%
\pgfpathmoveto{\pgfqpoint{4.471111in}{2.555740in}}%
\pgfpathlineto{\pgfqpoint{4.746667in}{2.555740in}}%
\pgfpathlineto{\pgfqpoint{4.746667in}{4.235740in}}%
\pgfpathlineto{\pgfqpoint{4.471111in}{4.235740in}}%
\pgfpathlineto{\pgfqpoint{4.471111in}{2.555740in}}%
\pgfpathclose%
\pgfusepath{stroke,fill}%
\end{pgfscope}%
\begin{pgfscope}%
\pgfpathrectangle{\pgfqpoint{0.475556in}{1.127740in}}{\pgfqpoint{4.960000in}{3.696000in}}%
\pgfusepath{clip}%
\pgfsetbuttcap%
\pgfsetmiterjoin%
\definecolor{currentfill}{rgb}{0.199462,0.528950,0.739100}%
\pgfsetfillcolor{currentfill}%
\pgfsetlinewidth{1.003750pt}%
\definecolor{currentstroke}{rgb}{0.000000,0.000000,0.000000}%
\pgfsetstrokecolor{currentstroke}%
\pgfsetdash{}{0pt}%
\pgfpathmoveto{\pgfqpoint{5.022222in}{2.555740in}}%
\pgfpathlineto{\pgfqpoint{5.297778in}{2.555740in}}%
\pgfpathlineto{\pgfqpoint{5.297778in}{4.235740in}}%
\pgfpathlineto{\pgfqpoint{5.022222in}{4.235740in}}%
\pgfpathlineto{\pgfqpoint{5.022222in}{2.555740in}}%
\pgfpathclose%
\pgfusepath{stroke,fill}%
\end{pgfscope}%
\begin{pgfscope}%
\pgfpathrectangle{\pgfqpoint{0.475556in}{1.127740in}}{\pgfqpoint{4.960000in}{3.696000in}}%
\pgfusepath{clip}%
\pgfsetrectcap%
\pgfsetroundjoin%
\pgfsetlinewidth{1.505625pt}%
\definecolor{currentstroke}{rgb}{0.054902,0.560784,0.027451}%
\pgfsetstrokecolor{currentstroke}%
\pgfsetdash{}{0pt}%
\pgfpathmoveto{\pgfqpoint{0.613334in}{2.975740in}}%
\pgfpathlineto{\pgfqpoint{0.888889in}{2.975740in}}%
\pgfusepath{stroke}%
\end{pgfscope}%
\begin{pgfscope}%
\pgfpathrectangle{\pgfqpoint{0.475556in}{1.127740in}}{\pgfqpoint{4.960000in}{3.696000in}}%
\pgfusepath{clip}%
\pgfsetbuttcap%
\pgfsetroundjoin%
\pgfsetlinewidth{1.505625pt}%
\definecolor{currentstroke}{rgb}{0.019608,0.078431,0.470588}%
\pgfsetstrokecolor{currentstroke}%
\pgfsetdash{{5.550000pt}{2.400000pt}}{0.000000pt}%
\pgfpathmoveto{\pgfqpoint{0.613334in}{3.194613in}}%
\pgfpathlineto{\pgfqpoint{0.888889in}{3.194613in}}%
\pgfusepath{stroke}%
\end{pgfscope}%
\begin{pgfscope}%
\pgfpathrectangle{\pgfqpoint{0.475556in}{1.127740in}}{\pgfqpoint{4.960000in}{3.696000in}}%
\pgfusepath{clip}%
\pgfsetrectcap%
\pgfsetroundjoin%
\pgfsetlinewidth{1.505625pt}%
\definecolor{currentstroke}{rgb}{0.054902,0.560784,0.027451}%
\pgfsetstrokecolor{currentstroke}%
\pgfsetdash{}{0pt}%
\pgfpathmoveto{\pgfqpoint{1.164445in}{2.975740in}}%
\pgfpathlineto{\pgfqpoint{1.440000in}{2.975740in}}%
\pgfusepath{stroke}%
\end{pgfscope}%
\begin{pgfscope}%
\pgfpathrectangle{\pgfqpoint{0.475556in}{1.127740in}}{\pgfqpoint{4.960000in}{3.696000in}}%
\pgfusepath{clip}%
\pgfsetbuttcap%
\pgfsetroundjoin%
\pgfsetlinewidth{1.505625pt}%
\definecolor{currentstroke}{rgb}{0.019608,0.078431,0.470588}%
\pgfsetstrokecolor{currentstroke}%
\pgfsetdash{{5.550000pt}{2.400000pt}}{0.000000pt}%
\pgfpathmoveto{\pgfqpoint{1.164445in}{2.983627in}}%
\pgfpathlineto{\pgfqpoint{1.440000in}{2.983627in}}%
\pgfusepath{stroke}%
\end{pgfscope}%
\begin{pgfscope}%
\pgfpathrectangle{\pgfqpoint{0.475556in}{1.127740in}}{\pgfqpoint{4.960000in}{3.696000in}}%
\pgfusepath{clip}%
\pgfsetrectcap%
\pgfsetroundjoin%
\pgfsetlinewidth{1.505625pt}%
\definecolor{currentstroke}{rgb}{0.054902,0.560784,0.027451}%
\pgfsetstrokecolor{currentstroke}%
\pgfsetdash{}{0pt}%
\pgfpathmoveto{\pgfqpoint{1.715556in}{3.815740in}}%
\pgfpathlineto{\pgfqpoint{1.991111in}{3.815740in}}%
\pgfusepath{stroke}%
\end{pgfscope}%
\begin{pgfscope}%
\pgfpathrectangle{\pgfqpoint{0.475556in}{1.127740in}}{\pgfqpoint{4.960000in}{3.696000in}}%
\pgfusepath{clip}%
\pgfsetbuttcap%
\pgfsetroundjoin%
\pgfsetlinewidth{1.505625pt}%
\definecolor{currentstroke}{rgb}{0.019608,0.078431,0.470588}%
\pgfsetstrokecolor{currentstroke}%
\pgfsetdash{{5.550000pt}{2.400000pt}}{0.000000pt}%
\pgfpathmoveto{\pgfqpoint{1.715556in}{3.596867in}}%
\pgfpathlineto{\pgfqpoint{1.991111in}{3.596867in}}%
\pgfusepath{stroke}%
\end{pgfscope}%
\begin{pgfscope}%
\pgfpathrectangle{\pgfqpoint{0.475556in}{1.127740in}}{\pgfqpoint{4.960000in}{3.696000in}}%
\pgfusepath{clip}%
\pgfsetrectcap%
\pgfsetroundjoin%
\pgfsetlinewidth{1.505625pt}%
\definecolor{currentstroke}{rgb}{0.054902,0.560784,0.027451}%
\pgfsetstrokecolor{currentstroke}%
\pgfsetdash{}{0pt}%
\pgfpathmoveto{\pgfqpoint{2.266667in}{2.555740in}}%
\pgfpathlineto{\pgfqpoint{2.542222in}{2.555740in}}%
\pgfusepath{stroke}%
\end{pgfscope}%
\begin{pgfscope}%
\pgfpathrectangle{\pgfqpoint{0.475556in}{1.127740in}}{\pgfqpoint{4.960000in}{3.696000in}}%
\pgfusepath{clip}%
\pgfsetbuttcap%
\pgfsetroundjoin%
\pgfsetlinewidth{1.505625pt}%
\definecolor{currentstroke}{rgb}{0.019608,0.078431,0.470588}%
\pgfsetstrokecolor{currentstroke}%
\pgfsetdash{{5.550000pt}{2.400000pt}}{0.000000pt}%
\pgfpathmoveto{\pgfqpoint{2.266667in}{2.514332in}}%
\pgfpathlineto{\pgfqpoint{2.542222in}{2.514332in}}%
\pgfusepath{stroke}%
\end{pgfscope}%
\begin{pgfscope}%
\pgfpathrectangle{\pgfqpoint{0.475556in}{1.127740in}}{\pgfqpoint{4.960000in}{3.696000in}}%
\pgfusepath{clip}%
\pgfsetrectcap%
\pgfsetroundjoin%
\pgfsetlinewidth{1.505625pt}%
\definecolor{currentstroke}{rgb}{0.054902,0.560784,0.027451}%
\pgfsetstrokecolor{currentstroke}%
\pgfsetdash{}{0pt}%
\pgfpathmoveto{\pgfqpoint{2.817778in}{1.715740in}}%
\pgfpathlineto{\pgfqpoint{3.093334in}{1.715740in}}%
\pgfusepath{stroke}%
\end{pgfscope}%
\begin{pgfscope}%
\pgfpathrectangle{\pgfqpoint{0.475556in}{1.127740in}}{\pgfqpoint{4.960000in}{3.696000in}}%
\pgfusepath{clip}%
\pgfsetbuttcap%
\pgfsetroundjoin%
\pgfsetlinewidth{1.505625pt}%
\definecolor{currentstroke}{rgb}{0.019608,0.078431,0.470588}%
\pgfsetstrokecolor{currentstroke}%
\pgfsetdash{{5.550000pt}{2.400000pt}}{0.000000pt}%
\pgfpathmoveto{\pgfqpoint{2.817778in}{2.187008in}}%
\pgfpathlineto{\pgfqpoint{3.093334in}{2.187008in}}%
\pgfusepath{stroke}%
\end{pgfscope}%
\begin{pgfscope}%
\pgfpathrectangle{\pgfqpoint{0.475556in}{1.127740in}}{\pgfqpoint{4.960000in}{3.696000in}}%
\pgfusepath{clip}%
\pgfsetrectcap%
\pgfsetroundjoin%
\pgfsetlinewidth{1.505625pt}%
\definecolor{currentstroke}{rgb}{0.054902,0.560784,0.027451}%
\pgfsetstrokecolor{currentstroke}%
\pgfsetdash{}{0pt}%
\pgfpathmoveto{\pgfqpoint{3.368889in}{3.815740in}}%
\pgfpathlineto{\pgfqpoint{3.644445in}{3.815740in}}%
\pgfusepath{stroke}%
\end{pgfscope}%
\begin{pgfscope}%
\pgfpathrectangle{\pgfqpoint{0.475556in}{1.127740in}}{\pgfqpoint{4.960000in}{3.696000in}}%
\pgfusepath{clip}%
\pgfsetbuttcap%
\pgfsetroundjoin%
\pgfsetlinewidth{1.505625pt}%
\definecolor{currentstroke}{rgb}{0.019608,0.078431,0.470588}%
\pgfsetstrokecolor{currentstroke}%
\pgfsetdash{{5.550000pt}{2.400000pt}}{0.000000pt}%
\pgfpathmoveto{\pgfqpoint{3.368889in}{3.575177in}}%
\pgfpathlineto{\pgfqpoint{3.644445in}{3.575177in}}%
\pgfusepath{stroke}%
\end{pgfscope}%
\begin{pgfscope}%
\pgfpathrectangle{\pgfqpoint{0.475556in}{1.127740in}}{\pgfqpoint{4.960000in}{3.696000in}}%
\pgfusepath{clip}%
\pgfsetrectcap%
\pgfsetroundjoin%
\pgfsetlinewidth{1.505625pt}%
\definecolor{currentstroke}{rgb}{0.054902,0.560784,0.027451}%
\pgfsetstrokecolor{currentstroke}%
\pgfsetdash{}{0pt}%
\pgfpathmoveto{\pgfqpoint{3.920000in}{1.715740in}}%
\pgfpathlineto{\pgfqpoint{4.195556in}{1.715740in}}%
\pgfusepath{stroke}%
\end{pgfscope}%
\begin{pgfscope}%
\pgfpathrectangle{\pgfqpoint{0.475556in}{1.127740in}}{\pgfqpoint{4.960000in}{3.696000in}}%
\pgfusepath{clip}%
\pgfsetbuttcap%
\pgfsetroundjoin%
\pgfsetlinewidth{1.505625pt}%
\definecolor{currentstroke}{rgb}{0.019608,0.078431,0.470588}%
\pgfsetstrokecolor{currentstroke}%
\pgfsetdash{{5.550000pt}{2.400000pt}}{0.000000pt}%
\pgfpathmoveto{\pgfqpoint{3.920000in}{1.920811in}}%
\pgfpathlineto{\pgfqpoint{4.195556in}{1.920811in}}%
\pgfusepath{stroke}%
\end{pgfscope}%
\begin{pgfscope}%
\pgfpathrectangle{\pgfqpoint{0.475556in}{1.127740in}}{\pgfqpoint{4.960000in}{3.696000in}}%
\pgfusepath{clip}%
\pgfsetrectcap%
\pgfsetroundjoin%
\pgfsetlinewidth{1.505625pt}%
\definecolor{currentstroke}{rgb}{0.054902,0.560784,0.027451}%
\pgfsetstrokecolor{currentstroke}%
\pgfsetdash{}{0pt}%
\pgfpathmoveto{\pgfqpoint{4.471111in}{3.395740in}}%
\pgfpathlineto{\pgfqpoint{4.746667in}{3.395740in}}%
\pgfusepath{stroke}%
\end{pgfscope}%
\begin{pgfscope}%
\pgfpathrectangle{\pgfqpoint{0.475556in}{1.127740in}}{\pgfqpoint{4.960000in}{3.696000in}}%
\pgfusepath{clip}%
\pgfsetbuttcap%
\pgfsetroundjoin%
\pgfsetlinewidth{1.505625pt}%
\definecolor{currentstroke}{rgb}{0.019608,0.078431,0.470588}%
\pgfsetstrokecolor{currentstroke}%
\pgfsetdash{{5.550000pt}{2.400000pt}}{0.000000pt}%
\pgfpathmoveto{\pgfqpoint{4.471111in}{3.376022in}}%
\pgfpathlineto{\pgfqpoint{4.746667in}{3.376022in}}%
\pgfusepath{stroke}%
\end{pgfscope}%
\begin{pgfscope}%
\pgfpathrectangle{\pgfqpoint{0.475556in}{1.127740in}}{\pgfqpoint{4.960000in}{3.696000in}}%
\pgfusepath{clip}%
\pgfsetrectcap%
\pgfsetroundjoin%
\pgfsetlinewidth{1.505625pt}%
\definecolor{currentstroke}{rgb}{0.054902,0.560784,0.027451}%
\pgfsetstrokecolor{currentstroke}%
\pgfsetdash{}{0pt}%
\pgfpathmoveto{\pgfqpoint{5.022222in}{3.395740in}}%
\pgfpathlineto{\pgfqpoint{5.297778in}{3.395740in}}%
\pgfusepath{stroke}%
\end{pgfscope}%
\begin{pgfscope}%
\pgfpathrectangle{\pgfqpoint{0.475556in}{1.127740in}}{\pgfqpoint{4.960000in}{3.696000in}}%
\pgfusepath{clip}%
\pgfsetbuttcap%
\pgfsetroundjoin%
\pgfsetlinewidth{1.505625pt}%
\definecolor{currentstroke}{rgb}{0.019608,0.078431,0.470588}%
\pgfsetstrokecolor{currentstroke}%
\pgfsetdash{{5.550000pt}{2.400000pt}}{0.000000pt}%
\pgfpathmoveto{\pgfqpoint{5.022222in}{3.433205in}}%
\pgfpathlineto{\pgfqpoint{5.297778in}{3.433205in}}%
\pgfusepath{stroke}%
\end{pgfscope}%
\begin{pgfscope}%
\pgfsetrectcap%
\pgfsetmiterjoin%
\pgfsetlinewidth{0.803000pt}%
\definecolor{currentstroke}{rgb}{0.000000,0.000000,0.000000}%
\pgfsetstrokecolor{currentstroke}%
\pgfsetdash{}{0pt}%
\pgfpathmoveto{\pgfqpoint{0.475556in}{1.127740in}}%
\pgfpathlineto{\pgfqpoint{0.475556in}{4.823740in}}%
\pgfusepath{stroke}%
\end{pgfscope}%
\begin{pgfscope}%
\pgfsetrectcap%
\pgfsetmiterjoin%
\pgfsetlinewidth{0.803000pt}%
\definecolor{currentstroke}{rgb}{0.000000,0.000000,0.000000}%
\pgfsetstrokecolor{currentstroke}%
\pgfsetdash{}{0pt}%
\pgfpathmoveto{\pgfqpoint{5.435556in}{1.127740in}}%
\pgfpathlineto{\pgfqpoint{5.435556in}{4.823740in}}%
\pgfusepath{stroke}%
\end{pgfscope}%
\begin{pgfscope}%
\pgfsetrectcap%
\pgfsetmiterjoin%
\pgfsetlinewidth{0.803000pt}%
\definecolor{currentstroke}{rgb}{0.000000,0.000000,0.000000}%
\pgfsetstrokecolor{currentstroke}%
\pgfsetdash{}{0pt}%
\pgfpathmoveto{\pgfqpoint{0.475556in}{1.127740in}}%
\pgfpathlineto{\pgfqpoint{5.435556in}{1.127740in}}%
\pgfusepath{stroke}%
\end{pgfscope}%
\begin{pgfscope}%
\pgfsetrectcap%
\pgfsetmiterjoin%
\pgfsetlinewidth{0.803000pt}%
\definecolor{currentstroke}{rgb}{0.000000,0.000000,0.000000}%
\pgfsetstrokecolor{currentstroke}%
\pgfsetdash{}{0pt}%
\pgfpathmoveto{\pgfqpoint{0.475556in}{4.823740in}}%
\pgfpathlineto{\pgfqpoint{5.435556in}{4.823740in}}%
\pgfusepath{stroke}%
\end{pgfscope}%
\begin{pgfscope}%
\definecolor{textcolor}{rgb}{0.000000,0.000000,0.000000}%
\pgfsetstrokecolor{textcolor}%
\pgfsetfillcolor{textcolor}%
\pgftext[x=2.955556in,y=4.907073in,,base]{\color{textcolor}\sffamily\fontsize{12.000000}{14.400000}\selectfont Rectangular box plot}%
\end{pgfscope}%
\end{pgfpicture}%
\makeatother%
\endgroup%
}
    \caption[Average Rankings for Survey:Section 3]{The figure represents box plot for section 3. The orange horizontal line within the box indicates median, the blue dotted line indicates mean.
    The bolder boxes represents automated dataset, while lighter boxes represent manually created dataset. \Gls{free} has highest mean among the automated datasets.}
    \label{fig:question3_2}
\end{figure}

\subsection{Summary for Survey}
In the survey, we observe Pix3D was rightly chosen as a photorealistic image because it is a real-world image collection.
Though the proposed \gls{free} dataset falls behind in comparison to manually designed images from Hyperism, InteriorNet, and \gls{front},
it is comparable to automated datasets from Blenderproc, OpenRooms and SceneNet.
It also trumps over AI2THOR, which is manually designed using Unity game engines.

\section{Domain Gaps}\label{sec:domain-gaps}

This section verifies if the synthetic dataset; \gls{free} has a domain gap with the real dataset(Pix3D).
Along with the new synthetic dataset, we will compare the datasets used for the survey in \autoref{sec:a-survey-on-photorealism}.
Qualitatively we will visualize dataset embeddings using \gls{tsne} in \autoref{subsec:qualitative}; we will compare the distributions of all the synthetic datasets and the real dataset as in \autoref{subsec:quantitative}.

\subsection{Qualitative}\label{subsec:qualitative}

For qualitative assessment of the domains for each dataset, we utilize \gls{tsne} visualizations of embedding space from \gls{vgg}~\cite{simonyan2015deep}.
We consider a \gls{vgg}16 model pre-trained on ImageNet~\cite{Deng2009ImageNetAL} and use it as an encoder to embed the image space of all the images from each dataset.
This latent space embedding is then converted to 2-Dimensional representation using \gls{tsne} visualization.
A model trained on ImageNet can be used for encoding the images since it contains all the furniture categories present in Pix3D\@.
Hence the images will be respectfully embedded and mapped to 2D space.
As indicated in \autoref{subsec:visualizing-with-tsne}, the distances of the clusters, here dataset, is not represented by the \gls{tsne} visualization.
However, the overlap can be considered as an inference of occupying the same latent space.

\begin{figure}[ht]
    \centering
    \resizebox{0.9\textwidth}{8.5cm}{%% Creator: Matplotlib, PGF backend
%%
%% To include the figure in your LaTeX document, write
%%   \input{<filename>.pgf}
%%
%% Make sure the required packages are loaded in your preamble
%%   \usepackage{pgf}
%%
%% Figures using additional raster images can only be included by \input if
%% they are in the same directory as the main LaTeX file. For loading figures
%% from other directories you can use the `import` package
%%   \usepackage{import}
%%
%% and then include the figures with
%%   \import{<path to file>}{<filename>.pgf}
%%
%% Matplotlib used the following preamble
%%   \usepackage{fontspec}
%%   \setmainfont{DejaVuSerif.ttf}[Path=\detokenize{/Users/apple/opt/anaconda3/envs/kaolin/lib/python3.7/site-packages/matplotlib/mpl-data/fonts/ttf/}]
%%   \setsansfont{DejaVuSans.ttf}[Path=\detokenize{/Users/apple/opt/anaconda3/envs/kaolin/lib/python3.7/site-packages/matplotlib/mpl-data/fonts/ttf/}]
%%   \setmonofont{DejaVuSansMono.ttf}[Path=\detokenize{/Users/apple/opt/anaconda3/envs/kaolin/lib/python3.7/site-packages/matplotlib/mpl-data/fonts/ttf/}]
%%
\begingroup%
\makeatletter%
\begin{pgfpicture}%
\pgfpathrectangle{\pgfpointorigin}{\pgfqpoint{11.330057in}{8.341596in}}%
\pgfusepath{use as bounding box, clip}%
\begin{pgfscope}%
\pgfsetbuttcap%
\pgfsetmiterjoin%
\definecolor{currentfill}{rgb}{1.000000,1.000000,1.000000}%
\pgfsetfillcolor{currentfill}%
\pgfsetlinewidth{0.000000pt}%
\definecolor{currentstroke}{rgb}{1.000000,1.000000,1.000000}%
\pgfsetstrokecolor{currentstroke}%
\pgfsetdash{}{0pt}%
\pgfpathmoveto{\pgfqpoint{0.000000in}{0.000000in}}%
\pgfpathlineto{\pgfqpoint{11.330057in}{0.000000in}}%
\pgfpathlineto{\pgfqpoint{11.330057in}{8.341596in}}%
\pgfpathlineto{\pgfqpoint{0.000000in}{8.341596in}}%
\pgfpathclose%
\pgfusepath{fill}%
\end{pgfscope}%
\begin{pgfscope}%
\pgfsetbuttcap%
\pgfsetmiterjoin%
\definecolor{currentfill}{rgb}{1.000000,1.000000,1.000000}%
\pgfsetfillcolor{currentfill}%
\pgfsetlinewidth{0.000000pt}%
\definecolor{currentstroke}{rgb}{0.000000,0.000000,0.000000}%
\pgfsetstrokecolor{currentstroke}%
\pgfsetstrokeopacity{0.000000}%
\pgfsetdash{}{0pt}%
\pgfpathmoveto{\pgfqpoint{0.526127in}{0.331635in}}%
\pgfpathlineto{\pgfqpoint{9.826127in}{0.331635in}}%
\pgfpathlineto{\pgfqpoint{9.826127in}{8.031635in}}%
\pgfpathlineto{\pgfqpoint{0.526127in}{8.031635in}}%
\pgfpathclose%
\pgfusepath{fill}%
\end{pgfscope}%
\begin{pgfscope}%
\pgfpathrectangle{\pgfqpoint{0.526127in}{0.331635in}}{\pgfqpoint{9.300000in}{7.700000in}}%
\pgfusepath{clip}%
\pgfsetbuttcap%
\pgfsetroundjoin%
\definecolor{currentfill}{rgb}{0.631373,0.788235,0.956863}%
\pgfsetfillcolor{currentfill}%
\pgfsetlinewidth{0.481800pt}%
\definecolor{currentstroke}{rgb}{1.000000,1.000000,1.000000}%
\pgfsetstrokecolor{currentstroke}%
\pgfsetdash{}{0pt}%
\pgfpathmoveto{\pgfqpoint{8.423233in}{5.749929in}}%
\pgfpathcurveto{\pgfqpoint{8.434283in}{5.749929in}}{\pgfqpoint{8.444882in}{5.754319in}}{\pgfqpoint{8.452696in}{5.762133in}}%
\pgfpathcurveto{\pgfqpoint{8.460509in}{5.769946in}}{\pgfqpoint{8.464899in}{5.780545in}}{\pgfqpoint{8.464899in}{5.791596in}}%
\pgfpathcurveto{\pgfqpoint{8.464899in}{5.802646in}}{\pgfqpoint{8.460509in}{5.813245in}}{\pgfqpoint{8.452696in}{5.821058in}}%
\pgfpathcurveto{\pgfqpoint{8.444882in}{5.828872in}}{\pgfqpoint{8.434283in}{5.833262in}}{\pgfqpoint{8.423233in}{5.833262in}}%
\pgfpathcurveto{\pgfqpoint{8.412183in}{5.833262in}}{\pgfqpoint{8.401584in}{5.828872in}}{\pgfqpoint{8.393770in}{5.821058in}}%
\pgfpathcurveto{\pgfqpoint{8.385956in}{5.813245in}}{\pgfqpoint{8.381566in}{5.802646in}}{\pgfqpoint{8.381566in}{5.791596in}}%
\pgfpathcurveto{\pgfqpoint{8.381566in}{5.780545in}}{\pgfqpoint{8.385956in}{5.769946in}}{\pgfqpoint{8.393770in}{5.762133in}}%
\pgfpathcurveto{\pgfqpoint{8.401584in}{5.754319in}}{\pgfqpoint{8.412183in}{5.749929in}}{\pgfqpoint{8.423233in}{5.749929in}}%
\pgfpathclose%
\pgfusepath{stroke,fill}%
\end{pgfscope}%
\begin{pgfscope}%
\pgfpathrectangle{\pgfqpoint{0.526127in}{0.331635in}}{\pgfqpoint{9.300000in}{7.700000in}}%
\pgfusepath{clip}%
\pgfsetbuttcap%
\pgfsetroundjoin%
\definecolor{currentfill}{rgb}{0.631373,0.788235,0.956863}%
\pgfsetfillcolor{currentfill}%
\pgfsetlinewidth{0.481800pt}%
\definecolor{currentstroke}{rgb}{1.000000,1.000000,1.000000}%
\pgfsetstrokecolor{currentstroke}%
\pgfsetdash{}{0pt}%
\pgfpathmoveto{\pgfqpoint{4.747368in}{3.873443in}}%
\pgfpathcurveto{\pgfqpoint{4.758419in}{3.873443in}}{\pgfqpoint{4.769018in}{3.877833in}}{\pgfqpoint{4.776831in}{3.885647in}}%
\pgfpathcurveto{\pgfqpoint{4.784645in}{3.893461in}}{\pgfqpoint{4.789035in}{3.904060in}}{\pgfqpoint{4.789035in}{3.915110in}}%
\pgfpathcurveto{\pgfqpoint{4.789035in}{3.926160in}}{\pgfqpoint{4.784645in}{3.936759in}}{\pgfqpoint{4.776831in}{3.944573in}}%
\pgfpathcurveto{\pgfqpoint{4.769018in}{3.952386in}}{\pgfqpoint{4.758419in}{3.956777in}}{\pgfqpoint{4.747368in}{3.956777in}}%
\pgfpathcurveto{\pgfqpoint{4.736318in}{3.956777in}}{\pgfqpoint{4.725719in}{3.952386in}}{\pgfqpoint{4.717906in}{3.944573in}}%
\pgfpathcurveto{\pgfqpoint{4.710092in}{3.936759in}}{\pgfqpoint{4.705702in}{3.926160in}}{\pgfqpoint{4.705702in}{3.915110in}}%
\pgfpathcurveto{\pgfqpoint{4.705702in}{3.904060in}}{\pgfqpoint{4.710092in}{3.893461in}}{\pgfqpoint{4.717906in}{3.885647in}}%
\pgfpathcurveto{\pgfqpoint{4.725719in}{3.877833in}}{\pgfqpoint{4.736318in}{3.873443in}}{\pgfqpoint{4.747368in}{3.873443in}}%
\pgfpathclose%
\pgfusepath{stroke,fill}%
\end{pgfscope}%
\begin{pgfscope}%
\pgfpathrectangle{\pgfqpoint{0.526127in}{0.331635in}}{\pgfqpoint{9.300000in}{7.700000in}}%
\pgfusepath{clip}%
\pgfsetbuttcap%
\pgfsetroundjoin%
\definecolor{currentfill}{rgb}{0.631373,0.788235,0.956863}%
\pgfsetfillcolor{currentfill}%
\pgfsetlinewidth{0.481800pt}%
\definecolor{currentstroke}{rgb}{1.000000,1.000000,1.000000}%
\pgfsetstrokecolor{currentstroke}%
\pgfsetdash{}{0pt}%
\pgfpathmoveto{\pgfqpoint{6.866294in}{4.133304in}}%
\pgfpathcurveto{\pgfqpoint{6.877344in}{4.133304in}}{\pgfqpoint{6.887943in}{4.137694in}}{\pgfqpoint{6.895757in}{4.145508in}}%
\pgfpathcurveto{\pgfqpoint{6.903570in}{4.153322in}}{\pgfqpoint{6.907961in}{4.163921in}}{\pgfqpoint{6.907961in}{4.174971in}}%
\pgfpathcurveto{\pgfqpoint{6.907961in}{4.186021in}}{\pgfqpoint{6.903570in}{4.196620in}}{\pgfqpoint{6.895757in}{4.204433in}}%
\pgfpathcurveto{\pgfqpoint{6.887943in}{4.212247in}}{\pgfqpoint{6.877344in}{4.216637in}}{\pgfqpoint{6.866294in}{4.216637in}}%
\pgfpathcurveto{\pgfqpoint{6.855244in}{4.216637in}}{\pgfqpoint{6.844645in}{4.212247in}}{\pgfqpoint{6.836831in}{4.204433in}}%
\pgfpathcurveto{\pgfqpoint{6.829018in}{4.196620in}}{\pgfqpoint{6.824627in}{4.186021in}}{\pgfqpoint{6.824627in}{4.174971in}}%
\pgfpathcurveto{\pgfqpoint{6.824627in}{4.163921in}}{\pgfqpoint{6.829018in}{4.153322in}}{\pgfqpoint{6.836831in}{4.145508in}}%
\pgfpathcurveto{\pgfqpoint{6.844645in}{4.137694in}}{\pgfqpoint{6.855244in}{4.133304in}}{\pgfqpoint{6.866294in}{4.133304in}}%
\pgfpathclose%
\pgfusepath{stroke,fill}%
\end{pgfscope}%
\begin{pgfscope}%
\pgfpathrectangle{\pgfqpoint{0.526127in}{0.331635in}}{\pgfqpoint{9.300000in}{7.700000in}}%
\pgfusepath{clip}%
\pgfsetbuttcap%
\pgfsetroundjoin%
\definecolor{currentfill}{rgb}{0.631373,0.788235,0.956863}%
\pgfsetfillcolor{currentfill}%
\pgfsetlinewidth{0.481800pt}%
\definecolor{currentstroke}{rgb}{1.000000,1.000000,1.000000}%
\pgfsetstrokecolor{currentstroke}%
\pgfsetdash{}{0pt}%
\pgfpathmoveto{\pgfqpoint{8.257844in}{5.197892in}}%
\pgfpathcurveto{\pgfqpoint{8.268894in}{5.197892in}}{\pgfqpoint{8.279493in}{5.202283in}}{\pgfqpoint{8.287306in}{5.210096in}}%
\pgfpathcurveto{\pgfqpoint{8.295120in}{5.217910in}}{\pgfqpoint{8.299510in}{5.228509in}}{\pgfqpoint{8.299510in}{5.239559in}}%
\pgfpathcurveto{\pgfqpoint{8.299510in}{5.250609in}}{\pgfqpoint{8.295120in}{5.261208in}}{\pgfqpoint{8.287306in}{5.269022in}}%
\pgfpathcurveto{\pgfqpoint{8.279493in}{5.276835in}}{\pgfqpoint{8.268894in}{5.281226in}}{\pgfqpoint{8.257844in}{5.281226in}}%
\pgfpathcurveto{\pgfqpoint{8.246794in}{5.281226in}}{\pgfqpoint{8.236195in}{5.276835in}}{\pgfqpoint{8.228381in}{5.269022in}}%
\pgfpathcurveto{\pgfqpoint{8.220567in}{5.261208in}}{\pgfqpoint{8.216177in}{5.250609in}}{\pgfqpoint{8.216177in}{5.239559in}}%
\pgfpathcurveto{\pgfqpoint{8.216177in}{5.228509in}}{\pgfqpoint{8.220567in}{5.217910in}}{\pgfqpoint{8.228381in}{5.210096in}}%
\pgfpathcurveto{\pgfqpoint{8.236195in}{5.202283in}}{\pgfqpoint{8.246794in}{5.197892in}}{\pgfqpoint{8.257844in}{5.197892in}}%
\pgfpathclose%
\pgfusepath{stroke,fill}%
\end{pgfscope}%
\begin{pgfscope}%
\pgfpathrectangle{\pgfqpoint{0.526127in}{0.331635in}}{\pgfqpoint{9.300000in}{7.700000in}}%
\pgfusepath{clip}%
\pgfsetbuttcap%
\pgfsetroundjoin%
\definecolor{currentfill}{rgb}{0.631373,0.788235,0.956863}%
\pgfsetfillcolor{currentfill}%
\pgfsetlinewidth{0.481800pt}%
\definecolor{currentstroke}{rgb}{1.000000,1.000000,1.000000}%
\pgfsetstrokecolor{currentstroke}%
\pgfsetdash{}{0pt}%
\pgfpathmoveto{\pgfqpoint{3.377597in}{2.706317in}}%
\pgfpathcurveto{\pgfqpoint{3.388647in}{2.706317in}}{\pgfqpoint{3.399246in}{2.710708in}}{\pgfqpoint{3.407060in}{2.718521in}}%
\pgfpathcurveto{\pgfqpoint{3.414873in}{2.726335in}}{\pgfqpoint{3.419263in}{2.736934in}}{\pgfqpoint{3.419263in}{2.747984in}}%
\pgfpathcurveto{\pgfqpoint{3.419263in}{2.759034in}}{\pgfqpoint{3.414873in}{2.769633in}}{\pgfqpoint{3.407060in}{2.777447in}}%
\pgfpathcurveto{\pgfqpoint{3.399246in}{2.785260in}}{\pgfqpoint{3.388647in}{2.789651in}}{\pgfqpoint{3.377597in}{2.789651in}}%
\pgfpathcurveto{\pgfqpoint{3.366547in}{2.789651in}}{\pgfqpoint{3.355948in}{2.785260in}}{\pgfqpoint{3.348134in}{2.777447in}}%
\pgfpathcurveto{\pgfqpoint{3.340320in}{2.769633in}}{\pgfqpoint{3.335930in}{2.759034in}}{\pgfqpoint{3.335930in}{2.747984in}}%
\pgfpathcurveto{\pgfqpoint{3.335930in}{2.736934in}}{\pgfqpoint{3.340320in}{2.726335in}}{\pgfqpoint{3.348134in}{2.718521in}}%
\pgfpathcurveto{\pgfqpoint{3.355948in}{2.710708in}}{\pgfqpoint{3.366547in}{2.706317in}}{\pgfqpoint{3.377597in}{2.706317in}}%
\pgfpathclose%
\pgfusepath{stroke,fill}%
\end{pgfscope}%
\begin{pgfscope}%
\pgfpathrectangle{\pgfqpoint{0.526127in}{0.331635in}}{\pgfqpoint{9.300000in}{7.700000in}}%
\pgfusepath{clip}%
\pgfsetbuttcap%
\pgfsetroundjoin%
\definecolor{currentfill}{rgb}{0.631373,0.788235,0.956863}%
\pgfsetfillcolor{currentfill}%
\pgfsetlinewidth{0.481800pt}%
\definecolor{currentstroke}{rgb}{1.000000,1.000000,1.000000}%
\pgfsetstrokecolor{currentstroke}%
\pgfsetdash{}{0pt}%
\pgfpathmoveto{\pgfqpoint{5.374384in}{6.532835in}}%
\pgfpathcurveto{\pgfqpoint{5.385434in}{6.532835in}}{\pgfqpoint{5.396033in}{6.537225in}}{\pgfqpoint{5.403847in}{6.545039in}}%
\pgfpathcurveto{\pgfqpoint{5.411661in}{6.552852in}}{\pgfqpoint{5.416051in}{6.563451in}}{\pgfqpoint{5.416051in}{6.574501in}}%
\pgfpathcurveto{\pgfqpoint{5.416051in}{6.585552in}}{\pgfqpoint{5.411661in}{6.596151in}}{\pgfqpoint{5.403847in}{6.603964in}}%
\pgfpathcurveto{\pgfqpoint{5.396033in}{6.611778in}}{\pgfqpoint{5.385434in}{6.616168in}}{\pgfqpoint{5.374384in}{6.616168in}}%
\pgfpathcurveto{\pgfqpoint{5.363334in}{6.616168in}}{\pgfqpoint{5.352735in}{6.611778in}}{\pgfqpoint{5.344922in}{6.603964in}}%
\pgfpathcurveto{\pgfqpoint{5.337108in}{6.596151in}}{\pgfqpoint{5.332718in}{6.585552in}}{\pgfqpoint{5.332718in}{6.574501in}}%
\pgfpathcurveto{\pgfqpoint{5.332718in}{6.563451in}}{\pgfqpoint{5.337108in}{6.552852in}}{\pgfqpoint{5.344922in}{6.545039in}}%
\pgfpathcurveto{\pgfqpoint{5.352735in}{6.537225in}}{\pgfqpoint{5.363334in}{6.532835in}}{\pgfqpoint{5.374384in}{6.532835in}}%
\pgfpathclose%
\pgfusepath{stroke,fill}%
\end{pgfscope}%
\begin{pgfscope}%
\pgfpathrectangle{\pgfqpoint{0.526127in}{0.331635in}}{\pgfqpoint{9.300000in}{7.700000in}}%
\pgfusepath{clip}%
\pgfsetbuttcap%
\pgfsetroundjoin%
\definecolor{currentfill}{rgb}{0.631373,0.788235,0.956863}%
\pgfsetfillcolor{currentfill}%
\pgfsetlinewidth{0.481800pt}%
\definecolor{currentstroke}{rgb}{1.000000,1.000000,1.000000}%
\pgfsetstrokecolor{currentstroke}%
\pgfsetdash{}{0pt}%
\pgfpathmoveto{\pgfqpoint{3.945884in}{4.821923in}}%
\pgfpathcurveto{\pgfqpoint{3.956934in}{4.821923in}}{\pgfqpoint{3.967533in}{4.826313in}}{\pgfqpoint{3.975347in}{4.834127in}}%
\pgfpathcurveto{\pgfqpoint{3.983161in}{4.841941in}}{\pgfqpoint{3.987551in}{4.852540in}}{\pgfqpoint{3.987551in}{4.863590in}}%
\pgfpathcurveto{\pgfqpoint{3.987551in}{4.874640in}}{\pgfqpoint{3.983161in}{4.885239in}}{\pgfqpoint{3.975347in}{4.893053in}}%
\pgfpathcurveto{\pgfqpoint{3.967533in}{4.900866in}}{\pgfqpoint{3.956934in}{4.905256in}}{\pgfqpoint{3.945884in}{4.905256in}}%
\pgfpathcurveto{\pgfqpoint{3.934834in}{4.905256in}}{\pgfqpoint{3.924235in}{4.900866in}}{\pgfqpoint{3.916421in}{4.893053in}}%
\pgfpathcurveto{\pgfqpoint{3.908608in}{4.885239in}}{\pgfqpoint{3.904218in}{4.874640in}}{\pgfqpoint{3.904218in}{4.863590in}}%
\pgfpathcurveto{\pgfqpoint{3.904218in}{4.852540in}}{\pgfqpoint{3.908608in}{4.841941in}}{\pgfqpoint{3.916421in}{4.834127in}}%
\pgfpathcurveto{\pgfqpoint{3.924235in}{4.826313in}}{\pgfqpoint{3.934834in}{4.821923in}}{\pgfqpoint{3.945884in}{4.821923in}}%
\pgfpathclose%
\pgfusepath{stroke,fill}%
\end{pgfscope}%
\begin{pgfscope}%
\pgfpathrectangle{\pgfqpoint{0.526127in}{0.331635in}}{\pgfqpoint{9.300000in}{7.700000in}}%
\pgfusepath{clip}%
\pgfsetbuttcap%
\pgfsetroundjoin%
\definecolor{currentfill}{rgb}{0.631373,0.788235,0.956863}%
\pgfsetfillcolor{currentfill}%
\pgfsetlinewidth{0.481800pt}%
\definecolor{currentstroke}{rgb}{1.000000,1.000000,1.000000}%
\pgfsetstrokecolor{currentstroke}%
\pgfsetdash{}{0pt}%
\pgfpathmoveto{\pgfqpoint{6.422655in}{6.020429in}}%
\pgfpathcurveto{\pgfqpoint{6.433705in}{6.020429in}}{\pgfqpoint{6.444304in}{6.024819in}}{\pgfqpoint{6.452117in}{6.032633in}}%
\pgfpathcurveto{\pgfqpoint{6.459931in}{6.040447in}}{\pgfqpoint{6.464321in}{6.051046in}}{\pgfqpoint{6.464321in}{6.062096in}}%
\pgfpathcurveto{\pgfqpoint{6.464321in}{6.073146in}}{\pgfqpoint{6.459931in}{6.083745in}}{\pgfqpoint{6.452117in}{6.091559in}}%
\pgfpathcurveto{\pgfqpoint{6.444304in}{6.099372in}}{\pgfqpoint{6.433705in}{6.103762in}}{\pgfqpoint{6.422655in}{6.103762in}}%
\pgfpathcurveto{\pgfqpoint{6.411604in}{6.103762in}}{\pgfqpoint{6.401005in}{6.099372in}}{\pgfqpoint{6.393192in}{6.091559in}}%
\pgfpathcurveto{\pgfqpoint{6.385378in}{6.083745in}}{\pgfqpoint{6.380988in}{6.073146in}}{\pgfqpoint{6.380988in}{6.062096in}}%
\pgfpathcurveto{\pgfqpoint{6.380988in}{6.051046in}}{\pgfqpoint{6.385378in}{6.040447in}}{\pgfqpoint{6.393192in}{6.032633in}}%
\pgfpathcurveto{\pgfqpoint{6.401005in}{6.024819in}}{\pgfqpoint{6.411604in}{6.020429in}}{\pgfqpoint{6.422655in}{6.020429in}}%
\pgfpathclose%
\pgfusepath{stroke,fill}%
\end{pgfscope}%
\begin{pgfscope}%
\pgfpathrectangle{\pgfqpoint{0.526127in}{0.331635in}}{\pgfqpoint{9.300000in}{7.700000in}}%
\pgfusepath{clip}%
\pgfsetbuttcap%
\pgfsetroundjoin%
\definecolor{currentfill}{rgb}{0.631373,0.788235,0.956863}%
\pgfsetfillcolor{currentfill}%
\pgfsetlinewidth{0.481800pt}%
\definecolor{currentstroke}{rgb}{1.000000,1.000000,1.000000}%
\pgfsetstrokecolor{currentstroke}%
\pgfsetdash{}{0pt}%
\pgfpathmoveto{\pgfqpoint{3.784322in}{5.153163in}}%
\pgfpathcurveto{\pgfqpoint{3.795372in}{5.153163in}}{\pgfqpoint{3.805971in}{5.157553in}}{\pgfqpoint{3.813785in}{5.165367in}}%
\pgfpathcurveto{\pgfqpoint{3.821599in}{5.173180in}}{\pgfqpoint{3.825989in}{5.183779in}}{\pgfqpoint{3.825989in}{5.194829in}}%
\pgfpathcurveto{\pgfqpoint{3.825989in}{5.205880in}}{\pgfqpoint{3.821599in}{5.216479in}}{\pgfqpoint{3.813785in}{5.224292in}}%
\pgfpathcurveto{\pgfqpoint{3.805971in}{5.232106in}}{\pgfqpoint{3.795372in}{5.236496in}}{\pgfqpoint{3.784322in}{5.236496in}}%
\pgfpathcurveto{\pgfqpoint{3.773272in}{5.236496in}}{\pgfqpoint{3.762673in}{5.232106in}}{\pgfqpoint{3.754860in}{5.224292in}}%
\pgfpathcurveto{\pgfqpoint{3.747046in}{5.216479in}}{\pgfqpoint{3.742656in}{5.205880in}}{\pgfqpoint{3.742656in}{5.194829in}}%
\pgfpathcurveto{\pgfqpoint{3.742656in}{5.183779in}}{\pgfqpoint{3.747046in}{5.173180in}}{\pgfqpoint{3.754860in}{5.165367in}}%
\pgfpathcurveto{\pgfqpoint{3.762673in}{5.157553in}}{\pgfqpoint{3.773272in}{5.153163in}}{\pgfqpoint{3.784322in}{5.153163in}}%
\pgfpathclose%
\pgfusepath{stroke,fill}%
\end{pgfscope}%
\begin{pgfscope}%
\pgfpathrectangle{\pgfqpoint{0.526127in}{0.331635in}}{\pgfqpoint{9.300000in}{7.700000in}}%
\pgfusepath{clip}%
\pgfsetbuttcap%
\pgfsetroundjoin%
\definecolor{currentfill}{rgb}{0.631373,0.788235,0.956863}%
\pgfsetfillcolor{currentfill}%
\pgfsetlinewidth{0.481800pt}%
\definecolor{currentstroke}{rgb}{1.000000,1.000000,1.000000}%
\pgfsetstrokecolor{currentstroke}%
\pgfsetdash{}{0pt}%
\pgfpathmoveto{\pgfqpoint{4.807076in}{4.359110in}}%
\pgfpathcurveto{\pgfqpoint{4.818126in}{4.359110in}}{\pgfqpoint{4.828725in}{4.363500in}}{\pgfqpoint{4.836539in}{4.371314in}}%
\pgfpathcurveto{\pgfqpoint{4.844352in}{4.379128in}}{\pgfqpoint{4.848743in}{4.389727in}}{\pgfqpoint{4.848743in}{4.400777in}}%
\pgfpathcurveto{\pgfqpoint{4.848743in}{4.411827in}}{\pgfqpoint{4.844352in}{4.422426in}}{\pgfqpoint{4.836539in}{4.430240in}}%
\pgfpathcurveto{\pgfqpoint{4.828725in}{4.438053in}}{\pgfqpoint{4.818126in}{4.442444in}}{\pgfqpoint{4.807076in}{4.442444in}}%
\pgfpathcurveto{\pgfqpoint{4.796026in}{4.442444in}}{\pgfqpoint{4.785427in}{4.438053in}}{\pgfqpoint{4.777613in}{4.430240in}}%
\pgfpathcurveto{\pgfqpoint{4.769800in}{4.422426in}}{\pgfqpoint{4.765409in}{4.411827in}}{\pgfqpoint{4.765409in}{4.400777in}}%
\pgfpathcurveto{\pgfqpoint{4.765409in}{4.389727in}}{\pgfqpoint{4.769800in}{4.379128in}}{\pgfqpoint{4.777613in}{4.371314in}}%
\pgfpathcurveto{\pgfqpoint{4.785427in}{4.363500in}}{\pgfqpoint{4.796026in}{4.359110in}}{\pgfqpoint{4.807076in}{4.359110in}}%
\pgfpathclose%
\pgfusepath{stroke,fill}%
\end{pgfscope}%
\begin{pgfscope}%
\pgfpathrectangle{\pgfqpoint{0.526127in}{0.331635in}}{\pgfqpoint{9.300000in}{7.700000in}}%
\pgfusepath{clip}%
\pgfsetbuttcap%
\pgfsetroundjoin%
\definecolor{currentfill}{rgb}{0.631373,0.788235,0.956863}%
\pgfsetfillcolor{currentfill}%
\pgfsetlinewidth{0.481800pt}%
\definecolor{currentstroke}{rgb}{1.000000,1.000000,1.000000}%
\pgfsetstrokecolor{currentstroke}%
\pgfsetdash{}{0pt}%
\pgfpathmoveto{\pgfqpoint{7.809864in}{6.873138in}}%
\pgfpathcurveto{\pgfqpoint{7.820915in}{6.873138in}}{\pgfqpoint{7.831514in}{6.877528in}}{\pgfqpoint{7.839327in}{6.885342in}}%
\pgfpathcurveto{\pgfqpoint{7.847141in}{6.893155in}}{\pgfqpoint{7.851531in}{6.903754in}}{\pgfqpoint{7.851531in}{6.914805in}}%
\pgfpathcurveto{\pgfqpoint{7.851531in}{6.925855in}}{\pgfqpoint{7.847141in}{6.936454in}}{\pgfqpoint{7.839327in}{6.944267in}}%
\pgfpathcurveto{\pgfqpoint{7.831514in}{6.952081in}}{\pgfqpoint{7.820915in}{6.956471in}}{\pgfqpoint{7.809864in}{6.956471in}}%
\pgfpathcurveto{\pgfqpoint{7.798814in}{6.956471in}}{\pgfqpoint{7.788215in}{6.952081in}}{\pgfqpoint{7.780402in}{6.944267in}}%
\pgfpathcurveto{\pgfqpoint{7.772588in}{6.936454in}}{\pgfqpoint{7.768198in}{6.925855in}}{\pgfqpoint{7.768198in}{6.914805in}}%
\pgfpathcurveto{\pgfqpoint{7.768198in}{6.903754in}}{\pgfqpoint{7.772588in}{6.893155in}}{\pgfqpoint{7.780402in}{6.885342in}}%
\pgfpathcurveto{\pgfqpoint{7.788215in}{6.877528in}}{\pgfqpoint{7.798814in}{6.873138in}}{\pgfqpoint{7.809864in}{6.873138in}}%
\pgfpathclose%
\pgfusepath{stroke,fill}%
\end{pgfscope}%
\begin{pgfscope}%
\pgfpathrectangle{\pgfqpoint{0.526127in}{0.331635in}}{\pgfqpoint{9.300000in}{7.700000in}}%
\pgfusepath{clip}%
\pgfsetbuttcap%
\pgfsetroundjoin%
\definecolor{currentfill}{rgb}{0.631373,0.788235,0.956863}%
\pgfsetfillcolor{currentfill}%
\pgfsetlinewidth{0.481800pt}%
\definecolor{currentstroke}{rgb}{1.000000,1.000000,1.000000}%
\pgfsetstrokecolor{currentstroke}%
\pgfsetdash{}{0pt}%
\pgfpathmoveto{\pgfqpoint{7.744328in}{2.730779in}}%
\pgfpathcurveto{\pgfqpoint{7.755378in}{2.730779in}}{\pgfqpoint{7.765977in}{2.735169in}}{\pgfqpoint{7.773791in}{2.742983in}}%
\pgfpathcurveto{\pgfqpoint{7.781604in}{2.750796in}}{\pgfqpoint{7.785995in}{2.761395in}}{\pgfqpoint{7.785995in}{2.772445in}}%
\pgfpathcurveto{\pgfqpoint{7.785995in}{2.783495in}}{\pgfqpoint{7.781604in}{2.794095in}}{\pgfqpoint{7.773791in}{2.801908in}}%
\pgfpathcurveto{\pgfqpoint{7.765977in}{2.809722in}}{\pgfqpoint{7.755378in}{2.814112in}}{\pgfqpoint{7.744328in}{2.814112in}}%
\pgfpathcurveto{\pgfqpoint{7.733278in}{2.814112in}}{\pgfqpoint{7.722679in}{2.809722in}}{\pgfqpoint{7.714865in}{2.801908in}}%
\pgfpathcurveto{\pgfqpoint{7.707051in}{2.794095in}}{\pgfqpoint{7.702661in}{2.783495in}}{\pgfqpoint{7.702661in}{2.772445in}}%
\pgfpathcurveto{\pgfqpoint{7.702661in}{2.761395in}}{\pgfqpoint{7.707051in}{2.750796in}}{\pgfqpoint{7.714865in}{2.742983in}}%
\pgfpathcurveto{\pgfqpoint{7.722679in}{2.735169in}}{\pgfqpoint{7.733278in}{2.730779in}}{\pgfqpoint{7.744328in}{2.730779in}}%
\pgfpathclose%
\pgfusepath{stroke,fill}%
\end{pgfscope}%
\begin{pgfscope}%
\pgfpathrectangle{\pgfqpoint{0.526127in}{0.331635in}}{\pgfqpoint{9.300000in}{7.700000in}}%
\pgfusepath{clip}%
\pgfsetbuttcap%
\pgfsetroundjoin%
\definecolor{currentfill}{rgb}{0.631373,0.788235,0.956863}%
\pgfsetfillcolor{currentfill}%
\pgfsetlinewidth{0.481800pt}%
\definecolor{currentstroke}{rgb}{1.000000,1.000000,1.000000}%
\pgfsetstrokecolor{currentstroke}%
\pgfsetdash{}{0pt}%
\pgfpathmoveto{\pgfqpoint{6.917985in}{5.041600in}}%
\pgfpathcurveto{\pgfqpoint{6.929036in}{5.041600in}}{\pgfqpoint{6.939635in}{5.045990in}}{\pgfqpoint{6.947448in}{5.053804in}}%
\pgfpathcurveto{\pgfqpoint{6.955262in}{5.061617in}}{\pgfqpoint{6.959652in}{5.072216in}}{\pgfqpoint{6.959652in}{5.083266in}}%
\pgfpathcurveto{\pgfqpoint{6.959652in}{5.094316in}}{\pgfqpoint{6.955262in}{5.104915in}}{\pgfqpoint{6.947448in}{5.112729in}}%
\pgfpathcurveto{\pgfqpoint{6.939635in}{5.120543in}}{\pgfqpoint{6.929036in}{5.124933in}}{\pgfqpoint{6.917985in}{5.124933in}}%
\pgfpathcurveto{\pgfqpoint{6.906935in}{5.124933in}}{\pgfqpoint{6.896336in}{5.120543in}}{\pgfqpoint{6.888523in}{5.112729in}}%
\pgfpathcurveto{\pgfqpoint{6.880709in}{5.104915in}}{\pgfqpoint{6.876319in}{5.094316in}}{\pgfqpoint{6.876319in}{5.083266in}}%
\pgfpathcurveto{\pgfqpoint{6.876319in}{5.072216in}}{\pgfqpoint{6.880709in}{5.061617in}}{\pgfqpoint{6.888523in}{5.053804in}}%
\pgfpathcurveto{\pgfqpoint{6.896336in}{5.045990in}}{\pgfqpoint{6.906935in}{5.041600in}}{\pgfqpoint{6.917985in}{5.041600in}}%
\pgfpathclose%
\pgfusepath{stroke,fill}%
\end{pgfscope}%
\begin{pgfscope}%
\pgfpathrectangle{\pgfqpoint{0.526127in}{0.331635in}}{\pgfqpoint{9.300000in}{7.700000in}}%
\pgfusepath{clip}%
\pgfsetbuttcap%
\pgfsetroundjoin%
\definecolor{currentfill}{rgb}{0.631373,0.788235,0.956863}%
\pgfsetfillcolor{currentfill}%
\pgfsetlinewidth{0.481800pt}%
\definecolor{currentstroke}{rgb}{1.000000,1.000000,1.000000}%
\pgfsetstrokecolor{currentstroke}%
\pgfsetdash{}{0pt}%
\pgfpathmoveto{\pgfqpoint{5.740650in}{6.378204in}}%
\pgfpathcurveto{\pgfqpoint{5.751700in}{6.378204in}}{\pgfqpoint{5.762299in}{6.382595in}}{\pgfqpoint{5.770113in}{6.390408in}}%
\pgfpathcurveto{\pgfqpoint{5.777926in}{6.398222in}}{\pgfqpoint{5.782317in}{6.408821in}}{\pgfqpoint{5.782317in}{6.419871in}}%
\pgfpathcurveto{\pgfqpoint{5.782317in}{6.430921in}}{\pgfqpoint{5.777926in}{6.441520in}}{\pgfqpoint{5.770113in}{6.449334in}}%
\pgfpathcurveto{\pgfqpoint{5.762299in}{6.457148in}}{\pgfqpoint{5.751700in}{6.461538in}}{\pgfqpoint{5.740650in}{6.461538in}}%
\pgfpathcurveto{\pgfqpoint{5.729600in}{6.461538in}}{\pgfqpoint{5.719001in}{6.457148in}}{\pgfqpoint{5.711187in}{6.449334in}}%
\pgfpathcurveto{\pgfqpoint{5.703374in}{6.441520in}}{\pgfqpoint{5.698983in}{6.430921in}}{\pgfqpoint{5.698983in}{6.419871in}}%
\pgfpathcurveto{\pgfqpoint{5.698983in}{6.408821in}}{\pgfqpoint{5.703374in}{6.398222in}}{\pgfqpoint{5.711187in}{6.390408in}}%
\pgfpathcurveto{\pgfqpoint{5.719001in}{6.382595in}}{\pgfqpoint{5.729600in}{6.378204in}}{\pgfqpoint{5.740650in}{6.378204in}}%
\pgfpathclose%
\pgfusepath{stroke,fill}%
\end{pgfscope}%
\begin{pgfscope}%
\pgfpathrectangle{\pgfqpoint{0.526127in}{0.331635in}}{\pgfqpoint{9.300000in}{7.700000in}}%
\pgfusepath{clip}%
\pgfsetbuttcap%
\pgfsetroundjoin%
\definecolor{currentfill}{rgb}{0.631373,0.788235,0.956863}%
\pgfsetfillcolor{currentfill}%
\pgfsetlinewidth{0.481800pt}%
\definecolor{currentstroke}{rgb}{1.000000,1.000000,1.000000}%
\pgfsetstrokecolor{currentstroke}%
\pgfsetdash{}{0pt}%
\pgfpathmoveto{\pgfqpoint{7.681588in}{6.050549in}}%
\pgfpathcurveto{\pgfqpoint{7.692638in}{6.050549in}}{\pgfqpoint{7.703237in}{6.054939in}}{\pgfqpoint{7.711051in}{6.062753in}}%
\pgfpathcurveto{\pgfqpoint{7.718865in}{6.070566in}}{\pgfqpoint{7.723255in}{6.081165in}}{\pgfqpoint{7.723255in}{6.092216in}}%
\pgfpathcurveto{\pgfqpoint{7.723255in}{6.103266in}}{\pgfqpoint{7.718865in}{6.113865in}}{\pgfqpoint{7.711051in}{6.121678in}}%
\pgfpathcurveto{\pgfqpoint{7.703237in}{6.129492in}}{\pgfqpoint{7.692638in}{6.133882in}}{\pgfqpoint{7.681588in}{6.133882in}}%
\pgfpathcurveto{\pgfqpoint{7.670538in}{6.133882in}}{\pgfqpoint{7.659939in}{6.129492in}}{\pgfqpoint{7.652126in}{6.121678in}}%
\pgfpathcurveto{\pgfqpoint{7.644312in}{6.113865in}}{\pgfqpoint{7.639922in}{6.103266in}}{\pgfqpoint{7.639922in}{6.092216in}}%
\pgfpathcurveto{\pgfqpoint{7.639922in}{6.081165in}}{\pgfqpoint{7.644312in}{6.070566in}}{\pgfqpoint{7.652126in}{6.062753in}}%
\pgfpathcurveto{\pgfqpoint{7.659939in}{6.054939in}}{\pgfqpoint{7.670538in}{6.050549in}}{\pgfqpoint{7.681588in}{6.050549in}}%
\pgfpathclose%
\pgfusepath{stroke,fill}%
\end{pgfscope}%
\begin{pgfscope}%
\pgfpathrectangle{\pgfqpoint{0.526127in}{0.331635in}}{\pgfqpoint{9.300000in}{7.700000in}}%
\pgfusepath{clip}%
\pgfsetbuttcap%
\pgfsetroundjoin%
\definecolor{currentfill}{rgb}{0.631373,0.788235,0.956863}%
\pgfsetfillcolor{currentfill}%
\pgfsetlinewidth{0.481800pt}%
\definecolor{currentstroke}{rgb}{1.000000,1.000000,1.000000}%
\pgfsetstrokecolor{currentstroke}%
\pgfsetdash{}{0pt}%
\pgfpathmoveto{\pgfqpoint{6.865837in}{4.587417in}}%
\pgfpathcurveto{\pgfqpoint{6.876887in}{4.587417in}}{\pgfqpoint{6.887486in}{4.591808in}}{\pgfqpoint{6.895300in}{4.599621in}}%
\pgfpathcurveto{\pgfqpoint{6.903113in}{4.607435in}}{\pgfqpoint{6.907504in}{4.618034in}}{\pgfqpoint{6.907504in}{4.629084in}}%
\pgfpathcurveto{\pgfqpoint{6.907504in}{4.640134in}}{\pgfqpoint{6.903113in}{4.650733in}}{\pgfqpoint{6.895300in}{4.658547in}}%
\pgfpathcurveto{\pgfqpoint{6.887486in}{4.666360in}}{\pgfqpoint{6.876887in}{4.670751in}}{\pgfqpoint{6.865837in}{4.670751in}}%
\pgfpathcurveto{\pgfqpoint{6.854787in}{4.670751in}}{\pgfqpoint{6.844188in}{4.666360in}}{\pgfqpoint{6.836374in}{4.658547in}}%
\pgfpathcurveto{\pgfqpoint{6.828561in}{4.650733in}}{\pgfqpoint{6.824170in}{4.640134in}}{\pgfqpoint{6.824170in}{4.629084in}}%
\pgfpathcurveto{\pgfqpoint{6.824170in}{4.618034in}}{\pgfqpoint{6.828561in}{4.607435in}}{\pgfqpoint{6.836374in}{4.599621in}}%
\pgfpathcurveto{\pgfqpoint{6.844188in}{4.591808in}}{\pgfqpoint{6.854787in}{4.587417in}}{\pgfqpoint{6.865837in}{4.587417in}}%
\pgfpathclose%
\pgfusepath{stroke,fill}%
\end{pgfscope}%
\begin{pgfscope}%
\pgfpathrectangle{\pgfqpoint{0.526127in}{0.331635in}}{\pgfqpoint{9.300000in}{7.700000in}}%
\pgfusepath{clip}%
\pgfsetbuttcap%
\pgfsetroundjoin%
\definecolor{currentfill}{rgb}{0.631373,0.788235,0.956863}%
\pgfsetfillcolor{currentfill}%
\pgfsetlinewidth{0.481800pt}%
\definecolor{currentstroke}{rgb}{1.000000,1.000000,1.000000}%
\pgfsetstrokecolor{currentstroke}%
\pgfsetdash{}{0pt}%
\pgfpathmoveto{\pgfqpoint{7.703805in}{3.254938in}}%
\pgfpathcurveto{\pgfqpoint{7.714856in}{3.254938in}}{\pgfqpoint{7.725455in}{3.259328in}}{\pgfqpoint{7.733268in}{3.267142in}}%
\pgfpathcurveto{\pgfqpoint{7.741082in}{3.274956in}}{\pgfqpoint{7.745472in}{3.285555in}}{\pgfqpoint{7.745472in}{3.296605in}}%
\pgfpathcurveto{\pgfqpoint{7.745472in}{3.307655in}}{\pgfqpoint{7.741082in}{3.318254in}}{\pgfqpoint{7.733268in}{3.326068in}}%
\pgfpathcurveto{\pgfqpoint{7.725455in}{3.333881in}}{\pgfqpoint{7.714856in}{3.338271in}}{\pgfqpoint{7.703805in}{3.338271in}}%
\pgfpathcurveto{\pgfqpoint{7.692755in}{3.338271in}}{\pgfqpoint{7.682156in}{3.333881in}}{\pgfqpoint{7.674343in}{3.326068in}}%
\pgfpathcurveto{\pgfqpoint{7.666529in}{3.318254in}}{\pgfqpoint{7.662139in}{3.307655in}}{\pgfqpoint{7.662139in}{3.296605in}}%
\pgfpathcurveto{\pgfqpoint{7.662139in}{3.285555in}}{\pgfqpoint{7.666529in}{3.274956in}}{\pgfqpoint{7.674343in}{3.267142in}}%
\pgfpathcurveto{\pgfqpoint{7.682156in}{3.259328in}}{\pgfqpoint{7.692755in}{3.254938in}}{\pgfqpoint{7.703805in}{3.254938in}}%
\pgfpathclose%
\pgfusepath{stroke,fill}%
\end{pgfscope}%
\begin{pgfscope}%
\pgfpathrectangle{\pgfqpoint{0.526127in}{0.331635in}}{\pgfqpoint{9.300000in}{7.700000in}}%
\pgfusepath{clip}%
\pgfsetbuttcap%
\pgfsetroundjoin%
\definecolor{currentfill}{rgb}{0.631373,0.788235,0.956863}%
\pgfsetfillcolor{currentfill}%
\pgfsetlinewidth{0.481800pt}%
\definecolor{currentstroke}{rgb}{1.000000,1.000000,1.000000}%
\pgfsetstrokecolor{currentstroke}%
\pgfsetdash{}{0pt}%
\pgfpathmoveto{\pgfqpoint{2.458995in}{0.988152in}}%
\pgfpathcurveto{\pgfqpoint{2.470045in}{0.988152in}}{\pgfqpoint{2.480644in}{0.992543in}}{\pgfqpoint{2.488458in}{1.000356in}}%
\pgfpathcurveto{\pgfqpoint{2.496271in}{1.008170in}}{\pgfqpoint{2.500662in}{1.018769in}}{\pgfqpoint{2.500662in}{1.029819in}}%
\pgfpathcurveto{\pgfqpoint{2.500662in}{1.040869in}}{\pgfqpoint{2.496271in}{1.051468in}}{\pgfqpoint{2.488458in}{1.059282in}}%
\pgfpathcurveto{\pgfqpoint{2.480644in}{1.067095in}}{\pgfqpoint{2.470045in}{1.071486in}}{\pgfqpoint{2.458995in}{1.071486in}}%
\pgfpathcurveto{\pgfqpoint{2.447945in}{1.071486in}}{\pgfqpoint{2.437346in}{1.067095in}}{\pgfqpoint{2.429532in}{1.059282in}}%
\pgfpathcurveto{\pgfqpoint{2.421719in}{1.051468in}}{\pgfqpoint{2.417328in}{1.040869in}}{\pgfqpoint{2.417328in}{1.029819in}}%
\pgfpathcurveto{\pgfqpoint{2.417328in}{1.018769in}}{\pgfqpoint{2.421719in}{1.008170in}}{\pgfqpoint{2.429532in}{1.000356in}}%
\pgfpathcurveto{\pgfqpoint{2.437346in}{0.992543in}}{\pgfqpoint{2.447945in}{0.988152in}}{\pgfqpoint{2.458995in}{0.988152in}}%
\pgfpathclose%
\pgfusepath{stroke,fill}%
\end{pgfscope}%
\begin{pgfscope}%
\pgfpathrectangle{\pgfqpoint{0.526127in}{0.331635in}}{\pgfqpoint{9.300000in}{7.700000in}}%
\pgfusepath{clip}%
\pgfsetbuttcap%
\pgfsetroundjoin%
\definecolor{currentfill}{rgb}{0.631373,0.788235,0.956863}%
\pgfsetfillcolor{currentfill}%
\pgfsetlinewidth{0.481800pt}%
\definecolor{currentstroke}{rgb}{1.000000,1.000000,1.000000}%
\pgfsetstrokecolor{currentstroke}%
\pgfsetdash{}{0pt}%
\pgfpathmoveto{\pgfqpoint{4.621174in}{4.016137in}}%
\pgfpathcurveto{\pgfqpoint{4.632224in}{4.016137in}}{\pgfqpoint{4.642823in}{4.020527in}}{\pgfqpoint{4.650636in}{4.028341in}}%
\pgfpathcurveto{\pgfqpoint{4.658450in}{4.036155in}}{\pgfqpoint{4.662840in}{4.046754in}}{\pgfqpoint{4.662840in}{4.057804in}}%
\pgfpathcurveto{\pgfqpoint{4.662840in}{4.068854in}}{\pgfqpoint{4.658450in}{4.079453in}}{\pgfqpoint{4.650636in}{4.087266in}}%
\pgfpathcurveto{\pgfqpoint{4.642823in}{4.095080in}}{\pgfqpoint{4.632224in}{4.099470in}}{\pgfqpoint{4.621174in}{4.099470in}}%
\pgfpathcurveto{\pgfqpoint{4.610123in}{4.099470in}}{\pgfqpoint{4.599524in}{4.095080in}}{\pgfqpoint{4.591711in}{4.087266in}}%
\pgfpathcurveto{\pgfqpoint{4.583897in}{4.079453in}}{\pgfqpoint{4.579507in}{4.068854in}}{\pgfqpoint{4.579507in}{4.057804in}}%
\pgfpathcurveto{\pgfqpoint{4.579507in}{4.046754in}}{\pgfqpoint{4.583897in}{4.036155in}}{\pgfqpoint{4.591711in}{4.028341in}}%
\pgfpathcurveto{\pgfqpoint{4.599524in}{4.020527in}}{\pgfqpoint{4.610123in}{4.016137in}}{\pgfqpoint{4.621174in}{4.016137in}}%
\pgfpathclose%
\pgfusepath{stroke,fill}%
\end{pgfscope}%
\begin{pgfscope}%
\pgfpathrectangle{\pgfqpoint{0.526127in}{0.331635in}}{\pgfqpoint{9.300000in}{7.700000in}}%
\pgfusepath{clip}%
\pgfsetbuttcap%
\pgfsetroundjoin%
\definecolor{currentfill}{rgb}{0.631373,0.788235,0.956863}%
\pgfsetfillcolor{currentfill}%
\pgfsetlinewidth{0.481800pt}%
\definecolor{currentstroke}{rgb}{1.000000,1.000000,1.000000}%
\pgfsetstrokecolor{currentstroke}%
\pgfsetdash{}{0pt}%
\pgfpathmoveto{\pgfqpoint{5.970043in}{6.856085in}}%
\pgfpathcurveto{\pgfqpoint{5.981093in}{6.856085in}}{\pgfqpoint{5.991692in}{6.860475in}}{\pgfqpoint{5.999505in}{6.868289in}}%
\pgfpathcurveto{\pgfqpoint{6.007319in}{6.876102in}}{\pgfqpoint{6.011709in}{6.886701in}}{\pgfqpoint{6.011709in}{6.897752in}}%
\pgfpathcurveto{\pgfqpoint{6.011709in}{6.908802in}}{\pgfqpoint{6.007319in}{6.919401in}}{\pgfqpoint{5.999505in}{6.927214in}}%
\pgfpathcurveto{\pgfqpoint{5.991692in}{6.935028in}}{\pgfqpoint{5.981093in}{6.939418in}}{\pgfqpoint{5.970043in}{6.939418in}}%
\pgfpathcurveto{\pgfqpoint{5.958992in}{6.939418in}}{\pgfqpoint{5.948393in}{6.935028in}}{\pgfqpoint{5.940580in}{6.927214in}}%
\pgfpathcurveto{\pgfqpoint{5.932766in}{6.919401in}}{\pgfqpoint{5.928376in}{6.908802in}}{\pgfqpoint{5.928376in}{6.897752in}}%
\pgfpathcurveto{\pgfqpoint{5.928376in}{6.886701in}}{\pgfqpoint{5.932766in}{6.876102in}}{\pgfqpoint{5.940580in}{6.868289in}}%
\pgfpathcurveto{\pgfqpoint{5.948393in}{6.860475in}}{\pgfqpoint{5.958992in}{6.856085in}}{\pgfqpoint{5.970043in}{6.856085in}}%
\pgfpathclose%
\pgfusepath{stroke,fill}%
\end{pgfscope}%
\begin{pgfscope}%
\pgfpathrectangle{\pgfqpoint{0.526127in}{0.331635in}}{\pgfqpoint{9.300000in}{7.700000in}}%
\pgfusepath{clip}%
\pgfsetbuttcap%
\pgfsetroundjoin%
\definecolor{currentfill}{rgb}{0.631373,0.788235,0.956863}%
\pgfsetfillcolor{currentfill}%
\pgfsetlinewidth{0.481800pt}%
\definecolor{currentstroke}{rgb}{1.000000,1.000000,1.000000}%
\pgfsetstrokecolor{currentstroke}%
\pgfsetdash{}{0pt}%
\pgfpathmoveto{\pgfqpoint{5.368377in}{3.368977in}}%
\pgfpathcurveto{\pgfqpoint{5.379427in}{3.368977in}}{\pgfqpoint{5.390026in}{3.373367in}}{\pgfqpoint{5.397840in}{3.381181in}}%
\pgfpathcurveto{\pgfqpoint{5.405654in}{3.388994in}}{\pgfqpoint{5.410044in}{3.399593in}}{\pgfqpoint{5.410044in}{3.410644in}}%
\pgfpathcurveto{\pgfqpoint{5.410044in}{3.421694in}}{\pgfqpoint{5.405654in}{3.432293in}}{\pgfqpoint{5.397840in}{3.440106in}}%
\pgfpathcurveto{\pgfqpoint{5.390026in}{3.447920in}}{\pgfqpoint{5.379427in}{3.452310in}}{\pgfqpoint{5.368377in}{3.452310in}}%
\pgfpathcurveto{\pgfqpoint{5.357327in}{3.452310in}}{\pgfqpoint{5.346728in}{3.447920in}}{\pgfqpoint{5.338915in}{3.440106in}}%
\pgfpathcurveto{\pgfqpoint{5.331101in}{3.432293in}}{\pgfqpoint{5.326711in}{3.421694in}}{\pgfqpoint{5.326711in}{3.410644in}}%
\pgfpathcurveto{\pgfqpoint{5.326711in}{3.399593in}}{\pgfqpoint{5.331101in}{3.388994in}}{\pgfqpoint{5.338915in}{3.381181in}}%
\pgfpathcurveto{\pgfqpoint{5.346728in}{3.373367in}}{\pgfqpoint{5.357327in}{3.368977in}}{\pgfqpoint{5.368377in}{3.368977in}}%
\pgfpathclose%
\pgfusepath{stroke,fill}%
\end{pgfscope}%
\begin{pgfscope}%
\pgfpathrectangle{\pgfqpoint{0.526127in}{0.331635in}}{\pgfqpoint{9.300000in}{7.700000in}}%
\pgfusepath{clip}%
\pgfsetbuttcap%
\pgfsetroundjoin%
\definecolor{currentfill}{rgb}{0.631373,0.788235,0.956863}%
\pgfsetfillcolor{currentfill}%
\pgfsetlinewidth{0.481800pt}%
\definecolor{currentstroke}{rgb}{1.000000,1.000000,1.000000}%
\pgfsetstrokecolor{currentstroke}%
\pgfsetdash{}{0pt}%
\pgfpathmoveto{\pgfqpoint{6.364240in}{6.562434in}}%
\pgfpathcurveto{\pgfqpoint{6.375290in}{6.562434in}}{\pgfqpoint{6.385889in}{6.566825in}}{\pgfqpoint{6.393703in}{6.574638in}}%
\pgfpathcurveto{\pgfqpoint{6.401517in}{6.582452in}}{\pgfqpoint{6.405907in}{6.593051in}}{\pgfqpoint{6.405907in}{6.604101in}}%
\pgfpathcurveto{\pgfqpoint{6.405907in}{6.615151in}}{\pgfqpoint{6.401517in}{6.625750in}}{\pgfqpoint{6.393703in}{6.633564in}}%
\pgfpathcurveto{\pgfqpoint{6.385889in}{6.641378in}}{\pgfqpoint{6.375290in}{6.645768in}}{\pgfqpoint{6.364240in}{6.645768in}}%
\pgfpathcurveto{\pgfqpoint{6.353190in}{6.645768in}}{\pgfqpoint{6.342591in}{6.641378in}}{\pgfqpoint{6.334778in}{6.633564in}}%
\pgfpathcurveto{\pgfqpoint{6.326964in}{6.625750in}}{\pgfqpoint{6.322574in}{6.615151in}}{\pgfqpoint{6.322574in}{6.604101in}}%
\pgfpathcurveto{\pgfqpoint{6.322574in}{6.593051in}}{\pgfqpoint{6.326964in}{6.582452in}}{\pgfqpoint{6.334778in}{6.574638in}}%
\pgfpathcurveto{\pgfqpoint{6.342591in}{6.566825in}}{\pgfqpoint{6.353190in}{6.562434in}}{\pgfqpoint{6.364240in}{6.562434in}}%
\pgfpathclose%
\pgfusepath{stroke,fill}%
\end{pgfscope}%
\begin{pgfscope}%
\pgfpathrectangle{\pgfqpoint{0.526127in}{0.331635in}}{\pgfqpoint{9.300000in}{7.700000in}}%
\pgfusepath{clip}%
\pgfsetbuttcap%
\pgfsetroundjoin%
\definecolor{currentfill}{rgb}{0.631373,0.788235,0.956863}%
\pgfsetfillcolor{currentfill}%
\pgfsetlinewidth{0.481800pt}%
\definecolor{currentstroke}{rgb}{1.000000,1.000000,1.000000}%
\pgfsetstrokecolor{currentstroke}%
\pgfsetdash{}{0pt}%
\pgfpathmoveto{\pgfqpoint{6.258495in}{6.773269in}}%
\pgfpathcurveto{\pgfqpoint{6.269546in}{6.773269in}}{\pgfqpoint{6.280145in}{6.777659in}}{\pgfqpoint{6.287958in}{6.785472in}}%
\pgfpathcurveto{\pgfqpoint{6.295772in}{6.793286in}}{\pgfqpoint{6.300162in}{6.803885in}}{\pgfqpoint{6.300162in}{6.814935in}}%
\pgfpathcurveto{\pgfqpoint{6.300162in}{6.825985in}}{\pgfqpoint{6.295772in}{6.836584in}}{\pgfqpoint{6.287958in}{6.844398in}}%
\pgfpathcurveto{\pgfqpoint{6.280145in}{6.852212in}}{\pgfqpoint{6.269546in}{6.856602in}}{\pgfqpoint{6.258495in}{6.856602in}}%
\pgfpathcurveto{\pgfqpoint{6.247445in}{6.856602in}}{\pgfqpoint{6.236846in}{6.852212in}}{\pgfqpoint{6.229033in}{6.844398in}}%
\pgfpathcurveto{\pgfqpoint{6.221219in}{6.836584in}}{\pgfqpoint{6.216829in}{6.825985in}}{\pgfqpoint{6.216829in}{6.814935in}}%
\pgfpathcurveto{\pgfqpoint{6.216829in}{6.803885in}}{\pgfqpoint{6.221219in}{6.793286in}}{\pgfqpoint{6.229033in}{6.785472in}}%
\pgfpathcurveto{\pgfqpoint{6.236846in}{6.777659in}}{\pgfqpoint{6.247445in}{6.773269in}}{\pgfqpoint{6.258495in}{6.773269in}}%
\pgfpathclose%
\pgfusepath{stroke,fill}%
\end{pgfscope}%
\begin{pgfscope}%
\pgfpathrectangle{\pgfqpoint{0.526127in}{0.331635in}}{\pgfqpoint{9.300000in}{7.700000in}}%
\pgfusepath{clip}%
\pgfsetbuttcap%
\pgfsetroundjoin%
\definecolor{currentfill}{rgb}{0.631373,0.788235,0.956863}%
\pgfsetfillcolor{currentfill}%
\pgfsetlinewidth{0.481800pt}%
\definecolor{currentstroke}{rgb}{1.000000,1.000000,1.000000}%
\pgfsetstrokecolor{currentstroke}%
\pgfsetdash{}{0pt}%
\pgfpathmoveto{\pgfqpoint{7.197145in}{4.652940in}}%
\pgfpathcurveto{\pgfqpoint{7.208195in}{4.652940in}}{\pgfqpoint{7.218794in}{4.657330in}}{\pgfqpoint{7.226607in}{4.665144in}}%
\pgfpathcurveto{\pgfqpoint{7.234421in}{4.672958in}}{\pgfqpoint{7.238811in}{4.683557in}}{\pgfqpoint{7.238811in}{4.694607in}}%
\pgfpathcurveto{\pgfqpoint{7.238811in}{4.705657in}}{\pgfqpoint{7.234421in}{4.716256in}}{\pgfqpoint{7.226607in}{4.724069in}}%
\pgfpathcurveto{\pgfqpoint{7.218794in}{4.731883in}}{\pgfqpoint{7.208195in}{4.736273in}}{\pgfqpoint{7.197145in}{4.736273in}}%
\pgfpathcurveto{\pgfqpoint{7.186095in}{4.736273in}}{\pgfqpoint{7.175496in}{4.731883in}}{\pgfqpoint{7.167682in}{4.724069in}}%
\pgfpathcurveto{\pgfqpoint{7.159868in}{4.716256in}}{\pgfqpoint{7.155478in}{4.705657in}}{\pgfqpoint{7.155478in}{4.694607in}}%
\pgfpathcurveto{\pgfqpoint{7.155478in}{4.683557in}}{\pgfqpoint{7.159868in}{4.672958in}}{\pgfqpoint{7.167682in}{4.665144in}}%
\pgfpathcurveto{\pgfqpoint{7.175496in}{4.657330in}}{\pgfqpoint{7.186095in}{4.652940in}}{\pgfqpoint{7.197145in}{4.652940in}}%
\pgfpathclose%
\pgfusepath{stroke,fill}%
\end{pgfscope}%
\begin{pgfscope}%
\pgfpathrectangle{\pgfqpoint{0.526127in}{0.331635in}}{\pgfqpoint{9.300000in}{7.700000in}}%
\pgfusepath{clip}%
\pgfsetbuttcap%
\pgfsetroundjoin%
\definecolor{currentfill}{rgb}{0.631373,0.788235,0.956863}%
\pgfsetfillcolor{currentfill}%
\pgfsetlinewidth{0.481800pt}%
\definecolor{currentstroke}{rgb}{1.000000,1.000000,1.000000}%
\pgfsetstrokecolor{currentstroke}%
\pgfsetdash{}{0pt}%
\pgfpathmoveto{\pgfqpoint{7.051687in}{6.205810in}}%
\pgfpathcurveto{\pgfqpoint{7.062738in}{6.205810in}}{\pgfqpoint{7.073337in}{6.210200in}}{\pgfqpoint{7.081150in}{6.218013in}}%
\pgfpathcurveto{\pgfqpoint{7.088964in}{6.225827in}}{\pgfqpoint{7.093354in}{6.236426in}}{\pgfqpoint{7.093354in}{6.247476in}}%
\pgfpathcurveto{\pgfqpoint{7.093354in}{6.258526in}}{\pgfqpoint{7.088964in}{6.269125in}}{\pgfqpoint{7.081150in}{6.276939in}}%
\pgfpathcurveto{\pgfqpoint{7.073337in}{6.284753in}}{\pgfqpoint{7.062738in}{6.289143in}}{\pgfqpoint{7.051687in}{6.289143in}}%
\pgfpathcurveto{\pgfqpoint{7.040637in}{6.289143in}}{\pgfqpoint{7.030038in}{6.284753in}}{\pgfqpoint{7.022225in}{6.276939in}}%
\pgfpathcurveto{\pgfqpoint{7.014411in}{6.269125in}}{\pgfqpoint{7.010021in}{6.258526in}}{\pgfqpoint{7.010021in}{6.247476in}}%
\pgfpathcurveto{\pgfqpoint{7.010021in}{6.236426in}}{\pgfqpoint{7.014411in}{6.225827in}}{\pgfqpoint{7.022225in}{6.218013in}}%
\pgfpathcurveto{\pgfqpoint{7.030038in}{6.210200in}}{\pgfqpoint{7.040637in}{6.205810in}}{\pgfqpoint{7.051687in}{6.205810in}}%
\pgfpathclose%
\pgfusepath{stroke,fill}%
\end{pgfscope}%
\begin{pgfscope}%
\pgfpathrectangle{\pgfqpoint{0.526127in}{0.331635in}}{\pgfqpoint{9.300000in}{7.700000in}}%
\pgfusepath{clip}%
\pgfsetbuttcap%
\pgfsetroundjoin%
\definecolor{currentfill}{rgb}{0.631373,0.788235,0.956863}%
\pgfsetfillcolor{currentfill}%
\pgfsetlinewidth{0.481800pt}%
\definecolor{currentstroke}{rgb}{1.000000,1.000000,1.000000}%
\pgfsetstrokecolor{currentstroke}%
\pgfsetdash{}{0pt}%
\pgfpathmoveto{\pgfqpoint{6.775438in}{7.196897in}}%
\pgfpathcurveto{\pgfqpoint{6.786488in}{7.196897in}}{\pgfqpoint{6.797087in}{7.201287in}}{\pgfqpoint{6.804901in}{7.209101in}}%
\pgfpathcurveto{\pgfqpoint{6.812715in}{7.216914in}}{\pgfqpoint{6.817105in}{7.227513in}}{\pgfqpoint{6.817105in}{7.238564in}}%
\pgfpathcurveto{\pgfqpoint{6.817105in}{7.249614in}}{\pgfqpoint{6.812715in}{7.260213in}}{\pgfqpoint{6.804901in}{7.268026in}}%
\pgfpathcurveto{\pgfqpoint{6.797087in}{7.275840in}}{\pgfqpoint{6.786488in}{7.280230in}}{\pgfqpoint{6.775438in}{7.280230in}}%
\pgfpathcurveto{\pgfqpoint{6.764388in}{7.280230in}}{\pgfqpoint{6.753789in}{7.275840in}}{\pgfqpoint{6.745975in}{7.268026in}}%
\pgfpathcurveto{\pgfqpoint{6.738162in}{7.260213in}}{\pgfqpoint{6.733772in}{7.249614in}}{\pgfqpoint{6.733772in}{7.238564in}}%
\pgfpathcurveto{\pgfqpoint{6.733772in}{7.227513in}}{\pgfqpoint{6.738162in}{7.216914in}}{\pgfqpoint{6.745975in}{7.209101in}}%
\pgfpathcurveto{\pgfqpoint{6.753789in}{7.201287in}}{\pgfqpoint{6.764388in}{7.196897in}}{\pgfqpoint{6.775438in}{7.196897in}}%
\pgfpathclose%
\pgfusepath{stroke,fill}%
\end{pgfscope}%
\begin{pgfscope}%
\pgfpathrectangle{\pgfqpoint{0.526127in}{0.331635in}}{\pgfqpoint{9.300000in}{7.700000in}}%
\pgfusepath{clip}%
\pgfsetbuttcap%
\pgfsetroundjoin%
\definecolor{currentfill}{rgb}{0.631373,0.788235,0.956863}%
\pgfsetfillcolor{currentfill}%
\pgfsetlinewidth{0.481800pt}%
\definecolor{currentstroke}{rgb}{1.000000,1.000000,1.000000}%
\pgfsetstrokecolor{currentstroke}%
\pgfsetdash{}{0pt}%
\pgfpathmoveto{\pgfqpoint{4.179928in}{4.064430in}}%
\pgfpathcurveto{\pgfqpoint{4.190978in}{4.064430in}}{\pgfqpoint{4.201577in}{4.068820in}}{\pgfqpoint{4.209391in}{4.076634in}}%
\pgfpathcurveto{\pgfqpoint{4.217204in}{4.084447in}}{\pgfqpoint{4.221595in}{4.095046in}}{\pgfqpoint{4.221595in}{4.106096in}}%
\pgfpathcurveto{\pgfqpoint{4.221595in}{4.117147in}}{\pgfqpoint{4.217204in}{4.127746in}}{\pgfqpoint{4.209391in}{4.135559in}}%
\pgfpathcurveto{\pgfqpoint{4.201577in}{4.143373in}}{\pgfqpoint{4.190978in}{4.147763in}}{\pgfqpoint{4.179928in}{4.147763in}}%
\pgfpathcurveto{\pgfqpoint{4.168878in}{4.147763in}}{\pgfqpoint{4.158279in}{4.143373in}}{\pgfqpoint{4.150465in}{4.135559in}}%
\pgfpathcurveto{\pgfqpoint{4.142652in}{4.127746in}}{\pgfqpoint{4.138261in}{4.117147in}}{\pgfqpoint{4.138261in}{4.106096in}}%
\pgfpathcurveto{\pgfqpoint{4.138261in}{4.095046in}}{\pgfqpoint{4.142652in}{4.084447in}}{\pgfqpoint{4.150465in}{4.076634in}}%
\pgfpathcurveto{\pgfqpoint{4.158279in}{4.068820in}}{\pgfqpoint{4.168878in}{4.064430in}}{\pgfqpoint{4.179928in}{4.064430in}}%
\pgfpathclose%
\pgfusepath{stroke,fill}%
\end{pgfscope}%
\begin{pgfscope}%
\pgfpathrectangle{\pgfqpoint{0.526127in}{0.331635in}}{\pgfqpoint{9.300000in}{7.700000in}}%
\pgfusepath{clip}%
\pgfsetbuttcap%
\pgfsetroundjoin%
\definecolor{currentfill}{rgb}{0.631373,0.788235,0.956863}%
\pgfsetfillcolor{currentfill}%
\pgfsetlinewidth{0.481800pt}%
\definecolor{currentstroke}{rgb}{1.000000,1.000000,1.000000}%
\pgfsetstrokecolor{currentstroke}%
\pgfsetdash{}{0pt}%
\pgfpathmoveto{\pgfqpoint{8.098361in}{3.028530in}}%
\pgfpathcurveto{\pgfqpoint{8.109412in}{3.028530in}}{\pgfqpoint{8.120011in}{3.032920in}}{\pgfqpoint{8.127824in}{3.040734in}}%
\pgfpathcurveto{\pgfqpoint{8.135638in}{3.048548in}}{\pgfqpoint{8.140028in}{3.059147in}}{\pgfqpoint{8.140028in}{3.070197in}}%
\pgfpathcurveto{\pgfqpoint{8.140028in}{3.081247in}}{\pgfqpoint{8.135638in}{3.091846in}}{\pgfqpoint{8.127824in}{3.099660in}}%
\pgfpathcurveto{\pgfqpoint{8.120011in}{3.107473in}}{\pgfqpoint{8.109412in}{3.111864in}}{\pgfqpoint{8.098361in}{3.111864in}}%
\pgfpathcurveto{\pgfqpoint{8.087311in}{3.111864in}}{\pgfqpoint{8.076712in}{3.107473in}}{\pgfqpoint{8.068899in}{3.099660in}}%
\pgfpathcurveto{\pgfqpoint{8.061085in}{3.091846in}}{\pgfqpoint{8.056695in}{3.081247in}}{\pgfqpoint{8.056695in}{3.070197in}}%
\pgfpathcurveto{\pgfqpoint{8.056695in}{3.059147in}}{\pgfqpoint{8.061085in}{3.048548in}}{\pgfqpoint{8.068899in}{3.040734in}}%
\pgfpathcurveto{\pgfqpoint{8.076712in}{3.032920in}}{\pgfqpoint{8.087311in}{3.028530in}}{\pgfqpoint{8.098361in}{3.028530in}}%
\pgfpathclose%
\pgfusepath{stroke,fill}%
\end{pgfscope}%
\begin{pgfscope}%
\pgfpathrectangle{\pgfqpoint{0.526127in}{0.331635in}}{\pgfqpoint{9.300000in}{7.700000in}}%
\pgfusepath{clip}%
\pgfsetbuttcap%
\pgfsetroundjoin%
\definecolor{currentfill}{rgb}{1.000000,0.705882,0.509804}%
\pgfsetfillcolor{currentfill}%
\pgfsetlinewidth{0.481800pt}%
\definecolor{currentstroke}{rgb}{1.000000,1.000000,1.000000}%
\pgfsetstrokecolor{currentstroke}%
\pgfsetdash{}{0pt}%
\pgfpathmoveto{\pgfqpoint{6.611324in}{4.087918in}}%
\pgfpathcurveto{\pgfqpoint{6.622374in}{4.087918in}}{\pgfqpoint{6.632973in}{4.092308in}}{\pgfqpoint{6.640787in}{4.100122in}}%
\pgfpathcurveto{\pgfqpoint{6.648601in}{4.107935in}}{\pgfqpoint{6.652991in}{4.118534in}}{\pgfqpoint{6.652991in}{4.129584in}}%
\pgfpathcurveto{\pgfqpoint{6.652991in}{4.140634in}}{\pgfqpoint{6.648601in}{4.151234in}}{\pgfqpoint{6.640787in}{4.159047in}}%
\pgfpathcurveto{\pgfqpoint{6.632973in}{4.166861in}}{\pgfqpoint{6.622374in}{4.171251in}}{\pgfqpoint{6.611324in}{4.171251in}}%
\pgfpathcurveto{\pgfqpoint{6.600274in}{4.171251in}}{\pgfqpoint{6.589675in}{4.166861in}}{\pgfqpoint{6.581861in}{4.159047in}}%
\pgfpathcurveto{\pgfqpoint{6.574048in}{4.151234in}}{\pgfqpoint{6.569657in}{4.140634in}}{\pgfqpoint{6.569657in}{4.129584in}}%
\pgfpathcurveto{\pgfqpoint{6.569657in}{4.118534in}}{\pgfqpoint{6.574048in}{4.107935in}}{\pgfqpoint{6.581861in}{4.100122in}}%
\pgfpathcurveto{\pgfqpoint{6.589675in}{4.092308in}}{\pgfqpoint{6.600274in}{4.087918in}}{\pgfqpoint{6.611324in}{4.087918in}}%
\pgfpathclose%
\pgfusepath{stroke,fill}%
\end{pgfscope}%
\begin{pgfscope}%
\pgfpathrectangle{\pgfqpoint{0.526127in}{0.331635in}}{\pgfqpoint{9.300000in}{7.700000in}}%
\pgfusepath{clip}%
\pgfsetbuttcap%
\pgfsetroundjoin%
\definecolor{currentfill}{rgb}{1.000000,0.705882,0.509804}%
\pgfsetfillcolor{currentfill}%
\pgfsetlinewidth{0.481800pt}%
\definecolor{currentstroke}{rgb}{1.000000,1.000000,1.000000}%
\pgfsetstrokecolor{currentstroke}%
\pgfsetdash{}{0pt}%
\pgfpathmoveto{\pgfqpoint{4.265030in}{6.967810in}}%
\pgfpathcurveto{\pgfqpoint{4.276080in}{6.967810in}}{\pgfqpoint{4.286679in}{6.972200in}}{\pgfqpoint{4.294493in}{6.980014in}}%
\pgfpathcurveto{\pgfqpoint{4.302306in}{6.987827in}}{\pgfqpoint{4.306697in}{6.998426in}}{\pgfqpoint{4.306697in}{7.009476in}}%
\pgfpathcurveto{\pgfqpoint{4.306697in}{7.020527in}}{\pgfqpoint{4.302306in}{7.031126in}}{\pgfqpoint{4.294493in}{7.038939in}}%
\pgfpathcurveto{\pgfqpoint{4.286679in}{7.046753in}}{\pgfqpoint{4.276080in}{7.051143in}}{\pgfqpoint{4.265030in}{7.051143in}}%
\pgfpathcurveto{\pgfqpoint{4.253980in}{7.051143in}}{\pgfqpoint{4.243381in}{7.046753in}}{\pgfqpoint{4.235567in}{7.038939in}}%
\pgfpathcurveto{\pgfqpoint{4.227753in}{7.031126in}}{\pgfqpoint{4.223363in}{7.020527in}}{\pgfqpoint{4.223363in}{7.009476in}}%
\pgfpathcurveto{\pgfqpoint{4.223363in}{6.998426in}}{\pgfqpoint{4.227753in}{6.987827in}}{\pgfqpoint{4.235567in}{6.980014in}}%
\pgfpathcurveto{\pgfqpoint{4.243381in}{6.972200in}}{\pgfqpoint{4.253980in}{6.967810in}}{\pgfqpoint{4.265030in}{6.967810in}}%
\pgfpathclose%
\pgfusepath{stroke,fill}%
\end{pgfscope}%
\begin{pgfscope}%
\pgfpathrectangle{\pgfqpoint{0.526127in}{0.331635in}}{\pgfqpoint{9.300000in}{7.700000in}}%
\pgfusepath{clip}%
\pgfsetbuttcap%
\pgfsetroundjoin%
\definecolor{currentfill}{rgb}{1.000000,0.705882,0.509804}%
\pgfsetfillcolor{currentfill}%
\pgfsetlinewidth{0.481800pt}%
\definecolor{currentstroke}{rgb}{1.000000,1.000000,1.000000}%
\pgfsetstrokecolor{currentstroke}%
\pgfsetdash{}{0pt}%
\pgfpathmoveto{\pgfqpoint{5.530805in}{6.526749in}}%
\pgfpathcurveto{\pgfqpoint{5.541855in}{6.526749in}}{\pgfqpoint{5.552454in}{6.531140in}}{\pgfqpoint{5.560268in}{6.538953in}}%
\pgfpathcurveto{\pgfqpoint{5.568082in}{6.546767in}}{\pgfqpoint{5.572472in}{6.557366in}}{\pgfqpoint{5.572472in}{6.568416in}}%
\pgfpathcurveto{\pgfqpoint{5.572472in}{6.579466in}}{\pgfqpoint{5.568082in}{6.590065in}}{\pgfqpoint{5.560268in}{6.597879in}}%
\pgfpathcurveto{\pgfqpoint{5.552454in}{6.605692in}}{\pgfqpoint{5.541855in}{6.610083in}}{\pgfqpoint{5.530805in}{6.610083in}}%
\pgfpathcurveto{\pgfqpoint{5.519755in}{6.610083in}}{\pgfqpoint{5.509156in}{6.605692in}}{\pgfqpoint{5.501342in}{6.597879in}}%
\pgfpathcurveto{\pgfqpoint{5.493529in}{6.590065in}}{\pgfqpoint{5.489139in}{6.579466in}}{\pgfqpoint{5.489139in}{6.568416in}}%
\pgfpathcurveto{\pgfqpoint{5.489139in}{6.557366in}}{\pgfqpoint{5.493529in}{6.546767in}}{\pgfqpoint{5.501342in}{6.538953in}}%
\pgfpathcurveto{\pgfqpoint{5.509156in}{6.531140in}}{\pgfqpoint{5.519755in}{6.526749in}}{\pgfqpoint{5.530805in}{6.526749in}}%
\pgfpathclose%
\pgfusepath{stroke,fill}%
\end{pgfscope}%
\begin{pgfscope}%
\pgfpathrectangle{\pgfqpoint{0.526127in}{0.331635in}}{\pgfqpoint{9.300000in}{7.700000in}}%
\pgfusepath{clip}%
\pgfsetbuttcap%
\pgfsetroundjoin%
\definecolor{currentfill}{rgb}{1.000000,0.705882,0.509804}%
\pgfsetfillcolor{currentfill}%
\pgfsetlinewidth{0.481800pt}%
\definecolor{currentstroke}{rgb}{1.000000,1.000000,1.000000}%
\pgfsetstrokecolor{currentstroke}%
\pgfsetdash{}{0pt}%
\pgfpathmoveto{\pgfqpoint{6.584340in}{5.691002in}}%
\pgfpathcurveto{\pgfqpoint{6.595390in}{5.691002in}}{\pgfqpoint{6.605989in}{5.695392in}}{\pgfqpoint{6.613803in}{5.703206in}}%
\pgfpathcurveto{\pgfqpoint{6.621616in}{5.711019in}}{\pgfqpoint{6.626007in}{5.721618in}}{\pgfqpoint{6.626007in}{5.732668in}}%
\pgfpathcurveto{\pgfqpoint{6.626007in}{5.743719in}}{\pgfqpoint{6.621616in}{5.754318in}}{\pgfqpoint{6.613803in}{5.762131in}}%
\pgfpathcurveto{\pgfqpoint{6.605989in}{5.769945in}}{\pgfqpoint{6.595390in}{5.774335in}}{\pgfqpoint{6.584340in}{5.774335in}}%
\pgfpathcurveto{\pgfqpoint{6.573290in}{5.774335in}}{\pgfqpoint{6.562691in}{5.769945in}}{\pgfqpoint{6.554877in}{5.762131in}}%
\pgfpathcurveto{\pgfqpoint{6.547064in}{5.754318in}}{\pgfqpoint{6.542673in}{5.743719in}}{\pgfqpoint{6.542673in}{5.732668in}}%
\pgfpathcurveto{\pgfqpoint{6.542673in}{5.721618in}}{\pgfqpoint{6.547064in}{5.711019in}}{\pgfqpoint{6.554877in}{5.703206in}}%
\pgfpathcurveto{\pgfqpoint{6.562691in}{5.695392in}}{\pgfqpoint{6.573290in}{5.691002in}}{\pgfqpoint{6.584340in}{5.691002in}}%
\pgfpathclose%
\pgfusepath{stroke,fill}%
\end{pgfscope}%
\begin{pgfscope}%
\pgfpathrectangle{\pgfqpoint{0.526127in}{0.331635in}}{\pgfqpoint{9.300000in}{7.700000in}}%
\pgfusepath{clip}%
\pgfsetbuttcap%
\pgfsetroundjoin%
\definecolor{currentfill}{rgb}{1.000000,0.705882,0.509804}%
\pgfsetfillcolor{currentfill}%
\pgfsetlinewidth{0.481800pt}%
\definecolor{currentstroke}{rgb}{1.000000,1.000000,1.000000}%
\pgfsetstrokecolor{currentstroke}%
\pgfsetdash{}{0pt}%
\pgfpathmoveto{\pgfqpoint{6.306902in}{2.479853in}}%
\pgfpathcurveto{\pgfqpoint{6.317952in}{2.479853in}}{\pgfqpoint{6.328551in}{2.484244in}}{\pgfqpoint{6.336365in}{2.492057in}}%
\pgfpathcurveto{\pgfqpoint{6.344179in}{2.499871in}}{\pgfqpoint{6.348569in}{2.510470in}}{\pgfqpoint{6.348569in}{2.521520in}}%
\pgfpathcurveto{\pgfqpoint{6.348569in}{2.532570in}}{\pgfqpoint{6.344179in}{2.543169in}}{\pgfqpoint{6.336365in}{2.550983in}}%
\pgfpathcurveto{\pgfqpoint{6.328551in}{2.558797in}}{\pgfqpoint{6.317952in}{2.563187in}}{\pgfqpoint{6.306902in}{2.563187in}}%
\pgfpathcurveto{\pgfqpoint{6.295852in}{2.563187in}}{\pgfqpoint{6.285253in}{2.558797in}}{\pgfqpoint{6.277439in}{2.550983in}}%
\pgfpathcurveto{\pgfqpoint{6.269626in}{2.543169in}}{\pgfqpoint{6.265235in}{2.532570in}}{\pgfqpoint{6.265235in}{2.521520in}}%
\pgfpathcurveto{\pgfqpoint{6.265235in}{2.510470in}}{\pgfqpoint{6.269626in}{2.499871in}}{\pgfqpoint{6.277439in}{2.492057in}}%
\pgfpathcurveto{\pgfqpoint{6.285253in}{2.484244in}}{\pgfqpoint{6.295852in}{2.479853in}}{\pgfqpoint{6.306902in}{2.479853in}}%
\pgfpathclose%
\pgfusepath{stroke,fill}%
\end{pgfscope}%
\begin{pgfscope}%
\pgfpathrectangle{\pgfqpoint{0.526127in}{0.331635in}}{\pgfqpoint{9.300000in}{7.700000in}}%
\pgfusepath{clip}%
\pgfsetbuttcap%
\pgfsetroundjoin%
\definecolor{currentfill}{rgb}{1.000000,0.705882,0.509804}%
\pgfsetfillcolor{currentfill}%
\pgfsetlinewidth{0.481800pt}%
\definecolor{currentstroke}{rgb}{1.000000,1.000000,1.000000}%
\pgfsetstrokecolor{currentstroke}%
\pgfsetdash{}{0pt}%
\pgfpathmoveto{\pgfqpoint{7.529402in}{3.679205in}}%
\pgfpathcurveto{\pgfqpoint{7.540452in}{3.679205in}}{\pgfqpoint{7.551051in}{3.683596in}}{\pgfqpoint{7.558865in}{3.691409in}}%
\pgfpathcurveto{\pgfqpoint{7.566678in}{3.699223in}}{\pgfqpoint{7.571069in}{3.709822in}}{\pgfqpoint{7.571069in}{3.720872in}}%
\pgfpathcurveto{\pgfqpoint{7.571069in}{3.731922in}}{\pgfqpoint{7.566678in}{3.742521in}}{\pgfqpoint{7.558865in}{3.750335in}}%
\pgfpathcurveto{\pgfqpoint{7.551051in}{3.758148in}}{\pgfqpoint{7.540452in}{3.762539in}}{\pgfqpoint{7.529402in}{3.762539in}}%
\pgfpathcurveto{\pgfqpoint{7.518352in}{3.762539in}}{\pgfqpoint{7.507753in}{3.758148in}}{\pgfqpoint{7.499939in}{3.750335in}}%
\pgfpathcurveto{\pgfqpoint{7.492126in}{3.742521in}}{\pgfqpoint{7.487735in}{3.731922in}}{\pgfqpoint{7.487735in}{3.720872in}}%
\pgfpathcurveto{\pgfqpoint{7.487735in}{3.709822in}}{\pgfqpoint{7.492126in}{3.699223in}}{\pgfqpoint{7.499939in}{3.691409in}}%
\pgfpathcurveto{\pgfqpoint{7.507753in}{3.683596in}}{\pgfqpoint{7.518352in}{3.679205in}}{\pgfqpoint{7.529402in}{3.679205in}}%
\pgfpathclose%
\pgfusepath{stroke,fill}%
\end{pgfscope}%
\begin{pgfscope}%
\pgfpathrectangle{\pgfqpoint{0.526127in}{0.331635in}}{\pgfqpoint{9.300000in}{7.700000in}}%
\pgfusepath{clip}%
\pgfsetbuttcap%
\pgfsetroundjoin%
\definecolor{currentfill}{rgb}{1.000000,0.705882,0.509804}%
\pgfsetfillcolor{currentfill}%
\pgfsetlinewidth{0.481800pt}%
\definecolor{currentstroke}{rgb}{1.000000,1.000000,1.000000}%
\pgfsetstrokecolor{currentstroke}%
\pgfsetdash{}{0pt}%
\pgfpathmoveto{\pgfqpoint{7.857488in}{3.630724in}}%
\pgfpathcurveto{\pgfqpoint{7.868538in}{3.630724in}}{\pgfqpoint{7.879137in}{3.635114in}}{\pgfqpoint{7.886951in}{3.642928in}}%
\pgfpathcurveto{\pgfqpoint{7.894764in}{3.650741in}}{\pgfqpoint{7.899155in}{3.661340in}}{\pgfqpoint{7.899155in}{3.672391in}}%
\pgfpathcurveto{\pgfqpoint{7.899155in}{3.683441in}}{\pgfqpoint{7.894764in}{3.694040in}}{\pgfqpoint{7.886951in}{3.701853in}}%
\pgfpathcurveto{\pgfqpoint{7.879137in}{3.709667in}}{\pgfqpoint{7.868538in}{3.714057in}}{\pgfqpoint{7.857488in}{3.714057in}}%
\pgfpathcurveto{\pgfqpoint{7.846438in}{3.714057in}}{\pgfqpoint{7.835839in}{3.709667in}}{\pgfqpoint{7.828025in}{3.701853in}}%
\pgfpathcurveto{\pgfqpoint{7.820212in}{3.694040in}}{\pgfqpoint{7.815821in}{3.683441in}}{\pgfqpoint{7.815821in}{3.672391in}}%
\pgfpathcurveto{\pgfqpoint{7.815821in}{3.661340in}}{\pgfqpoint{7.820212in}{3.650741in}}{\pgfqpoint{7.828025in}{3.642928in}}%
\pgfpathcurveto{\pgfqpoint{7.835839in}{3.635114in}}{\pgfqpoint{7.846438in}{3.630724in}}{\pgfqpoint{7.857488in}{3.630724in}}%
\pgfpathclose%
\pgfusepath{stroke,fill}%
\end{pgfscope}%
\begin{pgfscope}%
\pgfpathrectangle{\pgfqpoint{0.526127in}{0.331635in}}{\pgfqpoint{9.300000in}{7.700000in}}%
\pgfusepath{clip}%
\pgfsetbuttcap%
\pgfsetroundjoin%
\definecolor{currentfill}{rgb}{1.000000,0.705882,0.509804}%
\pgfsetfillcolor{currentfill}%
\pgfsetlinewidth{0.481800pt}%
\definecolor{currentstroke}{rgb}{1.000000,1.000000,1.000000}%
\pgfsetstrokecolor{currentstroke}%
\pgfsetdash{}{0pt}%
\pgfpathmoveto{\pgfqpoint{6.572528in}{4.183325in}}%
\pgfpathcurveto{\pgfqpoint{6.583578in}{4.183325in}}{\pgfqpoint{6.594177in}{4.187715in}}{\pgfqpoint{6.601990in}{4.195528in}}%
\pgfpathcurveto{\pgfqpoint{6.609804in}{4.203342in}}{\pgfqpoint{6.614194in}{4.213941in}}{\pgfqpoint{6.614194in}{4.224991in}}%
\pgfpathcurveto{\pgfqpoint{6.614194in}{4.236041in}}{\pgfqpoint{6.609804in}{4.246640in}}{\pgfqpoint{6.601990in}{4.254454in}}%
\pgfpathcurveto{\pgfqpoint{6.594177in}{4.262268in}}{\pgfqpoint{6.583578in}{4.266658in}}{\pgfqpoint{6.572528in}{4.266658in}}%
\pgfpathcurveto{\pgfqpoint{6.561477in}{4.266658in}}{\pgfqpoint{6.550878in}{4.262268in}}{\pgfqpoint{6.543065in}{4.254454in}}%
\pgfpathcurveto{\pgfqpoint{6.535251in}{4.246640in}}{\pgfqpoint{6.530861in}{4.236041in}}{\pgfqpoint{6.530861in}{4.224991in}}%
\pgfpathcurveto{\pgfqpoint{6.530861in}{4.213941in}}{\pgfqpoint{6.535251in}{4.203342in}}{\pgfqpoint{6.543065in}{4.195528in}}%
\pgfpathcurveto{\pgfqpoint{6.550878in}{4.187715in}}{\pgfqpoint{6.561477in}{4.183325in}}{\pgfqpoint{6.572528in}{4.183325in}}%
\pgfpathclose%
\pgfusepath{stroke,fill}%
\end{pgfscope}%
\begin{pgfscope}%
\pgfpathrectangle{\pgfqpoint{0.526127in}{0.331635in}}{\pgfqpoint{9.300000in}{7.700000in}}%
\pgfusepath{clip}%
\pgfsetbuttcap%
\pgfsetroundjoin%
\definecolor{currentfill}{rgb}{1.000000,0.705882,0.509804}%
\pgfsetfillcolor{currentfill}%
\pgfsetlinewidth{0.481800pt}%
\definecolor{currentstroke}{rgb}{1.000000,1.000000,1.000000}%
\pgfsetstrokecolor{currentstroke}%
\pgfsetdash{}{0pt}%
\pgfpathmoveto{\pgfqpoint{5.154620in}{3.724228in}}%
\pgfpathcurveto{\pgfqpoint{5.165670in}{3.724228in}}{\pgfqpoint{5.176269in}{3.728618in}}{\pgfqpoint{5.184083in}{3.736431in}}%
\pgfpathcurveto{\pgfqpoint{5.191896in}{3.744245in}}{\pgfqpoint{5.196287in}{3.754844in}}{\pgfqpoint{5.196287in}{3.765894in}}%
\pgfpathcurveto{\pgfqpoint{5.196287in}{3.776944in}}{\pgfqpoint{5.191896in}{3.787543in}}{\pgfqpoint{5.184083in}{3.795357in}}%
\pgfpathcurveto{\pgfqpoint{5.176269in}{3.803171in}}{\pgfqpoint{5.165670in}{3.807561in}}{\pgfqpoint{5.154620in}{3.807561in}}%
\pgfpathcurveto{\pgfqpoint{5.143570in}{3.807561in}}{\pgfqpoint{5.132971in}{3.803171in}}{\pgfqpoint{5.125157in}{3.795357in}}%
\pgfpathcurveto{\pgfqpoint{5.117344in}{3.787543in}}{\pgfqpoint{5.112953in}{3.776944in}}{\pgfqpoint{5.112953in}{3.765894in}}%
\pgfpathcurveto{\pgfqpoint{5.112953in}{3.754844in}}{\pgfqpoint{5.117344in}{3.744245in}}{\pgfqpoint{5.125157in}{3.736431in}}%
\pgfpathcurveto{\pgfqpoint{5.132971in}{3.728618in}}{\pgfqpoint{5.143570in}{3.724228in}}{\pgfqpoint{5.154620in}{3.724228in}}%
\pgfpathclose%
\pgfusepath{stroke,fill}%
\end{pgfscope}%
\begin{pgfscope}%
\pgfpathrectangle{\pgfqpoint{0.526127in}{0.331635in}}{\pgfqpoint{9.300000in}{7.700000in}}%
\pgfusepath{clip}%
\pgfsetbuttcap%
\pgfsetroundjoin%
\definecolor{currentfill}{rgb}{1.000000,0.705882,0.509804}%
\pgfsetfillcolor{currentfill}%
\pgfsetlinewidth{0.481800pt}%
\definecolor{currentstroke}{rgb}{1.000000,1.000000,1.000000}%
\pgfsetstrokecolor{currentstroke}%
\pgfsetdash{}{0pt}%
\pgfpathmoveto{\pgfqpoint{5.190149in}{3.797647in}}%
\pgfpathcurveto{\pgfqpoint{5.201199in}{3.797647in}}{\pgfqpoint{5.211798in}{3.802038in}}{\pgfqpoint{5.219612in}{3.809851in}}%
\pgfpathcurveto{\pgfqpoint{5.227426in}{3.817665in}}{\pgfqpoint{5.231816in}{3.828264in}}{\pgfqpoint{5.231816in}{3.839314in}}%
\pgfpathcurveto{\pgfqpoint{5.231816in}{3.850364in}}{\pgfqpoint{5.227426in}{3.860963in}}{\pgfqpoint{5.219612in}{3.868777in}}%
\pgfpathcurveto{\pgfqpoint{5.211798in}{3.876591in}}{\pgfqpoint{5.201199in}{3.880981in}}{\pgfqpoint{5.190149in}{3.880981in}}%
\pgfpathcurveto{\pgfqpoint{5.179099in}{3.880981in}}{\pgfqpoint{5.168500in}{3.876591in}}{\pgfqpoint{5.160687in}{3.868777in}}%
\pgfpathcurveto{\pgfqpoint{5.152873in}{3.860963in}}{\pgfqpoint{5.148483in}{3.850364in}}{\pgfqpoint{5.148483in}{3.839314in}}%
\pgfpathcurveto{\pgfqpoint{5.148483in}{3.828264in}}{\pgfqpoint{5.152873in}{3.817665in}}{\pgfqpoint{5.160687in}{3.809851in}}%
\pgfpathcurveto{\pgfqpoint{5.168500in}{3.802038in}}{\pgfqpoint{5.179099in}{3.797647in}}{\pgfqpoint{5.190149in}{3.797647in}}%
\pgfpathclose%
\pgfusepath{stroke,fill}%
\end{pgfscope}%
\begin{pgfscope}%
\pgfpathrectangle{\pgfqpoint{0.526127in}{0.331635in}}{\pgfqpoint{9.300000in}{7.700000in}}%
\pgfusepath{clip}%
\pgfsetbuttcap%
\pgfsetroundjoin%
\definecolor{currentfill}{rgb}{1.000000,0.705882,0.509804}%
\pgfsetfillcolor{currentfill}%
\pgfsetlinewidth{0.481800pt}%
\definecolor{currentstroke}{rgb}{1.000000,1.000000,1.000000}%
\pgfsetstrokecolor{currentstroke}%
\pgfsetdash{}{0pt}%
\pgfpathmoveto{\pgfqpoint{6.190070in}{4.268157in}}%
\pgfpathcurveto{\pgfqpoint{6.201120in}{4.268157in}}{\pgfqpoint{6.211719in}{4.272548in}}{\pgfqpoint{6.219532in}{4.280361in}}%
\pgfpathcurveto{\pgfqpoint{6.227346in}{4.288175in}}{\pgfqpoint{6.231736in}{4.298774in}}{\pgfqpoint{6.231736in}{4.309824in}}%
\pgfpathcurveto{\pgfqpoint{6.231736in}{4.320874in}}{\pgfqpoint{6.227346in}{4.331473in}}{\pgfqpoint{6.219532in}{4.339287in}}%
\pgfpathcurveto{\pgfqpoint{6.211719in}{4.347101in}}{\pgfqpoint{6.201120in}{4.351491in}}{\pgfqpoint{6.190070in}{4.351491in}}%
\pgfpathcurveto{\pgfqpoint{6.179019in}{4.351491in}}{\pgfqpoint{6.168420in}{4.347101in}}{\pgfqpoint{6.160607in}{4.339287in}}%
\pgfpathcurveto{\pgfqpoint{6.152793in}{4.331473in}}{\pgfqpoint{6.148403in}{4.320874in}}{\pgfqpoint{6.148403in}{4.309824in}}%
\pgfpathcurveto{\pgfqpoint{6.148403in}{4.298774in}}{\pgfqpoint{6.152793in}{4.288175in}}{\pgfqpoint{6.160607in}{4.280361in}}%
\pgfpathcurveto{\pgfqpoint{6.168420in}{4.272548in}}{\pgfqpoint{6.179019in}{4.268157in}}{\pgfqpoint{6.190070in}{4.268157in}}%
\pgfpathclose%
\pgfusepath{stroke,fill}%
\end{pgfscope}%
\begin{pgfscope}%
\pgfpathrectangle{\pgfqpoint{0.526127in}{0.331635in}}{\pgfqpoint{9.300000in}{7.700000in}}%
\pgfusepath{clip}%
\pgfsetbuttcap%
\pgfsetroundjoin%
\definecolor{currentfill}{rgb}{1.000000,0.705882,0.509804}%
\pgfsetfillcolor{currentfill}%
\pgfsetlinewidth{0.481800pt}%
\definecolor{currentstroke}{rgb}{1.000000,1.000000,1.000000}%
\pgfsetstrokecolor{currentstroke}%
\pgfsetdash{}{0pt}%
\pgfpathmoveto{\pgfqpoint{6.008150in}{4.950389in}}%
\pgfpathcurveto{\pgfqpoint{6.019200in}{4.950389in}}{\pgfqpoint{6.029799in}{4.954779in}}{\pgfqpoint{6.037612in}{4.962593in}}%
\pgfpathcurveto{\pgfqpoint{6.045426in}{4.970406in}}{\pgfqpoint{6.049816in}{4.981005in}}{\pgfqpoint{6.049816in}{4.992055in}}%
\pgfpathcurveto{\pgfqpoint{6.049816in}{5.003105in}}{\pgfqpoint{6.045426in}{5.013705in}}{\pgfqpoint{6.037612in}{5.021518in}}%
\pgfpathcurveto{\pgfqpoint{6.029799in}{5.029332in}}{\pgfqpoint{6.019200in}{5.033722in}}{\pgfqpoint{6.008150in}{5.033722in}}%
\pgfpathcurveto{\pgfqpoint{5.997099in}{5.033722in}}{\pgfqpoint{5.986500in}{5.029332in}}{\pgfqpoint{5.978687in}{5.021518in}}%
\pgfpathcurveto{\pgfqpoint{5.970873in}{5.013705in}}{\pgfqpoint{5.966483in}{5.003105in}}{\pgfqpoint{5.966483in}{4.992055in}}%
\pgfpathcurveto{\pgfqpoint{5.966483in}{4.981005in}}{\pgfqpoint{5.970873in}{4.970406in}}{\pgfqpoint{5.978687in}{4.962593in}}%
\pgfpathcurveto{\pgfqpoint{5.986500in}{4.954779in}}{\pgfqpoint{5.997099in}{4.950389in}}{\pgfqpoint{6.008150in}{4.950389in}}%
\pgfpathclose%
\pgfusepath{stroke,fill}%
\end{pgfscope}%
\begin{pgfscope}%
\pgfpathrectangle{\pgfqpoint{0.526127in}{0.331635in}}{\pgfqpoint{9.300000in}{7.700000in}}%
\pgfusepath{clip}%
\pgfsetbuttcap%
\pgfsetroundjoin%
\definecolor{currentfill}{rgb}{1.000000,0.705882,0.509804}%
\pgfsetfillcolor{currentfill}%
\pgfsetlinewidth{0.481800pt}%
\definecolor{currentstroke}{rgb}{1.000000,1.000000,1.000000}%
\pgfsetstrokecolor{currentstroke}%
\pgfsetdash{}{0pt}%
\pgfpathmoveto{\pgfqpoint{3.566645in}{0.639968in}}%
\pgfpathcurveto{\pgfqpoint{3.577695in}{0.639968in}}{\pgfqpoint{3.588294in}{0.644359in}}{\pgfqpoint{3.596108in}{0.652172in}}%
\pgfpathcurveto{\pgfqpoint{3.603921in}{0.659986in}}{\pgfqpoint{3.608311in}{0.670585in}}{\pgfqpoint{3.608311in}{0.681635in}}%
\pgfpathcurveto{\pgfqpoint{3.608311in}{0.692685in}}{\pgfqpoint{3.603921in}{0.703284in}}{\pgfqpoint{3.596108in}{0.711098in}}%
\pgfpathcurveto{\pgfqpoint{3.588294in}{0.718911in}}{\pgfqpoint{3.577695in}{0.723302in}}{\pgfqpoint{3.566645in}{0.723302in}}%
\pgfpathcurveto{\pgfqpoint{3.555595in}{0.723302in}}{\pgfqpoint{3.544996in}{0.718911in}}{\pgfqpoint{3.537182in}{0.711098in}}%
\pgfpathcurveto{\pgfqpoint{3.529368in}{0.703284in}}{\pgfqpoint{3.524978in}{0.692685in}}{\pgfqpoint{3.524978in}{0.681635in}}%
\pgfpathcurveto{\pgfqpoint{3.524978in}{0.670585in}}{\pgfqpoint{3.529368in}{0.659986in}}{\pgfqpoint{3.537182in}{0.652172in}}%
\pgfpathcurveto{\pgfqpoint{3.544996in}{0.644359in}}{\pgfqpoint{3.555595in}{0.639968in}}{\pgfqpoint{3.566645in}{0.639968in}}%
\pgfpathclose%
\pgfusepath{stroke,fill}%
\end{pgfscope}%
\begin{pgfscope}%
\pgfpathrectangle{\pgfqpoint{0.526127in}{0.331635in}}{\pgfqpoint{9.300000in}{7.700000in}}%
\pgfusepath{clip}%
\pgfsetbuttcap%
\pgfsetroundjoin%
\definecolor{currentfill}{rgb}{1.000000,0.705882,0.509804}%
\pgfsetfillcolor{currentfill}%
\pgfsetlinewidth{0.481800pt}%
\definecolor{currentstroke}{rgb}{1.000000,1.000000,1.000000}%
\pgfsetstrokecolor{currentstroke}%
\pgfsetdash{}{0pt}%
\pgfpathmoveto{\pgfqpoint{7.924332in}{3.429377in}}%
\pgfpathcurveto{\pgfqpoint{7.935383in}{3.429377in}}{\pgfqpoint{7.945982in}{3.433767in}}{\pgfqpoint{7.953795in}{3.441580in}}%
\pgfpathcurveto{\pgfqpoint{7.961609in}{3.449394in}}{\pgfqpoint{7.965999in}{3.459993in}}{\pgfqpoint{7.965999in}{3.471043in}}%
\pgfpathcurveto{\pgfqpoint{7.965999in}{3.482093in}}{\pgfqpoint{7.961609in}{3.492692in}}{\pgfqpoint{7.953795in}{3.500506in}}%
\pgfpathcurveto{\pgfqpoint{7.945982in}{3.508320in}}{\pgfqpoint{7.935383in}{3.512710in}}{\pgfqpoint{7.924332in}{3.512710in}}%
\pgfpathcurveto{\pgfqpoint{7.913282in}{3.512710in}}{\pgfqpoint{7.902683in}{3.508320in}}{\pgfqpoint{7.894870in}{3.500506in}}%
\pgfpathcurveto{\pgfqpoint{7.887056in}{3.492692in}}{\pgfqpoint{7.882666in}{3.482093in}}{\pgfqpoint{7.882666in}{3.471043in}}%
\pgfpathcurveto{\pgfqpoint{7.882666in}{3.459993in}}{\pgfqpoint{7.887056in}{3.449394in}}{\pgfqpoint{7.894870in}{3.441580in}}%
\pgfpathcurveto{\pgfqpoint{7.902683in}{3.433767in}}{\pgfqpoint{7.913282in}{3.429377in}}{\pgfqpoint{7.924332in}{3.429377in}}%
\pgfpathclose%
\pgfusepath{stroke,fill}%
\end{pgfscope}%
\begin{pgfscope}%
\pgfpathrectangle{\pgfqpoint{0.526127in}{0.331635in}}{\pgfqpoint{9.300000in}{7.700000in}}%
\pgfusepath{clip}%
\pgfsetbuttcap%
\pgfsetroundjoin%
\definecolor{currentfill}{rgb}{1.000000,0.705882,0.509804}%
\pgfsetfillcolor{currentfill}%
\pgfsetlinewidth{0.481800pt}%
\definecolor{currentstroke}{rgb}{1.000000,1.000000,1.000000}%
\pgfsetstrokecolor{currentstroke}%
\pgfsetdash{}{0pt}%
\pgfpathmoveto{\pgfqpoint{6.132760in}{3.833143in}}%
\pgfpathcurveto{\pgfqpoint{6.143811in}{3.833143in}}{\pgfqpoint{6.154410in}{3.837534in}}{\pgfqpoint{6.162223in}{3.845347in}}%
\pgfpathcurveto{\pgfqpoint{6.170037in}{3.853161in}}{\pgfqpoint{6.174427in}{3.863760in}}{\pgfqpoint{6.174427in}{3.874810in}}%
\pgfpathcurveto{\pgfqpoint{6.174427in}{3.885860in}}{\pgfqpoint{6.170037in}{3.896459in}}{\pgfqpoint{6.162223in}{3.904273in}}%
\pgfpathcurveto{\pgfqpoint{6.154410in}{3.912086in}}{\pgfqpoint{6.143811in}{3.916477in}}{\pgfqpoint{6.132760in}{3.916477in}}%
\pgfpathcurveto{\pgfqpoint{6.121710in}{3.916477in}}{\pgfqpoint{6.111111in}{3.912086in}}{\pgfqpoint{6.103298in}{3.904273in}}%
\pgfpathcurveto{\pgfqpoint{6.095484in}{3.896459in}}{\pgfqpoint{6.091094in}{3.885860in}}{\pgfqpoint{6.091094in}{3.874810in}}%
\pgfpathcurveto{\pgfqpoint{6.091094in}{3.863760in}}{\pgfqpoint{6.095484in}{3.853161in}}{\pgfqpoint{6.103298in}{3.845347in}}%
\pgfpathcurveto{\pgfqpoint{6.111111in}{3.837534in}}{\pgfqpoint{6.121710in}{3.833143in}}{\pgfqpoint{6.132760in}{3.833143in}}%
\pgfpathclose%
\pgfusepath{stroke,fill}%
\end{pgfscope}%
\begin{pgfscope}%
\pgfpathrectangle{\pgfqpoint{0.526127in}{0.331635in}}{\pgfqpoint{9.300000in}{7.700000in}}%
\pgfusepath{clip}%
\pgfsetbuttcap%
\pgfsetroundjoin%
\definecolor{currentfill}{rgb}{1.000000,0.705882,0.509804}%
\pgfsetfillcolor{currentfill}%
\pgfsetlinewidth{0.481800pt}%
\definecolor{currentstroke}{rgb}{1.000000,1.000000,1.000000}%
\pgfsetstrokecolor{currentstroke}%
\pgfsetdash{}{0pt}%
\pgfpathmoveto{\pgfqpoint{6.193019in}{4.731699in}}%
\pgfpathcurveto{\pgfqpoint{6.204069in}{4.731699in}}{\pgfqpoint{6.214668in}{4.736090in}}{\pgfqpoint{6.222482in}{4.743903in}}%
\pgfpathcurveto{\pgfqpoint{6.230295in}{4.751717in}}{\pgfqpoint{6.234686in}{4.762316in}}{\pgfqpoint{6.234686in}{4.773366in}}%
\pgfpathcurveto{\pgfqpoint{6.234686in}{4.784416in}}{\pgfqpoint{6.230295in}{4.795015in}}{\pgfqpoint{6.222482in}{4.802829in}}%
\pgfpathcurveto{\pgfqpoint{6.214668in}{4.810642in}}{\pgfqpoint{6.204069in}{4.815033in}}{\pgfqpoint{6.193019in}{4.815033in}}%
\pgfpathcurveto{\pgfqpoint{6.181969in}{4.815033in}}{\pgfqpoint{6.171370in}{4.810642in}}{\pgfqpoint{6.163556in}{4.802829in}}%
\pgfpathcurveto{\pgfqpoint{6.155743in}{4.795015in}}{\pgfqpoint{6.151352in}{4.784416in}}{\pgfqpoint{6.151352in}{4.773366in}}%
\pgfpathcurveto{\pgfqpoint{6.151352in}{4.762316in}}{\pgfqpoint{6.155743in}{4.751717in}}{\pgfqpoint{6.163556in}{4.743903in}}%
\pgfpathcurveto{\pgfqpoint{6.171370in}{4.736090in}}{\pgfqpoint{6.181969in}{4.731699in}}{\pgfqpoint{6.193019in}{4.731699in}}%
\pgfpathclose%
\pgfusepath{stroke,fill}%
\end{pgfscope}%
\begin{pgfscope}%
\pgfpathrectangle{\pgfqpoint{0.526127in}{0.331635in}}{\pgfqpoint{9.300000in}{7.700000in}}%
\pgfusepath{clip}%
\pgfsetbuttcap%
\pgfsetroundjoin%
\definecolor{currentfill}{rgb}{1.000000,0.705882,0.509804}%
\pgfsetfillcolor{currentfill}%
\pgfsetlinewidth{0.481800pt}%
\definecolor{currentstroke}{rgb}{1.000000,1.000000,1.000000}%
\pgfsetstrokecolor{currentstroke}%
\pgfsetdash{}{0pt}%
\pgfpathmoveto{\pgfqpoint{6.027104in}{5.585847in}}%
\pgfpathcurveto{\pgfqpoint{6.038154in}{5.585847in}}{\pgfqpoint{6.048753in}{5.590238in}}{\pgfqpoint{6.056567in}{5.598051in}}%
\pgfpathcurveto{\pgfqpoint{6.064380in}{5.605865in}}{\pgfqpoint{6.068770in}{5.616464in}}{\pgfqpoint{6.068770in}{5.627514in}}%
\pgfpathcurveto{\pgfqpoint{6.068770in}{5.638564in}}{\pgfqpoint{6.064380in}{5.649163in}}{\pgfqpoint{6.056567in}{5.656977in}}%
\pgfpathcurveto{\pgfqpoint{6.048753in}{5.664790in}}{\pgfqpoint{6.038154in}{5.669181in}}{\pgfqpoint{6.027104in}{5.669181in}}%
\pgfpathcurveto{\pgfqpoint{6.016054in}{5.669181in}}{\pgfqpoint{6.005455in}{5.664790in}}{\pgfqpoint{5.997641in}{5.656977in}}%
\pgfpathcurveto{\pgfqpoint{5.989827in}{5.649163in}}{\pgfqpoint{5.985437in}{5.638564in}}{\pgfqpoint{5.985437in}{5.627514in}}%
\pgfpathcurveto{\pgfqpoint{5.985437in}{5.616464in}}{\pgfqpoint{5.989827in}{5.605865in}}{\pgfqpoint{5.997641in}{5.598051in}}%
\pgfpathcurveto{\pgfqpoint{6.005455in}{5.590238in}}{\pgfqpoint{6.016054in}{5.585847in}}{\pgfqpoint{6.027104in}{5.585847in}}%
\pgfpathclose%
\pgfusepath{stroke,fill}%
\end{pgfscope}%
\begin{pgfscope}%
\pgfpathrectangle{\pgfqpoint{0.526127in}{0.331635in}}{\pgfqpoint{9.300000in}{7.700000in}}%
\pgfusepath{clip}%
\pgfsetbuttcap%
\pgfsetroundjoin%
\definecolor{currentfill}{rgb}{1.000000,0.705882,0.509804}%
\pgfsetfillcolor{currentfill}%
\pgfsetlinewidth{0.481800pt}%
\definecolor{currentstroke}{rgb}{1.000000,1.000000,1.000000}%
\pgfsetstrokecolor{currentstroke}%
\pgfsetdash{}{0pt}%
\pgfpathmoveto{\pgfqpoint{6.988150in}{5.526950in}}%
\pgfpathcurveto{\pgfqpoint{6.999201in}{5.526950in}}{\pgfqpoint{7.009800in}{5.531340in}}{\pgfqpoint{7.017613in}{5.539154in}}%
\pgfpathcurveto{\pgfqpoint{7.025427in}{5.546967in}}{\pgfqpoint{7.029817in}{5.557566in}}{\pgfqpoint{7.029817in}{5.568616in}}%
\pgfpathcurveto{\pgfqpoint{7.029817in}{5.579667in}}{\pgfqpoint{7.025427in}{5.590266in}}{\pgfqpoint{7.017613in}{5.598079in}}%
\pgfpathcurveto{\pgfqpoint{7.009800in}{5.605893in}}{\pgfqpoint{6.999201in}{5.610283in}}{\pgfqpoint{6.988150in}{5.610283in}}%
\pgfpathcurveto{\pgfqpoint{6.977100in}{5.610283in}}{\pgfqpoint{6.966501in}{5.605893in}}{\pgfqpoint{6.958688in}{5.598079in}}%
\pgfpathcurveto{\pgfqpoint{6.950874in}{5.590266in}}{\pgfqpoint{6.946484in}{5.579667in}}{\pgfqpoint{6.946484in}{5.568616in}}%
\pgfpathcurveto{\pgfqpoint{6.946484in}{5.557566in}}{\pgfqpoint{6.950874in}{5.546967in}}{\pgfqpoint{6.958688in}{5.539154in}}%
\pgfpathcurveto{\pgfqpoint{6.966501in}{5.531340in}}{\pgfqpoint{6.977100in}{5.526950in}}{\pgfqpoint{6.988150in}{5.526950in}}%
\pgfpathclose%
\pgfusepath{stroke,fill}%
\end{pgfscope}%
\begin{pgfscope}%
\pgfpathrectangle{\pgfqpoint{0.526127in}{0.331635in}}{\pgfqpoint{9.300000in}{7.700000in}}%
\pgfusepath{clip}%
\pgfsetbuttcap%
\pgfsetroundjoin%
\definecolor{currentfill}{rgb}{1.000000,0.705882,0.509804}%
\pgfsetfillcolor{currentfill}%
\pgfsetlinewidth{0.481800pt}%
\definecolor{currentstroke}{rgb}{1.000000,1.000000,1.000000}%
\pgfsetstrokecolor{currentstroke}%
\pgfsetdash{}{0pt}%
\pgfpathmoveto{\pgfqpoint{6.794558in}{3.513100in}}%
\pgfpathcurveto{\pgfqpoint{6.805608in}{3.513100in}}{\pgfqpoint{6.816207in}{3.517491in}}{\pgfqpoint{6.824021in}{3.525304in}}%
\pgfpathcurveto{\pgfqpoint{6.831835in}{3.533118in}}{\pgfqpoint{6.836225in}{3.543717in}}{\pgfqpoint{6.836225in}{3.554767in}}%
\pgfpathcurveto{\pgfqpoint{6.836225in}{3.565817in}}{\pgfqpoint{6.831835in}{3.576416in}}{\pgfqpoint{6.824021in}{3.584230in}}%
\pgfpathcurveto{\pgfqpoint{6.816207in}{3.592043in}}{\pgfqpoint{6.805608in}{3.596434in}}{\pgfqpoint{6.794558in}{3.596434in}}%
\pgfpathcurveto{\pgfqpoint{6.783508in}{3.596434in}}{\pgfqpoint{6.772909in}{3.592043in}}{\pgfqpoint{6.765096in}{3.584230in}}%
\pgfpathcurveto{\pgfqpoint{6.757282in}{3.576416in}}{\pgfqpoint{6.752892in}{3.565817in}}{\pgfqpoint{6.752892in}{3.554767in}}%
\pgfpathcurveto{\pgfqpoint{6.752892in}{3.543717in}}{\pgfqpoint{6.757282in}{3.533118in}}{\pgfqpoint{6.765096in}{3.525304in}}%
\pgfpathcurveto{\pgfqpoint{6.772909in}{3.517491in}}{\pgfqpoint{6.783508in}{3.513100in}}{\pgfqpoint{6.794558in}{3.513100in}}%
\pgfpathclose%
\pgfusepath{stroke,fill}%
\end{pgfscope}%
\begin{pgfscope}%
\pgfpathrectangle{\pgfqpoint{0.526127in}{0.331635in}}{\pgfqpoint{9.300000in}{7.700000in}}%
\pgfusepath{clip}%
\pgfsetbuttcap%
\pgfsetroundjoin%
\definecolor{currentfill}{rgb}{1.000000,0.705882,0.509804}%
\pgfsetfillcolor{currentfill}%
\pgfsetlinewidth{0.481800pt}%
\definecolor{currentstroke}{rgb}{1.000000,1.000000,1.000000}%
\pgfsetstrokecolor{currentstroke}%
\pgfsetdash{}{0pt}%
\pgfpathmoveto{\pgfqpoint{7.563920in}{3.414147in}}%
\pgfpathcurveto{\pgfqpoint{7.574970in}{3.414147in}}{\pgfqpoint{7.585570in}{3.418537in}}{\pgfqpoint{7.593383in}{3.426350in}}%
\pgfpathcurveto{\pgfqpoint{7.601197in}{3.434164in}}{\pgfqpoint{7.605587in}{3.444763in}}{\pgfqpoint{7.605587in}{3.455813in}}%
\pgfpathcurveto{\pgfqpoint{7.605587in}{3.466863in}}{\pgfqpoint{7.601197in}{3.477462in}}{\pgfqpoint{7.593383in}{3.485276in}}%
\pgfpathcurveto{\pgfqpoint{7.585570in}{3.493090in}}{\pgfqpoint{7.574970in}{3.497480in}}{\pgfqpoint{7.563920in}{3.497480in}}%
\pgfpathcurveto{\pgfqpoint{7.552870in}{3.497480in}}{\pgfqpoint{7.542271in}{3.493090in}}{\pgfqpoint{7.534458in}{3.485276in}}%
\pgfpathcurveto{\pgfqpoint{7.526644in}{3.477462in}}{\pgfqpoint{7.522254in}{3.466863in}}{\pgfqpoint{7.522254in}{3.455813in}}%
\pgfpathcurveto{\pgfqpoint{7.522254in}{3.444763in}}{\pgfqpoint{7.526644in}{3.434164in}}{\pgfqpoint{7.534458in}{3.426350in}}%
\pgfpathcurveto{\pgfqpoint{7.542271in}{3.418537in}}{\pgfqpoint{7.552870in}{3.414147in}}{\pgfqpoint{7.563920in}{3.414147in}}%
\pgfpathclose%
\pgfusepath{stroke,fill}%
\end{pgfscope}%
\begin{pgfscope}%
\pgfpathrectangle{\pgfqpoint{0.526127in}{0.331635in}}{\pgfqpoint{9.300000in}{7.700000in}}%
\pgfusepath{clip}%
\pgfsetbuttcap%
\pgfsetroundjoin%
\definecolor{currentfill}{rgb}{1.000000,0.705882,0.509804}%
\pgfsetfillcolor{currentfill}%
\pgfsetlinewidth{0.481800pt}%
\definecolor{currentstroke}{rgb}{1.000000,1.000000,1.000000}%
\pgfsetstrokecolor{currentstroke}%
\pgfsetdash{}{0pt}%
\pgfpathmoveto{\pgfqpoint{5.276797in}{5.708763in}}%
\pgfpathcurveto{\pgfqpoint{5.287847in}{5.708763in}}{\pgfqpoint{5.298446in}{5.713153in}}{\pgfqpoint{5.306260in}{5.720967in}}%
\pgfpathcurveto{\pgfqpoint{5.314073in}{5.728780in}}{\pgfqpoint{5.318464in}{5.739379in}}{\pgfqpoint{5.318464in}{5.750429in}}%
\pgfpathcurveto{\pgfqpoint{5.318464in}{5.761480in}}{\pgfqpoint{5.314073in}{5.772079in}}{\pgfqpoint{5.306260in}{5.779892in}}%
\pgfpathcurveto{\pgfqpoint{5.298446in}{5.787706in}}{\pgfqpoint{5.287847in}{5.792096in}}{\pgfqpoint{5.276797in}{5.792096in}}%
\pgfpathcurveto{\pgfqpoint{5.265747in}{5.792096in}}{\pgfqpoint{5.255148in}{5.787706in}}{\pgfqpoint{5.247334in}{5.779892in}}%
\pgfpathcurveto{\pgfqpoint{5.239521in}{5.772079in}}{\pgfqpoint{5.235130in}{5.761480in}}{\pgfqpoint{5.235130in}{5.750429in}}%
\pgfpathcurveto{\pgfqpoint{5.235130in}{5.739379in}}{\pgfqpoint{5.239521in}{5.728780in}}{\pgfqpoint{5.247334in}{5.720967in}}%
\pgfpathcurveto{\pgfqpoint{5.255148in}{5.713153in}}{\pgfqpoint{5.265747in}{5.708763in}}{\pgfqpoint{5.276797in}{5.708763in}}%
\pgfpathclose%
\pgfusepath{stroke,fill}%
\end{pgfscope}%
\begin{pgfscope}%
\pgfpathrectangle{\pgfqpoint{0.526127in}{0.331635in}}{\pgfqpoint{9.300000in}{7.700000in}}%
\pgfusepath{clip}%
\pgfsetbuttcap%
\pgfsetroundjoin%
\definecolor{currentfill}{rgb}{1.000000,0.705882,0.509804}%
\pgfsetfillcolor{currentfill}%
\pgfsetlinewidth{0.481800pt}%
\definecolor{currentstroke}{rgb}{1.000000,1.000000,1.000000}%
\pgfsetstrokecolor{currentstroke}%
\pgfsetdash{}{0pt}%
\pgfpathmoveto{\pgfqpoint{7.155517in}{3.669209in}}%
\pgfpathcurveto{\pgfqpoint{7.166567in}{3.669209in}}{\pgfqpoint{7.177166in}{3.673599in}}{\pgfqpoint{7.184980in}{3.681413in}}%
\pgfpathcurveto{\pgfqpoint{7.192794in}{3.689227in}}{\pgfqpoint{7.197184in}{3.699826in}}{\pgfqpoint{7.197184in}{3.710876in}}%
\pgfpathcurveto{\pgfqpoint{7.197184in}{3.721926in}}{\pgfqpoint{7.192794in}{3.732525in}}{\pgfqpoint{7.184980in}{3.740339in}}%
\pgfpathcurveto{\pgfqpoint{7.177166in}{3.748152in}}{\pgfqpoint{7.166567in}{3.752543in}}{\pgfqpoint{7.155517in}{3.752543in}}%
\pgfpathcurveto{\pgfqpoint{7.144467in}{3.752543in}}{\pgfqpoint{7.133868in}{3.748152in}}{\pgfqpoint{7.126055in}{3.740339in}}%
\pgfpathcurveto{\pgfqpoint{7.118241in}{3.732525in}}{\pgfqpoint{7.113851in}{3.721926in}}{\pgfqpoint{7.113851in}{3.710876in}}%
\pgfpathcurveto{\pgfqpoint{7.113851in}{3.699826in}}{\pgfqpoint{7.118241in}{3.689227in}}{\pgfqpoint{7.126055in}{3.681413in}}%
\pgfpathcurveto{\pgfqpoint{7.133868in}{3.673599in}}{\pgfqpoint{7.144467in}{3.669209in}}{\pgfqpoint{7.155517in}{3.669209in}}%
\pgfpathclose%
\pgfusepath{stroke,fill}%
\end{pgfscope}%
\begin{pgfscope}%
\pgfpathrectangle{\pgfqpoint{0.526127in}{0.331635in}}{\pgfqpoint{9.300000in}{7.700000in}}%
\pgfusepath{clip}%
\pgfsetbuttcap%
\pgfsetroundjoin%
\definecolor{currentfill}{rgb}{1.000000,0.705882,0.509804}%
\pgfsetfillcolor{currentfill}%
\pgfsetlinewidth{0.481800pt}%
\definecolor{currentstroke}{rgb}{1.000000,1.000000,1.000000}%
\pgfsetstrokecolor{currentstroke}%
\pgfsetdash{}{0pt}%
\pgfpathmoveto{\pgfqpoint{6.008087in}{6.360043in}}%
\pgfpathcurveto{\pgfqpoint{6.019137in}{6.360043in}}{\pgfqpoint{6.029736in}{6.364433in}}{\pgfqpoint{6.037549in}{6.372246in}}%
\pgfpathcurveto{\pgfqpoint{6.045363in}{6.380060in}}{\pgfqpoint{6.049753in}{6.390659in}}{\pgfqpoint{6.049753in}{6.401709in}}%
\pgfpathcurveto{\pgfqpoint{6.049753in}{6.412759in}}{\pgfqpoint{6.045363in}{6.423358in}}{\pgfqpoint{6.037549in}{6.431172in}}%
\pgfpathcurveto{\pgfqpoint{6.029736in}{6.438986in}}{\pgfqpoint{6.019137in}{6.443376in}}{\pgfqpoint{6.008087in}{6.443376in}}%
\pgfpathcurveto{\pgfqpoint{5.997037in}{6.443376in}}{\pgfqpoint{5.986438in}{6.438986in}}{\pgfqpoint{5.978624in}{6.431172in}}%
\pgfpathcurveto{\pgfqpoint{5.970810in}{6.423358in}}{\pgfqpoint{5.966420in}{6.412759in}}{\pgfqpoint{5.966420in}{6.401709in}}%
\pgfpathcurveto{\pgfqpoint{5.966420in}{6.390659in}}{\pgfqpoint{5.970810in}{6.380060in}}{\pgfqpoint{5.978624in}{6.372246in}}%
\pgfpathcurveto{\pgfqpoint{5.986438in}{6.364433in}}{\pgfqpoint{5.997037in}{6.360043in}}{\pgfqpoint{6.008087in}{6.360043in}}%
\pgfpathclose%
\pgfusepath{stroke,fill}%
\end{pgfscope}%
\begin{pgfscope}%
\pgfpathrectangle{\pgfqpoint{0.526127in}{0.331635in}}{\pgfqpoint{9.300000in}{7.700000in}}%
\pgfusepath{clip}%
\pgfsetbuttcap%
\pgfsetroundjoin%
\definecolor{currentfill}{rgb}{1.000000,0.705882,0.509804}%
\pgfsetfillcolor{currentfill}%
\pgfsetlinewidth{0.481800pt}%
\definecolor{currentstroke}{rgb}{1.000000,1.000000,1.000000}%
\pgfsetstrokecolor{currentstroke}%
\pgfsetdash{}{0pt}%
\pgfpathmoveto{\pgfqpoint{7.439477in}{5.247476in}}%
\pgfpathcurveto{\pgfqpoint{7.450527in}{5.247476in}}{\pgfqpoint{7.461126in}{5.251867in}}{\pgfqpoint{7.468940in}{5.259680in}}%
\pgfpathcurveto{\pgfqpoint{7.476754in}{5.267494in}}{\pgfqpoint{7.481144in}{5.278093in}}{\pgfqpoint{7.481144in}{5.289143in}}%
\pgfpathcurveto{\pgfqpoint{7.481144in}{5.300193in}}{\pgfqpoint{7.476754in}{5.310792in}}{\pgfqpoint{7.468940in}{5.318606in}}%
\pgfpathcurveto{\pgfqpoint{7.461126in}{5.326419in}}{\pgfqpoint{7.450527in}{5.330810in}}{\pgfqpoint{7.439477in}{5.330810in}}%
\pgfpathcurveto{\pgfqpoint{7.428427in}{5.330810in}}{\pgfqpoint{7.417828in}{5.326419in}}{\pgfqpoint{7.410014in}{5.318606in}}%
\pgfpathcurveto{\pgfqpoint{7.402201in}{5.310792in}}{\pgfqpoint{7.397811in}{5.300193in}}{\pgfqpoint{7.397811in}{5.289143in}}%
\pgfpathcurveto{\pgfqpoint{7.397811in}{5.278093in}}{\pgfqpoint{7.402201in}{5.267494in}}{\pgfqpoint{7.410014in}{5.259680in}}%
\pgfpathcurveto{\pgfqpoint{7.417828in}{5.251867in}}{\pgfqpoint{7.428427in}{5.247476in}}{\pgfqpoint{7.439477in}{5.247476in}}%
\pgfpathclose%
\pgfusepath{stroke,fill}%
\end{pgfscope}%
\begin{pgfscope}%
\pgfpathrectangle{\pgfqpoint{0.526127in}{0.331635in}}{\pgfqpoint{9.300000in}{7.700000in}}%
\pgfusepath{clip}%
\pgfsetbuttcap%
\pgfsetroundjoin%
\definecolor{currentfill}{rgb}{1.000000,0.705882,0.509804}%
\pgfsetfillcolor{currentfill}%
\pgfsetlinewidth{0.481800pt}%
\definecolor{currentstroke}{rgb}{1.000000,1.000000,1.000000}%
\pgfsetstrokecolor{currentstroke}%
\pgfsetdash{}{0pt}%
\pgfpathmoveto{\pgfqpoint{7.682374in}{4.833782in}}%
\pgfpathcurveto{\pgfqpoint{7.693424in}{4.833782in}}{\pgfqpoint{7.704023in}{4.838173in}}{\pgfqpoint{7.711837in}{4.845986in}}%
\pgfpathcurveto{\pgfqpoint{7.719650in}{4.853800in}}{\pgfqpoint{7.724041in}{4.864399in}}{\pgfqpoint{7.724041in}{4.875449in}}%
\pgfpathcurveto{\pgfqpoint{7.724041in}{4.886499in}}{\pgfqpoint{7.719650in}{4.897098in}}{\pgfqpoint{7.711837in}{4.904912in}}%
\pgfpathcurveto{\pgfqpoint{7.704023in}{4.912725in}}{\pgfqpoint{7.693424in}{4.917116in}}{\pgfqpoint{7.682374in}{4.917116in}}%
\pgfpathcurveto{\pgfqpoint{7.671324in}{4.917116in}}{\pgfqpoint{7.660725in}{4.912725in}}{\pgfqpoint{7.652911in}{4.904912in}}%
\pgfpathcurveto{\pgfqpoint{7.645098in}{4.897098in}}{\pgfqpoint{7.640707in}{4.886499in}}{\pgfqpoint{7.640707in}{4.875449in}}%
\pgfpathcurveto{\pgfqpoint{7.640707in}{4.864399in}}{\pgfqpoint{7.645098in}{4.853800in}}{\pgfqpoint{7.652911in}{4.845986in}}%
\pgfpathcurveto{\pgfqpoint{7.660725in}{4.838173in}}{\pgfqpoint{7.671324in}{4.833782in}}{\pgfqpoint{7.682374in}{4.833782in}}%
\pgfpathclose%
\pgfusepath{stroke,fill}%
\end{pgfscope}%
\begin{pgfscope}%
\pgfpathrectangle{\pgfqpoint{0.526127in}{0.331635in}}{\pgfqpoint{9.300000in}{7.700000in}}%
\pgfusepath{clip}%
\pgfsetbuttcap%
\pgfsetroundjoin%
\definecolor{currentfill}{rgb}{1.000000,0.705882,0.509804}%
\pgfsetfillcolor{currentfill}%
\pgfsetlinewidth{0.481800pt}%
\definecolor{currentstroke}{rgb}{1.000000,1.000000,1.000000}%
\pgfsetstrokecolor{currentstroke}%
\pgfsetdash{}{0pt}%
\pgfpathmoveto{\pgfqpoint{5.929161in}{6.085863in}}%
\pgfpathcurveto{\pgfqpoint{5.940211in}{6.085863in}}{\pgfqpoint{5.950810in}{6.090253in}}{\pgfqpoint{5.958624in}{6.098067in}}%
\pgfpathcurveto{\pgfqpoint{5.966437in}{6.105880in}}{\pgfqpoint{5.970827in}{6.116479in}}{\pgfqpoint{5.970827in}{6.127529in}}%
\pgfpathcurveto{\pgfqpoint{5.970827in}{6.138579in}}{\pgfqpoint{5.966437in}{6.149178in}}{\pgfqpoint{5.958624in}{6.156992in}}%
\pgfpathcurveto{\pgfqpoint{5.950810in}{6.164806in}}{\pgfqpoint{5.940211in}{6.169196in}}{\pgfqpoint{5.929161in}{6.169196in}}%
\pgfpathcurveto{\pgfqpoint{5.918111in}{6.169196in}}{\pgfqpoint{5.907512in}{6.164806in}}{\pgfqpoint{5.899698in}{6.156992in}}%
\pgfpathcurveto{\pgfqpoint{5.891884in}{6.149178in}}{\pgfqpoint{5.887494in}{6.138579in}}{\pgfqpoint{5.887494in}{6.127529in}}%
\pgfpathcurveto{\pgfqpoint{5.887494in}{6.116479in}}{\pgfqpoint{5.891884in}{6.105880in}}{\pgfqpoint{5.899698in}{6.098067in}}%
\pgfpathcurveto{\pgfqpoint{5.907512in}{6.090253in}}{\pgfqpoint{5.918111in}{6.085863in}}{\pgfqpoint{5.929161in}{6.085863in}}%
\pgfpathclose%
\pgfusepath{stroke,fill}%
\end{pgfscope}%
\begin{pgfscope}%
\pgfpathrectangle{\pgfqpoint{0.526127in}{0.331635in}}{\pgfqpoint{9.300000in}{7.700000in}}%
\pgfusepath{clip}%
\pgfsetbuttcap%
\pgfsetroundjoin%
\definecolor{currentfill}{rgb}{1.000000,0.705882,0.509804}%
\pgfsetfillcolor{currentfill}%
\pgfsetlinewidth{0.481800pt}%
\definecolor{currentstroke}{rgb}{1.000000,1.000000,1.000000}%
\pgfsetstrokecolor{currentstroke}%
\pgfsetdash{}{0pt}%
\pgfpathmoveto{\pgfqpoint{8.724020in}{3.258644in}}%
\pgfpathcurveto{\pgfqpoint{8.735071in}{3.258644in}}{\pgfqpoint{8.745670in}{3.263034in}}{\pgfqpoint{8.753483in}{3.270848in}}%
\pgfpathcurveto{\pgfqpoint{8.761297in}{3.278662in}}{\pgfqpoint{8.765687in}{3.289261in}}{\pgfqpoint{8.765687in}{3.300311in}}%
\pgfpathcurveto{\pgfqpoint{8.765687in}{3.311361in}}{\pgfqpoint{8.761297in}{3.321960in}}{\pgfqpoint{8.753483in}{3.329774in}}%
\pgfpathcurveto{\pgfqpoint{8.745670in}{3.337587in}}{\pgfqpoint{8.735071in}{3.341978in}}{\pgfqpoint{8.724020in}{3.341978in}}%
\pgfpathcurveto{\pgfqpoint{8.712970in}{3.341978in}}{\pgfqpoint{8.702371in}{3.337587in}}{\pgfqpoint{8.694558in}{3.329774in}}%
\pgfpathcurveto{\pgfqpoint{8.686744in}{3.321960in}}{\pgfqpoint{8.682354in}{3.311361in}}{\pgfqpoint{8.682354in}{3.300311in}}%
\pgfpathcurveto{\pgfqpoint{8.682354in}{3.289261in}}{\pgfqpoint{8.686744in}{3.278662in}}{\pgfqpoint{8.694558in}{3.270848in}}%
\pgfpathcurveto{\pgfqpoint{8.702371in}{3.263034in}}{\pgfqpoint{8.712970in}{3.258644in}}{\pgfqpoint{8.724020in}{3.258644in}}%
\pgfpathclose%
\pgfusepath{stroke,fill}%
\end{pgfscope}%
\begin{pgfscope}%
\pgfpathrectangle{\pgfqpoint{0.526127in}{0.331635in}}{\pgfqpoint{9.300000in}{7.700000in}}%
\pgfusepath{clip}%
\pgfsetbuttcap%
\pgfsetroundjoin%
\definecolor{currentfill}{rgb}{1.000000,0.705882,0.509804}%
\pgfsetfillcolor{currentfill}%
\pgfsetlinewidth{0.481800pt}%
\definecolor{currentstroke}{rgb}{1.000000,1.000000,1.000000}%
\pgfsetstrokecolor{currentstroke}%
\pgfsetdash{}{0pt}%
\pgfpathmoveto{\pgfqpoint{6.318203in}{3.404579in}}%
\pgfpathcurveto{\pgfqpoint{6.329253in}{3.404579in}}{\pgfqpoint{6.339852in}{3.408969in}}{\pgfqpoint{6.347666in}{3.416783in}}%
\pgfpathcurveto{\pgfqpoint{6.355479in}{3.424596in}}{\pgfqpoint{6.359870in}{3.435196in}}{\pgfqpoint{6.359870in}{3.446246in}}%
\pgfpathcurveto{\pgfqpoint{6.359870in}{3.457296in}}{\pgfqpoint{6.355479in}{3.467895in}}{\pgfqpoint{6.347666in}{3.475708in}}%
\pgfpathcurveto{\pgfqpoint{6.339852in}{3.483522in}}{\pgfqpoint{6.329253in}{3.487912in}}{\pgfqpoint{6.318203in}{3.487912in}}%
\pgfpathcurveto{\pgfqpoint{6.307153in}{3.487912in}}{\pgfqpoint{6.296554in}{3.483522in}}{\pgfqpoint{6.288740in}{3.475708in}}%
\pgfpathcurveto{\pgfqpoint{6.280927in}{3.467895in}}{\pgfqpoint{6.276536in}{3.457296in}}{\pgfqpoint{6.276536in}{3.446246in}}%
\pgfpathcurveto{\pgfqpoint{6.276536in}{3.435196in}}{\pgfqpoint{6.280927in}{3.424596in}}{\pgfqpoint{6.288740in}{3.416783in}}%
\pgfpathcurveto{\pgfqpoint{6.296554in}{3.408969in}}{\pgfqpoint{6.307153in}{3.404579in}}{\pgfqpoint{6.318203in}{3.404579in}}%
\pgfpathclose%
\pgfusepath{stroke,fill}%
\end{pgfscope}%
\begin{pgfscope}%
\pgfpathrectangle{\pgfqpoint{0.526127in}{0.331635in}}{\pgfqpoint{9.300000in}{7.700000in}}%
\pgfusepath{clip}%
\pgfsetbuttcap%
\pgfsetroundjoin%
\definecolor{currentfill}{rgb}{0.552941,0.898039,0.631373}%
\pgfsetfillcolor{currentfill}%
\pgfsetlinewidth{0.481800pt}%
\definecolor{currentstroke}{rgb}{1.000000,1.000000,1.000000}%
\pgfsetstrokecolor{currentstroke}%
\pgfsetdash{}{0pt}%
\pgfpathmoveto{\pgfqpoint{7.719822in}{2.245611in}}%
\pgfpathcurveto{\pgfqpoint{7.730872in}{2.245611in}}{\pgfqpoint{7.741471in}{2.250001in}}{\pgfqpoint{7.749285in}{2.257815in}}%
\pgfpathcurveto{\pgfqpoint{7.757098in}{2.265628in}}{\pgfqpoint{7.761489in}{2.276227in}}{\pgfqpoint{7.761489in}{2.287277in}}%
\pgfpathcurveto{\pgfqpoint{7.761489in}{2.298327in}}{\pgfqpoint{7.757098in}{2.308926in}}{\pgfqpoint{7.749285in}{2.316740in}}%
\pgfpathcurveto{\pgfqpoint{7.741471in}{2.324554in}}{\pgfqpoint{7.730872in}{2.328944in}}{\pgfqpoint{7.719822in}{2.328944in}}%
\pgfpathcurveto{\pgfqpoint{7.708772in}{2.328944in}}{\pgfqpoint{7.698173in}{2.324554in}}{\pgfqpoint{7.690359in}{2.316740in}}%
\pgfpathcurveto{\pgfqpoint{7.682546in}{2.308926in}}{\pgfqpoint{7.678155in}{2.298327in}}{\pgfqpoint{7.678155in}{2.287277in}}%
\pgfpathcurveto{\pgfqpoint{7.678155in}{2.276227in}}{\pgfqpoint{7.682546in}{2.265628in}}{\pgfqpoint{7.690359in}{2.257815in}}%
\pgfpathcurveto{\pgfqpoint{7.698173in}{2.250001in}}{\pgfqpoint{7.708772in}{2.245611in}}{\pgfqpoint{7.719822in}{2.245611in}}%
\pgfpathclose%
\pgfusepath{stroke,fill}%
\end{pgfscope}%
\begin{pgfscope}%
\pgfpathrectangle{\pgfqpoint{0.526127in}{0.331635in}}{\pgfqpoint{9.300000in}{7.700000in}}%
\pgfusepath{clip}%
\pgfsetbuttcap%
\pgfsetroundjoin%
\definecolor{currentfill}{rgb}{0.552941,0.898039,0.631373}%
\pgfsetfillcolor{currentfill}%
\pgfsetlinewidth{0.481800pt}%
\definecolor{currentstroke}{rgb}{1.000000,1.000000,1.000000}%
\pgfsetstrokecolor{currentstroke}%
\pgfsetdash{}{0pt}%
\pgfpathmoveto{\pgfqpoint{5.865422in}{1.569557in}}%
\pgfpathcurveto{\pgfqpoint{5.876472in}{1.569557in}}{\pgfqpoint{5.887071in}{1.573947in}}{\pgfqpoint{5.894885in}{1.581761in}}%
\pgfpathcurveto{\pgfqpoint{5.902698in}{1.589574in}}{\pgfqpoint{5.907089in}{1.600173in}}{\pgfqpoint{5.907089in}{1.611223in}}%
\pgfpathcurveto{\pgfqpoint{5.907089in}{1.622274in}}{\pgfqpoint{5.902698in}{1.632873in}}{\pgfqpoint{5.894885in}{1.640686in}}%
\pgfpathcurveto{\pgfqpoint{5.887071in}{1.648500in}}{\pgfqpoint{5.876472in}{1.652890in}}{\pgfqpoint{5.865422in}{1.652890in}}%
\pgfpathcurveto{\pgfqpoint{5.854372in}{1.652890in}}{\pgfqpoint{5.843773in}{1.648500in}}{\pgfqpoint{5.835959in}{1.640686in}}%
\pgfpathcurveto{\pgfqpoint{5.828145in}{1.632873in}}{\pgfqpoint{5.823755in}{1.622274in}}{\pgfqpoint{5.823755in}{1.611223in}}%
\pgfpathcurveto{\pgfqpoint{5.823755in}{1.600173in}}{\pgfqpoint{5.828145in}{1.589574in}}{\pgfqpoint{5.835959in}{1.581761in}}%
\pgfpathcurveto{\pgfqpoint{5.843773in}{1.573947in}}{\pgfqpoint{5.854372in}{1.569557in}}{\pgfqpoint{5.865422in}{1.569557in}}%
\pgfpathclose%
\pgfusepath{stroke,fill}%
\end{pgfscope}%
\begin{pgfscope}%
\pgfpathrectangle{\pgfqpoint{0.526127in}{0.331635in}}{\pgfqpoint{9.300000in}{7.700000in}}%
\pgfusepath{clip}%
\pgfsetbuttcap%
\pgfsetroundjoin%
\definecolor{currentfill}{rgb}{0.552941,0.898039,0.631373}%
\pgfsetfillcolor{currentfill}%
\pgfsetlinewidth{0.481800pt}%
\definecolor{currentstroke}{rgb}{1.000000,1.000000,1.000000}%
\pgfsetstrokecolor{currentstroke}%
\pgfsetdash{}{0pt}%
\pgfpathmoveto{\pgfqpoint{6.692745in}{2.167260in}}%
\pgfpathcurveto{\pgfqpoint{6.703795in}{2.167260in}}{\pgfqpoint{6.714394in}{2.171651in}}{\pgfqpoint{6.722208in}{2.179464in}}%
\pgfpathcurveto{\pgfqpoint{6.730022in}{2.187278in}}{\pgfqpoint{6.734412in}{2.197877in}}{\pgfqpoint{6.734412in}{2.208927in}}%
\pgfpathcurveto{\pgfqpoint{6.734412in}{2.219977in}}{\pgfqpoint{6.730022in}{2.230576in}}{\pgfqpoint{6.722208in}{2.238390in}}%
\pgfpathcurveto{\pgfqpoint{6.714394in}{2.246203in}}{\pgfqpoint{6.703795in}{2.250594in}}{\pgfqpoint{6.692745in}{2.250594in}}%
\pgfpathcurveto{\pgfqpoint{6.681695in}{2.250594in}}{\pgfqpoint{6.671096in}{2.246203in}}{\pgfqpoint{6.663282in}{2.238390in}}%
\pgfpathcurveto{\pgfqpoint{6.655469in}{2.230576in}}{\pgfqpoint{6.651078in}{2.219977in}}{\pgfqpoint{6.651078in}{2.208927in}}%
\pgfpathcurveto{\pgfqpoint{6.651078in}{2.197877in}}{\pgfqpoint{6.655469in}{2.187278in}}{\pgfqpoint{6.663282in}{2.179464in}}%
\pgfpathcurveto{\pgfqpoint{6.671096in}{2.171651in}}{\pgfqpoint{6.681695in}{2.167260in}}{\pgfqpoint{6.692745in}{2.167260in}}%
\pgfpathclose%
\pgfusepath{stroke,fill}%
\end{pgfscope}%
\begin{pgfscope}%
\pgfpathrectangle{\pgfqpoint{0.526127in}{0.331635in}}{\pgfqpoint{9.300000in}{7.700000in}}%
\pgfusepath{clip}%
\pgfsetbuttcap%
\pgfsetroundjoin%
\definecolor{currentfill}{rgb}{0.552941,0.898039,0.631373}%
\pgfsetfillcolor{currentfill}%
\pgfsetlinewidth{0.481800pt}%
\definecolor{currentstroke}{rgb}{1.000000,1.000000,1.000000}%
\pgfsetstrokecolor{currentstroke}%
\pgfsetdash{}{0pt}%
\pgfpathmoveto{\pgfqpoint{2.147474in}{3.354463in}}%
\pgfpathcurveto{\pgfqpoint{2.158524in}{3.354463in}}{\pgfqpoint{2.169123in}{3.358853in}}{\pgfqpoint{2.176937in}{3.366667in}}%
\pgfpathcurveto{\pgfqpoint{2.184750in}{3.374481in}}{\pgfqpoint{2.189140in}{3.385080in}}{\pgfqpoint{2.189140in}{3.396130in}}%
\pgfpathcurveto{\pgfqpoint{2.189140in}{3.407180in}}{\pgfqpoint{2.184750in}{3.417779in}}{\pgfqpoint{2.176937in}{3.425593in}}%
\pgfpathcurveto{\pgfqpoint{2.169123in}{3.433406in}}{\pgfqpoint{2.158524in}{3.437796in}}{\pgfqpoint{2.147474in}{3.437796in}}%
\pgfpathcurveto{\pgfqpoint{2.136424in}{3.437796in}}{\pgfqpoint{2.125825in}{3.433406in}}{\pgfqpoint{2.118011in}{3.425593in}}%
\pgfpathcurveto{\pgfqpoint{2.110197in}{3.417779in}}{\pgfqpoint{2.105807in}{3.407180in}}{\pgfqpoint{2.105807in}{3.396130in}}%
\pgfpathcurveto{\pgfqpoint{2.105807in}{3.385080in}}{\pgfqpoint{2.110197in}{3.374481in}}{\pgfqpoint{2.118011in}{3.366667in}}%
\pgfpathcurveto{\pgfqpoint{2.125825in}{3.358853in}}{\pgfqpoint{2.136424in}{3.354463in}}{\pgfqpoint{2.147474in}{3.354463in}}%
\pgfpathclose%
\pgfusepath{stroke,fill}%
\end{pgfscope}%
\begin{pgfscope}%
\pgfpathrectangle{\pgfqpoint{0.526127in}{0.331635in}}{\pgfqpoint{9.300000in}{7.700000in}}%
\pgfusepath{clip}%
\pgfsetbuttcap%
\pgfsetroundjoin%
\definecolor{currentfill}{rgb}{0.552941,0.898039,0.631373}%
\pgfsetfillcolor{currentfill}%
\pgfsetlinewidth{0.481800pt}%
\definecolor{currentstroke}{rgb}{1.000000,1.000000,1.000000}%
\pgfsetstrokecolor{currentstroke}%
\pgfsetdash{}{0pt}%
\pgfpathmoveto{\pgfqpoint{7.116859in}{3.099327in}}%
\pgfpathcurveto{\pgfqpoint{7.127909in}{3.099327in}}{\pgfqpoint{7.138508in}{3.103717in}}{\pgfqpoint{7.146322in}{3.111531in}}%
\pgfpathcurveto{\pgfqpoint{7.154135in}{3.119345in}}{\pgfqpoint{7.158526in}{3.129944in}}{\pgfqpoint{7.158526in}{3.140994in}}%
\pgfpathcurveto{\pgfqpoint{7.158526in}{3.152044in}}{\pgfqpoint{7.154135in}{3.162643in}}{\pgfqpoint{7.146322in}{3.170456in}}%
\pgfpathcurveto{\pgfqpoint{7.138508in}{3.178270in}}{\pgfqpoint{7.127909in}{3.182660in}}{\pgfqpoint{7.116859in}{3.182660in}}%
\pgfpathcurveto{\pgfqpoint{7.105809in}{3.182660in}}{\pgfqpoint{7.095210in}{3.178270in}}{\pgfqpoint{7.087396in}{3.170456in}}%
\pgfpathcurveto{\pgfqpoint{7.079583in}{3.162643in}}{\pgfqpoint{7.075192in}{3.152044in}}{\pgfqpoint{7.075192in}{3.140994in}}%
\pgfpathcurveto{\pgfqpoint{7.075192in}{3.129944in}}{\pgfqpoint{7.079583in}{3.119345in}}{\pgfqpoint{7.087396in}{3.111531in}}%
\pgfpathcurveto{\pgfqpoint{7.095210in}{3.103717in}}{\pgfqpoint{7.105809in}{3.099327in}}{\pgfqpoint{7.116859in}{3.099327in}}%
\pgfpathclose%
\pgfusepath{stroke,fill}%
\end{pgfscope}%
\begin{pgfscope}%
\pgfpathrectangle{\pgfqpoint{0.526127in}{0.331635in}}{\pgfqpoint{9.300000in}{7.700000in}}%
\pgfusepath{clip}%
\pgfsetbuttcap%
\pgfsetroundjoin%
\definecolor{currentfill}{rgb}{0.552941,0.898039,0.631373}%
\pgfsetfillcolor{currentfill}%
\pgfsetlinewidth{0.481800pt}%
\definecolor{currentstroke}{rgb}{1.000000,1.000000,1.000000}%
\pgfsetstrokecolor{currentstroke}%
\pgfsetdash{}{0pt}%
\pgfpathmoveto{\pgfqpoint{3.558492in}{4.103567in}}%
\pgfpathcurveto{\pgfqpoint{3.569542in}{4.103567in}}{\pgfqpoint{3.580141in}{4.107957in}}{\pgfqpoint{3.587955in}{4.115771in}}%
\pgfpathcurveto{\pgfqpoint{3.595769in}{4.123584in}}{\pgfqpoint{3.600159in}{4.134183in}}{\pgfqpoint{3.600159in}{4.145234in}}%
\pgfpathcurveto{\pgfqpoint{3.600159in}{4.156284in}}{\pgfqpoint{3.595769in}{4.166883in}}{\pgfqpoint{3.587955in}{4.174696in}}%
\pgfpathcurveto{\pgfqpoint{3.580141in}{4.182510in}}{\pgfqpoint{3.569542in}{4.186900in}}{\pgfqpoint{3.558492in}{4.186900in}}%
\pgfpathcurveto{\pgfqpoint{3.547442in}{4.186900in}}{\pgfqpoint{3.536843in}{4.182510in}}{\pgfqpoint{3.529029in}{4.174696in}}%
\pgfpathcurveto{\pgfqpoint{3.521216in}{4.166883in}}{\pgfqpoint{3.516826in}{4.156284in}}{\pgfqpoint{3.516826in}{4.145234in}}%
\pgfpathcurveto{\pgfqpoint{3.516826in}{4.134183in}}{\pgfqpoint{3.521216in}{4.123584in}}{\pgfqpoint{3.529029in}{4.115771in}}%
\pgfpathcurveto{\pgfqpoint{3.536843in}{4.107957in}}{\pgfqpoint{3.547442in}{4.103567in}}{\pgfqpoint{3.558492in}{4.103567in}}%
\pgfpathclose%
\pgfusepath{stroke,fill}%
\end{pgfscope}%
\begin{pgfscope}%
\pgfpathrectangle{\pgfqpoint{0.526127in}{0.331635in}}{\pgfqpoint{9.300000in}{7.700000in}}%
\pgfusepath{clip}%
\pgfsetbuttcap%
\pgfsetroundjoin%
\definecolor{currentfill}{rgb}{0.552941,0.898039,0.631373}%
\pgfsetfillcolor{currentfill}%
\pgfsetlinewidth{0.481800pt}%
\definecolor{currentstroke}{rgb}{1.000000,1.000000,1.000000}%
\pgfsetstrokecolor{currentstroke}%
\pgfsetdash{}{0pt}%
\pgfpathmoveto{\pgfqpoint{3.383051in}{1.527628in}}%
\pgfpathcurveto{\pgfqpoint{3.394101in}{1.527628in}}{\pgfqpoint{3.404700in}{1.532019in}}{\pgfqpoint{3.412514in}{1.539832in}}%
\pgfpathcurveto{\pgfqpoint{3.420328in}{1.547646in}}{\pgfqpoint{3.424718in}{1.558245in}}{\pgfqpoint{3.424718in}{1.569295in}}%
\pgfpathcurveto{\pgfqpoint{3.424718in}{1.580345in}}{\pgfqpoint{3.420328in}{1.590944in}}{\pgfqpoint{3.412514in}{1.598758in}}%
\pgfpathcurveto{\pgfqpoint{3.404700in}{1.606571in}}{\pgfqpoint{3.394101in}{1.610962in}}{\pgfqpoint{3.383051in}{1.610962in}}%
\pgfpathcurveto{\pgfqpoint{3.372001in}{1.610962in}}{\pgfqpoint{3.361402in}{1.606571in}}{\pgfqpoint{3.353588in}{1.598758in}}%
\pgfpathcurveto{\pgfqpoint{3.345775in}{1.590944in}}{\pgfqpoint{3.341385in}{1.580345in}}{\pgfqpoint{3.341385in}{1.569295in}}%
\pgfpathcurveto{\pgfqpoint{3.341385in}{1.558245in}}{\pgfqpoint{3.345775in}{1.547646in}}{\pgfqpoint{3.353588in}{1.539832in}}%
\pgfpathcurveto{\pgfqpoint{3.361402in}{1.532019in}}{\pgfqpoint{3.372001in}{1.527628in}}{\pgfqpoint{3.383051in}{1.527628in}}%
\pgfpathclose%
\pgfusepath{stroke,fill}%
\end{pgfscope}%
\begin{pgfscope}%
\pgfpathrectangle{\pgfqpoint{0.526127in}{0.331635in}}{\pgfqpoint{9.300000in}{7.700000in}}%
\pgfusepath{clip}%
\pgfsetbuttcap%
\pgfsetroundjoin%
\definecolor{currentfill}{rgb}{0.552941,0.898039,0.631373}%
\pgfsetfillcolor{currentfill}%
\pgfsetlinewidth{0.481800pt}%
\definecolor{currentstroke}{rgb}{1.000000,1.000000,1.000000}%
\pgfsetstrokecolor{currentstroke}%
\pgfsetdash{}{0pt}%
\pgfpathmoveto{\pgfqpoint{6.701250in}{2.564722in}}%
\pgfpathcurveto{\pgfqpoint{6.712300in}{2.564722in}}{\pgfqpoint{6.722899in}{2.569113in}}{\pgfqpoint{6.730713in}{2.576926in}}%
\pgfpathcurveto{\pgfqpoint{6.738526in}{2.584740in}}{\pgfqpoint{6.742917in}{2.595339in}}{\pgfqpoint{6.742917in}{2.606389in}}%
\pgfpathcurveto{\pgfqpoint{6.742917in}{2.617439in}}{\pgfqpoint{6.738526in}{2.628038in}}{\pgfqpoint{6.730713in}{2.635852in}}%
\pgfpathcurveto{\pgfqpoint{6.722899in}{2.643666in}}{\pgfqpoint{6.712300in}{2.648056in}}{\pgfqpoint{6.701250in}{2.648056in}}%
\pgfpathcurveto{\pgfqpoint{6.690200in}{2.648056in}}{\pgfqpoint{6.679601in}{2.643666in}}{\pgfqpoint{6.671787in}{2.635852in}}%
\pgfpathcurveto{\pgfqpoint{6.663973in}{2.628038in}}{\pgfqpoint{6.659583in}{2.617439in}}{\pgfqpoint{6.659583in}{2.606389in}}%
\pgfpathcurveto{\pgfqpoint{6.659583in}{2.595339in}}{\pgfqpoint{6.663973in}{2.584740in}}{\pgfqpoint{6.671787in}{2.576926in}}%
\pgfpathcurveto{\pgfqpoint{6.679601in}{2.569113in}}{\pgfqpoint{6.690200in}{2.564722in}}{\pgfqpoint{6.701250in}{2.564722in}}%
\pgfpathclose%
\pgfusepath{stroke,fill}%
\end{pgfscope}%
\begin{pgfscope}%
\pgfpathrectangle{\pgfqpoint{0.526127in}{0.331635in}}{\pgfqpoint{9.300000in}{7.700000in}}%
\pgfusepath{clip}%
\pgfsetbuttcap%
\pgfsetroundjoin%
\definecolor{currentfill}{rgb}{0.552941,0.898039,0.631373}%
\pgfsetfillcolor{currentfill}%
\pgfsetlinewidth{0.481800pt}%
\definecolor{currentstroke}{rgb}{1.000000,1.000000,1.000000}%
\pgfsetstrokecolor{currentstroke}%
\pgfsetdash{}{0pt}%
\pgfpathmoveto{\pgfqpoint{6.060531in}{3.385094in}}%
\pgfpathcurveto{\pgfqpoint{6.071581in}{3.385094in}}{\pgfqpoint{6.082180in}{3.389484in}}{\pgfqpoint{6.089994in}{3.397298in}}%
\pgfpathcurveto{\pgfqpoint{6.097808in}{3.405111in}}{\pgfqpoint{6.102198in}{3.415711in}}{\pgfqpoint{6.102198in}{3.426761in}}%
\pgfpathcurveto{\pgfqpoint{6.102198in}{3.437811in}}{\pgfqpoint{6.097808in}{3.448410in}}{\pgfqpoint{6.089994in}{3.456223in}}%
\pgfpathcurveto{\pgfqpoint{6.082180in}{3.464037in}}{\pgfqpoint{6.071581in}{3.468427in}}{\pgfqpoint{6.060531in}{3.468427in}}%
\pgfpathcurveto{\pgfqpoint{6.049481in}{3.468427in}}{\pgfqpoint{6.038882in}{3.464037in}}{\pgfqpoint{6.031068in}{3.456223in}}%
\pgfpathcurveto{\pgfqpoint{6.023255in}{3.448410in}}{\pgfqpoint{6.018865in}{3.437811in}}{\pgfqpoint{6.018865in}{3.426761in}}%
\pgfpathcurveto{\pgfqpoint{6.018865in}{3.415711in}}{\pgfqpoint{6.023255in}{3.405111in}}{\pgfqpoint{6.031068in}{3.397298in}}%
\pgfpathcurveto{\pgfqpoint{6.038882in}{3.389484in}}{\pgfqpoint{6.049481in}{3.385094in}}{\pgfqpoint{6.060531in}{3.385094in}}%
\pgfpathclose%
\pgfusepath{stroke,fill}%
\end{pgfscope}%
\begin{pgfscope}%
\pgfpathrectangle{\pgfqpoint{0.526127in}{0.331635in}}{\pgfqpoint{9.300000in}{7.700000in}}%
\pgfusepath{clip}%
\pgfsetbuttcap%
\pgfsetroundjoin%
\definecolor{currentfill}{rgb}{0.552941,0.898039,0.631373}%
\pgfsetfillcolor{currentfill}%
\pgfsetlinewidth{0.481800pt}%
\definecolor{currentstroke}{rgb}{1.000000,1.000000,1.000000}%
\pgfsetstrokecolor{currentstroke}%
\pgfsetdash{}{0pt}%
\pgfpathmoveto{\pgfqpoint{3.545333in}{4.472892in}}%
\pgfpathcurveto{\pgfqpoint{3.556383in}{4.472892in}}{\pgfqpoint{3.566982in}{4.477282in}}{\pgfqpoint{3.574796in}{4.485096in}}%
\pgfpathcurveto{\pgfqpoint{3.582609in}{4.492910in}}{\pgfqpoint{3.586999in}{4.503509in}}{\pgfqpoint{3.586999in}{4.514559in}}%
\pgfpathcurveto{\pgfqpoint{3.586999in}{4.525609in}}{\pgfqpoint{3.582609in}{4.536208in}}{\pgfqpoint{3.574796in}{4.544021in}}%
\pgfpathcurveto{\pgfqpoint{3.566982in}{4.551835in}}{\pgfqpoint{3.556383in}{4.556225in}}{\pgfqpoint{3.545333in}{4.556225in}}%
\pgfpathcurveto{\pgfqpoint{3.534283in}{4.556225in}}{\pgfqpoint{3.523684in}{4.551835in}}{\pgfqpoint{3.515870in}{4.544021in}}%
\pgfpathcurveto{\pgfqpoint{3.508056in}{4.536208in}}{\pgfqpoint{3.503666in}{4.525609in}}{\pgfqpoint{3.503666in}{4.514559in}}%
\pgfpathcurveto{\pgfqpoint{3.503666in}{4.503509in}}{\pgfqpoint{3.508056in}{4.492910in}}{\pgfqpoint{3.515870in}{4.485096in}}%
\pgfpathcurveto{\pgfqpoint{3.523684in}{4.477282in}}{\pgfqpoint{3.534283in}{4.472892in}}{\pgfqpoint{3.545333in}{4.472892in}}%
\pgfpathclose%
\pgfusepath{stroke,fill}%
\end{pgfscope}%
\begin{pgfscope}%
\pgfpathrectangle{\pgfqpoint{0.526127in}{0.331635in}}{\pgfqpoint{9.300000in}{7.700000in}}%
\pgfusepath{clip}%
\pgfsetbuttcap%
\pgfsetroundjoin%
\definecolor{currentfill}{rgb}{0.552941,0.898039,0.631373}%
\pgfsetfillcolor{currentfill}%
\pgfsetlinewidth{0.481800pt}%
\definecolor{currentstroke}{rgb}{1.000000,1.000000,1.000000}%
\pgfsetstrokecolor{currentstroke}%
\pgfsetdash{}{0pt}%
\pgfpathmoveto{\pgfqpoint{2.060713in}{2.880644in}}%
\pgfpathcurveto{\pgfqpoint{2.071764in}{2.880644in}}{\pgfqpoint{2.082363in}{2.885034in}}{\pgfqpoint{2.090176in}{2.892848in}}%
\pgfpathcurveto{\pgfqpoint{2.097990in}{2.900661in}}{\pgfqpoint{2.102380in}{2.911261in}}{\pgfqpoint{2.102380in}{2.922311in}}%
\pgfpathcurveto{\pgfqpoint{2.102380in}{2.933361in}}{\pgfqpoint{2.097990in}{2.943960in}}{\pgfqpoint{2.090176in}{2.951773in}}%
\pgfpathcurveto{\pgfqpoint{2.082363in}{2.959587in}}{\pgfqpoint{2.071764in}{2.963977in}}{\pgfqpoint{2.060713in}{2.963977in}}%
\pgfpathcurveto{\pgfqpoint{2.049663in}{2.963977in}}{\pgfqpoint{2.039064in}{2.959587in}}{\pgfqpoint{2.031251in}{2.951773in}}%
\pgfpathcurveto{\pgfqpoint{2.023437in}{2.943960in}}{\pgfqpoint{2.019047in}{2.933361in}}{\pgfqpoint{2.019047in}{2.922311in}}%
\pgfpathcurveto{\pgfqpoint{2.019047in}{2.911261in}}{\pgfqpoint{2.023437in}{2.900661in}}{\pgfqpoint{2.031251in}{2.892848in}}%
\pgfpathcurveto{\pgfqpoint{2.039064in}{2.885034in}}{\pgfqpoint{2.049663in}{2.880644in}}{\pgfqpoint{2.060713in}{2.880644in}}%
\pgfpathclose%
\pgfusepath{stroke,fill}%
\end{pgfscope}%
\begin{pgfscope}%
\pgfpathrectangle{\pgfqpoint{0.526127in}{0.331635in}}{\pgfqpoint{9.300000in}{7.700000in}}%
\pgfusepath{clip}%
\pgfsetbuttcap%
\pgfsetroundjoin%
\definecolor{currentfill}{rgb}{0.552941,0.898039,0.631373}%
\pgfsetfillcolor{currentfill}%
\pgfsetlinewidth{0.481800pt}%
\definecolor{currentstroke}{rgb}{1.000000,1.000000,1.000000}%
\pgfsetstrokecolor{currentstroke}%
\pgfsetdash{}{0pt}%
\pgfpathmoveto{\pgfqpoint{9.178578in}{4.662248in}}%
\pgfpathcurveto{\pgfqpoint{9.189628in}{4.662248in}}{\pgfqpoint{9.200227in}{4.666638in}}{\pgfqpoint{9.208041in}{4.674452in}}%
\pgfpathcurveto{\pgfqpoint{9.215854in}{4.682265in}}{\pgfqpoint{9.220244in}{4.692865in}}{\pgfqpoint{9.220244in}{4.703915in}}%
\pgfpathcurveto{\pgfqpoint{9.220244in}{4.714965in}}{\pgfqpoint{9.215854in}{4.725564in}}{\pgfqpoint{9.208041in}{4.733377in}}%
\pgfpathcurveto{\pgfqpoint{9.200227in}{4.741191in}}{\pgfqpoint{9.189628in}{4.745581in}}{\pgfqpoint{9.178578in}{4.745581in}}%
\pgfpathcurveto{\pgfqpoint{9.167528in}{4.745581in}}{\pgfqpoint{9.156929in}{4.741191in}}{\pgfqpoint{9.149115in}{4.733377in}}%
\pgfpathcurveto{\pgfqpoint{9.141301in}{4.725564in}}{\pgfqpoint{9.136911in}{4.714965in}}{\pgfqpoint{9.136911in}{4.703915in}}%
\pgfpathcurveto{\pgfqpoint{9.136911in}{4.692865in}}{\pgfqpoint{9.141301in}{4.682265in}}{\pgfqpoint{9.149115in}{4.674452in}}%
\pgfpathcurveto{\pgfqpoint{9.156929in}{4.666638in}}{\pgfqpoint{9.167528in}{4.662248in}}{\pgfqpoint{9.178578in}{4.662248in}}%
\pgfpathclose%
\pgfusepath{stroke,fill}%
\end{pgfscope}%
\begin{pgfscope}%
\pgfpathrectangle{\pgfqpoint{0.526127in}{0.331635in}}{\pgfqpoint{9.300000in}{7.700000in}}%
\pgfusepath{clip}%
\pgfsetbuttcap%
\pgfsetroundjoin%
\definecolor{currentfill}{rgb}{0.552941,0.898039,0.631373}%
\pgfsetfillcolor{currentfill}%
\pgfsetlinewidth{0.481800pt}%
\definecolor{currentstroke}{rgb}{1.000000,1.000000,1.000000}%
\pgfsetstrokecolor{currentstroke}%
\pgfsetdash{}{0pt}%
\pgfpathmoveto{\pgfqpoint{2.528674in}{2.413856in}}%
\pgfpathcurveto{\pgfqpoint{2.539724in}{2.413856in}}{\pgfqpoint{2.550323in}{2.418246in}}{\pgfqpoint{2.558137in}{2.426060in}}%
\pgfpathcurveto{\pgfqpoint{2.565950in}{2.433873in}}{\pgfqpoint{2.570341in}{2.444472in}}{\pgfqpoint{2.570341in}{2.455523in}}%
\pgfpathcurveto{\pgfqpoint{2.570341in}{2.466573in}}{\pgfqpoint{2.565950in}{2.477172in}}{\pgfqpoint{2.558137in}{2.484985in}}%
\pgfpathcurveto{\pgfqpoint{2.550323in}{2.492799in}}{\pgfqpoint{2.539724in}{2.497189in}}{\pgfqpoint{2.528674in}{2.497189in}}%
\pgfpathcurveto{\pgfqpoint{2.517624in}{2.497189in}}{\pgfqpoint{2.507025in}{2.492799in}}{\pgfqpoint{2.499211in}{2.484985in}}%
\pgfpathcurveto{\pgfqpoint{2.491397in}{2.477172in}}{\pgfqpoint{2.487007in}{2.466573in}}{\pgfqpoint{2.487007in}{2.455523in}}%
\pgfpathcurveto{\pgfqpoint{2.487007in}{2.444472in}}{\pgfqpoint{2.491397in}{2.433873in}}{\pgfqpoint{2.499211in}{2.426060in}}%
\pgfpathcurveto{\pgfqpoint{2.507025in}{2.418246in}}{\pgfqpoint{2.517624in}{2.413856in}}{\pgfqpoint{2.528674in}{2.413856in}}%
\pgfpathclose%
\pgfusepath{stroke,fill}%
\end{pgfscope}%
\begin{pgfscope}%
\pgfpathrectangle{\pgfqpoint{0.526127in}{0.331635in}}{\pgfqpoint{9.300000in}{7.700000in}}%
\pgfusepath{clip}%
\pgfsetbuttcap%
\pgfsetroundjoin%
\definecolor{currentfill}{rgb}{0.552941,0.898039,0.631373}%
\pgfsetfillcolor{currentfill}%
\pgfsetlinewidth{0.481800pt}%
\definecolor{currentstroke}{rgb}{1.000000,1.000000,1.000000}%
\pgfsetstrokecolor{currentstroke}%
\pgfsetdash{}{0pt}%
\pgfpathmoveto{\pgfqpoint{2.638599in}{2.924571in}}%
\pgfpathcurveto{\pgfqpoint{2.649649in}{2.924571in}}{\pgfqpoint{2.660248in}{2.928962in}}{\pgfqpoint{2.668062in}{2.936775in}}%
\pgfpathcurveto{\pgfqpoint{2.675876in}{2.944589in}}{\pgfqpoint{2.680266in}{2.955188in}}{\pgfqpoint{2.680266in}{2.966238in}}%
\pgfpathcurveto{\pgfqpoint{2.680266in}{2.977288in}}{\pgfqpoint{2.675876in}{2.987887in}}{\pgfqpoint{2.668062in}{2.995701in}}%
\pgfpathcurveto{\pgfqpoint{2.660248in}{3.003515in}}{\pgfqpoint{2.649649in}{3.007905in}}{\pgfqpoint{2.638599in}{3.007905in}}%
\pgfpathcurveto{\pgfqpoint{2.627549in}{3.007905in}}{\pgfqpoint{2.616950in}{3.003515in}}{\pgfqpoint{2.609137in}{2.995701in}}%
\pgfpathcurveto{\pgfqpoint{2.601323in}{2.987887in}}{\pgfqpoint{2.596933in}{2.977288in}}{\pgfqpoint{2.596933in}{2.966238in}}%
\pgfpathcurveto{\pgfqpoint{2.596933in}{2.955188in}}{\pgfqpoint{2.601323in}{2.944589in}}{\pgfqpoint{2.609137in}{2.936775in}}%
\pgfpathcurveto{\pgfqpoint{2.616950in}{2.928962in}}{\pgfqpoint{2.627549in}{2.924571in}}{\pgfqpoint{2.638599in}{2.924571in}}%
\pgfpathclose%
\pgfusepath{stroke,fill}%
\end{pgfscope}%
\begin{pgfscope}%
\pgfpathrectangle{\pgfqpoint{0.526127in}{0.331635in}}{\pgfqpoint{9.300000in}{7.700000in}}%
\pgfusepath{clip}%
\pgfsetbuttcap%
\pgfsetroundjoin%
\definecolor{currentfill}{rgb}{0.552941,0.898039,0.631373}%
\pgfsetfillcolor{currentfill}%
\pgfsetlinewidth{0.481800pt}%
\definecolor{currentstroke}{rgb}{1.000000,1.000000,1.000000}%
\pgfsetstrokecolor{currentstroke}%
\pgfsetdash{}{0pt}%
\pgfpathmoveto{\pgfqpoint{2.718231in}{2.230056in}}%
\pgfpathcurveto{\pgfqpoint{2.729281in}{2.230056in}}{\pgfqpoint{2.739880in}{2.234446in}}{\pgfqpoint{2.747694in}{2.242259in}}%
\pgfpathcurveto{\pgfqpoint{2.755508in}{2.250073in}}{\pgfqpoint{2.759898in}{2.260672in}}{\pgfqpoint{2.759898in}{2.271722in}}%
\pgfpathcurveto{\pgfqpoint{2.759898in}{2.282772in}}{\pgfqpoint{2.755508in}{2.293371in}}{\pgfqpoint{2.747694in}{2.301185in}}%
\pgfpathcurveto{\pgfqpoint{2.739880in}{2.308999in}}{\pgfqpoint{2.729281in}{2.313389in}}{\pgfqpoint{2.718231in}{2.313389in}}%
\pgfpathcurveto{\pgfqpoint{2.707181in}{2.313389in}}{\pgfqpoint{2.696582in}{2.308999in}}{\pgfqpoint{2.688768in}{2.301185in}}%
\pgfpathcurveto{\pgfqpoint{2.680955in}{2.293371in}}{\pgfqpoint{2.676564in}{2.282772in}}{\pgfqpoint{2.676564in}{2.271722in}}%
\pgfpathcurveto{\pgfqpoint{2.676564in}{2.260672in}}{\pgfqpoint{2.680955in}{2.250073in}}{\pgfqpoint{2.688768in}{2.242259in}}%
\pgfpathcurveto{\pgfqpoint{2.696582in}{2.234446in}}{\pgfqpoint{2.707181in}{2.230056in}}{\pgfqpoint{2.718231in}{2.230056in}}%
\pgfpathclose%
\pgfusepath{stroke,fill}%
\end{pgfscope}%
\begin{pgfscope}%
\pgfpathrectangle{\pgfqpoint{0.526127in}{0.331635in}}{\pgfqpoint{9.300000in}{7.700000in}}%
\pgfusepath{clip}%
\pgfsetbuttcap%
\pgfsetroundjoin%
\definecolor{currentfill}{rgb}{0.552941,0.898039,0.631373}%
\pgfsetfillcolor{currentfill}%
\pgfsetlinewidth{0.481800pt}%
\definecolor{currentstroke}{rgb}{1.000000,1.000000,1.000000}%
\pgfsetstrokecolor{currentstroke}%
\pgfsetdash{}{0pt}%
\pgfpathmoveto{\pgfqpoint{1.976182in}{2.480597in}}%
\pgfpathcurveto{\pgfqpoint{1.987232in}{2.480597in}}{\pgfqpoint{1.997831in}{2.484987in}}{\pgfqpoint{2.005645in}{2.492801in}}%
\pgfpathcurveto{\pgfqpoint{2.013459in}{2.500614in}}{\pgfqpoint{2.017849in}{2.511213in}}{\pgfqpoint{2.017849in}{2.522263in}}%
\pgfpathcurveto{\pgfqpoint{2.017849in}{2.533314in}}{\pgfqpoint{2.013459in}{2.543913in}}{\pgfqpoint{2.005645in}{2.551726in}}%
\pgfpathcurveto{\pgfqpoint{1.997831in}{2.559540in}}{\pgfqpoint{1.987232in}{2.563930in}}{\pgfqpoint{1.976182in}{2.563930in}}%
\pgfpathcurveto{\pgfqpoint{1.965132in}{2.563930in}}{\pgfqpoint{1.954533in}{2.559540in}}{\pgfqpoint{1.946719in}{2.551726in}}%
\pgfpathcurveto{\pgfqpoint{1.938906in}{2.543913in}}{\pgfqpoint{1.934515in}{2.533314in}}{\pgfqpoint{1.934515in}{2.522263in}}%
\pgfpathcurveto{\pgfqpoint{1.934515in}{2.511213in}}{\pgfqpoint{1.938906in}{2.500614in}}{\pgfqpoint{1.946719in}{2.492801in}}%
\pgfpathcurveto{\pgfqpoint{1.954533in}{2.484987in}}{\pgfqpoint{1.965132in}{2.480597in}}{\pgfqpoint{1.976182in}{2.480597in}}%
\pgfpathclose%
\pgfusepath{stroke,fill}%
\end{pgfscope}%
\begin{pgfscope}%
\pgfpathrectangle{\pgfqpoint{0.526127in}{0.331635in}}{\pgfqpoint{9.300000in}{7.700000in}}%
\pgfusepath{clip}%
\pgfsetbuttcap%
\pgfsetroundjoin%
\definecolor{currentfill}{rgb}{0.552941,0.898039,0.631373}%
\pgfsetfillcolor{currentfill}%
\pgfsetlinewidth{0.481800pt}%
\definecolor{currentstroke}{rgb}{1.000000,1.000000,1.000000}%
\pgfsetstrokecolor{currentstroke}%
\pgfsetdash{}{0pt}%
\pgfpathmoveto{\pgfqpoint{2.309460in}{2.873355in}}%
\pgfpathcurveto{\pgfqpoint{2.320510in}{2.873355in}}{\pgfqpoint{2.331109in}{2.877745in}}{\pgfqpoint{2.338923in}{2.885559in}}%
\pgfpathcurveto{\pgfqpoint{2.346736in}{2.893372in}}{\pgfqpoint{2.351127in}{2.903971in}}{\pgfqpoint{2.351127in}{2.915022in}}%
\pgfpathcurveto{\pgfqpoint{2.351127in}{2.926072in}}{\pgfqpoint{2.346736in}{2.936671in}}{\pgfqpoint{2.338923in}{2.944484in}}%
\pgfpathcurveto{\pgfqpoint{2.331109in}{2.952298in}}{\pgfqpoint{2.320510in}{2.956688in}}{\pgfqpoint{2.309460in}{2.956688in}}%
\pgfpathcurveto{\pgfqpoint{2.298410in}{2.956688in}}{\pgfqpoint{2.287811in}{2.952298in}}{\pgfqpoint{2.279997in}{2.944484in}}%
\pgfpathcurveto{\pgfqpoint{2.272184in}{2.936671in}}{\pgfqpoint{2.267793in}{2.926072in}}{\pgfqpoint{2.267793in}{2.915022in}}%
\pgfpathcurveto{\pgfqpoint{2.267793in}{2.903971in}}{\pgfqpoint{2.272184in}{2.893372in}}{\pgfqpoint{2.279997in}{2.885559in}}%
\pgfpathcurveto{\pgfqpoint{2.287811in}{2.877745in}}{\pgfqpoint{2.298410in}{2.873355in}}{\pgfqpoint{2.309460in}{2.873355in}}%
\pgfpathclose%
\pgfusepath{stroke,fill}%
\end{pgfscope}%
\begin{pgfscope}%
\pgfpathrectangle{\pgfqpoint{0.526127in}{0.331635in}}{\pgfqpoint{9.300000in}{7.700000in}}%
\pgfusepath{clip}%
\pgfsetbuttcap%
\pgfsetroundjoin%
\definecolor{currentfill}{rgb}{0.552941,0.898039,0.631373}%
\pgfsetfillcolor{currentfill}%
\pgfsetlinewidth{0.481800pt}%
\definecolor{currentstroke}{rgb}{1.000000,1.000000,1.000000}%
\pgfsetstrokecolor{currentstroke}%
\pgfsetdash{}{0pt}%
\pgfpathmoveto{\pgfqpoint{1.801343in}{3.465174in}}%
\pgfpathcurveto{\pgfqpoint{1.812393in}{3.465174in}}{\pgfqpoint{1.822992in}{3.469565in}}{\pgfqpoint{1.830806in}{3.477378in}}%
\pgfpathcurveto{\pgfqpoint{1.838619in}{3.485192in}}{\pgfqpoint{1.843009in}{3.495791in}}{\pgfqpoint{1.843009in}{3.506841in}}%
\pgfpathcurveto{\pgfqpoint{1.843009in}{3.517891in}}{\pgfqpoint{1.838619in}{3.528490in}}{\pgfqpoint{1.830806in}{3.536304in}}%
\pgfpathcurveto{\pgfqpoint{1.822992in}{3.544117in}}{\pgfqpoint{1.812393in}{3.548508in}}{\pgfqpoint{1.801343in}{3.548508in}}%
\pgfpathcurveto{\pgfqpoint{1.790293in}{3.548508in}}{\pgfqpoint{1.779694in}{3.544117in}}{\pgfqpoint{1.771880in}{3.536304in}}%
\pgfpathcurveto{\pgfqpoint{1.764066in}{3.528490in}}{\pgfqpoint{1.759676in}{3.517891in}}{\pgfqpoint{1.759676in}{3.506841in}}%
\pgfpathcurveto{\pgfqpoint{1.759676in}{3.495791in}}{\pgfqpoint{1.764066in}{3.485192in}}{\pgfqpoint{1.771880in}{3.477378in}}%
\pgfpathcurveto{\pgfqpoint{1.779694in}{3.469565in}}{\pgfqpoint{1.790293in}{3.465174in}}{\pgfqpoint{1.801343in}{3.465174in}}%
\pgfpathclose%
\pgfusepath{stroke,fill}%
\end{pgfscope}%
\begin{pgfscope}%
\pgfpathrectangle{\pgfqpoint{0.526127in}{0.331635in}}{\pgfqpoint{9.300000in}{7.700000in}}%
\pgfusepath{clip}%
\pgfsetbuttcap%
\pgfsetroundjoin%
\definecolor{currentfill}{rgb}{0.552941,0.898039,0.631373}%
\pgfsetfillcolor{currentfill}%
\pgfsetlinewidth{0.481800pt}%
\definecolor{currentstroke}{rgb}{1.000000,1.000000,1.000000}%
\pgfsetstrokecolor{currentstroke}%
\pgfsetdash{}{0pt}%
\pgfpathmoveto{\pgfqpoint{2.821384in}{2.771711in}}%
\pgfpathcurveto{\pgfqpoint{2.832434in}{2.771711in}}{\pgfqpoint{2.843033in}{2.776101in}}{\pgfqpoint{2.850847in}{2.783914in}}%
\pgfpathcurveto{\pgfqpoint{2.858660in}{2.791728in}}{\pgfqpoint{2.863051in}{2.802327in}}{\pgfqpoint{2.863051in}{2.813377in}}%
\pgfpathcurveto{\pgfqpoint{2.863051in}{2.824427in}}{\pgfqpoint{2.858660in}{2.835026in}}{\pgfqpoint{2.850847in}{2.842840in}}%
\pgfpathcurveto{\pgfqpoint{2.843033in}{2.850654in}}{\pgfqpoint{2.832434in}{2.855044in}}{\pgfqpoint{2.821384in}{2.855044in}}%
\pgfpathcurveto{\pgfqpoint{2.810334in}{2.855044in}}{\pgfqpoint{2.799735in}{2.850654in}}{\pgfqpoint{2.791921in}{2.842840in}}%
\pgfpathcurveto{\pgfqpoint{2.784107in}{2.835026in}}{\pgfqpoint{2.779717in}{2.824427in}}{\pgfqpoint{2.779717in}{2.813377in}}%
\pgfpathcurveto{\pgfqpoint{2.779717in}{2.802327in}}{\pgfqpoint{2.784107in}{2.791728in}}{\pgfqpoint{2.791921in}{2.783914in}}%
\pgfpathcurveto{\pgfqpoint{2.799735in}{2.776101in}}{\pgfqpoint{2.810334in}{2.771711in}}{\pgfqpoint{2.821384in}{2.771711in}}%
\pgfpathclose%
\pgfusepath{stroke,fill}%
\end{pgfscope}%
\begin{pgfscope}%
\pgfpathrectangle{\pgfqpoint{0.526127in}{0.331635in}}{\pgfqpoint{9.300000in}{7.700000in}}%
\pgfusepath{clip}%
\pgfsetbuttcap%
\pgfsetroundjoin%
\definecolor{currentfill}{rgb}{0.552941,0.898039,0.631373}%
\pgfsetfillcolor{currentfill}%
\pgfsetlinewidth{0.481800pt}%
\definecolor{currentstroke}{rgb}{1.000000,1.000000,1.000000}%
\pgfsetstrokecolor{currentstroke}%
\pgfsetdash{}{0pt}%
\pgfpathmoveto{\pgfqpoint{7.035760in}{1.010655in}}%
\pgfpathcurveto{\pgfqpoint{7.046810in}{1.010655in}}{\pgfqpoint{7.057409in}{1.015045in}}{\pgfqpoint{7.065223in}{1.022858in}}%
\pgfpathcurveto{\pgfqpoint{7.073037in}{1.030672in}}{\pgfqpoint{7.077427in}{1.041271in}}{\pgfqpoint{7.077427in}{1.052321in}}%
\pgfpathcurveto{\pgfqpoint{7.077427in}{1.063371in}}{\pgfqpoint{7.073037in}{1.073970in}}{\pgfqpoint{7.065223in}{1.081784in}}%
\pgfpathcurveto{\pgfqpoint{7.057409in}{1.089598in}}{\pgfqpoint{7.046810in}{1.093988in}}{\pgfqpoint{7.035760in}{1.093988in}}%
\pgfpathcurveto{\pgfqpoint{7.024710in}{1.093988in}}{\pgfqpoint{7.014111in}{1.089598in}}{\pgfqpoint{7.006297in}{1.081784in}}%
\pgfpathcurveto{\pgfqpoint{6.998484in}{1.073970in}}{\pgfqpoint{6.994094in}{1.063371in}}{\pgfqpoint{6.994094in}{1.052321in}}%
\pgfpathcurveto{\pgfqpoint{6.994094in}{1.041271in}}{\pgfqpoint{6.998484in}{1.030672in}}{\pgfqpoint{7.006297in}{1.022858in}}%
\pgfpathcurveto{\pgfqpoint{7.014111in}{1.015045in}}{\pgfqpoint{7.024710in}{1.010655in}}{\pgfqpoint{7.035760in}{1.010655in}}%
\pgfpathclose%
\pgfusepath{stroke,fill}%
\end{pgfscope}%
\begin{pgfscope}%
\pgfpathrectangle{\pgfqpoint{0.526127in}{0.331635in}}{\pgfqpoint{9.300000in}{7.700000in}}%
\pgfusepath{clip}%
\pgfsetbuttcap%
\pgfsetroundjoin%
\definecolor{currentfill}{rgb}{0.552941,0.898039,0.631373}%
\pgfsetfillcolor{currentfill}%
\pgfsetlinewidth{0.481800pt}%
\definecolor{currentstroke}{rgb}{1.000000,1.000000,1.000000}%
\pgfsetstrokecolor{currentstroke}%
\pgfsetdash{}{0pt}%
\pgfpathmoveto{\pgfqpoint{6.494262in}{1.134274in}}%
\pgfpathcurveto{\pgfqpoint{6.505313in}{1.134274in}}{\pgfqpoint{6.515912in}{1.138664in}}{\pgfqpoint{6.523725in}{1.146477in}}%
\pgfpathcurveto{\pgfqpoint{6.531539in}{1.154291in}}{\pgfqpoint{6.535929in}{1.164890in}}{\pgfqpoint{6.535929in}{1.175940in}}%
\pgfpathcurveto{\pgfqpoint{6.535929in}{1.186990in}}{\pgfqpoint{6.531539in}{1.197589in}}{\pgfqpoint{6.523725in}{1.205403in}}%
\pgfpathcurveto{\pgfqpoint{6.515912in}{1.213217in}}{\pgfqpoint{6.505313in}{1.217607in}}{\pgfqpoint{6.494262in}{1.217607in}}%
\pgfpathcurveto{\pgfqpoint{6.483212in}{1.217607in}}{\pgfqpoint{6.472613in}{1.213217in}}{\pgfqpoint{6.464800in}{1.205403in}}%
\pgfpathcurveto{\pgfqpoint{6.456986in}{1.197589in}}{\pgfqpoint{6.452596in}{1.186990in}}{\pgfqpoint{6.452596in}{1.175940in}}%
\pgfpathcurveto{\pgfqpoint{6.452596in}{1.164890in}}{\pgfqpoint{6.456986in}{1.154291in}}{\pgfqpoint{6.464800in}{1.146477in}}%
\pgfpathcurveto{\pgfqpoint{6.472613in}{1.138664in}}{\pgfqpoint{6.483212in}{1.134274in}}{\pgfqpoint{6.494262in}{1.134274in}}%
\pgfpathclose%
\pgfusepath{stroke,fill}%
\end{pgfscope}%
\begin{pgfscope}%
\pgfpathrectangle{\pgfqpoint{0.526127in}{0.331635in}}{\pgfqpoint{9.300000in}{7.700000in}}%
\pgfusepath{clip}%
\pgfsetbuttcap%
\pgfsetroundjoin%
\definecolor{currentfill}{rgb}{0.552941,0.898039,0.631373}%
\pgfsetfillcolor{currentfill}%
\pgfsetlinewidth{0.481800pt}%
\definecolor{currentstroke}{rgb}{1.000000,1.000000,1.000000}%
\pgfsetstrokecolor{currentstroke}%
\pgfsetdash{}{0pt}%
\pgfpathmoveto{\pgfqpoint{5.499012in}{2.290551in}}%
\pgfpathcurveto{\pgfqpoint{5.510062in}{2.290551in}}{\pgfqpoint{5.520661in}{2.294941in}}{\pgfqpoint{5.528474in}{2.302755in}}%
\pgfpathcurveto{\pgfqpoint{5.536288in}{2.310568in}}{\pgfqpoint{5.540678in}{2.321167in}}{\pgfqpoint{5.540678in}{2.332218in}}%
\pgfpathcurveto{\pgfqpoint{5.540678in}{2.343268in}}{\pgfqpoint{5.536288in}{2.353867in}}{\pgfqpoint{5.528474in}{2.361680in}}%
\pgfpathcurveto{\pgfqpoint{5.520661in}{2.369494in}}{\pgfqpoint{5.510062in}{2.373884in}}{\pgfqpoint{5.499012in}{2.373884in}}%
\pgfpathcurveto{\pgfqpoint{5.487961in}{2.373884in}}{\pgfqpoint{5.477362in}{2.369494in}}{\pgfqpoint{5.469549in}{2.361680in}}%
\pgfpathcurveto{\pgfqpoint{5.461735in}{2.353867in}}{\pgfqpoint{5.457345in}{2.343268in}}{\pgfqpoint{5.457345in}{2.332218in}}%
\pgfpathcurveto{\pgfqpoint{5.457345in}{2.321167in}}{\pgfqpoint{5.461735in}{2.310568in}}{\pgfqpoint{5.469549in}{2.302755in}}%
\pgfpathcurveto{\pgfqpoint{5.477362in}{2.294941in}}{\pgfqpoint{5.487961in}{2.290551in}}{\pgfqpoint{5.499012in}{2.290551in}}%
\pgfpathclose%
\pgfusepath{stroke,fill}%
\end{pgfscope}%
\begin{pgfscope}%
\pgfpathrectangle{\pgfqpoint{0.526127in}{0.331635in}}{\pgfqpoint{9.300000in}{7.700000in}}%
\pgfusepath{clip}%
\pgfsetbuttcap%
\pgfsetroundjoin%
\definecolor{currentfill}{rgb}{0.552941,0.898039,0.631373}%
\pgfsetfillcolor{currentfill}%
\pgfsetlinewidth{0.481800pt}%
\definecolor{currentstroke}{rgb}{1.000000,1.000000,1.000000}%
\pgfsetstrokecolor{currentstroke}%
\pgfsetdash{}{0pt}%
\pgfpathmoveto{\pgfqpoint{6.082548in}{1.279265in}}%
\pgfpathcurveto{\pgfqpoint{6.093598in}{1.279265in}}{\pgfqpoint{6.104197in}{1.283655in}}{\pgfqpoint{6.112011in}{1.291469in}}%
\pgfpathcurveto{\pgfqpoint{6.119825in}{1.299282in}}{\pgfqpoint{6.124215in}{1.309881in}}{\pgfqpoint{6.124215in}{1.320932in}}%
\pgfpathcurveto{\pgfqpoint{6.124215in}{1.331982in}}{\pgfqpoint{6.119825in}{1.342581in}}{\pgfqpoint{6.112011in}{1.350394in}}%
\pgfpathcurveto{\pgfqpoint{6.104197in}{1.358208in}}{\pgfqpoint{6.093598in}{1.362598in}}{\pgfqpoint{6.082548in}{1.362598in}}%
\pgfpathcurveto{\pgfqpoint{6.071498in}{1.362598in}}{\pgfqpoint{6.060899in}{1.358208in}}{\pgfqpoint{6.053086in}{1.350394in}}%
\pgfpathcurveto{\pgfqpoint{6.045272in}{1.342581in}}{\pgfqpoint{6.040882in}{1.331982in}}{\pgfqpoint{6.040882in}{1.320932in}}%
\pgfpathcurveto{\pgfqpoint{6.040882in}{1.309881in}}{\pgfqpoint{6.045272in}{1.299282in}}{\pgfqpoint{6.053086in}{1.291469in}}%
\pgfpathcurveto{\pgfqpoint{6.060899in}{1.283655in}}{\pgfqpoint{6.071498in}{1.279265in}}{\pgfqpoint{6.082548in}{1.279265in}}%
\pgfpathclose%
\pgfusepath{stroke,fill}%
\end{pgfscope}%
\begin{pgfscope}%
\pgfpathrectangle{\pgfqpoint{0.526127in}{0.331635in}}{\pgfqpoint{9.300000in}{7.700000in}}%
\pgfusepath{clip}%
\pgfsetbuttcap%
\pgfsetroundjoin%
\definecolor{currentfill}{rgb}{0.552941,0.898039,0.631373}%
\pgfsetfillcolor{currentfill}%
\pgfsetlinewidth{0.481800pt}%
\definecolor{currentstroke}{rgb}{1.000000,1.000000,1.000000}%
\pgfsetstrokecolor{currentstroke}%
\pgfsetdash{}{0pt}%
\pgfpathmoveto{\pgfqpoint{1.873484in}{3.865802in}}%
\pgfpathcurveto{\pgfqpoint{1.884534in}{3.865802in}}{\pgfqpoint{1.895133in}{3.870192in}}{\pgfqpoint{1.902947in}{3.878006in}}%
\pgfpathcurveto{\pgfqpoint{1.910760in}{3.885819in}}{\pgfqpoint{1.915151in}{3.896419in}}{\pgfqpoint{1.915151in}{3.907469in}}%
\pgfpathcurveto{\pgfqpoint{1.915151in}{3.918519in}}{\pgfqpoint{1.910760in}{3.929118in}}{\pgfqpoint{1.902947in}{3.936931in}}%
\pgfpathcurveto{\pgfqpoint{1.895133in}{3.944745in}}{\pgfqpoint{1.884534in}{3.949135in}}{\pgfqpoint{1.873484in}{3.949135in}}%
\pgfpathcurveto{\pgfqpoint{1.862434in}{3.949135in}}{\pgfqpoint{1.851835in}{3.944745in}}{\pgfqpoint{1.844021in}{3.936931in}}%
\pgfpathcurveto{\pgfqpoint{1.836207in}{3.929118in}}{\pgfqpoint{1.831817in}{3.918519in}}{\pgfqpoint{1.831817in}{3.907469in}}%
\pgfpathcurveto{\pgfqpoint{1.831817in}{3.896419in}}{\pgfqpoint{1.836207in}{3.885819in}}{\pgfqpoint{1.844021in}{3.878006in}}%
\pgfpathcurveto{\pgfqpoint{1.851835in}{3.870192in}}{\pgfqpoint{1.862434in}{3.865802in}}{\pgfqpoint{1.873484in}{3.865802in}}%
\pgfpathclose%
\pgfusepath{stroke,fill}%
\end{pgfscope}%
\begin{pgfscope}%
\pgfpathrectangle{\pgfqpoint{0.526127in}{0.331635in}}{\pgfqpoint{9.300000in}{7.700000in}}%
\pgfusepath{clip}%
\pgfsetbuttcap%
\pgfsetroundjoin%
\definecolor{currentfill}{rgb}{0.552941,0.898039,0.631373}%
\pgfsetfillcolor{currentfill}%
\pgfsetlinewidth{0.481800pt}%
\definecolor{currentstroke}{rgb}{1.000000,1.000000,1.000000}%
\pgfsetstrokecolor{currentstroke}%
\pgfsetdash{}{0pt}%
\pgfpathmoveto{\pgfqpoint{5.027322in}{2.986482in}}%
\pgfpathcurveto{\pgfqpoint{5.038372in}{2.986482in}}{\pgfqpoint{5.048971in}{2.990872in}}{\pgfqpoint{5.056784in}{2.998686in}}%
\pgfpathcurveto{\pgfqpoint{5.064598in}{3.006499in}}{\pgfqpoint{5.068988in}{3.017098in}}{\pgfqpoint{5.068988in}{3.028149in}}%
\pgfpathcurveto{\pgfqpoint{5.068988in}{3.039199in}}{\pgfqpoint{5.064598in}{3.049798in}}{\pgfqpoint{5.056784in}{3.057611in}}%
\pgfpathcurveto{\pgfqpoint{5.048971in}{3.065425in}}{\pgfqpoint{5.038372in}{3.069815in}}{\pgfqpoint{5.027322in}{3.069815in}}%
\pgfpathcurveto{\pgfqpoint{5.016272in}{3.069815in}}{\pgfqpoint{5.005673in}{3.065425in}}{\pgfqpoint{4.997859in}{3.057611in}}%
\pgfpathcurveto{\pgfqpoint{4.990045in}{3.049798in}}{\pgfqpoint{4.985655in}{3.039199in}}{\pgfqpoint{4.985655in}{3.028149in}}%
\pgfpathcurveto{\pgfqpoint{4.985655in}{3.017098in}}{\pgfqpoint{4.990045in}{3.006499in}}{\pgfqpoint{4.997859in}{2.998686in}}%
\pgfpathcurveto{\pgfqpoint{5.005673in}{2.990872in}}{\pgfqpoint{5.016272in}{2.986482in}}{\pgfqpoint{5.027322in}{2.986482in}}%
\pgfpathclose%
\pgfusepath{stroke,fill}%
\end{pgfscope}%
\begin{pgfscope}%
\pgfpathrectangle{\pgfqpoint{0.526127in}{0.331635in}}{\pgfqpoint{9.300000in}{7.700000in}}%
\pgfusepath{clip}%
\pgfsetbuttcap%
\pgfsetroundjoin%
\definecolor{currentfill}{rgb}{0.552941,0.898039,0.631373}%
\pgfsetfillcolor{currentfill}%
\pgfsetlinewidth{0.481800pt}%
\definecolor{currentstroke}{rgb}{1.000000,1.000000,1.000000}%
\pgfsetstrokecolor{currentstroke}%
\pgfsetdash{}{0pt}%
\pgfpathmoveto{\pgfqpoint{1.754896in}{3.458290in}}%
\pgfpathcurveto{\pgfqpoint{1.765946in}{3.458290in}}{\pgfqpoint{1.776545in}{3.462680in}}{\pgfqpoint{1.784359in}{3.470494in}}%
\pgfpathcurveto{\pgfqpoint{1.792173in}{3.478307in}}{\pgfqpoint{1.796563in}{3.488906in}}{\pgfqpoint{1.796563in}{3.499956in}}%
\pgfpathcurveto{\pgfqpoint{1.796563in}{3.511007in}}{\pgfqpoint{1.792173in}{3.521606in}}{\pgfqpoint{1.784359in}{3.529419in}}%
\pgfpathcurveto{\pgfqpoint{1.776545in}{3.537233in}}{\pgfqpoint{1.765946in}{3.541623in}}{\pgfqpoint{1.754896in}{3.541623in}}%
\pgfpathcurveto{\pgfqpoint{1.743846in}{3.541623in}}{\pgfqpoint{1.733247in}{3.537233in}}{\pgfqpoint{1.725433in}{3.529419in}}%
\pgfpathcurveto{\pgfqpoint{1.717620in}{3.521606in}}{\pgfqpoint{1.713230in}{3.511007in}}{\pgfqpoint{1.713230in}{3.499956in}}%
\pgfpathcurveto{\pgfqpoint{1.713230in}{3.488906in}}{\pgfqpoint{1.717620in}{3.478307in}}{\pgfqpoint{1.725433in}{3.470494in}}%
\pgfpathcurveto{\pgfqpoint{1.733247in}{3.462680in}}{\pgfqpoint{1.743846in}{3.458290in}}{\pgfqpoint{1.754896in}{3.458290in}}%
\pgfpathclose%
\pgfusepath{stroke,fill}%
\end{pgfscope}%
\begin{pgfscope}%
\pgfpathrectangle{\pgfqpoint{0.526127in}{0.331635in}}{\pgfqpoint{9.300000in}{7.700000in}}%
\pgfusepath{clip}%
\pgfsetbuttcap%
\pgfsetroundjoin%
\definecolor{currentfill}{rgb}{0.552941,0.898039,0.631373}%
\pgfsetfillcolor{currentfill}%
\pgfsetlinewidth{0.481800pt}%
\definecolor{currentstroke}{rgb}{1.000000,1.000000,1.000000}%
\pgfsetstrokecolor{currentstroke}%
\pgfsetdash{}{0pt}%
\pgfpathmoveto{\pgfqpoint{3.811390in}{5.303206in}}%
\pgfpathcurveto{\pgfqpoint{3.822441in}{5.303206in}}{\pgfqpoint{3.833040in}{5.307596in}}{\pgfqpoint{3.840853in}{5.315410in}}%
\pgfpathcurveto{\pgfqpoint{3.848667in}{5.323224in}}{\pgfqpoint{3.853057in}{5.333823in}}{\pgfqpoint{3.853057in}{5.344873in}}%
\pgfpathcurveto{\pgfqpoint{3.853057in}{5.355923in}}{\pgfqpoint{3.848667in}{5.366522in}}{\pgfqpoint{3.840853in}{5.374335in}}%
\pgfpathcurveto{\pgfqpoint{3.833040in}{5.382149in}}{\pgfqpoint{3.822441in}{5.386539in}}{\pgfqpoint{3.811390in}{5.386539in}}%
\pgfpathcurveto{\pgfqpoint{3.800340in}{5.386539in}}{\pgfqpoint{3.789741in}{5.382149in}}{\pgfqpoint{3.781928in}{5.374335in}}%
\pgfpathcurveto{\pgfqpoint{3.774114in}{5.366522in}}{\pgfqpoint{3.769724in}{5.355923in}}{\pgfqpoint{3.769724in}{5.344873in}}%
\pgfpathcurveto{\pgfqpoint{3.769724in}{5.333823in}}{\pgfqpoint{3.774114in}{5.323224in}}{\pgfqpoint{3.781928in}{5.315410in}}%
\pgfpathcurveto{\pgfqpoint{3.789741in}{5.307596in}}{\pgfqpoint{3.800340in}{5.303206in}}{\pgfqpoint{3.811390in}{5.303206in}}%
\pgfpathclose%
\pgfusepath{stroke,fill}%
\end{pgfscope}%
\begin{pgfscope}%
\pgfpathrectangle{\pgfqpoint{0.526127in}{0.331635in}}{\pgfqpoint{9.300000in}{7.700000in}}%
\pgfusepath{clip}%
\pgfsetbuttcap%
\pgfsetroundjoin%
\definecolor{currentfill}{rgb}{0.552941,0.898039,0.631373}%
\pgfsetfillcolor{currentfill}%
\pgfsetlinewidth{0.481800pt}%
\definecolor{currentstroke}{rgb}{1.000000,1.000000,1.000000}%
\pgfsetstrokecolor{currentstroke}%
\pgfsetdash{}{0pt}%
\pgfpathmoveto{\pgfqpoint{4.820098in}{3.381921in}}%
\pgfpathcurveto{\pgfqpoint{4.831148in}{3.381921in}}{\pgfqpoint{4.841747in}{3.386311in}}{\pgfqpoint{4.849561in}{3.394125in}}%
\pgfpathcurveto{\pgfqpoint{4.857375in}{3.401939in}}{\pgfqpoint{4.861765in}{3.412538in}}{\pgfqpoint{4.861765in}{3.423588in}}%
\pgfpathcurveto{\pgfqpoint{4.861765in}{3.434638in}}{\pgfqpoint{4.857375in}{3.445237in}}{\pgfqpoint{4.849561in}{3.453051in}}%
\pgfpathcurveto{\pgfqpoint{4.841747in}{3.460864in}}{\pgfqpoint{4.831148in}{3.465255in}}{\pgfqpoint{4.820098in}{3.465255in}}%
\pgfpathcurveto{\pgfqpoint{4.809048in}{3.465255in}}{\pgfqpoint{4.798449in}{3.460864in}}{\pgfqpoint{4.790635in}{3.453051in}}%
\pgfpathcurveto{\pgfqpoint{4.782822in}{3.445237in}}{\pgfqpoint{4.778432in}{3.434638in}}{\pgfqpoint{4.778432in}{3.423588in}}%
\pgfpathcurveto{\pgfqpoint{4.778432in}{3.412538in}}{\pgfqpoint{4.782822in}{3.401939in}}{\pgfqpoint{4.790635in}{3.394125in}}%
\pgfpathcurveto{\pgfqpoint{4.798449in}{3.386311in}}{\pgfqpoint{4.809048in}{3.381921in}}{\pgfqpoint{4.820098in}{3.381921in}}%
\pgfpathclose%
\pgfusepath{stroke,fill}%
\end{pgfscope}%
\begin{pgfscope}%
\pgfpathrectangle{\pgfqpoint{0.526127in}{0.331635in}}{\pgfqpoint{9.300000in}{7.700000in}}%
\pgfusepath{clip}%
\pgfsetbuttcap%
\pgfsetroundjoin%
\definecolor{currentfill}{rgb}{1.000000,0.623529,0.607843}%
\pgfsetfillcolor{currentfill}%
\pgfsetlinewidth{0.481800pt}%
\definecolor{currentstroke}{rgb}{1.000000,1.000000,1.000000}%
\pgfsetstrokecolor{currentstroke}%
\pgfsetdash{}{0pt}%
\pgfpathmoveto{\pgfqpoint{6.036344in}{7.639968in}}%
\pgfpathcurveto{\pgfqpoint{6.047394in}{7.639968in}}{\pgfqpoint{6.057993in}{7.644359in}}{\pgfqpoint{6.065807in}{7.652172in}}%
\pgfpathcurveto{\pgfqpoint{6.073621in}{7.659986in}}{\pgfqpoint{6.078011in}{7.670585in}}{\pgfqpoint{6.078011in}{7.681635in}}%
\pgfpathcurveto{\pgfqpoint{6.078011in}{7.692685in}}{\pgfqpoint{6.073621in}{7.703284in}}{\pgfqpoint{6.065807in}{7.711098in}}%
\pgfpathcurveto{\pgfqpoint{6.057993in}{7.718911in}}{\pgfqpoint{6.047394in}{7.723302in}}{\pgfqpoint{6.036344in}{7.723302in}}%
\pgfpathcurveto{\pgfqpoint{6.025294in}{7.723302in}}{\pgfqpoint{6.014695in}{7.718911in}}{\pgfqpoint{6.006881in}{7.711098in}}%
\pgfpathcurveto{\pgfqpoint{5.999068in}{7.703284in}}{\pgfqpoint{5.994678in}{7.692685in}}{\pgfqpoint{5.994678in}{7.681635in}}%
\pgfpathcurveto{\pgfqpoint{5.994678in}{7.670585in}}{\pgfqpoint{5.999068in}{7.659986in}}{\pgfqpoint{6.006881in}{7.652172in}}%
\pgfpathcurveto{\pgfqpoint{6.014695in}{7.644359in}}{\pgfqpoint{6.025294in}{7.639968in}}{\pgfqpoint{6.036344in}{7.639968in}}%
\pgfpathclose%
\pgfusepath{stroke,fill}%
\end{pgfscope}%
\begin{pgfscope}%
\pgfpathrectangle{\pgfqpoint{0.526127in}{0.331635in}}{\pgfqpoint{9.300000in}{7.700000in}}%
\pgfusepath{clip}%
\pgfsetbuttcap%
\pgfsetroundjoin%
\definecolor{currentfill}{rgb}{1.000000,0.623529,0.607843}%
\pgfsetfillcolor{currentfill}%
\pgfsetlinewidth{0.481800pt}%
\definecolor{currentstroke}{rgb}{1.000000,1.000000,1.000000}%
\pgfsetstrokecolor{currentstroke}%
\pgfsetdash{}{0pt}%
\pgfpathmoveto{\pgfqpoint{5.059438in}{4.695820in}}%
\pgfpathcurveto{\pgfqpoint{5.070488in}{4.695820in}}{\pgfqpoint{5.081087in}{4.700210in}}{\pgfqpoint{5.088900in}{4.708024in}}%
\pgfpathcurveto{\pgfqpoint{5.096714in}{4.715837in}}{\pgfqpoint{5.101104in}{4.726436in}}{\pgfqpoint{5.101104in}{4.737486in}}%
\pgfpathcurveto{\pgfqpoint{5.101104in}{4.748537in}}{\pgfqpoint{5.096714in}{4.759136in}}{\pgfqpoint{5.088900in}{4.766949in}}%
\pgfpathcurveto{\pgfqpoint{5.081087in}{4.774763in}}{\pgfqpoint{5.070488in}{4.779153in}}{\pgfqpoint{5.059438in}{4.779153in}}%
\pgfpathcurveto{\pgfqpoint{5.048387in}{4.779153in}}{\pgfqpoint{5.037788in}{4.774763in}}{\pgfqpoint{5.029975in}{4.766949in}}%
\pgfpathcurveto{\pgfqpoint{5.022161in}{4.759136in}}{\pgfqpoint{5.017771in}{4.748537in}}{\pgfqpoint{5.017771in}{4.737486in}}%
\pgfpathcurveto{\pgfqpoint{5.017771in}{4.726436in}}{\pgfqpoint{5.022161in}{4.715837in}}{\pgfqpoint{5.029975in}{4.708024in}}%
\pgfpathcurveto{\pgfqpoint{5.037788in}{4.700210in}}{\pgfqpoint{5.048387in}{4.695820in}}{\pgfqpoint{5.059438in}{4.695820in}}%
\pgfpathclose%
\pgfusepath{stroke,fill}%
\end{pgfscope}%
\begin{pgfscope}%
\pgfpathrectangle{\pgfqpoint{0.526127in}{0.331635in}}{\pgfqpoint{9.300000in}{7.700000in}}%
\pgfusepath{clip}%
\pgfsetbuttcap%
\pgfsetroundjoin%
\definecolor{currentfill}{rgb}{1.000000,0.623529,0.607843}%
\pgfsetfillcolor{currentfill}%
\pgfsetlinewidth{0.481800pt}%
\definecolor{currentstroke}{rgb}{1.000000,1.000000,1.000000}%
\pgfsetstrokecolor{currentstroke}%
\pgfsetdash{}{0pt}%
\pgfpathmoveto{\pgfqpoint{3.537682in}{4.723912in}}%
\pgfpathcurveto{\pgfqpoint{3.548733in}{4.723912in}}{\pgfqpoint{3.559332in}{4.728303in}}{\pgfqpoint{3.567145in}{4.736116in}}%
\pgfpathcurveto{\pgfqpoint{3.574959in}{4.743930in}}{\pgfqpoint{3.579349in}{4.754529in}}{\pgfqpoint{3.579349in}{4.765579in}}%
\pgfpathcurveto{\pgfqpoint{3.579349in}{4.776629in}}{\pgfqpoint{3.574959in}{4.787228in}}{\pgfqpoint{3.567145in}{4.795042in}}%
\pgfpathcurveto{\pgfqpoint{3.559332in}{4.802856in}}{\pgfqpoint{3.548733in}{4.807246in}}{\pgfqpoint{3.537682in}{4.807246in}}%
\pgfpathcurveto{\pgfqpoint{3.526632in}{4.807246in}}{\pgfqpoint{3.516033in}{4.802856in}}{\pgfqpoint{3.508220in}{4.795042in}}%
\pgfpathcurveto{\pgfqpoint{3.500406in}{4.787228in}}{\pgfqpoint{3.496016in}{4.776629in}}{\pgfqpoint{3.496016in}{4.765579in}}%
\pgfpathcurveto{\pgfqpoint{3.496016in}{4.754529in}}{\pgfqpoint{3.500406in}{4.743930in}}{\pgfqpoint{3.508220in}{4.736116in}}%
\pgfpathcurveto{\pgfqpoint{3.516033in}{4.728303in}}{\pgfqpoint{3.526632in}{4.723912in}}{\pgfqpoint{3.537682in}{4.723912in}}%
\pgfpathclose%
\pgfusepath{stroke,fill}%
\end{pgfscope}%
\begin{pgfscope}%
\pgfpathrectangle{\pgfqpoint{0.526127in}{0.331635in}}{\pgfqpoint{9.300000in}{7.700000in}}%
\pgfusepath{clip}%
\pgfsetbuttcap%
\pgfsetroundjoin%
\definecolor{currentfill}{rgb}{1.000000,0.623529,0.607843}%
\pgfsetfillcolor{currentfill}%
\pgfsetlinewidth{0.481800pt}%
\definecolor{currentstroke}{rgb}{1.000000,1.000000,1.000000}%
\pgfsetstrokecolor{currentstroke}%
\pgfsetdash{}{0pt}%
\pgfpathmoveto{\pgfqpoint{3.454692in}{5.724756in}}%
\pgfpathcurveto{\pgfqpoint{3.465742in}{5.724756in}}{\pgfqpoint{3.476341in}{5.729147in}}{\pgfqpoint{3.484155in}{5.736960in}}%
\pgfpathcurveto{\pgfqpoint{3.491968in}{5.744774in}}{\pgfqpoint{3.496359in}{5.755373in}}{\pgfqpoint{3.496359in}{5.766423in}}%
\pgfpathcurveto{\pgfqpoint{3.496359in}{5.777473in}}{\pgfqpoint{3.491968in}{5.788072in}}{\pgfqpoint{3.484155in}{5.795886in}}%
\pgfpathcurveto{\pgfqpoint{3.476341in}{5.803699in}}{\pgfqpoint{3.465742in}{5.808090in}}{\pgfqpoint{3.454692in}{5.808090in}}%
\pgfpathcurveto{\pgfqpoint{3.443642in}{5.808090in}}{\pgfqpoint{3.433043in}{5.803699in}}{\pgfqpoint{3.425229in}{5.795886in}}%
\pgfpathcurveto{\pgfqpoint{3.417416in}{5.788072in}}{\pgfqpoint{3.413025in}{5.777473in}}{\pgfqpoint{3.413025in}{5.766423in}}%
\pgfpathcurveto{\pgfqpoint{3.413025in}{5.755373in}}{\pgfqpoint{3.417416in}{5.744774in}}{\pgfqpoint{3.425229in}{5.736960in}}%
\pgfpathcurveto{\pgfqpoint{3.433043in}{5.729147in}}{\pgfqpoint{3.443642in}{5.724756in}}{\pgfqpoint{3.454692in}{5.724756in}}%
\pgfpathclose%
\pgfusepath{stroke,fill}%
\end{pgfscope}%
\begin{pgfscope}%
\pgfpathrectangle{\pgfqpoint{0.526127in}{0.331635in}}{\pgfqpoint{9.300000in}{7.700000in}}%
\pgfusepath{clip}%
\pgfsetbuttcap%
\pgfsetroundjoin%
\definecolor{currentfill}{rgb}{1.000000,0.623529,0.607843}%
\pgfsetfillcolor{currentfill}%
\pgfsetlinewidth{0.481800pt}%
\definecolor{currentstroke}{rgb}{1.000000,1.000000,1.000000}%
\pgfsetstrokecolor{currentstroke}%
\pgfsetdash{}{0pt}%
\pgfpathmoveto{\pgfqpoint{5.176399in}{7.364906in}}%
\pgfpathcurveto{\pgfqpoint{5.187449in}{7.364906in}}{\pgfqpoint{5.198048in}{7.369297in}}{\pgfqpoint{5.205862in}{7.377110in}}%
\pgfpathcurveto{\pgfqpoint{5.213675in}{7.384924in}}{\pgfqpoint{5.218066in}{7.395523in}}{\pgfqpoint{5.218066in}{7.406573in}}%
\pgfpathcurveto{\pgfqpoint{5.218066in}{7.417623in}}{\pgfqpoint{5.213675in}{7.428222in}}{\pgfqpoint{5.205862in}{7.436036in}}%
\pgfpathcurveto{\pgfqpoint{5.198048in}{7.443849in}}{\pgfqpoint{5.187449in}{7.448240in}}{\pgfqpoint{5.176399in}{7.448240in}}%
\pgfpathcurveto{\pgfqpoint{5.165349in}{7.448240in}}{\pgfqpoint{5.154750in}{7.443849in}}{\pgfqpoint{5.146936in}{7.436036in}}%
\pgfpathcurveto{\pgfqpoint{5.139123in}{7.428222in}}{\pgfqpoint{5.134732in}{7.417623in}}{\pgfqpoint{5.134732in}{7.406573in}}%
\pgfpathcurveto{\pgfqpoint{5.134732in}{7.395523in}}{\pgfqpoint{5.139123in}{7.384924in}}{\pgfqpoint{5.146936in}{7.377110in}}%
\pgfpathcurveto{\pgfqpoint{5.154750in}{7.369297in}}{\pgfqpoint{5.165349in}{7.364906in}}{\pgfqpoint{5.176399in}{7.364906in}}%
\pgfpathclose%
\pgfusepath{stroke,fill}%
\end{pgfscope}%
\begin{pgfscope}%
\pgfpathrectangle{\pgfqpoint{0.526127in}{0.331635in}}{\pgfqpoint{9.300000in}{7.700000in}}%
\pgfusepath{clip}%
\pgfsetbuttcap%
\pgfsetroundjoin%
\definecolor{currentfill}{rgb}{1.000000,0.623529,0.607843}%
\pgfsetfillcolor{currentfill}%
\pgfsetlinewidth{0.481800pt}%
\definecolor{currentstroke}{rgb}{1.000000,1.000000,1.000000}%
\pgfsetstrokecolor{currentstroke}%
\pgfsetdash{}{0pt}%
\pgfpathmoveto{\pgfqpoint{5.896285in}{3.450847in}}%
\pgfpathcurveto{\pgfqpoint{5.907335in}{3.450847in}}{\pgfqpoint{5.917934in}{3.455238in}}{\pgfqpoint{5.925748in}{3.463051in}}%
\pgfpathcurveto{\pgfqpoint{5.933562in}{3.470865in}}{\pgfqpoint{5.937952in}{3.481464in}}{\pgfqpoint{5.937952in}{3.492514in}}%
\pgfpathcurveto{\pgfqpoint{5.937952in}{3.503564in}}{\pgfqpoint{5.933562in}{3.514163in}}{\pgfqpoint{5.925748in}{3.521977in}}%
\pgfpathcurveto{\pgfqpoint{5.917934in}{3.529790in}}{\pgfqpoint{5.907335in}{3.534181in}}{\pgfqpoint{5.896285in}{3.534181in}}%
\pgfpathcurveto{\pgfqpoint{5.885235in}{3.534181in}}{\pgfqpoint{5.874636in}{3.529790in}}{\pgfqpoint{5.866822in}{3.521977in}}%
\pgfpathcurveto{\pgfqpoint{5.859009in}{3.514163in}}{\pgfqpoint{5.854618in}{3.503564in}}{\pgfqpoint{5.854618in}{3.492514in}}%
\pgfpathcurveto{\pgfqpoint{5.854618in}{3.481464in}}{\pgfqpoint{5.859009in}{3.470865in}}{\pgfqpoint{5.866822in}{3.463051in}}%
\pgfpathcurveto{\pgfqpoint{5.874636in}{3.455238in}}{\pgfqpoint{5.885235in}{3.450847in}}{\pgfqpoint{5.896285in}{3.450847in}}%
\pgfpathclose%
\pgfusepath{stroke,fill}%
\end{pgfscope}%
\begin{pgfscope}%
\pgfpathrectangle{\pgfqpoint{0.526127in}{0.331635in}}{\pgfqpoint{9.300000in}{7.700000in}}%
\pgfusepath{clip}%
\pgfsetbuttcap%
\pgfsetroundjoin%
\definecolor{currentfill}{rgb}{1.000000,0.623529,0.607843}%
\pgfsetfillcolor{currentfill}%
\pgfsetlinewidth{0.481800pt}%
\definecolor{currentstroke}{rgb}{1.000000,1.000000,1.000000}%
\pgfsetstrokecolor{currentstroke}%
\pgfsetdash{}{0pt}%
\pgfpathmoveto{\pgfqpoint{5.765801in}{5.086687in}}%
\pgfpathcurveto{\pgfqpoint{5.776851in}{5.086687in}}{\pgfqpoint{5.787450in}{5.091077in}}{\pgfqpoint{5.795264in}{5.098891in}}%
\pgfpathcurveto{\pgfqpoint{5.803077in}{5.106705in}}{\pgfqpoint{5.807467in}{5.117304in}}{\pgfqpoint{5.807467in}{5.128354in}}%
\pgfpathcurveto{\pgfqpoint{5.807467in}{5.139404in}}{\pgfqpoint{5.803077in}{5.150003in}}{\pgfqpoint{5.795264in}{5.157817in}}%
\pgfpathcurveto{\pgfqpoint{5.787450in}{5.165630in}}{\pgfqpoint{5.776851in}{5.170020in}}{\pgfqpoint{5.765801in}{5.170020in}}%
\pgfpathcurveto{\pgfqpoint{5.754751in}{5.170020in}}{\pgfqpoint{5.744152in}{5.165630in}}{\pgfqpoint{5.736338in}{5.157817in}}%
\pgfpathcurveto{\pgfqpoint{5.728524in}{5.150003in}}{\pgfqpoint{5.724134in}{5.139404in}}{\pgfqpoint{5.724134in}{5.128354in}}%
\pgfpathcurveto{\pgfqpoint{5.724134in}{5.117304in}}{\pgfqpoint{5.728524in}{5.106705in}}{\pgfqpoint{5.736338in}{5.098891in}}%
\pgfpathcurveto{\pgfqpoint{5.744152in}{5.091077in}}{\pgfqpoint{5.754751in}{5.086687in}}{\pgfqpoint{5.765801in}{5.086687in}}%
\pgfpathclose%
\pgfusepath{stroke,fill}%
\end{pgfscope}%
\begin{pgfscope}%
\pgfpathrectangle{\pgfqpoint{0.526127in}{0.331635in}}{\pgfqpoint{9.300000in}{7.700000in}}%
\pgfusepath{clip}%
\pgfsetbuttcap%
\pgfsetroundjoin%
\definecolor{currentfill}{rgb}{1.000000,0.623529,0.607843}%
\pgfsetfillcolor{currentfill}%
\pgfsetlinewidth{0.481800pt}%
\definecolor{currentstroke}{rgb}{1.000000,1.000000,1.000000}%
\pgfsetstrokecolor{currentstroke}%
\pgfsetdash{}{0pt}%
\pgfpathmoveto{\pgfqpoint{3.919124in}{4.234069in}}%
\pgfpathcurveto{\pgfqpoint{3.930174in}{4.234069in}}{\pgfqpoint{3.940773in}{4.238460in}}{\pgfqpoint{3.948587in}{4.246273in}}%
\pgfpathcurveto{\pgfqpoint{3.956400in}{4.254087in}}{\pgfqpoint{3.960791in}{4.264686in}}{\pgfqpoint{3.960791in}{4.275736in}}%
\pgfpathcurveto{\pgfqpoint{3.960791in}{4.286786in}}{\pgfqpoint{3.956400in}{4.297385in}}{\pgfqpoint{3.948587in}{4.305199in}}%
\pgfpathcurveto{\pgfqpoint{3.940773in}{4.313013in}}{\pgfqpoint{3.930174in}{4.317403in}}{\pgfqpoint{3.919124in}{4.317403in}}%
\pgfpathcurveto{\pgfqpoint{3.908074in}{4.317403in}}{\pgfqpoint{3.897475in}{4.313013in}}{\pgfqpoint{3.889661in}{4.305199in}}%
\pgfpathcurveto{\pgfqpoint{3.881848in}{4.297385in}}{\pgfqpoint{3.877457in}{4.286786in}}{\pgfqpoint{3.877457in}{4.275736in}}%
\pgfpathcurveto{\pgfqpoint{3.877457in}{4.264686in}}{\pgfqpoint{3.881848in}{4.254087in}}{\pgfqpoint{3.889661in}{4.246273in}}%
\pgfpathcurveto{\pgfqpoint{3.897475in}{4.238460in}}{\pgfqpoint{3.908074in}{4.234069in}}{\pgfqpoint{3.919124in}{4.234069in}}%
\pgfpathclose%
\pgfusepath{stroke,fill}%
\end{pgfscope}%
\begin{pgfscope}%
\pgfpathrectangle{\pgfqpoint{0.526127in}{0.331635in}}{\pgfqpoint{9.300000in}{7.700000in}}%
\pgfusepath{clip}%
\pgfsetbuttcap%
\pgfsetroundjoin%
\definecolor{currentfill}{rgb}{1.000000,0.623529,0.607843}%
\pgfsetfillcolor{currentfill}%
\pgfsetlinewidth{0.481800pt}%
\definecolor{currentstroke}{rgb}{1.000000,1.000000,1.000000}%
\pgfsetstrokecolor{currentstroke}%
\pgfsetdash{}{0pt}%
\pgfpathmoveto{\pgfqpoint{7.219680in}{3.360479in}}%
\pgfpathcurveto{\pgfqpoint{7.230730in}{3.360479in}}{\pgfqpoint{7.241329in}{3.364869in}}{\pgfqpoint{7.249143in}{3.372683in}}%
\pgfpathcurveto{\pgfqpoint{7.256956in}{3.380497in}}{\pgfqpoint{7.261346in}{3.391096in}}{\pgfqpoint{7.261346in}{3.402146in}}%
\pgfpathcurveto{\pgfqpoint{7.261346in}{3.413196in}}{\pgfqpoint{7.256956in}{3.423795in}}{\pgfqpoint{7.249143in}{3.431609in}}%
\pgfpathcurveto{\pgfqpoint{7.241329in}{3.439422in}}{\pgfqpoint{7.230730in}{3.443812in}}{\pgfqpoint{7.219680in}{3.443812in}}%
\pgfpathcurveto{\pgfqpoint{7.208630in}{3.443812in}}{\pgfqpoint{7.198031in}{3.439422in}}{\pgfqpoint{7.190217in}{3.431609in}}%
\pgfpathcurveto{\pgfqpoint{7.182403in}{3.423795in}}{\pgfqpoint{7.178013in}{3.413196in}}{\pgfqpoint{7.178013in}{3.402146in}}%
\pgfpathcurveto{\pgfqpoint{7.178013in}{3.391096in}}{\pgfqpoint{7.182403in}{3.380497in}}{\pgfqpoint{7.190217in}{3.372683in}}%
\pgfpathcurveto{\pgfqpoint{7.198031in}{3.364869in}}{\pgfqpoint{7.208630in}{3.360479in}}{\pgfqpoint{7.219680in}{3.360479in}}%
\pgfpathclose%
\pgfusepath{stroke,fill}%
\end{pgfscope}%
\begin{pgfscope}%
\pgfpathrectangle{\pgfqpoint{0.526127in}{0.331635in}}{\pgfqpoint{9.300000in}{7.700000in}}%
\pgfusepath{clip}%
\pgfsetbuttcap%
\pgfsetroundjoin%
\definecolor{currentfill}{rgb}{1.000000,0.623529,0.607843}%
\pgfsetfillcolor{currentfill}%
\pgfsetlinewidth{0.481800pt}%
\definecolor{currentstroke}{rgb}{1.000000,1.000000,1.000000}%
\pgfsetstrokecolor{currentstroke}%
\pgfsetdash{}{0pt}%
\pgfpathmoveto{\pgfqpoint{4.459787in}{4.742341in}}%
\pgfpathcurveto{\pgfqpoint{4.470837in}{4.742341in}}{\pgfqpoint{4.481437in}{4.746732in}}{\pgfqpoint{4.489250in}{4.754545in}}%
\pgfpathcurveto{\pgfqpoint{4.497064in}{4.762359in}}{\pgfqpoint{4.501454in}{4.772958in}}{\pgfqpoint{4.501454in}{4.784008in}}%
\pgfpathcurveto{\pgfqpoint{4.501454in}{4.795058in}}{\pgfqpoint{4.497064in}{4.805657in}}{\pgfqpoint{4.489250in}{4.813471in}}%
\pgfpathcurveto{\pgfqpoint{4.481437in}{4.821284in}}{\pgfqpoint{4.470837in}{4.825675in}}{\pgfqpoint{4.459787in}{4.825675in}}%
\pgfpathcurveto{\pgfqpoint{4.448737in}{4.825675in}}{\pgfqpoint{4.438138in}{4.821284in}}{\pgfqpoint{4.430325in}{4.813471in}}%
\pgfpathcurveto{\pgfqpoint{4.422511in}{4.805657in}}{\pgfqpoint{4.418121in}{4.795058in}}{\pgfqpoint{4.418121in}{4.784008in}}%
\pgfpathcurveto{\pgfqpoint{4.418121in}{4.772958in}}{\pgfqpoint{4.422511in}{4.762359in}}{\pgfqpoint{4.430325in}{4.754545in}}%
\pgfpathcurveto{\pgfqpoint{4.438138in}{4.746732in}}{\pgfqpoint{4.448737in}{4.742341in}}{\pgfqpoint{4.459787in}{4.742341in}}%
\pgfpathclose%
\pgfusepath{stroke,fill}%
\end{pgfscope}%
\begin{pgfscope}%
\pgfpathrectangle{\pgfqpoint{0.526127in}{0.331635in}}{\pgfqpoint{9.300000in}{7.700000in}}%
\pgfusepath{clip}%
\pgfsetbuttcap%
\pgfsetroundjoin%
\definecolor{currentfill}{rgb}{1.000000,0.623529,0.607843}%
\pgfsetfillcolor{currentfill}%
\pgfsetlinewidth{0.481800pt}%
\definecolor{currentstroke}{rgb}{1.000000,1.000000,1.000000}%
\pgfsetstrokecolor{currentstroke}%
\pgfsetdash{}{0pt}%
\pgfpathmoveto{\pgfqpoint{7.438369in}{3.950682in}}%
\pgfpathcurveto{\pgfqpoint{7.449419in}{3.950682in}}{\pgfqpoint{7.460018in}{3.955072in}}{\pgfqpoint{7.467832in}{3.962886in}}%
\pgfpathcurveto{\pgfqpoint{7.475645in}{3.970700in}}{\pgfqpoint{7.480035in}{3.981299in}}{\pgfqpoint{7.480035in}{3.992349in}}%
\pgfpathcurveto{\pgfqpoint{7.480035in}{4.003399in}}{\pgfqpoint{7.475645in}{4.013998in}}{\pgfqpoint{7.467832in}{4.021812in}}%
\pgfpathcurveto{\pgfqpoint{7.460018in}{4.029625in}}{\pgfqpoint{7.449419in}{4.034016in}}{\pgfqpoint{7.438369in}{4.034016in}}%
\pgfpathcurveto{\pgfqpoint{7.427319in}{4.034016in}}{\pgfqpoint{7.416720in}{4.029625in}}{\pgfqpoint{7.408906in}{4.021812in}}%
\pgfpathcurveto{\pgfqpoint{7.401092in}{4.013998in}}{\pgfqpoint{7.396702in}{4.003399in}}{\pgfqpoint{7.396702in}{3.992349in}}%
\pgfpathcurveto{\pgfqpoint{7.396702in}{3.981299in}}{\pgfqpoint{7.401092in}{3.970700in}}{\pgfqpoint{7.408906in}{3.962886in}}%
\pgfpathcurveto{\pgfqpoint{7.416720in}{3.955072in}}{\pgfqpoint{7.427319in}{3.950682in}}{\pgfqpoint{7.438369in}{3.950682in}}%
\pgfpathclose%
\pgfusepath{stroke,fill}%
\end{pgfscope}%
\begin{pgfscope}%
\pgfpathrectangle{\pgfqpoint{0.526127in}{0.331635in}}{\pgfqpoint{9.300000in}{7.700000in}}%
\pgfusepath{clip}%
\pgfsetbuttcap%
\pgfsetroundjoin%
\definecolor{currentfill}{rgb}{1.000000,0.623529,0.607843}%
\pgfsetfillcolor{currentfill}%
\pgfsetlinewidth{0.481800pt}%
\definecolor{currentstroke}{rgb}{1.000000,1.000000,1.000000}%
\pgfsetstrokecolor{currentstroke}%
\pgfsetdash{}{0pt}%
\pgfpathmoveto{\pgfqpoint{4.906391in}{4.922709in}}%
\pgfpathcurveto{\pgfqpoint{4.917441in}{4.922709in}}{\pgfqpoint{4.928040in}{4.927099in}}{\pgfqpoint{4.935854in}{4.934913in}}%
\pgfpathcurveto{\pgfqpoint{4.943667in}{4.942726in}}{\pgfqpoint{4.948058in}{4.953326in}}{\pgfqpoint{4.948058in}{4.964376in}}%
\pgfpathcurveto{\pgfqpoint{4.948058in}{4.975426in}}{\pgfqpoint{4.943667in}{4.986025in}}{\pgfqpoint{4.935854in}{4.993838in}}%
\pgfpathcurveto{\pgfqpoint{4.928040in}{5.001652in}}{\pgfqpoint{4.917441in}{5.006042in}}{\pgfqpoint{4.906391in}{5.006042in}}%
\pgfpathcurveto{\pgfqpoint{4.895341in}{5.006042in}}{\pgfqpoint{4.884742in}{5.001652in}}{\pgfqpoint{4.876928in}{4.993838in}}%
\pgfpathcurveto{\pgfqpoint{4.869115in}{4.986025in}}{\pgfqpoint{4.864724in}{4.975426in}}{\pgfqpoint{4.864724in}{4.964376in}}%
\pgfpathcurveto{\pgfqpoint{4.864724in}{4.953326in}}{\pgfqpoint{4.869115in}{4.942726in}}{\pgfqpoint{4.876928in}{4.934913in}}%
\pgfpathcurveto{\pgfqpoint{4.884742in}{4.927099in}}{\pgfqpoint{4.895341in}{4.922709in}}{\pgfqpoint{4.906391in}{4.922709in}}%
\pgfpathclose%
\pgfusepath{stroke,fill}%
\end{pgfscope}%
\begin{pgfscope}%
\pgfpathrectangle{\pgfqpoint{0.526127in}{0.331635in}}{\pgfqpoint{9.300000in}{7.700000in}}%
\pgfusepath{clip}%
\pgfsetbuttcap%
\pgfsetroundjoin%
\definecolor{currentfill}{rgb}{1.000000,0.623529,0.607843}%
\pgfsetfillcolor{currentfill}%
\pgfsetlinewidth{0.481800pt}%
\definecolor{currentstroke}{rgb}{1.000000,1.000000,1.000000}%
\pgfsetstrokecolor{currentstroke}%
\pgfsetdash{}{0pt}%
\pgfpathmoveto{\pgfqpoint{5.942000in}{5.870150in}}%
\pgfpathcurveto{\pgfqpoint{5.953050in}{5.870150in}}{\pgfqpoint{5.963649in}{5.874540in}}{\pgfqpoint{5.971463in}{5.882354in}}%
\pgfpathcurveto{\pgfqpoint{5.979276in}{5.890167in}}{\pgfqpoint{5.983666in}{5.900766in}}{\pgfqpoint{5.983666in}{5.911816in}}%
\pgfpathcurveto{\pgfqpoint{5.983666in}{5.922867in}}{\pgfqpoint{5.979276in}{5.933466in}}{\pgfqpoint{5.971463in}{5.941279in}}%
\pgfpathcurveto{\pgfqpoint{5.963649in}{5.949093in}}{\pgfqpoint{5.953050in}{5.953483in}}{\pgfqpoint{5.942000in}{5.953483in}}%
\pgfpathcurveto{\pgfqpoint{5.930950in}{5.953483in}}{\pgfqpoint{5.920351in}{5.949093in}}{\pgfqpoint{5.912537in}{5.941279in}}%
\pgfpathcurveto{\pgfqpoint{5.904723in}{5.933466in}}{\pgfqpoint{5.900333in}{5.922867in}}{\pgfqpoint{5.900333in}{5.911816in}}%
\pgfpathcurveto{\pgfqpoint{5.900333in}{5.900766in}}{\pgfqpoint{5.904723in}{5.890167in}}{\pgfqpoint{5.912537in}{5.882354in}}%
\pgfpathcurveto{\pgfqpoint{5.920351in}{5.874540in}}{\pgfqpoint{5.930950in}{5.870150in}}{\pgfqpoint{5.942000in}{5.870150in}}%
\pgfpathclose%
\pgfusepath{stroke,fill}%
\end{pgfscope}%
\begin{pgfscope}%
\pgfpathrectangle{\pgfqpoint{0.526127in}{0.331635in}}{\pgfqpoint{9.300000in}{7.700000in}}%
\pgfusepath{clip}%
\pgfsetbuttcap%
\pgfsetroundjoin%
\definecolor{currentfill}{rgb}{1.000000,0.623529,0.607843}%
\pgfsetfillcolor{currentfill}%
\pgfsetlinewidth{0.481800pt}%
\definecolor{currentstroke}{rgb}{1.000000,1.000000,1.000000}%
\pgfsetstrokecolor{currentstroke}%
\pgfsetdash{}{0pt}%
\pgfpathmoveto{\pgfqpoint{5.122725in}{5.229954in}}%
\pgfpathcurveto{\pgfqpoint{5.133775in}{5.229954in}}{\pgfqpoint{5.144374in}{5.234344in}}{\pgfqpoint{5.152188in}{5.242157in}}%
\pgfpathcurveto{\pgfqpoint{5.160002in}{5.249971in}}{\pgfqpoint{5.164392in}{5.260570in}}{\pgfqpoint{5.164392in}{5.271620in}}%
\pgfpathcurveto{\pgfqpoint{5.164392in}{5.282670in}}{\pgfqpoint{5.160002in}{5.293269in}}{\pgfqpoint{5.152188in}{5.301083in}}%
\pgfpathcurveto{\pgfqpoint{5.144374in}{5.308897in}}{\pgfqpoint{5.133775in}{5.313287in}}{\pgfqpoint{5.122725in}{5.313287in}}%
\pgfpathcurveto{\pgfqpoint{5.111675in}{5.313287in}}{\pgfqpoint{5.101076in}{5.308897in}}{\pgfqpoint{5.093263in}{5.301083in}}%
\pgfpathcurveto{\pgfqpoint{5.085449in}{5.293269in}}{\pgfqpoint{5.081059in}{5.282670in}}{\pgfqpoint{5.081059in}{5.271620in}}%
\pgfpathcurveto{\pgfqpoint{5.081059in}{5.260570in}}{\pgfqpoint{5.085449in}{5.249971in}}{\pgfqpoint{5.093263in}{5.242157in}}%
\pgfpathcurveto{\pgfqpoint{5.101076in}{5.234344in}}{\pgfqpoint{5.111675in}{5.229954in}}{\pgfqpoint{5.122725in}{5.229954in}}%
\pgfpathclose%
\pgfusepath{stroke,fill}%
\end{pgfscope}%
\begin{pgfscope}%
\pgfpathrectangle{\pgfqpoint{0.526127in}{0.331635in}}{\pgfqpoint{9.300000in}{7.700000in}}%
\pgfusepath{clip}%
\pgfsetbuttcap%
\pgfsetroundjoin%
\definecolor{currentfill}{rgb}{1.000000,0.623529,0.607843}%
\pgfsetfillcolor{currentfill}%
\pgfsetlinewidth{0.481800pt}%
\definecolor{currentstroke}{rgb}{1.000000,1.000000,1.000000}%
\pgfsetstrokecolor{currentstroke}%
\pgfsetdash{}{0pt}%
\pgfpathmoveto{\pgfqpoint{6.430193in}{5.068925in}}%
\pgfpathcurveto{\pgfqpoint{6.441243in}{5.068925in}}{\pgfqpoint{6.451842in}{5.073315in}}{\pgfqpoint{6.459656in}{5.081129in}}%
\pgfpathcurveto{\pgfqpoint{6.467469in}{5.088942in}}{\pgfqpoint{6.471860in}{5.099541in}}{\pgfqpoint{6.471860in}{5.110591in}}%
\pgfpathcurveto{\pgfqpoint{6.471860in}{5.121641in}}{\pgfqpoint{6.467469in}{5.132240in}}{\pgfqpoint{6.459656in}{5.140054in}}%
\pgfpathcurveto{\pgfqpoint{6.451842in}{5.147868in}}{\pgfqpoint{6.441243in}{5.152258in}}{\pgfqpoint{6.430193in}{5.152258in}}%
\pgfpathcurveto{\pgfqpoint{6.419143in}{5.152258in}}{\pgfqpoint{6.408544in}{5.147868in}}{\pgfqpoint{6.400730in}{5.140054in}}%
\pgfpathcurveto{\pgfqpoint{6.392917in}{5.132240in}}{\pgfqpoint{6.388526in}{5.121641in}}{\pgfqpoint{6.388526in}{5.110591in}}%
\pgfpathcurveto{\pgfqpoint{6.388526in}{5.099541in}}{\pgfqpoint{6.392917in}{5.088942in}}{\pgfqpoint{6.400730in}{5.081129in}}%
\pgfpathcurveto{\pgfqpoint{6.408544in}{5.073315in}}{\pgfqpoint{6.419143in}{5.068925in}}{\pgfqpoint{6.430193in}{5.068925in}}%
\pgfpathclose%
\pgfusepath{stroke,fill}%
\end{pgfscope}%
\begin{pgfscope}%
\pgfpathrectangle{\pgfqpoint{0.526127in}{0.331635in}}{\pgfqpoint{9.300000in}{7.700000in}}%
\pgfusepath{clip}%
\pgfsetbuttcap%
\pgfsetroundjoin%
\definecolor{currentfill}{rgb}{1.000000,0.623529,0.607843}%
\pgfsetfillcolor{currentfill}%
\pgfsetlinewidth{0.481800pt}%
\definecolor{currentstroke}{rgb}{1.000000,1.000000,1.000000}%
\pgfsetstrokecolor{currentstroke}%
\pgfsetdash{}{0pt}%
\pgfpathmoveto{\pgfqpoint{5.393295in}{4.009107in}}%
\pgfpathcurveto{\pgfqpoint{5.404345in}{4.009107in}}{\pgfqpoint{5.414944in}{4.013497in}}{\pgfqpoint{5.422758in}{4.021311in}}%
\pgfpathcurveto{\pgfqpoint{5.430571in}{4.029125in}}{\pgfqpoint{5.434962in}{4.039724in}}{\pgfqpoint{5.434962in}{4.050774in}}%
\pgfpathcurveto{\pgfqpoint{5.434962in}{4.061824in}}{\pgfqpoint{5.430571in}{4.072423in}}{\pgfqpoint{5.422758in}{4.080237in}}%
\pgfpathcurveto{\pgfqpoint{5.414944in}{4.088050in}}{\pgfqpoint{5.404345in}{4.092441in}}{\pgfqpoint{5.393295in}{4.092441in}}%
\pgfpathcurveto{\pgfqpoint{5.382245in}{4.092441in}}{\pgfqpoint{5.371646in}{4.088050in}}{\pgfqpoint{5.363832in}{4.080237in}}%
\pgfpathcurveto{\pgfqpoint{5.356019in}{4.072423in}}{\pgfqpoint{5.351628in}{4.061824in}}{\pgfqpoint{5.351628in}{4.050774in}}%
\pgfpathcurveto{\pgfqpoint{5.351628in}{4.039724in}}{\pgfqpoint{5.356019in}{4.029125in}}{\pgfqpoint{5.363832in}{4.021311in}}%
\pgfpathcurveto{\pgfqpoint{5.371646in}{4.013497in}}{\pgfqpoint{5.382245in}{4.009107in}}{\pgfqpoint{5.393295in}{4.009107in}}%
\pgfpathclose%
\pgfusepath{stroke,fill}%
\end{pgfscope}%
\begin{pgfscope}%
\pgfpathrectangle{\pgfqpoint{0.526127in}{0.331635in}}{\pgfqpoint{9.300000in}{7.700000in}}%
\pgfusepath{clip}%
\pgfsetbuttcap%
\pgfsetroundjoin%
\definecolor{currentfill}{rgb}{1.000000,0.623529,0.607843}%
\pgfsetfillcolor{currentfill}%
\pgfsetlinewidth{0.481800pt}%
\definecolor{currentstroke}{rgb}{1.000000,1.000000,1.000000}%
\pgfsetstrokecolor{currentstroke}%
\pgfsetdash{}{0pt}%
\pgfpathmoveto{\pgfqpoint{4.106444in}{4.492949in}}%
\pgfpathcurveto{\pgfqpoint{4.117494in}{4.492949in}}{\pgfqpoint{4.128093in}{4.497339in}}{\pgfqpoint{4.135907in}{4.505153in}}%
\pgfpathcurveto{\pgfqpoint{4.143720in}{4.512967in}}{\pgfqpoint{4.148111in}{4.523566in}}{\pgfqpoint{4.148111in}{4.534616in}}%
\pgfpathcurveto{\pgfqpoint{4.148111in}{4.545666in}}{\pgfqpoint{4.143720in}{4.556265in}}{\pgfqpoint{4.135907in}{4.564079in}}%
\pgfpathcurveto{\pgfqpoint{4.128093in}{4.571892in}}{\pgfqpoint{4.117494in}{4.576283in}}{\pgfqpoint{4.106444in}{4.576283in}}%
\pgfpathcurveto{\pgfqpoint{4.095394in}{4.576283in}}{\pgfqpoint{4.084795in}{4.571892in}}{\pgfqpoint{4.076981in}{4.564079in}}%
\pgfpathcurveto{\pgfqpoint{4.069168in}{4.556265in}}{\pgfqpoint{4.064777in}{4.545666in}}{\pgfqpoint{4.064777in}{4.534616in}}%
\pgfpathcurveto{\pgfqpoint{4.064777in}{4.523566in}}{\pgfqpoint{4.069168in}{4.512967in}}{\pgfqpoint{4.076981in}{4.505153in}}%
\pgfpathcurveto{\pgfqpoint{4.084795in}{4.497339in}}{\pgfqpoint{4.095394in}{4.492949in}}{\pgfqpoint{4.106444in}{4.492949in}}%
\pgfpathclose%
\pgfusepath{stroke,fill}%
\end{pgfscope}%
\begin{pgfscope}%
\pgfpathrectangle{\pgfqpoint{0.526127in}{0.331635in}}{\pgfqpoint{9.300000in}{7.700000in}}%
\pgfusepath{clip}%
\pgfsetbuttcap%
\pgfsetroundjoin%
\definecolor{currentfill}{rgb}{1.000000,0.623529,0.607843}%
\pgfsetfillcolor{currentfill}%
\pgfsetlinewidth{0.481800pt}%
\definecolor{currentstroke}{rgb}{1.000000,1.000000,1.000000}%
\pgfsetstrokecolor{currentstroke}%
\pgfsetdash{}{0pt}%
\pgfpathmoveto{\pgfqpoint{3.340315in}{5.074842in}}%
\pgfpathcurveto{\pgfqpoint{3.351365in}{5.074842in}}{\pgfqpoint{3.361964in}{5.079233in}}{\pgfqpoint{3.369777in}{5.087046in}}%
\pgfpathcurveto{\pgfqpoint{3.377591in}{5.094860in}}{\pgfqpoint{3.381981in}{5.105459in}}{\pgfqpoint{3.381981in}{5.116509in}}%
\pgfpathcurveto{\pgfqpoint{3.381981in}{5.127559in}}{\pgfqpoint{3.377591in}{5.138158in}}{\pgfqpoint{3.369777in}{5.145972in}}%
\pgfpathcurveto{\pgfqpoint{3.361964in}{5.153785in}}{\pgfqpoint{3.351365in}{5.158176in}}{\pgfqpoint{3.340315in}{5.158176in}}%
\pgfpathcurveto{\pgfqpoint{3.329264in}{5.158176in}}{\pgfqpoint{3.318665in}{5.153785in}}{\pgfqpoint{3.310852in}{5.145972in}}%
\pgfpathcurveto{\pgfqpoint{3.303038in}{5.138158in}}{\pgfqpoint{3.298648in}{5.127559in}}{\pgfqpoint{3.298648in}{5.116509in}}%
\pgfpathcurveto{\pgfqpoint{3.298648in}{5.105459in}}{\pgfqpoint{3.303038in}{5.094860in}}{\pgfqpoint{3.310852in}{5.087046in}}%
\pgfpathcurveto{\pgfqpoint{3.318665in}{5.079233in}}{\pgfqpoint{3.329264in}{5.074842in}}{\pgfqpoint{3.340315in}{5.074842in}}%
\pgfpathclose%
\pgfusepath{stroke,fill}%
\end{pgfscope}%
\begin{pgfscope}%
\pgfpathrectangle{\pgfqpoint{0.526127in}{0.331635in}}{\pgfqpoint{9.300000in}{7.700000in}}%
\pgfusepath{clip}%
\pgfsetbuttcap%
\pgfsetroundjoin%
\definecolor{currentfill}{rgb}{1.000000,0.623529,0.607843}%
\pgfsetfillcolor{currentfill}%
\pgfsetlinewidth{0.481800pt}%
\definecolor{currentstroke}{rgb}{1.000000,1.000000,1.000000}%
\pgfsetstrokecolor{currentstroke}%
\pgfsetdash{}{0pt}%
\pgfpathmoveto{\pgfqpoint{5.209743in}{4.817984in}}%
\pgfpathcurveto{\pgfqpoint{5.220793in}{4.817984in}}{\pgfqpoint{5.231392in}{4.822374in}}{\pgfqpoint{5.239206in}{4.830188in}}%
\pgfpathcurveto{\pgfqpoint{5.247020in}{4.838001in}}{\pgfqpoint{5.251410in}{4.848600in}}{\pgfqpoint{5.251410in}{4.859650in}}%
\pgfpathcurveto{\pgfqpoint{5.251410in}{4.870701in}}{\pgfqpoint{5.247020in}{4.881300in}}{\pgfqpoint{5.239206in}{4.889113in}}%
\pgfpathcurveto{\pgfqpoint{5.231392in}{4.896927in}}{\pgfqpoint{5.220793in}{4.901317in}}{\pgfqpoint{5.209743in}{4.901317in}}%
\pgfpathcurveto{\pgfqpoint{5.198693in}{4.901317in}}{\pgfqpoint{5.188094in}{4.896927in}}{\pgfqpoint{5.180280in}{4.889113in}}%
\pgfpathcurveto{\pgfqpoint{5.172467in}{4.881300in}}{\pgfqpoint{5.168077in}{4.870701in}}{\pgfqpoint{5.168077in}{4.859650in}}%
\pgfpathcurveto{\pgfqpoint{5.168077in}{4.848600in}}{\pgfqpoint{5.172467in}{4.838001in}}{\pgfqpoint{5.180280in}{4.830188in}}%
\pgfpathcurveto{\pgfqpoint{5.188094in}{4.822374in}}{\pgfqpoint{5.198693in}{4.817984in}}{\pgfqpoint{5.209743in}{4.817984in}}%
\pgfpathclose%
\pgfusepath{stroke,fill}%
\end{pgfscope}%
\begin{pgfscope}%
\pgfpathrectangle{\pgfqpoint{0.526127in}{0.331635in}}{\pgfqpoint{9.300000in}{7.700000in}}%
\pgfusepath{clip}%
\pgfsetbuttcap%
\pgfsetroundjoin%
\definecolor{currentfill}{rgb}{1.000000,0.623529,0.607843}%
\pgfsetfillcolor{currentfill}%
\pgfsetlinewidth{0.481800pt}%
\definecolor{currentstroke}{rgb}{1.000000,1.000000,1.000000}%
\pgfsetstrokecolor{currentstroke}%
\pgfsetdash{}{0pt}%
\pgfpathmoveto{\pgfqpoint{4.735213in}{4.833487in}}%
\pgfpathcurveto{\pgfqpoint{4.746263in}{4.833487in}}{\pgfqpoint{4.756862in}{4.837877in}}{\pgfqpoint{4.764676in}{4.845691in}}%
\pgfpathcurveto{\pgfqpoint{4.772489in}{4.853505in}}{\pgfqpoint{4.776879in}{4.864104in}}{\pgfqpoint{4.776879in}{4.875154in}}%
\pgfpathcurveto{\pgfqpoint{4.776879in}{4.886204in}}{\pgfqpoint{4.772489in}{4.896803in}}{\pgfqpoint{4.764676in}{4.904616in}}%
\pgfpathcurveto{\pgfqpoint{4.756862in}{4.912430in}}{\pgfqpoint{4.746263in}{4.916820in}}{\pgfqpoint{4.735213in}{4.916820in}}%
\pgfpathcurveto{\pgfqpoint{4.724163in}{4.916820in}}{\pgfqpoint{4.713564in}{4.912430in}}{\pgfqpoint{4.705750in}{4.904616in}}%
\pgfpathcurveto{\pgfqpoint{4.697936in}{4.896803in}}{\pgfqpoint{4.693546in}{4.886204in}}{\pgfqpoint{4.693546in}{4.875154in}}%
\pgfpathcurveto{\pgfqpoint{4.693546in}{4.864104in}}{\pgfqpoint{4.697936in}{4.853505in}}{\pgfqpoint{4.705750in}{4.845691in}}%
\pgfpathcurveto{\pgfqpoint{4.713564in}{4.837877in}}{\pgfqpoint{4.724163in}{4.833487in}}{\pgfqpoint{4.735213in}{4.833487in}}%
\pgfpathclose%
\pgfusepath{stroke,fill}%
\end{pgfscope}%
\begin{pgfscope}%
\pgfpathrectangle{\pgfqpoint{0.526127in}{0.331635in}}{\pgfqpoint{9.300000in}{7.700000in}}%
\pgfusepath{clip}%
\pgfsetbuttcap%
\pgfsetroundjoin%
\definecolor{currentfill}{rgb}{1.000000,0.623529,0.607843}%
\pgfsetfillcolor{currentfill}%
\pgfsetlinewidth{0.481800pt}%
\definecolor{currentstroke}{rgb}{1.000000,1.000000,1.000000}%
\pgfsetstrokecolor{currentstroke}%
\pgfsetdash{}{0pt}%
\pgfpathmoveto{\pgfqpoint{3.352726in}{5.641625in}}%
\pgfpathcurveto{\pgfqpoint{3.363777in}{5.641625in}}{\pgfqpoint{3.374376in}{5.646015in}}{\pgfqpoint{3.382189in}{5.653829in}}%
\pgfpathcurveto{\pgfqpoint{3.390003in}{5.661642in}}{\pgfqpoint{3.394393in}{5.672241in}}{\pgfqpoint{3.394393in}{5.683291in}}%
\pgfpathcurveto{\pgfqpoint{3.394393in}{5.694342in}}{\pgfqpoint{3.390003in}{5.704941in}}{\pgfqpoint{3.382189in}{5.712754in}}%
\pgfpathcurveto{\pgfqpoint{3.374376in}{5.720568in}}{\pgfqpoint{3.363777in}{5.724958in}}{\pgfqpoint{3.352726in}{5.724958in}}%
\pgfpathcurveto{\pgfqpoint{3.341676in}{5.724958in}}{\pgfqpoint{3.331077in}{5.720568in}}{\pgfqpoint{3.323264in}{5.712754in}}%
\pgfpathcurveto{\pgfqpoint{3.315450in}{5.704941in}}{\pgfqpoint{3.311060in}{5.694342in}}{\pgfqpoint{3.311060in}{5.683291in}}%
\pgfpathcurveto{\pgfqpoint{3.311060in}{5.672241in}}{\pgfqpoint{3.315450in}{5.661642in}}{\pgfqpoint{3.323264in}{5.653829in}}%
\pgfpathcurveto{\pgfqpoint{3.331077in}{5.646015in}}{\pgfqpoint{3.341676in}{5.641625in}}{\pgfqpoint{3.352726in}{5.641625in}}%
\pgfpathclose%
\pgfusepath{stroke,fill}%
\end{pgfscope}%
\begin{pgfscope}%
\pgfpathrectangle{\pgfqpoint{0.526127in}{0.331635in}}{\pgfqpoint{9.300000in}{7.700000in}}%
\pgfusepath{clip}%
\pgfsetbuttcap%
\pgfsetroundjoin%
\definecolor{currentfill}{rgb}{1.000000,0.623529,0.607843}%
\pgfsetfillcolor{currentfill}%
\pgfsetlinewidth{0.481800pt}%
\definecolor{currentstroke}{rgb}{1.000000,1.000000,1.000000}%
\pgfsetstrokecolor{currentstroke}%
\pgfsetdash{}{0pt}%
\pgfpathmoveto{\pgfqpoint{4.527282in}{5.262823in}}%
\pgfpathcurveto{\pgfqpoint{4.538332in}{5.262823in}}{\pgfqpoint{4.548931in}{5.267214in}}{\pgfqpoint{4.556744in}{5.275027in}}%
\pgfpathcurveto{\pgfqpoint{4.564558in}{5.282841in}}{\pgfqpoint{4.568948in}{5.293440in}}{\pgfqpoint{4.568948in}{5.304490in}}%
\pgfpathcurveto{\pgfqpoint{4.568948in}{5.315540in}}{\pgfqpoint{4.564558in}{5.326139in}}{\pgfqpoint{4.556744in}{5.333953in}}%
\pgfpathcurveto{\pgfqpoint{4.548931in}{5.341766in}}{\pgfqpoint{4.538332in}{5.346157in}}{\pgfqpoint{4.527282in}{5.346157in}}%
\pgfpathcurveto{\pgfqpoint{4.516232in}{5.346157in}}{\pgfqpoint{4.505633in}{5.341766in}}{\pgfqpoint{4.497819in}{5.333953in}}%
\pgfpathcurveto{\pgfqpoint{4.490005in}{5.326139in}}{\pgfqpoint{4.485615in}{5.315540in}}{\pgfqpoint{4.485615in}{5.304490in}}%
\pgfpathcurveto{\pgfqpoint{4.485615in}{5.293440in}}{\pgfqpoint{4.490005in}{5.282841in}}{\pgfqpoint{4.497819in}{5.275027in}}%
\pgfpathcurveto{\pgfqpoint{4.505633in}{5.267214in}}{\pgfqpoint{4.516232in}{5.262823in}}{\pgfqpoint{4.527282in}{5.262823in}}%
\pgfpathclose%
\pgfusepath{stroke,fill}%
\end{pgfscope}%
\begin{pgfscope}%
\pgfpathrectangle{\pgfqpoint{0.526127in}{0.331635in}}{\pgfqpoint{9.300000in}{7.700000in}}%
\pgfusepath{clip}%
\pgfsetbuttcap%
\pgfsetroundjoin%
\definecolor{currentfill}{rgb}{1.000000,0.623529,0.607843}%
\pgfsetfillcolor{currentfill}%
\pgfsetlinewidth{0.481800pt}%
\definecolor{currentstroke}{rgb}{1.000000,1.000000,1.000000}%
\pgfsetstrokecolor{currentstroke}%
\pgfsetdash{}{0pt}%
\pgfpathmoveto{\pgfqpoint{4.542592in}{3.093998in}}%
\pgfpathcurveto{\pgfqpoint{4.553642in}{3.093998in}}{\pgfqpoint{4.564241in}{3.098388in}}{\pgfqpoint{4.572055in}{3.106201in}}%
\pgfpathcurveto{\pgfqpoint{4.579868in}{3.114015in}}{\pgfqpoint{4.584259in}{3.124614in}}{\pgfqpoint{4.584259in}{3.135664in}}%
\pgfpathcurveto{\pgfqpoint{4.584259in}{3.146714in}}{\pgfqpoint{4.579868in}{3.157313in}}{\pgfqpoint{4.572055in}{3.165127in}}%
\pgfpathcurveto{\pgfqpoint{4.564241in}{3.172941in}}{\pgfqpoint{4.553642in}{3.177331in}}{\pgfqpoint{4.542592in}{3.177331in}}%
\pgfpathcurveto{\pgfqpoint{4.531542in}{3.177331in}}{\pgfqpoint{4.520943in}{3.172941in}}{\pgfqpoint{4.513129in}{3.165127in}}%
\pgfpathcurveto{\pgfqpoint{4.505316in}{3.157313in}}{\pgfqpoint{4.500925in}{3.146714in}}{\pgfqpoint{4.500925in}{3.135664in}}%
\pgfpathcurveto{\pgfqpoint{4.500925in}{3.124614in}}{\pgfqpoint{4.505316in}{3.114015in}}{\pgfqpoint{4.513129in}{3.106201in}}%
\pgfpathcurveto{\pgfqpoint{4.520943in}{3.098388in}}{\pgfqpoint{4.531542in}{3.093998in}}{\pgfqpoint{4.542592in}{3.093998in}}%
\pgfpathclose%
\pgfusepath{stroke,fill}%
\end{pgfscope}%
\begin{pgfscope}%
\pgfpathrectangle{\pgfqpoint{0.526127in}{0.331635in}}{\pgfqpoint{9.300000in}{7.700000in}}%
\pgfusepath{clip}%
\pgfsetbuttcap%
\pgfsetroundjoin%
\definecolor{currentfill}{rgb}{1.000000,0.623529,0.607843}%
\pgfsetfillcolor{currentfill}%
\pgfsetlinewidth{0.481800pt}%
\definecolor{currentstroke}{rgb}{1.000000,1.000000,1.000000}%
\pgfsetstrokecolor{currentstroke}%
\pgfsetdash{}{0pt}%
\pgfpathmoveto{\pgfqpoint{6.055654in}{5.581923in}}%
\pgfpathcurveto{\pgfqpoint{6.066704in}{5.581923in}}{\pgfqpoint{6.077303in}{5.586314in}}{\pgfqpoint{6.085117in}{5.594127in}}%
\pgfpathcurveto{\pgfqpoint{6.092930in}{5.601941in}}{\pgfqpoint{6.097320in}{5.612540in}}{\pgfqpoint{6.097320in}{5.623590in}}%
\pgfpathcurveto{\pgfqpoint{6.097320in}{5.634640in}}{\pgfqpoint{6.092930in}{5.645239in}}{\pgfqpoint{6.085117in}{5.653053in}}%
\pgfpathcurveto{\pgfqpoint{6.077303in}{5.660866in}}{\pgfqpoint{6.066704in}{5.665257in}}{\pgfqpoint{6.055654in}{5.665257in}}%
\pgfpathcurveto{\pgfqpoint{6.044604in}{5.665257in}}{\pgfqpoint{6.034005in}{5.660866in}}{\pgfqpoint{6.026191in}{5.653053in}}%
\pgfpathcurveto{\pgfqpoint{6.018377in}{5.645239in}}{\pgfqpoint{6.013987in}{5.634640in}}{\pgfqpoint{6.013987in}{5.623590in}}%
\pgfpathcurveto{\pgfqpoint{6.013987in}{5.612540in}}{\pgfqpoint{6.018377in}{5.601941in}}{\pgfqpoint{6.026191in}{5.594127in}}%
\pgfpathcurveto{\pgfqpoint{6.034005in}{5.586314in}}{\pgfqpoint{6.044604in}{5.581923in}}{\pgfqpoint{6.055654in}{5.581923in}}%
\pgfpathclose%
\pgfusepath{stroke,fill}%
\end{pgfscope}%
\begin{pgfscope}%
\pgfpathrectangle{\pgfqpoint{0.526127in}{0.331635in}}{\pgfqpoint{9.300000in}{7.700000in}}%
\pgfusepath{clip}%
\pgfsetbuttcap%
\pgfsetroundjoin%
\definecolor{currentfill}{rgb}{1.000000,0.623529,0.607843}%
\pgfsetfillcolor{currentfill}%
\pgfsetlinewidth{0.481800pt}%
\definecolor{currentstroke}{rgb}{1.000000,1.000000,1.000000}%
\pgfsetstrokecolor{currentstroke}%
\pgfsetdash{}{0pt}%
\pgfpathmoveto{\pgfqpoint{6.518607in}{5.384306in}}%
\pgfpathcurveto{\pgfqpoint{6.529657in}{5.384306in}}{\pgfqpoint{6.540256in}{5.388696in}}{\pgfqpoint{6.548070in}{5.396510in}}%
\pgfpathcurveto{\pgfqpoint{6.555884in}{5.404324in}}{\pgfqpoint{6.560274in}{5.414923in}}{\pgfqpoint{6.560274in}{5.425973in}}%
\pgfpathcurveto{\pgfqpoint{6.560274in}{5.437023in}}{\pgfqpoint{6.555884in}{5.447622in}}{\pgfqpoint{6.548070in}{5.455435in}}%
\pgfpathcurveto{\pgfqpoint{6.540256in}{5.463249in}}{\pgfqpoint{6.529657in}{5.467639in}}{\pgfqpoint{6.518607in}{5.467639in}}%
\pgfpathcurveto{\pgfqpoint{6.507557in}{5.467639in}}{\pgfqpoint{6.496958in}{5.463249in}}{\pgfqpoint{6.489144in}{5.455435in}}%
\pgfpathcurveto{\pgfqpoint{6.481331in}{5.447622in}}{\pgfqpoint{6.476941in}{5.437023in}}{\pgfqpoint{6.476941in}{5.425973in}}%
\pgfpathcurveto{\pgfqpoint{6.476941in}{5.414923in}}{\pgfqpoint{6.481331in}{5.404324in}}{\pgfqpoint{6.489144in}{5.396510in}}%
\pgfpathcurveto{\pgfqpoint{6.496958in}{5.388696in}}{\pgfqpoint{6.507557in}{5.384306in}}{\pgfqpoint{6.518607in}{5.384306in}}%
\pgfpathclose%
\pgfusepath{stroke,fill}%
\end{pgfscope}%
\begin{pgfscope}%
\pgfpathrectangle{\pgfqpoint{0.526127in}{0.331635in}}{\pgfqpoint{9.300000in}{7.700000in}}%
\pgfusepath{clip}%
\pgfsetbuttcap%
\pgfsetroundjoin%
\definecolor{currentfill}{rgb}{1.000000,0.623529,0.607843}%
\pgfsetfillcolor{currentfill}%
\pgfsetlinewidth{0.481800pt}%
\definecolor{currentstroke}{rgb}{1.000000,1.000000,1.000000}%
\pgfsetstrokecolor{currentstroke}%
\pgfsetdash{}{0pt}%
\pgfpathmoveto{\pgfqpoint{4.977203in}{5.447654in}}%
\pgfpathcurveto{\pgfqpoint{4.988253in}{5.447654in}}{\pgfqpoint{4.998852in}{5.452044in}}{\pgfqpoint{5.006666in}{5.459857in}}%
\pgfpathcurveto{\pgfqpoint{5.014480in}{5.467671in}}{\pgfqpoint{5.018870in}{5.478270in}}{\pgfqpoint{5.018870in}{5.489320in}}%
\pgfpathcurveto{\pgfqpoint{5.018870in}{5.500370in}}{\pgfqpoint{5.014480in}{5.510969in}}{\pgfqpoint{5.006666in}{5.518783in}}%
\pgfpathcurveto{\pgfqpoint{4.998852in}{5.526597in}}{\pgfqpoint{4.988253in}{5.530987in}}{\pgfqpoint{4.977203in}{5.530987in}}%
\pgfpathcurveto{\pgfqpoint{4.966153in}{5.530987in}}{\pgfqpoint{4.955554in}{5.526597in}}{\pgfqpoint{4.947740in}{5.518783in}}%
\pgfpathcurveto{\pgfqpoint{4.939927in}{5.510969in}}{\pgfqpoint{4.935537in}{5.500370in}}{\pgfqpoint{4.935537in}{5.489320in}}%
\pgfpathcurveto{\pgfqpoint{4.935537in}{5.478270in}}{\pgfqpoint{4.939927in}{5.467671in}}{\pgfqpoint{4.947740in}{5.459857in}}%
\pgfpathcurveto{\pgfqpoint{4.955554in}{5.452044in}}{\pgfqpoint{4.966153in}{5.447654in}}{\pgfqpoint{4.977203in}{5.447654in}}%
\pgfpathclose%
\pgfusepath{stroke,fill}%
\end{pgfscope}%
\begin{pgfscope}%
\pgfpathrectangle{\pgfqpoint{0.526127in}{0.331635in}}{\pgfqpoint{9.300000in}{7.700000in}}%
\pgfusepath{clip}%
\pgfsetbuttcap%
\pgfsetroundjoin%
\definecolor{currentfill}{rgb}{1.000000,0.623529,0.607843}%
\pgfsetfillcolor{currentfill}%
\pgfsetlinewidth{0.481800pt}%
\definecolor{currentstroke}{rgb}{1.000000,1.000000,1.000000}%
\pgfsetstrokecolor{currentstroke}%
\pgfsetdash{}{0pt}%
\pgfpathmoveto{\pgfqpoint{4.655568in}{5.295323in}}%
\pgfpathcurveto{\pgfqpoint{4.666618in}{5.295323in}}{\pgfqpoint{4.677217in}{5.299713in}}{\pgfqpoint{4.685031in}{5.307527in}}%
\pgfpathcurveto{\pgfqpoint{4.692845in}{5.315340in}}{\pgfqpoint{4.697235in}{5.325939in}}{\pgfqpoint{4.697235in}{5.336989in}}%
\pgfpathcurveto{\pgfqpoint{4.697235in}{5.348040in}}{\pgfqpoint{4.692845in}{5.358639in}}{\pgfqpoint{4.685031in}{5.366452in}}%
\pgfpathcurveto{\pgfqpoint{4.677217in}{5.374266in}}{\pgfqpoint{4.666618in}{5.378656in}}{\pgfqpoint{4.655568in}{5.378656in}}%
\pgfpathcurveto{\pgfqpoint{4.644518in}{5.378656in}}{\pgfqpoint{4.633919in}{5.374266in}}{\pgfqpoint{4.626105in}{5.366452in}}%
\pgfpathcurveto{\pgfqpoint{4.618292in}{5.358639in}}{\pgfqpoint{4.613902in}{5.348040in}}{\pgfqpoint{4.613902in}{5.336989in}}%
\pgfpathcurveto{\pgfqpoint{4.613902in}{5.325939in}}{\pgfqpoint{4.618292in}{5.315340in}}{\pgfqpoint{4.626105in}{5.307527in}}%
\pgfpathcurveto{\pgfqpoint{4.633919in}{5.299713in}}{\pgfqpoint{4.644518in}{5.295323in}}{\pgfqpoint{4.655568in}{5.295323in}}%
\pgfpathclose%
\pgfusepath{stroke,fill}%
\end{pgfscope}%
\begin{pgfscope}%
\pgfpathrectangle{\pgfqpoint{0.526127in}{0.331635in}}{\pgfqpoint{9.300000in}{7.700000in}}%
\pgfusepath{clip}%
\pgfsetbuttcap%
\pgfsetroundjoin%
\definecolor{currentfill}{rgb}{1.000000,0.623529,0.607843}%
\pgfsetfillcolor{currentfill}%
\pgfsetlinewidth{0.481800pt}%
\definecolor{currentstroke}{rgb}{1.000000,1.000000,1.000000}%
\pgfsetstrokecolor{currentstroke}%
\pgfsetdash{}{0pt}%
\pgfpathmoveto{\pgfqpoint{5.314689in}{4.304703in}}%
\pgfpathcurveto{\pgfqpoint{5.325739in}{4.304703in}}{\pgfqpoint{5.336338in}{4.309093in}}{\pgfqpoint{5.344152in}{4.316907in}}%
\pgfpathcurveto{\pgfqpoint{5.351965in}{4.324720in}}{\pgfqpoint{5.356356in}{4.335319in}}{\pgfqpoint{5.356356in}{4.346369in}}%
\pgfpathcurveto{\pgfqpoint{5.356356in}{4.357420in}}{\pgfqpoint{5.351965in}{4.368019in}}{\pgfqpoint{5.344152in}{4.375832in}}%
\pgfpathcurveto{\pgfqpoint{5.336338in}{4.383646in}}{\pgfqpoint{5.325739in}{4.388036in}}{\pgfqpoint{5.314689in}{4.388036in}}%
\pgfpathcurveto{\pgfqpoint{5.303639in}{4.388036in}}{\pgfqpoint{5.293040in}{4.383646in}}{\pgfqpoint{5.285226in}{4.375832in}}%
\pgfpathcurveto{\pgfqpoint{5.277413in}{4.368019in}}{\pgfqpoint{5.273022in}{4.357420in}}{\pgfqpoint{5.273022in}{4.346369in}}%
\pgfpathcurveto{\pgfqpoint{5.273022in}{4.335319in}}{\pgfqpoint{5.277413in}{4.324720in}}{\pgfqpoint{5.285226in}{4.316907in}}%
\pgfpathcurveto{\pgfqpoint{5.293040in}{4.309093in}}{\pgfqpoint{5.303639in}{4.304703in}}{\pgfqpoint{5.314689in}{4.304703in}}%
\pgfpathclose%
\pgfusepath{stroke,fill}%
\end{pgfscope}%
\begin{pgfscope}%
\pgfpathrectangle{\pgfqpoint{0.526127in}{0.331635in}}{\pgfqpoint{9.300000in}{7.700000in}}%
\pgfusepath{clip}%
\pgfsetbuttcap%
\pgfsetroundjoin%
\definecolor{currentfill}{rgb}{0.815686,0.733333,1.000000}%
\pgfsetfillcolor{currentfill}%
\pgfsetlinewidth{0.481800pt}%
\definecolor{currentstroke}{rgb}{1.000000,1.000000,1.000000}%
\pgfsetstrokecolor{currentstroke}%
\pgfsetdash{}{0pt}%
\pgfpathmoveto{\pgfqpoint{6.917849in}{2.661457in}}%
\pgfpathcurveto{\pgfqpoint{6.928899in}{2.661457in}}{\pgfqpoint{6.939498in}{2.665847in}}{\pgfqpoint{6.947312in}{2.673661in}}%
\pgfpathcurveto{\pgfqpoint{6.955125in}{2.681475in}}{\pgfqpoint{6.959516in}{2.692074in}}{\pgfqpoint{6.959516in}{2.703124in}}%
\pgfpathcurveto{\pgfqpoint{6.959516in}{2.714174in}}{\pgfqpoint{6.955125in}{2.724773in}}{\pgfqpoint{6.947312in}{2.732587in}}%
\pgfpathcurveto{\pgfqpoint{6.939498in}{2.740400in}}{\pgfqpoint{6.928899in}{2.744790in}}{\pgfqpoint{6.917849in}{2.744790in}}%
\pgfpathcurveto{\pgfqpoint{6.906799in}{2.744790in}}{\pgfqpoint{6.896200in}{2.740400in}}{\pgfqpoint{6.888386in}{2.732587in}}%
\pgfpathcurveto{\pgfqpoint{6.880572in}{2.724773in}}{\pgfqpoint{6.876182in}{2.714174in}}{\pgfqpoint{6.876182in}{2.703124in}}%
\pgfpathcurveto{\pgfqpoint{6.876182in}{2.692074in}}{\pgfqpoint{6.880572in}{2.681475in}}{\pgfqpoint{6.888386in}{2.673661in}}%
\pgfpathcurveto{\pgfqpoint{6.896200in}{2.665847in}}{\pgfqpoint{6.906799in}{2.661457in}}{\pgfqpoint{6.917849in}{2.661457in}}%
\pgfpathclose%
\pgfusepath{stroke,fill}%
\end{pgfscope}%
\begin{pgfscope}%
\pgfpathrectangle{\pgfqpoint{0.526127in}{0.331635in}}{\pgfqpoint{9.300000in}{7.700000in}}%
\pgfusepath{clip}%
\pgfsetbuttcap%
\pgfsetroundjoin%
\definecolor{currentfill}{rgb}{0.815686,0.733333,1.000000}%
\pgfsetfillcolor{currentfill}%
\pgfsetlinewidth{0.481800pt}%
\definecolor{currentstroke}{rgb}{1.000000,1.000000,1.000000}%
\pgfsetstrokecolor{currentstroke}%
\pgfsetdash{}{0pt}%
\pgfpathmoveto{\pgfqpoint{6.254342in}{2.041288in}}%
\pgfpathcurveto{\pgfqpoint{6.265392in}{2.041288in}}{\pgfqpoint{6.275991in}{2.045678in}}{\pgfqpoint{6.283804in}{2.053492in}}%
\pgfpathcurveto{\pgfqpoint{6.291618in}{2.061305in}}{\pgfqpoint{6.296008in}{2.071904in}}{\pgfqpoint{6.296008in}{2.082954in}}%
\pgfpathcurveto{\pgfqpoint{6.296008in}{2.094005in}}{\pgfqpoint{6.291618in}{2.104604in}}{\pgfqpoint{6.283804in}{2.112417in}}%
\pgfpathcurveto{\pgfqpoint{6.275991in}{2.120231in}}{\pgfqpoint{6.265392in}{2.124621in}}{\pgfqpoint{6.254342in}{2.124621in}}%
\pgfpathcurveto{\pgfqpoint{6.243291in}{2.124621in}}{\pgfqpoint{6.232692in}{2.120231in}}{\pgfqpoint{6.224879in}{2.112417in}}%
\pgfpathcurveto{\pgfqpoint{6.217065in}{2.104604in}}{\pgfqpoint{6.212675in}{2.094005in}}{\pgfqpoint{6.212675in}{2.082954in}}%
\pgfpathcurveto{\pgfqpoint{6.212675in}{2.071904in}}{\pgfqpoint{6.217065in}{2.061305in}}{\pgfqpoint{6.224879in}{2.053492in}}%
\pgfpathcurveto{\pgfqpoint{6.232692in}{2.045678in}}{\pgfqpoint{6.243291in}{2.041288in}}{\pgfqpoint{6.254342in}{2.041288in}}%
\pgfpathclose%
\pgfusepath{stroke,fill}%
\end{pgfscope}%
\begin{pgfscope}%
\pgfpathrectangle{\pgfqpoint{0.526127in}{0.331635in}}{\pgfqpoint{9.300000in}{7.700000in}}%
\pgfusepath{clip}%
\pgfsetbuttcap%
\pgfsetroundjoin%
\definecolor{currentfill}{rgb}{0.815686,0.733333,1.000000}%
\pgfsetfillcolor{currentfill}%
\pgfsetlinewidth{0.481800pt}%
\definecolor{currentstroke}{rgb}{1.000000,1.000000,1.000000}%
\pgfsetstrokecolor{currentstroke}%
\pgfsetdash{}{0pt}%
\pgfpathmoveto{\pgfqpoint{4.500103in}{1.819196in}}%
\pgfpathcurveto{\pgfqpoint{4.511153in}{1.819196in}}{\pgfqpoint{4.521752in}{1.823586in}}{\pgfqpoint{4.529566in}{1.831400in}}%
\pgfpathcurveto{\pgfqpoint{4.537379in}{1.839214in}}{\pgfqpoint{4.541770in}{1.849813in}}{\pgfqpoint{4.541770in}{1.860863in}}%
\pgfpathcurveto{\pgfqpoint{4.541770in}{1.871913in}}{\pgfqpoint{4.537379in}{1.882512in}}{\pgfqpoint{4.529566in}{1.890326in}}%
\pgfpathcurveto{\pgfqpoint{4.521752in}{1.898139in}}{\pgfqpoint{4.511153in}{1.902529in}}{\pgfqpoint{4.500103in}{1.902529in}}%
\pgfpathcurveto{\pgfqpoint{4.489053in}{1.902529in}}{\pgfqpoint{4.478454in}{1.898139in}}{\pgfqpoint{4.470640in}{1.890326in}}%
\pgfpathcurveto{\pgfqpoint{4.462826in}{1.882512in}}{\pgfqpoint{4.458436in}{1.871913in}}{\pgfqpoint{4.458436in}{1.860863in}}%
\pgfpathcurveto{\pgfqpoint{4.458436in}{1.849813in}}{\pgfqpoint{4.462826in}{1.839214in}}{\pgfqpoint{4.470640in}{1.831400in}}%
\pgfpathcurveto{\pgfqpoint{4.478454in}{1.823586in}}{\pgfqpoint{4.489053in}{1.819196in}}{\pgfqpoint{4.500103in}{1.819196in}}%
\pgfpathclose%
\pgfusepath{stroke,fill}%
\end{pgfscope}%
\begin{pgfscope}%
\pgfpathrectangle{\pgfqpoint{0.526127in}{0.331635in}}{\pgfqpoint{9.300000in}{7.700000in}}%
\pgfusepath{clip}%
\pgfsetbuttcap%
\pgfsetroundjoin%
\definecolor{currentfill}{rgb}{0.815686,0.733333,1.000000}%
\pgfsetfillcolor{currentfill}%
\pgfsetlinewidth{0.481800pt}%
\definecolor{currentstroke}{rgb}{1.000000,1.000000,1.000000}%
\pgfsetstrokecolor{currentstroke}%
\pgfsetdash{}{0pt}%
\pgfpathmoveto{\pgfqpoint{3.205256in}{1.791054in}}%
\pgfpathcurveto{\pgfqpoint{3.216306in}{1.791054in}}{\pgfqpoint{3.226905in}{1.795444in}}{\pgfqpoint{3.234719in}{1.803258in}}%
\pgfpathcurveto{\pgfqpoint{3.242532in}{1.811072in}}{\pgfqpoint{3.246923in}{1.821671in}}{\pgfqpoint{3.246923in}{1.832721in}}%
\pgfpathcurveto{\pgfqpoint{3.246923in}{1.843771in}}{\pgfqpoint{3.242532in}{1.854370in}}{\pgfqpoint{3.234719in}{1.862184in}}%
\pgfpathcurveto{\pgfqpoint{3.226905in}{1.869997in}}{\pgfqpoint{3.216306in}{1.874387in}}{\pgfqpoint{3.205256in}{1.874387in}}%
\pgfpathcurveto{\pgfqpoint{3.194206in}{1.874387in}}{\pgfqpoint{3.183607in}{1.869997in}}{\pgfqpoint{3.175793in}{1.862184in}}%
\pgfpathcurveto{\pgfqpoint{3.167980in}{1.854370in}}{\pgfqpoint{3.163589in}{1.843771in}}{\pgfqpoint{3.163589in}{1.832721in}}%
\pgfpathcurveto{\pgfqpoint{3.163589in}{1.821671in}}{\pgfqpoint{3.167980in}{1.811072in}}{\pgfqpoint{3.175793in}{1.803258in}}%
\pgfpathcurveto{\pgfqpoint{3.183607in}{1.795444in}}{\pgfqpoint{3.194206in}{1.791054in}}{\pgfqpoint{3.205256in}{1.791054in}}%
\pgfpathclose%
\pgfusepath{stroke,fill}%
\end{pgfscope}%
\begin{pgfscope}%
\pgfpathrectangle{\pgfqpoint{0.526127in}{0.331635in}}{\pgfqpoint{9.300000in}{7.700000in}}%
\pgfusepath{clip}%
\pgfsetbuttcap%
\pgfsetroundjoin%
\definecolor{currentfill}{rgb}{0.815686,0.733333,1.000000}%
\pgfsetfillcolor{currentfill}%
\pgfsetlinewidth{0.481800pt}%
\definecolor{currentstroke}{rgb}{1.000000,1.000000,1.000000}%
\pgfsetstrokecolor{currentstroke}%
\pgfsetdash{}{0pt}%
\pgfpathmoveto{\pgfqpoint{6.199623in}{2.176425in}}%
\pgfpathcurveto{\pgfqpoint{6.210674in}{2.176425in}}{\pgfqpoint{6.221273in}{2.180815in}}{\pgfqpoint{6.229086in}{2.188629in}}%
\pgfpathcurveto{\pgfqpoint{6.236900in}{2.196442in}}{\pgfqpoint{6.241290in}{2.207041in}}{\pgfqpoint{6.241290in}{2.218091in}}%
\pgfpathcurveto{\pgfqpoint{6.241290in}{2.229142in}}{\pgfqpoint{6.236900in}{2.239741in}}{\pgfqpoint{6.229086in}{2.247554in}}%
\pgfpathcurveto{\pgfqpoint{6.221273in}{2.255368in}}{\pgfqpoint{6.210674in}{2.259758in}}{\pgfqpoint{6.199623in}{2.259758in}}%
\pgfpathcurveto{\pgfqpoint{6.188573in}{2.259758in}}{\pgfqpoint{6.177974in}{2.255368in}}{\pgfqpoint{6.170161in}{2.247554in}}%
\pgfpathcurveto{\pgfqpoint{6.162347in}{2.239741in}}{\pgfqpoint{6.157957in}{2.229142in}}{\pgfqpoint{6.157957in}{2.218091in}}%
\pgfpathcurveto{\pgfqpoint{6.157957in}{2.207041in}}{\pgfqpoint{6.162347in}{2.196442in}}{\pgfqpoint{6.170161in}{2.188629in}}%
\pgfpathcurveto{\pgfqpoint{6.177974in}{2.180815in}}{\pgfqpoint{6.188573in}{2.176425in}}{\pgfqpoint{6.199623in}{2.176425in}}%
\pgfpathclose%
\pgfusepath{stroke,fill}%
\end{pgfscope}%
\begin{pgfscope}%
\pgfpathrectangle{\pgfqpoint{0.526127in}{0.331635in}}{\pgfqpoint{9.300000in}{7.700000in}}%
\pgfusepath{clip}%
\pgfsetbuttcap%
\pgfsetroundjoin%
\definecolor{currentfill}{rgb}{0.815686,0.733333,1.000000}%
\pgfsetfillcolor{currentfill}%
\pgfsetlinewidth{0.481800pt}%
\definecolor{currentstroke}{rgb}{1.000000,1.000000,1.000000}%
\pgfsetstrokecolor{currentstroke}%
\pgfsetdash{}{0pt}%
\pgfpathmoveto{\pgfqpoint{7.384542in}{2.688185in}}%
\pgfpathcurveto{\pgfqpoint{7.395592in}{2.688185in}}{\pgfqpoint{7.406191in}{2.692576in}}{\pgfqpoint{7.414005in}{2.700389in}}%
\pgfpathcurveto{\pgfqpoint{7.421818in}{2.708203in}}{\pgfqpoint{7.426208in}{2.718802in}}{\pgfqpoint{7.426208in}{2.729852in}}%
\pgfpathcurveto{\pgfqpoint{7.426208in}{2.740902in}}{\pgfqpoint{7.421818in}{2.751501in}}{\pgfqpoint{7.414005in}{2.759315in}}%
\pgfpathcurveto{\pgfqpoint{7.406191in}{2.767128in}}{\pgfqpoint{7.395592in}{2.771519in}}{\pgfqpoint{7.384542in}{2.771519in}}%
\pgfpathcurveto{\pgfqpoint{7.373492in}{2.771519in}}{\pgfqpoint{7.362893in}{2.767128in}}{\pgfqpoint{7.355079in}{2.759315in}}%
\pgfpathcurveto{\pgfqpoint{7.347265in}{2.751501in}}{\pgfqpoint{7.342875in}{2.740902in}}{\pgfqpoint{7.342875in}{2.729852in}}%
\pgfpathcurveto{\pgfqpoint{7.342875in}{2.718802in}}{\pgfqpoint{7.347265in}{2.708203in}}{\pgfqpoint{7.355079in}{2.700389in}}%
\pgfpathcurveto{\pgfqpoint{7.362893in}{2.692576in}}{\pgfqpoint{7.373492in}{2.688185in}}{\pgfqpoint{7.384542in}{2.688185in}}%
\pgfpathclose%
\pgfusepath{stroke,fill}%
\end{pgfscope}%
\begin{pgfscope}%
\pgfpathrectangle{\pgfqpoint{0.526127in}{0.331635in}}{\pgfqpoint{9.300000in}{7.700000in}}%
\pgfusepath{clip}%
\pgfsetbuttcap%
\pgfsetroundjoin%
\definecolor{currentfill}{rgb}{0.815686,0.733333,1.000000}%
\pgfsetfillcolor{currentfill}%
\pgfsetlinewidth{0.481800pt}%
\definecolor{currentstroke}{rgb}{1.000000,1.000000,1.000000}%
\pgfsetstrokecolor{currentstroke}%
\pgfsetdash{}{0pt}%
\pgfpathmoveto{\pgfqpoint{2.373781in}{3.541707in}}%
\pgfpathcurveto{\pgfqpoint{2.384831in}{3.541707in}}{\pgfqpoint{2.395430in}{3.546098in}}{\pgfqpoint{2.403244in}{3.553911in}}%
\pgfpathcurveto{\pgfqpoint{2.411058in}{3.561725in}}{\pgfqpoint{2.415448in}{3.572324in}}{\pgfqpoint{2.415448in}{3.583374in}}%
\pgfpathcurveto{\pgfqpoint{2.415448in}{3.594424in}}{\pgfqpoint{2.411058in}{3.605023in}}{\pgfqpoint{2.403244in}{3.612837in}}%
\pgfpathcurveto{\pgfqpoint{2.395430in}{3.620650in}}{\pgfqpoint{2.384831in}{3.625041in}}{\pgfqpoint{2.373781in}{3.625041in}}%
\pgfpathcurveto{\pgfqpoint{2.362731in}{3.625041in}}{\pgfqpoint{2.352132in}{3.620650in}}{\pgfqpoint{2.344318in}{3.612837in}}%
\pgfpathcurveto{\pgfqpoint{2.336505in}{3.605023in}}{\pgfqpoint{2.332115in}{3.594424in}}{\pgfqpoint{2.332115in}{3.583374in}}%
\pgfpathcurveto{\pgfqpoint{2.332115in}{3.572324in}}{\pgfqpoint{2.336505in}{3.561725in}}{\pgfqpoint{2.344318in}{3.553911in}}%
\pgfpathcurveto{\pgfqpoint{2.352132in}{3.546098in}}{\pgfqpoint{2.362731in}{3.541707in}}{\pgfqpoint{2.373781in}{3.541707in}}%
\pgfpathclose%
\pgfusepath{stroke,fill}%
\end{pgfscope}%
\begin{pgfscope}%
\pgfpathrectangle{\pgfqpoint{0.526127in}{0.331635in}}{\pgfqpoint{9.300000in}{7.700000in}}%
\pgfusepath{clip}%
\pgfsetbuttcap%
\pgfsetroundjoin%
\definecolor{currentfill}{rgb}{0.815686,0.733333,1.000000}%
\pgfsetfillcolor{currentfill}%
\pgfsetlinewidth{0.481800pt}%
\definecolor{currentstroke}{rgb}{1.000000,1.000000,1.000000}%
\pgfsetstrokecolor{currentstroke}%
\pgfsetdash{}{0pt}%
\pgfpathmoveto{\pgfqpoint{7.393947in}{1.637329in}}%
\pgfpathcurveto{\pgfqpoint{7.404997in}{1.637329in}}{\pgfqpoint{7.415596in}{1.641719in}}{\pgfqpoint{7.423410in}{1.649532in}}%
\pgfpathcurveto{\pgfqpoint{7.431224in}{1.657346in}}{\pgfqpoint{7.435614in}{1.667945in}}{\pgfqpoint{7.435614in}{1.678995in}}%
\pgfpathcurveto{\pgfqpoint{7.435614in}{1.690045in}}{\pgfqpoint{7.431224in}{1.700644in}}{\pgfqpoint{7.423410in}{1.708458in}}%
\pgfpathcurveto{\pgfqpoint{7.415596in}{1.716272in}}{\pgfqpoint{7.404997in}{1.720662in}}{\pgfqpoint{7.393947in}{1.720662in}}%
\pgfpathcurveto{\pgfqpoint{7.382897in}{1.720662in}}{\pgfqpoint{7.372298in}{1.716272in}}{\pgfqpoint{7.364485in}{1.708458in}}%
\pgfpathcurveto{\pgfqpoint{7.356671in}{1.700644in}}{\pgfqpoint{7.352281in}{1.690045in}}{\pgfqpoint{7.352281in}{1.678995in}}%
\pgfpathcurveto{\pgfqpoint{7.352281in}{1.667945in}}{\pgfqpoint{7.356671in}{1.657346in}}{\pgfqpoint{7.364485in}{1.649532in}}%
\pgfpathcurveto{\pgfqpoint{7.372298in}{1.641719in}}{\pgfqpoint{7.382897in}{1.637329in}}{\pgfqpoint{7.393947in}{1.637329in}}%
\pgfpathclose%
\pgfusepath{stroke,fill}%
\end{pgfscope}%
\begin{pgfscope}%
\pgfpathrectangle{\pgfqpoint{0.526127in}{0.331635in}}{\pgfqpoint{9.300000in}{7.700000in}}%
\pgfusepath{clip}%
\pgfsetbuttcap%
\pgfsetroundjoin%
\definecolor{currentfill}{rgb}{0.815686,0.733333,1.000000}%
\pgfsetfillcolor{currentfill}%
\pgfsetlinewidth{0.481800pt}%
\definecolor{currentstroke}{rgb}{1.000000,1.000000,1.000000}%
\pgfsetstrokecolor{currentstroke}%
\pgfsetdash{}{0pt}%
\pgfpathmoveto{\pgfqpoint{7.162597in}{2.816526in}}%
\pgfpathcurveto{\pgfqpoint{7.173647in}{2.816526in}}{\pgfqpoint{7.184247in}{2.820916in}}{\pgfqpoint{7.192060in}{2.828730in}}%
\pgfpathcurveto{\pgfqpoint{7.199874in}{2.836543in}}{\pgfqpoint{7.204264in}{2.847142in}}{\pgfqpoint{7.204264in}{2.858193in}}%
\pgfpathcurveto{\pgfqpoint{7.204264in}{2.869243in}}{\pgfqpoint{7.199874in}{2.879842in}}{\pgfqpoint{7.192060in}{2.887655in}}%
\pgfpathcurveto{\pgfqpoint{7.184247in}{2.895469in}}{\pgfqpoint{7.173647in}{2.899859in}}{\pgfqpoint{7.162597in}{2.899859in}}%
\pgfpathcurveto{\pgfqpoint{7.151547in}{2.899859in}}{\pgfqpoint{7.140948in}{2.895469in}}{\pgfqpoint{7.133135in}{2.887655in}}%
\pgfpathcurveto{\pgfqpoint{7.125321in}{2.879842in}}{\pgfqpoint{7.120931in}{2.869243in}}{\pgfqpoint{7.120931in}{2.858193in}}%
\pgfpathcurveto{\pgfqpoint{7.120931in}{2.847142in}}{\pgfqpoint{7.125321in}{2.836543in}}{\pgfqpoint{7.133135in}{2.828730in}}%
\pgfpathcurveto{\pgfqpoint{7.140948in}{2.820916in}}{\pgfqpoint{7.151547in}{2.816526in}}{\pgfqpoint{7.162597in}{2.816526in}}%
\pgfpathclose%
\pgfusepath{stroke,fill}%
\end{pgfscope}%
\begin{pgfscope}%
\pgfpathrectangle{\pgfqpoint{0.526127in}{0.331635in}}{\pgfqpoint{9.300000in}{7.700000in}}%
\pgfusepath{clip}%
\pgfsetbuttcap%
\pgfsetroundjoin%
\definecolor{currentfill}{rgb}{0.815686,0.733333,1.000000}%
\pgfsetfillcolor{currentfill}%
\pgfsetlinewidth{0.481800pt}%
\definecolor{currentstroke}{rgb}{1.000000,1.000000,1.000000}%
\pgfsetstrokecolor{currentstroke}%
\pgfsetdash{}{0pt}%
\pgfpathmoveto{\pgfqpoint{6.664367in}{4.855590in}}%
\pgfpathcurveto{\pgfqpoint{6.675418in}{4.855590in}}{\pgfqpoint{6.686017in}{4.859980in}}{\pgfqpoint{6.693830in}{4.867794in}}%
\pgfpathcurveto{\pgfqpoint{6.701644in}{4.875607in}}{\pgfqpoint{6.706034in}{4.886206in}}{\pgfqpoint{6.706034in}{4.897256in}}%
\pgfpathcurveto{\pgfqpoint{6.706034in}{4.908306in}}{\pgfqpoint{6.701644in}{4.918906in}}{\pgfqpoint{6.693830in}{4.926719in}}%
\pgfpathcurveto{\pgfqpoint{6.686017in}{4.934533in}}{\pgfqpoint{6.675418in}{4.938923in}}{\pgfqpoint{6.664367in}{4.938923in}}%
\pgfpathcurveto{\pgfqpoint{6.653317in}{4.938923in}}{\pgfqpoint{6.642718in}{4.934533in}}{\pgfqpoint{6.634905in}{4.926719in}}%
\pgfpathcurveto{\pgfqpoint{6.627091in}{4.918906in}}{\pgfqpoint{6.622701in}{4.908306in}}{\pgfqpoint{6.622701in}{4.897256in}}%
\pgfpathcurveto{\pgfqpoint{6.622701in}{4.886206in}}{\pgfqpoint{6.627091in}{4.875607in}}{\pgfqpoint{6.634905in}{4.867794in}}%
\pgfpathcurveto{\pgfqpoint{6.642718in}{4.859980in}}{\pgfqpoint{6.653317in}{4.855590in}}{\pgfqpoint{6.664367in}{4.855590in}}%
\pgfpathclose%
\pgfusepath{stroke,fill}%
\end{pgfscope}%
\begin{pgfscope}%
\pgfpathrectangle{\pgfqpoint{0.526127in}{0.331635in}}{\pgfqpoint{9.300000in}{7.700000in}}%
\pgfusepath{clip}%
\pgfsetbuttcap%
\pgfsetroundjoin%
\definecolor{currentfill}{rgb}{0.815686,0.733333,1.000000}%
\pgfsetfillcolor{currentfill}%
\pgfsetlinewidth{0.481800pt}%
\definecolor{currentstroke}{rgb}{1.000000,1.000000,1.000000}%
\pgfsetstrokecolor{currentstroke}%
\pgfsetdash{}{0pt}%
\pgfpathmoveto{\pgfqpoint{5.711097in}{2.461358in}}%
\pgfpathcurveto{\pgfqpoint{5.722148in}{2.461358in}}{\pgfqpoint{5.732747in}{2.465749in}}{\pgfqpoint{5.740560in}{2.473562in}}%
\pgfpathcurveto{\pgfqpoint{5.748374in}{2.481376in}}{\pgfqpoint{5.752764in}{2.491975in}}{\pgfqpoint{5.752764in}{2.503025in}}%
\pgfpathcurveto{\pgfqpoint{5.752764in}{2.514075in}}{\pgfqpoint{5.748374in}{2.524674in}}{\pgfqpoint{5.740560in}{2.532488in}}%
\pgfpathcurveto{\pgfqpoint{5.732747in}{2.540301in}}{\pgfqpoint{5.722148in}{2.544692in}}{\pgfqpoint{5.711097in}{2.544692in}}%
\pgfpathcurveto{\pgfqpoint{5.700047in}{2.544692in}}{\pgfqpoint{5.689448in}{2.540301in}}{\pgfqpoint{5.681635in}{2.532488in}}%
\pgfpathcurveto{\pgfqpoint{5.673821in}{2.524674in}}{\pgfqpoint{5.669431in}{2.514075in}}{\pgfqpoint{5.669431in}{2.503025in}}%
\pgfpathcurveto{\pgfqpoint{5.669431in}{2.491975in}}{\pgfqpoint{5.673821in}{2.481376in}}{\pgfqpoint{5.681635in}{2.473562in}}%
\pgfpathcurveto{\pgfqpoint{5.689448in}{2.465749in}}{\pgfqpoint{5.700047in}{2.461358in}}{\pgfqpoint{5.711097in}{2.461358in}}%
\pgfpathclose%
\pgfusepath{stroke,fill}%
\end{pgfscope}%
\begin{pgfscope}%
\pgfpathrectangle{\pgfqpoint{0.526127in}{0.331635in}}{\pgfqpoint{9.300000in}{7.700000in}}%
\pgfusepath{clip}%
\pgfsetbuttcap%
\pgfsetroundjoin%
\definecolor{currentfill}{rgb}{0.815686,0.733333,1.000000}%
\pgfsetfillcolor{currentfill}%
\pgfsetlinewidth{0.481800pt}%
\definecolor{currentstroke}{rgb}{1.000000,1.000000,1.000000}%
\pgfsetstrokecolor{currentstroke}%
\pgfsetdash{}{0pt}%
\pgfpathmoveto{\pgfqpoint{6.805509in}{2.175050in}}%
\pgfpathcurveto{\pgfqpoint{6.816559in}{2.175050in}}{\pgfqpoint{6.827158in}{2.179440in}}{\pgfqpoint{6.834972in}{2.187254in}}%
\pgfpathcurveto{\pgfqpoint{6.842785in}{2.195068in}}{\pgfqpoint{6.847175in}{2.205667in}}{\pgfqpoint{6.847175in}{2.216717in}}%
\pgfpathcurveto{\pgfqpoint{6.847175in}{2.227767in}}{\pgfqpoint{6.842785in}{2.238366in}}{\pgfqpoint{6.834972in}{2.246180in}}%
\pgfpathcurveto{\pgfqpoint{6.827158in}{2.253993in}}{\pgfqpoint{6.816559in}{2.258383in}}{\pgfqpoint{6.805509in}{2.258383in}}%
\pgfpathcurveto{\pgfqpoint{6.794459in}{2.258383in}}{\pgfqpoint{6.783860in}{2.253993in}}{\pgfqpoint{6.776046in}{2.246180in}}%
\pgfpathcurveto{\pgfqpoint{6.768232in}{2.238366in}}{\pgfqpoint{6.763842in}{2.227767in}}{\pgfqpoint{6.763842in}{2.216717in}}%
\pgfpathcurveto{\pgfqpoint{6.763842in}{2.205667in}}{\pgfqpoint{6.768232in}{2.195068in}}{\pgfqpoint{6.776046in}{2.187254in}}%
\pgfpathcurveto{\pgfqpoint{6.783860in}{2.179440in}}{\pgfqpoint{6.794459in}{2.175050in}}{\pgfqpoint{6.805509in}{2.175050in}}%
\pgfpathclose%
\pgfusepath{stroke,fill}%
\end{pgfscope}%
\begin{pgfscope}%
\pgfpathrectangle{\pgfqpoint{0.526127in}{0.331635in}}{\pgfqpoint{9.300000in}{7.700000in}}%
\pgfusepath{clip}%
\pgfsetbuttcap%
\pgfsetroundjoin%
\definecolor{currentfill}{rgb}{0.815686,0.733333,1.000000}%
\pgfsetfillcolor{currentfill}%
\pgfsetlinewidth{0.481800pt}%
\definecolor{currentstroke}{rgb}{1.000000,1.000000,1.000000}%
\pgfsetstrokecolor{currentstroke}%
\pgfsetdash{}{0pt}%
\pgfpathmoveto{\pgfqpoint{3.875538in}{2.264873in}}%
\pgfpathcurveto{\pgfqpoint{3.886588in}{2.264873in}}{\pgfqpoint{3.897187in}{2.269263in}}{\pgfqpoint{3.905000in}{2.277077in}}%
\pgfpathcurveto{\pgfqpoint{3.912814in}{2.284890in}}{\pgfqpoint{3.917204in}{2.295490in}}{\pgfqpoint{3.917204in}{2.306540in}}%
\pgfpathcurveto{\pgfqpoint{3.917204in}{2.317590in}}{\pgfqpoint{3.912814in}{2.328189in}}{\pgfqpoint{3.905000in}{2.336002in}}%
\pgfpathcurveto{\pgfqpoint{3.897187in}{2.343816in}}{\pgfqpoint{3.886588in}{2.348206in}}{\pgfqpoint{3.875538in}{2.348206in}}%
\pgfpathcurveto{\pgfqpoint{3.864487in}{2.348206in}}{\pgfqpoint{3.853888in}{2.343816in}}{\pgfqpoint{3.846075in}{2.336002in}}%
\pgfpathcurveto{\pgfqpoint{3.838261in}{2.328189in}}{\pgfqpoint{3.833871in}{2.317590in}}{\pgfqpoint{3.833871in}{2.306540in}}%
\pgfpathcurveto{\pgfqpoint{3.833871in}{2.295490in}}{\pgfqpoint{3.838261in}{2.284890in}}{\pgfqpoint{3.846075in}{2.277077in}}%
\pgfpathcurveto{\pgfqpoint{3.853888in}{2.269263in}}{\pgfqpoint{3.864487in}{2.264873in}}{\pgfqpoint{3.875538in}{2.264873in}}%
\pgfpathclose%
\pgfusepath{stroke,fill}%
\end{pgfscope}%
\begin{pgfscope}%
\pgfpathrectangle{\pgfqpoint{0.526127in}{0.331635in}}{\pgfqpoint{9.300000in}{7.700000in}}%
\pgfusepath{clip}%
\pgfsetbuttcap%
\pgfsetroundjoin%
\definecolor{currentfill}{rgb}{0.815686,0.733333,1.000000}%
\pgfsetfillcolor{currentfill}%
\pgfsetlinewidth{0.481800pt}%
\definecolor{currentstroke}{rgb}{1.000000,1.000000,1.000000}%
\pgfsetstrokecolor{currentstroke}%
\pgfsetdash{}{0pt}%
\pgfpathmoveto{\pgfqpoint{1.970636in}{4.841715in}}%
\pgfpathcurveto{\pgfqpoint{1.981686in}{4.841715in}}{\pgfqpoint{1.992285in}{4.846105in}}{\pgfqpoint{2.000099in}{4.853919in}}%
\pgfpathcurveto{\pgfqpoint{2.007912in}{4.861733in}}{\pgfqpoint{2.012303in}{4.872332in}}{\pgfqpoint{2.012303in}{4.883382in}}%
\pgfpathcurveto{\pgfqpoint{2.012303in}{4.894432in}}{\pgfqpoint{2.007912in}{4.905031in}}{\pgfqpoint{2.000099in}{4.912845in}}%
\pgfpathcurveto{\pgfqpoint{1.992285in}{4.920658in}}{\pgfqpoint{1.981686in}{4.925048in}}{\pgfqpoint{1.970636in}{4.925048in}}%
\pgfpathcurveto{\pgfqpoint{1.959586in}{4.925048in}}{\pgfqpoint{1.948987in}{4.920658in}}{\pgfqpoint{1.941173in}{4.912845in}}%
\pgfpathcurveto{\pgfqpoint{1.933359in}{4.905031in}}{\pgfqpoint{1.928969in}{4.894432in}}{\pgfqpoint{1.928969in}{4.883382in}}%
\pgfpathcurveto{\pgfqpoint{1.928969in}{4.872332in}}{\pgfqpoint{1.933359in}{4.861733in}}{\pgfqpoint{1.941173in}{4.853919in}}%
\pgfpathcurveto{\pgfqpoint{1.948987in}{4.846105in}}{\pgfqpoint{1.959586in}{4.841715in}}{\pgfqpoint{1.970636in}{4.841715in}}%
\pgfpathclose%
\pgfusepath{stroke,fill}%
\end{pgfscope}%
\begin{pgfscope}%
\pgfpathrectangle{\pgfqpoint{0.526127in}{0.331635in}}{\pgfqpoint{9.300000in}{7.700000in}}%
\pgfusepath{clip}%
\pgfsetbuttcap%
\pgfsetroundjoin%
\definecolor{currentfill}{rgb}{0.815686,0.733333,1.000000}%
\pgfsetfillcolor{currentfill}%
\pgfsetlinewidth{0.481800pt}%
\definecolor{currentstroke}{rgb}{1.000000,1.000000,1.000000}%
\pgfsetstrokecolor{currentstroke}%
\pgfsetdash{}{0pt}%
\pgfpathmoveto{\pgfqpoint{3.343614in}{2.625512in}}%
\pgfpathcurveto{\pgfqpoint{3.354664in}{2.625512in}}{\pgfqpoint{3.365263in}{2.629902in}}{\pgfqpoint{3.373077in}{2.637716in}}%
\pgfpathcurveto{\pgfqpoint{3.380890in}{2.645530in}}{\pgfqpoint{3.385281in}{2.656129in}}{\pgfqpoint{3.385281in}{2.667179in}}%
\pgfpathcurveto{\pgfqpoint{3.385281in}{2.678229in}}{\pgfqpoint{3.380890in}{2.688828in}}{\pgfqpoint{3.373077in}{2.696641in}}%
\pgfpathcurveto{\pgfqpoint{3.365263in}{2.704455in}}{\pgfqpoint{3.354664in}{2.708845in}}{\pgfqpoint{3.343614in}{2.708845in}}%
\pgfpathcurveto{\pgfqpoint{3.332564in}{2.708845in}}{\pgfqpoint{3.321965in}{2.704455in}}{\pgfqpoint{3.314151in}{2.696641in}}%
\pgfpathcurveto{\pgfqpoint{3.306337in}{2.688828in}}{\pgfqpoint{3.301947in}{2.678229in}}{\pgfqpoint{3.301947in}{2.667179in}}%
\pgfpathcurveto{\pgfqpoint{3.301947in}{2.656129in}}{\pgfqpoint{3.306337in}{2.645530in}}{\pgfqpoint{3.314151in}{2.637716in}}%
\pgfpathcurveto{\pgfqpoint{3.321965in}{2.629902in}}{\pgfqpoint{3.332564in}{2.625512in}}{\pgfqpoint{3.343614in}{2.625512in}}%
\pgfpathclose%
\pgfusepath{stroke,fill}%
\end{pgfscope}%
\begin{pgfscope}%
\pgfpathrectangle{\pgfqpoint{0.526127in}{0.331635in}}{\pgfqpoint{9.300000in}{7.700000in}}%
\pgfusepath{clip}%
\pgfsetbuttcap%
\pgfsetroundjoin%
\definecolor{currentfill}{rgb}{0.815686,0.733333,1.000000}%
\pgfsetfillcolor{currentfill}%
\pgfsetlinewidth{0.481800pt}%
\definecolor{currentstroke}{rgb}{1.000000,1.000000,1.000000}%
\pgfsetstrokecolor{currentstroke}%
\pgfsetdash{}{0pt}%
\pgfpathmoveto{\pgfqpoint{6.735659in}{3.815807in}}%
\pgfpathcurveto{\pgfqpoint{6.746709in}{3.815807in}}{\pgfqpoint{6.757308in}{3.820198in}}{\pgfqpoint{6.765122in}{3.828011in}}%
\pgfpathcurveto{\pgfqpoint{6.772935in}{3.835825in}}{\pgfqpoint{6.777326in}{3.846424in}}{\pgfqpoint{6.777326in}{3.857474in}}%
\pgfpathcurveto{\pgfqpoint{6.777326in}{3.868524in}}{\pgfqpoint{6.772935in}{3.879123in}}{\pgfqpoint{6.765122in}{3.886937in}}%
\pgfpathcurveto{\pgfqpoint{6.757308in}{3.894751in}}{\pgfqpoint{6.746709in}{3.899141in}}{\pgfqpoint{6.735659in}{3.899141in}}%
\pgfpathcurveto{\pgfqpoint{6.724609in}{3.899141in}}{\pgfqpoint{6.714010in}{3.894751in}}{\pgfqpoint{6.706196in}{3.886937in}}%
\pgfpathcurveto{\pgfqpoint{6.698383in}{3.879123in}}{\pgfqpoint{6.693992in}{3.868524in}}{\pgfqpoint{6.693992in}{3.857474in}}%
\pgfpathcurveto{\pgfqpoint{6.693992in}{3.846424in}}{\pgfqpoint{6.698383in}{3.835825in}}{\pgfqpoint{6.706196in}{3.828011in}}%
\pgfpathcurveto{\pgfqpoint{6.714010in}{3.820198in}}{\pgfqpoint{6.724609in}{3.815807in}}{\pgfqpoint{6.735659in}{3.815807in}}%
\pgfpathclose%
\pgfusepath{stroke,fill}%
\end{pgfscope}%
\begin{pgfscope}%
\pgfpathrectangle{\pgfqpoint{0.526127in}{0.331635in}}{\pgfqpoint{9.300000in}{7.700000in}}%
\pgfusepath{clip}%
\pgfsetbuttcap%
\pgfsetroundjoin%
\definecolor{currentfill}{rgb}{0.815686,0.733333,1.000000}%
\pgfsetfillcolor{currentfill}%
\pgfsetlinewidth{0.481800pt}%
\definecolor{currentstroke}{rgb}{1.000000,1.000000,1.000000}%
\pgfsetstrokecolor{currentstroke}%
\pgfsetdash{}{0pt}%
\pgfpathmoveto{\pgfqpoint{2.507937in}{5.074246in}}%
\pgfpathcurveto{\pgfqpoint{2.518987in}{5.074246in}}{\pgfqpoint{2.529586in}{5.078636in}}{\pgfqpoint{2.537400in}{5.086450in}}%
\pgfpathcurveto{\pgfqpoint{2.545214in}{5.094263in}}{\pgfqpoint{2.549604in}{5.104862in}}{\pgfqpoint{2.549604in}{5.115912in}}%
\pgfpathcurveto{\pgfqpoint{2.549604in}{5.126963in}}{\pgfqpoint{2.545214in}{5.137562in}}{\pgfqpoint{2.537400in}{5.145375in}}%
\pgfpathcurveto{\pgfqpoint{2.529586in}{5.153189in}}{\pgfqpoint{2.518987in}{5.157579in}}{\pgfqpoint{2.507937in}{5.157579in}}%
\pgfpathcurveto{\pgfqpoint{2.496887in}{5.157579in}}{\pgfqpoint{2.486288in}{5.153189in}}{\pgfqpoint{2.478475in}{5.145375in}}%
\pgfpathcurveto{\pgfqpoint{2.470661in}{5.137562in}}{\pgfqpoint{2.466271in}{5.126963in}}{\pgfqpoint{2.466271in}{5.115912in}}%
\pgfpathcurveto{\pgfqpoint{2.466271in}{5.104862in}}{\pgfqpoint{2.470661in}{5.094263in}}{\pgfqpoint{2.478475in}{5.086450in}}%
\pgfpathcurveto{\pgfqpoint{2.486288in}{5.078636in}}{\pgfqpoint{2.496887in}{5.074246in}}{\pgfqpoint{2.507937in}{5.074246in}}%
\pgfpathclose%
\pgfusepath{stroke,fill}%
\end{pgfscope}%
\begin{pgfscope}%
\pgfpathrectangle{\pgfqpoint{0.526127in}{0.331635in}}{\pgfqpoint{9.300000in}{7.700000in}}%
\pgfusepath{clip}%
\pgfsetbuttcap%
\pgfsetroundjoin%
\definecolor{currentfill}{rgb}{0.815686,0.733333,1.000000}%
\pgfsetfillcolor{currentfill}%
\pgfsetlinewidth{0.481800pt}%
\definecolor{currentstroke}{rgb}{1.000000,1.000000,1.000000}%
\pgfsetstrokecolor{currentstroke}%
\pgfsetdash{}{0pt}%
\pgfpathmoveto{\pgfqpoint{8.582505in}{3.915961in}}%
\pgfpathcurveto{\pgfqpoint{8.593555in}{3.915961in}}{\pgfqpoint{8.604154in}{3.920351in}}{\pgfqpoint{8.611968in}{3.928165in}}%
\pgfpathcurveto{\pgfqpoint{8.619781in}{3.935978in}}{\pgfqpoint{8.624171in}{3.946577in}}{\pgfqpoint{8.624171in}{3.957628in}}%
\pgfpathcurveto{\pgfqpoint{8.624171in}{3.968678in}}{\pgfqpoint{8.619781in}{3.979277in}}{\pgfqpoint{8.611968in}{3.987090in}}%
\pgfpathcurveto{\pgfqpoint{8.604154in}{3.994904in}}{\pgfqpoint{8.593555in}{3.999294in}}{\pgfqpoint{8.582505in}{3.999294in}}%
\pgfpathcurveto{\pgfqpoint{8.571455in}{3.999294in}}{\pgfqpoint{8.560856in}{3.994904in}}{\pgfqpoint{8.553042in}{3.987090in}}%
\pgfpathcurveto{\pgfqpoint{8.545228in}{3.979277in}}{\pgfqpoint{8.540838in}{3.968678in}}{\pgfqpoint{8.540838in}{3.957628in}}%
\pgfpathcurveto{\pgfqpoint{8.540838in}{3.946577in}}{\pgfqpoint{8.545228in}{3.935978in}}{\pgfqpoint{8.553042in}{3.928165in}}%
\pgfpathcurveto{\pgfqpoint{8.560856in}{3.920351in}}{\pgfqpoint{8.571455in}{3.915961in}}{\pgfqpoint{8.582505in}{3.915961in}}%
\pgfpathclose%
\pgfusepath{stroke,fill}%
\end{pgfscope}%
\begin{pgfscope}%
\pgfpathrectangle{\pgfqpoint{0.526127in}{0.331635in}}{\pgfqpoint{9.300000in}{7.700000in}}%
\pgfusepath{clip}%
\pgfsetbuttcap%
\pgfsetroundjoin%
\definecolor{currentfill}{rgb}{0.815686,0.733333,1.000000}%
\pgfsetfillcolor{currentfill}%
\pgfsetlinewidth{0.481800pt}%
\definecolor{currentstroke}{rgb}{1.000000,1.000000,1.000000}%
\pgfsetstrokecolor{currentstroke}%
\pgfsetdash{}{0pt}%
\pgfpathmoveto{\pgfqpoint{5.435254in}{1.741114in}}%
\pgfpathcurveto{\pgfqpoint{5.446304in}{1.741114in}}{\pgfqpoint{5.456903in}{1.745505in}}{\pgfqpoint{5.464717in}{1.753318in}}%
\pgfpathcurveto{\pgfqpoint{5.472530in}{1.761132in}}{\pgfqpoint{5.476921in}{1.771731in}}{\pgfqpoint{5.476921in}{1.782781in}}%
\pgfpathcurveto{\pgfqpoint{5.476921in}{1.793831in}}{\pgfqpoint{5.472530in}{1.804430in}}{\pgfqpoint{5.464717in}{1.812244in}}%
\pgfpathcurveto{\pgfqpoint{5.456903in}{1.820057in}}{\pgfqpoint{5.446304in}{1.824448in}}{\pgfqpoint{5.435254in}{1.824448in}}%
\pgfpathcurveto{\pgfqpoint{5.424204in}{1.824448in}}{\pgfqpoint{5.413605in}{1.820057in}}{\pgfqpoint{5.405791in}{1.812244in}}%
\pgfpathcurveto{\pgfqpoint{5.397978in}{1.804430in}}{\pgfqpoint{5.393587in}{1.793831in}}{\pgfqpoint{5.393587in}{1.782781in}}%
\pgfpathcurveto{\pgfqpoint{5.393587in}{1.771731in}}{\pgfqpoint{5.397978in}{1.761132in}}{\pgfqpoint{5.405791in}{1.753318in}}%
\pgfpathcurveto{\pgfqpoint{5.413605in}{1.745505in}}{\pgfqpoint{5.424204in}{1.741114in}}{\pgfqpoint{5.435254in}{1.741114in}}%
\pgfpathclose%
\pgfusepath{stroke,fill}%
\end{pgfscope}%
\begin{pgfscope}%
\pgfpathrectangle{\pgfqpoint{0.526127in}{0.331635in}}{\pgfqpoint{9.300000in}{7.700000in}}%
\pgfusepath{clip}%
\pgfsetbuttcap%
\pgfsetroundjoin%
\definecolor{currentfill}{rgb}{0.815686,0.733333,1.000000}%
\pgfsetfillcolor{currentfill}%
\pgfsetlinewidth{0.481800pt}%
\definecolor{currentstroke}{rgb}{1.000000,1.000000,1.000000}%
\pgfsetstrokecolor{currentstroke}%
\pgfsetdash{}{0pt}%
\pgfpathmoveto{\pgfqpoint{5.661250in}{4.627366in}}%
\pgfpathcurveto{\pgfqpoint{5.672300in}{4.627366in}}{\pgfqpoint{5.682899in}{4.631756in}}{\pgfqpoint{5.690713in}{4.639570in}}%
\pgfpathcurveto{\pgfqpoint{5.698527in}{4.647383in}}{\pgfqpoint{5.702917in}{4.657982in}}{\pgfqpoint{5.702917in}{4.669033in}}%
\pgfpathcurveto{\pgfqpoint{5.702917in}{4.680083in}}{\pgfqpoint{5.698527in}{4.690682in}}{\pgfqpoint{5.690713in}{4.698495in}}%
\pgfpathcurveto{\pgfqpoint{5.682899in}{4.706309in}}{\pgfqpoint{5.672300in}{4.710699in}}{\pgfqpoint{5.661250in}{4.710699in}}%
\pgfpathcurveto{\pgfqpoint{5.650200in}{4.710699in}}{\pgfqpoint{5.639601in}{4.706309in}}{\pgfqpoint{5.631787in}{4.698495in}}%
\pgfpathcurveto{\pgfqpoint{5.623974in}{4.690682in}}{\pgfqpoint{5.619583in}{4.680083in}}{\pgfqpoint{5.619583in}{4.669033in}}%
\pgfpathcurveto{\pgfqpoint{5.619583in}{4.657982in}}{\pgfqpoint{5.623974in}{4.647383in}}{\pgfqpoint{5.631787in}{4.639570in}}%
\pgfpathcurveto{\pgfqpoint{5.639601in}{4.631756in}}{\pgfqpoint{5.650200in}{4.627366in}}{\pgfqpoint{5.661250in}{4.627366in}}%
\pgfpathclose%
\pgfusepath{stroke,fill}%
\end{pgfscope}%
\begin{pgfscope}%
\pgfpathrectangle{\pgfqpoint{0.526127in}{0.331635in}}{\pgfqpoint{9.300000in}{7.700000in}}%
\pgfusepath{clip}%
\pgfsetbuttcap%
\pgfsetroundjoin%
\definecolor{currentfill}{rgb}{0.815686,0.733333,1.000000}%
\pgfsetfillcolor{currentfill}%
\pgfsetlinewidth{0.481800pt}%
\definecolor{currentstroke}{rgb}{1.000000,1.000000,1.000000}%
\pgfsetstrokecolor{currentstroke}%
\pgfsetdash{}{0pt}%
\pgfpathmoveto{\pgfqpoint{2.380624in}{2.662548in}}%
\pgfpathcurveto{\pgfqpoint{2.391674in}{2.662548in}}{\pgfqpoint{2.402273in}{2.666939in}}{\pgfqpoint{2.410087in}{2.674752in}}%
\pgfpathcurveto{\pgfqpoint{2.417900in}{2.682566in}}{\pgfqpoint{2.422290in}{2.693165in}}{\pgfqpoint{2.422290in}{2.704215in}}%
\pgfpathcurveto{\pgfqpoint{2.422290in}{2.715265in}}{\pgfqpoint{2.417900in}{2.725864in}}{\pgfqpoint{2.410087in}{2.733678in}}%
\pgfpathcurveto{\pgfqpoint{2.402273in}{2.741491in}}{\pgfqpoint{2.391674in}{2.745882in}}{\pgfqpoint{2.380624in}{2.745882in}}%
\pgfpathcurveto{\pgfqpoint{2.369574in}{2.745882in}}{\pgfqpoint{2.358975in}{2.741491in}}{\pgfqpoint{2.351161in}{2.733678in}}%
\pgfpathcurveto{\pgfqpoint{2.343347in}{2.725864in}}{\pgfqpoint{2.338957in}{2.715265in}}{\pgfqpoint{2.338957in}{2.704215in}}%
\pgfpathcurveto{\pgfqpoint{2.338957in}{2.693165in}}{\pgfqpoint{2.343347in}{2.682566in}}{\pgfqpoint{2.351161in}{2.674752in}}%
\pgfpathcurveto{\pgfqpoint{2.358975in}{2.666939in}}{\pgfqpoint{2.369574in}{2.662548in}}{\pgfqpoint{2.380624in}{2.662548in}}%
\pgfpathclose%
\pgfusepath{stroke,fill}%
\end{pgfscope}%
\begin{pgfscope}%
\pgfpathrectangle{\pgfqpoint{0.526127in}{0.331635in}}{\pgfqpoint{9.300000in}{7.700000in}}%
\pgfusepath{clip}%
\pgfsetbuttcap%
\pgfsetroundjoin%
\definecolor{currentfill}{rgb}{0.815686,0.733333,1.000000}%
\pgfsetfillcolor{currentfill}%
\pgfsetlinewidth{0.481800pt}%
\definecolor{currentstroke}{rgb}{1.000000,1.000000,1.000000}%
\pgfsetstrokecolor{currentstroke}%
\pgfsetdash{}{0pt}%
\pgfpathmoveto{\pgfqpoint{3.490954in}{3.314820in}}%
\pgfpathcurveto{\pgfqpoint{3.502004in}{3.314820in}}{\pgfqpoint{3.512603in}{3.319210in}}{\pgfqpoint{3.520417in}{3.327024in}}%
\pgfpathcurveto{\pgfqpoint{3.528231in}{3.334838in}}{\pgfqpoint{3.532621in}{3.345437in}}{\pgfqpoint{3.532621in}{3.356487in}}%
\pgfpathcurveto{\pgfqpoint{3.532621in}{3.367537in}}{\pgfqpoint{3.528231in}{3.378136in}}{\pgfqpoint{3.520417in}{3.385950in}}%
\pgfpathcurveto{\pgfqpoint{3.512603in}{3.393763in}}{\pgfqpoint{3.502004in}{3.398154in}}{\pgfqpoint{3.490954in}{3.398154in}}%
\pgfpathcurveto{\pgfqpoint{3.479904in}{3.398154in}}{\pgfqpoint{3.469305in}{3.393763in}}{\pgfqpoint{3.461491in}{3.385950in}}%
\pgfpathcurveto{\pgfqpoint{3.453678in}{3.378136in}}{\pgfqpoint{3.449287in}{3.367537in}}{\pgfqpoint{3.449287in}{3.356487in}}%
\pgfpathcurveto{\pgfqpoint{3.449287in}{3.345437in}}{\pgfqpoint{3.453678in}{3.334838in}}{\pgfqpoint{3.461491in}{3.327024in}}%
\pgfpathcurveto{\pgfqpoint{3.469305in}{3.319210in}}{\pgfqpoint{3.479904in}{3.314820in}}{\pgfqpoint{3.490954in}{3.314820in}}%
\pgfpathclose%
\pgfusepath{stroke,fill}%
\end{pgfscope}%
\begin{pgfscope}%
\pgfpathrectangle{\pgfqpoint{0.526127in}{0.331635in}}{\pgfqpoint{9.300000in}{7.700000in}}%
\pgfusepath{clip}%
\pgfsetbuttcap%
\pgfsetroundjoin%
\definecolor{currentfill}{rgb}{0.815686,0.733333,1.000000}%
\pgfsetfillcolor{currentfill}%
\pgfsetlinewidth{0.481800pt}%
\definecolor{currentstroke}{rgb}{1.000000,1.000000,1.000000}%
\pgfsetstrokecolor{currentstroke}%
\pgfsetdash{}{0pt}%
\pgfpathmoveto{\pgfqpoint{4.801136in}{0.732417in}}%
\pgfpathcurveto{\pgfqpoint{4.812186in}{0.732417in}}{\pgfqpoint{4.822785in}{0.736808in}}{\pgfqpoint{4.830599in}{0.744621in}}%
\pgfpathcurveto{\pgfqpoint{4.838412in}{0.752435in}}{\pgfqpoint{4.842803in}{0.763034in}}{\pgfqpoint{4.842803in}{0.774084in}}%
\pgfpathcurveto{\pgfqpoint{4.842803in}{0.785134in}}{\pgfqpoint{4.838412in}{0.795733in}}{\pgfqpoint{4.830599in}{0.803547in}}%
\pgfpathcurveto{\pgfqpoint{4.822785in}{0.811360in}}{\pgfqpoint{4.812186in}{0.815751in}}{\pgfqpoint{4.801136in}{0.815751in}}%
\pgfpathcurveto{\pgfqpoint{4.790086in}{0.815751in}}{\pgfqpoint{4.779487in}{0.811360in}}{\pgfqpoint{4.771673in}{0.803547in}}%
\pgfpathcurveto{\pgfqpoint{4.763860in}{0.795733in}}{\pgfqpoint{4.759469in}{0.785134in}}{\pgfqpoint{4.759469in}{0.774084in}}%
\pgfpathcurveto{\pgfqpoint{4.759469in}{0.763034in}}{\pgfqpoint{4.763860in}{0.752435in}}{\pgfqpoint{4.771673in}{0.744621in}}%
\pgfpathcurveto{\pgfqpoint{4.779487in}{0.736808in}}{\pgfqpoint{4.790086in}{0.732417in}}{\pgfqpoint{4.801136in}{0.732417in}}%
\pgfpathclose%
\pgfusepath{stroke,fill}%
\end{pgfscope}%
\begin{pgfscope}%
\pgfpathrectangle{\pgfqpoint{0.526127in}{0.331635in}}{\pgfqpoint{9.300000in}{7.700000in}}%
\pgfusepath{clip}%
\pgfsetbuttcap%
\pgfsetroundjoin%
\definecolor{currentfill}{rgb}{0.815686,0.733333,1.000000}%
\pgfsetfillcolor{currentfill}%
\pgfsetlinewidth{0.481800pt}%
\definecolor{currentstroke}{rgb}{1.000000,1.000000,1.000000}%
\pgfsetstrokecolor{currentstroke}%
\pgfsetdash{}{0pt}%
\pgfpathmoveto{\pgfqpoint{7.283490in}{1.678495in}}%
\pgfpathcurveto{\pgfqpoint{7.294540in}{1.678495in}}{\pgfqpoint{7.305139in}{1.682885in}}{\pgfqpoint{7.312953in}{1.690699in}}%
\pgfpathcurveto{\pgfqpoint{7.320766in}{1.698512in}}{\pgfqpoint{7.325157in}{1.709111in}}{\pgfqpoint{7.325157in}{1.720161in}}%
\pgfpathcurveto{\pgfqpoint{7.325157in}{1.731212in}}{\pgfqpoint{7.320766in}{1.741811in}}{\pgfqpoint{7.312953in}{1.749624in}}%
\pgfpathcurveto{\pgfqpoint{7.305139in}{1.757438in}}{\pgfqpoint{7.294540in}{1.761828in}}{\pgfqpoint{7.283490in}{1.761828in}}%
\pgfpathcurveto{\pgfqpoint{7.272440in}{1.761828in}}{\pgfqpoint{7.261841in}{1.757438in}}{\pgfqpoint{7.254027in}{1.749624in}}%
\pgfpathcurveto{\pgfqpoint{7.246214in}{1.741811in}}{\pgfqpoint{7.241823in}{1.731212in}}{\pgfqpoint{7.241823in}{1.720161in}}%
\pgfpathcurveto{\pgfqpoint{7.241823in}{1.709111in}}{\pgfqpoint{7.246214in}{1.698512in}}{\pgfqpoint{7.254027in}{1.690699in}}%
\pgfpathcurveto{\pgfqpoint{7.261841in}{1.682885in}}{\pgfqpoint{7.272440in}{1.678495in}}{\pgfqpoint{7.283490in}{1.678495in}}%
\pgfpathclose%
\pgfusepath{stroke,fill}%
\end{pgfscope}%
\begin{pgfscope}%
\pgfpathrectangle{\pgfqpoint{0.526127in}{0.331635in}}{\pgfqpoint{9.300000in}{7.700000in}}%
\pgfusepath{clip}%
\pgfsetbuttcap%
\pgfsetroundjoin%
\definecolor{currentfill}{rgb}{0.815686,0.733333,1.000000}%
\pgfsetfillcolor{currentfill}%
\pgfsetlinewidth{0.481800pt}%
\definecolor{currentstroke}{rgb}{1.000000,1.000000,1.000000}%
\pgfsetstrokecolor{currentstroke}%
\pgfsetdash{}{0pt}%
\pgfpathmoveto{\pgfqpoint{4.932772in}{1.200226in}}%
\pgfpathcurveto{\pgfqpoint{4.943822in}{1.200226in}}{\pgfqpoint{4.954421in}{1.204617in}}{\pgfqpoint{4.962235in}{1.212430in}}%
\pgfpathcurveto{\pgfqpoint{4.970048in}{1.220244in}}{\pgfqpoint{4.974438in}{1.230843in}}{\pgfqpoint{4.974438in}{1.241893in}}%
\pgfpathcurveto{\pgfqpoint{4.974438in}{1.252943in}}{\pgfqpoint{4.970048in}{1.263542in}}{\pgfqpoint{4.962235in}{1.271356in}}%
\pgfpathcurveto{\pgfqpoint{4.954421in}{1.279169in}}{\pgfqpoint{4.943822in}{1.283560in}}{\pgfqpoint{4.932772in}{1.283560in}}%
\pgfpathcurveto{\pgfqpoint{4.921722in}{1.283560in}}{\pgfqpoint{4.911123in}{1.279169in}}{\pgfqpoint{4.903309in}{1.271356in}}%
\pgfpathcurveto{\pgfqpoint{4.895495in}{1.263542in}}{\pgfqpoint{4.891105in}{1.252943in}}{\pgfqpoint{4.891105in}{1.241893in}}%
\pgfpathcurveto{\pgfqpoint{4.891105in}{1.230843in}}{\pgfqpoint{4.895495in}{1.220244in}}{\pgfqpoint{4.903309in}{1.212430in}}%
\pgfpathcurveto{\pgfqpoint{4.911123in}{1.204617in}}{\pgfqpoint{4.921722in}{1.200226in}}{\pgfqpoint{4.932772in}{1.200226in}}%
\pgfpathclose%
\pgfusepath{stroke,fill}%
\end{pgfscope}%
\begin{pgfscope}%
\pgfpathrectangle{\pgfqpoint{0.526127in}{0.331635in}}{\pgfqpoint{9.300000in}{7.700000in}}%
\pgfusepath{clip}%
\pgfsetbuttcap%
\pgfsetroundjoin%
\definecolor{currentfill}{rgb}{0.815686,0.733333,1.000000}%
\pgfsetfillcolor{currentfill}%
\pgfsetlinewidth{0.481800pt}%
\definecolor{currentstroke}{rgb}{1.000000,1.000000,1.000000}%
\pgfsetstrokecolor{currentstroke}%
\pgfsetdash{}{0pt}%
\pgfpathmoveto{\pgfqpoint{7.494242in}{1.650056in}}%
\pgfpathcurveto{\pgfqpoint{7.505292in}{1.650056in}}{\pgfqpoint{7.515891in}{1.654446in}}{\pgfqpoint{7.523705in}{1.662260in}}%
\pgfpathcurveto{\pgfqpoint{7.531519in}{1.670074in}}{\pgfqpoint{7.535909in}{1.680673in}}{\pgfqpoint{7.535909in}{1.691723in}}%
\pgfpathcurveto{\pgfqpoint{7.535909in}{1.702773in}}{\pgfqpoint{7.531519in}{1.713372in}}{\pgfqpoint{7.523705in}{1.721186in}}%
\pgfpathcurveto{\pgfqpoint{7.515891in}{1.728999in}}{\pgfqpoint{7.505292in}{1.733390in}}{\pgfqpoint{7.494242in}{1.733390in}}%
\pgfpathcurveto{\pgfqpoint{7.483192in}{1.733390in}}{\pgfqpoint{7.472593in}{1.728999in}}{\pgfqpoint{7.464779in}{1.721186in}}%
\pgfpathcurveto{\pgfqpoint{7.456966in}{1.713372in}}{\pgfqpoint{7.452576in}{1.702773in}}{\pgfqpoint{7.452576in}{1.691723in}}%
\pgfpathcurveto{\pgfqpoint{7.452576in}{1.680673in}}{\pgfqpoint{7.456966in}{1.670074in}}{\pgfqpoint{7.464779in}{1.662260in}}%
\pgfpathcurveto{\pgfqpoint{7.472593in}{1.654446in}}{\pgfqpoint{7.483192in}{1.650056in}}{\pgfqpoint{7.494242in}{1.650056in}}%
\pgfpathclose%
\pgfusepath{stroke,fill}%
\end{pgfscope}%
\begin{pgfscope}%
\pgfpathrectangle{\pgfqpoint{0.526127in}{0.331635in}}{\pgfqpoint{9.300000in}{7.700000in}}%
\pgfusepath{clip}%
\pgfsetbuttcap%
\pgfsetroundjoin%
\definecolor{currentfill}{rgb}{0.815686,0.733333,1.000000}%
\pgfsetfillcolor{currentfill}%
\pgfsetlinewidth{0.481800pt}%
\definecolor{currentstroke}{rgb}{1.000000,1.000000,1.000000}%
\pgfsetstrokecolor{currentstroke}%
\pgfsetdash{}{0pt}%
\pgfpathmoveto{\pgfqpoint{8.328870in}{2.285356in}}%
\pgfpathcurveto{\pgfqpoint{8.339920in}{2.285356in}}{\pgfqpoint{8.350519in}{2.289746in}}{\pgfqpoint{8.358333in}{2.297560in}}%
\pgfpathcurveto{\pgfqpoint{8.366146in}{2.305374in}}{\pgfqpoint{8.370537in}{2.315973in}}{\pgfqpoint{8.370537in}{2.327023in}}%
\pgfpathcurveto{\pgfqpoint{8.370537in}{2.338073in}}{\pgfqpoint{8.366146in}{2.348672in}}{\pgfqpoint{8.358333in}{2.356486in}}%
\pgfpathcurveto{\pgfqpoint{8.350519in}{2.364299in}}{\pgfqpoint{8.339920in}{2.368689in}}{\pgfqpoint{8.328870in}{2.368689in}}%
\pgfpathcurveto{\pgfqpoint{8.317820in}{2.368689in}}{\pgfqpoint{8.307221in}{2.364299in}}{\pgfqpoint{8.299407in}{2.356486in}}%
\pgfpathcurveto{\pgfqpoint{8.291593in}{2.348672in}}{\pgfqpoint{8.287203in}{2.338073in}}{\pgfqpoint{8.287203in}{2.327023in}}%
\pgfpathcurveto{\pgfqpoint{8.287203in}{2.315973in}}{\pgfqpoint{8.291593in}{2.305374in}}{\pgfqpoint{8.299407in}{2.297560in}}%
\pgfpathcurveto{\pgfqpoint{8.307221in}{2.289746in}}{\pgfqpoint{8.317820in}{2.285356in}}{\pgfqpoint{8.328870in}{2.285356in}}%
\pgfpathclose%
\pgfusepath{stroke,fill}%
\end{pgfscope}%
\begin{pgfscope}%
\pgfpathrectangle{\pgfqpoint{0.526127in}{0.331635in}}{\pgfqpoint{9.300000in}{7.700000in}}%
\pgfusepath{clip}%
\pgfsetbuttcap%
\pgfsetroundjoin%
\definecolor{currentfill}{rgb}{0.815686,0.733333,1.000000}%
\pgfsetfillcolor{currentfill}%
\pgfsetlinewidth{0.481800pt}%
\definecolor{currentstroke}{rgb}{1.000000,1.000000,1.000000}%
\pgfsetstrokecolor{currentstroke}%
\pgfsetdash{}{0pt}%
\pgfpathmoveto{\pgfqpoint{2.658761in}{4.375525in}}%
\pgfpathcurveto{\pgfqpoint{2.669811in}{4.375525in}}{\pgfqpoint{2.680410in}{4.379916in}}{\pgfqpoint{2.688224in}{4.387729in}}%
\pgfpathcurveto{\pgfqpoint{2.696038in}{4.395543in}}{\pgfqpoint{2.700428in}{4.406142in}}{\pgfqpoint{2.700428in}{4.417192in}}%
\pgfpathcurveto{\pgfqpoint{2.700428in}{4.428242in}}{\pgfqpoint{2.696038in}{4.438841in}}{\pgfqpoint{2.688224in}{4.446655in}}%
\pgfpathcurveto{\pgfqpoint{2.680410in}{4.454468in}}{\pgfqpoint{2.669811in}{4.458859in}}{\pgfqpoint{2.658761in}{4.458859in}}%
\pgfpathcurveto{\pgfqpoint{2.647711in}{4.458859in}}{\pgfqpoint{2.637112in}{4.454468in}}{\pgfqpoint{2.629298in}{4.446655in}}%
\pgfpathcurveto{\pgfqpoint{2.621485in}{4.438841in}}{\pgfqpoint{2.617095in}{4.428242in}}{\pgfqpoint{2.617095in}{4.417192in}}%
\pgfpathcurveto{\pgfqpoint{2.617095in}{4.406142in}}{\pgfqpoint{2.621485in}{4.395543in}}{\pgfqpoint{2.629298in}{4.387729in}}%
\pgfpathcurveto{\pgfqpoint{2.637112in}{4.379916in}}{\pgfqpoint{2.647711in}{4.375525in}}{\pgfqpoint{2.658761in}{4.375525in}}%
\pgfpathclose%
\pgfusepath{stroke,fill}%
\end{pgfscope}%
\begin{pgfscope}%
\pgfpathrectangle{\pgfqpoint{0.526127in}{0.331635in}}{\pgfqpoint{9.300000in}{7.700000in}}%
\pgfusepath{clip}%
\pgfsetbuttcap%
\pgfsetroundjoin%
\definecolor{currentfill}{rgb}{0.870588,0.733333,0.607843}%
\pgfsetfillcolor{currentfill}%
\pgfsetlinewidth{0.481800pt}%
\definecolor{currentstroke}{rgb}{1.000000,1.000000,1.000000}%
\pgfsetstrokecolor{currentstroke}%
\pgfsetdash{}{0pt}%
\pgfpathmoveto{\pgfqpoint{3.293241in}{4.272945in}}%
\pgfpathcurveto{\pgfqpoint{3.304291in}{4.272945in}}{\pgfqpoint{3.314890in}{4.277336in}}{\pgfqpoint{3.322704in}{4.285149in}}%
\pgfpathcurveto{\pgfqpoint{3.330518in}{4.292963in}}{\pgfqpoint{3.334908in}{4.303562in}}{\pgfqpoint{3.334908in}{4.314612in}}%
\pgfpathcurveto{\pgfqpoint{3.334908in}{4.325662in}}{\pgfqpoint{3.330518in}{4.336261in}}{\pgfqpoint{3.322704in}{4.344075in}}%
\pgfpathcurveto{\pgfqpoint{3.314890in}{4.351889in}}{\pgfqpoint{3.304291in}{4.356279in}}{\pgfqpoint{3.293241in}{4.356279in}}%
\pgfpathcurveto{\pgfqpoint{3.282191in}{4.356279in}}{\pgfqpoint{3.271592in}{4.351889in}}{\pgfqpoint{3.263778in}{4.344075in}}%
\pgfpathcurveto{\pgfqpoint{3.255965in}{4.336261in}}{\pgfqpoint{3.251575in}{4.325662in}}{\pgfqpoint{3.251575in}{4.314612in}}%
\pgfpathcurveto{\pgfqpoint{3.251575in}{4.303562in}}{\pgfqpoint{3.255965in}{4.292963in}}{\pgfqpoint{3.263778in}{4.285149in}}%
\pgfpathcurveto{\pgfqpoint{3.271592in}{4.277336in}}{\pgfqpoint{3.282191in}{4.272945in}}{\pgfqpoint{3.293241in}{4.272945in}}%
\pgfpathclose%
\pgfusepath{stroke,fill}%
\end{pgfscope}%
\begin{pgfscope}%
\pgfpathrectangle{\pgfqpoint{0.526127in}{0.331635in}}{\pgfqpoint{9.300000in}{7.700000in}}%
\pgfusepath{clip}%
\pgfsetbuttcap%
\pgfsetroundjoin%
\definecolor{currentfill}{rgb}{0.870588,0.733333,0.607843}%
\pgfsetfillcolor{currentfill}%
\pgfsetlinewidth{0.481800pt}%
\definecolor{currentstroke}{rgb}{1.000000,1.000000,1.000000}%
\pgfsetstrokecolor{currentstroke}%
\pgfsetdash{}{0pt}%
\pgfpathmoveto{\pgfqpoint{4.449865in}{2.592672in}}%
\pgfpathcurveto{\pgfqpoint{4.460915in}{2.592672in}}{\pgfqpoint{4.471514in}{2.597063in}}{\pgfqpoint{4.479327in}{2.604876in}}%
\pgfpathcurveto{\pgfqpoint{4.487141in}{2.612690in}}{\pgfqpoint{4.491531in}{2.623289in}}{\pgfqpoint{4.491531in}{2.634339in}}%
\pgfpathcurveto{\pgfqpoint{4.491531in}{2.645389in}}{\pgfqpoint{4.487141in}{2.655988in}}{\pgfqpoint{4.479327in}{2.663802in}}%
\pgfpathcurveto{\pgfqpoint{4.471514in}{2.671615in}}{\pgfqpoint{4.460915in}{2.676006in}}{\pgfqpoint{4.449865in}{2.676006in}}%
\pgfpathcurveto{\pgfqpoint{4.438814in}{2.676006in}}{\pgfqpoint{4.428215in}{2.671615in}}{\pgfqpoint{4.420402in}{2.663802in}}%
\pgfpathcurveto{\pgfqpoint{4.412588in}{2.655988in}}{\pgfqpoint{4.408198in}{2.645389in}}{\pgfqpoint{4.408198in}{2.634339in}}%
\pgfpathcurveto{\pgfqpoint{4.408198in}{2.623289in}}{\pgfqpoint{4.412588in}{2.612690in}}{\pgfqpoint{4.420402in}{2.604876in}}%
\pgfpathcurveto{\pgfqpoint{4.428215in}{2.597063in}}{\pgfqpoint{4.438814in}{2.592672in}}{\pgfqpoint{4.449865in}{2.592672in}}%
\pgfpathclose%
\pgfusepath{stroke,fill}%
\end{pgfscope}%
\begin{pgfscope}%
\pgfpathrectangle{\pgfqpoint{0.526127in}{0.331635in}}{\pgfqpoint{9.300000in}{7.700000in}}%
\pgfusepath{clip}%
\pgfsetbuttcap%
\pgfsetroundjoin%
\definecolor{currentfill}{rgb}{0.870588,0.733333,0.607843}%
\pgfsetfillcolor{currentfill}%
\pgfsetlinewidth{0.481800pt}%
\definecolor{currentstroke}{rgb}{1.000000,1.000000,1.000000}%
\pgfsetstrokecolor{currentstroke}%
\pgfsetdash{}{0pt}%
\pgfpathmoveto{\pgfqpoint{5.505397in}{0.933762in}}%
\pgfpathcurveto{\pgfqpoint{5.516447in}{0.933762in}}{\pgfqpoint{5.527046in}{0.938152in}}{\pgfqpoint{5.534860in}{0.945966in}}%
\pgfpathcurveto{\pgfqpoint{5.542673in}{0.953780in}}{\pgfqpoint{5.547064in}{0.964379in}}{\pgfqpoint{5.547064in}{0.975429in}}%
\pgfpathcurveto{\pgfqpoint{5.547064in}{0.986479in}}{\pgfqpoint{5.542673in}{0.997078in}}{\pgfqpoint{5.534860in}{1.004892in}}%
\pgfpathcurveto{\pgfqpoint{5.527046in}{1.012705in}}{\pgfqpoint{5.516447in}{1.017096in}}{\pgfqpoint{5.505397in}{1.017096in}}%
\pgfpathcurveto{\pgfqpoint{5.494347in}{1.017096in}}{\pgfqpoint{5.483748in}{1.012705in}}{\pgfqpoint{5.475934in}{1.004892in}}%
\pgfpathcurveto{\pgfqpoint{5.468121in}{0.997078in}}{\pgfqpoint{5.463730in}{0.986479in}}{\pgfqpoint{5.463730in}{0.975429in}}%
\pgfpathcurveto{\pgfqpoint{5.463730in}{0.964379in}}{\pgfqpoint{5.468121in}{0.953780in}}{\pgfqpoint{5.475934in}{0.945966in}}%
\pgfpathcurveto{\pgfqpoint{5.483748in}{0.938152in}}{\pgfqpoint{5.494347in}{0.933762in}}{\pgfqpoint{5.505397in}{0.933762in}}%
\pgfpathclose%
\pgfusepath{stroke,fill}%
\end{pgfscope}%
\begin{pgfscope}%
\pgfpathrectangle{\pgfqpoint{0.526127in}{0.331635in}}{\pgfqpoint{9.300000in}{7.700000in}}%
\pgfusepath{clip}%
\pgfsetbuttcap%
\pgfsetroundjoin%
\definecolor{currentfill}{rgb}{0.870588,0.733333,0.607843}%
\pgfsetfillcolor{currentfill}%
\pgfsetlinewidth{0.481800pt}%
\definecolor{currentstroke}{rgb}{1.000000,1.000000,1.000000}%
\pgfsetstrokecolor{currentstroke}%
\pgfsetdash{}{0pt}%
\pgfpathmoveto{\pgfqpoint{3.975782in}{2.759819in}}%
\pgfpathcurveto{\pgfqpoint{3.986832in}{2.759819in}}{\pgfqpoint{3.997431in}{2.764209in}}{\pgfqpoint{4.005245in}{2.772023in}}%
\pgfpathcurveto{\pgfqpoint{4.013059in}{2.779837in}}{\pgfqpoint{4.017449in}{2.790436in}}{\pgfqpoint{4.017449in}{2.801486in}}%
\pgfpathcurveto{\pgfqpoint{4.017449in}{2.812536in}}{\pgfqpoint{4.013059in}{2.823135in}}{\pgfqpoint{4.005245in}{2.830949in}}%
\pgfpathcurveto{\pgfqpoint{3.997431in}{2.838762in}}{\pgfqpoint{3.986832in}{2.843153in}}{\pgfqpoint{3.975782in}{2.843153in}}%
\pgfpathcurveto{\pgfqpoint{3.964732in}{2.843153in}}{\pgfqpoint{3.954133in}{2.838762in}}{\pgfqpoint{3.946319in}{2.830949in}}%
\pgfpathcurveto{\pgfqpoint{3.938506in}{2.823135in}}{\pgfqpoint{3.934116in}{2.812536in}}{\pgfqpoint{3.934116in}{2.801486in}}%
\pgfpathcurveto{\pgfqpoint{3.934116in}{2.790436in}}{\pgfqpoint{3.938506in}{2.779837in}}{\pgfqpoint{3.946319in}{2.772023in}}%
\pgfpathcurveto{\pgfqpoint{3.954133in}{2.764209in}}{\pgfqpoint{3.964732in}{2.759819in}}{\pgfqpoint{3.975782in}{2.759819in}}%
\pgfpathclose%
\pgfusepath{stroke,fill}%
\end{pgfscope}%
\begin{pgfscope}%
\pgfpathrectangle{\pgfqpoint{0.526127in}{0.331635in}}{\pgfqpoint{9.300000in}{7.700000in}}%
\pgfusepath{clip}%
\pgfsetbuttcap%
\pgfsetroundjoin%
\definecolor{currentfill}{rgb}{0.870588,0.733333,0.607843}%
\pgfsetfillcolor{currentfill}%
\pgfsetlinewidth{0.481800pt}%
\definecolor{currentstroke}{rgb}{1.000000,1.000000,1.000000}%
\pgfsetstrokecolor{currentstroke}%
\pgfsetdash{}{0pt}%
\pgfpathmoveto{\pgfqpoint{5.150340in}{3.107207in}}%
\pgfpathcurveto{\pgfqpoint{5.161390in}{3.107207in}}{\pgfqpoint{5.171989in}{3.111597in}}{\pgfqpoint{5.179802in}{3.119411in}}%
\pgfpathcurveto{\pgfqpoint{5.187616in}{3.127224in}}{\pgfqpoint{5.192006in}{3.137823in}}{\pgfqpoint{5.192006in}{3.148874in}}%
\pgfpathcurveto{\pgfqpoint{5.192006in}{3.159924in}}{\pgfqpoint{5.187616in}{3.170523in}}{\pgfqpoint{5.179802in}{3.178336in}}%
\pgfpathcurveto{\pgfqpoint{5.171989in}{3.186150in}}{\pgfqpoint{5.161390in}{3.190540in}}{\pgfqpoint{5.150340in}{3.190540in}}%
\pgfpathcurveto{\pgfqpoint{5.139289in}{3.190540in}}{\pgfqpoint{5.128690in}{3.186150in}}{\pgfqpoint{5.120877in}{3.178336in}}%
\pgfpathcurveto{\pgfqpoint{5.113063in}{3.170523in}}{\pgfqpoint{5.108673in}{3.159924in}}{\pgfqpoint{5.108673in}{3.148874in}}%
\pgfpathcurveto{\pgfqpoint{5.108673in}{3.137823in}}{\pgfqpoint{5.113063in}{3.127224in}}{\pgfqpoint{5.120877in}{3.119411in}}%
\pgfpathcurveto{\pgfqpoint{5.128690in}{3.111597in}}{\pgfqpoint{5.139289in}{3.107207in}}{\pgfqpoint{5.150340in}{3.107207in}}%
\pgfpathclose%
\pgfusepath{stroke,fill}%
\end{pgfscope}%
\begin{pgfscope}%
\pgfpathrectangle{\pgfqpoint{0.526127in}{0.331635in}}{\pgfqpoint{9.300000in}{7.700000in}}%
\pgfusepath{clip}%
\pgfsetbuttcap%
\pgfsetroundjoin%
\definecolor{currentfill}{rgb}{0.870588,0.733333,0.607843}%
\pgfsetfillcolor{currentfill}%
\pgfsetlinewidth{0.481800pt}%
\definecolor{currentstroke}{rgb}{1.000000,1.000000,1.000000}%
\pgfsetstrokecolor{currentstroke}%
\pgfsetdash{}{0pt}%
\pgfpathmoveto{\pgfqpoint{6.303025in}{4.215839in}}%
\pgfpathcurveto{\pgfqpoint{6.314076in}{4.215839in}}{\pgfqpoint{6.324675in}{4.220230in}}{\pgfqpoint{6.332488in}{4.228043in}}%
\pgfpathcurveto{\pgfqpoint{6.340302in}{4.235857in}}{\pgfqpoint{6.344692in}{4.246456in}}{\pgfqpoint{6.344692in}{4.257506in}}%
\pgfpathcurveto{\pgfqpoint{6.344692in}{4.268556in}}{\pgfqpoint{6.340302in}{4.279155in}}{\pgfqpoint{6.332488in}{4.286969in}}%
\pgfpathcurveto{\pgfqpoint{6.324675in}{4.294782in}}{\pgfqpoint{6.314076in}{4.299173in}}{\pgfqpoint{6.303025in}{4.299173in}}%
\pgfpathcurveto{\pgfqpoint{6.291975in}{4.299173in}}{\pgfqpoint{6.281376in}{4.294782in}}{\pgfqpoint{6.273563in}{4.286969in}}%
\pgfpathcurveto{\pgfqpoint{6.265749in}{4.279155in}}{\pgfqpoint{6.261359in}{4.268556in}}{\pgfqpoint{6.261359in}{4.257506in}}%
\pgfpathcurveto{\pgfqpoint{6.261359in}{4.246456in}}{\pgfqpoint{6.265749in}{4.235857in}}{\pgfqpoint{6.273563in}{4.228043in}}%
\pgfpathcurveto{\pgfqpoint{6.281376in}{4.220230in}}{\pgfqpoint{6.291975in}{4.215839in}}{\pgfqpoint{6.303025in}{4.215839in}}%
\pgfpathclose%
\pgfusepath{stroke,fill}%
\end{pgfscope}%
\begin{pgfscope}%
\pgfpathrectangle{\pgfqpoint{0.526127in}{0.331635in}}{\pgfqpoint{9.300000in}{7.700000in}}%
\pgfusepath{clip}%
\pgfsetbuttcap%
\pgfsetroundjoin%
\definecolor{currentfill}{rgb}{0.870588,0.733333,0.607843}%
\pgfsetfillcolor{currentfill}%
\pgfsetlinewidth{0.481800pt}%
\definecolor{currentstroke}{rgb}{1.000000,1.000000,1.000000}%
\pgfsetstrokecolor{currentstroke}%
\pgfsetdash{}{0pt}%
\pgfpathmoveto{\pgfqpoint{3.101691in}{7.304554in}}%
\pgfpathcurveto{\pgfqpoint{3.112741in}{7.304554in}}{\pgfqpoint{3.123340in}{7.308944in}}{\pgfqpoint{3.131154in}{7.316758in}}%
\pgfpathcurveto{\pgfqpoint{3.138968in}{7.324571in}}{\pgfqpoint{3.143358in}{7.335170in}}{\pgfqpoint{3.143358in}{7.346220in}}%
\pgfpathcurveto{\pgfqpoint{3.143358in}{7.357270in}}{\pgfqpoint{3.138968in}{7.367869in}}{\pgfqpoint{3.131154in}{7.375683in}}%
\pgfpathcurveto{\pgfqpoint{3.123340in}{7.383497in}}{\pgfqpoint{3.112741in}{7.387887in}}{\pgfqpoint{3.101691in}{7.387887in}}%
\pgfpathcurveto{\pgfqpoint{3.090641in}{7.387887in}}{\pgfqpoint{3.080042in}{7.383497in}}{\pgfqpoint{3.072229in}{7.375683in}}%
\pgfpathcurveto{\pgfqpoint{3.064415in}{7.367869in}}{\pgfqpoint{3.060025in}{7.357270in}}{\pgfqpoint{3.060025in}{7.346220in}}%
\pgfpathcurveto{\pgfqpoint{3.060025in}{7.335170in}}{\pgfqpoint{3.064415in}{7.324571in}}{\pgfqpoint{3.072229in}{7.316758in}}%
\pgfpathcurveto{\pgfqpoint{3.080042in}{7.308944in}}{\pgfqpoint{3.090641in}{7.304554in}}{\pgfqpoint{3.101691in}{7.304554in}}%
\pgfpathclose%
\pgfusepath{stroke,fill}%
\end{pgfscope}%
\begin{pgfscope}%
\pgfpathrectangle{\pgfqpoint{0.526127in}{0.331635in}}{\pgfqpoint{9.300000in}{7.700000in}}%
\pgfusepath{clip}%
\pgfsetbuttcap%
\pgfsetroundjoin%
\definecolor{currentfill}{rgb}{0.870588,0.733333,0.607843}%
\pgfsetfillcolor{currentfill}%
\pgfsetlinewidth{0.481800pt}%
\definecolor{currentstroke}{rgb}{1.000000,1.000000,1.000000}%
\pgfsetstrokecolor{currentstroke}%
\pgfsetdash{}{0pt}%
\pgfpathmoveto{\pgfqpoint{6.025455in}{1.506152in}}%
\pgfpathcurveto{\pgfqpoint{6.036505in}{1.506152in}}{\pgfqpoint{6.047104in}{1.510542in}}{\pgfqpoint{6.054918in}{1.518356in}}%
\pgfpathcurveto{\pgfqpoint{6.062731in}{1.526170in}}{\pgfqpoint{6.067122in}{1.536769in}}{\pgfqpoint{6.067122in}{1.547819in}}%
\pgfpathcurveto{\pgfqpoint{6.067122in}{1.558869in}}{\pgfqpoint{6.062731in}{1.569468in}}{\pgfqpoint{6.054918in}{1.577282in}}%
\pgfpathcurveto{\pgfqpoint{6.047104in}{1.585095in}}{\pgfqpoint{6.036505in}{1.589485in}}{\pgfqpoint{6.025455in}{1.589485in}}%
\pgfpathcurveto{\pgfqpoint{6.014405in}{1.589485in}}{\pgfqpoint{6.003806in}{1.585095in}}{\pgfqpoint{5.995992in}{1.577282in}}%
\pgfpathcurveto{\pgfqpoint{5.988179in}{1.569468in}}{\pgfqpoint{5.983788in}{1.558869in}}{\pgfqpoint{5.983788in}{1.547819in}}%
\pgfpathcurveto{\pgfqpoint{5.983788in}{1.536769in}}{\pgfqpoint{5.988179in}{1.526170in}}{\pgfqpoint{5.995992in}{1.518356in}}%
\pgfpathcurveto{\pgfqpoint{6.003806in}{1.510542in}}{\pgfqpoint{6.014405in}{1.506152in}}{\pgfqpoint{6.025455in}{1.506152in}}%
\pgfpathclose%
\pgfusepath{stroke,fill}%
\end{pgfscope}%
\begin{pgfscope}%
\pgfpathrectangle{\pgfqpoint{0.526127in}{0.331635in}}{\pgfqpoint{9.300000in}{7.700000in}}%
\pgfusepath{clip}%
\pgfsetbuttcap%
\pgfsetroundjoin%
\definecolor{currentfill}{rgb}{0.870588,0.733333,0.607843}%
\pgfsetfillcolor{currentfill}%
\pgfsetlinewidth{0.481800pt}%
\definecolor{currentstroke}{rgb}{1.000000,1.000000,1.000000}%
\pgfsetstrokecolor{currentstroke}%
\pgfsetdash{}{0pt}%
\pgfpathmoveto{\pgfqpoint{2.005207in}{1.954811in}}%
\pgfpathcurveto{\pgfqpoint{2.016257in}{1.954811in}}{\pgfqpoint{2.026856in}{1.959201in}}{\pgfqpoint{2.034670in}{1.967015in}}%
\pgfpathcurveto{\pgfqpoint{2.042483in}{1.974828in}}{\pgfqpoint{2.046874in}{1.985427in}}{\pgfqpoint{2.046874in}{1.996477in}}%
\pgfpathcurveto{\pgfqpoint{2.046874in}{2.007528in}}{\pgfqpoint{2.042483in}{2.018127in}}{\pgfqpoint{2.034670in}{2.025940in}}%
\pgfpathcurveto{\pgfqpoint{2.026856in}{2.033754in}}{\pgfqpoint{2.016257in}{2.038144in}}{\pgfqpoint{2.005207in}{2.038144in}}%
\pgfpathcurveto{\pgfqpoint{1.994157in}{2.038144in}}{\pgfqpoint{1.983558in}{2.033754in}}{\pgfqpoint{1.975744in}{2.025940in}}%
\pgfpathcurveto{\pgfqpoint{1.967930in}{2.018127in}}{\pgfqpoint{1.963540in}{2.007528in}}{\pgfqpoint{1.963540in}{1.996477in}}%
\pgfpathcurveto{\pgfqpoint{1.963540in}{1.985427in}}{\pgfqpoint{1.967930in}{1.974828in}}{\pgfqpoint{1.975744in}{1.967015in}}%
\pgfpathcurveto{\pgfqpoint{1.983558in}{1.959201in}}{\pgfqpoint{1.994157in}{1.954811in}}{\pgfqpoint{2.005207in}{1.954811in}}%
\pgfpathclose%
\pgfusepath{stroke,fill}%
\end{pgfscope}%
\begin{pgfscope}%
\pgfpathrectangle{\pgfqpoint{0.526127in}{0.331635in}}{\pgfqpoint{9.300000in}{7.700000in}}%
\pgfusepath{clip}%
\pgfsetbuttcap%
\pgfsetroundjoin%
\definecolor{currentfill}{rgb}{0.870588,0.733333,0.607843}%
\pgfsetfillcolor{currentfill}%
\pgfsetlinewidth{0.481800pt}%
\definecolor{currentstroke}{rgb}{1.000000,1.000000,1.000000}%
\pgfsetstrokecolor{currentstroke}%
\pgfsetdash{}{0pt}%
\pgfpathmoveto{\pgfqpoint{0.948854in}{3.258509in}}%
\pgfpathcurveto{\pgfqpoint{0.959904in}{3.258509in}}{\pgfqpoint{0.970503in}{3.262899in}}{\pgfqpoint{0.978317in}{3.270713in}}%
\pgfpathcurveto{\pgfqpoint{0.986130in}{3.278527in}}{\pgfqpoint{0.990521in}{3.289126in}}{\pgfqpoint{0.990521in}{3.300176in}}%
\pgfpathcurveto{\pgfqpoint{0.990521in}{3.311226in}}{\pgfqpoint{0.986130in}{3.321825in}}{\pgfqpoint{0.978317in}{3.329639in}}%
\pgfpathcurveto{\pgfqpoint{0.970503in}{3.337452in}}{\pgfqpoint{0.959904in}{3.341842in}}{\pgfqpoint{0.948854in}{3.341842in}}%
\pgfpathcurveto{\pgfqpoint{0.937804in}{3.341842in}}{\pgfqpoint{0.927205in}{3.337452in}}{\pgfqpoint{0.919391in}{3.329639in}}%
\pgfpathcurveto{\pgfqpoint{0.911578in}{3.321825in}}{\pgfqpoint{0.907187in}{3.311226in}}{\pgfqpoint{0.907187in}{3.300176in}}%
\pgfpathcurveto{\pgfqpoint{0.907187in}{3.289126in}}{\pgfqpoint{0.911578in}{3.278527in}}{\pgfqpoint{0.919391in}{3.270713in}}%
\pgfpathcurveto{\pgfqpoint{0.927205in}{3.262899in}}{\pgfqpoint{0.937804in}{3.258509in}}{\pgfqpoint{0.948854in}{3.258509in}}%
\pgfpathclose%
\pgfusepath{stroke,fill}%
\end{pgfscope}%
\begin{pgfscope}%
\pgfpathrectangle{\pgfqpoint{0.526127in}{0.331635in}}{\pgfqpoint{9.300000in}{7.700000in}}%
\pgfusepath{clip}%
\pgfsetbuttcap%
\pgfsetroundjoin%
\definecolor{currentfill}{rgb}{0.870588,0.733333,0.607843}%
\pgfsetfillcolor{currentfill}%
\pgfsetlinewidth{0.481800pt}%
\definecolor{currentstroke}{rgb}{1.000000,1.000000,1.000000}%
\pgfsetstrokecolor{currentstroke}%
\pgfsetdash{}{0pt}%
\pgfpathmoveto{\pgfqpoint{2.231357in}{2.117700in}}%
\pgfpathcurveto{\pgfqpoint{2.242407in}{2.117700in}}{\pgfqpoint{2.253006in}{2.122090in}}{\pgfqpoint{2.260820in}{2.129904in}}%
\pgfpathcurveto{\pgfqpoint{2.268634in}{2.137718in}}{\pgfqpoint{2.273024in}{2.148317in}}{\pgfqpoint{2.273024in}{2.159367in}}%
\pgfpathcurveto{\pgfqpoint{2.273024in}{2.170417in}}{\pgfqpoint{2.268634in}{2.181016in}}{\pgfqpoint{2.260820in}{2.188830in}}%
\pgfpathcurveto{\pgfqpoint{2.253006in}{2.196643in}}{\pgfqpoint{2.242407in}{2.201034in}}{\pgfqpoint{2.231357in}{2.201034in}}%
\pgfpathcurveto{\pgfqpoint{2.220307in}{2.201034in}}{\pgfqpoint{2.209708in}{2.196643in}}{\pgfqpoint{2.201894in}{2.188830in}}%
\pgfpathcurveto{\pgfqpoint{2.194081in}{2.181016in}}{\pgfqpoint{2.189691in}{2.170417in}}{\pgfqpoint{2.189691in}{2.159367in}}%
\pgfpathcurveto{\pgfqpoint{2.189691in}{2.148317in}}{\pgfqpoint{2.194081in}{2.137718in}}{\pgfqpoint{2.201894in}{2.129904in}}%
\pgfpathcurveto{\pgfqpoint{2.209708in}{2.122090in}}{\pgfqpoint{2.220307in}{2.117700in}}{\pgfqpoint{2.231357in}{2.117700in}}%
\pgfpathclose%
\pgfusepath{stroke,fill}%
\end{pgfscope}%
\begin{pgfscope}%
\pgfpathrectangle{\pgfqpoint{0.526127in}{0.331635in}}{\pgfqpoint{9.300000in}{7.700000in}}%
\pgfusepath{clip}%
\pgfsetbuttcap%
\pgfsetroundjoin%
\definecolor{currentfill}{rgb}{0.870588,0.733333,0.607843}%
\pgfsetfillcolor{currentfill}%
\pgfsetlinewidth{0.481800pt}%
\definecolor{currentstroke}{rgb}{1.000000,1.000000,1.000000}%
\pgfsetstrokecolor{currentstroke}%
\pgfsetdash{}{0pt}%
\pgfpathmoveto{\pgfqpoint{5.273237in}{2.683641in}}%
\pgfpathcurveto{\pgfqpoint{5.284287in}{2.683641in}}{\pgfqpoint{5.294886in}{2.688031in}}{\pgfqpoint{5.302700in}{2.695845in}}%
\pgfpathcurveto{\pgfqpoint{5.310513in}{2.703658in}}{\pgfqpoint{5.314903in}{2.714257in}}{\pgfqpoint{5.314903in}{2.725307in}}%
\pgfpathcurveto{\pgfqpoint{5.314903in}{2.736358in}}{\pgfqpoint{5.310513in}{2.746957in}}{\pgfqpoint{5.302700in}{2.754770in}}%
\pgfpathcurveto{\pgfqpoint{5.294886in}{2.762584in}}{\pgfqpoint{5.284287in}{2.766974in}}{\pgfqpoint{5.273237in}{2.766974in}}%
\pgfpathcurveto{\pgfqpoint{5.262187in}{2.766974in}}{\pgfqpoint{5.251588in}{2.762584in}}{\pgfqpoint{5.243774in}{2.754770in}}%
\pgfpathcurveto{\pgfqpoint{5.235960in}{2.746957in}}{\pgfqpoint{5.231570in}{2.736358in}}{\pgfqpoint{5.231570in}{2.725307in}}%
\pgfpathcurveto{\pgfqpoint{5.231570in}{2.714257in}}{\pgfqpoint{5.235960in}{2.703658in}}{\pgfqpoint{5.243774in}{2.695845in}}%
\pgfpathcurveto{\pgfqpoint{5.251588in}{2.688031in}}{\pgfqpoint{5.262187in}{2.683641in}}{\pgfqpoint{5.273237in}{2.683641in}}%
\pgfpathclose%
\pgfusepath{stroke,fill}%
\end{pgfscope}%
\begin{pgfscope}%
\pgfpathrectangle{\pgfqpoint{0.526127in}{0.331635in}}{\pgfqpoint{9.300000in}{7.700000in}}%
\pgfusepath{clip}%
\pgfsetbuttcap%
\pgfsetroundjoin%
\definecolor{currentfill}{rgb}{0.870588,0.733333,0.607843}%
\pgfsetfillcolor{currentfill}%
\pgfsetlinewidth{0.481800pt}%
\definecolor{currentstroke}{rgb}{1.000000,1.000000,1.000000}%
\pgfsetstrokecolor{currentstroke}%
\pgfsetdash{}{0pt}%
\pgfpathmoveto{\pgfqpoint{4.721664in}{2.278825in}}%
\pgfpathcurveto{\pgfqpoint{4.732714in}{2.278825in}}{\pgfqpoint{4.743313in}{2.283215in}}{\pgfqpoint{4.751127in}{2.291028in}}%
\pgfpathcurveto{\pgfqpoint{4.758940in}{2.298842in}}{\pgfqpoint{4.763331in}{2.309441in}}{\pgfqpoint{4.763331in}{2.320491in}}%
\pgfpathcurveto{\pgfqpoint{4.763331in}{2.331541in}}{\pgfqpoint{4.758940in}{2.342140in}}{\pgfqpoint{4.751127in}{2.349954in}}%
\pgfpathcurveto{\pgfqpoint{4.743313in}{2.357768in}}{\pgfqpoint{4.732714in}{2.362158in}}{\pgfqpoint{4.721664in}{2.362158in}}%
\pgfpathcurveto{\pgfqpoint{4.710614in}{2.362158in}}{\pgfqpoint{4.700015in}{2.357768in}}{\pgfqpoint{4.692201in}{2.349954in}}%
\pgfpathcurveto{\pgfqpoint{4.684388in}{2.342140in}}{\pgfqpoint{4.679997in}{2.331541in}}{\pgfqpoint{4.679997in}{2.320491in}}%
\pgfpathcurveto{\pgfqpoint{4.679997in}{2.309441in}}{\pgfqpoint{4.684388in}{2.298842in}}{\pgfqpoint{4.692201in}{2.291028in}}%
\pgfpathcurveto{\pgfqpoint{4.700015in}{2.283215in}}{\pgfqpoint{4.710614in}{2.278825in}}{\pgfqpoint{4.721664in}{2.278825in}}%
\pgfpathclose%
\pgfusepath{stroke,fill}%
\end{pgfscope}%
\begin{pgfscope}%
\pgfpathrectangle{\pgfqpoint{0.526127in}{0.331635in}}{\pgfqpoint{9.300000in}{7.700000in}}%
\pgfusepath{clip}%
\pgfsetbuttcap%
\pgfsetroundjoin%
\definecolor{currentfill}{rgb}{0.870588,0.733333,0.607843}%
\pgfsetfillcolor{currentfill}%
\pgfsetlinewidth{0.481800pt}%
\definecolor{currentstroke}{rgb}{1.000000,1.000000,1.000000}%
\pgfsetstrokecolor{currentstroke}%
\pgfsetdash{}{0pt}%
\pgfpathmoveto{\pgfqpoint{5.294260in}{1.656223in}}%
\pgfpathcurveto{\pgfqpoint{5.305310in}{1.656223in}}{\pgfqpoint{5.315909in}{1.660613in}}{\pgfqpoint{5.323723in}{1.668427in}}%
\pgfpathcurveto{\pgfqpoint{5.331537in}{1.676240in}}{\pgfqpoint{5.335927in}{1.686839in}}{\pgfqpoint{5.335927in}{1.697890in}}%
\pgfpathcurveto{\pgfqpoint{5.335927in}{1.708940in}}{\pgfqpoint{5.331537in}{1.719539in}}{\pgfqpoint{5.323723in}{1.727352in}}%
\pgfpathcurveto{\pgfqpoint{5.315909in}{1.735166in}}{\pgfqpoint{5.305310in}{1.739556in}}{\pgfqpoint{5.294260in}{1.739556in}}%
\pgfpathcurveto{\pgfqpoint{5.283210in}{1.739556in}}{\pgfqpoint{5.272611in}{1.735166in}}{\pgfqpoint{5.264797in}{1.727352in}}%
\pgfpathcurveto{\pgfqpoint{5.256984in}{1.719539in}}{\pgfqpoint{5.252594in}{1.708940in}}{\pgfqpoint{5.252594in}{1.697890in}}%
\pgfpathcurveto{\pgfqpoint{5.252594in}{1.686839in}}{\pgfqpoint{5.256984in}{1.676240in}}{\pgfqpoint{5.264797in}{1.668427in}}%
\pgfpathcurveto{\pgfqpoint{5.272611in}{1.660613in}}{\pgfqpoint{5.283210in}{1.656223in}}{\pgfqpoint{5.294260in}{1.656223in}}%
\pgfpathclose%
\pgfusepath{stroke,fill}%
\end{pgfscope}%
\begin{pgfscope}%
\pgfpathrectangle{\pgfqpoint{0.526127in}{0.331635in}}{\pgfqpoint{9.300000in}{7.700000in}}%
\pgfusepath{clip}%
\pgfsetbuttcap%
\pgfsetroundjoin%
\definecolor{currentfill}{rgb}{0.870588,0.733333,0.607843}%
\pgfsetfillcolor{currentfill}%
\pgfsetlinewidth{0.481800pt}%
\definecolor{currentstroke}{rgb}{1.000000,1.000000,1.000000}%
\pgfsetstrokecolor{currentstroke}%
\pgfsetdash{}{0pt}%
\pgfpathmoveto{\pgfqpoint{3.016007in}{6.448638in}}%
\pgfpathcurveto{\pgfqpoint{3.027057in}{6.448638in}}{\pgfqpoint{3.037657in}{6.453028in}}{\pgfqpoint{3.045470in}{6.460842in}}%
\pgfpathcurveto{\pgfqpoint{3.053284in}{6.468655in}}{\pgfqpoint{3.057674in}{6.479254in}}{\pgfqpoint{3.057674in}{6.490305in}}%
\pgfpathcurveto{\pgfqpoint{3.057674in}{6.501355in}}{\pgfqpoint{3.053284in}{6.511954in}}{\pgfqpoint{3.045470in}{6.519767in}}%
\pgfpathcurveto{\pgfqpoint{3.037657in}{6.527581in}}{\pgfqpoint{3.027057in}{6.531971in}}{\pgfqpoint{3.016007in}{6.531971in}}%
\pgfpathcurveto{\pgfqpoint{3.004957in}{6.531971in}}{\pgfqpoint{2.994358in}{6.527581in}}{\pgfqpoint{2.986545in}{6.519767in}}%
\pgfpathcurveto{\pgfqpoint{2.978731in}{6.511954in}}{\pgfqpoint{2.974341in}{6.501355in}}{\pgfqpoint{2.974341in}{6.490305in}}%
\pgfpathcurveto{\pgfqpoint{2.974341in}{6.479254in}}{\pgfqpoint{2.978731in}{6.468655in}}{\pgfqpoint{2.986545in}{6.460842in}}%
\pgfpathcurveto{\pgfqpoint{2.994358in}{6.453028in}}{\pgfqpoint{3.004957in}{6.448638in}}{\pgfqpoint{3.016007in}{6.448638in}}%
\pgfpathclose%
\pgfusepath{stroke,fill}%
\end{pgfscope}%
\begin{pgfscope}%
\pgfpathrectangle{\pgfqpoint{0.526127in}{0.331635in}}{\pgfqpoint{9.300000in}{7.700000in}}%
\pgfusepath{clip}%
\pgfsetbuttcap%
\pgfsetroundjoin%
\definecolor{currentfill}{rgb}{0.870588,0.733333,0.607843}%
\pgfsetfillcolor{currentfill}%
\pgfsetlinewidth{0.481800pt}%
\definecolor{currentstroke}{rgb}{1.000000,1.000000,1.000000}%
\pgfsetstrokecolor{currentstroke}%
\pgfsetdash{}{0pt}%
\pgfpathmoveto{\pgfqpoint{0.975724in}{3.271591in}}%
\pgfpathcurveto{\pgfqpoint{0.986774in}{3.271591in}}{\pgfqpoint{0.997373in}{3.275982in}}{\pgfqpoint{1.005187in}{3.283795in}}%
\pgfpathcurveto{\pgfqpoint{1.013000in}{3.291609in}}{\pgfqpoint{1.017391in}{3.302208in}}{\pgfqpoint{1.017391in}{3.313258in}}%
\pgfpathcurveto{\pgfqpoint{1.017391in}{3.324308in}}{\pgfqpoint{1.013000in}{3.334907in}}{\pgfqpoint{1.005187in}{3.342721in}}%
\pgfpathcurveto{\pgfqpoint{0.997373in}{3.350534in}}{\pgfqpoint{0.986774in}{3.354925in}}{\pgfqpoint{0.975724in}{3.354925in}}%
\pgfpathcurveto{\pgfqpoint{0.964674in}{3.354925in}}{\pgfqpoint{0.954075in}{3.350534in}}{\pgfqpoint{0.946261in}{3.342721in}}%
\pgfpathcurveto{\pgfqpoint{0.938448in}{3.334907in}}{\pgfqpoint{0.934057in}{3.324308in}}{\pgfqpoint{0.934057in}{3.313258in}}%
\pgfpathcurveto{\pgfqpoint{0.934057in}{3.302208in}}{\pgfqpoint{0.938448in}{3.291609in}}{\pgfqpoint{0.946261in}{3.283795in}}%
\pgfpathcurveto{\pgfqpoint{0.954075in}{3.275982in}}{\pgfqpoint{0.964674in}{3.271591in}}{\pgfqpoint{0.975724in}{3.271591in}}%
\pgfpathclose%
\pgfusepath{stroke,fill}%
\end{pgfscope}%
\begin{pgfscope}%
\pgfpathrectangle{\pgfqpoint{0.526127in}{0.331635in}}{\pgfqpoint{9.300000in}{7.700000in}}%
\pgfusepath{clip}%
\pgfsetbuttcap%
\pgfsetroundjoin%
\definecolor{currentfill}{rgb}{0.870588,0.733333,0.607843}%
\pgfsetfillcolor{currentfill}%
\pgfsetlinewidth{0.481800pt}%
\definecolor{currentstroke}{rgb}{1.000000,1.000000,1.000000}%
\pgfsetstrokecolor{currentstroke}%
\pgfsetdash{}{0pt}%
\pgfpathmoveto{\pgfqpoint{6.443064in}{3.164056in}}%
\pgfpathcurveto{\pgfqpoint{6.454114in}{3.164056in}}{\pgfqpoint{6.464713in}{3.168446in}}{\pgfqpoint{6.472527in}{3.176260in}}%
\pgfpathcurveto{\pgfqpoint{6.480341in}{3.184073in}}{\pgfqpoint{6.484731in}{3.194672in}}{\pgfqpoint{6.484731in}{3.205723in}}%
\pgfpathcurveto{\pgfqpoint{6.484731in}{3.216773in}}{\pgfqpoint{6.480341in}{3.227372in}}{\pgfqpoint{6.472527in}{3.235185in}}%
\pgfpathcurveto{\pgfqpoint{6.464713in}{3.242999in}}{\pgfqpoint{6.454114in}{3.247389in}}{\pgfqpoint{6.443064in}{3.247389in}}%
\pgfpathcurveto{\pgfqpoint{6.432014in}{3.247389in}}{\pgfqpoint{6.421415in}{3.242999in}}{\pgfqpoint{6.413601in}{3.235185in}}%
\pgfpathcurveto{\pgfqpoint{6.405788in}{3.227372in}}{\pgfqpoint{6.401397in}{3.216773in}}{\pgfqpoint{6.401397in}{3.205723in}}%
\pgfpathcurveto{\pgfqpoint{6.401397in}{3.194672in}}{\pgfqpoint{6.405788in}{3.184073in}}{\pgfqpoint{6.413601in}{3.176260in}}%
\pgfpathcurveto{\pgfqpoint{6.421415in}{3.168446in}}{\pgfqpoint{6.432014in}{3.164056in}}{\pgfqpoint{6.443064in}{3.164056in}}%
\pgfpathclose%
\pgfusepath{stroke,fill}%
\end{pgfscope}%
\begin{pgfscope}%
\pgfpathrectangle{\pgfqpoint{0.526127in}{0.331635in}}{\pgfqpoint{9.300000in}{7.700000in}}%
\pgfusepath{clip}%
\pgfsetbuttcap%
\pgfsetroundjoin%
\definecolor{currentfill}{rgb}{0.870588,0.733333,0.607843}%
\pgfsetfillcolor{currentfill}%
\pgfsetlinewidth{0.481800pt}%
\definecolor{currentstroke}{rgb}{1.000000,1.000000,1.000000}%
\pgfsetstrokecolor{currentstroke}%
\pgfsetdash{}{0pt}%
\pgfpathmoveto{\pgfqpoint{5.818974in}{2.915205in}}%
\pgfpathcurveto{\pgfqpoint{5.830024in}{2.915205in}}{\pgfqpoint{5.840623in}{2.919595in}}{\pgfqpoint{5.848437in}{2.927408in}}%
\pgfpathcurveto{\pgfqpoint{5.856250in}{2.935222in}}{\pgfqpoint{5.860641in}{2.945821in}}{\pgfqpoint{5.860641in}{2.956871in}}%
\pgfpathcurveto{\pgfqpoint{5.860641in}{2.967921in}}{\pgfqpoint{5.856250in}{2.978520in}}{\pgfqpoint{5.848437in}{2.986334in}}%
\pgfpathcurveto{\pgfqpoint{5.840623in}{2.994148in}}{\pgfqpoint{5.830024in}{2.998538in}}{\pgfqpoint{5.818974in}{2.998538in}}%
\pgfpathcurveto{\pgfqpoint{5.807924in}{2.998538in}}{\pgfqpoint{5.797325in}{2.994148in}}{\pgfqpoint{5.789511in}{2.986334in}}%
\pgfpathcurveto{\pgfqpoint{5.781698in}{2.978520in}}{\pgfqpoint{5.777307in}{2.967921in}}{\pgfqpoint{5.777307in}{2.956871in}}%
\pgfpathcurveto{\pgfqpoint{5.777307in}{2.945821in}}{\pgfqpoint{5.781698in}{2.935222in}}{\pgfqpoint{5.789511in}{2.927408in}}%
\pgfpathcurveto{\pgfqpoint{5.797325in}{2.919595in}}{\pgfqpoint{5.807924in}{2.915205in}}{\pgfqpoint{5.818974in}{2.915205in}}%
\pgfpathclose%
\pgfusepath{stroke,fill}%
\end{pgfscope}%
\begin{pgfscope}%
\pgfpathrectangle{\pgfqpoint{0.526127in}{0.331635in}}{\pgfqpoint{9.300000in}{7.700000in}}%
\pgfusepath{clip}%
\pgfsetbuttcap%
\pgfsetroundjoin%
\definecolor{currentfill}{rgb}{0.870588,0.733333,0.607843}%
\pgfsetfillcolor{currentfill}%
\pgfsetlinewidth{0.481800pt}%
\definecolor{currentstroke}{rgb}{1.000000,1.000000,1.000000}%
\pgfsetstrokecolor{currentstroke}%
\pgfsetdash{}{0pt}%
\pgfpathmoveto{\pgfqpoint{3.186993in}{2.546735in}}%
\pgfpathcurveto{\pgfqpoint{3.198043in}{2.546735in}}{\pgfqpoint{3.208642in}{2.551125in}}{\pgfqpoint{3.216456in}{2.558938in}}%
\pgfpathcurveto{\pgfqpoint{3.224269in}{2.566752in}}{\pgfqpoint{3.228659in}{2.577351in}}{\pgfqpoint{3.228659in}{2.588401in}}%
\pgfpathcurveto{\pgfqpoint{3.228659in}{2.599451in}}{\pgfqpoint{3.224269in}{2.610050in}}{\pgfqpoint{3.216456in}{2.617864in}}%
\pgfpathcurveto{\pgfqpoint{3.208642in}{2.625678in}}{\pgfqpoint{3.198043in}{2.630068in}}{\pgfqpoint{3.186993in}{2.630068in}}%
\pgfpathcurveto{\pgfqpoint{3.175943in}{2.630068in}}{\pgfqpoint{3.165344in}{2.625678in}}{\pgfqpoint{3.157530in}{2.617864in}}%
\pgfpathcurveto{\pgfqpoint{3.149716in}{2.610050in}}{\pgfqpoint{3.145326in}{2.599451in}}{\pgfqpoint{3.145326in}{2.588401in}}%
\pgfpathcurveto{\pgfqpoint{3.145326in}{2.577351in}}{\pgfqpoint{3.149716in}{2.566752in}}{\pgfqpoint{3.157530in}{2.558938in}}%
\pgfpathcurveto{\pgfqpoint{3.165344in}{2.551125in}}{\pgfqpoint{3.175943in}{2.546735in}}{\pgfqpoint{3.186993in}{2.546735in}}%
\pgfpathclose%
\pgfusepath{stroke,fill}%
\end{pgfscope}%
\begin{pgfscope}%
\pgfpathrectangle{\pgfqpoint{0.526127in}{0.331635in}}{\pgfqpoint{9.300000in}{7.700000in}}%
\pgfusepath{clip}%
\pgfsetbuttcap%
\pgfsetroundjoin%
\definecolor{currentfill}{rgb}{0.870588,0.733333,0.607843}%
\pgfsetfillcolor{currentfill}%
\pgfsetlinewidth{0.481800pt}%
\definecolor{currentstroke}{rgb}{1.000000,1.000000,1.000000}%
\pgfsetstrokecolor{currentstroke}%
\pgfsetdash{}{0pt}%
\pgfpathmoveto{\pgfqpoint{5.009171in}{2.382315in}}%
\pgfpathcurveto{\pgfqpoint{5.020222in}{2.382315in}}{\pgfqpoint{5.030821in}{2.386706in}}{\pgfqpoint{5.038634in}{2.394519in}}%
\pgfpathcurveto{\pgfqpoint{5.046448in}{2.402333in}}{\pgfqpoint{5.050838in}{2.412932in}}{\pgfqpoint{5.050838in}{2.423982in}}%
\pgfpathcurveto{\pgfqpoint{5.050838in}{2.435032in}}{\pgfqpoint{5.046448in}{2.445631in}}{\pgfqpoint{5.038634in}{2.453445in}}%
\pgfpathcurveto{\pgfqpoint{5.030821in}{2.461259in}}{\pgfqpoint{5.020222in}{2.465649in}}{\pgfqpoint{5.009171in}{2.465649in}}%
\pgfpathcurveto{\pgfqpoint{4.998121in}{2.465649in}}{\pgfqpoint{4.987522in}{2.461259in}}{\pgfqpoint{4.979709in}{2.453445in}}%
\pgfpathcurveto{\pgfqpoint{4.971895in}{2.445631in}}{\pgfqpoint{4.967505in}{2.435032in}}{\pgfqpoint{4.967505in}{2.423982in}}%
\pgfpathcurveto{\pgfqpoint{4.967505in}{2.412932in}}{\pgfqpoint{4.971895in}{2.402333in}}{\pgfqpoint{4.979709in}{2.394519in}}%
\pgfpathcurveto{\pgfqpoint{4.987522in}{2.386706in}}{\pgfqpoint{4.998121in}{2.382315in}}{\pgfqpoint{5.009171in}{2.382315in}}%
\pgfpathclose%
\pgfusepath{stroke,fill}%
\end{pgfscope}%
\begin{pgfscope}%
\pgfpathrectangle{\pgfqpoint{0.526127in}{0.331635in}}{\pgfqpoint{9.300000in}{7.700000in}}%
\pgfusepath{clip}%
\pgfsetbuttcap%
\pgfsetroundjoin%
\definecolor{currentfill}{rgb}{0.870588,0.733333,0.607843}%
\pgfsetfillcolor{currentfill}%
\pgfsetlinewidth{0.481800pt}%
\definecolor{currentstroke}{rgb}{1.000000,1.000000,1.000000}%
\pgfsetstrokecolor{currentstroke}%
\pgfsetdash{}{0pt}%
\pgfpathmoveto{\pgfqpoint{5.648571in}{1.105803in}}%
\pgfpathcurveto{\pgfqpoint{5.659622in}{1.105803in}}{\pgfqpoint{5.670221in}{1.110194in}}{\pgfqpoint{5.678034in}{1.118007in}}%
\pgfpathcurveto{\pgfqpoint{5.685848in}{1.125821in}}{\pgfqpoint{5.690238in}{1.136420in}}{\pgfqpoint{5.690238in}{1.147470in}}%
\pgfpathcurveto{\pgfqpoint{5.690238in}{1.158520in}}{\pgfqpoint{5.685848in}{1.169119in}}{\pgfqpoint{5.678034in}{1.176933in}}%
\pgfpathcurveto{\pgfqpoint{5.670221in}{1.184747in}}{\pgfqpoint{5.659622in}{1.189137in}}{\pgfqpoint{5.648571in}{1.189137in}}%
\pgfpathcurveto{\pgfqpoint{5.637521in}{1.189137in}}{\pgfqpoint{5.626922in}{1.184747in}}{\pgfqpoint{5.619109in}{1.176933in}}%
\pgfpathcurveto{\pgfqpoint{5.611295in}{1.169119in}}{\pgfqpoint{5.606905in}{1.158520in}}{\pgfqpoint{5.606905in}{1.147470in}}%
\pgfpathcurveto{\pgfqpoint{5.606905in}{1.136420in}}{\pgfqpoint{5.611295in}{1.125821in}}{\pgfqpoint{5.619109in}{1.118007in}}%
\pgfpathcurveto{\pgfqpoint{5.626922in}{1.110194in}}{\pgfqpoint{5.637521in}{1.105803in}}{\pgfqpoint{5.648571in}{1.105803in}}%
\pgfpathclose%
\pgfusepath{stroke,fill}%
\end{pgfscope}%
\begin{pgfscope}%
\pgfpathrectangle{\pgfqpoint{0.526127in}{0.331635in}}{\pgfqpoint{9.300000in}{7.700000in}}%
\pgfusepath{clip}%
\pgfsetbuttcap%
\pgfsetroundjoin%
\definecolor{currentfill}{rgb}{0.870588,0.733333,0.607843}%
\pgfsetfillcolor{currentfill}%
\pgfsetlinewidth{0.481800pt}%
\definecolor{currentstroke}{rgb}{1.000000,1.000000,1.000000}%
\pgfsetstrokecolor{currentstroke}%
\pgfsetdash{}{0pt}%
\pgfpathmoveto{\pgfqpoint{4.947462in}{1.651842in}}%
\pgfpathcurveto{\pgfqpoint{4.958512in}{1.651842in}}{\pgfqpoint{4.969111in}{1.656233in}}{\pgfqpoint{4.976925in}{1.664046in}}%
\pgfpathcurveto{\pgfqpoint{4.984739in}{1.671860in}}{\pgfqpoint{4.989129in}{1.682459in}}{\pgfqpoint{4.989129in}{1.693509in}}%
\pgfpathcurveto{\pgfqpoint{4.989129in}{1.704559in}}{\pgfqpoint{4.984739in}{1.715158in}}{\pgfqpoint{4.976925in}{1.722972in}}%
\pgfpathcurveto{\pgfqpoint{4.969111in}{1.730785in}}{\pgfqpoint{4.958512in}{1.735176in}}{\pgfqpoint{4.947462in}{1.735176in}}%
\pgfpathcurveto{\pgfqpoint{4.936412in}{1.735176in}}{\pgfqpoint{4.925813in}{1.730785in}}{\pgfqpoint{4.917999in}{1.722972in}}%
\pgfpathcurveto{\pgfqpoint{4.910186in}{1.715158in}}{\pgfqpoint{4.905795in}{1.704559in}}{\pgfqpoint{4.905795in}{1.693509in}}%
\pgfpathcurveto{\pgfqpoint{4.905795in}{1.682459in}}{\pgfqpoint{4.910186in}{1.671860in}}{\pgfqpoint{4.917999in}{1.664046in}}%
\pgfpathcurveto{\pgfqpoint{4.925813in}{1.656233in}}{\pgfqpoint{4.936412in}{1.651842in}}{\pgfqpoint{4.947462in}{1.651842in}}%
\pgfpathclose%
\pgfusepath{stroke,fill}%
\end{pgfscope}%
\begin{pgfscope}%
\pgfpathrectangle{\pgfqpoint{0.526127in}{0.331635in}}{\pgfqpoint{9.300000in}{7.700000in}}%
\pgfusepath{clip}%
\pgfsetbuttcap%
\pgfsetroundjoin%
\definecolor{currentfill}{rgb}{0.870588,0.733333,0.607843}%
\pgfsetfillcolor{currentfill}%
\pgfsetlinewidth{0.481800pt}%
\definecolor{currentstroke}{rgb}{1.000000,1.000000,1.000000}%
\pgfsetstrokecolor{currentstroke}%
\pgfsetdash{}{0pt}%
\pgfpathmoveto{\pgfqpoint{5.178715in}{1.632599in}}%
\pgfpathcurveto{\pgfqpoint{5.189765in}{1.632599in}}{\pgfqpoint{5.200364in}{1.636990in}}{\pgfqpoint{5.208177in}{1.644803in}}%
\pgfpathcurveto{\pgfqpoint{5.215991in}{1.652617in}}{\pgfqpoint{5.220381in}{1.663216in}}{\pgfqpoint{5.220381in}{1.674266in}}%
\pgfpathcurveto{\pgfqpoint{5.220381in}{1.685316in}}{\pgfqpoint{5.215991in}{1.695915in}}{\pgfqpoint{5.208177in}{1.703729in}}%
\pgfpathcurveto{\pgfqpoint{5.200364in}{1.711543in}}{\pgfqpoint{5.189765in}{1.715933in}}{\pgfqpoint{5.178715in}{1.715933in}}%
\pgfpathcurveto{\pgfqpoint{5.167664in}{1.715933in}}{\pgfqpoint{5.157065in}{1.711543in}}{\pgfqpoint{5.149252in}{1.703729in}}%
\pgfpathcurveto{\pgfqpoint{5.141438in}{1.695915in}}{\pgfqpoint{5.137048in}{1.685316in}}{\pgfqpoint{5.137048in}{1.674266in}}%
\pgfpathcurveto{\pgfqpoint{5.137048in}{1.663216in}}{\pgfqpoint{5.141438in}{1.652617in}}{\pgfqpoint{5.149252in}{1.644803in}}%
\pgfpathcurveto{\pgfqpoint{5.157065in}{1.636990in}}{\pgfqpoint{5.167664in}{1.632599in}}{\pgfqpoint{5.178715in}{1.632599in}}%
\pgfpathclose%
\pgfusepath{stroke,fill}%
\end{pgfscope}%
\begin{pgfscope}%
\pgfpathrectangle{\pgfqpoint{0.526127in}{0.331635in}}{\pgfqpoint{9.300000in}{7.700000in}}%
\pgfusepath{clip}%
\pgfsetbuttcap%
\pgfsetroundjoin%
\definecolor{currentfill}{rgb}{0.870588,0.733333,0.607843}%
\pgfsetfillcolor{currentfill}%
\pgfsetlinewidth{0.481800pt}%
\definecolor{currentstroke}{rgb}{1.000000,1.000000,1.000000}%
\pgfsetstrokecolor{currentstroke}%
\pgfsetdash{}{0pt}%
\pgfpathmoveto{\pgfqpoint{6.316003in}{1.896725in}}%
\pgfpathcurveto{\pgfqpoint{6.327054in}{1.896725in}}{\pgfqpoint{6.337653in}{1.901115in}}{\pgfqpoint{6.345466in}{1.908929in}}%
\pgfpathcurveto{\pgfqpoint{6.353280in}{1.916742in}}{\pgfqpoint{6.357670in}{1.927341in}}{\pgfqpoint{6.357670in}{1.938392in}}%
\pgfpathcurveto{\pgfqpoint{6.357670in}{1.949442in}}{\pgfqpoint{6.353280in}{1.960041in}}{\pgfqpoint{6.345466in}{1.967854in}}%
\pgfpathcurveto{\pgfqpoint{6.337653in}{1.975668in}}{\pgfqpoint{6.327054in}{1.980058in}}{\pgfqpoint{6.316003in}{1.980058in}}%
\pgfpathcurveto{\pgfqpoint{6.304953in}{1.980058in}}{\pgfqpoint{6.294354in}{1.975668in}}{\pgfqpoint{6.286541in}{1.967854in}}%
\pgfpathcurveto{\pgfqpoint{6.278727in}{1.960041in}}{\pgfqpoint{6.274337in}{1.949442in}}{\pgfqpoint{6.274337in}{1.938392in}}%
\pgfpathcurveto{\pgfqpoint{6.274337in}{1.927341in}}{\pgfqpoint{6.278727in}{1.916742in}}{\pgfqpoint{6.286541in}{1.908929in}}%
\pgfpathcurveto{\pgfqpoint{6.294354in}{1.901115in}}{\pgfqpoint{6.304953in}{1.896725in}}{\pgfqpoint{6.316003in}{1.896725in}}%
\pgfpathclose%
\pgfusepath{stroke,fill}%
\end{pgfscope}%
\begin{pgfscope}%
\pgfpathrectangle{\pgfqpoint{0.526127in}{0.331635in}}{\pgfqpoint{9.300000in}{7.700000in}}%
\pgfusepath{clip}%
\pgfsetbuttcap%
\pgfsetroundjoin%
\definecolor{currentfill}{rgb}{0.870588,0.733333,0.607843}%
\pgfsetfillcolor{currentfill}%
\pgfsetlinewidth{0.481800pt}%
\definecolor{currentstroke}{rgb}{1.000000,1.000000,1.000000}%
\pgfsetstrokecolor{currentstroke}%
\pgfsetdash{}{0pt}%
\pgfpathmoveto{\pgfqpoint{4.799485in}{2.332859in}}%
\pgfpathcurveto{\pgfqpoint{4.810535in}{2.332859in}}{\pgfqpoint{4.821134in}{2.337250in}}{\pgfqpoint{4.828947in}{2.345063in}}%
\pgfpathcurveto{\pgfqpoint{4.836761in}{2.352877in}}{\pgfqpoint{4.841151in}{2.363476in}}{\pgfqpoint{4.841151in}{2.374526in}}%
\pgfpathcurveto{\pgfqpoint{4.841151in}{2.385576in}}{\pgfqpoint{4.836761in}{2.396175in}}{\pgfqpoint{4.828947in}{2.403989in}}%
\pgfpathcurveto{\pgfqpoint{4.821134in}{2.411802in}}{\pgfqpoint{4.810535in}{2.416193in}}{\pgfqpoint{4.799485in}{2.416193in}}%
\pgfpathcurveto{\pgfqpoint{4.788434in}{2.416193in}}{\pgfqpoint{4.777835in}{2.411802in}}{\pgfqpoint{4.770022in}{2.403989in}}%
\pgfpathcurveto{\pgfqpoint{4.762208in}{2.396175in}}{\pgfqpoint{4.757818in}{2.385576in}}{\pgfqpoint{4.757818in}{2.374526in}}%
\pgfpathcurveto{\pgfqpoint{4.757818in}{2.363476in}}{\pgfqpoint{4.762208in}{2.352877in}}{\pgfqpoint{4.770022in}{2.345063in}}%
\pgfpathcurveto{\pgfqpoint{4.777835in}{2.337250in}}{\pgfqpoint{4.788434in}{2.332859in}}{\pgfqpoint{4.799485in}{2.332859in}}%
\pgfpathclose%
\pgfusepath{stroke,fill}%
\end{pgfscope}%
\begin{pgfscope}%
\pgfpathrectangle{\pgfqpoint{0.526127in}{0.331635in}}{\pgfqpoint{9.300000in}{7.700000in}}%
\pgfusepath{clip}%
\pgfsetbuttcap%
\pgfsetroundjoin%
\definecolor{currentfill}{rgb}{0.870588,0.733333,0.607843}%
\pgfsetfillcolor{currentfill}%
\pgfsetlinewidth{0.481800pt}%
\definecolor{currentstroke}{rgb}{1.000000,1.000000,1.000000}%
\pgfsetstrokecolor{currentstroke}%
\pgfsetdash{}{0pt}%
\pgfpathmoveto{\pgfqpoint{3.925973in}{2.826021in}}%
\pgfpathcurveto{\pgfqpoint{3.937023in}{2.826021in}}{\pgfqpoint{3.947622in}{2.830411in}}{\pgfqpoint{3.955436in}{2.838225in}}%
\pgfpathcurveto{\pgfqpoint{3.963249in}{2.846038in}}{\pgfqpoint{3.967639in}{2.856637in}}{\pgfqpoint{3.967639in}{2.867687in}}%
\pgfpathcurveto{\pgfqpoint{3.967639in}{2.878738in}}{\pgfqpoint{3.963249in}{2.889337in}}{\pgfqpoint{3.955436in}{2.897150in}}%
\pgfpathcurveto{\pgfqpoint{3.947622in}{2.904964in}}{\pgfqpoint{3.937023in}{2.909354in}}{\pgfqpoint{3.925973in}{2.909354in}}%
\pgfpathcurveto{\pgfqpoint{3.914923in}{2.909354in}}{\pgfqpoint{3.904324in}{2.904964in}}{\pgfqpoint{3.896510in}{2.897150in}}%
\pgfpathcurveto{\pgfqpoint{3.888696in}{2.889337in}}{\pgfqpoint{3.884306in}{2.878738in}}{\pgfqpoint{3.884306in}{2.867687in}}%
\pgfpathcurveto{\pgfqpoint{3.884306in}{2.856637in}}{\pgfqpoint{3.888696in}{2.846038in}}{\pgfqpoint{3.896510in}{2.838225in}}%
\pgfpathcurveto{\pgfqpoint{3.904324in}{2.830411in}}{\pgfqpoint{3.914923in}{2.826021in}}{\pgfqpoint{3.925973in}{2.826021in}}%
\pgfpathclose%
\pgfusepath{stroke,fill}%
\end{pgfscope}%
\begin{pgfscope}%
\pgfpathrectangle{\pgfqpoint{0.526127in}{0.331635in}}{\pgfqpoint{9.300000in}{7.700000in}}%
\pgfusepath{clip}%
\pgfsetbuttcap%
\pgfsetroundjoin%
\definecolor{currentfill}{rgb}{0.870588,0.733333,0.607843}%
\pgfsetfillcolor{currentfill}%
\pgfsetlinewidth{0.481800pt}%
\definecolor{currentstroke}{rgb}{1.000000,1.000000,1.000000}%
\pgfsetstrokecolor{currentstroke}%
\pgfsetdash{}{0pt}%
\pgfpathmoveto{\pgfqpoint{2.545411in}{1.753691in}}%
\pgfpathcurveto{\pgfqpoint{2.556462in}{1.753691in}}{\pgfqpoint{2.567061in}{1.758081in}}{\pgfqpoint{2.574874in}{1.765895in}}%
\pgfpathcurveto{\pgfqpoint{2.582688in}{1.773708in}}{\pgfqpoint{2.587078in}{1.784307in}}{\pgfqpoint{2.587078in}{1.795358in}}%
\pgfpathcurveto{\pgfqpoint{2.587078in}{1.806408in}}{\pgfqpoint{2.582688in}{1.817007in}}{\pgfqpoint{2.574874in}{1.824820in}}%
\pgfpathcurveto{\pgfqpoint{2.567061in}{1.832634in}}{\pgfqpoint{2.556462in}{1.837024in}}{\pgfqpoint{2.545411in}{1.837024in}}%
\pgfpathcurveto{\pgfqpoint{2.534361in}{1.837024in}}{\pgfqpoint{2.523762in}{1.832634in}}{\pgfqpoint{2.515949in}{1.824820in}}%
\pgfpathcurveto{\pgfqpoint{2.508135in}{1.817007in}}{\pgfqpoint{2.503745in}{1.806408in}}{\pgfqpoint{2.503745in}{1.795358in}}%
\pgfpathcurveto{\pgfqpoint{2.503745in}{1.784307in}}{\pgfqpoint{2.508135in}{1.773708in}}{\pgfqpoint{2.515949in}{1.765895in}}%
\pgfpathcurveto{\pgfqpoint{2.523762in}{1.758081in}}{\pgfqpoint{2.534361in}{1.753691in}}{\pgfqpoint{2.545411in}{1.753691in}}%
\pgfpathclose%
\pgfusepath{stroke,fill}%
\end{pgfscope}%
\begin{pgfscope}%
\pgfpathrectangle{\pgfqpoint{0.526127in}{0.331635in}}{\pgfqpoint{9.300000in}{7.700000in}}%
\pgfusepath{clip}%
\pgfsetbuttcap%
\pgfsetroundjoin%
\definecolor{currentfill}{rgb}{0.870588,0.733333,0.607843}%
\pgfsetfillcolor{currentfill}%
\pgfsetlinewidth{0.481800pt}%
\definecolor{currentstroke}{rgb}{1.000000,1.000000,1.000000}%
\pgfsetstrokecolor{currentstroke}%
\pgfsetdash{}{0pt}%
\pgfpathmoveto{\pgfqpoint{4.257376in}{1.879369in}}%
\pgfpathcurveto{\pgfqpoint{4.268426in}{1.879369in}}{\pgfqpoint{4.279025in}{1.883759in}}{\pgfqpoint{4.286839in}{1.891572in}}%
\pgfpathcurveto{\pgfqpoint{4.294653in}{1.899386in}}{\pgfqpoint{4.299043in}{1.909985in}}{\pgfqpoint{4.299043in}{1.921035in}}%
\pgfpathcurveto{\pgfqpoint{4.299043in}{1.932085in}}{\pgfqpoint{4.294653in}{1.942684in}}{\pgfqpoint{4.286839in}{1.950498in}}%
\pgfpathcurveto{\pgfqpoint{4.279025in}{1.958312in}}{\pgfqpoint{4.268426in}{1.962702in}}{\pgfqpoint{4.257376in}{1.962702in}}%
\pgfpathcurveto{\pgfqpoint{4.246326in}{1.962702in}}{\pgfqpoint{4.235727in}{1.958312in}}{\pgfqpoint{4.227913in}{1.950498in}}%
\pgfpathcurveto{\pgfqpoint{4.220100in}{1.942684in}}{\pgfqpoint{4.215709in}{1.932085in}}{\pgfqpoint{4.215709in}{1.921035in}}%
\pgfpathcurveto{\pgfqpoint{4.215709in}{1.909985in}}{\pgfqpoint{4.220100in}{1.899386in}}{\pgfqpoint{4.227913in}{1.891572in}}%
\pgfpathcurveto{\pgfqpoint{4.235727in}{1.883759in}}{\pgfqpoint{4.246326in}{1.879369in}}{\pgfqpoint{4.257376in}{1.879369in}}%
\pgfpathclose%
\pgfusepath{stroke,fill}%
\end{pgfscope}%
\begin{pgfscope}%
\pgfpathrectangle{\pgfqpoint{0.526127in}{0.331635in}}{\pgfqpoint{9.300000in}{7.700000in}}%
\pgfusepath{clip}%
\pgfsetbuttcap%
\pgfsetroundjoin%
\definecolor{currentfill}{rgb}{0.980392,0.690196,0.894118}%
\pgfsetfillcolor{currentfill}%
\pgfsetlinewidth{0.481800pt}%
\definecolor{currentstroke}{rgb}{1.000000,1.000000,1.000000}%
\pgfsetstrokecolor{currentstroke}%
\pgfsetdash{}{0pt}%
\pgfpathmoveto{\pgfqpoint{8.127617in}{4.787796in}}%
\pgfpathcurveto{\pgfqpoint{8.138667in}{4.787796in}}{\pgfqpoint{8.149266in}{4.792186in}}{\pgfqpoint{8.157079in}{4.800000in}}%
\pgfpathcurveto{\pgfqpoint{8.164893in}{4.807814in}}{\pgfqpoint{8.169283in}{4.818413in}}{\pgfqpoint{8.169283in}{4.829463in}}%
\pgfpathcurveto{\pgfqpoint{8.169283in}{4.840513in}}{\pgfqpoint{8.164893in}{4.851112in}}{\pgfqpoint{8.157079in}{4.858926in}}%
\pgfpathcurveto{\pgfqpoint{8.149266in}{4.866739in}}{\pgfqpoint{8.138667in}{4.871129in}}{\pgfqpoint{8.127617in}{4.871129in}}%
\pgfpathcurveto{\pgfqpoint{8.116566in}{4.871129in}}{\pgfqpoint{8.105967in}{4.866739in}}{\pgfqpoint{8.098154in}{4.858926in}}%
\pgfpathcurveto{\pgfqpoint{8.090340in}{4.851112in}}{\pgfqpoint{8.085950in}{4.840513in}}{\pgfqpoint{8.085950in}{4.829463in}}%
\pgfpathcurveto{\pgfqpoint{8.085950in}{4.818413in}}{\pgfqpoint{8.090340in}{4.807814in}}{\pgfqpoint{8.098154in}{4.800000in}}%
\pgfpathcurveto{\pgfqpoint{8.105967in}{4.792186in}}{\pgfqpoint{8.116566in}{4.787796in}}{\pgfqpoint{8.127617in}{4.787796in}}%
\pgfpathclose%
\pgfusepath{stroke,fill}%
\end{pgfscope}%
\begin{pgfscope}%
\pgfpathrectangle{\pgfqpoint{0.526127in}{0.331635in}}{\pgfqpoint{9.300000in}{7.700000in}}%
\pgfusepath{clip}%
\pgfsetbuttcap%
\pgfsetroundjoin%
\definecolor{currentfill}{rgb}{0.980392,0.690196,0.894118}%
\pgfsetfillcolor{currentfill}%
\pgfsetlinewidth{0.481800pt}%
\definecolor{currentstroke}{rgb}{1.000000,1.000000,1.000000}%
\pgfsetstrokecolor{currentstroke}%
\pgfsetdash{}{0pt}%
\pgfpathmoveto{\pgfqpoint{5.250982in}{7.293098in}}%
\pgfpathcurveto{\pgfqpoint{5.262032in}{7.293098in}}{\pgfqpoint{5.272631in}{7.297488in}}{\pgfqpoint{5.280445in}{7.305302in}}%
\pgfpathcurveto{\pgfqpoint{5.288258in}{7.313115in}}{\pgfqpoint{5.292648in}{7.323714in}}{\pgfqpoint{5.292648in}{7.334764in}}%
\pgfpathcurveto{\pgfqpoint{5.292648in}{7.345814in}}{\pgfqpoint{5.288258in}{7.356414in}}{\pgfqpoint{5.280445in}{7.364227in}}%
\pgfpathcurveto{\pgfqpoint{5.272631in}{7.372041in}}{\pgfqpoint{5.262032in}{7.376431in}}{\pgfqpoint{5.250982in}{7.376431in}}%
\pgfpathcurveto{\pgfqpoint{5.239932in}{7.376431in}}{\pgfqpoint{5.229333in}{7.372041in}}{\pgfqpoint{5.221519in}{7.364227in}}%
\pgfpathcurveto{\pgfqpoint{5.213705in}{7.356414in}}{\pgfqpoint{5.209315in}{7.345814in}}{\pgfqpoint{5.209315in}{7.334764in}}%
\pgfpathcurveto{\pgfqpoint{5.209315in}{7.323714in}}{\pgfqpoint{5.213705in}{7.313115in}}{\pgfqpoint{5.221519in}{7.305302in}}%
\pgfpathcurveto{\pgfqpoint{5.229333in}{7.297488in}}{\pgfqpoint{5.239932in}{7.293098in}}{\pgfqpoint{5.250982in}{7.293098in}}%
\pgfpathclose%
\pgfusepath{stroke,fill}%
\end{pgfscope}%
\begin{pgfscope}%
\pgfpathrectangle{\pgfqpoint{0.526127in}{0.331635in}}{\pgfqpoint{9.300000in}{7.700000in}}%
\pgfusepath{clip}%
\pgfsetbuttcap%
\pgfsetroundjoin%
\definecolor{currentfill}{rgb}{0.980392,0.690196,0.894118}%
\pgfsetfillcolor{currentfill}%
\pgfsetlinewidth{0.481800pt}%
\definecolor{currentstroke}{rgb}{1.000000,1.000000,1.000000}%
\pgfsetstrokecolor{currentstroke}%
\pgfsetdash{}{0pt}%
\pgfpathmoveto{\pgfqpoint{4.648550in}{2.894499in}}%
\pgfpathcurveto{\pgfqpoint{4.659600in}{2.894499in}}{\pgfqpoint{4.670199in}{2.898889in}}{\pgfqpoint{4.678012in}{2.906703in}}%
\pgfpathcurveto{\pgfqpoint{4.685826in}{2.914516in}}{\pgfqpoint{4.690216in}{2.925115in}}{\pgfqpoint{4.690216in}{2.936165in}}%
\pgfpathcurveto{\pgfqpoint{4.690216in}{2.947216in}}{\pgfqpoint{4.685826in}{2.957815in}}{\pgfqpoint{4.678012in}{2.965628in}}%
\pgfpathcurveto{\pgfqpoint{4.670199in}{2.973442in}}{\pgfqpoint{4.659600in}{2.977832in}}{\pgfqpoint{4.648550in}{2.977832in}}%
\pgfpathcurveto{\pgfqpoint{4.637499in}{2.977832in}}{\pgfqpoint{4.626900in}{2.973442in}}{\pgfqpoint{4.619087in}{2.965628in}}%
\pgfpathcurveto{\pgfqpoint{4.611273in}{2.957815in}}{\pgfqpoint{4.606883in}{2.947216in}}{\pgfqpoint{4.606883in}{2.936165in}}%
\pgfpathcurveto{\pgfqpoint{4.606883in}{2.925115in}}{\pgfqpoint{4.611273in}{2.914516in}}{\pgfqpoint{4.619087in}{2.906703in}}%
\pgfpathcurveto{\pgfqpoint{4.626900in}{2.898889in}}{\pgfqpoint{4.637499in}{2.894499in}}{\pgfqpoint{4.648550in}{2.894499in}}%
\pgfpathclose%
\pgfusepath{stroke,fill}%
\end{pgfscope}%
\begin{pgfscope}%
\pgfpathrectangle{\pgfqpoint{0.526127in}{0.331635in}}{\pgfqpoint{9.300000in}{7.700000in}}%
\pgfusepath{clip}%
\pgfsetbuttcap%
\pgfsetroundjoin%
\definecolor{currentfill}{rgb}{0.980392,0.690196,0.894118}%
\pgfsetfillcolor{currentfill}%
\pgfsetlinewidth{0.481800pt}%
\definecolor{currentstroke}{rgb}{1.000000,1.000000,1.000000}%
\pgfsetstrokecolor{currentstroke}%
\pgfsetdash{}{0pt}%
\pgfpathmoveto{\pgfqpoint{6.518469in}{6.081978in}}%
\pgfpathcurveto{\pgfqpoint{6.529519in}{6.081978in}}{\pgfqpoint{6.540118in}{6.086368in}}{\pgfqpoint{6.547932in}{6.094182in}}%
\pgfpathcurveto{\pgfqpoint{6.555745in}{6.101995in}}{\pgfqpoint{6.560136in}{6.112594in}}{\pgfqpoint{6.560136in}{6.123645in}}%
\pgfpathcurveto{\pgfqpoint{6.560136in}{6.134695in}}{\pgfqpoint{6.555745in}{6.145294in}}{\pgfqpoint{6.547932in}{6.153107in}}%
\pgfpathcurveto{\pgfqpoint{6.540118in}{6.160921in}}{\pgfqpoint{6.529519in}{6.165311in}}{\pgfqpoint{6.518469in}{6.165311in}}%
\pgfpathcurveto{\pgfqpoint{6.507419in}{6.165311in}}{\pgfqpoint{6.496820in}{6.160921in}}{\pgfqpoint{6.489006in}{6.153107in}}%
\pgfpathcurveto{\pgfqpoint{6.481193in}{6.145294in}}{\pgfqpoint{6.476802in}{6.134695in}}{\pgfqpoint{6.476802in}{6.123645in}}%
\pgfpathcurveto{\pgfqpoint{6.476802in}{6.112594in}}{\pgfqpoint{6.481193in}{6.101995in}}{\pgfqpoint{6.489006in}{6.094182in}}%
\pgfpathcurveto{\pgfqpoint{6.496820in}{6.086368in}}{\pgfqpoint{6.507419in}{6.081978in}}{\pgfqpoint{6.518469in}{6.081978in}}%
\pgfpathclose%
\pgfusepath{stroke,fill}%
\end{pgfscope}%
\begin{pgfscope}%
\pgfpathrectangle{\pgfqpoint{0.526127in}{0.331635in}}{\pgfqpoint{9.300000in}{7.700000in}}%
\pgfusepath{clip}%
\pgfsetbuttcap%
\pgfsetroundjoin%
\definecolor{currentfill}{rgb}{0.980392,0.690196,0.894118}%
\pgfsetfillcolor{currentfill}%
\pgfsetlinewidth{0.481800pt}%
\definecolor{currentstroke}{rgb}{1.000000,1.000000,1.000000}%
\pgfsetstrokecolor{currentstroke}%
\pgfsetdash{}{0pt}%
\pgfpathmoveto{\pgfqpoint{8.083074in}{4.699371in}}%
\pgfpathcurveto{\pgfqpoint{8.094124in}{4.699371in}}{\pgfqpoint{8.104723in}{4.703761in}}{\pgfqpoint{8.112537in}{4.711575in}}%
\pgfpathcurveto{\pgfqpoint{8.120350in}{4.719388in}}{\pgfqpoint{8.124741in}{4.729987in}}{\pgfqpoint{8.124741in}{4.741037in}}%
\pgfpathcurveto{\pgfqpoint{8.124741in}{4.752088in}}{\pgfqpoint{8.120350in}{4.762687in}}{\pgfqpoint{8.112537in}{4.770500in}}%
\pgfpathcurveto{\pgfqpoint{8.104723in}{4.778314in}}{\pgfqpoint{8.094124in}{4.782704in}}{\pgfqpoint{8.083074in}{4.782704in}}%
\pgfpathcurveto{\pgfqpoint{8.072024in}{4.782704in}}{\pgfqpoint{8.061425in}{4.778314in}}{\pgfqpoint{8.053611in}{4.770500in}}%
\pgfpathcurveto{\pgfqpoint{8.045798in}{4.762687in}}{\pgfqpoint{8.041407in}{4.752088in}}{\pgfqpoint{8.041407in}{4.741037in}}%
\pgfpathcurveto{\pgfqpoint{8.041407in}{4.729987in}}{\pgfqpoint{8.045798in}{4.719388in}}{\pgfqpoint{8.053611in}{4.711575in}}%
\pgfpathcurveto{\pgfqpoint{8.061425in}{4.703761in}}{\pgfqpoint{8.072024in}{4.699371in}}{\pgfqpoint{8.083074in}{4.699371in}}%
\pgfpathclose%
\pgfusepath{stroke,fill}%
\end{pgfscope}%
\begin{pgfscope}%
\pgfpathrectangle{\pgfqpoint{0.526127in}{0.331635in}}{\pgfqpoint{9.300000in}{7.700000in}}%
\pgfusepath{clip}%
\pgfsetbuttcap%
\pgfsetroundjoin%
\definecolor{currentfill}{rgb}{0.980392,0.690196,0.894118}%
\pgfsetfillcolor{currentfill}%
\pgfsetlinewidth{0.481800pt}%
\definecolor{currentstroke}{rgb}{1.000000,1.000000,1.000000}%
\pgfsetstrokecolor{currentstroke}%
\pgfsetdash{}{0pt}%
\pgfpathmoveto{\pgfqpoint{7.058761in}{4.090315in}}%
\pgfpathcurveto{\pgfqpoint{7.069811in}{4.090315in}}{\pgfqpoint{7.080410in}{4.094705in}}{\pgfqpoint{7.088224in}{4.102518in}}%
\pgfpathcurveto{\pgfqpoint{7.096038in}{4.110332in}}{\pgfqpoint{7.100428in}{4.120931in}}{\pgfqpoint{7.100428in}{4.131981in}}%
\pgfpathcurveto{\pgfqpoint{7.100428in}{4.143031in}}{\pgfqpoint{7.096038in}{4.153630in}}{\pgfqpoint{7.088224in}{4.161444in}}%
\pgfpathcurveto{\pgfqpoint{7.080410in}{4.169258in}}{\pgfqpoint{7.069811in}{4.173648in}}{\pgfqpoint{7.058761in}{4.173648in}}%
\pgfpathcurveto{\pgfqpoint{7.047711in}{4.173648in}}{\pgfqpoint{7.037112in}{4.169258in}}{\pgfqpoint{7.029298in}{4.161444in}}%
\pgfpathcurveto{\pgfqpoint{7.021485in}{4.153630in}}{\pgfqpoint{7.017094in}{4.143031in}}{\pgfqpoint{7.017094in}{4.131981in}}%
\pgfpathcurveto{\pgfqpoint{7.017094in}{4.120931in}}{\pgfqpoint{7.021485in}{4.110332in}}{\pgfqpoint{7.029298in}{4.102518in}}%
\pgfpathcurveto{\pgfqpoint{7.037112in}{4.094705in}}{\pgfqpoint{7.047711in}{4.090315in}}{\pgfqpoint{7.058761in}{4.090315in}}%
\pgfpathclose%
\pgfusepath{stroke,fill}%
\end{pgfscope}%
\begin{pgfscope}%
\pgfpathrectangle{\pgfqpoint{0.526127in}{0.331635in}}{\pgfqpoint{9.300000in}{7.700000in}}%
\pgfusepath{clip}%
\pgfsetbuttcap%
\pgfsetroundjoin%
\definecolor{currentfill}{rgb}{0.980392,0.690196,0.894118}%
\pgfsetfillcolor{currentfill}%
\pgfsetlinewidth{0.481800pt}%
\definecolor{currentstroke}{rgb}{1.000000,1.000000,1.000000}%
\pgfsetstrokecolor{currentstroke}%
\pgfsetdash{}{0pt}%
\pgfpathmoveto{\pgfqpoint{8.288664in}{4.840301in}}%
\pgfpathcurveto{\pgfqpoint{8.299714in}{4.840301in}}{\pgfqpoint{8.310313in}{4.844691in}}{\pgfqpoint{8.318127in}{4.852504in}}%
\pgfpathcurveto{\pgfqpoint{8.325940in}{4.860318in}}{\pgfqpoint{8.330331in}{4.870917in}}{\pgfqpoint{8.330331in}{4.881967in}}%
\pgfpathcurveto{\pgfqpoint{8.330331in}{4.893017in}}{\pgfqpoint{8.325940in}{4.903616in}}{\pgfqpoint{8.318127in}{4.911430in}}%
\pgfpathcurveto{\pgfqpoint{8.310313in}{4.919244in}}{\pgfqpoint{8.299714in}{4.923634in}}{\pgfqpoint{8.288664in}{4.923634in}}%
\pgfpathcurveto{\pgfqpoint{8.277614in}{4.923634in}}{\pgfqpoint{8.267015in}{4.919244in}}{\pgfqpoint{8.259201in}{4.911430in}}%
\pgfpathcurveto{\pgfqpoint{8.251388in}{4.903616in}}{\pgfqpoint{8.246997in}{4.893017in}}{\pgfqpoint{8.246997in}{4.881967in}}%
\pgfpathcurveto{\pgfqpoint{8.246997in}{4.870917in}}{\pgfqpoint{8.251388in}{4.860318in}}{\pgfqpoint{8.259201in}{4.852504in}}%
\pgfpathcurveto{\pgfqpoint{8.267015in}{4.844691in}}{\pgfqpoint{8.277614in}{4.840301in}}{\pgfqpoint{8.288664in}{4.840301in}}%
\pgfpathclose%
\pgfusepath{stroke,fill}%
\end{pgfscope}%
\begin{pgfscope}%
\pgfpathrectangle{\pgfqpoint{0.526127in}{0.331635in}}{\pgfqpoint{9.300000in}{7.700000in}}%
\pgfusepath{clip}%
\pgfsetbuttcap%
\pgfsetroundjoin%
\definecolor{currentfill}{rgb}{0.980392,0.690196,0.894118}%
\pgfsetfillcolor{currentfill}%
\pgfsetlinewidth{0.481800pt}%
\definecolor{currentstroke}{rgb}{1.000000,1.000000,1.000000}%
\pgfsetstrokecolor{currentstroke}%
\pgfsetdash{}{0pt}%
\pgfpathmoveto{\pgfqpoint{8.365658in}{4.173936in}}%
\pgfpathcurveto{\pgfqpoint{8.376708in}{4.173936in}}{\pgfqpoint{8.387307in}{4.178326in}}{\pgfqpoint{8.395121in}{4.186139in}}%
\pgfpathcurveto{\pgfqpoint{8.402935in}{4.193953in}}{\pgfqpoint{8.407325in}{4.204552in}}{\pgfqpoint{8.407325in}{4.215602in}}%
\pgfpathcurveto{\pgfqpoint{8.407325in}{4.226652in}}{\pgfqpoint{8.402935in}{4.237251in}}{\pgfqpoint{8.395121in}{4.245065in}}%
\pgfpathcurveto{\pgfqpoint{8.387307in}{4.252879in}}{\pgfqpoint{8.376708in}{4.257269in}}{\pgfqpoint{8.365658in}{4.257269in}}%
\pgfpathcurveto{\pgfqpoint{8.354608in}{4.257269in}}{\pgfqpoint{8.344009in}{4.252879in}}{\pgfqpoint{8.336195in}{4.245065in}}%
\pgfpathcurveto{\pgfqpoint{8.328382in}{4.237251in}}{\pgfqpoint{8.323991in}{4.226652in}}{\pgfqpoint{8.323991in}{4.215602in}}%
\pgfpathcurveto{\pgfqpoint{8.323991in}{4.204552in}}{\pgfqpoint{8.328382in}{4.193953in}}{\pgfqpoint{8.336195in}{4.186139in}}%
\pgfpathcurveto{\pgfqpoint{8.344009in}{4.178326in}}{\pgfqpoint{8.354608in}{4.173936in}}{\pgfqpoint{8.365658in}{4.173936in}}%
\pgfpathclose%
\pgfusepath{stroke,fill}%
\end{pgfscope}%
\begin{pgfscope}%
\pgfpathrectangle{\pgfqpoint{0.526127in}{0.331635in}}{\pgfqpoint{9.300000in}{7.700000in}}%
\pgfusepath{clip}%
\pgfsetbuttcap%
\pgfsetroundjoin%
\definecolor{currentfill}{rgb}{0.980392,0.690196,0.894118}%
\pgfsetfillcolor{currentfill}%
\pgfsetlinewidth{0.481800pt}%
\definecolor{currentstroke}{rgb}{1.000000,1.000000,1.000000}%
\pgfsetstrokecolor{currentstroke}%
\pgfsetdash{}{0pt}%
\pgfpathmoveto{\pgfqpoint{5.586667in}{5.980161in}}%
\pgfpathcurveto{\pgfqpoint{5.597717in}{5.980161in}}{\pgfqpoint{5.608316in}{5.984552in}}{\pgfqpoint{5.616130in}{5.992365in}}%
\pgfpathcurveto{\pgfqpoint{5.623944in}{6.000179in}}{\pgfqpoint{5.628334in}{6.010778in}}{\pgfqpoint{5.628334in}{6.021828in}}%
\pgfpathcurveto{\pgfqpoint{5.628334in}{6.032878in}}{\pgfqpoint{5.623944in}{6.043477in}}{\pgfqpoint{5.616130in}{6.051291in}}%
\pgfpathcurveto{\pgfqpoint{5.608316in}{6.059104in}}{\pgfqpoint{5.597717in}{6.063495in}}{\pgfqpoint{5.586667in}{6.063495in}}%
\pgfpathcurveto{\pgfqpoint{5.575617in}{6.063495in}}{\pgfqpoint{5.565018in}{6.059104in}}{\pgfqpoint{5.557205in}{6.051291in}}%
\pgfpathcurveto{\pgfqpoint{5.549391in}{6.043477in}}{\pgfqpoint{5.545001in}{6.032878in}}{\pgfqpoint{5.545001in}{6.021828in}}%
\pgfpathcurveto{\pgfqpoint{5.545001in}{6.010778in}}{\pgfqpoint{5.549391in}{6.000179in}}{\pgfqpoint{5.557205in}{5.992365in}}%
\pgfpathcurveto{\pgfqpoint{5.565018in}{5.984552in}}{\pgfqpoint{5.575617in}{5.980161in}}{\pgfqpoint{5.586667in}{5.980161in}}%
\pgfpathclose%
\pgfusepath{stroke,fill}%
\end{pgfscope}%
\begin{pgfscope}%
\pgfpathrectangle{\pgfqpoint{0.526127in}{0.331635in}}{\pgfqpoint{9.300000in}{7.700000in}}%
\pgfusepath{clip}%
\pgfsetbuttcap%
\pgfsetroundjoin%
\definecolor{currentfill}{rgb}{0.980392,0.690196,0.894118}%
\pgfsetfillcolor{currentfill}%
\pgfsetlinewidth{0.481800pt}%
\definecolor{currentstroke}{rgb}{1.000000,1.000000,1.000000}%
\pgfsetstrokecolor{currentstroke}%
\pgfsetdash{}{0pt}%
\pgfpathmoveto{\pgfqpoint{7.411729in}{4.222944in}}%
\pgfpathcurveto{\pgfqpoint{7.422780in}{4.222944in}}{\pgfqpoint{7.433379in}{4.227334in}}{\pgfqpoint{7.441192in}{4.235148in}}%
\pgfpathcurveto{\pgfqpoint{7.449006in}{4.242961in}}{\pgfqpoint{7.453396in}{4.253560in}}{\pgfqpoint{7.453396in}{4.264611in}}%
\pgfpathcurveto{\pgfqpoint{7.453396in}{4.275661in}}{\pgfqpoint{7.449006in}{4.286260in}}{\pgfqpoint{7.441192in}{4.294073in}}%
\pgfpathcurveto{\pgfqpoint{7.433379in}{4.301887in}}{\pgfqpoint{7.422780in}{4.306277in}}{\pgfqpoint{7.411729in}{4.306277in}}%
\pgfpathcurveto{\pgfqpoint{7.400679in}{4.306277in}}{\pgfqpoint{7.390080in}{4.301887in}}{\pgfqpoint{7.382267in}{4.294073in}}%
\pgfpathcurveto{\pgfqpoint{7.374453in}{4.286260in}}{\pgfqpoint{7.370063in}{4.275661in}}{\pgfqpoint{7.370063in}{4.264611in}}%
\pgfpathcurveto{\pgfqpoint{7.370063in}{4.253560in}}{\pgfqpoint{7.374453in}{4.242961in}}{\pgfqpoint{7.382267in}{4.235148in}}%
\pgfpathcurveto{\pgfqpoint{7.390080in}{4.227334in}}{\pgfqpoint{7.400679in}{4.222944in}}{\pgfqpoint{7.411729in}{4.222944in}}%
\pgfpathclose%
\pgfusepath{stroke,fill}%
\end{pgfscope}%
\begin{pgfscope}%
\pgfpathrectangle{\pgfqpoint{0.526127in}{0.331635in}}{\pgfqpoint{9.300000in}{7.700000in}}%
\pgfusepath{clip}%
\pgfsetbuttcap%
\pgfsetroundjoin%
\definecolor{currentfill}{rgb}{0.980392,0.690196,0.894118}%
\pgfsetfillcolor{currentfill}%
\pgfsetlinewidth{0.481800pt}%
\definecolor{currentstroke}{rgb}{1.000000,1.000000,1.000000}%
\pgfsetstrokecolor{currentstroke}%
\pgfsetdash{}{0pt}%
\pgfpathmoveto{\pgfqpoint{5.552197in}{5.456007in}}%
\pgfpathcurveto{\pgfqpoint{5.563247in}{5.456007in}}{\pgfqpoint{5.573846in}{5.460398in}}{\pgfqpoint{5.581660in}{5.468211in}}%
\pgfpathcurveto{\pgfqpoint{5.589473in}{5.476025in}}{\pgfqpoint{5.593864in}{5.486624in}}{\pgfqpoint{5.593864in}{5.497674in}}%
\pgfpathcurveto{\pgfqpoint{5.593864in}{5.508724in}}{\pgfqpoint{5.589473in}{5.519323in}}{\pgfqpoint{5.581660in}{5.527137in}}%
\pgfpathcurveto{\pgfqpoint{5.573846in}{5.534951in}}{\pgfqpoint{5.563247in}{5.539341in}}{\pgfqpoint{5.552197in}{5.539341in}}%
\pgfpathcurveto{\pgfqpoint{5.541147in}{5.539341in}}{\pgfqpoint{5.530548in}{5.534951in}}{\pgfqpoint{5.522734in}{5.527137in}}%
\pgfpathcurveto{\pgfqpoint{5.514921in}{5.519323in}}{\pgfqpoint{5.510530in}{5.508724in}}{\pgfqpoint{5.510530in}{5.497674in}}%
\pgfpathcurveto{\pgfqpoint{5.510530in}{5.486624in}}{\pgfqpoint{5.514921in}{5.476025in}}{\pgfqpoint{5.522734in}{5.468211in}}%
\pgfpathcurveto{\pgfqpoint{5.530548in}{5.460398in}}{\pgfqpoint{5.541147in}{5.456007in}}{\pgfqpoint{5.552197in}{5.456007in}}%
\pgfpathclose%
\pgfusepath{stroke,fill}%
\end{pgfscope}%
\begin{pgfscope}%
\pgfpathrectangle{\pgfqpoint{0.526127in}{0.331635in}}{\pgfqpoint{9.300000in}{7.700000in}}%
\pgfusepath{clip}%
\pgfsetbuttcap%
\pgfsetroundjoin%
\definecolor{currentfill}{rgb}{0.980392,0.690196,0.894118}%
\pgfsetfillcolor{currentfill}%
\pgfsetlinewidth{0.481800pt}%
\definecolor{currentstroke}{rgb}{1.000000,1.000000,1.000000}%
\pgfsetstrokecolor{currentstroke}%
\pgfsetdash{}{0pt}%
\pgfpathmoveto{\pgfqpoint{5.112613in}{5.981865in}}%
\pgfpathcurveto{\pgfqpoint{5.123663in}{5.981865in}}{\pgfqpoint{5.134262in}{5.986255in}}{\pgfqpoint{5.142076in}{5.994069in}}%
\pgfpathcurveto{\pgfqpoint{5.149889in}{6.001882in}}{\pgfqpoint{5.154280in}{6.012481in}}{\pgfqpoint{5.154280in}{6.023531in}}%
\pgfpathcurveto{\pgfqpoint{5.154280in}{6.034582in}}{\pgfqpoint{5.149889in}{6.045181in}}{\pgfqpoint{5.142076in}{6.052994in}}%
\pgfpathcurveto{\pgfqpoint{5.134262in}{6.060808in}}{\pgfqpoint{5.123663in}{6.065198in}}{\pgfqpoint{5.112613in}{6.065198in}}%
\pgfpathcurveto{\pgfqpoint{5.101563in}{6.065198in}}{\pgfqpoint{5.090964in}{6.060808in}}{\pgfqpoint{5.083150in}{6.052994in}}%
\pgfpathcurveto{\pgfqpoint{5.075337in}{6.045181in}}{\pgfqpoint{5.070946in}{6.034582in}}{\pgfqpoint{5.070946in}{6.023531in}}%
\pgfpathcurveto{\pgfqpoint{5.070946in}{6.012481in}}{\pgfqpoint{5.075337in}{6.001882in}}{\pgfqpoint{5.083150in}{5.994069in}}%
\pgfpathcurveto{\pgfqpoint{5.090964in}{5.986255in}}{\pgfqpoint{5.101563in}{5.981865in}}{\pgfqpoint{5.112613in}{5.981865in}}%
\pgfpathclose%
\pgfusepath{stroke,fill}%
\end{pgfscope}%
\begin{pgfscope}%
\pgfpathrectangle{\pgfqpoint{0.526127in}{0.331635in}}{\pgfqpoint{9.300000in}{7.700000in}}%
\pgfusepath{clip}%
\pgfsetbuttcap%
\pgfsetroundjoin%
\definecolor{currentfill}{rgb}{0.980392,0.690196,0.894118}%
\pgfsetfillcolor{currentfill}%
\pgfsetlinewidth{0.481800pt}%
\definecolor{currentstroke}{rgb}{1.000000,1.000000,1.000000}%
\pgfsetstrokecolor{currentstroke}%
\pgfsetdash{}{0pt}%
\pgfpathmoveto{\pgfqpoint{3.785574in}{6.619565in}}%
\pgfpathcurveto{\pgfqpoint{3.796624in}{6.619565in}}{\pgfqpoint{3.807223in}{6.623955in}}{\pgfqpoint{3.815037in}{6.631768in}}%
\pgfpathcurveto{\pgfqpoint{3.822850in}{6.639582in}}{\pgfqpoint{3.827241in}{6.650181in}}{\pgfqpoint{3.827241in}{6.661231in}}%
\pgfpathcurveto{\pgfqpoint{3.827241in}{6.672281in}}{\pgfqpoint{3.822850in}{6.682880in}}{\pgfqpoint{3.815037in}{6.690694in}}%
\pgfpathcurveto{\pgfqpoint{3.807223in}{6.698508in}}{\pgfqpoint{3.796624in}{6.702898in}}{\pgfqpoint{3.785574in}{6.702898in}}%
\pgfpathcurveto{\pgfqpoint{3.774524in}{6.702898in}}{\pgfqpoint{3.763925in}{6.698508in}}{\pgfqpoint{3.756111in}{6.690694in}}%
\pgfpathcurveto{\pgfqpoint{3.748298in}{6.682880in}}{\pgfqpoint{3.743907in}{6.672281in}}{\pgfqpoint{3.743907in}{6.661231in}}%
\pgfpathcurveto{\pgfqpoint{3.743907in}{6.650181in}}{\pgfqpoint{3.748298in}{6.639582in}}{\pgfqpoint{3.756111in}{6.631768in}}%
\pgfpathcurveto{\pgfqpoint{3.763925in}{6.623955in}}{\pgfqpoint{3.774524in}{6.619565in}}{\pgfqpoint{3.785574in}{6.619565in}}%
\pgfpathclose%
\pgfusepath{stroke,fill}%
\end{pgfscope}%
\begin{pgfscope}%
\pgfpathrectangle{\pgfqpoint{0.526127in}{0.331635in}}{\pgfqpoint{9.300000in}{7.700000in}}%
\pgfusepath{clip}%
\pgfsetbuttcap%
\pgfsetroundjoin%
\definecolor{currentfill}{rgb}{0.980392,0.690196,0.894118}%
\pgfsetfillcolor{currentfill}%
\pgfsetlinewidth{0.481800pt}%
\definecolor{currentstroke}{rgb}{1.000000,1.000000,1.000000}%
\pgfsetstrokecolor{currentstroke}%
\pgfsetdash{}{0pt}%
\pgfpathmoveto{\pgfqpoint{3.136695in}{6.197030in}}%
\pgfpathcurveto{\pgfqpoint{3.147746in}{6.197030in}}{\pgfqpoint{3.158345in}{6.201420in}}{\pgfqpoint{3.166158in}{6.209234in}}%
\pgfpathcurveto{\pgfqpoint{3.173972in}{6.217047in}}{\pgfqpoint{3.178362in}{6.227646in}}{\pgfqpoint{3.178362in}{6.238697in}}%
\pgfpathcurveto{\pgfqpoint{3.178362in}{6.249747in}}{\pgfqpoint{3.173972in}{6.260346in}}{\pgfqpoint{3.166158in}{6.268159in}}%
\pgfpathcurveto{\pgfqpoint{3.158345in}{6.275973in}}{\pgfqpoint{3.147746in}{6.280363in}}{\pgfqpoint{3.136695in}{6.280363in}}%
\pgfpathcurveto{\pgfqpoint{3.125645in}{6.280363in}}{\pgfqpoint{3.115046in}{6.275973in}}{\pgfqpoint{3.107233in}{6.268159in}}%
\pgfpathcurveto{\pgfqpoint{3.099419in}{6.260346in}}{\pgfqpoint{3.095029in}{6.249747in}}{\pgfqpoint{3.095029in}{6.238697in}}%
\pgfpathcurveto{\pgfqpoint{3.095029in}{6.227646in}}{\pgfqpoint{3.099419in}{6.217047in}}{\pgfqpoint{3.107233in}{6.209234in}}%
\pgfpathcurveto{\pgfqpoint{3.115046in}{6.201420in}}{\pgfqpoint{3.125645in}{6.197030in}}{\pgfqpoint{3.136695in}{6.197030in}}%
\pgfpathclose%
\pgfusepath{stroke,fill}%
\end{pgfscope}%
\begin{pgfscope}%
\pgfpathrectangle{\pgfqpoint{0.526127in}{0.331635in}}{\pgfqpoint{9.300000in}{7.700000in}}%
\pgfusepath{clip}%
\pgfsetbuttcap%
\pgfsetroundjoin%
\definecolor{currentfill}{rgb}{0.980392,0.690196,0.894118}%
\pgfsetfillcolor{currentfill}%
\pgfsetlinewidth{0.481800pt}%
\definecolor{currentstroke}{rgb}{1.000000,1.000000,1.000000}%
\pgfsetstrokecolor{currentstroke}%
\pgfsetdash{}{0pt}%
\pgfpathmoveto{\pgfqpoint{8.338475in}{3.480006in}}%
\pgfpathcurveto{\pgfqpoint{8.349526in}{3.480006in}}{\pgfqpoint{8.360125in}{3.484396in}}{\pgfqpoint{8.367938in}{3.492210in}}%
\pgfpathcurveto{\pgfqpoint{8.375752in}{3.500024in}}{\pgfqpoint{8.380142in}{3.510623in}}{\pgfqpoint{8.380142in}{3.521673in}}%
\pgfpathcurveto{\pgfqpoint{8.380142in}{3.532723in}}{\pgfqpoint{8.375752in}{3.543322in}}{\pgfqpoint{8.367938in}{3.551136in}}%
\pgfpathcurveto{\pgfqpoint{8.360125in}{3.558949in}}{\pgfqpoint{8.349526in}{3.563340in}}{\pgfqpoint{8.338475in}{3.563340in}}%
\pgfpathcurveto{\pgfqpoint{8.327425in}{3.563340in}}{\pgfqpoint{8.316826in}{3.558949in}}{\pgfqpoint{8.309013in}{3.551136in}}%
\pgfpathcurveto{\pgfqpoint{8.301199in}{3.543322in}}{\pgfqpoint{8.296809in}{3.532723in}}{\pgfqpoint{8.296809in}{3.521673in}}%
\pgfpathcurveto{\pgfqpoint{8.296809in}{3.510623in}}{\pgfqpoint{8.301199in}{3.500024in}}{\pgfqpoint{8.309013in}{3.492210in}}%
\pgfpathcurveto{\pgfqpoint{8.316826in}{3.484396in}}{\pgfqpoint{8.327425in}{3.480006in}}{\pgfqpoint{8.338475in}{3.480006in}}%
\pgfpathclose%
\pgfusepath{stroke,fill}%
\end{pgfscope}%
\begin{pgfscope}%
\pgfpathrectangle{\pgfqpoint{0.526127in}{0.331635in}}{\pgfqpoint{9.300000in}{7.700000in}}%
\pgfusepath{clip}%
\pgfsetbuttcap%
\pgfsetroundjoin%
\definecolor{currentfill}{rgb}{0.980392,0.690196,0.894118}%
\pgfsetfillcolor{currentfill}%
\pgfsetlinewidth{0.481800pt}%
\definecolor{currentstroke}{rgb}{1.000000,1.000000,1.000000}%
\pgfsetstrokecolor{currentstroke}%
\pgfsetdash{}{0pt}%
\pgfpathmoveto{\pgfqpoint{2.725282in}{4.706332in}}%
\pgfpathcurveto{\pgfqpoint{2.736332in}{4.706332in}}{\pgfqpoint{2.746931in}{4.710723in}}{\pgfqpoint{2.754745in}{4.718536in}}%
\pgfpathcurveto{\pgfqpoint{2.762559in}{4.726350in}}{\pgfqpoint{2.766949in}{4.736949in}}{\pgfqpoint{2.766949in}{4.747999in}}%
\pgfpathcurveto{\pgfqpoint{2.766949in}{4.759049in}}{\pgfqpoint{2.762559in}{4.769648in}}{\pgfqpoint{2.754745in}{4.777462in}}%
\pgfpathcurveto{\pgfqpoint{2.746931in}{4.785275in}}{\pgfqpoint{2.736332in}{4.789666in}}{\pgfqpoint{2.725282in}{4.789666in}}%
\pgfpathcurveto{\pgfqpoint{2.714232in}{4.789666in}}{\pgfqpoint{2.703633in}{4.785275in}}{\pgfqpoint{2.695819in}{4.777462in}}%
\pgfpathcurveto{\pgfqpoint{2.688006in}{4.769648in}}{\pgfqpoint{2.683616in}{4.759049in}}{\pgfqpoint{2.683616in}{4.747999in}}%
\pgfpathcurveto{\pgfqpoint{2.683616in}{4.736949in}}{\pgfqpoint{2.688006in}{4.726350in}}{\pgfqpoint{2.695819in}{4.718536in}}%
\pgfpathcurveto{\pgfqpoint{2.703633in}{4.710723in}}{\pgfqpoint{2.714232in}{4.706332in}}{\pgfqpoint{2.725282in}{4.706332in}}%
\pgfpathclose%
\pgfusepath{stroke,fill}%
\end{pgfscope}%
\begin{pgfscope}%
\pgfpathrectangle{\pgfqpoint{0.526127in}{0.331635in}}{\pgfqpoint{9.300000in}{7.700000in}}%
\pgfusepath{clip}%
\pgfsetbuttcap%
\pgfsetroundjoin%
\definecolor{currentfill}{rgb}{0.980392,0.690196,0.894118}%
\pgfsetfillcolor{currentfill}%
\pgfsetlinewidth{0.481800pt}%
\definecolor{currentstroke}{rgb}{1.000000,1.000000,1.000000}%
\pgfsetstrokecolor{currentstroke}%
\pgfsetdash{}{0pt}%
\pgfpathmoveto{\pgfqpoint{5.804992in}{2.150742in}}%
\pgfpathcurveto{\pgfqpoint{5.816042in}{2.150742in}}{\pgfqpoint{5.826641in}{2.155133in}}{\pgfqpoint{5.834454in}{2.162946in}}%
\pgfpathcurveto{\pgfqpoint{5.842268in}{2.170760in}}{\pgfqpoint{5.846658in}{2.181359in}}{\pgfqpoint{5.846658in}{2.192409in}}%
\pgfpathcurveto{\pgfqpoint{5.846658in}{2.203459in}}{\pgfqpoint{5.842268in}{2.214058in}}{\pgfqpoint{5.834454in}{2.221872in}}%
\pgfpathcurveto{\pgfqpoint{5.826641in}{2.229686in}}{\pgfqpoint{5.816042in}{2.234076in}}{\pgfqpoint{5.804992in}{2.234076in}}%
\pgfpathcurveto{\pgfqpoint{5.793942in}{2.234076in}}{\pgfqpoint{5.783342in}{2.229686in}}{\pgfqpoint{5.775529in}{2.221872in}}%
\pgfpathcurveto{\pgfqpoint{5.767715in}{2.214058in}}{\pgfqpoint{5.763325in}{2.203459in}}{\pgfqpoint{5.763325in}{2.192409in}}%
\pgfpathcurveto{\pgfqpoint{5.763325in}{2.181359in}}{\pgfqpoint{5.767715in}{2.170760in}}{\pgfqpoint{5.775529in}{2.162946in}}%
\pgfpathcurveto{\pgfqpoint{5.783342in}{2.155133in}}{\pgfqpoint{5.793942in}{2.150742in}}{\pgfqpoint{5.804992in}{2.150742in}}%
\pgfpathclose%
\pgfusepath{stroke,fill}%
\end{pgfscope}%
\begin{pgfscope}%
\pgfpathrectangle{\pgfqpoint{0.526127in}{0.331635in}}{\pgfqpoint{9.300000in}{7.700000in}}%
\pgfusepath{clip}%
\pgfsetbuttcap%
\pgfsetroundjoin%
\definecolor{currentfill}{rgb}{0.980392,0.690196,0.894118}%
\pgfsetfillcolor{currentfill}%
\pgfsetlinewidth{0.481800pt}%
\definecolor{currentstroke}{rgb}{1.000000,1.000000,1.000000}%
\pgfsetstrokecolor{currentstroke}%
\pgfsetdash{}{0pt}%
\pgfpathmoveto{\pgfqpoint{4.387621in}{6.270253in}}%
\pgfpathcurveto{\pgfqpoint{4.398672in}{6.270253in}}{\pgfqpoint{4.409271in}{6.274643in}}{\pgfqpoint{4.417084in}{6.282457in}}%
\pgfpathcurveto{\pgfqpoint{4.424898in}{6.290270in}}{\pgfqpoint{4.429288in}{6.300869in}}{\pgfqpoint{4.429288in}{6.311919in}}%
\pgfpathcurveto{\pgfqpoint{4.429288in}{6.322970in}}{\pgfqpoint{4.424898in}{6.333569in}}{\pgfqpoint{4.417084in}{6.341382in}}%
\pgfpathcurveto{\pgfqpoint{4.409271in}{6.349196in}}{\pgfqpoint{4.398672in}{6.353586in}}{\pgfqpoint{4.387621in}{6.353586in}}%
\pgfpathcurveto{\pgfqpoint{4.376571in}{6.353586in}}{\pgfqpoint{4.365972in}{6.349196in}}{\pgfqpoint{4.358159in}{6.341382in}}%
\pgfpathcurveto{\pgfqpoint{4.350345in}{6.333569in}}{\pgfqpoint{4.345955in}{6.322970in}}{\pgfqpoint{4.345955in}{6.311919in}}%
\pgfpathcurveto{\pgfqpoint{4.345955in}{6.300869in}}{\pgfqpoint{4.350345in}{6.290270in}}{\pgfqpoint{4.358159in}{6.282457in}}%
\pgfpathcurveto{\pgfqpoint{4.365972in}{6.274643in}}{\pgfqpoint{4.376571in}{6.270253in}}{\pgfqpoint{4.387621in}{6.270253in}}%
\pgfpathclose%
\pgfusepath{stroke,fill}%
\end{pgfscope}%
\begin{pgfscope}%
\pgfpathrectangle{\pgfqpoint{0.526127in}{0.331635in}}{\pgfqpoint{9.300000in}{7.700000in}}%
\pgfusepath{clip}%
\pgfsetbuttcap%
\pgfsetroundjoin%
\definecolor{currentfill}{rgb}{0.980392,0.690196,0.894118}%
\pgfsetfillcolor{currentfill}%
\pgfsetlinewidth{0.481800pt}%
\definecolor{currentstroke}{rgb}{1.000000,1.000000,1.000000}%
\pgfsetstrokecolor{currentstroke}%
\pgfsetdash{}{0pt}%
\pgfpathmoveto{\pgfqpoint{4.353160in}{5.759122in}}%
\pgfpathcurveto{\pgfqpoint{4.364210in}{5.759122in}}{\pgfqpoint{4.374810in}{5.763512in}}{\pgfqpoint{4.382623in}{5.771326in}}%
\pgfpathcurveto{\pgfqpoint{4.390437in}{5.779140in}}{\pgfqpoint{4.394827in}{5.789739in}}{\pgfqpoint{4.394827in}{5.800789in}}%
\pgfpathcurveto{\pgfqpoint{4.394827in}{5.811839in}}{\pgfqpoint{4.390437in}{5.822438in}}{\pgfqpoint{4.382623in}{5.830252in}}%
\pgfpathcurveto{\pgfqpoint{4.374810in}{5.838065in}}{\pgfqpoint{4.364210in}{5.842456in}}{\pgfqpoint{4.353160in}{5.842456in}}%
\pgfpathcurveto{\pgfqpoint{4.342110in}{5.842456in}}{\pgfqpoint{4.331511in}{5.838065in}}{\pgfqpoint{4.323698in}{5.830252in}}%
\pgfpathcurveto{\pgfqpoint{4.315884in}{5.822438in}}{\pgfqpoint{4.311494in}{5.811839in}}{\pgfqpoint{4.311494in}{5.800789in}}%
\pgfpathcurveto{\pgfqpoint{4.311494in}{5.789739in}}{\pgfqpoint{4.315884in}{5.779140in}}{\pgfqpoint{4.323698in}{5.771326in}}%
\pgfpathcurveto{\pgfqpoint{4.331511in}{5.763512in}}{\pgfqpoint{4.342110in}{5.759122in}}{\pgfqpoint{4.353160in}{5.759122in}}%
\pgfpathclose%
\pgfusepath{stroke,fill}%
\end{pgfscope}%
\begin{pgfscope}%
\pgfpathrectangle{\pgfqpoint{0.526127in}{0.331635in}}{\pgfqpoint{9.300000in}{7.700000in}}%
\pgfusepath{clip}%
\pgfsetbuttcap%
\pgfsetroundjoin%
\definecolor{currentfill}{rgb}{0.980392,0.690196,0.894118}%
\pgfsetfillcolor{currentfill}%
\pgfsetlinewidth{0.481800pt}%
\definecolor{currentstroke}{rgb}{1.000000,1.000000,1.000000}%
\pgfsetstrokecolor{currentstroke}%
\pgfsetdash{}{0pt}%
\pgfpathmoveto{\pgfqpoint{4.693280in}{5.787436in}}%
\pgfpathcurveto{\pgfqpoint{4.704330in}{5.787436in}}{\pgfqpoint{4.714929in}{5.791826in}}{\pgfqpoint{4.722743in}{5.799640in}}%
\pgfpathcurveto{\pgfqpoint{4.730556in}{5.807453in}}{\pgfqpoint{4.734947in}{5.818052in}}{\pgfqpoint{4.734947in}{5.829103in}}%
\pgfpathcurveto{\pgfqpoint{4.734947in}{5.840153in}}{\pgfqpoint{4.730556in}{5.850752in}}{\pgfqpoint{4.722743in}{5.858565in}}%
\pgfpathcurveto{\pgfqpoint{4.714929in}{5.866379in}}{\pgfqpoint{4.704330in}{5.870769in}}{\pgfqpoint{4.693280in}{5.870769in}}%
\pgfpathcurveto{\pgfqpoint{4.682230in}{5.870769in}}{\pgfqpoint{4.671631in}{5.866379in}}{\pgfqpoint{4.663817in}{5.858565in}}%
\pgfpathcurveto{\pgfqpoint{4.656003in}{5.850752in}}{\pgfqpoint{4.651613in}{5.840153in}}{\pgfqpoint{4.651613in}{5.829103in}}%
\pgfpathcurveto{\pgfqpoint{4.651613in}{5.818052in}}{\pgfqpoint{4.656003in}{5.807453in}}{\pgfqpoint{4.663817in}{5.799640in}}%
\pgfpathcurveto{\pgfqpoint{4.671631in}{5.791826in}}{\pgfqpoint{4.682230in}{5.787436in}}{\pgfqpoint{4.693280in}{5.787436in}}%
\pgfpathclose%
\pgfusepath{stroke,fill}%
\end{pgfscope}%
\begin{pgfscope}%
\pgfpathrectangle{\pgfqpoint{0.526127in}{0.331635in}}{\pgfqpoint{9.300000in}{7.700000in}}%
\pgfusepath{clip}%
\pgfsetbuttcap%
\pgfsetroundjoin%
\definecolor{currentfill}{rgb}{0.980392,0.690196,0.894118}%
\pgfsetfillcolor{currentfill}%
\pgfsetlinewidth{0.481800pt}%
\definecolor{currentstroke}{rgb}{1.000000,1.000000,1.000000}%
\pgfsetstrokecolor{currentstroke}%
\pgfsetdash{}{0pt}%
\pgfpathmoveto{\pgfqpoint{9.403399in}{5.614452in}}%
\pgfpathcurveto{\pgfqpoint{9.414450in}{5.614452in}}{\pgfqpoint{9.425049in}{5.618842in}}{\pgfqpoint{9.432862in}{5.626656in}}%
\pgfpathcurveto{\pgfqpoint{9.440676in}{5.634469in}}{\pgfqpoint{9.445066in}{5.645068in}}{\pgfqpoint{9.445066in}{5.656119in}}%
\pgfpathcurveto{\pgfqpoint{9.445066in}{5.667169in}}{\pgfqpoint{9.440676in}{5.677768in}}{\pgfqpoint{9.432862in}{5.685581in}}%
\pgfpathcurveto{\pgfqpoint{9.425049in}{5.693395in}}{\pgfqpoint{9.414450in}{5.697785in}}{\pgfqpoint{9.403399in}{5.697785in}}%
\pgfpathcurveto{\pgfqpoint{9.392349in}{5.697785in}}{\pgfqpoint{9.381750in}{5.693395in}}{\pgfqpoint{9.373937in}{5.685581in}}%
\pgfpathcurveto{\pgfqpoint{9.366123in}{5.677768in}}{\pgfqpoint{9.361733in}{5.667169in}}{\pgfqpoint{9.361733in}{5.656119in}}%
\pgfpathcurveto{\pgfqpoint{9.361733in}{5.645068in}}{\pgfqpoint{9.366123in}{5.634469in}}{\pgfqpoint{9.373937in}{5.626656in}}%
\pgfpathcurveto{\pgfqpoint{9.381750in}{5.618842in}}{\pgfqpoint{9.392349in}{5.614452in}}{\pgfqpoint{9.403399in}{5.614452in}}%
\pgfpathclose%
\pgfusepath{stroke,fill}%
\end{pgfscope}%
\begin{pgfscope}%
\pgfpathrectangle{\pgfqpoint{0.526127in}{0.331635in}}{\pgfqpoint{9.300000in}{7.700000in}}%
\pgfusepath{clip}%
\pgfsetbuttcap%
\pgfsetroundjoin%
\definecolor{currentfill}{rgb}{0.980392,0.690196,0.894118}%
\pgfsetfillcolor{currentfill}%
\pgfsetlinewidth{0.481800pt}%
\definecolor{currentstroke}{rgb}{1.000000,1.000000,1.000000}%
\pgfsetstrokecolor{currentstroke}%
\pgfsetdash{}{0pt}%
\pgfpathmoveto{\pgfqpoint{4.320810in}{3.309729in}}%
\pgfpathcurveto{\pgfqpoint{4.331861in}{3.309729in}}{\pgfqpoint{4.342460in}{3.314120in}}{\pgfqpoint{4.350273in}{3.321933in}}%
\pgfpathcurveto{\pgfqpoint{4.358087in}{3.329747in}}{\pgfqpoint{4.362477in}{3.340346in}}{\pgfqpoint{4.362477in}{3.351396in}}%
\pgfpathcurveto{\pgfqpoint{4.362477in}{3.362446in}}{\pgfqpoint{4.358087in}{3.373045in}}{\pgfqpoint{4.350273in}{3.380859in}}%
\pgfpathcurveto{\pgfqpoint{4.342460in}{3.388672in}}{\pgfqpoint{4.331861in}{3.393063in}}{\pgfqpoint{4.320810in}{3.393063in}}%
\pgfpathcurveto{\pgfqpoint{4.309760in}{3.393063in}}{\pgfqpoint{4.299161in}{3.388672in}}{\pgfqpoint{4.291348in}{3.380859in}}%
\pgfpathcurveto{\pgfqpoint{4.283534in}{3.373045in}}{\pgfqpoint{4.279144in}{3.362446in}}{\pgfqpoint{4.279144in}{3.351396in}}%
\pgfpathcurveto{\pgfqpoint{4.279144in}{3.340346in}}{\pgfqpoint{4.283534in}{3.329747in}}{\pgfqpoint{4.291348in}{3.321933in}}%
\pgfpathcurveto{\pgfqpoint{4.299161in}{3.314120in}}{\pgfqpoint{4.309760in}{3.309729in}}{\pgfqpoint{4.320810in}{3.309729in}}%
\pgfpathclose%
\pgfusepath{stroke,fill}%
\end{pgfscope}%
\begin{pgfscope}%
\pgfpathrectangle{\pgfqpoint{0.526127in}{0.331635in}}{\pgfqpoint{9.300000in}{7.700000in}}%
\pgfusepath{clip}%
\pgfsetbuttcap%
\pgfsetroundjoin%
\definecolor{currentfill}{rgb}{0.980392,0.690196,0.894118}%
\pgfsetfillcolor{currentfill}%
\pgfsetlinewidth{0.481800pt}%
\definecolor{currentstroke}{rgb}{1.000000,1.000000,1.000000}%
\pgfsetstrokecolor{currentstroke}%
\pgfsetdash{}{0pt}%
\pgfpathmoveto{\pgfqpoint{1.703182in}{5.215231in}}%
\pgfpathcurveto{\pgfqpoint{1.714232in}{5.215231in}}{\pgfqpoint{1.724831in}{5.219621in}}{\pgfqpoint{1.732644in}{5.227435in}}%
\pgfpathcurveto{\pgfqpoint{1.740458in}{5.235248in}}{\pgfqpoint{1.744848in}{5.245847in}}{\pgfqpoint{1.744848in}{5.256898in}}%
\pgfpathcurveto{\pgfqpoint{1.744848in}{5.267948in}}{\pgfqpoint{1.740458in}{5.278547in}}{\pgfqpoint{1.732644in}{5.286360in}}%
\pgfpathcurveto{\pgfqpoint{1.724831in}{5.294174in}}{\pgfqpoint{1.714232in}{5.298564in}}{\pgfqpoint{1.703182in}{5.298564in}}%
\pgfpathcurveto{\pgfqpoint{1.692131in}{5.298564in}}{\pgfqpoint{1.681532in}{5.294174in}}{\pgfqpoint{1.673719in}{5.286360in}}%
\pgfpathcurveto{\pgfqpoint{1.665905in}{5.278547in}}{\pgfqpoint{1.661515in}{5.267948in}}{\pgfqpoint{1.661515in}{5.256898in}}%
\pgfpathcurveto{\pgfqpoint{1.661515in}{5.245847in}}{\pgfqpoint{1.665905in}{5.235248in}}{\pgfqpoint{1.673719in}{5.227435in}}%
\pgfpathcurveto{\pgfqpoint{1.681532in}{5.219621in}}{\pgfqpoint{1.692131in}{5.215231in}}{\pgfqpoint{1.703182in}{5.215231in}}%
\pgfpathclose%
\pgfusepath{stroke,fill}%
\end{pgfscope}%
\begin{pgfscope}%
\pgfpathrectangle{\pgfqpoint{0.526127in}{0.331635in}}{\pgfqpoint{9.300000in}{7.700000in}}%
\pgfusepath{clip}%
\pgfsetbuttcap%
\pgfsetroundjoin%
\definecolor{currentfill}{rgb}{0.980392,0.690196,0.894118}%
\pgfsetfillcolor{currentfill}%
\pgfsetlinewidth{0.481800pt}%
\definecolor{currentstroke}{rgb}{1.000000,1.000000,1.000000}%
\pgfsetstrokecolor{currentstroke}%
\pgfsetdash{}{0pt}%
\pgfpathmoveto{\pgfqpoint{4.829760in}{6.437133in}}%
\pgfpathcurveto{\pgfqpoint{4.840811in}{6.437133in}}{\pgfqpoint{4.851410in}{6.441524in}}{\pgfqpoint{4.859223in}{6.449337in}}%
\pgfpathcurveto{\pgfqpoint{4.867037in}{6.457151in}}{\pgfqpoint{4.871427in}{6.467750in}}{\pgfqpoint{4.871427in}{6.478800in}}%
\pgfpathcurveto{\pgfqpoint{4.871427in}{6.489850in}}{\pgfqpoint{4.867037in}{6.500449in}}{\pgfqpoint{4.859223in}{6.508263in}}%
\pgfpathcurveto{\pgfqpoint{4.851410in}{6.516076in}}{\pgfqpoint{4.840811in}{6.520467in}}{\pgfqpoint{4.829760in}{6.520467in}}%
\pgfpathcurveto{\pgfqpoint{4.818710in}{6.520467in}}{\pgfqpoint{4.808111in}{6.516076in}}{\pgfqpoint{4.800298in}{6.508263in}}%
\pgfpathcurveto{\pgfqpoint{4.792484in}{6.500449in}}{\pgfqpoint{4.788094in}{6.489850in}}{\pgfqpoint{4.788094in}{6.478800in}}%
\pgfpathcurveto{\pgfqpoint{4.788094in}{6.467750in}}{\pgfqpoint{4.792484in}{6.457151in}}{\pgfqpoint{4.800298in}{6.449337in}}%
\pgfpathcurveto{\pgfqpoint{4.808111in}{6.441524in}}{\pgfqpoint{4.818710in}{6.437133in}}{\pgfqpoint{4.829760in}{6.437133in}}%
\pgfpathclose%
\pgfusepath{stroke,fill}%
\end{pgfscope}%
\begin{pgfscope}%
\pgfpathrectangle{\pgfqpoint{0.526127in}{0.331635in}}{\pgfqpoint{9.300000in}{7.700000in}}%
\pgfusepath{clip}%
\pgfsetbuttcap%
\pgfsetroundjoin%
\definecolor{currentfill}{rgb}{0.980392,0.690196,0.894118}%
\pgfsetfillcolor{currentfill}%
\pgfsetlinewidth{0.481800pt}%
\definecolor{currentstroke}{rgb}{1.000000,1.000000,1.000000}%
\pgfsetstrokecolor{currentstroke}%
\pgfsetdash{}{0pt}%
\pgfpathmoveto{\pgfqpoint{5.312702in}{4.293977in}}%
\pgfpathcurveto{\pgfqpoint{5.323753in}{4.293977in}}{\pgfqpoint{5.334352in}{4.298367in}}{\pgfqpoint{5.342165in}{4.306181in}}%
\pgfpathcurveto{\pgfqpoint{5.349979in}{4.313995in}}{\pgfqpoint{5.354369in}{4.324594in}}{\pgfqpoint{5.354369in}{4.335644in}}%
\pgfpathcurveto{\pgfqpoint{5.354369in}{4.346694in}}{\pgfqpoint{5.349979in}{4.357293in}}{\pgfqpoint{5.342165in}{4.365107in}}%
\pgfpathcurveto{\pgfqpoint{5.334352in}{4.372920in}}{\pgfqpoint{5.323753in}{4.377310in}}{\pgfqpoint{5.312702in}{4.377310in}}%
\pgfpathcurveto{\pgfqpoint{5.301652in}{4.377310in}}{\pgfqpoint{5.291053in}{4.372920in}}{\pgfqpoint{5.283240in}{4.365107in}}%
\pgfpathcurveto{\pgfqpoint{5.275426in}{4.357293in}}{\pgfqpoint{5.271036in}{4.346694in}}{\pgfqpoint{5.271036in}{4.335644in}}%
\pgfpathcurveto{\pgfqpoint{5.271036in}{4.324594in}}{\pgfqpoint{5.275426in}{4.313995in}}{\pgfqpoint{5.283240in}{4.306181in}}%
\pgfpathcurveto{\pgfqpoint{5.291053in}{4.298367in}}{\pgfqpoint{5.301652in}{4.293977in}}{\pgfqpoint{5.312702in}{4.293977in}}%
\pgfpathclose%
\pgfusepath{stroke,fill}%
\end{pgfscope}%
\begin{pgfscope}%
\pgfpathrectangle{\pgfqpoint{0.526127in}{0.331635in}}{\pgfqpoint{9.300000in}{7.700000in}}%
\pgfusepath{clip}%
\pgfsetbuttcap%
\pgfsetroundjoin%
\definecolor{currentfill}{rgb}{0.980392,0.690196,0.894118}%
\pgfsetfillcolor{currentfill}%
\pgfsetlinewidth{0.481800pt}%
\definecolor{currentstroke}{rgb}{1.000000,1.000000,1.000000}%
\pgfsetstrokecolor{currentstroke}%
\pgfsetdash{}{0pt}%
\pgfpathmoveto{\pgfqpoint{8.614970in}{4.727871in}}%
\pgfpathcurveto{\pgfqpoint{8.626020in}{4.727871in}}{\pgfqpoint{8.636619in}{4.732261in}}{\pgfqpoint{8.644433in}{4.740075in}}%
\pgfpathcurveto{\pgfqpoint{8.652247in}{4.747889in}}{\pgfqpoint{8.656637in}{4.758488in}}{\pgfqpoint{8.656637in}{4.769538in}}%
\pgfpathcurveto{\pgfqpoint{8.656637in}{4.780588in}}{\pgfqpoint{8.652247in}{4.791187in}}{\pgfqpoint{8.644433in}{4.799001in}}%
\pgfpathcurveto{\pgfqpoint{8.636619in}{4.806814in}}{\pgfqpoint{8.626020in}{4.811204in}}{\pgfqpoint{8.614970in}{4.811204in}}%
\pgfpathcurveto{\pgfqpoint{8.603920in}{4.811204in}}{\pgfqpoint{8.593321in}{4.806814in}}{\pgfqpoint{8.585508in}{4.799001in}}%
\pgfpathcurveto{\pgfqpoint{8.577694in}{4.791187in}}{\pgfqpoint{8.573304in}{4.780588in}}{\pgfqpoint{8.573304in}{4.769538in}}%
\pgfpathcurveto{\pgfqpoint{8.573304in}{4.758488in}}{\pgfqpoint{8.577694in}{4.747889in}}{\pgfqpoint{8.585508in}{4.740075in}}%
\pgfpathcurveto{\pgfqpoint{8.593321in}{4.732261in}}{\pgfqpoint{8.603920in}{4.727871in}}{\pgfqpoint{8.614970in}{4.727871in}}%
\pgfpathclose%
\pgfusepath{stroke,fill}%
\end{pgfscope}%
\begin{pgfscope}%
\pgfpathrectangle{\pgfqpoint{0.526127in}{0.331635in}}{\pgfqpoint{9.300000in}{7.700000in}}%
\pgfusepath{clip}%
\pgfsetbuttcap%
\pgfsetroundjoin%
\definecolor{currentfill}{rgb}{0.980392,0.690196,0.894118}%
\pgfsetfillcolor{currentfill}%
\pgfsetlinewidth{0.481800pt}%
\definecolor{currentstroke}{rgb}{1.000000,1.000000,1.000000}%
\pgfsetstrokecolor{currentstroke}%
\pgfsetdash{}{0pt}%
\pgfpathmoveto{\pgfqpoint{8.457929in}{5.790002in}}%
\pgfpathcurveto{\pgfqpoint{8.468979in}{5.790002in}}{\pgfqpoint{8.479578in}{5.794393in}}{\pgfqpoint{8.487392in}{5.802206in}}%
\pgfpathcurveto{\pgfqpoint{8.495205in}{5.810020in}}{\pgfqpoint{8.499596in}{5.820619in}}{\pgfqpoint{8.499596in}{5.831669in}}%
\pgfpathcurveto{\pgfqpoint{8.499596in}{5.842719in}}{\pgfqpoint{8.495205in}{5.853318in}}{\pgfqpoint{8.487392in}{5.861132in}}%
\pgfpathcurveto{\pgfqpoint{8.479578in}{5.868945in}}{\pgfqpoint{8.468979in}{5.873336in}}{\pgfqpoint{8.457929in}{5.873336in}}%
\pgfpathcurveto{\pgfqpoint{8.446879in}{5.873336in}}{\pgfqpoint{8.436280in}{5.868945in}}{\pgfqpoint{8.428466in}{5.861132in}}%
\pgfpathcurveto{\pgfqpoint{8.420653in}{5.853318in}}{\pgfqpoint{8.416262in}{5.842719in}}{\pgfqpoint{8.416262in}{5.831669in}}%
\pgfpathcurveto{\pgfqpoint{8.416262in}{5.820619in}}{\pgfqpoint{8.420653in}{5.810020in}}{\pgfqpoint{8.428466in}{5.802206in}}%
\pgfpathcurveto{\pgfqpoint{8.436280in}{5.794393in}}{\pgfqpoint{8.446879in}{5.790002in}}{\pgfqpoint{8.457929in}{5.790002in}}%
\pgfpathclose%
\pgfusepath{stroke,fill}%
\end{pgfscope}%
\begin{pgfscope}%
\pgfpathrectangle{\pgfqpoint{0.526127in}{0.331635in}}{\pgfqpoint{9.300000in}{7.700000in}}%
\pgfusepath{clip}%
\pgfsetbuttcap%
\pgfsetroundjoin%
\definecolor{currentfill}{rgb}{0.980392,0.690196,0.894118}%
\pgfsetfillcolor{currentfill}%
\pgfsetlinewidth{0.481800pt}%
\definecolor{currentstroke}{rgb}{1.000000,1.000000,1.000000}%
\pgfsetstrokecolor{currentstroke}%
\pgfsetdash{}{0pt}%
\pgfpathmoveto{\pgfqpoint{7.835370in}{4.136992in}}%
\pgfpathcurveto{\pgfqpoint{7.846421in}{4.136992in}}{\pgfqpoint{7.857020in}{4.141382in}}{\pgfqpoint{7.864833in}{4.149196in}}%
\pgfpathcurveto{\pgfqpoint{7.872647in}{4.157010in}}{\pgfqpoint{7.877037in}{4.167609in}}{\pgfqpoint{7.877037in}{4.178659in}}%
\pgfpathcurveto{\pgfqpoint{7.877037in}{4.189709in}}{\pgfqpoint{7.872647in}{4.200308in}}{\pgfqpoint{7.864833in}{4.208122in}}%
\pgfpathcurveto{\pgfqpoint{7.857020in}{4.215935in}}{\pgfqpoint{7.846421in}{4.220326in}}{\pgfqpoint{7.835370in}{4.220326in}}%
\pgfpathcurveto{\pgfqpoint{7.824320in}{4.220326in}}{\pgfqpoint{7.813721in}{4.215935in}}{\pgfqpoint{7.805908in}{4.208122in}}%
\pgfpathcurveto{\pgfqpoint{7.798094in}{4.200308in}}{\pgfqpoint{7.793704in}{4.189709in}}{\pgfqpoint{7.793704in}{4.178659in}}%
\pgfpathcurveto{\pgfqpoint{7.793704in}{4.167609in}}{\pgfqpoint{7.798094in}{4.157010in}}{\pgfqpoint{7.805908in}{4.149196in}}%
\pgfpathcurveto{\pgfqpoint{7.813721in}{4.141382in}}{\pgfqpoint{7.824320in}{4.136992in}}{\pgfqpoint{7.835370in}{4.136992in}}%
\pgfpathclose%
\pgfusepath{stroke,fill}%
\end{pgfscope}%
\begin{pgfscope}%
\pgfpathrectangle{\pgfqpoint{0.526127in}{0.331635in}}{\pgfqpoint{9.300000in}{7.700000in}}%
\pgfusepath{clip}%
\pgfsetbuttcap%
\pgfsetroundjoin%
\definecolor{currentfill}{rgb}{0.721569,0.521569,0.039216}%
\pgfsetfillcolor{currentfill}%
\pgfsetlinewidth{0.481800pt}%
\definecolor{currentstroke}{rgb}{1.000000,1.000000,1.000000}%
\pgfsetstrokecolor{currentstroke}%
\pgfsetdash{}{0pt}%
\pgfpathmoveto{\pgfqpoint{5.776420in}{4.037481in}}%
\pgfpathcurveto{\pgfqpoint{5.787470in}{4.037481in}}{\pgfqpoint{5.798069in}{4.041871in}}{\pgfqpoint{5.805883in}{4.049684in}}%
\pgfpathcurveto{\pgfqpoint{5.813696in}{4.057498in}}{\pgfqpoint{5.818087in}{4.068097in}}{\pgfqpoint{5.818087in}{4.079147in}}%
\pgfpathcurveto{\pgfqpoint{5.818087in}{4.090197in}}{\pgfqpoint{5.813696in}{4.100796in}}{\pgfqpoint{5.805883in}{4.108610in}}%
\pgfpathcurveto{\pgfqpoint{5.798069in}{4.116424in}}{\pgfqpoint{5.787470in}{4.120814in}}{\pgfqpoint{5.776420in}{4.120814in}}%
\pgfpathcurveto{\pgfqpoint{5.765370in}{4.120814in}}{\pgfqpoint{5.754771in}{4.116424in}}{\pgfqpoint{5.746957in}{4.108610in}}%
\pgfpathcurveto{\pgfqpoint{5.739144in}{4.100796in}}{\pgfqpoint{5.734753in}{4.090197in}}{\pgfqpoint{5.734753in}{4.079147in}}%
\pgfpathcurveto{\pgfqpoint{5.734753in}{4.068097in}}{\pgfqpoint{5.739144in}{4.057498in}}{\pgfqpoint{5.746957in}{4.049684in}}%
\pgfpathcurveto{\pgfqpoint{5.754771in}{4.041871in}}{\pgfqpoint{5.765370in}{4.037481in}}{\pgfqpoint{5.776420in}{4.037481in}}%
\pgfpathclose%
\pgfusepath{stroke,fill}%
\end{pgfscope}%
\begin{pgfscope}%
\pgfpathrectangle{\pgfqpoint{0.526127in}{0.331635in}}{\pgfqpoint{9.300000in}{7.700000in}}%
\pgfusepath{clip}%
\pgfsetbuttcap%
\pgfsetroundjoin%
\definecolor{currentfill}{rgb}{0.721569,0.521569,0.039216}%
\pgfsetfillcolor{currentfill}%
\pgfsetlinewidth{0.481800pt}%
\definecolor{currentstroke}{rgb}{1.000000,1.000000,1.000000}%
\pgfsetstrokecolor{currentstroke}%
\pgfsetdash{}{0pt}%
\pgfpathmoveto{\pgfqpoint{8.729922in}{5.014306in}}%
\pgfpathcurveto{\pgfqpoint{8.740972in}{5.014306in}}{\pgfqpoint{8.751571in}{5.018696in}}{\pgfqpoint{8.759385in}{5.026510in}}%
\pgfpathcurveto{\pgfqpoint{8.767198in}{5.034323in}}{\pgfqpoint{8.771589in}{5.044922in}}{\pgfqpoint{8.771589in}{5.055972in}}%
\pgfpathcurveto{\pgfqpoint{8.771589in}{5.067022in}}{\pgfqpoint{8.767198in}{5.077622in}}{\pgfqpoint{8.759385in}{5.085435in}}%
\pgfpathcurveto{\pgfqpoint{8.751571in}{5.093249in}}{\pgfqpoint{8.740972in}{5.097639in}}{\pgfqpoint{8.729922in}{5.097639in}}%
\pgfpathcurveto{\pgfqpoint{8.718872in}{5.097639in}}{\pgfqpoint{8.708273in}{5.093249in}}{\pgfqpoint{8.700459in}{5.085435in}}%
\pgfpathcurveto{\pgfqpoint{8.692646in}{5.077622in}}{\pgfqpoint{8.688255in}{5.067022in}}{\pgfqpoint{8.688255in}{5.055972in}}%
\pgfpathcurveto{\pgfqpoint{8.688255in}{5.044922in}}{\pgfqpoint{8.692646in}{5.034323in}}{\pgfqpoint{8.700459in}{5.026510in}}%
\pgfpathcurveto{\pgfqpoint{8.708273in}{5.018696in}}{\pgfqpoint{8.718872in}{5.014306in}}{\pgfqpoint{8.729922in}{5.014306in}}%
\pgfpathclose%
\pgfusepath{stroke,fill}%
\end{pgfscope}%
\begin{pgfscope}%
\pgfpathrectangle{\pgfqpoint{0.526127in}{0.331635in}}{\pgfqpoint{9.300000in}{7.700000in}}%
\pgfusepath{clip}%
\pgfsetbuttcap%
\pgfsetroundjoin%
\definecolor{currentfill}{rgb}{0.721569,0.521569,0.039216}%
\pgfsetfillcolor{currentfill}%
\pgfsetlinewidth{0.481800pt}%
\definecolor{currentstroke}{rgb}{1.000000,1.000000,1.000000}%
\pgfsetstrokecolor{currentstroke}%
\pgfsetdash{}{0pt}%
\pgfpathmoveto{\pgfqpoint{2.913096in}{4.616803in}}%
\pgfpathcurveto{\pgfqpoint{2.924146in}{4.616803in}}{\pgfqpoint{2.934745in}{4.621194in}}{\pgfqpoint{2.942558in}{4.629007in}}%
\pgfpathcurveto{\pgfqpoint{2.950372in}{4.636821in}}{\pgfqpoint{2.954762in}{4.647420in}}{\pgfqpoint{2.954762in}{4.658470in}}%
\pgfpathcurveto{\pgfqpoint{2.954762in}{4.669520in}}{\pgfqpoint{2.950372in}{4.680119in}}{\pgfqpoint{2.942558in}{4.687933in}}%
\pgfpathcurveto{\pgfqpoint{2.934745in}{4.695746in}}{\pgfqpoint{2.924146in}{4.700137in}}{\pgfqpoint{2.913096in}{4.700137in}}%
\pgfpathcurveto{\pgfqpoint{2.902045in}{4.700137in}}{\pgfqpoint{2.891446in}{4.695746in}}{\pgfqpoint{2.883633in}{4.687933in}}%
\pgfpathcurveto{\pgfqpoint{2.875819in}{4.680119in}}{\pgfqpoint{2.871429in}{4.669520in}}{\pgfqpoint{2.871429in}{4.658470in}}%
\pgfpathcurveto{\pgfqpoint{2.871429in}{4.647420in}}{\pgfqpoint{2.875819in}{4.636821in}}{\pgfqpoint{2.883633in}{4.629007in}}%
\pgfpathcurveto{\pgfqpoint{2.891446in}{4.621194in}}{\pgfqpoint{2.902045in}{4.616803in}}{\pgfqpoint{2.913096in}{4.616803in}}%
\pgfpathclose%
\pgfusepath{stroke,fill}%
\end{pgfscope}%
\begin{pgfscope}%
\pgfpathrectangle{\pgfqpoint{0.526127in}{0.331635in}}{\pgfqpoint{9.300000in}{7.700000in}}%
\pgfusepath{clip}%
\pgfsetbuttcap%
\pgfsetroundjoin%
\definecolor{currentfill}{rgb}{0.721569,0.521569,0.039216}%
\pgfsetfillcolor{currentfill}%
\pgfsetlinewidth{0.481800pt}%
\definecolor{currentstroke}{rgb}{1.000000,1.000000,1.000000}%
\pgfsetstrokecolor{currentstroke}%
\pgfsetdash{}{0pt}%
\pgfpathmoveto{\pgfqpoint{3.918807in}{5.738183in}}%
\pgfpathcurveto{\pgfqpoint{3.929858in}{5.738183in}}{\pgfqpoint{3.940457in}{5.742573in}}{\pgfqpoint{3.948270in}{5.750387in}}%
\pgfpathcurveto{\pgfqpoint{3.956084in}{5.758200in}}{\pgfqpoint{3.960474in}{5.768799in}}{\pgfqpoint{3.960474in}{5.779850in}}%
\pgfpathcurveto{\pgfqpoint{3.960474in}{5.790900in}}{\pgfqpoint{3.956084in}{5.801499in}}{\pgfqpoint{3.948270in}{5.809312in}}%
\pgfpathcurveto{\pgfqpoint{3.940457in}{5.817126in}}{\pgfqpoint{3.929858in}{5.821516in}}{\pgfqpoint{3.918807in}{5.821516in}}%
\pgfpathcurveto{\pgfqpoint{3.907757in}{5.821516in}}{\pgfqpoint{3.897158in}{5.817126in}}{\pgfqpoint{3.889345in}{5.809312in}}%
\pgfpathcurveto{\pgfqpoint{3.881531in}{5.801499in}}{\pgfqpoint{3.877141in}{5.790900in}}{\pgfqpoint{3.877141in}{5.779850in}}%
\pgfpathcurveto{\pgfqpoint{3.877141in}{5.768799in}}{\pgfqpoint{3.881531in}{5.758200in}}{\pgfqpoint{3.889345in}{5.750387in}}%
\pgfpathcurveto{\pgfqpoint{3.897158in}{5.742573in}}{\pgfqpoint{3.907757in}{5.738183in}}{\pgfqpoint{3.918807in}{5.738183in}}%
\pgfpathclose%
\pgfusepath{stroke,fill}%
\end{pgfscope}%
\begin{pgfscope}%
\pgfpathrectangle{\pgfqpoint{0.526127in}{0.331635in}}{\pgfqpoint{9.300000in}{7.700000in}}%
\pgfusepath{clip}%
\pgfsetbuttcap%
\pgfsetroundjoin%
\definecolor{currentfill}{rgb}{0.721569,0.521569,0.039216}%
\pgfsetfillcolor{currentfill}%
\pgfsetlinewidth{0.481800pt}%
\definecolor{currentstroke}{rgb}{1.000000,1.000000,1.000000}%
\pgfsetstrokecolor{currentstroke}%
\pgfsetdash{}{0pt}%
\pgfpathmoveto{\pgfqpoint{2.663268in}{3.834370in}}%
\pgfpathcurveto{\pgfqpoint{2.674319in}{3.834370in}}{\pgfqpoint{2.684918in}{3.838760in}}{\pgfqpoint{2.692731in}{3.846574in}}%
\pgfpathcurveto{\pgfqpoint{2.700545in}{3.854387in}}{\pgfqpoint{2.704935in}{3.864986in}}{\pgfqpoint{2.704935in}{3.876036in}}%
\pgfpathcurveto{\pgfqpoint{2.704935in}{3.887087in}}{\pgfqpoint{2.700545in}{3.897686in}}{\pgfqpoint{2.692731in}{3.905499in}}%
\pgfpathcurveto{\pgfqpoint{2.684918in}{3.913313in}}{\pgfqpoint{2.674319in}{3.917703in}}{\pgfqpoint{2.663268in}{3.917703in}}%
\pgfpathcurveto{\pgfqpoint{2.652218in}{3.917703in}}{\pgfqpoint{2.641619in}{3.913313in}}{\pgfqpoint{2.633806in}{3.905499in}}%
\pgfpathcurveto{\pgfqpoint{2.625992in}{3.897686in}}{\pgfqpoint{2.621602in}{3.887087in}}{\pgfqpoint{2.621602in}{3.876036in}}%
\pgfpathcurveto{\pgfqpoint{2.621602in}{3.864986in}}{\pgfqpoint{2.625992in}{3.854387in}}{\pgfqpoint{2.633806in}{3.846574in}}%
\pgfpathcurveto{\pgfqpoint{2.641619in}{3.838760in}}{\pgfqpoint{2.652218in}{3.834370in}}{\pgfqpoint{2.663268in}{3.834370in}}%
\pgfpathclose%
\pgfusepath{stroke,fill}%
\end{pgfscope}%
\begin{pgfscope}%
\pgfpathrectangle{\pgfqpoint{0.526127in}{0.331635in}}{\pgfqpoint{9.300000in}{7.700000in}}%
\pgfusepath{clip}%
\pgfsetbuttcap%
\pgfsetroundjoin%
\definecolor{currentfill}{rgb}{0.721569,0.521569,0.039216}%
\pgfsetfillcolor{currentfill}%
\pgfsetlinewidth{0.481800pt}%
\definecolor{currentstroke}{rgb}{1.000000,1.000000,1.000000}%
\pgfsetstrokecolor{currentstroke}%
\pgfsetdash{}{0pt}%
\pgfpathmoveto{\pgfqpoint{6.282605in}{4.913350in}}%
\pgfpathcurveto{\pgfqpoint{6.293655in}{4.913350in}}{\pgfqpoint{6.304254in}{4.917741in}}{\pgfqpoint{6.312068in}{4.925554in}}%
\pgfpathcurveto{\pgfqpoint{6.319881in}{4.933368in}}{\pgfqpoint{6.324272in}{4.943967in}}{\pgfqpoint{6.324272in}{4.955017in}}%
\pgfpathcurveto{\pgfqpoint{6.324272in}{4.966067in}}{\pgfqpoint{6.319881in}{4.976666in}}{\pgfqpoint{6.312068in}{4.984480in}}%
\pgfpathcurveto{\pgfqpoint{6.304254in}{4.992293in}}{\pgfqpoint{6.293655in}{4.996684in}}{\pgfqpoint{6.282605in}{4.996684in}}%
\pgfpathcurveto{\pgfqpoint{6.271555in}{4.996684in}}{\pgfqpoint{6.260956in}{4.992293in}}{\pgfqpoint{6.253142in}{4.984480in}}%
\pgfpathcurveto{\pgfqpoint{6.245328in}{4.976666in}}{\pgfqpoint{6.240938in}{4.966067in}}{\pgfqpoint{6.240938in}{4.955017in}}%
\pgfpathcurveto{\pgfqpoint{6.240938in}{4.943967in}}{\pgfqpoint{6.245328in}{4.933368in}}{\pgfqpoint{6.253142in}{4.925554in}}%
\pgfpathcurveto{\pgfqpoint{6.260956in}{4.917741in}}{\pgfqpoint{6.271555in}{4.913350in}}{\pgfqpoint{6.282605in}{4.913350in}}%
\pgfpathclose%
\pgfusepath{stroke,fill}%
\end{pgfscope}%
\begin{pgfscope}%
\pgfpathrectangle{\pgfqpoint{0.526127in}{0.331635in}}{\pgfqpoint{9.300000in}{7.700000in}}%
\pgfusepath{clip}%
\pgfsetbuttcap%
\pgfsetroundjoin%
\definecolor{currentfill}{rgb}{0.721569,0.521569,0.039216}%
\pgfsetfillcolor{currentfill}%
\pgfsetlinewidth{0.481800pt}%
\definecolor{currentstroke}{rgb}{1.000000,1.000000,1.000000}%
\pgfsetstrokecolor{currentstroke}%
\pgfsetdash{}{0pt}%
\pgfpathmoveto{\pgfqpoint{3.179398in}{3.411320in}}%
\pgfpathcurveto{\pgfqpoint{3.190448in}{3.411320in}}{\pgfqpoint{3.201047in}{3.415710in}}{\pgfqpoint{3.208861in}{3.423524in}}%
\pgfpathcurveto{\pgfqpoint{3.216674in}{3.431338in}}{\pgfqpoint{3.221065in}{3.441937in}}{\pgfqpoint{3.221065in}{3.452987in}}%
\pgfpathcurveto{\pgfqpoint{3.221065in}{3.464037in}}{\pgfqpoint{3.216674in}{3.474636in}}{\pgfqpoint{3.208861in}{3.482450in}}%
\pgfpathcurveto{\pgfqpoint{3.201047in}{3.490263in}}{\pgfqpoint{3.190448in}{3.494653in}}{\pgfqpoint{3.179398in}{3.494653in}}%
\pgfpathcurveto{\pgfqpoint{3.168348in}{3.494653in}}{\pgfqpoint{3.157749in}{3.490263in}}{\pgfqpoint{3.149935in}{3.482450in}}%
\pgfpathcurveto{\pgfqpoint{3.142122in}{3.474636in}}{\pgfqpoint{3.137731in}{3.464037in}}{\pgfqpoint{3.137731in}{3.452987in}}%
\pgfpathcurveto{\pgfqpoint{3.137731in}{3.441937in}}{\pgfqpoint{3.142122in}{3.431338in}}{\pgfqpoint{3.149935in}{3.423524in}}%
\pgfpathcurveto{\pgfqpoint{3.157749in}{3.415710in}}{\pgfqpoint{3.168348in}{3.411320in}}{\pgfqpoint{3.179398in}{3.411320in}}%
\pgfpathclose%
\pgfusepath{stroke,fill}%
\end{pgfscope}%
\begin{pgfscope}%
\pgfpathrectangle{\pgfqpoint{0.526127in}{0.331635in}}{\pgfqpoint{9.300000in}{7.700000in}}%
\pgfusepath{clip}%
\pgfsetbuttcap%
\pgfsetroundjoin%
\definecolor{currentfill}{rgb}{0.721569,0.521569,0.039216}%
\pgfsetfillcolor{currentfill}%
\pgfsetlinewidth{0.481800pt}%
\definecolor{currentstroke}{rgb}{1.000000,1.000000,1.000000}%
\pgfsetstrokecolor{currentstroke}%
\pgfsetdash{}{0pt}%
\pgfpathmoveto{\pgfqpoint{4.240309in}{7.078791in}}%
\pgfpathcurveto{\pgfqpoint{4.251359in}{7.078791in}}{\pgfqpoint{4.261958in}{7.083182in}}{\pgfqpoint{4.269772in}{7.090995in}}%
\pgfpathcurveto{\pgfqpoint{4.277585in}{7.098809in}}{\pgfqpoint{4.281976in}{7.109408in}}{\pgfqpoint{4.281976in}{7.120458in}}%
\pgfpathcurveto{\pgfqpoint{4.281976in}{7.131508in}}{\pgfqpoint{4.277585in}{7.142107in}}{\pgfqpoint{4.269772in}{7.149921in}}%
\pgfpathcurveto{\pgfqpoint{4.261958in}{7.157734in}}{\pgfqpoint{4.251359in}{7.162125in}}{\pgfqpoint{4.240309in}{7.162125in}}%
\pgfpathcurveto{\pgfqpoint{4.229259in}{7.162125in}}{\pgfqpoint{4.218660in}{7.157734in}}{\pgfqpoint{4.210846in}{7.149921in}}%
\pgfpathcurveto{\pgfqpoint{4.203033in}{7.142107in}}{\pgfqpoint{4.198642in}{7.131508in}}{\pgfqpoint{4.198642in}{7.120458in}}%
\pgfpathcurveto{\pgfqpoint{4.198642in}{7.109408in}}{\pgfqpoint{4.203033in}{7.098809in}}{\pgfqpoint{4.210846in}{7.090995in}}%
\pgfpathcurveto{\pgfqpoint{4.218660in}{7.083182in}}{\pgfqpoint{4.229259in}{7.078791in}}{\pgfqpoint{4.240309in}{7.078791in}}%
\pgfpathclose%
\pgfusepath{stroke,fill}%
\end{pgfscope}%
\begin{pgfscope}%
\pgfpathrectangle{\pgfqpoint{0.526127in}{0.331635in}}{\pgfqpoint{9.300000in}{7.700000in}}%
\pgfusepath{clip}%
\pgfsetbuttcap%
\pgfsetroundjoin%
\definecolor{currentfill}{rgb}{0.721569,0.521569,0.039216}%
\pgfsetfillcolor{currentfill}%
\pgfsetlinewidth{0.481800pt}%
\definecolor{currentstroke}{rgb}{1.000000,1.000000,1.000000}%
\pgfsetstrokecolor{currentstroke}%
\pgfsetdash{}{0pt}%
\pgfpathmoveto{\pgfqpoint{5.395356in}{4.851976in}}%
\pgfpathcurveto{\pgfqpoint{5.406406in}{4.851976in}}{\pgfqpoint{5.417005in}{4.856367in}}{\pgfqpoint{5.424819in}{4.864180in}}%
\pgfpathcurveto{\pgfqpoint{5.432632in}{4.871994in}}{\pgfqpoint{5.437023in}{4.882593in}}{\pgfqpoint{5.437023in}{4.893643in}}%
\pgfpathcurveto{\pgfqpoint{5.437023in}{4.904693in}}{\pgfqpoint{5.432632in}{4.915292in}}{\pgfqpoint{5.424819in}{4.923106in}}%
\pgfpathcurveto{\pgfqpoint{5.417005in}{4.930919in}}{\pgfqpoint{5.406406in}{4.935310in}}{\pgfqpoint{5.395356in}{4.935310in}}%
\pgfpathcurveto{\pgfqpoint{5.384306in}{4.935310in}}{\pgfqpoint{5.373707in}{4.930919in}}{\pgfqpoint{5.365893in}{4.923106in}}%
\pgfpathcurveto{\pgfqpoint{5.358080in}{4.915292in}}{\pgfqpoint{5.353689in}{4.904693in}}{\pgfqpoint{5.353689in}{4.893643in}}%
\pgfpathcurveto{\pgfqpoint{5.353689in}{4.882593in}}{\pgfqpoint{5.358080in}{4.871994in}}{\pgfqpoint{5.365893in}{4.864180in}}%
\pgfpathcurveto{\pgfqpoint{5.373707in}{4.856367in}}{\pgfqpoint{5.384306in}{4.851976in}}{\pgfqpoint{5.395356in}{4.851976in}}%
\pgfpathclose%
\pgfusepath{stroke,fill}%
\end{pgfscope}%
\begin{pgfscope}%
\pgfpathrectangle{\pgfqpoint{0.526127in}{0.331635in}}{\pgfqpoint{9.300000in}{7.700000in}}%
\pgfusepath{clip}%
\pgfsetbuttcap%
\pgfsetroundjoin%
\definecolor{currentfill}{rgb}{0.721569,0.521569,0.039216}%
\pgfsetfillcolor{currentfill}%
\pgfsetlinewidth{0.481800pt}%
\definecolor{currentstroke}{rgb}{1.000000,1.000000,1.000000}%
\pgfsetstrokecolor{currentstroke}%
\pgfsetdash{}{0pt}%
\pgfpathmoveto{\pgfqpoint{4.154359in}{5.088748in}}%
\pgfpathcurveto{\pgfqpoint{4.165409in}{5.088748in}}{\pgfqpoint{4.176008in}{5.093138in}}{\pgfqpoint{4.183822in}{5.100952in}}%
\pgfpathcurveto{\pgfqpoint{4.191636in}{5.108766in}}{\pgfqpoint{4.196026in}{5.119365in}}{\pgfqpoint{4.196026in}{5.130415in}}%
\pgfpathcurveto{\pgfqpoint{4.196026in}{5.141465in}}{\pgfqpoint{4.191636in}{5.152064in}}{\pgfqpoint{4.183822in}{5.159878in}}%
\pgfpathcurveto{\pgfqpoint{4.176008in}{5.167691in}}{\pgfqpoint{4.165409in}{5.172081in}}{\pgfqpoint{4.154359in}{5.172081in}}%
\pgfpathcurveto{\pgfqpoint{4.143309in}{5.172081in}}{\pgfqpoint{4.132710in}{5.167691in}}{\pgfqpoint{4.124896in}{5.159878in}}%
\pgfpathcurveto{\pgfqpoint{4.117083in}{5.152064in}}{\pgfqpoint{4.112693in}{5.141465in}}{\pgfqpoint{4.112693in}{5.130415in}}%
\pgfpathcurveto{\pgfqpoint{4.112693in}{5.119365in}}{\pgfqpoint{4.117083in}{5.108766in}}{\pgfqpoint{4.124896in}{5.100952in}}%
\pgfpathcurveto{\pgfqpoint{4.132710in}{5.093138in}}{\pgfqpoint{4.143309in}{5.088748in}}{\pgfqpoint{4.154359in}{5.088748in}}%
\pgfpathclose%
\pgfusepath{stroke,fill}%
\end{pgfscope}%
\begin{pgfscope}%
\pgfpathrectangle{\pgfqpoint{0.526127in}{0.331635in}}{\pgfqpoint{9.300000in}{7.700000in}}%
\pgfusepath{clip}%
\pgfsetbuttcap%
\pgfsetroundjoin%
\definecolor{currentfill}{rgb}{0.721569,0.521569,0.039216}%
\pgfsetfillcolor{currentfill}%
\pgfsetlinewidth{0.481800pt}%
\definecolor{currentstroke}{rgb}{1.000000,1.000000,1.000000}%
\pgfsetstrokecolor{currentstroke}%
\pgfsetdash{}{0pt}%
\pgfpathmoveto{\pgfqpoint{3.012801in}{4.435592in}}%
\pgfpathcurveto{\pgfqpoint{3.023851in}{4.435592in}}{\pgfqpoint{3.034450in}{4.439982in}}{\pgfqpoint{3.042264in}{4.447795in}}%
\pgfpathcurveto{\pgfqpoint{3.050078in}{4.455609in}}{\pgfqpoint{3.054468in}{4.466208in}}{\pgfqpoint{3.054468in}{4.477258in}}%
\pgfpathcurveto{\pgfqpoint{3.054468in}{4.488308in}}{\pgfqpoint{3.050078in}{4.498907in}}{\pgfqpoint{3.042264in}{4.506721in}}%
\pgfpathcurveto{\pgfqpoint{3.034450in}{4.514535in}}{\pgfqpoint{3.023851in}{4.518925in}}{\pgfqpoint{3.012801in}{4.518925in}}%
\pgfpathcurveto{\pgfqpoint{3.001751in}{4.518925in}}{\pgfqpoint{2.991152in}{4.514535in}}{\pgfqpoint{2.983338in}{4.506721in}}%
\pgfpathcurveto{\pgfqpoint{2.975525in}{4.498907in}}{\pgfqpoint{2.971134in}{4.488308in}}{\pgfqpoint{2.971134in}{4.477258in}}%
\pgfpathcurveto{\pgfqpoint{2.971134in}{4.466208in}}{\pgfqpoint{2.975525in}{4.455609in}}{\pgfqpoint{2.983338in}{4.447795in}}%
\pgfpathcurveto{\pgfqpoint{2.991152in}{4.439982in}}{\pgfqpoint{3.001751in}{4.435592in}}{\pgfqpoint{3.012801in}{4.435592in}}%
\pgfpathclose%
\pgfusepath{stroke,fill}%
\end{pgfscope}%
\begin{pgfscope}%
\pgfpathrectangle{\pgfqpoint{0.526127in}{0.331635in}}{\pgfqpoint{9.300000in}{7.700000in}}%
\pgfusepath{clip}%
\pgfsetbuttcap%
\pgfsetroundjoin%
\definecolor{currentfill}{rgb}{0.721569,0.521569,0.039216}%
\pgfsetfillcolor{currentfill}%
\pgfsetlinewidth{0.481800pt}%
\definecolor{currentstroke}{rgb}{1.000000,1.000000,1.000000}%
\pgfsetstrokecolor{currentstroke}%
\pgfsetdash{}{0pt}%
\pgfpathmoveto{\pgfqpoint{6.132477in}{0.970105in}}%
\pgfpathcurveto{\pgfqpoint{6.143527in}{0.970105in}}{\pgfqpoint{6.154126in}{0.974495in}}{\pgfqpoint{6.161940in}{0.982309in}}%
\pgfpathcurveto{\pgfqpoint{6.169754in}{0.990123in}}{\pgfqpoint{6.174144in}{1.000722in}}{\pgfqpoint{6.174144in}{1.011772in}}%
\pgfpathcurveto{\pgfqpoint{6.174144in}{1.022822in}}{\pgfqpoint{6.169754in}{1.033421in}}{\pgfqpoint{6.161940in}{1.041235in}}%
\pgfpathcurveto{\pgfqpoint{6.154126in}{1.049048in}}{\pgfqpoint{6.143527in}{1.053438in}}{\pgfqpoint{6.132477in}{1.053438in}}%
\pgfpathcurveto{\pgfqpoint{6.121427in}{1.053438in}}{\pgfqpoint{6.110828in}{1.049048in}}{\pgfqpoint{6.103015in}{1.041235in}}%
\pgfpathcurveto{\pgfqpoint{6.095201in}{1.033421in}}{\pgfqpoint{6.090811in}{1.022822in}}{\pgfqpoint{6.090811in}{1.011772in}}%
\pgfpathcurveto{\pgfqpoint{6.090811in}{1.000722in}}{\pgfqpoint{6.095201in}{0.990123in}}{\pgfqpoint{6.103015in}{0.982309in}}%
\pgfpathcurveto{\pgfqpoint{6.110828in}{0.974495in}}{\pgfqpoint{6.121427in}{0.970105in}}{\pgfqpoint{6.132477in}{0.970105in}}%
\pgfpathclose%
\pgfusepath{stroke,fill}%
\end{pgfscope}%
\begin{pgfscope}%
\pgfpathrectangle{\pgfqpoint{0.526127in}{0.331635in}}{\pgfqpoint{9.300000in}{7.700000in}}%
\pgfusepath{clip}%
\pgfsetbuttcap%
\pgfsetroundjoin%
\definecolor{currentfill}{rgb}{0.721569,0.521569,0.039216}%
\pgfsetfillcolor{currentfill}%
\pgfsetlinewidth{0.481800pt}%
\definecolor{currentstroke}{rgb}{1.000000,1.000000,1.000000}%
\pgfsetstrokecolor{currentstroke}%
\pgfsetdash{}{0pt}%
\pgfpathmoveto{\pgfqpoint{8.967963in}{2.858023in}}%
\pgfpathcurveto{\pgfqpoint{8.979013in}{2.858023in}}{\pgfqpoint{8.989612in}{2.862414in}}{\pgfqpoint{8.997426in}{2.870227in}}%
\pgfpathcurveto{\pgfqpoint{9.005239in}{2.878041in}}{\pgfqpoint{9.009629in}{2.888640in}}{\pgfqpoint{9.009629in}{2.899690in}}%
\pgfpathcurveto{\pgfqpoint{9.009629in}{2.910740in}}{\pgfqpoint{9.005239in}{2.921339in}}{\pgfqpoint{8.997426in}{2.929153in}}%
\pgfpathcurveto{\pgfqpoint{8.989612in}{2.936966in}}{\pgfqpoint{8.979013in}{2.941357in}}{\pgfqpoint{8.967963in}{2.941357in}}%
\pgfpathcurveto{\pgfqpoint{8.956913in}{2.941357in}}{\pgfqpoint{8.946314in}{2.936966in}}{\pgfqpoint{8.938500in}{2.929153in}}%
\pgfpathcurveto{\pgfqpoint{8.930686in}{2.921339in}}{\pgfqpoint{8.926296in}{2.910740in}}{\pgfqpoint{8.926296in}{2.899690in}}%
\pgfpathcurveto{\pgfqpoint{8.926296in}{2.888640in}}{\pgfqpoint{8.930686in}{2.878041in}}{\pgfqpoint{8.938500in}{2.870227in}}%
\pgfpathcurveto{\pgfqpoint{8.946314in}{2.862414in}}{\pgfqpoint{8.956913in}{2.858023in}}{\pgfqpoint{8.967963in}{2.858023in}}%
\pgfpathclose%
\pgfusepath{stroke,fill}%
\end{pgfscope}%
\begin{pgfscope}%
\pgfpathrectangle{\pgfqpoint{0.526127in}{0.331635in}}{\pgfqpoint{9.300000in}{7.700000in}}%
\pgfusepath{clip}%
\pgfsetbuttcap%
\pgfsetroundjoin%
\definecolor{currentfill}{rgb}{0.721569,0.521569,0.039216}%
\pgfsetfillcolor{currentfill}%
\pgfsetlinewidth{0.481800pt}%
\definecolor{currentstroke}{rgb}{1.000000,1.000000,1.000000}%
\pgfsetstrokecolor{currentstroke}%
\pgfsetdash{}{0pt}%
\pgfpathmoveto{\pgfqpoint{2.684948in}{3.897660in}}%
\pgfpathcurveto{\pgfqpoint{2.695998in}{3.897660in}}{\pgfqpoint{2.706597in}{3.902050in}}{\pgfqpoint{2.714411in}{3.909864in}}%
\pgfpathcurveto{\pgfqpoint{2.722225in}{3.917678in}}{\pgfqpoint{2.726615in}{3.928277in}}{\pgfqpoint{2.726615in}{3.939327in}}%
\pgfpathcurveto{\pgfqpoint{2.726615in}{3.950377in}}{\pgfqpoint{2.722225in}{3.960976in}}{\pgfqpoint{2.714411in}{3.968790in}}%
\pgfpathcurveto{\pgfqpoint{2.706597in}{3.976603in}}{\pgfqpoint{2.695998in}{3.980993in}}{\pgfqpoint{2.684948in}{3.980993in}}%
\pgfpathcurveto{\pgfqpoint{2.673898in}{3.980993in}}{\pgfqpoint{2.663299in}{3.976603in}}{\pgfqpoint{2.655485in}{3.968790in}}%
\pgfpathcurveto{\pgfqpoint{2.647672in}{3.960976in}}{\pgfqpoint{2.643281in}{3.950377in}}{\pgfqpoint{2.643281in}{3.939327in}}%
\pgfpathcurveto{\pgfqpoint{2.643281in}{3.928277in}}{\pgfqpoint{2.647672in}{3.917678in}}{\pgfqpoint{2.655485in}{3.909864in}}%
\pgfpathcurveto{\pgfqpoint{2.663299in}{3.902050in}}{\pgfqpoint{2.673898in}{3.897660in}}{\pgfqpoint{2.684948in}{3.897660in}}%
\pgfpathclose%
\pgfusepath{stroke,fill}%
\end{pgfscope}%
\begin{pgfscope}%
\pgfpathrectangle{\pgfqpoint{0.526127in}{0.331635in}}{\pgfqpoint{9.300000in}{7.700000in}}%
\pgfusepath{clip}%
\pgfsetbuttcap%
\pgfsetroundjoin%
\definecolor{currentfill}{rgb}{0.721569,0.521569,0.039216}%
\pgfsetfillcolor{currentfill}%
\pgfsetlinewidth{0.481800pt}%
\definecolor{currentstroke}{rgb}{1.000000,1.000000,1.000000}%
\pgfsetstrokecolor{currentstroke}%
\pgfsetdash{}{0pt}%
\pgfpathmoveto{\pgfqpoint{7.144222in}{6.603868in}}%
\pgfpathcurveto{\pgfqpoint{7.155272in}{6.603868in}}{\pgfqpoint{7.165871in}{6.608258in}}{\pgfqpoint{7.173685in}{6.616072in}}%
\pgfpathcurveto{\pgfqpoint{7.181499in}{6.623885in}}{\pgfqpoint{7.185889in}{6.634484in}}{\pgfqpoint{7.185889in}{6.645535in}}%
\pgfpathcurveto{\pgfqpoint{7.185889in}{6.656585in}}{\pgfqpoint{7.181499in}{6.667184in}}{\pgfqpoint{7.173685in}{6.674997in}}%
\pgfpathcurveto{\pgfqpoint{7.165871in}{6.682811in}}{\pgfqpoint{7.155272in}{6.687201in}}{\pgfqpoint{7.144222in}{6.687201in}}%
\pgfpathcurveto{\pgfqpoint{7.133172in}{6.687201in}}{\pgfqpoint{7.122573in}{6.682811in}}{\pgfqpoint{7.114759in}{6.674997in}}%
\pgfpathcurveto{\pgfqpoint{7.106946in}{6.667184in}}{\pgfqpoint{7.102556in}{6.656585in}}{\pgfqpoint{7.102556in}{6.645535in}}%
\pgfpathcurveto{\pgfqpoint{7.102556in}{6.634484in}}{\pgfqpoint{7.106946in}{6.623885in}}{\pgfqpoint{7.114759in}{6.616072in}}%
\pgfpathcurveto{\pgfqpoint{7.122573in}{6.608258in}}{\pgfqpoint{7.133172in}{6.603868in}}{\pgfqpoint{7.144222in}{6.603868in}}%
\pgfpathclose%
\pgfusepath{stroke,fill}%
\end{pgfscope}%
\begin{pgfscope}%
\pgfpathrectangle{\pgfqpoint{0.526127in}{0.331635in}}{\pgfqpoint{9.300000in}{7.700000in}}%
\pgfusepath{clip}%
\pgfsetbuttcap%
\pgfsetroundjoin%
\definecolor{currentfill}{rgb}{0.721569,0.521569,0.039216}%
\pgfsetfillcolor{currentfill}%
\pgfsetlinewidth{0.481800pt}%
\definecolor{currentstroke}{rgb}{1.000000,1.000000,1.000000}%
\pgfsetstrokecolor{currentstroke}%
\pgfsetdash{}{0pt}%
\pgfpathmoveto{\pgfqpoint{5.755207in}{1.704388in}}%
\pgfpathcurveto{\pgfqpoint{5.766257in}{1.704388in}}{\pgfqpoint{5.776856in}{1.708778in}}{\pgfqpoint{5.784670in}{1.716591in}}%
\pgfpathcurveto{\pgfqpoint{5.792484in}{1.724405in}}{\pgfqpoint{5.796874in}{1.735004in}}{\pgfqpoint{5.796874in}{1.746054in}}%
\pgfpathcurveto{\pgfqpoint{5.796874in}{1.757104in}}{\pgfqpoint{5.792484in}{1.767703in}}{\pgfqpoint{5.784670in}{1.775517in}}%
\pgfpathcurveto{\pgfqpoint{5.776856in}{1.783331in}}{\pgfqpoint{5.766257in}{1.787721in}}{\pgfqpoint{5.755207in}{1.787721in}}%
\pgfpathcurveto{\pgfqpoint{5.744157in}{1.787721in}}{\pgfqpoint{5.733558in}{1.783331in}}{\pgfqpoint{5.725744in}{1.775517in}}%
\pgfpathcurveto{\pgfqpoint{5.717931in}{1.767703in}}{\pgfqpoint{5.713541in}{1.757104in}}{\pgfqpoint{5.713541in}{1.746054in}}%
\pgfpathcurveto{\pgfqpoint{5.713541in}{1.735004in}}{\pgfqpoint{5.717931in}{1.724405in}}{\pgfqpoint{5.725744in}{1.716591in}}%
\pgfpathcurveto{\pgfqpoint{5.733558in}{1.708778in}}{\pgfqpoint{5.744157in}{1.704388in}}{\pgfqpoint{5.755207in}{1.704388in}}%
\pgfpathclose%
\pgfusepath{stroke,fill}%
\end{pgfscope}%
\begin{pgfscope}%
\pgfpathrectangle{\pgfqpoint{0.526127in}{0.331635in}}{\pgfqpoint{9.300000in}{7.700000in}}%
\pgfusepath{clip}%
\pgfsetbuttcap%
\pgfsetroundjoin%
\definecolor{currentfill}{rgb}{0.721569,0.521569,0.039216}%
\pgfsetfillcolor{currentfill}%
\pgfsetlinewidth{0.481800pt}%
\definecolor{currentstroke}{rgb}{1.000000,1.000000,1.000000}%
\pgfsetstrokecolor{currentstroke}%
\pgfsetdash{}{0pt}%
\pgfpathmoveto{\pgfqpoint{9.079707in}{4.375025in}}%
\pgfpathcurveto{\pgfqpoint{9.090757in}{4.375025in}}{\pgfqpoint{9.101356in}{4.379415in}}{\pgfqpoint{9.109170in}{4.387229in}}%
\pgfpathcurveto{\pgfqpoint{9.116984in}{4.395042in}}{\pgfqpoint{9.121374in}{4.405641in}}{\pgfqpoint{9.121374in}{4.416691in}}%
\pgfpathcurveto{\pgfqpoint{9.121374in}{4.427741in}}{\pgfqpoint{9.116984in}{4.438341in}}{\pgfqpoint{9.109170in}{4.446154in}}%
\pgfpathcurveto{\pgfqpoint{9.101356in}{4.453968in}}{\pgfqpoint{9.090757in}{4.458358in}}{\pgfqpoint{9.079707in}{4.458358in}}%
\pgfpathcurveto{\pgfqpoint{9.068657in}{4.458358in}}{\pgfqpoint{9.058058in}{4.453968in}}{\pgfqpoint{9.050244in}{4.446154in}}%
\pgfpathcurveto{\pgfqpoint{9.042431in}{4.438341in}}{\pgfqpoint{9.038040in}{4.427741in}}{\pgfqpoint{9.038040in}{4.416691in}}%
\pgfpathcurveto{\pgfqpoint{9.038040in}{4.405641in}}{\pgfqpoint{9.042431in}{4.395042in}}{\pgfqpoint{9.050244in}{4.387229in}}%
\pgfpathcurveto{\pgfqpoint{9.058058in}{4.379415in}}{\pgfqpoint{9.068657in}{4.375025in}}{\pgfqpoint{9.079707in}{4.375025in}}%
\pgfpathclose%
\pgfusepath{stroke,fill}%
\end{pgfscope}%
\begin{pgfscope}%
\pgfpathrectangle{\pgfqpoint{0.526127in}{0.331635in}}{\pgfqpoint{9.300000in}{7.700000in}}%
\pgfusepath{clip}%
\pgfsetbuttcap%
\pgfsetroundjoin%
\definecolor{currentfill}{rgb}{0.721569,0.521569,0.039216}%
\pgfsetfillcolor{currentfill}%
\pgfsetlinewidth{0.481800pt}%
\definecolor{currentstroke}{rgb}{1.000000,1.000000,1.000000}%
\pgfsetstrokecolor{currentstroke}%
\pgfsetdash{}{0pt}%
\pgfpathmoveto{\pgfqpoint{4.104581in}{1.271214in}}%
\pgfpathcurveto{\pgfqpoint{4.115632in}{1.271214in}}{\pgfqpoint{4.126231in}{1.275604in}}{\pgfqpoint{4.134044in}{1.283418in}}%
\pgfpathcurveto{\pgfqpoint{4.141858in}{1.291231in}}{\pgfqpoint{4.146248in}{1.301831in}}{\pgfqpoint{4.146248in}{1.312881in}}%
\pgfpathcurveto{\pgfqpoint{4.146248in}{1.323931in}}{\pgfqpoint{4.141858in}{1.334530in}}{\pgfqpoint{4.134044in}{1.342343in}}%
\pgfpathcurveto{\pgfqpoint{4.126231in}{1.350157in}}{\pgfqpoint{4.115632in}{1.354547in}}{\pgfqpoint{4.104581in}{1.354547in}}%
\pgfpathcurveto{\pgfqpoint{4.093531in}{1.354547in}}{\pgfqpoint{4.082932in}{1.350157in}}{\pgfqpoint{4.075119in}{1.342343in}}%
\pgfpathcurveto{\pgfqpoint{4.067305in}{1.334530in}}{\pgfqpoint{4.062915in}{1.323931in}}{\pgfqpoint{4.062915in}{1.312881in}}%
\pgfpathcurveto{\pgfqpoint{4.062915in}{1.301831in}}{\pgfqpoint{4.067305in}{1.291231in}}{\pgfqpoint{4.075119in}{1.283418in}}%
\pgfpathcurveto{\pgfqpoint{4.082932in}{1.275604in}}{\pgfqpoint{4.093531in}{1.271214in}}{\pgfqpoint{4.104581in}{1.271214in}}%
\pgfpathclose%
\pgfusepath{stroke,fill}%
\end{pgfscope}%
\begin{pgfscope}%
\pgfpathrectangle{\pgfqpoint{0.526127in}{0.331635in}}{\pgfqpoint{9.300000in}{7.700000in}}%
\pgfusepath{clip}%
\pgfsetbuttcap%
\pgfsetroundjoin%
\definecolor{currentfill}{rgb}{0.721569,0.521569,0.039216}%
\pgfsetfillcolor{currentfill}%
\pgfsetlinewidth{0.481800pt}%
\definecolor{currentstroke}{rgb}{1.000000,1.000000,1.000000}%
\pgfsetstrokecolor{currentstroke}%
\pgfsetdash{}{0pt}%
\pgfpathmoveto{\pgfqpoint{1.641796in}{4.114808in}}%
\pgfpathcurveto{\pgfqpoint{1.652846in}{4.114808in}}{\pgfqpoint{1.663445in}{4.119198in}}{\pgfqpoint{1.671258in}{4.127012in}}%
\pgfpathcurveto{\pgfqpoint{1.679072in}{4.134825in}}{\pgfqpoint{1.683462in}{4.145424in}}{\pgfqpoint{1.683462in}{4.156474in}}%
\pgfpathcurveto{\pgfqpoint{1.683462in}{4.167525in}}{\pgfqpoint{1.679072in}{4.178124in}}{\pgfqpoint{1.671258in}{4.185937in}}%
\pgfpathcurveto{\pgfqpoint{1.663445in}{4.193751in}}{\pgfqpoint{1.652846in}{4.198141in}}{\pgfqpoint{1.641796in}{4.198141in}}%
\pgfpathcurveto{\pgfqpoint{1.630745in}{4.198141in}}{\pgfqpoint{1.620146in}{4.193751in}}{\pgfqpoint{1.612333in}{4.185937in}}%
\pgfpathcurveto{\pgfqpoint{1.604519in}{4.178124in}}{\pgfqpoint{1.600129in}{4.167525in}}{\pgfqpoint{1.600129in}{4.156474in}}%
\pgfpathcurveto{\pgfqpoint{1.600129in}{4.145424in}}{\pgfqpoint{1.604519in}{4.134825in}}{\pgfqpoint{1.612333in}{4.127012in}}%
\pgfpathcurveto{\pgfqpoint{1.620146in}{4.119198in}}{\pgfqpoint{1.630745in}{4.114808in}}{\pgfqpoint{1.641796in}{4.114808in}}%
\pgfpathclose%
\pgfusepath{stroke,fill}%
\end{pgfscope}%
\begin{pgfscope}%
\pgfpathrectangle{\pgfqpoint{0.526127in}{0.331635in}}{\pgfqpoint{9.300000in}{7.700000in}}%
\pgfusepath{clip}%
\pgfsetbuttcap%
\pgfsetroundjoin%
\definecolor{currentfill}{rgb}{0.721569,0.521569,0.039216}%
\pgfsetfillcolor{currentfill}%
\pgfsetlinewidth{0.481800pt}%
\definecolor{currentstroke}{rgb}{1.000000,1.000000,1.000000}%
\pgfsetstrokecolor{currentstroke}%
\pgfsetdash{}{0pt}%
\pgfpathmoveto{\pgfqpoint{2.312627in}{5.957100in}}%
\pgfpathcurveto{\pgfqpoint{2.323677in}{5.957100in}}{\pgfqpoint{2.334276in}{5.961490in}}{\pgfqpoint{2.342089in}{5.969304in}}%
\pgfpathcurveto{\pgfqpoint{2.349903in}{5.977117in}}{\pgfqpoint{2.354293in}{5.987717in}}{\pgfqpoint{2.354293in}{5.998767in}}%
\pgfpathcurveto{\pgfqpoint{2.354293in}{6.009817in}}{\pgfqpoint{2.349903in}{6.020416in}}{\pgfqpoint{2.342089in}{6.028229in}}%
\pgfpathcurveto{\pgfqpoint{2.334276in}{6.036043in}}{\pgfqpoint{2.323677in}{6.040433in}}{\pgfqpoint{2.312627in}{6.040433in}}%
\pgfpathcurveto{\pgfqpoint{2.301576in}{6.040433in}}{\pgfqpoint{2.290977in}{6.036043in}}{\pgfqpoint{2.283164in}{6.028229in}}%
\pgfpathcurveto{\pgfqpoint{2.275350in}{6.020416in}}{\pgfqpoint{2.270960in}{6.009817in}}{\pgfqpoint{2.270960in}{5.998767in}}%
\pgfpathcurveto{\pgfqpoint{2.270960in}{5.987717in}}{\pgfqpoint{2.275350in}{5.977117in}}{\pgfqpoint{2.283164in}{5.969304in}}%
\pgfpathcurveto{\pgfqpoint{2.290977in}{5.961490in}}{\pgfqpoint{2.301576in}{5.957100in}}{\pgfqpoint{2.312627in}{5.957100in}}%
\pgfpathclose%
\pgfusepath{stroke,fill}%
\end{pgfscope}%
\begin{pgfscope}%
\pgfpathrectangle{\pgfqpoint{0.526127in}{0.331635in}}{\pgfqpoint{9.300000in}{7.700000in}}%
\pgfusepath{clip}%
\pgfsetbuttcap%
\pgfsetroundjoin%
\definecolor{currentfill}{rgb}{0.721569,0.521569,0.039216}%
\pgfsetfillcolor{currentfill}%
\pgfsetlinewidth{0.481800pt}%
\definecolor{currentstroke}{rgb}{1.000000,1.000000,1.000000}%
\pgfsetstrokecolor{currentstroke}%
\pgfsetdash{}{0pt}%
\pgfpathmoveto{\pgfqpoint{7.299025in}{5.711229in}}%
\pgfpathcurveto{\pgfqpoint{7.310075in}{5.711229in}}{\pgfqpoint{7.320674in}{5.715620in}}{\pgfqpoint{7.328487in}{5.723433in}}%
\pgfpathcurveto{\pgfqpoint{7.336301in}{5.731247in}}{\pgfqpoint{7.340691in}{5.741846in}}{\pgfqpoint{7.340691in}{5.752896in}}%
\pgfpathcurveto{\pgfqpoint{7.340691in}{5.763946in}}{\pgfqpoint{7.336301in}{5.774545in}}{\pgfqpoint{7.328487in}{5.782359in}}%
\pgfpathcurveto{\pgfqpoint{7.320674in}{5.790172in}}{\pgfqpoint{7.310075in}{5.794563in}}{\pgfqpoint{7.299025in}{5.794563in}}%
\pgfpathcurveto{\pgfqpoint{7.287975in}{5.794563in}}{\pgfqpoint{7.277375in}{5.790172in}}{\pgfqpoint{7.269562in}{5.782359in}}%
\pgfpathcurveto{\pgfqpoint{7.261748in}{5.774545in}}{\pgfqpoint{7.257358in}{5.763946in}}{\pgfqpoint{7.257358in}{5.752896in}}%
\pgfpathcurveto{\pgfqpoint{7.257358in}{5.741846in}}{\pgfqpoint{7.261748in}{5.731247in}}{\pgfqpoint{7.269562in}{5.723433in}}%
\pgfpathcurveto{\pgfqpoint{7.277375in}{5.715620in}}{\pgfqpoint{7.287975in}{5.711229in}}{\pgfqpoint{7.299025in}{5.711229in}}%
\pgfpathclose%
\pgfusepath{stroke,fill}%
\end{pgfscope}%
\begin{pgfscope}%
\pgfpathrectangle{\pgfqpoint{0.526127in}{0.331635in}}{\pgfqpoint{9.300000in}{7.700000in}}%
\pgfusepath{clip}%
\pgfsetbuttcap%
\pgfsetroundjoin%
\definecolor{currentfill}{rgb}{0.721569,0.521569,0.039216}%
\pgfsetfillcolor{currentfill}%
\pgfsetlinewidth{0.481800pt}%
\definecolor{currentstroke}{rgb}{1.000000,1.000000,1.000000}%
\pgfsetstrokecolor{currentstroke}%
\pgfsetdash{}{0pt}%
\pgfpathmoveto{\pgfqpoint{1.663441in}{4.064401in}}%
\pgfpathcurveto{\pgfqpoint{1.674491in}{4.064401in}}{\pgfqpoint{1.685090in}{4.068791in}}{\pgfqpoint{1.692904in}{4.076605in}}%
\pgfpathcurveto{\pgfqpoint{1.700717in}{4.084419in}}{\pgfqpoint{1.705108in}{4.095018in}}{\pgfqpoint{1.705108in}{4.106068in}}%
\pgfpathcurveto{\pgfqpoint{1.705108in}{4.117118in}}{\pgfqpoint{1.700717in}{4.127717in}}{\pgfqpoint{1.692904in}{4.135531in}}%
\pgfpathcurveto{\pgfqpoint{1.685090in}{4.143344in}}{\pgfqpoint{1.674491in}{4.147734in}}{\pgfqpoint{1.663441in}{4.147734in}}%
\pgfpathcurveto{\pgfqpoint{1.652391in}{4.147734in}}{\pgfqpoint{1.641792in}{4.143344in}}{\pgfqpoint{1.633978in}{4.135531in}}%
\pgfpathcurveto{\pgfqpoint{1.626165in}{4.127717in}}{\pgfqpoint{1.621774in}{4.117118in}}{\pgfqpoint{1.621774in}{4.106068in}}%
\pgfpathcurveto{\pgfqpoint{1.621774in}{4.095018in}}{\pgfqpoint{1.626165in}{4.084419in}}{\pgfqpoint{1.633978in}{4.076605in}}%
\pgfpathcurveto{\pgfqpoint{1.641792in}{4.068791in}}{\pgfqpoint{1.652391in}{4.064401in}}{\pgfqpoint{1.663441in}{4.064401in}}%
\pgfpathclose%
\pgfusepath{stroke,fill}%
\end{pgfscope}%
\begin{pgfscope}%
\pgfpathrectangle{\pgfqpoint{0.526127in}{0.331635in}}{\pgfqpoint{9.300000in}{7.700000in}}%
\pgfusepath{clip}%
\pgfsetbuttcap%
\pgfsetroundjoin%
\definecolor{currentfill}{rgb}{0.721569,0.521569,0.039216}%
\pgfsetfillcolor{currentfill}%
\pgfsetlinewidth{0.481800pt}%
\definecolor{currentstroke}{rgb}{1.000000,1.000000,1.000000}%
\pgfsetstrokecolor{currentstroke}%
\pgfsetdash{}{0pt}%
\pgfpathmoveto{\pgfqpoint{9.274981in}{5.517114in}}%
\pgfpathcurveto{\pgfqpoint{9.286031in}{5.517114in}}{\pgfqpoint{9.296630in}{5.521504in}}{\pgfqpoint{9.304444in}{5.529318in}}%
\pgfpathcurveto{\pgfqpoint{9.312257in}{5.537132in}}{\pgfqpoint{9.316647in}{5.547731in}}{\pgfqpoint{9.316647in}{5.558781in}}%
\pgfpathcurveto{\pgfqpoint{9.316647in}{5.569831in}}{\pgfqpoint{9.312257in}{5.580430in}}{\pgfqpoint{9.304444in}{5.588244in}}%
\pgfpathcurveto{\pgfqpoint{9.296630in}{5.596057in}}{\pgfqpoint{9.286031in}{5.600448in}}{\pgfqpoint{9.274981in}{5.600448in}}%
\pgfpathcurveto{\pgfqpoint{9.263931in}{5.600448in}}{\pgfqpoint{9.253332in}{5.596057in}}{\pgfqpoint{9.245518in}{5.588244in}}%
\pgfpathcurveto{\pgfqpoint{9.237704in}{5.580430in}}{\pgfqpoint{9.233314in}{5.569831in}}{\pgfqpoint{9.233314in}{5.558781in}}%
\pgfpathcurveto{\pgfqpoint{9.233314in}{5.547731in}}{\pgfqpoint{9.237704in}{5.537132in}}{\pgfqpoint{9.245518in}{5.529318in}}%
\pgfpathcurveto{\pgfqpoint{9.253332in}{5.521504in}}{\pgfqpoint{9.263931in}{5.517114in}}{\pgfqpoint{9.274981in}{5.517114in}}%
\pgfpathclose%
\pgfusepath{stroke,fill}%
\end{pgfscope}%
\begin{pgfscope}%
\pgfpathrectangle{\pgfqpoint{0.526127in}{0.331635in}}{\pgfqpoint{9.300000in}{7.700000in}}%
\pgfusepath{clip}%
\pgfsetbuttcap%
\pgfsetroundjoin%
\definecolor{currentfill}{rgb}{0.721569,0.521569,0.039216}%
\pgfsetfillcolor{currentfill}%
\pgfsetlinewidth{0.481800pt}%
\definecolor{currentstroke}{rgb}{1.000000,1.000000,1.000000}%
\pgfsetstrokecolor{currentstroke}%
\pgfsetdash{}{0pt}%
\pgfpathmoveto{\pgfqpoint{9.099248in}{4.648975in}}%
\pgfpathcurveto{\pgfqpoint{9.110298in}{4.648975in}}{\pgfqpoint{9.120897in}{4.653365in}}{\pgfqpoint{9.128711in}{4.661179in}}%
\pgfpathcurveto{\pgfqpoint{9.136524in}{4.668992in}}{\pgfqpoint{9.140915in}{4.679591in}}{\pgfqpoint{9.140915in}{4.690642in}}%
\pgfpathcurveto{\pgfqpoint{9.140915in}{4.701692in}}{\pgfqpoint{9.136524in}{4.712291in}}{\pgfqpoint{9.128711in}{4.720104in}}%
\pgfpathcurveto{\pgfqpoint{9.120897in}{4.727918in}}{\pgfqpoint{9.110298in}{4.732308in}}{\pgfqpoint{9.099248in}{4.732308in}}%
\pgfpathcurveto{\pgfqpoint{9.088198in}{4.732308in}}{\pgfqpoint{9.077599in}{4.727918in}}{\pgfqpoint{9.069785in}{4.720104in}}%
\pgfpathcurveto{\pgfqpoint{9.061972in}{4.712291in}}{\pgfqpoint{9.057581in}{4.701692in}}{\pgfqpoint{9.057581in}{4.690642in}}%
\pgfpathcurveto{\pgfqpoint{9.057581in}{4.679591in}}{\pgfqpoint{9.061972in}{4.668992in}}{\pgfqpoint{9.069785in}{4.661179in}}%
\pgfpathcurveto{\pgfqpoint{9.077599in}{4.653365in}}{\pgfqpoint{9.088198in}{4.648975in}}{\pgfqpoint{9.099248in}{4.648975in}}%
\pgfpathclose%
\pgfusepath{stroke,fill}%
\end{pgfscope}%
\begin{pgfscope}%
\pgfpathrectangle{\pgfqpoint{0.526127in}{0.331635in}}{\pgfqpoint{9.300000in}{7.700000in}}%
\pgfusepath{clip}%
\pgfsetbuttcap%
\pgfsetroundjoin%
\definecolor{currentfill}{rgb}{0.721569,0.521569,0.039216}%
\pgfsetfillcolor{currentfill}%
\pgfsetlinewidth{0.481800pt}%
\definecolor{currentstroke}{rgb}{1.000000,1.000000,1.000000}%
\pgfsetstrokecolor{currentstroke}%
\pgfsetdash{}{0pt}%
\pgfpathmoveto{\pgfqpoint{4.144255in}{5.844477in}}%
\pgfpathcurveto{\pgfqpoint{4.155306in}{5.844477in}}{\pgfqpoint{4.165905in}{5.848867in}}{\pgfqpoint{4.173718in}{5.856681in}}%
\pgfpathcurveto{\pgfqpoint{4.181532in}{5.864494in}}{\pgfqpoint{4.185922in}{5.875093in}}{\pgfqpoint{4.185922in}{5.886144in}}%
\pgfpathcurveto{\pgfqpoint{4.185922in}{5.897194in}}{\pgfqpoint{4.181532in}{5.907793in}}{\pgfqpoint{4.173718in}{5.915606in}}%
\pgfpathcurveto{\pgfqpoint{4.165905in}{5.923420in}}{\pgfqpoint{4.155306in}{5.927810in}}{\pgfqpoint{4.144255in}{5.927810in}}%
\pgfpathcurveto{\pgfqpoint{4.133205in}{5.927810in}}{\pgfqpoint{4.122606in}{5.923420in}}{\pgfqpoint{4.114793in}{5.915606in}}%
\pgfpathcurveto{\pgfqpoint{4.106979in}{5.907793in}}{\pgfqpoint{4.102589in}{5.897194in}}{\pgfqpoint{4.102589in}{5.886144in}}%
\pgfpathcurveto{\pgfqpoint{4.102589in}{5.875093in}}{\pgfqpoint{4.106979in}{5.864494in}}{\pgfqpoint{4.114793in}{5.856681in}}%
\pgfpathcurveto{\pgfqpoint{4.122606in}{5.848867in}}{\pgfqpoint{4.133205in}{5.844477in}}{\pgfqpoint{4.144255in}{5.844477in}}%
\pgfpathclose%
\pgfusepath{stroke,fill}%
\end{pgfscope}%
\begin{pgfscope}%
\pgfpathrectangle{\pgfqpoint{0.526127in}{0.331635in}}{\pgfqpoint{9.300000in}{7.700000in}}%
\pgfusepath{clip}%
\pgfsetbuttcap%
\pgfsetroundjoin%
\definecolor{currentfill}{rgb}{0.721569,0.521569,0.039216}%
\pgfsetfillcolor{currentfill}%
\pgfsetlinewidth{0.481800pt}%
\definecolor{currentstroke}{rgb}{1.000000,1.000000,1.000000}%
\pgfsetstrokecolor{currentstroke}%
\pgfsetdash{}{0pt}%
\pgfpathmoveto{\pgfqpoint{6.389607in}{4.472796in}}%
\pgfpathcurveto{\pgfqpoint{6.400657in}{4.472796in}}{\pgfqpoint{6.411256in}{4.477187in}}{\pgfqpoint{6.419069in}{4.485000in}}%
\pgfpathcurveto{\pgfqpoint{6.426883in}{4.492814in}}{\pgfqpoint{6.431273in}{4.503413in}}{\pgfqpoint{6.431273in}{4.514463in}}%
\pgfpathcurveto{\pgfqpoint{6.431273in}{4.525513in}}{\pgfqpoint{6.426883in}{4.536112in}}{\pgfqpoint{6.419069in}{4.543926in}}%
\pgfpathcurveto{\pgfqpoint{6.411256in}{4.551739in}}{\pgfqpoint{6.400657in}{4.556130in}}{\pgfqpoint{6.389607in}{4.556130in}}%
\pgfpathcurveto{\pgfqpoint{6.378556in}{4.556130in}}{\pgfqpoint{6.367957in}{4.551739in}}{\pgfqpoint{6.360144in}{4.543926in}}%
\pgfpathcurveto{\pgfqpoint{6.352330in}{4.536112in}}{\pgfqpoint{6.347940in}{4.525513in}}{\pgfqpoint{6.347940in}{4.514463in}}%
\pgfpathcurveto{\pgfqpoint{6.347940in}{4.503413in}}{\pgfqpoint{6.352330in}{4.492814in}}{\pgfqpoint{6.360144in}{4.485000in}}%
\pgfpathcurveto{\pgfqpoint{6.367957in}{4.477187in}}{\pgfqpoint{6.378556in}{4.472796in}}{\pgfqpoint{6.389607in}{4.472796in}}%
\pgfpathclose%
\pgfusepath{stroke,fill}%
\end{pgfscope}%
\begin{pgfscope}%
\pgfpathrectangle{\pgfqpoint{0.526127in}{0.331635in}}{\pgfqpoint{9.300000in}{7.700000in}}%
\pgfusepath{clip}%
\pgfsetbuttcap%
\pgfsetroundjoin%
\definecolor{currentfill}{rgb}{0.721569,0.521569,0.039216}%
\pgfsetfillcolor{currentfill}%
\pgfsetlinewidth{0.481800pt}%
\definecolor{currentstroke}{rgb}{1.000000,1.000000,1.000000}%
\pgfsetstrokecolor{currentstroke}%
\pgfsetdash{}{0pt}%
\pgfpathmoveto{\pgfqpoint{3.057300in}{5.078843in}}%
\pgfpathcurveto{\pgfqpoint{3.068351in}{5.078843in}}{\pgfqpoint{3.078950in}{5.083233in}}{\pgfqpoint{3.086763in}{5.091046in}}%
\pgfpathcurveto{\pgfqpoint{3.094577in}{5.098860in}}{\pgfqpoint{3.098967in}{5.109459in}}{\pgfqpoint{3.098967in}{5.120509in}}%
\pgfpathcurveto{\pgfqpoint{3.098967in}{5.131559in}}{\pgfqpoint{3.094577in}{5.142158in}}{\pgfqpoint{3.086763in}{5.149972in}}%
\pgfpathcurveto{\pgfqpoint{3.078950in}{5.157786in}}{\pgfqpoint{3.068351in}{5.162176in}}{\pgfqpoint{3.057300in}{5.162176in}}%
\pgfpathcurveto{\pgfqpoint{3.046250in}{5.162176in}}{\pgfqpoint{3.035651in}{5.157786in}}{\pgfqpoint{3.027838in}{5.149972in}}%
\pgfpathcurveto{\pgfqpoint{3.020024in}{5.142158in}}{\pgfqpoint{3.015634in}{5.131559in}}{\pgfqpoint{3.015634in}{5.120509in}}%
\pgfpathcurveto{\pgfqpoint{3.015634in}{5.109459in}}{\pgfqpoint{3.020024in}{5.098860in}}{\pgfqpoint{3.027838in}{5.091046in}}%
\pgfpathcurveto{\pgfqpoint{3.035651in}{5.083233in}}{\pgfqpoint{3.046250in}{5.078843in}}{\pgfqpoint{3.057300in}{5.078843in}}%
\pgfpathclose%
\pgfusepath{stroke,fill}%
\end{pgfscope}%
\begin{pgfscope}%
\pgfpathrectangle{\pgfqpoint{0.526127in}{0.331635in}}{\pgfqpoint{9.300000in}{7.700000in}}%
\pgfusepath{clip}%
\pgfsetbuttcap%
\pgfsetroundjoin%
\definecolor{currentfill}{rgb}{0.721569,0.521569,0.039216}%
\pgfsetfillcolor{currentfill}%
\pgfsetlinewidth{0.481800pt}%
\definecolor{currentstroke}{rgb}{1.000000,1.000000,1.000000}%
\pgfsetstrokecolor{currentstroke}%
\pgfsetdash{}{0pt}%
\pgfpathmoveto{\pgfqpoint{1.903215in}{4.884775in}}%
\pgfpathcurveto{\pgfqpoint{1.914265in}{4.884775in}}{\pgfqpoint{1.924864in}{4.889165in}}{\pgfqpoint{1.932678in}{4.896979in}}%
\pgfpathcurveto{\pgfqpoint{1.940492in}{4.904793in}}{\pgfqpoint{1.944882in}{4.915392in}}{\pgfqpoint{1.944882in}{4.926442in}}%
\pgfpathcurveto{\pgfqpoint{1.944882in}{4.937492in}}{\pgfqpoint{1.940492in}{4.948091in}}{\pgfqpoint{1.932678in}{4.955904in}}%
\pgfpathcurveto{\pgfqpoint{1.924864in}{4.963718in}}{\pgfqpoint{1.914265in}{4.968108in}}{\pgfqpoint{1.903215in}{4.968108in}}%
\pgfpathcurveto{\pgfqpoint{1.892165in}{4.968108in}}{\pgfqpoint{1.881566in}{4.963718in}}{\pgfqpoint{1.873752in}{4.955904in}}%
\pgfpathcurveto{\pgfqpoint{1.865939in}{4.948091in}}{\pgfqpoint{1.861549in}{4.937492in}}{\pgfqpoint{1.861549in}{4.926442in}}%
\pgfpathcurveto{\pgfqpoint{1.861549in}{4.915392in}}{\pgfqpoint{1.865939in}{4.904793in}}{\pgfqpoint{1.873752in}{4.896979in}}%
\pgfpathcurveto{\pgfqpoint{1.881566in}{4.889165in}}{\pgfqpoint{1.892165in}{4.884775in}}{\pgfqpoint{1.903215in}{4.884775in}}%
\pgfpathclose%
\pgfusepath{stroke,fill}%
\end{pgfscope}%
\begin{pgfscope}%
\pgfpathrectangle{\pgfqpoint{0.526127in}{0.331635in}}{\pgfqpoint{9.300000in}{7.700000in}}%
\pgfusepath{clip}%
\pgfsetbuttcap%
\pgfsetroundjoin%
\definecolor{currentfill}{rgb}{0.631373,0.788235,0.956863}%
\pgfsetfillcolor{currentfill}%
\pgfsetlinewidth{1.003750pt}%
\definecolor{currentstroke}{rgb}{0.631373,0.788235,0.956863}%
\pgfsetstrokecolor{currentstroke}%
\pgfsetdash{}{0pt}%
\pgfsys@defobject{currentmarker}{\pgfqpoint{-0.041667in}{-0.041667in}}{\pgfqpoint{0.041667in}{0.041667in}}{%
\pgfpathmoveto{\pgfqpoint{0.000000in}{-0.041667in}}%
\pgfpathcurveto{\pgfqpoint{0.011050in}{-0.041667in}}{\pgfqpoint{0.021649in}{-0.037276in}}{\pgfqpoint{0.029463in}{-0.029463in}}%
\pgfpathcurveto{\pgfqpoint{0.037276in}{-0.021649in}}{\pgfqpoint{0.041667in}{-0.011050in}}{\pgfqpoint{0.041667in}{0.000000in}}%
\pgfpathcurveto{\pgfqpoint{0.041667in}{0.011050in}}{\pgfqpoint{0.037276in}{0.021649in}}{\pgfqpoint{0.029463in}{0.029463in}}%
\pgfpathcurveto{\pgfqpoint{0.021649in}{0.037276in}}{\pgfqpoint{0.011050in}{0.041667in}}{\pgfqpoint{0.000000in}{0.041667in}}%
\pgfpathcurveto{\pgfqpoint{-0.011050in}{0.041667in}}{\pgfqpoint{-0.021649in}{0.037276in}}{\pgfqpoint{-0.029463in}{0.029463in}}%
\pgfpathcurveto{\pgfqpoint{-0.037276in}{0.021649in}}{\pgfqpoint{-0.041667in}{0.011050in}}{\pgfqpoint{-0.041667in}{0.000000in}}%
\pgfpathcurveto{\pgfqpoint{-0.041667in}{-0.011050in}}{\pgfqpoint{-0.037276in}{-0.021649in}}{\pgfqpoint{-0.029463in}{-0.029463in}}%
\pgfpathcurveto{\pgfqpoint{-0.021649in}{-0.037276in}}{\pgfqpoint{-0.011050in}{-0.041667in}}{\pgfqpoint{0.000000in}{-0.041667in}}%
\pgfpathclose%
\pgfusepath{stroke,fill}%
}%
\end{pgfscope}%
\begin{pgfscope}%
\pgfpathrectangle{\pgfqpoint{0.526127in}{0.331635in}}{\pgfqpoint{9.300000in}{7.700000in}}%
\pgfusepath{clip}%
\pgfsetbuttcap%
\pgfsetroundjoin%
\definecolor{currentfill}{rgb}{1.000000,0.705882,0.509804}%
\pgfsetfillcolor{currentfill}%
\pgfsetlinewidth{1.003750pt}%
\definecolor{currentstroke}{rgb}{1.000000,0.705882,0.509804}%
\pgfsetstrokecolor{currentstroke}%
\pgfsetdash{}{0pt}%
\pgfsys@defobject{currentmarker}{\pgfqpoint{-0.041667in}{-0.041667in}}{\pgfqpoint{0.041667in}{0.041667in}}{%
\pgfpathmoveto{\pgfqpoint{0.000000in}{-0.041667in}}%
\pgfpathcurveto{\pgfqpoint{0.011050in}{-0.041667in}}{\pgfqpoint{0.021649in}{-0.037276in}}{\pgfqpoint{0.029463in}{-0.029463in}}%
\pgfpathcurveto{\pgfqpoint{0.037276in}{-0.021649in}}{\pgfqpoint{0.041667in}{-0.011050in}}{\pgfqpoint{0.041667in}{0.000000in}}%
\pgfpathcurveto{\pgfqpoint{0.041667in}{0.011050in}}{\pgfqpoint{0.037276in}{0.021649in}}{\pgfqpoint{0.029463in}{0.029463in}}%
\pgfpathcurveto{\pgfqpoint{0.021649in}{0.037276in}}{\pgfqpoint{0.011050in}{0.041667in}}{\pgfqpoint{0.000000in}{0.041667in}}%
\pgfpathcurveto{\pgfqpoint{-0.011050in}{0.041667in}}{\pgfqpoint{-0.021649in}{0.037276in}}{\pgfqpoint{-0.029463in}{0.029463in}}%
\pgfpathcurveto{\pgfqpoint{-0.037276in}{0.021649in}}{\pgfqpoint{-0.041667in}{0.011050in}}{\pgfqpoint{-0.041667in}{0.000000in}}%
\pgfpathcurveto{\pgfqpoint{-0.041667in}{-0.011050in}}{\pgfqpoint{-0.037276in}{-0.021649in}}{\pgfqpoint{-0.029463in}{-0.029463in}}%
\pgfpathcurveto{\pgfqpoint{-0.021649in}{-0.037276in}}{\pgfqpoint{-0.011050in}{-0.041667in}}{\pgfqpoint{0.000000in}{-0.041667in}}%
\pgfpathclose%
\pgfusepath{stroke,fill}%
}%
\end{pgfscope}%
\begin{pgfscope}%
\pgfpathrectangle{\pgfqpoint{0.526127in}{0.331635in}}{\pgfqpoint{9.300000in}{7.700000in}}%
\pgfusepath{clip}%
\pgfsetbuttcap%
\pgfsetroundjoin%
\definecolor{currentfill}{rgb}{0.552941,0.898039,0.631373}%
\pgfsetfillcolor{currentfill}%
\pgfsetlinewidth{1.003750pt}%
\definecolor{currentstroke}{rgb}{0.552941,0.898039,0.631373}%
\pgfsetstrokecolor{currentstroke}%
\pgfsetdash{}{0pt}%
\pgfsys@defobject{currentmarker}{\pgfqpoint{-0.041667in}{-0.041667in}}{\pgfqpoint{0.041667in}{0.041667in}}{%
\pgfpathmoveto{\pgfqpoint{0.000000in}{-0.041667in}}%
\pgfpathcurveto{\pgfqpoint{0.011050in}{-0.041667in}}{\pgfqpoint{0.021649in}{-0.037276in}}{\pgfqpoint{0.029463in}{-0.029463in}}%
\pgfpathcurveto{\pgfqpoint{0.037276in}{-0.021649in}}{\pgfqpoint{0.041667in}{-0.011050in}}{\pgfqpoint{0.041667in}{0.000000in}}%
\pgfpathcurveto{\pgfqpoint{0.041667in}{0.011050in}}{\pgfqpoint{0.037276in}{0.021649in}}{\pgfqpoint{0.029463in}{0.029463in}}%
\pgfpathcurveto{\pgfqpoint{0.021649in}{0.037276in}}{\pgfqpoint{0.011050in}{0.041667in}}{\pgfqpoint{0.000000in}{0.041667in}}%
\pgfpathcurveto{\pgfqpoint{-0.011050in}{0.041667in}}{\pgfqpoint{-0.021649in}{0.037276in}}{\pgfqpoint{-0.029463in}{0.029463in}}%
\pgfpathcurveto{\pgfqpoint{-0.037276in}{0.021649in}}{\pgfqpoint{-0.041667in}{0.011050in}}{\pgfqpoint{-0.041667in}{0.000000in}}%
\pgfpathcurveto{\pgfqpoint{-0.041667in}{-0.011050in}}{\pgfqpoint{-0.037276in}{-0.021649in}}{\pgfqpoint{-0.029463in}{-0.029463in}}%
\pgfpathcurveto{\pgfqpoint{-0.021649in}{-0.037276in}}{\pgfqpoint{-0.011050in}{-0.041667in}}{\pgfqpoint{0.000000in}{-0.041667in}}%
\pgfpathclose%
\pgfusepath{stroke,fill}%
}%
\end{pgfscope}%
\begin{pgfscope}%
\pgfpathrectangle{\pgfqpoint{0.526127in}{0.331635in}}{\pgfqpoint{9.300000in}{7.700000in}}%
\pgfusepath{clip}%
\pgfsetbuttcap%
\pgfsetroundjoin%
\definecolor{currentfill}{rgb}{1.000000,0.623529,0.607843}%
\pgfsetfillcolor{currentfill}%
\pgfsetlinewidth{1.003750pt}%
\definecolor{currentstroke}{rgb}{1.000000,0.623529,0.607843}%
\pgfsetstrokecolor{currentstroke}%
\pgfsetdash{}{0pt}%
\pgfsys@defobject{currentmarker}{\pgfqpoint{-0.041667in}{-0.041667in}}{\pgfqpoint{0.041667in}{0.041667in}}{%
\pgfpathmoveto{\pgfqpoint{0.000000in}{-0.041667in}}%
\pgfpathcurveto{\pgfqpoint{0.011050in}{-0.041667in}}{\pgfqpoint{0.021649in}{-0.037276in}}{\pgfqpoint{0.029463in}{-0.029463in}}%
\pgfpathcurveto{\pgfqpoint{0.037276in}{-0.021649in}}{\pgfqpoint{0.041667in}{-0.011050in}}{\pgfqpoint{0.041667in}{0.000000in}}%
\pgfpathcurveto{\pgfqpoint{0.041667in}{0.011050in}}{\pgfqpoint{0.037276in}{0.021649in}}{\pgfqpoint{0.029463in}{0.029463in}}%
\pgfpathcurveto{\pgfqpoint{0.021649in}{0.037276in}}{\pgfqpoint{0.011050in}{0.041667in}}{\pgfqpoint{0.000000in}{0.041667in}}%
\pgfpathcurveto{\pgfqpoint{-0.011050in}{0.041667in}}{\pgfqpoint{-0.021649in}{0.037276in}}{\pgfqpoint{-0.029463in}{0.029463in}}%
\pgfpathcurveto{\pgfqpoint{-0.037276in}{0.021649in}}{\pgfqpoint{-0.041667in}{0.011050in}}{\pgfqpoint{-0.041667in}{0.000000in}}%
\pgfpathcurveto{\pgfqpoint{-0.041667in}{-0.011050in}}{\pgfqpoint{-0.037276in}{-0.021649in}}{\pgfqpoint{-0.029463in}{-0.029463in}}%
\pgfpathcurveto{\pgfqpoint{-0.021649in}{-0.037276in}}{\pgfqpoint{-0.011050in}{-0.041667in}}{\pgfqpoint{0.000000in}{-0.041667in}}%
\pgfpathclose%
\pgfusepath{stroke,fill}%
}%
\end{pgfscope}%
\begin{pgfscope}%
\pgfpathrectangle{\pgfqpoint{0.526127in}{0.331635in}}{\pgfqpoint{9.300000in}{7.700000in}}%
\pgfusepath{clip}%
\pgfsetbuttcap%
\pgfsetroundjoin%
\definecolor{currentfill}{rgb}{0.815686,0.733333,1.000000}%
\pgfsetfillcolor{currentfill}%
\pgfsetlinewidth{1.003750pt}%
\definecolor{currentstroke}{rgb}{0.815686,0.733333,1.000000}%
\pgfsetstrokecolor{currentstroke}%
\pgfsetdash{}{0pt}%
\pgfsys@defobject{currentmarker}{\pgfqpoint{-0.041667in}{-0.041667in}}{\pgfqpoint{0.041667in}{0.041667in}}{%
\pgfpathmoveto{\pgfqpoint{0.000000in}{-0.041667in}}%
\pgfpathcurveto{\pgfqpoint{0.011050in}{-0.041667in}}{\pgfqpoint{0.021649in}{-0.037276in}}{\pgfqpoint{0.029463in}{-0.029463in}}%
\pgfpathcurveto{\pgfqpoint{0.037276in}{-0.021649in}}{\pgfqpoint{0.041667in}{-0.011050in}}{\pgfqpoint{0.041667in}{0.000000in}}%
\pgfpathcurveto{\pgfqpoint{0.041667in}{0.011050in}}{\pgfqpoint{0.037276in}{0.021649in}}{\pgfqpoint{0.029463in}{0.029463in}}%
\pgfpathcurveto{\pgfqpoint{0.021649in}{0.037276in}}{\pgfqpoint{0.011050in}{0.041667in}}{\pgfqpoint{0.000000in}{0.041667in}}%
\pgfpathcurveto{\pgfqpoint{-0.011050in}{0.041667in}}{\pgfqpoint{-0.021649in}{0.037276in}}{\pgfqpoint{-0.029463in}{0.029463in}}%
\pgfpathcurveto{\pgfqpoint{-0.037276in}{0.021649in}}{\pgfqpoint{-0.041667in}{0.011050in}}{\pgfqpoint{-0.041667in}{0.000000in}}%
\pgfpathcurveto{\pgfqpoint{-0.041667in}{-0.011050in}}{\pgfqpoint{-0.037276in}{-0.021649in}}{\pgfqpoint{-0.029463in}{-0.029463in}}%
\pgfpathcurveto{\pgfqpoint{-0.021649in}{-0.037276in}}{\pgfqpoint{-0.011050in}{-0.041667in}}{\pgfqpoint{0.000000in}{-0.041667in}}%
\pgfpathclose%
\pgfusepath{stroke,fill}%
}%
\end{pgfscope}%
\begin{pgfscope}%
\pgfpathrectangle{\pgfqpoint{0.526127in}{0.331635in}}{\pgfqpoint{9.300000in}{7.700000in}}%
\pgfusepath{clip}%
\pgfsetbuttcap%
\pgfsetroundjoin%
\definecolor{currentfill}{rgb}{0.870588,0.733333,0.607843}%
\pgfsetfillcolor{currentfill}%
\pgfsetlinewidth{1.003750pt}%
\definecolor{currentstroke}{rgb}{0.870588,0.733333,0.607843}%
\pgfsetstrokecolor{currentstroke}%
\pgfsetdash{}{0pt}%
\pgfsys@defobject{currentmarker}{\pgfqpoint{-0.041667in}{-0.041667in}}{\pgfqpoint{0.041667in}{0.041667in}}{%
\pgfpathmoveto{\pgfqpoint{0.000000in}{-0.041667in}}%
\pgfpathcurveto{\pgfqpoint{0.011050in}{-0.041667in}}{\pgfqpoint{0.021649in}{-0.037276in}}{\pgfqpoint{0.029463in}{-0.029463in}}%
\pgfpathcurveto{\pgfqpoint{0.037276in}{-0.021649in}}{\pgfqpoint{0.041667in}{-0.011050in}}{\pgfqpoint{0.041667in}{0.000000in}}%
\pgfpathcurveto{\pgfqpoint{0.041667in}{0.011050in}}{\pgfqpoint{0.037276in}{0.021649in}}{\pgfqpoint{0.029463in}{0.029463in}}%
\pgfpathcurveto{\pgfqpoint{0.021649in}{0.037276in}}{\pgfqpoint{0.011050in}{0.041667in}}{\pgfqpoint{0.000000in}{0.041667in}}%
\pgfpathcurveto{\pgfqpoint{-0.011050in}{0.041667in}}{\pgfqpoint{-0.021649in}{0.037276in}}{\pgfqpoint{-0.029463in}{0.029463in}}%
\pgfpathcurveto{\pgfqpoint{-0.037276in}{0.021649in}}{\pgfqpoint{-0.041667in}{0.011050in}}{\pgfqpoint{-0.041667in}{0.000000in}}%
\pgfpathcurveto{\pgfqpoint{-0.041667in}{-0.011050in}}{\pgfqpoint{-0.037276in}{-0.021649in}}{\pgfqpoint{-0.029463in}{-0.029463in}}%
\pgfpathcurveto{\pgfqpoint{-0.021649in}{-0.037276in}}{\pgfqpoint{-0.011050in}{-0.041667in}}{\pgfqpoint{0.000000in}{-0.041667in}}%
\pgfpathclose%
\pgfusepath{stroke,fill}%
}%
\end{pgfscope}%
\begin{pgfscope}%
\pgfpathrectangle{\pgfqpoint{0.526127in}{0.331635in}}{\pgfqpoint{9.300000in}{7.700000in}}%
\pgfusepath{clip}%
\pgfsetbuttcap%
\pgfsetroundjoin%
\definecolor{currentfill}{rgb}{0.980392,0.690196,0.894118}%
\pgfsetfillcolor{currentfill}%
\pgfsetlinewidth{1.003750pt}%
\definecolor{currentstroke}{rgb}{0.980392,0.690196,0.894118}%
\pgfsetstrokecolor{currentstroke}%
\pgfsetdash{}{0pt}%
\pgfsys@defobject{currentmarker}{\pgfqpoint{-0.041667in}{-0.041667in}}{\pgfqpoint{0.041667in}{0.041667in}}{%
\pgfpathmoveto{\pgfqpoint{0.000000in}{-0.041667in}}%
\pgfpathcurveto{\pgfqpoint{0.011050in}{-0.041667in}}{\pgfqpoint{0.021649in}{-0.037276in}}{\pgfqpoint{0.029463in}{-0.029463in}}%
\pgfpathcurveto{\pgfqpoint{0.037276in}{-0.021649in}}{\pgfqpoint{0.041667in}{-0.011050in}}{\pgfqpoint{0.041667in}{0.000000in}}%
\pgfpathcurveto{\pgfqpoint{0.041667in}{0.011050in}}{\pgfqpoint{0.037276in}{0.021649in}}{\pgfqpoint{0.029463in}{0.029463in}}%
\pgfpathcurveto{\pgfqpoint{0.021649in}{0.037276in}}{\pgfqpoint{0.011050in}{0.041667in}}{\pgfqpoint{0.000000in}{0.041667in}}%
\pgfpathcurveto{\pgfqpoint{-0.011050in}{0.041667in}}{\pgfqpoint{-0.021649in}{0.037276in}}{\pgfqpoint{-0.029463in}{0.029463in}}%
\pgfpathcurveto{\pgfqpoint{-0.037276in}{0.021649in}}{\pgfqpoint{-0.041667in}{0.011050in}}{\pgfqpoint{-0.041667in}{0.000000in}}%
\pgfpathcurveto{\pgfqpoint{-0.041667in}{-0.011050in}}{\pgfqpoint{-0.037276in}{-0.021649in}}{\pgfqpoint{-0.029463in}{-0.029463in}}%
\pgfpathcurveto{\pgfqpoint{-0.021649in}{-0.037276in}}{\pgfqpoint{-0.011050in}{-0.041667in}}{\pgfqpoint{0.000000in}{-0.041667in}}%
\pgfpathclose%
\pgfusepath{stroke,fill}%
}%
\end{pgfscope}%
\begin{pgfscope}%
\pgfpathrectangle{\pgfqpoint{0.526127in}{0.331635in}}{\pgfqpoint{9.300000in}{7.700000in}}%
\pgfusepath{clip}%
\pgfsetbuttcap%
\pgfsetroundjoin%
\definecolor{currentfill}{rgb}{0.721569,0.521569,0.039216}%
\pgfsetfillcolor{currentfill}%
\pgfsetlinewidth{1.003750pt}%
\definecolor{currentstroke}{rgb}{0.721569,0.521569,0.039216}%
\pgfsetstrokecolor{currentstroke}%
\pgfsetdash{}{0pt}%
\pgfsys@defobject{currentmarker}{\pgfqpoint{-0.041667in}{-0.041667in}}{\pgfqpoint{0.041667in}{0.041667in}}{%
\pgfpathmoveto{\pgfqpoint{0.000000in}{-0.041667in}}%
\pgfpathcurveto{\pgfqpoint{0.011050in}{-0.041667in}}{\pgfqpoint{0.021649in}{-0.037276in}}{\pgfqpoint{0.029463in}{-0.029463in}}%
\pgfpathcurveto{\pgfqpoint{0.037276in}{-0.021649in}}{\pgfqpoint{0.041667in}{-0.011050in}}{\pgfqpoint{0.041667in}{0.000000in}}%
\pgfpathcurveto{\pgfqpoint{0.041667in}{0.011050in}}{\pgfqpoint{0.037276in}{0.021649in}}{\pgfqpoint{0.029463in}{0.029463in}}%
\pgfpathcurveto{\pgfqpoint{0.021649in}{0.037276in}}{\pgfqpoint{0.011050in}{0.041667in}}{\pgfqpoint{0.000000in}{0.041667in}}%
\pgfpathcurveto{\pgfqpoint{-0.011050in}{0.041667in}}{\pgfqpoint{-0.021649in}{0.037276in}}{\pgfqpoint{-0.029463in}{0.029463in}}%
\pgfpathcurveto{\pgfqpoint{-0.037276in}{0.021649in}}{\pgfqpoint{-0.041667in}{0.011050in}}{\pgfqpoint{-0.041667in}{0.000000in}}%
\pgfpathcurveto{\pgfqpoint{-0.041667in}{-0.011050in}}{\pgfqpoint{-0.037276in}{-0.021649in}}{\pgfqpoint{-0.029463in}{-0.029463in}}%
\pgfpathcurveto{\pgfqpoint{-0.021649in}{-0.037276in}}{\pgfqpoint{-0.011050in}{-0.041667in}}{\pgfqpoint{0.000000in}{-0.041667in}}%
\pgfpathclose%
\pgfusepath{stroke,fill}%
}%
\end{pgfscope}%
\begin{pgfscope}%
\pgfsetbuttcap%
\pgfsetroundjoin%
\definecolor{currentfill}{rgb}{0.000000,0.000000,0.000000}%
\pgfsetfillcolor{currentfill}%
\pgfsetlinewidth{0.803000pt}%
\definecolor{currentstroke}{rgb}{0.000000,0.000000,0.000000}%
\pgfsetstrokecolor{currentstroke}%
\pgfsetdash{}{0pt}%
\pgfsys@defobject{currentmarker}{\pgfqpoint{0.000000in}{-0.048611in}}{\pgfqpoint{0.000000in}{0.000000in}}{%
\pgfpathmoveto{\pgfqpoint{0.000000in}{0.000000in}}%
\pgfpathlineto{\pgfqpoint{0.000000in}{-0.048611in}}%
\pgfusepath{stroke,fill}%
}%
\begin{pgfscope}%
\pgfsys@transformshift{1.954606in}{0.331635in}%
\pgfsys@useobject{currentmarker}{}%
\end{pgfscope}%
\end{pgfscope}%
\begin{pgfscope}%
\definecolor{textcolor}{rgb}{0.000000,0.000000,0.000000}%
\pgfsetstrokecolor{textcolor}%
\pgfsetfillcolor{textcolor}%
\pgftext[x=1.954606in,y=0.234413in,,top]{\color{textcolor}\sffamily\fontsize{10.000000}{12.000000}\selectfont \ensuremath{-}10}%
\end{pgfscope}%
\begin{pgfscope}%
\pgfsetbuttcap%
\pgfsetroundjoin%
\definecolor{currentfill}{rgb}{0.000000,0.000000,0.000000}%
\pgfsetfillcolor{currentfill}%
\pgfsetlinewidth{0.803000pt}%
\definecolor{currentstroke}{rgb}{0.000000,0.000000,0.000000}%
\pgfsetstrokecolor{currentstroke}%
\pgfsetdash{}{0pt}%
\pgfsys@defobject{currentmarker}{\pgfqpoint{0.000000in}{-0.048611in}}{\pgfqpoint{0.000000in}{0.000000in}}{%
\pgfpathmoveto{\pgfqpoint{0.000000in}{0.000000in}}%
\pgfpathlineto{\pgfqpoint{0.000000in}{-0.048611in}}%
\pgfusepath{stroke,fill}%
}%
\begin{pgfscope}%
\pgfsys@transformshift{3.838891in}{0.331635in}%
\pgfsys@useobject{currentmarker}{}%
\end{pgfscope}%
\end{pgfscope}%
\begin{pgfscope}%
\definecolor{textcolor}{rgb}{0.000000,0.000000,0.000000}%
\pgfsetstrokecolor{textcolor}%
\pgfsetfillcolor{textcolor}%
\pgftext[x=3.838891in,y=0.234413in,,top]{\color{textcolor}\sffamily\fontsize{10.000000}{12.000000}\selectfont \ensuremath{-}5}%
\end{pgfscope}%
\begin{pgfscope}%
\pgfsetbuttcap%
\pgfsetroundjoin%
\definecolor{currentfill}{rgb}{0.000000,0.000000,0.000000}%
\pgfsetfillcolor{currentfill}%
\pgfsetlinewidth{0.803000pt}%
\definecolor{currentstroke}{rgb}{0.000000,0.000000,0.000000}%
\pgfsetstrokecolor{currentstroke}%
\pgfsetdash{}{0pt}%
\pgfsys@defobject{currentmarker}{\pgfqpoint{0.000000in}{-0.048611in}}{\pgfqpoint{0.000000in}{0.000000in}}{%
\pgfpathmoveto{\pgfqpoint{0.000000in}{0.000000in}}%
\pgfpathlineto{\pgfqpoint{0.000000in}{-0.048611in}}%
\pgfusepath{stroke,fill}%
}%
\begin{pgfscope}%
\pgfsys@transformshift{5.723177in}{0.331635in}%
\pgfsys@useobject{currentmarker}{}%
\end{pgfscope}%
\end{pgfscope}%
\begin{pgfscope}%
\definecolor{textcolor}{rgb}{0.000000,0.000000,0.000000}%
\pgfsetstrokecolor{textcolor}%
\pgfsetfillcolor{textcolor}%
\pgftext[x=5.723177in,y=0.234413in,,top]{\color{textcolor}\sffamily\fontsize{10.000000}{12.000000}\selectfont 0}%
\end{pgfscope}%
\begin{pgfscope}%
\pgfsetbuttcap%
\pgfsetroundjoin%
\definecolor{currentfill}{rgb}{0.000000,0.000000,0.000000}%
\pgfsetfillcolor{currentfill}%
\pgfsetlinewidth{0.803000pt}%
\definecolor{currentstroke}{rgb}{0.000000,0.000000,0.000000}%
\pgfsetstrokecolor{currentstroke}%
\pgfsetdash{}{0pt}%
\pgfsys@defobject{currentmarker}{\pgfqpoint{0.000000in}{-0.048611in}}{\pgfqpoint{0.000000in}{0.000000in}}{%
\pgfpathmoveto{\pgfqpoint{0.000000in}{0.000000in}}%
\pgfpathlineto{\pgfqpoint{0.000000in}{-0.048611in}}%
\pgfusepath{stroke,fill}%
}%
\begin{pgfscope}%
\pgfsys@transformshift{7.607462in}{0.331635in}%
\pgfsys@useobject{currentmarker}{}%
\end{pgfscope}%
\end{pgfscope}%
\begin{pgfscope}%
\definecolor{textcolor}{rgb}{0.000000,0.000000,0.000000}%
\pgfsetstrokecolor{textcolor}%
\pgfsetfillcolor{textcolor}%
\pgftext[x=7.607462in,y=0.234413in,,top]{\color{textcolor}\sffamily\fontsize{10.000000}{12.000000}\selectfont 5}%
\end{pgfscope}%
\begin{pgfscope}%
\pgfsetbuttcap%
\pgfsetroundjoin%
\definecolor{currentfill}{rgb}{0.000000,0.000000,0.000000}%
\pgfsetfillcolor{currentfill}%
\pgfsetlinewidth{0.803000pt}%
\definecolor{currentstroke}{rgb}{0.000000,0.000000,0.000000}%
\pgfsetstrokecolor{currentstroke}%
\pgfsetdash{}{0pt}%
\pgfsys@defobject{currentmarker}{\pgfqpoint{0.000000in}{-0.048611in}}{\pgfqpoint{0.000000in}{0.000000in}}{%
\pgfpathmoveto{\pgfqpoint{0.000000in}{0.000000in}}%
\pgfpathlineto{\pgfqpoint{0.000000in}{-0.048611in}}%
\pgfusepath{stroke,fill}%
}%
\begin{pgfscope}%
\pgfsys@transformshift{9.491748in}{0.331635in}%
\pgfsys@useobject{currentmarker}{}%
\end{pgfscope}%
\end{pgfscope}%
\begin{pgfscope}%
\definecolor{textcolor}{rgb}{0.000000,0.000000,0.000000}%
\pgfsetstrokecolor{textcolor}%
\pgfsetfillcolor{textcolor}%
\pgftext[x=9.491748in,y=0.234413in,,top]{\color{textcolor}\sffamily\fontsize{10.000000}{12.000000}\selectfont 10}%
\end{pgfscope}%
\begin{pgfscope}%
\pgfsetbuttcap%
\pgfsetroundjoin%
\definecolor{currentfill}{rgb}{0.000000,0.000000,0.000000}%
\pgfsetfillcolor{currentfill}%
\pgfsetlinewidth{0.803000pt}%
\definecolor{currentstroke}{rgb}{0.000000,0.000000,0.000000}%
\pgfsetstrokecolor{currentstroke}%
\pgfsetdash{}{0pt}%
\pgfsys@defobject{currentmarker}{\pgfqpoint{-0.048611in}{0.000000in}}{\pgfqpoint{-0.000000in}{0.000000in}}{%
\pgfpathmoveto{\pgfqpoint{-0.000000in}{0.000000in}}%
\pgfpathlineto{\pgfqpoint{-0.048611in}{0.000000in}}%
\pgfusepath{stroke,fill}%
}%
\begin{pgfscope}%
\pgfsys@transformshift{0.526127in}{1.120852in}%
\pgfsys@useobject{currentmarker}{}%
\end{pgfscope}%
\end{pgfscope}%
\begin{pgfscope}%
\definecolor{textcolor}{rgb}{0.000000,0.000000,0.000000}%
\pgfsetstrokecolor{textcolor}%
\pgfsetfillcolor{textcolor}%
\pgftext[x=0.100000in, y=1.068091in, left, base]{\color{textcolor}\sffamily\fontsize{10.000000}{12.000000}\selectfont \ensuremath{-}7.5}%
\end{pgfscope}%
\begin{pgfscope}%
\pgfsetbuttcap%
\pgfsetroundjoin%
\definecolor{currentfill}{rgb}{0.000000,0.000000,0.000000}%
\pgfsetfillcolor{currentfill}%
\pgfsetlinewidth{0.803000pt}%
\definecolor{currentstroke}{rgb}{0.000000,0.000000,0.000000}%
\pgfsetstrokecolor{currentstroke}%
\pgfsetdash{}{0pt}%
\pgfsys@defobject{currentmarker}{\pgfqpoint{-0.048611in}{0.000000in}}{\pgfqpoint{-0.000000in}{0.000000in}}{%
\pgfpathmoveto{\pgfqpoint{-0.000000in}{0.000000in}}%
\pgfpathlineto{\pgfqpoint{-0.048611in}{0.000000in}}%
\pgfusepath{stroke,fill}%
}%
\begin{pgfscope}%
\pgfsys@transformshift{0.526127in}{2.007930in}%
\pgfsys@useobject{currentmarker}{}%
\end{pgfscope}%
\end{pgfscope}%
\begin{pgfscope}%
\definecolor{textcolor}{rgb}{0.000000,0.000000,0.000000}%
\pgfsetstrokecolor{textcolor}%
\pgfsetfillcolor{textcolor}%
\pgftext[x=0.100000in, y=1.955168in, left, base]{\color{textcolor}\sffamily\fontsize{10.000000}{12.000000}\selectfont \ensuremath{-}5.0}%
\end{pgfscope}%
\begin{pgfscope}%
\pgfsetbuttcap%
\pgfsetroundjoin%
\definecolor{currentfill}{rgb}{0.000000,0.000000,0.000000}%
\pgfsetfillcolor{currentfill}%
\pgfsetlinewidth{0.803000pt}%
\definecolor{currentstroke}{rgb}{0.000000,0.000000,0.000000}%
\pgfsetstrokecolor{currentstroke}%
\pgfsetdash{}{0pt}%
\pgfsys@defobject{currentmarker}{\pgfqpoint{-0.048611in}{0.000000in}}{\pgfqpoint{-0.000000in}{0.000000in}}{%
\pgfpathmoveto{\pgfqpoint{-0.000000in}{0.000000in}}%
\pgfpathlineto{\pgfqpoint{-0.048611in}{0.000000in}}%
\pgfusepath{stroke,fill}%
}%
\begin{pgfscope}%
\pgfsys@transformshift{0.526127in}{2.895007in}%
\pgfsys@useobject{currentmarker}{}%
\end{pgfscope}%
\end{pgfscope}%
\begin{pgfscope}%
\definecolor{textcolor}{rgb}{0.000000,0.000000,0.000000}%
\pgfsetstrokecolor{textcolor}%
\pgfsetfillcolor{textcolor}%
\pgftext[x=0.100000in, y=2.842246in, left, base]{\color{textcolor}\sffamily\fontsize{10.000000}{12.000000}\selectfont \ensuremath{-}2.5}%
\end{pgfscope}%
\begin{pgfscope}%
\pgfsetbuttcap%
\pgfsetroundjoin%
\definecolor{currentfill}{rgb}{0.000000,0.000000,0.000000}%
\pgfsetfillcolor{currentfill}%
\pgfsetlinewidth{0.803000pt}%
\definecolor{currentstroke}{rgb}{0.000000,0.000000,0.000000}%
\pgfsetstrokecolor{currentstroke}%
\pgfsetdash{}{0pt}%
\pgfsys@defobject{currentmarker}{\pgfqpoint{-0.048611in}{0.000000in}}{\pgfqpoint{-0.000000in}{0.000000in}}{%
\pgfpathmoveto{\pgfqpoint{-0.000000in}{0.000000in}}%
\pgfpathlineto{\pgfqpoint{-0.048611in}{0.000000in}}%
\pgfusepath{stroke,fill}%
}%
\begin{pgfscope}%
\pgfsys@transformshift{0.526127in}{3.782085in}%
\pgfsys@useobject{currentmarker}{}%
\end{pgfscope}%
\end{pgfscope}%
\begin{pgfscope}%
\definecolor{textcolor}{rgb}{0.000000,0.000000,0.000000}%
\pgfsetstrokecolor{textcolor}%
\pgfsetfillcolor{textcolor}%
\pgftext[x=0.208025in, y=3.729323in, left, base]{\color{textcolor}\sffamily\fontsize{10.000000}{12.000000}\selectfont 0.0}%
\end{pgfscope}%
\begin{pgfscope}%
\pgfsetbuttcap%
\pgfsetroundjoin%
\definecolor{currentfill}{rgb}{0.000000,0.000000,0.000000}%
\pgfsetfillcolor{currentfill}%
\pgfsetlinewidth{0.803000pt}%
\definecolor{currentstroke}{rgb}{0.000000,0.000000,0.000000}%
\pgfsetstrokecolor{currentstroke}%
\pgfsetdash{}{0pt}%
\pgfsys@defobject{currentmarker}{\pgfqpoint{-0.048611in}{0.000000in}}{\pgfqpoint{-0.000000in}{0.000000in}}{%
\pgfpathmoveto{\pgfqpoint{-0.000000in}{0.000000in}}%
\pgfpathlineto{\pgfqpoint{-0.048611in}{0.000000in}}%
\pgfusepath{stroke,fill}%
}%
\begin{pgfscope}%
\pgfsys@transformshift{0.526127in}{4.669162in}%
\pgfsys@useobject{currentmarker}{}%
\end{pgfscope}%
\end{pgfscope}%
\begin{pgfscope}%
\definecolor{textcolor}{rgb}{0.000000,0.000000,0.000000}%
\pgfsetstrokecolor{textcolor}%
\pgfsetfillcolor{textcolor}%
\pgftext[x=0.208025in, y=4.616401in, left, base]{\color{textcolor}\sffamily\fontsize{10.000000}{12.000000}\selectfont 2.5}%
\end{pgfscope}%
\begin{pgfscope}%
\pgfsetbuttcap%
\pgfsetroundjoin%
\definecolor{currentfill}{rgb}{0.000000,0.000000,0.000000}%
\pgfsetfillcolor{currentfill}%
\pgfsetlinewidth{0.803000pt}%
\definecolor{currentstroke}{rgb}{0.000000,0.000000,0.000000}%
\pgfsetstrokecolor{currentstroke}%
\pgfsetdash{}{0pt}%
\pgfsys@defobject{currentmarker}{\pgfqpoint{-0.048611in}{0.000000in}}{\pgfqpoint{-0.000000in}{0.000000in}}{%
\pgfpathmoveto{\pgfqpoint{-0.000000in}{0.000000in}}%
\pgfpathlineto{\pgfqpoint{-0.048611in}{0.000000in}}%
\pgfusepath{stroke,fill}%
}%
\begin{pgfscope}%
\pgfsys@transformshift{0.526127in}{5.556240in}%
\pgfsys@useobject{currentmarker}{}%
\end{pgfscope}%
\end{pgfscope}%
\begin{pgfscope}%
\definecolor{textcolor}{rgb}{0.000000,0.000000,0.000000}%
\pgfsetstrokecolor{textcolor}%
\pgfsetfillcolor{textcolor}%
\pgftext[x=0.208025in, y=5.503478in, left, base]{\color{textcolor}\sffamily\fontsize{10.000000}{12.000000}\selectfont 5.0}%
\end{pgfscope}%
\begin{pgfscope}%
\pgfsetbuttcap%
\pgfsetroundjoin%
\definecolor{currentfill}{rgb}{0.000000,0.000000,0.000000}%
\pgfsetfillcolor{currentfill}%
\pgfsetlinewidth{0.803000pt}%
\definecolor{currentstroke}{rgb}{0.000000,0.000000,0.000000}%
\pgfsetstrokecolor{currentstroke}%
\pgfsetdash{}{0pt}%
\pgfsys@defobject{currentmarker}{\pgfqpoint{-0.048611in}{0.000000in}}{\pgfqpoint{-0.000000in}{0.000000in}}{%
\pgfpathmoveto{\pgfqpoint{-0.000000in}{0.000000in}}%
\pgfpathlineto{\pgfqpoint{-0.048611in}{0.000000in}}%
\pgfusepath{stroke,fill}%
}%
\begin{pgfscope}%
\pgfsys@transformshift{0.526127in}{6.443317in}%
\pgfsys@useobject{currentmarker}{}%
\end{pgfscope}%
\end{pgfscope}%
\begin{pgfscope}%
\definecolor{textcolor}{rgb}{0.000000,0.000000,0.000000}%
\pgfsetstrokecolor{textcolor}%
\pgfsetfillcolor{textcolor}%
\pgftext[x=0.208025in, y=6.390556in, left, base]{\color{textcolor}\sffamily\fontsize{10.000000}{12.000000}\selectfont 7.5}%
\end{pgfscope}%
\begin{pgfscope}%
\pgfsetbuttcap%
\pgfsetroundjoin%
\definecolor{currentfill}{rgb}{0.000000,0.000000,0.000000}%
\pgfsetfillcolor{currentfill}%
\pgfsetlinewidth{0.803000pt}%
\definecolor{currentstroke}{rgb}{0.000000,0.000000,0.000000}%
\pgfsetstrokecolor{currentstroke}%
\pgfsetdash{}{0pt}%
\pgfsys@defobject{currentmarker}{\pgfqpoint{-0.048611in}{0.000000in}}{\pgfqpoint{-0.000000in}{0.000000in}}{%
\pgfpathmoveto{\pgfqpoint{-0.000000in}{0.000000in}}%
\pgfpathlineto{\pgfqpoint{-0.048611in}{0.000000in}}%
\pgfusepath{stroke,fill}%
}%
\begin{pgfscope}%
\pgfsys@transformshift{0.526127in}{7.330395in}%
\pgfsys@useobject{currentmarker}{}%
\end{pgfscope}%
\end{pgfscope}%
\begin{pgfscope}%
\definecolor{textcolor}{rgb}{0.000000,0.000000,0.000000}%
\pgfsetstrokecolor{textcolor}%
\pgfsetfillcolor{textcolor}%
\pgftext[x=0.119660in, y=7.277633in, left, base]{\color{textcolor}\sffamily\fontsize{10.000000}{12.000000}\selectfont 10.0}%
\end{pgfscope}%
\begin{pgfscope}%
\pgfpathrectangle{\pgfqpoint{0.526127in}{0.331635in}}{\pgfqpoint{9.300000in}{7.700000in}}%
\pgfusepath{clip}%
\pgfsetrectcap%
\pgfsetroundjoin%
\pgfsetlinewidth{1.505625pt}%
\definecolor{currentstroke}{rgb}{0.631373,0.788235,0.956863}%
\pgfsetstrokecolor{currentstroke}%
\pgfsetstrokeopacity{0.200000}%
\pgfsetdash{}{0pt}%
\pgfpathmoveto{\pgfqpoint{8.423233in}{5.791596in}}%
\pgfpathlineto{\pgfqpoint{6.100521in}{4.940903in}}%
\pgfusepath{stroke}%
\end{pgfscope}%
\begin{pgfscope}%
\pgfpathrectangle{\pgfqpoint{0.526127in}{0.331635in}}{\pgfqpoint{9.300000in}{7.700000in}}%
\pgfusepath{clip}%
\pgfsetrectcap%
\pgfsetroundjoin%
\pgfsetlinewidth{1.505625pt}%
\definecolor{currentstroke}{rgb}{0.631373,0.788235,0.956863}%
\pgfsetstrokecolor{currentstroke}%
\pgfsetstrokeopacity{0.200000}%
\pgfsetdash{}{0pt}%
\pgfpathmoveto{\pgfqpoint{4.747368in}{3.915110in}}%
\pgfpathlineto{\pgfqpoint{6.100521in}{4.940903in}}%
\pgfusepath{stroke}%
\end{pgfscope}%
\begin{pgfscope}%
\pgfpathrectangle{\pgfqpoint{0.526127in}{0.331635in}}{\pgfqpoint{9.300000in}{7.700000in}}%
\pgfusepath{clip}%
\pgfsetrectcap%
\pgfsetroundjoin%
\pgfsetlinewidth{1.505625pt}%
\definecolor{currentstroke}{rgb}{0.631373,0.788235,0.956863}%
\pgfsetstrokecolor{currentstroke}%
\pgfsetstrokeopacity{0.200000}%
\pgfsetdash{}{0pt}%
\pgfpathmoveto{\pgfqpoint{6.866294in}{4.174971in}}%
\pgfpathlineto{\pgfqpoint{6.100521in}{4.940903in}}%
\pgfusepath{stroke}%
\end{pgfscope}%
\begin{pgfscope}%
\pgfpathrectangle{\pgfqpoint{0.526127in}{0.331635in}}{\pgfqpoint{9.300000in}{7.700000in}}%
\pgfusepath{clip}%
\pgfsetrectcap%
\pgfsetroundjoin%
\pgfsetlinewidth{1.505625pt}%
\definecolor{currentstroke}{rgb}{0.631373,0.788235,0.956863}%
\pgfsetstrokecolor{currentstroke}%
\pgfsetstrokeopacity{0.200000}%
\pgfsetdash{}{0pt}%
\pgfpathmoveto{\pgfqpoint{8.257844in}{5.239559in}}%
\pgfpathlineto{\pgfqpoint{6.100521in}{4.940903in}}%
\pgfusepath{stroke}%
\end{pgfscope}%
\begin{pgfscope}%
\pgfpathrectangle{\pgfqpoint{0.526127in}{0.331635in}}{\pgfqpoint{9.300000in}{7.700000in}}%
\pgfusepath{clip}%
\pgfsetrectcap%
\pgfsetroundjoin%
\pgfsetlinewidth{1.505625pt}%
\definecolor{currentstroke}{rgb}{0.631373,0.788235,0.956863}%
\pgfsetstrokecolor{currentstroke}%
\pgfsetstrokeopacity{0.200000}%
\pgfsetdash{}{0pt}%
\pgfpathmoveto{\pgfqpoint{3.377597in}{2.747984in}}%
\pgfpathlineto{\pgfqpoint{6.100521in}{4.940903in}}%
\pgfusepath{stroke}%
\end{pgfscope}%
\begin{pgfscope}%
\pgfpathrectangle{\pgfqpoint{0.526127in}{0.331635in}}{\pgfqpoint{9.300000in}{7.700000in}}%
\pgfusepath{clip}%
\pgfsetrectcap%
\pgfsetroundjoin%
\pgfsetlinewidth{1.505625pt}%
\definecolor{currentstroke}{rgb}{0.631373,0.788235,0.956863}%
\pgfsetstrokecolor{currentstroke}%
\pgfsetstrokeopacity{0.200000}%
\pgfsetdash{}{0pt}%
\pgfpathmoveto{\pgfqpoint{5.374384in}{6.574501in}}%
\pgfpathlineto{\pgfqpoint{6.100521in}{4.940903in}}%
\pgfusepath{stroke}%
\end{pgfscope}%
\begin{pgfscope}%
\pgfpathrectangle{\pgfqpoint{0.526127in}{0.331635in}}{\pgfqpoint{9.300000in}{7.700000in}}%
\pgfusepath{clip}%
\pgfsetrectcap%
\pgfsetroundjoin%
\pgfsetlinewidth{1.505625pt}%
\definecolor{currentstroke}{rgb}{0.631373,0.788235,0.956863}%
\pgfsetstrokecolor{currentstroke}%
\pgfsetstrokeopacity{0.200000}%
\pgfsetdash{}{0pt}%
\pgfpathmoveto{\pgfqpoint{3.945884in}{4.863590in}}%
\pgfpathlineto{\pgfqpoint{6.100521in}{4.940903in}}%
\pgfusepath{stroke}%
\end{pgfscope}%
\begin{pgfscope}%
\pgfpathrectangle{\pgfqpoint{0.526127in}{0.331635in}}{\pgfqpoint{9.300000in}{7.700000in}}%
\pgfusepath{clip}%
\pgfsetrectcap%
\pgfsetroundjoin%
\pgfsetlinewidth{1.505625pt}%
\definecolor{currentstroke}{rgb}{0.631373,0.788235,0.956863}%
\pgfsetstrokecolor{currentstroke}%
\pgfsetstrokeopacity{0.200000}%
\pgfsetdash{}{0pt}%
\pgfpathmoveto{\pgfqpoint{6.422655in}{6.062096in}}%
\pgfpathlineto{\pgfqpoint{6.100521in}{4.940903in}}%
\pgfusepath{stroke}%
\end{pgfscope}%
\begin{pgfscope}%
\pgfpathrectangle{\pgfqpoint{0.526127in}{0.331635in}}{\pgfqpoint{9.300000in}{7.700000in}}%
\pgfusepath{clip}%
\pgfsetrectcap%
\pgfsetroundjoin%
\pgfsetlinewidth{1.505625pt}%
\definecolor{currentstroke}{rgb}{0.631373,0.788235,0.956863}%
\pgfsetstrokecolor{currentstroke}%
\pgfsetstrokeopacity{0.200000}%
\pgfsetdash{}{0pt}%
\pgfpathmoveto{\pgfqpoint{3.784322in}{5.194829in}}%
\pgfpathlineto{\pgfqpoint{6.100521in}{4.940903in}}%
\pgfusepath{stroke}%
\end{pgfscope}%
\begin{pgfscope}%
\pgfpathrectangle{\pgfqpoint{0.526127in}{0.331635in}}{\pgfqpoint{9.300000in}{7.700000in}}%
\pgfusepath{clip}%
\pgfsetrectcap%
\pgfsetroundjoin%
\pgfsetlinewidth{1.505625pt}%
\definecolor{currentstroke}{rgb}{0.631373,0.788235,0.956863}%
\pgfsetstrokecolor{currentstroke}%
\pgfsetstrokeopacity{0.200000}%
\pgfsetdash{}{0pt}%
\pgfpathmoveto{\pgfqpoint{4.807076in}{4.400777in}}%
\pgfpathlineto{\pgfqpoint{6.100521in}{4.940903in}}%
\pgfusepath{stroke}%
\end{pgfscope}%
\begin{pgfscope}%
\pgfpathrectangle{\pgfqpoint{0.526127in}{0.331635in}}{\pgfqpoint{9.300000in}{7.700000in}}%
\pgfusepath{clip}%
\pgfsetrectcap%
\pgfsetroundjoin%
\pgfsetlinewidth{1.505625pt}%
\definecolor{currentstroke}{rgb}{0.631373,0.788235,0.956863}%
\pgfsetstrokecolor{currentstroke}%
\pgfsetstrokeopacity{0.200000}%
\pgfsetdash{}{0pt}%
\pgfpathmoveto{\pgfqpoint{7.809864in}{6.914805in}}%
\pgfpathlineto{\pgfqpoint{6.100521in}{4.940903in}}%
\pgfusepath{stroke}%
\end{pgfscope}%
\begin{pgfscope}%
\pgfpathrectangle{\pgfqpoint{0.526127in}{0.331635in}}{\pgfqpoint{9.300000in}{7.700000in}}%
\pgfusepath{clip}%
\pgfsetrectcap%
\pgfsetroundjoin%
\pgfsetlinewidth{1.505625pt}%
\definecolor{currentstroke}{rgb}{0.631373,0.788235,0.956863}%
\pgfsetstrokecolor{currentstroke}%
\pgfsetstrokeopacity{0.200000}%
\pgfsetdash{}{0pt}%
\pgfpathmoveto{\pgfqpoint{7.744328in}{2.772445in}}%
\pgfpathlineto{\pgfqpoint{6.100521in}{4.940903in}}%
\pgfusepath{stroke}%
\end{pgfscope}%
\begin{pgfscope}%
\pgfpathrectangle{\pgfqpoint{0.526127in}{0.331635in}}{\pgfqpoint{9.300000in}{7.700000in}}%
\pgfusepath{clip}%
\pgfsetrectcap%
\pgfsetroundjoin%
\pgfsetlinewidth{1.505625pt}%
\definecolor{currentstroke}{rgb}{0.631373,0.788235,0.956863}%
\pgfsetstrokecolor{currentstroke}%
\pgfsetstrokeopacity{0.200000}%
\pgfsetdash{}{0pt}%
\pgfpathmoveto{\pgfqpoint{6.917985in}{5.083266in}}%
\pgfpathlineto{\pgfqpoint{6.100521in}{4.940903in}}%
\pgfusepath{stroke}%
\end{pgfscope}%
\begin{pgfscope}%
\pgfpathrectangle{\pgfqpoint{0.526127in}{0.331635in}}{\pgfqpoint{9.300000in}{7.700000in}}%
\pgfusepath{clip}%
\pgfsetrectcap%
\pgfsetroundjoin%
\pgfsetlinewidth{1.505625pt}%
\definecolor{currentstroke}{rgb}{0.631373,0.788235,0.956863}%
\pgfsetstrokecolor{currentstroke}%
\pgfsetstrokeopacity{0.200000}%
\pgfsetdash{}{0pt}%
\pgfpathmoveto{\pgfqpoint{5.740650in}{6.419871in}}%
\pgfpathlineto{\pgfqpoint{6.100521in}{4.940903in}}%
\pgfusepath{stroke}%
\end{pgfscope}%
\begin{pgfscope}%
\pgfpathrectangle{\pgfqpoint{0.526127in}{0.331635in}}{\pgfqpoint{9.300000in}{7.700000in}}%
\pgfusepath{clip}%
\pgfsetrectcap%
\pgfsetroundjoin%
\pgfsetlinewidth{1.505625pt}%
\definecolor{currentstroke}{rgb}{0.631373,0.788235,0.956863}%
\pgfsetstrokecolor{currentstroke}%
\pgfsetstrokeopacity{0.200000}%
\pgfsetdash{}{0pt}%
\pgfpathmoveto{\pgfqpoint{7.681588in}{6.092216in}}%
\pgfpathlineto{\pgfqpoint{6.100521in}{4.940903in}}%
\pgfusepath{stroke}%
\end{pgfscope}%
\begin{pgfscope}%
\pgfpathrectangle{\pgfqpoint{0.526127in}{0.331635in}}{\pgfqpoint{9.300000in}{7.700000in}}%
\pgfusepath{clip}%
\pgfsetrectcap%
\pgfsetroundjoin%
\pgfsetlinewidth{1.505625pt}%
\definecolor{currentstroke}{rgb}{0.631373,0.788235,0.956863}%
\pgfsetstrokecolor{currentstroke}%
\pgfsetstrokeopacity{0.200000}%
\pgfsetdash{}{0pt}%
\pgfpathmoveto{\pgfqpoint{6.865837in}{4.629084in}}%
\pgfpathlineto{\pgfqpoint{6.100521in}{4.940903in}}%
\pgfusepath{stroke}%
\end{pgfscope}%
\begin{pgfscope}%
\pgfpathrectangle{\pgfqpoint{0.526127in}{0.331635in}}{\pgfqpoint{9.300000in}{7.700000in}}%
\pgfusepath{clip}%
\pgfsetrectcap%
\pgfsetroundjoin%
\pgfsetlinewidth{1.505625pt}%
\definecolor{currentstroke}{rgb}{0.631373,0.788235,0.956863}%
\pgfsetstrokecolor{currentstroke}%
\pgfsetstrokeopacity{0.200000}%
\pgfsetdash{}{0pt}%
\pgfpathmoveto{\pgfqpoint{7.703805in}{3.296605in}}%
\pgfpathlineto{\pgfqpoint{6.100521in}{4.940903in}}%
\pgfusepath{stroke}%
\end{pgfscope}%
\begin{pgfscope}%
\pgfpathrectangle{\pgfqpoint{0.526127in}{0.331635in}}{\pgfqpoint{9.300000in}{7.700000in}}%
\pgfusepath{clip}%
\pgfsetrectcap%
\pgfsetroundjoin%
\pgfsetlinewidth{1.505625pt}%
\definecolor{currentstroke}{rgb}{0.631373,0.788235,0.956863}%
\pgfsetstrokecolor{currentstroke}%
\pgfsetstrokeopacity{0.200000}%
\pgfsetdash{}{0pt}%
\pgfpathmoveto{\pgfqpoint{2.458995in}{1.029819in}}%
\pgfpathlineto{\pgfqpoint{6.100521in}{4.940903in}}%
\pgfusepath{stroke}%
\end{pgfscope}%
\begin{pgfscope}%
\pgfpathrectangle{\pgfqpoint{0.526127in}{0.331635in}}{\pgfqpoint{9.300000in}{7.700000in}}%
\pgfusepath{clip}%
\pgfsetrectcap%
\pgfsetroundjoin%
\pgfsetlinewidth{1.505625pt}%
\definecolor{currentstroke}{rgb}{0.631373,0.788235,0.956863}%
\pgfsetstrokecolor{currentstroke}%
\pgfsetstrokeopacity{0.200000}%
\pgfsetdash{}{0pt}%
\pgfpathmoveto{\pgfqpoint{4.621174in}{4.057804in}}%
\pgfpathlineto{\pgfqpoint{6.100521in}{4.940903in}}%
\pgfusepath{stroke}%
\end{pgfscope}%
\begin{pgfscope}%
\pgfpathrectangle{\pgfqpoint{0.526127in}{0.331635in}}{\pgfqpoint{9.300000in}{7.700000in}}%
\pgfusepath{clip}%
\pgfsetrectcap%
\pgfsetroundjoin%
\pgfsetlinewidth{1.505625pt}%
\definecolor{currentstroke}{rgb}{0.631373,0.788235,0.956863}%
\pgfsetstrokecolor{currentstroke}%
\pgfsetstrokeopacity{0.200000}%
\pgfsetdash{}{0pt}%
\pgfpathmoveto{\pgfqpoint{5.970043in}{6.897752in}}%
\pgfpathlineto{\pgfqpoint{6.100521in}{4.940903in}}%
\pgfusepath{stroke}%
\end{pgfscope}%
\begin{pgfscope}%
\pgfpathrectangle{\pgfqpoint{0.526127in}{0.331635in}}{\pgfqpoint{9.300000in}{7.700000in}}%
\pgfusepath{clip}%
\pgfsetrectcap%
\pgfsetroundjoin%
\pgfsetlinewidth{1.505625pt}%
\definecolor{currentstroke}{rgb}{0.631373,0.788235,0.956863}%
\pgfsetstrokecolor{currentstroke}%
\pgfsetstrokeopacity{0.200000}%
\pgfsetdash{}{0pt}%
\pgfpathmoveto{\pgfqpoint{5.368377in}{3.410644in}}%
\pgfpathlineto{\pgfqpoint{6.100521in}{4.940903in}}%
\pgfusepath{stroke}%
\end{pgfscope}%
\begin{pgfscope}%
\pgfpathrectangle{\pgfqpoint{0.526127in}{0.331635in}}{\pgfqpoint{9.300000in}{7.700000in}}%
\pgfusepath{clip}%
\pgfsetrectcap%
\pgfsetroundjoin%
\pgfsetlinewidth{1.505625pt}%
\definecolor{currentstroke}{rgb}{0.631373,0.788235,0.956863}%
\pgfsetstrokecolor{currentstroke}%
\pgfsetstrokeopacity{0.200000}%
\pgfsetdash{}{0pt}%
\pgfpathmoveto{\pgfqpoint{6.364240in}{6.604101in}}%
\pgfpathlineto{\pgfqpoint{6.100521in}{4.940903in}}%
\pgfusepath{stroke}%
\end{pgfscope}%
\begin{pgfscope}%
\pgfpathrectangle{\pgfqpoint{0.526127in}{0.331635in}}{\pgfqpoint{9.300000in}{7.700000in}}%
\pgfusepath{clip}%
\pgfsetrectcap%
\pgfsetroundjoin%
\pgfsetlinewidth{1.505625pt}%
\definecolor{currentstroke}{rgb}{0.631373,0.788235,0.956863}%
\pgfsetstrokecolor{currentstroke}%
\pgfsetstrokeopacity{0.200000}%
\pgfsetdash{}{0pt}%
\pgfpathmoveto{\pgfqpoint{6.258495in}{6.814935in}}%
\pgfpathlineto{\pgfqpoint{6.100521in}{4.940903in}}%
\pgfusepath{stroke}%
\end{pgfscope}%
\begin{pgfscope}%
\pgfpathrectangle{\pgfqpoint{0.526127in}{0.331635in}}{\pgfqpoint{9.300000in}{7.700000in}}%
\pgfusepath{clip}%
\pgfsetrectcap%
\pgfsetroundjoin%
\pgfsetlinewidth{1.505625pt}%
\definecolor{currentstroke}{rgb}{0.631373,0.788235,0.956863}%
\pgfsetstrokecolor{currentstroke}%
\pgfsetstrokeopacity{0.200000}%
\pgfsetdash{}{0pt}%
\pgfpathmoveto{\pgfqpoint{7.197145in}{4.694607in}}%
\pgfpathlineto{\pgfqpoint{6.100521in}{4.940903in}}%
\pgfusepath{stroke}%
\end{pgfscope}%
\begin{pgfscope}%
\pgfpathrectangle{\pgfqpoint{0.526127in}{0.331635in}}{\pgfqpoint{9.300000in}{7.700000in}}%
\pgfusepath{clip}%
\pgfsetrectcap%
\pgfsetroundjoin%
\pgfsetlinewidth{1.505625pt}%
\definecolor{currentstroke}{rgb}{0.631373,0.788235,0.956863}%
\pgfsetstrokecolor{currentstroke}%
\pgfsetstrokeopacity{0.200000}%
\pgfsetdash{}{0pt}%
\pgfpathmoveto{\pgfqpoint{7.051687in}{6.247476in}}%
\pgfpathlineto{\pgfqpoint{6.100521in}{4.940903in}}%
\pgfusepath{stroke}%
\end{pgfscope}%
\begin{pgfscope}%
\pgfpathrectangle{\pgfqpoint{0.526127in}{0.331635in}}{\pgfqpoint{9.300000in}{7.700000in}}%
\pgfusepath{clip}%
\pgfsetrectcap%
\pgfsetroundjoin%
\pgfsetlinewidth{1.505625pt}%
\definecolor{currentstroke}{rgb}{0.631373,0.788235,0.956863}%
\pgfsetstrokecolor{currentstroke}%
\pgfsetstrokeopacity{0.200000}%
\pgfsetdash{}{0pt}%
\pgfpathmoveto{\pgfqpoint{6.775438in}{7.238564in}}%
\pgfpathlineto{\pgfqpoint{6.100521in}{4.940903in}}%
\pgfusepath{stroke}%
\end{pgfscope}%
\begin{pgfscope}%
\pgfpathrectangle{\pgfqpoint{0.526127in}{0.331635in}}{\pgfqpoint{9.300000in}{7.700000in}}%
\pgfusepath{clip}%
\pgfsetrectcap%
\pgfsetroundjoin%
\pgfsetlinewidth{1.505625pt}%
\definecolor{currentstroke}{rgb}{0.631373,0.788235,0.956863}%
\pgfsetstrokecolor{currentstroke}%
\pgfsetstrokeopacity{0.200000}%
\pgfsetdash{}{0pt}%
\pgfpathmoveto{\pgfqpoint{4.179928in}{4.106096in}}%
\pgfpathlineto{\pgfqpoint{6.100521in}{4.940903in}}%
\pgfusepath{stroke}%
\end{pgfscope}%
\begin{pgfscope}%
\pgfpathrectangle{\pgfqpoint{0.526127in}{0.331635in}}{\pgfqpoint{9.300000in}{7.700000in}}%
\pgfusepath{clip}%
\pgfsetrectcap%
\pgfsetroundjoin%
\pgfsetlinewidth{1.505625pt}%
\definecolor{currentstroke}{rgb}{0.631373,0.788235,0.956863}%
\pgfsetstrokecolor{currentstroke}%
\pgfsetstrokeopacity{0.200000}%
\pgfsetdash{}{0pt}%
\pgfpathmoveto{\pgfqpoint{8.098361in}{3.070197in}}%
\pgfpathlineto{\pgfqpoint{6.100521in}{4.940903in}}%
\pgfusepath{stroke}%
\end{pgfscope}%
\begin{pgfscope}%
\pgfpathrectangle{\pgfqpoint{0.526127in}{0.331635in}}{\pgfqpoint{9.300000in}{7.700000in}}%
\pgfusepath{clip}%
\pgfsetrectcap%
\pgfsetroundjoin%
\pgfsetlinewidth{1.505625pt}%
\definecolor{currentstroke}{rgb}{1.000000,0.705882,0.509804}%
\pgfsetstrokecolor{currentstroke}%
\pgfsetstrokeopacity{0.200000}%
\pgfsetdash{}{0pt}%
\pgfpathmoveto{\pgfqpoint{6.611324in}{4.129584in}}%
\pgfpathlineto{\pgfqpoint{6.411605in}{4.442724in}}%
\pgfusepath{stroke}%
\end{pgfscope}%
\begin{pgfscope}%
\pgfpathrectangle{\pgfqpoint{0.526127in}{0.331635in}}{\pgfqpoint{9.300000in}{7.700000in}}%
\pgfusepath{clip}%
\pgfsetrectcap%
\pgfsetroundjoin%
\pgfsetlinewidth{1.505625pt}%
\definecolor{currentstroke}{rgb}{1.000000,0.705882,0.509804}%
\pgfsetstrokecolor{currentstroke}%
\pgfsetstrokeopacity{0.200000}%
\pgfsetdash{}{0pt}%
\pgfpathmoveto{\pgfqpoint{4.265030in}{7.009476in}}%
\pgfpathlineto{\pgfqpoint{6.411605in}{4.442724in}}%
\pgfusepath{stroke}%
\end{pgfscope}%
\begin{pgfscope}%
\pgfpathrectangle{\pgfqpoint{0.526127in}{0.331635in}}{\pgfqpoint{9.300000in}{7.700000in}}%
\pgfusepath{clip}%
\pgfsetrectcap%
\pgfsetroundjoin%
\pgfsetlinewidth{1.505625pt}%
\definecolor{currentstroke}{rgb}{1.000000,0.705882,0.509804}%
\pgfsetstrokecolor{currentstroke}%
\pgfsetstrokeopacity{0.200000}%
\pgfsetdash{}{0pt}%
\pgfpathmoveto{\pgfqpoint{5.530805in}{6.568416in}}%
\pgfpathlineto{\pgfqpoint{6.411605in}{4.442724in}}%
\pgfusepath{stroke}%
\end{pgfscope}%
\begin{pgfscope}%
\pgfpathrectangle{\pgfqpoint{0.526127in}{0.331635in}}{\pgfqpoint{9.300000in}{7.700000in}}%
\pgfusepath{clip}%
\pgfsetrectcap%
\pgfsetroundjoin%
\pgfsetlinewidth{1.505625pt}%
\definecolor{currentstroke}{rgb}{1.000000,0.705882,0.509804}%
\pgfsetstrokecolor{currentstroke}%
\pgfsetstrokeopacity{0.200000}%
\pgfsetdash{}{0pt}%
\pgfpathmoveto{\pgfqpoint{6.584340in}{5.732668in}}%
\pgfpathlineto{\pgfqpoint{6.411605in}{4.442724in}}%
\pgfusepath{stroke}%
\end{pgfscope}%
\begin{pgfscope}%
\pgfpathrectangle{\pgfqpoint{0.526127in}{0.331635in}}{\pgfqpoint{9.300000in}{7.700000in}}%
\pgfusepath{clip}%
\pgfsetrectcap%
\pgfsetroundjoin%
\pgfsetlinewidth{1.505625pt}%
\definecolor{currentstroke}{rgb}{1.000000,0.705882,0.509804}%
\pgfsetstrokecolor{currentstroke}%
\pgfsetstrokeopacity{0.200000}%
\pgfsetdash{}{0pt}%
\pgfpathmoveto{\pgfqpoint{6.306902in}{2.521520in}}%
\pgfpathlineto{\pgfqpoint{6.411605in}{4.442724in}}%
\pgfusepath{stroke}%
\end{pgfscope}%
\begin{pgfscope}%
\pgfpathrectangle{\pgfqpoint{0.526127in}{0.331635in}}{\pgfqpoint{9.300000in}{7.700000in}}%
\pgfusepath{clip}%
\pgfsetrectcap%
\pgfsetroundjoin%
\pgfsetlinewidth{1.505625pt}%
\definecolor{currentstroke}{rgb}{1.000000,0.705882,0.509804}%
\pgfsetstrokecolor{currentstroke}%
\pgfsetstrokeopacity{0.200000}%
\pgfsetdash{}{0pt}%
\pgfpathmoveto{\pgfqpoint{7.529402in}{3.720872in}}%
\pgfpathlineto{\pgfqpoint{6.411605in}{4.442724in}}%
\pgfusepath{stroke}%
\end{pgfscope}%
\begin{pgfscope}%
\pgfpathrectangle{\pgfqpoint{0.526127in}{0.331635in}}{\pgfqpoint{9.300000in}{7.700000in}}%
\pgfusepath{clip}%
\pgfsetrectcap%
\pgfsetroundjoin%
\pgfsetlinewidth{1.505625pt}%
\definecolor{currentstroke}{rgb}{1.000000,0.705882,0.509804}%
\pgfsetstrokecolor{currentstroke}%
\pgfsetstrokeopacity{0.200000}%
\pgfsetdash{}{0pt}%
\pgfpathmoveto{\pgfqpoint{7.857488in}{3.672391in}}%
\pgfpathlineto{\pgfqpoint{6.411605in}{4.442724in}}%
\pgfusepath{stroke}%
\end{pgfscope}%
\begin{pgfscope}%
\pgfpathrectangle{\pgfqpoint{0.526127in}{0.331635in}}{\pgfqpoint{9.300000in}{7.700000in}}%
\pgfusepath{clip}%
\pgfsetrectcap%
\pgfsetroundjoin%
\pgfsetlinewidth{1.505625pt}%
\definecolor{currentstroke}{rgb}{1.000000,0.705882,0.509804}%
\pgfsetstrokecolor{currentstroke}%
\pgfsetstrokeopacity{0.200000}%
\pgfsetdash{}{0pt}%
\pgfpathmoveto{\pgfqpoint{6.572528in}{4.224991in}}%
\pgfpathlineto{\pgfqpoint{6.411605in}{4.442724in}}%
\pgfusepath{stroke}%
\end{pgfscope}%
\begin{pgfscope}%
\pgfpathrectangle{\pgfqpoint{0.526127in}{0.331635in}}{\pgfqpoint{9.300000in}{7.700000in}}%
\pgfusepath{clip}%
\pgfsetrectcap%
\pgfsetroundjoin%
\pgfsetlinewidth{1.505625pt}%
\definecolor{currentstroke}{rgb}{1.000000,0.705882,0.509804}%
\pgfsetstrokecolor{currentstroke}%
\pgfsetstrokeopacity{0.200000}%
\pgfsetdash{}{0pt}%
\pgfpathmoveto{\pgfqpoint{5.154620in}{3.765894in}}%
\pgfpathlineto{\pgfqpoint{6.411605in}{4.442724in}}%
\pgfusepath{stroke}%
\end{pgfscope}%
\begin{pgfscope}%
\pgfpathrectangle{\pgfqpoint{0.526127in}{0.331635in}}{\pgfqpoint{9.300000in}{7.700000in}}%
\pgfusepath{clip}%
\pgfsetrectcap%
\pgfsetroundjoin%
\pgfsetlinewidth{1.505625pt}%
\definecolor{currentstroke}{rgb}{1.000000,0.705882,0.509804}%
\pgfsetstrokecolor{currentstroke}%
\pgfsetstrokeopacity{0.200000}%
\pgfsetdash{}{0pt}%
\pgfpathmoveto{\pgfqpoint{5.190149in}{3.839314in}}%
\pgfpathlineto{\pgfqpoint{6.411605in}{4.442724in}}%
\pgfusepath{stroke}%
\end{pgfscope}%
\begin{pgfscope}%
\pgfpathrectangle{\pgfqpoint{0.526127in}{0.331635in}}{\pgfqpoint{9.300000in}{7.700000in}}%
\pgfusepath{clip}%
\pgfsetrectcap%
\pgfsetroundjoin%
\pgfsetlinewidth{1.505625pt}%
\definecolor{currentstroke}{rgb}{1.000000,0.705882,0.509804}%
\pgfsetstrokecolor{currentstroke}%
\pgfsetstrokeopacity{0.200000}%
\pgfsetdash{}{0pt}%
\pgfpathmoveto{\pgfqpoint{6.190070in}{4.309824in}}%
\pgfpathlineto{\pgfqpoint{6.411605in}{4.442724in}}%
\pgfusepath{stroke}%
\end{pgfscope}%
\begin{pgfscope}%
\pgfpathrectangle{\pgfqpoint{0.526127in}{0.331635in}}{\pgfqpoint{9.300000in}{7.700000in}}%
\pgfusepath{clip}%
\pgfsetrectcap%
\pgfsetroundjoin%
\pgfsetlinewidth{1.505625pt}%
\definecolor{currentstroke}{rgb}{1.000000,0.705882,0.509804}%
\pgfsetstrokecolor{currentstroke}%
\pgfsetstrokeopacity{0.200000}%
\pgfsetdash{}{0pt}%
\pgfpathmoveto{\pgfqpoint{6.008150in}{4.992055in}}%
\pgfpathlineto{\pgfqpoint{6.411605in}{4.442724in}}%
\pgfusepath{stroke}%
\end{pgfscope}%
\begin{pgfscope}%
\pgfpathrectangle{\pgfqpoint{0.526127in}{0.331635in}}{\pgfqpoint{9.300000in}{7.700000in}}%
\pgfusepath{clip}%
\pgfsetrectcap%
\pgfsetroundjoin%
\pgfsetlinewidth{1.505625pt}%
\definecolor{currentstroke}{rgb}{1.000000,0.705882,0.509804}%
\pgfsetstrokecolor{currentstroke}%
\pgfsetstrokeopacity{0.200000}%
\pgfsetdash{}{0pt}%
\pgfpathmoveto{\pgfqpoint{3.566645in}{0.681635in}}%
\pgfpathlineto{\pgfqpoint{6.411605in}{4.442724in}}%
\pgfusepath{stroke}%
\end{pgfscope}%
\begin{pgfscope}%
\pgfpathrectangle{\pgfqpoint{0.526127in}{0.331635in}}{\pgfqpoint{9.300000in}{7.700000in}}%
\pgfusepath{clip}%
\pgfsetrectcap%
\pgfsetroundjoin%
\pgfsetlinewidth{1.505625pt}%
\definecolor{currentstroke}{rgb}{1.000000,0.705882,0.509804}%
\pgfsetstrokecolor{currentstroke}%
\pgfsetstrokeopacity{0.200000}%
\pgfsetdash{}{0pt}%
\pgfpathmoveto{\pgfqpoint{7.924332in}{3.471043in}}%
\pgfpathlineto{\pgfqpoint{6.411605in}{4.442724in}}%
\pgfusepath{stroke}%
\end{pgfscope}%
\begin{pgfscope}%
\pgfpathrectangle{\pgfqpoint{0.526127in}{0.331635in}}{\pgfqpoint{9.300000in}{7.700000in}}%
\pgfusepath{clip}%
\pgfsetrectcap%
\pgfsetroundjoin%
\pgfsetlinewidth{1.505625pt}%
\definecolor{currentstroke}{rgb}{1.000000,0.705882,0.509804}%
\pgfsetstrokecolor{currentstroke}%
\pgfsetstrokeopacity{0.200000}%
\pgfsetdash{}{0pt}%
\pgfpathmoveto{\pgfqpoint{6.132760in}{3.874810in}}%
\pgfpathlineto{\pgfqpoint{6.411605in}{4.442724in}}%
\pgfusepath{stroke}%
\end{pgfscope}%
\begin{pgfscope}%
\pgfpathrectangle{\pgfqpoint{0.526127in}{0.331635in}}{\pgfqpoint{9.300000in}{7.700000in}}%
\pgfusepath{clip}%
\pgfsetrectcap%
\pgfsetroundjoin%
\pgfsetlinewidth{1.505625pt}%
\definecolor{currentstroke}{rgb}{1.000000,0.705882,0.509804}%
\pgfsetstrokecolor{currentstroke}%
\pgfsetstrokeopacity{0.200000}%
\pgfsetdash{}{0pt}%
\pgfpathmoveto{\pgfqpoint{6.193019in}{4.773366in}}%
\pgfpathlineto{\pgfqpoint{6.411605in}{4.442724in}}%
\pgfusepath{stroke}%
\end{pgfscope}%
\begin{pgfscope}%
\pgfpathrectangle{\pgfqpoint{0.526127in}{0.331635in}}{\pgfqpoint{9.300000in}{7.700000in}}%
\pgfusepath{clip}%
\pgfsetrectcap%
\pgfsetroundjoin%
\pgfsetlinewidth{1.505625pt}%
\definecolor{currentstroke}{rgb}{1.000000,0.705882,0.509804}%
\pgfsetstrokecolor{currentstroke}%
\pgfsetstrokeopacity{0.200000}%
\pgfsetdash{}{0pt}%
\pgfpathmoveto{\pgfqpoint{6.027104in}{5.627514in}}%
\pgfpathlineto{\pgfqpoint{6.411605in}{4.442724in}}%
\pgfusepath{stroke}%
\end{pgfscope}%
\begin{pgfscope}%
\pgfpathrectangle{\pgfqpoint{0.526127in}{0.331635in}}{\pgfqpoint{9.300000in}{7.700000in}}%
\pgfusepath{clip}%
\pgfsetrectcap%
\pgfsetroundjoin%
\pgfsetlinewidth{1.505625pt}%
\definecolor{currentstroke}{rgb}{1.000000,0.705882,0.509804}%
\pgfsetstrokecolor{currentstroke}%
\pgfsetstrokeopacity{0.200000}%
\pgfsetdash{}{0pt}%
\pgfpathmoveto{\pgfqpoint{6.988150in}{5.568616in}}%
\pgfpathlineto{\pgfqpoint{6.411605in}{4.442724in}}%
\pgfusepath{stroke}%
\end{pgfscope}%
\begin{pgfscope}%
\pgfpathrectangle{\pgfqpoint{0.526127in}{0.331635in}}{\pgfqpoint{9.300000in}{7.700000in}}%
\pgfusepath{clip}%
\pgfsetrectcap%
\pgfsetroundjoin%
\pgfsetlinewidth{1.505625pt}%
\definecolor{currentstroke}{rgb}{1.000000,0.705882,0.509804}%
\pgfsetstrokecolor{currentstroke}%
\pgfsetstrokeopacity{0.200000}%
\pgfsetdash{}{0pt}%
\pgfpathmoveto{\pgfqpoint{6.794558in}{3.554767in}}%
\pgfpathlineto{\pgfqpoint{6.411605in}{4.442724in}}%
\pgfusepath{stroke}%
\end{pgfscope}%
\begin{pgfscope}%
\pgfpathrectangle{\pgfqpoint{0.526127in}{0.331635in}}{\pgfqpoint{9.300000in}{7.700000in}}%
\pgfusepath{clip}%
\pgfsetrectcap%
\pgfsetroundjoin%
\pgfsetlinewidth{1.505625pt}%
\definecolor{currentstroke}{rgb}{1.000000,0.705882,0.509804}%
\pgfsetstrokecolor{currentstroke}%
\pgfsetstrokeopacity{0.200000}%
\pgfsetdash{}{0pt}%
\pgfpathmoveto{\pgfqpoint{7.563920in}{3.455813in}}%
\pgfpathlineto{\pgfqpoint{6.411605in}{4.442724in}}%
\pgfusepath{stroke}%
\end{pgfscope}%
\begin{pgfscope}%
\pgfpathrectangle{\pgfqpoint{0.526127in}{0.331635in}}{\pgfqpoint{9.300000in}{7.700000in}}%
\pgfusepath{clip}%
\pgfsetrectcap%
\pgfsetroundjoin%
\pgfsetlinewidth{1.505625pt}%
\definecolor{currentstroke}{rgb}{1.000000,0.705882,0.509804}%
\pgfsetstrokecolor{currentstroke}%
\pgfsetstrokeopacity{0.200000}%
\pgfsetdash{}{0pt}%
\pgfpathmoveto{\pgfqpoint{5.276797in}{5.750429in}}%
\pgfpathlineto{\pgfqpoint{6.411605in}{4.442724in}}%
\pgfusepath{stroke}%
\end{pgfscope}%
\begin{pgfscope}%
\pgfpathrectangle{\pgfqpoint{0.526127in}{0.331635in}}{\pgfqpoint{9.300000in}{7.700000in}}%
\pgfusepath{clip}%
\pgfsetrectcap%
\pgfsetroundjoin%
\pgfsetlinewidth{1.505625pt}%
\definecolor{currentstroke}{rgb}{1.000000,0.705882,0.509804}%
\pgfsetstrokecolor{currentstroke}%
\pgfsetstrokeopacity{0.200000}%
\pgfsetdash{}{0pt}%
\pgfpathmoveto{\pgfqpoint{7.155517in}{3.710876in}}%
\pgfpathlineto{\pgfqpoint{6.411605in}{4.442724in}}%
\pgfusepath{stroke}%
\end{pgfscope}%
\begin{pgfscope}%
\pgfpathrectangle{\pgfqpoint{0.526127in}{0.331635in}}{\pgfqpoint{9.300000in}{7.700000in}}%
\pgfusepath{clip}%
\pgfsetrectcap%
\pgfsetroundjoin%
\pgfsetlinewidth{1.505625pt}%
\definecolor{currentstroke}{rgb}{1.000000,0.705882,0.509804}%
\pgfsetstrokecolor{currentstroke}%
\pgfsetstrokeopacity{0.200000}%
\pgfsetdash{}{0pt}%
\pgfpathmoveto{\pgfqpoint{6.008087in}{6.401709in}}%
\pgfpathlineto{\pgfqpoint{6.411605in}{4.442724in}}%
\pgfusepath{stroke}%
\end{pgfscope}%
\begin{pgfscope}%
\pgfpathrectangle{\pgfqpoint{0.526127in}{0.331635in}}{\pgfqpoint{9.300000in}{7.700000in}}%
\pgfusepath{clip}%
\pgfsetrectcap%
\pgfsetroundjoin%
\pgfsetlinewidth{1.505625pt}%
\definecolor{currentstroke}{rgb}{1.000000,0.705882,0.509804}%
\pgfsetstrokecolor{currentstroke}%
\pgfsetstrokeopacity{0.200000}%
\pgfsetdash{}{0pt}%
\pgfpathmoveto{\pgfqpoint{7.439477in}{5.289143in}}%
\pgfpathlineto{\pgfqpoint{6.411605in}{4.442724in}}%
\pgfusepath{stroke}%
\end{pgfscope}%
\begin{pgfscope}%
\pgfpathrectangle{\pgfqpoint{0.526127in}{0.331635in}}{\pgfqpoint{9.300000in}{7.700000in}}%
\pgfusepath{clip}%
\pgfsetrectcap%
\pgfsetroundjoin%
\pgfsetlinewidth{1.505625pt}%
\definecolor{currentstroke}{rgb}{1.000000,0.705882,0.509804}%
\pgfsetstrokecolor{currentstroke}%
\pgfsetstrokeopacity{0.200000}%
\pgfsetdash{}{0pt}%
\pgfpathmoveto{\pgfqpoint{7.682374in}{4.875449in}}%
\pgfpathlineto{\pgfqpoint{6.411605in}{4.442724in}}%
\pgfusepath{stroke}%
\end{pgfscope}%
\begin{pgfscope}%
\pgfpathrectangle{\pgfqpoint{0.526127in}{0.331635in}}{\pgfqpoint{9.300000in}{7.700000in}}%
\pgfusepath{clip}%
\pgfsetrectcap%
\pgfsetroundjoin%
\pgfsetlinewidth{1.505625pt}%
\definecolor{currentstroke}{rgb}{1.000000,0.705882,0.509804}%
\pgfsetstrokecolor{currentstroke}%
\pgfsetstrokeopacity{0.200000}%
\pgfsetdash{}{0pt}%
\pgfpathmoveto{\pgfqpoint{5.929161in}{6.127529in}}%
\pgfpathlineto{\pgfqpoint{6.411605in}{4.442724in}}%
\pgfusepath{stroke}%
\end{pgfscope}%
\begin{pgfscope}%
\pgfpathrectangle{\pgfqpoint{0.526127in}{0.331635in}}{\pgfqpoint{9.300000in}{7.700000in}}%
\pgfusepath{clip}%
\pgfsetrectcap%
\pgfsetroundjoin%
\pgfsetlinewidth{1.505625pt}%
\definecolor{currentstroke}{rgb}{1.000000,0.705882,0.509804}%
\pgfsetstrokecolor{currentstroke}%
\pgfsetstrokeopacity{0.200000}%
\pgfsetdash{}{0pt}%
\pgfpathmoveto{\pgfqpoint{8.724020in}{3.300311in}}%
\pgfpathlineto{\pgfqpoint{6.411605in}{4.442724in}}%
\pgfusepath{stroke}%
\end{pgfscope}%
\begin{pgfscope}%
\pgfpathrectangle{\pgfqpoint{0.526127in}{0.331635in}}{\pgfqpoint{9.300000in}{7.700000in}}%
\pgfusepath{clip}%
\pgfsetrectcap%
\pgfsetroundjoin%
\pgfsetlinewidth{1.505625pt}%
\definecolor{currentstroke}{rgb}{1.000000,0.705882,0.509804}%
\pgfsetstrokecolor{currentstroke}%
\pgfsetstrokeopacity{0.200000}%
\pgfsetdash{}{0pt}%
\pgfpathmoveto{\pgfqpoint{6.318203in}{3.446246in}}%
\pgfpathlineto{\pgfqpoint{6.411605in}{4.442724in}}%
\pgfusepath{stroke}%
\end{pgfscope}%
\begin{pgfscope}%
\pgfpathrectangle{\pgfqpoint{0.526127in}{0.331635in}}{\pgfqpoint{9.300000in}{7.700000in}}%
\pgfusepath{clip}%
\pgfsetrectcap%
\pgfsetroundjoin%
\pgfsetlinewidth{1.505625pt}%
\definecolor{currentstroke}{rgb}{0.552941,0.898039,0.631373}%
\pgfsetstrokecolor{currentstroke}%
\pgfsetstrokeopacity{0.200000}%
\pgfsetdash{}{0pt}%
\pgfpathmoveto{\pgfqpoint{7.719822in}{2.287277in}}%
\pgfpathlineto{\pgfqpoint{4.400818in}{2.895337in}}%
\pgfusepath{stroke}%
\end{pgfscope}%
\begin{pgfscope}%
\pgfpathrectangle{\pgfqpoint{0.526127in}{0.331635in}}{\pgfqpoint{9.300000in}{7.700000in}}%
\pgfusepath{clip}%
\pgfsetrectcap%
\pgfsetroundjoin%
\pgfsetlinewidth{1.505625pt}%
\definecolor{currentstroke}{rgb}{0.552941,0.898039,0.631373}%
\pgfsetstrokecolor{currentstroke}%
\pgfsetstrokeopacity{0.200000}%
\pgfsetdash{}{0pt}%
\pgfpathmoveto{\pgfqpoint{5.865422in}{1.611223in}}%
\pgfpathlineto{\pgfqpoint{4.400818in}{2.895337in}}%
\pgfusepath{stroke}%
\end{pgfscope}%
\begin{pgfscope}%
\pgfpathrectangle{\pgfqpoint{0.526127in}{0.331635in}}{\pgfqpoint{9.300000in}{7.700000in}}%
\pgfusepath{clip}%
\pgfsetrectcap%
\pgfsetroundjoin%
\pgfsetlinewidth{1.505625pt}%
\definecolor{currentstroke}{rgb}{0.552941,0.898039,0.631373}%
\pgfsetstrokecolor{currentstroke}%
\pgfsetstrokeopacity{0.200000}%
\pgfsetdash{}{0pt}%
\pgfpathmoveto{\pgfqpoint{6.692745in}{2.208927in}}%
\pgfpathlineto{\pgfqpoint{4.400818in}{2.895337in}}%
\pgfusepath{stroke}%
\end{pgfscope}%
\begin{pgfscope}%
\pgfpathrectangle{\pgfqpoint{0.526127in}{0.331635in}}{\pgfqpoint{9.300000in}{7.700000in}}%
\pgfusepath{clip}%
\pgfsetrectcap%
\pgfsetroundjoin%
\pgfsetlinewidth{1.505625pt}%
\definecolor{currentstroke}{rgb}{0.552941,0.898039,0.631373}%
\pgfsetstrokecolor{currentstroke}%
\pgfsetstrokeopacity{0.200000}%
\pgfsetdash{}{0pt}%
\pgfpathmoveto{\pgfqpoint{2.147474in}{3.396130in}}%
\pgfpathlineto{\pgfqpoint{4.400818in}{2.895337in}}%
\pgfusepath{stroke}%
\end{pgfscope}%
\begin{pgfscope}%
\pgfpathrectangle{\pgfqpoint{0.526127in}{0.331635in}}{\pgfqpoint{9.300000in}{7.700000in}}%
\pgfusepath{clip}%
\pgfsetrectcap%
\pgfsetroundjoin%
\pgfsetlinewidth{1.505625pt}%
\definecolor{currentstroke}{rgb}{0.552941,0.898039,0.631373}%
\pgfsetstrokecolor{currentstroke}%
\pgfsetstrokeopacity{0.200000}%
\pgfsetdash{}{0pt}%
\pgfpathmoveto{\pgfqpoint{7.116859in}{3.140994in}}%
\pgfpathlineto{\pgfqpoint{4.400818in}{2.895337in}}%
\pgfusepath{stroke}%
\end{pgfscope}%
\begin{pgfscope}%
\pgfpathrectangle{\pgfqpoint{0.526127in}{0.331635in}}{\pgfqpoint{9.300000in}{7.700000in}}%
\pgfusepath{clip}%
\pgfsetrectcap%
\pgfsetroundjoin%
\pgfsetlinewidth{1.505625pt}%
\definecolor{currentstroke}{rgb}{0.552941,0.898039,0.631373}%
\pgfsetstrokecolor{currentstroke}%
\pgfsetstrokeopacity{0.200000}%
\pgfsetdash{}{0pt}%
\pgfpathmoveto{\pgfqpoint{3.558492in}{4.145234in}}%
\pgfpathlineto{\pgfqpoint{4.400818in}{2.895337in}}%
\pgfusepath{stroke}%
\end{pgfscope}%
\begin{pgfscope}%
\pgfpathrectangle{\pgfqpoint{0.526127in}{0.331635in}}{\pgfqpoint{9.300000in}{7.700000in}}%
\pgfusepath{clip}%
\pgfsetrectcap%
\pgfsetroundjoin%
\pgfsetlinewidth{1.505625pt}%
\definecolor{currentstroke}{rgb}{0.552941,0.898039,0.631373}%
\pgfsetstrokecolor{currentstroke}%
\pgfsetstrokeopacity{0.200000}%
\pgfsetdash{}{0pt}%
\pgfpathmoveto{\pgfqpoint{3.383051in}{1.569295in}}%
\pgfpathlineto{\pgfqpoint{4.400818in}{2.895337in}}%
\pgfusepath{stroke}%
\end{pgfscope}%
\begin{pgfscope}%
\pgfpathrectangle{\pgfqpoint{0.526127in}{0.331635in}}{\pgfqpoint{9.300000in}{7.700000in}}%
\pgfusepath{clip}%
\pgfsetrectcap%
\pgfsetroundjoin%
\pgfsetlinewidth{1.505625pt}%
\definecolor{currentstroke}{rgb}{0.552941,0.898039,0.631373}%
\pgfsetstrokecolor{currentstroke}%
\pgfsetstrokeopacity{0.200000}%
\pgfsetdash{}{0pt}%
\pgfpathmoveto{\pgfqpoint{6.701250in}{2.606389in}}%
\pgfpathlineto{\pgfqpoint{4.400818in}{2.895337in}}%
\pgfusepath{stroke}%
\end{pgfscope}%
\begin{pgfscope}%
\pgfpathrectangle{\pgfqpoint{0.526127in}{0.331635in}}{\pgfqpoint{9.300000in}{7.700000in}}%
\pgfusepath{clip}%
\pgfsetrectcap%
\pgfsetroundjoin%
\pgfsetlinewidth{1.505625pt}%
\definecolor{currentstroke}{rgb}{0.552941,0.898039,0.631373}%
\pgfsetstrokecolor{currentstroke}%
\pgfsetstrokeopacity{0.200000}%
\pgfsetdash{}{0pt}%
\pgfpathmoveto{\pgfqpoint{6.060531in}{3.426761in}}%
\pgfpathlineto{\pgfqpoint{4.400818in}{2.895337in}}%
\pgfusepath{stroke}%
\end{pgfscope}%
\begin{pgfscope}%
\pgfpathrectangle{\pgfqpoint{0.526127in}{0.331635in}}{\pgfqpoint{9.300000in}{7.700000in}}%
\pgfusepath{clip}%
\pgfsetrectcap%
\pgfsetroundjoin%
\pgfsetlinewidth{1.505625pt}%
\definecolor{currentstroke}{rgb}{0.552941,0.898039,0.631373}%
\pgfsetstrokecolor{currentstroke}%
\pgfsetstrokeopacity{0.200000}%
\pgfsetdash{}{0pt}%
\pgfpathmoveto{\pgfqpoint{3.545333in}{4.514559in}}%
\pgfpathlineto{\pgfqpoint{4.400818in}{2.895337in}}%
\pgfusepath{stroke}%
\end{pgfscope}%
\begin{pgfscope}%
\pgfpathrectangle{\pgfqpoint{0.526127in}{0.331635in}}{\pgfqpoint{9.300000in}{7.700000in}}%
\pgfusepath{clip}%
\pgfsetrectcap%
\pgfsetroundjoin%
\pgfsetlinewidth{1.505625pt}%
\definecolor{currentstroke}{rgb}{0.552941,0.898039,0.631373}%
\pgfsetstrokecolor{currentstroke}%
\pgfsetstrokeopacity{0.200000}%
\pgfsetdash{}{0pt}%
\pgfpathmoveto{\pgfqpoint{2.060713in}{2.922311in}}%
\pgfpathlineto{\pgfqpoint{4.400818in}{2.895337in}}%
\pgfusepath{stroke}%
\end{pgfscope}%
\begin{pgfscope}%
\pgfpathrectangle{\pgfqpoint{0.526127in}{0.331635in}}{\pgfqpoint{9.300000in}{7.700000in}}%
\pgfusepath{clip}%
\pgfsetrectcap%
\pgfsetroundjoin%
\pgfsetlinewidth{1.505625pt}%
\definecolor{currentstroke}{rgb}{0.552941,0.898039,0.631373}%
\pgfsetstrokecolor{currentstroke}%
\pgfsetstrokeopacity{0.200000}%
\pgfsetdash{}{0pt}%
\pgfpathmoveto{\pgfqpoint{9.178578in}{4.703915in}}%
\pgfpathlineto{\pgfqpoint{4.400818in}{2.895337in}}%
\pgfusepath{stroke}%
\end{pgfscope}%
\begin{pgfscope}%
\pgfpathrectangle{\pgfqpoint{0.526127in}{0.331635in}}{\pgfqpoint{9.300000in}{7.700000in}}%
\pgfusepath{clip}%
\pgfsetrectcap%
\pgfsetroundjoin%
\pgfsetlinewidth{1.505625pt}%
\definecolor{currentstroke}{rgb}{0.552941,0.898039,0.631373}%
\pgfsetstrokecolor{currentstroke}%
\pgfsetstrokeopacity{0.200000}%
\pgfsetdash{}{0pt}%
\pgfpathmoveto{\pgfqpoint{2.528674in}{2.455523in}}%
\pgfpathlineto{\pgfqpoint{4.400818in}{2.895337in}}%
\pgfusepath{stroke}%
\end{pgfscope}%
\begin{pgfscope}%
\pgfpathrectangle{\pgfqpoint{0.526127in}{0.331635in}}{\pgfqpoint{9.300000in}{7.700000in}}%
\pgfusepath{clip}%
\pgfsetrectcap%
\pgfsetroundjoin%
\pgfsetlinewidth{1.505625pt}%
\definecolor{currentstroke}{rgb}{0.552941,0.898039,0.631373}%
\pgfsetstrokecolor{currentstroke}%
\pgfsetstrokeopacity{0.200000}%
\pgfsetdash{}{0pt}%
\pgfpathmoveto{\pgfqpoint{2.638599in}{2.966238in}}%
\pgfpathlineto{\pgfqpoint{4.400818in}{2.895337in}}%
\pgfusepath{stroke}%
\end{pgfscope}%
\begin{pgfscope}%
\pgfpathrectangle{\pgfqpoint{0.526127in}{0.331635in}}{\pgfqpoint{9.300000in}{7.700000in}}%
\pgfusepath{clip}%
\pgfsetrectcap%
\pgfsetroundjoin%
\pgfsetlinewidth{1.505625pt}%
\definecolor{currentstroke}{rgb}{0.552941,0.898039,0.631373}%
\pgfsetstrokecolor{currentstroke}%
\pgfsetstrokeopacity{0.200000}%
\pgfsetdash{}{0pt}%
\pgfpathmoveto{\pgfqpoint{2.718231in}{2.271722in}}%
\pgfpathlineto{\pgfqpoint{4.400818in}{2.895337in}}%
\pgfusepath{stroke}%
\end{pgfscope}%
\begin{pgfscope}%
\pgfpathrectangle{\pgfqpoint{0.526127in}{0.331635in}}{\pgfqpoint{9.300000in}{7.700000in}}%
\pgfusepath{clip}%
\pgfsetrectcap%
\pgfsetroundjoin%
\pgfsetlinewidth{1.505625pt}%
\definecolor{currentstroke}{rgb}{0.552941,0.898039,0.631373}%
\pgfsetstrokecolor{currentstroke}%
\pgfsetstrokeopacity{0.200000}%
\pgfsetdash{}{0pt}%
\pgfpathmoveto{\pgfqpoint{1.976182in}{2.522263in}}%
\pgfpathlineto{\pgfqpoint{4.400818in}{2.895337in}}%
\pgfusepath{stroke}%
\end{pgfscope}%
\begin{pgfscope}%
\pgfpathrectangle{\pgfqpoint{0.526127in}{0.331635in}}{\pgfqpoint{9.300000in}{7.700000in}}%
\pgfusepath{clip}%
\pgfsetrectcap%
\pgfsetroundjoin%
\pgfsetlinewidth{1.505625pt}%
\definecolor{currentstroke}{rgb}{0.552941,0.898039,0.631373}%
\pgfsetstrokecolor{currentstroke}%
\pgfsetstrokeopacity{0.200000}%
\pgfsetdash{}{0pt}%
\pgfpathmoveto{\pgfqpoint{2.309460in}{2.915022in}}%
\pgfpathlineto{\pgfqpoint{4.400818in}{2.895337in}}%
\pgfusepath{stroke}%
\end{pgfscope}%
\begin{pgfscope}%
\pgfpathrectangle{\pgfqpoint{0.526127in}{0.331635in}}{\pgfqpoint{9.300000in}{7.700000in}}%
\pgfusepath{clip}%
\pgfsetrectcap%
\pgfsetroundjoin%
\pgfsetlinewidth{1.505625pt}%
\definecolor{currentstroke}{rgb}{0.552941,0.898039,0.631373}%
\pgfsetstrokecolor{currentstroke}%
\pgfsetstrokeopacity{0.200000}%
\pgfsetdash{}{0pt}%
\pgfpathmoveto{\pgfqpoint{1.801343in}{3.506841in}}%
\pgfpathlineto{\pgfqpoint{4.400818in}{2.895337in}}%
\pgfusepath{stroke}%
\end{pgfscope}%
\begin{pgfscope}%
\pgfpathrectangle{\pgfqpoint{0.526127in}{0.331635in}}{\pgfqpoint{9.300000in}{7.700000in}}%
\pgfusepath{clip}%
\pgfsetrectcap%
\pgfsetroundjoin%
\pgfsetlinewidth{1.505625pt}%
\definecolor{currentstroke}{rgb}{0.552941,0.898039,0.631373}%
\pgfsetstrokecolor{currentstroke}%
\pgfsetstrokeopacity{0.200000}%
\pgfsetdash{}{0pt}%
\pgfpathmoveto{\pgfqpoint{2.821384in}{2.813377in}}%
\pgfpathlineto{\pgfqpoint{4.400818in}{2.895337in}}%
\pgfusepath{stroke}%
\end{pgfscope}%
\begin{pgfscope}%
\pgfpathrectangle{\pgfqpoint{0.526127in}{0.331635in}}{\pgfqpoint{9.300000in}{7.700000in}}%
\pgfusepath{clip}%
\pgfsetrectcap%
\pgfsetroundjoin%
\pgfsetlinewidth{1.505625pt}%
\definecolor{currentstroke}{rgb}{0.552941,0.898039,0.631373}%
\pgfsetstrokecolor{currentstroke}%
\pgfsetstrokeopacity{0.200000}%
\pgfsetdash{}{0pt}%
\pgfpathmoveto{\pgfqpoint{7.035760in}{1.052321in}}%
\pgfpathlineto{\pgfqpoint{4.400818in}{2.895337in}}%
\pgfusepath{stroke}%
\end{pgfscope}%
\begin{pgfscope}%
\pgfpathrectangle{\pgfqpoint{0.526127in}{0.331635in}}{\pgfqpoint{9.300000in}{7.700000in}}%
\pgfusepath{clip}%
\pgfsetrectcap%
\pgfsetroundjoin%
\pgfsetlinewidth{1.505625pt}%
\definecolor{currentstroke}{rgb}{0.552941,0.898039,0.631373}%
\pgfsetstrokecolor{currentstroke}%
\pgfsetstrokeopacity{0.200000}%
\pgfsetdash{}{0pt}%
\pgfpathmoveto{\pgfqpoint{6.494262in}{1.175940in}}%
\pgfpathlineto{\pgfqpoint{4.400818in}{2.895337in}}%
\pgfusepath{stroke}%
\end{pgfscope}%
\begin{pgfscope}%
\pgfpathrectangle{\pgfqpoint{0.526127in}{0.331635in}}{\pgfqpoint{9.300000in}{7.700000in}}%
\pgfusepath{clip}%
\pgfsetrectcap%
\pgfsetroundjoin%
\pgfsetlinewidth{1.505625pt}%
\definecolor{currentstroke}{rgb}{0.552941,0.898039,0.631373}%
\pgfsetstrokecolor{currentstroke}%
\pgfsetstrokeopacity{0.200000}%
\pgfsetdash{}{0pt}%
\pgfpathmoveto{\pgfqpoint{5.499012in}{2.332218in}}%
\pgfpathlineto{\pgfqpoint{4.400818in}{2.895337in}}%
\pgfusepath{stroke}%
\end{pgfscope}%
\begin{pgfscope}%
\pgfpathrectangle{\pgfqpoint{0.526127in}{0.331635in}}{\pgfqpoint{9.300000in}{7.700000in}}%
\pgfusepath{clip}%
\pgfsetrectcap%
\pgfsetroundjoin%
\pgfsetlinewidth{1.505625pt}%
\definecolor{currentstroke}{rgb}{0.552941,0.898039,0.631373}%
\pgfsetstrokecolor{currentstroke}%
\pgfsetstrokeopacity{0.200000}%
\pgfsetdash{}{0pt}%
\pgfpathmoveto{\pgfqpoint{6.082548in}{1.320932in}}%
\pgfpathlineto{\pgfqpoint{4.400818in}{2.895337in}}%
\pgfusepath{stroke}%
\end{pgfscope}%
\begin{pgfscope}%
\pgfpathrectangle{\pgfqpoint{0.526127in}{0.331635in}}{\pgfqpoint{9.300000in}{7.700000in}}%
\pgfusepath{clip}%
\pgfsetrectcap%
\pgfsetroundjoin%
\pgfsetlinewidth{1.505625pt}%
\definecolor{currentstroke}{rgb}{0.552941,0.898039,0.631373}%
\pgfsetstrokecolor{currentstroke}%
\pgfsetstrokeopacity{0.200000}%
\pgfsetdash{}{0pt}%
\pgfpathmoveto{\pgfqpoint{1.873484in}{3.907469in}}%
\pgfpathlineto{\pgfqpoint{4.400818in}{2.895337in}}%
\pgfusepath{stroke}%
\end{pgfscope}%
\begin{pgfscope}%
\pgfpathrectangle{\pgfqpoint{0.526127in}{0.331635in}}{\pgfqpoint{9.300000in}{7.700000in}}%
\pgfusepath{clip}%
\pgfsetrectcap%
\pgfsetroundjoin%
\pgfsetlinewidth{1.505625pt}%
\definecolor{currentstroke}{rgb}{0.552941,0.898039,0.631373}%
\pgfsetstrokecolor{currentstroke}%
\pgfsetstrokeopacity{0.200000}%
\pgfsetdash{}{0pt}%
\pgfpathmoveto{\pgfqpoint{5.027322in}{3.028149in}}%
\pgfpathlineto{\pgfqpoint{4.400818in}{2.895337in}}%
\pgfusepath{stroke}%
\end{pgfscope}%
\begin{pgfscope}%
\pgfpathrectangle{\pgfqpoint{0.526127in}{0.331635in}}{\pgfqpoint{9.300000in}{7.700000in}}%
\pgfusepath{clip}%
\pgfsetrectcap%
\pgfsetroundjoin%
\pgfsetlinewidth{1.505625pt}%
\definecolor{currentstroke}{rgb}{0.552941,0.898039,0.631373}%
\pgfsetstrokecolor{currentstroke}%
\pgfsetstrokeopacity{0.200000}%
\pgfsetdash{}{0pt}%
\pgfpathmoveto{\pgfqpoint{1.754896in}{3.499956in}}%
\pgfpathlineto{\pgfqpoint{4.400818in}{2.895337in}}%
\pgfusepath{stroke}%
\end{pgfscope}%
\begin{pgfscope}%
\pgfpathrectangle{\pgfqpoint{0.526127in}{0.331635in}}{\pgfqpoint{9.300000in}{7.700000in}}%
\pgfusepath{clip}%
\pgfsetrectcap%
\pgfsetroundjoin%
\pgfsetlinewidth{1.505625pt}%
\definecolor{currentstroke}{rgb}{0.552941,0.898039,0.631373}%
\pgfsetstrokecolor{currentstroke}%
\pgfsetstrokeopacity{0.200000}%
\pgfsetdash{}{0pt}%
\pgfpathmoveto{\pgfqpoint{3.811390in}{5.344873in}}%
\pgfpathlineto{\pgfqpoint{4.400818in}{2.895337in}}%
\pgfusepath{stroke}%
\end{pgfscope}%
\begin{pgfscope}%
\pgfpathrectangle{\pgfqpoint{0.526127in}{0.331635in}}{\pgfqpoint{9.300000in}{7.700000in}}%
\pgfusepath{clip}%
\pgfsetrectcap%
\pgfsetroundjoin%
\pgfsetlinewidth{1.505625pt}%
\definecolor{currentstroke}{rgb}{0.552941,0.898039,0.631373}%
\pgfsetstrokecolor{currentstroke}%
\pgfsetstrokeopacity{0.200000}%
\pgfsetdash{}{0pt}%
\pgfpathmoveto{\pgfqpoint{4.820098in}{3.423588in}}%
\pgfpathlineto{\pgfqpoint{4.400818in}{2.895337in}}%
\pgfusepath{stroke}%
\end{pgfscope}%
\begin{pgfscope}%
\pgfpathrectangle{\pgfqpoint{0.526127in}{0.331635in}}{\pgfqpoint{9.300000in}{7.700000in}}%
\pgfusepath{clip}%
\pgfsetrectcap%
\pgfsetroundjoin%
\pgfsetlinewidth{1.505625pt}%
\definecolor{currentstroke}{rgb}{1.000000,0.623529,0.607843}%
\pgfsetstrokecolor{currentstroke}%
\pgfsetstrokeopacity{0.200000}%
\pgfsetdash{}{0pt}%
\pgfpathmoveto{\pgfqpoint{6.036344in}{7.681635in}}%
\pgfpathlineto{\pgfqpoint{5.110509in}{5.016914in}}%
\pgfusepath{stroke}%
\end{pgfscope}%
\begin{pgfscope}%
\pgfpathrectangle{\pgfqpoint{0.526127in}{0.331635in}}{\pgfqpoint{9.300000in}{7.700000in}}%
\pgfusepath{clip}%
\pgfsetrectcap%
\pgfsetroundjoin%
\pgfsetlinewidth{1.505625pt}%
\definecolor{currentstroke}{rgb}{1.000000,0.623529,0.607843}%
\pgfsetstrokecolor{currentstroke}%
\pgfsetstrokeopacity{0.200000}%
\pgfsetdash{}{0pt}%
\pgfpathmoveto{\pgfqpoint{5.059438in}{4.737486in}}%
\pgfpathlineto{\pgfqpoint{5.110509in}{5.016914in}}%
\pgfusepath{stroke}%
\end{pgfscope}%
\begin{pgfscope}%
\pgfpathrectangle{\pgfqpoint{0.526127in}{0.331635in}}{\pgfqpoint{9.300000in}{7.700000in}}%
\pgfusepath{clip}%
\pgfsetrectcap%
\pgfsetroundjoin%
\pgfsetlinewidth{1.505625pt}%
\definecolor{currentstroke}{rgb}{1.000000,0.623529,0.607843}%
\pgfsetstrokecolor{currentstroke}%
\pgfsetstrokeopacity{0.200000}%
\pgfsetdash{}{0pt}%
\pgfpathmoveto{\pgfqpoint{3.537682in}{4.765579in}}%
\pgfpathlineto{\pgfqpoint{5.110509in}{5.016914in}}%
\pgfusepath{stroke}%
\end{pgfscope}%
\begin{pgfscope}%
\pgfpathrectangle{\pgfqpoint{0.526127in}{0.331635in}}{\pgfqpoint{9.300000in}{7.700000in}}%
\pgfusepath{clip}%
\pgfsetrectcap%
\pgfsetroundjoin%
\pgfsetlinewidth{1.505625pt}%
\definecolor{currentstroke}{rgb}{1.000000,0.623529,0.607843}%
\pgfsetstrokecolor{currentstroke}%
\pgfsetstrokeopacity{0.200000}%
\pgfsetdash{}{0pt}%
\pgfpathmoveto{\pgfqpoint{3.454692in}{5.766423in}}%
\pgfpathlineto{\pgfqpoint{5.110509in}{5.016914in}}%
\pgfusepath{stroke}%
\end{pgfscope}%
\begin{pgfscope}%
\pgfpathrectangle{\pgfqpoint{0.526127in}{0.331635in}}{\pgfqpoint{9.300000in}{7.700000in}}%
\pgfusepath{clip}%
\pgfsetrectcap%
\pgfsetroundjoin%
\pgfsetlinewidth{1.505625pt}%
\definecolor{currentstroke}{rgb}{1.000000,0.623529,0.607843}%
\pgfsetstrokecolor{currentstroke}%
\pgfsetstrokeopacity{0.200000}%
\pgfsetdash{}{0pt}%
\pgfpathmoveto{\pgfqpoint{5.176399in}{7.406573in}}%
\pgfpathlineto{\pgfqpoint{5.110509in}{5.016914in}}%
\pgfusepath{stroke}%
\end{pgfscope}%
\begin{pgfscope}%
\pgfpathrectangle{\pgfqpoint{0.526127in}{0.331635in}}{\pgfqpoint{9.300000in}{7.700000in}}%
\pgfusepath{clip}%
\pgfsetrectcap%
\pgfsetroundjoin%
\pgfsetlinewidth{1.505625pt}%
\definecolor{currentstroke}{rgb}{1.000000,0.623529,0.607843}%
\pgfsetstrokecolor{currentstroke}%
\pgfsetstrokeopacity{0.200000}%
\pgfsetdash{}{0pt}%
\pgfpathmoveto{\pgfqpoint{5.896285in}{3.492514in}}%
\pgfpathlineto{\pgfqpoint{5.110509in}{5.016914in}}%
\pgfusepath{stroke}%
\end{pgfscope}%
\begin{pgfscope}%
\pgfpathrectangle{\pgfqpoint{0.526127in}{0.331635in}}{\pgfqpoint{9.300000in}{7.700000in}}%
\pgfusepath{clip}%
\pgfsetrectcap%
\pgfsetroundjoin%
\pgfsetlinewidth{1.505625pt}%
\definecolor{currentstroke}{rgb}{1.000000,0.623529,0.607843}%
\pgfsetstrokecolor{currentstroke}%
\pgfsetstrokeopacity{0.200000}%
\pgfsetdash{}{0pt}%
\pgfpathmoveto{\pgfqpoint{5.765801in}{5.128354in}}%
\pgfpathlineto{\pgfqpoint{5.110509in}{5.016914in}}%
\pgfusepath{stroke}%
\end{pgfscope}%
\begin{pgfscope}%
\pgfpathrectangle{\pgfqpoint{0.526127in}{0.331635in}}{\pgfqpoint{9.300000in}{7.700000in}}%
\pgfusepath{clip}%
\pgfsetrectcap%
\pgfsetroundjoin%
\pgfsetlinewidth{1.505625pt}%
\definecolor{currentstroke}{rgb}{1.000000,0.623529,0.607843}%
\pgfsetstrokecolor{currentstroke}%
\pgfsetstrokeopacity{0.200000}%
\pgfsetdash{}{0pt}%
\pgfpathmoveto{\pgfqpoint{3.919124in}{4.275736in}}%
\pgfpathlineto{\pgfqpoint{5.110509in}{5.016914in}}%
\pgfusepath{stroke}%
\end{pgfscope}%
\begin{pgfscope}%
\pgfpathrectangle{\pgfqpoint{0.526127in}{0.331635in}}{\pgfqpoint{9.300000in}{7.700000in}}%
\pgfusepath{clip}%
\pgfsetrectcap%
\pgfsetroundjoin%
\pgfsetlinewidth{1.505625pt}%
\definecolor{currentstroke}{rgb}{1.000000,0.623529,0.607843}%
\pgfsetstrokecolor{currentstroke}%
\pgfsetstrokeopacity{0.200000}%
\pgfsetdash{}{0pt}%
\pgfpathmoveto{\pgfqpoint{7.219680in}{3.402146in}}%
\pgfpathlineto{\pgfqpoint{5.110509in}{5.016914in}}%
\pgfusepath{stroke}%
\end{pgfscope}%
\begin{pgfscope}%
\pgfpathrectangle{\pgfqpoint{0.526127in}{0.331635in}}{\pgfqpoint{9.300000in}{7.700000in}}%
\pgfusepath{clip}%
\pgfsetrectcap%
\pgfsetroundjoin%
\pgfsetlinewidth{1.505625pt}%
\definecolor{currentstroke}{rgb}{1.000000,0.623529,0.607843}%
\pgfsetstrokecolor{currentstroke}%
\pgfsetstrokeopacity{0.200000}%
\pgfsetdash{}{0pt}%
\pgfpathmoveto{\pgfqpoint{4.459787in}{4.784008in}}%
\pgfpathlineto{\pgfqpoint{5.110509in}{5.016914in}}%
\pgfusepath{stroke}%
\end{pgfscope}%
\begin{pgfscope}%
\pgfpathrectangle{\pgfqpoint{0.526127in}{0.331635in}}{\pgfqpoint{9.300000in}{7.700000in}}%
\pgfusepath{clip}%
\pgfsetrectcap%
\pgfsetroundjoin%
\pgfsetlinewidth{1.505625pt}%
\definecolor{currentstroke}{rgb}{1.000000,0.623529,0.607843}%
\pgfsetstrokecolor{currentstroke}%
\pgfsetstrokeopacity{0.200000}%
\pgfsetdash{}{0pt}%
\pgfpathmoveto{\pgfqpoint{7.438369in}{3.992349in}}%
\pgfpathlineto{\pgfqpoint{5.110509in}{5.016914in}}%
\pgfusepath{stroke}%
\end{pgfscope}%
\begin{pgfscope}%
\pgfpathrectangle{\pgfqpoint{0.526127in}{0.331635in}}{\pgfqpoint{9.300000in}{7.700000in}}%
\pgfusepath{clip}%
\pgfsetrectcap%
\pgfsetroundjoin%
\pgfsetlinewidth{1.505625pt}%
\definecolor{currentstroke}{rgb}{1.000000,0.623529,0.607843}%
\pgfsetstrokecolor{currentstroke}%
\pgfsetstrokeopacity{0.200000}%
\pgfsetdash{}{0pt}%
\pgfpathmoveto{\pgfqpoint{4.906391in}{4.964376in}}%
\pgfpathlineto{\pgfqpoint{5.110509in}{5.016914in}}%
\pgfusepath{stroke}%
\end{pgfscope}%
\begin{pgfscope}%
\pgfpathrectangle{\pgfqpoint{0.526127in}{0.331635in}}{\pgfqpoint{9.300000in}{7.700000in}}%
\pgfusepath{clip}%
\pgfsetrectcap%
\pgfsetroundjoin%
\pgfsetlinewidth{1.505625pt}%
\definecolor{currentstroke}{rgb}{1.000000,0.623529,0.607843}%
\pgfsetstrokecolor{currentstroke}%
\pgfsetstrokeopacity{0.200000}%
\pgfsetdash{}{0pt}%
\pgfpathmoveto{\pgfqpoint{5.942000in}{5.911816in}}%
\pgfpathlineto{\pgfqpoint{5.110509in}{5.016914in}}%
\pgfusepath{stroke}%
\end{pgfscope}%
\begin{pgfscope}%
\pgfpathrectangle{\pgfqpoint{0.526127in}{0.331635in}}{\pgfqpoint{9.300000in}{7.700000in}}%
\pgfusepath{clip}%
\pgfsetrectcap%
\pgfsetroundjoin%
\pgfsetlinewidth{1.505625pt}%
\definecolor{currentstroke}{rgb}{1.000000,0.623529,0.607843}%
\pgfsetstrokecolor{currentstroke}%
\pgfsetstrokeopacity{0.200000}%
\pgfsetdash{}{0pt}%
\pgfpathmoveto{\pgfqpoint{5.122725in}{5.271620in}}%
\pgfpathlineto{\pgfqpoint{5.110509in}{5.016914in}}%
\pgfusepath{stroke}%
\end{pgfscope}%
\begin{pgfscope}%
\pgfpathrectangle{\pgfqpoint{0.526127in}{0.331635in}}{\pgfqpoint{9.300000in}{7.700000in}}%
\pgfusepath{clip}%
\pgfsetrectcap%
\pgfsetroundjoin%
\pgfsetlinewidth{1.505625pt}%
\definecolor{currentstroke}{rgb}{1.000000,0.623529,0.607843}%
\pgfsetstrokecolor{currentstroke}%
\pgfsetstrokeopacity{0.200000}%
\pgfsetdash{}{0pt}%
\pgfpathmoveto{\pgfqpoint{6.430193in}{5.110591in}}%
\pgfpathlineto{\pgfqpoint{5.110509in}{5.016914in}}%
\pgfusepath{stroke}%
\end{pgfscope}%
\begin{pgfscope}%
\pgfpathrectangle{\pgfqpoint{0.526127in}{0.331635in}}{\pgfqpoint{9.300000in}{7.700000in}}%
\pgfusepath{clip}%
\pgfsetrectcap%
\pgfsetroundjoin%
\pgfsetlinewidth{1.505625pt}%
\definecolor{currentstroke}{rgb}{1.000000,0.623529,0.607843}%
\pgfsetstrokecolor{currentstroke}%
\pgfsetstrokeopacity{0.200000}%
\pgfsetdash{}{0pt}%
\pgfpathmoveto{\pgfqpoint{5.393295in}{4.050774in}}%
\pgfpathlineto{\pgfqpoint{5.110509in}{5.016914in}}%
\pgfusepath{stroke}%
\end{pgfscope}%
\begin{pgfscope}%
\pgfpathrectangle{\pgfqpoint{0.526127in}{0.331635in}}{\pgfqpoint{9.300000in}{7.700000in}}%
\pgfusepath{clip}%
\pgfsetrectcap%
\pgfsetroundjoin%
\pgfsetlinewidth{1.505625pt}%
\definecolor{currentstroke}{rgb}{1.000000,0.623529,0.607843}%
\pgfsetstrokecolor{currentstroke}%
\pgfsetstrokeopacity{0.200000}%
\pgfsetdash{}{0pt}%
\pgfpathmoveto{\pgfqpoint{4.106444in}{4.534616in}}%
\pgfpathlineto{\pgfqpoint{5.110509in}{5.016914in}}%
\pgfusepath{stroke}%
\end{pgfscope}%
\begin{pgfscope}%
\pgfpathrectangle{\pgfqpoint{0.526127in}{0.331635in}}{\pgfqpoint{9.300000in}{7.700000in}}%
\pgfusepath{clip}%
\pgfsetrectcap%
\pgfsetroundjoin%
\pgfsetlinewidth{1.505625pt}%
\definecolor{currentstroke}{rgb}{1.000000,0.623529,0.607843}%
\pgfsetstrokecolor{currentstroke}%
\pgfsetstrokeopacity{0.200000}%
\pgfsetdash{}{0pt}%
\pgfpathmoveto{\pgfqpoint{3.340315in}{5.116509in}}%
\pgfpathlineto{\pgfqpoint{5.110509in}{5.016914in}}%
\pgfusepath{stroke}%
\end{pgfscope}%
\begin{pgfscope}%
\pgfpathrectangle{\pgfqpoint{0.526127in}{0.331635in}}{\pgfqpoint{9.300000in}{7.700000in}}%
\pgfusepath{clip}%
\pgfsetrectcap%
\pgfsetroundjoin%
\pgfsetlinewidth{1.505625pt}%
\definecolor{currentstroke}{rgb}{1.000000,0.623529,0.607843}%
\pgfsetstrokecolor{currentstroke}%
\pgfsetstrokeopacity{0.200000}%
\pgfsetdash{}{0pt}%
\pgfpathmoveto{\pgfqpoint{5.209743in}{4.859650in}}%
\pgfpathlineto{\pgfqpoint{5.110509in}{5.016914in}}%
\pgfusepath{stroke}%
\end{pgfscope}%
\begin{pgfscope}%
\pgfpathrectangle{\pgfqpoint{0.526127in}{0.331635in}}{\pgfqpoint{9.300000in}{7.700000in}}%
\pgfusepath{clip}%
\pgfsetrectcap%
\pgfsetroundjoin%
\pgfsetlinewidth{1.505625pt}%
\definecolor{currentstroke}{rgb}{1.000000,0.623529,0.607843}%
\pgfsetstrokecolor{currentstroke}%
\pgfsetstrokeopacity{0.200000}%
\pgfsetdash{}{0pt}%
\pgfpathmoveto{\pgfqpoint{4.735213in}{4.875154in}}%
\pgfpathlineto{\pgfqpoint{5.110509in}{5.016914in}}%
\pgfusepath{stroke}%
\end{pgfscope}%
\begin{pgfscope}%
\pgfpathrectangle{\pgfqpoint{0.526127in}{0.331635in}}{\pgfqpoint{9.300000in}{7.700000in}}%
\pgfusepath{clip}%
\pgfsetrectcap%
\pgfsetroundjoin%
\pgfsetlinewidth{1.505625pt}%
\definecolor{currentstroke}{rgb}{1.000000,0.623529,0.607843}%
\pgfsetstrokecolor{currentstroke}%
\pgfsetstrokeopacity{0.200000}%
\pgfsetdash{}{0pt}%
\pgfpathmoveto{\pgfqpoint{3.352726in}{5.683291in}}%
\pgfpathlineto{\pgfqpoint{5.110509in}{5.016914in}}%
\pgfusepath{stroke}%
\end{pgfscope}%
\begin{pgfscope}%
\pgfpathrectangle{\pgfqpoint{0.526127in}{0.331635in}}{\pgfqpoint{9.300000in}{7.700000in}}%
\pgfusepath{clip}%
\pgfsetrectcap%
\pgfsetroundjoin%
\pgfsetlinewidth{1.505625pt}%
\definecolor{currentstroke}{rgb}{1.000000,0.623529,0.607843}%
\pgfsetstrokecolor{currentstroke}%
\pgfsetstrokeopacity{0.200000}%
\pgfsetdash{}{0pt}%
\pgfpathmoveto{\pgfqpoint{4.527282in}{5.304490in}}%
\pgfpathlineto{\pgfqpoint{5.110509in}{5.016914in}}%
\pgfusepath{stroke}%
\end{pgfscope}%
\begin{pgfscope}%
\pgfpathrectangle{\pgfqpoint{0.526127in}{0.331635in}}{\pgfqpoint{9.300000in}{7.700000in}}%
\pgfusepath{clip}%
\pgfsetrectcap%
\pgfsetroundjoin%
\pgfsetlinewidth{1.505625pt}%
\definecolor{currentstroke}{rgb}{1.000000,0.623529,0.607843}%
\pgfsetstrokecolor{currentstroke}%
\pgfsetstrokeopacity{0.200000}%
\pgfsetdash{}{0pt}%
\pgfpathmoveto{\pgfqpoint{4.542592in}{3.135664in}}%
\pgfpathlineto{\pgfqpoint{5.110509in}{5.016914in}}%
\pgfusepath{stroke}%
\end{pgfscope}%
\begin{pgfscope}%
\pgfpathrectangle{\pgfqpoint{0.526127in}{0.331635in}}{\pgfqpoint{9.300000in}{7.700000in}}%
\pgfusepath{clip}%
\pgfsetrectcap%
\pgfsetroundjoin%
\pgfsetlinewidth{1.505625pt}%
\definecolor{currentstroke}{rgb}{1.000000,0.623529,0.607843}%
\pgfsetstrokecolor{currentstroke}%
\pgfsetstrokeopacity{0.200000}%
\pgfsetdash{}{0pt}%
\pgfpathmoveto{\pgfqpoint{6.055654in}{5.623590in}}%
\pgfpathlineto{\pgfqpoint{5.110509in}{5.016914in}}%
\pgfusepath{stroke}%
\end{pgfscope}%
\begin{pgfscope}%
\pgfpathrectangle{\pgfqpoint{0.526127in}{0.331635in}}{\pgfqpoint{9.300000in}{7.700000in}}%
\pgfusepath{clip}%
\pgfsetrectcap%
\pgfsetroundjoin%
\pgfsetlinewidth{1.505625pt}%
\definecolor{currentstroke}{rgb}{1.000000,0.623529,0.607843}%
\pgfsetstrokecolor{currentstroke}%
\pgfsetstrokeopacity{0.200000}%
\pgfsetdash{}{0pt}%
\pgfpathmoveto{\pgfqpoint{6.518607in}{5.425973in}}%
\pgfpathlineto{\pgfqpoint{5.110509in}{5.016914in}}%
\pgfusepath{stroke}%
\end{pgfscope}%
\begin{pgfscope}%
\pgfpathrectangle{\pgfqpoint{0.526127in}{0.331635in}}{\pgfqpoint{9.300000in}{7.700000in}}%
\pgfusepath{clip}%
\pgfsetrectcap%
\pgfsetroundjoin%
\pgfsetlinewidth{1.505625pt}%
\definecolor{currentstroke}{rgb}{1.000000,0.623529,0.607843}%
\pgfsetstrokecolor{currentstroke}%
\pgfsetstrokeopacity{0.200000}%
\pgfsetdash{}{0pt}%
\pgfpathmoveto{\pgfqpoint{4.977203in}{5.489320in}}%
\pgfpathlineto{\pgfqpoint{5.110509in}{5.016914in}}%
\pgfusepath{stroke}%
\end{pgfscope}%
\begin{pgfscope}%
\pgfpathrectangle{\pgfqpoint{0.526127in}{0.331635in}}{\pgfqpoint{9.300000in}{7.700000in}}%
\pgfusepath{clip}%
\pgfsetrectcap%
\pgfsetroundjoin%
\pgfsetlinewidth{1.505625pt}%
\definecolor{currentstroke}{rgb}{1.000000,0.623529,0.607843}%
\pgfsetstrokecolor{currentstroke}%
\pgfsetstrokeopacity{0.200000}%
\pgfsetdash{}{0pt}%
\pgfpathmoveto{\pgfqpoint{4.655568in}{5.336989in}}%
\pgfpathlineto{\pgfqpoint{5.110509in}{5.016914in}}%
\pgfusepath{stroke}%
\end{pgfscope}%
\begin{pgfscope}%
\pgfpathrectangle{\pgfqpoint{0.526127in}{0.331635in}}{\pgfqpoint{9.300000in}{7.700000in}}%
\pgfusepath{clip}%
\pgfsetrectcap%
\pgfsetroundjoin%
\pgfsetlinewidth{1.505625pt}%
\definecolor{currentstroke}{rgb}{1.000000,0.623529,0.607843}%
\pgfsetstrokecolor{currentstroke}%
\pgfsetstrokeopacity{0.200000}%
\pgfsetdash{}{0pt}%
\pgfpathmoveto{\pgfqpoint{5.314689in}{4.346369in}}%
\pgfpathlineto{\pgfqpoint{5.110509in}{5.016914in}}%
\pgfusepath{stroke}%
\end{pgfscope}%
\begin{pgfscope}%
\pgfpathrectangle{\pgfqpoint{0.526127in}{0.331635in}}{\pgfqpoint{9.300000in}{7.700000in}}%
\pgfusepath{clip}%
\pgfsetrectcap%
\pgfsetroundjoin%
\pgfsetlinewidth{1.505625pt}%
\definecolor{currentstroke}{rgb}{0.815686,0.733333,1.000000}%
\pgfsetstrokecolor{currentstroke}%
\pgfsetstrokeopacity{0.200000}%
\pgfsetdash{}{0pt}%
\pgfpathmoveto{\pgfqpoint{6.917849in}{2.703124in}}%
\pgfpathlineto{\pgfqpoint{5.359152in}{2.808495in}}%
\pgfusepath{stroke}%
\end{pgfscope}%
\begin{pgfscope}%
\pgfpathrectangle{\pgfqpoint{0.526127in}{0.331635in}}{\pgfqpoint{9.300000in}{7.700000in}}%
\pgfusepath{clip}%
\pgfsetrectcap%
\pgfsetroundjoin%
\pgfsetlinewidth{1.505625pt}%
\definecolor{currentstroke}{rgb}{0.815686,0.733333,1.000000}%
\pgfsetstrokecolor{currentstroke}%
\pgfsetstrokeopacity{0.200000}%
\pgfsetdash{}{0pt}%
\pgfpathmoveto{\pgfqpoint{6.254342in}{2.082954in}}%
\pgfpathlineto{\pgfqpoint{5.359152in}{2.808495in}}%
\pgfusepath{stroke}%
\end{pgfscope}%
\begin{pgfscope}%
\pgfpathrectangle{\pgfqpoint{0.526127in}{0.331635in}}{\pgfqpoint{9.300000in}{7.700000in}}%
\pgfusepath{clip}%
\pgfsetrectcap%
\pgfsetroundjoin%
\pgfsetlinewidth{1.505625pt}%
\definecolor{currentstroke}{rgb}{0.815686,0.733333,1.000000}%
\pgfsetstrokecolor{currentstroke}%
\pgfsetstrokeopacity{0.200000}%
\pgfsetdash{}{0pt}%
\pgfpathmoveto{\pgfqpoint{4.500103in}{1.860863in}}%
\pgfpathlineto{\pgfqpoint{5.359152in}{2.808495in}}%
\pgfusepath{stroke}%
\end{pgfscope}%
\begin{pgfscope}%
\pgfpathrectangle{\pgfqpoint{0.526127in}{0.331635in}}{\pgfqpoint{9.300000in}{7.700000in}}%
\pgfusepath{clip}%
\pgfsetrectcap%
\pgfsetroundjoin%
\pgfsetlinewidth{1.505625pt}%
\definecolor{currentstroke}{rgb}{0.815686,0.733333,1.000000}%
\pgfsetstrokecolor{currentstroke}%
\pgfsetstrokeopacity{0.200000}%
\pgfsetdash{}{0pt}%
\pgfpathmoveto{\pgfqpoint{3.205256in}{1.832721in}}%
\pgfpathlineto{\pgfqpoint{5.359152in}{2.808495in}}%
\pgfusepath{stroke}%
\end{pgfscope}%
\begin{pgfscope}%
\pgfpathrectangle{\pgfqpoint{0.526127in}{0.331635in}}{\pgfqpoint{9.300000in}{7.700000in}}%
\pgfusepath{clip}%
\pgfsetrectcap%
\pgfsetroundjoin%
\pgfsetlinewidth{1.505625pt}%
\definecolor{currentstroke}{rgb}{0.815686,0.733333,1.000000}%
\pgfsetstrokecolor{currentstroke}%
\pgfsetstrokeopacity{0.200000}%
\pgfsetdash{}{0pt}%
\pgfpathmoveto{\pgfqpoint{6.199623in}{2.218091in}}%
\pgfpathlineto{\pgfqpoint{5.359152in}{2.808495in}}%
\pgfusepath{stroke}%
\end{pgfscope}%
\begin{pgfscope}%
\pgfpathrectangle{\pgfqpoint{0.526127in}{0.331635in}}{\pgfqpoint{9.300000in}{7.700000in}}%
\pgfusepath{clip}%
\pgfsetrectcap%
\pgfsetroundjoin%
\pgfsetlinewidth{1.505625pt}%
\definecolor{currentstroke}{rgb}{0.815686,0.733333,1.000000}%
\pgfsetstrokecolor{currentstroke}%
\pgfsetstrokeopacity{0.200000}%
\pgfsetdash{}{0pt}%
\pgfpathmoveto{\pgfqpoint{7.384542in}{2.729852in}}%
\pgfpathlineto{\pgfqpoint{5.359152in}{2.808495in}}%
\pgfusepath{stroke}%
\end{pgfscope}%
\begin{pgfscope}%
\pgfpathrectangle{\pgfqpoint{0.526127in}{0.331635in}}{\pgfqpoint{9.300000in}{7.700000in}}%
\pgfusepath{clip}%
\pgfsetrectcap%
\pgfsetroundjoin%
\pgfsetlinewidth{1.505625pt}%
\definecolor{currentstroke}{rgb}{0.815686,0.733333,1.000000}%
\pgfsetstrokecolor{currentstroke}%
\pgfsetstrokeopacity{0.200000}%
\pgfsetdash{}{0pt}%
\pgfpathmoveto{\pgfqpoint{2.373781in}{3.583374in}}%
\pgfpathlineto{\pgfqpoint{5.359152in}{2.808495in}}%
\pgfusepath{stroke}%
\end{pgfscope}%
\begin{pgfscope}%
\pgfpathrectangle{\pgfqpoint{0.526127in}{0.331635in}}{\pgfqpoint{9.300000in}{7.700000in}}%
\pgfusepath{clip}%
\pgfsetrectcap%
\pgfsetroundjoin%
\pgfsetlinewidth{1.505625pt}%
\definecolor{currentstroke}{rgb}{0.815686,0.733333,1.000000}%
\pgfsetstrokecolor{currentstroke}%
\pgfsetstrokeopacity{0.200000}%
\pgfsetdash{}{0pt}%
\pgfpathmoveto{\pgfqpoint{7.393947in}{1.678995in}}%
\pgfpathlineto{\pgfqpoint{5.359152in}{2.808495in}}%
\pgfusepath{stroke}%
\end{pgfscope}%
\begin{pgfscope}%
\pgfpathrectangle{\pgfqpoint{0.526127in}{0.331635in}}{\pgfqpoint{9.300000in}{7.700000in}}%
\pgfusepath{clip}%
\pgfsetrectcap%
\pgfsetroundjoin%
\pgfsetlinewidth{1.505625pt}%
\definecolor{currentstroke}{rgb}{0.815686,0.733333,1.000000}%
\pgfsetstrokecolor{currentstroke}%
\pgfsetstrokeopacity{0.200000}%
\pgfsetdash{}{0pt}%
\pgfpathmoveto{\pgfqpoint{7.162597in}{2.858193in}}%
\pgfpathlineto{\pgfqpoint{5.359152in}{2.808495in}}%
\pgfusepath{stroke}%
\end{pgfscope}%
\begin{pgfscope}%
\pgfpathrectangle{\pgfqpoint{0.526127in}{0.331635in}}{\pgfqpoint{9.300000in}{7.700000in}}%
\pgfusepath{clip}%
\pgfsetrectcap%
\pgfsetroundjoin%
\pgfsetlinewidth{1.505625pt}%
\definecolor{currentstroke}{rgb}{0.815686,0.733333,1.000000}%
\pgfsetstrokecolor{currentstroke}%
\pgfsetstrokeopacity{0.200000}%
\pgfsetdash{}{0pt}%
\pgfpathmoveto{\pgfqpoint{6.664367in}{4.897256in}}%
\pgfpathlineto{\pgfqpoint{5.359152in}{2.808495in}}%
\pgfusepath{stroke}%
\end{pgfscope}%
\begin{pgfscope}%
\pgfpathrectangle{\pgfqpoint{0.526127in}{0.331635in}}{\pgfqpoint{9.300000in}{7.700000in}}%
\pgfusepath{clip}%
\pgfsetrectcap%
\pgfsetroundjoin%
\pgfsetlinewidth{1.505625pt}%
\definecolor{currentstroke}{rgb}{0.815686,0.733333,1.000000}%
\pgfsetstrokecolor{currentstroke}%
\pgfsetstrokeopacity{0.200000}%
\pgfsetdash{}{0pt}%
\pgfpathmoveto{\pgfqpoint{5.711097in}{2.503025in}}%
\pgfpathlineto{\pgfqpoint{5.359152in}{2.808495in}}%
\pgfusepath{stroke}%
\end{pgfscope}%
\begin{pgfscope}%
\pgfpathrectangle{\pgfqpoint{0.526127in}{0.331635in}}{\pgfqpoint{9.300000in}{7.700000in}}%
\pgfusepath{clip}%
\pgfsetrectcap%
\pgfsetroundjoin%
\pgfsetlinewidth{1.505625pt}%
\definecolor{currentstroke}{rgb}{0.815686,0.733333,1.000000}%
\pgfsetstrokecolor{currentstroke}%
\pgfsetstrokeopacity{0.200000}%
\pgfsetdash{}{0pt}%
\pgfpathmoveto{\pgfqpoint{6.805509in}{2.216717in}}%
\pgfpathlineto{\pgfqpoint{5.359152in}{2.808495in}}%
\pgfusepath{stroke}%
\end{pgfscope}%
\begin{pgfscope}%
\pgfpathrectangle{\pgfqpoint{0.526127in}{0.331635in}}{\pgfqpoint{9.300000in}{7.700000in}}%
\pgfusepath{clip}%
\pgfsetrectcap%
\pgfsetroundjoin%
\pgfsetlinewidth{1.505625pt}%
\definecolor{currentstroke}{rgb}{0.815686,0.733333,1.000000}%
\pgfsetstrokecolor{currentstroke}%
\pgfsetstrokeopacity{0.200000}%
\pgfsetdash{}{0pt}%
\pgfpathmoveto{\pgfqpoint{3.875538in}{2.306540in}}%
\pgfpathlineto{\pgfqpoint{5.359152in}{2.808495in}}%
\pgfusepath{stroke}%
\end{pgfscope}%
\begin{pgfscope}%
\pgfpathrectangle{\pgfqpoint{0.526127in}{0.331635in}}{\pgfqpoint{9.300000in}{7.700000in}}%
\pgfusepath{clip}%
\pgfsetrectcap%
\pgfsetroundjoin%
\pgfsetlinewidth{1.505625pt}%
\definecolor{currentstroke}{rgb}{0.815686,0.733333,1.000000}%
\pgfsetstrokecolor{currentstroke}%
\pgfsetstrokeopacity{0.200000}%
\pgfsetdash{}{0pt}%
\pgfpathmoveto{\pgfqpoint{1.970636in}{4.883382in}}%
\pgfpathlineto{\pgfqpoint{5.359152in}{2.808495in}}%
\pgfusepath{stroke}%
\end{pgfscope}%
\begin{pgfscope}%
\pgfpathrectangle{\pgfqpoint{0.526127in}{0.331635in}}{\pgfqpoint{9.300000in}{7.700000in}}%
\pgfusepath{clip}%
\pgfsetrectcap%
\pgfsetroundjoin%
\pgfsetlinewidth{1.505625pt}%
\definecolor{currentstroke}{rgb}{0.815686,0.733333,1.000000}%
\pgfsetstrokecolor{currentstroke}%
\pgfsetstrokeopacity{0.200000}%
\pgfsetdash{}{0pt}%
\pgfpathmoveto{\pgfqpoint{3.343614in}{2.667179in}}%
\pgfpathlineto{\pgfqpoint{5.359152in}{2.808495in}}%
\pgfusepath{stroke}%
\end{pgfscope}%
\begin{pgfscope}%
\pgfpathrectangle{\pgfqpoint{0.526127in}{0.331635in}}{\pgfqpoint{9.300000in}{7.700000in}}%
\pgfusepath{clip}%
\pgfsetrectcap%
\pgfsetroundjoin%
\pgfsetlinewidth{1.505625pt}%
\definecolor{currentstroke}{rgb}{0.815686,0.733333,1.000000}%
\pgfsetstrokecolor{currentstroke}%
\pgfsetstrokeopacity{0.200000}%
\pgfsetdash{}{0pt}%
\pgfpathmoveto{\pgfqpoint{6.735659in}{3.857474in}}%
\pgfpathlineto{\pgfqpoint{5.359152in}{2.808495in}}%
\pgfusepath{stroke}%
\end{pgfscope}%
\begin{pgfscope}%
\pgfpathrectangle{\pgfqpoint{0.526127in}{0.331635in}}{\pgfqpoint{9.300000in}{7.700000in}}%
\pgfusepath{clip}%
\pgfsetrectcap%
\pgfsetroundjoin%
\pgfsetlinewidth{1.505625pt}%
\definecolor{currentstroke}{rgb}{0.815686,0.733333,1.000000}%
\pgfsetstrokecolor{currentstroke}%
\pgfsetstrokeopacity{0.200000}%
\pgfsetdash{}{0pt}%
\pgfpathmoveto{\pgfqpoint{2.507937in}{5.115912in}}%
\pgfpathlineto{\pgfqpoint{5.359152in}{2.808495in}}%
\pgfusepath{stroke}%
\end{pgfscope}%
\begin{pgfscope}%
\pgfpathrectangle{\pgfqpoint{0.526127in}{0.331635in}}{\pgfqpoint{9.300000in}{7.700000in}}%
\pgfusepath{clip}%
\pgfsetrectcap%
\pgfsetroundjoin%
\pgfsetlinewidth{1.505625pt}%
\definecolor{currentstroke}{rgb}{0.815686,0.733333,1.000000}%
\pgfsetstrokecolor{currentstroke}%
\pgfsetstrokeopacity{0.200000}%
\pgfsetdash{}{0pt}%
\pgfpathmoveto{\pgfqpoint{8.582505in}{3.957628in}}%
\pgfpathlineto{\pgfqpoint{5.359152in}{2.808495in}}%
\pgfusepath{stroke}%
\end{pgfscope}%
\begin{pgfscope}%
\pgfpathrectangle{\pgfqpoint{0.526127in}{0.331635in}}{\pgfqpoint{9.300000in}{7.700000in}}%
\pgfusepath{clip}%
\pgfsetrectcap%
\pgfsetroundjoin%
\pgfsetlinewidth{1.505625pt}%
\definecolor{currentstroke}{rgb}{0.815686,0.733333,1.000000}%
\pgfsetstrokecolor{currentstroke}%
\pgfsetstrokeopacity{0.200000}%
\pgfsetdash{}{0pt}%
\pgfpathmoveto{\pgfqpoint{5.435254in}{1.782781in}}%
\pgfpathlineto{\pgfqpoint{5.359152in}{2.808495in}}%
\pgfusepath{stroke}%
\end{pgfscope}%
\begin{pgfscope}%
\pgfpathrectangle{\pgfqpoint{0.526127in}{0.331635in}}{\pgfqpoint{9.300000in}{7.700000in}}%
\pgfusepath{clip}%
\pgfsetrectcap%
\pgfsetroundjoin%
\pgfsetlinewidth{1.505625pt}%
\definecolor{currentstroke}{rgb}{0.815686,0.733333,1.000000}%
\pgfsetstrokecolor{currentstroke}%
\pgfsetstrokeopacity{0.200000}%
\pgfsetdash{}{0pt}%
\pgfpathmoveto{\pgfqpoint{5.661250in}{4.669033in}}%
\pgfpathlineto{\pgfqpoint{5.359152in}{2.808495in}}%
\pgfusepath{stroke}%
\end{pgfscope}%
\begin{pgfscope}%
\pgfpathrectangle{\pgfqpoint{0.526127in}{0.331635in}}{\pgfqpoint{9.300000in}{7.700000in}}%
\pgfusepath{clip}%
\pgfsetrectcap%
\pgfsetroundjoin%
\pgfsetlinewidth{1.505625pt}%
\definecolor{currentstroke}{rgb}{0.815686,0.733333,1.000000}%
\pgfsetstrokecolor{currentstroke}%
\pgfsetstrokeopacity{0.200000}%
\pgfsetdash{}{0pt}%
\pgfpathmoveto{\pgfqpoint{2.380624in}{2.704215in}}%
\pgfpathlineto{\pgfqpoint{5.359152in}{2.808495in}}%
\pgfusepath{stroke}%
\end{pgfscope}%
\begin{pgfscope}%
\pgfpathrectangle{\pgfqpoint{0.526127in}{0.331635in}}{\pgfqpoint{9.300000in}{7.700000in}}%
\pgfusepath{clip}%
\pgfsetrectcap%
\pgfsetroundjoin%
\pgfsetlinewidth{1.505625pt}%
\definecolor{currentstroke}{rgb}{0.815686,0.733333,1.000000}%
\pgfsetstrokecolor{currentstroke}%
\pgfsetstrokeopacity{0.200000}%
\pgfsetdash{}{0pt}%
\pgfpathmoveto{\pgfqpoint{3.490954in}{3.356487in}}%
\pgfpathlineto{\pgfqpoint{5.359152in}{2.808495in}}%
\pgfusepath{stroke}%
\end{pgfscope}%
\begin{pgfscope}%
\pgfpathrectangle{\pgfqpoint{0.526127in}{0.331635in}}{\pgfqpoint{9.300000in}{7.700000in}}%
\pgfusepath{clip}%
\pgfsetrectcap%
\pgfsetroundjoin%
\pgfsetlinewidth{1.505625pt}%
\definecolor{currentstroke}{rgb}{0.815686,0.733333,1.000000}%
\pgfsetstrokecolor{currentstroke}%
\pgfsetstrokeopacity{0.200000}%
\pgfsetdash{}{0pt}%
\pgfpathmoveto{\pgfqpoint{4.801136in}{0.774084in}}%
\pgfpathlineto{\pgfqpoint{5.359152in}{2.808495in}}%
\pgfusepath{stroke}%
\end{pgfscope}%
\begin{pgfscope}%
\pgfpathrectangle{\pgfqpoint{0.526127in}{0.331635in}}{\pgfqpoint{9.300000in}{7.700000in}}%
\pgfusepath{clip}%
\pgfsetrectcap%
\pgfsetroundjoin%
\pgfsetlinewidth{1.505625pt}%
\definecolor{currentstroke}{rgb}{0.815686,0.733333,1.000000}%
\pgfsetstrokecolor{currentstroke}%
\pgfsetstrokeopacity{0.200000}%
\pgfsetdash{}{0pt}%
\pgfpathmoveto{\pgfqpoint{7.283490in}{1.720161in}}%
\pgfpathlineto{\pgfqpoint{5.359152in}{2.808495in}}%
\pgfusepath{stroke}%
\end{pgfscope}%
\begin{pgfscope}%
\pgfpathrectangle{\pgfqpoint{0.526127in}{0.331635in}}{\pgfqpoint{9.300000in}{7.700000in}}%
\pgfusepath{clip}%
\pgfsetrectcap%
\pgfsetroundjoin%
\pgfsetlinewidth{1.505625pt}%
\definecolor{currentstroke}{rgb}{0.815686,0.733333,1.000000}%
\pgfsetstrokecolor{currentstroke}%
\pgfsetstrokeopacity{0.200000}%
\pgfsetdash{}{0pt}%
\pgfpathmoveto{\pgfqpoint{4.932772in}{1.241893in}}%
\pgfpathlineto{\pgfqpoint{5.359152in}{2.808495in}}%
\pgfusepath{stroke}%
\end{pgfscope}%
\begin{pgfscope}%
\pgfpathrectangle{\pgfqpoint{0.526127in}{0.331635in}}{\pgfqpoint{9.300000in}{7.700000in}}%
\pgfusepath{clip}%
\pgfsetrectcap%
\pgfsetroundjoin%
\pgfsetlinewidth{1.505625pt}%
\definecolor{currentstroke}{rgb}{0.815686,0.733333,1.000000}%
\pgfsetstrokecolor{currentstroke}%
\pgfsetstrokeopacity{0.200000}%
\pgfsetdash{}{0pt}%
\pgfpathmoveto{\pgfqpoint{7.494242in}{1.691723in}}%
\pgfpathlineto{\pgfqpoint{5.359152in}{2.808495in}}%
\pgfusepath{stroke}%
\end{pgfscope}%
\begin{pgfscope}%
\pgfpathrectangle{\pgfqpoint{0.526127in}{0.331635in}}{\pgfqpoint{9.300000in}{7.700000in}}%
\pgfusepath{clip}%
\pgfsetrectcap%
\pgfsetroundjoin%
\pgfsetlinewidth{1.505625pt}%
\definecolor{currentstroke}{rgb}{0.815686,0.733333,1.000000}%
\pgfsetstrokecolor{currentstroke}%
\pgfsetstrokeopacity{0.200000}%
\pgfsetdash{}{0pt}%
\pgfpathmoveto{\pgfqpoint{8.328870in}{2.327023in}}%
\pgfpathlineto{\pgfqpoint{5.359152in}{2.808495in}}%
\pgfusepath{stroke}%
\end{pgfscope}%
\begin{pgfscope}%
\pgfpathrectangle{\pgfqpoint{0.526127in}{0.331635in}}{\pgfqpoint{9.300000in}{7.700000in}}%
\pgfusepath{clip}%
\pgfsetrectcap%
\pgfsetroundjoin%
\pgfsetlinewidth{1.505625pt}%
\definecolor{currentstroke}{rgb}{0.815686,0.733333,1.000000}%
\pgfsetstrokecolor{currentstroke}%
\pgfsetstrokeopacity{0.200000}%
\pgfsetdash{}{0pt}%
\pgfpathmoveto{\pgfqpoint{2.658761in}{4.417192in}}%
\pgfpathlineto{\pgfqpoint{5.359152in}{2.808495in}}%
\pgfusepath{stroke}%
\end{pgfscope}%
\begin{pgfscope}%
\pgfpathrectangle{\pgfqpoint{0.526127in}{0.331635in}}{\pgfqpoint{9.300000in}{7.700000in}}%
\pgfusepath{clip}%
\pgfsetrectcap%
\pgfsetroundjoin%
\pgfsetlinewidth{1.505625pt}%
\definecolor{currentstroke}{rgb}{0.870588,0.733333,0.607843}%
\pgfsetstrokecolor{currentstroke}%
\pgfsetstrokeopacity{0.200000}%
\pgfsetdash{}{0pt}%
\pgfpathmoveto{\pgfqpoint{3.293241in}{4.314612in}}%
\pgfpathlineto{\pgfqpoint{4.298154in}{2.772028in}}%
\pgfusepath{stroke}%
\end{pgfscope}%
\begin{pgfscope}%
\pgfpathrectangle{\pgfqpoint{0.526127in}{0.331635in}}{\pgfqpoint{9.300000in}{7.700000in}}%
\pgfusepath{clip}%
\pgfsetrectcap%
\pgfsetroundjoin%
\pgfsetlinewidth{1.505625pt}%
\definecolor{currentstroke}{rgb}{0.870588,0.733333,0.607843}%
\pgfsetstrokecolor{currentstroke}%
\pgfsetstrokeopacity{0.200000}%
\pgfsetdash{}{0pt}%
\pgfpathmoveto{\pgfqpoint{4.449865in}{2.634339in}}%
\pgfpathlineto{\pgfqpoint{4.298154in}{2.772028in}}%
\pgfusepath{stroke}%
\end{pgfscope}%
\begin{pgfscope}%
\pgfpathrectangle{\pgfqpoint{0.526127in}{0.331635in}}{\pgfqpoint{9.300000in}{7.700000in}}%
\pgfusepath{clip}%
\pgfsetrectcap%
\pgfsetroundjoin%
\pgfsetlinewidth{1.505625pt}%
\definecolor{currentstroke}{rgb}{0.870588,0.733333,0.607843}%
\pgfsetstrokecolor{currentstroke}%
\pgfsetstrokeopacity{0.200000}%
\pgfsetdash{}{0pt}%
\pgfpathmoveto{\pgfqpoint{5.505397in}{0.975429in}}%
\pgfpathlineto{\pgfqpoint{4.298154in}{2.772028in}}%
\pgfusepath{stroke}%
\end{pgfscope}%
\begin{pgfscope}%
\pgfpathrectangle{\pgfqpoint{0.526127in}{0.331635in}}{\pgfqpoint{9.300000in}{7.700000in}}%
\pgfusepath{clip}%
\pgfsetrectcap%
\pgfsetroundjoin%
\pgfsetlinewidth{1.505625pt}%
\definecolor{currentstroke}{rgb}{0.870588,0.733333,0.607843}%
\pgfsetstrokecolor{currentstroke}%
\pgfsetstrokeopacity{0.200000}%
\pgfsetdash{}{0pt}%
\pgfpathmoveto{\pgfqpoint{3.975782in}{2.801486in}}%
\pgfpathlineto{\pgfqpoint{4.298154in}{2.772028in}}%
\pgfusepath{stroke}%
\end{pgfscope}%
\begin{pgfscope}%
\pgfpathrectangle{\pgfqpoint{0.526127in}{0.331635in}}{\pgfqpoint{9.300000in}{7.700000in}}%
\pgfusepath{clip}%
\pgfsetrectcap%
\pgfsetroundjoin%
\pgfsetlinewidth{1.505625pt}%
\definecolor{currentstroke}{rgb}{0.870588,0.733333,0.607843}%
\pgfsetstrokecolor{currentstroke}%
\pgfsetstrokeopacity{0.200000}%
\pgfsetdash{}{0pt}%
\pgfpathmoveto{\pgfqpoint{5.150340in}{3.148874in}}%
\pgfpathlineto{\pgfqpoint{4.298154in}{2.772028in}}%
\pgfusepath{stroke}%
\end{pgfscope}%
\begin{pgfscope}%
\pgfpathrectangle{\pgfqpoint{0.526127in}{0.331635in}}{\pgfqpoint{9.300000in}{7.700000in}}%
\pgfusepath{clip}%
\pgfsetrectcap%
\pgfsetroundjoin%
\pgfsetlinewidth{1.505625pt}%
\definecolor{currentstroke}{rgb}{0.870588,0.733333,0.607843}%
\pgfsetstrokecolor{currentstroke}%
\pgfsetstrokeopacity{0.200000}%
\pgfsetdash{}{0pt}%
\pgfpathmoveto{\pgfqpoint{6.303025in}{4.257506in}}%
\pgfpathlineto{\pgfqpoint{4.298154in}{2.772028in}}%
\pgfusepath{stroke}%
\end{pgfscope}%
\begin{pgfscope}%
\pgfpathrectangle{\pgfqpoint{0.526127in}{0.331635in}}{\pgfqpoint{9.300000in}{7.700000in}}%
\pgfusepath{clip}%
\pgfsetrectcap%
\pgfsetroundjoin%
\pgfsetlinewidth{1.505625pt}%
\definecolor{currentstroke}{rgb}{0.870588,0.733333,0.607843}%
\pgfsetstrokecolor{currentstroke}%
\pgfsetstrokeopacity{0.200000}%
\pgfsetdash{}{0pt}%
\pgfpathmoveto{\pgfqpoint{3.101691in}{7.346220in}}%
\pgfpathlineto{\pgfqpoint{4.298154in}{2.772028in}}%
\pgfusepath{stroke}%
\end{pgfscope}%
\begin{pgfscope}%
\pgfpathrectangle{\pgfqpoint{0.526127in}{0.331635in}}{\pgfqpoint{9.300000in}{7.700000in}}%
\pgfusepath{clip}%
\pgfsetrectcap%
\pgfsetroundjoin%
\pgfsetlinewidth{1.505625pt}%
\definecolor{currentstroke}{rgb}{0.870588,0.733333,0.607843}%
\pgfsetstrokecolor{currentstroke}%
\pgfsetstrokeopacity{0.200000}%
\pgfsetdash{}{0pt}%
\pgfpathmoveto{\pgfqpoint{6.025455in}{1.547819in}}%
\pgfpathlineto{\pgfqpoint{4.298154in}{2.772028in}}%
\pgfusepath{stroke}%
\end{pgfscope}%
\begin{pgfscope}%
\pgfpathrectangle{\pgfqpoint{0.526127in}{0.331635in}}{\pgfqpoint{9.300000in}{7.700000in}}%
\pgfusepath{clip}%
\pgfsetrectcap%
\pgfsetroundjoin%
\pgfsetlinewidth{1.505625pt}%
\definecolor{currentstroke}{rgb}{0.870588,0.733333,0.607843}%
\pgfsetstrokecolor{currentstroke}%
\pgfsetstrokeopacity{0.200000}%
\pgfsetdash{}{0pt}%
\pgfpathmoveto{\pgfqpoint{2.005207in}{1.996477in}}%
\pgfpathlineto{\pgfqpoint{4.298154in}{2.772028in}}%
\pgfusepath{stroke}%
\end{pgfscope}%
\begin{pgfscope}%
\pgfpathrectangle{\pgfqpoint{0.526127in}{0.331635in}}{\pgfqpoint{9.300000in}{7.700000in}}%
\pgfusepath{clip}%
\pgfsetrectcap%
\pgfsetroundjoin%
\pgfsetlinewidth{1.505625pt}%
\definecolor{currentstroke}{rgb}{0.870588,0.733333,0.607843}%
\pgfsetstrokecolor{currentstroke}%
\pgfsetstrokeopacity{0.200000}%
\pgfsetdash{}{0pt}%
\pgfpathmoveto{\pgfqpoint{0.948854in}{3.300176in}}%
\pgfpathlineto{\pgfqpoint{4.298154in}{2.772028in}}%
\pgfusepath{stroke}%
\end{pgfscope}%
\begin{pgfscope}%
\pgfpathrectangle{\pgfqpoint{0.526127in}{0.331635in}}{\pgfqpoint{9.300000in}{7.700000in}}%
\pgfusepath{clip}%
\pgfsetrectcap%
\pgfsetroundjoin%
\pgfsetlinewidth{1.505625pt}%
\definecolor{currentstroke}{rgb}{0.870588,0.733333,0.607843}%
\pgfsetstrokecolor{currentstroke}%
\pgfsetstrokeopacity{0.200000}%
\pgfsetdash{}{0pt}%
\pgfpathmoveto{\pgfqpoint{2.231357in}{2.159367in}}%
\pgfpathlineto{\pgfqpoint{4.298154in}{2.772028in}}%
\pgfusepath{stroke}%
\end{pgfscope}%
\begin{pgfscope}%
\pgfpathrectangle{\pgfqpoint{0.526127in}{0.331635in}}{\pgfqpoint{9.300000in}{7.700000in}}%
\pgfusepath{clip}%
\pgfsetrectcap%
\pgfsetroundjoin%
\pgfsetlinewidth{1.505625pt}%
\definecolor{currentstroke}{rgb}{0.870588,0.733333,0.607843}%
\pgfsetstrokecolor{currentstroke}%
\pgfsetstrokeopacity{0.200000}%
\pgfsetdash{}{0pt}%
\pgfpathmoveto{\pgfqpoint{5.273237in}{2.725307in}}%
\pgfpathlineto{\pgfqpoint{4.298154in}{2.772028in}}%
\pgfusepath{stroke}%
\end{pgfscope}%
\begin{pgfscope}%
\pgfpathrectangle{\pgfqpoint{0.526127in}{0.331635in}}{\pgfqpoint{9.300000in}{7.700000in}}%
\pgfusepath{clip}%
\pgfsetrectcap%
\pgfsetroundjoin%
\pgfsetlinewidth{1.505625pt}%
\definecolor{currentstroke}{rgb}{0.870588,0.733333,0.607843}%
\pgfsetstrokecolor{currentstroke}%
\pgfsetstrokeopacity{0.200000}%
\pgfsetdash{}{0pt}%
\pgfpathmoveto{\pgfqpoint{4.721664in}{2.320491in}}%
\pgfpathlineto{\pgfqpoint{4.298154in}{2.772028in}}%
\pgfusepath{stroke}%
\end{pgfscope}%
\begin{pgfscope}%
\pgfpathrectangle{\pgfqpoint{0.526127in}{0.331635in}}{\pgfqpoint{9.300000in}{7.700000in}}%
\pgfusepath{clip}%
\pgfsetrectcap%
\pgfsetroundjoin%
\pgfsetlinewidth{1.505625pt}%
\definecolor{currentstroke}{rgb}{0.870588,0.733333,0.607843}%
\pgfsetstrokecolor{currentstroke}%
\pgfsetstrokeopacity{0.200000}%
\pgfsetdash{}{0pt}%
\pgfpathmoveto{\pgfqpoint{5.294260in}{1.697890in}}%
\pgfpathlineto{\pgfqpoint{4.298154in}{2.772028in}}%
\pgfusepath{stroke}%
\end{pgfscope}%
\begin{pgfscope}%
\pgfpathrectangle{\pgfqpoint{0.526127in}{0.331635in}}{\pgfqpoint{9.300000in}{7.700000in}}%
\pgfusepath{clip}%
\pgfsetrectcap%
\pgfsetroundjoin%
\pgfsetlinewidth{1.505625pt}%
\definecolor{currentstroke}{rgb}{0.870588,0.733333,0.607843}%
\pgfsetstrokecolor{currentstroke}%
\pgfsetstrokeopacity{0.200000}%
\pgfsetdash{}{0pt}%
\pgfpathmoveto{\pgfqpoint{3.016007in}{6.490305in}}%
\pgfpathlineto{\pgfqpoint{4.298154in}{2.772028in}}%
\pgfusepath{stroke}%
\end{pgfscope}%
\begin{pgfscope}%
\pgfpathrectangle{\pgfqpoint{0.526127in}{0.331635in}}{\pgfqpoint{9.300000in}{7.700000in}}%
\pgfusepath{clip}%
\pgfsetrectcap%
\pgfsetroundjoin%
\pgfsetlinewidth{1.505625pt}%
\definecolor{currentstroke}{rgb}{0.870588,0.733333,0.607843}%
\pgfsetstrokecolor{currentstroke}%
\pgfsetstrokeopacity{0.200000}%
\pgfsetdash{}{0pt}%
\pgfpathmoveto{\pgfqpoint{0.975724in}{3.313258in}}%
\pgfpathlineto{\pgfqpoint{4.298154in}{2.772028in}}%
\pgfusepath{stroke}%
\end{pgfscope}%
\begin{pgfscope}%
\pgfpathrectangle{\pgfqpoint{0.526127in}{0.331635in}}{\pgfqpoint{9.300000in}{7.700000in}}%
\pgfusepath{clip}%
\pgfsetrectcap%
\pgfsetroundjoin%
\pgfsetlinewidth{1.505625pt}%
\definecolor{currentstroke}{rgb}{0.870588,0.733333,0.607843}%
\pgfsetstrokecolor{currentstroke}%
\pgfsetstrokeopacity{0.200000}%
\pgfsetdash{}{0pt}%
\pgfpathmoveto{\pgfqpoint{6.443064in}{3.205723in}}%
\pgfpathlineto{\pgfqpoint{4.298154in}{2.772028in}}%
\pgfusepath{stroke}%
\end{pgfscope}%
\begin{pgfscope}%
\pgfpathrectangle{\pgfqpoint{0.526127in}{0.331635in}}{\pgfqpoint{9.300000in}{7.700000in}}%
\pgfusepath{clip}%
\pgfsetrectcap%
\pgfsetroundjoin%
\pgfsetlinewidth{1.505625pt}%
\definecolor{currentstroke}{rgb}{0.870588,0.733333,0.607843}%
\pgfsetstrokecolor{currentstroke}%
\pgfsetstrokeopacity{0.200000}%
\pgfsetdash{}{0pt}%
\pgfpathmoveto{\pgfqpoint{5.818974in}{2.956871in}}%
\pgfpathlineto{\pgfqpoint{4.298154in}{2.772028in}}%
\pgfusepath{stroke}%
\end{pgfscope}%
\begin{pgfscope}%
\pgfpathrectangle{\pgfqpoint{0.526127in}{0.331635in}}{\pgfqpoint{9.300000in}{7.700000in}}%
\pgfusepath{clip}%
\pgfsetrectcap%
\pgfsetroundjoin%
\pgfsetlinewidth{1.505625pt}%
\definecolor{currentstroke}{rgb}{0.870588,0.733333,0.607843}%
\pgfsetstrokecolor{currentstroke}%
\pgfsetstrokeopacity{0.200000}%
\pgfsetdash{}{0pt}%
\pgfpathmoveto{\pgfqpoint{3.186993in}{2.588401in}}%
\pgfpathlineto{\pgfqpoint{4.298154in}{2.772028in}}%
\pgfusepath{stroke}%
\end{pgfscope}%
\begin{pgfscope}%
\pgfpathrectangle{\pgfqpoint{0.526127in}{0.331635in}}{\pgfqpoint{9.300000in}{7.700000in}}%
\pgfusepath{clip}%
\pgfsetrectcap%
\pgfsetroundjoin%
\pgfsetlinewidth{1.505625pt}%
\definecolor{currentstroke}{rgb}{0.870588,0.733333,0.607843}%
\pgfsetstrokecolor{currentstroke}%
\pgfsetstrokeopacity{0.200000}%
\pgfsetdash{}{0pt}%
\pgfpathmoveto{\pgfqpoint{5.009171in}{2.423982in}}%
\pgfpathlineto{\pgfqpoint{4.298154in}{2.772028in}}%
\pgfusepath{stroke}%
\end{pgfscope}%
\begin{pgfscope}%
\pgfpathrectangle{\pgfqpoint{0.526127in}{0.331635in}}{\pgfqpoint{9.300000in}{7.700000in}}%
\pgfusepath{clip}%
\pgfsetrectcap%
\pgfsetroundjoin%
\pgfsetlinewidth{1.505625pt}%
\definecolor{currentstroke}{rgb}{0.870588,0.733333,0.607843}%
\pgfsetstrokecolor{currentstroke}%
\pgfsetstrokeopacity{0.200000}%
\pgfsetdash{}{0pt}%
\pgfpathmoveto{\pgfqpoint{5.648571in}{1.147470in}}%
\pgfpathlineto{\pgfqpoint{4.298154in}{2.772028in}}%
\pgfusepath{stroke}%
\end{pgfscope}%
\begin{pgfscope}%
\pgfpathrectangle{\pgfqpoint{0.526127in}{0.331635in}}{\pgfqpoint{9.300000in}{7.700000in}}%
\pgfusepath{clip}%
\pgfsetrectcap%
\pgfsetroundjoin%
\pgfsetlinewidth{1.505625pt}%
\definecolor{currentstroke}{rgb}{0.870588,0.733333,0.607843}%
\pgfsetstrokecolor{currentstroke}%
\pgfsetstrokeopacity{0.200000}%
\pgfsetdash{}{0pt}%
\pgfpathmoveto{\pgfqpoint{4.947462in}{1.693509in}}%
\pgfpathlineto{\pgfqpoint{4.298154in}{2.772028in}}%
\pgfusepath{stroke}%
\end{pgfscope}%
\begin{pgfscope}%
\pgfpathrectangle{\pgfqpoint{0.526127in}{0.331635in}}{\pgfqpoint{9.300000in}{7.700000in}}%
\pgfusepath{clip}%
\pgfsetrectcap%
\pgfsetroundjoin%
\pgfsetlinewidth{1.505625pt}%
\definecolor{currentstroke}{rgb}{0.870588,0.733333,0.607843}%
\pgfsetstrokecolor{currentstroke}%
\pgfsetstrokeopacity{0.200000}%
\pgfsetdash{}{0pt}%
\pgfpathmoveto{\pgfqpoint{5.178715in}{1.674266in}}%
\pgfpathlineto{\pgfqpoint{4.298154in}{2.772028in}}%
\pgfusepath{stroke}%
\end{pgfscope}%
\begin{pgfscope}%
\pgfpathrectangle{\pgfqpoint{0.526127in}{0.331635in}}{\pgfqpoint{9.300000in}{7.700000in}}%
\pgfusepath{clip}%
\pgfsetrectcap%
\pgfsetroundjoin%
\pgfsetlinewidth{1.505625pt}%
\definecolor{currentstroke}{rgb}{0.870588,0.733333,0.607843}%
\pgfsetstrokecolor{currentstroke}%
\pgfsetstrokeopacity{0.200000}%
\pgfsetdash{}{0pt}%
\pgfpathmoveto{\pgfqpoint{6.316003in}{1.938392in}}%
\pgfpathlineto{\pgfqpoint{4.298154in}{2.772028in}}%
\pgfusepath{stroke}%
\end{pgfscope}%
\begin{pgfscope}%
\pgfpathrectangle{\pgfqpoint{0.526127in}{0.331635in}}{\pgfqpoint{9.300000in}{7.700000in}}%
\pgfusepath{clip}%
\pgfsetrectcap%
\pgfsetroundjoin%
\pgfsetlinewidth{1.505625pt}%
\definecolor{currentstroke}{rgb}{0.870588,0.733333,0.607843}%
\pgfsetstrokecolor{currentstroke}%
\pgfsetstrokeopacity{0.200000}%
\pgfsetdash{}{0pt}%
\pgfpathmoveto{\pgfqpoint{4.799485in}{2.374526in}}%
\pgfpathlineto{\pgfqpoint{4.298154in}{2.772028in}}%
\pgfusepath{stroke}%
\end{pgfscope}%
\begin{pgfscope}%
\pgfpathrectangle{\pgfqpoint{0.526127in}{0.331635in}}{\pgfqpoint{9.300000in}{7.700000in}}%
\pgfusepath{clip}%
\pgfsetrectcap%
\pgfsetroundjoin%
\pgfsetlinewidth{1.505625pt}%
\definecolor{currentstroke}{rgb}{0.870588,0.733333,0.607843}%
\pgfsetstrokecolor{currentstroke}%
\pgfsetstrokeopacity{0.200000}%
\pgfsetdash{}{0pt}%
\pgfpathmoveto{\pgfqpoint{3.925973in}{2.867687in}}%
\pgfpathlineto{\pgfqpoint{4.298154in}{2.772028in}}%
\pgfusepath{stroke}%
\end{pgfscope}%
\begin{pgfscope}%
\pgfpathrectangle{\pgfqpoint{0.526127in}{0.331635in}}{\pgfqpoint{9.300000in}{7.700000in}}%
\pgfusepath{clip}%
\pgfsetrectcap%
\pgfsetroundjoin%
\pgfsetlinewidth{1.505625pt}%
\definecolor{currentstroke}{rgb}{0.870588,0.733333,0.607843}%
\pgfsetstrokecolor{currentstroke}%
\pgfsetstrokeopacity{0.200000}%
\pgfsetdash{}{0pt}%
\pgfpathmoveto{\pgfqpoint{2.545411in}{1.795358in}}%
\pgfpathlineto{\pgfqpoint{4.298154in}{2.772028in}}%
\pgfusepath{stroke}%
\end{pgfscope}%
\begin{pgfscope}%
\pgfpathrectangle{\pgfqpoint{0.526127in}{0.331635in}}{\pgfqpoint{9.300000in}{7.700000in}}%
\pgfusepath{clip}%
\pgfsetrectcap%
\pgfsetroundjoin%
\pgfsetlinewidth{1.505625pt}%
\definecolor{currentstroke}{rgb}{0.870588,0.733333,0.607843}%
\pgfsetstrokecolor{currentstroke}%
\pgfsetstrokeopacity{0.200000}%
\pgfsetdash{}{0pt}%
\pgfpathmoveto{\pgfqpoint{4.257376in}{1.921035in}}%
\pgfpathlineto{\pgfqpoint{4.298154in}{2.772028in}}%
\pgfusepath{stroke}%
\end{pgfscope}%
\begin{pgfscope}%
\pgfpathrectangle{\pgfqpoint{0.526127in}{0.331635in}}{\pgfqpoint{9.300000in}{7.700000in}}%
\pgfusepath{clip}%
\pgfsetrectcap%
\pgfsetroundjoin%
\pgfsetlinewidth{1.505625pt}%
\definecolor{currentstroke}{rgb}{0.980392,0.690196,0.894118}%
\pgfsetstrokecolor{currentstroke}%
\pgfsetstrokeopacity{0.800000}%
\pgfsetdash{}{0pt}%
\pgfpathmoveto{\pgfqpoint{8.127617in}{4.829463in}}%
\pgfpathlineto{\pgfqpoint{5.989578in}{5.077315in}}%
\pgfusepath{stroke}%
\end{pgfscope}%
\begin{pgfscope}%
\pgfpathrectangle{\pgfqpoint{0.526127in}{0.331635in}}{\pgfqpoint{9.300000in}{7.700000in}}%
\pgfusepath{clip}%
\pgfsetrectcap%
\pgfsetroundjoin%
\pgfsetlinewidth{1.505625pt}%
\definecolor{currentstroke}{rgb}{0.980392,0.690196,0.894118}%
\pgfsetstrokecolor{currentstroke}%
\pgfsetstrokeopacity{0.800000}%
\pgfsetdash{}{0pt}%
\pgfpathmoveto{\pgfqpoint{5.250982in}{7.334764in}}%
\pgfpathlineto{\pgfqpoint{5.989578in}{5.077315in}}%
\pgfusepath{stroke}%
\end{pgfscope}%
\begin{pgfscope}%
\pgfpathrectangle{\pgfqpoint{0.526127in}{0.331635in}}{\pgfqpoint{9.300000in}{7.700000in}}%
\pgfusepath{clip}%
\pgfsetrectcap%
\pgfsetroundjoin%
\pgfsetlinewidth{1.505625pt}%
\definecolor{currentstroke}{rgb}{0.980392,0.690196,0.894118}%
\pgfsetstrokecolor{currentstroke}%
\pgfsetstrokeopacity{0.800000}%
\pgfsetdash{}{0pt}%
\pgfpathmoveto{\pgfqpoint{4.648550in}{2.936165in}}%
\pgfpathlineto{\pgfqpoint{5.989578in}{5.077315in}}%
\pgfusepath{stroke}%
\end{pgfscope}%
\begin{pgfscope}%
\pgfpathrectangle{\pgfqpoint{0.526127in}{0.331635in}}{\pgfqpoint{9.300000in}{7.700000in}}%
\pgfusepath{clip}%
\pgfsetrectcap%
\pgfsetroundjoin%
\pgfsetlinewidth{1.505625pt}%
\definecolor{currentstroke}{rgb}{0.980392,0.690196,0.894118}%
\pgfsetstrokecolor{currentstroke}%
\pgfsetstrokeopacity{0.800000}%
\pgfsetdash{}{0pt}%
\pgfpathmoveto{\pgfqpoint{6.518469in}{6.123645in}}%
\pgfpathlineto{\pgfqpoint{5.989578in}{5.077315in}}%
\pgfusepath{stroke}%
\end{pgfscope}%
\begin{pgfscope}%
\pgfpathrectangle{\pgfqpoint{0.526127in}{0.331635in}}{\pgfqpoint{9.300000in}{7.700000in}}%
\pgfusepath{clip}%
\pgfsetrectcap%
\pgfsetroundjoin%
\pgfsetlinewidth{1.505625pt}%
\definecolor{currentstroke}{rgb}{0.980392,0.690196,0.894118}%
\pgfsetstrokecolor{currentstroke}%
\pgfsetstrokeopacity{0.800000}%
\pgfsetdash{}{0pt}%
\pgfpathmoveto{\pgfqpoint{8.083074in}{4.741037in}}%
\pgfpathlineto{\pgfqpoint{5.989578in}{5.077315in}}%
\pgfusepath{stroke}%
\end{pgfscope}%
\begin{pgfscope}%
\pgfpathrectangle{\pgfqpoint{0.526127in}{0.331635in}}{\pgfqpoint{9.300000in}{7.700000in}}%
\pgfusepath{clip}%
\pgfsetrectcap%
\pgfsetroundjoin%
\pgfsetlinewidth{1.505625pt}%
\definecolor{currentstroke}{rgb}{0.980392,0.690196,0.894118}%
\pgfsetstrokecolor{currentstroke}%
\pgfsetstrokeopacity{0.800000}%
\pgfsetdash{}{0pt}%
\pgfpathmoveto{\pgfqpoint{7.058761in}{4.131981in}}%
\pgfpathlineto{\pgfqpoint{5.989578in}{5.077315in}}%
\pgfusepath{stroke}%
\end{pgfscope}%
\begin{pgfscope}%
\pgfpathrectangle{\pgfqpoint{0.526127in}{0.331635in}}{\pgfqpoint{9.300000in}{7.700000in}}%
\pgfusepath{clip}%
\pgfsetrectcap%
\pgfsetroundjoin%
\pgfsetlinewidth{1.505625pt}%
\definecolor{currentstroke}{rgb}{0.980392,0.690196,0.894118}%
\pgfsetstrokecolor{currentstroke}%
\pgfsetstrokeopacity{0.800000}%
\pgfsetdash{}{0pt}%
\pgfpathmoveto{\pgfqpoint{8.288664in}{4.881967in}}%
\pgfpathlineto{\pgfqpoint{5.989578in}{5.077315in}}%
\pgfusepath{stroke}%
\end{pgfscope}%
\begin{pgfscope}%
\pgfpathrectangle{\pgfqpoint{0.526127in}{0.331635in}}{\pgfqpoint{9.300000in}{7.700000in}}%
\pgfusepath{clip}%
\pgfsetrectcap%
\pgfsetroundjoin%
\pgfsetlinewidth{1.505625pt}%
\definecolor{currentstroke}{rgb}{0.980392,0.690196,0.894118}%
\pgfsetstrokecolor{currentstroke}%
\pgfsetstrokeopacity{0.800000}%
\pgfsetdash{}{0pt}%
\pgfpathmoveto{\pgfqpoint{8.365658in}{4.215602in}}%
\pgfpathlineto{\pgfqpoint{5.989578in}{5.077315in}}%
\pgfusepath{stroke}%
\end{pgfscope}%
\begin{pgfscope}%
\pgfpathrectangle{\pgfqpoint{0.526127in}{0.331635in}}{\pgfqpoint{9.300000in}{7.700000in}}%
\pgfusepath{clip}%
\pgfsetrectcap%
\pgfsetroundjoin%
\pgfsetlinewidth{1.505625pt}%
\definecolor{currentstroke}{rgb}{0.980392,0.690196,0.894118}%
\pgfsetstrokecolor{currentstroke}%
\pgfsetstrokeopacity{0.800000}%
\pgfsetdash{}{0pt}%
\pgfpathmoveto{\pgfqpoint{5.586667in}{6.021828in}}%
\pgfpathlineto{\pgfqpoint{5.989578in}{5.077315in}}%
\pgfusepath{stroke}%
\end{pgfscope}%
\begin{pgfscope}%
\pgfpathrectangle{\pgfqpoint{0.526127in}{0.331635in}}{\pgfqpoint{9.300000in}{7.700000in}}%
\pgfusepath{clip}%
\pgfsetrectcap%
\pgfsetroundjoin%
\pgfsetlinewidth{1.505625pt}%
\definecolor{currentstroke}{rgb}{0.980392,0.690196,0.894118}%
\pgfsetstrokecolor{currentstroke}%
\pgfsetstrokeopacity{0.800000}%
\pgfsetdash{}{0pt}%
\pgfpathmoveto{\pgfqpoint{7.411729in}{4.264611in}}%
\pgfpathlineto{\pgfqpoint{5.989578in}{5.077315in}}%
\pgfusepath{stroke}%
\end{pgfscope}%
\begin{pgfscope}%
\pgfpathrectangle{\pgfqpoint{0.526127in}{0.331635in}}{\pgfqpoint{9.300000in}{7.700000in}}%
\pgfusepath{clip}%
\pgfsetrectcap%
\pgfsetroundjoin%
\pgfsetlinewidth{1.505625pt}%
\definecolor{currentstroke}{rgb}{0.980392,0.690196,0.894118}%
\pgfsetstrokecolor{currentstroke}%
\pgfsetstrokeopacity{0.800000}%
\pgfsetdash{}{0pt}%
\pgfpathmoveto{\pgfqpoint{5.552197in}{5.497674in}}%
\pgfpathlineto{\pgfqpoint{5.989578in}{5.077315in}}%
\pgfusepath{stroke}%
\end{pgfscope}%
\begin{pgfscope}%
\pgfpathrectangle{\pgfqpoint{0.526127in}{0.331635in}}{\pgfqpoint{9.300000in}{7.700000in}}%
\pgfusepath{clip}%
\pgfsetrectcap%
\pgfsetroundjoin%
\pgfsetlinewidth{1.505625pt}%
\definecolor{currentstroke}{rgb}{0.980392,0.690196,0.894118}%
\pgfsetstrokecolor{currentstroke}%
\pgfsetstrokeopacity{0.800000}%
\pgfsetdash{}{0pt}%
\pgfpathmoveto{\pgfqpoint{5.112613in}{6.023531in}}%
\pgfpathlineto{\pgfqpoint{5.989578in}{5.077315in}}%
\pgfusepath{stroke}%
\end{pgfscope}%
\begin{pgfscope}%
\pgfpathrectangle{\pgfqpoint{0.526127in}{0.331635in}}{\pgfqpoint{9.300000in}{7.700000in}}%
\pgfusepath{clip}%
\pgfsetrectcap%
\pgfsetroundjoin%
\pgfsetlinewidth{1.505625pt}%
\definecolor{currentstroke}{rgb}{0.980392,0.690196,0.894118}%
\pgfsetstrokecolor{currentstroke}%
\pgfsetstrokeopacity{0.800000}%
\pgfsetdash{}{0pt}%
\pgfpathmoveto{\pgfqpoint{3.785574in}{6.661231in}}%
\pgfpathlineto{\pgfqpoint{5.989578in}{5.077315in}}%
\pgfusepath{stroke}%
\end{pgfscope}%
\begin{pgfscope}%
\pgfpathrectangle{\pgfqpoint{0.526127in}{0.331635in}}{\pgfqpoint{9.300000in}{7.700000in}}%
\pgfusepath{clip}%
\pgfsetrectcap%
\pgfsetroundjoin%
\pgfsetlinewidth{1.505625pt}%
\definecolor{currentstroke}{rgb}{0.980392,0.690196,0.894118}%
\pgfsetstrokecolor{currentstroke}%
\pgfsetstrokeopacity{0.800000}%
\pgfsetdash{}{0pt}%
\pgfpathmoveto{\pgfqpoint{3.136695in}{6.238697in}}%
\pgfpathlineto{\pgfqpoint{5.989578in}{5.077315in}}%
\pgfusepath{stroke}%
\end{pgfscope}%
\begin{pgfscope}%
\pgfpathrectangle{\pgfqpoint{0.526127in}{0.331635in}}{\pgfqpoint{9.300000in}{7.700000in}}%
\pgfusepath{clip}%
\pgfsetrectcap%
\pgfsetroundjoin%
\pgfsetlinewidth{1.505625pt}%
\definecolor{currentstroke}{rgb}{0.980392,0.690196,0.894118}%
\pgfsetstrokecolor{currentstroke}%
\pgfsetstrokeopacity{0.800000}%
\pgfsetdash{}{0pt}%
\pgfpathmoveto{\pgfqpoint{8.338475in}{3.521673in}}%
\pgfpathlineto{\pgfqpoint{5.989578in}{5.077315in}}%
\pgfusepath{stroke}%
\end{pgfscope}%
\begin{pgfscope}%
\pgfpathrectangle{\pgfqpoint{0.526127in}{0.331635in}}{\pgfqpoint{9.300000in}{7.700000in}}%
\pgfusepath{clip}%
\pgfsetrectcap%
\pgfsetroundjoin%
\pgfsetlinewidth{1.505625pt}%
\definecolor{currentstroke}{rgb}{0.980392,0.690196,0.894118}%
\pgfsetstrokecolor{currentstroke}%
\pgfsetstrokeopacity{0.800000}%
\pgfsetdash{}{0pt}%
\pgfpathmoveto{\pgfqpoint{2.725282in}{4.747999in}}%
\pgfpathlineto{\pgfqpoint{5.989578in}{5.077315in}}%
\pgfusepath{stroke}%
\end{pgfscope}%
\begin{pgfscope}%
\pgfpathrectangle{\pgfqpoint{0.526127in}{0.331635in}}{\pgfqpoint{9.300000in}{7.700000in}}%
\pgfusepath{clip}%
\pgfsetrectcap%
\pgfsetroundjoin%
\pgfsetlinewidth{1.505625pt}%
\definecolor{currentstroke}{rgb}{0.980392,0.690196,0.894118}%
\pgfsetstrokecolor{currentstroke}%
\pgfsetstrokeopacity{0.800000}%
\pgfsetdash{}{0pt}%
\pgfpathmoveto{\pgfqpoint{5.804992in}{2.192409in}}%
\pgfpathlineto{\pgfqpoint{5.989578in}{5.077315in}}%
\pgfusepath{stroke}%
\end{pgfscope}%
\begin{pgfscope}%
\pgfpathrectangle{\pgfqpoint{0.526127in}{0.331635in}}{\pgfqpoint{9.300000in}{7.700000in}}%
\pgfusepath{clip}%
\pgfsetrectcap%
\pgfsetroundjoin%
\pgfsetlinewidth{1.505625pt}%
\definecolor{currentstroke}{rgb}{0.980392,0.690196,0.894118}%
\pgfsetstrokecolor{currentstroke}%
\pgfsetstrokeopacity{0.800000}%
\pgfsetdash{}{0pt}%
\pgfpathmoveto{\pgfqpoint{4.387621in}{6.311919in}}%
\pgfpathlineto{\pgfqpoint{5.989578in}{5.077315in}}%
\pgfusepath{stroke}%
\end{pgfscope}%
\begin{pgfscope}%
\pgfpathrectangle{\pgfqpoint{0.526127in}{0.331635in}}{\pgfqpoint{9.300000in}{7.700000in}}%
\pgfusepath{clip}%
\pgfsetrectcap%
\pgfsetroundjoin%
\pgfsetlinewidth{1.505625pt}%
\definecolor{currentstroke}{rgb}{0.980392,0.690196,0.894118}%
\pgfsetstrokecolor{currentstroke}%
\pgfsetstrokeopacity{0.800000}%
\pgfsetdash{}{0pt}%
\pgfpathmoveto{\pgfqpoint{4.353160in}{5.800789in}}%
\pgfpathlineto{\pgfqpoint{5.989578in}{5.077315in}}%
\pgfusepath{stroke}%
\end{pgfscope}%
\begin{pgfscope}%
\pgfpathrectangle{\pgfqpoint{0.526127in}{0.331635in}}{\pgfqpoint{9.300000in}{7.700000in}}%
\pgfusepath{clip}%
\pgfsetrectcap%
\pgfsetroundjoin%
\pgfsetlinewidth{1.505625pt}%
\definecolor{currentstroke}{rgb}{0.980392,0.690196,0.894118}%
\pgfsetstrokecolor{currentstroke}%
\pgfsetstrokeopacity{0.800000}%
\pgfsetdash{}{0pt}%
\pgfpathmoveto{\pgfqpoint{4.693280in}{5.829103in}}%
\pgfpathlineto{\pgfqpoint{5.989578in}{5.077315in}}%
\pgfusepath{stroke}%
\end{pgfscope}%
\begin{pgfscope}%
\pgfpathrectangle{\pgfqpoint{0.526127in}{0.331635in}}{\pgfqpoint{9.300000in}{7.700000in}}%
\pgfusepath{clip}%
\pgfsetrectcap%
\pgfsetroundjoin%
\pgfsetlinewidth{1.505625pt}%
\definecolor{currentstroke}{rgb}{0.980392,0.690196,0.894118}%
\pgfsetstrokecolor{currentstroke}%
\pgfsetstrokeopacity{0.800000}%
\pgfsetdash{}{0pt}%
\pgfpathmoveto{\pgfqpoint{9.403399in}{5.656119in}}%
\pgfpathlineto{\pgfqpoint{5.989578in}{5.077315in}}%
\pgfusepath{stroke}%
\end{pgfscope}%
\begin{pgfscope}%
\pgfpathrectangle{\pgfqpoint{0.526127in}{0.331635in}}{\pgfqpoint{9.300000in}{7.700000in}}%
\pgfusepath{clip}%
\pgfsetrectcap%
\pgfsetroundjoin%
\pgfsetlinewidth{1.505625pt}%
\definecolor{currentstroke}{rgb}{0.980392,0.690196,0.894118}%
\pgfsetstrokecolor{currentstroke}%
\pgfsetstrokeopacity{0.800000}%
\pgfsetdash{}{0pt}%
\pgfpathmoveto{\pgfqpoint{4.320810in}{3.351396in}}%
\pgfpathlineto{\pgfqpoint{5.989578in}{5.077315in}}%
\pgfusepath{stroke}%
\end{pgfscope}%
\begin{pgfscope}%
\pgfpathrectangle{\pgfqpoint{0.526127in}{0.331635in}}{\pgfqpoint{9.300000in}{7.700000in}}%
\pgfusepath{clip}%
\pgfsetrectcap%
\pgfsetroundjoin%
\pgfsetlinewidth{1.505625pt}%
\definecolor{currentstroke}{rgb}{0.980392,0.690196,0.894118}%
\pgfsetstrokecolor{currentstroke}%
\pgfsetstrokeopacity{0.800000}%
\pgfsetdash{}{0pt}%
\pgfpathmoveto{\pgfqpoint{1.703182in}{5.256898in}}%
\pgfpathlineto{\pgfqpoint{5.989578in}{5.077315in}}%
\pgfusepath{stroke}%
\end{pgfscope}%
\begin{pgfscope}%
\pgfpathrectangle{\pgfqpoint{0.526127in}{0.331635in}}{\pgfqpoint{9.300000in}{7.700000in}}%
\pgfusepath{clip}%
\pgfsetrectcap%
\pgfsetroundjoin%
\pgfsetlinewidth{1.505625pt}%
\definecolor{currentstroke}{rgb}{0.980392,0.690196,0.894118}%
\pgfsetstrokecolor{currentstroke}%
\pgfsetstrokeopacity{0.800000}%
\pgfsetdash{}{0pt}%
\pgfpathmoveto{\pgfqpoint{4.829760in}{6.478800in}}%
\pgfpathlineto{\pgfqpoint{5.989578in}{5.077315in}}%
\pgfusepath{stroke}%
\end{pgfscope}%
\begin{pgfscope}%
\pgfpathrectangle{\pgfqpoint{0.526127in}{0.331635in}}{\pgfqpoint{9.300000in}{7.700000in}}%
\pgfusepath{clip}%
\pgfsetrectcap%
\pgfsetroundjoin%
\pgfsetlinewidth{1.505625pt}%
\definecolor{currentstroke}{rgb}{0.980392,0.690196,0.894118}%
\pgfsetstrokecolor{currentstroke}%
\pgfsetstrokeopacity{0.800000}%
\pgfsetdash{}{0pt}%
\pgfpathmoveto{\pgfqpoint{5.312702in}{4.335644in}}%
\pgfpathlineto{\pgfqpoint{5.989578in}{5.077315in}}%
\pgfusepath{stroke}%
\end{pgfscope}%
\begin{pgfscope}%
\pgfpathrectangle{\pgfqpoint{0.526127in}{0.331635in}}{\pgfqpoint{9.300000in}{7.700000in}}%
\pgfusepath{clip}%
\pgfsetrectcap%
\pgfsetroundjoin%
\pgfsetlinewidth{1.505625pt}%
\definecolor{currentstroke}{rgb}{0.980392,0.690196,0.894118}%
\pgfsetstrokecolor{currentstroke}%
\pgfsetstrokeopacity{0.800000}%
\pgfsetdash{}{0pt}%
\pgfpathmoveto{\pgfqpoint{8.614970in}{4.769538in}}%
\pgfpathlineto{\pgfqpoint{5.989578in}{5.077315in}}%
\pgfusepath{stroke}%
\end{pgfscope}%
\begin{pgfscope}%
\pgfpathrectangle{\pgfqpoint{0.526127in}{0.331635in}}{\pgfqpoint{9.300000in}{7.700000in}}%
\pgfusepath{clip}%
\pgfsetrectcap%
\pgfsetroundjoin%
\pgfsetlinewidth{1.505625pt}%
\definecolor{currentstroke}{rgb}{0.980392,0.690196,0.894118}%
\pgfsetstrokecolor{currentstroke}%
\pgfsetstrokeopacity{0.800000}%
\pgfsetdash{}{0pt}%
\pgfpathmoveto{\pgfqpoint{8.457929in}{5.831669in}}%
\pgfpathlineto{\pgfqpoint{5.989578in}{5.077315in}}%
\pgfusepath{stroke}%
\end{pgfscope}%
\begin{pgfscope}%
\pgfpathrectangle{\pgfqpoint{0.526127in}{0.331635in}}{\pgfqpoint{9.300000in}{7.700000in}}%
\pgfusepath{clip}%
\pgfsetrectcap%
\pgfsetroundjoin%
\pgfsetlinewidth{1.505625pt}%
\definecolor{currentstroke}{rgb}{0.980392,0.690196,0.894118}%
\pgfsetstrokecolor{currentstroke}%
\pgfsetstrokeopacity{0.800000}%
\pgfsetdash{}{0pt}%
\pgfpathmoveto{\pgfqpoint{7.835370in}{4.178659in}}%
\pgfpathlineto{\pgfqpoint{5.989578in}{5.077315in}}%
\pgfusepath{stroke}%
\end{pgfscope}%
\begin{pgfscope}%
\pgfpathrectangle{\pgfqpoint{0.526127in}{0.331635in}}{\pgfqpoint{9.300000in}{7.700000in}}%
\pgfusepath{clip}%
\pgfsetrectcap%
\pgfsetroundjoin%
\pgfsetlinewidth{1.505625pt}%
\definecolor{currentstroke}{rgb}{0.721569,0.521569,0.039216}%
\pgfsetstrokecolor{currentstroke}%
\pgfsetstrokeopacity{0.800000}%
\pgfsetdash{}{0pt}%
\pgfpathmoveto{\pgfqpoint{5.776420in}{4.079147in}}%
\pgfpathlineto{\pgfqpoint{5.032891in}{4.505800in}}%
\pgfusepath{stroke}%
\end{pgfscope}%
\begin{pgfscope}%
\pgfpathrectangle{\pgfqpoint{0.526127in}{0.331635in}}{\pgfqpoint{9.300000in}{7.700000in}}%
\pgfusepath{clip}%
\pgfsetrectcap%
\pgfsetroundjoin%
\pgfsetlinewidth{1.505625pt}%
\definecolor{currentstroke}{rgb}{0.721569,0.521569,0.039216}%
\pgfsetstrokecolor{currentstroke}%
\pgfsetstrokeopacity{0.800000}%
\pgfsetdash{}{0pt}%
\pgfpathmoveto{\pgfqpoint{8.729922in}{5.055972in}}%
\pgfpathlineto{\pgfqpoint{5.032891in}{4.505800in}}%
\pgfusepath{stroke}%
\end{pgfscope}%
\begin{pgfscope}%
\pgfpathrectangle{\pgfqpoint{0.526127in}{0.331635in}}{\pgfqpoint{9.300000in}{7.700000in}}%
\pgfusepath{clip}%
\pgfsetrectcap%
\pgfsetroundjoin%
\pgfsetlinewidth{1.505625pt}%
\definecolor{currentstroke}{rgb}{0.721569,0.521569,0.039216}%
\pgfsetstrokecolor{currentstroke}%
\pgfsetstrokeopacity{0.800000}%
\pgfsetdash{}{0pt}%
\pgfpathmoveto{\pgfqpoint{2.913096in}{4.658470in}}%
\pgfpathlineto{\pgfqpoint{5.032891in}{4.505800in}}%
\pgfusepath{stroke}%
\end{pgfscope}%
\begin{pgfscope}%
\pgfpathrectangle{\pgfqpoint{0.526127in}{0.331635in}}{\pgfqpoint{9.300000in}{7.700000in}}%
\pgfusepath{clip}%
\pgfsetrectcap%
\pgfsetroundjoin%
\pgfsetlinewidth{1.505625pt}%
\definecolor{currentstroke}{rgb}{0.721569,0.521569,0.039216}%
\pgfsetstrokecolor{currentstroke}%
\pgfsetstrokeopacity{0.800000}%
\pgfsetdash{}{0pt}%
\pgfpathmoveto{\pgfqpoint{3.918807in}{5.779850in}}%
\pgfpathlineto{\pgfqpoint{5.032891in}{4.505800in}}%
\pgfusepath{stroke}%
\end{pgfscope}%
\begin{pgfscope}%
\pgfpathrectangle{\pgfqpoint{0.526127in}{0.331635in}}{\pgfqpoint{9.300000in}{7.700000in}}%
\pgfusepath{clip}%
\pgfsetrectcap%
\pgfsetroundjoin%
\pgfsetlinewidth{1.505625pt}%
\definecolor{currentstroke}{rgb}{0.721569,0.521569,0.039216}%
\pgfsetstrokecolor{currentstroke}%
\pgfsetstrokeopacity{0.800000}%
\pgfsetdash{}{0pt}%
\pgfpathmoveto{\pgfqpoint{2.663268in}{3.876036in}}%
\pgfpathlineto{\pgfqpoint{5.032891in}{4.505800in}}%
\pgfusepath{stroke}%
\end{pgfscope}%
\begin{pgfscope}%
\pgfpathrectangle{\pgfqpoint{0.526127in}{0.331635in}}{\pgfqpoint{9.300000in}{7.700000in}}%
\pgfusepath{clip}%
\pgfsetrectcap%
\pgfsetroundjoin%
\pgfsetlinewidth{1.505625pt}%
\definecolor{currentstroke}{rgb}{0.721569,0.521569,0.039216}%
\pgfsetstrokecolor{currentstroke}%
\pgfsetstrokeopacity{0.800000}%
\pgfsetdash{}{0pt}%
\pgfpathmoveto{\pgfqpoint{6.282605in}{4.955017in}}%
\pgfpathlineto{\pgfqpoint{5.032891in}{4.505800in}}%
\pgfusepath{stroke}%
\end{pgfscope}%
\begin{pgfscope}%
\pgfpathrectangle{\pgfqpoint{0.526127in}{0.331635in}}{\pgfqpoint{9.300000in}{7.700000in}}%
\pgfusepath{clip}%
\pgfsetrectcap%
\pgfsetroundjoin%
\pgfsetlinewidth{1.505625pt}%
\definecolor{currentstroke}{rgb}{0.721569,0.521569,0.039216}%
\pgfsetstrokecolor{currentstroke}%
\pgfsetstrokeopacity{0.800000}%
\pgfsetdash{}{0pt}%
\pgfpathmoveto{\pgfqpoint{3.179398in}{3.452987in}}%
\pgfpathlineto{\pgfqpoint{5.032891in}{4.505800in}}%
\pgfusepath{stroke}%
\end{pgfscope}%
\begin{pgfscope}%
\pgfpathrectangle{\pgfqpoint{0.526127in}{0.331635in}}{\pgfqpoint{9.300000in}{7.700000in}}%
\pgfusepath{clip}%
\pgfsetrectcap%
\pgfsetroundjoin%
\pgfsetlinewidth{1.505625pt}%
\definecolor{currentstroke}{rgb}{0.721569,0.521569,0.039216}%
\pgfsetstrokecolor{currentstroke}%
\pgfsetstrokeopacity{0.800000}%
\pgfsetdash{}{0pt}%
\pgfpathmoveto{\pgfqpoint{4.240309in}{7.120458in}}%
\pgfpathlineto{\pgfqpoint{5.032891in}{4.505800in}}%
\pgfusepath{stroke}%
\end{pgfscope}%
\begin{pgfscope}%
\pgfpathrectangle{\pgfqpoint{0.526127in}{0.331635in}}{\pgfqpoint{9.300000in}{7.700000in}}%
\pgfusepath{clip}%
\pgfsetrectcap%
\pgfsetroundjoin%
\pgfsetlinewidth{1.505625pt}%
\definecolor{currentstroke}{rgb}{0.721569,0.521569,0.039216}%
\pgfsetstrokecolor{currentstroke}%
\pgfsetstrokeopacity{0.800000}%
\pgfsetdash{}{0pt}%
\pgfpathmoveto{\pgfqpoint{5.395356in}{4.893643in}}%
\pgfpathlineto{\pgfqpoint{5.032891in}{4.505800in}}%
\pgfusepath{stroke}%
\end{pgfscope}%
\begin{pgfscope}%
\pgfpathrectangle{\pgfqpoint{0.526127in}{0.331635in}}{\pgfqpoint{9.300000in}{7.700000in}}%
\pgfusepath{clip}%
\pgfsetrectcap%
\pgfsetroundjoin%
\pgfsetlinewidth{1.505625pt}%
\definecolor{currentstroke}{rgb}{0.721569,0.521569,0.039216}%
\pgfsetstrokecolor{currentstroke}%
\pgfsetstrokeopacity{0.800000}%
\pgfsetdash{}{0pt}%
\pgfpathmoveto{\pgfqpoint{4.154359in}{5.130415in}}%
\pgfpathlineto{\pgfqpoint{5.032891in}{4.505800in}}%
\pgfusepath{stroke}%
\end{pgfscope}%
\begin{pgfscope}%
\pgfpathrectangle{\pgfqpoint{0.526127in}{0.331635in}}{\pgfqpoint{9.300000in}{7.700000in}}%
\pgfusepath{clip}%
\pgfsetrectcap%
\pgfsetroundjoin%
\pgfsetlinewidth{1.505625pt}%
\definecolor{currentstroke}{rgb}{0.721569,0.521569,0.039216}%
\pgfsetstrokecolor{currentstroke}%
\pgfsetstrokeopacity{0.800000}%
\pgfsetdash{}{0pt}%
\pgfpathmoveto{\pgfqpoint{3.012801in}{4.477258in}}%
\pgfpathlineto{\pgfqpoint{5.032891in}{4.505800in}}%
\pgfusepath{stroke}%
\end{pgfscope}%
\begin{pgfscope}%
\pgfpathrectangle{\pgfqpoint{0.526127in}{0.331635in}}{\pgfqpoint{9.300000in}{7.700000in}}%
\pgfusepath{clip}%
\pgfsetrectcap%
\pgfsetroundjoin%
\pgfsetlinewidth{1.505625pt}%
\definecolor{currentstroke}{rgb}{0.721569,0.521569,0.039216}%
\pgfsetstrokecolor{currentstroke}%
\pgfsetstrokeopacity{0.800000}%
\pgfsetdash{}{0pt}%
\pgfpathmoveto{\pgfqpoint{6.132477in}{1.011772in}}%
\pgfpathlineto{\pgfqpoint{5.032891in}{4.505800in}}%
\pgfusepath{stroke}%
\end{pgfscope}%
\begin{pgfscope}%
\pgfpathrectangle{\pgfqpoint{0.526127in}{0.331635in}}{\pgfqpoint{9.300000in}{7.700000in}}%
\pgfusepath{clip}%
\pgfsetrectcap%
\pgfsetroundjoin%
\pgfsetlinewidth{1.505625pt}%
\definecolor{currentstroke}{rgb}{0.721569,0.521569,0.039216}%
\pgfsetstrokecolor{currentstroke}%
\pgfsetstrokeopacity{0.800000}%
\pgfsetdash{}{0pt}%
\pgfpathmoveto{\pgfqpoint{8.967963in}{2.899690in}}%
\pgfpathlineto{\pgfqpoint{5.032891in}{4.505800in}}%
\pgfusepath{stroke}%
\end{pgfscope}%
\begin{pgfscope}%
\pgfpathrectangle{\pgfqpoint{0.526127in}{0.331635in}}{\pgfqpoint{9.300000in}{7.700000in}}%
\pgfusepath{clip}%
\pgfsetrectcap%
\pgfsetroundjoin%
\pgfsetlinewidth{1.505625pt}%
\definecolor{currentstroke}{rgb}{0.721569,0.521569,0.039216}%
\pgfsetstrokecolor{currentstroke}%
\pgfsetstrokeopacity{0.800000}%
\pgfsetdash{}{0pt}%
\pgfpathmoveto{\pgfqpoint{2.684948in}{3.939327in}}%
\pgfpathlineto{\pgfqpoint{5.032891in}{4.505800in}}%
\pgfusepath{stroke}%
\end{pgfscope}%
\begin{pgfscope}%
\pgfpathrectangle{\pgfqpoint{0.526127in}{0.331635in}}{\pgfqpoint{9.300000in}{7.700000in}}%
\pgfusepath{clip}%
\pgfsetrectcap%
\pgfsetroundjoin%
\pgfsetlinewidth{1.505625pt}%
\definecolor{currentstroke}{rgb}{0.721569,0.521569,0.039216}%
\pgfsetstrokecolor{currentstroke}%
\pgfsetstrokeopacity{0.800000}%
\pgfsetdash{}{0pt}%
\pgfpathmoveto{\pgfqpoint{7.144222in}{6.645535in}}%
\pgfpathlineto{\pgfqpoint{5.032891in}{4.505800in}}%
\pgfusepath{stroke}%
\end{pgfscope}%
\begin{pgfscope}%
\pgfpathrectangle{\pgfqpoint{0.526127in}{0.331635in}}{\pgfqpoint{9.300000in}{7.700000in}}%
\pgfusepath{clip}%
\pgfsetrectcap%
\pgfsetroundjoin%
\pgfsetlinewidth{1.505625pt}%
\definecolor{currentstroke}{rgb}{0.721569,0.521569,0.039216}%
\pgfsetstrokecolor{currentstroke}%
\pgfsetstrokeopacity{0.800000}%
\pgfsetdash{}{0pt}%
\pgfpathmoveto{\pgfqpoint{5.755207in}{1.746054in}}%
\pgfpathlineto{\pgfqpoint{5.032891in}{4.505800in}}%
\pgfusepath{stroke}%
\end{pgfscope}%
\begin{pgfscope}%
\pgfpathrectangle{\pgfqpoint{0.526127in}{0.331635in}}{\pgfqpoint{9.300000in}{7.700000in}}%
\pgfusepath{clip}%
\pgfsetrectcap%
\pgfsetroundjoin%
\pgfsetlinewidth{1.505625pt}%
\definecolor{currentstroke}{rgb}{0.721569,0.521569,0.039216}%
\pgfsetstrokecolor{currentstroke}%
\pgfsetstrokeopacity{0.800000}%
\pgfsetdash{}{0pt}%
\pgfpathmoveto{\pgfqpoint{9.079707in}{4.416691in}}%
\pgfpathlineto{\pgfqpoint{5.032891in}{4.505800in}}%
\pgfusepath{stroke}%
\end{pgfscope}%
\begin{pgfscope}%
\pgfpathrectangle{\pgfqpoint{0.526127in}{0.331635in}}{\pgfqpoint{9.300000in}{7.700000in}}%
\pgfusepath{clip}%
\pgfsetrectcap%
\pgfsetroundjoin%
\pgfsetlinewidth{1.505625pt}%
\definecolor{currentstroke}{rgb}{0.721569,0.521569,0.039216}%
\pgfsetstrokecolor{currentstroke}%
\pgfsetstrokeopacity{0.800000}%
\pgfsetdash{}{0pt}%
\pgfpathmoveto{\pgfqpoint{4.104581in}{1.312881in}}%
\pgfpathlineto{\pgfqpoint{5.032891in}{4.505800in}}%
\pgfusepath{stroke}%
\end{pgfscope}%
\begin{pgfscope}%
\pgfpathrectangle{\pgfqpoint{0.526127in}{0.331635in}}{\pgfqpoint{9.300000in}{7.700000in}}%
\pgfusepath{clip}%
\pgfsetrectcap%
\pgfsetroundjoin%
\pgfsetlinewidth{1.505625pt}%
\definecolor{currentstroke}{rgb}{0.721569,0.521569,0.039216}%
\pgfsetstrokecolor{currentstroke}%
\pgfsetstrokeopacity{0.800000}%
\pgfsetdash{}{0pt}%
\pgfpathmoveto{\pgfqpoint{1.641796in}{4.156474in}}%
\pgfpathlineto{\pgfqpoint{5.032891in}{4.505800in}}%
\pgfusepath{stroke}%
\end{pgfscope}%
\begin{pgfscope}%
\pgfpathrectangle{\pgfqpoint{0.526127in}{0.331635in}}{\pgfqpoint{9.300000in}{7.700000in}}%
\pgfusepath{clip}%
\pgfsetrectcap%
\pgfsetroundjoin%
\pgfsetlinewidth{1.505625pt}%
\definecolor{currentstroke}{rgb}{0.721569,0.521569,0.039216}%
\pgfsetstrokecolor{currentstroke}%
\pgfsetstrokeopacity{0.800000}%
\pgfsetdash{}{0pt}%
\pgfpathmoveto{\pgfqpoint{2.312627in}{5.998767in}}%
\pgfpathlineto{\pgfqpoint{5.032891in}{4.505800in}}%
\pgfusepath{stroke}%
\end{pgfscope}%
\begin{pgfscope}%
\pgfpathrectangle{\pgfqpoint{0.526127in}{0.331635in}}{\pgfqpoint{9.300000in}{7.700000in}}%
\pgfusepath{clip}%
\pgfsetrectcap%
\pgfsetroundjoin%
\pgfsetlinewidth{1.505625pt}%
\definecolor{currentstroke}{rgb}{0.721569,0.521569,0.039216}%
\pgfsetstrokecolor{currentstroke}%
\pgfsetstrokeopacity{0.800000}%
\pgfsetdash{}{0pt}%
\pgfpathmoveto{\pgfqpoint{7.299025in}{5.752896in}}%
\pgfpathlineto{\pgfqpoint{5.032891in}{4.505800in}}%
\pgfusepath{stroke}%
\end{pgfscope}%
\begin{pgfscope}%
\pgfpathrectangle{\pgfqpoint{0.526127in}{0.331635in}}{\pgfqpoint{9.300000in}{7.700000in}}%
\pgfusepath{clip}%
\pgfsetrectcap%
\pgfsetroundjoin%
\pgfsetlinewidth{1.505625pt}%
\definecolor{currentstroke}{rgb}{0.721569,0.521569,0.039216}%
\pgfsetstrokecolor{currentstroke}%
\pgfsetstrokeopacity{0.800000}%
\pgfsetdash{}{0pt}%
\pgfpathmoveto{\pgfqpoint{1.663441in}{4.106068in}}%
\pgfpathlineto{\pgfqpoint{5.032891in}{4.505800in}}%
\pgfusepath{stroke}%
\end{pgfscope}%
\begin{pgfscope}%
\pgfpathrectangle{\pgfqpoint{0.526127in}{0.331635in}}{\pgfqpoint{9.300000in}{7.700000in}}%
\pgfusepath{clip}%
\pgfsetrectcap%
\pgfsetroundjoin%
\pgfsetlinewidth{1.505625pt}%
\definecolor{currentstroke}{rgb}{0.721569,0.521569,0.039216}%
\pgfsetstrokecolor{currentstroke}%
\pgfsetstrokeopacity{0.800000}%
\pgfsetdash{}{0pt}%
\pgfpathmoveto{\pgfqpoint{9.274981in}{5.558781in}}%
\pgfpathlineto{\pgfqpoint{5.032891in}{4.505800in}}%
\pgfusepath{stroke}%
\end{pgfscope}%
\begin{pgfscope}%
\pgfpathrectangle{\pgfqpoint{0.526127in}{0.331635in}}{\pgfqpoint{9.300000in}{7.700000in}}%
\pgfusepath{clip}%
\pgfsetrectcap%
\pgfsetroundjoin%
\pgfsetlinewidth{1.505625pt}%
\definecolor{currentstroke}{rgb}{0.721569,0.521569,0.039216}%
\pgfsetstrokecolor{currentstroke}%
\pgfsetstrokeopacity{0.800000}%
\pgfsetdash{}{0pt}%
\pgfpathmoveto{\pgfqpoint{9.099248in}{4.690642in}}%
\pgfpathlineto{\pgfqpoint{5.032891in}{4.505800in}}%
\pgfusepath{stroke}%
\end{pgfscope}%
\begin{pgfscope}%
\pgfpathrectangle{\pgfqpoint{0.526127in}{0.331635in}}{\pgfqpoint{9.300000in}{7.700000in}}%
\pgfusepath{clip}%
\pgfsetrectcap%
\pgfsetroundjoin%
\pgfsetlinewidth{1.505625pt}%
\definecolor{currentstroke}{rgb}{0.721569,0.521569,0.039216}%
\pgfsetstrokecolor{currentstroke}%
\pgfsetstrokeopacity{0.800000}%
\pgfsetdash{}{0pt}%
\pgfpathmoveto{\pgfqpoint{4.144255in}{5.886144in}}%
\pgfpathlineto{\pgfqpoint{5.032891in}{4.505800in}}%
\pgfusepath{stroke}%
\end{pgfscope}%
\begin{pgfscope}%
\pgfpathrectangle{\pgfqpoint{0.526127in}{0.331635in}}{\pgfqpoint{9.300000in}{7.700000in}}%
\pgfusepath{clip}%
\pgfsetrectcap%
\pgfsetroundjoin%
\pgfsetlinewidth{1.505625pt}%
\definecolor{currentstroke}{rgb}{0.721569,0.521569,0.039216}%
\pgfsetstrokecolor{currentstroke}%
\pgfsetstrokeopacity{0.800000}%
\pgfsetdash{}{0pt}%
\pgfpathmoveto{\pgfqpoint{6.389607in}{4.514463in}}%
\pgfpathlineto{\pgfqpoint{5.032891in}{4.505800in}}%
\pgfusepath{stroke}%
\end{pgfscope}%
\begin{pgfscope}%
\pgfpathrectangle{\pgfqpoint{0.526127in}{0.331635in}}{\pgfqpoint{9.300000in}{7.700000in}}%
\pgfusepath{clip}%
\pgfsetrectcap%
\pgfsetroundjoin%
\pgfsetlinewidth{1.505625pt}%
\definecolor{currentstroke}{rgb}{0.721569,0.521569,0.039216}%
\pgfsetstrokecolor{currentstroke}%
\pgfsetstrokeopacity{0.800000}%
\pgfsetdash{}{0pt}%
\pgfpathmoveto{\pgfqpoint{3.057300in}{5.120509in}}%
\pgfpathlineto{\pgfqpoint{5.032891in}{4.505800in}}%
\pgfusepath{stroke}%
\end{pgfscope}%
\begin{pgfscope}%
\pgfpathrectangle{\pgfqpoint{0.526127in}{0.331635in}}{\pgfqpoint{9.300000in}{7.700000in}}%
\pgfusepath{clip}%
\pgfsetrectcap%
\pgfsetroundjoin%
\pgfsetlinewidth{1.505625pt}%
\definecolor{currentstroke}{rgb}{0.721569,0.521569,0.039216}%
\pgfsetstrokecolor{currentstroke}%
\pgfsetstrokeopacity{0.800000}%
\pgfsetdash{}{0pt}%
\pgfpathmoveto{\pgfqpoint{1.903215in}{4.926442in}}%
\pgfpathlineto{\pgfqpoint{5.032891in}{4.505800in}}%
\pgfusepath{stroke}%
\end{pgfscope}%
\begin{pgfscope}%
\pgfsetrectcap%
\pgfsetmiterjoin%
\pgfsetlinewidth{0.803000pt}%
\definecolor{currentstroke}{rgb}{0.000000,0.000000,0.000000}%
\pgfsetstrokecolor{currentstroke}%
\pgfsetdash{}{0pt}%
\pgfpathmoveto{\pgfqpoint{0.526127in}{0.331635in}}%
\pgfpathlineto{\pgfqpoint{0.526127in}{8.031635in}}%
\pgfusepath{stroke}%
\end{pgfscope}%
\begin{pgfscope}%
\pgfsetrectcap%
\pgfsetmiterjoin%
\pgfsetlinewidth{0.803000pt}%
\definecolor{currentstroke}{rgb}{0.000000,0.000000,0.000000}%
\pgfsetstrokecolor{currentstroke}%
\pgfsetdash{}{0pt}%
\pgfpathmoveto{\pgfqpoint{9.826127in}{0.331635in}}%
\pgfpathlineto{\pgfqpoint{9.826127in}{8.031635in}}%
\pgfusepath{stroke}%
\end{pgfscope}%
\begin{pgfscope}%
\pgfsetrectcap%
\pgfsetmiterjoin%
\pgfsetlinewidth{0.803000pt}%
\definecolor{currentstroke}{rgb}{0.000000,0.000000,0.000000}%
\pgfsetstrokecolor{currentstroke}%
\pgfsetdash{}{0pt}%
\pgfpathmoveto{\pgfqpoint{0.526127in}{0.331635in}}%
\pgfpathlineto{\pgfqpoint{9.826127in}{0.331635in}}%
\pgfusepath{stroke}%
\end{pgfscope}%
\begin{pgfscope}%
\pgfsetrectcap%
\pgfsetmiterjoin%
\pgfsetlinewidth{0.803000pt}%
\definecolor{currentstroke}{rgb}{0.000000,0.000000,0.000000}%
\pgfsetstrokecolor{currentstroke}%
\pgfsetdash{}{0pt}%
\pgfpathmoveto{\pgfqpoint{0.526127in}{8.031635in}}%
\pgfpathlineto{\pgfqpoint{9.826127in}{8.031635in}}%
\pgfusepath{stroke}%
\end{pgfscope}%
\begin{pgfscope}%
\definecolor{textcolor}{rgb}{0.000000,0.000000,0.000000}%
\pgfsetstrokecolor{textcolor}%
\pgfsetfillcolor{textcolor}%
\pgftext[x=5.176127in,y=8.114968in,,base]{\color{textcolor}\sffamily\fontsize{12.000000}{14.400000}\selectfont Photo-Realistic Images}%
\end{pgfscope}%
\begin{pgfscope}%
\pgfsetbuttcap%
\pgfsetmiterjoin%
\definecolor{currentfill}{rgb}{1.000000,1.000000,1.000000}%
\pgfsetfillcolor{currentfill}%
\pgfsetfillopacity{0.800000}%
\pgfsetlinewidth{1.003750pt}%
\definecolor{currentstroke}{rgb}{0.800000,0.800000,0.800000}%
\pgfsetstrokecolor{currentstroke}%
\pgfsetstrokeopacity{0.800000}%
\pgfsetdash{}{0pt}%
\pgfpathmoveto{\pgfqpoint{9.923349in}{3.345373in}}%
\pgfpathlineto{\pgfqpoint{11.202280in}{3.345373in}}%
\pgfpathquadraticcurveto{\pgfqpoint{11.230057in}{3.345373in}}{\pgfqpoint{11.230057in}{3.373151in}}%
\pgfpathlineto{\pgfqpoint{11.230057in}{4.990119in}}%
\pgfpathquadraticcurveto{\pgfqpoint{11.230057in}{5.017897in}}{\pgfqpoint{11.202280in}{5.017897in}}%
\pgfpathlineto{\pgfqpoint{9.923349in}{5.017897in}}%
\pgfpathquadraticcurveto{\pgfqpoint{9.895571in}{5.017897in}}{\pgfqpoint{9.895571in}{4.990119in}}%
\pgfpathlineto{\pgfqpoint{9.895571in}{3.373151in}}%
\pgfpathquadraticcurveto{\pgfqpoint{9.895571in}{3.345373in}}{\pgfqpoint{9.923349in}{3.345373in}}%
\pgfpathclose%
\pgfusepath{stroke,fill}%
\end{pgfscope}%
\begin{pgfscope}%
\pgfsetbuttcap%
\pgfsetroundjoin%
\definecolor{currentfill}{rgb}{0.631373,0.788235,0.956863}%
\pgfsetfillcolor{currentfill}%
\pgfsetlinewidth{1.003750pt}%
\definecolor{currentstroke}{rgb}{0.631373,0.788235,0.956863}%
\pgfsetstrokecolor{currentstroke}%
\pgfsetdash{}{0pt}%
\pgfsys@defobject{currentmarker}{\pgfqpoint{-0.041667in}{-0.041667in}}{\pgfqpoint{0.041667in}{0.041667in}}{%
\pgfpathmoveto{\pgfqpoint{0.000000in}{-0.041667in}}%
\pgfpathcurveto{\pgfqpoint{0.011050in}{-0.041667in}}{\pgfqpoint{0.021649in}{-0.037276in}}{\pgfqpoint{0.029463in}{-0.029463in}}%
\pgfpathcurveto{\pgfqpoint{0.037276in}{-0.021649in}}{\pgfqpoint{0.041667in}{-0.011050in}}{\pgfqpoint{0.041667in}{0.000000in}}%
\pgfpathcurveto{\pgfqpoint{0.041667in}{0.011050in}}{\pgfqpoint{0.037276in}{0.021649in}}{\pgfqpoint{0.029463in}{0.029463in}}%
\pgfpathcurveto{\pgfqpoint{0.021649in}{0.037276in}}{\pgfqpoint{0.011050in}{0.041667in}}{\pgfqpoint{0.000000in}{0.041667in}}%
\pgfpathcurveto{\pgfqpoint{-0.011050in}{0.041667in}}{\pgfqpoint{-0.021649in}{0.037276in}}{\pgfqpoint{-0.029463in}{0.029463in}}%
\pgfpathcurveto{\pgfqpoint{-0.037276in}{0.021649in}}{\pgfqpoint{-0.041667in}{0.011050in}}{\pgfqpoint{-0.041667in}{0.000000in}}%
\pgfpathcurveto{\pgfqpoint{-0.041667in}{-0.011050in}}{\pgfqpoint{-0.037276in}{-0.021649in}}{\pgfqpoint{-0.029463in}{-0.029463in}}%
\pgfpathcurveto{\pgfqpoint{-0.021649in}{-0.037276in}}{\pgfqpoint{-0.011050in}{-0.041667in}}{\pgfqpoint{0.000000in}{-0.041667in}}%
\pgfpathclose%
\pgfusepath{stroke,fill}%
}%
\begin{pgfscope}%
\pgfsys@transformshift{10.090016in}{4.893277in}%
\pgfsys@useobject{currentmarker}{}%
\end{pgfscope}%
\end{pgfscope}%
\begin{pgfscope}%
\definecolor{textcolor}{rgb}{0.000000,0.000000,0.000000}%
\pgfsetstrokecolor{textcolor}%
\pgfsetfillcolor{textcolor}%
\pgftext[x=10.340016in,y=4.856819in,left,base]{\color{textcolor}\sffamily\fontsize{10.000000}{12.000000}\selectfont openrooms}%
\end{pgfscope}%
\begin{pgfscope}%
\pgfsetbuttcap%
\pgfsetroundjoin%
\definecolor{currentfill}{rgb}{1.000000,0.705882,0.509804}%
\pgfsetfillcolor{currentfill}%
\pgfsetlinewidth{1.003750pt}%
\definecolor{currentstroke}{rgb}{1.000000,0.705882,0.509804}%
\pgfsetstrokecolor{currentstroke}%
\pgfsetdash{}{0pt}%
\pgfsys@defobject{currentmarker}{\pgfqpoint{-0.041667in}{-0.041667in}}{\pgfqpoint{0.041667in}{0.041667in}}{%
\pgfpathmoveto{\pgfqpoint{0.000000in}{-0.041667in}}%
\pgfpathcurveto{\pgfqpoint{0.011050in}{-0.041667in}}{\pgfqpoint{0.021649in}{-0.037276in}}{\pgfqpoint{0.029463in}{-0.029463in}}%
\pgfpathcurveto{\pgfqpoint{0.037276in}{-0.021649in}}{\pgfqpoint{0.041667in}{-0.011050in}}{\pgfqpoint{0.041667in}{0.000000in}}%
\pgfpathcurveto{\pgfqpoint{0.041667in}{0.011050in}}{\pgfqpoint{0.037276in}{0.021649in}}{\pgfqpoint{0.029463in}{0.029463in}}%
\pgfpathcurveto{\pgfqpoint{0.021649in}{0.037276in}}{\pgfqpoint{0.011050in}{0.041667in}}{\pgfqpoint{0.000000in}{0.041667in}}%
\pgfpathcurveto{\pgfqpoint{-0.011050in}{0.041667in}}{\pgfqpoint{-0.021649in}{0.037276in}}{\pgfqpoint{-0.029463in}{0.029463in}}%
\pgfpathcurveto{\pgfqpoint{-0.037276in}{0.021649in}}{\pgfqpoint{-0.041667in}{0.011050in}}{\pgfqpoint{-0.041667in}{0.000000in}}%
\pgfpathcurveto{\pgfqpoint{-0.041667in}{-0.011050in}}{\pgfqpoint{-0.037276in}{-0.021649in}}{\pgfqpoint{-0.029463in}{-0.029463in}}%
\pgfpathcurveto{\pgfqpoint{-0.021649in}{-0.037276in}}{\pgfqpoint{-0.011050in}{-0.041667in}}{\pgfqpoint{0.000000in}{-0.041667in}}%
\pgfpathclose%
\pgfusepath{stroke,fill}%
}%
\begin{pgfscope}%
\pgfsys@transformshift{10.090016in}{4.689420in}%
\pgfsys@useobject{currentmarker}{}%
\end{pgfscope}%
\end{pgfscope}%
\begin{pgfscope}%
\definecolor{textcolor}{rgb}{0.000000,0.000000,0.000000}%
\pgfsetstrokecolor{textcolor}%
\pgfsetfillcolor{textcolor}%
\pgftext[x=10.340016in,y=4.652961in,left,base]{\color{textcolor}\sffamily\fontsize{10.000000}{12.000000}\selectfont scenenet}%
\end{pgfscope}%
\begin{pgfscope}%
\pgfsetbuttcap%
\pgfsetroundjoin%
\definecolor{currentfill}{rgb}{0.552941,0.898039,0.631373}%
\pgfsetfillcolor{currentfill}%
\pgfsetlinewidth{1.003750pt}%
\definecolor{currentstroke}{rgb}{0.552941,0.898039,0.631373}%
\pgfsetstrokecolor{currentstroke}%
\pgfsetdash{}{0pt}%
\pgfsys@defobject{currentmarker}{\pgfqpoint{-0.041667in}{-0.041667in}}{\pgfqpoint{0.041667in}{0.041667in}}{%
\pgfpathmoveto{\pgfqpoint{0.000000in}{-0.041667in}}%
\pgfpathcurveto{\pgfqpoint{0.011050in}{-0.041667in}}{\pgfqpoint{0.021649in}{-0.037276in}}{\pgfqpoint{0.029463in}{-0.029463in}}%
\pgfpathcurveto{\pgfqpoint{0.037276in}{-0.021649in}}{\pgfqpoint{0.041667in}{-0.011050in}}{\pgfqpoint{0.041667in}{0.000000in}}%
\pgfpathcurveto{\pgfqpoint{0.041667in}{0.011050in}}{\pgfqpoint{0.037276in}{0.021649in}}{\pgfqpoint{0.029463in}{0.029463in}}%
\pgfpathcurveto{\pgfqpoint{0.021649in}{0.037276in}}{\pgfqpoint{0.011050in}{0.041667in}}{\pgfqpoint{0.000000in}{0.041667in}}%
\pgfpathcurveto{\pgfqpoint{-0.011050in}{0.041667in}}{\pgfqpoint{-0.021649in}{0.037276in}}{\pgfqpoint{-0.029463in}{0.029463in}}%
\pgfpathcurveto{\pgfqpoint{-0.037276in}{0.021649in}}{\pgfqpoint{-0.041667in}{0.011050in}}{\pgfqpoint{-0.041667in}{0.000000in}}%
\pgfpathcurveto{\pgfqpoint{-0.041667in}{-0.011050in}}{\pgfqpoint{-0.037276in}{-0.021649in}}{\pgfqpoint{-0.029463in}{-0.029463in}}%
\pgfpathcurveto{\pgfqpoint{-0.021649in}{-0.037276in}}{\pgfqpoint{-0.011050in}{-0.041667in}}{\pgfqpoint{0.000000in}{-0.041667in}}%
\pgfpathclose%
\pgfusepath{stroke,fill}%
}%
\begin{pgfscope}%
\pgfsys@transformshift{10.090016in}{4.485562in}%
\pgfsys@useobject{currentmarker}{}%
\end{pgfscope}%
\end{pgfscope}%
\begin{pgfscope}%
\definecolor{textcolor}{rgb}{0.000000,0.000000,0.000000}%
\pgfsetstrokecolor{textcolor}%
\pgfsetfillcolor{textcolor}%
\pgftext[x=10.340016in,y=4.449104in,left,base]{\color{textcolor}\sffamily\fontsize{10.000000}{12.000000}\selectfont ai2thor}%
\end{pgfscope}%
\begin{pgfscope}%
\pgfsetbuttcap%
\pgfsetroundjoin%
\definecolor{currentfill}{rgb}{1.000000,0.623529,0.607843}%
\pgfsetfillcolor{currentfill}%
\pgfsetlinewidth{1.003750pt}%
\definecolor{currentstroke}{rgb}{1.000000,0.623529,0.607843}%
\pgfsetstrokecolor{currentstroke}%
\pgfsetdash{}{0pt}%
\pgfsys@defobject{currentmarker}{\pgfqpoint{-0.041667in}{-0.041667in}}{\pgfqpoint{0.041667in}{0.041667in}}{%
\pgfpathmoveto{\pgfqpoint{0.000000in}{-0.041667in}}%
\pgfpathcurveto{\pgfqpoint{0.011050in}{-0.041667in}}{\pgfqpoint{0.021649in}{-0.037276in}}{\pgfqpoint{0.029463in}{-0.029463in}}%
\pgfpathcurveto{\pgfqpoint{0.037276in}{-0.021649in}}{\pgfqpoint{0.041667in}{-0.011050in}}{\pgfqpoint{0.041667in}{0.000000in}}%
\pgfpathcurveto{\pgfqpoint{0.041667in}{0.011050in}}{\pgfqpoint{0.037276in}{0.021649in}}{\pgfqpoint{0.029463in}{0.029463in}}%
\pgfpathcurveto{\pgfqpoint{0.021649in}{0.037276in}}{\pgfqpoint{0.011050in}{0.041667in}}{\pgfqpoint{0.000000in}{0.041667in}}%
\pgfpathcurveto{\pgfqpoint{-0.011050in}{0.041667in}}{\pgfqpoint{-0.021649in}{0.037276in}}{\pgfqpoint{-0.029463in}{0.029463in}}%
\pgfpathcurveto{\pgfqpoint{-0.037276in}{0.021649in}}{\pgfqpoint{-0.041667in}{0.011050in}}{\pgfqpoint{-0.041667in}{0.000000in}}%
\pgfpathcurveto{\pgfqpoint{-0.041667in}{-0.011050in}}{\pgfqpoint{-0.037276in}{-0.021649in}}{\pgfqpoint{-0.029463in}{-0.029463in}}%
\pgfpathcurveto{\pgfqpoint{-0.021649in}{-0.037276in}}{\pgfqpoint{-0.011050in}{-0.041667in}}{\pgfqpoint{0.000000in}{-0.041667in}}%
\pgfpathclose%
\pgfusepath{stroke,fill}%
}%
\begin{pgfscope}%
\pgfsys@transformshift{10.090016in}{4.281705in}%
\pgfsys@useobject{currentmarker}{}%
\end{pgfscope}%
\end{pgfscope}%
\begin{pgfscope}%
\definecolor{textcolor}{rgb}{0.000000,0.000000,0.000000}%
\pgfsetstrokecolor{textcolor}%
\pgfsetfillcolor{textcolor}%
\pgftext[x=10.340016in,y=4.245247in,left,base]{\color{textcolor}\sffamily\fontsize{10.000000}{12.000000}\selectfont blenderproc}%
\end{pgfscope}%
\begin{pgfscope}%
\pgfsetbuttcap%
\pgfsetroundjoin%
\definecolor{currentfill}{rgb}{0.815686,0.733333,1.000000}%
\pgfsetfillcolor{currentfill}%
\pgfsetlinewidth{1.003750pt}%
\definecolor{currentstroke}{rgb}{0.815686,0.733333,1.000000}%
\pgfsetstrokecolor{currentstroke}%
\pgfsetdash{}{0pt}%
\pgfsys@defobject{currentmarker}{\pgfqpoint{-0.041667in}{-0.041667in}}{\pgfqpoint{0.041667in}{0.041667in}}{%
\pgfpathmoveto{\pgfqpoint{0.000000in}{-0.041667in}}%
\pgfpathcurveto{\pgfqpoint{0.011050in}{-0.041667in}}{\pgfqpoint{0.021649in}{-0.037276in}}{\pgfqpoint{0.029463in}{-0.029463in}}%
\pgfpathcurveto{\pgfqpoint{0.037276in}{-0.021649in}}{\pgfqpoint{0.041667in}{-0.011050in}}{\pgfqpoint{0.041667in}{0.000000in}}%
\pgfpathcurveto{\pgfqpoint{0.041667in}{0.011050in}}{\pgfqpoint{0.037276in}{0.021649in}}{\pgfqpoint{0.029463in}{0.029463in}}%
\pgfpathcurveto{\pgfqpoint{0.021649in}{0.037276in}}{\pgfqpoint{0.011050in}{0.041667in}}{\pgfqpoint{0.000000in}{0.041667in}}%
\pgfpathcurveto{\pgfqpoint{-0.011050in}{0.041667in}}{\pgfqpoint{-0.021649in}{0.037276in}}{\pgfqpoint{-0.029463in}{0.029463in}}%
\pgfpathcurveto{\pgfqpoint{-0.037276in}{0.021649in}}{\pgfqpoint{-0.041667in}{0.011050in}}{\pgfqpoint{-0.041667in}{0.000000in}}%
\pgfpathcurveto{\pgfqpoint{-0.041667in}{-0.011050in}}{\pgfqpoint{-0.037276in}{-0.021649in}}{\pgfqpoint{-0.029463in}{-0.029463in}}%
\pgfpathcurveto{\pgfqpoint{-0.021649in}{-0.037276in}}{\pgfqpoint{-0.011050in}{-0.041667in}}{\pgfqpoint{0.000000in}{-0.041667in}}%
\pgfpathclose%
\pgfusepath{stroke,fill}%
}%
\begin{pgfscope}%
\pgfsys@transformshift{10.090016in}{4.077848in}%
\pgfsys@useobject{currentmarker}{}%
\end{pgfscope}%
\end{pgfscope}%
\begin{pgfscope}%
\definecolor{textcolor}{rgb}{0.000000,0.000000,0.000000}%
\pgfsetstrokecolor{textcolor}%
\pgfsetfillcolor{textcolor}%
\pgftext[x=10.340016in,y=4.041390in,left,base]{\color{textcolor}\sffamily\fontsize{10.000000}{12.000000}\selectfont hypersim}%
\end{pgfscope}%
\begin{pgfscope}%
\pgfsetbuttcap%
\pgfsetroundjoin%
\definecolor{currentfill}{rgb}{0.870588,0.733333,0.607843}%
\pgfsetfillcolor{currentfill}%
\pgfsetlinewidth{1.003750pt}%
\definecolor{currentstroke}{rgb}{0.870588,0.733333,0.607843}%
\pgfsetstrokecolor{currentstroke}%
\pgfsetdash{}{0pt}%
\pgfsys@defobject{currentmarker}{\pgfqpoint{-0.041667in}{-0.041667in}}{\pgfqpoint{0.041667in}{0.041667in}}{%
\pgfpathmoveto{\pgfqpoint{0.000000in}{-0.041667in}}%
\pgfpathcurveto{\pgfqpoint{0.011050in}{-0.041667in}}{\pgfqpoint{0.021649in}{-0.037276in}}{\pgfqpoint{0.029463in}{-0.029463in}}%
\pgfpathcurveto{\pgfqpoint{0.037276in}{-0.021649in}}{\pgfqpoint{0.041667in}{-0.011050in}}{\pgfqpoint{0.041667in}{0.000000in}}%
\pgfpathcurveto{\pgfqpoint{0.041667in}{0.011050in}}{\pgfqpoint{0.037276in}{0.021649in}}{\pgfqpoint{0.029463in}{0.029463in}}%
\pgfpathcurveto{\pgfqpoint{0.021649in}{0.037276in}}{\pgfqpoint{0.011050in}{0.041667in}}{\pgfqpoint{0.000000in}{0.041667in}}%
\pgfpathcurveto{\pgfqpoint{-0.011050in}{0.041667in}}{\pgfqpoint{-0.021649in}{0.037276in}}{\pgfqpoint{-0.029463in}{0.029463in}}%
\pgfpathcurveto{\pgfqpoint{-0.037276in}{0.021649in}}{\pgfqpoint{-0.041667in}{0.011050in}}{\pgfqpoint{-0.041667in}{0.000000in}}%
\pgfpathcurveto{\pgfqpoint{-0.041667in}{-0.011050in}}{\pgfqpoint{-0.037276in}{-0.021649in}}{\pgfqpoint{-0.029463in}{-0.029463in}}%
\pgfpathcurveto{\pgfqpoint{-0.021649in}{-0.037276in}}{\pgfqpoint{-0.011050in}{-0.041667in}}{\pgfqpoint{0.000000in}{-0.041667in}}%
\pgfpathclose%
\pgfusepath{stroke,fill}%
}%
\begin{pgfscope}%
\pgfsys@transformshift{10.090016in}{3.873991in}%
\pgfsys@useobject{currentmarker}{}%
\end{pgfscope}%
\end{pgfscope}%
\begin{pgfscope}%
\definecolor{textcolor}{rgb}{0.000000,0.000000,0.000000}%
\pgfsetstrokecolor{textcolor}%
\pgfsetfillcolor{textcolor}%
\pgftext[x=10.340016in,y=3.837533in,left,base]{\color{textcolor}\sffamily\fontsize{10.000000}{12.000000}\selectfont 3dfront}%
\end{pgfscope}%
\begin{pgfscope}%
\pgfsetbuttcap%
\pgfsetroundjoin%
\definecolor{currentfill}{rgb}{0.980392,0.690196,0.894118}%
\pgfsetfillcolor{currentfill}%
\pgfsetlinewidth{1.003750pt}%
\definecolor{currentstroke}{rgb}{0.980392,0.690196,0.894118}%
\pgfsetstrokecolor{currentstroke}%
\pgfsetdash{}{0pt}%
\pgfsys@defobject{currentmarker}{\pgfqpoint{-0.041667in}{-0.041667in}}{\pgfqpoint{0.041667in}{0.041667in}}{%
\pgfpathmoveto{\pgfqpoint{0.000000in}{-0.041667in}}%
\pgfpathcurveto{\pgfqpoint{0.011050in}{-0.041667in}}{\pgfqpoint{0.021649in}{-0.037276in}}{\pgfqpoint{0.029463in}{-0.029463in}}%
\pgfpathcurveto{\pgfqpoint{0.037276in}{-0.021649in}}{\pgfqpoint{0.041667in}{-0.011050in}}{\pgfqpoint{0.041667in}{0.000000in}}%
\pgfpathcurveto{\pgfqpoint{0.041667in}{0.011050in}}{\pgfqpoint{0.037276in}{0.021649in}}{\pgfqpoint{0.029463in}{0.029463in}}%
\pgfpathcurveto{\pgfqpoint{0.021649in}{0.037276in}}{\pgfqpoint{0.011050in}{0.041667in}}{\pgfqpoint{0.000000in}{0.041667in}}%
\pgfpathcurveto{\pgfqpoint{-0.011050in}{0.041667in}}{\pgfqpoint{-0.021649in}{0.037276in}}{\pgfqpoint{-0.029463in}{0.029463in}}%
\pgfpathcurveto{\pgfqpoint{-0.037276in}{0.021649in}}{\pgfqpoint{-0.041667in}{0.011050in}}{\pgfqpoint{-0.041667in}{0.000000in}}%
\pgfpathcurveto{\pgfqpoint{-0.041667in}{-0.011050in}}{\pgfqpoint{-0.037276in}{-0.021649in}}{\pgfqpoint{-0.029463in}{-0.029463in}}%
\pgfpathcurveto{\pgfqpoint{-0.021649in}{-0.037276in}}{\pgfqpoint{-0.011050in}{-0.041667in}}{\pgfqpoint{0.000000in}{-0.041667in}}%
\pgfpathclose%
\pgfusepath{stroke,fill}%
}%
\begin{pgfscope}%
\pgfsys@transformshift{10.090016in}{3.670134in}%
\pgfsys@useobject{currentmarker}{}%
\end{pgfscope}%
\end{pgfscope}%
\begin{pgfscope}%
\definecolor{textcolor}{rgb}{0.000000,0.000000,0.000000}%
\pgfsetstrokecolor{textcolor}%
\pgfsetfillcolor{textcolor}%
\pgftext[x=10.340016in,y=3.633675in,left,base]{\color{textcolor}\sffamily\fontsize{10.000000}{12.000000}\selectfont s2r-3dfree}%
\end{pgfscope}%
\begin{pgfscope}%
\pgfsetbuttcap%
\pgfsetroundjoin%
\definecolor{currentfill}{rgb}{0.721569,0.521569,0.039216}%
\pgfsetfillcolor{currentfill}%
\pgfsetlinewidth{1.003750pt}%
\definecolor{currentstroke}{rgb}{0.721569,0.521569,0.039216}%
\pgfsetstrokecolor{currentstroke}%
\pgfsetdash{}{0pt}%
\pgfsys@defobject{currentmarker}{\pgfqpoint{-0.041667in}{-0.041667in}}{\pgfqpoint{0.041667in}{0.041667in}}{%
\pgfpathmoveto{\pgfqpoint{0.000000in}{-0.041667in}}%
\pgfpathcurveto{\pgfqpoint{0.011050in}{-0.041667in}}{\pgfqpoint{0.021649in}{-0.037276in}}{\pgfqpoint{0.029463in}{-0.029463in}}%
\pgfpathcurveto{\pgfqpoint{0.037276in}{-0.021649in}}{\pgfqpoint{0.041667in}{-0.011050in}}{\pgfqpoint{0.041667in}{0.000000in}}%
\pgfpathcurveto{\pgfqpoint{0.041667in}{0.011050in}}{\pgfqpoint{0.037276in}{0.021649in}}{\pgfqpoint{0.029463in}{0.029463in}}%
\pgfpathcurveto{\pgfqpoint{0.021649in}{0.037276in}}{\pgfqpoint{0.011050in}{0.041667in}}{\pgfqpoint{0.000000in}{0.041667in}}%
\pgfpathcurveto{\pgfqpoint{-0.011050in}{0.041667in}}{\pgfqpoint{-0.021649in}{0.037276in}}{\pgfqpoint{-0.029463in}{0.029463in}}%
\pgfpathcurveto{\pgfqpoint{-0.037276in}{0.021649in}}{\pgfqpoint{-0.041667in}{0.011050in}}{\pgfqpoint{-0.041667in}{0.000000in}}%
\pgfpathcurveto{\pgfqpoint{-0.041667in}{-0.011050in}}{\pgfqpoint{-0.037276in}{-0.021649in}}{\pgfqpoint{-0.029463in}{-0.029463in}}%
\pgfpathcurveto{\pgfqpoint{-0.021649in}{-0.037276in}}{\pgfqpoint{-0.011050in}{-0.041667in}}{\pgfqpoint{0.000000in}{-0.041667in}}%
\pgfpathclose%
\pgfusepath{stroke,fill}%
}%
\begin{pgfscope}%
\pgfsys@transformshift{10.090016in}{3.466276in}%
\pgfsys@useobject{currentmarker}{}%
\end{pgfscope}%
\end{pgfscope}%
\begin{pgfscope}%
\definecolor{textcolor}{rgb}{0.000000,0.000000,0.000000}%
\pgfsetstrokecolor{textcolor}%
\pgfsetfillcolor{textcolor}%
\pgftext[x=10.340016in,y=3.429818in,left,base]{\color{textcolor}\sffamily\fontsize{10.000000}{12.000000}\selectfont pix3d}%
\end{pgfscope}%
\end{pgfpicture}%
\makeatother%
\endgroup%
}
    \caption[\gls{tsne} for Photorealistic Synthetic datasets.]{\gls{tsne} visualization for images from various photo-realistic synthetic dataset. Pix3D and \gls{free} are highlighted with bolder colors.
    Both these datasets have wide spread in the embedding space.}
    \label{fig:photorealistic tsne}
\end{figure}

\begin{figure}[!ht]
    \centering
    \resizebox{0.49\linewidth}{5cm}{%% Creator: Matplotlib, PGF backend
%%
%% To include the figure in your LaTeX document, write
%%   \input{<filename>.pgf}
%%
%% Make sure the required packages are loaded in your preamble
%%   \usepackage{pgf}
%%
%% Figures using additional raster images can only be included by \input if
%% they are in the same directory as the main LaTeX file. For loading figures
%% from other directories you can use the `import` package
%%   \usepackage{import}
%%
%% and then include the figures with
%%   \import{<path to file>}{<filename>.pgf}
%%
%% Matplotlib used the following preamble
%%   \usepackage{fontspec}
%%   \setmainfont{DejaVuSerif.ttf}[Path=\detokenize{/Users/apple/opt/anaconda3/envs/kaolin/lib/python3.7/site-packages/matplotlib/mpl-data/fonts/ttf/}]
%%   \setsansfont{DejaVuSans.ttf}[Path=\detokenize{/Users/apple/opt/anaconda3/envs/kaolin/lib/python3.7/site-packages/matplotlib/mpl-data/fonts/ttf/}]
%%   \setmonofont{DejaVuSansMono.ttf}[Path=\detokenize{/Users/apple/opt/anaconda3/envs/kaolin/lib/python3.7/site-packages/matplotlib/mpl-data/fonts/ttf/}]
%%
\begingroup%
\makeatletter%
\begin{pgfpicture}%
\pgfpathrectangle{\pgfpointorigin}{\pgfqpoint{11.229688in}{8.341596in}}%
\pgfusepath{use as bounding box, clip}%
\begin{pgfscope}%
\pgfsetbuttcap%
\pgfsetmiterjoin%
\definecolor{currentfill}{rgb}{1.000000,1.000000,1.000000}%
\pgfsetfillcolor{currentfill}%
\pgfsetlinewidth{0.000000pt}%
\definecolor{currentstroke}{rgb}{1.000000,1.000000,1.000000}%
\pgfsetstrokecolor{currentstroke}%
\pgfsetdash{}{0pt}%
\pgfpathmoveto{\pgfqpoint{0.000000in}{0.000000in}}%
\pgfpathlineto{\pgfqpoint{11.229688in}{0.000000in}}%
\pgfpathlineto{\pgfqpoint{11.229688in}{8.341596in}}%
\pgfpathlineto{\pgfqpoint{0.000000in}{8.341596in}}%
\pgfpathclose%
\pgfusepath{fill}%
\end{pgfscope}%
\begin{pgfscope}%
\pgfsetbuttcap%
\pgfsetmiterjoin%
\definecolor{currentfill}{rgb}{1.000000,1.000000,1.000000}%
\pgfsetfillcolor{currentfill}%
\pgfsetlinewidth{0.000000pt}%
\definecolor{currentstroke}{rgb}{0.000000,0.000000,0.000000}%
\pgfsetstrokecolor{currentstroke}%
\pgfsetstrokeopacity{0.000000}%
\pgfsetdash{}{0pt}%
\pgfpathmoveto{\pgfqpoint{0.481978in}{0.331635in}}%
\pgfpathlineto{\pgfqpoint{9.781978in}{0.331635in}}%
\pgfpathlineto{\pgfqpoint{9.781978in}{8.031635in}}%
\pgfpathlineto{\pgfqpoint{0.481978in}{8.031635in}}%
\pgfpathclose%
\pgfusepath{fill}%
\end{pgfscope}%
\begin{pgfscope}%
\pgfpathrectangle{\pgfqpoint{0.481978in}{0.331635in}}{\pgfqpoint{9.300000in}{7.700000in}}%
\pgfusepath{clip}%
\pgfsetbuttcap%
\pgfsetroundjoin%
\definecolor{currentfill}{rgb}{0.631373,0.788235,0.956863}%
\pgfsetfillcolor{currentfill}%
\pgfsetlinewidth{0.481800pt}%
\definecolor{currentstroke}{rgb}{1.000000,1.000000,1.000000}%
\pgfsetstrokecolor{currentstroke}%
\pgfsetdash{}{0pt}%
\pgfpathmoveto{\pgfqpoint{5.968461in}{3.642287in}}%
\pgfpathcurveto{\pgfqpoint{5.979511in}{3.642287in}}{\pgfqpoint{5.990110in}{3.646677in}}{\pgfqpoint{5.997924in}{3.654491in}}%
\pgfpathcurveto{\pgfqpoint{6.005737in}{3.662304in}}{\pgfqpoint{6.010127in}{3.672903in}}{\pgfqpoint{6.010127in}{3.683953in}}%
\pgfpathcurveto{\pgfqpoint{6.010127in}{3.695004in}}{\pgfqpoint{6.005737in}{3.705603in}}{\pgfqpoint{5.997924in}{3.713416in}}%
\pgfpathcurveto{\pgfqpoint{5.990110in}{3.721230in}}{\pgfqpoint{5.979511in}{3.725620in}}{\pgfqpoint{5.968461in}{3.725620in}}%
\pgfpathcurveto{\pgfqpoint{5.957411in}{3.725620in}}{\pgfqpoint{5.946812in}{3.721230in}}{\pgfqpoint{5.938998in}{3.713416in}}%
\pgfpathcurveto{\pgfqpoint{5.931184in}{3.705603in}}{\pgfqpoint{5.926794in}{3.695004in}}{\pgfqpoint{5.926794in}{3.683953in}}%
\pgfpathcurveto{\pgfqpoint{5.926794in}{3.672903in}}{\pgfqpoint{5.931184in}{3.662304in}}{\pgfqpoint{5.938998in}{3.654491in}}%
\pgfpathcurveto{\pgfqpoint{5.946812in}{3.646677in}}{\pgfqpoint{5.957411in}{3.642287in}}{\pgfqpoint{5.968461in}{3.642287in}}%
\pgfpathclose%
\pgfusepath{stroke,fill}%
\end{pgfscope}%
\begin{pgfscope}%
\pgfpathrectangle{\pgfqpoint{0.481978in}{0.331635in}}{\pgfqpoint{9.300000in}{7.700000in}}%
\pgfusepath{clip}%
\pgfsetbuttcap%
\pgfsetroundjoin%
\definecolor{currentfill}{rgb}{0.631373,0.788235,0.956863}%
\pgfsetfillcolor{currentfill}%
\pgfsetlinewidth{0.481800pt}%
\definecolor{currentstroke}{rgb}{1.000000,1.000000,1.000000}%
\pgfsetstrokecolor{currentstroke}%
\pgfsetdash{}{0pt}%
\pgfpathmoveto{\pgfqpoint{6.716760in}{0.639968in}}%
\pgfpathcurveto{\pgfqpoint{6.727810in}{0.639968in}}{\pgfqpoint{6.738409in}{0.644359in}}{\pgfqpoint{6.746222in}{0.652172in}}%
\pgfpathcurveto{\pgfqpoint{6.754036in}{0.659986in}}{\pgfqpoint{6.758426in}{0.670585in}}{\pgfqpoint{6.758426in}{0.681635in}}%
\pgfpathcurveto{\pgfqpoint{6.758426in}{0.692685in}}{\pgfqpoint{6.754036in}{0.703284in}}{\pgfqpoint{6.746222in}{0.711098in}}%
\pgfpathcurveto{\pgfqpoint{6.738409in}{0.718911in}}{\pgfqpoint{6.727810in}{0.723302in}}{\pgfqpoint{6.716760in}{0.723302in}}%
\pgfpathcurveto{\pgfqpoint{6.705710in}{0.723302in}}{\pgfqpoint{6.695110in}{0.718911in}}{\pgfqpoint{6.687297in}{0.711098in}}%
\pgfpathcurveto{\pgfqpoint{6.679483in}{0.703284in}}{\pgfqpoint{6.675093in}{0.692685in}}{\pgfqpoint{6.675093in}{0.681635in}}%
\pgfpathcurveto{\pgfqpoint{6.675093in}{0.670585in}}{\pgfqpoint{6.679483in}{0.659986in}}{\pgfqpoint{6.687297in}{0.652172in}}%
\pgfpathcurveto{\pgfqpoint{6.695110in}{0.644359in}}{\pgfqpoint{6.705710in}{0.639968in}}{\pgfqpoint{6.716760in}{0.639968in}}%
\pgfpathclose%
\pgfusepath{stroke,fill}%
\end{pgfscope}%
\begin{pgfscope}%
\pgfpathrectangle{\pgfqpoint{0.481978in}{0.331635in}}{\pgfqpoint{9.300000in}{7.700000in}}%
\pgfusepath{clip}%
\pgfsetbuttcap%
\pgfsetroundjoin%
\definecolor{currentfill}{rgb}{0.631373,0.788235,0.956863}%
\pgfsetfillcolor{currentfill}%
\pgfsetlinewidth{0.481800pt}%
\definecolor{currentstroke}{rgb}{1.000000,1.000000,1.000000}%
\pgfsetstrokecolor{currentstroke}%
\pgfsetdash{}{0pt}%
\pgfpathmoveto{\pgfqpoint{4.030414in}{3.987941in}}%
\pgfpathcurveto{\pgfqpoint{4.041464in}{3.987941in}}{\pgfqpoint{4.052063in}{3.992331in}}{\pgfqpoint{4.059877in}{4.000145in}}%
\pgfpathcurveto{\pgfqpoint{4.067690in}{4.007958in}}{\pgfqpoint{4.072080in}{4.018557in}}{\pgfqpoint{4.072080in}{4.029607in}}%
\pgfpathcurveto{\pgfqpoint{4.072080in}{4.040658in}}{\pgfqpoint{4.067690in}{4.051257in}}{\pgfqpoint{4.059877in}{4.059070in}}%
\pgfpathcurveto{\pgfqpoint{4.052063in}{4.066884in}}{\pgfqpoint{4.041464in}{4.071274in}}{\pgfqpoint{4.030414in}{4.071274in}}%
\pgfpathcurveto{\pgfqpoint{4.019364in}{4.071274in}}{\pgfqpoint{4.008765in}{4.066884in}}{\pgfqpoint{4.000951in}{4.059070in}}%
\pgfpathcurveto{\pgfqpoint{3.993137in}{4.051257in}}{\pgfqpoint{3.988747in}{4.040658in}}{\pgfqpoint{3.988747in}{4.029607in}}%
\pgfpathcurveto{\pgfqpoint{3.988747in}{4.018557in}}{\pgfqpoint{3.993137in}{4.007958in}}{\pgfqpoint{4.000951in}{4.000145in}}%
\pgfpathcurveto{\pgfqpoint{4.008765in}{3.992331in}}{\pgfqpoint{4.019364in}{3.987941in}}{\pgfqpoint{4.030414in}{3.987941in}}%
\pgfpathclose%
\pgfusepath{stroke,fill}%
\end{pgfscope}%
\begin{pgfscope}%
\pgfpathrectangle{\pgfqpoint{0.481978in}{0.331635in}}{\pgfqpoint{9.300000in}{7.700000in}}%
\pgfusepath{clip}%
\pgfsetbuttcap%
\pgfsetroundjoin%
\definecolor{currentfill}{rgb}{0.631373,0.788235,0.956863}%
\pgfsetfillcolor{currentfill}%
\pgfsetlinewidth{0.481800pt}%
\definecolor{currentstroke}{rgb}{1.000000,1.000000,1.000000}%
\pgfsetstrokecolor{currentstroke}%
\pgfsetdash{}{0pt}%
\pgfpathmoveto{\pgfqpoint{3.807365in}{2.300877in}}%
\pgfpathcurveto{\pgfqpoint{3.818415in}{2.300877in}}{\pgfqpoint{3.829014in}{2.305268in}}{\pgfqpoint{3.836828in}{2.313081in}}%
\pgfpathcurveto{\pgfqpoint{3.844642in}{2.320895in}}{\pgfqpoint{3.849032in}{2.331494in}}{\pgfqpoint{3.849032in}{2.342544in}}%
\pgfpathcurveto{\pgfqpoint{3.849032in}{2.353594in}}{\pgfqpoint{3.844642in}{2.364193in}}{\pgfqpoint{3.836828in}{2.372007in}}%
\pgfpathcurveto{\pgfqpoint{3.829014in}{2.379820in}}{\pgfqpoint{3.818415in}{2.384211in}}{\pgfqpoint{3.807365in}{2.384211in}}%
\pgfpathcurveto{\pgfqpoint{3.796315in}{2.384211in}}{\pgfqpoint{3.785716in}{2.379820in}}{\pgfqpoint{3.777903in}{2.372007in}}%
\pgfpathcurveto{\pgfqpoint{3.770089in}{2.364193in}}{\pgfqpoint{3.765699in}{2.353594in}}{\pgfqpoint{3.765699in}{2.342544in}}%
\pgfpathcurveto{\pgfqpoint{3.765699in}{2.331494in}}{\pgfqpoint{3.770089in}{2.320895in}}{\pgfqpoint{3.777903in}{2.313081in}}%
\pgfpathcurveto{\pgfqpoint{3.785716in}{2.305268in}}{\pgfqpoint{3.796315in}{2.300877in}}{\pgfqpoint{3.807365in}{2.300877in}}%
\pgfpathclose%
\pgfusepath{stroke,fill}%
\end{pgfscope}%
\begin{pgfscope}%
\pgfpathrectangle{\pgfqpoint{0.481978in}{0.331635in}}{\pgfqpoint{9.300000in}{7.700000in}}%
\pgfusepath{clip}%
\pgfsetbuttcap%
\pgfsetroundjoin%
\definecolor{currentfill}{rgb}{0.631373,0.788235,0.956863}%
\pgfsetfillcolor{currentfill}%
\pgfsetlinewidth{0.481800pt}%
\definecolor{currentstroke}{rgb}{1.000000,1.000000,1.000000}%
\pgfsetstrokecolor{currentstroke}%
\pgfsetdash{}{0pt}%
\pgfpathmoveto{\pgfqpoint{2.350199in}{2.814087in}}%
\pgfpathcurveto{\pgfqpoint{2.361249in}{2.814087in}}{\pgfqpoint{2.371848in}{2.818477in}}{\pgfqpoint{2.379662in}{2.826291in}}%
\pgfpathcurveto{\pgfqpoint{2.387475in}{2.834104in}}{\pgfqpoint{2.391866in}{2.844703in}}{\pgfqpoint{2.391866in}{2.855754in}}%
\pgfpathcurveto{\pgfqpoint{2.391866in}{2.866804in}}{\pgfqpoint{2.387475in}{2.877403in}}{\pgfqpoint{2.379662in}{2.885216in}}%
\pgfpathcurveto{\pgfqpoint{2.371848in}{2.893030in}}{\pgfqpoint{2.361249in}{2.897420in}}{\pgfqpoint{2.350199in}{2.897420in}}%
\pgfpathcurveto{\pgfqpoint{2.339149in}{2.897420in}}{\pgfqpoint{2.328550in}{2.893030in}}{\pgfqpoint{2.320736in}{2.885216in}}%
\pgfpathcurveto{\pgfqpoint{2.312923in}{2.877403in}}{\pgfqpoint{2.308532in}{2.866804in}}{\pgfqpoint{2.308532in}{2.855754in}}%
\pgfpathcurveto{\pgfqpoint{2.308532in}{2.844703in}}{\pgfqpoint{2.312923in}{2.834104in}}{\pgfqpoint{2.320736in}{2.826291in}}%
\pgfpathcurveto{\pgfqpoint{2.328550in}{2.818477in}}{\pgfqpoint{2.339149in}{2.814087in}}{\pgfqpoint{2.350199in}{2.814087in}}%
\pgfpathclose%
\pgfusepath{stroke,fill}%
\end{pgfscope}%
\begin{pgfscope}%
\pgfpathrectangle{\pgfqpoint{0.481978in}{0.331635in}}{\pgfqpoint{9.300000in}{7.700000in}}%
\pgfusepath{clip}%
\pgfsetbuttcap%
\pgfsetroundjoin%
\definecolor{currentfill}{rgb}{0.631373,0.788235,0.956863}%
\pgfsetfillcolor{currentfill}%
\pgfsetlinewidth{0.481800pt}%
\definecolor{currentstroke}{rgb}{1.000000,1.000000,1.000000}%
\pgfsetstrokecolor{currentstroke}%
\pgfsetdash{}{0pt}%
\pgfpathmoveto{\pgfqpoint{5.589222in}{5.930958in}}%
\pgfpathcurveto{\pgfqpoint{5.600272in}{5.930958in}}{\pgfqpoint{5.610871in}{5.935348in}}{\pgfqpoint{5.618685in}{5.943162in}}%
\pgfpathcurveto{\pgfqpoint{5.626498in}{5.950975in}}{\pgfqpoint{5.630888in}{5.961574in}}{\pgfqpoint{5.630888in}{5.972624in}}%
\pgfpathcurveto{\pgfqpoint{5.630888in}{5.983674in}}{\pgfqpoint{5.626498in}{5.994274in}}{\pgfqpoint{5.618685in}{6.002087in}}%
\pgfpathcurveto{\pgfqpoint{5.610871in}{6.009901in}}{\pgfqpoint{5.600272in}{6.014291in}}{\pgfqpoint{5.589222in}{6.014291in}}%
\pgfpathcurveto{\pgfqpoint{5.578172in}{6.014291in}}{\pgfqpoint{5.567573in}{6.009901in}}{\pgfqpoint{5.559759in}{6.002087in}}%
\pgfpathcurveto{\pgfqpoint{5.551945in}{5.994274in}}{\pgfqpoint{5.547555in}{5.983674in}}{\pgfqpoint{5.547555in}{5.972624in}}%
\pgfpathcurveto{\pgfqpoint{5.547555in}{5.961574in}}{\pgfqpoint{5.551945in}{5.950975in}}{\pgfqpoint{5.559759in}{5.943162in}}%
\pgfpathcurveto{\pgfqpoint{5.567573in}{5.935348in}}{\pgfqpoint{5.578172in}{5.930958in}}{\pgfqpoint{5.589222in}{5.930958in}}%
\pgfpathclose%
\pgfusepath{stroke,fill}%
\end{pgfscope}%
\begin{pgfscope}%
\pgfpathrectangle{\pgfqpoint{0.481978in}{0.331635in}}{\pgfqpoint{9.300000in}{7.700000in}}%
\pgfusepath{clip}%
\pgfsetbuttcap%
\pgfsetroundjoin%
\definecolor{currentfill}{rgb}{0.631373,0.788235,0.956863}%
\pgfsetfillcolor{currentfill}%
\pgfsetlinewidth{0.481800pt}%
\definecolor{currentstroke}{rgb}{1.000000,1.000000,1.000000}%
\pgfsetstrokecolor{currentstroke}%
\pgfsetdash{}{0pt}%
\pgfpathmoveto{\pgfqpoint{9.359251in}{4.404130in}}%
\pgfpathcurveto{\pgfqpoint{9.370301in}{4.404130in}}{\pgfqpoint{9.380900in}{4.408521in}}{\pgfqpoint{9.388713in}{4.416334in}}%
\pgfpathcurveto{\pgfqpoint{9.396527in}{4.424148in}}{\pgfqpoint{9.400917in}{4.434747in}}{\pgfqpoint{9.400917in}{4.445797in}}%
\pgfpathcurveto{\pgfqpoint{9.400917in}{4.456847in}}{\pgfqpoint{9.396527in}{4.467446in}}{\pgfqpoint{9.388713in}{4.475260in}}%
\pgfpathcurveto{\pgfqpoint{9.380900in}{4.483074in}}{\pgfqpoint{9.370301in}{4.487464in}}{\pgfqpoint{9.359251in}{4.487464in}}%
\pgfpathcurveto{\pgfqpoint{9.348201in}{4.487464in}}{\pgfqpoint{9.337601in}{4.483074in}}{\pgfqpoint{9.329788in}{4.475260in}}%
\pgfpathcurveto{\pgfqpoint{9.321974in}{4.467446in}}{\pgfqpoint{9.317584in}{4.456847in}}{\pgfqpoint{9.317584in}{4.445797in}}%
\pgfpathcurveto{\pgfqpoint{9.317584in}{4.434747in}}{\pgfqpoint{9.321974in}{4.424148in}}{\pgfqpoint{9.329788in}{4.416334in}}%
\pgfpathcurveto{\pgfqpoint{9.337601in}{4.408521in}}{\pgfqpoint{9.348201in}{4.404130in}}{\pgfqpoint{9.359251in}{4.404130in}}%
\pgfpathclose%
\pgfusepath{stroke,fill}%
\end{pgfscope}%
\begin{pgfscope}%
\pgfpathrectangle{\pgfqpoint{0.481978in}{0.331635in}}{\pgfqpoint{9.300000in}{7.700000in}}%
\pgfusepath{clip}%
\pgfsetbuttcap%
\pgfsetroundjoin%
\definecolor{currentfill}{rgb}{0.631373,0.788235,0.956863}%
\pgfsetfillcolor{currentfill}%
\pgfsetlinewidth{0.481800pt}%
\definecolor{currentstroke}{rgb}{1.000000,1.000000,1.000000}%
\pgfsetstrokecolor{currentstroke}%
\pgfsetdash{}{0pt}%
\pgfpathmoveto{\pgfqpoint{5.412406in}{5.045738in}}%
\pgfpathcurveto{\pgfqpoint{5.423456in}{5.045738in}}{\pgfqpoint{5.434055in}{5.050128in}}{\pgfqpoint{5.441869in}{5.057942in}}%
\pgfpathcurveto{\pgfqpoint{5.449682in}{5.065756in}}{\pgfqpoint{5.454073in}{5.076355in}}{\pgfqpoint{5.454073in}{5.087405in}}%
\pgfpathcurveto{\pgfqpoint{5.454073in}{5.098455in}}{\pgfqpoint{5.449682in}{5.109054in}}{\pgfqpoint{5.441869in}{5.116868in}}%
\pgfpathcurveto{\pgfqpoint{5.434055in}{5.124681in}}{\pgfqpoint{5.423456in}{5.129071in}}{\pgfqpoint{5.412406in}{5.129071in}}%
\pgfpathcurveto{\pgfqpoint{5.401356in}{5.129071in}}{\pgfqpoint{5.390757in}{5.124681in}}{\pgfqpoint{5.382943in}{5.116868in}}%
\pgfpathcurveto{\pgfqpoint{5.375130in}{5.109054in}}{\pgfqpoint{5.370739in}{5.098455in}}{\pgfqpoint{5.370739in}{5.087405in}}%
\pgfpathcurveto{\pgfqpoint{5.370739in}{5.076355in}}{\pgfqpoint{5.375130in}{5.065756in}}{\pgfqpoint{5.382943in}{5.057942in}}%
\pgfpathcurveto{\pgfqpoint{5.390757in}{5.050128in}}{\pgfqpoint{5.401356in}{5.045738in}}{\pgfqpoint{5.412406in}{5.045738in}}%
\pgfpathclose%
\pgfusepath{stroke,fill}%
\end{pgfscope}%
\begin{pgfscope}%
\pgfpathrectangle{\pgfqpoint{0.481978in}{0.331635in}}{\pgfqpoint{9.300000in}{7.700000in}}%
\pgfusepath{clip}%
\pgfsetbuttcap%
\pgfsetroundjoin%
\definecolor{currentfill}{rgb}{0.631373,0.788235,0.956863}%
\pgfsetfillcolor{currentfill}%
\pgfsetlinewidth{0.481800pt}%
\definecolor{currentstroke}{rgb}{1.000000,1.000000,1.000000}%
\pgfsetstrokecolor{currentstroke}%
\pgfsetdash{}{0pt}%
\pgfpathmoveto{\pgfqpoint{3.390760in}{6.231096in}}%
\pgfpathcurveto{\pgfqpoint{3.401810in}{6.231096in}}{\pgfqpoint{3.412409in}{6.235486in}}{\pgfqpoint{3.420223in}{6.243300in}}%
\pgfpathcurveto{\pgfqpoint{3.428037in}{6.251113in}}{\pgfqpoint{3.432427in}{6.261712in}}{\pgfqpoint{3.432427in}{6.272762in}}%
\pgfpathcurveto{\pgfqpoint{3.432427in}{6.283813in}}{\pgfqpoint{3.428037in}{6.294412in}}{\pgfqpoint{3.420223in}{6.302225in}}%
\pgfpathcurveto{\pgfqpoint{3.412409in}{6.310039in}}{\pgfqpoint{3.401810in}{6.314429in}}{\pgfqpoint{3.390760in}{6.314429in}}%
\pgfpathcurveto{\pgfqpoint{3.379710in}{6.314429in}}{\pgfqpoint{3.369111in}{6.310039in}}{\pgfqpoint{3.361297in}{6.302225in}}%
\pgfpathcurveto{\pgfqpoint{3.353484in}{6.294412in}}{\pgfqpoint{3.349094in}{6.283813in}}{\pgfqpoint{3.349094in}{6.272762in}}%
\pgfpathcurveto{\pgfqpoint{3.349094in}{6.261712in}}{\pgfqpoint{3.353484in}{6.251113in}}{\pgfqpoint{3.361297in}{6.243300in}}%
\pgfpathcurveto{\pgfqpoint{3.369111in}{6.235486in}}{\pgfqpoint{3.379710in}{6.231096in}}{\pgfqpoint{3.390760in}{6.231096in}}%
\pgfpathclose%
\pgfusepath{stroke,fill}%
\end{pgfscope}%
\begin{pgfscope}%
\pgfpathrectangle{\pgfqpoint{0.481978in}{0.331635in}}{\pgfqpoint{9.300000in}{7.700000in}}%
\pgfusepath{clip}%
\pgfsetbuttcap%
\pgfsetroundjoin%
\definecolor{currentfill}{rgb}{0.631373,0.788235,0.956863}%
\pgfsetfillcolor{currentfill}%
\pgfsetlinewidth{0.481800pt}%
\definecolor{currentstroke}{rgb}{1.000000,1.000000,1.000000}%
\pgfsetstrokecolor{currentstroke}%
\pgfsetdash{}{0pt}%
\pgfpathmoveto{\pgfqpoint{7.065739in}{4.651877in}}%
\pgfpathcurveto{\pgfqpoint{7.076789in}{4.651877in}}{\pgfqpoint{7.087388in}{4.656267in}}{\pgfqpoint{7.095202in}{4.664081in}}%
\pgfpathcurveto{\pgfqpoint{7.103015in}{4.671894in}}{\pgfqpoint{7.107405in}{4.682493in}}{\pgfqpoint{7.107405in}{4.693543in}}%
\pgfpathcurveto{\pgfqpoint{7.107405in}{4.704594in}}{\pgfqpoint{7.103015in}{4.715193in}}{\pgfqpoint{7.095202in}{4.723006in}}%
\pgfpathcurveto{\pgfqpoint{7.087388in}{4.730820in}}{\pgfqpoint{7.076789in}{4.735210in}}{\pgfqpoint{7.065739in}{4.735210in}}%
\pgfpathcurveto{\pgfqpoint{7.054689in}{4.735210in}}{\pgfqpoint{7.044090in}{4.730820in}}{\pgfqpoint{7.036276in}{4.723006in}}%
\pgfpathcurveto{\pgfqpoint{7.028462in}{4.715193in}}{\pgfqpoint{7.024072in}{4.704594in}}{\pgfqpoint{7.024072in}{4.693543in}}%
\pgfpathcurveto{\pgfqpoint{7.024072in}{4.682493in}}{\pgfqpoint{7.028462in}{4.671894in}}{\pgfqpoint{7.036276in}{4.664081in}}%
\pgfpathcurveto{\pgfqpoint{7.044090in}{4.656267in}}{\pgfqpoint{7.054689in}{4.651877in}}{\pgfqpoint{7.065739in}{4.651877in}}%
\pgfpathclose%
\pgfusepath{stroke,fill}%
\end{pgfscope}%
\begin{pgfscope}%
\pgfpathrectangle{\pgfqpoint{0.481978in}{0.331635in}}{\pgfqpoint{9.300000in}{7.700000in}}%
\pgfusepath{clip}%
\pgfsetbuttcap%
\pgfsetroundjoin%
\definecolor{currentfill}{rgb}{0.631373,0.788235,0.956863}%
\pgfsetfillcolor{currentfill}%
\pgfsetlinewidth{0.481800pt}%
\definecolor{currentstroke}{rgb}{1.000000,1.000000,1.000000}%
\pgfsetstrokecolor{currentstroke}%
\pgfsetdash{}{0pt}%
\pgfpathmoveto{\pgfqpoint{6.796596in}{5.826111in}}%
\pgfpathcurveto{\pgfqpoint{6.807646in}{5.826111in}}{\pgfqpoint{6.818245in}{5.830501in}}{\pgfqpoint{6.826058in}{5.838315in}}%
\pgfpathcurveto{\pgfqpoint{6.833872in}{5.846128in}}{\pgfqpoint{6.838262in}{5.856728in}}{\pgfqpoint{6.838262in}{5.867778in}}%
\pgfpathcurveto{\pgfqpoint{6.838262in}{5.878828in}}{\pgfqpoint{6.833872in}{5.889427in}}{\pgfqpoint{6.826058in}{5.897240in}}%
\pgfpathcurveto{\pgfqpoint{6.818245in}{5.905054in}}{\pgfqpoint{6.807646in}{5.909444in}}{\pgfqpoint{6.796596in}{5.909444in}}%
\pgfpathcurveto{\pgfqpoint{6.785545in}{5.909444in}}{\pgfqpoint{6.774946in}{5.905054in}}{\pgfqpoint{6.767133in}{5.897240in}}%
\pgfpathcurveto{\pgfqpoint{6.759319in}{5.889427in}}{\pgfqpoint{6.754929in}{5.878828in}}{\pgfqpoint{6.754929in}{5.867778in}}%
\pgfpathcurveto{\pgfqpoint{6.754929in}{5.856728in}}{\pgfqpoint{6.759319in}{5.846128in}}{\pgfqpoint{6.767133in}{5.838315in}}%
\pgfpathcurveto{\pgfqpoint{6.774946in}{5.830501in}}{\pgfqpoint{6.785545in}{5.826111in}}{\pgfqpoint{6.796596in}{5.826111in}}%
\pgfpathclose%
\pgfusepath{stroke,fill}%
\end{pgfscope}%
\begin{pgfscope}%
\pgfpathrectangle{\pgfqpoint{0.481978in}{0.331635in}}{\pgfqpoint{9.300000in}{7.700000in}}%
\pgfusepath{clip}%
\pgfsetbuttcap%
\pgfsetroundjoin%
\definecolor{currentfill}{rgb}{0.631373,0.788235,0.956863}%
\pgfsetfillcolor{currentfill}%
\pgfsetlinewidth{0.481800pt}%
\definecolor{currentstroke}{rgb}{1.000000,1.000000,1.000000}%
\pgfsetstrokecolor{currentstroke}%
\pgfsetdash{}{0pt}%
\pgfpathmoveto{\pgfqpoint{6.170106in}{5.271216in}}%
\pgfpathcurveto{\pgfqpoint{6.181156in}{5.271216in}}{\pgfqpoint{6.191755in}{5.275606in}}{\pgfqpoint{6.199569in}{5.283419in}}%
\pgfpathcurveto{\pgfqpoint{6.207382in}{5.291233in}}{\pgfqpoint{6.211772in}{5.301832in}}{\pgfqpoint{6.211772in}{5.312882in}}%
\pgfpathcurveto{\pgfqpoint{6.211772in}{5.323932in}}{\pgfqpoint{6.207382in}{5.334531in}}{\pgfqpoint{6.199569in}{5.342345in}}%
\pgfpathcurveto{\pgfqpoint{6.191755in}{5.350159in}}{\pgfqpoint{6.181156in}{5.354549in}}{\pgfqpoint{6.170106in}{5.354549in}}%
\pgfpathcurveto{\pgfqpoint{6.159056in}{5.354549in}}{\pgfqpoint{6.148457in}{5.350159in}}{\pgfqpoint{6.140643in}{5.342345in}}%
\pgfpathcurveto{\pgfqpoint{6.132829in}{5.334531in}}{\pgfqpoint{6.128439in}{5.323932in}}{\pgfqpoint{6.128439in}{5.312882in}}%
\pgfpathcurveto{\pgfqpoint{6.128439in}{5.301832in}}{\pgfqpoint{6.132829in}{5.291233in}}{\pgfqpoint{6.140643in}{5.283419in}}%
\pgfpathcurveto{\pgfqpoint{6.148457in}{5.275606in}}{\pgfqpoint{6.159056in}{5.271216in}}{\pgfqpoint{6.170106in}{5.271216in}}%
\pgfpathclose%
\pgfusepath{stroke,fill}%
\end{pgfscope}%
\begin{pgfscope}%
\pgfpathrectangle{\pgfqpoint{0.481978in}{0.331635in}}{\pgfqpoint{9.300000in}{7.700000in}}%
\pgfusepath{clip}%
\pgfsetbuttcap%
\pgfsetroundjoin%
\definecolor{currentfill}{rgb}{0.631373,0.788235,0.956863}%
\pgfsetfillcolor{currentfill}%
\pgfsetlinewidth{0.481800pt}%
\definecolor{currentstroke}{rgb}{1.000000,1.000000,1.000000}%
\pgfsetstrokecolor{currentstroke}%
\pgfsetdash{}{0pt}%
\pgfpathmoveto{\pgfqpoint{7.510480in}{3.033169in}}%
\pgfpathcurveto{\pgfqpoint{7.521530in}{3.033169in}}{\pgfqpoint{7.532130in}{3.037559in}}{\pgfqpoint{7.539943in}{3.045372in}}%
\pgfpathcurveto{\pgfqpoint{7.547757in}{3.053186in}}{\pgfqpoint{7.552147in}{3.063785in}}{\pgfqpoint{7.552147in}{3.074835in}}%
\pgfpathcurveto{\pgfqpoint{7.552147in}{3.085885in}}{\pgfqpoint{7.547757in}{3.096484in}}{\pgfqpoint{7.539943in}{3.104298in}}%
\pgfpathcurveto{\pgfqpoint{7.532130in}{3.112112in}}{\pgfqpoint{7.521530in}{3.116502in}}{\pgfqpoint{7.510480in}{3.116502in}}%
\pgfpathcurveto{\pgfqpoint{7.499430in}{3.116502in}}{\pgfqpoint{7.488831in}{3.112112in}}{\pgfqpoint{7.481018in}{3.104298in}}%
\pgfpathcurveto{\pgfqpoint{7.473204in}{3.096484in}}{\pgfqpoint{7.468814in}{3.085885in}}{\pgfqpoint{7.468814in}{3.074835in}}%
\pgfpathcurveto{\pgfqpoint{7.468814in}{3.063785in}}{\pgfqpoint{7.473204in}{3.053186in}}{\pgfqpoint{7.481018in}{3.045372in}}%
\pgfpathcurveto{\pgfqpoint{7.488831in}{3.037559in}}{\pgfqpoint{7.499430in}{3.033169in}}{\pgfqpoint{7.510480in}{3.033169in}}%
\pgfpathclose%
\pgfusepath{stroke,fill}%
\end{pgfscope}%
\begin{pgfscope}%
\pgfpathrectangle{\pgfqpoint{0.481978in}{0.331635in}}{\pgfqpoint{9.300000in}{7.700000in}}%
\pgfusepath{clip}%
\pgfsetbuttcap%
\pgfsetroundjoin%
\definecolor{currentfill}{rgb}{0.631373,0.788235,0.956863}%
\pgfsetfillcolor{currentfill}%
\pgfsetlinewidth{0.481800pt}%
\definecolor{currentstroke}{rgb}{1.000000,1.000000,1.000000}%
\pgfsetstrokecolor{currentstroke}%
\pgfsetdash{}{0pt}%
\pgfpathmoveto{\pgfqpoint{6.540018in}{1.774081in}}%
\pgfpathcurveto{\pgfqpoint{6.551068in}{1.774081in}}{\pgfqpoint{6.561667in}{1.778471in}}{\pgfqpoint{6.569480in}{1.786285in}}%
\pgfpathcurveto{\pgfqpoint{6.577294in}{1.794098in}}{\pgfqpoint{6.581684in}{1.804697in}}{\pgfqpoint{6.581684in}{1.815747in}}%
\pgfpathcurveto{\pgfqpoint{6.581684in}{1.826797in}}{\pgfqpoint{6.577294in}{1.837396in}}{\pgfqpoint{6.569480in}{1.845210in}}%
\pgfpathcurveto{\pgfqpoint{6.561667in}{1.853024in}}{\pgfqpoint{6.551068in}{1.857414in}}{\pgfqpoint{6.540018in}{1.857414in}}%
\pgfpathcurveto{\pgfqpoint{6.528968in}{1.857414in}}{\pgfqpoint{6.518369in}{1.853024in}}{\pgfqpoint{6.510555in}{1.845210in}}%
\pgfpathcurveto{\pgfqpoint{6.502741in}{1.837396in}}{\pgfqpoint{6.498351in}{1.826797in}}{\pgfqpoint{6.498351in}{1.815747in}}%
\pgfpathcurveto{\pgfqpoint{6.498351in}{1.804697in}}{\pgfqpoint{6.502741in}{1.794098in}}{\pgfqpoint{6.510555in}{1.786285in}}%
\pgfpathcurveto{\pgfqpoint{6.518369in}{1.778471in}}{\pgfqpoint{6.528968in}{1.774081in}}{\pgfqpoint{6.540018in}{1.774081in}}%
\pgfpathclose%
\pgfusepath{stroke,fill}%
\end{pgfscope}%
\begin{pgfscope}%
\pgfpathrectangle{\pgfqpoint{0.481978in}{0.331635in}}{\pgfqpoint{9.300000in}{7.700000in}}%
\pgfusepath{clip}%
\pgfsetbuttcap%
\pgfsetroundjoin%
\definecolor{currentfill}{rgb}{0.631373,0.788235,0.956863}%
\pgfsetfillcolor{currentfill}%
\pgfsetlinewidth{0.481800pt}%
\definecolor{currentstroke}{rgb}{1.000000,1.000000,1.000000}%
\pgfsetstrokecolor{currentstroke}%
\pgfsetdash{}{0pt}%
\pgfpathmoveto{\pgfqpoint{6.818751in}{2.766920in}}%
\pgfpathcurveto{\pgfqpoint{6.829801in}{2.766920in}}{\pgfqpoint{6.840400in}{2.771310in}}{\pgfqpoint{6.848214in}{2.779124in}}%
\pgfpathcurveto{\pgfqpoint{6.856027in}{2.786938in}}{\pgfqpoint{6.860418in}{2.797537in}}{\pgfqpoint{6.860418in}{2.808587in}}%
\pgfpathcurveto{\pgfqpoint{6.860418in}{2.819637in}}{\pgfqpoint{6.856027in}{2.830236in}}{\pgfqpoint{6.848214in}{2.838050in}}%
\pgfpathcurveto{\pgfqpoint{6.840400in}{2.845863in}}{\pgfqpoint{6.829801in}{2.850254in}}{\pgfqpoint{6.818751in}{2.850254in}}%
\pgfpathcurveto{\pgfqpoint{6.807701in}{2.850254in}}{\pgfqpoint{6.797102in}{2.845863in}}{\pgfqpoint{6.789288in}{2.838050in}}%
\pgfpathcurveto{\pgfqpoint{6.781474in}{2.830236in}}{\pgfqpoint{6.777084in}{2.819637in}}{\pgfqpoint{6.777084in}{2.808587in}}%
\pgfpathcurveto{\pgfqpoint{6.777084in}{2.797537in}}{\pgfqpoint{6.781474in}{2.786938in}}{\pgfqpoint{6.789288in}{2.779124in}}%
\pgfpathcurveto{\pgfqpoint{6.797102in}{2.771310in}}{\pgfqpoint{6.807701in}{2.766920in}}{\pgfqpoint{6.818751in}{2.766920in}}%
\pgfpathclose%
\pgfusepath{stroke,fill}%
\end{pgfscope}%
\begin{pgfscope}%
\pgfpathrectangle{\pgfqpoint{0.481978in}{0.331635in}}{\pgfqpoint{9.300000in}{7.700000in}}%
\pgfusepath{clip}%
\pgfsetbuttcap%
\pgfsetroundjoin%
\definecolor{currentfill}{rgb}{0.631373,0.788235,0.956863}%
\pgfsetfillcolor{currentfill}%
\pgfsetlinewidth{0.481800pt}%
\definecolor{currentstroke}{rgb}{1.000000,1.000000,1.000000}%
\pgfsetstrokecolor{currentstroke}%
\pgfsetdash{}{0pt}%
\pgfpathmoveto{\pgfqpoint{4.820436in}{6.196549in}}%
\pgfpathcurveto{\pgfqpoint{4.831487in}{6.196549in}}{\pgfqpoint{4.842086in}{6.200939in}}{\pgfqpoint{4.849899in}{6.208752in}}%
\pgfpathcurveto{\pgfqpoint{4.857713in}{6.216566in}}{\pgfqpoint{4.862103in}{6.227165in}}{\pgfqpoint{4.862103in}{6.238215in}}%
\pgfpathcurveto{\pgfqpoint{4.862103in}{6.249265in}}{\pgfqpoint{4.857713in}{6.259864in}}{\pgfqpoint{4.849899in}{6.267678in}}%
\pgfpathcurveto{\pgfqpoint{4.842086in}{6.275492in}}{\pgfqpoint{4.831487in}{6.279882in}}{\pgfqpoint{4.820436in}{6.279882in}}%
\pgfpathcurveto{\pgfqpoint{4.809386in}{6.279882in}}{\pgfqpoint{4.798787in}{6.275492in}}{\pgfqpoint{4.790974in}{6.267678in}}%
\pgfpathcurveto{\pgfqpoint{4.783160in}{6.259864in}}{\pgfqpoint{4.778770in}{6.249265in}}{\pgfqpoint{4.778770in}{6.238215in}}%
\pgfpathcurveto{\pgfqpoint{4.778770in}{6.227165in}}{\pgfqpoint{4.783160in}{6.216566in}}{\pgfqpoint{4.790974in}{6.208752in}}%
\pgfpathcurveto{\pgfqpoint{4.798787in}{6.200939in}}{\pgfqpoint{4.809386in}{6.196549in}}{\pgfqpoint{4.820436in}{6.196549in}}%
\pgfpathclose%
\pgfusepath{stroke,fill}%
\end{pgfscope}%
\begin{pgfscope}%
\pgfpathrectangle{\pgfqpoint{0.481978in}{0.331635in}}{\pgfqpoint{9.300000in}{7.700000in}}%
\pgfusepath{clip}%
\pgfsetbuttcap%
\pgfsetroundjoin%
\definecolor{currentfill}{rgb}{0.631373,0.788235,0.956863}%
\pgfsetfillcolor{currentfill}%
\pgfsetlinewidth{0.481800pt}%
\definecolor{currentstroke}{rgb}{1.000000,1.000000,1.000000}%
\pgfsetstrokecolor{currentstroke}%
\pgfsetdash{}{0pt}%
\pgfpathmoveto{\pgfqpoint{5.842861in}{4.427537in}}%
\pgfpathcurveto{\pgfqpoint{5.853911in}{4.427537in}}{\pgfqpoint{5.864510in}{4.431928in}}{\pgfqpoint{5.872324in}{4.439741in}}%
\pgfpathcurveto{\pgfqpoint{5.880137in}{4.447555in}}{\pgfqpoint{5.884528in}{4.458154in}}{\pgfqpoint{5.884528in}{4.469204in}}%
\pgfpathcurveto{\pgfqpoint{5.884528in}{4.480254in}}{\pgfqpoint{5.880137in}{4.490853in}}{\pgfqpoint{5.872324in}{4.498667in}}%
\pgfpathcurveto{\pgfqpoint{5.864510in}{4.506481in}}{\pgfqpoint{5.853911in}{4.510871in}}{\pgfqpoint{5.842861in}{4.510871in}}%
\pgfpathcurveto{\pgfqpoint{5.831811in}{4.510871in}}{\pgfqpoint{5.821212in}{4.506481in}}{\pgfqpoint{5.813398in}{4.498667in}}%
\pgfpathcurveto{\pgfqpoint{5.805585in}{4.490853in}}{\pgfqpoint{5.801194in}{4.480254in}}{\pgfqpoint{5.801194in}{4.469204in}}%
\pgfpathcurveto{\pgfqpoint{5.801194in}{4.458154in}}{\pgfqpoint{5.805585in}{4.447555in}}{\pgfqpoint{5.813398in}{4.439741in}}%
\pgfpathcurveto{\pgfqpoint{5.821212in}{4.431928in}}{\pgfqpoint{5.831811in}{4.427537in}}{\pgfqpoint{5.842861in}{4.427537in}}%
\pgfpathclose%
\pgfusepath{stroke,fill}%
\end{pgfscope}%
\begin{pgfscope}%
\pgfpathrectangle{\pgfqpoint{0.481978in}{0.331635in}}{\pgfqpoint{9.300000in}{7.700000in}}%
\pgfusepath{clip}%
\pgfsetbuttcap%
\pgfsetroundjoin%
\definecolor{currentfill}{rgb}{0.631373,0.788235,0.956863}%
\pgfsetfillcolor{currentfill}%
\pgfsetlinewidth{0.481800pt}%
\definecolor{currentstroke}{rgb}{1.000000,1.000000,1.000000}%
\pgfsetstrokecolor{currentstroke}%
\pgfsetdash{}{0pt}%
\pgfpathmoveto{\pgfqpoint{6.322821in}{6.783128in}}%
\pgfpathcurveto{\pgfqpoint{6.333871in}{6.783128in}}{\pgfqpoint{6.344470in}{6.787518in}}{\pgfqpoint{6.352284in}{6.795331in}}%
\pgfpathcurveto{\pgfqpoint{6.360097in}{6.803145in}}{\pgfqpoint{6.364487in}{6.813744in}}{\pgfqpoint{6.364487in}{6.824794in}}%
\pgfpathcurveto{\pgfqpoint{6.364487in}{6.835844in}}{\pgfqpoint{6.360097in}{6.846443in}}{\pgfqpoint{6.352284in}{6.854257in}}%
\pgfpathcurveto{\pgfqpoint{6.344470in}{6.862071in}}{\pgfqpoint{6.333871in}{6.866461in}}{\pgfqpoint{6.322821in}{6.866461in}}%
\pgfpathcurveto{\pgfqpoint{6.311771in}{6.866461in}}{\pgfqpoint{6.301172in}{6.862071in}}{\pgfqpoint{6.293358in}{6.854257in}}%
\pgfpathcurveto{\pgfqpoint{6.285544in}{6.846443in}}{\pgfqpoint{6.281154in}{6.835844in}}{\pgfqpoint{6.281154in}{6.824794in}}%
\pgfpathcurveto{\pgfqpoint{6.281154in}{6.813744in}}{\pgfqpoint{6.285544in}{6.803145in}}{\pgfqpoint{6.293358in}{6.795331in}}%
\pgfpathcurveto{\pgfqpoint{6.301172in}{6.787518in}}{\pgfqpoint{6.311771in}{6.783128in}}{\pgfqpoint{6.322821in}{6.783128in}}%
\pgfpathclose%
\pgfusepath{stroke,fill}%
\end{pgfscope}%
\begin{pgfscope}%
\pgfpathrectangle{\pgfqpoint{0.481978in}{0.331635in}}{\pgfqpoint{9.300000in}{7.700000in}}%
\pgfusepath{clip}%
\pgfsetbuttcap%
\pgfsetroundjoin%
\definecolor{currentfill}{rgb}{0.631373,0.788235,0.956863}%
\pgfsetfillcolor{currentfill}%
\pgfsetlinewidth{0.481800pt}%
\definecolor{currentstroke}{rgb}{1.000000,1.000000,1.000000}%
\pgfsetstrokecolor{currentstroke}%
\pgfsetdash{}{0pt}%
\pgfpathmoveto{\pgfqpoint{7.830146in}{0.979624in}}%
\pgfpathcurveto{\pgfqpoint{7.841196in}{0.979624in}}{\pgfqpoint{7.851795in}{0.984014in}}{\pgfqpoint{7.859608in}{0.991828in}}%
\pgfpathcurveto{\pgfqpoint{7.867422in}{0.999641in}}{\pgfqpoint{7.871812in}{1.010240in}}{\pgfqpoint{7.871812in}{1.021290in}}%
\pgfpathcurveto{\pgfqpoint{7.871812in}{1.032340in}}{\pgfqpoint{7.867422in}{1.042939in}}{\pgfqpoint{7.859608in}{1.050753in}}%
\pgfpathcurveto{\pgfqpoint{7.851795in}{1.058567in}}{\pgfqpoint{7.841196in}{1.062957in}}{\pgfqpoint{7.830146in}{1.062957in}}%
\pgfpathcurveto{\pgfqpoint{7.819095in}{1.062957in}}{\pgfqpoint{7.808496in}{1.058567in}}{\pgfqpoint{7.800683in}{1.050753in}}%
\pgfpathcurveto{\pgfqpoint{7.792869in}{1.042939in}}{\pgfqpoint{7.788479in}{1.032340in}}{\pgfqpoint{7.788479in}{1.021290in}}%
\pgfpathcurveto{\pgfqpoint{7.788479in}{1.010240in}}{\pgfqpoint{7.792869in}{0.999641in}}{\pgfqpoint{7.800683in}{0.991828in}}%
\pgfpathcurveto{\pgfqpoint{7.808496in}{0.984014in}}{\pgfqpoint{7.819095in}{0.979624in}}{\pgfqpoint{7.830146in}{0.979624in}}%
\pgfpathclose%
\pgfusepath{stroke,fill}%
\end{pgfscope}%
\begin{pgfscope}%
\pgfpathrectangle{\pgfqpoint{0.481978in}{0.331635in}}{\pgfqpoint{9.300000in}{7.700000in}}%
\pgfusepath{clip}%
\pgfsetbuttcap%
\pgfsetroundjoin%
\definecolor{currentfill}{rgb}{0.631373,0.788235,0.956863}%
\pgfsetfillcolor{currentfill}%
\pgfsetlinewidth{0.481800pt}%
\definecolor{currentstroke}{rgb}{1.000000,1.000000,1.000000}%
\pgfsetstrokecolor{currentstroke}%
\pgfsetdash{}{0pt}%
\pgfpathmoveto{\pgfqpoint{5.762941in}{2.425077in}}%
\pgfpathcurveto{\pgfqpoint{5.773991in}{2.425077in}}{\pgfqpoint{5.784590in}{2.429467in}}{\pgfqpoint{5.792404in}{2.437281in}}%
\pgfpathcurveto{\pgfqpoint{5.800217in}{2.445094in}}{\pgfqpoint{5.804608in}{2.455693in}}{\pgfqpoint{5.804608in}{2.466744in}}%
\pgfpathcurveto{\pgfqpoint{5.804608in}{2.477794in}}{\pgfqpoint{5.800217in}{2.488393in}}{\pgfqpoint{5.792404in}{2.496206in}}%
\pgfpathcurveto{\pgfqpoint{5.784590in}{2.504020in}}{\pgfqpoint{5.773991in}{2.508410in}}{\pgfqpoint{5.762941in}{2.508410in}}%
\pgfpathcurveto{\pgfqpoint{5.751891in}{2.508410in}}{\pgfqpoint{5.741292in}{2.504020in}}{\pgfqpoint{5.733478in}{2.496206in}}%
\pgfpathcurveto{\pgfqpoint{5.725665in}{2.488393in}}{\pgfqpoint{5.721274in}{2.477794in}}{\pgfqpoint{5.721274in}{2.466744in}}%
\pgfpathcurveto{\pgfqpoint{5.721274in}{2.455693in}}{\pgfqpoint{5.725665in}{2.445094in}}{\pgfqpoint{5.733478in}{2.437281in}}%
\pgfpathcurveto{\pgfqpoint{5.741292in}{2.429467in}}{\pgfqpoint{5.751891in}{2.425077in}}{\pgfqpoint{5.762941in}{2.425077in}}%
\pgfpathclose%
\pgfusepath{stroke,fill}%
\end{pgfscope}%
\begin{pgfscope}%
\pgfpathrectangle{\pgfqpoint{0.481978in}{0.331635in}}{\pgfqpoint{9.300000in}{7.700000in}}%
\pgfusepath{clip}%
\pgfsetbuttcap%
\pgfsetroundjoin%
\definecolor{currentfill}{rgb}{0.631373,0.788235,0.956863}%
\pgfsetfillcolor{currentfill}%
\pgfsetlinewidth{0.481800pt}%
\definecolor{currentstroke}{rgb}{1.000000,1.000000,1.000000}%
\pgfsetstrokecolor{currentstroke}%
\pgfsetdash{}{0pt}%
\pgfpathmoveto{\pgfqpoint{6.414379in}{4.456302in}}%
\pgfpathcurveto{\pgfqpoint{6.425429in}{4.456302in}}{\pgfqpoint{6.436028in}{4.460692in}}{\pgfqpoint{6.443842in}{4.468506in}}%
\pgfpathcurveto{\pgfqpoint{6.451656in}{4.476319in}}{\pgfqpoint{6.456046in}{4.486918in}}{\pgfqpoint{6.456046in}{4.497968in}}%
\pgfpathcurveto{\pgfqpoint{6.456046in}{4.509018in}}{\pgfqpoint{6.451656in}{4.519618in}}{\pgfqpoint{6.443842in}{4.527431in}}%
\pgfpathcurveto{\pgfqpoint{6.436028in}{4.535245in}}{\pgfqpoint{6.425429in}{4.539635in}}{\pgfqpoint{6.414379in}{4.539635in}}%
\pgfpathcurveto{\pgfqpoint{6.403329in}{4.539635in}}{\pgfqpoint{6.392730in}{4.535245in}}{\pgfqpoint{6.384916in}{4.527431in}}%
\pgfpathcurveto{\pgfqpoint{6.377103in}{4.519618in}}{\pgfqpoint{6.372713in}{4.509018in}}{\pgfqpoint{6.372713in}{4.497968in}}%
\pgfpathcurveto{\pgfqpoint{6.372713in}{4.486918in}}{\pgfqpoint{6.377103in}{4.476319in}}{\pgfqpoint{6.384916in}{4.468506in}}%
\pgfpathcurveto{\pgfqpoint{6.392730in}{4.460692in}}{\pgfqpoint{6.403329in}{4.456302in}}{\pgfqpoint{6.414379in}{4.456302in}}%
\pgfpathclose%
\pgfusepath{stroke,fill}%
\end{pgfscope}%
\begin{pgfscope}%
\pgfpathrectangle{\pgfqpoint{0.481978in}{0.331635in}}{\pgfqpoint{9.300000in}{7.700000in}}%
\pgfusepath{clip}%
\pgfsetbuttcap%
\pgfsetroundjoin%
\definecolor{currentfill}{rgb}{0.631373,0.788235,0.956863}%
\pgfsetfillcolor{currentfill}%
\pgfsetlinewidth{0.481800pt}%
\definecolor{currentstroke}{rgb}{1.000000,1.000000,1.000000}%
\pgfsetstrokecolor{currentstroke}%
\pgfsetdash{}{0pt}%
\pgfpathmoveto{\pgfqpoint{5.330843in}{3.109933in}}%
\pgfpathcurveto{\pgfqpoint{5.341893in}{3.109933in}}{\pgfqpoint{5.352492in}{3.114323in}}{\pgfqpoint{5.360306in}{3.122137in}}%
\pgfpathcurveto{\pgfqpoint{5.368120in}{3.129950in}}{\pgfqpoint{5.372510in}{3.140549in}}{\pgfqpoint{5.372510in}{3.151599in}}%
\pgfpathcurveto{\pgfqpoint{5.372510in}{3.162650in}}{\pgfqpoint{5.368120in}{3.173249in}}{\pgfqpoint{5.360306in}{3.181062in}}%
\pgfpathcurveto{\pgfqpoint{5.352492in}{3.188876in}}{\pgfqpoint{5.341893in}{3.193266in}}{\pgfqpoint{5.330843in}{3.193266in}}%
\pgfpathcurveto{\pgfqpoint{5.319793in}{3.193266in}}{\pgfqpoint{5.309194in}{3.188876in}}{\pgfqpoint{5.301380in}{3.181062in}}%
\pgfpathcurveto{\pgfqpoint{5.293567in}{3.173249in}}{\pgfqpoint{5.289177in}{3.162650in}}{\pgfqpoint{5.289177in}{3.151599in}}%
\pgfpathcurveto{\pgfqpoint{5.289177in}{3.140549in}}{\pgfqpoint{5.293567in}{3.129950in}}{\pgfqpoint{5.301380in}{3.122137in}}%
\pgfpathcurveto{\pgfqpoint{5.309194in}{3.114323in}}{\pgfqpoint{5.319793in}{3.109933in}}{\pgfqpoint{5.330843in}{3.109933in}}%
\pgfpathclose%
\pgfusepath{stroke,fill}%
\end{pgfscope}%
\begin{pgfscope}%
\pgfpathrectangle{\pgfqpoint{0.481978in}{0.331635in}}{\pgfqpoint{9.300000in}{7.700000in}}%
\pgfusepath{clip}%
\pgfsetbuttcap%
\pgfsetroundjoin%
\definecolor{currentfill}{rgb}{0.631373,0.788235,0.956863}%
\pgfsetfillcolor{currentfill}%
\pgfsetlinewidth{0.481800pt}%
\definecolor{currentstroke}{rgb}{1.000000,1.000000,1.000000}%
\pgfsetstrokecolor{currentstroke}%
\pgfsetdash{}{0pt}%
\pgfpathmoveto{\pgfqpoint{5.723444in}{1.091644in}}%
\pgfpathcurveto{\pgfqpoint{5.734494in}{1.091644in}}{\pgfqpoint{5.745093in}{1.096034in}}{\pgfqpoint{5.752907in}{1.103848in}}%
\pgfpathcurveto{\pgfqpoint{5.760720in}{1.111661in}}{\pgfqpoint{5.765110in}{1.122260in}}{\pgfqpoint{5.765110in}{1.133310in}}%
\pgfpathcurveto{\pgfqpoint{5.765110in}{1.144361in}}{\pgfqpoint{5.760720in}{1.154960in}}{\pgfqpoint{5.752907in}{1.162773in}}%
\pgfpathcurveto{\pgfqpoint{5.745093in}{1.170587in}}{\pgfqpoint{5.734494in}{1.174977in}}{\pgfqpoint{5.723444in}{1.174977in}}%
\pgfpathcurveto{\pgfqpoint{5.712394in}{1.174977in}}{\pgfqpoint{5.701795in}{1.170587in}}{\pgfqpoint{5.693981in}{1.162773in}}%
\pgfpathcurveto{\pgfqpoint{5.686167in}{1.154960in}}{\pgfqpoint{5.681777in}{1.144361in}}{\pgfqpoint{5.681777in}{1.133310in}}%
\pgfpathcurveto{\pgfqpoint{5.681777in}{1.122260in}}{\pgfqpoint{5.686167in}{1.111661in}}{\pgfqpoint{5.693981in}{1.103848in}}%
\pgfpathcurveto{\pgfqpoint{5.701795in}{1.096034in}}{\pgfqpoint{5.712394in}{1.091644in}}{\pgfqpoint{5.723444in}{1.091644in}}%
\pgfpathclose%
\pgfusepath{stroke,fill}%
\end{pgfscope}%
\begin{pgfscope}%
\pgfpathrectangle{\pgfqpoint{0.481978in}{0.331635in}}{\pgfqpoint{9.300000in}{7.700000in}}%
\pgfusepath{clip}%
\pgfsetbuttcap%
\pgfsetroundjoin%
\definecolor{currentfill}{rgb}{0.631373,0.788235,0.956863}%
\pgfsetfillcolor{currentfill}%
\pgfsetlinewidth{0.481800pt}%
\definecolor{currentstroke}{rgb}{1.000000,1.000000,1.000000}%
\pgfsetstrokecolor{currentstroke}%
\pgfsetdash{}{0pt}%
\pgfpathmoveto{\pgfqpoint{3.540419in}{3.324689in}}%
\pgfpathcurveto{\pgfqpoint{3.551469in}{3.324689in}}{\pgfqpoint{3.562068in}{3.329079in}}{\pgfqpoint{3.569881in}{3.336893in}}%
\pgfpathcurveto{\pgfqpoint{3.577695in}{3.344707in}}{\pgfqpoint{3.582085in}{3.355306in}}{\pgfqpoint{3.582085in}{3.366356in}}%
\pgfpathcurveto{\pgfqpoint{3.582085in}{3.377406in}}{\pgfqpoint{3.577695in}{3.388005in}}{\pgfqpoint{3.569881in}{3.395819in}}%
\pgfpathcurveto{\pgfqpoint{3.562068in}{3.403632in}}{\pgfqpoint{3.551469in}{3.408022in}}{\pgfqpoint{3.540419in}{3.408022in}}%
\pgfpathcurveto{\pgfqpoint{3.529368in}{3.408022in}}{\pgfqpoint{3.518769in}{3.403632in}}{\pgfqpoint{3.510956in}{3.395819in}}%
\pgfpathcurveto{\pgfqpoint{3.503142in}{3.388005in}}{\pgfqpoint{3.498752in}{3.377406in}}{\pgfqpoint{3.498752in}{3.366356in}}%
\pgfpathcurveto{\pgfqpoint{3.498752in}{3.355306in}}{\pgfqpoint{3.503142in}{3.344707in}}{\pgfqpoint{3.510956in}{3.336893in}}%
\pgfpathcurveto{\pgfqpoint{3.518769in}{3.329079in}}{\pgfqpoint{3.529368in}{3.324689in}}{\pgfqpoint{3.540419in}{3.324689in}}%
\pgfpathclose%
\pgfusepath{stroke,fill}%
\end{pgfscope}%
\begin{pgfscope}%
\pgfpathrectangle{\pgfqpoint{0.481978in}{0.331635in}}{\pgfqpoint{9.300000in}{7.700000in}}%
\pgfusepath{clip}%
\pgfsetbuttcap%
\pgfsetroundjoin%
\definecolor{currentfill}{rgb}{0.631373,0.788235,0.956863}%
\pgfsetfillcolor{currentfill}%
\pgfsetlinewidth{0.481800pt}%
\definecolor{currentstroke}{rgb}{1.000000,1.000000,1.000000}%
\pgfsetstrokecolor{currentstroke}%
\pgfsetdash{}{0pt}%
\pgfpathmoveto{\pgfqpoint{7.439774in}{5.499083in}}%
\pgfpathcurveto{\pgfqpoint{7.450825in}{5.499083in}}{\pgfqpoint{7.461424in}{5.503473in}}{\pgfqpoint{7.469237in}{5.511287in}}%
\pgfpathcurveto{\pgfqpoint{7.477051in}{5.519101in}}{\pgfqpoint{7.481441in}{5.529700in}}{\pgfqpoint{7.481441in}{5.540750in}}%
\pgfpathcurveto{\pgfqpoint{7.481441in}{5.551800in}}{\pgfqpoint{7.477051in}{5.562399in}}{\pgfqpoint{7.469237in}{5.570213in}}%
\pgfpathcurveto{\pgfqpoint{7.461424in}{5.578026in}}{\pgfqpoint{7.450825in}{5.582417in}}{\pgfqpoint{7.439774in}{5.582417in}}%
\pgfpathcurveto{\pgfqpoint{7.428724in}{5.582417in}}{\pgfqpoint{7.418125in}{5.578026in}}{\pgfqpoint{7.410312in}{5.570213in}}%
\pgfpathcurveto{\pgfqpoint{7.402498in}{5.562399in}}{\pgfqpoint{7.398108in}{5.551800in}}{\pgfqpoint{7.398108in}{5.540750in}}%
\pgfpathcurveto{\pgfqpoint{7.398108in}{5.529700in}}{\pgfqpoint{7.402498in}{5.519101in}}{\pgfqpoint{7.410312in}{5.511287in}}%
\pgfpathcurveto{\pgfqpoint{7.418125in}{5.503473in}}{\pgfqpoint{7.428724in}{5.499083in}}{\pgfqpoint{7.439774in}{5.499083in}}%
\pgfpathclose%
\pgfusepath{stroke,fill}%
\end{pgfscope}%
\begin{pgfscope}%
\pgfpathrectangle{\pgfqpoint{0.481978in}{0.331635in}}{\pgfqpoint{9.300000in}{7.700000in}}%
\pgfusepath{clip}%
\pgfsetbuttcap%
\pgfsetroundjoin%
\definecolor{currentfill}{rgb}{0.631373,0.788235,0.956863}%
\pgfsetfillcolor{currentfill}%
\pgfsetlinewidth{0.481800pt}%
\definecolor{currentstroke}{rgb}{1.000000,1.000000,1.000000}%
\pgfsetstrokecolor{currentstroke}%
\pgfsetdash{}{0pt}%
\pgfpathmoveto{\pgfqpoint{8.272345in}{2.122362in}}%
\pgfpathcurveto{\pgfqpoint{8.283395in}{2.122362in}}{\pgfqpoint{8.293994in}{2.126752in}}{\pgfqpoint{8.301808in}{2.134566in}}%
\pgfpathcurveto{\pgfqpoint{8.309621in}{2.142380in}}{\pgfqpoint{8.314011in}{2.152979in}}{\pgfqpoint{8.314011in}{2.164029in}}%
\pgfpathcurveto{\pgfqpoint{8.314011in}{2.175079in}}{\pgfqpoint{8.309621in}{2.185678in}}{\pgfqpoint{8.301808in}{2.193492in}}%
\pgfpathcurveto{\pgfqpoint{8.293994in}{2.201305in}}{\pgfqpoint{8.283395in}{2.205696in}}{\pgfqpoint{8.272345in}{2.205696in}}%
\pgfpathcurveto{\pgfqpoint{8.261295in}{2.205696in}}{\pgfqpoint{8.250696in}{2.201305in}}{\pgfqpoint{8.242882in}{2.193492in}}%
\pgfpathcurveto{\pgfqpoint{8.235068in}{2.185678in}}{\pgfqpoint{8.230678in}{2.175079in}}{\pgfqpoint{8.230678in}{2.164029in}}%
\pgfpathcurveto{\pgfqpoint{8.230678in}{2.152979in}}{\pgfqpoint{8.235068in}{2.142380in}}{\pgfqpoint{8.242882in}{2.134566in}}%
\pgfpathcurveto{\pgfqpoint{8.250696in}{2.126752in}}{\pgfqpoint{8.261295in}{2.122362in}}{\pgfqpoint{8.272345in}{2.122362in}}%
\pgfpathclose%
\pgfusepath{stroke,fill}%
\end{pgfscope}%
\begin{pgfscope}%
\pgfpathrectangle{\pgfqpoint{0.481978in}{0.331635in}}{\pgfqpoint{9.300000in}{7.700000in}}%
\pgfusepath{clip}%
\pgfsetbuttcap%
\pgfsetroundjoin%
\definecolor{currentfill}{rgb}{0.631373,0.788235,0.956863}%
\pgfsetfillcolor{currentfill}%
\pgfsetlinewidth{0.481800pt}%
\definecolor{currentstroke}{rgb}{1.000000,1.000000,1.000000}%
\pgfsetstrokecolor{currentstroke}%
\pgfsetdash{}{0pt}%
\pgfpathmoveto{\pgfqpoint{7.470382in}{1.755818in}}%
\pgfpathcurveto{\pgfqpoint{7.481433in}{1.755818in}}{\pgfqpoint{7.492032in}{1.760209in}}{\pgfqpoint{7.499845in}{1.768022in}}%
\pgfpathcurveto{\pgfqpoint{7.507659in}{1.775836in}}{\pgfqpoint{7.512049in}{1.786435in}}{\pgfqpoint{7.512049in}{1.797485in}}%
\pgfpathcurveto{\pgfqpoint{7.512049in}{1.808535in}}{\pgfqpoint{7.507659in}{1.819134in}}{\pgfqpoint{7.499845in}{1.826948in}}%
\pgfpathcurveto{\pgfqpoint{7.492032in}{1.834761in}}{\pgfqpoint{7.481433in}{1.839152in}}{\pgfqpoint{7.470382in}{1.839152in}}%
\pgfpathcurveto{\pgfqpoint{7.459332in}{1.839152in}}{\pgfqpoint{7.448733in}{1.834761in}}{\pgfqpoint{7.440920in}{1.826948in}}%
\pgfpathcurveto{\pgfqpoint{7.433106in}{1.819134in}}{\pgfqpoint{7.428716in}{1.808535in}}{\pgfqpoint{7.428716in}{1.797485in}}%
\pgfpathcurveto{\pgfqpoint{7.428716in}{1.786435in}}{\pgfqpoint{7.433106in}{1.775836in}}{\pgfqpoint{7.440920in}{1.768022in}}%
\pgfpathcurveto{\pgfqpoint{7.448733in}{1.760209in}}{\pgfqpoint{7.459332in}{1.755818in}}{\pgfqpoint{7.470382in}{1.755818in}}%
\pgfpathclose%
\pgfusepath{stroke,fill}%
\end{pgfscope}%
\begin{pgfscope}%
\pgfpathrectangle{\pgfqpoint{0.481978in}{0.331635in}}{\pgfqpoint{9.300000in}{7.700000in}}%
\pgfusepath{clip}%
\pgfsetbuttcap%
\pgfsetroundjoin%
\definecolor{currentfill}{rgb}{0.631373,0.788235,0.956863}%
\pgfsetfillcolor{currentfill}%
\pgfsetlinewidth{0.481800pt}%
\definecolor{currentstroke}{rgb}{1.000000,1.000000,1.000000}%
\pgfsetstrokecolor{currentstroke}%
\pgfsetdash{}{0pt}%
\pgfpathmoveto{\pgfqpoint{7.403075in}{6.729377in}}%
\pgfpathcurveto{\pgfqpoint{7.414125in}{6.729377in}}{\pgfqpoint{7.424724in}{6.733767in}}{\pgfqpoint{7.432538in}{6.741580in}}%
\pgfpathcurveto{\pgfqpoint{7.440351in}{6.749394in}}{\pgfqpoint{7.444741in}{6.759993in}}{\pgfqpoint{7.444741in}{6.771043in}}%
\pgfpathcurveto{\pgfqpoint{7.444741in}{6.782093in}}{\pgfqpoint{7.440351in}{6.792692in}}{\pgfqpoint{7.432538in}{6.800506in}}%
\pgfpathcurveto{\pgfqpoint{7.424724in}{6.808320in}}{\pgfqpoint{7.414125in}{6.812710in}}{\pgfqpoint{7.403075in}{6.812710in}}%
\pgfpathcurveto{\pgfqpoint{7.392025in}{6.812710in}}{\pgfqpoint{7.381426in}{6.808320in}}{\pgfqpoint{7.373612in}{6.800506in}}%
\pgfpathcurveto{\pgfqpoint{7.365798in}{6.792692in}}{\pgfqpoint{7.361408in}{6.782093in}}{\pgfqpoint{7.361408in}{6.771043in}}%
\pgfpathcurveto{\pgfqpoint{7.361408in}{6.759993in}}{\pgfqpoint{7.365798in}{6.749394in}}{\pgfqpoint{7.373612in}{6.741580in}}%
\pgfpathcurveto{\pgfqpoint{7.381426in}{6.733767in}}{\pgfqpoint{7.392025in}{6.729377in}}{\pgfqpoint{7.403075in}{6.729377in}}%
\pgfpathclose%
\pgfusepath{stroke,fill}%
\end{pgfscope}%
\begin{pgfscope}%
\pgfpathrectangle{\pgfqpoint{0.481978in}{0.331635in}}{\pgfqpoint{9.300000in}{7.700000in}}%
\pgfusepath{clip}%
\pgfsetbuttcap%
\pgfsetroundjoin%
\definecolor{currentfill}{rgb}{1.000000,0.705882,0.509804}%
\pgfsetfillcolor{currentfill}%
\pgfsetlinewidth{0.481800pt}%
\definecolor{currentstroke}{rgb}{1.000000,1.000000,1.000000}%
\pgfsetstrokecolor{currentstroke}%
\pgfsetdash{}{0pt}%
\pgfpathmoveto{\pgfqpoint{4.737094in}{4.504340in}}%
\pgfpathcurveto{\pgfqpoint{4.748144in}{4.504340in}}{\pgfqpoint{4.758743in}{4.508730in}}{\pgfqpoint{4.766557in}{4.516544in}}%
\pgfpathcurveto{\pgfqpoint{4.774370in}{4.524357in}}{\pgfqpoint{4.778761in}{4.534956in}}{\pgfqpoint{4.778761in}{4.546006in}}%
\pgfpathcurveto{\pgfqpoint{4.778761in}{4.557057in}}{\pgfqpoint{4.774370in}{4.567656in}}{\pgfqpoint{4.766557in}{4.575469in}}%
\pgfpathcurveto{\pgfqpoint{4.758743in}{4.583283in}}{\pgfqpoint{4.748144in}{4.587673in}}{\pgfqpoint{4.737094in}{4.587673in}}%
\pgfpathcurveto{\pgfqpoint{4.726044in}{4.587673in}}{\pgfqpoint{4.715445in}{4.583283in}}{\pgfqpoint{4.707631in}{4.575469in}}%
\pgfpathcurveto{\pgfqpoint{4.699817in}{4.567656in}}{\pgfqpoint{4.695427in}{4.557057in}}{\pgfqpoint{4.695427in}{4.546006in}}%
\pgfpathcurveto{\pgfqpoint{4.695427in}{4.534956in}}{\pgfqpoint{4.699817in}{4.524357in}}{\pgfqpoint{4.707631in}{4.516544in}}%
\pgfpathcurveto{\pgfqpoint{4.715445in}{4.508730in}}{\pgfqpoint{4.726044in}{4.504340in}}{\pgfqpoint{4.737094in}{4.504340in}}%
\pgfpathclose%
\pgfusepath{stroke,fill}%
\end{pgfscope}%
\begin{pgfscope}%
\pgfpathrectangle{\pgfqpoint{0.481978in}{0.331635in}}{\pgfqpoint{9.300000in}{7.700000in}}%
\pgfusepath{clip}%
\pgfsetbuttcap%
\pgfsetroundjoin%
\definecolor{currentfill}{rgb}{1.000000,0.705882,0.509804}%
\pgfsetfillcolor{currentfill}%
\pgfsetlinewidth{0.481800pt}%
\definecolor{currentstroke}{rgb}{1.000000,1.000000,1.000000}%
\pgfsetstrokecolor{currentstroke}%
\pgfsetdash{}{0pt}%
\pgfpathmoveto{\pgfqpoint{4.102620in}{7.153439in}}%
\pgfpathcurveto{\pgfqpoint{4.113670in}{7.153439in}}{\pgfqpoint{4.124269in}{7.157829in}}{\pgfqpoint{4.132083in}{7.165642in}}%
\pgfpathcurveto{\pgfqpoint{4.139896in}{7.173456in}}{\pgfqpoint{4.144287in}{7.184055in}}{\pgfqpoint{4.144287in}{7.195105in}}%
\pgfpathcurveto{\pgfqpoint{4.144287in}{7.206155in}}{\pgfqpoint{4.139896in}{7.216754in}}{\pgfqpoint{4.132083in}{7.224568in}}%
\pgfpathcurveto{\pgfqpoint{4.124269in}{7.232382in}}{\pgfqpoint{4.113670in}{7.236772in}}{\pgfqpoint{4.102620in}{7.236772in}}%
\pgfpathcurveto{\pgfqpoint{4.091570in}{7.236772in}}{\pgfqpoint{4.080971in}{7.232382in}}{\pgfqpoint{4.073157in}{7.224568in}}%
\pgfpathcurveto{\pgfqpoint{4.065344in}{7.216754in}}{\pgfqpoint{4.060953in}{7.206155in}}{\pgfqpoint{4.060953in}{7.195105in}}%
\pgfpathcurveto{\pgfqpoint{4.060953in}{7.184055in}}{\pgfqpoint{4.065344in}{7.173456in}}{\pgfqpoint{4.073157in}{7.165642in}}%
\pgfpathcurveto{\pgfqpoint{4.080971in}{7.157829in}}{\pgfqpoint{4.091570in}{7.153439in}}{\pgfqpoint{4.102620in}{7.153439in}}%
\pgfpathclose%
\pgfusepath{stroke,fill}%
\end{pgfscope}%
\begin{pgfscope}%
\pgfpathrectangle{\pgfqpoint{0.481978in}{0.331635in}}{\pgfqpoint{9.300000in}{7.700000in}}%
\pgfusepath{clip}%
\pgfsetbuttcap%
\pgfsetroundjoin%
\definecolor{currentfill}{rgb}{1.000000,0.705882,0.509804}%
\pgfsetfillcolor{currentfill}%
\pgfsetlinewidth{0.481800pt}%
\definecolor{currentstroke}{rgb}{1.000000,1.000000,1.000000}%
\pgfsetstrokecolor{currentstroke}%
\pgfsetdash{}{0pt}%
\pgfpathmoveto{\pgfqpoint{3.960956in}{4.752480in}}%
\pgfpathcurveto{\pgfqpoint{3.972006in}{4.752480in}}{\pgfqpoint{3.982605in}{4.756870in}}{\pgfqpoint{3.990418in}{4.764684in}}%
\pgfpathcurveto{\pgfqpoint{3.998232in}{4.772497in}}{\pgfqpoint{4.002622in}{4.783097in}}{\pgfqpoint{4.002622in}{4.794147in}}%
\pgfpathcurveto{\pgfqpoint{4.002622in}{4.805197in}}{\pgfqpoint{3.998232in}{4.815796in}}{\pgfqpoint{3.990418in}{4.823609in}}%
\pgfpathcurveto{\pgfqpoint{3.982605in}{4.831423in}}{\pgfqpoint{3.972006in}{4.835813in}}{\pgfqpoint{3.960956in}{4.835813in}}%
\pgfpathcurveto{\pgfqpoint{3.949905in}{4.835813in}}{\pgfqpoint{3.939306in}{4.831423in}}{\pgfqpoint{3.931493in}{4.823609in}}%
\pgfpathcurveto{\pgfqpoint{3.923679in}{4.815796in}}{\pgfqpoint{3.919289in}{4.805197in}}{\pgfqpoint{3.919289in}{4.794147in}}%
\pgfpathcurveto{\pgfqpoint{3.919289in}{4.783097in}}{\pgfqpoint{3.923679in}{4.772497in}}{\pgfqpoint{3.931493in}{4.764684in}}%
\pgfpathcurveto{\pgfqpoint{3.939306in}{4.756870in}}{\pgfqpoint{3.949905in}{4.752480in}}{\pgfqpoint{3.960956in}{4.752480in}}%
\pgfpathclose%
\pgfusepath{stroke,fill}%
\end{pgfscope}%
\begin{pgfscope}%
\pgfpathrectangle{\pgfqpoint{0.481978in}{0.331635in}}{\pgfqpoint{9.300000in}{7.700000in}}%
\pgfusepath{clip}%
\pgfsetbuttcap%
\pgfsetroundjoin%
\definecolor{currentfill}{rgb}{1.000000,0.705882,0.509804}%
\pgfsetfillcolor{currentfill}%
\pgfsetlinewidth{0.481800pt}%
\definecolor{currentstroke}{rgb}{1.000000,1.000000,1.000000}%
\pgfsetstrokecolor{currentstroke}%
\pgfsetdash{}{0pt}%
\pgfpathmoveto{\pgfqpoint{3.386233in}{5.143212in}}%
\pgfpathcurveto{\pgfqpoint{3.397283in}{5.143212in}}{\pgfqpoint{3.407882in}{5.147603in}}{\pgfqpoint{3.415695in}{5.155416in}}%
\pgfpathcurveto{\pgfqpoint{3.423509in}{5.163230in}}{\pgfqpoint{3.427899in}{5.173829in}}{\pgfqpoint{3.427899in}{5.184879in}}%
\pgfpathcurveto{\pgfqpoint{3.427899in}{5.195929in}}{\pgfqpoint{3.423509in}{5.206528in}}{\pgfqpoint{3.415695in}{5.214342in}}%
\pgfpathcurveto{\pgfqpoint{3.407882in}{5.222155in}}{\pgfqpoint{3.397283in}{5.226546in}}{\pgfqpoint{3.386233in}{5.226546in}}%
\pgfpathcurveto{\pgfqpoint{3.375183in}{5.226546in}}{\pgfqpoint{3.364584in}{5.222155in}}{\pgfqpoint{3.356770in}{5.214342in}}%
\pgfpathcurveto{\pgfqpoint{3.348956in}{5.206528in}}{\pgfqpoint{3.344566in}{5.195929in}}{\pgfqpoint{3.344566in}{5.184879in}}%
\pgfpathcurveto{\pgfqpoint{3.344566in}{5.173829in}}{\pgfqpoint{3.348956in}{5.163230in}}{\pgfqpoint{3.356770in}{5.155416in}}%
\pgfpathcurveto{\pgfqpoint{3.364584in}{5.147603in}}{\pgfqpoint{3.375183in}{5.143212in}}{\pgfqpoint{3.386233in}{5.143212in}}%
\pgfpathclose%
\pgfusepath{stroke,fill}%
\end{pgfscope}%
\begin{pgfscope}%
\pgfpathrectangle{\pgfqpoint{0.481978in}{0.331635in}}{\pgfqpoint{9.300000in}{7.700000in}}%
\pgfusepath{clip}%
\pgfsetbuttcap%
\pgfsetroundjoin%
\definecolor{currentfill}{rgb}{1.000000,0.705882,0.509804}%
\pgfsetfillcolor{currentfill}%
\pgfsetlinewidth{0.481800pt}%
\definecolor{currentstroke}{rgb}{1.000000,1.000000,1.000000}%
\pgfsetstrokecolor{currentstroke}%
\pgfsetdash{}{0pt}%
\pgfpathmoveto{\pgfqpoint{2.506952in}{3.859071in}}%
\pgfpathcurveto{\pgfqpoint{2.518002in}{3.859071in}}{\pgfqpoint{2.528601in}{3.863462in}}{\pgfqpoint{2.536415in}{3.871275in}}%
\pgfpathcurveto{\pgfqpoint{2.544228in}{3.879089in}}{\pgfqpoint{2.548618in}{3.889688in}}{\pgfqpoint{2.548618in}{3.900738in}}%
\pgfpathcurveto{\pgfqpoint{2.548618in}{3.911788in}}{\pgfqpoint{2.544228in}{3.922387in}}{\pgfqpoint{2.536415in}{3.930201in}}%
\pgfpathcurveto{\pgfqpoint{2.528601in}{3.938015in}}{\pgfqpoint{2.518002in}{3.942405in}}{\pgfqpoint{2.506952in}{3.942405in}}%
\pgfpathcurveto{\pgfqpoint{2.495902in}{3.942405in}}{\pgfqpoint{2.485303in}{3.938015in}}{\pgfqpoint{2.477489in}{3.930201in}}%
\pgfpathcurveto{\pgfqpoint{2.469675in}{3.922387in}}{\pgfqpoint{2.465285in}{3.911788in}}{\pgfqpoint{2.465285in}{3.900738in}}%
\pgfpathcurveto{\pgfqpoint{2.465285in}{3.889688in}}{\pgfqpoint{2.469675in}{3.879089in}}{\pgfqpoint{2.477489in}{3.871275in}}%
\pgfpathcurveto{\pgfqpoint{2.485303in}{3.863462in}}{\pgfqpoint{2.495902in}{3.859071in}}{\pgfqpoint{2.506952in}{3.859071in}}%
\pgfpathclose%
\pgfusepath{stroke,fill}%
\end{pgfscope}%
\begin{pgfscope}%
\pgfpathrectangle{\pgfqpoint{0.481978in}{0.331635in}}{\pgfqpoint{9.300000in}{7.700000in}}%
\pgfusepath{clip}%
\pgfsetbuttcap%
\pgfsetroundjoin%
\definecolor{currentfill}{rgb}{1.000000,0.705882,0.509804}%
\pgfsetfillcolor{currentfill}%
\pgfsetlinewidth{0.481800pt}%
\definecolor{currentstroke}{rgb}{1.000000,1.000000,1.000000}%
\pgfsetstrokecolor{currentstroke}%
\pgfsetdash{}{0pt}%
\pgfpathmoveto{\pgfqpoint{7.641446in}{3.984517in}}%
\pgfpathcurveto{\pgfqpoint{7.652496in}{3.984517in}}{\pgfqpoint{7.663095in}{3.988907in}}{\pgfqpoint{7.670909in}{3.996720in}}%
\pgfpathcurveto{\pgfqpoint{7.678722in}{4.004534in}}{\pgfqpoint{7.683113in}{4.015133in}}{\pgfqpoint{7.683113in}{4.026183in}}%
\pgfpathcurveto{\pgfqpoint{7.683113in}{4.037233in}}{\pgfqpoint{7.678722in}{4.047832in}}{\pgfqpoint{7.670909in}{4.055646in}}%
\pgfpathcurveto{\pgfqpoint{7.663095in}{4.063460in}}{\pgfqpoint{7.652496in}{4.067850in}}{\pgfqpoint{7.641446in}{4.067850in}}%
\pgfpathcurveto{\pgfqpoint{7.630396in}{4.067850in}}{\pgfqpoint{7.619797in}{4.063460in}}{\pgfqpoint{7.611983in}{4.055646in}}%
\pgfpathcurveto{\pgfqpoint{7.604170in}{4.047832in}}{\pgfqpoint{7.599779in}{4.037233in}}{\pgfqpoint{7.599779in}{4.026183in}}%
\pgfpathcurveto{\pgfqpoint{7.599779in}{4.015133in}}{\pgfqpoint{7.604170in}{4.004534in}}{\pgfqpoint{7.611983in}{3.996720in}}%
\pgfpathcurveto{\pgfqpoint{7.619797in}{3.988907in}}{\pgfqpoint{7.630396in}{3.984517in}}{\pgfqpoint{7.641446in}{3.984517in}}%
\pgfpathclose%
\pgfusepath{stroke,fill}%
\end{pgfscope}%
\begin{pgfscope}%
\pgfpathrectangle{\pgfqpoint{0.481978in}{0.331635in}}{\pgfqpoint{9.300000in}{7.700000in}}%
\pgfusepath{clip}%
\pgfsetbuttcap%
\pgfsetroundjoin%
\definecolor{currentfill}{rgb}{1.000000,0.705882,0.509804}%
\pgfsetfillcolor{currentfill}%
\pgfsetlinewidth{0.481800pt}%
\definecolor{currentstroke}{rgb}{1.000000,1.000000,1.000000}%
\pgfsetstrokecolor{currentstroke}%
\pgfsetdash{}{0pt}%
\pgfpathmoveto{\pgfqpoint{2.626934in}{4.727491in}}%
\pgfpathcurveto{\pgfqpoint{2.637984in}{4.727491in}}{\pgfqpoint{2.648583in}{4.731882in}}{\pgfqpoint{2.656397in}{4.739695in}}%
\pgfpathcurveto{\pgfqpoint{2.664210in}{4.747509in}}{\pgfqpoint{2.668601in}{4.758108in}}{\pgfqpoint{2.668601in}{4.769158in}}%
\pgfpathcurveto{\pgfqpoint{2.668601in}{4.780208in}}{\pgfqpoint{2.664210in}{4.790807in}}{\pgfqpoint{2.656397in}{4.798621in}}%
\pgfpathcurveto{\pgfqpoint{2.648583in}{4.806434in}}{\pgfqpoint{2.637984in}{4.810825in}}{\pgfqpoint{2.626934in}{4.810825in}}%
\pgfpathcurveto{\pgfqpoint{2.615884in}{4.810825in}}{\pgfqpoint{2.605285in}{4.806434in}}{\pgfqpoint{2.597471in}{4.798621in}}%
\pgfpathcurveto{\pgfqpoint{2.589658in}{4.790807in}}{\pgfqpoint{2.585267in}{4.780208in}}{\pgfqpoint{2.585267in}{4.769158in}}%
\pgfpathcurveto{\pgfqpoint{2.585267in}{4.758108in}}{\pgfqpoint{2.589658in}{4.747509in}}{\pgfqpoint{2.597471in}{4.739695in}}%
\pgfpathcurveto{\pgfqpoint{2.605285in}{4.731882in}}{\pgfqpoint{2.615884in}{4.727491in}}{\pgfqpoint{2.626934in}{4.727491in}}%
\pgfpathclose%
\pgfusepath{stroke,fill}%
\end{pgfscope}%
\begin{pgfscope}%
\pgfpathrectangle{\pgfqpoint{0.481978in}{0.331635in}}{\pgfqpoint{9.300000in}{7.700000in}}%
\pgfusepath{clip}%
\pgfsetbuttcap%
\pgfsetroundjoin%
\definecolor{currentfill}{rgb}{1.000000,0.705882,0.509804}%
\pgfsetfillcolor{currentfill}%
\pgfsetlinewidth{0.481800pt}%
\definecolor{currentstroke}{rgb}{1.000000,1.000000,1.000000}%
\pgfsetstrokecolor{currentstroke}%
\pgfsetdash{}{0pt}%
\pgfpathmoveto{\pgfqpoint{4.292588in}{0.962886in}}%
\pgfpathcurveto{\pgfqpoint{4.303638in}{0.962886in}}{\pgfqpoint{4.314237in}{0.967276in}}{\pgfqpoint{4.322051in}{0.975090in}}%
\pgfpathcurveto{\pgfqpoint{4.329865in}{0.982903in}}{\pgfqpoint{4.334255in}{0.993502in}}{\pgfqpoint{4.334255in}{1.004552in}}%
\pgfpathcurveto{\pgfqpoint{4.334255in}{1.015602in}}{\pgfqpoint{4.329865in}{1.026202in}}{\pgfqpoint{4.322051in}{1.034015in}}%
\pgfpathcurveto{\pgfqpoint{4.314237in}{1.041829in}}{\pgfqpoint{4.303638in}{1.046219in}}{\pgfqpoint{4.292588in}{1.046219in}}%
\pgfpathcurveto{\pgfqpoint{4.281538in}{1.046219in}}{\pgfqpoint{4.270939in}{1.041829in}}{\pgfqpoint{4.263125in}{1.034015in}}%
\pgfpathcurveto{\pgfqpoint{4.255312in}{1.026202in}}{\pgfqpoint{4.250922in}{1.015602in}}{\pgfqpoint{4.250922in}{1.004552in}}%
\pgfpathcurveto{\pgfqpoint{4.250922in}{0.993502in}}{\pgfqpoint{4.255312in}{0.982903in}}{\pgfqpoint{4.263125in}{0.975090in}}%
\pgfpathcurveto{\pgfqpoint{4.270939in}{0.967276in}}{\pgfqpoint{4.281538in}{0.962886in}}{\pgfqpoint{4.292588in}{0.962886in}}%
\pgfpathclose%
\pgfusepath{stroke,fill}%
\end{pgfscope}%
\begin{pgfscope}%
\pgfpathrectangle{\pgfqpoint{0.481978in}{0.331635in}}{\pgfqpoint{9.300000in}{7.700000in}}%
\pgfusepath{clip}%
\pgfsetbuttcap%
\pgfsetroundjoin%
\definecolor{currentfill}{rgb}{1.000000,0.705882,0.509804}%
\pgfsetfillcolor{currentfill}%
\pgfsetlinewidth{0.481800pt}%
\definecolor{currentstroke}{rgb}{1.000000,1.000000,1.000000}%
\pgfsetstrokecolor{currentstroke}%
\pgfsetdash{}{0pt}%
\pgfpathmoveto{\pgfqpoint{4.914941in}{2.209440in}}%
\pgfpathcurveto{\pgfqpoint{4.925991in}{2.209440in}}{\pgfqpoint{4.936590in}{2.213830in}}{\pgfqpoint{4.944403in}{2.221643in}}%
\pgfpathcurveto{\pgfqpoint{4.952217in}{2.229457in}}{\pgfqpoint{4.956607in}{2.240056in}}{\pgfqpoint{4.956607in}{2.251106in}}%
\pgfpathcurveto{\pgfqpoint{4.956607in}{2.262156in}}{\pgfqpoint{4.952217in}{2.272755in}}{\pgfqpoint{4.944403in}{2.280569in}}%
\pgfpathcurveto{\pgfqpoint{4.936590in}{2.288383in}}{\pgfqpoint{4.925991in}{2.292773in}}{\pgfqpoint{4.914941in}{2.292773in}}%
\pgfpathcurveto{\pgfqpoint{4.903891in}{2.292773in}}{\pgfqpoint{4.893292in}{2.288383in}}{\pgfqpoint{4.885478in}{2.280569in}}%
\pgfpathcurveto{\pgfqpoint{4.877664in}{2.272755in}}{\pgfqpoint{4.873274in}{2.262156in}}{\pgfqpoint{4.873274in}{2.251106in}}%
\pgfpathcurveto{\pgfqpoint{4.873274in}{2.240056in}}{\pgfqpoint{4.877664in}{2.229457in}}{\pgfqpoint{4.885478in}{2.221643in}}%
\pgfpathcurveto{\pgfqpoint{4.893292in}{2.213830in}}{\pgfqpoint{4.903891in}{2.209440in}}{\pgfqpoint{4.914941in}{2.209440in}}%
\pgfpathclose%
\pgfusepath{stroke,fill}%
\end{pgfscope}%
\begin{pgfscope}%
\pgfpathrectangle{\pgfqpoint{0.481978in}{0.331635in}}{\pgfqpoint{9.300000in}{7.700000in}}%
\pgfusepath{clip}%
\pgfsetbuttcap%
\pgfsetroundjoin%
\definecolor{currentfill}{rgb}{1.000000,0.705882,0.509804}%
\pgfsetfillcolor{currentfill}%
\pgfsetlinewidth{0.481800pt}%
\definecolor{currentstroke}{rgb}{1.000000,1.000000,1.000000}%
\pgfsetstrokecolor{currentstroke}%
\pgfsetdash{}{0pt}%
\pgfpathmoveto{\pgfqpoint{1.360419in}{3.188475in}}%
\pgfpathcurveto{\pgfqpoint{1.371469in}{3.188475in}}{\pgfqpoint{1.382068in}{3.192866in}}{\pgfqpoint{1.389882in}{3.200679in}}%
\pgfpathcurveto{\pgfqpoint{1.397695in}{3.208493in}}{\pgfqpoint{1.402086in}{3.219092in}}{\pgfqpoint{1.402086in}{3.230142in}}%
\pgfpathcurveto{\pgfqpoint{1.402086in}{3.241192in}}{\pgfqpoint{1.397695in}{3.251791in}}{\pgfqpoint{1.389882in}{3.259605in}}%
\pgfpathcurveto{\pgfqpoint{1.382068in}{3.267418in}}{\pgfqpoint{1.371469in}{3.271809in}}{\pgfqpoint{1.360419in}{3.271809in}}%
\pgfpathcurveto{\pgfqpoint{1.349369in}{3.271809in}}{\pgfqpoint{1.338770in}{3.267418in}}{\pgfqpoint{1.330956in}{3.259605in}}%
\pgfpathcurveto{\pgfqpoint{1.323143in}{3.251791in}}{\pgfqpoint{1.318752in}{3.241192in}}{\pgfqpoint{1.318752in}{3.230142in}}%
\pgfpathcurveto{\pgfqpoint{1.318752in}{3.219092in}}{\pgfqpoint{1.323143in}{3.208493in}}{\pgfqpoint{1.330956in}{3.200679in}}%
\pgfpathcurveto{\pgfqpoint{1.338770in}{3.192866in}}{\pgfqpoint{1.349369in}{3.188475in}}{\pgfqpoint{1.360419in}{3.188475in}}%
\pgfpathclose%
\pgfusepath{stroke,fill}%
\end{pgfscope}%
\begin{pgfscope}%
\pgfpathrectangle{\pgfqpoint{0.481978in}{0.331635in}}{\pgfqpoint{9.300000in}{7.700000in}}%
\pgfusepath{clip}%
\pgfsetbuttcap%
\pgfsetroundjoin%
\definecolor{currentfill}{rgb}{1.000000,0.705882,0.509804}%
\pgfsetfillcolor{currentfill}%
\pgfsetlinewidth{0.481800pt}%
\definecolor{currentstroke}{rgb}{1.000000,1.000000,1.000000}%
\pgfsetstrokecolor{currentstroke}%
\pgfsetdash{}{0pt}%
\pgfpathmoveto{\pgfqpoint{2.680958in}{5.597608in}}%
\pgfpathcurveto{\pgfqpoint{2.692008in}{5.597608in}}{\pgfqpoint{2.702607in}{5.601998in}}{\pgfqpoint{2.710420in}{5.609812in}}%
\pgfpathcurveto{\pgfqpoint{2.718234in}{5.617625in}}{\pgfqpoint{2.722624in}{5.628224in}}{\pgfqpoint{2.722624in}{5.639274in}}%
\pgfpathcurveto{\pgfqpoint{2.722624in}{5.650325in}}{\pgfqpoint{2.718234in}{5.660924in}}{\pgfqpoint{2.710420in}{5.668737in}}%
\pgfpathcurveto{\pgfqpoint{2.702607in}{5.676551in}}{\pgfqpoint{2.692008in}{5.680941in}}{\pgfqpoint{2.680958in}{5.680941in}}%
\pgfpathcurveto{\pgfqpoint{2.669908in}{5.680941in}}{\pgfqpoint{2.659309in}{5.676551in}}{\pgfqpoint{2.651495in}{5.668737in}}%
\pgfpathcurveto{\pgfqpoint{2.643681in}{5.660924in}}{\pgfqpoint{2.639291in}{5.650325in}}{\pgfqpoint{2.639291in}{5.639274in}}%
\pgfpathcurveto{\pgfqpoint{2.639291in}{5.628224in}}{\pgfqpoint{2.643681in}{5.617625in}}{\pgfqpoint{2.651495in}{5.609812in}}%
\pgfpathcurveto{\pgfqpoint{2.659309in}{5.601998in}}{\pgfqpoint{2.669908in}{5.597608in}}{\pgfqpoint{2.680958in}{5.597608in}}%
\pgfpathclose%
\pgfusepath{stroke,fill}%
\end{pgfscope}%
\begin{pgfscope}%
\pgfpathrectangle{\pgfqpoint{0.481978in}{0.331635in}}{\pgfqpoint{9.300000in}{7.700000in}}%
\pgfusepath{clip}%
\pgfsetbuttcap%
\pgfsetroundjoin%
\definecolor{currentfill}{rgb}{1.000000,0.705882,0.509804}%
\pgfsetfillcolor{currentfill}%
\pgfsetlinewidth{0.481800pt}%
\definecolor{currentstroke}{rgb}{1.000000,1.000000,1.000000}%
\pgfsetstrokecolor{currentstroke}%
\pgfsetdash{}{0pt}%
\pgfpathmoveto{\pgfqpoint{8.439887in}{6.135148in}}%
\pgfpathcurveto{\pgfqpoint{8.450937in}{6.135148in}}{\pgfqpoint{8.461536in}{6.139538in}}{\pgfqpoint{8.469350in}{6.147352in}}%
\pgfpathcurveto{\pgfqpoint{8.477163in}{6.155166in}}{\pgfqpoint{8.481553in}{6.165765in}}{\pgfqpoint{8.481553in}{6.176815in}}%
\pgfpathcurveto{\pgfqpoint{8.481553in}{6.187865in}}{\pgfqpoint{8.477163in}{6.198464in}}{\pgfqpoint{8.469350in}{6.206278in}}%
\pgfpathcurveto{\pgfqpoint{8.461536in}{6.214091in}}{\pgfqpoint{8.450937in}{6.218481in}}{\pgfqpoint{8.439887in}{6.218481in}}%
\pgfpathcurveto{\pgfqpoint{8.428837in}{6.218481in}}{\pgfqpoint{8.418238in}{6.214091in}}{\pgfqpoint{8.410424in}{6.206278in}}%
\pgfpathcurveto{\pgfqpoint{8.402610in}{6.198464in}}{\pgfqpoint{8.398220in}{6.187865in}}{\pgfqpoint{8.398220in}{6.176815in}}%
\pgfpathcurveto{\pgfqpoint{8.398220in}{6.165765in}}{\pgfqpoint{8.402610in}{6.155166in}}{\pgfqpoint{8.410424in}{6.147352in}}%
\pgfpathcurveto{\pgfqpoint{8.418238in}{6.139538in}}{\pgfqpoint{8.428837in}{6.135148in}}{\pgfqpoint{8.439887in}{6.135148in}}%
\pgfpathclose%
\pgfusepath{stroke,fill}%
\end{pgfscope}%
\begin{pgfscope}%
\pgfpathrectangle{\pgfqpoint{0.481978in}{0.331635in}}{\pgfqpoint{9.300000in}{7.700000in}}%
\pgfusepath{clip}%
\pgfsetbuttcap%
\pgfsetroundjoin%
\definecolor{currentfill}{rgb}{1.000000,0.705882,0.509804}%
\pgfsetfillcolor{currentfill}%
\pgfsetlinewidth{0.481800pt}%
\definecolor{currentstroke}{rgb}{1.000000,1.000000,1.000000}%
\pgfsetstrokecolor{currentstroke}%
\pgfsetdash{}{0pt}%
\pgfpathmoveto{\pgfqpoint{5.343023in}{7.639968in}}%
\pgfpathcurveto{\pgfqpoint{5.354073in}{7.639968in}}{\pgfqpoint{5.364672in}{7.644359in}}{\pgfqpoint{5.372486in}{7.652172in}}%
\pgfpathcurveto{\pgfqpoint{5.380299in}{7.659986in}}{\pgfqpoint{5.384690in}{7.670585in}}{\pgfqpoint{5.384690in}{7.681635in}}%
\pgfpathcurveto{\pgfqpoint{5.384690in}{7.692685in}}{\pgfqpoint{5.380299in}{7.703284in}}{\pgfqpoint{5.372486in}{7.711098in}}%
\pgfpathcurveto{\pgfqpoint{5.364672in}{7.718911in}}{\pgfqpoint{5.354073in}{7.723302in}}{\pgfqpoint{5.343023in}{7.723302in}}%
\pgfpathcurveto{\pgfqpoint{5.331973in}{7.723302in}}{\pgfqpoint{5.321374in}{7.718911in}}{\pgfqpoint{5.313560in}{7.711098in}}%
\pgfpathcurveto{\pgfqpoint{5.305747in}{7.703284in}}{\pgfqpoint{5.301356in}{7.692685in}}{\pgfqpoint{5.301356in}{7.681635in}}%
\pgfpathcurveto{\pgfqpoint{5.301356in}{7.670585in}}{\pgfqpoint{5.305747in}{7.659986in}}{\pgfqpoint{5.313560in}{7.652172in}}%
\pgfpathcurveto{\pgfqpoint{5.321374in}{7.644359in}}{\pgfqpoint{5.331973in}{7.639968in}}{\pgfqpoint{5.343023in}{7.639968in}}%
\pgfpathclose%
\pgfusepath{stroke,fill}%
\end{pgfscope}%
\begin{pgfscope}%
\pgfpathrectangle{\pgfqpoint{0.481978in}{0.331635in}}{\pgfqpoint{9.300000in}{7.700000in}}%
\pgfusepath{clip}%
\pgfsetbuttcap%
\pgfsetroundjoin%
\definecolor{currentfill}{rgb}{1.000000,0.705882,0.509804}%
\pgfsetfillcolor{currentfill}%
\pgfsetlinewidth{0.481800pt}%
\definecolor{currentstroke}{rgb}{1.000000,1.000000,1.000000}%
\pgfsetstrokecolor{currentstroke}%
\pgfsetdash{}{0pt}%
\pgfpathmoveto{\pgfqpoint{1.984154in}{4.127682in}}%
\pgfpathcurveto{\pgfqpoint{1.995204in}{4.127682in}}{\pgfqpoint{2.005803in}{4.132072in}}{\pgfqpoint{2.013617in}{4.139886in}}%
\pgfpathcurveto{\pgfqpoint{2.021431in}{4.147700in}}{\pgfqpoint{2.025821in}{4.158299in}}{\pgfqpoint{2.025821in}{4.169349in}}%
\pgfpathcurveto{\pgfqpoint{2.025821in}{4.180399in}}{\pgfqpoint{2.021431in}{4.190998in}}{\pgfqpoint{2.013617in}{4.198812in}}%
\pgfpathcurveto{\pgfqpoint{2.005803in}{4.206625in}}{\pgfqpoint{1.995204in}{4.211015in}}{\pgfqpoint{1.984154in}{4.211015in}}%
\pgfpathcurveto{\pgfqpoint{1.973104in}{4.211015in}}{\pgfqpoint{1.962505in}{4.206625in}}{\pgfqpoint{1.954691in}{4.198812in}}%
\pgfpathcurveto{\pgfqpoint{1.946878in}{4.190998in}}{\pgfqpoint{1.942487in}{4.180399in}}{\pgfqpoint{1.942487in}{4.169349in}}%
\pgfpathcurveto{\pgfqpoint{1.942487in}{4.158299in}}{\pgfqpoint{1.946878in}{4.147700in}}{\pgfqpoint{1.954691in}{4.139886in}}%
\pgfpathcurveto{\pgfqpoint{1.962505in}{4.132072in}}{\pgfqpoint{1.973104in}{4.127682in}}{\pgfqpoint{1.984154in}{4.127682in}}%
\pgfpathclose%
\pgfusepath{stroke,fill}%
\end{pgfscope}%
\begin{pgfscope}%
\pgfpathrectangle{\pgfqpoint{0.481978in}{0.331635in}}{\pgfqpoint{9.300000in}{7.700000in}}%
\pgfusepath{clip}%
\pgfsetbuttcap%
\pgfsetroundjoin%
\definecolor{currentfill}{rgb}{1.000000,0.705882,0.509804}%
\pgfsetfillcolor{currentfill}%
\pgfsetlinewidth{0.481800pt}%
\definecolor{currentstroke}{rgb}{1.000000,1.000000,1.000000}%
\pgfsetstrokecolor{currentstroke}%
\pgfsetdash{}{0pt}%
\pgfpathmoveto{\pgfqpoint{8.035769in}{4.906239in}}%
\pgfpathcurveto{\pgfqpoint{8.046819in}{4.906239in}}{\pgfqpoint{8.057418in}{4.910629in}}{\pgfqpoint{8.065232in}{4.918443in}}%
\pgfpathcurveto{\pgfqpoint{8.073046in}{4.926256in}}{\pgfqpoint{8.077436in}{4.936855in}}{\pgfqpoint{8.077436in}{4.947906in}}%
\pgfpathcurveto{\pgfqpoint{8.077436in}{4.958956in}}{\pgfqpoint{8.073046in}{4.969555in}}{\pgfqpoint{8.065232in}{4.977368in}}%
\pgfpathcurveto{\pgfqpoint{8.057418in}{4.985182in}}{\pgfqpoint{8.046819in}{4.989572in}}{\pgfqpoint{8.035769in}{4.989572in}}%
\pgfpathcurveto{\pgfqpoint{8.024719in}{4.989572in}}{\pgfqpoint{8.014120in}{4.985182in}}{\pgfqpoint{8.006307in}{4.977368in}}%
\pgfpathcurveto{\pgfqpoint{7.998493in}{4.969555in}}{\pgfqpoint{7.994103in}{4.958956in}}{\pgfqpoint{7.994103in}{4.947906in}}%
\pgfpathcurveto{\pgfqpoint{7.994103in}{4.936855in}}{\pgfqpoint{7.998493in}{4.926256in}}{\pgfqpoint{8.006307in}{4.918443in}}%
\pgfpathcurveto{\pgfqpoint{8.014120in}{4.910629in}}{\pgfqpoint{8.024719in}{4.906239in}}{\pgfqpoint{8.035769in}{4.906239in}}%
\pgfpathclose%
\pgfusepath{stroke,fill}%
\end{pgfscope}%
\begin{pgfscope}%
\pgfpathrectangle{\pgfqpoint{0.481978in}{0.331635in}}{\pgfqpoint{9.300000in}{7.700000in}}%
\pgfusepath{clip}%
\pgfsetbuttcap%
\pgfsetroundjoin%
\definecolor{currentfill}{rgb}{1.000000,0.705882,0.509804}%
\pgfsetfillcolor{currentfill}%
\pgfsetlinewidth{0.481800pt}%
\definecolor{currentstroke}{rgb}{1.000000,1.000000,1.000000}%
\pgfsetstrokecolor{currentstroke}%
\pgfsetdash{}{0pt}%
\pgfpathmoveto{\pgfqpoint{6.113857in}{3.264397in}}%
\pgfpathcurveto{\pgfqpoint{6.124907in}{3.264397in}}{\pgfqpoint{6.135506in}{3.268787in}}{\pgfqpoint{6.143320in}{3.276601in}}%
\pgfpathcurveto{\pgfqpoint{6.151133in}{3.284414in}}{\pgfqpoint{6.155524in}{3.295013in}}{\pgfqpoint{6.155524in}{3.306063in}}%
\pgfpathcurveto{\pgfqpoint{6.155524in}{3.317114in}}{\pgfqpoint{6.151133in}{3.327713in}}{\pgfqpoint{6.143320in}{3.335526in}}%
\pgfpathcurveto{\pgfqpoint{6.135506in}{3.343340in}}{\pgfqpoint{6.124907in}{3.347730in}}{\pgfqpoint{6.113857in}{3.347730in}}%
\pgfpathcurveto{\pgfqpoint{6.102807in}{3.347730in}}{\pgfqpoint{6.092208in}{3.343340in}}{\pgfqpoint{6.084394in}{3.335526in}}%
\pgfpathcurveto{\pgfqpoint{6.076581in}{3.327713in}}{\pgfqpoint{6.072190in}{3.317114in}}{\pgfqpoint{6.072190in}{3.306063in}}%
\pgfpathcurveto{\pgfqpoint{6.072190in}{3.295013in}}{\pgfqpoint{6.076581in}{3.284414in}}{\pgfqpoint{6.084394in}{3.276601in}}%
\pgfpathcurveto{\pgfqpoint{6.092208in}{3.268787in}}{\pgfqpoint{6.102807in}{3.264397in}}{\pgfqpoint{6.113857in}{3.264397in}}%
\pgfpathclose%
\pgfusepath{stroke,fill}%
\end{pgfscope}%
\begin{pgfscope}%
\pgfpathrectangle{\pgfqpoint{0.481978in}{0.331635in}}{\pgfqpoint{9.300000in}{7.700000in}}%
\pgfusepath{clip}%
\pgfsetbuttcap%
\pgfsetroundjoin%
\definecolor{currentfill}{rgb}{1.000000,0.705882,0.509804}%
\pgfsetfillcolor{currentfill}%
\pgfsetlinewidth{0.481800pt}%
\definecolor{currentstroke}{rgb}{1.000000,1.000000,1.000000}%
\pgfsetstrokecolor{currentstroke}%
\pgfsetdash{}{0pt}%
\pgfpathmoveto{\pgfqpoint{5.267481in}{4.055813in}}%
\pgfpathcurveto{\pgfqpoint{5.278531in}{4.055813in}}{\pgfqpoint{5.289130in}{4.060203in}}{\pgfqpoint{5.296944in}{4.068017in}}%
\pgfpathcurveto{\pgfqpoint{5.304758in}{4.075831in}}{\pgfqpoint{5.309148in}{4.086430in}}{\pgfqpoint{5.309148in}{4.097480in}}%
\pgfpathcurveto{\pgfqpoint{5.309148in}{4.108530in}}{\pgfqpoint{5.304758in}{4.119129in}}{\pgfqpoint{5.296944in}{4.126943in}}%
\pgfpathcurveto{\pgfqpoint{5.289130in}{4.134756in}}{\pgfqpoint{5.278531in}{4.139146in}}{\pgfqpoint{5.267481in}{4.139146in}}%
\pgfpathcurveto{\pgfqpoint{5.256431in}{4.139146in}}{\pgfqpoint{5.245832in}{4.134756in}}{\pgfqpoint{5.238018in}{4.126943in}}%
\pgfpathcurveto{\pgfqpoint{5.230205in}{4.119129in}}{\pgfqpoint{5.225815in}{4.108530in}}{\pgfqpoint{5.225815in}{4.097480in}}%
\pgfpathcurveto{\pgfqpoint{5.225815in}{4.086430in}}{\pgfqpoint{5.230205in}{4.075831in}}{\pgfqpoint{5.238018in}{4.068017in}}%
\pgfpathcurveto{\pgfqpoint{5.245832in}{4.060203in}}{\pgfqpoint{5.256431in}{4.055813in}}{\pgfqpoint{5.267481in}{4.055813in}}%
\pgfpathclose%
\pgfusepath{stroke,fill}%
\end{pgfscope}%
\begin{pgfscope}%
\pgfpathrectangle{\pgfqpoint{0.481978in}{0.331635in}}{\pgfqpoint{9.300000in}{7.700000in}}%
\pgfusepath{clip}%
\pgfsetbuttcap%
\pgfsetroundjoin%
\definecolor{currentfill}{rgb}{1.000000,0.705882,0.509804}%
\pgfsetfillcolor{currentfill}%
\pgfsetlinewidth{0.481800pt}%
\definecolor{currentstroke}{rgb}{1.000000,1.000000,1.000000}%
\pgfsetstrokecolor{currentstroke}%
\pgfsetdash{}{0pt}%
\pgfpathmoveto{\pgfqpoint{2.715921in}{1.829469in}}%
\pgfpathcurveto{\pgfqpoint{2.726972in}{1.829469in}}{\pgfqpoint{2.737571in}{1.833859in}}{\pgfqpoint{2.745384in}{1.841673in}}%
\pgfpathcurveto{\pgfqpoint{2.753198in}{1.849487in}}{\pgfqpoint{2.757588in}{1.860086in}}{\pgfqpoint{2.757588in}{1.871136in}}%
\pgfpathcurveto{\pgfqpoint{2.757588in}{1.882186in}}{\pgfqpoint{2.753198in}{1.892785in}}{\pgfqpoint{2.745384in}{1.900599in}}%
\pgfpathcurveto{\pgfqpoint{2.737571in}{1.908412in}}{\pgfqpoint{2.726972in}{1.912803in}}{\pgfqpoint{2.715921in}{1.912803in}}%
\pgfpathcurveto{\pgfqpoint{2.704871in}{1.912803in}}{\pgfqpoint{2.694272in}{1.908412in}}{\pgfqpoint{2.686459in}{1.900599in}}%
\pgfpathcurveto{\pgfqpoint{2.678645in}{1.892785in}}{\pgfqpoint{2.674255in}{1.882186in}}{\pgfqpoint{2.674255in}{1.871136in}}%
\pgfpathcurveto{\pgfqpoint{2.674255in}{1.860086in}}{\pgfqpoint{2.678645in}{1.849487in}}{\pgfqpoint{2.686459in}{1.841673in}}%
\pgfpathcurveto{\pgfqpoint{2.694272in}{1.833859in}}{\pgfqpoint{2.704871in}{1.829469in}}{\pgfqpoint{2.715921in}{1.829469in}}%
\pgfpathclose%
\pgfusepath{stroke,fill}%
\end{pgfscope}%
\begin{pgfscope}%
\pgfpathrectangle{\pgfqpoint{0.481978in}{0.331635in}}{\pgfqpoint{9.300000in}{7.700000in}}%
\pgfusepath{clip}%
\pgfsetbuttcap%
\pgfsetroundjoin%
\definecolor{currentfill}{rgb}{1.000000,0.705882,0.509804}%
\pgfsetfillcolor{currentfill}%
\pgfsetlinewidth{0.481800pt}%
\definecolor{currentstroke}{rgb}{1.000000,1.000000,1.000000}%
\pgfsetstrokecolor{currentstroke}%
\pgfsetdash{}{0pt}%
\pgfpathmoveto{\pgfqpoint{0.904705in}{5.054642in}}%
\pgfpathcurveto{\pgfqpoint{0.915755in}{5.054642in}}{\pgfqpoint{0.926354in}{5.059032in}}{\pgfqpoint{0.934168in}{5.066846in}}%
\pgfpathcurveto{\pgfqpoint{0.941982in}{5.074660in}}{\pgfqpoint{0.946372in}{5.085259in}}{\pgfqpoint{0.946372in}{5.096309in}}%
\pgfpathcurveto{\pgfqpoint{0.946372in}{5.107359in}}{\pgfqpoint{0.941982in}{5.117958in}}{\pgfqpoint{0.934168in}{5.125772in}}%
\pgfpathcurveto{\pgfqpoint{0.926354in}{5.133585in}}{\pgfqpoint{0.915755in}{5.137975in}}{\pgfqpoint{0.904705in}{5.137975in}}%
\pgfpathcurveto{\pgfqpoint{0.893655in}{5.137975in}}{\pgfqpoint{0.883056in}{5.133585in}}{\pgfqpoint{0.875242in}{5.125772in}}%
\pgfpathcurveto{\pgfqpoint{0.867429in}{5.117958in}}{\pgfqpoint{0.863039in}{5.107359in}}{\pgfqpoint{0.863039in}{5.096309in}}%
\pgfpathcurveto{\pgfqpoint{0.863039in}{5.085259in}}{\pgfqpoint{0.867429in}{5.074660in}}{\pgfqpoint{0.875242in}{5.066846in}}%
\pgfpathcurveto{\pgfqpoint{0.883056in}{5.059032in}}{\pgfqpoint{0.893655in}{5.054642in}}{\pgfqpoint{0.904705in}{5.054642in}}%
\pgfpathclose%
\pgfusepath{stroke,fill}%
\end{pgfscope}%
\begin{pgfscope}%
\pgfpathrectangle{\pgfqpoint{0.481978in}{0.331635in}}{\pgfqpoint{9.300000in}{7.700000in}}%
\pgfusepath{clip}%
\pgfsetbuttcap%
\pgfsetroundjoin%
\definecolor{currentfill}{rgb}{1.000000,0.705882,0.509804}%
\pgfsetfillcolor{currentfill}%
\pgfsetlinewidth{0.481800pt}%
\definecolor{currentstroke}{rgb}{1.000000,1.000000,1.000000}%
\pgfsetstrokecolor{currentstroke}%
\pgfsetdash{}{0pt}%
\pgfpathmoveto{\pgfqpoint{8.533110in}{3.540648in}}%
\pgfpathcurveto{\pgfqpoint{8.544160in}{3.540648in}}{\pgfqpoint{8.554759in}{3.545038in}}{\pgfqpoint{8.562572in}{3.552852in}}%
\pgfpathcurveto{\pgfqpoint{8.570386in}{3.560666in}}{\pgfqpoint{8.574776in}{3.571265in}}{\pgfqpoint{8.574776in}{3.582315in}}%
\pgfpathcurveto{\pgfqpoint{8.574776in}{3.593365in}}{\pgfqpoint{8.570386in}{3.603964in}}{\pgfqpoint{8.562572in}{3.611777in}}%
\pgfpathcurveto{\pgfqpoint{8.554759in}{3.619591in}}{\pgfqpoint{8.544160in}{3.623981in}}{\pgfqpoint{8.533110in}{3.623981in}}%
\pgfpathcurveto{\pgfqpoint{8.522059in}{3.623981in}}{\pgfqpoint{8.511460in}{3.619591in}}{\pgfqpoint{8.503647in}{3.611777in}}%
\pgfpathcurveto{\pgfqpoint{8.495833in}{3.603964in}}{\pgfqpoint{8.491443in}{3.593365in}}{\pgfqpoint{8.491443in}{3.582315in}}%
\pgfpathcurveto{\pgfqpoint{8.491443in}{3.571265in}}{\pgfqpoint{8.495833in}{3.560666in}}{\pgfqpoint{8.503647in}{3.552852in}}%
\pgfpathcurveto{\pgfqpoint{8.511460in}{3.545038in}}{\pgfqpoint{8.522059in}{3.540648in}}{\pgfqpoint{8.533110in}{3.540648in}}%
\pgfpathclose%
\pgfusepath{stroke,fill}%
\end{pgfscope}%
\begin{pgfscope}%
\pgfpathrectangle{\pgfqpoint{0.481978in}{0.331635in}}{\pgfqpoint{9.300000in}{7.700000in}}%
\pgfusepath{clip}%
\pgfsetbuttcap%
\pgfsetroundjoin%
\definecolor{currentfill}{rgb}{1.000000,0.705882,0.509804}%
\pgfsetfillcolor{currentfill}%
\pgfsetlinewidth{0.481800pt}%
\definecolor{currentstroke}{rgb}{1.000000,1.000000,1.000000}%
\pgfsetstrokecolor{currentstroke}%
\pgfsetdash{}{0pt}%
\pgfpathmoveto{\pgfqpoint{4.391134in}{3.007273in}}%
\pgfpathcurveto{\pgfqpoint{4.402184in}{3.007273in}}{\pgfqpoint{4.412783in}{3.011663in}}{\pgfqpoint{4.420597in}{3.019477in}}%
\pgfpathcurveto{\pgfqpoint{4.428410in}{3.027290in}}{\pgfqpoint{4.432800in}{3.037889in}}{\pgfqpoint{4.432800in}{3.048940in}}%
\pgfpathcurveto{\pgfqpoint{4.432800in}{3.059990in}}{\pgfqpoint{4.428410in}{3.070589in}}{\pgfqpoint{4.420597in}{3.078402in}}%
\pgfpathcurveto{\pgfqpoint{4.412783in}{3.086216in}}{\pgfqpoint{4.402184in}{3.090606in}}{\pgfqpoint{4.391134in}{3.090606in}}%
\pgfpathcurveto{\pgfqpoint{4.380084in}{3.090606in}}{\pgfqpoint{4.369485in}{3.086216in}}{\pgfqpoint{4.361671in}{3.078402in}}%
\pgfpathcurveto{\pgfqpoint{4.353857in}{3.070589in}}{\pgfqpoint{4.349467in}{3.059990in}}{\pgfqpoint{4.349467in}{3.048940in}}%
\pgfpathcurveto{\pgfqpoint{4.349467in}{3.037889in}}{\pgfqpoint{4.353857in}{3.027290in}}{\pgfqpoint{4.361671in}{3.019477in}}%
\pgfpathcurveto{\pgfqpoint{4.369485in}{3.011663in}}{\pgfqpoint{4.380084in}{3.007273in}}{\pgfqpoint{4.391134in}{3.007273in}}%
\pgfpathclose%
\pgfusepath{stroke,fill}%
\end{pgfscope}%
\begin{pgfscope}%
\pgfpathrectangle{\pgfqpoint{0.481978in}{0.331635in}}{\pgfqpoint{9.300000in}{7.700000in}}%
\pgfusepath{clip}%
\pgfsetbuttcap%
\pgfsetroundjoin%
\definecolor{currentfill}{rgb}{1.000000,0.705882,0.509804}%
\pgfsetfillcolor{currentfill}%
\pgfsetlinewidth{0.481800pt}%
\definecolor{currentstroke}{rgb}{1.000000,1.000000,1.000000}%
\pgfsetstrokecolor{currentstroke}%
\pgfsetdash{}{0pt}%
\pgfpathmoveto{\pgfqpoint{1.537381in}{5.295533in}}%
\pgfpathcurveto{\pgfqpoint{1.548432in}{5.295533in}}{\pgfqpoint{1.559031in}{5.299923in}}{\pgfqpoint{1.566844in}{5.307737in}}%
\pgfpathcurveto{\pgfqpoint{1.574658in}{5.315551in}}{\pgfqpoint{1.579048in}{5.326150in}}{\pgfqpoint{1.579048in}{5.337200in}}%
\pgfpathcurveto{\pgfqpoint{1.579048in}{5.348250in}}{\pgfqpoint{1.574658in}{5.358849in}}{\pgfqpoint{1.566844in}{5.366662in}}%
\pgfpathcurveto{\pgfqpoint{1.559031in}{5.374476in}}{\pgfqpoint{1.548432in}{5.378866in}}{\pgfqpoint{1.537381in}{5.378866in}}%
\pgfpathcurveto{\pgfqpoint{1.526331in}{5.378866in}}{\pgfqpoint{1.515732in}{5.374476in}}{\pgfqpoint{1.507919in}{5.366662in}}%
\pgfpathcurveto{\pgfqpoint{1.500105in}{5.358849in}}{\pgfqpoint{1.495715in}{5.348250in}}{\pgfqpoint{1.495715in}{5.337200in}}%
\pgfpathcurveto{\pgfqpoint{1.495715in}{5.326150in}}{\pgfqpoint{1.500105in}{5.315551in}}{\pgfqpoint{1.507919in}{5.307737in}}%
\pgfpathcurveto{\pgfqpoint{1.515732in}{5.299923in}}{\pgfqpoint{1.526331in}{5.295533in}}{\pgfqpoint{1.537381in}{5.295533in}}%
\pgfpathclose%
\pgfusepath{stroke,fill}%
\end{pgfscope}%
\begin{pgfscope}%
\pgfpathrectangle{\pgfqpoint{0.481978in}{0.331635in}}{\pgfqpoint{9.300000in}{7.700000in}}%
\pgfusepath{clip}%
\pgfsetbuttcap%
\pgfsetroundjoin%
\definecolor{currentfill}{rgb}{1.000000,0.705882,0.509804}%
\pgfsetfillcolor{currentfill}%
\pgfsetlinewidth{0.481800pt}%
\definecolor{currentstroke}{rgb}{1.000000,1.000000,1.000000}%
\pgfsetstrokecolor{currentstroke}%
\pgfsetdash{}{0pt}%
\pgfpathmoveto{\pgfqpoint{6.879175in}{3.698562in}}%
\pgfpathcurveto{\pgfqpoint{6.890225in}{3.698562in}}{\pgfqpoint{6.900824in}{3.702952in}}{\pgfqpoint{6.908638in}{3.710766in}}%
\pgfpathcurveto{\pgfqpoint{6.916452in}{3.718579in}}{\pgfqpoint{6.920842in}{3.729178in}}{\pgfqpoint{6.920842in}{3.740228in}}%
\pgfpathcurveto{\pgfqpoint{6.920842in}{3.751278in}}{\pgfqpoint{6.916452in}{3.761877in}}{\pgfqpoint{6.908638in}{3.769691in}}%
\pgfpathcurveto{\pgfqpoint{6.900824in}{3.777505in}}{\pgfqpoint{6.890225in}{3.781895in}}{\pgfqpoint{6.879175in}{3.781895in}}%
\pgfpathcurveto{\pgfqpoint{6.868125in}{3.781895in}}{\pgfqpoint{6.857526in}{3.777505in}}{\pgfqpoint{6.849712in}{3.769691in}}%
\pgfpathcurveto{\pgfqpoint{6.841899in}{3.761877in}}{\pgfqpoint{6.837509in}{3.751278in}}{\pgfqpoint{6.837509in}{3.740228in}}%
\pgfpathcurveto{\pgfqpoint{6.837509in}{3.729178in}}{\pgfqpoint{6.841899in}{3.718579in}}{\pgfqpoint{6.849712in}{3.710766in}}%
\pgfpathcurveto{\pgfqpoint{6.857526in}{3.702952in}}{\pgfqpoint{6.868125in}{3.698562in}}{\pgfqpoint{6.879175in}{3.698562in}}%
\pgfpathclose%
\pgfusepath{stroke,fill}%
\end{pgfscope}%
\begin{pgfscope}%
\pgfpathrectangle{\pgfqpoint{0.481978in}{0.331635in}}{\pgfqpoint{9.300000in}{7.700000in}}%
\pgfusepath{clip}%
\pgfsetbuttcap%
\pgfsetroundjoin%
\definecolor{currentfill}{rgb}{1.000000,0.705882,0.509804}%
\pgfsetfillcolor{currentfill}%
\pgfsetlinewidth{0.481800pt}%
\definecolor{currentstroke}{rgb}{1.000000,1.000000,1.000000}%
\pgfsetstrokecolor{currentstroke}%
\pgfsetdash{}{0pt}%
\pgfpathmoveto{\pgfqpoint{4.800335in}{3.680630in}}%
\pgfpathcurveto{\pgfqpoint{4.811385in}{3.680630in}}{\pgfqpoint{4.821984in}{3.685020in}}{\pgfqpoint{4.829798in}{3.692834in}}%
\pgfpathcurveto{\pgfqpoint{4.837611in}{3.700647in}}{\pgfqpoint{4.842002in}{3.711246in}}{\pgfqpoint{4.842002in}{3.722296in}}%
\pgfpathcurveto{\pgfqpoint{4.842002in}{3.733346in}}{\pgfqpoint{4.837611in}{3.743945in}}{\pgfqpoint{4.829798in}{3.751759in}}%
\pgfpathcurveto{\pgfqpoint{4.821984in}{3.759573in}}{\pgfqpoint{4.811385in}{3.763963in}}{\pgfqpoint{4.800335in}{3.763963in}}%
\pgfpathcurveto{\pgfqpoint{4.789285in}{3.763963in}}{\pgfqpoint{4.778686in}{3.759573in}}{\pgfqpoint{4.770872in}{3.751759in}}%
\pgfpathcurveto{\pgfqpoint{4.763059in}{3.743945in}}{\pgfqpoint{4.758668in}{3.733346in}}{\pgfqpoint{4.758668in}{3.722296in}}%
\pgfpathcurveto{\pgfqpoint{4.758668in}{3.711246in}}{\pgfqpoint{4.763059in}{3.700647in}}{\pgfqpoint{4.770872in}{3.692834in}}%
\pgfpathcurveto{\pgfqpoint{4.778686in}{3.685020in}}{\pgfqpoint{4.789285in}{3.680630in}}{\pgfqpoint{4.800335in}{3.680630in}}%
\pgfpathclose%
\pgfusepath{stroke,fill}%
\end{pgfscope}%
\begin{pgfscope}%
\pgfpathrectangle{\pgfqpoint{0.481978in}{0.331635in}}{\pgfqpoint{9.300000in}{7.700000in}}%
\pgfusepath{clip}%
\pgfsetbuttcap%
\pgfsetroundjoin%
\definecolor{currentfill}{rgb}{1.000000,0.705882,0.509804}%
\pgfsetfillcolor{currentfill}%
\pgfsetlinewidth{0.481800pt}%
\definecolor{currentstroke}{rgb}{1.000000,1.000000,1.000000}%
\pgfsetstrokecolor{currentstroke}%
\pgfsetdash{}{0pt}%
\pgfpathmoveto{\pgfqpoint{4.757910in}{5.288221in}}%
\pgfpathcurveto{\pgfqpoint{4.768960in}{5.288221in}}{\pgfqpoint{4.779559in}{5.292611in}}{\pgfqpoint{4.787373in}{5.300424in}}%
\pgfpathcurveto{\pgfqpoint{4.795186in}{5.308238in}}{\pgfqpoint{4.799577in}{5.318837in}}{\pgfqpoint{4.799577in}{5.329887in}}%
\pgfpathcurveto{\pgfqpoint{4.799577in}{5.340937in}}{\pgfqpoint{4.795186in}{5.351536in}}{\pgfqpoint{4.787373in}{5.359350in}}%
\pgfpathcurveto{\pgfqpoint{4.779559in}{5.367164in}}{\pgfqpoint{4.768960in}{5.371554in}}{\pgfqpoint{4.757910in}{5.371554in}}%
\pgfpathcurveto{\pgfqpoint{4.746860in}{5.371554in}}{\pgfqpoint{4.736261in}{5.367164in}}{\pgfqpoint{4.728447in}{5.359350in}}%
\pgfpathcurveto{\pgfqpoint{4.720633in}{5.351536in}}{\pgfqpoint{4.716243in}{5.340937in}}{\pgfqpoint{4.716243in}{5.329887in}}%
\pgfpathcurveto{\pgfqpoint{4.716243in}{5.318837in}}{\pgfqpoint{4.720633in}{5.308238in}}{\pgfqpoint{4.728447in}{5.300424in}}%
\pgfpathcurveto{\pgfqpoint{4.736261in}{5.292611in}}{\pgfqpoint{4.746860in}{5.288221in}}{\pgfqpoint{4.757910in}{5.288221in}}%
\pgfpathclose%
\pgfusepath{stroke,fill}%
\end{pgfscope}%
\begin{pgfscope}%
\pgfpathrectangle{\pgfqpoint{0.481978in}{0.331635in}}{\pgfqpoint{9.300000in}{7.700000in}}%
\pgfusepath{clip}%
\pgfsetbuttcap%
\pgfsetroundjoin%
\definecolor{currentfill}{rgb}{1.000000,0.705882,0.509804}%
\pgfsetfillcolor{currentfill}%
\pgfsetlinewidth{0.481800pt}%
\definecolor{currentstroke}{rgb}{1.000000,1.000000,1.000000}%
\pgfsetstrokecolor{currentstroke}%
\pgfsetdash{}{0pt}%
\pgfpathmoveto{\pgfqpoint{4.104994in}{5.646590in}}%
\pgfpathcurveto{\pgfqpoint{4.116044in}{5.646590in}}{\pgfqpoint{4.126643in}{5.650980in}}{\pgfqpoint{4.134457in}{5.658794in}}%
\pgfpathcurveto{\pgfqpoint{4.142271in}{5.666607in}}{\pgfqpoint{4.146661in}{5.677206in}}{\pgfqpoint{4.146661in}{5.688256in}}%
\pgfpathcurveto{\pgfqpoint{4.146661in}{5.699306in}}{\pgfqpoint{4.142271in}{5.709906in}}{\pgfqpoint{4.134457in}{5.717719in}}%
\pgfpathcurveto{\pgfqpoint{4.126643in}{5.725533in}}{\pgfqpoint{4.116044in}{5.729923in}}{\pgfqpoint{4.104994in}{5.729923in}}%
\pgfpathcurveto{\pgfqpoint{4.093944in}{5.729923in}}{\pgfqpoint{4.083345in}{5.725533in}}{\pgfqpoint{4.075531in}{5.717719in}}%
\pgfpathcurveto{\pgfqpoint{4.067718in}{5.709906in}}{\pgfqpoint{4.063328in}{5.699306in}}{\pgfqpoint{4.063328in}{5.688256in}}%
\pgfpathcurveto{\pgfqpoint{4.063328in}{5.677206in}}{\pgfqpoint{4.067718in}{5.666607in}}{\pgfqpoint{4.075531in}{5.658794in}}%
\pgfpathcurveto{\pgfqpoint{4.083345in}{5.650980in}}{\pgfqpoint{4.093944in}{5.646590in}}{\pgfqpoint{4.104994in}{5.646590in}}%
\pgfpathclose%
\pgfusepath{stroke,fill}%
\end{pgfscope}%
\begin{pgfscope}%
\pgfpathrectangle{\pgfqpoint{0.481978in}{0.331635in}}{\pgfqpoint{9.300000in}{7.700000in}}%
\pgfusepath{clip}%
\pgfsetbuttcap%
\pgfsetroundjoin%
\definecolor{currentfill}{rgb}{1.000000,0.705882,0.509804}%
\pgfsetfillcolor{currentfill}%
\pgfsetlinewidth{0.481800pt}%
\definecolor{currentstroke}{rgb}{1.000000,1.000000,1.000000}%
\pgfsetstrokecolor{currentstroke}%
\pgfsetdash{}{0pt}%
\pgfpathmoveto{\pgfqpoint{1.344247in}{4.328959in}}%
\pgfpathcurveto{\pgfqpoint{1.355297in}{4.328959in}}{\pgfqpoint{1.365896in}{4.333349in}}{\pgfqpoint{1.373710in}{4.341163in}}%
\pgfpathcurveto{\pgfqpoint{1.381524in}{4.348977in}}{\pgfqpoint{1.385914in}{4.359576in}}{\pgfqpoint{1.385914in}{4.370626in}}%
\pgfpathcurveto{\pgfqpoint{1.385914in}{4.381676in}}{\pgfqpoint{1.381524in}{4.392275in}}{\pgfqpoint{1.373710in}{4.400089in}}%
\pgfpathcurveto{\pgfqpoint{1.365896in}{4.407902in}}{\pgfqpoint{1.355297in}{4.412292in}}{\pgfqpoint{1.344247in}{4.412292in}}%
\pgfpathcurveto{\pgfqpoint{1.333197in}{4.412292in}}{\pgfqpoint{1.322598in}{4.407902in}}{\pgfqpoint{1.314784in}{4.400089in}}%
\pgfpathcurveto{\pgfqpoint{1.306971in}{4.392275in}}{\pgfqpoint{1.302580in}{4.381676in}}{\pgfqpoint{1.302580in}{4.370626in}}%
\pgfpathcurveto{\pgfqpoint{1.302580in}{4.359576in}}{\pgfqpoint{1.306971in}{4.348977in}}{\pgfqpoint{1.314784in}{4.341163in}}%
\pgfpathcurveto{\pgfqpoint{1.322598in}{4.333349in}}{\pgfqpoint{1.333197in}{4.328959in}}{\pgfqpoint{1.344247in}{4.328959in}}%
\pgfpathclose%
\pgfusepath{stroke,fill}%
\end{pgfscope}%
\begin{pgfscope}%
\pgfpathrectangle{\pgfqpoint{0.481978in}{0.331635in}}{\pgfqpoint{9.300000in}{7.700000in}}%
\pgfusepath{clip}%
\pgfsetbuttcap%
\pgfsetroundjoin%
\definecolor{currentfill}{rgb}{1.000000,0.705882,0.509804}%
\pgfsetfillcolor{currentfill}%
\pgfsetlinewidth{0.481800pt}%
\definecolor{currentstroke}{rgb}{1.000000,1.000000,1.000000}%
\pgfsetstrokecolor{currentstroke}%
\pgfsetdash{}{0pt}%
\pgfpathmoveto{\pgfqpoint{3.248962in}{4.199832in}}%
\pgfpathcurveto{\pgfqpoint{3.260013in}{4.199832in}}{\pgfqpoint{3.270612in}{4.204223in}}{\pgfqpoint{3.278425in}{4.212036in}}%
\pgfpathcurveto{\pgfqpoint{3.286239in}{4.219850in}}{\pgfqpoint{3.290629in}{4.230449in}}{\pgfqpoint{3.290629in}{4.241499in}}%
\pgfpathcurveto{\pgfqpoint{3.290629in}{4.252549in}}{\pgfqpoint{3.286239in}{4.263148in}}{\pgfqpoint{3.278425in}{4.270962in}}%
\pgfpathcurveto{\pgfqpoint{3.270612in}{4.278775in}}{\pgfqpoint{3.260013in}{4.283166in}}{\pgfqpoint{3.248962in}{4.283166in}}%
\pgfpathcurveto{\pgfqpoint{3.237912in}{4.283166in}}{\pgfqpoint{3.227313in}{4.278775in}}{\pgfqpoint{3.219500in}{4.270962in}}%
\pgfpathcurveto{\pgfqpoint{3.211686in}{4.263148in}}{\pgfqpoint{3.207296in}{4.252549in}}{\pgfqpoint{3.207296in}{4.241499in}}%
\pgfpathcurveto{\pgfqpoint{3.207296in}{4.230449in}}{\pgfqpoint{3.211686in}{4.219850in}}{\pgfqpoint{3.219500in}{4.212036in}}%
\pgfpathcurveto{\pgfqpoint{3.227313in}{4.204223in}}{\pgfqpoint{3.237912in}{4.199832in}}{\pgfqpoint{3.248962in}{4.199832in}}%
\pgfpathclose%
\pgfusepath{stroke,fill}%
\end{pgfscope}%
\begin{pgfscope}%
\pgfpathrectangle{\pgfqpoint{0.481978in}{0.331635in}}{\pgfqpoint{9.300000in}{7.700000in}}%
\pgfusepath{clip}%
\pgfsetbuttcap%
\pgfsetroundjoin%
\definecolor{currentfill}{rgb}{0.631373,0.788235,0.956863}%
\pgfsetfillcolor{currentfill}%
\pgfsetlinewidth{1.003750pt}%
\definecolor{currentstroke}{rgb}{0.631373,0.788235,0.956863}%
\pgfsetstrokecolor{currentstroke}%
\pgfsetdash{}{0pt}%
\pgfsys@defobject{currentmarker}{\pgfqpoint{-0.041667in}{-0.041667in}}{\pgfqpoint{0.041667in}{0.041667in}}{%
\pgfpathmoveto{\pgfqpoint{0.000000in}{-0.041667in}}%
\pgfpathcurveto{\pgfqpoint{0.011050in}{-0.041667in}}{\pgfqpoint{0.021649in}{-0.037276in}}{\pgfqpoint{0.029463in}{-0.029463in}}%
\pgfpathcurveto{\pgfqpoint{0.037276in}{-0.021649in}}{\pgfqpoint{0.041667in}{-0.011050in}}{\pgfqpoint{0.041667in}{0.000000in}}%
\pgfpathcurveto{\pgfqpoint{0.041667in}{0.011050in}}{\pgfqpoint{0.037276in}{0.021649in}}{\pgfqpoint{0.029463in}{0.029463in}}%
\pgfpathcurveto{\pgfqpoint{0.021649in}{0.037276in}}{\pgfqpoint{0.011050in}{0.041667in}}{\pgfqpoint{0.000000in}{0.041667in}}%
\pgfpathcurveto{\pgfqpoint{-0.011050in}{0.041667in}}{\pgfqpoint{-0.021649in}{0.037276in}}{\pgfqpoint{-0.029463in}{0.029463in}}%
\pgfpathcurveto{\pgfqpoint{-0.037276in}{0.021649in}}{\pgfqpoint{-0.041667in}{0.011050in}}{\pgfqpoint{-0.041667in}{0.000000in}}%
\pgfpathcurveto{\pgfqpoint{-0.041667in}{-0.011050in}}{\pgfqpoint{-0.037276in}{-0.021649in}}{\pgfqpoint{-0.029463in}{-0.029463in}}%
\pgfpathcurveto{\pgfqpoint{-0.021649in}{-0.037276in}}{\pgfqpoint{-0.011050in}{-0.041667in}}{\pgfqpoint{0.000000in}{-0.041667in}}%
\pgfpathclose%
\pgfusepath{stroke,fill}%
}%
\end{pgfscope}%
\begin{pgfscope}%
\pgfpathrectangle{\pgfqpoint{0.481978in}{0.331635in}}{\pgfqpoint{9.300000in}{7.700000in}}%
\pgfusepath{clip}%
\pgfsetbuttcap%
\pgfsetroundjoin%
\definecolor{currentfill}{rgb}{1.000000,0.705882,0.509804}%
\pgfsetfillcolor{currentfill}%
\pgfsetlinewidth{1.003750pt}%
\definecolor{currentstroke}{rgb}{1.000000,0.705882,0.509804}%
\pgfsetstrokecolor{currentstroke}%
\pgfsetdash{}{0pt}%
\pgfsys@defobject{currentmarker}{\pgfqpoint{-0.041667in}{-0.041667in}}{\pgfqpoint{0.041667in}{0.041667in}}{%
\pgfpathmoveto{\pgfqpoint{0.000000in}{-0.041667in}}%
\pgfpathcurveto{\pgfqpoint{0.011050in}{-0.041667in}}{\pgfqpoint{0.021649in}{-0.037276in}}{\pgfqpoint{0.029463in}{-0.029463in}}%
\pgfpathcurveto{\pgfqpoint{0.037276in}{-0.021649in}}{\pgfqpoint{0.041667in}{-0.011050in}}{\pgfqpoint{0.041667in}{0.000000in}}%
\pgfpathcurveto{\pgfqpoint{0.041667in}{0.011050in}}{\pgfqpoint{0.037276in}{0.021649in}}{\pgfqpoint{0.029463in}{0.029463in}}%
\pgfpathcurveto{\pgfqpoint{0.021649in}{0.037276in}}{\pgfqpoint{0.011050in}{0.041667in}}{\pgfqpoint{0.000000in}{0.041667in}}%
\pgfpathcurveto{\pgfqpoint{-0.011050in}{0.041667in}}{\pgfqpoint{-0.021649in}{0.037276in}}{\pgfqpoint{-0.029463in}{0.029463in}}%
\pgfpathcurveto{\pgfqpoint{-0.037276in}{0.021649in}}{\pgfqpoint{-0.041667in}{0.011050in}}{\pgfqpoint{-0.041667in}{0.000000in}}%
\pgfpathcurveto{\pgfqpoint{-0.041667in}{-0.011050in}}{\pgfqpoint{-0.037276in}{-0.021649in}}{\pgfqpoint{-0.029463in}{-0.029463in}}%
\pgfpathcurveto{\pgfqpoint{-0.021649in}{-0.037276in}}{\pgfqpoint{-0.011050in}{-0.041667in}}{\pgfqpoint{0.000000in}{-0.041667in}}%
\pgfpathclose%
\pgfusepath{stroke,fill}%
}%
\end{pgfscope}%
\begin{pgfscope}%
\pgfsetbuttcap%
\pgfsetroundjoin%
\definecolor{currentfill}{rgb}{0.000000,0.000000,0.000000}%
\pgfsetfillcolor{currentfill}%
\pgfsetlinewidth{0.803000pt}%
\definecolor{currentstroke}{rgb}{0.000000,0.000000,0.000000}%
\pgfsetstrokecolor{currentstroke}%
\pgfsetdash{}{0pt}%
\pgfsys@defobject{currentmarker}{\pgfqpoint{0.000000in}{-0.048611in}}{\pgfqpoint{0.000000in}{0.000000in}}{%
\pgfpathmoveto{\pgfqpoint{0.000000in}{0.000000in}}%
\pgfpathlineto{\pgfqpoint{0.000000in}{-0.048611in}}%
\pgfusepath{stroke,fill}%
}%
\begin{pgfscope}%
\pgfsys@transformshift{1.232596in}{0.331635in}%
\pgfsys@useobject{currentmarker}{}%
\end{pgfscope}%
\end{pgfscope}%
\begin{pgfscope}%
\definecolor{textcolor}{rgb}{0.000000,0.000000,0.000000}%
\pgfsetstrokecolor{textcolor}%
\pgfsetfillcolor{textcolor}%
\pgftext[x=1.232596in,y=0.234413in,,top]{\color{textcolor}\sffamily\fontsize{10.000000}{12.000000}\selectfont \ensuremath{-}100}%
\end{pgfscope}%
\begin{pgfscope}%
\pgfsetbuttcap%
\pgfsetroundjoin%
\definecolor{currentfill}{rgb}{0.000000,0.000000,0.000000}%
\pgfsetfillcolor{currentfill}%
\pgfsetlinewidth{0.803000pt}%
\definecolor{currentstroke}{rgb}{0.000000,0.000000,0.000000}%
\pgfsetstrokecolor{currentstroke}%
\pgfsetdash{}{0pt}%
\pgfsys@defobject{currentmarker}{\pgfqpoint{0.000000in}{-0.048611in}}{\pgfqpoint{0.000000in}{0.000000in}}{%
\pgfpathmoveto{\pgfqpoint{0.000000in}{0.000000in}}%
\pgfpathlineto{\pgfqpoint{0.000000in}{-0.048611in}}%
\pgfusepath{stroke,fill}%
}%
\begin{pgfscope}%
\pgfsys@transformshift{2.286320in}{0.331635in}%
\pgfsys@useobject{currentmarker}{}%
\end{pgfscope}%
\end{pgfscope}%
\begin{pgfscope}%
\definecolor{textcolor}{rgb}{0.000000,0.000000,0.000000}%
\pgfsetstrokecolor{textcolor}%
\pgfsetfillcolor{textcolor}%
\pgftext[x=2.286320in,y=0.234413in,,top]{\color{textcolor}\sffamily\fontsize{10.000000}{12.000000}\selectfont \ensuremath{-}75}%
\end{pgfscope}%
\begin{pgfscope}%
\pgfsetbuttcap%
\pgfsetroundjoin%
\definecolor{currentfill}{rgb}{0.000000,0.000000,0.000000}%
\pgfsetfillcolor{currentfill}%
\pgfsetlinewidth{0.803000pt}%
\definecolor{currentstroke}{rgb}{0.000000,0.000000,0.000000}%
\pgfsetstrokecolor{currentstroke}%
\pgfsetdash{}{0pt}%
\pgfsys@defobject{currentmarker}{\pgfqpoint{0.000000in}{-0.048611in}}{\pgfqpoint{0.000000in}{0.000000in}}{%
\pgfpathmoveto{\pgfqpoint{0.000000in}{0.000000in}}%
\pgfpathlineto{\pgfqpoint{0.000000in}{-0.048611in}}%
\pgfusepath{stroke,fill}%
}%
\begin{pgfscope}%
\pgfsys@transformshift{3.340044in}{0.331635in}%
\pgfsys@useobject{currentmarker}{}%
\end{pgfscope}%
\end{pgfscope}%
\begin{pgfscope}%
\definecolor{textcolor}{rgb}{0.000000,0.000000,0.000000}%
\pgfsetstrokecolor{textcolor}%
\pgfsetfillcolor{textcolor}%
\pgftext[x=3.340044in,y=0.234413in,,top]{\color{textcolor}\sffamily\fontsize{10.000000}{12.000000}\selectfont \ensuremath{-}50}%
\end{pgfscope}%
\begin{pgfscope}%
\pgfsetbuttcap%
\pgfsetroundjoin%
\definecolor{currentfill}{rgb}{0.000000,0.000000,0.000000}%
\pgfsetfillcolor{currentfill}%
\pgfsetlinewidth{0.803000pt}%
\definecolor{currentstroke}{rgb}{0.000000,0.000000,0.000000}%
\pgfsetstrokecolor{currentstroke}%
\pgfsetdash{}{0pt}%
\pgfsys@defobject{currentmarker}{\pgfqpoint{0.000000in}{-0.048611in}}{\pgfqpoint{0.000000in}{0.000000in}}{%
\pgfpathmoveto{\pgfqpoint{0.000000in}{0.000000in}}%
\pgfpathlineto{\pgfqpoint{0.000000in}{-0.048611in}}%
\pgfusepath{stroke,fill}%
}%
\begin{pgfscope}%
\pgfsys@transformshift{4.393768in}{0.331635in}%
\pgfsys@useobject{currentmarker}{}%
\end{pgfscope}%
\end{pgfscope}%
\begin{pgfscope}%
\definecolor{textcolor}{rgb}{0.000000,0.000000,0.000000}%
\pgfsetstrokecolor{textcolor}%
\pgfsetfillcolor{textcolor}%
\pgftext[x=4.393768in,y=0.234413in,,top]{\color{textcolor}\sffamily\fontsize{10.000000}{12.000000}\selectfont \ensuremath{-}25}%
\end{pgfscope}%
\begin{pgfscope}%
\pgfsetbuttcap%
\pgfsetroundjoin%
\definecolor{currentfill}{rgb}{0.000000,0.000000,0.000000}%
\pgfsetfillcolor{currentfill}%
\pgfsetlinewidth{0.803000pt}%
\definecolor{currentstroke}{rgb}{0.000000,0.000000,0.000000}%
\pgfsetstrokecolor{currentstroke}%
\pgfsetdash{}{0pt}%
\pgfsys@defobject{currentmarker}{\pgfqpoint{0.000000in}{-0.048611in}}{\pgfqpoint{0.000000in}{0.000000in}}{%
\pgfpathmoveto{\pgfqpoint{0.000000in}{0.000000in}}%
\pgfpathlineto{\pgfqpoint{0.000000in}{-0.048611in}}%
\pgfusepath{stroke,fill}%
}%
\begin{pgfscope}%
\pgfsys@transformshift{5.447492in}{0.331635in}%
\pgfsys@useobject{currentmarker}{}%
\end{pgfscope}%
\end{pgfscope}%
\begin{pgfscope}%
\definecolor{textcolor}{rgb}{0.000000,0.000000,0.000000}%
\pgfsetstrokecolor{textcolor}%
\pgfsetfillcolor{textcolor}%
\pgftext[x=5.447492in,y=0.234413in,,top]{\color{textcolor}\sffamily\fontsize{10.000000}{12.000000}\selectfont 0}%
\end{pgfscope}%
\begin{pgfscope}%
\pgfsetbuttcap%
\pgfsetroundjoin%
\definecolor{currentfill}{rgb}{0.000000,0.000000,0.000000}%
\pgfsetfillcolor{currentfill}%
\pgfsetlinewidth{0.803000pt}%
\definecolor{currentstroke}{rgb}{0.000000,0.000000,0.000000}%
\pgfsetstrokecolor{currentstroke}%
\pgfsetdash{}{0pt}%
\pgfsys@defobject{currentmarker}{\pgfqpoint{0.000000in}{-0.048611in}}{\pgfqpoint{0.000000in}{0.000000in}}{%
\pgfpathmoveto{\pgfqpoint{0.000000in}{0.000000in}}%
\pgfpathlineto{\pgfqpoint{0.000000in}{-0.048611in}}%
\pgfusepath{stroke,fill}%
}%
\begin{pgfscope}%
\pgfsys@transformshift{6.501216in}{0.331635in}%
\pgfsys@useobject{currentmarker}{}%
\end{pgfscope}%
\end{pgfscope}%
\begin{pgfscope}%
\definecolor{textcolor}{rgb}{0.000000,0.000000,0.000000}%
\pgfsetstrokecolor{textcolor}%
\pgfsetfillcolor{textcolor}%
\pgftext[x=6.501216in,y=0.234413in,,top]{\color{textcolor}\sffamily\fontsize{10.000000}{12.000000}\selectfont 25}%
\end{pgfscope}%
\begin{pgfscope}%
\pgfsetbuttcap%
\pgfsetroundjoin%
\definecolor{currentfill}{rgb}{0.000000,0.000000,0.000000}%
\pgfsetfillcolor{currentfill}%
\pgfsetlinewidth{0.803000pt}%
\definecolor{currentstroke}{rgb}{0.000000,0.000000,0.000000}%
\pgfsetstrokecolor{currentstroke}%
\pgfsetdash{}{0pt}%
\pgfsys@defobject{currentmarker}{\pgfqpoint{0.000000in}{-0.048611in}}{\pgfqpoint{0.000000in}{0.000000in}}{%
\pgfpathmoveto{\pgfqpoint{0.000000in}{0.000000in}}%
\pgfpathlineto{\pgfqpoint{0.000000in}{-0.048611in}}%
\pgfusepath{stroke,fill}%
}%
\begin{pgfscope}%
\pgfsys@transformshift{7.554940in}{0.331635in}%
\pgfsys@useobject{currentmarker}{}%
\end{pgfscope}%
\end{pgfscope}%
\begin{pgfscope}%
\definecolor{textcolor}{rgb}{0.000000,0.000000,0.000000}%
\pgfsetstrokecolor{textcolor}%
\pgfsetfillcolor{textcolor}%
\pgftext[x=7.554940in,y=0.234413in,,top]{\color{textcolor}\sffamily\fontsize{10.000000}{12.000000}\selectfont 50}%
\end{pgfscope}%
\begin{pgfscope}%
\pgfsetbuttcap%
\pgfsetroundjoin%
\definecolor{currentfill}{rgb}{0.000000,0.000000,0.000000}%
\pgfsetfillcolor{currentfill}%
\pgfsetlinewidth{0.803000pt}%
\definecolor{currentstroke}{rgb}{0.000000,0.000000,0.000000}%
\pgfsetstrokecolor{currentstroke}%
\pgfsetdash{}{0pt}%
\pgfsys@defobject{currentmarker}{\pgfqpoint{0.000000in}{-0.048611in}}{\pgfqpoint{0.000000in}{0.000000in}}{%
\pgfpathmoveto{\pgfqpoint{0.000000in}{0.000000in}}%
\pgfpathlineto{\pgfqpoint{0.000000in}{-0.048611in}}%
\pgfusepath{stroke,fill}%
}%
\begin{pgfscope}%
\pgfsys@transformshift{8.608664in}{0.331635in}%
\pgfsys@useobject{currentmarker}{}%
\end{pgfscope}%
\end{pgfscope}%
\begin{pgfscope}%
\definecolor{textcolor}{rgb}{0.000000,0.000000,0.000000}%
\pgfsetstrokecolor{textcolor}%
\pgfsetfillcolor{textcolor}%
\pgftext[x=8.608664in,y=0.234413in,,top]{\color{textcolor}\sffamily\fontsize{10.000000}{12.000000}\selectfont 75}%
\end{pgfscope}%
\begin{pgfscope}%
\pgfsetbuttcap%
\pgfsetroundjoin%
\definecolor{currentfill}{rgb}{0.000000,0.000000,0.000000}%
\pgfsetfillcolor{currentfill}%
\pgfsetlinewidth{0.803000pt}%
\definecolor{currentstroke}{rgb}{0.000000,0.000000,0.000000}%
\pgfsetstrokecolor{currentstroke}%
\pgfsetdash{}{0pt}%
\pgfsys@defobject{currentmarker}{\pgfqpoint{0.000000in}{-0.048611in}}{\pgfqpoint{0.000000in}{0.000000in}}{%
\pgfpathmoveto{\pgfqpoint{0.000000in}{0.000000in}}%
\pgfpathlineto{\pgfqpoint{0.000000in}{-0.048611in}}%
\pgfusepath{stroke,fill}%
}%
\begin{pgfscope}%
\pgfsys@transformshift{9.662388in}{0.331635in}%
\pgfsys@useobject{currentmarker}{}%
\end{pgfscope}%
\end{pgfscope}%
\begin{pgfscope}%
\definecolor{textcolor}{rgb}{0.000000,0.000000,0.000000}%
\pgfsetstrokecolor{textcolor}%
\pgfsetfillcolor{textcolor}%
\pgftext[x=9.662388in,y=0.234413in,,top]{\color{textcolor}\sffamily\fontsize{10.000000}{12.000000}\selectfont 100}%
\end{pgfscope}%
\begin{pgfscope}%
\pgfsetbuttcap%
\pgfsetroundjoin%
\definecolor{currentfill}{rgb}{0.000000,0.000000,0.000000}%
\pgfsetfillcolor{currentfill}%
\pgfsetlinewidth{0.803000pt}%
\definecolor{currentstroke}{rgb}{0.000000,0.000000,0.000000}%
\pgfsetstrokecolor{currentstroke}%
\pgfsetdash{}{0pt}%
\pgfsys@defobject{currentmarker}{\pgfqpoint{-0.048611in}{0.000000in}}{\pgfqpoint{-0.000000in}{0.000000in}}{%
\pgfpathmoveto{\pgfqpoint{-0.000000in}{0.000000in}}%
\pgfpathlineto{\pgfqpoint{-0.048611in}{0.000000in}}%
\pgfusepath{stroke,fill}%
}%
\begin{pgfscope}%
\pgfsys@transformshift{0.481978in}{0.363094in}%
\pgfsys@useobject{currentmarker}{}%
\end{pgfscope}%
\end{pgfscope}%
\begin{pgfscope}%
\definecolor{textcolor}{rgb}{0.000000,0.000000,0.000000}%
\pgfsetstrokecolor{textcolor}%
\pgfsetfillcolor{textcolor}%
\pgftext[x=0.100000in, y=0.310333in, left, base]{\color{textcolor}\sffamily\fontsize{10.000000}{12.000000}\selectfont \ensuremath{-}80}%
\end{pgfscope}%
\begin{pgfscope}%
\pgfsetbuttcap%
\pgfsetroundjoin%
\definecolor{currentfill}{rgb}{0.000000,0.000000,0.000000}%
\pgfsetfillcolor{currentfill}%
\pgfsetlinewidth{0.803000pt}%
\definecolor{currentstroke}{rgb}{0.000000,0.000000,0.000000}%
\pgfsetstrokecolor{currentstroke}%
\pgfsetdash{}{0pt}%
\pgfsys@defobject{currentmarker}{\pgfqpoint{-0.048611in}{0.000000in}}{\pgfqpoint{-0.000000in}{0.000000in}}{%
\pgfpathmoveto{\pgfqpoint{-0.000000in}{0.000000in}}%
\pgfpathlineto{\pgfqpoint{-0.048611in}{0.000000in}}%
\pgfusepath{stroke,fill}%
}%
\begin{pgfscope}%
\pgfsys@transformshift{0.481978in}{1.398454in}%
\pgfsys@useobject{currentmarker}{}%
\end{pgfscope}%
\end{pgfscope}%
\begin{pgfscope}%
\definecolor{textcolor}{rgb}{0.000000,0.000000,0.000000}%
\pgfsetstrokecolor{textcolor}%
\pgfsetfillcolor{textcolor}%
\pgftext[x=0.100000in, y=1.345693in, left, base]{\color{textcolor}\sffamily\fontsize{10.000000}{12.000000}\selectfont \ensuremath{-}60}%
\end{pgfscope}%
\begin{pgfscope}%
\pgfsetbuttcap%
\pgfsetroundjoin%
\definecolor{currentfill}{rgb}{0.000000,0.000000,0.000000}%
\pgfsetfillcolor{currentfill}%
\pgfsetlinewidth{0.803000pt}%
\definecolor{currentstroke}{rgb}{0.000000,0.000000,0.000000}%
\pgfsetstrokecolor{currentstroke}%
\pgfsetdash{}{0pt}%
\pgfsys@defobject{currentmarker}{\pgfqpoint{-0.048611in}{0.000000in}}{\pgfqpoint{-0.000000in}{0.000000in}}{%
\pgfpathmoveto{\pgfqpoint{-0.000000in}{0.000000in}}%
\pgfpathlineto{\pgfqpoint{-0.048611in}{0.000000in}}%
\pgfusepath{stroke,fill}%
}%
\begin{pgfscope}%
\pgfsys@transformshift{0.481978in}{2.433815in}%
\pgfsys@useobject{currentmarker}{}%
\end{pgfscope}%
\end{pgfscope}%
\begin{pgfscope}%
\definecolor{textcolor}{rgb}{0.000000,0.000000,0.000000}%
\pgfsetstrokecolor{textcolor}%
\pgfsetfillcolor{textcolor}%
\pgftext[x=0.100000in, y=2.381053in, left, base]{\color{textcolor}\sffamily\fontsize{10.000000}{12.000000}\selectfont \ensuremath{-}40}%
\end{pgfscope}%
\begin{pgfscope}%
\pgfsetbuttcap%
\pgfsetroundjoin%
\definecolor{currentfill}{rgb}{0.000000,0.000000,0.000000}%
\pgfsetfillcolor{currentfill}%
\pgfsetlinewidth{0.803000pt}%
\definecolor{currentstroke}{rgb}{0.000000,0.000000,0.000000}%
\pgfsetstrokecolor{currentstroke}%
\pgfsetdash{}{0pt}%
\pgfsys@defobject{currentmarker}{\pgfqpoint{-0.048611in}{0.000000in}}{\pgfqpoint{-0.000000in}{0.000000in}}{%
\pgfpathmoveto{\pgfqpoint{-0.000000in}{0.000000in}}%
\pgfpathlineto{\pgfqpoint{-0.048611in}{0.000000in}}%
\pgfusepath{stroke,fill}%
}%
\begin{pgfscope}%
\pgfsys@transformshift{0.481978in}{3.469175in}%
\pgfsys@useobject{currentmarker}{}%
\end{pgfscope}%
\end{pgfscope}%
\begin{pgfscope}%
\definecolor{textcolor}{rgb}{0.000000,0.000000,0.000000}%
\pgfsetstrokecolor{textcolor}%
\pgfsetfillcolor{textcolor}%
\pgftext[x=0.100000in, y=3.416413in, left, base]{\color{textcolor}\sffamily\fontsize{10.000000}{12.000000}\selectfont \ensuremath{-}20}%
\end{pgfscope}%
\begin{pgfscope}%
\pgfsetbuttcap%
\pgfsetroundjoin%
\definecolor{currentfill}{rgb}{0.000000,0.000000,0.000000}%
\pgfsetfillcolor{currentfill}%
\pgfsetlinewidth{0.803000pt}%
\definecolor{currentstroke}{rgb}{0.000000,0.000000,0.000000}%
\pgfsetstrokecolor{currentstroke}%
\pgfsetdash{}{0pt}%
\pgfsys@defobject{currentmarker}{\pgfqpoint{-0.048611in}{0.000000in}}{\pgfqpoint{-0.000000in}{0.000000in}}{%
\pgfpathmoveto{\pgfqpoint{-0.000000in}{0.000000in}}%
\pgfpathlineto{\pgfqpoint{-0.048611in}{0.000000in}}%
\pgfusepath{stroke,fill}%
}%
\begin{pgfscope}%
\pgfsys@transformshift{0.481978in}{4.504535in}%
\pgfsys@useobject{currentmarker}{}%
\end{pgfscope}%
\end{pgfscope}%
\begin{pgfscope}%
\definecolor{textcolor}{rgb}{0.000000,0.000000,0.000000}%
\pgfsetstrokecolor{textcolor}%
\pgfsetfillcolor{textcolor}%
\pgftext[x=0.296390in, y=4.451774in, left, base]{\color{textcolor}\sffamily\fontsize{10.000000}{12.000000}\selectfont 0}%
\end{pgfscope}%
\begin{pgfscope}%
\pgfsetbuttcap%
\pgfsetroundjoin%
\definecolor{currentfill}{rgb}{0.000000,0.000000,0.000000}%
\pgfsetfillcolor{currentfill}%
\pgfsetlinewidth{0.803000pt}%
\definecolor{currentstroke}{rgb}{0.000000,0.000000,0.000000}%
\pgfsetstrokecolor{currentstroke}%
\pgfsetdash{}{0pt}%
\pgfsys@defobject{currentmarker}{\pgfqpoint{-0.048611in}{0.000000in}}{\pgfqpoint{-0.000000in}{0.000000in}}{%
\pgfpathmoveto{\pgfqpoint{-0.000000in}{0.000000in}}%
\pgfpathlineto{\pgfqpoint{-0.048611in}{0.000000in}}%
\pgfusepath{stroke,fill}%
}%
\begin{pgfscope}%
\pgfsys@transformshift{0.481978in}{5.539895in}%
\pgfsys@useobject{currentmarker}{}%
\end{pgfscope}%
\end{pgfscope}%
\begin{pgfscope}%
\definecolor{textcolor}{rgb}{0.000000,0.000000,0.000000}%
\pgfsetstrokecolor{textcolor}%
\pgfsetfillcolor{textcolor}%
\pgftext[x=0.208025in, y=5.487134in, left, base]{\color{textcolor}\sffamily\fontsize{10.000000}{12.000000}\selectfont 20}%
\end{pgfscope}%
\begin{pgfscope}%
\pgfsetbuttcap%
\pgfsetroundjoin%
\definecolor{currentfill}{rgb}{0.000000,0.000000,0.000000}%
\pgfsetfillcolor{currentfill}%
\pgfsetlinewidth{0.803000pt}%
\definecolor{currentstroke}{rgb}{0.000000,0.000000,0.000000}%
\pgfsetstrokecolor{currentstroke}%
\pgfsetdash{}{0pt}%
\pgfsys@defobject{currentmarker}{\pgfqpoint{-0.048611in}{0.000000in}}{\pgfqpoint{-0.000000in}{0.000000in}}{%
\pgfpathmoveto{\pgfqpoint{-0.000000in}{0.000000in}}%
\pgfpathlineto{\pgfqpoint{-0.048611in}{0.000000in}}%
\pgfusepath{stroke,fill}%
}%
\begin{pgfscope}%
\pgfsys@transformshift{0.481978in}{6.575256in}%
\pgfsys@useobject{currentmarker}{}%
\end{pgfscope}%
\end{pgfscope}%
\begin{pgfscope}%
\definecolor{textcolor}{rgb}{0.000000,0.000000,0.000000}%
\pgfsetstrokecolor{textcolor}%
\pgfsetfillcolor{textcolor}%
\pgftext[x=0.208025in, y=6.522494in, left, base]{\color{textcolor}\sffamily\fontsize{10.000000}{12.000000}\selectfont 40}%
\end{pgfscope}%
\begin{pgfscope}%
\pgfsetbuttcap%
\pgfsetroundjoin%
\definecolor{currentfill}{rgb}{0.000000,0.000000,0.000000}%
\pgfsetfillcolor{currentfill}%
\pgfsetlinewidth{0.803000pt}%
\definecolor{currentstroke}{rgb}{0.000000,0.000000,0.000000}%
\pgfsetstrokecolor{currentstroke}%
\pgfsetdash{}{0pt}%
\pgfsys@defobject{currentmarker}{\pgfqpoint{-0.048611in}{0.000000in}}{\pgfqpoint{-0.000000in}{0.000000in}}{%
\pgfpathmoveto{\pgfqpoint{-0.000000in}{0.000000in}}%
\pgfpathlineto{\pgfqpoint{-0.048611in}{0.000000in}}%
\pgfusepath{stroke,fill}%
}%
\begin{pgfscope}%
\pgfsys@transformshift{0.481978in}{7.610616in}%
\pgfsys@useobject{currentmarker}{}%
\end{pgfscope}%
\end{pgfscope}%
\begin{pgfscope}%
\definecolor{textcolor}{rgb}{0.000000,0.000000,0.000000}%
\pgfsetstrokecolor{textcolor}%
\pgfsetfillcolor{textcolor}%
\pgftext[x=0.208025in, y=7.557854in, left, base]{\color{textcolor}\sffamily\fontsize{10.000000}{12.000000}\selectfont 60}%
\end{pgfscope}%
\begin{pgfscope}%
\pgfpathrectangle{\pgfqpoint{0.481978in}{0.331635in}}{\pgfqpoint{9.300000in}{7.700000in}}%
\pgfusepath{clip}%
\pgfsetrectcap%
\pgfsetroundjoin%
\pgfsetlinewidth{1.505625pt}%
\definecolor{currentstroke}{rgb}{0.631373,0.788235,0.956863}%
\pgfsetstrokecolor{currentstroke}%
\pgfsetstrokeopacity{0.800000}%
\pgfsetdash{}{0pt}%
\pgfpathmoveto{\pgfqpoint{5.968461in}{3.683953in}}%
\pgfpathlineto{\pgfqpoint{6.060728in}{3.871009in}}%
\pgfusepath{stroke}%
\end{pgfscope}%
\begin{pgfscope}%
\pgfpathrectangle{\pgfqpoint{0.481978in}{0.331635in}}{\pgfqpoint{9.300000in}{7.700000in}}%
\pgfusepath{clip}%
\pgfsetrectcap%
\pgfsetroundjoin%
\pgfsetlinewidth{1.505625pt}%
\definecolor{currentstroke}{rgb}{0.631373,0.788235,0.956863}%
\pgfsetstrokecolor{currentstroke}%
\pgfsetstrokeopacity{0.800000}%
\pgfsetdash{}{0pt}%
\pgfpathmoveto{\pgfqpoint{6.716760in}{0.681635in}}%
\pgfpathlineto{\pgfqpoint{6.060728in}{3.871009in}}%
\pgfusepath{stroke}%
\end{pgfscope}%
\begin{pgfscope}%
\pgfpathrectangle{\pgfqpoint{0.481978in}{0.331635in}}{\pgfqpoint{9.300000in}{7.700000in}}%
\pgfusepath{clip}%
\pgfsetrectcap%
\pgfsetroundjoin%
\pgfsetlinewidth{1.505625pt}%
\definecolor{currentstroke}{rgb}{0.631373,0.788235,0.956863}%
\pgfsetstrokecolor{currentstroke}%
\pgfsetstrokeopacity{0.800000}%
\pgfsetdash{}{0pt}%
\pgfpathmoveto{\pgfqpoint{4.030414in}{4.029607in}}%
\pgfpathlineto{\pgfqpoint{6.060728in}{3.871009in}}%
\pgfusepath{stroke}%
\end{pgfscope}%
\begin{pgfscope}%
\pgfpathrectangle{\pgfqpoint{0.481978in}{0.331635in}}{\pgfqpoint{9.300000in}{7.700000in}}%
\pgfusepath{clip}%
\pgfsetrectcap%
\pgfsetroundjoin%
\pgfsetlinewidth{1.505625pt}%
\definecolor{currentstroke}{rgb}{0.631373,0.788235,0.956863}%
\pgfsetstrokecolor{currentstroke}%
\pgfsetstrokeopacity{0.800000}%
\pgfsetdash{}{0pt}%
\pgfpathmoveto{\pgfqpoint{3.807365in}{2.342544in}}%
\pgfpathlineto{\pgfqpoint{6.060728in}{3.871009in}}%
\pgfusepath{stroke}%
\end{pgfscope}%
\begin{pgfscope}%
\pgfpathrectangle{\pgfqpoint{0.481978in}{0.331635in}}{\pgfqpoint{9.300000in}{7.700000in}}%
\pgfusepath{clip}%
\pgfsetrectcap%
\pgfsetroundjoin%
\pgfsetlinewidth{1.505625pt}%
\definecolor{currentstroke}{rgb}{0.631373,0.788235,0.956863}%
\pgfsetstrokecolor{currentstroke}%
\pgfsetstrokeopacity{0.800000}%
\pgfsetdash{}{0pt}%
\pgfpathmoveto{\pgfqpoint{2.350199in}{2.855754in}}%
\pgfpathlineto{\pgfqpoint{6.060728in}{3.871009in}}%
\pgfusepath{stroke}%
\end{pgfscope}%
\begin{pgfscope}%
\pgfpathrectangle{\pgfqpoint{0.481978in}{0.331635in}}{\pgfqpoint{9.300000in}{7.700000in}}%
\pgfusepath{clip}%
\pgfsetrectcap%
\pgfsetroundjoin%
\pgfsetlinewidth{1.505625pt}%
\definecolor{currentstroke}{rgb}{0.631373,0.788235,0.956863}%
\pgfsetstrokecolor{currentstroke}%
\pgfsetstrokeopacity{0.800000}%
\pgfsetdash{}{0pt}%
\pgfpathmoveto{\pgfqpoint{5.589222in}{5.972624in}}%
\pgfpathlineto{\pgfqpoint{6.060728in}{3.871009in}}%
\pgfusepath{stroke}%
\end{pgfscope}%
\begin{pgfscope}%
\pgfpathrectangle{\pgfqpoint{0.481978in}{0.331635in}}{\pgfqpoint{9.300000in}{7.700000in}}%
\pgfusepath{clip}%
\pgfsetrectcap%
\pgfsetroundjoin%
\pgfsetlinewidth{1.505625pt}%
\definecolor{currentstroke}{rgb}{0.631373,0.788235,0.956863}%
\pgfsetstrokecolor{currentstroke}%
\pgfsetstrokeopacity{0.800000}%
\pgfsetdash{}{0pt}%
\pgfpathmoveto{\pgfqpoint{9.359251in}{4.445797in}}%
\pgfpathlineto{\pgfqpoint{6.060728in}{3.871009in}}%
\pgfusepath{stroke}%
\end{pgfscope}%
\begin{pgfscope}%
\pgfpathrectangle{\pgfqpoint{0.481978in}{0.331635in}}{\pgfqpoint{9.300000in}{7.700000in}}%
\pgfusepath{clip}%
\pgfsetrectcap%
\pgfsetroundjoin%
\pgfsetlinewidth{1.505625pt}%
\definecolor{currentstroke}{rgb}{0.631373,0.788235,0.956863}%
\pgfsetstrokecolor{currentstroke}%
\pgfsetstrokeopacity{0.800000}%
\pgfsetdash{}{0pt}%
\pgfpathmoveto{\pgfqpoint{5.412406in}{5.087405in}}%
\pgfpathlineto{\pgfqpoint{6.060728in}{3.871009in}}%
\pgfusepath{stroke}%
\end{pgfscope}%
\begin{pgfscope}%
\pgfpathrectangle{\pgfqpoint{0.481978in}{0.331635in}}{\pgfqpoint{9.300000in}{7.700000in}}%
\pgfusepath{clip}%
\pgfsetrectcap%
\pgfsetroundjoin%
\pgfsetlinewidth{1.505625pt}%
\definecolor{currentstroke}{rgb}{0.631373,0.788235,0.956863}%
\pgfsetstrokecolor{currentstroke}%
\pgfsetstrokeopacity{0.800000}%
\pgfsetdash{}{0pt}%
\pgfpathmoveto{\pgfqpoint{3.390760in}{6.272762in}}%
\pgfpathlineto{\pgfqpoint{6.060728in}{3.871009in}}%
\pgfusepath{stroke}%
\end{pgfscope}%
\begin{pgfscope}%
\pgfpathrectangle{\pgfqpoint{0.481978in}{0.331635in}}{\pgfqpoint{9.300000in}{7.700000in}}%
\pgfusepath{clip}%
\pgfsetrectcap%
\pgfsetroundjoin%
\pgfsetlinewidth{1.505625pt}%
\definecolor{currentstroke}{rgb}{0.631373,0.788235,0.956863}%
\pgfsetstrokecolor{currentstroke}%
\pgfsetstrokeopacity{0.800000}%
\pgfsetdash{}{0pt}%
\pgfpathmoveto{\pgfqpoint{7.065739in}{4.693543in}}%
\pgfpathlineto{\pgfqpoint{6.060728in}{3.871009in}}%
\pgfusepath{stroke}%
\end{pgfscope}%
\begin{pgfscope}%
\pgfpathrectangle{\pgfqpoint{0.481978in}{0.331635in}}{\pgfqpoint{9.300000in}{7.700000in}}%
\pgfusepath{clip}%
\pgfsetrectcap%
\pgfsetroundjoin%
\pgfsetlinewidth{1.505625pt}%
\definecolor{currentstroke}{rgb}{0.631373,0.788235,0.956863}%
\pgfsetstrokecolor{currentstroke}%
\pgfsetstrokeopacity{0.800000}%
\pgfsetdash{}{0pt}%
\pgfpathmoveto{\pgfqpoint{6.796596in}{5.867778in}}%
\pgfpathlineto{\pgfqpoint{6.060728in}{3.871009in}}%
\pgfusepath{stroke}%
\end{pgfscope}%
\begin{pgfscope}%
\pgfpathrectangle{\pgfqpoint{0.481978in}{0.331635in}}{\pgfqpoint{9.300000in}{7.700000in}}%
\pgfusepath{clip}%
\pgfsetrectcap%
\pgfsetroundjoin%
\pgfsetlinewidth{1.505625pt}%
\definecolor{currentstroke}{rgb}{0.631373,0.788235,0.956863}%
\pgfsetstrokecolor{currentstroke}%
\pgfsetstrokeopacity{0.800000}%
\pgfsetdash{}{0pt}%
\pgfpathmoveto{\pgfqpoint{6.170106in}{5.312882in}}%
\pgfpathlineto{\pgfqpoint{6.060728in}{3.871009in}}%
\pgfusepath{stroke}%
\end{pgfscope}%
\begin{pgfscope}%
\pgfpathrectangle{\pgfqpoint{0.481978in}{0.331635in}}{\pgfqpoint{9.300000in}{7.700000in}}%
\pgfusepath{clip}%
\pgfsetrectcap%
\pgfsetroundjoin%
\pgfsetlinewidth{1.505625pt}%
\definecolor{currentstroke}{rgb}{0.631373,0.788235,0.956863}%
\pgfsetstrokecolor{currentstroke}%
\pgfsetstrokeopacity{0.800000}%
\pgfsetdash{}{0pt}%
\pgfpathmoveto{\pgfqpoint{7.510480in}{3.074835in}}%
\pgfpathlineto{\pgfqpoint{6.060728in}{3.871009in}}%
\pgfusepath{stroke}%
\end{pgfscope}%
\begin{pgfscope}%
\pgfpathrectangle{\pgfqpoint{0.481978in}{0.331635in}}{\pgfqpoint{9.300000in}{7.700000in}}%
\pgfusepath{clip}%
\pgfsetrectcap%
\pgfsetroundjoin%
\pgfsetlinewidth{1.505625pt}%
\definecolor{currentstroke}{rgb}{0.631373,0.788235,0.956863}%
\pgfsetstrokecolor{currentstroke}%
\pgfsetstrokeopacity{0.800000}%
\pgfsetdash{}{0pt}%
\pgfpathmoveto{\pgfqpoint{6.540018in}{1.815747in}}%
\pgfpathlineto{\pgfqpoint{6.060728in}{3.871009in}}%
\pgfusepath{stroke}%
\end{pgfscope}%
\begin{pgfscope}%
\pgfpathrectangle{\pgfqpoint{0.481978in}{0.331635in}}{\pgfqpoint{9.300000in}{7.700000in}}%
\pgfusepath{clip}%
\pgfsetrectcap%
\pgfsetroundjoin%
\pgfsetlinewidth{1.505625pt}%
\definecolor{currentstroke}{rgb}{0.631373,0.788235,0.956863}%
\pgfsetstrokecolor{currentstroke}%
\pgfsetstrokeopacity{0.800000}%
\pgfsetdash{}{0pt}%
\pgfpathmoveto{\pgfqpoint{6.818751in}{2.808587in}}%
\pgfpathlineto{\pgfqpoint{6.060728in}{3.871009in}}%
\pgfusepath{stroke}%
\end{pgfscope}%
\begin{pgfscope}%
\pgfpathrectangle{\pgfqpoint{0.481978in}{0.331635in}}{\pgfqpoint{9.300000in}{7.700000in}}%
\pgfusepath{clip}%
\pgfsetrectcap%
\pgfsetroundjoin%
\pgfsetlinewidth{1.505625pt}%
\definecolor{currentstroke}{rgb}{0.631373,0.788235,0.956863}%
\pgfsetstrokecolor{currentstroke}%
\pgfsetstrokeopacity{0.800000}%
\pgfsetdash{}{0pt}%
\pgfpathmoveto{\pgfqpoint{4.820436in}{6.238215in}}%
\pgfpathlineto{\pgfqpoint{6.060728in}{3.871009in}}%
\pgfusepath{stroke}%
\end{pgfscope}%
\begin{pgfscope}%
\pgfpathrectangle{\pgfqpoint{0.481978in}{0.331635in}}{\pgfqpoint{9.300000in}{7.700000in}}%
\pgfusepath{clip}%
\pgfsetrectcap%
\pgfsetroundjoin%
\pgfsetlinewidth{1.505625pt}%
\definecolor{currentstroke}{rgb}{0.631373,0.788235,0.956863}%
\pgfsetstrokecolor{currentstroke}%
\pgfsetstrokeopacity{0.800000}%
\pgfsetdash{}{0pt}%
\pgfpathmoveto{\pgfqpoint{5.842861in}{4.469204in}}%
\pgfpathlineto{\pgfqpoint{6.060728in}{3.871009in}}%
\pgfusepath{stroke}%
\end{pgfscope}%
\begin{pgfscope}%
\pgfpathrectangle{\pgfqpoint{0.481978in}{0.331635in}}{\pgfqpoint{9.300000in}{7.700000in}}%
\pgfusepath{clip}%
\pgfsetrectcap%
\pgfsetroundjoin%
\pgfsetlinewidth{1.505625pt}%
\definecolor{currentstroke}{rgb}{0.631373,0.788235,0.956863}%
\pgfsetstrokecolor{currentstroke}%
\pgfsetstrokeopacity{0.800000}%
\pgfsetdash{}{0pt}%
\pgfpathmoveto{\pgfqpoint{6.322821in}{6.824794in}}%
\pgfpathlineto{\pgfqpoint{6.060728in}{3.871009in}}%
\pgfusepath{stroke}%
\end{pgfscope}%
\begin{pgfscope}%
\pgfpathrectangle{\pgfqpoint{0.481978in}{0.331635in}}{\pgfqpoint{9.300000in}{7.700000in}}%
\pgfusepath{clip}%
\pgfsetrectcap%
\pgfsetroundjoin%
\pgfsetlinewidth{1.505625pt}%
\definecolor{currentstroke}{rgb}{0.631373,0.788235,0.956863}%
\pgfsetstrokecolor{currentstroke}%
\pgfsetstrokeopacity{0.800000}%
\pgfsetdash{}{0pt}%
\pgfpathmoveto{\pgfqpoint{7.830146in}{1.021290in}}%
\pgfpathlineto{\pgfqpoint{6.060728in}{3.871009in}}%
\pgfusepath{stroke}%
\end{pgfscope}%
\begin{pgfscope}%
\pgfpathrectangle{\pgfqpoint{0.481978in}{0.331635in}}{\pgfqpoint{9.300000in}{7.700000in}}%
\pgfusepath{clip}%
\pgfsetrectcap%
\pgfsetroundjoin%
\pgfsetlinewidth{1.505625pt}%
\definecolor{currentstroke}{rgb}{0.631373,0.788235,0.956863}%
\pgfsetstrokecolor{currentstroke}%
\pgfsetstrokeopacity{0.800000}%
\pgfsetdash{}{0pt}%
\pgfpathmoveto{\pgfqpoint{5.762941in}{2.466744in}}%
\pgfpathlineto{\pgfqpoint{6.060728in}{3.871009in}}%
\pgfusepath{stroke}%
\end{pgfscope}%
\begin{pgfscope}%
\pgfpathrectangle{\pgfqpoint{0.481978in}{0.331635in}}{\pgfqpoint{9.300000in}{7.700000in}}%
\pgfusepath{clip}%
\pgfsetrectcap%
\pgfsetroundjoin%
\pgfsetlinewidth{1.505625pt}%
\definecolor{currentstroke}{rgb}{0.631373,0.788235,0.956863}%
\pgfsetstrokecolor{currentstroke}%
\pgfsetstrokeopacity{0.800000}%
\pgfsetdash{}{0pt}%
\pgfpathmoveto{\pgfqpoint{6.414379in}{4.497968in}}%
\pgfpathlineto{\pgfqpoint{6.060728in}{3.871009in}}%
\pgfusepath{stroke}%
\end{pgfscope}%
\begin{pgfscope}%
\pgfpathrectangle{\pgfqpoint{0.481978in}{0.331635in}}{\pgfqpoint{9.300000in}{7.700000in}}%
\pgfusepath{clip}%
\pgfsetrectcap%
\pgfsetroundjoin%
\pgfsetlinewidth{1.505625pt}%
\definecolor{currentstroke}{rgb}{0.631373,0.788235,0.956863}%
\pgfsetstrokecolor{currentstroke}%
\pgfsetstrokeopacity{0.800000}%
\pgfsetdash{}{0pt}%
\pgfpathmoveto{\pgfqpoint{5.330843in}{3.151599in}}%
\pgfpathlineto{\pgfqpoint{6.060728in}{3.871009in}}%
\pgfusepath{stroke}%
\end{pgfscope}%
\begin{pgfscope}%
\pgfpathrectangle{\pgfqpoint{0.481978in}{0.331635in}}{\pgfqpoint{9.300000in}{7.700000in}}%
\pgfusepath{clip}%
\pgfsetrectcap%
\pgfsetroundjoin%
\pgfsetlinewidth{1.505625pt}%
\definecolor{currentstroke}{rgb}{0.631373,0.788235,0.956863}%
\pgfsetstrokecolor{currentstroke}%
\pgfsetstrokeopacity{0.800000}%
\pgfsetdash{}{0pt}%
\pgfpathmoveto{\pgfqpoint{5.723444in}{1.133310in}}%
\pgfpathlineto{\pgfqpoint{6.060728in}{3.871009in}}%
\pgfusepath{stroke}%
\end{pgfscope}%
\begin{pgfscope}%
\pgfpathrectangle{\pgfqpoint{0.481978in}{0.331635in}}{\pgfqpoint{9.300000in}{7.700000in}}%
\pgfusepath{clip}%
\pgfsetrectcap%
\pgfsetroundjoin%
\pgfsetlinewidth{1.505625pt}%
\definecolor{currentstroke}{rgb}{0.631373,0.788235,0.956863}%
\pgfsetstrokecolor{currentstroke}%
\pgfsetstrokeopacity{0.800000}%
\pgfsetdash{}{0pt}%
\pgfpathmoveto{\pgfqpoint{3.540419in}{3.366356in}}%
\pgfpathlineto{\pgfqpoint{6.060728in}{3.871009in}}%
\pgfusepath{stroke}%
\end{pgfscope}%
\begin{pgfscope}%
\pgfpathrectangle{\pgfqpoint{0.481978in}{0.331635in}}{\pgfqpoint{9.300000in}{7.700000in}}%
\pgfusepath{clip}%
\pgfsetrectcap%
\pgfsetroundjoin%
\pgfsetlinewidth{1.505625pt}%
\definecolor{currentstroke}{rgb}{0.631373,0.788235,0.956863}%
\pgfsetstrokecolor{currentstroke}%
\pgfsetstrokeopacity{0.800000}%
\pgfsetdash{}{0pt}%
\pgfpathmoveto{\pgfqpoint{7.439774in}{5.540750in}}%
\pgfpathlineto{\pgfqpoint{6.060728in}{3.871009in}}%
\pgfusepath{stroke}%
\end{pgfscope}%
\begin{pgfscope}%
\pgfpathrectangle{\pgfqpoint{0.481978in}{0.331635in}}{\pgfqpoint{9.300000in}{7.700000in}}%
\pgfusepath{clip}%
\pgfsetrectcap%
\pgfsetroundjoin%
\pgfsetlinewidth{1.505625pt}%
\definecolor{currentstroke}{rgb}{0.631373,0.788235,0.956863}%
\pgfsetstrokecolor{currentstroke}%
\pgfsetstrokeopacity{0.800000}%
\pgfsetdash{}{0pt}%
\pgfpathmoveto{\pgfqpoint{8.272345in}{2.164029in}}%
\pgfpathlineto{\pgfqpoint{6.060728in}{3.871009in}}%
\pgfusepath{stroke}%
\end{pgfscope}%
\begin{pgfscope}%
\pgfpathrectangle{\pgfqpoint{0.481978in}{0.331635in}}{\pgfqpoint{9.300000in}{7.700000in}}%
\pgfusepath{clip}%
\pgfsetrectcap%
\pgfsetroundjoin%
\pgfsetlinewidth{1.505625pt}%
\definecolor{currentstroke}{rgb}{0.631373,0.788235,0.956863}%
\pgfsetstrokecolor{currentstroke}%
\pgfsetstrokeopacity{0.800000}%
\pgfsetdash{}{0pt}%
\pgfpathmoveto{\pgfqpoint{7.470382in}{1.797485in}}%
\pgfpathlineto{\pgfqpoint{6.060728in}{3.871009in}}%
\pgfusepath{stroke}%
\end{pgfscope}%
\begin{pgfscope}%
\pgfpathrectangle{\pgfqpoint{0.481978in}{0.331635in}}{\pgfqpoint{9.300000in}{7.700000in}}%
\pgfusepath{clip}%
\pgfsetrectcap%
\pgfsetroundjoin%
\pgfsetlinewidth{1.505625pt}%
\definecolor{currentstroke}{rgb}{0.631373,0.788235,0.956863}%
\pgfsetstrokecolor{currentstroke}%
\pgfsetstrokeopacity{0.800000}%
\pgfsetdash{}{0pt}%
\pgfpathmoveto{\pgfqpoint{7.403075in}{6.771043in}}%
\pgfpathlineto{\pgfqpoint{6.060728in}{3.871009in}}%
\pgfusepath{stroke}%
\end{pgfscope}%
\begin{pgfscope}%
\pgfpathrectangle{\pgfqpoint{0.481978in}{0.331635in}}{\pgfqpoint{9.300000in}{7.700000in}}%
\pgfusepath{clip}%
\pgfsetrectcap%
\pgfsetroundjoin%
\pgfsetlinewidth{1.505625pt}%
\definecolor{currentstroke}{rgb}{1.000000,0.705882,0.509804}%
\pgfsetstrokecolor{currentstroke}%
\pgfsetstrokeopacity{0.800000}%
\pgfsetdash{}{0pt}%
\pgfpathmoveto{\pgfqpoint{4.737094in}{4.546006in}}%
\pgfpathlineto{\pgfqpoint{4.307614in}{4.391044in}}%
\pgfusepath{stroke}%
\end{pgfscope}%
\begin{pgfscope}%
\pgfpathrectangle{\pgfqpoint{0.481978in}{0.331635in}}{\pgfqpoint{9.300000in}{7.700000in}}%
\pgfusepath{clip}%
\pgfsetrectcap%
\pgfsetroundjoin%
\pgfsetlinewidth{1.505625pt}%
\definecolor{currentstroke}{rgb}{1.000000,0.705882,0.509804}%
\pgfsetstrokecolor{currentstroke}%
\pgfsetstrokeopacity{0.800000}%
\pgfsetdash{}{0pt}%
\pgfpathmoveto{\pgfqpoint{4.102620in}{7.195105in}}%
\pgfpathlineto{\pgfqpoint{4.307614in}{4.391044in}}%
\pgfusepath{stroke}%
\end{pgfscope}%
\begin{pgfscope}%
\pgfpathrectangle{\pgfqpoint{0.481978in}{0.331635in}}{\pgfqpoint{9.300000in}{7.700000in}}%
\pgfusepath{clip}%
\pgfsetrectcap%
\pgfsetroundjoin%
\pgfsetlinewidth{1.505625pt}%
\definecolor{currentstroke}{rgb}{1.000000,0.705882,0.509804}%
\pgfsetstrokecolor{currentstroke}%
\pgfsetstrokeopacity{0.800000}%
\pgfsetdash{}{0pt}%
\pgfpathmoveto{\pgfqpoint{3.960956in}{4.794147in}}%
\pgfpathlineto{\pgfqpoint{4.307614in}{4.391044in}}%
\pgfusepath{stroke}%
\end{pgfscope}%
\begin{pgfscope}%
\pgfpathrectangle{\pgfqpoint{0.481978in}{0.331635in}}{\pgfqpoint{9.300000in}{7.700000in}}%
\pgfusepath{clip}%
\pgfsetrectcap%
\pgfsetroundjoin%
\pgfsetlinewidth{1.505625pt}%
\definecolor{currentstroke}{rgb}{1.000000,0.705882,0.509804}%
\pgfsetstrokecolor{currentstroke}%
\pgfsetstrokeopacity{0.800000}%
\pgfsetdash{}{0pt}%
\pgfpathmoveto{\pgfqpoint{3.386233in}{5.184879in}}%
\pgfpathlineto{\pgfqpoint{4.307614in}{4.391044in}}%
\pgfusepath{stroke}%
\end{pgfscope}%
\begin{pgfscope}%
\pgfpathrectangle{\pgfqpoint{0.481978in}{0.331635in}}{\pgfqpoint{9.300000in}{7.700000in}}%
\pgfusepath{clip}%
\pgfsetrectcap%
\pgfsetroundjoin%
\pgfsetlinewidth{1.505625pt}%
\definecolor{currentstroke}{rgb}{1.000000,0.705882,0.509804}%
\pgfsetstrokecolor{currentstroke}%
\pgfsetstrokeopacity{0.800000}%
\pgfsetdash{}{0pt}%
\pgfpathmoveto{\pgfqpoint{2.506952in}{3.900738in}}%
\pgfpathlineto{\pgfqpoint{4.307614in}{4.391044in}}%
\pgfusepath{stroke}%
\end{pgfscope}%
\begin{pgfscope}%
\pgfpathrectangle{\pgfqpoint{0.481978in}{0.331635in}}{\pgfqpoint{9.300000in}{7.700000in}}%
\pgfusepath{clip}%
\pgfsetrectcap%
\pgfsetroundjoin%
\pgfsetlinewidth{1.505625pt}%
\definecolor{currentstroke}{rgb}{1.000000,0.705882,0.509804}%
\pgfsetstrokecolor{currentstroke}%
\pgfsetstrokeopacity{0.800000}%
\pgfsetdash{}{0pt}%
\pgfpathmoveto{\pgfqpoint{7.641446in}{4.026183in}}%
\pgfpathlineto{\pgfqpoint{4.307614in}{4.391044in}}%
\pgfusepath{stroke}%
\end{pgfscope}%
\begin{pgfscope}%
\pgfpathrectangle{\pgfqpoint{0.481978in}{0.331635in}}{\pgfqpoint{9.300000in}{7.700000in}}%
\pgfusepath{clip}%
\pgfsetrectcap%
\pgfsetroundjoin%
\pgfsetlinewidth{1.505625pt}%
\definecolor{currentstroke}{rgb}{1.000000,0.705882,0.509804}%
\pgfsetstrokecolor{currentstroke}%
\pgfsetstrokeopacity{0.800000}%
\pgfsetdash{}{0pt}%
\pgfpathmoveto{\pgfqpoint{2.626934in}{4.769158in}}%
\pgfpathlineto{\pgfqpoint{4.307614in}{4.391044in}}%
\pgfusepath{stroke}%
\end{pgfscope}%
\begin{pgfscope}%
\pgfpathrectangle{\pgfqpoint{0.481978in}{0.331635in}}{\pgfqpoint{9.300000in}{7.700000in}}%
\pgfusepath{clip}%
\pgfsetrectcap%
\pgfsetroundjoin%
\pgfsetlinewidth{1.505625pt}%
\definecolor{currentstroke}{rgb}{1.000000,0.705882,0.509804}%
\pgfsetstrokecolor{currentstroke}%
\pgfsetstrokeopacity{0.800000}%
\pgfsetdash{}{0pt}%
\pgfpathmoveto{\pgfqpoint{4.292588in}{1.004552in}}%
\pgfpathlineto{\pgfqpoint{4.307614in}{4.391044in}}%
\pgfusepath{stroke}%
\end{pgfscope}%
\begin{pgfscope}%
\pgfpathrectangle{\pgfqpoint{0.481978in}{0.331635in}}{\pgfqpoint{9.300000in}{7.700000in}}%
\pgfusepath{clip}%
\pgfsetrectcap%
\pgfsetroundjoin%
\pgfsetlinewidth{1.505625pt}%
\definecolor{currentstroke}{rgb}{1.000000,0.705882,0.509804}%
\pgfsetstrokecolor{currentstroke}%
\pgfsetstrokeopacity{0.800000}%
\pgfsetdash{}{0pt}%
\pgfpathmoveto{\pgfqpoint{4.914941in}{2.251106in}}%
\pgfpathlineto{\pgfqpoint{4.307614in}{4.391044in}}%
\pgfusepath{stroke}%
\end{pgfscope}%
\begin{pgfscope}%
\pgfpathrectangle{\pgfqpoint{0.481978in}{0.331635in}}{\pgfqpoint{9.300000in}{7.700000in}}%
\pgfusepath{clip}%
\pgfsetrectcap%
\pgfsetroundjoin%
\pgfsetlinewidth{1.505625pt}%
\definecolor{currentstroke}{rgb}{1.000000,0.705882,0.509804}%
\pgfsetstrokecolor{currentstroke}%
\pgfsetstrokeopacity{0.800000}%
\pgfsetdash{}{0pt}%
\pgfpathmoveto{\pgfqpoint{1.360419in}{3.230142in}}%
\pgfpathlineto{\pgfqpoint{4.307614in}{4.391044in}}%
\pgfusepath{stroke}%
\end{pgfscope}%
\begin{pgfscope}%
\pgfpathrectangle{\pgfqpoint{0.481978in}{0.331635in}}{\pgfqpoint{9.300000in}{7.700000in}}%
\pgfusepath{clip}%
\pgfsetrectcap%
\pgfsetroundjoin%
\pgfsetlinewidth{1.505625pt}%
\definecolor{currentstroke}{rgb}{1.000000,0.705882,0.509804}%
\pgfsetstrokecolor{currentstroke}%
\pgfsetstrokeopacity{0.800000}%
\pgfsetdash{}{0pt}%
\pgfpathmoveto{\pgfqpoint{2.680958in}{5.639274in}}%
\pgfpathlineto{\pgfqpoint{4.307614in}{4.391044in}}%
\pgfusepath{stroke}%
\end{pgfscope}%
\begin{pgfscope}%
\pgfpathrectangle{\pgfqpoint{0.481978in}{0.331635in}}{\pgfqpoint{9.300000in}{7.700000in}}%
\pgfusepath{clip}%
\pgfsetrectcap%
\pgfsetroundjoin%
\pgfsetlinewidth{1.505625pt}%
\definecolor{currentstroke}{rgb}{1.000000,0.705882,0.509804}%
\pgfsetstrokecolor{currentstroke}%
\pgfsetstrokeopacity{0.800000}%
\pgfsetdash{}{0pt}%
\pgfpathmoveto{\pgfqpoint{8.439887in}{6.176815in}}%
\pgfpathlineto{\pgfqpoint{4.307614in}{4.391044in}}%
\pgfusepath{stroke}%
\end{pgfscope}%
\begin{pgfscope}%
\pgfpathrectangle{\pgfqpoint{0.481978in}{0.331635in}}{\pgfqpoint{9.300000in}{7.700000in}}%
\pgfusepath{clip}%
\pgfsetrectcap%
\pgfsetroundjoin%
\pgfsetlinewidth{1.505625pt}%
\definecolor{currentstroke}{rgb}{1.000000,0.705882,0.509804}%
\pgfsetstrokecolor{currentstroke}%
\pgfsetstrokeopacity{0.800000}%
\pgfsetdash{}{0pt}%
\pgfpathmoveto{\pgfqpoint{5.343023in}{7.681635in}}%
\pgfpathlineto{\pgfqpoint{4.307614in}{4.391044in}}%
\pgfusepath{stroke}%
\end{pgfscope}%
\begin{pgfscope}%
\pgfpathrectangle{\pgfqpoint{0.481978in}{0.331635in}}{\pgfqpoint{9.300000in}{7.700000in}}%
\pgfusepath{clip}%
\pgfsetrectcap%
\pgfsetroundjoin%
\pgfsetlinewidth{1.505625pt}%
\definecolor{currentstroke}{rgb}{1.000000,0.705882,0.509804}%
\pgfsetstrokecolor{currentstroke}%
\pgfsetstrokeopacity{0.800000}%
\pgfsetdash{}{0pt}%
\pgfpathmoveto{\pgfqpoint{1.984154in}{4.169349in}}%
\pgfpathlineto{\pgfqpoint{4.307614in}{4.391044in}}%
\pgfusepath{stroke}%
\end{pgfscope}%
\begin{pgfscope}%
\pgfpathrectangle{\pgfqpoint{0.481978in}{0.331635in}}{\pgfqpoint{9.300000in}{7.700000in}}%
\pgfusepath{clip}%
\pgfsetrectcap%
\pgfsetroundjoin%
\pgfsetlinewidth{1.505625pt}%
\definecolor{currentstroke}{rgb}{1.000000,0.705882,0.509804}%
\pgfsetstrokecolor{currentstroke}%
\pgfsetstrokeopacity{0.800000}%
\pgfsetdash{}{0pt}%
\pgfpathmoveto{\pgfqpoint{8.035769in}{4.947906in}}%
\pgfpathlineto{\pgfqpoint{4.307614in}{4.391044in}}%
\pgfusepath{stroke}%
\end{pgfscope}%
\begin{pgfscope}%
\pgfpathrectangle{\pgfqpoint{0.481978in}{0.331635in}}{\pgfqpoint{9.300000in}{7.700000in}}%
\pgfusepath{clip}%
\pgfsetrectcap%
\pgfsetroundjoin%
\pgfsetlinewidth{1.505625pt}%
\definecolor{currentstroke}{rgb}{1.000000,0.705882,0.509804}%
\pgfsetstrokecolor{currentstroke}%
\pgfsetstrokeopacity{0.800000}%
\pgfsetdash{}{0pt}%
\pgfpathmoveto{\pgfqpoint{6.113857in}{3.306063in}}%
\pgfpathlineto{\pgfqpoint{4.307614in}{4.391044in}}%
\pgfusepath{stroke}%
\end{pgfscope}%
\begin{pgfscope}%
\pgfpathrectangle{\pgfqpoint{0.481978in}{0.331635in}}{\pgfqpoint{9.300000in}{7.700000in}}%
\pgfusepath{clip}%
\pgfsetrectcap%
\pgfsetroundjoin%
\pgfsetlinewidth{1.505625pt}%
\definecolor{currentstroke}{rgb}{1.000000,0.705882,0.509804}%
\pgfsetstrokecolor{currentstroke}%
\pgfsetstrokeopacity{0.800000}%
\pgfsetdash{}{0pt}%
\pgfpathmoveto{\pgfqpoint{5.267481in}{4.097480in}}%
\pgfpathlineto{\pgfqpoint{4.307614in}{4.391044in}}%
\pgfusepath{stroke}%
\end{pgfscope}%
\begin{pgfscope}%
\pgfpathrectangle{\pgfqpoint{0.481978in}{0.331635in}}{\pgfqpoint{9.300000in}{7.700000in}}%
\pgfusepath{clip}%
\pgfsetrectcap%
\pgfsetroundjoin%
\pgfsetlinewidth{1.505625pt}%
\definecolor{currentstroke}{rgb}{1.000000,0.705882,0.509804}%
\pgfsetstrokecolor{currentstroke}%
\pgfsetstrokeopacity{0.800000}%
\pgfsetdash{}{0pt}%
\pgfpathmoveto{\pgfqpoint{2.715921in}{1.871136in}}%
\pgfpathlineto{\pgfqpoint{4.307614in}{4.391044in}}%
\pgfusepath{stroke}%
\end{pgfscope}%
\begin{pgfscope}%
\pgfpathrectangle{\pgfqpoint{0.481978in}{0.331635in}}{\pgfqpoint{9.300000in}{7.700000in}}%
\pgfusepath{clip}%
\pgfsetrectcap%
\pgfsetroundjoin%
\pgfsetlinewidth{1.505625pt}%
\definecolor{currentstroke}{rgb}{1.000000,0.705882,0.509804}%
\pgfsetstrokecolor{currentstroke}%
\pgfsetstrokeopacity{0.800000}%
\pgfsetdash{}{0pt}%
\pgfpathmoveto{\pgfqpoint{0.904705in}{5.096309in}}%
\pgfpathlineto{\pgfqpoint{4.307614in}{4.391044in}}%
\pgfusepath{stroke}%
\end{pgfscope}%
\begin{pgfscope}%
\pgfpathrectangle{\pgfqpoint{0.481978in}{0.331635in}}{\pgfqpoint{9.300000in}{7.700000in}}%
\pgfusepath{clip}%
\pgfsetrectcap%
\pgfsetroundjoin%
\pgfsetlinewidth{1.505625pt}%
\definecolor{currentstroke}{rgb}{1.000000,0.705882,0.509804}%
\pgfsetstrokecolor{currentstroke}%
\pgfsetstrokeopacity{0.800000}%
\pgfsetdash{}{0pt}%
\pgfpathmoveto{\pgfqpoint{8.533110in}{3.582315in}}%
\pgfpathlineto{\pgfqpoint{4.307614in}{4.391044in}}%
\pgfusepath{stroke}%
\end{pgfscope}%
\begin{pgfscope}%
\pgfpathrectangle{\pgfqpoint{0.481978in}{0.331635in}}{\pgfqpoint{9.300000in}{7.700000in}}%
\pgfusepath{clip}%
\pgfsetrectcap%
\pgfsetroundjoin%
\pgfsetlinewidth{1.505625pt}%
\definecolor{currentstroke}{rgb}{1.000000,0.705882,0.509804}%
\pgfsetstrokecolor{currentstroke}%
\pgfsetstrokeopacity{0.800000}%
\pgfsetdash{}{0pt}%
\pgfpathmoveto{\pgfqpoint{4.391134in}{3.048940in}}%
\pgfpathlineto{\pgfqpoint{4.307614in}{4.391044in}}%
\pgfusepath{stroke}%
\end{pgfscope}%
\begin{pgfscope}%
\pgfpathrectangle{\pgfqpoint{0.481978in}{0.331635in}}{\pgfqpoint{9.300000in}{7.700000in}}%
\pgfusepath{clip}%
\pgfsetrectcap%
\pgfsetroundjoin%
\pgfsetlinewidth{1.505625pt}%
\definecolor{currentstroke}{rgb}{1.000000,0.705882,0.509804}%
\pgfsetstrokecolor{currentstroke}%
\pgfsetstrokeopacity{0.800000}%
\pgfsetdash{}{0pt}%
\pgfpathmoveto{\pgfqpoint{1.537381in}{5.337200in}}%
\pgfpathlineto{\pgfqpoint{4.307614in}{4.391044in}}%
\pgfusepath{stroke}%
\end{pgfscope}%
\begin{pgfscope}%
\pgfpathrectangle{\pgfqpoint{0.481978in}{0.331635in}}{\pgfqpoint{9.300000in}{7.700000in}}%
\pgfusepath{clip}%
\pgfsetrectcap%
\pgfsetroundjoin%
\pgfsetlinewidth{1.505625pt}%
\definecolor{currentstroke}{rgb}{1.000000,0.705882,0.509804}%
\pgfsetstrokecolor{currentstroke}%
\pgfsetstrokeopacity{0.800000}%
\pgfsetdash{}{0pt}%
\pgfpathmoveto{\pgfqpoint{6.879175in}{3.740228in}}%
\pgfpathlineto{\pgfqpoint{4.307614in}{4.391044in}}%
\pgfusepath{stroke}%
\end{pgfscope}%
\begin{pgfscope}%
\pgfpathrectangle{\pgfqpoint{0.481978in}{0.331635in}}{\pgfqpoint{9.300000in}{7.700000in}}%
\pgfusepath{clip}%
\pgfsetrectcap%
\pgfsetroundjoin%
\pgfsetlinewidth{1.505625pt}%
\definecolor{currentstroke}{rgb}{1.000000,0.705882,0.509804}%
\pgfsetstrokecolor{currentstroke}%
\pgfsetstrokeopacity{0.800000}%
\pgfsetdash{}{0pt}%
\pgfpathmoveto{\pgfqpoint{4.800335in}{3.722296in}}%
\pgfpathlineto{\pgfqpoint{4.307614in}{4.391044in}}%
\pgfusepath{stroke}%
\end{pgfscope}%
\begin{pgfscope}%
\pgfpathrectangle{\pgfqpoint{0.481978in}{0.331635in}}{\pgfqpoint{9.300000in}{7.700000in}}%
\pgfusepath{clip}%
\pgfsetrectcap%
\pgfsetroundjoin%
\pgfsetlinewidth{1.505625pt}%
\definecolor{currentstroke}{rgb}{1.000000,0.705882,0.509804}%
\pgfsetstrokecolor{currentstroke}%
\pgfsetstrokeopacity{0.800000}%
\pgfsetdash{}{0pt}%
\pgfpathmoveto{\pgfqpoint{4.757910in}{5.329887in}}%
\pgfpathlineto{\pgfqpoint{4.307614in}{4.391044in}}%
\pgfusepath{stroke}%
\end{pgfscope}%
\begin{pgfscope}%
\pgfpathrectangle{\pgfqpoint{0.481978in}{0.331635in}}{\pgfqpoint{9.300000in}{7.700000in}}%
\pgfusepath{clip}%
\pgfsetrectcap%
\pgfsetroundjoin%
\pgfsetlinewidth{1.505625pt}%
\definecolor{currentstroke}{rgb}{1.000000,0.705882,0.509804}%
\pgfsetstrokecolor{currentstroke}%
\pgfsetstrokeopacity{0.800000}%
\pgfsetdash{}{0pt}%
\pgfpathmoveto{\pgfqpoint{4.104994in}{5.688256in}}%
\pgfpathlineto{\pgfqpoint{4.307614in}{4.391044in}}%
\pgfusepath{stroke}%
\end{pgfscope}%
\begin{pgfscope}%
\pgfpathrectangle{\pgfqpoint{0.481978in}{0.331635in}}{\pgfqpoint{9.300000in}{7.700000in}}%
\pgfusepath{clip}%
\pgfsetrectcap%
\pgfsetroundjoin%
\pgfsetlinewidth{1.505625pt}%
\definecolor{currentstroke}{rgb}{1.000000,0.705882,0.509804}%
\pgfsetstrokecolor{currentstroke}%
\pgfsetstrokeopacity{0.800000}%
\pgfsetdash{}{0pt}%
\pgfpathmoveto{\pgfqpoint{1.344247in}{4.370626in}}%
\pgfpathlineto{\pgfqpoint{4.307614in}{4.391044in}}%
\pgfusepath{stroke}%
\end{pgfscope}%
\begin{pgfscope}%
\pgfpathrectangle{\pgfqpoint{0.481978in}{0.331635in}}{\pgfqpoint{9.300000in}{7.700000in}}%
\pgfusepath{clip}%
\pgfsetrectcap%
\pgfsetroundjoin%
\pgfsetlinewidth{1.505625pt}%
\definecolor{currentstroke}{rgb}{1.000000,0.705882,0.509804}%
\pgfsetstrokecolor{currentstroke}%
\pgfsetstrokeopacity{0.800000}%
\pgfsetdash{}{0pt}%
\pgfpathmoveto{\pgfqpoint{3.248962in}{4.241499in}}%
\pgfpathlineto{\pgfqpoint{4.307614in}{4.391044in}}%
\pgfusepath{stroke}%
\end{pgfscope}%
\begin{pgfscope}%
\pgfsetrectcap%
\pgfsetmiterjoin%
\pgfsetlinewidth{0.803000pt}%
\definecolor{currentstroke}{rgb}{0.000000,0.000000,0.000000}%
\pgfsetstrokecolor{currentstroke}%
\pgfsetdash{}{0pt}%
\pgfpathmoveto{\pgfqpoint{0.481978in}{0.331635in}}%
\pgfpathlineto{\pgfqpoint{0.481978in}{8.031635in}}%
\pgfusepath{stroke}%
\end{pgfscope}%
\begin{pgfscope}%
\pgfsetrectcap%
\pgfsetmiterjoin%
\pgfsetlinewidth{0.803000pt}%
\definecolor{currentstroke}{rgb}{0.000000,0.000000,0.000000}%
\pgfsetstrokecolor{currentstroke}%
\pgfsetdash{}{0pt}%
\pgfpathmoveto{\pgfqpoint{9.781978in}{0.331635in}}%
\pgfpathlineto{\pgfqpoint{9.781978in}{8.031635in}}%
\pgfusepath{stroke}%
\end{pgfscope}%
\begin{pgfscope}%
\pgfsetrectcap%
\pgfsetmiterjoin%
\pgfsetlinewidth{0.803000pt}%
\definecolor{currentstroke}{rgb}{0.000000,0.000000,0.000000}%
\pgfsetstrokecolor{currentstroke}%
\pgfsetdash{}{0pt}%
\pgfpathmoveto{\pgfqpoint{0.481978in}{0.331635in}}%
\pgfpathlineto{\pgfqpoint{9.781978in}{0.331635in}}%
\pgfusepath{stroke}%
\end{pgfscope}%
\begin{pgfscope}%
\pgfsetrectcap%
\pgfsetmiterjoin%
\pgfsetlinewidth{0.803000pt}%
\definecolor{currentstroke}{rgb}{0.000000,0.000000,0.000000}%
\pgfsetstrokecolor{currentstroke}%
\pgfsetdash{}{0pt}%
\pgfpathmoveto{\pgfqpoint{0.481978in}{8.031635in}}%
\pgfpathlineto{\pgfqpoint{9.781978in}{8.031635in}}%
\pgfusepath{stroke}%
\end{pgfscope}%
\begin{pgfscope}%
\definecolor{textcolor}{rgb}{0.000000,0.000000,0.000000}%
\pgfsetstrokecolor{textcolor}%
\pgfsetfillcolor{textcolor}%
\pgftext[x=5.131978in,y=8.114968in,,base]{\color{textcolor}\sffamily\fontsize{12.000000}{14.400000}\selectfont Photo-Realistic Images}%
\end{pgfscope}%
\begin{pgfscope}%
\pgfsetbuttcap%
\pgfsetmiterjoin%
\definecolor{currentfill}{rgb}{1.000000,1.000000,1.000000}%
\pgfsetfillcolor{currentfill}%
\pgfsetfillopacity{0.800000}%
\pgfsetlinewidth{1.003750pt}%
\definecolor{currentstroke}{rgb}{0.800000,0.800000,0.800000}%
\pgfsetstrokecolor{currentstroke}%
\pgfsetstrokeopacity{0.800000}%
\pgfsetdash{}{0pt}%
\pgfpathmoveto{\pgfqpoint{9.879200in}{3.956944in}}%
\pgfpathlineto{\pgfqpoint{11.101911in}{3.956944in}}%
\pgfpathquadraticcurveto{\pgfqpoint{11.129688in}{3.956944in}}{\pgfqpoint{11.129688in}{3.984722in}}%
\pgfpathlineto{\pgfqpoint{11.129688in}{4.378548in}}%
\pgfpathquadraticcurveto{\pgfqpoint{11.129688in}{4.406326in}}{\pgfqpoint{11.101911in}{4.406326in}}%
\pgfpathlineto{\pgfqpoint{9.879200in}{4.406326in}}%
\pgfpathquadraticcurveto{\pgfqpoint{9.851422in}{4.406326in}}{\pgfqpoint{9.851422in}{4.378548in}}%
\pgfpathlineto{\pgfqpoint{9.851422in}{3.984722in}}%
\pgfpathquadraticcurveto{\pgfqpoint{9.851422in}{3.956944in}}{\pgfqpoint{9.879200in}{3.956944in}}%
\pgfpathclose%
\pgfusepath{stroke,fill}%
\end{pgfscope}%
\begin{pgfscope}%
\pgfsetbuttcap%
\pgfsetroundjoin%
\definecolor{currentfill}{rgb}{0.631373,0.788235,0.956863}%
\pgfsetfillcolor{currentfill}%
\pgfsetlinewidth{1.003750pt}%
\definecolor{currentstroke}{rgb}{0.631373,0.788235,0.956863}%
\pgfsetstrokecolor{currentstroke}%
\pgfsetdash{}{0pt}%
\pgfsys@defobject{currentmarker}{\pgfqpoint{-0.041667in}{-0.041667in}}{\pgfqpoint{0.041667in}{0.041667in}}{%
\pgfpathmoveto{\pgfqpoint{0.000000in}{-0.041667in}}%
\pgfpathcurveto{\pgfqpoint{0.011050in}{-0.041667in}}{\pgfqpoint{0.021649in}{-0.037276in}}{\pgfqpoint{0.029463in}{-0.029463in}}%
\pgfpathcurveto{\pgfqpoint{0.037276in}{-0.021649in}}{\pgfqpoint{0.041667in}{-0.011050in}}{\pgfqpoint{0.041667in}{0.000000in}}%
\pgfpathcurveto{\pgfqpoint{0.041667in}{0.011050in}}{\pgfqpoint{0.037276in}{0.021649in}}{\pgfqpoint{0.029463in}{0.029463in}}%
\pgfpathcurveto{\pgfqpoint{0.021649in}{0.037276in}}{\pgfqpoint{0.011050in}{0.041667in}}{\pgfqpoint{0.000000in}{0.041667in}}%
\pgfpathcurveto{\pgfqpoint{-0.011050in}{0.041667in}}{\pgfqpoint{-0.021649in}{0.037276in}}{\pgfqpoint{-0.029463in}{0.029463in}}%
\pgfpathcurveto{\pgfqpoint{-0.037276in}{0.021649in}}{\pgfqpoint{-0.041667in}{0.011050in}}{\pgfqpoint{-0.041667in}{0.000000in}}%
\pgfpathcurveto{\pgfqpoint{-0.041667in}{-0.011050in}}{\pgfqpoint{-0.037276in}{-0.021649in}}{\pgfqpoint{-0.029463in}{-0.029463in}}%
\pgfpathcurveto{\pgfqpoint{-0.021649in}{-0.037276in}}{\pgfqpoint{-0.011050in}{-0.041667in}}{\pgfqpoint{0.000000in}{-0.041667in}}%
\pgfpathclose%
\pgfusepath{stroke,fill}%
}%
\begin{pgfscope}%
\pgfsys@transformshift{10.045867in}{4.281705in}%
\pgfsys@useobject{currentmarker}{}%
\end{pgfscope}%
\end{pgfscope}%
\begin{pgfscope}%
\definecolor{textcolor}{rgb}{0.000000,0.000000,0.000000}%
\pgfsetstrokecolor{textcolor}%
\pgfsetfillcolor{textcolor}%
\pgftext[x=10.295867in,y=4.245247in,left,base]{\color{textcolor}\sffamily\fontsize{10.000000}{12.000000}\selectfont openrooms}%
\end{pgfscope}%
\begin{pgfscope}%
\pgfsetbuttcap%
\pgfsetroundjoin%
\definecolor{currentfill}{rgb}{1.000000,0.705882,0.509804}%
\pgfsetfillcolor{currentfill}%
\pgfsetlinewidth{1.003750pt}%
\definecolor{currentstroke}{rgb}{1.000000,0.705882,0.509804}%
\pgfsetstrokecolor{currentstroke}%
\pgfsetdash{}{0pt}%
\pgfsys@defobject{currentmarker}{\pgfqpoint{-0.041667in}{-0.041667in}}{\pgfqpoint{0.041667in}{0.041667in}}{%
\pgfpathmoveto{\pgfqpoint{0.000000in}{-0.041667in}}%
\pgfpathcurveto{\pgfqpoint{0.011050in}{-0.041667in}}{\pgfqpoint{0.021649in}{-0.037276in}}{\pgfqpoint{0.029463in}{-0.029463in}}%
\pgfpathcurveto{\pgfqpoint{0.037276in}{-0.021649in}}{\pgfqpoint{0.041667in}{-0.011050in}}{\pgfqpoint{0.041667in}{0.000000in}}%
\pgfpathcurveto{\pgfqpoint{0.041667in}{0.011050in}}{\pgfqpoint{0.037276in}{0.021649in}}{\pgfqpoint{0.029463in}{0.029463in}}%
\pgfpathcurveto{\pgfqpoint{0.021649in}{0.037276in}}{\pgfqpoint{0.011050in}{0.041667in}}{\pgfqpoint{0.000000in}{0.041667in}}%
\pgfpathcurveto{\pgfqpoint{-0.011050in}{0.041667in}}{\pgfqpoint{-0.021649in}{0.037276in}}{\pgfqpoint{-0.029463in}{0.029463in}}%
\pgfpathcurveto{\pgfqpoint{-0.037276in}{0.021649in}}{\pgfqpoint{-0.041667in}{0.011050in}}{\pgfqpoint{-0.041667in}{0.000000in}}%
\pgfpathcurveto{\pgfqpoint{-0.041667in}{-0.011050in}}{\pgfqpoint{-0.037276in}{-0.021649in}}{\pgfqpoint{-0.029463in}{-0.029463in}}%
\pgfpathcurveto{\pgfqpoint{-0.021649in}{-0.037276in}}{\pgfqpoint{-0.011050in}{-0.041667in}}{\pgfqpoint{0.000000in}{-0.041667in}}%
\pgfpathclose%
\pgfusepath{stroke,fill}%
}%
\begin{pgfscope}%
\pgfsys@transformshift{10.045867in}{4.077848in}%
\pgfsys@useobject{currentmarker}{}%
\end{pgfscope}%
\end{pgfscope}%
\begin{pgfscope}%
\definecolor{textcolor}{rgb}{0.000000,0.000000,0.000000}%
\pgfsetstrokecolor{textcolor}%
\pgfsetfillcolor{textcolor}%
\pgftext[x=10.295867in,y=4.041390in,left,base]{\color{textcolor}\sffamily\fontsize{10.000000}{12.000000}\selectfont pix3d}%
\end{pgfscope}%
\end{pgfpicture}%
\makeatother%
\endgroup%
}
    \resizebox{0.49\linewidth}{5cm}{%% Creator: Matplotlib, PGF backend
%%
%% To include the figure in your LaTeX document, write
%%   \input{<filename>.pgf}
%%
%% Make sure the required packages are loaded in your preamble
%%   \usepackage{pgf}
%%
%% Figures using additional raster images can only be included by \input if
%% they are in the same directory as the main LaTeX file. For loading figures
%% from other directories you can use the `import` package
%%   \usepackage{import}
%%
%% and then include the figures with
%%   \import{<path to file>}{<filename>.pgf}
%%
%% Matplotlib used the following preamble
%%   \usepackage{fontspec}
%%   \setmainfont{DejaVuSerif.ttf}[Path=\detokenize{/Users/apple/opt/anaconda3/envs/kaolin/lib/python3.7/site-packages/matplotlib/mpl-data/fonts/ttf/}]
%%   \setsansfont{DejaVuSans.ttf}[Path=\detokenize{/Users/apple/opt/anaconda3/envs/kaolin/lib/python3.7/site-packages/matplotlib/mpl-data/fonts/ttf/}]
%%   \setmonofont{DejaVuSansMono.ttf}[Path=\detokenize{/Users/apple/opt/anaconda3/envs/kaolin/lib/python3.7/site-packages/matplotlib/mpl-data/fonts/ttf/}]
%%
\begingroup%
\makeatletter%
\begin{pgfpicture}%
\pgfpathrectangle{\pgfpointorigin}{\pgfqpoint{5.541978in}{4.337596in}}%
\pgfusepath{use as bounding box, clip}%
\begin{pgfscope}%
\pgfsetbuttcap%
\pgfsetmiterjoin%
\definecolor{currentfill}{rgb}{1.000000,1.000000,1.000000}%
\pgfsetfillcolor{currentfill}%
\pgfsetlinewidth{0.000000pt}%
\definecolor{currentstroke}{rgb}{1.000000,1.000000,1.000000}%
\pgfsetstrokecolor{currentstroke}%
\pgfsetdash{}{0pt}%
\pgfpathmoveto{\pgfqpoint{0.000000in}{0.000000in}}%
\pgfpathlineto{\pgfqpoint{5.541978in}{0.000000in}}%
\pgfpathlineto{\pgfqpoint{5.541978in}{4.337596in}}%
\pgfpathlineto{\pgfqpoint{0.000000in}{4.337596in}}%
\pgfpathclose%
\pgfusepath{fill}%
\end{pgfscope}%
\begin{pgfscope}%
\pgfsetbuttcap%
\pgfsetmiterjoin%
\definecolor{currentfill}{rgb}{1.000000,1.000000,1.000000}%
\pgfsetfillcolor{currentfill}%
\pgfsetlinewidth{0.000000pt}%
\definecolor{currentstroke}{rgb}{0.000000,0.000000,0.000000}%
\pgfsetstrokecolor{currentstroke}%
\pgfsetstrokeopacity{0.000000}%
\pgfsetdash{}{0pt}%
\pgfpathmoveto{\pgfqpoint{0.481978in}{0.331635in}}%
\pgfpathlineto{\pgfqpoint{5.441978in}{0.331635in}}%
\pgfpathlineto{\pgfqpoint{5.441978in}{4.027635in}}%
\pgfpathlineto{\pgfqpoint{0.481978in}{4.027635in}}%
\pgfpathclose%
\pgfusepath{fill}%
\end{pgfscope}%
\begin{pgfscope}%
\pgfpathrectangle{\pgfqpoint{0.481978in}{0.331635in}}{\pgfqpoint{4.960000in}{3.696000in}}%
\pgfusepath{clip}%
\pgfsetbuttcap%
\pgfsetroundjoin%
\definecolor{currentfill}{rgb}{0.631373,0.788235,0.956863}%
\pgfsetfillcolor{currentfill}%
\pgfsetlinewidth{0.481800pt}%
\definecolor{currentstroke}{rgb}{1.000000,1.000000,1.000000}%
\pgfsetstrokecolor{currentstroke}%
\pgfsetdash{}{0pt}%
\pgfpathmoveto{\pgfqpoint{3.170276in}{2.389021in}}%
\pgfpathcurveto{\pgfqpoint{3.181326in}{2.389021in}}{\pgfqpoint{3.191925in}{2.393411in}}{\pgfqpoint{3.199739in}{2.401225in}}%
\pgfpathcurveto{\pgfqpoint{3.207552in}{2.409038in}}{\pgfqpoint{3.211942in}{2.419637in}}{\pgfqpoint{3.211942in}{2.430688in}}%
\pgfpathcurveto{\pgfqpoint{3.211942in}{2.441738in}}{\pgfqpoint{3.207552in}{2.452337in}}{\pgfqpoint{3.199739in}{2.460150in}}%
\pgfpathcurveto{\pgfqpoint{3.191925in}{2.467964in}}{\pgfqpoint{3.181326in}{2.472354in}}{\pgfqpoint{3.170276in}{2.472354in}}%
\pgfpathcurveto{\pgfqpoint{3.159226in}{2.472354in}}{\pgfqpoint{3.148627in}{2.467964in}}{\pgfqpoint{3.140813in}{2.460150in}}%
\pgfpathcurveto{\pgfqpoint{3.132999in}{2.452337in}}{\pgfqpoint{3.128609in}{2.441738in}}{\pgfqpoint{3.128609in}{2.430688in}}%
\pgfpathcurveto{\pgfqpoint{3.128609in}{2.419637in}}{\pgfqpoint{3.132999in}{2.409038in}}{\pgfqpoint{3.140813in}{2.401225in}}%
\pgfpathcurveto{\pgfqpoint{3.148627in}{2.393411in}}{\pgfqpoint{3.159226in}{2.389021in}}{\pgfqpoint{3.170276in}{2.389021in}}%
\pgfpathclose%
\pgfusepath{stroke,fill}%
\end{pgfscope}%
\begin{pgfscope}%
\pgfpathrectangle{\pgfqpoint{0.481978in}{0.331635in}}{\pgfqpoint{4.960000in}{3.696000in}}%
\pgfusepath{clip}%
\pgfsetbuttcap%
\pgfsetroundjoin%
\definecolor{currentfill}{rgb}{0.631373,0.788235,0.956863}%
\pgfsetfillcolor{currentfill}%
\pgfsetlinewidth{0.481800pt}%
\definecolor{currentstroke}{rgb}{1.000000,1.000000,1.000000}%
\pgfsetstrokecolor{currentstroke}%
\pgfsetdash{}{0pt}%
\pgfpathmoveto{\pgfqpoint{3.797277in}{1.208051in}}%
\pgfpathcurveto{\pgfqpoint{3.808327in}{1.208051in}}{\pgfqpoint{3.818926in}{1.212442in}}{\pgfqpoint{3.826739in}{1.220255in}}%
\pgfpathcurveto{\pgfqpoint{3.834553in}{1.228069in}}{\pgfqpoint{3.838943in}{1.238668in}}{\pgfqpoint{3.838943in}{1.249718in}}%
\pgfpathcurveto{\pgfqpoint{3.838943in}{1.260768in}}{\pgfqpoint{3.834553in}{1.271367in}}{\pgfqpoint{3.826739in}{1.279181in}}%
\pgfpathcurveto{\pgfqpoint{3.818926in}{1.286995in}}{\pgfqpoint{3.808327in}{1.291385in}}{\pgfqpoint{3.797277in}{1.291385in}}%
\pgfpathcurveto{\pgfqpoint{3.786227in}{1.291385in}}{\pgfqpoint{3.775627in}{1.286995in}}{\pgfqpoint{3.767814in}{1.279181in}}%
\pgfpathcurveto{\pgfqpoint{3.760000in}{1.271367in}}{\pgfqpoint{3.755610in}{1.260768in}}{\pgfqpoint{3.755610in}{1.249718in}}%
\pgfpathcurveto{\pgfqpoint{3.755610in}{1.238668in}}{\pgfqpoint{3.760000in}{1.228069in}}{\pgfqpoint{3.767814in}{1.220255in}}%
\pgfpathcurveto{\pgfqpoint{3.775627in}{1.212442in}}{\pgfqpoint{3.786227in}{1.208051in}}{\pgfqpoint{3.797277in}{1.208051in}}%
\pgfpathclose%
\pgfusepath{stroke,fill}%
\end{pgfscope}%
\begin{pgfscope}%
\pgfpathrectangle{\pgfqpoint{0.481978in}{0.331635in}}{\pgfqpoint{4.960000in}{3.696000in}}%
\pgfusepath{clip}%
\pgfsetbuttcap%
\pgfsetroundjoin%
\definecolor{currentfill}{rgb}{0.631373,0.788235,0.956863}%
\pgfsetfillcolor{currentfill}%
\pgfsetlinewidth{0.481800pt}%
\definecolor{currentstroke}{rgb}{1.000000,1.000000,1.000000}%
\pgfsetstrokecolor{currentstroke}%
\pgfsetdash{}{0pt}%
\pgfpathmoveto{\pgfqpoint{2.431161in}{2.118790in}}%
\pgfpathcurveto{\pgfqpoint{2.442211in}{2.118790in}}{\pgfqpoint{2.452810in}{2.123181in}}{\pgfqpoint{2.460624in}{2.130994in}}%
\pgfpathcurveto{\pgfqpoint{2.468437in}{2.138808in}}{\pgfqpoint{2.472828in}{2.149407in}}{\pgfqpoint{2.472828in}{2.160457in}}%
\pgfpathcurveto{\pgfqpoint{2.472828in}{2.171507in}}{\pgfqpoint{2.468437in}{2.182106in}}{\pgfqpoint{2.460624in}{2.189920in}}%
\pgfpathcurveto{\pgfqpoint{2.452810in}{2.197733in}}{\pgfqpoint{2.442211in}{2.202124in}}{\pgfqpoint{2.431161in}{2.202124in}}%
\pgfpathcurveto{\pgfqpoint{2.420111in}{2.202124in}}{\pgfqpoint{2.409512in}{2.197733in}}{\pgfqpoint{2.401698in}{2.189920in}}%
\pgfpathcurveto{\pgfqpoint{2.393884in}{2.182106in}}{\pgfqpoint{2.389494in}{2.171507in}}{\pgfqpoint{2.389494in}{2.160457in}}%
\pgfpathcurveto{\pgfqpoint{2.389494in}{2.149407in}}{\pgfqpoint{2.393884in}{2.138808in}}{\pgfqpoint{2.401698in}{2.130994in}}%
\pgfpathcurveto{\pgfqpoint{2.409512in}{2.123181in}}{\pgfqpoint{2.420111in}{2.118790in}}{\pgfqpoint{2.431161in}{2.118790in}}%
\pgfpathclose%
\pgfusepath{stroke,fill}%
\end{pgfscope}%
\begin{pgfscope}%
\pgfpathrectangle{\pgfqpoint{0.481978in}{0.331635in}}{\pgfqpoint{4.960000in}{3.696000in}}%
\pgfusepath{clip}%
\pgfsetbuttcap%
\pgfsetroundjoin%
\definecolor{currentfill}{rgb}{0.631373,0.788235,0.956863}%
\pgfsetfillcolor{currentfill}%
\pgfsetlinewidth{0.481800pt}%
\definecolor{currentstroke}{rgb}{1.000000,1.000000,1.000000}%
\pgfsetstrokecolor{currentstroke}%
\pgfsetdash{}{0pt}%
\pgfpathmoveto{\pgfqpoint{3.988216in}{3.317337in}}%
\pgfpathcurveto{\pgfqpoint{3.999266in}{3.317337in}}{\pgfqpoint{4.009865in}{3.321727in}}{\pgfqpoint{4.017678in}{3.329541in}}%
\pgfpathcurveto{\pgfqpoint{4.025492in}{3.337354in}}{\pgfqpoint{4.029882in}{3.347953in}}{\pgfqpoint{4.029882in}{3.359003in}}%
\pgfpathcurveto{\pgfqpoint{4.029882in}{3.370053in}}{\pgfqpoint{4.025492in}{3.380652in}}{\pgfqpoint{4.017678in}{3.388466in}}%
\pgfpathcurveto{\pgfqpoint{4.009865in}{3.396280in}}{\pgfqpoint{3.999266in}{3.400670in}}{\pgfqpoint{3.988216in}{3.400670in}}%
\pgfpathcurveto{\pgfqpoint{3.977165in}{3.400670in}}{\pgfqpoint{3.966566in}{3.396280in}}{\pgfqpoint{3.958753in}{3.388466in}}%
\pgfpathcurveto{\pgfqpoint{3.950939in}{3.380652in}}{\pgfqpoint{3.946549in}{3.370053in}}{\pgfqpoint{3.946549in}{3.359003in}}%
\pgfpathcurveto{\pgfqpoint{3.946549in}{3.347953in}}{\pgfqpoint{3.950939in}{3.337354in}}{\pgfqpoint{3.958753in}{3.329541in}}%
\pgfpathcurveto{\pgfqpoint{3.966566in}{3.321727in}}{\pgfqpoint{3.977165in}{3.317337in}}{\pgfqpoint{3.988216in}{3.317337in}}%
\pgfpathclose%
\pgfusepath{stroke,fill}%
\end{pgfscope}%
\begin{pgfscope}%
\pgfpathrectangle{\pgfqpoint{0.481978in}{0.331635in}}{\pgfqpoint{4.960000in}{3.696000in}}%
\pgfusepath{clip}%
\pgfsetbuttcap%
\pgfsetroundjoin%
\definecolor{currentfill}{rgb}{0.631373,0.788235,0.956863}%
\pgfsetfillcolor{currentfill}%
\pgfsetlinewidth{0.481800pt}%
\definecolor{currentstroke}{rgb}{1.000000,1.000000,1.000000}%
\pgfsetstrokecolor{currentstroke}%
\pgfsetdash{}{0pt}%
\pgfpathmoveto{\pgfqpoint{4.509417in}{1.343708in}}%
\pgfpathcurveto{\pgfqpoint{4.520467in}{1.343708in}}{\pgfqpoint{4.531066in}{1.348099in}}{\pgfqpoint{4.538879in}{1.355912in}}%
\pgfpathcurveto{\pgfqpoint{4.546693in}{1.363726in}}{\pgfqpoint{4.551083in}{1.374325in}}{\pgfqpoint{4.551083in}{1.385375in}}%
\pgfpathcurveto{\pgfqpoint{4.551083in}{1.396425in}}{\pgfqpoint{4.546693in}{1.407024in}}{\pgfqpoint{4.538879in}{1.414838in}}%
\pgfpathcurveto{\pgfqpoint{4.531066in}{1.422651in}}{\pgfqpoint{4.520467in}{1.427042in}}{\pgfqpoint{4.509417in}{1.427042in}}%
\pgfpathcurveto{\pgfqpoint{4.498367in}{1.427042in}}{\pgfqpoint{4.487767in}{1.422651in}}{\pgfqpoint{4.479954in}{1.414838in}}%
\pgfpathcurveto{\pgfqpoint{4.472140in}{1.407024in}}{\pgfqpoint{4.467750in}{1.396425in}}{\pgfqpoint{4.467750in}{1.385375in}}%
\pgfpathcurveto{\pgfqpoint{4.467750in}{1.374325in}}{\pgfqpoint{4.472140in}{1.363726in}}{\pgfqpoint{4.479954in}{1.355912in}}%
\pgfpathcurveto{\pgfqpoint{4.487767in}{1.348099in}}{\pgfqpoint{4.498367in}{1.343708in}}{\pgfqpoint{4.509417in}{1.343708in}}%
\pgfpathclose%
\pgfusepath{stroke,fill}%
\end{pgfscope}%
\begin{pgfscope}%
\pgfpathrectangle{\pgfqpoint{0.481978in}{0.331635in}}{\pgfqpoint{4.960000in}{3.696000in}}%
\pgfusepath{clip}%
\pgfsetbuttcap%
\pgfsetroundjoin%
\definecolor{currentfill}{rgb}{0.631373,0.788235,0.956863}%
\pgfsetfillcolor{currentfill}%
\pgfsetlinewidth{0.481800pt}%
\definecolor{currentstroke}{rgb}{1.000000,1.000000,1.000000}%
\pgfsetstrokecolor{currentstroke}%
\pgfsetdash{}{0pt}%
\pgfpathmoveto{\pgfqpoint{3.252727in}{2.098088in}}%
\pgfpathcurveto{\pgfqpoint{3.263777in}{2.098088in}}{\pgfqpoint{3.274376in}{2.102478in}}{\pgfqpoint{3.282190in}{2.110292in}}%
\pgfpathcurveto{\pgfqpoint{3.290003in}{2.118105in}}{\pgfqpoint{3.294394in}{2.128704in}}{\pgfqpoint{3.294394in}{2.139754in}}%
\pgfpathcurveto{\pgfqpoint{3.294394in}{2.150805in}}{\pgfqpoint{3.290003in}{2.161404in}}{\pgfqpoint{3.282190in}{2.169217in}}%
\pgfpathcurveto{\pgfqpoint{3.274376in}{2.177031in}}{\pgfqpoint{3.263777in}{2.181421in}}{\pgfqpoint{3.252727in}{2.181421in}}%
\pgfpathcurveto{\pgfqpoint{3.241677in}{2.181421in}}{\pgfqpoint{3.231078in}{2.177031in}}{\pgfqpoint{3.223264in}{2.169217in}}%
\pgfpathcurveto{\pgfqpoint{3.215450in}{2.161404in}}{\pgfqpoint{3.211060in}{2.150805in}}{\pgfqpoint{3.211060in}{2.139754in}}%
\pgfpathcurveto{\pgfqpoint{3.211060in}{2.128704in}}{\pgfqpoint{3.215450in}{2.118105in}}{\pgfqpoint{3.223264in}{2.110292in}}%
\pgfpathcurveto{\pgfqpoint{3.231078in}{2.102478in}}{\pgfqpoint{3.241677in}{2.098088in}}{\pgfqpoint{3.252727in}{2.098088in}}%
\pgfpathclose%
\pgfusepath{stroke,fill}%
\end{pgfscope}%
\begin{pgfscope}%
\pgfpathrectangle{\pgfqpoint{0.481978in}{0.331635in}}{\pgfqpoint{4.960000in}{3.696000in}}%
\pgfusepath{clip}%
\pgfsetbuttcap%
\pgfsetroundjoin%
\definecolor{currentfill}{rgb}{0.631373,0.788235,0.956863}%
\pgfsetfillcolor{currentfill}%
\pgfsetlinewidth{0.481800pt}%
\definecolor{currentstroke}{rgb}{1.000000,1.000000,1.000000}%
\pgfsetstrokecolor{currentstroke}%
\pgfsetdash{}{0pt}%
\pgfpathmoveto{\pgfqpoint{4.536927in}{3.153019in}}%
\pgfpathcurveto{\pgfqpoint{4.547977in}{3.153019in}}{\pgfqpoint{4.558576in}{3.157409in}}{\pgfqpoint{4.566389in}{3.165223in}}%
\pgfpathcurveto{\pgfqpoint{4.574203in}{3.173036in}}{\pgfqpoint{4.578593in}{3.183635in}}{\pgfqpoint{4.578593in}{3.194685in}}%
\pgfpathcurveto{\pgfqpoint{4.578593in}{3.205735in}}{\pgfqpoint{4.574203in}{3.216334in}}{\pgfqpoint{4.566389in}{3.224148in}}%
\pgfpathcurveto{\pgfqpoint{4.558576in}{3.231962in}}{\pgfqpoint{4.547977in}{3.236352in}}{\pgfqpoint{4.536927in}{3.236352in}}%
\pgfpathcurveto{\pgfqpoint{4.525877in}{3.236352in}}{\pgfqpoint{4.515278in}{3.231962in}}{\pgfqpoint{4.507464in}{3.224148in}}%
\pgfpathcurveto{\pgfqpoint{4.499650in}{3.216334in}}{\pgfqpoint{4.495260in}{3.205735in}}{\pgfqpoint{4.495260in}{3.194685in}}%
\pgfpathcurveto{\pgfqpoint{4.495260in}{3.183635in}}{\pgfqpoint{4.499650in}{3.173036in}}{\pgfqpoint{4.507464in}{3.165223in}}%
\pgfpathcurveto{\pgfqpoint{4.515278in}{3.157409in}}{\pgfqpoint{4.525877in}{3.153019in}}{\pgfqpoint{4.536927in}{3.153019in}}%
\pgfpathclose%
\pgfusepath{stroke,fill}%
\end{pgfscope}%
\begin{pgfscope}%
\pgfpathrectangle{\pgfqpoint{0.481978in}{0.331635in}}{\pgfqpoint{4.960000in}{3.696000in}}%
\pgfusepath{clip}%
\pgfsetbuttcap%
\pgfsetroundjoin%
\definecolor{currentfill}{rgb}{0.631373,0.788235,0.956863}%
\pgfsetfillcolor{currentfill}%
\pgfsetlinewidth{0.481800pt}%
\definecolor{currentstroke}{rgb}{1.000000,1.000000,1.000000}%
\pgfsetstrokecolor{currentstroke}%
\pgfsetdash{}{0pt}%
\pgfpathmoveto{\pgfqpoint{3.471510in}{2.570085in}}%
\pgfpathcurveto{\pgfqpoint{3.482560in}{2.570085in}}{\pgfqpoint{3.493159in}{2.574475in}}{\pgfqpoint{3.500973in}{2.582289in}}%
\pgfpathcurveto{\pgfqpoint{3.508786in}{2.590102in}}{\pgfqpoint{3.513177in}{2.600701in}}{\pgfqpoint{3.513177in}{2.611751in}}%
\pgfpathcurveto{\pgfqpoint{3.513177in}{2.622802in}}{\pgfqpoint{3.508786in}{2.633401in}}{\pgfqpoint{3.500973in}{2.641214in}}%
\pgfpathcurveto{\pgfqpoint{3.493159in}{2.649028in}}{\pgfqpoint{3.482560in}{2.653418in}}{\pgfqpoint{3.471510in}{2.653418in}}%
\pgfpathcurveto{\pgfqpoint{3.460460in}{2.653418in}}{\pgfqpoint{3.449861in}{2.649028in}}{\pgfqpoint{3.442047in}{2.641214in}}%
\pgfpathcurveto{\pgfqpoint{3.434234in}{2.633401in}}{\pgfqpoint{3.429843in}{2.622802in}}{\pgfqpoint{3.429843in}{2.611751in}}%
\pgfpathcurveto{\pgfqpoint{3.429843in}{2.600701in}}{\pgfqpoint{3.434234in}{2.590102in}}{\pgfqpoint{3.442047in}{2.582289in}}%
\pgfpathcurveto{\pgfqpoint{3.449861in}{2.574475in}}{\pgfqpoint{3.460460in}{2.570085in}}{\pgfqpoint{3.471510in}{2.570085in}}%
\pgfpathclose%
\pgfusepath{stroke,fill}%
\end{pgfscope}%
\begin{pgfscope}%
\pgfpathrectangle{\pgfqpoint{0.481978in}{0.331635in}}{\pgfqpoint{4.960000in}{3.696000in}}%
\pgfusepath{clip}%
\pgfsetbuttcap%
\pgfsetroundjoin%
\definecolor{currentfill}{rgb}{0.631373,0.788235,0.956863}%
\pgfsetfillcolor{currentfill}%
\pgfsetlinewidth{0.481800pt}%
\definecolor{currentstroke}{rgb}{1.000000,1.000000,1.000000}%
\pgfsetstrokecolor{currentstroke}%
\pgfsetdash{}{0pt}%
\pgfpathmoveto{\pgfqpoint{2.459931in}{1.717452in}}%
\pgfpathcurveto{\pgfqpoint{2.470981in}{1.717452in}}{\pgfqpoint{2.481580in}{1.721842in}}{\pgfqpoint{2.489394in}{1.729656in}}%
\pgfpathcurveto{\pgfqpoint{2.497208in}{1.737470in}}{\pgfqpoint{2.501598in}{1.748069in}}{\pgfqpoint{2.501598in}{1.759119in}}%
\pgfpathcurveto{\pgfqpoint{2.501598in}{1.770169in}}{\pgfqpoint{2.497208in}{1.780768in}}{\pgfqpoint{2.489394in}{1.788582in}}%
\pgfpathcurveto{\pgfqpoint{2.481580in}{1.796395in}}{\pgfqpoint{2.470981in}{1.800785in}}{\pgfqpoint{2.459931in}{1.800785in}}%
\pgfpathcurveto{\pgfqpoint{2.448881in}{1.800785in}}{\pgfqpoint{2.438282in}{1.796395in}}{\pgfqpoint{2.430468in}{1.788582in}}%
\pgfpathcurveto{\pgfqpoint{2.422655in}{1.780768in}}{\pgfqpoint{2.418265in}{1.770169in}}{\pgfqpoint{2.418265in}{1.759119in}}%
\pgfpathcurveto{\pgfqpoint{2.418265in}{1.748069in}}{\pgfqpoint{2.422655in}{1.737470in}}{\pgfqpoint{2.430468in}{1.729656in}}%
\pgfpathcurveto{\pgfqpoint{2.438282in}{1.721842in}}{\pgfqpoint{2.448881in}{1.717452in}}{\pgfqpoint{2.459931in}{1.717452in}}%
\pgfpathclose%
\pgfusepath{stroke,fill}%
\end{pgfscope}%
\begin{pgfscope}%
\pgfpathrectangle{\pgfqpoint{0.481978in}{0.331635in}}{\pgfqpoint{4.960000in}{3.696000in}}%
\pgfusepath{clip}%
\pgfsetbuttcap%
\pgfsetroundjoin%
\definecolor{currentfill}{rgb}{0.631373,0.788235,0.956863}%
\pgfsetfillcolor{currentfill}%
\pgfsetlinewidth{0.481800pt}%
\definecolor{currentstroke}{rgb}{1.000000,1.000000,1.000000}%
\pgfsetstrokecolor{currentstroke}%
\pgfsetdash{}{0pt}%
\pgfpathmoveto{\pgfqpoint{2.937570in}{1.854576in}}%
\pgfpathcurveto{\pgfqpoint{2.948621in}{1.854576in}}{\pgfqpoint{2.959220in}{1.858966in}}{\pgfqpoint{2.967033in}{1.866779in}}%
\pgfpathcurveto{\pgfqpoint{2.974847in}{1.874593in}}{\pgfqpoint{2.979237in}{1.885192in}}{\pgfqpoint{2.979237in}{1.896242in}}%
\pgfpathcurveto{\pgfqpoint{2.979237in}{1.907292in}}{\pgfqpoint{2.974847in}{1.917891in}}{\pgfqpoint{2.967033in}{1.925705in}}%
\pgfpathcurveto{\pgfqpoint{2.959220in}{1.933519in}}{\pgfqpoint{2.948621in}{1.937909in}}{\pgfqpoint{2.937570in}{1.937909in}}%
\pgfpathcurveto{\pgfqpoint{2.926520in}{1.937909in}}{\pgfqpoint{2.915921in}{1.933519in}}{\pgfqpoint{2.908108in}{1.925705in}}%
\pgfpathcurveto{\pgfqpoint{2.900294in}{1.917891in}}{\pgfqpoint{2.895904in}{1.907292in}}{\pgfqpoint{2.895904in}{1.896242in}}%
\pgfpathcurveto{\pgfqpoint{2.895904in}{1.885192in}}{\pgfqpoint{2.900294in}{1.874593in}}{\pgfqpoint{2.908108in}{1.866779in}}%
\pgfpathcurveto{\pgfqpoint{2.915921in}{1.858966in}}{\pgfqpoint{2.926520in}{1.854576in}}{\pgfqpoint{2.937570in}{1.854576in}}%
\pgfpathclose%
\pgfusepath{stroke,fill}%
\end{pgfscope}%
\begin{pgfscope}%
\pgfpathrectangle{\pgfqpoint{0.481978in}{0.331635in}}{\pgfqpoint{4.960000in}{3.696000in}}%
\pgfusepath{clip}%
\pgfsetbuttcap%
\pgfsetroundjoin%
\definecolor{currentfill}{rgb}{0.631373,0.788235,0.956863}%
\pgfsetfillcolor{currentfill}%
\pgfsetlinewidth{0.481800pt}%
\definecolor{currentstroke}{rgb}{1.000000,1.000000,1.000000}%
\pgfsetstrokecolor{currentstroke}%
\pgfsetdash{}{0pt}%
\pgfpathmoveto{\pgfqpoint{3.802689in}{1.746808in}}%
\pgfpathcurveto{\pgfqpoint{3.813739in}{1.746808in}}{\pgfqpoint{3.824338in}{1.751198in}}{\pgfqpoint{3.832152in}{1.759012in}}%
\pgfpathcurveto{\pgfqpoint{3.839965in}{1.766825in}}{\pgfqpoint{3.844355in}{1.777424in}}{\pgfqpoint{3.844355in}{1.788474in}}%
\pgfpathcurveto{\pgfqpoint{3.844355in}{1.799525in}}{\pgfqpoint{3.839965in}{1.810124in}}{\pgfqpoint{3.832152in}{1.817937in}}%
\pgfpathcurveto{\pgfqpoint{3.824338in}{1.825751in}}{\pgfqpoint{3.813739in}{1.830141in}}{\pgfqpoint{3.802689in}{1.830141in}}%
\pgfpathcurveto{\pgfqpoint{3.791639in}{1.830141in}}{\pgfqpoint{3.781040in}{1.825751in}}{\pgfqpoint{3.773226in}{1.817937in}}%
\pgfpathcurveto{\pgfqpoint{3.765412in}{1.810124in}}{\pgfqpoint{3.761022in}{1.799525in}}{\pgfqpoint{3.761022in}{1.788474in}}%
\pgfpathcurveto{\pgfqpoint{3.761022in}{1.777424in}}{\pgfqpoint{3.765412in}{1.766825in}}{\pgfqpoint{3.773226in}{1.759012in}}%
\pgfpathcurveto{\pgfqpoint{3.781040in}{1.751198in}}{\pgfqpoint{3.791639in}{1.746808in}}{\pgfqpoint{3.802689in}{1.746808in}}%
\pgfpathclose%
\pgfusepath{stroke,fill}%
\end{pgfscope}%
\begin{pgfscope}%
\pgfpathrectangle{\pgfqpoint{0.481978in}{0.331635in}}{\pgfqpoint{4.960000in}{3.696000in}}%
\pgfusepath{clip}%
\pgfsetbuttcap%
\pgfsetroundjoin%
\definecolor{currentfill}{rgb}{0.631373,0.788235,0.956863}%
\pgfsetfillcolor{currentfill}%
\pgfsetlinewidth{0.481800pt}%
\definecolor{currentstroke}{rgb}{1.000000,1.000000,1.000000}%
\pgfsetstrokecolor{currentstroke}%
\pgfsetdash{}{0pt}%
\pgfpathmoveto{\pgfqpoint{3.541785in}{1.498159in}}%
\pgfpathcurveto{\pgfqpoint{3.552835in}{1.498159in}}{\pgfqpoint{3.563434in}{1.502550in}}{\pgfqpoint{3.571248in}{1.510363in}}%
\pgfpathcurveto{\pgfqpoint{3.579061in}{1.518177in}}{\pgfqpoint{3.583452in}{1.528776in}}{\pgfqpoint{3.583452in}{1.539826in}}%
\pgfpathcurveto{\pgfqpoint{3.583452in}{1.550876in}}{\pgfqpoint{3.579061in}{1.561475in}}{\pgfqpoint{3.571248in}{1.569289in}}%
\pgfpathcurveto{\pgfqpoint{3.563434in}{1.577102in}}{\pgfqpoint{3.552835in}{1.581493in}}{\pgfqpoint{3.541785in}{1.581493in}}%
\pgfpathcurveto{\pgfqpoint{3.530735in}{1.581493in}}{\pgfqpoint{3.520136in}{1.577102in}}{\pgfqpoint{3.512322in}{1.569289in}}%
\pgfpathcurveto{\pgfqpoint{3.504509in}{1.561475in}}{\pgfqpoint{3.500118in}{1.550876in}}{\pgfqpoint{3.500118in}{1.539826in}}%
\pgfpathcurveto{\pgfqpoint{3.500118in}{1.528776in}}{\pgfqpoint{3.504509in}{1.518177in}}{\pgfqpoint{3.512322in}{1.510363in}}%
\pgfpathcurveto{\pgfqpoint{3.520136in}{1.502550in}}{\pgfqpoint{3.530735in}{1.498159in}}{\pgfqpoint{3.541785in}{1.498159in}}%
\pgfpathclose%
\pgfusepath{stroke,fill}%
\end{pgfscope}%
\begin{pgfscope}%
\pgfpathrectangle{\pgfqpoint{0.481978in}{0.331635in}}{\pgfqpoint{4.960000in}{3.696000in}}%
\pgfusepath{clip}%
\pgfsetbuttcap%
\pgfsetroundjoin%
\definecolor{currentfill}{rgb}{0.631373,0.788235,0.956863}%
\pgfsetfillcolor{currentfill}%
\pgfsetlinewidth{0.481800pt}%
\definecolor{currentstroke}{rgb}{1.000000,1.000000,1.000000}%
\pgfsetstrokecolor{currentstroke}%
\pgfsetdash{}{0pt}%
\pgfpathmoveto{\pgfqpoint{5.216523in}{2.392773in}}%
\pgfpathcurveto{\pgfqpoint{5.227574in}{2.392773in}}{\pgfqpoint{5.238173in}{2.397163in}}{\pgfqpoint{5.245986in}{2.404976in}}%
\pgfpathcurveto{\pgfqpoint{5.253800in}{2.412790in}}{\pgfqpoint{5.258190in}{2.423389in}}{\pgfqpoint{5.258190in}{2.434439in}}%
\pgfpathcurveto{\pgfqpoint{5.258190in}{2.445489in}}{\pgfqpoint{5.253800in}{2.456088in}}{\pgfqpoint{5.245986in}{2.463902in}}%
\pgfpathcurveto{\pgfqpoint{5.238173in}{2.471716in}}{\pgfqpoint{5.227574in}{2.476106in}}{\pgfqpoint{5.216523in}{2.476106in}}%
\pgfpathcurveto{\pgfqpoint{5.205473in}{2.476106in}}{\pgfqpoint{5.194874in}{2.471716in}}{\pgfqpoint{5.187061in}{2.463902in}}%
\pgfpathcurveto{\pgfqpoint{5.179247in}{2.456088in}}{\pgfqpoint{5.174857in}{2.445489in}}{\pgfqpoint{5.174857in}{2.434439in}}%
\pgfpathcurveto{\pgfqpoint{5.174857in}{2.423389in}}{\pgfqpoint{5.179247in}{2.412790in}}{\pgfqpoint{5.187061in}{2.404976in}}%
\pgfpathcurveto{\pgfqpoint{5.194874in}{2.397163in}}{\pgfqpoint{5.205473in}{2.392773in}}{\pgfqpoint{5.216523in}{2.392773in}}%
\pgfpathclose%
\pgfusepath{stroke,fill}%
\end{pgfscope}%
\begin{pgfscope}%
\pgfpathrectangle{\pgfqpoint{0.481978in}{0.331635in}}{\pgfqpoint{4.960000in}{3.696000in}}%
\pgfusepath{clip}%
\pgfsetbuttcap%
\pgfsetroundjoin%
\definecolor{currentfill}{rgb}{0.631373,0.788235,0.956863}%
\pgfsetfillcolor{currentfill}%
\pgfsetlinewidth{0.481800pt}%
\definecolor{currentstroke}{rgb}{1.000000,1.000000,1.000000}%
\pgfsetstrokecolor{currentstroke}%
\pgfsetdash{}{0pt}%
\pgfpathmoveto{\pgfqpoint{3.043134in}{1.544907in}}%
\pgfpathcurveto{\pgfqpoint{3.054184in}{1.544907in}}{\pgfqpoint{3.064783in}{1.549297in}}{\pgfqpoint{3.072597in}{1.557111in}}%
\pgfpathcurveto{\pgfqpoint{3.080410in}{1.564924in}}{\pgfqpoint{3.084801in}{1.575523in}}{\pgfqpoint{3.084801in}{1.586573in}}%
\pgfpathcurveto{\pgfqpoint{3.084801in}{1.597624in}}{\pgfqpoint{3.080410in}{1.608223in}}{\pgfqpoint{3.072597in}{1.616036in}}%
\pgfpathcurveto{\pgfqpoint{3.064783in}{1.623850in}}{\pgfqpoint{3.054184in}{1.628240in}}{\pgfqpoint{3.043134in}{1.628240in}}%
\pgfpathcurveto{\pgfqpoint{3.032084in}{1.628240in}}{\pgfqpoint{3.021485in}{1.623850in}}{\pgfqpoint{3.013671in}{1.616036in}}%
\pgfpathcurveto{\pgfqpoint{3.005858in}{1.608223in}}{\pgfqpoint{3.001467in}{1.597624in}}{\pgfqpoint{3.001467in}{1.586573in}}%
\pgfpathcurveto{\pgfqpoint{3.001467in}{1.575523in}}{\pgfqpoint{3.005858in}{1.564924in}}{\pgfqpoint{3.013671in}{1.557111in}}%
\pgfpathcurveto{\pgfqpoint{3.021485in}{1.549297in}}{\pgfqpoint{3.032084in}{1.544907in}}{\pgfqpoint{3.043134in}{1.544907in}}%
\pgfpathclose%
\pgfusepath{stroke,fill}%
\end{pgfscope}%
\begin{pgfscope}%
\pgfpathrectangle{\pgfqpoint{0.481978in}{0.331635in}}{\pgfqpoint{4.960000in}{3.696000in}}%
\pgfusepath{clip}%
\pgfsetbuttcap%
\pgfsetroundjoin%
\definecolor{currentfill}{rgb}{0.631373,0.788235,0.956863}%
\pgfsetfillcolor{currentfill}%
\pgfsetlinewidth{0.481800pt}%
\definecolor{currentstroke}{rgb}{1.000000,1.000000,1.000000}%
\pgfsetstrokecolor{currentstroke}%
\pgfsetdash{}{0pt}%
\pgfpathmoveto{\pgfqpoint{2.580652in}{3.142864in}}%
\pgfpathcurveto{\pgfqpoint{2.591703in}{3.142864in}}{\pgfqpoint{2.602302in}{3.147254in}}{\pgfqpoint{2.610115in}{3.155068in}}%
\pgfpathcurveto{\pgfqpoint{2.617929in}{3.162881in}}{\pgfqpoint{2.622319in}{3.173480in}}{\pgfqpoint{2.622319in}{3.184530in}}%
\pgfpathcurveto{\pgfqpoint{2.622319in}{3.195580in}}{\pgfqpoint{2.617929in}{3.206180in}}{\pgfqpoint{2.610115in}{3.213993in}}%
\pgfpathcurveto{\pgfqpoint{2.602302in}{3.221807in}}{\pgfqpoint{2.591703in}{3.226197in}}{\pgfqpoint{2.580652in}{3.226197in}}%
\pgfpathcurveto{\pgfqpoint{2.569602in}{3.226197in}}{\pgfqpoint{2.559003in}{3.221807in}}{\pgfqpoint{2.551190in}{3.213993in}}%
\pgfpathcurveto{\pgfqpoint{2.543376in}{3.206180in}}{\pgfqpoint{2.538986in}{3.195580in}}{\pgfqpoint{2.538986in}{3.184530in}}%
\pgfpathcurveto{\pgfqpoint{2.538986in}{3.173480in}}{\pgfqpoint{2.543376in}{3.162881in}}{\pgfqpoint{2.551190in}{3.155068in}}%
\pgfpathcurveto{\pgfqpoint{2.559003in}{3.147254in}}{\pgfqpoint{2.569602in}{3.142864in}}{\pgfqpoint{2.580652in}{3.142864in}}%
\pgfpathclose%
\pgfusepath{stroke,fill}%
\end{pgfscope}%
\begin{pgfscope}%
\pgfpathrectangle{\pgfqpoint{0.481978in}{0.331635in}}{\pgfqpoint{4.960000in}{3.696000in}}%
\pgfusepath{clip}%
\pgfsetbuttcap%
\pgfsetroundjoin%
\definecolor{currentfill}{rgb}{0.631373,0.788235,0.956863}%
\pgfsetfillcolor{currentfill}%
\pgfsetlinewidth{0.481800pt}%
\definecolor{currentstroke}{rgb}{1.000000,1.000000,1.000000}%
\pgfsetstrokecolor{currentstroke}%
\pgfsetdash{}{0pt}%
\pgfpathmoveto{\pgfqpoint{4.001840in}{1.536698in}}%
\pgfpathcurveto{\pgfqpoint{4.012890in}{1.536698in}}{\pgfqpoint{4.023489in}{1.541088in}}{\pgfqpoint{4.031303in}{1.548902in}}%
\pgfpathcurveto{\pgfqpoint{4.039117in}{1.556715in}}{\pgfqpoint{4.043507in}{1.567314in}}{\pgfqpoint{4.043507in}{1.578365in}}%
\pgfpathcurveto{\pgfqpoint{4.043507in}{1.589415in}}{\pgfqpoint{4.039117in}{1.600014in}}{\pgfqpoint{4.031303in}{1.607827in}}%
\pgfpathcurveto{\pgfqpoint{4.023489in}{1.615641in}}{\pgfqpoint{4.012890in}{1.620031in}}{\pgfqpoint{4.001840in}{1.620031in}}%
\pgfpathcurveto{\pgfqpoint{3.990790in}{1.620031in}}{\pgfqpoint{3.980191in}{1.615641in}}{\pgfqpoint{3.972377in}{1.607827in}}%
\pgfpathcurveto{\pgfqpoint{3.964564in}{1.600014in}}{\pgfqpoint{3.960174in}{1.589415in}}{\pgfqpoint{3.960174in}{1.578365in}}%
\pgfpathcurveto{\pgfqpoint{3.960174in}{1.567314in}}{\pgfqpoint{3.964564in}{1.556715in}}{\pgfqpoint{3.972377in}{1.548902in}}%
\pgfpathcurveto{\pgfqpoint{3.980191in}{1.541088in}}{\pgfqpoint{3.990790in}{1.536698in}}{\pgfqpoint{4.001840in}{1.536698in}}%
\pgfpathclose%
\pgfusepath{stroke,fill}%
\end{pgfscope}%
\begin{pgfscope}%
\pgfpathrectangle{\pgfqpoint{0.481978in}{0.331635in}}{\pgfqpoint{4.960000in}{3.696000in}}%
\pgfusepath{clip}%
\pgfsetbuttcap%
\pgfsetroundjoin%
\definecolor{currentfill}{rgb}{0.631373,0.788235,0.956863}%
\pgfsetfillcolor{currentfill}%
\pgfsetlinewidth{0.481800pt}%
\definecolor{currentstroke}{rgb}{1.000000,1.000000,1.000000}%
\pgfsetstrokecolor{currentstroke}%
\pgfsetdash{}{0pt}%
\pgfpathmoveto{\pgfqpoint{3.872808in}{2.514458in}}%
\pgfpathcurveto{\pgfqpoint{3.883858in}{2.514458in}}{\pgfqpoint{3.894457in}{2.518848in}}{\pgfqpoint{3.902271in}{2.526662in}}%
\pgfpathcurveto{\pgfqpoint{3.910084in}{2.534475in}}{\pgfqpoint{3.914475in}{2.545074in}}{\pgfqpoint{3.914475in}{2.556124in}}%
\pgfpathcurveto{\pgfqpoint{3.914475in}{2.567175in}}{\pgfqpoint{3.910084in}{2.577774in}}{\pgfqpoint{3.902271in}{2.585587in}}%
\pgfpathcurveto{\pgfqpoint{3.894457in}{2.593401in}}{\pgfqpoint{3.883858in}{2.597791in}}{\pgfqpoint{3.872808in}{2.597791in}}%
\pgfpathcurveto{\pgfqpoint{3.861758in}{2.597791in}}{\pgfqpoint{3.851159in}{2.593401in}}{\pgfqpoint{3.843345in}{2.585587in}}%
\pgfpathcurveto{\pgfqpoint{3.835532in}{2.577774in}}{\pgfqpoint{3.831141in}{2.567175in}}{\pgfqpoint{3.831141in}{2.556124in}}%
\pgfpathcurveto{\pgfqpoint{3.831141in}{2.545074in}}{\pgfqpoint{3.835532in}{2.534475in}}{\pgfqpoint{3.843345in}{2.526662in}}%
\pgfpathcurveto{\pgfqpoint{3.851159in}{2.518848in}}{\pgfqpoint{3.861758in}{2.514458in}}{\pgfqpoint{3.872808in}{2.514458in}}%
\pgfpathclose%
\pgfusepath{stroke,fill}%
\end{pgfscope}%
\begin{pgfscope}%
\pgfpathrectangle{\pgfqpoint{0.481978in}{0.331635in}}{\pgfqpoint{4.960000in}{3.696000in}}%
\pgfusepath{clip}%
\pgfsetbuttcap%
\pgfsetroundjoin%
\definecolor{currentfill}{rgb}{0.631373,0.788235,0.956863}%
\pgfsetfillcolor{currentfill}%
\pgfsetlinewidth{0.481800pt}%
\definecolor{currentstroke}{rgb}{1.000000,1.000000,1.000000}%
\pgfsetstrokecolor{currentstroke}%
\pgfsetdash{}{0pt}%
\pgfpathmoveto{\pgfqpoint{3.135350in}{2.736846in}}%
\pgfpathcurveto{\pgfqpoint{3.146400in}{2.736846in}}{\pgfqpoint{3.156999in}{2.741237in}}{\pgfqpoint{3.164813in}{2.749050in}}%
\pgfpathcurveto{\pgfqpoint{3.172627in}{2.756864in}}{\pgfqpoint{3.177017in}{2.767463in}}{\pgfqpoint{3.177017in}{2.778513in}}%
\pgfpathcurveto{\pgfqpoint{3.177017in}{2.789563in}}{\pgfqpoint{3.172627in}{2.800162in}}{\pgfqpoint{3.164813in}{2.807976in}}%
\pgfpathcurveto{\pgfqpoint{3.156999in}{2.815789in}}{\pgfqpoint{3.146400in}{2.820180in}}{\pgfqpoint{3.135350in}{2.820180in}}%
\pgfpathcurveto{\pgfqpoint{3.124300in}{2.820180in}}{\pgfqpoint{3.113701in}{2.815789in}}{\pgfqpoint{3.105888in}{2.807976in}}%
\pgfpathcurveto{\pgfqpoint{3.098074in}{2.800162in}}{\pgfqpoint{3.093684in}{2.789563in}}{\pgfqpoint{3.093684in}{2.778513in}}%
\pgfpathcurveto{\pgfqpoint{3.093684in}{2.767463in}}{\pgfqpoint{3.098074in}{2.756864in}}{\pgfqpoint{3.105888in}{2.749050in}}%
\pgfpathcurveto{\pgfqpoint{3.113701in}{2.741237in}}{\pgfqpoint{3.124300in}{2.736846in}}{\pgfqpoint{3.135350in}{2.736846in}}%
\pgfpathclose%
\pgfusepath{stroke,fill}%
\end{pgfscope}%
\begin{pgfscope}%
\pgfpathrectangle{\pgfqpoint{0.481978in}{0.331635in}}{\pgfqpoint{4.960000in}{3.696000in}}%
\pgfusepath{clip}%
\pgfsetbuttcap%
\pgfsetroundjoin%
\definecolor{currentfill}{rgb}{0.631373,0.788235,0.956863}%
\pgfsetfillcolor{currentfill}%
\pgfsetlinewidth{0.481800pt}%
\definecolor{currentstroke}{rgb}{1.000000,1.000000,1.000000}%
\pgfsetstrokecolor{currentstroke}%
\pgfsetdash{}{0pt}%
\pgfpathmoveto{\pgfqpoint{3.378442in}{1.825081in}}%
\pgfpathcurveto{\pgfqpoint{3.389492in}{1.825081in}}{\pgfqpoint{3.400091in}{1.829471in}}{\pgfqpoint{3.407905in}{1.837285in}}%
\pgfpathcurveto{\pgfqpoint{3.415719in}{1.845098in}}{\pgfqpoint{3.420109in}{1.855697in}}{\pgfqpoint{3.420109in}{1.866748in}}%
\pgfpathcurveto{\pgfqpoint{3.420109in}{1.877798in}}{\pgfqpoint{3.415719in}{1.888397in}}{\pgfqpoint{3.407905in}{1.896210in}}%
\pgfpathcurveto{\pgfqpoint{3.400091in}{1.904024in}}{\pgfqpoint{3.389492in}{1.908414in}}{\pgfqpoint{3.378442in}{1.908414in}}%
\pgfpathcurveto{\pgfqpoint{3.367392in}{1.908414in}}{\pgfqpoint{3.356793in}{1.904024in}}{\pgfqpoint{3.348979in}{1.896210in}}%
\pgfpathcurveto{\pgfqpoint{3.341166in}{1.888397in}}{\pgfqpoint{3.336775in}{1.877798in}}{\pgfqpoint{3.336775in}{1.866748in}}%
\pgfpathcurveto{\pgfqpoint{3.336775in}{1.855697in}}{\pgfqpoint{3.341166in}{1.845098in}}{\pgfqpoint{3.348979in}{1.837285in}}%
\pgfpathcurveto{\pgfqpoint{3.356793in}{1.829471in}}{\pgfqpoint{3.367392in}{1.825081in}}{\pgfqpoint{3.378442in}{1.825081in}}%
\pgfpathclose%
\pgfusepath{stroke,fill}%
\end{pgfscope}%
\begin{pgfscope}%
\pgfpathrectangle{\pgfqpoint{0.481978in}{0.331635in}}{\pgfqpoint{4.960000in}{3.696000in}}%
\pgfusepath{clip}%
\pgfsetbuttcap%
\pgfsetroundjoin%
\definecolor{currentfill}{rgb}{0.631373,0.788235,0.956863}%
\pgfsetfillcolor{currentfill}%
\pgfsetlinewidth{0.481800pt}%
\definecolor{currentstroke}{rgb}{1.000000,1.000000,1.000000}%
\pgfsetstrokecolor{currentstroke}%
\pgfsetdash{}{0pt}%
\pgfpathmoveto{\pgfqpoint{3.656219in}{2.033922in}}%
\pgfpathcurveto{\pgfqpoint{3.667269in}{2.033922in}}{\pgfqpoint{3.677868in}{2.038312in}}{\pgfqpoint{3.685682in}{2.046126in}}%
\pgfpathcurveto{\pgfqpoint{3.693495in}{2.053939in}}{\pgfqpoint{3.697885in}{2.064538in}}{\pgfqpoint{3.697885in}{2.075588in}}%
\pgfpathcurveto{\pgfqpoint{3.697885in}{2.086638in}}{\pgfqpoint{3.693495in}{2.097238in}}{\pgfqpoint{3.685682in}{2.105051in}}%
\pgfpathcurveto{\pgfqpoint{3.677868in}{2.112865in}}{\pgfqpoint{3.667269in}{2.117255in}}{\pgfqpoint{3.656219in}{2.117255in}}%
\pgfpathcurveto{\pgfqpoint{3.645169in}{2.117255in}}{\pgfqpoint{3.634570in}{2.112865in}}{\pgfqpoint{3.626756in}{2.105051in}}%
\pgfpathcurveto{\pgfqpoint{3.618942in}{2.097238in}}{\pgfqpoint{3.614552in}{2.086638in}}{\pgfqpoint{3.614552in}{2.075588in}}%
\pgfpathcurveto{\pgfqpoint{3.614552in}{2.064538in}}{\pgfqpoint{3.618942in}{2.053939in}}{\pgfqpoint{3.626756in}{2.046126in}}%
\pgfpathcurveto{\pgfqpoint{3.634570in}{2.038312in}}{\pgfqpoint{3.645169in}{2.033922in}}{\pgfqpoint{3.656219in}{2.033922in}}%
\pgfpathclose%
\pgfusepath{stroke,fill}%
\end{pgfscope}%
\begin{pgfscope}%
\pgfpathrectangle{\pgfqpoint{0.481978in}{0.331635in}}{\pgfqpoint{4.960000in}{3.696000in}}%
\pgfusepath{clip}%
\pgfsetbuttcap%
\pgfsetroundjoin%
\definecolor{currentfill}{rgb}{0.631373,0.788235,0.956863}%
\pgfsetfillcolor{currentfill}%
\pgfsetlinewidth{0.481800pt}%
\definecolor{currentstroke}{rgb}{1.000000,1.000000,1.000000}%
\pgfsetstrokecolor{currentstroke}%
\pgfsetdash{}{0pt}%
\pgfpathmoveto{\pgfqpoint{3.440189in}{2.967561in}}%
\pgfpathcurveto{\pgfqpoint{3.451239in}{2.967561in}}{\pgfqpoint{3.461838in}{2.971951in}}{\pgfqpoint{3.469652in}{2.979765in}}%
\pgfpathcurveto{\pgfqpoint{3.477466in}{2.987578in}}{\pgfqpoint{3.481856in}{2.998177in}}{\pgfqpoint{3.481856in}{3.009227in}}%
\pgfpathcurveto{\pgfqpoint{3.481856in}{3.020278in}}{\pgfqpoint{3.477466in}{3.030877in}}{\pgfqpoint{3.469652in}{3.038690in}}%
\pgfpathcurveto{\pgfqpoint{3.461838in}{3.046504in}}{\pgfqpoint{3.451239in}{3.050894in}}{\pgfqpoint{3.440189in}{3.050894in}}%
\pgfpathcurveto{\pgfqpoint{3.429139in}{3.050894in}}{\pgfqpoint{3.418540in}{3.046504in}}{\pgfqpoint{3.410727in}{3.038690in}}%
\pgfpathcurveto{\pgfqpoint{3.402913in}{3.030877in}}{\pgfqpoint{3.398523in}{3.020278in}}{\pgfqpoint{3.398523in}{3.009227in}}%
\pgfpathcurveto{\pgfqpoint{3.398523in}{2.998177in}}{\pgfqpoint{3.402913in}{2.987578in}}{\pgfqpoint{3.410727in}{2.979765in}}%
\pgfpathcurveto{\pgfqpoint{3.418540in}{2.971951in}}{\pgfqpoint{3.429139in}{2.967561in}}{\pgfqpoint{3.440189in}{2.967561in}}%
\pgfpathclose%
\pgfusepath{stroke,fill}%
\end{pgfscope}%
\begin{pgfscope}%
\pgfpathrectangle{\pgfqpoint{0.481978in}{0.331635in}}{\pgfqpoint{4.960000in}{3.696000in}}%
\pgfusepath{clip}%
\pgfsetbuttcap%
\pgfsetroundjoin%
\definecolor{currentfill}{rgb}{0.631373,0.788235,0.956863}%
\pgfsetfillcolor{currentfill}%
\pgfsetlinewidth{0.481800pt}%
\definecolor{currentstroke}{rgb}{1.000000,1.000000,1.000000}%
\pgfsetstrokecolor{currentstroke}%
\pgfsetdash{}{0pt}%
\pgfpathmoveto{\pgfqpoint{4.280646in}{1.811485in}}%
\pgfpathcurveto{\pgfqpoint{4.291696in}{1.811485in}}{\pgfqpoint{4.302295in}{1.815876in}}{\pgfqpoint{4.310109in}{1.823689in}}%
\pgfpathcurveto{\pgfqpoint{4.317922in}{1.831503in}}{\pgfqpoint{4.322313in}{1.842102in}}{\pgfqpoint{4.322313in}{1.853152in}}%
\pgfpathcurveto{\pgfqpoint{4.322313in}{1.864202in}}{\pgfqpoint{4.317922in}{1.874801in}}{\pgfqpoint{4.310109in}{1.882615in}}%
\pgfpathcurveto{\pgfqpoint{4.302295in}{1.890428in}}{\pgfqpoint{4.291696in}{1.894819in}}{\pgfqpoint{4.280646in}{1.894819in}}%
\pgfpathcurveto{\pgfqpoint{4.269596in}{1.894819in}}{\pgfqpoint{4.258997in}{1.890428in}}{\pgfqpoint{4.251183in}{1.882615in}}%
\pgfpathcurveto{\pgfqpoint{4.243370in}{1.874801in}}{\pgfqpoint{4.238979in}{1.864202in}}{\pgfqpoint{4.238979in}{1.853152in}}%
\pgfpathcurveto{\pgfqpoint{4.238979in}{1.842102in}}{\pgfqpoint{4.243370in}{1.831503in}}{\pgfqpoint{4.251183in}{1.823689in}}%
\pgfpathcurveto{\pgfqpoint{4.258997in}{1.815876in}}{\pgfqpoint{4.269596in}{1.811485in}}{\pgfqpoint{4.280646in}{1.811485in}}%
\pgfpathclose%
\pgfusepath{stroke,fill}%
\end{pgfscope}%
\begin{pgfscope}%
\pgfpathrectangle{\pgfqpoint{0.481978in}{0.331635in}}{\pgfqpoint{4.960000in}{3.696000in}}%
\pgfusepath{clip}%
\pgfsetbuttcap%
\pgfsetroundjoin%
\definecolor{currentfill}{rgb}{0.631373,0.788235,0.956863}%
\pgfsetfillcolor{currentfill}%
\pgfsetlinewidth{0.481800pt}%
\definecolor{currentstroke}{rgb}{1.000000,1.000000,1.000000}%
\pgfsetstrokecolor{currentstroke}%
\pgfsetdash{}{0pt}%
\pgfpathmoveto{\pgfqpoint{2.790063in}{2.195412in}}%
\pgfpathcurveto{\pgfqpoint{2.801113in}{2.195412in}}{\pgfqpoint{2.811712in}{2.199802in}}{\pgfqpoint{2.819526in}{2.207616in}}%
\pgfpathcurveto{\pgfqpoint{2.827340in}{2.215429in}}{\pgfqpoint{2.831730in}{2.226028in}}{\pgfqpoint{2.831730in}{2.237079in}}%
\pgfpathcurveto{\pgfqpoint{2.831730in}{2.248129in}}{\pgfqpoint{2.827340in}{2.258728in}}{\pgfqpoint{2.819526in}{2.266541in}}%
\pgfpathcurveto{\pgfqpoint{2.811712in}{2.274355in}}{\pgfqpoint{2.801113in}{2.278745in}}{\pgfqpoint{2.790063in}{2.278745in}}%
\pgfpathcurveto{\pgfqpoint{2.779013in}{2.278745in}}{\pgfqpoint{2.768414in}{2.274355in}}{\pgfqpoint{2.760600in}{2.266541in}}%
\pgfpathcurveto{\pgfqpoint{2.752787in}{2.258728in}}{\pgfqpoint{2.748397in}{2.248129in}}{\pgfqpoint{2.748397in}{2.237079in}}%
\pgfpathcurveto{\pgfqpoint{2.748397in}{2.226028in}}{\pgfqpoint{2.752787in}{2.215429in}}{\pgfqpoint{2.760600in}{2.207616in}}%
\pgfpathcurveto{\pgfqpoint{2.768414in}{2.199802in}}{\pgfqpoint{2.779013in}{2.195412in}}{\pgfqpoint{2.790063in}{2.195412in}}%
\pgfpathclose%
\pgfusepath{stroke,fill}%
\end{pgfscope}%
\begin{pgfscope}%
\pgfpathrectangle{\pgfqpoint{0.481978in}{0.331635in}}{\pgfqpoint{4.960000in}{3.696000in}}%
\pgfusepath{clip}%
\pgfsetbuttcap%
\pgfsetroundjoin%
\definecolor{currentfill}{rgb}{0.631373,0.788235,0.956863}%
\pgfsetfillcolor{currentfill}%
\pgfsetlinewidth{0.481800pt}%
\definecolor{currentstroke}{rgb}{1.000000,1.000000,1.000000}%
\pgfsetstrokecolor{currentstroke}%
\pgfsetdash{}{0pt}%
\pgfpathmoveto{\pgfqpoint{4.256551in}{2.832602in}}%
\pgfpathcurveto{\pgfqpoint{4.267602in}{2.832602in}}{\pgfqpoint{4.278201in}{2.836992in}}{\pgfqpoint{4.286014in}{2.844806in}}%
\pgfpathcurveto{\pgfqpoint{4.293828in}{2.852619in}}{\pgfqpoint{4.298218in}{2.863218in}}{\pgfqpoint{4.298218in}{2.874269in}}%
\pgfpathcurveto{\pgfqpoint{4.298218in}{2.885319in}}{\pgfqpoint{4.293828in}{2.895918in}}{\pgfqpoint{4.286014in}{2.903731in}}%
\pgfpathcurveto{\pgfqpoint{4.278201in}{2.911545in}}{\pgfqpoint{4.267602in}{2.915935in}}{\pgfqpoint{4.256551in}{2.915935in}}%
\pgfpathcurveto{\pgfqpoint{4.245501in}{2.915935in}}{\pgfqpoint{4.234902in}{2.911545in}}{\pgfqpoint{4.227089in}{2.903731in}}%
\pgfpathcurveto{\pgfqpoint{4.219275in}{2.895918in}}{\pgfqpoint{4.214885in}{2.885319in}}{\pgfqpoint{4.214885in}{2.874269in}}%
\pgfpathcurveto{\pgfqpoint{4.214885in}{2.863218in}}{\pgfqpoint{4.219275in}{2.852619in}}{\pgfqpoint{4.227089in}{2.844806in}}%
\pgfpathcurveto{\pgfqpoint{4.234902in}{2.836992in}}{\pgfqpoint{4.245501in}{2.832602in}}{\pgfqpoint{4.256551in}{2.832602in}}%
\pgfpathclose%
\pgfusepath{stroke,fill}%
\end{pgfscope}%
\begin{pgfscope}%
\pgfpathrectangle{\pgfqpoint{0.481978in}{0.331635in}}{\pgfqpoint{4.960000in}{3.696000in}}%
\pgfusepath{clip}%
\pgfsetbuttcap%
\pgfsetroundjoin%
\definecolor{currentfill}{rgb}{0.631373,0.788235,0.956863}%
\pgfsetfillcolor{currentfill}%
\pgfsetlinewidth{0.481800pt}%
\definecolor{currentstroke}{rgb}{1.000000,1.000000,1.000000}%
\pgfsetstrokecolor{currentstroke}%
\pgfsetdash{}{0pt}%
\pgfpathmoveto{\pgfqpoint{4.575205in}{2.151472in}}%
\pgfpathcurveto{\pgfqpoint{4.586255in}{2.151472in}}{\pgfqpoint{4.596854in}{2.155862in}}{\pgfqpoint{4.604668in}{2.163676in}}%
\pgfpathcurveto{\pgfqpoint{4.612482in}{2.171490in}}{\pgfqpoint{4.616872in}{2.182089in}}{\pgfqpoint{4.616872in}{2.193139in}}%
\pgfpathcurveto{\pgfqpoint{4.616872in}{2.204189in}}{\pgfqpoint{4.612482in}{2.214788in}}{\pgfqpoint{4.604668in}{2.222601in}}%
\pgfpathcurveto{\pgfqpoint{4.596854in}{2.230415in}}{\pgfqpoint{4.586255in}{2.234805in}}{\pgfqpoint{4.575205in}{2.234805in}}%
\pgfpathcurveto{\pgfqpoint{4.564155in}{2.234805in}}{\pgfqpoint{4.553556in}{2.230415in}}{\pgfqpoint{4.545742in}{2.222601in}}%
\pgfpathcurveto{\pgfqpoint{4.537929in}{2.214788in}}{\pgfqpoint{4.533539in}{2.204189in}}{\pgfqpoint{4.533539in}{2.193139in}}%
\pgfpathcurveto{\pgfqpoint{4.533539in}{2.182089in}}{\pgfqpoint{4.537929in}{2.171490in}}{\pgfqpoint{4.545742in}{2.163676in}}%
\pgfpathcurveto{\pgfqpoint{4.553556in}{2.155862in}}{\pgfqpoint{4.564155in}{2.151472in}}{\pgfqpoint{4.575205in}{2.151472in}}%
\pgfpathclose%
\pgfusepath{stroke,fill}%
\end{pgfscope}%
\begin{pgfscope}%
\pgfpathrectangle{\pgfqpoint{0.481978in}{0.331635in}}{\pgfqpoint{4.960000in}{3.696000in}}%
\pgfusepath{clip}%
\pgfsetbuttcap%
\pgfsetroundjoin%
\definecolor{currentfill}{rgb}{0.631373,0.788235,0.956863}%
\pgfsetfillcolor{currentfill}%
\pgfsetlinewidth{0.481800pt}%
\definecolor{currentstroke}{rgb}{1.000000,1.000000,1.000000}%
\pgfsetstrokecolor{currentstroke}%
\pgfsetdash{}{0pt}%
\pgfpathmoveto{\pgfqpoint{3.448245in}{3.271854in}}%
\pgfpathcurveto{\pgfqpoint{3.459295in}{3.271854in}}{\pgfqpoint{3.469894in}{3.276244in}}{\pgfqpoint{3.477708in}{3.284058in}}%
\pgfpathcurveto{\pgfqpoint{3.485521in}{3.291871in}}{\pgfqpoint{3.489911in}{3.302470in}}{\pgfqpoint{3.489911in}{3.313520in}}%
\pgfpathcurveto{\pgfqpoint{3.489911in}{3.324571in}}{\pgfqpoint{3.485521in}{3.335170in}}{\pgfqpoint{3.477708in}{3.342983in}}%
\pgfpathcurveto{\pgfqpoint{3.469894in}{3.350797in}}{\pgfqpoint{3.459295in}{3.355187in}}{\pgfqpoint{3.448245in}{3.355187in}}%
\pgfpathcurveto{\pgfqpoint{3.437195in}{3.355187in}}{\pgfqpoint{3.426596in}{3.350797in}}{\pgfqpoint{3.418782in}{3.342983in}}%
\pgfpathcurveto{\pgfqpoint{3.410968in}{3.335170in}}{\pgfqpoint{3.406578in}{3.324571in}}{\pgfqpoint{3.406578in}{3.313520in}}%
\pgfpathcurveto{\pgfqpoint{3.406578in}{3.302470in}}{\pgfqpoint{3.410968in}{3.291871in}}{\pgfqpoint{3.418782in}{3.284058in}}%
\pgfpathcurveto{\pgfqpoint{3.426596in}{3.276244in}}{\pgfqpoint{3.437195in}{3.271854in}}{\pgfqpoint{3.448245in}{3.271854in}}%
\pgfpathclose%
\pgfusepath{stroke,fill}%
\end{pgfscope}%
\begin{pgfscope}%
\pgfpathrectangle{\pgfqpoint{0.481978in}{0.331635in}}{\pgfqpoint{4.960000in}{3.696000in}}%
\pgfusepath{clip}%
\pgfsetbuttcap%
\pgfsetroundjoin%
\definecolor{currentfill}{rgb}{0.631373,0.788235,0.956863}%
\pgfsetfillcolor{currentfill}%
\pgfsetlinewidth{0.481800pt}%
\definecolor{currentstroke}{rgb}{1.000000,1.000000,1.000000}%
\pgfsetstrokecolor{currentstroke}%
\pgfsetdash{}{0pt}%
\pgfpathmoveto{\pgfqpoint{2.657795in}{1.338091in}}%
\pgfpathcurveto{\pgfqpoint{2.668845in}{1.338091in}}{\pgfqpoint{2.679444in}{1.342481in}}{\pgfqpoint{2.687258in}{1.350295in}}%
\pgfpathcurveto{\pgfqpoint{2.695071in}{1.358109in}}{\pgfqpoint{2.699462in}{1.368708in}}{\pgfqpoint{2.699462in}{1.379758in}}%
\pgfpathcurveto{\pgfqpoint{2.699462in}{1.390808in}}{\pgfqpoint{2.695071in}{1.401407in}}{\pgfqpoint{2.687258in}{1.409221in}}%
\pgfpathcurveto{\pgfqpoint{2.679444in}{1.417034in}}{\pgfqpoint{2.668845in}{1.421424in}}{\pgfqpoint{2.657795in}{1.421424in}}%
\pgfpathcurveto{\pgfqpoint{2.646745in}{1.421424in}}{\pgfqpoint{2.636146in}{1.417034in}}{\pgfqpoint{2.628332in}{1.409221in}}%
\pgfpathcurveto{\pgfqpoint{2.620519in}{1.401407in}}{\pgfqpoint{2.616128in}{1.390808in}}{\pgfqpoint{2.616128in}{1.379758in}}%
\pgfpathcurveto{\pgfqpoint{2.616128in}{1.368708in}}{\pgfqpoint{2.620519in}{1.358109in}}{\pgfqpoint{2.628332in}{1.350295in}}%
\pgfpathcurveto{\pgfqpoint{2.636146in}{1.342481in}}{\pgfqpoint{2.646745in}{1.338091in}}{\pgfqpoint{2.657795in}{1.338091in}}%
\pgfpathclose%
\pgfusepath{stroke,fill}%
\end{pgfscope}%
\begin{pgfscope}%
\pgfpathrectangle{\pgfqpoint{0.481978in}{0.331635in}}{\pgfqpoint{4.960000in}{3.696000in}}%
\pgfusepath{clip}%
\pgfsetbuttcap%
\pgfsetroundjoin%
\definecolor{currentfill}{rgb}{0.631373,0.788235,0.956863}%
\pgfsetfillcolor{currentfill}%
\pgfsetlinewidth{0.481800pt}%
\definecolor{currentstroke}{rgb}{1.000000,1.000000,1.000000}%
\pgfsetstrokecolor{currentstroke}%
\pgfsetdash{}{0pt}%
\pgfpathmoveto{\pgfqpoint{3.546339in}{2.299935in}}%
\pgfpathcurveto{\pgfqpoint{3.557389in}{2.299935in}}{\pgfqpoint{3.567988in}{2.304326in}}{\pgfqpoint{3.575802in}{2.312139in}}%
\pgfpathcurveto{\pgfqpoint{3.583616in}{2.319953in}}{\pgfqpoint{3.588006in}{2.330552in}}{\pgfqpoint{3.588006in}{2.341602in}}%
\pgfpathcurveto{\pgfqpoint{3.588006in}{2.352652in}}{\pgfqpoint{3.583616in}{2.363251in}}{\pgfqpoint{3.575802in}{2.371065in}}%
\pgfpathcurveto{\pgfqpoint{3.567988in}{2.378878in}}{\pgfqpoint{3.557389in}{2.383269in}}{\pgfqpoint{3.546339in}{2.383269in}}%
\pgfpathcurveto{\pgfqpoint{3.535289in}{2.383269in}}{\pgfqpoint{3.524690in}{2.378878in}}{\pgfqpoint{3.516876in}{2.371065in}}%
\pgfpathcurveto{\pgfqpoint{3.509063in}{2.363251in}}{\pgfqpoint{3.504673in}{2.352652in}}{\pgfqpoint{3.504673in}{2.341602in}}%
\pgfpathcurveto{\pgfqpoint{3.504673in}{2.330552in}}{\pgfqpoint{3.509063in}{2.319953in}}{\pgfqpoint{3.516876in}{2.312139in}}%
\pgfpathcurveto{\pgfqpoint{3.524690in}{2.304326in}}{\pgfqpoint{3.535289in}{2.299935in}}{\pgfqpoint{3.546339in}{2.299935in}}%
\pgfpathclose%
\pgfusepath{stroke,fill}%
\end{pgfscope}%
\begin{pgfscope}%
\pgfpathrectangle{\pgfqpoint{0.481978in}{0.331635in}}{\pgfqpoint{4.960000in}{3.696000in}}%
\pgfusepath{clip}%
\pgfsetbuttcap%
\pgfsetroundjoin%
\definecolor{currentfill}{rgb}{1.000000,0.705882,0.509804}%
\pgfsetfillcolor{currentfill}%
\pgfsetlinewidth{0.481800pt}%
\definecolor{currentstroke}{rgb}{1.000000,1.000000,1.000000}%
\pgfsetstrokecolor{currentstroke}%
\pgfsetdash{}{0pt}%
\pgfpathmoveto{\pgfqpoint{3.014752in}{3.153624in}}%
\pgfpathcurveto{\pgfqpoint{3.025802in}{3.153624in}}{\pgfqpoint{3.036401in}{3.158014in}}{\pgfqpoint{3.044215in}{3.165828in}}%
\pgfpathcurveto{\pgfqpoint{3.052028in}{3.173642in}}{\pgfqpoint{3.056418in}{3.184241in}}{\pgfqpoint{3.056418in}{3.195291in}}%
\pgfpathcurveto{\pgfqpoint{3.056418in}{3.206341in}}{\pgfqpoint{3.052028in}{3.216940in}}{\pgfqpoint{3.044215in}{3.224754in}}%
\pgfpathcurveto{\pgfqpoint{3.036401in}{3.232567in}}{\pgfqpoint{3.025802in}{3.236957in}}{\pgfqpoint{3.014752in}{3.236957in}}%
\pgfpathcurveto{\pgfqpoint{3.003702in}{3.236957in}}{\pgfqpoint{2.993103in}{3.232567in}}{\pgfqpoint{2.985289in}{3.224754in}}%
\pgfpathcurveto{\pgfqpoint{2.977475in}{3.216940in}}{\pgfqpoint{2.973085in}{3.206341in}}{\pgfqpoint{2.973085in}{3.195291in}}%
\pgfpathcurveto{\pgfqpoint{2.973085in}{3.184241in}}{\pgfqpoint{2.977475in}{3.173642in}}{\pgfqpoint{2.985289in}{3.165828in}}%
\pgfpathcurveto{\pgfqpoint{2.993103in}{3.158014in}}{\pgfqpoint{3.003702in}{3.153624in}}{\pgfqpoint{3.014752in}{3.153624in}}%
\pgfpathclose%
\pgfusepath{stroke,fill}%
\end{pgfscope}%
\begin{pgfscope}%
\pgfpathrectangle{\pgfqpoint{0.481978in}{0.331635in}}{\pgfqpoint{4.960000in}{3.696000in}}%
\pgfusepath{clip}%
\pgfsetbuttcap%
\pgfsetroundjoin%
\definecolor{currentfill}{rgb}{1.000000,0.705882,0.509804}%
\pgfsetfillcolor{currentfill}%
\pgfsetlinewidth{0.481800pt}%
\definecolor{currentstroke}{rgb}{1.000000,1.000000,1.000000}%
\pgfsetstrokecolor{currentstroke}%
\pgfsetdash{}{0pt}%
\pgfpathmoveto{\pgfqpoint{5.085754in}{1.751706in}}%
\pgfpathcurveto{\pgfqpoint{5.096804in}{1.751706in}}{\pgfqpoint{5.107403in}{1.756096in}}{\pgfqpoint{5.115216in}{1.763910in}}%
\pgfpathcurveto{\pgfqpoint{5.123030in}{1.771723in}}{\pgfqpoint{5.127420in}{1.782322in}}{\pgfqpoint{5.127420in}{1.793372in}}%
\pgfpathcurveto{\pgfqpoint{5.127420in}{1.804422in}}{\pgfqpoint{5.123030in}{1.815021in}}{\pgfqpoint{5.115216in}{1.822835in}}%
\pgfpathcurveto{\pgfqpoint{5.107403in}{1.830649in}}{\pgfqpoint{5.096804in}{1.835039in}}{\pgfqpoint{5.085754in}{1.835039in}}%
\pgfpathcurveto{\pgfqpoint{5.074703in}{1.835039in}}{\pgfqpoint{5.064104in}{1.830649in}}{\pgfqpoint{5.056291in}{1.822835in}}%
\pgfpathcurveto{\pgfqpoint{5.048477in}{1.815021in}}{\pgfqpoint{5.044087in}{1.804422in}}{\pgfqpoint{5.044087in}{1.793372in}}%
\pgfpathcurveto{\pgfqpoint{5.044087in}{1.782322in}}{\pgfqpoint{5.048477in}{1.771723in}}{\pgfqpoint{5.056291in}{1.763910in}}%
\pgfpathcurveto{\pgfqpoint{5.064104in}{1.756096in}}{\pgfqpoint{5.074703in}{1.751706in}}{\pgfqpoint{5.085754in}{1.751706in}}%
\pgfpathclose%
\pgfusepath{stroke,fill}%
\end{pgfscope}%
\begin{pgfscope}%
\pgfpathrectangle{\pgfqpoint{0.481978in}{0.331635in}}{\pgfqpoint{4.960000in}{3.696000in}}%
\pgfusepath{clip}%
\pgfsetbuttcap%
\pgfsetroundjoin%
\definecolor{currentfill}{rgb}{1.000000,0.705882,0.509804}%
\pgfsetfillcolor{currentfill}%
\pgfsetlinewidth{0.481800pt}%
\definecolor{currentstroke}{rgb}{1.000000,1.000000,1.000000}%
\pgfsetstrokecolor{currentstroke}%
\pgfsetdash{}{0pt}%
\pgfpathmoveto{\pgfqpoint{2.284015in}{2.558577in}}%
\pgfpathcurveto{\pgfqpoint{2.295066in}{2.558577in}}{\pgfqpoint{2.305665in}{2.562967in}}{\pgfqpoint{2.313478in}{2.570781in}}%
\pgfpathcurveto{\pgfqpoint{2.321292in}{2.578594in}}{\pgfqpoint{2.325682in}{2.589193in}}{\pgfqpoint{2.325682in}{2.600243in}}%
\pgfpathcurveto{\pgfqpoint{2.325682in}{2.611293in}}{\pgfqpoint{2.321292in}{2.621892in}}{\pgfqpoint{2.313478in}{2.629706in}}%
\pgfpathcurveto{\pgfqpoint{2.305665in}{2.637520in}}{\pgfqpoint{2.295066in}{2.641910in}}{\pgfqpoint{2.284015in}{2.641910in}}%
\pgfpathcurveto{\pgfqpoint{2.272965in}{2.641910in}}{\pgfqpoint{2.262366in}{2.637520in}}{\pgfqpoint{2.254553in}{2.629706in}}%
\pgfpathcurveto{\pgfqpoint{2.246739in}{2.621892in}}{\pgfqpoint{2.242349in}{2.611293in}}{\pgfqpoint{2.242349in}{2.600243in}}%
\pgfpathcurveto{\pgfqpoint{2.242349in}{2.589193in}}{\pgfqpoint{2.246739in}{2.578594in}}{\pgfqpoint{2.254553in}{2.570781in}}%
\pgfpathcurveto{\pgfqpoint{2.262366in}{2.562967in}}{\pgfqpoint{2.272965in}{2.558577in}}{\pgfqpoint{2.284015in}{2.558577in}}%
\pgfpathclose%
\pgfusepath{stroke,fill}%
\end{pgfscope}%
\begin{pgfscope}%
\pgfpathrectangle{\pgfqpoint{0.481978in}{0.331635in}}{\pgfqpoint{4.960000in}{3.696000in}}%
\pgfusepath{clip}%
\pgfsetbuttcap%
\pgfsetroundjoin%
\definecolor{currentfill}{rgb}{1.000000,0.705882,0.509804}%
\pgfsetfillcolor{currentfill}%
\pgfsetlinewidth{0.481800pt}%
\definecolor{currentstroke}{rgb}{1.000000,1.000000,1.000000}%
\pgfsetstrokecolor{currentstroke}%
\pgfsetdash{}{0pt}%
\pgfpathmoveto{\pgfqpoint{1.922934in}{2.324891in}}%
\pgfpathcurveto{\pgfqpoint{1.933984in}{2.324891in}}{\pgfqpoint{1.944583in}{2.329282in}}{\pgfqpoint{1.952397in}{2.337095in}}%
\pgfpathcurveto{\pgfqpoint{1.960211in}{2.344909in}}{\pgfqpoint{1.964601in}{2.355508in}}{\pgfqpoint{1.964601in}{2.366558in}}%
\pgfpathcurveto{\pgfqpoint{1.964601in}{2.377608in}}{\pgfqpoint{1.960211in}{2.388207in}}{\pgfqpoint{1.952397in}{2.396021in}}%
\pgfpathcurveto{\pgfqpoint{1.944583in}{2.403834in}}{\pgfqpoint{1.933984in}{2.408225in}}{\pgfqpoint{1.922934in}{2.408225in}}%
\pgfpathcurveto{\pgfqpoint{1.911884in}{2.408225in}}{\pgfqpoint{1.901285in}{2.403834in}}{\pgfqpoint{1.893472in}{2.396021in}}%
\pgfpathcurveto{\pgfqpoint{1.885658in}{2.388207in}}{\pgfqpoint{1.881268in}{2.377608in}}{\pgfqpoint{1.881268in}{2.366558in}}%
\pgfpathcurveto{\pgfqpoint{1.881268in}{2.355508in}}{\pgfqpoint{1.885658in}{2.344909in}}{\pgfqpoint{1.893472in}{2.337095in}}%
\pgfpathcurveto{\pgfqpoint{1.901285in}{2.329282in}}{\pgfqpoint{1.911884in}{2.324891in}}{\pgfqpoint{1.922934in}{2.324891in}}%
\pgfpathclose%
\pgfusepath{stroke,fill}%
\end{pgfscope}%
\begin{pgfscope}%
\pgfpathrectangle{\pgfqpoint{0.481978in}{0.331635in}}{\pgfqpoint{4.960000in}{3.696000in}}%
\pgfusepath{clip}%
\pgfsetbuttcap%
\pgfsetroundjoin%
\definecolor{currentfill}{rgb}{1.000000,0.705882,0.509804}%
\pgfsetfillcolor{currentfill}%
\pgfsetlinewidth{0.481800pt}%
\definecolor{currentstroke}{rgb}{1.000000,1.000000,1.000000}%
\pgfsetstrokecolor{currentstroke}%
\pgfsetdash{}{0pt}%
\pgfpathmoveto{\pgfqpoint{1.598169in}{2.677784in}}%
\pgfpathcurveto{\pgfqpoint{1.609219in}{2.677784in}}{\pgfqpoint{1.619818in}{2.682175in}}{\pgfqpoint{1.627631in}{2.689988in}}%
\pgfpathcurveto{\pgfqpoint{1.635445in}{2.697802in}}{\pgfqpoint{1.639835in}{2.708401in}}{\pgfqpoint{1.639835in}{2.719451in}}%
\pgfpathcurveto{\pgfqpoint{1.639835in}{2.730501in}}{\pgfqpoint{1.635445in}{2.741100in}}{\pgfqpoint{1.627631in}{2.748914in}}%
\pgfpathcurveto{\pgfqpoint{1.619818in}{2.756728in}}{\pgfqpoint{1.609219in}{2.761118in}}{\pgfqpoint{1.598169in}{2.761118in}}%
\pgfpathcurveto{\pgfqpoint{1.587119in}{2.761118in}}{\pgfqpoint{1.576520in}{2.756728in}}{\pgfqpoint{1.568706in}{2.748914in}}%
\pgfpathcurveto{\pgfqpoint{1.560892in}{2.741100in}}{\pgfqpoint{1.556502in}{2.730501in}}{\pgfqpoint{1.556502in}{2.719451in}}%
\pgfpathcurveto{\pgfqpoint{1.556502in}{2.708401in}}{\pgfqpoint{1.560892in}{2.697802in}}{\pgfqpoint{1.568706in}{2.689988in}}%
\pgfpathcurveto{\pgfqpoint{1.576520in}{2.682175in}}{\pgfqpoint{1.587119in}{2.677784in}}{\pgfqpoint{1.598169in}{2.677784in}}%
\pgfpathclose%
\pgfusepath{stroke,fill}%
\end{pgfscope}%
\begin{pgfscope}%
\pgfpathrectangle{\pgfqpoint{0.481978in}{0.331635in}}{\pgfqpoint{4.960000in}{3.696000in}}%
\pgfusepath{clip}%
\pgfsetbuttcap%
\pgfsetroundjoin%
\definecolor{currentfill}{rgb}{1.000000,0.705882,0.509804}%
\pgfsetfillcolor{currentfill}%
\pgfsetlinewidth{0.481800pt}%
\definecolor{currentstroke}{rgb}{1.000000,1.000000,1.000000}%
\pgfsetstrokecolor{currentstroke}%
\pgfsetdash{}{0pt}%
\pgfpathmoveto{\pgfqpoint{3.105033in}{0.769246in}}%
\pgfpathcurveto{\pgfqpoint{3.116083in}{0.769246in}}{\pgfqpoint{3.126682in}{0.773636in}}{\pgfqpoint{3.134495in}{0.781450in}}%
\pgfpathcurveto{\pgfqpoint{3.142309in}{0.789263in}}{\pgfqpoint{3.146699in}{0.799862in}}{\pgfqpoint{3.146699in}{0.810912in}}%
\pgfpathcurveto{\pgfqpoint{3.146699in}{0.821962in}}{\pgfqpoint{3.142309in}{0.832562in}}{\pgfqpoint{3.134495in}{0.840375in}}%
\pgfpathcurveto{\pgfqpoint{3.126682in}{0.848189in}}{\pgfqpoint{3.116083in}{0.852579in}}{\pgfqpoint{3.105033in}{0.852579in}}%
\pgfpathcurveto{\pgfqpoint{3.093983in}{0.852579in}}{\pgfqpoint{3.083384in}{0.848189in}}{\pgfqpoint{3.075570in}{0.840375in}}%
\pgfpathcurveto{\pgfqpoint{3.067756in}{0.832562in}}{\pgfqpoint{3.063366in}{0.821962in}}{\pgfqpoint{3.063366in}{0.810912in}}%
\pgfpathcurveto{\pgfqpoint{3.063366in}{0.799862in}}{\pgfqpoint{3.067756in}{0.789263in}}{\pgfqpoint{3.075570in}{0.781450in}}%
\pgfpathcurveto{\pgfqpoint{3.083384in}{0.773636in}}{\pgfqpoint{3.093983in}{0.769246in}}{\pgfqpoint{3.105033in}{0.769246in}}%
\pgfpathclose%
\pgfusepath{stroke,fill}%
\end{pgfscope}%
\begin{pgfscope}%
\pgfpathrectangle{\pgfqpoint{0.481978in}{0.331635in}}{\pgfqpoint{4.960000in}{3.696000in}}%
\pgfusepath{clip}%
\pgfsetbuttcap%
\pgfsetroundjoin%
\definecolor{currentfill}{rgb}{1.000000,0.705882,0.509804}%
\pgfsetfillcolor{currentfill}%
\pgfsetlinewidth{0.481800pt}%
\definecolor{currentstroke}{rgb}{1.000000,1.000000,1.000000}%
\pgfsetstrokecolor{currentstroke}%
\pgfsetdash{}{0pt}%
\pgfpathmoveto{\pgfqpoint{1.895234in}{3.273531in}}%
\pgfpathcurveto{\pgfqpoint{1.906284in}{3.273531in}}{\pgfqpoint{1.916883in}{3.277921in}}{\pgfqpoint{1.924697in}{3.285735in}}%
\pgfpathcurveto{\pgfqpoint{1.932510in}{3.293548in}}{\pgfqpoint{1.936901in}{3.304147in}}{\pgfqpoint{1.936901in}{3.315197in}}%
\pgfpathcurveto{\pgfqpoint{1.936901in}{3.326248in}}{\pgfqpoint{1.932510in}{3.336847in}}{\pgfqpoint{1.924697in}{3.344660in}}%
\pgfpathcurveto{\pgfqpoint{1.916883in}{3.352474in}}{\pgfqpoint{1.906284in}{3.356864in}}{\pgfqpoint{1.895234in}{3.356864in}}%
\pgfpathcurveto{\pgfqpoint{1.884184in}{3.356864in}}{\pgfqpoint{1.873585in}{3.352474in}}{\pgfqpoint{1.865771in}{3.344660in}}%
\pgfpathcurveto{\pgfqpoint{1.857958in}{3.336847in}}{\pgfqpoint{1.853567in}{3.326248in}}{\pgfqpoint{1.853567in}{3.315197in}}%
\pgfpathcurveto{\pgfqpoint{1.853567in}{3.304147in}}{\pgfqpoint{1.857958in}{3.293548in}}{\pgfqpoint{1.865771in}{3.285735in}}%
\pgfpathcurveto{\pgfqpoint{1.873585in}{3.277921in}}{\pgfqpoint{1.884184in}{3.273531in}}{\pgfqpoint{1.895234in}{3.273531in}}%
\pgfpathclose%
\pgfusepath{stroke,fill}%
\end{pgfscope}%
\begin{pgfscope}%
\pgfpathrectangle{\pgfqpoint{0.481978in}{0.331635in}}{\pgfqpoint{4.960000in}{3.696000in}}%
\pgfusepath{clip}%
\pgfsetbuttcap%
\pgfsetroundjoin%
\definecolor{currentfill}{rgb}{1.000000,0.705882,0.509804}%
\pgfsetfillcolor{currentfill}%
\pgfsetlinewidth{0.481800pt}%
\definecolor{currentstroke}{rgb}{1.000000,1.000000,1.000000}%
\pgfsetstrokecolor{currentstroke}%
\pgfsetdash{}{0pt}%
\pgfpathmoveto{\pgfqpoint{3.360808in}{1.083842in}}%
\pgfpathcurveto{\pgfqpoint{3.371858in}{1.083842in}}{\pgfqpoint{3.382457in}{1.088232in}}{\pgfqpoint{3.390271in}{1.096046in}}%
\pgfpathcurveto{\pgfqpoint{3.398084in}{1.103859in}}{\pgfqpoint{3.402474in}{1.114458in}}{\pgfqpoint{3.402474in}{1.125508in}}%
\pgfpathcurveto{\pgfqpoint{3.402474in}{1.136558in}}{\pgfqpoint{3.398084in}{1.147157in}}{\pgfqpoint{3.390271in}{1.154971in}}%
\pgfpathcurveto{\pgfqpoint{3.382457in}{1.162785in}}{\pgfqpoint{3.371858in}{1.167175in}}{\pgfqpoint{3.360808in}{1.167175in}}%
\pgfpathcurveto{\pgfqpoint{3.349758in}{1.167175in}}{\pgfqpoint{3.339159in}{1.162785in}}{\pgfqpoint{3.331345in}{1.154971in}}%
\pgfpathcurveto{\pgfqpoint{3.323531in}{1.147157in}}{\pgfqpoint{3.319141in}{1.136558in}}{\pgfqpoint{3.319141in}{1.125508in}}%
\pgfpathcurveto{\pgfqpoint{3.319141in}{1.114458in}}{\pgfqpoint{3.323531in}{1.103859in}}{\pgfqpoint{3.331345in}{1.096046in}}%
\pgfpathcurveto{\pgfqpoint{3.339159in}{1.088232in}}{\pgfqpoint{3.349758in}{1.083842in}}{\pgfqpoint{3.360808in}{1.083842in}}%
\pgfpathclose%
\pgfusepath{stroke,fill}%
\end{pgfscope}%
\begin{pgfscope}%
\pgfpathrectangle{\pgfqpoint{0.481978in}{0.331635in}}{\pgfqpoint{4.960000in}{3.696000in}}%
\pgfusepath{clip}%
\pgfsetbuttcap%
\pgfsetroundjoin%
\definecolor{currentfill}{rgb}{1.000000,0.705882,0.509804}%
\pgfsetfillcolor{currentfill}%
\pgfsetlinewidth{0.481800pt}%
\definecolor{currentstroke}{rgb}{1.000000,1.000000,1.000000}%
\pgfsetstrokecolor{currentstroke}%
\pgfsetdash{}{0pt}%
\pgfpathmoveto{\pgfqpoint{4.027785in}{2.127986in}}%
\pgfpathcurveto{\pgfqpoint{4.038835in}{2.127986in}}{\pgfqpoint{4.049434in}{2.132376in}}{\pgfqpoint{4.057247in}{2.140190in}}%
\pgfpathcurveto{\pgfqpoint{4.065061in}{2.148003in}}{\pgfqpoint{4.069451in}{2.158602in}}{\pgfqpoint{4.069451in}{2.169652in}}%
\pgfpathcurveto{\pgfqpoint{4.069451in}{2.180703in}}{\pgfqpoint{4.065061in}{2.191302in}}{\pgfqpoint{4.057247in}{2.199115in}}%
\pgfpathcurveto{\pgfqpoint{4.049434in}{2.206929in}}{\pgfqpoint{4.038835in}{2.211319in}}{\pgfqpoint{4.027785in}{2.211319in}}%
\pgfpathcurveto{\pgfqpoint{4.016735in}{2.211319in}}{\pgfqpoint{4.006136in}{2.206929in}}{\pgfqpoint{3.998322in}{2.199115in}}%
\pgfpathcurveto{\pgfqpoint{3.990508in}{2.191302in}}{\pgfqpoint{3.986118in}{2.180703in}}{\pgfqpoint{3.986118in}{2.169652in}}%
\pgfpathcurveto{\pgfqpoint{3.986118in}{2.158602in}}{\pgfqpoint{3.990508in}{2.148003in}}{\pgfqpoint{3.998322in}{2.140190in}}%
\pgfpathcurveto{\pgfqpoint{4.006136in}{2.132376in}}{\pgfqpoint{4.016735in}{2.127986in}}{\pgfqpoint{4.027785in}{2.127986in}}%
\pgfpathclose%
\pgfusepath{stroke,fill}%
\end{pgfscope}%
\begin{pgfscope}%
\pgfpathrectangle{\pgfqpoint{0.481978in}{0.331635in}}{\pgfqpoint{4.960000in}{3.696000in}}%
\pgfusepath{clip}%
\pgfsetbuttcap%
\pgfsetroundjoin%
\definecolor{currentfill}{rgb}{1.000000,0.705882,0.509804}%
\pgfsetfillcolor{currentfill}%
\pgfsetlinewidth{0.481800pt}%
\definecolor{currentstroke}{rgb}{1.000000,1.000000,1.000000}%
\pgfsetstrokecolor{currentstroke}%
\pgfsetdash{}{0pt}%
\pgfpathmoveto{\pgfqpoint{1.211108in}{2.423451in}}%
\pgfpathcurveto{\pgfqpoint{1.222158in}{2.423451in}}{\pgfqpoint{1.232757in}{2.427842in}}{\pgfqpoint{1.240570in}{2.435655in}}%
\pgfpathcurveto{\pgfqpoint{1.248384in}{2.443469in}}{\pgfqpoint{1.252774in}{2.454068in}}{\pgfqpoint{1.252774in}{2.465118in}}%
\pgfpathcurveto{\pgfqpoint{1.252774in}{2.476168in}}{\pgfqpoint{1.248384in}{2.486767in}}{\pgfqpoint{1.240570in}{2.494581in}}%
\pgfpathcurveto{\pgfqpoint{1.232757in}{2.502395in}}{\pgfqpoint{1.222158in}{2.506785in}}{\pgfqpoint{1.211108in}{2.506785in}}%
\pgfpathcurveto{\pgfqpoint{1.200057in}{2.506785in}}{\pgfqpoint{1.189458in}{2.502395in}}{\pgfqpoint{1.181645in}{2.494581in}}%
\pgfpathcurveto{\pgfqpoint{1.173831in}{2.486767in}}{\pgfqpoint{1.169441in}{2.476168in}}{\pgfqpoint{1.169441in}{2.465118in}}%
\pgfpathcurveto{\pgfqpoint{1.169441in}{2.454068in}}{\pgfqpoint{1.173831in}{2.443469in}}{\pgfqpoint{1.181645in}{2.435655in}}%
\pgfpathcurveto{\pgfqpoint{1.189458in}{2.427842in}}{\pgfqpoint{1.200057in}{2.423451in}}{\pgfqpoint{1.211108in}{2.423451in}}%
\pgfpathclose%
\pgfusepath{stroke,fill}%
\end{pgfscope}%
\begin{pgfscope}%
\pgfpathrectangle{\pgfqpoint{0.481978in}{0.331635in}}{\pgfqpoint{4.960000in}{3.696000in}}%
\pgfusepath{clip}%
\pgfsetbuttcap%
\pgfsetroundjoin%
\definecolor{currentfill}{rgb}{1.000000,0.705882,0.509804}%
\pgfsetfillcolor{currentfill}%
\pgfsetlinewidth{0.481800pt}%
\definecolor{currentstroke}{rgb}{1.000000,1.000000,1.000000}%
\pgfsetstrokecolor{currentstroke}%
\pgfsetdash{}{0pt}%
\pgfpathmoveto{\pgfqpoint{4.184791in}{0.901415in}}%
\pgfpathcurveto{\pgfqpoint{4.195841in}{0.901415in}}{\pgfqpoint{4.206440in}{0.905805in}}{\pgfqpoint{4.214253in}{0.913619in}}%
\pgfpathcurveto{\pgfqpoint{4.222067in}{0.921433in}}{\pgfqpoint{4.226457in}{0.932032in}}{\pgfqpoint{4.226457in}{0.943082in}}%
\pgfpathcurveto{\pgfqpoint{4.226457in}{0.954132in}}{\pgfqpoint{4.222067in}{0.964731in}}{\pgfqpoint{4.214253in}{0.972545in}}%
\pgfpathcurveto{\pgfqpoint{4.206440in}{0.980358in}}{\pgfqpoint{4.195841in}{0.984749in}}{\pgfqpoint{4.184791in}{0.984749in}}%
\pgfpathcurveto{\pgfqpoint{4.173740in}{0.984749in}}{\pgfqpoint{4.163141in}{0.980358in}}{\pgfqpoint{4.155328in}{0.972545in}}%
\pgfpathcurveto{\pgfqpoint{4.147514in}{0.964731in}}{\pgfqpoint{4.143124in}{0.954132in}}{\pgfqpoint{4.143124in}{0.943082in}}%
\pgfpathcurveto{\pgfqpoint{4.143124in}{0.932032in}}{\pgfqpoint{4.147514in}{0.921433in}}{\pgfqpoint{4.155328in}{0.913619in}}%
\pgfpathcurveto{\pgfqpoint{4.163141in}{0.905805in}}{\pgfqpoint{4.173740in}{0.901415in}}{\pgfqpoint{4.184791in}{0.901415in}}%
\pgfpathclose%
\pgfusepath{stroke,fill}%
\end{pgfscope}%
\begin{pgfscope}%
\pgfpathrectangle{\pgfqpoint{0.481978in}{0.331635in}}{\pgfqpoint{4.960000in}{3.696000in}}%
\pgfusepath{clip}%
\pgfsetbuttcap%
\pgfsetroundjoin%
\definecolor{currentfill}{rgb}{1.000000,0.705882,0.509804}%
\pgfsetfillcolor{currentfill}%
\pgfsetlinewidth{0.481800pt}%
\definecolor{currentstroke}{rgb}{1.000000,1.000000,1.000000}%
\pgfsetstrokecolor{currentstroke}%
\pgfsetdash{}{0pt}%
\pgfpathmoveto{\pgfqpoint{1.789027in}{1.264166in}}%
\pgfpathcurveto{\pgfqpoint{1.800077in}{1.264166in}}{\pgfqpoint{1.810677in}{1.268556in}}{\pgfqpoint{1.818490in}{1.276370in}}%
\pgfpathcurveto{\pgfqpoint{1.826304in}{1.284183in}}{\pgfqpoint{1.830694in}{1.294782in}}{\pgfqpoint{1.830694in}{1.305832in}}%
\pgfpathcurveto{\pgfqpoint{1.830694in}{1.316882in}}{\pgfqpoint{1.826304in}{1.327482in}}{\pgfqpoint{1.818490in}{1.335295in}}%
\pgfpathcurveto{\pgfqpoint{1.810677in}{1.343109in}}{\pgfqpoint{1.800077in}{1.347499in}}{\pgfqpoint{1.789027in}{1.347499in}}%
\pgfpathcurveto{\pgfqpoint{1.777977in}{1.347499in}}{\pgfqpoint{1.767378in}{1.343109in}}{\pgfqpoint{1.759565in}{1.335295in}}%
\pgfpathcurveto{\pgfqpoint{1.751751in}{1.327482in}}{\pgfqpoint{1.747361in}{1.316882in}}{\pgfqpoint{1.747361in}{1.305832in}}%
\pgfpathcurveto{\pgfqpoint{1.747361in}{1.294782in}}{\pgfqpoint{1.751751in}{1.284183in}}{\pgfqpoint{1.759565in}{1.276370in}}%
\pgfpathcurveto{\pgfqpoint{1.767378in}{1.268556in}}{\pgfqpoint{1.777977in}{1.264166in}}{\pgfqpoint{1.789027in}{1.264166in}}%
\pgfpathclose%
\pgfusepath{stroke,fill}%
\end{pgfscope}%
\begin{pgfscope}%
\pgfpathrectangle{\pgfqpoint{0.481978in}{0.331635in}}{\pgfqpoint{4.960000in}{3.696000in}}%
\pgfusepath{clip}%
\pgfsetbuttcap%
\pgfsetroundjoin%
\definecolor{currentfill}{rgb}{1.000000,0.705882,0.509804}%
\pgfsetfillcolor{currentfill}%
\pgfsetlinewidth{0.481800pt}%
\definecolor{currentstroke}{rgb}{1.000000,1.000000,1.000000}%
\pgfsetstrokecolor{currentstroke}%
\pgfsetdash{}{0pt}%
\pgfpathmoveto{\pgfqpoint{2.368419in}{0.963626in}}%
\pgfpathcurveto{\pgfqpoint{2.379470in}{0.963626in}}{\pgfqpoint{2.390069in}{0.968016in}}{\pgfqpoint{2.397882in}{0.975829in}}%
\pgfpathcurveto{\pgfqpoint{2.405696in}{0.983643in}}{\pgfqpoint{2.410086in}{0.994242in}}{\pgfqpoint{2.410086in}{1.005292in}}%
\pgfpathcurveto{\pgfqpoint{2.410086in}{1.016342in}}{\pgfqpoint{2.405696in}{1.026941in}}{\pgfqpoint{2.397882in}{1.034755in}}%
\pgfpathcurveto{\pgfqpoint{2.390069in}{1.042569in}}{\pgfqpoint{2.379470in}{1.046959in}}{\pgfqpoint{2.368419in}{1.046959in}}%
\pgfpathcurveto{\pgfqpoint{2.357369in}{1.046959in}}{\pgfqpoint{2.346770in}{1.042569in}}{\pgfqpoint{2.338957in}{1.034755in}}%
\pgfpathcurveto{\pgfqpoint{2.331143in}{1.026941in}}{\pgfqpoint{2.326753in}{1.016342in}}{\pgfqpoint{2.326753in}{1.005292in}}%
\pgfpathcurveto{\pgfqpoint{2.326753in}{0.994242in}}{\pgfqpoint{2.331143in}{0.983643in}}{\pgfqpoint{2.338957in}{0.975829in}}%
\pgfpathcurveto{\pgfqpoint{2.346770in}{0.968016in}}{\pgfqpoint{2.357369in}{0.963626in}}{\pgfqpoint{2.368419in}{0.963626in}}%
\pgfpathclose%
\pgfusepath{stroke,fill}%
\end{pgfscope}%
\begin{pgfscope}%
\pgfpathrectangle{\pgfqpoint{0.481978in}{0.331635in}}{\pgfqpoint{4.960000in}{3.696000in}}%
\pgfusepath{clip}%
\pgfsetbuttcap%
\pgfsetroundjoin%
\definecolor{currentfill}{rgb}{1.000000,0.705882,0.509804}%
\pgfsetfillcolor{currentfill}%
\pgfsetlinewidth{0.481800pt}%
\definecolor{currentstroke}{rgb}{1.000000,1.000000,1.000000}%
\pgfsetstrokecolor{currentstroke}%
\pgfsetdash{}{0pt}%
\pgfpathmoveto{\pgfqpoint{1.427964in}{2.922774in}}%
\pgfpathcurveto{\pgfqpoint{1.439014in}{2.922774in}}{\pgfqpoint{1.449613in}{2.927164in}}{\pgfqpoint{1.457427in}{2.934978in}}%
\pgfpathcurveto{\pgfqpoint{1.465240in}{2.942791in}}{\pgfqpoint{1.469631in}{2.953390in}}{\pgfqpoint{1.469631in}{2.964440in}}%
\pgfpathcurveto{\pgfqpoint{1.469631in}{2.975490in}}{\pgfqpoint{1.465240in}{2.986089in}}{\pgfqpoint{1.457427in}{2.993903in}}%
\pgfpathcurveto{\pgfqpoint{1.449613in}{3.001717in}}{\pgfqpoint{1.439014in}{3.006107in}}{\pgfqpoint{1.427964in}{3.006107in}}%
\pgfpathcurveto{\pgfqpoint{1.416914in}{3.006107in}}{\pgfqpoint{1.406315in}{3.001717in}}{\pgfqpoint{1.398501in}{2.993903in}}%
\pgfpathcurveto{\pgfqpoint{1.390687in}{2.986089in}}{\pgfqpoint{1.386297in}{2.975490in}}{\pgfqpoint{1.386297in}{2.964440in}}%
\pgfpathcurveto{\pgfqpoint{1.386297in}{2.953390in}}{\pgfqpoint{1.390687in}{2.942791in}}{\pgfqpoint{1.398501in}{2.934978in}}%
\pgfpathcurveto{\pgfqpoint{1.406315in}{2.927164in}}{\pgfqpoint{1.416914in}{2.922774in}}{\pgfqpoint{1.427964in}{2.922774in}}%
\pgfpathclose%
\pgfusepath{stroke,fill}%
\end{pgfscope}%
\begin{pgfscope}%
\pgfpathrectangle{\pgfqpoint{0.481978in}{0.331635in}}{\pgfqpoint{4.960000in}{3.696000in}}%
\pgfusepath{clip}%
\pgfsetbuttcap%
\pgfsetroundjoin%
\definecolor{currentfill}{rgb}{1.000000,0.705882,0.509804}%
\pgfsetfillcolor{currentfill}%
\pgfsetlinewidth{0.481800pt}%
\definecolor{currentstroke}{rgb}{1.000000,1.000000,1.000000}%
\pgfsetstrokecolor{currentstroke}%
\pgfsetdash{}{0pt}%
\pgfpathmoveto{\pgfqpoint{3.393546in}{3.817968in}}%
\pgfpathcurveto{\pgfqpoint{3.404596in}{3.817968in}}{\pgfqpoint{3.415195in}{3.822359in}}{\pgfqpoint{3.423008in}{3.830172in}}%
\pgfpathcurveto{\pgfqpoint{3.430822in}{3.837986in}}{\pgfqpoint{3.435212in}{3.848585in}}{\pgfqpoint{3.435212in}{3.859635in}}%
\pgfpathcurveto{\pgfqpoint{3.435212in}{3.870685in}}{\pgfqpoint{3.430822in}{3.881284in}}{\pgfqpoint{3.423008in}{3.889098in}}%
\pgfpathcurveto{\pgfqpoint{3.415195in}{3.896911in}}{\pgfqpoint{3.404596in}{3.901302in}}{\pgfqpoint{3.393546in}{3.901302in}}%
\pgfpathcurveto{\pgfqpoint{3.382495in}{3.901302in}}{\pgfqpoint{3.371896in}{3.896911in}}{\pgfqpoint{3.364083in}{3.889098in}}%
\pgfpathcurveto{\pgfqpoint{3.356269in}{3.881284in}}{\pgfqpoint{3.351879in}{3.870685in}}{\pgfqpoint{3.351879in}{3.859635in}}%
\pgfpathcurveto{\pgfqpoint{3.351879in}{3.848585in}}{\pgfqpoint{3.356269in}{3.837986in}}{\pgfqpoint{3.364083in}{3.830172in}}%
\pgfpathcurveto{\pgfqpoint{3.371896in}{3.822359in}}{\pgfqpoint{3.382495in}{3.817968in}}{\pgfqpoint{3.393546in}{3.817968in}}%
\pgfpathclose%
\pgfusepath{stroke,fill}%
\end{pgfscope}%
\begin{pgfscope}%
\pgfpathrectangle{\pgfqpoint{0.481978in}{0.331635in}}{\pgfqpoint{4.960000in}{3.696000in}}%
\pgfusepath{clip}%
\pgfsetbuttcap%
\pgfsetroundjoin%
\definecolor{currentfill}{rgb}{1.000000,0.705882,0.509804}%
\pgfsetfillcolor{currentfill}%
\pgfsetlinewidth{0.481800pt}%
\definecolor{currentstroke}{rgb}{1.000000,1.000000,1.000000}%
\pgfsetstrokecolor{currentstroke}%
\pgfsetdash{}{0pt}%
\pgfpathmoveto{\pgfqpoint{1.952110in}{1.586388in}}%
\pgfpathcurveto{\pgfqpoint{1.963160in}{1.586388in}}{\pgfqpoint{1.973759in}{1.590779in}}{\pgfqpoint{1.981573in}{1.598592in}}%
\pgfpathcurveto{\pgfqpoint{1.989386in}{1.606406in}}{\pgfqpoint{1.993777in}{1.617005in}}{\pgfqpoint{1.993777in}{1.628055in}}%
\pgfpathcurveto{\pgfqpoint{1.993777in}{1.639105in}}{\pgfqpoint{1.989386in}{1.649704in}}{\pgfqpoint{1.981573in}{1.657518in}}%
\pgfpathcurveto{\pgfqpoint{1.973759in}{1.665331in}}{\pgfqpoint{1.963160in}{1.669722in}}{\pgfqpoint{1.952110in}{1.669722in}}%
\pgfpathcurveto{\pgfqpoint{1.941060in}{1.669722in}}{\pgfqpoint{1.930461in}{1.665331in}}{\pgfqpoint{1.922647in}{1.657518in}}%
\pgfpathcurveto{\pgfqpoint{1.914833in}{1.649704in}}{\pgfqpoint{1.910443in}{1.639105in}}{\pgfqpoint{1.910443in}{1.628055in}}%
\pgfpathcurveto{\pgfqpoint{1.910443in}{1.617005in}}{\pgfqpoint{1.914833in}{1.606406in}}{\pgfqpoint{1.922647in}{1.598592in}}%
\pgfpathcurveto{\pgfqpoint{1.930461in}{1.590779in}}{\pgfqpoint{1.941060in}{1.586388in}}{\pgfqpoint{1.952110in}{1.586388in}}%
\pgfpathclose%
\pgfusepath{stroke,fill}%
\end{pgfscope}%
\begin{pgfscope}%
\pgfpathrectangle{\pgfqpoint{0.481978in}{0.331635in}}{\pgfqpoint{4.960000in}{3.696000in}}%
\pgfusepath{clip}%
\pgfsetbuttcap%
\pgfsetroundjoin%
\definecolor{currentfill}{rgb}{1.000000,0.705882,0.509804}%
\pgfsetfillcolor{currentfill}%
\pgfsetlinewidth{0.481800pt}%
\definecolor{currentstroke}{rgb}{1.000000,1.000000,1.000000}%
\pgfsetstrokecolor{currentstroke}%
\pgfsetdash{}{0pt}%
\pgfpathmoveto{\pgfqpoint{2.726428in}{2.815925in}}%
\pgfpathcurveto{\pgfqpoint{2.737479in}{2.815925in}}{\pgfqpoint{2.748078in}{2.820315in}}{\pgfqpoint{2.755891in}{2.828128in}}%
\pgfpathcurveto{\pgfqpoint{2.763705in}{2.835942in}}{\pgfqpoint{2.768095in}{2.846541in}}{\pgfqpoint{2.768095in}{2.857591in}}%
\pgfpathcurveto{\pgfqpoint{2.768095in}{2.868641in}}{\pgfqpoint{2.763705in}{2.879240in}}{\pgfqpoint{2.755891in}{2.887054in}}%
\pgfpathcurveto{\pgfqpoint{2.748078in}{2.894868in}}{\pgfqpoint{2.737479in}{2.899258in}}{\pgfqpoint{2.726428in}{2.899258in}}%
\pgfpathcurveto{\pgfqpoint{2.715378in}{2.899258in}}{\pgfqpoint{2.704779in}{2.894868in}}{\pgfqpoint{2.696966in}{2.887054in}}%
\pgfpathcurveto{\pgfqpoint{2.689152in}{2.879240in}}{\pgfqpoint{2.684762in}{2.868641in}}{\pgfqpoint{2.684762in}{2.857591in}}%
\pgfpathcurveto{\pgfqpoint{2.684762in}{2.846541in}}{\pgfqpoint{2.689152in}{2.835942in}}{\pgfqpoint{2.696966in}{2.828128in}}%
\pgfpathcurveto{\pgfqpoint{2.704779in}{2.820315in}}{\pgfqpoint{2.715378in}{2.815925in}}{\pgfqpoint{2.726428in}{2.815925in}}%
\pgfpathclose%
\pgfusepath{stroke,fill}%
\end{pgfscope}%
\begin{pgfscope}%
\pgfpathrectangle{\pgfqpoint{0.481978in}{0.331635in}}{\pgfqpoint{4.960000in}{3.696000in}}%
\pgfusepath{clip}%
\pgfsetbuttcap%
\pgfsetroundjoin%
\definecolor{currentfill}{rgb}{1.000000,0.705882,0.509804}%
\pgfsetfillcolor{currentfill}%
\pgfsetlinewidth{0.481800pt}%
\definecolor{currentstroke}{rgb}{1.000000,1.000000,1.000000}%
\pgfsetstrokecolor{currentstroke}%
\pgfsetdash{}{0pt}%
\pgfpathmoveto{\pgfqpoint{1.372997in}{1.814536in}}%
\pgfpathcurveto{\pgfqpoint{1.384048in}{1.814536in}}{\pgfqpoint{1.394647in}{1.818926in}}{\pgfqpoint{1.402460in}{1.826740in}}%
\pgfpathcurveto{\pgfqpoint{1.410274in}{1.834554in}}{\pgfqpoint{1.414664in}{1.845153in}}{\pgfqpoint{1.414664in}{1.856203in}}%
\pgfpathcurveto{\pgfqpoint{1.414664in}{1.867253in}}{\pgfqpoint{1.410274in}{1.877852in}}{\pgfqpoint{1.402460in}{1.885666in}}%
\pgfpathcurveto{\pgfqpoint{1.394647in}{1.893479in}}{\pgfqpoint{1.384048in}{1.897869in}}{\pgfqpoint{1.372997in}{1.897869in}}%
\pgfpathcurveto{\pgfqpoint{1.361947in}{1.897869in}}{\pgfqpoint{1.351348in}{1.893479in}}{\pgfqpoint{1.343535in}{1.885666in}}%
\pgfpathcurveto{\pgfqpoint{1.335721in}{1.877852in}}{\pgfqpoint{1.331331in}{1.867253in}}{\pgfqpoint{1.331331in}{1.856203in}}%
\pgfpathcurveto{\pgfqpoint{1.331331in}{1.845153in}}{\pgfqpoint{1.335721in}{1.834554in}}{\pgfqpoint{1.343535in}{1.826740in}}%
\pgfpathcurveto{\pgfqpoint{1.351348in}{1.818926in}}{\pgfqpoint{1.361947in}{1.814536in}}{\pgfqpoint{1.372997in}{1.814536in}}%
\pgfpathclose%
\pgfusepath{stroke,fill}%
\end{pgfscope}%
\begin{pgfscope}%
\pgfpathrectangle{\pgfqpoint{0.481978in}{0.331635in}}{\pgfqpoint{4.960000in}{3.696000in}}%
\pgfusepath{clip}%
\pgfsetbuttcap%
\pgfsetroundjoin%
\definecolor{currentfill}{rgb}{1.000000,0.705882,0.509804}%
\pgfsetfillcolor{currentfill}%
\pgfsetlinewidth{0.481800pt}%
\definecolor{currentstroke}{rgb}{1.000000,1.000000,1.000000}%
\pgfsetstrokecolor{currentstroke}%
\pgfsetdash{}{0pt}%
\pgfpathmoveto{\pgfqpoint{0.707432in}{3.102544in}}%
\pgfpathcurveto{\pgfqpoint{0.718483in}{3.102544in}}{\pgfqpoint{0.729082in}{3.106934in}}{\pgfqpoint{0.736895in}{3.114748in}}%
\pgfpathcurveto{\pgfqpoint{0.744709in}{3.122561in}}{\pgfqpoint{0.749099in}{3.133160in}}{\pgfqpoint{0.749099in}{3.144211in}}%
\pgfpathcurveto{\pgfqpoint{0.749099in}{3.155261in}}{\pgfqpoint{0.744709in}{3.165860in}}{\pgfqpoint{0.736895in}{3.173673in}}%
\pgfpathcurveto{\pgfqpoint{0.729082in}{3.181487in}}{\pgfqpoint{0.718483in}{3.185877in}}{\pgfqpoint{0.707432in}{3.185877in}}%
\pgfpathcurveto{\pgfqpoint{0.696382in}{3.185877in}}{\pgfqpoint{0.685783in}{3.181487in}}{\pgfqpoint{0.677970in}{3.173673in}}%
\pgfpathcurveto{\pgfqpoint{0.670156in}{3.165860in}}{\pgfqpoint{0.665766in}{3.155261in}}{\pgfqpoint{0.665766in}{3.144211in}}%
\pgfpathcurveto{\pgfqpoint{0.665766in}{3.133160in}}{\pgfqpoint{0.670156in}{3.122561in}}{\pgfqpoint{0.677970in}{3.114748in}}%
\pgfpathcurveto{\pgfqpoint{0.685783in}{3.106934in}}{\pgfqpoint{0.696382in}{3.102544in}}{\pgfqpoint{0.707432in}{3.102544in}}%
\pgfpathclose%
\pgfusepath{stroke,fill}%
\end{pgfscope}%
\begin{pgfscope}%
\pgfpathrectangle{\pgfqpoint{0.481978in}{0.331635in}}{\pgfqpoint{4.960000in}{3.696000in}}%
\pgfusepath{clip}%
\pgfsetbuttcap%
\pgfsetroundjoin%
\definecolor{currentfill}{rgb}{1.000000,0.705882,0.509804}%
\pgfsetfillcolor{currentfill}%
\pgfsetlinewidth{0.481800pt}%
\definecolor{currentstroke}{rgb}{1.000000,1.000000,1.000000}%
\pgfsetstrokecolor{currentstroke}%
\pgfsetdash{}{0pt}%
\pgfpathmoveto{\pgfqpoint{3.782152in}{0.457968in}}%
\pgfpathcurveto{\pgfqpoint{3.793202in}{0.457968in}}{\pgfqpoint{3.803801in}{0.462359in}}{\pgfqpoint{3.811614in}{0.470172in}}%
\pgfpathcurveto{\pgfqpoint{3.819428in}{0.477986in}}{\pgfqpoint{3.823818in}{0.488585in}}{\pgfqpoint{3.823818in}{0.499635in}}%
\pgfpathcurveto{\pgfqpoint{3.823818in}{0.510685in}}{\pgfqpoint{3.819428in}{0.521284in}}{\pgfqpoint{3.811614in}{0.529098in}}%
\pgfpathcurveto{\pgfqpoint{3.803801in}{0.536911in}}{\pgfqpoint{3.793202in}{0.541302in}}{\pgfqpoint{3.782152in}{0.541302in}}%
\pgfpathcurveto{\pgfqpoint{3.771101in}{0.541302in}}{\pgfqpoint{3.760502in}{0.536911in}}{\pgfqpoint{3.752689in}{0.529098in}}%
\pgfpathcurveto{\pgfqpoint{3.744875in}{0.521284in}}{\pgfqpoint{3.740485in}{0.510685in}}{\pgfqpoint{3.740485in}{0.499635in}}%
\pgfpathcurveto{\pgfqpoint{3.740485in}{0.488585in}}{\pgfqpoint{3.744875in}{0.477986in}}{\pgfqpoint{3.752689in}{0.470172in}}%
\pgfpathcurveto{\pgfqpoint{3.760502in}{0.462359in}}{\pgfqpoint{3.771101in}{0.457968in}}{\pgfqpoint{3.782152in}{0.457968in}}%
\pgfpathclose%
\pgfusepath{stroke,fill}%
\end{pgfscope}%
\begin{pgfscope}%
\pgfpathrectangle{\pgfqpoint{0.481978in}{0.331635in}}{\pgfqpoint{4.960000in}{3.696000in}}%
\pgfusepath{clip}%
\pgfsetbuttcap%
\pgfsetroundjoin%
\definecolor{currentfill}{rgb}{1.000000,0.705882,0.509804}%
\pgfsetfillcolor{currentfill}%
\pgfsetlinewidth{0.481800pt}%
\definecolor{currentstroke}{rgb}{1.000000,1.000000,1.000000}%
\pgfsetstrokecolor{currentstroke}%
\pgfsetdash{}{0pt}%
\pgfpathmoveto{\pgfqpoint{4.764267in}{2.705857in}}%
\pgfpathcurveto{\pgfqpoint{4.775318in}{2.705857in}}{\pgfqpoint{4.785917in}{2.710247in}}{\pgfqpoint{4.793730in}{2.718061in}}%
\pgfpathcurveto{\pgfqpoint{4.801544in}{2.725875in}}{\pgfqpoint{4.805934in}{2.736474in}}{\pgfqpoint{4.805934in}{2.747524in}}%
\pgfpathcurveto{\pgfqpoint{4.805934in}{2.758574in}}{\pgfqpoint{4.801544in}{2.769173in}}{\pgfqpoint{4.793730in}{2.776986in}}%
\pgfpathcurveto{\pgfqpoint{4.785917in}{2.784800in}}{\pgfqpoint{4.775318in}{2.789190in}}{\pgfqpoint{4.764267in}{2.789190in}}%
\pgfpathcurveto{\pgfqpoint{4.753217in}{2.789190in}}{\pgfqpoint{4.742618in}{2.784800in}}{\pgfqpoint{4.734805in}{2.776986in}}%
\pgfpathcurveto{\pgfqpoint{4.726991in}{2.769173in}}{\pgfqpoint{4.722601in}{2.758574in}}{\pgfqpoint{4.722601in}{2.747524in}}%
\pgfpathcurveto{\pgfqpoint{4.722601in}{2.736474in}}{\pgfqpoint{4.726991in}{2.725875in}}{\pgfqpoint{4.734805in}{2.718061in}}%
\pgfpathcurveto{\pgfqpoint{4.742618in}{2.710247in}}{\pgfqpoint{4.753217in}{2.705857in}}{\pgfqpoint{4.764267in}{2.705857in}}%
\pgfpathclose%
\pgfusepath{stroke,fill}%
\end{pgfscope}%
\begin{pgfscope}%
\pgfpathrectangle{\pgfqpoint{0.481978in}{0.331635in}}{\pgfqpoint{4.960000in}{3.696000in}}%
\pgfusepath{clip}%
\pgfsetbuttcap%
\pgfsetroundjoin%
\definecolor{currentfill}{rgb}{1.000000,0.705882,0.509804}%
\pgfsetfillcolor{currentfill}%
\pgfsetlinewidth{0.481800pt}%
\definecolor{currentstroke}{rgb}{1.000000,1.000000,1.000000}%
\pgfsetstrokecolor{currentstroke}%
\pgfsetdash{}{0pt}%
\pgfpathmoveto{\pgfqpoint{1.027884in}{3.268783in}}%
\pgfpathcurveto{\pgfqpoint{1.038934in}{3.268783in}}{\pgfqpoint{1.049533in}{3.273173in}}{\pgfqpoint{1.057347in}{3.280987in}}%
\pgfpathcurveto{\pgfqpoint{1.065160in}{3.288800in}}{\pgfqpoint{1.069550in}{3.299399in}}{\pgfqpoint{1.069550in}{3.310449in}}%
\pgfpathcurveto{\pgfqpoint{1.069550in}{3.321500in}}{\pgfqpoint{1.065160in}{3.332099in}}{\pgfqpoint{1.057347in}{3.339912in}}%
\pgfpathcurveto{\pgfqpoint{1.049533in}{3.347726in}}{\pgfqpoint{1.038934in}{3.352116in}}{\pgfqpoint{1.027884in}{3.352116in}}%
\pgfpathcurveto{\pgfqpoint{1.016834in}{3.352116in}}{\pgfqpoint{1.006235in}{3.347726in}}{\pgfqpoint{0.998421in}{3.339912in}}%
\pgfpathcurveto{\pgfqpoint{0.990607in}{3.332099in}}{\pgfqpoint{0.986217in}{3.321500in}}{\pgfqpoint{0.986217in}{3.310449in}}%
\pgfpathcurveto{\pgfqpoint{0.986217in}{3.299399in}}{\pgfqpoint{0.990607in}{3.288800in}}{\pgfqpoint{0.998421in}{3.280987in}}%
\pgfpathcurveto{\pgfqpoint{1.006235in}{3.273173in}}{\pgfqpoint{1.016834in}{3.268783in}}{\pgfqpoint{1.027884in}{3.268783in}}%
\pgfpathclose%
\pgfusepath{stroke,fill}%
\end{pgfscope}%
\begin{pgfscope}%
\pgfpathrectangle{\pgfqpoint{0.481978in}{0.331635in}}{\pgfqpoint{4.960000in}{3.696000in}}%
\pgfusepath{clip}%
\pgfsetbuttcap%
\pgfsetroundjoin%
\definecolor{currentfill}{rgb}{1.000000,0.705882,0.509804}%
\pgfsetfillcolor{currentfill}%
\pgfsetlinewidth{0.481800pt}%
\definecolor{currentstroke}{rgb}{1.000000,1.000000,1.000000}%
\pgfsetstrokecolor{currentstroke}%
\pgfsetdash{}{0pt}%
\pgfpathmoveto{\pgfqpoint{2.571490in}{3.546079in}}%
\pgfpathcurveto{\pgfqpoint{2.582540in}{3.546079in}}{\pgfqpoint{2.593139in}{3.550469in}}{\pgfqpoint{2.600953in}{3.558283in}}%
\pgfpathcurveto{\pgfqpoint{2.608766in}{3.566096in}}{\pgfqpoint{2.613157in}{3.576695in}}{\pgfqpoint{2.613157in}{3.587745in}}%
\pgfpathcurveto{\pgfqpoint{2.613157in}{3.598796in}}{\pgfqpoint{2.608766in}{3.609395in}}{\pgfqpoint{2.600953in}{3.617208in}}%
\pgfpathcurveto{\pgfqpoint{2.593139in}{3.625022in}}{\pgfqpoint{2.582540in}{3.629412in}}{\pgfqpoint{2.571490in}{3.629412in}}%
\pgfpathcurveto{\pgfqpoint{2.560440in}{3.629412in}}{\pgfqpoint{2.549841in}{3.625022in}}{\pgfqpoint{2.542027in}{3.617208in}}%
\pgfpathcurveto{\pgfqpoint{2.534214in}{3.609395in}}{\pgfqpoint{2.529823in}{3.598796in}}{\pgfqpoint{2.529823in}{3.587745in}}%
\pgfpathcurveto{\pgfqpoint{2.529823in}{3.576695in}}{\pgfqpoint{2.534214in}{3.566096in}}{\pgfqpoint{2.542027in}{3.558283in}}%
\pgfpathcurveto{\pgfqpoint{2.549841in}{3.550469in}}{\pgfqpoint{2.560440in}{3.546079in}}{\pgfqpoint{2.571490in}{3.546079in}}%
\pgfpathclose%
\pgfusepath{stroke,fill}%
\end{pgfscope}%
\begin{pgfscope}%
\pgfpathrectangle{\pgfqpoint{0.481978in}{0.331635in}}{\pgfqpoint{4.960000in}{3.696000in}}%
\pgfusepath{clip}%
\pgfsetbuttcap%
\pgfsetroundjoin%
\definecolor{currentfill}{rgb}{1.000000,0.705882,0.509804}%
\pgfsetfillcolor{currentfill}%
\pgfsetlinewidth{0.481800pt}%
\definecolor{currentstroke}{rgb}{1.000000,1.000000,1.000000}%
\pgfsetstrokecolor{currentstroke}%
\pgfsetdash{}{0pt}%
\pgfpathmoveto{\pgfqpoint{2.788619in}{2.545126in}}%
\pgfpathcurveto{\pgfqpoint{2.799669in}{2.545126in}}{\pgfqpoint{2.810268in}{2.549516in}}{\pgfqpoint{2.818081in}{2.557330in}}%
\pgfpathcurveto{\pgfqpoint{2.825895in}{2.565144in}}{\pgfqpoint{2.830285in}{2.575743in}}{\pgfqpoint{2.830285in}{2.586793in}}%
\pgfpathcurveto{\pgfqpoint{2.830285in}{2.597843in}}{\pgfqpoint{2.825895in}{2.608442in}}{\pgfqpoint{2.818081in}{2.616256in}}%
\pgfpathcurveto{\pgfqpoint{2.810268in}{2.624069in}}{\pgfqpoint{2.799669in}{2.628459in}}{\pgfqpoint{2.788619in}{2.628459in}}%
\pgfpathcurveto{\pgfqpoint{2.777568in}{2.628459in}}{\pgfqpoint{2.766969in}{2.624069in}}{\pgfqpoint{2.759156in}{2.616256in}}%
\pgfpathcurveto{\pgfqpoint{2.751342in}{2.608442in}}{\pgfqpoint{2.746952in}{2.597843in}}{\pgfqpoint{2.746952in}{2.586793in}}%
\pgfpathcurveto{\pgfqpoint{2.746952in}{2.575743in}}{\pgfqpoint{2.751342in}{2.565144in}}{\pgfqpoint{2.759156in}{2.557330in}}%
\pgfpathcurveto{\pgfqpoint{2.766969in}{2.549516in}}{\pgfqpoint{2.777568in}{2.545126in}}{\pgfqpoint{2.788619in}{2.545126in}}%
\pgfpathclose%
\pgfusepath{stroke,fill}%
\end{pgfscope}%
\begin{pgfscope}%
\pgfpathrectangle{\pgfqpoint{0.481978in}{0.331635in}}{\pgfqpoint{4.960000in}{3.696000in}}%
\pgfusepath{clip}%
\pgfsetbuttcap%
\pgfsetroundjoin%
\definecolor{currentfill}{rgb}{1.000000,0.705882,0.509804}%
\pgfsetfillcolor{currentfill}%
\pgfsetlinewidth{0.481800pt}%
\definecolor{currentstroke}{rgb}{1.000000,1.000000,1.000000}%
\pgfsetstrokecolor{currentstroke}%
\pgfsetdash{}{0pt}%
\pgfpathmoveto{\pgfqpoint{3.846440in}{2.897939in}}%
\pgfpathcurveto{\pgfqpoint{3.857490in}{2.897939in}}{\pgfqpoint{3.868089in}{2.902329in}}{\pgfqpoint{3.875902in}{2.910143in}}%
\pgfpathcurveto{\pgfqpoint{3.883716in}{2.917956in}}{\pgfqpoint{3.888106in}{2.928555in}}{\pgfqpoint{3.888106in}{2.939605in}}%
\pgfpathcurveto{\pgfqpoint{3.888106in}{2.950656in}}{\pgfqpoint{3.883716in}{2.961255in}}{\pgfqpoint{3.875902in}{2.969068in}}%
\pgfpathcurveto{\pgfqpoint{3.868089in}{2.976882in}}{\pgfqpoint{3.857490in}{2.981272in}}{\pgfqpoint{3.846440in}{2.981272in}}%
\pgfpathcurveto{\pgfqpoint{3.835390in}{2.981272in}}{\pgfqpoint{3.824790in}{2.976882in}}{\pgfqpoint{3.816977in}{2.969068in}}%
\pgfpathcurveto{\pgfqpoint{3.809163in}{2.961255in}}{\pgfqpoint{3.804773in}{2.950656in}}{\pgfqpoint{3.804773in}{2.939605in}}%
\pgfpathcurveto{\pgfqpoint{3.804773in}{2.928555in}}{\pgfqpoint{3.809163in}{2.917956in}}{\pgfqpoint{3.816977in}{2.910143in}}%
\pgfpathcurveto{\pgfqpoint{3.824790in}{2.902329in}}{\pgfqpoint{3.835390in}{2.897939in}}{\pgfqpoint{3.846440in}{2.897939in}}%
\pgfpathclose%
\pgfusepath{stroke,fill}%
\end{pgfscope}%
\begin{pgfscope}%
\pgfpathrectangle{\pgfqpoint{0.481978in}{0.331635in}}{\pgfqpoint{4.960000in}{3.696000in}}%
\pgfusepath{clip}%
\pgfsetbuttcap%
\pgfsetroundjoin%
\definecolor{currentfill}{rgb}{1.000000,0.705882,0.509804}%
\pgfsetfillcolor{currentfill}%
\pgfsetlinewidth{0.481800pt}%
\definecolor{currentstroke}{rgb}{1.000000,1.000000,1.000000}%
\pgfsetstrokecolor{currentstroke}%
\pgfsetdash{}{0pt}%
\pgfpathmoveto{\pgfqpoint{4.322794in}{2.423244in}}%
\pgfpathcurveto{\pgfqpoint{4.333844in}{2.423244in}}{\pgfqpoint{4.344443in}{2.427634in}}{\pgfqpoint{4.352257in}{2.435448in}}%
\pgfpathcurveto{\pgfqpoint{4.360070in}{2.443261in}}{\pgfqpoint{4.364461in}{2.453860in}}{\pgfqpoint{4.364461in}{2.464911in}}%
\pgfpathcurveto{\pgfqpoint{4.364461in}{2.475961in}}{\pgfqpoint{4.360070in}{2.486560in}}{\pgfqpoint{4.352257in}{2.494373in}}%
\pgfpathcurveto{\pgfqpoint{4.344443in}{2.502187in}}{\pgfqpoint{4.333844in}{2.506577in}}{\pgfqpoint{4.322794in}{2.506577in}}%
\pgfpathcurveto{\pgfqpoint{4.311744in}{2.506577in}}{\pgfqpoint{4.301145in}{2.502187in}}{\pgfqpoint{4.293331in}{2.494373in}}%
\pgfpathcurveto{\pgfqpoint{4.285518in}{2.486560in}}{\pgfqpoint{4.281127in}{2.475961in}}{\pgfqpoint{4.281127in}{2.464911in}}%
\pgfpathcurveto{\pgfqpoint{4.281127in}{2.453860in}}{\pgfqpoint{4.285518in}{2.443261in}}{\pgfqpoint{4.293331in}{2.435448in}}%
\pgfpathcurveto{\pgfqpoint{4.301145in}{2.427634in}}{\pgfqpoint{4.311744in}{2.423244in}}{\pgfqpoint{4.322794in}{2.423244in}}%
\pgfpathclose%
\pgfusepath{stroke,fill}%
\end{pgfscope}%
\begin{pgfscope}%
\pgfpathrectangle{\pgfqpoint{0.481978in}{0.331635in}}{\pgfqpoint{4.960000in}{3.696000in}}%
\pgfusepath{clip}%
\pgfsetbuttcap%
\pgfsetroundjoin%
\definecolor{currentfill}{rgb}{1.000000,0.705882,0.509804}%
\pgfsetfillcolor{currentfill}%
\pgfsetlinewidth{0.481800pt}%
\definecolor{currentstroke}{rgb}{1.000000,1.000000,1.000000}%
\pgfsetstrokecolor{currentstroke}%
\pgfsetdash{}{0pt}%
\pgfpathmoveto{\pgfqpoint{1.013445in}{2.853600in}}%
\pgfpathcurveto{\pgfqpoint{1.024495in}{2.853600in}}{\pgfqpoint{1.035094in}{2.857991in}}{\pgfqpoint{1.042907in}{2.865804in}}%
\pgfpathcurveto{\pgfqpoint{1.050721in}{2.873618in}}{\pgfqpoint{1.055111in}{2.884217in}}{\pgfqpoint{1.055111in}{2.895267in}}%
\pgfpathcurveto{\pgfqpoint{1.055111in}{2.906317in}}{\pgfqpoint{1.050721in}{2.916916in}}{\pgfqpoint{1.042907in}{2.924730in}}%
\pgfpathcurveto{\pgfqpoint{1.035094in}{2.932544in}}{\pgfqpoint{1.024495in}{2.936934in}}{\pgfqpoint{1.013445in}{2.936934in}}%
\pgfpathcurveto{\pgfqpoint{1.002394in}{2.936934in}}{\pgfqpoint{0.991795in}{2.932544in}}{\pgfqpoint{0.983982in}{2.924730in}}%
\pgfpathcurveto{\pgfqpoint{0.976168in}{2.916916in}}{\pgfqpoint{0.971778in}{2.906317in}}{\pgfqpoint{0.971778in}{2.895267in}}%
\pgfpathcurveto{\pgfqpoint{0.971778in}{2.884217in}}{\pgfqpoint{0.976168in}{2.873618in}}{\pgfqpoint{0.983982in}{2.865804in}}%
\pgfpathcurveto{\pgfqpoint{0.991795in}{2.857991in}}{\pgfqpoint{1.002394in}{2.853600in}}{\pgfqpoint{1.013445in}{2.853600in}}%
\pgfpathclose%
\pgfusepath{stroke,fill}%
\end{pgfscope}%
\begin{pgfscope}%
\pgfpathrectangle{\pgfqpoint{0.481978in}{0.331635in}}{\pgfqpoint{4.960000in}{3.696000in}}%
\pgfusepath{clip}%
\pgfsetbuttcap%
\pgfsetroundjoin%
\definecolor{currentfill}{rgb}{1.000000,0.705882,0.509804}%
\pgfsetfillcolor{currentfill}%
\pgfsetlinewidth{0.481800pt}%
\definecolor{currentstroke}{rgb}{1.000000,1.000000,1.000000}%
\pgfsetstrokecolor{currentstroke}%
\pgfsetdash{}{0pt}%
\pgfpathmoveto{\pgfqpoint{2.049477in}{2.898169in}}%
\pgfpathcurveto{\pgfqpoint{2.060527in}{2.898169in}}{\pgfqpoint{2.071126in}{2.902559in}}{\pgfqpoint{2.078940in}{2.910373in}}%
\pgfpathcurveto{\pgfqpoint{2.086754in}{2.918186in}}{\pgfqpoint{2.091144in}{2.928785in}}{\pgfqpoint{2.091144in}{2.939835in}}%
\pgfpathcurveto{\pgfqpoint{2.091144in}{2.950886in}}{\pgfqpoint{2.086754in}{2.961485in}}{\pgfqpoint{2.078940in}{2.969298in}}%
\pgfpathcurveto{\pgfqpoint{2.071126in}{2.977112in}}{\pgfqpoint{2.060527in}{2.981502in}}{\pgfqpoint{2.049477in}{2.981502in}}%
\pgfpathcurveto{\pgfqpoint{2.038427in}{2.981502in}}{\pgfqpoint{2.027828in}{2.977112in}}{\pgfqpoint{2.020014in}{2.969298in}}%
\pgfpathcurveto{\pgfqpoint{2.012201in}{2.961485in}}{\pgfqpoint{2.007810in}{2.950886in}}{\pgfqpoint{2.007810in}{2.939835in}}%
\pgfpathcurveto{\pgfqpoint{2.007810in}{2.928785in}}{\pgfqpoint{2.012201in}{2.918186in}}{\pgfqpoint{2.020014in}{2.910373in}}%
\pgfpathcurveto{\pgfqpoint{2.027828in}{2.902559in}}{\pgfqpoint{2.038427in}{2.898169in}}{\pgfqpoint{2.049477in}{2.898169in}}%
\pgfpathclose%
\pgfusepath{stroke,fill}%
\end{pgfscope}%
\begin{pgfscope}%
\pgfpathrectangle{\pgfqpoint{0.481978in}{0.331635in}}{\pgfqpoint{4.960000in}{3.696000in}}%
\pgfusepath{clip}%
\pgfsetbuttcap%
\pgfsetroundjoin%
\definecolor{currentfill}{rgb}{0.631373,0.788235,0.956863}%
\pgfsetfillcolor{currentfill}%
\pgfsetlinewidth{1.003750pt}%
\definecolor{currentstroke}{rgb}{0.631373,0.788235,0.956863}%
\pgfsetstrokecolor{currentstroke}%
\pgfsetdash{}{0pt}%
\pgfsys@defobject{currentmarker}{\pgfqpoint{-0.041667in}{-0.041667in}}{\pgfqpoint{0.041667in}{0.041667in}}{%
\pgfpathmoveto{\pgfqpoint{0.000000in}{-0.041667in}}%
\pgfpathcurveto{\pgfqpoint{0.011050in}{-0.041667in}}{\pgfqpoint{0.021649in}{-0.037276in}}{\pgfqpoint{0.029463in}{-0.029463in}}%
\pgfpathcurveto{\pgfqpoint{0.037276in}{-0.021649in}}{\pgfqpoint{0.041667in}{-0.011050in}}{\pgfqpoint{0.041667in}{0.000000in}}%
\pgfpathcurveto{\pgfqpoint{0.041667in}{0.011050in}}{\pgfqpoint{0.037276in}{0.021649in}}{\pgfqpoint{0.029463in}{0.029463in}}%
\pgfpathcurveto{\pgfqpoint{0.021649in}{0.037276in}}{\pgfqpoint{0.011050in}{0.041667in}}{\pgfqpoint{0.000000in}{0.041667in}}%
\pgfpathcurveto{\pgfqpoint{-0.011050in}{0.041667in}}{\pgfqpoint{-0.021649in}{0.037276in}}{\pgfqpoint{-0.029463in}{0.029463in}}%
\pgfpathcurveto{\pgfqpoint{-0.037276in}{0.021649in}}{\pgfqpoint{-0.041667in}{0.011050in}}{\pgfqpoint{-0.041667in}{0.000000in}}%
\pgfpathcurveto{\pgfqpoint{-0.041667in}{-0.011050in}}{\pgfqpoint{-0.037276in}{-0.021649in}}{\pgfqpoint{-0.029463in}{-0.029463in}}%
\pgfpathcurveto{\pgfqpoint{-0.021649in}{-0.037276in}}{\pgfqpoint{-0.011050in}{-0.041667in}}{\pgfqpoint{0.000000in}{-0.041667in}}%
\pgfpathclose%
\pgfusepath{stroke,fill}%
}%
\end{pgfscope}%
\begin{pgfscope}%
\pgfpathrectangle{\pgfqpoint{0.481978in}{0.331635in}}{\pgfqpoint{4.960000in}{3.696000in}}%
\pgfusepath{clip}%
\pgfsetbuttcap%
\pgfsetroundjoin%
\definecolor{currentfill}{rgb}{1.000000,0.705882,0.509804}%
\pgfsetfillcolor{currentfill}%
\pgfsetlinewidth{1.003750pt}%
\definecolor{currentstroke}{rgb}{1.000000,0.705882,0.509804}%
\pgfsetstrokecolor{currentstroke}%
\pgfsetdash{}{0pt}%
\pgfsys@defobject{currentmarker}{\pgfqpoint{-0.041667in}{-0.041667in}}{\pgfqpoint{0.041667in}{0.041667in}}{%
\pgfpathmoveto{\pgfqpoint{0.000000in}{-0.041667in}}%
\pgfpathcurveto{\pgfqpoint{0.011050in}{-0.041667in}}{\pgfqpoint{0.021649in}{-0.037276in}}{\pgfqpoint{0.029463in}{-0.029463in}}%
\pgfpathcurveto{\pgfqpoint{0.037276in}{-0.021649in}}{\pgfqpoint{0.041667in}{-0.011050in}}{\pgfqpoint{0.041667in}{0.000000in}}%
\pgfpathcurveto{\pgfqpoint{0.041667in}{0.011050in}}{\pgfqpoint{0.037276in}{0.021649in}}{\pgfqpoint{0.029463in}{0.029463in}}%
\pgfpathcurveto{\pgfqpoint{0.021649in}{0.037276in}}{\pgfqpoint{0.011050in}{0.041667in}}{\pgfqpoint{0.000000in}{0.041667in}}%
\pgfpathcurveto{\pgfqpoint{-0.011050in}{0.041667in}}{\pgfqpoint{-0.021649in}{0.037276in}}{\pgfqpoint{-0.029463in}{0.029463in}}%
\pgfpathcurveto{\pgfqpoint{-0.037276in}{0.021649in}}{\pgfqpoint{-0.041667in}{0.011050in}}{\pgfqpoint{-0.041667in}{0.000000in}}%
\pgfpathcurveto{\pgfqpoint{-0.041667in}{-0.011050in}}{\pgfqpoint{-0.037276in}{-0.021649in}}{\pgfqpoint{-0.029463in}{-0.029463in}}%
\pgfpathcurveto{\pgfqpoint{-0.021649in}{-0.037276in}}{\pgfqpoint{-0.011050in}{-0.041667in}}{\pgfqpoint{0.000000in}{-0.041667in}}%
\pgfpathclose%
\pgfusepath{stroke,fill}%
}%
\end{pgfscope}%
\begin{pgfscope}%
\pgfsetbuttcap%
\pgfsetroundjoin%
\definecolor{currentfill}{rgb}{0.000000,0.000000,0.000000}%
\pgfsetfillcolor{currentfill}%
\pgfsetlinewidth{0.803000pt}%
\definecolor{currentstroke}{rgb}{0.000000,0.000000,0.000000}%
\pgfsetstrokecolor{currentstroke}%
\pgfsetdash{}{0pt}%
\pgfsys@defobject{currentmarker}{\pgfqpoint{0.000000in}{-0.048611in}}{\pgfqpoint{0.000000in}{0.000000in}}{%
\pgfpathmoveto{\pgfqpoint{0.000000in}{0.000000in}}%
\pgfpathlineto{\pgfqpoint{0.000000in}{-0.048611in}}%
\pgfusepath{stroke,fill}%
}%
\begin{pgfscope}%
\pgfsys@transformshift{0.651717in}{0.331635in}%
\pgfsys@useobject{currentmarker}{}%
\end{pgfscope}%
\end{pgfscope}%
\begin{pgfscope}%
\definecolor{textcolor}{rgb}{0.000000,0.000000,0.000000}%
\pgfsetstrokecolor{textcolor}%
\pgfsetfillcolor{textcolor}%
\pgftext[x=0.651717in,y=0.234413in,,top]{\color{textcolor}\sffamily\fontsize{10.000000}{12.000000}\selectfont \ensuremath{-}60}%
\end{pgfscope}%
\begin{pgfscope}%
\pgfsetbuttcap%
\pgfsetroundjoin%
\definecolor{currentfill}{rgb}{0.000000,0.000000,0.000000}%
\pgfsetfillcolor{currentfill}%
\pgfsetlinewidth{0.803000pt}%
\definecolor{currentstroke}{rgb}{0.000000,0.000000,0.000000}%
\pgfsetstrokecolor{currentstroke}%
\pgfsetdash{}{0pt}%
\pgfsys@defobject{currentmarker}{\pgfqpoint{0.000000in}{-0.048611in}}{\pgfqpoint{0.000000in}{0.000000in}}{%
\pgfpathmoveto{\pgfqpoint{0.000000in}{0.000000in}}%
\pgfpathlineto{\pgfqpoint{0.000000in}{-0.048611in}}%
\pgfusepath{stroke,fill}%
}%
\begin{pgfscope}%
\pgfsys@transformshift{1.501372in}{0.331635in}%
\pgfsys@useobject{currentmarker}{}%
\end{pgfscope}%
\end{pgfscope}%
\begin{pgfscope}%
\definecolor{textcolor}{rgb}{0.000000,0.000000,0.000000}%
\pgfsetstrokecolor{textcolor}%
\pgfsetfillcolor{textcolor}%
\pgftext[x=1.501372in,y=0.234413in,,top]{\color{textcolor}\sffamily\fontsize{10.000000}{12.000000}\selectfont \ensuremath{-}40}%
\end{pgfscope}%
\begin{pgfscope}%
\pgfsetbuttcap%
\pgfsetroundjoin%
\definecolor{currentfill}{rgb}{0.000000,0.000000,0.000000}%
\pgfsetfillcolor{currentfill}%
\pgfsetlinewidth{0.803000pt}%
\definecolor{currentstroke}{rgb}{0.000000,0.000000,0.000000}%
\pgfsetstrokecolor{currentstroke}%
\pgfsetdash{}{0pt}%
\pgfsys@defobject{currentmarker}{\pgfqpoint{0.000000in}{-0.048611in}}{\pgfqpoint{0.000000in}{0.000000in}}{%
\pgfpathmoveto{\pgfqpoint{0.000000in}{0.000000in}}%
\pgfpathlineto{\pgfqpoint{0.000000in}{-0.048611in}}%
\pgfusepath{stroke,fill}%
}%
\begin{pgfscope}%
\pgfsys@transformshift{2.351027in}{0.331635in}%
\pgfsys@useobject{currentmarker}{}%
\end{pgfscope}%
\end{pgfscope}%
\begin{pgfscope}%
\definecolor{textcolor}{rgb}{0.000000,0.000000,0.000000}%
\pgfsetstrokecolor{textcolor}%
\pgfsetfillcolor{textcolor}%
\pgftext[x=2.351027in,y=0.234413in,,top]{\color{textcolor}\sffamily\fontsize{10.000000}{12.000000}\selectfont \ensuremath{-}20}%
\end{pgfscope}%
\begin{pgfscope}%
\pgfsetbuttcap%
\pgfsetroundjoin%
\definecolor{currentfill}{rgb}{0.000000,0.000000,0.000000}%
\pgfsetfillcolor{currentfill}%
\pgfsetlinewidth{0.803000pt}%
\definecolor{currentstroke}{rgb}{0.000000,0.000000,0.000000}%
\pgfsetstrokecolor{currentstroke}%
\pgfsetdash{}{0pt}%
\pgfsys@defobject{currentmarker}{\pgfqpoint{0.000000in}{-0.048611in}}{\pgfqpoint{0.000000in}{0.000000in}}{%
\pgfpathmoveto{\pgfqpoint{0.000000in}{0.000000in}}%
\pgfpathlineto{\pgfqpoint{0.000000in}{-0.048611in}}%
\pgfusepath{stroke,fill}%
}%
\begin{pgfscope}%
\pgfsys@transformshift{3.200683in}{0.331635in}%
\pgfsys@useobject{currentmarker}{}%
\end{pgfscope}%
\end{pgfscope}%
\begin{pgfscope}%
\definecolor{textcolor}{rgb}{0.000000,0.000000,0.000000}%
\pgfsetstrokecolor{textcolor}%
\pgfsetfillcolor{textcolor}%
\pgftext[x=3.200683in,y=0.234413in,,top]{\color{textcolor}\sffamily\fontsize{10.000000}{12.000000}\selectfont 0}%
\end{pgfscope}%
\begin{pgfscope}%
\pgfsetbuttcap%
\pgfsetroundjoin%
\definecolor{currentfill}{rgb}{0.000000,0.000000,0.000000}%
\pgfsetfillcolor{currentfill}%
\pgfsetlinewidth{0.803000pt}%
\definecolor{currentstroke}{rgb}{0.000000,0.000000,0.000000}%
\pgfsetstrokecolor{currentstroke}%
\pgfsetdash{}{0pt}%
\pgfsys@defobject{currentmarker}{\pgfqpoint{0.000000in}{-0.048611in}}{\pgfqpoint{0.000000in}{0.000000in}}{%
\pgfpathmoveto{\pgfqpoint{0.000000in}{0.000000in}}%
\pgfpathlineto{\pgfqpoint{0.000000in}{-0.048611in}}%
\pgfusepath{stroke,fill}%
}%
\begin{pgfscope}%
\pgfsys@transformshift{4.050338in}{0.331635in}%
\pgfsys@useobject{currentmarker}{}%
\end{pgfscope}%
\end{pgfscope}%
\begin{pgfscope}%
\definecolor{textcolor}{rgb}{0.000000,0.000000,0.000000}%
\pgfsetstrokecolor{textcolor}%
\pgfsetfillcolor{textcolor}%
\pgftext[x=4.050338in,y=0.234413in,,top]{\color{textcolor}\sffamily\fontsize{10.000000}{12.000000}\selectfont 20}%
\end{pgfscope}%
\begin{pgfscope}%
\pgfsetbuttcap%
\pgfsetroundjoin%
\definecolor{currentfill}{rgb}{0.000000,0.000000,0.000000}%
\pgfsetfillcolor{currentfill}%
\pgfsetlinewidth{0.803000pt}%
\definecolor{currentstroke}{rgb}{0.000000,0.000000,0.000000}%
\pgfsetstrokecolor{currentstroke}%
\pgfsetdash{}{0pt}%
\pgfsys@defobject{currentmarker}{\pgfqpoint{0.000000in}{-0.048611in}}{\pgfqpoint{0.000000in}{0.000000in}}{%
\pgfpathmoveto{\pgfqpoint{0.000000in}{0.000000in}}%
\pgfpathlineto{\pgfqpoint{0.000000in}{-0.048611in}}%
\pgfusepath{stroke,fill}%
}%
\begin{pgfscope}%
\pgfsys@transformshift{4.899993in}{0.331635in}%
\pgfsys@useobject{currentmarker}{}%
\end{pgfscope}%
\end{pgfscope}%
\begin{pgfscope}%
\definecolor{textcolor}{rgb}{0.000000,0.000000,0.000000}%
\pgfsetstrokecolor{textcolor}%
\pgfsetfillcolor{textcolor}%
\pgftext[x=4.899993in,y=0.234413in,,top]{\color{textcolor}\sffamily\fontsize{10.000000}{12.000000}\selectfont 40}%
\end{pgfscope}%
\begin{pgfscope}%
\pgfsetbuttcap%
\pgfsetroundjoin%
\definecolor{currentfill}{rgb}{0.000000,0.000000,0.000000}%
\pgfsetfillcolor{currentfill}%
\pgfsetlinewidth{0.803000pt}%
\definecolor{currentstroke}{rgb}{0.000000,0.000000,0.000000}%
\pgfsetstrokecolor{currentstroke}%
\pgfsetdash{}{0pt}%
\pgfsys@defobject{currentmarker}{\pgfqpoint{-0.048611in}{0.000000in}}{\pgfqpoint{-0.000000in}{0.000000in}}{%
\pgfpathmoveto{\pgfqpoint{-0.000000in}{0.000000in}}%
\pgfpathlineto{\pgfqpoint{-0.048611in}{0.000000in}}%
\pgfusepath{stroke,fill}%
}%
\begin{pgfscope}%
\pgfsys@transformshift{0.481978in}{0.459608in}%
\pgfsys@useobject{currentmarker}{}%
\end{pgfscope}%
\end{pgfscope}%
\begin{pgfscope}%
\definecolor{textcolor}{rgb}{0.000000,0.000000,0.000000}%
\pgfsetstrokecolor{textcolor}%
\pgfsetfillcolor{textcolor}%
\pgftext[x=0.100000in, y=0.406846in, left, base]{\color{textcolor}\sffamily\fontsize{10.000000}{12.000000}\selectfont \ensuremath{-}60}%
\end{pgfscope}%
\begin{pgfscope}%
\pgfsetbuttcap%
\pgfsetroundjoin%
\definecolor{currentfill}{rgb}{0.000000,0.000000,0.000000}%
\pgfsetfillcolor{currentfill}%
\pgfsetlinewidth{0.803000pt}%
\definecolor{currentstroke}{rgb}{0.000000,0.000000,0.000000}%
\pgfsetstrokecolor{currentstroke}%
\pgfsetdash{}{0pt}%
\pgfsys@defobject{currentmarker}{\pgfqpoint{-0.048611in}{0.000000in}}{\pgfqpoint{-0.000000in}{0.000000in}}{%
\pgfpathmoveto{\pgfqpoint{-0.000000in}{0.000000in}}%
\pgfpathlineto{\pgfqpoint{-0.048611in}{0.000000in}}%
\pgfusepath{stroke,fill}%
}%
\begin{pgfscope}%
\pgfsys@transformshift{0.481978in}{1.157602in}%
\pgfsys@useobject{currentmarker}{}%
\end{pgfscope}%
\end{pgfscope}%
\begin{pgfscope}%
\definecolor{textcolor}{rgb}{0.000000,0.000000,0.000000}%
\pgfsetstrokecolor{textcolor}%
\pgfsetfillcolor{textcolor}%
\pgftext[x=0.100000in, y=1.104840in, left, base]{\color{textcolor}\sffamily\fontsize{10.000000}{12.000000}\selectfont \ensuremath{-}40}%
\end{pgfscope}%
\begin{pgfscope}%
\pgfsetbuttcap%
\pgfsetroundjoin%
\definecolor{currentfill}{rgb}{0.000000,0.000000,0.000000}%
\pgfsetfillcolor{currentfill}%
\pgfsetlinewidth{0.803000pt}%
\definecolor{currentstroke}{rgb}{0.000000,0.000000,0.000000}%
\pgfsetstrokecolor{currentstroke}%
\pgfsetdash{}{0pt}%
\pgfsys@defobject{currentmarker}{\pgfqpoint{-0.048611in}{0.000000in}}{\pgfqpoint{-0.000000in}{0.000000in}}{%
\pgfpathmoveto{\pgfqpoint{-0.000000in}{0.000000in}}%
\pgfpathlineto{\pgfqpoint{-0.048611in}{0.000000in}}%
\pgfusepath{stroke,fill}%
}%
\begin{pgfscope}%
\pgfsys@transformshift{0.481978in}{1.855596in}%
\pgfsys@useobject{currentmarker}{}%
\end{pgfscope}%
\end{pgfscope}%
\begin{pgfscope}%
\definecolor{textcolor}{rgb}{0.000000,0.000000,0.000000}%
\pgfsetstrokecolor{textcolor}%
\pgfsetfillcolor{textcolor}%
\pgftext[x=0.100000in, y=1.802834in, left, base]{\color{textcolor}\sffamily\fontsize{10.000000}{12.000000}\selectfont \ensuremath{-}20}%
\end{pgfscope}%
\begin{pgfscope}%
\pgfsetbuttcap%
\pgfsetroundjoin%
\definecolor{currentfill}{rgb}{0.000000,0.000000,0.000000}%
\pgfsetfillcolor{currentfill}%
\pgfsetlinewidth{0.803000pt}%
\definecolor{currentstroke}{rgb}{0.000000,0.000000,0.000000}%
\pgfsetstrokecolor{currentstroke}%
\pgfsetdash{}{0pt}%
\pgfsys@defobject{currentmarker}{\pgfqpoint{-0.048611in}{0.000000in}}{\pgfqpoint{-0.000000in}{0.000000in}}{%
\pgfpathmoveto{\pgfqpoint{-0.000000in}{0.000000in}}%
\pgfpathlineto{\pgfqpoint{-0.048611in}{0.000000in}}%
\pgfusepath{stroke,fill}%
}%
\begin{pgfscope}%
\pgfsys@transformshift{0.481978in}{2.553589in}%
\pgfsys@useobject{currentmarker}{}%
\end{pgfscope}%
\end{pgfscope}%
\begin{pgfscope}%
\definecolor{textcolor}{rgb}{0.000000,0.000000,0.000000}%
\pgfsetstrokecolor{textcolor}%
\pgfsetfillcolor{textcolor}%
\pgftext[x=0.296390in, y=2.500828in, left, base]{\color{textcolor}\sffamily\fontsize{10.000000}{12.000000}\selectfont 0}%
\end{pgfscope}%
\begin{pgfscope}%
\pgfsetbuttcap%
\pgfsetroundjoin%
\definecolor{currentfill}{rgb}{0.000000,0.000000,0.000000}%
\pgfsetfillcolor{currentfill}%
\pgfsetlinewidth{0.803000pt}%
\definecolor{currentstroke}{rgb}{0.000000,0.000000,0.000000}%
\pgfsetstrokecolor{currentstroke}%
\pgfsetdash{}{0pt}%
\pgfsys@defobject{currentmarker}{\pgfqpoint{-0.048611in}{0.000000in}}{\pgfqpoint{-0.000000in}{0.000000in}}{%
\pgfpathmoveto{\pgfqpoint{-0.000000in}{0.000000in}}%
\pgfpathlineto{\pgfqpoint{-0.048611in}{0.000000in}}%
\pgfusepath{stroke,fill}%
}%
\begin{pgfscope}%
\pgfsys@transformshift{0.481978in}{3.251583in}%
\pgfsys@useobject{currentmarker}{}%
\end{pgfscope}%
\end{pgfscope}%
\begin{pgfscope}%
\definecolor{textcolor}{rgb}{0.000000,0.000000,0.000000}%
\pgfsetstrokecolor{textcolor}%
\pgfsetfillcolor{textcolor}%
\pgftext[x=0.208025in, y=3.198822in, left, base]{\color{textcolor}\sffamily\fontsize{10.000000}{12.000000}\selectfont 20}%
\end{pgfscope}%
\begin{pgfscope}%
\pgfsetbuttcap%
\pgfsetroundjoin%
\definecolor{currentfill}{rgb}{0.000000,0.000000,0.000000}%
\pgfsetfillcolor{currentfill}%
\pgfsetlinewidth{0.803000pt}%
\definecolor{currentstroke}{rgb}{0.000000,0.000000,0.000000}%
\pgfsetstrokecolor{currentstroke}%
\pgfsetdash{}{0pt}%
\pgfsys@defobject{currentmarker}{\pgfqpoint{-0.048611in}{0.000000in}}{\pgfqpoint{-0.000000in}{0.000000in}}{%
\pgfpathmoveto{\pgfqpoint{-0.000000in}{0.000000in}}%
\pgfpathlineto{\pgfqpoint{-0.048611in}{0.000000in}}%
\pgfusepath{stroke,fill}%
}%
\begin{pgfscope}%
\pgfsys@transformshift{0.481978in}{3.949577in}%
\pgfsys@useobject{currentmarker}{}%
\end{pgfscope}%
\end{pgfscope}%
\begin{pgfscope}%
\definecolor{textcolor}{rgb}{0.000000,0.000000,0.000000}%
\pgfsetstrokecolor{textcolor}%
\pgfsetfillcolor{textcolor}%
\pgftext[x=0.208025in, y=3.896816in, left, base]{\color{textcolor}\sffamily\fontsize{10.000000}{12.000000}\selectfont 40}%
\end{pgfscope}%
\begin{pgfscope}%
\pgfpathrectangle{\pgfqpoint{0.481978in}{0.331635in}}{\pgfqpoint{4.960000in}{3.696000in}}%
\pgfusepath{clip}%
\pgfsetrectcap%
\pgfsetroundjoin%
\pgfsetlinewidth{1.505625pt}%
\definecolor{currentstroke}{rgb}{0.631373,0.788235,0.956863}%
\pgfsetstrokecolor{currentstroke}%
\pgfsetstrokeopacity{0.800000}%
\pgfsetdash{}{0pt}%
\pgfpathmoveto{\pgfqpoint{3.170276in}{2.430688in}}%
\pgfpathlineto{\pgfqpoint{3.563553in}{2.242061in}}%
\pgfusepath{stroke}%
\end{pgfscope}%
\begin{pgfscope}%
\pgfpathrectangle{\pgfqpoint{0.481978in}{0.331635in}}{\pgfqpoint{4.960000in}{3.696000in}}%
\pgfusepath{clip}%
\pgfsetrectcap%
\pgfsetroundjoin%
\pgfsetlinewidth{1.505625pt}%
\definecolor{currentstroke}{rgb}{0.631373,0.788235,0.956863}%
\pgfsetstrokecolor{currentstroke}%
\pgfsetstrokeopacity{0.800000}%
\pgfsetdash{}{0pt}%
\pgfpathmoveto{\pgfqpoint{3.797277in}{1.249718in}}%
\pgfpathlineto{\pgfqpoint{3.563553in}{2.242061in}}%
\pgfusepath{stroke}%
\end{pgfscope}%
\begin{pgfscope}%
\pgfpathrectangle{\pgfqpoint{0.481978in}{0.331635in}}{\pgfqpoint{4.960000in}{3.696000in}}%
\pgfusepath{clip}%
\pgfsetrectcap%
\pgfsetroundjoin%
\pgfsetlinewidth{1.505625pt}%
\definecolor{currentstroke}{rgb}{0.631373,0.788235,0.956863}%
\pgfsetstrokecolor{currentstroke}%
\pgfsetstrokeopacity{0.800000}%
\pgfsetdash{}{0pt}%
\pgfpathmoveto{\pgfqpoint{2.431161in}{2.160457in}}%
\pgfpathlineto{\pgfqpoint{3.563553in}{2.242061in}}%
\pgfusepath{stroke}%
\end{pgfscope}%
\begin{pgfscope}%
\pgfpathrectangle{\pgfqpoint{0.481978in}{0.331635in}}{\pgfqpoint{4.960000in}{3.696000in}}%
\pgfusepath{clip}%
\pgfsetrectcap%
\pgfsetroundjoin%
\pgfsetlinewidth{1.505625pt}%
\definecolor{currentstroke}{rgb}{0.631373,0.788235,0.956863}%
\pgfsetstrokecolor{currentstroke}%
\pgfsetstrokeopacity{0.800000}%
\pgfsetdash{}{0pt}%
\pgfpathmoveto{\pgfqpoint{3.988216in}{3.359003in}}%
\pgfpathlineto{\pgfqpoint{3.563553in}{2.242061in}}%
\pgfusepath{stroke}%
\end{pgfscope}%
\begin{pgfscope}%
\pgfpathrectangle{\pgfqpoint{0.481978in}{0.331635in}}{\pgfqpoint{4.960000in}{3.696000in}}%
\pgfusepath{clip}%
\pgfsetrectcap%
\pgfsetroundjoin%
\pgfsetlinewidth{1.505625pt}%
\definecolor{currentstroke}{rgb}{0.631373,0.788235,0.956863}%
\pgfsetstrokecolor{currentstroke}%
\pgfsetstrokeopacity{0.800000}%
\pgfsetdash{}{0pt}%
\pgfpathmoveto{\pgfqpoint{4.509417in}{1.385375in}}%
\pgfpathlineto{\pgfqpoint{3.563553in}{2.242061in}}%
\pgfusepath{stroke}%
\end{pgfscope}%
\begin{pgfscope}%
\pgfpathrectangle{\pgfqpoint{0.481978in}{0.331635in}}{\pgfqpoint{4.960000in}{3.696000in}}%
\pgfusepath{clip}%
\pgfsetrectcap%
\pgfsetroundjoin%
\pgfsetlinewidth{1.505625pt}%
\definecolor{currentstroke}{rgb}{0.631373,0.788235,0.956863}%
\pgfsetstrokecolor{currentstroke}%
\pgfsetstrokeopacity{0.800000}%
\pgfsetdash{}{0pt}%
\pgfpathmoveto{\pgfqpoint{3.252727in}{2.139754in}}%
\pgfpathlineto{\pgfqpoint{3.563553in}{2.242061in}}%
\pgfusepath{stroke}%
\end{pgfscope}%
\begin{pgfscope}%
\pgfpathrectangle{\pgfqpoint{0.481978in}{0.331635in}}{\pgfqpoint{4.960000in}{3.696000in}}%
\pgfusepath{clip}%
\pgfsetrectcap%
\pgfsetroundjoin%
\pgfsetlinewidth{1.505625pt}%
\definecolor{currentstroke}{rgb}{0.631373,0.788235,0.956863}%
\pgfsetstrokecolor{currentstroke}%
\pgfsetstrokeopacity{0.800000}%
\pgfsetdash{}{0pt}%
\pgfpathmoveto{\pgfqpoint{4.536927in}{3.194685in}}%
\pgfpathlineto{\pgfqpoint{3.563553in}{2.242061in}}%
\pgfusepath{stroke}%
\end{pgfscope}%
\begin{pgfscope}%
\pgfpathrectangle{\pgfqpoint{0.481978in}{0.331635in}}{\pgfqpoint{4.960000in}{3.696000in}}%
\pgfusepath{clip}%
\pgfsetrectcap%
\pgfsetroundjoin%
\pgfsetlinewidth{1.505625pt}%
\definecolor{currentstroke}{rgb}{0.631373,0.788235,0.956863}%
\pgfsetstrokecolor{currentstroke}%
\pgfsetstrokeopacity{0.800000}%
\pgfsetdash{}{0pt}%
\pgfpathmoveto{\pgfqpoint{3.471510in}{2.611751in}}%
\pgfpathlineto{\pgfqpoint{3.563553in}{2.242061in}}%
\pgfusepath{stroke}%
\end{pgfscope}%
\begin{pgfscope}%
\pgfpathrectangle{\pgfqpoint{0.481978in}{0.331635in}}{\pgfqpoint{4.960000in}{3.696000in}}%
\pgfusepath{clip}%
\pgfsetrectcap%
\pgfsetroundjoin%
\pgfsetlinewidth{1.505625pt}%
\definecolor{currentstroke}{rgb}{0.631373,0.788235,0.956863}%
\pgfsetstrokecolor{currentstroke}%
\pgfsetstrokeopacity{0.800000}%
\pgfsetdash{}{0pt}%
\pgfpathmoveto{\pgfqpoint{2.459931in}{1.759119in}}%
\pgfpathlineto{\pgfqpoint{3.563553in}{2.242061in}}%
\pgfusepath{stroke}%
\end{pgfscope}%
\begin{pgfscope}%
\pgfpathrectangle{\pgfqpoint{0.481978in}{0.331635in}}{\pgfqpoint{4.960000in}{3.696000in}}%
\pgfusepath{clip}%
\pgfsetrectcap%
\pgfsetroundjoin%
\pgfsetlinewidth{1.505625pt}%
\definecolor{currentstroke}{rgb}{0.631373,0.788235,0.956863}%
\pgfsetstrokecolor{currentstroke}%
\pgfsetstrokeopacity{0.800000}%
\pgfsetdash{}{0pt}%
\pgfpathmoveto{\pgfqpoint{2.937570in}{1.896242in}}%
\pgfpathlineto{\pgfqpoint{3.563553in}{2.242061in}}%
\pgfusepath{stroke}%
\end{pgfscope}%
\begin{pgfscope}%
\pgfpathrectangle{\pgfqpoint{0.481978in}{0.331635in}}{\pgfqpoint{4.960000in}{3.696000in}}%
\pgfusepath{clip}%
\pgfsetrectcap%
\pgfsetroundjoin%
\pgfsetlinewidth{1.505625pt}%
\definecolor{currentstroke}{rgb}{0.631373,0.788235,0.956863}%
\pgfsetstrokecolor{currentstroke}%
\pgfsetstrokeopacity{0.800000}%
\pgfsetdash{}{0pt}%
\pgfpathmoveto{\pgfqpoint{3.802689in}{1.788474in}}%
\pgfpathlineto{\pgfqpoint{3.563553in}{2.242061in}}%
\pgfusepath{stroke}%
\end{pgfscope}%
\begin{pgfscope}%
\pgfpathrectangle{\pgfqpoint{0.481978in}{0.331635in}}{\pgfqpoint{4.960000in}{3.696000in}}%
\pgfusepath{clip}%
\pgfsetrectcap%
\pgfsetroundjoin%
\pgfsetlinewidth{1.505625pt}%
\definecolor{currentstroke}{rgb}{0.631373,0.788235,0.956863}%
\pgfsetstrokecolor{currentstroke}%
\pgfsetstrokeopacity{0.800000}%
\pgfsetdash{}{0pt}%
\pgfpathmoveto{\pgfqpoint{3.541785in}{1.539826in}}%
\pgfpathlineto{\pgfqpoint{3.563553in}{2.242061in}}%
\pgfusepath{stroke}%
\end{pgfscope}%
\begin{pgfscope}%
\pgfpathrectangle{\pgfqpoint{0.481978in}{0.331635in}}{\pgfqpoint{4.960000in}{3.696000in}}%
\pgfusepath{clip}%
\pgfsetrectcap%
\pgfsetroundjoin%
\pgfsetlinewidth{1.505625pt}%
\definecolor{currentstroke}{rgb}{0.631373,0.788235,0.956863}%
\pgfsetstrokecolor{currentstroke}%
\pgfsetstrokeopacity{0.800000}%
\pgfsetdash{}{0pt}%
\pgfpathmoveto{\pgfqpoint{5.216523in}{2.434439in}}%
\pgfpathlineto{\pgfqpoint{3.563553in}{2.242061in}}%
\pgfusepath{stroke}%
\end{pgfscope}%
\begin{pgfscope}%
\pgfpathrectangle{\pgfqpoint{0.481978in}{0.331635in}}{\pgfqpoint{4.960000in}{3.696000in}}%
\pgfusepath{clip}%
\pgfsetrectcap%
\pgfsetroundjoin%
\pgfsetlinewidth{1.505625pt}%
\definecolor{currentstroke}{rgb}{0.631373,0.788235,0.956863}%
\pgfsetstrokecolor{currentstroke}%
\pgfsetstrokeopacity{0.800000}%
\pgfsetdash{}{0pt}%
\pgfpathmoveto{\pgfqpoint{3.043134in}{1.586573in}}%
\pgfpathlineto{\pgfqpoint{3.563553in}{2.242061in}}%
\pgfusepath{stroke}%
\end{pgfscope}%
\begin{pgfscope}%
\pgfpathrectangle{\pgfqpoint{0.481978in}{0.331635in}}{\pgfqpoint{4.960000in}{3.696000in}}%
\pgfusepath{clip}%
\pgfsetrectcap%
\pgfsetroundjoin%
\pgfsetlinewidth{1.505625pt}%
\definecolor{currentstroke}{rgb}{0.631373,0.788235,0.956863}%
\pgfsetstrokecolor{currentstroke}%
\pgfsetstrokeopacity{0.800000}%
\pgfsetdash{}{0pt}%
\pgfpathmoveto{\pgfqpoint{2.580652in}{3.184530in}}%
\pgfpathlineto{\pgfqpoint{3.563553in}{2.242061in}}%
\pgfusepath{stroke}%
\end{pgfscope}%
\begin{pgfscope}%
\pgfpathrectangle{\pgfqpoint{0.481978in}{0.331635in}}{\pgfqpoint{4.960000in}{3.696000in}}%
\pgfusepath{clip}%
\pgfsetrectcap%
\pgfsetroundjoin%
\pgfsetlinewidth{1.505625pt}%
\definecolor{currentstroke}{rgb}{0.631373,0.788235,0.956863}%
\pgfsetstrokecolor{currentstroke}%
\pgfsetstrokeopacity{0.800000}%
\pgfsetdash{}{0pt}%
\pgfpathmoveto{\pgfqpoint{4.001840in}{1.578365in}}%
\pgfpathlineto{\pgfqpoint{3.563553in}{2.242061in}}%
\pgfusepath{stroke}%
\end{pgfscope}%
\begin{pgfscope}%
\pgfpathrectangle{\pgfqpoint{0.481978in}{0.331635in}}{\pgfqpoint{4.960000in}{3.696000in}}%
\pgfusepath{clip}%
\pgfsetrectcap%
\pgfsetroundjoin%
\pgfsetlinewidth{1.505625pt}%
\definecolor{currentstroke}{rgb}{0.631373,0.788235,0.956863}%
\pgfsetstrokecolor{currentstroke}%
\pgfsetstrokeopacity{0.800000}%
\pgfsetdash{}{0pt}%
\pgfpathmoveto{\pgfqpoint{3.872808in}{2.556124in}}%
\pgfpathlineto{\pgfqpoint{3.563553in}{2.242061in}}%
\pgfusepath{stroke}%
\end{pgfscope}%
\begin{pgfscope}%
\pgfpathrectangle{\pgfqpoint{0.481978in}{0.331635in}}{\pgfqpoint{4.960000in}{3.696000in}}%
\pgfusepath{clip}%
\pgfsetrectcap%
\pgfsetroundjoin%
\pgfsetlinewidth{1.505625pt}%
\definecolor{currentstroke}{rgb}{0.631373,0.788235,0.956863}%
\pgfsetstrokecolor{currentstroke}%
\pgfsetstrokeopacity{0.800000}%
\pgfsetdash{}{0pt}%
\pgfpathmoveto{\pgfqpoint{3.135350in}{2.778513in}}%
\pgfpathlineto{\pgfqpoint{3.563553in}{2.242061in}}%
\pgfusepath{stroke}%
\end{pgfscope}%
\begin{pgfscope}%
\pgfpathrectangle{\pgfqpoint{0.481978in}{0.331635in}}{\pgfqpoint{4.960000in}{3.696000in}}%
\pgfusepath{clip}%
\pgfsetrectcap%
\pgfsetroundjoin%
\pgfsetlinewidth{1.505625pt}%
\definecolor{currentstroke}{rgb}{0.631373,0.788235,0.956863}%
\pgfsetstrokecolor{currentstroke}%
\pgfsetstrokeopacity{0.800000}%
\pgfsetdash{}{0pt}%
\pgfpathmoveto{\pgfqpoint{3.378442in}{1.866748in}}%
\pgfpathlineto{\pgfqpoint{3.563553in}{2.242061in}}%
\pgfusepath{stroke}%
\end{pgfscope}%
\begin{pgfscope}%
\pgfpathrectangle{\pgfqpoint{0.481978in}{0.331635in}}{\pgfqpoint{4.960000in}{3.696000in}}%
\pgfusepath{clip}%
\pgfsetrectcap%
\pgfsetroundjoin%
\pgfsetlinewidth{1.505625pt}%
\definecolor{currentstroke}{rgb}{0.631373,0.788235,0.956863}%
\pgfsetstrokecolor{currentstroke}%
\pgfsetstrokeopacity{0.800000}%
\pgfsetdash{}{0pt}%
\pgfpathmoveto{\pgfqpoint{3.656219in}{2.075588in}}%
\pgfpathlineto{\pgfqpoint{3.563553in}{2.242061in}}%
\pgfusepath{stroke}%
\end{pgfscope}%
\begin{pgfscope}%
\pgfpathrectangle{\pgfqpoint{0.481978in}{0.331635in}}{\pgfqpoint{4.960000in}{3.696000in}}%
\pgfusepath{clip}%
\pgfsetrectcap%
\pgfsetroundjoin%
\pgfsetlinewidth{1.505625pt}%
\definecolor{currentstroke}{rgb}{0.631373,0.788235,0.956863}%
\pgfsetstrokecolor{currentstroke}%
\pgfsetstrokeopacity{0.800000}%
\pgfsetdash{}{0pt}%
\pgfpathmoveto{\pgfqpoint{3.440189in}{3.009227in}}%
\pgfpathlineto{\pgfqpoint{3.563553in}{2.242061in}}%
\pgfusepath{stroke}%
\end{pgfscope}%
\begin{pgfscope}%
\pgfpathrectangle{\pgfqpoint{0.481978in}{0.331635in}}{\pgfqpoint{4.960000in}{3.696000in}}%
\pgfusepath{clip}%
\pgfsetrectcap%
\pgfsetroundjoin%
\pgfsetlinewidth{1.505625pt}%
\definecolor{currentstroke}{rgb}{0.631373,0.788235,0.956863}%
\pgfsetstrokecolor{currentstroke}%
\pgfsetstrokeopacity{0.800000}%
\pgfsetdash{}{0pt}%
\pgfpathmoveto{\pgfqpoint{4.280646in}{1.853152in}}%
\pgfpathlineto{\pgfqpoint{3.563553in}{2.242061in}}%
\pgfusepath{stroke}%
\end{pgfscope}%
\begin{pgfscope}%
\pgfpathrectangle{\pgfqpoint{0.481978in}{0.331635in}}{\pgfqpoint{4.960000in}{3.696000in}}%
\pgfusepath{clip}%
\pgfsetrectcap%
\pgfsetroundjoin%
\pgfsetlinewidth{1.505625pt}%
\definecolor{currentstroke}{rgb}{0.631373,0.788235,0.956863}%
\pgfsetstrokecolor{currentstroke}%
\pgfsetstrokeopacity{0.800000}%
\pgfsetdash{}{0pt}%
\pgfpathmoveto{\pgfqpoint{2.790063in}{2.237079in}}%
\pgfpathlineto{\pgfqpoint{3.563553in}{2.242061in}}%
\pgfusepath{stroke}%
\end{pgfscope}%
\begin{pgfscope}%
\pgfpathrectangle{\pgfqpoint{0.481978in}{0.331635in}}{\pgfqpoint{4.960000in}{3.696000in}}%
\pgfusepath{clip}%
\pgfsetrectcap%
\pgfsetroundjoin%
\pgfsetlinewidth{1.505625pt}%
\definecolor{currentstroke}{rgb}{0.631373,0.788235,0.956863}%
\pgfsetstrokecolor{currentstroke}%
\pgfsetstrokeopacity{0.800000}%
\pgfsetdash{}{0pt}%
\pgfpathmoveto{\pgfqpoint{4.256551in}{2.874269in}}%
\pgfpathlineto{\pgfqpoint{3.563553in}{2.242061in}}%
\pgfusepath{stroke}%
\end{pgfscope}%
\begin{pgfscope}%
\pgfpathrectangle{\pgfqpoint{0.481978in}{0.331635in}}{\pgfqpoint{4.960000in}{3.696000in}}%
\pgfusepath{clip}%
\pgfsetrectcap%
\pgfsetroundjoin%
\pgfsetlinewidth{1.505625pt}%
\definecolor{currentstroke}{rgb}{0.631373,0.788235,0.956863}%
\pgfsetstrokecolor{currentstroke}%
\pgfsetstrokeopacity{0.800000}%
\pgfsetdash{}{0pt}%
\pgfpathmoveto{\pgfqpoint{4.575205in}{2.193139in}}%
\pgfpathlineto{\pgfqpoint{3.563553in}{2.242061in}}%
\pgfusepath{stroke}%
\end{pgfscope}%
\begin{pgfscope}%
\pgfpathrectangle{\pgfqpoint{0.481978in}{0.331635in}}{\pgfqpoint{4.960000in}{3.696000in}}%
\pgfusepath{clip}%
\pgfsetrectcap%
\pgfsetroundjoin%
\pgfsetlinewidth{1.505625pt}%
\definecolor{currentstroke}{rgb}{0.631373,0.788235,0.956863}%
\pgfsetstrokecolor{currentstroke}%
\pgfsetstrokeopacity{0.800000}%
\pgfsetdash{}{0pt}%
\pgfpathmoveto{\pgfqpoint{3.448245in}{3.313520in}}%
\pgfpathlineto{\pgfqpoint{3.563553in}{2.242061in}}%
\pgfusepath{stroke}%
\end{pgfscope}%
\begin{pgfscope}%
\pgfpathrectangle{\pgfqpoint{0.481978in}{0.331635in}}{\pgfqpoint{4.960000in}{3.696000in}}%
\pgfusepath{clip}%
\pgfsetrectcap%
\pgfsetroundjoin%
\pgfsetlinewidth{1.505625pt}%
\definecolor{currentstroke}{rgb}{0.631373,0.788235,0.956863}%
\pgfsetstrokecolor{currentstroke}%
\pgfsetstrokeopacity{0.800000}%
\pgfsetdash{}{0pt}%
\pgfpathmoveto{\pgfqpoint{2.657795in}{1.379758in}}%
\pgfpathlineto{\pgfqpoint{3.563553in}{2.242061in}}%
\pgfusepath{stroke}%
\end{pgfscope}%
\begin{pgfscope}%
\pgfpathrectangle{\pgfqpoint{0.481978in}{0.331635in}}{\pgfqpoint{4.960000in}{3.696000in}}%
\pgfusepath{clip}%
\pgfsetrectcap%
\pgfsetroundjoin%
\pgfsetlinewidth{1.505625pt}%
\definecolor{currentstroke}{rgb}{0.631373,0.788235,0.956863}%
\pgfsetstrokecolor{currentstroke}%
\pgfsetstrokeopacity{0.800000}%
\pgfsetdash{}{0pt}%
\pgfpathmoveto{\pgfqpoint{3.546339in}{2.341602in}}%
\pgfpathlineto{\pgfqpoint{3.563553in}{2.242061in}}%
\pgfusepath{stroke}%
\end{pgfscope}%
\begin{pgfscope}%
\pgfpathrectangle{\pgfqpoint{0.481978in}{0.331635in}}{\pgfqpoint{4.960000in}{3.696000in}}%
\pgfusepath{clip}%
\pgfsetrectcap%
\pgfsetroundjoin%
\pgfsetlinewidth{1.505625pt}%
\definecolor{currentstroke}{rgb}{1.000000,0.705882,0.509804}%
\pgfsetstrokecolor{currentstroke}%
\pgfsetstrokeopacity{0.800000}%
\pgfsetdash{}{0pt}%
\pgfpathmoveto{\pgfqpoint{3.014752in}{3.195291in}}%
\pgfpathlineto{\pgfqpoint{2.628388in}{2.360622in}}%
\pgfusepath{stroke}%
\end{pgfscope}%
\begin{pgfscope}%
\pgfpathrectangle{\pgfqpoint{0.481978in}{0.331635in}}{\pgfqpoint{4.960000in}{3.696000in}}%
\pgfusepath{clip}%
\pgfsetrectcap%
\pgfsetroundjoin%
\pgfsetlinewidth{1.505625pt}%
\definecolor{currentstroke}{rgb}{1.000000,0.705882,0.509804}%
\pgfsetstrokecolor{currentstroke}%
\pgfsetstrokeopacity{0.800000}%
\pgfsetdash{}{0pt}%
\pgfpathmoveto{\pgfqpoint{5.085754in}{1.793372in}}%
\pgfpathlineto{\pgfqpoint{2.628388in}{2.360622in}}%
\pgfusepath{stroke}%
\end{pgfscope}%
\begin{pgfscope}%
\pgfpathrectangle{\pgfqpoint{0.481978in}{0.331635in}}{\pgfqpoint{4.960000in}{3.696000in}}%
\pgfusepath{clip}%
\pgfsetrectcap%
\pgfsetroundjoin%
\pgfsetlinewidth{1.505625pt}%
\definecolor{currentstroke}{rgb}{1.000000,0.705882,0.509804}%
\pgfsetstrokecolor{currentstroke}%
\pgfsetstrokeopacity{0.800000}%
\pgfsetdash{}{0pt}%
\pgfpathmoveto{\pgfqpoint{2.284015in}{2.600243in}}%
\pgfpathlineto{\pgfqpoint{2.628388in}{2.360622in}}%
\pgfusepath{stroke}%
\end{pgfscope}%
\begin{pgfscope}%
\pgfpathrectangle{\pgfqpoint{0.481978in}{0.331635in}}{\pgfqpoint{4.960000in}{3.696000in}}%
\pgfusepath{clip}%
\pgfsetrectcap%
\pgfsetroundjoin%
\pgfsetlinewidth{1.505625pt}%
\definecolor{currentstroke}{rgb}{1.000000,0.705882,0.509804}%
\pgfsetstrokecolor{currentstroke}%
\pgfsetstrokeopacity{0.800000}%
\pgfsetdash{}{0pt}%
\pgfpathmoveto{\pgfqpoint{1.922934in}{2.366558in}}%
\pgfpathlineto{\pgfqpoint{2.628388in}{2.360622in}}%
\pgfusepath{stroke}%
\end{pgfscope}%
\begin{pgfscope}%
\pgfpathrectangle{\pgfqpoint{0.481978in}{0.331635in}}{\pgfqpoint{4.960000in}{3.696000in}}%
\pgfusepath{clip}%
\pgfsetrectcap%
\pgfsetroundjoin%
\pgfsetlinewidth{1.505625pt}%
\definecolor{currentstroke}{rgb}{1.000000,0.705882,0.509804}%
\pgfsetstrokecolor{currentstroke}%
\pgfsetstrokeopacity{0.800000}%
\pgfsetdash{}{0pt}%
\pgfpathmoveto{\pgfqpoint{1.598169in}{2.719451in}}%
\pgfpathlineto{\pgfqpoint{2.628388in}{2.360622in}}%
\pgfusepath{stroke}%
\end{pgfscope}%
\begin{pgfscope}%
\pgfpathrectangle{\pgfqpoint{0.481978in}{0.331635in}}{\pgfqpoint{4.960000in}{3.696000in}}%
\pgfusepath{clip}%
\pgfsetrectcap%
\pgfsetroundjoin%
\pgfsetlinewidth{1.505625pt}%
\definecolor{currentstroke}{rgb}{1.000000,0.705882,0.509804}%
\pgfsetstrokecolor{currentstroke}%
\pgfsetstrokeopacity{0.800000}%
\pgfsetdash{}{0pt}%
\pgfpathmoveto{\pgfqpoint{3.105033in}{0.810912in}}%
\pgfpathlineto{\pgfqpoint{2.628388in}{2.360622in}}%
\pgfusepath{stroke}%
\end{pgfscope}%
\begin{pgfscope}%
\pgfpathrectangle{\pgfqpoint{0.481978in}{0.331635in}}{\pgfqpoint{4.960000in}{3.696000in}}%
\pgfusepath{clip}%
\pgfsetrectcap%
\pgfsetroundjoin%
\pgfsetlinewidth{1.505625pt}%
\definecolor{currentstroke}{rgb}{1.000000,0.705882,0.509804}%
\pgfsetstrokecolor{currentstroke}%
\pgfsetstrokeopacity{0.800000}%
\pgfsetdash{}{0pt}%
\pgfpathmoveto{\pgfqpoint{1.895234in}{3.315197in}}%
\pgfpathlineto{\pgfqpoint{2.628388in}{2.360622in}}%
\pgfusepath{stroke}%
\end{pgfscope}%
\begin{pgfscope}%
\pgfpathrectangle{\pgfqpoint{0.481978in}{0.331635in}}{\pgfqpoint{4.960000in}{3.696000in}}%
\pgfusepath{clip}%
\pgfsetrectcap%
\pgfsetroundjoin%
\pgfsetlinewidth{1.505625pt}%
\definecolor{currentstroke}{rgb}{1.000000,0.705882,0.509804}%
\pgfsetstrokecolor{currentstroke}%
\pgfsetstrokeopacity{0.800000}%
\pgfsetdash{}{0pt}%
\pgfpathmoveto{\pgfqpoint{3.360808in}{1.125508in}}%
\pgfpathlineto{\pgfqpoint{2.628388in}{2.360622in}}%
\pgfusepath{stroke}%
\end{pgfscope}%
\begin{pgfscope}%
\pgfpathrectangle{\pgfqpoint{0.481978in}{0.331635in}}{\pgfqpoint{4.960000in}{3.696000in}}%
\pgfusepath{clip}%
\pgfsetrectcap%
\pgfsetroundjoin%
\pgfsetlinewidth{1.505625pt}%
\definecolor{currentstroke}{rgb}{1.000000,0.705882,0.509804}%
\pgfsetstrokecolor{currentstroke}%
\pgfsetstrokeopacity{0.800000}%
\pgfsetdash{}{0pt}%
\pgfpathmoveto{\pgfqpoint{4.027785in}{2.169652in}}%
\pgfpathlineto{\pgfqpoint{2.628388in}{2.360622in}}%
\pgfusepath{stroke}%
\end{pgfscope}%
\begin{pgfscope}%
\pgfpathrectangle{\pgfqpoint{0.481978in}{0.331635in}}{\pgfqpoint{4.960000in}{3.696000in}}%
\pgfusepath{clip}%
\pgfsetrectcap%
\pgfsetroundjoin%
\pgfsetlinewidth{1.505625pt}%
\definecolor{currentstroke}{rgb}{1.000000,0.705882,0.509804}%
\pgfsetstrokecolor{currentstroke}%
\pgfsetstrokeopacity{0.800000}%
\pgfsetdash{}{0pt}%
\pgfpathmoveto{\pgfqpoint{1.211108in}{2.465118in}}%
\pgfpathlineto{\pgfqpoint{2.628388in}{2.360622in}}%
\pgfusepath{stroke}%
\end{pgfscope}%
\begin{pgfscope}%
\pgfpathrectangle{\pgfqpoint{0.481978in}{0.331635in}}{\pgfqpoint{4.960000in}{3.696000in}}%
\pgfusepath{clip}%
\pgfsetrectcap%
\pgfsetroundjoin%
\pgfsetlinewidth{1.505625pt}%
\definecolor{currentstroke}{rgb}{1.000000,0.705882,0.509804}%
\pgfsetstrokecolor{currentstroke}%
\pgfsetstrokeopacity{0.800000}%
\pgfsetdash{}{0pt}%
\pgfpathmoveto{\pgfqpoint{4.184791in}{0.943082in}}%
\pgfpathlineto{\pgfqpoint{2.628388in}{2.360622in}}%
\pgfusepath{stroke}%
\end{pgfscope}%
\begin{pgfscope}%
\pgfpathrectangle{\pgfqpoint{0.481978in}{0.331635in}}{\pgfqpoint{4.960000in}{3.696000in}}%
\pgfusepath{clip}%
\pgfsetrectcap%
\pgfsetroundjoin%
\pgfsetlinewidth{1.505625pt}%
\definecolor{currentstroke}{rgb}{1.000000,0.705882,0.509804}%
\pgfsetstrokecolor{currentstroke}%
\pgfsetstrokeopacity{0.800000}%
\pgfsetdash{}{0pt}%
\pgfpathmoveto{\pgfqpoint{1.789027in}{1.305832in}}%
\pgfpathlineto{\pgfqpoint{2.628388in}{2.360622in}}%
\pgfusepath{stroke}%
\end{pgfscope}%
\begin{pgfscope}%
\pgfpathrectangle{\pgfqpoint{0.481978in}{0.331635in}}{\pgfqpoint{4.960000in}{3.696000in}}%
\pgfusepath{clip}%
\pgfsetrectcap%
\pgfsetroundjoin%
\pgfsetlinewidth{1.505625pt}%
\definecolor{currentstroke}{rgb}{1.000000,0.705882,0.509804}%
\pgfsetstrokecolor{currentstroke}%
\pgfsetstrokeopacity{0.800000}%
\pgfsetdash{}{0pt}%
\pgfpathmoveto{\pgfqpoint{2.368419in}{1.005292in}}%
\pgfpathlineto{\pgfqpoint{2.628388in}{2.360622in}}%
\pgfusepath{stroke}%
\end{pgfscope}%
\begin{pgfscope}%
\pgfpathrectangle{\pgfqpoint{0.481978in}{0.331635in}}{\pgfqpoint{4.960000in}{3.696000in}}%
\pgfusepath{clip}%
\pgfsetrectcap%
\pgfsetroundjoin%
\pgfsetlinewidth{1.505625pt}%
\definecolor{currentstroke}{rgb}{1.000000,0.705882,0.509804}%
\pgfsetstrokecolor{currentstroke}%
\pgfsetstrokeopacity{0.800000}%
\pgfsetdash{}{0pt}%
\pgfpathmoveto{\pgfqpoint{1.427964in}{2.964440in}}%
\pgfpathlineto{\pgfqpoint{2.628388in}{2.360622in}}%
\pgfusepath{stroke}%
\end{pgfscope}%
\begin{pgfscope}%
\pgfpathrectangle{\pgfqpoint{0.481978in}{0.331635in}}{\pgfqpoint{4.960000in}{3.696000in}}%
\pgfusepath{clip}%
\pgfsetrectcap%
\pgfsetroundjoin%
\pgfsetlinewidth{1.505625pt}%
\definecolor{currentstroke}{rgb}{1.000000,0.705882,0.509804}%
\pgfsetstrokecolor{currentstroke}%
\pgfsetstrokeopacity{0.800000}%
\pgfsetdash{}{0pt}%
\pgfpathmoveto{\pgfqpoint{3.393546in}{3.859635in}}%
\pgfpathlineto{\pgfqpoint{2.628388in}{2.360622in}}%
\pgfusepath{stroke}%
\end{pgfscope}%
\begin{pgfscope}%
\pgfpathrectangle{\pgfqpoint{0.481978in}{0.331635in}}{\pgfqpoint{4.960000in}{3.696000in}}%
\pgfusepath{clip}%
\pgfsetrectcap%
\pgfsetroundjoin%
\pgfsetlinewidth{1.505625pt}%
\definecolor{currentstroke}{rgb}{1.000000,0.705882,0.509804}%
\pgfsetstrokecolor{currentstroke}%
\pgfsetstrokeopacity{0.800000}%
\pgfsetdash{}{0pt}%
\pgfpathmoveto{\pgfqpoint{1.952110in}{1.628055in}}%
\pgfpathlineto{\pgfqpoint{2.628388in}{2.360622in}}%
\pgfusepath{stroke}%
\end{pgfscope}%
\begin{pgfscope}%
\pgfpathrectangle{\pgfqpoint{0.481978in}{0.331635in}}{\pgfqpoint{4.960000in}{3.696000in}}%
\pgfusepath{clip}%
\pgfsetrectcap%
\pgfsetroundjoin%
\pgfsetlinewidth{1.505625pt}%
\definecolor{currentstroke}{rgb}{1.000000,0.705882,0.509804}%
\pgfsetstrokecolor{currentstroke}%
\pgfsetstrokeopacity{0.800000}%
\pgfsetdash{}{0pt}%
\pgfpathmoveto{\pgfqpoint{2.726428in}{2.857591in}}%
\pgfpathlineto{\pgfqpoint{2.628388in}{2.360622in}}%
\pgfusepath{stroke}%
\end{pgfscope}%
\begin{pgfscope}%
\pgfpathrectangle{\pgfqpoint{0.481978in}{0.331635in}}{\pgfqpoint{4.960000in}{3.696000in}}%
\pgfusepath{clip}%
\pgfsetrectcap%
\pgfsetroundjoin%
\pgfsetlinewidth{1.505625pt}%
\definecolor{currentstroke}{rgb}{1.000000,0.705882,0.509804}%
\pgfsetstrokecolor{currentstroke}%
\pgfsetstrokeopacity{0.800000}%
\pgfsetdash{}{0pt}%
\pgfpathmoveto{\pgfqpoint{1.372997in}{1.856203in}}%
\pgfpathlineto{\pgfqpoint{2.628388in}{2.360622in}}%
\pgfusepath{stroke}%
\end{pgfscope}%
\begin{pgfscope}%
\pgfpathrectangle{\pgfqpoint{0.481978in}{0.331635in}}{\pgfqpoint{4.960000in}{3.696000in}}%
\pgfusepath{clip}%
\pgfsetrectcap%
\pgfsetroundjoin%
\pgfsetlinewidth{1.505625pt}%
\definecolor{currentstroke}{rgb}{1.000000,0.705882,0.509804}%
\pgfsetstrokecolor{currentstroke}%
\pgfsetstrokeopacity{0.800000}%
\pgfsetdash{}{0pt}%
\pgfpathmoveto{\pgfqpoint{0.707432in}{3.144211in}}%
\pgfpathlineto{\pgfqpoint{2.628388in}{2.360622in}}%
\pgfusepath{stroke}%
\end{pgfscope}%
\begin{pgfscope}%
\pgfpathrectangle{\pgfqpoint{0.481978in}{0.331635in}}{\pgfqpoint{4.960000in}{3.696000in}}%
\pgfusepath{clip}%
\pgfsetrectcap%
\pgfsetroundjoin%
\pgfsetlinewidth{1.505625pt}%
\definecolor{currentstroke}{rgb}{1.000000,0.705882,0.509804}%
\pgfsetstrokecolor{currentstroke}%
\pgfsetstrokeopacity{0.800000}%
\pgfsetdash{}{0pt}%
\pgfpathmoveto{\pgfqpoint{3.782152in}{0.499635in}}%
\pgfpathlineto{\pgfqpoint{2.628388in}{2.360622in}}%
\pgfusepath{stroke}%
\end{pgfscope}%
\begin{pgfscope}%
\pgfpathrectangle{\pgfqpoint{0.481978in}{0.331635in}}{\pgfqpoint{4.960000in}{3.696000in}}%
\pgfusepath{clip}%
\pgfsetrectcap%
\pgfsetroundjoin%
\pgfsetlinewidth{1.505625pt}%
\definecolor{currentstroke}{rgb}{1.000000,0.705882,0.509804}%
\pgfsetstrokecolor{currentstroke}%
\pgfsetstrokeopacity{0.800000}%
\pgfsetdash{}{0pt}%
\pgfpathmoveto{\pgfqpoint{4.764267in}{2.747524in}}%
\pgfpathlineto{\pgfqpoint{2.628388in}{2.360622in}}%
\pgfusepath{stroke}%
\end{pgfscope}%
\begin{pgfscope}%
\pgfpathrectangle{\pgfqpoint{0.481978in}{0.331635in}}{\pgfqpoint{4.960000in}{3.696000in}}%
\pgfusepath{clip}%
\pgfsetrectcap%
\pgfsetroundjoin%
\pgfsetlinewidth{1.505625pt}%
\definecolor{currentstroke}{rgb}{1.000000,0.705882,0.509804}%
\pgfsetstrokecolor{currentstroke}%
\pgfsetstrokeopacity{0.800000}%
\pgfsetdash{}{0pt}%
\pgfpathmoveto{\pgfqpoint{1.027884in}{3.310449in}}%
\pgfpathlineto{\pgfqpoint{2.628388in}{2.360622in}}%
\pgfusepath{stroke}%
\end{pgfscope}%
\begin{pgfscope}%
\pgfpathrectangle{\pgfqpoint{0.481978in}{0.331635in}}{\pgfqpoint{4.960000in}{3.696000in}}%
\pgfusepath{clip}%
\pgfsetrectcap%
\pgfsetroundjoin%
\pgfsetlinewidth{1.505625pt}%
\definecolor{currentstroke}{rgb}{1.000000,0.705882,0.509804}%
\pgfsetstrokecolor{currentstroke}%
\pgfsetstrokeopacity{0.800000}%
\pgfsetdash{}{0pt}%
\pgfpathmoveto{\pgfqpoint{2.571490in}{3.587745in}}%
\pgfpathlineto{\pgfqpoint{2.628388in}{2.360622in}}%
\pgfusepath{stroke}%
\end{pgfscope}%
\begin{pgfscope}%
\pgfpathrectangle{\pgfqpoint{0.481978in}{0.331635in}}{\pgfqpoint{4.960000in}{3.696000in}}%
\pgfusepath{clip}%
\pgfsetrectcap%
\pgfsetroundjoin%
\pgfsetlinewidth{1.505625pt}%
\definecolor{currentstroke}{rgb}{1.000000,0.705882,0.509804}%
\pgfsetstrokecolor{currentstroke}%
\pgfsetstrokeopacity{0.800000}%
\pgfsetdash{}{0pt}%
\pgfpathmoveto{\pgfqpoint{2.788619in}{2.586793in}}%
\pgfpathlineto{\pgfqpoint{2.628388in}{2.360622in}}%
\pgfusepath{stroke}%
\end{pgfscope}%
\begin{pgfscope}%
\pgfpathrectangle{\pgfqpoint{0.481978in}{0.331635in}}{\pgfqpoint{4.960000in}{3.696000in}}%
\pgfusepath{clip}%
\pgfsetrectcap%
\pgfsetroundjoin%
\pgfsetlinewidth{1.505625pt}%
\definecolor{currentstroke}{rgb}{1.000000,0.705882,0.509804}%
\pgfsetstrokecolor{currentstroke}%
\pgfsetstrokeopacity{0.800000}%
\pgfsetdash{}{0pt}%
\pgfpathmoveto{\pgfqpoint{3.846440in}{2.939605in}}%
\pgfpathlineto{\pgfqpoint{2.628388in}{2.360622in}}%
\pgfusepath{stroke}%
\end{pgfscope}%
\begin{pgfscope}%
\pgfpathrectangle{\pgfqpoint{0.481978in}{0.331635in}}{\pgfqpoint{4.960000in}{3.696000in}}%
\pgfusepath{clip}%
\pgfsetrectcap%
\pgfsetroundjoin%
\pgfsetlinewidth{1.505625pt}%
\definecolor{currentstroke}{rgb}{1.000000,0.705882,0.509804}%
\pgfsetstrokecolor{currentstroke}%
\pgfsetstrokeopacity{0.800000}%
\pgfsetdash{}{0pt}%
\pgfpathmoveto{\pgfqpoint{4.322794in}{2.464911in}}%
\pgfpathlineto{\pgfqpoint{2.628388in}{2.360622in}}%
\pgfusepath{stroke}%
\end{pgfscope}%
\begin{pgfscope}%
\pgfpathrectangle{\pgfqpoint{0.481978in}{0.331635in}}{\pgfqpoint{4.960000in}{3.696000in}}%
\pgfusepath{clip}%
\pgfsetrectcap%
\pgfsetroundjoin%
\pgfsetlinewidth{1.505625pt}%
\definecolor{currentstroke}{rgb}{1.000000,0.705882,0.509804}%
\pgfsetstrokecolor{currentstroke}%
\pgfsetstrokeopacity{0.800000}%
\pgfsetdash{}{0pt}%
\pgfpathmoveto{\pgfqpoint{1.013445in}{2.895267in}}%
\pgfpathlineto{\pgfqpoint{2.628388in}{2.360622in}}%
\pgfusepath{stroke}%
\end{pgfscope}%
\begin{pgfscope}%
\pgfpathrectangle{\pgfqpoint{0.481978in}{0.331635in}}{\pgfqpoint{4.960000in}{3.696000in}}%
\pgfusepath{clip}%
\pgfsetrectcap%
\pgfsetroundjoin%
\pgfsetlinewidth{1.505625pt}%
\definecolor{currentstroke}{rgb}{1.000000,0.705882,0.509804}%
\pgfsetstrokecolor{currentstroke}%
\pgfsetstrokeopacity{0.800000}%
\pgfsetdash{}{0pt}%
\pgfpathmoveto{\pgfqpoint{2.049477in}{2.939835in}}%
\pgfpathlineto{\pgfqpoint{2.628388in}{2.360622in}}%
\pgfusepath{stroke}%
\end{pgfscope}%
\begin{pgfscope}%
\pgfsetrectcap%
\pgfsetmiterjoin%
\pgfsetlinewidth{0.803000pt}%
\definecolor{currentstroke}{rgb}{0.000000,0.000000,0.000000}%
\pgfsetstrokecolor{currentstroke}%
\pgfsetdash{}{0pt}%
\pgfpathmoveto{\pgfqpoint{0.481978in}{0.331635in}}%
\pgfpathlineto{\pgfqpoint{0.481978in}{4.027635in}}%
\pgfusepath{stroke}%
\end{pgfscope}%
\begin{pgfscope}%
\pgfsetrectcap%
\pgfsetmiterjoin%
\pgfsetlinewidth{0.803000pt}%
\definecolor{currentstroke}{rgb}{0.000000,0.000000,0.000000}%
\pgfsetstrokecolor{currentstroke}%
\pgfsetdash{}{0pt}%
\pgfpathmoveto{\pgfqpoint{5.441978in}{0.331635in}}%
\pgfpathlineto{\pgfqpoint{5.441978in}{4.027635in}}%
\pgfusepath{stroke}%
\end{pgfscope}%
\begin{pgfscope}%
\pgfsetrectcap%
\pgfsetmiterjoin%
\pgfsetlinewidth{0.803000pt}%
\definecolor{currentstroke}{rgb}{0.000000,0.000000,0.000000}%
\pgfsetstrokecolor{currentstroke}%
\pgfsetdash{}{0pt}%
\pgfpathmoveto{\pgfqpoint{0.481978in}{0.331635in}}%
\pgfpathlineto{\pgfqpoint{5.441978in}{0.331635in}}%
\pgfusepath{stroke}%
\end{pgfscope}%
\begin{pgfscope}%
\pgfsetrectcap%
\pgfsetmiterjoin%
\pgfsetlinewidth{0.803000pt}%
\definecolor{currentstroke}{rgb}{0.000000,0.000000,0.000000}%
\pgfsetstrokecolor{currentstroke}%
\pgfsetdash{}{0pt}%
\pgfpathmoveto{\pgfqpoint{0.481978in}{4.027635in}}%
\pgfpathlineto{\pgfqpoint{5.441978in}{4.027635in}}%
\pgfusepath{stroke}%
\end{pgfscope}%
\begin{pgfscope}%
\definecolor{textcolor}{rgb}{0.000000,0.000000,0.000000}%
\pgfsetstrokecolor{textcolor}%
\pgfsetfillcolor{textcolor}%
\pgftext[x=2.961978in,y=4.110968in,,base]{\color{textcolor}\sffamily\fontsize{12.000000}{14.400000}\selectfont t-SNE for pix3d and scenenet}%
\end{pgfscope}%
\begin{pgfscope}%
\pgfsetbuttcap%
\pgfsetmiterjoin%
\definecolor{currentfill}{rgb}{1.000000,1.000000,1.000000}%
\pgfsetfillcolor{currentfill}%
\pgfsetfillopacity{0.800000}%
\pgfsetlinewidth{1.003750pt}%
\definecolor{currentstroke}{rgb}{0.800000,0.800000,0.800000}%
\pgfsetstrokecolor{currentstroke}%
\pgfsetstrokeopacity{0.800000}%
\pgfsetdash{}{0pt}%
\pgfpathmoveto{\pgfqpoint{4.264732in}{3.508809in}}%
\pgfpathlineto{\pgfqpoint{5.344756in}{3.508809in}}%
\pgfpathquadraticcurveto{\pgfqpoint{5.372533in}{3.508809in}}{\pgfqpoint{5.372533in}{3.536587in}}%
\pgfpathlineto{\pgfqpoint{5.372533in}{3.930413in}}%
\pgfpathquadraticcurveto{\pgfqpoint{5.372533in}{3.958191in}}{\pgfqpoint{5.344756in}{3.958191in}}%
\pgfpathlineto{\pgfqpoint{4.264732in}{3.958191in}}%
\pgfpathquadraticcurveto{\pgfqpoint{4.236954in}{3.958191in}}{\pgfqpoint{4.236954in}{3.930413in}}%
\pgfpathlineto{\pgfqpoint{4.236954in}{3.536587in}}%
\pgfpathquadraticcurveto{\pgfqpoint{4.236954in}{3.508809in}}{\pgfqpoint{4.264732in}{3.508809in}}%
\pgfpathclose%
\pgfusepath{stroke,fill}%
\end{pgfscope}%
\begin{pgfscope}%
\pgfsetbuttcap%
\pgfsetroundjoin%
\definecolor{currentfill}{rgb}{0.631373,0.788235,0.956863}%
\pgfsetfillcolor{currentfill}%
\pgfsetlinewidth{1.003750pt}%
\definecolor{currentstroke}{rgb}{0.631373,0.788235,0.956863}%
\pgfsetstrokecolor{currentstroke}%
\pgfsetdash{}{0pt}%
\pgfsys@defobject{currentmarker}{\pgfqpoint{-0.041667in}{-0.041667in}}{\pgfqpoint{0.041667in}{0.041667in}}{%
\pgfpathmoveto{\pgfqpoint{0.000000in}{-0.041667in}}%
\pgfpathcurveto{\pgfqpoint{0.011050in}{-0.041667in}}{\pgfqpoint{0.021649in}{-0.037276in}}{\pgfqpoint{0.029463in}{-0.029463in}}%
\pgfpathcurveto{\pgfqpoint{0.037276in}{-0.021649in}}{\pgfqpoint{0.041667in}{-0.011050in}}{\pgfqpoint{0.041667in}{0.000000in}}%
\pgfpathcurveto{\pgfqpoint{0.041667in}{0.011050in}}{\pgfqpoint{0.037276in}{0.021649in}}{\pgfqpoint{0.029463in}{0.029463in}}%
\pgfpathcurveto{\pgfqpoint{0.021649in}{0.037276in}}{\pgfqpoint{0.011050in}{0.041667in}}{\pgfqpoint{0.000000in}{0.041667in}}%
\pgfpathcurveto{\pgfqpoint{-0.011050in}{0.041667in}}{\pgfqpoint{-0.021649in}{0.037276in}}{\pgfqpoint{-0.029463in}{0.029463in}}%
\pgfpathcurveto{\pgfqpoint{-0.037276in}{0.021649in}}{\pgfqpoint{-0.041667in}{0.011050in}}{\pgfqpoint{-0.041667in}{0.000000in}}%
\pgfpathcurveto{\pgfqpoint{-0.041667in}{-0.011050in}}{\pgfqpoint{-0.037276in}{-0.021649in}}{\pgfqpoint{-0.029463in}{-0.029463in}}%
\pgfpathcurveto{\pgfqpoint{-0.021649in}{-0.037276in}}{\pgfqpoint{-0.011050in}{-0.041667in}}{\pgfqpoint{0.000000in}{-0.041667in}}%
\pgfpathclose%
\pgfusepath{stroke,fill}%
}%
\begin{pgfscope}%
\pgfsys@transformshift{4.431398in}{3.833570in}%
\pgfsys@useobject{currentmarker}{}%
\end{pgfscope}%
\end{pgfscope}%
\begin{pgfscope}%
\definecolor{textcolor}{rgb}{0.000000,0.000000,0.000000}%
\pgfsetstrokecolor{textcolor}%
\pgfsetfillcolor{textcolor}%
\pgftext[x=4.681398in,y=3.797112in,left,base]{\color{textcolor}\sffamily\fontsize{10.000000}{12.000000}\selectfont scenenet}%
\end{pgfscope}%
\begin{pgfscope}%
\pgfsetbuttcap%
\pgfsetroundjoin%
\definecolor{currentfill}{rgb}{1.000000,0.705882,0.509804}%
\pgfsetfillcolor{currentfill}%
\pgfsetlinewidth{1.003750pt}%
\definecolor{currentstroke}{rgb}{1.000000,0.705882,0.509804}%
\pgfsetstrokecolor{currentstroke}%
\pgfsetdash{}{0pt}%
\pgfsys@defobject{currentmarker}{\pgfqpoint{-0.041667in}{-0.041667in}}{\pgfqpoint{0.041667in}{0.041667in}}{%
\pgfpathmoveto{\pgfqpoint{0.000000in}{-0.041667in}}%
\pgfpathcurveto{\pgfqpoint{0.011050in}{-0.041667in}}{\pgfqpoint{0.021649in}{-0.037276in}}{\pgfqpoint{0.029463in}{-0.029463in}}%
\pgfpathcurveto{\pgfqpoint{0.037276in}{-0.021649in}}{\pgfqpoint{0.041667in}{-0.011050in}}{\pgfqpoint{0.041667in}{0.000000in}}%
\pgfpathcurveto{\pgfqpoint{0.041667in}{0.011050in}}{\pgfqpoint{0.037276in}{0.021649in}}{\pgfqpoint{0.029463in}{0.029463in}}%
\pgfpathcurveto{\pgfqpoint{0.021649in}{0.037276in}}{\pgfqpoint{0.011050in}{0.041667in}}{\pgfqpoint{0.000000in}{0.041667in}}%
\pgfpathcurveto{\pgfqpoint{-0.011050in}{0.041667in}}{\pgfqpoint{-0.021649in}{0.037276in}}{\pgfqpoint{-0.029463in}{0.029463in}}%
\pgfpathcurveto{\pgfqpoint{-0.037276in}{0.021649in}}{\pgfqpoint{-0.041667in}{0.011050in}}{\pgfqpoint{-0.041667in}{0.000000in}}%
\pgfpathcurveto{\pgfqpoint{-0.041667in}{-0.011050in}}{\pgfqpoint{-0.037276in}{-0.021649in}}{\pgfqpoint{-0.029463in}{-0.029463in}}%
\pgfpathcurveto{\pgfqpoint{-0.021649in}{-0.037276in}}{\pgfqpoint{-0.011050in}{-0.041667in}}{\pgfqpoint{0.000000in}{-0.041667in}}%
\pgfpathclose%
\pgfusepath{stroke,fill}%
}%
\begin{pgfscope}%
\pgfsys@transformshift{4.431398in}{3.629713in}%
\pgfsys@useobject{currentmarker}{}%
\end{pgfscope}%
\end{pgfscope}%
\begin{pgfscope}%
\definecolor{textcolor}{rgb}{0.000000,0.000000,0.000000}%
\pgfsetstrokecolor{textcolor}%
\pgfsetfillcolor{textcolor}%
\pgftext[x=4.681398in,y=3.593255in,left,base]{\color{textcolor}\sffamily\fontsize{10.000000}{12.000000}\selectfont pix3d}%
\end{pgfscope}%
\end{pgfpicture}%
\makeatother%
\endgroup%
}\\
    \resizebox{0.49\linewidth}{5cm}{%% Creator: Matplotlib, PGF backend
%%
%% To include the figure in your LaTeX document, write
%%   \input{<filename>.pgf}
%%
%% Make sure the required packages are loaded in your preamble
%%   \usepackage{pgf}
%%
%% Figures using additional raster images can only be included by \input if
%% they are in the same directory as the main LaTeX file. For loading figures
%% from other directories you can use the `import` package
%%   \usepackage{import}
%%
%% and then include the figures with
%%   \import{<path to file>}{<filename>.pgf}
%%
%% Matplotlib used the following preamble
%%   \usepackage{fontspec}
%%   \setmainfont{DejaVuSerif.ttf}[Path=\detokenize{/Users/apple/opt/anaconda3/envs/kaolin/lib/python3.7/site-packages/matplotlib/mpl-data/fonts/ttf/}]
%%   \setsansfont{DejaVuSans.ttf}[Path=\detokenize{/Users/apple/opt/anaconda3/envs/kaolin/lib/python3.7/site-packages/matplotlib/mpl-data/fonts/ttf/}]
%%   \setmonofont{DejaVuSansMono.ttf}[Path=\detokenize{/Users/apple/opt/anaconda3/envs/kaolin/lib/python3.7/site-packages/matplotlib/mpl-data/fonts/ttf/}]
%%
\begingroup%
\makeatletter%
\begin{pgfpicture}%
\pgfpathrectangle{\pgfpointorigin}{\pgfqpoint{11.374274in}{8.341596in}}%
\pgfusepath{use as bounding box, clip}%
\begin{pgfscope}%
\pgfsetbuttcap%
\pgfsetmiterjoin%
\definecolor{currentfill}{rgb}{1.000000,1.000000,1.000000}%
\pgfsetfillcolor{currentfill}%
\pgfsetlinewidth{0.000000pt}%
\definecolor{currentstroke}{rgb}{1.000000,1.000000,1.000000}%
\pgfsetstrokecolor{currentstroke}%
\pgfsetdash{}{0pt}%
\pgfpathmoveto{\pgfqpoint{0.000000in}{0.000000in}}%
\pgfpathlineto{\pgfqpoint{11.374274in}{0.000000in}}%
\pgfpathlineto{\pgfqpoint{11.374274in}{8.341596in}}%
\pgfpathlineto{\pgfqpoint{0.000000in}{8.341596in}}%
\pgfpathclose%
\pgfusepath{fill}%
\end{pgfscope}%
\begin{pgfscope}%
\pgfsetbuttcap%
\pgfsetmiterjoin%
\definecolor{currentfill}{rgb}{1.000000,1.000000,1.000000}%
\pgfsetfillcolor{currentfill}%
\pgfsetlinewidth{0.000000pt}%
\definecolor{currentstroke}{rgb}{0.000000,0.000000,0.000000}%
\pgfsetstrokecolor{currentstroke}%
\pgfsetstrokeopacity{0.000000}%
\pgfsetdash{}{0pt}%
\pgfpathmoveto{\pgfqpoint{0.570343in}{0.331635in}}%
\pgfpathlineto{\pgfqpoint{9.870343in}{0.331635in}}%
\pgfpathlineto{\pgfqpoint{9.870343in}{8.031635in}}%
\pgfpathlineto{\pgfqpoint{0.570343in}{8.031635in}}%
\pgfpathclose%
\pgfusepath{fill}%
\end{pgfscope}%
\begin{pgfscope}%
\pgfpathrectangle{\pgfqpoint{0.570343in}{0.331635in}}{\pgfqpoint{9.300000in}{7.700000in}}%
\pgfusepath{clip}%
\pgfsetbuttcap%
\pgfsetroundjoin%
\definecolor{currentfill}{rgb}{0.631373,0.788235,0.956863}%
\pgfsetfillcolor{currentfill}%
\pgfsetlinewidth{0.481800pt}%
\definecolor{currentstroke}{rgb}{1.000000,1.000000,1.000000}%
\pgfsetstrokecolor{currentstroke}%
\pgfsetdash{}{0pt}%
\pgfpathmoveto{\pgfqpoint{2.738718in}{6.340074in}}%
\pgfpathcurveto{\pgfqpoint{2.749768in}{6.340074in}}{\pgfqpoint{2.760367in}{6.344464in}}{\pgfqpoint{2.768181in}{6.352277in}}%
\pgfpathcurveto{\pgfqpoint{2.775994in}{6.360091in}}{\pgfqpoint{2.780384in}{6.370690in}}{\pgfqpoint{2.780384in}{6.381740in}}%
\pgfpathcurveto{\pgfqpoint{2.780384in}{6.392790in}}{\pgfqpoint{2.775994in}{6.403389in}}{\pgfqpoint{2.768181in}{6.411203in}}%
\pgfpathcurveto{\pgfqpoint{2.760367in}{6.419017in}}{\pgfqpoint{2.749768in}{6.423407in}}{\pgfqpoint{2.738718in}{6.423407in}}%
\pgfpathcurveto{\pgfqpoint{2.727668in}{6.423407in}}{\pgfqpoint{2.717069in}{6.419017in}}{\pgfqpoint{2.709255in}{6.411203in}}%
\pgfpathcurveto{\pgfqpoint{2.701441in}{6.403389in}}{\pgfqpoint{2.697051in}{6.392790in}}{\pgfqpoint{2.697051in}{6.381740in}}%
\pgfpathcurveto{\pgfqpoint{2.697051in}{6.370690in}}{\pgfqpoint{2.701441in}{6.360091in}}{\pgfqpoint{2.709255in}{6.352277in}}%
\pgfpathcurveto{\pgfqpoint{2.717069in}{6.344464in}}{\pgfqpoint{2.727668in}{6.340074in}}{\pgfqpoint{2.738718in}{6.340074in}}%
\pgfpathclose%
\pgfusepath{stroke,fill}%
\end{pgfscope}%
\begin{pgfscope}%
\pgfpathrectangle{\pgfqpoint{0.570343in}{0.331635in}}{\pgfqpoint{9.300000in}{7.700000in}}%
\pgfusepath{clip}%
\pgfsetbuttcap%
\pgfsetroundjoin%
\definecolor{currentfill}{rgb}{0.631373,0.788235,0.956863}%
\pgfsetfillcolor{currentfill}%
\pgfsetlinewidth{0.481800pt}%
\definecolor{currentstroke}{rgb}{1.000000,1.000000,1.000000}%
\pgfsetstrokecolor{currentstroke}%
\pgfsetdash{}{0pt}%
\pgfpathmoveto{\pgfqpoint{4.645211in}{3.331282in}}%
\pgfpathcurveto{\pgfqpoint{4.656261in}{3.331282in}}{\pgfqpoint{4.666860in}{3.335672in}}{\pgfqpoint{4.674674in}{3.343485in}}%
\pgfpathcurveto{\pgfqpoint{4.682487in}{3.351299in}}{\pgfqpoint{4.686878in}{3.361898in}}{\pgfqpoint{4.686878in}{3.372948in}}%
\pgfpathcurveto{\pgfqpoint{4.686878in}{3.383998in}}{\pgfqpoint{4.682487in}{3.394597in}}{\pgfqpoint{4.674674in}{3.402411in}}%
\pgfpathcurveto{\pgfqpoint{4.666860in}{3.410225in}}{\pgfqpoint{4.656261in}{3.414615in}}{\pgfqpoint{4.645211in}{3.414615in}}%
\pgfpathcurveto{\pgfqpoint{4.634161in}{3.414615in}}{\pgfqpoint{4.623562in}{3.410225in}}{\pgfqpoint{4.615748in}{3.402411in}}%
\pgfpathcurveto{\pgfqpoint{4.607935in}{3.394597in}}{\pgfqpoint{4.603544in}{3.383998in}}{\pgfqpoint{4.603544in}{3.372948in}}%
\pgfpathcurveto{\pgfqpoint{4.603544in}{3.361898in}}{\pgfqpoint{4.607935in}{3.351299in}}{\pgfqpoint{4.615748in}{3.343485in}}%
\pgfpathcurveto{\pgfqpoint{4.623562in}{3.335672in}}{\pgfqpoint{4.634161in}{3.331282in}}{\pgfqpoint{4.645211in}{3.331282in}}%
\pgfpathclose%
\pgfusepath{stroke,fill}%
\end{pgfscope}%
\begin{pgfscope}%
\pgfpathrectangle{\pgfqpoint{0.570343in}{0.331635in}}{\pgfqpoint{9.300000in}{7.700000in}}%
\pgfusepath{clip}%
\pgfsetbuttcap%
\pgfsetroundjoin%
\definecolor{currentfill}{rgb}{0.631373,0.788235,0.956863}%
\pgfsetfillcolor{currentfill}%
\pgfsetlinewidth{0.481800pt}%
\definecolor{currentstroke}{rgb}{1.000000,1.000000,1.000000}%
\pgfsetstrokecolor{currentstroke}%
\pgfsetdash{}{0pt}%
\pgfpathmoveto{\pgfqpoint{5.785764in}{4.741451in}}%
\pgfpathcurveto{\pgfqpoint{5.796815in}{4.741451in}}{\pgfqpoint{5.807414in}{4.745842in}}{\pgfqpoint{5.815227in}{4.753655in}}%
\pgfpathcurveto{\pgfqpoint{5.823041in}{4.761469in}}{\pgfqpoint{5.827431in}{4.772068in}}{\pgfqpoint{5.827431in}{4.783118in}}%
\pgfpathcurveto{\pgfqpoint{5.827431in}{4.794168in}}{\pgfqpoint{5.823041in}{4.804767in}}{\pgfqpoint{5.815227in}{4.812581in}}%
\pgfpathcurveto{\pgfqpoint{5.807414in}{4.820394in}}{\pgfqpoint{5.796815in}{4.824785in}}{\pgfqpoint{5.785764in}{4.824785in}}%
\pgfpathcurveto{\pgfqpoint{5.774714in}{4.824785in}}{\pgfqpoint{5.764115in}{4.820394in}}{\pgfqpoint{5.756302in}{4.812581in}}%
\pgfpathcurveto{\pgfqpoint{5.748488in}{4.804767in}}{\pgfqpoint{5.744098in}{4.794168in}}{\pgfqpoint{5.744098in}{4.783118in}}%
\pgfpathcurveto{\pgfqpoint{5.744098in}{4.772068in}}{\pgfqpoint{5.748488in}{4.761469in}}{\pgfqpoint{5.756302in}{4.753655in}}%
\pgfpathcurveto{\pgfqpoint{5.764115in}{4.745842in}}{\pgfqpoint{5.774714in}{4.741451in}}{\pgfqpoint{5.785764in}{4.741451in}}%
\pgfpathclose%
\pgfusepath{stroke,fill}%
\end{pgfscope}%
\begin{pgfscope}%
\pgfpathrectangle{\pgfqpoint{0.570343in}{0.331635in}}{\pgfqpoint{9.300000in}{7.700000in}}%
\pgfusepath{clip}%
\pgfsetbuttcap%
\pgfsetroundjoin%
\definecolor{currentfill}{rgb}{0.631373,0.788235,0.956863}%
\pgfsetfillcolor{currentfill}%
\pgfsetlinewidth{0.481800pt}%
\definecolor{currentstroke}{rgb}{1.000000,1.000000,1.000000}%
\pgfsetstrokecolor{currentstroke}%
\pgfsetdash{}{0pt}%
\pgfpathmoveto{\pgfqpoint{5.100277in}{4.600045in}}%
\pgfpathcurveto{\pgfqpoint{5.111327in}{4.600045in}}{\pgfqpoint{5.121926in}{4.604435in}}{\pgfqpoint{5.129740in}{4.612249in}}%
\pgfpathcurveto{\pgfqpoint{5.137553in}{4.620062in}}{\pgfqpoint{5.141944in}{4.630661in}}{\pgfqpoint{5.141944in}{4.641711in}}%
\pgfpathcurveto{\pgfqpoint{5.141944in}{4.652761in}}{\pgfqpoint{5.137553in}{4.663360in}}{\pgfqpoint{5.129740in}{4.671174in}}%
\pgfpathcurveto{\pgfqpoint{5.121926in}{4.678988in}}{\pgfqpoint{5.111327in}{4.683378in}}{\pgfqpoint{5.100277in}{4.683378in}}%
\pgfpathcurveto{\pgfqpoint{5.089227in}{4.683378in}}{\pgfqpoint{5.078628in}{4.678988in}}{\pgfqpoint{5.070814in}{4.671174in}}%
\pgfpathcurveto{\pgfqpoint{5.063001in}{4.663360in}}{\pgfqpoint{5.058610in}{4.652761in}}{\pgfqpoint{5.058610in}{4.641711in}}%
\pgfpathcurveto{\pgfqpoint{5.058610in}{4.630661in}}{\pgfqpoint{5.063001in}{4.620062in}}{\pgfqpoint{5.070814in}{4.612249in}}%
\pgfpathcurveto{\pgfqpoint{5.078628in}{4.604435in}}{\pgfqpoint{5.089227in}{4.600045in}}{\pgfqpoint{5.100277in}{4.600045in}}%
\pgfpathclose%
\pgfusepath{stroke,fill}%
\end{pgfscope}%
\begin{pgfscope}%
\pgfpathrectangle{\pgfqpoint{0.570343in}{0.331635in}}{\pgfqpoint{9.300000in}{7.700000in}}%
\pgfusepath{clip}%
\pgfsetbuttcap%
\pgfsetroundjoin%
\definecolor{currentfill}{rgb}{0.631373,0.788235,0.956863}%
\pgfsetfillcolor{currentfill}%
\pgfsetlinewidth{0.481800pt}%
\definecolor{currentstroke}{rgb}{1.000000,1.000000,1.000000}%
\pgfsetstrokecolor{currentstroke}%
\pgfsetdash{}{0pt}%
\pgfpathmoveto{\pgfqpoint{0.993071in}{4.256195in}}%
\pgfpathcurveto{\pgfqpoint{1.004121in}{4.256195in}}{\pgfqpoint{1.014720in}{4.260585in}}{\pgfqpoint{1.022533in}{4.268399in}}%
\pgfpathcurveto{\pgfqpoint{1.030347in}{4.276213in}}{\pgfqpoint{1.034737in}{4.286812in}}{\pgfqpoint{1.034737in}{4.297862in}}%
\pgfpathcurveto{\pgfqpoint{1.034737in}{4.308912in}}{\pgfqpoint{1.030347in}{4.319511in}}{\pgfqpoint{1.022533in}{4.327325in}}%
\pgfpathcurveto{\pgfqpoint{1.014720in}{4.335138in}}{\pgfqpoint{1.004121in}{4.339528in}}{\pgfqpoint{0.993071in}{4.339528in}}%
\pgfpathcurveto{\pgfqpoint{0.982020in}{4.339528in}}{\pgfqpoint{0.971421in}{4.335138in}}{\pgfqpoint{0.963608in}{4.327325in}}%
\pgfpathcurveto{\pgfqpoint{0.955794in}{4.319511in}}{\pgfqpoint{0.951404in}{4.308912in}}{\pgfqpoint{0.951404in}{4.297862in}}%
\pgfpathcurveto{\pgfqpoint{0.951404in}{4.286812in}}{\pgfqpoint{0.955794in}{4.276213in}}{\pgfqpoint{0.963608in}{4.268399in}}%
\pgfpathcurveto{\pgfqpoint{0.971421in}{4.260585in}}{\pgfqpoint{0.982020in}{4.256195in}}{\pgfqpoint{0.993071in}{4.256195in}}%
\pgfpathclose%
\pgfusepath{stroke,fill}%
\end{pgfscope}%
\begin{pgfscope}%
\pgfpathrectangle{\pgfqpoint{0.570343in}{0.331635in}}{\pgfqpoint{9.300000in}{7.700000in}}%
\pgfusepath{clip}%
\pgfsetbuttcap%
\pgfsetroundjoin%
\definecolor{currentfill}{rgb}{0.631373,0.788235,0.956863}%
\pgfsetfillcolor{currentfill}%
\pgfsetlinewidth{0.481800pt}%
\definecolor{currentstroke}{rgb}{1.000000,1.000000,1.000000}%
\pgfsetstrokecolor{currentstroke}%
\pgfsetdash{}{0pt}%
\pgfpathmoveto{\pgfqpoint{6.176499in}{3.572166in}}%
\pgfpathcurveto{\pgfqpoint{6.187549in}{3.572166in}}{\pgfqpoint{6.198148in}{3.576557in}}{\pgfqpoint{6.205962in}{3.584370in}}%
\pgfpathcurveto{\pgfqpoint{6.213775in}{3.592184in}}{\pgfqpoint{6.218166in}{3.602783in}}{\pgfqpoint{6.218166in}{3.613833in}}%
\pgfpathcurveto{\pgfqpoint{6.218166in}{3.624883in}}{\pgfqpoint{6.213775in}{3.635482in}}{\pgfqpoint{6.205962in}{3.643296in}}%
\pgfpathcurveto{\pgfqpoint{6.198148in}{3.651109in}}{\pgfqpoint{6.187549in}{3.655500in}}{\pgfqpoint{6.176499in}{3.655500in}}%
\pgfpathcurveto{\pgfqpoint{6.165449in}{3.655500in}}{\pgfqpoint{6.154850in}{3.651109in}}{\pgfqpoint{6.147036in}{3.643296in}}%
\pgfpathcurveto{\pgfqpoint{6.139223in}{3.635482in}}{\pgfqpoint{6.134832in}{3.624883in}}{\pgfqpoint{6.134832in}{3.613833in}}%
\pgfpathcurveto{\pgfqpoint{6.134832in}{3.602783in}}{\pgfqpoint{6.139223in}{3.592184in}}{\pgfqpoint{6.147036in}{3.584370in}}%
\pgfpathcurveto{\pgfqpoint{6.154850in}{3.576557in}}{\pgfqpoint{6.165449in}{3.572166in}}{\pgfqpoint{6.176499in}{3.572166in}}%
\pgfpathclose%
\pgfusepath{stroke,fill}%
\end{pgfscope}%
\begin{pgfscope}%
\pgfpathrectangle{\pgfqpoint{0.570343in}{0.331635in}}{\pgfqpoint{9.300000in}{7.700000in}}%
\pgfusepath{clip}%
\pgfsetbuttcap%
\pgfsetroundjoin%
\definecolor{currentfill}{rgb}{0.631373,0.788235,0.956863}%
\pgfsetfillcolor{currentfill}%
\pgfsetlinewidth{0.481800pt}%
\definecolor{currentstroke}{rgb}{1.000000,1.000000,1.000000}%
\pgfsetstrokecolor{currentstroke}%
\pgfsetdash{}{0pt}%
\pgfpathmoveto{\pgfqpoint{3.773953in}{2.538414in}}%
\pgfpathcurveto{\pgfqpoint{3.785003in}{2.538414in}}{\pgfqpoint{3.795602in}{2.542804in}}{\pgfqpoint{3.803416in}{2.550618in}}%
\pgfpathcurveto{\pgfqpoint{3.811229in}{2.558431in}}{\pgfqpoint{3.815620in}{2.569030in}}{\pgfqpoint{3.815620in}{2.580081in}}%
\pgfpathcurveto{\pgfqpoint{3.815620in}{2.591131in}}{\pgfqpoint{3.811229in}{2.601730in}}{\pgfqpoint{3.803416in}{2.609543in}}%
\pgfpathcurveto{\pgfqpoint{3.795602in}{2.617357in}}{\pgfqpoint{3.785003in}{2.621747in}}{\pgfqpoint{3.773953in}{2.621747in}}%
\pgfpathcurveto{\pgfqpoint{3.762903in}{2.621747in}}{\pgfqpoint{3.752304in}{2.617357in}}{\pgfqpoint{3.744490in}{2.609543in}}%
\pgfpathcurveto{\pgfqpoint{3.736677in}{2.601730in}}{\pgfqpoint{3.732286in}{2.591131in}}{\pgfqpoint{3.732286in}{2.580081in}}%
\pgfpathcurveto{\pgfqpoint{3.732286in}{2.569030in}}{\pgfqpoint{3.736677in}{2.558431in}}{\pgfqpoint{3.744490in}{2.550618in}}%
\pgfpathcurveto{\pgfqpoint{3.752304in}{2.542804in}}{\pgfqpoint{3.762903in}{2.538414in}}{\pgfqpoint{3.773953in}{2.538414in}}%
\pgfpathclose%
\pgfusepath{stroke,fill}%
\end{pgfscope}%
\begin{pgfscope}%
\pgfpathrectangle{\pgfqpoint{0.570343in}{0.331635in}}{\pgfqpoint{9.300000in}{7.700000in}}%
\pgfusepath{clip}%
\pgfsetbuttcap%
\pgfsetroundjoin%
\definecolor{currentfill}{rgb}{0.631373,0.788235,0.956863}%
\pgfsetfillcolor{currentfill}%
\pgfsetlinewidth{0.481800pt}%
\definecolor{currentstroke}{rgb}{1.000000,1.000000,1.000000}%
\pgfsetstrokecolor{currentstroke}%
\pgfsetdash{}{0pt}%
\pgfpathmoveto{\pgfqpoint{7.797264in}{4.727564in}}%
\pgfpathcurveto{\pgfqpoint{7.808314in}{4.727564in}}{\pgfqpoint{7.818913in}{4.731954in}}{\pgfqpoint{7.826727in}{4.739768in}}%
\pgfpathcurveto{\pgfqpoint{7.834541in}{4.747581in}}{\pgfqpoint{7.838931in}{4.758180in}}{\pgfqpoint{7.838931in}{4.769230in}}%
\pgfpathcurveto{\pgfqpoint{7.838931in}{4.780280in}}{\pgfqpoint{7.834541in}{4.790880in}}{\pgfqpoint{7.826727in}{4.798693in}}%
\pgfpathcurveto{\pgfqpoint{7.818913in}{4.806507in}}{\pgfqpoint{7.808314in}{4.810897in}}{\pgfqpoint{7.797264in}{4.810897in}}%
\pgfpathcurveto{\pgfqpoint{7.786214in}{4.810897in}}{\pgfqpoint{7.775615in}{4.806507in}}{\pgfqpoint{7.767801in}{4.798693in}}%
\pgfpathcurveto{\pgfqpoint{7.759988in}{4.790880in}}{\pgfqpoint{7.755597in}{4.780280in}}{\pgfqpoint{7.755597in}{4.769230in}}%
\pgfpathcurveto{\pgfqpoint{7.755597in}{4.758180in}}{\pgfqpoint{7.759988in}{4.747581in}}{\pgfqpoint{7.767801in}{4.739768in}}%
\pgfpathcurveto{\pgfqpoint{7.775615in}{4.731954in}}{\pgfqpoint{7.786214in}{4.727564in}}{\pgfqpoint{7.797264in}{4.727564in}}%
\pgfpathclose%
\pgfusepath{stroke,fill}%
\end{pgfscope}%
\begin{pgfscope}%
\pgfpathrectangle{\pgfqpoint{0.570343in}{0.331635in}}{\pgfqpoint{9.300000in}{7.700000in}}%
\pgfusepath{clip}%
\pgfsetbuttcap%
\pgfsetroundjoin%
\definecolor{currentfill}{rgb}{0.631373,0.788235,0.956863}%
\pgfsetfillcolor{currentfill}%
\pgfsetlinewidth{0.481800pt}%
\definecolor{currentstroke}{rgb}{1.000000,1.000000,1.000000}%
\pgfsetstrokecolor{currentstroke}%
\pgfsetdash{}{0pt}%
\pgfpathmoveto{\pgfqpoint{2.799738in}{3.089874in}}%
\pgfpathcurveto{\pgfqpoint{2.810788in}{3.089874in}}{\pgfqpoint{2.821387in}{3.094264in}}{\pgfqpoint{2.829201in}{3.102078in}}%
\pgfpathcurveto{\pgfqpoint{2.837014in}{3.109892in}}{\pgfqpoint{2.841405in}{3.120491in}}{\pgfqpoint{2.841405in}{3.131541in}}%
\pgfpathcurveto{\pgfqpoint{2.841405in}{3.142591in}}{\pgfqpoint{2.837014in}{3.153190in}}{\pgfqpoint{2.829201in}{3.161004in}}%
\pgfpathcurveto{\pgfqpoint{2.821387in}{3.168817in}}{\pgfqpoint{2.810788in}{3.173207in}}{\pgfqpoint{2.799738in}{3.173207in}}%
\pgfpathcurveto{\pgfqpoint{2.788688in}{3.173207in}}{\pgfqpoint{2.778089in}{3.168817in}}{\pgfqpoint{2.770275in}{3.161004in}}%
\pgfpathcurveto{\pgfqpoint{2.762462in}{3.153190in}}{\pgfqpoint{2.758071in}{3.142591in}}{\pgfqpoint{2.758071in}{3.131541in}}%
\pgfpathcurveto{\pgfqpoint{2.758071in}{3.120491in}}{\pgfqpoint{2.762462in}{3.109892in}}{\pgfqpoint{2.770275in}{3.102078in}}%
\pgfpathcurveto{\pgfqpoint{2.778089in}{3.094264in}}{\pgfqpoint{2.788688in}{3.089874in}}{\pgfqpoint{2.799738in}{3.089874in}}%
\pgfpathclose%
\pgfusepath{stroke,fill}%
\end{pgfscope}%
\begin{pgfscope}%
\pgfpathrectangle{\pgfqpoint{0.570343in}{0.331635in}}{\pgfqpoint{9.300000in}{7.700000in}}%
\pgfusepath{clip}%
\pgfsetbuttcap%
\pgfsetroundjoin%
\definecolor{currentfill}{rgb}{0.631373,0.788235,0.956863}%
\pgfsetfillcolor{currentfill}%
\pgfsetlinewidth{0.481800pt}%
\definecolor{currentstroke}{rgb}{1.000000,1.000000,1.000000}%
\pgfsetstrokecolor{currentstroke}%
\pgfsetdash{}{0pt}%
\pgfpathmoveto{\pgfqpoint{6.185154in}{2.850809in}}%
\pgfpathcurveto{\pgfqpoint{6.196204in}{2.850809in}}{\pgfqpoint{6.206803in}{2.855199in}}{\pgfqpoint{6.214617in}{2.863013in}}%
\pgfpathcurveto{\pgfqpoint{6.222431in}{2.870827in}}{\pgfqpoint{6.226821in}{2.881426in}}{\pgfqpoint{6.226821in}{2.892476in}}%
\pgfpathcurveto{\pgfqpoint{6.226821in}{2.903526in}}{\pgfqpoint{6.222431in}{2.914125in}}{\pgfqpoint{6.214617in}{2.921939in}}%
\pgfpathcurveto{\pgfqpoint{6.206803in}{2.929752in}}{\pgfqpoint{6.196204in}{2.934143in}}{\pgfqpoint{6.185154in}{2.934143in}}%
\pgfpathcurveto{\pgfqpoint{6.174104in}{2.934143in}}{\pgfqpoint{6.163505in}{2.929752in}}{\pgfqpoint{6.155691in}{2.921939in}}%
\pgfpathcurveto{\pgfqpoint{6.147878in}{2.914125in}}{\pgfqpoint{6.143487in}{2.903526in}}{\pgfqpoint{6.143487in}{2.892476in}}%
\pgfpathcurveto{\pgfqpoint{6.143487in}{2.881426in}}{\pgfqpoint{6.147878in}{2.870827in}}{\pgfqpoint{6.155691in}{2.863013in}}%
\pgfpathcurveto{\pgfqpoint{6.163505in}{2.855199in}}{\pgfqpoint{6.174104in}{2.850809in}}{\pgfqpoint{6.185154in}{2.850809in}}%
\pgfpathclose%
\pgfusepath{stroke,fill}%
\end{pgfscope}%
\begin{pgfscope}%
\pgfpathrectangle{\pgfqpoint{0.570343in}{0.331635in}}{\pgfqpoint{9.300000in}{7.700000in}}%
\pgfusepath{clip}%
\pgfsetbuttcap%
\pgfsetroundjoin%
\definecolor{currentfill}{rgb}{0.631373,0.788235,0.956863}%
\pgfsetfillcolor{currentfill}%
\pgfsetlinewidth{0.481800pt}%
\definecolor{currentstroke}{rgb}{1.000000,1.000000,1.000000}%
\pgfsetstrokecolor{currentstroke}%
\pgfsetdash{}{0pt}%
\pgfpathmoveto{\pgfqpoint{2.095236in}{3.470631in}}%
\pgfpathcurveto{\pgfqpoint{2.106287in}{3.470631in}}{\pgfqpoint{2.116886in}{3.475022in}}{\pgfqpoint{2.124699in}{3.482835in}}%
\pgfpathcurveto{\pgfqpoint{2.132513in}{3.490649in}}{\pgfqpoint{2.136903in}{3.501248in}}{\pgfqpoint{2.136903in}{3.512298in}}%
\pgfpathcurveto{\pgfqpoint{2.136903in}{3.523348in}}{\pgfqpoint{2.132513in}{3.533947in}}{\pgfqpoint{2.124699in}{3.541761in}}%
\pgfpathcurveto{\pgfqpoint{2.116886in}{3.549574in}}{\pgfqpoint{2.106287in}{3.553965in}}{\pgfqpoint{2.095236in}{3.553965in}}%
\pgfpathcurveto{\pgfqpoint{2.084186in}{3.553965in}}{\pgfqpoint{2.073587in}{3.549574in}}{\pgfqpoint{2.065774in}{3.541761in}}%
\pgfpathcurveto{\pgfqpoint{2.057960in}{3.533947in}}{\pgfqpoint{2.053570in}{3.523348in}}{\pgfqpoint{2.053570in}{3.512298in}}%
\pgfpathcurveto{\pgfqpoint{2.053570in}{3.501248in}}{\pgfqpoint{2.057960in}{3.490649in}}{\pgfqpoint{2.065774in}{3.482835in}}%
\pgfpathcurveto{\pgfqpoint{2.073587in}{3.475022in}}{\pgfqpoint{2.084186in}{3.470631in}}{\pgfqpoint{2.095236in}{3.470631in}}%
\pgfpathclose%
\pgfusepath{stroke,fill}%
\end{pgfscope}%
\begin{pgfscope}%
\pgfpathrectangle{\pgfqpoint{0.570343in}{0.331635in}}{\pgfqpoint{9.300000in}{7.700000in}}%
\pgfusepath{clip}%
\pgfsetbuttcap%
\pgfsetroundjoin%
\definecolor{currentfill}{rgb}{0.631373,0.788235,0.956863}%
\pgfsetfillcolor{currentfill}%
\pgfsetlinewidth{0.481800pt}%
\definecolor{currentstroke}{rgb}{1.000000,1.000000,1.000000}%
\pgfsetstrokecolor{currentstroke}%
\pgfsetdash{}{0pt}%
\pgfpathmoveto{\pgfqpoint{4.706525in}{5.260494in}}%
\pgfpathcurveto{\pgfqpoint{4.717575in}{5.260494in}}{\pgfqpoint{4.728174in}{5.264885in}}{\pgfqpoint{4.735988in}{5.272698in}}%
\pgfpathcurveto{\pgfqpoint{4.743802in}{5.280512in}}{\pgfqpoint{4.748192in}{5.291111in}}{\pgfqpoint{4.748192in}{5.302161in}}%
\pgfpathcurveto{\pgfqpoint{4.748192in}{5.313211in}}{\pgfqpoint{4.743802in}{5.323810in}}{\pgfqpoint{4.735988in}{5.331624in}}%
\pgfpathcurveto{\pgfqpoint{4.728174in}{5.339437in}}{\pgfqpoint{4.717575in}{5.343828in}}{\pgfqpoint{4.706525in}{5.343828in}}%
\pgfpathcurveto{\pgfqpoint{4.695475in}{5.343828in}}{\pgfqpoint{4.684876in}{5.339437in}}{\pgfqpoint{4.677062in}{5.331624in}}%
\pgfpathcurveto{\pgfqpoint{4.669249in}{5.323810in}}{\pgfqpoint{4.664858in}{5.313211in}}{\pgfqpoint{4.664858in}{5.302161in}}%
\pgfpathcurveto{\pgfqpoint{4.664858in}{5.291111in}}{\pgfqpoint{4.669249in}{5.280512in}}{\pgfqpoint{4.677062in}{5.272698in}}%
\pgfpathcurveto{\pgfqpoint{4.684876in}{5.264885in}}{\pgfqpoint{4.695475in}{5.260494in}}{\pgfqpoint{4.706525in}{5.260494in}}%
\pgfpathclose%
\pgfusepath{stroke,fill}%
\end{pgfscope}%
\begin{pgfscope}%
\pgfpathrectangle{\pgfqpoint{0.570343in}{0.331635in}}{\pgfqpoint{9.300000in}{7.700000in}}%
\pgfusepath{clip}%
\pgfsetbuttcap%
\pgfsetroundjoin%
\definecolor{currentfill}{rgb}{0.631373,0.788235,0.956863}%
\pgfsetfillcolor{currentfill}%
\pgfsetlinewidth{0.481800pt}%
\definecolor{currentstroke}{rgb}{1.000000,1.000000,1.000000}%
\pgfsetstrokecolor{currentstroke}%
\pgfsetdash{}{0pt}%
\pgfpathmoveto{\pgfqpoint{3.943202in}{3.894706in}}%
\pgfpathcurveto{\pgfqpoint{3.954252in}{3.894706in}}{\pgfqpoint{3.964851in}{3.899097in}}{\pgfqpoint{3.972665in}{3.906910in}}%
\pgfpathcurveto{\pgfqpoint{3.980479in}{3.914724in}}{\pgfqpoint{3.984869in}{3.925323in}}{\pgfqpoint{3.984869in}{3.936373in}}%
\pgfpathcurveto{\pgfqpoint{3.984869in}{3.947423in}}{\pgfqpoint{3.980479in}{3.958022in}}{\pgfqpoint{3.972665in}{3.965836in}}%
\pgfpathcurveto{\pgfqpoint{3.964851in}{3.973649in}}{\pgfqpoint{3.954252in}{3.978040in}}{\pgfqpoint{3.943202in}{3.978040in}}%
\pgfpathcurveto{\pgfqpoint{3.932152in}{3.978040in}}{\pgfqpoint{3.921553in}{3.973649in}}{\pgfqpoint{3.913740in}{3.965836in}}%
\pgfpathcurveto{\pgfqpoint{3.905926in}{3.958022in}}{\pgfqpoint{3.901536in}{3.947423in}}{\pgfqpoint{3.901536in}{3.936373in}}%
\pgfpathcurveto{\pgfqpoint{3.901536in}{3.925323in}}{\pgfqpoint{3.905926in}{3.914724in}}{\pgfqpoint{3.913740in}{3.906910in}}%
\pgfpathcurveto{\pgfqpoint{3.921553in}{3.899097in}}{\pgfqpoint{3.932152in}{3.894706in}}{\pgfqpoint{3.943202in}{3.894706in}}%
\pgfpathclose%
\pgfusepath{stroke,fill}%
\end{pgfscope}%
\begin{pgfscope}%
\pgfpathrectangle{\pgfqpoint{0.570343in}{0.331635in}}{\pgfqpoint{9.300000in}{7.700000in}}%
\pgfusepath{clip}%
\pgfsetbuttcap%
\pgfsetroundjoin%
\definecolor{currentfill}{rgb}{0.631373,0.788235,0.956863}%
\pgfsetfillcolor{currentfill}%
\pgfsetlinewidth{0.481800pt}%
\definecolor{currentstroke}{rgb}{1.000000,1.000000,1.000000}%
\pgfsetstrokecolor{currentstroke}%
\pgfsetdash{}{0pt}%
\pgfpathmoveto{\pgfqpoint{4.635731in}{3.961694in}}%
\pgfpathcurveto{\pgfqpoint{4.646781in}{3.961694in}}{\pgfqpoint{4.657380in}{3.966084in}}{\pgfqpoint{4.665193in}{3.973897in}}%
\pgfpathcurveto{\pgfqpoint{4.673007in}{3.981711in}}{\pgfqpoint{4.677397in}{3.992310in}}{\pgfqpoint{4.677397in}{4.003360in}}%
\pgfpathcurveto{\pgfqpoint{4.677397in}{4.014410in}}{\pgfqpoint{4.673007in}{4.025009in}}{\pgfqpoint{4.665193in}{4.032823in}}%
\pgfpathcurveto{\pgfqpoint{4.657380in}{4.040637in}}{\pgfqpoint{4.646781in}{4.045027in}}{\pgfqpoint{4.635731in}{4.045027in}}%
\pgfpathcurveto{\pgfqpoint{4.624680in}{4.045027in}}{\pgfqpoint{4.614081in}{4.040637in}}{\pgfqpoint{4.606268in}{4.032823in}}%
\pgfpathcurveto{\pgfqpoint{4.598454in}{4.025009in}}{\pgfqpoint{4.594064in}{4.014410in}}{\pgfqpoint{4.594064in}{4.003360in}}%
\pgfpathcurveto{\pgfqpoint{4.594064in}{3.992310in}}{\pgfqpoint{4.598454in}{3.981711in}}{\pgfqpoint{4.606268in}{3.973897in}}%
\pgfpathcurveto{\pgfqpoint{4.614081in}{3.966084in}}{\pgfqpoint{4.624680in}{3.961694in}}{\pgfqpoint{4.635731in}{3.961694in}}%
\pgfpathclose%
\pgfusepath{stroke,fill}%
\end{pgfscope}%
\begin{pgfscope}%
\pgfpathrectangle{\pgfqpoint{0.570343in}{0.331635in}}{\pgfqpoint{9.300000in}{7.700000in}}%
\pgfusepath{clip}%
\pgfsetbuttcap%
\pgfsetroundjoin%
\definecolor{currentfill}{rgb}{0.631373,0.788235,0.956863}%
\pgfsetfillcolor{currentfill}%
\pgfsetlinewidth{0.481800pt}%
\definecolor{currentstroke}{rgb}{1.000000,1.000000,1.000000}%
\pgfsetstrokecolor{currentstroke}%
\pgfsetdash{}{0pt}%
\pgfpathmoveto{\pgfqpoint{3.827407in}{3.262886in}}%
\pgfpathcurveto{\pgfqpoint{3.838458in}{3.262886in}}{\pgfqpoint{3.849057in}{3.267277in}}{\pgfqpoint{3.856870in}{3.275090in}}%
\pgfpathcurveto{\pgfqpoint{3.864684in}{3.282904in}}{\pgfqpoint{3.869074in}{3.293503in}}{\pgfqpoint{3.869074in}{3.304553in}}%
\pgfpathcurveto{\pgfqpoint{3.869074in}{3.315603in}}{\pgfqpoint{3.864684in}{3.326202in}}{\pgfqpoint{3.856870in}{3.334016in}}%
\pgfpathcurveto{\pgfqpoint{3.849057in}{3.341829in}}{\pgfqpoint{3.838458in}{3.346220in}}{\pgfqpoint{3.827407in}{3.346220in}}%
\pgfpathcurveto{\pgfqpoint{3.816357in}{3.346220in}}{\pgfqpoint{3.805758in}{3.341829in}}{\pgfqpoint{3.797945in}{3.334016in}}%
\pgfpathcurveto{\pgfqpoint{3.790131in}{3.326202in}}{\pgfqpoint{3.785741in}{3.315603in}}{\pgfqpoint{3.785741in}{3.304553in}}%
\pgfpathcurveto{\pgfqpoint{3.785741in}{3.293503in}}{\pgfqpoint{3.790131in}{3.282904in}}{\pgfqpoint{3.797945in}{3.275090in}}%
\pgfpathcurveto{\pgfqpoint{3.805758in}{3.267277in}}{\pgfqpoint{3.816357in}{3.262886in}}{\pgfqpoint{3.827407in}{3.262886in}}%
\pgfpathclose%
\pgfusepath{stroke,fill}%
\end{pgfscope}%
\begin{pgfscope}%
\pgfpathrectangle{\pgfqpoint{0.570343in}{0.331635in}}{\pgfqpoint{9.300000in}{7.700000in}}%
\pgfusepath{clip}%
\pgfsetbuttcap%
\pgfsetroundjoin%
\definecolor{currentfill}{rgb}{0.631373,0.788235,0.956863}%
\pgfsetfillcolor{currentfill}%
\pgfsetlinewidth{0.481800pt}%
\definecolor{currentstroke}{rgb}{1.000000,1.000000,1.000000}%
\pgfsetstrokecolor{currentstroke}%
\pgfsetdash{}{0pt}%
\pgfpathmoveto{\pgfqpoint{3.605912in}{4.449800in}}%
\pgfpathcurveto{\pgfqpoint{3.616962in}{4.449800in}}{\pgfqpoint{3.627562in}{4.454190in}}{\pgfqpoint{3.635375in}{4.462004in}}%
\pgfpathcurveto{\pgfqpoint{3.643189in}{4.469818in}}{\pgfqpoint{3.647579in}{4.480417in}}{\pgfqpoint{3.647579in}{4.491467in}}%
\pgfpathcurveto{\pgfqpoint{3.647579in}{4.502517in}}{\pgfqpoint{3.643189in}{4.513116in}}{\pgfqpoint{3.635375in}{4.520930in}}%
\pgfpathcurveto{\pgfqpoint{3.627562in}{4.528743in}}{\pgfqpoint{3.616962in}{4.533133in}}{\pgfqpoint{3.605912in}{4.533133in}}%
\pgfpathcurveto{\pgfqpoint{3.594862in}{4.533133in}}{\pgfqpoint{3.584263in}{4.528743in}}{\pgfqpoint{3.576450in}{4.520930in}}%
\pgfpathcurveto{\pgfqpoint{3.568636in}{4.513116in}}{\pgfqpoint{3.564246in}{4.502517in}}{\pgfqpoint{3.564246in}{4.491467in}}%
\pgfpathcurveto{\pgfqpoint{3.564246in}{4.480417in}}{\pgfqpoint{3.568636in}{4.469818in}}{\pgfqpoint{3.576450in}{4.462004in}}%
\pgfpathcurveto{\pgfqpoint{3.584263in}{4.454190in}}{\pgfqpoint{3.594862in}{4.449800in}}{\pgfqpoint{3.605912in}{4.449800in}}%
\pgfpathclose%
\pgfusepath{stroke,fill}%
\end{pgfscope}%
\begin{pgfscope}%
\pgfpathrectangle{\pgfqpoint{0.570343in}{0.331635in}}{\pgfqpoint{9.300000in}{7.700000in}}%
\pgfusepath{clip}%
\pgfsetbuttcap%
\pgfsetroundjoin%
\definecolor{currentfill}{rgb}{0.631373,0.788235,0.956863}%
\pgfsetfillcolor{currentfill}%
\pgfsetlinewidth{0.481800pt}%
\definecolor{currentstroke}{rgb}{1.000000,1.000000,1.000000}%
\pgfsetstrokecolor{currentstroke}%
\pgfsetdash{}{0pt}%
\pgfpathmoveto{\pgfqpoint{5.549912in}{4.145442in}}%
\pgfpathcurveto{\pgfqpoint{5.560962in}{4.145442in}}{\pgfqpoint{5.571561in}{4.149832in}}{\pgfqpoint{5.579374in}{4.157646in}}%
\pgfpathcurveto{\pgfqpoint{5.587188in}{4.165459in}}{\pgfqpoint{5.591578in}{4.176058in}}{\pgfqpoint{5.591578in}{4.187109in}}%
\pgfpathcurveto{\pgfqpoint{5.591578in}{4.198159in}}{\pgfqpoint{5.587188in}{4.208758in}}{\pgfqpoint{5.579374in}{4.216571in}}%
\pgfpathcurveto{\pgfqpoint{5.571561in}{4.224385in}}{\pgfqpoint{5.560962in}{4.228775in}}{\pgfqpoint{5.549912in}{4.228775in}}%
\pgfpathcurveto{\pgfqpoint{5.538862in}{4.228775in}}{\pgfqpoint{5.528262in}{4.224385in}}{\pgfqpoint{5.520449in}{4.216571in}}%
\pgfpathcurveto{\pgfqpoint{5.512635in}{4.208758in}}{\pgfqpoint{5.508245in}{4.198159in}}{\pgfqpoint{5.508245in}{4.187109in}}%
\pgfpathcurveto{\pgfqpoint{5.508245in}{4.176058in}}{\pgfqpoint{5.512635in}{4.165459in}}{\pgfqpoint{5.520449in}{4.157646in}}%
\pgfpathcurveto{\pgfqpoint{5.528262in}{4.149832in}}{\pgfqpoint{5.538862in}{4.145442in}}{\pgfqpoint{5.549912in}{4.145442in}}%
\pgfpathclose%
\pgfusepath{stroke,fill}%
\end{pgfscope}%
\begin{pgfscope}%
\pgfpathrectangle{\pgfqpoint{0.570343in}{0.331635in}}{\pgfqpoint{9.300000in}{7.700000in}}%
\pgfusepath{clip}%
\pgfsetbuttcap%
\pgfsetroundjoin%
\definecolor{currentfill}{rgb}{0.631373,0.788235,0.956863}%
\pgfsetfillcolor{currentfill}%
\pgfsetlinewidth{0.481800pt}%
\definecolor{currentstroke}{rgb}{1.000000,1.000000,1.000000}%
\pgfsetstrokecolor{currentstroke}%
\pgfsetdash{}{0pt}%
\pgfpathmoveto{\pgfqpoint{6.429778in}{5.520341in}}%
\pgfpathcurveto{\pgfqpoint{6.440828in}{5.520341in}}{\pgfqpoint{6.451427in}{5.524732in}}{\pgfqpoint{6.459241in}{5.532545in}}%
\pgfpathcurveto{\pgfqpoint{6.467054in}{5.540359in}}{\pgfqpoint{6.471445in}{5.550958in}}{\pgfqpoint{6.471445in}{5.562008in}}%
\pgfpathcurveto{\pgfqpoint{6.471445in}{5.573058in}}{\pgfqpoint{6.467054in}{5.583657in}}{\pgfqpoint{6.459241in}{5.591471in}}%
\pgfpathcurveto{\pgfqpoint{6.451427in}{5.599284in}}{\pgfqpoint{6.440828in}{5.603675in}}{\pgfqpoint{6.429778in}{5.603675in}}%
\pgfpathcurveto{\pgfqpoint{6.418728in}{5.603675in}}{\pgfqpoint{6.408129in}{5.599284in}}{\pgfqpoint{6.400315in}{5.591471in}}%
\pgfpathcurveto{\pgfqpoint{6.392502in}{5.583657in}}{\pgfqpoint{6.388111in}{5.573058in}}{\pgfqpoint{6.388111in}{5.562008in}}%
\pgfpathcurveto{\pgfqpoint{6.388111in}{5.550958in}}{\pgfqpoint{6.392502in}{5.540359in}}{\pgfqpoint{6.400315in}{5.532545in}}%
\pgfpathcurveto{\pgfqpoint{6.408129in}{5.524732in}}{\pgfqpoint{6.418728in}{5.520341in}}{\pgfqpoint{6.429778in}{5.520341in}}%
\pgfpathclose%
\pgfusepath{stroke,fill}%
\end{pgfscope}%
\begin{pgfscope}%
\pgfpathrectangle{\pgfqpoint{0.570343in}{0.331635in}}{\pgfqpoint{9.300000in}{7.700000in}}%
\pgfusepath{clip}%
\pgfsetbuttcap%
\pgfsetroundjoin%
\definecolor{currentfill}{rgb}{0.631373,0.788235,0.956863}%
\pgfsetfillcolor{currentfill}%
\pgfsetlinewidth{0.481800pt}%
\definecolor{currentstroke}{rgb}{1.000000,1.000000,1.000000}%
\pgfsetstrokecolor{currentstroke}%
\pgfsetdash{}{0pt}%
\pgfpathmoveto{\pgfqpoint{4.510732in}{2.688622in}}%
\pgfpathcurveto{\pgfqpoint{4.521782in}{2.688622in}}{\pgfqpoint{4.532381in}{2.693012in}}{\pgfqpoint{4.540195in}{2.700826in}}%
\pgfpathcurveto{\pgfqpoint{4.548008in}{2.708640in}}{\pgfqpoint{4.552399in}{2.719239in}}{\pgfqpoint{4.552399in}{2.730289in}}%
\pgfpathcurveto{\pgfqpoint{4.552399in}{2.741339in}}{\pgfqpoint{4.548008in}{2.751938in}}{\pgfqpoint{4.540195in}{2.759752in}}%
\pgfpathcurveto{\pgfqpoint{4.532381in}{2.767565in}}{\pgfqpoint{4.521782in}{2.771956in}}{\pgfqpoint{4.510732in}{2.771956in}}%
\pgfpathcurveto{\pgfqpoint{4.499682in}{2.771956in}}{\pgfqpoint{4.489083in}{2.767565in}}{\pgfqpoint{4.481269in}{2.759752in}}%
\pgfpathcurveto{\pgfqpoint{4.473456in}{2.751938in}}{\pgfqpoint{4.469065in}{2.741339in}}{\pgfqpoint{4.469065in}{2.730289in}}%
\pgfpathcurveto{\pgfqpoint{4.469065in}{2.719239in}}{\pgfqpoint{4.473456in}{2.708640in}}{\pgfqpoint{4.481269in}{2.700826in}}%
\pgfpathcurveto{\pgfqpoint{4.489083in}{2.693012in}}{\pgfqpoint{4.499682in}{2.688622in}}{\pgfqpoint{4.510732in}{2.688622in}}%
\pgfpathclose%
\pgfusepath{stroke,fill}%
\end{pgfscope}%
\begin{pgfscope}%
\pgfpathrectangle{\pgfqpoint{0.570343in}{0.331635in}}{\pgfqpoint{9.300000in}{7.700000in}}%
\pgfusepath{clip}%
\pgfsetbuttcap%
\pgfsetroundjoin%
\definecolor{currentfill}{rgb}{0.631373,0.788235,0.956863}%
\pgfsetfillcolor{currentfill}%
\pgfsetlinewidth{0.481800pt}%
\definecolor{currentstroke}{rgb}{1.000000,1.000000,1.000000}%
\pgfsetstrokecolor{currentstroke}%
\pgfsetdash{}{0pt}%
\pgfpathmoveto{\pgfqpoint{5.386210in}{3.505671in}}%
\pgfpathcurveto{\pgfqpoint{5.397260in}{3.505671in}}{\pgfqpoint{5.407859in}{3.510061in}}{\pgfqpoint{5.415673in}{3.517875in}}%
\pgfpathcurveto{\pgfqpoint{5.423486in}{3.525688in}}{\pgfqpoint{5.427877in}{3.536287in}}{\pgfqpoint{5.427877in}{3.547338in}}%
\pgfpathcurveto{\pgfqpoint{5.427877in}{3.558388in}}{\pgfqpoint{5.423486in}{3.568987in}}{\pgfqpoint{5.415673in}{3.576800in}}%
\pgfpathcurveto{\pgfqpoint{5.407859in}{3.584614in}}{\pgfqpoint{5.397260in}{3.589004in}}{\pgfqpoint{5.386210in}{3.589004in}}%
\pgfpathcurveto{\pgfqpoint{5.375160in}{3.589004in}}{\pgfqpoint{5.364561in}{3.584614in}}{\pgfqpoint{5.356747in}{3.576800in}}%
\pgfpathcurveto{\pgfqpoint{5.348934in}{3.568987in}}{\pgfqpoint{5.344543in}{3.558388in}}{\pgfqpoint{5.344543in}{3.547338in}}%
\pgfpathcurveto{\pgfqpoint{5.344543in}{3.536287in}}{\pgfqpoint{5.348934in}{3.525688in}}{\pgfqpoint{5.356747in}{3.517875in}}%
\pgfpathcurveto{\pgfqpoint{5.364561in}{3.510061in}}{\pgfqpoint{5.375160in}{3.505671in}}{\pgfqpoint{5.386210in}{3.505671in}}%
\pgfpathclose%
\pgfusepath{stroke,fill}%
\end{pgfscope}%
\begin{pgfscope}%
\pgfpathrectangle{\pgfqpoint{0.570343in}{0.331635in}}{\pgfqpoint{9.300000in}{7.700000in}}%
\pgfusepath{clip}%
\pgfsetbuttcap%
\pgfsetroundjoin%
\definecolor{currentfill}{rgb}{0.631373,0.788235,0.956863}%
\pgfsetfillcolor{currentfill}%
\pgfsetlinewidth{0.481800pt}%
\definecolor{currentstroke}{rgb}{1.000000,1.000000,1.000000}%
\pgfsetstrokecolor{currentstroke}%
\pgfsetdash{}{0pt}%
\pgfpathmoveto{\pgfqpoint{8.348041in}{5.455977in}}%
\pgfpathcurveto{\pgfqpoint{8.359091in}{5.455977in}}{\pgfqpoint{8.369690in}{5.460367in}}{\pgfqpoint{8.377503in}{5.468181in}}%
\pgfpathcurveto{\pgfqpoint{8.385317in}{5.475994in}}{\pgfqpoint{8.389707in}{5.486593in}}{\pgfqpoint{8.389707in}{5.497643in}}%
\pgfpathcurveto{\pgfqpoint{8.389707in}{5.508694in}}{\pgfqpoint{8.385317in}{5.519293in}}{\pgfqpoint{8.377503in}{5.527106in}}%
\pgfpathcurveto{\pgfqpoint{8.369690in}{5.534920in}}{\pgfqpoint{8.359091in}{5.539310in}}{\pgfqpoint{8.348041in}{5.539310in}}%
\pgfpathcurveto{\pgfqpoint{8.336991in}{5.539310in}}{\pgfqpoint{8.326391in}{5.534920in}}{\pgfqpoint{8.318578in}{5.527106in}}%
\pgfpathcurveto{\pgfqpoint{8.310764in}{5.519293in}}{\pgfqpoint{8.306374in}{5.508694in}}{\pgfqpoint{8.306374in}{5.497643in}}%
\pgfpathcurveto{\pgfqpoint{8.306374in}{5.486593in}}{\pgfqpoint{8.310764in}{5.475994in}}{\pgfqpoint{8.318578in}{5.468181in}}%
\pgfpathcurveto{\pgfqpoint{8.326391in}{5.460367in}}{\pgfqpoint{8.336991in}{5.455977in}}{\pgfqpoint{8.348041in}{5.455977in}}%
\pgfpathclose%
\pgfusepath{stroke,fill}%
\end{pgfscope}%
\begin{pgfscope}%
\pgfpathrectangle{\pgfqpoint{0.570343in}{0.331635in}}{\pgfqpoint{9.300000in}{7.700000in}}%
\pgfusepath{clip}%
\pgfsetbuttcap%
\pgfsetroundjoin%
\definecolor{currentfill}{rgb}{0.631373,0.788235,0.956863}%
\pgfsetfillcolor{currentfill}%
\pgfsetlinewidth{0.481800pt}%
\definecolor{currentstroke}{rgb}{1.000000,1.000000,1.000000}%
\pgfsetstrokecolor{currentstroke}%
\pgfsetdash{}{0pt}%
\pgfpathmoveto{\pgfqpoint{5.507412in}{5.364816in}}%
\pgfpathcurveto{\pgfqpoint{5.518463in}{5.364816in}}{\pgfqpoint{5.529062in}{5.369206in}}{\pgfqpoint{5.536875in}{5.377020in}}%
\pgfpathcurveto{\pgfqpoint{5.544689in}{5.384833in}}{\pgfqpoint{5.549079in}{5.395432in}}{\pgfqpoint{5.549079in}{5.406482in}}%
\pgfpathcurveto{\pgfqpoint{5.549079in}{5.417533in}}{\pgfqpoint{5.544689in}{5.428132in}}{\pgfqpoint{5.536875in}{5.435945in}}%
\pgfpathcurveto{\pgfqpoint{5.529062in}{5.443759in}}{\pgfqpoint{5.518463in}{5.448149in}}{\pgfqpoint{5.507412in}{5.448149in}}%
\pgfpathcurveto{\pgfqpoint{5.496362in}{5.448149in}}{\pgfqpoint{5.485763in}{5.443759in}}{\pgfqpoint{5.477950in}{5.435945in}}%
\pgfpathcurveto{\pgfqpoint{5.470136in}{5.428132in}}{\pgfqpoint{5.465746in}{5.417533in}}{\pgfqpoint{5.465746in}{5.406482in}}%
\pgfpathcurveto{\pgfqpoint{5.465746in}{5.395432in}}{\pgfqpoint{5.470136in}{5.384833in}}{\pgfqpoint{5.477950in}{5.377020in}}%
\pgfpathcurveto{\pgfqpoint{5.485763in}{5.369206in}}{\pgfqpoint{5.496362in}{5.364816in}}{\pgfqpoint{5.507412in}{5.364816in}}%
\pgfpathclose%
\pgfusepath{stroke,fill}%
\end{pgfscope}%
\begin{pgfscope}%
\pgfpathrectangle{\pgfqpoint{0.570343in}{0.331635in}}{\pgfqpoint{9.300000in}{7.700000in}}%
\pgfusepath{clip}%
\pgfsetbuttcap%
\pgfsetroundjoin%
\definecolor{currentfill}{rgb}{0.631373,0.788235,0.956863}%
\pgfsetfillcolor{currentfill}%
\pgfsetlinewidth{0.481800pt}%
\definecolor{currentstroke}{rgb}{1.000000,1.000000,1.000000}%
\pgfsetstrokecolor{currentstroke}%
\pgfsetdash{}{0pt}%
\pgfpathmoveto{\pgfqpoint{5.886021in}{6.235301in}}%
\pgfpathcurveto{\pgfqpoint{5.897071in}{6.235301in}}{\pgfqpoint{5.907670in}{6.239691in}}{\pgfqpoint{5.915484in}{6.247505in}}%
\pgfpathcurveto{\pgfqpoint{5.923297in}{6.255319in}}{\pgfqpoint{5.927688in}{6.265918in}}{\pgfqpoint{5.927688in}{6.276968in}}%
\pgfpathcurveto{\pgfqpoint{5.927688in}{6.288018in}}{\pgfqpoint{5.923297in}{6.298617in}}{\pgfqpoint{5.915484in}{6.306430in}}%
\pgfpathcurveto{\pgfqpoint{5.907670in}{6.314244in}}{\pgfqpoint{5.897071in}{6.318634in}}{\pgfqpoint{5.886021in}{6.318634in}}%
\pgfpathcurveto{\pgfqpoint{5.874971in}{6.318634in}}{\pgfqpoint{5.864372in}{6.314244in}}{\pgfqpoint{5.856558in}{6.306430in}}%
\pgfpathcurveto{\pgfqpoint{5.848744in}{6.298617in}}{\pgfqpoint{5.844354in}{6.288018in}}{\pgfqpoint{5.844354in}{6.276968in}}%
\pgfpathcurveto{\pgfqpoint{5.844354in}{6.265918in}}{\pgfqpoint{5.848744in}{6.255319in}}{\pgfqpoint{5.856558in}{6.247505in}}%
\pgfpathcurveto{\pgfqpoint{5.864372in}{6.239691in}}{\pgfqpoint{5.874971in}{6.235301in}}{\pgfqpoint{5.886021in}{6.235301in}}%
\pgfpathclose%
\pgfusepath{stroke,fill}%
\end{pgfscope}%
\begin{pgfscope}%
\pgfpathrectangle{\pgfqpoint{0.570343in}{0.331635in}}{\pgfqpoint{9.300000in}{7.700000in}}%
\pgfusepath{clip}%
\pgfsetbuttcap%
\pgfsetroundjoin%
\definecolor{currentfill}{rgb}{0.631373,0.788235,0.956863}%
\pgfsetfillcolor{currentfill}%
\pgfsetlinewidth{0.481800pt}%
\definecolor{currentstroke}{rgb}{1.000000,1.000000,1.000000}%
\pgfsetstrokecolor{currentstroke}%
\pgfsetdash{}{0pt}%
\pgfpathmoveto{\pgfqpoint{6.703980in}{2.093785in}}%
\pgfpathcurveto{\pgfqpoint{6.715031in}{2.093785in}}{\pgfqpoint{6.725630in}{2.098175in}}{\pgfqpoint{6.733443in}{2.105989in}}%
\pgfpathcurveto{\pgfqpoint{6.741257in}{2.113803in}}{\pgfqpoint{6.745647in}{2.124402in}}{\pgfqpoint{6.745647in}{2.135452in}}%
\pgfpathcurveto{\pgfqpoint{6.745647in}{2.146502in}}{\pgfqpoint{6.741257in}{2.157101in}}{\pgfqpoint{6.733443in}{2.164915in}}%
\pgfpathcurveto{\pgfqpoint{6.725630in}{2.172728in}}{\pgfqpoint{6.715031in}{2.177119in}}{\pgfqpoint{6.703980in}{2.177119in}}%
\pgfpathcurveto{\pgfqpoint{6.692930in}{2.177119in}}{\pgfqpoint{6.682331in}{2.172728in}}{\pgfqpoint{6.674518in}{2.164915in}}%
\pgfpathcurveto{\pgfqpoint{6.666704in}{2.157101in}}{\pgfqpoint{6.662314in}{2.146502in}}{\pgfqpoint{6.662314in}{2.135452in}}%
\pgfpathcurveto{\pgfqpoint{6.662314in}{2.124402in}}{\pgfqpoint{6.666704in}{2.113803in}}{\pgfqpoint{6.674518in}{2.105989in}}%
\pgfpathcurveto{\pgfqpoint{6.682331in}{2.098175in}}{\pgfqpoint{6.692930in}{2.093785in}}{\pgfqpoint{6.703980in}{2.093785in}}%
\pgfpathclose%
\pgfusepath{stroke,fill}%
\end{pgfscope}%
\begin{pgfscope}%
\pgfpathrectangle{\pgfqpoint{0.570343in}{0.331635in}}{\pgfqpoint{9.300000in}{7.700000in}}%
\pgfusepath{clip}%
\pgfsetbuttcap%
\pgfsetroundjoin%
\definecolor{currentfill}{rgb}{0.631373,0.788235,0.956863}%
\pgfsetfillcolor{currentfill}%
\pgfsetlinewidth{0.481800pt}%
\definecolor{currentstroke}{rgb}{1.000000,1.000000,1.000000}%
\pgfsetstrokecolor{currentstroke}%
\pgfsetdash{}{0pt}%
\pgfpathmoveto{\pgfqpoint{3.152829in}{3.769352in}}%
\pgfpathcurveto{\pgfqpoint{3.163879in}{3.769352in}}{\pgfqpoint{3.174478in}{3.773742in}}{\pgfqpoint{3.182292in}{3.781556in}}%
\pgfpathcurveto{\pgfqpoint{3.190105in}{3.789369in}}{\pgfqpoint{3.194496in}{3.799968in}}{\pgfqpoint{3.194496in}{3.811018in}}%
\pgfpathcurveto{\pgfqpoint{3.194496in}{3.822069in}}{\pgfqpoint{3.190105in}{3.832668in}}{\pgfqpoint{3.182292in}{3.840481in}}%
\pgfpathcurveto{\pgfqpoint{3.174478in}{3.848295in}}{\pgfqpoint{3.163879in}{3.852685in}}{\pgfqpoint{3.152829in}{3.852685in}}%
\pgfpathcurveto{\pgfqpoint{3.141779in}{3.852685in}}{\pgfqpoint{3.131180in}{3.848295in}}{\pgfqpoint{3.123366in}{3.840481in}}%
\pgfpathcurveto{\pgfqpoint{3.115552in}{3.832668in}}{\pgfqpoint{3.111162in}{3.822069in}}{\pgfqpoint{3.111162in}{3.811018in}}%
\pgfpathcurveto{\pgfqpoint{3.111162in}{3.799968in}}{\pgfqpoint{3.115552in}{3.789369in}}{\pgfqpoint{3.123366in}{3.781556in}}%
\pgfpathcurveto{\pgfqpoint{3.131180in}{3.773742in}}{\pgfqpoint{3.141779in}{3.769352in}}{\pgfqpoint{3.152829in}{3.769352in}}%
\pgfpathclose%
\pgfusepath{stroke,fill}%
\end{pgfscope}%
\begin{pgfscope}%
\pgfpathrectangle{\pgfqpoint{0.570343in}{0.331635in}}{\pgfqpoint{9.300000in}{7.700000in}}%
\pgfusepath{clip}%
\pgfsetbuttcap%
\pgfsetroundjoin%
\definecolor{currentfill}{rgb}{0.631373,0.788235,0.956863}%
\pgfsetfillcolor{currentfill}%
\pgfsetlinewidth{0.481800pt}%
\definecolor{currentstroke}{rgb}{1.000000,1.000000,1.000000}%
\pgfsetstrokecolor{currentstroke}%
\pgfsetdash{}{0pt}%
\pgfpathmoveto{\pgfqpoint{4.382416in}{4.593957in}}%
\pgfpathcurveto{\pgfqpoint{4.393466in}{4.593957in}}{\pgfqpoint{4.404065in}{4.598348in}}{\pgfqpoint{4.411878in}{4.606161in}}%
\pgfpathcurveto{\pgfqpoint{4.419692in}{4.613975in}}{\pgfqpoint{4.424082in}{4.624574in}}{\pgfqpoint{4.424082in}{4.635624in}}%
\pgfpathcurveto{\pgfqpoint{4.424082in}{4.646674in}}{\pgfqpoint{4.419692in}{4.657273in}}{\pgfqpoint{4.411878in}{4.665087in}}%
\pgfpathcurveto{\pgfqpoint{4.404065in}{4.672900in}}{\pgfqpoint{4.393466in}{4.677291in}}{\pgfqpoint{4.382416in}{4.677291in}}%
\pgfpathcurveto{\pgfqpoint{4.371365in}{4.677291in}}{\pgfqpoint{4.360766in}{4.672900in}}{\pgfqpoint{4.352953in}{4.665087in}}%
\pgfpathcurveto{\pgfqpoint{4.345139in}{4.657273in}}{\pgfqpoint{4.340749in}{4.646674in}}{\pgfqpoint{4.340749in}{4.635624in}}%
\pgfpathcurveto{\pgfqpoint{4.340749in}{4.624574in}}{\pgfqpoint{4.345139in}{4.613975in}}{\pgfqpoint{4.352953in}{4.606161in}}%
\pgfpathcurveto{\pgfqpoint{4.360766in}{4.598348in}}{\pgfqpoint{4.371365in}{4.593957in}}{\pgfqpoint{4.382416in}{4.593957in}}%
\pgfpathclose%
\pgfusepath{stroke,fill}%
\end{pgfscope}%
\begin{pgfscope}%
\pgfpathrectangle{\pgfqpoint{0.570343in}{0.331635in}}{\pgfqpoint{9.300000in}{7.700000in}}%
\pgfusepath{clip}%
\pgfsetbuttcap%
\pgfsetroundjoin%
\definecolor{currentfill}{rgb}{0.631373,0.788235,0.956863}%
\pgfsetfillcolor{currentfill}%
\pgfsetlinewidth{0.481800pt}%
\definecolor{currentstroke}{rgb}{1.000000,1.000000,1.000000}%
\pgfsetstrokecolor{currentstroke}%
\pgfsetdash{}{0pt}%
\pgfpathmoveto{\pgfqpoint{6.950477in}{6.262727in}}%
\pgfpathcurveto{\pgfqpoint{6.961527in}{6.262727in}}{\pgfqpoint{6.972126in}{6.267117in}}{\pgfqpoint{6.979940in}{6.274931in}}%
\pgfpathcurveto{\pgfqpoint{6.987753in}{6.282745in}}{\pgfqpoint{6.992144in}{6.293344in}}{\pgfqpoint{6.992144in}{6.304394in}}%
\pgfpathcurveto{\pgfqpoint{6.992144in}{6.315444in}}{\pgfqpoint{6.987753in}{6.326043in}}{\pgfqpoint{6.979940in}{6.333857in}}%
\pgfpathcurveto{\pgfqpoint{6.972126in}{6.341670in}}{\pgfqpoint{6.961527in}{6.346060in}}{\pgfqpoint{6.950477in}{6.346060in}}%
\pgfpathcurveto{\pgfqpoint{6.939427in}{6.346060in}}{\pgfqpoint{6.928828in}{6.341670in}}{\pgfqpoint{6.921014in}{6.333857in}}%
\pgfpathcurveto{\pgfqpoint{6.913201in}{6.326043in}}{\pgfqpoint{6.908810in}{6.315444in}}{\pgfqpoint{6.908810in}{6.304394in}}%
\pgfpathcurveto{\pgfqpoint{6.908810in}{6.293344in}}{\pgfqpoint{6.913201in}{6.282745in}}{\pgfqpoint{6.921014in}{6.274931in}}%
\pgfpathcurveto{\pgfqpoint{6.928828in}{6.267117in}}{\pgfqpoint{6.939427in}{6.262727in}}{\pgfqpoint{6.950477in}{6.262727in}}%
\pgfpathclose%
\pgfusepath{stroke,fill}%
\end{pgfscope}%
\begin{pgfscope}%
\pgfpathrectangle{\pgfqpoint{0.570343in}{0.331635in}}{\pgfqpoint{9.300000in}{7.700000in}}%
\pgfusepath{clip}%
\pgfsetbuttcap%
\pgfsetroundjoin%
\definecolor{currentfill}{rgb}{0.631373,0.788235,0.956863}%
\pgfsetfillcolor{currentfill}%
\pgfsetlinewidth{0.481800pt}%
\definecolor{currentstroke}{rgb}{1.000000,1.000000,1.000000}%
\pgfsetstrokecolor{currentstroke}%
\pgfsetdash{}{0pt}%
\pgfpathmoveto{\pgfqpoint{2.893408in}{4.607151in}}%
\pgfpathcurveto{\pgfqpoint{2.904458in}{4.607151in}}{\pgfqpoint{2.915057in}{4.611542in}}{\pgfqpoint{2.922871in}{4.619355in}}%
\pgfpathcurveto{\pgfqpoint{2.930685in}{4.627169in}}{\pgfqpoint{2.935075in}{4.637768in}}{\pgfqpoint{2.935075in}{4.648818in}}%
\pgfpathcurveto{\pgfqpoint{2.935075in}{4.659868in}}{\pgfqpoint{2.930685in}{4.670467in}}{\pgfqpoint{2.922871in}{4.678281in}}%
\pgfpathcurveto{\pgfqpoint{2.915057in}{4.686094in}}{\pgfqpoint{2.904458in}{4.690485in}}{\pgfqpoint{2.893408in}{4.690485in}}%
\pgfpathcurveto{\pgfqpoint{2.882358in}{4.690485in}}{\pgfqpoint{2.871759in}{4.686094in}}{\pgfqpoint{2.863945in}{4.678281in}}%
\pgfpathcurveto{\pgfqpoint{2.856132in}{4.670467in}}{\pgfqpoint{2.851741in}{4.659868in}}{\pgfqpoint{2.851741in}{4.648818in}}%
\pgfpathcurveto{\pgfqpoint{2.851741in}{4.637768in}}{\pgfqpoint{2.856132in}{4.627169in}}{\pgfqpoint{2.863945in}{4.619355in}}%
\pgfpathcurveto{\pgfqpoint{2.871759in}{4.611542in}}{\pgfqpoint{2.882358in}{4.607151in}}{\pgfqpoint{2.893408in}{4.607151in}}%
\pgfpathclose%
\pgfusepath{stroke,fill}%
\end{pgfscope}%
\begin{pgfscope}%
\pgfpathrectangle{\pgfqpoint{0.570343in}{0.331635in}}{\pgfqpoint{9.300000in}{7.700000in}}%
\pgfusepath{clip}%
\pgfsetbuttcap%
\pgfsetroundjoin%
\definecolor{currentfill}{rgb}{1.000000,0.705882,0.509804}%
\pgfsetfillcolor{currentfill}%
\pgfsetlinewidth{0.481800pt}%
\definecolor{currentstroke}{rgb}{1.000000,1.000000,1.000000}%
\pgfsetstrokecolor{currentstroke}%
\pgfsetdash{}{0pt}%
\pgfpathmoveto{\pgfqpoint{3.817222in}{5.982446in}}%
\pgfpathcurveto{\pgfqpoint{3.828272in}{5.982446in}}{\pgfqpoint{3.838871in}{5.986836in}}{\pgfqpoint{3.846685in}{5.994650in}}%
\pgfpathcurveto{\pgfqpoint{3.854498in}{6.002464in}}{\pgfqpoint{3.858889in}{6.013063in}}{\pgfqpoint{3.858889in}{6.024113in}}%
\pgfpathcurveto{\pgfqpoint{3.858889in}{6.035163in}}{\pgfqpoint{3.854498in}{6.045762in}}{\pgfqpoint{3.846685in}{6.053576in}}%
\pgfpathcurveto{\pgfqpoint{3.838871in}{6.061389in}}{\pgfqpoint{3.828272in}{6.065779in}}{\pgfqpoint{3.817222in}{6.065779in}}%
\pgfpathcurveto{\pgfqpoint{3.806172in}{6.065779in}}{\pgfqpoint{3.795573in}{6.061389in}}{\pgfqpoint{3.787759in}{6.053576in}}%
\pgfpathcurveto{\pgfqpoint{3.779946in}{6.045762in}}{\pgfqpoint{3.775555in}{6.035163in}}{\pgfqpoint{3.775555in}{6.024113in}}%
\pgfpathcurveto{\pgfqpoint{3.775555in}{6.013063in}}{\pgfqpoint{3.779946in}{6.002464in}}{\pgfqpoint{3.787759in}{5.994650in}}%
\pgfpathcurveto{\pgfqpoint{3.795573in}{5.986836in}}{\pgfqpoint{3.806172in}{5.982446in}}{\pgfqpoint{3.817222in}{5.982446in}}%
\pgfpathclose%
\pgfusepath{stroke,fill}%
\end{pgfscope}%
\begin{pgfscope}%
\pgfpathrectangle{\pgfqpoint{0.570343in}{0.331635in}}{\pgfqpoint{9.300000in}{7.700000in}}%
\pgfusepath{clip}%
\pgfsetbuttcap%
\pgfsetroundjoin%
\definecolor{currentfill}{rgb}{1.000000,0.705882,0.509804}%
\pgfsetfillcolor{currentfill}%
\pgfsetlinewidth{0.481800pt}%
\definecolor{currentstroke}{rgb}{1.000000,1.000000,1.000000}%
\pgfsetstrokecolor{currentstroke}%
\pgfsetdash{}{0pt}%
\pgfpathmoveto{\pgfqpoint{7.659561in}{1.344530in}}%
\pgfpathcurveto{\pgfqpoint{7.670611in}{1.344530in}}{\pgfqpoint{7.681210in}{1.348920in}}{\pgfqpoint{7.689024in}{1.356734in}}%
\pgfpathcurveto{\pgfqpoint{7.696838in}{1.364548in}}{\pgfqpoint{7.701228in}{1.375147in}}{\pgfqpoint{7.701228in}{1.386197in}}%
\pgfpathcurveto{\pgfqpoint{7.701228in}{1.397247in}}{\pgfqpoint{7.696838in}{1.407846in}}{\pgfqpoint{7.689024in}{1.415660in}}%
\pgfpathcurveto{\pgfqpoint{7.681210in}{1.423473in}}{\pgfqpoint{7.670611in}{1.427864in}}{\pgfqpoint{7.659561in}{1.427864in}}%
\pgfpathcurveto{\pgfqpoint{7.648511in}{1.427864in}}{\pgfqpoint{7.637912in}{1.423473in}}{\pgfqpoint{7.630098in}{1.415660in}}%
\pgfpathcurveto{\pgfqpoint{7.622285in}{1.407846in}}{\pgfqpoint{7.617895in}{1.397247in}}{\pgfqpoint{7.617895in}{1.386197in}}%
\pgfpathcurveto{\pgfqpoint{7.617895in}{1.375147in}}{\pgfqpoint{7.622285in}{1.364548in}}{\pgfqpoint{7.630098in}{1.356734in}}%
\pgfpathcurveto{\pgfqpoint{7.637912in}{1.348920in}}{\pgfqpoint{7.648511in}{1.344530in}}{\pgfqpoint{7.659561in}{1.344530in}}%
\pgfpathclose%
\pgfusepath{stroke,fill}%
\end{pgfscope}%
\begin{pgfscope}%
\pgfpathrectangle{\pgfqpoint{0.570343in}{0.331635in}}{\pgfqpoint{9.300000in}{7.700000in}}%
\pgfusepath{clip}%
\pgfsetbuttcap%
\pgfsetroundjoin%
\definecolor{currentfill}{rgb}{1.000000,0.705882,0.509804}%
\pgfsetfillcolor{currentfill}%
\pgfsetlinewidth{0.481800pt}%
\definecolor{currentstroke}{rgb}{1.000000,1.000000,1.000000}%
\pgfsetstrokecolor{currentstroke}%
\pgfsetdash{}{0pt}%
\pgfpathmoveto{\pgfqpoint{6.328447in}{4.263692in}}%
\pgfpathcurveto{\pgfqpoint{6.339497in}{4.263692in}}{\pgfqpoint{6.350096in}{4.268082in}}{\pgfqpoint{6.357910in}{4.275896in}}%
\pgfpathcurveto{\pgfqpoint{6.365724in}{4.283709in}}{\pgfqpoint{6.370114in}{4.294308in}}{\pgfqpoint{6.370114in}{4.305358in}}%
\pgfpathcurveto{\pgfqpoint{6.370114in}{4.316408in}}{\pgfqpoint{6.365724in}{4.327007in}}{\pgfqpoint{6.357910in}{4.334821in}}%
\pgfpathcurveto{\pgfqpoint{6.350096in}{4.342635in}}{\pgfqpoint{6.339497in}{4.347025in}}{\pgfqpoint{6.328447in}{4.347025in}}%
\pgfpathcurveto{\pgfqpoint{6.317397in}{4.347025in}}{\pgfqpoint{6.306798in}{4.342635in}}{\pgfqpoint{6.298984in}{4.334821in}}%
\pgfpathcurveto{\pgfqpoint{6.291171in}{4.327007in}}{\pgfqpoint{6.286780in}{4.316408in}}{\pgfqpoint{6.286780in}{4.305358in}}%
\pgfpathcurveto{\pgfqpoint{6.286780in}{4.294308in}}{\pgfqpoint{6.291171in}{4.283709in}}{\pgfqpoint{6.298984in}{4.275896in}}%
\pgfpathcurveto{\pgfqpoint{6.306798in}{4.268082in}}{\pgfqpoint{6.317397in}{4.263692in}}{\pgfqpoint{6.328447in}{4.263692in}}%
\pgfpathclose%
\pgfusepath{stroke,fill}%
\end{pgfscope}%
\begin{pgfscope}%
\pgfpathrectangle{\pgfqpoint{0.570343in}{0.331635in}}{\pgfqpoint{9.300000in}{7.700000in}}%
\pgfusepath{clip}%
\pgfsetbuttcap%
\pgfsetroundjoin%
\definecolor{currentfill}{rgb}{1.000000,0.705882,0.509804}%
\pgfsetfillcolor{currentfill}%
\pgfsetlinewidth{0.481800pt}%
\definecolor{currentstroke}{rgb}{1.000000,1.000000,1.000000}%
\pgfsetstrokecolor{currentstroke}%
\pgfsetdash{}{0pt}%
\pgfpathmoveto{\pgfqpoint{8.131383in}{3.712424in}}%
\pgfpathcurveto{\pgfqpoint{8.142433in}{3.712424in}}{\pgfqpoint{8.153032in}{3.716814in}}{\pgfqpoint{8.160846in}{3.724627in}}%
\pgfpathcurveto{\pgfqpoint{8.168659in}{3.732441in}}{\pgfqpoint{8.173049in}{3.743040in}}{\pgfqpoint{8.173049in}{3.754090in}}%
\pgfpathcurveto{\pgfqpoint{8.173049in}{3.765140in}}{\pgfqpoint{8.168659in}{3.775739in}}{\pgfqpoint{8.160846in}{3.783553in}}%
\pgfpathcurveto{\pgfqpoint{8.153032in}{3.791367in}}{\pgfqpoint{8.142433in}{3.795757in}}{\pgfqpoint{8.131383in}{3.795757in}}%
\pgfpathcurveto{\pgfqpoint{8.120333in}{3.795757in}}{\pgfqpoint{8.109734in}{3.791367in}}{\pgfqpoint{8.101920in}{3.783553in}}%
\pgfpathcurveto{\pgfqpoint{8.094106in}{3.775739in}}{\pgfqpoint{8.089716in}{3.765140in}}{\pgfqpoint{8.089716in}{3.754090in}}%
\pgfpathcurveto{\pgfqpoint{8.089716in}{3.743040in}}{\pgfqpoint{8.094106in}{3.732441in}}{\pgfqpoint{8.101920in}{3.724627in}}%
\pgfpathcurveto{\pgfqpoint{8.109734in}{3.716814in}}{\pgfqpoint{8.120333in}{3.712424in}}{\pgfqpoint{8.131383in}{3.712424in}}%
\pgfpathclose%
\pgfusepath{stroke,fill}%
\end{pgfscope}%
\begin{pgfscope}%
\pgfpathrectangle{\pgfqpoint{0.570343in}{0.331635in}}{\pgfqpoint{9.300000in}{7.700000in}}%
\pgfusepath{clip}%
\pgfsetbuttcap%
\pgfsetroundjoin%
\definecolor{currentfill}{rgb}{1.000000,0.705882,0.509804}%
\pgfsetfillcolor{currentfill}%
\pgfsetlinewidth{0.481800pt}%
\definecolor{currentstroke}{rgb}{1.000000,1.000000,1.000000}%
\pgfsetstrokecolor{currentstroke}%
\pgfsetdash{}{0pt}%
\pgfpathmoveto{\pgfqpoint{6.702968in}{4.863493in}}%
\pgfpathcurveto{\pgfqpoint{6.714018in}{4.863493in}}{\pgfqpoint{6.724617in}{4.867883in}}{\pgfqpoint{6.732431in}{4.875697in}}%
\pgfpathcurveto{\pgfqpoint{6.740245in}{4.883510in}}{\pgfqpoint{6.744635in}{4.894109in}}{\pgfqpoint{6.744635in}{4.905160in}}%
\pgfpathcurveto{\pgfqpoint{6.744635in}{4.916210in}}{\pgfqpoint{6.740245in}{4.926809in}}{\pgfqpoint{6.732431in}{4.934622in}}%
\pgfpathcurveto{\pgfqpoint{6.724617in}{4.942436in}}{\pgfqpoint{6.714018in}{4.946826in}}{\pgfqpoint{6.702968in}{4.946826in}}%
\pgfpathcurveto{\pgfqpoint{6.691918in}{4.946826in}}{\pgfqpoint{6.681319in}{4.942436in}}{\pgfqpoint{6.673505in}{4.934622in}}%
\pgfpathcurveto{\pgfqpoint{6.665692in}{4.926809in}}{\pgfqpoint{6.661302in}{4.916210in}}{\pgfqpoint{6.661302in}{4.905160in}}%
\pgfpathcurveto{\pgfqpoint{6.661302in}{4.894109in}}{\pgfqpoint{6.665692in}{4.883510in}}{\pgfqpoint{6.673505in}{4.875697in}}%
\pgfpathcurveto{\pgfqpoint{6.681319in}{4.867883in}}{\pgfqpoint{6.691918in}{4.863493in}}{\pgfqpoint{6.702968in}{4.863493in}}%
\pgfpathclose%
\pgfusepath{stroke,fill}%
\end{pgfscope}%
\begin{pgfscope}%
\pgfpathrectangle{\pgfqpoint{0.570343in}{0.331635in}}{\pgfqpoint{9.300000in}{7.700000in}}%
\pgfusepath{clip}%
\pgfsetbuttcap%
\pgfsetroundjoin%
\definecolor{currentfill}{rgb}{1.000000,0.705882,0.509804}%
\pgfsetfillcolor{currentfill}%
\pgfsetlinewidth{0.481800pt}%
\definecolor{currentstroke}{rgb}{1.000000,1.000000,1.000000}%
\pgfsetstrokecolor{currentstroke}%
\pgfsetdash{}{0pt}%
\pgfpathmoveto{\pgfqpoint{5.355699in}{2.026578in}}%
\pgfpathcurveto{\pgfqpoint{5.366750in}{2.026578in}}{\pgfqpoint{5.377349in}{2.030968in}}{\pgfqpoint{5.385162in}{2.038782in}}%
\pgfpathcurveto{\pgfqpoint{5.392976in}{2.046595in}}{\pgfqpoint{5.397366in}{2.057194in}}{\pgfqpoint{5.397366in}{2.068245in}}%
\pgfpathcurveto{\pgfqpoint{5.397366in}{2.079295in}}{\pgfqpoint{5.392976in}{2.089894in}}{\pgfqpoint{5.385162in}{2.097707in}}%
\pgfpathcurveto{\pgfqpoint{5.377349in}{2.105521in}}{\pgfqpoint{5.366750in}{2.109911in}}{\pgfqpoint{5.355699in}{2.109911in}}%
\pgfpathcurveto{\pgfqpoint{5.344649in}{2.109911in}}{\pgfqpoint{5.334050in}{2.105521in}}{\pgfqpoint{5.326237in}{2.097707in}}%
\pgfpathcurveto{\pgfqpoint{5.318423in}{2.089894in}}{\pgfqpoint{5.314033in}{2.079295in}}{\pgfqpoint{5.314033in}{2.068245in}}%
\pgfpathcurveto{\pgfqpoint{5.314033in}{2.057194in}}{\pgfqpoint{5.318423in}{2.046595in}}{\pgfqpoint{5.326237in}{2.038782in}}%
\pgfpathcurveto{\pgfqpoint{5.334050in}{2.030968in}}{\pgfqpoint{5.344649in}{2.026578in}}{\pgfqpoint{5.355699in}{2.026578in}}%
\pgfpathclose%
\pgfusepath{stroke,fill}%
\end{pgfscope}%
\begin{pgfscope}%
\pgfpathrectangle{\pgfqpoint{0.570343in}{0.331635in}}{\pgfqpoint{9.300000in}{7.700000in}}%
\pgfusepath{clip}%
\pgfsetbuttcap%
\pgfsetroundjoin%
\definecolor{currentfill}{rgb}{1.000000,0.705882,0.509804}%
\pgfsetfillcolor{currentfill}%
\pgfsetlinewidth{0.481800pt}%
\definecolor{currentstroke}{rgb}{1.000000,1.000000,1.000000}%
\pgfsetstrokecolor{currentstroke}%
\pgfsetdash{}{0pt}%
\pgfpathmoveto{\pgfqpoint{2.219532in}{1.446127in}}%
\pgfpathcurveto{\pgfqpoint{2.230582in}{1.446127in}}{\pgfqpoint{2.241181in}{1.450517in}}{\pgfqpoint{2.248995in}{1.458330in}}%
\pgfpathcurveto{\pgfqpoint{2.256809in}{1.466144in}}{\pgfqpoint{2.261199in}{1.476743in}}{\pgfqpoint{2.261199in}{1.487793in}}%
\pgfpathcurveto{\pgfqpoint{2.261199in}{1.498843in}}{\pgfqpoint{2.256809in}{1.509442in}}{\pgfqpoint{2.248995in}{1.517256in}}%
\pgfpathcurveto{\pgfqpoint{2.241181in}{1.525070in}}{\pgfqpoint{2.230582in}{1.529460in}}{\pgfqpoint{2.219532in}{1.529460in}}%
\pgfpathcurveto{\pgfqpoint{2.208482in}{1.529460in}}{\pgfqpoint{2.197883in}{1.525070in}}{\pgfqpoint{2.190069in}{1.517256in}}%
\pgfpathcurveto{\pgfqpoint{2.182256in}{1.509442in}}{\pgfqpoint{2.177865in}{1.498843in}}{\pgfqpoint{2.177865in}{1.487793in}}%
\pgfpathcurveto{\pgfqpoint{2.177865in}{1.476743in}}{\pgfqpoint{2.182256in}{1.466144in}}{\pgfqpoint{2.190069in}{1.458330in}}%
\pgfpathcurveto{\pgfqpoint{2.197883in}{1.450517in}}{\pgfqpoint{2.208482in}{1.446127in}}{\pgfqpoint{2.219532in}{1.446127in}}%
\pgfpathclose%
\pgfusepath{stroke,fill}%
\end{pgfscope}%
\begin{pgfscope}%
\pgfpathrectangle{\pgfqpoint{0.570343in}{0.331635in}}{\pgfqpoint{9.300000in}{7.700000in}}%
\pgfusepath{clip}%
\pgfsetbuttcap%
\pgfsetroundjoin%
\definecolor{currentfill}{rgb}{1.000000,0.705882,0.509804}%
\pgfsetfillcolor{currentfill}%
\pgfsetlinewidth{0.481800pt}%
\definecolor{currentstroke}{rgb}{1.000000,1.000000,1.000000}%
\pgfsetstrokecolor{currentstroke}%
\pgfsetdash{}{0pt}%
\pgfpathmoveto{\pgfqpoint{4.514146in}{1.406180in}}%
\pgfpathcurveto{\pgfqpoint{4.525196in}{1.406180in}}{\pgfqpoint{4.535795in}{1.410570in}}{\pgfqpoint{4.543609in}{1.418384in}}%
\pgfpathcurveto{\pgfqpoint{4.551422in}{1.426198in}}{\pgfqpoint{4.555813in}{1.436797in}}{\pgfqpoint{4.555813in}{1.447847in}}%
\pgfpathcurveto{\pgfqpoint{4.555813in}{1.458897in}}{\pgfqpoint{4.551422in}{1.469496in}}{\pgfqpoint{4.543609in}{1.477310in}}%
\pgfpathcurveto{\pgfqpoint{4.535795in}{1.485123in}}{\pgfqpoint{4.525196in}{1.489514in}}{\pgfqpoint{4.514146in}{1.489514in}}%
\pgfpathcurveto{\pgfqpoint{4.503096in}{1.489514in}}{\pgfqpoint{4.492497in}{1.485123in}}{\pgfqpoint{4.484683in}{1.477310in}}%
\pgfpathcurveto{\pgfqpoint{4.476870in}{1.469496in}}{\pgfqpoint{4.472479in}{1.458897in}}{\pgfqpoint{4.472479in}{1.447847in}}%
\pgfpathcurveto{\pgfqpoint{4.472479in}{1.436797in}}{\pgfqpoint{4.476870in}{1.426198in}}{\pgfqpoint{4.484683in}{1.418384in}}%
\pgfpathcurveto{\pgfqpoint{4.492497in}{1.410570in}}{\pgfqpoint{4.503096in}{1.406180in}}{\pgfqpoint{4.514146in}{1.406180in}}%
\pgfpathclose%
\pgfusepath{stroke,fill}%
\end{pgfscope}%
\begin{pgfscope}%
\pgfpathrectangle{\pgfqpoint{0.570343in}{0.331635in}}{\pgfqpoint{9.300000in}{7.700000in}}%
\pgfusepath{clip}%
\pgfsetbuttcap%
\pgfsetroundjoin%
\definecolor{currentfill}{rgb}{1.000000,0.705882,0.509804}%
\pgfsetfillcolor{currentfill}%
\pgfsetlinewidth{0.481800pt}%
\definecolor{currentstroke}{rgb}{1.000000,1.000000,1.000000}%
\pgfsetstrokecolor{currentstroke}%
\pgfsetdash{}{0pt}%
\pgfpathmoveto{\pgfqpoint{5.290381in}{2.832328in}}%
\pgfpathcurveto{\pgfqpoint{5.301431in}{2.832328in}}{\pgfqpoint{5.312030in}{2.836718in}}{\pgfqpoint{5.319843in}{2.844532in}}%
\pgfpathcurveto{\pgfqpoint{5.327657in}{2.852345in}}{\pgfqpoint{5.332047in}{2.862944in}}{\pgfqpoint{5.332047in}{2.873994in}}%
\pgfpathcurveto{\pgfqpoint{5.332047in}{2.885044in}}{\pgfqpoint{5.327657in}{2.895643in}}{\pgfqpoint{5.319843in}{2.903457in}}%
\pgfpathcurveto{\pgfqpoint{5.312030in}{2.911271in}}{\pgfqpoint{5.301431in}{2.915661in}}{\pgfqpoint{5.290381in}{2.915661in}}%
\pgfpathcurveto{\pgfqpoint{5.279331in}{2.915661in}}{\pgfqpoint{5.268731in}{2.911271in}}{\pgfqpoint{5.260918in}{2.903457in}}%
\pgfpathcurveto{\pgfqpoint{5.253104in}{2.895643in}}{\pgfqpoint{5.248714in}{2.885044in}}{\pgfqpoint{5.248714in}{2.873994in}}%
\pgfpathcurveto{\pgfqpoint{5.248714in}{2.862944in}}{\pgfqpoint{5.253104in}{2.852345in}}{\pgfqpoint{5.260918in}{2.844532in}}%
\pgfpathcurveto{\pgfqpoint{5.268731in}{2.836718in}}{\pgfqpoint{5.279331in}{2.832328in}}{\pgfqpoint{5.290381in}{2.832328in}}%
\pgfpathclose%
\pgfusepath{stroke,fill}%
\end{pgfscope}%
\begin{pgfscope}%
\pgfpathrectangle{\pgfqpoint{0.570343in}{0.331635in}}{\pgfqpoint{9.300000in}{7.700000in}}%
\pgfusepath{clip}%
\pgfsetbuttcap%
\pgfsetroundjoin%
\definecolor{currentfill}{rgb}{1.000000,0.705882,0.509804}%
\pgfsetfillcolor{currentfill}%
\pgfsetlinewidth{0.481800pt}%
\definecolor{currentstroke}{rgb}{1.000000,1.000000,1.000000}%
\pgfsetstrokecolor{currentstroke}%
\pgfsetdash{}{0pt}%
\pgfpathmoveto{\pgfqpoint{4.901182in}{6.182179in}}%
\pgfpathcurveto{\pgfqpoint{4.912232in}{6.182179in}}{\pgfqpoint{4.922831in}{6.186569in}}{\pgfqpoint{4.930645in}{6.194383in}}%
\pgfpathcurveto{\pgfqpoint{4.938459in}{6.202197in}}{\pgfqpoint{4.942849in}{6.212796in}}{\pgfqpoint{4.942849in}{6.223846in}}%
\pgfpathcurveto{\pgfqpoint{4.942849in}{6.234896in}}{\pgfqpoint{4.938459in}{6.245495in}}{\pgfqpoint{4.930645in}{6.253308in}}%
\pgfpathcurveto{\pgfqpoint{4.922831in}{6.261122in}}{\pgfqpoint{4.912232in}{6.265512in}}{\pgfqpoint{4.901182in}{6.265512in}}%
\pgfpathcurveto{\pgfqpoint{4.890132in}{6.265512in}}{\pgfqpoint{4.879533in}{6.261122in}}{\pgfqpoint{4.871719in}{6.253308in}}%
\pgfpathcurveto{\pgfqpoint{4.863906in}{6.245495in}}{\pgfqpoint{4.859516in}{6.234896in}}{\pgfqpoint{4.859516in}{6.223846in}}%
\pgfpathcurveto{\pgfqpoint{4.859516in}{6.212796in}}{\pgfqpoint{4.863906in}{6.202197in}}{\pgfqpoint{4.871719in}{6.194383in}}%
\pgfpathcurveto{\pgfqpoint{4.879533in}{6.186569in}}{\pgfqpoint{4.890132in}{6.182179in}}{\pgfqpoint{4.901182in}{6.182179in}}%
\pgfpathclose%
\pgfusepath{stroke,fill}%
\end{pgfscope}%
\begin{pgfscope}%
\pgfpathrectangle{\pgfqpoint{0.570343in}{0.331635in}}{\pgfqpoint{9.300000in}{7.700000in}}%
\pgfusepath{clip}%
\pgfsetbuttcap%
\pgfsetroundjoin%
\definecolor{currentfill}{rgb}{1.000000,0.705882,0.509804}%
\pgfsetfillcolor{currentfill}%
\pgfsetlinewidth{0.481800pt}%
\definecolor{currentstroke}{rgb}{1.000000,1.000000,1.000000}%
\pgfsetstrokecolor{currentstroke}%
\pgfsetdash{}{0pt}%
\pgfpathmoveto{\pgfqpoint{8.177594in}{2.886332in}}%
\pgfpathcurveto{\pgfqpoint{8.188644in}{2.886332in}}{\pgfqpoint{8.199243in}{2.890723in}}{\pgfqpoint{8.207057in}{2.898536in}}%
\pgfpathcurveto{\pgfqpoint{8.214870in}{2.906350in}}{\pgfqpoint{8.219260in}{2.916949in}}{\pgfqpoint{8.219260in}{2.927999in}}%
\pgfpathcurveto{\pgfqpoint{8.219260in}{2.939049in}}{\pgfqpoint{8.214870in}{2.949648in}}{\pgfqpoint{8.207057in}{2.957462in}}%
\pgfpathcurveto{\pgfqpoint{8.199243in}{2.965276in}}{\pgfqpoint{8.188644in}{2.969666in}}{\pgfqpoint{8.177594in}{2.969666in}}%
\pgfpathcurveto{\pgfqpoint{8.166544in}{2.969666in}}{\pgfqpoint{8.155945in}{2.965276in}}{\pgfqpoint{8.148131in}{2.957462in}}%
\pgfpathcurveto{\pgfqpoint{8.140317in}{2.949648in}}{\pgfqpoint{8.135927in}{2.939049in}}{\pgfqpoint{8.135927in}{2.927999in}}%
\pgfpathcurveto{\pgfqpoint{8.135927in}{2.916949in}}{\pgfqpoint{8.140317in}{2.906350in}}{\pgfqpoint{8.148131in}{2.898536in}}%
\pgfpathcurveto{\pgfqpoint{8.155945in}{2.890723in}}{\pgfqpoint{8.166544in}{2.886332in}}{\pgfqpoint{8.177594in}{2.886332in}}%
\pgfpathclose%
\pgfusepath{stroke,fill}%
\end{pgfscope}%
\begin{pgfscope}%
\pgfpathrectangle{\pgfqpoint{0.570343in}{0.331635in}}{\pgfqpoint{9.300000in}{7.700000in}}%
\pgfusepath{clip}%
\pgfsetbuttcap%
\pgfsetroundjoin%
\definecolor{currentfill}{rgb}{1.000000,0.705882,0.509804}%
\pgfsetfillcolor{currentfill}%
\pgfsetlinewidth{0.481800pt}%
\definecolor{currentstroke}{rgb}{1.000000,1.000000,1.000000}%
\pgfsetstrokecolor{currentstroke}%
\pgfsetdash{}{0pt}%
\pgfpathmoveto{\pgfqpoint{7.239046in}{7.639968in}}%
\pgfpathcurveto{\pgfqpoint{7.250096in}{7.639968in}}{\pgfqpoint{7.260695in}{7.644359in}}{\pgfqpoint{7.268508in}{7.652172in}}%
\pgfpathcurveto{\pgfqpoint{7.276322in}{7.659986in}}{\pgfqpoint{7.280712in}{7.670585in}}{\pgfqpoint{7.280712in}{7.681635in}}%
\pgfpathcurveto{\pgfqpoint{7.280712in}{7.692685in}}{\pgfqpoint{7.276322in}{7.703284in}}{\pgfqpoint{7.268508in}{7.711098in}}%
\pgfpathcurveto{\pgfqpoint{7.260695in}{7.718911in}}{\pgfqpoint{7.250096in}{7.723302in}}{\pgfqpoint{7.239046in}{7.723302in}}%
\pgfpathcurveto{\pgfqpoint{7.227996in}{7.723302in}}{\pgfqpoint{7.217397in}{7.718911in}}{\pgfqpoint{7.209583in}{7.711098in}}%
\pgfpathcurveto{\pgfqpoint{7.201769in}{7.703284in}}{\pgfqpoint{7.197379in}{7.692685in}}{\pgfqpoint{7.197379in}{7.681635in}}%
\pgfpathcurveto{\pgfqpoint{7.197379in}{7.670585in}}{\pgfqpoint{7.201769in}{7.659986in}}{\pgfqpoint{7.209583in}{7.652172in}}%
\pgfpathcurveto{\pgfqpoint{7.217397in}{7.644359in}}{\pgfqpoint{7.227996in}{7.639968in}}{\pgfqpoint{7.239046in}{7.639968in}}%
\pgfpathclose%
\pgfusepath{stroke,fill}%
\end{pgfscope}%
\begin{pgfscope}%
\pgfpathrectangle{\pgfqpoint{0.570343in}{0.331635in}}{\pgfqpoint{9.300000in}{7.700000in}}%
\pgfusepath{clip}%
\pgfsetbuttcap%
\pgfsetroundjoin%
\definecolor{currentfill}{rgb}{1.000000,0.705882,0.509804}%
\pgfsetfillcolor{currentfill}%
\pgfsetlinewidth{0.481800pt}%
\definecolor{currentstroke}{rgb}{1.000000,1.000000,1.000000}%
\pgfsetstrokecolor{currentstroke}%
\pgfsetdash{}{0pt}%
\pgfpathmoveto{\pgfqpoint{3.182112in}{0.639968in}}%
\pgfpathcurveto{\pgfqpoint{3.193162in}{0.639968in}}{\pgfqpoint{3.203761in}{0.644359in}}{\pgfqpoint{3.211575in}{0.652172in}}%
\pgfpathcurveto{\pgfqpoint{3.219389in}{0.659986in}}{\pgfqpoint{3.223779in}{0.670585in}}{\pgfqpoint{3.223779in}{0.681635in}}%
\pgfpathcurveto{\pgfqpoint{3.223779in}{0.692685in}}{\pgfqpoint{3.219389in}{0.703284in}}{\pgfqpoint{3.211575in}{0.711098in}}%
\pgfpathcurveto{\pgfqpoint{3.203761in}{0.718911in}}{\pgfqpoint{3.193162in}{0.723302in}}{\pgfqpoint{3.182112in}{0.723302in}}%
\pgfpathcurveto{\pgfqpoint{3.171062in}{0.723302in}}{\pgfqpoint{3.160463in}{0.718911in}}{\pgfqpoint{3.152649in}{0.711098in}}%
\pgfpathcurveto{\pgfqpoint{3.144836in}{0.703284in}}{\pgfqpoint{3.140446in}{0.692685in}}{\pgfqpoint{3.140446in}{0.681635in}}%
\pgfpathcurveto{\pgfqpoint{3.140446in}{0.670585in}}{\pgfqpoint{3.144836in}{0.659986in}}{\pgfqpoint{3.152649in}{0.652172in}}%
\pgfpathcurveto{\pgfqpoint{3.160463in}{0.644359in}}{\pgfqpoint{3.171062in}{0.639968in}}{\pgfqpoint{3.182112in}{0.639968in}}%
\pgfpathclose%
\pgfusepath{stroke,fill}%
\end{pgfscope}%
\begin{pgfscope}%
\pgfpathrectangle{\pgfqpoint{0.570343in}{0.331635in}}{\pgfqpoint{9.300000in}{7.700000in}}%
\pgfusepath{clip}%
\pgfsetbuttcap%
\pgfsetroundjoin%
\definecolor{currentfill}{rgb}{1.000000,0.705882,0.509804}%
\pgfsetfillcolor{currentfill}%
\pgfsetlinewidth{0.481800pt}%
\definecolor{currentstroke}{rgb}{1.000000,1.000000,1.000000}%
\pgfsetstrokecolor{currentstroke}%
\pgfsetdash{}{0pt}%
\pgfpathmoveto{\pgfqpoint{7.238874in}{5.429662in}}%
\pgfpathcurveto{\pgfqpoint{7.249925in}{5.429662in}}{\pgfqpoint{7.260524in}{5.434052in}}{\pgfqpoint{7.268337in}{5.441866in}}%
\pgfpathcurveto{\pgfqpoint{7.276151in}{5.449680in}}{\pgfqpoint{7.280541in}{5.460279in}}{\pgfqpoint{7.280541in}{5.471329in}}%
\pgfpathcurveto{\pgfqpoint{7.280541in}{5.482379in}}{\pgfqpoint{7.276151in}{5.492978in}}{\pgfqpoint{7.268337in}{5.500792in}}%
\pgfpathcurveto{\pgfqpoint{7.260524in}{5.508605in}}{\pgfqpoint{7.249925in}{5.512995in}}{\pgfqpoint{7.238874in}{5.512995in}}%
\pgfpathcurveto{\pgfqpoint{7.227824in}{5.512995in}}{\pgfqpoint{7.217225in}{5.508605in}}{\pgfqpoint{7.209412in}{5.500792in}}%
\pgfpathcurveto{\pgfqpoint{7.201598in}{5.492978in}}{\pgfqpoint{7.197208in}{5.482379in}}{\pgfqpoint{7.197208in}{5.471329in}}%
\pgfpathcurveto{\pgfqpoint{7.197208in}{5.460279in}}{\pgfqpoint{7.201598in}{5.449680in}}{\pgfqpoint{7.209412in}{5.441866in}}%
\pgfpathcurveto{\pgfqpoint{7.217225in}{5.434052in}}{\pgfqpoint{7.227824in}{5.429662in}}{\pgfqpoint{7.238874in}{5.429662in}}%
\pgfpathclose%
\pgfusepath{stroke,fill}%
\end{pgfscope}%
\begin{pgfscope}%
\pgfpathrectangle{\pgfqpoint{0.570343in}{0.331635in}}{\pgfqpoint{9.300000in}{7.700000in}}%
\pgfusepath{clip}%
\pgfsetbuttcap%
\pgfsetroundjoin%
\definecolor{currentfill}{rgb}{1.000000,0.705882,0.509804}%
\pgfsetfillcolor{currentfill}%
\pgfsetlinewidth{0.481800pt}%
\definecolor{currentstroke}{rgb}{1.000000,1.000000,1.000000}%
\pgfsetstrokecolor{currentstroke}%
\pgfsetdash{}{0pt}%
\pgfpathmoveto{\pgfqpoint{3.375695in}{1.877218in}}%
\pgfpathcurveto{\pgfqpoint{3.386745in}{1.877218in}}{\pgfqpoint{3.397344in}{1.881608in}}{\pgfqpoint{3.405158in}{1.889422in}}%
\pgfpathcurveto{\pgfqpoint{3.412971in}{1.897236in}}{\pgfqpoint{3.417362in}{1.907835in}}{\pgfqpoint{3.417362in}{1.918885in}}%
\pgfpathcurveto{\pgfqpoint{3.417362in}{1.929935in}}{\pgfqpoint{3.412971in}{1.940534in}}{\pgfqpoint{3.405158in}{1.948348in}}%
\pgfpathcurveto{\pgfqpoint{3.397344in}{1.956161in}}{\pgfqpoint{3.386745in}{1.960551in}}{\pgfqpoint{3.375695in}{1.960551in}}%
\pgfpathcurveto{\pgfqpoint{3.364645in}{1.960551in}}{\pgfqpoint{3.354046in}{1.956161in}}{\pgfqpoint{3.346232in}{1.948348in}}%
\pgfpathcurveto{\pgfqpoint{3.338419in}{1.940534in}}{\pgfqpoint{3.334028in}{1.929935in}}{\pgfqpoint{3.334028in}{1.918885in}}%
\pgfpathcurveto{\pgfqpoint{3.334028in}{1.907835in}}{\pgfqpoint{3.338419in}{1.897236in}}{\pgfqpoint{3.346232in}{1.889422in}}%
\pgfpathcurveto{\pgfqpoint{3.354046in}{1.881608in}}{\pgfqpoint{3.364645in}{1.877218in}}{\pgfqpoint{3.375695in}{1.877218in}}%
\pgfpathclose%
\pgfusepath{stroke,fill}%
\end{pgfscope}%
\begin{pgfscope}%
\pgfpathrectangle{\pgfqpoint{0.570343in}{0.331635in}}{\pgfqpoint{9.300000in}{7.700000in}}%
\pgfusepath{clip}%
\pgfsetbuttcap%
\pgfsetroundjoin%
\definecolor{currentfill}{rgb}{1.000000,0.705882,0.509804}%
\pgfsetfillcolor{currentfill}%
\pgfsetlinewidth{0.481800pt}%
\definecolor{currentstroke}{rgb}{1.000000,1.000000,1.000000}%
\pgfsetstrokecolor{currentstroke}%
\pgfsetdash{}{0pt}%
\pgfpathmoveto{\pgfqpoint{5.763337in}{1.269409in}}%
\pgfpathcurveto{\pgfqpoint{5.774387in}{1.269409in}}{\pgfqpoint{5.784986in}{1.273799in}}{\pgfqpoint{5.792800in}{1.281613in}}%
\pgfpathcurveto{\pgfqpoint{5.800613in}{1.289427in}}{\pgfqpoint{5.805004in}{1.300026in}}{\pgfqpoint{5.805004in}{1.311076in}}%
\pgfpathcurveto{\pgfqpoint{5.805004in}{1.322126in}}{\pgfqpoint{5.800613in}{1.332725in}}{\pgfqpoint{5.792800in}{1.340539in}}%
\pgfpathcurveto{\pgfqpoint{5.784986in}{1.348352in}}{\pgfqpoint{5.774387in}{1.352742in}}{\pgfqpoint{5.763337in}{1.352742in}}%
\pgfpathcurveto{\pgfqpoint{5.752287in}{1.352742in}}{\pgfqpoint{5.741688in}{1.348352in}}{\pgfqpoint{5.733874in}{1.340539in}}%
\pgfpathcurveto{\pgfqpoint{5.726061in}{1.332725in}}{\pgfqpoint{5.721670in}{1.322126in}}{\pgfqpoint{5.721670in}{1.311076in}}%
\pgfpathcurveto{\pgfqpoint{5.721670in}{1.300026in}}{\pgfqpoint{5.726061in}{1.289427in}}{\pgfqpoint{5.733874in}{1.281613in}}%
\pgfpathcurveto{\pgfqpoint{5.741688in}{1.273799in}}{\pgfqpoint{5.752287in}{1.269409in}}{\pgfqpoint{5.763337in}{1.269409in}}%
\pgfpathclose%
\pgfusepath{stroke,fill}%
\end{pgfscope}%
\begin{pgfscope}%
\pgfpathrectangle{\pgfqpoint{0.570343in}{0.331635in}}{\pgfqpoint{9.300000in}{7.700000in}}%
\pgfusepath{clip}%
\pgfsetbuttcap%
\pgfsetroundjoin%
\definecolor{currentfill}{rgb}{1.000000,0.705882,0.509804}%
\pgfsetfillcolor{currentfill}%
\pgfsetlinewidth{0.481800pt}%
\definecolor{currentstroke}{rgb}{1.000000,1.000000,1.000000}%
\pgfsetstrokecolor{currentstroke}%
\pgfsetdash{}{0pt}%
\pgfpathmoveto{\pgfqpoint{2.235429in}{2.293475in}}%
\pgfpathcurveto{\pgfqpoint{2.246479in}{2.293475in}}{\pgfqpoint{2.257078in}{2.297865in}}{\pgfqpoint{2.264891in}{2.305678in}}%
\pgfpathcurveto{\pgfqpoint{2.272705in}{2.313492in}}{\pgfqpoint{2.277095in}{2.324091in}}{\pgfqpoint{2.277095in}{2.335141in}}%
\pgfpathcurveto{\pgfqpoint{2.277095in}{2.346191in}}{\pgfqpoint{2.272705in}{2.356790in}}{\pgfqpoint{2.264891in}{2.364604in}}%
\pgfpathcurveto{\pgfqpoint{2.257078in}{2.372418in}}{\pgfqpoint{2.246479in}{2.376808in}}{\pgfqpoint{2.235429in}{2.376808in}}%
\pgfpathcurveto{\pgfqpoint{2.224378in}{2.376808in}}{\pgfqpoint{2.213779in}{2.372418in}}{\pgfqpoint{2.205966in}{2.364604in}}%
\pgfpathcurveto{\pgfqpoint{2.198152in}{2.356790in}}{\pgfqpoint{2.193762in}{2.346191in}}{\pgfqpoint{2.193762in}{2.335141in}}%
\pgfpathcurveto{\pgfqpoint{2.193762in}{2.324091in}}{\pgfqpoint{2.198152in}{2.313492in}}{\pgfqpoint{2.205966in}{2.305678in}}%
\pgfpathcurveto{\pgfqpoint{2.213779in}{2.297865in}}{\pgfqpoint{2.224378in}{2.293475in}}{\pgfqpoint{2.235429in}{2.293475in}}%
\pgfpathclose%
\pgfusepath{stroke,fill}%
\end{pgfscope}%
\begin{pgfscope}%
\pgfpathrectangle{\pgfqpoint{0.570343in}{0.331635in}}{\pgfqpoint{9.300000in}{7.700000in}}%
\pgfusepath{clip}%
\pgfsetbuttcap%
\pgfsetroundjoin%
\definecolor{currentfill}{rgb}{1.000000,0.705882,0.509804}%
\pgfsetfillcolor{currentfill}%
\pgfsetlinewidth{0.481800pt}%
\definecolor{currentstroke}{rgb}{1.000000,1.000000,1.000000}%
\pgfsetstrokecolor{currentstroke}%
\pgfsetdash{}{0pt}%
\pgfpathmoveto{\pgfqpoint{8.228735in}{6.525549in}}%
\pgfpathcurveto{\pgfqpoint{8.239785in}{6.525549in}}{\pgfqpoint{8.250384in}{6.529939in}}{\pgfqpoint{8.258198in}{6.537753in}}%
\pgfpathcurveto{\pgfqpoint{8.266011in}{6.545566in}}{\pgfqpoint{8.270402in}{6.556165in}}{\pgfqpoint{8.270402in}{6.567215in}}%
\pgfpathcurveto{\pgfqpoint{8.270402in}{6.578265in}}{\pgfqpoint{8.266011in}{6.588864in}}{\pgfqpoint{8.258198in}{6.596678in}}%
\pgfpathcurveto{\pgfqpoint{8.250384in}{6.604492in}}{\pgfqpoint{8.239785in}{6.608882in}}{\pgfqpoint{8.228735in}{6.608882in}}%
\pgfpathcurveto{\pgfqpoint{8.217685in}{6.608882in}}{\pgfqpoint{8.207086in}{6.604492in}}{\pgfqpoint{8.199272in}{6.596678in}}%
\pgfpathcurveto{\pgfqpoint{8.191459in}{6.588864in}}{\pgfqpoint{8.187068in}{6.578265in}}{\pgfqpoint{8.187068in}{6.567215in}}%
\pgfpathcurveto{\pgfqpoint{8.187068in}{6.556165in}}{\pgfqpoint{8.191459in}{6.545566in}}{\pgfqpoint{8.199272in}{6.537753in}}%
\pgfpathcurveto{\pgfqpoint{8.207086in}{6.529939in}}{\pgfqpoint{8.217685in}{6.525549in}}{\pgfqpoint{8.228735in}{6.525549in}}%
\pgfpathclose%
\pgfusepath{stroke,fill}%
\end{pgfscope}%
\begin{pgfscope}%
\pgfpathrectangle{\pgfqpoint{0.570343in}{0.331635in}}{\pgfqpoint{9.300000in}{7.700000in}}%
\pgfusepath{clip}%
\pgfsetbuttcap%
\pgfsetroundjoin%
\definecolor{currentfill}{rgb}{1.000000,0.705882,0.509804}%
\pgfsetfillcolor{currentfill}%
\pgfsetlinewidth{0.481800pt}%
\definecolor{currentstroke}{rgb}{1.000000,1.000000,1.000000}%
\pgfsetstrokecolor{currentstroke}%
\pgfsetdash{}{0pt}%
\pgfpathmoveto{\pgfqpoint{4.520708in}{7.381404in}}%
\pgfpathcurveto{\pgfqpoint{4.531758in}{7.381404in}}{\pgfqpoint{4.542358in}{7.385794in}}{\pgfqpoint{4.550171in}{7.393608in}}%
\pgfpathcurveto{\pgfqpoint{4.557985in}{7.401421in}}{\pgfqpoint{4.562375in}{7.412020in}}{\pgfqpoint{4.562375in}{7.423071in}}%
\pgfpathcurveto{\pgfqpoint{4.562375in}{7.434121in}}{\pgfqpoint{4.557985in}{7.444720in}}{\pgfqpoint{4.550171in}{7.452533in}}%
\pgfpathcurveto{\pgfqpoint{4.542358in}{7.460347in}}{\pgfqpoint{4.531758in}{7.464737in}}{\pgfqpoint{4.520708in}{7.464737in}}%
\pgfpathcurveto{\pgfqpoint{4.509658in}{7.464737in}}{\pgfqpoint{4.499059in}{7.460347in}}{\pgfqpoint{4.491246in}{7.452533in}}%
\pgfpathcurveto{\pgfqpoint{4.483432in}{7.444720in}}{\pgfqpoint{4.479042in}{7.434121in}}{\pgfqpoint{4.479042in}{7.423071in}}%
\pgfpathcurveto{\pgfqpoint{4.479042in}{7.412020in}}{\pgfqpoint{4.483432in}{7.401421in}}{\pgfqpoint{4.491246in}{7.393608in}}%
\pgfpathcurveto{\pgfqpoint{4.499059in}{7.385794in}}{\pgfqpoint{4.509658in}{7.381404in}}{\pgfqpoint{4.520708in}{7.381404in}}%
\pgfpathclose%
\pgfusepath{stroke,fill}%
\end{pgfscope}%
\begin{pgfscope}%
\pgfpathrectangle{\pgfqpoint{0.570343in}{0.331635in}}{\pgfqpoint{9.300000in}{7.700000in}}%
\pgfusepath{clip}%
\pgfsetbuttcap%
\pgfsetroundjoin%
\definecolor{currentfill}{rgb}{1.000000,0.705882,0.509804}%
\pgfsetfillcolor{currentfill}%
\pgfsetlinewidth{0.481800pt}%
\definecolor{currentstroke}{rgb}{1.000000,1.000000,1.000000}%
\pgfsetstrokecolor{currentstroke}%
\pgfsetdash{}{0pt}%
\pgfpathmoveto{\pgfqpoint{5.816987in}{7.290765in}}%
\pgfpathcurveto{\pgfqpoint{5.828037in}{7.290765in}}{\pgfqpoint{5.838636in}{7.295155in}}{\pgfqpoint{5.846449in}{7.302969in}}%
\pgfpathcurveto{\pgfqpoint{5.854263in}{7.310783in}}{\pgfqpoint{5.858653in}{7.321382in}}{\pgfqpoint{5.858653in}{7.332432in}}%
\pgfpathcurveto{\pgfqpoint{5.858653in}{7.343482in}}{\pgfqpoint{5.854263in}{7.354081in}}{\pgfqpoint{5.846449in}{7.361895in}}%
\pgfpathcurveto{\pgfqpoint{5.838636in}{7.369708in}}{\pgfqpoint{5.828037in}{7.374099in}}{\pgfqpoint{5.816987in}{7.374099in}}%
\pgfpathcurveto{\pgfqpoint{5.805937in}{7.374099in}}{\pgfqpoint{5.795337in}{7.369708in}}{\pgfqpoint{5.787524in}{7.361895in}}%
\pgfpathcurveto{\pgfqpoint{5.779710in}{7.354081in}}{\pgfqpoint{5.775320in}{7.343482in}}{\pgfqpoint{5.775320in}{7.332432in}}%
\pgfpathcurveto{\pgfqpoint{5.775320in}{7.321382in}}{\pgfqpoint{5.779710in}{7.310783in}}{\pgfqpoint{5.787524in}{7.302969in}}%
\pgfpathcurveto{\pgfqpoint{5.795337in}{7.295155in}}{\pgfqpoint{5.805937in}{7.290765in}}{\pgfqpoint{5.816987in}{7.290765in}}%
\pgfpathclose%
\pgfusepath{stroke,fill}%
\end{pgfscope}%
\begin{pgfscope}%
\pgfpathrectangle{\pgfqpoint{0.570343in}{0.331635in}}{\pgfqpoint{9.300000in}{7.700000in}}%
\pgfusepath{clip}%
\pgfsetbuttcap%
\pgfsetroundjoin%
\definecolor{currentfill}{rgb}{1.000000,0.705882,0.509804}%
\pgfsetfillcolor{currentfill}%
\pgfsetlinewidth{0.481800pt}%
\definecolor{currentstroke}{rgb}{1.000000,1.000000,1.000000}%
\pgfsetstrokecolor{currentstroke}%
\pgfsetdash{}{0pt}%
\pgfpathmoveto{\pgfqpoint{1.860590in}{5.242580in}}%
\pgfpathcurveto{\pgfqpoint{1.871640in}{5.242580in}}{\pgfqpoint{1.882239in}{5.246970in}}{\pgfqpoint{1.890053in}{5.254784in}}%
\pgfpathcurveto{\pgfqpoint{1.897867in}{5.262598in}}{\pgfqpoint{1.902257in}{5.273197in}}{\pgfqpoint{1.902257in}{5.284247in}}%
\pgfpathcurveto{\pgfqpoint{1.902257in}{5.295297in}}{\pgfqpoint{1.897867in}{5.305896in}}{\pgfqpoint{1.890053in}{5.313710in}}%
\pgfpathcurveto{\pgfqpoint{1.882239in}{5.321523in}}{\pgfqpoint{1.871640in}{5.325913in}}{\pgfqpoint{1.860590in}{5.325913in}}%
\pgfpathcurveto{\pgfqpoint{1.849540in}{5.325913in}}{\pgfqpoint{1.838941in}{5.321523in}}{\pgfqpoint{1.831127in}{5.313710in}}%
\pgfpathcurveto{\pgfqpoint{1.823314in}{5.305896in}}{\pgfqpoint{1.818923in}{5.295297in}}{\pgfqpoint{1.818923in}{5.284247in}}%
\pgfpathcurveto{\pgfqpoint{1.818923in}{5.273197in}}{\pgfqpoint{1.823314in}{5.262598in}}{\pgfqpoint{1.831127in}{5.254784in}}%
\pgfpathcurveto{\pgfqpoint{1.838941in}{5.246970in}}{\pgfqpoint{1.849540in}{5.242580in}}{\pgfqpoint{1.860590in}{5.242580in}}%
\pgfpathclose%
\pgfusepath{stroke,fill}%
\end{pgfscope}%
\begin{pgfscope}%
\pgfpathrectangle{\pgfqpoint{0.570343in}{0.331635in}}{\pgfqpoint{9.300000in}{7.700000in}}%
\pgfusepath{clip}%
\pgfsetbuttcap%
\pgfsetroundjoin%
\definecolor{currentfill}{rgb}{1.000000,0.705882,0.509804}%
\pgfsetfillcolor{currentfill}%
\pgfsetlinewidth{0.481800pt}%
\definecolor{currentstroke}{rgb}{1.000000,1.000000,1.000000}%
\pgfsetstrokecolor{currentstroke}%
\pgfsetdash{}{0pt}%
\pgfpathmoveto{\pgfqpoint{7.102567in}{4.100170in}}%
\pgfpathcurveto{\pgfqpoint{7.113617in}{4.100170in}}{\pgfqpoint{7.124216in}{4.104560in}}{\pgfqpoint{7.132030in}{4.112374in}}%
\pgfpathcurveto{\pgfqpoint{7.139844in}{4.120188in}}{\pgfqpoint{7.144234in}{4.130787in}}{\pgfqpoint{7.144234in}{4.141837in}}%
\pgfpathcurveto{\pgfqpoint{7.144234in}{4.152887in}}{\pgfqpoint{7.139844in}{4.163486in}}{\pgfqpoint{7.132030in}{4.171300in}}%
\pgfpathcurveto{\pgfqpoint{7.124216in}{4.179113in}}{\pgfqpoint{7.113617in}{4.183503in}}{\pgfqpoint{7.102567in}{4.183503in}}%
\pgfpathcurveto{\pgfqpoint{7.091517in}{4.183503in}}{\pgfqpoint{7.080918in}{4.179113in}}{\pgfqpoint{7.073104in}{4.171300in}}%
\pgfpathcurveto{\pgfqpoint{7.065291in}{4.163486in}}{\pgfqpoint{7.060901in}{4.152887in}}{\pgfqpoint{7.060901in}{4.141837in}}%
\pgfpathcurveto{\pgfqpoint{7.060901in}{4.130787in}}{\pgfqpoint{7.065291in}{4.120188in}}{\pgfqpoint{7.073104in}{4.112374in}}%
\pgfpathcurveto{\pgfqpoint{7.080918in}{4.104560in}}{\pgfqpoint{7.091517in}{4.100170in}}{\pgfqpoint{7.102567in}{4.100170in}}%
\pgfpathclose%
\pgfusepath{stroke,fill}%
\end{pgfscope}%
\begin{pgfscope}%
\pgfpathrectangle{\pgfqpoint{0.570343in}{0.331635in}}{\pgfqpoint{9.300000in}{7.700000in}}%
\pgfusepath{clip}%
\pgfsetbuttcap%
\pgfsetroundjoin%
\definecolor{currentfill}{rgb}{1.000000,0.705882,0.509804}%
\pgfsetfillcolor{currentfill}%
\pgfsetlinewidth{0.481800pt}%
\definecolor{currentstroke}{rgb}{1.000000,1.000000,1.000000}%
\pgfsetstrokecolor{currentstroke}%
\pgfsetdash{}{0pt}%
\pgfpathmoveto{\pgfqpoint{1.082655in}{2.149923in}}%
\pgfpathcurveto{\pgfqpoint{1.093705in}{2.149923in}}{\pgfqpoint{1.104304in}{2.154313in}}{\pgfqpoint{1.112118in}{2.162126in}}%
\pgfpathcurveto{\pgfqpoint{1.119931in}{2.169940in}}{\pgfqpoint{1.124322in}{2.180539in}}{\pgfqpoint{1.124322in}{2.191589in}}%
\pgfpathcurveto{\pgfqpoint{1.124322in}{2.202639in}}{\pgfqpoint{1.119931in}{2.213238in}}{\pgfqpoint{1.112118in}{2.221052in}}%
\pgfpathcurveto{\pgfqpoint{1.104304in}{2.228866in}}{\pgfqpoint{1.093705in}{2.233256in}}{\pgfqpoint{1.082655in}{2.233256in}}%
\pgfpathcurveto{\pgfqpoint{1.071605in}{2.233256in}}{\pgfqpoint{1.061006in}{2.228866in}}{\pgfqpoint{1.053192in}{2.221052in}}%
\pgfpathcurveto{\pgfqpoint{1.045379in}{2.213238in}}{\pgfqpoint{1.040988in}{2.202639in}}{\pgfqpoint{1.040988in}{2.191589in}}%
\pgfpathcurveto{\pgfqpoint{1.040988in}{2.180539in}}{\pgfqpoint{1.045379in}{2.169940in}}{\pgfqpoint{1.053192in}{2.162126in}}%
\pgfpathcurveto{\pgfqpoint{1.061006in}{2.154313in}}{\pgfqpoint{1.071605in}{2.149923in}}{\pgfqpoint{1.082655in}{2.149923in}}%
\pgfpathclose%
\pgfusepath{stroke,fill}%
\end{pgfscope}%
\begin{pgfscope}%
\pgfpathrectangle{\pgfqpoint{0.570343in}{0.331635in}}{\pgfqpoint{9.300000in}{7.700000in}}%
\pgfusepath{clip}%
\pgfsetbuttcap%
\pgfsetroundjoin%
\definecolor{currentfill}{rgb}{1.000000,0.705882,0.509804}%
\pgfsetfillcolor{currentfill}%
\pgfsetlinewidth{0.481800pt}%
\definecolor{currentstroke}{rgb}{1.000000,1.000000,1.000000}%
\pgfsetstrokecolor{currentstroke}%
\pgfsetdash{}{0pt}%
\pgfpathmoveto{\pgfqpoint{9.447616in}{3.296455in}}%
\pgfpathcurveto{\pgfqpoint{9.458666in}{3.296455in}}{\pgfqpoint{9.469265in}{3.300845in}}{\pgfqpoint{9.477079in}{3.308658in}}%
\pgfpathcurveto{\pgfqpoint{9.484892in}{3.316472in}}{\pgfqpoint{9.489283in}{3.327071in}}{\pgfqpoint{9.489283in}{3.338121in}}%
\pgfpathcurveto{\pgfqpoint{9.489283in}{3.349171in}}{\pgfqpoint{9.484892in}{3.359770in}}{\pgfqpoint{9.477079in}{3.367584in}}%
\pgfpathcurveto{\pgfqpoint{9.469265in}{3.375398in}}{\pgfqpoint{9.458666in}{3.379788in}}{\pgfqpoint{9.447616in}{3.379788in}}%
\pgfpathcurveto{\pgfqpoint{9.436566in}{3.379788in}}{\pgfqpoint{9.425967in}{3.375398in}}{\pgfqpoint{9.418153in}{3.367584in}}%
\pgfpathcurveto{\pgfqpoint{9.410340in}{3.359770in}}{\pgfqpoint{9.405949in}{3.349171in}}{\pgfqpoint{9.405949in}{3.338121in}}%
\pgfpathcurveto{\pgfqpoint{9.405949in}{3.327071in}}{\pgfqpoint{9.410340in}{3.316472in}}{\pgfqpoint{9.418153in}{3.308658in}}%
\pgfpathcurveto{\pgfqpoint{9.425967in}{3.300845in}}{\pgfqpoint{9.436566in}{3.296455in}}{\pgfqpoint{9.447616in}{3.296455in}}%
\pgfpathclose%
\pgfusepath{stroke,fill}%
\end{pgfscope}%
\begin{pgfscope}%
\pgfpathrectangle{\pgfqpoint{0.570343in}{0.331635in}}{\pgfqpoint{9.300000in}{7.700000in}}%
\pgfusepath{clip}%
\pgfsetbuttcap%
\pgfsetroundjoin%
\definecolor{currentfill}{rgb}{1.000000,0.705882,0.509804}%
\pgfsetfillcolor{currentfill}%
\pgfsetlinewidth{0.481800pt}%
\definecolor{currentstroke}{rgb}{1.000000,1.000000,1.000000}%
\pgfsetstrokecolor{currentstroke}%
\pgfsetdash{}{0pt}%
\pgfpathmoveto{\pgfqpoint{3.829395in}{5.151485in}}%
\pgfpathcurveto{\pgfqpoint{3.840445in}{5.151485in}}{\pgfqpoint{3.851044in}{5.155875in}}{\pgfqpoint{3.858857in}{5.163689in}}%
\pgfpathcurveto{\pgfqpoint{3.866671in}{5.171502in}}{\pgfqpoint{3.871061in}{5.182101in}}{\pgfqpoint{3.871061in}{5.193151in}}%
\pgfpathcurveto{\pgfqpoint{3.871061in}{5.204202in}}{\pgfqpoint{3.866671in}{5.214801in}}{\pgfqpoint{3.858857in}{5.222614in}}%
\pgfpathcurveto{\pgfqpoint{3.851044in}{5.230428in}}{\pgfqpoint{3.840445in}{5.234818in}}{\pgfqpoint{3.829395in}{5.234818in}}%
\pgfpathcurveto{\pgfqpoint{3.818345in}{5.234818in}}{\pgfqpoint{3.807746in}{5.230428in}}{\pgfqpoint{3.799932in}{5.222614in}}%
\pgfpathcurveto{\pgfqpoint{3.792118in}{5.214801in}}{\pgfqpoint{3.787728in}{5.204202in}}{\pgfqpoint{3.787728in}{5.193151in}}%
\pgfpathcurveto{\pgfqpoint{3.787728in}{5.182101in}}{\pgfqpoint{3.792118in}{5.171502in}}{\pgfqpoint{3.799932in}{5.163689in}}%
\pgfpathcurveto{\pgfqpoint{3.807746in}{5.155875in}}{\pgfqpoint{3.818345in}{5.151485in}}{\pgfqpoint{3.829395in}{5.151485in}}%
\pgfpathclose%
\pgfusepath{stroke,fill}%
\end{pgfscope}%
\begin{pgfscope}%
\pgfpathrectangle{\pgfqpoint{0.570343in}{0.331635in}}{\pgfqpoint{9.300000in}{7.700000in}}%
\pgfusepath{clip}%
\pgfsetbuttcap%
\pgfsetroundjoin%
\definecolor{currentfill}{rgb}{1.000000,0.705882,0.509804}%
\pgfsetfillcolor{currentfill}%
\pgfsetlinewidth{0.481800pt}%
\definecolor{currentstroke}{rgb}{1.000000,1.000000,1.000000}%
\pgfsetstrokecolor{currentstroke}%
\pgfsetdash{}{0pt}%
\pgfpathmoveto{\pgfqpoint{2.176389in}{4.174756in}}%
\pgfpathcurveto{\pgfqpoint{2.187439in}{4.174756in}}{\pgfqpoint{2.198038in}{4.179147in}}{\pgfqpoint{2.205852in}{4.186960in}}%
\pgfpathcurveto{\pgfqpoint{2.213665in}{4.194774in}}{\pgfqpoint{2.218056in}{4.205373in}}{\pgfqpoint{2.218056in}{4.216423in}}%
\pgfpathcurveto{\pgfqpoint{2.218056in}{4.227473in}}{\pgfqpoint{2.213665in}{4.238072in}}{\pgfqpoint{2.205852in}{4.245886in}}%
\pgfpathcurveto{\pgfqpoint{2.198038in}{4.253699in}}{\pgfqpoint{2.187439in}{4.258090in}}{\pgfqpoint{2.176389in}{4.258090in}}%
\pgfpathcurveto{\pgfqpoint{2.165339in}{4.258090in}}{\pgfqpoint{2.154740in}{4.253699in}}{\pgfqpoint{2.146926in}{4.245886in}}%
\pgfpathcurveto{\pgfqpoint{2.139113in}{4.238072in}}{\pgfqpoint{2.134722in}{4.227473in}}{\pgfqpoint{2.134722in}{4.216423in}}%
\pgfpathcurveto{\pgfqpoint{2.134722in}{4.205373in}}{\pgfqpoint{2.139113in}{4.194774in}}{\pgfqpoint{2.146926in}{4.186960in}}%
\pgfpathcurveto{\pgfqpoint{2.154740in}{4.179147in}}{\pgfqpoint{2.165339in}{4.174756in}}{\pgfqpoint{2.176389in}{4.174756in}}%
\pgfpathclose%
\pgfusepath{stroke,fill}%
\end{pgfscope}%
\begin{pgfscope}%
\pgfpathrectangle{\pgfqpoint{0.570343in}{0.331635in}}{\pgfqpoint{9.300000in}{7.700000in}}%
\pgfusepath{clip}%
\pgfsetbuttcap%
\pgfsetroundjoin%
\definecolor{currentfill}{rgb}{1.000000,0.705882,0.509804}%
\pgfsetfillcolor{currentfill}%
\pgfsetlinewidth{0.481800pt}%
\definecolor{currentstroke}{rgb}{1.000000,1.000000,1.000000}%
\pgfsetstrokecolor{currentstroke}%
\pgfsetdash{}{0pt}%
\pgfpathmoveto{\pgfqpoint{7.119633in}{3.356574in}}%
\pgfpathcurveto{\pgfqpoint{7.130684in}{3.356574in}}{\pgfqpoint{7.141283in}{3.360964in}}{\pgfqpoint{7.149096in}{3.368778in}}%
\pgfpathcurveto{\pgfqpoint{7.156910in}{3.376591in}}{\pgfqpoint{7.161300in}{3.387190in}}{\pgfqpoint{7.161300in}{3.398240in}}%
\pgfpathcurveto{\pgfqpoint{7.161300in}{3.409291in}}{\pgfqpoint{7.156910in}{3.419890in}}{\pgfqpoint{7.149096in}{3.427703in}}%
\pgfpathcurveto{\pgfqpoint{7.141283in}{3.435517in}}{\pgfqpoint{7.130684in}{3.439907in}}{\pgfqpoint{7.119633in}{3.439907in}}%
\pgfpathcurveto{\pgfqpoint{7.108583in}{3.439907in}}{\pgfqpoint{7.097984in}{3.435517in}}{\pgfqpoint{7.090171in}{3.427703in}}%
\pgfpathcurveto{\pgfqpoint{7.082357in}{3.419890in}}{\pgfqpoint{7.077967in}{3.409291in}}{\pgfqpoint{7.077967in}{3.398240in}}%
\pgfpathcurveto{\pgfqpoint{7.077967in}{3.387190in}}{\pgfqpoint{7.082357in}{3.376591in}}{\pgfqpoint{7.090171in}{3.368778in}}%
\pgfpathcurveto{\pgfqpoint{7.097984in}{3.360964in}}{\pgfqpoint{7.108583in}{3.356574in}}{\pgfqpoint{7.119633in}{3.356574in}}%
\pgfpathclose%
\pgfusepath{stroke,fill}%
\end{pgfscope}%
\begin{pgfscope}%
\pgfpathrectangle{\pgfqpoint{0.570343in}{0.331635in}}{\pgfqpoint{9.300000in}{7.700000in}}%
\pgfusepath{clip}%
\pgfsetbuttcap%
\pgfsetroundjoin%
\definecolor{currentfill}{rgb}{1.000000,0.705882,0.509804}%
\pgfsetfillcolor{currentfill}%
\pgfsetlinewidth{0.481800pt}%
\definecolor{currentstroke}{rgb}{1.000000,1.000000,1.000000}%
\pgfsetstrokecolor{currentstroke}%
\pgfsetdash{}{0pt}%
\pgfpathmoveto{\pgfqpoint{2.977371in}{5.463396in}}%
\pgfpathcurveto{\pgfqpoint{2.988421in}{5.463396in}}{\pgfqpoint{2.999020in}{5.467787in}}{\pgfqpoint{3.006833in}{5.475600in}}%
\pgfpathcurveto{\pgfqpoint{3.014647in}{5.483414in}}{\pgfqpoint{3.019037in}{5.494013in}}{\pgfqpoint{3.019037in}{5.505063in}}%
\pgfpathcurveto{\pgfqpoint{3.019037in}{5.516113in}}{\pgfqpoint{3.014647in}{5.526712in}}{\pgfqpoint{3.006833in}{5.534526in}}%
\pgfpathcurveto{\pgfqpoint{2.999020in}{5.542339in}}{\pgfqpoint{2.988421in}{5.546730in}}{\pgfqpoint{2.977371in}{5.546730in}}%
\pgfpathcurveto{\pgfqpoint{2.966321in}{5.546730in}}{\pgfqpoint{2.955721in}{5.542339in}}{\pgfqpoint{2.947908in}{5.534526in}}%
\pgfpathcurveto{\pgfqpoint{2.940094in}{5.526712in}}{\pgfqpoint{2.935704in}{5.516113in}}{\pgfqpoint{2.935704in}{5.505063in}}%
\pgfpathcurveto{\pgfqpoint{2.935704in}{5.494013in}}{\pgfqpoint{2.940094in}{5.483414in}}{\pgfqpoint{2.947908in}{5.475600in}}%
\pgfpathcurveto{\pgfqpoint{2.955721in}{5.467787in}}{\pgfqpoint{2.966321in}{5.463396in}}{\pgfqpoint{2.977371in}{5.463396in}}%
\pgfpathclose%
\pgfusepath{stroke,fill}%
\end{pgfscope}%
\begin{pgfscope}%
\pgfpathrectangle{\pgfqpoint{0.570343in}{0.331635in}}{\pgfqpoint{9.300000in}{7.700000in}}%
\pgfusepath{clip}%
\pgfsetbuttcap%
\pgfsetroundjoin%
\definecolor{currentfill}{rgb}{0.631373,0.788235,0.956863}%
\pgfsetfillcolor{currentfill}%
\pgfsetlinewidth{1.003750pt}%
\definecolor{currentstroke}{rgb}{0.631373,0.788235,0.956863}%
\pgfsetstrokecolor{currentstroke}%
\pgfsetdash{}{0pt}%
\pgfsys@defobject{currentmarker}{\pgfqpoint{-0.041667in}{-0.041667in}}{\pgfqpoint{0.041667in}{0.041667in}}{%
\pgfpathmoveto{\pgfqpoint{0.000000in}{-0.041667in}}%
\pgfpathcurveto{\pgfqpoint{0.011050in}{-0.041667in}}{\pgfqpoint{0.021649in}{-0.037276in}}{\pgfqpoint{0.029463in}{-0.029463in}}%
\pgfpathcurveto{\pgfqpoint{0.037276in}{-0.021649in}}{\pgfqpoint{0.041667in}{-0.011050in}}{\pgfqpoint{0.041667in}{0.000000in}}%
\pgfpathcurveto{\pgfqpoint{0.041667in}{0.011050in}}{\pgfqpoint{0.037276in}{0.021649in}}{\pgfqpoint{0.029463in}{0.029463in}}%
\pgfpathcurveto{\pgfqpoint{0.021649in}{0.037276in}}{\pgfqpoint{0.011050in}{0.041667in}}{\pgfqpoint{0.000000in}{0.041667in}}%
\pgfpathcurveto{\pgfqpoint{-0.011050in}{0.041667in}}{\pgfqpoint{-0.021649in}{0.037276in}}{\pgfqpoint{-0.029463in}{0.029463in}}%
\pgfpathcurveto{\pgfqpoint{-0.037276in}{0.021649in}}{\pgfqpoint{-0.041667in}{0.011050in}}{\pgfqpoint{-0.041667in}{0.000000in}}%
\pgfpathcurveto{\pgfqpoint{-0.041667in}{-0.011050in}}{\pgfqpoint{-0.037276in}{-0.021649in}}{\pgfqpoint{-0.029463in}{-0.029463in}}%
\pgfpathcurveto{\pgfqpoint{-0.021649in}{-0.037276in}}{\pgfqpoint{-0.011050in}{-0.041667in}}{\pgfqpoint{0.000000in}{-0.041667in}}%
\pgfpathclose%
\pgfusepath{stroke,fill}%
}%
\end{pgfscope}%
\begin{pgfscope}%
\pgfpathrectangle{\pgfqpoint{0.570343in}{0.331635in}}{\pgfqpoint{9.300000in}{7.700000in}}%
\pgfusepath{clip}%
\pgfsetbuttcap%
\pgfsetroundjoin%
\definecolor{currentfill}{rgb}{1.000000,0.705882,0.509804}%
\pgfsetfillcolor{currentfill}%
\pgfsetlinewidth{1.003750pt}%
\definecolor{currentstroke}{rgb}{1.000000,0.705882,0.509804}%
\pgfsetstrokecolor{currentstroke}%
\pgfsetdash{}{0pt}%
\pgfsys@defobject{currentmarker}{\pgfqpoint{-0.041667in}{-0.041667in}}{\pgfqpoint{0.041667in}{0.041667in}}{%
\pgfpathmoveto{\pgfqpoint{0.000000in}{-0.041667in}}%
\pgfpathcurveto{\pgfqpoint{0.011050in}{-0.041667in}}{\pgfqpoint{0.021649in}{-0.037276in}}{\pgfqpoint{0.029463in}{-0.029463in}}%
\pgfpathcurveto{\pgfqpoint{0.037276in}{-0.021649in}}{\pgfqpoint{0.041667in}{-0.011050in}}{\pgfqpoint{0.041667in}{0.000000in}}%
\pgfpathcurveto{\pgfqpoint{0.041667in}{0.011050in}}{\pgfqpoint{0.037276in}{0.021649in}}{\pgfqpoint{0.029463in}{0.029463in}}%
\pgfpathcurveto{\pgfqpoint{0.021649in}{0.037276in}}{\pgfqpoint{0.011050in}{0.041667in}}{\pgfqpoint{0.000000in}{0.041667in}}%
\pgfpathcurveto{\pgfqpoint{-0.011050in}{0.041667in}}{\pgfqpoint{-0.021649in}{0.037276in}}{\pgfqpoint{-0.029463in}{0.029463in}}%
\pgfpathcurveto{\pgfqpoint{-0.037276in}{0.021649in}}{\pgfqpoint{-0.041667in}{0.011050in}}{\pgfqpoint{-0.041667in}{0.000000in}}%
\pgfpathcurveto{\pgfqpoint{-0.041667in}{-0.011050in}}{\pgfqpoint{-0.037276in}{-0.021649in}}{\pgfqpoint{-0.029463in}{-0.029463in}}%
\pgfpathcurveto{\pgfqpoint{-0.021649in}{-0.037276in}}{\pgfqpoint{-0.011050in}{-0.041667in}}{\pgfqpoint{0.000000in}{-0.041667in}}%
\pgfpathclose%
\pgfusepath{stroke,fill}%
}%
\end{pgfscope}%
\begin{pgfscope}%
\pgfsetbuttcap%
\pgfsetroundjoin%
\definecolor{currentfill}{rgb}{0.000000,0.000000,0.000000}%
\pgfsetfillcolor{currentfill}%
\pgfsetlinewidth{0.803000pt}%
\definecolor{currentstroke}{rgb}{0.000000,0.000000,0.000000}%
\pgfsetstrokecolor{currentstroke}%
\pgfsetdash{}{0pt}%
\pgfsys@defobject{currentmarker}{\pgfqpoint{0.000000in}{-0.048611in}}{\pgfqpoint{0.000000in}{0.000000in}}{%
\pgfpathmoveto{\pgfqpoint{0.000000in}{0.000000in}}%
\pgfpathlineto{\pgfqpoint{0.000000in}{-0.048611in}}%
\pgfusepath{stroke,fill}%
}%
\begin{pgfscope}%
\pgfsys@transformshift{1.089085in}{0.331635in}%
\pgfsys@useobject{currentmarker}{}%
\end{pgfscope}%
\end{pgfscope}%
\begin{pgfscope}%
\definecolor{textcolor}{rgb}{0.000000,0.000000,0.000000}%
\pgfsetstrokecolor{textcolor}%
\pgfsetfillcolor{textcolor}%
\pgftext[x=1.089085in,y=0.234413in,,top]{\color{textcolor}\sffamily\fontsize{10.000000}{12.000000}\selectfont \ensuremath{-}150}%
\end{pgfscope}%
\begin{pgfscope}%
\pgfsetbuttcap%
\pgfsetroundjoin%
\definecolor{currentfill}{rgb}{0.000000,0.000000,0.000000}%
\pgfsetfillcolor{currentfill}%
\pgfsetlinewidth{0.803000pt}%
\definecolor{currentstroke}{rgb}{0.000000,0.000000,0.000000}%
\pgfsetstrokecolor{currentstroke}%
\pgfsetdash{}{0pt}%
\pgfsys@defobject{currentmarker}{\pgfqpoint{0.000000in}{-0.048611in}}{\pgfqpoint{0.000000in}{0.000000in}}{%
\pgfpathmoveto{\pgfqpoint{0.000000in}{0.000000in}}%
\pgfpathlineto{\pgfqpoint{0.000000in}{-0.048611in}}%
\pgfusepath{stroke,fill}%
}%
\begin{pgfscope}%
\pgfsys@transformshift{2.469704in}{0.331635in}%
\pgfsys@useobject{currentmarker}{}%
\end{pgfscope}%
\end{pgfscope}%
\begin{pgfscope}%
\definecolor{textcolor}{rgb}{0.000000,0.000000,0.000000}%
\pgfsetstrokecolor{textcolor}%
\pgfsetfillcolor{textcolor}%
\pgftext[x=2.469704in,y=0.234413in,,top]{\color{textcolor}\sffamily\fontsize{10.000000}{12.000000}\selectfont \ensuremath{-}100}%
\end{pgfscope}%
\begin{pgfscope}%
\pgfsetbuttcap%
\pgfsetroundjoin%
\definecolor{currentfill}{rgb}{0.000000,0.000000,0.000000}%
\pgfsetfillcolor{currentfill}%
\pgfsetlinewidth{0.803000pt}%
\definecolor{currentstroke}{rgb}{0.000000,0.000000,0.000000}%
\pgfsetstrokecolor{currentstroke}%
\pgfsetdash{}{0pt}%
\pgfsys@defobject{currentmarker}{\pgfqpoint{0.000000in}{-0.048611in}}{\pgfqpoint{0.000000in}{0.000000in}}{%
\pgfpathmoveto{\pgfqpoint{0.000000in}{0.000000in}}%
\pgfpathlineto{\pgfqpoint{0.000000in}{-0.048611in}}%
\pgfusepath{stroke,fill}%
}%
\begin{pgfscope}%
\pgfsys@transformshift{3.850323in}{0.331635in}%
\pgfsys@useobject{currentmarker}{}%
\end{pgfscope}%
\end{pgfscope}%
\begin{pgfscope}%
\definecolor{textcolor}{rgb}{0.000000,0.000000,0.000000}%
\pgfsetstrokecolor{textcolor}%
\pgfsetfillcolor{textcolor}%
\pgftext[x=3.850323in,y=0.234413in,,top]{\color{textcolor}\sffamily\fontsize{10.000000}{12.000000}\selectfont \ensuremath{-}50}%
\end{pgfscope}%
\begin{pgfscope}%
\pgfsetbuttcap%
\pgfsetroundjoin%
\definecolor{currentfill}{rgb}{0.000000,0.000000,0.000000}%
\pgfsetfillcolor{currentfill}%
\pgfsetlinewidth{0.803000pt}%
\definecolor{currentstroke}{rgb}{0.000000,0.000000,0.000000}%
\pgfsetstrokecolor{currentstroke}%
\pgfsetdash{}{0pt}%
\pgfsys@defobject{currentmarker}{\pgfqpoint{0.000000in}{-0.048611in}}{\pgfqpoint{0.000000in}{0.000000in}}{%
\pgfpathmoveto{\pgfqpoint{0.000000in}{0.000000in}}%
\pgfpathlineto{\pgfqpoint{0.000000in}{-0.048611in}}%
\pgfusepath{stroke,fill}%
}%
\begin{pgfscope}%
\pgfsys@transformshift{5.230942in}{0.331635in}%
\pgfsys@useobject{currentmarker}{}%
\end{pgfscope}%
\end{pgfscope}%
\begin{pgfscope}%
\definecolor{textcolor}{rgb}{0.000000,0.000000,0.000000}%
\pgfsetstrokecolor{textcolor}%
\pgfsetfillcolor{textcolor}%
\pgftext[x=5.230942in,y=0.234413in,,top]{\color{textcolor}\sffamily\fontsize{10.000000}{12.000000}\selectfont 0}%
\end{pgfscope}%
\begin{pgfscope}%
\pgfsetbuttcap%
\pgfsetroundjoin%
\definecolor{currentfill}{rgb}{0.000000,0.000000,0.000000}%
\pgfsetfillcolor{currentfill}%
\pgfsetlinewidth{0.803000pt}%
\definecolor{currentstroke}{rgb}{0.000000,0.000000,0.000000}%
\pgfsetstrokecolor{currentstroke}%
\pgfsetdash{}{0pt}%
\pgfsys@defobject{currentmarker}{\pgfqpoint{0.000000in}{-0.048611in}}{\pgfqpoint{0.000000in}{0.000000in}}{%
\pgfpathmoveto{\pgfqpoint{0.000000in}{0.000000in}}%
\pgfpathlineto{\pgfqpoint{0.000000in}{-0.048611in}}%
\pgfusepath{stroke,fill}%
}%
\begin{pgfscope}%
\pgfsys@transformshift{6.611561in}{0.331635in}%
\pgfsys@useobject{currentmarker}{}%
\end{pgfscope}%
\end{pgfscope}%
\begin{pgfscope}%
\definecolor{textcolor}{rgb}{0.000000,0.000000,0.000000}%
\pgfsetstrokecolor{textcolor}%
\pgfsetfillcolor{textcolor}%
\pgftext[x=6.611561in,y=0.234413in,,top]{\color{textcolor}\sffamily\fontsize{10.000000}{12.000000}\selectfont 50}%
\end{pgfscope}%
\begin{pgfscope}%
\pgfsetbuttcap%
\pgfsetroundjoin%
\definecolor{currentfill}{rgb}{0.000000,0.000000,0.000000}%
\pgfsetfillcolor{currentfill}%
\pgfsetlinewidth{0.803000pt}%
\definecolor{currentstroke}{rgb}{0.000000,0.000000,0.000000}%
\pgfsetstrokecolor{currentstroke}%
\pgfsetdash{}{0pt}%
\pgfsys@defobject{currentmarker}{\pgfqpoint{0.000000in}{-0.048611in}}{\pgfqpoint{0.000000in}{0.000000in}}{%
\pgfpathmoveto{\pgfqpoint{0.000000in}{0.000000in}}%
\pgfpathlineto{\pgfqpoint{0.000000in}{-0.048611in}}%
\pgfusepath{stroke,fill}%
}%
\begin{pgfscope}%
\pgfsys@transformshift{7.992180in}{0.331635in}%
\pgfsys@useobject{currentmarker}{}%
\end{pgfscope}%
\end{pgfscope}%
\begin{pgfscope}%
\definecolor{textcolor}{rgb}{0.000000,0.000000,0.000000}%
\pgfsetstrokecolor{textcolor}%
\pgfsetfillcolor{textcolor}%
\pgftext[x=7.992180in,y=0.234413in,,top]{\color{textcolor}\sffamily\fontsize{10.000000}{12.000000}\selectfont 100}%
\end{pgfscope}%
\begin{pgfscope}%
\pgfsetbuttcap%
\pgfsetroundjoin%
\definecolor{currentfill}{rgb}{0.000000,0.000000,0.000000}%
\pgfsetfillcolor{currentfill}%
\pgfsetlinewidth{0.803000pt}%
\definecolor{currentstroke}{rgb}{0.000000,0.000000,0.000000}%
\pgfsetstrokecolor{currentstroke}%
\pgfsetdash{}{0pt}%
\pgfsys@defobject{currentmarker}{\pgfqpoint{0.000000in}{-0.048611in}}{\pgfqpoint{0.000000in}{0.000000in}}{%
\pgfpathmoveto{\pgfqpoint{0.000000in}{0.000000in}}%
\pgfpathlineto{\pgfqpoint{0.000000in}{-0.048611in}}%
\pgfusepath{stroke,fill}%
}%
\begin{pgfscope}%
\pgfsys@transformshift{9.372799in}{0.331635in}%
\pgfsys@useobject{currentmarker}{}%
\end{pgfscope}%
\end{pgfscope}%
\begin{pgfscope}%
\definecolor{textcolor}{rgb}{0.000000,0.000000,0.000000}%
\pgfsetstrokecolor{textcolor}%
\pgfsetfillcolor{textcolor}%
\pgftext[x=9.372799in,y=0.234413in,,top]{\color{textcolor}\sffamily\fontsize{10.000000}{12.000000}\selectfont 150}%
\end{pgfscope}%
\begin{pgfscope}%
\pgfsetbuttcap%
\pgfsetroundjoin%
\definecolor{currentfill}{rgb}{0.000000,0.000000,0.000000}%
\pgfsetfillcolor{currentfill}%
\pgfsetlinewidth{0.803000pt}%
\definecolor{currentstroke}{rgb}{0.000000,0.000000,0.000000}%
\pgfsetstrokecolor{currentstroke}%
\pgfsetdash{}{0pt}%
\pgfsys@defobject{currentmarker}{\pgfqpoint{-0.048611in}{0.000000in}}{\pgfqpoint{-0.000000in}{0.000000in}}{%
\pgfpathmoveto{\pgfqpoint{-0.000000in}{0.000000in}}%
\pgfpathlineto{\pgfqpoint{-0.048611in}{0.000000in}}%
\pgfusepath{stroke,fill}%
}%
\begin{pgfscope}%
\pgfsys@transformshift{0.570343in}{0.447685in}%
\pgfsys@useobject{currentmarker}{}%
\end{pgfscope}%
\end{pgfscope}%
\begin{pgfscope}%
\definecolor{textcolor}{rgb}{0.000000,0.000000,0.000000}%
\pgfsetstrokecolor{textcolor}%
\pgfsetfillcolor{textcolor}%
\pgftext[x=0.100000in, y=0.394924in, left, base]{\color{textcolor}\sffamily\fontsize{10.000000}{12.000000}\selectfont \ensuremath{-}150}%
\end{pgfscope}%
\begin{pgfscope}%
\pgfsetbuttcap%
\pgfsetroundjoin%
\definecolor{currentfill}{rgb}{0.000000,0.000000,0.000000}%
\pgfsetfillcolor{currentfill}%
\pgfsetlinewidth{0.803000pt}%
\definecolor{currentstroke}{rgb}{0.000000,0.000000,0.000000}%
\pgfsetstrokecolor{currentstroke}%
\pgfsetdash{}{0pt}%
\pgfsys@defobject{currentmarker}{\pgfqpoint{-0.048611in}{0.000000in}}{\pgfqpoint{-0.000000in}{0.000000in}}{%
\pgfpathmoveto{\pgfqpoint{-0.000000in}{0.000000in}}%
\pgfpathlineto{\pgfqpoint{-0.048611in}{0.000000in}}%
\pgfusepath{stroke,fill}%
}%
\begin{pgfscope}%
\pgfsys@transformshift{0.570343in}{1.694204in}%
\pgfsys@useobject{currentmarker}{}%
\end{pgfscope}%
\end{pgfscope}%
\begin{pgfscope}%
\definecolor{textcolor}{rgb}{0.000000,0.000000,0.000000}%
\pgfsetstrokecolor{textcolor}%
\pgfsetfillcolor{textcolor}%
\pgftext[x=0.100000in, y=1.641443in, left, base]{\color{textcolor}\sffamily\fontsize{10.000000}{12.000000}\selectfont \ensuremath{-}100}%
\end{pgfscope}%
\begin{pgfscope}%
\pgfsetbuttcap%
\pgfsetroundjoin%
\definecolor{currentfill}{rgb}{0.000000,0.000000,0.000000}%
\pgfsetfillcolor{currentfill}%
\pgfsetlinewidth{0.803000pt}%
\definecolor{currentstroke}{rgb}{0.000000,0.000000,0.000000}%
\pgfsetstrokecolor{currentstroke}%
\pgfsetdash{}{0pt}%
\pgfsys@defobject{currentmarker}{\pgfqpoint{-0.048611in}{0.000000in}}{\pgfqpoint{-0.000000in}{0.000000in}}{%
\pgfpathmoveto{\pgfqpoint{-0.000000in}{0.000000in}}%
\pgfpathlineto{\pgfqpoint{-0.048611in}{0.000000in}}%
\pgfusepath{stroke,fill}%
}%
\begin{pgfscope}%
\pgfsys@transformshift{0.570343in}{2.940723in}%
\pgfsys@useobject{currentmarker}{}%
\end{pgfscope}%
\end{pgfscope}%
\begin{pgfscope}%
\definecolor{textcolor}{rgb}{0.000000,0.000000,0.000000}%
\pgfsetstrokecolor{textcolor}%
\pgfsetfillcolor{textcolor}%
\pgftext[x=0.188365in, y=2.887962in, left, base]{\color{textcolor}\sffamily\fontsize{10.000000}{12.000000}\selectfont \ensuremath{-}50}%
\end{pgfscope}%
\begin{pgfscope}%
\pgfsetbuttcap%
\pgfsetroundjoin%
\definecolor{currentfill}{rgb}{0.000000,0.000000,0.000000}%
\pgfsetfillcolor{currentfill}%
\pgfsetlinewidth{0.803000pt}%
\definecolor{currentstroke}{rgb}{0.000000,0.000000,0.000000}%
\pgfsetstrokecolor{currentstroke}%
\pgfsetdash{}{0pt}%
\pgfsys@defobject{currentmarker}{\pgfqpoint{-0.048611in}{0.000000in}}{\pgfqpoint{-0.000000in}{0.000000in}}{%
\pgfpathmoveto{\pgfqpoint{-0.000000in}{0.000000in}}%
\pgfpathlineto{\pgfqpoint{-0.048611in}{0.000000in}}%
\pgfusepath{stroke,fill}%
}%
\begin{pgfscope}%
\pgfsys@transformshift{0.570343in}{4.187242in}%
\pgfsys@useobject{currentmarker}{}%
\end{pgfscope}%
\end{pgfscope}%
\begin{pgfscope}%
\definecolor{textcolor}{rgb}{0.000000,0.000000,0.000000}%
\pgfsetstrokecolor{textcolor}%
\pgfsetfillcolor{textcolor}%
\pgftext[x=0.384756in, y=4.134481in, left, base]{\color{textcolor}\sffamily\fontsize{10.000000}{12.000000}\selectfont 0}%
\end{pgfscope}%
\begin{pgfscope}%
\pgfsetbuttcap%
\pgfsetroundjoin%
\definecolor{currentfill}{rgb}{0.000000,0.000000,0.000000}%
\pgfsetfillcolor{currentfill}%
\pgfsetlinewidth{0.803000pt}%
\definecolor{currentstroke}{rgb}{0.000000,0.000000,0.000000}%
\pgfsetstrokecolor{currentstroke}%
\pgfsetdash{}{0pt}%
\pgfsys@defobject{currentmarker}{\pgfqpoint{-0.048611in}{0.000000in}}{\pgfqpoint{-0.000000in}{0.000000in}}{%
\pgfpathmoveto{\pgfqpoint{-0.000000in}{0.000000in}}%
\pgfpathlineto{\pgfqpoint{-0.048611in}{0.000000in}}%
\pgfusepath{stroke,fill}%
}%
\begin{pgfscope}%
\pgfsys@transformshift{0.570343in}{5.433761in}%
\pgfsys@useobject{currentmarker}{}%
\end{pgfscope}%
\end{pgfscope}%
\begin{pgfscope}%
\definecolor{textcolor}{rgb}{0.000000,0.000000,0.000000}%
\pgfsetstrokecolor{textcolor}%
\pgfsetfillcolor{textcolor}%
\pgftext[x=0.296390in, y=5.381000in, left, base]{\color{textcolor}\sffamily\fontsize{10.000000}{12.000000}\selectfont 50}%
\end{pgfscope}%
\begin{pgfscope}%
\pgfsetbuttcap%
\pgfsetroundjoin%
\definecolor{currentfill}{rgb}{0.000000,0.000000,0.000000}%
\pgfsetfillcolor{currentfill}%
\pgfsetlinewidth{0.803000pt}%
\definecolor{currentstroke}{rgb}{0.000000,0.000000,0.000000}%
\pgfsetstrokecolor{currentstroke}%
\pgfsetdash{}{0pt}%
\pgfsys@defobject{currentmarker}{\pgfqpoint{-0.048611in}{0.000000in}}{\pgfqpoint{-0.000000in}{0.000000in}}{%
\pgfpathmoveto{\pgfqpoint{-0.000000in}{0.000000in}}%
\pgfpathlineto{\pgfqpoint{-0.048611in}{0.000000in}}%
\pgfusepath{stroke,fill}%
}%
\begin{pgfscope}%
\pgfsys@transformshift{0.570343in}{6.680280in}%
\pgfsys@useobject{currentmarker}{}%
\end{pgfscope}%
\end{pgfscope}%
\begin{pgfscope}%
\definecolor{textcolor}{rgb}{0.000000,0.000000,0.000000}%
\pgfsetstrokecolor{textcolor}%
\pgfsetfillcolor{textcolor}%
\pgftext[x=0.208025in, y=6.627519in, left, base]{\color{textcolor}\sffamily\fontsize{10.000000}{12.000000}\selectfont 100}%
\end{pgfscope}%
\begin{pgfscope}%
\pgfsetbuttcap%
\pgfsetroundjoin%
\definecolor{currentfill}{rgb}{0.000000,0.000000,0.000000}%
\pgfsetfillcolor{currentfill}%
\pgfsetlinewidth{0.803000pt}%
\definecolor{currentstroke}{rgb}{0.000000,0.000000,0.000000}%
\pgfsetstrokecolor{currentstroke}%
\pgfsetdash{}{0pt}%
\pgfsys@defobject{currentmarker}{\pgfqpoint{-0.048611in}{0.000000in}}{\pgfqpoint{-0.000000in}{0.000000in}}{%
\pgfpathmoveto{\pgfqpoint{-0.000000in}{0.000000in}}%
\pgfpathlineto{\pgfqpoint{-0.048611in}{0.000000in}}%
\pgfusepath{stroke,fill}%
}%
\begin{pgfscope}%
\pgfsys@transformshift{0.570343in}{7.926799in}%
\pgfsys@useobject{currentmarker}{}%
\end{pgfscope}%
\end{pgfscope}%
\begin{pgfscope}%
\definecolor{textcolor}{rgb}{0.000000,0.000000,0.000000}%
\pgfsetstrokecolor{textcolor}%
\pgfsetfillcolor{textcolor}%
\pgftext[x=0.208025in, y=7.874038in, left, base]{\color{textcolor}\sffamily\fontsize{10.000000}{12.000000}\selectfont 150}%
\end{pgfscope}%
\begin{pgfscope}%
\pgfpathrectangle{\pgfqpoint{0.570343in}{0.331635in}}{\pgfqpoint{9.300000in}{7.700000in}}%
\pgfusepath{clip}%
\pgfsetrectcap%
\pgfsetroundjoin%
\pgfsetlinewidth{1.505625pt}%
\definecolor{currentstroke}{rgb}{0.631373,0.788235,0.956863}%
\pgfsetstrokecolor{currentstroke}%
\pgfsetstrokeopacity{0.800000}%
\pgfsetdash{}{0pt}%
\pgfpathmoveto{\pgfqpoint{2.738718in}{6.381740in}}%
\pgfpathlineto{\pgfqpoint{4.803960in}{4.277068in}}%
\pgfusepath{stroke}%
\end{pgfscope}%
\begin{pgfscope}%
\pgfpathrectangle{\pgfqpoint{0.570343in}{0.331635in}}{\pgfqpoint{9.300000in}{7.700000in}}%
\pgfusepath{clip}%
\pgfsetrectcap%
\pgfsetroundjoin%
\pgfsetlinewidth{1.505625pt}%
\definecolor{currentstroke}{rgb}{0.631373,0.788235,0.956863}%
\pgfsetstrokecolor{currentstroke}%
\pgfsetstrokeopacity{0.800000}%
\pgfsetdash{}{0pt}%
\pgfpathmoveto{\pgfqpoint{4.645211in}{3.372948in}}%
\pgfpathlineto{\pgfqpoint{4.803960in}{4.277068in}}%
\pgfusepath{stroke}%
\end{pgfscope}%
\begin{pgfscope}%
\pgfpathrectangle{\pgfqpoint{0.570343in}{0.331635in}}{\pgfqpoint{9.300000in}{7.700000in}}%
\pgfusepath{clip}%
\pgfsetrectcap%
\pgfsetroundjoin%
\pgfsetlinewidth{1.505625pt}%
\definecolor{currentstroke}{rgb}{0.631373,0.788235,0.956863}%
\pgfsetstrokecolor{currentstroke}%
\pgfsetstrokeopacity{0.800000}%
\pgfsetdash{}{0pt}%
\pgfpathmoveto{\pgfqpoint{5.785764in}{4.783118in}}%
\pgfpathlineto{\pgfqpoint{4.803960in}{4.277068in}}%
\pgfusepath{stroke}%
\end{pgfscope}%
\begin{pgfscope}%
\pgfpathrectangle{\pgfqpoint{0.570343in}{0.331635in}}{\pgfqpoint{9.300000in}{7.700000in}}%
\pgfusepath{clip}%
\pgfsetrectcap%
\pgfsetroundjoin%
\pgfsetlinewidth{1.505625pt}%
\definecolor{currentstroke}{rgb}{0.631373,0.788235,0.956863}%
\pgfsetstrokecolor{currentstroke}%
\pgfsetstrokeopacity{0.800000}%
\pgfsetdash{}{0pt}%
\pgfpathmoveto{\pgfqpoint{5.100277in}{4.641711in}}%
\pgfpathlineto{\pgfqpoint{4.803960in}{4.277068in}}%
\pgfusepath{stroke}%
\end{pgfscope}%
\begin{pgfscope}%
\pgfpathrectangle{\pgfqpoint{0.570343in}{0.331635in}}{\pgfqpoint{9.300000in}{7.700000in}}%
\pgfusepath{clip}%
\pgfsetrectcap%
\pgfsetroundjoin%
\pgfsetlinewidth{1.505625pt}%
\definecolor{currentstroke}{rgb}{0.631373,0.788235,0.956863}%
\pgfsetstrokecolor{currentstroke}%
\pgfsetstrokeopacity{0.800000}%
\pgfsetdash{}{0pt}%
\pgfpathmoveto{\pgfqpoint{0.993071in}{4.297862in}}%
\pgfpathlineto{\pgfqpoint{4.803960in}{4.277068in}}%
\pgfusepath{stroke}%
\end{pgfscope}%
\begin{pgfscope}%
\pgfpathrectangle{\pgfqpoint{0.570343in}{0.331635in}}{\pgfqpoint{9.300000in}{7.700000in}}%
\pgfusepath{clip}%
\pgfsetrectcap%
\pgfsetroundjoin%
\pgfsetlinewidth{1.505625pt}%
\definecolor{currentstroke}{rgb}{0.631373,0.788235,0.956863}%
\pgfsetstrokecolor{currentstroke}%
\pgfsetstrokeopacity{0.800000}%
\pgfsetdash{}{0pt}%
\pgfpathmoveto{\pgfqpoint{6.176499in}{3.613833in}}%
\pgfpathlineto{\pgfqpoint{4.803960in}{4.277068in}}%
\pgfusepath{stroke}%
\end{pgfscope}%
\begin{pgfscope}%
\pgfpathrectangle{\pgfqpoint{0.570343in}{0.331635in}}{\pgfqpoint{9.300000in}{7.700000in}}%
\pgfusepath{clip}%
\pgfsetrectcap%
\pgfsetroundjoin%
\pgfsetlinewidth{1.505625pt}%
\definecolor{currentstroke}{rgb}{0.631373,0.788235,0.956863}%
\pgfsetstrokecolor{currentstroke}%
\pgfsetstrokeopacity{0.800000}%
\pgfsetdash{}{0pt}%
\pgfpathmoveto{\pgfqpoint{3.773953in}{2.580081in}}%
\pgfpathlineto{\pgfqpoint{4.803960in}{4.277068in}}%
\pgfusepath{stroke}%
\end{pgfscope}%
\begin{pgfscope}%
\pgfpathrectangle{\pgfqpoint{0.570343in}{0.331635in}}{\pgfqpoint{9.300000in}{7.700000in}}%
\pgfusepath{clip}%
\pgfsetrectcap%
\pgfsetroundjoin%
\pgfsetlinewidth{1.505625pt}%
\definecolor{currentstroke}{rgb}{0.631373,0.788235,0.956863}%
\pgfsetstrokecolor{currentstroke}%
\pgfsetstrokeopacity{0.800000}%
\pgfsetdash{}{0pt}%
\pgfpathmoveto{\pgfqpoint{7.797264in}{4.769230in}}%
\pgfpathlineto{\pgfqpoint{4.803960in}{4.277068in}}%
\pgfusepath{stroke}%
\end{pgfscope}%
\begin{pgfscope}%
\pgfpathrectangle{\pgfqpoint{0.570343in}{0.331635in}}{\pgfqpoint{9.300000in}{7.700000in}}%
\pgfusepath{clip}%
\pgfsetrectcap%
\pgfsetroundjoin%
\pgfsetlinewidth{1.505625pt}%
\definecolor{currentstroke}{rgb}{0.631373,0.788235,0.956863}%
\pgfsetstrokecolor{currentstroke}%
\pgfsetstrokeopacity{0.800000}%
\pgfsetdash{}{0pt}%
\pgfpathmoveto{\pgfqpoint{2.799738in}{3.131541in}}%
\pgfpathlineto{\pgfqpoint{4.803960in}{4.277068in}}%
\pgfusepath{stroke}%
\end{pgfscope}%
\begin{pgfscope}%
\pgfpathrectangle{\pgfqpoint{0.570343in}{0.331635in}}{\pgfqpoint{9.300000in}{7.700000in}}%
\pgfusepath{clip}%
\pgfsetrectcap%
\pgfsetroundjoin%
\pgfsetlinewidth{1.505625pt}%
\definecolor{currentstroke}{rgb}{0.631373,0.788235,0.956863}%
\pgfsetstrokecolor{currentstroke}%
\pgfsetstrokeopacity{0.800000}%
\pgfsetdash{}{0pt}%
\pgfpathmoveto{\pgfqpoint{6.185154in}{2.892476in}}%
\pgfpathlineto{\pgfqpoint{4.803960in}{4.277068in}}%
\pgfusepath{stroke}%
\end{pgfscope}%
\begin{pgfscope}%
\pgfpathrectangle{\pgfqpoint{0.570343in}{0.331635in}}{\pgfqpoint{9.300000in}{7.700000in}}%
\pgfusepath{clip}%
\pgfsetrectcap%
\pgfsetroundjoin%
\pgfsetlinewidth{1.505625pt}%
\definecolor{currentstroke}{rgb}{0.631373,0.788235,0.956863}%
\pgfsetstrokecolor{currentstroke}%
\pgfsetstrokeopacity{0.800000}%
\pgfsetdash{}{0pt}%
\pgfpathmoveto{\pgfqpoint{2.095236in}{3.512298in}}%
\pgfpathlineto{\pgfqpoint{4.803960in}{4.277068in}}%
\pgfusepath{stroke}%
\end{pgfscope}%
\begin{pgfscope}%
\pgfpathrectangle{\pgfqpoint{0.570343in}{0.331635in}}{\pgfqpoint{9.300000in}{7.700000in}}%
\pgfusepath{clip}%
\pgfsetrectcap%
\pgfsetroundjoin%
\pgfsetlinewidth{1.505625pt}%
\definecolor{currentstroke}{rgb}{0.631373,0.788235,0.956863}%
\pgfsetstrokecolor{currentstroke}%
\pgfsetstrokeopacity{0.800000}%
\pgfsetdash{}{0pt}%
\pgfpathmoveto{\pgfqpoint{4.706525in}{5.302161in}}%
\pgfpathlineto{\pgfqpoint{4.803960in}{4.277068in}}%
\pgfusepath{stroke}%
\end{pgfscope}%
\begin{pgfscope}%
\pgfpathrectangle{\pgfqpoint{0.570343in}{0.331635in}}{\pgfqpoint{9.300000in}{7.700000in}}%
\pgfusepath{clip}%
\pgfsetrectcap%
\pgfsetroundjoin%
\pgfsetlinewidth{1.505625pt}%
\definecolor{currentstroke}{rgb}{0.631373,0.788235,0.956863}%
\pgfsetstrokecolor{currentstroke}%
\pgfsetstrokeopacity{0.800000}%
\pgfsetdash{}{0pt}%
\pgfpathmoveto{\pgfqpoint{3.943202in}{3.936373in}}%
\pgfpathlineto{\pgfqpoint{4.803960in}{4.277068in}}%
\pgfusepath{stroke}%
\end{pgfscope}%
\begin{pgfscope}%
\pgfpathrectangle{\pgfqpoint{0.570343in}{0.331635in}}{\pgfqpoint{9.300000in}{7.700000in}}%
\pgfusepath{clip}%
\pgfsetrectcap%
\pgfsetroundjoin%
\pgfsetlinewidth{1.505625pt}%
\definecolor{currentstroke}{rgb}{0.631373,0.788235,0.956863}%
\pgfsetstrokecolor{currentstroke}%
\pgfsetstrokeopacity{0.800000}%
\pgfsetdash{}{0pt}%
\pgfpathmoveto{\pgfqpoint{4.635731in}{4.003360in}}%
\pgfpathlineto{\pgfqpoint{4.803960in}{4.277068in}}%
\pgfusepath{stroke}%
\end{pgfscope}%
\begin{pgfscope}%
\pgfpathrectangle{\pgfqpoint{0.570343in}{0.331635in}}{\pgfqpoint{9.300000in}{7.700000in}}%
\pgfusepath{clip}%
\pgfsetrectcap%
\pgfsetroundjoin%
\pgfsetlinewidth{1.505625pt}%
\definecolor{currentstroke}{rgb}{0.631373,0.788235,0.956863}%
\pgfsetstrokecolor{currentstroke}%
\pgfsetstrokeopacity{0.800000}%
\pgfsetdash{}{0pt}%
\pgfpathmoveto{\pgfqpoint{3.827407in}{3.304553in}}%
\pgfpathlineto{\pgfqpoint{4.803960in}{4.277068in}}%
\pgfusepath{stroke}%
\end{pgfscope}%
\begin{pgfscope}%
\pgfpathrectangle{\pgfqpoint{0.570343in}{0.331635in}}{\pgfqpoint{9.300000in}{7.700000in}}%
\pgfusepath{clip}%
\pgfsetrectcap%
\pgfsetroundjoin%
\pgfsetlinewidth{1.505625pt}%
\definecolor{currentstroke}{rgb}{0.631373,0.788235,0.956863}%
\pgfsetstrokecolor{currentstroke}%
\pgfsetstrokeopacity{0.800000}%
\pgfsetdash{}{0pt}%
\pgfpathmoveto{\pgfqpoint{3.605912in}{4.491467in}}%
\pgfpathlineto{\pgfqpoint{4.803960in}{4.277068in}}%
\pgfusepath{stroke}%
\end{pgfscope}%
\begin{pgfscope}%
\pgfpathrectangle{\pgfqpoint{0.570343in}{0.331635in}}{\pgfqpoint{9.300000in}{7.700000in}}%
\pgfusepath{clip}%
\pgfsetrectcap%
\pgfsetroundjoin%
\pgfsetlinewidth{1.505625pt}%
\definecolor{currentstroke}{rgb}{0.631373,0.788235,0.956863}%
\pgfsetstrokecolor{currentstroke}%
\pgfsetstrokeopacity{0.800000}%
\pgfsetdash{}{0pt}%
\pgfpathmoveto{\pgfqpoint{5.549912in}{4.187109in}}%
\pgfpathlineto{\pgfqpoint{4.803960in}{4.277068in}}%
\pgfusepath{stroke}%
\end{pgfscope}%
\begin{pgfscope}%
\pgfpathrectangle{\pgfqpoint{0.570343in}{0.331635in}}{\pgfqpoint{9.300000in}{7.700000in}}%
\pgfusepath{clip}%
\pgfsetrectcap%
\pgfsetroundjoin%
\pgfsetlinewidth{1.505625pt}%
\definecolor{currentstroke}{rgb}{0.631373,0.788235,0.956863}%
\pgfsetstrokecolor{currentstroke}%
\pgfsetstrokeopacity{0.800000}%
\pgfsetdash{}{0pt}%
\pgfpathmoveto{\pgfqpoint{6.429778in}{5.562008in}}%
\pgfpathlineto{\pgfqpoint{4.803960in}{4.277068in}}%
\pgfusepath{stroke}%
\end{pgfscope}%
\begin{pgfscope}%
\pgfpathrectangle{\pgfqpoint{0.570343in}{0.331635in}}{\pgfqpoint{9.300000in}{7.700000in}}%
\pgfusepath{clip}%
\pgfsetrectcap%
\pgfsetroundjoin%
\pgfsetlinewidth{1.505625pt}%
\definecolor{currentstroke}{rgb}{0.631373,0.788235,0.956863}%
\pgfsetstrokecolor{currentstroke}%
\pgfsetstrokeopacity{0.800000}%
\pgfsetdash{}{0pt}%
\pgfpathmoveto{\pgfqpoint{4.510732in}{2.730289in}}%
\pgfpathlineto{\pgfqpoint{4.803960in}{4.277068in}}%
\pgfusepath{stroke}%
\end{pgfscope}%
\begin{pgfscope}%
\pgfpathrectangle{\pgfqpoint{0.570343in}{0.331635in}}{\pgfqpoint{9.300000in}{7.700000in}}%
\pgfusepath{clip}%
\pgfsetrectcap%
\pgfsetroundjoin%
\pgfsetlinewidth{1.505625pt}%
\definecolor{currentstroke}{rgb}{0.631373,0.788235,0.956863}%
\pgfsetstrokecolor{currentstroke}%
\pgfsetstrokeopacity{0.800000}%
\pgfsetdash{}{0pt}%
\pgfpathmoveto{\pgfqpoint{5.386210in}{3.547338in}}%
\pgfpathlineto{\pgfqpoint{4.803960in}{4.277068in}}%
\pgfusepath{stroke}%
\end{pgfscope}%
\begin{pgfscope}%
\pgfpathrectangle{\pgfqpoint{0.570343in}{0.331635in}}{\pgfqpoint{9.300000in}{7.700000in}}%
\pgfusepath{clip}%
\pgfsetrectcap%
\pgfsetroundjoin%
\pgfsetlinewidth{1.505625pt}%
\definecolor{currentstroke}{rgb}{0.631373,0.788235,0.956863}%
\pgfsetstrokecolor{currentstroke}%
\pgfsetstrokeopacity{0.800000}%
\pgfsetdash{}{0pt}%
\pgfpathmoveto{\pgfqpoint{8.348041in}{5.497643in}}%
\pgfpathlineto{\pgfqpoint{4.803960in}{4.277068in}}%
\pgfusepath{stroke}%
\end{pgfscope}%
\begin{pgfscope}%
\pgfpathrectangle{\pgfqpoint{0.570343in}{0.331635in}}{\pgfqpoint{9.300000in}{7.700000in}}%
\pgfusepath{clip}%
\pgfsetrectcap%
\pgfsetroundjoin%
\pgfsetlinewidth{1.505625pt}%
\definecolor{currentstroke}{rgb}{0.631373,0.788235,0.956863}%
\pgfsetstrokecolor{currentstroke}%
\pgfsetstrokeopacity{0.800000}%
\pgfsetdash{}{0pt}%
\pgfpathmoveto{\pgfqpoint{5.507412in}{5.406482in}}%
\pgfpathlineto{\pgfqpoint{4.803960in}{4.277068in}}%
\pgfusepath{stroke}%
\end{pgfscope}%
\begin{pgfscope}%
\pgfpathrectangle{\pgfqpoint{0.570343in}{0.331635in}}{\pgfqpoint{9.300000in}{7.700000in}}%
\pgfusepath{clip}%
\pgfsetrectcap%
\pgfsetroundjoin%
\pgfsetlinewidth{1.505625pt}%
\definecolor{currentstroke}{rgb}{0.631373,0.788235,0.956863}%
\pgfsetstrokecolor{currentstroke}%
\pgfsetstrokeopacity{0.800000}%
\pgfsetdash{}{0pt}%
\pgfpathmoveto{\pgfqpoint{5.886021in}{6.276968in}}%
\pgfpathlineto{\pgfqpoint{4.803960in}{4.277068in}}%
\pgfusepath{stroke}%
\end{pgfscope}%
\begin{pgfscope}%
\pgfpathrectangle{\pgfqpoint{0.570343in}{0.331635in}}{\pgfqpoint{9.300000in}{7.700000in}}%
\pgfusepath{clip}%
\pgfsetrectcap%
\pgfsetroundjoin%
\pgfsetlinewidth{1.505625pt}%
\definecolor{currentstroke}{rgb}{0.631373,0.788235,0.956863}%
\pgfsetstrokecolor{currentstroke}%
\pgfsetstrokeopacity{0.800000}%
\pgfsetdash{}{0pt}%
\pgfpathmoveto{\pgfqpoint{6.703980in}{2.135452in}}%
\pgfpathlineto{\pgfqpoint{4.803960in}{4.277068in}}%
\pgfusepath{stroke}%
\end{pgfscope}%
\begin{pgfscope}%
\pgfpathrectangle{\pgfqpoint{0.570343in}{0.331635in}}{\pgfqpoint{9.300000in}{7.700000in}}%
\pgfusepath{clip}%
\pgfsetrectcap%
\pgfsetroundjoin%
\pgfsetlinewidth{1.505625pt}%
\definecolor{currentstroke}{rgb}{0.631373,0.788235,0.956863}%
\pgfsetstrokecolor{currentstroke}%
\pgfsetstrokeopacity{0.800000}%
\pgfsetdash{}{0pt}%
\pgfpathmoveto{\pgfqpoint{3.152829in}{3.811018in}}%
\pgfpathlineto{\pgfqpoint{4.803960in}{4.277068in}}%
\pgfusepath{stroke}%
\end{pgfscope}%
\begin{pgfscope}%
\pgfpathrectangle{\pgfqpoint{0.570343in}{0.331635in}}{\pgfqpoint{9.300000in}{7.700000in}}%
\pgfusepath{clip}%
\pgfsetrectcap%
\pgfsetroundjoin%
\pgfsetlinewidth{1.505625pt}%
\definecolor{currentstroke}{rgb}{0.631373,0.788235,0.956863}%
\pgfsetstrokecolor{currentstroke}%
\pgfsetstrokeopacity{0.800000}%
\pgfsetdash{}{0pt}%
\pgfpathmoveto{\pgfqpoint{4.382416in}{4.635624in}}%
\pgfpathlineto{\pgfqpoint{4.803960in}{4.277068in}}%
\pgfusepath{stroke}%
\end{pgfscope}%
\begin{pgfscope}%
\pgfpathrectangle{\pgfqpoint{0.570343in}{0.331635in}}{\pgfqpoint{9.300000in}{7.700000in}}%
\pgfusepath{clip}%
\pgfsetrectcap%
\pgfsetroundjoin%
\pgfsetlinewidth{1.505625pt}%
\definecolor{currentstroke}{rgb}{0.631373,0.788235,0.956863}%
\pgfsetstrokecolor{currentstroke}%
\pgfsetstrokeopacity{0.800000}%
\pgfsetdash{}{0pt}%
\pgfpathmoveto{\pgfqpoint{6.950477in}{6.304394in}}%
\pgfpathlineto{\pgfqpoint{4.803960in}{4.277068in}}%
\pgfusepath{stroke}%
\end{pgfscope}%
\begin{pgfscope}%
\pgfpathrectangle{\pgfqpoint{0.570343in}{0.331635in}}{\pgfqpoint{9.300000in}{7.700000in}}%
\pgfusepath{clip}%
\pgfsetrectcap%
\pgfsetroundjoin%
\pgfsetlinewidth{1.505625pt}%
\definecolor{currentstroke}{rgb}{0.631373,0.788235,0.956863}%
\pgfsetstrokecolor{currentstroke}%
\pgfsetstrokeopacity{0.800000}%
\pgfsetdash{}{0pt}%
\pgfpathmoveto{\pgfqpoint{2.893408in}{4.648818in}}%
\pgfpathlineto{\pgfqpoint{4.803960in}{4.277068in}}%
\pgfusepath{stroke}%
\end{pgfscope}%
\begin{pgfscope}%
\pgfpathrectangle{\pgfqpoint{0.570343in}{0.331635in}}{\pgfqpoint{9.300000in}{7.700000in}}%
\pgfusepath{clip}%
\pgfsetrectcap%
\pgfsetroundjoin%
\pgfsetlinewidth{1.505625pt}%
\definecolor{currentstroke}{rgb}{1.000000,0.705882,0.509804}%
\pgfsetstrokecolor{currentstroke}%
\pgfsetstrokeopacity{0.800000}%
\pgfsetdash{}{0pt}%
\pgfpathmoveto{\pgfqpoint{3.817222in}{6.024113in}}%
\pgfpathlineto{\pgfqpoint{5.224831in}{3.978419in}}%
\pgfusepath{stroke}%
\end{pgfscope}%
\begin{pgfscope}%
\pgfpathrectangle{\pgfqpoint{0.570343in}{0.331635in}}{\pgfqpoint{9.300000in}{7.700000in}}%
\pgfusepath{clip}%
\pgfsetrectcap%
\pgfsetroundjoin%
\pgfsetlinewidth{1.505625pt}%
\definecolor{currentstroke}{rgb}{1.000000,0.705882,0.509804}%
\pgfsetstrokecolor{currentstroke}%
\pgfsetstrokeopacity{0.800000}%
\pgfsetdash{}{0pt}%
\pgfpathmoveto{\pgfqpoint{7.659561in}{1.386197in}}%
\pgfpathlineto{\pgfqpoint{5.224831in}{3.978419in}}%
\pgfusepath{stroke}%
\end{pgfscope}%
\begin{pgfscope}%
\pgfpathrectangle{\pgfqpoint{0.570343in}{0.331635in}}{\pgfqpoint{9.300000in}{7.700000in}}%
\pgfusepath{clip}%
\pgfsetrectcap%
\pgfsetroundjoin%
\pgfsetlinewidth{1.505625pt}%
\definecolor{currentstroke}{rgb}{1.000000,0.705882,0.509804}%
\pgfsetstrokecolor{currentstroke}%
\pgfsetstrokeopacity{0.800000}%
\pgfsetdash{}{0pt}%
\pgfpathmoveto{\pgfqpoint{6.328447in}{4.305358in}}%
\pgfpathlineto{\pgfqpoint{5.224831in}{3.978419in}}%
\pgfusepath{stroke}%
\end{pgfscope}%
\begin{pgfscope}%
\pgfpathrectangle{\pgfqpoint{0.570343in}{0.331635in}}{\pgfqpoint{9.300000in}{7.700000in}}%
\pgfusepath{clip}%
\pgfsetrectcap%
\pgfsetroundjoin%
\pgfsetlinewidth{1.505625pt}%
\definecolor{currentstroke}{rgb}{1.000000,0.705882,0.509804}%
\pgfsetstrokecolor{currentstroke}%
\pgfsetstrokeopacity{0.800000}%
\pgfsetdash{}{0pt}%
\pgfpathmoveto{\pgfqpoint{8.131383in}{3.754090in}}%
\pgfpathlineto{\pgfqpoint{5.224831in}{3.978419in}}%
\pgfusepath{stroke}%
\end{pgfscope}%
\begin{pgfscope}%
\pgfpathrectangle{\pgfqpoint{0.570343in}{0.331635in}}{\pgfqpoint{9.300000in}{7.700000in}}%
\pgfusepath{clip}%
\pgfsetrectcap%
\pgfsetroundjoin%
\pgfsetlinewidth{1.505625pt}%
\definecolor{currentstroke}{rgb}{1.000000,0.705882,0.509804}%
\pgfsetstrokecolor{currentstroke}%
\pgfsetstrokeopacity{0.800000}%
\pgfsetdash{}{0pt}%
\pgfpathmoveto{\pgfqpoint{6.702968in}{4.905160in}}%
\pgfpathlineto{\pgfqpoint{5.224831in}{3.978419in}}%
\pgfusepath{stroke}%
\end{pgfscope}%
\begin{pgfscope}%
\pgfpathrectangle{\pgfqpoint{0.570343in}{0.331635in}}{\pgfqpoint{9.300000in}{7.700000in}}%
\pgfusepath{clip}%
\pgfsetrectcap%
\pgfsetroundjoin%
\pgfsetlinewidth{1.505625pt}%
\definecolor{currentstroke}{rgb}{1.000000,0.705882,0.509804}%
\pgfsetstrokecolor{currentstroke}%
\pgfsetstrokeopacity{0.800000}%
\pgfsetdash{}{0pt}%
\pgfpathmoveto{\pgfqpoint{5.355699in}{2.068245in}}%
\pgfpathlineto{\pgfqpoint{5.224831in}{3.978419in}}%
\pgfusepath{stroke}%
\end{pgfscope}%
\begin{pgfscope}%
\pgfpathrectangle{\pgfqpoint{0.570343in}{0.331635in}}{\pgfqpoint{9.300000in}{7.700000in}}%
\pgfusepath{clip}%
\pgfsetrectcap%
\pgfsetroundjoin%
\pgfsetlinewidth{1.505625pt}%
\definecolor{currentstroke}{rgb}{1.000000,0.705882,0.509804}%
\pgfsetstrokecolor{currentstroke}%
\pgfsetstrokeopacity{0.800000}%
\pgfsetdash{}{0pt}%
\pgfpathmoveto{\pgfqpoint{2.219532in}{1.487793in}}%
\pgfpathlineto{\pgfqpoint{5.224831in}{3.978419in}}%
\pgfusepath{stroke}%
\end{pgfscope}%
\begin{pgfscope}%
\pgfpathrectangle{\pgfqpoint{0.570343in}{0.331635in}}{\pgfqpoint{9.300000in}{7.700000in}}%
\pgfusepath{clip}%
\pgfsetrectcap%
\pgfsetroundjoin%
\pgfsetlinewidth{1.505625pt}%
\definecolor{currentstroke}{rgb}{1.000000,0.705882,0.509804}%
\pgfsetstrokecolor{currentstroke}%
\pgfsetstrokeopacity{0.800000}%
\pgfsetdash{}{0pt}%
\pgfpathmoveto{\pgfqpoint{4.514146in}{1.447847in}}%
\pgfpathlineto{\pgfqpoint{5.224831in}{3.978419in}}%
\pgfusepath{stroke}%
\end{pgfscope}%
\begin{pgfscope}%
\pgfpathrectangle{\pgfqpoint{0.570343in}{0.331635in}}{\pgfqpoint{9.300000in}{7.700000in}}%
\pgfusepath{clip}%
\pgfsetrectcap%
\pgfsetroundjoin%
\pgfsetlinewidth{1.505625pt}%
\definecolor{currentstroke}{rgb}{1.000000,0.705882,0.509804}%
\pgfsetstrokecolor{currentstroke}%
\pgfsetstrokeopacity{0.800000}%
\pgfsetdash{}{0pt}%
\pgfpathmoveto{\pgfqpoint{5.290381in}{2.873994in}}%
\pgfpathlineto{\pgfqpoint{5.224831in}{3.978419in}}%
\pgfusepath{stroke}%
\end{pgfscope}%
\begin{pgfscope}%
\pgfpathrectangle{\pgfqpoint{0.570343in}{0.331635in}}{\pgfqpoint{9.300000in}{7.700000in}}%
\pgfusepath{clip}%
\pgfsetrectcap%
\pgfsetroundjoin%
\pgfsetlinewidth{1.505625pt}%
\definecolor{currentstroke}{rgb}{1.000000,0.705882,0.509804}%
\pgfsetstrokecolor{currentstroke}%
\pgfsetstrokeopacity{0.800000}%
\pgfsetdash{}{0pt}%
\pgfpathmoveto{\pgfqpoint{4.901182in}{6.223846in}}%
\pgfpathlineto{\pgfqpoint{5.224831in}{3.978419in}}%
\pgfusepath{stroke}%
\end{pgfscope}%
\begin{pgfscope}%
\pgfpathrectangle{\pgfqpoint{0.570343in}{0.331635in}}{\pgfqpoint{9.300000in}{7.700000in}}%
\pgfusepath{clip}%
\pgfsetrectcap%
\pgfsetroundjoin%
\pgfsetlinewidth{1.505625pt}%
\definecolor{currentstroke}{rgb}{1.000000,0.705882,0.509804}%
\pgfsetstrokecolor{currentstroke}%
\pgfsetstrokeopacity{0.800000}%
\pgfsetdash{}{0pt}%
\pgfpathmoveto{\pgfqpoint{8.177594in}{2.927999in}}%
\pgfpathlineto{\pgfqpoint{5.224831in}{3.978419in}}%
\pgfusepath{stroke}%
\end{pgfscope}%
\begin{pgfscope}%
\pgfpathrectangle{\pgfqpoint{0.570343in}{0.331635in}}{\pgfqpoint{9.300000in}{7.700000in}}%
\pgfusepath{clip}%
\pgfsetrectcap%
\pgfsetroundjoin%
\pgfsetlinewidth{1.505625pt}%
\definecolor{currentstroke}{rgb}{1.000000,0.705882,0.509804}%
\pgfsetstrokecolor{currentstroke}%
\pgfsetstrokeopacity{0.800000}%
\pgfsetdash{}{0pt}%
\pgfpathmoveto{\pgfqpoint{7.239046in}{7.681635in}}%
\pgfpathlineto{\pgfqpoint{5.224831in}{3.978419in}}%
\pgfusepath{stroke}%
\end{pgfscope}%
\begin{pgfscope}%
\pgfpathrectangle{\pgfqpoint{0.570343in}{0.331635in}}{\pgfqpoint{9.300000in}{7.700000in}}%
\pgfusepath{clip}%
\pgfsetrectcap%
\pgfsetroundjoin%
\pgfsetlinewidth{1.505625pt}%
\definecolor{currentstroke}{rgb}{1.000000,0.705882,0.509804}%
\pgfsetstrokecolor{currentstroke}%
\pgfsetstrokeopacity{0.800000}%
\pgfsetdash{}{0pt}%
\pgfpathmoveto{\pgfqpoint{3.182112in}{0.681635in}}%
\pgfpathlineto{\pgfqpoint{5.224831in}{3.978419in}}%
\pgfusepath{stroke}%
\end{pgfscope}%
\begin{pgfscope}%
\pgfpathrectangle{\pgfqpoint{0.570343in}{0.331635in}}{\pgfqpoint{9.300000in}{7.700000in}}%
\pgfusepath{clip}%
\pgfsetrectcap%
\pgfsetroundjoin%
\pgfsetlinewidth{1.505625pt}%
\definecolor{currentstroke}{rgb}{1.000000,0.705882,0.509804}%
\pgfsetstrokecolor{currentstroke}%
\pgfsetstrokeopacity{0.800000}%
\pgfsetdash{}{0pt}%
\pgfpathmoveto{\pgfqpoint{7.238874in}{5.471329in}}%
\pgfpathlineto{\pgfqpoint{5.224831in}{3.978419in}}%
\pgfusepath{stroke}%
\end{pgfscope}%
\begin{pgfscope}%
\pgfpathrectangle{\pgfqpoint{0.570343in}{0.331635in}}{\pgfqpoint{9.300000in}{7.700000in}}%
\pgfusepath{clip}%
\pgfsetrectcap%
\pgfsetroundjoin%
\pgfsetlinewidth{1.505625pt}%
\definecolor{currentstroke}{rgb}{1.000000,0.705882,0.509804}%
\pgfsetstrokecolor{currentstroke}%
\pgfsetstrokeopacity{0.800000}%
\pgfsetdash{}{0pt}%
\pgfpathmoveto{\pgfqpoint{3.375695in}{1.918885in}}%
\pgfpathlineto{\pgfqpoint{5.224831in}{3.978419in}}%
\pgfusepath{stroke}%
\end{pgfscope}%
\begin{pgfscope}%
\pgfpathrectangle{\pgfqpoint{0.570343in}{0.331635in}}{\pgfqpoint{9.300000in}{7.700000in}}%
\pgfusepath{clip}%
\pgfsetrectcap%
\pgfsetroundjoin%
\pgfsetlinewidth{1.505625pt}%
\definecolor{currentstroke}{rgb}{1.000000,0.705882,0.509804}%
\pgfsetstrokecolor{currentstroke}%
\pgfsetstrokeopacity{0.800000}%
\pgfsetdash{}{0pt}%
\pgfpathmoveto{\pgfqpoint{5.763337in}{1.311076in}}%
\pgfpathlineto{\pgfqpoint{5.224831in}{3.978419in}}%
\pgfusepath{stroke}%
\end{pgfscope}%
\begin{pgfscope}%
\pgfpathrectangle{\pgfqpoint{0.570343in}{0.331635in}}{\pgfqpoint{9.300000in}{7.700000in}}%
\pgfusepath{clip}%
\pgfsetrectcap%
\pgfsetroundjoin%
\pgfsetlinewidth{1.505625pt}%
\definecolor{currentstroke}{rgb}{1.000000,0.705882,0.509804}%
\pgfsetstrokecolor{currentstroke}%
\pgfsetstrokeopacity{0.800000}%
\pgfsetdash{}{0pt}%
\pgfpathmoveto{\pgfqpoint{2.235429in}{2.335141in}}%
\pgfpathlineto{\pgfqpoint{5.224831in}{3.978419in}}%
\pgfusepath{stroke}%
\end{pgfscope}%
\begin{pgfscope}%
\pgfpathrectangle{\pgfqpoint{0.570343in}{0.331635in}}{\pgfqpoint{9.300000in}{7.700000in}}%
\pgfusepath{clip}%
\pgfsetrectcap%
\pgfsetroundjoin%
\pgfsetlinewidth{1.505625pt}%
\definecolor{currentstroke}{rgb}{1.000000,0.705882,0.509804}%
\pgfsetstrokecolor{currentstroke}%
\pgfsetstrokeopacity{0.800000}%
\pgfsetdash{}{0pt}%
\pgfpathmoveto{\pgfqpoint{8.228735in}{6.567215in}}%
\pgfpathlineto{\pgfqpoint{5.224831in}{3.978419in}}%
\pgfusepath{stroke}%
\end{pgfscope}%
\begin{pgfscope}%
\pgfpathrectangle{\pgfqpoint{0.570343in}{0.331635in}}{\pgfqpoint{9.300000in}{7.700000in}}%
\pgfusepath{clip}%
\pgfsetrectcap%
\pgfsetroundjoin%
\pgfsetlinewidth{1.505625pt}%
\definecolor{currentstroke}{rgb}{1.000000,0.705882,0.509804}%
\pgfsetstrokecolor{currentstroke}%
\pgfsetstrokeopacity{0.800000}%
\pgfsetdash{}{0pt}%
\pgfpathmoveto{\pgfqpoint{4.520708in}{7.423071in}}%
\pgfpathlineto{\pgfqpoint{5.224831in}{3.978419in}}%
\pgfusepath{stroke}%
\end{pgfscope}%
\begin{pgfscope}%
\pgfpathrectangle{\pgfqpoint{0.570343in}{0.331635in}}{\pgfqpoint{9.300000in}{7.700000in}}%
\pgfusepath{clip}%
\pgfsetrectcap%
\pgfsetroundjoin%
\pgfsetlinewidth{1.505625pt}%
\definecolor{currentstroke}{rgb}{1.000000,0.705882,0.509804}%
\pgfsetstrokecolor{currentstroke}%
\pgfsetstrokeopacity{0.800000}%
\pgfsetdash{}{0pt}%
\pgfpathmoveto{\pgfqpoint{5.816987in}{7.332432in}}%
\pgfpathlineto{\pgfqpoint{5.224831in}{3.978419in}}%
\pgfusepath{stroke}%
\end{pgfscope}%
\begin{pgfscope}%
\pgfpathrectangle{\pgfqpoint{0.570343in}{0.331635in}}{\pgfqpoint{9.300000in}{7.700000in}}%
\pgfusepath{clip}%
\pgfsetrectcap%
\pgfsetroundjoin%
\pgfsetlinewidth{1.505625pt}%
\definecolor{currentstroke}{rgb}{1.000000,0.705882,0.509804}%
\pgfsetstrokecolor{currentstroke}%
\pgfsetstrokeopacity{0.800000}%
\pgfsetdash{}{0pt}%
\pgfpathmoveto{\pgfqpoint{1.860590in}{5.284247in}}%
\pgfpathlineto{\pgfqpoint{5.224831in}{3.978419in}}%
\pgfusepath{stroke}%
\end{pgfscope}%
\begin{pgfscope}%
\pgfpathrectangle{\pgfqpoint{0.570343in}{0.331635in}}{\pgfqpoint{9.300000in}{7.700000in}}%
\pgfusepath{clip}%
\pgfsetrectcap%
\pgfsetroundjoin%
\pgfsetlinewidth{1.505625pt}%
\definecolor{currentstroke}{rgb}{1.000000,0.705882,0.509804}%
\pgfsetstrokecolor{currentstroke}%
\pgfsetstrokeopacity{0.800000}%
\pgfsetdash{}{0pt}%
\pgfpathmoveto{\pgfqpoint{7.102567in}{4.141837in}}%
\pgfpathlineto{\pgfqpoint{5.224831in}{3.978419in}}%
\pgfusepath{stroke}%
\end{pgfscope}%
\begin{pgfscope}%
\pgfpathrectangle{\pgfqpoint{0.570343in}{0.331635in}}{\pgfqpoint{9.300000in}{7.700000in}}%
\pgfusepath{clip}%
\pgfsetrectcap%
\pgfsetroundjoin%
\pgfsetlinewidth{1.505625pt}%
\definecolor{currentstroke}{rgb}{1.000000,0.705882,0.509804}%
\pgfsetstrokecolor{currentstroke}%
\pgfsetstrokeopacity{0.800000}%
\pgfsetdash{}{0pt}%
\pgfpathmoveto{\pgfqpoint{1.082655in}{2.191589in}}%
\pgfpathlineto{\pgfqpoint{5.224831in}{3.978419in}}%
\pgfusepath{stroke}%
\end{pgfscope}%
\begin{pgfscope}%
\pgfpathrectangle{\pgfqpoint{0.570343in}{0.331635in}}{\pgfqpoint{9.300000in}{7.700000in}}%
\pgfusepath{clip}%
\pgfsetrectcap%
\pgfsetroundjoin%
\pgfsetlinewidth{1.505625pt}%
\definecolor{currentstroke}{rgb}{1.000000,0.705882,0.509804}%
\pgfsetstrokecolor{currentstroke}%
\pgfsetstrokeopacity{0.800000}%
\pgfsetdash{}{0pt}%
\pgfpathmoveto{\pgfqpoint{9.447616in}{3.338121in}}%
\pgfpathlineto{\pgfqpoint{5.224831in}{3.978419in}}%
\pgfusepath{stroke}%
\end{pgfscope}%
\begin{pgfscope}%
\pgfpathrectangle{\pgfqpoint{0.570343in}{0.331635in}}{\pgfqpoint{9.300000in}{7.700000in}}%
\pgfusepath{clip}%
\pgfsetrectcap%
\pgfsetroundjoin%
\pgfsetlinewidth{1.505625pt}%
\definecolor{currentstroke}{rgb}{1.000000,0.705882,0.509804}%
\pgfsetstrokecolor{currentstroke}%
\pgfsetstrokeopacity{0.800000}%
\pgfsetdash{}{0pt}%
\pgfpathmoveto{\pgfqpoint{3.829395in}{5.193151in}}%
\pgfpathlineto{\pgfqpoint{5.224831in}{3.978419in}}%
\pgfusepath{stroke}%
\end{pgfscope}%
\begin{pgfscope}%
\pgfpathrectangle{\pgfqpoint{0.570343in}{0.331635in}}{\pgfqpoint{9.300000in}{7.700000in}}%
\pgfusepath{clip}%
\pgfsetrectcap%
\pgfsetroundjoin%
\pgfsetlinewidth{1.505625pt}%
\definecolor{currentstroke}{rgb}{1.000000,0.705882,0.509804}%
\pgfsetstrokecolor{currentstroke}%
\pgfsetstrokeopacity{0.800000}%
\pgfsetdash{}{0pt}%
\pgfpathmoveto{\pgfqpoint{2.176389in}{4.216423in}}%
\pgfpathlineto{\pgfqpoint{5.224831in}{3.978419in}}%
\pgfusepath{stroke}%
\end{pgfscope}%
\begin{pgfscope}%
\pgfpathrectangle{\pgfqpoint{0.570343in}{0.331635in}}{\pgfqpoint{9.300000in}{7.700000in}}%
\pgfusepath{clip}%
\pgfsetrectcap%
\pgfsetroundjoin%
\pgfsetlinewidth{1.505625pt}%
\definecolor{currentstroke}{rgb}{1.000000,0.705882,0.509804}%
\pgfsetstrokecolor{currentstroke}%
\pgfsetstrokeopacity{0.800000}%
\pgfsetdash{}{0pt}%
\pgfpathmoveto{\pgfqpoint{7.119633in}{3.398240in}}%
\pgfpathlineto{\pgfqpoint{5.224831in}{3.978419in}}%
\pgfusepath{stroke}%
\end{pgfscope}%
\begin{pgfscope}%
\pgfpathrectangle{\pgfqpoint{0.570343in}{0.331635in}}{\pgfqpoint{9.300000in}{7.700000in}}%
\pgfusepath{clip}%
\pgfsetrectcap%
\pgfsetroundjoin%
\pgfsetlinewidth{1.505625pt}%
\definecolor{currentstroke}{rgb}{1.000000,0.705882,0.509804}%
\pgfsetstrokecolor{currentstroke}%
\pgfsetstrokeopacity{0.800000}%
\pgfsetdash{}{0pt}%
\pgfpathmoveto{\pgfqpoint{2.977371in}{5.505063in}}%
\pgfpathlineto{\pgfqpoint{5.224831in}{3.978419in}}%
\pgfusepath{stroke}%
\end{pgfscope}%
\begin{pgfscope}%
\pgfsetrectcap%
\pgfsetmiterjoin%
\pgfsetlinewidth{0.803000pt}%
\definecolor{currentstroke}{rgb}{0.000000,0.000000,0.000000}%
\pgfsetstrokecolor{currentstroke}%
\pgfsetdash{}{0pt}%
\pgfpathmoveto{\pgfqpoint{0.570343in}{0.331635in}}%
\pgfpathlineto{\pgfqpoint{0.570343in}{8.031635in}}%
\pgfusepath{stroke}%
\end{pgfscope}%
\begin{pgfscope}%
\pgfsetrectcap%
\pgfsetmiterjoin%
\pgfsetlinewidth{0.803000pt}%
\definecolor{currentstroke}{rgb}{0.000000,0.000000,0.000000}%
\pgfsetstrokecolor{currentstroke}%
\pgfsetdash{}{0pt}%
\pgfpathmoveto{\pgfqpoint{9.870343in}{0.331635in}}%
\pgfpathlineto{\pgfqpoint{9.870343in}{8.031635in}}%
\pgfusepath{stroke}%
\end{pgfscope}%
\begin{pgfscope}%
\pgfsetrectcap%
\pgfsetmiterjoin%
\pgfsetlinewidth{0.803000pt}%
\definecolor{currentstroke}{rgb}{0.000000,0.000000,0.000000}%
\pgfsetstrokecolor{currentstroke}%
\pgfsetdash{}{0pt}%
\pgfpathmoveto{\pgfqpoint{0.570343in}{0.331635in}}%
\pgfpathlineto{\pgfqpoint{9.870343in}{0.331635in}}%
\pgfusepath{stroke}%
\end{pgfscope}%
\begin{pgfscope}%
\pgfsetrectcap%
\pgfsetmiterjoin%
\pgfsetlinewidth{0.803000pt}%
\definecolor{currentstroke}{rgb}{0.000000,0.000000,0.000000}%
\pgfsetstrokecolor{currentstroke}%
\pgfsetdash{}{0pt}%
\pgfpathmoveto{\pgfqpoint{0.570343in}{8.031635in}}%
\pgfpathlineto{\pgfqpoint{9.870343in}{8.031635in}}%
\pgfusepath{stroke}%
\end{pgfscope}%
\begin{pgfscope}%
\definecolor{textcolor}{rgb}{0.000000,0.000000,0.000000}%
\pgfsetstrokecolor{textcolor}%
\pgfsetfillcolor{textcolor}%
\pgftext[x=5.220343in,y=8.114968in,,base]{\color{textcolor}\sffamily\fontsize{12.000000}{14.400000}\selectfont Photo-Realistic Images}%
\end{pgfscope}%
\begin{pgfscope}%
\pgfsetbuttcap%
\pgfsetmiterjoin%
\definecolor{currentfill}{rgb}{1.000000,1.000000,1.000000}%
\pgfsetfillcolor{currentfill}%
\pgfsetfillopacity{0.800000}%
\pgfsetlinewidth{1.003750pt}%
\definecolor{currentstroke}{rgb}{0.800000,0.800000,0.800000}%
\pgfsetstrokecolor{currentstroke}%
\pgfsetstrokeopacity{0.800000}%
\pgfsetdash{}{0pt}%
\pgfpathmoveto{\pgfqpoint{9.967566in}{3.956944in}}%
\pgfpathlineto{\pgfqpoint{11.246496in}{3.956944in}}%
\pgfpathquadraticcurveto{\pgfqpoint{11.274274in}{3.956944in}}{\pgfqpoint{11.274274in}{3.984722in}}%
\pgfpathlineto{\pgfqpoint{11.274274in}{4.378548in}}%
\pgfpathquadraticcurveto{\pgfqpoint{11.274274in}{4.406326in}}{\pgfqpoint{11.246496in}{4.406326in}}%
\pgfpathlineto{\pgfqpoint{9.967566in}{4.406326in}}%
\pgfpathquadraticcurveto{\pgfqpoint{9.939788in}{4.406326in}}{\pgfqpoint{9.939788in}{4.378548in}}%
\pgfpathlineto{\pgfqpoint{9.939788in}{3.984722in}}%
\pgfpathquadraticcurveto{\pgfqpoint{9.939788in}{3.956944in}}{\pgfqpoint{9.967566in}{3.956944in}}%
\pgfpathclose%
\pgfusepath{stroke,fill}%
\end{pgfscope}%
\begin{pgfscope}%
\pgfsetbuttcap%
\pgfsetroundjoin%
\definecolor{currentfill}{rgb}{0.631373,0.788235,0.956863}%
\pgfsetfillcolor{currentfill}%
\pgfsetlinewidth{1.003750pt}%
\definecolor{currentstroke}{rgb}{0.631373,0.788235,0.956863}%
\pgfsetstrokecolor{currentstroke}%
\pgfsetdash{}{0pt}%
\pgfsys@defobject{currentmarker}{\pgfqpoint{-0.041667in}{-0.041667in}}{\pgfqpoint{0.041667in}{0.041667in}}{%
\pgfpathmoveto{\pgfqpoint{0.000000in}{-0.041667in}}%
\pgfpathcurveto{\pgfqpoint{0.011050in}{-0.041667in}}{\pgfqpoint{0.021649in}{-0.037276in}}{\pgfqpoint{0.029463in}{-0.029463in}}%
\pgfpathcurveto{\pgfqpoint{0.037276in}{-0.021649in}}{\pgfqpoint{0.041667in}{-0.011050in}}{\pgfqpoint{0.041667in}{0.000000in}}%
\pgfpathcurveto{\pgfqpoint{0.041667in}{0.011050in}}{\pgfqpoint{0.037276in}{0.021649in}}{\pgfqpoint{0.029463in}{0.029463in}}%
\pgfpathcurveto{\pgfqpoint{0.021649in}{0.037276in}}{\pgfqpoint{0.011050in}{0.041667in}}{\pgfqpoint{0.000000in}{0.041667in}}%
\pgfpathcurveto{\pgfqpoint{-0.011050in}{0.041667in}}{\pgfqpoint{-0.021649in}{0.037276in}}{\pgfqpoint{-0.029463in}{0.029463in}}%
\pgfpathcurveto{\pgfqpoint{-0.037276in}{0.021649in}}{\pgfqpoint{-0.041667in}{0.011050in}}{\pgfqpoint{-0.041667in}{0.000000in}}%
\pgfpathcurveto{\pgfqpoint{-0.041667in}{-0.011050in}}{\pgfqpoint{-0.037276in}{-0.021649in}}{\pgfqpoint{-0.029463in}{-0.029463in}}%
\pgfpathcurveto{\pgfqpoint{-0.021649in}{-0.037276in}}{\pgfqpoint{-0.011050in}{-0.041667in}}{\pgfqpoint{0.000000in}{-0.041667in}}%
\pgfpathclose%
\pgfusepath{stroke,fill}%
}%
\begin{pgfscope}%
\pgfsys@transformshift{10.134232in}{4.281705in}%
\pgfsys@useobject{currentmarker}{}%
\end{pgfscope}%
\end{pgfscope}%
\begin{pgfscope}%
\definecolor{textcolor}{rgb}{0.000000,0.000000,0.000000}%
\pgfsetstrokecolor{textcolor}%
\pgfsetfillcolor{textcolor}%
\pgftext[x=10.384232in,y=4.245247in,left,base]{\color{textcolor}\sffamily\fontsize{10.000000}{12.000000}\selectfont blenderproc}%
\end{pgfscope}%
\begin{pgfscope}%
\pgfsetbuttcap%
\pgfsetroundjoin%
\definecolor{currentfill}{rgb}{1.000000,0.705882,0.509804}%
\pgfsetfillcolor{currentfill}%
\pgfsetlinewidth{1.003750pt}%
\definecolor{currentstroke}{rgb}{1.000000,0.705882,0.509804}%
\pgfsetstrokecolor{currentstroke}%
\pgfsetdash{}{0pt}%
\pgfsys@defobject{currentmarker}{\pgfqpoint{-0.041667in}{-0.041667in}}{\pgfqpoint{0.041667in}{0.041667in}}{%
\pgfpathmoveto{\pgfqpoint{0.000000in}{-0.041667in}}%
\pgfpathcurveto{\pgfqpoint{0.011050in}{-0.041667in}}{\pgfqpoint{0.021649in}{-0.037276in}}{\pgfqpoint{0.029463in}{-0.029463in}}%
\pgfpathcurveto{\pgfqpoint{0.037276in}{-0.021649in}}{\pgfqpoint{0.041667in}{-0.011050in}}{\pgfqpoint{0.041667in}{0.000000in}}%
\pgfpathcurveto{\pgfqpoint{0.041667in}{0.011050in}}{\pgfqpoint{0.037276in}{0.021649in}}{\pgfqpoint{0.029463in}{0.029463in}}%
\pgfpathcurveto{\pgfqpoint{0.021649in}{0.037276in}}{\pgfqpoint{0.011050in}{0.041667in}}{\pgfqpoint{0.000000in}{0.041667in}}%
\pgfpathcurveto{\pgfqpoint{-0.011050in}{0.041667in}}{\pgfqpoint{-0.021649in}{0.037276in}}{\pgfqpoint{-0.029463in}{0.029463in}}%
\pgfpathcurveto{\pgfqpoint{-0.037276in}{0.021649in}}{\pgfqpoint{-0.041667in}{0.011050in}}{\pgfqpoint{-0.041667in}{0.000000in}}%
\pgfpathcurveto{\pgfqpoint{-0.041667in}{-0.011050in}}{\pgfqpoint{-0.037276in}{-0.021649in}}{\pgfqpoint{-0.029463in}{-0.029463in}}%
\pgfpathcurveto{\pgfqpoint{-0.021649in}{-0.037276in}}{\pgfqpoint{-0.011050in}{-0.041667in}}{\pgfqpoint{0.000000in}{-0.041667in}}%
\pgfpathclose%
\pgfusepath{stroke,fill}%
}%
\begin{pgfscope}%
\pgfsys@transformshift{10.134232in}{4.077848in}%
\pgfsys@useobject{currentmarker}{}%
\end{pgfscope}%
\end{pgfscope}%
\begin{pgfscope}%
\definecolor{textcolor}{rgb}{0.000000,0.000000,0.000000}%
\pgfsetstrokecolor{textcolor}%
\pgfsetfillcolor{textcolor}%
\pgftext[x=10.384232in,y=4.041390in,left,base]{\color{textcolor}\sffamily\fontsize{10.000000}{12.000000}\selectfont pix3d}%
\end{pgfscope}%
\end{pgfpicture}%
\makeatother%
\endgroup%
}
    \resizebox{0.49\linewidth}{5cm}{%% Creator: Matplotlib, PGF backend
%%
%% To include the figure in your LaTeX document, write
%%   \input{<filename>.pgf}
%%
%% Make sure the required packages are loaded in your preamble
%%   \usepackage{pgf}
%%
%% Figures using additional raster images can only be included by \input if
%% they are in the same directory as the main LaTeX file. For loading figures
%% from other directories you can use the `import` package
%%   \usepackage{import}
%%
%% and then include the figures with
%%   \import{<path to file>}{<filename>.pgf}
%%
%% Matplotlib used the following preamble
%%   \usepackage{fontspec}
%%   \setmainfont{DejaVuSerif.ttf}[Path=\detokenize{/Users/apple/opt/anaconda3/envs/kaolin/lib/python3.7/site-packages/matplotlib/mpl-data/fonts/ttf/}]
%%   \setsansfont{DejaVuSans.ttf}[Path=\detokenize{/Users/apple/opt/anaconda3/envs/kaolin/lib/python3.7/site-packages/matplotlib/mpl-data/fonts/ttf/}]
%%   \setmonofont{DejaVuSansMono.ttf}[Path=\detokenize{/Users/apple/opt/anaconda3/envs/kaolin/lib/python3.7/site-packages/matplotlib/mpl-data/fonts/ttf/}]
%%
\begingroup%
\makeatletter%
\begin{pgfpicture}%
\pgfpathrectangle{\pgfpointorigin}{\pgfqpoint{11.036411in}{8.341596in}}%
\pgfusepath{use as bounding box, clip}%
\begin{pgfscope}%
\pgfsetbuttcap%
\pgfsetmiterjoin%
\definecolor{currentfill}{rgb}{1.000000,1.000000,1.000000}%
\pgfsetfillcolor{currentfill}%
\pgfsetlinewidth{0.000000pt}%
\definecolor{currentstroke}{rgb}{1.000000,1.000000,1.000000}%
\pgfsetstrokecolor{currentstroke}%
\pgfsetdash{}{0pt}%
\pgfpathmoveto{\pgfqpoint{0.000000in}{0.000000in}}%
\pgfpathlineto{\pgfqpoint{11.036411in}{0.000000in}}%
\pgfpathlineto{\pgfqpoint{11.036411in}{8.341596in}}%
\pgfpathlineto{\pgfqpoint{0.000000in}{8.341596in}}%
\pgfpathclose%
\pgfusepath{fill}%
\end{pgfscope}%
\begin{pgfscope}%
\pgfsetbuttcap%
\pgfsetmiterjoin%
\definecolor{currentfill}{rgb}{1.000000,1.000000,1.000000}%
\pgfsetfillcolor{currentfill}%
\pgfsetlinewidth{0.000000pt}%
\definecolor{currentstroke}{rgb}{0.000000,0.000000,0.000000}%
\pgfsetstrokecolor{currentstroke}%
\pgfsetstrokeopacity{0.000000}%
\pgfsetdash{}{0pt}%
\pgfpathmoveto{\pgfqpoint{0.570343in}{0.331635in}}%
\pgfpathlineto{\pgfqpoint{9.870343in}{0.331635in}}%
\pgfpathlineto{\pgfqpoint{9.870343in}{8.031635in}}%
\pgfpathlineto{\pgfqpoint{0.570343in}{8.031635in}}%
\pgfpathclose%
\pgfusepath{fill}%
\end{pgfscope}%
\begin{pgfscope}%
\pgfpathrectangle{\pgfqpoint{0.570343in}{0.331635in}}{\pgfqpoint{9.300000in}{7.700000in}}%
\pgfusepath{clip}%
\pgfsetbuttcap%
\pgfsetroundjoin%
\definecolor{currentfill}{rgb}{0.631373,0.788235,0.956863}%
\pgfsetfillcolor{currentfill}%
\pgfsetlinewidth{0.481800pt}%
\definecolor{currentstroke}{rgb}{1.000000,1.000000,1.000000}%
\pgfsetstrokecolor{currentstroke}%
\pgfsetdash{}{0pt}%
\pgfpathmoveto{\pgfqpoint{2.375246in}{6.898172in}}%
\pgfpathcurveto{\pgfqpoint{2.386296in}{6.898172in}}{\pgfqpoint{2.396896in}{6.902562in}}{\pgfqpoint{2.404709in}{6.910376in}}%
\pgfpathcurveto{\pgfqpoint{2.412523in}{6.918190in}}{\pgfqpoint{2.416913in}{6.928789in}}{\pgfqpoint{2.416913in}{6.939839in}}%
\pgfpathcurveto{\pgfqpoint{2.416913in}{6.950889in}}{\pgfqpoint{2.412523in}{6.961488in}}{\pgfqpoint{2.404709in}{6.969302in}}%
\pgfpathcurveto{\pgfqpoint{2.396896in}{6.977115in}}{\pgfqpoint{2.386296in}{6.981506in}}{\pgfqpoint{2.375246in}{6.981506in}}%
\pgfpathcurveto{\pgfqpoint{2.364196in}{6.981506in}}{\pgfqpoint{2.353597in}{6.977115in}}{\pgfqpoint{2.345784in}{6.969302in}}%
\pgfpathcurveto{\pgfqpoint{2.337970in}{6.961488in}}{\pgfqpoint{2.333580in}{6.950889in}}{\pgfqpoint{2.333580in}{6.939839in}}%
\pgfpathcurveto{\pgfqpoint{2.333580in}{6.928789in}}{\pgfqpoint{2.337970in}{6.918190in}}{\pgfqpoint{2.345784in}{6.910376in}}%
\pgfpathcurveto{\pgfqpoint{2.353597in}{6.902562in}}{\pgfqpoint{2.364196in}{6.898172in}}{\pgfqpoint{2.375246in}{6.898172in}}%
\pgfpathclose%
\pgfusepath{stroke,fill}%
\end{pgfscope}%
\begin{pgfscope}%
\pgfpathrectangle{\pgfqpoint{0.570343in}{0.331635in}}{\pgfqpoint{9.300000in}{7.700000in}}%
\pgfusepath{clip}%
\pgfsetbuttcap%
\pgfsetroundjoin%
\definecolor{currentfill}{rgb}{0.631373,0.788235,0.956863}%
\pgfsetfillcolor{currentfill}%
\pgfsetlinewidth{0.481800pt}%
\definecolor{currentstroke}{rgb}{1.000000,1.000000,1.000000}%
\pgfsetstrokecolor{currentstroke}%
\pgfsetdash{}{0pt}%
\pgfpathmoveto{\pgfqpoint{3.492960in}{4.419701in}}%
\pgfpathcurveto{\pgfqpoint{3.504010in}{4.419701in}}{\pgfqpoint{3.514609in}{4.424091in}}{\pgfqpoint{3.522422in}{4.431905in}}%
\pgfpathcurveto{\pgfqpoint{3.530236in}{4.439718in}}{\pgfqpoint{3.534626in}{4.450317in}}{\pgfqpoint{3.534626in}{4.461367in}}%
\pgfpathcurveto{\pgfqpoint{3.534626in}{4.472418in}}{\pgfqpoint{3.530236in}{4.483017in}}{\pgfqpoint{3.522422in}{4.490830in}}%
\pgfpathcurveto{\pgfqpoint{3.514609in}{4.498644in}}{\pgfqpoint{3.504010in}{4.503034in}}{\pgfqpoint{3.492960in}{4.503034in}}%
\pgfpathcurveto{\pgfqpoint{3.481909in}{4.503034in}}{\pgfqpoint{3.471310in}{4.498644in}}{\pgfqpoint{3.463497in}{4.490830in}}%
\pgfpathcurveto{\pgfqpoint{3.455683in}{4.483017in}}{\pgfqpoint{3.451293in}{4.472418in}}{\pgfqpoint{3.451293in}{4.461367in}}%
\pgfpathcurveto{\pgfqpoint{3.451293in}{4.450317in}}{\pgfqpoint{3.455683in}{4.439718in}}{\pgfqpoint{3.463497in}{4.431905in}}%
\pgfpathcurveto{\pgfqpoint{3.471310in}{4.424091in}}{\pgfqpoint{3.481909in}{4.419701in}}{\pgfqpoint{3.492960in}{4.419701in}}%
\pgfpathclose%
\pgfusepath{stroke,fill}%
\end{pgfscope}%
\begin{pgfscope}%
\pgfpathrectangle{\pgfqpoint{0.570343in}{0.331635in}}{\pgfqpoint{9.300000in}{7.700000in}}%
\pgfusepath{clip}%
\pgfsetbuttcap%
\pgfsetroundjoin%
\definecolor{currentfill}{rgb}{0.631373,0.788235,0.956863}%
\pgfsetfillcolor{currentfill}%
\pgfsetlinewidth{0.481800pt}%
\definecolor{currentstroke}{rgb}{1.000000,1.000000,1.000000}%
\pgfsetstrokecolor{currentstroke}%
\pgfsetdash{}{0pt}%
\pgfpathmoveto{\pgfqpoint{5.673563in}{5.030838in}}%
\pgfpathcurveto{\pgfqpoint{5.684613in}{5.030838in}}{\pgfqpoint{5.695212in}{5.035228in}}{\pgfqpoint{5.703026in}{5.043042in}}%
\pgfpathcurveto{\pgfqpoint{5.710839in}{5.050856in}}{\pgfqpoint{5.715229in}{5.061455in}}{\pgfqpoint{5.715229in}{5.072505in}}%
\pgfpathcurveto{\pgfqpoint{5.715229in}{5.083555in}}{\pgfqpoint{5.710839in}{5.094154in}}{\pgfqpoint{5.703026in}{5.101967in}}%
\pgfpathcurveto{\pgfqpoint{5.695212in}{5.109781in}}{\pgfqpoint{5.684613in}{5.114171in}}{\pgfqpoint{5.673563in}{5.114171in}}%
\pgfpathcurveto{\pgfqpoint{5.662513in}{5.114171in}}{\pgfqpoint{5.651914in}{5.109781in}}{\pgfqpoint{5.644100in}{5.101967in}}%
\pgfpathcurveto{\pgfqpoint{5.636286in}{5.094154in}}{\pgfqpoint{5.631896in}{5.083555in}}{\pgfqpoint{5.631896in}{5.072505in}}%
\pgfpathcurveto{\pgfqpoint{5.631896in}{5.061455in}}{\pgfqpoint{5.636286in}{5.050856in}}{\pgfqpoint{5.644100in}{5.043042in}}%
\pgfpathcurveto{\pgfqpoint{5.651914in}{5.035228in}}{\pgfqpoint{5.662513in}{5.030838in}}{\pgfqpoint{5.673563in}{5.030838in}}%
\pgfpathclose%
\pgfusepath{stroke,fill}%
\end{pgfscope}%
\begin{pgfscope}%
\pgfpathrectangle{\pgfqpoint{0.570343in}{0.331635in}}{\pgfqpoint{9.300000in}{7.700000in}}%
\pgfusepath{clip}%
\pgfsetbuttcap%
\pgfsetroundjoin%
\definecolor{currentfill}{rgb}{0.631373,0.788235,0.956863}%
\pgfsetfillcolor{currentfill}%
\pgfsetlinewidth{0.481800pt}%
\definecolor{currentstroke}{rgb}{1.000000,1.000000,1.000000}%
\pgfsetstrokecolor{currentstroke}%
\pgfsetdash{}{0pt}%
\pgfpathmoveto{\pgfqpoint{3.420050in}{0.639968in}}%
\pgfpathcurveto{\pgfqpoint{3.431100in}{0.639968in}}{\pgfqpoint{3.441699in}{0.644359in}}{\pgfqpoint{3.449513in}{0.652172in}}%
\pgfpathcurveto{\pgfqpoint{3.457326in}{0.659986in}}{\pgfqpoint{3.461717in}{0.670585in}}{\pgfqpoint{3.461717in}{0.681635in}}%
\pgfpathcurveto{\pgfqpoint{3.461717in}{0.692685in}}{\pgfqpoint{3.457326in}{0.703284in}}{\pgfqpoint{3.449513in}{0.711098in}}%
\pgfpathcurveto{\pgfqpoint{3.441699in}{0.718911in}}{\pgfqpoint{3.431100in}{0.723302in}}{\pgfqpoint{3.420050in}{0.723302in}}%
\pgfpathcurveto{\pgfqpoint{3.409000in}{0.723302in}}{\pgfqpoint{3.398401in}{0.718911in}}{\pgfqpoint{3.390587in}{0.711098in}}%
\pgfpathcurveto{\pgfqpoint{3.382774in}{0.703284in}}{\pgfqpoint{3.378383in}{0.692685in}}{\pgfqpoint{3.378383in}{0.681635in}}%
\pgfpathcurveto{\pgfqpoint{3.378383in}{0.670585in}}{\pgfqpoint{3.382774in}{0.659986in}}{\pgfqpoint{3.390587in}{0.652172in}}%
\pgfpathcurveto{\pgfqpoint{3.398401in}{0.644359in}}{\pgfqpoint{3.409000in}{0.639968in}}{\pgfqpoint{3.420050in}{0.639968in}}%
\pgfpathclose%
\pgfusepath{stroke,fill}%
\end{pgfscope}%
\begin{pgfscope}%
\pgfpathrectangle{\pgfqpoint{0.570343in}{0.331635in}}{\pgfqpoint{9.300000in}{7.700000in}}%
\pgfusepath{clip}%
\pgfsetbuttcap%
\pgfsetroundjoin%
\definecolor{currentfill}{rgb}{0.631373,0.788235,0.956863}%
\pgfsetfillcolor{currentfill}%
\pgfsetlinewidth{0.481800pt}%
\definecolor{currentstroke}{rgb}{1.000000,1.000000,1.000000}%
\pgfsetstrokecolor{currentstroke}%
\pgfsetdash{}{0pt}%
\pgfpathmoveto{\pgfqpoint{4.927031in}{7.639968in}}%
\pgfpathcurveto{\pgfqpoint{4.938081in}{7.639968in}}{\pgfqpoint{4.948680in}{7.644359in}}{\pgfqpoint{4.956494in}{7.652172in}}%
\pgfpathcurveto{\pgfqpoint{4.964307in}{7.659986in}}{\pgfqpoint{4.968698in}{7.670585in}}{\pgfqpoint{4.968698in}{7.681635in}}%
\pgfpathcurveto{\pgfqpoint{4.968698in}{7.692685in}}{\pgfqpoint{4.964307in}{7.703284in}}{\pgfqpoint{4.956494in}{7.711098in}}%
\pgfpathcurveto{\pgfqpoint{4.948680in}{7.718911in}}{\pgfqpoint{4.938081in}{7.723302in}}{\pgfqpoint{4.927031in}{7.723302in}}%
\pgfpathcurveto{\pgfqpoint{4.915981in}{7.723302in}}{\pgfqpoint{4.905382in}{7.718911in}}{\pgfqpoint{4.897568in}{7.711098in}}%
\pgfpathcurveto{\pgfqpoint{4.889755in}{7.703284in}}{\pgfqpoint{4.885364in}{7.692685in}}{\pgfqpoint{4.885364in}{7.681635in}}%
\pgfpathcurveto{\pgfqpoint{4.885364in}{7.670585in}}{\pgfqpoint{4.889755in}{7.659986in}}{\pgfqpoint{4.897568in}{7.652172in}}%
\pgfpathcurveto{\pgfqpoint{4.905382in}{7.644359in}}{\pgfqpoint{4.915981in}{7.639968in}}{\pgfqpoint{4.927031in}{7.639968in}}%
\pgfpathclose%
\pgfusepath{stroke,fill}%
\end{pgfscope}%
\begin{pgfscope}%
\pgfpathrectangle{\pgfqpoint{0.570343in}{0.331635in}}{\pgfqpoint{9.300000in}{7.700000in}}%
\pgfusepath{clip}%
\pgfsetbuttcap%
\pgfsetroundjoin%
\definecolor{currentfill}{rgb}{0.631373,0.788235,0.956863}%
\pgfsetfillcolor{currentfill}%
\pgfsetlinewidth{0.481800pt}%
\definecolor{currentstroke}{rgb}{1.000000,1.000000,1.000000}%
\pgfsetstrokecolor{currentstroke}%
\pgfsetdash{}{0pt}%
\pgfpathmoveto{\pgfqpoint{8.854047in}{5.540076in}}%
\pgfpathcurveto{\pgfqpoint{8.865097in}{5.540076in}}{\pgfqpoint{8.875696in}{5.544466in}}{\pgfqpoint{8.883509in}{5.552280in}}%
\pgfpathcurveto{\pgfqpoint{8.891323in}{5.560093in}}{\pgfqpoint{8.895713in}{5.570692in}}{\pgfqpoint{8.895713in}{5.581742in}}%
\pgfpathcurveto{\pgfqpoint{8.895713in}{5.592793in}}{\pgfqpoint{8.891323in}{5.603392in}}{\pgfqpoint{8.883509in}{5.611205in}}%
\pgfpathcurveto{\pgfqpoint{8.875696in}{5.619019in}}{\pgfqpoint{8.865097in}{5.623409in}}{\pgfqpoint{8.854047in}{5.623409in}}%
\pgfpathcurveto{\pgfqpoint{8.842996in}{5.623409in}}{\pgfqpoint{8.832397in}{5.619019in}}{\pgfqpoint{8.824584in}{5.611205in}}%
\pgfpathcurveto{\pgfqpoint{8.816770in}{5.603392in}}{\pgfqpoint{8.812380in}{5.592793in}}{\pgfqpoint{8.812380in}{5.581742in}}%
\pgfpathcurveto{\pgfqpoint{8.812380in}{5.570692in}}{\pgfqpoint{8.816770in}{5.560093in}}{\pgfqpoint{8.824584in}{5.552280in}}%
\pgfpathcurveto{\pgfqpoint{8.832397in}{5.544466in}}{\pgfqpoint{8.842996in}{5.540076in}}{\pgfqpoint{8.854047in}{5.540076in}}%
\pgfpathclose%
\pgfusepath{stroke,fill}%
\end{pgfscope}%
\begin{pgfscope}%
\pgfpathrectangle{\pgfqpoint{0.570343in}{0.331635in}}{\pgfqpoint{9.300000in}{7.700000in}}%
\pgfusepath{clip}%
\pgfsetbuttcap%
\pgfsetroundjoin%
\definecolor{currentfill}{rgb}{0.631373,0.788235,0.956863}%
\pgfsetfillcolor{currentfill}%
\pgfsetlinewidth{0.481800pt}%
\definecolor{currentstroke}{rgb}{1.000000,1.000000,1.000000}%
\pgfsetstrokecolor{currentstroke}%
\pgfsetdash{}{0pt}%
\pgfpathmoveto{\pgfqpoint{1.547559in}{3.962355in}}%
\pgfpathcurveto{\pgfqpoint{1.558609in}{3.962355in}}{\pgfqpoint{1.569208in}{3.966746in}}{\pgfqpoint{1.577021in}{3.974559in}}%
\pgfpathcurveto{\pgfqpoint{1.584835in}{3.982373in}}{\pgfqpoint{1.589225in}{3.992972in}}{\pgfqpoint{1.589225in}{4.004022in}}%
\pgfpathcurveto{\pgfqpoint{1.589225in}{4.015072in}}{\pgfqpoint{1.584835in}{4.025671in}}{\pgfqpoint{1.577021in}{4.033485in}}%
\pgfpathcurveto{\pgfqpoint{1.569208in}{4.041299in}}{\pgfqpoint{1.558609in}{4.045689in}}{\pgfqpoint{1.547559in}{4.045689in}}%
\pgfpathcurveto{\pgfqpoint{1.536508in}{4.045689in}}{\pgfqpoint{1.525909in}{4.041299in}}{\pgfqpoint{1.518096in}{4.033485in}}%
\pgfpathcurveto{\pgfqpoint{1.510282in}{4.025671in}}{\pgfqpoint{1.505892in}{4.015072in}}{\pgfqpoint{1.505892in}{4.004022in}}%
\pgfpathcurveto{\pgfqpoint{1.505892in}{3.992972in}}{\pgfqpoint{1.510282in}{3.982373in}}{\pgfqpoint{1.518096in}{3.974559in}}%
\pgfpathcurveto{\pgfqpoint{1.525909in}{3.966746in}}{\pgfqpoint{1.536508in}{3.962355in}}{\pgfqpoint{1.547559in}{3.962355in}}%
\pgfpathclose%
\pgfusepath{stroke,fill}%
\end{pgfscope}%
\begin{pgfscope}%
\pgfpathrectangle{\pgfqpoint{0.570343in}{0.331635in}}{\pgfqpoint{9.300000in}{7.700000in}}%
\pgfusepath{clip}%
\pgfsetbuttcap%
\pgfsetroundjoin%
\definecolor{currentfill}{rgb}{0.631373,0.788235,0.956863}%
\pgfsetfillcolor{currentfill}%
\pgfsetlinewidth{0.481800pt}%
\definecolor{currentstroke}{rgb}{1.000000,1.000000,1.000000}%
\pgfsetstrokecolor{currentstroke}%
\pgfsetdash{}{0pt}%
\pgfpathmoveto{\pgfqpoint{2.739938in}{3.672409in}}%
\pgfpathcurveto{\pgfqpoint{2.750988in}{3.672409in}}{\pgfqpoint{2.761587in}{3.676799in}}{\pgfqpoint{2.769401in}{3.684612in}}%
\pgfpathcurveto{\pgfqpoint{2.777215in}{3.692426in}}{\pgfqpoint{2.781605in}{3.703025in}}{\pgfqpoint{2.781605in}{3.714075in}}%
\pgfpathcurveto{\pgfqpoint{2.781605in}{3.725125in}}{\pgfqpoint{2.777215in}{3.735724in}}{\pgfqpoint{2.769401in}{3.743538in}}%
\pgfpathcurveto{\pgfqpoint{2.761587in}{3.751352in}}{\pgfqpoint{2.750988in}{3.755742in}}{\pgfqpoint{2.739938in}{3.755742in}}%
\pgfpathcurveto{\pgfqpoint{2.728888in}{3.755742in}}{\pgfqpoint{2.718289in}{3.751352in}}{\pgfqpoint{2.710475in}{3.743538in}}%
\pgfpathcurveto{\pgfqpoint{2.702662in}{3.735724in}}{\pgfqpoint{2.698272in}{3.725125in}}{\pgfqpoint{2.698272in}{3.714075in}}%
\pgfpathcurveto{\pgfqpoint{2.698272in}{3.703025in}}{\pgfqpoint{2.702662in}{3.692426in}}{\pgfqpoint{2.710475in}{3.684612in}}%
\pgfpathcurveto{\pgfqpoint{2.718289in}{3.676799in}}{\pgfqpoint{2.728888in}{3.672409in}}{\pgfqpoint{2.739938in}{3.672409in}}%
\pgfpathclose%
\pgfusepath{stroke,fill}%
\end{pgfscope}%
\begin{pgfscope}%
\pgfpathrectangle{\pgfqpoint{0.570343in}{0.331635in}}{\pgfqpoint{9.300000in}{7.700000in}}%
\pgfusepath{clip}%
\pgfsetbuttcap%
\pgfsetroundjoin%
\definecolor{currentfill}{rgb}{0.631373,0.788235,0.956863}%
\pgfsetfillcolor{currentfill}%
\pgfsetlinewidth{0.481800pt}%
\definecolor{currentstroke}{rgb}{1.000000,1.000000,1.000000}%
\pgfsetstrokecolor{currentstroke}%
\pgfsetdash{}{0pt}%
\pgfpathmoveto{\pgfqpoint{7.703434in}{5.198382in}}%
\pgfpathcurveto{\pgfqpoint{7.714485in}{5.198382in}}{\pgfqpoint{7.725084in}{5.202772in}}{\pgfqpoint{7.732897in}{5.210586in}}%
\pgfpathcurveto{\pgfqpoint{7.740711in}{5.218399in}}{\pgfqpoint{7.745101in}{5.228998in}}{\pgfqpoint{7.745101in}{5.240049in}}%
\pgfpathcurveto{\pgfqpoint{7.745101in}{5.251099in}}{\pgfqpoint{7.740711in}{5.261698in}}{\pgfqpoint{7.732897in}{5.269511in}}%
\pgfpathcurveto{\pgfqpoint{7.725084in}{5.277325in}}{\pgfqpoint{7.714485in}{5.281715in}}{\pgfqpoint{7.703434in}{5.281715in}}%
\pgfpathcurveto{\pgfqpoint{7.692384in}{5.281715in}}{\pgfqpoint{7.681785in}{5.277325in}}{\pgfqpoint{7.673972in}{5.269511in}}%
\pgfpathcurveto{\pgfqpoint{7.666158in}{5.261698in}}{\pgfqpoint{7.661768in}{5.251099in}}{\pgfqpoint{7.661768in}{5.240049in}}%
\pgfpathcurveto{\pgfqpoint{7.661768in}{5.228998in}}{\pgfqpoint{7.666158in}{5.218399in}}{\pgfqpoint{7.673972in}{5.210586in}}%
\pgfpathcurveto{\pgfqpoint{7.681785in}{5.202772in}}{\pgfqpoint{7.692384in}{5.198382in}}{\pgfqpoint{7.703434in}{5.198382in}}%
\pgfpathclose%
\pgfusepath{stroke,fill}%
\end{pgfscope}%
\begin{pgfscope}%
\pgfpathrectangle{\pgfqpoint{0.570343in}{0.331635in}}{\pgfqpoint{9.300000in}{7.700000in}}%
\pgfusepath{clip}%
\pgfsetbuttcap%
\pgfsetroundjoin%
\definecolor{currentfill}{rgb}{0.631373,0.788235,0.956863}%
\pgfsetfillcolor{currentfill}%
\pgfsetlinewidth{0.481800pt}%
\definecolor{currentstroke}{rgb}{1.000000,1.000000,1.000000}%
\pgfsetstrokecolor{currentstroke}%
\pgfsetdash{}{0pt}%
\pgfpathmoveto{\pgfqpoint{7.113284in}{2.769399in}}%
\pgfpathcurveto{\pgfqpoint{7.124334in}{2.769399in}}{\pgfqpoint{7.134933in}{2.773790in}}{\pgfqpoint{7.142746in}{2.781603in}}%
\pgfpathcurveto{\pgfqpoint{7.150560in}{2.789417in}}{\pgfqpoint{7.154950in}{2.800016in}}{\pgfqpoint{7.154950in}{2.811066in}}%
\pgfpathcurveto{\pgfqpoint{7.154950in}{2.822116in}}{\pgfqpoint{7.150560in}{2.832715in}}{\pgfqpoint{7.142746in}{2.840529in}}%
\pgfpathcurveto{\pgfqpoint{7.134933in}{2.848343in}}{\pgfqpoint{7.124334in}{2.852733in}}{\pgfqpoint{7.113284in}{2.852733in}}%
\pgfpathcurveto{\pgfqpoint{7.102233in}{2.852733in}}{\pgfqpoint{7.091634in}{2.848343in}}{\pgfqpoint{7.083821in}{2.840529in}}%
\pgfpathcurveto{\pgfqpoint{7.076007in}{2.832715in}}{\pgfqpoint{7.071617in}{2.822116in}}{\pgfqpoint{7.071617in}{2.811066in}}%
\pgfpathcurveto{\pgfqpoint{7.071617in}{2.800016in}}{\pgfqpoint{7.076007in}{2.789417in}}{\pgfqpoint{7.083821in}{2.781603in}}%
\pgfpathcurveto{\pgfqpoint{7.091634in}{2.773790in}}{\pgfqpoint{7.102233in}{2.769399in}}{\pgfqpoint{7.113284in}{2.769399in}}%
\pgfpathclose%
\pgfusepath{stroke,fill}%
\end{pgfscope}%
\begin{pgfscope}%
\pgfpathrectangle{\pgfqpoint{0.570343in}{0.331635in}}{\pgfqpoint{9.300000in}{7.700000in}}%
\pgfusepath{clip}%
\pgfsetbuttcap%
\pgfsetroundjoin%
\definecolor{currentfill}{rgb}{0.631373,0.788235,0.956863}%
\pgfsetfillcolor{currentfill}%
\pgfsetlinewidth{0.481800pt}%
\definecolor{currentstroke}{rgb}{1.000000,1.000000,1.000000}%
\pgfsetstrokecolor{currentstroke}%
\pgfsetdash{}{0pt}%
\pgfpathmoveto{\pgfqpoint{2.082625in}{2.090137in}}%
\pgfpathcurveto{\pgfqpoint{2.093675in}{2.090137in}}{\pgfqpoint{2.104275in}{2.094527in}}{\pgfqpoint{2.112088in}{2.102341in}}%
\pgfpathcurveto{\pgfqpoint{2.119902in}{2.110155in}}{\pgfqpoint{2.124292in}{2.120754in}}{\pgfqpoint{2.124292in}{2.131804in}}%
\pgfpathcurveto{\pgfqpoint{2.124292in}{2.142854in}}{\pgfqpoint{2.119902in}{2.153453in}}{\pgfqpoint{2.112088in}{2.161267in}}%
\pgfpathcurveto{\pgfqpoint{2.104275in}{2.169080in}}{\pgfqpoint{2.093675in}{2.173470in}}{\pgfqpoint{2.082625in}{2.173470in}}%
\pgfpathcurveto{\pgfqpoint{2.071575in}{2.173470in}}{\pgfqpoint{2.060976in}{2.169080in}}{\pgfqpoint{2.053163in}{2.161267in}}%
\pgfpathcurveto{\pgfqpoint{2.045349in}{2.153453in}}{\pgfqpoint{2.040959in}{2.142854in}}{\pgfqpoint{2.040959in}{2.131804in}}%
\pgfpathcurveto{\pgfqpoint{2.040959in}{2.120754in}}{\pgfqpoint{2.045349in}{2.110155in}}{\pgfqpoint{2.053163in}{2.102341in}}%
\pgfpathcurveto{\pgfqpoint{2.060976in}{2.094527in}}{\pgfqpoint{2.071575in}{2.090137in}}{\pgfqpoint{2.082625in}{2.090137in}}%
\pgfpathclose%
\pgfusepath{stroke,fill}%
\end{pgfscope}%
\begin{pgfscope}%
\pgfpathrectangle{\pgfqpoint{0.570343in}{0.331635in}}{\pgfqpoint{9.300000in}{7.700000in}}%
\pgfusepath{clip}%
\pgfsetbuttcap%
\pgfsetroundjoin%
\definecolor{currentfill}{rgb}{0.631373,0.788235,0.956863}%
\pgfsetfillcolor{currentfill}%
\pgfsetlinewidth{0.481800pt}%
\definecolor{currentstroke}{rgb}{1.000000,1.000000,1.000000}%
\pgfsetstrokecolor{currentstroke}%
\pgfsetdash{}{0pt}%
\pgfpathmoveto{\pgfqpoint{8.230336in}{6.091749in}}%
\pgfpathcurveto{\pgfqpoint{8.241386in}{6.091749in}}{\pgfqpoint{8.251985in}{6.096139in}}{\pgfqpoint{8.259799in}{6.103953in}}%
\pgfpathcurveto{\pgfqpoint{8.267612in}{6.111767in}}{\pgfqpoint{8.272003in}{6.122366in}}{\pgfqpoint{8.272003in}{6.133416in}}%
\pgfpathcurveto{\pgfqpoint{8.272003in}{6.144466in}}{\pgfqpoint{8.267612in}{6.155065in}}{\pgfqpoint{8.259799in}{6.162879in}}%
\pgfpathcurveto{\pgfqpoint{8.251985in}{6.170692in}}{\pgfqpoint{8.241386in}{6.175082in}}{\pgfqpoint{8.230336in}{6.175082in}}%
\pgfpathcurveto{\pgfqpoint{8.219286in}{6.175082in}}{\pgfqpoint{8.208687in}{6.170692in}}{\pgfqpoint{8.200873in}{6.162879in}}%
\pgfpathcurveto{\pgfqpoint{8.193060in}{6.155065in}}{\pgfqpoint{8.188669in}{6.144466in}}{\pgfqpoint{8.188669in}{6.133416in}}%
\pgfpathcurveto{\pgfqpoint{8.188669in}{6.122366in}}{\pgfqpoint{8.193060in}{6.111767in}}{\pgfqpoint{8.200873in}{6.103953in}}%
\pgfpathcurveto{\pgfqpoint{8.208687in}{6.096139in}}{\pgfqpoint{8.219286in}{6.091749in}}{\pgfqpoint{8.230336in}{6.091749in}}%
\pgfpathclose%
\pgfusepath{stroke,fill}%
\end{pgfscope}%
\begin{pgfscope}%
\pgfpathrectangle{\pgfqpoint{0.570343in}{0.331635in}}{\pgfqpoint{9.300000in}{7.700000in}}%
\pgfusepath{clip}%
\pgfsetbuttcap%
\pgfsetroundjoin%
\definecolor{currentfill}{rgb}{0.631373,0.788235,0.956863}%
\pgfsetfillcolor{currentfill}%
\pgfsetlinewidth{0.481800pt}%
\definecolor{currentstroke}{rgb}{1.000000,1.000000,1.000000}%
\pgfsetstrokecolor{currentstroke}%
\pgfsetdash{}{0pt}%
\pgfpathmoveto{\pgfqpoint{4.884974in}{2.588954in}}%
\pgfpathcurveto{\pgfqpoint{4.896024in}{2.588954in}}{\pgfqpoint{4.906623in}{2.593344in}}{\pgfqpoint{4.914437in}{2.601158in}}%
\pgfpathcurveto{\pgfqpoint{4.922250in}{2.608971in}}{\pgfqpoint{4.926641in}{2.619570in}}{\pgfqpoint{4.926641in}{2.630620in}}%
\pgfpathcurveto{\pgfqpoint{4.926641in}{2.641671in}}{\pgfqpoint{4.922250in}{2.652270in}}{\pgfqpoint{4.914437in}{2.660083in}}%
\pgfpathcurveto{\pgfqpoint{4.906623in}{2.667897in}}{\pgfqpoint{4.896024in}{2.672287in}}{\pgfqpoint{4.884974in}{2.672287in}}%
\pgfpathcurveto{\pgfqpoint{4.873924in}{2.672287in}}{\pgfqpoint{4.863325in}{2.667897in}}{\pgfqpoint{4.855511in}{2.660083in}}%
\pgfpathcurveto{\pgfqpoint{4.847698in}{2.652270in}}{\pgfqpoint{4.843307in}{2.641671in}}{\pgfqpoint{4.843307in}{2.630620in}}%
\pgfpathcurveto{\pgfqpoint{4.843307in}{2.619570in}}{\pgfqpoint{4.847698in}{2.608971in}}{\pgfqpoint{4.855511in}{2.601158in}}%
\pgfpathcurveto{\pgfqpoint{4.863325in}{2.593344in}}{\pgfqpoint{4.873924in}{2.588954in}}{\pgfqpoint{4.884974in}{2.588954in}}%
\pgfpathclose%
\pgfusepath{stroke,fill}%
\end{pgfscope}%
\begin{pgfscope}%
\pgfpathrectangle{\pgfqpoint{0.570343in}{0.331635in}}{\pgfqpoint{9.300000in}{7.700000in}}%
\pgfusepath{clip}%
\pgfsetbuttcap%
\pgfsetroundjoin%
\definecolor{currentfill}{rgb}{0.631373,0.788235,0.956863}%
\pgfsetfillcolor{currentfill}%
\pgfsetlinewidth{0.481800pt}%
\definecolor{currentstroke}{rgb}{1.000000,1.000000,1.000000}%
\pgfsetstrokecolor{currentstroke}%
\pgfsetdash{}{0pt}%
\pgfpathmoveto{\pgfqpoint{1.225108in}{2.741075in}}%
\pgfpathcurveto{\pgfqpoint{1.236158in}{2.741075in}}{\pgfqpoint{1.246757in}{2.745465in}}{\pgfqpoint{1.254570in}{2.753279in}}%
\pgfpathcurveto{\pgfqpoint{1.262384in}{2.761093in}}{\pgfqpoint{1.266774in}{2.771692in}}{\pgfqpoint{1.266774in}{2.782742in}}%
\pgfpathcurveto{\pgfqpoint{1.266774in}{2.793792in}}{\pgfqpoint{1.262384in}{2.804391in}}{\pgfqpoint{1.254570in}{2.812205in}}%
\pgfpathcurveto{\pgfqpoint{1.246757in}{2.820018in}}{\pgfqpoint{1.236158in}{2.824408in}}{\pgfqpoint{1.225108in}{2.824408in}}%
\pgfpathcurveto{\pgfqpoint{1.214057in}{2.824408in}}{\pgfqpoint{1.203458in}{2.820018in}}{\pgfqpoint{1.195645in}{2.812205in}}%
\pgfpathcurveto{\pgfqpoint{1.187831in}{2.804391in}}{\pgfqpoint{1.183441in}{2.793792in}}{\pgfqpoint{1.183441in}{2.782742in}}%
\pgfpathcurveto{\pgfqpoint{1.183441in}{2.771692in}}{\pgfqpoint{1.187831in}{2.761093in}}{\pgfqpoint{1.195645in}{2.753279in}}%
\pgfpathcurveto{\pgfqpoint{1.203458in}{2.745465in}}{\pgfqpoint{1.214057in}{2.741075in}}{\pgfqpoint{1.225108in}{2.741075in}}%
\pgfpathclose%
\pgfusepath{stroke,fill}%
\end{pgfscope}%
\begin{pgfscope}%
\pgfpathrectangle{\pgfqpoint{0.570343in}{0.331635in}}{\pgfqpoint{9.300000in}{7.700000in}}%
\pgfusepath{clip}%
\pgfsetbuttcap%
\pgfsetroundjoin%
\definecolor{currentfill}{rgb}{0.631373,0.788235,0.956863}%
\pgfsetfillcolor{currentfill}%
\pgfsetlinewidth{0.481800pt}%
\definecolor{currentstroke}{rgb}{1.000000,1.000000,1.000000}%
\pgfsetstrokecolor{currentstroke}%
\pgfsetdash{}{0pt}%
\pgfpathmoveto{\pgfqpoint{3.946393in}{3.437064in}}%
\pgfpathcurveto{\pgfqpoint{3.957443in}{3.437064in}}{\pgfqpoint{3.968042in}{3.441454in}}{\pgfqpoint{3.975856in}{3.449267in}}%
\pgfpathcurveto{\pgfqpoint{3.983670in}{3.457081in}}{\pgfqpoint{3.988060in}{3.467680in}}{\pgfqpoint{3.988060in}{3.478730in}}%
\pgfpathcurveto{\pgfqpoint{3.988060in}{3.489780in}}{\pgfqpoint{3.983670in}{3.500379in}}{\pgfqpoint{3.975856in}{3.508193in}}%
\pgfpathcurveto{\pgfqpoint{3.968042in}{3.516007in}}{\pgfqpoint{3.957443in}{3.520397in}}{\pgfqpoint{3.946393in}{3.520397in}}%
\pgfpathcurveto{\pgfqpoint{3.935343in}{3.520397in}}{\pgfqpoint{3.924744in}{3.516007in}}{\pgfqpoint{3.916930in}{3.508193in}}%
\pgfpathcurveto{\pgfqpoint{3.909117in}{3.500379in}}{\pgfqpoint{3.904726in}{3.489780in}}{\pgfqpoint{3.904726in}{3.478730in}}%
\pgfpathcurveto{\pgfqpoint{3.904726in}{3.467680in}}{\pgfqpoint{3.909117in}{3.457081in}}{\pgfqpoint{3.916930in}{3.449267in}}%
\pgfpathcurveto{\pgfqpoint{3.924744in}{3.441454in}}{\pgfqpoint{3.935343in}{3.437064in}}{\pgfqpoint{3.946393in}{3.437064in}}%
\pgfpathclose%
\pgfusepath{stroke,fill}%
\end{pgfscope}%
\begin{pgfscope}%
\pgfpathrectangle{\pgfqpoint{0.570343in}{0.331635in}}{\pgfqpoint{9.300000in}{7.700000in}}%
\pgfusepath{clip}%
\pgfsetbuttcap%
\pgfsetroundjoin%
\definecolor{currentfill}{rgb}{0.631373,0.788235,0.956863}%
\pgfsetfillcolor{currentfill}%
\pgfsetlinewidth{0.481800pt}%
\definecolor{currentstroke}{rgb}{1.000000,1.000000,1.000000}%
\pgfsetstrokecolor{currentstroke}%
\pgfsetdash{}{0pt}%
\pgfpathmoveto{\pgfqpoint{4.886393in}{1.848117in}}%
\pgfpathcurveto{\pgfqpoint{4.897443in}{1.848117in}}{\pgfqpoint{4.908042in}{1.852507in}}{\pgfqpoint{4.915856in}{1.860321in}}%
\pgfpathcurveto{\pgfqpoint{4.923669in}{1.868135in}}{\pgfqpoint{4.928059in}{1.878734in}}{\pgfqpoint{4.928059in}{1.889784in}}%
\pgfpathcurveto{\pgfqpoint{4.928059in}{1.900834in}}{\pgfqpoint{4.923669in}{1.911433in}}{\pgfqpoint{4.915856in}{1.919247in}}%
\pgfpathcurveto{\pgfqpoint{4.908042in}{1.927060in}}{\pgfqpoint{4.897443in}{1.931450in}}{\pgfqpoint{4.886393in}{1.931450in}}%
\pgfpathcurveto{\pgfqpoint{4.875343in}{1.931450in}}{\pgfqpoint{4.864744in}{1.927060in}}{\pgfqpoint{4.856930in}{1.919247in}}%
\pgfpathcurveto{\pgfqpoint{4.849116in}{1.911433in}}{\pgfqpoint{4.844726in}{1.900834in}}{\pgfqpoint{4.844726in}{1.889784in}}%
\pgfpathcurveto{\pgfqpoint{4.844726in}{1.878734in}}{\pgfqpoint{4.849116in}{1.868135in}}{\pgfqpoint{4.856930in}{1.860321in}}%
\pgfpathcurveto{\pgfqpoint{4.864744in}{1.852507in}}{\pgfqpoint{4.875343in}{1.848117in}}{\pgfqpoint{4.886393in}{1.848117in}}%
\pgfpathclose%
\pgfusepath{stroke,fill}%
\end{pgfscope}%
\begin{pgfscope}%
\pgfpathrectangle{\pgfqpoint{0.570343in}{0.331635in}}{\pgfqpoint{9.300000in}{7.700000in}}%
\pgfusepath{clip}%
\pgfsetbuttcap%
\pgfsetroundjoin%
\definecolor{currentfill}{rgb}{0.631373,0.788235,0.956863}%
\pgfsetfillcolor{currentfill}%
\pgfsetlinewidth{0.481800pt}%
\definecolor{currentstroke}{rgb}{1.000000,1.000000,1.000000}%
\pgfsetstrokecolor{currentstroke}%
\pgfsetdash{}{0pt}%
\pgfpathmoveto{\pgfqpoint{3.784493in}{1.682234in}}%
\pgfpathcurveto{\pgfqpoint{3.795543in}{1.682234in}}{\pgfqpoint{3.806142in}{1.686624in}}{\pgfqpoint{3.813956in}{1.694438in}}%
\pgfpathcurveto{\pgfqpoint{3.821769in}{1.702252in}}{\pgfqpoint{3.826160in}{1.712851in}}{\pgfqpoint{3.826160in}{1.723901in}}%
\pgfpathcurveto{\pgfqpoint{3.826160in}{1.734951in}}{\pgfqpoint{3.821769in}{1.745550in}}{\pgfqpoint{3.813956in}{1.753364in}}%
\pgfpathcurveto{\pgfqpoint{3.806142in}{1.761177in}}{\pgfqpoint{3.795543in}{1.765567in}}{\pgfqpoint{3.784493in}{1.765567in}}%
\pgfpathcurveto{\pgfqpoint{3.773443in}{1.765567in}}{\pgfqpoint{3.762844in}{1.761177in}}{\pgfqpoint{3.755030in}{1.753364in}}%
\pgfpathcurveto{\pgfqpoint{3.747217in}{1.745550in}}{\pgfqpoint{3.742826in}{1.734951in}}{\pgfqpoint{3.742826in}{1.723901in}}%
\pgfpathcurveto{\pgfqpoint{3.742826in}{1.712851in}}{\pgfqpoint{3.747217in}{1.702252in}}{\pgfqpoint{3.755030in}{1.694438in}}%
\pgfpathcurveto{\pgfqpoint{3.762844in}{1.686624in}}{\pgfqpoint{3.773443in}{1.682234in}}{\pgfqpoint{3.784493in}{1.682234in}}%
\pgfpathclose%
\pgfusepath{stroke,fill}%
\end{pgfscope}%
\begin{pgfscope}%
\pgfpathrectangle{\pgfqpoint{0.570343in}{0.331635in}}{\pgfqpoint{9.300000in}{7.700000in}}%
\pgfusepath{clip}%
\pgfsetbuttcap%
\pgfsetroundjoin%
\definecolor{currentfill}{rgb}{0.631373,0.788235,0.956863}%
\pgfsetfillcolor{currentfill}%
\pgfsetlinewidth{0.481800pt}%
\definecolor{currentstroke}{rgb}{1.000000,1.000000,1.000000}%
\pgfsetstrokecolor{currentstroke}%
\pgfsetdash{}{0pt}%
\pgfpathmoveto{\pgfqpoint{2.706989in}{2.493185in}}%
\pgfpathcurveto{\pgfqpoint{2.718039in}{2.493185in}}{\pgfqpoint{2.728638in}{2.497575in}}{\pgfqpoint{2.736452in}{2.505389in}}%
\pgfpathcurveto{\pgfqpoint{2.744265in}{2.513202in}}{\pgfqpoint{2.748656in}{2.523802in}}{\pgfqpoint{2.748656in}{2.534852in}}%
\pgfpathcurveto{\pgfqpoint{2.748656in}{2.545902in}}{\pgfqpoint{2.744265in}{2.556501in}}{\pgfqpoint{2.736452in}{2.564314in}}%
\pgfpathcurveto{\pgfqpoint{2.728638in}{2.572128in}}{\pgfqpoint{2.718039in}{2.576518in}}{\pgfqpoint{2.706989in}{2.576518in}}%
\pgfpathcurveto{\pgfqpoint{2.695939in}{2.576518in}}{\pgfqpoint{2.685340in}{2.572128in}}{\pgfqpoint{2.677526in}{2.564314in}}%
\pgfpathcurveto{\pgfqpoint{2.669712in}{2.556501in}}{\pgfqpoint{2.665322in}{2.545902in}}{\pgfqpoint{2.665322in}{2.534852in}}%
\pgfpathcurveto{\pgfqpoint{2.665322in}{2.523802in}}{\pgfqpoint{2.669712in}{2.513202in}}{\pgfqpoint{2.677526in}{2.505389in}}%
\pgfpathcurveto{\pgfqpoint{2.685340in}{2.497575in}}{\pgfqpoint{2.695939in}{2.493185in}}{\pgfqpoint{2.706989in}{2.493185in}}%
\pgfpathclose%
\pgfusepath{stroke,fill}%
\end{pgfscope}%
\begin{pgfscope}%
\pgfpathrectangle{\pgfqpoint{0.570343in}{0.331635in}}{\pgfqpoint{9.300000in}{7.700000in}}%
\pgfusepath{clip}%
\pgfsetbuttcap%
\pgfsetroundjoin%
\definecolor{currentfill}{rgb}{0.631373,0.788235,0.956863}%
\pgfsetfillcolor{currentfill}%
\pgfsetlinewidth{0.481800pt}%
\definecolor{currentstroke}{rgb}{1.000000,1.000000,1.000000}%
\pgfsetstrokecolor{currentstroke}%
\pgfsetdash{}{0pt}%
\pgfpathmoveto{\pgfqpoint{3.495356in}{4.079677in}}%
\pgfpathcurveto{\pgfqpoint{3.506406in}{4.079677in}}{\pgfqpoint{3.517005in}{4.084068in}}{\pgfqpoint{3.524818in}{4.091881in}}%
\pgfpathcurveto{\pgfqpoint{3.532632in}{4.099695in}}{\pgfqpoint{3.537022in}{4.110294in}}{\pgfqpoint{3.537022in}{4.121344in}}%
\pgfpathcurveto{\pgfqpoint{3.537022in}{4.132394in}}{\pgfqpoint{3.532632in}{4.142993in}}{\pgfqpoint{3.524818in}{4.150807in}}%
\pgfpathcurveto{\pgfqpoint{3.517005in}{4.158621in}}{\pgfqpoint{3.506406in}{4.163011in}}{\pgfqpoint{3.495356in}{4.163011in}}%
\pgfpathcurveto{\pgfqpoint{3.484306in}{4.163011in}}{\pgfqpoint{3.473706in}{4.158621in}}{\pgfqpoint{3.465893in}{4.150807in}}%
\pgfpathcurveto{\pgfqpoint{3.458079in}{4.142993in}}{\pgfqpoint{3.453689in}{4.132394in}}{\pgfqpoint{3.453689in}{4.121344in}}%
\pgfpathcurveto{\pgfqpoint{3.453689in}{4.110294in}}{\pgfqpoint{3.458079in}{4.099695in}}{\pgfqpoint{3.465893in}{4.091881in}}%
\pgfpathcurveto{\pgfqpoint{3.473706in}{4.084068in}}{\pgfqpoint{3.484306in}{4.079677in}}{\pgfqpoint{3.495356in}{4.079677in}}%
\pgfpathclose%
\pgfusepath{stroke,fill}%
\end{pgfscope}%
\begin{pgfscope}%
\pgfpathrectangle{\pgfqpoint{0.570343in}{0.331635in}}{\pgfqpoint{9.300000in}{7.700000in}}%
\pgfusepath{clip}%
\pgfsetbuttcap%
\pgfsetroundjoin%
\definecolor{currentfill}{rgb}{0.631373,0.788235,0.956863}%
\pgfsetfillcolor{currentfill}%
\pgfsetlinewidth{0.481800pt}%
\definecolor{currentstroke}{rgb}{1.000000,1.000000,1.000000}%
\pgfsetstrokecolor{currentstroke}%
\pgfsetdash{}{0pt}%
\pgfpathmoveto{\pgfqpoint{0.993071in}{4.504689in}}%
\pgfpathcurveto{\pgfqpoint{1.004121in}{4.504689in}}{\pgfqpoint{1.014720in}{4.509079in}}{\pgfqpoint{1.022533in}{4.516893in}}%
\pgfpathcurveto{\pgfqpoint{1.030347in}{4.524707in}}{\pgfqpoint{1.034737in}{4.535306in}}{\pgfqpoint{1.034737in}{4.546356in}}%
\pgfpathcurveto{\pgfqpoint{1.034737in}{4.557406in}}{\pgfqpoint{1.030347in}{4.568005in}}{\pgfqpoint{1.022533in}{4.575818in}}%
\pgfpathcurveto{\pgfqpoint{1.014720in}{4.583632in}}{\pgfqpoint{1.004121in}{4.588022in}}{\pgfqpoint{0.993071in}{4.588022in}}%
\pgfpathcurveto{\pgfqpoint{0.982020in}{4.588022in}}{\pgfqpoint{0.971421in}{4.583632in}}{\pgfqpoint{0.963608in}{4.575818in}}%
\pgfpathcurveto{\pgfqpoint{0.955794in}{4.568005in}}{\pgfqpoint{0.951404in}{4.557406in}}{\pgfqpoint{0.951404in}{4.546356in}}%
\pgfpathcurveto{\pgfqpoint{0.951404in}{4.535306in}}{\pgfqpoint{0.955794in}{4.524707in}}{\pgfqpoint{0.963608in}{4.516893in}}%
\pgfpathcurveto{\pgfqpoint{0.971421in}{4.509079in}}{\pgfqpoint{0.982020in}{4.504689in}}{\pgfqpoint{0.993071in}{4.504689in}}%
\pgfpathclose%
\pgfusepath{stroke,fill}%
\end{pgfscope}%
\begin{pgfscope}%
\pgfpathrectangle{\pgfqpoint{0.570343in}{0.331635in}}{\pgfqpoint{9.300000in}{7.700000in}}%
\pgfusepath{clip}%
\pgfsetbuttcap%
\pgfsetroundjoin%
\definecolor{currentfill}{rgb}{0.631373,0.788235,0.956863}%
\pgfsetfillcolor{currentfill}%
\pgfsetlinewidth{0.481800pt}%
\definecolor{currentstroke}{rgb}{1.000000,1.000000,1.000000}%
\pgfsetstrokecolor{currentstroke}%
\pgfsetdash{}{0pt}%
\pgfpathmoveto{\pgfqpoint{2.362144in}{4.853320in}}%
\pgfpathcurveto{\pgfqpoint{2.373195in}{4.853320in}}{\pgfqpoint{2.383794in}{4.857710in}}{\pgfqpoint{2.391607in}{4.865524in}}%
\pgfpathcurveto{\pgfqpoint{2.399421in}{4.873338in}}{\pgfqpoint{2.403811in}{4.883937in}}{\pgfqpoint{2.403811in}{4.894987in}}%
\pgfpathcurveto{\pgfqpoint{2.403811in}{4.906037in}}{\pgfqpoint{2.399421in}{4.916636in}}{\pgfqpoint{2.391607in}{4.924450in}}%
\pgfpathcurveto{\pgfqpoint{2.383794in}{4.932263in}}{\pgfqpoint{2.373195in}{4.936654in}}{\pgfqpoint{2.362144in}{4.936654in}}%
\pgfpathcurveto{\pgfqpoint{2.351094in}{4.936654in}}{\pgfqpoint{2.340495in}{4.932263in}}{\pgfqpoint{2.332682in}{4.924450in}}%
\pgfpathcurveto{\pgfqpoint{2.324868in}{4.916636in}}{\pgfqpoint{2.320478in}{4.906037in}}{\pgfqpoint{2.320478in}{4.894987in}}%
\pgfpathcurveto{\pgfqpoint{2.320478in}{4.883937in}}{\pgfqpoint{2.324868in}{4.873338in}}{\pgfqpoint{2.332682in}{4.865524in}}%
\pgfpathcurveto{\pgfqpoint{2.340495in}{4.857710in}}{\pgfqpoint{2.351094in}{4.853320in}}{\pgfqpoint{2.362144in}{4.853320in}}%
\pgfpathclose%
\pgfusepath{stroke,fill}%
\end{pgfscope}%
\begin{pgfscope}%
\pgfpathrectangle{\pgfqpoint{0.570343in}{0.331635in}}{\pgfqpoint{9.300000in}{7.700000in}}%
\pgfusepath{clip}%
\pgfsetbuttcap%
\pgfsetroundjoin%
\definecolor{currentfill}{rgb}{0.631373,0.788235,0.956863}%
\pgfsetfillcolor{currentfill}%
\pgfsetlinewidth{0.481800pt}%
\definecolor{currentstroke}{rgb}{1.000000,1.000000,1.000000}%
\pgfsetstrokecolor{currentstroke}%
\pgfsetdash{}{0pt}%
\pgfpathmoveto{\pgfqpoint{4.437533in}{3.921461in}}%
\pgfpathcurveto{\pgfqpoint{4.448583in}{3.921461in}}{\pgfqpoint{4.459182in}{3.925852in}}{\pgfqpoint{4.466996in}{3.933665in}}%
\pgfpathcurveto{\pgfqpoint{4.474810in}{3.941479in}}{\pgfqpoint{4.479200in}{3.952078in}}{\pgfqpoint{4.479200in}{3.963128in}}%
\pgfpathcurveto{\pgfqpoint{4.479200in}{3.974178in}}{\pgfqpoint{4.474810in}{3.984777in}}{\pgfqpoint{4.466996in}{3.992591in}}%
\pgfpathcurveto{\pgfqpoint{4.459182in}{4.000404in}}{\pgfqpoint{4.448583in}{4.004795in}}{\pgfqpoint{4.437533in}{4.004795in}}%
\pgfpathcurveto{\pgfqpoint{4.426483in}{4.004795in}}{\pgfqpoint{4.415884in}{4.000404in}}{\pgfqpoint{4.408070in}{3.992591in}}%
\pgfpathcurveto{\pgfqpoint{4.400257in}{3.984777in}}{\pgfqpoint{4.395867in}{3.974178in}}{\pgfqpoint{4.395867in}{3.963128in}}%
\pgfpathcurveto{\pgfqpoint{4.395867in}{3.952078in}}{\pgfqpoint{4.400257in}{3.941479in}}{\pgfqpoint{4.408070in}{3.933665in}}%
\pgfpathcurveto{\pgfqpoint{4.415884in}{3.925852in}}{\pgfqpoint{4.426483in}{3.921461in}}{\pgfqpoint{4.437533in}{3.921461in}}%
\pgfpathclose%
\pgfusepath{stroke,fill}%
\end{pgfscope}%
\begin{pgfscope}%
\pgfpathrectangle{\pgfqpoint{0.570343in}{0.331635in}}{\pgfqpoint{9.300000in}{7.700000in}}%
\pgfusepath{clip}%
\pgfsetbuttcap%
\pgfsetroundjoin%
\definecolor{currentfill}{rgb}{0.631373,0.788235,0.956863}%
\pgfsetfillcolor{currentfill}%
\pgfsetlinewidth{0.481800pt}%
\definecolor{currentstroke}{rgb}{1.000000,1.000000,1.000000}%
\pgfsetstrokecolor{currentstroke}%
\pgfsetdash{}{0pt}%
\pgfpathmoveto{\pgfqpoint{3.510625in}{5.131958in}}%
\pgfpathcurveto{\pgfqpoint{3.521675in}{5.131958in}}{\pgfqpoint{3.532274in}{5.136348in}}{\pgfqpoint{3.540088in}{5.144162in}}%
\pgfpathcurveto{\pgfqpoint{3.547901in}{5.151976in}}{\pgfqpoint{3.552291in}{5.162575in}}{\pgfqpoint{3.552291in}{5.173625in}}%
\pgfpathcurveto{\pgfqpoint{3.552291in}{5.184675in}}{\pgfqpoint{3.547901in}{5.195274in}}{\pgfqpoint{3.540088in}{5.203087in}}%
\pgfpathcurveto{\pgfqpoint{3.532274in}{5.210901in}}{\pgfqpoint{3.521675in}{5.215291in}}{\pgfqpoint{3.510625in}{5.215291in}}%
\pgfpathcurveto{\pgfqpoint{3.499575in}{5.215291in}}{\pgfqpoint{3.488976in}{5.210901in}}{\pgfqpoint{3.481162in}{5.203087in}}%
\pgfpathcurveto{\pgfqpoint{3.473348in}{5.195274in}}{\pgfqpoint{3.468958in}{5.184675in}}{\pgfqpoint{3.468958in}{5.173625in}}%
\pgfpathcurveto{\pgfqpoint{3.468958in}{5.162575in}}{\pgfqpoint{3.473348in}{5.151976in}}{\pgfqpoint{3.481162in}{5.144162in}}%
\pgfpathcurveto{\pgfqpoint{3.488976in}{5.136348in}}{\pgfqpoint{3.499575in}{5.131958in}}{\pgfqpoint{3.510625in}{5.131958in}}%
\pgfpathclose%
\pgfusepath{stroke,fill}%
\end{pgfscope}%
\begin{pgfscope}%
\pgfpathrectangle{\pgfqpoint{0.570343in}{0.331635in}}{\pgfqpoint{9.300000in}{7.700000in}}%
\pgfusepath{clip}%
\pgfsetbuttcap%
\pgfsetroundjoin%
\definecolor{currentfill}{rgb}{0.631373,0.788235,0.956863}%
\pgfsetfillcolor{currentfill}%
\pgfsetlinewidth{0.481800pt}%
\definecolor{currentstroke}{rgb}{1.000000,1.000000,1.000000}%
\pgfsetstrokecolor{currentstroke}%
\pgfsetdash{}{0pt}%
\pgfpathmoveto{\pgfqpoint{4.054617in}{2.257712in}}%
\pgfpathcurveto{\pgfqpoint{4.065667in}{2.257712in}}{\pgfqpoint{4.076266in}{2.262103in}}{\pgfqpoint{4.084079in}{2.269916in}}%
\pgfpathcurveto{\pgfqpoint{4.091893in}{2.277730in}}{\pgfqpoint{4.096283in}{2.288329in}}{\pgfqpoint{4.096283in}{2.299379in}}%
\pgfpathcurveto{\pgfqpoint{4.096283in}{2.310429in}}{\pgfqpoint{4.091893in}{2.321028in}}{\pgfqpoint{4.084079in}{2.328842in}}%
\pgfpathcurveto{\pgfqpoint{4.076266in}{2.336655in}}{\pgfqpoint{4.065667in}{2.341046in}}{\pgfqpoint{4.054617in}{2.341046in}}%
\pgfpathcurveto{\pgfqpoint{4.043567in}{2.341046in}}{\pgfqpoint{4.032968in}{2.336655in}}{\pgfqpoint{4.025154in}{2.328842in}}%
\pgfpathcurveto{\pgfqpoint{4.017340in}{2.321028in}}{\pgfqpoint{4.012950in}{2.310429in}}{\pgfqpoint{4.012950in}{2.299379in}}%
\pgfpathcurveto{\pgfqpoint{4.012950in}{2.288329in}}{\pgfqpoint{4.017340in}{2.277730in}}{\pgfqpoint{4.025154in}{2.269916in}}%
\pgfpathcurveto{\pgfqpoint{4.032968in}{2.262103in}}{\pgfqpoint{4.043567in}{2.257712in}}{\pgfqpoint{4.054617in}{2.257712in}}%
\pgfpathclose%
\pgfusepath{stroke,fill}%
\end{pgfscope}%
\begin{pgfscope}%
\pgfpathrectangle{\pgfqpoint{0.570343in}{0.331635in}}{\pgfqpoint{9.300000in}{7.700000in}}%
\pgfusepath{clip}%
\pgfsetbuttcap%
\pgfsetroundjoin%
\definecolor{currentfill}{rgb}{0.631373,0.788235,0.956863}%
\pgfsetfillcolor{currentfill}%
\pgfsetlinewidth{0.481800pt}%
\definecolor{currentstroke}{rgb}{1.000000,1.000000,1.000000}%
\pgfsetstrokecolor{currentstroke}%
\pgfsetdash{}{0pt}%
\pgfpathmoveto{\pgfqpoint{4.954557in}{3.304660in}}%
\pgfpathcurveto{\pgfqpoint{4.965608in}{3.304660in}}{\pgfqpoint{4.976207in}{3.309051in}}{\pgfqpoint{4.984020in}{3.316864in}}%
\pgfpathcurveto{\pgfqpoint{4.991834in}{3.324678in}}{\pgfqpoint{4.996224in}{3.335277in}}{\pgfqpoint{4.996224in}{3.346327in}}%
\pgfpathcurveto{\pgfqpoint{4.996224in}{3.357377in}}{\pgfqpoint{4.991834in}{3.367976in}}{\pgfqpoint{4.984020in}{3.375790in}}%
\pgfpathcurveto{\pgfqpoint{4.976207in}{3.383603in}}{\pgfqpoint{4.965608in}{3.387994in}}{\pgfqpoint{4.954557in}{3.387994in}}%
\pgfpathcurveto{\pgfqpoint{4.943507in}{3.387994in}}{\pgfqpoint{4.932908in}{3.383603in}}{\pgfqpoint{4.925095in}{3.375790in}}%
\pgfpathcurveto{\pgfqpoint{4.917281in}{3.367976in}}{\pgfqpoint{4.912891in}{3.357377in}}{\pgfqpoint{4.912891in}{3.346327in}}%
\pgfpathcurveto{\pgfqpoint{4.912891in}{3.335277in}}{\pgfqpoint{4.917281in}{3.324678in}}{\pgfqpoint{4.925095in}{3.316864in}}%
\pgfpathcurveto{\pgfqpoint{4.932908in}{3.309051in}}{\pgfqpoint{4.943507in}{3.304660in}}{\pgfqpoint{4.954557in}{3.304660in}}%
\pgfpathclose%
\pgfusepath{stroke,fill}%
\end{pgfscope}%
\begin{pgfscope}%
\pgfpathrectangle{\pgfqpoint{0.570343in}{0.331635in}}{\pgfqpoint{9.300000in}{7.700000in}}%
\pgfusepath{clip}%
\pgfsetbuttcap%
\pgfsetroundjoin%
\definecolor{currentfill}{rgb}{0.631373,0.788235,0.956863}%
\pgfsetfillcolor{currentfill}%
\pgfsetlinewidth{0.481800pt}%
\definecolor{currentstroke}{rgb}{1.000000,1.000000,1.000000}%
\pgfsetstrokecolor{currentstroke}%
\pgfsetdash{}{0pt}%
\pgfpathmoveto{\pgfqpoint{2.539065in}{1.355485in}}%
\pgfpathcurveto{\pgfqpoint{2.550115in}{1.355485in}}{\pgfqpoint{2.560714in}{1.359876in}}{\pgfqpoint{2.568527in}{1.367689in}}%
\pgfpathcurveto{\pgfqpoint{2.576341in}{1.375503in}}{\pgfqpoint{2.580731in}{1.386102in}}{\pgfqpoint{2.580731in}{1.397152in}}%
\pgfpathcurveto{\pgfqpoint{2.580731in}{1.408202in}}{\pgfqpoint{2.576341in}{1.418801in}}{\pgfqpoint{2.568527in}{1.426615in}}%
\pgfpathcurveto{\pgfqpoint{2.560714in}{1.434429in}}{\pgfqpoint{2.550115in}{1.438819in}}{\pgfqpoint{2.539065in}{1.438819in}}%
\pgfpathcurveto{\pgfqpoint{2.528015in}{1.438819in}}{\pgfqpoint{2.517415in}{1.434429in}}{\pgfqpoint{2.509602in}{1.426615in}}%
\pgfpathcurveto{\pgfqpoint{2.501788in}{1.418801in}}{\pgfqpoint{2.497398in}{1.408202in}}{\pgfqpoint{2.497398in}{1.397152in}}%
\pgfpathcurveto{\pgfqpoint{2.497398in}{1.386102in}}{\pgfqpoint{2.501788in}{1.375503in}}{\pgfqpoint{2.509602in}{1.367689in}}%
\pgfpathcurveto{\pgfqpoint{2.517415in}{1.359876in}}{\pgfqpoint{2.528015in}{1.355485in}}{\pgfqpoint{2.539065in}{1.355485in}}%
\pgfpathclose%
\pgfusepath{stroke,fill}%
\end{pgfscope}%
\begin{pgfscope}%
\pgfpathrectangle{\pgfqpoint{0.570343in}{0.331635in}}{\pgfqpoint{9.300000in}{7.700000in}}%
\pgfusepath{clip}%
\pgfsetbuttcap%
\pgfsetroundjoin%
\definecolor{currentfill}{rgb}{0.631373,0.788235,0.956863}%
\pgfsetfillcolor{currentfill}%
\pgfsetlinewidth{0.481800pt}%
\definecolor{currentstroke}{rgb}{1.000000,1.000000,1.000000}%
\pgfsetstrokecolor{currentstroke}%
\pgfsetdash{}{0pt}%
\pgfpathmoveto{\pgfqpoint{6.972947in}{3.668887in}}%
\pgfpathcurveto{\pgfqpoint{6.983997in}{3.668887in}}{\pgfqpoint{6.994596in}{3.673277in}}{\pgfqpoint{7.002409in}{3.681091in}}%
\pgfpathcurveto{\pgfqpoint{7.010223in}{3.688904in}}{\pgfqpoint{7.014613in}{3.699503in}}{\pgfqpoint{7.014613in}{3.710554in}}%
\pgfpathcurveto{\pgfqpoint{7.014613in}{3.721604in}}{\pgfqpoint{7.010223in}{3.732203in}}{\pgfqpoint{7.002409in}{3.740016in}}%
\pgfpathcurveto{\pgfqpoint{6.994596in}{3.747830in}}{\pgfqpoint{6.983997in}{3.752220in}}{\pgfqpoint{6.972947in}{3.752220in}}%
\pgfpathcurveto{\pgfqpoint{6.961897in}{3.752220in}}{\pgfqpoint{6.951298in}{3.747830in}}{\pgfqpoint{6.943484in}{3.740016in}}%
\pgfpathcurveto{\pgfqpoint{6.935670in}{3.732203in}}{\pgfqpoint{6.931280in}{3.721604in}}{\pgfqpoint{6.931280in}{3.710554in}}%
\pgfpathcurveto{\pgfqpoint{6.931280in}{3.699503in}}{\pgfqpoint{6.935670in}{3.688904in}}{\pgfqpoint{6.943484in}{3.681091in}}%
\pgfpathcurveto{\pgfqpoint{6.951298in}{3.673277in}}{\pgfqpoint{6.961897in}{3.668887in}}{\pgfqpoint{6.972947in}{3.668887in}}%
\pgfpathclose%
\pgfusepath{stroke,fill}%
\end{pgfscope}%
\begin{pgfscope}%
\pgfpathrectangle{\pgfqpoint{0.570343in}{0.331635in}}{\pgfqpoint{9.300000in}{7.700000in}}%
\pgfusepath{clip}%
\pgfsetbuttcap%
\pgfsetroundjoin%
\definecolor{currentfill}{rgb}{0.631373,0.788235,0.956863}%
\pgfsetfillcolor{currentfill}%
\pgfsetlinewidth{0.481800pt}%
\definecolor{currentstroke}{rgb}{1.000000,1.000000,1.000000}%
\pgfsetstrokecolor{currentstroke}%
\pgfsetdash{}{0pt}%
\pgfpathmoveto{\pgfqpoint{5.923406in}{3.054205in}}%
\pgfpathcurveto{\pgfqpoint{5.934456in}{3.054205in}}{\pgfqpoint{5.945055in}{3.058595in}}{\pgfqpoint{5.952869in}{3.066409in}}%
\pgfpathcurveto{\pgfqpoint{5.960682in}{3.074222in}}{\pgfqpoint{5.965073in}{3.084821in}}{\pgfqpoint{5.965073in}{3.095871in}}%
\pgfpathcurveto{\pgfqpoint{5.965073in}{3.106922in}}{\pgfqpoint{5.960682in}{3.117521in}}{\pgfqpoint{5.952869in}{3.125334in}}%
\pgfpathcurveto{\pgfqpoint{5.945055in}{3.133148in}}{\pgfqpoint{5.934456in}{3.137538in}}{\pgfqpoint{5.923406in}{3.137538in}}%
\pgfpathcurveto{\pgfqpoint{5.912356in}{3.137538in}}{\pgfqpoint{5.901757in}{3.133148in}}{\pgfqpoint{5.893943in}{3.125334in}}%
\pgfpathcurveto{\pgfqpoint{5.886130in}{3.117521in}}{\pgfqpoint{5.881739in}{3.106922in}}{\pgfqpoint{5.881739in}{3.095871in}}%
\pgfpathcurveto{\pgfqpoint{5.881739in}{3.084821in}}{\pgfqpoint{5.886130in}{3.074222in}}{\pgfqpoint{5.893943in}{3.066409in}}%
\pgfpathcurveto{\pgfqpoint{5.901757in}{3.058595in}}{\pgfqpoint{5.912356in}{3.054205in}}{\pgfqpoint{5.923406in}{3.054205in}}%
\pgfpathclose%
\pgfusepath{stroke,fill}%
\end{pgfscope}%
\begin{pgfscope}%
\pgfpathrectangle{\pgfqpoint{0.570343in}{0.331635in}}{\pgfqpoint{9.300000in}{7.700000in}}%
\pgfusepath{clip}%
\pgfsetbuttcap%
\pgfsetroundjoin%
\definecolor{currentfill}{rgb}{1.000000,0.705882,0.509804}%
\pgfsetfillcolor{currentfill}%
\pgfsetlinewidth{0.481800pt}%
\definecolor{currentstroke}{rgb}{1.000000,1.000000,1.000000}%
\pgfsetstrokecolor{currentstroke}%
\pgfsetdash{}{0pt}%
\pgfpathmoveto{\pgfqpoint{6.817529in}{4.793902in}}%
\pgfpathcurveto{\pgfqpoint{6.828579in}{4.793902in}}{\pgfqpoint{6.839178in}{4.798293in}}{\pgfqpoint{6.846992in}{4.806106in}}%
\pgfpathcurveto{\pgfqpoint{6.854805in}{4.813920in}}{\pgfqpoint{6.859196in}{4.824519in}}{\pgfqpoint{6.859196in}{4.835569in}}%
\pgfpathcurveto{\pgfqpoint{6.859196in}{4.846619in}}{\pgfqpoint{6.854805in}{4.857218in}}{\pgfqpoint{6.846992in}{4.865032in}}%
\pgfpathcurveto{\pgfqpoint{6.839178in}{4.872845in}}{\pgfqpoint{6.828579in}{4.877236in}}{\pgfqpoint{6.817529in}{4.877236in}}%
\pgfpathcurveto{\pgfqpoint{6.806479in}{4.877236in}}{\pgfqpoint{6.795880in}{4.872845in}}{\pgfqpoint{6.788066in}{4.865032in}}%
\pgfpathcurveto{\pgfqpoint{6.780253in}{4.857218in}}{\pgfqpoint{6.775862in}{4.846619in}}{\pgfqpoint{6.775862in}{4.835569in}}%
\pgfpathcurveto{\pgfqpoint{6.775862in}{4.824519in}}{\pgfqpoint{6.780253in}{4.813920in}}{\pgfqpoint{6.788066in}{4.806106in}}%
\pgfpathcurveto{\pgfqpoint{6.795880in}{4.798293in}}{\pgfqpoint{6.806479in}{4.793902in}}{\pgfqpoint{6.817529in}{4.793902in}}%
\pgfpathclose%
\pgfusepath{stroke,fill}%
\end{pgfscope}%
\begin{pgfscope}%
\pgfpathrectangle{\pgfqpoint{0.570343in}{0.331635in}}{\pgfqpoint{9.300000in}{7.700000in}}%
\pgfusepath{clip}%
\pgfsetbuttcap%
\pgfsetroundjoin%
\definecolor{currentfill}{rgb}{1.000000,0.705882,0.509804}%
\pgfsetfillcolor{currentfill}%
\pgfsetlinewidth{0.481800pt}%
\definecolor{currentstroke}{rgb}{1.000000,1.000000,1.000000}%
\pgfsetstrokecolor{currentstroke}%
\pgfsetdash{}{0pt}%
\pgfpathmoveto{\pgfqpoint{6.117416in}{7.368199in}}%
\pgfpathcurveto{\pgfqpoint{6.128466in}{7.368199in}}{\pgfqpoint{6.139065in}{7.372590in}}{\pgfqpoint{6.146878in}{7.380403in}}%
\pgfpathcurveto{\pgfqpoint{6.154692in}{7.388217in}}{\pgfqpoint{6.159082in}{7.398816in}}{\pgfqpoint{6.159082in}{7.409866in}}%
\pgfpathcurveto{\pgfqpoint{6.159082in}{7.420916in}}{\pgfqpoint{6.154692in}{7.431515in}}{\pgfqpoint{6.146878in}{7.439329in}}%
\pgfpathcurveto{\pgfqpoint{6.139065in}{7.447142in}}{\pgfqpoint{6.128466in}{7.451533in}}{\pgfqpoint{6.117416in}{7.451533in}}%
\pgfpathcurveto{\pgfqpoint{6.106365in}{7.451533in}}{\pgfqpoint{6.095766in}{7.447142in}}{\pgfqpoint{6.087953in}{7.439329in}}%
\pgfpathcurveto{\pgfqpoint{6.080139in}{7.431515in}}{\pgfqpoint{6.075749in}{7.420916in}}{\pgfqpoint{6.075749in}{7.409866in}}%
\pgfpathcurveto{\pgfqpoint{6.075749in}{7.398816in}}{\pgfqpoint{6.080139in}{7.388217in}}{\pgfqpoint{6.087953in}{7.380403in}}%
\pgfpathcurveto{\pgfqpoint{6.095766in}{7.372590in}}{\pgfqpoint{6.106365in}{7.368199in}}{\pgfqpoint{6.117416in}{7.368199in}}%
\pgfpathclose%
\pgfusepath{stroke,fill}%
\end{pgfscope}%
\begin{pgfscope}%
\pgfpathrectangle{\pgfqpoint{0.570343in}{0.331635in}}{\pgfqpoint{9.300000in}{7.700000in}}%
\pgfusepath{clip}%
\pgfsetbuttcap%
\pgfsetroundjoin%
\definecolor{currentfill}{rgb}{1.000000,0.705882,0.509804}%
\pgfsetfillcolor{currentfill}%
\pgfsetlinewidth{0.481800pt}%
\definecolor{currentstroke}{rgb}{1.000000,1.000000,1.000000}%
\pgfsetstrokecolor{currentstroke}%
\pgfsetdash{}{0pt}%
\pgfpathmoveto{\pgfqpoint{7.933795in}{3.368900in}}%
\pgfpathcurveto{\pgfqpoint{7.944845in}{3.368900in}}{\pgfqpoint{7.955444in}{3.373290in}}{\pgfqpoint{7.963258in}{3.381104in}}%
\pgfpathcurveto{\pgfqpoint{7.971071in}{3.388917in}}{\pgfqpoint{7.975462in}{3.399516in}}{\pgfqpoint{7.975462in}{3.410567in}}%
\pgfpathcurveto{\pgfqpoint{7.975462in}{3.421617in}}{\pgfqpoint{7.971071in}{3.432216in}}{\pgfqpoint{7.963258in}{3.440029in}}%
\pgfpathcurveto{\pgfqpoint{7.955444in}{3.447843in}}{\pgfqpoint{7.944845in}{3.452233in}}{\pgfqpoint{7.933795in}{3.452233in}}%
\pgfpathcurveto{\pgfqpoint{7.922745in}{3.452233in}}{\pgfqpoint{7.912146in}{3.447843in}}{\pgfqpoint{7.904332in}{3.440029in}}%
\pgfpathcurveto{\pgfqpoint{7.896519in}{3.432216in}}{\pgfqpoint{7.892128in}{3.421617in}}{\pgfqpoint{7.892128in}{3.410567in}}%
\pgfpathcurveto{\pgfqpoint{7.892128in}{3.399516in}}{\pgfqpoint{7.896519in}{3.388917in}}{\pgfqpoint{7.904332in}{3.381104in}}%
\pgfpathcurveto{\pgfqpoint{7.912146in}{3.373290in}}{\pgfqpoint{7.922745in}{3.368900in}}{\pgfqpoint{7.933795in}{3.368900in}}%
\pgfpathclose%
\pgfusepath{stroke,fill}%
\end{pgfscope}%
\begin{pgfscope}%
\pgfpathrectangle{\pgfqpoint{0.570343in}{0.331635in}}{\pgfqpoint{9.300000in}{7.700000in}}%
\pgfusepath{clip}%
\pgfsetbuttcap%
\pgfsetroundjoin%
\definecolor{currentfill}{rgb}{1.000000,0.705882,0.509804}%
\pgfsetfillcolor{currentfill}%
\pgfsetlinewidth{0.481800pt}%
\definecolor{currentstroke}{rgb}{1.000000,1.000000,1.000000}%
\pgfsetstrokecolor{currentstroke}%
\pgfsetdash{}{0pt}%
\pgfpathmoveto{\pgfqpoint{7.377901in}{4.220747in}}%
\pgfpathcurveto{\pgfqpoint{7.388951in}{4.220747in}}{\pgfqpoint{7.399550in}{4.225138in}}{\pgfqpoint{7.407364in}{4.232951in}}%
\pgfpathcurveto{\pgfqpoint{7.415178in}{4.240765in}}{\pgfqpoint{7.419568in}{4.251364in}}{\pgfqpoint{7.419568in}{4.262414in}}%
\pgfpathcurveto{\pgfqpoint{7.419568in}{4.273464in}}{\pgfqpoint{7.415178in}{4.284063in}}{\pgfqpoint{7.407364in}{4.291877in}}%
\pgfpathcurveto{\pgfqpoint{7.399550in}{4.299690in}}{\pgfqpoint{7.388951in}{4.304081in}}{\pgfqpoint{7.377901in}{4.304081in}}%
\pgfpathcurveto{\pgfqpoint{7.366851in}{4.304081in}}{\pgfqpoint{7.356252in}{4.299690in}}{\pgfqpoint{7.348438in}{4.291877in}}%
\pgfpathcurveto{\pgfqpoint{7.340625in}{4.284063in}}{\pgfqpoint{7.336235in}{4.273464in}}{\pgfqpoint{7.336235in}{4.262414in}}%
\pgfpathcurveto{\pgfqpoint{7.336235in}{4.251364in}}{\pgfqpoint{7.340625in}{4.240765in}}{\pgfqpoint{7.348438in}{4.232951in}}%
\pgfpathcurveto{\pgfqpoint{7.356252in}{4.225138in}}{\pgfqpoint{7.366851in}{4.220747in}}{\pgfqpoint{7.377901in}{4.220747in}}%
\pgfpathclose%
\pgfusepath{stroke,fill}%
\end{pgfscope}%
\begin{pgfscope}%
\pgfpathrectangle{\pgfqpoint{0.570343in}{0.331635in}}{\pgfqpoint{9.300000in}{7.700000in}}%
\pgfusepath{clip}%
\pgfsetbuttcap%
\pgfsetroundjoin%
\definecolor{currentfill}{rgb}{1.000000,0.705882,0.509804}%
\pgfsetfillcolor{currentfill}%
\pgfsetlinewidth{0.481800pt}%
\definecolor{currentstroke}{rgb}{1.000000,1.000000,1.000000}%
\pgfsetstrokecolor{currentstroke}%
\pgfsetdash{}{0pt}%
\pgfpathmoveto{\pgfqpoint{7.284262in}{2.105459in}}%
\pgfpathcurveto{\pgfqpoint{7.295312in}{2.105459in}}{\pgfqpoint{7.305911in}{2.109849in}}{\pgfqpoint{7.313724in}{2.117662in}}%
\pgfpathcurveto{\pgfqpoint{7.321538in}{2.125476in}}{\pgfqpoint{7.325928in}{2.136075in}}{\pgfqpoint{7.325928in}{2.147125in}}%
\pgfpathcurveto{\pgfqpoint{7.325928in}{2.158175in}}{\pgfqpoint{7.321538in}{2.168774in}}{\pgfqpoint{7.313724in}{2.176588in}}%
\pgfpathcurveto{\pgfqpoint{7.305911in}{2.184402in}}{\pgfqpoint{7.295312in}{2.188792in}}{\pgfqpoint{7.284262in}{2.188792in}}%
\pgfpathcurveto{\pgfqpoint{7.273211in}{2.188792in}}{\pgfqpoint{7.262612in}{2.184402in}}{\pgfqpoint{7.254799in}{2.176588in}}%
\pgfpathcurveto{\pgfqpoint{7.246985in}{2.168774in}}{\pgfqpoint{7.242595in}{2.158175in}}{\pgfqpoint{7.242595in}{2.147125in}}%
\pgfpathcurveto{\pgfqpoint{7.242595in}{2.136075in}}{\pgfqpoint{7.246985in}{2.125476in}}{\pgfqpoint{7.254799in}{2.117662in}}%
\pgfpathcurveto{\pgfqpoint{7.262612in}{2.109849in}}{\pgfqpoint{7.273211in}{2.105459in}}{\pgfqpoint{7.284262in}{2.105459in}}%
\pgfpathclose%
\pgfusepath{stroke,fill}%
\end{pgfscope}%
\begin{pgfscope}%
\pgfpathrectangle{\pgfqpoint{0.570343in}{0.331635in}}{\pgfqpoint{9.300000in}{7.700000in}}%
\pgfusepath{clip}%
\pgfsetbuttcap%
\pgfsetroundjoin%
\definecolor{currentfill}{rgb}{1.000000,0.705882,0.509804}%
\pgfsetfillcolor{currentfill}%
\pgfsetlinewidth{0.481800pt}%
\definecolor{currentstroke}{rgb}{1.000000,1.000000,1.000000}%
\pgfsetstrokecolor{currentstroke}%
\pgfsetdash{}{0pt}%
\pgfpathmoveto{\pgfqpoint{5.804302in}{4.421587in}}%
\pgfpathcurveto{\pgfqpoint{5.815352in}{4.421587in}}{\pgfqpoint{5.825951in}{4.425977in}}{\pgfqpoint{5.833765in}{4.433791in}}%
\pgfpathcurveto{\pgfqpoint{5.841579in}{4.441605in}}{\pgfqpoint{5.845969in}{4.452204in}}{\pgfqpoint{5.845969in}{4.463254in}}%
\pgfpathcurveto{\pgfqpoint{5.845969in}{4.474304in}}{\pgfqpoint{5.841579in}{4.484903in}}{\pgfqpoint{5.833765in}{4.492717in}}%
\pgfpathcurveto{\pgfqpoint{5.825951in}{4.500530in}}{\pgfqpoint{5.815352in}{4.504920in}}{\pgfqpoint{5.804302in}{4.504920in}}%
\pgfpathcurveto{\pgfqpoint{5.793252in}{4.504920in}}{\pgfqpoint{5.782653in}{4.500530in}}{\pgfqpoint{5.774839in}{4.492717in}}%
\pgfpathcurveto{\pgfqpoint{5.767026in}{4.484903in}}{\pgfqpoint{5.762635in}{4.474304in}}{\pgfqpoint{5.762635in}{4.463254in}}%
\pgfpathcurveto{\pgfqpoint{5.762635in}{4.452204in}}{\pgfqpoint{5.767026in}{4.441605in}}{\pgfqpoint{5.774839in}{4.433791in}}%
\pgfpathcurveto{\pgfqpoint{5.782653in}{4.425977in}}{\pgfqpoint{5.793252in}{4.421587in}}{\pgfqpoint{5.804302in}{4.421587in}}%
\pgfpathclose%
\pgfusepath{stroke,fill}%
\end{pgfscope}%
\begin{pgfscope}%
\pgfpathrectangle{\pgfqpoint{0.570343in}{0.331635in}}{\pgfqpoint{9.300000in}{7.700000in}}%
\pgfusepath{clip}%
\pgfsetbuttcap%
\pgfsetroundjoin%
\definecolor{currentfill}{rgb}{1.000000,0.705882,0.509804}%
\pgfsetfillcolor{currentfill}%
\pgfsetlinewidth{0.481800pt}%
\definecolor{currentstroke}{rgb}{1.000000,1.000000,1.000000}%
\pgfsetstrokecolor{currentstroke}%
\pgfsetdash{}{0pt}%
\pgfpathmoveto{\pgfqpoint{9.282814in}{3.497977in}}%
\pgfpathcurveto{\pgfqpoint{9.293864in}{3.497977in}}{\pgfqpoint{9.304463in}{3.502367in}}{\pgfqpoint{9.312277in}{3.510181in}}%
\pgfpathcurveto{\pgfqpoint{9.320091in}{3.517994in}}{\pgfqpoint{9.324481in}{3.528593in}}{\pgfqpoint{9.324481in}{3.539643in}}%
\pgfpathcurveto{\pgfqpoint{9.324481in}{3.550694in}}{\pgfqpoint{9.320091in}{3.561293in}}{\pgfqpoint{9.312277in}{3.569106in}}%
\pgfpathcurveto{\pgfqpoint{9.304463in}{3.576920in}}{\pgfqpoint{9.293864in}{3.581310in}}{\pgfqpoint{9.282814in}{3.581310in}}%
\pgfpathcurveto{\pgfqpoint{9.271764in}{3.581310in}}{\pgfqpoint{9.261165in}{3.576920in}}{\pgfqpoint{9.253351in}{3.569106in}}%
\pgfpathcurveto{\pgfqpoint{9.245538in}{3.561293in}}{\pgfqpoint{9.241148in}{3.550694in}}{\pgfqpoint{9.241148in}{3.539643in}}%
\pgfpathcurveto{\pgfqpoint{9.241148in}{3.528593in}}{\pgfqpoint{9.245538in}{3.517994in}}{\pgfqpoint{9.253351in}{3.510181in}}%
\pgfpathcurveto{\pgfqpoint{9.261165in}{3.502367in}}{\pgfqpoint{9.271764in}{3.497977in}}{\pgfqpoint{9.282814in}{3.497977in}}%
\pgfpathclose%
\pgfusepath{stroke,fill}%
\end{pgfscope}%
\begin{pgfscope}%
\pgfpathrectangle{\pgfqpoint{0.570343in}{0.331635in}}{\pgfqpoint{9.300000in}{7.700000in}}%
\pgfusepath{clip}%
\pgfsetbuttcap%
\pgfsetroundjoin%
\definecolor{currentfill}{rgb}{1.000000,0.705882,0.509804}%
\pgfsetfillcolor{currentfill}%
\pgfsetlinewidth{0.481800pt}%
\definecolor{currentstroke}{rgb}{1.000000,1.000000,1.000000}%
\pgfsetstrokecolor{currentstroke}%
\pgfsetdash{}{0pt}%
\pgfpathmoveto{\pgfqpoint{5.814279in}{3.876417in}}%
\pgfpathcurveto{\pgfqpoint{5.825329in}{3.876417in}}{\pgfqpoint{5.835928in}{3.880807in}}{\pgfqpoint{5.843741in}{3.888620in}}%
\pgfpathcurveto{\pgfqpoint{5.851555in}{3.896434in}}{\pgfqpoint{5.855945in}{3.907033in}}{\pgfqpoint{5.855945in}{3.918083in}}%
\pgfpathcurveto{\pgfqpoint{5.855945in}{3.929133in}}{\pgfqpoint{5.851555in}{3.939732in}}{\pgfqpoint{5.843741in}{3.947546in}}%
\pgfpathcurveto{\pgfqpoint{5.835928in}{3.955360in}}{\pgfqpoint{5.825329in}{3.959750in}}{\pgfqpoint{5.814279in}{3.959750in}}%
\pgfpathcurveto{\pgfqpoint{5.803229in}{3.959750in}}{\pgfqpoint{5.792630in}{3.955360in}}{\pgfqpoint{5.784816in}{3.947546in}}%
\pgfpathcurveto{\pgfqpoint{5.777002in}{3.939732in}}{\pgfqpoint{5.772612in}{3.929133in}}{\pgfqpoint{5.772612in}{3.918083in}}%
\pgfpathcurveto{\pgfqpoint{5.772612in}{3.907033in}}{\pgfqpoint{5.777002in}{3.896434in}}{\pgfqpoint{5.784816in}{3.888620in}}%
\pgfpathcurveto{\pgfqpoint{5.792630in}{3.880807in}}{\pgfqpoint{5.803229in}{3.876417in}}{\pgfqpoint{5.814279in}{3.876417in}}%
\pgfpathclose%
\pgfusepath{stroke,fill}%
\end{pgfscope}%
\begin{pgfscope}%
\pgfpathrectangle{\pgfqpoint{0.570343in}{0.331635in}}{\pgfqpoint{9.300000in}{7.700000in}}%
\pgfusepath{clip}%
\pgfsetbuttcap%
\pgfsetroundjoin%
\definecolor{currentfill}{rgb}{1.000000,0.705882,0.509804}%
\pgfsetfillcolor{currentfill}%
\pgfsetlinewidth{0.481800pt}%
\definecolor{currentstroke}{rgb}{1.000000,1.000000,1.000000}%
\pgfsetstrokecolor{currentstroke}%
\pgfsetdash{}{0pt}%
\pgfpathmoveto{\pgfqpoint{6.347009in}{5.397079in}}%
\pgfpathcurveto{\pgfqpoint{6.358059in}{5.397079in}}{\pgfqpoint{6.368658in}{5.401470in}}{\pgfqpoint{6.376472in}{5.409283in}}%
\pgfpathcurveto{\pgfqpoint{6.384285in}{5.417097in}}{\pgfqpoint{6.388676in}{5.427696in}}{\pgfqpoint{6.388676in}{5.438746in}}%
\pgfpathcurveto{\pgfqpoint{6.388676in}{5.449796in}}{\pgfqpoint{6.384285in}{5.460395in}}{\pgfqpoint{6.376472in}{5.468209in}}%
\pgfpathcurveto{\pgfqpoint{6.368658in}{5.476022in}}{\pgfqpoint{6.358059in}{5.480413in}}{\pgfqpoint{6.347009in}{5.480413in}}%
\pgfpathcurveto{\pgfqpoint{6.335959in}{5.480413in}}{\pgfqpoint{6.325360in}{5.476022in}}{\pgfqpoint{6.317546in}{5.468209in}}%
\pgfpathcurveto{\pgfqpoint{6.309733in}{5.460395in}}{\pgfqpoint{6.305342in}{5.449796in}}{\pgfqpoint{6.305342in}{5.438746in}}%
\pgfpathcurveto{\pgfqpoint{6.305342in}{5.427696in}}{\pgfqpoint{6.309733in}{5.417097in}}{\pgfqpoint{6.317546in}{5.409283in}}%
\pgfpathcurveto{\pgfqpoint{6.325360in}{5.401470in}}{\pgfqpoint{6.335959in}{5.397079in}}{\pgfqpoint{6.347009in}{5.397079in}}%
\pgfpathclose%
\pgfusepath{stroke,fill}%
\end{pgfscope}%
\begin{pgfscope}%
\pgfpathrectangle{\pgfqpoint{0.570343in}{0.331635in}}{\pgfqpoint{9.300000in}{7.700000in}}%
\pgfusepath{clip}%
\pgfsetbuttcap%
\pgfsetroundjoin%
\definecolor{currentfill}{rgb}{1.000000,0.705882,0.509804}%
\pgfsetfillcolor{currentfill}%
\pgfsetlinewidth{0.481800pt}%
\definecolor{currentstroke}{rgb}{1.000000,1.000000,1.000000}%
\pgfsetstrokecolor{currentstroke}%
\pgfsetdash{}{0pt}%
\pgfpathmoveto{\pgfqpoint{8.386537in}{2.506416in}}%
\pgfpathcurveto{\pgfqpoint{8.397587in}{2.506416in}}{\pgfqpoint{8.408186in}{2.510806in}}{\pgfqpoint{8.415999in}{2.518619in}}%
\pgfpathcurveto{\pgfqpoint{8.423813in}{2.526433in}}{\pgfqpoint{8.428203in}{2.537032in}}{\pgfqpoint{8.428203in}{2.548082in}}%
\pgfpathcurveto{\pgfqpoint{8.428203in}{2.559132in}}{\pgfqpoint{8.423813in}{2.569731in}}{\pgfqpoint{8.415999in}{2.577545in}}%
\pgfpathcurveto{\pgfqpoint{8.408186in}{2.585359in}}{\pgfqpoint{8.397587in}{2.589749in}}{\pgfqpoint{8.386537in}{2.589749in}}%
\pgfpathcurveto{\pgfqpoint{8.375486in}{2.589749in}}{\pgfqpoint{8.364887in}{2.585359in}}{\pgfqpoint{8.357074in}{2.577545in}}%
\pgfpathcurveto{\pgfqpoint{8.349260in}{2.569731in}}{\pgfqpoint{8.344870in}{2.559132in}}{\pgfqpoint{8.344870in}{2.548082in}}%
\pgfpathcurveto{\pgfqpoint{8.344870in}{2.537032in}}{\pgfqpoint{8.349260in}{2.526433in}}{\pgfqpoint{8.357074in}{2.518619in}}%
\pgfpathcurveto{\pgfqpoint{8.364887in}{2.510806in}}{\pgfqpoint{8.375486in}{2.506416in}}{\pgfqpoint{8.386537in}{2.506416in}}%
\pgfpathclose%
\pgfusepath{stroke,fill}%
\end{pgfscope}%
\begin{pgfscope}%
\pgfpathrectangle{\pgfqpoint{0.570343in}{0.331635in}}{\pgfqpoint{9.300000in}{7.700000in}}%
\pgfusepath{clip}%
\pgfsetbuttcap%
\pgfsetroundjoin%
\definecolor{currentfill}{rgb}{1.000000,0.705882,0.509804}%
\pgfsetfillcolor{currentfill}%
\pgfsetlinewidth{0.481800pt}%
\definecolor{currentstroke}{rgb}{1.000000,1.000000,1.000000}%
\pgfsetstrokecolor{currentstroke}%
\pgfsetdash{}{0pt}%
\pgfpathmoveto{\pgfqpoint{8.388882in}{3.844007in}}%
\pgfpathcurveto{\pgfqpoint{8.399932in}{3.844007in}}{\pgfqpoint{8.410531in}{3.848398in}}{\pgfqpoint{8.418345in}{3.856211in}}%
\pgfpathcurveto{\pgfqpoint{8.426159in}{3.864025in}}{\pgfqpoint{8.430549in}{3.874624in}}{\pgfqpoint{8.430549in}{3.885674in}}%
\pgfpathcurveto{\pgfqpoint{8.430549in}{3.896724in}}{\pgfqpoint{8.426159in}{3.907323in}}{\pgfqpoint{8.418345in}{3.915137in}}%
\pgfpathcurveto{\pgfqpoint{8.410531in}{3.922950in}}{\pgfqpoint{8.399932in}{3.927341in}}{\pgfqpoint{8.388882in}{3.927341in}}%
\pgfpathcurveto{\pgfqpoint{8.377832in}{3.927341in}}{\pgfqpoint{8.367233in}{3.922950in}}{\pgfqpoint{8.359419in}{3.915137in}}%
\pgfpathcurveto{\pgfqpoint{8.351606in}{3.907323in}}{\pgfqpoint{8.347216in}{3.896724in}}{\pgfqpoint{8.347216in}{3.885674in}}%
\pgfpathcurveto{\pgfqpoint{8.347216in}{3.874624in}}{\pgfqpoint{8.351606in}{3.864025in}}{\pgfqpoint{8.359419in}{3.856211in}}%
\pgfpathcurveto{\pgfqpoint{8.367233in}{3.848398in}}{\pgfqpoint{8.377832in}{3.844007in}}{\pgfqpoint{8.388882in}{3.844007in}}%
\pgfpathclose%
\pgfusepath{stroke,fill}%
\end{pgfscope}%
\begin{pgfscope}%
\pgfpathrectangle{\pgfqpoint{0.570343in}{0.331635in}}{\pgfqpoint{9.300000in}{7.700000in}}%
\pgfusepath{clip}%
\pgfsetbuttcap%
\pgfsetroundjoin%
\definecolor{currentfill}{rgb}{1.000000,0.705882,0.509804}%
\pgfsetfillcolor{currentfill}%
\pgfsetlinewidth{0.481800pt}%
\definecolor{currentstroke}{rgb}{1.000000,1.000000,1.000000}%
\pgfsetstrokecolor{currentstroke}%
\pgfsetdash{}{0pt}%
\pgfpathmoveto{\pgfqpoint{3.136766in}{5.808604in}}%
\pgfpathcurveto{\pgfqpoint{3.147816in}{5.808604in}}{\pgfqpoint{3.158415in}{5.812994in}}{\pgfqpoint{3.166229in}{5.820807in}}%
\pgfpathcurveto{\pgfqpoint{3.174042in}{5.828621in}}{\pgfqpoint{3.178432in}{5.839220in}}{\pgfqpoint{3.178432in}{5.850270in}}%
\pgfpathcurveto{\pgfqpoint{3.178432in}{5.861320in}}{\pgfqpoint{3.174042in}{5.871919in}}{\pgfqpoint{3.166229in}{5.879733in}}%
\pgfpathcurveto{\pgfqpoint{3.158415in}{5.887547in}}{\pgfqpoint{3.147816in}{5.891937in}}{\pgfqpoint{3.136766in}{5.891937in}}%
\pgfpathcurveto{\pgfqpoint{3.125716in}{5.891937in}}{\pgfqpoint{3.115117in}{5.887547in}}{\pgfqpoint{3.107303in}{5.879733in}}%
\pgfpathcurveto{\pgfqpoint{3.099489in}{5.871919in}}{\pgfqpoint{3.095099in}{5.861320in}}{\pgfqpoint{3.095099in}{5.850270in}}%
\pgfpathcurveto{\pgfqpoint{3.095099in}{5.839220in}}{\pgfqpoint{3.099489in}{5.828621in}}{\pgfqpoint{3.107303in}{5.820807in}}%
\pgfpathcurveto{\pgfqpoint{3.115117in}{5.812994in}}{\pgfqpoint{3.125716in}{5.808604in}}{\pgfqpoint{3.136766in}{5.808604in}}%
\pgfpathclose%
\pgfusepath{stroke,fill}%
\end{pgfscope}%
\begin{pgfscope}%
\pgfpathrectangle{\pgfqpoint{0.570343in}{0.331635in}}{\pgfqpoint{9.300000in}{7.700000in}}%
\pgfusepath{clip}%
\pgfsetbuttcap%
\pgfsetroundjoin%
\definecolor{currentfill}{rgb}{1.000000,0.705882,0.509804}%
\pgfsetfillcolor{currentfill}%
\pgfsetlinewidth{0.481800pt}%
\definecolor{currentstroke}{rgb}{1.000000,1.000000,1.000000}%
\pgfsetstrokecolor{currentstroke}%
\pgfsetdash{}{0pt}%
\pgfpathmoveto{\pgfqpoint{5.512434in}{5.962742in}}%
\pgfpathcurveto{\pgfqpoint{5.523484in}{5.962742in}}{\pgfqpoint{5.534083in}{5.967132in}}{\pgfqpoint{5.541896in}{5.974946in}}%
\pgfpathcurveto{\pgfqpoint{5.549710in}{5.982760in}}{\pgfqpoint{5.554100in}{5.993359in}}{\pgfqpoint{5.554100in}{6.004409in}}%
\pgfpathcurveto{\pgfqpoint{5.554100in}{6.015459in}}{\pgfqpoint{5.549710in}{6.026058in}}{\pgfqpoint{5.541896in}{6.033872in}}%
\pgfpathcurveto{\pgfqpoint{5.534083in}{6.041685in}}{\pgfqpoint{5.523484in}{6.046075in}}{\pgfqpoint{5.512434in}{6.046075in}}%
\pgfpathcurveto{\pgfqpoint{5.501384in}{6.046075in}}{\pgfqpoint{5.490785in}{6.041685in}}{\pgfqpoint{5.482971in}{6.033872in}}%
\pgfpathcurveto{\pgfqpoint{5.475157in}{6.026058in}}{\pgfqpoint{5.470767in}{6.015459in}}{\pgfqpoint{5.470767in}{6.004409in}}%
\pgfpathcurveto{\pgfqpoint{5.470767in}{5.993359in}}{\pgfqpoint{5.475157in}{5.982760in}}{\pgfqpoint{5.482971in}{5.974946in}}%
\pgfpathcurveto{\pgfqpoint{5.490785in}{5.967132in}}{\pgfqpoint{5.501384in}{5.962742in}}{\pgfqpoint{5.512434in}{5.962742in}}%
\pgfpathclose%
\pgfusepath{stroke,fill}%
\end{pgfscope}%
\begin{pgfscope}%
\pgfpathrectangle{\pgfqpoint{0.570343in}{0.331635in}}{\pgfqpoint{9.300000in}{7.700000in}}%
\pgfusepath{clip}%
\pgfsetbuttcap%
\pgfsetroundjoin%
\definecolor{currentfill}{rgb}{1.000000,0.705882,0.509804}%
\pgfsetfillcolor{currentfill}%
\pgfsetlinewidth{0.481800pt}%
\definecolor{currentstroke}{rgb}{1.000000,1.000000,1.000000}%
\pgfsetstrokecolor{currentstroke}%
\pgfsetdash{}{0pt}%
\pgfpathmoveto{\pgfqpoint{7.456205in}{1.508541in}}%
\pgfpathcurveto{\pgfqpoint{7.467255in}{1.508541in}}{\pgfqpoint{7.477854in}{1.512931in}}{\pgfqpoint{7.485668in}{1.520745in}}%
\pgfpathcurveto{\pgfqpoint{7.493481in}{1.528558in}}{\pgfqpoint{7.497871in}{1.539157in}}{\pgfqpoint{7.497871in}{1.550207in}}%
\pgfpathcurveto{\pgfqpoint{7.497871in}{1.561258in}}{\pgfqpoint{7.493481in}{1.571857in}}{\pgfqpoint{7.485668in}{1.579670in}}%
\pgfpathcurveto{\pgfqpoint{7.477854in}{1.587484in}}{\pgfqpoint{7.467255in}{1.591874in}}{\pgfqpoint{7.456205in}{1.591874in}}%
\pgfpathcurveto{\pgfqpoint{7.445155in}{1.591874in}}{\pgfqpoint{7.434556in}{1.587484in}}{\pgfqpoint{7.426742in}{1.579670in}}%
\pgfpathcurveto{\pgfqpoint{7.418928in}{1.571857in}}{\pgfqpoint{7.414538in}{1.561258in}}{\pgfqpoint{7.414538in}{1.550207in}}%
\pgfpathcurveto{\pgfqpoint{7.414538in}{1.539157in}}{\pgfqpoint{7.418928in}{1.528558in}}{\pgfqpoint{7.426742in}{1.520745in}}%
\pgfpathcurveto{\pgfqpoint{7.434556in}{1.512931in}}{\pgfqpoint{7.445155in}{1.508541in}}{\pgfqpoint{7.456205in}{1.508541in}}%
\pgfpathclose%
\pgfusepath{stroke,fill}%
\end{pgfscope}%
\begin{pgfscope}%
\pgfpathrectangle{\pgfqpoint{0.570343in}{0.331635in}}{\pgfqpoint{9.300000in}{7.700000in}}%
\pgfusepath{clip}%
\pgfsetbuttcap%
\pgfsetroundjoin%
\definecolor{currentfill}{rgb}{1.000000,0.705882,0.509804}%
\pgfsetfillcolor{currentfill}%
\pgfsetlinewidth{0.481800pt}%
\definecolor{currentstroke}{rgb}{1.000000,1.000000,1.000000}%
\pgfsetstrokecolor{currentstroke}%
\pgfsetdash{}{0pt}%
\pgfpathmoveto{\pgfqpoint{4.644507in}{6.725659in}}%
\pgfpathcurveto{\pgfqpoint{4.655557in}{6.725659in}}{\pgfqpoint{4.666156in}{6.730049in}}{\pgfqpoint{4.673970in}{6.737863in}}%
\pgfpathcurveto{\pgfqpoint{4.681783in}{6.745676in}}{\pgfqpoint{4.686174in}{6.756275in}}{\pgfqpoint{4.686174in}{6.767325in}}%
\pgfpathcurveto{\pgfqpoint{4.686174in}{6.778376in}}{\pgfqpoint{4.681783in}{6.788975in}}{\pgfqpoint{4.673970in}{6.796788in}}%
\pgfpathcurveto{\pgfqpoint{4.666156in}{6.804602in}}{\pgfqpoint{4.655557in}{6.808992in}}{\pgfqpoint{4.644507in}{6.808992in}}%
\pgfpathcurveto{\pgfqpoint{4.633457in}{6.808992in}}{\pgfqpoint{4.622858in}{6.804602in}}{\pgfqpoint{4.615044in}{6.796788in}}%
\pgfpathcurveto{\pgfqpoint{4.607230in}{6.788975in}}{\pgfqpoint{4.602840in}{6.778376in}}{\pgfqpoint{4.602840in}{6.767325in}}%
\pgfpathcurveto{\pgfqpoint{4.602840in}{6.756275in}}{\pgfqpoint{4.607230in}{6.745676in}}{\pgfqpoint{4.615044in}{6.737863in}}%
\pgfpathcurveto{\pgfqpoint{4.622858in}{6.730049in}}{\pgfqpoint{4.633457in}{6.725659in}}{\pgfqpoint{4.644507in}{6.725659in}}%
\pgfpathclose%
\pgfusepath{stroke,fill}%
\end{pgfscope}%
\begin{pgfscope}%
\pgfpathrectangle{\pgfqpoint{0.570343in}{0.331635in}}{\pgfqpoint{9.300000in}{7.700000in}}%
\pgfusepath{clip}%
\pgfsetbuttcap%
\pgfsetroundjoin%
\definecolor{currentfill}{rgb}{1.000000,0.705882,0.509804}%
\pgfsetfillcolor{currentfill}%
\pgfsetlinewidth{0.481800pt}%
\definecolor{currentstroke}{rgb}{1.000000,1.000000,1.000000}%
\pgfsetstrokecolor{currentstroke}%
\pgfsetdash{}{0pt}%
\pgfpathmoveto{\pgfqpoint{4.552181in}{5.485402in}}%
\pgfpathcurveto{\pgfqpoint{4.563231in}{5.485402in}}{\pgfqpoint{4.573831in}{5.489792in}}{\pgfqpoint{4.581644in}{5.497606in}}%
\pgfpathcurveto{\pgfqpoint{4.589458in}{5.505420in}}{\pgfqpoint{4.593848in}{5.516019in}}{\pgfqpoint{4.593848in}{5.527069in}}%
\pgfpathcurveto{\pgfqpoint{4.593848in}{5.538119in}}{\pgfqpoint{4.589458in}{5.548718in}}{\pgfqpoint{4.581644in}{5.556532in}}%
\pgfpathcurveto{\pgfqpoint{4.573831in}{5.564345in}}{\pgfqpoint{4.563231in}{5.568736in}}{\pgfqpoint{4.552181in}{5.568736in}}%
\pgfpathcurveto{\pgfqpoint{4.541131in}{5.568736in}}{\pgfqpoint{4.530532in}{5.564345in}}{\pgfqpoint{4.522719in}{5.556532in}}%
\pgfpathcurveto{\pgfqpoint{4.514905in}{5.548718in}}{\pgfqpoint{4.510515in}{5.538119in}}{\pgfqpoint{4.510515in}{5.527069in}}%
\pgfpathcurveto{\pgfqpoint{4.510515in}{5.516019in}}{\pgfqpoint{4.514905in}{5.505420in}}{\pgfqpoint{4.522719in}{5.497606in}}%
\pgfpathcurveto{\pgfqpoint{4.530532in}{5.489792in}}{\pgfqpoint{4.541131in}{5.485402in}}{\pgfqpoint{4.552181in}{5.485402in}}%
\pgfpathclose%
\pgfusepath{stroke,fill}%
\end{pgfscope}%
\begin{pgfscope}%
\pgfpathrectangle{\pgfqpoint{0.570343in}{0.331635in}}{\pgfqpoint{9.300000in}{7.700000in}}%
\pgfusepath{clip}%
\pgfsetbuttcap%
\pgfsetroundjoin%
\definecolor{currentfill}{rgb}{1.000000,0.705882,0.509804}%
\pgfsetfillcolor{currentfill}%
\pgfsetlinewidth{0.481800pt}%
\definecolor{currentstroke}{rgb}{1.000000,1.000000,1.000000}%
\pgfsetstrokecolor{currentstroke}%
\pgfsetdash{}{0pt}%
\pgfpathmoveto{\pgfqpoint{7.082491in}{7.108311in}}%
\pgfpathcurveto{\pgfqpoint{7.093541in}{7.108311in}}{\pgfqpoint{7.104140in}{7.112701in}}{\pgfqpoint{7.111954in}{7.120515in}}%
\pgfpathcurveto{\pgfqpoint{7.119768in}{7.128329in}}{\pgfqpoint{7.124158in}{7.138928in}}{\pgfqpoint{7.124158in}{7.149978in}}%
\pgfpathcurveto{\pgfqpoint{7.124158in}{7.161028in}}{\pgfqpoint{7.119768in}{7.171627in}}{\pgfqpoint{7.111954in}{7.179441in}}%
\pgfpathcurveto{\pgfqpoint{7.104140in}{7.187254in}}{\pgfqpoint{7.093541in}{7.191645in}}{\pgfqpoint{7.082491in}{7.191645in}}%
\pgfpathcurveto{\pgfqpoint{7.071441in}{7.191645in}}{\pgfqpoint{7.060842in}{7.187254in}}{\pgfqpoint{7.053028in}{7.179441in}}%
\pgfpathcurveto{\pgfqpoint{7.045215in}{7.171627in}}{\pgfqpoint{7.040825in}{7.161028in}}{\pgfqpoint{7.040825in}{7.149978in}}%
\pgfpathcurveto{\pgfqpoint{7.040825in}{7.138928in}}{\pgfqpoint{7.045215in}{7.128329in}}{\pgfqpoint{7.053028in}{7.120515in}}%
\pgfpathcurveto{\pgfqpoint{7.060842in}{7.112701in}}{\pgfqpoint{7.071441in}{7.108311in}}{\pgfqpoint{7.082491in}{7.108311in}}%
\pgfpathclose%
\pgfusepath{stroke,fill}%
\end{pgfscope}%
\begin{pgfscope}%
\pgfpathrectangle{\pgfqpoint{0.570343in}{0.331635in}}{\pgfqpoint{9.300000in}{7.700000in}}%
\pgfusepath{clip}%
\pgfsetbuttcap%
\pgfsetroundjoin%
\definecolor{currentfill}{rgb}{1.000000,0.705882,0.509804}%
\pgfsetfillcolor{currentfill}%
\pgfsetlinewidth{0.481800pt}%
\definecolor{currentstroke}{rgb}{1.000000,1.000000,1.000000}%
\pgfsetstrokecolor{currentstroke}%
\pgfsetdash{}{0pt}%
\pgfpathmoveto{\pgfqpoint{6.077373in}{2.291342in}}%
\pgfpathcurveto{\pgfqpoint{6.088423in}{2.291342in}}{\pgfqpoint{6.099022in}{2.295732in}}{\pgfqpoint{6.106836in}{2.303546in}}%
\pgfpathcurveto{\pgfqpoint{6.114649in}{2.311359in}}{\pgfqpoint{6.119040in}{2.321958in}}{\pgfqpoint{6.119040in}{2.333008in}}%
\pgfpathcurveto{\pgfqpoint{6.119040in}{2.344059in}}{\pgfqpoint{6.114649in}{2.354658in}}{\pgfqpoint{6.106836in}{2.362471in}}%
\pgfpathcurveto{\pgfqpoint{6.099022in}{2.370285in}}{\pgfqpoint{6.088423in}{2.374675in}}{\pgfqpoint{6.077373in}{2.374675in}}%
\pgfpathcurveto{\pgfqpoint{6.066323in}{2.374675in}}{\pgfqpoint{6.055724in}{2.370285in}}{\pgfqpoint{6.047910in}{2.362471in}}%
\pgfpathcurveto{\pgfqpoint{6.040096in}{2.354658in}}{\pgfqpoint{6.035706in}{2.344059in}}{\pgfqpoint{6.035706in}{2.333008in}}%
\pgfpathcurveto{\pgfqpoint{6.035706in}{2.321958in}}{\pgfqpoint{6.040096in}{2.311359in}}{\pgfqpoint{6.047910in}{2.303546in}}%
\pgfpathcurveto{\pgfqpoint{6.055724in}{2.295732in}}{\pgfqpoint{6.066323in}{2.291342in}}{\pgfqpoint{6.077373in}{2.291342in}}%
\pgfpathclose%
\pgfusepath{stroke,fill}%
\end{pgfscope}%
\begin{pgfscope}%
\pgfpathrectangle{\pgfqpoint{0.570343in}{0.331635in}}{\pgfqpoint{9.300000in}{7.700000in}}%
\pgfusepath{clip}%
\pgfsetbuttcap%
\pgfsetroundjoin%
\definecolor{currentfill}{rgb}{1.000000,0.705882,0.509804}%
\pgfsetfillcolor{currentfill}%
\pgfsetlinewidth{0.481800pt}%
\definecolor{currentstroke}{rgb}{1.000000,1.000000,1.000000}%
\pgfsetstrokecolor{currentstroke}%
\pgfsetdash{}{0pt}%
\pgfpathmoveto{\pgfqpoint{4.868813in}{0.938917in}}%
\pgfpathcurveto{\pgfqpoint{4.879863in}{0.938917in}}{\pgfqpoint{4.890462in}{0.943308in}}{\pgfqpoint{4.898276in}{0.951121in}}%
\pgfpathcurveto{\pgfqpoint{4.906089in}{0.958935in}}{\pgfqpoint{4.910480in}{0.969534in}}{\pgfqpoint{4.910480in}{0.980584in}}%
\pgfpathcurveto{\pgfqpoint{4.910480in}{0.991634in}}{\pgfqpoint{4.906089in}{1.002233in}}{\pgfqpoint{4.898276in}{1.010047in}}%
\pgfpathcurveto{\pgfqpoint{4.890462in}{1.017860in}}{\pgfqpoint{4.879863in}{1.022251in}}{\pgfqpoint{4.868813in}{1.022251in}}%
\pgfpathcurveto{\pgfqpoint{4.857763in}{1.022251in}}{\pgfqpoint{4.847164in}{1.017860in}}{\pgfqpoint{4.839350in}{1.010047in}}%
\pgfpathcurveto{\pgfqpoint{4.831537in}{1.002233in}}{\pgfqpoint{4.827146in}{0.991634in}}{\pgfqpoint{4.827146in}{0.980584in}}%
\pgfpathcurveto{\pgfqpoint{4.827146in}{0.969534in}}{\pgfqpoint{4.831537in}{0.958935in}}{\pgfqpoint{4.839350in}{0.951121in}}%
\pgfpathcurveto{\pgfqpoint{4.847164in}{0.943308in}}{\pgfqpoint{4.857763in}{0.938917in}}{\pgfqpoint{4.868813in}{0.938917in}}%
\pgfpathclose%
\pgfusepath{stroke,fill}%
\end{pgfscope}%
\begin{pgfscope}%
\pgfpathrectangle{\pgfqpoint{0.570343in}{0.331635in}}{\pgfqpoint{9.300000in}{7.700000in}}%
\pgfusepath{clip}%
\pgfsetbuttcap%
\pgfsetroundjoin%
\definecolor{currentfill}{rgb}{1.000000,0.705882,0.509804}%
\pgfsetfillcolor{currentfill}%
\pgfsetlinewidth{0.481800pt}%
\definecolor{currentstroke}{rgb}{1.000000,1.000000,1.000000}%
\pgfsetstrokecolor{currentstroke}%
\pgfsetdash{}{0pt}%
\pgfpathmoveto{\pgfqpoint{4.793914in}{4.747558in}}%
\pgfpathcurveto{\pgfqpoint{4.804964in}{4.747558in}}{\pgfqpoint{4.815564in}{4.751948in}}{\pgfqpoint{4.823377in}{4.759762in}}%
\pgfpathcurveto{\pgfqpoint{4.831191in}{4.767576in}}{\pgfqpoint{4.835581in}{4.778175in}}{\pgfqpoint{4.835581in}{4.789225in}}%
\pgfpathcurveto{\pgfqpoint{4.835581in}{4.800275in}}{\pgfqpoint{4.831191in}{4.810874in}}{\pgfqpoint{4.823377in}{4.818687in}}%
\pgfpathcurveto{\pgfqpoint{4.815564in}{4.826501in}}{\pgfqpoint{4.804964in}{4.830891in}}{\pgfqpoint{4.793914in}{4.830891in}}%
\pgfpathcurveto{\pgfqpoint{4.782864in}{4.830891in}}{\pgfqpoint{4.772265in}{4.826501in}}{\pgfqpoint{4.764452in}{4.818687in}}%
\pgfpathcurveto{\pgfqpoint{4.756638in}{4.810874in}}{\pgfqpoint{4.752248in}{4.800275in}}{\pgfqpoint{4.752248in}{4.789225in}}%
\pgfpathcurveto{\pgfqpoint{4.752248in}{4.778175in}}{\pgfqpoint{4.756638in}{4.767576in}}{\pgfqpoint{4.764452in}{4.759762in}}%
\pgfpathcurveto{\pgfqpoint{4.772265in}{4.751948in}}{\pgfqpoint{4.782864in}{4.747558in}}{\pgfqpoint{4.793914in}{4.747558in}}%
\pgfpathclose%
\pgfusepath{stroke,fill}%
\end{pgfscope}%
\begin{pgfscope}%
\pgfpathrectangle{\pgfqpoint{0.570343in}{0.331635in}}{\pgfqpoint{9.300000in}{7.700000in}}%
\pgfusepath{clip}%
\pgfsetbuttcap%
\pgfsetroundjoin%
\definecolor{currentfill}{rgb}{1.000000,0.705882,0.509804}%
\pgfsetfillcolor{currentfill}%
\pgfsetlinewidth{0.481800pt}%
\definecolor{currentstroke}{rgb}{1.000000,1.000000,1.000000}%
\pgfsetstrokecolor{currentstroke}%
\pgfsetdash{}{0pt}%
\pgfpathmoveto{\pgfqpoint{6.545799in}{6.354922in}}%
\pgfpathcurveto{\pgfqpoint{6.556849in}{6.354922in}}{\pgfqpoint{6.567448in}{6.359312in}}{\pgfqpoint{6.575262in}{6.367126in}}%
\pgfpathcurveto{\pgfqpoint{6.583076in}{6.374939in}}{\pgfqpoint{6.587466in}{6.385538in}}{\pgfqpoint{6.587466in}{6.396589in}}%
\pgfpathcurveto{\pgfqpoint{6.587466in}{6.407639in}}{\pgfqpoint{6.583076in}{6.418238in}}{\pgfqpoint{6.575262in}{6.426051in}}%
\pgfpathcurveto{\pgfqpoint{6.567448in}{6.433865in}}{\pgfqpoint{6.556849in}{6.438255in}}{\pgfqpoint{6.545799in}{6.438255in}}%
\pgfpathcurveto{\pgfqpoint{6.534749in}{6.438255in}}{\pgfqpoint{6.524150in}{6.433865in}}{\pgfqpoint{6.516336in}{6.426051in}}%
\pgfpathcurveto{\pgfqpoint{6.508523in}{6.418238in}}{\pgfqpoint{6.504132in}{6.407639in}}{\pgfqpoint{6.504132in}{6.396589in}}%
\pgfpathcurveto{\pgfqpoint{6.504132in}{6.385538in}}{\pgfqpoint{6.508523in}{6.374939in}}{\pgfqpoint{6.516336in}{6.367126in}}%
\pgfpathcurveto{\pgfqpoint{6.524150in}{6.359312in}}{\pgfqpoint{6.534749in}{6.354922in}}{\pgfqpoint{6.545799in}{6.354922in}}%
\pgfpathclose%
\pgfusepath{stroke,fill}%
\end{pgfscope}%
\begin{pgfscope}%
\pgfpathrectangle{\pgfqpoint{0.570343in}{0.331635in}}{\pgfqpoint{9.300000in}{7.700000in}}%
\pgfusepath{clip}%
\pgfsetbuttcap%
\pgfsetroundjoin%
\definecolor{currentfill}{rgb}{1.000000,0.705882,0.509804}%
\pgfsetfillcolor{currentfill}%
\pgfsetlinewidth{0.481800pt}%
\definecolor{currentstroke}{rgb}{1.000000,1.000000,1.000000}%
\pgfsetstrokecolor{currentstroke}%
\pgfsetdash{}{0pt}%
\pgfpathmoveto{\pgfqpoint{3.662589in}{2.790271in}}%
\pgfpathcurveto{\pgfqpoint{3.673640in}{2.790271in}}{\pgfqpoint{3.684239in}{2.794661in}}{\pgfqpoint{3.692052in}{2.802475in}}%
\pgfpathcurveto{\pgfqpoint{3.699866in}{2.810288in}}{\pgfqpoint{3.704256in}{2.820887in}}{\pgfqpoint{3.704256in}{2.831937in}}%
\pgfpathcurveto{\pgfqpoint{3.704256in}{2.842988in}}{\pgfqpoint{3.699866in}{2.853587in}}{\pgfqpoint{3.692052in}{2.861400in}}%
\pgfpathcurveto{\pgfqpoint{3.684239in}{2.869214in}}{\pgfqpoint{3.673640in}{2.873604in}}{\pgfqpoint{3.662589in}{2.873604in}}%
\pgfpathcurveto{\pgfqpoint{3.651539in}{2.873604in}}{\pgfqpoint{3.640940in}{2.869214in}}{\pgfqpoint{3.633127in}{2.861400in}}%
\pgfpathcurveto{\pgfqpoint{3.625313in}{2.853587in}}{\pgfqpoint{3.620923in}{2.842988in}}{\pgfqpoint{3.620923in}{2.831937in}}%
\pgfpathcurveto{\pgfqpoint{3.620923in}{2.820887in}}{\pgfqpoint{3.625313in}{2.810288in}}{\pgfqpoint{3.633127in}{2.802475in}}%
\pgfpathcurveto{\pgfqpoint{3.640940in}{2.794661in}}{\pgfqpoint{3.651539in}{2.790271in}}{\pgfqpoint{3.662589in}{2.790271in}}%
\pgfpathclose%
\pgfusepath{stroke,fill}%
\end{pgfscope}%
\begin{pgfscope}%
\pgfpathrectangle{\pgfqpoint{0.570343in}{0.331635in}}{\pgfqpoint{9.300000in}{7.700000in}}%
\pgfusepath{clip}%
\pgfsetbuttcap%
\pgfsetroundjoin%
\definecolor{currentfill}{rgb}{1.000000,0.705882,0.509804}%
\pgfsetfillcolor{currentfill}%
\pgfsetlinewidth{0.481800pt}%
\definecolor{currentstroke}{rgb}{1.000000,1.000000,1.000000}%
\pgfsetstrokecolor{currentstroke}%
\pgfsetdash{}{0pt}%
\pgfpathmoveto{\pgfqpoint{8.266806in}{7.400156in}}%
\pgfpathcurveto{\pgfqpoint{8.277856in}{7.400156in}}{\pgfqpoint{8.288455in}{7.404546in}}{\pgfqpoint{8.296269in}{7.412360in}}%
\pgfpathcurveto{\pgfqpoint{8.304083in}{7.420174in}}{\pgfqpoint{8.308473in}{7.430773in}}{\pgfqpoint{8.308473in}{7.441823in}}%
\pgfpathcurveto{\pgfqpoint{8.308473in}{7.452873in}}{\pgfqpoint{8.304083in}{7.463472in}}{\pgfqpoint{8.296269in}{7.471286in}}%
\pgfpathcurveto{\pgfqpoint{8.288455in}{7.479099in}}{\pgfqpoint{8.277856in}{7.483489in}}{\pgfqpoint{8.266806in}{7.483489in}}%
\pgfpathcurveto{\pgfqpoint{8.255756in}{7.483489in}}{\pgfqpoint{8.245157in}{7.479099in}}{\pgfqpoint{8.237343in}{7.471286in}}%
\pgfpathcurveto{\pgfqpoint{8.229530in}{7.463472in}}{\pgfqpoint{8.225140in}{7.452873in}}{\pgfqpoint{8.225140in}{7.441823in}}%
\pgfpathcurveto{\pgfqpoint{8.225140in}{7.430773in}}{\pgfqpoint{8.229530in}{7.420174in}}{\pgfqpoint{8.237343in}{7.412360in}}%
\pgfpathcurveto{\pgfqpoint{8.245157in}{7.404546in}}{\pgfqpoint{8.255756in}{7.400156in}}{\pgfqpoint{8.266806in}{7.400156in}}%
\pgfpathclose%
\pgfusepath{stroke,fill}%
\end{pgfscope}%
\begin{pgfscope}%
\pgfpathrectangle{\pgfqpoint{0.570343in}{0.331635in}}{\pgfqpoint{9.300000in}{7.700000in}}%
\pgfusepath{clip}%
\pgfsetbuttcap%
\pgfsetroundjoin%
\definecolor{currentfill}{rgb}{1.000000,0.705882,0.509804}%
\pgfsetfillcolor{currentfill}%
\pgfsetlinewidth{0.481800pt}%
\definecolor{currentstroke}{rgb}{1.000000,1.000000,1.000000}%
\pgfsetstrokecolor{currentstroke}%
\pgfsetdash{}{0pt}%
\pgfpathmoveto{\pgfqpoint{1.678869in}{5.939593in}}%
\pgfpathcurveto{\pgfqpoint{1.689919in}{5.939593in}}{\pgfqpoint{1.700518in}{5.943983in}}{\pgfqpoint{1.708332in}{5.951797in}}%
\pgfpathcurveto{\pgfqpoint{1.716145in}{5.959611in}}{\pgfqpoint{1.720536in}{5.970210in}}{\pgfqpoint{1.720536in}{5.981260in}}%
\pgfpathcurveto{\pgfqpoint{1.720536in}{5.992310in}}{\pgfqpoint{1.716145in}{6.002909in}}{\pgfqpoint{1.708332in}{6.010722in}}%
\pgfpathcurveto{\pgfqpoint{1.700518in}{6.018536in}}{\pgfqpoint{1.689919in}{6.022926in}}{\pgfqpoint{1.678869in}{6.022926in}}%
\pgfpathcurveto{\pgfqpoint{1.667819in}{6.022926in}}{\pgfqpoint{1.657220in}{6.018536in}}{\pgfqpoint{1.649406in}{6.010722in}}%
\pgfpathcurveto{\pgfqpoint{1.641593in}{6.002909in}}{\pgfqpoint{1.637202in}{5.992310in}}{\pgfqpoint{1.637202in}{5.981260in}}%
\pgfpathcurveto{\pgfqpoint{1.637202in}{5.970210in}}{\pgfqpoint{1.641593in}{5.959611in}}{\pgfqpoint{1.649406in}{5.951797in}}%
\pgfpathcurveto{\pgfqpoint{1.657220in}{5.943983in}}{\pgfqpoint{1.667819in}{5.939593in}}{\pgfqpoint{1.678869in}{5.939593in}}%
\pgfpathclose%
\pgfusepath{stroke,fill}%
\end{pgfscope}%
\begin{pgfscope}%
\pgfpathrectangle{\pgfqpoint{0.570343in}{0.331635in}}{\pgfqpoint{9.300000in}{7.700000in}}%
\pgfusepath{clip}%
\pgfsetbuttcap%
\pgfsetroundjoin%
\definecolor{currentfill}{rgb}{1.000000,0.705882,0.509804}%
\pgfsetfillcolor{currentfill}%
\pgfsetlinewidth{0.481800pt}%
\definecolor{currentstroke}{rgb}{1.000000,1.000000,1.000000}%
\pgfsetstrokecolor{currentstroke}%
\pgfsetdash{}{0pt}%
\pgfpathmoveto{\pgfqpoint{8.307790in}{4.537423in}}%
\pgfpathcurveto{\pgfqpoint{8.318840in}{4.537423in}}{\pgfqpoint{8.329439in}{4.541813in}}{\pgfqpoint{8.337253in}{4.549627in}}%
\pgfpathcurveto{\pgfqpoint{8.345066in}{4.557440in}}{\pgfqpoint{8.349457in}{4.568039in}}{\pgfqpoint{8.349457in}{4.579090in}}%
\pgfpathcurveto{\pgfqpoint{8.349457in}{4.590140in}}{\pgfqpoint{8.345066in}{4.600739in}}{\pgfqpoint{8.337253in}{4.608552in}}%
\pgfpathcurveto{\pgfqpoint{8.329439in}{4.616366in}}{\pgfqpoint{8.318840in}{4.620756in}}{\pgfqpoint{8.307790in}{4.620756in}}%
\pgfpathcurveto{\pgfqpoint{8.296740in}{4.620756in}}{\pgfqpoint{8.286141in}{4.616366in}}{\pgfqpoint{8.278327in}{4.608552in}}%
\pgfpathcurveto{\pgfqpoint{8.270514in}{4.600739in}}{\pgfqpoint{8.266123in}{4.590140in}}{\pgfqpoint{8.266123in}{4.579090in}}%
\pgfpathcurveto{\pgfqpoint{8.266123in}{4.568039in}}{\pgfqpoint{8.270514in}{4.557440in}}{\pgfqpoint{8.278327in}{4.549627in}}%
\pgfpathcurveto{\pgfqpoint{8.286141in}{4.541813in}}{\pgfqpoint{8.296740in}{4.537423in}}{\pgfqpoint{8.307790in}{4.537423in}}%
\pgfpathclose%
\pgfusepath{stroke,fill}%
\end{pgfscope}%
\begin{pgfscope}%
\pgfpathrectangle{\pgfqpoint{0.570343in}{0.331635in}}{\pgfqpoint{9.300000in}{7.700000in}}%
\pgfusepath{clip}%
\pgfsetbuttcap%
\pgfsetroundjoin%
\definecolor{currentfill}{rgb}{1.000000,0.705882,0.509804}%
\pgfsetfillcolor{currentfill}%
\pgfsetlinewidth{0.481800pt}%
\definecolor{currentstroke}{rgb}{1.000000,1.000000,1.000000}%
\pgfsetstrokecolor{currentstroke}%
\pgfsetdash{}{0pt}%
\pgfpathmoveto{\pgfqpoint{7.187407in}{5.783280in}}%
\pgfpathcurveto{\pgfqpoint{7.198457in}{5.783280in}}{\pgfqpoint{7.209056in}{5.787670in}}{\pgfqpoint{7.216870in}{5.795483in}}%
\pgfpathcurveto{\pgfqpoint{7.224683in}{5.803297in}}{\pgfqpoint{7.229074in}{5.813896in}}{\pgfqpoint{7.229074in}{5.824946in}}%
\pgfpathcurveto{\pgfqpoint{7.229074in}{5.835996in}}{\pgfqpoint{7.224683in}{5.846595in}}{\pgfqpoint{7.216870in}{5.854409in}}%
\pgfpathcurveto{\pgfqpoint{7.209056in}{5.862223in}}{\pgfqpoint{7.198457in}{5.866613in}}{\pgfqpoint{7.187407in}{5.866613in}}%
\pgfpathcurveto{\pgfqpoint{7.176357in}{5.866613in}}{\pgfqpoint{7.165758in}{5.862223in}}{\pgfqpoint{7.157944in}{5.854409in}}%
\pgfpathcurveto{\pgfqpoint{7.150131in}{5.846595in}}{\pgfqpoint{7.145740in}{5.835996in}}{\pgfqpoint{7.145740in}{5.824946in}}%
\pgfpathcurveto{\pgfqpoint{7.145740in}{5.813896in}}{\pgfqpoint{7.150131in}{5.803297in}}{\pgfqpoint{7.157944in}{5.795483in}}%
\pgfpathcurveto{\pgfqpoint{7.165758in}{5.787670in}}{\pgfqpoint{7.176357in}{5.783280in}}{\pgfqpoint{7.187407in}{5.783280in}}%
\pgfpathclose%
\pgfusepath{stroke,fill}%
\end{pgfscope}%
\begin{pgfscope}%
\pgfpathrectangle{\pgfqpoint{0.570343in}{0.331635in}}{\pgfqpoint{9.300000in}{7.700000in}}%
\pgfusepath{clip}%
\pgfsetbuttcap%
\pgfsetroundjoin%
\definecolor{currentfill}{rgb}{1.000000,0.705882,0.509804}%
\pgfsetfillcolor{currentfill}%
\pgfsetlinewidth{0.481800pt}%
\definecolor{currentstroke}{rgb}{1.000000,1.000000,1.000000}%
\pgfsetstrokecolor{currentstroke}%
\pgfsetdash{}{0pt}%
\pgfpathmoveto{\pgfqpoint{6.326121in}{1.360791in}}%
\pgfpathcurveto{\pgfqpoint{6.337172in}{1.360791in}}{\pgfqpoint{6.347771in}{1.365181in}}{\pgfqpoint{6.355584in}{1.372995in}}%
\pgfpathcurveto{\pgfqpoint{6.363398in}{1.380809in}}{\pgfqpoint{6.367788in}{1.391408in}}{\pgfqpoint{6.367788in}{1.402458in}}%
\pgfpathcurveto{\pgfqpoint{6.367788in}{1.413508in}}{\pgfqpoint{6.363398in}{1.424107in}}{\pgfqpoint{6.355584in}{1.431921in}}%
\pgfpathcurveto{\pgfqpoint{6.347771in}{1.439734in}}{\pgfqpoint{6.337172in}{1.444124in}}{\pgfqpoint{6.326121in}{1.444124in}}%
\pgfpathcurveto{\pgfqpoint{6.315071in}{1.444124in}}{\pgfqpoint{6.304472in}{1.439734in}}{\pgfqpoint{6.296659in}{1.431921in}}%
\pgfpathcurveto{\pgfqpoint{6.288845in}{1.424107in}}{\pgfqpoint{6.284455in}{1.413508in}}{\pgfqpoint{6.284455in}{1.402458in}}%
\pgfpathcurveto{\pgfqpoint{6.284455in}{1.391408in}}{\pgfqpoint{6.288845in}{1.380809in}}{\pgfqpoint{6.296659in}{1.372995in}}%
\pgfpathcurveto{\pgfqpoint{6.304472in}{1.365181in}}{\pgfqpoint{6.315071in}{1.360791in}}{\pgfqpoint{6.326121in}{1.360791in}}%
\pgfpathclose%
\pgfusepath{stroke,fill}%
\end{pgfscope}%
\begin{pgfscope}%
\pgfpathrectangle{\pgfqpoint{0.570343in}{0.331635in}}{\pgfqpoint{9.300000in}{7.700000in}}%
\pgfusepath{clip}%
\pgfsetbuttcap%
\pgfsetroundjoin%
\definecolor{currentfill}{rgb}{1.000000,0.705882,0.509804}%
\pgfsetfillcolor{currentfill}%
\pgfsetlinewidth{0.481800pt}%
\definecolor{currentstroke}{rgb}{1.000000,1.000000,1.000000}%
\pgfsetstrokecolor{currentstroke}%
\pgfsetdash{}{0pt}%
\pgfpathmoveto{\pgfqpoint{9.447616in}{4.615425in}}%
\pgfpathcurveto{\pgfqpoint{9.458666in}{4.615425in}}{\pgfqpoint{9.469265in}{4.619816in}}{\pgfqpoint{9.477079in}{4.627629in}}%
\pgfpathcurveto{\pgfqpoint{9.484892in}{4.635443in}}{\pgfqpoint{9.489283in}{4.646042in}}{\pgfqpoint{9.489283in}{4.657092in}}%
\pgfpathcurveto{\pgfqpoint{9.489283in}{4.668142in}}{\pgfqpoint{9.484892in}{4.678741in}}{\pgfqpoint{9.477079in}{4.686555in}}%
\pgfpathcurveto{\pgfqpoint{9.469265in}{4.694369in}}{\pgfqpoint{9.458666in}{4.698759in}}{\pgfqpoint{9.447616in}{4.698759in}}%
\pgfpathcurveto{\pgfqpoint{9.436566in}{4.698759in}}{\pgfqpoint{9.425967in}{4.694369in}}{\pgfqpoint{9.418153in}{4.686555in}}%
\pgfpathcurveto{\pgfqpoint{9.410340in}{4.678741in}}{\pgfqpoint{9.405949in}{4.668142in}}{\pgfqpoint{9.405949in}{4.657092in}}%
\pgfpathcurveto{\pgfqpoint{9.405949in}{4.646042in}}{\pgfqpoint{9.410340in}{4.635443in}}{\pgfqpoint{9.418153in}{4.627629in}}%
\pgfpathcurveto{\pgfqpoint{9.425967in}{4.619816in}}{\pgfqpoint{9.436566in}{4.615425in}}{\pgfqpoint{9.447616in}{4.615425in}}%
\pgfpathclose%
\pgfusepath{stroke,fill}%
\end{pgfscope}%
\begin{pgfscope}%
\pgfpathrectangle{\pgfqpoint{0.570343in}{0.331635in}}{\pgfqpoint{9.300000in}{7.700000in}}%
\pgfusepath{clip}%
\pgfsetbuttcap%
\pgfsetroundjoin%
\definecolor{currentfill}{rgb}{0.631373,0.788235,0.956863}%
\pgfsetfillcolor{currentfill}%
\pgfsetlinewidth{1.003750pt}%
\definecolor{currentstroke}{rgb}{0.631373,0.788235,0.956863}%
\pgfsetstrokecolor{currentstroke}%
\pgfsetdash{}{0pt}%
\pgfsys@defobject{currentmarker}{\pgfqpoint{-0.041667in}{-0.041667in}}{\pgfqpoint{0.041667in}{0.041667in}}{%
\pgfpathmoveto{\pgfqpoint{0.000000in}{-0.041667in}}%
\pgfpathcurveto{\pgfqpoint{0.011050in}{-0.041667in}}{\pgfqpoint{0.021649in}{-0.037276in}}{\pgfqpoint{0.029463in}{-0.029463in}}%
\pgfpathcurveto{\pgfqpoint{0.037276in}{-0.021649in}}{\pgfqpoint{0.041667in}{-0.011050in}}{\pgfqpoint{0.041667in}{0.000000in}}%
\pgfpathcurveto{\pgfqpoint{0.041667in}{0.011050in}}{\pgfqpoint{0.037276in}{0.021649in}}{\pgfqpoint{0.029463in}{0.029463in}}%
\pgfpathcurveto{\pgfqpoint{0.021649in}{0.037276in}}{\pgfqpoint{0.011050in}{0.041667in}}{\pgfqpoint{0.000000in}{0.041667in}}%
\pgfpathcurveto{\pgfqpoint{-0.011050in}{0.041667in}}{\pgfqpoint{-0.021649in}{0.037276in}}{\pgfqpoint{-0.029463in}{0.029463in}}%
\pgfpathcurveto{\pgfqpoint{-0.037276in}{0.021649in}}{\pgfqpoint{-0.041667in}{0.011050in}}{\pgfqpoint{-0.041667in}{0.000000in}}%
\pgfpathcurveto{\pgfqpoint{-0.041667in}{-0.011050in}}{\pgfqpoint{-0.037276in}{-0.021649in}}{\pgfqpoint{-0.029463in}{-0.029463in}}%
\pgfpathcurveto{\pgfqpoint{-0.021649in}{-0.037276in}}{\pgfqpoint{-0.011050in}{-0.041667in}}{\pgfqpoint{0.000000in}{-0.041667in}}%
\pgfpathclose%
\pgfusepath{stroke,fill}%
}%
\end{pgfscope}%
\begin{pgfscope}%
\pgfpathrectangle{\pgfqpoint{0.570343in}{0.331635in}}{\pgfqpoint{9.300000in}{7.700000in}}%
\pgfusepath{clip}%
\pgfsetbuttcap%
\pgfsetroundjoin%
\definecolor{currentfill}{rgb}{1.000000,0.705882,0.509804}%
\pgfsetfillcolor{currentfill}%
\pgfsetlinewidth{1.003750pt}%
\definecolor{currentstroke}{rgb}{1.000000,0.705882,0.509804}%
\pgfsetstrokecolor{currentstroke}%
\pgfsetdash{}{0pt}%
\pgfsys@defobject{currentmarker}{\pgfqpoint{-0.041667in}{-0.041667in}}{\pgfqpoint{0.041667in}{0.041667in}}{%
\pgfpathmoveto{\pgfqpoint{0.000000in}{-0.041667in}}%
\pgfpathcurveto{\pgfqpoint{0.011050in}{-0.041667in}}{\pgfqpoint{0.021649in}{-0.037276in}}{\pgfqpoint{0.029463in}{-0.029463in}}%
\pgfpathcurveto{\pgfqpoint{0.037276in}{-0.021649in}}{\pgfqpoint{0.041667in}{-0.011050in}}{\pgfqpoint{0.041667in}{0.000000in}}%
\pgfpathcurveto{\pgfqpoint{0.041667in}{0.011050in}}{\pgfqpoint{0.037276in}{0.021649in}}{\pgfqpoint{0.029463in}{0.029463in}}%
\pgfpathcurveto{\pgfqpoint{0.021649in}{0.037276in}}{\pgfqpoint{0.011050in}{0.041667in}}{\pgfqpoint{0.000000in}{0.041667in}}%
\pgfpathcurveto{\pgfqpoint{-0.011050in}{0.041667in}}{\pgfqpoint{-0.021649in}{0.037276in}}{\pgfqpoint{-0.029463in}{0.029463in}}%
\pgfpathcurveto{\pgfqpoint{-0.037276in}{0.021649in}}{\pgfqpoint{-0.041667in}{0.011050in}}{\pgfqpoint{-0.041667in}{0.000000in}}%
\pgfpathcurveto{\pgfqpoint{-0.041667in}{-0.011050in}}{\pgfqpoint{-0.037276in}{-0.021649in}}{\pgfqpoint{-0.029463in}{-0.029463in}}%
\pgfpathcurveto{\pgfqpoint{-0.021649in}{-0.037276in}}{\pgfqpoint{-0.011050in}{-0.041667in}}{\pgfqpoint{0.000000in}{-0.041667in}}%
\pgfpathclose%
\pgfusepath{stroke,fill}%
}%
\end{pgfscope}%
\begin{pgfscope}%
\pgfsetbuttcap%
\pgfsetroundjoin%
\definecolor{currentfill}{rgb}{0.000000,0.000000,0.000000}%
\pgfsetfillcolor{currentfill}%
\pgfsetlinewidth{0.803000pt}%
\definecolor{currentstroke}{rgb}{0.000000,0.000000,0.000000}%
\pgfsetstrokecolor{currentstroke}%
\pgfsetdash{}{0pt}%
\pgfsys@defobject{currentmarker}{\pgfqpoint{0.000000in}{-0.048611in}}{\pgfqpoint{0.000000in}{0.000000in}}{%
\pgfpathmoveto{\pgfqpoint{0.000000in}{0.000000in}}%
\pgfpathlineto{\pgfqpoint{0.000000in}{-0.048611in}}%
\pgfusepath{stroke,fill}%
}%
\begin{pgfscope}%
\pgfsys@transformshift{1.237108in}{0.331635in}%
\pgfsys@useobject{currentmarker}{}%
\end{pgfscope}%
\end{pgfscope}%
\begin{pgfscope}%
\definecolor{textcolor}{rgb}{0.000000,0.000000,0.000000}%
\pgfsetstrokecolor{textcolor}%
\pgfsetfillcolor{textcolor}%
\pgftext[x=1.237108in,y=0.234413in,,top]{\color{textcolor}\sffamily\fontsize{10.000000}{12.000000}\selectfont \ensuremath{-}80}%
\end{pgfscope}%
\begin{pgfscope}%
\pgfsetbuttcap%
\pgfsetroundjoin%
\definecolor{currentfill}{rgb}{0.000000,0.000000,0.000000}%
\pgfsetfillcolor{currentfill}%
\pgfsetlinewidth{0.803000pt}%
\definecolor{currentstroke}{rgb}{0.000000,0.000000,0.000000}%
\pgfsetstrokecolor{currentstroke}%
\pgfsetdash{}{0pt}%
\pgfsys@defobject{currentmarker}{\pgfqpoint{0.000000in}{-0.048611in}}{\pgfqpoint{0.000000in}{0.000000in}}{%
\pgfpathmoveto{\pgfqpoint{0.000000in}{0.000000in}}%
\pgfpathlineto{\pgfqpoint{0.000000in}{-0.048611in}}%
\pgfusepath{stroke,fill}%
}%
\begin{pgfscope}%
\pgfsys@transformshift{2.352555in}{0.331635in}%
\pgfsys@useobject{currentmarker}{}%
\end{pgfscope}%
\end{pgfscope}%
\begin{pgfscope}%
\definecolor{textcolor}{rgb}{0.000000,0.000000,0.000000}%
\pgfsetstrokecolor{textcolor}%
\pgfsetfillcolor{textcolor}%
\pgftext[x=2.352555in,y=0.234413in,,top]{\color{textcolor}\sffamily\fontsize{10.000000}{12.000000}\selectfont \ensuremath{-}60}%
\end{pgfscope}%
\begin{pgfscope}%
\pgfsetbuttcap%
\pgfsetroundjoin%
\definecolor{currentfill}{rgb}{0.000000,0.000000,0.000000}%
\pgfsetfillcolor{currentfill}%
\pgfsetlinewidth{0.803000pt}%
\definecolor{currentstroke}{rgb}{0.000000,0.000000,0.000000}%
\pgfsetstrokecolor{currentstroke}%
\pgfsetdash{}{0pt}%
\pgfsys@defobject{currentmarker}{\pgfqpoint{0.000000in}{-0.048611in}}{\pgfqpoint{0.000000in}{0.000000in}}{%
\pgfpathmoveto{\pgfqpoint{0.000000in}{0.000000in}}%
\pgfpathlineto{\pgfqpoint{0.000000in}{-0.048611in}}%
\pgfusepath{stroke,fill}%
}%
\begin{pgfscope}%
\pgfsys@transformshift{3.468002in}{0.331635in}%
\pgfsys@useobject{currentmarker}{}%
\end{pgfscope}%
\end{pgfscope}%
\begin{pgfscope}%
\definecolor{textcolor}{rgb}{0.000000,0.000000,0.000000}%
\pgfsetstrokecolor{textcolor}%
\pgfsetfillcolor{textcolor}%
\pgftext[x=3.468002in,y=0.234413in,,top]{\color{textcolor}\sffamily\fontsize{10.000000}{12.000000}\selectfont \ensuremath{-}40}%
\end{pgfscope}%
\begin{pgfscope}%
\pgfsetbuttcap%
\pgfsetroundjoin%
\definecolor{currentfill}{rgb}{0.000000,0.000000,0.000000}%
\pgfsetfillcolor{currentfill}%
\pgfsetlinewidth{0.803000pt}%
\definecolor{currentstroke}{rgb}{0.000000,0.000000,0.000000}%
\pgfsetstrokecolor{currentstroke}%
\pgfsetdash{}{0pt}%
\pgfsys@defobject{currentmarker}{\pgfqpoint{0.000000in}{-0.048611in}}{\pgfqpoint{0.000000in}{0.000000in}}{%
\pgfpathmoveto{\pgfqpoint{0.000000in}{0.000000in}}%
\pgfpathlineto{\pgfqpoint{0.000000in}{-0.048611in}}%
\pgfusepath{stroke,fill}%
}%
\begin{pgfscope}%
\pgfsys@transformshift{4.583449in}{0.331635in}%
\pgfsys@useobject{currentmarker}{}%
\end{pgfscope}%
\end{pgfscope}%
\begin{pgfscope}%
\definecolor{textcolor}{rgb}{0.000000,0.000000,0.000000}%
\pgfsetstrokecolor{textcolor}%
\pgfsetfillcolor{textcolor}%
\pgftext[x=4.583449in,y=0.234413in,,top]{\color{textcolor}\sffamily\fontsize{10.000000}{12.000000}\selectfont \ensuremath{-}20}%
\end{pgfscope}%
\begin{pgfscope}%
\pgfsetbuttcap%
\pgfsetroundjoin%
\definecolor{currentfill}{rgb}{0.000000,0.000000,0.000000}%
\pgfsetfillcolor{currentfill}%
\pgfsetlinewidth{0.803000pt}%
\definecolor{currentstroke}{rgb}{0.000000,0.000000,0.000000}%
\pgfsetstrokecolor{currentstroke}%
\pgfsetdash{}{0pt}%
\pgfsys@defobject{currentmarker}{\pgfqpoint{0.000000in}{-0.048611in}}{\pgfqpoint{0.000000in}{0.000000in}}{%
\pgfpathmoveto{\pgfqpoint{0.000000in}{0.000000in}}%
\pgfpathlineto{\pgfqpoint{0.000000in}{-0.048611in}}%
\pgfusepath{stroke,fill}%
}%
\begin{pgfscope}%
\pgfsys@transformshift{5.698896in}{0.331635in}%
\pgfsys@useobject{currentmarker}{}%
\end{pgfscope}%
\end{pgfscope}%
\begin{pgfscope}%
\definecolor{textcolor}{rgb}{0.000000,0.000000,0.000000}%
\pgfsetstrokecolor{textcolor}%
\pgfsetfillcolor{textcolor}%
\pgftext[x=5.698896in,y=0.234413in,,top]{\color{textcolor}\sffamily\fontsize{10.000000}{12.000000}\selectfont 0}%
\end{pgfscope}%
\begin{pgfscope}%
\pgfsetbuttcap%
\pgfsetroundjoin%
\definecolor{currentfill}{rgb}{0.000000,0.000000,0.000000}%
\pgfsetfillcolor{currentfill}%
\pgfsetlinewidth{0.803000pt}%
\definecolor{currentstroke}{rgb}{0.000000,0.000000,0.000000}%
\pgfsetstrokecolor{currentstroke}%
\pgfsetdash{}{0pt}%
\pgfsys@defobject{currentmarker}{\pgfqpoint{0.000000in}{-0.048611in}}{\pgfqpoint{0.000000in}{0.000000in}}{%
\pgfpathmoveto{\pgfqpoint{0.000000in}{0.000000in}}%
\pgfpathlineto{\pgfqpoint{0.000000in}{-0.048611in}}%
\pgfusepath{stroke,fill}%
}%
\begin{pgfscope}%
\pgfsys@transformshift{6.814343in}{0.331635in}%
\pgfsys@useobject{currentmarker}{}%
\end{pgfscope}%
\end{pgfscope}%
\begin{pgfscope}%
\definecolor{textcolor}{rgb}{0.000000,0.000000,0.000000}%
\pgfsetstrokecolor{textcolor}%
\pgfsetfillcolor{textcolor}%
\pgftext[x=6.814343in,y=0.234413in,,top]{\color{textcolor}\sffamily\fontsize{10.000000}{12.000000}\selectfont 20}%
\end{pgfscope}%
\begin{pgfscope}%
\pgfsetbuttcap%
\pgfsetroundjoin%
\definecolor{currentfill}{rgb}{0.000000,0.000000,0.000000}%
\pgfsetfillcolor{currentfill}%
\pgfsetlinewidth{0.803000pt}%
\definecolor{currentstroke}{rgb}{0.000000,0.000000,0.000000}%
\pgfsetstrokecolor{currentstroke}%
\pgfsetdash{}{0pt}%
\pgfsys@defobject{currentmarker}{\pgfqpoint{0.000000in}{-0.048611in}}{\pgfqpoint{0.000000in}{0.000000in}}{%
\pgfpathmoveto{\pgfqpoint{0.000000in}{0.000000in}}%
\pgfpathlineto{\pgfqpoint{0.000000in}{-0.048611in}}%
\pgfusepath{stroke,fill}%
}%
\begin{pgfscope}%
\pgfsys@transformshift{7.929790in}{0.331635in}%
\pgfsys@useobject{currentmarker}{}%
\end{pgfscope}%
\end{pgfscope}%
\begin{pgfscope}%
\definecolor{textcolor}{rgb}{0.000000,0.000000,0.000000}%
\pgfsetstrokecolor{textcolor}%
\pgfsetfillcolor{textcolor}%
\pgftext[x=7.929790in,y=0.234413in,,top]{\color{textcolor}\sffamily\fontsize{10.000000}{12.000000}\selectfont 40}%
\end{pgfscope}%
\begin{pgfscope}%
\pgfsetbuttcap%
\pgfsetroundjoin%
\definecolor{currentfill}{rgb}{0.000000,0.000000,0.000000}%
\pgfsetfillcolor{currentfill}%
\pgfsetlinewidth{0.803000pt}%
\definecolor{currentstroke}{rgb}{0.000000,0.000000,0.000000}%
\pgfsetstrokecolor{currentstroke}%
\pgfsetdash{}{0pt}%
\pgfsys@defobject{currentmarker}{\pgfqpoint{0.000000in}{-0.048611in}}{\pgfqpoint{0.000000in}{0.000000in}}{%
\pgfpathmoveto{\pgfqpoint{0.000000in}{0.000000in}}%
\pgfpathlineto{\pgfqpoint{0.000000in}{-0.048611in}}%
\pgfusepath{stroke,fill}%
}%
\begin{pgfscope}%
\pgfsys@transformshift{9.045237in}{0.331635in}%
\pgfsys@useobject{currentmarker}{}%
\end{pgfscope}%
\end{pgfscope}%
\begin{pgfscope}%
\definecolor{textcolor}{rgb}{0.000000,0.000000,0.000000}%
\pgfsetstrokecolor{textcolor}%
\pgfsetfillcolor{textcolor}%
\pgftext[x=9.045237in,y=0.234413in,,top]{\color{textcolor}\sffamily\fontsize{10.000000}{12.000000}\selectfont 60}%
\end{pgfscope}%
\begin{pgfscope}%
\pgfsetbuttcap%
\pgfsetroundjoin%
\definecolor{currentfill}{rgb}{0.000000,0.000000,0.000000}%
\pgfsetfillcolor{currentfill}%
\pgfsetlinewidth{0.803000pt}%
\definecolor{currentstroke}{rgb}{0.000000,0.000000,0.000000}%
\pgfsetstrokecolor{currentstroke}%
\pgfsetdash{}{0pt}%
\pgfsys@defobject{currentmarker}{\pgfqpoint{-0.048611in}{0.000000in}}{\pgfqpoint{-0.000000in}{0.000000in}}{%
\pgfpathmoveto{\pgfqpoint{-0.000000in}{0.000000in}}%
\pgfpathlineto{\pgfqpoint{-0.048611in}{0.000000in}}%
\pgfusepath{stroke,fill}%
}%
\begin{pgfscope}%
\pgfsys@transformshift{0.570343in}{0.580690in}%
\pgfsys@useobject{currentmarker}{}%
\end{pgfscope}%
\end{pgfscope}%
\begin{pgfscope}%
\definecolor{textcolor}{rgb}{0.000000,0.000000,0.000000}%
\pgfsetstrokecolor{textcolor}%
\pgfsetfillcolor{textcolor}%
\pgftext[x=0.100000in, y=0.527928in, left, base]{\color{textcolor}\sffamily\fontsize{10.000000}{12.000000}\selectfont \ensuremath{-}100}%
\end{pgfscope}%
\begin{pgfscope}%
\pgfsetbuttcap%
\pgfsetroundjoin%
\definecolor{currentfill}{rgb}{0.000000,0.000000,0.000000}%
\pgfsetfillcolor{currentfill}%
\pgfsetlinewidth{0.803000pt}%
\definecolor{currentstroke}{rgb}{0.000000,0.000000,0.000000}%
\pgfsetstrokecolor{currentstroke}%
\pgfsetdash{}{0pt}%
\pgfsys@defobject{currentmarker}{\pgfqpoint{-0.048611in}{0.000000in}}{\pgfqpoint{-0.000000in}{0.000000in}}{%
\pgfpathmoveto{\pgfqpoint{-0.000000in}{0.000000in}}%
\pgfpathlineto{\pgfqpoint{-0.048611in}{0.000000in}}%
\pgfusepath{stroke,fill}%
}%
\begin{pgfscope}%
\pgfsys@transformshift{0.570343in}{1.444910in}%
\pgfsys@useobject{currentmarker}{}%
\end{pgfscope}%
\end{pgfscope}%
\begin{pgfscope}%
\definecolor{textcolor}{rgb}{0.000000,0.000000,0.000000}%
\pgfsetstrokecolor{textcolor}%
\pgfsetfillcolor{textcolor}%
\pgftext[x=0.188365in, y=1.392149in, left, base]{\color{textcolor}\sffamily\fontsize{10.000000}{12.000000}\selectfont \ensuremath{-}75}%
\end{pgfscope}%
\begin{pgfscope}%
\pgfsetbuttcap%
\pgfsetroundjoin%
\definecolor{currentfill}{rgb}{0.000000,0.000000,0.000000}%
\pgfsetfillcolor{currentfill}%
\pgfsetlinewidth{0.803000pt}%
\definecolor{currentstroke}{rgb}{0.000000,0.000000,0.000000}%
\pgfsetstrokecolor{currentstroke}%
\pgfsetdash{}{0pt}%
\pgfsys@defobject{currentmarker}{\pgfqpoint{-0.048611in}{0.000000in}}{\pgfqpoint{-0.000000in}{0.000000in}}{%
\pgfpathmoveto{\pgfqpoint{-0.000000in}{0.000000in}}%
\pgfpathlineto{\pgfqpoint{-0.048611in}{0.000000in}}%
\pgfusepath{stroke,fill}%
}%
\begin{pgfscope}%
\pgfsys@transformshift{0.570343in}{2.309131in}%
\pgfsys@useobject{currentmarker}{}%
\end{pgfscope}%
\end{pgfscope}%
\begin{pgfscope}%
\definecolor{textcolor}{rgb}{0.000000,0.000000,0.000000}%
\pgfsetstrokecolor{textcolor}%
\pgfsetfillcolor{textcolor}%
\pgftext[x=0.188365in, y=2.256369in, left, base]{\color{textcolor}\sffamily\fontsize{10.000000}{12.000000}\selectfont \ensuremath{-}50}%
\end{pgfscope}%
\begin{pgfscope}%
\pgfsetbuttcap%
\pgfsetroundjoin%
\definecolor{currentfill}{rgb}{0.000000,0.000000,0.000000}%
\pgfsetfillcolor{currentfill}%
\pgfsetlinewidth{0.803000pt}%
\definecolor{currentstroke}{rgb}{0.000000,0.000000,0.000000}%
\pgfsetstrokecolor{currentstroke}%
\pgfsetdash{}{0pt}%
\pgfsys@defobject{currentmarker}{\pgfqpoint{-0.048611in}{0.000000in}}{\pgfqpoint{-0.000000in}{0.000000in}}{%
\pgfpathmoveto{\pgfqpoint{-0.000000in}{0.000000in}}%
\pgfpathlineto{\pgfqpoint{-0.048611in}{0.000000in}}%
\pgfusepath{stroke,fill}%
}%
\begin{pgfscope}%
\pgfsys@transformshift{0.570343in}{3.173352in}%
\pgfsys@useobject{currentmarker}{}%
\end{pgfscope}%
\end{pgfscope}%
\begin{pgfscope}%
\definecolor{textcolor}{rgb}{0.000000,0.000000,0.000000}%
\pgfsetstrokecolor{textcolor}%
\pgfsetfillcolor{textcolor}%
\pgftext[x=0.188365in, y=3.120590in, left, base]{\color{textcolor}\sffamily\fontsize{10.000000}{12.000000}\selectfont \ensuremath{-}25}%
\end{pgfscope}%
\begin{pgfscope}%
\pgfsetbuttcap%
\pgfsetroundjoin%
\definecolor{currentfill}{rgb}{0.000000,0.000000,0.000000}%
\pgfsetfillcolor{currentfill}%
\pgfsetlinewidth{0.803000pt}%
\definecolor{currentstroke}{rgb}{0.000000,0.000000,0.000000}%
\pgfsetstrokecolor{currentstroke}%
\pgfsetdash{}{0pt}%
\pgfsys@defobject{currentmarker}{\pgfqpoint{-0.048611in}{0.000000in}}{\pgfqpoint{-0.000000in}{0.000000in}}{%
\pgfpathmoveto{\pgfqpoint{-0.000000in}{0.000000in}}%
\pgfpathlineto{\pgfqpoint{-0.048611in}{0.000000in}}%
\pgfusepath{stroke,fill}%
}%
\begin{pgfscope}%
\pgfsys@transformshift{0.570343in}{4.037572in}%
\pgfsys@useobject{currentmarker}{}%
\end{pgfscope}%
\end{pgfscope}%
\begin{pgfscope}%
\definecolor{textcolor}{rgb}{0.000000,0.000000,0.000000}%
\pgfsetstrokecolor{textcolor}%
\pgfsetfillcolor{textcolor}%
\pgftext[x=0.384756in, y=3.984811in, left, base]{\color{textcolor}\sffamily\fontsize{10.000000}{12.000000}\selectfont 0}%
\end{pgfscope}%
\begin{pgfscope}%
\pgfsetbuttcap%
\pgfsetroundjoin%
\definecolor{currentfill}{rgb}{0.000000,0.000000,0.000000}%
\pgfsetfillcolor{currentfill}%
\pgfsetlinewidth{0.803000pt}%
\definecolor{currentstroke}{rgb}{0.000000,0.000000,0.000000}%
\pgfsetstrokecolor{currentstroke}%
\pgfsetdash{}{0pt}%
\pgfsys@defobject{currentmarker}{\pgfqpoint{-0.048611in}{0.000000in}}{\pgfqpoint{-0.000000in}{0.000000in}}{%
\pgfpathmoveto{\pgfqpoint{-0.000000in}{0.000000in}}%
\pgfpathlineto{\pgfqpoint{-0.048611in}{0.000000in}}%
\pgfusepath{stroke,fill}%
}%
\begin{pgfscope}%
\pgfsys@transformshift{0.570343in}{4.901793in}%
\pgfsys@useobject{currentmarker}{}%
\end{pgfscope}%
\end{pgfscope}%
\begin{pgfscope}%
\definecolor{textcolor}{rgb}{0.000000,0.000000,0.000000}%
\pgfsetstrokecolor{textcolor}%
\pgfsetfillcolor{textcolor}%
\pgftext[x=0.296390in, y=4.849031in, left, base]{\color{textcolor}\sffamily\fontsize{10.000000}{12.000000}\selectfont 25}%
\end{pgfscope}%
\begin{pgfscope}%
\pgfsetbuttcap%
\pgfsetroundjoin%
\definecolor{currentfill}{rgb}{0.000000,0.000000,0.000000}%
\pgfsetfillcolor{currentfill}%
\pgfsetlinewidth{0.803000pt}%
\definecolor{currentstroke}{rgb}{0.000000,0.000000,0.000000}%
\pgfsetstrokecolor{currentstroke}%
\pgfsetdash{}{0pt}%
\pgfsys@defobject{currentmarker}{\pgfqpoint{-0.048611in}{0.000000in}}{\pgfqpoint{-0.000000in}{0.000000in}}{%
\pgfpathmoveto{\pgfqpoint{-0.000000in}{0.000000in}}%
\pgfpathlineto{\pgfqpoint{-0.048611in}{0.000000in}}%
\pgfusepath{stroke,fill}%
}%
\begin{pgfscope}%
\pgfsys@transformshift{0.570343in}{5.766014in}%
\pgfsys@useobject{currentmarker}{}%
\end{pgfscope}%
\end{pgfscope}%
\begin{pgfscope}%
\definecolor{textcolor}{rgb}{0.000000,0.000000,0.000000}%
\pgfsetstrokecolor{textcolor}%
\pgfsetfillcolor{textcolor}%
\pgftext[x=0.296390in, y=5.713252in, left, base]{\color{textcolor}\sffamily\fontsize{10.000000}{12.000000}\selectfont 50}%
\end{pgfscope}%
\begin{pgfscope}%
\pgfsetbuttcap%
\pgfsetroundjoin%
\definecolor{currentfill}{rgb}{0.000000,0.000000,0.000000}%
\pgfsetfillcolor{currentfill}%
\pgfsetlinewidth{0.803000pt}%
\definecolor{currentstroke}{rgb}{0.000000,0.000000,0.000000}%
\pgfsetstrokecolor{currentstroke}%
\pgfsetdash{}{0pt}%
\pgfsys@defobject{currentmarker}{\pgfqpoint{-0.048611in}{0.000000in}}{\pgfqpoint{-0.000000in}{0.000000in}}{%
\pgfpathmoveto{\pgfqpoint{-0.000000in}{0.000000in}}%
\pgfpathlineto{\pgfqpoint{-0.048611in}{0.000000in}}%
\pgfusepath{stroke,fill}%
}%
\begin{pgfscope}%
\pgfsys@transformshift{0.570343in}{6.630234in}%
\pgfsys@useobject{currentmarker}{}%
\end{pgfscope}%
\end{pgfscope}%
\begin{pgfscope}%
\definecolor{textcolor}{rgb}{0.000000,0.000000,0.000000}%
\pgfsetstrokecolor{textcolor}%
\pgfsetfillcolor{textcolor}%
\pgftext[x=0.296390in, y=6.577473in, left, base]{\color{textcolor}\sffamily\fontsize{10.000000}{12.000000}\selectfont 75}%
\end{pgfscope}%
\begin{pgfscope}%
\pgfsetbuttcap%
\pgfsetroundjoin%
\definecolor{currentfill}{rgb}{0.000000,0.000000,0.000000}%
\pgfsetfillcolor{currentfill}%
\pgfsetlinewidth{0.803000pt}%
\definecolor{currentstroke}{rgb}{0.000000,0.000000,0.000000}%
\pgfsetstrokecolor{currentstroke}%
\pgfsetdash{}{0pt}%
\pgfsys@defobject{currentmarker}{\pgfqpoint{-0.048611in}{0.000000in}}{\pgfqpoint{-0.000000in}{0.000000in}}{%
\pgfpathmoveto{\pgfqpoint{-0.000000in}{0.000000in}}%
\pgfpathlineto{\pgfqpoint{-0.048611in}{0.000000in}}%
\pgfusepath{stroke,fill}%
}%
\begin{pgfscope}%
\pgfsys@transformshift{0.570343in}{7.494455in}%
\pgfsys@useobject{currentmarker}{}%
\end{pgfscope}%
\end{pgfscope}%
\begin{pgfscope}%
\definecolor{textcolor}{rgb}{0.000000,0.000000,0.000000}%
\pgfsetstrokecolor{textcolor}%
\pgfsetfillcolor{textcolor}%
\pgftext[x=0.208025in, y=7.441694in, left, base]{\color{textcolor}\sffamily\fontsize{10.000000}{12.000000}\selectfont 100}%
\end{pgfscope}%
\begin{pgfscope}%
\pgfpathrectangle{\pgfqpoint{0.570343in}{0.331635in}}{\pgfqpoint{9.300000in}{7.700000in}}%
\pgfusepath{clip}%
\pgfsetrectcap%
\pgfsetroundjoin%
\pgfsetlinewidth{1.505625pt}%
\definecolor{currentstroke}{rgb}{0.631373,0.788235,0.956863}%
\pgfsetstrokecolor{currentstroke}%
\pgfsetstrokeopacity{0.800000}%
\pgfsetdash{}{0pt}%
\pgfpathmoveto{\pgfqpoint{2.375246in}{6.939839in}}%
\pgfpathlineto{\pgfqpoint{4.244205in}{3.787232in}}%
\pgfusepath{stroke}%
\end{pgfscope}%
\begin{pgfscope}%
\pgfpathrectangle{\pgfqpoint{0.570343in}{0.331635in}}{\pgfqpoint{9.300000in}{7.700000in}}%
\pgfusepath{clip}%
\pgfsetrectcap%
\pgfsetroundjoin%
\pgfsetlinewidth{1.505625pt}%
\definecolor{currentstroke}{rgb}{0.631373,0.788235,0.956863}%
\pgfsetstrokecolor{currentstroke}%
\pgfsetstrokeopacity{0.800000}%
\pgfsetdash{}{0pt}%
\pgfpathmoveto{\pgfqpoint{3.492960in}{4.461367in}}%
\pgfpathlineto{\pgfqpoint{4.244205in}{3.787232in}}%
\pgfusepath{stroke}%
\end{pgfscope}%
\begin{pgfscope}%
\pgfpathrectangle{\pgfqpoint{0.570343in}{0.331635in}}{\pgfqpoint{9.300000in}{7.700000in}}%
\pgfusepath{clip}%
\pgfsetrectcap%
\pgfsetroundjoin%
\pgfsetlinewidth{1.505625pt}%
\definecolor{currentstroke}{rgb}{0.631373,0.788235,0.956863}%
\pgfsetstrokecolor{currentstroke}%
\pgfsetstrokeopacity{0.800000}%
\pgfsetdash{}{0pt}%
\pgfpathmoveto{\pgfqpoint{5.673563in}{5.072505in}}%
\pgfpathlineto{\pgfqpoint{4.244205in}{3.787232in}}%
\pgfusepath{stroke}%
\end{pgfscope}%
\begin{pgfscope}%
\pgfpathrectangle{\pgfqpoint{0.570343in}{0.331635in}}{\pgfqpoint{9.300000in}{7.700000in}}%
\pgfusepath{clip}%
\pgfsetrectcap%
\pgfsetroundjoin%
\pgfsetlinewidth{1.505625pt}%
\definecolor{currentstroke}{rgb}{0.631373,0.788235,0.956863}%
\pgfsetstrokecolor{currentstroke}%
\pgfsetstrokeopacity{0.800000}%
\pgfsetdash{}{0pt}%
\pgfpathmoveto{\pgfqpoint{3.420050in}{0.681635in}}%
\pgfpathlineto{\pgfqpoint{4.244205in}{3.787232in}}%
\pgfusepath{stroke}%
\end{pgfscope}%
\begin{pgfscope}%
\pgfpathrectangle{\pgfqpoint{0.570343in}{0.331635in}}{\pgfqpoint{9.300000in}{7.700000in}}%
\pgfusepath{clip}%
\pgfsetrectcap%
\pgfsetroundjoin%
\pgfsetlinewidth{1.505625pt}%
\definecolor{currentstroke}{rgb}{0.631373,0.788235,0.956863}%
\pgfsetstrokecolor{currentstroke}%
\pgfsetstrokeopacity{0.800000}%
\pgfsetdash{}{0pt}%
\pgfpathmoveto{\pgfqpoint{4.927031in}{7.681635in}}%
\pgfpathlineto{\pgfqpoint{4.244205in}{3.787232in}}%
\pgfusepath{stroke}%
\end{pgfscope}%
\begin{pgfscope}%
\pgfpathrectangle{\pgfqpoint{0.570343in}{0.331635in}}{\pgfqpoint{9.300000in}{7.700000in}}%
\pgfusepath{clip}%
\pgfsetrectcap%
\pgfsetroundjoin%
\pgfsetlinewidth{1.505625pt}%
\definecolor{currentstroke}{rgb}{0.631373,0.788235,0.956863}%
\pgfsetstrokecolor{currentstroke}%
\pgfsetstrokeopacity{0.800000}%
\pgfsetdash{}{0pt}%
\pgfpathmoveto{\pgfqpoint{8.854047in}{5.581742in}}%
\pgfpathlineto{\pgfqpoint{4.244205in}{3.787232in}}%
\pgfusepath{stroke}%
\end{pgfscope}%
\begin{pgfscope}%
\pgfpathrectangle{\pgfqpoint{0.570343in}{0.331635in}}{\pgfqpoint{9.300000in}{7.700000in}}%
\pgfusepath{clip}%
\pgfsetrectcap%
\pgfsetroundjoin%
\pgfsetlinewidth{1.505625pt}%
\definecolor{currentstroke}{rgb}{0.631373,0.788235,0.956863}%
\pgfsetstrokecolor{currentstroke}%
\pgfsetstrokeopacity{0.800000}%
\pgfsetdash{}{0pt}%
\pgfpathmoveto{\pgfqpoint{1.547559in}{4.004022in}}%
\pgfpathlineto{\pgfqpoint{4.244205in}{3.787232in}}%
\pgfusepath{stroke}%
\end{pgfscope}%
\begin{pgfscope}%
\pgfpathrectangle{\pgfqpoint{0.570343in}{0.331635in}}{\pgfqpoint{9.300000in}{7.700000in}}%
\pgfusepath{clip}%
\pgfsetrectcap%
\pgfsetroundjoin%
\pgfsetlinewidth{1.505625pt}%
\definecolor{currentstroke}{rgb}{0.631373,0.788235,0.956863}%
\pgfsetstrokecolor{currentstroke}%
\pgfsetstrokeopacity{0.800000}%
\pgfsetdash{}{0pt}%
\pgfpathmoveto{\pgfqpoint{2.739938in}{3.714075in}}%
\pgfpathlineto{\pgfqpoint{4.244205in}{3.787232in}}%
\pgfusepath{stroke}%
\end{pgfscope}%
\begin{pgfscope}%
\pgfpathrectangle{\pgfqpoint{0.570343in}{0.331635in}}{\pgfqpoint{9.300000in}{7.700000in}}%
\pgfusepath{clip}%
\pgfsetrectcap%
\pgfsetroundjoin%
\pgfsetlinewidth{1.505625pt}%
\definecolor{currentstroke}{rgb}{0.631373,0.788235,0.956863}%
\pgfsetstrokecolor{currentstroke}%
\pgfsetstrokeopacity{0.800000}%
\pgfsetdash{}{0pt}%
\pgfpathmoveto{\pgfqpoint{7.703434in}{5.240049in}}%
\pgfpathlineto{\pgfqpoint{4.244205in}{3.787232in}}%
\pgfusepath{stroke}%
\end{pgfscope}%
\begin{pgfscope}%
\pgfpathrectangle{\pgfqpoint{0.570343in}{0.331635in}}{\pgfqpoint{9.300000in}{7.700000in}}%
\pgfusepath{clip}%
\pgfsetrectcap%
\pgfsetroundjoin%
\pgfsetlinewidth{1.505625pt}%
\definecolor{currentstroke}{rgb}{0.631373,0.788235,0.956863}%
\pgfsetstrokecolor{currentstroke}%
\pgfsetstrokeopacity{0.800000}%
\pgfsetdash{}{0pt}%
\pgfpathmoveto{\pgfqpoint{7.113284in}{2.811066in}}%
\pgfpathlineto{\pgfqpoint{4.244205in}{3.787232in}}%
\pgfusepath{stroke}%
\end{pgfscope}%
\begin{pgfscope}%
\pgfpathrectangle{\pgfqpoint{0.570343in}{0.331635in}}{\pgfqpoint{9.300000in}{7.700000in}}%
\pgfusepath{clip}%
\pgfsetrectcap%
\pgfsetroundjoin%
\pgfsetlinewidth{1.505625pt}%
\definecolor{currentstroke}{rgb}{0.631373,0.788235,0.956863}%
\pgfsetstrokecolor{currentstroke}%
\pgfsetstrokeopacity{0.800000}%
\pgfsetdash{}{0pt}%
\pgfpathmoveto{\pgfqpoint{2.082625in}{2.131804in}}%
\pgfpathlineto{\pgfqpoint{4.244205in}{3.787232in}}%
\pgfusepath{stroke}%
\end{pgfscope}%
\begin{pgfscope}%
\pgfpathrectangle{\pgfqpoint{0.570343in}{0.331635in}}{\pgfqpoint{9.300000in}{7.700000in}}%
\pgfusepath{clip}%
\pgfsetrectcap%
\pgfsetroundjoin%
\pgfsetlinewidth{1.505625pt}%
\definecolor{currentstroke}{rgb}{0.631373,0.788235,0.956863}%
\pgfsetstrokecolor{currentstroke}%
\pgfsetstrokeopacity{0.800000}%
\pgfsetdash{}{0pt}%
\pgfpathmoveto{\pgfqpoint{8.230336in}{6.133416in}}%
\pgfpathlineto{\pgfqpoint{4.244205in}{3.787232in}}%
\pgfusepath{stroke}%
\end{pgfscope}%
\begin{pgfscope}%
\pgfpathrectangle{\pgfqpoint{0.570343in}{0.331635in}}{\pgfqpoint{9.300000in}{7.700000in}}%
\pgfusepath{clip}%
\pgfsetrectcap%
\pgfsetroundjoin%
\pgfsetlinewidth{1.505625pt}%
\definecolor{currentstroke}{rgb}{0.631373,0.788235,0.956863}%
\pgfsetstrokecolor{currentstroke}%
\pgfsetstrokeopacity{0.800000}%
\pgfsetdash{}{0pt}%
\pgfpathmoveto{\pgfqpoint{4.884974in}{2.630620in}}%
\pgfpathlineto{\pgfqpoint{4.244205in}{3.787232in}}%
\pgfusepath{stroke}%
\end{pgfscope}%
\begin{pgfscope}%
\pgfpathrectangle{\pgfqpoint{0.570343in}{0.331635in}}{\pgfqpoint{9.300000in}{7.700000in}}%
\pgfusepath{clip}%
\pgfsetrectcap%
\pgfsetroundjoin%
\pgfsetlinewidth{1.505625pt}%
\definecolor{currentstroke}{rgb}{0.631373,0.788235,0.956863}%
\pgfsetstrokecolor{currentstroke}%
\pgfsetstrokeopacity{0.800000}%
\pgfsetdash{}{0pt}%
\pgfpathmoveto{\pgfqpoint{1.225108in}{2.782742in}}%
\pgfpathlineto{\pgfqpoint{4.244205in}{3.787232in}}%
\pgfusepath{stroke}%
\end{pgfscope}%
\begin{pgfscope}%
\pgfpathrectangle{\pgfqpoint{0.570343in}{0.331635in}}{\pgfqpoint{9.300000in}{7.700000in}}%
\pgfusepath{clip}%
\pgfsetrectcap%
\pgfsetroundjoin%
\pgfsetlinewidth{1.505625pt}%
\definecolor{currentstroke}{rgb}{0.631373,0.788235,0.956863}%
\pgfsetstrokecolor{currentstroke}%
\pgfsetstrokeopacity{0.800000}%
\pgfsetdash{}{0pt}%
\pgfpathmoveto{\pgfqpoint{3.946393in}{3.478730in}}%
\pgfpathlineto{\pgfqpoint{4.244205in}{3.787232in}}%
\pgfusepath{stroke}%
\end{pgfscope}%
\begin{pgfscope}%
\pgfpathrectangle{\pgfqpoint{0.570343in}{0.331635in}}{\pgfqpoint{9.300000in}{7.700000in}}%
\pgfusepath{clip}%
\pgfsetrectcap%
\pgfsetroundjoin%
\pgfsetlinewidth{1.505625pt}%
\definecolor{currentstroke}{rgb}{0.631373,0.788235,0.956863}%
\pgfsetstrokecolor{currentstroke}%
\pgfsetstrokeopacity{0.800000}%
\pgfsetdash{}{0pt}%
\pgfpathmoveto{\pgfqpoint{4.886393in}{1.889784in}}%
\pgfpathlineto{\pgfqpoint{4.244205in}{3.787232in}}%
\pgfusepath{stroke}%
\end{pgfscope}%
\begin{pgfscope}%
\pgfpathrectangle{\pgfqpoint{0.570343in}{0.331635in}}{\pgfqpoint{9.300000in}{7.700000in}}%
\pgfusepath{clip}%
\pgfsetrectcap%
\pgfsetroundjoin%
\pgfsetlinewidth{1.505625pt}%
\definecolor{currentstroke}{rgb}{0.631373,0.788235,0.956863}%
\pgfsetstrokecolor{currentstroke}%
\pgfsetstrokeopacity{0.800000}%
\pgfsetdash{}{0pt}%
\pgfpathmoveto{\pgfqpoint{3.784493in}{1.723901in}}%
\pgfpathlineto{\pgfqpoint{4.244205in}{3.787232in}}%
\pgfusepath{stroke}%
\end{pgfscope}%
\begin{pgfscope}%
\pgfpathrectangle{\pgfqpoint{0.570343in}{0.331635in}}{\pgfqpoint{9.300000in}{7.700000in}}%
\pgfusepath{clip}%
\pgfsetrectcap%
\pgfsetroundjoin%
\pgfsetlinewidth{1.505625pt}%
\definecolor{currentstroke}{rgb}{0.631373,0.788235,0.956863}%
\pgfsetstrokecolor{currentstroke}%
\pgfsetstrokeopacity{0.800000}%
\pgfsetdash{}{0pt}%
\pgfpathmoveto{\pgfqpoint{2.706989in}{2.534852in}}%
\pgfpathlineto{\pgfqpoint{4.244205in}{3.787232in}}%
\pgfusepath{stroke}%
\end{pgfscope}%
\begin{pgfscope}%
\pgfpathrectangle{\pgfqpoint{0.570343in}{0.331635in}}{\pgfqpoint{9.300000in}{7.700000in}}%
\pgfusepath{clip}%
\pgfsetrectcap%
\pgfsetroundjoin%
\pgfsetlinewidth{1.505625pt}%
\definecolor{currentstroke}{rgb}{0.631373,0.788235,0.956863}%
\pgfsetstrokecolor{currentstroke}%
\pgfsetstrokeopacity{0.800000}%
\pgfsetdash{}{0pt}%
\pgfpathmoveto{\pgfqpoint{3.495356in}{4.121344in}}%
\pgfpathlineto{\pgfqpoint{4.244205in}{3.787232in}}%
\pgfusepath{stroke}%
\end{pgfscope}%
\begin{pgfscope}%
\pgfpathrectangle{\pgfqpoint{0.570343in}{0.331635in}}{\pgfqpoint{9.300000in}{7.700000in}}%
\pgfusepath{clip}%
\pgfsetrectcap%
\pgfsetroundjoin%
\pgfsetlinewidth{1.505625pt}%
\definecolor{currentstroke}{rgb}{0.631373,0.788235,0.956863}%
\pgfsetstrokecolor{currentstroke}%
\pgfsetstrokeopacity{0.800000}%
\pgfsetdash{}{0pt}%
\pgfpathmoveto{\pgfqpoint{0.993071in}{4.546356in}}%
\pgfpathlineto{\pgfqpoint{4.244205in}{3.787232in}}%
\pgfusepath{stroke}%
\end{pgfscope}%
\begin{pgfscope}%
\pgfpathrectangle{\pgfqpoint{0.570343in}{0.331635in}}{\pgfqpoint{9.300000in}{7.700000in}}%
\pgfusepath{clip}%
\pgfsetrectcap%
\pgfsetroundjoin%
\pgfsetlinewidth{1.505625pt}%
\definecolor{currentstroke}{rgb}{0.631373,0.788235,0.956863}%
\pgfsetstrokecolor{currentstroke}%
\pgfsetstrokeopacity{0.800000}%
\pgfsetdash{}{0pt}%
\pgfpathmoveto{\pgfqpoint{2.362144in}{4.894987in}}%
\pgfpathlineto{\pgfqpoint{4.244205in}{3.787232in}}%
\pgfusepath{stroke}%
\end{pgfscope}%
\begin{pgfscope}%
\pgfpathrectangle{\pgfqpoint{0.570343in}{0.331635in}}{\pgfqpoint{9.300000in}{7.700000in}}%
\pgfusepath{clip}%
\pgfsetrectcap%
\pgfsetroundjoin%
\pgfsetlinewidth{1.505625pt}%
\definecolor{currentstroke}{rgb}{0.631373,0.788235,0.956863}%
\pgfsetstrokecolor{currentstroke}%
\pgfsetstrokeopacity{0.800000}%
\pgfsetdash{}{0pt}%
\pgfpathmoveto{\pgfqpoint{4.437533in}{3.963128in}}%
\pgfpathlineto{\pgfqpoint{4.244205in}{3.787232in}}%
\pgfusepath{stroke}%
\end{pgfscope}%
\begin{pgfscope}%
\pgfpathrectangle{\pgfqpoint{0.570343in}{0.331635in}}{\pgfqpoint{9.300000in}{7.700000in}}%
\pgfusepath{clip}%
\pgfsetrectcap%
\pgfsetroundjoin%
\pgfsetlinewidth{1.505625pt}%
\definecolor{currentstroke}{rgb}{0.631373,0.788235,0.956863}%
\pgfsetstrokecolor{currentstroke}%
\pgfsetstrokeopacity{0.800000}%
\pgfsetdash{}{0pt}%
\pgfpathmoveto{\pgfqpoint{3.510625in}{5.173625in}}%
\pgfpathlineto{\pgfqpoint{4.244205in}{3.787232in}}%
\pgfusepath{stroke}%
\end{pgfscope}%
\begin{pgfscope}%
\pgfpathrectangle{\pgfqpoint{0.570343in}{0.331635in}}{\pgfqpoint{9.300000in}{7.700000in}}%
\pgfusepath{clip}%
\pgfsetrectcap%
\pgfsetroundjoin%
\pgfsetlinewidth{1.505625pt}%
\definecolor{currentstroke}{rgb}{0.631373,0.788235,0.956863}%
\pgfsetstrokecolor{currentstroke}%
\pgfsetstrokeopacity{0.800000}%
\pgfsetdash{}{0pt}%
\pgfpathmoveto{\pgfqpoint{4.054617in}{2.299379in}}%
\pgfpathlineto{\pgfqpoint{4.244205in}{3.787232in}}%
\pgfusepath{stroke}%
\end{pgfscope}%
\begin{pgfscope}%
\pgfpathrectangle{\pgfqpoint{0.570343in}{0.331635in}}{\pgfqpoint{9.300000in}{7.700000in}}%
\pgfusepath{clip}%
\pgfsetrectcap%
\pgfsetroundjoin%
\pgfsetlinewidth{1.505625pt}%
\definecolor{currentstroke}{rgb}{0.631373,0.788235,0.956863}%
\pgfsetstrokecolor{currentstroke}%
\pgfsetstrokeopacity{0.800000}%
\pgfsetdash{}{0pt}%
\pgfpathmoveto{\pgfqpoint{4.954557in}{3.346327in}}%
\pgfpathlineto{\pgfqpoint{4.244205in}{3.787232in}}%
\pgfusepath{stroke}%
\end{pgfscope}%
\begin{pgfscope}%
\pgfpathrectangle{\pgfqpoint{0.570343in}{0.331635in}}{\pgfqpoint{9.300000in}{7.700000in}}%
\pgfusepath{clip}%
\pgfsetrectcap%
\pgfsetroundjoin%
\pgfsetlinewidth{1.505625pt}%
\definecolor{currentstroke}{rgb}{0.631373,0.788235,0.956863}%
\pgfsetstrokecolor{currentstroke}%
\pgfsetstrokeopacity{0.800000}%
\pgfsetdash{}{0pt}%
\pgfpathmoveto{\pgfqpoint{2.539065in}{1.397152in}}%
\pgfpathlineto{\pgfqpoint{4.244205in}{3.787232in}}%
\pgfusepath{stroke}%
\end{pgfscope}%
\begin{pgfscope}%
\pgfpathrectangle{\pgfqpoint{0.570343in}{0.331635in}}{\pgfqpoint{9.300000in}{7.700000in}}%
\pgfusepath{clip}%
\pgfsetrectcap%
\pgfsetroundjoin%
\pgfsetlinewidth{1.505625pt}%
\definecolor{currentstroke}{rgb}{0.631373,0.788235,0.956863}%
\pgfsetstrokecolor{currentstroke}%
\pgfsetstrokeopacity{0.800000}%
\pgfsetdash{}{0pt}%
\pgfpathmoveto{\pgfqpoint{6.972947in}{3.710554in}}%
\pgfpathlineto{\pgfqpoint{4.244205in}{3.787232in}}%
\pgfusepath{stroke}%
\end{pgfscope}%
\begin{pgfscope}%
\pgfpathrectangle{\pgfqpoint{0.570343in}{0.331635in}}{\pgfqpoint{9.300000in}{7.700000in}}%
\pgfusepath{clip}%
\pgfsetrectcap%
\pgfsetroundjoin%
\pgfsetlinewidth{1.505625pt}%
\definecolor{currentstroke}{rgb}{0.631373,0.788235,0.956863}%
\pgfsetstrokecolor{currentstroke}%
\pgfsetstrokeopacity{0.800000}%
\pgfsetdash{}{0pt}%
\pgfpathmoveto{\pgfqpoint{5.923406in}{3.095871in}}%
\pgfpathlineto{\pgfqpoint{4.244205in}{3.787232in}}%
\pgfusepath{stroke}%
\end{pgfscope}%
\begin{pgfscope}%
\pgfpathrectangle{\pgfqpoint{0.570343in}{0.331635in}}{\pgfqpoint{9.300000in}{7.700000in}}%
\pgfusepath{clip}%
\pgfsetrectcap%
\pgfsetroundjoin%
\pgfsetlinewidth{1.505625pt}%
\definecolor{currentstroke}{rgb}{1.000000,0.705882,0.509804}%
\pgfsetstrokecolor{currentstroke}%
\pgfsetstrokeopacity{0.800000}%
\pgfsetdash{}{0pt}%
\pgfpathmoveto{\pgfqpoint{6.817529in}{4.835569in}}%
\pgfpathlineto{\pgfqpoint{6.396515in}{4.497368in}}%
\pgfusepath{stroke}%
\end{pgfscope}%
\begin{pgfscope}%
\pgfpathrectangle{\pgfqpoint{0.570343in}{0.331635in}}{\pgfqpoint{9.300000in}{7.700000in}}%
\pgfusepath{clip}%
\pgfsetrectcap%
\pgfsetroundjoin%
\pgfsetlinewidth{1.505625pt}%
\definecolor{currentstroke}{rgb}{1.000000,0.705882,0.509804}%
\pgfsetstrokecolor{currentstroke}%
\pgfsetstrokeopacity{0.800000}%
\pgfsetdash{}{0pt}%
\pgfpathmoveto{\pgfqpoint{6.117416in}{7.409866in}}%
\pgfpathlineto{\pgfqpoint{6.396515in}{4.497368in}}%
\pgfusepath{stroke}%
\end{pgfscope}%
\begin{pgfscope}%
\pgfpathrectangle{\pgfqpoint{0.570343in}{0.331635in}}{\pgfqpoint{9.300000in}{7.700000in}}%
\pgfusepath{clip}%
\pgfsetrectcap%
\pgfsetroundjoin%
\pgfsetlinewidth{1.505625pt}%
\definecolor{currentstroke}{rgb}{1.000000,0.705882,0.509804}%
\pgfsetstrokecolor{currentstroke}%
\pgfsetstrokeopacity{0.800000}%
\pgfsetdash{}{0pt}%
\pgfpathmoveto{\pgfqpoint{7.933795in}{3.410567in}}%
\pgfpathlineto{\pgfqpoint{6.396515in}{4.497368in}}%
\pgfusepath{stroke}%
\end{pgfscope}%
\begin{pgfscope}%
\pgfpathrectangle{\pgfqpoint{0.570343in}{0.331635in}}{\pgfqpoint{9.300000in}{7.700000in}}%
\pgfusepath{clip}%
\pgfsetrectcap%
\pgfsetroundjoin%
\pgfsetlinewidth{1.505625pt}%
\definecolor{currentstroke}{rgb}{1.000000,0.705882,0.509804}%
\pgfsetstrokecolor{currentstroke}%
\pgfsetstrokeopacity{0.800000}%
\pgfsetdash{}{0pt}%
\pgfpathmoveto{\pgfqpoint{7.377901in}{4.262414in}}%
\pgfpathlineto{\pgfqpoint{6.396515in}{4.497368in}}%
\pgfusepath{stroke}%
\end{pgfscope}%
\begin{pgfscope}%
\pgfpathrectangle{\pgfqpoint{0.570343in}{0.331635in}}{\pgfqpoint{9.300000in}{7.700000in}}%
\pgfusepath{clip}%
\pgfsetrectcap%
\pgfsetroundjoin%
\pgfsetlinewidth{1.505625pt}%
\definecolor{currentstroke}{rgb}{1.000000,0.705882,0.509804}%
\pgfsetstrokecolor{currentstroke}%
\pgfsetstrokeopacity{0.800000}%
\pgfsetdash{}{0pt}%
\pgfpathmoveto{\pgfqpoint{7.284262in}{2.147125in}}%
\pgfpathlineto{\pgfqpoint{6.396515in}{4.497368in}}%
\pgfusepath{stroke}%
\end{pgfscope}%
\begin{pgfscope}%
\pgfpathrectangle{\pgfqpoint{0.570343in}{0.331635in}}{\pgfqpoint{9.300000in}{7.700000in}}%
\pgfusepath{clip}%
\pgfsetrectcap%
\pgfsetroundjoin%
\pgfsetlinewidth{1.505625pt}%
\definecolor{currentstroke}{rgb}{1.000000,0.705882,0.509804}%
\pgfsetstrokecolor{currentstroke}%
\pgfsetstrokeopacity{0.800000}%
\pgfsetdash{}{0pt}%
\pgfpathmoveto{\pgfqpoint{5.804302in}{4.463254in}}%
\pgfpathlineto{\pgfqpoint{6.396515in}{4.497368in}}%
\pgfusepath{stroke}%
\end{pgfscope}%
\begin{pgfscope}%
\pgfpathrectangle{\pgfqpoint{0.570343in}{0.331635in}}{\pgfqpoint{9.300000in}{7.700000in}}%
\pgfusepath{clip}%
\pgfsetrectcap%
\pgfsetroundjoin%
\pgfsetlinewidth{1.505625pt}%
\definecolor{currentstroke}{rgb}{1.000000,0.705882,0.509804}%
\pgfsetstrokecolor{currentstroke}%
\pgfsetstrokeopacity{0.800000}%
\pgfsetdash{}{0pt}%
\pgfpathmoveto{\pgfqpoint{9.282814in}{3.539643in}}%
\pgfpathlineto{\pgfqpoint{6.396515in}{4.497368in}}%
\pgfusepath{stroke}%
\end{pgfscope}%
\begin{pgfscope}%
\pgfpathrectangle{\pgfqpoint{0.570343in}{0.331635in}}{\pgfqpoint{9.300000in}{7.700000in}}%
\pgfusepath{clip}%
\pgfsetrectcap%
\pgfsetroundjoin%
\pgfsetlinewidth{1.505625pt}%
\definecolor{currentstroke}{rgb}{1.000000,0.705882,0.509804}%
\pgfsetstrokecolor{currentstroke}%
\pgfsetstrokeopacity{0.800000}%
\pgfsetdash{}{0pt}%
\pgfpathmoveto{\pgfqpoint{5.814279in}{3.918083in}}%
\pgfpathlineto{\pgfqpoint{6.396515in}{4.497368in}}%
\pgfusepath{stroke}%
\end{pgfscope}%
\begin{pgfscope}%
\pgfpathrectangle{\pgfqpoint{0.570343in}{0.331635in}}{\pgfqpoint{9.300000in}{7.700000in}}%
\pgfusepath{clip}%
\pgfsetrectcap%
\pgfsetroundjoin%
\pgfsetlinewidth{1.505625pt}%
\definecolor{currentstroke}{rgb}{1.000000,0.705882,0.509804}%
\pgfsetstrokecolor{currentstroke}%
\pgfsetstrokeopacity{0.800000}%
\pgfsetdash{}{0pt}%
\pgfpathmoveto{\pgfqpoint{6.347009in}{5.438746in}}%
\pgfpathlineto{\pgfqpoint{6.396515in}{4.497368in}}%
\pgfusepath{stroke}%
\end{pgfscope}%
\begin{pgfscope}%
\pgfpathrectangle{\pgfqpoint{0.570343in}{0.331635in}}{\pgfqpoint{9.300000in}{7.700000in}}%
\pgfusepath{clip}%
\pgfsetrectcap%
\pgfsetroundjoin%
\pgfsetlinewidth{1.505625pt}%
\definecolor{currentstroke}{rgb}{1.000000,0.705882,0.509804}%
\pgfsetstrokecolor{currentstroke}%
\pgfsetstrokeopacity{0.800000}%
\pgfsetdash{}{0pt}%
\pgfpathmoveto{\pgfqpoint{8.386537in}{2.548082in}}%
\pgfpathlineto{\pgfqpoint{6.396515in}{4.497368in}}%
\pgfusepath{stroke}%
\end{pgfscope}%
\begin{pgfscope}%
\pgfpathrectangle{\pgfqpoint{0.570343in}{0.331635in}}{\pgfqpoint{9.300000in}{7.700000in}}%
\pgfusepath{clip}%
\pgfsetrectcap%
\pgfsetroundjoin%
\pgfsetlinewidth{1.505625pt}%
\definecolor{currentstroke}{rgb}{1.000000,0.705882,0.509804}%
\pgfsetstrokecolor{currentstroke}%
\pgfsetstrokeopacity{0.800000}%
\pgfsetdash{}{0pt}%
\pgfpathmoveto{\pgfqpoint{8.388882in}{3.885674in}}%
\pgfpathlineto{\pgfqpoint{6.396515in}{4.497368in}}%
\pgfusepath{stroke}%
\end{pgfscope}%
\begin{pgfscope}%
\pgfpathrectangle{\pgfqpoint{0.570343in}{0.331635in}}{\pgfqpoint{9.300000in}{7.700000in}}%
\pgfusepath{clip}%
\pgfsetrectcap%
\pgfsetroundjoin%
\pgfsetlinewidth{1.505625pt}%
\definecolor{currentstroke}{rgb}{1.000000,0.705882,0.509804}%
\pgfsetstrokecolor{currentstroke}%
\pgfsetstrokeopacity{0.800000}%
\pgfsetdash{}{0pt}%
\pgfpathmoveto{\pgfqpoint{3.136766in}{5.850270in}}%
\pgfpathlineto{\pgfqpoint{6.396515in}{4.497368in}}%
\pgfusepath{stroke}%
\end{pgfscope}%
\begin{pgfscope}%
\pgfpathrectangle{\pgfqpoint{0.570343in}{0.331635in}}{\pgfqpoint{9.300000in}{7.700000in}}%
\pgfusepath{clip}%
\pgfsetrectcap%
\pgfsetroundjoin%
\pgfsetlinewidth{1.505625pt}%
\definecolor{currentstroke}{rgb}{1.000000,0.705882,0.509804}%
\pgfsetstrokecolor{currentstroke}%
\pgfsetstrokeopacity{0.800000}%
\pgfsetdash{}{0pt}%
\pgfpathmoveto{\pgfqpoint{5.512434in}{6.004409in}}%
\pgfpathlineto{\pgfqpoint{6.396515in}{4.497368in}}%
\pgfusepath{stroke}%
\end{pgfscope}%
\begin{pgfscope}%
\pgfpathrectangle{\pgfqpoint{0.570343in}{0.331635in}}{\pgfqpoint{9.300000in}{7.700000in}}%
\pgfusepath{clip}%
\pgfsetrectcap%
\pgfsetroundjoin%
\pgfsetlinewidth{1.505625pt}%
\definecolor{currentstroke}{rgb}{1.000000,0.705882,0.509804}%
\pgfsetstrokecolor{currentstroke}%
\pgfsetstrokeopacity{0.800000}%
\pgfsetdash{}{0pt}%
\pgfpathmoveto{\pgfqpoint{7.456205in}{1.550207in}}%
\pgfpathlineto{\pgfqpoint{6.396515in}{4.497368in}}%
\pgfusepath{stroke}%
\end{pgfscope}%
\begin{pgfscope}%
\pgfpathrectangle{\pgfqpoint{0.570343in}{0.331635in}}{\pgfqpoint{9.300000in}{7.700000in}}%
\pgfusepath{clip}%
\pgfsetrectcap%
\pgfsetroundjoin%
\pgfsetlinewidth{1.505625pt}%
\definecolor{currentstroke}{rgb}{1.000000,0.705882,0.509804}%
\pgfsetstrokecolor{currentstroke}%
\pgfsetstrokeopacity{0.800000}%
\pgfsetdash{}{0pt}%
\pgfpathmoveto{\pgfqpoint{4.644507in}{6.767325in}}%
\pgfpathlineto{\pgfqpoint{6.396515in}{4.497368in}}%
\pgfusepath{stroke}%
\end{pgfscope}%
\begin{pgfscope}%
\pgfpathrectangle{\pgfqpoint{0.570343in}{0.331635in}}{\pgfqpoint{9.300000in}{7.700000in}}%
\pgfusepath{clip}%
\pgfsetrectcap%
\pgfsetroundjoin%
\pgfsetlinewidth{1.505625pt}%
\definecolor{currentstroke}{rgb}{1.000000,0.705882,0.509804}%
\pgfsetstrokecolor{currentstroke}%
\pgfsetstrokeopacity{0.800000}%
\pgfsetdash{}{0pt}%
\pgfpathmoveto{\pgfqpoint{4.552181in}{5.527069in}}%
\pgfpathlineto{\pgfqpoint{6.396515in}{4.497368in}}%
\pgfusepath{stroke}%
\end{pgfscope}%
\begin{pgfscope}%
\pgfpathrectangle{\pgfqpoint{0.570343in}{0.331635in}}{\pgfqpoint{9.300000in}{7.700000in}}%
\pgfusepath{clip}%
\pgfsetrectcap%
\pgfsetroundjoin%
\pgfsetlinewidth{1.505625pt}%
\definecolor{currentstroke}{rgb}{1.000000,0.705882,0.509804}%
\pgfsetstrokecolor{currentstroke}%
\pgfsetstrokeopacity{0.800000}%
\pgfsetdash{}{0pt}%
\pgfpathmoveto{\pgfqpoint{7.082491in}{7.149978in}}%
\pgfpathlineto{\pgfqpoint{6.396515in}{4.497368in}}%
\pgfusepath{stroke}%
\end{pgfscope}%
\begin{pgfscope}%
\pgfpathrectangle{\pgfqpoint{0.570343in}{0.331635in}}{\pgfqpoint{9.300000in}{7.700000in}}%
\pgfusepath{clip}%
\pgfsetrectcap%
\pgfsetroundjoin%
\pgfsetlinewidth{1.505625pt}%
\definecolor{currentstroke}{rgb}{1.000000,0.705882,0.509804}%
\pgfsetstrokecolor{currentstroke}%
\pgfsetstrokeopacity{0.800000}%
\pgfsetdash{}{0pt}%
\pgfpathmoveto{\pgfqpoint{6.077373in}{2.333008in}}%
\pgfpathlineto{\pgfqpoint{6.396515in}{4.497368in}}%
\pgfusepath{stroke}%
\end{pgfscope}%
\begin{pgfscope}%
\pgfpathrectangle{\pgfqpoint{0.570343in}{0.331635in}}{\pgfqpoint{9.300000in}{7.700000in}}%
\pgfusepath{clip}%
\pgfsetrectcap%
\pgfsetroundjoin%
\pgfsetlinewidth{1.505625pt}%
\definecolor{currentstroke}{rgb}{1.000000,0.705882,0.509804}%
\pgfsetstrokecolor{currentstroke}%
\pgfsetstrokeopacity{0.800000}%
\pgfsetdash{}{0pt}%
\pgfpathmoveto{\pgfqpoint{4.868813in}{0.980584in}}%
\pgfpathlineto{\pgfqpoint{6.396515in}{4.497368in}}%
\pgfusepath{stroke}%
\end{pgfscope}%
\begin{pgfscope}%
\pgfpathrectangle{\pgfqpoint{0.570343in}{0.331635in}}{\pgfqpoint{9.300000in}{7.700000in}}%
\pgfusepath{clip}%
\pgfsetrectcap%
\pgfsetroundjoin%
\pgfsetlinewidth{1.505625pt}%
\definecolor{currentstroke}{rgb}{1.000000,0.705882,0.509804}%
\pgfsetstrokecolor{currentstroke}%
\pgfsetstrokeopacity{0.800000}%
\pgfsetdash{}{0pt}%
\pgfpathmoveto{\pgfqpoint{4.793914in}{4.789225in}}%
\pgfpathlineto{\pgfqpoint{6.396515in}{4.497368in}}%
\pgfusepath{stroke}%
\end{pgfscope}%
\begin{pgfscope}%
\pgfpathrectangle{\pgfqpoint{0.570343in}{0.331635in}}{\pgfqpoint{9.300000in}{7.700000in}}%
\pgfusepath{clip}%
\pgfsetrectcap%
\pgfsetroundjoin%
\pgfsetlinewidth{1.505625pt}%
\definecolor{currentstroke}{rgb}{1.000000,0.705882,0.509804}%
\pgfsetstrokecolor{currentstroke}%
\pgfsetstrokeopacity{0.800000}%
\pgfsetdash{}{0pt}%
\pgfpathmoveto{\pgfqpoint{6.545799in}{6.396589in}}%
\pgfpathlineto{\pgfqpoint{6.396515in}{4.497368in}}%
\pgfusepath{stroke}%
\end{pgfscope}%
\begin{pgfscope}%
\pgfpathrectangle{\pgfqpoint{0.570343in}{0.331635in}}{\pgfqpoint{9.300000in}{7.700000in}}%
\pgfusepath{clip}%
\pgfsetrectcap%
\pgfsetroundjoin%
\pgfsetlinewidth{1.505625pt}%
\definecolor{currentstroke}{rgb}{1.000000,0.705882,0.509804}%
\pgfsetstrokecolor{currentstroke}%
\pgfsetstrokeopacity{0.800000}%
\pgfsetdash{}{0pt}%
\pgfpathmoveto{\pgfqpoint{3.662589in}{2.831937in}}%
\pgfpathlineto{\pgfqpoint{6.396515in}{4.497368in}}%
\pgfusepath{stroke}%
\end{pgfscope}%
\begin{pgfscope}%
\pgfpathrectangle{\pgfqpoint{0.570343in}{0.331635in}}{\pgfqpoint{9.300000in}{7.700000in}}%
\pgfusepath{clip}%
\pgfsetrectcap%
\pgfsetroundjoin%
\pgfsetlinewidth{1.505625pt}%
\definecolor{currentstroke}{rgb}{1.000000,0.705882,0.509804}%
\pgfsetstrokecolor{currentstroke}%
\pgfsetstrokeopacity{0.800000}%
\pgfsetdash{}{0pt}%
\pgfpathmoveto{\pgfqpoint{8.266806in}{7.441823in}}%
\pgfpathlineto{\pgfqpoint{6.396515in}{4.497368in}}%
\pgfusepath{stroke}%
\end{pgfscope}%
\begin{pgfscope}%
\pgfpathrectangle{\pgfqpoint{0.570343in}{0.331635in}}{\pgfqpoint{9.300000in}{7.700000in}}%
\pgfusepath{clip}%
\pgfsetrectcap%
\pgfsetroundjoin%
\pgfsetlinewidth{1.505625pt}%
\definecolor{currentstroke}{rgb}{1.000000,0.705882,0.509804}%
\pgfsetstrokecolor{currentstroke}%
\pgfsetstrokeopacity{0.800000}%
\pgfsetdash{}{0pt}%
\pgfpathmoveto{\pgfqpoint{1.678869in}{5.981260in}}%
\pgfpathlineto{\pgfqpoint{6.396515in}{4.497368in}}%
\pgfusepath{stroke}%
\end{pgfscope}%
\begin{pgfscope}%
\pgfpathrectangle{\pgfqpoint{0.570343in}{0.331635in}}{\pgfqpoint{9.300000in}{7.700000in}}%
\pgfusepath{clip}%
\pgfsetrectcap%
\pgfsetroundjoin%
\pgfsetlinewidth{1.505625pt}%
\definecolor{currentstroke}{rgb}{1.000000,0.705882,0.509804}%
\pgfsetstrokecolor{currentstroke}%
\pgfsetstrokeopacity{0.800000}%
\pgfsetdash{}{0pt}%
\pgfpathmoveto{\pgfqpoint{8.307790in}{4.579090in}}%
\pgfpathlineto{\pgfqpoint{6.396515in}{4.497368in}}%
\pgfusepath{stroke}%
\end{pgfscope}%
\begin{pgfscope}%
\pgfpathrectangle{\pgfqpoint{0.570343in}{0.331635in}}{\pgfqpoint{9.300000in}{7.700000in}}%
\pgfusepath{clip}%
\pgfsetrectcap%
\pgfsetroundjoin%
\pgfsetlinewidth{1.505625pt}%
\definecolor{currentstroke}{rgb}{1.000000,0.705882,0.509804}%
\pgfsetstrokecolor{currentstroke}%
\pgfsetstrokeopacity{0.800000}%
\pgfsetdash{}{0pt}%
\pgfpathmoveto{\pgfqpoint{7.187407in}{5.824946in}}%
\pgfpathlineto{\pgfqpoint{6.396515in}{4.497368in}}%
\pgfusepath{stroke}%
\end{pgfscope}%
\begin{pgfscope}%
\pgfpathrectangle{\pgfqpoint{0.570343in}{0.331635in}}{\pgfqpoint{9.300000in}{7.700000in}}%
\pgfusepath{clip}%
\pgfsetrectcap%
\pgfsetroundjoin%
\pgfsetlinewidth{1.505625pt}%
\definecolor{currentstroke}{rgb}{1.000000,0.705882,0.509804}%
\pgfsetstrokecolor{currentstroke}%
\pgfsetstrokeopacity{0.800000}%
\pgfsetdash{}{0pt}%
\pgfpathmoveto{\pgfqpoint{6.326121in}{1.402458in}}%
\pgfpathlineto{\pgfqpoint{6.396515in}{4.497368in}}%
\pgfusepath{stroke}%
\end{pgfscope}%
\begin{pgfscope}%
\pgfpathrectangle{\pgfqpoint{0.570343in}{0.331635in}}{\pgfqpoint{9.300000in}{7.700000in}}%
\pgfusepath{clip}%
\pgfsetrectcap%
\pgfsetroundjoin%
\pgfsetlinewidth{1.505625pt}%
\definecolor{currentstroke}{rgb}{1.000000,0.705882,0.509804}%
\pgfsetstrokecolor{currentstroke}%
\pgfsetstrokeopacity{0.800000}%
\pgfsetdash{}{0pt}%
\pgfpathmoveto{\pgfqpoint{9.447616in}{4.657092in}}%
\pgfpathlineto{\pgfqpoint{6.396515in}{4.497368in}}%
\pgfusepath{stroke}%
\end{pgfscope}%
\begin{pgfscope}%
\pgfsetrectcap%
\pgfsetmiterjoin%
\pgfsetlinewidth{0.803000pt}%
\definecolor{currentstroke}{rgb}{0.000000,0.000000,0.000000}%
\pgfsetstrokecolor{currentstroke}%
\pgfsetdash{}{0pt}%
\pgfpathmoveto{\pgfqpoint{0.570343in}{0.331635in}}%
\pgfpathlineto{\pgfqpoint{0.570343in}{8.031635in}}%
\pgfusepath{stroke}%
\end{pgfscope}%
\begin{pgfscope}%
\pgfsetrectcap%
\pgfsetmiterjoin%
\pgfsetlinewidth{0.803000pt}%
\definecolor{currentstroke}{rgb}{0.000000,0.000000,0.000000}%
\pgfsetstrokecolor{currentstroke}%
\pgfsetdash{}{0pt}%
\pgfpathmoveto{\pgfqpoint{9.870343in}{0.331635in}}%
\pgfpathlineto{\pgfqpoint{9.870343in}{8.031635in}}%
\pgfusepath{stroke}%
\end{pgfscope}%
\begin{pgfscope}%
\pgfsetrectcap%
\pgfsetmiterjoin%
\pgfsetlinewidth{0.803000pt}%
\definecolor{currentstroke}{rgb}{0.000000,0.000000,0.000000}%
\pgfsetstrokecolor{currentstroke}%
\pgfsetdash{}{0pt}%
\pgfpathmoveto{\pgfqpoint{0.570343in}{0.331635in}}%
\pgfpathlineto{\pgfqpoint{9.870343in}{0.331635in}}%
\pgfusepath{stroke}%
\end{pgfscope}%
\begin{pgfscope}%
\pgfsetrectcap%
\pgfsetmiterjoin%
\pgfsetlinewidth{0.803000pt}%
\definecolor{currentstroke}{rgb}{0.000000,0.000000,0.000000}%
\pgfsetstrokecolor{currentstroke}%
\pgfsetdash{}{0pt}%
\pgfpathmoveto{\pgfqpoint{0.570343in}{8.031635in}}%
\pgfpathlineto{\pgfqpoint{9.870343in}{8.031635in}}%
\pgfusepath{stroke}%
\end{pgfscope}%
\begin{pgfscope}%
\definecolor{textcolor}{rgb}{0.000000,0.000000,0.000000}%
\pgfsetstrokecolor{textcolor}%
\pgfsetfillcolor{textcolor}%
\pgftext[x=5.220343in,y=8.114968in,,base]{\color{textcolor}\sffamily\fontsize{12.000000}{14.400000}\selectfont Photo-Realistic Images}%
\end{pgfscope}%
\begin{pgfscope}%
\pgfsetbuttcap%
\pgfsetmiterjoin%
\definecolor{currentfill}{rgb}{1.000000,1.000000,1.000000}%
\pgfsetfillcolor{currentfill}%
\pgfsetfillopacity{0.800000}%
\pgfsetlinewidth{1.003750pt}%
\definecolor{currentstroke}{rgb}{0.800000,0.800000,0.800000}%
\pgfsetstrokecolor{currentstroke}%
\pgfsetstrokeopacity{0.800000}%
\pgfsetdash{}{0pt}%
\pgfpathmoveto{\pgfqpoint{9.967566in}{3.956944in}}%
\pgfpathlineto{\pgfqpoint{10.908633in}{3.956944in}}%
\pgfpathquadraticcurveto{\pgfqpoint{10.936411in}{3.956944in}}{\pgfqpoint{10.936411in}{3.984722in}}%
\pgfpathlineto{\pgfqpoint{10.936411in}{4.378548in}}%
\pgfpathquadraticcurveto{\pgfqpoint{10.936411in}{4.406326in}}{\pgfqpoint{10.908633in}{4.406326in}}%
\pgfpathlineto{\pgfqpoint{9.967566in}{4.406326in}}%
\pgfpathquadraticcurveto{\pgfqpoint{9.939788in}{4.406326in}}{\pgfqpoint{9.939788in}{4.378548in}}%
\pgfpathlineto{\pgfqpoint{9.939788in}{3.984722in}}%
\pgfpathquadraticcurveto{\pgfqpoint{9.939788in}{3.956944in}}{\pgfqpoint{9.967566in}{3.956944in}}%
\pgfpathclose%
\pgfusepath{stroke,fill}%
\end{pgfscope}%
\begin{pgfscope}%
\pgfsetbuttcap%
\pgfsetroundjoin%
\definecolor{currentfill}{rgb}{0.631373,0.788235,0.956863}%
\pgfsetfillcolor{currentfill}%
\pgfsetlinewidth{1.003750pt}%
\definecolor{currentstroke}{rgb}{0.631373,0.788235,0.956863}%
\pgfsetstrokecolor{currentstroke}%
\pgfsetdash{}{0pt}%
\pgfsys@defobject{currentmarker}{\pgfqpoint{-0.041667in}{-0.041667in}}{\pgfqpoint{0.041667in}{0.041667in}}{%
\pgfpathmoveto{\pgfqpoint{0.000000in}{-0.041667in}}%
\pgfpathcurveto{\pgfqpoint{0.011050in}{-0.041667in}}{\pgfqpoint{0.021649in}{-0.037276in}}{\pgfqpoint{0.029463in}{-0.029463in}}%
\pgfpathcurveto{\pgfqpoint{0.037276in}{-0.021649in}}{\pgfqpoint{0.041667in}{-0.011050in}}{\pgfqpoint{0.041667in}{0.000000in}}%
\pgfpathcurveto{\pgfqpoint{0.041667in}{0.011050in}}{\pgfqpoint{0.037276in}{0.021649in}}{\pgfqpoint{0.029463in}{0.029463in}}%
\pgfpathcurveto{\pgfqpoint{0.021649in}{0.037276in}}{\pgfqpoint{0.011050in}{0.041667in}}{\pgfqpoint{0.000000in}{0.041667in}}%
\pgfpathcurveto{\pgfqpoint{-0.011050in}{0.041667in}}{\pgfqpoint{-0.021649in}{0.037276in}}{\pgfqpoint{-0.029463in}{0.029463in}}%
\pgfpathcurveto{\pgfqpoint{-0.037276in}{0.021649in}}{\pgfqpoint{-0.041667in}{0.011050in}}{\pgfqpoint{-0.041667in}{0.000000in}}%
\pgfpathcurveto{\pgfqpoint{-0.041667in}{-0.011050in}}{\pgfqpoint{-0.037276in}{-0.021649in}}{\pgfqpoint{-0.029463in}{-0.029463in}}%
\pgfpathcurveto{\pgfqpoint{-0.021649in}{-0.037276in}}{\pgfqpoint{-0.011050in}{-0.041667in}}{\pgfqpoint{0.000000in}{-0.041667in}}%
\pgfpathclose%
\pgfusepath{stroke,fill}%
}%
\begin{pgfscope}%
\pgfsys@transformshift{10.134232in}{4.281705in}%
\pgfsys@useobject{currentmarker}{}%
\end{pgfscope}%
\end{pgfscope}%
\begin{pgfscope}%
\definecolor{textcolor}{rgb}{0.000000,0.000000,0.000000}%
\pgfsetstrokecolor{textcolor}%
\pgfsetfillcolor{textcolor}%
\pgftext[x=10.384232in,y=4.245247in,left,base]{\color{textcolor}\sffamily\fontsize{10.000000}{12.000000}\selectfont ai2thor}%
\end{pgfscope}%
\begin{pgfscope}%
\pgfsetbuttcap%
\pgfsetroundjoin%
\definecolor{currentfill}{rgb}{1.000000,0.705882,0.509804}%
\pgfsetfillcolor{currentfill}%
\pgfsetlinewidth{1.003750pt}%
\definecolor{currentstroke}{rgb}{1.000000,0.705882,0.509804}%
\pgfsetstrokecolor{currentstroke}%
\pgfsetdash{}{0pt}%
\pgfsys@defobject{currentmarker}{\pgfqpoint{-0.041667in}{-0.041667in}}{\pgfqpoint{0.041667in}{0.041667in}}{%
\pgfpathmoveto{\pgfqpoint{0.000000in}{-0.041667in}}%
\pgfpathcurveto{\pgfqpoint{0.011050in}{-0.041667in}}{\pgfqpoint{0.021649in}{-0.037276in}}{\pgfqpoint{0.029463in}{-0.029463in}}%
\pgfpathcurveto{\pgfqpoint{0.037276in}{-0.021649in}}{\pgfqpoint{0.041667in}{-0.011050in}}{\pgfqpoint{0.041667in}{0.000000in}}%
\pgfpathcurveto{\pgfqpoint{0.041667in}{0.011050in}}{\pgfqpoint{0.037276in}{0.021649in}}{\pgfqpoint{0.029463in}{0.029463in}}%
\pgfpathcurveto{\pgfqpoint{0.021649in}{0.037276in}}{\pgfqpoint{0.011050in}{0.041667in}}{\pgfqpoint{0.000000in}{0.041667in}}%
\pgfpathcurveto{\pgfqpoint{-0.011050in}{0.041667in}}{\pgfqpoint{-0.021649in}{0.037276in}}{\pgfqpoint{-0.029463in}{0.029463in}}%
\pgfpathcurveto{\pgfqpoint{-0.037276in}{0.021649in}}{\pgfqpoint{-0.041667in}{0.011050in}}{\pgfqpoint{-0.041667in}{0.000000in}}%
\pgfpathcurveto{\pgfqpoint{-0.041667in}{-0.011050in}}{\pgfqpoint{-0.037276in}{-0.021649in}}{\pgfqpoint{-0.029463in}{-0.029463in}}%
\pgfpathcurveto{\pgfqpoint{-0.021649in}{-0.037276in}}{\pgfqpoint{-0.011050in}{-0.041667in}}{\pgfqpoint{0.000000in}{-0.041667in}}%
\pgfpathclose%
\pgfusepath{stroke,fill}%
}%
\begin{pgfscope}%
\pgfsys@transformshift{10.134232in}{4.077848in}%
\pgfsys@useobject{currentmarker}{}%
\end{pgfscope}%
\end{pgfscope}%
\begin{pgfscope}%
\definecolor{textcolor}{rgb}{0.000000,0.000000,0.000000}%
\pgfsetstrokecolor{textcolor}%
\pgfsetfillcolor{textcolor}%
\pgftext[x=10.384232in,y=4.041390in,left,base]{\color{textcolor}\sffamily\fontsize{10.000000}{12.000000}\selectfont pix3d}%
\end{pgfscope}%
\end{pgfpicture}%
\makeatother%
\endgroup%
}\\
    \resizebox{0.49\linewidth}{5cm}{%% Creator: Matplotlib, PGF backend
%%
%% To include the figure in your LaTeX document, write
%%   \input{<filename>.pgf}
%%
%% Make sure the required packages are loaded in your preamble
%%   \usepackage{pgf}
%%
%% Figures using additional raster images can only be included by \input if
%% they are in the same directory as the main LaTeX file. For loading figures
%% from other directories you can use the `import` package
%%   \usepackage{import}
%%
%% and then include the figures with
%%   \import{<path to file>}{<filename>.pgf}
%%
%% Matplotlib used the following preamble
%%   \usepackage{fontspec}
%%   \setmainfont{DejaVuSerif.ttf}[Path=\detokenize{/Users/apple/opt/anaconda3/envs/kaolin/lib/python3.7/site-packages/matplotlib/mpl-data/fonts/ttf/}]
%%   \setsansfont{DejaVuSans.ttf}[Path=\detokenize{/Users/apple/opt/anaconda3/envs/kaolin/lib/python3.7/site-packages/matplotlib/mpl-data/fonts/ttf/}]
%%   \setmonofont{DejaVuSansMono.ttf}[Path=\detokenize{/Users/apple/opt/anaconda3/envs/kaolin/lib/python3.7/site-packages/matplotlib/mpl-data/fonts/ttf/}]
%%
\begingroup%
\makeatletter%
\begin{pgfpicture}%
\pgfpathrectangle{\pgfpointorigin}{\pgfqpoint{11.046719in}{8.341596in}}%
\pgfusepath{use as bounding box, clip}%
\begin{pgfscope}%
\pgfsetbuttcap%
\pgfsetmiterjoin%
\definecolor{currentfill}{rgb}{1.000000,1.000000,1.000000}%
\pgfsetfillcolor{currentfill}%
\pgfsetlinewidth{0.000000pt}%
\definecolor{currentstroke}{rgb}{1.000000,1.000000,1.000000}%
\pgfsetstrokecolor{currentstroke}%
\pgfsetdash{}{0pt}%
\pgfpathmoveto{\pgfqpoint{0.000000in}{0.000000in}}%
\pgfpathlineto{\pgfqpoint{11.046719in}{0.000000in}}%
\pgfpathlineto{\pgfqpoint{11.046719in}{8.341596in}}%
\pgfpathlineto{\pgfqpoint{0.000000in}{8.341596in}}%
\pgfpathclose%
\pgfusepath{fill}%
\end{pgfscope}%
\begin{pgfscope}%
\pgfsetbuttcap%
\pgfsetmiterjoin%
\definecolor{currentfill}{rgb}{1.000000,1.000000,1.000000}%
\pgfsetfillcolor{currentfill}%
\pgfsetlinewidth{0.000000pt}%
\definecolor{currentstroke}{rgb}{0.000000,0.000000,0.000000}%
\pgfsetstrokecolor{currentstroke}%
\pgfsetstrokeopacity{0.000000}%
\pgfsetdash{}{0pt}%
\pgfpathmoveto{\pgfqpoint{0.570343in}{0.331635in}}%
\pgfpathlineto{\pgfqpoint{9.870343in}{0.331635in}}%
\pgfpathlineto{\pgfqpoint{9.870343in}{8.031635in}}%
\pgfpathlineto{\pgfqpoint{0.570343in}{8.031635in}}%
\pgfpathclose%
\pgfusepath{fill}%
\end{pgfscope}%
\begin{pgfscope}%
\pgfpathrectangle{\pgfqpoint{0.570343in}{0.331635in}}{\pgfqpoint{9.300000in}{7.700000in}}%
\pgfusepath{clip}%
\pgfsetbuttcap%
\pgfsetroundjoin%
\definecolor{currentfill}{rgb}{0.631373,0.788235,0.956863}%
\pgfsetfillcolor{currentfill}%
\pgfsetlinewidth{0.481800pt}%
\definecolor{currentstroke}{rgb}{1.000000,1.000000,1.000000}%
\pgfsetstrokecolor{currentstroke}%
\pgfsetdash{}{0pt}%
\pgfpathmoveto{\pgfqpoint{7.239021in}{5.072456in}}%
\pgfpathcurveto{\pgfqpoint{7.250071in}{5.072456in}}{\pgfqpoint{7.260670in}{5.076846in}}{\pgfqpoint{7.268483in}{5.084660in}}%
\pgfpathcurveto{\pgfqpoint{7.276297in}{5.092474in}}{\pgfqpoint{7.280687in}{5.103073in}}{\pgfqpoint{7.280687in}{5.114123in}}%
\pgfpathcurveto{\pgfqpoint{7.280687in}{5.125173in}}{\pgfqpoint{7.276297in}{5.135772in}}{\pgfqpoint{7.268483in}{5.143586in}}%
\pgfpathcurveto{\pgfqpoint{7.260670in}{5.151399in}}{\pgfqpoint{7.250071in}{5.155789in}}{\pgfqpoint{7.239021in}{5.155789in}}%
\pgfpathcurveto{\pgfqpoint{7.227970in}{5.155789in}}{\pgfqpoint{7.217371in}{5.151399in}}{\pgfqpoint{7.209558in}{5.143586in}}%
\pgfpathcurveto{\pgfqpoint{7.201744in}{5.135772in}}{\pgfqpoint{7.197354in}{5.125173in}}{\pgfqpoint{7.197354in}{5.114123in}}%
\pgfpathcurveto{\pgfqpoint{7.197354in}{5.103073in}}{\pgfqpoint{7.201744in}{5.092474in}}{\pgfqpoint{7.209558in}{5.084660in}}%
\pgfpathcurveto{\pgfqpoint{7.217371in}{5.076846in}}{\pgfqpoint{7.227970in}{5.072456in}}{\pgfqpoint{7.239021in}{5.072456in}}%
\pgfpathclose%
\pgfusepath{stroke,fill}%
\end{pgfscope}%
\begin{pgfscope}%
\pgfpathrectangle{\pgfqpoint{0.570343in}{0.331635in}}{\pgfqpoint{9.300000in}{7.700000in}}%
\pgfusepath{clip}%
\pgfsetbuttcap%
\pgfsetroundjoin%
\definecolor{currentfill}{rgb}{0.631373,0.788235,0.956863}%
\pgfsetfillcolor{currentfill}%
\pgfsetlinewidth{0.481800pt}%
\definecolor{currentstroke}{rgb}{1.000000,1.000000,1.000000}%
\pgfsetstrokecolor{currentstroke}%
\pgfsetdash{}{0pt}%
\pgfpathmoveto{\pgfqpoint{6.513392in}{3.968650in}}%
\pgfpathcurveto{\pgfqpoint{6.524442in}{3.968650in}}{\pgfqpoint{6.535041in}{3.973041in}}{\pgfqpoint{6.542855in}{3.980854in}}%
\pgfpathcurveto{\pgfqpoint{6.550668in}{3.988668in}}{\pgfqpoint{6.555059in}{3.999267in}}{\pgfqpoint{6.555059in}{4.010317in}}%
\pgfpathcurveto{\pgfqpoint{6.555059in}{4.021367in}}{\pgfqpoint{6.550668in}{4.031966in}}{\pgfqpoint{6.542855in}{4.039780in}}%
\pgfpathcurveto{\pgfqpoint{6.535041in}{4.047593in}}{\pgfqpoint{6.524442in}{4.051984in}}{\pgfqpoint{6.513392in}{4.051984in}}%
\pgfpathcurveto{\pgfqpoint{6.502342in}{4.051984in}}{\pgfqpoint{6.491743in}{4.047593in}}{\pgfqpoint{6.483929in}{4.039780in}}%
\pgfpathcurveto{\pgfqpoint{6.476116in}{4.031966in}}{\pgfqpoint{6.471725in}{4.021367in}}{\pgfqpoint{6.471725in}{4.010317in}}%
\pgfpathcurveto{\pgfqpoint{6.471725in}{3.999267in}}{\pgfqpoint{6.476116in}{3.988668in}}{\pgfqpoint{6.483929in}{3.980854in}}%
\pgfpathcurveto{\pgfqpoint{6.491743in}{3.973041in}}{\pgfqpoint{6.502342in}{3.968650in}}{\pgfqpoint{6.513392in}{3.968650in}}%
\pgfpathclose%
\pgfusepath{stroke,fill}%
\end{pgfscope}%
\begin{pgfscope}%
\pgfpathrectangle{\pgfqpoint{0.570343in}{0.331635in}}{\pgfqpoint{9.300000in}{7.700000in}}%
\pgfusepath{clip}%
\pgfsetbuttcap%
\pgfsetroundjoin%
\definecolor{currentfill}{rgb}{0.631373,0.788235,0.956863}%
\pgfsetfillcolor{currentfill}%
\pgfsetlinewidth{0.481800pt}%
\definecolor{currentstroke}{rgb}{1.000000,1.000000,1.000000}%
\pgfsetstrokecolor{currentstroke}%
\pgfsetdash{}{0pt}%
\pgfpathmoveto{\pgfqpoint{2.367651in}{5.480048in}}%
\pgfpathcurveto{\pgfqpoint{2.378701in}{5.480048in}}{\pgfqpoint{2.389300in}{5.484438in}}{\pgfqpoint{2.397114in}{5.492252in}}%
\pgfpathcurveto{\pgfqpoint{2.404928in}{5.500066in}}{\pgfqpoint{2.409318in}{5.510665in}}{\pgfqpoint{2.409318in}{5.521715in}}%
\pgfpathcurveto{\pgfqpoint{2.409318in}{5.532765in}}{\pgfqpoint{2.404928in}{5.543364in}}{\pgfqpoint{2.397114in}{5.551177in}}%
\pgfpathcurveto{\pgfqpoint{2.389300in}{5.558991in}}{\pgfqpoint{2.378701in}{5.563381in}}{\pgfqpoint{2.367651in}{5.563381in}}%
\pgfpathcurveto{\pgfqpoint{2.356601in}{5.563381in}}{\pgfqpoint{2.346002in}{5.558991in}}{\pgfqpoint{2.338188in}{5.551177in}}%
\pgfpathcurveto{\pgfqpoint{2.330375in}{5.543364in}}{\pgfqpoint{2.325985in}{5.532765in}}{\pgfqpoint{2.325985in}{5.521715in}}%
\pgfpathcurveto{\pgfqpoint{2.325985in}{5.510665in}}{\pgfqpoint{2.330375in}{5.500066in}}{\pgfqpoint{2.338188in}{5.492252in}}%
\pgfpathcurveto{\pgfqpoint{2.346002in}{5.484438in}}{\pgfqpoint{2.356601in}{5.480048in}}{\pgfqpoint{2.367651in}{5.480048in}}%
\pgfpathclose%
\pgfusepath{stroke,fill}%
\end{pgfscope}%
\begin{pgfscope}%
\pgfpathrectangle{\pgfqpoint{0.570343in}{0.331635in}}{\pgfqpoint{9.300000in}{7.700000in}}%
\pgfusepath{clip}%
\pgfsetbuttcap%
\pgfsetroundjoin%
\definecolor{currentfill}{rgb}{0.631373,0.788235,0.956863}%
\pgfsetfillcolor{currentfill}%
\pgfsetlinewidth{0.481800pt}%
\definecolor{currentstroke}{rgb}{1.000000,1.000000,1.000000}%
\pgfsetstrokecolor{currentstroke}%
\pgfsetdash{}{0pt}%
\pgfpathmoveto{\pgfqpoint{6.809391in}{4.507992in}}%
\pgfpathcurveto{\pgfqpoint{6.820441in}{4.507992in}}{\pgfqpoint{6.831040in}{4.512382in}}{\pgfqpoint{6.838854in}{4.520196in}}%
\pgfpathcurveto{\pgfqpoint{6.846668in}{4.528009in}}{\pgfqpoint{6.851058in}{4.538608in}}{\pgfqpoint{6.851058in}{4.549658in}}%
\pgfpathcurveto{\pgfqpoint{6.851058in}{4.560708in}}{\pgfqpoint{6.846668in}{4.571307in}}{\pgfqpoint{6.838854in}{4.579121in}}%
\pgfpathcurveto{\pgfqpoint{6.831040in}{4.586935in}}{\pgfqpoint{6.820441in}{4.591325in}}{\pgfqpoint{6.809391in}{4.591325in}}%
\pgfpathcurveto{\pgfqpoint{6.798341in}{4.591325in}}{\pgfqpoint{6.787742in}{4.586935in}}{\pgfqpoint{6.779929in}{4.579121in}}%
\pgfpathcurveto{\pgfqpoint{6.772115in}{4.571307in}}{\pgfqpoint{6.767725in}{4.560708in}}{\pgfqpoint{6.767725in}{4.549658in}}%
\pgfpathcurveto{\pgfqpoint{6.767725in}{4.538608in}}{\pgfqpoint{6.772115in}{4.528009in}}{\pgfqpoint{6.779929in}{4.520196in}}%
\pgfpathcurveto{\pgfqpoint{6.787742in}{4.512382in}}{\pgfqpoint{6.798341in}{4.507992in}}{\pgfqpoint{6.809391in}{4.507992in}}%
\pgfpathclose%
\pgfusepath{stroke,fill}%
\end{pgfscope}%
\begin{pgfscope}%
\pgfpathrectangle{\pgfqpoint{0.570343in}{0.331635in}}{\pgfqpoint{9.300000in}{7.700000in}}%
\pgfusepath{clip}%
\pgfsetbuttcap%
\pgfsetroundjoin%
\definecolor{currentfill}{rgb}{0.631373,0.788235,0.956863}%
\pgfsetfillcolor{currentfill}%
\pgfsetlinewidth{0.481800pt}%
\definecolor{currentstroke}{rgb}{1.000000,1.000000,1.000000}%
\pgfsetstrokecolor{currentstroke}%
\pgfsetdash{}{0pt}%
\pgfpathmoveto{\pgfqpoint{4.564595in}{4.371940in}}%
\pgfpathcurveto{\pgfqpoint{4.575645in}{4.371940in}}{\pgfqpoint{4.586244in}{4.376330in}}{\pgfqpoint{4.594058in}{4.384144in}}%
\pgfpathcurveto{\pgfqpoint{4.601872in}{4.391957in}}{\pgfqpoint{4.606262in}{4.402556in}}{\pgfqpoint{4.606262in}{4.413606in}}%
\pgfpathcurveto{\pgfqpoint{4.606262in}{4.424657in}}{\pgfqpoint{4.601872in}{4.435256in}}{\pgfqpoint{4.594058in}{4.443069in}}%
\pgfpathcurveto{\pgfqpoint{4.586244in}{4.450883in}}{\pgfqpoint{4.575645in}{4.455273in}}{\pgfqpoint{4.564595in}{4.455273in}}%
\pgfpathcurveto{\pgfqpoint{4.553545in}{4.455273in}}{\pgfqpoint{4.542946in}{4.450883in}}{\pgfqpoint{4.535132in}{4.443069in}}%
\pgfpathcurveto{\pgfqpoint{4.527319in}{4.435256in}}{\pgfqpoint{4.522928in}{4.424657in}}{\pgfqpoint{4.522928in}{4.413606in}}%
\pgfpathcurveto{\pgfqpoint{4.522928in}{4.402556in}}{\pgfqpoint{4.527319in}{4.391957in}}{\pgfqpoint{4.535132in}{4.384144in}}%
\pgfpathcurveto{\pgfqpoint{4.542946in}{4.376330in}}{\pgfqpoint{4.553545in}{4.371940in}}{\pgfqpoint{4.564595in}{4.371940in}}%
\pgfpathclose%
\pgfusepath{stroke,fill}%
\end{pgfscope}%
\begin{pgfscope}%
\pgfpathrectangle{\pgfqpoint{0.570343in}{0.331635in}}{\pgfqpoint{9.300000in}{7.700000in}}%
\pgfusepath{clip}%
\pgfsetbuttcap%
\pgfsetroundjoin%
\definecolor{currentfill}{rgb}{0.631373,0.788235,0.956863}%
\pgfsetfillcolor{currentfill}%
\pgfsetlinewidth{0.481800pt}%
\definecolor{currentstroke}{rgb}{1.000000,1.000000,1.000000}%
\pgfsetstrokecolor{currentstroke}%
\pgfsetdash{}{0pt}%
\pgfpathmoveto{\pgfqpoint{5.991395in}{5.235590in}}%
\pgfpathcurveto{\pgfqpoint{6.002446in}{5.235590in}}{\pgfqpoint{6.013045in}{5.239980in}}{\pgfqpoint{6.020858in}{5.247794in}}%
\pgfpathcurveto{\pgfqpoint{6.028672in}{5.255608in}}{\pgfqpoint{6.033062in}{5.266207in}}{\pgfqpoint{6.033062in}{5.277257in}}%
\pgfpathcurveto{\pgfqpoint{6.033062in}{5.288307in}}{\pgfqpoint{6.028672in}{5.298906in}}{\pgfqpoint{6.020858in}{5.306720in}}%
\pgfpathcurveto{\pgfqpoint{6.013045in}{5.314533in}}{\pgfqpoint{6.002446in}{5.318924in}}{\pgfqpoint{5.991395in}{5.318924in}}%
\pgfpathcurveto{\pgfqpoint{5.980345in}{5.318924in}}{\pgfqpoint{5.969746in}{5.314533in}}{\pgfqpoint{5.961933in}{5.306720in}}%
\pgfpathcurveto{\pgfqpoint{5.954119in}{5.298906in}}{\pgfqpoint{5.949729in}{5.288307in}}{\pgfqpoint{5.949729in}{5.277257in}}%
\pgfpathcurveto{\pgfqpoint{5.949729in}{5.266207in}}{\pgfqpoint{5.954119in}{5.255608in}}{\pgfqpoint{5.961933in}{5.247794in}}%
\pgfpathcurveto{\pgfqpoint{5.969746in}{5.239980in}}{\pgfqpoint{5.980345in}{5.235590in}}{\pgfqpoint{5.991395in}{5.235590in}}%
\pgfpathclose%
\pgfusepath{stroke,fill}%
\end{pgfscope}%
\begin{pgfscope}%
\pgfpathrectangle{\pgfqpoint{0.570343in}{0.331635in}}{\pgfqpoint{9.300000in}{7.700000in}}%
\pgfusepath{clip}%
\pgfsetbuttcap%
\pgfsetroundjoin%
\definecolor{currentfill}{rgb}{0.631373,0.788235,0.956863}%
\pgfsetfillcolor{currentfill}%
\pgfsetlinewidth{0.481800pt}%
\definecolor{currentstroke}{rgb}{1.000000,1.000000,1.000000}%
\pgfsetstrokecolor{currentstroke}%
\pgfsetdash{}{0pt}%
\pgfpathmoveto{\pgfqpoint{3.383575in}{7.102117in}}%
\pgfpathcurveto{\pgfqpoint{3.394625in}{7.102117in}}{\pgfqpoint{3.405224in}{7.106507in}}{\pgfqpoint{3.413038in}{7.114321in}}%
\pgfpathcurveto{\pgfqpoint{3.420851in}{7.122134in}}{\pgfqpoint{3.425241in}{7.132733in}}{\pgfqpoint{3.425241in}{7.143783in}}%
\pgfpathcurveto{\pgfqpoint{3.425241in}{7.154834in}}{\pgfqpoint{3.420851in}{7.165433in}}{\pgfqpoint{3.413038in}{7.173246in}}%
\pgfpathcurveto{\pgfqpoint{3.405224in}{7.181060in}}{\pgfqpoint{3.394625in}{7.185450in}}{\pgfqpoint{3.383575in}{7.185450in}}%
\pgfpathcurveto{\pgfqpoint{3.372525in}{7.185450in}}{\pgfqpoint{3.361926in}{7.181060in}}{\pgfqpoint{3.354112in}{7.173246in}}%
\pgfpathcurveto{\pgfqpoint{3.346298in}{7.165433in}}{\pgfqpoint{3.341908in}{7.154834in}}{\pgfqpoint{3.341908in}{7.143783in}}%
\pgfpathcurveto{\pgfqpoint{3.341908in}{7.132733in}}{\pgfqpoint{3.346298in}{7.122134in}}{\pgfqpoint{3.354112in}{7.114321in}}%
\pgfpathcurveto{\pgfqpoint{3.361926in}{7.106507in}}{\pgfqpoint{3.372525in}{7.102117in}}{\pgfqpoint{3.383575in}{7.102117in}}%
\pgfpathclose%
\pgfusepath{stroke,fill}%
\end{pgfscope}%
\begin{pgfscope}%
\pgfpathrectangle{\pgfqpoint{0.570343in}{0.331635in}}{\pgfqpoint{9.300000in}{7.700000in}}%
\pgfusepath{clip}%
\pgfsetbuttcap%
\pgfsetroundjoin%
\definecolor{currentfill}{rgb}{0.631373,0.788235,0.956863}%
\pgfsetfillcolor{currentfill}%
\pgfsetlinewidth{0.481800pt}%
\definecolor{currentstroke}{rgb}{1.000000,1.000000,1.000000}%
\pgfsetstrokecolor{currentstroke}%
\pgfsetdash{}{0pt}%
\pgfpathmoveto{\pgfqpoint{3.614236in}{4.220995in}}%
\pgfpathcurveto{\pgfqpoint{3.625286in}{4.220995in}}{\pgfqpoint{3.635885in}{4.225385in}}{\pgfqpoint{3.643698in}{4.233199in}}%
\pgfpathcurveto{\pgfqpoint{3.651512in}{4.241013in}}{\pgfqpoint{3.655902in}{4.251612in}}{\pgfqpoint{3.655902in}{4.262662in}}%
\pgfpathcurveto{\pgfqpoint{3.655902in}{4.273712in}}{\pgfqpoint{3.651512in}{4.284311in}}{\pgfqpoint{3.643698in}{4.292125in}}%
\pgfpathcurveto{\pgfqpoint{3.635885in}{4.299938in}}{\pgfqpoint{3.625286in}{4.304328in}}{\pgfqpoint{3.614236in}{4.304328in}}%
\pgfpathcurveto{\pgfqpoint{3.603185in}{4.304328in}}{\pgfqpoint{3.592586in}{4.299938in}}{\pgfqpoint{3.584773in}{4.292125in}}%
\pgfpathcurveto{\pgfqpoint{3.576959in}{4.284311in}}{\pgfqpoint{3.572569in}{4.273712in}}{\pgfqpoint{3.572569in}{4.262662in}}%
\pgfpathcurveto{\pgfqpoint{3.572569in}{4.251612in}}{\pgfqpoint{3.576959in}{4.241013in}}{\pgfqpoint{3.584773in}{4.233199in}}%
\pgfpathcurveto{\pgfqpoint{3.592586in}{4.225385in}}{\pgfqpoint{3.603185in}{4.220995in}}{\pgfqpoint{3.614236in}{4.220995in}}%
\pgfpathclose%
\pgfusepath{stroke,fill}%
\end{pgfscope}%
\begin{pgfscope}%
\pgfpathrectangle{\pgfqpoint{0.570343in}{0.331635in}}{\pgfqpoint{9.300000in}{7.700000in}}%
\pgfusepath{clip}%
\pgfsetbuttcap%
\pgfsetroundjoin%
\definecolor{currentfill}{rgb}{0.631373,0.788235,0.956863}%
\pgfsetfillcolor{currentfill}%
\pgfsetlinewidth{0.481800pt}%
\definecolor{currentstroke}{rgb}{1.000000,1.000000,1.000000}%
\pgfsetstrokecolor{currentstroke}%
\pgfsetdash{}{0pt}%
\pgfpathmoveto{\pgfqpoint{8.088914in}{1.897247in}}%
\pgfpathcurveto{\pgfqpoint{8.099964in}{1.897247in}}{\pgfqpoint{8.110563in}{1.901637in}}{\pgfqpoint{8.118377in}{1.909451in}}%
\pgfpathcurveto{\pgfqpoint{8.126190in}{1.917265in}}{\pgfqpoint{8.130581in}{1.927864in}}{\pgfqpoint{8.130581in}{1.938914in}}%
\pgfpathcurveto{\pgfqpoint{8.130581in}{1.949964in}}{\pgfqpoint{8.126190in}{1.960563in}}{\pgfqpoint{8.118377in}{1.968377in}}%
\pgfpathcurveto{\pgfqpoint{8.110563in}{1.976190in}}{\pgfqpoint{8.099964in}{1.980580in}}{\pgfqpoint{8.088914in}{1.980580in}}%
\pgfpathcurveto{\pgfqpoint{8.077864in}{1.980580in}}{\pgfqpoint{8.067265in}{1.976190in}}{\pgfqpoint{8.059451in}{1.968377in}}%
\pgfpathcurveto{\pgfqpoint{8.051637in}{1.960563in}}{\pgfqpoint{8.047247in}{1.949964in}}{\pgfqpoint{8.047247in}{1.938914in}}%
\pgfpathcurveto{\pgfqpoint{8.047247in}{1.927864in}}{\pgfqpoint{8.051637in}{1.917265in}}{\pgfqpoint{8.059451in}{1.909451in}}%
\pgfpathcurveto{\pgfqpoint{8.067265in}{1.901637in}}{\pgfqpoint{8.077864in}{1.897247in}}{\pgfqpoint{8.088914in}{1.897247in}}%
\pgfpathclose%
\pgfusepath{stroke,fill}%
\end{pgfscope}%
\begin{pgfscope}%
\pgfpathrectangle{\pgfqpoint{0.570343in}{0.331635in}}{\pgfqpoint{9.300000in}{7.700000in}}%
\pgfusepath{clip}%
\pgfsetbuttcap%
\pgfsetroundjoin%
\definecolor{currentfill}{rgb}{0.631373,0.788235,0.956863}%
\pgfsetfillcolor{currentfill}%
\pgfsetlinewidth{0.481800pt}%
\definecolor{currentstroke}{rgb}{1.000000,1.000000,1.000000}%
\pgfsetstrokecolor{currentstroke}%
\pgfsetdash{}{0pt}%
\pgfpathmoveto{\pgfqpoint{6.893373in}{7.586771in}}%
\pgfpathcurveto{\pgfqpoint{6.904424in}{7.586771in}}{\pgfqpoint{6.915023in}{7.591161in}}{\pgfqpoint{6.922836in}{7.598975in}}%
\pgfpathcurveto{\pgfqpoint{6.930650in}{7.606788in}}{\pgfqpoint{6.935040in}{7.617387in}}{\pgfqpoint{6.935040in}{7.628437in}}%
\pgfpathcurveto{\pgfqpoint{6.935040in}{7.639487in}}{\pgfqpoint{6.930650in}{7.650086in}}{\pgfqpoint{6.922836in}{7.657900in}}%
\pgfpathcurveto{\pgfqpoint{6.915023in}{7.665714in}}{\pgfqpoint{6.904424in}{7.670104in}}{\pgfqpoint{6.893373in}{7.670104in}}%
\pgfpathcurveto{\pgfqpoint{6.882323in}{7.670104in}}{\pgfqpoint{6.871724in}{7.665714in}}{\pgfqpoint{6.863911in}{7.657900in}}%
\pgfpathcurveto{\pgfqpoint{6.856097in}{7.650086in}}{\pgfqpoint{6.851707in}{7.639487in}}{\pgfqpoint{6.851707in}{7.628437in}}%
\pgfpathcurveto{\pgfqpoint{6.851707in}{7.617387in}}{\pgfqpoint{6.856097in}{7.606788in}}{\pgfqpoint{6.863911in}{7.598975in}}%
\pgfpathcurveto{\pgfqpoint{6.871724in}{7.591161in}}{\pgfqpoint{6.882323in}{7.586771in}}{\pgfqpoint{6.893373in}{7.586771in}}%
\pgfpathclose%
\pgfusepath{stroke,fill}%
\end{pgfscope}%
\begin{pgfscope}%
\pgfpathrectangle{\pgfqpoint{0.570343in}{0.331635in}}{\pgfqpoint{9.300000in}{7.700000in}}%
\pgfusepath{clip}%
\pgfsetbuttcap%
\pgfsetroundjoin%
\definecolor{currentfill}{rgb}{0.631373,0.788235,0.956863}%
\pgfsetfillcolor{currentfill}%
\pgfsetlinewidth{0.481800pt}%
\definecolor{currentstroke}{rgb}{1.000000,1.000000,1.000000}%
\pgfsetstrokecolor{currentstroke}%
\pgfsetdash{}{0pt}%
\pgfpathmoveto{\pgfqpoint{8.123953in}{2.895214in}}%
\pgfpathcurveto{\pgfqpoint{8.135003in}{2.895214in}}{\pgfqpoint{8.145602in}{2.899604in}}{\pgfqpoint{8.153416in}{2.907418in}}%
\pgfpathcurveto{\pgfqpoint{8.161229in}{2.915231in}}{\pgfqpoint{8.165620in}{2.925830in}}{\pgfqpoint{8.165620in}{2.936881in}}%
\pgfpathcurveto{\pgfqpoint{8.165620in}{2.947931in}}{\pgfqpoint{8.161229in}{2.958530in}}{\pgfqpoint{8.153416in}{2.966343in}}%
\pgfpathcurveto{\pgfqpoint{8.145602in}{2.974157in}}{\pgfqpoint{8.135003in}{2.978547in}}{\pgfqpoint{8.123953in}{2.978547in}}%
\pgfpathcurveto{\pgfqpoint{8.112903in}{2.978547in}}{\pgfqpoint{8.102304in}{2.974157in}}{\pgfqpoint{8.094490in}{2.966343in}}%
\pgfpathcurveto{\pgfqpoint{8.086676in}{2.958530in}}{\pgfqpoint{8.082286in}{2.947931in}}{\pgfqpoint{8.082286in}{2.936881in}}%
\pgfpathcurveto{\pgfqpoint{8.082286in}{2.925830in}}{\pgfqpoint{8.086676in}{2.915231in}}{\pgfqpoint{8.094490in}{2.907418in}}%
\pgfpathcurveto{\pgfqpoint{8.102304in}{2.899604in}}{\pgfqpoint{8.112903in}{2.895214in}}{\pgfqpoint{8.123953in}{2.895214in}}%
\pgfpathclose%
\pgfusepath{stroke,fill}%
\end{pgfscope}%
\begin{pgfscope}%
\pgfpathrectangle{\pgfqpoint{0.570343in}{0.331635in}}{\pgfqpoint{9.300000in}{7.700000in}}%
\pgfusepath{clip}%
\pgfsetbuttcap%
\pgfsetroundjoin%
\definecolor{currentfill}{rgb}{0.631373,0.788235,0.956863}%
\pgfsetfillcolor{currentfill}%
\pgfsetlinewidth{0.481800pt}%
\definecolor{currentstroke}{rgb}{1.000000,1.000000,1.000000}%
\pgfsetstrokecolor{currentstroke}%
\pgfsetdash{}{0pt}%
\pgfpathmoveto{\pgfqpoint{6.741009in}{5.773028in}}%
\pgfpathcurveto{\pgfqpoint{6.752059in}{5.773028in}}{\pgfqpoint{6.762658in}{5.777418in}}{\pgfqpoint{6.770472in}{5.785231in}}%
\pgfpathcurveto{\pgfqpoint{6.778285in}{5.793045in}}{\pgfqpoint{6.782675in}{5.803644in}}{\pgfqpoint{6.782675in}{5.814694in}}%
\pgfpathcurveto{\pgfqpoint{6.782675in}{5.825744in}}{\pgfqpoint{6.778285in}{5.836343in}}{\pgfqpoint{6.770472in}{5.844157in}}%
\pgfpathcurveto{\pgfqpoint{6.762658in}{5.851971in}}{\pgfqpoint{6.752059in}{5.856361in}}{\pgfqpoint{6.741009in}{5.856361in}}%
\pgfpathcurveto{\pgfqpoint{6.729959in}{5.856361in}}{\pgfqpoint{6.719360in}{5.851971in}}{\pgfqpoint{6.711546in}{5.844157in}}%
\pgfpathcurveto{\pgfqpoint{6.703732in}{5.836343in}}{\pgfqpoint{6.699342in}{5.825744in}}{\pgfqpoint{6.699342in}{5.814694in}}%
\pgfpathcurveto{\pgfqpoint{6.699342in}{5.803644in}}{\pgfqpoint{6.703732in}{5.793045in}}{\pgfqpoint{6.711546in}{5.785231in}}%
\pgfpathcurveto{\pgfqpoint{6.719360in}{5.777418in}}{\pgfqpoint{6.729959in}{5.773028in}}{\pgfqpoint{6.741009in}{5.773028in}}%
\pgfpathclose%
\pgfusepath{stroke,fill}%
\end{pgfscope}%
\begin{pgfscope}%
\pgfpathrectangle{\pgfqpoint{0.570343in}{0.331635in}}{\pgfqpoint{9.300000in}{7.700000in}}%
\pgfusepath{clip}%
\pgfsetbuttcap%
\pgfsetroundjoin%
\definecolor{currentfill}{rgb}{0.631373,0.788235,0.956863}%
\pgfsetfillcolor{currentfill}%
\pgfsetlinewidth{0.481800pt}%
\definecolor{currentstroke}{rgb}{1.000000,1.000000,1.000000}%
\pgfsetstrokecolor{currentstroke}%
\pgfsetdash{}{0pt}%
\pgfpathmoveto{\pgfqpoint{4.238932in}{6.370567in}}%
\pgfpathcurveto{\pgfqpoint{4.249983in}{6.370567in}}{\pgfqpoint{4.260582in}{6.374957in}}{\pgfqpoint{4.268395in}{6.382771in}}%
\pgfpathcurveto{\pgfqpoint{4.276209in}{6.390585in}}{\pgfqpoint{4.280599in}{6.401184in}}{\pgfqpoint{4.280599in}{6.412234in}}%
\pgfpathcurveto{\pgfqpoint{4.280599in}{6.423284in}}{\pgfqpoint{4.276209in}{6.433883in}}{\pgfqpoint{4.268395in}{6.441697in}}%
\pgfpathcurveto{\pgfqpoint{4.260582in}{6.449510in}}{\pgfqpoint{4.249983in}{6.453901in}}{\pgfqpoint{4.238932in}{6.453901in}}%
\pgfpathcurveto{\pgfqpoint{4.227882in}{6.453901in}}{\pgfqpoint{4.217283in}{6.449510in}}{\pgfqpoint{4.209470in}{6.441697in}}%
\pgfpathcurveto{\pgfqpoint{4.201656in}{6.433883in}}{\pgfqpoint{4.197266in}{6.423284in}}{\pgfqpoint{4.197266in}{6.412234in}}%
\pgfpathcurveto{\pgfqpoint{4.197266in}{6.401184in}}{\pgfqpoint{4.201656in}{6.390585in}}{\pgfqpoint{4.209470in}{6.382771in}}%
\pgfpathcurveto{\pgfqpoint{4.217283in}{6.374957in}}{\pgfqpoint{4.227882in}{6.370567in}}{\pgfqpoint{4.238932in}{6.370567in}}%
\pgfpathclose%
\pgfusepath{stroke,fill}%
\end{pgfscope}%
\begin{pgfscope}%
\pgfpathrectangle{\pgfqpoint{0.570343in}{0.331635in}}{\pgfqpoint{9.300000in}{7.700000in}}%
\pgfusepath{clip}%
\pgfsetbuttcap%
\pgfsetroundjoin%
\definecolor{currentfill}{rgb}{0.631373,0.788235,0.956863}%
\pgfsetfillcolor{currentfill}%
\pgfsetlinewidth{0.481800pt}%
\definecolor{currentstroke}{rgb}{1.000000,1.000000,1.000000}%
\pgfsetstrokecolor{currentstroke}%
\pgfsetdash{}{0pt}%
\pgfpathmoveto{\pgfqpoint{3.437634in}{3.457680in}}%
\pgfpathcurveto{\pgfqpoint{3.448684in}{3.457680in}}{\pgfqpoint{3.459283in}{3.462070in}}{\pgfqpoint{3.467097in}{3.469884in}}%
\pgfpathcurveto{\pgfqpoint{3.474911in}{3.477697in}}{\pgfqpoint{3.479301in}{3.488296in}}{\pgfqpoint{3.479301in}{3.499346in}}%
\pgfpathcurveto{\pgfqpoint{3.479301in}{3.510396in}}{\pgfqpoint{3.474911in}{3.520996in}}{\pgfqpoint{3.467097in}{3.528809in}}%
\pgfpathcurveto{\pgfqpoint{3.459283in}{3.536623in}}{\pgfqpoint{3.448684in}{3.541013in}}{\pgfqpoint{3.437634in}{3.541013in}}%
\pgfpathcurveto{\pgfqpoint{3.426584in}{3.541013in}}{\pgfqpoint{3.415985in}{3.536623in}}{\pgfqpoint{3.408171in}{3.528809in}}%
\pgfpathcurveto{\pgfqpoint{3.400358in}{3.520996in}}{\pgfqpoint{3.395968in}{3.510396in}}{\pgfqpoint{3.395968in}{3.499346in}}%
\pgfpathcurveto{\pgfqpoint{3.395968in}{3.488296in}}{\pgfqpoint{3.400358in}{3.477697in}}{\pgfqpoint{3.408171in}{3.469884in}}%
\pgfpathcurveto{\pgfqpoint{3.415985in}{3.462070in}}{\pgfqpoint{3.426584in}{3.457680in}}{\pgfqpoint{3.437634in}{3.457680in}}%
\pgfpathclose%
\pgfusepath{stroke,fill}%
\end{pgfscope}%
\begin{pgfscope}%
\pgfpathrectangle{\pgfqpoint{0.570343in}{0.331635in}}{\pgfqpoint{9.300000in}{7.700000in}}%
\pgfusepath{clip}%
\pgfsetbuttcap%
\pgfsetroundjoin%
\definecolor{currentfill}{rgb}{0.631373,0.788235,0.956863}%
\pgfsetfillcolor{currentfill}%
\pgfsetlinewidth{0.481800pt}%
\definecolor{currentstroke}{rgb}{1.000000,1.000000,1.000000}%
\pgfsetstrokecolor{currentstroke}%
\pgfsetdash{}{0pt}%
\pgfpathmoveto{\pgfqpoint{1.464475in}{6.181012in}}%
\pgfpathcurveto{\pgfqpoint{1.475525in}{6.181012in}}{\pgfqpoint{1.486125in}{6.185403in}}{\pgfqpoint{1.493938in}{6.193216in}}%
\pgfpathcurveto{\pgfqpoint{1.501752in}{6.201030in}}{\pgfqpoint{1.506142in}{6.211629in}}{\pgfqpoint{1.506142in}{6.222679in}}%
\pgfpathcurveto{\pgfqpoint{1.506142in}{6.233729in}}{\pgfqpoint{1.501752in}{6.244328in}}{\pgfqpoint{1.493938in}{6.252142in}}%
\pgfpathcurveto{\pgfqpoint{1.486125in}{6.259955in}}{\pgfqpoint{1.475525in}{6.264346in}}{\pgfqpoint{1.464475in}{6.264346in}}%
\pgfpathcurveto{\pgfqpoint{1.453425in}{6.264346in}}{\pgfqpoint{1.442826in}{6.259955in}}{\pgfqpoint{1.435013in}{6.252142in}}%
\pgfpathcurveto{\pgfqpoint{1.427199in}{6.244328in}}{\pgfqpoint{1.422809in}{6.233729in}}{\pgfqpoint{1.422809in}{6.222679in}}%
\pgfpathcurveto{\pgfqpoint{1.422809in}{6.211629in}}{\pgfqpoint{1.427199in}{6.201030in}}{\pgfqpoint{1.435013in}{6.193216in}}%
\pgfpathcurveto{\pgfqpoint{1.442826in}{6.185403in}}{\pgfqpoint{1.453425in}{6.181012in}}{\pgfqpoint{1.464475in}{6.181012in}}%
\pgfpathclose%
\pgfusepath{stroke,fill}%
\end{pgfscope}%
\begin{pgfscope}%
\pgfpathrectangle{\pgfqpoint{0.570343in}{0.331635in}}{\pgfqpoint{9.300000in}{7.700000in}}%
\pgfusepath{clip}%
\pgfsetbuttcap%
\pgfsetroundjoin%
\definecolor{currentfill}{rgb}{0.631373,0.788235,0.956863}%
\pgfsetfillcolor{currentfill}%
\pgfsetlinewidth{0.481800pt}%
\definecolor{currentstroke}{rgb}{1.000000,1.000000,1.000000}%
\pgfsetstrokecolor{currentstroke}%
\pgfsetdash{}{0pt}%
\pgfpathmoveto{\pgfqpoint{0.993071in}{4.919062in}}%
\pgfpathcurveto{\pgfqpoint{1.004121in}{4.919062in}}{\pgfqpoint{1.014720in}{4.923452in}}{\pgfqpoint{1.022533in}{4.931266in}}%
\pgfpathcurveto{\pgfqpoint{1.030347in}{4.939080in}}{\pgfqpoint{1.034737in}{4.949679in}}{\pgfqpoint{1.034737in}{4.960729in}}%
\pgfpathcurveto{\pgfqpoint{1.034737in}{4.971779in}}{\pgfqpoint{1.030347in}{4.982378in}}{\pgfqpoint{1.022533in}{4.990192in}}%
\pgfpathcurveto{\pgfqpoint{1.014720in}{4.998005in}}{\pgfqpoint{1.004121in}{5.002395in}}{\pgfqpoint{0.993071in}{5.002395in}}%
\pgfpathcurveto{\pgfqpoint{0.982020in}{5.002395in}}{\pgfqpoint{0.971421in}{4.998005in}}{\pgfqpoint{0.963608in}{4.990192in}}%
\pgfpathcurveto{\pgfqpoint{0.955794in}{4.982378in}}{\pgfqpoint{0.951404in}{4.971779in}}{\pgfqpoint{0.951404in}{4.960729in}}%
\pgfpathcurveto{\pgfqpoint{0.951404in}{4.949679in}}{\pgfqpoint{0.955794in}{4.939080in}}{\pgfqpoint{0.963608in}{4.931266in}}%
\pgfpathcurveto{\pgfqpoint{0.971421in}{4.923452in}}{\pgfqpoint{0.982020in}{4.919062in}}{\pgfqpoint{0.993071in}{4.919062in}}%
\pgfpathclose%
\pgfusepath{stroke,fill}%
\end{pgfscope}%
\begin{pgfscope}%
\pgfpathrectangle{\pgfqpoint{0.570343in}{0.331635in}}{\pgfqpoint{9.300000in}{7.700000in}}%
\pgfusepath{clip}%
\pgfsetbuttcap%
\pgfsetroundjoin%
\definecolor{currentfill}{rgb}{0.631373,0.788235,0.956863}%
\pgfsetfillcolor{currentfill}%
\pgfsetlinewidth{0.481800pt}%
\definecolor{currentstroke}{rgb}{1.000000,1.000000,1.000000}%
\pgfsetstrokecolor{currentstroke}%
\pgfsetdash{}{0pt}%
\pgfpathmoveto{\pgfqpoint{5.578651in}{5.921356in}}%
\pgfpathcurveto{\pgfqpoint{5.589701in}{5.921356in}}{\pgfqpoint{5.600300in}{5.925746in}}{\pgfqpoint{5.608114in}{5.933560in}}%
\pgfpathcurveto{\pgfqpoint{5.615927in}{5.941374in}}{\pgfqpoint{5.620318in}{5.951973in}}{\pgfqpoint{5.620318in}{5.963023in}}%
\pgfpathcurveto{\pgfqpoint{5.620318in}{5.974073in}}{\pgfqpoint{5.615927in}{5.984672in}}{\pgfqpoint{5.608114in}{5.992485in}}%
\pgfpathcurveto{\pgfqpoint{5.600300in}{6.000299in}}{\pgfqpoint{5.589701in}{6.004689in}}{\pgfqpoint{5.578651in}{6.004689in}}%
\pgfpathcurveto{\pgfqpoint{5.567601in}{6.004689in}}{\pgfqpoint{5.557002in}{6.000299in}}{\pgfqpoint{5.549188in}{5.992485in}}%
\pgfpathcurveto{\pgfqpoint{5.541375in}{5.984672in}}{\pgfqpoint{5.536984in}{5.974073in}}{\pgfqpoint{5.536984in}{5.963023in}}%
\pgfpathcurveto{\pgfqpoint{5.536984in}{5.951973in}}{\pgfqpoint{5.541375in}{5.941374in}}{\pgfqpoint{5.549188in}{5.933560in}}%
\pgfpathcurveto{\pgfqpoint{5.557002in}{5.925746in}}{\pgfqpoint{5.567601in}{5.921356in}}{\pgfqpoint{5.578651in}{5.921356in}}%
\pgfpathclose%
\pgfusepath{stroke,fill}%
\end{pgfscope}%
\begin{pgfscope}%
\pgfpathrectangle{\pgfqpoint{0.570343in}{0.331635in}}{\pgfqpoint{9.300000in}{7.700000in}}%
\pgfusepath{clip}%
\pgfsetbuttcap%
\pgfsetroundjoin%
\definecolor{currentfill}{rgb}{0.631373,0.788235,0.956863}%
\pgfsetfillcolor{currentfill}%
\pgfsetlinewidth{0.481800pt}%
\definecolor{currentstroke}{rgb}{1.000000,1.000000,1.000000}%
\pgfsetstrokecolor{currentstroke}%
\pgfsetdash{}{0pt}%
\pgfpathmoveto{\pgfqpoint{3.551278in}{5.591021in}}%
\pgfpathcurveto{\pgfqpoint{3.562328in}{5.591021in}}{\pgfqpoint{3.572927in}{5.595411in}}{\pgfqpoint{3.580741in}{5.603225in}}%
\pgfpathcurveto{\pgfqpoint{3.588554in}{5.611038in}}{\pgfqpoint{3.592945in}{5.621637in}}{\pgfqpoint{3.592945in}{5.632687in}}%
\pgfpathcurveto{\pgfqpoint{3.592945in}{5.643738in}}{\pgfqpoint{3.588554in}{5.654337in}}{\pgfqpoint{3.580741in}{5.662150in}}%
\pgfpathcurveto{\pgfqpoint{3.572927in}{5.669964in}}{\pgfqpoint{3.562328in}{5.674354in}}{\pgfqpoint{3.551278in}{5.674354in}}%
\pgfpathcurveto{\pgfqpoint{3.540228in}{5.674354in}}{\pgfqpoint{3.529629in}{5.669964in}}{\pgfqpoint{3.521815in}{5.662150in}}%
\pgfpathcurveto{\pgfqpoint{3.514002in}{5.654337in}}{\pgfqpoint{3.509611in}{5.643738in}}{\pgfqpoint{3.509611in}{5.632687in}}%
\pgfpathcurveto{\pgfqpoint{3.509611in}{5.621637in}}{\pgfqpoint{3.514002in}{5.611038in}}{\pgfqpoint{3.521815in}{5.603225in}}%
\pgfpathcurveto{\pgfqpoint{3.529629in}{5.595411in}}{\pgfqpoint{3.540228in}{5.591021in}}{\pgfqpoint{3.551278in}{5.591021in}}%
\pgfpathclose%
\pgfusepath{stroke,fill}%
\end{pgfscope}%
\begin{pgfscope}%
\pgfpathrectangle{\pgfqpoint{0.570343in}{0.331635in}}{\pgfqpoint{9.300000in}{7.700000in}}%
\pgfusepath{clip}%
\pgfsetbuttcap%
\pgfsetroundjoin%
\definecolor{currentfill}{rgb}{0.631373,0.788235,0.956863}%
\pgfsetfillcolor{currentfill}%
\pgfsetlinewidth{0.481800pt}%
\definecolor{currentstroke}{rgb}{1.000000,1.000000,1.000000}%
\pgfsetstrokecolor{currentstroke}%
\pgfsetdash{}{0pt}%
\pgfpathmoveto{\pgfqpoint{7.539850in}{4.394674in}}%
\pgfpathcurveto{\pgfqpoint{7.550900in}{4.394674in}}{\pgfqpoint{7.561499in}{4.399064in}}{\pgfqpoint{7.569313in}{4.406877in}}%
\pgfpathcurveto{\pgfqpoint{7.577126in}{4.414691in}}{\pgfqpoint{7.581517in}{4.425290in}}{\pgfqpoint{7.581517in}{4.436340in}}%
\pgfpathcurveto{\pgfqpoint{7.581517in}{4.447390in}}{\pgfqpoint{7.577126in}{4.457989in}}{\pgfqpoint{7.569313in}{4.465803in}}%
\pgfpathcurveto{\pgfqpoint{7.561499in}{4.473617in}}{\pgfqpoint{7.550900in}{4.478007in}}{\pgfqpoint{7.539850in}{4.478007in}}%
\pgfpathcurveto{\pgfqpoint{7.528800in}{4.478007in}}{\pgfqpoint{7.518201in}{4.473617in}}{\pgfqpoint{7.510387in}{4.465803in}}%
\pgfpathcurveto{\pgfqpoint{7.502574in}{4.457989in}}{\pgfqpoint{7.498183in}{4.447390in}}{\pgfqpoint{7.498183in}{4.436340in}}%
\pgfpathcurveto{\pgfqpoint{7.498183in}{4.425290in}}{\pgfqpoint{7.502574in}{4.414691in}}{\pgfqpoint{7.510387in}{4.406877in}}%
\pgfpathcurveto{\pgfqpoint{7.518201in}{4.399064in}}{\pgfqpoint{7.528800in}{4.394674in}}{\pgfqpoint{7.539850in}{4.394674in}}%
\pgfpathclose%
\pgfusepath{stroke,fill}%
\end{pgfscope}%
\begin{pgfscope}%
\pgfpathrectangle{\pgfqpoint{0.570343in}{0.331635in}}{\pgfqpoint{9.300000in}{7.700000in}}%
\pgfusepath{clip}%
\pgfsetbuttcap%
\pgfsetroundjoin%
\definecolor{currentfill}{rgb}{0.631373,0.788235,0.956863}%
\pgfsetfillcolor{currentfill}%
\pgfsetlinewidth{0.481800pt}%
\definecolor{currentstroke}{rgb}{1.000000,1.000000,1.000000}%
\pgfsetstrokecolor{currentstroke}%
\pgfsetdash{}{0pt}%
\pgfpathmoveto{\pgfqpoint{6.633358in}{6.377637in}}%
\pgfpathcurveto{\pgfqpoint{6.644408in}{6.377637in}}{\pgfqpoint{6.655007in}{6.382027in}}{\pgfqpoint{6.662820in}{6.389841in}}%
\pgfpathcurveto{\pgfqpoint{6.670634in}{6.397654in}}{\pgfqpoint{6.675024in}{6.408253in}}{\pgfqpoint{6.675024in}{6.419304in}}%
\pgfpathcurveto{\pgfqpoint{6.675024in}{6.430354in}}{\pgfqpoint{6.670634in}{6.440953in}}{\pgfqpoint{6.662820in}{6.448766in}}%
\pgfpathcurveto{\pgfqpoint{6.655007in}{6.456580in}}{\pgfqpoint{6.644408in}{6.460970in}}{\pgfqpoint{6.633358in}{6.460970in}}%
\pgfpathcurveto{\pgfqpoint{6.622307in}{6.460970in}}{\pgfqpoint{6.611708in}{6.456580in}}{\pgfqpoint{6.603895in}{6.448766in}}%
\pgfpathcurveto{\pgfqpoint{6.596081in}{6.440953in}}{\pgfqpoint{6.591691in}{6.430354in}}{\pgfqpoint{6.591691in}{6.419304in}}%
\pgfpathcurveto{\pgfqpoint{6.591691in}{6.408253in}}{\pgfqpoint{6.596081in}{6.397654in}}{\pgfqpoint{6.603895in}{6.389841in}}%
\pgfpathcurveto{\pgfqpoint{6.611708in}{6.382027in}}{\pgfqpoint{6.622307in}{6.377637in}}{\pgfqpoint{6.633358in}{6.377637in}}%
\pgfpathclose%
\pgfusepath{stroke,fill}%
\end{pgfscope}%
\begin{pgfscope}%
\pgfpathrectangle{\pgfqpoint{0.570343in}{0.331635in}}{\pgfqpoint{9.300000in}{7.700000in}}%
\pgfusepath{clip}%
\pgfsetbuttcap%
\pgfsetroundjoin%
\definecolor{currentfill}{rgb}{0.631373,0.788235,0.956863}%
\pgfsetfillcolor{currentfill}%
\pgfsetlinewidth{0.481800pt}%
\definecolor{currentstroke}{rgb}{1.000000,1.000000,1.000000}%
\pgfsetstrokecolor{currentstroke}%
\pgfsetdash{}{0pt}%
\pgfpathmoveto{\pgfqpoint{9.092676in}{7.043292in}}%
\pgfpathcurveto{\pgfqpoint{9.103727in}{7.043292in}}{\pgfqpoint{9.114326in}{7.047682in}}{\pgfqpoint{9.122139in}{7.055496in}}%
\pgfpathcurveto{\pgfqpoint{9.129953in}{7.063310in}}{\pgfqpoint{9.134343in}{7.073909in}}{\pgfqpoint{9.134343in}{7.084959in}}%
\pgfpathcurveto{\pgfqpoint{9.134343in}{7.096009in}}{\pgfqpoint{9.129953in}{7.106608in}}{\pgfqpoint{9.122139in}{7.114422in}}%
\pgfpathcurveto{\pgfqpoint{9.114326in}{7.122235in}}{\pgfqpoint{9.103727in}{7.126625in}}{\pgfqpoint{9.092676in}{7.126625in}}%
\pgfpathcurveto{\pgfqpoint{9.081626in}{7.126625in}}{\pgfqpoint{9.071027in}{7.122235in}}{\pgfqpoint{9.063214in}{7.114422in}}%
\pgfpathcurveto{\pgfqpoint{9.055400in}{7.106608in}}{\pgfqpoint{9.051010in}{7.096009in}}{\pgfqpoint{9.051010in}{7.084959in}}%
\pgfpathcurveto{\pgfqpoint{9.051010in}{7.073909in}}{\pgfqpoint{9.055400in}{7.063310in}}{\pgfqpoint{9.063214in}{7.055496in}}%
\pgfpathcurveto{\pgfqpoint{9.071027in}{7.047682in}}{\pgfqpoint{9.081626in}{7.043292in}}{\pgfqpoint{9.092676in}{7.043292in}}%
\pgfpathclose%
\pgfusepath{stroke,fill}%
\end{pgfscope}%
\begin{pgfscope}%
\pgfpathrectangle{\pgfqpoint{0.570343in}{0.331635in}}{\pgfqpoint{9.300000in}{7.700000in}}%
\pgfusepath{clip}%
\pgfsetbuttcap%
\pgfsetroundjoin%
\definecolor{currentfill}{rgb}{0.631373,0.788235,0.956863}%
\pgfsetfillcolor{currentfill}%
\pgfsetlinewidth{0.481800pt}%
\definecolor{currentstroke}{rgb}{1.000000,1.000000,1.000000}%
\pgfsetstrokecolor{currentstroke}%
\pgfsetdash{}{0pt}%
\pgfpathmoveto{\pgfqpoint{4.705737in}{5.569235in}}%
\pgfpathcurveto{\pgfqpoint{4.716787in}{5.569235in}}{\pgfqpoint{4.727386in}{5.573625in}}{\pgfqpoint{4.735200in}{5.581439in}}%
\pgfpathcurveto{\pgfqpoint{4.743013in}{5.589253in}}{\pgfqpoint{4.747404in}{5.599852in}}{\pgfqpoint{4.747404in}{5.610902in}}%
\pgfpathcurveto{\pgfqpoint{4.747404in}{5.621952in}}{\pgfqpoint{4.743013in}{5.632551in}}{\pgfqpoint{4.735200in}{5.640365in}}%
\pgfpathcurveto{\pgfqpoint{4.727386in}{5.648178in}}{\pgfqpoint{4.716787in}{5.652569in}}{\pgfqpoint{4.705737in}{5.652569in}}%
\pgfpathcurveto{\pgfqpoint{4.694687in}{5.652569in}}{\pgfqpoint{4.684088in}{5.648178in}}{\pgfqpoint{4.676274in}{5.640365in}}%
\pgfpathcurveto{\pgfqpoint{4.668461in}{5.632551in}}{\pgfqpoint{4.664070in}{5.621952in}}{\pgfqpoint{4.664070in}{5.610902in}}%
\pgfpathcurveto{\pgfqpoint{4.664070in}{5.599852in}}{\pgfqpoint{4.668461in}{5.589253in}}{\pgfqpoint{4.676274in}{5.581439in}}%
\pgfpathcurveto{\pgfqpoint{4.684088in}{5.573625in}}{\pgfqpoint{4.694687in}{5.569235in}}{\pgfqpoint{4.705737in}{5.569235in}}%
\pgfpathclose%
\pgfusepath{stroke,fill}%
\end{pgfscope}%
\begin{pgfscope}%
\pgfpathrectangle{\pgfqpoint{0.570343in}{0.331635in}}{\pgfqpoint{9.300000in}{7.700000in}}%
\pgfusepath{clip}%
\pgfsetbuttcap%
\pgfsetroundjoin%
\definecolor{currentfill}{rgb}{0.631373,0.788235,0.956863}%
\pgfsetfillcolor{currentfill}%
\pgfsetlinewidth{0.481800pt}%
\definecolor{currentstroke}{rgb}{1.000000,1.000000,1.000000}%
\pgfsetstrokecolor{currentstroke}%
\pgfsetdash{}{0pt}%
\pgfpathmoveto{\pgfqpoint{3.116901in}{2.790322in}}%
\pgfpathcurveto{\pgfqpoint{3.127951in}{2.790322in}}{\pgfqpoint{3.138550in}{2.794712in}}{\pgfqpoint{3.146364in}{2.802525in}}%
\pgfpathcurveto{\pgfqpoint{3.154177in}{2.810339in}}{\pgfqpoint{3.158568in}{2.820938in}}{\pgfqpoint{3.158568in}{2.831988in}}%
\pgfpathcurveto{\pgfqpoint{3.158568in}{2.843038in}}{\pgfqpoint{3.154177in}{2.853637in}}{\pgfqpoint{3.146364in}{2.861451in}}%
\pgfpathcurveto{\pgfqpoint{3.138550in}{2.869265in}}{\pgfqpoint{3.127951in}{2.873655in}}{\pgfqpoint{3.116901in}{2.873655in}}%
\pgfpathcurveto{\pgfqpoint{3.105851in}{2.873655in}}{\pgfqpoint{3.095252in}{2.869265in}}{\pgfqpoint{3.087438in}{2.861451in}}%
\pgfpathcurveto{\pgfqpoint{3.079625in}{2.853637in}}{\pgfqpoint{3.075234in}{2.843038in}}{\pgfqpoint{3.075234in}{2.831988in}}%
\pgfpathcurveto{\pgfqpoint{3.075234in}{2.820938in}}{\pgfqpoint{3.079625in}{2.810339in}}{\pgfqpoint{3.087438in}{2.802525in}}%
\pgfpathcurveto{\pgfqpoint{3.095252in}{2.794712in}}{\pgfqpoint{3.105851in}{2.790322in}}{\pgfqpoint{3.116901in}{2.790322in}}%
\pgfpathclose%
\pgfusepath{stroke,fill}%
\end{pgfscope}%
\begin{pgfscope}%
\pgfpathrectangle{\pgfqpoint{0.570343in}{0.331635in}}{\pgfqpoint{9.300000in}{7.700000in}}%
\pgfusepath{clip}%
\pgfsetbuttcap%
\pgfsetroundjoin%
\definecolor{currentfill}{rgb}{0.631373,0.788235,0.956863}%
\pgfsetfillcolor{currentfill}%
\pgfsetlinewidth{0.481800pt}%
\definecolor{currentstroke}{rgb}{1.000000,1.000000,1.000000}%
\pgfsetstrokecolor{currentstroke}%
\pgfsetdash{}{0pt}%
\pgfpathmoveto{\pgfqpoint{4.125517in}{2.805137in}}%
\pgfpathcurveto{\pgfqpoint{4.136568in}{2.805137in}}{\pgfqpoint{4.147167in}{2.809527in}}{\pgfqpoint{4.154980in}{2.817340in}}%
\pgfpathcurveto{\pgfqpoint{4.162794in}{2.825154in}}{\pgfqpoint{4.167184in}{2.835753in}}{\pgfqpoint{4.167184in}{2.846803in}}%
\pgfpathcurveto{\pgfqpoint{4.167184in}{2.857853in}}{\pgfqpoint{4.162794in}{2.868452in}}{\pgfqpoint{4.154980in}{2.876266in}}%
\pgfpathcurveto{\pgfqpoint{4.147167in}{2.884080in}}{\pgfqpoint{4.136568in}{2.888470in}}{\pgfqpoint{4.125517in}{2.888470in}}%
\pgfpathcurveto{\pgfqpoint{4.114467in}{2.888470in}}{\pgfqpoint{4.103868in}{2.884080in}}{\pgfqpoint{4.096055in}{2.876266in}}%
\pgfpathcurveto{\pgfqpoint{4.088241in}{2.868452in}}{\pgfqpoint{4.083851in}{2.857853in}}{\pgfqpoint{4.083851in}{2.846803in}}%
\pgfpathcurveto{\pgfqpoint{4.083851in}{2.835753in}}{\pgfqpoint{4.088241in}{2.825154in}}{\pgfqpoint{4.096055in}{2.817340in}}%
\pgfpathcurveto{\pgfqpoint{4.103868in}{2.809527in}}{\pgfqpoint{4.114467in}{2.805137in}}{\pgfqpoint{4.125517in}{2.805137in}}%
\pgfpathclose%
\pgfusepath{stroke,fill}%
\end{pgfscope}%
\begin{pgfscope}%
\pgfpathrectangle{\pgfqpoint{0.570343in}{0.331635in}}{\pgfqpoint{9.300000in}{7.700000in}}%
\pgfusepath{clip}%
\pgfsetbuttcap%
\pgfsetroundjoin%
\definecolor{currentfill}{rgb}{0.631373,0.788235,0.956863}%
\pgfsetfillcolor{currentfill}%
\pgfsetlinewidth{0.481800pt}%
\definecolor{currentstroke}{rgb}{1.000000,1.000000,1.000000}%
\pgfsetstrokecolor{currentstroke}%
\pgfsetdash{}{0pt}%
\pgfpathmoveto{\pgfqpoint{3.950923in}{4.944942in}}%
\pgfpathcurveto{\pgfqpoint{3.961973in}{4.944942in}}{\pgfqpoint{3.972572in}{4.949332in}}{\pgfqpoint{3.980385in}{4.957146in}}%
\pgfpathcurveto{\pgfqpoint{3.988199in}{4.964959in}}{\pgfqpoint{3.992589in}{4.975558in}}{\pgfqpoint{3.992589in}{4.986608in}}%
\pgfpathcurveto{\pgfqpoint{3.992589in}{4.997659in}}{\pgfqpoint{3.988199in}{5.008258in}}{\pgfqpoint{3.980385in}{5.016071in}}%
\pgfpathcurveto{\pgfqpoint{3.972572in}{5.023885in}}{\pgfqpoint{3.961973in}{5.028275in}}{\pgfqpoint{3.950923in}{5.028275in}}%
\pgfpathcurveto{\pgfqpoint{3.939873in}{5.028275in}}{\pgfqpoint{3.929274in}{5.023885in}}{\pgfqpoint{3.921460in}{5.016071in}}%
\pgfpathcurveto{\pgfqpoint{3.913646in}{5.008258in}}{\pgfqpoint{3.909256in}{4.997659in}}{\pgfqpoint{3.909256in}{4.986608in}}%
\pgfpathcurveto{\pgfqpoint{3.909256in}{4.975558in}}{\pgfqpoint{3.913646in}{4.964959in}}{\pgfqpoint{3.921460in}{4.957146in}}%
\pgfpathcurveto{\pgfqpoint{3.929274in}{4.949332in}}{\pgfqpoint{3.939873in}{4.944942in}}{\pgfqpoint{3.950923in}{4.944942in}}%
\pgfpathclose%
\pgfusepath{stroke,fill}%
\end{pgfscope}%
\begin{pgfscope}%
\pgfpathrectangle{\pgfqpoint{0.570343in}{0.331635in}}{\pgfqpoint{9.300000in}{7.700000in}}%
\pgfusepath{clip}%
\pgfsetbuttcap%
\pgfsetroundjoin%
\definecolor{currentfill}{rgb}{0.631373,0.788235,0.956863}%
\pgfsetfillcolor{currentfill}%
\pgfsetlinewidth{0.481800pt}%
\definecolor{currentstroke}{rgb}{1.000000,1.000000,1.000000}%
\pgfsetstrokecolor{currentstroke}%
\pgfsetdash{}{0pt}%
\pgfpathmoveto{\pgfqpoint{2.997649in}{1.895736in}}%
\pgfpathcurveto{\pgfqpoint{3.008699in}{1.895736in}}{\pgfqpoint{3.019298in}{1.900126in}}{\pgfqpoint{3.027112in}{1.907940in}}%
\pgfpathcurveto{\pgfqpoint{3.034926in}{1.915753in}}{\pgfqpoint{3.039316in}{1.926352in}}{\pgfqpoint{3.039316in}{1.937402in}}%
\pgfpathcurveto{\pgfqpoint{3.039316in}{1.948452in}}{\pgfqpoint{3.034926in}{1.959051in}}{\pgfqpoint{3.027112in}{1.966865in}}%
\pgfpathcurveto{\pgfqpoint{3.019298in}{1.974679in}}{\pgfqpoint{3.008699in}{1.979069in}}{\pgfqpoint{2.997649in}{1.979069in}}%
\pgfpathcurveto{\pgfqpoint{2.986599in}{1.979069in}}{\pgfqpoint{2.976000in}{1.974679in}}{\pgfqpoint{2.968186in}{1.966865in}}%
\pgfpathcurveto{\pgfqpoint{2.960373in}{1.959051in}}{\pgfqpoint{2.955983in}{1.948452in}}{\pgfqpoint{2.955983in}{1.937402in}}%
\pgfpathcurveto{\pgfqpoint{2.955983in}{1.926352in}}{\pgfqpoint{2.960373in}{1.915753in}}{\pgfqpoint{2.968186in}{1.907940in}}%
\pgfpathcurveto{\pgfqpoint{2.976000in}{1.900126in}}{\pgfqpoint{2.986599in}{1.895736in}}{\pgfqpoint{2.997649in}{1.895736in}}%
\pgfpathclose%
\pgfusepath{stroke,fill}%
\end{pgfscope}%
\begin{pgfscope}%
\pgfpathrectangle{\pgfqpoint{0.570343in}{0.331635in}}{\pgfqpoint{9.300000in}{7.700000in}}%
\pgfusepath{clip}%
\pgfsetbuttcap%
\pgfsetroundjoin%
\definecolor{currentfill}{rgb}{0.631373,0.788235,0.956863}%
\pgfsetfillcolor{currentfill}%
\pgfsetlinewidth{0.481800pt}%
\definecolor{currentstroke}{rgb}{1.000000,1.000000,1.000000}%
\pgfsetstrokecolor{currentstroke}%
\pgfsetdash{}{0pt}%
\pgfpathmoveto{\pgfqpoint{5.501193in}{6.813359in}}%
\pgfpathcurveto{\pgfqpoint{5.512243in}{6.813359in}}{\pgfqpoint{5.522842in}{6.817749in}}{\pgfqpoint{5.530656in}{6.825563in}}%
\pgfpathcurveto{\pgfqpoint{5.538469in}{6.833377in}}{\pgfqpoint{5.542859in}{6.843976in}}{\pgfqpoint{5.542859in}{6.855026in}}%
\pgfpathcurveto{\pgfqpoint{5.542859in}{6.866076in}}{\pgfqpoint{5.538469in}{6.876675in}}{\pgfqpoint{5.530656in}{6.884489in}}%
\pgfpathcurveto{\pgfqpoint{5.522842in}{6.892302in}}{\pgfqpoint{5.512243in}{6.896692in}}{\pgfqpoint{5.501193in}{6.896692in}}%
\pgfpathcurveto{\pgfqpoint{5.490143in}{6.896692in}}{\pgfqpoint{5.479544in}{6.892302in}}{\pgfqpoint{5.471730in}{6.884489in}}%
\pgfpathcurveto{\pgfqpoint{5.463916in}{6.876675in}}{\pgfqpoint{5.459526in}{6.866076in}}{\pgfqpoint{5.459526in}{6.855026in}}%
\pgfpathcurveto{\pgfqpoint{5.459526in}{6.843976in}}{\pgfqpoint{5.463916in}{6.833377in}}{\pgfqpoint{5.471730in}{6.825563in}}%
\pgfpathcurveto{\pgfqpoint{5.479544in}{6.817749in}}{\pgfqpoint{5.490143in}{6.813359in}}{\pgfqpoint{5.501193in}{6.813359in}}%
\pgfpathclose%
\pgfusepath{stroke,fill}%
\end{pgfscope}%
\begin{pgfscope}%
\pgfpathrectangle{\pgfqpoint{0.570343in}{0.331635in}}{\pgfqpoint{9.300000in}{7.700000in}}%
\pgfusepath{clip}%
\pgfsetbuttcap%
\pgfsetroundjoin%
\definecolor{currentfill}{rgb}{0.631373,0.788235,0.956863}%
\pgfsetfillcolor{currentfill}%
\pgfsetlinewidth{0.481800pt}%
\definecolor{currentstroke}{rgb}{1.000000,1.000000,1.000000}%
\pgfsetstrokecolor{currentstroke}%
\pgfsetdash{}{0pt}%
\pgfpathmoveto{\pgfqpoint{4.900969in}{5.299919in}}%
\pgfpathcurveto{\pgfqpoint{4.912019in}{5.299919in}}{\pgfqpoint{4.922618in}{5.304309in}}{\pgfqpoint{4.930431in}{5.312123in}}%
\pgfpathcurveto{\pgfqpoint{4.938245in}{5.319936in}}{\pgfqpoint{4.942635in}{5.330536in}}{\pgfqpoint{4.942635in}{5.341586in}}%
\pgfpathcurveto{\pgfqpoint{4.942635in}{5.352636in}}{\pgfqpoint{4.938245in}{5.363235in}}{\pgfqpoint{4.930431in}{5.371048in}}%
\pgfpathcurveto{\pgfqpoint{4.922618in}{5.378862in}}{\pgfqpoint{4.912019in}{5.383252in}}{\pgfqpoint{4.900969in}{5.383252in}}%
\pgfpathcurveto{\pgfqpoint{4.889918in}{5.383252in}}{\pgfqpoint{4.879319in}{5.378862in}}{\pgfqpoint{4.871506in}{5.371048in}}%
\pgfpathcurveto{\pgfqpoint{4.863692in}{5.363235in}}{\pgfqpoint{4.859302in}{5.352636in}}{\pgfqpoint{4.859302in}{5.341586in}}%
\pgfpathcurveto{\pgfqpoint{4.859302in}{5.330536in}}{\pgfqpoint{4.863692in}{5.319936in}}{\pgfqpoint{4.871506in}{5.312123in}}%
\pgfpathcurveto{\pgfqpoint{4.879319in}{5.304309in}}{\pgfqpoint{4.889918in}{5.299919in}}{\pgfqpoint{4.900969in}{5.299919in}}%
\pgfpathclose%
\pgfusepath{stroke,fill}%
\end{pgfscope}%
\begin{pgfscope}%
\pgfpathrectangle{\pgfqpoint{0.570343in}{0.331635in}}{\pgfqpoint{9.300000in}{7.700000in}}%
\pgfusepath{clip}%
\pgfsetbuttcap%
\pgfsetroundjoin%
\definecolor{currentfill}{rgb}{1.000000,0.705882,0.509804}%
\pgfsetfillcolor{currentfill}%
\pgfsetlinewidth{0.481800pt}%
\definecolor{currentstroke}{rgb}{1.000000,1.000000,1.000000}%
\pgfsetstrokecolor{currentstroke}%
\pgfsetdash{}{0pt}%
\pgfpathmoveto{\pgfqpoint{5.688827in}{3.673198in}}%
\pgfpathcurveto{\pgfqpoint{5.699877in}{3.673198in}}{\pgfqpoint{5.710476in}{3.677588in}}{\pgfqpoint{5.718290in}{3.685402in}}%
\pgfpathcurveto{\pgfqpoint{5.726104in}{3.693216in}}{\pgfqpoint{5.730494in}{3.703815in}}{\pgfqpoint{5.730494in}{3.714865in}}%
\pgfpathcurveto{\pgfqpoint{5.730494in}{3.725915in}}{\pgfqpoint{5.726104in}{3.736514in}}{\pgfqpoint{5.718290in}{3.744328in}}%
\pgfpathcurveto{\pgfqpoint{5.710476in}{3.752141in}}{\pgfqpoint{5.699877in}{3.756532in}}{\pgfqpoint{5.688827in}{3.756532in}}%
\pgfpathcurveto{\pgfqpoint{5.677777in}{3.756532in}}{\pgfqpoint{5.667178in}{3.752141in}}{\pgfqpoint{5.659364in}{3.744328in}}%
\pgfpathcurveto{\pgfqpoint{5.651551in}{3.736514in}}{\pgfqpoint{5.647161in}{3.725915in}}{\pgfqpoint{5.647161in}{3.714865in}}%
\pgfpathcurveto{\pgfqpoint{5.647161in}{3.703815in}}{\pgfqpoint{5.651551in}{3.693216in}}{\pgfqpoint{5.659364in}{3.685402in}}%
\pgfpathcurveto{\pgfqpoint{5.667178in}{3.677588in}}{\pgfqpoint{5.677777in}{3.673198in}}{\pgfqpoint{5.688827in}{3.673198in}}%
\pgfpathclose%
\pgfusepath{stroke,fill}%
\end{pgfscope}%
\begin{pgfscope}%
\pgfpathrectangle{\pgfqpoint{0.570343in}{0.331635in}}{\pgfqpoint{9.300000in}{7.700000in}}%
\pgfusepath{clip}%
\pgfsetbuttcap%
\pgfsetroundjoin%
\definecolor{currentfill}{rgb}{1.000000,0.705882,0.509804}%
\pgfsetfillcolor{currentfill}%
\pgfsetlinewidth{0.481800pt}%
\definecolor{currentstroke}{rgb}{1.000000,1.000000,1.000000}%
\pgfsetstrokecolor{currentstroke}%
\pgfsetdash{}{0pt}%
\pgfpathmoveto{\pgfqpoint{7.831480in}{6.697908in}}%
\pgfpathcurveto{\pgfqpoint{7.842530in}{6.697908in}}{\pgfqpoint{7.853129in}{6.702298in}}{\pgfqpoint{7.860943in}{6.710112in}}%
\pgfpathcurveto{\pgfqpoint{7.868757in}{6.717925in}}{\pgfqpoint{7.873147in}{6.728524in}}{\pgfqpoint{7.873147in}{6.739574in}}%
\pgfpathcurveto{\pgfqpoint{7.873147in}{6.750625in}}{\pgfqpoint{7.868757in}{6.761224in}}{\pgfqpoint{7.860943in}{6.769037in}}%
\pgfpathcurveto{\pgfqpoint{7.853129in}{6.776851in}}{\pgfqpoint{7.842530in}{6.781241in}}{\pgfqpoint{7.831480in}{6.781241in}}%
\pgfpathcurveto{\pgfqpoint{7.820430in}{6.781241in}}{\pgfqpoint{7.809831in}{6.776851in}}{\pgfqpoint{7.802017in}{6.769037in}}%
\pgfpathcurveto{\pgfqpoint{7.794204in}{6.761224in}}{\pgfqpoint{7.789813in}{6.750625in}}{\pgfqpoint{7.789813in}{6.739574in}}%
\pgfpathcurveto{\pgfqpoint{7.789813in}{6.728524in}}{\pgfqpoint{7.794204in}{6.717925in}}{\pgfqpoint{7.802017in}{6.710112in}}%
\pgfpathcurveto{\pgfqpoint{7.809831in}{6.702298in}}{\pgfqpoint{7.820430in}{6.697908in}}{\pgfqpoint{7.831480in}{6.697908in}}%
\pgfpathclose%
\pgfusepath{stroke,fill}%
\end{pgfscope}%
\begin{pgfscope}%
\pgfpathrectangle{\pgfqpoint{0.570343in}{0.331635in}}{\pgfqpoint{9.300000in}{7.700000in}}%
\pgfusepath{clip}%
\pgfsetbuttcap%
\pgfsetroundjoin%
\definecolor{currentfill}{rgb}{1.000000,0.705882,0.509804}%
\pgfsetfillcolor{currentfill}%
\pgfsetlinewidth{0.481800pt}%
\definecolor{currentstroke}{rgb}{1.000000,1.000000,1.000000}%
\pgfsetstrokecolor{currentstroke}%
\pgfsetdash{}{0pt}%
\pgfpathmoveto{\pgfqpoint{5.338526in}{4.857417in}}%
\pgfpathcurveto{\pgfqpoint{5.349576in}{4.857417in}}{\pgfqpoint{5.360175in}{4.861808in}}{\pgfqpoint{5.367989in}{4.869621in}}%
\pgfpathcurveto{\pgfqpoint{5.375802in}{4.877435in}}{\pgfqpoint{5.380192in}{4.888034in}}{\pgfqpoint{5.380192in}{4.899084in}}%
\pgfpathcurveto{\pgfqpoint{5.380192in}{4.910134in}}{\pgfqpoint{5.375802in}{4.920733in}}{\pgfqpoint{5.367989in}{4.928547in}}%
\pgfpathcurveto{\pgfqpoint{5.360175in}{4.936360in}}{\pgfqpoint{5.349576in}{4.940751in}}{\pgfqpoint{5.338526in}{4.940751in}}%
\pgfpathcurveto{\pgfqpoint{5.327476in}{4.940751in}}{\pgfqpoint{5.316877in}{4.936360in}}{\pgfqpoint{5.309063in}{4.928547in}}%
\pgfpathcurveto{\pgfqpoint{5.301249in}{4.920733in}}{\pgfqpoint{5.296859in}{4.910134in}}{\pgfqpoint{5.296859in}{4.899084in}}%
\pgfpathcurveto{\pgfqpoint{5.296859in}{4.888034in}}{\pgfqpoint{5.301249in}{4.877435in}}{\pgfqpoint{5.309063in}{4.869621in}}%
\pgfpathcurveto{\pgfqpoint{5.316877in}{4.861808in}}{\pgfqpoint{5.327476in}{4.857417in}}{\pgfqpoint{5.338526in}{4.857417in}}%
\pgfpathclose%
\pgfusepath{stroke,fill}%
\end{pgfscope}%
\begin{pgfscope}%
\pgfpathrectangle{\pgfqpoint{0.570343in}{0.331635in}}{\pgfqpoint{9.300000in}{7.700000in}}%
\pgfusepath{clip}%
\pgfsetbuttcap%
\pgfsetroundjoin%
\definecolor{currentfill}{rgb}{1.000000,0.705882,0.509804}%
\pgfsetfillcolor{currentfill}%
\pgfsetlinewidth{0.481800pt}%
\definecolor{currentstroke}{rgb}{1.000000,1.000000,1.000000}%
\pgfsetstrokecolor{currentstroke}%
\pgfsetdash{}{0pt}%
\pgfpathmoveto{\pgfqpoint{6.546058in}{3.193196in}}%
\pgfpathcurveto{\pgfqpoint{6.557108in}{3.193196in}}{\pgfqpoint{6.567707in}{3.197586in}}{\pgfqpoint{6.575520in}{3.205399in}}%
\pgfpathcurveto{\pgfqpoint{6.583334in}{3.213213in}}{\pgfqpoint{6.587724in}{3.223812in}}{\pgfqpoint{6.587724in}{3.234862in}}%
\pgfpathcurveto{\pgfqpoint{6.587724in}{3.245912in}}{\pgfqpoint{6.583334in}{3.256511in}}{\pgfqpoint{6.575520in}{3.264325in}}%
\pgfpathcurveto{\pgfqpoint{6.567707in}{3.272139in}}{\pgfqpoint{6.557108in}{3.276529in}}{\pgfqpoint{6.546058in}{3.276529in}}%
\pgfpathcurveto{\pgfqpoint{6.535008in}{3.276529in}}{\pgfqpoint{6.524408in}{3.272139in}}{\pgfqpoint{6.516595in}{3.264325in}}%
\pgfpathcurveto{\pgfqpoint{6.508781in}{3.256511in}}{\pgfqpoint{6.504391in}{3.245912in}}{\pgfqpoint{6.504391in}{3.234862in}}%
\pgfpathcurveto{\pgfqpoint{6.504391in}{3.223812in}}{\pgfqpoint{6.508781in}{3.213213in}}{\pgfqpoint{6.516595in}{3.205399in}}%
\pgfpathcurveto{\pgfqpoint{6.524408in}{3.197586in}}{\pgfqpoint{6.535008in}{3.193196in}}{\pgfqpoint{6.546058in}{3.193196in}}%
\pgfpathclose%
\pgfusepath{stroke,fill}%
\end{pgfscope}%
\begin{pgfscope}%
\pgfpathrectangle{\pgfqpoint{0.570343in}{0.331635in}}{\pgfqpoint{9.300000in}{7.700000in}}%
\pgfusepath{clip}%
\pgfsetbuttcap%
\pgfsetroundjoin%
\definecolor{currentfill}{rgb}{1.000000,0.705882,0.509804}%
\pgfsetfillcolor{currentfill}%
\pgfsetlinewidth{0.481800pt}%
\definecolor{currentstroke}{rgb}{1.000000,1.000000,1.000000}%
\pgfsetstrokecolor{currentstroke}%
\pgfsetdash{}{0pt}%
\pgfpathmoveto{\pgfqpoint{8.931283in}{5.352643in}}%
\pgfpathcurveto{\pgfqpoint{8.942334in}{5.352643in}}{\pgfqpoint{8.952933in}{5.357033in}}{\pgfqpoint{8.960746in}{5.364847in}}%
\pgfpathcurveto{\pgfqpoint{8.968560in}{5.372660in}}{\pgfqpoint{8.972950in}{5.383259in}}{\pgfqpoint{8.972950in}{5.394309in}}%
\pgfpathcurveto{\pgfqpoint{8.972950in}{5.405360in}}{\pgfqpoint{8.968560in}{5.415959in}}{\pgfqpoint{8.960746in}{5.423772in}}%
\pgfpathcurveto{\pgfqpoint{8.952933in}{5.431586in}}{\pgfqpoint{8.942334in}{5.435976in}}{\pgfqpoint{8.931283in}{5.435976in}}%
\pgfpathcurveto{\pgfqpoint{8.920233in}{5.435976in}}{\pgfqpoint{8.909634in}{5.431586in}}{\pgfqpoint{8.901821in}{5.423772in}}%
\pgfpathcurveto{\pgfqpoint{8.894007in}{5.415959in}}{\pgfqpoint{8.889617in}{5.405360in}}{\pgfqpoint{8.889617in}{5.394309in}}%
\pgfpathcurveto{\pgfqpoint{8.889617in}{5.383259in}}{\pgfqpoint{8.894007in}{5.372660in}}{\pgfqpoint{8.901821in}{5.364847in}}%
\pgfpathcurveto{\pgfqpoint{8.909634in}{5.357033in}}{\pgfqpoint{8.920233in}{5.352643in}}{\pgfqpoint{8.931283in}{5.352643in}}%
\pgfpathclose%
\pgfusepath{stroke,fill}%
\end{pgfscope}%
\begin{pgfscope}%
\pgfpathrectangle{\pgfqpoint{0.570343in}{0.331635in}}{\pgfqpoint{9.300000in}{7.700000in}}%
\pgfusepath{clip}%
\pgfsetbuttcap%
\pgfsetroundjoin%
\definecolor{currentfill}{rgb}{1.000000,0.705882,0.509804}%
\pgfsetfillcolor{currentfill}%
\pgfsetlinewidth{0.481800pt}%
\definecolor{currentstroke}{rgb}{1.000000,1.000000,1.000000}%
\pgfsetstrokecolor{currentstroke}%
\pgfsetdash{}{0pt}%
\pgfpathmoveto{\pgfqpoint{4.854875in}{3.413194in}}%
\pgfpathcurveto{\pgfqpoint{4.865925in}{3.413194in}}{\pgfqpoint{4.876524in}{3.417585in}}{\pgfqpoint{4.884338in}{3.425398in}}%
\pgfpathcurveto{\pgfqpoint{4.892152in}{3.433212in}}{\pgfqpoint{4.896542in}{3.443811in}}{\pgfqpoint{4.896542in}{3.454861in}}%
\pgfpathcurveto{\pgfqpoint{4.896542in}{3.465911in}}{\pgfqpoint{4.892152in}{3.476510in}}{\pgfqpoint{4.884338in}{3.484324in}}%
\pgfpathcurveto{\pgfqpoint{4.876524in}{3.492138in}}{\pgfqpoint{4.865925in}{3.496528in}}{\pgfqpoint{4.854875in}{3.496528in}}%
\pgfpathcurveto{\pgfqpoint{4.843825in}{3.496528in}}{\pgfqpoint{4.833226in}{3.492138in}}{\pgfqpoint{4.825412in}{3.484324in}}%
\pgfpathcurveto{\pgfqpoint{4.817599in}{3.476510in}}{\pgfqpoint{4.813208in}{3.465911in}}{\pgfqpoint{4.813208in}{3.454861in}}%
\pgfpathcurveto{\pgfqpoint{4.813208in}{3.443811in}}{\pgfqpoint{4.817599in}{3.433212in}}{\pgfqpoint{4.825412in}{3.425398in}}%
\pgfpathcurveto{\pgfqpoint{4.833226in}{3.417585in}}{\pgfqpoint{4.843825in}{3.413194in}}{\pgfqpoint{4.854875in}{3.413194in}}%
\pgfpathclose%
\pgfusepath{stroke,fill}%
\end{pgfscope}%
\begin{pgfscope}%
\pgfpathrectangle{\pgfqpoint{0.570343in}{0.331635in}}{\pgfqpoint{9.300000in}{7.700000in}}%
\pgfusepath{clip}%
\pgfsetbuttcap%
\pgfsetroundjoin%
\definecolor{currentfill}{rgb}{1.000000,0.705882,0.509804}%
\pgfsetfillcolor{currentfill}%
\pgfsetlinewidth{0.481800pt}%
\definecolor{currentstroke}{rgb}{1.000000,1.000000,1.000000}%
\pgfsetstrokecolor{currentstroke}%
\pgfsetdash{}{0pt}%
\pgfpathmoveto{\pgfqpoint{7.413796in}{3.602347in}}%
\pgfpathcurveto{\pgfqpoint{7.424846in}{3.602347in}}{\pgfqpoint{7.435445in}{3.606737in}}{\pgfqpoint{7.443258in}{3.614550in}}%
\pgfpathcurveto{\pgfqpoint{7.451072in}{3.622364in}}{\pgfqpoint{7.455462in}{3.632963in}}{\pgfqpoint{7.455462in}{3.644013in}}%
\pgfpathcurveto{\pgfqpoint{7.455462in}{3.655063in}}{\pgfqpoint{7.451072in}{3.665662in}}{\pgfqpoint{7.443258in}{3.673476in}}%
\pgfpathcurveto{\pgfqpoint{7.435445in}{3.681290in}}{\pgfqpoint{7.424846in}{3.685680in}}{\pgfqpoint{7.413796in}{3.685680in}}%
\pgfpathcurveto{\pgfqpoint{7.402745in}{3.685680in}}{\pgfqpoint{7.392146in}{3.681290in}}{\pgfqpoint{7.384333in}{3.673476in}}%
\pgfpathcurveto{\pgfqpoint{7.376519in}{3.665662in}}{\pgfqpoint{7.372129in}{3.655063in}}{\pgfqpoint{7.372129in}{3.644013in}}%
\pgfpathcurveto{\pgfqpoint{7.372129in}{3.632963in}}{\pgfqpoint{7.376519in}{3.622364in}}{\pgfqpoint{7.384333in}{3.614550in}}%
\pgfpathcurveto{\pgfqpoint{7.392146in}{3.606737in}}{\pgfqpoint{7.402745in}{3.602347in}}{\pgfqpoint{7.413796in}{3.602347in}}%
\pgfpathclose%
\pgfusepath{stroke,fill}%
\end{pgfscope}%
\begin{pgfscope}%
\pgfpathrectangle{\pgfqpoint{0.570343in}{0.331635in}}{\pgfqpoint{9.300000in}{7.700000in}}%
\pgfusepath{clip}%
\pgfsetbuttcap%
\pgfsetroundjoin%
\definecolor{currentfill}{rgb}{1.000000,0.705882,0.509804}%
\pgfsetfillcolor{currentfill}%
\pgfsetlinewidth{0.481800pt}%
\definecolor{currentstroke}{rgb}{1.000000,1.000000,1.000000}%
\pgfsetstrokecolor{currentstroke}%
\pgfsetdash{}{0pt}%
\pgfpathmoveto{\pgfqpoint{1.671602in}{3.865489in}}%
\pgfpathcurveto{\pgfqpoint{1.682652in}{3.865489in}}{\pgfqpoint{1.693252in}{3.869879in}}{\pgfqpoint{1.701065in}{3.877693in}}%
\pgfpathcurveto{\pgfqpoint{1.708879in}{3.885507in}}{\pgfqpoint{1.713269in}{3.896106in}}{\pgfqpoint{1.713269in}{3.907156in}}%
\pgfpathcurveto{\pgfqpoint{1.713269in}{3.918206in}}{\pgfqpoint{1.708879in}{3.928805in}}{\pgfqpoint{1.701065in}{3.936619in}}%
\pgfpathcurveto{\pgfqpoint{1.693252in}{3.944432in}}{\pgfqpoint{1.682652in}{3.948823in}}{\pgfqpoint{1.671602in}{3.948823in}}%
\pgfpathcurveto{\pgfqpoint{1.660552in}{3.948823in}}{\pgfqpoint{1.649953in}{3.944432in}}{\pgfqpoint{1.642140in}{3.936619in}}%
\pgfpathcurveto{\pgfqpoint{1.634326in}{3.928805in}}{\pgfqpoint{1.629936in}{3.918206in}}{\pgfqpoint{1.629936in}{3.907156in}}%
\pgfpathcurveto{\pgfqpoint{1.629936in}{3.896106in}}{\pgfqpoint{1.634326in}{3.885507in}}{\pgfqpoint{1.642140in}{3.877693in}}%
\pgfpathcurveto{\pgfqpoint{1.649953in}{3.869879in}}{\pgfqpoint{1.660552in}{3.865489in}}{\pgfqpoint{1.671602in}{3.865489in}}%
\pgfpathclose%
\pgfusepath{stroke,fill}%
\end{pgfscope}%
\begin{pgfscope}%
\pgfpathrectangle{\pgfqpoint{0.570343in}{0.331635in}}{\pgfqpoint{9.300000in}{7.700000in}}%
\pgfusepath{clip}%
\pgfsetbuttcap%
\pgfsetroundjoin%
\definecolor{currentfill}{rgb}{1.000000,0.705882,0.509804}%
\pgfsetfillcolor{currentfill}%
\pgfsetlinewidth{0.481800pt}%
\definecolor{currentstroke}{rgb}{1.000000,1.000000,1.000000}%
\pgfsetstrokecolor{currentstroke}%
\pgfsetdash{}{0pt}%
\pgfpathmoveto{\pgfqpoint{1.885787in}{2.890873in}}%
\pgfpathcurveto{\pgfqpoint{1.896837in}{2.890873in}}{\pgfqpoint{1.907436in}{2.895263in}}{\pgfqpoint{1.915250in}{2.903077in}}%
\pgfpathcurveto{\pgfqpoint{1.923064in}{2.910891in}}{\pgfqpoint{1.927454in}{2.921490in}}{\pgfqpoint{1.927454in}{2.932540in}}%
\pgfpathcurveto{\pgfqpoint{1.927454in}{2.943590in}}{\pgfqpoint{1.923064in}{2.954189in}}{\pgfqpoint{1.915250in}{2.962003in}}%
\pgfpathcurveto{\pgfqpoint{1.907436in}{2.969816in}}{\pgfqpoint{1.896837in}{2.974206in}}{\pgfqpoint{1.885787in}{2.974206in}}%
\pgfpathcurveto{\pgfqpoint{1.874737in}{2.974206in}}{\pgfqpoint{1.864138in}{2.969816in}}{\pgfqpoint{1.856325in}{2.962003in}}%
\pgfpathcurveto{\pgfqpoint{1.848511in}{2.954189in}}{\pgfqpoint{1.844121in}{2.943590in}}{\pgfqpoint{1.844121in}{2.932540in}}%
\pgfpathcurveto{\pgfqpoint{1.844121in}{2.921490in}}{\pgfqpoint{1.848511in}{2.910891in}}{\pgfqpoint{1.856325in}{2.903077in}}%
\pgfpathcurveto{\pgfqpoint{1.864138in}{2.895263in}}{\pgfqpoint{1.874737in}{2.890873in}}{\pgfqpoint{1.885787in}{2.890873in}}%
\pgfpathclose%
\pgfusepath{stroke,fill}%
\end{pgfscope}%
\begin{pgfscope}%
\pgfpathrectangle{\pgfqpoint{0.570343in}{0.331635in}}{\pgfqpoint{9.300000in}{7.700000in}}%
\pgfusepath{clip}%
\pgfsetbuttcap%
\pgfsetroundjoin%
\definecolor{currentfill}{rgb}{1.000000,0.705882,0.509804}%
\pgfsetfillcolor{currentfill}%
\pgfsetlinewidth{0.481800pt}%
\definecolor{currentstroke}{rgb}{1.000000,1.000000,1.000000}%
\pgfsetstrokecolor{currentstroke}%
\pgfsetdash{}{0pt}%
\pgfpathmoveto{\pgfqpoint{5.353276in}{4.246754in}}%
\pgfpathcurveto{\pgfqpoint{5.364327in}{4.246754in}}{\pgfqpoint{5.374926in}{4.251144in}}{\pgfqpoint{5.382739in}{4.258958in}}%
\pgfpathcurveto{\pgfqpoint{5.390553in}{4.266771in}}{\pgfqpoint{5.394943in}{4.277370in}}{\pgfqpoint{5.394943in}{4.288421in}}%
\pgfpathcurveto{\pgfqpoint{5.394943in}{4.299471in}}{\pgfqpoint{5.390553in}{4.310070in}}{\pgfqpoint{5.382739in}{4.317883in}}%
\pgfpathcurveto{\pgfqpoint{5.374926in}{4.325697in}}{\pgfqpoint{5.364327in}{4.330087in}}{\pgfqpoint{5.353276in}{4.330087in}}%
\pgfpathcurveto{\pgfqpoint{5.342226in}{4.330087in}}{\pgfqpoint{5.331627in}{4.325697in}}{\pgfqpoint{5.323814in}{4.317883in}}%
\pgfpathcurveto{\pgfqpoint{5.316000in}{4.310070in}}{\pgfqpoint{5.311610in}{4.299471in}}{\pgfqpoint{5.311610in}{4.288421in}}%
\pgfpathcurveto{\pgfqpoint{5.311610in}{4.277370in}}{\pgfqpoint{5.316000in}{4.266771in}}{\pgfqpoint{5.323814in}{4.258958in}}%
\pgfpathcurveto{\pgfqpoint{5.331627in}{4.251144in}}{\pgfqpoint{5.342226in}{4.246754in}}{\pgfqpoint{5.353276in}{4.246754in}}%
\pgfpathclose%
\pgfusepath{stroke,fill}%
\end{pgfscope}%
\begin{pgfscope}%
\pgfpathrectangle{\pgfqpoint{0.570343in}{0.331635in}}{\pgfqpoint{9.300000in}{7.700000in}}%
\pgfusepath{clip}%
\pgfsetbuttcap%
\pgfsetroundjoin%
\definecolor{currentfill}{rgb}{1.000000,0.705882,0.509804}%
\pgfsetfillcolor{currentfill}%
\pgfsetlinewidth{0.481800pt}%
\definecolor{currentstroke}{rgb}{1.000000,1.000000,1.000000}%
\pgfsetstrokecolor{currentstroke}%
\pgfsetdash{}{0pt}%
\pgfpathmoveto{\pgfqpoint{6.076381in}{4.577443in}}%
\pgfpathcurveto{\pgfqpoint{6.087432in}{4.577443in}}{\pgfqpoint{6.098031in}{4.581833in}}{\pgfqpoint{6.105844in}{4.589647in}}%
\pgfpathcurveto{\pgfqpoint{6.113658in}{4.597461in}}{\pgfqpoint{6.118048in}{4.608060in}}{\pgfqpoint{6.118048in}{4.619110in}}%
\pgfpathcurveto{\pgfqpoint{6.118048in}{4.630160in}}{\pgfqpoint{6.113658in}{4.640759in}}{\pgfqpoint{6.105844in}{4.648572in}}%
\pgfpathcurveto{\pgfqpoint{6.098031in}{4.656386in}}{\pgfqpoint{6.087432in}{4.660776in}}{\pgfqpoint{6.076381in}{4.660776in}}%
\pgfpathcurveto{\pgfqpoint{6.065331in}{4.660776in}}{\pgfqpoint{6.054732in}{4.656386in}}{\pgfqpoint{6.046919in}{4.648572in}}%
\pgfpathcurveto{\pgfqpoint{6.039105in}{4.640759in}}{\pgfqpoint{6.034715in}{4.630160in}}{\pgfqpoint{6.034715in}{4.619110in}}%
\pgfpathcurveto{\pgfqpoint{6.034715in}{4.608060in}}{\pgfqpoint{6.039105in}{4.597461in}}{\pgfqpoint{6.046919in}{4.589647in}}%
\pgfpathcurveto{\pgfqpoint{6.054732in}{4.581833in}}{\pgfqpoint{6.065331in}{4.577443in}}{\pgfqpoint{6.076381in}{4.577443in}}%
\pgfpathclose%
\pgfusepath{stroke,fill}%
\end{pgfscope}%
\begin{pgfscope}%
\pgfpathrectangle{\pgfqpoint{0.570343in}{0.331635in}}{\pgfqpoint{9.300000in}{7.700000in}}%
\pgfusepath{clip}%
\pgfsetbuttcap%
\pgfsetroundjoin%
\definecolor{currentfill}{rgb}{1.000000,0.705882,0.509804}%
\pgfsetfillcolor{currentfill}%
\pgfsetlinewidth{0.481800pt}%
\definecolor{currentstroke}{rgb}{1.000000,1.000000,1.000000}%
\pgfsetstrokecolor{currentstroke}%
\pgfsetdash{}{0pt}%
\pgfpathmoveto{\pgfqpoint{4.577524in}{7.639968in}}%
\pgfpathcurveto{\pgfqpoint{4.588574in}{7.639968in}}{\pgfqpoint{4.599173in}{7.644359in}}{\pgfqpoint{4.606986in}{7.652172in}}%
\pgfpathcurveto{\pgfqpoint{4.614800in}{7.659986in}}{\pgfqpoint{4.619190in}{7.670585in}}{\pgfqpoint{4.619190in}{7.681635in}}%
\pgfpathcurveto{\pgfqpoint{4.619190in}{7.692685in}}{\pgfqpoint{4.614800in}{7.703284in}}{\pgfqpoint{4.606986in}{7.711098in}}%
\pgfpathcurveto{\pgfqpoint{4.599173in}{7.718911in}}{\pgfqpoint{4.588574in}{7.723302in}}{\pgfqpoint{4.577524in}{7.723302in}}%
\pgfpathcurveto{\pgfqpoint{4.566474in}{7.723302in}}{\pgfqpoint{4.555875in}{7.718911in}}{\pgfqpoint{4.548061in}{7.711098in}}%
\pgfpathcurveto{\pgfqpoint{4.540247in}{7.703284in}}{\pgfqpoint{4.535857in}{7.692685in}}{\pgfqpoint{4.535857in}{7.681635in}}%
\pgfpathcurveto{\pgfqpoint{4.535857in}{7.670585in}}{\pgfqpoint{4.540247in}{7.659986in}}{\pgfqpoint{4.548061in}{7.652172in}}%
\pgfpathcurveto{\pgfqpoint{4.555875in}{7.644359in}}{\pgfqpoint{4.566474in}{7.639968in}}{\pgfqpoint{4.577524in}{7.639968in}}%
\pgfpathclose%
\pgfusepath{stroke,fill}%
\end{pgfscope}%
\begin{pgfscope}%
\pgfpathrectangle{\pgfqpoint{0.570343in}{0.331635in}}{\pgfqpoint{9.300000in}{7.700000in}}%
\pgfusepath{clip}%
\pgfsetbuttcap%
\pgfsetroundjoin%
\definecolor{currentfill}{rgb}{1.000000,0.705882,0.509804}%
\pgfsetfillcolor{currentfill}%
\pgfsetlinewidth{0.481800pt}%
\definecolor{currentstroke}{rgb}{1.000000,1.000000,1.000000}%
\pgfsetstrokecolor{currentstroke}%
\pgfsetdash{}{0pt}%
\pgfpathmoveto{\pgfqpoint{2.562027in}{6.241590in}}%
\pgfpathcurveto{\pgfqpoint{2.573077in}{6.241590in}}{\pgfqpoint{2.583676in}{6.245980in}}{\pgfqpoint{2.591490in}{6.253794in}}%
\pgfpathcurveto{\pgfqpoint{2.599303in}{6.261608in}}{\pgfqpoint{2.603694in}{6.272207in}}{\pgfqpoint{2.603694in}{6.283257in}}%
\pgfpathcurveto{\pgfqpoint{2.603694in}{6.294307in}}{\pgfqpoint{2.599303in}{6.304906in}}{\pgfqpoint{2.591490in}{6.312720in}}%
\pgfpathcurveto{\pgfqpoint{2.583676in}{6.320533in}}{\pgfqpoint{2.573077in}{6.324924in}}{\pgfqpoint{2.562027in}{6.324924in}}%
\pgfpathcurveto{\pgfqpoint{2.550977in}{6.324924in}}{\pgfqpoint{2.540378in}{6.320533in}}{\pgfqpoint{2.532564in}{6.312720in}}%
\pgfpathcurveto{\pgfqpoint{2.524751in}{6.304906in}}{\pgfqpoint{2.520360in}{6.294307in}}{\pgfqpoint{2.520360in}{6.283257in}}%
\pgfpathcurveto{\pgfqpoint{2.520360in}{6.272207in}}{\pgfqpoint{2.524751in}{6.261608in}}{\pgfqpoint{2.532564in}{6.253794in}}%
\pgfpathcurveto{\pgfqpoint{2.540378in}{6.245980in}}{\pgfqpoint{2.550977in}{6.241590in}}{\pgfqpoint{2.562027in}{6.241590in}}%
\pgfpathclose%
\pgfusepath{stroke,fill}%
\end{pgfscope}%
\begin{pgfscope}%
\pgfpathrectangle{\pgfqpoint{0.570343in}{0.331635in}}{\pgfqpoint{9.300000in}{7.700000in}}%
\pgfusepath{clip}%
\pgfsetbuttcap%
\pgfsetroundjoin%
\definecolor{currentfill}{rgb}{1.000000,0.705882,0.509804}%
\pgfsetfillcolor{currentfill}%
\pgfsetlinewidth{0.481800pt}%
\definecolor{currentstroke}{rgb}{1.000000,1.000000,1.000000}%
\pgfsetstrokecolor{currentstroke}%
\pgfsetdash{}{0pt}%
\pgfpathmoveto{\pgfqpoint{8.392100in}{4.860635in}}%
\pgfpathcurveto{\pgfqpoint{8.403150in}{4.860635in}}{\pgfqpoint{8.413749in}{4.865025in}}{\pgfqpoint{8.421563in}{4.872839in}}%
\pgfpathcurveto{\pgfqpoint{8.429376in}{4.880652in}}{\pgfqpoint{8.433767in}{4.891251in}}{\pgfqpoint{8.433767in}{4.902301in}}%
\pgfpathcurveto{\pgfqpoint{8.433767in}{4.913352in}}{\pgfqpoint{8.429376in}{4.923951in}}{\pgfqpoint{8.421563in}{4.931764in}}%
\pgfpathcurveto{\pgfqpoint{8.413749in}{4.939578in}}{\pgfqpoint{8.403150in}{4.943968in}}{\pgfqpoint{8.392100in}{4.943968in}}%
\pgfpathcurveto{\pgfqpoint{8.381050in}{4.943968in}}{\pgfqpoint{8.370451in}{4.939578in}}{\pgfqpoint{8.362637in}{4.931764in}}%
\pgfpathcurveto{\pgfqpoint{8.354824in}{4.923951in}}{\pgfqpoint{8.350433in}{4.913352in}}{\pgfqpoint{8.350433in}{4.902301in}}%
\pgfpathcurveto{\pgfqpoint{8.350433in}{4.891251in}}{\pgfqpoint{8.354824in}{4.880652in}}{\pgfqpoint{8.362637in}{4.872839in}}%
\pgfpathcurveto{\pgfqpoint{8.370451in}{4.865025in}}{\pgfqpoint{8.381050in}{4.860635in}}{\pgfqpoint{8.392100in}{4.860635in}}%
\pgfpathclose%
\pgfusepath{stroke,fill}%
\end{pgfscope}%
\begin{pgfscope}%
\pgfpathrectangle{\pgfqpoint{0.570343in}{0.331635in}}{\pgfqpoint{9.300000in}{7.700000in}}%
\pgfusepath{clip}%
\pgfsetbuttcap%
\pgfsetroundjoin%
\definecolor{currentfill}{rgb}{1.000000,0.705882,0.509804}%
\pgfsetfillcolor{currentfill}%
\pgfsetlinewidth{0.481800pt}%
\definecolor{currentstroke}{rgb}{1.000000,1.000000,1.000000}%
\pgfsetstrokecolor{currentstroke}%
\pgfsetdash{}{0pt}%
\pgfpathmoveto{\pgfqpoint{5.806842in}{1.666048in}}%
\pgfpathcurveto{\pgfqpoint{5.817893in}{1.666048in}}{\pgfqpoint{5.828492in}{1.670438in}}{\pgfqpoint{5.836305in}{1.678252in}}%
\pgfpathcurveto{\pgfqpoint{5.844119in}{1.686066in}}{\pgfqpoint{5.848509in}{1.696665in}}{\pgfqpoint{5.848509in}{1.707715in}}%
\pgfpathcurveto{\pgfqpoint{5.848509in}{1.718765in}}{\pgfqpoint{5.844119in}{1.729364in}}{\pgfqpoint{5.836305in}{1.737178in}}%
\pgfpathcurveto{\pgfqpoint{5.828492in}{1.744991in}}{\pgfqpoint{5.817893in}{1.749382in}}{\pgfqpoint{5.806842in}{1.749382in}}%
\pgfpathcurveto{\pgfqpoint{5.795792in}{1.749382in}}{\pgfqpoint{5.785193in}{1.744991in}}{\pgfqpoint{5.777380in}{1.737178in}}%
\pgfpathcurveto{\pgfqpoint{5.769566in}{1.729364in}}{\pgfqpoint{5.765176in}{1.718765in}}{\pgfqpoint{5.765176in}{1.707715in}}%
\pgfpathcurveto{\pgfqpoint{5.765176in}{1.696665in}}{\pgfqpoint{5.769566in}{1.686066in}}{\pgfqpoint{5.777380in}{1.678252in}}%
\pgfpathcurveto{\pgfqpoint{5.785193in}{1.670438in}}{\pgfqpoint{5.795792in}{1.666048in}}{\pgfqpoint{5.806842in}{1.666048in}}%
\pgfpathclose%
\pgfusepath{stroke,fill}%
\end{pgfscope}%
\begin{pgfscope}%
\pgfpathrectangle{\pgfqpoint{0.570343in}{0.331635in}}{\pgfqpoint{9.300000in}{7.700000in}}%
\pgfusepath{clip}%
\pgfsetbuttcap%
\pgfsetroundjoin%
\definecolor{currentfill}{rgb}{1.000000,0.705882,0.509804}%
\pgfsetfillcolor{currentfill}%
\pgfsetlinewidth{0.481800pt}%
\definecolor{currentstroke}{rgb}{1.000000,1.000000,1.000000}%
\pgfsetstrokecolor{currentstroke}%
\pgfsetdash{}{0pt}%
\pgfpathmoveto{\pgfqpoint{4.120169in}{3.694300in}}%
\pgfpathcurveto{\pgfqpoint{4.131219in}{3.694300in}}{\pgfqpoint{4.141818in}{3.698690in}}{\pgfqpoint{4.149632in}{3.706504in}}%
\pgfpathcurveto{\pgfqpoint{4.157446in}{3.714318in}}{\pgfqpoint{4.161836in}{3.724917in}}{\pgfqpoint{4.161836in}{3.735967in}}%
\pgfpathcurveto{\pgfqpoint{4.161836in}{3.747017in}}{\pgfqpoint{4.157446in}{3.757616in}}{\pgfqpoint{4.149632in}{3.765430in}}%
\pgfpathcurveto{\pgfqpoint{4.141818in}{3.773243in}}{\pgfqpoint{4.131219in}{3.777634in}}{\pgfqpoint{4.120169in}{3.777634in}}%
\pgfpathcurveto{\pgfqpoint{4.109119in}{3.777634in}}{\pgfqpoint{4.098520in}{3.773243in}}{\pgfqpoint{4.090707in}{3.765430in}}%
\pgfpathcurveto{\pgfqpoint{4.082893in}{3.757616in}}{\pgfqpoint{4.078503in}{3.747017in}}{\pgfqpoint{4.078503in}{3.735967in}}%
\pgfpathcurveto{\pgfqpoint{4.078503in}{3.724917in}}{\pgfqpoint{4.082893in}{3.714318in}}{\pgfqpoint{4.090707in}{3.706504in}}%
\pgfpathcurveto{\pgfqpoint{4.098520in}{3.698690in}}{\pgfqpoint{4.109119in}{3.694300in}}{\pgfqpoint{4.120169in}{3.694300in}}%
\pgfpathclose%
\pgfusepath{stroke,fill}%
\end{pgfscope}%
\begin{pgfscope}%
\pgfpathrectangle{\pgfqpoint{0.570343in}{0.331635in}}{\pgfqpoint{9.300000in}{7.700000in}}%
\pgfusepath{clip}%
\pgfsetbuttcap%
\pgfsetroundjoin%
\definecolor{currentfill}{rgb}{1.000000,0.705882,0.509804}%
\pgfsetfillcolor{currentfill}%
\pgfsetlinewidth{0.481800pt}%
\definecolor{currentstroke}{rgb}{1.000000,1.000000,1.000000}%
\pgfsetstrokecolor{currentstroke}%
\pgfsetdash{}{0pt}%
\pgfpathmoveto{\pgfqpoint{3.866323in}{1.796214in}}%
\pgfpathcurveto{\pgfqpoint{3.877374in}{1.796214in}}{\pgfqpoint{3.887973in}{1.800604in}}{\pgfqpoint{3.895786in}{1.808417in}}%
\pgfpathcurveto{\pgfqpoint{3.903600in}{1.816231in}}{\pgfqpoint{3.907990in}{1.826830in}}{\pgfqpoint{3.907990in}{1.837880in}}%
\pgfpathcurveto{\pgfqpoint{3.907990in}{1.848930in}}{\pgfqpoint{3.903600in}{1.859529in}}{\pgfqpoint{3.895786in}{1.867343in}}%
\pgfpathcurveto{\pgfqpoint{3.887973in}{1.875157in}}{\pgfqpoint{3.877374in}{1.879547in}}{\pgfqpoint{3.866323in}{1.879547in}}%
\pgfpathcurveto{\pgfqpoint{3.855273in}{1.879547in}}{\pgfqpoint{3.844674in}{1.875157in}}{\pgfqpoint{3.836861in}{1.867343in}}%
\pgfpathcurveto{\pgfqpoint{3.829047in}{1.859529in}}{\pgfqpoint{3.824657in}{1.848930in}}{\pgfqpoint{3.824657in}{1.837880in}}%
\pgfpathcurveto{\pgfqpoint{3.824657in}{1.826830in}}{\pgfqpoint{3.829047in}{1.816231in}}{\pgfqpoint{3.836861in}{1.808417in}}%
\pgfpathcurveto{\pgfqpoint{3.844674in}{1.800604in}}{\pgfqpoint{3.855273in}{1.796214in}}{\pgfqpoint{3.866323in}{1.796214in}}%
\pgfpathclose%
\pgfusepath{stroke,fill}%
\end{pgfscope}%
\begin{pgfscope}%
\pgfpathrectangle{\pgfqpoint{0.570343in}{0.331635in}}{\pgfqpoint{9.300000in}{7.700000in}}%
\pgfusepath{clip}%
\pgfsetbuttcap%
\pgfsetroundjoin%
\definecolor{currentfill}{rgb}{1.000000,0.705882,0.509804}%
\pgfsetfillcolor{currentfill}%
\pgfsetlinewidth{0.481800pt}%
\definecolor{currentstroke}{rgb}{1.000000,1.000000,1.000000}%
\pgfsetstrokecolor{currentstroke}%
\pgfsetdash{}{0pt}%
\pgfpathmoveto{\pgfqpoint{2.788649in}{3.977237in}}%
\pgfpathcurveto{\pgfqpoint{2.799699in}{3.977237in}}{\pgfqpoint{2.810298in}{3.981627in}}{\pgfqpoint{2.818112in}{3.989440in}}%
\pgfpathcurveto{\pgfqpoint{2.825926in}{3.997254in}}{\pgfqpoint{2.830316in}{4.007853in}}{\pgfqpoint{2.830316in}{4.018903in}}%
\pgfpathcurveto{\pgfqpoint{2.830316in}{4.029953in}}{\pgfqpoint{2.825926in}{4.040552in}}{\pgfqpoint{2.818112in}{4.048366in}}%
\pgfpathcurveto{\pgfqpoint{2.810298in}{4.056180in}}{\pgfqpoint{2.799699in}{4.060570in}}{\pgfqpoint{2.788649in}{4.060570in}}%
\pgfpathcurveto{\pgfqpoint{2.777599in}{4.060570in}}{\pgfqpoint{2.767000in}{4.056180in}}{\pgfqpoint{2.759187in}{4.048366in}}%
\pgfpathcurveto{\pgfqpoint{2.751373in}{4.040552in}}{\pgfqpoint{2.746983in}{4.029953in}}{\pgfqpoint{2.746983in}{4.018903in}}%
\pgfpathcurveto{\pgfqpoint{2.746983in}{4.007853in}}{\pgfqpoint{2.751373in}{3.997254in}}{\pgfqpoint{2.759187in}{3.989440in}}%
\pgfpathcurveto{\pgfqpoint{2.767000in}{3.981627in}}{\pgfqpoint{2.777599in}{3.977237in}}{\pgfqpoint{2.788649in}{3.977237in}}%
\pgfpathclose%
\pgfusepath{stroke,fill}%
\end{pgfscope}%
\begin{pgfscope}%
\pgfpathrectangle{\pgfqpoint{0.570343in}{0.331635in}}{\pgfqpoint{9.300000in}{7.700000in}}%
\pgfusepath{clip}%
\pgfsetbuttcap%
\pgfsetroundjoin%
\definecolor{currentfill}{rgb}{1.000000,0.705882,0.509804}%
\pgfsetfillcolor{currentfill}%
\pgfsetlinewidth{0.481800pt}%
\definecolor{currentstroke}{rgb}{1.000000,1.000000,1.000000}%
\pgfsetstrokecolor{currentstroke}%
\pgfsetdash{}{0pt}%
\pgfpathmoveto{\pgfqpoint{9.447616in}{3.622428in}}%
\pgfpathcurveto{\pgfqpoint{9.458666in}{3.622428in}}{\pgfqpoint{9.469265in}{3.626818in}}{\pgfqpoint{9.477079in}{3.634632in}}%
\pgfpathcurveto{\pgfqpoint{9.484892in}{3.642445in}}{\pgfqpoint{9.489283in}{3.653044in}}{\pgfqpoint{9.489283in}{3.664094in}}%
\pgfpathcurveto{\pgfqpoint{9.489283in}{3.675144in}}{\pgfqpoint{9.484892in}{3.685744in}}{\pgfqpoint{9.477079in}{3.693557in}}%
\pgfpathcurveto{\pgfqpoint{9.469265in}{3.701371in}}{\pgfqpoint{9.458666in}{3.705761in}}{\pgfqpoint{9.447616in}{3.705761in}}%
\pgfpathcurveto{\pgfqpoint{9.436566in}{3.705761in}}{\pgfqpoint{9.425967in}{3.701371in}}{\pgfqpoint{9.418153in}{3.693557in}}%
\pgfpathcurveto{\pgfqpoint{9.410340in}{3.685744in}}{\pgfqpoint{9.405949in}{3.675144in}}{\pgfqpoint{9.405949in}{3.664094in}}%
\pgfpathcurveto{\pgfqpoint{9.405949in}{3.653044in}}{\pgfqpoint{9.410340in}{3.642445in}}{\pgfqpoint{9.418153in}{3.634632in}}%
\pgfpathcurveto{\pgfqpoint{9.425967in}{3.626818in}}{\pgfqpoint{9.436566in}{3.622428in}}{\pgfqpoint{9.447616in}{3.622428in}}%
\pgfpathclose%
\pgfusepath{stroke,fill}%
\end{pgfscope}%
\begin{pgfscope}%
\pgfpathrectangle{\pgfqpoint{0.570343in}{0.331635in}}{\pgfqpoint{9.300000in}{7.700000in}}%
\pgfusepath{clip}%
\pgfsetbuttcap%
\pgfsetroundjoin%
\definecolor{currentfill}{rgb}{1.000000,0.705882,0.509804}%
\pgfsetfillcolor{currentfill}%
\pgfsetlinewidth{0.481800pt}%
\definecolor{currentstroke}{rgb}{1.000000,1.000000,1.000000}%
\pgfsetstrokecolor{currentstroke}%
\pgfsetdash{}{0pt}%
\pgfpathmoveto{\pgfqpoint{5.814101in}{0.639968in}}%
\pgfpathcurveto{\pgfqpoint{5.825151in}{0.639968in}}{\pgfqpoint{5.835750in}{0.644359in}}{\pgfqpoint{5.843564in}{0.652172in}}%
\pgfpathcurveto{\pgfqpoint{5.851377in}{0.659986in}}{\pgfqpoint{5.855767in}{0.670585in}}{\pgfqpoint{5.855767in}{0.681635in}}%
\pgfpathcurveto{\pgfqpoint{5.855767in}{0.692685in}}{\pgfqpoint{5.851377in}{0.703284in}}{\pgfqpoint{5.843564in}{0.711098in}}%
\pgfpathcurveto{\pgfqpoint{5.835750in}{0.718911in}}{\pgfqpoint{5.825151in}{0.723302in}}{\pgfqpoint{5.814101in}{0.723302in}}%
\pgfpathcurveto{\pgfqpoint{5.803051in}{0.723302in}}{\pgfqpoint{5.792452in}{0.718911in}}{\pgfqpoint{5.784638in}{0.711098in}}%
\pgfpathcurveto{\pgfqpoint{5.776824in}{0.703284in}}{\pgfqpoint{5.772434in}{0.692685in}}{\pgfqpoint{5.772434in}{0.681635in}}%
\pgfpathcurveto{\pgfqpoint{5.772434in}{0.670585in}}{\pgfqpoint{5.776824in}{0.659986in}}{\pgfqpoint{5.784638in}{0.652172in}}%
\pgfpathcurveto{\pgfqpoint{5.792452in}{0.644359in}}{\pgfqpoint{5.803051in}{0.639968in}}{\pgfqpoint{5.814101in}{0.639968in}}%
\pgfpathclose%
\pgfusepath{stroke,fill}%
\end{pgfscope}%
\begin{pgfscope}%
\pgfpathrectangle{\pgfqpoint{0.570343in}{0.331635in}}{\pgfqpoint{9.300000in}{7.700000in}}%
\pgfusepath{clip}%
\pgfsetbuttcap%
\pgfsetroundjoin%
\definecolor{currentfill}{rgb}{1.000000,0.705882,0.509804}%
\pgfsetfillcolor{currentfill}%
\pgfsetlinewidth{0.481800pt}%
\definecolor{currentstroke}{rgb}{1.000000,1.000000,1.000000}%
\pgfsetstrokecolor{currentstroke}%
\pgfsetdash{}{0pt}%
\pgfpathmoveto{\pgfqpoint{5.056091in}{2.233045in}}%
\pgfpathcurveto{\pgfqpoint{5.067141in}{2.233045in}}{\pgfqpoint{5.077740in}{2.237435in}}{\pgfqpoint{5.085554in}{2.245248in}}%
\pgfpathcurveto{\pgfqpoint{5.093368in}{2.253062in}}{\pgfqpoint{5.097758in}{2.263661in}}{\pgfqpoint{5.097758in}{2.274711in}}%
\pgfpathcurveto{\pgfqpoint{5.097758in}{2.285761in}}{\pgfqpoint{5.093368in}{2.296360in}}{\pgfqpoint{5.085554in}{2.304174in}}%
\pgfpathcurveto{\pgfqpoint{5.077740in}{2.311988in}}{\pgfqpoint{5.067141in}{2.316378in}}{\pgfqpoint{5.056091in}{2.316378in}}%
\pgfpathcurveto{\pgfqpoint{5.045041in}{2.316378in}}{\pgfqpoint{5.034442in}{2.311988in}}{\pgfqpoint{5.026628in}{2.304174in}}%
\pgfpathcurveto{\pgfqpoint{5.018815in}{2.296360in}}{\pgfqpoint{5.014425in}{2.285761in}}{\pgfqpoint{5.014425in}{2.274711in}}%
\pgfpathcurveto{\pgfqpoint{5.014425in}{2.263661in}}{\pgfqpoint{5.018815in}{2.253062in}}{\pgfqpoint{5.026628in}{2.245248in}}%
\pgfpathcurveto{\pgfqpoint{5.034442in}{2.237435in}}{\pgfqpoint{5.045041in}{2.233045in}}{\pgfqpoint{5.056091in}{2.233045in}}%
\pgfpathclose%
\pgfusepath{stroke,fill}%
\end{pgfscope}%
\begin{pgfscope}%
\pgfpathrectangle{\pgfqpoint{0.570343in}{0.331635in}}{\pgfqpoint{9.300000in}{7.700000in}}%
\pgfusepath{clip}%
\pgfsetbuttcap%
\pgfsetroundjoin%
\definecolor{currentfill}{rgb}{1.000000,0.705882,0.509804}%
\pgfsetfillcolor{currentfill}%
\pgfsetlinewidth{0.481800pt}%
\definecolor{currentstroke}{rgb}{1.000000,1.000000,1.000000}%
\pgfsetstrokecolor{currentstroke}%
\pgfsetdash{}{0pt}%
\pgfpathmoveto{\pgfqpoint{8.597790in}{3.777488in}}%
\pgfpathcurveto{\pgfqpoint{8.608840in}{3.777488in}}{\pgfqpoint{8.619439in}{3.781879in}}{\pgfqpoint{8.627252in}{3.789692in}}%
\pgfpathcurveto{\pgfqpoint{8.635066in}{3.797506in}}{\pgfqpoint{8.639456in}{3.808105in}}{\pgfqpoint{8.639456in}{3.819155in}}%
\pgfpathcurveto{\pgfqpoint{8.639456in}{3.830205in}}{\pgfqpoint{8.635066in}{3.840804in}}{\pgfqpoint{8.627252in}{3.848618in}}%
\pgfpathcurveto{\pgfqpoint{8.619439in}{3.856431in}}{\pgfqpoint{8.608840in}{3.860822in}}{\pgfqpoint{8.597790in}{3.860822in}}%
\pgfpathcurveto{\pgfqpoint{8.586740in}{3.860822in}}{\pgfqpoint{8.576141in}{3.856431in}}{\pgfqpoint{8.568327in}{3.848618in}}%
\pgfpathcurveto{\pgfqpoint{8.560513in}{3.840804in}}{\pgfqpoint{8.556123in}{3.830205in}}{\pgfqpoint{8.556123in}{3.819155in}}%
\pgfpathcurveto{\pgfqpoint{8.556123in}{3.808105in}}{\pgfqpoint{8.560513in}{3.797506in}}{\pgfqpoint{8.568327in}{3.789692in}}%
\pgfpathcurveto{\pgfqpoint{8.576141in}{3.781879in}}{\pgfqpoint{8.586740in}{3.777488in}}{\pgfqpoint{8.597790in}{3.777488in}}%
\pgfpathclose%
\pgfusepath{stroke,fill}%
\end{pgfscope}%
\begin{pgfscope}%
\pgfpathrectangle{\pgfqpoint{0.570343in}{0.331635in}}{\pgfqpoint{9.300000in}{7.700000in}}%
\pgfusepath{clip}%
\pgfsetbuttcap%
\pgfsetroundjoin%
\definecolor{currentfill}{rgb}{1.000000,0.705882,0.509804}%
\pgfsetfillcolor{currentfill}%
\pgfsetlinewidth{0.481800pt}%
\definecolor{currentstroke}{rgb}{1.000000,1.000000,1.000000}%
\pgfsetstrokecolor{currentstroke}%
\pgfsetdash{}{0pt}%
\pgfpathmoveto{\pgfqpoint{4.061550in}{0.813270in}}%
\pgfpathcurveto{\pgfqpoint{4.072601in}{0.813270in}}{\pgfqpoint{4.083200in}{0.817660in}}{\pgfqpoint{4.091013in}{0.825474in}}%
\pgfpathcurveto{\pgfqpoint{4.098827in}{0.833287in}}{\pgfqpoint{4.103217in}{0.843886in}}{\pgfqpoint{4.103217in}{0.854936in}}%
\pgfpathcurveto{\pgfqpoint{4.103217in}{0.865987in}}{\pgfqpoint{4.098827in}{0.876586in}}{\pgfqpoint{4.091013in}{0.884399in}}%
\pgfpathcurveto{\pgfqpoint{4.083200in}{0.892213in}}{\pgfqpoint{4.072601in}{0.896603in}}{\pgfqpoint{4.061550in}{0.896603in}}%
\pgfpathcurveto{\pgfqpoint{4.050500in}{0.896603in}}{\pgfqpoint{4.039901in}{0.892213in}}{\pgfqpoint{4.032088in}{0.884399in}}%
\pgfpathcurveto{\pgfqpoint{4.024274in}{0.876586in}}{\pgfqpoint{4.019884in}{0.865987in}}{\pgfqpoint{4.019884in}{0.854936in}}%
\pgfpathcurveto{\pgfqpoint{4.019884in}{0.843886in}}{\pgfqpoint{4.024274in}{0.833287in}}{\pgfqpoint{4.032088in}{0.825474in}}%
\pgfpathcurveto{\pgfqpoint{4.039901in}{0.817660in}}{\pgfqpoint{4.050500in}{0.813270in}}{\pgfqpoint{4.061550in}{0.813270in}}%
\pgfpathclose%
\pgfusepath{stroke,fill}%
\end{pgfscope}%
\begin{pgfscope}%
\pgfpathrectangle{\pgfqpoint{0.570343in}{0.331635in}}{\pgfqpoint{9.300000in}{7.700000in}}%
\pgfusepath{clip}%
\pgfsetbuttcap%
\pgfsetroundjoin%
\definecolor{currentfill}{rgb}{1.000000,0.705882,0.509804}%
\pgfsetfillcolor{currentfill}%
\pgfsetlinewidth{0.481800pt}%
\definecolor{currentstroke}{rgb}{1.000000,1.000000,1.000000}%
\pgfsetstrokecolor{currentstroke}%
\pgfsetdash{}{0pt}%
\pgfpathmoveto{\pgfqpoint{7.863027in}{5.764293in}}%
\pgfpathcurveto{\pgfqpoint{7.874077in}{5.764293in}}{\pgfqpoint{7.884676in}{5.768683in}}{\pgfqpoint{7.892490in}{5.776497in}}%
\pgfpathcurveto{\pgfqpoint{7.900303in}{5.784310in}}{\pgfqpoint{7.904694in}{5.794909in}}{\pgfqpoint{7.904694in}{5.805960in}}%
\pgfpathcurveto{\pgfqpoint{7.904694in}{5.817010in}}{\pgfqpoint{7.900303in}{5.827609in}}{\pgfqpoint{7.892490in}{5.835422in}}%
\pgfpathcurveto{\pgfqpoint{7.884676in}{5.843236in}}{\pgfqpoint{7.874077in}{5.847626in}}{\pgfqpoint{7.863027in}{5.847626in}}%
\pgfpathcurveto{\pgfqpoint{7.851977in}{5.847626in}}{\pgfqpoint{7.841378in}{5.843236in}}{\pgfqpoint{7.833564in}{5.835422in}}%
\pgfpathcurveto{\pgfqpoint{7.825750in}{5.827609in}}{\pgfqpoint{7.821360in}{5.817010in}}{\pgfqpoint{7.821360in}{5.805960in}}%
\pgfpathcurveto{\pgfqpoint{7.821360in}{5.794909in}}{\pgfqpoint{7.825750in}{5.784310in}}{\pgfqpoint{7.833564in}{5.776497in}}%
\pgfpathcurveto{\pgfqpoint{7.841378in}{5.768683in}}{\pgfqpoint{7.851977in}{5.764293in}}{\pgfqpoint{7.863027in}{5.764293in}}%
\pgfpathclose%
\pgfusepath{stroke,fill}%
\end{pgfscope}%
\begin{pgfscope}%
\pgfpathrectangle{\pgfqpoint{0.570343in}{0.331635in}}{\pgfqpoint{9.300000in}{7.700000in}}%
\pgfusepath{clip}%
\pgfsetbuttcap%
\pgfsetroundjoin%
\definecolor{currentfill}{rgb}{1.000000,0.705882,0.509804}%
\pgfsetfillcolor{currentfill}%
\pgfsetlinewidth{0.481800pt}%
\definecolor{currentstroke}{rgb}{1.000000,1.000000,1.000000}%
\pgfsetstrokecolor{currentstroke}%
\pgfsetdash{}{0pt}%
\pgfpathmoveto{\pgfqpoint{6.615719in}{2.465016in}}%
\pgfpathcurveto{\pgfqpoint{6.626770in}{2.465016in}}{\pgfqpoint{6.637369in}{2.469406in}}{\pgfqpoint{6.645182in}{2.477220in}}%
\pgfpathcurveto{\pgfqpoint{6.652996in}{2.485034in}}{\pgfqpoint{6.657386in}{2.495633in}}{\pgfqpoint{6.657386in}{2.506683in}}%
\pgfpathcurveto{\pgfqpoint{6.657386in}{2.517733in}}{\pgfqpoint{6.652996in}{2.528332in}}{\pgfqpoint{6.645182in}{2.536146in}}%
\pgfpathcurveto{\pgfqpoint{6.637369in}{2.543959in}}{\pgfqpoint{6.626770in}{2.548350in}}{\pgfqpoint{6.615719in}{2.548350in}}%
\pgfpathcurveto{\pgfqpoint{6.604669in}{2.548350in}}{\pgfqpoint{6.594070in}{2.543959in}}{\pgfqpoint{6.586257in}{2.536146in}}%
\pgfpathcurveto{\pgfqpoint{6.578443in}{2.528332in}}{\pgfqpoint{6.574053in}{2.517733in}}{\pgfqpoint{6.574053in}{2.506683in}}%
\pgfpathcurveto{\pgfqpoint{6.574053in}{2.495633in}}{\pgfqpoint{6.578443in}{2.485034in}}{\pgfqpoint{6.586257in}{2.477220in}}%
\pgfpathcurveto{\pgfqpoint{6.594070in}{2.469406in}}{\pgfqpoint{6.604669in}{2.465016in}}{\pgfqpoint{6.615719in}{2.465016in}}%
\pgfpathclose%
\pgfusepath{stroke,fill}%
\end{pgfscope}%
\begin{pgfscope}%
\pgfpathrectangle{\pgfqpoint{0.570343in}{0.331635in}}{\pgfqpoint{9.300000in}{7.700000in}}%
\pgfusepath{clip}%
\pgfsetbuttcap%
\pgfsetroundjoin%
\definecolor{currentfill}{rgb}{1.000000,0.705882,0.509804}%
\pgfsetfillcolor{currentfill}%
\pgfsetlinewidth{0.481800pt}%
\definecolor{currentstroke}{rgb}{1.000000,1.000000,1.000000}%
\pgfsetstrokecolor{currentstroke}%
\pgfsetdash{}{0pt}%
\pgfpathmoveto{\pgfqpoint{3.036089in}{4.902839in}}%
\pgfpathcurveto{\pgfqpoint{3.047139in}{4.902839in}}{\pgfqpoint{3.057738in}{4.907229in}}{\pgfqpoint{3.065551in}{4.915043in}}%
\pgfpathcurveto{\pgfqpoint{3.073365in}{4.922856in}}{\pgfqpoint{3.077755in}{4.933455in}}{\pgfqpoint{3.077755in}{4.944505in}}%
\pgfpathcurveto{\pgfqpoint{3.077755in}{4.955555in}}{\pgfqpoint{3.073365in}{4.966155in}}{\pgfqpoint{3.065551in}{4.973968in}}%
\pgfpathcurveto{\pgfqpoint{3.057738in}{4.981782in}}{\pgfqpoint{3.047139in}{4.986172in}}{\pgfqpoint{3.036089in}{4.986172in}}%
\pgfpathcurveto{\pgfqpoint{3.025038in}{4.986172in}}{\pgfqpoint{3.014439in}{4.981782in}}{\pgfqpoint{3.006626in}{4.973968in}}%
\pgfpathcurveto{\pgfqpoint{2.998812in}{4.966155in}}{\pgfqpoint{2.994422in}{4.955555in}}{\pgfqpoint{2.994422in}{4.944505in}}%
\pgfpathcurveto{\pgfqpoint{2.994422in}{4.933455in}}{\pgfqpoint{2.998812in}{4.922856in}}{\pgfqpoint{3.006626in}{4.915043in}}%
\pgfpathcurveto{\pgfqpoint{3.014439in}{4.907229in}}{\pgfqpoint{3.025038in}{4.902839in}}{\pgfqpoint{3.036089in}{4.902839in}}%
\pgfpathclose%
\pgfusepath{stroke,fill}%
\end{pgfscope}%
\begin{pgfscope}%
\pgfpathrectangle{\pgfqpoint{0.570343in}{0.331635in}}{\pgfqpoint{9.300000in}{7.700000in}}%
\pgfusepath{clip}%
\pgfsetbuttcap%
\pgfsetroundjoin%
\definecolor{currentfill}{rgb}{1.000000,0.705882,0.509804}%
\pgfsetfillcolor{currentfill}%
\pgfsetlinewidth{0.481800pt}%
\definecolor{currentstroke}{rgb}{1.000000,1.000000,1.000000}%
\pgfsetstrokecolor{currentstroke}%
\pgfsetdash{}{0pt}%
\pgfpathmoveto{\pgfqpoint{9.333987in}{4.576789in}}%
\pgfpathcurveto{\pgfqpoint{9.345037in}{4.576789in}}{\pgfqpoint{9.355636in}{4.581180in}}{\pgfqpoint{9.363450in}{4.588993in}}%
\pgfpathcurveto{\pgfqpoint{9.371264in}{4.596807in}}{\pgfqpoint{9.375654in}{4.607406in}}{\pgfqpoint{9.375654in}{4.618456in}}%
\pgfpathcurveto{\pgfqpoint{9.375654in}{4.629506in}}{\pgfqpoint{9.371264in}{4.640105in}}{\pgfqpoint{9.363450in}{4.647919in}}%
\pgfpathcurveto{\pgfqpoint{9.355636in}{4.655733in}}{\pgfqpoint{9.345037in}{4.660123in}}{\pgfqpoint{9.333987in}{4.660123in}}%
\pgfpathcurveto{\pgfqpoint{9.322937in}{4.660123in}}{\pgfqpoint{9.312338in}{4.655733in}}{\pgfqpoint{9.304524in}{4.647919in}}%
\pgfpathcurveto{\pgfqpoint{9.296711in}{4.640105in}}{\pgfqpoint{9.292321in}{4.629506in}}{\pgfqpoint{9.292321in}{4.618456in}}%
\pgfpathcurveto{\pgfqpoint{9.292321in}{4.607406in}}{\pgfqpoint{9.296711in}{4.596807in}}{\pgfqpoint{9.304524in}{4.588993in}}%
\pgfpathcurveto{\pgfqpoint{9.312338in}{4.581180in}}{\pgfqpoint{9.322937in}{4.576789in}}{\pgfqpoint{9.333987in}{4.576789in}}%
\pgfpathclose%
\pgfusepath{stroke,fill}%
\end{pgfscope}%
\begin{pgfscope}%
\pgfpathrectangle{\pgfqpoint{0.570343in}{0.331635in}}{\pgfqpoint{9.300000in}{7.700000in}}%
\pgfusepath{clip}%
\pgfsetbuttcap%
\pgfsetroundjoin%
\definecolor{currentfill}{rgb}{1.000000,0.705882,0.509804}%
\pgfsetfillcolor{currentfill}%
\pgfsetlinewidth{0.481800pt}%
\definecolor{currentstroke}{rgb}{1.000000,1.000000,1.000000}%
\pgfsetstrokecolor{currentstroke}%
\pgfsetdash{}{0pt}%
\pgfpathmoveto{\pgfqpoint{5.652131in}{2.830374in}}%
\pgfpathcurveto{\pgfqpoint{5.663181in}{2.830374in}}{\pgfqpoint{5.673780in}{2.834765in}}{\pgfqpoint{5.681594in}{2.842578in}}%
\pgfpathcurveto{\pgfqpoint{5.689407in}{2.850392in}}{\pgfqpoint{5.693798in}{2.860991in}}{\pgfqpoint{5.693798in}{2.872041in}}%
\pgfpathcurveto{\pgfqpoint{5.693798in}{2.883091in}}{\pgfqpoint{5.689407in}{2.893690in}}{\pgfqpoint{5.681594in}{2.901504in}}%
\pgfpathcurveto{\pgfqpoint{5.673780in}{2.909318in}}{\pgfqpoint{5.663181in}{2.913708in}}{\pgfqpoint{5.652131in}{2.913708in}}%
\pgfpathcurveto{\pgfqpoint{5.641081in}{2.913708in}}{\pgfqpoint{5.630482in}{2.909318in}}{\pgfqpoint{5.622668in}{2.901504in}}%
\pgfpathcurveto{\pgfqpoint{5.614855in}{2.893690in}}{\pgfqpoint{5.610464in}{2.883091in}}{\pgfqpoint{5.610464in}{2.872041in}}%
\pgfpathcurveto{\pgfqpoint{5.610464in}{2.860991in}}{\pgfqpoint{5.614855in}{2.850392in}}{\pgfqpoint{5.622668in}{2.842578in}}%
\pgfpathcurveto{\pgfqpoint{5.630482in}{2.834765in}}{\pgfqpoint{5.641081in}{2.830374in}}{\pgfqpoint{5.652131in}{2.830374in}}%
\pgfpathclose%
\pgfusepath{stroke,fill}%
\end{pgfscope}%
\begin{pgfscope}%
\pgfpathrectangle{\pgfqpoint{0.570343in}{0.331635in}}{\pgfqpoint{9.300000in}{7.700000in}}%
\pgfusepath{clip}%
\pgfsetbuttcap%
\pgfsetroundjoin%
\definecolor{currentfill}{rgb}{0.631373,0.788235,0.956863}%
\pgfsetfillcolor{currentfill}%
\pgfsetlinewidth{1.003750pt}%
\definecolor{currentstroke}{rgb}{0.631373,0.788235,0.956863}%
\pgfsetstrokecolor{currentstroke}%
\pgfsetdash{}{0pt}%
\pgfsys@defobject{currentmarker}{\pgfqpoint{-0.041667in}{-0.041667in}}{\pgfqpoint{0.041667in}{0.041667in}}{%
\pgfpathmoveto{\pgfqpoint{0.000000in}{-0.041667in}}%
\pgfpathcurveto{\pgfqpoint{0.011050in}{-0.041667in}}{\pgfqpoint{0.021649in}{-0.037276in}}{\pgfqpoint{0.029463in}{-0.029463in}}%
\pgfpathcurveto{\pgfqpoint{0.037276in}{-0.021649in}}{\pgfqpoint{0.041667in}{-0.011050in}}{\pgfqpoint{0.041667in}{0.000000in}}%
\pgfpathcurveto{\pgfqpoint{0.041667in}{0.011050in}}{\pgfqpoint{0.037276in}{0.021649in}}{\pgfqpoint{0.029463in}{0.029463in}}%
\pgfpathcurveto{\pgfqpoint{0.021649in}{0.037276in}}{\pgfqpoint{0.011050in}{0.041667in}}{\pgfqpoint{0.000000in}{0.041667in}}%
\pgfpathcurveto{\pgfqpoint{-0.011050in}{0.041667in}}{\pgfqpoint{-0.021649in}{0.037276in}}{\pgfqpoint{-0.029463in}{0.029463in}}%
\pgfpathcurveto{\pgfqpoint{-0.037276in}{0.021649in}}{\pgfqpoint{-0.041667in}{0.011050in}}{\pgfqpoint{-0.041667in}{0.000000in}}%
\pgfpathcurveto{\pgfqpoint{-0.041667in}{-0.011050in}}{\pgfqpoint{-0.037276in}{-0.021649in}}{\pgfqpoint{-0.029463in}{-0.029463in}}%
\pgfpathcurveto{\pgfqpoint{-0.021649in}{-0.037276in}}{\pgfqpoint{-0.011050in}{-0.041667in}}{\pgfqpoint{0.000000in}{-0.041667in}}%
\pgfpathclose%
\pgfusepath{stroke,fill}%
}%
\end{pgfscope}%
\begin{pgfscope}%
\pgfpathrectangle{\pgfqpoint{0.570343in}{0.331635in}}{\pgfqpoint{9.300000in}{7.700000in}}%
\pgfusepath{clip}%
\pgfsetbuttcap%
\pgfsetroundjoin%
\definecolor{currentfill}{rgb}{1.000000,0.705882,0.509804}%
\pgfsetfillcolor{currentfill}%
\pgfsetlinewidth{1.003750pt}%
\definecolor{currentstroke}{rgb}{1.000000,0.705882,0.509804}%
\pgfsetstrokecolor{currentstroke}%
\pgfsetdash{}{0pt}%
\pgfsys@defobject{currentmarker}{\pgfqpoint{-0.041667in}{-0.041667in}}{\pgfqpoint{0.041667in}{0.041667in}}{%
\pgfpathmoveto{\pgfqpoint{0.000000in}{-0.041667in}}%
\pgfpathcurveto{\pgfqpoint{0.011050in}{-0.041667in}}{\pgfqpoint{0.021649in}{-0.037276in}}{\pgfqpoint{0.029463in}{-0.029463in}}%
\pgfpathcurveto{\pgfqpoint{0.037276in}{-0.021649in}}{\pgfqpoint{0.041667in}{-0.011050in}}{\pgfqpoint{0.041667in}{0.000000in}}%
\pgfpathcurveto{\pgfqpoint{0.041667in}{0.011050in}}{\pgfqpoint{0.037276in}{0.021649in}}{\pgfqpoint{0.029463in}{0.029463in}}%
\pgfpathcurveto{\pgfqpoint{0.021649in}{0.037276in}}{\pgfqpoint{0.011050in}{0.041667in}}{\pgfqpoint{0.000000in}{0.041667in}}%
\pgfpathcurveto{\pgfqpoint{-0.011050in}{0.041667in}}{\pgfqpoint{-0.021649in}{0.037276in}}{\pgfqpoint{-0.029463in}{0.029463in}}%
\pgfpathcurveto{\pgfqpoint{-0.037276in}{0.021649in}}{\pgfqpoint{-0.041667in}{0.011050in}}{\pgfqpoint{-0.041667in}{0.000000in}}%
\pgfpathcurveto{\pgfqpoint{-0.041667in}{-0.011050in}}{\pgfqpoint{-0.037276in}{-0.021649in}}{\pgfqpoint{-0.029463in}{-0.029463in}}%
\pgfpathcurveto{\pgfqpoint{-0.021649in}{-0.037276in}}{\pgfqpoint{-0.011050in}{-0.041667in}}{\pgfqpoint{0.000000in}{-0.041667in}}%
\pgfpathclose%
\pgfusepath{stroke,fill}%
}%
\end{pgfscope}%
\begin{pgfscope}%
\pgfsetbuttcap%
\pgfsetroundjoin%
\definecolor{currentfill}{rgb}{0.000000,0.000000,0.000000}%
\pgfsetfillcolor{currentfill}%
\pgfsetlinewidth{0.803000pt}%
\definecolor{currentstroke}{rgb}{0.000000,0.000000,0.000000}%
\pgfsetstrokecolor{currentstroke}%
\pgfsetdash{}{0pt}%
\pgfsys@defobject{currentmarker}{\pgfqpoint{0.000000in}{-0.048611in}}{\pgfqpoint{0.000000in}{0.000000in}}{%
\pgfpathmoveto{\pgfqpoint{0.000000in}{0.000000in}}%
\pgfpathlineto{\pgfqpoint{0.000000in}{-0.048611in}}%
\pgfusepath{stroke,fill}%
}%
\begin{pgfscope}%
\pgfsys@transformshift{0.607701in}{0.331635in}%
\pgfsys@useobject{currentmarker}{}%
\end{pgfscope}%
\end{pgfscope}%
\begin{pgfscope}%
\definecolor{textcolor}{rgb}{0.000000,0.000000,0.000000}%
\pgfsetstrokecolor{textcolor}%
\pgfsetfillcolor{textcolor}%
\pgftext[x=0.607701in,y=0.234413in,,top]{\color{textcolor}\sffamily\fontsize{10.000000}{12.000000}\selectfont \ensuremath{-}100}%
\end{pgfscope}%
\begin{pgfscope}%
\pgfsetbuttcap%
\pgfsetroundjoin%
\definecolor{currentfill}{rgb}{0.000000,0.000000,0.000000}%
\pgfsetfillcolor{currentfill}%
\pgfsetlinewidth{0.803000pt}%
\definecolor{currentstroke}{rgb}{0.000000,0.000000,0.000000}%
\pgfsetstrokecolor{currentstroke}%
\pgfsetdash{}{0pt}%
\pgfsys@defobject{currentmarker}{\pgfqpoint{0.000000in}{-0.048611in}}{\pgfqpoint{0.000000in}{0.000000in}}{%
\pgfpathmoveto{\pgfqpoint{0.000000in}{0.000000in}}%
\pgfpathlineto{\pgfqpoint{0.000000in}{-0.048611in}}%
\pgfusepath{stroke,fill}%
}%
\begin{pgfscope}%
\pgfsys@transformshift{1.748621in}{0.331635in}%
\pgfsys@useobject{currentmarker}{}%
\end{pgfscope}%
\end{pgfscope}%
\begin{pgfscope}%
\definecolor{textcolor}{rgb}{0.000000,0.000000,0.000000}%
\pgfsetstrokecolor{textcolor}%
\pgfsetfillcolor{textcolor}%
\pgftext[x=1.748621in,y=0.234413in,,top]{\color{textcolor}\sffamily\fontsize{10.000000}{12.000000}\selectfont \ensuremath{-}75}%
\end{pgfscope}%
\begin{pgfscope}%
\pgfsetbuttcap%
\pgfsetroundjoin%
\definecolor{currentfill}{rgb}{0.000000,0.000000,0.000000}%
\pgfsetfillcolor{currentfill}%
\pgfsetlinewidth{0.803000pt}%
\definecolor{currentstroke}{rgb}{0.000000,0.000000,0.000000}%
\pgfsetstrokecolor{currentstroke}%
\pgfsetdash{}{0pt}%
\pgfsys@defobject{currentmarker}{\pgfqpoint{0.000000in}{-0.048611in}}{\pgfqpoint{0.000000in}{0.000000in}}{%
\pgfpathmoveto{\pgfqpoint{0.000000in}{0.000000in}}%
\pgfpathlineto{\pgfqpoint{0.000000in}{-0.048611in}}%
\pgfusepath{stroke,fill}%
}%
\begin{pgfscope}%
\pgfsys@transformshift{2.889540in}{0.331635in}%
\pgfsys@useobject{currentmarker}{}%
\end{pgfscope}%
\end{pgfscope}%
\begin{pgfscope}%
\definecolor{textcolor}{rgb}{0.000000,0.000000,0.000000}%
\pgfsetstrokecolor{textcolor}%
\pgfsetfillcolor{textcolor}%
\pgftext[x=2.889540in,y=0.234413in,,top]{\color{textcolor}\sffamily\fontsize{10.000000}{12.000000}\selectfont \ensuremath{-}50}%
\end{pgfscope}%
\begin{pgfscope}%
\pgfsetbuttcap%
\pgfsetroundjoin%
\definecolor{currentfill}{rgb}{0.000000,0.000000,0.000000}%
\pgfsetfillcolor{currentfill}%
\pgfsetlinewidth{0.803000pt}%
\definecolor{currentstroke}{rgb}{0.000000,0.000000,0.000000}%
\pgfsetstrokecolor{currentstroke}%
\pgfsetdash{}{0pt}%
\pgfsys@defobject{currentmarker}{\pgfqpoint{0.000000in}{-0.048611in}}{\pgfqpoint{0.000000in}{0.000000in}}{%
\pgfpathmoveto{\pgfqpoint{0.000000in}{0.000000in}}%
\pgfpathlineto{\pgfqpoint{0.000000in}{-0.048611in}}%
\pgfusepath{stroke,fill}%
}%
\begin{pgfscope}%
\pgfsys@transformshift{4.030460in}{0.331635in}%
\pgfsys@useobject{currentmarker}{}%
\end{pgfscope}%
\end{pgfscope}%
\begin{pgfscope}%
\definecolor{textcolor}{rgb}{0.000000,0.000000,0.000000}%
\pgfsetstrokecolor{textcolor}%
\pgfsetfillcolor{textcolor}%
\pgftext[x=4.030460in,y=0.234413in,,top]{\color{textcolor}\sffamily\fontsize{10.000000}{12.000000}\selectfont \ensuremath{-}25}%
\end{pgfscope}%
\begin{pgfscope}%
\pgfsetbuttcap%
\pgfsetroundjoin%
\definecolor{currentfill}{rgb}{0.000000,0.000000,0.000000}%
\pgfsetfillcolor{currentfill}%
\pgfsetlinewidth{0.803000pt}%
\definecolor{currentstroke}{rgb}{0.000000,0.000000,0.000000}%
\pgfsetstrokecolor{currentstroke}%
\pgfsetdash{}{0pt}%
\pgfsys@defobject{currentmarker}{\pgfqpoint{0.000000in}{-0.048611in}}{\pgfqpoint{0.000000in}{0.000000in}}{%
\pgfpathmoveto{\pgfqpoint{0.000000in}{0.000000in}}%
\pgfpathlineto{\pgfqpoint{0.000000in}{-0.048611in}}%
\pgfusepath{stroke,fill}%
}%
\begin{pgfscope}%
\pgfsys@transformshift{5.171380in}{0.331635in}%
\pgfsys@useobject{currentmarker}{}%
\end{pgfscope}%
\end{pgfscope}%
\begin{pgfscope}%
\definecolor{textcolor}{rgb}{0.000000,0.000000,0.000000}%
\pgfsetstrokecolor{textcolor}%
\pgfsetfillcolor{textcolor}%
\pgftext[x=5.171380in,y=0.234413in,,top]{\color{textcolor}\sffamily\fontsize{10.000000}{12.000000}\selectfont 0}%
\end{pgfscope}%
\begin{pgfscope}%
\pgfsetbuttcap%
\pgfsetroundjoin%
\definecolor{currentfill}{rgb}{0.000000,0.000000,0.000000}%
\pgfsetfillcolor{currentfill}%
\pgfsetlinewidth{0.803000pt}%
\definecolor{currentstroke}{rgb}{0.000000,0.000000,0.000000}%
\pgfsetstrokecolor{currentstroke}%
\pgfsetdash{}{0pt}%
\pgfsys@defobject{currentmarker}{\pgfqpoint{0.000000in}{-0.048611in}}{\pgfqpoint{0.000000in}{0.000000in}}{%
\pgfpathmoveto{\pgfqpoint{0.000000in}{0.000000in}}%
\pgfpathlineto{\pgfqpoint{0.000000in}{-0.048611in}}%
\pgfusepath{stroke,fill}%
}%
\begin{pgfscope}%
\pgfsys@transformshift{6.312299in}{0.331635in}%
\pgfsys@useobject{currentmarker}{}%
\end{pgfscope}%
\end{pgfscope}%
\begin{pgfscope}%
\definecolor{textcolor}{rgb}{0.000000,0.000000,0.000000}%
\pgfsetstrokecolor{textcolor}%
\pgfsetfillcolor{textcolor}%
\pgftext[x=6.312299in,y=0.234413in,,top]{\color{textcolor}\sffamily\fontsize{10.000000}{12.000000}\selectfont 25}%
\end{pgfscope}%
\begin{pgfscope}%
\pgfsetbuttcap%
\pgfsetroundjoin%
\definecolor{currentfill}{rgb}{0.000000,0.000000,0.000000}%
\pgfsetfillcolor{currentfill}%
\pgfsetlinewidth{0.803000pt}%
\definecolor{currentstroke}{rgb}{0.000000,0.000000,0.000000}%
\pgfsetstrokecolor{currentstroke}%
\pgfsetdash{}{0pt}%
\pgfsys@defobject{currentmarker}{\pgfqpoint{0.000000in}{-0.048611in}}{\pgfqpoint{0.000000in}{0.000000in}}{%
\pgfpathmoveto{\pgfqpoint{0.000000in}{0.000000in}}%
\pgfpathlineto{\pgfqpoint{0.000000in}{-0.048611in}}%
\pgfusepath{stroke,fill}%
}%
\begin{pgfscope}%
\pgfsys@transformshift{7.453219in}{0.331635in}%
\pgfsys@useobject{currentmarker}{}%
\end{pgfscope}%
\end{pgfscope}%
\begin{pgfscope}%
\definecolor{textcolor}{rgb}{0.000000,0.000000,0.000000}%
\pgfsetstrokecolor{textcolor}%
\pgfsetfillcolor{textcolor}%
\pgftext[x=7.453219in,y=0.234413in,,top]{\color{textcolor}\sffamily\fontsize{10.000000}{12.000000}\selectfont 50}%
\end{pgfscope}%
\begin{pgfscope}%
\pgfsetbuttcap%
\pgfsetroundjoin%
\definecolor{currentfill}{rgb}{0.000000,0.000000,0.000000}%
\pgfsetfillcolor{currentfill}%
\pgfsetlinewidth{0.803000pt}%
\definecolor{currentstroke}{rgb}{0.000000,0.000000,0.000000}%
\pgfsetstrokecolor{currentstroke}%
\pgfsetdash{}{0pt}%
\pgfsys@defobject{currentmarker}{\pgfqpoint{0.000000in}{-0.048611in}}{\pgfqpoint{0.000000in}{0.000000in}}{%
\pgfpathmoveto{\pgfqpoint{0.000000in}{0.000000in}}%
\pgfpathlineto{\pgfqpoint{0.000000in}{-0.048611in}}%
\pgfusepath{stroke,fill}%
}%
\begin{pgfscope}%
\pgfsys@transformshift{8.594139in}{0.331635in}%
\pgfsys@useobject{currentmarker}{}%
\end{pgfscope}%
\end{pgfscope}%
\begin{pgfscope}%
\definecolor{textcolor}{rgb}{0.000000,0.000000,0.000000}%
\pgfsetstrokecolor{textcolor}%
\pgfsetfillcolor{textcolor}%
\pgftext[x=8.594139in,y=0.234413in,,top]{\color{textcolor}\sffamily\fontsize{10.000000}{12.000000}\selectfont 75}%
\end{pgfscope}%
\begin{pgfscope}%
\pgfsetbuttcap%
\pgfsetroundjoin%
\definecolor{currentfill}{rgb}{0.000000,0.000000,0.000000}%
\pgfsetfillcolor{currentfill}%
\pgfsetlinewidth{0.803000pt}%
\definecolor{currentstroke}{rgb}{0.000000,0.000000,0.000000}%
\pgfsetstrokecolor{currentstroke}%
\pgfsetdash{}{0pt}%
\pgfsys@defobject{currentmarker}{\pgfqpoint{0.000000in}{-0.048611in}}{\pgfqpoint{0.000000in}{0.000000in}}{%
\pgfpathmoveto{\pgfqpoint{0.000000in}{0.000000in}}%
\pgfpathlineto{\pgfqpoint{0.000000in}{-0.048611in}}%
\pgfusepath{stroke,fill}%
}%
\begin{pgfscope}%
\pgfsys@transformshift{9.735058in}{0.331635in}%
\pgfsys@useobject{currentmarker}{}%
\end{pgfscope}%
\end{pgfscope}%
\begin{pgfscope}%
\definecolor{textcolor}{rgb}{0.000000,0.000000,0.000000}%
\pgfsetstrokecolor{textcolor}%
\pgfsetfillcolor{textcolor}%
\pgftext[x=9.735058in,y=0.234413in,,top]{\color{textcolor}\sffamily\fontsize{10.000000}{12.000000}\selectfont 100}%
\end{pgfscope}%
\begin{pgfscope}%
\pgfsetbuttcap%
\pgfsetroundjoin%
\definecolor{currentfill}{rgb}{0.000000,0.000000,0.000000}%
\pgfsetfillcolor{currentfill}%
\pgfsetlinewidth{0.803000pt}%
\definecolor{currentstroke}{rgb}{0.000000,0.000000,0.000000}%
\pgfsetstrokecolor{currentstroke}%
\pgfsetdash{}{0pt}%
\pgfsys@defobject{currentmarker}{\pgfqpoint{-0.048611in}{0.000000in}}{\pgfqpoint{-0.000000in}{0.000000in}}{%
\pgfpathmoveto{\pgfqpoint{-0.000000in}{0.000000in}}%
\pgfpathlineto{\pgfqpoint{-0.048611in}{0.000000in}}%
\pgfusepath{stroke,fill}%
}%
\begin{pgfscope}%
\pgfsys@transformshift{0.570343in}{0.590557in}%
\pgfsys@useobject{currentmarker}{}%
\end{pgfscope}%
\end{pgfscope}%
\begin{pgfscope}%
\definecolor{textcolor}{rgb}{0.000000,0.000000,0.000000}%
\pgfsetstrokecolor{textcolor}%
\pgfsetfillcolor{textcolor}%
\pgftext[x=0.100000in, y=0.537796in, left, base]{\color{textcolor}\sffamily\fontsize{10.000000}{12.000000}\selectfont \ensuremath{-}100}%
\end{pgfscope}%
\begin{pgfscope}%
\pgfsetbuttcap%
\pgfsetroundjoin%
\definecolor{currentfill}{rgb}{0.000000,0.000000,0.000000}%
\pgfsetfillcolor{currentfill}%
\pgfsetlinewidth{0.803000pt}%
\definecolor{currentstroke}{rgb}{0.000000,0.000000,0.000000}%
\pgfsetstrokecolor{currentstroke}%
\pgfsetdash{}{0pt}%
\pgfsys@defobject{currentmarker}{\pgfqpoint{-0.048611in}{0.000000in}}{\pgfqpoint{-0.000000in}{0.000000in}}{%
\pgfpathmoveto{\pgfqpoint{-0.000000in}{0.000000in}}%
\pgfpathlineto{\pgfqpoint{-0.048611in}{0.000000in}}%
\pgfusepath{stroke,fill}%
}%
\begin{pgfscope}%
\pgfsys@transformshift{0.570343in}{1.514832in}%
\pgfsys@useobject{currentmarker}{}%
\end{pgfscope}%
\end{pgfscope}%
\begin{pgfscope}%
\definecolor{textcolor}{rgb}{0.000000,0.000000,0.000000}%
\pgfsetstrokecolor{textcolor}%
\pgfsetfillcolor{textcolor}%
\pgftext[x=0.188365in, y=1.462070in, left, base]{\color{textcolor}\sffamily\fontsize{10.000000}{12.000000}\selectfont \ensuremath{-}75}%
\end{pgfscope}%
\begin{pgfscope}%
\pgfsetbuttcap%
\pgfsetroundjoin%
\definecolor{currentfill}{rgb}{0.000000,0.000000,0.000000}%
\pgfsetfillcolor{currentfill}%
\pgfsetlinewidth{0.803000pt}%
\definecolor{currentstroke}{rgb}{0.000000,0.000000,0.000000}%
\pgfsetstrokecolor{currentstroke}%
\pgfsetdash{}{0pt}%
\pgfsys@defobject{currentmarker}{\pgfqpoint{-0.048611in}{0.000000in}}{\pgfqpoint{-0.000000in}{0.000000in}}{%
\pgfpathmoveto{\pgfqpoint{-0.000000in}{0.000000in}}%
\pgfpathlineto{\pgfqpoint{-0.048611in}{0.000000in}}%
\pgfusepath{stroke,fill}%
}%
\begin{pgfscope}%
\pgfsys@transformshift{0.570343in}{2.439106in}%
\pgfsys@useobject{currentmarker}{}%
\end{pgfscope}%
\end{pgfscope}%
\begin{pgfscope}%
\definecolor{textcolor}{rgb}{0.000000,0.000000,0.000000}%
\pgfsetstrokecolor{textcolor}%
\pgfsetfillcolor{textcolor}%
\pgftext[x=0.188365in, y=2.386344in, left, base]{\color{textcolor}\sffamily\fontsize{10.000000}{12.000000}\selectfont \ensuremath{-}50}%
\end{pgfscope}%
\begin{pgfscope}%
\pgfsetbuttcap%
\pgfsetroundjoin%
\definecolor{currentfill}{rgb}{0.000000,0.000000,0.000000}%
\pgfsetfillcolor{currentfill}%
\pgfsetlinewidth{0.803000pt}%
\definecolor{currentstroke}{rgb}{0.000000,0.000000,0.000000}%
\pgfsetstrokecolor{currentstroke}%
\pgfsetdash{}{0pt}%
\pgfsys@defobject{currentmarker}{\pgfqpoint{-0.048611in}{0.000000in}}{\pgfqpoint{-0.000000in}{0.000000in}}{%
\pgfpathmoveto{\pgfqpoint{-0.000000in}{0.000000in}}%
\pgfpathlineto{\pgfqpoint{-0.048611in}{0.000000in}}%
\pgfusepath{stroke,fill}%
}%
\begin{pgfscope}%
\pgfsys@transformshift{0.570343in}{3.363380in}%
\pgfsys@useobject{currentmarker}{}%
\end{pgfscope}%
\end{pgfscope}%
\begin{pgfscope}%
\definecolor{textcolor}{rgb}{0.000000,0.000000,0.000000}%
\pgfsetstrokecolor{textcolor}%
\pgfsetfillcolor{textcolor}%
\pgftext[x=0.188365in, y=3.310619in, left, base]{\color{textcolor}\sffamily\fontsize{10.000000}{12.000000}\selectfont \ensuremath{-}25}%
\end{pgfscope}%
\begin{pgfscope}%
\pgfsetbuttcap%
\pgfsetroundjoin%
\definecolor{currentfill}{rgb}{0.000000,0.000000,0.000000}%
\pgfsetfillcolor{currentfill}%
\pgfsetlinewidth{0.803000pt}%
\definecolor{currentstroke}{rgb}{0.000000,0.000000,0.000000}%
\pgfsetstrokecolor{currentstroke}%
\pgfsetdash{}{0pt}%
\pgfsys@defobject{currentmarker}{\pgfqpoint{-0.048611in}{0.000000in}}{\pgfqpoint{-0.000000in}{0.000000in}}{%
\pgfpathmoveto{\pgfqpoint{-0.000000in}{0.000000in}}%
\pgfpathlineto{\pgfqpoint{-0.048611in}{0.000000in}}%
\pgfusepath{stroke,fill}%
}%
\begin{pgfscope}%
\pgfsys@transformshift{0.570343in}{4.287654in}%
\pgfsys@useobject{currentmarker}{}%
\end{pgfscope}%
\end{pgfscope}%
\begin{pgfscope}%
\definecolor{textcolor}{rgb}{0.000000,0.000000,0.000000}%
\pgfsetstrokecolor{textcolor}%
\pgfsetfillcolor{textcolor}%
\pgftext[x=0.384756in, y=4.234893in, left, base]{\color{textcolor}\sffamily\fontsize{10.000000}{12.000000}\selectfont 0}%
\end{pgfscope}%
\begin{pgfscope}%
\pgfsetbuttcap%
\pgfsetroundjoin%
\definecolor{currentfill}{rgb}{0.000000,0.000000,0.000000}%
\pgfsetfillcolor{currentfill}%
\pgfsetlinewidth{0.803000pt}%
\definecolor{currentstroke}{rgb}{0.000000,0.000000,0.000000}%
\pgfsetstrokecolor{currentstroke}%
\pgfsetdash{}{0pt}%
\pgfsys@defobject{currentmarker}{\pgfqpoint{-0.048611in}{0.000000in}}{\pgfqpoint{-0.000000in}{0.000000in}}{%
\pgfpathmoveto{\pgfqpoint{-0.000000in}{0.000000in}}%
\pgfpathlineto{\pgfqpoint{-0.048611in}{0.000000in}}%
\pgfusepath{stroke,fill}%
}%
\begin{pgfscope}%
\pgfsys@transformshift{0.570343in}{5.211929in}%
\pgfsys@useobject{currentmarker}{}%
\end{pgfscope}%
\end{pgfscope}%
\begin{pgfscope}%
\definecolor{textcolor}{rgb}{0.000000,0.000000,0.000000}%
\pgfsetstrokecolor{textcolor}%
\pgfsetfillcolor{textcolor}%
\pgftext[x=0.296390in, y=5.159167in, left, base]{\color{textcolor}\sffamily\fontsize{10.000000}{12.000000}\selectfont 25}%
\end{pgfscope}%
\begin{pgfscope}%
\pgfsetbuttcap%
\pgfsetroundjoin%
\definecolor{currentfill}{rgb}{0.000000,0.000000,0.000000}%
\pgfsetfillcolor{currentfill}%
\pgfsetlinewidth{0.803000pt}%
\definecolor{currentstroke}{rgb}{0.000000,0.000000,0.000000}%
\pgfsetstrokecolor{currentstroke}%
\pgfsetdash{}{0pt}%
\pgfsys@defobject{currentmarker}{\pgfqpoint{-0.048611in}{0.000000in}}{\pgfqpoint{-0.000000in}{0.000000in}}{%
\pgfpathmoveto{\pgfqpoint{-0.000000in}{0.000000in}}%
\pgfpathlineto{\pgfqpoint{-0.048611in}{0.000000in}}%
\pgfusepath{stroke,fill}%
}%
\begin{pgfscope}%
\pgfsys@transformshift{0.570343in}{6.136203in}%
\pgfsys@useobject{currentmarker}{}%
\end{pgfscope}%
\end{pgfscope}%
\begin{pgfscope}%
\definecolor{textcolor}{rgb}{0.000000,0.000000,0.000000}%
\pgfsetstrokecolor{textcolor}%
\pgfsetfillcolor{textcolor}%
\pgftext[x=0.296390in, y=6.083442in, left, base]{\color{textcolor}\sffamily\fontsize{10.000000}{12.000000}\selectfont 50}%
\end{pgfscope}%
\begin{pgfscope}%
\pgfsetbuttcap%
\pgfsetroundjoin%
\definecolor{currentfill}{rgb}{0.000000,0.000000,0.000000}%
\pgfsetfillcolor{currentfill}%
\pgfsetlinewidth{0.803000pt}%
\definecolor{currentstroke}{rgb}{0.000000,0.000000,0.000000}%
\pgfsetstrokecolor{currentstroke}%
\pgfsetdash{}{0pt}%
\pgfsys@defobject{currentmarker}{\pgfqpoint{-0.048611in}{0.000000in}}{\pgfqpoint{-0.000000in}{0.000000in}}{%
\pgfpathmoveto{\pgfqpoint{-0.000000in}{0.000000in}}%
\pgfpathlineto{\pgfqpoint{-0.048611in}{0.000000in}}%
\pgfusepath{stroke,fill}%
}%
\begin{pgfscope}%
\pgfsys@transformshift{0.570343in}{7.060477in}%
\pgfsys@useobject{currentmarker}{}%
\end{pgfscope}%
\end{pgfscope}%
\begin{pgfscope}%
\definecolor{textcolor}{rgb}{0.000000,0.000000,0.000000}%
\pgfsetstrokecolor{textcolor}%
\pgfsetfillcolor{textcolor}%
\pgftext[x=0.296390in, y=7.007716in, left, base]{\color{textcolor}\sffamily\fontsize{10.000000}{12.000000}\selectfont 75}%
\end{pgfscope}%
\begin{pgfscope}%
\pgfsetbuttcap%
\pgfsetroundjoin%
\definecolor{currentfill}{rgb}{0.000000,0.000000,0.000000}%
\pgfsetfillcolor{currentfill}%
\pgfsetlinewidth{0.803000pt}%
\definecolor{currentstroke}{rgb}{0.000000,0.000000,0.000000}%
\pgfsetstrokecolor{currentstroke}%
\pgfsetdash{}{0pt}%
\pgfsys@defobject{currentmarker}{\pgfqpoint{-0.048611in}{0.000000in}}{\pgfqpoint{-0.000000in}{0.000000in}}{%
\pgfpathmoveto{\pgfqpoint{-0.000000in}{0.000000in}}%
\pgfpathlineto{\pgfqpoint{-0.048611in}{0.000000in}}%
\pgfusepath{stroke,fill}%
}%
\begin{pgfscope}%
\pgfsys@transformshift{0.570343in}{7.984752in}%
\pgfsys@useobject{currentmarker}{}%
\end{pgfscope}%
\end{pgfscope}%
\begin{pgfscope}%
\definecolor{textcolor}{rgb}{0.000000,0.000000,0.000000}%
\pgfsetstrokecolor{textcolor}%
\pgfsetfillcolor{textcolor}%
\pgftext[x=0.208025in, y=7.931990in, left, base]{\color{textcolor}\sffamily\fontsize{10.000000}{12.000000}\selectfont 100}%
\end{pgfscope}%
\begin{pgfscope}%
\pgfpathrectangle{\pgfqpoint{0.570343in}{0.331635in}}{\pgfqpoint{9.300000in}{7.700000in}}%
\pgfusepath{clip}%
\pgfsetrectcap%
\pgfsetroundjoin%
\pgfsetlinewidth{1.505625pt}%
\definecolor{currentstroke}{rgb}{0.631373,0.788235,0.956863}%
\pgfsetstrokecolor{currentstroke}%
\pgfsetstrokeopacity{0.800000}%
\pgfsetdash{}{0pt}%
\pgfpathmoveto{\pgfqpoint{7.239021in}{5.114123in}}%
\pgfpathlineto{\pgfqpoint{5.077119in}{4.987631in}}%
\pgfusepath{stroke}%
\end{pgfscope}%
\begin{pgfscope}%
\pgfpathrectangle{\pgfqpoint{0.570343in}{0.331635in}}{\pgfqpoint{9.300000in}{7.700000in}}%
\pgfusepath{clip}%
\pgfsetrectcap%
\pgfsetroundjoin%
\pgfsetlinewidth{1.505625pt}%
\definecolor{currentstroke}{rgb}{0.631373,0.788235,0.956863}%
\pgfsetstrokecolor{currentstroke}%
\pgfsetstrokeopacity{0.800000}%
\pgfsetdash{}{0pt}%
\pgfpathmoveto{\pgfqpoint{6.513392in}{4.010317in}}%
\pgfpathlineto{\pgfqpoint{5.077119in}{4.987631in}}%
\pgfusepath{stroke}%
\end{pgfscope}%
\begin{pgfscope}%
\pgfpathrectangle{\pgfqpoint{0.570343in}{0.331635in}}{\pgfqpoint{9.300000in}{7.700000in}}%
\pgfusepath{clip}%
\pgfsetrectcap%
\pgfsetroundjoin%
\pgfsetlinewidth{1.505625pt}%
\definecolor{currentstroke}{rgb}{0.631373,0.788235,0.956863}%
\pgfsetstrokecolor{currentstroke}%
\pgfsetstrokeopacity{0.800000}%
\pgfsetdash{}{0pt}%
\pgfpathmoveto{\pgfqpoint{2.367651in}{5.521715in}}%
\pgfpathlineto{\pgfqpoint{5.077119in}{4.987631in}}%
\pgfusepath{stroke}%
\end{pgfscope}%
\begin{pgfscope}%
\pgfpathrectangle{\pgfqpoint{0.570343in}{0.331635in}}{\pgfqpoint{9.300000in}{7.700000in}}%
\pgfusepath{clip}%
\pgfsetrectcap%
\pgfsetroundjoin%
\pgfsetlinewidth{1.505625pt}%
\definecolor{currentstroke}{rgb}{0.631373,0.788235,0.956863}%
\pgfsetstrokecolor{currentstroke}%
\pgfsetstrokeopacity{0.800000}%
\pgfsetdash{}{0pt}%
\pgfpathmoveto{\pgfqpoint{6.809391in}{4.549658in}}%
\pgfpathlineto{\pgfqpoint{5.077119in}{4.987631in}}%
\pgfusepath{stroke}%
\end{pgfscope}%
\begin{pgfscope}%
\pgfpathrectangle{\pgfqpoint{0.570343in}{0.331635in}}{\pgfqpoint{9.300000in}{7.700000in}}%
\pgfusepath{clip}%
\pgfsetrectcap%
\pgfsetroundjoin%
\pgfsetlinewidth{1.505625pt}%
\definecolor{currentstroke}{rgb}{0.631373,0.788235,0.956863}%
\pgfsetstrokecolor{currentstroke}%
\pgfsetstrokeopacity{0.800000}%
\pgfsetdash{}{0pt}%
\pgfpathmoveto{\pgfqpoint{4.564595in}{4.413606in}}%
\pgfpathlineto{\pgfqpoint{5.077119in}{4.987631in}}%
\pgfusepath{stroke}%
\end{pgfscope}%
\begin{pgfscope}%
\pgfpathrectangle{\pgfqpoint{0.570343in}{0.331635in}}{\pgfqpoint{9.300000in}{7.700000in}}%
\pgfusepath{clip}%
\pgfsetrectcap%
\pgfsetroundjoin%
\pgfsetlinewidth{1.505625pt}%
\definecolor{currentstroke}{rgb}{0.631373,0.788235,0.956863}%
\pgfsetstrokecolor{currentstroke}%
\pgfsetstrokeopacity{0.800000}%
\pgfsetdash{}{0pt}%
\pgfpathmoveto{\pgfqpoint{5.991395in}{5.277257in}}%
\pgfpathlineto{\pgfqpoint{5.077119in}{4.987631in}}%
\pgfusepath{stroke}%
\end{pgfscope}%
\begin{pgfscope}%
\pgfpathrectangle{\pgfqpoint{0.570343in}{0.331635in}}{\pgfqpoint{9.300000in}{7.700000in}}%
\pgfusepath{clip}%
\pgfsetrectcap%
\pgfsetroundjoin%
\pgfsetlinewidth{1.505625pt}%
\definecolor{currentstroke}{rgb}{0.631373,0.788235,0.956863}%
\pgfsetstrokecolor{currentstroke}%
\pgfsetstrokeopacity{0.800000}%
\pgfsetdash{}{0pt}%
\pgfpathmoveto{\pgfqpoint{3.383575in}{7.143783in}}%
\pgfpathlineto{\pgfqpoint{5.077119in}{4.987631in}}%
\pgfusepath{stroke}%
\end{pgfscope}%
\begin{pgfscope}%
\pgfpathrectangle{\pgfqpoint{0.570343in}{0.331635in}}{\pgfqpoint{9.300000in}{7.700000in}}%
\pgfusepath{clip}%
\pgfsetrectcap%
\pgfsetroundjoin%
\pgfsetlinewidth{1.505625pt}%
\definecolor{currentstroke}{rgb}{0.631373,0.788235,0.956863}%
\pgfsetstrokecolor{currentstroke}%
\pgfsetstrokeopacity{0.800000}%
\pgfsetdash{}{0pt}%
\pgfpathmoveto{\pgfqpoint{3.614236in}{4.262662in}}%
\pgfpathlineto{\pgfqpoint{5.077119in}{4.987631in}}%
\pgfusepath{stroke}%
\end{pgfscope}%
\begin{pgfscope}%
\pgfpathrectangle{\pgfqpoint{0.570343in}{0.331635in}}{\pgfqpoint{9.300000in}{7.700000in}}%
\pgfusepath{clip}%
\pgfsetrectcap%
\pgfsetroundjoin%
\pgfsetlinewidth{1.505625pt}%
\definecolor{currentstroke}{rgb}{0.631373,0.788235,0.956863}%
\pgfsetstrokecolor{currentstroke}%
\pgfsetstrokeopacity{0.800000}%
\pgfsetdash{}{0pt}%
\pgfpathmoveto{\pgfqpoint{8.088914in}{1.938914in}}%
\pgfpathlineto{\pgfqpoint{5.077119in}{4.987631in}}%
\pgfusepath{stroke}%
\end{pgfscope}%
\begin{pgfscope}%
\pgfpathrectangle{\pgfqpoint{0.570343in}{0.331635in}}{\pgfqpoint{9.300000in}{7.700000in}}%
\pgfusepath{clip}%
\pgfsetrectcap%
\pgfsetroundjoin%
\pgfsetlinewidth{1.505625pt}%
\definecolor{currentstroke}{rgb}{0.631373,0.788235,0.956863}%
\pgfsetstrokecolor{currentstroke}%
\pgfsetstrokeopacity{0.800000}%
\pgfsetdash{}{0pt}%
\pgfpathmoveto{\pgfqpoint{6.893373in}{7.628437in}}%
\pgfpathlineto{\pgfqpoint{5.077119in}{4.987631in}}%
\pgfusepath{stroke}%
\end{pgfscope}%
\begin{pgfscope}%
\pgfpathrectangle{\pgfqpoint{0.570343in}{0.331635in}}{\pgfqpoint{9.300000in}{7.700000in}}%
\pgfusepath{clip}%
\pgfsetrectcap%
\pgfsetroundjoin%
\pgfsetlinewidth{1.505625pt}%
\definecolor{currentstroke}{rgb}{0.631373,0.788235,0.956863}%
\pgfsetstrokecolor{currentstroke}%
\pgfsetstrokeopacity{0.800000}%
\pgfsetdash{}{0pt}%
\pgfpathmoveto{\pgfqpoint{8.123953in}{2.936881in}}%
\pgfpathlineto{\pgfqpoint{5.077119in}{4.987631in}}%
\pgfusepath{stroke}%
\end{pgfscope}%
\begin{pgfscope}%
\pgfpathrectangle{\pgfqpoint{0.570343in}{0.331635in}}{\pgfqpoint{9.300000in}{7.700000in}}%
\pgfusepath{clip}%
\pgfsetrectcap%
\pgfsetroundjoin%
\pgfsetlinewidth{1.505625pt}%
\definecolor{currentstroke}{rgb}{0.631373,0.788235,0.956863}%
\pgfsetstrokecolor{currentstroke}%
\pgfsetstrokeopacity{0.800000}%
\pgfsetdash{}{0pt}%
\pgfpathmoveto{\pgfqpoint{6.741009in}{5.814694in}}%
\pgfpathlineto{\pgfqpoint{5.077119in}{4.987631in}}%
\pgfusepath{stroke}%
\end{pgfscope}%
\begin{pgfscope}%
\pgfpathrectangle{\pgfqpoint{0.570343in}{0.331635in}}{\pgfqpoint{9.300000in}{7.700000in}}%
\pgfusepath{clip}%
\pgfsetrectcap%
\pgfsetroundjoin%
\pgfsetlinewidth{1.505625pt}%
\definecolor{currentstroke}{rgb}{0.631373,0.788235,0.956863}%
\pgfsetstrokecolor{currentstroke}%
\pgfsetstrokeopacity{0.800000}%
\pgfsetdash{}{0pt}%
\pgfpathmoveto{\pgfqpoint{4.238932in}{6.412234in}}%
\pgfpathlineto{\pgfqpoint{5.077119in}{4.987631in}}%
\pgfusepath{stroke}%
\end{pgfscope}%
\begin{pgfscope}%
\pgfpathrectangle{\pgfqpoint{0.570343in}{0.331635in}}{\pgfqpoint{9.300000in}{7.700000in}}%
\pgfusepath{clip}%
\pgfsetrectcap%
\pgfsetroundjoin%
\pgfsetlinewidth{1.505625pt}%
\definecolor{currentstroke}{rgb}{0.631373,0.788235,0.956863}%
\pgfsetstrokecolor{currentstroke}%
\pgfsetstrokeopacity{0.800000}%
\pgfsetdash{}{0pt}%
\pgfpathmoveto{\pgfqpoint{3.437634in}{3.499346in}}%
\pgfpathlineto{\pgfqpoint{5.077119in}{4.987631in}}%
\pgfusepath{stroke}%
\end{pgfscope}%
\begin{pgfscope}%
\pgfpathrectangle{\pgfqpoint{0.570343in}{0.331635in}}{\pgfqpoint{9.300000in}{7.700000in}}%
\pgfusepath{clip}%
\pgfsetrectcap%
\pgfsetroundjoin%
\pgfsetlinewidth{1.505625pt}%
\definecolor{currentstroke}{rgb}{0.631373,0.788235,0.956863}%
\pgfsetstrokecolor{currentstroke}%
\pgfsetstrokeopacity{0.800000}%
\pgfsetdash{}{0pt}%
\pgfpathmoveto{\pgfqpoint{1.464475in}{6.222679in}}%
\pgfpathlineto{\pgfqpoint{5.077119in}{4.987631in}}%
\pgfusepath{stroke}%
\end{pgfscope}%
\begin{pgfscope}%
\pgfpathrectangle{\pgfqpoint{0.570343in}{0.331635in}}{\pgfqpoint{9.300000in}{7.700000in}}%
\pgfusepath{clip}%
\pgfsetrectcap%
\pgfsetroundjoin%
\pgfsetlinewidth{1.505625pt}%
\definecolor{currentstroke}{rgb}{0.631373,0.788235,0.956863}%
\pgfsetstrokecolor{currentstroke}%
\pgfsetstrokeopacity{0.800000}%
\pgfsetdash{}{0pt}%
\pgfpathmoveto{\pgfqpoint{0.993071in}{4.960729in}}%
\pgfpathlineto{\pgfqpoint{5.077119in}{4.987631in}}%
\pgfusepath{stroke}%
\end{pgfscope}%
\begin{pgfscope}%
\pgfpathrectangle{\pgfqpoint{0.570343in}{0.331635in}}{\pgfqpoint{9.300000in}{7.700000in}}%
\pgfusepath{clip}%
\pgfsetrectcap%
\pgfsetroundjoin%
\pgfsetlinewidth{1.505625pt}%
\definecolor{currentstroke}{rgb}{0.631373,0.788235,0.956863}%
\pgfsetstrokecolor{currentstroke}%
\pgfsetstrokeopacity{0.800000}%
\pgfsetdash{}{0pt}%
\pgfpathmoveto{\pgfqpoint{5.578651in}{5.963023in}}%
\pgfpathlineto{\pgfqpoint{5.077119in}{4.987631in}}%
\pgfusepath{stroke}%
\end{pgfscope}%
\begin{pgfscope}%
\pgfpathrectangle{\pgfqpoint{0.570343in}{0.331635in}}{\pgfqpoint{9.300000in}{7.700000in}}%
\pgfusepath{clip}%
\pgfsetrectcap%
\pgfsetroundjoin%
\pgfsetlinewidth{1.505625pt}%
\definecolor{currentstroke}{rgb}{0.631373,0.788235,0.956863}%
\pgfsetstrokecolor{currentstroke}%
\pgfsetstrokeopacity{0.800000}%
\pgfsetdash{}{0pt}%
\pgfpathmoveto{\pgfqpoint{3.551278in}{5.632687in}}%
\pgfpathlineto{\pgfqpoint{5.077119in}{4.987631in}}%
\pgfusepath{stroke}%
\end{pgfscope}%
\begin{pgfscope}%
\pgfpathrectangle{\pgfqpoint{0.570343in}{0.331635in}}{\pgfqpoint{9.300000in}{7.700000in}}%
\pgfusepath{clip}%
\pgfsetrectcap%
\pgfsetroundjoin%
\pgfsetlinewidth{1.505625pt}%
\definecolor{currentstroke}{rgb}{0.631373,0.788235,0.956863}%
\pgfsetstrokecolor{currentstroke}%
\pgfsetstrokeopacity{0.800000}%
\pgfsetdash{}{0pt}%
\pgfpathmoveto{\pgfqpoint{7.539850in}{4.436340in}}%
\pgfpathlineto{\pgfqpoint{5.077119in}{4.987631in}}%
\pgfusepath{stroke}%
\end{pgfscope}%
\begin{pgfscope}%
\pgfpathrectangle{\pgfqpoint{0.570343in}{0.331635in}}{\pgfqpoint{9.300000in}{7.700000in}}%
\pgfusepath{clip}%
\pgfsetrectcap%
\pgfsetroundjoin%
\pgfsetlinewidth{1.505625pt}%
\definecolor{currentstroke}{rgb}{0.631373,0.788235,0.956863}%
\pgfsetstrokecolor{currentstroke}%
\pgfsetstrokeopacity{0.800000}%
\pgfsetdash{}{0pt}%
\pgfpathmoveto{\pgfqpoint{6.633358in}{6.419304in}}%
\pgfpathlineto{\pgfqpoint{5.077119in}{4.987631in}}%
\pgfusepath{stroke}%
\end{pgfscope}%
\begin{pgfscope}%
\pgfpathrectangle{\pgfqpoint{0.570343in}{0.331635in}}{\pgfqpoint{9.300000in}{7.700000in}}%
\pgfusepath{clip}%
\pgfsetrectcap%
\pgfsetroundjoin%
\pgfsetlinewidth{1.505625pt}%
\definecolor{currentstroke}{rgb}{0.631373,0.788235,0.956863}%
\pgfsetstrokecolor{currentstroke}%
\pgfsetstrokeopacity{0.800000}%
\pgfsetdash{}{0pt}%
\pgfpathmoveto{\pgfqpoint{9.092676in}{7.084959in}}%
\pgfpathlineto{\pgfqpoint{5.077119in}{4.987631in}}%
\pgfusepath{stroke}%
\end{pgfscope}%
\begin{pgfscope}%
\pgfpathrectangle{\pgfqpoint{0.570343in}{0.331635in}}{\pgfqpoint{9.300000in}{7.700000in}}%
\pgfusepath{clip}%
\pgfsetrectcap%
\pgfsetroundjoin%
\pgfsetlinewidth{1.505625pt}%
\definecolor{currentstroke}{rgb}{0.631373,0.788235,0.956863}%
\pgfsetstrokecolor{currentstroke}%
\pgfsetstrokeopacity{0.800000}%
\pgfsetdash{}{0pt}%
\pgfpathmoveto{\pgfqpoint{4.705737in}{5.610902in}}%
\pgfpathlineto{\pgfqpoint{5.077119in}{4.987631in}}%
\pgfusepath{stroke}%
\end{pgfscope}%
\begin{pgfscope}%
\pgfpathrectangle{\pgfqpoint{0.570343in}{0.331635in}}{\pgfqpoint{9.300000in}{7.700000in}}%
\pgfusepath{clip}%
\pgfsetrectcap%
\pgfsetroundjoin%
\pgfsetlinewidth{1.505625pt}%
\definecolor{currentstroke}{rgb}{0.631373,0.788235,0.956863}%
\pgfsetstrokecolor{currentstroke}%
\pgfsetstrokeopacity{0.800000}%
\pgfsetdash{}{0pt}%
\pgfpathmoveto{\pgfqpoint{3.116901in}{2.831988in}}%
\pgfpathlineto{\pgfqpoint{5.077119in}{4.987631in}}%
\pgfusepath{stroke}%
\end{pgfscope}%
\begin{pgfscope}%
\pgfpathrectangle{\pgfqpoint{0.570343in}{0.331635in}}{\pgfqpoint{9.300000in}{7.700000in}}%
\pgfusepath{clip}%
\pgfsetrectcap%
\pgfsetroundjoin%
\pgfsetlinewidth{1.505625pt}%
\definecolor{currentstroke}{rgb}{0.631373,0.788235,0.956863}%
\pgfsetstrokecolor{currentstroke}%
\pgfsetstrokeopacity{0.800000}%
\pgfsetdash{}{0pt}%
\pgfpathmoveto{\pgfqpoint{4.125517in}{2.846803in}}%
\pgfpathlineto{\pgfqpoint{5.077119in}{4.987631in}}%
\pgfusepath{stroke}%
\end{pgfscope}%
\begin{pgfscope}%
\pgfpathrectangle{\pgfqpoint{0.570343in}{0.331635in}}{\pgfqpoint{9.300000in}{7.700000in}}%
\pgfusepath{clip}%
\pgfsetrectcap%
\pgfsetroundjoin%
\pgfsetlinewidth{1.505625pt}%
\definecolor{currentstroke}{rgb}{0.631373,0.788235,0.956863}%
\pgfsetstrokecolor{currentstroke}%
\pgfsetstrokeopacity{0.800000}%
\pgfsetdash{}{0pt}%
\pgfpathmoveto{\pgfqpoint{3.950923in}{4.986608in}}%
\pgfpathlineto{\pgfqpoint{5.077119in}{4.987631in}}%
\pgfusepath{stroke}%
\end{pgfscope}%
\begin{pgfscope}%
\pgfpathrectangle{\pgfqpoint{0.570343in}{0.331635in}}{\pgfqpoint{9.300000in}{7.700000in}}%
\pgfusepath{clip}%
\pgfsetrectcap%
\pgfsetroundjoin%
\pgfsetlinewidth{1.505625pt}%
\definecolor{currentstroke}{rgb}{0.631373,0.788235,0.956863}%
\pgfsetstrokecolor{currentstroke}%
\pgfsetstrokeopacity{0.800000}%
\pgfsetdash{}{0pt}%
\pgfpathmoveto{\pgfqpoint{2.997649in}{1.937402in}}%
\pgfpathlineto{\pgfqpoint{5.077119in}{4.987631in}}%
\pgfusepath{stroke}%
\end{pgfscope}%
\begin{pgfscope}%
\pgfpathrectangle{\pgfqpoint{0.570343in}{0.331635in}}{\pgfqpoint{9.300000in}{7.700000in}}%
\pgfusepath{clip}%
\pgfsetrectcap%
\pgfsetroundjoin%
\pgfsetlinewidth{1.505625pt}%
\definecolor{currentstroke}{rgb}{0.631373,0.788235,0.956863}%
\pgfsetstrokecolor{currentstroke}%
\pgfsetstrokeopacity{0.800000}%
\pgfsetdash{}{0pt}%
\pgfpathmoveto{\pgfqpoint{5.501193in}{6.855026in}}%
\pgfpathlineto{\pgfqpoint{5.077119in}{4.987631in}}%
\pgfusepath{stroke}%
\end{pgfscope}%
\begin{pgfscope}%
\pgfpathrectangle{\pgfqpoint{0.570343in}{0.331635in}}{\pgfqpoint{9.300000in}{7.700000in}}%
\pgfusepath{clip}%
\pgfsetrectcap%
\pgfsetroundjoin%
\pgfsetlinewidth{1.505625pt}%
\definecolor{currentstroke}{rgb}{0.631373,0.788235,0.956863}%
\pgfsetstrokecolor{currentstroke}%
\pgfsetstrokeopacity{0.800000}%
\pgfsetdash{}{0pt}%
\pgfpathmoveto{\pgfqpoint{4.900969in}{5.341586in}}%
\pgfpathlineto{\pgfqpoint{5.077119in}{4.987631in}}%
\pgfusepath{stroke}%
\end{pgfscope}%
\begin{pgfscope}%
\pgfpathrectangle{\pgfqpoint{0.570343in}{0.331635in}}{\pgfqpoint{9.300000in}{7.700000in}}%
\pgfusepath{clip}%
\pgfsetrectcap%
\pgfsetroundjoin%
\pgfsetlinewidth{1.505625pt}%
\definecolor{currentstroke}{rgb}{1.000000,0.705882,0.509804}%
\pgfsetstrokecolor{currentstroke}%
\pgfsetstrokeopacity{0.800000}%
\pgfsetdash{}{0pt}%
\pgfpathmoveto{\pgfqpoint{5.688827in}{3.714865in}}%
\pgfpathlineto{\pgfqpoint{5.685130in}{3.894237in}}%
\pgfusepath{stroke}%
\end{pgfscope}%
\begin{pgfscope}%
\pgfpathrectangle{\pgfqpoint{0.570343in}{0.331635in}}{\pgfqpoint{9.300000in}{7.700000in}}%
\pgfusepath{clip}%
\pgfsetrectcap%
\pgfsetroundjoin%
\pgfsetlinewidth{1.505625pt}%
\definecolor{currentstroke}{rgb}{1.000000,0.705882,0.509804}%
\pgfsetstrokecolor{currentstroke}%
\pgfsetstrokeopacity{0.800000}%
\pgfsetdash{}{0pt}%
\pgfpathmoveto{\pgfqpoint{7.831480in}{6.739574in}}%
\pgfpathlineto{\pgfqpoint{5.685130in}{3.894237in}}%
\pgfusepath{stroke}%
\end{pgfscope}%
\begin{pgfscope}%
\pgfpathrectangle{\pgfqpoint{0.570343in}{0.331635in}}{\pgfqpoint{9.300000in}{7.700000in}}%
\pgfusepath{clip}%
\pgfsetrectcap%
\pgfsetroundjoin%
\pgfsetlinewidth{1.505625pt}%
\definecolor{currentstroke}{rgb}{1.000000,0.705882,0.509804}%
\pgfsetstrokecolor{currentstroke}%
\pgfsetstrokeopacity{0.800000}%
\pgfsetdash{}{0pt}%
\pgfpathmoveto{\pgfqpoint{5.338526in}{4.899084in}}%
\pgfpathlineto{\pgfqpoint{5.685130in}{3.894237in}}%
\pgfusepath{stroke}%
\end{pgfscope}%
\begin{pgfscope}%
\pgfpathrectangle{\pgfqpoint{0.570343in}{0.331635in}}{\pgfqpoint{9.300000in}{7.700000in}}%
\pgfusepath{clip}%
\pgfsetrectcap%
\pgfsetroundjoin%
\pgfsetlinewidth{1.505625pt}%
\definecolor{currentstroke}{rgb}{1.000000,0.705882,0.509804}%
\pgfsetstrokecolor{currentstroke}%
\pgfsetstrokeopacity{0.800000}%
\pgfsetdash{}{0pt}%
\pgfpathmoveto{\pgfqpoint{6.546058in}{3.234862in}}%
\pgfpathlineto{\pgfqpoint{5.685130in}{3.894237in}}%
\pgfusepath{stroke}%
\end{pgfscope}%
\begin{pgfscope}%
\pgfpathrectangle{\pgfqpoint{0.570343in}{0.331635in}}{\pgfqpoint{9.300000in}{7.700000in}}%
\pgfusepath{clip}%
\pgfsetrectcap%
\pgfsetroundjoin%
\pgfsetlinewidth{1.505625pt}%
\definecolor{currentstroke}{rgb}{1.000000,0.705882,0.509804}%
\pgfsetstrokecolor{currentstroke}%
\pgfsetstrokeopacity{0.800000}%
\pgfsetdash{}{0pt}%
\pgfpathmoveto{\pgfqpoint{8.931283in}{5.394309in}}%
\pgfpathlineto{\pgfqpoint{5.685130in}{3.894237in}}%
\pgfusepath{stroke}%
\end{pgfscope}%
\begin{pgfscope}%
\pgfpathrectangle{\pgfqpoint{0.570343in}{0.331635in}}{\pgfqpoint{9.300000in}{7.700000in}}%
\pgfusepath{clip}%
\pgfsetrectcap%
\pgfsetroundjoin%
\pgfsetlinewidth{1.505625pt}%
\definecolor{currentstroke}{rgb}{1.000000,0.705882,0.509804}%
\pgfsetstrokecolor{currentstroke}%
\pgfsetstrokeopacity{0.800000}%
\pgfsetdash{}{0pt}%
\pgfpathmoveto{\pgfqpoint{4.854875in}{3.454861in}}%
\pgfpathlineto{\pgfqpoint{5.685130in}{3.894237in}}%
\pgfusepath{stroke}%
\end{pgfscope}%
\begin{pgfscope}%
\pgfpathrectangle{\pgfqpoint{0.570343in}{0.331635in}}{\pgfqpoint{9.300000in}{7.700000in}}%
\pgfusepath{clip}%
\pgfsetrectcap%
\pgfsetroundjoin%
\pgfsetlinewidth{1.505625pt}%
\definecolor{currentstroke}{rgb}{1.000000,0.705882,0.509804}%
\pgfsetstrokecolor{currentstroke}%
\pgfsetstrokeopacity{0.800000}%
\pgfsetdash{}{0pt}%
\pgfpathmoveto{\pgfqpoint{7.413796in}{3.644013in}}%
\pgfpathlineto{\pgfqpoint{5.685130in}{3.894237in}}%
\pgfusepath{stroke}%
\end{pgfscope}%
\begin{pgfscope}%
\pgfpathrectangle{\pgfqpoint{0.570343in}{0.331635in}}{\pgfqpoint{9.300000in}{7.700000in}}%
\pgfusepath{clip}%
\pgfsetrectcap%
\pgfsetroundjoin%
\pgfsetlinewidth{1.505625pt}%
\definecolor{currentstroke}{rgb}{1.000000,0.705882,0.509804}%
\pgfsetstrokecolor{currentstroke}%
\pgfsetstrokeopacity{0.800000}%
\pgfsetdash{}{0pt}%
\pgfpathmoveto{\pgfqpoint{1.671602in}{3.907156in}}%
\pgfpathlineto{\pgfqpoint{5.685130in}{3.894237in}}%
\pgfusepath{stroke}%
\end{pgfscope}%
\begin{pgfscope}%
\pgfpathrectangle{\pgfqpoint{0.570343in}{0.331635in}}{\pgfqpoint{9.300000in}{7.700000in}}%
\pgfusepath{clip}%
\pgfsetrectcap%
\pgfsetroundjoin%
\pgfsetlinewidth{1.505625pt}%
\definecolor{currentstroke}{rgb}{1.000000,0.705882,0.509804}%
\pgfsetstrokecolor{currentstroke}%
\pgfsetstrokeopacity{0.800000}%
\pgfsetdash{}{0pt}%
\pgfpathmoveto{\pgfqpoint{1.885787in}{2.932540in}}%
\pgfpathlineto{\pgfqpoint{5.685130in}{3.894237in}}%
\pgfusepath{stroke}%
\end{pgfscope}%
\begin{pgfscope}%
\pgfpathrectangle{\pgfqpoint{0.570343in}{0.331635in}}{\pgfqpoint{9.300000in}{7.700000in}}%
\pgfusepath{clip}%
\pgfsetrectcap%
\pgfsetroundjoin%
\pgfsetlinewidth{1.505625pt}%
\definecolor{currentstroke}{rgb}{1.000000,0.705882,0.509804}%
\pgfsetstrokecolor{currentstroke}%
\pgfsetstrokeopacity{0.800000}%
\pgfsetdash{}{0pt}%
\pgfpathmoveto{\pgfqpoint{5.353276in}{4.288421in}}%
\pgfpathlineto{\pgfqpoint{5.685130in}{3.894237in}}%
\pgfusepath{stroke}%
\end{pgfscope}%
\begin{pgfscope}%
\pgfpathrectangle{\pgfqpoint{0.570343in}{0.331635in}}{\pgfqpoint{9.300000in}{7.700000in}}%
\pgfusepath{clip}%
\pgfsetrectcap%
\pgfsetroundjoin%
\pgfsetlinewidth{1.505625pt}%
\definecolor{currentstroke}{rgb}{1.000000,0.705882,0.509804}%
\pgfsetstrokecolor{currentstroke}%
\pgfsetstrokeopacity{0.800000}%
\pgfsetdash{}{0pt}%
\pgfpathmoveto{\pgfqpoint{6.076381in}{4.619110in}}%
\pgfpathlineto{\pgfqpoint{5.685130in}{3.894237in}}%
\pgfusepath{stroke}%
\end{pgfscope}%
\begin{pgfscope}%
\pgfpathrectangle{\pgfqpoint{0.570343in}{0.331635in}}{\pgfqpoint{9.300000in}{7.700000in}}%
\pgfusepath{clip}%
\pgfsetrectcap%
\pgfsetroundjoin%
\pgfsetlinewidth{1.505625pt}%
\definecolor{currentstroke}{rgb}{1.000000,0.705882,0.509804}%
\pgfsetstrokecolor{currentstroke}%
\pgfsetstrokeopacity{0.800000}%
\pgfsetdash{}{0pt}%
\pgfpathmoveto{\pgfqpoint{4.577524in}{7.681635in}}%
\pgfpathlineto{\pgfqpoint{5.685130in}{3.894237in}}%
\pgfusepath{stroke}%
\end{pgfscope}%
\begin{pgfscope}%
\pgfpathrectangle{\pgfqpoint{0.570343in}{0.331635in}}{\pgfqpoint{9.300000in}{7.700000in}}%
\pgfusepath{clip}%
\pgfsetrectcap%
\pgfsetroundjoin%
\pgfsetlinewidth{1.505625pt}%
\definecolor{currentstroke}{rgb}{1.000000,0.705882,0.509804}%
\pgfsetstrokecolor{currentstroke}%
\pgfsetstrokeopacity{0.800000}%
\pgfsetdash{}{0pt}%
\pgfpathmoveto{\pgfqpoint{2.562027in}{6.283257in}}%
\pgfpathlineto{\pgfqpoint{5.685130in}{3.894237in}}%
\pgfusepath{stroke}%
\end{pgfscope}%
\begin{pgfscope}%
\pgfpathrectangle{\pgfqpoint{0.570343in}{0.331635in}}{\pgfqpoint{9.300000in}{7.700000in}}%
\pgfusepath{clip}%
\pgfsetrectcap%
\pgfsetroundjoin%
\pgfsetlinewidth{1.505625pt}%
\definecolor{currentstroke}{rgb}{1.000000,0.705882,0.509804}%
\pgfsetstrokecolor{currentstroke}%
\pgfsetstrokeopacity{0.800000}%
\pgfsetdash{}{0pt}%
\pgfpathmoveto{\pgfqpoint{8.392100in}{4.902301in}}%
\pgfpathlineto{\pgfqpoint{5.685130in}{3.894237in}}%
\pgfusepath{stroke}%
\end{pgfscope}%
\begin{pgfscope}%
\pgfpathrectangle{\pgfqpoint{0.570343in}{0.331635in}}{\pgfqpoint{9.300000in}{7.700000in}}%
\pgfusepath{clip}%
\pgfsetrectcap%
\pgfsetroundjoin%
\pgfsetlinewidth{1.505625pt}%
\definecolor{currentstroke}{rgb}{1.000000,0.705882,0.509804}%
\pgfsetstrokecolor{currentstroke}%
\pgfsetstrokeopacity{0.800000}%
\pgfsetdash{}{0pt}%
\pgfpathmoveto{\pgfqpoint{5.806842in}{1.707715in}}%
\pgfpathlineto{\pgfqpoint{5.685130in}{3.894237in}}%
\pgfusepath{stroke}%
\end{pgfscope}%
\begin{pgfscope}%
\pgfpathrectangle{\pgfqpoint{0.570343in}{0.331635in}}{\pgfqpoint{9.300000in}{7.700000in}}%
\pgfusepath{clip}%
\pgfsetrectcap%
\pgfsetroundjoin%
\pgfsetlinewidth{1.505625pt}%
\definecolor{currentstroke}{rgb}{1.000000,0.705882,0.509804}%
\pgfsetstrokecolor{currentstroke}%
\pgfsetstrokeopacity{0.800000}%
\pgfsetdash{}{0pt}%
\pgfpathmoveto{\pgfqpoint{4.120169in}{3.735967in}}%
\pgfpathlineto{\pgfqpoint{5.685130in}{3.894237in}}%
\pgfusepath{stroke}%
\end{pgfscope}%
\begin{pgfscope}%
\pgfpathrectangle{\pgfqpoint{0.570343in}{0.331635in}}{\pgfqpoint{9.300000in}{7.700000in}}%
\pgfusepath{clip}%
\pgfsetrectcap%
\pgfsetroundjoin%
\pgfsetlinewidth{1.505625pt}%
\definecolor{currentstroke}{rgb}{1.000000,0.705882,0.509804}%
\pgfsetstrokecolor{currentstroke}%
\pgfsetstrokeopacity{0.800000}%
\pgfsetdash{}{0pt}%
\pgfpathmoveto{\pgfqpoint{3.866323in}{1.837880in}}%
\pgfpathlineto{\pgfqpoint{5.685130in}{3.894237in}}%
\pgfusepath{stroke}%
\end{pgfscope}%
\begin{pgfscope}%
\pgfpathrectangle{\pgfqpoint{0.570343in}{0.331635in}}{\pgfqpoint{9.300000in}{7.700000in}}%
\pgfusepath{clip}%
\pgfsetrectcap%
\pgfsetroundjoin%
\pgfsetlinewidth{1.505625pt}%
\definecolor{currentstroke}{rgb}{1.000000,0.705882,0.509804}%
\pgfsetstrokecolor{currentstroke}%
\pgfsetstrokeopacity{0.800000}%
\pgfsetdash{}{0pt}%
\pgfpathmoveto{\pgfqpoint{2.788649in}{4.018903in}}%
\pgfpathlineto{\pgfqpoint{5.685130in}{3.894237in}}%
\pgfusepath{stroke}%
\end{pgfscope}%
\begin{pgfscope}%
\pgfpathrectangle{\pgfqpoint{0.570343in}{0.331635in}}{\pgfqpoint{9.300000in}{7.700000in}}%
\pgfusepath{clip}%
\pgfsetrectcap%
\pgfsetroundjoin%
\pgfsetlinewidth{1.505625pt}%
\definecolor{currentstroke}{rgb}{1.000000,0.705882,0.509804}%
\pgfsetstrokecolor{currentstroke}%
\pgfsetstrokeopacity{0.800000}%
\pgfsetdash{}{0pt}%
\pgfpathmoveto{\pgfqpoint{9.447616in}{3.664094in}}%
\pgfpathlineto{\pgfqpoint{5.685130in}{3.894237in}}%
\pgfusepath{stroke}%
\end{pgfscope}%
\begin{pgfscope}%
\pgfpathrectangle{\pgfqpoint{0.570343in}{0.331635in}}{\pgfqpoint{9.300000in}{7.700000in}}%
\pgfusepath{clip}%
\pgfsetrectcap%
\pgfsetroundjoin%
\pgfsetlinewidth{1.505625pt}%
\definecolor{currentstroke}{rgb}{1.000000,0.705882,0.509804}%
\pgfsetstrokecolor{currentstroke}%
\pgfsetstrokeopacity{0.800000}%
\pgfsetdash{}{0pt}%
\pgfpathmoveto{\pgfqpoint{5.814101in}{0.681635in}}%
\pgfpathlineto{\pgfqpoint{5.685130in}{3.894237in}}%
\pgfusepath{stroke}%
\end{pgfscope}%
\begin{pgfscope}%
\pgfpathrectangle{\pgfqpoint{0.570343in}{0.331635in}}{\pgfqpoint{9.300000in}{7.700000in}}%
\pgfusepath{clip}%
\pgfsetrectcap%
\pgfsetroundjoin%
\pgfsetlinewidth{1.505625pt}%
\definecolor{currentstroke}{rgb}{1.000000,0.705882,0.509804}%
\pgfsetstrokecolor{currentstroke}%
\pgfsetstrokeopacity{0.800000}%
\pgfsetdash{}{0pt}%
\pgfpathmoveto{\pgfqpoint{5.056091in}{2.274711in}}%
\pgfpathlineto{\pgfqpoint{5.685130in}{3.894237in}}%
\pgfusepath{stroke}%
\end{pgfscope}%
\begin{pgfscope}%
\pgfpathrectangle{\pgfqpoint{0.570343in}{0.331635in}}{\pgfqpoint{9.300000in}{7.700000in}}%
\pgfusepath{clip}%
\pgfsetrectcap%
\pgfsetroundjoin%
\pgfsetlinewidth{1.505625pt}%
\definecolor{currentstroke}{rgb}{1.000000,0.705882,0.509804}%
\pgfsetstrokecolor{currentstroke}%
\pgfsetstrokeopacity{0.800000}%
\pgfsetdash{}{0pt}%
\pgfpathmoveto{\pgfqpoint{8.597790in}{3.819155in}}%
\pgfpathlineto{\pgfqpoint{5.685130in}{3.894237in}}%
\pgfusepath{stroke}%
\end{pgfscope}%
\begin{pgfscope}%
\pgfpathrectangle{\pgfqpoint{0.570343in}{0.331635in}}{\pgfqpoint{9.300000in}{7.700000in}}%
\pgfusepath{clip}%
\pgfsetrectcap%
\pgfsetroundjoin%
\pgfsetlinewidth{1.505625pt}%
\definecolor{currentstroke}{rgb}{1.000000,0.705882,0.509804}%
\pgfsetstrokecolor{currentstroke}%
\pgfsetstrokeopacity{0.800000}%
\pgfsetdash{}{0pt}%
\pgfpathmoveto{\pgfqpoint{4.061550in}{0.854936in}}%
\pgfpathlineto{\pgfqpoint{5.685130in}{3.894237in}}%
\pgfusepath{stroke}%
\end{pgfscope}%
\begin{pgfscope}%
\pgfpathrectangle{\pgfqpoint{0.570343in}{0.331635in}}{\pgfqpoint{9.300000in}{7.700000in}}%
\pgfusepath{clip}%
\pgfsetrectcap%
\pgfsetroundjoin%
\pgfsetlinewidth{1.505625pt}%
\definecolor{currentstroke}{rgb}{1.000000,0.705882,0.509804}%
\pgfsetstrokecolor{currentstroke}%
\pgfsetstrokeopacity{0.800000}%
\pgfsetdash{}{0pt}%
\pgfpathmoveto{\pgfqpoint{7.863027in}{5.805960in}}%
\pgfpathlineto{\pgfqpoint{5.685130in}{3.894237in}}%
\pgfusepath{stroke}%
\end{pgfscope}%
\begin{pgfscope}%
\pgfpathrectangle{\pgfqpoint{0.570343in}{0.331635in}}{\pgfqpoint{9.300000in}{7.700000in}}%
\pgfusepath{clip}%
\pgfsetrectcap%
\pgfsetroundjoin%
\pgfsetlinewidth{1.505625pt}%
\definecolor{currentstroke}{rgb}{1.000000,0.705882,0.509804}%
\pgfsetstrokecolor{currentstroke}%
\pgfsetstrokeopacity{0.800000}%
\pgfsetdash{}{0pt}%
\pgfpathmoveto{\pgfqpoint{6.615719in}{2.506683in}}%
\pgfpathlineto{\pgfqpoint{5.685130in}{3.894237in}}%
\pgfusepath{stroke}%
\end{pgfscope}%
\begin{pgfscope}%
\pgfpathrectangle{\pgfqpoint{0.570343in}{0.331635in}}{\pgfqpoint{9.300000in}{7.700000in}}%
\pgfusepath{clip}%
\pgfsetrectcap%
\pgfsetroundjoin%
\pgfsetlinewidth{1.505625pt}%
\definecolor{currentstroke}{rgb}{1.000000,0.705882,0.509804}%
\pgfsetstrokecolor{currentstroke}%
\pgfsetstrokeopacity{0.800000}%
\pgfsetdash{}{0pt}%
\pgfpathmoveto{\pgfqpoint{3.036089in}{4.944505in}}%
\pgfpathlineto{\pgfqpoint{5.685130in}{3.894237in}}%
\pgfusepath{stroke}%
\end{pgfscope}%
\begin{pgfscope}%
\pgfpathrectangle{\pgfqpoint{0.570343in}{0.331635in}}{\pgfqpoint{9.300000in}{7.700000in}}%
\pgfusepath{clip}%
\pgfsetrectcap%
\pgfsetroundjoin%
\pgfsetlinewidth{1.505625pt}%
\definecolor{currentstroke}{rgb}{1.000000,0.705882,0.509804}%
\pgfsetstrokecolor{currentstroke}%
\pgfsetstrokeopacity{0.800000}%
\pgfsetdash{}{0pt}%
\pgfpathmoveto{\pgfqpoint{9.333987in}{4.618456in}}%
\pgfpathlineto{\pgfqpoint{5.685130in}{3.894237in}}%
\pgfusepath{stroke}%
\end{pgfscope}%
\begin{pgfscope}%
\pgfpathrectangle{\pgfqpoint{0.570343in}{0.331635in}}{\pgfqpoint{9.300000in}{7.700000in}}%
\pgfusepath{clip}%
\pgfsetrectcap%
\pgfsetroundjoin%
\pgfsetlinewidth{1.505625pt}%
\definecolor{currentstroke}{rgb}{1.000000,0.705882,0.509804}%
\pgfsetstrokecolor{currentstroke}%
\pgfsetstrokeopacity{0.800000}%
\pgfsetdash{}{0pt}%
\pgfpathmoveto{\pgfqpoint{5.652131in}{2.872041in}}%
\pgfpathlineto{\pgfqpoint{5.685130in}{3.894237in}}%
\pgfusepath{stroke}%
\end{pgfscope}%
\begin{pgfscope}%
\pgfsetrectcap%
\pgfsetmiterjoin%
\pgfsetlinewidth{0.803000pt}%
\definecolor{currentstroke}{rgb}{0.000000,0.000000,0.000000}%
\pgfsetstrokecolor{currentstroke}%
\pgfsetdash{}{0pt}%
\pgfpathmoveto{\pgfqpoint{0.570343in}{0.331635in}}%
\pgfpathlineto{\pgfqpoint{0.570343in}{8.031635in}}%
\pgfusepath{stroke}%
\end{pgfscope}%
\begin{pgfscope}%
\pgfsetrectcap%
\pgfsetmiterjoin%
\pgfsetlinewidth{0.803000pt}%
\definecolor{currentstroke}{rgb}{0.000000,0.000000,0.000000}%
\pgfsetstrokecolor{currentstroke}%
\pgfsetdash{}{0pt}%
\pgfpathmoveto{\pgfqpoint{9.870343in}{0.331635in}}%
\pgfpathlineto{\pgfqpoint{9.870343in}{8.031635in}}%
\pgfusepath{stroke}%
\end{pgfscope}%
\begin{pgfscope}%
\pgfsetrectcap%
\pgfsetmiterjoin%
\pgfsetlinewidth{0.803000pt}%
\definecolor{currentstroke}{rgb}{0.000000,0.000000,0.000000}%
\pgfsetstrokecolor{currentstroke}%
\pgfsetdash{}{0pt}%
\pgfpathmoveto{\pgfqpoint{0.570343in}{0.331635in}}%
\pgfpathlineto{\pgfqpoint{9.870343in}{0.331635in}}%
\pgfusepath{stroke}%
\end{pgfscope}%
\begin{pgfscope}%
\pgfsetrectcap%
\pgfsetmiterjoin%
\pgfsetlinewidth{0.803000pt}%
\definecolor{currentstroke}{rgb}{0.000000,0.000000,0.000000}%
\pgfsetstrokecolor{currentstroke}%
\pgfsetdash{}{0pt}%
\pgfpathmoveto{\pgfqpoint{0.570343in}{8.031635in}}%
\pgfpathlineto{\pgfqpoint{9.870343in}{8.031635in}}%
\pgfusepath{stroke}%
\end{pgfscope}%
\begin{pgfscope}%
\definecolor{textcolor}{rgb}{0.000000,0.000000,0.000000}%
\pgfsetstrokecolor{textcolor}%
\pgfsetfillcolor{textcolor}%
\pgftext[x=5.220343in,y=8.114968in,,base]{\color{textcolor}\sffamily\fontsize{12.000000}{14.400000}\selectfont Photo-Realistic Images}%
\end{pgfscope}%
\begin{pgfscope}%
\pgfsetbuttcap%
\pgfsetmiterjoin%
\definecolor{currentfill}{rgb}{1.000000,1.000000,1.000000}%
\pgfsetfillcolor{currentfill}%
\pgfsetfillopacity{0.800000}%
\pgfsetlinewidth{1.003750pt}%
\definecolor{currentstroke}{rgb}{0.800000,0.800000,0.800000}%
\pgfsetstrokecolor{currentstroke}%
\pgfsetstrokeopacity{0.800000}%
\pgfsetdash{}{0pt}%
\pgfpathmoveto{\pgfqpoint{9.967566in}{3.956944in}}%
\pgfpathlineto{\pgfqpoint{10.918941in}{3.956944in}}%
\pgfpathquadraticcurveto{\pgfqpoint{10.946719in}{3.956944in}}{\pgfqpoint{10.946719in}{3.984722in}}%
\pgfpathlineto{\pgfqpoint{10.946719in}{4.378548in}}%
\pgfpathquadraticcurveto{\pgfqpoint{10.946719in}{4.406326in}}{\pgfqpoint{10.918941in}{4.406326in}}%
\pgfpathlineto{\pgfqpoint{9.967566in}{4.406326in}}%
\pgfpathquadraticcurveto{\pgfqpoint{9.939788in}{4.406326in}}{\pgfqpoint{9.939788in}{4.378548in}}%
\pgfpathlineto{\pgfqpoint{9.939788in}{3.984722in}}%
\pgfpathquadraticcurveto{\pgfqpoint{9.939788in}{3.956944in}}{\pgfqpoint{9.967566in}{3.956944in}}%
\pgfpathclose%
\pgfusepath{stroke,fill}%
\end{pgfscope}%
\begin{pgfscope}%
\pgfsetbuttcap%
\pgfsetroundjoin%
\definecolor{currentfill}{rgb}{0.631373,0.788235,0.956863}%
\pgfsetfillcolor{currentfill}%
\pgfsetlinewidth{1.003750pt}%
\definecolor{currentstroke}{rgb}{0.631373,0.788235,0.956863}%
\pgfsetstrokecolor{currentstroke}%
\pgfsetdash{}{0pt}%
\pgfsys@defobject{currentmarker}{\pgfqpoint{-0.041667in}{-0.041667in}}{\pgfqpoint{0.041667in}{0.041667in}}{%
\pgfpathmoveto{\pgfqpoint{0.000000in}{-0.041667in}}%
\pgfpathcurveto{\pgfqpoint{0.011050in}{-0.041667in}}{\pgfqpoint{0.021649in}{-0.037276in}}{\pgfqpoint{0.029463in}{-0.029463in}}%
\pgfpathcurveto{\pgfqpoint{0.037276in}{-0.021649in}}{\pgfqpoint{0.041667in}{-0.011050in}}{\pgfqpoint{0.041667in}{0.000000in}}%
\pgfpathcurveto{\pgfqpoint{0.041667in}{0.011050in}}{\pgfqpoint{0.037276in}{0.021649in}}{\pgfqpoint{0.029463in}{0.029463in}}%
\pgfpathcurveto{\pgfqpoint{0.021649in}{0.037276in}}{\pgfqpoint{0.011050in}{0.041667in}}{\pgfqpoint{0.000000in}{0.041667in}}%
\pgfpathcurveto{\pgfqpoint{-0.011050in}{0.041667in}}{\pgfqpoint{-0.021649in}{0.037276in}}{\pgfqpoint{-0.029463in}{0.029463in}}%
\pgfpathcurveto{\pgfqpoint{-0.037276in}{0.021649in}}{\pgfqpoint{-0.041667in}{0.011050in}}{\pgfqpoint{-0.041667in}{0.000000in}}%
\pgfpathcurveto{\pgfqpoint{-0.041667in}{-0.011050in}}{\pgfqpoint{-0.037276in}{-0.021649in}}{\pgfqpoint{-0.029463in}{-0.029463in}}%
\pgfpathcurveto{\pgfqpoint{-0.021649in}{-0.037276in}}{\pgfqpoint{-0.011050in}{-0.041667in}}{\pgfqpoint{0.000000in}{-0.041667in}}%
\pgfpathclose%
\pgfusepath{stroke,fill}%
}%
\begin{pgfscope}%
\pgfsys@transformshift{10.134232in}{4.281705in}%
\pgfsys@useobject{currentmarker}{}%
\end{pgfscope}%
\end{pgfscope}%
\begin{pgfscope}%
\definecolor{textcolor}{rgb}{0.000000,0.000000,0.000000}%
\pgfsetstrokecolor{textcolor}%
\pgfsetfillcolor{textcolor}%
\pgftext[x=10.384232in,y=4.245247in,left,base]{\color{textcolor}\sffamily\fontsize{10.000000}{12.000000}\selectfont 3dfront}%
\end{pgfscope}%
\begin{pgfscope}%
\pgfsetbuttcap%
\pgfsetroundjoin%
\definecolor{currentfill}{rgb}{1.000000,0.705882,0.509804}%
\pgfsetfillcolor{currentfill}%
\pgfsetlinewidth{1.003750pt}%
\definecolor{currentstroke}{rgb}{1.000000,0.705882,0.509804}%
\pgfsetstrokecolor{currentstroke}%
\pgfsetdash{}{0pt}%
\pgfsys@defobject{currentmarker}{\pgfqpoint{-0.041667in}{-0.041667in}}{\pgfqpoint{0.041667in}{0.041667in}}{%
\pgfpathmoveto{\pgfqpoint{0.000000in}{-0.041667in}}%
\pgfpathcurveto{\pgfqpoint{0.011050in}{-0.041667in}}{\pgfqpoint{0.021649in}{-0.037276in}}{\pgfqpoint{0.029463in}{-0.029463in}}%
\pgfpathcurveto{\pgfqpoint{0.037276in}{-0.021649in}}{\pgfqpoint{0.041667in}{-0.011050in}}{\pgfqpoint{0.041667in}{0.000000in}}%
\pgfpathcurveto{\pgfqpoint{0.041667in}{0.011050in}}{\pgfqpoint{0.037276in}{0.021649in}}{\pgfqpoint{0.029463in}{0.029463in}}%
\pgfpathcurveto{\pgfqpoint{0.021649in}{0.037276in}}{\pgfqpoint{0.011050in}{0.041667in}}{\pgfqpoint{0.000000in}{0.041667in}}%
\pgfpathcurveto{\pgfqpoint{-0.011050in}{0.041667in}}{\pgfqpoint{-0.021649in}{0.037276in}}{\pgfqpoint{-0.029463in}{0.029463in}}%
\pgfpathcurveto{\pgfqpoint{-0.037276in}{0.021649in}}{\pgfqpoint{-0.041667in}{0.011050in}}{\pgfqpoint{-0.041667in}{0.000000in}}%
\pgfpathcurveto{\pgfqpoint{-0.041667in}{-0.011050in}}{\pgfqpoint{-0.037276in}{-0.021649in}}{\pgfqpoint{-0.029463in}{-0.029463in}}%
\pgfpathcurveto{\pgfqpoint{-0.021649in}{-0.037276in}}{\pgfqpoint{-0.011050in}{-0.041667in}}{\pgfqpoint{0.000000in}{-0.041667in}}%
\pgfpathclose%
\pgfusepath{stroke,fill}%
}%
\begin{pgfscope}%
\pgfsys@transformshift{10.134232in}{4.077848in}%
\pgfsys@useobject{currentmarker}{}%
\end{pgfscope}%
\end{pgfscope}%
\begin{pgfscope}%
\definecolor{textcolor}{rgb}{0.000000,0.000000,0.000000}%
\pgfsetstrokecolor{textcolor}%
\pgfsetfillcolor{textcolor}%
\pgftext[x=10.384232in,y=4.041390in,left,base]{\color{textcolor}\sffamily\fontsize{10.000000}{12.000000}\selectfont pix3d}%
\end{pgfscope}%
\end{pgfpicture}%
\makeatother%
\endgroup%
}
    \resizebox{0.49\linewidth}{5cm}{%% Creator: Matplotlib, PGF backend
%%
%% To include the figure in your LaTeX document, write
%%   \input{<filename>.pgf}
%%
%% Make sure the required packages are loaded in your preamble
%%   \usepackage{pgf}
%%
%% Figures using additional raster images can only be included by \input if
%% they are in the same directory as the main LaTeX file. For loading figures
%% from other directories you can use the `import` package
%%   \usepackage{import}
%%
%% and then include the figures with
%%   \import{<path to file>}{<filename>.pgf}
%%
%% Matplotlib used the following preamble
%%   \usepackage{fontspec}
%%   \setmainfont{DejaVuSerif.ttf}[Path=\detokenize{/Users/apple/opt/anaconda3/envs/kaolin/lib/python3.7/site-packages/matplotlib/mpl-data/fonts/ttf/}]
%%   \setsansfont{DejaVuSans.ttf}[Path=\detokenize{/Users/apple/opt/anaconda3/envs/kaolin/lib/python3.7/site-packages/matplotlib/mpl-data/fonts/ttf/}]
%%   \setmonofont{DejaVuSansMono.ttf}[Path=\detokenize{/Users/apple/opt/anaconda3/envs/kaolin/lib/python3.7/site-packages/matplotlib/mpl-data/fonts/ttf/}]
%%
\begingroup%
\makeatletter%
\begin{pgfpicture}%
\pgfpathrectangle{\pgfpointorigin}{\pgfqpoint{11.186964in}{8.341596in}}%
\pgfusepath{use as bounding box, clip}%
\begin{pgfscope}%
\pgfsetbuttcap%
\pgfsetmiterjoin%
\definecolor{currentfill}{rgb}{1.000000,1.000000,1.000000}%
\pgfsetfillcolor{currentfill}%
\pgfsetlinewidth{0.000000pt}%
\definecolor{currentstroke}{rgb}{1.000000,1.000000,1.000000}%
\pgfsetstrokecolor{currentstroke}%
\pgfsetdash{}{0pt}%
\pgfpathmoveto{\pgfqpoint{0.000000in}{0.000000in}}%
\pgfpathlineto{\pgfqpoint{11.186964in}{0.000000in}}%
\pgfpathlineto{\pgfqpoint{11.186964in}{8.341596in}}%
\pgfpathlineto{\pgfqpoint{0.000000in}{8.341596in}}%
\pgfpathclose%
\pgfusepath{fill}%
\end{pgfscope}%
\begin{pgfscope}%
\pgfsetbuttcap%
\pgfsetmiterjoin%
\definecolor{currentfill}{rgb}{1.000000,1.000000,1.000000}%
\pgfsetfillcolor{currentfill}%
\pgfsetlinewidth{0.000000pt}%
\definecolor{currentstroke}{rgb}{0.000000,0.000000,0.000000}%
\pgfsetstrokecolor{currentstroke}%
\pgfsetstrokeopacity{0.000000}%
\pgfsetdash{}{0pt}%
\pgfpathmoveto{\pgfqpoint{0.570343in}{0.331635in}}%
\pgfpathlineto{\pgfqpoint{9.870343in}{0.331635in}}%
\pgfpathlineto{\pgfqpoint{9.870343in}{8.031635in}}%
\pgfpathlineto{\pgfqpoint{0.570343in}{8.031635in}}%
\pgfpathclose%
\pgfusepath{fill}%
\end{pgfscope}%
\begin{pgfscope}%
\pgfpathrectangle{\pgfqpoint{0.570343in}{0.331635in}}{\pgfqpoint{9.300000in}{7.700000in}}%
\pgfusepath{clip}%
\pgfsetbuttcap%
\pgfsetroundjoin%
\definecolor{currentfill}{rgb}{0.631373,0.788235,0.956863}%
\pgfsetfillcolor{currentfill}%
\pgfsetlinewidth{0.481800pt}%
\definecolor{currentstroke}{rgb}{1.000000,1.000000,1.000000}%
\pgfsetstrokecolor{currentstroke}%
\pgfsetdash{}{0pt}%
\pgfpathmoveto{\pgfqpoint{6.993440in}{5.386807in}}%
\pgfpathcurveto{\pgfqpoint{7.004490in}{5.386807in}}{\pgfqpoint{7.015089in}{5.391197in}}{\pgfqpoint{7.022903in}{5.399011in}}%
\pgfpathcurveto{\pgfqpoint{7.030716in}{5.406825in}}{\pgfqpoint{7.035106in}{5.417424in}}{\pgfqpoint{7.035106in}{5.428474in}}%
\pgfpathcurveto{\pgfqpoint{7.035106in}{5.439524in}}{\pgfqpoint{7.030716in}{5.450123in}}{\pgfqpoint{7.022903in}{5.457937in}}%
\pgfpathcurveto{\pgfqpoint{7.015089in}{5.465750in}}{\pgfqpoint{7.004490in}{5.470140in}}{\pgfqpoint{6.993440in}{5.470140in}}%
\pgfpathcurveto{\pgfqpoint{6.982390in}{5.470140in}}{\pgfqpoint{6.971791in}{5.465750in}}{\pgfqpoint{6.963977in}{5.457937in}}%
\pgfpathcurveto{\pgfqpoint{6.956163in}{5.450123in}}{\pgfqpoint{6.951773in}{5.439524in}}{\pgfqpoint{6.951773in}{5.428474in}}%
\pgfpathcurveto{\pgfqpoint{6.951773in}{5.417424in}}{\pgfqpoint{6.956163in}{5.406825in}}{\pgfqpoint{6.963977in}{5.399011in}}%
\pgfpathcurveto{\pgfqpoint{6.971791in}{5.391197in}}{\pgfqpoint{6.982390in}{5.386807in}}{\pgfqpoint{6.993440in}{5.386807in}}%
\pgfpathclose%
\pgfusepath{stroke,fill}%
\end{pgfscope}%
\begin{pgfscope}%
\pgfpathrectangle{\pgfqpoint{0.570343in}{0.331635in}}{\pgfqpoint{9.300000in}{7.700000in}}%
\pgfusepath{clip}%
\pgfsetbuttcap%
\pgfsetroundjoin%
\definecolor{currentfill}{rgb}{0.631373,0.788235,0.956863}%
\pgfsetfillcolor{currentfill}%
\pgfsetlinewidth{0.481800pt}%
\definecolor{currentstroke}{rgb}{1.000000,1.000000,1.000000}%
\pgfsetstrokecolor{currentstroke}%
\pgfsetdash{}{0pt}%
\pgfpathmoveto{\pgfqpoint{7.514448in}{1.636778in}}%
\pgfpathcurveto{\pgfqpoint{7.525498in}{1.636778in}}{\pgfqpoint{7.536097in}{1.641169in}}{\pgfqpoint{7.543910in}{1.648982in}}%
\pgfpathcurveto{\pgfqpoint{7.551724in}{1.656796in}}{\pgfqpoint{7.556114in}{1.667395in}}{\pgfqpoint{7.556114in}{1.678445in}}%
\pgfpathcurveto{\pgfqpoint{7.556114in}{1.689495in}}{\pgfqpoint{7.551724in}{1.700094in}}{\pgfqpoint{7.543910in}{1.707908in}}%
\pgfpathcurveto{\pgfqpoint{7.536097in}{1.715721in}}{\pgfqpoint{7.525498in}{1.720112in}}{\pgfqpoint{7.514448in}{1.720112in}}%
\pgfpathcurveto{\pgfqpoint{7.503397in}{1.720112in}}{\pgfqpoint{7.492798in}{1.715721in}}{\pgfqpoint{7.484985in}{1.707908in}}%
\pgfpathcurveto{\pgfqpoint{7.477171in}{1.700094in}}{\pgfqpoint{7.472781in}{1.689495in}}{\pgfqpoint{7.472781in}{1.678445in}}%
\pgfpathcurveto{\pgfqpoint{7.472781in}{1.667395in}}{\pgfqpoint{7.477171in}{1.656796in}}{\pgfqpoint{7.484985in}{1.648982in}}%
\pgfpathcurveto{\pgfqpoint{7.492798in}{1.641169in}}{\pgfqpoint{7.503397in}{1.636778in}}{\pgfqpoint{7.514448in}{1.636778in}}%
\pgfpathclose%
\pgfusepath{stroke,fill}%
\end{pgfscope}%
\begin{pgfscope}%
\pgfpathrectangle{\pgfqpoint{0.570343in}{0.331635in}}{\pgfqpoint{9.300000in}{7.700000in}}%
\pgfusepath{clip}%
\pgfsetbuttcap%
\pgfsetroundjoin%
\definecolor{currentfill}{rgb}{0.631373,0.788235,0.956863}%
\pgfsetfillcolor{currentfill}%
\pgfsetlinewidth{0.481800pt}%
\definecolor{currentstroke}{rgb}{1.000000,1.000000,1.000000}%
\pgfsetstrokecolor{currentstroke}%
\pgfsetdash{}{0pt}%
\pgfpathmoveto{\pgfqpoint{5.023692in}{7.139318in}}%
\pgfpathcurveto{\pgfqpoint{5.034742in}{7.139318in}}{\pgfqpoint{5.045342in}{7.143708in}}{\pgfqpoint{5.053155in}{7.151522in}}%
\pgfpathcurveto{\pgfqpoint{5.060969in}{7.159335in}}{\pgfqpoint{5.065359in}{7.169934in}}{\pgfqpoint{5.065359in}{7.180984in}}%
\pgfpathcurveto{\pgfqpoint{5.065359in}{7.192035in}}{\pgfqpoint{5.060969in}{7.202634in}}{\pgfqpoint{5.053155in}{7.210447in}}%
\pgfpathcurveto{\pgfqpoint{5.045342in}{7.218261in}}{\pgfqpoint{5.034742in}{7.222651in}}{\pgfqpoint{5.023692in}{7.222651in}}%
\pgfpathcurveto{\pgfqpoint{5.012642in}{7.222651in}}{\pgfqpoint{5.002043in}{7.218261in}}{\pgfqpoint{4.994230in}{7.210447in}}%
\pgfpathcurveto{\pgfqpoint{4.986416in}{7.202634in}}{\pgfqpoint{4.982026in}{7.192035in}}{\pgfqpoint{4.982026in}{7.180984in}}%
\pgfpathcurveto{\pgfqpoint{4.982026in}{7.169934in}}{\pgfqpoint{4.986416in}{7.159335in}}{\pgfqpoint{4.994230in}{7.151522in}}%
\pgfpathcurveto{\pgfqpoint{5.002043in}{7.143708in}}{\pgfqpoint{5.012642in}{7.139318in}}{\pgfqpoint{5.023692in}{7.139318in}}%
\pgfpathclose%
\pgfusepath{stroke,fill}%
\end{pgfscope}%
\begin{pgfscope}%
\pgfpathrectangle{\pgfqpoint{0.570343in}{0.331635in}}{\pgfqpoint{9.300000in}{7.700000in}}%
\pgfusepath{clip}%
\pgfsetbuttcap%
\pgfsetroundjoin%
\definecolor{currentfill}{rgb}{0.631373,0.788235,0.956863}%
\pgfsetfillcolor{currentfill}%
\pgfsetlinewidth{0.481800pt}%
\definecolor{currentstroke}{rgb}{1.000000,1.000000,1.000000}%
\pgfsetstrokecolor{currentstroke}%
\pgfsetdash{}{0pt}%
\pgfpathmoveto{\pgfqpoint{3.832172in}{4.260196in}}%
\pgfpathcurveto{\pgfqpoint{3.843222in}{4.260196in}}{\pgfqpoint{3.853821in}{4.264586in}}{\pgfqpoint{3.861635in}{4.272399in}}%
\pgfpathcurveto{\pgfqpoint{3.869449in}{4.280213in}}{\pgfqpoint{3.873839in}{4.290812in}}{\pgfqpoint{3.873839in}{4.301862in}}%
\pgfpathcurveto{\pgfqpoint{3.873839in}{4.312912in}}{\pgfqpoint{3.869449in}{4.323511in}}{\pgfqpoint{3.861635in}{4.331325in}}%
\pgfpathcurveto{\pgfqpoint{3.853821in}{4.339139in}}{\pgfqpoint{3.843222in}{4.343529in}}{\pgfqpoint{3.832172in}{4.343529in}}%
\pgfpathcurveto{\pgfqpoint{3.821122in}{4.343529in}}{\pgfqpoint{3.810523in}{4.339139in}}{\pgfqpoint{3.802709in}{4.331325in}}%
\pgfpathcurveto{\pgfqpoint{3.794896in}{4.323511in}}{\pgfqpoint{3.790506in}{4.312912in}}{\pgfqpoint{3.790506in}{4.301862in}}%
\pgfpathcurveto{\pgfqpoint{3.790506in}{4.290812in}}{\pgfqpoint{3.794896in}{4.280213in}}{\pgfqpoint{3.802709in}{4.272399in}}%
\pgfpathcurveto{\pgfqpoint{3.810523in}{4.264586in}}{\pgfqpoint{3.821122in}{4.260196in}}{\pgfqpoint{3.832172in}{4.260196in}}%
\pgfpathclose%
\pgfusepath{stroke,fill}%
\end{pgfscope}%
\begin{pgfscope}%
\pgfpathrectangle{\pgfqpoint{0.570343in}{0.331635in}}{\pgfqpoint{9.300000in}{7.700000in}}%
\pgfusepath{clip}%
\pgfsetbuttcap%
\pgfsetroundjoin%
\definecolor{currentfill}{rgb}{0.631373,0.788235,0.956863}%
\pgfsetfillcolor{currentfill}%
\pgfsetlinewidth{0.481800pt}%
\definecolor{currentstroke}{rgb}{1.000000,1.000000,1.000000}%
\pgfsetstrokecolor{currentstroke}%
\pgfsetdash{}{0pt}%
\pgfpathmoveto{\pgfqpoint{5.168249in}{4.633667in}}%
\pgfpathcurveto{\pgfqpoint{5.179299in}{4.633667in}}{\pgfqpoint{5.189898in}{4.638057in}}{\pgfqpoint{5.197712in}{4.645870in}}%
\pgfpathcurveto{\pgfqpoint{5.205525in}{4.653684in}}{\pgfqpoint{5.209915in}{4.664283in}}{\pgfqpoint{5.209915in}{4.675333in}}%
\pgfpathcurveto{\pgfqpoint{5.209915in}{4.686383in}}{\pgfqpoint{5.205525in}{4.696982in}}{\pgfqpoint{5.197712in}{4.704796in}}%
\pgfpathcurveto{\pgfqpoint{5.189898in}{4.712610in}}{\pgfqpoint{5.179299in}{4.717000in}}{\pgfqpoint{5.168249in}{4.717000in}}%
\pgfpathcurveto{\pgfqpoint{5.157199in}{4.717000in}}{\pgfqpoint{5.146600in}{4.712610in}}{\pgfqpoint{5.138786in}{4.704796in}}%
\pgfpathcurveto{\pgfqpoint{5.130972in}{4.696982in}}{\pgfqpoint{5.126582in}{4.686383in}}{\pgfqpoint{5.126582in}{4.675333in}}%
\pgfpathcurveto{\pgfqpoint{5.126582in}{4.664283in}}{\pgfqpoint{5.130972in}{4.653684in}}{\pgfqpoint{5.138786in}{4.645870in}}%
\pgfpathcurveto{\pgfqpoint{5.146600in}{4.638057in}}{\pgfqpoint{5.157199in}{4.633667in}}{\pgfqpoint{5.168249in}{4.633667in}}%
\pgfpathclose%
\pgfusepath{stroke,fill}%
\end{pgfscope}%
\begin{pgfscope}%
\pgfpathrectangle{\pgfqpoint{0.570343in}{0.331635in}}{\pgfqpoint{9.300000in}{7.700000in}}%
\pgfusepath{clip}%
\pgfsetbuttcap%
\pgfsetroundjoin%
\definecolor{currentfill}{rgb}{0.631373,0.788235,0.956863}%
\pgfsetfillcolor{currentfill}%
\pgfsetlinewidth{0.481800pt}%
\definecolor{currentstroke}{rgb}{1.000000,1.000000,1.000000}%
\pgfsetstrokecolor{currentstroke}%
\pgfsetdash{}{0pt}%
\pgfpathmoveto{\pgfqpoint{4.194348in}{1.372368in}}%
\pgfpathcurveto{\pgfqpoint{4.205398in}{1.372368in}}{\pgfqpoint{4.215997in}{1.376758in}}{\pgfqpoint{4.223811in}{1.384572in}}%
\pgfpathcurveto{\pgfqpoint{4.231624in}{1.392385in}}{\pgfqpoint{4.236015in}{1.402984in}}{\pgfqpoint{4.236015in}{1.414034in}}%
\pgfpathcurveto{\pgfqpoint{4.236015in}{1.425084in}}{\pgfqpoint{4.231624in}{1.435684in}}{\pgfqpoint{4.223811in}{1.443497in}}%
\pgfpathcurveto{\pgfqpoint{4.215997in}{1.451311in}}{\pgfqpoint{4.205398in}{1.455701in}}{\pgfqpoint{4.194348in}{1.455701in}}%
\pgfpathcurveto{\pgfqpoint{4.183298in}{1.455701in}}{\pgfqpoint{4.172699in}{1.451311in}}{\pgfqpoint{4.164885in}{1.443497in}}%
\pgfpathcurveto{\pgfqpoint{4.157072in}{1.435684in}}{\pgfqpoint{4.152681in}{1.425084in}}{\pgfqpoint{4.152681in}{1.414034in}}%
\pgfpathcurveto{\pgfqpoint{4.152681in}{1.402984in}}{\pgfqpoint{4.157072in}{1.392385in}}{\pgfqpoint{4.164885in}{1.384572in}}%
\pgfpathcurveto{\pgfqpoint{4.172699in}{1.376758in}}{\pgfqpoint{4.183298in}{1.372368in}}{\pgfqpoint{4.194348in}{1.372368in}}%
\pgfpathclose%
\pgfusepath{stroke,fill}%
\end{pgfscope}%
\begin{pgfscope}%
\pgfpathrectangle{\pgfqpoint{0.570343in}{0.331635in}}{\pgfqpoint{9.300000in}{7.700000in}}%
\pgfusepath{clip}%
\pgfsetbuttcap%
\pgfsetroundjoin%
\definecolor{currentfill}{rgb}{0.631373,0.788235,0.956863}%
\pgfsetfillcolor{currentfill}%
\pgfsetlinewidth{0.481800pt}%
\definecolor{currentstroke}{rgb}{1.000000,1.000000,1.000000}%
\pgfsetstrokecolor{currentstroke}%
\pgfsetdash{}{0pt}%
\pgfpathmoveto{\pgfqpoint{2.750613in}{5.800171in}}%
\pgfpathcurveto{\pgfqpoint{2.761663in}{5.800171in}}{\pgfqpoint{2.772263in}{5.804561in}}{\pgfqpoint{2.780076in}{5.812374in}}%
\pgfpathcurveto{\pgfqpoint{2.787890in}{5.820188in}}{\pgfqpoint{2.792280in}{5.830787in}}{\pgfqpoint{2.792280in}{5.841837in}}%
\pgfpathcurveto{\pgfqpoint{2.792280in}{5.852887in}}{\pgfqpoint{2.787890in}{5.863486in}}{\pgfqpoint{2.780076in}{5.871300in}}%
\pgfpathcurveto{\pgfqpoint{2.772263in}{5.879114in}}{\pgfqpoint{2.761663in}{5.883504in}}{\pgfqpoint{2.750613in}{5.883504in}}%
\pgfpathcurveto{\pgfqpoint{2.739563in}{5.883504in}}{\pgfqpoint{2.728964in}{5.879114in}}{\pgfqpoint{2.721151in}{5.871300in}}%
\pgfpathcurveto{\pgfqpoint{2.713337in}{5.863486in}}{\pgfqpoint{2.708947in}{5.852887in}}{\pgfqpoint{2.708947in}{5.841837in}}%
\pgfpathcurveto{\pgfqpoint{2.708947in}{5.830787in}}{\pgfqpoint{2.713337in}{5.820188in}}{\pgfqpoint{2.721151in}{5.812374in}}%
\pgfpathcurveto{\pgfqpoint{2.728964in}{5.804561in}}{\pgfqpoint{2.739563in}{5.800171in}}{\pgfqpoint{2.750613in}{5.800171in}}%
\pgfpathclose%
\pgfusepath{stroke,fill}%
\end{pgfscope}%
\begin{pgfscope}%
\pgfpathrectangle{\pgfqpoint{0.570343in}{0.331635in}}{\pgfqpoint{9.300000in}{7.700000in}}%
\pgfusepath{clip}%
\pgfsetbuttcap%
\pgfsetroundjoin%
\definecolor{currentfill}{rgb}{0.631373,0.788235,0.956863}%
\pgfsetfillcolor{currentfill}%
\pgfsetlinewidth{0.481800pt}%
\definecolor{currentstroke}{rgb}{1.000000,1.000000,1.000000}%
\pgfsetstrokecolor{currentstroke}%
\pgfsetdash{}{0pt}%
\pgfpathmoveto{\pgfqpoint{3.118617in}{2.356278in}}%
\pgfpathcurveto{\pgfqpoint{3.129667in}{2.356278in}}{\pgfqpoint{3.140266in}{2.360668in}}{\pgfqpoint{3.148080in}{2.368482in}}%
\pgfpathcurveto{\pgfqpoint{3.155893in}{2.376295in}}{\pgfqpoint{3.160284in}{2.386894in}}{\pgfqpoint{3.160284in}{2.397944in}}%
\pgfpathcurveto{\pgfqpoint{3.160284in}{2.408994in}}{\pgfqpoint{3.155893in}{2.419593in}}{\pgfqpoint{3.148080in}{2.427407in}}%
\pgfpathcurveto{\pgfqpoint{3.140266in}{2.435221in}}{\pgfqpoint{3.129667in}{2.439611in}}{\pgfqpoint{3.118617in}{2.439611in}}%
\pgfpathcurveto{\pgfqpoint{3.107567in}{2.439611in}}{\pgfqpoint{3.096968in}{2.435221in}}{\pgfqpoint{3.089154in}{2.427407in}}%
\pgfpathcurveto{\pgfqpoint{3.081341in}{2.419593in}}{\pgfqpoint{3.076950in}{2.408994in}}{\pgfqpoint{3.076950in}{2.397944in}}%
\pgfpathcurveto{\pgfqpoint{3.076950in}{2.386894in}}{\pgfqpoint{3.081341in}{2.376295in}}{\pgfqpoint{3.089154in}{2.368482in}}%
\pgfpathcurveto{\pgfqpoint{3.096968in}{2.360668in}}{\pgfqpoint{3.107567in}{2.356278in}}{\pgfqpoint{3.118617in}{2.356278in}}%
\pgfpathclose%
\pgfusepath{stroke,fill}%
\end{pgfscope}%
\begin{pgfscope}%
\pgfpathrectangle{\pgfqpoint{0.570343in}{0.331635in}}{\pgfqpoint{9.300000in}{7.700000in}}%
\pgfusepath{clip}%
\pgfsetbuttcap%
\pgfsetroundjoin%
\definecolor{currentfill}{rgb}{0.631373,0.788235,0.956863}%
\pgfsetfillcolor{currentfill}%
\pgfsetlinewidth{0.481800pt}%
\definecolor{currentstroke}{rgb}{1.000000,1.000000,1.000000}%
\pgfsetstrokecolor{currentstroke}%
\pgfsetdash{}{0pt}%
\pgfpathmoveto{\pgfqpoint{5.057402in}{1.848505in}}%
\pgfpathcurveto{\pgfqpoint{5.068453in}{1.848505in}}{\pgfqpoint{5.079052in}{1.852895in}}{\pgfqpoint{5.086865in}{1.860709in}}%
\pgfpathcurveto{\pgfqpoint{5.094679in}{1.868522in}}{\pgfqpoint{5.099069in}{1.879121in}}{\pgfqpoint{5.099069in}{1.890172in}}%
\pgfpathcurveto{\pgfqpoint{5.099069in}{1.901222in}}{\pgfqpoint{5.094679in}{1.911821in}}{\pgfqpoint{5.086865in}{1.919634in}}%
\pgfpathcurveto{\pgfqpoint{5.079052in}{1.927448in}}{\pgfqpoint{5.068453in}{1.931838in}}{\pgfqpoint{5.057402in}{1.931838in}}%
\pgfpathcurveto{\pgfqpoint{5.046352in}{1.931838in}}{\pgfqpoint{5.035753in}{1.927448in}}{\pgfqpoint{5.027940in}{1.919634in}}%
\pgfpathcurveto{\pgfqpoint{5.020126in}{1.911821in}}{\pgfqpoint{5.015736in}{1.901222in}}{\pgfqpoint{5.015736in}{1.890172in}}%
\pgfpathcurveto{\pgfqpoint{5.015736in}{1.879121in}}{\pgfqpoint{5.020126in}{1.868522in}}{\pgfqpoint{5.027940in}{1.860709in}}%
\pgfpathcurveto{\pgfqpoint{5.035753in}{1.852895in}}{\pgfqpoint{5.046352in}{1.848505in}}{\pgfqpoint{5.057402in}{1.848505in}}%
\pgfpathclose%
\pgfusepath{stroke,fill}%
\end{pgfscope}%
\begin{pgfscope}%
\pgfpathrectangle{\pgfqpoint{0.570343in}{0.331635in}}{\pgfqpoint{9.300000in}{7.700000in}}%
\pgfusepath{clip}%
\pgfsetbuttcap%
\pgfsetroundjoin%
\definecolor{currentfill}{rgb}{0.631373,0.788235,0.956863}%
\pgfsetfillcolor{currentfill}%
\pgfsetlinewidth{0.481800pt}%
\definecolor{currentstroke}{rgb}{1.000000,1.000000,1.000000}%
\pgfsetstrokecolor{currentstroke}%
\pgfsetdash{}{0pt}%
\pgfpathmoveto{\pgfqpoint{5.910435in}{2.316238in}}%
\pgfpathcurveto{\pgfqpoint{5.921485in}{2.316238in}}{\pgfqpoint{5.932084in}{2.320628in}}{\pgfqpoint{5.939898in}{2.328441in}}%
\pgfpathcurveto{\pgfqpoint{5.947712in}{2.336255in}}{\pgfqpoint{5.952102in}{2.346854in}}{\pgfqpoint{5.952102in}{2.357904in}}%
\pgfpathcurveto{\pgfqpoint{5.952102in}{2.368954in}}{\pgfqpoint{5.947712in}{2.379553in}}{\pgfqpoint{5.939898in}{2.387367in}}%
\pgfpathcurveto{\pgfqpoint{5.932084in}{2.395181in}}{\pgfqpoint{5.921485in}{2.399571in}}{\pgfqpoint{5.910435in}{2.399571in}}%
\pgfpathcurveto{\pgfqpoint{5.899385in}{2.399571in}}{\pgfqpoint{5.888786in}{2.395181in}}{\pgfqpoint{5.880972in}{2.387367in}}%
\pgfpathcurveto{\pgfqpoint{5.873159in}{2.379553in}}{\pgfqpoint{5.868769in}{2.368954in}}{\pgfqpoint{5.868769in}{2.357904in}}%
\pgfpathcurveto{\pgfqpoint{5.868769in}{2.346854in}}{\pgfqpoint{5.873159in}{2.336255in}}{\pgfqpoint{5.880972in}{2.328441in}}%
\pgfpathcurveto{\pgfqpoint{5.888786in}{2.320628in}}{\pgfqpoint{5.899385in}{2.316238in}}{\pgfqpoint{5.910435in}{2.316238in}}%
\pgfpathclose%
\pgfusepath{stroke,fill}%
\end{pgfscope}%
\begin{pgfscope}%
\pgfpathrectangle{\pgfqpoint{0.570343in}{0.331635in}}{\pgfqpoint{9.300000in}{7.700000in}}%
\pgfusepath{clip}%
\pgfsetbuttcap%
\pgfsetroundjoin%
\definecolor{currentfill}{rgb}{0.631373,0.788235,0.956863}%
\pgfsetfillcolor{currentfill}%
\pgfsetlinewidth{0.481800pt}%
\definecolor{currentstroke}{rgb}{1.000000,1.000000,1.000000}%
\pgfsetstrokecolor{currentstroke}%
\pgfsetdash{}{0pt}%
\pgfpathmoveto{\pgfqpoint{5.846427in}{5.254826in}}%
\pgfpathcurveto{\pgfqpoint{5.857477in}{5.254826in}}{\pgfqpoint{5.868076in}{5.259217in}}{\pgfqpoint{5.875890in}{5.267030in}}%
\pgfpathcurveto{\pgfqpoint{5.883704in}{5.274844in}}{\pgfqpoint{5.888094in}{5.285443in}}{\pgfqpoint{5.888094in}{5.296493in}}%
\pgfpathcurveto{\pgfqpoint{5.888094in}{5.307543in}}{\pgfqpoint{5.883704in}{5.318142in}}{\pgfqpoint{5.875890in}{5.325956in}}%
\pgfpathcurveto{\pgfqpoint{5.868076in}{5.333769in}}{\pgfqpoint{5.857477in}{5.338160in}}{\pgfqpoint{5.846427in}{5.338160in}}%
\pgfpathcurveto{\pgfqpoint{5.835377in}{5.338160in}}{\pgfqpoint{5.824778in}{5.333769in}}{\pgfqpoint{5.816964in}{5.325956in}}%
\pgfpathcurveto{\pgfqpoint{5.809151in}{5.318142in}}{\pgfqpoint{5.804761in}{5.307543in}}{\pgfqpoint{5.804761in}{5.296493in}}%
\pgfpathcurveto{\pgfqpoint{5.804761in}{5.285443in}}{\pgfqpoint{5.809151in}{5.274844in}}{\pgfqpoint{5.816964in}{5.267030in}}%
\pgfpathcurveto{\pgfqpoint{5.824778in}{5.259217in}}{\pgfqpoint{5.835377in}{5.254826in}}{\pgfqpoint{5.846427in}{5.254826in}}%
\pgfpathclose%
\pgfusepath{stroke,fill}%
\end{pgfscope}%
\begin{pgfscope}%
\pgfpathrectangle{\pgfqpoint{0.570343in}{0.331635in}}{\pgfqpoint{9.300000in}{7.700000in}}%
\pgfusepath{clip}%
\pgfsetbuttcap%
\pgfsetroundjoin%
\definecolor{currentfill}{rgb}{0.631373,0.788235,0.956863}%
\pgfsetfillcolor{currentfill}%
\pgfsetlinewidth{0.481800pt}%
\definecolor{currentstroke}{rgb}{1.000000,1.000000,1.000000}%
\pgfsetstrokecolor{currentstroke}%
\pgfsetdash{}{0pt}%
\pgfpathmoveto{\pgfqpoint{4.284170in}{4.749655in}}%
\pgfpathcurveto{\pgfqpoint{4.295220in}{4.749655in}}{\pgfqpoint{4.305820in}{4.754045in}}{\pgfqpoint{4.313633in}{4.761859in}}%
\pgfpathcurveto{\pgfqpoint{4.321447in}{4.769672in}}{\pgfqpoint{4.325837in}{4.780271in}}{\pgfqpoint{4.325837in}{4.791321in}}%
\pgfpathcurveto{\pgfqpoint{4.325837in}{4.802372in}}{\pgfqpoint{4.321447in}{4.812971in}}{\pgfqpoint{4.313633in}{4.820784in}}%
\pgfpathcurveto{\pgfqpoint{4.305820in}{4.828598in}}{\pgfqpoint{4.295220in}{4.832988in}}{\pgfqpoint{4.284170in}{4.832988in}}%
\pgfpathcurveto{\pgfqpoint{4.273120in}{4.832988in}}{\pgfqpoint{4.262521in}{4.828598in}}{\pgfqpoint{4.254708in}{4.820784in}}%
\pgfpathcurveto{\pgfqpoint{4.246894in}{4.812971in}}{\pgfqpoint{4.242504in}{4.802372in}}{\pgfqpoint{4.242504in}{4.791321in}}%
\pgfpathcurveto{\pgfqpoint{4.242504in}{4.780271in}}{\pgfqpoint{4.246894in}{4.769672in}}{\pgfqpoint{4.254708in}{4.761859in}}%
\pgfpathcurveto{\pgfqpoint{4.262521in}{4.754045in}}{\pgfqpoint{4.273120in}{4.749655in}}{\pgfqpoint{4.284170in}{4.749655in}}%
\pgfpathclose%
\pgfusepath{stroke,fill}%
\end{pgfscope}%
\begin{pgfscope}%
\pgfpathrectangle{\pgfqpoint{0.570343in}{0.331635in}}{\pgfqpoint{9.300000in}{7.700000in}}%
\pgfusepath{clip}%
\pgfsetbuttcap%
\pgfsetroundjoin%
\definecolor{currentfill}{rgb}{0.631373,0.788235,0.956863}%
\pgfsetfillcolor{currentfill}%
\pgfsetlinewidth{0.481800pt}%
\definecolor{currentstroke}{rgb}{1.000000,1.000000,1.000000}%
\pgfsetstrokecolor{currentstroke}%
\pgfsetdash{}{0pt}%
\pgfpathmoveto{\pgfqpoint{1.673868in}{2.866832in}}%
\pgfpathcurveto{\pgfqpoint{1.684918in}{2.866832in}}{\pgfqpoint{1.695517in}{2.871222in}}{\pgfqpoint{1.703331in}{2.879035in}}%
\pgfpathcurveto{\pgfqpoint{1.711144in}{2.886849in}}{\pgfqpoint{1.715535in}{2.897448in}}{\pgfqpoint{1.715535in}{2.908498in}}%
\pgfpathcurveto{\pgfqpoint{1.715535in}{2.919548in}}{\pgfqpoint{1.711144in}{2.930147in}}{\pgfqpoint{1.703331in}{2.937961in}}%
\pgfpathcurveto{\pgfqpoint{1.695517in}{2.945775in}}{\pgfqpoint{1.684918in}{2.950165in}}{\pgfqpoint{1.673868in}{2.950165in}}%
\pgfpathcurveto{\pgfqpoint{1.662818in}{2.950165in}}{\pgfqpoint{1.652219in}{2.945775in}}{\pgfqpoint{1.644405in}{2.937961in}}%
\pgfpathcurveto{\pgfqpoint{1.636592in}{2.930147in}}{\pgfqpoint{1.632201in}{2.919548in}}{\pgfqpoint{1.632201in}{2.908498in}}%
\pgfpathcurveto{\pgfqpoint{1.632201in}{2.897448in}}{\pgfqpoint{1.636592in}{2.886849in}}{\pgfqpoint{1.644405in}{2.879035in}}%
\pgfpathcurveto{\pgfqpoint{1.652219in}{2.871222in}}{\pgfqpoint{1.662818in}{2.866832in}}{\pgfqpoint{1.673868in}{2.866832in}}%
\pgfpathclose%
\pgfusepath{stroke,fill}%
\end{pgfscope}%
\begin{pgfscope}%
\pgfpathrectangle{\pgfqpoint{0.570343in}{0.331635in}}{\pgfqpoint{9.300000in}{7.700000in}}%
\pgfusepath{clip}%
\pgfsetbuttcap%
\pgfsetroundjoin%
\definecolor{currentfill}{rgb}{0.631373,0.788235,0.956863}%
\pgfsetfillcolor{currentfill}%
\pgfsetlinewidth{0.481800pt}%
\definecolor{currentstroke}{rgb}{1.000000,1.000000,1.000000}%
\pgfsetstrokecolor{currentstroke}%
\pgfsetdash{}{0pt}%
\pgfpathmoveto{\pgfqpoint{4.453376in}{3.392004in}}%
\pgfpathcurveto{\pgfqpoint{4.464426in}{3.392004in}}{\pgfqpoint{4.475025in}{3.396394in}}{\pgfqpoint{4.482839in}{3.404208in}}%
\pgfpathcurveto{\pgfqpoint{4.490652in}{3.412022in}}{\pgfqpoint{4.495042in}{3.422621in}}{\pgfqpoint{4.495042in}{3.433671in}}%
\pgfpathcurveto{\pgfqpoint{4.495042in}{3.444721in}}{\pgfqpoint{4.490652in}{3.455320in}}{\pgfqpoint{4.482839in}{3.463134in}}%
\pgfpathcurveto{\pgfqpoint{4.475025in}{3.470947in}}{\pgfqpoint{4.464426in}{3.475337in}}{\pgfqpoint{4.453376in}{3.475337in}}%
\pgfpathcurveto{\pgfqpoint{4.442326in}{3.475337in}}{\pgfqpoint{4.431727in}{3.470947in}}{\pgfqpoint{4.423913in}{3.463134in}}%
\pgfpathcurveto{\pgfqpoint{4.416099in}{3.455320in}}{\pgfqpoint{4.411709in}{3.444721in}}{\pgfqpoint{4.411709in}{3.433671in}}%
\pgfpathcurveto{\pgfqpoint{4.411709in}{3.422621in}}{\pgfqpoint{4.416099in}{3.412022in}}{\pgfqpoint{4.423913in}{3.404208in}}%
\pgfpathcurveto{\pgfqpoint{4.431727in}{3.396394in}}{\pgfqpoint{4.442326in}{3.392004in}}{\pgfqpoint{4.453376in}{3.392004in}}%
\pgfpathclose%
\pgfusepath{stroke,fill}%
\end{pgfscope}%
\begin{pgfscope}%
\pgfpathrectangle{\pgfqpoint{0.570343in}{0.331635in}}{\pgfqpoint{9.300000in}{7.700000in}}%
\pgfusepath{clip}%
\pgfsetbuttcap%
\pgfsetroundjoin%
\definecolor{currentfill}{rgb}{0.631373,0.788235,0.956863}%
\pgfsetfillcolor{currentfill}%
\pgfsetlinewidth{0.481800pt}%
\definecolor{currentstroke}{rgb}{1.000000,1.000000,1.000000}%
\pgfsetstrokecolor{currentstroke}%
\pgfsetdash{}{0pt}%
\pgfpathmoveto{\pgfqpoint{5.511077in}{3.346155in}}%
\pgfpathcurveto{\pgfqpoint{5.522127in}{3.346155in}}{\pgfqpoint{5.532726in}{3.350545in}}{\pgfqpoint{5.540540in}{3.358359in}}%
\pgfpathcurveto{\pgfqpoint{5.548353in}{3.366173in}}{\pgfqpoint{5.552744in}{3.376772in}}{\pgfqpoint{5.552744in}{3.387822in}}%
\pgfpathcurveto{\pgfqpoint{5.552744in}{3.398872in}}{\pgfqpoint{5.548353in}{3.409471in}}{\pgfqpoint{5.540540in}{3.417284in}}%
\pgfpathcurveto{\pgfqpoint{5.532726in}{3.425098in}}{\pgfqpoint{5.522127in}{3.429488in}}{\pgfqpoint{5.511077in}{3.429488in}}%
\pgfpathcurveto{\pgfqpoint{5.500027in}{3.429488in}}{\pgfqpoint{5.489428in}{3.425098in}}{\pgfqpoint{5.481614in}{3.417284in}}%
\pgfpathcurveto{\pgfqpoint{5.473801in}{3.409471in}}{\pgfqpoint{5.469410in}{3.398872in}}{\pgfqpoint{5.469410in}{3.387822in}}%
\pgfpathcurveto{\pgfqpoint{5.469410in}{3.376772in}}{\pgfqpoint{5.473801in}{3.366173in}}{\pgfqpoint{5.481614in}{3.358359in}}%
\pgfpathcurveto{\pgfqpoint{5.489428in}{3.350545in}}{\pgfqpoint{5.500027in}{3.346155in}}{\pgfqpoint{5.511077in}{3.346155in}}%
\pgfpathclose%
\pgfusepath{stroke,fill}%
\end{pgfscope}%
\begin{pgfscope}%
\pgfpathrectangle{\pgfqpoint{0.570343in}{0.331635in}}{\pgfqpoint{9.300000in}{7.700000in}}%
\pgfusepath{clip}%
\pgfsetbuttcap%
\pgfsetroundjoin%
\definecolor{currentfill}{rgb}{0.631373,0.788235,0.956863}%
\pgfsetfillcolor{currentfill}%
\pgfsetlinewidth{0.481800pt}%
\definecolor{currentstroke}{rgb}{1.000000,1.000000,1.000000}%
\pgfsetstrokecolor{currentstroke}%
\pgfsetdash{}{0pt}%
\pgfpathmoveto{\pgfqpoint{6.465570in}{3.537469in}}%
\pgfpathcurveto{\pgfqpoint{6.476620in}{3.537469in}}{\pgfqpoint{6.487219in}{3.541859in}}{\pgfqpoint{6.495033in}{3.549673in}}%
\pgfpathcurveto{\pgfqpoint{6.502846in}{3.557486in}}{\pgfqpoint{6.507237in}{3.568085in}}{\pgfqpoint{6.507237in}{3.579135in}}%
\pgfpathcurveto{\pgfqpoint{6.507237in}{3.590186in}}{\pgfqpoint{6.502846in}{3.600785in}}{\pgfqpoint{6.495033in}{3.608598in}}%
\pgfpathcurveto{\pgfqpoint{6.487219in}{3.616412in}}{\pgfqpoint{6.476620in}{3.620802in}}{\pgfqpoint{6.465570in}{3.620802in}}%
\pgfpathcurveto{\pgfqpoint{6.454520in}{3.620802in}}{\pgfqpoint{6.443921in}{3.616412in}}{\pgfqpoint{6.436107in}{3.608598in}}%
\pgfpathcurveto{\pgfqpoint{6.428294in}{3.600785in}}{\pgfqpoint{6.423903in}{3.590186in}}{\pgfqpoint{6.423903in}{3.579135in}}%
\pgfpathcurveto{\pgfqpoint{6.423903in}{3.568085in}}{\pgfqpoint{6.428294in}{3.557486in}}{\pgfqpoint{6.436107in}{3.549673in}}%
\pgfpathcurveto{\pgfqpoint{6.443921in}{3.541859in}}{\pgfqpoint{6.454520in}{3.537469in}}{\pgfqpoint{6.465570in}{3.537469in}}%
\pgfpathclose%
\pgfusepath{stroke,fill}%
\end{pgfscope}%
\begin{pgfscope}%
\pgfpathrectangle{\pgfqpoint{0.570343in}{0.331635in}}{\pgfqpoint{9.300000in}{7.700000in}}%
\pgfusepath{clip}%
\pgfsetbuttcap%
\pgfsetroundjoin%
\definecolor{currentfill}{rgb}{0.631373,0.788235,0.956863}%
\pgfsetfillcolor{currentfill}%
\pgfsetlinewidth{0.481800pt}%
\definecolor{currentstroke}{rgb}{1.000000,1.000000,1.000000}%
\pgfsetstrokecolor{currentstroke}%
\pgfsetdash{}{0pt}%
\pgfpathmoveto{\pgfqpoint{5.213120in}{0.639968in}}%
\pgfpathcurveto{\pgfqpoint{5.224170in}{0.639968in}}{\pgfqpoint{5.234769in}{0.644359in}}{\pgfqpoint{5.242583in}{0.652172in}}%
\pgfpathcurveto{\pgfqpoint{5.250396in}{0.659986in}}{\pgfqpoint{5.254787in}{0.670585in}}{\pgfqpoint{5.254787in}{0.681635in}}%
\pgfpathcurveto{\pgfqpoint{5.254787in}{0.692685in}}{\pgfqpoint{5.250396in}{0.703284in}}{\pgfqpoint{5.242583in}{0.711098in}}%
\pgfpathcurveto{\pgfqpoint{5.234769in}{0.718911in}}{\pgfqpoint{5.224170in}{0.723302in}}{\pgfqpoint{5.213120in}{0.723302in}}%
\pgfpathcurveto{\pgfqpoint{5.202070in}{0.723302in}}{\pgfqpoint{5.191471in}{0.718911in}}{\pgfqpoint{5.183657in}{0.711098in}}%
\pgfpathcurveto{\pgfqpoint{5.175844in}{0.703284in}}{\pgfqpoint{5.171453in}{0.692685in}}{\pgfqpoint{5.171453in}{0.681635in}}%
\pgfpathcurveto{\pgfqpoint{5.171453in}{0.670585in}}{\pgfqpoint{5.175844in}{0.659986in}}{\pgfqpoint{5.183657in}{0.652172in}}%
\pgfpathcurveto{\pgfqpoint{5.191471in}{0.644359in}}{\pgfqpoint{5.202070in}{0.639968in}}{\pgfqpoint{5.213120in}{0.639968in}}%
\pgfpathclose%
\pgfusepath{stroke,fill}%
\end{pgfscope}%
\begin{pgfscope}%
\pgfpathrectangle{\pgfqpoint{0.570343in}{0.331635in}}{\pgfqpoint{9.300000in}{7.700000in}}%
\pgfusepath{clip}%
\pgfsetbuttcap%
\pgfsetroundjoin%
\definecolor{currentfill}{rgb}{0.631373,0.788235,0.956863}%
\pgfsetfillcolor{currentfill}%
\pgfsetlinewidth{0.481800pt}%
\definecolor{currentstroke}{rgb}{1.000000,1.000000,1.000000}%
\pgfsetstrokecolor{currentstroke}%
\pgfsetdash{}{0pt}%
\pgfpathmoveto{\pgfqpoint{9.447616in}{4.687116in}}%
\pgfpathcurveto{\pgfqpoint{9.458666in}{4.687116in}}{\pgfqpoint{9.469265in}{4.691506in}}{\pgfqpoint{9.477079in}{4.699320in}}%
\pgfpathcurveto{\pgfqpoint{9.484892in}{4.707133in}}{\pgfqpoint{9.489283in}{4.717732in}}{\pgfqpoint{9.489283in}{4.728782in}}%
\pgfpathcurveto{\pgfqpoint{9.489283in}{4.739833in}}{\pgfqpoint{9.484892in}{4.750432in}}{\pgfqpoint{9.477079in}{4.758245in}}%
\pgfpathcurveto{\pgfqpoint{9.469265in}{4.766059in}}{\pgfqpoint{9.458666in}{4.770449in}}{\pgfqpoint{9.447616in}{4.770449in}}%
\pgfpathcurveto{\pgfqpoint{9.436566in}{4.770449in}}{\pgfqpoint{9.425967in}{4.766059in}}{\pgfqpoint{9.418153in}{4.758245in}}%
\pgfpathcurveto{\pgfqpoint{9.410340in}{4.750432in}}{\pgfqpoint{9.405949in}{4.739833in}}{\pgfqpoint{9.405949in}{4.728782in}}%
\pgfpathcurveto{\pgfqpoint{9.405949in}{4.717732in}}{\pgfqpoint{9.410340in}{4.707133in}}{\pgfqpoint{9.418153in}{4.699320in}}%
\pgfpathcurveto{\pgfqpoint{9.425967in}{4.691506in}}{\pgfqpoint{9.436566in}{4.687116in}}{\pgfqpoint{9.447616in}{4.687116in}}%
\pgfpathclose%
\pgfusepath{stroke,fill}%
\end{pgfscope}%
\begin{pgfscope}%
\pgfpathrectangle{\pgfqpoint{0.570343in}{0.331635in}}{\pgfqpoint{9.300000in}{7.700000in}}%
\pgfusepath{clip}%
\pgfsetbuttcap%
\pgfsetroundjoin%
\definecolor{currentfill}{rgb}{0.631373,0.788235,0.956863}%
\pgfsetfillcolor{currentfill}%
\pgfsetlinewidth{0.481800pt}%
\definecolor{currentstroke}{rgb}{1.000000,1.000000,1.000000}%
\pgfsetstrokecolor{currentstroke}%
\pgfsetdash{}{0pt}%
\pgfpathmoveto{\pgfqpoint{6.337177in}{4.231956in}}%
\pgfpathcurveto{\pgfqpoint{6.348227in}{4.231956in}}{\pgfqpoint{6.358826in}{4.236346in}}{\pgfqpoint{6.366639in}{4.244160in}}%
\pgfpathcurveto{\pgfqpoint{6.374453in}{4.251974in}}{\pgfqpoint{6.378843in}{4.262573in}}{\pgfqpoint{6.378843in}{4.273623in}}%
\pgfpathcurveto{\pgfqpoint{6.378843in}{4.284673in}}{\pgfqpoint{6.374453in}{4.295272in}}{\pgfqpoint{6.366639in}{4.303085in}}%
\pgfpathcurveto{\pgfqpoint{6.358826in}{4.310899in}}{\pgfqpoint{6.348227in}{4.315289in}}{\pgfqpoint{6.337177in}{4.315289in}}%
\pgfpathcurveto{\pgfqpoint{6.326127in}{4.315289in}}{\pgfqpoint{6.315528in}{4.310899in}}{\pgfqpoint{6.307714in}{4.303085in}}%
\pgfpathcurveto{\pgfqpoint{6.299900in}{4.295272in}}{\pgfqpoint{6.295510in}{4.284673in}}{\pgfqpoint{6.295510in}{4.273623in}}%
\pgfpathcurveto{\pgfqpoint{6.295510in}{4.262573in}}{\pgfqpoint{6.299900in}{4.251974in}}{\pgfqpoint{6.307714in}{4.244160in}}%
\pgfpathcurveto{\pgfqpoint{6.315528in}{4.236346in}}{\pgfqpoint{6.326127in}{4.231956in}}{\pgfqpoint{6.337177in}{4.231956in}}%
\pgfpathclose%
\pgfusepath{stroke,fill}%
\end{pgfscope}%
\begin{pgfscope}%
\pgfpathrectangle{\pgfqpoint{0.570343in}{0.331635in}}{\pgfqpoint{9.300000in}{7.700000in}}%
\pgfusepath{clip}%
\pgfsetbuttcap%
\pgfsetroundjoin%
\definecolor{currentfill}{rgb}{0.631373,0.788235,0.956863}%
\pgfsetfillcolor{currentfill}%
\pgfsetlinewidth{0.481800pt}%
\definecolor{currentstroke}{rgb}{1.000000,1.000000,1.000000}%
\pgfsetstrokecolor{currentstroke}%
\pgfsetdash{}{0pt}%
\pgfpathmoveto{\pgfqpoint{3.060240in}{4.505893in}}%
\pgfpathcurveto{\pgfqpoint{3.071291in}{4.505893in}}{\pgfqpoint{3.081890in}{4.510284in}}{\pgfqpoint{3.089703in}{4.518097in}}%
\pgfpathcurveto{\pgfqpoint{3.097517in}{4.525911in}}{\pgfqpoint{3.101907in}{4.536510in}}{\pgfqpoint{3.101907in}{4.547560in}}%
\pgfpathcurveto{\pgfqpoint{3.101907in}{4.558610in}}{\pgfqpoint{3.097517in}{4.569209in}}{\pgfqpoint{3.089703in}{4.577023in}}%
\pgfpathcurveto{\pgfqpoint{3.081890in}{4.584836in}}{\pgfqpoint{3.071291in}{4.589227in}}{\pgfqpoint{3.060240in}{4.589227in}}%
\pgfpathcurveto{\pgfqpoint{3.049190in}{4.589227in}}{\pgfqpoint{3.038591in}{4.584836in}}{\pgfqpoint{3.030778in}{4.577023in}}%
\pgfpathcurveto{\pgfqpoint{3.022964in}{4.569209in}}{\pgfqpoint{3.018574in}{4.558610in}}{\pgfqpoint{3.018574in}{4.547560in}}%
\pgfpathcurveto{\pgfqpoint{3.018574in}{4.536510in}}{\pgfqpoint{3.022964in}{4.525911in}}{\pgfqpoint{3.030778in}{4.518097in}}%
\pgfpathcurveto{\pgfqpoint{3.038591in}{4.510284in}}{\pgfqpoint{3.049190in}{4.505893in}}{\pgfqpoint{3.060240in}{4.505893in}}%
\pgfpathclose%
\pgfusepath{stroke,fill}%
\end{pgfscope}%
\begin{pgfscope}%
\pgfpathrectangle{\pgfqpoint{0.570343in}{0.331635in}}{\pgfqpoint{9.300000in}{7.700000in}}%
\pgfusepath{clip}%
\pgfsetbuttcap%
\pgfsetroundjoin%
\definecolor{currentfill}{rgb}{0.631373,0.788235,0.956863}%
\pgfsetfillcolor{currentfill}%
\pgfsetlinewidth{0.481800pt}%
\definecolor{currentstroke}{rgb}{1.000000,1.000000,1.000000}%
\pgfsetstrokecolor{currentstroke}%
\pgfsetdash{}{0pt}%
\pgfpathmoveto{\pgfqpoint{0.993071in}{5.114428in}}%
\pgfpathcurveto{\pgfqpoint{1.004121in}{5.114428in}}{\pgfqpoint{1.014720in}{5.118818in}}{\pgfqpoint{1.022533in}{5.126632in}}%
\pgfpathcurveto{\pgfqpoint{1.030347in}{5.134446in}}{\pgfqpoint{1.034737in}{5.145045in}}{\pgfqpoint{1.034737in}{5.156095in}}%
\pgfpathcurveto{\pgfqpoint{1.034737in}{5.167145in}}{\pgfqpoint{1.030347in}{5.177744in}}{\pgfqpoint{1.022533in}{5.185558in}}%
\pgfpathcurveto{\pgfqpoint{1.014720in}{5.193371in}}{\pgfqpoint{1.004121in}{5.197762in}}{\pgfqpoint{0.993071in}{5.197762in}}%
\pgfpathcurveto{\pgfqpoint{0.982020in}{5.197762in}}{\pgfqpoint{0.971421in}{5.193371in}}{\pgfqpoint{0.963608in}{5.185558in}}%
\pgfpathcurveto{\pgfqpoint{0.955794in}{5.177744in}}{\pgfqpoint{0.951404in}{5.167145in}}{\pgfqpoint{0.951404in}{5.156095in}}%
\pgfpathcurveto{\pgfqpoint{0.951404in}{5.145045in}}{\pgfqpoint{0.955794in}{5.134446in}}{\pgfqpoint{0.963608in}{5.126632in}}%
\pgfpathcurveto{\pgfqpoint{0.971421in}{5.118818in}}{\pgfqpoint{0.982020in}{5.114428in}}{\pgfqpoint{0.993071in}{5.114428in}}%
\pgfpathclose%
\pgfusepath{stroke,fill}%
\end{pgfscope}%
\begin{pgfscope}%
\pgfpathrectangle{\pgfqpoint{0.570343in}{0.331635in}}{\pgfqpoint{9.300000in}{7.700000in}}%
\pgfusepath{clip}%
\pgfsetbuttcap%
\pgfsetroundjoin%
\definecolor{currentfill}{rgb}{0.631373,0.788235,0.956863}%
\pgfsetfillcolor{currentfill}%
\pgfsetlinewidth{0.481800pt}%
\definecolor{currentstroke}{rgb}{1.000000,1.000000,1.000000}%
\pgfsetstrokecolor{currentstroke}%
\pgfsetdash{}{0pt}%
\pgfpathmoveto{\pgfqpoint{6.861115in}{0.792873in}}%
\pgfpathcurveto{\pgfqpoint{6.872165in}{0.792873in}}{\pgfqpoint{6.882764in}{0.797263in}}{\pgfqpoint{6.890577in}{0.805077in}}%
\pgfpathcurveto{\pgfqpoint{6.898391in}{0.812891in}}{\pgfqpoint{6.902781in}{0.823490in}}{\pgfqpoint{6.902781in}{0.834540in}}%
\pgfpathcurveto{\pgfqpoint{6.902781in}{0.845590in}}{\pgfqpoint{6.898391in}{0.856189in}}{\pgfqpoint{6.890577in}{0.864003in}}%
\pgfpathcurveto{\pgfqpoint{6.882764in}{0.871816in}}{\pgfqpoint{6.872165in}{0.876207in}}{\pgfqpoint{6.861115in}{0.876207in}}%
\pgfpathcurveto{\pgfqpoint{6.850065in}{0.876207in}}{\pgfqpoint{6.839466in}{0.871816in}}{\pgfqpoint{6.831652in}{0.864003in}}%
\pgfpathcurveto{\pgfqpoint{6.823838in}{0.856189in}}{\pgfqpoint{6.819448in}{0.845590in}}{\pgfqpoint{6.819448in}{0.834540in}}%
\pgfpathcurveto{\pgfqpoint{6.819448in}{0.823490in}}{\pgfqpoint{6.823838in}{0.812891in}}{\pgfqpoint{6.831652in}{0.805077in}}%
\pgfpathcurveto{\pgfqpoint{6.839466in}{0.797263in}}{\pgfqpoint{6.850065in}{0.792873in}}{\pgfqpoint{6.861115in}{0.792873in}}%
\pgfpathclose%
\pgfusepath{stroke,fill}%
\end{pgfscope}%
\begin{pgfscope}%
\pgfpathrectangle{\pgfqpoint{0.570343in}{0.331635in}}{\pgfqpoint{9.300000in}{7.700000in}}%
\pgfusepath{clip}%
\pgfsetbuttcap%
\pgfsetroundjoin%
\definecolor{currentfill}{rgb}{0.631373,0.788235,0.956863}%
\pgfsetfillcolor{currentfill}%
\pgfsetlinewidth{0.481800pt}%
\definecolor{currentstroke}{rgb}{1.000000,1.000000,1.000000}%
\pgfsetstrokecolor{currentstroke}%
\pgfsetdash{}{0pt}%
\pgfpathmoveto{\pgfqpoint{4.067058in}{2.345881in}}%
\pgfpathcurveto{\pgfqpoint{4.078108in}{2.345881in}}{\pgfqpoint{4.088707in}{2.350271in}}{\pgfqpoint{4.096521in}{2.358085in}}%
\pgfpathcurveto{\pgfqpoint{4.104335in}{2.365899in}}{\pgfqpoint{4.108725in}{2.376498in}}{\pgfqpoint{4.108725in}{2.387548in}}%
\pgfpathcurveto{\pgfqpoint{4.108725in}{2.398598in}}{\pgfqpoint{4.104335in}{2.409197in}}{\pgfqpoint{4.096521in}{2.417010in}}%
\pgfpathcurveto{\pgfqpoint{4.088707in}{2.424824in}}{\pgfqpoint{4.078108in}{2.429214in}}{\pgfqpoint{4.067058in}{2.429214in}}%
\pgfpathcurveto{\pgfqpoint{4.056008in}{2.429214in}}{\pgfqpoint{4.045409in}{2.424824in}}{\pgfqpoint{4.037595in}{2.417010in}}%
\pgfpathcurveto{\pgfqpoint{4.029782in}{2.409197in}}{\pgfqpoint{4.025391in}{2.398598in}}{\pgfqpoint{4.025391in}{2.387548in}}%
\pgfpathcurveto{\pgfqpoint{4.025391in}{2.376498in}}{\pgfqpoint{4.029782in}{2.365899in}}{\pgfqpoint{4.037595in}{2.358085in}}%
\pgfpathcurveto{\pgfqpoint{4.045409in}{2.350271in}}{\pgfqpoint{4.056008in}{2.345881in}}{\pgfqpoint{4.067058in}{2.345881in}}%
\pgfpathclose%
\pgfusepath{stroke,fill}%
\end{pgfscope}%
\begin{pgfscope}%
\pgfpathrectangle{\pgfqpoint{0.570343in}{0.331635in}}{\pgfqpoint{9.300000in}{7.700000in}}%
\pgfusepath{clip}%
\pgfsetbuttcap%
\pgfsetroundjoin%
\definecolor{currentfill}{rgb}{0.631373,0.788235,0.956863}%
\pgfsetfillcolor{currentfill}%
\pgfsetlinewidth{0.481800pt}%
\definecolor{currentstroke}{rgb}{1.000000,1.000000,1.000000}%
\pgfsetstrokecolor{currentstroke}%
\pgfsetdash{}{0pt}%
\pgfpathmoveto{\pgfqpoint{6.352480in}{5.846096in}}%
\pgfpathcurveto{\pgfqpoint{6.363530in}{5.846096in}}{\pgfqpoint{6.374129in}{5.850486in}}{\pgfqpoint{6.381943in}{5.858300in}}%
\pgfpathcurveto{\pgfqpoint{6.389757in}{5.866113in}}{\pgfqpoint{6.394147in}{5.876713in}}{\pgfqpoint{6.394147in}{5.887763in}}%
\pgfpathcurveto{\pgfqpoint{6.394147in}{5.898813in}}{\pgfqpoint{6.389757in}{5.909412in}}{\pgfqpoint{6.381943in}{5.917225in}}%
\pgfpathcurveto{\pgfqpoint{6.374129in}{5.925039in}}{\pgfqpoint{6.363530in}{5.929429in}}{\pgfqpoint{6.352480in}{5.929429in}}%
\pgfpathcurveto{\pgfqpoint{6.341430in}{5.929429in}}{\pgfqpoint{6.330831in}{5.925039in}}{\pgfqpoint{6.323017in}{5.917225in}}%
\pgfpathcurveto{\pgfqpoint{6.315204in}{5.909412in}}{\pgfqpoint{6.310814in}{5.898813in}}{\pgfqpoint{6.310814in}{5.887763in}}%
\pgfpathcurveto{\pgfqpoint{6.310814in}{5.876713in}}{\pgfqpoint{6.315204in}{5.866113in}}{\pgfqpoint{6.323017in}{5.858300in}}%
\pgfpathcurveto{\pgfqpoint{6.330831in}{5.850486in}}{\pgfqpoint{6.341430in}{5.846096in}}{\pgfqpoint{6.352480in}{5.846096in}}%
\pgfpathclose%
\pgfusepath{stroke,fill}%
\end{pgfscope}%
\begin{pgfscope}%
\pgfpathrectangle{\pgfqpoint{0.570343in}{0.331635in}}{\pgfqpoint{9.300000in}{7.700000in}}%
\pgfusepath{clip}%
\pgfsetbuttcap%
\pgfsetroundjoin%
\definecolor{currentfill}{rgb}{0.631373,0.788235,0.956863}%
\pgfsetfillcolor{currentfill}%
\pgfsetlinewidth{0.481800pt}%
\definecolor{currentstroke}{rgb}{1.000000,1.000000,1.000000}%
\pgfsetstrokecolor{currentstroke}%
\pgfsetdash{}{0pt}%
\pgfpathmoveto{\pgfqpoint{5.146154in}{5.452263in}}%
\pgfpathcurveto{\pgfqpoint{5.157204in}{5.452263in}}{\pgfqpoint{5.167803in}{5.456653in}}{\pgfqpoint{5.175617in}{5.464466in}}%
\pgfpathcurveto{\pgfqpoint{5.183431in}{5.472280in}}{\pgfqpoint{5.187821in}{5.482879in}}{\pgfqpoint{5.187821in}{5.493929in}}%
\pgfpathcurveto{\pgfqpoint{5.187821in}{5.504979in}}{\pgfqpoint{5.183431in}{5.515578in}}{\pgfqpoint{5.175617in}{5.523392in}}%
\pgfpathcurveto{\pgfqpoint{5.167803in}{5.531206in}}{\pgfqpoint{5.157204in}{5.535596in}}{\pgfqpoint{5.146154in}{5.535596in}}%
\pgfpathcurveto{\pgfqpoint{5.135104in}{5.535596in}}{\pgfqpoint{5.124505in}{5.531206in}}{\pgfqpoint{5.116692in}{5.523392in}}%
\pgfpathcurveto{\pgfqpoint{5.108878in}{5.515578in}}{\pgfqpoint{5.104488in}{5.504979in}}{\pgfqpoint{5.104488in}{5.493929in}}%
\pgfpathcurveto{\pgfqpoint{5.104488in}{5.482879in}}{\pgfqpoint{5.108878in}{5.472280in}}{\pgfqpoint{5.116692in}{5.464466in}}%
\pgfpathcurveto{\pgfqpoint{5.124505in}{5.456653in}}{\pgfqpoint{5.135104in}{5.452263in}}{\pgfqpoint{5.146154in}{5.452263in}}%
\pgfpathclose%
\pgfusepath{stroke,fill}%
\end{pgfscope}%
\begin{pgfscope}%
\pgfpathrectangle{\pgfqpoint{0.570343in}{0.331635in}}{\pgfqpoint{9.300000in}{7.700000in}}%
\pgfusepath{clip}%
\pgfsetbuttcap%
\pgfsetroundjoin%
\definecolor{currentfill}{rgb}{0.631373,0.788235,0.956863}%
\pgfsetfillcolor{currentfill}%
\pgfsetlinewidth{0.481800pt}%
\definecolor{currentstroke}{rgb}{1.000000,1.000000,1.000000}%
\pgfsetstrokecolor{currentstroke}%
\pgfsetdash{}{0pt}%
\pgfpathmoveto{\pgfqpoint{7.564757in}{2.443303in}}%
\pgfpathcurveto{\pgfqpoint{7.575807in}{2.443303in}}{\pgfqpoint{7.586406in}{2.447694in}}{\pgfqpoint{7.594220in}{2.455507in}}%
\pgfpathcurveto{\pgfqpoint{7.602034in}{2.463321in}}{\pgfqpoint{7.606424in}{2.473920in}}{\pgfqpoint{7.606424in}{2.484970in}}%
\pgfpathcurveto{\pgfqpoint{7.606424in}{2.496020in}}{\pgfqpoint{7.602034in}{2.506619in}}{\pgfqpoint{7.594220in}{2.514433in}}%
\pgfpathcurveto{\pgfqpoint{7.586406in}{2.522247in}}{\pgfqpoint{7.575807in}{2.526637in}}{\pgfqpoint{7.564757in}{2.526637in}}%
\pgfpathcurveto{\pgfqpoint{7.553707in}{2.526637in}}{\pgfqpoint{7.543108in}{2.522247in}}{\pgfqpoint{7.535294in}{2.514433in}}%
\pgfpathcurveto{\pgfqpoint{7.527481in}{2.506619in}}{\pgfqpoint{7.523091in}{2.496020in}}{\pgfqpoint{7.523091in}{2.484970in}}%
\pgfpathcurveto{\pgfqpoint{7.523091in}{2.473920in}}{\pgfqpoint{7.527481in}{2.463321in}}{\pgfqpoint{7.535294in}{2.455507in}}%
\pgfpathcurveto{\pgfqpoint{7.543108in}{2.447694in}}{\pgfqpoint{7.553707in}{2.443303in}}{\pgfqpoint{7.564757in}{2.443303in}}%
\pgfpathclose%
\pgfusepath{stroke,fill}%
\end{pgfscope}%
\begin{pgfscope}%
\pgfpathrectangle{\pgfqpoint{0.570343in}{0.331635in}}{\pgfqpoint{9.300000in}{7.700000in}}%
\pgfusepath{clip}%
\pgfsetbuttcap%
\pgfsetroundjoin%
\definecolor{currentfill}{rgb}{0.631373,0.788235,0.956863}%
\pgfsetfillcolor{currentfill}%
\pgfsetlinewidth{0.481800pt}%
\definecolor{currentstroke}{rgb}{1.000000,1.000000,1.000000}%
\pgfsetstrokecolor{currentstroke}%
\pgfsetdash{}{0pt}%
\pgfpathmoveto{\pgfqpoint{8.396171in}{4.660144in}}%
\pgfpathcurveto{\pgfqpoint{8.407221in}{4.660144in}}{\pgfqpoint{8.417820in}{4.664534in}}{\pgfqpoint{8.425634in}{4.672348in}}%
\pgfpathcurveto{\pgfqpoint{8.433448in}{4.680162in}}{\pgfqpoint{8.437838in}{4.690761in}}{\pgfqpoint{8.437838in}{4.701811in}}%
\pgfpathcurveto{\pgfqpoint{8.437838in}{4.712861in}}{\pgfqpoint{8.433448in}{4.723460in}}{\pgfqpoint{8.425634in}{4.731274in}}%
\pgfpathcurveto{\pgfqpoint{8.417820in}{4.739087in}}{\pgfqpoint{8.407221in}{4.743477in}}{\pgfqpoint{8.396171in}{4.743477in}}%
\pgfpathcurveto{\pgfqpoint{8.385121in}{4.743477in}}{\pgfqpoint{8.374522in}{4.739087in}}{\pgfqpoint{8.366708in}{4.731274in}}%
\pgfpathcurveto{\pgfqpoint{8.358895in}{4.723460in}}{\pgfqpoint{8.354505in}{4.712861in}}{\pgfqpoint{8.354505in}{4.701811in}}%
\pgfpathcurveto{\pgfqpoint{8.354505in}{4.690761in}}{\pgfqpoint{8.358895in}{4.680162in}}{\pgfqpoint{8.366708in}{4.672348in}}%
\pgfpathcurveto{\pgfqpoint{8.374522in}{4.664534in}}{\pgfqpoint{8.385121in}{4.660144in}}{\pgfqpoint{8.396171in}{4.660144in}}%
\pgfpathclose%
\pgfusepath{stroke,fill}%
\end{pgfscope}%
\begin{pgfscope}%
\pgfpathrectangle{\pgfqpoint{0.570343in}{0.331635in}}{\pgfqpoint{9.300000in}{7.700000in}}%
\pgfusepath{clip}%
\pgfsetbuttcap%
\pgfsetroundjoin%
\definecolor{currentfill}{rgb}{0.631373,0.788235,0.956863}%
\pgfsetfillcolor{currentfill}%
\pgfsetlinewidth{0.481800pt}%
\definecolor{currentstroke}{rgb}{1.000000,1.000000,1.000000}%
\pgfsetstrokecolor{currentstroke}%
\pgfsetdash{}{0pt}%
\pgfpathmoveto{\pgfqpoint{3.531441in}{5.702170in}}%
\pgfpathcurveto{\pgfqpoint{3.542491in}{5.702170in}}{\pgfqpoint{3.553090in}{5.706560in}}{\pgfqpoint{3.560904in}{5.714374in}}%
\pgfpathcurveto{\pgfqpoint{3.568718in}{5.722187in}}{\pgfqpoint{3.573108in}{5.732787in}}{\pgfqpoint{3.573108in}{5.743837in}}%
\pgfpathcurveto{\pgfqpoint{3.573108in}{5.754887in}}{\pgfqpoint{3.568718in}{5.765486in}}{\pgfqpoint{3.560904in}{5.773299in}}%
\pgfpathcurveto{\pgfqpoint{3.553090in}{5.781113in}}{\pgfqpoint{3.542491in}{5.785503in}}{\pgfqpoint{3.531441in}{5.785503in}}%
\pgfpathcurveto{\pgfqpoint{3.520391in}{5.785503in}}{\pgfqpoint{3.509792in}{5.781113in}}{\pgfqpoint{3.501978in}{5.773299in}}%
\pgfpathcurveto{\pgfqpoint{3.494165in}{5.765486in}}{\pgfqpoint{3.489775in}{5.754887in}}{\pgfqpoint{3.489775in}{5.743837in}}%
\pgfpathcurveto{\pgfqpoint{3.489775in}{5.732787in}}{\pgfqpoint{3.494165in}{5.722187in}}{\pgfqpoint{3.501978in}{5.714374in}}%
\pgfpathcurveto{\pgfqpoint{3.509792in}{5.706560in}}{\pgfqpoint{3.520391in}{5.702170in}}{\pgfqpoint{3.531441in}{5.702170in}}%
\pgfpathclose%
\pgfusepath{stroke,fill}%
\end{pgfscope}%
\begin{pgfscope}%
\pgfpathrectangle{\pgfqpoint{0.570343in}{0.331635in}}{\pgfqpoint{9.300000in}{7.700000in}}%
\pgfusepath{clip}%
\pgfsetbuttcap%
\pgfsetroundjoin%
\definecolor{currentfill}{rgb}{1.000000,0.705882,0.509804}%
\pgfsetfillcolor{currentfill}%
\pgfsetlinewidth{0.481800pt}%
\definecolor{currentstroke}{rgb}{1.000000,1.000000,1.000000}%
\pgfsetstrokecolor{currentstroke}%
\pgfsetdash{}{0pt}%
\pgfpathmoveto{\pgfqpoint{5.495895in}{6.139158in}}%
\pgfpathcurveto{\pgfqpoint{5.506945in}{6.139158in}}{\pgfqpoint{5.517544in}{6.143549in}}{\pgfqpoint{5.525358in}{6.151362in}}%
\pgfpathcurveto{\pgfqpoint{5.533172in}{6.159176in}}{\pgfqpoint{5.537562in}{6.169775in}}{\pgfqpoint{5.537562in}{6.180825in}}%
\pgfpathcurveto{\pgfqpoint{5.537562in}{6.191875in}}{\pgfqpoint{5.533172in}{6.202474in}}{\pgfqpoint{5.525358in}{6.210288in}}%
\pgfpathcurveto{\pgfqpoint{5.517544in}{6.218101in}}{\pgfqpoint{5.506945in}{6.222492in}}{\pgfqpoint{5.495895in}{6.222492in}}%
\pgfpathcurveto{\pgfqpoint{5.484845in}{6.222492in}}{\pgfqpoint{5.474246in}{6.218101in}}{\pgfqpoint{5.466432in}{6.210288in}}%
\pgfpathcurveto{\pgfqpoint{5.458619in}{6.202474in}}{\pgfqpoint{5.454229in}{6.191875in}}{\pgfqpoint{5.454229in}{6.180825in}}%
\pgfpathcurveto{\pgfqpoint{5.454229in}{6.169775in}}{\pgfqpoint{5.458619in}{6.159176in}}{\pgfqpoint{5.466432in}{6.151362in}}%
\pgfpathcurveto{\pgfqpoint{5.474246in}{6.143549in}}{\pgfqpoint{5.484845in}{6.139158in}}{\pgfqpoint{5.495895in}{6.139158in}}%
\pgfpathclose%
\pgfusepath{stroke,fill}%
\end{pgfscope}%
\begin{pgfscope}%
\pgfpathrectangle{\pgfqpoint{0.570343in}{0.331635in}}{\pgfqpoint{9.300000in}{7.700000in}}%
\pgfusepath{clip}%
\pgfsetbuttcap%
\pgfsetroundjoin%
\definecolor{currentfill}{rgb}{1.000000,0.705882,0.509804}%
\pgfsetfillcolor{currentfill}%
\pgfsetlinewidth{0.481800pt}%
\definecolor{currentstroke}{rgb}{1.000000,1.000000,1.000000}%
\pgfsetstrokecolor{currentstroke}%
\pgfsetdash{}{0pt}%
\pgfpathmoveto{\pgfqpoint{8.684047in}{3.619507in}}%
\pgfpathcurveto{\pgfqpoint{8.695097in}{3.619507in}}{\pgfqpoint{8.705696in}{3.623897in}}{\pgfqpoint{8.713510in}{3.631711in}}%
\pgfpathcurveto{\pgfqpoint{8.721324in}{3.639525in}}{\pgfqpoint{8.725714in}{3.650124in}}{\pgfqpoint{8.725714in}{3.661174in}}%
\pgfpathcurveto{\pgfqpoint{8.725714in}{3.672224in}}{\pgfqpoint{8.721324in}{3.682823in}}{\pgfqpoint{8.713510in}{3.690637in}}%
\pgfpathcurveto{\pgfqpoint{8.705696in}{3.698450in}}{\pgfqpoint{8.695097in}{3.702840in}}{\pgfqpoint{8.684047in}{3.702840in}}%
\pgfpathcurveto{\pgfqpoint{8.672997in}{3.702840in}}{\pgfqpoint{8.662398in}{3.698450in}}{\pgfqpoint{8.654584in}{3.690637in}}%
\pgfpathcurveto{\pgfqpoint{8.646771in}{3.682823in}}{\pgfqpoint{8.642380in}{3.672224in}}{\pgfqpoint{8.642380in}{3.661174in}}%
\pgfpathcurveto{\pgfqpoint{8.642380in}{3.650124in}}{\pgfqpoint{8.646771in}{3.639525in}}{\pgfqpoint{8.654584in}{3.631711in}}%
\pgfpathcurveto{\pgfqpoint{8.662398in}{3.623897in}}{\pgfqpoint{8.672997in}{3.619507in}}{\pgfqpoint{8.684047in}{3.619507in}}%
\pgfpathclose%
\pgfusepath{stroke,fill}%
\end{pgfscope}%
\begin{pgfscope}%
\pgfpathrectangle{\pgfqpoint{0.570343in}{0.331635in}}{\pgfqpoint{9.300000in}{7.700000in}}%
\pgfusepath{clip}%
\pgfsetbuttcap%
\pgfsetroundjoin%
\definecolor{currentfill}{rgb}{1.000000,0.705882,0.509804}%
\pgfsetfillcolor{currentfill}%
\pgfsetlinewidth{0.481800pt}%
\definecolor{currentstroke}{rgb}{1.000000,1.000000,1.000000}%
\pgfsetstrokecolor{currentstroke}%
\pgfsetdash{}{0pt}%
\pgfpathmoveto{\pgfqpoint{4.658532in}{4.140491in}}%
\pgfpathcurveto{\pgfqpoint{4.669583in}{4.140491in}}{\pgfqpoint{4.680182in}{4.144881in}}{\pgfqpoint{4.687995in}{4.152695in}}%
\pgfpathcurveto{\pgfqpoint{4.695809in}{4.160508in}}{\pgfqpoint{4.700199in}{4.171107in}}{\pgfqpoint{4.700199in}{4.182157in}}%
\pgfpathcurveto{\pgfqpoint{4.700199in}{4.193207in}}{\pgfqpoint{4.695809in}{4.203807in}}{\pgfqpoint{4.687995in}{4.211620in}}%
\pgfpathcurveto{\pgfqpoint{4.680182in}{4.219434in}}{\pgfqpoint{4.669583in}{4.223824in}}{\pgfqpoint{4.658532in}{4.223824in}}%
\pgfpathcurveto{\pgfqpoint{4.647482in}{4.223824in}}{\pgfqpoint{4.636883in}{4.219434in}}{\pgfqpoint{4.629070in}{4.211620in}}%
\pgfpathcurveto{\pgfqpoint{4.621256in}{4.203807in}}{\pgfqpoint{4.616866in}{4.193207in}}{\pgfqpoint{4.616866in}{4.182157in}}%
\pgfpathcurveto{\pgfqpoint{4.616866in}{4.171107in}}{\pgfqpoint{4.621256in}{4.160508in}}{\pgfqpoint{4.629070in}{4.152695in}}%
\pgfpathcurveto{\pgfqpoint{4.636883in}{4.144881in}}{\pgfqpoint{4.647482in}{4.140491in}}{\pgfqpoint{4.658532in}{4.140491in}}%
\pgfpathclose%
\pgfusepath{stroke,fill}%
\end{pgfscope}%
\begin{pgfscope}%
\pgfpathrectangle{\pgfqpoint{0.570343in}{0.331635in}}{\pgfqpoint{9.300000in}{7.700000in}}%
\pgfusepath{clip}%
\pgfsetbuttcap%
\pgfsetroundjoin%
\definecolor{currentfill}{rgb}{1.000000,0.705882,0.509804}%
\pgfsetfillcolor{currentfill}%
\pgfsetlinewidth{0.481800pt}%
\definecolor{currentstroke}{rgb}{1.000000,1.000000,1.000000}%
\pgfsetstrokecolor{currentstroke}%
\pgfsetdash{}{0pt}%
\pgfpathmoveto{\pgfqpoint{5.265486in}{3.699856in}}%
\pgfpathcurveto{\pgfqpoint{5.276536in}{3.699856in}}{\pgfqpoint{5.287135in}{3.704247in}}{\pgfqpoint{5.294948in}{3.712060in}}%
\pgfpathcurveto{\pgfqpoint{5.302762in}{3.719874in}}{\pgfqpoint{5.307152in}{3.730473in}}{\pgfqpoint{5.307152in}{3.741523in}}%
\pgfpathcurveto{\pgfqpoint{5.307152in}{3.752573in}}{\pgfqpoint{5.302762in}{3.763172in}}{\pgfqpoint{5.294948in}{3.770986in}}%
\pgfpathcurveto{\pgfqpoint{5.287135in}{3.778799in}}{\pgfqpoint{5.276536in}{3.783190in}}{\pgfqpoint{5.265486in}{3.783190in}}%
\pgfpathcurveto{\pgfqpoint{5.254436in}{3.783190in}}{\pgfqpoint{5.243837in}{3.778799in}}{\pgfqpoint{5.236023in}{3.770986in}}%
\pgfpathcurveto{\pgfqpoint{5.228209in}{3.763172in}}{\pgfqpoint{5.223819in}{3.752573in}}{\pgfqpoint{5.223819in}{3.741523in}}%
\pgfpathcurveto{\pgfqpoint{5.223819in}{3.730473in}}{\pgfqpoint{5.228209in}{3.719874in}}{\pgfqpoint{5.236023in}{3.712060in}}%
\pgfpathcurveto{\pgfqpoint{5.243837in}{3.704247in}}{\pgfqpoint{5.254436in}{3.699856in}}{\pgfqpoint{5.265486in}{3.699856in}}%
\pgfpathclose%
\pgfusepath{stroke,fill}%
\end{pgfscope}%
\begin{pgfscope}%
\pgfpathrectangle{\pgfqpoint{0.570343in}{0.331635in}}{\pgfqpoint{9.300000in}{7.700000in}}%
\pgfusepath{clip}%
\pgfsetbuttcap%
\pgfsetroundjoin%
\definecolor{currentfill}{rgb}{1.000000,0.705882,0.509804}%
\pgfsetfillcolor{currentfill}%
\pgfsetlinewidth{0.481800pt}%
\definecolor{currentstroke}{rgb}{1.000000,1.000000,1.000000}%
\pgfsetstrokecolor{currentstroke}%
\pgfsetdash{}{0pt}%
\pgfpathmoveto{\pgfqpoint{3.663113in}{6.535316in}}%
\pgfpathcurveto{\pgfqpoint{3.674163in}{6.535316in}}{\pgfqpoint{3.684763in}{6.539706in}}{\pgfqpoint{3.692576in}{6.547520in}}%
\pgfpathcurveto{\pgfqpoint{3.700390in}{6.555334in}}{\pgfqpoint{3.704780in}{6.565933in}}{\pgfqpoint{3.704780in}{6.576983in}}%
\pgfpathcurveto{\pgfqpoint{3.704780in}{6.588033in}}{\pgfqpoint{3.700390in}{6.598632in}}{\pgfqpoint{3.692576in}{6.606446in}}%
\pgfpathcurveto{\pgfqpoint{3.684763in}{6.614259in}}{\pgfqpoint{3.674163in}{6.618649in}}{\pgfqpoint{3.663113in}{6.618649in}}%
\pgfpathcurveto{\pgfqpoint{3.652063in}{6.618649in}}{\pgfqpoint{3.641464in}{6.614259in}}{\pgfqpoint{3.633651in}{6.606446in}}%
\pgfpathcurveto{\pgfqpoint{3.625837in}{6.598632in}}{\pgfqpoint{3.621447in}{6.588033in}}{\pgfqpoint{3.621447in}{6.576983in}}%
\pgfpathcurveto{\pgfqpoint{3.621447in}{6.565933in}}{\pgfqpoint{3.625837in}{6.555334in}}{\pgfqpoint{3.633651in}{6.547520in}}%
\pgfpathcurveto{\pgfqpoint{3.641464in}{6.539706in}}{\pgfqpoint{3.652063in}{6.535316in}}{\pgfqpoint{3.663113in}{6.535316in}}%
\pgfpathclose%
\pgfusepath{stroke,fill}%
\end{pgfscope}%
\begin{pgfscope}%
\pgfpathrectangle{\pgfqpoint{0.570343in}{0.331635in}}{\pgfqpoint{9.300000in}{7.700000in}}%
\pgfusepath{clip}%
\pgfsetbuttcap%
\pgfsetroundjoin%
\definecolor{currentfill}{rgb}{1.000000,0.705882,0.509804}%
\pgfsetfillcolor{currentfill}%
\pgfsetlinewidth{0.481800pt}%
\definecolor{currentstroke}{rgb}{1.000000,1.000000,1.000000}%
\pgfsetstrokecolor{currentstroke}%
\pgfsetdash{}{0pt}%
\pgfpathmoveto{\pgfqpoint{7.006137in}{4.502520in}}%
\pgfpathcurveto{\pgfqpoint{7.017188in}{4.502520in}}{\pgfqpoint{7.027787in}{4.506910in}}{\pgfqpoint{7.035600in}{4.514723in}}%
\pgfpathcurveto{\pgfqpoint{7.043414in}{4.522537in}}{\pgfqpoint{7.047804in}{4.533136in}}{\pgfqpoint{7.047804in}{4.544186in}}%
\pgfpathcurveto{\pgfqpoint{7.047804in}{4.555236in}}{\pgfqpoint{7.043414in}{4.565835in}}{\pgfqpoint{7.035600in}{4.573649in}}%
\pgfpathcurveto{\pgfqpoint{7.027787in}{4.581463in}}{\pgfqpoint{7.017188in}{4.585853in}}{\pgfqpoint{7.006137in}{4.585853in}}%
\pgfpathcurveto{\pgfqpoint{6.995087in}{4.585853in}}{\pgfqpoint{6.984488in}{4.581463in}}{\pgfqpoint{6.976675in}{4.573649in}}%
\pgfpathcurveto{\pgfqpoint{6.968861in}{4.565835in}}{\pgfqpoint{6.964471in}{4.555236in}}{\pgfqpoint{6.964471in}{4.544186in}}%
\pgfpathcurveto{\pgfqpoint{6.964471in}{4.533136in}}{\pgfqpoint{6.968861in}{4.522537in}}{\pgfqpoint{6.976675in}{4.514723in}}%
\pgfpathcurveto{\pgfqpoint{6.984488in}{4.506910in}}{\pgfqpoint{6.995087in}{4.502520in}}{\pgfqpoint{7.006137in}{4.502520in}}%
\pgfpathclose%
\pgfusepath{stroke,fill}%
\end{pgfscope}%
\begin{pgfscope}%
\pgfpathrectangle{\pgfqpoint{0.570343in}{0.331635in}}{\pgfqpoint{9.300000in}{7.700000in}}%
\pgfusepath{clip}%
\pgfsetbuttcap%
\pgfsetroundjoin%
\definecolor{currentfill}{rgb}{1.000000,0.705882,0.509804}%
\pgfsetfillcolor{currentfill}%
\pgfsetlinewidth{0.481800pt}%
\definecolor{currentstroke}{rgb}{1.000000,1.000000,1.000000}%
\pgfsetstrokecolor{currentstroke}%
\pgfsetdash{}{0pt}%
\pgfpathmoveto{\pgfqpoint{7.866554in}{4.490496in}}%
\pgfpathcurveto{\pgfqpoint{7.877604in}{4.490496in}}{\pgfqpoint{7.888203in}{4.494886in}}{\pgfqpoint{7.896017in}{4.502700in}}%
\pgfpathcurveto{\pgfqpoint{7.903830in}{4.510513in}}{\pgfqpoint{7.908220in}{4.521113in}}{\pgfqpoint{7.908220in}{4.532163in}}%
\pgfpathcurveto{\pgfqpoint{7.908220in}{4.543213in}}{\pgfqpoint{7.903830in}{4.553812in}}{\pgfqpoint{7.896017in}{4.561625in}}%
\pgfpathcurveto{\pgfqpoint{7.888203in}{4.569439in}}{\pgfqpoint{7.877604in}{4.573829in}}{\pgfqpoint{7.866554in}{4.573829in}}%
\pgfpathcurveto{\pgfqpoint{7.855504in}{4.573829in}}{\pgfqpoint{7.844905in}{4.569439in}}{\pgfqpoint{7.837091in}{4.561625in}}%
\pgfpathcurveto{\pgfqpoint{7.829277in}{4.553812in}}{\pgfqpoint{7.824887in}{4.543213in}}{\pgfqpoint{7.824887in}{4.532163in}}%
\pgfpathcurveto{\pgfqpoint{7.824887in}{4.521113in}}{\pgfqpoint{7.829277in}{4.510513in}}{\pgfqpoint{7.837091in}{4.502700in}}%
\pgfpathcurveto{\pgfqpoint{7.844905in}{4.494886in}}{\pgfqpoint{7.855504in}{4.490496in}}{\pgfqpoint{7.866554in}{4.490496in}}%
\pgfpathclose%
\pgfusepath{stroke,fill}%
\end{pgfscope}%
\begin{pgfscope}%
\pgfpathrectangle{\pgfqpoint{0.570343in}{0.331635in}}{\pgfqpoint{9.300000in}{7.700000in}}%
\pgfusepath{clip}%
\pgfsetbuttcap%
\pgfsetroundjoin%
\definecolor{currentfill}{rgb}{1.000000,0.705882,0.509804}%
\pgfsetfillcolor{currentfill}%
\pgfsetlinewidth{0.481800pt}%
\definecolor{currentstroke}{rgb}{1.000000,1.000000,1.000000}%
\pgfsetstrokecolor{currentstroke}%
\pgfsetdash{}{0pt}%
\pgfpathmoveto{\pgfqpoint{2.309950in}{4.798666in}}%
\pgfpathcurveto{\pgfqpoint{2.321001in}{4.798666in}}{\pgfqpoint{2.331600in}{4.803056in}}{\pgfqpoint{2.339413in}{4.810870in}}%
\pgfpathcurveto{\pgfqpoint{2.347227in}{4.818684in}}{\pgfqpoint{2.351617in}{4.829283in}}{\pgfqpoint{2.351617in}{4.840333in}}%
\pgfpathcurveto{\pgfqpoint{2.351617in}{4.851383in}}{\pgfqpoint{2.347227in}{4.861982in}}{\pgfqpoint{2.339413in}{4.869796in}}%
\pgfpathcurveto{\pgfqpoint{2.331600in}{4.877609in}}{\pgfqpoint{2.321001in}{4.881999in}}{\pgfqpoint{2.309950in}{4.881999in}}%
\pgfpathcurveto{\pgfqpoint{2.298900in}{4.881999in}}{\pgfqpoint{2.288301in}{4.877609in}}{\pgfqpoint{2.280488in}{4.869796in}}%
\pgfpathcurveto{\pgfqpoint{2.272674in}{4.861982in}}{\pgfqpoint{2.268284in}{4.851383in}}{\pgfqpoint{2.268284in}{4.840333in}}%
\pgfpathcurveto{\pgfqpoint{2.268284in}{4.829283in}}{\pgfqpoint{2.272674in}{4.818684in}}{\pgfqpoint{2.280488in}{4.810870in}}%
\pgfpathcurveto{\pgfqpoint{2.288301in}{4.803056in}}{\pgfqpoint{2.298900in}{4.798666in}}{\pgfqpoint{2.309950in}{4.798666in}}%
\pgfpathclose%
\pgfusepath{stroke,fill}%
\end{pgfscope}%
\begin{pgfscope}%
\pgfpathrectangle{\pgfqpoint{0.570343in}{0.331635in}}{\pgfqpoint{9.300000in}{7.700000in}}%
\pgfusepath{clip}%
\pgfsetbuttcap%
\pgfsetroundjoin%
\definecolor{currentfill}{rgb}{1.000000,0.705882,0.509804}%
\pgfsetfillcolor{currentfill}%
\pgfsetlinewidth{0.481800pt}%
\definecolor{currentstroke}{rgb}{1.000000,1.000000,1.000000}%
\pgfsetstrokecolor{currentstroke}%
\pgfsetdash{}{0pt}%
\pgfpathmoveto{\pgfqpoint{4.597903in}{6.246141in}}%
\pgfpathcurveto{\pgfqpoint{4.608953in}{6.246141in}}{\pgfqpoint{4.619552in}{6.250531in}}{\pgfqpoint{4.627366in}{6.258344in}}%
\pgfpathcurveto{\pgfqpoint{4.635179in}{6.266158in}}{\pgfqpoint{4.639570in}{6.276757in}}{\pgfqpoint{4.639570in}{6.287807in}}%
\pgfpathcurveto{\pgfqpoint{4.639570in}{6.298857in}}{\pgfqpoint{4.635179in}{6.309456in}}{\pgfqpoint{4.627366in}{6.317270in}}%
\pgfpathcurveto{\pgfqpoint{4.619552in}{6.325084in}}{\pgfqpoint{4.608953in}{6.329474in}}{\pgfqpoint{4.597903in}{6.329474in}}%
\pgfpathcurveto{\pgfqpoint{4.586853in}{6.329474in}}{\pgfqpoint{4.576254in}{6.325084in}}{\pgfqpoint{4.568440in}{6.317270in}}%
\pgfpathcurveto{\pgfqpoint{4.560627in}{6.309456in}}{\pgfqpoint{4.556236in}{6.298857in}}{\pgfqpoint{4.556236in}{6.287807in}}%
\pgfpathcurveto{\pgfqpoint{4.556236in}{6.276757in}}{\pgfqpoint{4.560627in}{6.266158in}}{\pgfqpoint{4.568440in}{6.258344in}}%
\pgfpathcurveto{\pgfqpoint{4.576254in}{6.250531in}}{\pgfqpoint{4.586853in}{6.246141in}}{\pgfqpoint{4.597903in}{6.246141in}}%
\pgfpathclose%
\pgfusepath{stroke,fill}%
\end{pgfscope}%
\begin{pgfscope}%
\pgfpathrectangle{\pgfqpoint{0.570343in}{0.331635in}}{\pgfqpoint{9.300000in}{7.700000in}}%
\pgfusepath{clip}%
\pgfsetbuttcap%
\pgfsetroundjoin%
\definecolor{currentfill}{rgb}{1.000000,0.705882,0.509804}%
\pgfsetfillcolor{currentfill}%
\pgfsetlinewidth{0.481800pt}%
\definecolor{currentstroke}{rgb}{1.000000,1.000000,1.000000}%
\pgfsetstrokecolor{currentstroke}%
\pgfsetdash{}{0pt}%
\pgfpathmoveto{\pgfqpoint{2.912749in}{6.686491in}}%
\pgfpathcurveto{\pgfqpoint{2.923799in}{6.686491in}}{\pgfqpoint{2.934398in}{6.690882in}}{\pgfqpoint{2.942212in}{6.698695in}}%
\pgfpathcurveto{\pgfqpoint{2.950026in}{6.706509in}}{\pgfqpoint{2.954416in}{6.717108in}}{\pgfqpoint{2.954416in}{6.728158in}}%
\pgfpathcurveto{\pgfqpoint{2.954416in}{6.739208in}}{\pgfqpoint{2.950026in}{6.749807in}}{\pgfqpoint{2.942212in}{6.757621in}}%
\pgfpathcurveto{\pgfqpoint{2.934398in}{6.765434in}}{\pgfqpoint{2.923799in}{6.769825in}}{\pgfqpoint{2.912749in}{6.769825in}}%
\pgfpathcurveto{\pgfqpoint{2.901699in}{6.769825in}}{\pgfqpoint{2.891100in}{6.765434in}}{\pgfqpoint{2.883287in}{6.757621in}}%
\pgfpathcurveto{\pgfqpoint{2.875473in}{6.749807in}}{\pgfqpoint{2.871083in}{6.739208in}}{\pgfqpoint{2.871083in}{6.728158in}}%
\pgfpathcurveto{\pgfqpoint{2.871083in}{6.717108in}}{\pgfqpoint{2.875473in}{6.706509in}}{\pgfqpoint{2.883287in}{6.698695in}}%
\pgfpathcurveto{\pgfqpoint{2.891100in}{6.690882in}}{\pgfqpoint{2.901699in}{6.686491in}}{\pgfqpoint{2.912749in}{6.686491in}}%
\pgfpathclose%
\pgfusepath{stroke,fill}%
\end{pgfscope}%
\begin{pgfscope}%
\pgfpathrectangle{\pgfqpoint{0.570343in}{0.331635in}}{\pgfqpoint{9.300000in}{7.700000in}}%
\pgfusepath{clip}%
\pgfsetbuttcap%
\pgfsetroundjoin%
\definecolor{currentfill}{rgb}{1.000000,0.705882,0.509804}%
\pgfsetfillcolor{currentfill}%
\pgfsetlinewidth{0.481800pt}%
\definecolor{currentstroke}{rgb}{1.000000,1.000000,1.000000}%
\pgfsetstrokecolor{currentstroke}%
\pgfsetdash{}{0pt}%
\pgfpathmoveto{\pgfqpoint{3.463668in}{5.030027in}}%
\pgfpathcurveto{\pgfqpoint{3.474718in}{5.030027in}}{\pgfqpoint{3.485317in}{5.034417in}}{\pgfqpoint{3.493131in}{5.042231in}}%
\pgfpathcurveto{\pgfqpoint{3.500945in}{5.050044in}}{\pgfqpoint{3.505335in}{5.060643in}}{\pgfqpoint{3.505335in}{5.071694in}}%
\pgfpathcurveto{\pgfqpoint{3.505335in}{5.082744in}}{\pgfqpoint{3.500945in}{5.093343in}}{\pgfqpoint{3.493131in}{5.101156in}}%
\pgfpathcurveto{\pgfqpoint{3.485317in}{5.108970in}}{\pgfqpoint{3.474718in}{5.113360in}}{\pgfqpoint{3.463668in}{5.113360in}}%
\pgfpathcurveto{\pgfqpoint{3.452618in}{5.113360in}}{\pgfqpoint{3.442019in}{5.108970in}}{\pgfqpoint{3.434205in}{5.101156in}}%
\pgfpathcurveto{\pgfqpoint{3.426392in}{5.093343in}}{\pgfqpoint{3.422002in}{5.082744in}}{\pgfqpoint{3.422002in}{5.071694in}}%
\pgfpathcurveto{\pgfqpoint{3.422002in}{5.060643in}}{\pgfqpoint{3.426392in}{5.050044in}}{\pgfqpoint{3.434205in}{5.042231in}}%
\pgfpathcurveto{\pgfqpoint{3.442019in}{5.034417in}}{\pgfqpoint{3.452618in}{5.030027in}}{\pgfqpoint{3.463668in}{5.030027in}}%
\pgfpathclose%
\pgfusepath{stroke,fill}%
\end{pgfscope}%
\begin{pgfscope}%
\pgfpathrectangle{\pgfqpoint{0.570343in}{0.331635in}}{\pgfqpoint{9.300000in}{7.700000in}}%
\pgfusepath{clip}%
\pgfsetbuttcap%
\pgfsetroundjoin%
\definecolor{currentfill}{rgb}{1.000000,0.705882,0.509804}%
\pgfsetfillcolor{currentfill}%
\pgfsetlinewidth{0.481800pt}%
\definecolor{currentstroke}{rgb}{1.000000,1.000000,1.000000}%
\pgfsetstrokecolor{currentstroke}%
\pgfsetdash{}{0pt}%
\pgfpathmoveto{\pgfqpoint{8.862844in}{2.392695in}}%
\pgfpathcurveto{\pgfqpoint{8.873894in}{2.392695in}}{\pgfqpoint{8.884493in}{2.397085in}}{\pgfqpoint{8.892307in}{2.404899in}}%
\pgfpathcurveto{\pgfqpoint{8.900120in}{2.412712in}}{\pgfqpoint{8.904510in}{2.423311in}}{\pgfqpoint{8.904510in}{2.434362in}}%
\pgfpathcurveto{\pgfqpoint{8.904510in}{2.445412in}}{\pgfqpoint{8.900120in}{2.456011in}}{\pgfqpoint{8.892307in}{2.463824in}}%
\pgfpathcurveto{\pgfqpoint{8.884493in}{2.471638in}}{\pgfqpoint{8.873894in}{2.476028in}}{\pgfqpoint{8.862844in}{2.476028in}}%
\pgfpathcurveto{\pgfqpoint{8.851794in}{2.476028in}}{\pgfqpoint{8.841195in}{2.471638in}}{\pgfqpoint{8.833381in}{2.463824in}}%
\pgfpathcurveto{\pgfqpoint{8.825567in}{2.456011in}}{\pgfqpoint{8.821177in}{2.445412in}}{\pgfqpoint{8.821177in}{2.434362in}}%
\pgfpathcurveto{\pgfqpoint{8.821177in}{2.423311in}}{\pgfqpoint{8.825567in}{2.412712in}}{\pgfqpoint{8.833381in}{2.404899in}}%
\pgfpathcurveto{\pgfqpoint{8.841195in}{2.397085in}}{\pgfqpoint{8.851794in}{2.392695in}}{\pgfqpoint{8.862844in}{2.392695in}}%
\pgfpathclose%
\pgfusepath{stroke,fill}%
\end{pgfscope}%
\begin{pgfscope}%
\pgfpathrectangle{\pgfqpoint{0.570343in}{0.331635in}}{\pgfqpoint{9.300000in}{7.700000in}}%
\pgfusepath{clip}%
\pgfsetbuttcap%
\pgfsetroundjoin%
\definecolor{currentfill}{rgb}{1.000000,0.705882,0.509804}%
\pgfsetfillcolor{currentfill}%
\pgfsetlinewidth{0.481800pt}%
\definecolor{currentstroke}{rgb}{1.000000,1.000000,1.000000}%
\pgfsetstrokecolor{currentstroke}%
\pgfsetdash{}{0pt}%
\pgfpathmoveto{\pgfqpoint{7.751788in}{5.266912in}}%
\pgfpathcurveto{\pgfqpoint{7.762838in}{5.266912in}}{\pgfqpoint{7.773437in}{5.271302in}}{\pgfqpoint{7.781251in}{5.279116in}}%
\pgfpathcurveto{\pgfqpoint{7.789065in}{5.286929in}}{\pgfqpoint{7.793455in}{5.297528in}}{\pgfqpoint{7.793455in}{5.308578in}}%
\pgfpathcurveto{\pgfqpoint{7.793455in}{5.319628in}}{\pgfqpoint{7.789065in}{5.330228in}}{\pgfqpoint{7.781251in}{5.338041in}}%
\pgfpathcurveto{\pgfqpoint{7.773437in}{5.345855in}}{\pgfqpoint{7.762838in}{5.350245in}}{\pgfqpoint{7.751788in}{5.350245in}}%
\pgfpathcurveto{\pgfqpoint{7.740738in}{5.350245in}}{\pgfqpoint{7.730139in}{5.345855in}}{\pgfqpoint{7.722326in}{5.338041in}}%
\pgfpathcurveto{\pgfqpoint{7.714512in}{5.330228in}}{\pgfqpoint{7.710122in}{5.319628in}}{\pgfqpoint{7.710122in}{5.308578in}}%
\pgfpathcurveto{\pgfqpoint{7.710122in}{5.297528in}}{\pgfqpoint{7.714512in}{5.286929in}}{\pgfqpoint{7.722326in}{5.279116in}}%
\pgfpathcurveto{\pgfqpoint{7.730139in}{5.271302in}}{\pgfqpoint{7.740738in}{5.266912in}}{\pgfqpoint{7.751788in}{5.266912in}}%
\pgfpathclose%
\pgfusepath{stroke,fill}%
\end{pgfscope}%
\begin{pgfscope}%
\pgfpathrectangle{\pgfqpoint{0.570343in}{0.331635in}}{\pgfqpoint{9.300000in}{7.700000in}}%
\pgfusepath{clip}%
\pgfsetbuttcap%
\pgfsetroundjoin%
\definecolor{currentfill}{rgb}{1.000000,0.705882,0.509804}%
\pgfsetfillcolor{currentfill}%
\pgfsetlinewidth{0.481800pt}%
\definecolor{currentstroke}{rgb}{1.000000,1.000000,1.000000}%
\pgfsetstrokecolor{currentstroke}%
\pgfsetdash{}{0pt}%
\pgfpathmoveto{\pgfqpoint{4.290232in}{5.503628in}}%
\pgfpathcurveto{\pgfqpoint{4.301282in}{5.503628in}}{\pgfqpoint{4.311881in}{5.508018in}}{\pgfqpoint{4.319695in}{5.515832in}}%
\pgfpathcurveto{\pgfqpoint{4.327508in}{5.523646in}}{\pgfqpoint{4.331899in}{5.534245in}}{\pgfqpoint{4.331899in}{5.545295in}}%
\pgfpathcurveto{\pgfqpoint{4.331899in}{5.556345in}}{\pgfqpoint{4.327508in}{5.566944in}}{\pgfqpoint{4.319695in}{5.574758in}}%
\pgfpathcurveto{\pgfqpoint{4.311881in}{5.582571in}}{\pgfqpoint{4.301282in}{5.586962in}}{\pgfqpoint{4.290232in}{5.586962in}}%
\pgfpathcurveto{\pgfqpoint{4.279182in}{5.586962in}}{\pgfqpoint{4.268583in}{5.582571in}}{\pgfqpoint{4.260769in}{5.574758in}}%
\pgfpathcurveto{\pgfqpoint{4.252955in}{5.566944in}}{\pgfqpoint{4.248565in}{5.556345in}}{\pgfqpoint{4.248565in}{5.545295in}}%
\pgfpathcurveto{\pgfqpoint{4.248565in}{5.534245in}}{\pgfqpoint{4.252955in}{5.523646in}}{\pgfqpoint{4.260769in}{5.515832in}}%
\pgfpathcurveto{\pgfqpoint{4.268583in}{5.508018in}}{\pgfqpoint{4.279182in}{5.503628in}}{\pgfqpoint{4.290232in}{5.503628in}}%
\pgfpathclose%
\pgfusepath{stroke,fill}%
\end{pgfscope}%
\begin{pgfscope}%
\pgfpathrectangle{\pgfqpoint{0.570343in}{0.331635in}}{\pgfqpoint{9.300000in}{7.700000in}}%
\pgfusepath{clip}%
\pgfsetbuttcap%
\pgfsetroundjoin%
\definecolor{currentfill}{rgb}{1.000000,0.705882,0.509804}%
\pgfsetfillcolor{currentfill}%
\pgfsetlinewidth{0.481800pt}%
\definecolor{currentstroke}{rgb}{1.000000,1.000000,1.000000}%
\pgfsetstrokecolor{currentstroke}%
\pgfsetdash{}{0pt}%
\pgfpathmoveto{\pgfqpoint{6.599542in}{6.850694in}}%
\pgfpathcurveto{\pgfqpoint{6.610593in}{6.850694in}}{\pgfqpoint{6.621192in}{6.855084in}}{\pgfqpoint{6.629005in}{6.862897in}}%
\pgfpathcurveto{\pgfqpoint{6.636819in}{6.870711in}}{\pgfqpoint{6.641209in}{6.881310in}}{\pgfqpoint{6.641209in}{6.892360in}}%
\pgfpathcurveto{\pgfqpoint{6.641209in}{6.903410in}}{\pgfqpoint{6.636819in}{6.914009in}}{\pgfqpoint{6.629005in}{6.921823in}}%
\pgfpathcurveto{\pgfqpoint{6.621192in}{6.929637in}}{\pgfqpoint{6.610593in}{6.934027in}}{\pgfqpoint{6.599542in}{6.934027in}}%
\pgfpathcurveto{\pgfqpoint{6.588492in}{6.934027in}}{\pgfqpoint{6.577893in}{6.929637in}}{\pgfqpoint{6.570080in}{6.921823in}}%
\pgfpathcurveto{\pgfqpoint{6.562266in}{6.914009in}}{\pgfqpoint{6.557876in}{6.903410in}}{\pgfqpoint{6.557876in}{6.892360in}}%
\pgfpathcurveto{\pgfqpoint{6.557876in}{6.881310in}}{\pgfqpoint{6.562266in}{6.870711in}}{\pgfqpoint{6.570080in}{6.862897in}}%
\pgfpathcurveto{\pgfqpoint{6.577893in}{6.855084in}}{\pgfqpoint{6.588492in}{6.850694in}}{\pgfqpoint{6.599542in}{6.850694in}}%
\pgfpathclose%
\pgfusepath{stroke,fill}%
\end{pgfscope}%
\begin{pgfscope}%
\pgfpathrectangle{\pgfqpoint{0.570343in}{0.331635in}}{\pgfqpoint{9.300000in}{7.700000in}}%
\pgfusepath{clip}%
\pgfsetbuttcap%
\pgfsetroundjoin%
\definecolor{currentfill}{rgb}{1.000000,0.705882,0.509804}%
\pgfsetfillcolor{currentfill}%
\pgfsetlinewidth{0.481800pt}%
\definecolor{currentstroke}{rgb}{1.000000,1.000000,1.000000}%
\pgfsetstrokecolor{currentstroke}%
\pgfsetdash{}{0pt}%
\pgfpathmoveto{\pgfqpoint{5.751027in}{4.391362in}}%
\pgfpathcurveto{\pgfqpoint{5.762077in}{4.391362in}}{\pgfqpoint{5.772676in}{4.395752in}}{\pgfqpoint{5.780490in}{4.403566in}}%
\pgfpathcurveto{\pgfqpoint{5.788304in}{4.411379in}}{\pgfqpoint{5.792694in}{4.421978in}}{\pgfqpoint{5.792694in}{4.433028in}}%
\pgfpathcurveto{\pgfqpoint{5.792694in}{4.444079in}}{\pgfqpoint{5.788304in}{4.454678in}}{\pgfqpoint{5.780490in}{4.462491in}}%
\pgfpathcurveto{\pgfqpoint{5.772676in}{4.470305in}}{\pgfqpoint{5.762077in}{4.474695in}}{\pgfqpoint{5.751027in}{4.474695in}}%
\pgfpathcurveto{\pgfqpoint{5.739977in}{4.474695in}}{\pgfqpoint{5.729378in}{4.470305in}}{\pgfqpoint{5.721564in}{4.462491in}}%
\pgfpathcurveto{\pgfqpoint{5.713751in}{4.454678in}}{\pgfqpoint{5.709361in}{4.444079in}}{\pgfqpoint{5.709361in}{4.433028in}}%
\pgfpathcurveto{\pgfqpoint{5.709361in}{4.421978in}}{\pgfqpoint{5.713751in}{4.411379in}}{\pgfqpoint{5.721564in}{4.403566in}}%
\pgfpathcurveto{\pgfqpoint{5.729378in}{4.395752in}}{\pgfqpoint{5.739977in}{4.391362in}}{\pgfqpoint{5.751027in}{4.391362in}}%
\pgfpathclose%
\pgfusepath{stroke,fill}%
\end{pgfscope}%
\begin{pgfscope}%
\pgfpathrectangle{\pgfqpoint{0.570343in}{0.331635in}}{\pgfqpoint{9.300000in}{7.700000in}}%
\pgfusepath{clip}%
\pgfsetbuttcap%
\pgfsetroundjoin%
\definecolor{currentfill}{rgb}{1.000000,0.705882,0.509804}%
\pgfsetfillcolor{currentfill}%
\pgfsetlinewidth{0.481800pt}%
\definecolor{currentstroke}{rgb}{1.000000,1.000000,1.000000}%
\pgfsetstrokecolor{currentstroke}%
\pgfsetdash{}{0pt}%
\pgfpathmoveto{\pgfqpoint{6.415032in}{4.959397in}}%
\pgfpathcurveto{\pgfqpoint{6.426082in}{4.959397in}}{\pgfqpoint{6.436681in}{4.963788in}}{\pgfqpoint{6.444495in}{4.971601in}}%
\pgfpathcurveto{\pgfqpoint{6.452309in}{4.979415in}}{\pgfqpoint{6.456699in}{4.990014in}}{\pgfqpoint{6.456699in}{5.001064in}}%
\pgfpathcurveto{\pgfqpoint{6.456699in}{5.012114in}}{\pgfqpoint{6.452309in}{5.022713in}}{\pgfqpoint{6.444495in}{5.030527in}}%
\pgfpathcurveto{\pgfqpoint{6.436681in}{5.038341in}}{\pgfqpoint{6.426082in}{5.042731in}}{\pgfqpoint{6.415032in}{5.042731in}}%
\pgfpathcurveto{\pgfqpoint{6.403982in}{5.042731in}}{\pgfqpoint{6.393383in}{5.038341in}}{\pgfqpoint{6.385569in}{5.030527in}}%
\pgfpathcurveto{\pgfqpoint{6.377756in}{5.022713in}}{\pgfqpoint{6.373366in}{5.012114in}}{\pgfqpoint{6.373366in}{5.001064in}}%
\pgfpathcurveto{\pgfqpoint{6.373366in}{4.990014in}}{\pgfqpoint{6.377756in}{4.979415in}}{\pgfqpoint{6.385569in}{4.971601in}}%
\pgfpathcurveto{\pgfqpoint{6.393383in}{4.963788in}}{\pgfqpoint{6.403982in}{4.959397in}}{\pgfqpoint{6.415032in}{4.959397in}}%
\pgfpathclose%
\pgfusepath{stroke,fill}%
\end{pgfscope}%
\begin{pgfscope}%
\pgfpathrectangle{\pgfqpoint{0.570343in}{0.331635in}}{\pgfqpoint{9.300000in}{7.700000in}}%
\pgfusepath{clip}%
\pgfsetbuttcap%
\pgfsetroundjoin%
\definecolor{currentfill}{rgb}{1.000000,0.705882,0.509804}%
\pgfsetfillcolor{currentfill}%
\pgfsetlinewidth{0.481800pt}%
\definecolor{currentstroke}{rgb}{1.000000,1.000000,1.000000}%
\pgfsetstrokecolor{currentstroke}%
\pgfsetdash{}{0pt}%
\pgfpathmoveto{\pgfqpoint{3.801528in}{7.639968in}}%
\pgfpathcurveto{\pgfqpoint{3.812578in}{7.639968in}}{\pgfqpoint{3.823177in}{7.644359in}}{\pgfqpoint{3.830990in}{7.652172in}}%
\pgfpathcurveto{\pgfqpoint{3.838804in}{7.659986in}}{\pgfqpoint{3.843194in}{7.670585in}}{\pgfqpoint{3.843194in}{7.681635in}}%
\pgfpathcurveto{\pgfqpoint{3.843194in}{7.692685in}}{\pgfqpoint{3.838804in}{7.703284in}}{\pgfqpoint{3.830990in}{7.711098in}}%
\pgfpathcurveto{\pgfqpoint{3.823177in}{7.718911in}}{\pgfqpoint{3.812578in}{7.723302in}}{\pgfqpoint{3.801528in}{7.723302in}}%
\pgfpathcurveto{\pgfqpoint{3.790477in}{7.723302in}}{\pgfqpoint{3.779878in}{7.718911in}}{\pgfqpoint{3.772065in}{7.711098in}}%
\pgfpathcurveto{\pgfqpoint{3.764251in}{7.703284in}}{\pgfqpoint{3.759861in}{7.692685in}}{\pgfqpoint{3.759861in}{7.681635in}}%
\pgfpathcurveto{\pgfqpoint{3.759861in}{7.670585in}}{\pgfqpoint{3.764251in}{7.659986in}}{\pgfqpoint{3.772065in}{7.652172in}}%
\pgfpathcurveto{\pgfqpoint{3.779878in}{7.644359in}}{\pgfqpoint{3.790477in}{7.639968in}}{\pgfqpoint{3.801528in}{7.639968in}}%
\pgfpathclose%
\pgfusepath{stroke,fill}%
\end{pgfscope}%
\begin{pgfscope}%
\pgfpathrectangle{\pgfqpoint{0.570343in}{0.331635in}}{\pgfqpoint{9.300000in}{7.700000in}}%
\pgfusepath{clip}%
\pgfsetbuttcap%
\pgfsetroundjoin%
\definecolor{currentfill}{rgb}{1.000000,0.705882,0.509804}%
\pgfsetfillcolor{currentfill}%
\pgfsetlinewidth{0.481800pt}%
\definecolor{currentstroke}{rgb}{1.000000,1.000000,1.000000}%
\pgfsetstrokecolor{currentstroke}%
\pgfsetdash{}{0pt}%
\pgfpathmoveto{\pgfqpoint{1.181459in}{6.288777in}}%
\pgfpathcurveto{\pgfqpoint{1.192509in}{6.288777in}}{\pgfqpoint{1.203108in}{6.293168in}}{\pgfqpoint{1.210922in}{6.300981in}}%
\pgfpathcurveto{\pgfqpoint{1.218735in}{6.308795in}}{\pgfqpoint{1.223125in}{6.319394in}}{\pgfqpoint{1.223125in}{6.330444in}}%
\pgfpathcurveto{\pgfqpoint{1.223125in}{6.341494in}}{\pgfqpoint{1.218735in}{6.352093in}}{\pgfqpoint{1.210922in}{6.359907in}}%
\pgfpathcurveto{\pgfqpoint{1.203108in}{6.367720in}}{\pgfqpoint{1.192509in}{6.372111in}}{\pgfqpoint{1.181459in}{6.372111in}}%
\pgfpathcurveto{\pgfqpoint{1.170409in}{6.372111in}}{\pgfqpoint{1.159810in}{6.367720in}}{\pgfqpoint{1.151996in}{6.359907in}}%
\pgfpathcurveto{\pgfqpoint{1.144182in}{6.352093in}}{\pgfqpoint{1.139792in}{6.341494in}}{\pgfqpoint{1.139792in}{6.330444in}}%
\pgfpathcurveto{\pgfqpoint{1.139792in}{6.319394in}}{\pgfqpoint{1.144182in}{6.308795in}}{\pgfqpoint{1.151996in}{6.300981in}}%
\pgfpathcurveto{\pgfqpoint{1.159810in}{6.293168in}}{\pgfqpoint{1.170409in}{6.288777in}}{\pgfqpoint{1.181459in}{6.288777in}}%
\pgfpathclose%
\pgfusepath{stroke,fill}%
\end{pgfscope}%
\begin{pgfscope}%
\pgfpathrectangle{\pgfqpoint{0.570343in}{0.331635in}}{\pgfqpoint{9.300000in}{7.700000in}}%
\pgfusepath{clip}%
\pgfsetbuttcap%
\pgfsetroundjoin%
\definecolor{currentfill}{rgb}{1.000000,0.705882,0.509804}%
\pgfsetfillcolor{currentfill}%
\pgfsetlinewidth{0.481800pt}%
\definecolor{currentstroke}{rgb}{1.000000,1.000000,1.000000}%
\pgfsetstrokecolor{currentstroke}%
\pgfsetdash{}{0pt}%
\pgfpathmoveto{\pgfqpoint{2.718007in}{3.604351in}}%
\pgfpathcurveto{\pgfqpoint{2.729058in}{3.604351in}}{\pgfqpoint{2.739657in}{3.608742in}}{\pgfqpoint{2.747470in}{3.616555in}}%
\pgfpathcurveto{\pgfqpoint{2.755284in}{3.624369in}}{\pgfqpoint{2.759674in}{3.634968in}}{\pgfqpoint{2.759674in}{3.646018in}}%
\pgfpathcurveto{\pgfqpoint{2.759674in}{3.657068in}}{\pgfqpoint{2.755284in}{3.667667in}}{\pgfqpoint{2.747470in}{3.675481in}}%
\pgfpathcurveto{\pgfqpoint{2.739657in}{3.683295in}}{\pgfqpoint{2.729058in}{3.687685in}}{\pgfqpoint{2.718007in}{3.687685in}}%
\pgfpathcurveto{\pgfqpoint{2.706957in}{3.687685in}}{\pgfqpoint{2.696358in}{3.683295in}}{\pgfqpoint{2.688545in}{3.675481in}}%
\pgfpathcurveto{\pgfqpoint{2.680731in}{3.667667in}}{\pgfqpoint{2.676341in}{3.657068in}}{\pgfqpoint{2.676341in}{3.646018in}}%
\pgfpathcurveto{\pgfqpoint{2.676341in}{3.634968in}}{\pgfqpoint{2.680731in}{3.624369in}}{\pgfqpoint{2.688545in}{3.616555in}}%
\pgfpathcurveto{\pgfqpoint{2.696358in}{3.608742in}}{\pgfqpoint{2.706957in}{3.604351in}}{\pgfqpoint{2.718007in}{3.604351in}}%
\pgfpathclose%
\pgfusepath{stroke,fill}%
\end{pgfscope}%
\begin{pgfscope}%
\pgfpathrectangle{\pgfqpoint{0.570343in}{0.331635in}}{\pgfqpoint{9.300000in}{7.700000in}}%
\pgfusepath{clip}%
\pgfsetbuttcap%
\pgfsetroundjoin%
\definecolor{currentfill}{rgb}{1.000000,0.705882,0.509804}%
\pgfsetfillcolor{currentfill}%
\pgfsetlinewidth{0.481800pt}%
\definecolor{currentstroke}{rgb}{1.000000,1.000000,1.000000}%
\pgfsetstrokecolor{currentstroke}%
\pgfsetdash{}{0pt}%
\pgfpathmoveto{\pgfqpoint{6.602631in}{2.817343in}}%
\pgfpathcurveto{\pgfqpoint{6.613681in}{2.817343in}}{\pgfqpoint{6.624280in}{2.821733in}}{\pgfqpoint{6.632093in}{2.829547in}}%
\pgfpathcurveto{\pgfqpoint{6.639907in}{2.837360in}}{\pgfqpoint{6.644297in}{2.847959in}}{\pgfqpoint{6.644297in}{2.859010in}}%
\pgfpathcurveto{\pgfqpoint{6.644297in}{2.870060in}}{\pgfqpoint{6.639907in}{2.880659in}}{\pgfqpoint{6.632093in}{2.888472in}}%
\pgfpathcurveto{\pgfqpoint{6.624280in}{2.896286in}}{\pgfqpoint{6.613681in}{2.900676in}}{\pgfqpoint{6.602631in}{2.900676in}}%
\pgfpathcurveto{\pgfqpoint{6.591580in}{2.900676in}}{\pgfqpoint{6.580981in}{2.896286in}}{\pgfqpoint{6.573168in}{2.888472in}}%
\pgfpathcurveto{\pgfqpoint{6.565354in}{2.880659in}}{\pgfqpoint{6.560964in}{2.870060in}}{\pgfqpoint{6.560964in}{2.859010in}}%
\pgfpathcurveto{\pgfqpoint{6.560964in}{2.847959in}}{\pgfqpoint{6.565354in}{2.837360in}}{\pgfqpoint{6.573168in}{2.829547in}}%
\pgfpathcurveto{\pgfqpoint{6.580981in}{2.821733in}}{\pgfqpoint{6.591580in}{2.817343in}}{\pgfqpoint{6.602631in}{2.817343in}}%
\pgfpathclose%
\pgfusepath{stroke,fill}%
\end{pgfscope}%
\begin{pgfscope}%
\pgfpathrectangle{\pgfqpoint{0.570343in}{0.331635in}}{\pgfqpoint{9.300000in}{7.700000in}}%
\pgfusepath{clip}%
\pgfsetbuttcap%
\pgfsetroundjoin%
\definecolor{currentfill}{rgb}{1.000000,0.705882,0.509804}%
\pgfsetfillcolor{currentfill}%
\pgfsetlinewidth{0.481800pt}%
\definecolor{currentstroke}{rgb}{1.000000,1.000000,1.000000}%
\pgfsetstrokecolor{currentstroke}%
\pgfsetdash{}{0pt}%
\pgfpathmoveto{\pgfqpoint{1.957036in}{6.560089in}}%
\pgfpathcurveto{\pgfqpoint{1.968087in}{6.560089in}}{\pgfqpoint{1.978686in}{6.564479in}}{\pgfqpoint{1.986499in}{6.572292in}}%
\pgfpathcurveto{\pgfqpoint{1.994313in}{6.580106in}}{\pgfqpoint{1.998703in}{6.590705in}}{\pgfqpoint{1.998703in}{6.601755in}}%
\pgfpathcurveto{\pgfqpoint{1.998703in}{6.612805in}}{\pgfqpoint{1.994313in}{6.623404in}}{\pgfqpoint{1.986499in}{6.631218in}}%
\pgfpathcurveto{\pgfqpoint{1.978686in}{6.639032in}}{\pgfqpoint{1.968087in}{6.643422in}}{\pgfqpoint{1.957036in}{6.643422in}}%
\pgfpathcurveto{\pgfqpoint{1.945986in}{6.643422in}}{\pgfqpoint{1.935387in}{6.639032in}}{\pgfqpoint{1.927574in}{6.631218in}}%
\pgfpathcurveto{\pgfqpoint{1.919760in}{6.623404in}}{\pgfqpoint{1.915370in}{6.612805in}}{\pgfqpoint{1.915370in}{6.601755in}}%
\pgfpathcurveto{\pgfqpoint{1.915370in}{6.590705in}}{\pgfqpoint{1.919760in}{6.580106in}}{\pgfqpoint{1.927574in}{6.572292in}}%
\pgfpathcurveto{\pgfqpoint{1.935387in}{6.564479in}}{\pgfqpoint{1.945986in}{6.560089in}}{\pgfqpoint{1.957036in}{6.560089in}}%
\pgfpathclose%
\pgfusepath{stroke,fill}%
\end{pgfscope}%
\begin{pgfscope}%
\pgfpathrectangle{\pgfqpoint{0.570343in}{0.331635in}}{\pgfqpoint{9.300000in}{7.700000in}}%
\pgfusepath{clip}%
\pgfsetbuttcap%
\pgfsetroundjoin%
\definecolor{currentfill}{rgb}{1.000000,0.705882,0.509804}%
\pgfsetfillcolor{currentfill}%
\pgfsetlinewidth{0.481800pt}%
\definecolor{currentstroke}{rgb}{1.000000,1.000000,1.000000}%
\pgfsetstrokecolor{currentstroke}%
\pgfsetdash{}{0pt}%
\pgfpathmoveto{\pgfqpoint{7.683729in}{6.257111in}}%
\pgfpathcurveto{\pgfqpoint{7.694779in}{6.257111in}}{\pgfqpoint{7.705378in}{6.261501in}}{\pgfqpoint{7.713192in}{6.269315in}}%
\pgfpathcurveto{\pgfqpoint{7.721005in}{6.277128in}}{\pgfqpoint{7.725395in}{6.287727in}}{\pgfqpoint{7.725395in}{6.298777in}}%
\pgfpathcurveto{\pgfqpoint{7.725395in}{6.309827in}}{\pgfqpoint{7.721005in}{6.320427in}}{\pgfqpoint{7.713192in}{6.328240in}}%
\pgfpathcurveto{\pgfqpoint{7.705378in}{6.336054in}}{\pgfqpoint{7.694779in}{6.340444in}}{\pgfqpoint{7.683729in}{6.340444in}}%
\pgfpathcurveto{\pgfqpoint{7.672679in}{6.340444in}}{\pgfqpoint{7.662080in}{6.336054in}}{\pgfqpoint{7.654266in}{6.328240in}}%
\pgfpathcurveto{\pgfqpoint{7.646452in}{6.320427in}}{\pgfqpoint{7.642062in}{6.309827in}}{\pgfqpoint{7.642062in}{6.298777in}}%
\pgfpathcurveto{\pgfqpoint{7.642062in}{6.287727in}}{\pgfqpoint{7.646452in}{6.277128in}}{\pgfqpoint{7.654266in}{6.269315in}}%
\pgfpathcurveto{\pgfqpoint{7.662080in}{6.261501in}}{\pgfqpoint{7.672679in}{6.257111in}}{\pgfqpoint{7.683729in}{6.257111in}}%
\pgfpathclose%
\pgfusepath{stroke,fill}%
\end{pgfscope}%
\begin{pgfscope}%
\pgfpathrectangle{\pgfqpoint{0.570343in}{0.331635in}}{\pgfqpoint{9.300000in}{7.700000in}}%
\pgfusepath{clip}%
\pgfsetbuttcap%
\pgfsetroundjoin%
\definecolor{currentfill}{rgb}{1.000000,0.705882,0.509804}%
\pgfsetfillcolor{currentfill}%
\pgfsetlinewidth{0.481800pt}%
\definecolor{currentstroke}{rgb}{1.000000,1.000000,1.000000}%
\pgfsetstrokecolor{currentstroke}%
\pgfsetdash{}{0pt}%
\pgfpathmoveto{\pgfqpoint{8.418297in}{5.134422in}}%
\pgfpathcurveto{\pgfqpoint{8.429348in}{5.134422in}}{\pgfqpoint{8.439947in}{5.138813in}}{\pgfqpoint{8.447760in}{5.146626in}}%
\pgfpathcurveto{\pgfqpoint{8.455574in}{5.154440in}}{\pgfqpoint{8.459964in}{5.165039in}}{\pgfqpoint{8.459964in}{5.176089in}}%
\pgfpathcurveto{\pgfqpoint{8.459964in}{5.187139in}}{\pgfqpoint{8.455574in}{5.197738in}}{\pgfqpoint{8.447760in}{5.205552in}}%
\pgfpathcurveto{\pgfqpoint{8.439947in}{5.213365in}}{\pgfqpoint{8.429348in}{5.217756in}}{\pgfqpoint{8.418297in}{5.217756in}}%
\pgfpathcurveto{\pgfqpoint{8.407247in}{5.217756in}}{\pgfqpoint{8.396648in}{5.213365in}}{\pgfqpoint{8.388835in}{5.205552in}}%
\pgfpathcurveto{\pgfqpoint{8.381021in}{5.197738in}}{\pgfqpoint{8.376631in}{5.187139in}}{\pgfqpoint{8.376631in}{5.176089in}}%
\pgfpathcurveto{\pgfqpoint{8.376631in}{5.165039in}}{\pgfqpoint{8.381021in}{5.154440in}}{\pgfqpoint{8.388835in}{5.146626in}}%
\pgfpathcurveto{\pgfqpoint{8.396648in}{5.138813in}}{\pgfqpoint{8.407247in}{5.134422in}}{\pgfqpoint{8.418297in}{5.134422in}}%
\pgfpathclose%
\pgfusepath{stroke,fill}%
\end{pgfscope}%
\begin{pgfscope}%
\pgfpathrectangle{\pgfqpoint{0.570343in}{0.331635in}}{\pgfqpoint{9.300000in}{7.700000in}}%
\pgfusepath{clip}%
\pgfsetbuttcap%
\pgfsetroundjoin%
\definecolor{currentfill}{rgb}{1.000000,0.705882,0.509804}%
\pgfsetfillcolor{currentfill}%
\pgfsetlinewidth{0.481800pt}%
\definecolor{currentstroke}{rgb}{1.000000,1.000000,1.000000}%
\pgfsetstrokecolor{currentstroke}%
\pgfsetdash{}{0pt}%
\pgfpathmoveto{\pgfqpoint{1.819365in}{3.967926in}}%
\pgfpathcurveto{\pgfqpoint{1.830415in}{3.967926in}}{\pgfqpoint{1.841014in}{3.972317in}}{\pgfqpoint{1.848828in}{3.980130in}}%
\pgfpathcurveto{\pgfqpoint{1.856642in}{3.987944in}}{\pgfqpoint{1.861032in}{3.998543in}}{\pgfqpoint{1.861032in}{4.009593in}}%
\pgfpathcurveto{\pgfqpoint{1.861032in}{4.020643in}}{\pgfqpoint{1.856642in}{4.031242in}}{\pgfqpoint{1.848828in}{4.039056in}}%
\pgfpathcurveto{\pgfqpoint{1.841014in}{4.046870in}}{\pgfqpoint{1.830415in}{4.051260in}}{\pgfqpoint{1.819365in}{4.051260in}}%
\pgfpathcurveto{\pgfqpoint{1.808315in}{4.051260in}}{\pgfqpoint{1.797716in}{4.046870in}}{\pgfqpoint{1.789902in}{4.039056in}}%
\pgfpathcurveto{\pgfqpoint{1.782089in}{4.031242in}}{\pgfqpoint{1.777699in}{4.020643in}}{\pgfqpoint{1.777699in}{4.009593in}}%
\pgfpathcurveto{\pgfqpoint{1.777699in}{3.998543in}}{\pgfqpoint{1.782089in}{3.987944in}}{\pgfqpoint{1.789902in}{3.980130in}}%
\pgfpathcurveto{\pgfqpoint{1.797716in}{3.972317in}}{\pgfqpoint{1.808315in}{3.967926in}}{\pgfqpoint{1.819365in}{3.967926in}}%
\pgfpathclose%
\pgfusepath{stroke,fill}%
\end{pgfscope}%
\begin{pgfscope}%
\pgfpathrectangle{\pgfqpoint{0.570343in}{0.331635in}}{\pgfqpoint{9.300000in}{7.700000in}}%
\pgfusepath{clip}%
\pgfsetbuttcap%
\pgfsetroundjoin%
\definecolor{currentfill}{rgb}{1.000000,0.705882,0.509804}%
\pgfsetfillcolor{currentfill}%
\pgfsetlinewidth{0.481800pt}%
\definecolor{currentstroke}{rgb}{1.000000,1.000000,1.000000}%
\pgfsetstrokecolor{currentstroke}%
\pgfsetdash{}{0pt}%
\pgfpathmoveto{\pgfqpoint{3.672079in}{3.458964in}}%
\pgfpathcurveto{\pgfqpoint{3.683129in}{3.458964in}}{\pgfqpoint{3.693728in}{3.463354in}}{\pgfqpoint{3.701542in}{3.471168in}}%
\pgfpathcurveto{\pgfqpoint{3.709356in}{3.478982in}}{\pgfqpoint{3.713746in}{3.489581in}}{\pgfqpoint{3.713746in}{3.500631in}}%
\pgfpathcurveto{\pgfqpoint{3.713746in}{3.511681in}}{\pgfqpoint{3.709356in}{3.522280in}}{\pgfqpoint{3.701542in}{3.530094in}}%
\pgfpathcurveto{\pgfqpoint{3.693728in}{3.537907in}}{\pgfqpoint{3.683129in}{3.542297in}}{\pgfqpoint{3.672079in}{3.542297in}}%
\pgfpathcurveto{\pgfqpoint{3.661029in}{3.542297in}}{\pgfqpoint{3.650430in}{3.537907in}}{\pgfqpoint{3.642616in}{3.530094in}}%
\pgfpathcurveto{\pgfqpoint{3.634803in}{3.522280in}}{\pgfqpoint{3.630413in}{3.511681in}}{\pgfqpoint{3.630413in}{3.500631in}}%
\pgfpathcurveto{\pgfqpoint{3.630413in}{3.489581in}}{\pgfqpoint{3.634803in}{3.478982in}}{\pgfqpoint{3.642616in}{3.471168in}}%
\pgfpathcurveto{\pgfqpoint{3.650430in}{3.463354in}}{\pgfqpoint{3.661029in}{3.458964in}}{\pgfqpoint{3.672079in}{3.458964in}}%
\pgfpathclose%
\pgfusepath{stroke,fill}%
\end{pgfscope}%
\begin{pgfscope}%
\pgfpathrectangle{\pgfqpoint{0.570343in}{0.331635in}}{\pgfqpoint{9.300000in}{7.700000in}}%
\pgfusepath{clip}%
\pgfsetbuttcap%
\pgfsetroundjoin%
\definecolor{currentfill}{rgb}{1.000000,0.705882,0.509804}%
\pgfsetfillcolor{currentfill}%
\pgfsetlinewidth{0.481800pt}%
\definecolor{currentstroke}{rgb}{1.000000,1.000000,1.000000}%
\pgfsetstrokecolor{currentstroke}%
\pgfsetdash{}{0pt}%
\pgfpathmoveto{\pgfqpoint{2.042788in}{5.632611in}}%
\pgfpathcurveto{\pgfqpoint{2.053838in}{5.632611in}}{\pgfqpoint{2.064437in}{5.637001in}}{\pgfqpoint{2.072251in}{5.644815in}}%
\pgfpathcurveto{\pgfqpoint{2.080065in}{5.652628in}}{\pgfqpoint{2.084455in}{5.663227in}}{\pgfqpoint{2.084455in}{5.674277in}}%
\pgfpathcurveto{\pgfqpoint{2.084455in}{5.685328in}}{\pgfqpoint{2.080065in}{5.695927in}}{\pgfqpoint{2.072251in}{5.703740in}}%
\pgfpathcurveto{\pgfqpoint{2.064437in}{5.711554in}}{\pgfqpoint{2.053838in}{5.715944in}}{\pgfqpoint{2.042788in}{5.715944in}}%
\pgfpathcurveto{\pgfqpoint{2.031738in}{5.715944in}}{\pgfqpoint{2.021139in}{5.711554in}}{\pgfqpoint{2.013325in}{5.703740in}}%
\pgfpathcurveto{\pgfqpoint{2.005512in}{5.695927in}}{\pgfqpoint{2.001122in}{5.685328in}}{\pgfqpoint{2.001122in}{5.674277in}}%
\pgfpathcurveto{\pgfqpoint{2.001122in}{5.663227in}}{\pgfqpoint{2.005512in}{5.652628in}}{\pgfqpoint{2.013325in}{5.644815in}}%
\pgfpathcurveto{\pgfqpoint{2.021139in}{5.637001in}}{\pgfqpoint{2.031738in}{5.632611in}}{\pgfqpoint{2.042788in}{5.632611in}}%
\pgfpathclose%
\pgfusepath{stroke,fill}%
\end{pgfscope}%
\begin{pgfscope}%
\pgfpathrectangle{\pgfqpoint{0.570343in}{0.331635in}}{\pgfqpoint{9.300000in}{7.700000in}}%
\pgfusepath{clip}%
\pgfsetbuttcap%
\pgfsetroundjoin%
\definecolor{currentfill}{rgb}{1.000000,0.705882,0.509804}%
\pgfsetfillcolor{currentfill}%
\pgfsetlinewidth{0.481800pt}%
\definecolor{currentstroke}{rgb}{1.000000,1.000000,1.000000}%
\pgfsetstrokecolor{currentstroke}%
\pgfsetdash{}{0pt}%
\pgfpathmoveto{\pgfqpoint{4.877879in}{2.790593in}}%
\pgfpathcurveto{\pgfqpoint{4.888930in}{2.790593in}}{\pgfqpoint{4.899529in}{2.794983in}}{\pgfqpoint{4.907342in}{2.802797in}}%
\pgfpathcurveto{\pgfqpoint{4.915156in}{2.810611in}}{\pgfqpoint{4.919546in}{2.821210in}}{\pgfqpoint{4.919546in}{2.832260in}}%
\pgfpathcurveto{\pgfqpoint{4.919546in}{2.843310in}}{\pgfqpoint{4.915156in}{2.853909in}}{\pgfqpoint{4.907342in}{2.861723in}}%
\pgfpathcurveto{\pgfqpoint{4.899529in}{2.869536in}}{\pgfqpoint{4.888930in}{2.873926in}}{\pgfqpoint{4.877879in}{2.873926in}}%
\pgfpathcurveto{\pgfqpoint{4.866829in}{2.873926in}}{\pgfqpoint{4.856230in}{2.869536in}}{\pgfqpoint{4.848417in}{2.861723in}}%
\pgfpathcurveto{\pgfqpoint{4.840603in}{2.853909in}}{\pgfqpoint{4.836213in}{2.843310in}}{\pgfqpoint{4.836213in}{2.832260in}}%
\pgfpathcurveto{\pgfqpoint{4.836213in}{2.821210in}}{\pgfqpoint{4.840603in}{2.810611in}}{\pgfqpoint{4.848417in}{2.802797in}}%
\pgfpathcurveto{\pgfqpoint{4.856230in}{2.794983in}}{\pgfqpoint{4.866829in}{2.790593in}}{\pgfqpoint{4.877879in}{2.790593in}}%
\pgfpathclose%
\pgfusepath{stroke,fill}%
\end{pgfscope}%
\begin{pgfscope}%
\pgfpathrectangle{\pgfqpoint{0.570343in}{0.331635in}}{\pgfqpoint{9.300000in}{7.700000in}}%
\pgfusepath{clip}%
\pgfsetbuttcap%
\pgfsetroundjoin%
\definecolor{currentfill}{rgb}{0.631373,0.788235,0.956863}%
\pgfsetfillcolor{currentfill}%
\pgfsetlinewidth{1.003750pt}%
\definecolor{currentstroke}{rgb}{0.631373,0.788235,0.956863}%
\pgfsetstrokecolor{currentstroke}%
\pgfsetdash{}{0pt}%
\pgfsys@defobject{currentmarker}{\pgfqpoint{-0.041667in}{-0.041667in}}{\pgfqpoint{0.041667in}{0.041667in}}{%
\pgfpathmoveto{\pgfqpoint{0.000000in}{-0.041667in}}%
\pgfpathcurveto{\pgfqpoint{0.011050in}{-0.041667in}}{\pgfqpoint{0.021649in}{-0.037276in}}{\pgfqpoint{0.029463in}{-0.029463in}}%
\pgfpathcurveto{\pgfqpoint{0.037276in}{-0.021649in}}{\pgfqpoint{0.041667in}{-0.011050in}}{\pgfqpoint{0.041667in}{0.000000in}}%
\pgfpathcurveto{\pgfqpoint{0.041667in}{0.011050in}}{\pgfqpoint{0.037276in}{0.021649in}}{\pgfqpoint{0.029463in}{0.029463in}}%
\pgfpathcurveto{\pgfqpoint{0.021649in}{0.037276in}}{\pgfqpoint{0.011050in}{0.041667in}}{\pgfqpoint{0.000000in}{0.041667in}}%
\pgfpathcurveto{\pgfqpoint{-0.011050in}{0.041667in}}{\pgfqpoint{-0.021649in}{0.037276in}}{\pgfqpoint{-0.029463in}{0.029463in}}%
\pgfpathcurveto{\pgfqpoint{-0.037276in}{0.021649in}}{\pgfqpoint{-0.041667in}{0.011050in}}{\pgfqpoint{-0.041667in}{0.000000in}}%
\pgfpathcurveto{\pgfqpoint{-0.041667in}{-0.011050in}}{\pgfqpoint{-0.037276in}{-0.021649in}}{\pgfqpoint{-0.029463in}{-0.029463in}}%
\pgfpathcurveto{\pgfqpoint{-0.021649in}{-0.037276in}}{\pgfqpoint{-0.011050in}{-0.041667in}}{\pgfqpoint{0.000000in}{-0.041667in}}%
\pgfpathclose%
\pgfusepath{stroke,fill}%
}%
\end{pgfscope}%
\begin{pgfscope}%
\pgfpathrectangle{\pgfqpoint{0.570343in}{0.331635in}}{\pgfqpoint{9.300000in}{7.700000in}}%
\pgfusepath{clip}%
\pgfsetbuttcap%
\pgfsetroundjoin%
\definecolor{currentfill}{rgb}{1.000000,0.705882,0.509804}%
\pgfsetfillcolor{currentfill}%
\pgfsetlinewidth{1.003750pt}%
\definecolor{currentstroke}{rgb}{1.000000,0.705882,0.509804}%
\pgfsetstrokecolor{currentstroke}%
\pgfsetdash{}{0pt}%
\pgfsys@defobject{currentmarker}{\pgfqpoint{-0.041667in}{-0.041667in}}{\pgfqpoint{0.041667in}{0.041667in}}{%
\pgfpathmoveto{\pgfqpoint{0.000000in}{-0.041667in}}%
\pgfpathcurveto{\pgfqpoint{0.011050in}{-0.041667in}}{\pgfqpoint{0.021649in}{-0.037276in}}{\pgfqpoint{0.029463in}{-0.029463in}}%
\pgfpathcurveto{\pgfqpoint{0.037276in}{-0.021649in}}{\pgfqpoint{0.041667in}{-0.011050in}}{\pgfqpoint{0.041667in}{0.000000in}}%
\pgfpathcurveto{\pgfqpoint{0.041667in}{0.011050in}}{\pgfqpoint{0.037276in}{0.021649in}}{\pgfqpoint{0.029463in}{0.029463in}}%
\pgfpathcurveto{\pgfqpoint{0.021649in}{0.037276in}}{\pgfqpoint{0.011050in}{0.041667in}}{\pgfqpoint{0.000000in}{0.041667in}}%
\pgfpathcurveto{\pgfqpoint{-0.011050in}{0.041667in}}{\pgfqpoint{-0.021649in}{0.037276in}}{\pgfqpoint{-0.029463in}{0.029463in}}%
\pgfpathcurveto{\pgfqpoint{-0.037276in}{0.021649in}}{\pgfqpoint{-0.041667in}{0.011050in}}{\pgfqpoint{-0.041667in}{0.000000in}}%
\pgfpathcurveto{\pgfqpoint{-0.041667in}{-0.011050in}}{\pgfqpoint{-0.037276in}{-0.021649in}}{\pgfqpoint{-0.029463in}{-0.029463in}}%
\pgfpathcurveto{\pgfqpoint{-0.021649in}{-0.037276in}}{\pgfqpoint{-0.011050in}{-0.041667in}}{\pgfqpoint{0.000000in}{-0.041667in}}%
\pgfpathclose%
\pgfusepath{stroke,fill}%
}%
\end{pgfscope}%
\begin{pgfscope}%
\pgfsetbuttcap%
\pgfsetroundjoin%
\definecolor{currentfill}{rgb}{0.000000,0.000000,0.000000}%
\pgfsetfillcolor{currentfill}%
\pgfsetlinewidth{0.803000pt}%
\definecolor{currentstroke}{rgb}{0.000000,0.000000,0.000000}%
\pgfsetstrokecolor{currentstroke}%
\pgfsetdash{}{0pt}%
\pgfsys@defobject{currentmarker}{\pgfqpoint{0.000000in}{-0.048611in}}{\pgfqpoint{0.000000in}{0.000000in}}{%
\pgfpathmoveto{\pgfqpoint{0.000000in}{0.000000in}}%
\pgfpathlineto{\pgfqpoint{0.000000in}{-0.048611in}}%
\pgfusepath{stroke,fill}%
}%
\begin{pgfscope}%
\pgfsys@transformshift{1.291791in}{0.331635in}%
\pgfsys@useobject{currentmarker}{}%
\end{pgfscope}%
\end{pgfscope}%
\begin{pgfscope}%
\definecolor{textcolor}{rgb}{0.000000,0.000000,0.000000}%
\pgfsetstrokecolor{textcolor}%
\pgfsetfillcolor{textcolor}%
\pgftext[x=1.291791in,y=0.234413in,,top]{\color{textcolor}\sffamily\fontsize{10.000000}{12.000000}\selectfont \ensuremath{-}400}%
\end{pgfscope}%
\begin{pgfscope}%
\pgfsetbuttcap%
\pgfsetroundjoin%
\definecolor{currentfill}{rgb}{0.000000,0.000000,0.000000}%
\pgfsetfillcolor{currentfill}%
\pgfsetlinewidth{0.803000pt}%
\definecolor{currentstroke}{rgb}{0.000000,0.000000,0.000000}%
\pgfsetstrokecolor{currentstroke}%
\pgfsetdash{}{0pt}%
\pgfsys@defobject{currentmarker}{\pgfqpoint{0.000000in}{-0.048611in}}{\pgfqpoint{0.000000in}{0.000000in}}{%
\pgfpathmoveto{\pgfqpoint{0.000000in}{0.000000in}}%
\pgfpathlineto{\pgfqpoint{0.000000in}{-0.048611in}}%
\pgfusepath{stroke,fill}%
}%
\begin{pgfscope}%
\pgfsys@transformshift{3.107960in}{0.331635in}%
\pgfsys@useobject{currentmarker}{}%
\end{pgfscope}%
\end{pgfscope}%
\begin{pgfscope}%
\definecolor{textcolor}{rgb}{0.000000,0.000000,0.000000}%
\pgfsetstrokecolor{textcolor}%
\pgfsetfillcolor{textcolor}%
\pgftext[x=3.107960in,y=0.234413in,,top]{\color{textcolor}\sffamily\fontsize{10.000000}{12.000000}\selectfont \ensuremath{-}200}%
\end{pgfscope}%
\begin{pgfscope}%
\pgfsetbuttcap%
\pgfsetroundjoin%
\definecolor{currentfill}{rgb}{0.000000,0.000000,0.000000}%
\pgfsetfillcolor{currentfill}%
\pgfsetlinewidth{0.803000pt}%
\definecolor{currentstroke}{rgb}{0.000000,0.000000,0.000000}%
\pgfsetstrokecolor{currentstroke}%
\pgfsetdash{}{0pt}%
\pgfsys@defobject{currentmarker}{\pgfqpoint{0.000000in}{-0.048611in}}{\pgfqpoint{0.000000in}{0.000000in}}{%
\pgfpathmoveto{\pgfqpoint{0.000000in}{0.000000in}}%
\pgfpathlineto{\pgfqpoint{0.000000in}{-0.048611in}}%
\pgfusepath{stroke,fill}%
}%
\begin{pgfscope}%
\pgfsys@transformshift{4.924129in}{0.331635in}%
\pgfsys@useobject{currentmarker}{}%
\end{pgfscope}%
\end{pgfscope}%
\begin{pgfscope}%
\definecolor{textcolor}{rgb}{0.000000,0.000000,0.000000}%
\pgfsetstrokecolor{textcolor}%
\pgfsetfillcolor{textcolor}%
\pgftext[x=4.924129in,y=0.234413in,,top]{\color{textcolor}\sffamily\fontsize{10.000000}{12.000000}\selectfont 0}%
\end{pgfscope}%
\begin{pgfscope}%
\pgfsetbuttcap%
\pgfsetroundjoin%
\definecolor{currentfill}{rgb}{0.000000,0.000000,0.000000}%
\pgfsetfillcolor{currentfill}%
\pgfsetlinewidth{0.803000pt}%
\definecolor{currentstroke}{rgb}{0.000000,0.000000,0.000000}%
\pgfsetstrokecolor{currentstroke}%
\pgfsetdash{}{0pt}%
\pgfsys@defobject{currentmarker}{\pgfqpoint{0.000000in}{-0.048611in}}{\pgfqpoint{0.000000in}{0.000000in}}{%
\pgfpathmoveto{\pgfqpoint{0.000000in}{0.000000in}}%
\pgfpathlineto{\pgfqpoint{0.000000in}{-0.048611in}}%
\pgfusepath{stroke,fill}%
}%
\begin{pgfscope}%
\pgfsys@transformshift{6.740298in}{0.331635in}%
\pgfsys@useobject{currentmarker}{}%
\end{pgfscope}%
\end{pgfscope}%
\begin{pgfscope}%
\definecolor{textcolor}{rgb}{0.000000,0.000000,0.000000}%
\pgfsetstrokecolor{textcolor}%
\pgfsetfillcolor{textcolor}%
\pgftext[x=6.740298in,y=0.234413in,,top]{\color{textcolor}\sffamily\fontsize{10.000000}{12.000000}\selectfont 200}%
\end{pgfscope}%
\begin{pgfscope}%
\pgfsetbuttcap%
\pgfsetroundjoin%
\definecolor{currentfill}{rgb}{0.000000,0.000000,0.000000}%
\pgfsetfillcolor{currentfill}%
\pgfsetlinewidth{0.803000pt}%
\definecolor{currentstroke}{rgb}{0.000000,0.000000,0.000000}%
\pgfsetstrokecolor{currentstroke}%
\pgfsetdash{}{0pt}%
\pgfsys@defobject{currentmarker}{\pgfqpoint{0.000000in}{-0.048611in}}{\pgfqpoint{0.000000in}{0.000000in}}{%
\pgfpathmoveto{\pgfqpoint{0.000000in}{0.000000in}}%
\pgfpathlineto{\pgfqpoint{0.000000in}{-0.048611in}}%
\pgfusepath{stroke,fill}%
}%
\begin{pgfscope}%
\pgfsys@transformshift{8.556467in}{0.331635in}%
\pgfsys@useobject{currentmarker}{}%
\end{pgfscope}%
\end{pgfscope}%
\begin{pgfscope}%
\definecolor{textcolor}{rgb}{0.000000,0.000000,0.000000}%
\pgfsetstrokecolor{textcolor}%
\pgfsetfillcolor{textcolor}%
\pgftext[x=8.556467in,y=0.234413in,,top]{\color{textcolor}\sffamily\fontsize{10.000000}{12.000000}\selectfont 400}%
\end{pgfscope}%
\begin{pgfscope}%
\pgfsetbuttcap%
\pgfsetroundjoin%
\definecolor{currentfill}{rgb}{0.000000,0.000000,0.000000}%
\pgfsetfillcolor{currentfill}%
\pgfsetlinewidth{0.803000pt}%
\definecolor{currentstroke}{rgb}{0.000000,0.000000,0.000000}%
\pgfsetstrokecolor{currentstroke}%
\pgfsetdash{}{0pt}%
\pgfsys@defobject{currentmarker}{\pgfqpoint{-0.048611in}{0.000000in}}{\pgfqpoint{-0.000000in}{0.000000in}}{%
\pgfpathmoveto{\pgfqpoint{-0.000000in}{0.000000in}}%
\pgfpathlineto{\pgfqpoint{-0.048611in}{0.000000in}}%
\pgfusepath{stroke,fill}%
}%
\begin{pgfscope}%
\pgfsys@transformshift{0.570343in}{1.104137in}%
\pgfsys@useobject{currentmarker}{}%
\end{pgfscope}%
\end{pgfscope}%
\begin{pgfscope}%
\definecolor{textcolor}{rgb}{0.000000,0.000000,0.000000}%
\pgfsetstrokecolor{textcolor}%
\pgfsetfillcolor{textcolor}%
\pgftext[x=0.100000in, y=1.051375in, left, base]{\color{textcolor}\sffamily\fontsize{10.000000}{12.000000}\selectfont \ensuremath{-}400}%
\end{pgfscope}%
\begin{pgfscope}%
\pgfsetbuttcap%
\pgfsetroundjoin%
\definecolor{currentfill}{rgb}{0.000000,0.000000,0.000000}%
\pgfsetfillcolor{currentfill}%
\pgfsetlinewidth{0.803000pt}%
\definecolor{currentstroke}{rgb}{0.000000,0.000000,0.000000}%
\pgfsetstrokecolor{currentstroke}%
\pgfsetdash{}{0pt}%
\pgfsys@defobject{currentmarker}{\pgfqpoint{-0.048611in}{0.000000in}}{\pgfqpoint{-0.000000in}{0.000000in}}{%
\pgfpathmoveto{\pgfqpoint{-0.000000in}{0.000000in}}%
\pgfpathlineto{\pgfqpoint{-0.048611in}{0.000000in}}%
\pgfusepath{stroke,fill}%
}%
\begin{pgfscope}%
\pgfsys@transformshift{0.570343in}{2.777216in}%
\pgfsys@useobject{currentmarker}{}%
\end{pgfscope}%
\end{pgfscope}%
\begin{pgfscope}%
\definecolor{textcolor}{rgb}{0.000000,0.000000,0.000000}%
\pgfsetstrokecolor{textcolor}%
\pgfsetfillcolor{textcolor}%
\pgftext[x=0.100000in, y=2.724454in, left, base]{\color{textcolor}\sffamily\fontsize{10.000000}{12.000000}\selectfont \ensuremath{-}200}%
\end{pgfscope}%
\begin{pgfscope}%
\pgfsetbuttcap%
\pgfsetroundjoin%
\definecolor{currentfill}{rgb}{0.000000,0.000000,0.000000}%
\pgfsetfillcolor{currentfill}%
\pgfsetlinewidth{0.803000pt}%
\definecolor{currentstroke}{rgb}{0.000000,0.000000,0.000000}%
\pgfsetstrokecolor{currentstroke}%
\pgfsetdash{}{0pt}%
\pgfsys@defobject{currentmarker}{\pgfqpoint{-0.048611in}{0.000000in}}{\pgfqpoint{-0.000000in}{0.000000in}}{%
\pgfpathmoveto{\pgfqpoint{-0.000000in}{0.000000in}}%
\pgfpathlineto{\pgfqpoint{-0.048611in}{0.000000in}}%
\pgfusepath{stroke,fill}%
}%
\begin{pgfscope}%
\pgfsys@transformshift{0.570343in}{4.450295in}%
\pgfsys@useobject{currentmarker}{}%
\end{pgfscope}%
\end{pgfscope}%
\begin{pgfscope}%
\definecolor{textcolor}{rgb}{0.000000,0.000000,0.000000}%
\pgfsetstrokecolor{textcolor}%
\pgfsetfillcolor{textcolor}%
\pgftext[x=0.384756in, y=4.397533in, left, base]{\color{textcolor}\sffamily\fontsize{10.000000}{12.000000}\selectfont 0}%
\end{pgfscope}%
\begin{pgfscope}%
\pgfsetbuttcap%
\pgfsetroundjoin%
\definecolor{currentfill}{rgb}{0.000000,0.000000,0.000000}%
\pgfsetfillcolor{currentfill}%
\pgfsetlinewidth{0.803000pt}%
\definecolor{currentstroke}{rgb}{0.000000,0.000000,0.000000}%
\pgfsetstrokecolor{currentstroke}%
\pgfsetdash{}{0pt}%
\pgfsys@defobject{currentmarker}{\pgfqpoint{-0.048611in}{0.000000in}}{\pgfqpoint{-0.000000in}{0.000000in}}{%
\pgfpathmoveto{\pgfqpoint{-0.000000in}{0.000000in}}%
\pgfpathlineto{\pgfqpoint{-0.048611in}{0.000000in}}%
\pgfusepath{stroke,fill}%
}%
\begin{pgfscope}%
\pgfsys@transformshift{0.570343in}{6.123374in}%
\pgfsys@useobject{currentmarker}{}%
\end{pgfscope}%
\end{pgfscope}%
\begin{pgfscope}%
\definecolor{textcolor}{rgb}{0.000000,0.000000,0.000000}%
\pgfsetstrokecolor{textcolor}%
\pgfsetfillcolor{textcolor}%
\pgftext[x=0.208025in, y=6.070612in, left, base]{\color{textcolor}\sffamily\fontsize{10.000000}{12.000000}\selectfont 200}%
\end{pgfscope}%
\begin{pgfscope}%
\pgfsetbuttcap%
\pgfsetroundjoin%
\definecolor{currentfill}{rgb}{0.000000,0.000000,0.000000}%
\pgfsetfillcolor{currentfill}%
\pgfsetlinewidth{0.803000pt}%
\definecolor{currentstroke}{rgb}{0.000000,0.000000,0.000000}%
\pgfsetstrokecolor{currentstroke}%
\pgfsetdash{}{0pt}%
\pgfsys@defobject{currentmarker}{\pgfqpoint{-0.048611in}{0.000000in}}{\pgfqpoint{-0.000000in}{0.000000in}}{%
\pgfpathmoveto{\pgfqpoint{-0.000000in}{0.000000in}}%
\pgfpathlineto{\pgfqpoint{-0.048611in}{0.000000in}}%
\pgfusepath{stroke,fill}%
}%
\begin{pgfscope}%
\pgfsys@transformshift{0.570343in}{7.796452in}%
\pgfsys@useobject{currentmarker}{}%
\end{pgfscope}%
\end{pgfscope}%
\begin{pgfscope}%
\definecolor{textcolor}{rgb}{0.000000,0.000000,0.000000}%
\pgfsetstrokecolor{textcolor}%
\pgfsetfillcolor{textcolor}%
\pgftext[x=0.208025in, y=7.743691in, left, base]{\color{textcolor}\sffamily\fontsize{10.000000}{12.000000}\selectfont 400}%
\end{pgfscope}%
\begin{pgfscope}%
\pgfpathrectangle{\pgfqpoint{0.570343in}{0.331635in}}{\pgfqpoint{9.300000in}{7.700000in}}%
\pgfusepath{clip}%
\pgfsetrectcap%
\pgfsetroundjoin%
\pgfsetlinewidth{1.505625pt}%
\definecolor{currentstroke}{rgb}{0.631373,0.788235,0.956863}%
\pgfsetstrokecolor{currentstroke}%
\pgfsetstrokeopacity{0.800000}%
\pgfsetdash{}{0pt}%
\pgfpathmoveto{\pgfqpoint{6.993440in}{5.428474in}}%
\pgfpathlineto{\pgfqpoint{5.170297in}{3.838787in}}%
\pgfusepath{stroke}%
\end{pgfscope}%
\begin{pgfscope}%
\pgfpathrectangle{\pgfqpoint{0.570343in}{0.331635in}}{\pgfqpoint{9.300000in}{7.700000in}}%
\pgfusepath{clip}%
\pgfsetrectcap%
\pgfsetroundjoin%
\pgfsetlinewidth{1.505625pt}%
\definecolor{currentstroke}{rgb}{0.631373,0.788235,0.956863}%
\pgfsetstrokecolor{currentstroke}%
\pgfsetstrokeopacity{0.800000}%
\pgfsetdash{}{0pt}%
\pgfpathmoveto{\pgfqpoint{7.514448in}{1.678445in}}%
\pgfpathlineto{\pgfqpoint{5.170297in}{3.838787in}}%
\pgfusepath{stroke}%
\end{pgfscope}%
\begin{pgfscope}%
\pgfpathrectangle{\pgfqpoint{0.570343in}{0.331635in}}{\pgfqpoint{9.300000in}{7.700000in}}%
\pgfusepath{clip}%
\pgfsetrectcap%
\pgfsetroundjoin%
\pgfsetlinewidth{1.505625pt}%
\definecolor{currentstroke}{rgb}{0.631373,0.788235,0.956863}%
\pgfsetstrokecolor{currentstroke}%
\pgfsetstrokeopacity{0.800000}%
\pgfsetdash{}{0pt}%
\pgfpathmoveto{\pgfqpoint{5.023692in}{7.180984in}}%
\pgfpathlineto{\pgfqpoint{5.170297in}{3.838787in}}%
\pgfusepath{stroke}%
\end{pgfscope}%
\begin{pgfscope}%
\pgfpathrectangle{\pgfqpoint{0.570343in}{0.331635in}}{\pgfqpoint{9.300000in}{7.700000in}}%
\pgfusepath{clip}%
\pgfsetrectcap%
\pgfsetroundjoin%
\pgfsetlinewidth{1.505625pt}%
\definecolor{currentstroke}{rgb}{0.631373,0.788235,0.956863}%
\pgfsetstrokecolor{currentstroke}%
\pgfsetstrokeopacity{0.800000}%
\pgfsetdash{}{0pt}%
\pgfpathmoveto{\pgfqpoint{3.832172in}{4.301862in}}%
\pgfpathlineto{\pgfqpoint{5.170297in}{3.838787in}}%
\pgfusepath{stroke}%
\end{pgfscope}%
\begin{pgfscope}%
\pgfpathrectangle{\pgfqpoint{0.570343in}{0.331635in}}{\pgfqpoint{9.300000in}{7.700000in}}%
\pgfusepath{clip}%
\pgfsetrectcap%
\pgfsetroundjoin%
\pgfsetlinewidth{1.505625pt}%
\definecolor{currentstroke}{rgb}{0.631373,0.788235,0.956863}%
\pgfsetstrokecolor{currentstroke}%
\pgfsetstrokeopacity{0.800000}%
\pgfsetdash{}{0pt}%
\pgfpathmoveto{\pgfqpoint{5.168249in}{4.675333in}}%
\pgfpathlineto{\pgfqpoint{5.170297in}{3.838787in}}%
\pgfusepath{stroke}%
\end{pgfscope}%
\begin{pgfscope}%
\pgfpathrectangle{\pgfqpoint{0.570343in}{0.331635in}}{\pgfqpoint{9.300000in}{7.700000in}}%
\pgfusepath{clip}%
\pgfsetrectcap%
\pgfsetroundjoin%
\pgfsetlinewidth{1.505625pt}%
\definecolor{currentstroke}{rgb}{0.631373,0.788235,0.956863}%
\pgfsetstrokecolor{currentstroke}%
\pgfsetstrokeopacity{0.800000}%
\pgfsetdash{}{0pt}%
\pgfpathmoveto{\pgfqpoint{4.194348in}{1.414034in}}%
\pgfpathlineto{\pgfqpoint{5.170297in}{3.838787in}}%
\pgfusepath{stroke}%
\end{pgfscope}%
\begin{pgfscope}%
\pgfpathrectangle{\pgfqpoint{0.570343in}{0.331635in}}{\pgfqpoint{9.300000in}{7.700000in}}%
\pgfusepath{clip}%
\pgfsetrectcap%
\pgfsetroundjoin%
\pgfsetlinewidth{1.505625pt}%
\definecolor{currentstroke}{rgb}{0.631373,0.788235,0.956863}%
\pgfsetstrokecolor{currentstroke}%
\pgfsetstrokeopacity{0.800000}%
\pgfsetdash{}{0pt}%
\pgfpathmoveto{\pgfqpoint{2.750613in}{5.841837in}}%
\pgfpathlineto{\pgfqpoint{5.170297in}{3.838787in}}%
\pgfusepath{stroke}%
\end{pgfscope}%
\begin{pgfscope}%
\pgfpathrectangle{\pgfqpoint{0.570343in}{0.331635in}}{\pgfqpoint{9.300000in}{7.700000in}}%
\pgfusepath{clip}%
\pgfsetrectcap%
\pgfsetroundjoin%
\pgfsetlinewidth{1.505625pt}%
\definecolor{currentstroke}{rgb}{0.631373,0.788235,0.956863}%
\pgfsetstrokecolor{currentstroke}%
\pgfsetstrokeopacity{0.800000}%
\pgfsetdash{}{0pt}%
\pgfpathmoveto{\pgfqpoint{3.118617in}{2.397944in}}%
\pgfpathlineto{\pgfqpoint{5.170297in}{3.838787in}}%
\pgfusepath{stroke}%
\end{pgfscope}%
\begin{pgfscope}%
\pgfpathrectangle{\pgfqpoint{0.570343in}{0.331635in}}{\pgfqpoint{9.300000in}{7.700000in}}%
\pgfusepath{clip}%
\pgfsetrectcap%
\pgfsetroundjoin%
\pgfsetlinewidth{1.505625pt}%
\definecolor{currentstroke}{rgb}{0.631373,0.788235,0.956863}%
\pgfsetstrokecolor{currentstroke}%
\pgfsetstrokeopacity{0.800000}%
\pgfsetdash{}{0pt}%
\pgfpathmoveto{\pgfqpoint{5.057402in}{1.890172in}}%
\pgfpathlineto{\pgfqpoint{5.170297in}{3.838787in}}%
\pgfusepath{stroke}%
\end{pgfscope}%
\begin{pgfscope}%
\pgfpathrectangle{\pgfqpoint{0.570343in}{0.331635in}}{\pgfqpoint{9.300000in}{7.700000in}}%
\pgfusepath{clip}%
\pgfsetrectcap%
\pgfsetroundjoin%
\pgfsetlinewidth{1.505625pt}%
\definecolor{currentstroke}{rgb}{0.631373,0.788235,0.956863}%
\pgfsetstrokecolor{currentstroke}%
\pgfsetstrokeopacity{0.800000}%
\pgfsetdash{}{0pt}%
\pgfpathmoveto{\pgfqpoint{5.910435in}{2.357904in}}%
\pgfpathlineto{\pgfqpoint{5.170297in}{3.838787in}}%
\pgfusepath{stroke}%
\end{pgfscope}%
\begin{pgfscope}%
\pgfpathrectangle{\pgfqpoint{0.570343in}{0.331635in}}{\pgfqpoint{9.300000in}{7.700000in}}%
\pgfusepath{clip}%
\pgfsetrectcap%
\pgfsetroundjoin%
\pgfsetlinewidth{1.505625pt}%
\definecolor{currentstroke}{rgb}{0.631373,0.788235,0.956863}%
\pgfsetstrokecolor{currentstroke}%
\pgfsetstrokeopacity{0.800000}%
\pgfsetdash{}{0pt}%
\pgfpathmoveto{\pgfqpoint{5.846427in}{5.296493in}}%
\pgfpathlineto{\pgfqpoint{5.170297in}{3.838787in}}%
\pgfusepath{stroke}%
\end{pgfscope}%
\begin{pgfscope}%
\pgfpathrectangle{\pgfqpoint{0.570343in}{0.331635in}}{\pgfqpoint{9.300000in}{7.700000in}}%
\pgfusepath{clip}%
\pgfsetrectcap%
\pgfsetroundjoin%
\pgfsetlinewidth{1.505625pt}%
\definecolor{currentstroke}{rgb}{0.631373,0.788235,0.956863}%
\pgfsetstrokecolor{currentstroke}%
\pgfsetstrokeopacity{0.800000}%
\pgfsetdash{}{0pt}%
\pgfpathmoveto{\pgfqpoint{4.284170in}{4.791321in}}%
\pgfpathlineto{\pgfqpoint{5.170297in}{3.838787in}}%
\pgfusepath{stroke}%
\end{pgfscope}%
\begin{pgfscope}%
\pgfpathrectangle{\pgfqpoint{0.570343in}{0.331635in}}{\pgfqpoint{9.300000in}{7.700000in}}%
\pgfusepath{clip}%
\pgfsetrectcap%
\pgfsetroundjoin%
\pgfsetlinewidth{1.505625pt}%
\definecolor{currentstroke}{rgb}{0.631373,0.788235,0.956863}%
\pgfsetstrokecolor{currentstroke}%
\pgfsetstrokeopacity{0.800000}%
\pgfsetdash{}{0pt}%
\pgfpathmoveto{\pgfqpoint{1.673868in}{2.908498in}}%
\pgfpathlineto{\pgfqpoint{5.170297in}{3.838787in}}%
\pgfusepath{stroke}%
\end{pgfscope}%
\begin{pgfscope}%
\pgfpathrectangle{\pgfqpoint{0.570343in}{0.331635in}}{\pgfqpoint{9.300000in}{7.700000in}}%
\pgfusepath{clip}%
\pgfsetrectcap%
\pgfsetroundjoin%
\pgfsetlinewidth{1.505625pt}%
\definecolor{currentstroke}{rgb}{0.631373,0.788235,0.956863}%
\pgfsetstrokecolor{currentstroke}%
\pgfsetstrokeopacity{0.800000}%
\pgfsetdash{}{0pt}%
\pgfpathmoveto{\pgfqpoint{4.453376in}{3.433671in}}%
\pgfpathlineto{\pgfqpoint{5.170297in}{3.838787in}}%
\pgfusepath{stroke}%
\end{pgfscope}%
\begin{pgfscope}%
\pgfpathrectangle{\pgfqpoint{0.570343in}{0.331635in}}{\pgfqpoint{9.300000in}{7.700000in}}%
\pgfusepath{clip}%
\pgfsetrectcap%
\pgfsetroundjoin%
\pgfsetlinewidth{1.505625pt}%
\definecolor{currentstroke}{rgb}{0.631373,0.788235,0.956863}%
\pgfsetstrokecolor{currentstroke}%
\pgfsetstrokeopacity{0.800000}%
\pgfsetdash{}{0pt}%
\pgfpathmoveto{\pgfqpoint{5.511077in}{3.387822in}}%
\pgfpathlineto{\pgfqpoint{5.170297in}{3.838787in}}%
\pgfusepath{stroke}%
\end{pgfscope}%
\begin{pgfscope}%
\pgfpathrectangle{\pgfqpoint{0.570343in}{0.331635in}}{\pgfqpoint{9.300000in}{7.700000in}}%
\pgfusepath{clip}%
\pgfsetrectcap%
\pgfsetroundjoin%
\pgfsetlinewidth{1.505625pt}%
\definecolor{currentstroke}{rgb}{0.631373,0.788235,0.956863}%
\pgfsetstrokecolor{currentstroke}%
\pgfsetstrokeopacity{0.800000}%
\pgfsetdash{}{0pt}%
\pgfpathmoveto{\pgfqpoint{6.465570in}{3.579135in}}%
\pgfpathlineto{\pgfqpoint{5.170297in}{3.838787in}}%
\pgfusepath{stroke}%
\end{pgfscope}%
\begin{pgfscope}%
\pgfpathrectangle{\pgfqpoint{0.570343in}{0.331635in}}{\pgfqpoint{9.300000in}{7.700000in}}%
\pgfusepath{clip}%
\pgfsetrectcap%
\pgfsetroundjoin%
\pgfsetlinewidth{1.505625pt}%
\definecolor{currentstroke}{rgb}{0.631373,0.788235,0.956863}%
\pgfsetstrokecolor{currentstroke}%
\pgfsetstrokeopacity{0.800000}%
\pgfsetdash{}{0pt}%
\pgfpathmoveto{\pgfqpoint{5.213120in}{0.681635in}}%
\pgfpathlineto{\pgfqpoint{5.170297in}{3.838787in}}%
\pgfusepath{stroke}%
\end{pgfscope}%
\begin{pgfscope}%
\pgfpathrectangle{\pgfqpoint{0.570343in}{0.331635in}}{\pgfqpoint{9.300000in}{7.700000in}}%
\pgfusepath{clip}%
\pgfsetrectcap%
\pgfsetroundjoin%
\pgfsetlinewidth{1.505625pt}%
\definecolor{currentstroke}{rgb}{0.631373,0.788235,0.956863}%
\pgfsetstrokecolor{currentstroke}%
\pgfsetstrokeopacity{0.800000}%
\pgfsetdash{}{0pt}%
\pgfpathmoveto{\pgfqpoint{9.447616in}{4.728782in}}%
\pgfpathlineto{\pgfqpoint{5.170297in}{3.838787in}}%
\pgfusepath{stroke}%
\end{pgfscope}%
\begin{pgfscope}%
\pgfpathrectangle{\pgfqpoint{0.570343in}{0.331635in}}{\pgfqpoint{9.300000in}{7.700000in}}%
\pgfusepath{clip}%
\pgfsetrectcap%
\pgfsetroundjoin%
\pgfsetlinewidth{1.505625pt}%
\definecolor{currentstroke}{rgb}{0.631373,0.788235,0.956863}%
\pgfsetstrokecolor{currentstroke}%
\pgfsetstrokeopacity{0.800000}%
\pgfsetdash{}{0pt}%
\pgfpathmoveto{\pgfqpoint{6.337177in}{4.273623in}}%
\pgfpathlineto{\pgfqpoint{5.170297in}{3.838787in}}%
\pgfusepath{stroke}%
\end{pgfscope}%
\begin{pgfscope}%
\pgfpathrectangle{\pgfqpoint{0.570343in}{0.331635in}}{\pgfqpoint{9.300000in}{7.700000in}}%
\pgfusepath{clip}%
\pgfsetrectcap%
\pgfsetroundjoin%
\pgfsetlinewidth{1.505625pt}%
\definecolor{currentstroke}{rgb}{0.631373,0.788235,0.956863}%
\pgfsetstrokecolor{currentstroke}%
\pgfsetstrokeopacity{0.800000}%
\pgfsetdash{}{0pt}%
\pgfpathmoveto{\pgfqpoint{3.060240in}{4.547560in}}%
\pgfpathlineto{\pgfqpoint{5.170297in}{3.838787in}}%
\pgfusepath{stroke}%
\end{pgfscope}%
\begin{pgfscope}%
\pgfpathrectangle{\pgfqpoint{0.570343in}{0.331635in}}{\pgfqpoint{9.300000in}{7.700000in}}%
\pgfusepath{clip}%
\pgfsetrectcap%
\pgfsetroundjoin%
\pgfsetlinewidth{1.505625pt}%
\definecolor{currentstroke}{rgb}{0.631373,0.788235,0.956863}%
\pgfsetstrokecolor{currentstroke}%
\pgfsetstrokeopacity{0.800000}%
\pgfsetdash{}{0pt}%
\pgfpathmoveto{\pgfqpoint{0.993071in}{5.156095in}}%
\pgfpathlineto{\pgfqpoint{5.170297in}{3.838787in}}%
\pgfusepath{stroke}%
\end{pgfscope}%
\begin{pgfscope}%
\pgfpathrectangle{\pgfqpoint{0.570343in}{0.331635in}}{\pgfqpoint{9.300000in}{7.700000in}}%
\pgfusepath{clip}%
\pgfsetrectcap%
\pgfsetroundjoin%
\pgfsetlinewidth{1.505625pt}%
\definecolor{currentstroke}{rgb}{0.631373,0.788235,0.956863}%
\pgfsetstrokecolor{currentstroke}%
\pgfsetstrokeopacity{0.800000}%
\pgfsetdash{}{0pt}%
\pgfpathmoveto{\pgfqpoint{6.861115in}{0.834540in}}%
\pgfpathlineto{\pgfqpoint{5.170297in}{3.838787in}}%
\pgfusepath{stroke}%
\end{pgfscope}%
\begin{pgfscope}%
\pgfpathrectangle{\pgfqpoint{0.570343in}{0.331635in}}{\pgfqpoint{9.300000in}{7.700000in}}%
\pgfusepath{clip}%
\pgfsetrectcap%
\pgfsetroundjoin%
\pgfsetlinewidth{1.505625pt}%
\definecolor{currentstroke}{rgb}{0.631373,0.788235,0.956863}%
\pgfsetstrokecolor{currentstroke}%
\pgfsetstrokeopacity{0.800000}%
\pgfsetdash{}{0pt}%
\pgfpathmoveto{\pgfqpoint{4.067058in}{2.387548in}}%
\pgfpathlineto{\pgfqpoint{5.170297in}{3.838787in}}%
\pgfusepath{stroke}%
\end{pgfscope}%
\begin{pgfscope}%
\pgfpathrectangle{\pgfqpoint{0.570343in}{0.331635in}}{\pgfqpoint{9.300000in}{7.700000in}}%
\pgfusepath{clip}%
\pgfsetrectcap%
\pgfsetroundjoin%
\pgfsetlinewidth{1.505625pt}%
\definecolor{currentstroke}{rgb}{0.631373,0.788235,0.956863}%
\pgfsetstrokecolor{currentstroke}%
\pgfsetstrokeopacity{0.800000}%
\pgfsetdash{}{0pt}%
\pgfpathmoveto{\pgfqpoint{6.352480in}{5.887763in}}%
\pgfpathlineto{\pgfqpoint{5.170297in}{3.838787in}}%
\pgfusepath{stroke}%
\end{pgfscope}%
\begin{pgfscope}%
\pgfpathrectangle{\pgfqpoint{0.570343in}{0.331635in}}{\pgfqpoint{9.300000in}{7.700000in}}%
\pgfusepath{clip}%
\pgfsetrectcap%
\pgfsetroundjoin%
\pgfsetlinewidth{1.505625pt}%
\definecolor{currentstroke}{rgb}{0.631373,0.788235,0.956863}%
\pgfsetstrokecolor{currentstroke}%
\pgfsetstrokeopacity{0.800000}%
\pgfsetdash{}{0pt}%
\pgfpathmoveto{\pgfqpoint{5.146154in}{5.493929in}}%
\pgfpathlineto{\pgfqpoint{5.170297in}{3.838787in}}%
\pgfusepath{stroke}%
\end{pgfscope}%
\begin{pgfscope}%
\pgfpathrectangle{\pgfqpoint{0.570343in}{0.331635in}}{\pgfqpoint{9.300000in}{7.700000in}}%
\pgfusepath{clip}%
\pgfsetrectcap%
\pgfsetroundjoin%
\pgfsetlinewidth{1.505625pt}%
\definecolor{currentstroke}{rgb}{0.631373,0.788235,0.956863}%
\pgfsetstrokecolor{currentstroke}%
\pgfsetstrokeopacity{0.800000}%
\pgfsetdash{}{0pt}%
\pgfpathmoveto{\pgfqpoint{7.564757in}{2.484970in}}%
\pgfpathlineto{\pgfqpoint{5.170297in}{3.838787in}}%
\pgfusepath{stroke}%
\end{pgfscope}%
\begin{pgfscope}%
\pgfpathrectangle{\pgfqpoint{0.570343in}{0.331635in}}{\pgfqpoint{9.300000in}{7.700000in}}%
\pgfusepath{clip}%
\pgfsetrectcap%
\pgfsetroundjoin%
\pgfsetlinewidth{1.505625pt}%
\definecolor{currentstroke}{rgb}{0.631373,0.788235,0.956863}%
\pgfsetstrokecolor{currentstroke}%
\pgfsetstrokeopacity{0.800000}%
\pgfsetdash{}{0pt}%
\pgfpathmoveto{\pgfqpoint{8.396171in}{4.701811in}}%
\pgfpathlineto{\pgfqpoint{5.170297in}{3.838787in}}%
\pgfusepath{stroke}%
\end{pgfscope}%
\begin{pgfscope}%
\pgfpathrectangle{\pgfqpoint{0.570343in}{0.331635in}}{\pgfqpoint{9.300000in}{7.700000in}}%
\pgfusepath{clip}%
\pgfsetrectcap%
\pgfsetroundjoin%
\pgfsetlinewidth{1.505625pt}%
\definecolor{currentstroke}{rgb}{0.631373,0.788235,0.956863}%
\pgfsetstrokecolor{currentstroke}%
\pgfsetstrokeopacity{0.800000}%
\pgfsetdash{}{0pt}%
\pgfpathmoveto{\pgfqpoint{3.531441in}{5.743837in}}%
\pgfpathlineto{\pgfqpoint{5.170297in}{3.838787in}}%
\pgfusepath{stroke}%
\end{pgfscope}%
\begin{pgfscope}%
\pgfpathrectangle{\pgfqpoint{0.570343in}{0.331635in}}{\pgfqpoint{9.300000in}{7.700000in}}%
\pgfusepath{clip}%
\pgfsetrectcap%
\pgfsetroundjoin%
\pgfsetlinewidth{1.505625pt}%
\definecolor{currentstroke}{rgb}{1.000000,0.705882,0.509804}%
\pgfsetstrokecolor{currentstroke}%
\pgfsetstrokeopacity{0.800000}%
\pgfsetdash{}{0pt}%
\pgfpathmoveto{\pgfqpoint{5.495895in}{6.180825in}}%
\pgfpathlineto{\pgfqpoint{5.013189in}{5.020435in}}%
\pgfusepath{stroke}%
\end{pgfscope}%
\begin{pgfscope}%
\pgfpathrectangle{\pgfqpoint{0.570343in}{0.331635in}}{\pgfqpoint{9.300000in}{7.700000in}}%
\pgfusepath{clip}%
\pgfsetrectcap%
\pgfsetroundjoin%
\pgfsetlinewidth{1.505625pt}%
\definecolor{currentstroke}{rgb}{1.000000,0.705882,0.509804}%
\pgfsetstrokecolor{currentstroke}%
\pgfsetstrokeopacity{0.800000}%
\pgfsetdash{}{0pt}%
\pgfpathmoveto{\pgfqpoint{8.684047in}{3.661174in}}%
\pgfpathlineto{\pgfqpoint{5.013189in}{5.020435in}}%
\pgfusepath{stroke}%
\end{pgfscope}%
\begin{pgfscope}%
\pgfpathrectangle{\pgfqpoint{0.570343in}{0.331635in}}{\pgfqpoint{9.300000in}{7.700000in}}%
\pgfusepath{clip}%
\pgfsetrectcap%
\pgfsetroundjoin%
\pgfsetlinewidth{1.505625pt}%
\definecolor{currentstroke}{rgb}{1.000000,0.705882,0.509804}%
\pgfsetstrokecolor{currentstroke}%
\pgfsetstrokeopacity{0.800000}%
\pgfsetdash{}{0pt}%
\pgfpathmoveto{\pgfqpoint{4.658532in}{4.182157in}}%
\pgfpathlineto{\pgfqpoint{5.013189in}{5.020435in}}%
\pgfusepath{stroke}%
\end{pgfscope}%
\begin{pgfscope}%
\pgfpathrectangle{\pgfqpoint{0.570343in}{0.331635in}}{\pgfqpoint{9.300000in}{7.700000in}}%
\pgfusepath{clip}%
\pgfsetrectcap%
\pgfsetroundjoin%
\pgfsetlinewidth{1.505625pt}%
\definecolor{currentstroke}{rgb}{1.000000,0.705882,0.509804}%
\pgfsetstrokecolor{currentstroke}%
\pgfsetstrokeopacity{0.800000}%
\pgfsetdash{}{0pt}%
\pgfpathmoveto{\pgfqpoint{5.265486in}{3.741523in}}%
\pgfpathlineto{\pgfqpoint{5.013189in}{5.020435in}}%
\pgfusepath{stroke}%
\end{pgfscope}%
\begin{pgfscope}%
\pgfpathrectangle{\pgfqpoint{0.570343in}{0.331635in}}{\pgfqpoint{9.300000in}{7.700000in}}%
\pgfusepath{clip}%
\pgfsetrectcap%
\pgfsetroundjoin%
\pgfsetlinewidth{1.505625pt}%
\definecolor{currentstroke}{rgb}{1.000000,0.705882,0.509804}%
\pgfsetstrokecolor{currentstroke}%
\pgfsetstrokeopacity{0.800000}%
\pgfsetdash{}{0pt}%
\pgfpathmoveto{\pgfqpoint{3.663113in}{6.576983in}}%
\pgfpathlineto{\pgfqpoint{5.013189in}{5.020435in}}%
\pgfusepath{stroke}%
\end{pgfscope}%
\begin{pgfscope}%
\pgfpathrectangle{\pgfqpoint{0.570343in}{0.331635in}}{\pgfqpoint{9.300000in}{7.700000in}}%
\pgfusepath{clip}%
\pgfsetrectcap%
\pgfsetroundjoin%
\pgfsetlinewidth{1.505625pt}%
\definecolor{currentstroke}{rgb}{1.000000,0.705882,0.509804}%
\pgfsetstrokecolor{currentstroke}%
\pgfsetstrokeopacity{0.800000}%
\pgfsetdash{}{0pt}%
\pgfpathmoveto{\pgfqpoint{7.006137in}{4.544186in}}%
\pgfpathlineto{\pgfqpoint{5.013189in}{5.020435in}}%
\pgfusepath{stroke}%
\end{pgfscope}%
\begin{pgfscope}%
\pgfpathrectangle{\pgfqpoint{0.570343in}{0.331635in}}{\pgfqpoint{9.300000in}{7.700000in}}%
\pgfusepath{clip}%
\pgfsetrectcap%
\pgfsetroundjoin%
\pgfsetlinewidth{1.505625pt}%
\definecolor{currentstroke}{rgb}{1.000000,0.705882,0.509804}%
\pgfsetstrokecolor{currentstroke}%
\pgfsetstrokeopacity{0.800000}%
\pgfsetdash{}{0pt}%
\pgfpathmoveto{\pgfqpoint{7.866554in}{4.532163in}}%
\pgfpathlineto{\pgfqpoint{5.013189in}{5.020435in}}%
\pgfusepath{stroke}%
\end{pgfscope}%
\begin{pgfscope}%
\pgfpathrectangle{\pgfqpoint{0.570343in}{0.331635in}}{\pgfqpoint{9.300000in}{7.700000in}}%
\pgfusepath{clip}%
\pgfsetrectcap%
\pgfsetroundjoin%
\pgfsetlinewidth{1.505625pt}%
\definecolor{currentstroke}{rgb}{1.000000,0.705882,0.509804}%
\pgfsetstrokecolor{currentstroke}%
\pgfsetstrokeopacity{0.800000}%
\pgfsetdash{}{0pt}%
\pgfpathmoveto{\pgfqpoint{2.309950in}{4.840333in}}%
\pgfpathlineto{\pgfqpoint{5.013189in}{5.020435in}}%
\pgfusepath{stroke}%
\end{pgfscope}%
\begin{pgfscope}%
\pgfpathrectangle{\pgfqpoint{0.570343in}{0.331635in}}{\pgfqpoint{9.300000in}{7.700000in}}%
\pgfusepath{clip}%
\pgfsetrectcap%
\pgfsetroundjoin%
\pgfsetlinewidth{1.505625pt}%
\definecolor{currentstroke}{rgb}{1.000000,0.705882,0.509804}%
\pgfsetstrokecolor{currentstroke}%
\pgfsetstrokeopacity{0.800000}%
\pgfsetdash{}{0pt}%
\pgfpathmoveto{\pgfqpoint{4.597903in}{6.287807in}}%
\pgfpathlineto{\pgfqpoint{5.013189in}{5.020435in}}%
\pgfusepath{stroke}%
\end{pgfscope}%
\begin{pgfscope}%
\pgfpathrectangle{\pgfqpoint{0.570343in}{0.331635in}}{\pgfqpoint{9.300000in}{7.700000in}}%
\pgfusepath{clip}%
\pgfsetrectcap%
\pgfsetroundjoin%
\pgfsetlinewidth{1.505625pt}%
\definecolor{currentstroke}{rgb}{1.000000,0.705882,0.509804}%
\pgfsetstrokecolor{currentstroke}%
\pgfsetstrokeopacity{0.800000}%
\pgfsetdash{}{0pt}%
\pgfpathmoveto{\pgfqpoint{2.912749in}{6.728158in}}%
\pgfpathlineto{\pgfqpoint{5.013189in}{5.020435in}}%
\pgfusepath{stroke}%
\end{pgfscope}%
\begin{pgfscope}%
\pgfpathrectangle{\pgfqpoint{0.570343in}{0.331635in}}{\pgfqpoint{9.300000in}{7.700000in}}%
\pgfusepath{clip}%
\pgfsetrectcap%
\pgfsetroundjoin%
\pgfsetlinewidth{1.505625pt}%
\definecolor{currentstroke}{rgb}{1.000000,0.705882,0.509804}%
\pgfsetstrokecolor{currentstroke}%
\pgfsetstrokeopacity{0.800000}%
\pgfsetdash{}{0pt}%
\pgfpathmoveto{\pgfqpoint{3.463668in}{5.071694in}}%
\pgfpathlineto{\pgfqpoint{5.013189in}{5.020435in}}%
\pgfusepath{stroke}%
\end{pgfscope}%
\begin{pgfscope}%
\pgfpathrectangle{\pgfqpoint{0.570343in}{0.331635in}}{\pgfqpoint{9.300000in}{7.700000in}}%
\pgfusepath{clip}%
\pgfsetrectcap%
\pgfsetroundjoin%
\pgfsetlinewidth{1.505625pt}%
\definecolor{currentstroke}{rgb}{1.000000,0.705882,0.509804}%
\pgfsetstrokecolor{currentstroke}%
\pgfsetstrokeopacity{0.800000}%
\pgfsetdash{}{0pt}%
\pgfpathmoveto{\pgfqpoint{8.862844in}{2.434362in}}%
\pgfpathlineto{\pgfqpoint{5.013189in}{5.020435in}}%
\pgfusepath{stroke}%
\end{pgfscope}%
\begin{pgfscope}%
\pgfpathrectangle{\pgfqpoint{0.570343in}{0.331635in}}{\pgfqpoint{9.300000in}{7.700000in}}%
\pgfusepath{clip}%
\pgfsetrectcap%
\pgfsetroundjoin%
\pgfsetlinewidth{1.505625pt}%
\definecolor{currentstroke}{rgb}{1.000000,0.705882,0.509804}%
\pgfsetstrokecolor{currentstroke}%
\pgfsetstrokeopacity{0.800000}%
\pgfsetdash{}{0pt}%
\pgfpathmoveto{\pgfqpoint{7.751788in}{5.308578in}}%
\pgfpathlineto{\pgfqpoint{5.013189in}{5.020435in}}%
\pgfusepath{stroke}%
\end{pgfscope}%
\begin{pgfscope}%
\pgfpathrectangle{\pgfqpoint{0.570343in}{0.331635in}}{\pgfqpoint{9.300000in}{7.700000in}}%
\pgfusepath{clip}%
\pgfsetrectcap%
\pgfsetroundjoin%
\pgfsetlinewidth{1.505625pt}%
\definecolor{currentstroke}{rgb}{1.000000,0.705882,0.509804}%
\pgfsetstrokecolor{currentstroke}%
\pgfsetstrokeopacity{0.800000}%
\pgfsetdash{}{0pt}%
\pgfpathmoveto{\pgfqpoint{4.290232in}{5.545295in}}%
\pgfpathlineto{\pgfqpoint{5.013189in}{5.020435in}}%
\pgfusepath{stroke}%
\end{pgfscope}%
\begin{pgfscope}%
\pgfpathrectangle{\pgfqpoint{0.570343in}{0.331635in}}{\pgfqpoint{9.300000in}{7.700000in}}%
\pgfusepath{clip}%
\pgfsetrectcap%
\pgfsetroundjoin%
\pgfsetlinewidth{1.505625pt}%
\definecolor{currentstroke}{rgb}{1.000000,0.705882,0.509804}%
\pgfsetstrokecolor{currentstroke}%
\pgfsetstrokeopacity{0.800000}%
\pgfsetdash{}{0pt}%
\pgfpathmoveto{\pgfqpoint{6.599542in}{6.892360in}}%
\pgfpathlineto{\pgfqpoint{5.013189in}{5.020435in}}%
\pgfusepath{stroke}%
\end{pgfscope}%
\begin{pgfscope}%
\pgfpathrectangle{\pgfqpoint{0.570343in}{0.331635in}}{\pgfqpoint{9.300000in}{7.700000in}}%
\pgfusepath{clip}%
\pgfsetrectcap%
\pgfsetroundjoin%
\pgfsetlinewidth{1.505625pt}%
\definecolor{currentstroke}{rgb}{1.000000,0.705882,0.509804}%
\pgfsetstrokecolor{currentstroke}%
\pgfsetstrokeopacity{0.800000}%
\pgfsetdash{}{0pt}%
\pgfpathmoveto{\pgfqpoint{5.751027in}{4.433028in}}%
\pgfpathlineto{\pgfqpoint{5.013189in}{5.020435in}}%
\pgfusepath{stroke}%
\end{pgfscope}%
\begin{pgfscope}%
\pgfpathrectangle{\pgfqpoint{0.570343in}{0.331635in}}{\pgfqpoint{9.300000in}{7.700000in}}%
\pgfusepath{clip}%
\pgfsetrectcap%
\pgfsetroundjoin%
\pgfsetlinewidth{1.505625pt}%
\definecolor{currentstroke}{rgb}{1.000000,0.705882,0.509804}%
\pgfsetstrokecolor{currentstroke}%
\pgfsetstrokeopacity{0.800000}%
\pgfsetdash{}{0pt}%
\pgfpathmoveto{\pgfqpoint{6.415032in}{5.001064in}}%
\pgfpathlineto{\pgfqpoint{5.013189in}{5.020435in}}%
\pgfusepath{stroke}%
\end{pgfscope}%
\begin{pgfscope}%
\pgfpathrectangle{\pgfqpoint{0.570343in}{0.331635in}}{\pgfqpoint{9.300000in}{7.700000in}}%
\pgfusepath{clip}%
\pgfsetrectcap%
\pgfsetroundjoin%
\pgfsetlinewidth{1.505625pt}%
\definecolor{currentstroke}{rgb}{1.000000,0.705882,0.509804}%
\pgfsetstrokecolor{currentstroke}%
\pgfsetstrokeopacity{0.800000}%
\pgfsetdash{}{0pt}%
\pgfpathmoveto{\pgfqpoint{3.801528in}{7.681635in}}%
\pgfpathlineto{\pgfqpoint{5.013189in}{5.020435in}}%
\pgfusepath{stroke}%
\end{pgfscope}%
\begin{pgfscope}%
\pgfpathrectangle{\pgfqpoint{0.570343in}{0.331635in}}{\pgfqpoint{9.300000in}{7.700000in}}%
\pgfusepath{clip}%
\pgfsetrectcap%
\pgfsetroundjoin%
\pgfsetlinewidth{1.505625pt}%
\definecolor{currentstroke}{rgb}{1.000000,0.705882,0.509804}%
\pgfsetstrokecolor{currentstroke}%
\pgfsetstrokeopacity{0.800000}%
\pgfsetdash{}{0pt}%
\pgfpathmoveto{\pgfqpoint{1.181459in}{6.330444in}}%
\pgfpathlineto{\pgfqpoint{5.013189in}{5.020435in}}%
\pgfusepath{stroke}%
\end{pgfscope}%
\begin{pgfscope}%
\pgfpathrectangle{\pgfqpoint{0.570343in}{0.331635in}}{\pgfqpoint{9.300000in}{7.700000in}}%
\pgfusepath{clip}%
\pgfsetrectcap%
\pgfsetroundjoin%
\pgfsetlinewidth{1.505625pt}%
\definecolor{currentstroke}{rgb}{1.000000,0.705882,0.509804}%
\pgfsetstrokecolor{currentstroke}%
\pgfsetstrokeopacity{0.800000}%
\pgfsetdash{}{0pt}%
\pgfpathmoveto{\pgfqpoint{2.718007in}{3.646018in}}%
\pgfpathlineto{\pgfqpoint{5.013189in}{5.020435in}}%
\pgfusepath{stroke}%
\end{pgfscope}%
\begin{pgfscope}%
\pgfpathrectangle{\pgfqpoint{0.570343in}{0.331635in}}{\pgfqpoint{9.300000in}{7.700000in}}%
\pgfusepath{clip}%
\pgfsetrectcap%
\pgfsetroundjoin%
\pgfsetlinewidth{1.505625pt}%
\definecolor{currentstroke}{rgb}{1.000000,0.705882,0.509804}%
\pgfsetstrokecolor{currentstroke}%
\pgfsetstrokeopacity{0.800000}%
\pgfsetdash{}{0pt}%
\pgfpathmoveto{\pgfqpoint{6.602631in}{2.859010in}}%
\pgfpathlineto{\pgfqpoint{5.013189in}{5.020435in}}%
\pgfusepath{stroke}%
\end{pgfscope}%
\begin{pgfscope}%
\pgfpathrectangle{\pgfqpoint{0.570343in}{0.331635in}}{\pgfqpoint{9.300000in}{7.700000in}}%
\pgfusepath{clip}%
\pgfsetrectcap%
\pgfsetroundjoin%
\pgfsetlinewidth{1.505625pt}%
\definecolor{currentstroke}{rgb}{1.000000,0.705882,0.509804}%
\pgfsetstrokecolor{currentstroke}%
\pgfsetstrokeopacity{0.800000}%
\pgfsetdash{}{0pt}%
\pgfpathmoveto{\pgfqpoint{1.957036in}{6.601755in}}%
\pgfpathlineto{\pgfqpoint{5.013189in}{5.020435in}}%
\pgfusepath{stroke}%
\end{pgfscope}%
\begin{pgfscope}%
\pgfpathrectangle{\pgfqpoint{0.570343in}{0.331635in}}{\pgfqpoint{9.300000in}{7.700000in}}%
\pgfusepath{clip}%
\pgfsetrectcap%
\pgfsetroundjoin%
\pgfsetlinewidth{1.505625pt}%
\definecolor{currentstroke}{rgb}{1.000000,0.705882,0.509804}%
\pgfsetstrokecolor{currentstroke}%
\pgfsetstrokeopacity{0.800000}%
\pgfsetdash{}{0pt}%
\pgfpathmoveto{\pgfqpoint{7.683729in}{6.298777in}}%
\pgfpathlineto{\pgfqpoint{5.013189in}{5.020435in}}%
\pgfusepath{stroke}%
\end{pgfscope}%
\begin{pgfscope}%
\pgfpathrectangle{\pgfqpoint{0.570343in}{0.331635in}}{\pgfqpoint{9.300000in}{7.700000in}}%
\pgfusepath{clip}%
\pgfsetrectcap%
\pgfsetroundjoin%
\pgfsetlinewidth{1.505625pt}%
\definecolor{currentstroke}{rgb}{1.000000,0.705882,0.509804}%
\pgfsetstrokecolor{currentstroke}%
\pgfsetstrokeopacity{0.800000}%
\pgfsetdash{}{0pt}%
\pgfpathmoveto{\pgfqpoint{8.418297in}{5.176089in}}%
\pgfpathlineto{\pgfqpoint{5.013189in}{5.020435in}}%
\pgfusepath{stroke}%
\end{pgfscope}%
\begin{pgfscope}%
\pgfpathrectangle{\pgfqpoint{0.570343in}{0.331635in}}{\pgfqpoint{9.300000in}{7.700000in}}%
\pgfusepath{clip}%
\pgfsetrectcap%
\pgfsetroundjoin%
\pgfsetlinewidth{1.505625pt}%
\definecolor{currentstroke}{rgb}{1.000000,0.705882,0.509804}%
\pgfsetstrokecolor{currentstroke}%
\pgfsetstrokeopacity{0.800000}%
\pgfsetdash{}{0pt}%
\pgfpathmoveto{\pgfqpoint{1.819365in}{4.009593in}}%
\pgfpathlineto{\pgfqpoint{5.013189in}{5.020435in}}%
\pgfusepath{stroke}%
\end{pgfscope}%
\begin{pgfscope}%
\pgfpathrectangle{\pgfqpoint{0.570343in}{0.331635in}}{\pgfqpoint{9.300000in}{7.700000in}}%
\pgfusepath{clip}%
\pgfsetrectcap%
\pgfsetroundjoin%
\pgfsetlinewidth{1.505625pt}%
\definecolor{currentstroke}{rgb}{1.000000,0.705882,0.509804}%
\pgfsetstrokecolor{currentstroke}%
\pgfsetstrokeopacity{0.800000}%
\pgfsetdash{}{0pt}%
\pgfpathmoveto{\pgfqpoint{3.672079in}{3.500631in}}%
\pgfpathlineto{\pgfqpoint{5.013189in}{5.020435in}}%
\pgfusepath{stroke}%
\end{pgfscope}%
\begin{pgfscope}%
\pgfpathrectangle{\pgfqpoint{0.570343in}{0.331635in}}{\pgfqpoint{9.300000in}{7.700000in}}%
\pgfusepath{clip}%
\pgfsetrectcap%
\pgfsetroundjoin%
\pgfsetlinewidth{1.505625pt}%
\definecolor{currentstroke}{rgb}{1.000000,0.705882,0.509804}%
\pgfsetstrokecolor{currentstroke}%
\pgfsetstrokeopacity{0.800000}%
\pgfsetdash{}{0pt}%
\pgfpathmoveto{\pgfqpoint{2.042788in}{5.674277in}}%
\pgfpathlineto{\pgfqpoint{5.013189in}{5.020435in}}%
\pgfusepath{stroke}%
\end{pgfscope}%
\begin{pgfscope}%
\pgfpathrectangle{\pgfqpoint{0.570343in}{0.331635in}}{\pgfqpoint{9.300000in}{7.700000in}}%
\pgfusepath{clip}%
\pgfsetrectcap%
\pgfsetroundjoin%
\pgfsetlinewidth{1.505625pt}%
\definecolor{currentstroke}{rgb}{1.000000,0.705882,0.509804}%
\pgfsetstrokecolor{currentstroke}%
\pgfsetstrokeopacity{0.800000}%
\pgfsetdash{}{0pt}%
\pgfpathmoveto{\pgfqpoint{4.877879in}{2.832260in}}%
\pgfpathlineto{\pgfqpoint{5.013189in}{5.020435in}}%
\pgfusepath{stroke}%
\end{pgfscope}%
\begin{pgfscope}%
\pgfsetrectcap%
\pgfsetmiterjoin%
\pgfsetlinewidth{0.803000pt}%
\definecolor{currentstroke}{rgb}{0.000000,0.000000,0.000000}%
\pgfsetstrokecolor{currentstroke}%
\pgfsetdash{}{0pt}%
\pgfpathmoveto{\pgfqpoint{0.570343in}{0.331635in}}%
\pgfpathlineto{\pgfqpoint{0.570343in}{8.031635in}}%
\pgfusepath{stroke}%
\end{pgfscope}%
\begin{pgfscope}%
\pgfsetrectcap%
\pgfsetmiterjoin%
\pgfsetlinewidth{0.803000pt}%
\definecolor{currentstroke}{rgb}{0.000000,0.000000,0.000000}%
\pgfsetstrokecolor{currentstroke}%
\pgfsetdash{}{0pt}%
\pgfpathmoveto{\pgfqpoint{9.870343in}{0.331635in}}%
\pgfpathlineto{\pgfqpoint{9.870343in}{8.031635in}}%
\pgfusepath{stroke}%
\end{pgfscope}%
\begin{pgfscope}%
\pgfsetrectcap%
\pgfsetmiterjoin%
\pgfsetlinewidth{0.803000pt}%
\definecolor{currentstroke}{rgb}{0.000000,0.000000,0.000000}%
\pgfsetstrokecolor{currentstroke}%
\pgfsetdash{}{0pt}%
\pgfpathmoveto{\pgfqpoint{0.570343in}{0.331635in}}%
\pgfpathlineto{\pgfqpoint{9.870343in}{0.331635in}}%
\pgfusepath{stroke}%
\end{pgfscope}%
\begin{pgfscope}%
\pgfsetrectcap%
\pgfsetmiterjoin%
\pgfsetlinewidth{0.803000pt}%
\definecolor{currentstroke}{rgb}{0.000000,0.000000,0.000000}%
\pgfsetstrokecolor{currentstroke}%
\pgfsetdash{}{0pt}%
\pgfpathmoveto{\pgfqpoint{0.570343in}{8.031635in}}%
\pgfpathlineto{\pgfqpoint{9.870343in}{8.031635in}}%
\pgfusepath{stroke}%
\end{pgfscope}%
\begin{pgfscope}%
\definecolor{textcolor}{rgb}{0.000000,0.000000,0.000000}%
\pgfsetstrokecolor{textcolor}%
\pgfsetfillcolor{textcolor}%
\pgftext[x=5.220343in,y=8.114968in,,base]{\color{textcolor}\sffamily\fontsize{12.000000}{14.400000}\selectfont Photo-Realistic Images}%
\end{pgfscope}%
\begin{pgfscope}%
\pgfsetbuttcap%
\pgfsetmiterjoin%
\definecolor{currentfill}{rgb}{1.000000,1.000000,1.000000}%
\pgfsetfillcolor{currentfill}%
\pgfsetfillopacity{0.800000}%
\pgfsetlinewidth{1.003750pt}%
\definecolor{currentstroke}{rgb}{0.800000,0.800000,0.800000}%
\pgfsetstrokecolor{currentstroke}%
\pgfsetstrokeopacity{0.800000}%
\pgfsetdash{}{0pt}%
\pgfpathmoveto{\pgfqpoint{9.967566in}{3.956944in}}%
\pgfpathlineto{\pgfqpoint{11.059186in}{3.956944in}}%
\pgfpathquadraticcurveto{\pgfqpoint{11.086964in}{3.956944in}}{\pgfqpoint{11.086964in}{3.984722in}}%
\pgfpathlineto{\pgfqpoint{11.086964in}{4.378548in}}%
\pgfpathquadraticcurveto{\pgfqpoint{11.086964in}{4.406326in}}{\pgfqpoint{11.059186in}{4.406326in}}%
\pgfpathlineto{\pgfqpoint{9.967566in}{4.406326in}}%
\pgfpathquadraticcurveto{\pgfqpoint{9.939788in}{4.406326in}}{\pgfqpoint{9.939788in}{4.378548in}}%
\pgfpathlineto{\pgfqpoint{9.939788in}{3.984722in}}%
\pgfpathquadraticcurveto{\pgfqpoint{9.939788in}{3.956944in}}{\pgfqpoint{9.967566in}{3.956944in}}%
\pgfpathclose%
\pgfusepath{stroke,fill}%
\end{pgfscope}%
\begin{pgfscope}%
\pgfsetbuttcap%
\pgfsetroundjoin%
\definecolor{currentfill}{rgb}{0.631373,0.788235,0.956863}%
\pgfsetfillcolor{currentfill}%
\pgfsetlinewidth{1.003750pt}%
\definecolor{currentstroke}{rgb}{0.631373,0.788235,0.956863}%
\pgfsetstrokecolor{currentstroke}%
\pgfsetdash{}{0pt}%
\pgfsys@defobject{currentmarker}{\pgfqpoint{-0.041667in}{-0.041667in}}{\pgfqpoint{0.041667in}{0.041667in}}{%
\pgfpathmoveto{\pgfqpoint{0.000000in}{-0.041667in}}%
\pgfpathcurveto{\pgfqpoint{0.011050in}{-0.041667in}}{\pgfqpoint{0.021649in}{-0.037276in}}{\pgfqpoint{0.029463in}{-0.029463in}}%
\pgfpathcurveto{\pgfqpoint{0.037276in}{-0.021649in}}{\pgfqpoint{0.041667in}{-0.011050in}}{\pgfqpoint{0.041667in}{0.000000in}}%
\pgfpathcurveto{\pgfqpoint{0.041667in}{0.011050in}}{\pgfqpoint{0.037276in}{0.021649in}}{\pgfqpoint{0.029463in}{0.029463in}}%
\pgfpathcurveto{\pgfqpoint{0.021649in}{0.037276in}}{\pgfqpoint{0.011050in}{0.041667in}}{\pgfqpoint{0.000000in}{0.041667in}}%
\pgfpathcurveto{\pgfqpoint{-0.011050in}{0.041667in}}{\pgfqpoint{-0.021649in}{0.037276in}}{\pgfqpoint{-0.029463in}{0.029463in}}%
\pgfpathcurveto{\pgfqpoint{-0.037276in}{0.021649in}}{\pgfqpoint{-0.041667in}{0.011050in}}{\pgfqpoint{-0.041667in}{0.000000in}}%
\pgfpathcurveto{\pgfqpoint{-0.041667in}{-0.011050in}}{\pgfqpoint{-0.037276in}{-0.021649in}}{\pgfqpoint{-0.029463in}{-0.029463in}}%
\pgfpathcurveto{\pgfqpoint{-0.021649in}{-0.037276in}}{\pgfqpoint{-0.011050in}{-0.041667in}}{\pgfqpoint{0.000000in}{-0.041667in}}%
\pgfpathclose%
\pgfusepath{stroke,fill}%
}%
\begin{pgfscope}%
\pgfsys@transformshift{10.134232in}{4.281705in}%
\pgfsys@useobject{currentmarker}{}%
\end{pgfscope}%
\end{pgfscope}%
\begin{pgfscope}%
\definecolor{textcolor}{rgb}{0.000000,0.000000,0.000000}%
\pgfsetstrokecolor{textcolor}%
\pgfsetfillcolor{textcolor}%
\pgftext[x=10.384232in,y=4.245247in,left,base]{\color{textcolor}\sffamily\fontsize{10.000000}{12.000000}\selectfont hypersim}%
\end{pgfscope}%
\begin{pgfscope}%
\pgfsetbuttcap%
\pgfsetroundjoin%
\definecolor{currentfill}{rgb}{1.000000,0.705882,0.509804}%
\pgfsetfillcolor{currentfill}%
\pgfsetlinewidth{1.003750pt}%
\definecolor{currentstroke}{rgb}{1.000000,0.705882,0.509804}%
\pgfsetstrokecolor{currentstroke}%
\pgfsetdash{}{0pt}%
\pgfsys@defobject{currentmarker}{\pgfqpoint{-0.041667in}{-0.041667in}}{\pgfqpoint{0.041667in}{0.041667in}}{%
\pgfpathmoveto{\pgfqpoint{0.000000in}{-0.041667in}}%
\pgfpathcurveto{\pgfqpoint{0.011050in}{-0.041667in}}{\pgfqpoint{0.021649in}{-0.037276in}}{\pgfqpoint{0.029463in}{-0.029463in}}%
\pgfpathcurveto{\pgfqpoint{0.037276in}{-0.021649in}}{\pgfqpoint{0.041667in}{-0.011050in}}{\pgfqpoint{0.041667in}{0.000000in}}%
\pgfpathcurveto{\pgfqpoint{0.041667in}{0.011050in}}{\pgfqpoint{0.037276in}{0.021649in}}{\pgfqpoint{0.029463in}{0.029463in}}%
\pgfpathcurveto{\pgfqpoint{0.021649in}{0.037276in}}{\pgfqpoint{0.011050in}{0.041667in}}{\pgfqpoint{0.000000in}{0.041667in}}%
\pgfpathcurveto{\pgfqpoint{-0.011050in}{0.041667in}}{\pgfqpoint{-0.021649in}{0.037276in}}{\pgfqpoint{-0.029463in}{0.029463in}}%
\pgfpathcurveto{\pgfqpoint{-0.037276in}{0.021649in}}{\pgfqpoint{-0.041667in}{0.011050in}}{\pgfqpoint{-0.041667in}{0.000000in}}%
\pgfpathcurveto{\pgfqpoint{-0.041667in}{-0.011050in}}{\pgfqpoint{-0.037276in}{-0.021649in}}{\pgfqpoint{-0.029463in}{-0.029463in}}%
\pgfpathcurveto{\pgfqpoint{-0.021649in}{-0.037276in}}{\pgfqpoint{-0.011050in}{-0.041667in}}{\pgfqpoint{0.000000in}{-0.041667in}}%
\pgfpathclose%
\pgfusepath{stroke,fill}%
}%
\begin{pgfscope}%
\pgfsys@transformshift{10.134232in}{4.077848in}%
\pgfsys@useobject{currentmarker}{}%
\end{pgfscope}%
\end{pgfscope}%
\begin{pgfscope}%
\definecolor{textcolor}{rgb}{0.000000,0.000000,0.000000}%
\pgfsetstrokecolor{textcolor}%
\pgfsetfillcolor{textcolor}%
\pgftext[x=10.384232in,y=4.041390in,left,base]{\color{textcolor}\sffamily\fontsize{10.000000}{12.000000}\selectfont pix3d}%
\end{pgfscope}%
\end{pgfpicture}%
\makeatother%
\endgroup%
}\\
    \resizebox{0.49\linewidth}{5cm}{%% Creator: Matplotlib, PGF backend
%%
%% To include the figure in your LaTeX document, write
%%   \input{<filename>.pgf}
%%
%% Make sure the required packages are loaded in your preamble
%%   \usepackage{pgf}
%%
%% Figures using additional raster images can only be included by \input if
%% they are in the same directory as the main LaTeX file. For loading figures
%% from other directories you can use the `import` package
%%   \usepackage{import}
%%
%% and then include the figures with
%%   \import{<path to file>}{<filename>.pgf}
%%
%% Matplotlib used the following preamble
%%   \usepackage{fontspec}
%%   \setmainfont{DejaVuSerif.ttf}[Path=\detokenize{/Users/apple/opt/anaconda3/envs/kaolin/lib/python3.7/site-packages/matplotlib/mpl-data/fonts/ttf/}]
%%   \setsansfont{DejaVuSans.ttf}[Path=\detokenize{/Users/apple/opt/anaconda3/envs/kaolin/lib/python3.7/site-packages/matplotlib/mpl-data/fonts/ttf/}]
%%   \setmonofont{DejaVuSansMono.ttf}[Path=\detokenize{/Users/apple/opt/anaconda3/envs/kaolin/lib/python3.7/site-packages/matplotlib/mpl-data/fonts/ttf/}]
%%
\begingroup%
\makeatletter%
\begin{pgfpicture}%
\pgfpathrectangle{\pgfpointorigin}{\pgfqpoint{5.630343in}{4.337596in}}%
\pgfusepath{use as bounding box, clip}%
\begin{pgfscope}%
\pgfsetbuttcap%
\pgfsetmiterjoin%
\definecolor{currentfill}{rgb}{1.000000,1.000000,1.000000}%
\pgfsetfillcolor{currentfill}%
\pgfsetlinewidth{0.000000pt}%
\definecolor{currentstroke}{rgb}{1.000000,1.000000,1.000000}%
\pgfsetstrokecolor{currentstroke}%
\pgfsetdash{}{0pt}%
\pgfpathmoveto{\pgfqpoint{0.000000in}{0.000000in}}%
\pgfpathlineto{\pgfqpoint{5.630343in}{0.000000in}}%
\pgfpathlineto{\pgfqpoint{5.630343in}{4.337596in}}%
\pgfpathlineto{\pgfqpoint{0.000000in}{4.337596in}}%
\pgfpathclose%
\pgfusepath{fill}%
\end{pgfscope}%
\begin{pgfscope}%
\pgfsetbuttcap%
\pgfsetmiterjoin%
\definecolor{currentfill}{rgb}{1.000000,1.000000,1.000000}%
\pgfsetfillcolor{currentfill}%
\pgfsetlinewidth{0.000000pt}%
\definecolor{currentstroke}{rgb}{0.000000,0.000000,0.000000}%
\pgfsetstrokecolor{currentstroke}%
\pgfsetstrokeopacity{0.000000}%
\pgfsetdash{}{0pt}%
\pgfpathmoveto{\pgfqpoint{0.570343in}{0.331635in}}%
\pgfpathlineto{\pgfqpoint{5.530343in}{0.331635in}}%
\pgfpathlineto{\pgfqpoint{5.530343in}{4.027635in}}%
\pgfpathlineto{\pgfqpoint{0.570343in}{4.027635in}}%
\pgfpathclose%
\pgfusepath{fill}%
\end{pgfscope}%
\begin{pgfscope}%
\pgfpathrectangle{\pgfqpoint{0.570343in}{0.331635in}}{\pgfqpoint{4.960000in}{3.696000in}}%
\pgfusepath{clip}%
\pgfsetbuttcap%
\pgfsetroundjoin%
\definecolor{currentfill}{rgb}{0.631373,0.788235,0.956863}%
\pgfsetfillcolor{currentfill}%
\pgfsetlinewidth{0.481800pt}%
\definecolor{currentstroke}{rgb}{1.000000,1.000000,1.000000}%
\pgfsetstrokecolor{currentstroke}%
\pgfsetdash{}{0pt}%
\pgfpathmoveto{\pgfqpoint{3.009597in}{1.331461in}}%
\pgfpathcurveto{\pgfqpoint{3.020647in}{1.331461in}}{\pgfqpoint{3.031246in}{1.335851in}}{\pgfqpoint{3.039060in}{1.343665in}}%
\pgfpathcurveto{\pgfqpoint{3.046874in}{1.351479in}}{\pgfqpoint{3.051264in}{1.362078in}}{\pgfqpoint{3.051264in}{1.373128in}}%
\pgfpathcurveto{\pgfqpoint{3.051264in}{1.384178in}}{\pgfqpoint{3.046874in}{1.394777in}}{\pgfqpoint{3.039060in}{1.402590in}}%
\pgfpathcurveto{\pgfqpoint{3.031246in}{1.410404in}}{\pgfqpoint{3.020647in}{1.414794in}}{\pgfqpoint{3.009597in}{1.414794in}}%
\pgfpathcurveto{\pgfqpoint{2.998547in}{1.414794in}}{\pgfqpoint{2.987948in}{1.410404in}}{\pgfqpoint{2.980135in}{1.402590in}}%
\pgfpathcurveto{\pgfqpoint{2.972321in}{1.394777in}}{\pgfqpoint{2.967931in}{1.384178in}}{\pgfqpoint{2.967931in}{1.373128in}}%
\pgfpathcurveto{\pgfqpoint{2.967931in}{1.362078in}}{\pgfqpoint{2.972321in}{1.351479in}}{\pgfqpoint{2.980135in}{1.343665in}}%
\pgfpathcurveto{\pgfqpoint{2.987948in}{1.335851in}}{\pgfqpoint{2.998547in}{1.331461in}}{\pgfqpoint{3.009597in}{1.331461in}}%
\pgfpathclose%
\pgfusepath{stroke,fill}%
\end{pgfscope}%
\begin{pgfscope}%
\pgfpathrectangle{\pgfqpoint{0.570343in}{0.331635in}}{\pgfqpoint{4.960000in}{3.696000in}}%
\pgfusepath{clip}%
\pgfsetbuttcap%
\pgfsetroundjoin%
\definecolor{currentfill}{rgb}{0.631373,0.788235,0.956863}%
\pgfsetfillcolor{currentfill}%
\pgfsetlinewidth{0.481800pt}%
\definecolor{currentstroke}{rgb}{1.000000,1.000000,1.000000}%
\pgfsetstrokecolor{currentstroke}%
\pgfsetdash{}{0pt}%
\pgfpathmoveto{\pgfqpoint{4.747431in}{2.652073in}}%
\pgfpathcurveto{\pgfqpoint{4.758481in}{2.652073in}}{\pgfqpoint{4.769080in}{2.656463in}}{\pgfqpoint{4.776894in}{2.664277in}}%
\pgfpathcurveto{\pgfqpoint{4.784707in}{2.672090in}}{\pgfqpoint{4.789097in}{2.682689in}}{\pgfqpoint{4.789097in}{2.693739in}}%
\pgfpathcurveto{\pgfqpoint{4.789097in}{2.704790in}}{\pgfqpoint{4.784707in}{2.715389in}}{\pgfqpoint{4.776894in}{2.723202in}}%
\pgfpathcurveto{\pgfqpoint{4.769080in}{2.731016in}}{\pgfqpoint{4.758481in}{2.735406in}}{\pgfqpoint{4.747431in}{2.735406in}}%
\pgfpathcurveto{\pgfqpoint{4.736381in}{2.735406in}}{\pgfqpoint{4.725782in}{2.731016in}}{\pgfqpoint{4.717968in}{2.723202in}}%
\pgfpathcurveto{\pgfqpoint{4.710154in}{2.715389in}}{\pgfqpoint{4.705764in}{2.704790in}}{\pgfqpoint{4.705764in}{2.693739in}}%
\pgfpathcurveto{\pgfqpoint{4.705764in}{2.682689in}}{\pgfqpoint{4.710154in}{2.672090in}}{\pgfqpoint{4.717968in}{2.664277in}}%
\pgfpathcurveto{\pgfqpoint{4.725782in}{2.656463in}}{\pgfqpoint{4.736381in}{2.652073in}}{\pgfqpoint{4.747431in}{2.652073in}}%
\pgfpathclose%
\pgfusepath{stroke,fill}%
\end{pgfscope}%
\begin{pgfscope}%
\pgfpathrectangle{\pgfqpoint{0.570343in}{0.331635in}}{\pgfqpoint{4.960000in}{3.696000in}}%
\pgfusepath{clip}%
\pgfsetbuttcap%
\pgfsetroundjoin%
\definecolor{currentfill}{rgb}{0.631373,0.788235,0.956863}%
\pgfsetfillcolor{currentfill}%
\pgfsetlinewidth{0.481800pt}%
\definecolor{currentstroke}{rgb}{1.000000,1.000000,1.000000}%
\pgfsetstrokecolor{currentstroke}%
\pgfsetdash{}{0pt}%
\pgfpathmoveto{\pgfqpoint{3.800801in}{1.122576in}}%
\pgfpathcurveto{\pgfqpoint{3.811851in}{1.122576in}}{\pgfqpoint{3.822450in}{1.126966in}}{\pgfqpoint{3.830264in}{1.134779in}}%
\pgfpathcurveto{\pgfqpoint{3.838077in}{1.142593in}}{\pgfqpoint{3.842468in}{1.153192in}}{\pgfqpoint{3.842468in}{1.164242in}}%
\pgfpathcurveto{\pgfqpoint{3.842468in}{1.175292in}}{\pgfqpoint{3.838077in}{1.185891in}}{\pgfqpoint{3.830264in}{1.193705in}}%
\pgfpathcurveto{\pgfqpoint{3.822450in}{1.201519in}}{\pgfqpoint{3.811851in}{1.205909in}}{\pgfqpoint{3.800801in}{1.205909in}}%
\pgfpathcurveto{\pgfqpoint{3.789751in}{1.205909in}}{\pgfqpoint{3.779152in}{1.201519in}}{\pgfqpoint{3.771338in}{1.193705in}}%
\pgfpathcurveto{\pgfqpoint{3.763525in}{1.185891in}}{\pgfqpoint{3.759134in}{1.175292in}}{\pgfqpoint{3.759134in}{1.164242in}}%
\pgfpathcurveto{\pgfqpoint{3.759134in}{1.153192in}}{\pgfqpoint{3.763525in}{1.142593in}}{\pgfqpoint{3.771338in}{1.134779in}}%
\pgfpathcurveto{\pgfqpoint{3.779152in}{1.126966in}}{\pgfqpoint{3.789751in}{1.122576in}}{\pgfqpoint{3.800801in}{1.122576in}}%
\pgfpathclose%
\pgfusepath{stroke,fill}%
\end{pgfscope}%
\begin{pgfscope}%
\pgfpathrectangle{\pgfqpoint{0.570343in}{0.331635in}}{\pgfqpoint{4.960000in}{3.696000in}}%
\pgfusepath{clip}%
\pgfsetbuttcap%
\pgfsetroundjoin%
\definecolor{currentfill}{rgb}{0.631373,0.788235,0.956863}%
\pgfsetfillcolor{currentfill}%
\pgfsetlinewidth{0.481800pt}%
\definecolor{currentstroke}{rgb}{1.000000,1.000000,1.000000}%
\pgfsetstrokecolor{currentstroke}%
\pgfsetdash{}{0pt}%
\pgfpathmoveto{\pgfqpoint{3.358861in}{1.907180in}}%
\pgfpathcurveto{\pgfqpoint{3.369911in}{1.907180in}}{\pgfqpoint{3.380510in}{1.911570in}}{\pgfqpoint{3.388324in}{1.919384in}}%
\pgfpathcurveto{\pgfqpoint{3.396138in}{1.927198in}}{\pgfqpoint{3.400528in}{1.937797in}}{\pgfqpoint{3.400528in}{1.948847in}}%
\pgfpathcurveto{\pgfqpoint{3.400528in}{1.959897in}}{\pgfqpoint{3.396138in}{1.970496in}}{\pgfqpoint{3.388324in}{1.978310in}}%
\pgfpathcurveto{\pgfqpoint{3.380510in}{1.986123in}}{\pgfqpoint{3.369911in}{1.990513in}}{\pgfqpoint{3.358861in}{1.990513in}}%
\pgfpathcurveto{\pgfqpoint{3.347811in}{1.990513in}}{\pgfqpoint{3.337212in}{1.986123in}}{\pgfqpoint{3.329398in}{1.978310in}}%
\pgfpathcurveto{\pgfqpoint{3.321585in}{1.970496in}}{\pgfqpoint{3.317195in}{1.959897in}}{\pgfqpoint{3.317195in}{1.948847in}}%
\pgfpathcurveto{\pgfqpoint{3.317195in}{1.937797in}}{\pgfqpoint{3.321585in}{1.927198in}}{\pgfqpoint{3.329398in}{1.919384in}}%
\pgfpathcurveto{\pgfqpoint{3.337212in}{1.911570in}}{\pgfqpoint{3.347811in}{1.907180in}}{\pgfqpoint{3.358861in}{1.907180in}}%
\pgfpathclose%
\pgfusepath{stroke,fill}%
\end{pgfscope}%
\begin{pgfscope}%
\pgfpathrectangle{\pgfqpoint{0.570343in}{0.331635in}}{\pgfqpoint{4.960000in}{3.696000in}}%
\pgfusepath{clip}%
\pgfsetbuttcap%
\pgfsetroundjoin%
\definecolor{currentfill}{rgb}{0.631373,0.788235,0.956863}%
\pgfsetfillcolor{currentfill}%
\pgfsetlinewidth{0.481800pt}%
\definecolor{currentstroke}{rgb}{1.000000,1.000000,1.000000}%
\pgfsetstrokecolor{currentstroke}%
\pgfsetdash{}{0pt}%
\pgfpathmoveto{\pgfqpoint{3.351516in}{1.046284in}}%
\pgfpathcurveto{\pgfqpoint{3.362566in}{1.046284in}}{\pgfqpoint{3.373165in}{1.050674in}}{\pgfqpoint{3.380979in}{1.058488in}}%
\pgfpathcurveto{\pgfqpoint{3.388792in}{1.066301in}}{\pgfqpoint{3.393183in}{1.076900in}}{\pgfqpoint{3.393183in}{1.087951in}}%
\pgfpathcurveto{\pgfqpoint{3.393183in}{1.099001in}}{\pgfqpoint{3.388792in}{1.109600in}}{\pgfqpoint{3.380979in}{1.117413in}}%
\pgfpathcurveto{\pgfqpoint{3.373165in}{1.125227in}}{\pgfqpoint{3.362566in}{1.129617in}}{\pgfqpoint{3.351516in}{1.129617in}}%
\pgfpathcurveto{\pgfqpoint{3.340466in}{1.129617in}}{\pgfqpoint{3.329867in}{1.125227in}}{\pgfqpoint{3.322053in}{1.117413in}}%
\pgfpathcurveto{\pgfqpoint{3.314240in}{1.109600in}}{\pgfqpoint{3.309849in}{1.099001in}}{\pgfqpoint{3.309849in}{1.087951in}}%
\pgfpathcurveto{\pgfqpoint{3.309849in}{1.076900in}}{\pgfqpoint{3.314240in}{1.066301in}}{\pgfqpoint{3.322053in}{1.058488in}}%
\pgfpathcurveto{\pgfqpoint{3.329867in}{1.050674in}}{\pgfqpoint{3.340466in}{1.046284in}}{\pgfqpoint{3.351516in}{1.046284in}}%
\pgfpathclose%
\pgfusepath{stroke,fill}%
\end{pgfscope}%
\begin{pgfscope}%
\pgfpathrectangle{\pgfqpoint{0.570343in}{0.331635in}}{\pgfqpoint{4.960000in}{3.696000in}}%
\pgfusepath{clip}%
\pgfsetbuttcap%
\pgfsetroundjoin%
\definecolor{currentfill}{rgb}{0.631373,0.788235,0.956863}%
\pgfsetfillcolor{currentfill}%
\pgfsetlinewidth{0.481800pt}%
\definecolor{currentstroke}{rgb}{1.000000,1.000000,1.000000}%
\pgfsetstrokecolor{currentstroke}%
\pgfsetdash{}{0pt}%
\pgfpathmoveto{\pgfqpoint{2.157332in}{3.115971in}}%
\pgfpathcurveto{\pgfqpoint{2.168382in}{3.115971in}}{\pgfqpoint{2.178981in}{3.120362in}}{\pgfqpoint{2.186795in}{3.128175in}}%
\pgfpathcurveto{\pgfqpoint{2.194609in}{3.135989in}}{\pgfqpoint{2.198999in}{3.146588in}}{\pgfqpoint{2.198999in}{3.157638in}}%
\pgfpathcurveto{\pgfqpoint{2.198999in}{3.168688in}}{\pgfqpoint{2.194609in}{3.179287in}}{\pgfqpoint{2.186795in}{3.187101in}}%
\pgfpathcurveto{\pgfqpoint{2.178981in}{3.194914in}}{\pgfqpoint{2.168382in}{3.199305in}}{\pgfqpoint{2.157332in}{3.199305in}}%
\pgfpathcurveto{\pgfqpoint{2.146282in}{3.199305in}}{\pgfqpoint{2.135683in}{3.194914in}}{\pgfqpoint{2.127870in}{3.187101in}}%
\pgfpathcurveto{\pgfqpoint{2.120056in}{3.179287in}}{\pgfqpoint{2.115666in}{3.168688in}}{\pgfqpoint{2.115666in}{3.157638in}}%
\pgfpathcurveto{\pgfqpoint{2.115666in}{3.146588in}}{\pgfqpoint{2.120056in}{3.135989in}}{\pgfqpoint{2.127870in}{3.128175in}}%
\pgfpathcurveto{\pgfqpoint{2.135683in}{3.120362in}}{\pgfqpoint{2.146282in}{3.115971in}}{\pgfqpoint{2.157332in}{3.115971in}}%
\pgfpathclose%
\pgfusepath{stroke,fill}%
\end{pgfscope}%
\begin{pgfscope}%
\pgfpathrectangle{\pgfqpoint{0.570343in}{0.331635in}}{\pgfqpoint{4.960000in}{3.696000in}}%
\pgfusepath{clip}%
\pgfsetbuttcap%
\pgfsetroundjoin%
\definecolor{currentfill}{rgb}{0.631373,0.788235,0.956863}%
\pgfsetfillcolor{currentfill}%
\pgfsetlinewidth{0.481800pt}%
\definecolor{currentstroke}{rgb}{1.000000,1.000000,1.000000}%
\pgfsetstrokecolor{currentstroke}%
\pgfsetdash{}{0pt}%
\pgfpathmoveto{\pgfqpoint{4.201874in}{1.874651in}}%
\pgfpathcurveto{\pgfqpoint{4.212924in}{1.874651in}}{\pgfqpoint{4.223523in}{1.879042in}}{\pgfqpoint{4.231337in}{1.886855in}}%
\pgfpathcurveto{\pgfqpoint{4.239150in}{1.894669in}}{\pgfqpoint{4.243541in}{1.905268in}}{\pgfqpoint{4.243541in}{1.916318in}}%
\pgfpathcurveto{\pgfqpoint{4.243541in}{1.927368in}}{\pgfqpoint{4.239150in}{1.937967in}}{\pgfqpoint{4.231337in}{1.945781in}}%
\pgfpathcurveto{\pgfqpoint{4.223523in}{1.953594in}}{\pgfqpoint{4.212924in}{1.957985in}}{\pgfqpoint{4.201874in}{1.957985in}}%
\pgfpathcurveto{\pgfqpoint{4.190824in}{1.957985in}}{\pgfqpoint{4.180225in}{1.953594in}}{\pgfqpoint{4.172411in}{1.945781in}}%
\pgfpathcurveto{\pgfqpoint{4.164598in}{1.937967in}}{\pgfqpoint{4.160207in}{1.927368in}}{\pgfqpoint{4.160207in}{1.916318in}}%
\pgfpathcurveto{\pgfqpoint{4.160207in}{1.905268in}}{\pgfqpoint{4.164598in}{1.894669in}}{\pgfqpoint{4.172411in}{1.886855in}}%
\pgfpathcurveto{\pgfqpoint{4.180225in}{1.879042in}}{\pgfqpoint{4.190824in}{1.874651in}}{\pgfqpoint{4.201874in}{1.874651in}}%
\pgfpathclose%
\pgfusepath{stroke,fill}%
\end{pgfscope}%
\begin{pgfscope}%
\pgfpathrectangle{\pgfqpoint{0.570343in}{0.331635in}}{\pgfqpoint{4.960000in}{3.696000in}}%
\pgfusepath{clip}%
\pgfsetbuttcap%
\pgfsetroundjoin%
\definecolor{currentfill}{rgb}{0.631373,0.788235,0.956863}%
\pgfsetfillcolor{currentfill}%
\pgfsetlinewidth{0.481800pt}%
\definecolor{currentstroke}{rgb}{1.000000,1.000000,1.000000}%
\pgfsetstrokecolor{currentstroke}%
\pgfsetdash{}{0pt}%
\pgfpathmoveto{\pgfqpoint{3.667396in}{0.643012in}}%
\pgfpathcurveto{\pgfqpoint{3.678446in}{0.643012in}}{\pgfqpoint{3.689045in}{0.647403in}}{\pgfqpoint{3.696859in}{0.655216in}}%
\pgfpathcurveto{\pgfqpoint{3.704672in}{0.663030in}}{\pgfqpoint{3.709062in}{0.673629in}}{\pgfqpoint{3.709062in}{0.684679in}}%
\pgfpathcurveto{\pgfqpoint{3.709062in}{0.695729in}}{\pgfqpoint{3.704672in}{0.706328in}}{\pgfqpoint{3.696859in}{0.714142in}}%
\pgfpathcurveto{\pgfqpoint{3.689045in}{0.721955in}}{\pgfqpoint{3.678446in}{0.726346in}}{\pgfqpoint{3.667396in}{0.726346in}}%
\pgfpathcurveto{\pgfqpoint{3.656346in}{0.726346in}}{\pgfqpoint{3.645747in}{0.721955in}}{\pgfqpoint{3.637933in}{0.714142in}}%
\pgfpathcurveto{\pgfqpoint{3.630119in}{0.706328in}}{\pgfqpoint{3.625729in}{0.695729in}}{\pgfqpoint{3.625729in}{0.684679in}}%
\pgfpathcurveto{\pgfqpoint{3.625729in}{0.673629in}}{\pgfqpoint{3.630119in}{0.663030in}}{\pgfqpoint{3.637933in}{0.655216in}}%
\pgfpathcurveto{\pgfqpoint{3.645747in}{0.647403in}}{\pgfqpoint{3.656346in}{0.643012in}}{\pgfqpoint{3.667396in}{0.643012in}}%
\pgfpathclose%
\pgfusepath{stroke,fill}%
\end{pgfscope}%
\begin{pgfscope}%
\pgfpathrectangle{\pgfqpoint{0.570343in}{0.331635in}}{\pgfqpoint{4.960000in}{3.696000in}}%
\pgfusepath{clip}%
\pgfsetbuttcap%
\pgfsetroundjoin%
\definecolor{currentfill}{rgb}{0.631373,0.788235,0.956863}%
\pgfsetfillcolor{currentfill}%
\pgfsetlinewidth{0.481800pt}%
\definecolor{currentstroke}{rgb}{1.000000,1.000000,1.000000}%
\pgfsetstrokecolor{currentstroke}%
\pgfsetdash{}{0pt}%
\pgfpathmoveto{\pgfqpoint{3.384863in}{1.519474in}}%
\pgfpathcurveto{\pgfqpoint{3.395913in}{1.519474in}}{\pgfqpoint{3.406512in}{1.523865in}}{\pgfqpoint{3.414326in}{1.531678in}}%
\pgfpathcurveto{\pgfqpoint{3.422140in}{1.539492in}}{\pgfqpoint{3.426530in}{1.550091in}}{\pgfqpoint{3.426530in}{1.561141in}}%
\pgfpathcurveto{\pgfqpoint{3.426530in}{1.572191in}}{\pgfqpoint{3.422140in}{1.582790in}}{\pgfqpoint{3.414326in}{1.590604in}}%
\pgfpathcurveto{\pgfqpoint{3.406512in}{1.598417in}}{\pgfqpoint{3.395913in}{1.602808in}}{\pgfqpoint{3.384863in}{1.602808in}}%
\pgfpathcurveto{\pgfqpoint{3.373813in}{1.602808in}}{\pgfqpoint{3.363214in}{1.598417in}}{\pgfqpoint{3.355400in}{1.590604in}}%
\pgfpathcurveto{\pgfqpoint{3.347587in}{1.582790in}}{\pgfqpoint{3.343196in}{1.572191in}}{\pgfqpoint{3.343196in}{1.561141in}}%
\pgfpathcurveto{\pgfqpoint{3.343196in}{1.550091in}}{\pgfqpoint{3.347587in}{1.539492in}}{\pgfqpoint{3.355400in}{1.531678in}}%
\pgfpathcurveto{\pgfqpoint{3.363214in}{1.523865in}}{\pgfqpoint{3.373813in}{1.519474in}}{\pgfqpoint{3.384863in}{1.519474in}}%
\pgfpathclose%
\pgfusepath{stroke,fill}%
\end{pgfscope}%
\begin{pgfscope}%
\pgfpathrectangle{\pgfqpoint{0.570343in}{0.331635in}}{\pgfqpoint{4.960000in}{3.696000in}}%
\pgfusepath{clip}%
\pgfsetbuttcap%
\pgfsetroundjoin%
\definecolor{currentfill}{rgb}{0.631373,0.788235,0.956863}%
\pgfsetfillcolor{currentfill}%
\pgfsetlinewidth{0.481800pt}%
\definecolor{currentstroke}{rgb}{1.000000,1.000000,1.000000}%
\pgfsetstrokecolor{currentstroke}%
\pgfsetdash{}{0pt}%
\pgfpathmoveto{\pgfqpoint{2.133517in}{2.131617in}}%
\pgfpathcurveto{\pgfqpoint{2.144567in}{2.131617in}}{\pgfqpoint{2.155166in}{2.136007in}}{\pgfqpoint{2.162979in}{2.143821in}}%
\pgfpathcurveto{\pgfqpoint{2.170793in}{2.151635in}}{\pgfqpoint{2.175183in}{2.162234in}}{\pgfqpoint{2.175183in}{2.173284in}}%
\pgfpathcurveto{\pgfqpoint{2.175183in}{2.184334in}}{\pgfqpoint{2.170793in}{2.194933in}}{\pgfqpoint{2.162979in}{2.202747in}}%
\pgfpathcurveto{\pgfqpoint{2.155166in}{2.210560in}}{\pgfqpoint{2.144567in}{2.214950in}}{\pgfqpoint{2.133517in}{2.214950in}}%
\pgfpathcurveto{\pgfqpoint{2.122466in}{2.214950in}}{\pgfqpoint{2.111867in}{2.210560in}}{\pgfqpoint{2.104054in}{2.202747in}}%
\pgfpathcurveto{\pgfqpoint{2.096240in}{2.194933in}}{\pgfqpoint{2.091850in}{2.184334in}}{\pgfqpoint{2.091850in}{2.173284in}}%
\pgfpathcurveto{\pgfqpoint{2.091850in}{2.162234in}}{\pgfqpoint{2.096240in}{2.151635in}}{\pgfqpoint{2.104054in}{2.143821in}}%
\pgfpathcurveto{\pgfqpoint{2.111867in}{2.136007in}}{\pgfqpoint{2.122466in}{2.131617in}}{\pgfqpoint{2.133517in}{2.131617in}}%
\pgfpathclose%
\pgfusepath{stroke,fill}%
\end{pgfscope}%
\begin{pgfscope}%
\pgfpathrectangle{\pgfqpoint{0.570343in}{0.331635in}}{\pgfqpoint{4.960000in}{3.696000in}}%
\pgfusepath{clip}%
\pgfsetbuttcap%
\pgfsetroundjoin%
\definecolor{currentfill}{rgb}{0.631373,0.788235,0.956863}%
\pgfsetfillcolor{currentfill}%
\pgfsetlinewidth{0.481800pt}%
\definecolor{currentstroke}{rgb}{1.000000,1.000000,1.000000}%
\pgfsetstrokecolor{currentstroke}%
\pgfsetdash{}{0pt}%
\pgfpathmoveto{\pgfqpoint{5.304889in}{1.738067in}}%
\pgfpathcurveto{\pgfqpoint{5.315939in}{1.738067in}}{\pgfqpoint{5.326538in}{1.742457in}}{\pgfqpoint{5.334352in}{1.750271in}}%
\pgfpathcurveto{\pgfqpoint{5.342165in}{1.758084in}}{\pgfqpoint{5.346555in}{1.768683in}}{\pgfqpoint{5.346555in}{1.779734in}}%
\pgfpathcurveto{\pgfqpoint{5.346555in}{1.790784in}}{\pgfqpoint{5.342165in}{1.801383in}}{\pgfqpoint{5.334352in}{1.809196in}}%
\pgfpathcurveto{\pgfqpoint{5.326538in}{1.817010in}}{\pgfqpoint{5.315939in}{1.821400in}}{\pgfqpoint{5.304889in}{1.821400in}}%
\pgfpathcurveto{\pgfqpoint{5.293839in}{1.821400in}}{\pgfqpoint{5.283240in}{1.817010in}}{\pgfqpoint{5.275426in}{1.809196in}}%
\pgfpathcurveto{\pgfqpoint{5.267612in}{1.801383in}}{\pgfqpoint{5.263222in}{1.790784in}}{\pgfqpoint{5.263222in}{1.779734in}}%
\pgfpathcurveto{\pgfqpoint{5.263222in}{1.768683in}}{\pgfqpoint{5.267612in}{1.758084in}}{\pgfqpoint{5.275426in}{1.750271in}}%
\pgfpathcurveto{\pgfqpoint{5.283240in}{1.742457in}}{\pgfqpoint{5.293839in}{1.738067in}}{\pgfqpoint{5.304889in}{1.738067in}}%
\pgfpathclose%
\pgfusepath{stroke,fill}%
\end{pgfscope}%
\begin{pgfscope}%
\pgfpathrectangle{\pgfqpoint{0.570343in}{0.331635in}}{\pgfqpoint{4.960000in}{3.696000in}}%
\pgfusepath{clip}%
\pgfsetbuttcap%
\pgfsetroundjoin%
\definecolor{currentfill}{rgb}{0.631373,0.788235,0.956863}%
\pgfsetfillcolor{currentfill}%
\pgfsetlinewidth{0.481800pt}%
\definecolor{currentstroke}{rgb}{1.000000,1.000000,1.000000}%
\pgfsetstrokecolor{currentstroke}%
\pgfsetdash{}{0pt}%
\pgfpathmoveto{\pgfqpoint{3.176792in}{2.210336in}}%
\pgfpathcurveto{\pgfqpoint{3.187842in}{2.210336in}}{\pgfqpoint{3.198441in}{2.214726in}}{\pgfqpoint{3.206255in}{2.222540in}}%
\pgfpathcurveto{\pgfqpoint{3.214068in}{2.230354in}}{\pgfqpoint{3.218459in}{2.240953in}}{\pgfqpoint{3.218459in}{2.252003in}}%
\pgfpathcurveto{\pgfqpoint{3.218459in}{2.263053in}}{\pgfqpoint{3.214068in}{2.273652in}}{\pgfqpoint{3.206255in}{2.281466in}}%
\pgfpathcurveto{\pgfqpoint{3.198441in}{2.289279in}}{\pgfqpoint{3.187842in}{2.293669in}}{\pgfqpoint{3.176792in}{2.293669in}}%
\pgfpathcurveto{\pgfqpoint{3.165742in}{2.293669in}}{\pgfqpoint{3.155143in}{2.289279in}}{\pgfqpoint{3.147329in}{2.281466in}}%
\pgfpathcurveto{\pgfqpoint{3.139516in}{2.273652in}}{\pgfqpoint{3.135125in}{2.263053in}}{\pgfqpoint{3.135125in}{2.252003in}}%
\pgfpathcurveto{\pgfqpoint{3.135125in}{2.240953in}}{\pgfqpoint{3.139516in}{2.230354in}}{\pgfqpoint{3.147329in}{2.222540in}}%
\pgfpathcurveto{\pgfqpoint{3.155143in}{2.214726in}}{\pgfqpoint{3.165742in}{2.210336in}}{\pgfqpoint{3.176792in}{2.210336in}}%
\pgfpathclose%
\pgfusepath{stroke,fill}%
\end{pgfscope}%
\begin{pgfscope}%
\pgfpathrectangle{\pgfqpoint{0.570343in}{0.331635in}}{\pgfqpoint{4.960000in}{3.696000in}}%
\pgfusepath{clip}%
\pgfsetbuttcap%
\pgfsetroundjoin%
\definecolor{currentfill}{rgb}{0.631373,0.788235,0.956863}%
\pgfsetfillcolor{currentfill}%
\pgfsetlinewidth{0.481800pt}%
\definecolor{currentstroke}{rgb}{1.000000,1.000000,1.000000}%
\pgfsetstrokecolor{currentstroke}%
\pgfsetdash{}{0pt}%
\pgfpathmoveto{\pgfqpoint{2.230769in}{1.168284in}}%
\pgfpathcurveto{\pgfqpoint{2.241819in}{1.168284in}}{\pgfqpoint{2.252418in}{1.172674in}}{\pgfqpoint{2.260232in}{1.180488in}}%
\pgfpathcurveto{\pgfqpoint{2.268046in}{1.188301in}}{\pgfqpoint{2.272436in}{1.198900in}}{\pgfqpoint{2.272436in}{1.209950in}}%
\pgfpathcurveto{\pgfqpoint{2.272436in}{1.221000in}}{\pgfqpoint{2.268046in}{1.231599in}}{\pgfqpoint{2.260232in}{1.239413in}}%
\pgfpathcurveto{\pgfqpoint{2.252418in}{1.247227in}}{\pgfqpoint{2.241819in}{1.251617in}}{\pgfqpoint{2.230769in}{1.251617in}}%
\pgfpathcurveto{\pgfqpoint{2.219719in}{1.251617in}}{\pgfqpoint{2.209120in}{1.247227in}}{\pgfqpoint{2.201306in}{1.239413in}}%
\pgfpathcurveto{\pgfqpoint{2.193493in}{1.231599in}}{\pgfqpoint{2.189102in}{1.221000in}}{\pgfqpoint{2.189102in}{1.209950in}}%
\pgfpathcurveto{\pgfqpoint{2.189102in}{1.198900in}}{\pgfqpoint{2.193493in}{1.188301in}}{\pgfqpoint{2.201306in}{1.180488in}}%
\pgfpathcurveto{\pgfqpoint{2.209120in}{1.172674in}}{\pgfqpoint{2.219719in}{1.168284in}}{\pgfqpoint{2.230769in}{1.168284in}}%
\pgfpathclose%
\pgfusepath{stroke,fill}%
\end{pgfscope}%
\begin{pgfscope}%
\pgfpathrectangle{\pgfqpoint{0.570343in}{0.331635in}}{\pgfqpoint{4.960000in}{3.696000in}}%
\pgfusepath{clip}%
\pgfsetbuttcap%
\pgfsetroundjoin%
\definecolor{currentfill}{rgb}{0.631373,0.788235,0.956863}%
\pgfsetfillcolor{currentfill}%
\pgfsetlinewidth{0.481800pt}%
\definecolor{currentstroke}{rgb}{1.000000,1.000000,1.000000}%
\pgfsetstrokecolor{currentstroke}%
\pgfsetdash{}{0pt}%
\pgfpathmoveto{\pgfqpoint{2.677188in}{2.172216in}}%
\pgfpathcurveto{\pgfqpoint{2.688238in}{2.172216in}}{\pgfqpoint{2.698837in}{2.176606in}}{\pgfqpoint{2.706651in}{2.184419in}}%
\pgfpathcurveto{\pgfqpoint{2.714465in}{2.192233in}}{\pgfqpoint{2.718855in}{2.202832in}}{\pgfqpoint{2.718855in}{2.213882in}}%
\pgfpathcurveto{\pgfqpoint{2.718855in}{2.224932in}}{\pgfqpoint{2.714465in}{2.235531in}}{\pgfqpoint{2.706651in}{2.243345in}}%
\pgfpathcurveto{\pgfqpoint{2.698837in}{2.251159in}}{\pgfqpoint{2.688238in}{2.255549in}}{\pgfqpoint{2.677188in}{2.255549in}}%
\pgfpathcurveto{\pgfqpoint{2.666138in}{2.255549in}}{\pgfqpoint{2.655539in}{2.251159in}}{\pgfqpoint{2.647726in}{2.243345in}}%
\pgfpathcurveto{\pgfqpoint{2.639912in}{2.235531in}}{\pgfqpoint{2.635522in}{2.224932in}}{\pgfqpoint{2.635522in}{2.213882in}}%
\pgfpathcurveto{\pgfqpoint{2.635522in}{2.202832in}}{\pgfqpoint{2.639912in}{2.192233in}}{\pgfqpoint{2.647726in}{2.184419in}}%
\pgfpathcurveto{\pgfqpoint{2.655539in}{2.176606in}}{\pgfqpoint{2.666138in}{2.172216in}}{\pgfqpoint{2.677188in}{2.172216in}}%
\pgfpathclose%
\pgfusepath{stroke,fill}%
\end{pgfscope}%
\begin{pgfscope}%
\pgfpathrectangle{\pgfqpoint{0.570343in}{0.331635in}}{\pgfqpoint{4.960000in}{3.696000in}}%
\pgfusepath{clip}%
\pgfsetbuttcap%
\pgfsetroundjoin%
\definecolor{currentfill}{rgb}{0.631373,0.788235,0.956863}%
\pgfsetfillcolor{currentfill}%
\pgfsetlinewidth{0.481800pt}%
\definecolor{currentstroke}{rgb}{1.000000,1.000000,1.000000}%
\pgfsetstrokecolor{currentstroke}%
\pgfsetdash{}{0pt}%
\pgfpathmoveto{\pgfqpoint{3.805258in}{1.743124in}}%
\pgfpathcurveto{\pgfqpoint{3.816308in}{1.743124in}}{\pgfqpoint{3.826907in}{1.747514in}}{\pgfqpoint{3.834721in}{1.755328in}}%
\pgfpathcurveto{\pgfqpoint{3.842535in}{1.763141in}}{\pgfqpoint{3.846925in}{1.773740in}}{\pgfqpoint{3.846925in}{1.784790in}}%
\pgfpathcurveto{\pgfqpoint{3.846925in}{1.795840in}}{\pgfqpoint{3.842535in}{1.806440in}}{\pgfqpoint{3.834721in}{1.814253in}}%
\pgfpathcurveto{\pgfqpoint{3.826907in}{1.822067in}}{\pgfqpoint{3.816308in}{1.826457in}}{\pgfqpoint{3.805258in}{1.826457in}}%
\pgfpathcurveto{\pgfqpoint{3.794208in}{1.826457in}}{\pgfqpoint{3.783609in}{1.822067in}}{\pgfqpoint{3.775795in}{1.814253in}}%
\pgfpathcurveto{\pgfqpoint{3.767982in}{1.806440in}}{\pgfqpoint{3.763591in}{1.795840in}}{\pgfqpoint{3.763591in}{1.784790in}}%
\pgfpathcurveto{\pgfqpoint{3.763591in}{1.773740in}}{\pgfqpoint{3.767982in}{1.763141in}}{\pgfqpoint{3.775795in}{1.755328in}}%
\pgfpathcurveto{\pgfqpoint{3.783609in}{1.747514in}}{\pgfqpoint{3.794208in}{1.743124in}}{\pgfqpoint{3.805258in}{1.743124in}}%
\pgfpathclose%
\pgfusepath{stroke,fill}%
\end{pgfscope}%
\begin{pgfscope}%
\pgfpathrectangle{\pgfqpoint{0.570343in}{0.331635in}}{\pgfqpoint{4.960000in}{3.696000in}}%
\pgfusepath{clip}%
\pgfsetbuttcap%
\pgfsetroundjoin%
\definecolor{currentfill}{rgb}{0.631373,0.788235,0.956863}%
\pgfsetfillcolor{currentfill}%
\pgfsetlinewidth{0.481800pt}%
\definecolor{currentstroke}{rgb}{1.000000,1.000000,1.000000}%
\pgfsetstrokecolor{currentstroke}%
\pgfsetdash{}{0pt}%
\pgfpathmoveto{\pgfqpoint{1.649989in}{2.063703in}}%
\pgfpathcurveto{\pgfqpoint{1.661039in}{2.063703in}}{\pgfqpoint{1.671638in}{2.068094in}}{\pgfqpoint{1.679451in}{2.075907in}}%
\pgfpathcurveto{\pgfqpoint{1.687265in}{2.083721in}}{\pgfqpoint{1.691655in}{2.094320in}}{\pgfqpoint{1.691655in}{2.105370in}}%
\pgfpathcurveto{\pgfqpoint{1.691655in}{2.116420in}}{\pgfqpoint{1.687265in}{2.127019in}}{\pgfqpoint{1.679451in}{2.134833in}}%
\pgfpathcurveto{\pgfqpoint{1.671638in}{2.142646in}}{\pgfqpoint{1.661039in}{2.147037in}}{\pgfqpoint{1.649989in}{2.147037in}}%
\pgfpathcurveto{\pgfqpoint{1.638939in}{2.147037in}}{\pgfqpoint{1.628340in}{2.142646in}}{\pgfqpoint{1.620526in}{2.134833in}}%
\pgfpathcurveto{\pgfqpoint{1.612712in}{2.127019in}}{\pgfqpoint{1.608322in}{2.116420in}}{\pgfqpoint{1.608322in}{2.105370in}}%
\pgfpathcurveto{\pgfqpoint{1.608322in}{2.094320in}}{\pgfqpoint{1.612712in}{2.083721in}}{\pgfqpoint{1.620526in}{2.075907in}}%
\pgfpathcurveto{\pgfqpoint{1.628340in}{2.068094in}}{\pgfqpoint{1.638939in}{2.063703in}}{\pgfqpoint{1.649989in}{2.063703in}}%
\pgfpathclose%
\pgfusepath{stroke,fill}%
\end{pgfscope}%
\begin{pgfscope}%
\pgfpathrectangle{\pgfqpoint{0.570343in}{0.331635in}}{\pgfqpoint{4.960000in}{3.696000in}}%
\pgfusepath{clip}%
\pgfsetbuttcap%
\pgfsetroundjoin%
\definecolor{currentfill}{rgb}{0.631373,0.788235,0.956863}%
\pgfsetfillcolor{currentfill}%
\pgfsetlinewidth{0.481800pt}%
\definecolor{currentstroke}{rgb}{1.000000,1.000000,1.000000}%
\pgfsetstrokecolor{currentstroke}%
\pgfsetdash{}{0pt}%
\pgfpathmoveto{\pgfqpoint{4.280136in}{1.475141in}}%
\pgfpathcurveto{\pgfqpoint{4.291186in}{1.475141in}}{\pgfqpoint{4.301785in}{1.479531in}}{\pgfqpoint{4.309599in}{1.487345in}}%
\pgfpathcurveto{\pgfqpoint{4.317412in}{1.495158in}}{\pgfqpoint{4.321803in}{1.505757in}}{\pgfqpoint{4.321803in}{1.516808in}}%
\pgfpathcurveto{\pgfqpoint{4.321803in}{1.527858in}}{\pgfqpoint{4.317412in}{1.538457in}}{\pgfqpoint{4.309599in}{1.546270in}}%
\pgfpathcurveto{\pgfqpoint{4.301785in}{1.554084in}}{\pgfqpoint{4.291186in}{1.558474in}}{\pgfqpoint{4.280136in}{1.558474in}}%
\pgfpathcurveto{\pgfqpoint{4.269086in}{1.558474in}}{\pgfqpoint{4.258487in}{1.554084in}}{\pgfqpoint{4.250673in}{1.546270in}}%
\pgfpathcurveto{\pgfqpoint{4.242860in}{1.538457in}}{\pgfqpoint{4.238469in}{1.527858in}}{\pgfqpoint{4.238469in}{1.516808in}}%
\pgfpathcurveto{\pgfqpoint{4.238469in}{1.505757in}}{\pgfqpoint{4.242860in}{1.495158in}}{\pgfqpoint{4.250673in}{1.487345in}}%
\pgfpathcurveto{\pgfqpoint{4.258487in}{1.479531in}}{\pgfqpoint{4.269086in}{1.475141in}}{\pgfqpoint{4.280136in}{1.475141in}}%
\pgfpathclose%
\pgfusepath{stroke,fill}%
\end{pgfscope}%
\begin{pgfscope}%
\pgfpathrectangle{\pgfqpoint{0.570343in}{0.331635in}}{\pgfqpoint{4.960000in}{3.696000in}}%
\pgfusepath{clip}%
\pgfsetbuttcap%
\pgfsetroundjoin%
\definecolor{currentfill}{rgb}{0.631373,0.788235,0.956863}%
\pgfsetfillcolor{currentfill}%
\pgfsetlinewidth{0.481800pt}%
\definecolor{currentstroke}{rgb}{1.000000,1.000000,1.000000}%
\pgfsetstrokecolor{currentstroke}%
\pgfsetdash{}{0pt}%
\pgfpathmoveto{\pgfqpoint{2.472600in}{1.865496in}}%
\pgfpathcurveto{\pgfqpoint{2.483650in}{1.865496in}}{\pgfqpoint{2.494249in}{1.869886in}}{\pgfqpoint{2.502062in}{1.877700in}}%
\pgfpathcurveto{\pgfqpoint{2.509876in}{1.885513in}}{\pgfqpoint{2.514266in}{1.896112in}}{\pgfqpoint{2.514266in}{1.907162in}}%
\pgfpathcurveto{\pgfqpoint{2.514266in}{1.918213in}}{\pgfqpoint{2.509876in}{1.928812in}}{\pgfqpoint{2.502062in}{1.936625in}}%
\pgfpathcurveto{\pgfqpoint{2.494249in}{1.944439in}}{\pgfqpoint{2.483650in}{1.948829in}}{\pgfqpoint{2.472600in}{1.948829in}}%
\pgfpathcurveto{\pgfqpoint{2.461549in}{1.948829in}}{\pgfqpoint{2.450950in}{1.944439in}}{\pgfqpoint{2.443137in}{1.936625in}}%
\pgfpathcurveto{\pgfqpoint{2.435323in}{1.928812in}}{\pgfqpoint{2.430933in}{1.918213in}}{\pgfqpoint{2.430933in}{1.907162in}}%
\pgfpathcurveto{\pgfqpoint{2.430933in}{1.896112in}}{\pgfqpoint{2.435323in}{1.885513in}}{\pgfqpoint{2.443137in}{1.877700in}}%
\pgfpathcurveto{\pgfqpoint{2.450950in}{1.869886in}}{\pgfqpoint{2.461549in}{1.865496in}}{\pgfqpoint{2.472600in}{1.865496in}}%
\pgfpathclose%
\pgfusepath{stroke,fill}%
\end{pgfscope}%
\begin{pgfscope}%
\pgfpathrectangle{\pgfqpoint{0.570343in}{0.331635in}}{\pgfqpoint{4.960000in}{3.696000in}}%
\pgfusepath{clip}%
\pgfsetbuttcap%
\pgfsetroundjoin%
\definecolor{currentfill}{rgb}{0.631373,0.788235,0.956863}%
\pgfsetfillcolor{currentfill}%
\pgfsetlinewidth{0.481800pt}%
\definecolor{currentstroke}{rgb}{1.000000,1.000000,1.000000}%
\pgfsetstrokecolor{currentstroke}%
\pgfsetdash{}{0pt}%
\pgfpathmoveto{\pgfqpoint{3.832697in}{2.104501in}}%
\pgfpathcurveto{\pgfqpoint{3.843747in}{2.104501in}}{\pgfqpoint{3.854346in}{2.108891in}}{\pgfqpoint{3.862159in}{2.116704in}}%
\pgfpathcurveto{\pgfqpoint{3.869973in}{2.124518in}}{\pgfqpoint{3.874363in}{2.135117in}}{\pgfqpoint{3.874363in}{2.146167in}}%
\pgfpathcurveto{\pgfqpoint{3.874363in}{2.157217in}}{\pgfqpoint{3.869973in}{2.167816in}}{\pgfqpoint{3.862159in}{2.175630in}}%
\pgfpathcurveto{\pgfqpoint{3.854346in}{2.183444in}}{\pgfqpoint{3.843747in}{2.187834in}}{\pgfqpoint{3.832697in}{2.187834in}}%
\pgfpathcurveto{\pgfqpoint{3.821646in}{2.187834in}}{\pgfqpoint{3.811047in}{2.183444in}}{\pgfqpoint{3.803234in}{2.175630in}}%
\pgfpathcurveto{\pgfqpoint{3.795420in}{2.167816in}}{\pgfqpoint{3.791030in}{2.157217in}}{\pgfqpoint{3.791030in}{2.146167in}}%
\pgfpathcurveto{\pgfqpoint{3.791030in}{2.135117in}}{\pgfqpoint{3.795420in}{2.124518in}}{\pgfqpoint{3.803234in}{2.116704in}}%
\pgfpathcurveto{\pgfqpoint{3.811047in}{2.108891in}}{\pgfqpoint{3.821646in}{2.104501in}}{\pgfqpoint{3.832697in}{2.104501in}}%
\pgfpathclose%
\pgfusepath{stroke,fill}%
\end{pgfscope}%
\begin{pgfscope}%
\pgfpathrectangle{\pgfqpoint{0.570343in}{0.331635in}}{\pgfqpoint{4.960000in}{3.696000in}}%
\pgfusepath{clip}%
\pgfsetbuttcap%
\pgfsetroundjoin%
\definecolor{currentfill}{rgb}{0.631373,0.788235,0.956863}%
\pgfsetfillcolor{currentfill}%
\pgfsetlinewidth{0.481800pt}%
\definecolor{currentstroke}{rgb}{1.000000,1.000000,1.000000}%
\pgfsetstrokecolor{currentstroke}%
\pgfsetdash{}{0pt}%
\pgfpathmoveto{\pgfqpoint{2.958348in}{1.963140in}}%
\pgfpathcurveto{\pgfqpoint{2.969398in}{1.963140in}}{\pgfqpoint{2.979997in}{1.967530in}}{\pgfqpoint{2.987810in}{1.975344in}}%
\pgfpathcurveto{\pgfqpoint{2.995624in}{1.983158in}}{\pgfqpoint{3.000014in}{1.993757in}}{\pgfqpoint{3.000014in}{2.004807in}}%
\pgfpathcurveto{\pgfqpoint{3.000014in}{2.015857in}}{\pgfqpoint{2.995624in}{2.026456in}}{\pgfqpoint{2.987810in}{2.034270in}}%
\pgfpathcurveto{\pgfqpoint{2.979997in}{2.042083in}}{\pgfqpoint{2.969398in}{2.046473in}}{\pgfqpoint{2.958348in}{2.046473in}}%
\pgfpathcurveto{\pgfqpoint{2.947297in}{2.046473in}}{\pgfqpoint{2.936698in}{2.042083in}}{\pgfqpoint{2.928885in}{2.034270in}}%
\pgfpathcurveto{\pgfqpoint{2.921071in}{2.026456in}}{\pgfqpoint{2.916681in}{2.015857in}}{\pgfqpoint{2.916681in}{2.004807in}}%
\pgfpathcurveto{\pgfqpoint{2.916681in}{1.993757in}}{\pgfqpoint{2.921071in}{1.983158in}}{\pgfqpoint{2.928885in}{1.975344in}}%
\pgfpathcurveto{\pgfqpoint{2.936698in}{1.967530in}}{\pgfqpoint{2.947297in}{1.963140in}}{\pgfqpoint{2.958348in}{1.963140in}}%
\pgfpathclose%
\pgfusepath{stroke,fill}%
\end{pgfscope}%
\begin{pgfscope}%
\pgfpathrectangle{\pgfqpoint{0.570343in}{0.331635in}}{\pgfqpoint{4.960000in}{3.696000in}}%
\pgfusepath{clip}%
\pgfsetbuttcap%
\pgfsetroundjoin%
\definecolor{currentfill}{rgb}{0.631373,0.788235,0.956863}%
\pgfsetfillcolor{currentfill}%
\pgfsetlinewidth{0.481800pt}%
\definecolor{currentstroke}{rgb}{1.000000,1.000000,1.000000}%
\pgfsetstrokecolor{currentstroke}%
\pgfsetdash{}{0pt}%
\pgfpathmoveto{\pgfqpoint{1.527720in}{1.130130in}}%
\pgfpathcurveto{\pgfqpoint{1.538771in}{1.130130in}}{\pgfqpoint{1.549370in}{1.134520in}}{\pgfqpoint{1.557183in}{1.142333in}}%
\pgfpathcurveto{\pgfqpoint{1.564997in}{1.150147in}}{\pgfqpoint{1.569387in}{1.160746in}}{\pgfqpoint{1.569387in}{1.171796in}}%
\pgfpathcurveto{\pgfqpoint{1.569387in}{1.182846in}}{\pgfqpoint{1.564997in}{1.193445in}}{\pgfqpoint{1.557183in}{1.201259in}}%
\pgfpathcurveto{\pgfqpoint{1.549370in}{1.209073in}}{\pgfqpoint{1.538771in}{1.213463in}}{\pgfqpoint{1.527720in}{1.213463in}}%
\pgfpathcurveto{\pgfqpoint{1.516670in}{1.213463in}}{\pgfqpoint{1.506071in}{1.209073in}}{\pgfqpoint{1.498258in}{1.201259in}}%
\pgfpathcurveto{\pgfqpoint{1.490444in}{1.193445in}}{\pgfqpoint{1.486054in}{1.182846in}}{\pgfqpoint{1.486054in}{1.171796in}}%
\pgfpathcurveto{\pgfqpoint{1.486054in}{1.160746in}}{\pgfqpoint{1.490444in}{1.150147in}}{\pgfqpoint{1.498258in}{1.142333in}}%
\pgfpathcurveto{\pgfqpoint{1.506071in}{1.134520in}}{\pgfqpoint{1.516670in}{1.130130in}}{\pgfqpoint{1.527720in}{1.130130in}}%
\pgfpathclose%
\pgfusepath{stroke,fill}%
\end{pgfscope}%
\begin{pgfscope}%
\pgfpathrectangle{\pgfqpoint{0.570343in}{0.331635in}}{\pgfqpoint{4.960000in}{3.696000in}}%
\pgfusepath{clip}%
\pgfsetbuttcap%
\pgfsetroundjoin%
\definecolor{currentfill}{rgb}{0.631373,0.788235,0.956863}%
\pgfsetfillcolor{currentfill}%
\pgfsetlinewidth{0.481800pt}%
\definecolor{currentstroke}{rgb}{1.000000,1.000000,1.000000}%
\pgfsetstrokecolor{currentstroke}%
\pgfsetdash{}{0pt}%
\pgfpathmoveto{\pgfqpoint{2.978471in}{0.598498in}}%
\pgfpathcurveto{\pgfqpoint{2.989521in}{0.598498in}}{\pgfqpoint{3.000120in}{0.602888in}}{\pgfqpoint{3.007934in}{0.610702in}}%
\pgfpathcurveto{\pgfqpoint{3.015747in}{0.618515in}}{\pgfqpoint{3.020138in}{0.629114in}}{\pgfqpoint{3.020138in}{0.640164in}}%
\pgfpathcurveto{\pgfqpoint{3.020138in}{0.651215in}}{\pgfqpoint{3.015747in}{0.661814in}}{\pgfqpoint{3.007934in}{0.669627in}}%
\pgfpathcurveto{\pgfqpoint{3.000120in}{0.677441in}}{\pgfqpoint{2.989521in}{0.681831in}}{\pgfqpoint{2.978471in}{0.681831in}}%
\pgfpathcurveto{\pgfqpoint{2.967421in}{0.681831in}}{\pgfqpoint{2.956822in}{0.677441in}}{\pgfqpoint{2.949008in}{0.669627in}}%
\pgfpathcurveto{\pgfqpoint{2.941195in}{0.661814in}}{\pgfqpoint{2.936804in}{0.651215in}}{\pgfqpoint{2.936804in}{0.640164in}}%
\pgfpathcurveto{\pgfqpoint{2.936804in}{0.629114in}}{\pgfqpoint{2.941195in}{0.618515in}}{\pgfqpoint{2.949008in}{0.610702in}}%
\pgfpathcurveto{\pgfqpoint{2.956822in}{0.602888in}}{\pgfqpoint{2.967421in}{0.598498in}}{\pgfqpoint{2.978471in}{0.598498in}}%
\pgfpathclose%
\pgfusepath{stroke,fill}%
\end{pgfscope}%
\begin{pgfscope}%
\pgfpathrectangle{\pgfqpoint{0.570343in}{0.331635in}}{\pgfqpoint{4.960000in}{3.696000in}}%
\pgfusepath{clip}%
\pgfsetbuttcap%
\pgfsetroundjoin%
\definecolor{currentfill}{rgb}{0.631373,0.788235,0.956863}%
\pgfsetfillcolor{currentfill}%
\pgfsetlinewidth{0.481800pt}%
\definecolor{currentstroke}{rgb}{1.000000,1.000000,1.000000}%
\pgfsetstrokecolor{currentstroke}%
\pgfsetdash{}{0pt}%
\pgfpathmoveto{\pgfqpoint{2.744542in}{1.066861in}}%
\pgfpathcurveto{\pgfqpoint{2.755592in}{1.066861in}}{\pgfqpoint{2.766191in}{1.071251in}}{\pgfqpoint{2.774005in}{1.079065in}}%
\pgfpathcurveto{\pgfqpoint{2.781819in}{1.086879in}}{\pgfqpoint{2.786209in}{1.097478in}}{\pgfqpoint{2.786209in}{1.108528in}}%
\pgfpathcurveto{\pgfqpoint{2.786209in}{1.119578in}}{\pgfqpoint{2.781819in}{1.130177in}}{\pgfqpoint{2.774005in}{1.137991in}}%
\pgfpathcurveto{\pgfqpoint{2.766191in}{1.145804in}}{\pgfqpoint{2.755592in}{1.150194in}}{\pgfqpoint{2.744542in}{1.150194in}}%
\pgfpathcurveto{\pgfqpoint{2.733492in}{1.150194in}}{\pgfqpoint{2.722893in}{1.145804in}}{\pgfqpoint{2.715080in}{1.137991in}}%
\pgfpathcurveto{\pgfqpoint{2.707266in}{1.130177in}}{\pgfqpoint{2.702876in}{1.119578in}}{\pgfqpoint{2.702876in}{1.108528in}}%
\pgfpathcurveto{\pgfqpoint{2.702876in}{1.097478in}}{\pgfqpoint{2.707266in}{1.086879in}}{\pgfqpoint{2.715080in}{1.079065in}}%
\pgfpathcurveto{\pgfqpoint{2.722893in}{1.071251in}}{\pgfqpoint{2.733492in}{1.066861in}}{\pgfqpoint{2.744542in}{1.066861in}}%
\pgfpathclose%
\pgfusepath{stroke,fill}%
\end{pgfscope}%
\begin{pgfscope}%
\pgfpathrectangle{\pgfqpoint{0.570343in}{0.331635in}}{\pgfqpoint{4.960000in}{3.696000in}}%
\pgfusepath{clip}%
\pgfsetbuttcap%
\pgfsetroundjoin%
\definecolor{currentfill}{rgb}{0.631373,0.788235,0.956863}%
\pgfsetfillcolor{currentfill}%
\pgfsetlinewidth{0.481800pt}%
\definecolor{currentstroke}{rgb}{1.000000,1.000000,1.000000}%
\pgfsetstrokecolor{currentstroke}%
\pgfsetdash{}{0pt}%
\pgfpathmoveto{\pgfqpoint{1.808176in}{1.610459in}}%
\pgfpathcurveto{\pgfqpoint{1.819226in}{1.610459in}}{\pgfqpoint{1.829825in}{1.614849in}}{\pgfqpoint{1.837639in}{1.622663in}}%
\pgfpathcurveto{\pgfqpoint{1.845452in}{1.630477in}}{\pgfqpoint{1.849843in}{1.641076in}}{\pgfqpoint{1.849843in}{1.652126in}}%
\pgfpathcurveto{\pgfqpoint{1.849843in}{1.663176in}}{\pgfqpoint{1.845452in}{1.673775in}}{\pgfqpoint{1.837639in}{1.681589in}}%
\pgfpathcurveto{\pgfqpoint{1.829825in}{1.689402in}}{\pgfqpoint{1.819226in}{1.693793in}}{\pgfqpoint{1.808176in}{1.693793in}}%
\pgfpathcurveto{\pgfqpoint{1.797126in}{1.693793in}}{\pgfqpoint{1.786527in}{1.689402in}}{\pgfqpoint{1.778713in}{1.681589in}}%
\pgfpathcurveto{\pgfqpoint{1.770900in}{1.673775in}}{\pgfqpoint{1.766509in}{1.663176in}}{\pgfqpoint{1.766509in}{1.652126in}}%
\pgfpathcurveto{\pgfqpoint{1.766509in}{1.641076in}}{\pgfqpoint{1.770900in}{1.630477in}}{\pgfqpoint{1.778713in}{1.622663in}}%
\pgfpathcurveto{\pgfqpoint{1.786527in}{1.614849in}}{\pgfqpoint{1.797126in}{1.610459in}}{\pgfqpoint{1.808176in}{1.610459in}}%
\pgfpathclose%
\pgfusepath{stroke,fill}%
\end{pgfscope}%
\begin{pgfscope}%
\pgfpathrectangle{\pgfqpoint{0.570343in}{0.331635in}}{\pgfqpoint{4.960000in}{3.696000in}}%
\pgfusepath{clip}%
\pgfsetbuttcap%
\pgfsetroundjoin%
\definecolor{currentfill}{rgb}{0.631373,0.788235,0.956863}%
\pgfsetfillcolor{currentfill}%
\pgfsetlinewidth{0.481800pt}%
\definecolor{currentstroke}{rgb}{1.000000,1.000000,1.000000}%
\pgfsetstrokecolor{currentstroke}%
\pgfsetdash{}{0pt}%
\pgfpathmoveto{\pgfqpoint{2.390127in}{1.538733in}}%
\pgfpathcurveto{\pgfqpoint{2.401177in}{1.538733in}}{\pgfqpoint{2.411776in}{1.543123in}}{\pgfqpoint{2.419590in}{1.550937in}}%
\pgfpathcurveto{\pgfqpoint{2.427403in}{1.558750in}}{\pgfqpoint{2.431793in}{1.569349in}}{\pgfqpoint{2.431793in}{1.580399in}}%
\pgfpathcurveto{\pgfqpoint{2.431793in}{1.591450in}}{\pgfqpoint{2.427403in}{1.602049in}}{\pgfqpoint{2.419590in}{1.609862in}}%
\pgfpathcurveto{\pgfqpoint{2.411776in}{1.617676in}}{\pgfqpoint{2.401177in}{1.622066in}}{\pgfqpoint{2.390127in}{1.622066in}}%
\pgfpathcurveto{\pgfqpoint{2.379077in}{1.622066in}}{\pgfqpoint{2.368478in}{1.617676in}}{\pgfqpoint{2.360664in}{1.609862in}}%
\pgfpathcurveto{\pgfqpoint{2.352850in}{1.602049in}}{\pgfqpoint{2.348460in}{1.591450in}}{\pgfqpoint{2.348460in}{1.580399in}}%
\pgfpathcurveto{\pgfqpoint{2.348460in}{1.569349in}}{\pgfqpoint{2.352850in}{1.558750in}}{\pgfqpoint{2.360664in}{1.550937in}}%
\pgfpathcurveto{\pgfqpoint{2.368478in}{1.543123in}}{\pgfqpoint{2.379077in}{1.538733in}}{\pgfqpoint{2.390127in}{1.538733in}}%
\pgfpathclose%
\pgfusepath{stroke,fill}%
\end{pgfscope}%
\begin{pgfscope}%
\pgfpathrectangle{\pgfqpoint{0.570343in}{0.331635in}}{\pgfqpoint{4.960000in}{3.696000in}}%
\pgfusepath{clip}%
\pgfsetbuttcap%
\pgfsetroundjoin%
\definecolor{currentfill}{rgb}{0.631373,0.788235,0.956863}%
\pgfsetfillcolor{currentfill}%
\pgfsetlinewidth{0.481800pt}%
\definecolor{currentstroke}{rgb}{1.000000,1.000000,1.000000}%
\pgfsetstrokecolor{currentstroke}%
\pgfsetdash{}{0pt}%
\pgfpathmoveto{\pgfqpoint{2.744744in}{3.138009in}}%
\pgfpathcurveto{\pgfqpoint{2.755795in}{3.138009in}}{\pgfqpoint{2.766394in}{3.142399in}}{\pgfqpoint{2.774207in}{3.150213in}}%
\pgfpathcurveto{\pgfqpoint{2.782021in}{3.158027in}}{\pgfqpoint{2.786411in}{3.168626in}}{\pgfqpoint{2.786411in}{3.179676in}}%
\pgfpathcurveto{\pgfqpoint{2.786411in}{3.190726in}}{\pgfqpoint{2.782021in}{3.201325in}}{\pgfqpoint{2.774207in}{3.209139in}}%
\pgfpathcurveto{\pgfqpoint{2.766394in}{3.216952in}}{\pgfqpoint{2.755795in}{3.221342in}}{\pgfqpoint{2.744744in}{3.221342in}}%
\pgfpathcurveto{\pgfqpoint{2.733694in}{3.221342in}}{\pgfqpoint{2.723095in}{3.216952in}}{\pgfqpoint{2.715282in}{3.209139in}}%
\pgfpathcurveto{\pgfqpoint{2.707468in}{3.201325in}}{\pgfqpoint{2.703078in}{3.190726in}}{\pgfqpoint{2.703078in}{3.179676in}}%
\pgfpathcurveto{\pgfqpoint{2.703078in}{3.168626in}}{\pgfqpoint{2.707468in}{3.158027in}}{\pgfqpoint{2.715282in}{3.150213in}}%
\pgfpathcurveto{\pgfqpoint{2.723095in}{3.142399in}}{\pgfqpoint{2.733694in}{3.138009in}}{\pgfqpoint{2.744744in}{3.138009in}}%
\pgfpathclose%
\pgfusepath{stroke,fill}%
\end{pgfscope}%
\begin{pgfscope}%
\pgfpathrectangle{\pgfqpoint{0.570343in}{0.331635in}}{\pgfqpoint{4.960000in}{3.696000in}}%
\pgfusepath{clip}%
\pgfsetbuttcap%
\pgfsetroundjoin%
\definecolor{currentfill}{rgb}{0.631373,0.788235,0.956863}%
\pgfsetfillcolor{currentfill}%
\pgfsetlinewidth{0.481800pt}%
\definecolor{currentstroke}{rgb}{1.000000,1.000000,1.000000}%
\pgfsetstrokecolor{currentstroke}%
\pgfsetdash{}{0pt}%
\pgfpathmoveto{\pgfqpoint{2.878184in}{1.662238in}}%
\pgfpathcurveto{\pgfqpoint{2.889234in}{1.662238in}}{\pgfqpoint{2.899833in}{1.666629in}}{\pgfqpoint{2.907647in}{1.674442in}}%
\pgfpathcurveto{\pgfqpoint{2.915460in}{1.682256in}}{\pgfqpoint{2.919851in}{1.692855in}}{\pgfqpoint{2.919851in}{1.703905in}}%
\pgfpathcurveto{\pgfqpoint{2.919851in}{1.714955in}}{\pgfqpoint{2.915460in}{1.725554in}}{\pgfqpoint{2.907647in}{1.733368in}}%
\pgfpathcurveto{\pgfqpoint{2.899833in}{1.741181in}}{\pgfqpoint{2.889234in}{1.745572in}}{\pgfqpoint{2.878184in}{1.745572in}}%
\pgfpathcurveto{\pgfqpoint{2.867134in}{1.745572in}}{\pgfqpoint{2.856535in}{1.741181in}}{\pgfqpoint{2.848721in}{1.733368in}}%
\pgfpathcurveto{\pgfqpoint{2.840908in}{1.725554in}}{\pgfqpoint{2.836517in}{1.714955in}}{\pgfqpoint{2.836517in}{1.703905in}}%
\pgfpathcurveto{\pgfqpoint{2.836517in}{1.692855in}}{\pgfqpoint{2.840908in}{1.682256in}}{\pgfqpoint{2.848721in}{1.674442in}}%
\pgfpathcurveto{\pgfqpoint{2.856535in}{1.666629in}}{\pgfqpoint{2.867134in}{1.662238in}}{\pgfqpoint{2.878184in}{1.662238in}}%
\pgfpathclose%
\pgfusepath{stroke,fill}%
\end{pgfscope}%
\begin{pgfscope}%
\pgfpathrectangle{\pgfqpoint{0.570343in}{0.331635in}}{\pgfqpoint{4.960000in}{3.696000in}}%
\pgfusepath{clip}%
\pgfsetbuttcap%
\pgfsetroundjoin%
\definecolor{currentfill}{rgb}{0.631373,0.788235,0.956863}%
\pgfsetfillcolor{currentfill}%
\pgfsetlinewidth{0.481800pt}%
\definecolor{currentstroke}{rgb}{1.000000,1.000000,1.000000}%
\pgfsetstrokecolor{currentstroke}%
\pgfsetdash{}{0pt}%
\pgfpathmoveto{\pgfqpoint{3.759465in}{1.432960in}}%
\pgfpathcurveto{\pgfqpoint{3.770515in}{1.432960in}}{\pgfqpoint{3.781114in}{1.437350in}}{\pgfqpoint{3.788928in}{1.445164in}}%
\pgfpathcurveto{\pgfqpoint{3.796741in}{1.452977in}}{\pgfqpoint{3.801132in}{1.463576in}}{\pgfqpoint{3.801132in}{1.474627in}}%
\pgfpathcurveto{\pgfqpoint{3.801132in}{1.485677in}}{\pgfqpoint{3.796741in}{1.496276in}}{\pgfqpoint{3.788928in}{1.504089in}}%
\pgfpathcurveto{\pgfqpoint{3.781114in}{1.511903in}}{\pgfqpoint{3.770515in}{1.516293in}}{\pgfqpoint{3.759465in}{1.516293in}}%
\pgfpathcurveto{\pgfqpoint{3.748415in}{1.516293in}}{\pgfqpoint{3.737816in}{1.511903in}}{\pgfqpoint{3.730002in}{1.504089in}}%
\pgfpathcurveto{\pgfqpoint{3.722188in}{1.496276in}}{\pgfqpoint{3.717798in}{1.485677in}}{\pgfqpoint{3.717798in}{1.474627in}}%
\pgfpathcurveto{\pgfqpoint{3.717798in}{1.463576in}}{\pgfqpoint{3.722188in}{1.452977in}}{\pgfqpoint{3.730002in}{1.445164in}}%
\pgfpathcurveto{\pgfqpoint{3.737816in}{1.437350in}}{\pgfqpoint{3.748415in}{1.432960in}}{\pgfqpoint{3.759465in}{1.432960in}}%
\pgfpathclose%
\pgfusepath{stroke,fill}%
\end{pgfscope}%
\begin{pgfscope}%
\pgfpathrectangle{\pgfqpoint{0.570343in}{0.331635in}}{\pgfqpoint{4.960000in}{3.696000in}}%
\pgfusepath{clip}%
\pgfsetbuttcap%
\pgfsetroundjoin%
\definecolor{currentfill}{rgb}{1.000000,0.705882,0.509804}%
\pgfsetfillcolor{currentfill}%
\pgfsetlinewidth{0.481800pt}%
\definecolor{currentstroke}{rgb}{1.000000,1.000000,1.000000}%
\pgfsetstrokecolor{currentstroke}%
\pgfsetdash{}{0pt}%
\pgfpathmoveto{\pgfqpoint{4.010786in}{2.428913in}}%
\pgfpathcurveto{\pgfqpoint{4.021837in}{2.428913in}}{\pgfqpoint{4.032436in}{2.433303in}}{\pgfqpoint{4.040249in}{2.441117in}}%
\pgfpathcurveto{\pgfqpoint{4.048063in}{2.448931in}}{\pgfqpoint{4.052453in}{2.459530in}}{\pgfqpoint{4.052453in}{2.470580in}}%
\pgfpathcurveto{\pgfqpoint{4.052453in}{2.481630in}}{\pgfqpoint{4.048063in}{2.492229in}}{\pgfqpoint{4.040249in}{2.500043in}}%
\pgfpathcurveto{\pgfqpoint{4.032436in}{2.507856in}}{\pgfqpoint{4.021837in}{2.512247in}}{\pgfqpoint{4.010786in}{2.512247in}}%
\pgfpathcurveto{\pgfqpoint{3.999736in}{2.512247in}}{\pgfqpoint{3.989137in}{2.507856in}}{\pgfqpoint{3.981324in}{2.500043in}}%
\pgfpathcurveto{\pgfqpoint{3.973510in}{2.492229in}}{\pgfqpoint{3.969120in}{2.481630in}}{\pgfqpoint{3.969120in}{2.470580in}}%
\pgfpathcurveto{\pgfqpoint{3.969120in}{2.459530in}}{\pgfqpoint{3.973510in}{2.448931in}}{\pgfqpoint{3.981324in}{2.441117in}}%
\pgfpathcurveto{\pgfqpoint{3.989137in}{2.433303in}}{\pgfqpoint{3.999736in}{2.428913in}}{\pgfqpoint{4.010786in}{2.428913in}}%
\pgfpathclose%
\pgfusepath{stroke,fill}%
\end{pgfscope}%
\begin{pgfscope}%
\pgfpathrectangle{\pgfqpoint{0.570343in}{0.331635in}}{\pgfqpoint{4.960000in}{3.696000in}}%
\pgfusepath{clip}%
\pgfsetbuttcap%
\pgfsetroundjoin%
\definecolor{currentfill}{rgb}{1.000000,0.705882,0.509804}%
\pgfsetfillcolor{currentfill}%
\pgfsetlinewidth{0.481800pt}%
\definecolor{currentstroke}{rgb}{1.000000,1.000000,1.000000}%
\pgfsetstrokecolor{currentstroke}%
\pgfsetdash{}{0pt}%
\pgfpathmoveto{\pgfqpoint{3.641382in}{3.235775in}}%
\pgfpathcurveto{\pgfqpoint{3.652432in}{3.235775in}}{\pgfqpoint{3.663031in}{3.240166in}}{\pgfqpoint{3.670845in}{3.247979in}}%
\pgfpathcurveto{\pgfqpoint{3.678659in}{3.255793in}}{\pgfqpoint{3.683049in}{3.266392in}}{\pgfqpoint{3.683049in}{3.277442in}}%
\pgfpathcurveto{\pgfqpoint{3.683049in}{3.288492in}}{\pgfqpoint{3.678659in}{3.299091in}}{\pgfqpoint{3.670845in}{3.306905in}}%
\pgfpathcurveto{\pgfqpoint{3.663031in}{3.314718in}}{\pgfqpoint{3.652432in}{3.319109in}}{\pgfqpoint{3.641382in}{3.319109in}}%
\pgfpathcurveto{\pgfqpoint{3.630332in}{3.319109in}}{\pgfqpoint{3.619733in}{3.314718in}}{\pgfqpoint{3.611919in}{3.306905in}}%
\pgfpathcurveto{\pgfqpoint{3.604106in}{3.299091in}}{\pgfqpoint{3.599716in}{3.288492in}}{\pgfqpoint{3.599716in}{3.277442in}}%
\pgfpathcurveto{\pgfqpoint{3.599716in}{3.266392in}}{\pgfqpoint{3.604106in}{3.255793in}}{\pgfqpoint{3.611919in}{3.247979in}}%
\pgfpathcurveto{\pgfqpoint{3.619733in}{3.240166in}}{\pgfqpoint{3.630332in}{3.235775in}}{\pgfqpoint{3.641382in}{3.235775in}}%
\pgfpathclose%
\pgfusepath{stroke,fill}%
\end{pgfscope}%
\begin{pgfscope}%
\pgfpathrectangle{\pgfqpoint{0.570343in}{0.331635in}}{\pgfqpoint{4.960000in}{3.696000in}}%
\pgfusepath{clip}%
\pgfsetbuttcap%
\pgfsetroundjoin%
\definecolor{currentfill}{rgb}{1.000000,0.705882,0.509804}%
\pgfsetfillcolor{currentfill}%
\pgfsetlinewidth{0.481800pt}%
\definecolor{currentstroke}{rgb}{1.000000,1.000000,1.000000}%
\pgfsetstrokecolor{currentstroke}%
\pgfsetdash{}{0pt}%
\pgfpathmoveto{\pgfqpoint{2.910306in}{2.493024in}}%
\pgfpathcurveto{\pgfqpoint{2.921356in}{2.493024in}}{\pgfqpoint{2.931955in}{2.497414in}}{\pgfqpoint{2.939768in}{2.505228in}}%
\pgfpathcurveto{\pgfqpoint{2.947582in}{2.513041in}}{\pgfqpoint{2.951972in}{2.523640in}}{\pgfqpoint{2.951972in}{2.534690in}}%
\pgfpathcurveto{\pgfqpoint{2.951972in}{2.545740in}}{\pgfqpoint{2.947582in}{2.556340in}}{\pgfqpoint{2.939768in}{2.564153in}}%
\pgfpathcurveto{\pgfqpoint{2.931955in}{2.571967in}}{\pgfqpoint{2.921356in}{2.576357in}}{\pgfqpoint{2.910306in}{2.576357in}}%
\pgfpathcurveto{\pgfqpoint{2.899256in}{2.576357in}}{\pgfqpoint{2.888656in}{2.571967in}}{\pgfqpoint{2.880843in}{2.564153in}}%
\pgfpathcurveto{\pgfqpoint{2.873029in}{2.556340in}}{\pgfqpoint{2.868639in}{2.545740in}}{\pgfqpoint{2.868639in}{2.534690in}}%
\pgfpathcurveto{\pgfqpoint{2.868639in}{2.523640in}}{\pgfqpoint{2.873029in}{2.513041in}}{\pgfqpoint{2.880843in}{2.505228in}}%
\pgfpathcurveto{\pgfqpoint{2.888656in}{2.497414in}}{\pgfqpoint{2.899256in}{2.493024in}}{\pgfqpoint{2.910306in}{2.493024in}}%
\pgfpathclose%
\pgfusepath{stroke,fill}%
\end{pgfscope}%
\begin{pgfscope}%
\pgfpathrectangle{\pgfqpoint{0.570343in}{0.331635in}}{\pgfqpoint{4.960000in}{3.696000in}}%
\pgfusepath{clip}%
\pgfsetbuttcap%
\pgfsetroundjoin%
\definecolor{currentfill}{rgb}{1.000000,0.705882,0.509804}%
\pgfsetfillcolor{currentfill}%
\pgfsetlinewidth{0.481800pt}%
\definecolor{currentstroke}{rgb}{1.000000,1.000000,1.000000}%
\pgfsetstrokecolor{currentstroke}%
\pgfsetdash{}{0pt}%
\pgfpathmoveto{\pgfqpoint{2.662402in}{2.751085in}}%
\pgfpathcurveto{\pgfqpoint{2.673452in}{2.751085in}}{\pgfqpoint{2.684051in}{2.755475in}}{\pgfqpoint{2.691864in}{2.763289in}}%
\pgfpathcurveto{\pgfqpoint{2.699678in}{2.771103in}}{\pgfqpoint{2.704068in}{2.781702in}}{\pgfqpoint{2.704068in}{2.792752in}}%
\pgfpathcurveto{\pgfqpoint{2.704068in}{2.803802in}}{\pgfqpoint{2.699678in}{2.814401in}}{\pgfqpoint{2.691864in}{2.822214in}}%
\pgfpathcurveto{\pgfqpoint{2.684051in}{2.830028in}}{\pgfqpoint{2.673452in}{2.834418in}}{\pgfqpoint{2.662402in}{2.834418in}}%
\pgfpathcurveto{\pgfqpoint{2.651351in}{2.834418in}}{\pgfqpoint{2.640752in}{2.830028in}}{\pgfqpoint{2.632939in}{2.822214in}}%
\pgfpathcurveto{\pgfqpoint{2.625125in}{2.814401in}}{\pgfqpoint{2.620735in}{2.803802in}}{\pgfqpoint{2.620735in}{2.792752in}}%
\pgfpathcurveto{\pgfqpoint{2.620735in}{2.781702in}}{\pgfqpoint{2.625125in}{2.771103in}}{\pgfqpoint{2.632939in}{2.763289in}}%
\pgfpathcurveto{\pgfqpoint{2.640752in}{2.755475in}}{\pgfqpoint{2.651351in}{2.751085in}}{\pgfqpoint{2.662402in}{2.751085in}}%
\pgfpathclose%
\pgfusepath{stroke,fill}%
\end{pgfscope}%
\begin{pgfscope}%
\pgfpathrectangle{\pgfqpoint{0.570343in}{0.331635in}}{\pgfqpoint{4.960000in}{3.696000in}}%
\pgfusepath{clip}%
\pgfsetbuttcap%
\pgfsetroundjoin%
\definecolor{currentfill}{rgb}{1.000000,0.705882,0.509804}%
\pgfsetfillcolor{currentfill}%
\pgfsetlinewidth{0.481800pt}%
\definecolor{currentstroke}{rgb}{1.000000,1.000000,1.000000}%
\pgfsetstrokecolor{currentstroke}%
\pgfsetdash{}{0pt}%
\pgfpathmoveto{\pgfqpoint{1.385602in}{3.454654in}}%
\pgfpathcurveto{\pgfqpoint{1.396652in}{3.454654in}}{\pgfqpoint{1.407251in}{3.459044in}}{\pgfqpoint{1.415065in}{3.466858in}}%
\pgfpathcurveto{\pgfqpoint{1.422879in}{3.474671in}}{\pgfqpoint{1.427269in}{3.485270in}}{\pgfqpoint{1.427269in}{3.496321in}}%
\pgfpathcurveto{\pgfqpoint{1.427269in}{3.507371in}}{\pgfqpoint{1.422879in}{3.517970in}}{\pgfqpoint{1.415065in}{3.525783in}}%
\pgfpathcurveto{\pgfqpoint{1.407251in}{3.533597in}}{\pgfqpoint{1.396652in}{3.537987in}}{\pgfqpoint{1.385602in}{3.537987in}}%
\pgfpathcurveto{\pgfqpoint{1.374552in}{3.537987in}}{\pgfqpoint{1.363953in}{3.533597in}}{\pgfqpoint{1.356139in}{3.525783in}}%
\pgfpathcurveto{\pgfqpoint{1.348326in}{3.517970in}}{\pgfqpoint{1.343936in}{3.507371in}}{\pgfqpoint{1.343936in}{3.496321in}}%
\pgfpathcurveto{\pgfqpoint{1.343936in}{3.485270in}}{\pgfqpoint{1.348326in}{3.474671in}}{\pgfqpoint{1.356139in}{3.466858in}}%
\pgfpathcurveto{\pgfqpoint{1.363953in}{3.459044in}}{\pgfqpoint{1.374552in}{3.454654in}}{\pgfqpoint{1.385602in}{3.454654in}}%
\pgfpathclose%
\pgfusepath{stroke,fill}%
\end{pgfscope}%
\begin{pgfscope}%
\pgfpathrectangle{\pgfqpoint{0.570343in}{0.331635in}}{\pgfqpoint{4.960000in}{3.696000in}}%
\pgfusepath{clip}%
\pgfsetbuttcap%
\pgfsetroundjoin%
\definecolor{currentfill}{rgb}{1.000000,0.705882,0.509804}%
\pgfsetfillcolor{currentfill}%
\pgfsetlinewidth{0.481800pt}%
\definecolor{currentstroke}{rgb}{1.000000,1.000000,1.000000}%
\pgfsetstrokecolor{currentstroke}%
\pgfsetdash{}{0pt}%
\pgfpathmoveto{\pgfqpoint{4.718935in}{1.875375in}}%
\pgfpathcurveto{\pgfqpoint{4.729985in}{1.875375in}}{\pgfqpoint{4.740584in}{1.879765in}}{\pgfqpoint{4.748398in}{1.887579in}}%
\pgfpathcurveto{\pgfqpoint{4.756212in}{1.895392in}}{\pgfqpoint{4.760602in}{1.905991in}}{\pgfqpoint{4.760602in}{1.917042in}}%
\pgfpathcurveto{\pgfqpoint{4.760602in}{1.928092in}}{\pgfqpoint{4.756212in}{1.938691in}}{\pgfqpoint{4.748398in}{1.946504in}}%
\pgfpathcurveto{\pgfqpoint{4.740584in}{1.954318in}}{\pgfqpoint{4.729985in}{1.958708in}}{\pgfqpoint{4.718935in}{1.958708in}}%
\pgfpathcurveto{\pgfqpoint{4.707885in}{1.958708in}}{\pgfqpoint{4.697286in}{1.954318in}}{\pgfqpoint{4.689472in}{1.946504in}}%
\pgfpathcurveto{\pgfqpoint{4.681659in}{1.938691in}}{\pgfqpoint{4.677268in}{1.928092in}}{\pgfqpoint{4.677268in}{1.917042in}}%
\pgfpathcurveto{\pgfqpoint{4.677268in}{1.905991in}}{\pgfqpoint{4.681659in}{1.895392in}}{\pgfqpoint{4.689472in}{1.887579in}}%
\pgfpathcurveto{\pgfqpoint{4.697286in}{1.879765in}}{\pgfqpoint{4.707885in}{1.875375in}}{\pgfqpoint{4.718935in}{1.875375in}}%
\pgfpathclose%
\pgfusepath{stroke,fill}%
\end{pgfscope}%
\begin{pgfscope}%
\pgfpathrectangle{\pgfqpoint{0.570343in}{0.331635in}}{\pgfqpoint{4.960000in}{3.696000in}}%
\pgfusepath{clip}%
\pgfsetbuttcap%
\pgfsetroundjoin%
\definecolor{currentfill}{rgb}{1.000000,0.705882,0.509804}%
\pgfsetfillcolor{currentfill}%
\pgfsetlinewidth{0.481800pt}%
\definecolor{currentstroke}{rgb}{1.000000,1.000000,1.000000}%
\pgfsetstrokecolor{currentstroke}%
\pgfsetdash{}{0pt}%
\pgfpathmoveto{\pgfqpoint{1.526138in}{2.872462in}}%
\pgfpathcurveto{\pgfqpoint{1.537188in}{2.872462in}}{\pgfqpoint{1.547787in}{2.876852in}}{\pgfqpoint{1.555601in}{2.884666in}}%
\pgfpathcurveto{\pgfqpoint{1.563414in}{2.892479in}}{\pgfqpoint{1.567805in}{2.903079in}}{\pgfqpoint{1.567805in}{2.914129in}}%
\pgfpathcurveto{\pgfqpoint{1.567805in}{2.925179in}}{\pgfqpoint{1.563414in}{2.935778in}}{\pgfqpoint{1.555601in}{2.943591in}}%
\pgfpathcurveto{\pgfqpoint{1.547787in}{2.951405in}}{\pgfqpoint{1.537188in}{2.955795in}}{\pgfqpoint{1.526138in}{2.955795in}}%
\pgfpathcurveto{\pgfqpoint{1.515088in}{2.955795in}}{\pgfqpoint{1.504489in}{2.951405in}}{\pgfqpoint{1.496675in}{2.943591in}}%
\pgfpathcurveto{\pgfqpoint{1.488862in}{2.935778in}}{\pgfqpoint{1.484471in}{2.925179in}}{\pgfqpoint{1.484471in}{2.914129in}}%
\pgfpathcurveto{\pgfqpoint{1.484471in}{2.903079in}}{\pgfqpoint{1.488862in}{2.892479in}}{\pgfqpoint{1.496675in}{2.884666in}}%
\pgfpathcurveto{\pgfqpoint{1.504489in}{2.876852in}}{\pgfqpoint{1.515088in}{2.872462in}}{\pgfqpoint{1.526138in}{2.872462in}}%
\pgfpathclose%
\pgfusepath{stroke,fill}%
\end{pgfscope}%
\begin{pgfscope}%
\pgfpathrectangle{\pgfqpoint{0.570343in}{0.331635in}}{\pgfqpoint{4.960000in}{3.696000in}}%
\pgfusepath{clip}%
\pgfsetbuttcap%
\pgfsetroundjoin%
\definecolor{currentfill}{rgb}{1.000000,0.705882,0.509804}%
\pgfsetfillcolor{currentfill}%
\pgfsetlinewidth{0.481800pt}%
\definecolor{currentstroke}{rgb}{1.000000,1.000000,1.000000}%
\pgfsetstrokecolor{currentstroke}%
\pgfsetdash{}{0pt}%
\pgfpathmoveto{\pgfqpoint{5.011810in}{2.229459in}}%
\pgfpathcurveto{\pgfqpoint{5.022860in}{2.229459in}}{\pgfqpoint{5.033459in}{2.233849in}}{\pgfqpoint{5.041273in}{2.241663in}}%
\pgfpathcurveto{\pgfqpoint{5.049086in}{2.249476in}}{\pgfqpoint{5.053477in}{2.260075in}}{\pgfqpoint{5.053477in}{2.271126in}}%
\pgfpathcurveto{\pgfqpoint{5.053477in}{2.282176in}}{\pgfqpoint{5.049086in}{2.292775in}}{\pgfqpoint{5.041273in}{2.300588in}}%
\pgfpathcurveto{\pgfqpoint{5.033459in}{2.308402in}}{\pgfqpoint{5.022860in}{2.312792in}}{\pgfqpoint{5.011810in}{2.312792in}}%
\pgfpathcurveto{\pgfqpoint{5.000760in}{2.312792in}}{\pgfqpoint{4.990161in}{2.308402in}}{\pgfqpoint{4.982347in}{2.300588in}}%
\pgfpathcurveto{\pgfqpoint{4.974533in}{2.292775in}}{\pgfqpoint{4.970143in}{2.282176in}}{\pgfqpoint{4.970143in}{2.271126in}}%
\pgfpathcurveto{\pgfqpoint{4.970143in}{2.260075in}}{\pgfqpoint{4.974533in}{2.249476in}}{\pgfqpoint{4.982347in}{2.241663in}}%
\pgfpathcurveto{\pgfqpoint{4.990161in}{2.233849in}}{\pgfqpoint{5.000760in}{2.229459in}}{\pgfqpoint{5.011810in}{2.229459in}}%
\pgfpathclose%
\pgfusepath{stroke,fill}%
\end{pgfscope}%
\begin{pgfscope}%
\pgfpathrectangle{\pgfqpoint{0.570343in}{0.331635in}}{\pgfqpoint{4.960000in}{3.696000in}}%
\pgfusepath{clip}%
\pgfsetbuttcap%
\pgfsetroundjoin%
\definecolor{currentfill}{rgb}{1.000000,0.705882,0.509804}%
\pgfsetfillcolor{currentfill}%
\pgfsetlinewidth{0.481800pt}%
\definecolor{currentstroke}{rgb}{1.000000,1.000000,1.000000}%
\pgfsetstrokecolor{currentstroke}%
\pgfsetdash{}{0pt}%
\pgfpathmoveto{\pgfqpoint{4.377389in}{1.083755in}}%
\pgfpathcurveto{\pgfqpoint{4.388439in}{1.083755in}}{\pgfqpoint{4.399038in}{1.088145in}}{\pgfqpoint{4.406852in}{1.095959in}}%
\pgfpathcurveto{\pgfqpoint{4.414665in}{1.103772in}}{\pgfqpoint{4.419056in}{1.114371in}}{\pgfqpoint{4.419056in}{1.125422in}}%
\pgfpathcurveto{\pgfqpoint{4.419056in}{1.136472in}}{\pgfqpoint{4.414665in}{1.147071in}}{\pgfqpoint{4.406852in}{1.154884in}}%
\pgfpathcurveto{\pgfqpoint{4.399038in}{1.162698in}}{\pgfqpoint{4.388439in}{1.167088in}}{\pgfqpoint{4.377389in}{1.167088in}}%
\pgfpathcurveto{\pgfqpoint{4.366339in}{1.167088in}}{\pgfqpoint{4.355740in}{1.162698in}}{\pgfqpoint{4.347926in}{1.154884in}}%
\pgfpathcurveto{\pgfqpoint{4.340113in}{1.147071in}}{\pgfqpoint{4.335722in}{1.136472in}}{\pgfqpoint{4.335722in}{1.125422in}}%
\pgfpathcurveto{\pgfqpoint{4.335722in}{1.114371in}}{\pgfqpoint{4.340113in}{1.103772in}}{\pgfqpoint{4.347926in}{1.095959in}}%
\pgfpathcurveto{\pgfqpoint{4.355740in}{1.088145in}}{\pgfqpoint{4.366339in}{1.083755in}}{\pgfqpoint{4.377389in}{1.083755in}}%
\pgfpathclose%
\pgfusepath{stroke,fill}%
\end{pgfscope}%
\begin{pgfscope}%
\pgfpathrectangle{\pgfqpoint{0.570343in}{0.331635in}}{\pgfqpoint{4.960000in}{3.696000in}}%
\pgfusepath{clip}%
\pgfsetbuttcap%
\pgfsetroundjoin%
\definecolor{currentfill}{rgb}{1.000000,0.705882,0.509804}%
\pgfsetfillcolor{currentfill}%
\pgfsetlinewidth{0.481800pt}%
\definecolor{currentstroke}{rgb}{1.000000,1.000000,1.000000}%
\pgfsetstrokecolor{currentstroke}%
\pgfsetdash{}{0pt}%
\pgfpathmoveto{\pgfqpoint{2.162297in}{2.708119in}}%
\pgfpathcurveto{\pgfqpoint{2.173347in}{2.708119in}}{\pgfqpoint{2.183946in}{2.712509in}}{\pgfqpoint{2.191760in}{2.720323in}}%
\pgfpathcurveto{\pgfqpoint{2.199574in}{2.728136in}}{\pgfqpoint{2.203964in}{2.738736in}}{\pgfqpoint{2.203964in}{2.749786in}}%
\pgfpathcurveto{\pgfqpoint{2.203964in}{2.760836in}}{\pgfqpoint{2.199574in}{2.771435in}}{\pgfqpoint{2.191760in}{2.779248in}}%
\pgfpathcurveto{\pgfqpoint{2.183946in}{2.787062in}}{\pgfqpoint{2.173347in}{2.791452in}}{\pgfqpoint{2.162297in}{2.791452in}}%
\pgfpathcurveto{\pgfqpoint{2.151247in}{2.791452in}}{\pgfqpoint{2.140648in}{2.787062in}}{\pgfqpoint{2.132834in}{2.779248in}}%
\pgfpathcurveto{\pgfqpoint{2.125021in}{2.771435in}}{\pgfqpoint{2.120631in}{2.760836in}}{\pgfqpoint{2.120631in}{2.749786in}}%
\pgfpathcurveto{\pgfqpoint{2.120631in}{2.738736in}}{\pgfqpoint{2.125021in}{2.728136in}}{\pgfqpoint{2.132834in}{2.720323in}}%
\pgfpathcurveto{\pgfqpoint{2.140648in}{2.712509in}}{\pgfqpoint{2.151247in}{2.708119in}}{\pgfqpoint{2.162297in}{2.708119in}}%
\pgfpathclose%
\pgfusepath{stroke,fill}%
\end{pgfscope}%
\begin{pgfscope}%
\pgfpathrectangle{\pgfqpoint{0.570343in}{0.331635in}}{\pgfqpoint{4.960000in}{3.696000in}}%
\pgfusepath{clip}%
\pgfsetbuttcap%
\pgfsetroundjoin%
\definecolor{currentfill}{rgb}{1.000000,0.705882,0.509804}%
\pgfsetfillcolor{currentfill}%
\pgfsetlinewidth{0.481800pt}%
\definecolor{currentstroke}{rgb}{1.000000,1.000000,1.000000}%
\pgfsetstrokecolor{currentstroke}%
\pgfsetdash{}{0pt}%
\pgfpathmoveto{\pgfqpoint{2.464678in}{2.431882in}}%
\pgfpathcurveto{\pgfqpoint{2.475729in}{2.431882in}}{\pgfqpoint{2.486328in}{2.436272in}}{\pgfqpoint{2.494141in}{2.444086in}}%
\pgfpathcurveto{\pgfqpoint{2.501955in}{2.451899in}}{\pgfqpoint{2.506345in}{2.462498in}}{\pgfqpoint{2.506345in}{2.473549in}}%
\pgfpathcurveto{\pgfqpoint{2.506345in}{2.484599in}}{\pgfqpoint{2.501955in}{2.495198in}}{\pgfqpoint{2.494141in}{2.503011in}}%
\pgfpathcurveto{\pgfqpoint{2.486328in}{2.510825in}}{\pgfqpoint{2.475729in}{2.515215in}}{\pgfqpoint{2.464678in}{2.515215in}}%
\pgfpathcurveto{\pgfqpoint{2.453628in}{2.515215in}}{\pgfqpoint{2.443029in}{2.510825in}}{\pgfqpoint{2.435216in}{2.503011in}}%
\pgfpathcurveto{\pgfqpoint{2.427402in}{2.495198in}}{\pgfqpoint{2.423012in}{2.484599in}}{\pgfqpoint{2.423012in}{2.473549in}}%
\pgfpathcurveto{\pgfqpoint{2.423012in}{2.462498in}}{\pgfqpoint{2.427402in}{2.451899in}}{\pgfqpoint{2.435216in}{2.444086in}}%
\pgfpathcurveto{\pgfqpoint{2.443029in}{2.436272in}}{\pgfqpoint{2.453628in}{2.431882in}}{\pgfqpoint{2.464678in}{2.431882in}}%
\pgfpathclose%
\pgfusepath{stroke,fill}%
\end{pgfscope}%
\begin{pgfscope}%
\pgfpathrectangle{\pgfqpoint{0.570343in}{0.331635in}}{\pgfqpoint{4.960000in}{3.696000in}}%
\pgfusepath{clip}%
\pgfsetbuttcap%
\pgfsetroundjoin%
\definecolor{currentfill}{rgb}{1.000000,0.705882,0.509804}%
\pgfsetfillcolor{currentfill}%
\pgfsetlinewidth{0.481800pt}%
\definecolor{currentstroke}{rgb}{1.000000,1.000000,1.000000}%
\pgfsetstrokecolor{currentstroke}%
\pgfsetdash{}{0pt}%
\pgfpathmoveto{\pgfqpoint{0.856192in}{1.529585in}}%
\pgfpathcurveto{\pgfqpoint{0.867242in}{1.529585in}}{\pgfqpoint{0.877841in}{1.533975in}}{\pgfqpoint{0.885655in}{1.541789in}}%
\pgfpathcurveto{\pgfqpoint{0.893468in}{1.549603in}}{\pgfqpoint{0.897859in}{1.560202in}}{\pgfqpoint{0.897859in}{1.571252in}}%
\pgfpathcurveto{\pgfqpoint{0.897859in}{1.582302in}}{\pgfqpoint{0.893468in}{1.592901in}}{\pgfqpoint{0.885655in}{1.600715in}}%
\pgfpathcurveto{\pgfqpoint{0.877841in}{1.608528in}}{\pgfqpoint{0.867242in}{1.612918in}}{\pgfqpoint{0.856192in}{1.612918in}}%
\pgfpathcurveto{\pgfqpoint{0.845142in}{1.612918in}}{\pgfqpoint{0.834543in}{1.608528in}}{\pgfqpoint{0.826729in}{1.600715in}}%
\pgfpathcurveto{\pgfqpoint{0.818916in}{1.592901in}}{\pgfqpoint{0.814525in}{1.582302in}}{\pgfqpoint{0.814525in}{1.571252in}}%
\pgfpathcurveto{\pgfqpoint{0.814525in}{1.560202in}}{\pgfqpoint{0.818916in}{1.549603in}}{\pgfqpoint{0.826729in}{1.541789in}}%
\pgfpathcurveto{\pgfqpoint{0.834543in}{1.533975in}}{\pgfqpoint{0.845142in}{1.529585in}}{\pgfqpoint{0.856192in}{1.529585in}}%
\pgfpathclose%
\pgfusepath{stroke,fill}%
\end{pgfscope}%
\begin{pgfscope}%
\pgfpathrectangle{\pgfqpoint{0.570343in}{0.331635in}}{\pgfqpoint{4.960000in}{3.696000in}}%
\pgfusepath{clip}%
\pgfsetbuttcap%
\pgfsetroundjoin%
\definecolor{currentfill}{rgb}{1.000000,0.705882,0.509804}%
\pgfsetfillcolor{currentfill}%
\pgfsetlinewidth{0.481800pt}%
\definecolor{currentstroke}{rgb}{1.000000,1.000000,1.000000}%
\pgfsetstrokecolor{currentstroke}%
\pgfsetdash{}{0pt}%
\pgfpathmoveto{\pgfqpoint{4.626113in}{3.279314in}}%
\pgfpathcurveto{\pgfqpoint{4.637163in}{3.279314in}}{\pgfqpoint{4.647762in}{3.283704in}}{\pgfqpoint{4.655575in}{3.291518in}}%
\pgfpathcurveto{\pgfqpoint{4.663389in}{3.299332in}}{\pgfqpoint{4.667779in}{3.309931in}}{\pgfqpoint{4.667779in}{3.320981in}}%
\pgfpathcurveto{\pgfqpoint{4.667779in}{3.332031in}}{\pgfqpoint{4.663389in}{3.342630in}}{\pgfqpoint{4.655575in}{3.350444in}}%
\pgfpathcurveto{\pgfqpoint{4.647762in}{3.358257in}}{\pgfqpoint{4.637163in}{3.362648in}}{\pgfqpoint{4.626113in}{3.362648in}}%
\pgfpathcurveto{\pgfqpoint{4.615062in}{3.362648in}}{\pgfqpoint{4.604463in}{3.358257in}}{\pgfqpoint{4.596650in}{3.350444in}}%
\pgfpathcurveto{\pgfqpoint{4.588836in}{3.342630in}}{\pgfqpoint{4.584446in}{3.332031in}}{\pgfqpoint{4.584446in}{3.320981in}}%
\pgfpathcurveto{\pgfqpoint{4.584446in}{3.309931in}}{\pgfqpoint{4.588836in}{3.299332in}}{\pgfqpoint{4.596650in}{3.291518in}}%
\pgfpathcurveto{\pgfqpoint{4.604463in}{3.283704in}}{\pgfqpoint{4.615062in}{3.279314in}}{\pgfqpoint{4.626113in}{3.279314in}}%
\pgfpathclose%
\pgfusepath{stroke,fill}%
\end{pgfscope}%
\begin{pgfscope}%
\pgfpathrectangle{\pgfqpoint{0.570343in}{0.331635in}}{\pgfqpoint{4.960000in}{3.696000in}}%
\pgfusepath{clip}%
\pgfsetbuttcap%
\pgfsetroundjoin%
\definecolor{currentfill}{rgb}{1.000000,0.705882,0.509804}%
\pgfsetfillcolor{currentfill}%
\pgfsetlinewidth{0.481800pt}%
\definecolor{currentstroke}{rgb}{1.000000,1.000000,1.000000}%
\pgfsetstrokecolor{currentstroke}%
\pgfsetdash{}{0pt}%
\pgfpathmoveto{\pgfqpoint{1.796127in}{3.311954in}}%
\pgfpathcurveto{\pgfqpoint{1.807177in}{3.311954in}}{\pgfqpoint{1.817776in}{3.316344in}}{\pgfqpoint{1.825589in}{3.324158in}}%
\pgfpathcurveto{\pgfqpoint{1.833403in}{3.331971in}}{\pgfqpoint{1.837793in}{3.342570in}}{\pgfqpoint{1.837793in}{3.353620in}}%
\pgfpathcurveto{\pgfqpoint{1.837793in}{3.364670in}}{\pgfqpoint{1.833403in}{3.375270in}}{\pgfqpoint{1.825589in}{3.383083in}}%
\pgfpathcurveto{\pgfqpoint{1.817776in}{3.390897in}}{\pgfqpoint{1.807177in}{3.395287in}}{\pgfqpoint{1.796127in}{3.395287in}}%
\pgfpathcurveto{\pgfqpoint{1.785076in}{3.395287in}}{\pgfqpoint{1.774477in}{3.390897in}}{\pgfqpoint{1.766664in}{3.383083in}}%
\pgfpathcurveto{\pgfqpoint{1.758850in}{3.375270in}}{\pgfqpoint{1.754460in}{3.364670in}}{\pgfqpoint{1.754460in}{3.353620in}}%
\pgfpathcurveto{\pgfqpoint{1.754460in}{3.342570in}}{\pgfqpoint{1.758850in}{3.331971in}}{\pgfqpoint{1.766664in}{3.324158in}}%
\pgfpathcurveto{\pgfqpoint{1.774477in}{3.316344in}}{\pgfqpoint{1.785076in}{3.311954in}}{\pgfqpoint{1.796127in}{3.311954in}}%
\pgfpathclose%
\pgfusepath{stroke,fill}%
\end{pgfscope}%
\begin{pgfscope}%
\pgfpathrectangle{\pgfqpoint{0.570343in}{0.331635in}}{\pgfqpoint{4.960000in}{3.696000in}}%
\pgfusepath{clip}%
\pgfsetbuttcap%
\pgfsetroundjoin%
\definecolor{currentfill}{rgb}{1.000000,0.705882,0.509804}%
\pgfsetfillcolor{currentfill}%
\pgfsetlinewidth{0.481800pt}%
\definecolor{currentstroke}{rgb}{1.000000,1.000000,1.000000}%
\pgfsetstrokecolor{currentstroke}%
\pgfsetdash{}{0pt}%
\pgfpathmoveto{\pgfqpoint{4.916919in}{1.019231in}}%
\pgfpathcurveto{\pgfqpoint{4.927969in}{1.019231in}}{\pgfqpoint{4.938568in}{1.023621in}}{\pgfqpoint{4.946382in}{1.031435in}}%
\pgfpathcurveto{\pgfqpoint{4.954195in}{1.039248in}}{\pgfqpoint{4.958586in}{1.049847in}}{\pgfqpoint{4.958586in}{1.060897in}}%
\pgfpathcurveto{\pgfqpoint{4.958586in}{1.071948in}}{\pgfqpoint{4.954195in}{1.082547in}}{\pgfqpoint{4.946382in}{1.090360in}}%
\pgfpathcurveto{\pgfqpoint{4.938568in}{1.098174in}}{\pgfqpoint{4.927969in}{1.102564in}}{\pgfqpoint{4.916919in}{1.102564in}}%
\pgfpathcurveto{\pgfqpoint{4.905869in}{1.102564in}}{\pgfqpoint{4.895270in}{1.098174in}}{\pgfqpoint{4.887456in}{1.090360in}}%
\pgfpathcurveto{\pgfqpoint{4.879643in}{1.082547in}}{\pgfqpoint{4.875252in}{1.071948in}}{\pgfqpoint{4.875252in}{1.060897in}}%
\pgfpathcurveto{\pgfqpoint{4.875252in}{1.049847in}}{\pgfqpoint{4.879643in}{1.039248in}}{\pgfqpoint{4.887456in}{1.031435in}}%
\pgfpathcurveto{\pgfqpoint{4.895270in}{1.023621in}}{\pgfqpoint{4.905869in}{1.019231in}}{\pgfqpoint{4.916919in}{1.019231in}}%
\pgfpathclose%
\pgfusepath{stroke,fill}%
\end{pgfscope}%
\begin{pgfscope}%
\pgfpathrectangle{\pgfqpoint{0.570343in}{0.331635in}}{\pgfqpoint{4.960000in}{3.696000in}}%
\pgfusepath{clip}%
\pgfsetbuttcap%
\pgfsetroundjoin%
\definecolor{currentfill}{rgb}{1.000000,0.705882,0.509804}%
\pgfsetfillcolor{currentfill}%
\pgfsetlinewidth{0.481800pt}%
\definecolor{currentstroke}{rgb}{1.000000,1.000000,1.000000}%
\pgfsetstrokecolor{currentstroke}%
\pgfsetdash{}{0pt}%
\pgfpathmoveto{\pgfqpoint{4.374254in}{2.132308in}}%
\pgfpathcurveto{\pgfqpoint{4.385304in}{2.132308in}}{\pgfqpoint{4.395903in}{2.136698in}}{\pgfqpoint{4.403717in}{2.144511in}}%
\pgfpathcurveto{\pgfqpoint{4.411531in}{2.152325in}}{\pgfqpoint{4.415921in}{2.162924in}}{\pgfqpoint{4.415921in}{2.173974in}}%
\pgfpathcurveto{\pgfqpoint{4.415921in}{2.185024in}}{\pgfqpoint{4.411531in}{2.195623in}}{\pgfqpoint{4.403717in}{2.203437in}}%
\pgfpathcurveto{\pgfqpoint{4.395903in}{2.211251in}}{\pgfqpoint{4.385304in}{2.215641in}}{\pgfqpoint{4.374254in}{2.215641in}}%
\pgfpathcurveto{\pgfqpoint{4.363204in}{2.215641in}}{\pgfqpoint{4.352605in}{2.211251in}}{\pgfqpoint{4.344792in}{2.203437in}}%
\pgfpathcurveto{\pgfqpoint{4.336978in}{2.195623in}}{\pgfqpoint{4.332588in}{2.185024in}}{\pgfqpoint{4.332588in}{2.173974in}}%
\pgfpathcurveto{\pgfqpoint{4.332588in}{2.162924in}}{\pgfqpoint{4.336978in}{2.152325in}}{\pgfqpoint{4.344792in}{2.144511in}}%
\pgfpathcurveto{\pgfqpoint{4.352605in}{2.136698in}}{\pgfqpoint{4.363204in}{2.132308in}}{\pgfqpoint{4.374254in}{2.132308in}}%
\pgfpathclose%
\pgfusepath{stroke,fill}%
\end{pgfscope}%
\begin{pgfscope}%
\pgfpathrectangle{\pgfqpoint{0.570343in}{0.331635in}}{\pgfqpoint{4.960000in}{3.696000in}}%
\pgfusepath{clip}%
\pgfsetbuttcap%
\pgfsetroundjoin%
\definecolor{currentfill}{rgb}{1.000000,0.705882,0.509804}%
\pgfsetfillcolor{currentfill}%
\pgfsetlinewidth{0.481800pt}%
\definecolor{currentstroke}{rgb}{1.000000,1.000000,1.000000}%
\pgfsetstrokecolor{currentstroke}%
\pgfsetdash{}{0pt}%
\pgfpathmoveto{\pgfqpoint{3.704677in}{2.903484in}}%
\pgfpathcurveto{\pgfqpoint{3.715727in}{2.903484in}}{\pgfqpoint{3.726326in}{2.907875in}}{\pgfqpoint{3.734140in}{2.915688in}}%
\pgfpathcurveto{\pgfqpoint{3.741953in}{2.923502in}}{\pgfqpoint{3.746344in}{2.934101in}}{\pgfqpoint{3.746344in}{2.945151in}}%
\pgfpathcurveto{\pgfqpoint{3.746344in}{2.956201in}}{\pgfqpoint{3.741953in}{2.966800in}}{\pgfqpoint{3.734140in}{2.974614in}}%
\pgfpathcurveto{\pgfqpoint{3.726326in}{2.982427in}}{\pgfqpoint{3.715727in}{2.986818in}}{\pgfqpoint{3.704677in}{2.986818in}}%
\pgfpathcurveto{\pgfqpoint{3.693627in}{2.986818in}}{\pgfqpoint{3.683028in}{2.982427in}}{\pgfqpoint{3.675214in}{2.974614in}}%
\pgfpathcurveto{\pgfqpoint{3.667401in}{2.966800in}}{\pgfqpoint{3.663010in}{2.956201in}}{\pgfqpoint{3.663010in}{2.945151in}}%
\pgfpathcurveto{\pgfqpoint{3.663010in}{2.934101in}}{\pgfqpoint{3.667401in}{2.923502in}}{\pgfqpoint{3.675214in}{2.915688in}}%
\pgfpathcurveto{\pgfqpoint{3.683028in}{2.907875in}}{\pgfqpoint{3.693627in}{2.903484in}}{\pgfqpoint{3.704677in}{2.903484in}}%
\pgfpathclose%
\pgfusepath{stroke,fill}%
\end{pgfscope}%
\begin{pgfscope}%
\pgfpathrectangle{\pgfqpoint{0.570343in}{0.331635in}}{\pgfqpoint{4.960000in}{3.696000in}}%
\pgfusepath{clip}%
\pgfsetbuttcap%
\pgfsetroundjoin%
\definecolor{currentfill}{rgb}{1.000000,0.705882,0.509804}%
\pgfsetfillcolor{currentfill}%
\pgfsetlinewidth{0.481800pt}%
\definecolor{currentstroke}{rgb}{1.000000,1.000000,1.000000}%
\pgfsetstrokecolor{currentstroke}%
\pgfsetdash{}{0pt}%
\pgfpathmoveto{\pgfqpoint{0.795798in}{2.919565in}}%
\pgfpathcurveto{\pgfqpoint{0.806848in}{2.919565in}}{\pgfqpoint{0.817447in}{2.923955in}}{\pgfqpoint{0.825261in}{2.931769in}}%
\pgfpathcurveto{\pgfqpoint{0.833074in}{2.939582in}}{\pgfqpoint{0.837465in}{2.950181in}}{\pgfqpoint{0.837465in}{2.961231in}}%
\pgfpathcurveto{\pgfqpoint{0.837465in}{2.972282in}}{\pgfqpoint{0.833074in}{2.982881in}}{\pgfqpoint{0.825261in}{2.990694in}}%
\pgfpathcurveto{\pgfqpoint{0.817447in}{2.998508in}}{\pgfqpoint{0.806848in}{3.002898in}}{\pgfqpoint{0.795798in}{3.002898in}}%
\pgfpathcurveto{\pgfqpoint{0.784748in}{3.002898in}}{\pgfqpoint{0.774149in}{2.998508in}}{\pgfqpoint{0.766335in}{2.990694in}}%
\pgfpathcurveto{\pgfqpoint{0.758521in}{2.982881in}}{\pgfqpoint{0.754131in}{2.972282in}}{\pgfqpoint{0.754131in}{2.961231in}}%
\pgfpathcurveto{\pgfqpoint{0.754131in}{2.950181in}}{\pgfqpoint{0.758521in}{2.939582in}}{\pgfqpoint{0.766335in}{2.931769in}}%
\pgfpathcurveto{\pgfqpoint{0.774149in}{2.923955in}}{\pgfqpoint{0.784748in}{2.919565in}}{\pgfqpoint{0.795798in}{2.919565in}}%
\pgfpathclose%
\pgfusepath{stroke,fill}%
\end{pgfscope}%
\begin{pgfscope}%
\pgfpathrectangle{\pgfqpoint{0.570343in}{0.331635in}}{\pgfqpoint{4.960000in}{3.696000in}}%
\pgfusepath{clip}%
\pgfsetbuttcap%
\pgfsetroundjoin%
\definecolor{currentfill}{rgb}{1.000000,0.705882,0.509804}%
\pgfsetfillcolor{currentfill}%
\pgfsetlinewidth{0.481800pt}%
\definecolor{currentstroke}{rgb}{1.000000,1.000000,1.000000}%
\pgfsetstrokecolor{currentstroke}%
\pgfsetdash{}{0pt}%
\pgfpathmoveto{\pgfqpoint{1.844531in}{3.817968in}}%
\pgfpathcurveto{\pgfqpoint{1.855581in}{3.817968in}}{\pgfqpoint{1.866180in}{3.822359in}}{\pgfqpoint{1.873994in}{3.830172in}}%
\pgfpathcurveto{\pgfqpoint{1.881807in}{3.837986in}}{\pgfqpoint{1.886198in}{3.848585in}}{\pgfqpoint{1.886198in}{3.859635in}}%
\pgfpathcurveto{\pgfqpoint{1.886198in}{3.870685in}}{\pgfqpoint{1.881807in}{3.881284in}}{\pgfqpoint{1.873994in}{3.889098in}}%
\pgfpathcurveto{\pgfqpoint{1.866180in}{3.896911in}}{\pgfqpoint{1.855581in}{3.901302in}}{\pgfqpoint{1.844531in}{3.901302in}}%
\pgfpathcurveto{\pgfqpoint{1.833481in}{3.901302in}}{\pgfqpoint{1.822882in}{3.896911in}}{\pgfqpoint{1.815068in}{3.889098in}}%
\pgfpathcurveto{\pgfqpoint{1.807254in}{3.881284in}}{\pgfqpoint{1.802864in}{3.870685in}}{\pgfqpoint{1.802864in}{3.859635in}}%
\pgfpathcurveto{\pgfqpoint{1.802864in}{3.848585in}}{\pgfqpoint{1.807254in}{3.837986in}}{\pgfqpoint{1.815068in}{3.830172in}}%
\pgfpathcurveto{\pgfqpoint{1.822882in}{3.822359in}}{\pgfqpoint{1.833481in}{3.817968in}}{\pgfqpoint{1.844531in}{3.817968in}}%
\pgfpathclose%
\pgfusepath{stroke,fill}%
\end{pgfscope}%
\begin{pgfscope}%
\pgfpathrectangle{\pgfqpoint{0.570343in}{0.331635in}}{\pgfqpoint{4.960000in}{3.696000in}}%
\pgfusepath{clip}%
\pgfsetbuttcap%
\pgfsetroundjoin%
\definecolor{currentfill}{rgb}{1.000000,0.705882,0.509804}%
\pgfsetfillcolor{currentfill}%
\pgfsetlinewidth{0.481800pt}%
\definecolor{currentstroke}{rgb}{1.000000,1.000000,1.000000}%
\pgfsetstrokecolor{currentstroke}%
\pgfsetdash{}{0pt}%
\pgfpathmoveto{\pgfqpoint{4.720753in}{1.443618in}}%
\pgfpathcurveto{\pgfqpoint{4.731803in}{1.443618in}}{\pgfqpoint{4.742402in}{1.448008in}}{\pgfqpoint{4.750216in}{1.455822in}}%
\pgfpathcurveto{\pgfqpoint{4.758030in}{1.463636in}}{\pgfqpoint{4.762420in}{1.474235in}}{\pgfqpoint{4.762420in}{1.485285in}}%
\pgfpathcurveto{\pgfqpoint{4.762420in}{1.496335in}}{\pgfqpoint{4.758030in}{1.506934in}}{\pgfqpoint{4.750216in}{1.514748in}}%
\pgfpathcurveto{\pgfqpoint{4.742402in}{1.522561in}}{\pgfqpoint{4.731803in}{1.526951in}}{\pgfqpoint{4.720753in}{1.526951in}}%
\pgfpathcurveto{\pgfqpoint{4.709703in}{1.526951in}}{\pgfqpoint{4.699104in}{1.522561in}}{\pgfqpoint{4.691290in}{1.514748in}}%
\pgfpathcurveto{\pgfqpoint{4.683477in}{1.506934in}}{\pgfqpoint{4.679086in}{1.496335in}}{\pgfqpoint{4.679086in}{1.485285in}}%
\pgfpathcurveto{\pgfqpoint{4.679086in}{1.474235in}}{\pgfqpoint{4.683477in}{1.463636in}}{\pgfqpoint{4.691290in}{1.455822in}}%
\pgfpathcurveto{\pgfqpoint{4.699104in}{1.448008in}}{\pgfqpoint{4.709703in}{1.443618in}}{\pgfqpoint{4.720753in}{1.443618in}}%
\pgfpathclose%
\pgfusepath{stroke,fill}%
\end{pgfscope}%
\begin{pgfscope}%
\pgfpathrectangle{\pgfqpoint{0.570343in}{0.331635in}}{\pgfqpoint{4.960000in}{3.696000in}}%
\pgfusepath{clip}%
\pgfsetbuttcap%
\pgfsetroundjoin%
\definecolor{currentfill}{rgb}{1.000000,0.705882,0.509804}%
\pgfsetfillcolor{currentfill}%
\pgfsetlinewidth{0.481800pt}%
\definecolor{currentstroke}{rgb}{1.000000,1.000000,1.000000}%
\pgfsetstrokecolor{currentstroke}%
\pgfsetdash{}{0pt}%
\pgfpathmoveto{\pgfqpoint{2.224277in}{0.457968in}}%
\pgfpathcurveto{\pgfqpoint{2.235327in}{0.457968in}}{\pgfqpoint{2.245926in}{0.462359in}}{\pgfqpoint{2.253740in}{0.470172in}}%
\pgfpathcurveto{\pgfqpoint{2.261554in}{0.477986in}}{\pgfqpoint{2.265944in}{0.488585in}}{\pgfqpoint{2.265944in}{0.499635in}}%
\pgfpathcurveto{\pgfqpoint{2.265944in}{0.510685in}}{\pgfqpoint{2.261554in}{0.521284in}}{\pgfqpoint{2.253740in}{0.529098in}}%
\pgfpathcurveto{\pgfqpoint{2.245926in}{0.536911in}}{\pgfqpoint{2.235327in}{0.541302in}}{\pgfqpoint{2.224277in}{0.541302in}}%
\pgfpathcurveto{\pgfqpoint{2.213227in}{0.541302in}}{\pgfqpoint{2.202628in}{0.536911in}}{\pgfqpoint{2.194815in}{0.529098in}}%
\pgfpathcurveto{\pgfqpoint{2.187001in}{0.521284in}}{\pgfqpoint{2.182611in}{0.510685in}}{\pgfqpoint{2.182611in}{0.499635in}}%
\pgfpathcurveto{\pgfqpoint{2.182611in}{0.488585in}}{\pgfqpoint{2.187001in}{0.477986in}}{\pgfqpoint{2.194815in}{0.470172in}}%
\pgfpathcurveto{\pgfqpoint{2.202628in}{0.462359in}}{\pgfqpoint{2.213227in}{0.457968in}}{\pgfqpoint{2.224277in}{0.457968in}}%
\pgfpathclose%
\pgfusepath{stroke,fill}%
\end{pgfscope}%
\begin{pgfscope}%
\pgfpathrectangle{\pgfqpoint{0.570343in}{0.331635in}}{\pgfqpoint{4.960000in}{3.696000in}}%
\pgfusepath{clip}%
\pgfsetbuttcap%
\pgfsetroundjoin%
\definecolor{currentfill}{rgb}{1.000000,0.705882,0.509804}%
\pgfsetfillcolor{currentfill}%
\pgfsetlinewidth{0.481800pt}%
\definecolor{currentstroke}{rgb}{1.000000,1.000000,1.000000}%
\pgfsetstrokecolor{currentstroke}%
\pgfsetdash{}{0pt}%
\pgfpathmoveto{\pgfqpoint{2.748227in}{3.639322in}}%
\pgfpathcurveto{\pgfqpoint{2.759277in}{3.639322in}}{\pgfqpoint{2.769876in}{3.643712in}}{\pgfqpoint{2.777690in}{3.651526in}}%
\pgfpathcurveto{\pgfqpoint{2.785503in}{3.659339in}}{\pgfqpoint{2.789894in}{3.669938in}}{\pgfqpoint{2.789894in}{3.680989in}}%
\pgfpathcurveto{\pgfqpoint{2.789894in}{3.692039in}}{\pgfqpoint{2.785503in}{3.702638in}}{\pgfqpoint{2.777690in}{3.710451in}}%
\pgfpathcurveto{\pgfqpoint{2.769876in}{3.718265in}}{\pgfqpoint{2.759277in}{3.722655in}}{\pgfqpoint{2.748227in}{3.722655in}}%
\pgfpathcurveto{\pgfqpoint{2.737177in}{3.722655in}}{\pgfqpoint{2.726578in}{3.718265in}}{\pgfqpoint{2.718764in}{3.710451in}}%
\pgfpathcurveto{\pgfqpoint{2.710951in}{3.702638in}}{\pgfqpoint{2.706560in}{3.692039in}}{\pgfqpoint{2.706560in}{3.680989in}}%
\pgfpathcurveto{\pgfqpoint{2.706560in}{3.669938in}}{\pgfqpoint{2.710951in}{3.659339in}}{\pgfqpoint{2.718764in}{3.651526in}}%
\pgfpathcurveto{\pgfqpoint{2.726578in}{3.643712in}}{\pgfqpoint{2.737177in}{3.639322in}}{\pgfqpoint{2.748227in}{3.639322in}}%
\pgfpathclose%
\pgfusepath{stroke,fill}%
\end{pgfscope}%
\begin{pgfscope}%
\pgfpathrectangle{\pgfqpoint{0.570343in}{0.331635in}}{\pgfqpoint{4.960000in}{3.696000in}}%
\pgfusepath{clip}%
\pgfsetbuttcap%
\pgfsetroundjoin%
\definecolor{currentfill}{rgb}{1.000000,0.705882,0.509804}%
\pgfsetfillcolor{currentfill}%
\pgfsetlinewidth{0.481800pt}%
\definecolor{currentstroke}{rgb}{1.000000,1.000000,1.000000}%
\pgfsetstrokecolor{currentstroke}%
\pgfsetdash{}{0pt}%
\pgfpathmoveto{\pgfqpoint{1.166400in}{2.379582in}}%
\pgfpathcurveto{\pgfqpoint{1.177450in}{2.379582in}}{\pgfqpoint{1.188049in}{2.383972in}}{\pgfqpoint{1.195863in}{2.391785in}}%
\pgfpathcurveto{\pgfqpoint{1.203676in}{2.399599in}}{\pgfqpoint{1.208067in}{2.410198in}}{\pgfqpoint{1.208067in}{2.421248in}}%
\pgfpathcurveto{\pgfqpoint{1.208067in}{2.432298in}}{\pgfqpoint{1.203676in}{2.442897in}}{\pgfqpoint{1.195863in}{2.450711in}}%
\pgfpathcurveto{\pgfqpoint{1.188049in}{2.458525in}}{\pgfqpoint{1.177450in}{2.462915in}}{\pgfqpoint{1.166400in}{2.462915in}}%
\pgfpathcurveto{\pgfqpoint{1.155350in}{2.462915in}}{\pgfqpoint{1.144751in}{2.458525in}}{\pgfqpoint{1.136937in}{2.450711in}}%
\pgfpathcurveto{\pgfqpoint{1.129123in}{2.442897in}}{\pgfqpoint{1.124733in}{2.432298in}}{\pgfqpoint{1.124733in}{2.421248in}}%
\pgfpathcurveto{\pgfqpoint{1.124733in}{2.410198in}}{\pgfqpoint{1.129123in}{2.399599in}}{\pgfqpoint{1.136937in}{2.391785in}}%
\pgfpathcurveto{\pgfqpoint{1.144751in}{2.383972in}}{\pgfqpoint{1.155350in}{2.379582in}}{\pgfqpoint{1.166400in}{2.379582in}}%
\pgfpathclose%
\pgfusepath{stroke,fill}%
\end{pgfscope}%
\begin{pgfscope}%
\pgfpathrectangle{\pgfqpoint{0.570343in}{0.331635in}}{\pgfqpoint{4.960000in}{3.696000in}}%
\pgfusepath{clip}%
\pgfsetbuttcap%
\pgfsetroundjoin%
\definecolor{currentfill}{rgb}{1.000000,0.705882,0.509804}%
\pgfsetfillcolor{currentfill}%
\pgfsetlinewidth{0.481800pt}%
\definecolor{currentstroke}{rgb}{1.000000,1.000000,1.000000}%
\pgfsetstrokecolor{currentstroke}%
\pgfsetdash{}{0pt}%
\pgfpathmoveto{\pgfqpoint{1.664925in}{2.487393in}}%
\pgfpathcurveto{\pgfqpoint{1.675975in}{2.487393in}}{\pgfqpoint{1.686574in}{2.491784in}}{\pgfqpoint{1.694388in}{2.499597in}}%
\pgfpathcurveto{\pgfqpoint{1.702201in}{2.507411in}}{\pgfqpoint{1.706592in}{2.518010in}}{\pgfqpoint{1.706592in}{2.529060in}}%
\pgfpathcurveto{\pgfqpoint{1.706592in}{2.540110in}}{\pgfqpoint{1.702201in}{2.550709in}}{\pgfqpoint{1.694388in}{2.558523in}}%
\pgfpathcurveto{\pgfqpoint{1.686574in}{2.566336in}}{\pgfqpoint{1.675975in}{2.570727in}}{\pgfqpoint{1.664925in}{2.570727in}}%
\pgfpathcurveto{\pgfqpoint{1.653875in}{2.570727in}}{\pgfqpoint{1.643276in}{2.566336in}}{\pgfqpoint{1.635462in}{2.558523in}}%
\pgfpathcurveto{\pgfqpoint{1.627649in}{2.550709in}}{\pgfqpoint{1.623258in}{2.540110in}}{\pgfqpoint{1.623258in}{2.529060in}}%
\pgfpathcurveto{\pgfqpoint{1.623258in}{2.518010in}}{\pgfqpoint{1.627649in}{2.507411in}}{\pgfqpoint{1.635462in}{2.499597in}}%
\pgfpathcurveto{\pgfqpoint{1.643276in}{2.491784in}}{\pgfqpoint{1.653875in}{2.487393in}}{\pgfqpoint{1.664925in}{2.487393in}}%
\pgfpathclose%
\pgfusepath{stroke,fill}%
\end{pgfscope}%
\begin{pgfscope}%
\pgfpathrectangle{\pgfqpoint{0.570343in}{0.331635in}}{\pgfqpoint{4.960000in}{3.696000in}}%
\pgfusepath{clip}%
\pgfsetbuttcap%
\pgfsetroundjoin%
\definecolor{currentfill}{rgb}{1.000000,0.705882,0.509804}%
\pgfsetfillcolor{currentfill}%
\pgfsetlinewidth{0.481800pt}%
\definecolor{currentstroke}{rgb}{1.000000,1.000000,1.000000}%
\pgfsetstrokecolor{currentstroke}%
\pgfsetdash{}{0pt}%
\pgfpathmoveto{\pgfqpoint{4.177702in}{2.785347in}}%
\pgfpathcurveto{\pgfqpoint{4.188752in}{2.785347in}}{\pgfqpoint{4.199351in}{2.789737in}}{\pgfqpoint{4.207165in}{2.797551in}}%
\pgfpathcurveto{\pgfqpoint{4.214979in}{2.805365in}}{\pgfqpoint{4.219369in}{2.815964in}}{\pgfqpoint{4.219369in}{2.827014in}}%
\pgfpathcurveto{\pgfqpoint{4.219369in}{2.838064in}}{\pgfqpoint{4.214979in}{2.848663in}}{\pgfqpoint{4.207165in}{2.856476in}}%
\pgfpathcurveto{\pgfqpoint{4.199351in}{2.864290in}}{\pgfqpoint{4.188752in}{2.868680in}}{\pgfqpoint{4.177702in}{2.868680in}}%
\pgfpathcurveto{\pgfqpoint{4.166652in}{2.868680in}}{\pgfqpoint{4.156053in}{2.864290in}}{\pgfqpoint{4.148239in}{2.856476in}}%
\pgfpathcurveto{\pgfqpoint{4.140426in}{2.848663in}}{\pgfqpoint{4.136036in}{2.838064in}}{\pgfqpoint{4.136036in}{2.827014in}}%
\pgfpathcurveto{\pgfqpoint{4.136036in}{2.815964in}}{\pgfqpoint{4.140426in}{2.805365in}}{\pgfqpoint{4.148239in}{2.797551in}}%
\pgfpathcurveto{\pgfqpoint{4.156053in}{2.789737in}}{\pgfqpoint{4.166652in}{2.785347in}}{\pgfqpoint{4.177702in}{2.785347in}}%
\pgfpathclose%
\pgfusepath{stroke,fill}%
\end{pgfscope}%
\begin{pgfscope}%
\pgfpathrectangle{\pgfqpoint{0.570343in}{0.331635in}}{\pgfqpoint{4.960000in}{3.696000in}}%
\pgfusepath{clip}%
\pgfsetbuttcap%
\pgfsetroundjoin%
\definecolor{currentfill}{rgb}{1.000000,0.705882,0.509804}%
\pgfsetfillcolor{currentfill}%
\pgfsetlinewidth{0.481800pt}%
\definecolor{currentstroke}{rgb}{1.000000,1.000000,1.000000}%
\pgfsetstrokecolor{currentstroke}%
\pgfsetdash{}{0pt}%
\pgfpathmoveto{\pgfqpoint{3.492167in}{2.375842in}}%
\pgfpathcurveto{\pgfqpoint{3.503218in}{2.375842in}}{\pgfqpoint{3.513817in}{2.380232in}}{\pgfqpoint{3.521630in}{2.388046in}}%
\pgfpathcurveto{\pgfqpoint{3.529444in}{2.395860in}}{\pgfqpoint{3.533834in}{2.406459in}}{\pgfqpoint{3.533834in}{2.417509in}}%
\pgfpathcurveto{\pgfqpoint{3.533834in}{2.428559in}}{\pgfqpoint{3.529444in}{2.439158in}}{\pgfqpoint{3.521630in}{2.446972in}}%
\pgfpathcurveto{\pgfqpoint{3.513817in}{2.454785in}}{\pgfqpoint{3.503218in}{2.459176in}}{\pgfqpoint{3.492167in}{2.459176in}}%
\pgfpathcurveto{\pgfqpoint{3.481117in}{2.459176in}}{\pgfqpoint{3.470518in}{2.454785in}}{\pgfqpoint{3.462705in}{2.446972in}}%
\pgfpathcurveto{\pgfqpoint{3.454891in}{2.439158in}}{\pgfqpoint{3.450501in}{2.428559in}}{\pgfqpoint{3.450501in}{2.417509in}}%
\pgfpathcurveto{\pgfqpoint{3.450501in}{2.406459in}}{\pgfqpoint{3.454891in}{2.395860in}}{\pgfqpoint{3.462705in}{2.388046in}}%
\pgfpathcurveto{\pgfqpoint{3.470518in}{2.380232in}}{\pgfqpoint{3.481117in}{2.375842in}}{\pgfqpoint{3.492167in}{2.375842in}}%
\pgfpathclose%
\pgfusepath{stroke,fill}%
\end{pgfscope}%
\begin{pgfscope}%
\pgfpathrectangle{\pgfqpoint{0.570343in}{0.331635in}}{\pgfqpoint{4.960000in}{3.696000in}}%
\pgfusepath{clip}%
\pgfsetbuttcap%
\pgfsetroundjoin%
\definecolor{currentfill}{rgb}{1.000000,0.705882,0.509804}%
\pgfsetfillcolor{currentfill}%
\pgfsetlinewidth{0.481800pt}%
\definecolor{currentstroke}{rgb}{1.000000,1.000000,1.000000}%
\pgfsetstrokecolor{currentstroke}%
\pgfsetdash{}{0pt}%
\pgfpathmoveto{\pgfqpoint{2.226219in}{3.480126in}}%
\pgfpathcurveto{\pgfqpoint{2.237269in}{3.480126in}}{\pgfqpoint{2.247868in}{3.484517in}}{\pgfqpoint{2.255682in}{3.492330in}}%
\pgfpathcurveto{\pgfqpoint{2.263495in}{3.500144in}}{\pgfqpoint{2.267885in}{3.510743in}}{\pgfqpoint{2.267885in}{3.521793in}}%
\pgfpathcurveto{\pgfqpoint{2.267885in}{3.532843in}}{\pgfqpoint{2.263495in}{3.543442in}}{\pgfqpoint{2.255682in}{3.551256in}}%
\pgfpathcurveto{\pgfqpoint{2.247868in}{3.559070in}}{\pgfqpoint{2.237269in}{3.563460in}}{\pgfqpoint{2.226219in}{3.563460in}}%
\pgfpathcurveto{\pgfqpoint{2.215169in}{3.563460in}}{\pgfqpoint{2.204570in}{3.559070in}}{\pgfqpoint{2.196756in}{3.551256in}}%
\pgfpathcurveto{\pgfqpoint{2.188942in}{3.543442in}}{\pgfqpoint{2.184552in}{3.532843in}}{\pgfqpoint{2.184552in}{3.521793in}}%
\pgfpathcurveto{\pgfqpoint{2.184552in}{3.510743in}}{\pgfqpoint{2.188942in}{3.500144in}}{\pgfqpoint{2.196756in}{3.492330in}}%
\pgfpathcurveto{\pgfqpoint{2.204570in}{3.484517in}}{\pgfqpoint{2.215169in}{3.480126in}}{\pgfqpoint{2.226219in}{3.480126in}}%
\pgfpathclose%
\pgfusepath{stroke,fill}%
\end{pgfscope}%
\begin{pgfscope}%
\pgfpathrectangle{\pgfqpoint{0.570343in}{0.331635in}}{\pgfqpoint{4.960000in}{3.696000in}}%
\pgfusepath{clip}%
\pgfsetbuttcap%
\pgfsetroundjoin%
\definecolor{currentfill}{rgb}{1.000000,0.705882,0.509804}%
\pgfsetfillcolor{currentfill}%
\pgfsetlinewidth{0.481800pt}%
\definecolor{currentstroke}{rgb}{1.000000,1.000000,1.000000}%
\pgfsetstrokecolor{currentstroke}%
\pgfsetdash{}{0pt}%
\pgfpathmoveto{\pgfqpoint{3.231645in}{2.827291in}}%
\pgfpathcurveto{\pgfqpoint{3.242695in}{2.827291in}}{\pgfqpoint{3.253294in}{2.831681in}}{\pgfqpoint{3.261107in}{2.839494in}}%
\pgfpathcurveto{\pgfqpoint{3.268921in}{2.847308in}}{\pgfqpoint{3.273311in}{2.857907in}}{\pgfqpoint{3.273311in}{2.868957in}}%
\pgfpathcurveto{\pgfqpoint{3.273311in}{2.880007in}}{\pgfqpoint{3.268921in}{2.890606in}}{\pgfqpoint{3.261107in}{2.898420in}}%
\pgfpathcurveto{\pgfqpoint{3.253294in}{2.906234in}}{\pgfqpoint{3.242695in}{2.910624in}}{\pgfqpoint{3.231645in}{2.910624in}}%
\pgfpathcurveto{\pgfqpoint{3.220595in}{2.910624in}}{\pgfqpoint{3.209996in}{2.906234in}}{\pgfqpoint{3.202182in}{2.898420in}}%
\pgfpathcurveto{\pgfqpoint{3.194368in}{2.890606in}}{\pgfqpoint{3.189978in}{2.880007in}}{\pgfqpoint{3.189978in}{2.868957in}}%
\pgfpathcurveto{\pgfqpoint{3.189978in}{2.857907in}}{\pgfqpoint{3.194368in}{2.847308in}}{\pgfqpoint{3.202182in}{2.839494in}}%
\pgfpathcurveto{\pgfqpoint{3.209996in}{2.831681in}}{\pgfqpoint{3.220595in}{2.827291in}}{\pgfqpoint{3.231645in}{2.827291in}}%
\pgfpathclose%
\pgfusepath{stroke,fill}%
\end{pgfscope}%
\begin{pgfscope}%
\pgfpathrectangle{\pgfqpoint{0.570343in}{0.331635in}}{\pgfqpoint{4.960000in}{3.696000in}}%
\pgfusepath{clip}%
\pgfsetbuttcap%
\pgfsetroundjoin%
\definecolor{currentfill}{rgb}{0.631373,0.788235,0.956863}%
\pgfsetfillcolor{currentfill}%
\pgfsetlinewidth{1.003750pt}%
\definecolor{currentstroke}{rgb}{0.631373,0.788235,0.956863}%
\pgfsetstrokecolor{currentstroke}%
\pgfsetdash{}{0pt}%
\pgfsys@defobject{currentmarker}{\pgfqpoint{-0.041667in}{-0.041667in}}{\pgfqpoint{0.041667in}{0.041667in}}{%
\pgfpathmoveto{\pgfqpoint{0.000000in}{-0.041667in}}%
\pgfpathcurveto{\pgfqpoint{0.011050in}{-0.041667in}}{\pgfqpoint{0.021649in}{-0.037276in}}{\pgfqpoint{0.029463in}{-0.029463in}}%
\pgfpathcurveto{\pgfqpoint{0.037276in}{-0.021649in}}{\pgfqpoint{0.041667in}{-0.011050in}}{\pgfqpoint{0.041667in}{0.000000in}}%
\pgfpathcurveto{\pgfqpoint{0.041667in}{0.011050in}}{\pgfqpoint{0.037276in}{0.021649in}}{\pgfqpoint{0.029463in}{0.029463in}}%
\pgfpathcurveto{\pgfqpoint{0.021649in}{0.037276in}}{\pgfqpoint{0.011050in}{0.041667in}}{\pgfqpoint{0.000000in}{0.041667in}}%
\pgfpathcurveto{\pgfqpoint{-0.011050in}{0.041667in}}{\pgfqpoint{-0.021649in}{0.037276in}}{\pgfqpoint{-0.029463in}{0.029463in}}%
\pgfpathcurveto{\pgfqpoint{-0.037276in}{0.021649in}}{\pgfqpoint{-0.041667in}{0.011050in}}{\pgfqpoint{-0.041667in}{0.000000in}}%
\pgfpathcurveto{\pgfqpoint{-0.041667in}{-0.011050in}}{\pgfqpoint{-0.037276in}{-0.021649in}}{\pgfqpoint{-0.029463in}{-0.029463in}}%
\pgfpathcurveto{\pgfqpoint{-0.021649in}{-0.037276in}}{\pgfqpoint{-0.011050in}{-0.041667in}}{\pgfqpoint{0.000000in}{-0.041667in}}%
\pgfpathclose%
\pgfusepath{stroke,fill}%
}%
\end{pgfscope}%
\begin{pgfscope}%
\pgfpathrectangle{\pgfqpoint{0.570343in}{0.331635in}}{\pgfqpoint{4.960000in}{3.696000in}}%
\pgfusepath{clip}%
\pgfsetbuttcap%
\pgfsetroundjoin%
\definecolor{currentfill}{rgb}{1.000000,0.705882,0.509804}%
\pgfsetfillcolor{currentfill}%
\pgfsetlinewidth{1.003750pt}%
\definecolor{currentstroke}{rgb}{1.000000,0.705882,0.509804}%
\pgfsetstrokecolor{currentstroke}%
\pgfsetdash{}{0pt}%
\pgfsys@defobject{currentmarker}{\pgfqpoint{-0.041667in}{-0.041667in}}{\pgfqpoint{0.041667in}{0.041667in}}{%
\pgfpathmoveto{\pgfqpoint{0.000000in}{-0.041667in}}%
\pgfpathcurveto{\pgfqpoint{0.011050in}{-0.041667in}}{\pgfqpoint{0.021649in}{-0.037276in}}{\pgfqpoint{0.029463in}{-0.029463in}}%
\pgfpathcurveto{\pgfqpoint{0.037276in}{-0.021649in}}{\pgfqpoint{0.041667in}{-0.011050in}}{\pgfqpoint{0.041667in}{0.000000in}}%
\pgfpathcurveto{\pgfqpoint{0.041667in}{0.011050in}}{\pgfqpoint{0.037276in}{0.021649in}}{\pgfqpoint{0.029463in}{0.029463in}}%
\pgfpathcurveto{\pgfqpoint{0.021649in}{0.037276in}}{\pgfqpoint{0.011050in}{0.041667in}}{\pgfqpoint{0.000000in}{0.041667in}}%
\pgfpathcurveto{\pgfqpoint{-0.011050in}{0.041667in}}{\pgfqpoint{-0.021649in}{0.037276in}}{\pgfqpoint{-0.029463in}{0.029463in}}%
\pgfpathcurveto{\pgfqpoint{-0.037276in}{0.021649in}}{\pgfqpoint{-0.041667in}{0.011050in}}{\pgfqpoint{-0.041667in}{0.000000in}}%
\pgfpathcurveto{\pgfqpoint{-0.041667in}{-0.011050in}}{\pgfqpoint{-0.037276in}{-0.021649in}}{\pgfqpoint{-0.029463in}{-0.029463in}}%
\pgfpathcurveto{\pgfqpoint{-0.021649in}{-0.037276in}}{\pgfqpoint{-0.011050in}{-0.041667in}}{\pgfqpoint{0.000000in}{-0.041667in}}%
\pgfpathclose%
\pgfusepath{stroke,fill}%
}%
\end{pgfscope}%
\begin{pgfscope}%
\pgfsetbuttcap%
\pgfsetroundjoin%
\definecolor{currentfill}{rgb}{0.000000,0.000000,0.000000}%
\pgfsetfillcolor{currentfill}%
\pgfsetlinewidth{0.803000pt}%
\definecolor{currentstroke}{rgb}{0.000000,0.000000,0.000000}%
\pgfsetstrokecolor{currentstroke}%
\pgfsetdash{}{0pt}%
\pgfsys@defobject{currentmarker}{\pgfqpoint{0.000000in}{-0.048611in}}{\pgfqpoint{0.000000in}{0.000000in}}{%
\pgfpathmoveto{\pgfqpoint{0.000000in}{0.000000in}}%
\pgfpathlineto{\pgfqpoint{0.000000in}{-0.048611in}}%
\pgfusepath{stroke,fill}%
}%
\begin{pgfscope}%
\pgfsys@transformshift{0.731994in}{0.331635in}%
\pgfsys@useobject{currentmarker}{}%
\end{pgfscope}%
\end{pgfscope}%
\begin{pgfscope}%
\definecolor{textcolor}{rgb}{0.000000,0.000000,0.000000}%
\pgfsetstrokecolor{textcolor}%
\pgfsetfillcolor{textcolor}%
\pgftext[x=0.731994in,y=0.234413in,,top]{\color{textcolor}\sffamily\fontsize{10.000000}{12.000000}\selectfont \ensuremath{-}150}%
\end{pgfscope}%
\begin{pgfscope}%
\pgfsetbuttcap%
\pgfsetroundjoin%
\definecolor{currentfill}{rgb}{0.000000,0.000000,0.000000}%
\pgfsetfillcolor{currentfill}%
\pgfsetlinewidth{0.803000pt}%
\definecolor{currentstroke}{rgb}{0.000000,0.000000,0.000000}%
\pgfsetstrokecolor{currentstroke}%
\pgfsetdash{}{0pt}%
\pgfsys@defobject{currentmarker}{\pgfqpoint{0.000000in}{-0.048611in}}{\pgfqpoint{0.000000in}{0.000000in}}{%
\pgfpathmoveto{\pgfqpoint{0.000000in}{0.000000in}}%
\pgfpathlineto{\pgfqpoint{0.000000in}{-0.048611in}}%
\pgfusepath{stroke,fill}%
}%
\begin{pgfscope}%
\pgfsys@transformshift{1.499921in}{0.331635in}%
\pgfsys@useobject{currentmarker}{}%
\end{pgfscope}%
\end{pgfscope}%
\begin{pgfscope}%
\definecolor{textcolor}{rgb}{0.000000,0.000000,0.000000}%
\pgfsetstrokecolor{textcolor}%
\pgfsetfillcolor{textcolor}%
\pgftext[x=1.499921in,y=0.234413in,,top]{\color{textcolor}\sffamily\fontsize{10.000000}{12.000000}\selectfont \ensuremath{-}100}%
\end{pgfscope}%
\begin{pgfscope}%
\pgfsetbuttcap%
\pgfsetroundjoin%
\definecolor{currentfill}{rgb}{0.000000,0.000000,0.000000}%
\pgfsetfillcolor{currentfill}%
\pgfsetlinewidth{0.803000pt}%
\definecolor{currentstroke}{rgb}{0.000000,0.000000,0.000000}%
\pgfsetstrokecolor{currentstroke}%
\pgfsetdash{}{0pt}%
\pgfsys@defobject{currentmarker}{\pgfqpoint{0.000000in}{-0.048611in}}{\pgfqpoint{0.000000in}{0.000000in}}{%
\pgfpathmoveto{\pgfqpoint{0.000000in}{0.000000in}}%
\pgfpathlineto{\pgfqpoint{0.000000in}{-0.048611in}}%
\pgfusepath{stroke,fill}%
}%
\begin{pgfscope}%
\pgfsys@transformshift{2.267848in}{0.331635in}%
\pgfsys@useobject{currentmarker}{}%
\end{pgfscope}%
\end{pgfscope}%
\begin{pgfscope}%
\definecolor{textcolor}{rgb}{0.000000,0.000000,0.000000}%
\pgfsetstrokecolor{textcolor}%
\pgfsetfillcolor{textcolor}%
\pgftext[x=2.267848in,y=0.234413in,,top]{\color{textcolor}\sffamily\fontsize{10.000000}{12.000000}\selectfont \ensuremath{-}50}%
\end{pgfscope}%
\begin{pgfscope}%
\pgfsetbuttcap%
\pgfsetroundjoin%
\definecolor{currentfill}{rgb}{0.000000,0.000000,0.000000}%
\pgfsetfillcolor{currentfill}%
\pgfsetlinewidth{0.803000pt}%
\definecolor{currentstroke}{rgb}{0.000000,0.000000,0.000000}%
\pgfsetstrokecolor{currentstroke}%
\pgfsetdash{}{0pt}%
\pgfsys@defobject{currentmarker}{\pgfqpoint{0.000000in}{-0.048611in}}{\pgfqpoint{0.000000in}{0.000000in}}{%
\pgfpathmoveto{\pgfqpoint{0.000000in}{0.000000in}}%
\pgfpathlineto{\pgfqpoint{0.000000in}{-0.048611in}}%
\pgfusepath{stroke,fill}%
}%
\begin{pgfscope}%
\pgfsys@transformshift{3.035775in}{0.331635in}%
\pgfsys@useobject{currentmarker}{}%
\end{pgfscope}%
\end{pgfscope}%
\begin{pgfscope}%
\definecolor{textcolor}{rgb}{0.000000,0.000000,0.000000}%
\pgfsetstrokecolor{textcolor}%
\pgfsetfillcolor{textcolor}%
\pgftext[x=3.035775in,y=0.234413in,,top]{\color{textcolor}\sffamily\fontsize{10.000000}{12.000000}\selectfont 0}%
\end{pgfscope}%
\begin{pgfscope}%
\pgfsetbuttcap%
\pgfsetroundjoin%
\definecolor{currentfill}{rgb}{0.000000,0.000000,0.000000}%
\pgfsetfillcolor{currentfill}%
\pgfsetlinewidth{0.803000pt}%
\definecolor{currentstroke}{rgb}{0.000000,0.000000,0.000000}%
\pgfsetstrokecolor{currentstroke}%
\pgfsetdash{}{0pt}%
\pgfsys@defobject{currentmarker}{\pgfqpoint{0.000000in}{-0.048611in}}{\pgfqpoint{0.000000in}{0.000000in}}{%
\pgfpathmoveto{\pgfqpoint{0.000000in}{0.000000in}}%
\pgfpathlineto{\pgfqpoint{0.000000in}{-0.048611in}}%
\pgfusepath{stroke,fill}%
}%
\begin{pgfscope}%
\pgfsys@transformshift{3.803702in}{0.331635in}%
\pgfsys@useobject{currentmarker}{}%
\end{pgfscope}%
\end{pgfscope}%
\begin{pgfscope}%
\definecolor{textcolor}{rgb}{0.000000,0.000000,0.000000}%
\pgfsetstrokecolor{textcolor}%
\pgfsetfillcolor{textcolor}%
\pgftext[x=3.803702in,y=0.234413in,,top]{\color{textcolor}\sffamily\fontsize{10.000000}{12.000000}\selectfont 50}%
\end{pgfscope}%
\begin{pgfscope}%
\pgfsetbuttcap%
\pgfsetroundjoin%
\definecolor{currentfill}{rgb}{0.000000,0.000000,0.000000}%
\pgfsetfillcolor{currentfill}%
\pgfsetlinewidth{0.803000pt}%
\definecolor{currentstroke}{rgb}{0.000000,0.000000,0.000000}%
\pgfsetstrokecolor{currentstroke}%
\pgfsetdash{}{0pt}%
\pgfsys@defobject{currentmarker}{\pgfqpoint{0.000000in}{-0.048611in}}{\pgfqpoint{0.000000in}{0.000000in}}{%
\pgfpathmoveto{\pgfqpoint{0.000000in}{0.000000in}}%
\pgfpathlineto{\pgfqpoint{0.000000in}{-0.048611in}}%
\pgfusepath{stroke,fill}%
}%
\begin{pgfscope}%
\pgfsys@transformshift{4.571629in}{0.331635in}%
\pgfsys@useobject{currentmarker}{}%
\end{pgfscope}%
\end{pgfscope}%
\begin{pgfscope}%
\definecolor{textcolor}{rgb}{0.000000,0.000000,0.000000}%
\pgfsetstrokecolor{textcolor}%
\pgfsetfillcolor{textcolor}%
\pgftext[x=4.571629in,y=0.234413in,,top]{\color{textcolor}\sffamily\fontsize{10.000000}{12.000000}\selectfont 100}%
\end{pgfscope}%
\begin{pgfscope}%
\pgfsetbuttcap%
\pgfsetroundjoin%
\definecolor{currentfill}{rgb}{0.000000,0.000000,0.000000}%
\pgfsetfillcolor{currentfill}%
\pgfsetlinewidth{0.803000pt}%
\definecolor{currentstroke}{rgb}{0.000000,0.000000,0.000000}%
\pgfsetstrokecolor{currentstroke}%
\pgfsetdash{}{0pt}%
\pgfsys@defobject{currentmarker}{\pgfqpoint{0.000000in}{-0.048611in}}{\pgfqpoint{0.000000in}{0.000000in}}{%
\pgfpathmoveto{\pgfqpoint{0.000000in}{0.000000in}}%
\pgfpathlineto{\pgfqpoint{0.000000in}{-0.048611in}}%
\pgfusepath{stroke,fill}%
}%
\begin{pgfscope}%
\pgfsys@transformshift{5.339556in}{0.331635in}%
\pgfsys@useobject{currentmarker}{}%
\end{pgfscope}%
\end{pgfscope}%
\begin{pgfscope}%
\definecolor{textcolor}{rgb}{0.000000,0.000000,0.000000}%
\pgfsetstrokecolor{textcolor}%
\pgfsetfillcolor{textcolor}%
\pgftext[x=5.339556in,y=0.234413in,,top]{\color{textcolor}\sffamily\fontsize{10.000000}{12.000000}\selectfont 150}%
\end{pgfscope}%
\begin{pgfscope}%
\pgfsetbuttcap%
\pgfsetroundjoin%
\definecolor{currentfill}{rgb}{0.000000,0.000000,0.000000}%
\pgfsetfillcolor{currentfill}%
\pgfsetlinewidth{0.803000pt}%
\definecolor{currentstroke}{rgb}{0.000000,0.000000,0.000000}%
\pgfsetstrokecolor{currentstroke}%
\pgfsetdash{}{0pt}%
\pgfsys@defobject{currentmarker}{\pgfqpoint{-0.048611in}{0.000000in}}{\pgfqpoint{-0.000000in}{0.000000in}}{%
\pgfpathmoveto{\pgfqpoint{-0.000000in}{0.000000in}}%
\pgfpathlineto{\pgfqpoint{-0.048611in}{0.000000in}}%
\pgfusepath{stroke,fill}%
}%
\begin{pgfscope}%
\pgfsys@transformshift{0.570343in}{0.859263in}%
\pgfsys@useobject{currentmarker}{}%
\end{pgfscope}%
\end{pgfscope}%
\begin{pgfscope}%
\definecolor{textcolor}{rgb}{0.000000,0.000000,0.000000}%
\pgfsetstrokecolor{textcolor}%
\pgfsetfillcolor{textcolor}%
\pgftext[x=0.100000in, y=0.806502in, left, base]{\color{textcolor}\sffamily\fontsize{10.000000}{12.000000}\selectfont \ensuremath{-}100}%
\end{pgfscope}%
\begin{pgfscope}%
\pgfsetbuttcap%
\pgfsetroundjoin%
\definecolor{currentfill}{rgb}{0.000000,0.000000,0.000000}%
\pgfsetfillcolor{currentfill}%
\pgfsetlinewidth{0.803000pt}%
\definecolor{currentstroke}{rgb}{0.000000,0.000000,0.000000}%
\pgfsetstrokecolor{currentstroke}%
\pgfsetdash{}{0pt}%
\pgfsys@defobject{currentmarker}{\pgfqpoint{-0.048611in}{0.000000in}}{\pgfqpoint{-0.000000in}{0.000000in}}{%
\pgfpathmoveto{\pgfqpoint{-0.000000in}{0.000000in}}%
\pgfpathlineto{\pgfqpoint{-0.048611in}{0.000000in}}%
\pgfusepath{stroke,fill}%
}%
\begin{pgfscope}%
\pgfsys@transformshift{0.570343in}{1.469591in}%
\pgfsys@useobject{currentmarker}{}%
\end{pgfscope}%
\end{pgfscope}%
\begin{pgfscope}%
\definecolor{textcolor}{rgb}{0.000000,0.000000,0.000000}%
\pgfsetstrokecolor{textcolor}%
\pgfsetfillcolor{textcolor}%
\pgftext[x=0.188365in, y=1.416830in, left, base]{\color{textcolor}\sffamily\fontsize{10.000000}{12.000000}\selectfont \ensuremath{-}50}%
\end{pgfscope}%
\begin{pgfscope}%
\pgfsetbuttcap%
\pgfsetroundjoin%
\definecolor{currentfill}{rgb}{0.000000,0.000000,0.000000}%
\pgfsetfillcolor{currentfill}%
\pgfsetlinewidth{0.803000pt}%
\definecolor{currentstroke}{rgb}{0.000000,0.000000,0.000000}%
\pgfsetstrokecolor{currentstroke}%
\pgfsetdash{}{0pt}%
\pgfsys@defobject{currentmarker}{\pgfqpoint{-0.048611in}{0.000000in}}{\pgfqpoint{-0.000000in}{0.000000in}}{%
\pgfpathmoveto{\pgfqpoint{-0.000000in}{0.000000in}}%
\pgfpathlineto{\pgfqpoint{-0.048611in}{0.000000in}}%
\pgfusepath{stroke,fill}%
}%
\begin{pgfscope}%
\pgfsys@transformshift{0.570343in}{2.079919in}%
\pgfsys@useobject{currentmarker}{}%
\end{pgfscope}%
\end{pgfscope}%
\begin{pgfscope}%
\definecolor{textcolor}{rgb}{0.000000,0.000000,0.000000}%
\pgfsetstrokecolor{textcolor}%
\pgfsetfillcolor{textcolor}%
\pgftext[x=0.384756in, y=2.027158in, left, base]{\color{textcolor}\sffamily\fontsize{10.000000}{12.000000}\selectfont 0}%
\end{pgfscope}%
\begin{pgfscope}%
\pgfsetbuttcap%
\pgfsetroundjoin%
\definecolor{currentfill}{rgb}{0.000000,0.000000,0.000000}%
\pgfsetfillcolor{currentfill}%
\pgfsetlinewidth{0.803000pt}%
\definecolor{currentstroke}{rgb}{0.000000,0.000000,0.000000}%
\pgfsetstrokecolor{currentstroke}%
\pgfsetdash{}{0pt}%
\pgfsys@defobject{currentmarker}{\pgfqpoint{-0.048611in}{0.000000in}}{\pgfqpoint{-0.000000in}{0.000000in}}{%
\pgfpathmoveto{\pgfqpoint{-0.000000in}{0.000000in}}%
\pgfpathlineto{\pgfqpoint{-0.048611in}{0.000000in}}%
\pgfusepath{stroke,fill}%
}%
\begin{pgfscope}%
\pgfsys@transformshift{0.570343in}{2.690247in}%
\pgfsys@useobject{currentmarker}{}%
\end{pgfscope}%
\end{pgfscope}%
\begin{pgfscope}%
\definecolor{textcolor}{rgb}{0.000000,0.000000,0.000000}%
\pgfsetstrokecolor{textcolor}%
\pgfsetfillcolor{textcolor}%
\pgftext[x=0.296390in, y=2.637486in, left, base]{\color{textcolor}\sffamily\fontsize{10.000000}{12.000000}\selectfont 50}%
\end{pgfscope}%
\begin{pgfscope}%
\pgfsetbuttcap%
\pgfsetroundjoin%
\definecolor{currentfill}{rgb}{0.000000,0.000000,0.000000}%
\pgfsetfillcolor{currentfill}%
\pgfsetlinewidth{0.803000pt}%
\definecolor{currentstroke}{rgb}{0.000000,0.000000,0.000000}%
\pgfsetstrokecolor{currentstroke}%
\pgfsetdash{}{0pt}%
\pgfsys@defobject{currentmarker}{\pgfqpoint{-0.048611in}{0.000000in}}{\pgfqpoint{-0.000000in}{0.000000in}}{%
\pgfpathmoveto{\pgfqpoint{-0.000000in}{0.000000in}}%
\pgfpathlineto{\pgfqpoint{-0.048611in}{0.000000in}}%
\pgfusepath{stroke,fill}%
}%
\begin{pgfscope}%
\pgfsys@transformshift{0.570343in}{3.300575in}%
\pgfsys@useobject{currentmarker}{}%
\end{pgfscope}%
\end{pgfscope}%
\begin{pgfscope}%
\definecolor{textcolor}{rgb}{0.000000,0.000000,0.000000}%
\pgfsetstrokecolor{textcolor}%
\pgfsetfillcolor{textcolor}%
\pgftext[x=0.208025in, y=3.247814in, left, base]{\color{textcolor}\sffamily\fontsize{10.000000}{12.000000}\selectfont 100}%
\end{pgfscope}%
\begin{pgfscope}%
\pgfsetbuttcap%
\pgfsetroundjoin%
\definecolor{currentfill}{rgb}{0.000000,0.000000,0.000000}%
\pgfsetfillcolor{currentfill}%
\pgfsetlinewidth{0.803000pt}%
\definecolor{currentstroke}{rgb}{0.000000,0.000000,0.000000}%
\pgfsetstrokecolor{currentstroke}%
\pgfsetdash{}{0pt}%
\pgfsys@defobject{currentmarker}{\pgfqpoint{-0.048611in}{0.000000in}}{\pgfqpoint{-0.000000in}{0.000000in}}{%
\pgfpathmoveto{\pgfqpoint{-0.000000in}{0.000000in}}%
\pgfpathlineto{\pgfqpoint{-0.048611in}{0.000000in}}%
\pgfusepath{stroke,fill}%
}%
\begin{pgfscope}%
\pgfsys@transformshift{0.570343in}{3.910903in}%
\pgfsys@useobject{currentmarker}{}%
\end{pgfscope}%
\end{pgfscope}%
\begin{pgfscope}%
\definecolor{textcolor}{rgb}{0.000000,0.000000,0.000000}%
\pgfsetstrokecolor{textcolor}%
\pgfsetfillcolor{textcolor}%
\pgftext[x=0.208025in, y=3.858142in, left, base]{\color{textcolor}\sffamily\fontsize{10.000000}{12.000000}\selectfont 150}%
\end{pgfscope}%
\begin{pgfscope}%
\pgfpathrectangle{\pgfqpoint{0.570343in}{0.331635in}}{\pgfqpoint{4.960000in}{3.696000in}}%
\pgfusepath{clip}%
\pgfsetrectcap%
\pgfsetroundjoin%
\pgfsetlinewidth{1.505625pt}%
\definecolor{currentstroke}{rgb}{0.631373,0.788235,0.956863}%
\pgfsetstrokecolor{currentstroke}%
\pgfsetstrokeopacity{0.800000}%
\pgfsetdash{}{0pt}%
\pgfpathmoveto{\pgfqpoint{3.009597in}{1.373128in}}%
\pgfpathlineto{\pgfqpoint{3.108332in}{1.756888in}}%
\pgfusepath{stroke}%
\end{pgfscope}%
\begin{pgfscope}%
\pgfpathrectangle{\pgfqpoint{0.570343in}{0.331635in}}{\pgfqpoint{4.960000in}{3.696000in}}%
\pgfusepath{clip}%
\pgfsetrectcap%
\pgfsetroundjoin%
\pgfsetlinewidth{1.505625pt}%
\definecolor{currentstroke}{rgb}{0.631373,0.788235,0.956863}%
\pgfsetstrokecolor{currentstroke}%
\pgfsetstrokeopacity{0.800000}%
\pgfsetdash{}{0pt}%
\pgfpathmoveto{\pgfqpoint{4.747431in}{2.693739in}}%
\pgfpathlineto{\pgfqpoint{3.108332in}{1.756888in}}%
\pgfusepath{stroke}%
\end{pgfscope}%
\begin{pgfscope}%
\pgfpathrectangle{\pgfqpoint{0.570343in}{0.331635in}}{\pgfqpoint{4.960000in}{3.696000in}}%
\pgfusepath{clip}%
\pgfsetrectcap%
\pgfsetroundjoin%
\pgfsetlinewidth{1.505625pt}%
\definecolor{currentstroke}{rgb}{0.631373,0.788235,0.956863}%
\pgfsetstrokecolor{currentstroke}%
\pgfsetstrokeopacity{0.800000}%
\pgfsetdash{}{0pt}%
\pgfpathmoveto{\pgfqpoint{3.800801in}{1.164242in}}%
\pgfpathlineto{\pgfqpoint{3.108332in}{1.756888in}}%
\pgfusepath{stroke}%
\end{pgfscope}%
\begin{pgfscope}%
\pgfpathrectangle{\pgfqpoint{0.570343in}{0.331635in}}{\pgfqpoint{4.960000in}{3.696000in}}%
\pgfusepath{clip}%
\pgfsetrectcap%
\pgfsetroundjoin%
\pgfsetlinewidth{1.505625pt}%
\definecolor{currentstroke}{rgb}{0.631373,0.788235,0.956863}%
\pgfsetstrokecolor{currentstroke}%
\pgfsetstrokeopacity{0.800000}%
\pgfsetdash{}{0pt}%
\pgfpathmoveto{\pgfqpoint{3.358861in}{1.948847in}}%
\pgfpathlineto{\pgfqpoint{3.108332in}{1.756888in}}%
\pgfusepath{stroke}%
\end{pgfscope}%
\begin{pgfscope}%
\pgfpathrectangle{\pgfqpoint{0.570343in}{0.331635in}}{\pgfqpoint{4.960000in}{3.696000in}}%
\pgfusepath{clip}%
\pgfsetrectcap%
\pgfsetroundjoin%
\pgfsetlinewidth{1.505625pt}%
\definecolor{currentstroke}{rgb}{0.631373,0.788235,0.956863}%
\pgfsetstrokecolor{currentstroke}%
\pgfsetstrokeopacity{0.800000}%
\pgfsetdash{}{0pt}%
\pgfpathmoveto{\pgfqpoint{3.351516in}{1.087951in}}%
\pgfpathlineto{\pgfqpoint{3.108332in}{1.756888in}}%
\pgfusepath{stroke}%
\end{pgfscope}%
\begin{pgfscope}%
\pgfpathrectangle{\pgfqpoint{0.570343in}{0.331635in}}{\pgfqpoint{4.960000in}{3.696000in}}%
\pgfusepath{clip}%
\pgfsetrectcap%
\pgfsetroundjoin%
\pgfsetlinewidth{1.505625pt}%
\definecolor{currentstroke}{rgb}{0.631373,0.788235,0.956863}%
\pgfsetstrokecolor{currentstroke}%
\pgfsetstrokeopacity{0.800000}%
\pgfsetdash{}{0pt}%
\pgfpathmoveto{\pgfqpoint{2.157332in}{3.157638in}}%
\pgfpathlineto{\pgfqpoint{3.108332in}{1.756888in}}%
\pgfusepath{stroke}%
\end{pgfscope}%
\begin{pgfscope}%
\pgfpathrectangle{\pgfqpoint{0.570343in}{0.331635in}}{\pgfqpoint{4.960000in}{3.696000in}}%
\pgfusepath{clip}%
\pgfsetrectcap%
\pgfsetroundjoin%
\pgfsetlinewidth{1.505625pt}%
\definecolor{currentstroke}{rgb}{0.631373,0.788235,0.956863}%
\pgfsetstrokecolor{currentstroke}%
\pgfsetstrokeopacity{0.800000}%
\pgfsetdash{}{0pt}%
\pgfpathmoveto{\pgfqpoint{4.201874in}{1.916318in}}%
\pgfpathlineto{\pgfqpoint{3.108332in}{1.756888in}}%
\pgfusepath{stroke}%
\end{pgfscope}%
\begin{pgfscope}%
\pgfpathrectangle{\pgfqpoint{0.570343in}{0.331635in}}{\pgfqpoint{4.960000in}{3.696000in}}%
\pgfusepath{clip}%
\pgfsetrectcap%
\pgfsetroundjoin%
\pgfsetlinewidth{1.505625pt}%
\definecolor{currentstroke}{rgb}{0.631373,0.788235,0.956863}%
\pgfsetstrokecolor{currentstroke}%
\pgfsetstrokeopacity{0.800000}%
\pgfsetdash{}{0pt}%
\pgfpathmoveto{\pgfqpoint{3.667396in}{0.684679in}}%
\pgfpathlineto{\pgfqpoint{3.108332in}{1.756888in}}%
\pgfusepath{stroke}%
\end{pgfscope}%
\begin{pgfscope}%
\pgfpathrectangle{\pgfqpoint{0.570343in}{0.331635in}}{\pgfqpoint{4.960000in}{3.696000in}}%
\pgfusepath{clip}%
\pgfsetrectcap%
\pgfsetroundjoin%
\pgfsetlinewidth{1.505625pt}%
\definecolor{currentstroke}{rgb}{0.631373,0.788235,0.956863}%
\pgfsetstrokecolor{currentstroke}%
\pgfsetstrokeopacity{0.800000}%
\pgfsetdash{}{0pt}%
\pgfpathmoveto{\pgfqpoint{3.384863in}{1.561141in}}%
\pgfpathlineto{\pgfqpoint{3.108332in}{1.756888in}}%
\pgfusepath{stroke}%
\end{pgfscope}%
\begin{pgfscope}%
\pgfpathrectangle{\pgfqpoint{0.570343in}{0.331635in}}{\pgfqpoint{4.960000in}{3.696000in}}%
\pgfusepath{clip}%
\pgfsetrectcap%
\pgfsetroundjoin%
\pgfsetlinewidth{1.505625pt}%
\definecolor{currentstroke}{rgb}{0.631373,0.788235,0.956863}%
\pgfsetstrokecolor{currentstroke}%
\pgfsetstrokeopacity{0.800000}%
\pgfsetdash{}{0pt}%
\pgfpathmoveto{\pgfqpoint{2.133517in}{2.173284in}}%
\pgfpathlineto{\pgfqpoint{3.108332in}{1.756888in}}%
\pgfusepath{stroke}%
\end{pgfscope}%
\begin{pgfscope}%
\pgfpathrectangle{\pgfqpoint{0.570343in}{0.331635in}}{\pgfqpoint{4.960000in}{3.696000in}}%
\pgfusepath{clip}%
\pgfsetrectcap%
\pgfsetroundjoin%
\pgfsetlinewidth{1.505625pt}%
\definecolor{currentstroke}{rgb}{0.631373,0.788235,0.956863}%
\pgfsetstrokecolor{currentstroke}%
\pgfsetstrokeopacity{0.800000}%
\pgfsetdash{}{0pt}%
\pgfpathmoveto{\pgfqpoint{5.304889in}{1.779734in}}%
\pgfpathlineto{\pgfqpoint{3.108332in}{1.756888in}}%
\pgfusepath{stroke}%
\end{pgfscope}%
\begin{pgfscope}%
\pgfpathrectangle{\pgfqpoint{0.570343in}{0.331635in}}{\pgfqpoint{4.960000in}{3.696000in}}%
\pgfusepath{clip}%
\pgfsetrectcap%
\pgfsetroundjoin%
\pgfsetlinewidth{1.505625pt}%
\definecolor{currentstroke}{rgb}{0.631373,0.788235,0.956863}%
\pgfsetstrokecolor{currentstroke}%
\pgfsetstrokeopacity{0.800000}%
\pgfsetdash{}{0pt}%
\pgfpathmoveto{\pgfqpoint{3.176792in}{2.252003in}}%
\pgfpathlineto{\pgfqpoint{3.108332in}{1.756888in}}%
\pgfusepath{stroke}%
\end{pgfscope}%
\begin{pgfscope}%
\pgfpathrectangle{\pgfqpoint{0.570343in}{0.331635in}}{\pgfqpoint{4.960000in}{3.696000in}}%
\pgfusepath{clip}%
\pgfsetrectcap%
\pgfsetroundjoin%
\pgfsetlinewidth{1.505625pt}%
\definecolor{currentstroke}{rgb}{0.631373,0.788235,0.956863}%
\pgfsetstrokecolor{currentstroke}%
\pgfsetstrokeopacity{0.800000}%
\pgfsetdash{}{0pt}%
\pgfpathmoveto{\pgfqpoint{2.230769in}{1.209950in}}%
\pgfpathlineto{\pgfqpoint{3.108332in}{1.756888in}}%
\pgfusepath{stroke}%
\end{pgfscope}%
\begin{pgfscope}%
\pgfpathrectangle{\pgfqpoint{0.570343in}{0.331635in}}{\pgfqpoint{4.960000in}{3.696000in}}%
\pgfusepath{clip}%
\pgfsetrectcap%
\pgfsetroundjoin%
\pgfsetlinewidth{1.505625pt}%
\definecolor{currentstroke}{rgb}{0.631373,0.788235,0.956863}%
\pgfsetstrokecolor{currentstroke}%
\pgfsetstrokeopacity{0.800000}%
\pgfsetdash{}{0pt}%
\pgfpathmoveto{\pgfqpoint{2.677188in}{2.213882in}}%
\pgfpathlineto{\pgfqpoint{3.108332in}{1.756888in}}%
\pgfusepath{stroke}%
\end{pgfscope}%
\begin{pgfscope}%
\pgfpathrectangle{\pgfqpoint{0.570343in}{0.331635in}}{\pgfqpoint{4.960000in}{3.696000in}}%
\pgfusepath{clip}%
\pgfsetrectcap%
\pgfsetroundjoin%
\pgfsetlinewidth{1.505625pt}%
\definecolor{currentstroke}{rgb}{0.631373,0.788235,0.956863}%
\pgfsetstrokecolor{currentstroke}%
\pgfsetstrokeopacity{0.800000}%
\pgfsetdash{}{0pt}%
\pgfpathmoveto{\pgfqpoint{3.805258in}{1.784790in}}%
\pgfpathlineto{\pgfqpoint{3.108332in}{1.756888in}}%
\pgfusepath{stroke}%
\end{pgfscope}%
\begin{pgfscope}%
\pgfpathrectangle{\pgfqpoint{0.570343in}{0.331635in}}{\pgfqpoint{4.960000in}{3.696000in}}%
\pgfusepath{clip}%
\pgfsetrectcap%
\pgfsetroundjoin%
\pgfsetlinewidth{1.505625pt}%
\definecolor{currentstroke}{rgb}{0.631373,0.788235,0.956863}%
\pgfsetstrokecolor{currentstroke}%
\pgfsetstrokeopacity{0.800000}%
\pgfsetdash{}{0pt}%
\pgfpathmoveto{\pgfqpoint{1.649989in}{2.105370in}}%
\pgfpathlineto{\pgfqpoint{3.108332in}{1.756888in}}%
\pgfusepath{stroke}%
\end{pgfscope}%
\begin{pgfscope}%
\pgfpathrectangle{\pgfqpoint{0.570343in}{0.331635in}}{\pgfqpoint{4.960000in}{3.696000in}}%
\pgfusepath{clip}%
\pgfsetrectcap%
\pgfsetroundjoin%
\pgfsetlinewidth{1.505625pt}%
\definecolor{currentstroke}{rgb}{0.631373,0.788235,0.956863}%
\pgfsetstrokecolor{currentstroke}%
\pgfsetstrokeopacity{0.800000}%
\pgfsetdash{}{0pt}%
\pgfpathmoveto{\pgfqpoint{4.280136in}{1.516808in}}%
\pgfpathlineto{\pgfqpoint{3.108332in}{1.756888in}}%
\pgfusepath{stroke}%
\end{pgfscope}%
\begin{pgfscope}%
\pgfpathrectangle{\pgfqpoint{0.570343in}{0.331635in}}{\pgfqpoint{4.960000in}{3.696000in}}%
\pgfusepath{clip}%
\pgfsetrectcap%
\pgfsetroundjoin%
\pgfsetlinewidth{1.505625pt}%
\definecolor{currentstroke}{rgb}{0.631373,0.788235,0.956863}%
\pgfsetstrokecolor{currentstroke}%
\pgfsetstrokeopacity{0.800000}%
\pgfsetdash{}{0pt}%
\pgfpathmoveto{\pgfqpoint{2.472600in}{1.907162in}}%
\pgfpathlineto{\pgfqpoint{3.108332in}{1.756888in}}%
\pgfusepath{stroke}%
\end{pgfscope}%
\begin{pgfscope}%
\pgfpathrectangle{\pgfqpoint{0.570343in}{0.331635in}}{\pgfqpoint{4.960000in}{3.696000in}}%
\pgfusepath{clip}%
\pgfsetrectcap%
\pgfsetroundjoin%
\pgfsetlinewidth{1.505625pt}%
\definecolor{currentstroke}{rgb}{0.631373,0.788235,0.956863}%
\pgfsetstrokecolor{currentstroke}%
\pgfsetstrokeopacity{0.800000}%
\pgfsetdash{}{0pt}%
\pgfpathmoveto{\pgfqpoint{3.832697in}{2.146167in}}%
\pgfpathlineto{\pgfqpoint{3.108332in}{1.756888in}}%
\pgfusepath{stroke}%
\end{pgfscope}%
\begin{pgfscope}%
\pgfpathrectangle{\pgfqpoint{0.570343in}{0.331635in}}{\pgfqpoint{4.960000in}{3.696000in}}%
\pgfusepath{clip}%
\pgfsetrectcap%
\pgfsetroundjoin%
\pgfsetlinewidth{1.505625pt}%
\definecolor{currentstroke}{rgb}{0.631373,0.788235,0.956863}%
\pgfsetstrokecolor{currentstroke}%
\pgfsetstrokeopacity{0.800000}%
\pgfsetdash{}{0pt}%
\pgfpathmoveto{\pgfqpoint{2.958348in}{2.004807in}}%
\pgfpathlineto{\pgfqpoint{3.108332in}{1.756888in}}%
\pgfusepath{stroke}%
\end{pgfscope}%
\begin{pgfscope}%
\pgfpathrectangle{\pgfqpoint{0.570343in}{0.331635in}}{\pgfqpoint{4.960000in}{3.696000in}}%
\pgfusepath{clip}%
\pgfsetrectcap%
\pgfsetroundjoin%
\pgfsetlinewidth{1.505625pt}%
\definecolor{currentstroke}{rgb}{0.631373,0.788235,0.956863}%
\pgfsetstrokecolor{currentstroke}%
\pgfsetstrokeopacity{0.800000}%
\pgfsetdash{}{0pt}%
\pgfpathmoveto{\pgfqpoint{1.527720in}{1.171796in}}%
\pgfpathlineto{\pgfqpoint{3.108332in}{1.756888in}}%
\pgfusepath{stroke}%
\end{pgfscope}%
\begin{pgfscope}%
\pgfpathrectangle{\pgfqpoint{0.570343in}{0.331635in}}{\pgfqpoint{4.960000in}{3.696000in}}%
\pgfusepath{clip}%
\pgfsetrectcap%
\pgfsetroundjoin%
\pgfsetlinewidth{1.505625pt}%
\definecolor{currentstroke}{rgb}{0.631373,0.788235,0.956863}%
\pgfsetstrokecolor{currentstroke}%
\pgfsetstrokeopacity{0.800000}%
\pgfsetdash{}{0pt}%
\pgfpathmoveto{\pgfqpoint{2.978471in}{0.640164in}}%
\pgfpathlineto{\pgfqpoint{3.108332in}{1.756888in}}%
\pgfusepath{stroke}%
\end{pgfscope}%
\begin{pgfscope}%
\pgfpathrectangle{\pgfqpoint{0.570343in}{0.331635in}}{\pgfqpoint{4.960000in}{3.696000in}}%
\pgfusepath{clip}%
\pgfsetrectcap%
\pgfsetroundjoin%
\pgfsetlinewidth{1.505625pt}%
\definecolor{currentstroke}{rgb}{0.631373,0.788235,0.956863}%
\pgfsetstrokecolor{currentstroke}%
\pgfsetstrokeopacity{0.800000}%
\pgfsetdash{}{0pt}%
\pgfpathmoveto{\pgfqpoint{2.744542in}{1.108528in}}%
\pgfpathlineto{\pgfqpoint{3.108332in}{1.756888in}}%
\pgfusepath{stroke}%
\end{pgfscope}%
\begin{pgfscope}%
\pgfpathrectangle{\pgfqpoint{0.570343in}{0.331635in}}{\pgfqpoint{4.960000in}{3.696000in}}%
\pgfusepath{clip}%
\pgfsetrectcap%
\pgfsetroundjoin%
\pgfsetlinewidth{1.505625pt}%
\definecolor{currentstroke}{rgb}{0.631373,0.788235,0.956863}%
\pgfsetstrokecolor{currentstroke}%
\pgfsetstrokeopacity{0.800000}%
\pgfsetdash{}{0pt}%
\pgfpathmoveto{\pgfqpoint{1.808176in}{1.652126in}}%
\pgfpathlineto{\pgfqpoint{3.108332in}{1.756888in}}%
\pgfusepath{stroke}%
\end{pgfscope}%
\begin{pgfscope}%
\pgfpathrectangle{\pgfqpoint{0.570343in}{0.331635in}}{\pgfqpoint{4.960000in}{3.696000in}}%
\pgfusepath{clip}%
\pgfsetrectcap%
\pgfsetroundjoin%
\pgfsetlinewidth{1.505625pt}%
\definecolor{currentstroke}{rgb}{0.631373,0.788235,0.956863}%
\pgfsetstrokecolor{currentstroke}%
\pgfsetstrokeopacity{0.800000}%
\pgfsetdash{}{0pt}%
\pgfpathmoveto{\pgfqpoint{2.390127in}{1.580399in}}%
\pgfpathlineto{\pgfqpoint{3.108332in}{1.756888in}}%
\pgfusepath{stroke}%
\end{pgfscope}%
\begin{pgfscope}%
\pgfpathrectangle{\pgfqpoint{0.570343in}{0.331635in}}{\pgfqpoint{4.960000in}{3.696000in}}%
\pgfusepath{clip}%
\pgfsetrectcap%
\pgfsetroundjoin%
\pgfsetlinewidth{1.505625pt}%
\definecolor{currentstroke}{rgb}{0.631373,0.788235,0.956863}%
\pgfsetstrokecolor{currentstroke}%
\pgfsetstrokeopacity{0.800000}%
\pgfsetdash{}{0pt}%
\pgfpathmoveto{\pgfqpoint{2.744744in}{3.179676in}}%
\pgfpathlineto{\pgfqpoint{3.108332in}{1.756888in}}%
\pgfusepath{stroke}%
\end{pgfscope}%
\begin{pgfscope}%
\pgfpathrectangle{\pgfqpoint{0.570343in}{0.331635in}}{\pgfqpoint{4.960000in}{3.696000in}}%
\pgfusepath{clip}%
\pgfsetrectcap%
\pgfsetroundjoin%
\pgfsetlinewidth{1.505625pt}%
\definecolor{currentstroke}{rgb}{0.631373,0.788235,0.956863}%
\pgfsetstrokecolor{currentstroke}%
\pgfsetstrokeopacity{0.800000}%
\pgfsetdash{}{0pt}%
\pgfpathmoveto{\pgfqpoint{2.878184in}{1.703905in}}%
\pgfpathlineto{\pgfqpoint{3.108332in}{1.756888in}}%
\pgfusepath{stroke}%
\end{pgfscope}%
\begin{pgfscope}%
\pgfpathrectangle{\pgfqpoint{0.570343in}{0.331635in}}{\pgfqpoint{4.960000in}{3.696000in}}%
\pgfusepath{clip}%
\pgfsetrectcap%
\pgfsetroundjoin%
\pgfsetlinewidth{1.505625pt}%
\definecolor{currentstroke}{rgb}{0.631373,0.788235,0.956863}%
\pgfsetstrokecolor{currentstroke}%
\pgfsetstrokeopacity{0.800000}%
\pgfsetdash{}{0pt}%
\pgfpathmoveto{\pgfqpoint{3.759465in}{1.474627in}}%
\pgfpathlineto{\pgfqpoint{3.108332in}{1.756888in}}%
\pgfusepath{stroke}%
\end{pgfscope}%
\begin{pgfscope}%
\pgfpathrectangle{\pgfqpoint{0.570343in}{0.331635in}}{\pgfqpoint{4.960000in}{3.696000in}}%
\pgfusepath{clip}%
\pgfsetrectcap%
\pgfsetroundjoin%
\pgfsetlinewidth{1.505625pt}%
\definecolor{currentstroke}{rgb}{1.000000,0.705882,0.509804}%
\pgfsetstrokecolor{currentstroke}%
\pgfsetstrokeopacity{0.800000}%
\pgfsetdash{}{0pt}%
\pgfpathmoveto{\pgfqpoint{4.010786in}{2.470580in}}%
\pgfpathlineto{\pgfqpoint{2.979952in}{2.554324in}}%
\pgfusepath{stroke}%
\end{pgfscope}%
\begin{pgfscope}%
\pgfpathrectangle{\pgfqpoint{0.570343in}{0.331635in}}{\pgfqpoint{4.960000in}{3.696000in}}%
\pgfusepath{clip}%
\pgfsetrectcap%
\pgfsetroundjoin%
\pgfsetlinewidth{1.505625pt}%
\definecolor{currentstroke}{rgb}{1.000000,0.705882,0.509804}%
\pgfsetstrokecolor{currentstroke}%
\pgfsetstrokeopacity{0.800000}%
\pgfsetdash{}{0pt}%
\pgfpathmoveto{\pgfqpoint{3.641382in}{3.277442in}}%
\pgfpathlineto{\pgfqpoint{2.979952in}{2.554324in}}%
\pgfusepath{stroke}%
\end{pgfscope}%
\begin{pgfscope}%
\pgfpathrectangle{\pgfqpoint{0.570343in}{0.331635in}}{\pgfqpoint{4.960000in}{3.696000in}}%
\pgfusepath{clip}%
\pgfsetrectcap%
\pgfsetroundjoin%
\pgfsetlinewidth{1.505625pt}%
\definecolor{currentstroke}{rgb}{1.000000,0.705882,0.509804}%
\pgfsetstrokecolor{currentstroke}%
\pgfsetstrokeopacity{0.800000}%
\pgfsetdash{}{0pt}%
\pgfpathmoveto{\pgfqpoint{2.910306in}{2.534690in}}%
\pgfpathlineto{\pgfqpoint{2.979952in}{2.554324in}}%
\pgfusepath{stroke}%
\end{pgfscope}%
\begin{pgfscope}%
\pgfpathrectangle{\pgfqpoint{0.570343in}{0.331635in}}{\pgfqpoint{4.960000in}{3.696000in}}%
\pgfusepath{clip}%
\pgfsetrectcap%
\pgfsetroundjoin%
\pgfsetlinewidth{1.505625pt}%
\definecolor{currentstroke}{rgb}{1.000000,0.705882,0.509804}%
\pgfsetstrokecolor{currentstroke}%
\pgfsetstrokeopacity{0.800000}%
\pgfsetdash{}{0pt}%
\pgfpathmoveto{\pgfqpoint{2.662402in}{2.792752in}}%
\pgfpathlineto{\pgfqpoint{2.979952in}{2.554324in}}%
\pgfusepath{stroke}%
\end{pgfscope}%
\begin{pgfscope}%
\pgfpathrectangle{\pgfqpoint{0.570343in}{0.331635in}}{\pgfqpoint{4.960000in}{3.696000in}}%
\pgfusepath{clip}%
\pgfsetrectcap%
\pgfsetroundjoin%
\pgfsetlinewidth{1.505625pt}%
\definecolor{currentstroke}{rgb}{1.000000,0.705882,0.509804}%
\pgfsetstrokecolor{currentstroke}%
\pgfsetstrokeopacity{0.800000}%
\pgfsetdash{}{0pt}%
\pgfpathmoveto{\pgfqpoint{1.385602in}{3.496321in}}%
\pgfpathlineto{\pgfqpoint{2.979952in}{2.554324in}}%
\pgfusepath{stroke}%
\end{pgfscope}%
\begin{pgfscope}%
\pgfpathrectangle{\pgfqpoint{0.570343in}{0.331635in}}{\pgfqpoint{4.960000in}{3.696000in}}%
\pgfusepath{clip}%
\pgfsetrectcap%
\pgfsetroundjoin%
\pgfsetlinewidth{1.505625pt}%
\definecolor{currentstroke}{rgb}{1.000000,0.705882,0.509804}%
\pgfsetstrokecolor{currentstroke}%
\pgfsetstrokeopacity{0.800000}%
\pgfsetdash{}{0pt}%
\pgfpathmoveto{\pgfqpoint{4.718935in}{1.917042in}}%
\pgfpathlineto{\pgfqpoint{2.979952in}{2.554324in}}%
\pgfusepath{stroke}%
\end{pgfscope}%
\begin{pgfscope}%
\pgfpathrectangle{\pgfqpoint{0.570343in}{0.331635in}}{\pgfqpoint{4.960000in}{3.696000in}}%
\pgfusepath{clip}%
\pgfsetrectcap%
\pgfsetroundjoin%
\pgfsetlinewidth{1.505625pt}%
\definecolor{currentstroke}{rgb}{1.000000,0.705882,0.509804}%
\pgfsetstrokecolor{currentstroke}%
\pgfsetstrokeopacity{0.800000}%
\pgfsetdash{}{0pt}%
\pgfpathmoveto{\pgfqpoint{1.526138in}{2.914129in}}%
\pgfpathlineto{\pgfqpoint{2.979952in}{2.554324in}}%
\pgfusepath{stroke}%
\end{pgfscope}%
\begin{pgfscope}%
\pgfpathrectangle{\pgfqpoint{0.570343in}{0.331635in}}{\pgfqpoint{4.960000in}{3.696000in}}%
\pgfusepath{clip}%
\pgfsetrectcap%
\pgfsetroundjoin%
\pgfsetlinewidth{1.505625pt}%
\definecolor{currentstroke}{rgb}{1.000000,0.705882,0.509804}%
\pgfsetstrokecolor{currentstroke}%
\pgfsetstrokeopacity{0.800000}%
\pgfsetdash{}{0pt}%
\pgfpathmoveto{\pgfqpoint{5.011810in}{2.271126in}}%
\pgfpathlineto{\pgfqpoint{2.979952in}{2.554324in}}%
\pgfusepath{stroke}%
\end{pgfscope}%
\begin{pgfscope}%
\pgfpathrectangle{\pgfqpoint{0.570343in}{0.331635in}}{\pgfqpoint{4.960000in}{3.696000in}}%
\pgfusepath{clip}%
\pgfsetrectcap%
\pgfsetroundjoin%
\pgfsetlinewidth{1.505625pt}%
\definecolor{currentstroke}{rgb}{1.000000,0.705882,0.509804}%
\pgfsetstrokecolor{currentstroke}%
\pgfsetstrokeopacity{0.800000}%
\pgfsetdash{}{0pt}%
\pgfpathmoveto{\pgfqpoint{4.377389in}{1.125422in}}%
\pgfpathlineto{\pgfqpoint{2.979952in}{2.554324in}}%
\pgfusepath{stroke}%
\end{pgfscope}%
\begin{pgfscope}%
\pgfpathrectangle{\pgfqpoint{0.570343in}{0.331635in}}{\pgfqpoint{4.960000in}{3.696000in}}%
\pgfusepath{clip}%
\pgfsetrectcap%
\pgfsetroundjoin%
\pgfsetlinewidth{1.505625pt}%
\definecolor{currentstroke}{rgb}{1.000000,0.705882,0.509804}%
\pgfsetstrokecolor{currentstroke}%
\pgfsetstrokeopacity{0.800000}%
\pgfsetdash{}{0pt}%
\pgfpathmoveto{\pgfqpoint{2.162297in}{2.749786in}}%
\pgfpathlineto{\pgfqpoint{2.979952in}{2.554324in}}%
\pgfusepath{stroke}%
\end{pgfscope}%
\begin{pgfscope}%
\pgfpathrectangle{\pgfqpoint{0.570343in}{0.331635in}}{\pgfqpoint{4.960000in}{3.696000in}}%
\pgfusepath{clip}%
\pgfsetrectcap%
\pgfsetroundjoin%
\pgfsetlinewidth{1.505625pt}%
\definecolor{currentstroke}{rgb}{1.000000,0.705882,0.509804}%
\pgfsetstrokecolor{currentstroke}%
\pgfsetstrokeopacity{0.800000}%
\pgfsetdash{}{0pt}%
\pgfpathmoveto{\pgfqpoint{2.464678in}{2.473549in}}%
\pgfpathlineto{\pgfqpoint{2.979952in}{2.554324in}}%
\pgfusepath{stroke}%
\end{pgfscope}%
\begin{pgfscope}%
\pgfpathrectangle{\pgfqpoint{0.570343in}{0.331635in}}{\pgfqpoint{4.960000in}{3.696000in}}%
\pgfusepath{clip}%
\pgfsetrectcap%
\pgfsetroundjoin%
\pgfsetlinewidth{1.505625pt}%
\definecolor{currentstroke}{rgb}{1.000000,0.705882,0.509804}%
\pgfsetstrokecolor{currentstroke}%
\pgfsetstrokeopacity{0.800000}%
\pgfsetdash{}{0pt}%
\pgfpathmoveto{\pgfqpoint{0.856192in}{1.571252in}}%
\pgfpathlineto{\pgfqpoint{2.979952in}{2.554324in}}%
\pgfusepath{stroke}%
\end{pgfscope}%
\begin{pgfscope}%
\pgfpathrectangle{\pgfqpoint{0.570343in}{0.331635in}}{\pgfqpoint{4.960000in}{3.696000in}}%
\pgfusepath{clip}%
\pgfsetrectcap%
\pgfsetroundjoin%
\pgfsetlinewidth{1.505625pt}%
\definecolor{currentstroke}{rgb}{1.000000,0.705882,0.509804}%
\pgfsetstrokecolor{currentstroke}%
\pgfsetstrokeopacity{0.800000}%
\pgfsetdash{}{0pt}%
\pgfpathmoveto{\pgfqpoint{4.626113in}{3.320981in}}%
\pgfpathlineto{\pgfqpoint{2.979952in}{2.554324in}}%
\pgfusepath{stroke}%
\end{pgfscope}%
\begin{pgfscope}%
\pgfpathrectangle{\pgfqpoint{0.570343in}{0.331635in}}{\pgfqpoint{4.960000in}{3.696000in}}%
\pgfusepath{clip}%
\pgfsetrectcap%
\pgfsetroundjoin%
\pgfsetlinewidth{1.505625pt}%
\definecolor{currentstroke}{rgb}{1.000000,0.705882,0.509804}%
\pgfsetstrokecolor{currentstroke}%
\pgfsetstrokeopacity{0.800000}%
\pgfsetdash{}{0pt}%
\pgfpathmoveto{\pgfqpoint{1.796127in}{3.353620in}}%
\pgfpathlineto{\pgfqpoint{2.979952in}{2.554324in}}%
\pgfusepath{stroke}%
\end{pgfscope}%
\begin{pgfscope}%
\pgfpathrectangle{\pgfqpoint{0.570343in}{0.331635in}}{\pgfqpoint{4.960000in}{3.696000in}}%
\pgfusepath{clip}%
\pgfsetrectcap%
\pgfsetroundjoin%
\pgfsetlinewidth{1.505625pt}%
\definecolor{currentstroke}{rgb}{1.000000,0.705882,0.509804}%
\pgfsetstrokecolor{currentstroke}%
\pgfsetstrokeopacity{0.800000}%
\pgfsetdash{}{0pt}%
\pgfpathmoveto{\pgfqpoint{4.916919in}{1.060897in}}%
\pgfpathlineto{\pgfqpoint{2.979952in}{2.554324in}}%
\pgfusepath{stroke}%
\end{pgfscope}%
\begin{pgfscope}%
\pgfpathrectangle{\pgfqpoint{0.570343in}{0.331635in}}{\pgfqpoint{4.960000in}{3.696000in}}%
\pgfusepath{clip}%
\pgfsetrectcap%
\pgfsetroundjoin%
\pgfsetlinewidth{1.505625pt}%
\definecolor{currentstroke}{rgb}{1.000000,0.705882,0.509804}%
\pgfsetstrokecolor{currentstroke}%
\pgfsetstrokeopacity{0.800000}%
\pgfsetdash{}{0pt}%
\pgfpathmoveto{\pgfqpoint{4.374254in}{2.173974in}}%
\pgfpathlineto{\pgfqpoint{2.979952in}{2.554324in}}%
\pgfusepath{stroke}%
\end{pgfscope}%
\begin{pgfscope}%
\pgfpathrectangle{\pgfqpoint{0.570343in}{0.331635in}}{\pgfqpoint{4.960000in}{3.696000in}}%
\pgfusepath{clip}%
\pgfsetrectcap%
\pgfsetroundjoin%
\pgfsetlinewidth{1.505625pt}%
\definecolor{currentstroke}{rgb}{1.000000,0.705882,0.509804}%
\pgfsetstrokecolor{currentstroke}%
\pgfsetstrokeopacity{0.800000}%
\pgfsetdash{}{0pt}%
\pgfpathmoveto{\pgfqpoint{3.704677in}{2.945151in}}%
\pgfpathlineto{\pgfqpoint{2.979952in}{2.554324in}}%
\pgfusepath{stroke}%
\end{pgfscope}%
\begin{pgfscope}%
\pgfpathrectangle{\pgfqpoint{0.570343in}{0.331635in}}{\pgfqpoint{4.960000in}{3.696000in}}%
\pgfusepath{clip}%
\pgfsetrectcap%
\pgfsetroundjoin%
\pgfsetlinewidth{1.505625pt}%
\definecolor{currentstroke}{rgb}{1.000000,0.705882,0.509804}%
\pgfsetstrokecolor{currentstroke}%
\pgfsetstrokeopacity{0.800000}%
\pgfsetdash{}{0pt}%
\pgfpathmoveto{\pgfqpoint{0.795798in}{2.961231in}}%
\pgfpathlineto{\pgfqpoint{2.979952in}{2.554324in}}%
\pgfusepath{stroke}%
\end{pgfscope}%
\begin{pgfscope}%
\pgfpathrectangle{\pgfqpoint{0.570343in}{0.331635in}}{\pgfqpoint{4.960000in}{3.696000in}}%
\pgfusepath{clip}%
\pgfsetrectcap%
\pgfsetroundjoin%
\pgfsetlinewidth{1.505625pt}%
\definecolor{currentstroke}{rgb}{1.000000,0.705882,0.509804}%
\pgfsetstrokecolor{currentstroke}%
\pgfsetstrokeopacity{0.800000}%
\pgfsetdash{}{0pt}%
\pgfpathmoveto{\pgfqpoint{1.844531in}{3.859635in}}%
\pgfpathlineto{\pgfqpoint{2.979952in}{2.554324in}}%
\pgfusepath{stroke}%
\end{pgfscope}%
\begin{pgfscope}%
\pgfpathrectangle{\pgfqpoint{0.570343in}{0.331635in}}{\pgfqpoint{4.960000in}{3.696000in}}%
\pgfusepath{clip}%
\pgfsetrectcap%
\pgfsetroundjoin%
\pgfsetlinewidth{1.505625pt}%
\definecolor{currentstroke}{rgb}{1.000000,0.705882,0.509804}%
\pgfsetstrokecolor{currentstroke}%
\pgfsetstrokeopacity{0.800000}%
\pgfsetdash{}{0pt}%
\pgfpathmoveto{\pgfqpoint{4.720753in}{1.485285in}}%
\pgfpathlineto{\pgfqpoint{2.979952in}{2.554324in}}%
\pgfusepath{stroke}%
\end{pgfscope}%
\begin{pgfscope}%
\pgfpathrectangle{\pgfqpoint{0.570343in}{0.331635in}}{\pgfqpoint{4.960000in}{3.696000in}}%
\pgfusepath{clip}%
\pgfsetrectcap%
\pgfsetroundjoin%
\pgfsetlinewidth{1.505625pt}%
\definecolor{currentstroke}{rgb}{1.000000,0.705882,0.509804}%
\pgfsetstrokecolor{currentstroke}%
\pgfsetstrokeopacity{0.800000}%
\pgfsetdash{}{0pt}%
\pgfpathmoveto{\pgfqpoint{2.224277in}{0.499635in}}%
\pgfpathlineto{\pgfqpoint{2.979952in}{2.554324in}}%
\pgfusepath{stroke}%
\end{pgfscope}%
\begin{pgfscope}%
\pgfpathrectangle{\pgfqpoint{0.570343in}{0.331635in}}{\pgfqpoint{4.960000in}{3.696000in}}%
\pgfusepath{clip}%
\pgfsetrectcap%
\pgfsetroundjoin%
\pgfsetlinewidth{1.505625pt}%
\definecolor{currentstroke}{rgb}{1.000000,0.705882,0.509804}%
\pgfsetstrokecolor{currentstroke}%
\pgfsetstrokeopacity{0.800000}%
\pgfsetdash{}{0pt}%
\pgfpathmoveto{\pgfqpoint{2.748227in}{3.680989in}}%
\pgfpathlineto{\pgfqpoint{2.979952in}{2.554324in}}%
\pgfusepath{stroke}%
\end{pgfscope}%
\begin{pgfscope}%
\pgfpathrectangle{\pgfqpoint{0.570343in}{0.331635in}}{\pgfqpoint{4.960000in}{3.696000in}}%
\pgfusepath{clip}%
\pgfsetrectcap%
\pgfsetroundjoin%
\pgfsetlinewidth{1.505625pt}%
\definecolor{currentstroke}{rgb}{1.000000,0.705882,0.509804}%
\pgfsetstrokecolor{currentstroke}%
\pgfsetstrokeopacity{0.800000}%
\pgfsetdash{}{0pt}%
\pgfpathmoveto{\pgfqpoint{1.166400in}{2.421248in}}%
\pgfpathlineto{\pgfqpoint{2.979952in}{2.554324in}}%
\pgfusepath{stroke}%
\end{pgfscope}%
\begin{pgfscope}%
\pgfpathrectangle{\pgfqpoint{0.570343in}{0.331635in}}{\pgfqpoint{4.960000in}{3.696000in}}%
\pgfusepath{clip}%
\pgfsetrectcap%
\pgfsetroundjoin%
\pgfsetlinewidth{1.505625pt}%
\definecolor{currentstroke}{rgb}{1.000000,0.705882,0.509804}%
\pgfsetstrokecolor{currentstroke}%
\pgfsetstrokeopacity{0.800000}%
\pgfsetdash{}{0pt}%
\pgfpathmoveto{\pgfqpoint{1.664925in}{2.529060in}}%
\pgfpathlineto{\pgfqpoint{2.979952in}{2.554324in}}%
\pgfusepath{stroke}%
\end{pgfscope}%
\begin{pgfscope}%
\pgfpathrectangle{\pgfqpoint{0.570343in}{0.331635in}}{\pgfqpoint{4.960000in}{3.696000in}}%
\pgfusepath{clip}%
\pgfsetrectcap%
\pgfsetroundjoin%
\pgfsetlinewidth{1.505625pt}%
\definecolor{currentstroke}{rgb}{1.000000,0.705882,0.509804}%
\pgfsetstrokecolor{currentstroke}%
\pgfsetstrokeopacity{0.800000}%
\pgfsetdash{}{0pt}%
\pgfpathmoveto{\pgfqpoint{4.177702in}{2.827014in}}%
\pgfpathlineto{\pgfqpoint{2.979952in}{2.554324in}}%
\pgfusepath{stroke}%
\end{pgfscope}%
\begin{pgfscope}%
\pgfpathrectangle{\pgfqpoint{0.570343in}{0.331635in}}{\pgfqpoint{4.960000in}{3.696000in}}%
\pgfusepath{clip}%
\pgfsetrectcap%
\pgfsetroundjoin%
\pgfsetlinewidth{1.505625pt}%
\definecolor{currentstroke}{rgb}{1.000000,0.705882,0.509804}%
\pgfsetstrokecolor{currentstroke}%
\pgfsetstrokeopacity{0.800000}%
\pgfsetdash{}{0pt}%
\pgfpathmoveto{\pgfqpoint{3.492167in}{2.417509in}}%
\pgfpathlineto{\pgfqpoint{2.979952in}{2.554324in}}%
\pgfusepath{stroke}%
\end{pgfscope}%
\begin{pgfscope}%
\pgfpathrectangle{\pgfqpoint{0.570343in}{0.331635in}}{\pgfqpoint{4.960000in}{3.696000in}}%
\pgfusepath{clip}%
\pgfsetrectcap%
\pgfsetroundjoin%
\pgfsetlinewidth{1.505625pt}%
\definecolor{currentstroke}{rgb}{1.000000,0.705882,0.509804}%
\pgfsetstrokecolor{currentstroke}%
\pgfsetstrokeopacity{0.800000}%
\pgfsetdash{}{0pt}%
\pgfpathmoveto{\pgfqpoint{2.226219in}{3.521793in}}%
\pgfpathlineto{\pgfqpoint{2.979952in}{2.554324in}}%
\pgfusepath{stroke}%
\end{pgfscope}%
\begin{pgfscope}%
\pgfpathrectangle{\pgfqpoint{0.570343in}{0.331635in}}{\pgfqpoint{4.960000in}{3.696000in}}%
\pgfusepath{clip}%
\pgfsetrectcap%
\pgfsetroundjoin%
\pgfsetlinewidth{1.505625pt}%
\definecolor{currentstroke}{rgb}{1.000000,0.705882,0.509804}%
\pgfsetstrokecolor{currentstroke}%
\pgfsetstrokeopacity{0.800000}%
\pgfsetdash{}{0pt}%
\pgfpathmoveto{\pgfqpoint{3.231645in}{2.868957in}}%
\pgfpathlineto{\pgfqpoint{2.979952in}{2.554324in}}%
\pgfusepath{stroke}%
\end{pgfscope}%
\begin{pgfscope}%
\pgfsetrectcap%
\pgfsetmiterjoin%
\pgfsetlinewidth{0.803000pt}%
\definecolor{currentstroke}{rgb}{0.000000,0.000000,0.000000}%
\pgfsetstrokecolor{currentstroke}%
\pgfsetdash{}{0pt}%
\pgfpathmoveto{\pgfqpoint{0.570343in}{0.331635in}}%
\pgfpathlineto{\pgfqpoint{0.570343in}{4.027635in}}%
\pgfusepath{stroke}%
\end{pgfscope}%
\begin{pgfscope}%
\pgfsetrectcap%
\pgfsetmiterjoin%
\pgfsetlinewidth{0.803000pt}%
\definecolor{currentstroke}{rgb}{0.000000,0.000000,0.000000}%
\pgfsetstrokecolor{currentstroke}%
\pgfsetdash{}{0pt}%
\pgfpathmoveto{\pgfqpoint{5.530343in}{0.331635in}}%
\pgfpathlineto{\pgfqpoint{5.530343in}{4.027635in}}%
\pgfusepath{stroke}%
\end{pgfscope}%
\begin{pgfscope}%
\pgfsetrectcap%
\pgfsetmiterjoin%
\pgfsetlinewidth{0.803000pt}%
\definecolor{currentstroke}{rgb}{0.000000,0.000000,0.000000}%
\pgfsetstrokecolor{currentstroke}%
\pgfsetdash{}{0pt}%
\pgfpathmoveto{\pgfqpoint{0.570343in}{0.331635in}}%
\pgfpathlineto{\pgfqpoint{5.530343in}{0.331635in}}%
\pgfusepath{stroke}%
\end{pgfscope}%
\begin{pgfscope}%
\pgfsetrectcap%
\pgfsetmiterjoin%
\pgfsetlinewidth{0.803000pt}%
\definecolor{currentstroke}{rgb}{0.000000,0.000000,0.000000}%
\pgfsetstrokecolor{currentstroke}%
\pgfsetdash{}{0pt}%
\pgfpathmoveto{\pgfqpoint{0.570343in}{4.027635in}}%
\pgfpathlineto{\pgfqpoint{5.530343in}{4.027635in}}%
\pgfusepath{stroke}%
\end{pgfscope}%
\begin{pgfscope}%
\definecolor{textcolor}{rgb}{0.000000,0.000000,0.000000}%
\pgfsetstrokecolor{textcolor}%
\pgfsetfillcolor{textcolor}%
\pgftext[x=3.050343in,y=4.110968in,,base]{\color{textcolor}\sffamily\fontsize{12.000000}{14.400000}\selectfont t-SNE for pix3d and interiornet}%
\end{pgfscope}%
\begin{pgfscope}%
\pgfsetbuttcap%
\pgfsetmiterjoin%
\definecolor{currentfill}{rgb}{1.000000,1.000000,1.000000}%
\pgfsetfillcolor{currentfill}%
\pgfsetfillopacity{0.800000}%
\pgfsetlinewidth{1.003750pt}%
\definecolor{currentstroke}{rgb}{0.800000,0.800000,0.800000}%
\pgfsetstrokecolor{currentstroke}%
\pgfsetstrokeopacity{0.800000}%
\pgfsetdash{}{0pt}%
\pgfpathmoveto{\pgfqpoint{4.258900in}{3.508809in}}%
\pgfpathlineto{\pgfqpoint{5.433121in}{3.508809in}}%
\pgfpathquadraticcurveto{\pgfqpoint{5.460899in}{3.508809in}}{\pgfqpoint{5.460899in}{3.536587in}}%
\pgfpathlineto{\pgfqpoint{5.460899in}{3.930413in}}%
\pgfpathquadraticcurveto{\pgfqpoint{5.460899in}{3.958191in}}{\pgfqpoint{5.433121in}{3.958191in}}%
\pgfpathlineto{\pgfqpoint{4.258900in}{3.958191in}}%
\pgfpathquadraticcurveto{\pgfqpoint{4.231122in}{3.958191in}}{\pgfqpoint{4.231122in}{3.930413in}}%
\pgfpathlineto{\pgfqpoint{4.231122in}{3.536587in}}%
\pgfpathquadraticcurveto{\pgfqpoint{4.231122in}{3.508809in}}{\pgfqpoint{4.258900in}{3.508809in}}%
\pgfpathclose%
\pgfusepath{stroke,fill}%
\end{pgfscope}%
\begin{pgfscope}%
\pgfsetbuttcap%
\pgfsetroundjoin%
\definecolor{currentfill}{rgb}{0.631373,0.788235,0.956863}%
\pgfsetfillcolor{currentfill}%
\pgfsetlinewidth{1.003750pt}%
\definecolor{currentstroke}{rgb}{0.631373,0.788235,0.956863}%
\pgfsetstrokecolor{currentstroke}%
\pgfsetdash{}{0pt}%
\pgfsys@defobject{currentmarker}{\pgfqpoint{-0.041667in}{-0.041667in}}{\pgfqpoint{0.041667in}{0.041667in}}{%
\pgfpathmoveto{\pgfqpoint{0.000000in}{-0.041667in}}%
\pgfpathcurveto{\pgfqpoint{0.011050in}{-0.041667in}}{\pgfqpoint{0.021649in}{-0.037276in}}{\pgfqpoint{0.029463in}{-0.029463in}}%
\pgfpathcurveto{\pgfqpoint{0.037276in}{-0.021649in}}{\pgfqpoint{0.041667in}{-0.011050in}}{\pgfqpoint{0.041667in}{0.000000in}}%
\pgfpathcurveto{\pgfqpoint{0.041667in}{0.011050in}}{\pgfqpoint{0.037276in}{0.021649in}}{\pgfqpoint{0.029463in}{0.029463in}}%
\pgfpathcurveto{\pgfqpoint{0.021649in}{0.037276in}}{\pgfqpoint{0.011050in}{0.041667in}}{\pgfqpoint{0.000000in}{0.041667in}}%
\pgfpathcurveto{\pgfqpoint{-0.011050in}{0.041667in}}{\pgfqpoint{-0.021649in}{0.037276in}}{\pgfqpoint{-0.029463in}{0.029463in}}%
\pgfpathcurveto{\pgfqpoint{-0.037276in}{0.021649in}}{\pgfqpoint{-0.041667in}{0.011050in}}{\pgfqpoint{-0.041667in}{0.000000in}}%
\pgfpathcurveto{\pgfqpoint{-0.041667in}{-0.011050in}}{\pgfqpoint{-0.037276in}{-0.021649in}}{\pgfqpoint{-0.029463in}{-0.029463in}}%
\pgfpathcurveto{\pgfqpoint{-0.021649in}{-0.037276in}}{\pgfqpoint{-0.011050in}{-0.041667in}}{\pgfqpoint{0.000000in}{-0.041667in}}%
\pgfpathclose%
\pgfusepath{stroke,fill}%
}%
\begin{pgfscope}%
\pgfsys@transformshift{4.425566in}{3.833570in}%
\pgfsys@useobject{currentmarker}{}%
\end{pgfscope}%
\end{pgfscope}%
\begin{pgfscope}%
\definecolor{textcolor}{rgb}{0.000000,0.000000,0.000000}%
\pgfsetstrokecolor{textcolor}%
\pgfsetfillcolor{textcolor}%
\pgftext[x=4.675566in,y=3.797112in,left,base]{\color{textcolor}\sffamily\fontsize{10.000000}{12.000000}\selectfont interiornet}%
\end{pgfscope}%
\begin{pgfscope}%
\pgfsetbuttcap%
\pgfsetroundjoin%
\definecolor{currentfill}{rgb}{1.000000,0.705882,0.509804}%
\pgfsetfillcolor{currentfill}%
\pgfsetlinewidth{1.003750pt}%
\definecolor{currentstroke}{rgb}{1.000000,0.705882,0.509804}%
\pgfsetstrokecolor{currentstroke}%
\pgfsetdash{}{0pt}%
\pgfsys@defobject{currentmarker}{\pgfqpoint{-0.041667in}{-0.041667in}}{\pgfqpoint{0.041667in}{0.041667in}}{%
\pgfpathmoveto{\pgfqpoint{0.000000in}{-0.041667in}}%
\pgfpathcurveto{\pgfqpoint{0.011050in}{-0.041667in}}{\pgfqpoint{0.021649in}{-0.037276in}}{\pgfqpoint{0.029463in}{-0.029463in}}%
\pgfpathcurveto{\pgfqpoint{0.037276in}{-0.021649in}}{\pgfqpoint{0.041667in}{-0.011050in}}{\pgfqpoint{0.041667in}{0.000000in}}%
\pgfpathcurveto{\pgfqpoint{0.041667in}{0.011050in}}{\pgfqpoint{0.037276in}{0.021649in}}{\pgfqpoint{0.029463in}{0.029463in}}%
\pgfpathcurveto{\pgfqpoint{0.021649in}{0.037276in}}{\pgfqpoint{0.011050in}{0.041667in}}{\pgfqpoint{0.000000in}{0.041667in}}%
\pgfpathcurveto{\pgfqpoint{-0.011050in}{0.041667in}}{\pgfqpoint{-0.021649in}{0.037276in}}{\pgfqpoint{-0.029463in}{0.029463in}}%
\pgfpathcurveto{\pgfqpoint{-0.037276in}{0.021649in}}{\pgfqpoint{-0.041667in}{0.011050in}}{\pgfqpoint{-0.041667in}{0.000000in}}%
\pgfpathcurveto{\pgfqpoint{-0.041667in}{-0.011050in}}{\pgfqpoint{-0.037276in}{-0.021649in}}{\pgfqpoint{-0.029463in}{-0.029463in}}%
\pgfpathcurveto{\pgfqpoint{-0.021649in}{-0.037276in}}{\pgfqpoint{-0.011050in}{-0.041667in}}{\pgfqpoint{0.000000in}{-0.041667in}}%
\pgfpathclose%
\pgfusepath{stroke,fill}%
}%
\begin{pgfscope}%
\pgfsys@transformshift{4.425566in}{3.629713in}%
\pgfsys@useobject{currentmarker}{}%
\end{pgfscope}%
\end{pgfscope}%
\begin{pgfscope}%
\definecolor{textcolor}{rgb}{0.000000,0.000000,0.000000}%
\pgfsetstrokecolor{textcolor}%
\pgfsetfillcolor{textcolor}%
\pgftext[x=4.675566in,y=3.593255in,left,base]{\color{textcolor}\sffamily\fontsize{10.000000}{12.000000}\selectfont pix3d}%
\end{pgfscope}%
\end{pgfpicture}%
\makeatother%
\endgroup%
}
    \resizebox{0.49\linewidth}{5cm}{%% Creator: Matplotlib, PGF backend
%%
%% To include the figure in your LaTeX document, write
%%   \input{<filename>.pgf}
%%
%% Make sure the required packages are loaded in your preamble
%%   \usepackage{pgf}
%%
%% Figures using additional raster images can only be included by \input if
%% they are in the same directory as the main LaTeX file. For loading figures
%% from other directories you can use the `import` package
%%   \usepackage{import}
%%
%% and then include the figures with
%%   \import{<path to file>}{<filename>.pgf}
%%
%% Matplotlib used the following preamble
%%   \usepackage{fontspec}
%%   \setmainfont{DejaVuSerif.ttf}[Path=\detokenize{/Users/apple/opt/anaconda3/envs/kaolin/lib/python3.7/site-packages/matplotlib/mpl-data/fonts/ttf/}]
%%   \setsansfont{DejaVuSans.ttf}[Path=\detokenize{/Users/apple/opt/anaconda3/envs/kaolin/lib/python3.7/site-packages/matplotlib/mpl-data/fonts/ttf/}]
%%   \setmonofont{DejaVuSansMono.ttf}[Path=\detokenize{/Users/apple/opt/anaconda3/envs/kaolin/lib/python3.7/site-packages/matplotlib/mpl-data/fonts/ttf/}]
%%
\begingroup%
\makeatletter%
\begin{pgfpicture}%
\pgfpathrectangle{\pgfpointorigin}{\pgfqpoint{11.249220in}{8.341596in}}%
\pgfusepath{use as bounding box, clip}%
\begin{pgfscope}%
\pgfsetbuttcap%
\pgfsetmiterjoin%
\definecolor{currentfill}{rgb}{1.000000,1.000000,1.000000}%
\pgfsetfillcolor{currentfill}%
\pgfsetlinewidth{0.000000pt}%
\definecolor{currentstroke}{rgb}{1.000000,1.000000,1.000000}%
\pgfsetstrokecolor{currentstroke}%
\pgfsetdash{}{0pt}%
\pgfpathmoveto{\pgfqpoint{0.000000in}{0.000000in}}%
\pgfpathlineto{\pgfqpoint{11.249220in}{0.000000in}}%
\pgfpathlineto{\pgfqpoint{11.249220in}{8.341596in}}%
\pgfpathlineto{\pgfqpoint{0.000000in}{8.341596in}}%
\pgfpathclose%
\pgfusepath{fill}%
\end{pgfscope}%
\begin{pgfscope}%
\pgfsetbuttcap%
\pgfsetmiterjoin%
\definecolor{currentfill}{rgb}{1.000000,1.000000,1.000000}%
\pgfsetfillcolor{currentfill}%
\pgfsetlinewidth{0.000000pt}%
\definecolor{currentstroke}{rgb}{0.000000,0.000000,0.000000}%
\pgfsetstrokecolor{currentstroke}%
\pgfsetstrokeopacity{0.000000}%
\pgfsetdash{}{0pt}%
\pgfpathmoveto{\pgfqpoint{0.570343in}{0.331635in}}%
\pgfpathlineto{\pgfqpoint{9.870343in}{0.331635in}}%
\pgfpathlineto{\pgfqpoint{9.870343in}{8.031635in}}%
\pgfpathlineto{\pgfqpoint{0.570343in}{8.031635in}}%
\pgfpathclose%
\pgfusepath{fill}%
\end{pgfscope}%
\begin{pgfscope}%
\pgfpathrectangle{\pgfqpoint{0.570343in}{0.331635in}}{\pgfqpoint{9.300000in}{7.700000in}}%
\pgfusepath{clip}%
\pgfsetbuttcap%
\pgfsetroundjoin%
\definecolor{currentfill}{rgb}{0.631373,0.788235,0.956863}%
\pgfsetfillcolor{currentfill}%
\pgfsetlinewidth{0.481800pt}%
\definecolor{currentstroke}{rgb}{1.000000,1.000000,1.000000}%
\pgfsetstrokecolor{currentstroke}%
\pgfsetdash{}{0pt}%
\pgfpathmoveto{\pgfqpoint{7.296937in}{5.048531in}}%
\pgfpathcurveto{\pgfqpoint{7.307987in}{5.048531in}}{\pgfqpoint{7.318586in}{5.052921in}}{\pgfqpoint{7.326400in}{5.060735in}}%
\pgfpathcurveto{\pgfqpoint{7.334213in}{5.068549in}}{\pgfqpoint{7.338604in}{5.079148in}}{\pgfqpoint{7.338604in}{5.090198in}}%
\pgfpathcurveto{\pgfqpoint{7.338604in}{5.101248in}}{\pgfqpoint{7.334213in}{5.111847in}}{\pgfqpoint{7.326400in}{5.119660in}}%
\pgfpathcurveto{\pgfqpoint{7.318586in}{5.127474in}}{\pgfqpoint{7.307987in}{5.131864in}}{\pgfqpoint{7.296937in}{5.131864in}}%
\pgfpathcurveto{\pgfqpoint{7.285887in}{5.131864in}}{\pgfqpoint{7.275288in}{5.127474in}}{\pgfqpoint{7.267474in}{5.119660in}}%
\pgfpathcurveto{\pgfqpoint{7.259661in}{5.111847in}}{\pgfqpoint{7.255270in}{5.101248in}}{\pgfqpoint{7.255270in}{5.090198in}}%
\pgfpathcurveto{\pgfqpoint{7.255270in}{5.079148in}}{\pgfqpoint{7.259661in}{5.068549in}}{\pgfqpoint{7.267474in}{5.060735in}}%
\pgfpathcurveto{\pgfqpoint{7.275288in}{5.052921in}}{\pgfqpoint{7.285887in}{5.048531in}}{\pgfqpoint{7.296937in}{5.048531in}}%
\pgfpathclose%
\pgfusepath{stroke,fill}%
\end{pgfscope}%
\begin{pgfscope}%
\pgfpathrectangle{\pgfqpoint{0.570343in}{0.331635in}}{\pgfqpoint{9.300000in}{7.700000in}}%
\pgfusepath{clip}%
\pgfsetbuttcap%
\pgfsetroundjoin%
\definecolor{currentfill}{rgb}{0.631373,0.788235,0.956863}%
\pgfsetfillcolor{currentfill}%
\pgfsetlinewidth{0.481800pt}%
\definecolor{currentstroke}{rgb}{1.000000,1.000000,1.000000}%
\pgfsetstrokecolor{currentstroke}%
\pgfsetdash{}{0pt}%
\pgfpathmoveto{\pgfqpoint{4.019475in}{3.676065in}}%
\pgfpathcurveto{\pgfqpoint{4.030525in}{3.676065in}}{\pgfqpoint{4.041124in}{3.680455in}}{\pgfqpoint{4.048938in}{3.688268in}}%
\pgfpathcurveto{\pgfqpoint{4.056752in}{3.696082in}}{\pgfqpoint{4.061142in}{3.706681in}}{\pgfqpoint{4.061142in}{3.717731in}}%
\pgfpathcurveto{\pgfqpoint{4.061142in}{3.728781in}}{\pgfqpoint{4.056752in}{3.739380in}}{\pgfqpoint{4.048938in}{3.747194in}}%
\pgfpathcurveto{\pgfqpoint{4.041124in}{3.755008in}}{\pgfqpoint{4.030525in}{3.759398in}}{\pgfqpoint{4.019475in}{3.759398in}}%
\pgfpathcurveto{\pgfqpoint{4.008425in}{3.759398in}}{\pgfqpoint{3.997826in}{3.755008in}}{\pgfqpoint{3.990012in}{3.747194in}}%
\pgfpathcurveto{\pgfqpoint{3.982199in}{3.739380in}}{\pgfqpoint{3.977809in}{3.728781in}}{\pgfqpoint{3.977809in}{3.717731in}}%
\pgfpathcurveto{\pgfqpoint{3.977809in}{3.706681in}}{\pgfqpoint{3.982199in}{3.696082in}}{\pgfqpoint{3.990012in}{3.688268in}}%
\pgfpathcurveto{\pgfqpoint{3.997826in}{3.680455in}}{\pgfqpoint{4.008425in}{3.676065in}}{\pgfqpoint{4.019475in}{3.676065in}}%
\pgfpathclose%
\pgfusepath{stroke,fill}%
\end{pgfscope}%
\begin{pgfscope}%
\pgfpathrectangle{\pgfqpoint{0.570343in}{0.331635in}}{\pgfqpoint{9.300000in}{7.700000in}}%
\pgfusepath{clip}%
\pgfsetbuttcap%
\pgfsetroundjoin%
\definecolor{currentfill}{rgb}{0.631373,0.788235,0.956863}%
\pgfsetfillcolor{currentfill}%
\pgfsetlinewidth{0.481800pt}%
\definecolor{currentstroke}{rgb}{1.000000,1.000000,1.000000}%
\pgfsetstrokecolor{currentstroke}%
\pgfsetdash{}{0pt}%
\pgfpathmoveto{\pgfqpoint{4.684473in}{3.314863in}}%
\pgfpathcurveto{\pgfqpoint{4.695523in}{3.314863in}}{\pgfqpoint{4.706122in}{3.319253in}}{\pgfqpoint{4.713936in}{3.327067in}}%
\pgfpathcurveto{\pgfqpoint{4.721749in}{3.334881in}}{\pgfqpoint{4.726140in}{3.345480in}}{\pgfqpoint{4.726140in}{3.356530in}}%
\pgfpathcurveto{\pgfqpoint{4.726140in}{3.367580in}}{\pgfqpoint{4.721749in}{3.378179in}}{\pgfqpoint{4.713936in}{3.385992in}}%
\pgfpathcurveto{\pgfqpoint{4.706122in}{3.393806in}}{\pgfqpoint{4.695523in}{3.398196in}}{\pgfqpoint{4.684473in}{3.398196in}}%
\pgfpathcurveto{\pgfqpoint{4.673423in}{3.398196in}}{\pgfqpoint{4.662824in}{3.393806in}}{\pgfqpoint{4.655010in}{3.385992in}}%
\pgfpathcurveto{\pgfqpoint{4.647196in}{3.378179in}}{\pgfqpoint{4.642806in}{3.367580in}}{\pgfqpoint{4.642806in}{3.356530in}}%
\pgfpathcurveto{\pgfqpoint{4.642806in}{3.345480in}}{\pgfqpoint{4.647196in}{3.334881in}}{\pgfqpoint{4.655010in}{3.327067in}}%
\pgfpathcurveto{\pgfqpoint{4.662824in}{3.319253in}}{\pgfqpoint{4.673423in}{3.314863in}}{\pgfqpoint{4.684473in}{3.314863in}}%
\pgfpathclose%
\pgfusepath{stroke,fill}%
\end{pgfscope}%
\begin{pgfscope}%
\pgfpathrectangle{\pgfqpoint{0.570343in}{0.331635in}}{\pgfqpoint{9.300000in}{7.700000in}}%
\pgfusepath{clip}%
\pgfsetbuttcap%
\pgfsetroundjoin%
\definecolor{currentfill}{rgb}{0.631373,0.788235,0.956863}%
\pgfsetfillcolor{currentfill}%
\pgfsetlinewidth{0.481800pt}%
\definecolor{currentstroke}{rgb}{1.000000,1.000000,1.000000}%
\pgfsetstrokecolor{currentstroke}%
\pgfsetdash{}{0pt}%
\pgfpathmoveto{\pgfqpoint{7.305182in}{2.797483in}}%
\pgfpathcurveto{\pgfqpoint{7.316232in}{2.797483in}}{\pgfqpoint{7.326831in}{2.801874in}}{\pgfqpoint{7.334644in}{2.809687in}}%
\pgfpathcurveto{\pgfqpoint{7.342458in}{2.817501in}}{\pgfqpoint{7.346848in}{2.828100in}}{\pgfqpoint{7.346848in}{2.839150in}}%
\pgfpathcurveto{\pgfqpoint{7.346848in}{2.850200in}}{\pgfqpoint{7.342458in}{2.860799in}}{\pgfqpoint{7.334644in}{2.868613in}}%
\pgfpathcurveto{\pgfqpoint{7.326831in}{2.876426in}}{\pgfqpoint{7.316232in}{2.880817in}}{\pgfqpoint{7.305182in}{2.880817in}}%
\pgfpathcurveto{\pgfqpoint{7.294131in}{2.880817in}}{\pgfqpoint{7.283532in}{2.876426in}}{\pgfqpoint{7.275719in}{2.868613in}}%
\pgfpathcurveto{\pgfqpoint{7.267905in}{2.860799in}}{\pgfqpoint{7.263515in}{2.850200in}}{\pgfqpoint{7.263515in}{2.839150in}}%
\pgfpathcurveto{\pgfqpoint{7.263515in}{2.828100in}}{\pgfqpoint{7.267905in}{2.817501in}}{\pgfqpoint{7.275719in}{2.809687in}}%
\pgfpathcurveto{\pgfqpoint{7.283532in}{2.801874in}}{\pgfqpoint{7.294131in}{2.797483in}}{\pgfqpoint{7.305182in}{2.797483in}}%
\pgfpathclose%
\pgfusepath{stroke,fill}%
\end{pgfscope}%
\begin{pgfscope}%
\pgfpathrectangle{\pgfqpoint{0.570343in}{0.331635in}}{\pgfqpoint{9.300000in}{7.700000in}}%
\pgfusepath{clip}%
\pgfsetbuttcap%
\pgfsetroundjoin%
\definecolor{currentfill}{rgb}{0.631373,0.788235,0.956863}%
\pgfsetfillcolor{currentfill}%
\pgfsetlinewidth{0.481800pt}%
\definecolor{currentstroke}{rgb}{1.000000,1.000000,1.000000}%
\pgfsetstrokecolor{currentstroke}%
\pgfsetdash{}{0pt}%
\pgfpathmoveto{\pgfqpoint{8.044602in}{5.249564in}}%
\pgfpathcurveto{\pgfqpoint{8.055652in}{5.249564in}}{\pgfqpoint{8.066251in}{5.253955in}}{\pgfqpoint{8.074065in}{5.261768in}}%
\pgfpathcurveto{\pgfqpoint{8.081878in}{5.269582in}}{\pgfqpoint{8.086269in}{5.280181in}}{\pgfqpoint{8.086269in}{5.291231in}}%
\pgfpathcurveto{\pgfqpoint{8.086269in}{5.302281in}}{\pgfqpoint{8.081878in}{5.312880in}}{\pgfqpoint{8.074065in}{5.320694in}}%
\pgfpathcurveto{\pgfqpoint{8.066251in}{5.328508in}}{\pgfqpoint{8.055652in}{5.332898in}}{\pgfqpoint{8.044602in}{5.332898in}}%
\pgfpathcurveto{\pgfqpoint{8.033552in}{5.332898in}}{\pgfqpoint{8.022953in}{5.328508in}}{\pgfqpoint{8.015139in}{5.320694in}}%
\pgfpathcurveto{\pgfqpoint{8.007326in}{5.312880in}}{\pgfqpoint{8.002935in}{5.302281in}}{\pgfqpoint{8.002935in}{5.291231in}}%
\pgfpathcurveto{\pgfqpoint{8.002935in}{5.280181in}}{\pgfqpoint{8.007326in}{5.269582in}}{\pgfqpoint{8.015139in}{5.261768in}}%
\pgfpathcurveto{\pgfqpoint{8.022953in}{5.253955in}}{\pgfqpoint{8.033552in}{5.249564in}}{\pgfqpoint{8.044602in}{5.249564in}}%
\pgfpathclose%
\pgfusepath{stroke,fill}%
\end{pgfscope}%
\begin{pgfscope}%
\pgfpathrectangle{\pgfqpoint{0.570343in}{0.331635in}}{\pgfqpoint{9.300000in}{7.700000in}}%
\pgfusepath{clip}%
\pgfsetbuttcap%
\pgfsetroundjoin%
\definecolor{currentfill}{rgb}{0.631373,0.788235,0.956863}%
\pgfsetfillcolor{currentfill}%
\pgfsetlinewidth{0.481800pt}%
\definecolor{currentstroke}{rgb}{1.000000,1.000000,1.000000}%
\pgfsetstrokecolor{currentstroke}%
\pgfsetdash{}{0pt}%
\pgfpathmoveto{\pgfqpoint{6.828282in}{4.387013in}}%
\pgfpathcurveto{\pgfqpoint{6.839332in}{4.387013in}}{\pgfqpoint{6.849931in}{4.391403in}}{\pgfqpoint{6.857745in}{4.399217in}}%
\pgfpathcurveto{\pgfqpoint{6.865558in}{4.407030in}}{\pgfqpoint{6.869948in}{4.417629in}}{\pgfqpoint{6.869948in}{4.428679in}}%
\pgfpathcurveto{\pgfqpoint{6.869948in}{4.439730in}}{\pgfqpoint{6.865558in}{4.450329in}}{\pgfqpoint{6.857745in}{4.458142in}}%
\pgfpathcurveto{\pgfqpoint{6.849931in}{4.465956in}}{\pgfqpoint{6.839332in}{4.470346in}}{\pgfqpoint{6.828282in}{4.470346in}}%
\pgfpathcurveto{\pgfqpoint{6.817232in}{4.470346in}}{\pgfqpoint{6.806633in}{4.465956in}}{\pgfqpoint{6.798819in}{4.458142in}}%
\pgfpathcurveto{\pgfqpoint{6.791005in}{4.450329in}}{\pgfqpoint{6.786615in}{4.439730in}}{\pgfqpoint{6.786615in}{4.428679in}}%
\pgfpathcurveto{\pgfqpoint{6.786615in}{4.417629in}}{\pgfqpoint{6.791005in}{4.407030in}}{\pgfqpoint{6.798819in}{4.399217in}}%
\pgfpathcurveto{\pgfqpoint{6.806633in}{4.391403in}}{\pgfqpoint{6.817232in}{4.387013in}}{\pgfqpoint{6.828282in}{4.387013in}}%
\pgfpathclose%
\pgfusepath{stroke,fill}%
\end{pgfscope}%
\begin{pgfscope}%
\pgfpathrectangle{\pgfqpoint{0.570343in}{0.331635in}}{\pgfqpoint{9.300000in}{7.700000in}}%
\pgfusepath{clip}%
\pgfsetbuttcap%
\pgfsetroundjoin%
\definecolor{currentfill}{rgb}{0.631373,0.788235,0.956863}%
\pgfsetfillcolor{currentfill}%
\pgfsetlinewidth{0.481800pt}%
\definecolor{currentstroke}{rgb}{1.000000,1.000000,1.000000}%
\pgfsetstrokecolor{currentstroke}%
\pgfsetdash{}{0pt}%
\pgfpathmoveto{\pgfqpoint{8.665804in}{4.777059in}}%
\pgfpathcurveto{\pgfqpoint{8.676854in}{4.777059in}}{\pgfqpoint{8.687453in}{4.781449in}}{\pgfqpoint{8.695266in}{4.789263in}}%
\pgfpathcurveto{\pgfqpoint{8.703080in}{4.797076in}}{\pgfqpoint{8.707470in}{4.807675in}}{\pgfqpoint{8.707470in}{4.818725in}}%
\pgfpathcurveto{\pgfqpoint{8.707470in}{4.829776in}}{\pgfqpoint{8.703080in}{4.840375in}}{\pgfqpoint{8.695266in}{4.848188in}}%
\pgfpathcurveto{\pgfqpoint{8.687453in}{4.856002in}}{\pgfqpoint{8.676854in}{4.860392in}}{\pgfqpoint{8.665804in}{4.860392in}}%
\pgfpathcurveto{\pgfqpoint{8.654754in}{4.860392in}}{\pgfqpoint{8.644154in}{4.856002in}}{\pgfqpoint{8.636341in}{4.848188in}}%
\pgfpathcurveto{\pgfqpoint{8.628527in}{4.840375in}}{\pgfqpoint{8.624137in}{4.829776in}}{\pgfqpoint{8.624137in}{4.818725in}}%
\pgfpathcurveto{\pgfqpoint{8.624137in}{4.807675in}}{\pgfqpoint{8.628527in}{4.797076in}}{\pgfqpoint{8.636341in}{4.789263in}}%
\pgfpathcurveto{\pgfqpoint{8.644154in}{4.781449in}}{\pgfqpoint{8.654754in}{4.777059in}}{\pgfqpoint{8.665804in}{4.777059in}}%
\pgfpathclose%
\pgfusepath{stroke,fill}%
\end{pgfscope}%
\begin{pgfscope}%
\pgfpathrectangle{\pgfqpoint{0.570343in}{0.331635in}}{\pgfqpoint{9.300000in}{7.700000in}}%
\pgfusepath{clip}%
\pgfsetbuttcap%
\pgfsetroundjoin%
\definecolor{currentfill}{rgb}{0.631373,0.788235,0.956863}%
\pgfsetfillcolor{currentfill}%
\pgfsetlinewidth{0.481800pt}%
\definecolor{currentstroke}{rgb}{1.000000,1.000000,1.000000}%
\pgfsetstrokecolor{currentstroke}%
\pgfsetdash{}{0pt}%
\pgfpathmoveto{\pgfqpoint{8.976651in}{5.644673in}}%
\pgfpathcurveto{\pgfqpoint{8.987701in}{5.644673in}}{\pgfqpoint{8.998301in}{5.649063in}}{\pgfqpoint{9.006114in}{5.656877in}}%
\pgfpathcurveto{\pgfqpoint{9.013928in}{5.664690in}}{\pgfqpoint{9.018318in}{5.675289in}}{\pgfqpoint{9.018318in}{5.686339in}}%
\pgfpathcurveto{\pgfqpoint{9.018318in}{5.697390in}}{\pgfqpoint{9.013928in}{5.707989in}}{\pgfqpoint{9.006114in}{5.715802in}}%
\pgfpathcurveto{\pgfqpoint{8.998301in}{5.723616in}}{\pgfqpoint{8.987701in}{5.728006in}}{\pgfqpoint{8.976651in}{5.728006in}}%
\pgfpathcurveto{\pgfqpoint{8.965601in}{5.728006in}}{\pgfqpoint{8.955002in}{5.723616in}}{\pgfqpoint{8.947189in}{5.715802in}}%
\pgfpathcurveto{\pgfqpoint{8.939375in}{5.707989in}}{\pgfqpoint{8.934985in}{5.697390in}}{\pgfqpoint{8.934985in}{5.686339in}}%
\pgfpathcurveto{\pgfqpoint{8.934985in}{5.675289in}}{\pgfqpoint{8.939375in}{5.664690in}}{\pgfqpoint{8.947189in}{5.656877in}}%
\pgfpathcurveto{\pgfqpoint{8.955002in}{5.649063in}}{\pgfqpoint{8.965601in}{5.644673in}}{\pgfqpoint{8.976651in}{5.644673in}}%
\pgfpathclose%
\pgfusepath{stroke,fill}%
\end{pgfscope}%
\begin{pgfscope}%
\pgfpathrectangle{\pgfqpoint{0.570343in}{0.331635in}}{\pgfqpoint{9.300000in}{7.700000in}}%
\pgfusepath{clip}%
\pgfsetbuttcap%
\pgfsetroundjoin%
\definecolor{currentfill}{rgb}{0.631373,0.788235,0.956863}%
\pgfsetfillcolor{currentfill}%
\pgfsetlinewidth{0.481800pt}%
\definecolor{currentstroke}{rgb}{1.000000,1.000000,1.000000}%
\pgfsetstrokecolor{currentstroke}%
\pgfsetdash{}{0pt}%
\pgfpathmoveto{\pgfqpoint{5.272172in}{4.776375in}}%
\pgfpathcurveto{\pgfqpoint{5.283222in}{4.776375in}}{\pgfqpoint{5.293821in}{4.780765in}}{\pgfqpoint{5.301634in}{4.788579in}}%
\pgfpathcurveto{\pgfqpoint{5.309448in}{4.796392in}}{\pgfqpoint{5.313838in}{4.806991in}}{\pgfqpoint{5.313838in}{4.818041in}}%
\pgfpathcurveto{\pgfqpoint{5.313838in}{4.829092in}}{\pgfqpoint{5.309448in}{4.839691in}}{\pgfqpoint{5.301634in}{4.847504in}}%
\pgfpathcurveto{\pgfqpoint{5.293821in}{4.855318in}}{\pgfqpoint{5.283222in}{4.859708in}}{\pgfqpoint{5.272172in}{4.859708in}}%
\pgfpathcurveto{\pgfqpoint{5.261121in}{4.859708in}}{\pgfqpoint{5.250522in}{4.855318in}}{\pgfqpoint{5.242709in}{4.847504in}}%
\pgfpathcurveto{\pgfqpoint{5.234895in}{4.839691in}}{\pgfqpoint{5.230505in}{4.829092in}}{\pgfqpoint{5.230505in}{4.818041in}}%
\pgfpathcurveto{\pgfqpoint{5.230505in}{4.806991in}}{\pgfqpoint{5.234895in}{4.796392in}}{\pgfqpoint{5.242709in}{4.788579in}}%
\pgfpathcurveto{\pgfqpoint{5.250522in}{4.780765in}}{\pgfqpoint{5.261121in}{4.776375in}}{\pgfqpoint{5.272172in}{4.776375in}}%
\pgfpathclose%
\pgfusepath{stroke,fill}%
\end{pgfscope}%
\begin{pgfscope}%
\pgfpathrectangle{\pgfqpoint{0.570343in}{0.331635in}}{\pgfqpoint{9.300000in}{7.700000in}}%
\pgfusepath{clip}%
\pgfsetbuttcap%
\pgfsetroundjoin%
\definecolor{currentfill}{rgb}{0.631373,0.788235,0.956863}%
\pgfsetfillcolor{currentfill}%
\pgfsetlinewidth{0.481800pt}%
\definecolor{currentstroke}{rgb}{1.000000,1.000000,1.000000}%
\pgfsetstrokecolor{currentstroke}%
\pgfsetdash{}{0pt}%
\pgfpathmoveto{\pgfqpoint{3.063486in}{6.430430in}}%
\pgfpathcurveto{\pgfqpoint{3.074536in}{6.430430in}}{\pgfqpoint{3.085135in}{6.434820in}}{\pgfqpoint{3.092949in}{6.442634in}}%
\pgfpathcurveto{\pgfqpoint{3.100762in}{6.450447in}}{\pgfqpoint{3.105152in}{6.461046in}}{\pgfqpoint{3.105152in}{6.472096in}}%
\pgfpathcurveto{\pgfqpoint{3.105152in}{6.483146in}}{\pgfqpoint{3.100762in}{6.493745in}}{\pgfqpoint{3.092949in}{6.501559in}}%
\pgfpathcurveto{\pgfqpoint{3.085135in}{6.509373in}}{\pgfqpoint{3.074536in}{6.513763in}}{\pgfqpoint{3.063486in}{6.513763in}}%
\pgfpathcurveto{\pgfqpoint{3.052436in}{6.513763in}}{\pgfqpoint{3.041837in}{6.509373in}}{\pgfqpoint{3.034023in}{6.501559in}}%
\pgfpathcurveto{\pgfqpoint{3.026209in}{6.493745in}}{\pgfqpoint{3.021819in}{6.483146in}}{\pgfqpoint{3.021819in}{6.472096in}}%
\pgfpathcurveto{\pgfqpoint{3.021819in}{6.461046in}}{\pgfqpoint{3.026209in}{6.450447in}}{\pgfqpoint{3.034023in}{6.442634in}}%
\pgfpathcurveto{\pgfqpoint{3.041837in}{6.434820in}}{\pgfqpoint{3.052436in}{6.430430in}}{\pgfqpoint{3.063486in}{6.430430in}}%
\pgfpathclose%
\pgfusepath{stroke,fill}%
\end{pgfscope}%
\begin{pgfscope}%
\pgfpathrectangle{\pgfqpoint{0.570343in}{0.331635in}}{\pgfqpoint{9.300000in}{7.700000in}}%
\pgfusepath{clip}%
\pgfsetbuttcap%
\pgfsetroundjoin%
\definecolor{currentfill}{rgb}{0.631373,0.788235,0.956863}%
\pgfsetfillcolor{currentfill}%
\pgfsetlinewidth{0.481800pt}%
\definecolor{currentstroke}{rgb}{1.000000,1.000000,1.000000}%
\pgfsetstrokecolor{currentstroke}%
\pgfsetdash{}{0pt}%
\pgfpathmoveto{\pgfqpoint{6.063540in}{4.763605in}}%
\pgfpathcurveto{\pgfqpoint{6.074590in}{4.763605in}}{\pgfqpoint{6.085189in}{4.767995in}}{\pgfqpoint{6.093003in}{4.775808in}}%
\pgfpathcurveto{\pgfqpoint{6.100816in}{4.783622in}}{\pgfqpoint{6.105207in}{4.794221in}}{\pgfqpoint{6.105207in}{4.805271in}}%
\pgfpathcurveto{\pgfqpoint{6.105207in}{4.816321in}}{\pgfqpoint{6.100816in}{4.826920in}}{\pgfqpoint{6.093003in}{4.834734in}}%
\pgfpathcurveto{\pgfqpoint{6.085189in}{4.842548in}}{\pgfqpoint{6.074590in}{4.846938in}}{\pgfqpoint{6.063540in}{4.846938in}}%
\pgfpathcurveto{\pgfqpoint{6.052490in}{4.846938in}}{\pgfqpoint{6.041891in}{4.842548in}}{\pgfqpoint{6.034077in}{4.834734in}}%
\pgfpathcurveto{\pgfqpoint{6.026264in}{4.826920in}}{\pgfqpoint{6.021873in}{4.816321in}}{\pgfqpoint{6.021873in}{4.805271in}}%
\pgfpathcurveto{\pgfqpoint{6.021873in}{4.794221in}}{\pgfqpoint{6.026264in}{4.783622in}}{\pgfqpoint{6.034077in}{4.775808in}}%
\pgfpathcurveto{\pgfqpoint{6.041891in}{4.767995in}}{\pgfqpoint{6.052490in}{4.763605in}}{\pgfqpoint{6.063540in}{4.763605in}}%
\pgfpathclose%
\pgfusepath{stroke,fill}%
\end{pgfscope}%
\begin{pgfscope}%
\pgfpathrectangle{\pgfqpoint{0.570343in}{0.331635in}}{\pgfqpoint{9.300000in}{7.700000in}}%
\pgfusepath{clip}%
\pgfsetbuttcap%
\pgfsetroundjoin%
\definecolor{currentfill}{rgb}{0.631373,0.788235,0.956863}%
\pgfsetfillcolor{currentfill}%
\pgfsetlinewidth{0.481800pt}%
\definecolor{currentstroke}{rgb}{1.000000,1.000000,1.000000}%
\pgfsetstrokecolor{currentstroke}%
\pgfsetdash{}{0pt}%
\pgfpathmoveto{\pgfqpoint{1.762911in}{3.854949in}}%
\pgfpathcurveto{\pgfqpoint{1.773961in}{3.854949in}}{\pgfqpoint{1.784560in}{3.859339in}}{\pgfqpoint{1.792374in}{3.867153in}}%
\pgfpathcurveto{\pgfqpoint{1.800187in}{3.874967in}}{\pgfqpoint{1.804577in}{3.885566in}}{\pgfqpoint{1.804577in}{3.896616in}}%
\pgfpathcurveto{\pgfqpoint{1.804577in}{3.907666in}}{\pgfqpoint{1.800187in}{3.918265in}}{\pgfqpoint{1.792374in}{3.926078in}}%
\pgfpathcurveto{\pgfqpoint{1.784560in}{3.933892in}}{\pgfqpoint{1.773961in}{3.938282in}}{\pgfqpoint{1.762911in}{3.938282in}}%
\pgfpathcurveto{\pgfqpoint{1.751861in}{3.938282in}}{\pgfqpoint{1.741262in}{3.933892in}}{\pgfqpoint{1.733448in}{3.926078in}}%
\pgfpathcurveto{\pgfqpoint{1.725634in}{3.918265in}}{\pgfqpoint{1.721244in}{3.907666in}}{\pgfqpoint{1.721244in}{3.896616in}}%
\pgfpathcurveto{\pgfqpoint{1.721244in}{3.885566in}}{\pgfqpoint{1.725634in}{3.874967in}}{\pgfqpoint{1.733448in}{3.867153in}}%
\pgfpathcurveto{\pgfqpoint{1.741262in}{3.859339in}}{\pgfqpoint{1.751861in}{3.854949in}}{\pgfqpoint{1.762911in}{3.854949in}}%
\pgfpathclose%
\pgfusepath{stroke,fill}%
\end{pgfscope}%
\begin{pgfscope}%
\pgfpathrectangle{\pgfqpoint{0.570343in}{0.331635in}}{\pgfqpoint{9.300000in}{7.700000in}}%
\pgfusepath{clip}%
\pgfsetbuttcap%
\pgfsetroundjoin%
\definecolor{currentfill}{rgb}{0.631373,0.788235,0.956863}%
\pgfsetfillcolor{currentfill}%
\pgfsetlinewidth{0.481800pt}%
\definecolor{currentstroke}{rgb}{1.000000,1.000000,1.000000}%
\pgfsetstrokecolor{currentstroke}%
\pgfsetdash{}{0pt}%
\pgfpathmoveto{\pgfqpoint{4.671643in}{5.114087in}}%
\pgfpathcurveto{\pgfqpoint{4.682693in}{5.114087in}}{\pgfqpoint{4.693292in}{5.118477in}}{\pgfqpoint{4.701106in}{5.126291in}}%
\pgfpathcurveto{\pgfqpoint{4.708919in}{5.134105in}}{\pgfqpoint{4.713310in}{5.144704in}}{\pgfqpoint{4.713310in}{5.155754in}}%
\pgfpathcurveto{\pgfqpoint{4.713310in}{5.166804in}}{\pgfqpoint{4.708919in}{5.177403in}}{\pgfqpoint{4.701106in}{5.185216in}}%
\pgfpathcurveto{\pgfqpoint{4.693292in}{5.193030in}}{\pgfqpoint{4.682693in}{5.197420in}}{\pgfqpoint{4.671643in}{5.197420in}}%
\pgfpathcurveto{\pgfqpoint{4.660593in}{5.197420in}}{\pgfqpoint{4.649994in}{5.193030in}}{\pgfqpoint{4.642180in}{5.185216in}}%
\pgfpathcurveto{\pgfqpoint{4.634367in}{5.177403in}}{\pgfqpoint{4.629976in}{5.166804in}}{\pgfqpoint{4.629976in}{5.155754in}}%
\pgfpathcurveto{\pgfqpoint{4.629976in}{5.144704in}}{\pgfqpoint{4.634367in}{5.134105in}}{\pgfqpoint{4.642180in}{5.126291in}}%
\pgfpathcurveto{\pgfqpoint{4.649994in}{5.118477in}}{\pgfqpoint{4.660593in}{5.114087in}}{\pgfqpoint{4.671643in}{5.114087in}}%
\pgfpathclose%
\pgfusepath{stroke,fill}%
\end{pgfscope}%
\begin{pgfscope}%
\pgfpathrectangle{\pgfqpoint{0.570343in}{0.331635in}}{\pgfqpoint{9.300000in}{7.700000in}}%
\pgfusepath{clip}%
\pgfsetbuttcap%
\pgfsetroundjoin%
\definecolor{currentfill}{rgb}{0.631373,0.788235,0.956863}%
\pgfsetfillcolor{currentfill}%
\pgfsetlinewidth{0.481800pt}%
\definecolor{currentstroke}{rgb}{1.000000,1.000000,1.000000}%
\pgfsetstrokecolor{currentstroke}%
\pgfsetdash{}{0pt}%
\pgfpathmoveto{\pgfqpoint{2.082714in}{3.013608in}}%
\pgfpathcurveto{\pgfqpoint{2.093764in}{3.013608in}}{\pgfqpoint{2.104363in}{3.017999in}}{\pgfqpoint{2.112177in}{3.025812in}}%
\pgfpathcurveto{\pgfqpoint{2.119990in}{3.033626in}}{\pgfqpoint{2.124380in}{3.044225in}}{\pgfqpoint{2.124380in}{3.055275in}}%
\pgfpathcurveto{\pgfqpoint{2.124380in}{3.066325in}}{\pgfqpoint{2.119990in}{3.076924in}}{\pgfqpoint{2.112177in}{3.084738in}}%
\pgfpathcurveto{\pgfqpoint{2.104363in}{3.092551in}}{\pgfqpoint{2.093764in}{3.096942in}}{\pgfqpoint{2.082714in}{3.096942in}}%
\pgfpathcurveto{\pgfqpoint{2.071664in}{3.096942in}}{\pgfqpoint{2.061065in}{3.092551in}}{\pgfqpoint{2.053251in}{3.084738in}}%
\pgfpathcurveto{\pgfqpoint{2.045437in}{3.076924in}}{\pgfqpoint{2.041047in}{3.066325in}}{\pgfqpoint{2.041047in}{3.055275in}}%
\pgfpathcurveto{\pgfqpoint{2.041047in}{3.044225in}}{\pgfqpoint{2.045437in}{3.033626in}}{\pgfqpoint{2.053251in}{3.025812in}}%
\pgfpathcurveto{\pgfqpoint{2.061065in}{3.017999in}}{\pgfqpoint{2.071664in}{3.013608in}}{\pgfqpoint{2.082714in}{3.013608in}}%
\pgfpathclose%
\pgfusepath{stroke,fill}%
\end{pgfscope}%
\begin{pgfscope}%
\pgfpathrectangle{\pgfqpoint{0.570343in}{0.331635in}}{\pgfqpoint{9.300000in}{7.700000in}}%
\pgfusepath{clip}%
\pgfsetbuttcap%
\pgfsetroundjoin%
\definecolor{currentfill}{rgb}{0.631373,0.788235,0.956863}%
\pgfsetfillcolor{currentfill}%
\pgfsetlinewidth{0.481800pt}%
\definecolor{currentstroke}{rgb}{1.000000,1.000000,1.000000}%
\pgfsetstrokecolor{currentstroke}%
\pgfsetdash{}{0pt}%
\pgfpathmoveto{\pgfqpoint{2.029248in}{1.536456in}}%
\pgfpathcurveto{\pgfqpoint{2.040298in}{1.536456in}}{\pgfqpoint{2.050897in}{1.540846in}}{\pgfqpoint{2.058711in}{1.548660in}}%
\pgfpathcurveto{\pgfqpoint{2.066524in}{1.556473in}}{\pgfqpoint{2.070915in}{1.567072in}}{\pgfqpoint{2.070915in}{1.578122in}}%
\pgfpathcurveto{\pgfqpoint{2.070915in}{1.589172in}}{\pgfqpoint{2.066524in}{1.599772in}}{\pgfqpoint{2.058711in}{1.607585in}}%
\pgfpathcurveto{\pgfqpoint{2.050897in}{1.615399in}}{\pgfqpoint{2.040298in}{1.619789in}}{\pgfqpoint{2.029248in}{1.619789in}}%
\pgfpathcurveto{\pgfqpoint{2.018198in}{1.619789in}}{\pgfqpoint{2.007599in}{1.615399in}}{\pgfqpoint{1.999785in}{1.607585in}}%
\pgfpathcurveto{\pgfqpoint{1.991972in}{1.599772in}}{\pgfqpoint{1.987581in}{1.589172in}}{\pgfqpoint{1.987581in}{1.578122in}}%
\pgfpathcurveto{\pgfqpoint{1.987581in}{1.567072in}}{\pgfqpoint{1.991972in}{1.556473in}}{\pgfqpoint{1.999785in}{1.548660in}}%
\pgfpathcurveto{\pgfqpoint{2.007599in}{1.540846in}}{\pgfqpoint{2.018198in}{1.536456in}}{\pgfqpoint{2.029248in}{1.536456in}}%
\pgfpathclose%
\pgfusepath{stroke,fill}%
\end{pgfscope}%
\begin{pgfscope}%
\pgfpathrectangle{\pgfqpoint{0.570343in}{0.331635in}}{\pgfqpoint{9.300000in}{7.700000in}}%
\pgfusepath{clip}%
\pgfsetbuttcap%
\pgfsetroundjoin%
\definecolor{currentfill}{rgb}{0.631373,0.788235,0.956863}%
\pgfsetfillcolor{currentfill}%
\pgfsetlinewidth{0.481800pt}%
\definecolor{currentstroke}{rgb}{1.000000,1.000000,1.000000}%
\pgfsetstrokecolor{currentstroke}%
\pgfsetdash{}{0pt}%
\pgfpathmoveto{\pgfqpoint{5.175896in}{3.975564in}}%
\pgfpathcurveto{\pgfqpoint{5.186946in}{3.975564in}}{\pgfqpoint{5.197545in}{3.979954in}}{\pgfqpoint{5.205358in}{3.987768in}}%
\pgfpathcurveto{\pgfqpoint{5.213172in}{3.995581in}}{\pgfqpoint{5.217562in}{4.006181in}}{\pgfqpoint{5.217562in}{4.017231in}}%
\pgfpathcurveto{\pgfqpoint{5.217562in}{4.028281in}}{\pgfqpoint{5.213172in}{4.038880in}}{\pgfqpoint{5.205358in}{4.046693in}}%
\pgfpathcurveto{\pgfqpoint{5.197545in}{4.054507in}}{\pgfqpoint{5.186946in}{4.058897in}}{\pgfqpoint{5.175896in}{4.058897in}}%
\pgfpathcurveto{\pgfqpoint{5.164845in}{4.058897in}}{\pgfqpoint{5.154246in}{4.054507in}}{\pgfqpoint{5.146433in}{4.046693in}}%
\pgfpathcurveto{\pgfqpoint{5.138619in}{4.038880in}}{\pgfqpoint{5.134229in}{4.028281in}}{\pgfqpoint{5.134229in}{4.017231in}}%
\pgfpathcurveto{\pgfqpoint{5.134229in}{4.006181in}}{\pgfqpoint{5.138619in}{3.995581in}}{\pgfqpoint{5.146433in}{3.987768in}}%
\pgfpathcurveto{\pgfqpoint{5.154246in}{3.979954in}}{\pgfqpoint{5.164845in}{3.975564in}}{\pgfqpoint{5.175896in}{3.975564in}}%
\pgfpathclose%
\pgfusepath{stroke,fill}%
\end{pgfscope}%
\begin{pgfscope}%
\pgfpathrectangle{\pgfqpoint{0.570343in}{0.331635in}}{\pgfqpoint{9.300000in}{7.700000in}}%
\pgfusepath{clip}%
\pgfsetbuttcap%
\pgfsetroundjoin%
\definecolor{currentfill}{rgb}{0.631373,0.788235,0.956863}%
\pgfsetfillcolor{currentfill}%
\pgfsetlinewidth{0.481800pt}%
\definecolor{currentstroke}{rgb}{1.000000,1.000000,1.000000}%
\pgfsetstrokecolor{currentstroke}%
\pgfsetdash{}{0pt}%
\pgfpathmoveto{\pgfqpoint{3.674629in}{5.500123in}}%
\pgfpathcurveto{\pgfqpoint{3.685679in}{5.500123in}}{\pgfqpoint{3.696278in}{5.504514in}}{\pgfqpoint{3.704092in}{5.512327in}}%
\pgfpathcurveto{\pgfqpoint{3.711905in}{5.520141in}}{\pgfqpoint{3.716296in}{5.530740in}}{\pgfqpoint{3.716296in}{5.541790in}}%
\pgfpathcurveto{\pgfqpoint{3.716296in}{5.552840in}}{\pgfqpoint{3.711905in}{5.563439in}}{\pgfqpoint{3.704092in}{5.571253in}}%
\pgfpathcurveto{\pgfqpoint{3.696278in}{5.579067in}}{\pgfqpoint{3.685679in}{5.583457in}}{\pgfqpoint{3.674629in}{5.583457in}}%
\pgfpathcurveto{\pgfqpoint{3.663579in}{5.583457in}}{\pgfqpoint{3.652980in}{5.579067in}}{\pgfqpoint{3.645166in}{5.571253in}}%
\pgfpathcurveto{\pgfqpoint{3.637353in}{5.563439in}}{\pgfqpoint{3.632962in}{5.552840in}}{\pgfqpoint{3.632962in}{5.541790in}}%
\pgfpathcurveto{\pgfqpoint{3.632962in}{5.530740in}}{\pgfqpoint{3.637353in}{5.520141in}}{\pgfqpoint{3.645166in}{5.512327in}}%
\pgfpathcurveto{\pgfqpoint{3.652980in}{5.504514in}}{\pgfqpoint{3.663579in}{5.500123in}}{\pgfqpoint{3.674629in}{5.500123in}}%
\pgfpathclose%
\pgfusepath{stroke,fill}%
\end{pgfscope}%
\begin{pgfscope}%
\pgfpathrectangle{\pgfqpoint{0.570343in}{0.331635in}}{\pgfqpoint{9.300000in}{7.700000in}}%
\pgfusepath{clip}%
\pgfsetbuttcap%
\pgfsetroundjoin%
\definecolor{currentfill}{rgb}{0.631373,0.788235,0.956863}%
\pgfsetfillcolor{currentfill}%
\pgfsetlinewidth{0.481800pt}%
\definecolor{currentstroke}{rgb}{1.000000,1.000000,1.000000}%
\pgfsetstrokecolor{currentstroke}%
\pgfsetdash{}{0pt}%
\pgfpathmoveto{\pgfqpoint{1.383929in}{4.998171in}}%
\pgfpathcurveto{\pgfqpoint{1.394980in}{4.998171in}}{\pgfqpoint{1.405579in}{5.002561in}}{\pgfqpoint{1.413392in}{5.010375in}}%
\pgfpathcurveto{\pgfqpoint{1.421206in}{5.018188in}}{\pgfqpoint{1.425596in}{5.028787in}}{\pgfqpoint{1.425596in}{5.039838in}}%
\pgfpathcurveto{\pgfqpoint{1.425596in}{5.050888in}}{\pgfqpoint{1.421206in}{5.061487in}}{\pgfqpoint{1.413392in}{5.069300in}}%
\pgfpathcurveto{\pgfqpoint{1.405579in}{5.077114in}}{\pgfqpoint{1.394980in}{5.081504in}}{\pgfqpoint{1.383929in}{5.081504in}}%
\pgfpathcurveto{\pgfqpoint{1.372879in}{5.081504in}}{\pgfqpoint{1.362280in}{5.077114in}}{\pgfqpoint{1.354467in}{5.069300in}}%
\pgfpathcurveto{\pgfqpoint{1.346653in}{5.061487in}}{\pgfqpoint{1.342263in}{5.050888in}}{\pgfqpoint{1.342263in}{5.039838in}}%
\pgfpathcurveto{\pgfqpoint{1.342263in}{5.028787in}}{\pgfqpoint{1.346653in}{5.018188in}}{\pgfqpoint{1.354467in}{5.010375in}}%
\pgfpathcurveto{\pgfqpoint{1.362280in}{5.002561in}}{\pgfqpoint{1.372879in}{4.998171in}}{\pgfqpoint{1.383929in}{4.998171in}}%
\pgfpathclose%
\pgfusepath{stroke,fill}%
\end{pgfscope}%
\begin{pgfscope}%
\pgfpathrectangle{\pgfqpoint{0.570343in}{0.331635in}}{\pgfqpoint{9.300000in}{7.700000in}}%
\pgfusepath{clip}%
\pgfsetbuttcap%
\pgfsetroundjoin%
\definecolor{currentfill}{rgb}{0.631373,0.788235,0.956863}%
\pgfsetfillcolor{currentfill}%
\pgfsetlinewidth{0.481800pt}%
\definecolor{currentstroke}{rgb}{1.000000,1.000000,1.000000}%
\pgfsetstrokecolor{currentstroke}%
\pgfsetdash{}{0pt}%
\pgfpathmoveto{\pgfqpoint{3.677550in}{4.732443in}}%
\pgfpathcurveto{\pgfqpoint{3.688600in}{4.732443in}}{\pgfqpoint{3.699199in}{4.736833in}}{\pgfqpoint{3.707013in}{4.744646in}}%
\pgfpathcurveto{\pgfqpoint{3.714826in}{4.752460in}}{\pgfqpoint{3.719216in}{4.763059in}}{\pgfqpoint{3.719216in}{4.774109in}}%
\pgfpathcurveto{\pgfqpoint{3.719216in}{4.785159in}}{\pgfqpoint{3.714826in}{4.795758in}}{\pgfqpoint{3.707013in}{4.803572in}}%
\pgfpathcurveto{\pgfqpoint{3.699199in}{4.811386in}}{\pgfqpoint{3.688600in}{4.815776in}}{\pgfqpoint{3.677550in}{4.815776in}}%
\pgfpathcurveto{\pgfqpoint{3.666500in}{4.815776in}}{\pgfqpoint{3.655901in}{4.811386in}}{\pgfqpoint{3.648087in}{4.803572in}}%
\pgfpathcurveto{\pgfqpoint{3.640273in}{4.795758in}}{\pgfqpoint{3.635883in}{4.785159in}}{\pgfqpoint{3.635883in}{4.774109in}}%
\pgfpathcurveto{\pgfqpoint{3.635883in}{4.763059in}}{\pgfqpoint{3.640273in}{4.752460in}}{\pgfqpoint{3.648087in}{4.744646in}}%
\pgfpathcurveto{\pgfqpoint{3.655901in}{4.736833in}}{\pgfqpoint{3.666500in}{4.732443in}}{\pgfqpoint{3.677550in}{4.732443in}}%
\pgfpathclose%
\pgfusepath{stroke,fill}%
\end{pgfscope}%
\begin{pgfscope}%
\pgfpathrectangle{\pgfqpoint{0.570343in}{0.331635in}}{\pgfqpoint{9.300000in}{7.700000in}}%
\pgfusepath{clip}%
\pgfsetbuttcap%
\pgfsetroundjoin%
\definecolor{currentfill}{rgb}{0.631373,0.788235,0.956863}%
\pgfsetfillcolor{currentfill}%
\pgfsetlinewidth{0.481800pt}%
\definecolor{currentstroke}{rgb}{1.000000,1.000000,1.000000}%
\pgfsetstrokecolor{currentstroke}%
\pgfsetdash{}{0pt}%
\pgfpathmoveto{\pgfqpoint{6.563827in}{3.604344in}}%
\pgfpathcurveto{\pgfqpoint{6.574877in}{3.604344in}}{\pgfqpoint{6.585476in}{3.608735in}}{\pgfqpoint{6.593290in}{3.616548in}}%
\pgfpathcurveto{\pgfqpoint{6.601103in}{3.624362in}}{\pgfqpoint{6.605493in}{3.634961in}}{\pgfqpoint{6.605493in}{3.646011in}}%
\pgfpathcurveto{\pgfqpoint{6.605493in}{3.657061in}}{\pgfqpoint{6.601103in}{3.667660in}}{\pgfqpoint{6.593290in}{3.675474in}}%
\pgfpathcurveto{\pgfqpoint{6.585476in}{3.683287in}}{\pgfqpoint{6.574877in}{3.687678in}}{\pgfqpoint{6.563827in}{3.687678in}}%
\pgfpathcurveto{\pgfqpoint{6.552777in}{3.687678in}}{\pgfqpoint{6.542178in}{3.683287in}}{\pgfqpoint{6.534364in}{3.675474in}}%
\pgfpathcurveto{\pgfqpoint{6.526550in}{3.667660in}}{\pgfqpoint{6.522160in}{3.657061in}}{\pgfqpoint{6.522160in}{3.646011in}}%
\pgfpathcurveto{\pgfqpoint{6.522160in}{3.634961in}}{\pgfqpoint{6.526550in}{3.624362in}}{\pgfqpoint{6.534364in}{3.616548in}}%
\pgfpathcurveto{\pgfqpoint{6.542178in}{3.608735in}}{\pgfqpoint{6.552777in}{3.604344in}}{\pgfqpoint{6.563827in}{3.604344in}}%
\pgfpathclose%
\pgfusepath{stroke,fill}%
\end{pgfscope}%
\begin{pgfscope}%
\pgfpathrectangle{\pgfqpoint{0.570343in}{0.331635in}}{\pgfqpoint{9.300000in}{7.700000in}}%
\pgfusepath{clip}%
\pgfsetbuttcap%
\pgfsetroundjoin%
\definecolor{currentfill}{rgb}{0.631373,0.788235,0.956863}%
\pgfsetfillcolor{currentfill}%
\pgfsetlinewidth{0.481800pt}%
\definecolor{currentstroke}{rgb}{1.000000,1.000000,1.000000}%
\pgfsetstrokecolor{currentstroke}%
\pgfsetdash{}{0pt}%
\pgfpathmoveto{\pgfqpoint{4.968812in}{5.867491in}}%
\pgfpathcurveto{\pgfqpoint{4.979862in}{5.867491in}}{\pgfqpoint{4.990461in}{5.871882in}}{\pgfqpoint{4.998275in}{5.879695in}}%
\pgfpathcurveto{\pgfqpoint{5.006088in}{5.887509in}}{\pgfqpoint{5.010479in}{5.898108in}}{\pgfqpoint{5.010479in}{5.909158in}}%
\pgfpathcurveto{\pgfqpoint{5.010479in}{5.920208in}}{\pgfqpoint{5.006088in}{5.930807in}}{\pgfqpoint{4.998275in}{5.938621in}}%
\pgfpathcurveto{\pgfqpoint{4.990461in}{5.946434in}}{\pgfqpoint{4.979862in}{5.950825in}}{\pgfqpoint{4.968812in}{5.950825in}}%
\pgfpathcurveto{\pgfqpoint{4.957762in}{5.950825in}}{\pgfqpoint{4.947163in}{5.946434in}}{\pgfqpoint{4.939349in}{5.938621in}}%
\pgfpathcurveto{\pgfqpoint{4.931535in}{5.930807in}}{\pgfqpoint{4.927145in}{5.920208in}}{\pgfqpoint{4.927145in}{5.909158in}}%
\pgfpathcurveto{\pgfqpoint{4.927145in}{5.898108in}}{\pgfqpoint{4.931535in}{5.887509in}}{\pgfqpoint{4.939349in}{5.879695in}}%
\pgfpathcurveto{\pgfqpoint{4.947163in}{5.871882in}}{\pgfqpoint{4.957762in}{5.867491in}}{\pgfqpoint{4.968812in}{5.867491in}}%
\pgfpathclose%
\pgfusepath{stroke,fill}%
\end{pgfscope}%
\begin{pgfscope}%
\pgfpathrectangle{\pgfqpoint{0.570343in}{0.331635in}}{\pgfqpoint{9.300000in}{7.700000in}}%
\pgfusepath{clip}%
\pgfsetbuttcap%
\pgfsetroundjoin%
\definecolor{currentfill}{rgb}{0.631373,0.788235,0.956863}%
\pgfsetfillcolor{currentfill}%
\pgfsetlinewidth{0.481800pt}%
\definecolor{currentstroke}{rgb}{1.000000,1.000000,1.000000}%
\pgfsetstrokecolor{currentstroke}%
\pgfsetdash{}{0pt}%
\pgfpathmoveto{\pgfqpoint{5.391700in}{1.625125in}}%
\pgfpathcurveto{\pgfqpoint{5.402750in}{1.625125in}}{\pgfqpoint{5.413349in}{1.629515in}}{\pgfqpoint{5.421162in}{1.637329in}}%
\pgfpathcurveto{\pgfqpoint{5.428976in}{1.645143in}}{\pgfqpoint{5.433366in}{1.655742in}}{\pgfqpoint{5.433366in}{1.666792in}}%
\pgfpathcurveto{\pgfqpoint{5.433366in}{1.677842in}}{\pgfqpoint{5.428976in}{1.688441in}}{\pgfqpoint{5.421162in}{1.696254in}}%
\pgfpathcurveto{\pgfqpoint{5.413349in}{1.704068in}}{\pgfqpoint{5.402750in}{1.708458in}}{\pgfqpoint{5.391700in}{1.708458in}}%
\pgfpathcurveto{\pgfqpoint{5.380650in}{1.708458in}}{\pgfqpoint{5.370051in}{1.704068in}}{\pgfqpoint{5.362237in}{1.696254in}}%
\pgfpathcurveto{\pgfqpoint{5.354423in}{1.688441in}}{\pgfqpoint{5.350033in}{1.677842in}}{\pgfqpoint{5.350033in}{1.666792in}}%
\pgfpathcurveto{\pgfqpoint{5.350033in}{1.655742in}}{\pgfqpoint{5.354423in}{1.645143in}}{\pgfqpoint{5.362237in}{1.637329in}}%
\pgfpathcurveto{\pgfqpoint{5.370051in}{1.629515in}}{\pgfqpoint{5.380650in}{1.625125in}}{\pgfqpoint{5.391700in}{1.625125in}}%
\pgfpathclose%
\pgfusepath{stroke,fill}%
\end{pgfscope}%
\begin{pgfscope}%
\pgfpathrectangle{\pgfqpoint{0.570343in}{0.331635in}}{\pgfqpoint{9.300000in}{7.700000in}}%
\pgfusepath{clip}%
\pgfsetbuttcap%
\pgfsetroundjoin%
\definecolor{currentfill}{rgb}{0.631373,0.788235,0.956863}%
\pgfsetfillcolor{currentfill}%
\pgfsetlinewidth{0.481800pt}%
\definecolor{currentstroke}{rgb}{1.000000,1.000000,1.000000}%
\pgfsetstrokecolor{currentstroke}%
\pgfsetdash{}{0pt}%
\pgfpathmoveto{\pgfqpoint{7.110525in}{6.170454in}}%
\pgfpathcurveto{\pgfqpoint{7.121575in}{6.170454in}}{\pgfqpoint{7.132174in}{6.174844in}}{\pgfqpoint{7.139988in}{6.182658in}}%
\pgfpathcurveto{\pgfqpoint{7.147801in}{6.190471in}}{\pgfqpoint{7.152192in}{6.201071in}}{\pgfqpoint{7.152192in}{6.212121in}}%
\pgfpathcurveto{\pgfqpoint{7.152192in}{6.223171in}}{\pgfqpoint{7.147801in}{6.233770in}}{\pgfqpoint{7.139988in}{6.241583in}}%
\pgfpathcurveto{\pgfqpoint{7.132174in}{6.249397in}}{\pgfqpoint{7.121575in}{6.253787in}}{\pgfqpoint{7.110525in}{6.253787in}}%
\pgfpathcurveto{\pgfqpoint{7.099475in}{6.253787in}}{\pgfqpoint{7.088876in}{6.249397in}}{\pgfqpoint{7.081062in}{6.241583in}}%
\pgfpathcurveto{\pgfqpoint{7.073249in}{6.233770in}}{\pgfqpoint{7.068858in}{6.223171in}}{\pgfqpoint{7.068858in}{6.212121in}}%
\pgfpathcurveto{\pgfqpoint{7.068858in}{6.201071in}}{\pgfqpoint{7.073249in}{6.190471in}}{\pgfqpoint{7.081062in}{6.182658in}}%
\pgfpathcurveto{\pgfqpoint{7.088876in}{6.174844in}}{\pgfqpoint{7.099475in}{6.170454in}}{\pgfqpoint{7.110525in}{6.170454in}}%
\pgfpathclose%
\pgfusepath{stroke,fill}%
\end{pgfscope}%
\begin{pgfscope}%
\pgfpathrectangle{\pgfqpoint{0.570343in}{0.331635in}}{\pgfqpoint{9.300000in}{7.700000in}}%
\pgfusepath{clip}%
\pgfsetbuttcap%
\pgfsetroundjoin%
\definecolor{currentfill}{rgb}{0.631373,0.788235,0.956863}%
\pgfsetfillcolor{currentfill}%
\pgfsetlinewidth{0.481800pt}%
\definecolor{currentstroke}{rgb}{1.000000,1.000000,1.000000}%
\pgfsetstrokecolor{currentstroke}%
\pgfsetdash{}{0pt}%
\pgfpathmoveto{\pgfqpoint{3.005187in}{2.523502in}}%
\pgfpathcurveto{\pgfqpoint{3.016237in}{2.523502in}}{\pgfqpoint{3.026836in}{2.527892in}}{\pgfqpoint{3.034650in}{2.535706in}}%
\pgfpathcurveto{\pgfqpoint{3.042463in}{2.543520in}}{\pgfqpoint{3.046854in}{2.554119in}}{\pgfqpoint{3.046854in}{2.565169in}}%
\pgfpathcurveto{\pgfqpoint{3.046854in}{2.576219in}}{\pgfqpoint{3.042463in}{2.586818in}}{\pgfqpoint{3.034650in}{2.594632in}}%
\pgfpathcurveto{\pgfqpoint{3.026836in}{2.602445in}}{\pgfqpoint{3.016237in}{2.606835in}}{\pgfqpoint{3.005187in}{2.606835in}}%
\pgfpathcurveto{\pgfqpoint{2.994137in}{2.606835in}}{\pgfqpoint{2.983538in}{2.602445in}}{\pgfqpoint{2.975724in}{2.594632in}}%
\pgfpathcurveto{\pgfqpoint{2.967911in}{2.586818in}}{\pgfqpoint{2.963520in}{2.576219in}}{\pgfqpoint{2.963520in}{2.565169in}}%
\pgfpathcurveto{\pgfqpoint{2.963520in}{2.554119in}}{\pgfqpoint{2.967911in}{2.543520in}}{\pgfqpoint{2.975724in}{2.535706in}}%
\pgfpathcurveto{\pgfqpoint{2.983538in}{2.527892in}}{\pgfqpoint{2.994137in}{2.523502in}}{\pgfqpoint{3.005187in}{2.523502in}}%
\pgfpathclose%
\pgfusepath{stroke,fill}%
\end{pgfscope}%
\begin{pgfscope}%
\pgfpathrectangle{\pgfqpoint{0.570343in}{0.331635in}}{\pgfqpoint{9.300000in}{7.700000in}}%
\pgfusepath{clip}%
\pgfsetbuttcap%
\pgfsetroundjoin%
\definecolor{currentfill}{rgb}{0.631373,0.788235,0.956863}%
\pgfsetfillcolor{currentfill}%
\pgfsetlinewidth{0.481800pt}%
\definecolor{currentstroke}{rgb}{1.000000,1.000000,1.000000}%
\pgfsetstrokecolor{currentstroke}%
\pgfsetdash{}{0pt}%
\pgfpathmoveto{\pgfqpoint{4.481129in}{4.356585in}}%
\pgfpathcurveto{\pgfqpoint{4.492179in}{4.356585in}}{\pgfqpoint{4.502778in}{4.360975in}}{\pgfqpoint{4.510592in}{4.368789in}}%
\pgfpathcurveto{\pgfqpoint{4.518405in}{4.376603in}}{\pgfqpoint{4.522796in}{4.387202in}}{\pgfqpoint{4.522796in}{4.398252in}}%
\pgfpathcurveto{\pgfqpoint{4.522796in}{4.409302in}}{\pgfqpoint{4.518405in}{4.419901in}}{\pgfqpoint{4.510592in}{4.427715in}}%
\pgfpathcurveto{\pgfqpoint{4.502778in}{4.435528in}}{\pgfqpoint{4.492179in}{4.439919in}}{\pgfqpoint{4.481129in}{4.439919in}}%
\pgfpathcurveto{\pgfqpoint{4.470079in}{4.439919in}}{\pgfqpoint{4.459480in}{4.435528in}}{\pgfqpoint{4.451666in}{4.427715in}}%
\pgfpathcurveto{\pgfqpoint{4.443853in}{4.419901in}}{\pgfqpoint{4.439462in}{4.409302in}}{\pgfqpoint{4.439462in}{4.398252in}}%
\pgfpathcurveto{\pgfqpoint{4.439462in}{4.387202in}}{\pgfqpoint{4.443853in}{4.376603in}}{\pgfqpoint{4.451666in}{4.368789in}}%
\pgfpathcurveto{\pgfqpoint{4.459480in}{4.360975in}}{\pgfqpoint{4.470079in}{4.356585in}}{\pgfqpoint{4.481129in}{4.356585in}}%
\pgfpathclose%
\pgfusepath{stroke,fill}%
\end{pgfscope}%
\begin{pgfscope}%
\pgfpathrectangle{\pgfqpoint{0.570343in}{0.331635in}}{\pgfqpoint{9.300000in}{7.700000in}}%
\pgfusepath{clip}%
\pgfsetbuttcap%
\pgfsetroundjoin%
\definecolor{currentfill}{rgb}{0.631373,0.788235,0.956863}%
\pgfsetfillcolor{currentfill}%
\pgfsetlinewidth{0.481800pt}%
\definecolor{currentstroke}{rgb}{1.000000,1.000000,1.000000}%
\pgfsetstrokecolor{currentstroke}%
\pgfsetdash{}{0pt}%
\pgfpathmoveto{\pgfqpoint{6.045238in}{6.374610in}}%
\pgfpathcurveto{\pgfqpoint{6.056288in}{6.374610in}}{\pgfqpoint{6.066887in}{6.379000in}}{\pgfqpoint{6.074700in}{6.386814in}}%
\pgfpathcurveto{\pgfqpoint{6.082514in}{6.394628in}}{\pgfqpoint{6.086904in}{6.405227in}}{\pgfqpoint{6.086904in}{6.416277in}}%
\pgfpathcurveto{\pgfqpoint{6.086904in}{6.427327in}}{\pgfqpoint{6.082514in}{6.437926in}}{\pgfqpoint{6.074700in}{6.445740in}}%
\pgfpathcurveto{\pgfqpoint{6.066887in}{6.453553in}}{\pgfqpoint{6.056288in}{6.457943in}}{\pgfqpoint{6.045238in}{6.457943in}}%
\pgfpathcurveto{\pgfqpoint{6.034188in}{6.457943in}}{\pgfqpoint{6.023588in}{6.453553in}}{\pgfqpoint{6.015775in}{6.445740in}}%
\pgfpathcurveto{\pgfqpoint{6.007961in}{6.437926in}}{\pgfqpoint{6.003571in}{6.427327in}}{\pgfqpoint{6.003571in}{6.416277in}}%
\pgfpathcurveto{\pgfqpoint{6.003571in}{6.405227in}}{\pgfqpoint{6.007961in}{6.394628in}}{\pgfqpoint{6.015775in}{6.386814in}}%
\pgfpathcurveto{\pgfqpoint{6.023588in}{6.379000in}}{\pgfqpoint{6.034188in}{6.374610in}}{\pgfqpoint{6.045238in}{6.374610in}}%
\pgfpathclose%
\pgfusepath{stroke,fill}%
\end{pgfscope}%
\begin{pgfscope}%
\pgfpathrectangle{\pgfqpoint{0.570343in}{0.331635in}}{\pgfqpoint{9.300000in}{7.700000in}}%
\pgfusepath{clip}%
\pgfsetbuttcap%
\pgfsetroundjoin%
\definecolor{currentfill}{rgb}{0.631373,0.788235,0.956863}%
\pgfsetfillcolor{currentfill}%
\pgfsetlinewidth{0.481800pt}%
\definecolor{currentstroke}{rgb}{1.000000,1.000000,1.000000}%
\pgfsetstrokecolor{currentstroke}%
\pgfsetdash{}{0pt}%
\pgfpathmoveto{\pgfqpoint{9.187938in}{4.053650in}}%
\pgfpathcurveto{\pgfqpoint{9.198989in}{4.053650in}}{\pgfqpoint{9.209588in}{4.058040in}}{\pgfqpoint{9.217401in}{4.065854in}}%
\pgfpathcurveto{\pgfqpoint{9.225215in}{4.073668in}}{\pgfqpoint{9.229605in}{4.084267in}}{\pgfqpoint{9.229605in}{4.095317in}}%
\pgfpathcurveto{\pgfqpoint{9.229605in}{4.106367in}}{\pgfqpoint{9.225215in}{4.116966in}}{\pgfqpoint{9.217401in}{4.124780in}}%
\pgfpathcurveto{\pgfqpoint{9.209588in}{4.132593in}}{\pgfqpoint{9.198989in}{4.136984in}}{\pgfqpoint{9.187938in}{4.136984in}}%
\pgfpathcurveto{\pgfqpoint{9.176888in}{4.136984in}}{\pgfqpoint{9.166289in}{4.132593in}}{\pgfqpoint{9.158476in}{4.124780in}}%
\pgfpathcurveto{\pgfqpoint{9.150662in}{4.116966in}}{\pgfqpoint{9.146272in}{4.106367in}}{\pgfqpoint{9.146272in}{4.095317in}}%
\pgfpathcurveto{\pgfqpoint{9.146272in}{4.084267in}}{\pgfqpoint{9.150662in}{4.073668in}}{\pgfqpoint{9.158476in}{4.065854in}}%
\pgfpathcurveto{\pgfqpoint{9.166289in}{4.058040in}}{\pgfqpoint{9.176888in}{4.053650in}}{\pgfqpoint{9.187938in}{4.053650in}}%
\pgfpathclose%
\pgfusepath{stroke,fill}%
\end{pgfscope}%
\begin{pgfscope}%
\pgfpathrectangle{\pgfqpoint{0.570343in}{0.331635in}}{\pgfqpoint{9.300000in}{7.700000in}}%
\pgfusepath{clip}%
\pgfsetbuttcap%
\pgfsetroundjoin%
\definecolor{currentfill}{rgb}{0.631373,0.788235,0.956863}%
\pgfsetfillcolor{currentfill}%
\pgfsetlinewidth{0.481800pt}%
\definecolor{currentstroke}{rgb}{1.000000,1.000000,1.000000}%
\pgfsetstrokecolor{currentstroke}%
\pgfsetdash{}{0pt}%
\pgfpathmoveto{\pgfqpoint{7.749803in}{4.456360in}}%
\pgfpathcurveto{\pgfqpoint{7.760853in}{4.456360in}}{\pgfqpoint{7.771452in}{4.460750in}}{\pgfqpoint{7.779265in}{4.468563in}}%
\pgfpathcurveto{\pgfqpoint{7.787079in}{4.476377in}}{\pgfqpoint{7.791469in}{4.486976in}}{\pgfqpoint{7.791469in}{4.498026in}}%
\pgfpathcurveto{\pgfqpoint{7.791469in}{4.509076in}}{\pgfqpoint{7.787079in}{4.519675in}}{\pgfqpoint{7.779265in}{4.527489in}}%
\pgfpathcurveto{\pgfqpoint{7.771452in}{4.535303in}}{\pgfqpoint{7.760853in}{4.539693in}}{\pgfqpoint{7.749803in}{4.539693in}}%
\pgfpathcurveto{\pgfqpoint{7.738753in}{4.539693in}}{\pgfqpoint{7.728154in}{4.535303in}}{\pgfqpoint{7.720340in}{4.527489in}}%
\pgfpathcurveto{\pgfqpoint{7.712526in}{4.519675in}}{\pgfqpoint{7.708136in}{4.509076in}}{\pgfqpoint{7.708136in}{4.498026in}}%
\pgfpathcurveto{\pgfqpoint{7.708136in}{4.486976in}}{\pgfqpoint{7.712526in}{4.476377in}}{\pgfqpoint{7.720340in}{4.468563in}}%
\pgfpathcurveto{\pgfqpoint{7.728154in}{4.460750in}}{\pgfqpoint{7.738753in}{4.456360in}}{\pgfqpoint{7.749803in}{4.456360in}}%
\pgfpathclose%
\pgfusepath{stroke,fill}%
\end{pgfscope}%
\begin{pgfscope}%
\pgfpathrectangle{\pgfqpoint{0.570343in}{0.331635in}}{\pgfqpoint{9.300000in}{7.700000in}}%
\pgfusepath{clip}%
\pgfsetbuttcap%
\pgfsetroundjoin%
\definecolor{currentfill}{rgb}{1.000000,0.705882,0.509804}%
\pgfsetfillcolor{currentfill}%
\pgfsetlinewidth{0.481800pt}%
\definecolor{currentstroke}{rgb}{1.000000,1.000000,1.000000}%
\pgfsetstrokecolor{currentstroke}%
\pgfsetdash{}{0pt}%
\pgfpathmoveto{\pgfqpoint{5.715735in}{3.294039in}}%
\pgfpathcurveto{\pgfqpoint{5.726785in}{3.294039in}}{\pgfqpoint{5.737384in}{3.298429in}}{\pgfqpoint{5.745198in}{3.306243in}}%
\pgfpathcurveto{\pgfqpoint{5.753012in}{3.314057in}}{\pgfqpoint{5.757402in}{3.324656in}}{\pgfqpoint{5.757402in}{3.335706in}}%
\pgfpathcurveto{\pgfqpoint{5.757402in}{3.346756in}}{\pgfqpoint{5.753012in}{3.357355in}}{\pgfqpoint{5.745198in}{3.365169in}}%
\pgfpathcurveto{\pgfqpoint{5.737384in}{3.372982in}}{\pgfqpoint{5.726785in}{3.377372in}}{\pgfqpoint{5.715735in}{3.377372in}}%
\pgfpathcurveto{\pgfqpoint{5.704685in}{3.377372in}}{\pgfqpoint{5.694086in}{3.372982in}}{\pgfqpoint{5.686272in}{3.365169in}}%
\pgfpathcurveto{\pgfqpoint{5.678459in}{3.357355in}}{\pgfqpoint{5.674069in}{3.346756in}}{\pgfqpoint{5.674069in}{3.335706in}}%
\pgfpathcurveto{\pgfqpoint{5.674069in}{3.324656in}}{\pgfqpoint{5.678459in}{3.314057in}}{\pgfqpoint{5.686272in}{3.306243in}}%
\pgfpathcurveto{\pgfqpoint{5.694086in}{3.298429in}}{\pgfqpoint{5.704685in}{3.294039in}}{\pgfqpoint{5.715735in}{3.294039in}}%
\pgfpathclose%
\pgfusepath{stroke,fill}%
\end{pgfscope}%
\begin{pgfscope}%
\pgfpathrectangle{\pgfqpoint{0.570343in}{0.331635in}}{\pgfqpoint{9.300000in}{7.700000in}}%
\pgfusepath{clip}%
\pgfsetbuttcap%
\pgfsetroundjoin%
\definecolor{currentfill}{rgb}{1.000000,0.705882,0.509804}%
\pgfsetfillcolor{currentfill}%
\pgfsetlinewidth{0.481800pt}%
\definecolor{currentstroke}{rgb}{1.000000,1.000000,1.000000}%
\pgfsetstrokecolor{currentstroke}%
\pgfsetdash{}{0pt}%
\pgfpathmoveto{\pgfqpoint{7.824153in}{5.921475in}}%
\pgfpathcurveto{\pgfqpoint{7.835204in}{5.921475in}}{\pgfqpoint{7.845803in}{5.925865in}}{\pgfqpoint{7.853616in}{5.933679in}}%
\pgfpathcurveto{\pgfqpoint{7.861430in}{5.941492in}}{\pgfqpoint{7.865820in}{5.952091in}}{\pgfqpoint{7.865820in}{5.963141in}}%
\pgfpathcurveto{\pgfqpoint{7.865820in}{5.974192in}}{\pgfqpoint{7.861430in}{5.984791in}}{\pgfqpoint{7.853616in}{5.992604in}}%
\pgfpathcurveto{\pgfqpoint{7.845803in}{6.000418in}}{\pgfqpoint{7.835204in}{6.004808in}}{\pgfqpoint{7.824153in}{6.004808in}}%
\pgfpathcurveto{\pgfqpoint{7.813103in}{6.004808in}}{\pgfqpoint{7.802504in}{6.000418in}}{\pgfqpoint{7.794691in}{5.992604in}}%
\pgfpathcurveto{\pgfqpoint{7.786877in}{5.984791in}}{\pgfqpoint{7.782487in}{5.974192in}}{\pgfqpoint{7.782487in}{5.963141in}}%
\pgfpathcurveto{\pgfqpoint{7.782487in}{5.952091in}}{\pgfqpoint{7.786877in}{5.941492in}}{\pgfqpoint{7.794691in}{5.933679in}}%
\pgfpathcurveto{\pgfqpoint{7.802504in}{5.925865in}}{\pgfqpoint{7.813103in}{5.921475in}}{\pgfqpoint{7.824153in}{5.921475in}}%
\pgfpathclose%
\pgfusepath{stroke,fill}%
\end{pgfscope}%
\begin{pgfscope}%
\pgfpathrectangle{\pgfqpoint{0.570343in}{0.331635in}}{\pgfqpoint{9.300000in}{7.700000in}}%
\pgfusepath{clip}%
\pgfsetbuttcap%
\pgfsetroundjoin%
\definecolor{currentfill}{rgb}{1.000000,0.705882,0.509804}%
\pgfsetfillcolor{currentfill}%
\pgfsetlinewidth{0.481800pt}%
\definecolor{currentstroke}{rgb}{1.000000,1.000000,1.000000}%
\pgfsetstrokecolor{currentstroke}%
\pgfsetdash{}{0pt}%
\pgfpathmoveto{\pgfqpoint{5.894391in}{4.089252in}}%
\pgfpathcurveto{\pgfqpoint{5.905441in}{4.089252in}}{\pgfqpoint{5.916040in}{4.093643in}}{\pgfqpoint{5.923854in}{4.101456in}}%
\pgfpathcurveto{\pgfqpoint{5.931667in}{4.109270in}}{\pgfqpoint{5.936057in}{4.119869in}}{\pgfqpoint{5.936057in}{4.130919in}}%
\pgfpathcurveto{\pgfqpoint{5.936057in}{4.141969in}}{\pgfqpoint{5.931667in}{4.152568in}}{\pgfqpoint{5.923854in}{4.160382in}}%
\pgfpathcurveto{\pgfqpoint{5.916040in}{4.168195in}}{\pgfqpoint{5.905441in}{4.172586in}}{\pgfqpoint{5.894391in}{4.172586in}}%
\pgfpathcurveto{\pgfqpoint{5.883341in}{4.172586in}}{\pgfqpoint{5.872742in}{4.168195in}}{\pgfqpoint{5.864928in}{4.160382in}}%
\pgfpathcurveto{\pgfqpoint{5.857114in}{4.152568in}}{\pgfqpoint{5.852724in}{4.141969in}}{\pgfqpoint{5.852724in}{4.130919in}}%
\pgfpathcurveto{\pgfqpoint{5.852724in}{4.119869in}}{\pgfqpoint{5.857114in}{4.109270in}}{\pgfqpoint{5.864928in}{4.101456in}}%
\pgfpathcurveto{\pgfqpoint{5.872742in}{4.093643in}}{\pgfqpoint{5.883341in}{4.089252in}}{\pgfqpoint{5.894391in}{4.089252in}}%
\pgfpathclose%
\pgfusepath{stroke,fill}%
\end{pgfscope}%
\begin{pgfscope}%
\pgfpathrectangle{\pgfqpoint{0.570343in}{0.331635in}}{\pgfqpoint{9.300000in}{7.700000in}}%
\pgfusepath{clip}%
\pgfsetbuttcap%
\pgfsetroundjoin%
\definecolor{currentfill}{rgb}{1.000000,0.705882,0.509804}%
\pgfsetfillcolor{currentfill}%
\pgfsetlinewidth{0.481800pt}%
\definecolor{currentstroke}{rgb}{1.000000,1.000000,1.000000}%
\pgfsetstrokecolor{currentstroke}%
\pgfsetdash{}{0pt}%
\pgfpathmoveto{\pgfqpoint{3.273962in}{4.188781in}}%
\pgfpathcurveto{\pgfqpoint{3.285012in}{4.188781in}}{\pgfqpoint{3.295611in}{4.193172in}}{\pgfqpoint{3.303425in}{4.200985in}}%
\pgfpathcurveto{\pgfqpoint{3.311239in}{4.208799in}}{\pgfqpoint{3.315629in}{4.219398in}}{\pgfqpoint{3.315629in}{4.230448in}}%
\pgfpathcurveto{\pgfqpoint{3.315629in}{4.241498in}}{\pgfqpoint{3.311239in}{4.252097in}}{\pgfqpoint{3.303425in}{4.259911in}}%
\pgfpathcurveto{\pgfqpoint{3.295611in}{4.267724in}}{\pgfqpoint{3.285012in}{4.272115in}}{\pgfqpoint{3.273962in}{4.272115in}}%
\pgfpathcurveto{\pgfqpoint{3.262912in}{4.272115in}}{\pgfqpoint{3.252313in}{4.267724in}}{\pgfqpoint{3.244499in}{4.259911in}}%
\pgfpathcurveto{\pgfqpoint{3.236686in}{4.252097in}}{\pgfqpoint{3.232295in}{4.241498in}}{\pgfqpoint{3.232295in}{4.230448in}}%
\pgfpathcurveto{\pgfqpoint{3.232295in}{4.219398in}}{\pgfqpoint{3.236686in}{4.208799in}}{\pgfqpoint{3.244499in}{4.200985in}}%
\pgfpathcurveto{\pgfqpoint{3.252313in}{4.193172in}}{\pgfqpoint{3.262912in}{4.188781in}}{\pgfqpoint{3.273962in}{4.188781in}}%
\pgfpathclose%
\pgfusepath{stroke,fill}%
\end{pgfscope}%
\begin{pgfscope}%
\pgfpathrectangle{\pgfqpoint{0.570343in}{0.331635in}}{\pgfqpoint{9.300000in}{7.700000in}}%
\pgfusepath{clip}%
\pgfsetbuttcap%
\pgfsetroundjoin%
\definecolor{currentfill}{rgb}{1.000000,0.705882,0.509804}%
\pgfsetfillcolor{currentfill}%
\pgfsetlinewidth{0.481800pt}%
\definecolor{currentstroke}{rgb}{1.000000,1.000000,1.000000}%
\pgfsetstrokecolor{currentstroke}%
\pgfsetdash{}{0pt}%
\pgfpathmoveto{\pgfqpoint{4.485603in}{2.482347in}}%
\pgfpathcurveto{\pgfqpoint{4.496653in}{2.482347in}}{\pgfqpoint{4.507252in}{2.486737in}}{\pgfqpoint{4.515066in}{2.494551in}}%
\pgfpathcurveto{\pgfqpoint{4.522879in}{2.502365in}}{\pgfqpoint{4.527269in}{2.512964in}}{\pgfqpoint{4.527269in}{2.524014in}}%
\pgfpathcurveto{\pgfqpoint{4.527269in}{2.535064in}}{\pgfqpoint{4.522879in}{2.545663in}}{\pgfqpoint{4.515066in}{2.553477in}}%
\pgfpathcurveto{\pgfqpoint{4.507252in}{2.561290in}}{\pgfqpoint{4.496653in}{2.565680in}}{\pgfqpoint{4.485603in}{2.565680in}}%
\pgfpathcurveto{\pgfqpoint{4.474553in}{2.565680in}}{\pgfqpoint{4.463954in}{2.561290in}}{\pgfqpoint{4.456140in}{2.553477in}}%
\pgfpathcurveto{\pgfqpoint{4.448326in}{2.545663in}}{\pgfqpoint{4.443936in}{2.535064in}}{\pgfqpoint{4.443936in}{2.524014in}}%
\pgfpathcurveto{\pgfqpoint{4.443936in}{2.512964in}}{\pgfqpoint{4.448326in}{2.502365in}}{\pgfqpoint{4.456140in}{2.494551in}}%
\pgfpathcurveto{\pgfqpoint{4.463954in}{2.486737in}}{\pgfqpoint{4.474553in}{2.482347in}}{\pgfqpoint{4.485603in}{2.482347in}}%
\pgfpathclose%
\pgfusepath{stroke,fill}%
\end{pgfscope}%
\begin{pgfscope}%
\pgfpathrectangle{\pgfqpoint{0.570343in}{0.331635in}}{\pgfqpoint{9.300000in}{7.700000in}}%
\pgfusepath{clip}%
\pgfsetbuttcap%
\pgfsetroundjoin%
\definecolor{currentfill}{rgb}{1.000000,0.705882,0.509804}%
\pgfsetfillcolor{currentfill}%
\pgfsetlinewidth{0.481800pt}%
\definecolor{currentstroke}{rgb}{1.000000,1.000000,1.000000}%
\pgfsetstrokecolor{currentstroke}%
\pgfsetdash{}{0pt}%
\pgfpathmoveto{\pgfqpoint{6.572664in}{5.197620in}}%
\pgfpathcurveto{\pgfqpoint{6.583714in}{5.197620in}}{\pgfqpoint{6.594313in}{5.202010in}}{\pgfqpoint{6.602127in}{5.209824in}}%
\pgfpathcurveto{\pgfqpoint{6.609940in}{5.217637in}}{\pgfqpoint{6.614330in}{5.228236in}}{\pgfqpoint{6.614330in}{5.239286in}}%
\pgfpathcurveto{\pgfqpoint{6.614330in}{5.250336in}}{\pgfqpoint{6.609940in}{5.260936in}}{\pgfqpoint{6.602127in}{5.268749in}}%
\pgfpathcurveto{\pgfqpoint{6.594313in}{5.276563in}}{\pgfqpoint{6.583714in}{5.280953in}}{\pgfqpoint{6.572664in}{5.280953in}}%
\pgfpathcurveto{\pgfqpoint{6.561614in}{5.280953in}}{\pgfqpoint{6.551015in}{5.276563in}}{\pgfqpoint{6.543201in}{5.268749in}}%
\pgfpathcurveto{\pgfqpoint{6.535387in}{5.260936in}}{\pgfqpoint{6.530997in}{5.250336in}}{\pgfqpoint{6.530997in}{5.239286in}}%
\pgfpathcurveto{\pgfqpoint{6.530997in}{5.228236in}}{\pgfqpoint{6.535387in}{5.217637in}}{\pgfqpoint{6.543201in}{5.209824in}}%
\pgfpathcurveto{\pgfqpoint{6.551015in}{5.202010in}}{\pgfqpoint{6.561614in}{5.197620in}}{\pgfqpoint{6.572664in}{5.197620in}}%
\pgfpathclose%
\pgfusepath{stroke,fill}%
\end{pgfscope}%
\begin{pgfscope}%
\pgfpathrectangle{\pgfqpoint{0.570343in}{0.331635in}}{\pgfqpoint{9.300000in}{7.700000in}}%
\pgfusepath{clip}%
\pgfsetbuttcap%
\pgfsetroundjoin%
\definecolor{currentfill}{rgb}{1.000000,0.705882,0.509804}%
\pgfsetfillcolor{currentfill}%
\pgfsetlinewidth{0.481800pt}%
\definecolor{currentstroke}{rgb}{1.000000,1.000000,1.000000}%
\pgfsetstrokecolor{currentstroke}%
\pgfsetdash{}{0pt}%
\pgfpathmoveto{\pgfqpoint{7.581009in}{1.719780in}}%
\pgfpathcurveto{\pgfqpoint{7.592059in}{1.719780in}}{\pgfqpoint{7.602658in}{1.724170in}}{\pgfqpoint{7.610472in}{1.731984in}}%
\pgfpathcurveto{\pgfqpoint{7.618285in}{1.739797in}}{\pgfqpoint{7.622676in}{1.750396in}}{\pgfqpoint{7.622676in}{1.761446in}}%
\pgfpathcurveto{\pgfqpoint{7.622676in}{1.772497in}}{\pgfqpoint{7.618285in}{1.783096in}}{\pgfqpoint{7.610472in}{1.790909in}}%
\pgfpathcurveto{\pgfqpoint{7.602658in}{1.798723in}}{\pgfqpoint{7.592059in}{1.803113in}}{\pgfqpoint{7.581009in}{1.803113in}}%
\pgfpathcurveto{\pgfqpoint{7.569959in}{1.803113in}}{\pgfqpoint{7.559360in}{1.798723in}}{\pgfqpoint{7.551546in}{1.790909in}}%
\pgfpathcurveto{\pgfqpoint{7.543732in}{1.783096in}}{\pgfqpoint{7.539342in}{1.772497in}}{\pgfqpoint{7.539342in}{1.761446in}}%
\pgfpathcurveto{\pgfqpoint{7.539342in}{1.750396in}}{\pgfqpoint{7.543732in}{1.739797in}}{\pgfqpoint{7.551546in}{1.731984in}}%
\pgfpathcurveto{\pgfqpoint{7.559360in}{1.724170in}}{\pgfqpoint{7.569959in}{1.719780in}}{\pgfqpoint{7.581009in}{1.719780in}}%
\pgfpathclose%
\pgfusepath{stroke,fill}%
\end{pgfscope}%
\begin{pgfscope}%
\pgfpathrectangle{\pgfqpoint{0.570343in}{0.331635in}}{\pgfqpoint{9.300000in}{7.700000in}}%
\pgfusepath{clip}%
\pgfsetbuttcap%
\pgfsetroundjoin%
\definecolor{currentfill}{rgb}{1.000000,0.705882,0.509804}%
\pgfsetfillcolor{currentfill}%
\pgfsetlinewidth{0.481800pt}%
\definecolor{currentstroke}{rgb}{1.000000,1.000000,1.000000}%
\pgfsetstrokecolor{currentstroke}%
\pgfsetdash{}{0pt}%
\pgfpathmoveto{\pgfqpoint{1.757394in}{6.484893in}}%
\pgfpathcurveto{\pgfqpoint{1.768445in}{6.484893in}}{\pgfqpoint{1.779044in}{6.489284in}}{\pgfqpoint{1.786857in}{6.497097in}}%
\pgfpathcurveto{\pgfqpoint{1.794671in}{6.504911in}}{\pgfqpoint{1.799061in}{6.515510in}}{\pgfqpoint{1.799061in}{6.526560in}}%
\pgfpathcurveto{\pgfqpoint{1.799061in}{6.537610in}}{\pgfqpoint{1.794671in}{6.548209in}}{\pgfqpoint{1.786857in}{6.556023in}}%
\pgfpathcurveto{\pgfqpoint{1.779044in}{6.563836in}}{\pgfqpoint{1.768445in}{6.568227in}}{\pgfqpoint{1.757394in}{6.568227in}}%
\pgfpathcurveto{\pgfqpoint{1.746344in}{6.568227in}}{\pgfqpoint{1.735745in}{6.563836in}}{\pgfqpoint{1.727932in}{6.556023in}}%
\pgfpathcurveto{\pgfqpoint{1.720118in}{6.548209in}}{\pgfqpoint{1.715728in}{6.537610in}}{\pgfqpoint{1.715728in}{6.526560in}}%
\pgfpathcurveto{\pgfqpoint{1.715728in}{6.515510in}}{\pgfqpoint{1.720118in}{6.504911in}}{\pgfqpoint{1.727932in}{6.497097in}}%
\pgfpathcurveto{\pgfqpoint{1.735745in}{6.489284in}}{\pgfqpoint{1.746344in}{6.484893in}}{\pgfqpoint{1.757394in}{6.484893in}}%
\pgfpathclose%
\pgfusepath{stroke,fill}%
\end{pgfscope}%
\begin{pgfscope}%
\pgfpathrectangle{\pgfqpoint{0.570343in}{0.331635in}}{\pgfqpoint{9.300000in}{7.700000in}}%
\pgfusepath{clip}%
\pgfsetbuttcap%
\pgfsetroundjoin%
\definecolor{currentfill}{rgb}{1.000000,0.705882,0.509804}%
\pgfsetfillcolor{currentfill}%
\pgfsetlinewidth{0.481800pt}%
\definecolor{currentstroke}{rgb}{1.000000,1.000000,1.000000}%
\pgfsetstrokecolor{currentstroke}%
\pgfsetdash{}{0pt}%
\pgfpathmoveto{\pgfqpoint{5.468750in}{7.308381in}}%
\pgfpathcurveto{\pgfqpoint{5.479800in}{7.308381in}}{\pgfqpoint{5.490399in}{7.312772in}}{\pgfqpoint{5.498213in}{7.320585in}}%
\pgfpathcurveto{\pgfqpoint{5.506027in}{7.328399in}}{\pgfqpoint{5.510417in}{7.338998in}}{\pgfqpoint{5.510417in}{7.350048in}}%
\pgfpathcurveto{\pgfqpoint{5.510417in}{7.361098in}}{\pgfqpoint{5.506027in}{7.371697in}}{\pgfqpoint{5.498213in}{7.379511in}}%
\pgfpathcurveto{\pgfqpoint{5.490399in}{7.387324in}}{\pgfqpoint{5.479800in}{7.391715in}}{\pgfqpoint{5.468750in}{7.391715in}}%
\pgfpathcurveto{\pgfqpoint{5.457700in}{7.391715in}}{\pgfqpoint{5.447101in}{7.387324in}}{\pgfqpoint{5.439287in}{7.379511in}}%
\pgfpathcurveto{\pgfqpoint{5.431474in}{7.371697in}}{\pgfqpoint{5.427084in}{7.361098in}}{\pgfqpoint{5.427084in}{7.350048in}}%
\pgfpathcurveto{\pgfqpoint{5.427084in}{7.338998in}}{\pgfqpoint{5.431474in}{7.328399in}}{\pgfqpoint{5.439287in}{7.320585in}}%
\pgfpathcurveto{\pgfqpoint{5.447101in}{7.312772in}}{\pgfqpoint{5.457700in}{7.308381in}}{\pgfqpoint{5.468750in}{7.308381in}}%
\pgfpathclose%
\pgfusepath{stroke,fill}%
\end{pgfscope}%
\begin{pgfscope}%
\pgfpathrectangle{\pgfqpoint{0.570343in}{0.331635in}}{\pgfqpoint{9.300000in}{7.700000in}}%
\pgfusepath{clip}%
\pgfsetbuttcap%
\pgfsetroundjoin%
\definecolor{currentfill}{rgb}{1.000000,0.705882,0.509804}%
\pgfsetfillcolor{currentfill}%
\pgfsetlinewidth{0.481800pt}%
\definecolor{currentstroke}{rgb}{1.000000,1.000000,1.000000}%
\pgfsetstrokecolor{currentstroke}%
\pgfsetdash{}{0pt}%
\pgfpathmoveto{\pgfqpoint{4.136121in}{6.363161in}}%
\pgfpathcurveto{\pgfqpoint{4.147171in}{6.363161in}}{\pgfqpoint{4.157770in}{6.367551in}}{\pgfqpoint{4.165584in}{6.375365in}}%
\pgfpathcurveto{\pgfqpoint{4.173398in}{6.383178in}}{\pgfqpoint{4.177788in}{6.393777in}}{\pgfqpoint{4.177788in}{6.404828in}}%
\pgfpathcurveto{\pgfqpoint{4.177788in}{6.415878in}}{\pgfqpoint{4.173398in}{6.426477in}}{\pgfqpoint{4.165584in}{6.434290in}}%
\pgfpathcurveto{\pgfqpoint{4.157770in}{6.442104in}}{\pgfqpoint{4.147171in}{6.446494in}}{\pgfqpoint{4.136121in}{6.446494in}}%
\pgfpathcurveto{\pgfqpoint{4.125071in}{6.446494in}}{\pgfqpoint{4.114472in}{6.442104in}}{\pgfqpoint{4.106658in}{6.434290in}}%
\pgfpathcurveto{\pgfqpoint{4.098845in}{6.426477in}}{\pgfqpoint{4.094454in}{6.415878in}}{\pgfqpoint{4.094454in}{6.404828in}}%
\pgfpathcurveto{\pgfqpoint{4.094454in}{6.393777in}}{\pgfqpoint{4.098845in}{6.383178in}}{\pgfqpoint{4.106658in}{6.375365in}}%
\pgfpathcurveto{\pgfqpoint{4.114472in}{6.367551in}}{\pgfqpoint{4.125071in}{6.363161in}}{\pgfqpoint{4.136121in}{6.363161in}}%
\pgfpathclose%
\pgfusepath{stroke,fill}%
\end{pgfscope}%
\begin{pgfscope}%
\pgfpathrectangle{\pgfqpoint{0.570343in}{0.331635in}}{\pgfqpoint{9.300000in}{7.700000in}}%
\pgfusepath{clip}%
\pgfsetbuttcap%
\pgfsetroundjoin%
\definecolor{currentfill}{rgb}{1.000000,0.705882,0.509804}%
\pgfsetfillcolor{currentfill}%
\pgfsetlinewidth{0.481800pt}%
\definecolor{currentstroke}{rgb}{1.000000,1.000000,1.000000}%
\pgfsetstrokecolor{currentstroke}%
\pgfsetdash{}{0pt}%
\pgfpathmoveto{\pgfqpoint{3.063964in}{3.535815in}}%
\pgfpathcurveto{\pgfqpoint{3.075014in}{3.535815in}}{\pgfqpoint{3.085613in}{3.540206in}}{\pgfqpoint{3.093427in}{3.548019in}}%
\pgfpathcurveto{\pgfqpoint{3.101241in}{3.555833in}}{\pgfqpoint{3.105631in}{3.566432in}}{\pgfqpoint{3.105631in}{3.577482in}}%
\pgfpathcurveto{\pgfqpoint{3.105631in}{3.588532in}}{\pgfqpoint{3.101241in}{3.599131in}}{\pgfqpoint{3.093427in}{3.606945in}}%
\pgfpathcurveto{\pgfqpoint{3.085613in}{3.614758in}}{\pgfqpoint{3.075014in}{3.619149in}}{\pgfqpoint{3.063964in}{3.619149in}}%
\pgfpathcurveto{\pgfqpoint{3.052914in}{3.619149in}}{\pgfqpoint{3.042315in}{3.614758in}}{\pgfqpoint{3.034501in}{3.606945in}}%
\pgfpathcurveto{\pgfqpoint{3.026688in}{3.599131in}}{\pgfqpoint{3.022297in}{3.588532in}}{\pgfqpoint{3.022297in}{3.577482in}}%
\pgfpathcurveto{\pgfqpoint{3.022297in}{3.566432in}}{\pgfqpoint{3.026688in}{3.555833in}}{\pgfqpoint{3.034501in}{3.548019in}}%
\pgfpathcurveto{\pgfqpoint{3.042315in}{3.540206in}}{\pgfqpoint{3.052914in}{3.535815in}}{\pgfqpoint{3.063964in}{3.535815in}}%
\pgfpathclose%
\pgfusepath{stroke,fill}%
\end{pgfscope}%
\begin{pgfscope}%
\pgfpathrectangle{\pgfqpoint{0.570343in}{0.331635in}}{\pgfqpoint{9.300000in}{7.700000in}}%
\pgfusepath{clip}%
\pgfsetbuttcap%
\pgfsetroundjoin%
\definecolor{currentfill}{rgb}{1.000000,0.705882,0.509804}%
\pgfsetfillcolor{currentfill}%
\pgfsetlinewidth{0.481800pt}%
\definecolor{currentstroke}{rgb}{1.000000,1.000000,1.000000}%
\pgfsetstrokecolor{currentstroke}%
\pgfsetdash{}{0pt}%
\pgfpathmoveto{\pgfqpoint{3.378790in}{1.183747in}}%
\pgfpathcurveto{\pgfqpoint{3.389840in}{1.183747in}}{\pgfqpoint{3.400439in}{1.188137in}}{\pgfqpoint{3.408252in}{1.195951in}}%
\pgfpathcurveto{\pgfqpoint{3.416066in}{1.203765in}}{\pgfqpoint{3.420456in}{1.214364in}}{\pgfqpoint{3.420456in}{1.225414in}}%
\pgfpathcurveto{\pgfqpoint{3.420456in}{1.236464in}}{\pgfqpoint{3.416066in}{1.247063in}}{\pgfqpoint{3.408252in}{1.254877in}}%
\pgfpathcurveto{\pgfqpoint{3.400439in}{1.262690in}}{\pgfqpoint{3.389840in}{1.267080in}}{\pgfqpoint{3.378790in}{1.267080in}}%
\pgfpathcurveto{\pgfqpoint{3.367740in}{1.267080in}}{\pgfqpoint{3.357140in}{1.262690in}}{\pgfqpoint{3.349327in}{1.254877in}}%
\pgfpathcurveto{\pgfqpoint{3.341513in}{1.247063in}}{\pgfqpoint{3.337123in}{1.236464in}}{\pgfqpoint{3.337123in}{1.225414in}}%
\pgfpathcurveto{\pgfqpoint{3.337123in}{1.214364in}}{\pgfqpoint{3.341513in}{1.203765in}}{\pgfqpoint{3.349327in}{1.195951in}}%
\pgfpathcurveto{\pgfqpoint{3.357140in}{1.188137in}}{\pgfqpoint{3.367740in}{1.183747in}}{\pgfqpoint{3.378790in}{1.183747in}}%
\pgfpathclose%
\pgfusepath{stroke,fill}%
\end{pgfscope}%
\begin{pgfscope}%
\pgfpathrectangle{\pgfqpoint{0.570343in}{0.331635in}}{\pgfqpoint{9.300000in}{7.700000in}}%
\pgfusepath{clip}%
\pgfsetbuttcap%
\pgfsetroundjoin%
\definecolor{currentfill}{rgb}{1.000000,0.705882,0.509804}%
\pgfsetfillcolor{currentfill}%
\pgfsetlinewidth{0.481800pt}%
\definecolor{currentstroke}{rgb}{1.000000,1.000000,1.000000}%
\pgfsetstrokecolor{currentstroke}%
\pgfsetdash{}{0pt}%
\pgfpathmoveto{\pgfqpoint{0.993071in}{1.736886in}}%
\pgfpathcurveto{\pgfqpoint{1.004121in}{1.736886in}}{\pgfqpoint{1.014720in}{1.741276in}}{\pgfqpoint{1.022533in}{1.749090in}}%
\pgfpathcurveto{\pgfqpoint{1.030347in}{1.756903in}}{\pgfqpoint{1.034737in}{1.767502in}}{\pgfqpoint{1.034737in}{1.778552in}}%
\pgfpathcurveto{\pgfqpoint{1.034737in}{1.789603in}}{\pgfqpoint{1.030347in}{1.800202in}}{\pgfqpoint{1.022533in}{1.808015in}}%
\pgfpathcurveto{\pgfqpoint{1.014720in}{1.815829in}}{\pgfqpoint{1.004121in}{1.820219in}}{\pgfqpoint{0.993071in}{1.820219in}}%
\pgfpathcurveto{\pgfqpoint{0.982020in}{1.820219in}}{\pgfqpoint{0.971421in}{1.815829in}}{\pgfqpoint{0.963608in}{1.808015in}}%
\pgfpathcurveto{\pgfqpoint{0.955794in}{1.800202in}}{\pgfqpoint{0.951404in}{1.789603in}}{\pgfqpoint{0.951404in}{1.778552in}}%
\pgfpathcurveto{\pgfqpoint{0.951404in}{1.767502in}}{\pgfqpoint{0.955794in}{1.756903in}}{\pgfqpoint{0.963608in}{1.749090in}}%
\pgfpathcurveto{\pgfqpoint{0.971421in}{1.741276in}}{\pgfqpoint{0.982020in}{1.736886in}}{\pgfqpoint{0.993071in}{1.736886in}}%
\pgfpathclose%
\pgfusepath{stroke,fill}%
\end{pgfscope}%
\begin{pgfscope}%
\pgfpathrectangle{\pgfqpoint{0.570343in}{0.331635in}}{\pgfqpoint{9.300000in}{7.700000in}}%
\pgfusepath{clip}%
\pgfsetbuttcap%
\pgfsetroundjoin%
\definecolor{currentfill}{rgb}{1.000000,0.705882,0.509804}%
\pgfsetfillcolor{currentfill}%
\pgfsetlinewidth{0.481800pt}%
\definecolor{currentstroke}{rgb}{1.000000,1.000000,1.000000}%
\pgfsetstrokecolor{currentstroke}%
\pgfsetdash{}{0pt}%
\pgfpathmoveto{\pgfqpoint{5.153229in}{2.615822in}}%
\pgfpathcurveto{\pgfqpoint{5.164279in}{2.615822in}}{\pgfqpoint{5.174878in}{2.620213in}}{\pgfqpoint{5.182691in}{2.628026in}}%
\pgfpathcurveto{\pgfqpoint{5.190505in}{2.635840in}}{\pgfqpoint{5.194895in}{2.646439in}}{\pgfqpoint{5.194895in}{2.657489in}}%
\pgfpathcurveto{\pgfqpoint{5.194895in}{2.668539in}}{\pgfqpoint{5.190505in}{2.679138in}}{\pgfqpoint{5.182691in}{2.686952in}}%
\pgfpathcurveto{\pgfqpoint{5.174878in}{2.694765in}}{\pgfqpoint{5.164279in}{2.699156in}}{\pgfqpoint{5.153229in}{2.699156in}}%
\pgfpathcurveto{\pgfqpoint{5.142178in}{2.699156in}}{\pgfqpoint{5.131579in}{2.694765in}}{\pgfqpoint{5.123766in}{2.686952in}}%
\pgfpathcurveto{\pgfqpoint{5.115952in}{2.679138in}}{\pgfqpoint{5.111562in}{2.668539in}}{\pgfqpoint{5.111562in}{2.657489in}}%
\pgfpathcurveto{\pgfqpoint{5.111562in}{2.646439in}}{\pgfqpoint{5.115952in}{2.635840in}}{\pgfqpoint{5.123766in}{2.628026in}}%
\pgfpathcurveto{\pgfqpoint{5.131579in}{2.620213in}}{\pgfqpoint{5.142178in}{2.615822in}}{\pgfqpoint{5.153229in}{2.615822in}}%
\pgfpathclose%
\pgfusepath{stroke,fill}%
\end{pgfscope}%
\begin{pgfscope}%
\pgfpathrectangle{\pgfqpoint{0.570343in}{0.331635in}}{\pgfqpoint{9.300000in}{7.700000in}}%
\pgfusepath{clip}%
\pgfsetbuttcap%
\pgfsetroundjoin%
\definecolor{currentfill}{rgb}{1.000000,0.705882,0.509804}%
\pgfsetfillcolor{currentfill}%
\pgfsetlinewidth{0.481800pt}%
\definecolor{currentstroke}{rgb}{1.000000,1.000000,1.000000}%
\pgfsetstrokecolor{currentstroke}%
\pgfsetdash{}{0pt}%
\pgfpathmoveto{\pgfqpoint{8.561752in}{3.062614in}}%
\pgfpathcurveto{\pgfqpoint{8.572802in}{3.062614in}}{\pgfqpoint{8.583401in}{3.067004in}}{\pgfqpoint{8.591215in}{3.074818in}}%
\pgfpathcurveto{\pgfqpoint{8.599028in}{3.082632in}}{\pgfqpoint{8.603419in}{3.093231in}}{\pgfqpoint{8.603419in}{3.104281in}}%
\pgfpathcurveto{\pgfqpoint{8.603419in}{3.115331in}}{\pgfqpoint{8.599028in}{3.125930in}}{\pgfqpoint{8.591215in}{3.133744in}}%
\pgfpathcurveto{\pgfqpoint{8.583401in}{3.141557in}}{\pgfqpoint{8.572802in}{3.145947in}}{\pgfqpoint{8.561752in}{3.145947in}}%
\pgfpathcurveto{\pgfqpoint{8.550702in}{3.145947in}}{\pgfqpoint{8.540103in}{3.141557in}}{\pgfqpoint{8.532289in}{3.133744in}}%
\pgfpathcurveto{\pgfqpoint{8.524476in}{3.125930in}}{\pgfqpoint{8.520085in}{3.115331in}}{\pgfqpoint{8.520085in}{3.104281in}}%
\pgfpathcurveto{\pgfqpoint{8.520085in}{3.093231in}}{\pgfqpoint{8.524476in}{3.082632in}}{\pgfqpoint{8.532289in}{3.074818in}}%
\pgfpathcurveto{\pgfqpoint{8.540103in}{3.067004in}}{\pgfqpoint{8.550702in}{3.062614in}}{\pgfqpoint{8.561752in}{3.062614in}}%
\pgfpathclose%
\pgfusepath{stroke,fill}%
\end{pgfscope}%
\begin{pgfscope}%
\pgfpathrectangle{\pgfqpoint{0.570343in}{0.331635in}}{\pgfqpoint{9.300000in}{7.700000in}}%
\pgfusepath{clip}%
\pgfsetbuttcap%
\pgfsetroundjoin%
\definecolor{currentfill}{rgb}{1.000000,0.705882,0.509804}%
\pgfsetfillcolor{currentfill}%
\pgfsetlinewidth{0.481800pt}%
\definecolor{currentstroke}{rgb}{1.000000,1.000000,1.000000}%
\pgfsetstrokecolor{currentstroke}%
\pgfsetdash{}{0pt}%
\pgfpathmoveto{\pgfqpoint{4.047696in}{1.828593in}}%
\pgfpathcurveto{\pgfqpoint{4.058746in}{1.828593in}}{\pgfqpoint{4.069345in}{1.832983in}}{\pgfqpoint{4.077158in}{1.840797in}}%
\pgfpathcurveto{\pgfqpoint{4.084972in}{1.848611in}}{\pgfqpoint{4.089362in}{1.859210in}}{\pgfqpoint{4.089362in}{1.870260in}}%
\pgfpathcurveto{\pgfqpoint{4.089362in}{1.881310in}}{\pgfqpoint{4.084972in}{1.891909in}}{\pgfqpoint{4.077158in}{1.899723in}}%
\pgfpathcurveto{\pgfqpoint{4.069345in}{1.907536in}}{\pgfqpoint{4.058746in}{1.911927in}}{\pgfqpoint{4.047696in}{1.911927in}}%
\pgfpathcurveto{\pgfqpoint{4.036646in}{1.911927in}}{\pgfqpoint{4.026046in}{1.907536in}}{\pgfqpoint{4.018233in}{1.899723in}}%
\pgfpathcurveto{\pgfqpoint{4.010419in}{1.891909in}}{\pgfqpoint{4.006029in}{1.881310in}}{\pgfqpoint{4.006029in}{1.870260in}}%
\pgfpathcurveto{\pgfqpoint{4.006029in}{1.859210in}}{\pgfqpoint{4.010419in}{1.848611in}}{\pgfqpoint{4.018233in}{1.840797in}}%
\pgfpathcurveto{\pgfqpoint{4.026046in}{1.832983in}}{\pgfqpoint{4.036646in}{1.828593in}}{\pgfqpoint{4.047696in}{1.828593in}}%
\pgfpathclose%
\pgfusepath{stroke,fill}%
\end{pgfscope}%
\begin{pgfscope}%
\pgfpathrectangle{\pgfqpoint{0.570343in}{0.331635in}}{\pgfqpoint{9.300000in}{7.700000in}}%
\pgfusepath{clip}%
\pgfsetbuttcap%
\pgfsetroundjoin%
\definecolor{currentfill}{rgb}{1.000000,0.705882,0.509804}%
\pgfsetfillcolor{currentfill}%
\pgfsetlinewidth{0.481800pt}%
\definecolor{currentstroke}{rgb}{1.000000,1.000000,1.000000}%
\pgfsetstrokecolor{currentstroke}%
\pgfsetdash{}{0pt}%
\pgfpathmoveto{\pgfqpoint{8.305186in}{3.919496in}}%
\pgfpathcurveto{\pgfqpoint{8.316236in}{3.919496in}}{\pgfqpoint{8.326835in}{3.923887in}}{\pgfqpoint{8.334648in}{3.931700in}}%
\pgfpathcurveto{\pgfqpoint{8.342462in}{3.939514in}}{\pgfqpoint{8.346852in}{3.950113in}}{\pgfqpoint{8.346852in}{3.961163in}}%
\pgfpathcurveto{\pgfqpoint{8.346852in}{3.972213in}}{\pgfqpoint{8.342462in}{3.982812in}}{\pgfqpoint{8.334648in}{3.990626in}}%
\pgfpathcurveto{\pgfqpoint{8.326835in}{3.998439in}}{\pgfqpoint{8.316236in}{4.002830in}}{\pgfqpoint{8.305186in}{4.002830in}}%
\pgfpathcurveto{\pgfqpoint{8.294136in}{4.002830in}}{\pgfqpoint{8.283537in}{3.998439in}}{\pgfqpoint{8.275723in}{3.990626in}}%
\pgfpathcurveto{\pgfqpoint{8.267909in}{3.982812in}}{\pgfqpoint{8.263519in}{3.972213in}}{\pgfqpoint{8.263519in}{3.961163in}}%
\pgfpathcurveto{\pgfqpoint{8.263519in}{3.950113in}}{\pgfqpoint{8.267909in}{3.939514in}}{\pgfqpoint{8.275723in}{3.931700in}}%
\pgfpathcurveto{\pgfqpoint{8.283537in}{3.923887in}}{\pgfqpoint{8.294136in}{3.919496in}}{\pgfqpoint{8.305186in}{3.919496in}}%
\pgfpathclose%
\pgfusepath{stroke,fill}%
\end{pgfscope}%
\begin{pgfscope}%
\pgfpathrectangle{\pgfqpoint{0.570343in}{0.331635in}}{\pgfqpoint{9.300000in}{7.700000in}}%
\pgfusepath{clip}%
\pgfsetbuttcap%
\pgfsetroundjoin%
\definecolor{currentfill}{rgb}{1.000000,0.705882,0.509804}%
\pgfsetfillcolor{currentfill}%
\pgfsetlinewidth{0.481800pt}%
\definecolor{currentstroke}{rgb}{1.000000,1.000000,1.000000}%
\pgfsetstrokecolor{currentstroke}%
\pgfsetdash{}{0pt}%
\pgfpathmoveto{\pgfqpoint{4.619077in}{0.639968in}}%
\pgfpathcurveto{\pgfqpoint{4.630127in}{0.639968in}}{\pgfqpoint{4.640726in}{0.644359in}}{\pgfqpoint{4.648539in}{0.652172in}}%
\pgfpathcurveto{\pgfqpoint{4.656353in}{0.659986in}}{\pgfqpoint{4.660743in}{0.670585in}}{\pgfqpoint{4.660743in}{0.681635in}}%
\pgfpathcurveto{\pgfqpoint{4.660743in}{0.692685in}}{\pgfqpoint{4.656353in}{0.703284in}}{\pgfqpoint{4.648539in}{0.711098in}}%
\pgfpathcurveto{\pgfqpoint{4.640726in}{0.718911in}}{\pgfqpoint{4.630127in}{0.723302in}}{\pgfqpoint{4.619077in}{0.723302in}}%
\pgfpathcurveto{\pgfqpoint{4.608026in}{0.723302in}}{\pgfqpoint{4.597427in}{0.718911in}}{\pgfqpoint{4.589614in}{0.711098in}}%
\pgfpathcurveto{\pgfqpoint{4.581800in}{0.703284in}}{\pgfqpoint{4.577410in}{0.692685in}}{\pgfqpoint{4.577410in}{0.681635in}}%
\pgfpathcurveto{\pgfqpoint{4.577410in}{0.670585in}}{\pgfqpoint{4.581800in}{0.659986in}}{\pgfqpoint{4.589614in}{0.652172in}}%
\pgfpathcurveto{\pgfqpoint{4.597427in}{0.644359in}}{\pgfqpoint{4.608026in}{0.639968in}}{\pgfqpoint{4.619077in}{0.639968in}}%
\pgfpathclose%
\pgfusepath{stroke,fill}%
\end{pgfscope}%
\begin{pgfscope}%
\pgfpathrectangle{\pgfqpoint{0.570343in}{0.331635in}}{\pgfqpoint{9.300000in}{7.700000in}}%
\pgfusepath{clip}%
\pgfsetbuttcap%
\pgfsetroundjoin%
\definecolor{currentfill}{rgb}{1.000000,0.705882,0.509804}%
\pgfsetfillcolor{currentfill}%
\pgfsetlinewidth{0.481800pt}%
\definecolor{currentstroke}{rgb}{1.000000,1.000000,1.000000}%
\pgfsetstrokecolor{currentstroke}%
\pgfsetdash{}{0pt}%
\pgfpathmoveto{\pgfqpoint{8.308279in}{7.639968in}}%
\pgfpathcurveto{\pgfqpoint{8.319329in}{7.639968in}}{\pgfqpoint{8.329928in}{7.644359in}}{\pgfqpoint{8.337742in}{7.652172in}}%
\pgfpathcurveto{\pgfqpoint{8.345555in}{7.659986in}}{\pgfqpoint{8.349945in}{7.670585in}}{\pgfqpoint{8.349945in}{7.681635in}}%
\pgfpathcurveto{\pgfqpoint{8.349945in}{7.692685in}}{\pgfqpoint{8.345555in}{7.703284in}}{\pgfqpoint{8.337742in}{7.711098in}}%
\pgfpathcurveto{\pgfqpoint{8.329928in}{7.718911in}}{\pgfqpoint{8.319329in}{7.723302in}}{\pgfqpoint{8.308279in}{7.723302in}}%
\pgfpathcurveto{\pgfqpoint{8.297229in}{7.723302in}}{\pgfqpoint{8.286630in}{7.718911in}}{\pgfqpoint{8.278816in}{7.711098in}}%
\pgfpathcurveto{\pgfqpoint{8.271002in}{7.703284in}}{\pgfqpoint{8.266612in}{7.692685in}}{\pgfqpoint{8.266612in}{7.681635in}}%
\pgfpathcurveto{\pgfqpoint{8.266612in}{7.670585in}}{\pgfqpoint{8.271002in}{7.659986in}}{\pgfqpoint{8.278816in}{7.652172in}}%
\pgfpathcurveto{\pgfqpoint{8.286630in}{7.644359in}}{\pgfqpoint{8.297229in}{7.639968in}}{\pgfqpoint{8.308279in}{7.639968in}}%
\pgfpathclose%
\pgfusepath{stroke,fill}%
\end{pgfscope}%
\begin{pgfscope}%
\pgfpathrectangle{\pgfqpoint{0.570343in}{0.331635in}}{\pgfqpoint{9.300000in}{7.700000in}}%
\pgfusepath{clip}%
\pgfsetbuttcap%
\pgfsetroundjoin%
\definecolor{currentfill}{rgb}{1.000000,0.705882,0.509804}%
\pgfsetfillcolor{currentfill}%
\pgfsetlinewidth{0.481800pt}%
\definecolor{currentstroke}{rgb}{1.000000,1.000000,1.000000}%
\pgfsetstrokecolor{currentstroke}%
\pgfsetdash{}{0pt}%
\pgfpathmoveto{\pgfqpoint{6.197001in}{1.217501in}}%
\pgfpathcurveto{\pgfqpoint{6.208051in}{1.217501in}}{\pgfqpoint{6.218650in}{1.221891in}}{\pgfqpoint{6.226463in}{1.229705in}}%
\pgfpathcurveto{\pgfqpoint{6.234277in}{1.237518in}}{\pgfqpoint{6.238667in}{1.248117in}}{\pgfqpoint{6.238667in}{1.259168in}}%
\pgfpathcurveto{\pgfqpoint{6.238667in}{1.270218in}}{\pgfqpoint{6.234277in}{1.280817in}}{\pgfqpoint{6.226463in}{1.288630in}}%
\pgfpathcurveto{\pgfqpoint{6.218650in}{1.296444in}}{\pgfqpoint{6.208051in}{1.300834in}}{\pgfqpoint{6.197001in}{1.300834in}}%
\pgfpathcurveto{\pgfqpoint{6.185951in}{1.300834in}}{\pgfqpoint{6.175352in}{1.296444in}}{\pgfqpoint{6.167538in}{1.288630in}}%
\pgfpathcurveto{\pgfqpoint{6.159724in}{1.280817in}}{\pgfqpoint{6.155334in}{1.270218in}}{\pgfqpoint{6.155334in}{1.259168in}}%
\pgfpathcurveto{\pgfqpoint{6.155334in}{1.248117in}}{\pgfqpoint{6.159724in}{1.237518in}}{\pgfqpoint{6.167538in}{1.229705in}}%
\pgfpathcurveto{\pgfqpoint{6.175352in}{1.221891in}}{\pgfqpoint{6.185951in}{1.217501in}}{\pgfqpoint{6.197001in}{1.217501in}}%
\pgfpathclose%
\pgfusepath{stroke,fill}%
\end{pgfscope}%
\begin{pgfscope}%
\pgfpathrectangle{\pgfqpoint{0.570343in}{0.331635in}}{\pgfqpoint{9.300000in}{7.700000in}}%
\pgfusepath{clip}%
\pgfsetbuttcap%
\pgfsetroundjoin%
\definecolor{currentfill}{rgb}{1.000000,0.705882,0.509804}%
\pgfsetfillcolor{currentfill}%
\pgfsetlinewidth{0.481800pt}%
\definecolor{currentstroke}{rgb}{1.000000,1.000000,1.000000}%
\pgfsetstrokecolor{currentstroke}%
\pgfsetdash{}{0pt}%
\pgfpathmoveto{\pgfqpoint{9.447616in}{5.083458in}}%
\pgfpathcurveto{\pgfqpoint{9.458666in}{5.083458in}}{\pgfqpoint{9.469265in}{5.087849in}}{\pgfqpoint{9.477079in}{5.095662in}}%
\pgfpathcurveto{\pgfqpoint{9.484892in}{5.103476in}}{\pgfqpoint{9.489283in}{5.114075in}}{\pgfqpoint{9.489283in}{5.125125in}}%
\pgfpathcurveto{\pgfqpoint{9.489283in}{5.136175in}}{\pgfqpoint{9.484892in}{5.146774in}}{\pgfqpoint{9.477079in}{5.154588in}}%
\pgfpathcurveto{\pgfqpoint{9.469265in}{5.162402in}}{\pgfqpoint{9.458666in}{5.166792in}}{\pgfqpoint{9.447616in}{5.166792in}}%
\pgfpathcurveto{\pgfqpoint{9.436566in}{5.166792in}}{\pgfqpoint{9.425967in}{5.162402in}}{\pgfqpoint{9.418153in}{5.154588in}}%
\pgfpathcurveto{\pgfqpoint{9.410340in}{5.146774in}}{\pgfqpoint{9.405949in}{5.136175in}}{\pgfqpoint{9.405949in}{5.125125in}}%
\pgfpathcurveto{\pgfqpoint{9.405949in}{5.114075in}}{\pgfqpoint{9.410340in}{5.103476in}}{\pgfqpoint{9.418153in}{5.095662in}}%
\pgfpathcurveto{\pgfqpoint{9.425967in}{5.087849in}}{\pgfqpoint{9.436566in}{5.083458in}}{\pgfqpoint{9.447616in}{5.083458in}}%
\pgfpathclose%
\pgfusepath{stroke,fill}%
\end{pgfscope}%
\begin{pgfscope}%
\pgfpathrectangle{\pgfqpoint{0.570343in}{0.331635in}}{\pgfqpoint{9.300000in}{7.700000in}}%
\pgfusepath{clip}%
\pgfsetbuttcap%
\pgfsetroundjoin%
\definecolor{currentfill}{rgb}{1.000000,0.705882,0.509804}%
\pgfsetfillcolor{currentfill}%
\pgfsetlinewidth{0.481800pt}%
\definecolor{currentstroke}{rgb}{1.000000,1.000000,1.000000}%
\pgfsetstrokecolor{currentstroke}%
\pgfsetdash{}{0pt}%
\pgfpathmoveto{\pgfqpoint{6.034006in}{2.397376in}}%
\pgfpathcurveto{\pgfqpoint{6.045056in}{2.397376in}}{\pgfqpoint{6.055655in}{2.401766in}}{\pgfqpoint{6.063468in}{2.409579in}}%
\pgfpathcurveto{\pgfqpoint{6.071282in}{2.417393in}}{\pgfqpoint{6.075672in}{2.427992in}}{\pgfqpoint{6.075672in}{2.439042in}}%
\pgfpathcurveto{\pgfqpoint{6.075672in}{2.450092in}}{\pgfqpoint{6.071282in}{2.460691in}}{\pgfqpoint{6.063468in}{2.468505in}}%
\pgfpathcurveto{\pgfqpoint{6.055655in}{2.476319in}}{\pgfqpoint{6.045056in}{2.480709in}}{\pgfqpoint{6.034006in}{2.480709in}}%
\pgfpathcurveto{\pgfqpoint{6.022955in}{2.480709in}}{\pgfqpoint{6.012356in}{2.476319in}}{\pgfqpoint{6.004543in}{2.468505in}}%
\pgfpathcurveto{\pgfqpoint{5.996729in}{2.460691in}}{\pgfqpoint{5.992339in}{2.450092in}}{\pgfqpoint{5.992339in}{2.439042in}}%
\pgfpathcurveto{\pgfqpoint{5.992339in}{2.427992in}}{\pgfqpoint{5.996729in}{2.417393in}}{\pgfqpoint{6.004543in}{2.409579in}}%
\pgfpathcurveto{\pgfqpoint{6.012356in}{2.401766in}}{\pgfqpoint{6.022955in}{2.397376in}}{\pgfqpoint{6.034006in}{2.397376in}}%
\pgfpathclose%
\pgfusepath{stroke,fill}%
\end{pgfscope}%
\begin{pgfscope}%
\pgfpathrectangle{\pgfqpoint{0.570343in}{0.331635in}}{\pgfqpoint{9.300000in}{7.700000in}}%
\pgfusepath{clip}%
\pgfsetbuttcap%
\pgfsetroundjoin%
\definecolor{currentfill}{rgb}{1.000000,0.705882,0.509804}%
\pgfsetfillcolor{currentfill}%
\pgfsetlinewidth{0.481800pt}%
\definecolor{currentstroke}{rgb}{1.000000,1.000000,1.000000}%
\pgfsetstrokecolor{currentstroke}%
\pgfsetdash{}{0pt}%
\pgfpathmoveto{\pgfqpoint{8.815623in}{2.157538in}}%
\pgfpathcurveto{\pgfqpoint{8.826674in}{2.157538in}}{\pgfqpoint{8.837273in}{2.161928in}}{\pgfqpoint{8.845086in}{2.169741in}}%
\pgfpathcurveto{\pgfqpoint{8.852900in}{2.177555in}}{\pgfqpoint{8.857290in}{2.188154in}}{\pgfqpoint{8.857290in}{2.199204in}}%
\pgfpathcurveto{\pgfqpoint{8.857290in}{2.210254in}}{\pgfqpoint{8.852900in}{2.220853in}}{\pgfqpoint{8.845086in}{2.228667in}}%
\pgfpathcurveto{\pgfqpoint{8.837273in}{2.236481in}}{\pgfqpoint{8.826674in}{2.240871in}}{\pgfqpoint{8.815623in}{2.240871in}}%
\pgfpathcurveto{\pgfqpoint{8.804573in}{2.240871in}}{\pgfqpoint{8.793974in}{2.236481in}}{\pgfqpoint{8.786161in}{2.228667in}}%
\pgfpathcurveto{\pgfqpoint{8.778347in}{2.220853in}}{\pgfqpoint{8.773957in}{2.210254in}}{\pgfqpoint{8.773957in}{2.199204in}}%
\pgfpathcurveto{\pgfqpoint{8.773957in}{2.188154in}}{\pgfqpoint{8.778347in}{2.177555in}}{\pgfqpoint{8.786161in}{2.169741in}}%
\pgfpathcurveto{\pgfqpoint{8.793974in}{2.161928in}}{\pgfqpoint{8.804573in}{2.157538in}}{\pgfqpoint{8.815623in}{2.157538in}}%
\pgfpathclose%
\pgfusepath{stroke,fill}%
\end{pgfscope}%
\begin{pgfscope}%
\pgfpathrectangle{\pgfqpoint{0.570343in}{0.331635in}}{\pgfqpoint{9.300000in}{7.700000in}}%
\pgfusepath{clip}%
\pgfsetbuttcap%
\pgfsetroundjoin%
\definecolor{currentfill}{rgb}{1.000000,0.705882,0.509804}%
\pgfsetfillcolor{currentfill}%
\pgfsetlinewidth{0.481800pt}%
\definecolor{currentstroke}{rgb}{1.000000,1.000000,1.000000}%
\pgfsetstrokecolor{currentstroke}%
\pgfsetdash{}{0pt}%
\pgfpathmoveto{\pgfqpoint{7.488123in}{3.678053in}}%
\pgfpathcurveto{\pgfqpoint{7.499173in}{3.678053in}}{\pgfqpoint{7.509772in}{3.682443in}}{\pgfqpoint{7.517586in}{3.690257in}}%
\pgfpathcurveto{\pgfqpoint{7.525399in}{3.698070in}}{\pgfqpoint{7.529790in}{3.708669in}}{\pgfqpoint{7.529790in}{3.719720in}}%
\pgfpathcurveto{\pgfqpoint{7.529790in}{3.730770in}}{\pgfqpoint{7.525399in}{3.741369in}}{\pgfqpoint{7.517586in}{3.749182in}}%
\pgfpathcurveto{\pgfqpoint{7.509772in}{3.756996in}}{\pgfqpoint{7.499173in}{3.761386in}}{\pgfqpoint{7.488123in}{3.761386in}}%
\pgfpathcurveto{\pgfqpoint{7.477073in}{3.761386in}}{\pgfqpoint{7.466474in}{3.756996in}}{\pgfqpoint{7.458660in}{3.749182in}}%
\pgfpathcurveto{\pgfqpoint{7.450847in}{3.741369in}}{\pgfqpoint{7.446456in}{3.730770in}}{\pgfqpoint{7.446456in}{3.719720in}}%
\pgfpathcurveto{\pgfqpoint{7.446456in}{3.708669in}}{\pgfqpoint{7.450847in}{3.698070in}}{\pgfqpoint{7.458660in}{3.690257in}}%
\pgfpathcurveto{\pgfqpoint{7.466474in}{3.682443in}}{\pgfqpoint{7.477073in}{3.678053in}}{\pgfqpoint{7.488123in}{3.678053in}}%
\pgfpathclose%
\pgfusepath{stroke,fill}%
\end{pgfscope}%
\begin{pgfscope}%
\pgfpathrectangle{\pgfqpoint{0.570343in}{0.331635in}}{\pgfqpoint{9.300000in}{7.700000in}}%
\pgfusepath{clip}%
\pgfsetbuttcap%
\pgfsetroundjoin%
\definecolor{currentfill}{rgb}{1.000000,0.705882,0.509804}%
\pgfsetfillcolor{currentfill}%
\pgfsetlinewidth{0.481800pt}%
\definecolor{currentstroke}{rgb}{1.000000,1.000000,1.000000}%
\pgfsetstrokecolor{currentstroke}%
\pgfsetdash{}{0pt}%
\pgfpathmoveto{\pgfqpoint{2.671621in}{4.803306in}}%
\pgfpathcurveto{\pgfqpoint{2.682671in}{4.803306in}}{\pgfqpoint{2.693270in}{4.807696in}}{\pgfqpoint{2.701084in}{4.815510in}}%
\pgfpathcurveto{\pgfqpoint{2.708898in}{4.823324in}}{\pgfqpoint{2.713288in}{4.833923in}}{\pgfqpoint{2.713288in}{4.844973in}}%
\pgfpathcurveto{\pgfqpoint{2.713288in}{4.856023in}}{\pgfqpoint{2.708898in}{4.866622in}}{\pgfqpoint{2.701084in}{4.874436in}}%
\pgfpathcurveto{\pgfqpoint{2.693270in}{4.882249in}}{\pgfqpoint{2.682671in}{4.886640in}}{\pgfqpoint{2.671621in}{4.886640in}}%
\pgfpathcurveto{\pgfqpoint{2.660571in}{4.886640in}}{\pgfqpoint{2.649972in}{4.882249in}}{\pgfqpoint{2.642159in}{4.874436in}}%
\pgfpathcurveto{\pgfqpoint{2.634345in}{4.866622in}}{\pgfqpoint{2.629955in}{4.856023in}}{\pgfqpoint{2.629955in}{4.844973in}}%
\pgfpathcurveto{\pgfqpoint{2.629955in}{4.833923in}}{\pgfqpoint{2.634345in}{4.823324in}}{\pgfqpoint{2.642159in}{4.815510in}}%
\pgfpathcurveto{\pgfqpoint{2.649972in}{4.807696in}}{\pgfqpoint{2.660571in}{4.803306in}}{\pgfqpoint{2.671621in}{4.803306in}}%
\pgfpathclose%
\pgfusepath{stroke,fill}%
\end{pgfscope}%
\begin{pgfscope}%
\pgfpathrectangle{\pgfqpoint{0.570343in}{0.331635in}}{\pgfqpoint{9.300000in}{7.700000in}}%
\pgfusepath{clip}%
\pgfsetbuttcap%
\pgfsetroundjoin%
\definecolor{currentfill}{rgb}{1.000000,0.705882,0.509804}%
\pgfsetfillcolor{currentfill}%
\pgfsetlinewidth{0.481800pt}%
\definecolor{currentstroke}{rgb}{1.000000,1.000000,1.000000}%
\pgfsetstrokecolor{currentstroke}%
\pgfsetdash{}{0pt}%
\pgfpathmoveto{\pgfqpoint{7.154996in}{6.987530in}}%
\pgfpathcurveto{\pgfqpoint{7.166046in}{6.987530in}}{\pgfqpoint{7.176645in}{6.991920in}}{\pgfqpoint{7.184459in}{6.999733in}}%
\pgfpathcurveto{\pgfqpoint{7.192272in}{7.007547in}}{\pgfqpoint{7.196663in}{7.018146in}}{\pgfqpoint{7.196663in}{7.029196in}}%
\pgfpathcurveto{\pgfqpoint{7.196663in}{7.040246in}}{\pgfqpoint{7.192272in}{7.050845in}}{\pgfqpoint{7.184459in}{7.058659in}}%
\pgfpathcurveto{\pgfqpoint{7.176645in}{7.066473in}}{\pgfqpoint{7.166046in}{7.070863in}}{\pgfqpoint{7.154996in}{7.070863in}}%
\pgfpathcurveto{\pgfqpoint{7.143946in}{7.070863in}}{\pgfqpoint{7.133347in}{7.066473in}}{\pgfqpoint{7.125533in}{7.058659in}}%
\pgfpathcurveto{\pgfqpoint{7.117720in}{7.050845in}}{\pgfqpoint{7.113329in}{7.040246in}}{\pgfqpoint{7.113329in}{7.029196in}}%
\pgfpathcurveto{\pgfqpoint{7.113329in}{7.018146in}}{\pgfqpoint{7.117720in}{7.007547in}}{\pgfqpoint{7.125533in}{6.999733in}}%
\pgfpathcurveto{\pgfqpoint{7.133347in}{6.991920in}}{\pgfqpoint{7.143946in}{6.987530in}}{\pgfqpoint{7.154996in}{6.987530in}}%
\pgfpathclose%
\pgfusepath{stroke,fill}%
\end{pgfscope}%
\begin{pgfscope}%
\pgfpathrectangle{\pgfqpoint{0.570343in}{0.331635in}}{\pgfqpoint{9.300000in}{7.700000in}}%
\pgfusepath{clip}%
\pgfsetbuttcap%
\pgfsetroundjoin%
\definecolor{currentfill}{rgb}{1.000000,0.705882,0.509804}%
\pgfsetfillcolor{currentfill}%
\pgfsetlinewidth{0.481800pt}%
\definecolor{currentstroke}{rgb}{1.000000,1.000000,1.000000}%
\pgfsetstrokecolor{currentstroke}%
\pgfsetdash{}{0pt}%
\pgfpathmoveto{\pgfqpoint{3.897376in}{2.961996in}}%
\pgfpathcurveto{\pgfqpoint{3.908426in}{2.961996in}}{\pgfqpoint{3.919025in}{2.966387in}}{\pgfqpoint{3.926838in}{2.974200in}}%
\pgfpathcurveto{\pgfqpoint{3.934652in}{2.982014in}}{\pgfqpoint{3.939042in}{2.992613in}}{\pgfqpoint{3.939042in}{3.003663in}}%
\pgfpathcurveto{\pgfqpoint{3.939042in}{3.014713in}}{\pgfqpoint{3.934652in}{3.025312in}}{\pgfqpoint{3.926838in}{3.033126in}}%
\pgfpathcurveto{\pgfqpoint{3.919025in}{3.040939in}}{\pgfqpoint{3.908426in}{3.045330in}}{\pgfqpoint{3.897376in}{3.045330in}}%
\pgfpathcurveto{\pgfqpoint{3.886326in}{3.045330in}}{\pgfqpoint{3.875726in}{3.040939in}}{\pgfqpoint{3.867913in}{3.033126in}}%
\pgfpathcurveto{\pgfqpoint{3.860099in}{3.025312in}}{\pgfqpoint{3.855709in}{3.014713in}}{\pgfqpoint{3.855709in}{3.003663in}}%
\pgfpathcurveto{\pgfqpoint{3.855709in}{2.992613in}}{\pgfqpoint{3.860099in}{2.982014in}}{\pgfqpoint{3.867913in}{2.974200in}}%
\pgfpathcurveto{\pgfqpoint{3.875726in}{2.966387in}}{\pgfqpoint{3.886326in}{2.961996in}}{\pgfqpoint{3.897376in}{2.961996in}}%
\pgfpathclose%
\pgfusepath{stroke,fill}%
\end{pgfscope}%
\begin{pgfscope}%
\pgfpathrectangle{\pgfqpoint{0.570343in}{0.331635in}}{\pgfqpoint{9.300000in}{7.700000in}}%
\pgfusepath{clip}%
\pgfsetbuttcap%
\pgfsetroundjoin%
\definecolor{currentfill}{rgb}{1.000000,0.705882,0.509804}%
\pgfsetfillcolor{currentfill}%
\pgfsetlinewidth{0.481800pt}%
\definecolor{currentstroke}{rgb}{1.000000,1.000000,1.000000}%
\pgfsetstrokecolor{currentstroke}%
\pgfsetdash{}{0pt}%
\pgfpathmoveto{\pgfqpoint{5.935607in}{5.697947in}}%
\pgfpathcurveto{\pgfqpoint{5.946657in}{5.697947in}}{\pgfqpoint{5.957256in}{5.702338in}}{\pgfqpoint{5.965070in}{5.710151in}}%
\pgfpathcurveto{\pgfqpoint{5.972883in}{5.717965in}}{\pgfqpoint{5.977273in}{5.728564in}}{\pgfqpoint{5.977273in}{5.739614in}}%
\pgfpathcurveto{\pgfqpoint{5.977273in}{5.750664in}}{\pgfqpoint{5.972883in}{5.761263in}}{\pgfqpoint{5.965070in}{5.769077in}}%
\pgfpathcurveto{\pgfqpoint{5.957256in}{5.776891in}}{\pgfqpoint{5.946657in}{5.781281in}}{\pgfqpoint{5.935607in}{5.781281in}}%
\pgfpathcurveto{\pgfqpoint{5.924557in}{5.781281in}}{\pgfqpoint{5.913958in}{5.776891in}}{\pgfqpoint{5.906144in}{5.769077in}}%
\pgfpathcurveto{\pgfqpoint{5.898330in}{5.761263in}}{\pgfqpoint{5.893940in}{5.750664in}}{\pgfqpoint{5.893940in}{5.739614in}}%
\pgfpathcurveto{\pgfqpoint{5.893940in}{5.728564in}}{\pgfqpoint{5.898330in}{5.717965in}}{\pgfqpoint{5.906144in}{5.710151in}}%
\pgfpathcurveto{\pgfqpoint{5.913958in}{5.702338in}}{\pgfqpoint{5.924557in}{5.697947in}}{\pgfqpoint{5.935607in}{5.697947in}}%
\pgfpathclose%
\pgfusepath{stroke,fill}%
\end{pgfscope}%
\begin{pgfscope}%
\pgfpathrectangle{\pgfqpoint{0.570343in}{0.331635in}}{\pgfqpoint{9.300000in}{7.700000in}}%
\pgfusepath{clip}%
\pgfsetbuttcap%
\pgfsetroundjoin%
\definecolor{currentfill}{rgb}{0.631373,0.788235,0.956863}%
\pgfsetfillcolor{currentfill}%
\pgfsetlinewidth{1.003750pt}%
\definecolor{currentstroke}{rgb}{0.631373,0.788235,0.956863}%
\pgfsetstrokecolor{currentstroke}%
\pgfsetdash{}{0pt}%
\pgfsys@defobject{currentmarker}{\pgfqpoint{-0.041667in}{-0.041667in}}{\pgfqpoint{0.041667in}{0.041667in}}{%
\pgfpathmoveto{\pgfqpoint{0.000000in}{-0.041667in}}%
\pgfpathcurveto{\pgfqpoint{0.011050in}{-0.041667in}}{\pgfqpoint{0.021649in}{-0.037276in}}{\pgfqpoint{0.029463in}{-0.029463in}}%
\pgfpathcurveto{\pgfqpoint{0.037276in}{-0.021649in}}{\pgfqpoint{0.041667in}{-0.011050in}}{\pgfqpoint{0.041667in}{0.000000in}}%
\pgfpathcurveto{\pgfqpoint{0.041667in}{0.011050in}}{\pgfqpoint{0.037276in}{0.021649in}}{\pgfqpoint{0.029463in}{0.029463in}}%
\pgfpathcurveto{\pgfqpoint{0.021649in}{0.037276in}}{\pgfqpoint{0.011050in}{0.041667in}}{\pgfqpoint{0.000000in}{0.041667in}}%
\pgfpathcurveto{\pgfqpoint{-0.011050in}{0.041667in}}{\pgfqpoint{-0.021649in}{0.037276in}}{\pgfqpoint{-0.029463in}{0.029463in}}%
\pgfpathcurveto{\pgfqpoint{-0.037276in}{0.021649in}}{\pgfqpoint{-0.041667in}{0.011050in}}{\pgfqpoint{-0.041667in}{0.000000in}}%
\pgfpathcurveto{\pgfqpoint{-0.041667in}{-0.011050in}}{\pgfqpoint{-0.037276in}{-0.021649in}}{\pgfqpoint{-0.029463in}{-0.029463in}}%
\pgfpathcurveto{\pgfqpoint{-0.021649in}{-0.037276in}}{\pgfqpoint{-0.011050in}{-0.041667in}}{\pgfqpoint{0.000000in}{-0.041667in}}%
\pgfpathclose%
\pgfusepath{stroke,fill}%
}%
\end{pgfscope}%
\begin{pgfscope}%
\pgfpathrectangle{\pgfqpoint{0.570343in}{0.331635in}}{\pgfqpoint{9.300000in}{7.700000in}}%
\pgfusepath{clip}%
\pgfsetbuttcap%
\pgfsetroundjoin%
\definecolor{currentfill}{rgb}{1.000000,0.705882,0.509804}%
\pgfsetfillcolor{currentfill}%
\pgfsetlinewidth{1.003750pt}%
\definecolor{currentstroke}{rgb}{1.000000,0.705882,0.509804}%
\pgfsetstrokecolor{currentstroke}%
\pgfsetdash{}{0pt}%
\pgfsys@defobject{currentmarker}{\pgfqpoint{-0.041667in}{-0.041667in}}{\pgfqpoint{0.041667in}{0.041667in}}{%
\pgfpathmoveto{\pgfqpoint{0.000000in}{-0.041667in}}%
\pgfpathcurveto{\pgfqpoint{0.011050in}{-0.041667in}}{\pgfqpoint{0.021649in}{-0.037276in}}{\pgfqpoint{0.029463in}{-0.029463in}}%
\pgfpathcurveto{\pgfqpoint{0.037276in}{-0.021649in}}{\pgfqpoint{0.041667in}{-0.011050in}}{\pgfqpoint{0.041667in}{0.000000in}}%
\pgfpathcurveto{\pgfqpoint{0.041667in}{0.011050in}}{\pgfqpoint{0.037276in}{0.021649in}}{\pgfqpoint{0.029463in}{0.029463in}}%
\pgfpathcurveto{\pgfqpoint{0.021649in}{0.037276in}}{\pgfqpoint{0.011050in}{0.041667in}}{\pgfqpoint{0.000000in}{0.041667in}}%
\pgfpathcurveto{\pgfqpoint{-0.011050in}{0.041667in}}{\pgfqpoint{-0.021649in}{0.037276in}}{\pgfqpoint{-0.029463in}{0.029463in}}%
\pgfpathcurveto{\pgfqpoint{-0.037276in}{0.021649in}}{\pgfqpoint{-0.041667in}{0.011050in}}{\pgfqpoint{-0.041667in}{0.000000in}}%
\pgfpathcurveto{\pgfqpoint{-0.041667in}{-0.011050in}}{\pgfqpoint{-0.037276in}{-0.021649in}}{\pgfqpoint{-0.029463in}{-0.029463in}}%
\pgfpathcurveto{\pgfqpoint{-0.021649in}{-0.037276in}}{\pgfqpoint{-0.011050in}{-0.041667in}}{\pgfqpoint{0.000000in}{-0.041667in}}%
\pgfpathclose%
\pgfusepath{stroke,fill}%
}%
\end{pgfscope}%
\begin{pgfscope}%
\pgfsetbuttcap%
\pgfsetroundjoin%
\definecolor{currentfill}{rgb}{0.000000,0.000000,0.000000}%
\pgfsetfillcolor{currentfill}%
\pgfsetlinewidth{0.803000pt}%
\definecolor{currentstroke}{rgb}{0.000000,0.000000,0.000000}%
\pgfsetstrokecolor{currentstroke}%
\pgfsetdash{}{0pt}%
\pgfsys@defobject{currentmarker}{\pgfqpoint{0.000000in}{-0.048611in}}{\pgfqpoint{0.000000in}{0.000000in}}{%
\pgfpathmoveto{\pgfqpoint{0.000000in}{0.000000in}}%
\pgfpathlineto{\pgfqpoint{0.000000in}{-0.048611in}}%
\pgfusepath{stroke,fill}%
}%
\begin{pgfscope}%
\pgfsys@transformshift{2.109765in}{0.331635in}%
\pgfsys@useobject{currentmarker}{}%
\end{pgfscope}%
\end{pgfscope}%
\begin{pgfscope}%
\definecolor{textcolor}{rgb}{0.000000,0.000000,0.000000}%
\pgfsetstrokecolor{textcolor}%
\pgfsetfillcolor{textcolor}%
\pgftext[x=2.109765in,y=0.234413in,,top]{\color{textcolor}\sffamily\fontsize{10.000000}{12.000000}\selectfont \ensuremath{-}100}%
\end{pgfscope}%
\begin{pgfscope}%
\pgfsetbuttcap%
\pgfsetroundjoin%
\definecolor{currentfill}{rgb}{0.000000,0.000000,0.000000}%
\pgfsetfillcolor{currentfill}%
\pgfsetlinewidth{0.803000pt}%
\definecolor{currentstroke}{rgb}{0.000000,0.000000,0.000000}%
\pgfsetstrokecolor{currentstroke}%
\pgfsetdash{}{0pt}%
\pgfsys@defobject{currentmarker}{\pgfqpoint{0.000000in}{-0.048611in}}{\pgfqpoint{0.000000in}{0.000000in}}{%
\pgfpathmoveto{\pgfqpoint{0.000000in}{0.000000in}}%
\pgfpathlineto{\pgfqpoint{0.000000in}{-0.048611in}}%
\pgfusepath{stroke,fill}%
}%
\begin{pgfscope}%
\pgfsys@transformshift{3.797034in}{0.331635in}%
\pgfsys@useobject{currentmarker}{}%
\end{pgfscope}%
\end{pgfscope}%
\begin{pgfscope}%
\definecolor{textcolor}{rgb}{0.000000,0.000000,0.000000}%
\pgfsetstrokecolor{textcolor}%
\pgfsetfillcolor{textcolor}%
\pgftext[x=3.797034in,y=0.234413in,,top]{\color{textcolor}\sffamily\fontsize{10.000000}{12.000000}\selectfont \ensuremath{-}50}%
\end{pgfscope}%
\begin{pgfscope}%
\pgfsetbuttcap%
\pgfsetroundjoin%
\definecolor{currentfill}{rgb}{0.000000,0.000000,0.000000}%
\pgfsetfillcolor{currentfill}%
\pgfsetlinewidth{0.803000pt}%
\definecolor{currentstroke}{rgb}{0.000000,0.000000,0.000000}%
\pgfsetstrokecolor{currentstroke}%
\pgfsetdash{}{0pt}%
\pgfsys@defobject{currentmarker}{\pgfqpoint{0.000000in}{-0.048611in}}{\pgfqpoint{0.000000in}{0.000000in}}{%
\pgfpathmoveto{\pgfqpoint{0.000000in}{0.000000in}}%
\pgfpathlineto{\pgfqpoint{0.000000in}{-0.048611in}}%
\pgfusepath{stroke,fill}%
}%
\begin{pgfscope}%
\pgfsys@transformshift{5.484303in}{0.331635in}%
\pgfsys@useobject{currentmarker}{}%
\end{pgfscope}%
\end{pgfscope}%
\begin{pgfscope}%
\definecolor{textcolor}{rgb}{0.000000,0.000000,0.000000}%
\pgfsetstrokecolor{textcolor}%
\pgfsetfillcolor{textcolor}%
\pgftext[x=5.484303in,y=0.234413in,,top]{\color{textcolor}\sffamily\fontsize{10.000000}{12.000000}\selectfont 0}%
\end{pgfscope}%
\begin{pgfscope}%
\pgfsetbuttcap%
\pgfsetroundjoin%
\definecolor{currentfill}{rgb}{0.000000,0.000000,0.000000}%
\pgfsetfillcolor{currentfill}%
\pgfsetlinewidth{0.803000pt}%
\definecolor{currentstroke}{rgb}{0.000000,0.000000,0.000000}%
\pgfsetstrokecolor{currentstroke}%
\pgfsetdash{}{0pt}%
\pgfsys@defobject{currentmarker}{\pgfqpoint{0.000000in}{-0.048611in}}{\pgfqpoint{0.000000in}{0.000000in}}{%
\pgfpathmoveto{\pgfqpoint{0.000000in}{0.000000in}}%
\pgfpathlineto{\pgfqpoint{0.000000in}{-0.048611in}}%
\pgfusepath{stroke,fill}%
}%
\begin{pgfscope}%
\pgfsys@transformshift{7.171572in}{0.331635in}%
\pgfsys@useobject{currentmarker}{}%
\end{pgfscope}%
\end{pgfscope}%
\begin{pgfscope}%
\definecolor{textcolor}{rgb}{0.000000,0.000000,0.000000}%
\pgfsetstrokecolor{textcolor}%
\pgfsetfillcolor{textcolor}%
\pgftext[x=7.171572in,y=0.234413in,,top]{\color{textcolor}\sffamily\fontsize{10.000000}{12.000000}\selectfont 50}%
\end{pgfscope}%
\begin{pgfscope}%
\pgfsetbuttcap%
\pgfsetroundjoin%
\definecolor{currentfill}{rgb}{0.000000,0.000000,0.000000}%
\pgfsetfillcolor{currentfill}%
\pgfsetlinewidth{0.803000pt}%
\definecolor{currentstroke}{rgb}{0.000000,0.000000,0.000000}%
\pgfsetstrokecolor{currentstroke}%
\pgfsetdash{}{0pt}%
\pgfsys@defobject{currentmarker}{\pgfqpoint{0.000000in}{-0.048611in}}{\pgfqpoint{0.000000in}{0.000000in}}{%
\pgfpathmoveto{\pgfqpoint{0.000000in}{0.000000in}}%
\pgfpathlineto{\pgfqpoint{0.000000in}{-0.048611in}}%
\pgfusepath{stroke,fill}%
}%
\begin{pgfscope}%
\pgfsys@transformshift{8.858841in}{0.331635in}%
\pgfsys@useobject{currentmarker}{}%
\end{pgfscope}%
\end{pgfscope}%
\begin{pgfscope}%
\definecolor{textcolor}{rgb}{0.000000,0.000000,0.000000}%
\pgfsetstrokecolor{textcolor}%
\pgfsetfillcolor{textcolor}%
\pgftext[x=8.858841in,y=0.234413in,,top]{\color{textcolor}\sffamily\fontsize{10.000000}{12.000000}\selectfont 100}%
\end{pgfscope}%
\begin{pgfscope}%
\pgfsetbuttcap%
\pgfsetroundjoin%
\definecolor{currentfill}{rgb}{0.000000,0.000000,0.000000}%
\pgfsetfillcolor{currentfill}%
\pgfsetlinewidth{0.803000pt}%
\definecolor{currentstroke}{rgb}{0.000000,0.000000,0.000000}%
\pgfsetstrokecolor{currentstroke}%
\pgfsetdash{}{0pt}%
\pgfsys@defobject{currentmarker}{\pgfqpoint{-0.048611in}{0.000000in}}{\pgfqpoint{-0.000000in}{0.000000in}}{%
\pgfpathmoveto{\pgfqpoint{-0.000000in}{0.000000in}}%
\pgfpathlineto{\pgfqpoint{-0.048611in}{0.000000in}}%
\pgfusepath{stroke,fill}%
}%
\begin{pgfscope}%
\pgfsys@transformshift{0.570343in}{1.425219in}%
\pgfsys@useobject{currentmarker}{}%
\end{pgfscope}%
\end{pgfscope}%
\begin{pgfscope}%
\definecolor{textcolor}{rgb}{0.000000,0.000000,0.000000}%
\pgfsetstrokecolor{textcolor}%
\pgfsetfillcolor{textcolor}%
\pgftext[x=0.100000in, y=1.372457in, left, base]{\color{textcolor}\sffamily\fontsize{10.000000}{12.000000}\selectfont \ensuremath{-}100}%
\end{pgfscope}%
\begin{pgfscope}%
\pgfsetbuttcap%
\pgfsetroundjoin%
\definecolor{currentfill}{rgb}{0.000000,0.000000,0.000000}%
\pgfsetfillcolor{currentfill}%
\pgfsetlinewidth{0.803000pt}%
\definecolor{currentstroke}{rgb}{0.000000,0.000000,0.000000}%
\pgfsetstrokecolor{currentstroke}%
\pgfsetdash{}{0pt}%
\pgfsys@defobject{currentmarker}{\pgfqpoint{-0.048611in}{0.000000in}}{\pgfqpoint{-0.000000in}{0.000000in}}{%
\pgfpathmoveto{\pgfqpoint{-0.000000in}{0.000000in}}%
\pgfpathlineto{\pgfqpoint{-0.048611in}{0.000000in}}%
\pgfusepath{stroke,fill}%
}%
\begin{pgfscope}%
\pgfsys@transformshift{0.570343in}{2.863412in}%
\pgfsys@useobject{currentmarker}{}%
\end{pgfscope}%
\end{pgfscope}%
\begin{pgfscope}%
\definecolor{textcolor}{rgb}{0.000000,0.000000,0.000000}%
\pgfsetstrokecolor{textcolor}%
\pgfsetfillcolor{textcolor}%
\pgftext[x=0.188365in, y=2.810650in, left, base]{\color{textcolor}\sffamily\fontsize{10.000000}{12.000000}\selectfont \ensuremath{-}50}%
\end{pgfscope}%
\begin{pgfscope}%
\pgfsetbuttcap%
\pgfsetroundjoin%
\definecolor{currentfill}{rgb}{0.000000,0.000000,0.000000}%
\pgfsetfillcolor{currentfill}%
\pgfsetlinewidth{0.803000pt}%
\definecolor{currentstroke}{rgb}{0.000000,0.000000,0.000000}%
\pgfsetstrokecolor{currentstroke}%
\pgfsetdash{}{0pt}%
\pgfsys@defobject{currentmarker}{\pgfqpoint{-0.048611in}{0.000000in}}{\pgfqpoint{-0.000000in}{0.000000in}}{%
\pgfpathmoveto{\pgfqpoint{-0.000000in}{0.000000in}}%
\pgfpathlineto{\pgfqpoint{-0.048611in}{0.000000in}}%
\pgfusepath{stroke,fill}%
}%
\begin{pgfscope}%
\pgfsys@transformshift{0.570343in}{4.301604in}%
\pgfsys@useobject{currentmarker}{}%
\end{pgfscope}%
\end{pgfscope}%
\begin{pgfscope}%
\definecolor{textcolor}{rgb}{0.000000,0.000000,0.000000}%
\pgfsetstrokecolor{textcolor}%
\pgfsetfillcolor{textcolor}%
\pgftext[x=0.384756in, y=4.248843in, left, base]{\color{textcolor}\sffamily\fontsize{10.000000}{12.000000}\selectfont 0}%
\end{pgfscope}%
\begin{pgfscope}%
\pgfsetbuttcap%
\pgfsetroundjoin%
\definecolor{currentfill}{rgb}{0.000000,0.000000,0.000000}%
\pgfsetfillcolor{currentfill}%
\pgfsetlinewidth{0.803000pt}%
\definecolor{currentstroke}{rgb}{0.000000,0.000000,0.000000}%
\pgfsetstrokecolor{currentstroke}%
\pgfsetdash{}{0pt}%
\pgfsys@defobject{currentmarker}{\pgfqpoint{-0.048611in}{0.000000in}}{\pgfqpoint{-0.000000in}{0.000000in}}{%
\pgfpathmoveto{\pgfqpoint{-0.000000in}{0.000000in}}%
\pgfpathlineto{\pgfqpoint{-0.048611in}{0.000000in}}%
\pgfusepath{stroke,fill}%
}%
\begin{pgfscope}%
\pgfsys@transformshift{0.570343in}{5.739797in}%
\pgfsys@useobject{currentmarker}{}%
\end{pgfscope}%
\end{pgfscope}%
\begin{pgfscope}%
\definecolor{textcolor}{rgb}{0.000000,0.000000,0.000000}%
\pgfsetstrokecolor{textcolor}%
\pgfsetfillcolor{textcolor}%
\pgftext[x=0.296390in, y=5.687035in, left, base]{\color{textcolor}\sffamily\fontsize{10.000000}{12.000000}\selectfont 50}%
\end{pgfscope}%
\begin{pgfscope}%
\pgfsetbuttcap%
\pgfsetroundjoin%
\definecolor{currentfill}{rgb}{0.000000,0.000000,0.000000}%
\pgfsetfillcolor{currentfill}%
\pgfsetlinewidth{0.803000pt}%
\definecolor{currentstroke}{rgb}{0.000000,0.000000,0.000000}%
\pgfsetstrokecolor{currentstroke}%
\pgfsetdash{}{0pt}%
\pgfsys@defobject{currentmarker}{\pgfqpoint{-0.048611in}{0.000000in}}{\pgfqpoint{-0.000000in}{0.000000in}}{%
\pgfpathmoveto{\pgfqpoint{-0.000000in}{0.000000in}}%
\pgfpathlineto{\pgfqpoint{-0.048611in}{0.000000in}}%
\pgfusepath{stroke,fill}%
}%
\begin{pgfscope}%
\pgfsys@transformshift{0.570343in}{7.177989in}%
\pgfsys@useobject{currentmarker}{}%
\end{pgfscope}%
\end{pgfscope}%
\begin{pgfscope}%
\definecolor{textcolor}{rgb}{0.000000,0.000000,0.000000}%
\pgfsetstrokecolor{textcolor}%
\pgfsetfillcolor{textcolor}%
\pgftext[x=0.208025in, y=7.125228in, left, base]{\color{textcolor}\sffamily\fontsize{10.000000}{12.000000}\selectfont 100}%
\end{pgfscope}%
\begin{pgfscope}%
\pgfpathrectangle{\pgfqpoint{0.570343in}{0.331635in}}{\pgfqpoint{9.300000in}{7.700000in}}%
\pgfusepath{clip}%
\pgfsetrectcap%
\pgfsetroundjoin%
\pgfsetlinewidth{1.505625pt}%
\definecolor{currentstroke}{rgb}{0.631373,0.788235,0.956863}%
\pgfsetstrokecolor{currentstroke}%
\pgfsetstrokeopacity{0.800000}%
\pgfsetdash{}{0pt}%
\pgfpathmoveto{\pgfqpoint{7.296937in}{5.090198in}}%
\pgfpathlineto{\pgfqpoint{5.327974in}{4.421066in}}%
\pgfusepath{stroke}%
\end{pgfscope}%
\begin{pgfscope}%
\pgfpathrectangle{\pgfqpoint{0.570343in}{0.331635in}}{\pgfqpoint{9.300000in}{7.700000in}}%
\pgfusepath{clip}%
\pgfsetrectcap%
\pgfsetroundjoin%
\pgfsetlinewidth{1.505625pt}%
\definecolor{currentstroke}{rgb}{0.631373,0.788235,0.956863}%
\pgfsetstrokecolor{currentstroke}%
\pgfsetstrokeopacity{0.800000}%
\pgfsetdash{}{0pt}%
\pgfpathmoveto{\pgfqpoint{4.019475in}{3.717731in}}%
\pgfpathlineto{\pgfqpoint{5.327974in}{4.421066in}}%
\pgfusepath{stroke}%
\end{pgfscope}%
\begin{pgfscope}%
\pgfpathrectangle{\pgfqpoint{0.570343in}{0.331635in}}{\pgfqpoint{9.300000in}{7.700000in}}%
\pgfusepath{clip}%
\pgfsetrectcap%
\pgfsetroundjoin%
\pgfsetlinewidth{1.505625pt}%
\definecolor{currentstroke}{rgb}{0.631373,0.788235,0.956863}%
\pgfsetstrokecolor{currentstroke}%
\pgfsetstrokeopacity{0.800000}%
\pgfsetdash{}{0pt}%
\pgfpathmoveto{\pgfqpoint{4.684473in}{3.356530in}}%
\pgfpathlineto{\pgfqpoint{5.327974in}{4.421066in}}%
\pgfusepath{stroke}%
\end{pgfscope}%
\begin{pgfscope}%
\pgfpathrectangle{\pgfqpoint{0.570343in}{0.331635in}}{\pgfqpoint{9.300000in}{7.700000in}}%
\pgfusepath{clip}%
\pgfsetrectcap%
\pgfsetroundjoin%
\pgfsetlinewidth{1.505625pt}%
\definecolor{currentstroke}{rgb}{0.631373,0.788235,0.956863}%
\pgfsetstrokecolor{currentstroke}%
\pgfsetstrokeopacity{0.800000}%
\pgfsetdash{}{0pt}%
\pgfpathmoveto{\pgfqpoint{7.305182in}{2.839150in}}%
\pgfpathlineto{\pgfqpoint{5.327974in}{4.421066in}}%
\pgfusepath{stroke}%
\end{pgfscope}%
\begin{pgfscope}%
\pgfpathrectangle{\pgfqpoint{0.570343in}{0.331635in}}{\pgfqpoint{9.300000in}{7.700000in}}%
\pgfusepath{clip}%
\pgfsetrectcap%
\pgfsetroundjoin%
\pgfsetlinewidth{1.505625pt}%
\definecolor{currentstroke}{rgb}{0.631373,0.788235,0.956863}%
\pgfsetstrokecolor{currentstroke}%
\pgfsetstrokeopacity{0.800000}%
\pgfsetdash{}{0pt}%
\pgfpathmoveto{\pgfqpoint{8.044602in}{5.291231in}}%
\pgfpathlineto{\pgfqpoint{5.327974in}{4.421066in}}%
\pgfusepath{stroke}%
\end{pgfscope}%
\begin{pgfscope}%
\pgfpathrectangle{\pgfqpoint{0.570343in}{0.331635in}}{\pgfqpoint{9.300000in}{7.700000in}}%
\pgfusepath{clip}%
\pgfsetrectcap%
\pgfsetroundjoin%
\pgfsetlinewidth{1.505625pt}%
\definecolor{currentstroke}{rgb}{0.631373,0.788235,0.956863}%
\pgfsetstrokecolor{currentstroke}%
\pgfsetstrokeopacity{0.800000}%
\pgfsetdash{}{0pt}%
\pgfpathmoveto{\pgfqpoint{6.828282in}{4.428679in}}%
\pgfpathlineto{\pgfqpoint{5.327974in}{4.421066in}}%
\pgfusepath{stroke}%
\end{pgfscope}%
\begin{pgfscope}%
\pgfpathrectangle{\pgfqpoint{0.570343in}{0.331635in}}{\pgfqpoint{9.300000in}{7.700000in}}%
\pgfusepath{clip}%
\pgfsetrectcap%
\pgfsetroundjoin%
\pgfsetlinewidth{1.505625pt}%
\definecolor{currentstroke}{rgb}{0.631373,0.788235,0.956863}%
\pgfsetstrokecolor{currentstroke}%
\pgfsetstrokeopacity{0.800000}%
\pgfsetdash{}{0pt}%
\pgfpathmoveto{\pgfqpoint{8.665804in}{4.818725in}}%
\pgfpathlineto{\pgfqpoint{5.327974in}{4.421066in}}%
\pgfusepath{stroke}%
\end{pgfscope}%
\begin{pgfscope}%
\pgfpathrectangle{\pgfqpoint{0.570343in}{0.331635in}}{\pgfqpoint{9.300000in}{7.700000in}}%
\pgfusepath{clip}%
\pgfsetrectcap%
\pgfsetroundjoin%
\pgfsetlinewidth{1.505625pt}%
\definecolor{currentstroke}{rgb}{0.631373,0.788235,0.956863}%
\pgfsetstrokecolor{currentstroke}%
\pgfsetstrokeopacity{0.800000}%
\pgfsetdash{}{0pt}%
\pgfpathmoveto{\pgfqpoint{8.976651in}{5.686339in}}%
\pgfpathlineto{\pgfqpoint{5.327974in}{4.421066in}}%
\pgfusepath{stroke}%
\end{pgfscope}%
\begin{pgfscope}%
\pgfpathrectangle{\pgfqpoint{0.570343in}{0.331635in}}{\pgfqpoint{9.300000in}{7.700000in}}%
\pgfusepath{clip}%
\pgfsetrectcap%
\pgfsetroundjoin%
\pgfsetlinewidth{1.505625pt}%
\definecolor{currentstroke}{rgb}{0.631373,0.788235,0.956863}%
\pgfsetstrokecolor{currentstroke}%
\pgfsetstrokeopacity{0.800000}%
\pgfsetdash{}{0pt}%
\pgfpathmoveto{\pgfqpoint{5.272172in}{4.818041in}}%
\pgfpathlineto{\pgfqpoint{5.327974in}{4.421066in}}%
\pgfusepath{stroke}%
\end{pgfscope}%
\begin{pgfscope}%
\pgfpathrectangle{\pgfqpoint{0.570343in}{0.331635in}}{\pgfqpoint{9.300000in}{7.700000in}}%
\pgfusepath{clip}%
\pgfsetrectcap%
\pgfsetroundjoin%
\pgfsetlinewidth{1.505625pt}%
\definecolor{currentstroke}{rgb}{0.631373,0.788235,0.956863}%
\pgfsetstrokecolor{currentstroke}%
\pgfsetstrokeopacity{0.800000}%
\pgfsetdash{}{0pt}%
\pgfpathmoveto{\pgfqpoint{3.063486in}{6.472096in}}%
\pgfpathlineto{\pgfqpoint{5.327974in}{4.421066in}}%
\pgfusepath{stroke}%
\end{pgfscope}%
\begin{pgfscope}%
\pgfpathrectangle{\pgfqpoint{0.570343in}{0.331635in}}{\pgfqpoint{9.300000in}{7.700000in}}%
\pgfusepath{clip}%
\pgfsetrectcap%
\pgfsetroundjoin%
\pgfsetlinewidth{1.505625pt}%
\definecolor{currentstroke}{rgb}{0.631373,0.788235,0.956863}%
\pgfsetstrokecolor{currentstroke}%
\pgfsetstrokeopacity{0.800000}%
\pgfsetdash{}{0pt}%
\pgfpathmoveto{\pgfqpoint{6.063540in}{4.805271in}}%
\pgfpathlineto{\pgfqpoint{5.327974in}{4.421066in}}%
\pgfusepath{stroke}%
\end{pgfscope}%
\begin{pgfscope}%
\pgfpathrectangle{\pgfqpoint{0.570343in}{0.331635in}}{\pgfqpoint{9.300000in}{7.700000in}}%
\pgfusepath{clip}%
\pgfsetrectcap%
\pgfsetroundjoin%
\pgfsetlinewidth{1.505625pt}%
\definecolor{currentstroke}{rgb}{0.631373,0.788235,0.956863}%
\pgfsetstrokecolor{currentstroke}%
\pgfsetstrokeopacity{0.800000}%
\pgfsetdash{}{0pt}%
\pgfpathmoveto{\pgfqpoint{1.762911in}{3.896616in}}%
\pgfpathlineto{\pgfqpoint{5.327974in}{4.421066in}}%
\pgfusepath{stroke}%
\end{pgfscope}%
\begin{pgfscope}%
\pgfpathrectangle{\pgfqpoint{0.570343in}{0.331635in}}{\pgfqpoint{9.300000in}{7.700000in}}%
\pgfusepath{clip}%
\pgfsetrectcap%
\pgfsetroundjoin%
\pgfsetlinewidth{1.505625pt}%
\definecolor{currentstroke}{rgb}{0.631373,0.788235,0.956863}%
\pgfsetstrokecolor{currentstroke}%
\pgfsetstrokeopacity{0.800000}%
\pgfsetdash{}{0pt}%
\pgfpathmoveto{\pgfqpoint{4.671643in}{5.155754in}}%
\pgfpathlineto{\pgfqpoint{5.327974in}{4.421066in}}%
\pgfusepath{stroke}%
\end{pgfscope}%
\begin{pgfscope}%
\pgfpathrectangle{\pgfqpoint{0.570343in}{0.331635in}}{\pgfqpoint{9.300000in}{7.700000in}}%
\pgfusepath{clip}%
\pgfsetrectcap%
\pgfsetroundjoin%
\pgfsetlinewidth{1.505625pt}%
\definecolor{currentstroke}{rgb}{0.631373,0.788235,0.956863}%
\pgfsetstrokecolor{currentstroke}%
\pgfsetstrokeopacity{0.800000}%
\pgfsetdash{}{0pt}%
\pgfpathmoveto{\pgfqpoint{2.082714in}{3.055275in}}%
\pgfpathlineto{\pgfqpoint{5.327974in}{4.421066in}}%
\pgfusepath{stroke}%
\end{pgfscope}%
\begin{pgfscope}%
\pgfpathrectangle{\pgfqpoint{0.570343in}{0.331635in}}{\pgfqpoint{9.300000in}{7.700000in}}%
\pgfusepath{clip}%
\pgfsetrectcap%
\pgfsetroundjoin%
\pgfsetlinewidth{1.505625pt}%
\definecolor{currentstroke}{rgb}{0.631373,0.788235,0.956863}%
\pgfsetstrokecolor{currentstroke}%
\pgfsetstrokeopacity{0.800000}%
\pgfsetdash{}{0pt}%
\pgfpathmoveto{\pgfqpoint{2.029248in}{1.578122in}}%
\pgfpathlineto{\pgfqpoint{5.327974in}{4.421066in}}%
\pgfusepath{stroke}%
\end{pgfscope}%
\begin{pgfscope}%
\pgfpathrectangle{\pgfqpoint{0.570343in}{0.331635in}}{\pgfqpoint{9.300000in}{7.700000in}}%
\pgfusepath{clip}%
\pgfsetrectcap%
\pgfsetroundjoin%
\pgfsetlinewidth{1.505625pt}%
\definecolor{currentstroke}{rgb}{0.631373,0.788235,0.956863}%
\pgfsetstrokecolor{currentstroke}%
\pgfsetstrokeopacity{0.800000}%
\pgfsetdash{}{0pt}%
\pgfpathmoveto{\pgfqpoint{5.175896in}{4.017231in}}%
\pgfpathlineto{\pgfqpoint{5.327974in}{4.421066in}}%
\pgfusepath{stroke}%
\end{pgfscope}%
\begin{pgfscope}%
\pgfpathrectangle{\pgfqpoint{0.570343in}{0.331635in}}{\pgfqpoint{9.300000in}{7.700000in}}%
\pgfusepath{clip}%
\pgfsetrectcap%
\pgfsetroundjoin%
\pgfsetlinewidth{1.505625pt}%
\definecolor{currentstroke}{rgb}{0.631373,0.788235,0.956863}%
\pgfsetstrokecolor{currentstroke}%
\pgfsetstrokeopacity{0.800000}%
\pgfsetdash{}{0pt}%
\pgfpathmoveto{\pgfqpoint{3.674629in}{5.541790in}}%
\pgfpathlineto{\pgfqpoint{5.327974in}{4.421066in}}%
\pgfusepath{stroke}%
\end{pgfscope}%
\begin{pgfscope}%
\pgfpathrectangle{\pgfqpoint{0.570343in}{0.331635in}}{\pgfqpoint{9.300000in}{7.700000in}}%
\pgfusepath{clip}%
\pgfsetrectcap%
\pgfsetroundjoin%
\pgfsetlinewidth{1.505625pt}%
\definecolor{currentstroke}{rgb}{0.631373,0.788235,0.956863}%
\pgfsetstrokecolor{currentstroke}%
\pgfsetstrokeopacity{0.800000}%
\pgfsetdash{}{0pt}%
\pgfpathmoveto{\pgfqpoint{1.383929in}{5.039838in}}%
\pgfpathlineto{\pgfqpoint{5.327974in}{4.421066in}}%
\pgfusepath{stroke}%
\end{pgfscope}%
\begin{pgfscope}%
\pgfpathrectangle{\pgfqpoint{0.570343in}{0.331635in}}{\pgfqpoint{9.300000in}{7.700000in}}%
\pgfusepath{clip}%
\pgfsetrectcap%
\pgfsetroundjoin%
\pgfsetlinewidth{1.505625pt}%
\definecolor{currentstroke}{rgb}{0.631373,0.788235,0.956863}%
\pgfsetstrokecolor{currentstroke}%
\pgfsetstrokeopacity{0.800000}%
\pgfsetdash{}{0pt}%
\pgfpathmoveto{\pgfqpoint{3.677550in}{4.774109in}}%
\pgfpathlineto{\pgfqpoint{5.327974in}{4.421066in}}%
\pgfusepath{stroke}%
\end{pgfscope}%
\begin{pgfscope}%
\pgfpathrectangle{\pgfqpoint{0.570343in}{0.331635in}}{\pgfqpoint{9.300000in}{7.700000in}}%
\pgfusepath{clip}%
\pgfsetrectcap%
\pgfsetroundjoin%
\pgfsetlinewidth{1.505625pt}%
\definecolor{currentstroke}{rgb}{0.631373,0.788235,0.956863}%
\pgfsetstrokecolor{currentstroke}%
\pgfsetstrokeopacity{0.800000}%
\pgfsetdash{}{0pt}%
\pgfpathmoveto{\pgfqpoint{6.563827in}{3.646011in}}%
\pgfpathlineto{\pgfqpoint{5.327974in}{4.421066in}}%
\pgfusepath{stroke}%
\end{pgfscope}%
\begin{pgfscope}%
\pgfpathrectangle{\pgfqpoint{0.570343in}{0.331635in}}{\pgfqpoint{9.300000in}{7.700000in}}%
\pgfusepath{clip}%
\pgfsetrectcap%
\pgfsetroundjoin%
\pgfsetlinewidth{1.505625pt}%
\definecolor{currentstroke}{rgb}{0.631373,0.788235,0.956863}%
\pgfsetstrokecolor{currentstroke}%
\pgfsetstrokeopacity{0.800000}%
\pgfsetdash{}{0pt}%
\pgfpathmoveto{\pgfqpoint{4.968812in}{5.909158in}}%
\pgfpathlineto{\pgfqpoint{5.327974in}{4.421066in}}%
\pgfusepath{stroke}%
\end{pgfscope}%
\begin{pgfscope}%
\pgfpathrectangle{\pgfqpoint{0.570343in}{0.331635in}}{\pgfqpoint{9.300000in}{7.700000in}}%
\pgfusepath{clip}%
\pgfsetrectcap%
\pgfsetroundjoin%
\pgfsetlinewidth{1.505625pt}%
\definecolor{currentstroke}{rgb}{0.631373,0.788235,0.956863}%
\pgfsetstrokecolor{currentstroke}%
\pgfsetstrokeopacity{0.800000}%
\pgfsetdash{}{0pt}%
\pgfpathmoveto{\pgfqpoint{5.391700in}{1.666792in}}%
\pgfpathlineto{\pgfqpoint{5.327974in}{4.421066in}}%
\pgfusepath{stroke}%
\end{pgfscope}%
\begin{pgfscope}%
\pgfpathrectangle{\pgfqpoint{0.570343in}{0.331635in}}{\pgfqpoint{9.300000in}{7.700000in}}%
\pgfusepath{clip}%
\pgfsetrectcap%
\pgfsetroundjoin%
\pgfsetlinewidth{1.505625pt}%
\definecolor{currentstroke}{rgb}{0.631373,0.788235,0.956863}%
\pgfsetstrokecolor{currentstroke}%
\pgfsetstrokeopacity{0.800000}%
\pgfsetdash{}{0pt}%
\pgfpathmoveto{\pgfqpoint{7.110525in}{6.212121in}}%
\pgfpathlineto{\pgfqpoint{5.327974in}{4.421066in}}%
\pgfusepath{stroke}%
\end{pgfscope}%
\begin{pgfscope}%
\pgfpathrectangle{\pgfqpoint{0.570343in}{0.331635in}}{\pgfqpoint{9.300000in}{7.700000in}}%
\pgfusepath{clip}%
\pgfsetrectcap%
\pgfsetroundjoin%
\pgfsetlinewidth{1.505625pt}%
\definecolor{currentstroke}{rgb}{0.631373,0.788235,0.956863}%
\pgfsetstrokecolor{currentstroke}%
\pgfsetstrokeopacity{0.800000}%
\pgfsetdash{}{0pt}%
\pgfpathmoveto{\pgfqpoint{3.005187in}{2.565169in}}%
\pgfpathlineto{\pgfqpoint{5.327974in}{4.421066in}}%
\pgfusepath{stroke}%
\end{pgfscope}%
\begin{pgfscope}%
\pgfpathrectangle{\pgfqpoint{0.570343in}{0.331635in}}{\pgfqpoint{9.300000in}{7.700000in}}%
\pgfusepath{clip}%
\pgfsetrectcap%
\pgfsetroundjoin%
\pgfsetlinewidth{1.505625pt}%
\definecolor{currentstroke}{rgb}{0.631373,0.788235,0.956863}%
\pgfsetstrokecolor{currentstroke}%
\pgfsetstrokeopacity{0.800000}%
\pgfsetdash{}{0pt}%
\pgfpathmoveto{\pgfqpoint{4.481129in}{4.398252in}}%
\pgfpathlineto{\pgfqpoint{5.327974in}{4.421066in}}%
\pgfusepath{stroke}%
\end{pgfscope}%
\begin{pgfscope}%
\pgfpathrectangle{\pgfqpoint{0.570343in}{0.331635in}}{\pgfqpoint{9.300000in}{7.700000in}}%
\pgfusepath{clip}%
\pgfsetrectcap%
\pgfsetroundjoin%
\pgfsetlinewidth{1.505625pt}%
\definecolor{currentstroke}{rgb}{0.631373,0.788235,0.956863}%
\pgfsetstrokecolor{currentstroke}%
\pgfsetstrokeopacity{0.800000}%
\pgfsetdash{}{0pt}%
\pgfpathmoveto{\pgfqpoint{6.045238in}{6.416277in}}%
\pgfpathlineto{\pgfqpoint{5.327974in}{4.421066in}}%
\pgfusepath{stroke}%
\end{pgfscope}%
\begin{pgfscope}%
\pgfpathrectangle{\pgfqpoint{0.570343in}{0.331635in}}{\pgfqpoint{9.300000in}{7.700000in}}%
\pgfusepath{clip}%
\pgfsetrectcap%
\pgfsetroundjoin%
\pgfsetlinewidth{1.505625pt}%
\definecolor{currentstroke}{rgb}{0.631373,0.788235,0.956863}%
\pgfsetstrokecolor{currentstroke}%
\pgfsetstrokeopacity{0.800000}%
\pgfsetdash{}{0pt}%
\pgfpathmoveto{\pgfqpoint{9.187938in}{4.095317in}}%
\pgfpathlineto{\pgfqpoint{5.327974in}{4.421066in}}%
\pgfusepath{stroke}%
\end{pgfscope}%
\begin{pgfscope}%
\pgfpathrectangle{\pgfqpoint{0.570343in}{0.331635in}}{\pgfqpoint{9.300000in}{7.700000in}}%
\pgfusepath{clip}%
\pgfsetrectcap%
\pgfsetroundjoin%
\pgfsetlinewidth{1.505625pt}%
\definecolor{currentstroke}{rgb}{0.631373,0.788235,0.956863}%
\pgfsetstrokecolor{currentstroke}%
\pgfsetstrokeopacity{0.800000}%
\pgfsetdash{}{0pt}%
\pgfpathmoveto{\pgfqpoint{7.749803in}{4.498026in}}%
\pgfpathlineto{\pgfqpoint{5.327974in}{4.421066in}}%
\pgfusepath{stroke}%
\end{pgfscope}%
\begin{pgfscope}%
\pgfpathrectangle{\pgfqpoint{0.570343in}{0.331635in}}{\pgfqpoint{9.300000in}{7.700000in}}%
\pgfusepath{clip}%
\pgfsetrectcap%
\pgfsetroundjoin%
\pgfsetlinewidth{1.505625pt}%
\definecolor{currentstroke}{rgb}{1.000000,0.705882,0.509804}%
\pgfsetstrokecolor{currentstroke}%
\pgfsetstrokeopacity{0.800000}%
\pgfsetdash{}{0pt}%
\pgfpathmoveto{\pgfqpoint{5.715735in}{3.335706in}}%
\pgfpathlineto{\pgfqpoint{5.599385in}{3.905858in}}%
\pgfusepath{stroke}%
\end{pgfscope}%
\begin{pgfscope}%
\pgfpathrectangle{\pgfqpoint{0.570343in}{0.331635in}}{\pgfqpoint{9.300000in}{7.700000in}}%
\pgfusepath{clip}%
\pgfsetrectcap%
\pgfsetroundjoin%
\pgfsetlinewidth{1.505625pt}%
\definecolor{currentstroke}{rgb}{1.000000,0.705882,0.509804}%
\pgfsetstrokecolor{currentstroke}%
\pgfsetstrokeopacity{0.800000}%
\pgfsetdash{}{0pt}%
\pgfpathmoveto{\pgfqpoint{7.824153in}{5.963141in}}%
\pgfpathlineto{\pgfqpoint{5.599385in}{3.905858in}}%
\pgfusepath{stroke}%
\end{pgfscope}%
\begin{pgfscope}%
\pgfpathrectangle{\pgfqpoint{0.570343in}{0.331635in}}{\pgfqpoint{9.300000in}{7.700000in}}%
\pgfusepath{clip}%
\pgfsetrectcap%
\pgfsetroundjoin%
\pgfsetlinewidth{1.505625pt}%
\definecolor{currentstroke}{rgb}{1.000000,0.705882,0.509804}%
\pgfsetstrokecolor{currentstroke}%
\pgfsetstrokeopacity{0.800000}%
\pgfsetdash{}{0pt}%
\pgfpathmoveto{\pgfqpoint{5.894391in}{4.130919in}}%
\pgfpathlineto{\pgfqpoint{5.599385in}{3.905858in}}%
\pgfusepath{stroke}%
\end{pgfscope}%
\begin{pgfscope}%
\pgfpathrectangle{\pgfqpoint{0.570343in}{0.331635in}}{\pgfqpoint{9.300000in}{7.700000in}}%
\pgfusepath{clip}%
\pgfsetrectcap%
\pgfsetroundjoin%
\pgfsetlinewidth{1.505625pt}%
\definecolor{currentstroke}{rgb}{1.000000,0.705882,0.509804}%
\pgfsetstrokecolor{currentstroke}%
\pgfsetstrokeopacity{0.800000}%
\pgfsetdash{}{0pt}%
\pgfpathmoveto{\pgfqpoint{3.273962in}{4.230448in}}%
\pgfpathlineto{\pgfqpoint{5.599385in}{3.905858in}}%
\pgfusepath{stroke}%
\end{pgfscope}%
\begin{pgfscope}%
\pgfpathrectangle{\pgfqpoint{0.570343in}{0.331635in}}{\pgfqpoint{9.300000in}{7.700000in}}%
\pgfusepath{clip}%
\pgfsetrectcap%
\pgfsetroundjoin%
\pgfsetlinewidth{1.505625pt}%
\definecolor{currentstroke}{rgb}{1.000000,0.705882,0.509804}%
\pgfsetstrokecolor{currentstroke}%
\pgfsetstrokeopacity{0.800000}%
\pgfsetdash{}{0pt}%
\pgfpathmoveto{\pgfqpoint{4.485603in}{2.524014in}}%
\pgfpathlineto{\pgfqpoint{5.599385in}{3.905858in}}%
\pgfusepath{stroke}%
\end{pgfscope}%
\begin{pgfscope}%
\pgfpathrectangle{\pgfqpoint{0.570343in}{0.331635in}}{\pgfqpoint{9.300000in}{7.700000in}}%
\pgfusepath{clip}%
\pgfsetrectcap%
\pgfsetroundjoin%
\pgfsetlinewidth{1.505625pt}%
\definecolor{currentstroke}{rgb}{1.000000,0.705882,0.509804}%
\pgfsetstrokecolor{currentstroke}%
\pgfsetstrokeopacity{0.800000}%
\pgfsetdash{}{0pt}%
\pgfpathmoveto{\pgfqpoint{6.572664in}{5.239286in}}%
\pgfpathlineto{\pgfqpoint{5.599385in}{3.905858in}}%
\pgfusepath{stroke}%
\end{pgfscope}%
\begin{pgfscope}%
\pgfpathrectangle{\pgfqpoint{0.570343in}{0.331635in}}{\pgfqpoint{9.300000in}{7.700000in}}%
\pgfusepath{clip}%
\pgfsetrectcap%
\pgfsetroundjoin%
\pgfsetlinewidth{1.505625pt}%
\definecolor{currentstroke}{rgb}{1.000000,0.705882,0.509804}%
\pgfsetstrokecolor{currentstroke}%
\pgfsetstrokeopacity{0.800000}%
\pgfsetdash{}{0pt}%
\pgfpathmoveto{\pgfqpoint{7.581009in}{1.761446in}}%
\pgfpathlineto{\pgfqpoint{5.599385in}{3.905858in}}%
\pgfusepath{stroke}%
\end{pgfscope}%
\begin{pgfscope}%
\pgfpathrectangle{\pgfqpoint{0.570343in}{0.331635in}}{\pgfqpoint{9.300000in}{7.700000in}}%
\pgfusepath{clip}%
\pgfsetrectcap%
\pgfsetroundjoin%
\pgfsetlinewidth{1.505625pt}%
\definecolor{currentstroke}{rgb}{1.000000,0.705882,0.509804}%
\pgfsetstrokecolor{currentstroke}%
\pgfsetstrokeopacity{0.800000}%
\pgfsetdash{}{0pt}%
\pgfpathmoveto{\pgfqpoint{1.757394in}{6.526560in}}%
\pgfpathlineto{\pgfqpoint{5.599385in}{3.905858in}}%
\pgfusepath{stroke}%
\end{pgfscope}%
\begin{pgfscope}%
\pgfpathrectangle{\pgfqpoint{0.570343in}{0.331635in}}{\pgfqpoint{9.300000in}{7.700000in}}%
\pgfusepath{clip}%
\pgfsetrectcap%
\pgfsetroundjoin%
\pgfsetlinewidth{1.505625pt}%
\definecolor{currentstroke}{rgb}{1.000000,0.705882,0.509804}%
\pgfsetstrokecolor{currentstroke}%
\pgfsetstrokeopacity{0.800000}%
\pgfsetdash{}{0pt}%
\pgfpathmoveto{\pgfqpoint{5.468750in}{7.350048in}}%
\pgfpathlineto{\pgfqpoint{5.599385in}{3.905858in}}%
\pgfusepath{stroke}%
\end{pgfscope}%
\begin{pgfscope}%
\pgfpathrectangle{\pgfqpoint{0.570343in}{0.331635in}}{\pgfqpoint{9.300000in}{7.700000in}}%
\pgfusepath{clip}%
\pgfsetrectcap%
\pgfsetroundjoin%
\pgfsetlinewidth{1.505625pt}%
\definecolor{currentstroke}{rgb}{1.000000,0.705882,0.509804}%
\pgfsetstrokecolor{currentstroke}%
\pgfsetstrokeopacity{0.800000}%
\pgfsetdash{}{0pt}%
\pgfpathmoveto{\pgfqpoint{4.136121in}{6.404828in}}%
\pgfpathlineto{\pgfqpoint{5.599385in}{3.905858in}}%
\pgfusepath{stroke}%
\end{pgfscope}%
\begin{pgfscope}%
\pgfpathrectangle{\pgfqpoint{0.570343in}{0.331635in}}{\pgfqpoint{9.300000in}{7.700000in}}%
\pgfusepath{clip}%
\pgfsetrectcap%
\pgfsetroundjoin%
\pgfsetlinewidth{1.505625pt}%
\definecolor{currentstroke}{rgb}{1.000000,0.705882,0.509804}%
\pgfsetstrokecolor{currentstroke}%
\pgfsetstrokeopacity{0.800000}%
\pgfsetdash{}{0pt}%
\pgfpathmoveto{\pgfqpoint{3.063964in}{3.577482in}}%
\pgfpathlineto{\pgfqpoint{5.599385in}{3.905858in}}%
\pgfusepath{stroke}%
\end{pgfscope}%
\begin{pgfscope}%
\pgfpathrectangle{\pgfqpoint{0.570343in}{0.331635in}}{\pgfqpoint{9.300000in}{7.700000in}}%
\pgfusepath{clip}%
\pgfsetrectcap%
\pgfsetroundjoin%
\pgfsetlinewidth{1.505625pt}%
\definecolor{currentstroke}{rgb}{1.000000,0.705882,0.509804}%
\pgfsetstrokecolor{currentstroke}%
\pgfsetstrokeopacity{0.800000}%
\pgfsetdash{}{0pt}%
\pgfpathmoveto{\pgfqpoint{3.378790in}{1.225414in}}%
\pgfpathlineto{\pgfqpoint{5.599385in}{3.905858in}}%
\pgfusepath{stroke}%
\end{pgfscope}%
\begin{pgfscope}%
\pgfpathrectangle{\pgfqpoint{0.570343in}{0.331635in}}{\pgfqpoint{9.300000in}{7.700000in}}%
\pgfusepath{clip}%
\pgfsetrectcap%
\pgfsetroundjoin%
\pgfsetlinewidth{1.505625pt}%
\definecolor{currentstroke}{rgb}{1.000000,0.705882,0.509804}%
\pgfsetstrokecolor{currentstroke}%
\pgfsetstrokeopacity{0.800000}%
\pgfsetdash{}{0pt}%
\pgfpathmoveto{\pgfqpoint{0.993071in}{1.778552in}}%
\pgfpathlineto{\pgfqpoint{5.599385in}{3.905858in}}%
\pgfusepath{stroke}%
\end{pgfscope}%
\begin{pgfscope}%
\pgfpathrectangle{\pgfqpoint{0.570343in}{0.331635in}}{\pgfqpoint{9.300000in}{7.700000in}}%
\pgfusepath{clip}%
\pgfsetrectcap%
\pgfsetroundjoin%
\pgfsetlinewidth{1.505625pt}%
\definecolor{currentstroke}{rgb}{1.000000,0.705882,0.509804}%
\pgfsetstrokecolor{currentstroke}%
\pgfsetstrokeopacity{0.800000}%
\pgfsetdash{}{0pt}%
\pgfpathmoveto{\pgfqpoint{5.153229in}{2.657489in}}%
\pgfpathlineto{\pgfqpoint{5.599385in}{3.905858in}}%
\pgfusepath{stroke}%
\end{pgfscope}%
\begin{pgfscope}%
\pgfpathrectangle{\pgfqpoint{0.570343in}{0.331635in}}{\pgfqpoint{9.300000in}{7.700000in}}%
\pgfusepath{clip}%
\pgfsetrectcap%
\pgfsetroundjoin%
\pgfsetlinewidth{1.505625pt}%
\definecolor{currentstroke}{rgb}{1.000000,0.705882,0.509804}%
\pgfsetstrokecolor{currentstroke}%
\pgfsetstrokeopacity{0.800000}%
\pgfsetdash{}{0pt}%
\pgfpathmoveto{\pgfqpoint{8.561752in}{3.104281in}}%
\pgfpathlineto{\pgfqpoint{5.599385in}{3.905858in}}%
\pgfusepath{stroke}%
\end{pgfscope}%
\begin{pgfscope}%
\pgfpathrectangle{\pgfqpoint{0.570343in}{0.331635in}}{\pgfqpoint{9.300000in}{7.700000in}}%
\pgfusepath{clip}%
\pgfsetrectcap%
\pgfsetroundjoin%
\pgfsetlinewidth{1.505625pt}%
\definecolor{currentstroke}{rgb}{1.000000,0.705882,0.509804}%
\pgfsetstrokecolor{currentstroke}%
\pgfsetstrokeopacity{0.800000}%
\pgfsetdash{}{0pt}%
\pgfpathmoveto{\pgfqpoint{4.047696in}{1.870260in}}%
\pgfpathlineto{\pgfqpoint{5.599385in}{3.905858in}}%
\pgfusepath{stroke}%
\end{pgfscope}%
\begin{pgfscope}%
\pgfpathrectangle{\pgfqpoint{0.570343in}{0.331635in}}{\pgfqpoint{9.300000in}{7.700000in}}%
\pgfusepath{clip}%
\pgfsetrectcap%
\pgfsetroundjoin%
\pgfsetlinewidth{1.505625pt}%
\definecolor{currentstroke}{rgb}{1.000000,0.705882,0.509804}%
\pgfsetstrokecolor{currentstroke}%
\pgfsetstrokeopacity{0.800000}%
\pgfsetdash{}{0pt}%
\pgfpathmoveto{\pgfqpoint{8.305186in}{3.961163in}}%
\pgfpathlineto{\pgfqpoint{5.599385in}{3.905858in}}%
\pgfusepath{stroke}%
\end{pgfscope}%
\begin{pgfscope}%
\pgfpathrectangle{\pgfqpoint{0.570343in}{0.331635in}}{\pgfqpoint{9.300000in}{7.700000in}}%
\pgfusepath{clip}%
\pgfsetrectcap%
\pgfsetroundjoin%
\pgfsetlinewidth{1.505625pt}%
\definecolor{currentstroke}{rgb}{1.000000,0.705882,0.509804}%
\pgfsetstrokecolor{currentstroke}%
\pgfsetstrokeopacity{0.800000}%
\pgfsetdash{}{0pt}%
\pgfpathmoveto{\pgfqpoint{4.619077in}{0.681635in}}%
\pgfpathlineto{\pgfqpoint{5.599385in}{3.905858in}}%
\pgfusepath{stroke}%
\end{pgfscope}%
\begin{pgfscope}%
\pgfpathrectangle{\pgfqpoint{0.570343in}{0.331635in}}{\pgfqpoint{9.300000in}{7.700000in}}%
\pgfusepath{clip}%
\pgfsetrectcap%
\pgfsetroundjoin%
\pgfsetlinewidth{1.505625pt}%
\definecolor{currentstroke}{rgb}{1.000000,0.705882,0.509804}%
\pgfsetstrokecolor{currentstroke}%
\pgfsetstrokeopacity{0.800000}%
\pgfsetdash{}{0pt}%
\pgfpathmoveto{\pgfqpoint{8.308279in}{7.681635in}}%
\pgfpathlineto{\pgfqpoint{5.599385in}{3.905858in}}%
\pgfusepath{stroke}%
\end{pgfscope}%
\begin{pgfscope}%
\pgfpathrectangle{\pgfqpoint{0.570343in}{0.331635in}}{\pgfqpoint{9.300000in}{7.700000in}}%
\pgfusepath{clip}%
\pgfsetrectcap%
\pgfsetroundjoin%
\pgfsetlinewidth{1.505625pt}%
\definecolor{currentstroke}{rgb}{1.000000,0.705882,0.509804}%
\pgfsetstrokecolor{currentstroke}%
\pgfsetstrokeopacity{0.800000}%
\pgfsetdash{}{0pt}%
\pgfpathmoveto{\pgfqpoint{6.197001in}{1.259168in}}%
\pgfpathlineto{\pgfqpoint{5.599385in}{3.905858in}}%
\pgfusepath{stroke}%
\end{pgfscope}%
\begin{pgfscope}%
\pgfpathrectangle{\pgfqpoint{0.570343in}{0.331635in}}{\pgfqpoint{9.300000in}{7.700000in}}%
\pgfusepath{clip}%
\pgfsetrectcap%
\pgfsetroundjoin%
\pgfsetlinewidth{1.505625pt}%
\definecolor{currentstroke}{rgb}{1.000000,0.705882,0.509804}%
\pgfsetstrokecolor{currentstroke}%
\pgfsetstrokeopacity{0.800000}%
\pgfsetdash{}{0pt}%
\pgfpathmoveto{\pgfqpoint{9.447616in}{5.125125in}}%
\pgfpathlineto{\pgfqpoint{5.599385in}{3.905858in}}%
\pgfusepath{stroke}%
\end{pgfscope}%
\begin{pgfscope}%
\pgfpathrectangle{\pgfqpoint{0.570343in}{0.331635in}}{\pgfqpoint{9.300000in}{7.700000in}}%
\pgfusepath{clip}%
\pgfsetrectcap%
\pgfsetroundjoin%
\pgfsetlinewidth{1.505625pt}%
\definecolor{currentstroke}{rgb}{1.000000,0.705882,0.509804}%
\pgfsetstrokecolor{currentstroke}%
\pgfsetstrokeopacity{0.800000}%
\pgfsetdash{}{0pt}%
\pgfpathmoveto{\pgfqpoint{6.034006in}{2.439042in}}%
\pgfpathlineto{\pgfqpoint{5.599385in}{3.905858in}}%
\pgfusepath{stroke}%
\end{pgfscope}%
\begin{pgfscope}%
\pgfpathrectangle{\pgfqpoint{0.570343in}{0.331635in}}{\pgfqpoint{9.300000in}{7.700000in}}%
\pgfusepath{clip}%
\pgfsetrectcap%
\pgfsetroundjoin%
\pgfsetlinewidth{1.505625pt}%
\definecolor{currentstroke}{rgb}{1.000000,0.705882,0.509804}%
\pgfsetstrokecolor{currentstroke}%
\pgfsetstrokeopacity{0.800000}%
\pgfsetdash{}{0pt}%
\pgfpathmoveto{\pgfqpoint{8.815623in}{2.199204in}}%
\pgfpathlineto{\pgfqpoint{5.599385in}{3.905858in}}%
\pgfusepath{stroke}%
\end{pgfscope}%
\begin{pgfscope}%
\pgfpathrectangle{\pgfqpoint{0.570343in}{0.331635in}}{\pgfqpoint{9.300000in}{7.700000in}}%
\pgfusepath{clip}%
\pgfsetrectcap%
\pgfsetroundjoin%
\pgfsetlinewidth{1.505625pt}%
\definecolor{currentstroke}{rgb}{1.000000,0.705882,0.509804}%
\pgfsetstrokecolor{currentstroke}%
\pgfsetstrokeopacity{0.800000}%
\pgfsetdash{}{0pt}%
\pgfpathmoveto{\pgfqpoint{7.488123in}{3.719720in}}%
\pgfpathlineto{\pgfqpoint{5.599385in}{3.905858in}}%
\pgfusepath{stroke}%
\end{pgfscope}%
\begin{pgfscope}%
\pgfpathrectangle{\pgfqpoint{0.570343in}{0.331635in}}{\pgfqpoint{9.300000in}{7.700000in}}%
\pgfusepath{clip}%
\pgfsetrectcap%
\pgfsetroundjoin%
\pgfsetlinewidth{1.505625pt}%
\definecolor{currentstroke}{rgb}{1.000000,0.705882,0.509804}%
\pgfsetstrokecolor{currentstroke}%
\pgfsetstrokeopacity{0.800000}%
\pgfsetdash{}{0pt}%
\pgfpathmoveto{\pgfqpoint{2.671621in}{4.844973in}}%
\pgfpathlineto{\pgfqpoint{5.599385in}{3.905858in}}%
\pgfusepath{stroke}%
\end{pgfscope}%
\begin{pgfscope}%
\pgfpathrectangle{\pgfqpoint{0.570343in}{0.331635in}}{\pgfqpoint{9.300000in}{7.700000in}}%
\pgfusepath{clip}%
\pgfsetrectcap%
\pgfsetroundjoin%
\pgfsetlinewidth{1.505625pt}%
\definecolor{currentstroke}{rgb}{1.000000,0.705882,0.509804}%
\pgfsetstrokecolor{currentstroke}%
\pgfsetstrokeopacity{0.800000}%
\pgfsetdash{}{0pt}%
\pgfpathmoveto{\pgfqpoint{7.154996in}{7.029196in}}%
\pgfpathlineto{\pgfqpoint{5.599385in}{3.905858in}}%
\pgfusepath{stroke}%
\end{pgfscope}%
\begin{pgfscope}%
\pgfpathrectangle{\pgfqpoint{0.570343in}{0.331635in}}{\pgfqpoint{9.300000in}{7.700000in}}%
\pgfusepath{clip}%
\pgfsetrectcap%
\pgfsetroundjoin%
\pgfsetlinewidth{1.505625pt}%
\definecolor{currentstroke}{rgb}{1.000000,0.705882,0.509804}%
\pgfsetstrokecolor{currentstroke}%
\pgfsetstrokeopacity{0.800000}%
\pgfsetdash{}{0pt}%
\pgfpathmoveto{\pgfqpoint{3.897376in}{3.003663in}}%
\pgfpathlineto{\pgfqpoint{5.599385in}{3.905858in}}%
\pgfusepath{stroke}%
\end{pgfscope}%
\begin{pgfscope}%
\pgfpathrectangle{\pgfqpoint{0.570343in}{0.331635in}}{\pgfqpoint{9.300000in}{7.700000in}}%
\pgfusepath{clip}%
\pgfsetrectcap%
\pgfsetroundjoin%
\pgfsetlinewidth{1.505625pt}%
\definecolor{currentstroke}{rgb}{1.000000,0.705882,0.509804}%
\pgfsetstrokecolor{currentstroke}%
\pgfsetstrokeopacity{0.800000}%
\pgfsetdash{}{0pt}%
\pgfpathmoveto{\pgfqpoint{5.935607in}{5.739614in}}%
\pgfpathlineto{\pgfqpoint{5.599385in}{3.905858in}}%
\pgfusepath{stroke}%
\end{pgfscope}%
\begin{pgfscope}%
\pgfsetrectcap%
\pgfsetmiterjoin%
\pgfsetlinewidth{0.803000pt}%
\definecolor{currentstroke}{rgb}{0.000000,0.000000,0.000000}%
\pgfsetstrokecolor{currentstroke}%
\pgfsetdash{}{0pt}%
\pgfpathmoveto{\pgfqpoint{0.570343in}{0.331635in}}%
\pgfpathlineto{\pgfqpoint{0.570343in}{8.031635in}}%
\pgfusepath{stroke}%
\end{pgfscope}%
\begin{pgfscope}%
\pgfsetrectcap%
\pgfsetmiterjoin%
\pgfsetlinewidth{0.803000pt}%
\definecolor{currentstroke}{rgb}{0.000000,0.000000,0.000000}%
\pgfsetstrokecolor{currentstroke}%
\pgfsetdash{}{0pt}%
\pgfpathmoveto{\pgfqpoint{9.870343in}{0.331635in}}%
\pgfpathlineto{\pgfqpoint{9.870343in}{8.031635in}}%
\pgfusepath{stroke}%
\end{pgfscope}%
\begin{pgfscope}%
\pgfsetrectcap%
\pgfsetmiterjoin%
\pgfsetlinewidth{0.803000pt}%
\definecolor{currentstroke}{rgb}{0.000000,0.000000,0.000000}%
\pgfsetstrokecolor{currentstroke}%
\pgfsetdash{}{0pt}%
\pgfpathmoveto{\pgfqpoint{0.570343in}{0.331635in}}%
\pgfpathlineto{\pgfqpoint{9.870343in}{0.331635in}}%
\pgfusepath{stroke}%
\end{pgfscope}%
\begin{pgfscope}%
\pgfsetrectcap%
\pgfsetmiterjoin%
\pgfsetlinewidth{0.803000pt}%
\definecolor{currentstroke}{rgb}{0.000000,0.000000,0.000000}%
\pgfsetstrokecolor{currentstroke}%
\pgfsetdash{}{0pt}%
\pgfpathmoveto{\pgfqpoint{0.570343in}{8.031635in}}%
\pgfpathlineto{\pgfqpoint{9.870343in}{8.031635in}}%
\pgfusepath{stroke}%
\end{pgfscope}%
\begin{pgfscope}%
\definecolor{textcolor}{rgb}{0.000000,0.000000,0.000000}%
\pgfsetstrokecolor{textcolor}%
\pgfsetfillcolor{textcolor}%
\pgftext[x=5.220343in,y=8.114968in,,base]{\color{textcolor}\sffamily\fontsize{12.000000}{14.400000}\selectfont Photo-Realistic Images}%
\end{pgfscope}%
\begin{pgfscope}%
\pgfsetbuttcap%
\pgfsetmiterjoin%
\definecolor{currentfill}{rgb}{1.000000,1.000000,1.000000}%
\pgfsetfillcolor{currentfill}%
\pgfsetfillopacity{0.800000}%
\pgfsetlinewidth{1.003750pt}%
\definecolor{currentstroke}{rgb}{0.800000,0.800000,0.800000}%
\pgfsetstrokecolor{currentstroke}%
\pgfsetstrokeopacity{0.800000}%
\pgfsetdash{}{0pt}%
\pgfpathmoveto{\pgfqpoint{9.967566in}{3.956944in}}%
\pgfpathlineto{\pgfqpoint{11.121442in}{3.956944in}}%
\pgfpathquadraticcurveto{\pgfqpoint{11.149220in}{3.956944in}}{\pgfqpoint{11.149220in}{3.984722in}}%
\pgfpathlineto{\pgfqpoint{11.149220in}{4.378548in}}%
\pgfpathquadraticcurveto{\pgfqpoint{11.149220in}{4.406326in}}{\pgfqpoint{11.121442in}{4.406326in}}%
\pgfpathlineto{\pgfqpoint{9.967566in}{4.406326in}}%
\pgfpathquadraticcurveto{\pgfqpoint{9.939788in}{4.406326in}}{\pgfqpoint{9.939788in}{4.378548in}}%
\pgfpathlineto{\pgfqpoint{9.939788in}{3.984722in}}%
\pgfpathquadraticcurveto{\pgfqpoint{9.939788in}{3.956944in}}{\pgfqpoint{9.967566in}{3.956944in}}%
\pgfpathclose%
\pgfusepath{stroke,fill}%
\end{pgfscope}%
\begin{pgfscope}%
\pgfsetbuttcap%
\pgfsetroundjoin%
\definecolor{currentfill}{rgb}{0.631373,0.788235,0.956863}%
\pgfsetfillcolor{currentfill}%
\pgfsetlinewidth{1.003750pt}%
\definecolor{currentstroke}{rgb}{0.631373,0.788235,0.956863}%
\pgfsetstrokecolor{currentstroke}%
\pgfsetdash{}{0pt}%
\pgfsys@defobject{currentmarker}{\pgfqpoint{-0.041667in}{-0.041667in}}{\pgfqpoint{0.041667in}{0.041667in}}{%
\pgfpathmoveto{\pgfqpoint{0.000000in}{-0.041667in}}%
\pgfpathcurveto{\pgfqpoint{0.011050in}{-0.041667in}}{\pgfqpoint{0.021649in}{-0.037276in}}{\pgfqpoint{0.029463in}{-0.029463in}}%
\pgfpathcurveto{\pgfqpoint{0.037276in}{-0.021649in}}{\pgfqpoint{0.041667in}{-0.011050in}}{\pgfqpoint{0.041667in}{0.000000in}}%
\pgfpathcurveto{\pgfqpoint{0.041667in}{0.011050in}}{\pgfqpoint{0.037276in}{0.021649in}}{\pgfqpoint{0.029463in}{0.029463in}}%
\pgfpathcurveto{\pgfqpoint{0.021649in}{0.037276in}}{\pgfqpoint{0.011050in}{0.041667in}}{\pgfqpoint{0.000000in}{0.041667in}}%
\pgfpathcurveto{\pgfqpoint{-0.011050in}{0.041667in}}{\pgfqpoint{-0.021649in}{0.037276in}}{\pgfqpoint{-0.029463in}{0.029463in}}%
\pgfpathcurveto{\pgfqpoint{-0.037276in}{0.021649in}}{\pgfqpoint{-0.041667in}{0.011050in}}{\pgfqpoint{-0.041667in}{0.000000in}}%
\pgfpathcurveto{\pgfqpoint{-0.041667in}{-0.011050in}}{\pgfqpoint{-0.037276in}{-0.021649in}}{\pgfqpoint{-0.029463in}{-0.029463in}}%
\pgfpathcurveto{\pgfqpoint{-0.021649in}{-0.037276in}}{\pgfqpoint{-0.011050in}{-0.041667in}}{\pgfqpoint{0.000000in}{-0.041667in}}%
\pgfpathclose%
\pgfusepath{stroke,fill}%
}%
\begin{pgfscope}%
\pgfsys@transformshift{10.134232in}{4.281705in}%
\pgfsys@useobject{currentmarker}{}%
\end{pgfscope}%
\end{pgfscope}%
\begin{pgfscope}%
\definecolor{textcolor}{rgb}{0.000000,0.000000,0.000000}%
\pgfsetstrokecolor{textcolor}%
\pgfsetfillcolor{textcolor}%
\pgftext[x=10.384232in,y=4.245247in,left,base]{\color{textcolor}\sffamily\fontsize{10.000000}{12.000000}\selectfont s2r-3dfree}%
\end{pgfscope}%
\begin{pgfscope}%
\pgfsetbuttcap%
\pgfsetroundjoin%
\definecolor{currentfill}{rgb}{1.000000,0.705882,0.509804}%
\pgfsetfillcolor{currentfill}%
\pgfsetlinewidth{1.003750pt}%
\definecolor{currentstroke}{rgb}{1.000000,0.705882,0.509804}%
\pgfsetstrokecolor{currentstroke}%
\pgfsetdash{}{0pt}%
\pgfsys@defobject{currentmarker}{\pgfqpoint{-0.041667in}{-0.041667in}}{\pgfqpoint{0.041667in}{0.041667in}}{%
\pgfpathmoveto{\pgfqpoint{0.000000in}{-0.041667in}}%
\pgfpathcurveto{\pgfqpoint{0.011050in}{-0.041667in}}{\pgfqpoint{0.021649in}{-0.037276in}}{\pgfqpoint{0.029463in}{-0.029463in}}%
\pgfpathcurveto{\pgfqpoint{0.037276in}{-0.021649in}}{\pgfqpoint{0.041667in}{-0.011050in}}{\pgfqpoint{0.041667in}{0.000000in}}%
\pgfpathcurveto{\pgfqpoint{0.041667in}{0.011050in}}{\pgfqpoint{0.037276in}{0.021649in}}{\pgfqpoint{0.029463in}{0.029463in}}%
\pgfpathcurveto{\pgfqpoint{0.021649in}{0.037276in}}{\pgfqpoint{0.011050in}{0.041667in}}{\pgfqpoint{0.000000in}{0.041667in}}%
\pgfpathcurveto{\pgfqpoint{-0.011050in}{0.041667in}}{\pgfqpoint{-0.021649in}{0.037276in}}{\pgfqpoint{-0.029463in}{0.029463in}}%
\pgfpathcurveto{\pgfqpoint{-0.037276in}{0.021649in}}{\pgfqpoint{-0.041667in}{0.011050in}}{\pgfqpoint{-0.041667in}{0.000000in}}%
\pgfpathcurveto{\pgfqpoint{-0.041667in}{-0.011050in}}{\pgfqpoint{-0.037276in}{-0.021649in}}{\pgfqpoint{-0.029463in}{-0.029463in}}%
\pgfpathcurveto{\pgfqpoint{-0.021649in}{-0.037276in}}{\pgfqpoint{-0.011050in}{-0.041667in}}{\pgfqpoint{0.000000in}{-0.041667in}}%
\pgfpathclose%
\pgfusepath{stroke,fill}%
}%
\begin{pgfscope}%
\pgfsys@transformshift{10.134232in}{4.077848in}%
\pgfsys@useobject{currentmarker}{}%
\end{pgfscope}%
\end{pgfscope}%
\begin{pgfscope}%
\definecolor{textcolor}{rgb}{0.000000,0.000000,0.000000}%
\pgfsetstrokecolor{textcolor}%
\pgfsetfillcolor{textcolor}%
\pgftext[x=10.384232in,y=4.041390in,left,base]{\color{textcolor}\sffamily\fontsize{10.000000}{12.000000}\selectfont pix3d}%
\end{pgfscope}%
\end{pgfpicture}%
\makeatother%
\endgroup%
}\\
    \caption[\gls{tsne} comparison for Real and Synthetic Dataset.]{\gls{tsne} visualization for images from individual photo-realistic synthetic dataset compared with Pix3D latent space.
        (Left to right, top to bottom) Openrooms, SceneNet, Blenderproc, \gls{ai2thor}, \gls{front}, Hyperism, InteriorNet, \gls{free} in blue;
        compared with Pix3D in orange.}
    \label{fig:tsne per dataset}
\end{figure}

For each dataset, 30 randomly chosen images were passed through the encoder at the same time.
The images are less in quantity because not all datasets provide images directly.
Some datasets like Hyperism, SceneNet, Openrooms are built for training \gls{slam} models, and thus not all frames contain furniture.
We had to filter the images containing furniture to appear to be in the same embedding space.

In \autoref{fig:photorealistic tsne}, we see the embedding for each dataset used for the survey discussed in \autoref{sec:a-survey-on-photorealism}.
For better visualization, the points are connected to the centroid of the dataset;
if not, we found it challenging to comprehend the scatterplot of all datasets in a single graph.
The proposed \gls{free} and the real dataset, Pix3D, are highlighted for better focus.

Each of the dataset have considerable overlap of latent space as seen in \autoref{fig:tsne per dataset}, which indicate that each of the dataset is closer to the photorealism of a real-world dataset.
From the \gls{tsne} visualization in \autoref{fig:photorealistic tsne}, we see that the real dataset(Pix3D) is spread across the space and lies at the center of the plot.
Interestingly, all the automated datasets(Openrooms, SceneNet, Blenderproc, \gls{free}) form cluster close to each other, while the non-automated datasets(Hyperism, InteriorNet, \gls{front}, \gls{ai2thor}) form cluster on the other side.
\gls{ai2thor}, openrooms, Hyperism, and \gls{front} have their embedding mapped in the outer region.
Openrooms seems to have least overlap with Pix3D which could indicate a gap in the two domains.
\gls{free} has a widespread and occupies much of the space occupied by Pix3D.
This can also be seen in \autoref{fig:pix3d_s2r3dfree}.
For this visualization, 280  images were randomly selected from each dataset, with 40 images belonging to each category.
However, the \gls{free} needs more randomization to encapsulate the latent space of Pix3D for better performance.

\begin{figure}[ht]
    \centering
    \resizebox{0.75\textwidth}{8.5cm}{%% Creator: Matplotlib, PGF backend
%%
%% To include the figure in your LaTeX document, write
%%   \input{<filename>.pgf}
%%
%% Make sure the required packages are loaded in your preamble
%%   \usepackage{pgf}
%%
%% Figures using additional raster images can only be included by \input if
%% they are in the same directory as the main LaTeX file. For loading figures
%% from other directories you can use the `import` package
%%   \usepackage{import}
%%
%% and then include the figures with
%%   \import{<path to file>}{<filename>.pgf}
%%
%% Matplotlib used the following preamble
%%   \usepackage{fontspec}
%%   \setmainfont{DejaVuSerif.ttf}[Path=\detokenize{/Users/apple/opt/anaconda3/envs/kaolin/lib/python3.7/site-packages/matplotlib/mpl-data/fonts/ttf/}]
%%   \setsansfont{DejaVuSans.ttf}[Path=\detokenize{/Users/apple/opt/anaconda3/envs/kaolin/lib/python3.7/site-packages/matplotlib/mpl-data/fonts/ttf/}]
%%   \setmonofont{DejaVuSansMono.ttf}[Path=\detokenize{/Users/apple/opt/anaconda3/envs/kaolin/lib/python3.7/site-packages/matplotlib/mpl-data/fonts/ttf/}]
%%
\begingroup%
\makeatletter%
\begin{pgfpicture}%
\pgfpathrectangle{\pgfpointorigin}{\pgfqpoint{11.314188in}{8.341596in}}%
\pgfusepath{use as bounding box, clip}%
\begin{pgfscope}%
\pgfsetbuttcap%
\pgfsetmiterjoin%
\definecolor{currentfill}{rgb}{1.000000,1.000000,1.000000}%
\pgfsetfillcolor{currentfill}%
\pgfsetlinewidth{0.000000pt}%
\definecolor{currentstroke}{rgb}{1.000000,1.000000,1.000000}%
\pgfsetstrokecolor{currentstroke}%
\pgfsetdash{}{0pt}%
\pgfpathmoveto{\pgfqpoint{0.000000in}{0.000000in}}%
\pgfpathlineto{\pgfqpoint{11.314188in}{0.000000in}}%
\pgfpathlineto{\pgfqpoint{11.314188in}{8.341596in}}%
\pgfpathlineto{\pgfqpoint{0.000000in}{8.341596in}}%
\pgfpathclose%
\pgfusepath{fill}%
\end{pgfscope}%
\begin{pgfscope}%
\pgfsetbuttcap%
\pgfsetmiterjoin%
\definecolor{currentfill}{rgb}{1.000000,1.000000,1.000000}%
\pgfsetfillcolor{currentfill}%
\pgfsetlinewidth{0.000000pt}%
\definecolor{currentstroke}{rgb}{0.000000,0.000000,0.000000}%
\pgfsetstrokecolor{currentstroke}%
\pgfsetstrokeopacity{0.000000}%
\pgfsetdash{}{0pt}%
\pgfpathmoveto{\pgfqpoint{0.481978in}{0.331635in}}%
\pgfpathlineto{\pgfqpoint{9.781978in}{0.331635in}}%
\pgfpathlineto{\pgfqpoint{9.781978in}{8.031635in}}%
\pgfpathlineto{\pgfqpoint{0.481978in}{8.031635in}}%
\pgfpathclose%
\pgfusepath{fill}%
\end{pgfscope}%
\begin{pgfscope}%
\pgfpathrectangle{\pgfqpoint{0.481978in}{0.331635in}}{\pgfqpoint{9.300000in}{7.700000in}}%
\pgfusepath{clip}%
\pgfsetbuttcap%
\pgfsetroundjoin%
\definecolor{currentfill}{rgb}{0.631373,0.788235,0.956863}%
\pgfsetfillcolor{currentfill}%
\pgfsetlinewidth{0.481800pt}%
\definecolor{currentstroke}{rgb}{1.000000,1.000000,1.000000}%
\pgfsetstrokecolor{currentstroke}%
\pgfsetdash{}{0pt}%
\pgfpathmoveto{\pgfqpoint{5.562292in}{2.004512in}}%
\pgfpathcurveto{\pgfqpoint{5.573342in}{2.004512in}}{\pgfqpoint{5.583941in}{2.008902in}}{\pgfqpoint{5.591755in}{2.016716in}}%
\pgfpathcurveto{\pgfqpoint{5.599569in}{2.024529in}}{\pgfqpoint{5.603959in}{2.035128in}}{\pgfqpoint{5.603959in}{2.046179in}}%
\pgfpathcurveto{\pgfqpoint{5.603959in}{2.057229in}}{\pgfqpoint{5.599569in}{2.067828in}}{\pgfqpoint{5.591755in}{2.075641in}}%
\pgfpathcurveto{\pgfqpoint{5.583941in}{2.083455in}}{\pgfqpoint{5.573342in}{2.087845in}}{\pgfqpoint{5.562292in}{2.087845in}}%
\pgfpathcurveto{\pgfqpoint{5.551242in}{2.087845in}}{\pgfqpoint{5.540643in}{2.083455in}}{\pgfqpoint{5.532829in}{2.075641in}}%
\pgfpathcurveto{\pgfqpoint{5.525016in}{2.067828in}}{\pgfqpoint{5.520625in}{2.057229in}}{\pgfqpoint{5.520625in}{2.046179in}}%
\pgfpathcurveto{\pgfqpoint{5.520625in}{2.035128in}}{\pgfqpoint{5.525016in}{2.024529in}}{\pgfqpoint{5.532829in}{2.016716in}}%
\pgfpathcurveto{\pgfqpoint{5.540643in}{2.008902in}}{\pgfqpoint{5.551242in}{2.004512in}}{\pgfqpoint{5.562292in}{2.004512in}}%
\pgfpathclose%
\pgfusepath{stroke,fill}%
\end{pgfscope}%
\begin{pgfscope}%
\pgfpathrectangle{\pgfqpoint{0.481978in}{0.331635in}}{\pgfqpoint{9.300000in}{7.700000in}}%
\pgfusepath{clip}%
\pgfsetbuttcap%
\pgfsetroundjoin%
\definecolor{currentfill}{rgb}{0.631373,0.788235,0.956863}%
\pgfsetfillcolor{currentfill}%
\pgfsetlinewidth{0.481800pt}%
\definecolor{currentstroke}{rgb}{1.000000,1.000000,1.000000}%
\pgfsetstrokecolor{currentstroke}%
\pgfsetdash{}{0pt}%
\pgfpathmoveto{\pgfqpoint{6.909912in}{4.418435in}}%
\pgfpathcurveto{\pgfqpoint{6.920962in}{4.418435in}}{\pgfqpoint{6.931561in}{4.422826in}}{\pgfqpoint{6.939375in}{4.430639in}}%
\pgfpathcurveto{\pgfqpoint{6.947188in}{4.438453in}}{\pgfqpoint{6.951578in}{4.449052in}}{\pgfqpoint{6.951578in}{4.460102in}}%
\pgfpathcurveto{\pgfqpoint{6.951578in}{4.471152in}}{\pgfqpoint{6.947188in}{4.481751in}}{\pgfqpoint{6.939375in}{4.489565in}}%
\pgfpathcurveto{\pgfqpoint{6.931561in}{4.497378in}}{\pgfqpoint{6.920962in}{4.501769in}}{\pgfqpoint{6.909912in}{4.501769in}}%
\pgfpathcurveto{\pgfqpoint{6.898862in}{4.501769in}}{\pgfqpoint{6.888263in}{4.497378in}}{\pgfqpoint{6.880449in}{4.489565in}}%
\pgfpathcurveto{\pgfqpoint{6.872635in}{4.481751in}}{\pgfqpoint{6.868245in}{4.471152in}}{\pgfqpoint{6.868245in}{4.460102in}}%
\pgfpathcurveto{\pgfqpoint{6.868245in}{4.449052in}}{\pgfqpoint{6.872635in}{4.438453in}}{\pgfqpoint{6.880449in}{4.430639in}}%
\pgfpathcurveto{\pgfqpoint{6.888263in}{4.422826in}}{\pgfqpoint{6.898862in}{4.418435in}}{\pgfqpoint{6.909912in}{4.418435in}}%
\pgfpathclose%
\pgfusepath{stroke,fill}%
\end{pgfscope}%
\begin{pgfscope}%
\pgfpathrectangle{\pgfqpoint{0.481978in}{0.331635in}}{\pgfqpoint{9.300000in}{7.700000in}}%
\pgfusepath{clip}%
\pgfsetbuttcap%
\pgfsetroundjoin%
\definecolor{currentfill}{rgb}{0.631373,0.788235,0.956863}%
\pgfsetfillcolor{currentfill}%
\pgfsetlinewidth{0.481800pt}%
\definecolor{currentstroke}{rgb}{1.000000,1.000000,1.000000}%
\pgfsetstrokecolor{currentstroke}%
\pgfsetdash{}{0pt}%
\pgfpathmoveto{\pgfqpoint{7.565874in}{5.841434in}}%
\pgfpathcurveto{\pgfqpoint{7.576924in}{5.841434in}}{\pgfqpoint{7.587523in}{5.845824in}}{\pgfqpoint{7.595337in}{5.853638in}}%
\pgfpathcurveto{\pgfqpoint{7.603150in}{5.861451in}}{\pgfqpoint{7.607541in}{5.872050in}}{\pgfqpoint{7.607541in}{5.883101in}}%
\pgfpathcurveto{\pgfqpoint{7.607541in}{5.894151in}}{\pgfqpoint{7.603150in}{5.904750in}}{\pgfqpoint{7.595337in}{5.912563in}}%
\pgfpathcurveto{\pgfqpoint{7.587523in}{5.920377in}}{\pgfqpoint{7.576924in}{5.924767in}}{\pgfqpoint{7.565874in}{5.924767in}}%
\pgfpathcurveto{\pgfqpoint{7.554824in}{5.924767in}}{\pgfqpoint{7.544225in}{5.920377in}}{\pgfqpoint{7.536411in}{5.912563in}}%
\pgfpathcurveto{\pgfqpoint{7.528598in}{5.904750in}}{\pgfqpoint{7.524207in}{5.894151in}}{\pgfqpoint{7.524207in}{5.883101in}}%
\pgfpathcurveto{\pgfqpoint{7.524207in}{5.872050in}}{\pgfqpoint{7.528598in}{5.861451in}}{\pgfqpoint{7.536411in}{5.853638in}}%
\pgfpathcurveto{\pgfqpoint{7.544225in}{5.845824in}}{\pgfqpoint{7.554824in}{5.841434in}}{\pgfqpoint{7.565874in}{5.841434in}}%
\pgfpathclose%
\pgfusepath{stroke,fill}%
\end{pgfscope}%
\begin{pgfscope}%
\pgfpathrectangle{\pgfqpoint{0.481978in}{0.331635in}}{\pgfqpoint{9.300000in}{7.700000in}}%
\pgfusepath{clip}%
\pgfsetbuttcap%
\pgfsetroundjoin%
\definecolor{currentfill}{rgb}{0.631373,0.788235,0.956863}%
\pgfsetfillcolor{currentfill}%
\pgfsetlinewidth{0.481800pt}%
\definecolor{currentstroke}{rgb}{1.000000,1.000000,1.000000}%
\pgfsetstrokecolor{currentstroke}%
\pgfsetdash{}{0pt}%
\pgfpathmoveto{\pgfqpoint{8.104384in}{5.147465in}}%
\pgfpathcurveto{\pgfqpoint{8.115434in}{5.147465in}}{\pgfqpoint{8.126033in}{5.151855in}}{\pgfqpoint{8.133846in}{5.159669in}}%
\pgfpathcurveto{\pgfqpoint{8.141660in}{5.167482in}}{\pgfqpoint{8.146050in}{5.178081in}}{\pgfqpoint{8.146050in}{5.189132in}}%
\pgfpathcurveto{\pgfqpoint{8.146050in}{5.200182in}}{\pgfqpoint{8.141660in}{5.210781in}}{\pgfqpoint{8.133846in}{5.218594in}}%
\pgfpathcurveto{\pgfqpoint{8.126033in}{5.226408in}}{\pgfqpoint{8.115434in}{5.230798in}}{\pgfqpoint{8.104384in}{5.230798in}}%
\pgfpathcurveto{\pgfqpoint{8.093334in}{5.230798in}}{\pgfqpoint{8.082735in}{5.226408in}}{\pgfqpoint{8.074921in}{5.218594in}}%
\pgfpathcurveto{\pgfqpoint{8.067107in}{5.210781in}}{\pgfqpoint{8.062717in}{5.200182in}}{\pgfqpoint{8.062717in}{5.189132in}}%
\pgfpathcurveto{\pgfqpoint{8.062717in}{5.178081in}}{\pgfqpoint{8.067107in}{5.167482in}}{\pgfqpoint{8.074921in}{5.159669in}}%
\pgfpathcurveto{\pgfqpoint{8.082735in}{5.151855in}}{\pgfqpoint{8.093334in}{5.147465in}}{\pgfqpoint{8.104384in}{5.147465in}}%
\pgfpathclose%
\pgfusepath{stroke,fill}%
\end{pgfscope}%
\begin{pgfscope}%
\pgfpathrectangle{\pgfqpoint{0.481978in}{0.331635in}}{\pgfqpoint{9.300000in}{7.700000in}}%
\pgfusepath{clip}%
\pgfsetbuttcap%
\pgfsetroundjoin%
\definecolor{currentfill}{rgb}{0.631373,0.788235,0.956863}%
\pgfsetfillcolor{currentfill}%
\pgfsetlinewidth{0.481800pt}%
\definecolor{currentstroke}{rgb}{1.000000,1.000000,1.000000}%
\pgfsetstrokecolor{currentstroke}%
\pgfsetdash{}{0pt}%
\pgfpathmoveto{\pgfqpoint{7.967295in}{4.784121in}}%
\pgfpathcurveto{\pgfqpoint{7.978345in}{4.784121in}}{\pgfqpoint{7.988944in}{4.788511in}}{\pgfqpoint{7.996758in}{4.796325in}}%
\pgfpathcurveto{\pgfqpoint{8.004571in}{4.804138in}}{\pgfqpoint{8.008962in}{4.814737in}}{\pgfqpoint{8.008962in}{4.825787in}}%
\pgfpathcurveto{\pgfqpoint{8.008962in}{4.836837in}}{\pgfqpoint{8.004571in}{4.847437in}}{\pgfqpoint{7.996758in}{4.855250in}}%
\pgfpathcurveto{\pgfqpoint{7.988944in}{4.863064in}}{\pgfqpoint{7.978345in}{4.867454in}}{\pgfqpoint{7.967295in}{4.867454in}}%
\pgfpathcurveto{\pgfqpoint{7.956245in}{4.867454in}}{\pgfqpoint{7.945646in}{4.863064in}}{\pgfqpoint{7.937832in}{4.855250in}}%
\pgfpathcurveto{\pgfqpoint{7.930019in}{4.847437in}}{\pgfqpoint{7.925628in}{4.836837in}}{\pgfqpoint{7.925628in}{4.825787in}}%
\pgfpathcurveto{\pgfqpoint{7.925628in}{4.814737in}}{\pgfqpoint{7.930019in}{4.804138in}}{\pgfqpoint{7.937832in}{4.796325in}}%
\pgfpathcurveto{\pgfqpoint{7.945646in}{4.788511in}}{\pgfqpoint{7.956245in}{4.784121in}}{\pgfqpoint{7.967295in}{4.784121in}}%
\pgfpathclose%
\pgfusepath{stroke,fill}%
\end{pgfscope}%
\begin{pgfscope}%
\pgfpathrectangle{\pgfqpoint{0.481978in}{0.331635in}}{\pgfqpoint{9.300000in}{7.700000in}}%
\pgfusepath{clip}%
\pgfsetbuttcap%
\pgfsetroundjoin%
\definecolor{currentfill}{rgb}{0.631373,0.788235,0.956863}%
\pgfsetfillcolor{currentfill}%
\pgfsetlinewidth{0.481800pt}%
\definecolor{currentstroke}{rgb}{1.000000,1.000000,1.000000}%
\pgfsetstrokecolor{currentstroke}%
\pgfsetdash{}{0pt}%
\pgfpathmoveto{\pgfqpoint{5.184843in}{3.663871in}}%
\pgfpathcurveto{\pgfqpoint{5.195893in}{3.663871in}}{\pgfqpoint{5.206492in}{3.668261in}}{\pgfqpoint{5.214305in}{3.676074in}}%
\pgfpathcurveto{\pgfqpoint{5.222119in}{3.683888in}}{\pgfqpoint{5.226509in}{3.694487in}}{\pgfqpoint{5.226509in}{3.705537in}}%
\pgfpathcurveto{\pgfqpoint{5.226509in}{3.716587in}}{\pgfqpoint{5.222119in}{3.727186in}}{\pgfqpoint{5.214305in}{3.735000in}}%
\pgfpathcurveto{\pgfqpoint{5.206492in}{3.742814in}}{\pgfqpoint{5.195893in}{3.747204in}}{\pgfqpoint{5.184843in}{3.747204in}}%
\pgfpathcurveto{\pgfqpoint{5.173793in}{3.747204in}}{\pgfqpoint{5.163194in}{3.742814in}}{\pgfqpoint{5.155380in}{3.735000in}}%
\pgfpathcurveto{\pgfqpoint{5.147566in}{3.727186in}}{\pgfqpoint{5.143176in}{3.716587in}}{\pgfqpoint{5.143176in}{3.705537in}}%
\pgfpathcurveto{\pgfqpoint{5.143176in}{3.694487in}}{\pgfqpoint{5.147566in}{3.683888in}}{\pgfqpoint{5.155380in}{3.676074in}}%
\pgfpathcurveto{\pgfqpoint{5.163194in}{3.668261in}}{\pgfqpoint{5.173793in}{3.663871in}}{\pgfqpoint{5.184843in}{3.663871in}}%
\pgfpathclose%
\pgfusepath{stroke,fill}%
\end{pgfscope}%
\begin{pgfscope}%
\pgfpathrectangle{\pgfqpoint{0.481978in}{0.331635in}}{\pgfqpoint{9.300000in}{7.700000in}}%
\pgfusepath{clip}%
\pgfsetbuttcap%
\pgfsetroundjoin%
\definecolor{currentfill}{rgb}{0.631373,0.788235,0.956863}%
\pgfsetfillcolor{currentfill}%
\pgfsetlinewidth{0.481800pt}%
\definecolor{currentstroke}{rgb}{1.000000,1.000000,1.000000}%
\pgfsetstrokecolor{currentstroke}%
\pgfsetdash{}{0pt}%
\pgfpathmoveto{\pgfqpoint{2.850031in}{2.106168in}}%
\pgfpathcurveto{\pgfqpoint{2.861081in}{2.106168in}}{\pgfqpoint{2.871680in}{2.110558in}}{\pgfqpoint{2.879493in}{2.118372in}}%
\pgfpathcurveto{\pgfqpoint{2.887307in}{2.126186in}}{\pgfqpoint{2.891697in}{2.136785in}}{\pgfqpoint{2.891697in}{2.147835in}}%
\pgfpathcurveto{\pgfqpoint{2.891697in}{2.158885in}}{\pgfqpoint{2.887307in}{2.169484in}}{\pgfqpoint{2.879493in}{2.177298in}}%
\pgfpathcurveto{\pgfqpoint{2.871680in}{2.185111in}}{\pgfqpoint{2.861081in}{2.189501in}}{\pgfqpoint{2.850031in}{2.189501in}}%
\pgfpathcurveto{\pgfqpoint{2.838980in}{2.189501in}}{\pgfqpoint{2.828381in}{2.185111in}}{\pgfqpoint{2.820568in}{2.177298in}}%
\pgfpathcurveto{\pgfqpoint{2.812754in}{2.169484in}}{\pgfqpoint{2.808364in}{2.158885in}}{\pgfqpoint{2.808364in}{2.147835in}}%
\pgfpathcurveto{\pgfqpoint{2.808364in}{2.136785in}}{\pgfqpoint{2.812754in}{2.126186in}}{\pgfqpoint{2.820568in}{2.118372in}}%
\pgfpathcurveto{\pgfqpoint{2.828381in}{2.110558in}}{\pgfqpoint{2.838980in}{2.106168in}}{\pgfqpoint{2.850031in}{2.106168in}}%
\pgfpathclose%
\pgfusepath{stroke,fill}%
\end{pgfscope}%
\begin{pgfscope}%
\pgfpathrectangle{\pgfqpoint{0.481978in}{0.331635in}}{\pgfqpoint{9.300000in}{7.700000in}}%
\pgfusepath{clip}%
\pgfsetbuttcap%
\pgfsetroundjoin%
\definecolor{currentfill}{rgb}{0.631373,0.788235,0.956863}%
\pgfsetfillcolor{currentfill}%
\pgfsetlinewidth{0.481800pt}%
\definecolor{currentstroke}{rgb}{1.000000,1.000000,1.000000}%
\pgfsetstrokecolor{currentstroke}%
\pgfsetdash{}{0pt}%
\pgfpathmoveto{\pgfqpoint{5.765949in}{2.898526in}}%
\pgfpathcurveto{\pgfqpoint{5.777000in}{2.898526in}}{\pgfqpoint{5.787599in}{2.902916in}}{\pgfqpoint{5.795412in}{2.910730in}}%
\pgfpathcurveto{\pgfqpoint{5.803226in}{2.918544in}}{\pgfqpoint{5.807616in}{2.929143in}}{\pgfqpoint{5.807616in}{2.940193in}}%
\pgfpathcurveto{\pgfqpoint{5.807616in}{2.951243in}}{\pgfqpoint{5.803226in}{2.961842in}}{\pgfqpoint{5.795412in}{2.969656in}}%
\pgfpathcurveto{\pgfqpoint{5.787599in}{2.977469in}}{\pgfqpoint{5.777000in}{2.981860in}}{\pgfqpoint{5.765949in}{2.981860in}}%
\pgfpathcurveto{\pgfqpoint{5.754899in}{2.981860in}}{\pgfqpoint{5.744300in}{2.977469in}}{\pgfqpoint{5.736487in}{2.969656in}}%
\pgfpathcurveto{\pgfqpoint{5.728673in}{2.961842in}}{\pgfqpoint{5.724283in}{2.951243in}}{\pgfqpoint{5.724283in}{2.940193in}}%
\pgfpathcurveto{\pgfqpoint{5.724283in}{2.929143in}}{\pgfqpoint{5.728673in}{2.918544in}}{\pgfqpoint{5.736487in}{2.910730in}}%
\pgfpathcurveto{\pgfqpoint{5.744300in}{2.902916in}}{\pgfqpoint{5.754899in}{2.898526in}}{\pgfqpoint{5.765949in}{2.898526in}}%
\pgfpathclose%
\pgfusepath{stroke,fill}%
\end{pgfscope}%
\begin{pgfscope}%
\pgfpathrectangle{\pgfqpoint{0.481978in}{0.331635in}}{\pgfqpoint{9.300000in}{7.700000in}}%
\pgfusepath{clip}%
\pgfsetbuttcap%
\pgfsetroundjoin%
\definecolor{currentfill}{rgb}{0.631373,0.788235,0.956863}%
\pgfsetfillcolor{currentfill}%
\pgfsetlinewidth{0.481800pt}%
\definecolor{currentstroke}{rgb}{1.000000,1.000000,1.000000}%
\pgfsetstrokecolor{currentstroke}%
\pgfsetdash{}{0pt}%
\pgfpathmoveto{\pgfqpoint{6.395914in}{3.188558in}}%
\pgfpathcurveto{\pgfqpoint{6.406964in}{3.188558in}}{\pgfqpoint{6.417563in}{3.192948in}}{\pgfqpoint{6.425377in}{3.200762in}}%
\pgfpathcurveto{\pgfqpoint{6.433191in}{3.208576in}}{\pgfqpoint{6.437581in}{3.219175in}}{\pgfqpoint{6.437581in}{3.230225in}}%
\pgfpathcurveto{\pgfqpoint{6.437581in}{3.241275in}}{\pgfqpoint{6.433191in}{3.251874in}}{\pgfqpoint{6.425377in}{3.259688in}}%
\pgfpathcurveto{\pgfqpoint{6.417563in}{3.267501in}}{\pgfqpoint{6.406964in}{3.271892in}}{\pgfqpoint{6.395914in}{3.271892in}}%
\pgfpathcurveto{\pgfqpoint{6.384864in}{3.271892in}}{\pgfqpoint{6.374265in}{3.267501in}}{\pgfqpoint{6.366451in}{3.259688in}}%
\pgfpathcurveto{\pgfqpoint{6.358638in}{3.251874in}}{\pgfqpoint{6.354248in}{3.241275in}}{\pgfqpoint{6.354248in}{3.230225in}}%
\pgfpathcurveto{\pgfqpoint{6.354248in}{3.219175in}}{\pgfqpoint{6.358638in}{3.208576in}}{\pgfqpoint{6.366451in}{3.200762in}}%
\pgfpathcurveto{\pgfqpoint{6.374265in}{3.192948in}}{\pgfqpoint{6.384864in}{3.188558in}}{\pgfqpoint{6.395914in}{3.188558in}}%
\pgfpathclose%
\pgfusepath{stroke,fill}%
\end{pgfscope}%
\begin{pgfscope}%
\pgfpathrectangle{\pgfqpoint{0.481978in}{0.331635in}}{\pgfqpoint{9.300000in}{7.700000in}}%
\pgfusepath{clip}%
\pgfsetbuttcap%
\pgfsetroundjoin%
\definecolor{currentfill}{rgb}{0.631373,0.788235,0.956863}%
\pgfsetfillcolor{currentfill}%
\pgfsetlinewidth{0.481800pt}%
\definecolor{currentstroke}{rgb}{1.000000,1.000000,1.000000}%
\pgfsetstrokecolor{currentstroke}%
\pgfsetdash{}{0pt}%
\pgfpathmoveto{\pgfqpoint{4.098452in}{6.229781in}}%
\pgfpathcurveto{\pgfqpoint{4.109502in}{6.229781in}}{\pgfqpoint{4.120101in}{6.234172in}}{\pgfqpoint{4.127915in}{6.241985in}}%
\pgfpathcurveto{\pgfqpoint{4.135729in}{6.249799in}}{\pgfqpoint{4.140119in}{6.260398in}}{\pgfqpoint{4.140119in}{6.271448in}}%
\pgfpathcurveto{\pgfqpoint{4.140119in}{6.282498in}}{\pgfqpoint{4.135729in}{6.293097in}}{\pgfqpoint{4.127915in}{6.300911in}}%
\pgfpathcurveto{\pgfqpoint{4.120101in}{6.308725in}}{\pgfqpoint{4.109502in}{6.313115in}}{\pgfqpoint{4.098452in}{6.313115in}}%
\pgfpathcurveto{\pgfqpoint{4.087402in}{6.313115in}}{\pgfqpoint{4.076803in}{6.308725in}}{\pgfqpoint{4.068989in}{6.300911in}}%
\pgfpathcurveto{\pgfqpoint{4.061176in}{6.293097in}}{\pgfqpoint{4.056785in}{6.282498in}}{\pgfqpoint{4.056785in}{6.271448in}}%
\pgfpathcurveto{\pgfqpoint{4.056785in}{6.260398in}}{\pgfqpoint{4.061176in}{6.249799in}}{\pgfqpoint{4.068989in}{6.241985in}}%
\pgfpathcurveto{\pgfqpoint{4.076803in}{6.234172in}}{\pgfqpoint{4.087402in}{6.229781in}}{\pgfqpoint{4.098452in}{6.229781in}}%
\pgfpathclose%
\pgfusepath{stroke,fill}%
\end{pgfscope}%
\begin{pgfscope}%
\pgfpathrectangle{\pgfqpoint{0.481978in}{0.331635in}}{\pgfqpoint{9.300000in}{7.700000in}}%
\pgfusepath{clip}%
\pgfsetbuttcap%
\pgfsetroundjoin%
\definecolor{currentfill}{rgb}{0.631373,0.788235,0.956863}%
\pgfsetfillcolor{currentfill}%
\pgfsetlinewidth{0.481800pt}%
\definecolor{currentstroke}{rgb}{1.000000,1.000000,1.000000}%
\pgfsetstrokecolor{currentstroke}%
\pgfsetdash{}{0pt}%
\pgfpathmoveto{\pgfqpoint{6.735020in}{4.467203in}}%
\pgfpathcurveto{\pgfqpoint{6.746071in}{4.467203in}}{\pgfqpoint{6.756670in}{4.471594in}}{\pgfqpoint{6.764483in}{4.479407in}}%
\pgfpathcurveto{\pgfqpoint{6.772297in}{4.487221in}}{\pgfqpoint{6.776687in}{4.497820in}}{\pgfqpoint{6.776687in}{4.508870in}}%
\pgfpathcurveto{\pgfqpoint{6.776687in}{4.519920in}}{\pgfqpoint{6.772297in}{4.530519in}}{\pgfqpoint{6.764483in}{4.538333in}}%
\pgfpathcurveto{\pgfqpoint{6.756670in}{4.546147in}}{\pgfqpoint{6.746071in}{4.550537in}}{\pgfqpoint{6.735020in}{4.550537in}}%
\pgfpathcurveto{\pgfqpoint{6.723970in}{4.550537in}}{\pgfqpoint{6.713371in}{4.546147in}}{\pgfqpoint{6.705558in}{4.538333in}}%
\pgfpathcurveto{\pgfqpoint{6.697744in}{4.530519in}}{\pgfqpoint{6.693354in}{4.519920in}}{\pgfqpoint{6.693354in}{4.508870in}}%
\pgfpathcurveto{\pgfqpoint{6.693354in}{4.497820in}}{\pgfqpoint{6.697744in}{4.487221in}}{\pgfqpoint{6.705558in}{4.479407in}}%
\pgfpathcurveto{\pgfqpoint{6.713371in}{4.471594in}}{\pgfqpoint{6.723970in}{4.467203in}}{\pgfqpoint{6.735020in}{4.467203in}}%
\pgfpathclose%
\pgfusepath{stroke,fill}%
\end{pgfscope}%
\begin{pgfscope}%
\pgfpathrectangle{\pgfqpoint{0.481978in}{0.331635in}}{\pgfqpoint{9.300000in}{7.700000in}}%
\pgfusepath{clip}%
\pgfsetbuttcap%
\pgfsetroundjoin%
\definecolor{currentfill}{rgb}{0.631373,0.788235,0.956863}%
\pgfsetfillcolor{currentfill}%
\pgfsetlinewidth{0.481800pt}%
\definecolor{currentstroke}{rgb}{1.000000,1.000000,1.000000}%
\pgfsetstrokecolor{currentstroke}%
\pgfsetdash{}{0pt}%
\pgfpathmoveto{\pgfqpoint{5.876608in}{2.487289in}}%
\pgfpathcurveto{\pgfqpoint{5.887658in}{2.487289in}}{\pgfqpoint{5.898257in}{2.491679in}}{\pgfqpoint{5.906071in}{2.499492in}}%
\pgfpathcurveto{\pgfqpoint{5.913885in}{2.507306in}}{\pgfqpoint{5.918275in}{2.517905in}}{\pgfqpoint{5.918275in}{2.528955in}}%
\pgfpathcurveto{\pgfqpoint{5.918275in}{2.540005in}}{\pgfqpoint{5.913885in}{2.550604in}}{\pgfqpoint{5.906071in}{2.558418in}}%
\pgfpathcurveto{\pgfqpoint{5.898257in}{2.566232in}}{\pgfqpoint{5.887658in}{2.570622in}}{\pgfqpoint{5.876608in}{2.570622in}}%
\pgfpathcurveto{\pgfqpoint{5.865558in}{2.570622in}}{\pgfqpoint{5.854959in}{2.566232in}}{\pgfqpoint{5.847146in}{2.558418in}}%
\pgfpathcurveto{\pgfqpoint{5.839332in}{2.550604in}}{\pgfqpoint{5.834942in}{2.540005in}}{\pgfqpoint{5.834942in}{2.528955in}}%
\pgfpathcurveto{\pgfqpoint{5.834942in}{2.517905in}}{\pgfqpoint{5.839332in}{2.507306in}}{\pgfqpoint{5.847146in}{2.499492in}}%
\pgfpathcurveto{\pgfqpoint{5.854959in}{2.491679in}}{\pgfqpoint{5.865558in}{2.487289in}}{\pgfqpoint{5.876608in}{2.487289in}}%
\pgfpathclose%
\pgfusepath{stroke,fill}%
\end{pgfscope}%
\begin{pgfscope}%
\pgfpathrectangle{\pgfqpoint{0.481978in}{0.331635in}}{\pgfqpoint{9.300000in}{7.700000in}}%
\pgfusepath{clip}%
\pgfsetbuttcap%
\pgfsetroundjoin%
\definecolor{currentfill}{rgb}{0.631373,0.788235,0.956863}%
\pgfsetfillcolor{currentfill}%
\pgfsetlinewidth{0.481800pt}%
\definecolor{currentstroke}{rgb}{1.000000,1.000000,1.000000}%
\pgfsetstrokecolor{currentstroke}%
\pgfsetdash{}{0pt}%
\pgfpathmoveto{\pgfqpoint{4.683187in}{4.383813in}}%
\pgfpathcurveto{\pgfqpoint{4.694237in}{4.383813in}}{\pgfqpoint{4.704837in}{4.388203in}}{\pgfqpoint{4.712650in}{4.396017in}}%
\pgfpathcurveto{\pgfqpoint{4.720464in}{4.403831in}}{\pgfqpoint{4.724854in}{4.414430in}}{\pgfqpoint{4.724854in}{4.425480in}}%
\pgfpathcurveto{\pgfqpoint{4.724854in}{4.436530in}}{\pgfqpoint{4.720464in}{4.447129in}}{\pgfqpoint{4.712650in}{4.454942in}}%
\pgfpathcurveto{\pgfqpoint{4.704837in}{4.462756in}}{\pgfqpoint{4.694237in}{4.467146in}}{\pgfqpoint{4.683187in}{4.467146in}}%
\pgfpathcurveto{\pgfqpoint{4.672137in}{4.467146in}}{\pgfqpoint{4.661538in}{4.462756in}}{\pgfqpoint{4.653725in}{4.454942in}}%
\pgfpathcurveto{\pgfqpoint{4.645911in}{4.447129in}}{\pgfqpoint{4.641521in}{4.436530in}}{\pgfqpoint{4.641521in}{4.425480in}}%
\pgfpathcurveto{\pgfqpoint{4.641521in}{4.414430in}}{\pgfqpoint{4.645911in}{4.403831in}}{\pgfqpoint{4.653725in}{4.396017in}}%
\pgfpathcurveto{\pgfqpoint{4.661538in}{4.388203in}}{\pgfqpoint{4.672137in}{4.383813in}}{\pgfqpoint{4.683187in}{4.383813in}}%
\pgfpathclose%
\pgfusepath{stroke,fill}%
\end{pgfscope}%
\begin{pgfscope}%
\pgfpathrectangle{\pgfqpoint{0.481978in}{0.331635in}}{\pgfqpoint{9.300000in}{7.700000in}}%
\pgfusepath{clip}%
\pgfsetbuttcap%
\pgfsetroundjoin%
\definecolor{currentfill}{rgb}{0.631373,0.788235,0.956863}%
\pgfsetfillcolor{currentfill}%
\pgfsetlinewidth{0.481800pt}%
\definecolor{currentstroke}{rgb}{1.000000,1.000000,1.000000}%
\pgfsetstrokecolor{currentstroke}%
\pgfsetdash{}{0pt}%
\pgfpathmoveto{\pgfqpoint{2.854341in}{1.999996in}}%
\pgfpathcurveto{\pgfqpoint{2.865391in}{1.999996in}}{\pgfqpoint{2.875990in}{2.004386in}}{\pgfqpoint{2.883804in}{2.012200in}}%
\pgfpathcurveto{\pgfqpoint{2.891618in}{2.020013in}}{\pgfqpoint{2.896008in}{2.030612in}}{\pgfqpoint{2.896008in}{2.041662in}}%
\pgfpathcurveto{\pgfqpoint{2.896008in}{2.052713in}}{\pgfqpoint{2.891618in}{2.063312in}}{\pgfqpoint{2.883804in}{2.071125in}}%
\pgfpathcurveto{\pgfqpoint{2.875990in}{2.078939in}}{\pgfqpoint{2.865391in}{2.083329in}}{\pgfqpoint{2.854341in}{2.083329in}}%
\pgfpathcurveto{\pgfqpoint{2.843291in}{2.083329in}}{\pgfqpoint{2.832692in}{2.078939in}}{\pgfqpoint{2.824878in}{2.071125in}}%
\pgfpathcurveto{\pgfqpoint{2.817065in}{2.063312in}}{\pgfqpoint{2.812674in}{2.052713in}}{\pgfqpoint{2.812674in}{2.041662in}}%
\pgfpathcurveto{\pgfqpoint{2.812674in}{2.030612in}}{\pgfqpoint{2.817065in}{2.020013in}}{\pgfqpoint{2.824878in}{2.012200in}}%
\pgfpathcurveto{\pgfqpoint{2.832692in}{2.004386in}}{\pgfqpoint{2.843291in}{1.999996in}}{\pgfqpoint{2.854341in}{1.999996in}}%
\pgfpathclose%
\pgfusepath{stroke,fill}%
\end{pgfscope}%
\begin{pgfscope}%
\pgfpathrectangle{\pgfqpoint{0.481978in}{0.331635in}}{\pgfqpoint{9.300000in}{7.700000in}}%
\pgfusepath{clip}%
\pgfsetbuttcap%
\pgfsetroundjoin%
\definecolor{currentfill}{rgb}{0.631373,0.788235,0.956863}%
\pgfsetfillcolor{currentfill}%
\pgfsetlinewidth{0.481800pt}%
\definecolor{currentstroke}{rgb}{1.000000,1.000000,1.000000}%
\pgfsetstrokecolor{currentstroke}%
\pgfsetdash{}{0pt}%
\pgfpathmoveto{\pgfqpoint{7.738499in}{4.278526in}}%
\pgfpathcurveto{\pgfqpoint{7.749549in}{4.278526in}}{\pgfqpoint{7.760148in}{4.282916in}}{\pgfqpoint{7.767962in}{4.290730in}}%
\pgfpathcurveto{\pgfqpoint{7.775775in}{4.298543in}}{\pgfqpoint{7.780165in}{4.309143in}}{\pgfqpoint{7.780165in}{4.320193in}}%
\pgfpathcurveto{\pgfqpoint{7.780165in}{4.331243in}}{\pgfqpoint{7.775775in}{4.341842in}}{\pgfqpoint{7.767962in}{4.349655in}}%
\pgfpathcurveto{\pgfqpoint{7.760148in}{4.357469in}}{\pgfqpoint{7.749549in}{4.361859in}}{\pgfqpoint{7.738499in}{4.361859in}}%
\pgfpathcurveto{\pgfqpoint{7.727449in}{4.361859in}}{\pgfqpoint{7.716850in}{4.357469in}}{\pgfqpoint{7.709036in}{4.349655in}}%
\pgfpathcurveto{\pgfqpoint{7.701222in}{4.341842in}}{\pgfqpoint{7.696832in}{4.331243in}}{\pgfqpoint{7.696832in}{4.320193in}}%
\pgfpathcurveto{\pgfqpoint{7.696832in}{4.309143in}}{\pgfqpoint{7.701222in}{4.298543in}}{\pgfqpoint{7.709036in}{4.290730in}}%
\pgfpathcurveto{\pgfqpoint{7.716850in}{4.282916in}}{\pgfqpoint{7.727449in}{4.278526in}}{\pgfqpoint{7.738499in}{4.278526in}}%
\pgfpathclose%
\pgfusepath{stroke,fill}%
\end{pgfscope}%
\begin{pgfscope}%
\pgfpathrectangle{\pgfqpoint{0.481978in}{0.331635in}}{\pgfqpoint{9.300000in}{7.700000in}}%
\pgfusepath{clip}%
\pgfsetbuttcap%
\pgfsetroundjoin%
\definecolor{currentfill}{rgb}{0.631373,0.788235,0.956863}%
\pgfsetfillcolor{currentfill}%
\pgfsetlinewidth{0.481800pt}%
\definecolor{currentstroke}{rgb}{1.000000,1.000000,1.000000}%
\pgfsetstrokecolor{currentstroke}%
\pgfsetdash{}{0pt}%
\pgfpathmoveto{\pgfqpoint{4.181272in}{7.628054in}}%
\pgfpathcurveto{\pgfqpoint{4.192323in}{7.628054in}}{\pgfqpoint{4.202922in}{7.632444in}}{\pgfqpoint{4.210735in}{7.640258in}}%
\pgfpathcurveto{\pgfqpoint{4.218549in}{7.648072in}}{\pgfqpoint{4.222939in}{7.658671in}}{\pgfqpoint{4.222939in}{7.669721in}}%
\pgfpathcurveto{\pgfqpoint{4.222939in}{7.680771in}}{\pgfqpoint{4.218549in}{7.691370in}}{\pgfqpoint{4.210735in}{7.699184in}}%
\pgfpathcurveto{\pgfqpoint{4.202922in}{7.706997in}}{\pgfqpoint{4.192323in}{7.711388in}}{\pgfqpoint{4.181272in}{7.711388in}}%
\pgfpathcurveto{\pgfqpoint{4.170222in}{7.711388in}}{\pgfqpoint{4.159623in}{7.706997in}}{\pgfqpoint{4.151810in}{7.699184in}}%
\pgfpathcurveto{\pgfqpoint{4.143996in}{7.691370in}}{\pgfqpoint{4.139606in}{7.680771in}}{\pgfqpoint{4.139606in}{7.669721in}}%
\pgfpathcurveto{\pgfqpoint{4.139606in}{7.658671in}}{\pgfqpoint{4.143996in}{7.648072in}}{\pgfqpoint{4.151810in}{7.640258in}}%
\pgfpathcurveto{\pgfqpoint{4.159623in}{7.632444in}}{\pgfqpoint{4.170222in}{7.628054in}}{\pgfqpoint{4.181272in}{7.628054in}}%
\pgfpathclose%
\pgfusepath{stroke,fill}%
\end{pgfscope}%
\begin{pgfscope}%
\pgfpathrectangle{\pgfqpoint{0.481978in}{0.331635in}}{\pgfqpoint{9.300000in}{7.700000in}}%
\pgfusepath{clip}%
\pgfsetbuttcap%
\pgfsetroundjoin%
\definecolor{currentfill}{rgb}{0.631373,0.788235,0.956863}%
\pgfsetfillcolor{currentfill}%
\pgfsetlinewidth{0.481800pt}%
\definecolor{currentstroke}{rgb}{1.000000,1.000000,1.000000}%
\pgfsetstrokecolor{currentstroke}%
\pgfsetdash{}{0pt}%
\pgfpathmoveto{\pgfqpoint{6.285628in}{4.415552in}}%
\pgfpathcurveto{\pgfqpoint{6.296678in}{4.415552in}}{\pgfqpoint{6.307277in}{4.419943in}}{\pgfqpoint{6.315090in}{4.427756in}}%
\pgfpathcurveto{\pgfqpoint{6.322904in}{4.435570in}}{\pgfqpoint{6.327294in}{4.446169in}}{\pgfqpoint{6.327294in}{4.457219in}}%
\pgfpathcurveto{\pgfqpoint{6.327294in}{4.468269in}}{\pgfqpoint{6.322904in}{4.478868in}}{\pgfqpoint{6.315090in}{4.486682in}}%
\pgfpathcurveto{\pgfqpoint{6.307277in}{4.494495in}}{\pgfqpoint{6.296678in}{4.498886in}}{\pgfqpoint{6.285628in}{4.498886in}}%
\pgfpathcurveto{\pgfqpoint{6.274578in}{4.498886in}}{\pgfqpoint{6.263979in}{4.494495in}}{\pgfqpoint{6.256165in}{4.486682in}}%
\pgfpathcurveto{\pgfqpoint{6.248351in}{4.478868in}}{\pgfqpoint{6.243961in}{4.468269in}}{\pgfqpoint{6.243961in}{4.457219in}}%
\pgfpathcurveto{\pgfqpoint{6.243961in}{4.446169in}}{\pgfqpoint{6.248351in}{4.435570in}}{\pgfqpoint{6.256165in}{4.427756in}}%
\pgfpathcurveto{\pgfqpoint{6.263979in}{4.419943in}}{\pgfqpoint{6.274578in}{4.415552in}}{\pgfqpoint{6.285628in}{4.415552in}}%
\pgfpathclose%
\pgfusepath{stroke,fill}%
\end{pgfscope}%
\begin{pgfscope}%
\pgfpathrectangle{\pgfqpoint{0.481978in}{0.331635in}}{\pgfqpoint{9.300000in}{7.700000in}}%
\pgfusepath{clip}%
\pgfsetbuttcap%
\pgfsetroundjoin%
\definecolor{currentfill}{rgb}{0.631373,0.788235,0.956863}%
\pgfsetfillcolor{currentfill}%
\pgfsetlinewidth{0.481800pt}%
\definecolor{currentstroke}{rgb}{1.000000,1.000000,1.000000}%
\pgfsetstrokecolor{currentstroke}%
\pgfsetdash{}{0pt}%
\pgfpathmoveto{\pgfqpoint{7.326306in}{5.190183in}}%
\pgfpathcurveto{\pgfqpoint{7.337356in}{5.190183in}}{\pgfqpoint{7.347955in}{5.194573in}}{\pgfqpoint{7.355769in}{5.202387in}}%
\pgfpathcurveto{\pgfqpoint{7.363583in}{5.210200in}}{\pgfqpoint{7.367973in}{5.220799in}}{\pgfqpoint{7.367973in}{5.231850in}}%
\pgfpathcurveto{\pgfqpoint{7.367973in}{5.242900in}}{\pgfqpoint{7.363583in}{5.253499in}}{\pgfqpoint{7.355769in}{5.261312in}}%
\pgfpathcurveto{\pgfqpoint{7.347955in}{5.269126in}}{\pgfqpoint{7.337356in}{5.273516in}}{\pgfqpoint{7.326306in}{5.273516in}}%
\pgfpathcurveto{\pgfqpoint{7.315256in}{5.273516in}}{\pgfqpoint{7.304657in}{5.269126in}}{\pgfqpoint{7.296843in}{5.261312in}}%
\pgfpathcurveto{\pgfqpoint{7.289030in}{5.253499in}}{\pgfqpoint{7.284639in}{5.242900in}}{\pgfqpoint{7.284639in}{5.231850in}}%
\pgfpathcurveto{\pgfqpoint{7.284639in}{5.220799in}}{\pgfqpoint{7.289030in}{5.210200in}}{\pgfqpoint{7.296843in}{5.202387in}}%
\pgfpathcurveto{\pgfqpoint{7.304657in}{5.194573in}}{\pgfqpoint{7.315256in}{5.190183in}}{\pgfqpoint{7.326306in}{5.190183in}}%
\pgfpathclose%
\pgfusepath{stroke,fill}%
\end{pgfscope}%
\begin{pgfscope}%
\pgfpathrectangle{\pgfqpoint{0.481978in}{0.331635in}}{\pgfqpoint{9.300000in}{7.700000in}}%
\pgfusepath{clip}%
\pgfsetbuttcap%
\pgfsetroundjoin%
\definecolor{currentfill}{rgb}{0.631373,0.788235,0.956863}%
\pgfsetfillcolor{currentfill}%
\pgfsetlinewidth{0.481800pt}%
\definecolor{currentstroke}{rgb}{1.000000,1.000000,1.000000}%
\pgfsetstrokecolor{currentstroke}%
\pgfsetdash{}{0pt}%
\pgfpathmoveto{\pgfqpoint{5.231346in}{2.233549in}}%
\pgfpathcurveto{\pgfqpoint{5.242397in}{2.233549in}}{\pgfqpoint{5.252996in}{2.237939in}}{\pgfqpoint{5.260809in}{2.245753in}}%
\pgfpathcurveto{\pgfqpoint{5.268623in}{2.253566in}}{\pgfqpoint{5.273013in}{2.264165in}}{\pgfqpoint{5.273013in}{2.275215in}}%
\pgfpathcurveto{\pgfqpoint{5.273013in}{2.286265in}}{\pgfqpoint{5.268623in}{2.296865in}}{\pgfqpoint{5.260809in}{2.304678in}}%
\pgfpathcurveto{\pgfqpoint{5.252996in}{2.312492in}}{\pgfqpoint{5.242397in}{2.316882in}}{\pgfqpoint{5.231346in}{2.316882in}}%
\pgfpathcurveto{\pgfqpoint{5.220296in}{2.316882in}}{\pgfqpoint{5.209697in}{2.312492in}}{\pgfqpoint{5.201884in}{2.304678in}}%
\pgfpathcurveto{\pgfqpoint{5.194070in}{2.296865in}}{\pgfqpoint{5.189680in}{2.286265in}}{\pgfqpoint{5.189680in}{2.275215in}}%
\pgfpathcurveto{\pgfqpoint{5.189680in}{2.264165in}}{\pgfqpoint{5.194070in}{2.253566in}}{\pgfqpoint{5.201884in}{2.245753in}}%
\pgfpathcurveto{\pgfqpoint{5.209697in}{2.237939in}}{\pgfqpoint{5.220296in}{2.233549in}}{\pgfqpoint{5.231346in}{2.233549in}}%
\pgfpathclose%
\pgfusepath{stroke,fill}%
\end{pgfscope}%
\begin{pgfscope}%
\pgfpathrectangle{\pgfqpoint{0.481978in}{0.331635in}}{\pgfqpoint{9.300000in}{7.700000in}}%
\pgfusepath{clip}%
\pgfsetbuttcap%
\pgfsetroundjoin%
\definecolor{currentfill}{rgb}{0.631373,0.788235,0.956863}%
\pgfsetfillcolor{currentfill}%
\pgfsetlinewidth{0.481800pt}%
\definecolor{currentstroke}{rgb}{1.000000,1.000000,1.000000}%
\pgfsetstrokecolor{currentstroke}%
\pgfsetdash{}{0pt}%
\pgfpathmoveto{\pgfqpoint{2.924270in}{1.890349in}}%
\pgfpathcurveto{\pgfqpoint{2.935320in}{1.890349in}}{\pgfqpoint{2.945919in}{1.894739in}}{\pgfqpoint{2.953733in}{1.902553in}}%
\pgfpathcurveto{\pgfqpoint{2.961547in}{1.910367in}}{\pgfqpoint{2.965937in}{1.920966in}}{\pgfqpoint{2.965937in}{1.932016in}}%
\pgfpathcurveto{\pgfqpoint{2.965937in}{1.943066in}}{\pgfqpoint{2.961547in}{1.953665in}}{\pgfqpoint{2.953733in}{1.961478in}}%
\pgfpathcurveto{\pgfqpoint{2.945919in}{1.969292in}}{\pgfqpoint{2.935320in}{1.973682in}}{\pgfqpoint{2.924270in}{1.973682in}}%
\pgfpathcurveto{\pgfqpoint{2.913220in}{1.973682in}}{\pgfqpoint{2.902621in}{1.969292in}}{\pgfqpoint{2.894807in}{1.961478in}}%
\pgfpathcurveto{\pgfqpoint{2.886994in}{1.953665in}}{\pgfqpoint{2.882604in}{1.943066in}}{\pgfqpoint{2.882604in}{1.932016in}}%
\pgfpathcurveto{\pgfqpoint{2.882604in}{1.920966in}}{\pgfqpoint{2.886994in}{1.910367in}}{\pgfqpoint{2.894807in}{1.902553in}}%
\pgfpathcurveto{\pgfqpoint{2.902621in}{1.894739in}}{\pgfqpoint{2.913220in}{1.890349in}}{\pgfqpoint{2.924270in}{1.890349in}}%
\pgfpathclose%
\pgfusepath{stroke,fill}%
\end{pgfscope}%
\begin{pgfscope}%
\pgfpathrectangle{\pgfqpoint{0.481978in}{0.331635in}}{\pgfqpoint{9.300000in}{7.700000in}}%
\pgfusepath{clip}%
\pgfsetbuttcap%
\pgfsetroundjoin%
\definecolor{currentfill}{rgb}{0.631373,0.788235,0.956863}%
\pgfsetfillcolor{currentfill}%
\pgfsetlinewidth{0.481800pt}%
\definecolor{currentstroke}{rgb}{1.000000,1.000000,1.000000}%
\pgfsetstrokecolor{currentstroke}%
\pgfsetdash{}{0pt}%
\pgfpathmoveto{\pgfqpoint{5.426953in}{7.035913in}}%
\pgfpathcurveto{\pgfqpoint{5.438003in}{7.035913in}}{\pgfqpoint{5.448602in}{7.040304in}}{\pgfqpoint{5.456415in}{7.048117in}}%
\pgfpathcurveto{\pgfqpoint{5.464229in}{7.055931in}}{\pgfqpoint{5.468619in}{7.066530in}}{\pgfqpoint{5.468619in}{7.077580in}}%
\pgfpathcurveto{\pgfqpoint{5.468619in}{7.088630in}}{\pgfqpoint{5.464229in}{7.099229in}}{\pgfqpoint{5.456415in}{7.107043in}}%
\pgfpathcurveto{\pgfqpoint{5.448602in}{7.114856in}}{\pgfqpoint{5.438003in}{7.119247in}}{\pgfqpoint{5.426953in}{7.119247in}}%
\pgfpathcurveto{\pgfqpoint{5.415902in}{7.119247in}}{\pgfqpoint{5.405303in}{7.114856in}}{\pgfqpoint{5.397490in}{7.107043in}}%
\pgfpathcurveto{\pgfqpoint{5.389676in}{7.099229in}}{\pgfqpoint{5.385286in}{7.088630in}}{\pgfqpoint{5.385286in}{7.077580in}}%
\pgfpathcurveto{\pgfqpoint{5.385286in}{7.066530in}}{\pgfqpoint{5.389676in}{7.055931in}}{\pgfqpoint{5.397490in}{7.048117in}}%
\pgfpathcurveto{\pgfqpoint{5.405303in}{7.040304in}}{\pgfqpoint{5.415902in}{7.035913in}}{\pgfqpoint{5.426953in}{7.035913in}}%
\pgfpathclose%
\pgfusepath{stroke,fill}%
\end{pgfscope}%
\begin{pgfscope}%
\pgfpathrectangle{\pgfqpoint{0.481978in}{0.331635in}}{\pgfqpoint{9.300000in}{7.700000in}}%
\pgfusepath{clip}%
\pgfsetbuttcap%
\pgfsetroundjoin%
\definecolor{currentfill}{rgb}{0.631373,0.788235,0.956863}%
\pgfsetfillcolor{currentfill}%
\pgfsetlinewidth{0.481800pt}%
\definecolor{currentstroke}{rgb}{1.000000,1.000000,1.000000}%
\pgfsetstrokecolor{currentstroke}%
\pgfsetdash{}{0pt}%
\pgfpathmoveto{\pgfqpoint{8.241979in}{5.227581in}}%
\pgfpathcurveto{\pgfqpoint{8.253029in}{5.227581in}}{\pgfqpoint{8.263628in}{5.231971in}}{\pgfqpoint{8.271442in}{5.239784in}}%
\pgfpathcurveto{\pgfqpoint{8.279256in}{5.247598in}}{\pgfqpoint{8.283646in}{5.258197in}}{\pgfqpoint{8.283646in}{5.269247in}}%
\pgfpathcurveto{\pgfqpoint{8.283646in}{5.280297in}}{\pgfqpoint{8.279256in}{5.290896in}}{\pgfqpoint{8.271442in}{5.298710in}}%
\pgfpathcurveto{\pgfqpoint{8.263628in}{5.306524in}}{\pgfqpoint{8.253029in}{5.310914in}}{\pgfqpoint{8.241979in}{5.310914in}}%
\pgfpathcurveto{\pgfqpoint{8.230929in}{5.310914in}}{\pgfqpoint{8.220330in}{5.306524in}}{\pgfqpoint{8.212516in}{5.298710in}}%
\pgfpathcurveto{\pgfqpoint{8.204703in}{5.290896in}}{\pgfqpoint{8.200313in}{5.280297in}}{\pgfqpoint{8.200313in}{5.269247in}}%
\pgfpathcurveto{\pgfqpoint{8.200313in}{5.258197in}}{\pgfqpoint{8.204703in}{5.247598in}}{\pgfqpoint{8.212516in}{5.239784in}}%
\pgfpathcurveto{\pgfqpoint{8.220330in}{5.231971in}}{\pgfqpoint{8.230929in}{5.227581in}}{\pgfqpoint{8.241979in}{5.227581in}}%
\pgfpathclose%
\pgfusepath{stroke,fill}%
\end{pgfscope}%
\begin{pgfscope}%
\pgfpathrectangle{\pgfqpoint{0.481978in}{0.331635in}}{\pgfqpoint{9.300000in}{7.700000in}}%
\pgfusepath{clip}%
\pgfsetbuttcap%
\pgfsetroundjoin%
\definecolor{currentfill}{rgb}{0.631373,0.788235,0.956863}%
\pgfsetfillcolor{currentfill}%
\pgfsetlinewidth{0.481800pt}%
\definecolor{currentstroke}{rgb}{1.000000,1.000000,1.000000}%
\pgfsetstrokecolor{currentstroke}%
\pgfsetdash{}{0pt}%
\pgfpathmoveto{\pgfqpoint{5.988042in}{2.698357in}}%
\pgfpathcurveto{\pgfqpoint{5.999092in}{2.698357in}}{\pgfqpoint{6.009691in}{2.702747in}}{\pgfqpoint{6.017505in}{2.710561in}}%
\pgfpathcurveto{\pgfqpoint{6.025319in}{2.718375in}}{\pgfqpoint{6.029709in}{2.728974in}}{\pgfqpoint{6.029709in}{2.740024in}}%
\pgfpathcurveto{\pgfqpoint{6.029709in}{2.751074in}}{\pgfqpoint{6.025319in}{2.761673in}}{\pgfqpoint{6.017505in}{2.769487in}}%
\pgfpathcurveto{\pgfqpoint{6.009691in}{2.777300in}}{\pgfqpoint{5.999092in}{2.781690in}}{\pgfqpoint{5.988042in}{2.781690in}}%
\pgfpathcurveto{\pgfqpoint{5.976992in}{2.781690in}}{\pgfqpoint{5.966393in}{2.777300in}}{\pgfqpoint{5.958579in}{2.769487in}}%
\pgfpathcurveto{\pgfqpoint{5.950766in}{2.761673in}}{\pgfqpoint{5.946375in}{2.751074in}}{\pgfqpoint{5.946375in}{2.740024in}}%
\pgfpathcurveto{\pgfqpoint{5.946375in}{2.728974in}}{\pgfqpoint{5.950766in}{2.718375in}}{\pgfqpoint{5.958579in}{2.710561in}}%
\pgfpathcurveto{\pgfqpoint{5.966393in}{2.702747in}}{\pgfqpoint{5.976992in}{2.698357in}}{\pgfqpoint{5.988042in}{2.698357in}}%
\pgfpathclose%
\pgfusepath{stroke,fill}%
\end{pgfscope}%
\begin{pgfscope}%
\pgfpathrectangle{\pgfqpoint{0.481978in}{0.331635in}}{\pgfqpoint{9.300000in}{7.700000in}}%
\pgfusepath{clip}%
\pgfsetbuttcap%
\pgfsetroundjoin%
\definecolor{currentfill}{rgb}{0.631373,0.788235,0.956863}%
\pgfsetfillcolor{currentfill}%
\pgfsetlinewidth{0.481800pt}%
\definecolor{currentstroke}{rgb}{1.000000,1.000000,1.000000}%
\pgfsetstrokecolor{currentstroke}%
\pgfsetdash{}{0pt}%
\pgfpathmoveto{\pgfqpoint{6.404379in}{5.365759in}}%
\pgfpathcurveto{\pgfqpoint{6.415429in}{5.365759in}}{\pgfqpoint{6.426028in}{5.370150in}}{\pgfqpoint{6.433842in}{5.377963in}}%
\pgfpathcurveto{\pgfqpoint{6.441656in}{5.385777in}}{\pgfqpoint{6.446046in}{5.396376in}}{\pgfqpoint{6.446046in}{5.407426in}}%
\pgfpathcurveto{\pgfqpoint{6.446046in}{5.418476in}}{\pgfqpoint{6.441656in}{5.429075in}}{\pgfqpoint{6.433842in}{5.436889in}}%
\pgfpathcurveto{\pgfqpoint{6.426028in}{5.444702in}}{\pgfqpoint{6.415429in}{5.449093in}}{\pgfqpoint{6.404379in}{5.449093in}}%
\pgfpathcurveto{\pgfqpoint{6.393329in}{5.449093in}}{\pgfqpoint{6.382730in}{5.444702in}}{\pgfqpoint{6.374916in}{5.436889in}}%
\pgfpathcurveto{\pgfqpoint{6.367103in}{5.429075in}}{\pgfqpoint{6.362713in}{5.418476in}}{\pgfqpoint{6.362713in}{5.407426in}}%
\pgfpathcurveto{\pgfqpoint{6.362713in}{5.396376in}}{\pgfqpoint{6.367103in}{5.385777in}}{\pgfqpoint{6.374916in}{5.377963in}}%
\pgfpathcurveto{\pgfqpoint{6.382730in}{5.370150in}}{\pgfqpoint{6.393329in}{5.365759in}}{\pgfqpoint{6.404379in}{5.365759in}}%
\pgfpathclose%
\pgfusepath{stroke,fill}%
\end{pgfscope}%
\begin{pgfscope}%
\pgfpathrectangle{\pgfqpoint{0.481978in}{0.331635in}}{\pgfqpoint{9.300000in}{7.700000in}}%
\pgfusepath{clip}%
\pgfsetbuttcap%
\pgfsetroundjoin%
\definecolor{currentfill}{rgb}{0.631373,0.788235,0.956863}%
\pgfsetfillcolor{currentfill}%
\pgfsetlinewidth{0.481800pt}%
\definecolor{currentstroke}{rgb}{1.000000,1.000000,1.000000}%
\pgfsetstrokecolor{currentstroke}%
\pgfsetdash{}{0pt}%
\pgfpathmoveto{\pgfqpoint{7.465449in}{4.115115in}}%
\pgfpathcurveto{\pgfqpoint{7.476499in}{4.115115in}}{\pgfqpoint{7.487098in}{4.119506in}}{\pgfqpoint{7.494912in}{4.127319in}}%
\pgfpathcurveto{\pgfqpoint{7.502725in}{4.135133in}}{\pgfqpoint{7.507115in}{4.145732in}}{\pgfqpoint{7.507115in}{4.156782in}}%
\pgfpathcurveto{\pgfqpoint{7.507115in}{4.167832in}}{\pgfqpoint{7.502725in}{4.178431in}}{\pgfqpoint{7.494912in}{4.186245in}}%
\pgfpathcurveto{\pgfqpoint{7.487098in}{4.194059in}}{\pgfqpoint{7.476499in}{4.198449in}}{\pgfqpoint{7.465449in}{4.198449in}}%
\pgfpathcurveto{\pgfqpoint{7.454399in}{4.198449in}}{\pgfqpoint{7.443800in}{4.194059in}}{\pgfqpoint{7.435986in}{4.186245in}}%
\pgfpathcurveto{\pgfqpoint{7.428172in}{4.178431in}}{\pgfqpoint{7.423782in}{4.167832in}}{\pgfqpoint{7.423782in}{4.156782in}}%
\pgfpathcurveto{\pgfqpoint{7.423782in}{4.145732in}}{\pgfqpoint{7.428172in}{4.135133in}}{\pgfqpoint{7.435986in}{4.127319in}}%
\pgfpathcurveto{\pgfqpoint{7.443800in}{4.119506in}}{\pgfqpoint{7.454399in}{4.115115in}}{\pgfqpoint{7.465449in}{4.115115in}}%
\pgfpathclose%
\pgfusepath{stroke,fill}%
\end{pgfscope}%
\begin{pgfscope}%
\pgfpathrectangle{\pgfqpoint{0.481978in}{0.331635in}}{\pgfqpoint{9.300000in}{7.700000in}}%
\pgfusepath{clip}%
\pgfsetbuttcap%
\pgfsetroundjoin%
\definecolor{currentfill}{rgb}{0.631373,0.788235,0.956863}%
\pgfsetfillcolor{currentfill}%
\pgfsetlinewidth{0.481800pt}%
\definecolor{currentstroke}{rgb}{1.000000,1.000000,1.000000}%
\pgfsetstrokecolor{currentstroke}%
\pgfsetdash{}{0pt}%
\pgfpathmoveto{\pgfqpoint{6.297588in}{4.817365in}}%
\pgfpathcurveto{\pgfqpoint{6.308638in}{4.817365in}}{\pgfqpoint{6.319237in}{4.821756in}}{\pgfqpoint{6.327051in}{4.829569in}}%
\pgfpathcurveto{\pgfqpoint{6.334864in}{4.837383in}}{\pgfqpoint{6.339255in}{4.847982in}}{\pgfqpoint{6.339255in}{4.859032in}}%
\pgfpathcurveto{\pgfqpoint{6.339255in}{4.870082in}}{\pgfqpoint{6.334864in}{4.880681in}}{\pgfqpoint{6.327051in}{4.888495in}}%
\pgfpathcurveto{\pgfqpoint{6.319237in}{4.896308in}}{\pgfqpoint{6.308638in}{4.900699in}}{\pgfqpoint{6.297588in}{4.900699in}}%
\pgfpathcurveto{\pgfqpoint{6.286538in}{4.900699in}}{\pgfqpoint{6.275939in}{4.896308in}}{\pgfqpoint{6.268125in}{4.888495in}}%
\pgfpathcurveto{\pgfqpoint{6.260312in}{4.880681in}}{\pgfqpoint{6.255921in}{4.870082in}}{\pgfqpoint{6.255921in}{4.859032in}}%
\pgfpathcurveto{\pgfqpoint{6.255921in}{4.847982in}}{\pgfqpoint{6.260312in}{4.837383in}}{\pgfqpoint{6.268125in}{4.829569in}}%
\pgfpathcurveto{\pgfqpoint{6.275939in}{4.821756in}}{\pgfqpoint{6.286538in}{4.817365in}}{\pgfqpoint{6.297588in}{4.817365in}}%
\pgfpathclose%
\pgfusepath{stroke,fill}%
\end{pgfscope}%
\begin{pgfscope}%
\pgfpathrectangle{\pgfqpoint{0.481978in}{0.331635in}}{\pgfqpoint{9.300000in}{7.700000in}}%
\pgfusepath{clip}%
\pgfsetbuttcap%
\pgfsetroundjoin%
\definecolor{currentfill}{rgb}{0.631373,0.788235,0.956863}%
\pgfsetfillcolor{currentfill}%
\pgfsetlinewidth{0.481800pt}%
\definecolor{currentstroke}{rgb}{1.000000,1.000000,1.000000}%
\pgfsetstrokecolor{currentstroke}%
\pgfsetdash{}{0pt}%
\pgfpathmoveto{\pgfqpoint{7.598727in}{4.490361in}}%
\pgfpathcurveto{\pgfqpoint{7.609778in}{4.490361in}}{\pgfqpoint{7.620377in}{4.494752in}}{\pgfqpoint{7.628190in}{4.502565in}}%
\pgfpathcurveto{\pgfqpoint{7.636004in}{4.510379in}}{\pgfqpoint{7.640394in}{4.520978in}}{\pgfqpoint{7.640394in}{4.532028in}}%
\pgfpathcurveto{\pgfqpoint{7.640394in}{4.543078in}}{\pgfqpoint{7.636004in}{4.553677in}}{\pgfqpoint{7.628190in}{4.561491in}}%
\pgfpathcurveto{\pgfqpoint{7.620377in}{4.569304in}}{\pgfqpoint{7.609778in}{4.573695in}}{\pgfqpoint{7.598727in}{4.573695in}}%
\pgfpathcurveto{\pgfqpoint{7.587677in}{4.573695in}}{\pgfqpoint{7.577078in}{4.569304in}}{\pgfqpoint{7.569265in}{4.561491in}}%
\pgfpathcurveto{\pgfqpoint{7.561451in}{4.553677in}}{\pgfqpoint{7.557061in}{4.543078in}}{\pgfqpoint{7.557061in}{4.532028in}}%
\pgfpathcurveto{\pgfqpoint{7.557061in}{4.520978in}}{\pgfqpoint{7.561451in}{4.510379in}}{\pgfqpoint{7.569265in}{4.502565in}}%
\pgfpathcurveto{\pgfqpoint{7.577078in}{4.494752in}}{\pgfqpoint{7.587677in}{4.490361in}}{\pgfqpoint{7.598727in}{4.490361in}}%
\pgfpathclose%
\pgfusepath{stroke,fill}%
\end{pgfscope}%
\begin{pgfscope}%
\pgfpathrectangle{\pgfqpoint{0.481978in}{0.331635in}}{\pgfqpoint{9.300000in}{7.700000in}}%
\pgfusepath{clip}%
\pgfsetbuttcap%
\pgfsetroundjoin%
\definecolor{currentfill}{rgb}{0.631373,0.788235,0.956863}%
\pgfsetfillcolor{currentfill}%
\pgfsetlinewidth{0.481800pt}%
\definecolor{currentstroke}{rgb}{1.000000,1.000000,1.000000}%
\pgfsetstrokecolor{currentstroke}%
\pgfsetdash{}{0pt}%
\pgfpathmoveto{\pgfqpoint{4.778349in}{5.109947in}}%
\pgfpathcurveto{\pgfqpoint{4.789399in}{5.109947in}}{\pgfqpoint{4.799998in}{5.114337in}}{\pgfqpoint{4.807812in}{5.122150in}}%
\pgfpathcurveto{\pgfqpoint{4.815625in}{5.129964in}}{\pgfqpoint{4.820015in}{5.140563in}}{\pgfqpoint{4.820015in}{5.151613in}}%
\pgfpathcurveto{\pgfqpoint{4.820015in}{5.162663in}}{\pgfqpoint{4.815625in}{5.173262in}}{\pgfqpoint{4.807812in}{5.181076in}}%
\pgfpathcurveto{\pgfqpoint{4.799998in}{5.188890in}}{\pgfqpoint{4.789399in}{5.193280in}}{\pgfqpoint{4.778349in}{5.193280in}}%
\pgfpathcurveto{\pgfqpoint{4.767299in}{5.193280in}}{\pgfqpoint{4.756700in}{5.188890in}}{\pgfqpoint{4.748886in}{5.181076in}}%
\pgfpathcurveto{\pgfqpoint{4.741072in}{5.173262in}}{\pgfqpoint{4.736682in}{5.162663in}}{\pgfqpoint{4.736682in}{5.151613in}}%
\pgfpathcurveto{\pgfqpoint{4.736682in}{5.140563in}}{\pgfqpoint{4.741072in}{5.129964in}}{\pgfqpoint{4.748886in}{5.122150in}}%
\pgfpathcurveto{\pgfqpoint{4.756700in}{5.114337in}}{\pgfqpoint{4.767299in}{5.109947in}}{\pgfqpoint{4.778349in}{5.109947in}}%
\pgfpathclose%
\pgfusepath{stroke,fill}%
\end{pgfscope}%
\begin{pgfscope}%
\pgfpathrectangle{\pgfqpoint{0.481978in}{0.331635in}}{\pgfqpoint{9.300000in}{7.700000in}}%
\pgfusepath{clip}%
\pgfsetbuttcap%
\pgfsetroundjoin%
\definecolor{currentfill}{rgb}{0.631373,0.788235,0.956863}%
\pgfsetfillcolor{currentfill}%
\pgfsetlinewidth{0.481800pt}%
\definecolor{currentstroke}{rgb}{1.000000,1.000000,1.000000}%
\pgfsetstrokecolor{currentstroke}%
\pgfsetdash{}{0pt}%
\pgfpathmoveto{\pgfqpoint{5.249516in}{5.477553in}}%
\pgfpathcurveto{\pgfqpoint{5.260566in}{5.477553in}}{\pgfqpoint{5.271166in}{5.481944in}}{\pgfqpoint{5.278979in}{5.489757in}}%
\pgfpathcurveto{\pgfqpoint{5.286793in}{5.497571in}}{\pgfqpoint{5.291183in}{5.508170in}}{\pgfqpoint{5.291183in}{5.519220in}}%
\pgfpathcurveto{\pgfqpoint{5.291183in}{5.530270in}}{\pgfqpoint{5.286793in}{5.540869in}}{\pgfqpoint{5.278979in}{5.548683in}}%
\pgfpathcurveto{\pgfqpoint{5.271166in}{5.556496in}}{\pgfqpoint{5.260566in}{5.560887in}}{\pgfqpoint{5.249516in}{5.560887in}}%
\pgfpathcurveto{\pgfqpoint{5.238466in}{5.560887in}}{\pgfqpoint{5.227867in}{5.556496in}}{\pgfqpoint{5.220054in}{5.548683in}}%
\pgfpathcurveto{\pgfqpoint{5.212240in}{5.540869in}}{\pgfqpoint{5.207850in}{5.530270in}}{\pgfqpoint{5.207850in}{5.519220in}}%
\pgfpathcurveto{\pgfqpoint{5.207850in}{5.508170in}}{\pgfqpoint{5.212240in}{5.497571in}}{\pgfqpoint{5.220054in}{5.489757in}}%
\pgfpathcurveto{\pgfqpoint{5.227867in}{5.481944in}}{\pgfqpoint{5.238466in}{5.477553in}}{\pgfqpoint{5.249516in}{5.477553in}}%
\pgfpathclose%
\pgfusepath{stroke,fill}%
\end{pgfscope}%
\begin{pgfscope}%
\pgfpathrectangle{\pgfqpoint{0.481978in}{0.331635in}}{\pgfqpoint{9.300000in}{7.700000in}}%
\pgfusepath{clip}%
\pgfsetbuttcap%
\pgfsetroundjoin%
\definecolor{currentfill}{rgb}{0.631373,0.788235,0.956863}%
\pgfsetfillcolor{currentfill}%
\pgfsetlinewidth{0.481800pt}%
\definecolor{currentstroke}{rgb}{1.000000,1.000000,1.000000}%
\pgfsetstrokecolor{currentstroke}%
\pgfsetdash{}{0pt}%
\pgfpathmoveto{\pgfqpoint{6.692326in}{1.648181in}}%
\pgfpathcurveto{\pgfqpoint{6.703376in}{1.648181in}}{\pgfqpoint{6.713976in}{1.652572in}}{\pgfqpoint{6.721789in}{1.660385in}}%
\pgfpathcurveto{\pgfqpoint{6.729603in}{1.668199in}}{\pgfqpoint{6.733993in}{1.678798in}}{\pgfqpoint{6.733993in}{1.689848in}}%
\pgfpathcurveto{\pgfqpoint{6.733993in}{1.700898in}}{\pgfqpoint{6.729603in}{1.711497in}}{\pgfqpoint{6.721789in}{1.719311in}}%
\pgfpathcurveto{\pgfqpoint{6.713976in}{1.727124in}}{\pgfqpoint{6.703376in}{1.731515in}}{\pgfqpoint{6.692326in}{1.731515in}}%
\pgfpathcurveto{\pgfqpoint{6.681276in}{1.731515in}}{\pgfqpoint{6.670677in}{1.727124in}}{\pgfqpoint{6.662864in}{1.719311in}}%
\pgfpathcurveto{\pgfqpoint{6.655050in}{1.711497in}}{\pgfqpoint{6.650660in}{1.700898in}}{\pgfqpoint{6.650660in}{1.689848in}}%
\pgfpathcurveto{\pgfqpoint{6.650660in}{1.678798in}}{\pgfqpoint{6.655050in}{1.668199in}}{\pgfqpoint{6.662864in}{1.660385in}}%
\pgfpathcurveto{\pgfqpoint{6.670677in}{1.652572in}}{\pgfqpoint{6.681276in}{1.648181in}}{\pgfqpoint{6.692326in}{1.648181in}}%
\pgfpathclose%
\pgfusepath{stroke,fill}%
\end{pgfscope}%
\begin{pgfscope}%
\pgfpathrectangle{\pgfqpoint{0.481978in}{0.331635in}}{\pgfqpoint{9.300000in}{7.700000in}}%
\pgfusepath{clip}%
\pgfsetbuttcap%
\pgfsetroundjoin%
\definecolor{currentfill}{rgb}{0.631373,0.788235,0.956863}%
\pgfsetfillcolor{currentfill}%
\pgfsetlinewidth{0.481800pt}%
\definecolor{currentstroke}{rgb}{1.000000,1.000000,1.000000}%
\pgfsetstrokecolor{currentstroke}%
\pgfsetdash{}{0pt}%
\pgfpathmoveto{\pgfqpoint{4.317631in}{1.907085in}}%
\pgfpathcurveto{\pgfqpoint{4.328682in}{1.907085in}}{\pgfqpoint{4.339281in}{1.911475in}}{\pgfqpoint{4.347094in}{1.919288in}}%
\pgfpathcurveto{\pgfqpoint{4.354908in}{1.927102in}}{\pgfqpoint{4.359298in}{1.937701in}}{\pgfqpoint{4.359298in}{1.948751in}}%
\pgfpathcurveto{\pgfqpoint{4.359298in}{1.959801in}}{\pgfqpoint{4.354908in}{1.970400in}}{\pgfqpoint{4.347094in}{1.978214in}}%
\pgfpathcurveto{\pgfqpoint{4.339281in}{1.986028in}}{\pgfqpoint{4.328682in}{1.990418in}}{\pgfqpoint{4.317631in}{1.990418in}}%
\pgfpathcurveto{\pgfqpoint{4.306581in}{1.990418in}}{\pgfqpoint{4.295982in}{1.986028in}}{\pgfqpoint{4.288169in}{1.978214in}}%
\pgfpathcurveto{\pgfqpoint{4.280355in}{1.970400in}}{\pgfqpoint{4.275965in}{1.959801in}}{\pgfqpoint{4.275965in}{1.948751in}}%
\pgfpathcurveto{\pgfqpoint{4.275965in}{1.937701in}}{\pgfqpoint{4.280355in}{1.927102in}}{\pgfqpoint{4.288169in}{1.919288in}}%
\pgfpathcurveto{\pgfqpoint{4.295982in}{1.911475in}}{\pgfqpoint{4.306581in}{1.907085in}}{\pgfqpoint{4.317631in}{1.907085in}}%
\pgfpathclose%
\pgfusepath{stroke,fill}%
\end{pgfscope}%
\begin{pgfscope}%
\pgfpathrectangle{\pgfqpoint{0.481978in}{0.331635in}}{\pgfqpoint{9.300000in}{7.700000in}}%
\pgfusepath{clip}%
\pgfsetbuttcap%
\pgfsetroundjoin%
\definecolor{currentfill}{rgb}{0.631373,0.788235,0.956863}%
\pgfsetfillcolor{currentfill}%
\pgfsetlinewidth{0.481800pt}%
\definecolor{currentstroke}{rgb}{1.000000,1.000000,1.000000}%
\pgfsetstrokecolor{currentstroke}%
\pgfsetdash{}{0pt}%
\pgfpathmoveto{\pgfqpoint{7.680936in}{6.007917in}}%
\pgfpathcurveto{\pgfqpoint{7.691986in}{6.007917in}}{\pgfqpoint{7.702585in}{6.012307in}}{\pgfqpoint{7.710398in}{6.020120in}}%
\pgfpathcurveto{\pgfqpoint{7.718212in}{6.027934in}}{\pgfqpoint{7.722602in}{6.038533in}}{\pgfqpoint{7.722602in}{6.049583in}}%
\pgfpathcurveto{\pgfqpoint{7.722602in}{6.060633in}}{\pgfqpoint{7.718212in}{6.071232in}}{\pgfqpoint{7.710398in}{6.079046in}}%
\pgfpathcurveto{\pgfqpoint{7.702585in}{6.086860in}}{\pgfqpoint{7.691986in}{6.091250in}}{\pgfqpoint{7.680936in}{6.091250in}}%
\pgfpathcurveto{\pgfqpoint{7.669885in}{6.091250in}}{\pgfqpoint{7.659286in}{6.086860in}}{\pgfqpoint{7.651473in}{6.079046in}}%
\pgfpathcurveto{\pgfqpoint{7.643659in}{6.071232in}}{\pgfqpoint{7.639269in}{6.060633in}}{\pgfqpoint{7.639269in}{6.049583in}}%
\pgfpathcurveto{\pgfqpoint{7.639269in}{6.038533in}}{\pgfqpoint{7.643659in}{6.027934in}}{\pgfqpoint{7.651473in}{6.020120in}}%
\pgfpathcurveto{\pgfqpoint{7.659286in}{6.012307in}}{\pgfqpoint{7.669885in}{6.007917in}}{\pgfqpoint{7.680936in}{6.007917in}}%
\pgfpathclose%
\pgfusepath{stroke,fill}%
\end{pgfscope}%
\begin{pgfscope}%
\pgfpathrectangle{\pgfqpoint{0.481978in}{0.331635in}}{\pgfqpoint{9.300000in}{7.700000in}}%
\pgfusepath{clip}%
\pgfsetbuttcap%
\pgfsetroundjoin%
\definecolor{currentfill}{rgb}{0.631373,0.788235,0.956863}%
\pgfsetfillcolor{currentfill}%
\pgfsetlinewidth{0.481800pt}%
\definecolor{currentstroke}{rgb}{1.000000,1.000000,1.000000}%
\pgfsetstrokecolor{currentstroke}%
\pgfsetdash{}{0pt}%
\pgfpathmoveto{\pgfqpoint{4.088639in}{5.974869in}}%
\pgfpathcurveto{\pgfqpoint{4.099689in}{5.974869in}}{\pgfqpoint{4.110289in}{5.979260in}}{\pgfqpoint{4.118102in}{5.987073in}}%
\pgfpathcurveto{\pgfqpoint{4.125916in}{5.994887in}}{\pgfqpoint{4.130306in}{6.005486in}}{\pgfqpoint{4.130306in}{6.016536in}}%
\pgfpathcurveto{\pgfqpoint{4.130306in}{6.027586in}}{\pgfqpoint{4.125916in}{6.038185in}}{\pgfqpoint{4.118102in}{6.045999in}}%
\pgfpathcurveto{\pgfqpoint{4.110289in}{6.053813in}}{\pgfqpoint{4.099689in}{6.058203in}}{\pgfqpoint{4.088639in}{6.058203in}}%
\pgfpathcurveto{\pgfqpoint{4.077589in}{6.058203in}}{\pgfqpoint{4.066990in}{6.053813in}}{\pgfqpoint{4.059177in}{6.045999in}}%
\pgfpathcurveto{\pgfqpoint{4.051363in}{6.038185in}}{\pgfqpoint{4.046973in}{6.027586in}}{\pgfqpoint{4.046973in}{6.016536in}}%
\pgfpathcurveto{\pgfqpoint{4.046973in}{6.005486in}}{\pgfqpoint{4.051363in}{5.994887in}}{\pgfqpoint{4.059177in}{5.987073in}}%
\pgfpathcurveto{\pgfqpoint{4.066990in}{5.979260in}}{\pgfqpoint{4.077589in}{5.974869in}}{\pgfqpoint{4.088639in}{5.974869in}}%
\pgfpathclose%
\pgfusepath{stroke,fill}%
\end{pgfscope}%
\begin{pgfscope}%
\pgfpathrectangle{\pgfqpoint{0.481978in}{0.331635in}}{\pgfqpoint{9.300000in}{7.700000in}}%
\pgfusepath{clip}%
\pgfsetbuttcap%
\pgfsetroundjoin%
\definecolor{currentfill}{rgb}{0.631373,0.788235,0.956863}%
\pgfsetfillcolor{currentfill}%
\pgfsetlinewidth{0.481800pt}%
\definecolor{currentstroke}{rgb}{1.000000,1.000000,1.000000}%
\pgfsetstrokecolor{currentstroke}%
\pgfsetdash{}{0pt}%
\pgfpathmoveto{\pgfqpoint{2.271824in}{3.772311in}}%
\pgfpathcurveto{\pgfqpoint{2.282874in}{3.772311in}}{\pgfqpoint{2.293473in}{3.776701in}}{\pgfqpoint{2.301287in}{3.784515in}}%
\pgfpathcurveto{\pgfqpoint{2.309101in}{3.792329in}}{\pgfqpoint{2.313491in}{3.802928in}}{\pgfqpoint{2.313491in}{3.813978in}}%
\pgfpathcurveto{\pgfqpoint{2.313491in}{3.825028in}}{\pgfqpoint{2.309101in}{3.835627in}}{\pgfqpoint{2.301287in}{3.843441in}}%
\pgfpathcurveto{\pgfqpoint{2.293473in}{3.851254in}}{\pgfqpoint{2.282874in}{3.855645in}}{\pgfqpoint{2.271824in}{3.855645in}}%
\pgfpathcurveto{\pgfqpoint{2.260774in}{3.855645in}}{\pgfqpoint{2.250175in}{3.851254in}}{\pgfqpoint{2.242361in}{3.843441in}}%
\pgfpathcurveto{\pgfqpoint{2.234548in}{3.835627in}}{\pgfqpoint{2.230158in}{3.825028in}}{\pgfqpoint{2.230158in}{3.813978in}}%
\pgfpathcurveto{\pgfqpoint{2.230158in}{3.802928in}}{\pgfqpoint{2.234548in}{3.792329in}}{\pgfqpoint{2.242361in}{3.784515in}}%
\pgfpathcurveto{\pgfqpoint{2.250175in}{3.776701in}}{\pgfqpoint{2.260774in}{3.772311in}}{\pgfqpoint{2.271824in}{3.772311in}}%
\pgfpathclose%
\pgfusepath{stroke,fill}%
\end{pgfscope}%
\begin{pgfscope}%
\pgfpathrectangle{\pgfqpoint{0.481978in}{0.331635in}}{\pgfqpoint{9.300000in}{7.700000in}}%
\pgfusepath{clip}%
\pgfsetbuttcap%
\pgfsetroundjoin%
\definecolor{currentfill}{rgb}{0.631373,0.788235,0.956863}%
\pgfsetfillcolor{currentfill}%
\pgfsetlinewidth{0.481800pt}%
\definecolor{currentstroke}{rgb}{1.000000,1.000000,1.000000}%
\pgfsetstrokecolor{currentstroke}%
\pgfsetdash{}{0pt}%
\pgfpathmoveto{\pgfqpoint{6.175909in}{2.642661in}}%
\pgfpathcurveto{\pgfqpoint{6.186959in}{2.642661in}}{\pgfqpoint{6.197558in}{2.647052in}}{\pgfqpoint{6.205371in}{2.654865in}}%
\pgfpathcurveto{\pgfqpoint{6.213185in}{2.662679in}}{\pgfqpoint{6.217575in}{2.673278in}}{\pgfqpoint{6.217575in}{2.684328in}}%
\pgfpathcurveto{\pgfqpoint{6.217575in}{2.695378in}}{\pgfqpoint{6.213185in}{2.705977in}}{\pgfqpoint{6.205371in}{2.713791in}}%
\pgfpathcurveto{\pgfqpoint{6.197558in}{2.721604in}}{\pgfqpoint{6.186959in}{2.725995in}}{\pgfqpoint{6.175909in}{2.725995in}}%
\pgfpathcurveto{\pgfqpoint{6.164859in}{2.725995in}}{\pgfqpoint{6.154259in}{2.721604in}}{\pgfqpoint{6.146446in}{2.713791in}}%
\pgfpathcurveto{\pgfqpoint{6.138632in}{2.705977in}}{\pgfqpoint{6.134242in}{2.695378in}}{\pgfqpoint{6.134242in}{2.684328in}}%
\pgfpathcurveto{\pgfqpoint{6.134242in}{2.673278in}}{\pgfqpoint{6.138632in}{2.662679in}}{\pgfqpoint{6.146446in}{2.654865in}}%
\pgfpathcurveto{\pgfqpoint{6.154259in}{2.647052in}}{\pgfqpoint{6.164859in}{2.642661in}}{\pgfqpoint{6.175909in}{2.642661in}}%
\pgfpathclose%
\pgfusepath{stroke,fill}%
\end{pgfscope}%
\begin{pgfscope}%
\pgfpathrectangle{\pgfqpoint{0.481978in}{0.331635in}}{\pgfqpoint{9.300000in}{7.700000in}}%
\pgfusepath{clip}%
\pgfsetbuttcap%
\pgfsetroundjoin%
\definecolor{currentfill}{rgb}{0.631373,0.788235,0.956863}%
\pgfsetfillcolor{currentfill}%
\pgfsetlinewidth{0.481800pt}%
\definecolor{currentstroke}{rgb}{1.000000,1.000000,1.000000}%
\pgfsetstrokecolor{currentstroke}%
\pgfsetdash{}{0pt}%
\pgfpathmoveto{\pgfqpoint{5.076404in}{1.722632in}}%
\pgfpathcurveto{\pgfqpoint{5.087454in}{1.722632in}}{\pgfqpoint{5.098053in}{1.727022in}}{\pgfqpoint{5.105867in}{1.734836in}}%
\pgfpathcurveto{\pgfqpoint{5.113681in}{1.742650in}}{\pgfqpoint{5.118071in}{1.753249in}}{\pgfqpoint{5.118071in}{1.764299in}}%
\pgfpathcurveto{\pgfqpoint{5.118071in}{1.775349in}}{\pgfqpoint{5.113681in}{1.785948in}}{\pgfqpoint{5.105867in}{1.793761in}}%
\pgfpathcurveto{\pgfqpoint{5.098053in}{1.801575in}}{\pgfqpoint{5.087454in}{1.805965in}}{\pgfqpoint{5.076404in}{1.805965in}}%
\pgfpathcurveto{\pgfqpoint{5.065354in}{1.805965in}}{\pgfqpoint{5.054755in}{1.801575in}}{\pgfqpoint{5.046941in}{1.793761in}}%
\pgfpathcurveto{\pgfqpoint{5.039128in}{1.785948in}}{\pgfqpoint{5.034738in}{1.775349in}}{\pgfqpoint{5.034738in}{1.764299in}}%
\pgfpathcurveto{\pgfqpoint{5.034738in}{1.753249in}}{\pgfqpoint{5.039128in}{1.742650in}}{\pgfqpoint{5.046941in}{1.734836in}}%
\pgfpathcurveto{\pgfqpoint{5.054755in}{1.727022in}}{\pgfqpoint{5.065354in}{1.722632in}}{\pgfqpoint{5.076404in}{1.722632in}}%
\pgfpathclose%
\pgfusepath{stroke,fill}%
\end{pgfscope}%
\begin{pgfscope}%
\pgfpathrectangle{\pgfqpoint{0.481978in}{0.331635in}}{\pgfqpoint{9.300000in}{7.700000in}}%
\pgfusepath{clip}%
\pgfsetbuttcap%
\pgfsetroundjoin%
\definecolor{currentfill}{rgb}{0.631373,0.788235,0.956863}%
\pgfsetfillcolor{currentfill}%
\pgfsetlinewidth{0.481800pt}%
\definecolor{currentstroke}{rgb}{1.000000,1.000000,1.000000}%
\pgfsetstrokecolor{currentstroke}%
\pgfsetdash{}{0pt}%
\pgfpathmoveto{\pgfqpoint{8.665313in}{4.817063in}}%
\pgfpathcurveto{\pgfqpoint{8.676363in}{4.817063in}}{\pgfqpoint{8.686962in}{4.821453in}}{\pgfqpoint{8.694776in}{4.829267in}}%
\pgfpathcurveto{\pgfqpoint{8.702589in}{4.837080in}}{\pgfqpoint{8.706980in}{4.847679in}}{\pgfqpoint{8.706980in}{4.858730in}}%
\pgfpathcurveto{\pgfqpoint{8.706980in}{4.869780in}}{\pgfqpoint{8.702589in}{4.880379in}}{\pgfqpoint{8.694776in}{4.888192in}}%
\pgfpathcurveto{\pgfqpoint{8.686962in}{4.896006in}}{\pgfqpoint{8.676363in}{4.900396in}}{\pgfqpoint{8.665313in}{4.900396in}}%
\pgfpathcurveto{\pgfqpoint{8.654263in}{4.900396in}}{\pgfqpoint{8.643664in}{4.896006in}}{\pgfqpoint{8.635850in}{4.888192in}}%
\pgfpathcurveto{\pgfqpoint{8.628037in}{4.880379in}}{\pgfqpoint{8.623646in}{4.869780in}}{\pgfqpoint{8.623646in}{4.858730in}}%
\pgfpathcurveto{\pgfqpoint{8.623646in}{4.847679in}}{\pgfqpoint{8.628037in}{4.837080in}}{\pgfqpoint{8.635850in}{4.829267in}}%
\pgfpathcurveto{\pgfqpoint{8.643664in}{4.821453in}}{\pgfqpoint{8.654263in}{4.817063in}}{\pgfqpoint{8.665313in}{4.817063in}}%
\pgfpathclose%
\pgfusepath{stroke,fill}%
\end{pgfscope}%
\begin{pgfscope}%
\pgfpathrectangle{\pgfqpoint{0.481978in}{0.331635in}}{\pgfqpoint{9.300000in}{7.700000in}}%
\pgfusepath{clip}%
\pgfsetbuttcap%
\pgfsetroundjoin%
\definecolor{currentfill}{rgb}{0.631373,0.788235,0.956863}%
\pgfsetfillcolor{currentfill}%
\pgfsetlinewidth{0.481800pt}%
\definecolor{currentstroke}{rgb}{1.000000,1.000000,1.000000}%
\pgfsetstrokecolor{currentstroke}%
\pgfsetdash{}{0pt}%
\pgfpathmoveto{\pgfqpoint{7.138229in}{2.559103in}}%
\pgfpathcurveto{\pgfqpoint{7.149279in}{2.559103in}}{\pgfqpoint{7.159878in}{2.563493in}}{\pgfqpoint{7.167692in}{2.571306in}}%
\pgfpathcurveto{\pgfqpoint{7.175505in}{2.579120in}}{\pgfqpoint{7.179895in}{2.589719in}}{\pgfqpoint{7.179895in}{2.600769in}}%
\pgfpathcurveto{\pgfqpoint{7.179895in}{2.611819in}}{\pgfqpoint{7.175505in}{2.622418in}}{\pgfqpoint{7.167692in}{2.630232in}}%
\pgfpathcurveto{\pgfqpoint{7.159878in}{2.638046in}}{\pgfqpoint{7.149279in}{2.642436in}}{\pgfqpoint{7.138229in}{2.642436in}}%
\pgfpathcurveto{\pgfqpoint{7.127179in}{2.642436in}}{\pgfqpoint{7.116580in}{2.638046in}}{\pgfqpoint{7.108766in}{2.630232in}}%
\pgfpathcurveto{\pgfqpoint{7.100952in}{2.622418in}}{\pgfqpoint{7.096562in}{2.611819in}}{\pgfqpoint{7.096562in}{2.600769in}}%
\pgfpathcurveto{\pgfqpoint{7.096562in}{2.589719in}}{\pgfqpoint{7.100952in}{2.579120in}}{\pgfqpoint{7.108766in}{2.571306in}}%
\pgfpathcurveto{\pgfqpoint{7.116580in}{2.563493in}}{\pgfqpoint{7.127179in}{2.559103in}}{\pgfqpoint{7.138229in}{2.559103in}}%
\pgfpathclose%
\pgfusepath{stroke,fill}%
\end{pgfscope}%
\begin{pgfscope}%
\pgfpathrectangle{\pgfqpoint{0.481978in}{0.331635in}}{\pgfqpoint{9.300000in}{7.700000in}}%
\pgfusepath{clip}%
\pgfsetbuttcap%
\pgfsetroundjoin%
\definecolor{currentfill}{rgb}{0.631373,0.788235,0.956863}%
\pgfsetfillcolor{currentfill}%
\pgfsetlinewidth{0.481800pt}%
\definecolor{currentstroke}{rgb}{1.000000,1.000000,1.000000}%
\pgfsetstrokecolor{currentstroke}%
\pgfsetdash{}{0pt}%
\pgfpathmoveto{\pgfqpoint{6.963010in}{1.813724in}}%
\pgfpathcurveto{\pgfqpoint{6.974060in}{1.813724in}}{\pgfqpoint{6.984659in}{1.818115in}}{\pgfqpoint{6.992472in}{1.825928in}}%
\pgfpathcurveto{\pgfqpoint{7.000286in}{1.833742in}}{\pgfqpoint{7.004676in}{1.844341in}}{\pgfqpoint{7.004676in}{1.855391in}}%
\pgfpathcurveto{\pgfqpoint{7.004676in}{1.866441in}}{\pgfqpoint{7.000286in}{1.877040in}}{\pgfqpoint{6.992472in}{1.884854in}}%
\pgfpathcurveto{\pgfqpoint{6.984659in}{1.892667in}}{\pgfqpoint{6.974060in}{1.897058in}}{\pgfqpoint{6.963010in}{1.897058in}}%
\pgfpathcurveto{\pgfqpoint{6.951960in}{1.897058in}}{\pgfqpoint{6.941361in}{1.892667in}}{\pgfqpoint{6.933547in}{1.884854in}}%
\pgfpathcurveto{\pgfqpoint{6.925733in}{1.877040in}}{\pgfqpoint{6.921343in}{1.866441in}}{\pgfqpoint{6.921343in}{1.855391in}}%
\pgfpathcurveto{\pgfqpoint{6.921343in}{1.844341in}}{\pgfqpoint{6.925733in}{1.833742in}}{\pgfqpoint{6.933547in}{1.825928in}}%
\pgfpathcurveto{\pgfqpoint{6.941361in}{1.818115in}}{\pgfqpoint{6.951960in}{1.813724in}}{\pgfqpoint{6.963010in}{1.813724in}}%
\pgfpathclose%
\pgfusepath{stroke,fill}%
\end{pgfscope}%
\begin{pgfscope}%
\pgfpathrectangle{\pgfqpoint{0.481978in}{0.331635in}}{\pgfqpoint{9.300000in}{7.700000in}}%
\pgfusepath{clip}%
\pgfsetbuttcap%
\pgfsetroundjoin%
\definecolor{currentfill}{rgb}{0.631373,0.788235,0.956863}%
\pgfsetfillcolor{currentfill}%
\pgfsetlinewidth{0.481800pt}%
\definecolor{currentstroke}{rgb}{1.000000,1.000000,1.000000}%
\pgfsetstrokecolor{currentstroke}%
\pgfsetdash{}{0pt}%
\pgfpathmoveto{\pgfqpoint{8.113494in}{5.267796in}}%
\pgfpathcurveto{\pgfqpoint{8.124544in}{5.267796in}}{\pgfqpoint{8.135143in}{5.272186in}}{\pgfqpoint{8.142957in}{5.280000in}}%
\pgfpathcurveto{\pgfqpoint{8.150771in}{5.287814in}}{\pgfqpoint{8.155161in}{5.298413in}}{\pgfqpoint{8.155161in}{5.309463in}}%
\pgfpathcurveto{\pgfqpoint{8.155161in}{5.320513in}}{\pgfqpoint{8.150771in}{5.331112in}}{\pgfqpoint{8.142957in}{5.338926in}}%
\pgfpathcurveto{\pgfqpoint{8.135143in}{5.346739in}}{\pgfqpoint{8.124544in}{5.351130in}}{\pgfqpoint{8.113494in}{5.351130in}}%
\pgfpathcurveto{\pgfqpoint{8.102444in}{5.351130in}}{\pgfqpoint{8.091845in}{5.346739in}}{\pgfqpoint{8.084031in}{5.338926in}}%
\pgfpathcurveto{\pgfqpoint{8.076218in}{5.331112in}}{\pgfqpoint{8.071827in}{5.320513in}}{\pgfqpoint{8.071827in}{5.309463in}}%
\pgfpathcurveto{\pgfqpoint{8.071827in}{5.298413in}}{\pgfqpoint{8.076218in}{5.287814in}}{\pgfqpoint{8.084031in}{5.280000in}}%
\pgfpathcurveto{\pgfqpoint{8.091845in}{5.272186in}}{\pgfqpoint{8.102444in}{5.267796in}}{\pgfqpoint{8.113494in}{5.267796in}}%
\pgfpathclose%
\pgfusepath{stroke,fill}%
\end{pgfscope}%
\begin{pgfscope}%
\pgfpathrectangle{\pgfqpoint{0.481978in}{0.331635in}}{\pgfqpoint{9.300000in}{7.700000in}}%
\pgfusepath{clip}%
\pgfsetbuttcap%
\pgfsetroundjoin%
\definecolor{currentfill}{rgb}{0.631373,0.788235,0.956863}%
\pgfsetfillcolor{currentfill}%
\pgfsetlinewidth{0.481800pt}%
\definecolor{currentstroke}{rgb}{1.000000,1.000000,1.000000}%
\pgfsetstrokecolor{currentstroke}%
\pgfsetdash{}{0pt}%
\pgfpathmoveto{\pgfqpoint{5.616150in}{1.832494in}}%
\pgfpathcurveto{\pgfqpoint{5.627200in}{1.832494in}}{\pgfqpoint{5.637799in}{1.836884in}}{\pgfqpoint{5.645613in}{1.844698in}}%
\pgfpathcurveto{\pgfqpoint{5.653426in}{1.852511in}}{\pgfqpoint{5.657816in}{1.863110in}}{\pgfqpoint{5.657816in}{1.874161in}}%
\pgfpathcurveto{\pgfqpoint{5.657816in}{1.885211in}}{\pgfqpoint{5.653426in}{1.895810in}}{\pgfqpoint{5.645613in}{1.903623in}}%
\pgfpathcurveto{\pgfqpoint{5.637799in}{1.911437in}}{\pgfqpoint{5.627200in}{1.915827in}}{\pgfqpoint{5.616150in}{1.915827in}}%
\pgfpathcurveto{\pgfqpoint{5.605100in}{1.915827in}}{\pgfqpoint{5.594501in}{1.911437in}}{\pgfqpoint{5.586687in}{1.903623in}}%
\pgfpathcurveto{\pgfqpoint{5.578873in}{1.895810in}}{\pgfqpoint{5.574483in}{1.885211in}}{\pgfqpoint{5.574483in}{1.874161in}}%
\pgfpathcurveto{\pgfqpoint{5.574483in}{1.863110in}}{\pgfqpoint{5.578873in}{1.852511in}}{\pgfqpoint{5.586687in}{1.844698in}}%
\pgfpathcurveto{\pgfqpoint{5.594501in}{1.836884in}}{\pgfqpoint{5.605100in}{1.832494in}}{\pgfqpoint{5.616150in}{1.832494in}}%
\pgfpathclose%
\pgfusepath{stroke,fill}%
\end{pgfscope}%
\begin{pgfscope}%
\pgfpathrectangle{\pgfqpoint{0.481978in}{0.331635in}}{\pgfqpoint{9.300000in}{7.700000in}}%
\pgfusepath{clip}%
\pgfsetbuttcap%
\pgfsetroundjoin%
\definecolor{currentfill}{rgb}{0.631373,0.788235,0.956863}%
\pgfsetfillcolor{currentfill}%
\pgfsetlinewidth{0.481800pt}%
\definecolor{currentstroke}{rgb}{1.000000,1.000000,1.000000}%
\pgfsetstrokecolor{currentstroke}%
\pgfsetdash{}{0pt}%
\pgfpathmoveto{\pgfqpoint{4.785427in}{1.270778in}}%
\pgfpathcurveto{\pgfqpoint{4.796478in}{1.270778in}}{\pgfqpoint{4.807077in}{1.275168in}}{\pgfqpoint{4.814890in}{1.282982in}}%
\pgfpathcurveto{\pgfqpoint{4.822704in}{1.290796in}}{\pgfqpoint{4.827094in}{1.301395in}}{\pgfqpoint{4.827094in}{1.312445in}}%
\pgfpathcurveto{\pgfqpoint{4.827094in}{1.323495in}}{\pgfqpoint{4.822704in}{1.334094in}}{\pgfqpoint{4.814890in}{1.341908in}}%
\pgfpathcurveto{\pgfqpoint{4.807077in}{1.349721in}}{\pgfqpoint{4.796478in}{1.354111in}}{\pgfqpoint{4.785427in}{1.354111in}}%
\pgfpathcurveto{\pgfqpoint{4.774377in}{1.354111in}}{\pgfqpoint{4.763778in}{1.349721in}}{\pgfqpoint{4.755965in}{1.341908in}}%
\pgfpathcurveto{\pgfqpoint{4.748151in}{1.334094in}}{\pgfqpoint{4.743761in}{1.323495in}}{\pgfqpoint{4.743761in}{1.312445in}}%
\pgfpathcurveto{\pgfqpoint{4.743761in}{1.301395in}}{\pgfqpoint{4.748151in}{1.290796in}}{\pgfqpoint{4.755965in}{1.282982in}}%
\pgfpathcurveto{\pgfqpoint{4.763778in}{1.275168in}}{\pgfqpoint{4.774377in}{1.270778in}}{\pgfqpoint{4.785427in}{1.270778in}}%
\pgfpathclose%
\pgfusepath{stroke,fill}%
\end{pgfscope}%
\begin{pgfscope}%
\pgfpathrectangle{\pgfqpoint{0.481978in}{0.331635in}}{\pgfqpoint{9.300000in}{7.700000in}}%
\pgfusepath{clip}%
\pgfsetbuttcap%
\pgfsetroundjoin%
\definecolor{currentfill}{rgb}{0.631373,0.788235,0.956863}%
\pgfsetfillcolor{currentfill}%
\pgfsetlinewidth{0.481800pt}%
\definecolor{currentstroke}{rgb}{1.000000,1.000000,1.000000}%
\pgfsetstrokecolor{currentstroke}%
\pgfsetdash{}{0pt}%
\pgfpathmoveto{\pgfqpoint{8.275701in}{4.967909in}}%
\pgfpathcurveto{\pgfqpoint{8.286751in}{4.967909in}}{\pgfqpoint{8.297350in}{4.972299in}}{\pgfqpoint{8.305164in}{4.980113in}}%
\pgfpathcurveto{\pgfqpoint{8.312978in}{4.987927in}}{\pgfqpoint{8.317368in}{4.998526in}}{\pgfqpoint{8.317368in}{5.009576in}}%
\pgfpathcurveto{\pgfqpoint{8.317368in}{5.020626in}}{\pgfqpoint{8.312978in}{5.031225in}}{\pgfqpoint{8.305164in}{5.039039in}}%
\pgfpathcurveto{\pgfqpoint{8.297350in}{5.046852in}}{\pgfqpoint{8.286751in}{5.051242in}}{\pgfqpoint{8.275701in}{5.051242in}}%
\pgfpathcurveto{\pgfqpoint{8.264651in}{5.051242in}}{\pgfqpoint{8.254052in}{5.046852in}}{\pgfqpoint{8.246238in}{5.039039in}}%
\pgfpathcurveto{\pgfqpoint{8.238425in}{5.031225in}}{\pgfqpoint{8.234035in}{5.020626in}}{\pgfqpoint{8.234035in}{5.009576in}}%
\pgfpathcurveto{\pgfqpoint{8.234035in}{4.998526in}}{\pgfqpoint{8.238425in}{4.987927in}}{\pgfqpoint{8.246238in}{4.980113in}}%
\pgfpathcurveto{\pgfqpoint{8.254052in}{4.972299in}}{\pgfqpoint{8.264651in}{4.967909in}}{\pgfqpoint{8.275701in}{4.967909in}}%
\pgfpathclose%
\pgfusepath{stroke,fill}%
\end{pgfscope}%
\begin{pgfscope}%
\pgfpathrectangle{\pgfqpoint{0.481978in}{0.331635in}}{\pgfqpoint{9.300000in}{7.700000in}}%
\pgfusepath{clip}%
\pgfsetbuttcap%
\pgfsetroundjoin%
\definecolor{currentfill}{rgb}{0.631373,0.788235,0.956863}%
\pgfsetfillcolor{currentfill}%
\pgfsetlinewidth{0.481800pt}%
\definecolor{currentstroke}{rgb}{1.000000,1.000000,1.000000}%
\pgfsetstrokecolor{currentstroke}%
\pgfsetdash{}{0pt}%
\pgfpathmoveto{\pgfqpoint{7.698866in}{5.076520in}}%
\pgfpathcurveto{\pgfqpoint{7.709916in}{5.076520in}}{\pgfqpoint{7.720515in}{5.080910in}}{\pgfqpoint{7.728329in}{5.088724in}}%
\pgfpathcurveto{\pgfqpoint{7.736142in}{5.096538in}}{\pgfqpoint{7.740533in}{5.107137in}}{\pgfqpoint{7.740533in}{5.118187in}}%
\pgfpathcurveto{\pgfqpoint{7.740533in}{5.129237in}}{\pgfqpoint{7.736142in}{5.139836in}}{\pgfqpoint{7.728329in}{5.147650in}}%
\pgfpathcurveto{\pgfqpoint{7.720515in}{5.155463in}}{\pgfqpoint{7.709916in}{5.159853in}}{\pgfqpoint{7.698866in}{5.159853in}}%
\pgfpathcurveto{\pgfqpoint{7.687816in}{5.159853in}}{\pgfqpoint{7.677217in}{5.155463in}}{\pgfqpoint{7.669403in}{5.147650in}}%
\pgfpathcurveto{\pgfqpoint{7.661589in}{5.139836in}}{\pgfqpoint{7.657199in}{5.129237in}}{\pgfqpoint{7.657199in}{5.118187in}}%
\pgfpathcurveto{\pgfqpoint{7.657199in}{5.107137in}}{\pgfqpoint{7.661589in}{5.096538in}}{\pgfqpoint{7.669403in}{5.088724in}}%
\pgfpathcurveto{\pgfqpoint{7.677217in}{5.080910in}}{\pgfqpoint{7.687816in}{5.076520in}}{\pgfqpoint{7.698866in}{5.076520in}}%
\pgfpathclose%
\pgfusepath{stroke,fill}%
\end{pgfscope}%
\begin{pgfscope}%
\pgfpathrectangle{\pgfqpoint{0.481978in}{0.331635in}}{\pgfqpoint{9.300000in}{7.700000in}}%
\pgfusepath{clip}%
\pgfsetbuttcap%
\pgfsetroundjoin%
\definecolor{currentfill}{rgb}{0.631373,0.788235,0.956863}%
\pgfsetfillcolor{currentfill}%
\pgfsetlinewidth{0.481800pt}%
\definecolor{currentstroke}{rgb}{1.000000,1.000000,1.000000}%
\pgfsetstrokecolor{currentstroke}%
\pgfsetdash{}{0pt}%
\pgfpathmoveto{\pgfqpoint{2.744094in}{5.008868in}}%
\pgfpathcurveto{\pgfqpoint{2.755144in}{5.008868in}}{\pgfqpoint{2.765743in}{5.013259in}}{\pgfqpoint{2.773557in}{5.021072in}}%
\pgfpathcurveto{\pgfqpoint{2.781370in}{5.028886in}}{\pgfqpoint{2.785760in}{5.039485in}}{\pgfqpoint{2.785760in}{5.050535in}}%
\pgfpathcurveto{\pgfqpoint{2.785760in}{5.061585in}}{\pgfqpoint{2.781370in}{5.072184in}}{\pgfqpoint{2.773557in}{5.079998in}}%
\pgfpathcurveto{\pgfqpoint{2.765743in}{5.087812in}}{\pgfqpoint{2.755144in}{5.092202in}}{\pgfqpoint{2.744094in}{5.092202in}}%
\pgfpathcurveto{\pgfqpoint{2.733044in}{5.092202in}}{\pgfqpoint{2.722445in}{5.087812in}}{\pgfqpoint{2.714631in}{5.079998in}}%
\pgfpathcurveto{\pgfqpoint{2.706817in}{5.072184in}}{\pgfqpoint{2.702427in}{5.061585in}}{\pgfqpoint{2.702427in}{5.050535in}}%
\pgfpathcurveto{\pgfqpoint{2.702427in}{5.039485in}}{\pgfqpoint{2.706817in}{5.028886in}}{\pgfqpoint{2.714631in}{5.021072in}}%
\pgfpathcurveto{\pgfqpoint{2.722445in}{5.013259in}}{\pgfqpoint{2.733044in}{5.008868in}}{\pgfqpoint{2.744094in}{5.008868in}}%
\pgfpathclose%
\pgfusepath{stroke,fill}%
\end{pgfscope}%
\begin{pgfscope}%
\pgfpathrectangle{\pgfqpoint{0.481978in}{0.331635in}}{\pgfqpoint{9.300000in}{7.700000in}}%
\pgfusepath{clip}%
\pgfsetbuttcap%
\pgfsetroundjoin%
\definecolor{currentfill}{rgb}{0.631373,0.788235,0.956863}%
\pgfsetfillcolor{currentfill}%
\pgfsetlinewidth{0.481800pt}%
\definecolor{currentstroke}{rgb}{1.000000,1.000000,1.000000}%
\pgfsetstrokecolor{currentstroke}%
\pgfsetdash{}{0pt}%
\pgfpathmoveto{\pgfqpoint{7.761828in}{4.277429in}}%
\pgfpathcurveto{\pgfqpoint{7.772878in}{4.277429in}}{\pgfqpoint{7.783477in}{4.281819in}}{\pgfqpoint{7.791291in}{4.289633in}}%
\pgfpathcurveto{\pgfqpoint{7.799104in}{4.297446in}}{\pgfqpoint{7.803495in}{4.308045in}}{\pgfqpoint{7.803495in}{4.319095in}}%
\pgfpathcurveto{\pgfqpoint{7.803495in}{4.330145in}}{\pgfqpoint{7.799104in}{4.340744in}}{\pgfqpoint{7.791291in}{4.348558in}}%
\pgfpathcurveto{\pgfqpoint{7.783477in}{4.356372in}}{\pgfqpoint{7.772878in}{4.360762in}}{\pgfqpoint{7.761828in}{4.360762in}}%
\pgfpathcurveto{\pgfqpoint{7.750778in}{4.360762in}}{\pgfqpoint{7.740179in}{4.356372in}}{\pgfqpoint{7.732365in}{4.348558in}}%
\pgfpathcurveto{\pgfqpoint{7.724551in}{4.340744in}}{\pgfqpoint{7.720161in}{4.330145in}}{\pgfqpoint{7.720161in}{4.319095in}}%
\pgfpathcurveto{\pgfqpoint{7.720161in}{4.308045in}}{\pgfqpoint{7.724551in}{4.297446in}}{\pgfqpoint{7.732365in}{4.289633in}}%
\pgfpathcurveto{\pgfqpoint{7.740179in}{4.281819in}}{\pgfqpoint{7.750778in}{4.277429in}}{\pgfqpoint{7.761828in}{4.277429in}}%
\pgfpathclose%
\pgfusepath{stroke,fill}%
\end{pgfscope}%
\begin{pgfscope}%
\pgfpathrectangle{\pgfqpoint{0.481978in}{0.331635in}}{\pgfqpoint{9.300000in}{7.700000in}}%
\pgfusepath{clip}%
\pgfsetbuttcap%
\pgfsetroundjoin%
\definecolor{currentfill}{rgb}{0.631373,0.788235,0.956863}%
\pgfsetfillcolor{currentfill}%
\pgfsetlinewidth{0.481800pt}%
\definecolor{currentstroke}{rgb}{1.000000,1.000000,1.000000}%
\pgfsetstrokecolor{currentstroke}%
\pgfsetdash{}{0pt}%
\pgfpathmoveto{\pgfqpoint{6.999874in}{2.207732in}}%
\pgfpathcurveto{\pgfqpoint{7.010924in}{2.207732in}}{\pgfqpoint{7.021523in}{2.212123in}}{\pgfqpoint{7.029337in}{2.219936in}}%
\pgfpathcurveto{\pgfqpoint{7.037150in}{2.227750in}}{\pgfqpoint{7.041540in}{2.238349in}}{\pgfqpoint{7.041540in}{2.249399in}}%
\pgfpathcurveto{\pgfqpoint{7.041540in}{2.260449in}}{\pgfqpoint{7.037150in}{2.271048in}}{\pgfqpoint{7.029337in}{2.278862in}}%
\pgfpathcurveto{\pgfqpoint{7.021523in}{2.286675in}}{\pgfqpoint{7.010924in}{2.291066in}}{\pgfqpoint{6.999874in}{2.291066in}}%
\pgfpathcurveto{\pgfqpoint{6.988824in}{2.291066in}}{\pgfqpoint{6.978225in}{2.286675in}}{\pgfqpoint{6.970411in}{2.278862in}}%
\pgfpathcurveto{\pgfqpoint{6.962597in}{2.271048in}}{\pgfqpoint{6.958207in}{2.260449in}}{\pgfqpoint{6.958207in}{2.249399in}}%
\pgfpathcurveto{\pgfqpoint{6.958207in}{2.238349in}}{\pgfqpoint{6.962597in}{2.227750in}}{\pgfqpoint{6.970411in}{2.219936in}}%
\pgfpathcurveto{\pgfqpoint{6.978225in}{2.212123in}}{\pgfqpoint{6.988824in}{2.207732in}}{\pgfqpoint{6.999874in}{2.207732in}}%
\pgfpathclose%
\pgfusepath{stroke,fill}%
\end{pgfscope}%
\begin{pgfscope}%
\pgfpathrectangle{\pgfqpoint{0.481978in}{0.331635in}}{\pgfqpoint{9.300000in}{7.700000in}}%
\pgfusepath{clip}%
\pgfsetbuttcap%
\pgfsetroundjoin%
\definecolor{currentfill}{rgb}{0.631373,0.788235,0.956863}%
\pgfsetfillcolor{currentfill}%
\pgfsetlinewidth{0.481800pt}%
\definecolor{currentstroke}{rgb}{1.000000,1.000000,1.000000}%
\pgfsetstrokecolor{currentstroke}%
\pgfsetdash{}{0pt}%
\pgfpathmoveto{\pgfqpoint{5.604350in}{5.389083in}}%
\pgfpathcurveto{\pgfqpoint{5.615400in}{5.389083in}}{\pgfqpoint{5.625999in}{5.393473in}}{\pgfqpoint{5.633813in}{5.401287in}}%
\pgfpathcurveto{\pgfqpoint{5.641627in}{5.409101in}}{\pgfqpoint{5.646017in}{5.419700in}}{\pgfqpoint{5.646017in}{5.430750in}}%
\pgfpathcurveto{\pgfqpoint{5.646017in}{5.441800in}}{\pgfqpoint{5.641627in}{5.452399in}}{\pgfqpoint{5.633813in}{5.460213in}}%
\pgfpathcurveto{\pgfqpoint{5.625999in}{5.468026in}}{\pgfqpoint{5.615400in}{5.472416in}}{\pgfqpoint{5.604350in}{5.472416in}}%
\pgfpathcurveto{\pgfqpoint{5.593300in}{5.472416in}}{\pgfqpoint{5.582701in}{5.468026in}}{\pgfqpoint{5.574887in}{5.460213in}}%
\pgfpathcurveto{\pgfqpoint{5.567074in}{5.452399in}}{\pgfqpoint{5.562684in}{5.441800in}}{\pgfqpoint{5.562684in}{5.430750in}}%
\pgfpathcurveto{\pgfqpoint{5.562684in}{5.419700in}}{\pgfqpoint{5.567074in}{5.409101in}}{\pgfqpoint{5.574887in}{5.401287in}}%
\pgfpathcurveto{\pgfqpoint{5.582701in}{5.393473in}}{\pgfqpoint{5.593300in}{5.389083in}}{\pgfqpoint{5.604350in}{5.389083in}}%
\pgfpathclose%
\pgfusepath{stroke,fill}%
\end{pgfscope}%
\begin{pgfscope}%
\pgfpathrectangle{\pgfqpoint{0.481978in}{0.331635in}}{\pgfqpoint{9.300000in}{7.700000in}}%
\pgfusepath{clip}%
\pgfsetbuttcap%
\pgfsetroundjoin%
\definecolor{currentfill}{rgb}{0.631373,0.788235,0.956863}%
\pgfsetfillcolor{currentfill}%
\pgfsetlinewidth{0.481800pt}%
\definecolor{currentstroke}{rgb}{1.000000,1.000000,1.000000}%
\pgfsetstrokecolor{currentstroke}%
\pgfsetdash{}{0pt}%
\pgfpathmoveto{\pgfqpoint{2.349323in}{2.233215in}}%
\pgfpathcurveto{\pgfqpoint{2.360373in}{2.233215in}}{\pgfqpoint{2.370972in}{2.237605in}}{\pgfqpoint{2.378786in}{2.245419in}}%
\pgfpathcurveto{\pgfqpoint{2.386599in}{2.253233in}}{\pgfqpoint{2.390990in}{2.263832in}}{\pgfqpoint{2.390990in}{2.274882in}}%
\pgfpathcurveto{\pgfqpoint{2.390990in}{2.285932in}}{\pgfqpoint{2.386599in}{2.296531in}}{\pgfqpoint{2.378786in}{2.304345in}}%
\pgfpathcurveto{\pgfqpoint{2.370972in}{2.312158in}}{\pgfqpoint{2.360373in}{2.316549in}}{\pgfqpoint{2.349323in}{2.316549in}}%
\pgfpathcurveto{\pgfqpoint{2.338273in}{2.316549in}}{\pgfqpoint{2.327674in}{2.312158in}}{\pgfqpoint{2.319860in}{2.304345in}}%
\pgfpathcurveto{\pgfqpoint{2.312046in}{2.296531in}}{\pgfqpoint{2.307656in}{2.285932in}}{\pgfqpoint{2.307656in}{2.274882in}}%
\pgfpathcurveto{\pgfqpoint{2.307656in}{2.263832in}}{\pgfqpoint{2.312046in}{2.253233in}}{\pgfqpoint{2.319860in}{2.245419in}}%
\pgfpathcurveto{\pgfqpoint{2.327674in}{2.237605in}}{\pgfqpoint{2.338273in}{2.233215in}}{\pgfqpoint{2.349323in}{2.233215in}}%
\pgfpathclose%
\pgfusepath{stroke,fill}%
\end{pgfscope}%
\begin{pgfscope}%
\pgfpathrectangle{\pgfqpoint{0.481978in}{0.331635in}}{\pgfqpoint{9.300000in}{7.700000in}}%
\pgfusepath{clip}%
\pgfsetbuttcap%
\pgfsetroundjoin%
\definecolor{currentfill}{rgb}{0.631373,0.788235,0.956863}%
\pgfsetfillcolor{currentfill}%
\pgfsetlinewidth{0.481800pt}%
\definecolor{currentstroke}{rgb}{1.000000,1.000000,1.000000}%
\pgfsetstrokecolor{currentstroke}%
\pgfsetdash{}{0pt}%
\pgfpathmoveto{\pgfqpoint{6.569928in}{3.215334in}}%
\pgfpathcurveto{\pgfqpoint{6.580979in}{3.215334in}}{\pgfqpoint{6.591578in}{3.219724in}}{\pgfqpoint{6.599391in}{3.227538in}}%
\pgfpathcurveto{\pgfqpoint{6.607205in}{3.235351in}}{\pgfqpoint{6.611595in}{3.245950in}}{\pgfqpoint{6.611595in}{3.257001in}}%
\pgfpathcurveto{\pgfqpoint{6.611595in}{3.268051in}}{\pgfqpoint{6.607205in}{3.278650in}}{\pgfqpoint{6.599391in}{3.286463in}}%
\pgfpathcurveto{\pgfqpoint{6.591578in}{3.294277in}}{\pgfqpoint{6.580979in}{3.298667in}}{\pgfqpoint{6.569928in}{3.298667in}}%
\pgfpathcurveto{\pgfqpoint{6.558878in}{3.298667in}}{\pgfqpoint{6.548279in}{3.294277in}}{\pgfqpoint{6.540466in}{3.286463in}}%
\pgfpathcurveto{\pgfqpoint{6.532652in}{3.278650in}}{\pgfqpoint{6.528262in}{3.268051in}}{\pgfqpoint{6.528262in}{3.257001in}}%
\pgfpathcurveto{\pgfqpoint{6.528262in}{3.245950in}}{\pgfqpoint{6.532652in}{3.235351in}}{\pgfqpoint{6.540466in}{3.227538in}}%
\pgfpathcurveto{\pgfqpoint{6.548279in}{3.219724in}}{\pgfqpoint{6.558878in}{3.215334in}}{\pgfqpoint{6.569928in}{3.215334in}}%
\pgfpathclose%
\pgfusepath{stroke,fill}%
\end{pgfscope}%
\begin{pgfscope}%
\pgfpathrectangle{\pgfqpoint{0.481978in}{0.331635in}}{\pgfqpoint{9.300000in}{7.700000in}}%
\pgfusepath{clip}%
\pgfsetbuttcap%
\pgfsetroundjoin%
\definecolor{currentfill}{rgb}{0.631373,0.788235,0.956863}%
\pgfsetfillcolor{currentfill}%
\pgfsetlinewidth{0.481800pt}%
\definecolor{currentstroke}{rgb}{1.000000,1.000000,1.000000}%
\pgfsetstrokecolor{currentstroke}%
\pgfsetdash{}{0pt}%
\pgfpathmoveto{\pgfqpoint{5.147663in}{2.571148in}}%
\pgfpathcurveto{\pgfqpoint{5.158713in}{2.571148in}}{\pgfqpoint{5.169312in}{2.575539in}}{\pgfqpoint{5.177125in}{2.583352in}}%
\pgfpathcurveto{\pgfqpoint{5.184939in}{2.591166in}}{\pgfqpoint{5.189329in}{2.601765in}}{\pgfqpoint{5.189329in}{2.612815in}}%
\pgfpathcurveto{\pgfqpoint{5.189329in}{2.623865in}}{\pgfqpoint{5.184939in}{2.634464in}}{\pgfqpoint{5.177125in}{2.642278in}}%
\pgfpathcurveto{\pgfqpoint{5.169312in}{2.650091in}}{\pgfqpoint{5.158713in}{2.654482in}}{\pgfqpoint{5.147663in}{2.654482in}}%
\pgfpathcurveto{\pgfqpoint{5.136613in}{2.654482in}}{\pgfqpoint{5.126014in}{2.650091in}}{\pgfqpoint{5.118200in}{2.642278in}}%
\pgfpathcurveto{\pgfqpoint{5.110386in}{2.634464in}}{\pgfqpoint{5.105996in}{2.623865in}}{\pgfqpoint{5.105996in}{2.612815in}}%
\pgfpathcurveto{\pgfqpoint{5.105996in}{2.601765in}}{\pgfqpoint{5.110386in}{2.591166in}}{\pgfqpoint{5.118200in}{2.583352in}}%
\pgfpathcurveto{\pgfqpoint{5.126014in}{2.575539in}}{\pgfqpoint{5.136613in}{2.571148in}}{\pgfqpoint{5.147663in}{2.571148in}}%
\pgfpathclose%
\pgfusepath{stroke,fill}%
\end{pgfscope}%
\begin{pgfscope}%
\pgfpathrectangle{\pgfqpoint{0.481978in}{0.331635in}}{\pgfqpoint{9.300000in}{7.700000in}}%
\pgfusepath{clip}%
\pgfsetbuttcap%
\pgfsetroundjoin%
\definecolor{currentfill}{rgb}{0.631373,0.788235,0.956863}%
\pgfsetfillcolor{currentfill}%
\pgfsetlinewidth{0.481800pt}%
\definecolor{currentstroke}{rgb}{1.000000,1.000000,1.000000}%
\pgfsetstrokecolor{currentstroke}%
\pgfsetdash{}{0pt}%
\pgfpathmoveto{\pgfqpoint{6.704806in}{2.191913in}}%
\pgfpathcurveto{\pgfqpoint{6.715856in}{2.191913in}}{\pgfqpoint{6.726455in}{2.196303in}}{\pgfqpoint{6.734269in}{2.204117in}}%
\pgfpathcurveto{\pgfqpoint{6.742083in}{2.211930in}}{\pgfqpoint{6.746473in}{2.222529in}}{\pgfqpoint{6.746473in}{2.233579in}}%
\pgfpathcurveto{\pgfqpoint{6.746473in}{2.244629in}}{\pgfqpoint{6.742083in}{2.255228in}}{\pgfqpoint{6.734269in}{2.263042in}}%
\pgfpathcurveto{\pgfqpoint{6.726455in}{2.270856in}}{\pgfqpoint{6.715856in}{2.275246in}}{\pgfqpoint{6.704806in}{2.275246in}}%
\pgfpathcurveto{\pgfqpoint{6.693756in}{2.275246in}}{\pgfqpoint{6.683157in}{2.270856in}}{\pgfqpoint{6.675343in}{2.263042in}}%
\pgfpathcurveto{\pgfqpoint{6.667530in}{2.255228in}}{\pgfqpoint{6.663140in}{2.244629in}}{\pgfqpoint{6.663140in}{2.233579in}}%
\pgfpathcurveto{\pgfqpoint{6.663140in}{2.222529in}}{\pgfqpoint{6.667530in}{2.211930in}}{\pgfqpoint{6.675343in}{2.204117in}}%
\pgfpathcurveto{\pgfqpoint{6.683157in}{2.196303in}}{\pgfqpoint{6.693756in}{2.191913in}}{\pgfqpoint{6.704806in}{2.191913in}}%
\pgfpathclose%
\pgfusepath{stroke,fill}%
\end{pgfscope}%
\begin{pgfscope}%
\pgfpathrectangle{\pgfqpoint{0.481978in}{0.331635in}}{\pgfqpoint{9.300000in}{7.700000in}}%
\pgfusepath{clip}%
\pgfsetbuttcap%
\pgfsetroundjoin%
\definecolor{currentfill}{rgb}{0.631373,0.788235,0.956863}%
\pgfsetfillcolor{currentfill}%
\pgfsetlinewidth{0.481800pt}%
\definecolor{currentstroke}{rgb}{1.000000,1.000000,1.000000}%
\pgfsetstrokecolor{currentstroke}%
\pgfsetdash{}{0pt}%
\pgfpathmoveto{\pgfqpoint{2.794290in}{5.331159in}}%
\pgfpathcurveto{\pgfqpoint{2.805340in}{5.331159in}}{\pgfqpoint{2.815940in}{5.335549in}}{\pgfqpoint{2.823753in}{5.343363in}}%
\pgfpathcurveto{\pgfqpoint{2.831567in}{5.351176in}}{\pgfqpoint{2.835957in}{5.361775in}}{\pgfqpoint{2.835957in}{5.372826in}}%
\pgfpathcurveto{\pgfqpoint{2.835957in}{5.383876in}}{\pgfqpoint{2.831567in}{5.394475in}}{\pgfqpoint{2.823753in}{5.402288in}}%
\pgfpathcurveto{\pgfqpoint{2.815940in}{5.410102in}}{\pgfqpoint{2.805340in}{5.414492in}}{\pgfqpoint{2.794290in}{5.414492in}}%
\pgfpathcurveto{\pgfqpoint{2.783240in}{5.414492in}}{\pgfqpoint{2.772641in}{5.410102in}}{\pgfqpoint{2.764828in}{5.402288in}}%
\pgfpathcurveto{\pgfqpoint{2.757014in}{5.394475in}}{\pgfqpoint{2.752624in}{5.383876in}}{\pgfqpoint{2.752624in}{5.372826in}}%
\pgfpathcurveto{\pgfqpoint{2.752624in}{5.361775in}}{\pgfqpoint{2.757014in}{5.351176in}}{\pgfqpoint{2.764828in}{5.343363in}}%
\pgfpathcurveto{\pgfqpoint{2.772641in}{5.335549in}}{\pgfqpoint{2.783240in}{5.331159in}}{\pgfqpoint{2.794290in}{5.331159in}}%
\pgfpathclose%
\pgfusepath{stroke,fill}%
\end{pgfscope}%
\begin{pgfscope}%
\pgfpathrectangle{\pgfqpoint{0.481978in}{0.331635in}}{\pgfqpoint{9.300000in}{7.700000in}}%
\pgfusepath{clip}%
\pgfsetbuttcap%
\pgfsetroundjoin%
\definecolor{currentfill}{rgb}{0.631373,0.788235,0.956863}%
\pgfsetfillcolor{currentfill}%
\pgfsetlinewidth{0.481800pt}%
\definecolor{currentstroke}{rgb}{1.000000,1.000000,1.000000}%
\pgfsetstrokecolor{currentstroke}%
\pgfsetdash{}{0pt}%
\pgfpathmoveto{\pgfqpoint{8.083200in}{4.965417in}}%
\pgfpathcurveto{\pgfqpoint{8.094250in}{4.965417in}}{\pgfqpoint{8.104849in}{4.969807in}}{\pgfqpoint{8.112663in}{4.977621in}}%
\pgfpathcurveto{\pgfqpoint{8.120477in}{4.985434in}}{\pgfqpoint{8.124867in}{4.996033in}}{\pgfqpoint{8.124867in}{5.007083in}}%
\pgfpathcurveto{\pgfqpoint{8.124867in}{5.018134in}}{\pgfqpoint{8.120477in}{5.028733in}}{\pgfqpoint{8.112663in}{5.036546in}}%
\pgfpathcurveto{\pgfqpoint{8.104849in}{5.044360in}}{\pgfqpoint{8.094250in}{5.048750in}}{\pgfqpoint{8.083200in}{5.048750in}}%
\pgfpathcurveto{\pgfqpoint{8.072150in}{5.048750in}}{\pgfqpoint{8.061551in}{5.044360in}}{\pgfqpoint{8.053738in}{5.036546in}}%
\pgfpathcurveto{\pgfqpoint{8.045924in}{5.028733in}}{\pgfqpoint{8.041534in}{5.018134in}}{\pgfqpoint{8.041534in}{5.007083in}}%
\pgfpathcurveto{\pgfqpoint{8.041534in}{4.996033in}}{\pgfqpoint{8.045924in}{4.985434in}}{\pgfqpoint{8.053738in}{4.977621in}}%
\pgfpathcurveto{\pgfqpoint{8.061551in}{4.969807in}}{\pgfqpoint{8.072150in}{4.965417in}}{\pgfqpoint{8.083200in}{4.965417in}}%
\pgfpathclose%
\pgfusepath{stroke,fill}%
\end{pgfscope}%
\begin{pgfscope}%
\pgfpathrectangle{\pgfqpoint{0.481978in}{0.331635in}}{\pgfqpoint{9.300000in}{7.700000in}}%
\pgfusepath{clip}%
\pgfsetbuttcap%
\pgfsetroundjoin%
\definecolor{currentfill}{rgb}{0.631373,0.788235,0.956863}%
\pgfsetfillcolor{currentfill}%
\pgfsetlinewidth{0.481800pt}%
\definecolor{currentstroke}{rgb}{1.000000,1.000000,1.000000}%
\pgfsetstrokecolor{currentstroke}%
\pgfsetdash{}{0pt}%
\pgfpathmoveto{\pgfqpoint{4.500321in}{1.281036in}}%
\pgfpathcurveto{\pgfqpoint{4.511371in}{1.281036in}}{\pgfqpoint{4.521970in}{1.285427in}}{\pgfqpoint{4.529783in}{1.293240in}}%
\pgfpathcurveto{\pgfqpoint{4.537597in}{1.301054in}}{\pgfqpoint{4.541987in}{1.311653in}}{\pgfqpoint{4.541987in}{1.322703in}}%
\pgfpathcurveto{\pgfqpoint{4.541987in}{1.333753in}}{\pgfqpoint{4.537597in}{1.344352in}}{\pgfqpoint{4.529783in}{1.352166in}}%
\pgfpathcurveto{\pgfqpoint{4.521970in}{1.359980in}}{\pgfqpoint{4.511371in}{1.364370in}}{\pgfqpoint{4.500321in}{1.364370in}}%
\pgfpathcurveto{\pgfqpoint{4.489270in}{1.364370in}}{\pgfqpoint{4.478671in}{1.359980in}}{\pgfqpoint{4.470858in}{1.352166in}}%
\pgfpathcurveto{\pgfqpoint{4.463044in}{1.344352in}}{\pgfqpoint{4.458654in}{1.333753in}}{\pgfqpoint{4.458654in}{1.322703in}}%
\pgfpathcurveto{\pgfqpoint{4.458654in}{1.311653in}}{\pgfqpoint{4.463044in}{1.301054in}}{\pgfqpoint{4.470858in}{1.293240in}}%
\pgfpathcurveto{\pgfqpoint{4.478671in}{1.285427in}}{\pgfqpoint{4.489270in}{1.281036in}}{\pgfqpoint{4.500321in}{1.281036in}}%
\pgfpathclose%
\pgfusepath{stroke,fill}%
\end{pgfscope}%
\begin{pgfscope}%
\pgfpathrectangle{\pgfqpoint{0.481978in}{0.331635in}}{\pgfqpoint{9.300000in}{7.700000in}}%
\pgfusepath{clip}%
\pgfsetbuttcap%
\pgfsetroundjoin%
\definecolor{currentfill}{rgb}{0.631373,0.788235,0.956863}%
\pgfsetfillcolor{currentfill}%
\pgfsetlinewidth{0.481800pt}%
\definecolor{currentstroke}{rgb}{1.000000,1.000000,1.000000}%
\pgfsetstrokecolor{currentstroke}%
\pgfsetdash{}{0pt}%
\pgfpathmoveto{\pgfqpoint{3.168751in}{1.333831in}}%
\pgfpathcurveto{\pgfqpoint{3.179801in}{1.333831in}}{\pgfqpoint{3.190400in}{1.338221in}}{\pgfqpoint{3.198214in}{1.346035in}}%
\pgfpathcurveto{\pgfqpoint{3.206027in}{1.353849in}}{\pgfqpoint{3.210418in}{1.364448in}}{\pgfqpoint{3.210418in}{1.375498in}}%
\pgfpathcurveto{\pgfqpoint{3.210418in}{1.386548in}}{\pgfqpoint{3.206027in}{1.397147in}}{\pgfqpoint{3.198214in}{1.404961in}}%
\pgfpathcurveto{\pgfqpoint{3.190400in}{1.412774in}}{\pgfqpoint{3.179801in}{1.417164in}}{\pgfqpoint{3.168751in}{1.417164in}}%
\pgfpathcurveto{\pgfqpoint{3.157701in}{1.417164in}}{\pgfqpoint{3.147102in}{1.412774in}}{\pgfqpoint{3.139288in}{1.404961in}}%
\pgfpathcurveto{\pgfqpoint{3.131475in}{1.397147in}}{\pgfqpoint{3.127084in}{1.386548in}}{\pgfqpoint{3.127084in}{1.375498in}}%
\pgfpathcurveto{\pgfqpoint{3.127084in}{1.364448in}}{\pgfqpoint{3.131475in}{1.353849in}}{\pgfqpoint{3.139288in}{1.346035in}}%
\pgfpathcurveto{\pgfqpoint{3.147102in}{1.338221in}}{\pgfqpoint{3.157701in}{1.333831in}}{\pgfqpoint{3.168751in}{1.333831in}}%
\pgfpathclose%
\pgfusepath{stroke,fill}%
\end{pgfscope}%
\begin{pgfscope}%
\pgfpathrectangle{\pgfqpoint{0.481978in}{0.331635in}}{\pgfqpoint{9.300000in}{7.700000in}}%
\pgfusepath{clip}%
\pgfsetbuttcap%
\pgfsetroundjoin%
\definecolor{currentfill}{rgb}{0.631373,0.788235,0.956863}%
\pgfsetfillcolor{currentfill}%
\pgfsetlinewidth{0.481800pt}%
\definecolor{currentstroke}{rgb}{1.000000,1.000000,1.000000}%
\pgfsetstrokecolor{currentstroke}%
\pgfsetdash{}{0pt}%
\pgfpathmoveto{\pgfqpoint{4.177608in}{7.639968in}}%
\pgfpathcurveto{\pgfqpoint{4.188659in}{7.639968in}}{\pgfqpoint{4.199258in}{7.644359in}}{\pgfqpoint{4.207071in}{7.652172in}}%
\pgfpathcurveto{\pgfqpoint{4.214885in}{7.659986in}}{\pgfqpoint{4.219275in}{7.670585in}}{\pgfqpoint{4.219275in}{7.681635in}}%
\pgfpathcurveto{\pgfqpoint{4.219275in}{7.692685in}}{\pgfqpoint{4.214885in}{7.703284in}}{\pgfqpoint{4.207071in}{7.711098in}}%
\pgfpathcurveto{\pgfqpoint{4.199258in}{7.718911in}}{\pgfqpoint{4.188659in}{7.723302in}}{\pgfqpoint{4.177608in}{7.723302in}}%
\pgfpathcurveto{\pgfqpoint{4.166558in}{7.723302in}}{\pgfqpoint{4.155959in}{7.718911in}}{\pgfqpoint{4.148146in}{7.711098in}}%
\pgfpathcurveto{\pgfqpoint{4.140332in}{7.703284in}}{\pgfqpoint{4.135942in}{7.692685in}}{\pgfqpoint{4.135942in}{7.681635in}}%
\pgfpathcurveto{\pgfqpoint{4.135942in}{7.670585in}}{\pgfqpoint{4.140332in}{7.659986in}}{\pgfqpoint{4.148146in}{7.652172in}}%
\pgfpathcurveto{\pgfqpoint{4.155959in}{7.644359in}}{\pgfqpoint{4.166558in}{7.639968in}}{\pgfqpoint{4.177608in}{7.639968in}}%
\pgfpathclose%
\pgfusepath{stroke,fill}%
\end{pgfscope}%
\begin{pgfscope}%
\pgfpathrectangle{\pgfqpoint{0.481978in}{0.331635in}}{\pgfqpoint{9.300000in}{7.700000in}}%
\pgfusepath{clip}%
\pgfsetbuttcap%
\pgfsetroundjoin%
\definecolor{currentfill}{rgb}{0.631373,0.788235,0.956863}%
\pgfsetfillcolor{currentfill}%
\pgfsetlinewidth{0.481800pt}%
\definecolor{currentstroke}{rgb}{1.000000,1.000000,1.000000}%
\pgfsetstrokecolor{currentstroke}%
\pgfsetdash{}{0pt}%
\pgfpathmoveto{\pgfqpoint{6.895459in}{3.133907in}}%
\pgfpathcurveto{\pgfqpoint{6.906509in}{3.133907in}}{\pgfqpoint{6.917108in}{3.138297in}}{\pgfqpoint{6.924921in}{3.146111in}}%
\pgfpathcurveto{\pgfqpoint{6.932735in}{3.153925in}}{\pgfqpoint{6.937125in}{3.164524in}}{\pgfqpoint{6.937125in}{3.175574in}}%
\pgfpathcurveto{\pgfqpoint{6.937125in}{3.186624in}}{\pgfqpoint{6.932735in}{3.197223in}}{\pgfqpoint{6.924921in}{3.205037in}}%
\pgfpathcurveto{\pgfqpoint{6.917108in}{3.212850in}}{\pgfqpoint{6.906509in}{3.217240in}}{\pgfqpoint{6.895459in}{3.217240in}}%
\pgfpathcurveto{\pgfqpoint{6.884409in}{3.217240in}}{\pgfqpoint{6.873810in}{3.212850in}}{\pgfqpoint{6.865996in}{3.205037in}}%
\pgfpathcurveto{\pgfqpoint{6.858182in}{3.197223in}}{\pgfqpoint{6.853792in}{3.186624in}}{\pgfqpoint{6.853792in}{3.175574in}}%
\pgfpathcurveto{\pgfqpoint{6.853792in}{3.164524in}}{\pgfqpoint{6.858182in}{3.153925in}}{\pgfqpoint{6.865996in}{3.146111in}}%
\pgfpathcurveto{\pgfqpoint{6.873810in}{3.138297in}}{\pgfqpoint{6.884409in}{3.133907in}}{\pgfqpoint{6.895459in}{3.133907in}}%
\pgfpathclose%
\pgfusepath{stroke,fill}%
\end{pgfscope}%
\begin{pgfscope}%
\pgfpathrectangle{\pgfqpoint{0.481978in}{0.331635in}}{\pgfqpoint{9.300000in}{7.700000in}}%
\pgfusepath{clip}%
\pgfsetbuttcap%
\pgfsetroundjoin%
\definecolor{currentfill}{rgb}{0.631373,0.788235,0.956863}%
\pgfsetfillcolor{currentfill}%
\pgfsetlinewidth{0.481800pt}%
\definecolor{currentstroke}{rgb}{1.000000,1.000000,1.000000}%
\pgfsetstrokecolor{currentstroke}%
\pgfsetdash{}{0pt}%
\pgfpathmoveto{\pgfqpoint{7.647915in}{1.867122in}}%
\pgfpathcurveto{\pgfqpoint{7.658966in}{1.867122in}}{\pgfqpoint{7.669565in}{1.871512in}}{\pgfqpoint{7.677378in}{1.879325in}}%
\pgfpathcurveto{\pgfqpoint{7.685192in}{1.887139in}}{\pgfqpoint{7.689582in}{1.897738in}}{\pgfqpoint{7.689582in}{1.908788in}}%
\pgfpathcurveto{\pgfqpoint{7.689582in}{1.919838in}}{\pgfqpoint{7.685192in}{1.930437in}}{\pgfqpoint{7.677378in}{1.938251in}}%
\pgfpathcurveto{\pgfqpoint{7.669565in}{1.946065in}}{\pgfqpoint{7.658966in}{1.950455in}}{\pgfqpoint{7.647915in}{1.950455in}}%
\pgfpathcurveto{\pgfqpoint{7.636865in}{1.950455in}}{\pgfqpoint{7.626266in}{1.946065in}}{\pgfqpoint{7.618453in}{1.938251in}}%
\pgfpathcurveto{\pgfqpoint{7.610639in}{1.930437in}}{\pgfqpoint{7.606249in}{1.919838in}}{\pgfqpoint{7.606249in}{1.908788in}}%
\pgfpathcurveto{\pgfqpoint{7.606249in}{1.897738in}}{\pgfqpoint{7.610639in}{1.887139in}}{\pgfqpoint{7.618453in}{1.879325in}}%
\pgfpathcurveto{\pgfqpoint{7.626266in}{1.871512in}}{\pgfqpoint{7.636865in}{1.867122in}}{\pgfqpoint{7.647915in}{1.867122in}}%
\pgfpathclose%
\pgfusepath{stroke,fill}%
\end{pgfscope}%
\begin{pgfscope}%
\pgfpathrectangle{\pgfqpoint{0.481978in}{0.331635in}}{\pgfqpoint{9.300000in}{7.700000in}}%
\pgfusepath{clip}%
\pgfsetbuttcap%
\pgfsetroundjoin%
\definecolor{currentfill}{rgb}{0.631373,0.788235,0.956863}%
\pgfsetfillcolor{currentfill}%
\pgfsetlinewidth{0.481800pt}%
\definecolor{currentstroke}{rgb}{1.000000,1.000000,1.000000}%
\pgfsetstrokecolor{currentstroke}%
\pgfsetdash{}{0pt}%
\pgfpathmoveto{\pgfqpoint{7.913291in}{4.938828in}}%
\pgfpathcurveto{\pgfqpoint{7.924341in}{4.938828in}}{\pgfqpoint{7.934940in}{4.943219in}}{\pgfqpoint{7.942753in}{4.951032in}}%
\pgfpathcurveto{\pgfqpoint{7.950567in}{4.958846in}}{\pgfqpoint{7.954957in}{4.969445in}}{\pgfqpoint{7.954957in}{4.980495in}}%
\pgfpathcurveto{\pgfqpoint{7.954957in}{4.991545in}}{\pgfqpoint{7.950567in}{5.002144in}}{\pgfqpoint{7.942753in}{5.009958in}}%
\pgfpathcurveto{\pgfqpoint{7.934940in}{5.017771in}}{\pgfqpoint{7.924341in}{5.022162in}}{\pgfqpoint{7.913291in}{5.022162in}}%
\pgfpathcurveto{\pgfqpoint{7.902240in}{5.022162in}}{\pgfqpoint{7.891641in}{5.017771in}}{\pgfqpoint{7.883828in}{5.009958in}}%
\pgfpathcurveto{\pgfqpoint{7.876014in}{5.002144in}}{\pgfqpoint{7.871624in}{4.991545in}}{\pgfqpoint{7.871624in}{4.980495in}}%
\pgfpathcurveto{\pgfqpoint{7.871624in}{4.969445in}}{\pgfqpoint{7.876014in}{4.958846in}}{\pgfqpoint{7.883828in}{4.951032in}}%
\pgfpathcurveto{\pgfqpoint{7.891641in}{4.943219in}}{\pgfqpoint{7.902240in}{4.938828in}}{\pgfqpoint{7.913291in}{4.938828in}}%
\pgfpathclose%
\pgfusepath{stroke,fill}%
\end{pgfscope}%
\begin{pgfscope}%
\pgfpathrectangle{\pgfqpoint{0.481978in}{0.331635in}}{\pgfqpoint{9.300000in}{7.700000in}}%
\pgfusepath{clip}%
\pgfsetbuttcap%
\pgfsetroundjoin%
\definecolor{currentfill}{rgb}{0.631373,0.788235,0.956863}%
\pgfsetfillcolor{currentfill}%
\pgfsetlinewidth{0.481800pt}%
\definecolor{currentstroke}{rgb}{1.000000,1.000000,1.000000}%
\pgfsetstrokecolor{currentstroke}%
\pgfsetdash{}{0pt}%
\pgfpathmoveto{\pgfqpoint{5.605585in}{2.822785in}}%
\pgfpathcurveto{\pgfqpoint{5.616635in}{2.822785in}}{\pgfqpoint{5.627234in}{2.827175in}}{\pgfqpoint{5.635048in}{2.834989in}}%
\pgfpathcurveto{\pgfqpoint{5.642861in}{2.842803in}}{\pgfqpoint{5.647252in}{2.853402in}}{\pgfqpoint{5.647252in}{2.864452in}}%
\pgfpathcurveto{\pgfqpoint{5.647252in}{2.875502in}}{\pgfqpoint{5.642861in}{2.886101in}}{\pgfqpoint{5.635048in}{2.893915in}}%
\pgfpathcurveto{\pgfqpoint{5.627234in}{2.901728in}}{\pgfqpoint{5.616635in}{2.906118in}}{\pgfqpoint{5.605585in}{2.906118in}}%
\pgfpathcurveto{\pgfqpoint{5.594535in}{2.906118in}}{\pgfqpoint{5.583936in}{2.901728in}}{\pgfqpoint{5.576122in}{2.893915in}}%
\pgfpathcurveto{\pgfqpoint{5.568309in}{2.886101in}}{\pgfqpoint{5.563918in}{2.875502in}}{\pgfqpoint{5.563918in}{2.864452in}}%
\pgfpathcurveto{\pgfqpoint{5.563918in}{2.853402in}}{\pgfqpoint{5.568309in}{2.842803in}}{\pgfqpoint{5.576122in}{2.834989in}}%
\pgfpathcurveto{\pgfqpoint{5.583936in}{2.827175in}}{\pgfqpoint{5.594535in}{2.822785in}}{\pgfqpoint{5.605585in}{2.822785in}}%
\pgfpathclose%
\pgfusepath{stroke,fill}%
\end{pgfscope}%
\begin{pgfscope}%
\pgfpathrectangle{\pgfqpoint{0.481978in}{0.331635in}}{\pgfqpoint{9.300000in}{7.700000in}}%
\pgfusepath{clip}%
\pgfsetbuttcap%
\pgfsetroundjoin%
\definecolor{currentfill}{rgb}{0.631373,0.788235,0.956863}%
\pgfsetfillcolor{currentfill}%
\pgfsetlinewidth{0.481800pt}%
\definecolor{currentstroke}{rgb}{1.000000,1.000000,1.000000}%
\pgfsetstrokecolor{currentstroke}%
\pgfsetdash{}{0pt}%
\pgfpathmoveto{\pgfqpoint{4.066101in}{5.114608in}}%
\pgfpathcurveto{\pgfqpoint{4.077151in}{5.114608in}}{\pgfqpoint{4.087750in}{5.118998in}}{\pgfqpoint{4.095563in}{5.126812in}}%
\pgfpathcurveto{\pgfqpoint{4.103377in}{5.134625in}}{\pgfqpoint{4.107767in}{5.145224in}}{\pgfqpoint{4.107767in}{5.156275in}}%
\pgfpathcurveto{\pgfqpoint{4.107767in}{5.167325in}}{\pgfqpoint{4.103377in}{5.177924in}}{\pgfqpoint{4.095563in}{5.185737in}}%
\pgfpathcurveto{\pgfqpoint{4.087750in}{5.193551in}}{\pgfqpoint{4.077151in}{5.197941in}}{\pgfqpoint{4.066101in}{5.197941in}}%
\pgfpathcurveto{\pgfqpoint{4.055051in}{5.197941in}}{\pgfqpoint{4.044452in}{5.193551in}}{\pgfqpoint{4.036638in}{5.185737in}}%
\pgfpathcurveto{\pgfqpoint{4.028824in}{5.177924in}}{\pgfqpoint{4.024434in}{5.167325in}}{\pgfqpoint{4.024434in}{5.156275in}}%
\pgfpathcurveto{\pgfqpoint{4.024434in}{5.145224in}}{\pgfqpoint{4.028824in}{5.134625in}}{\pgfqpoint{4.036638in}{5.126812in}}%
\pgfpathcurveto{\pgfqpoint{4.044452in}{5.118998in}}{\pgfqpoint{4.055051in}{5.114608in}}{\pgfqpoint{4.066101in}{5.114608in}}%
\pgfpathclose%
\pgfusepath{stroke,fill}%
\end{pgfscope}%
\begin{pgfscope}%
\pgfpathrectangle{\pgfqpoint{0.481978in}{0.331635in}}{\pgfqpoint{9.300000in}{7.700000in}}%
\pgfusepath{clip}%
\pgfsetbuttcap%
\pgfsetroundjoin%
\definecolor{currentfill}{rgb}{0.631373,0.788235,0.956863}%
\pgfsetfillcolor{currentfill}%
\pgfsetlinewidth{0.481800pt}%
\definecolor{currentstroke}{rgb}{1.000000,1.000000,1.000000}%
\pgfsetstrokecolor{currentstroke}%
\pgfsetdash{}{0pt}%
\pgfpathmoveto{\pgfqpoint{6.162715in}{2.512318in}}%
\pgfpathcurveto{\pgfqpoint{6.173765in}{2.512318in}}{\pgfqpoint{6.184364in}{2.516708in}}{\pgfqpoint{6.192178in}{2.524522in}}%
\pgfpathcurveto{\pgfqpoint{6.199991in}{2.532335in}}{\pgfqpoint{6.204382in}{2.542934in}}{\pgfqpoint{6.204382in}{2.553984in}}%
\pgfpathcurveto{\pgfqpoint{6.204382in}{2.565035in}}{\pgfqpoint{6.199991in}{2.575634in}}{\pgfqpoint{6.192178in}{2.583447in}}%
\pgfpathcurveto{\pgfqpoint{6.184364in}{2.591261in}}{\pgfqpoint{6.173765in}{2.595651in}}{\pgfqpoint{6.162715in}{2.595651in}}%
\pgfpathcurveto{\pgfqpoint{6.151665in}{2.595651in}}{\pgfqpoint{6.141066in}{2.591261in}}{\pgfqpoint{6.133252in}{2.583447in}}%
\pgfpathcurveto{\pgfqpoint{6.125439in}{2.575634in}}{\pgfqpoint{6.121048in}{2.565035in}}{\pgfqpoint{6.121048in}{2.553984in}}%
\pgfpathcurveto{\pgfqpoint{6.121048in}{2.542934in}}{\pgfqpoint{6.125439in}{2.532335in}}{\pgfqpoint{6.133252in}{2.524522in}}%
\pgfpathcurveto{\pgfqpoint{6.141066in}{2.516708in}}{\pgfqpoint{6.151665in}{2.512318in}}{\pgfqpoint{6.162715in}{2.512318in}}%
\pgfpathclose%
\pgfusepath{stroke,fill}%
\end{pgfscope}%
\begin{pgfscope}%
\pgfpathrectangle{\pgfqpoint{0.481978in}{0.331635in}}{\pgfqpoint{9.300000in}{7.700000in}}%
\pgfusepath{clip}%
\pgfsetbuttcap%
\pgfsetroundjoin%
\definecolor{currentfill}{rgb}{0.631373,0.788235,0.956863}%
\pgfsetfillcolor{currentfill}%
\pgfsetlinewidth{0.481800pt}%
\definecolor{currentstroke}{rgb}{1.000000,1.000000,1.000000}%
\pgfsetstrokecolor{currentstroke}%
\pgfsetdash{}{0pt}%
\pgfpathmoveto{\pgfqpoint{6.160576in}{3.090672in}}%
\pgfpathcurveto{\pgfqpoint{6.171626in}{3.090672in}}{\pgfqpoint{6.182225in}{3.095062in}}{\pgfqpoint{6.190039in}{3.102876in}}%
\pgfpathcurveto{\pgfqpoint{6.197852in}{3.110690in}}{\pgfqpoint{6.202242in}{3.121289in}}{\pgfqpoint{6.202242in}{3.132339in}}%
\pgfpathcurveto{\pgfqpoint{6.202242in}{3.143389in}}{\pgfqpoint{6.197852in}{3.153988in}}{\pgfqpoint{6.190039in}{3.161801in}}%
\pgfpathcurveto{\pgfqpoint{6.182225in}{3.169615in}}{\pgfqpoint{6.171626in}{3.174005in}}{\pgfqpoint{6.160576in}{3.174005in}}%
\pgfpathcurveto{\pgfqpoint{6.149526in}{3.174005in}}{\pgfqpoint{6.138927in}{3.169615in}}{\pgfqpoint{6.131113in}{3.161801in}}%
\pgfpathcurveto{\pgfqpoint{6.123299in}{3.153988in}}{\pgfqpoint{6.118909in}{3.143389in}}{\pgfqpoint{6.118909in}{3.132339in}}%
\pgfpathcurveto{\pgfqpoint{6.118909in}{3.121289in}}{\pgfqpoint{6.123299in}{3.110690in}}{\pgfqpoint{6.131113in}{3.102876in}}%
\pgfpathcurveto{\pgfqpoint{6.138927in}{3.095062in}}{\pgfqpoint{6.149526in}{3.090672in}}{\pgfqpoint{6.160576in}{3.090672in}}%
\pgfpathclose%
\pgfusepath{stroke,fill}%
\end{pgfscope}%
\begin{pgfscope}%
\pgfpathrectangle{\pgfqpoint{0.481978in}{0.331635in}}{\pgfqpoint{9.300000in}{7.700000in}}%
\pgfusepath{clip}%
\pgfsetbuttcap%
\pgfsetroundjoin%
\definecolor{currentfill}{rgb}{0.631373,0.788235,0.956863}%
\pgfsetfillcolor{currentfill}%
\pgfsetlinewidth{0.481800pt}%
\definecolor{currentstroke}{rgb}{1.000000,1.000000,1.000000}%
\pgfsetstrokecolor{currentstroke}%
\pgfsetdash{}{0pt}%
\pgfpathmoveto{\pgfqpoint{4.893800in}{6.368383in}}%
\pgfpathcurveto{\pgfqpoint{4.904850in}{6.368383in}}{\pgfqpoint{4.915449in}{6.372773in}}{\pgfqpoint{4.923262in}{6.380587in}}%
\pgfpathcurveto{\pgfqpoint{4.931076in}{6.388400in}}{\pgfqpoint{4.935466in}{6.398999in}}{\pgfqpoint{4.935466in}{6.410049in}}%
\pgfpathcurveto{\pgfqpoint{4.935466in}{6.421100in}}{\pgfqpoint{4.931076in}{6.431699in}}{\pgfqpoint{4.923262in}{6.439512in}}%
\pgfpathcurveto{\pgfqpoint{4.915449in}{6.447326in}}{\pgfqpoint{4.904850in}{6.451716in}}{\pgfqpoint{4.893800in}{6.451716in}}%
\pgfpathcurveto{\pgfqpoint{4.882749in}{6.451716in}}{\pgfqpoint{4.872150in}{6.447326in}}{\pgfqpoint{4.864337in}{6.439512in}}%
\pgfpathcurveto{\pgfqpoint{4.856523in}{6.431699in}}{\pgfqpoint{4.852133in}{6.421100in}}{\pgfqpoint{4.852133in}{6.410049in}}%
\pgfpathcurveto{\pgfqpoint{4.852133in}{6.398999in}}{\pgfqpoint{4.856523in}{6.388400in}}{\pgfqpoint{4.864337in}{6.380587in}}%
\pgfpathcurveto{\pgfqpoint{4.872150in}{6.372773in}}{\pgfqpoint{4.882749in}{6.368383in}}{\pgfqpoint{4.893800in}{6.368383in}}%
\pgfpathclose%
\pgfusepath{stroke,fill}%
\end{pgfscope}%
\begin{pgfscope}%
\pgfpathrectangle{\pgfqpoint{0.481978in}{0.331635in}}{\pgfqpoint{9.300000in}{7.700000in}}%
\pgfusepath{clip}%
\pgfsetbuttcap%
\pgfsetroundjoin%
\definecolor{currentfill}{rgb}{0.631373,0.788235,0.956863}%
\pgfsetfillcolor{currentfill}%
\pgfsetlinewidth{0.481800pt}%
\definecolor{currentstroke}{rgb}{1.000000,1.000000,1.000000}%
\pgfsetstrokecolor{currentstroke}%
\pgfsetdash{}{0pt}%
\pgfpathmoveto{\pgfqpoint{3.428502in}{1.813110in}}%
\pgfpathcurveto{\pgfqpoint{3.439552in}{1.813110in}}{\pgfqpoint{3.450151in}{1.817500in}}{\pgfqpoint{3.457965in}{1.825314in}}%
\pgfpathcurveto{\pgfqpoint{3.465779in}{1.833128in}}{\pgfqpoint{3.470169in}{1.843727in}}{\pgfqpoint{3.470169in}{1.854777in}}%
\pgfpathcurveto{\pgfqpoint{3.470169in}{1.865827in}}{\pgfqpoint{3.465779in}{1.876426in}}{\pgfqpoint{3.457965in}{1.884240in}}%
\pgfpathcurveto{\pgfqpoint{3.450151in}{1.892053in}}{\pgfqpoint{3.439552in}{1.896443in}}{\pgfqpoint{3.428502in}{1.896443in}}%
\pgfpathcurveto{\pgfqpoint{3.417452in}{1.896443in}}{\pgfqpoint{3.406853in}{1.892053in}}{\pgfqpoint{3.399039in}{1.884240in}}%
\pgfpathcurveto{\pgfqpoint{3.391226in}{1.876426in}}{\pgfqpoint{3.386836in}{1.865827in}}{\pgfqpoint{3.386836in}{1.854777in}}%
\pgfpathcurveto{\pgfqpoint{3.386836in}{1.843727in}}{\pgfqpoint{3.391226in}{1.833128in}}{\pgfqpoint{3.399039in}{1.825314in}}%
\pgfpathcurveto{\pgfqpoint{3.406853in}{1.817500in}}{\pgfqpoint{3.417452in}{1.813110in}}{\pgfqpoint{3.428502in}{1.813110in}}%
\pgfpathclose%
\pgfusepath{stroke,fill}%
\end{pgfscope}%
\begin{pgfscope}%
\pgfpathrectangle{\pgfqpoint{0.481978in}{0.331635in}}{\pgfqpoint{9.300000in}{7.700000in}}%
\pgfusepath{clip}%
\pgfsetbuttcap%
\pgfsetroundjoin%
\definecolor{currentfill}{rgb}{0.631373,0.788235,0.956863}%
\pgfsetfillcolor{currentfill}%
\pgfsetlinewidth{0.481800pt}%
\definecolor{currentstroke}{rgb}{1.000000,1.000000,1.000000}%
\pgfsetstrokecolor{currentstroke}%
\pgfsetdash{}{0pt}%
\pgfpathmoveto{\pgfqpoint{3.370561in}{7.113677in}}%
\pgfpathcurveto{\pgfqpoint{3.381612in}{7.113677in}}{\pgfqpoint{3.392211in}{7.118067in}}{\pgfqpoint{3.400024in}{7.125881in}}%
\pgfpathcurveto{\pgfqpoint{3.407838in}{7.133695in}}{\pgfqpoint{3.412228in}{7.144294in}}{\pgfqpoint{3.412228in}{7.155344in}}%
\pgfpathcurveto{\pgfqpoint{3.412228in}{7.166394in}}{\pgfqpoint{3.407838in}{7.176993in}}{\pgfqpoint{3.400024in}{7.184807in}}%
\pgfpathcurveto{\pgfqpoint{3.392211in}{7.192620in}}{\pgfqpoint{3.381612in}{7.197011in}}{\pgfqpoint{3.370561in}{7.197011in}}%
\pgfpathcurveto{\pgfqpoint{3.359511in}{7.197011in}}{\pgfqpoint{3.348912in}{7.192620in}}{\pgfqpoint{3.341099in}{7.184807in}}%
\pgfpathcurveto{\pgfqpoint{3.333285in}{7.176993in}}{\pgfqpoint{3.328895in}{7.166394in}}{\pgfqpoint{3.328895in}{7.155344in}}%
\pgfpathcurveto{\pgfqpoint{3.328895in}{7.144294in}}{\pgfqpoint{3.333285in}{7.133695in}}{\pgfqpoint{3.341099in}{7.125881in}}%
\pgfpathcurveto{\pgfqpoint{3.348912in}{7.118067in}}{\pgfqpoint{3.359511in}{7.113677in}}{\pgfqpoint{3.370561in}{7.113677in}}%
\pgfpathclose%
\pgfusepath{stroke,fill}%
\end{pgfscope}%
\begin{pgfscope}%
\pgfpathrectangle{\pgfqpoint{0.481978in}{0.331635in}}{\pgfqpoint{9.300000in}{7.700000in}}%
\pgfusepath{clip}%
\pgfsetbuttcap%
\pgfsetroundjoin%
\definecolor{currentfill}{rgb}{0.631373,0.788235,0.956863}%
\pgfsetfillcolor{currentfill}%
\pgfsetlinewidth{0.481800pt}%
\definecolor{currentstroke}{rgb}{1.000000,1.000000,1.000000}%
\pgfsetstrokecolor{currentstroke}%
\pgfsetdash{}{0pt}%
\pgfpathmoveto{\pgfqpoint{6.915286in}{4.156789in}}%
\pgfpathcurveto{\pgfqpoint{6.926336in}{4.156789in}}{\pgfqpoint{6.936935in}{4.161179in}}{\pgfqpoint{6.944748in}{4.168993in}}%
\pgfpathcurveto{\pgfqpoint{6.952562in}{4.176806in}}{\pgfqpoint{6.956952in}{4.187405in}}{\pgfqpoint{6.956952in}{4.198455in}}%
\pgfpathcurveto{\pgfqpoint{6.956952in}{4.209506in}}{\pgfqpoint{6.952562in}{4.220105in}}{\pgfqpoint{6.944748in}{4.227918in}}%
\pgfpathcurveto{\pgfqpoint{6.936935in}{4.235732in}}{\pgfqpoint{6.926336in}{4.240122in}}{\pgfqpoint{6.915286in}{4.240122in}}%
\pgfpathcurveto{\pgfqpoint{6.904235in}{4.240122in}}{\pgfqpoint{6.893636in}{4.235732in}}{\pgfqpoint{6.885823in}{4.227918in}}%
\pgfpathcurveto{\pgfqpoint{6.878009in}{4.220105in}}{\pgfqpoint{6.873619in}{4.209506in}}{\pgfqpoint{6.873619in}{4.198455in}}%
\pgfpathcurveto{\pgfqpoint{6.873619in}{4.187405in}}{\pgfqpoint{6.878009in}{4.176806in}}{\pgfqpoint{6.885823in}{4.168993in}}%
\pgfpathcurveto{\pgfqpoint{6.893636in}{4.161179in}}{\pgfqpoint{6.904235in}{4.156789in}}{\pgfqpoint{6.915286in}{4.156789in}}%
\pgfpathclose%
\pgfusepath{stroke,fill}%
\end{pgfscope}%
\begin{pgfscope}%
\pgfpathrectangle{\pgfqpoint{0.481978in}{0.331635in}}{\pgfqpoint{9.300000in}{7.700000in}}%
\pgfusepath{clip}%
\pgfsetbuttcap%
\pgfsetroundjoin%
\definecolor{currentfill}{rgb}{0.631373,0.788235,0.956863}%
\pgfsetfillcolor{currentfill}%
\pgfsetlinewidth{0.481800pt}%
\definecolor{currentstroke}{rgb}{1.000000,1.000000,1.000000}%
\pgfsetstrokecolor{currentstroke}%
\pgfsetdash{}{0pt}%
\pgfpathmoveto{\pgfqpoint{3.604723in}{5.484870in}}%
\pgfpathcurveto{\pgfqpoint{3.615773in}{5.484870in}}{\pgfqpoint{3.626372in}{5.489260in}}{\pgfqpoint{3.634185in}{5.497074in}}%
\pgfpathcurveto{\pgfqpoint{3.641999in}{5.504888in}}{\pgfqpoint{3.646389in}{5.515487in}}{\pgfqpoint{3.646389in}{5.526537in}}%
\pgfpathcurveto{\pgfqpoint{3.646389in}{5.537587in}}{\pgfqpoint{3.641999in}{5.548186in}}{\pgfqpoint{3.634185in}{5.556000in}}%
\pgfpathcurveto{\pgfqpoint{3.626372in}{5.563813in}}{\pgfqpoint{3.615773in}{5.568204in}}{\pgfqpoint{3.604723in}{5.568204in}}%
\pgfpathcurveto{\pgfqpoint{3.593672in}{5.568204in}}{\pgfqpoint{3.583073in}{5.563813in}}{\pgfqpoint{3.575260in}{5.556000in}}%
\pgfpathcurveto{\pgfqpoint{3.567446in}{5.548186in}}{\pgfqpoint{3.563056in}{5.537587in}}{\pgfqpoint{3.563056in}{5.526537in}}%
\pgfpathcurveto{\pgfqpoint{3.563056in}{5.515487in}}{\pgfqpoint{3.567446in}{5.504888in}}{\pgfqpoint{3.575260in}{5.497074in}}%
\pgfpathcurveto{\pgfqpoint{3.583073in}{5.489260in}}{\pgfqpoint{3.593672in}{5.484870in}}{\pgfqpoint{3.604723in}{5.484870in}}%
\pgfpathclose%
\pgfusepath{stroke,fill}%
\end{pgfscope}%
\begin{pgfscope}%
\pgfpathrectangle{\pgfqpoint{0.481978in}{0.331635in}}{\pgfqpoint{9.300000in}{7.700000in}}%
\pgfusepath{clip}%
\pgfsetbuttcap%
\pgfsetroundjoin%
\definecolor{currentfill}{rgb}{0.631373,0.788235,0.956863}%
\pgfsetfillcolor{currentfill}%
\pgfsetlinewidth{0.481800pt}%
\definecolor{currentstroke}{rgb}{1.000000,1.000000,1.000000}%
\pgfsetstrokecolor{currentstroke}%
\pgfsetdash{}{0pt}%
\pgfpathmoveto{\pgfqpoint{4.283816in}{3.497421in}}%
\pgfpathcurveto{\pgfqpoint{4.294866in}{3.497421in}}{\pgfqpoint{4.305465in}{3.501811in}}{\pgfqpoint{4.313279in}{3.509624in}}%
\pgfpathcurveto{\pgfqpoint{4.321093in}{3.517438in}}{\pgfqpoint{4.325483in}{3.528037in}}{\pgfqpoint{4.325483in}{3.539087in}}%
\pgfpathcurveto{\pgfqpoint{4.325483in}{3.550137in}}{\pgfqpoint{4.321093in}{3.560736in}}{\pgfqpoint{4.313279in}{3.568550in}}%
\pgfpathcurveto{\pgfqpoint{4.305465in}{3.576364in}}{\pgfqpoint{4.294866in}{3.580754in}}{\pgfqpoint{4.283816in}{3.580754in}}%
\pgfpathcurveto{\pgfqpoint{4.272766in}{3.580754in}}{\pgfqpoint{4.262167in}{3.576364in}}{\pgfqpoint{4.254353in}{3.568550in}}%
\pgfpathcurveto{\pgfqpoint{4.246540in}{3.560736in}}{\pgfqpoint{4.242150in}{3.550137in}}{\pgfqpoint{4.242150in}{3.539087in}}%
\pgfpathcurveto{\pgfqpoint{4.242150in}{3.528037in}}{\pgfqpoint{4.246540in}{3.517438in}}{\pgfqpoint{4.254353in}{3.509624in}}%
\pgfpathcurveto{\pgfqpoint{4.262167in}{3.501811in}}{\pgfqpoint{4.272766in}{3.497421in}}{\pgfqpoint{4.283816in}{3.497421in}}%
\pgfpathclose%
\pgfusepath{stroke,fill}%
\end{pgfscope}%
\begin{pgfscope}%
\pgfpathrectangle{\pgfqpoint{0.481978in}{0.331635in}}{\pgfqpoint{9.300000in}{7.700000in}}%
\pgfusepath{clip}%
\pgfsetbuttcap%
\pgfsetroundjoin%
\definecolor{currentfill}{rgb}{0.631373,0.788235,0.956863}%
\pgfsetfillcolor{currentfill}%
\pgfsetlinewidth{0.481800pt}%
\definecolor{currentstroke}{rgb}{1.000000,1.000000,1.000000}%
\pgfsetstrokecolor{currentstroke}%
\pgfsetdash{}{0pt}%
\pgfpathmoveto{\pgfqpoint{7.064950in}{1.902700in}}%
\pgfpathcurveto{\pgfqpoint{7.076000in}{1.902700in}}{\pgfqpoint{7.086599in}{1.907091in}}{\pgfqpoint{7.094413in}{1.914904in}}%
\pgfpathcurveto{\pgfqpoint{7.102226in}{1.922718in}}{\pgfqpoint{7.106617in}{1.933317in}}{\pgfqpoint{7.106617in}{1.944367in}}%
\pgfpathcurveto{\pgfqpoint{7.106617in}{1.955417in}}{\pgfqpoint{7.102226in}{1.966016in}}{\pgfqpoint{7.094413in}{1.973830in}}%
\pgfpathcurveto{\pgfqpoint{7.086599in}{1.981643in}}{\pgfqpoint{7.076000in}{1.986034in}}{\pgfqpoint{7.064950in}{1.986034in}}%
\pgfpathcurveto{\pgfqpoint{7.053900in}{1.986034in}}{\pgfqpoint{7.043301in}{1.981643in}}{\pgfqpoint{7.035487in}{1.973830in}}%
\pgfpathcurveto{\pgfqpoint{7.027673in}{1.966016in}}{\pgfqpoint{7.023283in}{1.955417in}}{\pgfqpoint{7.023283in}{1.944367in}}%
\pgfpathcurveto{\pgfqpoint{7.023283in}{1.933317in}}{\pgfqpoint{7.027673in}{1.922718in}}{\pgfqpoint{7.035487in}{1.914904in}}%
\pgfpathcurveto{\pgfqpoint{7.043301in}{1.907091in}}{\pgfqpoint{7.053900in}{1.902700in}}{\pgfqpoint{7.064950in}{1.902700in}}%
\pgfpathclose%
\pgfusepath{stroke,fill}%
\end{pgfscope}%
\begin{pgfscope}%
\pgfpathrectangle{\pgfqpoint{0.481978in}{0.331635in}}{\pgfqpoint{9.300000in}{7.700000in}}%
\pgfusepath{clip}%
\pgfsetbuttcap%
\pgfsetroundjoin%
\definecolor{currentfill}{rgb}{0.631373,0.788235,0.956863}%
\pgfsetfillcolor{currentfill}%
\pgfsetlinewidth{0.481800pt}%
\definecolor{currentstroke}{rgb}{1.000000,1.000000,1.000000}%
\pgfsetstrokecolor{currentstroke}%
\pgfsetdash{}{0pt}%
\pgfpathmoveto{\pgfqpoint{3.497950in}{6.601752in}}%
\pgfpathcurveto{\pgfqpoint{3.509000in}{6.601752in}}{\pgfqpoint{3.519599in}{6.606142in}}{\pgfqpoint{3.527412in}{6.613956in}}%
\pgfpathcurveto{\pgfqpoint{3.535226in}{6.621770in}}{\pgfqpoint{3.539616in}{6.632369in}}{\pgfqpoint{3.539616in}{6.643419in}}%
\pgfpathcurveto{\pgfqpoint{3.539616in}{6.654469in}}{\pgfqpoint{3.535226in}{6.665068in}}{\pgfqpoint{3.527412in}{6.672881in}}%
\pgfpathcurveto{\pgfqpoint{3.519599in}{6.680695in}}{\pgfqpoint{3.509000in}{6.685085in}}{\pgfqpoint{3.497950in}{6.685085in}}%
\pgfpathcurveto{\pgfqpoint{3.486900in}{6.685085in}}{\pgfqpoint{3.476301in}{6.680695in}}{\pgfqpoint{3.468487in}{6.672881in}}%
\pgfpathcurveto{\pgfqpoint{3.460673in}{6.665068in}}{\pgfqpoint{3.456283in}{6.654469in}}{\pgfqpoint{3.456283in}{6.643419in}}%
\pgfpathcurveto{\pgfqpoint{3.456283in}{6.632369in}}{\pgfqpoint{3.460673in}{6.621770in}}{\pgfqpoint{3.468487in}{6.613956in}}%
\pgfpathcurveto{\pgfqpoint{3.476301in}{6.606142in}}{\pgfqpoint{3.486900in}{6.601752in}}{\pgfqpoint{3.497950in}{6.601752in}}%
\pgfpathclose%
\pgfusepath{stroke,fill}%
\end{pgfscope}%
\begin{pgfscope}%
\pgfpathrectangle{\pgfqpoint{0.481978in}{0.331635in}}{\pgfqpoint{9.300000in}{7.700000in}}%
\pgfusepath{clip}%
\pgfsetbuttcap%
\pgfsetroundjoin%
\definecolor{currentfill}{rgb}{0.631373,0.788235,0.956863}%
\pgfsetfillcolor{currentfill}%
\pgfsetlinewidth{0.481800pt}%
\definecolor{currentstroke}{rgb}{1.000000,1.000000,1.000000}%
\pgfsetstrokecolor{currentstroke}%
\pgfsetdash{}{0pt}%
\pgfpathmoveto{\pgfqpoint{5.579377in}{3.166101in}}%
\pgfpathcurveto{\pgfqpoint{5.590427in}{3.166101in}}{\pgfqpoint{5.601026in}{3.170491in}}{\pgfqpoint{5.608840in}{3.178304in}}%
\pgfpathcurveto{\pgfqpoint{5.616653in}{3.186118in}}{\pgfqpoint{5.621043in}{3.196717in}}{\pgfqpoint{5.621043in}{3.207767in}}%
\pgfpathcurveto{\pgfqpoint{5.621043in}{3.218817in}}{\pgfqpoint{5.616653in}{3.229416in}}{\pgfqpoint{5.608840in}{3.237230in}}%
\pgfpathcurveto{\pgfqpoint{5.601026in}{3.245044in}}{\pgfqpoint{5.590427in}{3.249434in}}{\pgfqpoint{5.579377in}{3.249434in}}%
\pgfpathcurveto{\pgfqpoint{5.568327in}{3.249434in}}{\pgfqpoint{5.557728in}{3.245044in}}{\pgfqpoint{5.549914in}{3.237230in}}%
\pgfpathcurveto{\pgfqpoint{5.542100in}{3.229416in}}{\pgfqpoint{5.537710in}{3.218817in}}{\pgfqpoint{5.537710in}{3.207767in}}%
\pgfpathcurveto{\pgfqpoint{5.537710in}{3.196717in}}{\pgfqpoint{5.542100in}{3.186118in}}{\pgfqpoint{5.549914in}{3.178304in}}%
\pgfpathcurveto{\pgfqpoint{5.557728in}{3.170491in}}{\pgfqpoint{5.568327in}{3.166101in}}{\pgfqpoint{5.579377in}{3.166101in}}%
\pgfpathclose%
\pgfusepath{stroke,fill}%
\end{pgfscope}%
\begin{pgfscope}%
\pgfpathrectangle{\pgfqpoint{0.481978in}{0.331635in}}{\pgfqpoint{9.300000in}{7.700000in}}%
\pgfusepath{clip}%
\pgfsetbuttcap%
\pgfsetroundjoin%
\definecolor{currentfill}{rgb}{0.631373,0.788235,0.956863}%
\pgfsetfillcolor{currentfill}%
\pgfsetlinewidth{0.481800pt}%
\definecolor{currentstroke}{rgb}{1.000000,1.000000,1.000000}%
\pgfsetstrokecolor{currentstroke}%
\pgfsetdash{}{0pt}%
\pgfpathmoveto{\pgfqpoint{8.844215in}{5.030878in}}%
\pgfpathcurveto{\pgfqpoint{8.855265in}{5.030878in}}{\pgfqpoint{8.865864in}{5.035269in}}{\pgfqpoint{8.873678in}{5.043082in}}%
\pgfpathcurveto{\pgfqpoint{8.881491in}{5.050896in}}{\pgfqpoint{8.885882in}{5.061495in}}{\pgfqpoint{8.885882in}{5.072545in}}%
\pgfpathcurveto{\pgfqpoint{8.885882in}{5.083595in}}{\pgfqpoint{8.881491in}{5.094194in}}{\pgfqpoint{8.873678in}{5.102008in}}%
\pgfpathcurveto{\pgfqpoint{8.865864in}{5.109821in}}{\pgfqpoint{8.855265in}{5.114212in}}{\pgfqpoint{8.844215in}{5.114212in}}%
\pgfpathcurveto{\pgfqpoint{8.833165in}{5.114212in}}{\pgfqpoint{8.822566in}{5.109821in}}{\pgfqpoint{8.814752in}{5.102008in}}%
\pgfpathcurveto{\pgfqpoint{8.806939in}{5.094194in}}{\pgfqpoint{8.802548in}{5.083595in}}{\pgfqpoint{8.802548in}{5.072545in}}%
\pgfpathcurveto{\pgfqpoint{8.802548in}{5.061495in}}{\pgfqpoint{8.806939in}{5.050896in}}{\pgfqpoint{8.814752in}{5.043082in}}%
\pgfpathcurveto{\pgfqpoint{8.822566in}{5.035269in}}{\pgfqpoint{8.833165in}{5.030878in}}{\pgfqpoint{8.844215in}{5.030878in}}%
\pgfpathclose%
\pgfusepath{stroke,fill}%
\end{pgfscope}%
\begin{pgfscope}%
\pgfpathrectangle{\pgfqpoint{0.481978in}{0.331635in}}{\pgfqpoint{9.300000in}{7.700000in}}%
\pgfusepath{clip}%
\pgfsetbuttcap%
\pgfsetroundjoin%
\definecolor{currentfill}{rgb}{0.631373,0.788235,0.956863}%
\pgfsetfillcolor{currentfill}%
\pgfsetlinewidth{0.481800pt}%
\definecolor{currentstroke}{rgb}{1.000000,1.000000,1.000000}%
\pgfsetstrokecolor{currentstroke}%
\pgfsetdash{}{0pt}%
\pgfpathmoveto{\pgfqpoint{4.858261in}{1.811183in}}%
\pgfpathcurveto{\pgfqpoint{4.869311in}{1.811183in}}{\pgfqpoint{4.879910in}{1.815573in}}{\pgfqpoint{4.887723in}{1.823387in}}%
\pgfpathcurveto{\pgfqpoint{4.895537in}{1.831201in}}{\pgfqpoint{4.899927in}{1.841800in}}{\pgfqpoint{4.899927in}{1.852850in}}%
\pgfpathcurveto{\pgfqpoint{4.899927in}{1.863900in}}{\pgfqpoint{4.895537in}{1.874499in}}{\pgfqpoint{4.887723in}{1.882312in}}%
\pgfpathcurveto{\pgfqpoint{4.879910in}{1.890126in}}{\pgfqpoint{4.869311in}{1.894516in}}{\pgfqpoint{4.858261in}{1.894516in}}%
\pgfpathcurveto{\pgfqpoint{4.847210in}{1.894516in}}{\pgfqpoint{4.836611in}{1.890126in}}{\pgfqpoint{4.828798in}{1.882312in}}%
\pgfpathcurveto{\pgfqpoint{4.820984in}{1.874499in}}{\pgfqpoint{4.816594in}{1.863900in}}{\pgfqpoint{4.816594in}{1.852850in}}%
\pgfpathcurveto{\pgfqpoint{4.816594in}{1.841800in}}{\pgfqpoint{4.820984in}{1.831201in}}{\pgfqpoint{4.828798in}{1.823387in}}%
\pgfpathcurveto{\pgfqpoint{4.836611in}{1.815573in}}{\pgfqpoint{4.847210in}{1.811183in}}{\pgfqpoint{4.858261in}{1.811183in}}%
\pgfpathclose%
\pgfusepath{stroke,fill}%
\end{pgfscope}%
\begin{pgfscope}%
\pgfpathrectangle{\pgfqpoint{0.481978in}{0.331635in}}{\pgfqpoint{9.300000in}{7.700000in}}%
\pgfusepath{clip}%
\pgfsetbuttcap%
\pgfsetroundjoin%
\definecolor{currentfill}{rgb}{0.631373,0.788235,0.956863}%
\pgfsetfillcolor{currentfill}%
\pgfsetlinewidth{0.481800pt}%
\definecolor{currentstroke}{rgb}{1.000000,1.000000,1.000000}%
\pgfsetstrokecolor{currentstroke}%
\pgfsetdash{}{0pt}%
\pgfpathmoveto{\pgfqpoint{7.438148in}{1.928502in}}%
\pgfpathcurveto{\pgfqpoint{7.449198in}{1.928502in}}{\pgfqpoint{7.459797in}{1.932892in}}{\pgfqpoint{7.467611in}{1.940705in}}%
\pgfpathcurveto{\pgfqpoint{7.475425in}{1.948519in}}{\pgfqpoint{7.479815in}{1.959118in}}{\pgfqpoint{7.479815in}{1.970168in}}%
\pgfpathcurveto{\pgfqpoint{7.479815in}{1.981218in}}{\pgfqpoint{7.475425in}{1.991817in}}{\pgfqpoint{7.467611in}{1.999631in}}%
\pgfpathcurveto{\pgfqpoint{7.459797in}{2.007445in}}{\pgfqpoint{7.449198in}{2.011835in}}{\pgfqpoint{7.438148in}{2.011835in}}%
\pgfpathcurveto{\pgfqpoint{7.427098in}{2.011835in}}{\pgfqpoint{7.416499in}{2.007445in}}{\pgfqpoint{7.408685in}{1.999631in}}%
\pgfpathcurveto{\pgfqpoint{7.400872in}{1.991817in}}{\pgfqpoint{7.396481in}{1.981218in}}{\pgfqpoint{7.396481in}{1.970168in}}%
\pgfpathcurveto{\pgfqpoint{7.396481in}{1.959118in}}{\pgfqpoint{7.400872in}{1.948519in}}{\pgfqpoint{7.408685in}{1.940705in}}%
\pgfpathcurveto{\pgfqpoint{7.416499in}{1.932892in}}{\pgfqpoint{7.427098in}{1.928502in}}{\pgfqpoint{7.438148in}{1.928502in}}%
\pgfpathclose%
\pgfusepath{stroke,fill}%
\end{pgfscope}%
\begin{pgfscope}%
\pgfpathrectangle{\pgfqpoint{0.481978in}{0.331635in}}{\pgfqpoint{9.300000in}{7.700000in}}%
\pgfusepath{clip}%
\pgfsetbuttcap%
\pgfsetroundjoin%
\definecolor{currentfill}{rgb}{0.631373,0.788235,0.956863}%
\pgfsetfillcolor{currentfill}%
\pgfsetlinewidth{0.481800pt}%
\definecolor{currentstroke}{rgb}{1.000000,1.000000,1.000000}%
\pgfsetstrokecolor{currentstroke}%
\pgfsetdash{}{0pt}%
\pgfpathmoveto{\pgfqpoint{3.422792in}{0.998594in}}%
\pgfpathcurveto{\pgfqpoint{3.433842in}{0.998594in}}{\pgfqpoint{3.444441in}{1.002985in}}{\pgfqpoint{3.452255in}{1.010798in}}%
\pgfpathcurveto{\pgfqpoint{3.460069in}{1.018612in}}{\pgfqpoint{3.464459in}{1.029211in}}{\pgfqpoint{3.464459in}{1.040261in}}%
\pgfpathcurveto{\pgfqpoint{3.464459in}{1.051311in}}{\pgfqpoint{3.460069in}{1.061910in}}{\pgfqpoint{3.452255in}{1.069724in}}%
\pgfpathcurveto{\pgfqpoint{3.444441in}{1.077537in}}{\pgfqpoint{3.433842in}{1.081928in}}{\pgfqpoint{3.422792in}{1.081928in}}%
\pgfpathcurveto{\pgfqpoint{3.411742in}{1.081928in}}{\pgfqpoint{3.401143in}{1.077537in}}{\pgfqpoint{3.393329in}{1.069724in}}%
\pgfpathcurveto{\pgfqpoint{3.385516in}{1.061910in}}{\pgfqpoint{3.381126in}{1.051311in}}{\pgfqpoint{3.381126in}{1.040261in}}%
\pgfpathcurveto{\pgfqpoint{3.381126in}{1.029211in}}{\pgfqpoint{3.385516in}{1.018612in}}{\pgfqpoint{3.393329in}{1.010798in}}%
\pgfpathcurveto{\pgfqpoint{3.401143in}{1.002985in}}{\pgfqpoint{3.411742in}{0.998594in}}{\pgfqpoint{3.422792in}{0.998594in}}%
\pgfpathclose%
\pgfusepath{stroke,fill}%
\end{pgfscope}%
\begin{pgfscope}%
\pgfpathrectangle{\pgfqpoint{0.481978in}{0.331635in}}{\pgfqpoint{9.300000in}{7.700000in}}%
\pgfusepath{clip}%
\pgfsetbuttcap%
\pgfsetroundjoin%
\definecolor{currentfill}{rgb}{0.631373,0.788235,0.956863}%
\pgfsetfillcolor{currentfill}%
\pgfsetlinewidth{0.481800pt}%
\definecolor{currentstroke}{rgb}{1.000000,1.000000,1.000000}%
\pgfsetstrokecolor{currentstroke}%
\pgfsetdash{}{0pt}%
\pgfpathmoveto{\pgfqpoint{2.139723in}{2.443266in}}%
\pgfpathcurveto{\pgfqpoint{2.150773in}{2.443266in}}{\pgfqpoint{2.161372in}{2.447656in}}{\pgfqpoint{2.169186in}{2.455469in}}%
\pgfpathcurveto{\pgfqpoint{2.176999in}{2.463283in}}{\pgfqpoint{2.181390in}{2.473882in}}{\pgfqpoint{2.181390in}{2.484932in}}%
\pgfpathcurveto{\pgfqpoint{2.181390in}{2.495982in}}{\pgfqpoint{2.176999in}{2.506581in}}{\pgfqpoint{2.169186in}{2.514395in}}%
\pgfpathcurveto{\pgfqpoint{2.161372in}{2.522209in}}{\pgfqpoint{2.150773in}{2.526599in}}{\pgfqpoint{2.139723in}{2.526599in}}%
\pgfpathcurveto{\pgfqpoint{2.128673in}{2.526599in}}{\pgfqpoint{2.118074in}{2.522209in}}{\pgfqpoint{2.110260in}{2.514395in}}%
\pgfpathcurveto{\pgfqpoint{2.102447in}{2.506581in}}{\pgfqpoint{2.098056in}{2.495982in}}{\pgfqpoint{2.098056in}{2.484932in}}%
\pgfpathcurveto{\pgfqpoint{2.098056in}{2.473882in}}{\pgfqpoint{2.102447in}{2.463283in}}{\pgfqpoint{2.110260in}{2.455469in}}%
\pgfpathcurveto{\pgfqpoint{2.118074in}{2.447656in}}{\pgfqpoint{2.128673in}{2.443266in}}{\pgfqpoint{2.139723in}{2.443266in}}%
\pgfpathclose%
\pgfusepath{stroke,fill}%
\end{pgfscope}%
\begin{pgfscope}%
\pgfpathrectangle{\pgfqpoint{0.481978in}{0.331635in}}{\pgfqpoint{9.300000in}{7.700000in}}%
\pgfusepath{clip}%
\pgfsetbuttcap%
\pgfsetroundjoin%
\definecolor{currentfill}{rgb}{0.631373,0.788235,0.956863}%
\pgfsetfillcolor{currentfill}%
\pgfsetlinewidth{0.481800pt}%
\definecolor{currentstroke}{rgb}{1.000000,1.000000,1.000000}%
\pgfsetstrokecolor{currentstroke}%
\pgfsetdash{}{0pt}%
\pgfpathmoveto{\pgfqpoint{8.049554in}{5.424436in}}%
\pgfpathcurveto{\pgfqpoint{8.060604in}{5.424436in}}{\pgfqpoint{8.071203in}{5.428826in}}{\pgfqpoint{8.079017in}{5.436640in}}%
\pgfpathcurveto{\pgfqpoint{8.086831in}{5.444453in}}{\pgfqpoint{8.091221in}{5.455052in}}{\pgfqpoint{8.091221in}{5.466103in}}%
\pgfpathcurveto{\pgfqpoint{8.091221in}{5.477153in}}{\pgfqpoint{8.086831in}{5.487752in}}{\pgfqpoint{8.079017in}{5.495565in}}%
\pgfpathcurveto{\pgfqpoint{8.071203in}{5.503379in}}{\pgfqpoint{8.060604in}{5.507769in}}{\pgfqpoint{8.049554in}{5.507769in}}%
\pgfpathcurveto{\pgfqpoint{8.038504in}{5.507769in}}{\pgfqpoint{8.027905in}{5.503379in}}{\pgfqpoint{8.020091in}{5.495565in}}%
\pgfpathcurveto{\pgfqpoint{8.012278in}{5.487752in}}{\pgfqpoint{8.007887in}{5.477153in}}{\pgfqpoint{8.007887in}{5.466103in}}%
\pgfpathcurveto{\pgfqpoint{8.007887in}{5.455052in}}{\pgfqpoint{8.012278in}{5.444453in}}{\pgfqpoint{8.020091in}{5.436640in}}%
\pgfpathcurveto{\pgfqpoint{8.027905in}{5.428826in}}{\pgfqpoint{8.038504in}{5.424436in}}{\pgfqpoint{8.049554in}{5.424436in}}%
\pgfpathclose%
\pgfusepath{stroke,fill}%
\end{pgfscope}%
\begin{pgfscope}%
\pgfpathrectangle{\pgfqpoint{0.481978in}{0.331635in}}{\pgfqpoint{9.300000in}{7.700000in}}%
\pgfusepath{clip}%
\pgfsetbuttcap%
\pgfsetroundjoin%
\definecolor{currentfill}{rgb}{0.631373,0.788235,0.956863}%
\pgfsetfillcolor{currentfill}%
\pgfsetlinewidth{0.481800pt}%
\definecolor{currentstroke}{rgb}{1.000000,1.000000,1.000000}%
\pgfsetstrokecolor{currentstroke}%
\pgfsetdash{}{0pt}%
\pgfpathmoveto{\pgfqpoint{3.389691in}{1.585792in}}%
\pgfpathcurveto{\pgfqpoint{3.400741in}{1.585792in}}{\pgfqpoint{3.411340in}{1.590182in}}{\pgfqpoint{3.419153in}{1.597995in}}%
\pgfpathcurveto{\pgfqpoint{3.426967in}{1.605809in}}{\pgfqpoint{3.431357in}{1.616408in}}{\pgfqpoint{3.431357in}{1.627458in}}%
\pgfpathcurveto{\pgfqpoint{3.431357in}{1.638508in}}{\pgfqpoint{3.426967in}{1.649107in}}{\pgfqpoint{3.419153in}{1.656921in}}%
\pgfpathcurveto{\pgfqpoint{3.411340in}{1.664735in}}{\pgfqpoint{3.400741in}{1.669125in}}{\pgfqpoint{3.389691in}{1.669125in}}%
\pgfpathcurveto{\pgfqpoint{3.378640in}{1.669125in}}{\pgfqpoint{3.368041in}{1.664735in}}{\pgfqpoint{3.360228in}{1.656921in}}%
\pgfpathcurveto{\pgfqpoint{3.352414in}{1.649107in}}{\pgfqpoint{3.348024in}{1.638508in}}{\pgfqpoint{3.348024in}{1.627458in}}%
\pgfpathcurveto{\pgfqpoint{3.348024in}{1.616408in}}{\pgfqpoint{3.352414in}{1.605809in}}{\pgfqpoint{3.360228in}{1.597995in}}%
\pgfpathcurveto{\pgfqpoint{3.368041in}{1.590182in}}{\pgfqpoint{3.378640in}{1.585792in}}{\pgfqpoint{3.389691in}{1.585792in}}%
\pgfpathclose%
\pgfusepath{stroke,fill}%
\end{pgfscope}%
\begin{pgfscope}%
\pgfpathrectangle{\pgfqpoint{0.481978in}{0.331635in}}{\pgfqpoint{9.300000in}{7.700000in}}%
\pgfusepath{clip}%
\pgfsetbuttcap%
\pgfsetroundjoin%
\definecolor{currentfill}{rgb}{0.631373,0.788235,0.956863}%
\pgfsetfillcolor{currentfill}%
\pgfsetlinewidth{0.481800pt}%
\definecolor{currentstroke}{rgb}{1.000000,1.000000,1.000000}%
\pgfsetstrokecolor{currentstroke}%
\pgfsetdash{}{0pt}%
\pgfpathmoveto{\pgfqpoint{6.131345in}{5.208931in}}%
\pgfpathcurveto{\pgfqpoint{6.142395in}{5.208931in}}{\pgfqpoint{6.152994in}{5.213321in}}{\pgfqpoint{6.160808in}{5.221135in}}%
\pgfpathcurveto{\pgfqpoint{6.168621in}{5.228948in}}{\pgfqpoint{6.173012in}{5.239547in}}{\pgfqpoint{6.173012in}{5.250597in}}%
\pgfpathcurveto{\pgfqpoint{6.173012in}{5.261648in}}{\pgfqpoint{6.168621in}{5.272247in}}{\pgfqpoint{6.160808in}{5.280060in}}%
\pgfpathcurveto{\pgfqpoint{6.152994in}{5.287874in}}{\pgfqpoint{6.142395in}{5.292264in}}{\pgfqpoint{6.131345in}{5.292264in}}%
\pgfpathcurveto{\pgfqpoint{6.120295in}{5.292264in}}{\pgfqpoint{6.109696in}{5.287874in}}{\pgfqpoint{6.101882in}{5.280060in}}%
\pgfpathcurveto{\pgfqpoint{6.094069in}{5.272247in}}{\pgfqpoint{6.089678in}{5.261648in}}{\pgfqpoint{6.089678in}{5.250597in}}%
\pgfpathcurveto{\pgfqpoint{6.089678in}{5.239547in}}{\pgfqpoint{6.094069in}{5.228948in}}{\pgfqpoint{6.101882in}{5.221135in}}%
\pgfpathcurveto{\pgfqpoint{6.109696in}{5.213321in}}{\pgfqpoint{6.120295in}{5.208931in}}{\pgfqpoint{6.131345in}{5.208931in}}%
\pgfpathclose%
\pgfusepath{stroke,fill}%
\end{pgfscope}%
\begin{pgfscope}%
\pgfpathrectangle{\pgfqpoint{0.481978in}{0.331635in}}{\pgfqpoint{9.300000in}{7.700000in}}%
\pgfusepath{clip}%
\pgfsetbuttcap%
\pgfsetroundjoin%
\definecolor{currentfill}{rgb}{0.631373,0.788235,0.956863}%
\pgfsetfillcolor{currentfill}%
\pgfsetlinewidth{0.481800pt}%
\definecolor{currentstroke}{rgb}{1.000000,1.000000,1.000000}%
\pgfsetstrokecolor{currentstroke}%
\pgfsetdash{}{0pt}%
\pgfpathmoveto{\pgfqpoint{4.856079in}{6.373981in}}%
\pgfpathcurveto{\pgfqpoint{4.867130in}{6.373981in}}{\pgfqpoint{4.877729in}{6.378371in}}{\pgfqpoint{4.885542in}{6.386185in}}%
\pgfpathcurveto{\pgfqpoint{4.893356in}{6.393998in}}{\pgfqpoint{4.897746in}{6.404597in}}{\pgfqpoint{4.897746in}{6.415647in}}%
\pgfpathcurveto{\pgfqpoint{4.897746in}{6.426698in}}{\pgfqpoint{4.893356in}{6.437297in}}{\pgfqpoint{4.885542in}{6.445110in}}%
\pgfpathcurveto{\pgfqpoint{4.877729in}{6.452924in}}{\pgfqpoint{4.867130in}{6.457314in}}{\pgfqpoint{4.856079in}{6.457314in}}%
\pgfpathcurveto{\pgfqpoint{4.845029in}{6.457314in}}{\pgfqpoint{4.834430in}{6.452924in}}{\pgfqpoint{4.826617in}{6.445110in}}%
\pgfpathcurveto{\pgfqpoint{4.818803in}{6.437297in}}{\pgfqpoint{4.814413in}{6.426698in}}{\pgfqpoint{4.814413in}{6.415647in}}%
\pgfpathcurveto{\pgfqpoint{4.814413in}{6.404597in}}{\pgfqpoint{4.818803in}{6.393998in}}{\pgfqpoint{4.826617in}{6.386185in}}%
\pgfpathcurveto{\pgfqpoint{4.834430in}{6.378371in}}{\pgfqpoint{4.845029in}{6.373981in}}{\pgfqpoint{4.856079in}{6.373981in}}%
\pgfpathclose%
\pgfusepath{stroke,fill}%
\end{pgfscope}%
\begin{pgfscope}%
\pgfpathrectangle{\pgfqpoint{0.481978in}{0.331635in}}{\pgfqpoint{9.300000in}{7.700000in}}%
\pgfusepath{clip}%
\pgfsetbuttcap%
\pgfsetroundjoin%
\definecolor{currentfill}{rgb}{0.631373,0.788235,0.956863}%
\pgfsetfillcolor{currentfill}%
\pgfsetlinewidth{0.481800pt}%
\definecolor{currentstroke}{rgb}{1.000000,1.000000,1.000000}%
\pgfsetstrokecolor{currentstroke}%
\pgfsetdash{}{0pt}%
\pgfpathmoveto{\pgfqpoint{4.975752in}{4.578501in}}%
\pgfpathcurveto{\pgfqpoint{4.986802in}{4.578501in}}{\pgfqpoint{4.997401in}{4.582891in}}{\pgfqpoint{5.005215in}{4.590705in}}%
\pgfpathcurveto{\pgfqpoint{5.013028in}{4.598518in}}{\pgfqpoint{5.017418in}{4.609118in}}{\pgfqpoint{5.017418in}{4.620168in}}%
\pgfpathcurveto{\pgfqpoint{5.017418in}{4.631218in}}{\pgfqpoint{5.013028in}{4.641817in}}{\pgfqpoint{5.005215in}{4.649630in}}%
\pgfpathcurveto{\pgfqpoint{4.997401in}{4.657444in}}{\pgfqpoint{4.986802in}{4.661834in}}{\pgfqpoint{4.975752in}{4.661834in}}%
\pgfpathcurveto{\pgfqpoint{4.964702in}{4.661834in}}{\pgfqpoint{4.954103in}{4.657444in}}{\pgfqpoint{4.946289in}{4.649630in}}%
\pgfpathcurveto{\pgfqpoint{4.938475in}{4.641817in}}{\pgfqpoint{4.934085in}{4.631218in}}{\pgfqpoint{4.934085in}{4.620168in}}%
\pgfpathcurveto{\pgfqpoint{4.934085in}{4.609118in}}{\pgfqpoint{4.938475in}{4.598518in}}{\pgfqpoint{4.946289in}{4.590705in}}%
\pgfpathcurveto{\pgfqpoint{4.954103in}{4.582891in}}{\pgfqpoint{4.964702in}{4.578501in}}{\pgfqpoint{4.975752in}{4.578501in}}%
\pgfpathclose%
\pgfusepath{stroke,fill}%
\end{pgfscope}%
\begin{pgfscope}%
\pgfpathrectangle{\pgfqpoint{0.481978in}{0.331635in}}{\pgfqpoint{9.300000in}{7.700000in}}%
\pgfusepath{clip}%
\pgfsetbuttcap%
\pgfsetroundjoin%
\definecolor{currentfill}{rgb}{0.631373,0.788235,0.956863}%
\pgfsetfillcolor{currentfill}%
\pgfsetlinewidth{0.481800pt}%
\definecolor{currentstroke}{rgb}{1.000000,1.000000,1.000000}%
\pgfsetstrokecolor{currentstroke}%
\pgfsetdash{}{0pt}%
\pgfpathmoveto{\pgfqpoint{5.729756in}{5.704617in}}%
\pgfpathcurveto{\pgfqpoint{5.740806in}{5.704617in}}{\pgfqpoint{5.751405in}{5.709007in}}{\pgfqpoint{5.759219in}{5.716821in}}%
\pgfpathcurveto{\pgfqpoint{5.767033in}{5.724634in}}{\pgfqpoint{5.771423in}{5.735233in}}{\pgfqpoint{5.771423in}{5.746283in}}%
\pgfpathcurveto{\pgfqpoint{5.771423in}{5.757334in}}{\pgfqpoint{5.767033in}{5.767933in}}{\pgfqpoint{5.759219in}{5.775746in}}%
\pgfpathcurveto{\pgfqpoint{5.751405in}{5.783560in}}{\pgfqpoint{5.740806in}{5.787950in}}{\pgfqpoint{5.729756in}{5.787950in}}%
\pgfpathcurveto{\pgfqpoint{5.718706in}{5.787950in}}{\pgfqpoint{5.708107in}{5.783560in}}{\pgfqpoint{5.700293in}{5.775746in}}%
\pgfpathcurveto{\pgfqpoint{5.692480in}{5.767933in}}{\pgfqpoint{5.688089in}{5.757334in}}{\pgfqpoint{5.688089in}{5.746283in}}%
\pgfpathcurveto{\pgfqpoint{5.688089in}{5.735233in}}{\pgfqpoint{5.692480in}{5.724634in}}{\pgfqpoint{5.700293in}{5.716821in}}%
\pgfpathcurveto{\pgfqpoint{5.708107in}{5.709007in}}{\pgfqpoint{5.718706in}{5.704617in}}{\pgfqpoint{5.729756in}{5.704617in}}%
\pgfpathclose%
\pgfusepath{stroke,fill}%
\end{pgfscope}%
\begin{pgfscope}%
\pgfpathrectangle{\pgfqpoint{0.481978in}{0.331635in}}{\pgfqpoint{9.300000in}{7.700000in}}%
\pgfusepath{clip}%
\pgfsetbuttcap%
\pgfsetroundjoin%
\definecolor{currentfill}{rgb}{0.631373,0.788235,0.956863}%
\pgfsetfillcolor{currentfill}%
\pgfsetlinewidth{0.481800pt}%
\definecolor{currentstroke}{rgb}{1.000000,1.000000,1.000000}%
\pgfsetstrokecolor{currentstroke}%
\pgfsetdash{}{0pt}%
\pgfpathmoveto{\pgfqpoint{5.526353in}{1.621745in}}%
\pgfpathcurveto{\pgfqpoint{5.537403in}{1.621745in}}{\pgfqpoint{5.548002in}{1.626135in}}{\pgfqpoint{5.555816in}{1.633949in}}%
\pgfpathcurveto{\pgfqpoint{5.563629in}{1.641763in}}{\pgfqpoint{5.568019in}{1.652362in}}{\pgfqpoint{5.568019in}{1.663412in}}%
\pgfpathcurveto{\pgfqpoint{5.568019in}{1.674462in}}{\pgfqpoint{5.563629in}{1.685061in}}{\pgfqpoint{5.555816in}{1.692875in}}%
\pgfpathcurveto{\pgfqpoint{5.548002in}{1.700688in}}{\pgfqpoint{5.537403in}{1.705078in}}{\pgfqpoint{5.526353in}{1.705078in}}%
\pgfpathcurveto{\pgfqpoint{5.515303in}{1.705078in}}{\pgfqpoint{5.504704in}{1.700688in}}{\pgfqpoint{5.496890in}{1.692875in}}%
\pgfpathcurveto{\pgfqpoint{5.489076in}{1.685061in}}{\pgfqpoint{5.484686in}{1.674462in}}{\pgfqpoint{5.484686in}{1.663412in}}%
\pgfpathcurveto{\pgfqpoint{5.484686in}{1.652362in}}{\pgfqpoint{5.489076in}{1.641763in}}{\pgfqpoint{5.496890in}{1.633949in}}%
\pgfpathcurveto{\pgfqpoint{5.504704in}{1.626135in}}{\pgfqpoint{5.515303in}{1.621745in}}{\pgfqpoint{5.526353in}{1.621745in}}%
\pgfpathclose%
\pgfusepath{stroke,fill}%
\end{pgfscope}%
\begin{pgfscope}%
\pgfpathrectangle{\pgfqpoint{0.481978in}{0.331635in}}{\pgfqpoint{9.300000in}{7.700000in}}%
\pgfusepath{clip}%
\pgfsetbuttcap%
\pgfsetroundjoin%
\definecolor{currentfill}{rgb}{0.631373,0.788235,0.956863}%
\pgfsetfillcolor{currentfill}%
\pgfsetlinewidth{0.481800pt}%
\definecolor{currentstroke}{rgb}{1.000000,1.000000,1.000000}%
\pgfsetstrokecolor{currentstroke}%
\pgfsetdash{}{0pt}%
\pgfpathmoveto{\pgfqpoint{6.278735in}{2.388750in}}%
\pgfpathcurveto{\pgfqpoint{6.289785in}{2.388750in}}{\pgfqpoint{6.300384in}{2.393141in}}{\pgfqpoint{6.308198in}{2.400954in}}%
\pgfpathcurveto{\pgfqpoint{6.316012in}{2.408768in}}{\pgfqpoint{6.320402in}{2.419367in}}{\pgfqpoint{6.320402in}{2.430417in}}%
\pgfpathcurveto{\pgfqpoint{6.320402in}{2.441467in}}{\pgfqpoint{6.316012in}{2.452066in}}{\pgfqpoint{6.308198in}{2.459880in}}%
\pgfpathcurveto{\pgfqpoint{6.300384in}{2.467693in}}{\pgfqpoint{6.289785in}{2.472084in}}{\pgfqpoint{6.278735in}{2.472084in}}%
\pgfpathcurveto{\pgfqpoint{6.267685in}{2.472084in}}{\pgfqpoint{6.257086in}{2.467693in}}{\pgfqpoint{6.249272in}{2.459880in}}%
\pgfpathcurveto{\pgfqpoint{6.241459in}{2.452066in}}{\pgfqpoint{6.237069in}{2.441467in}}{\pgfqpoint{6.237069in}{2.430417in}}%
\pgfpathcurveto{\pgfqpoint{6.237069in}{2.419367in}}{\pgfqpoint{6.241459in}{2.408768in}}{\pgfqpoint{6.249272in}{2.400954in}}%
\pgfpathcurveto{\pgfqpoint{6.257086in}{2.393141in}}{\pgfqpoint{6.267685in}{2.388750in}}{\pgfqpoint{6.278735in}{2.388750in}}%
\pgfpathclose%
\pgfusepath{stroke,fill}%
\end{pgfscope}%
\begin{pgfscope}%
\pgfpathrectangle{\pgfqpoint{0.481978in}{0.331635in}}{\pgfqpoint{9.300000in}{7.700000in}}%
\pgfusepath{clip}%
\pgfsetbuttcap%
\pgfsetroundjoin%
\definecolor{currentfill}{rgb}{0.631373,0.788235,0.956863}%
\pgfsetfillcolor{currentfill}%
\pgfsetlinewidth{0.481800pt}%
\definecolor{currentstroke}{rgb}{1.000000,1.000000,1.000000}%
\pgfsetstrokecolor{currentstroke}%
\pgfsetdash{}{0pt}%
\pgfpathmoveto{\pgfqpoint{2.008732in}{2.104892in}}%
\pgfpathcurveto{\pgfqpoint{2.019782in}{2.104892in}}{\pgfqpoint{2.030381in}{2.109282in}}{\pgfqpoint{2.038195in}{2.117096in}}%
\pgfpathcurveto{\pgfqpoint{2.046008in}{2.124909in}}{\pgfqpoint{2.050399in}{2.135508in}}{\pgfqpoint{2.050399in}{2.146558in}}%
\pgfpathcurveto{\pgfqpoint{2.050399in}{2.157608in}}{\pgfqpoint{2.046008in}{2.168208in}}{\pgfqpoint{2.038195in}{2.176021in}}%
\pgfpathcurveto{\pgfqpoint{2.030381in}{2.183835in}}{\pgfqpoint{2.019782in}{2.188225in}}{\pgfqpoint{2.008732in}{2.188225in}}%
\pgfpathcurveto{\pgfqpoint{1.997682in}{2.188225in}}{\pgfqpoint{1.987083in}{2.183835in}}{\pgfqpoint{1.979269in}{2.176021in}}%
\pgfpathcurveto{\pgfqpoint{1.971456in}{2.168208in}}{\pgfqpoint{1.967065in}{2.157608in}}{\pgfqpoint{1.967065in}{2.146558in}}%
\pgfpathcurveto{\pgfqpoint{1.967065in}{2.135508in}}{\pgfqpoint{1.971456in}{2.124909in}}{\pgfqpoint{1.979269in}{2.117096in}}%
\pgfpathcurveto{\pgfqpoint{1.987083in}{2.109282in}}{\pgfqpoint{1.997682in}{2.104892in}}{\pgfqpoint{2.008732in}{2.104892in}}%
\pgfpathclose%
\pgfusepath{stroke,fill}%
\end{pgfscope}%
\begin{pgfscope}%
\pgfpathrectangle{\pgfqpoint{0.481978in}{0.331635in}}{\pgfqpoint{9.300000in}{7.700000in}}%
\pgfusepath{clip}%
\pgfsetbuttcap%
\pgfsetroundjoin%
\definecolor{currentfill}{rgb}{0.631373,0.788235,0.956863}%
\pgfsetfillcolor{currentfill}%
\pgfsetlinewidth{0.481800pt}%
\definecolor{currentstroke}{rgb}{1.000000,1.000000,1.000000}%
\pgfsetstrokecolor{currentstroke}%
\pgfsetdash{}{0pt}%
\pgfpathmoveto{\pgfqpoint{5.928515in}{1.734169in}}%
\pgfpathcurveto{\pgfqpoint{5.939565in}{1.734169in}}{\pgfqpoint{5.950164in}{1.738560in}}{\pgfqpoint{5.957978in}{1.746373in}}%
\pgfpathcurveto{\pgfqpoint{5.965791in}{1.754187in}}{\pgfqpoint{5.970182in}{1.764786in}}{\pgfqpoint{5.970182in}{1.775836in}}%
\pgfpathcurveto{\pgfqpoint{5.970182in}{1.786886in}}{\pgfqpoint{5.965791in}{1.797485in}}{\pgfqpoint{5.957978in}{1.805299in}}%
\pgfpathcurveto{\pgfqpoint{5.950164in}{1.813112in}}{\pgfqpoint{5.939565in}{1.817503in}}{\pgfqpoint{5.928515in}{1.817503in}}%
\pgfpathcurveto{\pgfqpoint{5.917465in}{1.817503in}}{\pgfqpoint{5.906866in}{1.813112in}}{\pgfqpoint{5.899052in}{1.805299in}}%
\pgfpathcurveto{\pgfqpoint{5.891238in}{1.797485in}}{\pgfqpoint{5.886848in}{1.786886in}}{\pgfqpoint{5.886848in}{1.775836in}}%
\pgfpathcurveto{\pgfqpoint{5.886848in}{1.764786in}}{\pgfqpoint{5.891238in}{1.754187in}}{\pgfqpoint{5.899052in}{1.746373in}}%
\pgfpathcurveto{\pgfqpoint{5.906866in}{1.738560in}}{\pgfqpoint{5.917465in}{1.734169in}}{\pgfqpoint{5.928515in}{1.734169in}}%
\pgfpathclose%
\pgfusepath{stroke,fill}%
\end{pgfscope}%
\begin{pgfscope}%
\pgfpathrectangle{\pgfqpoint{0.481978in}{0.331635in}}{\pgfqpoint{9.300000in}{7.700000in}}%
\pgfusepath{clip}%
\pgfsetbuttcap%
\pgfsetroundjoin%
\definecolor{currentfill}{rgb}{0.631373,0.788235,0.956863}%
\pgfsetfillcolor{currentfill}%
\pgfsetlinewidth{0.481800pt}%
\definecolor{currentstroke}{rgb}{1.000000,1.000000,1.000000}%
\pgfsetstrokecolor{currentstroke}%
\pgfsetdash{}{0pt}%
\pgfpathmoveto{\pgfqpoint{6.134955in}{2.431958in}}%
\pgfpathcurveto{\pgfqpoint{6.146005in}{2.431958in}}{\pgfqpoint{6.156604in}{2.436348in}}{\pgfqpoint{6.164418in}{2.444161in}}%
\pgfpathcurveto{\pgfqpoint{6.172231in}{2.451975in}}{\pgfqpoint{6.176621in}{2.462574in}}{\pgfqpoint{6.176621in}{2.473624in}}%
\pgfpathcurveto{\pgfqpoint{6.176621in}{2.484674in}}{\pgfqpoint{6.172231in}{2.495273in}}{\pgfqpoint{6.164418in}{2.503087in}}%
\pgfpathcurveto{\pgfqpoint{6.156604in}{2.510901in}}{\pgfqpoint{6.146005in}{2.515291in}}{\pgfqpoint{6.134955in}{2.515291in}}%
\pgfpathcurveto{\pgfqpoint{6.123905in}{2.515291in}}{\pgfqpoint{6.113306in}{2.510901in}}{\pgfqpoint{6.105492in}{2.503087in}}%
\pgfpathcurveto{\pgfqpoint{6.097678in}{2.495273in}}{\pgfqpoint{6.093288in}{2.484674in}}{\pgfqpoint{6.093288in}{2.473624in}}%
\pgfpathcurveto{\pgfqpoint{6.093288in}{2.462574in}}{\pgfqpoint{6.097678in}{2.451975in}}{\pgfqpoint{6.105492in}{2.444161in}}%
\pgfpathcurveto{\pgfqpoint{6.113306in}{2.436348in}}{\pgfqpoint{6.123905in}{2.431958in}}{\pgfqpoint{6.134955in}{2.431958in}}%
\pgfpathclose%
\pgfusepath{stroke,fill}%
\end{pgfscope}%
\begin{pgfscope}%
\pgfpathrectangle{\pgfqpoint{0.481978in}{0.331635in}}{\pgfqpoint{9.300000in}{7.700000in}}%
\pgfusepath{clip}%
\pgfsetbuttcap%
\pgfsetroundjoin%
\definecolor{currentfill}{rgb}{0.631373,0.788235,0.956863}%
\pgfsetfillcolor{currentfill}%
\pgfsetlinewidth{0.481800pt}%
\definecolor{currentstroke}{rgb}{1.000000,1.000000,1.000000}%
\pgfsetstrokecolor{currentstroke}%
\pgfsetdash{}{0pt}%
\pgfpathmoveto{\pgfqpoint{5.647865in}{3.622493in}}%
\pgfpathcurveto{\pgfqpoint{5.658915in}{3.622493in}}{\pgfqpoint{5.669514in}{3.626883in}}{\pgfqpoint{5.677327in}{3.634696in}}%
\pgfpathcurveto{\pgfqpoint{5.685141in}{3.642510in}}{\pgfqpoint{5.689531in}{3.653109in}}{\pgfqpoint{5.689531in}{3.664159in}}%
\pgfpathcurveto{\pgfqpoint{5.689531in}{3.675209in}}{\pgfqpoint{5.685141in}{3.685808in}}{\pgfqpoint{5.677327in}{3.693622in}}%
\pgfpathcurveto{\pgfqpoint{5.669514in}{3.701436in}}{\pgfqpoint{5.658915in}{3.705826in}}{\pgfqpoint{5.647865in}{3.705826in}}%
\pgfpathcurveto{\pgfqpoint{5.636814in}{3.705826in}}{\pgfqpoint{5.626215in}{3.701436in}}{\pgfqpoint{5.618402in}{3.693622in}}%
\pgfpathcurveto{\pgfqpoint{5.610588in}{3.685808in}}{\pgfqpoint{5.606198in}{3.675209in}}{\pgfqpoint{5.606198in}{3.664159in}}%
\pgfpathcurveto{\pgfqpoint{5.606198in}{3.653109in}}{\pgfqpoint{5.610588in}{3.642510in}}{\pgfqpoint{5.618402in}{3.634696in}}%
\pgfpathcurveto{\pgfqpoint{5.626215in}{3.626883in}}{\pgfqpoint{5.636814in}{3.622493in}}{\pgfqpoint{5.647865in}{3.622493in}}%
\pgfpathclose%
\pgfusepath{stroke,fill}%
\end{pgfscope}%
\begin{pgfscope}%
\pgfpathrectangle{\pgfqpoint{0.481978in}{0.331635in}}{\pgfqpoint{9.300000in}{7.700000in}}%
\pgfusepath{clip}%
\pgfsetbuttcap%
\pgfsetroundjoin%
\definecolor{currentfill}{rgb}{0.631373,0.788235,0.956863}%
\pgfsetfillcolor{currentfill}%
\pgfsetlinewidth{0.481800pt}%
\definecolor{currentstroke}{rgb}{1.000000,1.000000,1.000000}%
\pgfsetstrokecolor{currentstroke}%
\pgfsetdash{}{0pt}%
\pgfpathmoveto{\pgfqpoint{6.505006in}{5.405721in}}%
\pgfpathcurveto{\pgfqpoint{6.516056in}{5.405721in}}{\pgfqpoint{6.526655in}{5.410111in}}{\pgfqpoint{6.534469in}{5.417924in}}%
\pgfpathcurveto{\pgfqpoint{6.542282in}{5.425738in}}{\pgfqpoint{6.546672in}{5.436337in}}{\pgfqpoint{6.546672in}{5.447387in}}%
\pgfpathcurveto{\pgfqpoint{6.546672in}{5.458437in}}{\pgfqpoint{6.542282in}{5.469036in}}{\pgfqpoint{6.534469in}{5.476850in}}%
\pgfpathcurveto{\pgfqpoint{6.526655in}{5.484664in}}{\pgfqpoint{6.516056in}{5.489054in}}{\pgfqpoint{6.505006in}{5.489054in}}%
\pgfpathcurveto{\pgfqpoint{6.493956in}{5.489054in}}{\pgfqpoint{6.483357in}{5.484664in}}{\pgfqpoint{6.475543in}{5.476850in}}%
\pgfpathcurveto{\pgfqpoint{6.467729in}{5.469036in}}{\pgfqpoint{6.463339in}{5.458437in}}{\pgfqpoint{6.463339in}{5.447387in}}%
\pgfpathcurveto{\pgfqpoint{6.463339in}{5.436337in}}{\pgfqpoint{6.467729in}{5.425738in}}{\pgfqpoint{6.475543in}{5.417924in}}%
\pgfpathcurveto{\pgfqpoint{6.483357in}{5.410111in}}{\pgfqpoint{6.493956in}{5.405721in}}{\pgfqpoint{6.505006in}{5.405721in}}%
\pgfpathclose%
\pgfusepath{stroke,fill}%
\end{pgfscope}%
\begin{pgfscope}%
\pgfpathrectangle{\pgfqpoint{0.481978in}{0.331635in}}{\pgfqpoint{9.300000in}{7.700000in}}%
\pgfusepath{clip}%
\pgfsetbuttcap%
\pgfsetroundjoin%
\definecolor{currentfill}{rgb}{0.631373,0.788235,0.956863}%
\pgfsetfillcolor{currentfill}%
\pgfsetlinewidth{0.481800pt}%
\definecolor{currentstroke}{rgb}{1.000000,1.000000,1.000000}%
\pgfsetstrokecolor{currentstroke}%
\pgfsetdash{}{0pt}%
\pgfpathmoveto{\pgfqpoint{6.819209in}{3.642925in}}%
\pgfpathcurveto{\pgfqpoint{6.830259in}{3.642925in}}{\pgfqpoint{6.840859in}{3.647315in}}{\pgfqpoint{6.848672in}{3.655128in}}%
\pgfpathcurveto{\pgfqpoint{6.856486in}{3.662942in}}{\pgfqpoint{6.860876in}{3.673541in}}{\pgfqpoint{6.860876in}{3.684591in}}%
\pgfpathcurveto{\pgfqpoint{6.860876in}{3.695641in}}{\pgfqpoint{6.856486in}{3.706240in}}{\pgfqpoint{6.848672in}{3.714054in}}%
\pgfpathcurveto{\pgfqpoint{6.840859in}{3.721868in}}{\pgfqpoint{6.830259in}{3.726258in}}{\pgfqpoint{6.819209in}{3.726258in}}%
\pgfpathcurveto{\pgfqpoint{6.808159in}{3.726258in}}{\pgfqpoint{6.797560in}{3.721868in}}{\pgfqpoint{6.789747in}{3.714054in}}%
\pgfpathcurveto{\pgfqpoint{6.781933in}{3.706240in}}{\pgfqpoint{6.777543in}{3.695641in}}{\pgfqpoint{6.777543in}{3.684591in}}%
\pgfpathcurveto{\pgfqpoint{6.777543in}{3.673541in}}{\pgfqpoint{6.781933in}{3.662942in}}{\pgfqpoint{6.789747in}{3.655128in}}%
\pgfpathcurveto{\pgfqpoint{6.797560in}{3.647315in}}{\pgfqpoint{6.808159in}{3.642925in}}{\pgfqpoint{6.819209in}{3.642925in}}%
\pgfpathclose%
\pgfusepath{stroke,fill}%
\end{pgfscope}%
\begin{pgfscope}%
\pgfpathrectangle{\pgfqpoint{0.481978in}{0.331635in}}{\pgfqpoint{9.300000in}{7.700000in}}%
\pgfusepath{clip}%
\pgfsetbuttcap%
\pgfsetroundjoin%
\definecolor{currentfill}{rgb}{0.631373,0.788235,0.956863}%
\pgfsetfillcolor{currentfill}%
\pgfsetlinewidth{0.481800pt}%
\definecolor{currentstroke}{rgb}{1.000000,1.000000,1.000000}%
\pgfsetstrokecolor{currentstroke}%
\pgfsetdash{}{0pt}%
\pgfpathmoveto{\pgfqpoint{7.024686in}{3.292519in}}%
\pgfpathcurveto{\pgfqpoint{7.035736in}{3.292519in}}{\pgfqpoint{7.046335in}{3.296909in}}{\pgfqpoint{7.054149in}{3.304723in}}%
\pgfpathcurveto{\pgfqpoint{7.061963in}{3.312536in}}{\pgfqpoint{7.066353in}{3.323135in}}{\pgfqpoint{7.066353in}{3.334185in}}%
\pgfpathcurveto{\pgfqpoint{7.066353in}{3.345236in}}{\pgfqpoint{7.061963in}{3.355835in}}{\pgfqpoint{7.054149in}{3.363648in}}%
\pgfpathcurveto{\pgfqpoint{7.046335in}{3.371462in}}{\pgfqpoint{7.035736in}{3.375852in}}{\pgfqpoint{7.024686in}{3.375852in}}%
\pgfpathcurveto{\pgfqpoint{7.013636in}{3.375852in}}{\pgfqpoint{7.003037in}{3.371462in}}{\pgfqpoint{6.995223in}{3.363648in}}%
\pgfpathcurveto{\pgfqpoint{6.987410in}{3.355835in}}{\pgfqpoint{6.983019in}{3.345236in}}{\pgfqpoint{6.983019in}{3.334185in}}%
\pgfpathcurveto{\pgfqpoint{6.983019in}{3.323135in}}{\pgfqpoint{6.987410in}{3.312536in}}{\pgfqpoint{6.995223in}{3.304723in}}%
\pgfpathcurveto{\pgfqpoint{7.003037in}{3.296909in}}{\pgfqpoint{7.013636in}{3.292519in}}{\pgfqpoint{7.024686in}{3.292519in}}%
\pgfpathclose%
\pgfusepath{stroke,fill}%
\end{pgfscope}%
\begin{pgfscope}%
\pgfpathrectangle{\pgfqpoint{0.481978in}{0.331635in}}{\pgfqpoint{9.300000in}{7.700000in}}%
\pgfusepath{clip}%
\pgfsetbuttcap%
\pgfsetroundjoin%
\definecolor{currentfill}{rgb}{0.631373,0.788235,0.956863}%
\pgfsetfillcolor{currentfill}%
\pgfsetlinewidth{0.481800pt}%
\definecolor{currentstroke}{rgb}{1.000000,1.000000,1.000000}%
\pgfsetstrokecolor{currentstroke}%
\pgfsetdash{}{0pt}%
\pgfpathmoveto{\pgfqpoint{2.950033in}{6.667632in}}%
\pgfpathcurveto{\pgfqpoint{2.961083in}{6.667632in}}{\pgfqpoint{2.971682in}{6.672022in}}{\pgfqpoint{2.979495in}{6.679836in}}%
\pgfpathcurveto{\pgfqpoint{2.987309in}{6.687650in}}{\pgfqpoint{2.991699in}{6.698249in}}{\pgfqpoint{2.991699in}{6.709299in}}%
\pgfpathcurveto{\pgfqpoint{2.991699in}{6.720349in}}{\pgfqpoint{2.987309in}{6.730948in}}{\pgfqpoint{2.979495in}{6.738761in}}%
\pgfpathcurveto{\pgfqpoint{2.971682in}{6.746575in}}{\pgfqpoint{2.961083in}{6.750965in}}{\pgfqpoint{2.950033in}{6.750965in}}%
\pgfpathcurveto{\pgfqpoint{2.938983in}{6.750965in}}{\pgfqpoint{2.928383in}{6.746575in}}{\pgfqpoint{2.920570in}{6.738761in}}%
\pgfpathcurveto{\pgfqpoint{2.912756in}{6.730948in}}{\pgfqpoint{2.908366in}{6.720349in}}{\pgfqpoint{2.908366in}{6.709299in}}%
\pgfpathcurveto{\pgfqpoint{2.908366in}{6.698249in}}{\pgfqpoint{2.912756in}{6.687650in}}{\pgfqpoint{2.920570in}{6.679836in}}%
\pgfpathcurveto{\pgfqpoint{2.928383in}{6.672022in}}{\pgfqpoint{2.938983in}{6.667632in}}{\pgfqpoint{2.950033in}{6.667632in}}%
\pgfpathclose%
\pgfusepath{stroke,fill}%
\end{pgfscope}%
\begin{pgfscope}%
\pgfpathrectangle{\pgfqpoint{0.481978in}{0.331635in}}{\pgfqpoint{9.300000in}{7.700000in}}%
\pgfusepath{clip}%
\pgfsetbuttcap%
\pgfsetroundjoin%
\definecolor{currentfill}{rgb}{0.631373,0.788235,0.956863}%
\pgfsetfillcolor{currentfill}%
\pgfsetlinewidth{0.481800pt}%
\definecolor{currentstroke}{rgb}{1.000000,1.000000,1.000000}%
\pgfsetstrokecolor{currentstroke}%
\pgfsetdash{}{0pt}%
\pgfpathmoveto{\pgfqpoint{4.721839in}{7.137694in}}%
\pgfpathcurveto{\pgfqpoint{4.732889in}{7.137694in}}{\pgfqpoint{4.743488in}{7.142084in}}{\pgfqpoint{4.751301in}{7.149898in}}%
\pgfpathcurveto{\pgfqpoint{4.759115in}{7.157712in}}{\pgfqpoint{4.763505in}{7.168311in}}{\pgfqpoint{4.763505in}{7.179361in}}%
\pgfpathcurveto{\pgfqpoint{4.763505in}{7.190411in}}{\pgfqpoint{4.759115in}{7.201010in}}{\pgfqpoint{4.751301in}{7.208824in}}%
\pgfpathcurveto{\pgfqpoint{4.743488in}{7.216637in}}{\pgfqpoint{4.732889in}{7.221027in}}{\pgfqpoint{4.721839in}{7.221027in}}%
\pgfpathcurveto{\pgfqpoint{4.710788in}{7.221027in}}{\pgfqpoint{4.700189in}{7.216637in}}{\pgfqpoint{4.692376in}{7.208824in}}%
\pgfpathcurveto{\pgfqpoint{4.684562in}{7.201010in}}{\pgfqpoint{4.680172in}{7.190411in}}{\pgfqpoint{4.680172in}{7.179361in}}%
\pgfpathcurveto{\pgfqpoint{4.680172in}{7.168311in}}{\pgfqpoint{4.684562in}{7.157712in}}{\pgfqpoint{4.692376in}{7.149898in}}%
\pgfpathcurveto{\pgfqpoint{4.700189in}{7.142084in}}{\pgfqpoint{4.710788in}{7.137694in}}{\pgfqpoint{4.721839in}{7.137694in}}%
\pgfpathclose%
\pgfusepath{stroke,fill}%
\end{pgfscope}%
\begin{pgfscope}%
\pgfpathrectangle{\pgfqpoint{0.481978in}{0.331635in}}{\pgfqpoint{9.300000in}{7.700000in}}%
\pgfusepath{clip}%
\pgfsetbuttcap%
\pgfsetroundjoin%
\definecolor{currentfill}{rgb}{0.631373,0.788235,0.956863}%
\pgfsetfillcolor{currentfill}%
\pgfsetlinewidth{0.481800pt}%
\definecolor{currentstroke}{rgb}{1.000000,1.000000,1.000000}%
\pgfsetstrokecolor{currentstroke}%
\pgfsetdash{}{0pt}%
\pgfpathmoveto{\pgfqpoint{6.257325in}{5.295673in}}%
\pgfpathcurveto{\pgfqpoint{6.268375in}{5.295673in}}{\pgfqpoint{6.278974in}{5.300063in}}{\pgfqpoint{6.286787in}{5.307877in}}%
\pgfpathcurveto{\pgfqpoint{6.294601in}{5.315690in}}{\pgfqpoint{6.298991in}{5.326289in}}{\pgfqpoint{6.298991in}{5.337340in}}%
\pgfpathcurveto{\pgfqpoint{6.298991in}{5.348390in}}{\pgfqpoint{6.294601in}{5.358989in}}{\pgfqpoint{6.286787in}{5.366802in}}%
\pgfpathcurveto{\pgfqpoint{6.278974in}{5.374616in}}{\pgfqpoint{6.268375in}{5.379006in}}{\pgfqpoint{6.257325in}{5.379006in}}%
\pgfpathcurveto{\pgfqpoint{6.246275in}{5.379006in}}{\pgfqpoint{6.235676in}{5.374616in}}{\pgfqpoint{6.227862in}{5.366802in}}%
\pgfpathcurveto{\pgfqpoint{6.220048in}{5.358989in}}{\pgfqpoint{6.215658in}{5.348390in}}{\pgfqpoint{6.215658in}{5.337340in}}%
\pgfpathcurveto{\pgfqpoint{6.215658in}{5.326289in}}{\pgfqpoint{6.220048in}{5.315690in}}{\pgfqpoint{6.227862in}{5.307877in}}%
\pgfpathcurveto{\pgfqpoint{6.235676in}{5.300063in}}{\pgfqpoint{6.246275in}{5.295673in}}{\pgfqpoint{6.257325in}{5.295673in}}%
\pgfpathclose%
\pgfusepath{stroke,fill}%
\end{pgfscope}%
\begin{pgfscope}%
\pgfpathrectangle{\pgfqpoint{0.481978in}{0.331635in}}{\pgfqpoint{9.300000in}{7.700000in}}%
\pgfusepath{clip}%
\pgfsetbuttcap%
\pgfsetroundjoin%
\definecolor{currentfill}{rgb}{0.631373,0.788235,0.956863}%
\pgfsetfillcolor{currentfill}%
\pgfsetlinewidth{0.481800pt}%
\definecolor{currentstroke}{rgb}{1.000000,1.000000,1.000000}%
\pgfsetstrokecolor{currentstroke}%
\pgfsetdash{}{0pt}%
\pgfpathmoveto{\pgfqpoint{7.727200in}{4.813224in}}%
\pgfpathcurveto{\pgfqpoint{7.738250in}{4.813224in}}{\pgfqpoint{7.748849in}{4.817614in}}{\pgfqpoint{7.756662in}{4.825428in}}%
\pgfpathcurveto{\pgfqpoint{7.764476in}{4.833242in}}{\pgfqpoint{7.768866in}{4.843841in}}{\pgfqpoint{7.768866in}{4.854891in}}%
\pgfpathcurveto{\pgfqpoint{7.768866in}{4.865941in}}{\pgfqpoint{7.764476in}{4.876540in}}{\pgfqpoint{7.756662in}{4.884354in}}%
\pgfpathcurveto{\pgfqpoint{7.748849in}{4.892167in}}{\pgfqpoint{7.738250in}{4.896557in}}{\pgfqpoint{7.727200in}{4.896557in}}%
\pgfpathcurveto{\pgfqpoint{7.716150in}{4.896557in}}{\pgfqpoint{7.705550in}{4.892167in}}{\pgfqpoint{7.697737in}{4.884354in}}%
\pgfpathcurveto{\pgfqpoint{7.689923in}{4.876540in}}{\pgfqpoint{7.685533in}{4.865941in}}{\pgfqpoint{7.685533in}{4.854891in}}%
\pgfpathcurveto{\pgfqpoint{7.685533in}{4.843841in}}{\pgfqpoint{7.689923in}{4.833242in}}{\pgfqpoint{7.697737in}{4.825428in}}%
\pgfpathcurveto{\pgfqpoint{7.705550in}{4.817614in}}{\pgfqpoint{7.716150in}{4.813224in}}{\pgfqpoint{7.727200in}{4.813224in}}%
\pgfpathclose%
\pgfusepath{stroke,fill}%
\end{pgfscope}%
\begin{pgfscope}%
\pgfpathrectangle{\pgfqpoint{0.481978in}{0.331635in}}{\pgfqpoint{9.300000in}{7.700000in}}%
\pgfusepath{clip}%
\pgfsetbuttcap%
\pgfsetroundjoin%
\definecolor{currentfill}{rgb}{0.631373,0.788235,0.956863}%
\pgfsetfillcolor{currentfill}%
\pgfsetlinewidth{0.481800pt}%
\definecolor{currentstroke}{rgb}{1.000000,1.000000,1.000000}%
\pgfsetstrokecolor{currentstroke}%
\pgfsetdash{}{0pt}%
\pgfpathmoveto{\pgfqpoint{3.030295in}{1.686604in}}%
\pgfpathcurveto{\pgfqpoint{3.041345in}{1.686604in}}{\pgfqpoint{3.051944in}{1.690994in}}{\pgfqpoint{3.059758in}{1.698808in}}%
\pgfpathcurveto{\pgfqpoint{3.067571in}{1.706622in}}{\pgfqpoint{3.071962in}{1.717221in}}{\pgfqpoint{3.071962in}{1.728271in}}%
\pgfpathcurveto{\pgfqpoint{3.071962in}{1.739321in}}{\pgfqpoint{3.067571in}{1.749920in}}{\pgfqpoint{3.059758in}{1.757734in}}%
\pgfpathcurveto{\pgfqpoint{3.051944in}{1.765547in}}{\pgfqpoint{3.041345in}{1.769937in}}{\pgfqpoint{3.030295in}{1.769937in}}%
\pgfpathcurveto{\pgfqpoint{3.019245in}{1.769937in}}{\pgfqpoint{3.008646in}{1.765547in}}{\pgfqpoint{3.000832in}{1.757734in}}%
\pgfpathcurveto{\pgfqpoint{2.993019in}{1.749920in}}{\pgfqpoint{2.988628in}{1.739321in}}{\pgfqpoint{2.988628in}{1.728271in}}%
\pgfpathcurveto{\pgfqpoint{2.988628in}{1.717221in}}{\pgfqpoint{2.993019in}{1.706622in}}{\pgfqpoint{3.000832in}{1.698808in}}%
\pgfpathcurveto{\pgfqpoint{3.008646in}{1.690994in}}{\pgfqpoint{3.019245in}{1.686604in}}{\pgfqpoint{3.030295in}{1.686604in}}%
\pgfpathclose%
\pgfusepath{stroke,fill}%
\end{pgfscope}%
\begin{pgfscope}%
\pgfpathrectangle{\pgfqpoint{0.481978in}{0.331635in}}{\pgfqpoint{9.300000in}{7.700000in}}%
\pgfusepath{clip}%
\pgfsetbuttcap%
\pgfsetroundjoin%
\definecolor{currentfill}{rgb}{0.631373,0.788235,0.956863}%
\pgfsetfillcolor{currentfill}%
\pgfsetlinewidth{0.481800pt}%
\definecolor{currentstroke}{rgb}{1.000000,1.000000,1.000000}%
\pgfsetstrokecolor{currentstroke}%
\pgfsetdash{}{0pt}%
\pgfpathmoveto{\pgfqpoint{4.665732in}{6.455984in}}%
\pgfpathcurveto{\pgfqpoint{4.676782in}{6.455984in}}{\pgfqpoint{4.687381in}{6.460375in}}{\pgfqpoint{4.695194in}{6.468188in}}%
\pgfpathcurveto{\pgfqpoint{4.703008in}{6.476002in}}{\pgfqpoint{4.707398in}{6.486601in}}{\pgfqpoint{4.707398in}{6.497651in}}%
\pgfpathcurveto{\pgfqpoint{4.707398in}{6.508701in}}{\pgfqpoint{4.703008in}{6.519300in}}{\pgfqpoint{4.695194in}{6.527114in}}%
\pgfpathcurveto{\pgfqpoint{4.687381in}{6.534927in}}{\pgfqpoint{4.676782in}{6.539318in}}{\pgfqpoint{4.665732in}{6.539318in}}%
\pgfpathcurveto{\pgfqpoint{4.654681in}{6.539318in}}{\pgfqpoint{4.644082in}{6.534927in}}{\pgfqpoint{4.636269in}{6.527114in}}%
\pgfpathcurveto{\pgfqpoint{4.628455in}{6.519300in}}{\pgfqpoint{4.624065in}{6.508701in}}{\pgfqpoint{4.624065in}{6.497651in}}%
\pgfpathcurveto{\pgfqpoint{4.624065in}{6.486601in}}{\pgfqpoint{4.628455in}{6.476002in}}{\pgfqpoint{4.636269in}{6.468188in}}%
\pgfpathcurveto{\pgfqpoint{4.644082in}{6.460375in}}{\pgfqpoint{4.654681in}{6.455984in}}{\pgfqpoint{4.665732in}{6.455984in}}%
\pgfpathclose%
\pgfusepath{stroke,fill}%
\end{pgfscope}%
\begin{pgfscope}%
\pgfpathrectangle{\pgfqpoint{0.481978in}{0.331635in}}{\pgfqpoint{9.300000in}{7.700000in}}%
\pgfusepath{clip}%
\pgfsetbuttcap%
\pgfsetroundjoin%
\definecolor{currentfill}{rgb}{0.631373,0.788235,0.956863}%
\pgfsetfillcolor{currentfill}%
\pgfsetlinewidth{0.481800pt}%
\definecolor{currentstroke}{rgb}{1.000000,1.000000,1.000000}%
\pgfsetstrokecolor{currentstroke}%
\pgfsetdash{}{0pt}%
\pgfpathmoveto{\pgfqpoint{6.123240in}{1.999637in}}%
\pgfpathcurveto{\pgfqpoint{6.134290in}{1.999637in}}{\pgfqpoint{6.144890in}{2.004027in}}{\pgfqpoint{6.152703in}{2.011841in}}%
\pgfpathcurveto{\pgfqpoint{6.160517in}{2.019655in}}{\pgfqpoint{6.164907in}{2.030254in}}{\pgfqpoint{6.164907in}{2.041304in}}%
\pgfpathcurveto{\pgfqpoint{6.164907in}{2.052354in}}{\pgfqpoint{6.160517in}{2.062953in}}{\pgfqpoint{6.152703in}{2.070766in}}%
\pgfpathcurveto{\pgfqpoint{6.144890in}{2.078580in}}{\pgfqpoint{6.134290in}{2.082970in}}{\pgfqpoint{6.123240in}{2.082970in}}%
\pgfpathcurveto{\pgfqpoint{6.112190in}{2.082970in}}{\pgfqpoint{6.101591in}{2.078580in}}{\pgfqpoint{6.093778in}{2.070766in}}%
\pgfpathcurveto{\pgfqpoint{6.085964in}{2.062953in}}{\pgfqpoint{6.081574in}{2.052354in}}{\pgfqpoint{6.081574in}{2.041304in}}%
\pgfpathcurveto{\pgfqpoint{6.081574in}{2.030254in}}{\pgfqpoint{6.085964in}{2.019655in}}{\pgfqpoint{6.093778in}{2.011841in}}%
\pgfpathcurveto{\pgfqpoint{6.101591in}{2.004027in}}{\pgfqpoint{6.112190in}{1.999637in}}{\pgfqpoint{6.123240in}{1.999637in}}%
\pgfpathclose%
\pgfusepath{stroke,fill}%
\end{pgfscope}%
\begin{pgfscope}%
\pgfpathrectangle{\pgfqpoint{0.481978in}{0.331635in}}{\pgfqpoint{9.300000in}{7.700000in}}%
\pgfusepath{clip}%
\pgfsetbuttcap%
\pgfsetroundjoin%
\definecolor{currentfill}{rgb}{0.631373,0.788235,0.956863}%
\pgfsetfillcolor{currentfill}%
\pgfsetlinewidth{0.481800pt}%
\definecolor{currentstroke}{rgb}{1.000000,1.000000,1.000000}%
\pgfsetstrokecolor{currentstroke}%
\pgfsetdash{}{0pt}%
\pgfpathmoveto{\pgfqpoint{2.813753in}{6.481376in}}%
\pgfpathcurveto{\pgfqpoint{2.824803in}{6.481376in}}{\pgfqpoint{2.835402in}{6.485766in}}{\pgfqpoint{2.843216in}{6.493580in}}%
\pgfpathcurveto{\pgfqpoint{2.851030in}{6.501394in}}{\pgfqpoint{2.855420in}{6.511993in}}{\pgfqpoint{2.855420in}{6.523043in}}%
\pgfpathcurveto{\pgfqpoint{2.855420in}{6.534093in}}{\pgfqpoint{2.851030in}{6.544692in}}{\pgfqpoint{2.843216in}{6.552505in}}%
\pgfpathcurveto{\pgfqpoint{2.835402in}{6.560319in}}{\pgfqpoint{2.824803in}{6.564709in}}{\pgfqpoint{2.813753in}{6.564709in}}%
\pgfpathcurveto{\pgfqpoint{2.802703in}{6.564709in}}{\pgfqpoint{2.792104in}{6.560319in}}{\pgfqpoint{2.784290in}{6.552505in}}%
\pgfpathcurveto{\pgfqpoint{2.776477in}{6.544692in}}{\pgfqpoint{2.772086in}{6.534093in}}{\pgfqpoint{2.772086in}{6.523043in}}%
\pgfpathcurveto{\pgfqpoint{2.772086in}{6.511993in}}{\pgfqpoint{2.776477in}{6.501394in}}{\pgfqpoint{2.784290in}{6.493580in}}%
\pgfpathcurveto{\pgfqpoint{2.792104in}{6.485766in}}{\pgfqpoint{2.802703in}{6.481376in}}{\pgfqpoint{2.813753in}{6.481376in}}%
\pgfpathclose%
\pgfusepath{stroke,fill}%
\end{pgfscope}%
\begin{pgfscope}%
\pgfpathrectangle{\pgfqpoint{0.481978in}{0.331635in}}{\pgfqpoint{9.300000in}{7.700000in}}%
\pgfusepath{clip}%
\pgfsetbuttcap%
\pgfsetroundjoin%
\definecolor{currentfill}{rgb}{0.631373,0.788235,0.956863}%
\pgfsetfillcolor{currentfill}%
\pgfsetlinewidth{0.481800pt}%
\definecolor{currentstroke}{rgb}{1.000000,1.000000,1.000000}%
\pgfsetstrokecolor{currentstroke}%
\pgfsetdash{}{0pt}%
\pgfpathmoveto{\pgfqpoint{8.071034in}{4.382284in}}%
\pgfpathcurveto{\pgfqpoint{8.082085in}{4.382284in}}{\pgfqpoint{8.092684in}{4.386674in}}{\pgfqpoint{8.100497in}{4.394488in}}%
\pgfpathcurveto{\pgfqpoint{8.108311in}{4.402301in}}{\pgfqpoint{8.112701in}{4.412900in}}{\pgfqpoint{8.112701in}{4.423951in}}%
\pgfpathcurveto{\pgfqpoint{8.112701in}{4.435001in}}{\pgfqpoint{8.108311in}{4.445600in}}{\pgfqpoint{8.100497in}{4.453413in}}%
\pgfpathcurveto{\pgfqpoint{8.092684in}{4.461227in}}{\pgfqpoint{8.082085in}{4.465617in}}{\pgfqpoint{8.071034in}{4.465617in}}%
\pgfpathcurveto{\pgfqpoint{8.059984in}{4.465617in}}{\pgfqpoint{8.049385in}{4.461227in}}{\pgfqpoint{8.041572in}{4.453413in}}%
\pgfpathcurveto{\pgfqpoint{8.033758in}{4.445600in}}{\pgfqpoint{8.029368in}{4.435001in}}{\pgfqpoint{8.029368in}{4.423951in}}%
\pgfpathcurveto{\pgfqpoint{8.029368in}{4.412900in}}{\pgfqpoint{8.033758in}{4.402301in}}{\pgfqpoint{8.041572in}{4.394488in}}%
\pgfpathcurveto{\pgfqpoint{8.049385in}{4.386674in}}{\pgfqpoint{8.059984in}{4.382284in}}{\pgfqpoint{8.071034in}{4.382284in}}%
\pgfpathclose%
\pgfusepath{stroke,fill}%
\end{pgfscope}%
\begin{pgfscope}%
\pgfpathrectangle{\pgfqpoint{0.481978in}{0.331635in}}{\pgfqpoint{9.300000in}{7.700000in}}%
\pgfusepath{clip}%
\pgfsetbuttcap%
\pgfsetroundjoin%
\definecolor{currentfill}{rgb}{0.631373,0.788235,0.956863}%
\pgfsetfillcolor{currentfill}%
\pgfsetlinewidth{0.481800pt}%
\definecolor{currentstroke}{rgb}{1.000000,1.000000,1.000000}%
\pgfsetstrokecolor{currentstroke}%
\pgfsetdash{}{0pt}%
\pgfpathmoveto{\pgfqpoint{6.467051in}{2.323379in}}%
\pgfpathcurveto{\pgfqpoint{6.478101in}{2.323379in}}{\pgfqpoint{6.488700in}{2.327769in}}{\pgfqpoint{6.496514in}{2.335583in}}%
\pgfpathcurveto{\pgfqpoint{6.504328in}{2.343396in}}{\pgfqpoint{6.508718in}{2.353995in}}{\pgfqpoint{6.508718in}{2.365045in}}%
\pgfpathcurveto{\pgfqpoint{6.508718in}{2.376096in}}{\pgfqpoint{6.504328in}{2.386695in}}{\pgfqpoint{6.496514in}{2.394508in}}%
\pgfpathcurveto{\pgfqpoint{6.488700in}{2.402322in}}{\pgfqpoint{6.478101in}{2.406712in}}{\pgfqpoint{6.467051in}{2.406712in}}%
\pgfpathcurveto{\pgfqpoint{6.456001in}{2.406712in}}{\pgfqpoint{6.445402in}{2.402322in}}{\pgfqpoint{6.437588in}{2.394508in}}%
\pgfpathcurveto{\pgfqpoint{6.429775in}{2.386695in}}{\pgfqpoint{6.425384in}{2.376096in}}{\pgfqpoint{6.425384in}{2.365045in}}%
\pgfpathcurveto{\pgfqpoint{6.425384in}{2.353995in}}{\pgfqpoint{6.429775in}{2.343396in}}{\pgfqpoint{6.437588in}{2.335583in}}%
\pgfpathcurveto{\pgfqpoint{6.445402in}{2.327769in}}{\pgfqpoint{6.456001in}{2.323379in}}{\pgfqpoint{6.467051in}{2.323379in}}%
\pgfpathclose%
\pgfusepath{stroke,fill}%
\end{pgfscope}%
\begin{pgfscope}%
\pgfpathrectangle{\pgfqpoint{0.481978in}{0.331635in}}{\pgfqpoint{9.300000in}{7.700000in}}%
\pgfusepath{clip}%
\pgfsetbuttcap%
\pgfsetroundjoin%
\definecolor{currentfill}{rgb}{0.631373,0.788235,0.956863}%
\pgfsetfillcolor{currentfill}%
\pgfsetlinewidth{0.481800pt}%
\definecolor{currentstroke}{rgb}{1.000000,1.000000,1.000000}%
\pgfsetstrokecolor{currentstroke}%
\pgfsetdash{}{0pt}%
\pgfpathmoveto{\pgfqpoint{7.133111in}{4.672385in}}%
\pgfpathcurveto{\pgfqpoint{7.144161in}{4.672385in}}{\pgfqpoint{7.154760in}{4.676775in}}{\pgfqpoint{7.162573in}{4.684589in}}%
\pgfpathcurveto{\pgfqpoint{7.170387in}{4.692402in}}{\pgfqpoint{7.174777in}{4.703001in}}{\pgfqpoint{7.174777in}{4.714051in}}%
\pgfpathcurveto{\pgfqpoint{7.174777in}{4.725102in}}{\pgfqpoint{7.170387in}{4.735701in}}{\pgfqpoint{7.162573in}{4.743514in}}%
\pgfpathcurveto{\pgfqpoint{7.154760in}{4.751328in}}{\pgfqpoint{7.144161in}{4.755718in}}{\pgfqpoint{7.133111in}{4.755718in}}%
\pgfpathcurveto{\pgfqpoint{7.122061in}{4.755718in}}{\pgfqpoint{7.111462in}{4.751328in}}{\pgfqpoint{7.103648in}{4.743514in}}%
\pgfpathcurveto{\pgfqpoint{7.095834in}{4.735701in}}{\pgfqpoint{7.091444in}{4.725102in}}{\pgfqpoint{7.091444in}{4.714051in}}%
\pgfpathcurveto{\pgfqpoint{7.091444in}{4.703001in}}{\pgfqpoint{7.095834in}{4.692402in}}{\pgfqpoint{7.103648in}{4.684589in}}%
\pgfpathcurveto{\pgfqpoint{7.111462in}{4.676775in}}{\pgfqpoint{7.122061in}{4.672385in}}{\pgfqpoint{7.133111in}{4.672385in}}%
\pgfpathclose%
\pgfusepath{stroke,fill}%
\end{pgfscope}%
\begin{pgfscope}%
\pgfpathrectangle{\pgfqpoint{0.481978in}{0.331635in}}{\pgfqpoint{9.300000in}{7.700000in}}%
\pgfusepath{clip}%
\pgfsetbuttcap%
\pgfsetroundjoin%
\definecolor{currentfill}{rgb}{0.631373,0.788235,0.956863}%
\pgfsetfillcolor{currentfill}%
\pgfsetlinewidth{0.481800pt}%
\definecolor{currentstroke}{rgb}{1.000000,1.000000,1.000000}%
\pgfsetstrokecolor{currentstroke}%
\pgfsetdash{}{0pt}%
\pgfpathmoveto{\pgfqpoint{5.228856in}{2.150942in}}%
\pgfpathcurveto{\pgfqpoint{5.239906in}{2.150942in}}{\pgfqpoint{5.250505in}{2.155332in}}{\pgfqpoint{5.258319in}{2.163146in}}%
\pgfpathcurveto{\pgfqpoint{5.266132in}{2.170959in}}{\pgfqpoint{5.270523in}{2.181558in}}{\pgfqpoint{5.270523in}{2.192608in}}%
\pgfpathcurveto{\pgfqpoint{5.270523in}{2.203659in}}{\pgfqpoint{5.266132in}{2.214258in}}{\pgfqpoint{5.258319in}{2.222071in}}%
\pgfpathcurveto{\pgfqpoint{5.250505in}{2.229885in}}{\pgfqpoint{5.239906in}{2.234275in}}{\pgfqpoint{5.228856in}{2.234275in}}%
\pgfpathcurveto{\pgfqpoint{5.217806in}{2.234275in}}{\pgfqpoint{5.207207in}{2.229885in}}{\pgfqpoint{5.199393in}{2.222071in}}%
\pgfpathcurveto{\pgfqpoint{5.191580in}{2.214258in}}{\pgfqpoint{5.187189in}{2.203659in}}{\pgfqpoint{5.187189in}{2.192608in}}%
\pgfpathcurveto{\pgfqpoint{5.187189in}{2.181558in}}{\pgfqpoint{5.191580in}{2.170959in}}{\pgfqpoint{5.199393in}{2.163146in}}%
\pgfpathcurveto{\pgfqpoint{5.207207in}{2.155332in}}{\pgfqpoint{5.217806in}{2.150942in}}{\pgfqpoint{5.228856in}{2.150942in}}%
\pgfpathclose%
\pgfusepath{stroke,fill}%
\end{pgfscope}%
\begin{pgfscope}%
\pgfpathrectangle{\pgfqpoint{0.481978in}{0.331635in}}{\pgfqpoint{9.300000in}{7.700000in}}%
\pgfusepath{clip}%
\pgfsetbuttcap%
\pgfsetroundjoin%
\definecolor{currentfill}{rgb}{0.631373,0.788235,0.956863}%
\pgfsetfillcolor{currentfill}%
\pgfsetlinewidth{0.481800pt}%
\definecolor{currentstroke}{rgb}{1.000000,1.000000,1.000000}%
\pgfsetstrokecolor{currentstroke}%
\pgfsetdash{}{0pt}%
\pgfpathmoveto{\pgfqpoint{6.173996in}{3.277145in}}%
\pgfpathcurveto{\pgfqpoint{6.185046in}{3.277145in}}{\pgfqpoint{6.195645in}{3.281535in}}{\pgfqpoint{6.203459in}{3.289349in}}%
\pgfpathcurveto{\pgfqpoint{6.211273in}{3.297162in}}{\pgfqpoint{6.215663in}{3.307761in}}{\pgfqpoint{6.215663in}{3.318811in}}%
\pgfpathcurveto{\pgfqpoint{6.215663in}{3.329861in}}{\pgfqpoint{6.211273in}{3.340460in}}{\pgfqpoint{6.203459in}{3.348274in}}%
\pgfpathcurveto{\pgfqpoint{6.195645in}{3.356088in}}{\pgfqpoint{6.185046in}{3.360478in}}{\pgfqpoint{6.173996in}{3.360478in}}%
\pgfpathcurveto{\pgfqpoint{6.162946in}{3.360478in}}{\pgfqpoint{6.152347in}{3.356088in}}{\pgfqpoint{6.144533in}{3.348274in}}%
\pgfpathcurveto{\pgfqpoint{6.136720in}{3.340460in}}{\pgfqpoint{6.132330in}{3.329861in}}{\pgfqpoint{6.132330in}{3.318811in}}%
\pgfpathcurveto{\pgfqpoint{6.132330in}{3.307761in}}{\pgfqpoint{6.136720in}{3.297162in}}{\pgfqpoint{6.144533in}{3.289349in}}%
\pgfpathcurveto{\pgfqpoint{6.152347in}{3.281535in}}{\pgfqpoint{6.162946in}{3.277145in}}{\pgfqpoint{6.173996in}{3.277145in}}%
\pgfpathclose%
\pgfusepath{stroke,fill}%
\end{pgfscope}%
\begin{pgfscope}%
\pgfpathrectangle{\pgfqpoint{0.481978in}{0.331635in}}{\pgfqpoint{9.300000in}{7.700000in}}%
\pgfusepath{clip}%
\pgfsetbuttcap%
\pgfsetroundjoin%
\definecolor{currentfill}{rgb}{0.631373,0.788235,0.956863}%
\pgfsetfillcolor{currentfill}%
\pgfsetlinewidth{0.481800pt}%
\definecolor{currentstroke}{rgb}{1.000000,1.000000,1.000000}%
\pgfsetstrokecolor{currentstroke}%
\pgfsetdash{}{0pt}%
\pgfpathmoveto{\pgfqpoint{2.961409in}{1.495653in}}%
\pgfpathcurveto{\pgfqpoint{2.972459in}{1.495653in}}{\pgfqpoint{2.983058in}{1.500043in}}{\pgfqpoint{2.990872in}{1.507857in}}%
\pgfpathcurveto{\pgfqpoint{2.998686in}{1.515670in}}{\pgfqpoint{3.003076in}{1.526269in}}{\pgfqpoint{3.003076in}{1.537320in}}%
\pgfpathcurveto{\pgfqpoint{3.003076in}{1.548370in}}{\pgfqpoint{2.998686in}{1.558969in}}{\pgfqpoint{2.990872in}{1.566782in}}%
\pgfpathcurveto{\pgfqpoint{2.983058in}{1.574596in}}{\pgfqpoint{2.972459in}{1.578986in}}{\pgfqpoint{2.961409in}{1.578986in}}%
\pgfpathcurveto{\pgfqpoint{2.950359in}{1.578986in}}{\pgfqpoint{2.939760in}{1.574596in}}{\pgfqpoint{2.931946in}{1.566782in}}%
\pgfpathcurveto{\pgfqpoint{2.924133in}{1.558969in}}{\pgfqpoint{2.919743in}{1.548370in}}{\pgfqpoint{2.919743in}{1.537320in}}%
\pgfpathcurveto{\pgfqpoint{2.919743in}{1.526269in}}{\pgfqpoint{2.924133in}{1.515670in}}{\pgfqpoint{2.931946in}{1.507857in}}%
\pgfpathcurveto{\pgfqpoint{2.939760in}{1.500043in}}{\pgfqpoint{2.950359in}{1.495653in}}{\pgfqpoint{2.961409in}{1.495653in}}%
\pgfpathclose%
\pgfusepath{stroke,fill}%
\end{pgfscope}%
\begin{pgfscope}%
\pgfpathrectangle{\pgfqpoint{0.481978in}{0.331635in}}{\pgfqpoint{9.300000in}{7.700000in}}%
\pgfusepath{clip}%
\pgfsetbuttcap%
\pgfsetroundjoin%
\definecolor{currentfill}{rgb}{0.631373,0.788235,0.956863}%
\pgfsetfillcolor{currentfill}%
\pgfsetlinewidth{0.481800pt}%
\definecolor{currentstroke}{rgb}{1.000000,1.000000,1.000000}%
\pgfsetstrokecolor{currentstroke}%
\pgfsetdash{}{0pt}%
\pgfpathmoveto{\pgfqpoint{5.678749in}{4.846566in}}%
\pgfpathcurveto{\pgfqpoint{5.689799in}{4.846566in}}{\pgfqpoint{5.700398in}{4.850956in}}{\pgfqpoint{5.708212in}{4.858769in}}%
\pgfpathcurveto{\pgfqpoint{5.716026in}{4.866583in}}{\pgfqpoint{5.720416in}{4.877182in}}{\pgfqpoint{5.720416in}{4.888232in}}%
\pgfpathcurveto{\pgfqpoint{5.720416in}{4.899282in}}{\pgfqpoint{5.716026in}{4.909881in}}{\pgfqpoint{5.708212in}{4.917695in}}%
\pgfpathcurveto{\pgfqpoint{5.700398in}{4.925509in}}{\pgfqpoint{5.689799in}{4.929899in}}{\pgfqpoint{5.678749in}{4.929899in}}%
\pgfpathcurveto{\pgfqpoint{5.667699in}{4.929899in}}{\pgfqpoint{5.657100in}{4.925509in}}{\pgfqpoint{5.649286in}{4.917695in}}%
\pgfpathcurveto{\pgfqpoint{5.641473in}{4.909881in}}{\pgfqpoint{5.637082in}{4.899282in}}{\pgfqpoint{5.637082in}{4.888232in}}%
\pgfpathcurveto{\pgfqpoint{5.637082in}{4.877182in}}{\pgfqpoint{5.641473in}{4.866583in}}{\pgfqpoint{5.649286in}{4.858769in}}%
\pgfpathcurveto{\pgfqpoint{5.657100in}{4.850956in}}{\pgfqpoint{5.667699in}{4.846566in}}{\pgfqpoint{5.678749in}{4.846566in}}%
\pgfpathclose%
\pgfusepath{stroke,fill}%
\end{pgfscope}%
\begin{pgfscope}%
\pgfpathrectangle{\pgfqpoint{0.481978in}{0.331635in}}{\pgfqpoint{9.300000in}{7.700000in}}%
\pgfusepath{clip}%
\pgfsetbuttcap%
\pgfsetroundjoin%
\definecolor{currentfill}{rgb}{0.631373,0.788235,0.956863}%
\pgfsetfillcolor{currentfill}%
\pgfsetlinewidth{0.481800pt}%
\definecolor{currentstroke}{rgb}{1.000000,1.000000,1.000000}%
\pgfsetstrokecolor{currentstroke}%
\pgfsetdash{}{0pt}%
\pgfpathmoveto{\pgfqpoint{4.278462in}{1.270944in}}%
\pgfpathcurveto{\pgfqpoint{4.289512in}{1.270944in}}{\pgfqpoint{4.300111in}{1.275334in}}{\pgfqpoint{4.307924in}{1.283148in}}%
\pgfpathcurveto{\pgfqpoint{4.315738in}{1.290962in}}{\pgfqpoint{4.320128in}{1.301561in}}{\pgfqpoint{4.320128in}{1.312611in}}%
\pgfpathcurveto{\pgfqpoint{4.320128in}{1.323661in}}{\pgfqpoint{4.315738in}{1.334260in}}{\pgfqpoint{4.307924in}{1.342074in}}%
\pgfpathcurveto{\pgfqpoint{4.300111in}{1.349887in}}{\pgfqpoint{4.289512in}{1.354277in}}{\pgfqpoint{4.278462in}{1.354277in}}%
\pgfpathcurveto{\pgfqpoint{4.267411in}{1.354277in}}{\pgfqpoint{4.256812in}{1.349887in}}{\pgfqpoint{4.248999in}{1.342074in}}%
\pgfpathcurveto{\pgfqpoint{4.241185in}{1.334260in}}{\pgfqpoint{4.236795in}{1.323661in}}{\pgfqpoint{4.236795in}{1.312611in}}%
\pgfpathcurveto{\pgfqpoint{4.236795in}{1.301561in}}{\pgfqpoint{4.241185in}{1.290962in}}{\pgfqpoint{4.248999in}{1.283148in}}%
\pgfpathcurveto{\pgfqpoint{4.256812in}{1.275334in}}{\pgfqpoint{4.267411in}{1.270944in}}{\pgfqpoint{4.278462in}{1.270944in}}%
\pgfpathclose%
\pgfusepath{stroke,fill}%
\end{pgfscope}%
\begin{pgfscope}%
\pgfpathrectangle{\pgfqpoint{0.481978in}{0.331635in}}{\pgfqpoint{9.300000in}{7.700000in}}%
\pgfusepath{clip}%
\pgfsetbuttcap%
\pgfsetroundjoin%
\definecolor{currentfill}{rgb}{0.631373,0.788235,0.956863}%
\pgfsetfillcolor{currentfill}%
\pgfsetlinewidth{0.481800pt}%
\definecolor{currentstroke}{rgb}{1.000000,1.000000,1.000000}%
\pgfsetstrokecolor{currentstroke}%
\pgfsetdash{}{0pt}%
\pgfpathmoveto{\pgfqpoint{6.739909in}{2.256314in}}%
\pgfpathcurveto{\pgfqpoint{6.750959in}{2.256314in}}{\pgfqpoint{6.761558in}{2.260704in}}{\pgfqpoint{6.769372in}{2.268518in}}%
\pgfpathcurveto{\pgfqpoint{6.777185in}{2.276331in}}{\pgfqpoint{6.781576in}{2.286930in}}{\pgfqpoint{6.781576in}{2.297981in}}%
\pgfpathcurveto{\pgfqpoint{6.781576in}{2.309031in}}{\pgfqpoint{6.777185in}{2.319630in}}{\pgfqpoint{6.769372in}{2.327443in}}%
\pgfpathcurveto{\pgfqpoint{6.761558in}{2.335257in}}{\pgfqpoint{6.750959in}{2.339647in}}{\pgfqpoint{6.739909in}{2.339647in}}%
\pgfpathcurveto{\pgfqpoint{6.728859in}{2.339647in}}{\pgfqpoint{6.718260in}{2.335257in}}{\pgfqpoint{6.710446in}{2.327443in}}%
\pgfpathcurveto{\pgfqpoint{6.702632in}{2.319630in}}{\pgfqpoint{6.698242in}{2.309031in}}{\pgfqpoint{6.698242in}{2.297981in}}%
\pgfpathcurveto{\pgfqpoint{6.698242in}{2.286930in}}{\pgfqpoint{6.702632in}{2.276331in}}{\pgfqpoint{6.710446in}{2.268518in}}%
\pgfpathcurveto{\pgfqpoint{6.718260in}{2.260704in}}{\pgfqpoint{6.728859in}{2.256314in}}{\pgfqpoint{6.739909in}{2.256314in}}%
\pgfpathclose%
\pgfusepath{stroke,fill}%
\end{pgfscope}%
\begin{pgfscope}%
\pgfpathrectangle{\pgfqpoint{0.481978in}{0.331635in}}{\pgfqpoint{9.300000in}{7.700000in}}%
\pgfusepath{clip}%
\pgfsetbuttcap%
\pgfsetroundjoin%
\definecolor{currentfill}{rgb}{0.631373,0.788235,0.956863}%
\pgfsetfillcolor{currentfill}%
\pgfsetlinewidth{0.481800pt}%
\definecolor{currentstroke}{rgb}{1.000000,1.000000,1.000000}%
\pgfsetstrokecolor{currentstroke}%
\pgfsetdash{}{0pt}%
\pgfpathmoveto{\pgfqpoint{5.837904in}{2.538586in}}%
\pgfpathcurveto{\pgfqpoint{5.848954in}{2.538586in}}{\pgfqpoint{5.859553in}{2.542976in}}{\pgfqpoint{5.867366in}{2.550790in}}%
\pgfpathcurveto{\pgfqpoint{5.875180in}{2.558603in}}{\pgfqpoint{5.879570in}{2.569202in}}{\pgfqpoint{5.879570in}{2.580252in}}%
\pgfpathcurveto{\pgfqpoint{5.879570in}{2.591302in}}{\pgfqpoint{5.875180in}{2.601901in}}{\pgfqpoint{5.867366in}{2.609715in}}%
\pgfpathcurveto{\pgfqpoint{5.859553in}{2.617529in}}{\pgfqpoint{5.848954in}{2.621919in}}{\pgfqpoint{5.837904in}{2.621919in}}%
\pgfpathcurveto{\pgfqpoint{5.826854in}{2.621919in}}{\pgfqpoint{5.816254in}{2.617529in}}{\pgfqpoint{5.808441in}{2.609715in}}%
\pgfpathcurveto{\pgfqpoint{5.800627in}{2.601901in}}{\pgfqpoint{5.796237in}{2.591302in}}{\pgfqpoint{5.796237in}{2.580252in}}%
\pgfpathcurveto{\pgfqpoint{5.796237in}{2.569202in}}{\pgfqpoint{5.800627in}{2.558603in}}{\pgfqpoint{5.808441in}{2.550790in}}%
\pgfpathcurveto{\pgfqpoint{5.816254in}{2.542976in}}{\pgfqpoint{5.826854in}{2.538586in}}{\pgfqpoint{5.837904in}{2.538586in}}%
\pgfpathclose%
\pgfusepath{stroke,fill}%
\end{pgfscope}%
\begin{pgfscope}%
\pgfpathrectangle{\pgfqpoint{0.481978in}{0.331635in}}{\pgfqpoint{9.300000in}{7.700000in}}%
\pgfusepath{clip}%
\pgfsetbuttcap%
\pgfsetroundjoin%
\definecolor{currentfill}{rgb}{0.631373,0.788235,0.956863}%
\pgfsetfillcolor{currentfill}%
\pgfsetlinewidth{0.481800pt}%
\definecolor{currentstroke}{rgb}{1.000000,1.000000,1.000000}%
\pgfsetstrokecolor{currentstroke}%
\pgfsetdash{}{0pt}%
\pgfpathmoveto{\pgfqpoint{7.682296in}{2.269395in}}%
\pgfpathcurveto{\pgfqpoint{7.693346in}{2.269395in}}{\pgfqpoint{7.703945in}{2.273785in}}{\pgfqpoint{7.711758in}{2.281598in}}%
\pgfpathcurveto{\pgfqpoint{7.719572in}{2.289412in}}{\pgfqpoint{7.723962in}{2.300011in}}{\pgfqpoint{7.723962in}{2.311061in}}%
\pgfpathcurveto{\pgfqpoint{7.723962in}{2.322111in}}{\pgfqpoint{7.719572in}{2.332710in}}{\pgfqpoint{7.711758in}{2.340524in}}%
\pgfpathcurveto{\pgfqpoint{7.703945in}{2.348338in}}{\pgfqpoint{7.693346in}{2.352728in}}{\pgfqpoint{7.682296in}{2.352728in}}%
\pgfpathcurveto{\pgfqpoint{7.671245in}{2.352728in}}{\pgfqpoint{7.660646in}{2.348338in}}{\pgfqpoint{7.652833in}{2.340524in}}%
\pgfpathcurveto{\pgfqpoint{7.645019in}{2.332710in}}{\pgfqpoint{7.640629in}{2.322111in}}{\pgfqpoint{7.640629in}{2.311061in}}%
\pgfpathcurveto{\pgfqpoint{7.640629in}{2.300011in}}{\pgfqpoint{7.645019in}{2.289412in}}{\pgfqpoint{7.652833in}{2.281598in}}%
\pgfpathcurveto{\pgfqpoint{7.660646in}{2.273785in}}{\pgfqpoint{7.671245in}{2.269395in}}{\pgfqpoint{7.682296in}{2.269395in}}%
\pgfpathclose%
\pgfusepath{stroke,fill}%
\end{pgfscope}%
\begin{pgfscope}%
\pgfpathrectangle{\pgfqpoint{0.481978in}{0.331635in}}{\pgfqpoint{9.300000in}{7.700000in}}%
\pgfusepath{clip}%
\pgfsetbuttcap%
\pgfsetroundjoin%
\definecolor{currentfill}{rgb}{0.631373,0.788235,0.956863}%
\pgfsetfillcolor{currentfill}%
\pgfsetlinewidth{0.481800pt}%
\definecolor{currentstroke}{rgb}{1.000000,1.000000,1.000000}%
\pgfsetstrokecolor{currentstroke}%
\pgfsetdash{}{0pt}%
\pgfpathmoveto{\pgfqpoint{4.017749in}{1.858138in}}%
\pgfpathcurveto{\pgfqpoint{4.028799in}{1.858138in}}{\pgfqpoint{4.039398in}{1.862528in}}{\pgfqpoint{4.047212in}{1.870342in}}%
\pgfpathcurveto{\pgfqpoint{4.055026in}{1.878155in}}{\pgfqpoint{4.059416in}{1.888754in}}{\pgfqpoint{4.059416in}{1.899805in}}%
\pgfpathcurveto{\pgfqpoint{4.059416in}{1.910855in}}{\pgfqpoint{4.055026in}{1.921454in}}{\pgfqpoint{4.047212in}{1.929267in}}%
\pgfpathcurveto{\pgfqpoint{4.039398in}{1.937081in}}{\pgfqpoint{4.028799in}{1.941471in}}{\pgfqpoint{4.017749in}{1.941471in}}%
\pgfpathcurveto{\pgfqpoint{4.006699in}{1.941471in}}{\pgfqpoint{3.996100in}{1.937081in}}{\pgfqpoint{3.988286in}{1.929267in}}%
\pgfpathcurveto{\pgfqpoint{3.980473in}{1.921454in}}{\pgfqpoint{3.976082in}{1.910855in}}{\pgfqpoint{3.976082in}{1.899805in}}%
\pgfpathcurveto{\pgfqpoint{3.976082in}{1.888754in}}{\pgfqpoint{3.980473in}{1.878155in}}{\pgfqpoint{3.988286in}{1.870342in}}%
\pgfpathcurveto{\pgfqpoint{3.996100in}{1.862528in}}{\pgfqpoint{4.006699in}{1.858138in}}{\pgfqpoint{4.017749in}{1.858138in}}%
\pgfpathclose%
\pgfusepath{stroke,fill}%
\end{pgfscope}%
\begin{pgfscope}%
\pgfpathrectangle{\pgfqpoint{0.481978in}{0.331635in}}{\pgfqpoint{9.300000in}{7.700000in}}%
\pgfusepath{clip}%
\pgfsetbuttcap%
\pgfsetroundjoin%
\definecolor{currentfill}{rgb}{0.631373,0.788235,0.956863}%
\pgfsetfillcolor{currentfill}%
\pgfsetlinewidth{0.481800pt}%
\definecolor{currentstroke}{rgb}{1.000000,1.000000,1.000000}%
\pgfsetstrokecolor{currentstroke}%
\pgfsetdash{}{0pt}%
\pgfpathmoveto{\pgfqpoint{6.127545in}{4.842798in}}%
\pgfpathcurveto{\pgfqpoint{6.138595in}{4.842798in}}{\pgfqpoint{6.149194in}{4.847189in}}{\pgfqpoint{6.157008in}{4.855002in}}%
\pgfpathcurveto{\pgfqpoint{6.164821in}{4.862816in}}{\pgfqpoint{6.169212in}{4.873415in}}{\pgfqpoint{6.169212in}{4.884465in}}%
\pgfpathcurveto{\pgfqpoint{6.169212in}{4.895515in}}{\pgfqpoint{6.164821in}{4.906114in}}{\pgfqpoint{6.157008in}{4.913928in}}%
\pgfpathcurveto{\pgfqpoint{6.149194in}{4.921741in}}{\pgfqpoint{6.138595in}{4.926132in}}{\pgfqpoint{6.127545in}{4.926132in}}%
\pgfpathcurveto{\pgfqpoint{6.116495in}{4.926132in}}{\pgfqpoint{6.105896in}{4.921741in}}{\pgfqpoint{6.098082in}{4.913928in}}%
\pgfpathcurveto{\pgfqpoint{6.090269in}{4.906114in}}{\pgfqpoint{6.085878in}{4.895515in}}{\pgfqpoint{6.085878in}{4.884465in}}%
\pgfpathcurveto{\pgfqpoint{6.085878in}{4.873415in}}{\pgfqpoint{6.090269in}{4.862816in}}{\pgfqpoint{6.098082in}{4.855002in}}%
\pgfpathcurveto{\pgfqpoint{6.105896in}{4.847189in}}{\pgfqpoint{6.116495in}{4.842798in}}{\pgfqpoint{6.127545in}{4.842798in}}%
\pgfpathclose%
\pgfusepath{stroke,fill}%
\end{pgfscope}%
\begin{pgfscope}%
\pgfpathrectangle{\pgfqpoint{0.481978in}{0.331635in}}{\pgfqpoint{9.300000in}{7.700000in}}%
\pgfusepath{clip}%
\pgfsetbuttcap%
\pgfsetroundjoin%
\definecolor{currentfill}{rgb}{0.631373,0.788235,0.956863}%
\pgfsetfillcolor{currentfill}%
\pgfsetlinewidth{0.481800pt}%
\definecolor{currentstroke}{rgb}{1.000000,1.000000,1.000000}%
\pgfsetstrokecolor{currentstroke}%
\pgfsetdash{}{0pt}%
\pgfpathmoveto{\pgfqpoint{7.971575in}{5.247803in}}%
\pgfpathcurveto{\pgfqpoint{7.982626in}{5.247803in}}{\pgfqpoint{7.993225in}{5.252193in}}{\pgfqpoint{8.001038in}{5.260007in}}%
\pgfpathcurveto{\pgfqpoint{8.008852in}{5.267821in}}{\pgfqpoint{8.013242in}{5.278420in}}{\pgfqpoint{8.013242in}{5.289470in}}%
\pgfpathcurveto{\pgfqpoint{8.013242in}{5.300520in}}{\pgfqpoint{8.008852in}{5.311119in}}{\pgfqpoint{8.001038in}{5.318933in}}%
\pgfpathcurveto{\pgfqpoint{7.993225in}{5.326746in}}{\pgfqpoint{7.982626in}{5.331136in}}{\pgfqpoint{7.971575in}{5.331136in}}%
\pgfpathcurveto{\pgfqpoint{7.960525in}{5.331136in}}{\pgfqpoint{7.949926in}{5.326746in}}{\pgfqpoint{7.942113in}{5.318933in}}%
\pgfpathcurveto{\pgfqpoint{7.934299in}{5.311119in}}{\pgfqpoint{7.929909in}{5.300520in}}{\pgfqpoint{7.929909in}{5.289470in}}%
\pgfpathcurveto{\pgfqpoint{7.929909in}{5.278420in}}{\pgfqpoint{7.934299in}{5.267821in}}{\pgfqpoint{7.942113in}{5.260007in}}%
\pgfpathcurveto{\pgfqpoint{7.949926in}{5.252193in}}{\pgfqpoint{7.960525in}{5.247803in}}{\pgfqpoint{7.971575in}{5.247803in}}%
\pgfpathclose%
\pgfusepath{stroke,fill}%
\end{pgfscope}%
\begin{pgfscope}%
\pgfpathrectangle{\pgfqpoint{0.481978in}{0.331635in}}{\pgfqpoint{9.300000in}{7.700000in}}%
\pgfusepath{clip}%
\pgfsetbuttcap%
\pgfsetroundjoin%
\definecolor{currentfill}{rgb}{0.631373,0.788235,0.956863}%
\pgfsetfillcolor{currentfill}%
\pgfsetlinewidth{0.481800pt}%
\definecolor{currentstroke}{rgb}{1.000000,1.000000,1.000000}%
\pgfsetstrokecolor{currentstroke}%
\pgfsetdash{}{0pt}%
\pgfpathmoveto{\pgfqpoint{5.508663in}{2.377917in}}%
\pgfpathcurveto{\pgfqpoint{5.519713in}{2.377917in}}{\pgfqpoint{5.530312in}{2.382307in}}{\pgfqpoint{5.538125in}{2.390121in}}%
\pgfpathcurveto{\pgfqpoint{5.545939in}{2.397934in}}{\pgfqpoint{5.550329in}{2.408533in}}{\pgfqpoint{5.550329in}{2.419584in}}%
\pgfpathcurveto{\pgfqpoint{5.550329in}{2.430634in}}{\pgfqpoint{5.545939in}{2.441233in}}{\pgfqpoint{5.538125in}{2.449046in}}%
\pgfpathcurveto{\pgfqpoint{5.530312in}{2.456860in}}{\pgfqpoint{5.519713in}{2.461250in}}{\pgfqpoint{5.508663in}{2.461250in}}%
\pgfpathcurveto{\pgfqpoint{5.497612in}{2.461250in}}{\pgfqpoint{5.487013in}{2.456860in}}{\pgfqpoint{5.479200in}{2.449046in}}%
\pgfpathcurveto{\pgfqpoint{5.471386in}{2.441233in}}{\pgfqpoint{5.466996in}{2.430634in}}{\pgfqpoint{5.466996in}{2.419584in}}%
\pgfpathcurveto{\pgfqpoint{5.466996in}{2.408533in}}{\pgfqpoint{5.471386in}{2.397934in}}{\pgfqpoint{5.479200in}{2.390121in}}%
\pgfpathcurveto{\pgfqpoint{5.487013in}{2.382307in}}{\pgfqpoint{5.497612in}{2.377917in}}{\pgfqpoint{5.508663in}{2.377917in}}%
\pgfpathclose%
\pgfusepath{stroke,fill}%
\end{pgfscope}%
\begin{pgfscope}%
\pgfpathrectangle{\pgfqpoint{0.481978in}{0.331635in}}{\pgfqpoint{9.300000in}{7.700000in}}%
\pgfusepath{clip}%
\pgfsetbuttcap%
\pgfsetroundjoin%
\definecolor{currentfill}{rgb}{0.631373,0.788235,0.956863}%
\pgfsetfillcolor{currentfill}%
\pgfsetlinewidth{0.481800pt}%
\definecolor{currentstroke}{rgb}{1.000000,1.000000,1.000000}%
\pgfsetstrokecolor{currentstroke}%
\pgfsetdash{}{0pt}%
\pgfpathmoveto{\pgfqpoint{4.730932in}{7.139660in}}%
\pgfpathcurveto{\pgfqpoint{4.741982in}{7.139660in}}{\pgfqpoint{4.752581in}{7.144050in}}{\pgfqpoint{4.760395in}{7.151864in}}%
\pgfpathcurveto{\pgfqpoint{4.768209in}{7.159677in}}{\pgfqpoint{4.772599in}{7.170276in}}{\pgfqpoint{4.772599in}{7.181326in}}%
\pgfpathcurveto{\pgfqpoint{4.772599in}{7.192377in}}{\pgfqpoint{4.768209in}{7.202976in}}{\pgfqpoint{4.760395in}{7.210789in}}%
\pgfpathcurveto{\pgfqpoint{4.752581in}{7.218603in}}{\pgfqpoint{4.741982in}{7.222993in}}{\pgfqpoint{4.730932in}{7.222993in}}%
\pgfpathcurveto{\pgfqpoint{4.719882in}{7.222993in}}{\pgfqpoint{4.709283in}{7.218603in}}{\pgfqpoint{4.701469in}{7.210789in}}%
\pgfpathcurveto{\pgfqpoint{4.693656in}{7.202976in}}{\pgfqpoint{4.689266in}{7.192377in}}{\pgfqpoint{4.689266in}{7.181326in}}%
\pgfpathcurveto{\pgfqpoint{4.689266in}{7.170276in}}{\pgfqpoint{4.693656in}{7.159677in}}{\pgfqpoint{4.701469in}{7.151864in}}%
\pgfpathcurveto{\pgfqpoint{4.709283in}{7.144050in}}{\pgfqpoint{4.719882in}{7.139660in}}{\pgfqpoint{4.730932in}{7.139660in}}%
\pgfpathclose%
\pgfusepath{stroke,fill}%
\end{pgfscope}%
\begin{pgfscope}%
\pgfpathrectangle{\pgfqpoint{0.481978in}{0.331635in}}{\pgfqpoint{9.300000in}{7.700000in}}%
\pgfusepath{clip}%
\pgfsetbuttcap%
\pgfsetroundjoin%
\definecolor{currentfill}{rgb}{0.631373,0.788235,0.956863}%
\pgfsetfillcolor{currentfill}%
\pgfsetlinewidth{0.481800pt}%
\definecolor{currentstroke}{rgb}{1.000000,1.000000,1.000000}%
\pgfsetstrokecolor{currentstroke}%
\pgfsetdash{}{0pt}%
\pgfpathmoveto{\pgfqpoint{5.337518in}{5.909868in}}%
\pgfpathcurveto{\pgfqpoint{5.348568in}{5.909868in}}{\pgfqpoint{5.359167in}{5.914259in}}{\pgfqpoint{5.366981in}{5.922072in}}%
\pgfpathcurveto{\pgfqpoint{5.374795in}{5.929886in}}{\pgfqpoint{5.379185in}{5.940485in}}{\pgfqpoint{5.379185in}{5.951535in}}%
\pgfpathcurveto{\pgfqpoint{5.379185in}{5.962585in}}{\pgfqpoint{5.374795in}{5.973184in}}{\pgfqpoint{5.366981in}{5.980998in}}%
\pgfpathcurveto{\pgfqpoint{5.359167in}{5.988811in}}{\pgfqpoint{5.348568in}{5.993202in}}{\pgfqpoint{5.337518in}{5.993202in}}%
\pgfpathcurveto{\pgfqpoint{5.326468in}{5.993202in}}{\pgfqpoint{5.315869in}{5.988811in}}{\pgfqpoint{5.308055in}{5.980998in}}%
\pgfpathcurveto{\pgfqpoint{5.300242in}{5.973184in}}{\pgfqpoint{5.295851in}{5.962585in}}{\pgfqpoint{5.295851in}{5.951535in}}%
\pgfpathcurveto{\pgfqpoint{5.295851in}{5.940485in}}{\pgfqpoint{5.300242in}{5.929886in}}{\pgfqpoint{5.308055in}{5.922072in}}%
\pgfpathcurveto{\pgfqpoint{5.315869in}{5.914259in}}{\pgfqpoint{5.326468in}{5.909868in}}{\pgfqpoint{5.337518in}{5.909868in}}%
\pgfpathclose%
\pgfusepath{stroke,fill}%
\end{pgfscope}%
\begin{pgfscope}%
\pgfpathrectangle{\pgfqpoint{0.481978in}{0.331635in}}{\pgfqpoint{9.300000in}{7.700000in}}%
\pgfusepath{clip}%
\pgfsetbuttcap%
\pgfsetroundjoin%
\definecolor{currentfill}{rgb}{0.631373,0.788235,0.956863}%
\pgfsetfillcolor{currentfill}%
\pgfsetlinewidth{0.481800pt}%
\definecolor{currentstroke}{rgb}{1.000000,1.000000,1.000000}%
\pgfsetstrokecolor{currentstroke}%
\pgfsetdash{}{0pt}%
\pgfpathmoveto{\pgfqpoint{5.165140in}{3.088557in}}%
\pgfpathcurveto{\pgfqpoint{5.176190in}{3.088557in}}{\pgfqpoint{5.186789in}{3.092947in}}{\pgfqpoint{5.194603in}{3.100761in}}%
\pgfpathcurveto{\pgfqpoint{5.202416in}{3.108574in}}{\pgfqpoint{5.206807in}{3.119173in}}{\pgfqpoint{5.206807in}{3.130223in}}%
\pgfpathcurveto{\pgfqpoint{5.206807in}{3.141273in}}{\pgfqpoint{5.202416in}{3.151872in}}{\pgfqpoint{5.194603in}{3.159686in}}%
\pgfpathcurveto{\pgfqpoint{5.186789in}{3.167500in}}{\pgfqpoint{5.176190in}{3.171890in}}{\pgfqpoint{5.165140in}{3.171890in}}%
\pgfpathcurveto{\pgfqpoint{5.154090in}{3.171890in}}{\pgfqpoint{5.143491in}{3.167500in}}{\pgfqpoint{5.135677in}{3.159686in}}%
\pgfpathcurveto{\pgfqpoint{5.127863in}{3.151872in}}{\pgfqpoint{5.123473in}{3.141273in}}{\pgfqpoint{5.123473in}{3.130223in}}%
\pgfpathcurveto{\pgfqpoint{5.123473in}{3.119173in}}{\pgfqpoint{5.127863in}{3.108574in}}{\pgfqpoint{5.135677in}{3.100761in}}%
\pgfpathcurveto{\pgfqpoint{5.143491in}{3.092947in}}{\pgfqpoint{5.154090in}{3.088557in}}{\pgfqpoint{5.165140in}{3.088557in}}%
\pgfpathclose%
\pgfusepath{stroke,fill}%
\end{pgfscope}%
\begin{pgfscope}%
\pgfpathrectangle{\pgfqpoint{0.481978in}{0.331635in}}{\pgfqpoint{9.300000in}{7.700000in}}%
\pgfusepath{clip}%
\pgfsetbuttcap%
\pgfsetroundjoin%
\definecolor{currentfill}{rgb}{0.631373,0.788235,0.956863}%
\pgfsetfillcolor{currentfill}%
\pgfsetlinewidth{0.481800pt}%
\definecolor{currentstroke}{rgb}{1.000000,1.000000,1.000000}%
\pgfsetstrokecolor{currentstroke}%
\pgfsetdash{}{0pt}%
\pgfpathmoveto{\pgfqpoint{6.635187in}{2.582474in}}%
\pgfpathcurveto{\pgfqpoint{6.646237in}{2.582474in}}{\pgfqpoint{6.656836in}{2.586864in}}{\pgfqpoint{6.664649in}{2.594678in}}%
\pgfpathcurveto{\pgfqpoint{6.672463in}{2.602492in}}{\pgfqpoint{6.676853in}{2.613091in}}{\pgfqpoint{6.676853in}{2.624141in}}%
\pgfpathcurveto{\pgfqpoint{6.676853in}{2.635191in}}{\pgfqpoint{6.672463in}{2.645790in}}{\pgfqpoint{6.664649in}{2.653603in}}%
\pgfpathcurveto{\pgfqpoint{6.656836in}{2.661417in}}{\pgfqpoint{6.646237in}{2.665807in}}{\pgfqpoint{6.635187in}{2.665807in}}%
\pgfpathcurveto{\pgfqpoint{6.624136in}{2.665807in}}{\pgfqpoint{6.613537in}{2.661417in}}{\pgfqpoint{6.605724in}{2.653603in}}%
\pgfpathcurveto{\pgfqpoint{6.597910in}{2.645790in}}{\pgfqpoint{6.593520in}{2.635191in}}{\pgfqpoint{6.593520in}{2.624141in}}%
\pgfpathcurveto{\pgfqpoint{6.593520in}{2.613091in}}{\pgfqpoint{6.597910in}{2.602492in}}{\pgfqpoint{6.605724in}{2.594678in}}%
\pgfpathcurveto{\pgfqpoint{6.613537in}{2.586864in}}{\pgfqpoint{6.624136in}{2.582474in}}{\pgfqpoint{6.635187in}{2.582474in}}%
\pgfpathclose%
\pgfusepath{stroke,fill}%
\end{pgfscope}%
\begin{pgfscope}%
\pgfpathrectangle{\pgfqpoint{0.481978in}{0.331635in}}{\pgfqpoint{9.300000in}{7.700000in}}%
\pgfusepath{clip}%
\pgfsetbuttcap%
\pgfsetroundjoin%
\definecolor{currentfill}{rgb}{0.631373,0.788235,0.956863}%
\pgfsetfillcolor{currentfill}%
\pgfsetlinewidth{0.481800pt}%
\definecolor{currentstroke}{rgb}{1.000000,1.000000,1.000000}%
\pgfsetstrokecolor{currentstroke}%
\pgfsetdash{}{0pt}%
\pgfpathmoveto{\pgfqpoint{2.269929in}{6.433789in}}%
\pgfpathcurveto{\pgfqpoint{2.280979in}{6.433789in}}{\pgfqpoint{2.291578in}{6.438180in}}{\pgfqpoint{2.299392in}{6.445993in}}%
\pgfpathcurveto{\pgfqpoint{2.307206in}{6.453807in}}{\pgfqpoint{2.311596in}{6.464406in}}{\pgfqpoint{2.311596in}{6.475456in}}%
\pgfpathcurveto{\pgfqpoint{2.311596in}{6.486506in}}{\pgfqpoint{2.307206in}{6.497105in}}{\pgfqpoint{2.299392in}{6.504919in}}%
\pgfpathcurveto{\pgfqpoint{2.291578in}{6.512732in}}{\pgfqpoint{2.280979in}{6.517123in}}{\pgfqpoint{2.269929in}{6.517123in}}%
\pgfpathcurveto{\pgfqpoint{2.258879in}{6.517123in}}{\pgfqpoint{2.248280in}{6.512732in}}{\pgfqpoint{2.240467in}{6.504919in}}%
\pgfpathcurveto{\pgfqpoint{2.232653in}{6.497105in}}{\pgfqpoint{2.228263in}{6.486506in}}{\pgfqpoint{2.228263in}{6.475456in}}%
\pgfpathcurveto{\pgfqpoint{2.228263in}{6.464406in}}{\pgfqpoint{2.232653in}{6.453807in}}{\pgfqpoint{2.240467in}{6.445993in}}%
\pgfpathcurveto{\pgfqpoint{2.248280in}{6.438180in}}{\pgfqpoint{2.258879in}{6.433789in}}{\pgfqpoint{2.269929in}{6.433789in}}%
\pgfpathclose%
\pgfusepath{stroke,fill}%
\end{pgfscope}%
\begin{pgfscope}%
\pgfpathrectangle{\pgfqpoint{0.481978in}{0.331635in}}{\pgfqpoint{9.300000in}{7.700000in}}%
\pgfusepath{clip}%
\pgfsetbuttcap%
\pgfsetroundjoin%
\definecolor{currentfill}{rgb}{0.631373,0.788235,0.956863}%
\pgfsetfillcolor{currentfill}%
\pgfsetlinewidth{0.481800pt}%
\definecolor{currentstroke}{rgb}{1.000000,1.000000,1.000000}%
\pgfsetstrokecolor{currentstroke}%
\pgfsetdash{}{0pt}%
\pgfpathmoveto{\pgfqpoint{6.041629in}{4.109351in}}%
\pgfpathcurveto{\pgfqpoint{6.052679in}{4.109351in}}{\pgfqpoint{6.063278in}{4.113742in}}{\pgfqpoint{6.071092in}{4.121555in}}%
\pgfpathcurveto{\pgfqpoint{6.078905in}{4.129369in}}{\pgfqpoint{6.083295in}{4.139968in}}{\pgfqpoint{6.083295in}{4.151018in}}%
\pgfpathcurveto{\pgfqpoint{6.083295in}{4.162068in}}{\pgfqpoint{6.078905in}{4.172667in}}{\pgfqpoint{6.071092in}{4.180481in}}%
\pgfpathcurveto{\pgfqpoint{6.063278in}{4.188294in}}{\pgfqpoint{6.052679in}{4.192685in}}{\pgfqpoint{6.041629in}{4.192685in}}%
\pgfpathcurveto{\pgfqpoint{6.030579in}{4.192685in}}{\pgfqpoint{6.019980in}{4.188294in}}{\pgfqpoint{6.012166in}{4.180481in}}%
\pgfpathcurveto{\pgfqpoint{6.004352in}{4.172667in}}{\pgfqpoint{5.999962in}{4.162068in}}{\pgfqpoint{5.999962in}{4.151018in}}%
\pgfpathcurveto{\pgfqpoint{5.999962in}{4.139968in}}{\pgfqpoint{6.004352in}{4.129369in}}{\pgfqpoint{6.012166in}{4.121555in}}%
\pgfpathcurveto{\pgfqpoint{6.019980in}{4.113742in}}{\pgfqpoint{6.030579in}{4.109351in}}{\pgfqpoint{6.041629in}{4.109351in}}%
\pgfpathclose%
\pgfusepath{stroke,fill}%
\end{pgfscope}%
\begin{pgfscope}%
\pgfpathrectangle{\pgfqpoint{0.481978in}{0.331635in}}{\pgfqpoint{9.300000in}{7.700000in}}%
\pgfusepath{clip}%
\pgfsetbuttcap%
\pgfsetroundjoin%
\definecolor{currentfill}{rgb}{0.631373,0.788235,0.956863}%
\pgfsetfillcolor{currentfill}%
\pgfsetlinewidth{0.481800pt}%
\definecolor{currentstroke}{rgb}{1.000000,1.000000,1.000000}%
\pgfsetstrokecolor{currentstroke}%
\pgfsetdash{}{0pt}%
\pgfpathmoveto{\pgfqpoint{7.965678in}{3.983142in}}%
\pgfpathcurveto{\pgfqpoint{7.976728in}{3.983142in}}{\pgfqpoint{7.987327in}{3.987532in}}{\pgfqpoint{7.995140in}{3.995346in}}%
\pgfpathcurveto{\pgfqpoint{8.002954in}{4.003160in}}{\pgfqpoint{8.007344in}{4.013759in}}{\pgfqpoint{8.007344in}{4.024809in}}%
\pgfpathcurveto{\pgfqpoint{8.007344in}{4.035859in}}{\pgfqpoint{8.002954in}{4.046458in}}{\pgfqpoint{7.995140in}{4.054272in}}%
\pgfpathcurveto{\pgfqpoint{7.987327in}{4.062085in}}{\pgfqpoint{7.976728in}{4.066476in}}{\pgfqpoint{7.965678in}{4.066476in}}%
\pgfpathcurveto{\pgfqpoint{7.954627in}{4.066476in}}{\pgfqpoint{7.944028in}{4.062085in}}{\pgfqpoint{7.936215in}{4.054272in}}%
\pgfpathcurveto{\pgfqpoint{7.928401in}{4.046458in}}{\pgfqpoint{7.924011in}{4.035859in}}{\pgfqpoint{7.924011in}{4.024809in}}%
\pgfpathcurveto{\pgfqpoint{7.924011in}{4.013759in}}{\pgfqpoint{7.928401in}{4.003160in}}{\pgfqpoint{7.936215in}{3.995346in}}%
\pgfpathcurveto{\pgfqpoint{7.944028in}{3.987532in}}{\pgfqpoint{7.954627in}{3.983142in}}{\pgfqpoint{7.965678in}{3.983142in}}%
\pgfpathclose%
\pgfusepath{stroke,fill}%
\end{pgfscope}%
\begin{pgfscope}%
\pgfpathrectangle{\pgfqpoint{0.481978in}{0.331635in}}{\pgfqpoint{9.300000in}{7.700000in}}%
\pgfusepath{clip}%
\pgfsetbuttcap%
\pgfsetroundjoin%
\definecolor{currentfill}{rgb}{0.631373,0.788235,0.956863}%
\pgfsetfillcolor{currentfill}%
\pgfsetlinewidth{0.481800pt}%
\definecolor{currentstroke}{rgb}{1.000000,1.000000,1.000000}%
\pgfsetstrokecolor{currentstroke}%
\pgfsetdash{}{0pt}%
\pgfpathmoveto{\pgfqpoint{6.739115in}{1.447662in}}%
\pgfpathcurveto{\pgfqpoint{6.750166in}{1.447662in}}{\pgfqpoint{6.760765in}{1.452052in}}{\pgfqpoint{6.768578in}{1.459866in}}%
\pgfpathcurveto{\pgfqpoint{6.776392in}{1.467680in}}{\pgfqpoint{6.780782in}{1.478279in}}{\pgfqpoint{6.780782in}{1.489329in}}%
\pgfpathcurveto{\pgfqpoint{6.780782in}{1.500379in}}{\pgfqpoint{6.776392in}{1.510978in}}{\pgfqpoint{6.768578in}{1.518792in}}%
\pgfpathcurveto{\pgfqpoint{6.760765in}{1.526605in}}{\pgfqpoint{6.750166in}{1.530995in}}{\pgfqpoint{6.739115in}{1.530995in}}%
\pgfpathcurveto{\pgfqpoint{6.728065in}{1.530995in}}{\pgfqpoint{6.717466in}{1.526605in}}{\pgfqpoint{6.709653in}{1.518792in}}%
\pgfpathcurveto{\pgfqpoint{6.701839in}{1.510978in}}{\pgfqpoint{6.697449in}{1.500379in}}{\pgfqpoint{6.697449in}{1.489329in}}%
\pgfpathcurveto{\pgfqpoint{6.697449in}{1.478279in}}{\pgfqpoint{6.701839in}{1.467680in}}{\pgfqpoint{6.709653in}{1.459866in}}%
\pgfpathcurveto{\pgfqpoint{6.717466in}{1.452052in}}{\pgfqpoint{6.728065in}{1.447662in}}{\pgfqpoint{6.739115in}{1.447662in}}%
\pgfpathclose%
\pgfusepath{stroke,fill}%
\end{pgfscope}%
\begin{pgfscope}%
\pgfpathrectangle{\pgfqpoint{0.481978in}{0.331635in}}{\pgfqpoint{9.300000in}{7.700000in}}%
\pgfusepath{clip}%
\pgfsetbuttcap%
\pgfsetroundjoin%
\definecolor{currentfill}{rgb}{0.631373,0.788235,0.956863}%
\pgfsetfillcolor{currentfill}%
\pgfsetlinewidth{0.481800pt}%
\definecolor{currentstroke}{rgb}{1.000000,1.000000,1.000000}%
\pgfsetstrokecolor{currentstroke}%
\pgfsetdash{}{0pt}%
\pgfpathmoveto{\pgfqpoint{5.908920in}{1.523605in}}%
\pgfpathcurveto{\pgfqpoint{5.919970in}{1.523605in}}{\pgfqpoint{5.930569in}{1.527995in}}{\pgfqpoint{5.938383in}{1.535809in}}%
\pgfpathcurveto{\pgfqpoint{5.946197in}{1.543622in}}{\pgfqpoint{5.950587in}{1.554221in}}{\pgfqpoint{5.950587in}{1.565271in}}%
\pgfpathcurveto{\pgfqpoint{5.950587in}{1.576322in}}{\pgfqpoint{5.946197in}{1.586921in}}{\pgfqpoint{5.938383in}{1.594734in}}%
\pgfpathcurveto{\pgfqpoint{5.930569in}{1.602548in}}{\pgfqpoint{5.919970in}{1.606938in}}{\pgfqpoint{5.908920in}{1.606938in}}%
\pgfpathcurveto{\pgfqpoint{5.897870in}{1.606938in}}{\pgfqpoint{5.887271in}{1.602548in}}{\pgfqpoint{5.879457in}{1.594734in}}%
\pgfpathcurveto{\pgfqpoint{5.871644in}{1.586921in}}{\pgfqpoint{5.867254in}{1.576322in}}{\pgfqpoint{5.867254in}{1.565271in}}%
\pgfpathcurveto{\pgfqpoint{5.867254in}{1.554221in}}{\pgfqpoint{5.871644in}{1.543622in}}{\pgfqpoint{5.879457in}{1.535809in}}%
\pgfpathcurveto{\pgfqpoint{5.887271in}{1.527995in}}{\pgfqpoint{5.897870in}{1.523605in}}{\pgfqpoint{5.908920in}{1.523605in}}%
\pgfpathclose%
\pgfusepath{stroke,fill}%
\end{pgfscope}%
\begin{pgfscope}%
\pgfpathrectangle{\pgfqpoint{0.481978in}{0.331635in}}{\pgfqpoint{9.300000in}{7.700000in}}%
\pgfusepath{clip}%
\pgfsetbuttcap%
\pgfsetroundjoin%
\definecolor{currentfill}{rgb}{0.631373,0.788235,0.956863}%
\pgfsetfillcolor{currentfill}%
\pgfsetlinewidth{0.481800pt}%
\definecolor{currentstroke}{rgb}{1.000000,1.000000,1.000000}%
\pgfsetstrokecolor{currentstroke}%
\pgfsetdash{}{0pt}%
\pgfpathmoveto{\pgfqpoint{4.782057in}{1.286961in}}%
\pgfpathcurveto{\pgfqpoint{4.793108in}{1.286961in}}{\pgfqpoint{4.803707in}{1.291352in}}{\pgfqpoint{4.811520in}{1.299165in}}%
\pgfpathcurveto{\pgfqpoint{4.819334in}{1.306979in}}{\pgfqpoint{4.823724in}{1.317578in}}{\pgfqpoint{4.823724in}{1.328628in}}%
\pgfpathcurveto{\pgfqpoint{4.823724in}{1.339678in}}{\pgfqpoint{4.819334in}{1.350277in}}{\pgfqpoint{4.811520in}{1.358091in}}%
\pgfpathcurveto{\pgfqpoint{4.803707in}{1.365904in}}{\pgfqpoint{4.793108in}{1.370295in}}{\pgfqpoint{4.782057in}{1.370295in}}%
\pgfpathcurveto{\pgfqpoint{4.771007in}{1.370295in}}{\pgfqpoint{4.760408in}{1.365904in}}{\pgfqpoint{4.752595in}{1.358091in}}%
\pgfpathcurveto{\pgfqpoint{4.744781in}{1.350277in}}{\pgfqpoint{4.740391in}{1.339678in}}{\pgfqpoint{4.740391in}{1.328628in}}%
\pgfpathcurveto{\pgfqpoint{4.740391in}{1.317578in}}{\pgfqpoint{4.744781in}{1.306979in}}{\pgfqpoint{4.752595in}{1.299165in}}%
\pgfpathcurveto{\pgfqpoint{4.760408in}{1.291352in}}{\pgfqpoint{4.771007in}{1.286961in}}{\pgfqpoint{4.782057in}{1.286961in}}%
\pgfpathclose%
\pgfusepath{stroke,fill}%
\end{pgfscope}%
\begin{pgfscope}%
\pgfpathrectangle{\pgfqpoint{0.481978in}{0.331635in}}{\pgfqpoint{9.300000in}{7.700000in}}%
\pgfusepath{clip}%
\pgfsetbuttcap%
\pgfsetroundjoin%
\definecolor{currentfill}{rgb}{0.631373,0.788235,0.956863}%
\pgfsetfillcolor{currentfill}%
\pgfsetlinewidth{0.481800pt}%
\definecolor{currentstroke}{rgb}{1.000000,1.000000,1.000000}%
\pgfsetstrokecolor{currentstroke}%
\pgfsetdash{}{0pt}%
\pgfpathmoveto{\pgfqpoint{2.762676in}{1.397155in}}%
\pgfpathcurveto{\pgfqpoint{2.773726in}{1.397155in}}{\pgfqpoint{2.784325in}{1.401546in}}{\pgfqpoint{2.792138in}{1.409359in}}%
\pgfpathcurveto{\pgfqpoint{2.799952in}{1.417173in}}{\pgfqpoint{2.804342in}{1.427772in}}{\pgfqpoint{2.804342in}{1.438822in}}%
\pgfpathcurveto{\pgfqpoint{2.804342in}{1.449872in}}{\pgfqpoint{2.799952in}{1.460471in}}{\pgfqpoint{2.792138in}{1.468285in}}%
\pgfpathcurveto{\pgfqpoint{2.784325in}{1.476099in}}{\pgfqpoint{2.773726in}{1.480489in}}{\pgfqpoint{2.762676in}{1.480489in}}%
\pgfpathcurveto{\pgfqpoint{2.751625in}{1.480489in}}{\pgfqpoint{2.741026in}{1.476099in}}{\pgfqpoint{2.733213in}{1.468285in}}%
\pgfpathcurveto{\pgfqpoint{2.725399in}{1.460471in}}{\pgfqpoint{2.721009in}{1.449872in}}{\pgfqpoint{2.721009in}{1.438822in}}%
\pgfpathcurveto{\pgfqpoint{2.721009in}{1.427772in}}{\pgfqpoint{2.725399in}{1.417173in}}{\pgfqpoint{2.733213in}{1.409359in}}%
\pgfpathcurveto{\pgfqpoint{2.741026in}{1.401546in}}{\pgfqpoint{2.751625in}{1.397155in}}{\pgfqpoint{2.762676in}{1.397155in}}%
\pgfpathclose%
\pgfusepath{stroke,fill}%
\end{pgfscope}%
\begin{pgfscope}%
\pgfpathrectangle{\pgfqpoint{0.481978in}{0.331635in}}{\pgfqpoint{9.300000in}{7.700000in}}%
\pgfusepath{clip}%
\pgfsetbuttcap%
\pgfsetroundjoin%
\definecolor{currentfill}{rgb}{0.631373,0.788235,0.956863}%
\pgfsetfillcolor{currentfill}%
\pgfsetlinewidth{0.481800pt}%
\definecolor{currentstroke}{rgb}{1.000000,1.000000,1.000000}%
\pgfsetstrokecolor{currentstroke}%
\pgfsetdash{}{0pt}%
\pgfpathmoveto{\pgfqpoint{6.674444in}{1.834291in}}%
\pgfpathcurveto{\pgfqpoint{6.685494in}{1.834291in}}{\pgfqpoint{6.696093in}{1.838681in}}{\pgfqpoint{6.703907in}{1.846495in}}%
\pgfpathcurveto{\pgfqpoint{6.711720in}{1.854308in}}{\pgfqpoint{6.716111in}{1.864908in}}{\pgfqpoint{6.716111in}{1.875958in}}%
\pgfpathcurveto{\pgfqpoint{6.716111in}{1.887008in}}{\pgfqpoint{6.711720in}{1.897607in}}{\pgfqpoint{6.703907in}{1.905420in}}%
\pgfpathcurveto{\pgfqpoint{6.696093in}{1.913234in}}{\pgfqpoint{6.685494in}{1.917624in}}{\pgfqpoint{6.674444in}{1.917624in}}%
\pgfpathcurveto{\pgfqpoint{6.663394in}{1.917624in}}{\pgfqpoint{6.652795in}{1.913234in}}{\pgfqpoint{6.644981in}{1.905420in}}%
\pgfpathcurveto{\pgfqpoint{6.637167in}{1.897607in}}{\pgfqpoint{6.632777in}{1.887008in}}{\pgfqpoint{6.632777in}{1.875958in}}%
\pgfpathcurveto{\pgfqpoint{6.632777in}{1.864908in}}{\pgfqpoint{6.637167in}{1.854308in}}{\pgfqpoint{6.644981in}{1.846495in}}%
\pgfpathcurveto{\pgfqpoint{6.652795in}{1.838681in}}{\pgfqpoint{6.663394in}{1.834291in}}{\pgfqpoint{6.674444in}{1.834291in}}%
\pgfpathclose%
\pgfusepath{stroke,fill}%
\end{pgfscope}%
\begin{pgfscope}%
\pgfpathrectangle{\pgfqpoint{0.481978in}{0.331635in}}{\pgfqpoint{9.300000in}{7.700000in}}%
\pgfusepath{clip}%
\pgfsetbuttcap%
\pgfsetroundjoin%
\definecolor{currentfill}{rgb}{0.631373,0.788235,0.956863}%
\pgfsetfillcolor{currentfill}%
\pgfsetlinewidth{0.481800pt}%
\definecolor{currentstroke}{rgb}{1.000000,1.000000,1.000000}%
\pgfsetstrokecolor{currentstroke}%
\pgfsetdash{}{0pt}%
\pgfpathmoveto{\pgfqpoint{7.190919in}{2.075647in}}%
\pgfpathcurveto{\pgfqpoint{7.201969in}{2.075647in}}{\pgfqpoint{7.212568in}{2.080038in}}{\pgfqpoint{7.220381in}{2.087851in}}%
\pgfpathcurveto{\pgfqpoint{7.228195in}{2.095665in}}{\pgfqpoint{7.232585in}{2.106264in}}{\pgfqpoint{7.232585in}{2.117314in}}%
\pgfpathcurveto{\pgfqpoint{7.232585in}{2.128364in}}{\pgfqpoint{7.228195in}{2.138963in}}{\pgfqpoint{7.220381in}{2.146777in}}%
\pgfpathcurveto{\pgfqpoint{7.212568in}{2.154591in}}{\pgfqpoint{7.201969in}{2.158981in}}{\pgfqpoint{7.190919in}{2.158981in}}%
\pgfpathcurveto{\pgfqpoint{7.179868in}{2.158981in}}{\pgfqpoint{7.169269in}{2.154591in}}{\pgfqpoint{7.161456in}{2.146777in}}%
\pgfpathcurveto{\pgfqpoint{7.153642in}{2.138963in}}{\pgfqpoint{7.149252in}{2.128364in}}{\pgfqpoint{7.149252in}{2.117314in}}%
\pgfpathcurveto{\pgfqpoint{7.149252in}{2.106264in}}{\pgfqpoint{7.153642in}{2.095665in}}{\pgfqpoint{7.161456in}{2.087851in}}%
\pgfpathcurveto{\pgfqpoint{7.169269in}{2.080038in}}{\pgfqpoint{7.179868in}{2.075647in}}{\pgfqpoint{7.190919in}{2.075647in}}%
\pgfpathclose%
\pgfusepath{stroke,fill}%
\end{pgfscope}%
\begin{pgfscope}%
\pgfpathrectangle{\pgfqpoint{0.481978in}{0.331635in}}{\pgfqpoint{9.300000in}{7.700000in}}%
\pgfusepath{clip}%
\pgfsetbuttcap%
\pgfsetroundjoin%
\definecolor{currentfill}{rgb}{0.631373,0.788235,0.956863}%
\pgfsetfillcolor{currentfill}%
\pgfsetlinewidth{0.481800pt}%
\definecolor{currentstroke}{rgb}{1.000000,1.000000,1.000000}%
\pgfsetstrokecolor{currentstroke}%
\pgfsetdash{}{0pt}%
\pgfpathmoveto{\pgfqpoint{5.231440in}{6.850407in}}%
\pgfpathcurveto{\pgfqpoint{5.242490in}{6.850407in}}{\pgfqpoint{5.253089in}{6.854797in}}{\pgfqpoint{5.260903in}{6.862611in}}%
\pgfpathcurveto{\pgfqpoint{5.268716in}{6.870424in}}{\pgfqpoint{5.273107in}{6.881023in}}{\pgfqpoint{5.273107in}{6.892073in}}%
\pgfpathcurveto{\pgfqpoint{5.273107in}{6.903123in}}{\pgfqpoint{5.268716in}{6.913722in}}{\pgfqpoint{5.260903in}{6.921536in}}%
\pgfpathcurveto{\pgfqpoint{5.253089in}{6.929350in}}{\pgfqpoint{5.242490in}{6.933740in}}{\pgfqpoint{5.231440in}{6.933740in}}%
\pgfpathcurveto{\pgfqpoint{5.220390in}{6.933740in}}{\pgfqpoint{5.209791in}{6.929350in}}{\pgfqpoint{5.201977in}{6.921536in}}%
\pgfpathcurveto{\pgfqpoint{5.194164in}{6.913722in}}{\pgfqpoint{5.189773in}{6.903123in}}{\pgfqpoint{5.189773in}{6.892073in}}%
\pgfpathcurveto{\pgfqpoint{5.189773in}{6.881023in}}{\pgfqpoint{5.194164in}{6.870424in}}{\pgfqpoint{5.201977in}{6.862611in}}%
\pgfpathcurveto{\pgfqpoint{5.209791in}{6.854797in}}{\pgfqpoint{5.220390in}{6.850407in}}{\pgfqpoint{5.231440in}{6.850407in}}%
\pgfpathclose%
\pgfusepath{stroke,fill}%
\end{pgfscope}%
\begin{pgfscope}%
\pgfpathrectangle{\pgfqpoint{0.481978in}{0.331635in}}{\pgfqpoint{9.300000in}{7.700000in}}%
\pgfusepath{clip}%
\pgfsetbuttcap%
\pgfsetroundjoin%
\definecolor{currentfill}{rgb}{0.631373,0.788235,0.956863}%
\pgfsetfillcolor{currentfill}%
\pgfsetlinewidth{0.481800pt}%
\definecolor{currentstroke}{rgb}{1.000000,1.000000,1.000000}%
\pgfsetstrokecolor{currentstroke}%
\pgfsetdash{}{0pt}%
\pgfpathmoveto{\pgfqpoint{7.641158in}{4.350034in}}%
\pgfpathcurveto{\pgfqpoint{7.652208in}{4.350034in}}{\pgfqpoint{7.662807in}{4.354424in}}{\pgfqpoint{7.670621in}{4.362237in}}%
\pgfpathcurveto{\pgfqpoint{7.678434in}{4.370051in}}{\pgfqpoint{7.682825in}{4.380650in}}{\pgfqpoint{7.682825in}{4.391700in}}%
\pgfpathcurveto{\pgfqpoint{7.682825in}{4.402750in}}{\pgfqpoint{7.678434in}{4.413349in}}{\pgfqpoint{7.670621in}{4.421163in}}%
\pgfpathcurveto{\pgfqpoint{7.662807in}{4.428977in}}{\pgfqpoint{7.652208in}{4.433367in}}{\pgfqpoint{7.641158in}{4.433367in}}%
\pgfpathcurveto{\pgfqpoint{7.630108in}{4.433367in}}{\pgfqpoint{7.619509in}{4.428977in}}{\pgfqpoint{7.611695in}{4.421163in}}%
\pgfpathcurveto{\pgfqpoint{7.603882in}{4.413349in}}{\pgfqpoint{7.599491in}{4.402750in}}{\pgfqpoint{7.599491in}{4.391700in}}%
\pgfpathcurveto{\pgfqpoint{7.599491in}{4.380650in}}{\pgfqpoint{7.603882in}{4.370051in}}{\pgfqpoint{7.611695in}{4.362237in}}%
\pgfpathcurveto{\pgfqpoint{7.619509in}{4.354424in}}{\pgfqpoint{7.630108in}{4.350034in}}{\pgfqpoint{7.641158in}{4.350034in}}%
\pgfpathclose%
\pgfusepath{stroke,fill}%
\end{pgfscope}%
\begin{pgfscope}%
\pgfpathrectangle{\pgfqpoint{0.481978in}{0.331635in}}{\pgfqpoint{9.300000in}{7.700000in}}%
\pgfusepath{clip}%
\pgfsetbuttcap%
\pgfsetroundjoin%
\definecolor{currentfill}{rgb}{0.631373,0.788235,0.956863}%
\pgfsetfillcolor{currentfill}%
\pgfsetlinewidth{0.481800pt}%
\definecolor{currentstroke}{rgb}{1.000000,1.000000,1.000000}%
\pgfsetstrokecolor{currentstroke}%
\pgfsetdash{}{0pt}%
\pgfpathmoveto{\pgfqpoint{5.917655in}{4.335690in}}%
\pgfpathcurveto{\pgfqpoint{5.928705in}{4.335690in}}{\pgfqpoint{5.939305in}{4.340080in}}{\pgfqpoint{5.947118in}{4.347894in}}%
\pgfpathcurveto{\pgfqpoint{5.954932in}{4.355707in}}{\pgfqpoint{5.959322in}{4.366306in}}{\pgfqpoint{5.959322in}{4.377356in}}%
\pgfpathcurveto{\pgfqpoint{5.959322in}{4.388407in}}{\pgfqpoint{5.954932in}{4.399006in}}{\pgfqpoint{5.947118in}{4.406819in}}%
\pgfpathcurveto{\pgfqpoint{5.939305in}{4.414633in}}{\pgfqpoint{5.928705in}{4.419023in}}{\pgfqpoint{5.917655in}{4.419023in}}%
\pgfpathcurveto{\pgfqpoint{5.906605in}{4.419023in}}{\pgfqpoint{5.896006in}{4.414633in}}{\pgfqpoint{5.888193in}{4.406819in}}%
\pgfpathcurveto{\pgfqpoint{5.880379in}{4.399006in}}{\pgfqpoint{5.875989in}{4.388407in}}{\pgfqpoint{5.875989in}{4.377356in}}%
\pgfpathcurveto{\pgfqpoint{5.875989in}{4.366306in}}{\pgfqpoint{5.880379in}{4.355707in}}{\pgfqpoint{5.888193in}{4.347894in}}%
\pgfpathcurveto{\pgfqpoint{5.896006in}{4.340080in}}{\pgfqpoint{5.906605in}{4.335690in}}{\pgfqpoint{5.917655in}{4.335690in}}%
\pgfpathclose%
\pgfusepath{stroke,fill}%
\end{pgfscope}%
\begin{pgfscope}%
\pgfpathrectangle{\pgfqpoint{0.481978in}{0.331635in}}{\pgfqpoint{9.300000in}{7.700000in}}%
\pgfusepath{clip}%
\pgfsetbuttcap%
\pgfsetroundjoin%
\definecolor{currentfill}{rgb}{0.631373,0.788235,0.956863}%
\pgfsetfillcolor{currentfill}%
\pgfsetlinewidth{0.481800pt}%
\definecolor{currentstroke}{rgb}{1.000000,1.000000,1.000000}%
\pgfsetstrokecolor{currentstroke}%
\pgfsetdash{}{0pt}%
\pgfpathmoveto{\pgfqpoint{6.915293in}{3.926889in}}%
\pgfpathcurveto{\pgfqpoint{6.926343in}{3.926889in}}{\pgfqpoint{6.936942in}{3.931279in}}{\pgfqpoint{6.944756in}{3.939093in}}%
\pgfpathcurveto{\pgfqpoint{6.952570in}{3.946906in}}{\pgfqpoint{6.956960in}{3.957505in}}{\pgfqpoint{6.956960in}{3.968555in}}%
\pgfpathcurveto{\pgfqpoint{6.956960in}{3.979605in}}{\pgfqpoint{6.952570in}{3.990204in}}{\pgfqpoint{6.944756in}{3.998018in}}%
\pgfpathcurveto{\pgfqpoint{6.936942in}{4.005832in}}{\pgfqpoint{6.926343in}{4.010222in}}{\pgfqpoint{6.915293in}{4.010222in}}%
\pgfpathcurveto{\pgfqpoint{6.904243in}{4.010222in}}{\pgfqpoint{6.893644in}{4.005832in}}{\pgfqpoint{6.885830in}{3.998018in}}%
\pgfpathcurveto{\pgfqpoint{6.878017in}{3.990204in}}{\pgfqpoint{6.873627in}{3.979605in}}{\pgfqpoint{6.873627in}{3.968555in}}%
\pgfpathcurveto{\pgfqpoint{6.873627in}{3.957505in}}{\pgfqpoint{6.878017in}{3.946906in}}{\pgfqpoint{6.885830in}{3.939093in}}%
\pgfpathcurveto{\pgfqpoint{6.893644in}{3.931279in}}{\pgfqpoint{6.904243in}{3.926889in}}{\pgfqpoint{6.915293in}{3.926889in}}%
\pgfpathclose%
\pgfusepath{stroke,fill}%
\end{pgfscope}%
\begin{pgfscope}%
\pgfpathrectangle{\pgfqpoint{0.481978in}{0.331635in}}{\pgfqpoint{9.300000in}{7.700000in}}%
\pgfusepath{clip}%
\pgfsetbuttcap%
\pgfsetroundjoin%
\definecolor{currentfill}{rgb}{0.631373,0.788235,0.956863}%
\pgfsetfillcolor{currentfill}%
\pgfsetlinewidth{0.481800pt}%
\definecolor{currentstroke}{rgb}{1.000000,1.000000,1.000000}%
\pgfsetstrokecolor{currentstroke}%
\pgfsetdash{}{0pt}%
\pgfpathmoveto{\pgfqpoint{5.971393in}{2.344772in}}%
\pgfpathcurveto{\pgfqpoint{5.982443in}{2.344772in}}{\pgfqpoint{5.993042in}{2.349163in}}{\pgfqpoint{6.000856in}{2.356976in}}%
\pgfpathcurveto{\pgfqpoint{6.008669in}{2.364790in}}{\pgfqpoint{6.013060in}{2.375389in}}{\pgfqpoint{6.013060in}{2.386439in}}%
\pgfpathcurveto{\pgfqpoint{6.013060in}{2.397489in}}{\pgfqpoint{6.008669in}{2.408088in}}{\pgfqpoint{6.000856in}{2.415902in}}%
\pgfpathcurveto{\pgfqpoint{5.993042in}{2.423715in}}{\pgfqpoint{5.982443in}{2.428106in}}{\pgfqpoint{5.971393in}{2.428106in}}%
\pgfpathcurveto{\pgfqpoint{5.960343in}{2.428106in}}{\pgfqpoint{5.949744in}{2.423715in}}{\pgfqpoint{5.941930in}{2.415902in}}%
\pgfpathcurveto{\pgfqpoint{5.934117in}{2.408088in}}{\pgfqpoint{5.929726in}{2.397489in}}{\pgfqpoint{5.929726in}{2.386439in}}%
\pgfpathcurveto{\pgfqpoint{5.929726in}{2.375389in}}{\pgfqpoint{5.934117in}{2.364790in}}{\pgfqpoint{5.941930in}{2.356976in}}%
\pgfpathcurveto{\pgfqpoint{5.949744in}{2.349163in}}{\pgfqpoint{5.960343in}{2.344772in}}{\pgfqpoint{5.971393in}{2.344772in}}%
\pgfpathclose%
\pgfusepath{stroke,fill}%
\end{pgfscope}%
\begin{pgfscope}%
\pgfpathrectangle{\pgfqpoint{0.481978in}{0.331635in}}{\pgfqpoint{9.300000in}{7.700000in}}%
\pgfusepath{clip}%
\pgfsetbuttcap%
\pgfsetroundjoin%
\definecolor{currentfill}{rgb}{0.631373,0.788235,0.956863}%
\pgfsetfillcolor{currentfill}%
\pgfsetlinewidth{0.481800pt}%
\definecolor{currentstroke}{rgb}{1.000000,1.000000,1.000000}%
\pgfsetstrokecolor{currentstroke}%
\pgfsetdash{}{0pt}%
\pgfpathmoveto{\pgfqpoint{2.928552in}{6.577600in}}%
\pgfpathcurveto{\pgfqpoint{2.939602in}{6.577600in}}{\pgfqpoint{2.950201in}{6.581990in}}{\pgfqpoint{2.958014in}{6.589804in}}%
\pgfpathcurveto{\pgfqpoint{2.965828in}{6.597618in}}{\pgfqpoint{2.970218in}{6.608217in}}{\pgfqpoint{2.970218in}{6.619267in}}%
\pgfpathcurveto{\pgfqpoint{2.970218in}{6.630317in}}{\pgfqpoint{2.965828in}{6.640916in}}{\pgfqpoint{2.958014in}{6.648730in}}%
\pgfpathcurveto{\pgfqpoint{2.950201in}{6.656543in}}{\pgfqpoint{2.939602in}{6.660933in}}{\pgfqpoint{2.928552in}{6.660933in}}%
\pgfpathcurveto{\pgfqpoint{2.917501in}{6.660933in}}{\pgfqpoint{2.906902in}{6.656543in}}{\pgfqpoint{2.899089in}{6.648730in}}%
\pgfpathcurveto{\pgfqpoint{2.891275in}{6.640916in}}{\pgfqpoint{2.886885in}{6.630317in}}{\pgfqpoint{2.886885in}{6.619267in}}%
\pgfpathcurveto{\pgfqpoint{2.886885in}{6.608217in}}{\pgfqpoint{2.891275in}{6.597618in}}{\pgfqpoint{2.899089in}{6.589804in}}%
\pgfpathcurveto{\pgfqpoint{2.906902in}{6.581990in}}{\pgfqpoint{2.917501in}{6.577600in}}{\pgfqpoint{2.928552in}{6.577600in}}%
\pgfpathclose%
\pgfusepath{stroke,fill}%
\end{pgfscope}%
\begin{pgfscope}%
\pgfpathrectangle{\pgfqpoint{0.481978in}{0.331635in}}{\pgfqpoint{9.300000in}{7.700000in}}%
\pgfusepath{clip}%
\pgfsetbuttcap%
\pgfsetroundjoin%
\definecolor{currentfill}{rgb}{0.631373,0.788235,0.956863}%
\pgfsetfillcolor{currentfill}%
\pgfsetlinewidth{0.481800pt}%
\definecolor{currentstroke}{rgb}{1.000000,1.000000,1.000000}%
\pgfsetstrokecolor{currentstroke}%
\pgfsetdash{}{0pt}%
\pgfpathmoveto{\pgfqpoint{5.743990in}{2.572575in}}%
\pgfpathcurveto{\pgfqpoint{5.755041in}{2.572575in}}{\pgfqpoint{5.765640in}{2.576965in}}{\pgfqpoint{5.773453in}{2.584779in}}%
\pgfpathcurveto{\pgfqpoint{5.781267in}{2.592593in}}{\pgfqpoint{5.785657in}{2.603192in}}{\pgfqpoint{5.785657in}{2.614242in}}%
\pgfpathcurveto{\pgfqpoint{5.785657in}{2.625292in}}{\pgfqpoint{5.781267in}{2.635891in}}{\pgfqpoint{5.773453in}{2.643705in}}%
\pgfpathcurveto{\pgfqpoint{5.765640in}{2.651518in}}{\pgfqpoint{5.755041in}{2.655909in}}{\pgfqpoint{5.743990in}{2.655909in}}%
\pgfpathcurveto{\pgfqpoint{5.732940in}{2.655909in}}{\pgfqpoint{5.722341in}{2.651518in}}{\pgfqpoint{5.714528in}{2.643705in}}%
\pgfpathcurveto{\pgfqpoint{5.706714in}{2.635891in}}{\pgfqpoint{5.702324in}{2.625292in}}{\pgfqpoint{5.702324in}{2.614242in}}%
\pgfpathcurveto{\pgfqpoint{5.702324in}{2.603192in}}{\pgfqpoint{5.706714in}{2.592593in}}{\pgfqpoint{5.714528in}{2.584779in}}%
\pgfpathcurveto{\pgfqpoint{5.722341in}{2.576965in}}{\pgfqpoint{5.732940in}{2.572575in}}{\pgfqpoint{5.743990in}{2.572575in}}%
\pgfpathclose%
\pgfusepath{stroke,fill}%
\end{pgfscope}%
\begin{pgfscope}%
\pgfpathrectangle{\pgfqpoint{0.481978in}{0.331635in}}{\pgfqpoint{9.300000in}{7.700000in}}%
\pgfusepath{clip}%
\pgfsetbuttcap%
\pgfsetroundjoin%
\definecolor{currentfill}{rgb}{0.631373,0.788235,0.956863}%
\pgfsetfillcolor{currentfill}%
\pgfsetlinewidth{0.481800pt}%
\definecolor{currentstroke}{rgb}{1.000000,1.000000,1.000000}%
\pgfsetstrokecolor{currentstroke}%
\pgfsetdash{}{0pt}%
\pgfpathmoveto{\pgfqpoint{8.326824in}{4.882127in}}%
\pgfpathcurveto{\pgfqpoint{8.337874in}{4.882127in}}{\pgfqpoint{8.348473in}{4.886517in}}{\pgfqpoint{8.356287in}{4.894331in}}%
\pgfpathcurveto{\pgfqpoint{8.364100in}{4.902144in}}{\pgfqpoint{8.368491in}{4.912743in}}{\pgfqpoint{8.368491in}{4.923794in}}%
\pgfpathcurveto{\pgfqpoint{8.368491in}{4.934844in}}{\pgfqpoint{8.364100in}{4.945443in}}{\pgfqpoint{8.356287in}{4.953256in}}%
\pgfpathcurveto{\pgfqpoint{8.348473in}{4.961070in}}{\pgfqpoint{8.337874in}{4.965460in}}{\pgfqpoint{8.326824in}{4.965460in}}%
\pgfpathcurveto{\pgfqpoint{8.315774in}{4.965460in}}{\pgfqpoint{8.305175in}{4.961070in}}{\pgfqpoint{8.297361in}{4.953256in}}%
\pgfpathcurveto{\pgfqpoint{8.289548in}{4.945443in}}{\pgfqpoint{8.285157in}{4.934844in}}{\pgfqpoint{8.285157in}{4.923794in}}%
\pgfpathcurveto{\pgfqpoint{8.285157in}{4.912743in}}{\pgfqpoint{8.289548in}{4.902144in}}{\pgfqpoint{8.297361in}{4.894331in}}%
\pgfpathcurveto{\pgfqpoint{8.305175in}{4.886517in}}{\pgfqpoint{8.315774in}{4.882127in}}{\pgfqpoint{8.326824in}{4.882127in}}%
\pgfpathclose%
\pgfusepath{stroke,fill}%
\end{pgfscope}%
\begin{pgfscope}%
\pgfpathrectangle{\pgfqpoint{0.481978in}{0.331635in}}{\pgfqpoint{9.300000in}{7.700000in}}%
\pgfusepath{clip}%
\pgfsetbuttcap%
\pgfsetroundjoin%
\definecolor{currentfill}{rgb}{0.631373,0.788235,0.956863}%
\pgfsetfillcolor{currentfill}%
\pgfsetlinewidth{0.481800pt}%
\definecolor{currentstroke}{rgb}{1.000000,1.000000,1.000000}%
\pgfsetstrokecolor{currentstroke}%
\pgfsetdash{}{0pt}%
\pgfpathmoveto{\pgfqpoint{4.333517in}{5.742085in}}%
\pgfpathcurveto{\pgfqpoint{4.344567in}{5.742085in}}{\pgfqpoint{4.355166in}{5.746475in}}{\pgfqpoint{4.362979in}{5.754289in}}%
\pgfpathcurveto{\pgfqpoint{4.370793in}{5.762102in}}{\pgfqpoint{4.375183in}{5.772701in}}{\pgfqpoint{4.375183in}{5.783751in}}%
\pgfpathcurveto{\pgfqpoint{4.375183in}{5.794801in}}{\pgfqpoint{4.370793in}{5.805400in}}{\pgfqpoint{4.362979in}{5.813214in}}%
\pgfpathcurveto{\pgfqpoint{4.355166in}{5.821028in}}{\pgfqpoint{4.344567in}{5.825418in}}{\pgfqpoint{4.333517in}{5.825418in}}%
\pgfpathcurveto{\pgfqpoint{4.322466in}{5.825418in}}{\pgfqpoint{4.311867in}{5.821028in}}{\pgfqpoint{4.304054in}{5.813214in}}%
\pgfpathcurveto{\pgfqpoint{4.296240in}{5.805400in}}{\pgfqpoint{4.291850in}{5.794801in}}{\pgfqpoint{4.291850in}{5.783751in}}%
\pgfpathcurveto{\pgfqpoint{4.291850in}{5.772701in}}{\pgfqpoint{4.296240in}{5.762102in}}{\pgfqpoint{4.304054in}{5.754289in}}%
\pgfpathcurveto{\pgfqpoint{4.311867in}{5.746475in}}{\pgfqpoint{4.322466in}{5.742085in}}{\pgfqpoint{4.333517in}{5.742085in}}%
\pgfpathclose%
\pgfusepath{stroke,fill}%
\end{pgfscope}%
\begin{pgfscope}%
\pgfpathrectangle{\pgfqpoint{0.481978in}{0.331635in}}{\pgfqpoint{9.300000in}{7.700000in}}%
\pgfusepath{clip}%
\pgfsetbuttcap%
\pgfsetroundjoin%
\definecolor{currentfill}{rgb}{0.631373,0.788235,0.956863}%
\pgfsetfillcolor{currentfill}%
\pgfsetlinewidth{0.481800pt}%
\definecolor{currentstroke}{rgb}{1.000000,1.000000,1.000000}%
\pgfsetstrokecolor{currentstroke}%
\pgfsetdash{}{0pt}%
\pgfpathmoveto{\pgfqpoint{5.146855in}{2.018893in}}%
\pgfpathcurveto{\pgfqpoint{5.157905in}{2.018893in}}{\pgfqpoint{5.168504in}{2.023284in}}{\pgfqpoint{5.176317in}{2.031097in}}%
\pgfpathcurveto{\pgfqpoint{5.184131in}{2.038911in}}{\pgfqpoint{5.188521in}{2.049510in}}{\pgfqpoint{5.188521in}{2.060560in}}%
\pgfpathcurveto{\pgfqpoint{5.188521in}{2.071610in}}{\pgfqpoint{5.184131in}{2.082209in}}{\pgfqpoint{5.176317in}{2.090023in}}%
\pgfpathcurveto{\pgfqpoint{5.168504in}{2.097836in}}{\pgfqpoint{5.157905in}{2.102227in}}{\pgfqpoint{5.146855in}{2.102227in}}%
\pgfpathcurveto{\pgfqpoint{5.135805in}{2.102227in}}{\pgfqpoint{5.125206in}{2.097836in}}{\pgfqpoint{5.117392in}{2.090023in}}%
\pgfpathcurveto{\pgfqpoint{5.109578in}{2.082209in}}{\pgfqpoint{5.105188in}{2.071610in}}{\pgfqpoint{5.105188in}{2.060560in}}%
\pgfpathcurveto{\pgfqpoint{5.105188in}{2.049510in}}{\pgfqpoint{5.109578in}{2.038911in}}{\pgfqpoint{5.117392in}{2.031097in}}%
\pgfpathcurveto{\pgfqpoint{5.125206in}{2.023284in}}{\pgfqpoint{5.135805in}{2.018893in}}{\pgfqpoint{5.146855in}{2.018893in}}%
\pgfpathclose%
\pgfusepath{stroke,fill}%
\end{pgfscope}%
\begin{pgfscope}%
\pgfpathrectangle{\pgfqpoint{0.481978in}{0.331635in}}{\pgfqpoint{9.300000in}{7.700000in}}%
\pgfusepath{clip}%
\pgfsetbuttcap%
\pgfsetroundjoin%
\definecolor{currentfill}{rgb}{0.631373,0.788235,0.956863}%
\pgfsetfillcolor{currentfill}%
\pgfsetlinewidth{0.481800pt}%
\definecolor{currentstroke}{rgb}{1.000000,1.000000,1.000000}%
\pgfsetstrokecolor{currentstroke}%
\pgfsetdash{}{0pt}%
\pgfpathmoveto{\pgfqpoint{6.726675in}{5.368676in}}%
\pgfpathcurveto{\pgfqpoint{6.737725in}{5.368676in}}{\pgfqpoint{6.748324in}{5.373066in}}{\pgfqpoint{6.756138in}{5.380880in}}%
\pgfpathcurveto{\pgfqpoint{6.763951in}{5.388694in}}{\pgfqpoint{6.768342in}{5.399293in}}{\pgfqpoint{6.768342in}{5.410343in}}%
\pgfpathcurveto{\pgfqpoint{6.768342in}{5.421393in}}{\pgfqpoint{6.763951in}{5.431992in}}{\pgfqpoint{6.756138in}{5.439806in}}%
\pgfpathcurveto{\pgfqpoint{6.748324in}{5.447619in}}{\pgfqpoint{6.737725in}{5.452009in}}{\pgfqpoint{6.726675in}{5.452009in}}%
\pgfpathcurveto{\pgfqpoint{6.715625in}{5.452009in}}{\pgfqpoint{6.705026in}{5.447619in}}{\pgfqpoint{6.697212in}{5.439806in}}%
\pgfpathcurveto{\pgfqpoint{6.689399in}{5.431992in}}{\pgfqpoint{6.685008in}{5.421393in}}{\pgfqpoint{6.685008in}{5.410343in}}%
\pgfpathcurveto{\pgfqpoint{6.685008in}{5.399293in}}{\pgfqpoint{6.689399in}{5.388694in}}{\pgfqpoint{6.697212in}{5.380880in}}%
\pgfpathcurveto{\pgfqpoint{6.705026in}{5.373066in}}{\pgfqpoint{6.715625in}{5.368676in}}{\pgfqpoint{6.726675in}{5.368676in}}%
\pgfpathclose%
\pgfusepath{stroke,fill}%
\end{pgfscope}%
\begin{pgfscope}%
\pgfpathrectangle{\pgfqpoint{0.481978in}{0.331635in}}{\pgfqpoint{9.300000in}{7.700000in}}%
\pgfusepath{clip}%
\pgfsetbuttcap%
\pgfsetroundjoin%
\definecolor{currentfill}{rgb}{0.631373,0.788235,0.956863}%
\pgfsetfillcolor{currentfill}%
\pgfsetlinewidth{0.481800pt}%
\definecolor{currentstroke}{rgb}{1.000000,1.000000,1.000000}%
\pgfsetstrokecolor{currentstroke}%
\pgfsetdash{}{0pt}%
\pgfpathmoveto{\pgfqpoint{5.444209in}{2.533371in}}%
\pgfpathcurveto{\pgfqpoint{5.455260in}{2.533371in}}{\pgfqpoint{5.465859in}{2.537761in}}{\pgfqpoint{5.473672in}{2.545575in}}%
\pgfpathcurveto{\pgfqpoint{5.481486in}{2.553389in}}{\pgfqpoint{5.485876in}{2.563988in}}{\pgfqpoint{5.485876in}{2.575038in}}%
\pgfpathcurveto{\pgfqpoint{5.485876in}{2.586088in}}{\pgfqpoint{5.481486in}{2.596687in}}{\pgfqpoint{5.473672in}{2.604501in}}%
\pgfpathcurveto{\pgfqpoint{5.465859in}{2.612314in}}{\pgfqpoint{5.455260in}{2.616705in}}{\pgfqpoint{5.444209in}{2.616705in}}%
\pgfpathcurveto{\pgfqpoint{5.433159in}{2.616705in}}{\pgfqpoint{5.422560in}{2.612314in}}{\pgfqpoint{5.414747in}{2.604501in}}%
\pgfpathcurveto{\pgfqpoint{5.406933in}{2.596687in}}{\pgfqpoint{5.402543in}{2.586088in}}{\pgfqpoint{5.402543in}{2.575038in}}%
\pgfpathcurveto{\pgfqpoint{5.402543in}{2.563988in}}{\pgfqpoint{5.406933in}{2.553389in}}{\pgfqpoint{5.414747in}{2.545575in}}%
\pgfpathcurveto{\pgfqpoint{5.422560in}{2.537761in}}{\pgfqpoint{5.433159in}{2.533371in}}{\pgfqpoint{5.444209in}{2.533371in}}%
\pgfpathclose%
\pgfusepath{stroke,fill}%
\end{pgfscope}%
\begin{pgfscope}%
\pgfpathrectangle{\pgfqpoint{0.481978in}{0.331635in}}{\pgfqpoint{9.300000in}{7.700000in}}%
\pgfusepath{clip}%
\pgfsetbuttcap%
\pgfsetroundjoin%
\definecolor{currentfill}{rgb}{0.631373,0.788235,0.956863}%
\pgfsetfillcolor{currentfill}%
\pgfsetlinewidth{0.481800pt}%
\definecolor{currentstroke}{rgb}{1.000000,1.000000,1.000000}%
\pgfsetstrokecolor{currentstroke}%
\pgfsetdash{}{0pt}%
\pgfpathmoveto{\pgfqpoint{6.317681in}{2.742307in}}%
\pgfpathcurveto{\pgfqpoint{6.328731in}{2.742307in}}{\pgfqpoint{6.339330in}{2.746697in}}{\pgfqpoint{6.347143in}{2.754511in}}%
\pgfpathcurveto{\pgfqpoint{6.354957in}{2.762325in}}{\pgfqpoint{6.359347in}{2.772924in}}{\pgfqpoint{6.359347in}{2.783974in}}%
\pgfpathcurveto{\pgfqpoint{6.359347in}{2.795024in}}{\pgfqpoint{6.354957in}{2.805623in}}{\pgfqpoint{6.347143in}{2.813437in}}%
\pgfpathcurveto{\pgfqpoint{6.339330in}{2.821250in}}{\pgfqpoint{6.328731in}{2.825641in}}{\pgfqpoint{6.317681in}{2.825641in}}%
\pgfpathcurveto{\pgfqpoint{6.306631in}{2.825641in}}{\pgfqpoint{6.296032in}{2.821250in}}{\pgfqpoint{6.288218in}{2.813437in}}%
\pgfpathcurveto{\pgfqpoint{6.280404in}{2.805623in}}{\pgfqpoint{6.276014in}{2.795024in}}{\pgfqpoint{6.276014in}{2.783974in}}%
\pgfpathcurveto{\pgfqpoint{6.276014in}{2.772924in}}{\pgfqpoint{6.280404in}{2.762325in}}{\pgfqpoint{6.288218in}{2.754511in}}%
\pgfpathcurveto{\pgfqpoint{6.296032in}{2.746697in}}{\pgfqpoint{6.306631in}{2.742307in}}{\pgfqpoint{6.317681in}{2.742307in}}%
\pgfpathclose%
\pgfusepath{stroke,fill}%
\end{pgfscope}%
\begin{pgfscope}%
\pgfpathrectangle{\pgfqpoint{0.481978in}{0.331635in}}{\pgfqpoint{9.300000in}{7.700000in}}%
\pgfusepath{clip}%
\pgfsetbuttcap%
\pgfsetroundjoin%
\definecolor{currentfill}{rgb}{0.631373,0.788235,0.956863}%
\pgfsetfillcolor{currentfill}%
\pgfsetlinewidth{0.481800pt}%
\definecolor{currentstroke}{rgb}{1.000000,1.000000,1.000000}%
\pgfsetstrokecolor{currentstroke}%
\pgfsetdash{}{0pt}%
\pgfpathmoveto{\pgfqpoint{5.001156in}{3.852883in}}%
\pgfpathcurveto{\pgfqpoint{5.012206in}{3.852883in}}{\pgfqpoint{5.022805in}{3.857273in}}{\pgfqpoint{5.030618in}{3.865087in}}%
\pgfpathcurveto{\pgfqpoint{5.038432in}{3.872900in}}{\pgfqpoint{5.042822in}{3.883499in}}{\pgfqpoint{5.042822in}{3.894549in}}%
\pgfpathcurveto{\pgfqpoint{5.042822in}{3.905600in}}{\pgfqpoint{5.038432in}{3.916199in}}{\pgfqpoint{5.030618in}{3.924012in}}%
\pgfpathcurveto{\pgfqpoint{5.022805in}{3.931826in}}{\pgfqpoint{5.012206in}{3.936216in}}{\pgfqpoint{5.001156in}{3.936216in}}%
\pgfpathcurveto{\pgfqpoint{4.990105in}{3.936216in}}{\pgfqpoint{4.979506in}{3.931826in}}{\pgfqpoint{4.971693in}{3.924012in}}%
\pgfpathcurveto{\pgfqpoint{4.963879in}{3.916199in}}{\pgfqpoint{4.959489in}{3.905600in}}{\pgfqpoint{4.959489in}{3.894549in}}%
\pgfpathcurveto{\pgfqpoint{4.959489in}{3.883499in}}{\pgfqpoint{4.963879in}{3.872900in}}{\pgfqpoint{4.971693in}{3.865087in}}%
\pgfpathcurveto{\pgfqpoint{4.979506in}{3.857273in}}{\pgfqpoint{4.990105in}{3.852883in}}{\pgfqpoint{5.001156in}{3.852883in}}%
\pgfpathclose%
\pgfusepath{stroke,fill}%
\end{pgfscope}%
\begin{pgfscope}%
\pgfpathrectangle{\pgfqpoint{0.481978in}{0.331635in}}{\pgfqpoint{9.300000in}{7.700000in}}%
\pgfusepath{clip}%
\pgfsetbuttcap%
\pgfsetroundjoin%
\definecolor{currentfill}{rgb}{0.631373,0.788235,0.956863}%
\pgfsetfillcolor{currentfill}%
\pgfsetlinewidth{0.481800pt}%
\definecolor{currentstroke}{rgb}{1.000000,1.000000,1.000000}%
\pgfsetstrokecolor{currentstroke}%
\pgfsetdash{}{0pt}%
\pgfpathmoveto{\pgfqpoint{5.223174in}{3.234581in}}%
\pgfpathcurveto{\pgfqpoint{5.234224in}{3.234581in}}{\pgfqpoint{5.244823in}{3.238972in}}{\pgfqpoint{5.252637in}{3.246785in}}%
\pgfpathcurveto{\pgfqpoint{5.260450in}{3.254599in}}{\pgfqpoint{5.264841in}{3.265198in}}{\pgfqpoint{5.264841in}{3.276248in}}%
\pgfpathcurveto{\pgfqpoint{5.264841in}{3.287298in}}{\pgfqpoint{5.260450in}{3.297897in}}{\pgfqpoint{5.252637in}{3.305711in}}%
\pgfpathcurveto{\pgfqpoint{5.244823in}{3.313525in}}{\pgfqpoint{5.234224in}{3.317915in}}{\pgfqpoint{5.223174in}{3.317915in}}%
\pgfpathcurveto{\pgfqpoint{5.212124in}{3.317915in}}{\pgfqpoint{5.201525in}{3.313525in}}{\pgfqpoint{5.193711in}{3.305711in}}%
\pgfpathcurveto{\pgfqpoint{5.185897in}{3.297897in}}{\pgfqpoint{5.181507in}{3.287298in}}{\pgfqpoint{5.181507in}{3.276248in}}%
\pgfpathcurveto{\pgfqpoint{5.181507in}{3.265198in}}{\pgfqpoint{5.185897in}{3.254599in}}{\pgfqpoint{5.193711in}{3.246785in}}%
\pgfpathcurveto{\pgfqpoint{5.201525in}{3.238972in}}{\pgfqpoint{5.212124in}{3.234581in}}{\pgfqpoint{5.223174in}{3.234581in}}%
\pgfpathclose%
\pgfusepath{stroke,fill}%
\end{pgfscope}%
\begin{pgfscope}%
\pgfpathrectangle{\pgfqpoint{0.481978in}{0.331635in}}{\pgfqpoint{9.300000in}{7.700000in}}%
\pgfusepath{clip}%
\pgfsetbuttcap%
\pgfsetroundjoin%
\definecolor{currentfill}{rgb}{0.631373,0.788235,0.956863}%
\pgfsetfillcolor{currentfill}%
\pgfsetlinewidth{0.481800pt}%
\definecolor{currentstroke}{rgb}{1.000000,1.000000,1.000000}%
\pgfsetstrokecolor{currentstroke}%
\pgfsetdash{}{0pt}%
\pgfpathmoveto{\pgfqpoint{5.990771in}{2.510229in}}%
\pgfpathcurveto{\pgfqpoint{6.001822in}{2.510229in}}{\pgfqpoint{6.012421in}{2.514620in}}{\pgfqpoint{6.020234in}{2.522433in}}%
\pgfpathcurveto{\pgfqpoint{6.028048in}{2.530247in}}{\pgfqpoint{6.032438in}{2.540846in}}{\pgfqpoint{6.032438in}{2.551896in}}%
\pgfpathcurveto{\pgfqpoint{6.032438in}{2.562946in}}{\pgfqpoint{6.028048in}{2.573545in}}{\pgfqpoint{6.020234in}{2.581359in}}%
\pgfpathcurveto{\pgfqpoint{6.012421in}{2.589172in}}{\pgfqpoint{6.001822in}{2.593563in}}{\pgfqpoint{5.990771in}{2.593563in}}%
\pgfpathcurveto{\pgfqpoint{5.979721in}{2.593563in}}{\pgfqpoint{5.969122in}{2.589172in}}{\pgfqpoint{5.961309in}{2.581359in}}%
\pgfpathcurveto{\pgfqpoint{5.953495in}{2.573545in}}{\pgfqpoint{5.949105in}{2.562946in}}{\pgfqpoint{5.949105in}{2.551896in}}%
\pgfpathcurveto{\pgfqpoint{5.949105in}{2.540846in}}{\pgfqpoint{5.953495in}{2.530247in}}{\pgfqpoint{5.961309in}{2.522433in}}%
\pgfpathcurveto{\pgfqpoint{5.969122in}{2.514620in}}{\pgfqpoint{5.979721in}{2.510229in}}{\pgfqpoint{5.990771in}{2.510229in}}%
\pgfpathclose%
\pgfusepath{stroke,fill}%
\end{pgfscope}%
\begin{pgfscope}%
\pgfpathrectangle{\pgfqpoint{0.481978in}{0.331635in}}{\pgfqpoint{9.300000in}{7.700000in}}%
\pgfusepath{clip}%
\pgfsetbuttcap%
\pgfsetroundjoin%
\definecolor{currentfill}{rgb}{0.631373,0.788235,0.956863}%
\pgfsetfillcolor{currentfill}%
\pgfsetlinewidth{0.481800pt}%
\definecolor{currentstroke}{rgb}{1.000000,1.000000,1.000000}%
\pgfsetstrokecolor{currentstroke}%
\pgfsetdash{}{0pt}%
\pgfpathmoveto{\pgfqpoint{5.037316in}{4.288289in}}%
\pgfpathcurveto{\pgfqpoint{5.048366in}{4.288289in}}{\pgfqpoint{5.058965in}{4.292679in}}{\pgfqpoint{5.066779in}{4.300493in}}%
\pgfpathcurveto{\pgfqpoint{5.074592in}{4.308306in}}{\pgfqpoint{5.078983in}{4.318905in}}{\pgfqpoint{5.078983in}{4.329956in}}%
\pgfpathcurveto{\pgfqpoint{5.078983in}{4.341006in}}{\pgfqpoint{5.074592in}{4.351605in}}{\pgfqpoint{5.066779in}{4.359418in}}%
\pgfpathcurveto{\pgfqpoint{5.058965in}{4.367232in}}{\pgfqpoint{5.048366in}{4.371622in}}{\pgfqpoint{5.037316in}{4.371622in}}%
\pgfpathcurveto{\pgfqpoint{5.026266in}{4.371622in}}{\pgfqpoint{5.015667in}{4.367232in}}{\pgfqpoint{5.007853in}{4.359418in}}%
\pgfpathcurveto{\pgfqpoint{5.000040in}{4.351605in}}{\pgfqpoint{4.995649in}{4.341006in}}{\pgfqpoint{4.995649in}{4.329956in}}%
\pgfpathcurveto{\pgfqpoint{4.995649in}{4.318905in}}{\pgfqpoint{5.000040in}{4.308306in}}{\pgfqpoint{5.007853in}{4.300493in}}%
\pgfpathcurveto{\pgfqpoint{5.015667in}{4.292679in}}{\pgfqpoint{5.026266in}{4.288289in}}{\pgfqpoint{5.037316in}{4.288289in}}%
\pgfpathclose%
\pgfusepath{stroke,fill}%
\end{pgfscope}%
\begin{pgfscope}%
\pgfpathrectangle{\pgfqpoint{0.481978in}{0.331635in}}{\pgfqpoint{9.300000in}{7.700000in}}%
\pgfusepath{clip}%
\pgfsetbuttcap%
\pgfsetroundjoin%
\definecolor{currentfill}{rgb}{0.631373,0.788235,0.956863}%
\pgfsetfillcolor{currentfill}%
\pgfsetlinewidth{0.481800pt}%
\definecolor{currentstroke}{rgb}{1.000000,1.000000,1.000000}%
\pgfsetstrokecolor{currentstroke}%
\pgfsetdash{}{0pt}%
\pgfpathmoveto{\pgfqpoint{4.961622in}{0.843579in}}%
\pgfpathcurveto{\pgfqpoint{4.972673in}{0.843579in}}{\pgfqpoint{4.983272in}{0.847969in}}{\pgfqpoint{4.991085in}{0.855783in}}%
\pgfpathcurveto{\pgfqpoint{4.998899in}{0.863597in}}{\pgfqpoint{5.003289in}{0.874196in}}{\pgfqpoint{5.003289in}{0.885246in}}%
\pgfpathcurveto{\pgfqpoint{5.003289in}{0.896296in}}{\pgfqpoint{4.998899in}{0.906895in}}{\pgfqpoint{4.991085in}{0.914708in}}%
\pgfpathcurveto{\pgfqpoint{4.983272in}{0.922522in}}{\pgfqpoint{4.972673in}{0.926912in}}{\pgfqpoint{4.961622in}{0.926912in}}%
\pgfpathcurveto{\pgfqpoint{4.950572in}{0.926912in}}{\pgfqpoint{4.939973in}{0.922522in}}{\pgfqpoint{4.932160in}{0.914708in}}%
\pgfpathcurveto{\pgfqpoint{4.924346in}{0.906895in}}{\pgfqpoint{4.919956in}{0.896296in}}{\pgfqpoint{4.919956in}{0.885246in}}%
\pgfpathcurveto{\pgfqpoint{4.919956in}{0.874196in}}{\pgfqpoint{4.924346in}{0.863597in}}{\pgfqpoint{4.932160in}{0.855783in}}%
\pgfpathcurveto{\pgfqpoint{4.939973in}{0.847969in}}{\pgfqpoint{4.950572in}{0.843579in}}{\pgfqpoint{4.961622in}{0.843579in}}%
\pgfpathclose%
\pgfusepath{stroke,fill}%
\end{pgfscope}%
\begin{pgfscope}%
\pgfpathrectangle{\pgfqpoint{0.481978in}{0.331635in}}{\pgfqpoint{9.300000in}{7.700000in}}%
\pgfusepath{clip}%
\pgfsetbuttcap%
\pgfsetroundjoin%
\definecolor{currentfill}{rgb}{0.631373,0.788235,0.956863}%
\pgfsetfillcolor{currentfill}%
\pgfsetlinewidth{0.481800pt}%
\definecolor{currentstroke}{rgb}{1.000000,1.000000,1.000000}%
\pgfsetstrokecolor{currentstroke}%
\pgfsetdash{}{0pt}%
\pgfpathmoveto{\pgfqpoint{5.509130in}{2.049624in}}%
\pgfpathcurveto{\pgfqpoint{5.520180in}{2.049624in}}{\pgfqpoint{5.530779in}{2.054014in}}{\pgfqpoint{5.538593in}{2.061828in}}%
\pgfpathcurveto{\pgfqpoint{5.546406in}{2.069641in}}{\pgfqpoint{5.550797in}{2.080240in}}{\pgfqpoint{5.550797in}{2.091290in}}%
\pgfpathcurveto{\pgfqpoint{5.550797in}{2.102340in}}{\pgfqpoint{5.546406in}{2.112940in}}{\pgfqpoint{5.538593in}{2.120753in}}%
\pgfpathcurveto{\pgfqpoint{5.530779in}{2.128567in}}{\pgfqpoint{5.520180in}{2.132957in}}{\pgfqpoint{5.509130in}{2.132957in}}%
\pgfpathcurveto{\pgfqpoint{5.498080in}{2.132957in}}{\pgfqpoint{5.487481in}{2.128567in}}{\pgfqpoint{5.479667in}{2.120753in}}%
\pgfpathcurveto{\pgfqpoint{5.471854in}{2.112940in}}{\pgfqpoint{5.467463in}{2.102340in}}{\pgfqpoint{5.467463in}{2.091290in}}%
\pgfpathcurveto{\pgfqpoint{5.467463in}{2.080240in}}{\pgfqpoint{5.471854in}{2.069641in}}{\pgfqpoint{5.479667in}{2.061828in}}%
\pgfpathcurveto{\pgfqpoint{5.487481in}{2.054014in}}{\pgfqpoint{5.498080in}{2.049624in}}{\pgfqpoint{5.509130in}{2.049624in}}%
\pgfpathclose%
\pgfusepath{stroke,fill}%
\end{pgfscope}%
\begin{pgfscope}%
\pgfpathrectangle{\pgfqpoint{0.481978in}{0.331635in}}{\pgfqpoint{9.300000in}{7.700000in}}%
\pgfusepath{clip}%
\pgfsetbuttcap%
\pgfsetroundjoin%
\definecolor{currentfill}{rgb}{0.631373,0.788235,0.956863}%
\pgfsetfillcolor{currentfill}%
\pgfsetlinewidth{0.481800pt}%
\definecolor{currentstroke}{rgb}{1.000000,1.000000,1.000000}%
\pgfsetstrokecolor{currentstroke}%
\pgfsetdash{}{0pt}%
\pgfpathmoveto{\pgfqpoint{2.965360in}{6.380512in}}%
\pgfpathcurveto{\pgfqpoint{2.976410in}{6.380512in}}{\pgfqpoint{2.987009in}{6.384902in}}{\pgfqpoint{2.994823in}{6.392716in}}%
\pgfpathcurveto{\pgfqpoint{3.002636in}{6.400529in}}{\pgfqpoint{3.007027in}{6.411129in}}{\pgfqpoint{3.007027in}{6.422179in}}%
\pgfpathcurveto{\pgfqpoint{3.007027in}{6.433229in}}{\pgfqpoint{3.002636in}{6.443828in}}{\pgfqpoint{2.994823in}{6.451641in}}%
\pgfpathcurveto{\pgfqpoint{2.987009in}{6.459455in}}{\pgfqpoint{2.976410in}{6.463845in}}{\pgfqpoint{2.965360in}{6.463845in}}%
\pgfpathcurveto{\pgfqpoint{2.954310in}{6.463845in}}{\pgfqpoint{2.943711in}{6.459455in}}{\pgfqpoint{2.935897in}{6.451641in}}%
\pgfpathcurveto{\pgfqpoint{2.928083in}{6.443828in}}{\pgfqpoint{2.923693in}{6.433229in}}{\pgfqpoint{2.923693in}{6.422179in}}%
\pgfpathcurveto{\pgfqpoint{2.923693in}{6.411129in}}{\pgfqpoint{2.928083in}{6.400529in}}{\pgfqpoint{2.935897in}{6.392716in}}%
\pgfpathcurveto{\pgfqpoint{2.943711in}{6.384902in}}{\pgfqpoint{2.954310in}{6.380512in}}{\pgfqpoint{2.965360in}{6.380512in}}%
\pgfpathclose%
\pgfusepath{stroke,fill}%
\end{pgfscope}%
\begin{pgfscope}%
\pgfpathrectangle{\pgfqpoint{0.481978in}{0.331635in}}{\pgfqpoint{9.300000in}{7.700000in}}%
\pgfusepath{clip}%
\pgfsetbuttcap%
\pgfsetroundjoin%
\definecolor{currentfill}{rgb}{0.631373,0.788235,0.956863}%
\pgfsetfillcolor{currentfill}%
\pgfsetlinewidth{0.481800pt}%
\definecolor{currentstroke}{rgb}{1.000000,1.000000,1.000000}%
\pgfsetstrokecolor{currentstroke}%
\pgfsetdash{}{0pt}%
\pgfpathmoveto{\pgfqpoint{6.394060in}{2.525199in}}%
\pgfpathcurveto{\pgfqpoint{6.405110in}{2.525199in}}{\pgfqpoint{6.415709in}{2.529590in}}{\pgfqpoint{6.423523in}{2.537403in}}%
\pgfpathcurveto{\pgfqpoint{6.431336in}{2.545217in}}{\pgfqpoint{6.435727in}{2.555816in}}{\pgfqpoint{6.435727in}{2.566866in}}%
\pgfpathcurveto{\pgfqpoint{6.435727in}{2.577916in}}{\pgfqpoint{6.431336in}{2.588515in}}{\pgfqpoint{6.423523in}{2.596329in}}%
\pgfpathcurveto{\pgfqpoint{6.415709in}{2.604142in}}{\pgfqpoint{6.405110in}{2.608533in}}{\pgfqpoint{6.394060in}{2.608533in}}%
\pgfpathcurveto{\pgfqpoint{6.383010in}{2.608533in}}{\pgfqpoint{6.372411in}{2.604142in}}{\pgfqpoint{6.364597in}{2.596329in}}%
\pgfpathcurveto{\pgfqpoint{6.356784in}{2.588515in}}{\pgfqpoint{6.352393in}{2.577916in}}{\pgfqpoint{6.352393in}{2.566866in}}%
\pgfpathcurveto{\pgfqpoint{6.352393in}{2.555816in}}{\pgfqpoint{6.356784in}{2.545217in}}{\pgfqpoint{6.364597in}{2.537403in}}%
\pgfpathcurveto{\pgfqpoint{6.372411in}{2.529590in}}{\pgfqpoint{6.383010in}{2.525199in}}{\pgfqpoint{6.394060in}{2.525199in}}%
\pgfpathclose%
\pgfusepath{stroke,fill}%
\end{pgfscope}%
\begin{pgfscope}%
\pgfpathrectangle{\pgfqpoint{0.481978in}{0.331635in}}{\pgfqpoint{9.300000in}{7.700000in}}%
\pgfusepath{clip}%
\pgfsetbuttcap%
\pgfsetroundjoin%
\definecolor{currentfill}{rgb}{0.631373,0.788235,0.956863}%
\pgfsetfillcolor{currentfill}%
\pgfsetlinewidth{0.481800pt}%
\definecolor{currentstroke}{rgb}{1.000000,1.000000,1.000000}%
\pgfsetstrokecolor{currentstroke}%
\pgfsetdash{}{0pt}%
\pgfpathmoveto{\pgfqpoint{2.186468in}{6.771610in}}%
\pgfpathcurveto{\pgfqpoint{2.197518in}{6.771610in}}{\pgfqpoint{2.208117in}{6.776000in}}{\pgfqpoint{2.215931in}{6.783814in}}%
\pgfpathcurveto{\pgfqpoint{2.223745in}{6.791628in}}{\pgfqpoint{2.228135in}{6.802227in}}{\pgfqpoint{2.228135in}{6.813277in}}%
\pgfpathcurveto{\pgfqpoint{2.228135in}{6.824327in}}{\pgfqpoint{2.223745in}{6.834926in}}{\pgfqpoint{2.215931in}{6.842740in}}%
\pgfpathcurveto{\pgfqpoint{2.208117in}{6.850553in}}{\pgfqpoint{2.197518in}{6.854944in}}{\pgfqpoint{2.186468in}{6.854944in}}%
\pgfpathcurveto{\pgfqpoint{2.175418in}{6.854944in}}{\pgfqpoint{2.164819in}{6.850553in}}{\pgfqpoint{2.157005in}{6.842740in}}%
\pgfpathcurveto{\pgfqpoint{2.149192in}{6.834926in}}{\pgfqpoint{2.144802in}{6.824327in}}{\pgfqpoint{2.144802in}{6.813277in}}%
\pgfpathcurveto{\pgfqpoint{2.144802in}{6.802227in}}{\pgfqpoint{2.149192in}{6.791628in}}{\pgfqpoint{2.157005in}{6.783814in}}%
\pgfpathcurveto{\pgfqpoint{2.164819in}{6.776000in}}{\pgfqpoint{2.175418in}{6.771610in}}{\pgfqpoint{2.186468in}{6.771610in}}%
\pgfpathclose%
\pgfusepath{stroke,fill}%
\end{pgfscope}%
\begin{pgfscope}%
\pgfpathrectangle{\pgfqpoint{0.481978in}{0.331635in}}{\pgfqpoint{9.300000in}{7.700000in}}%
\pgfusepath{clip}%
\pgfsetbuttcap%
\pgfsetroundjoin%
\definecolor{currentfill}{rgb}{0.631373,0.788235,0.956863}%
\pgfsetfillcolor{currentfill}%
\pgfsetlinewidth{0.481800pt}%
\definecolor{currentstroke}{rgb}{1.000000,1.000000,1.000000}%
\pgfsetstrokecolor{currentstroke}%
\pgfsetdash{}{0pt}%
\pgfpathmoveto{\pgfqpoint{7.121234in}{2.875912in}}%
\pgfpathcurveto{\pgfqpoint{7.132284in}{2.875912in}}{\pgfqpoint{7.142883in}{2.880302in}}{\pgfqpoint{7.150697in}{2.888116in}}%
\pgfpathcurveto{\pgfqpoint{7.158510in}{2.895930in}}{\pgfqpoint{7.162901in}{2.906529in}}{\pgfqpoint{7.162901in}{2.917579in}}%
\pgfpathcurveto{\pgfqpoint{7.162901in}{2.928629in}}{\pgfqpoint{7.158510in}{2.939228in}}{\pgfqpoint{7.150697in}{2.947042in}}%
\pgfpathcurveto{\pgfqpoint{7.142883in}{2.954855in}}{\pgfqpoint{7.132284in}{2.959245in}}{\pgfqpoint{7.121234in}{2.959245in}}%
\pgfpathcurveto{\pgfqpoint{7.110184in}{2.959245in}}{\pgfqpoint{7.099585in}{2.954855in}}{\pgfqpoint{7.091771in}{2.947042in}}%
\pgfpathcurveto{\pgfqpoint{7.083958in}{2.939228in}}{\pgfqpoint{7.079567in}{2.928629in}}{\pgfqpoint{7.079567in}{2.917579in}}%
\pgfpathcurveto{\pgfqpoint{7.079567in}{2.906529in}}{\pgfqpoint{7.083958in}{2.895930in}}{\pgfqpoint{7.091771in}{2.888116in}}%
\pgfpathcurveto{\pgfqpoint{7.099585in}{2.880302in}}{\pgfqpoint{7.110184in}{2.875912in}}{\pgfqpoint{7.121234in}{2.875912in}}%
\pgfpathclose%
\pgfusepath{stroke,fill}%
\end{pgfscope}%
\begin{pgfscope}%
\pgfpathrectangle{\pgfqpoint{0.481978in}{0.331635in}}{\pgfqpoint{9.300000in}{7.700000in}}%
\pgfusepath{clip}%
\pgfsetbuttcap%
\pgfsetroundjoin%
\definecolor{currentfill}{rgb}{0.631373,0.788235,0.956863}%
\pgfsetfillcolor{currentfill}%
\pgfsetlinewidth{0.481800pt}%
\definecolor{currentstroke}{rgb}{1.000000,1.000000,1.000000}%
\pgfsetstrokecolor{currentstroke}%
\pgfsetdash{}{0pt}%
\pgfpathmoveto{\pgfqpoint{8.167603in}{4.962831in}}%
\pgfpathcurveto{\pgfqpoint{8.178653in}{4.962831in}}{\pgfqpoint{8.189252in}{4.967221in}}{\pgfqpoint{8.197066in}{4.975035in}}%
\pgfpathcurveto{\pgfqpoint{8.204879in}{4.982848in}}{\pgfqpoint{8.209270in}{4.993447in}}{\pgfqpoint{8.209270in}{5.004497in}}%
\pgfpathcurveto{\pgfqpoint{8.209270in}{5.015547in}}{\pgfqpoint{8.204879in}{5.026146in}}{\pgfqpoint{8.197066in}{5.033960in}}%
\pgfpathcurveto{\pgfqpoint{8.189252in}{5.041774in}}{\pgfqpoint{8.178653in}{5.046164in}}{\pgfqpoint{8.167603in}{5.046164in}}%
\pgfpathcurveto{\pgfqpoint{8.156553in}{5.046164in}}{\pgfqpoint{8.145954in}{5.041774in}}{\pgfqpoint{8.138140in}{5.033960in}}%
\pgfpathcurveto{\pgfqpoint{8.130327in}{5.026146in}}{\pgfqpoint{8.125936in}{5.015547in}}{\pgfqpoint{8.125936in}{5.004497in}}%
\pgfpathcurveto{\pgfqpoint{8.125936in}{4.993447in}}{\pgfqpoint{8.130327in}{4.982848in}}{\pgfqpoint{8.138140in}{4.975035in}}%
\pgfpathcurveto{\pgfqpoint{8.145954in}{4.967221in}}{\pgfqpoint{8.156553in}{4.962831in}}{\pgfqpoint{8.167603in}{4.962831in}}%
\pgfpathclose%
\pgfusepath{stroke,fill}%
\end{pgfscope}%
\begin{pgfscope}%
\pgfpathrectangle{\pgfqpoint{0.481978in}{0.331635in}}{\pgfqpoint{9.300000in}{7.700000in}}%
\pgfusepath{clip}%
\pgfsetbuttcap%
\pgfsetroundjoin%
\definecolor{currentfill}{rgb}{0.631373,0.788235,0.956863}%
\pgfsetfillcolor{currentfill}%
\pgfsetlinewidth{0.481800pt}%
\definecolor{currentstroke}{rgb}{1.000000,1.000000,1.000000}%
\pgfsetstrokecolor{currentstroke}%
\pgfsetdash{}{0pt}%
\pgfpathmoveto{\pgfqpoint{6.293261in}{4.405108in}}%
\pgfpathcurveto{\pgfqpoint{6.304311in}{4.405108in}}{\pgfqpoint{6.314910in}{4.409499in}}{\pgfqpoint{6.322724in}{4.417312in}}%
\pgfpathcurveto{\pgfqpoint{6.330538in}{4.425126in}}{\pgfqpoint{6.334928in}{4.435725in}}{\pgfqpoint{6.334928in}{4.446775in}}%
\pgfpathcurveto{\pgfqpoint{6.334928in}{4.457825in}}{\pgfqpoint{6.330538in}{4.468424in}}{\pgfqpoint{6.322724in}{4.476238in}}%
\pgfpathcurveto{\pgfqpoint{6.314910in}{4.484051in}}{\pgfqpoint{6.304311in}{4.488442in}}{\pgfqpoint{6.293261in}{4.488442in}}%
\pgfpathcurveto{\pgfqpoint{6.282211in}{4.488442in}}{\pgfqpoint{6.271612in}{4.484051in}}{\pgfqpoint{6.263799in}{4.476238in}}%
\pgfpathcurveto{\pgfqpoint{6.255985in}{4.468424in}}{\pgfqpoint{6.251595in}{4.457825in}}{\pgfqpoint{6.251595in}{4.446775in}}%
\pgfpathcurveto{\pgfqpoint{6.251595in}{4.435725in}}{\pgfqpoint{6.255985in}{4.425126in}}{\pgfqpoint{6.263799in}{4.417312in}}%
\pgfpathcurveto{\pgfqpoint{6.271612in}{4.409499in}}{\pgfqpoint{6.282211in}{4.405108in}}{\pgfqpoint{6.293261in}{4.405108in}}%
\pgfpathclose%
\pgfusepath{stroke,fill}%
\end{pgfscope}%
\begin{pgfscope}%
\pgfpathrectangle{\pgfqpoint{0.481978in}{0.331635in}}{\pgfqpoint{9.300000in}{7.700000in}}%
\pgfusepath{clip}%
\pgfsetbuttcap%
\pgfsetroundjoin%
\definecolor{currentfill}{rgb}{0.631373,0.788235,0.956863}%
\pgfsetfillcolor{currentfill}%
\pgfsetlinewidth{0.481800pt}%
\definecolor{currentstroke}{rgb}{1.000000,1.000000,1.000000}%
\pgfsetstrokecolor{currentstroke}%
\pgfsetdash{}{0pt}%
\pgfpathmoveto{\pgfqpoint{4.855717in}{7.162437in}}%
\pgfpathcurveto{\pgfqpoint{4.866767in}{7.162437in}}{\pgfqpoint{4.877366in}{7.166827in}}{\pgfqpoint{4.885179in}{7.174640in}}%
\pgfpathcurveto{\pgfqpoint{4.892993in}{7.182454in}}{\pgfqpoint{4.897383in}{7.193053in}}{\pgfqpoint{4.897383in}{7.204103in}}%
\pgfpathcurveto{\pgfqpoint{4.897383in}{7.215153in}}{\pgfqpoint{4.892993in}{7.225752in}}{\pgfqpoint{4.885179in}{7.233566in}}%
\pgfpathcurveto{\pgfqpoint{4.877366in}{7.241380in}}{\pgfqpoint{4.866767in}{7.245770in}}{\pgfqpoint{4.855717in}{7.245770in}}%
\pgfpathcurveto{\pgfqpoint{4.844666in}{7.245770in}}{\pgfqpoint{4.834067in}{7.241380in}}{\pgfqpoint{4.826254in}{7.233566in}}%
\pgfpathcurveto{\pgfqpoint{4.818440in}{7.225752in}}{\pgfqpoint{4.814050in}{7.215153in}}{\pgfqpoint{4.814050in}{7.204103in}}%
\pgfpathcurveto{\pgfqpoint{4.814050in}{7.193053in}}{\pgfqpoint{4.818440in}{7.182454in}}{\pgfqpoint{4.826254in}{7.174640in}}%
\pgfpathcurveto{\pgfqpoint{4.834067in}{7.166827in}}{\pgfqpoint{4.844666in}{7.162437in}}{\pgfqpoint{4.855717in}{7.162437in}}%
\pgfpathclose%
\pgfusepath{stroke,fill}%
\end{pgfscope}%
\begin{pgfscope}%
\pgfpathrectangle{\pgfqpoint{0.481978in}{0.331635in}}{\pgfqpoint{9.300000in}{7.700000in}}%
\pgfusepath{clip}%
\pgfsetbuttcap%
\pgfsetroundjoin%
\definecolor{currentfill}{rgb}{0.631373,0.788235,0.956863}%
\pgfsetfillcolor{currentfill}%
\pgfsetlinewidth{0.481800pt}%
\definecolor{currentstroke}{rgb}{1.000000,1.000000,1.000000}%
\pgfsetstrokecolor{currentstroke}%
\pgfsetdash{}{0pt}%
\pgfpathmoveto{\pgfqpoint{8.228148in}{4.788343in}}%
\pgfpathcurveto{\pgfqpoint{8.239199in}{4.788343in}}{\pgfqpoint{8.249798in}{4.792734in}}{\pgfqpoint{8.257611in}{4.800547in}}%
\pgfpathcurveto{\pgfqpoint{8.265425in}{4.808361in}}{\pgfqpoint{8.269815in}{4.818960in}}{\pgfqpoint{8.269815in}{4.830010in}}%
\pgfpathcurveto{\pgfqpoint{8.269815in}{4.841060in}}{\pgfqpoint{8.265425in}{4.851659in}}{\pgfqpoint{8.257611in}{4.859473in}}%
\pgfpathcurveto{\pgfqpoint{8.249798in}{4.867286in}}{\pgfqpoint{8.239199in}{4.871677in}}{\pgfqpoint{8.228148in}{4.871677in}}%
\pgfpathcurveto{\pgfqpoint{8.217098in}{4.871677in}}{\pgfqpoint{8.206499in}{4.867286in}}{\pgfqpoint{8.198686in}{4.859473in}}%
\pgfpathcurveto{\pgfqpoint{8.190872in}{4.851659in}}{\pgfqpoint{8.186482in}{4.841060in}}{\pgfqpoint{8.186482in}{4.830010in}}%
\pgfpathcurveto{\pgfqpoint{8.186482in}{4.818960in}}{\pgfqpoint{8.190872in}{4.808361in}}{\pgfqpoint{8.198686in}{4.800547in}}%
\pgfpathcurveto{\pgfqpoint{8.206499in}{4.792734in}}{\pgfqpoint{8.217098in}{4.788343in}}{\pgfqpoint{8.228148in}{4.788343in}}%
\pgfpathclose%
\pgfusepath{stroke,fill}%
\end{pgfscope}%
\begin{pgfscope}%
\pgfpathrectangle{\pgfqpoint{0.481978in}{0.331635in}}{\pgfqpoint{9.300000in}{7.700000in}}%
\pgfusepath{clip}%
\pgfsetbuttcap%
\pgfsetroundjoin%
\definecolor{currentfill}{rgb}{0.631373,0.788235,0.956863}%
\pgfsetfillcolor{currentfill}%
\pgfsetlinewidth{0.481800pt}%
\definecolor{currentstroke}{rgb}{1.000000,1.000000,1.000000}%
\pgfsetstrokecolor{currentstroke}%
\pgfsetdash{}{0pt}%
\pgfpathmoveto{\pgfqpoint{7.375231in}{1.910568in}}%
\pgfpathcurveto{\pgfqpoint{7.386281in}{1.910568in}}{\pgfqpoint{7.396880in}{1.914958in}}{\pgfqpoint{7.404694in}{1.922772in}}%
\pgfpathcurveto{\pgfqpoint{7.412507in}{1.930585in}}{\pgfqpoint{7.416898in}{1.941184in}}{\pgfqpoint{7.416898in}{1.952235in}}%
\pgfpathcurveto{\pgfqpoint{7.416898in}{1.963285in}}{\pgfqpoint{7.412507in}{1.973884in}}{\pgfqpoint{7.404694in}{1.981697in}}%
\pgfpathcurveto{\pgfqpoint{7.396880in}{1.989511in}}{\pgfqpoint{7.386281in}{1.993901in}}{\pgfqpoint{7.375231in}{1.993901in}}%
\pgfpathcurveto{\pgfqpoint{7.364181in}{1.993901in}}{\pgfqpoint{7.353582in}{1.989511in}}{\pgfqpoint{7.345768in}{1.981697in}}%
\pgfpathcurveto{\pgfqpoint{7.337954in}{1.973884in}}{\pgfqpoint{7.333564in}{1.963285in}}{\pgfqpoint{7.333564in}{1.952235in}}%
\pgfpathcurveto{\pgfqpoint{7.333564in}{1.941184in}}{\pgfqpoint{7.337954in}{1.930585in}}{\pgfqpoint{7.345768in}{1.922772in}}%
\pgfpathcurveto{\pgfqpoint{7.353582in}{1.914958in}}{\pgfqpoint{7.364181in}{1.910568in}}{\pgfqpoint{7.375231in}{1.910568in}}%
\pgfpathclose%
\pgfusepath{stroke,fill}%
\end{pgfscope}%
\begin{pgfscope}%
\pgfpathrectangle{\pgfqpoint{0.481978in}{0.331635in}}{\pgfqpoint{9.300000in}{7.700000in}}%
\pgfusepath{clip}%
\pgfsetbuttcap%
\pgfsetroundjoin%
\definecolor{currentfill}{rgb}{0.631373,0.788235,0.956863}%
\pgfsetfillcolor{currentfill}%
\pgfsetlinewidth{0.481800pt}%
\definecolor{currentstroke}{rgb}{1.000000,1.000000,1.000000}%
\pgfsetstrokecolor{currentstroke}%
\pgfsetdash{}{0pt}%
\pgfpathmoveto{\pgfqpoint{6.949155in}{1.608307in}}%
\pgfpathcurveto{\pgfqpoint{6.960205in}{1.608307in}}{\pgfqpoint{6.970804in}{1.612697in}}{\pgfqpoint{6.978618in}{1.620511in}}%
\pgfpathcurveto{\pgfqpoint{6.986431in}{1.628324in}}{\pgfqpoint{6.990822in}{1.638924in}}{\pgfqpoint{6.990822in}{1.649974in}}%
\pgfpathcurveto{\pgfqpoint{6.990822in}{1.661024in}}{\pgfqpoint{6.986431in}{1.671623in}}{\pgfqpoint{6.978618in}{1.679436in}}%
\pgfpathcurveto{\pgfqpoint{6.970804in}{1.687250in}}{\pgfqpoint{6.960205in}{1.691640in}}{\pgfqpoint{6.949155in}{1.691640in}}%
\pgfpathcurveto{\pgfqpoint{6.938105in}{1.691640in}}{\pgfqpoint{6.927506in}{1.687250in}}{\pgfqpoint{6.919692in}{1.679436in}}%
\pgfpathcurveto{\pgfqpoint{6.911879in}{1.671623in}}{\pgfqpoint{6.907488in}{1.661024in}}{\pgfqpoint{6.907488in}{1.649974in}}%
\pgfpathcurveto{\pgfqpoint{6.907488in}{1.638924in}}{\pgfqpoint{6.911879in}{1.628324in}}{\pgfqpoint{6.919692in}{1.620511in}}%
\pgfpathcurveto{\pgfqpoint{6.927506in}{1.612697in}}{\pgfqpoint{6.938105in}{1.608307in}}{\pgfqpoint{6.949155in}{1.608307in}}%
\pgfpathclose%
\pgfusepath{stroke,fill}%
\end{pgfscope}%
\begin{pgfscope}%
\pgfpathrectangle{\pgfqpoint{0.481978in}{0.331635in}}{\pgfqpoint{9.300000in}{7.700000in}}%
\pgfusepath{clip}%
\pgfsetbuttcap%
\pgfsetroundjoin%
\definecolor{currentfill}{rgb}{0.631373,0.788235,0.956863}%
\pgfsetfillcolor{currentfill}%
\pgfsetlinewidth{0.481800pt}%
\definecolor{currentstroke}{rgb}{1.000000,1.000000,1.000000}%
\pgfsetstrokecolor{currentstroke}%
\pgfsetdash{}{0pt}%
\pgfpathmoveto{\pgfqpoint{7.251466in}{2.478396in}}%
\pgfpathcurveto{\pgfqpoint{7.262516in}{2.478396in}}{\pgfqpoint{7.273115in}{2.482786in}}{\pgfqpoint{7.280929in}{2.490600in}}%
\pgfpathcurveto{\pgfqpoint{7.288742in}{2.498413in}}{\pgfqpoint{7.293133in}{2.509012in}}{\pgfqpoint{7.293133in}{2.520062in}}%
\pgfpathcurveto{\pgfqpoint{7.293133in}{2.531113in}}{\pgfqpoint{7.288742in}{2.541712in}}{\pgfqpoint{7.280929in}{2.549525in}}%
\pgfpathcurveto{\pgfqpoint{7.273115in}{2.557339in}}{\pgfqpoint{7.262516in}{2.561729in}}{\pgfqpoint{7.251466in}{2.561729in}}%
\pgfpathcurveto{\pgfqpoint{7.240416in}{2.561729in}}{\pgfqpoint{7.229817in}{2.557339in}}{\pgfqpoint{7.222003in}{2.549525in}}%
\pgfpathcurveto{\pgfqpoint{7.214189in}{2.541712in}}{\pgfqpoint{7.209799in}{2.531113in}}{\pgfqpoint{7.209799in}{2.520062in}}%
\pgfpathcurveto{\pgfqpoint{7.209799in}{2.509012in}}{\pgfqpoint{7.214189in}{2.498413in}}{\pgfqpoint{7.222003in}{2.490600in}}%
\pgfpathcurveto{\pgfqpoint{7.229817in}{2.482786in}}{\pgfqpoint{7.240416in}{2.478396in}}{\pgfqpoint{7.251466in}{2.478396in}}%
\pgfpathclose%
\pgfusepath{stroke,fill}%
\end{pgfscope}%
\begin{pgfscope}%
\pgfpathrectangle{\pgfqpoint{0.481978in}{0.331635in}}{\pgfqpoint{9.300000in}{7.700000in}}%
\pgfusepath{clip}%
\pgfsetbuttcap%
\pgfsetroundjoin%
\definecolor{currentfill}{rgb}{0.631373,0.788235,0.956863}%
\pgfsetfillcolor{currentfill}%
\pgfsetlinewidth{0.481800pt}%
\definecolor{currentstroke}{rgb}{1.000000,1.000000,1.000000}%
\pgfsetstrokecolor{currentstroke}%
\pgfsetdash{}{0pt}%
\pgfpathmoveto{\pgfqpoint{5.500453in}{2.132128in}}%
\pgfpathcurveto{\pgfqpoint{5.511503in}{2.132128in}}{\pgfqpoint{5.522102in}{2.136518in}}{\pgfqpoint{5.529915in}{2.144332in}}%
\pgfpathcurveto{\pgfqpoint{5.537729in}{2.152146in}}{\pgfqpoint{5.542119in}{2.162745in}}{\pgfqpoint{5.542119in}{2.173795in}}%
\pgfpathcurveto{\pgfqpoint{5.542119in}{2.184845in}}{\pgfqpoint{5.537729in}{2.195444in}}{\pgfqpoint{5.529915in}{2.203258in}}%
\pgfpathcurveto{\pgfqpoint{5.522102in}{2.211071in}}{\pgfqpoint{5.511503in}{2.215461in}}{\pgfqpoint{5.500453in}{2.215461in}}%
\pgfpathcurveto{\pgfqpoint{5.489402in}{2.215461in}}{\pgfqpoint{5.478803in}{2.211071in}}{\pgfqpoint{5.470990in}{2.203258in}}%
\pgfpathcurveto{\pgfqpoint{5.463176in}{2.195444in}}{\pgfqpoint{5.458786in}{2.184845in}}{\pgfqpoint{5.458786in}{2.173795in}}%
\pgfpathcurveto{\pgfqpoint{5.458786in}{2.162745in}}{\pgfqpoint{5.463176in}{2.152146in}}{\pgfqpoint{5.470990in}{2.144332in}}%
\pgfpathcurveto{\pgfqpoint{5.478803in}{2.136518in}}{\pgfqpoint{5.489402in}{2.132128in}}{\pgfqpoint{5.500453in}{2.132128in}}%
\pgfpathclose%
\pgfusepath{stroke,fill}%
\end{pgfscope}%
\begin{pgfscope}%
\pgfpathrectangle{\pgfqpoint{0.481978in}{0.331635in}}{\pgfqpoint{9.300000in}{7.700000in}}%
\pgfusepath{clip}%
\pgfsetbuttcap%
\pgfsetroundjoin%
\definecolor{currentfill}{rgb}{0.631373,0.788235,0.956863}%
\pgfsetfillcolor{currentfill}%
\pgfsetlinewidth{0.481800pt}%
\definecolor{currentstroke}{rgb}{1.000000,1.000000,1.000000}%
\pgfsetstrokecolor{currentstroke}%
\pgfsetdash{}{0pt}%
\pgfpathmoveto{\pgfqpoint{6.149200in}{2.277817in}}%
\pgfpathcurveto{\pgfqpoint{6.160250in}{2.277817in}}{\pgfqpoint{6.170849in}{2.282207in}}{\pgfqpoint{6.178663in}{2.290021in}}%
\pgfpathcurveto{\pgfqpoint{6.186476in}{2.297835in}}{\pgfqpoint{6.190867in}{2.308434in}}{\pgfqpoint{6.190867in}{2.319484in}}%
\pgfpathcurveto{\pgfqpoint{6.190867in}{2.330534in}}{\pgfqpoint{6.186476in}{2.341133in}}{\pgfqpoint{6.178663in}{2.348947in}}%
\pgfpathcurveto{\pgfqpoint{6.170849in}{2.356760in}}{\pgfqpoint{6.160250in}{2.361150in}}{\pgfqpoint{6.149200in}{2.361150in}}%
\pgfpathcurveto{\pgfqpoint{6.138150in}{2.361150in}}{\pgfqpoint{6.127551in}{2.356760in}}{\pgfqpoint{6.119737in}{2.348947in}}%
\pgfpathcurveto{\pgfqpoint{6.111923in}{2.341133in}}{\pgfqpoint{6.107533in}{2.330534in}}{\pgfqpoint{6.107533in}{2.319484in}}%
\pgfpathcurveto{\pgfqpoint{6.107533in}{2.308434in}}{\pgfqpoint{6.111923in}{2.297835in}}{\pgfqpoint{6.119737in}{2.290021in}}%
\pgfpathcurveto{\pgfqpoint{6.127551in}{2.282207in}}{\pgfqpoint{6.138150in}{2.277817in}}{\pgfqpoint{6.149200in}{2.277817in}}%
\pgfpathclose%
\pgfusepath{stroke,fill}%
\end{pgfscope}%
\begin{pgfscope}%
\pgfpathrectangle{\pgfqpoint{0.481978in}{0.331635in}}{\pgfqpoint{9.300000in}{7.700000in}}%
\pgfusepath{clip}%
\pgfsetbuttcap%
\pgfsetroundjoin%
\definecolor{currentfill}{rgb}{0.631373,0.788235,0.956863}%
\pgfsetfillcolor{currentfill}%
\pgfsetlinewidth{0.481800pt}%
\definecolor{currentstroke}{rgb}{1.000000,1.000000,1.000000}%
\pgfsetstrokecolor{currentstroke}%
\pgfsetdash{}{0pt}%
\pgfpathmoveto{\pgfqpoint{2.951748in}{1.113137in}}%
\pgfpathcurveto{\pgfqpoint{2.962798in}{1.113137in}}{\pgfqpoint{2.973397in}{1.117527in}}{\pgfqpoint{2.981211in}{1.125341in}}%
\pgfpathcurveto{\pgfqpoint{2.989024in}{1.133155in}}{\pgfqpoint{2.993415in}{1.143754in}}{\pgfqpoint{2.993415in}{1.154804in}}%
\pgfpathcurveto{\pgfqpoint{2.993415in}{1.165854in}}{\pgfqpoint{2.989024in}{1.176453in}}{\pgfqpoint{2.981211in}{1.184267in}}%
\pgfpathcurveto{\pgfqpoint{2.973397in}{1.192080in}}{\pgfqpoint{2.962798in}{1.196471in}}{\pgfqpoint{2.951748in}{1.196471in}}%
\pgfpathcurveto{\pgfqpoint{2.940698in}{1.196471in}}{\pgfqpoint{2.930099in}{1.192080in}}{\pgfqpoint{2.922285in}{1.184267in}}%
\pgfpathcurveto{\pgfqpoint{2.914471in}{1.176453in}}{\pgfqpoint{2.910081in}{1.165854in}}{\pgfqpoint{2.910081in}{1.154804in}}%
\pgfpathcurveto{\pgfqpoint{2.910081in}{1.143754in}}{\pgfqpoint{2.914471in}{1.133155in}}{\pgfqpoint{2.922285in}{1.125341in}}%
\pgfpathcurveto{\pgfqpoint{2.930099in}{1.117527in}}{\pgfqpoint{2.940698in}{1.113137in}}{\pgfqpoint{2.951748in}{1.113137in}}%
\pgfpathclose%
\pgfusepath{stroke,fill}%
\end{pgfscope}%
\begin{pgfscope}%
\pgfpathrectangle{\pgfqpoint{0.481978in}{0.331635in}}{\pgfqpoint{9.300000in}{7.700000in}}%
\pgfusepath{clip}%
\pgfsetbuttcap%
\pgfsetroundjoin%
\definecolor{currentfill}{rgb}{0.631373,0.788235,0.956863}%
\pgfsetfillcolor{currentfill}%
\pgfsetlinewidth{0.481800pt}%
\definecolor{currentstroke}{rgb}{1.000000,1.000000,1.000000}%
\pgfsetstrokecolor{currentstroke}%
\pgfsetdash{}{0pt}%
\pgfpathmoveto{\pgfqpoint{6.416375in}{3.193658in}}%
\pgfpathcurveto{\pgfqpoint{6.427425in}{3.193658in}}{\pgfqpoint{6.438024in}{3.198048in}}{\pgfqpoint{6.445838in}{3.205862in}}%
\pgfpathcurveto{\pgfqpoint{6.453651in}{3.213676in}}{\pgfqpoint{6.458041in}{3.224275in}}{\pgfqpoint{6.458041in}{3.235325in}}%
\pgfpathcurveto{\pgfqpoint{6.458041in}{3.246375in}}{\pgfqpoint{6.453651in}{3.256974in}}{\pgfqpoint{6.445838in}{3.264788in}}%
\pgfpathcurveto{\pgfqpoint{6.438024in}{3.272601in}}{\pgfqpoint{6.427425in}{3.276991in}}{\pgfqpoint{6.416375in}{3.276991in}}%
\pgfpathcurveto{\pgfqpoint{6.405325in}{3.276991in}}{\pgfqpoint{6.394726in}{3.272601in}}{\pgfqpoint{6.386912in}{3.264788in}}%
\pgfpathcurveto{\pgfqpoint{6.379098in}{3.256974in}}{\pgfqpoint{6.374708in}{3.246375in}}{\pgfqpoint{6.374708in}{3.235325in}}%
\pgfpathcurveto{\pgfqpoint{6.374708in}{3.224275in}}{\pgfqpoint{6.379098in}{3.213676in}}{\pgfqpoint{6.386912in}{3.205862in}}%
\pgfpathcurveto{\pgfqpoint{6.394726in}{3.198048in}}{\pgfqpoint{6.405325in}{3.193658in}}{\pgfqpoint{6.416375in}{3.193658in}}%
\pgfpathclose%
\pgfusepath{stroke,fill}%
\end{pgfscope}%
\begin{pgfscope}%
\pgfpathrectangle{\pgfqpoint{0.481978in}{0.331635in}}{\pgfqpoint{9.300000in}{7.700000in}}%
\pgfusepath{clip}%
\pgfsetbuttcap%
\pgfsetroundjoin%
\definecolor{currentfill}{rgb}{0.631373,0.788235,0.956863}%
\pgfsetfillcolor{currentfill}%
\pgfsetlinewidth{0.481800pt}%
\definecolor{currentstroke}{rgb}{1.000000,1.000000,1.000000}%
\pgfsetstrokecolor{currentstroke}%
\pgfsetdash{}{0pt}%
\pgfpathmoveto{\pgfqpoint{2.918703in}{1.224111in}}%
\pgfpathcurveto{\pgfqpoint{2.929753in}{1.224111in}}{\pgfqpoint{2.940352in}{1.228501in}}{\pgfqpoint{2.948166in}{1.236315in}}%
\pgfpathcurveto{\pgfqpoint{2.955979in}{1.244128in}}{\pgfqpoint{2.960370in}{1.254727in}}{\pgfqpoint{2.960370in}{1.265777in}}%
\pgfpathcurveto{\pgfqpoint{2.960370in}{1.276827in}}{\pgfqpoint{2.955979in}{1.287427in}}{\pgfqpoint{2.948166in}{1.295240in}}%
\pgfpathcurveto{\pgfqpoint{2.940352in}{1.303054in}}{\pgfqpoint{2.929753in}{1.307444in}}{\pgfqpoint{2.918703in}{1.307444in}}%
\pgfpathcurveto{\pgfqpoint{2.907653in}{1.307444in}}{\pgfqpoint{2.897054in}{1.303054in}}{\pgfqpoint{2.889240in}{1.295240in}}%
\pgfpathcurveto{\pgfqpoint{2.881427in}{1.287427in}}{\pgfqpoint{2.877036in}{1.276827in}}{\pgfqpoint{2.877036in}{1.265777in}}%
\pgfpathcurveto{\pgfqpoint{2.877036in}{1.254727in}}{\pgfqpoint{2.881427in}{1.244128in}}{\pgfqpoint{2.889240in}{1.236315in}}%
\pgfpathcurveto{\pgfqpoint{2.897054in}{1.228501in}}{\pgfqpoint{2.907653in}{1.224111in}}{\pgfqpoint{2.918703in}{1.224111in}}%
\pgfpathclose%
\pgfusepath{stroke,fill}%
\end{pgfscope}%
\begin{pgfscope}%
\pgfpathrectangle{\pgfqpoint{0.481978in}{0.331635in}}{\pgfqpoint{9.300000in}{7.700000in}}%
\pgfusepath{clip}%
\pgfsetbuttcap%
\pgfsetroundjoin%
\definecolor{currentfill}{rgb}{0.631373,0.788235,0.956863}%
\pgfsetfillcolor{currentfill}%
\pgfsetlinewidth{0.481800pt}%
\definecolor{currentstroke}{rgb}{1.000000,1.000000,1.000000}%
\pgfsetstrokecolor{currentstroke}%
\pgfsetdash{}{0pt}%
\pgfpathmoveto{\pgfqpoint{2.176114in}{5.829516in}}%
\pgfpathcurveto{\pgfqpoint{2.187164in}{5.829516in}}{\pgfqpoint{2.197764in}{5.833906in}}{\pgfqpoint{2.205577in}{5.841720in}}%
\pgfpathcurveto{\pgfqpoint{2.213391in}{5.849534in}}{\pgfqpoint{2.217781in}{5.860133in}}{\pgfqpoint{2.217781in}{5.871183in}}%
\pgfpathcurveto{\pgfqpoint{2.217781in}{5.882233in}}{\pgfqpoint{2.213391in}{5.892832in}}{\pgfqpoint{2.205577in}{5.900646in}}%
\pgfpathcurveto{\pgfqpoint{2.197764in}{5.908459in}}{\pgfqpoint{2.187164in}{5.912849in}}{\pgfqpoint{2.176114in}{5.912849in}}%
\pgfpathcurveto{\pgfqpoint{2.165064in}{5.912849in}}{\pgfqpoint{2.154465in}{5.908459in}}{\pgfqpoint{2.146652in}{5.900646in}}%
\pgfpathcurveto{\pgfqpoint{2.138838in}{5.892832in}}{\pgfqpoint{2.134448in}{5.882233in}}{\pgfqpoint{2.134448in}{5.871183in}}%
\pgfpathcurveto{\pgfqpoint{2.134448in}{5.860133in}}{\pgfqpoint{2.138838in}{5.849534in}}{\pgfqpoint{2.146652in}{5.841720in}}%
\pgfpathcurveto{\pgfqpoint{2.154465in}{5.833906in}}{\pgfqpoint{2.165064in}{5.829516in}}{\pgfqpoint{2.176114in}{5.829516in}}%
\pgfpathclose%
\pgfusepath{stroke,fill}%
\end{pgfscope}%
\begin{pgfscope}%
\pgfpathrectangle{\pgfqpoint{0.481978in}{0.331635in}}{\pgfqpoint{9.300000in}{7.700000in}}%
\pgfusepath{clip}%
\pgfsetbuttcap%
\pgfsetroundjoin%
\definecolor{currentfill}{rgb}{0.631373,0.788235,0.956863}%
\pgfsetfillcolor{currentfill}%
\pgfsetlinewidth{0.481800pt}%
\definecolor{currentstroke}{rgb}{1.000000,1.000000,1.000000}%
\pgfsetstrokecolor{currentstroke}%
\pgfsetdash{}{0pt}%
\pgfpathmoveto{\pgfqpoint{6.830655in}{1.534579in}}%
\pgfpathcurveto{\pgfqpoint{6.841705in}{1.534579in}}{\pgfqpoint{6.852304in}{1.538970in}}{\pgfqpoint{6.860118in}{1.546783in}}%
\pgfpathcurveto{\pgfqpoint{6.867931in}{1.554597in}}{\pgfqpoint{6.872321in}{1.565196in}}{\pgfqpoint{6.872321in}{1.576246in}}%
\pgfpathcurveto{\pgfqpoint{6.872321in}{1.587296in}}{\pgfqpoint{6.867931in}{1.597895in}}{\pgfqpoint{6.860118in}{1.605709in}}%
\pgfpathcurveto{\pgfqpoint{6.852304in}{1.613523in}}{\pgfqpoint{6.841705in}{1.617913in}}{\pgfqpoint{6.830655in}{1.617913in}}%
\pgfpathcurveto{\pgfqpoint{6.819605in}{1.617913in}}{\pgfqpoint{6.809006in}{1.613523in}}{\pgfqpoint{6.801192in}{1.605709in}}%
\pgfpathcurveto{\pgfqpoint{6.793378in}{1.597895in}}{\pgfqpoint{6.788988in}{1.587296in}}{\pgfqpoint{6.788988in}{1.576246in}}%
\pgfpathcurveto{\pgfqpoint{6.788988in}{1.565196in}}{\pgfqpoint{6.793378in}{1.554597in}}{\pgfqpoint{6.801192in}{1.546783in}}%
\pgfpathcurveto{\pgfqpoint{6.809006in}{1.538970in}}{\pgfqpoint{6.819605in}{1.534579in}}{\pgfqpoint{6.830655in}{1.534579in}}%
\pgfpathclose%
\pgfusepath{stroke,fill}%
\end{pgfscope}%
\begin{pgfscope}%
\pgfpathrectangle{\pgfqpoint{0.481978in}{0.331635in}}{\pgfqpoint{9.300000in}{7.700000in}}%
\pgfusepath{clip}%
\pgfsetbuttcap%
\pgfsetroundjoin%
\definecolor{currentfill}{rgb}{0.631373,0.788235,0.956863}%
\pgfsetfillcolor{currentfill}%
\pgfsetlinewidth{0.481800pt}%
\definecolor{currentstroke}{rgb}{1.000000,1.000000,1.000000}%
\pgfsetstrokecolor{currentstroke}%
\pgfsetdash{}{0pt}%
\pgfpathmoveto{\pgfqpoint{6.368029in}{1.338243in}}%
\pgfpathcurveto{\pgfqpoint{6.379079in}{1.338243in}}{\pgfqpoint{6.389678in}{1.342633in}}{\pgfqpoint{6.397492in}{1.350447in}}%
\pgfpathcurveto{\pgfqpoint{6.405305in}{1.358260in}}{\pgfqpoint{6.409696in}{1.368859in}}{\pgfqpoint{6.409696in}{1.379909in}}%
\pgfpathcurveto{\pgfqpoint{6.409696in}{1.390959in}}{\pgfqpoint{6.405305in}{1.401558in}}{\pgfqpoint{6.397492in}{1.409372in}}%
\pgfpathcurveto{\pgfqpoint{6.389678in}{1.417186in}}{\pgfqpoint{6.379079in}{1.421576in}}{\pgfqpoint{6.368029in}{1.421576in}}%
\pgfpathcurveto{\pgfqpoint{6.356979in}{1.421576in}}{\pgfqpoint{6.346380in}{1.417186in}}{\pgfqpoint{6.338566in}{1.409372in}}%
\pgfpathcurveto{\pgfqpoint{6.330753in}{1.401558in}}{\pgfqpoint{6.326362in}{1.390959in}}{\pgfqpoint{6.326362in}{1.379909in}}%
\pgfpathcurveto{\pgfqpoint{6.326362in}{1.368859in}}{\pgfqpoint{6.330753in}{1.358260in}}{\pgfqpoint{6.338566in}{1.350447in}}%
\pgfpathcurveto{\pgfqpoint{6.346380in}{1.342633in}}{\pgfqpoint{6.356979in}{1.338243in}}{\pgfqpoint{6.368029in}{1.338243in}}%
\pgfpathclose%
\pgfusepath{stroke,fill}%
\end{pgfscope}%
\begin{pgfscope}%
\pgfpathrectangle{\pgfqpoint{0.481978in}{0.331635in}}{\pgfqpoint{9.300000in}{7.700000in}}%
\pgfusepath{clip}%
\pgfsetbuttcap%
\pgfsetroundjoin%
\definecolor{currentfill}{rgb}{0.631373,0.788235,0.956863}%
\pgfsetfillcolor{currentfill}%
\pgfsetlinewidth{0.481800pt}%
\definecolor{currentstroke}{rgb}{1.000000,1.000000,1.000000}%
\pgfsetstrokecolor{currentstroke}%
\pgfsetdash{}{0pt}%
\pgfpathmoveto{\pgfqpoint{7.095890in}{2.657676in}}%
\pgfpathcurveto{\pgfqpoint{7.106940in}{2.657676in}}{\pgfqpoint{7.117539in}{2.662066in}}{\pgfqpoint{7.125353in}{2.669880in}}%
\pgfpathcurveto{\pgfqpoint{7.133166in}{2.677693in}}{\pgfqpoint{7.137557in}{2.688293in}}{\pgfqpoint{7.137557in}{2.699343in}}%
\pgfpathcurveto{\pgfqpoint{7.137557in}{2.710393in}}{\pgfqpoint{7.133166in}{2.720992in}}{\pgfqpoint{7.125353in}{2.728805in}}%
\pgfpathcurveto{\pgfqpoint{7.117539in}{2.736619in}}{\pgfqpoint{7.106940in}{2.741009in}}{\pgfqpoint{7.095890in}{2.741009in}}%
\pgfpathcurveto{\pgfqpoint{7.084840in}{2.741009in}}{\pgfqpoint{7.074241in}{2.736619in}}{\pgfqpoint{7.066427in}{2.728805in}}%
\pgfpathcurveto{\pgfqpoint{7.058613in}{2.720992in}}{\pgfqpoint{7.054223in}{2.710393in}}{\pgfqpoint{7.054223in}{2.699343in}}%
\pgfpathcurveto{\pgfqpoint{7.054223in}{2.688293in}}{\pgfqpoint{7.058613in}{2.677693in}}{\pgfqpoint{7.066427in}{2.669880in}}%
\pgfpathcurveto{\pgfqpoint{7.074241in}{2.662066in}}{\pgfqpoint{7.084840in}{2.657676in}}{\pgfqpoint{7.095890in}{2.657676in}}%
\pgfpathclose%
\pgfusepath{stroke,fill}%
\end{pgfscope}%
\begin{pgfscope}%
\pgfpathrectangle{\pgfqpoint{0.481978in}{0.331635in}}{\pgfqpoint{9.300000in}{7.700000in}}%
\pgfusepath{clip}%
\pgfsetbuttcap%
\pgfsetroundjoin%
\definecolor{currentfill}{rgb}{0.631373,0.788235,0.956863}%
\pgfsetfillcolor{currentfill}%
\pgfsetlinewidth{0.481800pt}%
\definecolor{currentstroke}{rgb}{1.000000,1.000000,1.000000}%
\pgfsetstrokecolor{currentstroke}%
\pgfsetdash{}{0pt}%
\pgfpathmoveto{\pgfqpoint{5.900690in}{3.631899in}}%
\pgfpathcurveto{\pgfqpoint{5.911740in}{3.631899in}}{\pgfqpoint{5.922339in}{3.636289in}}{\pgfqpoint{5.930153in}{3.644103in}}%
\pgfpathcurveto{\pgfqpoint{5.937966in}{3.651916in}}{\pgfqpoint{5.942356in}{3.662515in}}{\pgfqpoint{5.942356in}{3.673565in}}%
\pgfpathcurveto{\pgfqpoint{5.942356in}{3.684616in}}{\pgfqpoint{5.937966in}{3.695215in}}{\pgfqpoint{5.930153in}{3.703028in}}%
\pgfpathcurveto{\pgfqpoint{5.922339in}{3.710842in}}{\pgfqpoint{5.911740in}{3.715232in}}{\pgfqpoint{5.900690in}{3.715232in}}%
\pgfpathcurveto{\pgfqpoint{5.889640in}{3.715232in}}{\pgfqpoint{5.879041in}{3.710842in}}{\pgfqpoint{5.871227in}{3.703028in}}%
\pgfpathcurveto{\pgfqpoint{5.863413in}{3.695215in}}{\pgfqpoint{5.859023in}{3.684616in}}{\pgfqpoint{5.859023in}{3.673565in}}%
\pgfpathcurveto{\pgfqpoint{5.859023in}{3.662515in}}{\pgfqpoint{5.863413in}{3.651916in}}{\pgfqpoint{5.871227in}{3.644103in}}%
\pgfpathcurveto{\pgfqpoint{5.879041in}{3.636289in}}{\pgfqpoint{5.889640in}{3.631899in}}{\pgfqpoint{5.900690in}{3.631899in}}%
\pgfpathclose%
\pgfusepath{stroke,fill}%
\end{pgfscope}%
\begin{pgfscope}%
\pgfpathrectangle{\pgfqpoint{0.481978in}{0.331635in}}{\pgfqpoint{9.300000in}{7.700000in}}%
\pgfusepath{clip}%
\pgfsetbuttcap%
\pgfsetroundjoin%
\definecolor{currentfill}{rgb}{0.631373,0.788235,0.956863}%
\pgfsetfillcolor{currentfill}%
\pgfsetlinewidth{0.481800pt}%
\definecolor{currentstroke}{rgb}{1.000000,1.000000,1.000000}%
\pgfsetstrokecolor{currentstroke}%
\pgfsetdash{}{0pt}%
\pgfpathmoveto{\pgfqpoint{6.724845in}{4.569221in}}%
\pgfpathcurveto{\pgfqpoint{6.735895in}{4.569221in}}{\pgfqpoint{6.746495in}{4.573611in}}{\pgfqpoint{6.754308in}{4.581424in}}%
\pgfpathcurveto{\pgfqpoint{6.762122in}{4.589238in}}{\pgfqpoint{6.766512in}{4.599837in}}{\pgfqpoint{6.766512in}{4.610887in}}%
\pgfpathcurveto{\pgfqpoint{6.766512in}{4.621937in}}{\pgfqpoint{6.762122in}{4.632536in}}{\pgfqpoint{6.754308in}{4.640350in}}%
\pgfpathcurveto{\pgfqpoint{6.746495in}{4.648164in}}{\pgfqpoint{6.735895in}{4.652554in}}{\pgfqpoint{6.724845in}{4.652554in}}%
\pgfpathcurveto{\pgfqpoint{6.713795in}{4.652554in}}{\pgfqpoint{6.703196in}{4.648164in}}{\pgfqpoint{6.695383in}{4.640350in}}%
\pgfpathcurveto{\pgfqpoint{6.687569in}{4.632536in}}{\pgfqpoint{6.683179in}{4.621937in}}{\pgfqpoint{6.683179in}{4.610887in}}%
\pgfpathcurveto{\pgfqpoint{6.683179in}{4.599837in}}{\pgfqpoint{6.687569in}{4.589238in}}{\pgfqpoint{6.695383in}{4.581424in}}%
\pgfpathcurveto{\pgfqpoint{6.703196in}{4.573611in}}{\pgfqpoint{6.713795in}{4.569221in}}{\pgfqpoint{6.724845in}{4.569221in}}%
\pgfpathclose%
\pgfusepath{stroke,fill}%
\end{pgfscope}%
\begin{pgfscope}%
\pgfpathrectangle{\pgfqpoint{0.481978in}{0.331635in}}{\pgfqpoint{9.300000in}{7.700000in}}%
\pgfusepath{clip}%
\pgfsetbuttcap%
\pgfsetroundjoin%
\definecolor{currentfill}{rgb}{0.631373,0.788235,0.956863}%
\pgfsetfillcolor{currentfill}%
\pgfsetlinewidth{0.481800pt}%
\definecolor{currentstroke}{rgb}{1.000000,1.000000,1.000000}%
\pgfsetstrokecolor{currentstroke}%
\pgfsetdash{}{0pt}%
\pgfpathmoveto{\pgfqpoint{8.573001in}{5.110832in}}%
\pgfpathcurveto{\pgfqpoint{8.584051in}{5.110832in}}{\pgfqpoint{8.594650in}{5.115222in}}{\pgfqpoint{8.602463in}{5.123036in}}%
\pgfpathcurveto{\pgfqpoint{8.610277in}{5.130849in}}{\pgfqpoint{8.614667in}{5.141448in}}{\pgfqpoint{8.614667in}{5.152498in}}%
\pgfpathcurveto{\pgfqpoint{8.614667in}{5.163549in}}{\pgfqpoint{8.610277in}{5.174148in}}{\pgfqpoint{8.602463in}{5.181961in}}%
\pgfpathcurveto{\pgfqpoint{8.594650in}{5.189775in}}{\pgfqpoint{8.584051in}{5.194165in}}{\pgfqpoint{8.573001in}{5.194165in}}%
\pgfpathcurveto{\pgfqpoint{8.561950in}{5.194165in}}{\pgfqpoint{8.551351in}{5.189775in}}{\pgfqpoint{8.543538in}{5.181961in}}%
\pgfpathcurveto{\pgfqpoint{8.535724in}{5.174148in}}{\pgfqpoint{8.531334in}{5.163549in}}{\pgfqpoint{8.531334in}{5.152498in}}%
\pgfpathcurveto{\pgfqpoint{8.531334in}{5.141448in}}{\pgfqpoint{8.535724in}{5.130849in}}{\pgfqpoint{8.543538in}{5.123036in}}%
\pgfpathcurveto{\pgfqpoint{8.551351in}{5.115222in}}{\pgfqpoint{8.561950in}{5.110832in}}{\pgfqpoint{8.573001in}{5.110832in}}%
\pgfpathclose%
\pgfusepath{stroke,fill}%
\end{pgfscope}%
\begin{pgfscope}%
\pgfpathrectangle{\pgfqpoint{0.481978in}{0.331635in}}{\pgfqpoint{9.300000in}{7.700000in}}%
\pgfusepath{clip}%
\pgfsetbuttcap%
\pgfsetroundjoin%
\definecolor{currentfill}{rgb}{0.631373,0.788235,0.956863}%
\pgfsetfillcolor{currentfill}%
\pgfsetlinewidth{0.481800pt}%
\definecolor{currentstroke}{rgb}{1.000000,1.000000,1.000000}%
\pgfsetstrokecolor{currentstroke}%
\pgfsetdash{}{0pt}%
\pgfpathmoveto{\pgfqpoint{5.469587in}{6.991376in}}%
\pgfpathcurveto{\pgfqpoint{5.480637in}{6.991376in}}{\pgfqpoint{5.491236in}{6.995766in}}{\pgfqpoint{5.499050in}{7.003579in}}%
\pgfpathcurveto{\pgfqpoint{5.506863in}{7.011393in}}{\pgfqpoint{5.511253in}{7.021992in}}{\pgfqpoint{5.511253in}{7.033042in}}%
\pgfpathcurveto{\pgfqpoint{5.511253in}{7.044092in}}{\pgfqpoint{5.506863in}{7.054691in}}{\pgfqpoint{5.499050in}{7.062505in}}%
\pgfpathcurveto{\pgfqpoint{5.491236in}{7.070319in}}{\pgfqpoint{5.480637in}{7.074709in}}{\pgfqpoint{5.469587in}{7.074709in}}%
\pgfpathcurveto{\pgfqpoint{5.458537in}{7.074709in}}{\pgfqpoint{5.447938in}{7.070319in}}{\pgfqpoint{5.440124in}{7.062505in}}%
\pgfpathcurveto{\pgfqpoint{5.432310in}{7.054691in}}{\pgfqpoint{5.427920in}{7.044092in}}{\pgfqpoint{5.427920in}{7.033042in}}%
\pgfpathcurveto{\pgfqpoint{5.427920in}{7.021992in}}{\pgfqpoint{5.432310in}{7.011393in}}{\pgfqpoint{5.440124in}{7.003579in}}%
\pgfpathcurveto{\pgfqpoint{5.447938in}{6.995766in}}{\pgfqpoint{5.458537in}{6.991376in}}{\pgfqpoint{5.469587in}{6.991376in}}%
\pgfpathclose%
\pgfusepath{stroke,fill}%
\end{pgfscope}%
\begin{pgfscope}%
\pgfpathrectangle{\pgfqpoint{0.481978in}{0.331635in}}{\pgfqpoint{9.300000in}{7.700000in}}%
\pgfusepath{clip}%
\pgfsetbuttcap%
\pgfsetroundjoin%
\definecolor{currentfill}{rgb}{0.631373,0.788235,0.956863}%
\pgfsetfillcolor{currentfill}%
\pgfsetlinewidth{0.481800pt}%
\definecolor{currentstroke}{rgb}{1.000000,1.000000,1.000000}%
\pgfsetstrokecolor{currentstroke}%
\pgfsetdash{}{0pt}%
\pgfpathmoveto{\pgfqpoint{4.990883in}{4.317080in}}%
\pgfpathcurveto{\pgfqpoint{5.001933in}{4.317080in}}{\pgfqpoint{5.012532in}{4.321470in}}{\pgfqpoint{5.020346in}{4.329284in}}%
\pgfpathcurveto{\pgfqpoint{5.028160in}{4.337097in}}{\pgfqpoint{5.032550in}{4.347696in}}{\pgfqpoint{5.032550in}{4.358746in}}%
\pgfpathcurveto{\pgfqpoint{5.032550in}{4.369796in}}{\pgfqpoint{5.028160in}{4.380395in}}{\pgfqpoint{5.020346in}{4.388209in}}%
\pgfpathcurveto{\pgfqpoint{5.012532in}{4.396023in}}{\pgfqpoint{5.001933in}{4.400413in}}{\pgfqpoint{4.990883in}{4.400413in}}%
\pgfpathcurveto{\pgfqpoint{4.979833in}{4.400413in}}{\pgfqpoint{4.969234in}{4.396023in}}{\pgfqpoint{4.961420in}{4.388209in}}%
\pgfpathcurveto{\pgfqpoint{4.953607in}{4.380395in}}{\pgfqpoint{4.949217in}{4.369796in}}{\pgfqpoint{4.949217in}{4.358746in}}%
\pgfpathcurveto{\pgfqpoint{4.949217in}{4.347696in}}{\pgfqpoint{4.953607in}{4.337097in}}{\pgfqpoint{4.961420in}{4.329284in}}%
\pgfpathcurveto{\pgfqpoint{4.969234in}{4.321470in}}{\pgfqpoint{4.979833in}{4.317080in}}{\pgfqpoint{4.990883in}{4.317080in}}%
\pgfpathclose%
\pgfusepath{stroke,fill}%
\end{pgfscope}%
\begin{pgfscope}%
\pgfpathrectangle{\pgfqpoint{0.481978in}{0.331635in}}{\pgfqpoint{9.300000in}{7.700000in}}%
\pgfusepath{clip}%
\pgfsetbuttcap%
\pgfsetroundjoin%
\definecolor{currentfill}{rgb}{0.631373,0.788235,0.956863}%
\pgfsetfillcolor{currentfill}%
\pgfsetlinewidth{0.481800pt}%
\definecolor{currentstroke}{rgb}{1.000000,1.000000,1.000000}%
\pgfsetstrokecolor{currentstroke}%
\pgfsetdash{}{0pt}%
\pgfpathmoveto{\pgfqpoint{6.589980in}{1.645898in}}%
\pgfpathcurveto{\pgfqpoint{6.601030in}{1.645898in}}{\pgfqpoint{6.611629in}{1.650288in}}{\pgfqpoint{6.619443in}{1.658101in}}%
\pgfpathcurveto{\pgfqpoint{6.627256in}{1.665915in}}{\pgfqpoint{6.631647in}{1.676514in}}{\pgfqpoint{6.631647in}{1.687564in}}%
\pgfpathcurveto{\pgfqpoint{6.631647in}{1.698614in}}{\pgfqpoint{6.627256in}{1.709213in}}{\pgfqpoint{6.619443in}{1.717027in}}%
\pgfpathcurveto{\pgfqpoint{6.611629in}{1.724841in}}{\pgfqpoint{6.601030in}{1.729231in}}{\pgfqpoint{6.589980in}{1.729231in}}%
\pgfpathcurveto{\pgfqpoint{6.578930in}{1.729231in}}{\pgfqpoint{6.568331in}{1.724841in}}{\pgfqpoint{6.560517in}{1.717027in}}%
\pgfpathcurveto{\pgfqpoint{6.552704in}{1.709213in}}{\pgfqpoint{6.548313in}{1.698614in}}{\pgfqpoint{6.548313in}{1.687564in}}%
\pgfpathcurveto{\pgfqpoint{6.548313in}{1.676514in}}{\pgfqpoint{6.552704in}{1.665915in}}{\pgfqpoint{6.560517in}{1.658101in}}%
\pgfpathcurveto{\pgfqpoint{6.568331in}{1.650288in}}{\pgfqpoint{6.578930in}{1.645898in}}{\pgfqpoint{6.589980in}{1.645898in}}%
\pgfpathclose%
\pgfusepath{stroke,fill}%
\end{pgfscope}%
\begin{pgfscope}%
\pgfpathrectangle{\pgfqpoint{0.481978in}{0.331635in}}{\pgfqpoint{9.300000in}{7.700000in}}%
\pgfusepath{clip}%
\pgfsetbuttcap%
\pgfsetroundjoin%
\definecolor{currentfill}{rgb}{0.631373,0.788235,0.956863}%
\pgfsetfillcolor{currentfill}%
\pgfsetlinewidth{0.481800pt}%
\definecolor{currentstroke}{rgb}{1.000000,1.000000,1.000000}%
\pgfsetstrokecolor{currentstroke}%
\pgfsetdash{}{0pt}%
\pgfpathmoveto{\pgfqpoint{5.765884in}{2.285447in}}%
\pgfpathcurveto{\pgfqpoint{5.776934in}{2.285447in}}{\pgfqpoint{5.787533in}{2.289837in}}{\pgfqpoint{5.795347in}{2.297651in}}%
\pgfpathcurveto{\pgfqpoint{5.803161in}{2.305464in}}{\pgfqpoint{5.807551in}{2.316063in}}{\pgfqpoint{5.807551in}{2.327113in}}%
\pgfpathcurveto{\pgfqpoint{5.807551in}{2.338163in}}{\pgfqpoint{5.803161in}{2.348762in}}{\pgfqpoint{5.795347in}{2.356576in}}%
\pgfpathcurveto{\pgfqpoint{5.787533in}{2.364390in}}{\pgfqpoint{5.776934in}{2.368780in}}{\pgfqpoint{5.765884in}{2.368780in}}%
\pgfpathcurveto{\pgfqpoint{5.754834in}{2.368780in}}{\pgfqpoint{5.744235in}{2.364390in}}{\pgfqpoint{5.736421in}{2.356576in}}%
\pgfpathcurveto{\pgfqpoint{5.728608in}{2.348762in}}{\pgfqpoint{5.724218in}{2.338163in}}{\pgfqpoint{5.724218in}{2.327113in}}%
\pgfpathcurveto{\pgfqpoint{5.724218in}{2.316063in}}{\pgfqpoint{5.728608in}{2.305464in}}{\pgfqpoint{5.736421in}{2.297651in}}%
\pgfpathcurveto{\pgfqpoint{5.744235in}{2.289837in}}{\pgfqpoint{5.754834in}{2.285447in}}{\pgfqpoint{5.765884in}{2.285447in}}%
\pgfpathclose%
\pgfusepath{stroke,fill}%
\end{pgfscope}%
\begin{pgfscope}%
\pgfpathrectangle{\pgfqpoint{0.481978in}{0.331635in}}{\pgfqpoint{9.300000in}{7.700000in}}%
\pgfusepath{clip}%
\pgfsetbuttcap%
\pgfsetroundjoin%
\definecolor{currentfill}{rgb}{0.631373,0.788235,0.956863}%
\pgfsetfillcolor{currentfill}%
\pgfsetlinewidth{0.481800pt}%
\definecolor{currentstroke}{rgb}{1.000000,1.000000,1.000000}%
\pgfsetstrokecolor{currentstroke}%
\pgfsetdash{}{0pt}%
\pgfpathmoveto{\pgfqpoint{4.097733in}{6.266669in}}%
\pgfpathcurveto{\pgfqpoint{4.108783in}{6.266669in}}{\pgfqpoint{4.119382in}{6.271059in}}{\pgfqpoint{4.127196in}{6.278873in}}%
\pgfpathcurveto{\pgfqpoint{4.135009in}{6.286687in}}{\pgfqpoint{4.139400in}{6.297286in}}{\pgfqpoint{4.139400in}{6.308336in}}%
\pgfpathcurveto{\pgfqpoint{4.139400in}{6.319386in}}{\pgfqpoint{4.135009in}{6.329985in}}{\pgfqpoint{4.127196in}{6.337799in}}%
\pgfpathcurveto{\pgfqpoint{4.119382in}{6.345612in}}{\pgfqpoint{4.108783in}{6.350003in}}{\pgfqpoint{4.097733in}{6.350003in}}%
\pgfpathcurveto{\pgfqpoint{4.086683in}{6.350003in}}{\pgfqpoint{4.076084in}{6.345612in}}{\pgfqpoint{4.068270in}{6.337799in}}%
\pgfpathcurveto{\pgfqpoint{4.060456in}{6.329985in}}{\pgfqpoint{4.056066in}{6.319386in}}{\pgfqpoint{4.056066in}{6.308336in}}%
\pgfpathcurveto{\pgfqpoint{4.056066in}{6.297286in}}{\pgfqpoint{4.060456in}{6.286687in}}{\pgfqpoint{4.068270in}{6.278873in}}%
\pgfpathcurveto{\pgfqpoint{4.076084in}{6.271059in}}{\pgfqpoint{4.086683in}{6.266669in}}{\pgfqpoint{4.097733in}{6.266669in}}%
\pgfpathclose%
\pgfusepath{stroke,fill}%
\end{pgfscope}%
\begin{pgfscope}%
\pgfpathrectangle{\pgfqpoint{0.481978in}{0.331635in}}{\pgfqpoint{9.300000in}{7.700000in}}%
\pgfusepath{clip}%
\pgfsetbuttcap%
\pgfsetroundjoin%
\definecolor{currentfill}{rgb}{0.631373,0.788235,0.956863}%
\pgfsetfillcolor{currentfill}%
\pgfsetlinewidth{0.481800pt}%
\definecolor{currentstroke}{rgb}{1.000000,1.000000,1.000000}%
\pgfsetstrokecolor{currentstroke}%
\pgfsetdash{}{0pt}%
\pgfpathmoveto{\pgfqpoint{7.842322in}{5.264362in}}%
\pgfpathcurveto{\pgfqpoint{7.853372in}{5.264362in}}{\pgfqpoint{7.863971in}{5.268752in}}{\pgfqpoint{7.871784in}{5.276566in}}%
\pgfpathcurveto{\pgfqpoint{7.879598in}{5.284379in}}{\pgfqpoint{7.883988in}{5.294978in}}{\pgfqpoint{7.883988in}{5.306029in}}%
\pgfpathcurveto{\pgfqpoint{7.883988in}{5.317079in}}{\pgfqpoint{7.879598in}{5.327678in}}{\pgfqpoint{7.871784in}{5.335491in}}%
\pgfpathcurveto{\pgfqpoint{7.863971in}{5.343305in}}{\pgfqpoint{7.853372in}{5.347695in}}{\pgfqpoint{7.842322in}{5.347695in}}%
\pgfpathcurveto{\pgfqpoint{7.831271in}{5.347695in}}{\pgfqpoint{7.820672in}{5.343305in}}{\pgfqpoint{7.812859in}{5.335491in}}%
\pgfpathcurveto{\pgfqpoint{7.805045in}{5.327678in}}{\pgfqpoint{7.800655in}{5.317079in}}{\pgfqpoint{7.800655in}{5.306029in}}%
\pgfpathcurveto{\pgfqpoint{7.800655in}{5.294978in}}{\pgfqpoint{7.805045in}{5.284379in}}{\pgfqpoint{7.812859in}{5.276566in}}%
\pgfpathcurveto{\pgfqpoint{7.820672in}{5.268752in}}{\pgfqpoint{7.831271in}{5.264362in}}{\pgfqpoint{7.842322in}{5.264362in}}%
\pgfpathclose%
\pgfusepath{stroke,fill}%
\end{pgfscope}%
\begin{pgfscope}%
\pgfpathrectangle{\pgfqpoint{0.481978in}{0.331635in}}{\pgfqpoint{9.300000in}{7.700000in}}%
\pgfusepath{clip}%
\pgfsetbuttcap%
\pgfsetroundjoin%
\definecolor{currentfill}{rgb}{0.631373,0.788235,0.956863}%
\pgfsetfillcolor{currentfill}%
\pgfsetlinewidth{0.481800pt}%
\definecolor{currentstroke}{rgb}{1.000000,1.000000,1.000000}%
\pgfsetstrokecolor{currentstroke}%
\pgfsetdash{}{0pt}%
\pgfpathmoveto{\pgfqpoint{3.960820in}{4.581563in}}%
\pgfpathcurveto{\pgfqpoint{3.971870in}{4.581563in}}{\pgfqpoint{3.982469in}{4.585953in}}{\pgfqpoint{3.990283in}{4.593767in}}%
\pgfpathcurveto{\pgfqpoint{3.998097in}{4.601581in}}{\pgfqpoint{4.002487in}{4.612180in}}{\pgfqpoint{4.002487in}{4.623230in}}%
\pgfpathcurveto{\pgfqpoint{4.002487in}{4.634280in}}{\pgfqpoint{3.998097in}{4.644879in}}{\pgfqpoint{3.990283in}{4.652693in}}%
\pgfpathcurveto{\pgfqpoint{3.982469in}{4.660506in}}{\pgfqpoint{3.971870in}{4.664896in}}{\pgfqpoint{3.960820in}{4.664896in}}%
\pgfpathcurveto{\pgfqpoint{3.949770in}{4.664896in}}{\pgfqpoint{3.939171in}{4.660506in}}{\pgfqpoint{3.931357in}{4.652693in}}%
\pgfpathcurveto{\pgfqpoint{3.923544in}{4.644879in}}{\pgfqpoint{3.919153in}{4.634280in}}{\pgfqpoint{3.919153in}{4.623230in}}%
\pgfpathcurveto{\pgfqpoint{3.919153in}{4.612180in}}{\pgfqpoint{3.923544in}{4.601581in}}{\pgfqpoint{3.931357in}{4.593767in}}%
\pgfpathcurveto{\pgfqpoint{3.939171in}{4.585953in}}{\pgfqpoint{3.949770in}{4.581563in}}{\pgfqpoint{3.960820in}{4.581563in}}%
\pgfpathclose%
\pgfusepath{stroke,fill}%
\end{pgfscope}%
\begin{pgfscope}%
\pgfpathrectangle{\pgfqpoint{0.481978in}{0.331635in}}{\pgfqpoint{9.300000in}{7.700000in}}%
\pgfusepath{clip}%
\pgfsetbuttcap%
\pgfsetroundjoin%
\definecolor{currentfill}{rgb}{0.631373,0.788235,0.956863}%
\pgfsetfillcolor{currentfill}%
\pgfsetlinewidth{0.481800pt}%
\definecolor{currentstroke}{rgb}{1.000000,1.000000,1.000000}%
\pgfsetstrokecolor{currentstroke}%
\pgfsetdash{}{0pt}%
\pgfpathmoveto{\pgfqpoint{3.426685in}{4.624289in}}%
\pgfpathcurveto{\pgfqpoint{3.437735in}{4.624289in}}{\pgfqpoint{3.448334in}{4.628679in}}{\pgfqpoint{3.456148in}{4.636493in}}%
\pgfpathcurveto{\pgfqpoint{3.463962in}{4.644307in}}{\pgfqpoint{3.468352in}{4.654906in}}{\pgfqpoint{3.468352in}{4.665956in}}%
\pgfpathcurveto{\pgfqpoint{3.468352in}{4.677006in}}{\pgfqpoint{3.463962in}{4.687605in}}{\pgfqpoint{3.456148in}{4.695418in}}%
\pgfpathcurveto{\pgfqpoint{3.448334in}{4.703232in}}{\pgfqpoint{3.437735in}{4.707622in}}{\pgfqpoint{3.426685in}{4.707622in}}%
\pgfpathcurveto{\pgfqpoint{3.415635in}{4.707622in}}{\pgfqpoint{3.405036in}{4.703232in}}{\pgfqpoint{3.397222in}{4.695418in}}%
\pgfpathcurveto{\pgfqpoint{3.389409in}{4.687605in}}{\pgfqpoint{3.385019in}{4.677006in}}{\pgfqpoint{3.385019in}{4.665956in}}%
\pgfpathcurveto{\pgfqpoint{3.385019in}{4.654906in}}{\pgfqpoint{3.389409in}{4.644307in}}{\pgfqpoint{3.397222in}{4.636493in}}%
\pgfpathcurveto{\pgfqpoint{3.405036in}{4.628679in}}{\pgfqpoint{3.415635in}{4.624289in}}{\pgfqpoint{3.426685in}{4.624289in}}%
\pgfpathclose%
\pgfusepath{stroke,fill}%
\end{pgfscope}%
\begin{pgfscope}%
\pgfpathrectangle{\pgfqpoint{0.481978in}{0.331635in}}{\pgfqpoint{9.300000in}{7.700000in}}%
\pgfusepath{clip}%
\pgfsetbuttcap%
\pgfsetroundjoin%
\definecolor{currentfill}{rgb}{0.631373,0.788235,0.956863}%
\pgfsetfillcolor{currentfill}%
\pgfsetlinewidth{0.481800pt}%
\definecolor{currentstroke}{rgb}{1.000000,1.000000,1.000000}%
\pgfsetstrokecolor{currentstroke}%
\pgfsetdash{}{0pt}%
\pgfpathmoveto{\pgfqpoint{6.445366in}{2.049583in}}%
\pgfpathcurveto{\pgfqpoint{6.456416in}{2.049583in}}{\pgfqpoint{6.467016in}{2.053973in}}{\pgfqpoint{6.474829in}{2.061786in}}%
\pgfpathcurveto{\pgfqpoint{6.482643in}{2.069600in}}{\pgfqpoint{6.487033in}{2.080199in}}{\pgfqpoint{6.487033in}{2.091249in}}%
\pgfpathcurveto{\pgfqpoint{6.487033in}{2.102299in}}{\pgfqpoint{6.482643in}{2.112898in}}{\pgfqpoint{6.474829in}{2.120712in}}%
\pgfpathcurveto{\pgfqpoint{6.467016in}{2.128526in}}{\pgfqpoint{6.456416in}{2.132916in}}{\pgfqpoint{6.445366in}{2.132916in}}%
\pgfpathcurveto{\pgfqpoint{6.434316in}{2.132916in}}{\pgfqpoint{6.423717in}{2.128526in}}{\pgfqpoint{6.415904in}{2.120712in}}%
\pgfpathcurveto{\pgfqpoint{6.408090in}{2.112898in}}{\pgfqpoint{6.403700in}{2.102299in}}{\pgfqpoint{6.403700in}{2.091249in}}%
\pgfpathcurveto{\pgfqpoint{6.403700in}{2.080199in}}{\pgfqpoint{6.408090in}{2.069600in}}{\pgfqpoint{6.415904in}{2.061786in}}%
\pgfpathcurveto{\pgfqpoint{6.423717in}{2.053973in}}{\pgfqpoint{6.434316in}{2.049583in}}{\pgfqpoint{6.445366in}{2.049583in}}%
\pgfpathclose%
\pgfusepath{stroke,fill}%
\end{pgfscope}%
\begin{pgfscope}%
\pgfpathrectangle{\pgfqpoint{0.481978in}{0.331635in}}{\pgfqpoint{9.300000in}{7.700000in}}%
\pgfusepath{clip}%
\pgfsetbuttcap%
\pgfsetroundjoin%
\definecolor{currentfill}{rgb}{0.631373,0.788235,0.956863}%
\pgfsetfillcolor{currentfill}%
\pgfsetlinewidth{0.481800pt}%
\definecolor{currentstroke}{rgb}{1.000000,1.000000,1.000000}%
\pgfsetstrokecolor{currentstroke}%
\pgfsetdash{}{0pt}%
\pgfpathmoveto{\pgfqpoint{8.216289in}{4.696577in}}%
\pgfpathcurveto{\pgfqpoint{8.227339in}{4.696577in}}{\pgfqpoint{8.237938in}{4.700967in}}{\pgfqpoint{8.245752in}{4.708781in}}%
\pgfpathcurveto{\pgfqpoint{8.253565in}{4.716595in}}{\pgfqpoint{8.257956in}{4.727194in}}{\pgfqpoint{8.257956in}{4.738244in}}%
\pgfpathcurveto{\pgfqpoint{8.257956in}{4.749294in}}{\pgfqpoint{8.253565in}{4.759893in}}{\pgfqpoint{8.245752in}{4.767707in}}%
\pgfpathcurveto{\pgfqpoint{8.237938in}{4.775520in}}{\pgfqpoint{8.227339in}{4.779910in}}{\pgfqpoint{8.216289in}{4.779910in}}%
\pgfpathcurveto{\pgfqpoint{8.205239in}{4.779910in}}{\pgfqpoint{8.194640in}{4.775520in}}{\pgfqpoint{8.186826in}{4.767707in}}%
\pgfpathcurveto{\pgfqpoint{8.179013in}{4.759893in}}{\pgfqpoint{8.174622in}{4.749294in}}{\pgfqpoint{8.174622in}{4.738244in}}%
\pgfpathcurveto{\pgfqpoint{8.174622in}{4.727194in}}{\pgfqpoint{8.179013in}{4.716595in}}{\pgfqpoint{8.186826in}{4.708781in}}%
\pgfpathcurveto{\pgfqpoint{8.194640in}{4.700967in}}{\pgfqpoint{8.205239in}{4.696577in}}{\pgfqpoint{8.216289in}{4.696577in}}%
\pgfpathclose%
\pgfusepath{stroke,fill}%
\end{pgfscope}%
\begin{pgfscope}%
\pgfpathrectangle{\pgfqpoint{0.481978in}{0.331635in}}{\pgfqpoint{9.300000in}{7.700000in}}%
\pgfusepath{clip}%
\pgfsetbuttcap%
\pgfsetroundjoin%
\definecolor{currentfill}{rgb}{0.631373,0.788235,0.956863}%
\pgfsetfillcolor{currentfill}%
\pgfsetlinewidth{0.481800pt}%
\definecolor{currentstroke}{rgb}{1.000000,1.000000,1.000000}%
\pgfsetstrokecolor{currentstroke}%
\pgfsetdash{}{0pt}%
\pgfpathmoveto{\pgfqpoint{8.236668in}{4.377435in}}%
\pgfpathcurveto{\pgfqpoint{8.247718in}{4.377435in}}{\pgfqpoint{8.258317in}{4.381826in}}{\pgfqpoint{8.266131in}{4.389639in}}%
\pgfpathcurveto{\pgfqpoint{8.273945in}{4.397453in}}{\pgfqpoint{8.278335in}{4.408052in}}{\pgfqpoint{8.278335in}{4.419102in}}%
\pgfpathcurveto{\pgfqpoint{8.278335in}{4.430152in}}{\pgfqpoint{8.273945in}{4.440751in}}{\pgfqpoint{8.266131in}{4.448565in}}%
\pgfpathcurveto{\pgfqpoint{8.258317in}{4.456379in}}{\pgfqpoint{8.247718in}{4.460769in}}{\pgfqpoint{8.236668in}{4.460769in}}%
\pgfpathcurveto{\pgfqpoint{8.225618in}{4.460769in}}{\pgfqpoint{8.215019in}{4.456379in}}{\pgfqpoint{8.207205in}{4.448565in}}%
\pgfpathcurveto{\pgfqpoint{8.199392in}{4.440751in}}{\pgfqpoint{8.195001in}{4.430152in}}{\pgfqpoint{8.195001in}{4.419102in}}%
\pgfpathcurveto{\pgfqpoint{8.195001in}{4.408052in}}{\pgfqpoint{8.199392in}{4.397453in}}{\pgfqpoint{8.207205in}{4.389639in}}%
\pgfpathcurveto{\pgfqpoint{8.215019in}{4.381826in}}{\pgfqpoint{8.225618in}{4.377435in}}{\pgfqpoint{8.236668in}{4.377435in}}%
\pgfpathclose%
\pgfusepath{stroke,fill}%
\end{pgfscope}%
\begin{pgfscope}%
\pgfpathrectangle{\pgfqpoint{0.481978in}{0.331635in}}{\pgfqpoint{9.300000in}{7.700000in}}%
\pgfusepath{clip}%
\pgfsetbuttcap%
\pgfsetroundjoin%
\definecolor{currentfill}{rgb}{0.631373,0.788235,0.956863}%
\pgfsetfillcolor{currentfill}%
\pgfsetlinewidth{0.481800pt}%
\definecolor{currentstroke}{rgb}{1.000000,1.000000,1.000000}%
\pgfsetstrokecolor{currentstroke}%
\pgfsetdash{}{0pt}%
\pgfpathmoveto{\pgfqpoint{7.008316in}{2.323870in}}%
\pgfpathcurveto{\pgfqpoint{7.019366in}{2.323870in}}{\pgfqpoint{7.029965in}{2.328260in}}{\pgfqpoint{7.037779in}{2.336074in}}%
\pgfpathcurveto{\pgfqpoint{7.045593in}{2.343887in}}{\pgfqpoint{7.049983in}{2.354486in}}{\pgfqpoint{7.049983in}{2.365537in}}%
\pgfpathcurveto{\pgfqpoint{7.049983in}{2.376587in}}{\pgfqpoint{7.045593in}{2.387186in}}{\pgfqpoint{7.037779in}{2.394999in}}%
\pgfpathcurveto{\pgfqpoint{7.029965in}{2.402813in}}{\pgfqpoint{7.019366in}{2.407203in}}{\pgfqpoint{7.008316in}{2.407203in}}%
\pgfpathcurveto{\pgfqpoint{6.997266in}{2.407203in}}{\pgfqpoint{6.986667in}{2.402813in}}{\pgfqpoint{6.978853in}{2.394999in}}%
\pgfpathcurveto{\pgfqpoint{6.971040in}{2.387186in}}{\pgfqpoint{6.966650in}{2.376587in}}{\pgfqpoint{6.966650in}{2.365537in}}%
\pgfpathcurveto{\pgfqpoint{6.966650in}{2.354486in}}{\pgfqpoint{6.971040in}{2.343887in}}{\pgfqpoint{6.978853in}{2.336074in}}%
\pgfpathcurveto{\pgfqpoint{6.986667in}{2.328260in}}{\pgfqpoint{6.997266in}{2.323870in}}{\pgfqpoint{7.008316in}{2.323870in}}%
\pgfpathclose%
\pgfusepath{stroke,fill}%
\end{pgfscope}%
\begin{pgfscope}%
\pgfpathrectangle{\pgfqpoint{0.481978in}{0.331635in}}{\pgfqpoint{9.300000in}{7.700000in}}%
\pgfusepath{clip}%
\pgfsetbuttcap%
\pgfsetroundjoin%
\definecolor{currentfill}{rgb}{0.631373,0.788235,0.956863}%
\pgfsetfillcolor{currentfill}%
\pgfsetlinewidth{0.481800pt}%
\definecolor{currentstroke}{rgb}{1.000000,1.000000,1.000000}%
\pgfsetstrokecolor{currentstroke}%
\pgfsetdash{}{0pt}%
\pgfpathmoveto{\pgfqpoint{8.317641in}{5.122719in}}%
\pgfpathcurveto{\pgfqpoint{8.328691in}{5.122719in}}{\pgfqpoint{8.339290in}{5.127109in}}{\pgfqpoint{8.347104in}{5.134923in}}%
\pgfpathcurveto{\pgfqpoint{8.354917in}{5.142736in}}{\pgfqpoint{8.359307in}{5.153335in}}{\pgfqpoint{8.359307in}{5.164385in}}%
\pgfpathcurveto{\pgfqpoint{8.359307in}{5.175435in}}{\pgfqpoint{8.354917in}{5.186034in}}{\pgfqpoint{8.347104in}{5.193848in}}%
\pgfpathcurveto{\pgfqpoint{8.339290in}{5.201662in}}{\pgfqpoint{8.328691in}{5.206052in}}{\pgfqpoint{8.317641in}{5.206052in}}%
\pgfpathcurveto{\pgfqpoint{8.306591in}{5.206052in}}{\pgfqpoint{8.295992in}{5.201662in}}{\pgfqpoint{8.288178in}{5.193848in}}%
\pgfpathcurveto{\pgfqpoint{8.280364in}{5.186034in}}{\pgfqpoint{8.275974in}{5.175435in}}{\pgfqpoint{8.275974in}{5.164385in}}%
\pgfpathcurveto{\pgfqpoint{8.275974in}{5.153335in}}{\pgfqpoint{8.280364in}{5.142736in}}{\pgfqpoint{8.288178in}{5.134923in}}%
\pgfpathcurveto{\pgfqpoint{8.295992in}{5.127109in}}{\pgfqpoint{8.306591in}{5.122719in}}{\pgfqpoint{8.317641in}{5.122719in}}%
\pgfpathclose%
\pgfusepath{stroke,fill}%
\end{pgfscope}%
\begin{pgfscope}%
\pgfpathrectangle{\pgfqpoint{0.481978in}{0.331635in}}{\pgfqpoint{9.300000in}{7.700000in}}%
\pgfusepath{clip}%
\pgfsetbuttcap%
\pgfsetroundjoin%
\definecolor{currentfill}{rgb}{0.631373,0.788235,0.956863}%
\pgfsetfillcolor{currentfill}%
\pgfsetlinewidth{0.481800pt}%
\definecolor{currentstroke}{rgb}{1.000000,1.000000,1.000000}%
\pgfsetstrokecolor{currentstroke}%
\pgfsetdash{}{0pt}%
\pgfpathmoveto{\pgfqpoint{7.082145in}{1.699061in}}%
\pgfpathcurveto{\pgfqpoint{7.093195in}{1.699061in}}{\pgfqpoint{7.103794in}{1.703451in}}{\pgfqpoint{7.111608in}{1.711265in}}%
\pgfpathcurveto{\pgfqpoint{7.119422in}{1.719078in}}{\pgfqpoint{7.123812in}{1.729677in}}{\pgfqpoint{7.123812in}{1.740727in}}%
\pgfpathcurveto{\pgfqpoint{7.123812in}{1.751777in}}{\pgfqpoint{7.119422in}{1.762376in}}{\pgfqpoint{7.111608in}{1.770190in}}%
\pgfpathcurveto{\pgfqpoint{7.103794in}{1.778004in}}{\pgfqpoint{7.093195in}{1.782394in}}{\pgfqpoint{7.082145in}{1.782394in}}%
\pgfpathcurveto{\pgfqpoint{7.071095in}{1.782394in}}{\pgfqpoint{7.060496in}{1.778004in}}{\pgfqpoint{7.052683in}{1.770190in}}%
\pgfpathcurveto{\pgfqpoint{7.044869in}{1.762376in}}{\pgfqpoint{7.040479in}{1.751777in}}{\pgfqpoint{7.040479in}{1.740727in}}%
\pgfpathcurveto{\pgfqpoint{7.040479in}{1.729677in}}{\pgfqpoint{7.044869in}{1.719078in}}{\pgfqpoint{7.052683in}{1.711265in}}%
\pgfpathcurveto{\pgfqpoint{7.060496in}{1.703451in}}{\pgfqpoint{7.071095in}{1.699061in}}{\pgfqpoint{7.082145in}{1.699061in}}%
\pgfpathclose%
\pgfusepath{stroke,fill}%
\end{pgfscope}%
\begin{pgfscope}%
\pgfpathrectangle{\pgfqpoint{0.481978in}{0.331635in}}{\pgfqpoint{9.300000in}{7.700000in}}%
\pgfusepath{clip}%
\pgfsetbuttcap%
\pgfsetroundjoin%
\definecolor{currentfill}{rgb}{0.631373,0.788235,0.956863}%
\pgfsetfillcolor{currentfill}%
\pgfsetlinewidth{0.481800pt}%
\definecolor{currentstroke}{rgb}{1.000000,1.000000,1.000000}%
\pgfsetstrokecolor{currentstroke}%
\pgfsetdash{}{0pt}%
\pgfpathmoveto{\pgfqpoint{5.724298in}{1.762569in}}%
\pgfpathcurveto{\pgfqpoint{5.735348in}{1.762569in}}{\pgfqpoint{5.745947in}{1.766959in}}{\pgfqpoint{5.753761in}{1.774773in}}%
\pgfpathcurveto{\pgfqpoint{5.761575in}{1.782586in}}{\pgfqpoint{5.765965in}{1.793185in}}{\pgfqpoint{5.765965in}{1.804236in}}%
\pgfpathcurveto{\pgfqpoint{5.765965in}{1.815286in}}{\pgfqpoint{5.761575in}{1.825885in}}{\pgfqpoint{5.753761in}{1.833698in}}%
\pgfpathcurveto{\pgfqpoint{5.745947in}{1.841512in}}{\pgfqpoint{5.735348in}{1.845902in}}{\pgfqpoint{5.724298in}{1.845902in}}%
\pgfpathcurveto{\pgfqpoint{5.713248in}{1.845902in}}{\pgfqpoint{5.702649in}{1.841512in}}{\pgfqpoint{5.694835in}{1.833698in}}%
\pgfpathcurveto{\pgfqpoint{5.687022in}{1.825885in}}{\pgfqpoint{5.682632in}{1.815286in}}{\pgfqpoint{5.682632in}{1.804236in}}%
\pgfpathcurveto{\pgfqpoint{5.682632in}{1.793185in}}{\pgfqpoint{5.687022in}{1.782586in}}{\pgfqpoint{5.694835in}{1.774773in}}%
\pgfpathcurveto{\pgfqpoint{5.702649in}{1.766959in}}{\pgfqpoint{5.713248in}{1.762569in}}{\pgfqpoint{5.724298in}{1.762569in}}%
\pgfpathclose%
\pgfusepath{stroke,fill}%
\end{pgfscope}%
\begin{pgfscope}%
\pgfpathrectangle{\pgfqpoint{0.481978in}{0.331635in}}{\pgfqpoint{9.300000in}{7.700000in}}%
\pgfusepath{clip}%
\pgfsetbuttcap%
\pgfsetroundjoin%
\definecolor{currentfill}{rgb}{0.631373,0.788235,0.956863}%
\pgfsetfillcolor{currentfill}%
\pgfsetlinewidth{0.481800pt}%
\definecolor{currentstroke}{rgb}{1.000000,1.000000,1.000000}%
\pgfsetstrokecolor{currentstroke}%
\pgfsetdash{}{0pt}%
\pgfpathmoveto{\pgfqpoint{2.561461in}{1.982774in}}%
\pgfpathcurveto{\pgfqpoint{2.572512in}{1.982774in}}{\pgfqpoint{2.583111in}{1.987164in}}{\pgfqpoint{2.590924in}{1.994978in}}%
\pgfpathcurveto{\pgfqpoint{2.598738in}{2.002791in}}{\pgfqpoint{2.603128in}{2.013390in}}{\pgfqpoint{2.603128in}{2.024440in}}%
\pgfpathcurveto{\pgfqpoint{2.603128in}{2.035491in}}{\pgfqpoint{2.598738in}{2.046090in}}{\pgfqpoint{2.590924in}{2.053903in}}%
\pgfpathcurveto{\pgfqpoint{2.583111in}{2.061717in}}{\pgfqpoint{2.572512in}{2.066107in}}{\pgfqpoint{2.561461in}{2.066107in}}%
\pgfpathcurveto{\pgfqpoint{2.550411in}{2.066107in}}{\pgfqpoint{2.539812in}{2.061717in}}{\pgfqpoint{2.531999in}{2.053903in}}%
\pgfpathcurveto{\pgfqpoint{2.524185in}{2.046090in}}{\pgfqpoint{2.519795in}{2.035491in}}{\pgfqpoint{2.519795in}{2.024440in}}%
\pgfpathcurveto{\pgfqpoint{2.519795in}{2.013390in}}{\pgfqpoint{2.524185in}{2.002791in}}{\pgfqpoint{2.531999in}{1.994978in}}%
\pgfpathcurveto{\pgfqpoint{2.539812in}{1.987164in}}{\pgfqpoint{2.550411in}{1.982774in}}{\pgfqpoint{2.561461in}{1.982774in}}%
\pgfpathclose%
\pgfusepath{stroke,fill}%
\end{pgfscope}%
\begin{pgfscope}%
\pgfpathrectangle{\pgfqpoint{0.481978in}{0.331635in}}{\pgfqpoint{9.300000in}{7.700000in}}%
\pgfusepath{clip}%
\pgfsetbuttcap%
\pgfsetroundjoin%
\definecolor{currentfill}{rgb}{0.631373,0.788235,0.956863}%
\pgfsetfillcolor{currentfill}%
\pgfsetlinewidth{0.481800pt}%
\definecolor{currentstroke}{rgb}{1.000000,1.000000,1.000000}%
\pgfsetstrokecolor{currentstroke}%
\pgfsetdash{}{0pt}%
\pgfpathmoveto{\pgfqpoint{6.177807in}{1.835767in}}%
\pgfpathcurveto{\pgfqpoint{6.188857in}{1.835767in}}{\pgfqpoint{6.199456in}{1.840157in}}{\pgfqpoint{6.207270in}{1.847971in}}%
\pgfpathcurveto{\pgfqpoint{6.215084in}{1.855785in}}{\pgfqpoint{6.219474in}{1.866384in}}{\pgfqpoint{6.219474in}{1.877434in}}%
\pgfpathcurveto{\pgfqpoint{6.219474in}{1.888484in}}{\pgfqpoint{6.215084in}{1.899083in}}{\pgfqpoint{6.207270in}{1.906897in}}%
\pgfpathcurveto{\pgfqpoint{6.199456in}{1.914710in}}{\pgfqpoint{6.188857in}{1.919100in}}{\pgfqpoint{6.177807in}{1.919100in}}%
\pgfpathcurveto{\pgfqpoint{6.166757in}{1.919100in}}{\pgfqpoint{6.156158in}{1.914710in}}{\pgfqpoint{6.148344in}{1.906897in}}%
\pgfpathcurveto{\pgfqpoint{6.140531in}{1.899083in}}{\pgfqpoint{6.136141in}{1.888484in}}{\pgfqpoint{6.136141in}{1.877434in}}%
\pgfpathcurveto{\pgfqpoint{6.136141in}{1.866384in}}{\pgfqpoint{6.140531in}{1.855785in}}{\pgfqpoint{6.148344in}{1.847971in}}%
\pgfpathcurveto{\pgfqpoint{6.156158in}{1.840157in}}{\pgfqpoint{6.166757in}{1.835767in}}{\pgfqpoint{6.177807in}{1.835767in}}%
\pgfpathclose%
\pgfusepath{stroke,fill}%
\end{pgfscope}%
\begin{pgfscope}%
\pgfpathrectangle{\pgfqpoint{0.481978in}{0.331635in}}{\pgfqpoint{9.300000in}{7.700000in}}%
\pgfusepath{clip}%
\pgfsetbuttcap%
\pgfsetroundjoin%
\definecolor{currentfill}{rgb}{0.631373,0.788235,0.956863}%
\pgfsetfillcolor{currentfill}%
\pgfsetlinewidth{0.481800pt}%
\definecolor{currentstroke}{rgb}{1.000000,1.000000,1.000000}%
\pgfsetstrokecolor{currentstroke}%
\pgfsetdash{}{0pt}%
\pgfpathmoveto{\pgfqpoint{6.336424in}{0.639968in}}%
\pgfpathcurveto{\pgfqpoint{6.347474in}{0.639968in}}{\pgfqpoint{6.358073in}{0.644359in}}{\pgfqpoint{6.365887in}{0.652172in}}%
\pgfpathcurveto{\pgfqpoint{6.373700in}{0.659986in}}{\pgfqpoint{6.378091in}{0.670585in}}{\pgfqpoint{6.378091in}{0.681635in}}%
\pgfpathcurveto{\pgfqpoint{6.378091in}{0.692685in}}{\pgfqpoint{6.373700in}{0.703284in}}{\pgfqpoint{6.365887in}{0.711098in}}%
\pgfpathcurveto{\pgfqpoint{6.358073in}{0.718911in}}{\pgfqpoint{6.347474in}{0.723302in}}{\pgfqpoint{6.336424in}{0.723302in}}%
\pgfpathcurveto{\pgfqpoint{6.325374in}{0.723302in}}{\pgfqpoint{6.314775in}{0.718911in}}{\pgfqpoint{6.306961in}{0.711098in}}%
\pgfpathcurveto{\pgfqpoint{6.299148in}{0.703284in}}{\pgfqpoint{6.294757in}{0.692685in}}{\pgfqpoint{6.294757in}{0.681635in}}%
\pgfpathcurveto{\pgfqpoint{6.294757in}{0.670585in}}{\pgfqpoint{6.299148in}{0.659986in}}{\pgfqpoint{6.306961in}{0.652172in}}%
\pgfpathcurveto{\pgfqpoint{6.314775in}{0.644359in}}{\pgfqpoint{6.325374in}{0.639968in}}{\pgfqpoint{6.336424in}{0.639968in}}%
\pgfpathclose%
\pgfusepath{stroke,fill}%
\end{pgfscope}%
\begin{pgfscope}%
\pgfpathrectangle{\pgfqpoint{0.481978in}{0.331635in}}{\pgfqpoint{9.300000in}{7.700000in}}%
\pgfusepath{clip}%
\pgfsetbuttcap%
\pgfsetroundjoin%
\definecolor{currentfill}{rgb}{0.631373,0.788235,0.956863}%
\pgfsetfillcolor{currentfill}%
\pgfsetlinewidth{0.481800pt}%
\definecolor{currentstroke}{rgb}{1.000000,1.000000,1.000000}%
\pgfsetstrokecolor{currentstroke}%
\pgfsetdash{}{0pt}%
\pgfpathmoveto{\pgfqpoint{7.682062in}{5.911755in}}%
\pgfpathcurveto{\pgfqpoint{7.693112in}{5.911755in}}{\pgfqpoint{7.703711in}{5.916145in}}{\pgfqpoint{7.711524in}{5.923959in}}%
\pgfpathcurveto{\pgfqpoint{7.719338in}{5.931772in}}{\pgfqpoint{7.723728in}{5.942371in}}{\pgfqpoint{7.723728in}{5.953421in}}%
\pgfpathcurveto{\pgfqpoint{7.723728in}{5.964471in}}{\pgfqpoint{7.719338in}{5.975071in}}{\pgfqpoint{7.711524in}{5.982884in}}%
\pgfpathcurveto{\pgfqpoint{7.703711in}{5.990698in}}{\pgfqpoint{7.693112in}{5.995088in}}{\pgfqpoint{7.682062in}{5.995088in}}%
\pgfpathcurveto{\pgfqpoint{7.671011in}{5.995088in}}{\pgfqpoint{7.660412in}{5.990698in}}{\pgfqpoint{7.652599in}{5.982884in}}%
\pgfpathcurveto{\pgfqpoint{7.644785in}{5.975071in}}{\pgfqpoint{7.640395in}{5.964471in}}{\pgfqpoint{7.640395in}{5.953421in}}%
\pgfpathcurveto{\pgfqpoint{7.640395in}{5.942371in}}{\pgfqpoint{7.644785in}{5.931772in}}{\pgfqpoint{7.652599in}{5.923959in}}%
\pgfpathcurveto{\pgfqpoint{7.660412in}{5.916145in}}{\pgfqpoint{7.671011in}{5.911755in}}{\pgfqpoint{7.682062in}{5.911755in}}%
\pgfpathclose%
\pgfusepath{stroke,fill}%
\end{pgfscope}%
\begin{pgfscope}%
\pgfpathrectangle{\pgfqpoint{0.481978in}{0.331635in}}{\pgfqpoint{9.300000in}{7.700000in}}%
\pgfusepath{clip}%
\pgfsetbuttcap%
\pgfsetroundjoin%
\definecolor{currentfill}{rgb}{0.631373,0.788235,0.956863}%
\pgfsetfillcolor{currentfill}%
\pgfsetlinewidth{0.481800pt}%
\definecolor{currentstroke}{rgb}{1.000000,1.000000,1.000000}%
\pgfsetstrokecolor{currentstroke}%
\pgfsetdash{}{0pt}%
\pgfpathmoveto{\pgfqpoint{6.477636in}{3.838879in}}%
\pgfpathcurveto{\pgfqpoint{6.488687in}{3.838879in}}{\pgfqpoint{6.499286in}{3.843269in}}{\pgfqpoint{6.507099in}{3.851083in}}%
\pgfpathcurveto{\pgfqpoint{6.514913in}{3.858896in}}{\pgfqpoint{6.519303in}{3.869495in}}{\pgfqpoint{6.519303in}{3.880545in}}%
\pgfpathcurveto{\pgfqpoint{6.519303in}{3.891596in}}{\pgfqpoint{6.514913in}{3.902195in}}{\pgfqpoint{6.507099in}{3.910008in}}%
\pgfpathcurveto{\pgfqpoint{6.499286in}{3.917822in}}{\pgfqpoint{6.488687in}{3.922212in}}{\pgfqpoint{6.477636in}{3.922212in}}%
\pgfpathcurveto{\pgfqpoint{6.466586in}{3.922212in}}{\pgfqpoint{6.455987in}{3.917822in}}{\pgfqpoint{6.448174in}{3.910008in}}%
\pgfpathcurveto{\pgfqpoint{6.440360in}{3.902195in}}{\pgfqpoint{6.435970in}{3.891596in}}{\pgfqpoint{6.435970in}{3.880545in}}%
\pgfpathcurveto{\pgfqpoint{6.435970in}{3.869495in}}{\pgfqpoint{6.440360in}{3.858896in}}{\pgfqpoint{6.448174in}{3.851083in}}%
\pgfpathcurveto{\pgfqpoint{6.455987in}{3.843269in}}{\pgfqpoint{6.466586in}{3.838879in}}{\pgfqpoint{6.477636in}{3.838879in}}%
\pgfpathclose%
\pgfusepath{stroke,fill}%
\end{pgfscope}%
\begin{pgfscope}%
\pgfpathrectangle{\pgfqpoint{0.481978in}{0.331635in}}{\pgfqpoint{9.300000in}{7.700000in}}%
\pgfusepath{clip}%
\pgfsetbuttcap%
\pgfsetroundjoin%
\definecolor{currentfill}{rgb}{0.631373,0.788235,0.956863}%
\pgfsetfillcolor{currentfill}%
\pgfsetlinewidth{0.481800pt}%
\definecolor{currentstroke}{rgb}{1.000000,1.000000,1.000000}%
\pgfsetstrokecolor{currentstroke}%
\pgfsetdash{}{0pt}%
\pgfpathmoveto{\pgfqpoint{6.604622in}{2.909109in}}%
\pgfpathcurveto{\pgfqpoint{6.615672in}{2.909109in}}{\pgfqpoint{6.626271in}{2.913500in}}{\pgfqpoint{6.634085in}{2.921313in}}%
\pgfpathcurveto{\pgfqpoint{6.641898in}{2.929127in}}{\pgfqpoint{6.646289in}{2.939726in}}{\pgfqpoint{6.646289in}{2.950776in}}%
\pgfpathcurveto{\pgfqpoint{6.646289in}{2.961826in}}{\pgfqpoint{6.641898in}{2.972425in}}{\pgfqpoint{6.634085in}{2.980239in}}%
\pgfpathcurveto{\pgfqpoint{6.626271in}{2.988052in}}{\pgfqpoint{6.615672in}{2.992443in}}{\pgfqpoint{6.604622in}{2.992443in}}%
\pgfpathcurveto{\pgfqpoint{6.593572in}{2.992443in}}{\pgfqpoint{6.582973in}{2.988052in}}{\pgfqpoint{6.575159in}{2.980239in}}%
\pgfpathcurveto{\pgfqpoint{6.567346in}{2.972425in}}{\pgfqpoint{6.562955in}{2.961826in}}{\pgfqpoint{6.562955in}{2.950776in}}%
\pgfpathcurveto{\pgfqpoint{6.562955in}{2.939726in}}{\pgfqpoint{6.567346in}{2.929127in}}{\pgfqpoint{6.575159in}{2.921313in}}%
\pgfpathcurveto{\pgfqpoint{6.582973in}{2.913500in}}{\pgfqpoint{6.593572in}{2.909109in}}{\pgfqpoint{6.604622in}{2.909109in}}%
\pgfpathclose%
\pgfusepath{stroke,fill}%
\end{pgfscope}%
\begin{pgfscope}%
\pgfpathrectangle{\pgfqpoint{0.481978in}{0.331635in}}{\pgfqpoint{9.300000in}{7.700000in}}%
\pgfusepath{clip}%
\pgfsetbuttcap%
\pgfsetroundjoin%
\definecolor{currentfill}{rgb}{0.631373,0.788235,0.956863}%
\pgfsetfillcolor{currentfill}%
\pgfsetlinewidth{0.481800pt}%
\definecolor{currentstroke}{rgb}{1.000000,1.000000,1.000000}%
\pgfsetstrokecolor{currentstroke}%
\pgfsetdash{}{0pt}%
\pgfpathmoveto{\pgfqpoint{7.441282in}{2.667306in}}%
\pgfpathcurveto{\pgfqpoint{7.452332in}{2.667306in}}{\pgfqpoint{7.462931in}{2.671696in}}{\pgfqpoint{7.470745in}{2.679510in}}%
\pgfpathcurveto{\pgfqpoint{7.478558in}{2.687323in}}{\pgfqpoint{7.482948in}{2.697922in}}{\pgfqpoint{7.482948in}{2.708973in}}%
\pgfpathcurveto{\pgfqpoint{7.482948in}{2.720023in}}{\pgfqpoint{7.478558in}{2.730622in}}{\pgfqpoint{7.470745in}{2.738435in}}%
\pgfpathcurveto{\pgfqpoint{7.462931in}{2.746249in}}{\pgfqpoint{7.452332in}{2.750639in}}{\pgfqpoint{7.441282in}{2.750639in}}%
\pgfpathcurveto{\pgfqpoint{7.430232in}{2.750639in}}{\pgfqpoint{7.419633in}{2.746249in}}{\pgfqpoint{7.411819in}{2.738435in}}%
\pgfpathcurveto{\pgfqpoint{7.404005in}{2.730622in}}{\pgfqpoint{7.399615in}{2.720023in}}{\pgfqpoint{7.399615in}{2.708973in}}%
\pgfpathcurveto{\pgfqpoint{7.399615in}{2.697922in}}{\pgfqpoint{7.404005in}{2.687323in}}{\pgfqpoint{7.411819in}{2.679510in}}%
\pgfpathcurveto{\pgfqpoint{7.419633in}{2.671696in}}{\pgfqpoint{7.430232in}{2.667306in}}{\pgfqpoint{7.441282in}{2.667306in}}%
\pgfpathclose%
\pgfusepath{stroke,fill}%
\end{pgfscope}%
\begin{pgfscope}%
\pgfpathrectangle{\pgfqpoint{0.481978in}{0.331635in}}{\pgfqpoint{9.300000in}{7.700000in}}%
\pgfusepath{clip}%
\pgfsetbuttcap%
\pgfsetroundjoin%
\definecolor{currentfill}{rgb}{0.631373,0.788235,0.956863}%
\pgfsetfillcolor{currentfill}%
\pgfsetlinewidth{0.481800pt}%
\definecolor{currentstroke}{rgb}{1.000000,1.000000,1.000000}%
\pgfsetstrokecolor{currentstroke}%
\pgfsetdash{}{0pt}%
\pgfpathmoveto{\pgfqpoint{5.021448in}{5.556371in}}%
\pgfpathcurveto{\pgfqpoint{5.032498in}{5.556371in}}{\pgfqpoint{5.043097in}{5.560761in}}{\pgfqpoint{5.050910in}{5.568575in}}%
\pgfpathcurveto{\pgfqpoint{5.058724in}{5.576389in}}{\pgfqpoint{5.063114in}{5.586988in}}{\pgfqpoint{5.063114in}{5.598038in}}%
\pgfpathcurveto{\pgfqpoint{5.063114in}{5.609088in}}{\pgfqpoint{5.058724in}{5.619687in}}{\pgfqpoint{5.050910in}{5.627501in}}%
\pgfpathcurveto{\pgfqpoint{5.043097in}{5.635314in}}{\pgfqpoint{5.032498in}{5.639704in}}{\pgfqpoint{5.021448in}{5.639704in}}%
\pgfpathcurveto{\pgfqpoint{5.010398in}{5.639704in}}{\pgfqpoint{4.999799in}{5.635314in}}{\pgfqpoint{4.991985in}{5.627501in}}%
\pgfpathcurveto{\pgfqpoint{4.984171in}{5.619687in}}{\pgfqpoint{4.979781in}{5.609088in}}{\pgfqpoint{4.979781in}{5.598038in}}%
\pgfpathcurveto{\pgfqpoint{4.979781in}{5.586988in}}{\pgfqpoint{4.984171in}{5.576389in}}{\pgfqpoint{4.991985in}{5.568575in}}%
\pgfpathcurveto{\pgfqpoint{4.999799in}{5.560761in}}{\pgfqpoint{5.010398in}{5.556371in}}{\pgfqpoint{5.021448in}{5.556371in}}%
\pgfpathclose%
\pgfusepath{stroke,fill}%
\end{pgfscope}%
\begin{pgfscope}%
\pgfpathrectangle{\pgfqpoint{0.481978in}{0.331635in}}{\pgfqpoint{9.300000in}{7.700000in}}%
\pgfusepath{clip}%
\pgfsetbuttcap%
\pgfsetroundjoin%
\definecolor{currentfill}{rgb}{0.631373,0.788235,0.956863}%
\pgfsetfillcolor{currentfill}%
\pgfsetlinewidth{0.481800pt}%
\definecolor{currentstroke}{rgb}{1.000000,1.000000,1.000000}%
\pgfsetstrokecolor{currentstroke}%
\pgfsetdash{}{0pt}%
\pgfpathmoveto{\pgfqpoint{6.840343in}{4.402227in}}%
\pgfpathcurveto{\pgfqpoint{6.851393in}{4.402227in}}{\pgfqpoint{6.861992in}{4.406617in}}{\pgfqpoint{6.869806in}{4.414431in}}%
\pgfpathcurveto{\pgfqpoint{6.877619in}{4.422244in}}{\pgfqpoint{6.882010in}{4.432843in}}{\pgfqpoint{6.882010in}{4.443893in}}%
\pgfpathcurveto{\pgfqpoint{6.882010in}{4.454943in}}{\pgfqpoint{6.877619in}{4.465542in}}{\pgfqpoint{6.869806in}{4.473356in}}%
\pgfpathcurveto{\pgfqpoint{6.861992in}{4.481170in}}{\pgfqpoint{6.851393in}{4.485560in}}{\pgfqpoint{6.840343in}{4.485560in}}%
\pgfpathcurveto{\pgfqpoint{6.829293in}{4.485560in}}{\pgfqpoint{6.818694in}{4.481170in}}{\pgfqpoint{6.810880in}{4.473356in}}%
\pgfpathcurveto{\pgfqpoint{6.803066in}{4.465542in}}{\pgfqpoint{6.798676in}{4.454943in}}{\pgfqpoint{6.798676in}{4.443893in}}%
\pgfpathcurveto{\pgfqpoint{6.798676in}{4.432843in}}{\pgfqpoint{6.803066in}{4.422244in}}{\pgfqpoint{6.810880in}{4.414431in}}%
\pgfpathcurveto{\pgfqpoint{6.818694in}{4.406617in}}{\pgfqpoint{6.829293in}{4.402227in}}{\pgfqpoint{6.840343in}{4.402227in}}%
\pgfpathclose%
\pgfusepath{stroke,fill}%
\end{pgfscope}%
\begin{pgfscope}%
\pgfpathrectangle{\pgfqpoint{0.481978in}{0.331635in}}{\pgfqpoint{9.300000in}{7.700000in}}%
\pgfusepath{clip}%
\pgfsetbuttcap%
\pgfsetroundjoin%
\definecolor{currentfill}{rgb}{0.631373,0.788235,0.956863}%
\pgfsetfillcolor{currentfill}%
\pgfsetlinewidth{0.481800pt}%
\definecolor{currentstroke}{rgb}{1.000000,1.000000,1.000000}%
\pgfsetstrokecolor{currentstroke}%
\pgfsetdash{}{0pt}%
\pgfpathmoveto{\pgfqpoint{6.128623in}{5.960400in}}%
\pgfpathcurveto{\pgfqpoint{6.139673in}{5.960400in}}{\pgfqpoint{6.150272in}{5.964790in}}{\pgfqpoint{6.158086in}{5.972603in}}%
\pgfpathcurveto{\pgfqpoint{6.165900in}{5.980417in}}{\pgfqpoint{6.170290in}{5.991016in}}{\pgfqpoint{6.170290in}{6.002066in}}%
\pgfpathcurveto{\pgfqpoint{6.170290in}{6.013116in}}{\pgfqpoint{6.165900in}{6.023715in}}{\pgfqpoint{6.158086in}{6.031529in}}%
\pgfpathcurveto{\pgfqpoint{6.150272in}{6.039343in}}{\pgfqpoint{6.139673in}{6.043733in}}{\pgfqpoint{6.128623in}{6.043733in}}%
\pgfpathcurveto{\pgfqpoint{6.117573in}{6.043733in}}{\pgfqpoint{6.106974in}{6.039343in}}{\pgfqpoint{6.099160in}{6.031529in}}%
\pgfpathcurveto{\pgfqpoint{6.091347in}{6.023715in}}{\pgfqpoint{6.086957in}{6.013116in}}{\pgfqpoint{6.086957in}{6.002066in}}%
\pgfpathcurveto{\pgfqpoint{6.086957in}{5.991016in}}{\pgfqpoint{6.091347in}{5.980417in}}{\pgfqpoint{6.099160in}{5.972603in}}%
\pgfpathcurveto{\pgfqpoint{6.106974in}{5.964790in}}{\pgfqpoint{6.117573in}{5.960400in}}{\pgfqpoint{6.128623in}{5.960400in}}%
\pgfpathclose%
\pgfusepath{stroke,fill}%
\end{pgfscope}%
\begin{pgfscope}%
\pgfpathrectangle{\pgfqpoint{0.481978in}{0.331635in}}{\pgfqpoint{9.300000in}{7.700000in}}%
\pgfusepath{clip}%
\pgfsetbuttcap%
\pgfsetroundjoin%
\definecolor{currentfill}{rgb}{0.631373,0.788235,0.956863}%
\pgfsetfillcolor{currentfill}%
\pgfsetlinewidth{0.481800pt}%
\definecolor{currentstroke}{rgb}{1.000000,1.000000,1.000000}%
\pgfsetstrokecolor{currentstroke}%
\pgfsetdash{}{0pt}%
\pgfpathmoveto{\pgfqpoint{6.130665in}{5.955353in}}%
\pgfpathcurveto{\pgfqpoint{6.141715in}{5.955353in}}{\pgfqpoint{6.152314in}{5.959744in}}{\pgfqpoint{6.160127in}{5.967557in}}%
\pgfpathcurveto{\pgfqpoint{6.167941in}{5.975371in}}{\pgfqpoint{6.172331in}{5.985970in}}{\pgfqpoint{6.172331in}{5.997020in}}%
\pgfpathcurveto{\pgfqpoint{6.172331in}{6.008070in}}{\pgfqpoint{6.167941in}{6.018669in}}{\pgfqpoint{6.160127in}{6.026483in}}%
\pgfpathcurveto{\pgfqpoint{6.152314in}{6.034296in}}{\pgfqpoint{6.141715in}{6.038687in}}{\pgfqpoint{6.130665in}{6.038687in}}%
\pgfpathcurveto{\pgfqpoint{6.119614in}{6.038687in}}{\pgfqpoint{6.109015in}{6.034296in}}{\pgfqpoint{6.101202in}{6.026483in}}%
\pgfpathcurveto{\pgfqpoint{6.093388in}{6.018669in}}{\pgfqpoint{6.088998in}{6.008070in}}{\pgfqpoint{6.088998in}{5.997020in}}%
\pgfpathcurveto{\pgfqpoint{6.088998in}{5.985970in}}{\pgfqpoint{6.093388in}{5.975371in}}{\pgfqpoint{6.101202in}{5.967557in}}%
\pgfpathcurveto{\pgfqpoint{6.109015in}{5.959744in}}{\pgfqpoint{6.119614in}{5.955353in}}{\pgfqpoint{6.130665in}{5.955353in}}%
\pgfpathclose%
\pgfusepath{stroke,fill}%
\end{pgfscope}%
\begin{pgfscope}%
\pgfpathrectangle{\pgfqpoint{0.481978in}{0.331635in}}{\pgfqpoint{9.300000in}{7.700000in}}%
\pgfusepath{clip}%
\pgfsetbuttcap%
\pgfsetroundjoin%
\definecolor{currentfill}{rgb}{0.631373,0.788235,0.956863}%
\pgfsetfillcolor{currentfill}%
\pgfsetlinewidth{0.481800pt}%
\definecolor{currentstroke}{rgb}{1.000000,1.000000,1.000000}%
\pgfsetstrokecolor{currentstroke}%
\pgfsetdash{}{0pt}%
\pgfpathmoveto{\pgfqpoint{5.250013in}{5.297890in}}%
\pgfpathcurveto{\pgfqpoint{5.261063in}{5.297890in}}{\pgfqpoint{5.271662in}{5.302280in}}{\pgfqpoint{5.279476in}{5.310093in}}%
\pgfpathcurveto{\pgfqpoint{5.287289in}{5.317907in}}{\pgfqpoint{5.291679in}{5.328506in}}{\pgfqpoint{5.291679in}{5.339556in}}%
\pgfpathcurveto{\pgfqpoint{5.291679in}{5.350606in}}{\pgfqpoint{5.287289in}{5.361205in}}{\pgfqpoint{5.279476in}{5.369019in}}%
\pgfpathcurveto{\pgfqpoint{5.271662in}{5.376833in}}{\pgfqpoint{5.261063in}{5.381223in}}{\pgfqpoint{5.250013in}{5.381223in}}%
\pgfpathcurveto{\pgfqpoint{5.238963in}{5.381223in}}{\pgfqpoint{5.228364in}{5.376833in}}{\pgfqpoint{5.220550in}{5.369019in}}%
\pgfpathcurveto{\pgfqpoint{5.212736in}{5.361205in}}{\pgfqpoint{5.208346in}{5.350606in}}{\pgfqpoint{5.208346in}{5.339556in}}%
\pgfpathcurveto{\pgfqpoint{5.208346in}{5.328506in}}{\pgfqpoint{5.212736in}{5.317907in}}{\pgfqpoint{5.220550in}{5.310093in}}%
\pgfpathcurveto{\pgfqpoint{5.228364in}{5.302280in}}{\pgfqpoint{5.238963in}{5.297890in}}{\pgfqpoint{5.250013in}{5.297890in}}%
\pgfpathclose%
\pgfusepath{stroke,fill}%
\end{pgfscope}%
\begin{pgfscope}%
\pgfpathrectangle{\pgfqpoint{0.481978in}{0.331635in}}{\pgfqpoint{9.300000in}{7.700000in}}%
\pgfusepath{clip}%
\pgfsetbuttcap%
\pgfsetroundjoin%
\definecolor{currentfill}{rgb}{0.631373,0.788235,0.956863}%
\pgfsetfillcolor{currentfill}%
\pgfsetlinewidth{0.481800pt}%
\definecolor{currentstroke}{rgb}{1.000000,1.000000,1.000000}%
\pgfsetstrokecolor{currentstroke}%
\pgfsetdash{}{0pt}%
\pgfpathmoveto{\pgfqpoint{7.725554in}{5.784948in}}%
\pgfpathcurveto{\pgfqpoint{7.736604in}{5.784948in}}{\pgfqpoint{7.747203in}{5.789339in}}{\pgfqpoint{7.755017in}{5.797152in}}%
\pgfpathcurveto{\pgfqpoint{7.762831in}{5.804966in}}{\pgfqpoint{7.767221in}{5.815565in}}{\pgfqpoint{7.767221in}{5.826615in}}%
\pgfpathcurveto{\pgfqpoint{7.767221in}{5.837665in}}{\pgfqpoint{7.762831in}{5.848264in}}{\pgfqpoint{7.755017in}{5.856078in}}%
\pgfpathcurveto{\pgfqpoint{7.747203in}{5.863891in}}{\pgfqpoint{7.736604in}{5.868282in}}{\pgfqpoint{7.725554in}{5.868282in}}%
\pgfpathcurveto{\pgfqpoint{7.714504in}{5.868282in}}{\pgfqpoint{7.703905in}{5.863891in}}{\pgfqpoint{7.696091in}{5.856078in}}%
\pgfpathcurveto{\pgfqpoint{7.688278in}{5.848264in}}{\pgfqpoint{7.683888in}{5.837665in}}{\pgfqpoint{7.683888in}{5.826615in}}%
\pgfpathcurveto{\pgfqpoint{7.683888in}{5.815565in}}{\pgfqpoint{7.688278in}{5.804966in}}{\pgfqpoint{7.696091in}{5.797152in}}%
\pgfpathcurveto{\pgfqpoint{7.703905in}{5.789339in}}{\pgfqpoint{7.714504in}{5.784948in}}{\pgfqpoint{7.725554in}{5.784948in}}%
\pgfpathclose%
\pgfusepath{stroke,fill}%
\end{pgfscope}%
\begin{pgfscope}%
\pgfpathrectangle{\pgfqpoint{0.481978in}{0.331635in}}{\pgfqpoint{9.300000in}{7.700000in}}%
\pgfusepath{clip}%
\pgfsetbuttcap%
\pgfsetroundjoin%
\definecolor{currentfill}{rgb}{0.631373,0.788235,0.956863}%
\pgfsetfillcolor{currentfill}%
\pgfsetlinewidth{0.481800pt}%
\definecolor{currentstroke}{rgb}{1.000000,1.000000,1.000000}%
\pgfsetstrokecolor{currentstroke}%
\pgfsetdash{}{0pt}%
\pgfpathmoveto{\pgfqpoint{5.804196in}{2.892483in}}%
\pgfpathcurveto{\pgfqpoint{5.815246in}{2.892483in}}{\pgfqpoint{5.825845in}{2.896874in}}{\pgfqpoint{5.833659in}{2.904687in}}%
\pgfpathcurveto{\pgfqpoint{5.841472in}{2.912501in}}{\pgfqpoint{5.845863in}{2.923100in}}{\pgfqpoint{5.845863in}{2.934150in}}%
\pgfpathcurveto{\pgfqpoint{5.845863in}{2.945200in}}{\pgfqpoint{5.841472in}{2.955799in}}{\pgfqpoint{5.833659in}{2.963613in}}%
\pgfpathcurveto{\pgfqpoint{5.825845in}{2.971426in}}{\pgfqpoint{5.815246in}{2.975817in}}{\pgfqpoint{5.804196in}{2.975817in}}%
\pgfpathcurveto{\pgfqpoint{5.793146in}{2.975817in}}{\pgfqpoint{5.782547in}{2.971426in}}{\pgfqpoint{5.774733in}{2.963613in}}%
\pgfpathcurveto{\pgfqpoint{5.766920in}{2.955799in}}{\pgfqpoint{5.762529in}{2.945200in}}{\pgfqpoint{5.762529in}{2.934150in}}%
\pgfpathcurveto{\pgfqpoint{5.762529in}{2.923100in}}{\pgfqpoint{5.766920in}{2.912501in}}{\pgfqpoint{5.774733in}{2.904687in}}%
\pgfpathcurveto{\pgfqpoint{5.782547in}{2.896874in}}{\pgfqpoint{5.793146in}{2.892483in}}{\pgfqpoint{5.804196in}{2.892483in}}%
\pgfpathclose%
\pgfusepath{stroke,fill}%
\end{pgfscope}%
\begin{pgfscope}%
\pgfpathrectangle{\pgfqpoint{0.481978in}{0.331635in}}{\pgfqpoint{9.300000in}{7.700000in}}%
\pgfusepath{clip}%
\pgfsetbuttcap%
\pgfsetroundjoin%
\definecolor{currentfill}{rgb}{0.631373,0.788235,0.956863}%
\pgfsetfillcolor{currentfill}%
\pgfsetlinewidth{0.481800pt}%
\definecolor{currentstroke}{rgb}{1.000000,1.000000,1.000000}%
\pgfsetstrokecolor{currentstroke}%
\pgfsetdash{}{0pt}%
\pgfpathmoveto{\pgfqpoint{4.361412in}{6.101096in}}%
\pgfpathcurveto{\pgfqpoint{4.372462in}{6.101096in}}{\pgfqpoint{4.383061in}{6.105486in}}{\pgfqpoint{4.390875in}{6.113300in}}%
\pgfpathcurveto{\pgfqpoint{4.398689in}{6.121114in}}{\pgfqpoint{4.403079in}{6.131713in}}{\pgfqpoint{4.403079in}{6.142763in}}%
\pgfpathcurveto{\pgfqpoint{4.403079in}{6.153813in}}{\pgfqpoint{4.398689in}{6.164412in}}{\pgfqpoint{4.390875in}{6.172226in}}%
\pgfpathcurveto{\pgfqpoint{4.383061in}{6.180039in}}{\pgfqpoint{4.372462in}{6.184429in}}{\pgfqpoint{4.361412in}{6.184429in}}%
\pgfpathcurveto{\pgfqpoint{4.350362in}{6.184429in}}{\pgfqpoint{4.339763in}{6.180039in}}{\pgfqpoint{4.331949in}{6.172226in}}%
\pgfpathcurveto{\pgfqpoint{4.324136in}{6.164412in}}{\pgfqpoint{4.319745in}{6.153813in}}{\pgfqpoint{4.319745in}{6.142763in}}%
\pgfpathcurveto{\pgfqpoint{4.319745in}{6.131713in}}{\pgfqpoint{4.324136in}{6.121114in}}{\pgfqpoint{4.331949in}{6.113300in}}%
\pgfpathcurveto{\pgfqpoint{4.339763in}{6.105486in}}{\pgfqpoint{4.350362in}{6.101096in}}{\pgfqpoint{4.361412in}{6.101096in}}%
\pgfpathclose%
\pgfusepath{stroke,fill}%
\end{pgfscope}%
\begin{pgfscope}%
\pgfpathrectangle{\pgfqpoint{0.481978in}{0.331635in}}{\pgfqpoint{9.300000in}{7.700000in}}%
\pgfusepath{clip}%
\pgfsetbuttcap%
\pgfsetroundjoin%
\definecolor{currentfill}{rgb}{0.631373,0.788235,0.956863}%
\pgfsetfillcolor{currentfill}%
\pgfsetlinewidth{0.481800pt}%
\definecolor{currentstroke}{rgb}{1.000000,1.000000,1.000000}%
\pgfsetstrokecolor{currentstroke}%
\pgfsetdash{}{0pt}%
\pgfpathmoveto{\pgfqpoint{7.515857in}{5.108571in}}%
\pgfpathcurveto{\pgfqpoint{7.526908in}{5.108571in}}{\pgfqpoint{7.537507in}{5.112961in}}{\pgfqpoint{7.545320in}{5.120775in}}%
\pgfpathcurveto{\pgfqpoint{7.553134in}{5.128588in}}{\pgfqpoint{7.557524in}{5.139187in}}{\pgfqpoint{7.557524in}{5.150238in}}%
\pgfpathcurveto{\pgfqpoint{7.557524in}{5.161288in}}{\pgfqpoint{7.553134in}{5.171887in}}{\pgfqpoint{7.545320in}{5.179700in}}%
\pgfpathcurveto{\pgfqpoint{7.537507in}{5.187514in}}{\pgfqpoint{7.526908in}{5.191904in}}{\pgfqpoint{7.515857in}{5.191904in}}%
\pgfpathcurveto{\pgfqpoint{7.504807in}{5.191904in}}{\pgfqpoint{7.494208in}{5.187514in}}{\pgfqpoint{7.486395in}{5.179700in}}%
\pgfpathcurveto{\pgfqpoint{7.478581in}{5.171887in}}{\pgfqpoint{7.474191in}{5.161288in}}{\pgfqpoint{7.474191in}{5.150238in}}%
\pgfpathcurveto{\pgfqpoint{7.474191in}{5.139187in}}{\pgfqpoint{7.478581in}{5.128588in}}{\pgfqpoint{7.486395in}{5.120775in}}%
\pgfpathcurveto{\pgfqpoint{7.494208in}{5.112961in}}{\pgfqpoint{7.504807in}{5.108571in}}{\pgfqpoint{7.515857in}{5.108571in}}%
\pgfpathclose%
\pgfusepath{stroke,fill}%
\end{pgfscope}%
\begin{pgfscope}%
\pgfpathrectangle{\pgfqpoint{0.481978in}{0.331635in}}{\pgfqpoint{9.300000in}{7.700000in}}%
\pgfusepath{clip}%
\pgfsetbuttcap%
\pgfsetroundjoin%
\definecolor{currentfill}{rgb}{0.631373,0.788235,0.956863}%
\pgfsetfillcolor{currentfill}%
\pgfsetlinewidth{0.481800pt}%
\definecolor{currentstroke}{rgb}{1.000000,1.000000,1.000000}%
\pgfsetstrokecolor{currentstroke}%
\pgfsetdash{}{0pt}%
\pgfpathmoveto{\pgfqpoint{8.605266in}{4.793680in}}%
\pgfpathcurveto{\pgfqpoint{8.616316in}{4.793680in}}{\pgfqpoint{8.626915in}{4.798070in}}{\pgfqpoint{8.634729in}{4.805884in}}%
\pgfpathcurveto{\pgfqpoint{8.642542in}{4.813698in}}{\pgfqpoint{8.646933in}{4.824297in}}{\pgfqpoint{8.646933in}{4.835347in}}%
\pgfpathcurveto{\pgfqpoint{8.646933in}{4.846397in}}{\pgfqpoint{8.642542in}{4.856996in}}{\pgfqpoint{8.634729in}{4.864810in}}%
\pgfpathcurveto{\pgfqpoint{8.626915in}{4.872623in}}{\pgfqpoint{8.616316in}{4.877014in}}{\pgfqpoint{8.605266in}{4.877014in}}%
\pgfpathcurveto{\pgfqpoint{8.594216in}{4.877014in}}{\pgfqpoint{8.583617in}{4.872623in}}{\pgfqpoint{8.575803in}{4.864810in}}%
\pgfpathcurveto{\pgfqpoint{8.567990in}{4.856996in}}{\pgfqpoint{8.563599in}{4.846397in}}{\pgfqpoint{8.563599in}{4.835347in}}%
\pgfpathcurveto{\pgfqpoint{8.563599in}{4.824297in}}{\pgfqpoint{8.567990in}{4.813698in}}{\pgfqpoint{8.575803in}{4.805884in}}%
\pgfpathcurveto{\pgfqpoint{8.583617in}{4.798070in}}{\pgfqpoint{8.594216in}{4.793680in}}{\pgfqpoint{8.605266in}{4.793680in}}%
\pgfpathclose%
\pgfusepath{stroke,fill}%
\end{pgfscope}%
\begin{pgfscope}%
\pgfpathrectangle{\pgfqpoint{0.481978in}{0.331635in}}{\pgfqpoint{9.300000in}{7.700000in}}%
\pgfusepath{clip}%
\pgfsetbuttcap%
\pgfsetroundjoin%
\definecolor{currentfill}{rgb}{0.631373,0.788235,0.956863}%
\pgfsetfillcolor{currentfill}%
\pgfsetlinewidth{0.481800pt}%
\definecolor{currentstroke}{rgb}{1.000000,1.000000,1.000000}%
\pgfsetstrokecolor{currentstroke}%
\pgfsetdash{}{0pt}%
\pgfpathmoveto{\pgfqpoint{6.196847in}{1.907924in}}%
\pgfpathcurveto{\pgfqpoint{6.207897in}{1.907924in}}{\pgfqpoint{6.218496in}{1.912315in}}{\pgfqpoint{6.226310in}{1.920128in}}%
\pgfpathcurveto{\pgfqpoint{6.234124in}{1.927942in}}{\pgfqpoint{6.238514in}{1.938541in}}{\pgfqpoint{6.238514in}{1.949591in}}%
\pgfpathcurveto{\pgfqpoint{6.238514in}{1.960641in}}{\pgfqpoint{6.234124in}{1.971240in}}{\pgfqpoint{6.226310in}{1.979054in}}%
\pgfpathcurveto{\pgfqpoint{6.218496in}{1.986867in}}{\pgfqpoint{6.207897in}{1.991258in}}{\pgfqpoint{6.196847in}{1.991258in}}%
\pgfpathcurveto{\pgfqpoint{6.185797in}{1.991258in}}{\pgfqpoint{6.175198in}{1.986867in}}{\pgfqpoint{6.167384in}{1.979054in}}%
\pgfpathcurveto{\pgfqpoint{6.159571in}{1.971240in}}{\pgfqpoint{6.155181in}{1.960641in}}{\pgfqpoint{6.155181in}{1.949591in}}%
\pgfpathcurveto{\pgfqpoint{6.155181in}{1.938541in}}{\pgfqpoint{6.159571in}{1.927942in}}{\pgfqpoint{6.167384in}{1.920128in}}%
\pgfpathcurveto{\pgfqpoint{6.175198in}{1.912315in}}{\pgfqpoint{6.185797in}{1.907924in}}{\pgfqpoint{6.196847in}{1.907924in}}%
\pgfpathclose%
\pgfusepath{stroke,fill}%
\end{pgfscope}%
\begin{pgfscope}%
\pgfpathrectangle{\pgfqpoint{0.481978in}{0.331635in}}{\pgfqpoint{9.300000in}{7.700000in}}%
\pgfusepath{clip}%
\pgfsetbuttcap%
\pgfsetroundjoin%
\definecolor{currentfill}{rgb}{0.631373,0.788235,0.956863}%
\pgfsetfillcolor{currentfill}%
\pgfsetlinewidth{0.481800pt}%
\definecolor{currentstroke}{rgb}{1.000000,1.000000,1.000000}%
\pgfsetstrokecolor{currentstroke}%
\pgfsetdash{}{0pt}%
\pgfpathmoveto{\pgfqpoint{4.336260in}{2.021267in}}%
\pgfpathcurveto{\pgfqpoint{4.347310in}{2.021267in}}{\pgfqpoint{4.357909in}{2.025657in}}{\pgfqpoint{4.365723in}{2.033471in}}%
\pgfpathcurveto{\pgfqpoint{4.373537in}{2.041285in}}{\pgfqpoint{4.377927in}{2.051884in}}{\pgfqpoint{4.377927in}{2.062934in}}%
\pgfpathcurveto{\pgfqpoint{4.377927in}{2.073984in}}{\pgfqpoint{4.373537in}{2.084583in}}{\pgfqpoint{4.365723in}{2.092397in}}%
\pgfpathcurveto{\pgfqpoint{4.357909in}{2.100210in}}{\pgfqpoint{4.347310in}{2.104601in}}{\pgfqpoint{4.336260in}{2.104601in}}%
\pgfpathcurveto{\pgfqpoint{4.325210in}{2.104601in}}{\pgfqpoint{4.314611in}{2.100210in}}{\pgfqpoint{4.306797in}{2.092397in}}%
\pgfpathcurveto{\pgfqpoint{4.298984in}{2.084583in}}{\pgfqpoint{4.294593in}{2.073984in}}{\pgfqpoint{4.294593in}{2.062934in}}%
\pgfpathcurveto{\pgfqpoint{4.294593in}{2.051884in}}{\pgfqpoint{4.298984in}{2.041285in}}{\pgfqpoint{4.306797in}{2.033471in}}%
\pgfpathcurveto{\pgfqpoint{4.314611in}{2.025657in}}{\pgfqpoint{4.325210in}{2.021267in}}{\pgfqpoint{4.336260in}{2.021267in}}%
\pgfpathclose%
\pgfusepath{stroke,fill}%
\end{pgfscope}%
\begin{pgfscope}%
\pgfpathrectangle{\pgfqpoint{0.481978in}{0.331635in}}{\pgfqpoint{9.300000in}{7.700000in}}%
\pgfusepath{clip}%
\pgfsetbuttcap%
\pgfsetroundjoin%
\definecolor{currentfill}{rgb}{0.631373,0.788235,0.956863}%
\pgfsetfillcolor{currentfill}%
\pgfsetlinewidth{0.481800pt}%
\definecolor{currentstroke}{rgb}{1.000000,1.000000,1.000000}%
\pgfsetstrokecolor{currentstroke}%
\pgfsetdash{}{0pt}%
\pgfpathmoveto{\pgfqpoint{4.441169in}{5.370092in}}%
\pgfpathcurveto{\pgfqpoint{4.452219in}{5.370092in}}{\pgfqpoint{4.462818in}{5.374482in}}{\pgfqpoint{4.470632in}{5.382296in}}%
\pgfpathcurveto{\pgfqpoint{4.478445in}{5.390110in}}{\pgfqpoint{4.482836in}{5.400709in}}{\pgfqpoint{4.482836in}{5.411759in}}%
\pgfpathcurveto{\pgfqpoint{4.482836in}{5.422809in}}{\pgfqpoint{4.478445in}{5.433408in}}{\pgfqpoint{4.470632in}{5.441222in}}%
\pgfpathcurveto{\pgfqpoint{4.462818in}{5.449035in}}{\pgfqpoint{4.452219in}{5.453426in}}{\pgfqpoint{4.441169in}{5.453426in}}%
\pgfpathcurveto{\pgfqpoint{4.430119in}{5.453426in}}{\pgfqpoint{4.419520in}{5.449035in}}{\pgfqpoint{4.411706in}{5.441222in}}%
\pgfpathcurveto{\pgfqpoint{4.403893in}{5.433408in}}{\pgfqpoint{4.399502in}{5.422809in}}{\pgfqpoint{4.399502in}{5.411759in}}%
\pgfpathcurveto{\pgfqpoint{4.399502in}{5.400709in}}{\pgfqpoint{4.403893in}{5.390110in}}{\pgfqpoint{4.411706in}{5.382296in}}%
\pgfpathcurveto{\pgfqpoint{4.419520in}{5.374482in}}{\pgfqpoint{4.430119in}{5.370092in}}{\pgfqpoint{4.441169in}{5.370092in}}%
\pgfpathclose%
\pgfusepath{stroke,fill}%
\end{pgfscope}%
\begin{pgfscope}%
\pgfpathrectangle{\pgfqpoint{0.481978in}{0.331635in}}{\pgfqpoint{9.300000in}{7.700000in}}%
\pgfusepath{clip}%
\pgfsetbuttcap%
\pgfsetroundjoin%
\definecolor{currentfill}{rgb}{0.631373,0.788235,0.956863}%
\pgfsetfillcolor{currentfill}%
\pgfsetlinewidth{0.481800pt}%
\definecolor{currentstroke}{rgb}{1.000000,1.000000,1.000000}%
\pgfsetstrokecolor{currentstroke}%
\pgfsetdash{}{0pt}%
\pgfpathmoveto{\pgfqpoint{6.227427in}{1.656234in}}%
\pgfpathcurveto{\pgfqpoint{6.238477in}{1.656234in}}{\pgfqpoint{6.249076in}{1.660624in}}{\pgfqpoint{6.256889in}{1.668438in}}%
\pgfpathcurveto{\pgfqpoint{6.264703in}{1.676251in}}{\pgfqpoint{6.269093in}{1.686850in}}{\pgfqpoint{6.269093in}{1.697900in}}%
\pgfpathcurveto{\pgfqpoint{6.269093in}{1.708951in}}{\pgfqpoint{6.264703in}{1.719550in}}{\pgfqpoint{6.256889in}{1.727363in}}%
\pgfpathcurveto{\pgfqpoint{6.249076in}{1.735177in}}{\pgfqpoint{6.238477in}{1.739567in}}{\pgfqpoint{6.227427in}{1.739567in}}%
\pgfpathcurveto{\pgfqpoint{6.216376in}{1.739567in}}{\pgfqpoint{6.205777in}{1.735177in}}{\pgfqpoint{6.197964in}{1.727363in}}%
\pgfpathcurveto{\pgfqpoint{6.190150in}{1.719550in}}{\pgfqpoint{6.185760in}{1.708951in}}{\pgfqpoint{6.185760in}{1.697900in}}%
\pgfpathcurveto{\pgfqpoint{6.185760in}{1.686850in}}{\pgfqpoint{6.190150in}{1.676251in}}{\pgfqpoint{6.197964in}{1.668438in}}%
\pgfpathcurveto{\pgfqpoint{6.205777in}{1.660624in}}{\pgfqpoint{6.216376in}{1.656234in}}{\pgfqpoint{6.227427in}{1.656234in}}%
\pgfpathclose%
\pgfusepath{stroke,fill}%
\end{pgfscope}%
\begin{pgfscope}%
\pgfpathrectangle{\pgfqpoint{0.481978in}{0.331635in}}{\pgfqpoint{9.300000in}{7.700000in}}%
\pgfusepath{clip}%
\pgfsetbuttcap%
\pgfsetroundjoin%
\definecolor{currentfill}{rgb}{0.631373,0.788235,0.956863}%
\pgfsetfillcolor{currentfill}%
\pgfsetlinewidth{0.481800pt}%
\definecolor{currentstroke}{rgb}{1.000000,1.000000,1.000000}%
\pgfsetstrokecolor{currentstroke}%
\pgfsetdash{}{0pt}%
\pgfpathmoveto{\pgfqpoint{7.013858in}{2.218280in}}%
\pgfpathcurveto{\pgfqpoint{7.024908in}{2.218280in}}{\pgfqpoint{7.035507in}{2.222671in}}{\pgfqpoint{7.043321in}{2.230484in}}%
\pgfpathcurveto{\pgfqpoint{7.051135in}{2.238298in}}{\pgfqpoint{7.055525in}{2.248897in}}{\pgfqpoint{7.055525in}{2.259947in}}%
\pgfpathcurveto{\pgfqpoint{7.055525in}{2.270997in}}{\pgfqpoint{7.051135in}{2.281596in}}{\pgfqpoint{7.043321in}{2.289410in}}%
\pgfpathcurveto{\pgfqpoint{7.035507in}{2.297223in}}{\pgfqpoint{7.024908in}{2.301614in}}{\pgfqpoint{7.013858in}{2.301614in}}%
\pgfpathcurveto{\pgfqpoint{7.002808in}{2.301614in}}{\pgfqpoint{6.992209in}{2.297223in}}{\pgfqpoint{6.984395in}{2.289410in}}%
\pgfpathcurveto{\pgfqpoint{6.976582in}{2.281596in}}{\pgfqpoint{6.972191in}{2.270997in}}{\pgfqpoint{6.972191in}{2.259947in}}%
\pgfpathcurveto{\pgfqpoint{6.972191in}{2.248897in}}{\pgfqpoint{6.976582in}{2.238298in}}{\pgfqpoint{6.984395in}{2.230484in}}%
\pgfpathcurveto{\pgfqpoint{6.992209in}{2.222671in}}{\pgfqpoint{7.002808in}{2.218280in}}{\pgfqpoint{7.013858in}{2.218280in}}%
\pgfpathclose%
\pgfusepath{stroke,fill}%
\end{pgfscope}%
\begin{pgfscope}%
\pgfpathrectangle{\pgfqpoint{0.481978in}{0.331635in}}{\pgfqpoint{9.300000in}{7.700000in}}%
\pgfusepath{clip}%
\pgfsetbuttcap%
\pgfsetroundjoin%
\definecolor{currentfill}{rgb}{0.631373,0.788235,0.956863}%
\pgfsetfillcolor{currentfill}%
\pgfsetlinewidth{0.481800pt}%
\definecolor{currentstroke}{rgb}{1.000000,1.000000,1.000000}%
\pgfsetstrokecolor{currentstroke}%
\pgfsetdash{}{0pt}%
\pgfpathmoveto{\pgfqpoint{3.180339in}{6.470557in}}%
\pgfpathcurveto{\pgfqpoint{3.191389in}{6.470557in}}{\pgfqpoint{3.201988in}{6.474947in}}{\pgfqpoint{3.209801in}{6.482761in}}%
\pgfpathcurveto{\pgfqpoint{3.217615in}{6.490574in}}{\pgfqpoint{3.222005in}{6.501173in}}{\pgfqpoint{3.222005in}{6.512223in}}%
\pgfpathcurveto{\pgfqpoint{3.222005in}{6.523274in}}{\pgfqpoint{3.217615in}{6.533873in}}{\pgfqpoint{3.209801in}{6.541686in}}%
\pgfpathcurveto{\pgfqpoint{3.201988in}{6.549500in}}{\pgfqpoint{3.191389in}{6.553890in}}{\pgfqpoint{3.180339in}{6.553890in}}%
\pgfpathcurveto{\pgfqpoint{3.169289in}{6.553890in}}{\pgfqpoint{3.158689in}{6.549500in}}{\pgfqpoint{3.150876in}{6.541686in}}%
\pgfpathcurveto{\pgfqpoint{3.143062in}{6.533873in}}{\pgfqpoint{3.138672in}{6.523274in}}{\pgfqpoint{3.138672in}{6.512223in}}%
\pgfpathcurveto{\pgfqpoint{3.138672in}{6.501173in}}{\pgfqpoint{3.143062in}{6.490574in}}{\pgfqpoint{3.150876in}{6.482761in}}%
\pgfpathcurveto{\pgfqpoint{3.158689in}{6.474947in}}{\pgfqpoint{3.169289in}{6.470557in}}{\pgfqpoint{3.180339in}{6.470557in}}%
\pgfpathclose%
\pgfusepath{stroke,fill}%
\end{pgfscope}%
\begin{pgfscope}%
\pgfpathrectangle{\pgfqpoint{0.481978in}{0.331635in}}{\pgfqpoint{9.300000in}{7.700000in}}%
\pgfusepath{clip}%
\pgfsetbuttcap%
\pgfsetroundjoin%
\definecolor{currentfill}{rgb}{0.631373,0.788235,0.956863}%
\pgfsetfillcolor{currentfill}%
\pgfsetlinewidth{0.481800pt}%
\definecolor{currentstroke}{rgb}{1.000000,1.000000,1.000000}%
\pgfsetstrokecolor{currentstroke}%
\pgfsetdash{}{0pt}%
\pgfpathmoveto{\pgfqpoint{4.845738in}{0.823624in}}%
\pgfpathcurveto{\pgfqpoint{4.856788in}{0.823624in}}{\pgfqpoint{4.867387in}{0.828014in}}{\pgfqpoint{4.875201in}{0.835827in}}%
\pgfpathcurveto{\pgfqpoint{4.883015in}{0.843641in}}{\pgfqpoint{4.887405in}{0.854240in}}{\pgfqpoint{4.887405in}{0.865290in}}%
\pgfpathcurveto{\pgfqpoint{4.887405in}{0.876340in}}{\pgfqpoint{4.883015in}{0.886939in}}{\pgfqpoint{4.875201in}{0.894753in}}%
\pgfpathcurveto{\pgfqpoint{4.867387in}{0.902567in}}{\pgfqpoint{4.856788in}{0.906957in}}{\pgfqpoint{4.845738in}{0.906957in}}%
\pgfpathcurveto{\pgfqpoint{4.834688in}{0.906957in}}{\pgfqpoint{4.824089in}{0.902567in}}{\pgfqpoint{4.816275in}{0.894753in}}%
\pgfpathcurveto{\pgfqpoint{4.808462in}{0.886939in}}{\pgfqpoint{4.804071in}{0.876340in}}{\pgfqpoint{4.804071in}{0.865290in}}%
\pgfpathcurveto{\pgfqpoint{4.804071in}{0.854240in}}{\pgfqpoint{4.808462in}{0.843641in}}{\pgfqpoint{4.816275in}{0.835827in}}%
\pgfpathcurveto{\pgfqpoint{4.824089in}{0.828014in}}{\pgfqpoint{4.834688in}{0.823624in}}{\pgfqpoint{4.845738in}{0.823624in}}%
\pgfpathclose%
\pgfusepath{stroke,fill}%
\end{pgfscope}%
\begin{pgfscope}%
\pgfpathrectangle{\pgfqpoint{0.481978in}{0.331635in}}{\pgfqpoint{9.300000in}{7.700000in}}%
\pgfusepath{clip}%
\pgfsetbuttcap%
\pgfsetroundjoin%
\definecolor{currentfill}{rgb}{0.631373,0.788235,0.956863}%
\pgfsetfillcolor{currentfill}%
\pgfsetlinewidth{0.481800pt}%
\definecolor{currentstroke}{rgb}{1.000000,1.000000,1.000000}%
\pgfsetstrokecolor{currentstroke}%
\pgfsetdash{}{0pt}%
\pgfpathmoveto{\pgfqpoint{4.394203in}{1.807179in}}%
\pgfpathcurveto{\pgfqpoint{4.405253in}{1.807179in}}{\pgfqpoint{4.415852in}{1.811569in}}{\pgfqpoint{4.423666in}{1.819383in}}%
\pgfpathcurveto{\pgfqpoint{4.431480in}{1.827196in}}{\pgfqpoint{4.435870in}{1.837796in}}{\pgfqpoint{4.435870in}{1.848846in}}%
\pgfpathcurveto{\pgfqpoint{4.435870in}{1.859896in}}{\pgfqpoint{4.431480in}{1.870495in}}{\pgfqpoint{4.423666in}{1.878308in}}%
\pgfpathcurveto{\pgfqpoint{4.415852in}{1.886122in}}{\pgfqpoint{4.405253in}{1.890512in}}{\pgfqpoint{4.394203in}{1.890512in}}%
\pgfpathcurveto{\pgfqpoint{4.383153in}{1.890512in}}{\pgfqpoint{4.372554in}{1.886122in}}{\pgfqpoint{4.364740in}{1.878308in}}%
\pgfpathcurveto{\pgfqpoint{4.356927in}{1.870495in}}{\pgfqpoint{4.352537in}{1.859896in}}{\pgfqpoint{4.352537in}{1.848846in}}%
\pgfpathcurveto{\pgfqpoint{4.352537in}{1.837796in}}{\pgfqpoint{4.356927in}{1.827196in}}{\pgfqpoint{4.364740in}{1.819383in}}%
\pgfpathcurveto{\pgfqpoint{4.372554in}{1.811569in}}{\pgfqpoint{4.383153in}{1.807179in}}{\pgfqpoint{4.394203in}{1.807179in}}%
\pgfpathclose%
\pgfusepath{stroke,fill}%
\end{pgfscope}%
\begin{pgfscope}%
\pgfpathrectangle{\pgfqpoint{0.481978in}{0.331635in}}{\pgfqpoint{9.300000in}{7.700000in}}%
\pgfusepath{clip}%
\pgfsetbuttcap%
\pgfsetroundjoin%
\definecolor{currentfill}{rgb}{0.631373,0.788235,0.956863}%
\pgfsetfillcolor{currentfill}%
\pgfsetlinewidth{0.481800pt}%
\definecolor{currentstroke}{rgb}{1.000000,1.000000,1.000000}%
\pgfsetstrokecolor{currentstroke}%
\pgfsetdash{}{0pt}%
\pgfpathmoveto{\pgfqpoint{5.179568in}{5.314143in}}%
\pgfpathcurveto{\pgfqpoint{5.190618in}{5.314143in}}{\pgfqpoint{5.201217in}{5.318534in}}{\pgfqpoint{5.209031in}{5.326347in}}%
\pgfpathcurveto{\pgfqpoint{5.216845in}{5.334161in}}{\pgfqpoint{5.221235in}{5.344760in}}{\pgfqpoint{5.221235in}{5.355810in}}%
\pgfpathcurveto{\pgfqpoint{5.221235in}{5.366860in}}{\pgfqpoint{5.216845in}{5.377459in}}{\pgfqpoint{5.209031in}{5.385273in}}%
\pgfpathcurveto{\pgfqpoint{5.201217in}{5.393086in}}{\pgfqpoint{5.190618in}{5.397477in}}{\pgfqpoint{5.179568in}{5.397477in}}%
\pgfpathcurveto{\pgfqpoint{5.168518in}{5.397477in}}{\pgfqpoint{5.157919in}{5.393086in}}{\pgfqpoint{5.150105in}{5.385273in}}%
\pgfpathcurveto{\pgfqpoint{5.142292in}{5.377459in}}{\pgfqpoint{5.137901in}{5.366860in}}{\pgfqpoint{5.137901in}{5.355810in}}%
\pgfpathcurveto{\pgfqpoint{5.137901in}{5.344760in}}{\pgfqpoint{5.142292in}{5.334161in}}{\pgfqpoint{5.150105in}{5.326347in}}%
\pgfpathcurveto{\pgfqpoint{5.157919in}{5.318534in}}{\pgfqpoint{5.168518in}{5.314143in}}{\pgfqpoint{5.179568in}{5.314143in}}%
\pgfpathclose%
\pgfusepath{stroke,fill}%
\end{pgfscope}%
\begin{pgfscope}%
\pgfpathrectangle{\pgfqpoint{0.481978in}{0.331635in}}{\pgfqpoint{9.300000in}{7.700000in}}%
\pgfusepath{clip}%
\pgfsetbuttcap%
\pgfsetroundjoin%
\definecolor{currentfill}{rgb}{0.631373,0.788235,0.956863}%
\pgfsetfillcolor{currentfill}%
\pgfsetlinewidth{0.481800pt}%
\definecolor{currentstroke}{rgb}{1.000000,1.000000,1.000000}%
\pgfsetstrokecolor{currentstroke}%
\pgfsetdash{}{0pt}%
\pgfpathmoveto{\pgfqpoint{7.941055in}{4.945270in}}%
\pgfpathcurveto{\pgfqpoint{7.952105in}{4.945270in}}{\pgfqpoint{7.962704in}{4.949661in}}{\pgfqpoint{7.970518in}{4.957474in}}%
\pgfpathcurveto{\pgfqpoint{7.978331in}{4.965288in}}{\pgfqpoint{7.982722in}{4.975887in}}{\pgfqpoint{7.982722in}{4.986937in}}%
\pgfpathcurveto{\pgfqpoint{7.982722in}{4.997987in}}{\pgfqpoint{7.978331in}{5.008586in}}{\pgfqpoint{7.970518in}{5.016400in}}%
\pgfpathcurveto{\pgfqpoint{7.962704in}{5.024213in}}{\pgfqpoint{7.952105in}{5.028604in}}{\pgfqpoint{7.941055in}{5.028604in}}%
\pgfpathcurveto{\pgfqpoint{7.930005in}{5.028604in}}{\pgfqpoint{7.919406in}{5.024213in}}{\pgfqpoint{7.911592in}{5.016400in}}%
\pgfpathcurveto{\pgfqpoint{7.903779in}{5.008586in}}{\pgfqpoint{7.899388in}{4.997987in}}{\pgfqpoint{7.899388in}{4.986937in}}%
\pgfpathcurveto{\pgfqpoint{7.899388in}{4.975887in}}{\pgfqpoint{7.903779in}{4.965288in}}{\pgfqpoint{7.911592in}{4.957474in}}%
\pgfpathcurveto{\pgfqpoint{7.919406in}{4.949661in}}{\pgfqpoint{7.930005in}{4.945270in}}{\pgfqpoint{7.941055in}{4.945270in}}%
\pgfpathclose%
\pgfusepath{stroke,fill}%
\end{pgfscope}%
\begin{pgfscope}%
\pgfpathrectangle{\pgfqpoint{0.481978in}{0.331635in}}{\pgfqpoint{9.300000in}{7.700000in}}%
\pgfusepath{clip}%
\pgfsetbuttcap%
\pgfsetroundjoin%
\definecolor{currentfill}{rgb}{0.631373,0.788235,0.956863}%
\pgfsetfillcolor{currentfill}%
\pgfsetlinewidth{0.481800pt}%
\definecolor{currentstroke}{rgb}{1.000000,1.000000,1.000000}%
\pgfsetstrokecolor{currentstroke}%
\pgfsetdash{}{0pt}%
\pgfpathmoveto{\pgfqpoint{3.294139in}{1.535735in}}%
\pgfpathcurveto{\pgfqpoint{3.305189in}{1.535735in}}{\pgfqpoint{3.315788in}{1.540125in}}{\pgfqpoint{3.323601in}{1.547939in}}%
\pgfpathcurveto{\pgfqpoint{3.331415in}{1.555753in}}{\pgfqpoint{3.335805in}{1.566352in}}{\pgfqpoint{3.335805in}{1.577402in}}%
\pgfpathcurveto{\pgfqpoint{3.335805in}{1.588452in}}{\pgfqpoint{3.331415in}{1.599051in}}{\pgfqpoint{3.323601in}{1.606865in}}%
\pgfpathcurveto{\pgfqpoint{3.315788in}{1.614678in}}{\pgfqpoint{3.305189in}{1.619068in}}{\pgfqpoint{3.294139in}{1.619068in}}%
\pgfpathcurveto{\pgfqpoint{3.283088in}{1.619068in}}{\pgfqpoint{3.272489in}{1.614678in}}{\pgfqpoint{3.264676in}{1.606865in}}%
\pgfpathcurveto{\pgfqpoint{3.256862in}{1.599051in}}{\pgfqpoint{3.252472in}{1.588452in}}{\pgfqpoint{3.252472in}{1.577402in}}%
\pgfpathcurveto{\pgfqpoint{3.252472in}{1.566352in}}{\pgfqpoint{3.256862in}{1.555753in}}{\pgfqpoint{3.264676in}{1.547939in}}%
\pgfpathcurveto{\pgfqpoint{3.272489in}{1.540125in}}{\pgfqpoint{3.283088in}{1.535735in}}{\pgfqpoint{3.294139in}{1.535735in}}%
\pgfpathclose%
\pgfusepath{stroke,fill}%
\end{pgfscope}%
\begin{pgfscope}%
\pgfpathrectangle{\pgfqpoint{0.481978in}{0.331635in}}{\pgfqpoint{9.300000in}{7.700000in}}%
\pgfusepath{clip}%
\pgfsetbuttcap%
\pgfsetroundjoin%
\definecolor{currentfill}{rgb}{0.631373,0.788235,0.956863}%
\pgfsetfillcolor{currentfill}%
\pgfsetlinewidth{0.481800pt}%
\definecolor{currentstroke}{rgb}{1.000000,1.000000,1.000000}%
\pgfsetstrokecolor{currentstroke}%
\pgfsetdash{}{0pt}%
\pgfpathmoveto{\pgfqpoint{7.359994in}{3.989846in}}%
\pgfpathcurveto{\pgfqpoint{7.371044in}{3.989846in}}{\pgfqpoint{7.381643in}{3.994236in}}{\pgfqpoint{7.389457in}{4.002049in}}%
\pgfpathcurveto{\pgfqpoint{7.397270in}{4.009863in}}{\pgfqpoint{7.401661in}{4.020462in}}{\pgfqpoint{7.401661in}{4.031512in}}%
\pgfpathcurveto{\pgfqpoint{7.401661in}{4.042562in}}{\pgfqpoint{7.397270in}{4.053161in}}{\pgfqpoint{7.389457in}{4.060975in}}%
\pgfpathcurveto{\pgfqpoint{7.381643in}{4.068789in}}{\pgfqpoint{7.371044in}{4.073179in}}{\pgfqpoint{7.359994in}{4.073179in}}%
\pgfpathcurveto{\pgfqpoint{7.348944in}{4.073179in}}{\pgfqpoint{7.338345in}{4.068789in}}{\pgfqpoint{7.330531in}{4.060975in}}%
\pgfpathcurveto{\pgfqpoint{7.322718in}{4.053161in}}{\pgfqpoint{7.318327in}{4.042562in}}{\pgfqpoint{7.318327in}{4.031512in}}%
\pgfpathcurveto{\pgfqpoint{7.318327in}{4.020462in}}{\pgfqpoint{7.322718in}{4.009863in}}{\pgfqpoint{7.330531in}{4.002049in}}%
\pgfpathcurveto{\pgfqpoint{7.338345in}{3.994236in}}{\pgfqpoint{7.348944in}{3.989846in}}{\pgfqpoint{7.359994in}{3.989846in}}%
\pgfpathclose%
\pgfusepath{stroke,fill}%
\end{pgfscope}%
\begin{pgfscope}%
\pgfpathrectangle{\pgfqpoint{0.481978in}{0.331635in}}{\pgfqpoint{9.300000in}{7.700000in}}%
\pgfusepath{clip}%
\pgfsetbuttcap%
\pgfsetroundjoin%
\definecolor{currentfill}{rgb}{0.631373,0.788235,0.956863}%
\pgfsetfillcolor{currentfill}%
\pgfsetlinewidth{0.481800pt}%
\definecolor{currentstroke}{rgb}{1.000000,1.000000,1.000000}%
\pgfsetstrokecolor{currentstroke}%
\pgfsetdash{}{0pt}%
\pgfpathmoveto{\pgfqpoint{7.314751in}{4.873947in}}%
\pgfpathcurveto{\pgfqpoint{7.325801in}{4.873947in}}{\pgfqpoint{7.336400in}{4.878337in}}{\pgfqpoint{7.344213in}{4.886151in}}%
\pgfpathcurveto{\pgfqpoint{7.352027in}{4.893964in}}{\pgfqpoint{7.356417in}{4.904563in}}{\pgfqpoint{7.356417in}{4.915613in}}%
\pgfpathcurveto{\pgfqpoint{7.356417in}{4.926664in}}{\pgfqpoint{7.352027in}{4.937263in}}{\pgfqpoint{7.344213in}{4.945076in}}%
\pgfpathcurveto{\pgfqpoint{7.336400in}{4.952890in}}{\pgfqpoint{7.325801in}{4.957280in}}{\pgfqpoint{7.314751in}{4.957280in}}%
\pgfpathcurveto{\pgfqpoint{7.303700in}{4.957280in}}{\pgfqpoint{7.293101in}{4.952890in}}{\pgfqpoint{7.285288in}{4.945076in}}%
\pgfpathcurveto{\pgfqpoint{7.277474in}{4.937263in}}{\pgfqpoint{7.273084in}{4.926664in}}{\pgfqpoint{7.273084in}{4.915613in}}%
\pgfpathcurveto{\pgfqpoint{7.273084in}{4.904563in}}{\pgfqpoint{7.277474in}{4.893964in}}{\pgfqpoint{7.285288in}{4.886151in}}%
\pgfpathcurveto{\pgfqpoint{7.293101in}{4.878337in}}{\pgfqpoint{7.303700in}{4.873947in}}{\pgfqpoint{7.314751in}{4.873947in}}%
\pgfpathclose%
\pgfusepath{stroke,fill}%
\end{pgfscope}%
\begin{pgfscope}%
\pgfpathrectangle{\pgfqpoint{0.481978in}{0.331635in}}{\pgfqpoint{9.300000in}{7.700000in}}%
\pgfusepath{clip}%
\pgfsetbuttcap%
\pgfsetroundjoin%
\definecolor{currentfill}{rgb}{0.631373,0.788235,0.956863}%
\pgfsetfillcolor{currentfill}%
\pgfsetlinewidth{0.481800pt}%
\definecolor{currentstroke}{rgb}{1.000000,1.000000,1.000000}%
\pgfsetstrokecolor{currentstroke}%
\pgfsetdash{}{0pt}%
\pgfpathmoveto{\pgfqpoint{3.066979in}{6.640361in}}%
\pgfpathcurveto{\pgfqpoint{3.078029in}{6.640361in}}{\pgfqpoint{3.088628in}{6.644751in}}{\pgfqpoint{3.096442in}{6.652565in}}%
\pgfpathcurveto{\pgfqpoint{3.104255in}{6.660379in}}{\pgfqpoint{3.108646in}{6.670978in}}{\pgfqpoint{3.108646in}{6.682028in}}%
\pgfpathcurveto{\pgfqpoint{3.108646in}{6.693078in}}{\pgfqpoint{3.104255in}{6.703677in}}{\pgfqpoint{3.096442in}{6.711490in}}%
\pgfpathcurveto{\pgfqpoint{3.088628in}{6.719304in}}{\pgfqpoint{3.078029in}{6.723694in}}{\pgfqpoint{3.066979in}{6.723694in}}%
\pgfpathcurveto{\pgfqpoint{3.055929in}{6.723694in}}{\pgfqpoint{3.045330in}{6.719304in}}{\pgfqpoint{3.037516in}{6.711490in}}%
\pgfpathcurveto{\pgfqpoint{3.029702in}{6.703677in}}{\pgfqpoint{3.025312in}{6.693078in}}{\pgfqpoint{3.025312in}{6.682028in}}%
\pgfpathcurveto{\pgfqpoint{3.025312in}{6.670978in}}{\pgfqpoint{3.029702in}{6.660379in}}{\pgfqpoint{3.037516in}{6.652565in}}%
\pgfpathcurveto{\pgfqpoint{3.045330in}{6.644751in}}{\pgfqpoint{3.055929in}{6.640361in}}{\pgfqpoint{3.066979in}{6.640361in}}%
\pgfpathclose%
\pgfusepath{stroke,fill}%
\end{pgfscope}%
\begin{pgfscope}%
\pgfpathrectangle{\pgfqpoint{0.481978in}{0.331635in}}{\pgfqpoint{9.300000in}{7.700000in}}%
\pgfusepath{clip}%
\pgfsetbuttcap%
\pgfsetroundjoin%
\definecolor{currentfill}{rgb}{0.631373,0.788235,0.956863}%
\pgfsetfillcolor{currentfill}%
\pgfsetlinewidth{0.481800pt}%
\definecolor{currentstroke}{rgb}{1.000000,1.000000,1.000000}%
\pgfsetstrokecolor{currentstroke}%
\pgfsetdash{}{0pt}%
\pgfpathmoveto{\pgfqpoint{2.849648in}{6.868474in}}%
\pgfpathcurveto{\pgfqpoint{2.860698in}{6.868474in}}{\pgfqpoint{2.871297in}{6.872865in}}{\pgfqpoint{2.879110in}{6.880678in}}%
\pgfpathcurveto{\pgfqpoint{2.886924in}{6.888492in}}{\pgfqpoint{2.891314in}{6.899091in}}{\pgfqpoint{2.891314in}{6.910141in}}%
\pgfpathcurveto{\pgfqpoint{2.891314in}{6.921191in}}{\pgfqpoint{2.886924in}{6.931790in}}{\pgfqpoint{2.879110in}{6.939604in}}%
\pgfpathcurveto{\pgfqpoint{2.871297in}{6.947417in}}{\pgfqpoint{2.860698in}{6.951808in}}{\pgfqpoint{2.849648in}{6.951808in}}%
\pgfpathcurveto{\pgfqpoint{2.838597in}{6.951808in}}{\pgfqpoint{2.827998in}{6.947417in}}{\pgfqpoint{2.820185in}{6.939604in}}%
\pgfpathcurveto{\pgfqpoint{2.812371in}{6.931790in}}{\pgfqpoint{2.807981in}{6.921191in}}{\pgfqpoint{2.807981in}{6.910141in}}%
\pgfpathcurveto{\pgfqpoint{2.807981in}{6.899091in}}{\pgfqpoint{2.812371in}{6.888492in}}{\pgfqpoint{2.820185in}{6.880678in}}%
\pgfpathcurveto{\pgfqpoint{2.827998in}{6.872865in}}{\pgfqpoint{2.838597in}{6.868474in}}{\pgfqpoint{2.849648in}{6.868474in}}%
\pgfpathclose%
\pgfusepath{stroke,fill}%
\end{pgfscope}%
\begin{pgfscope}%
\pgfpathrectangle{\pgfqpoint{0.481978in}{0.331635in}}{\pgfqpoint{9.300000in}{7.700000in}}%
\pgfusepath{clip}%
\pgfsetbuttcap%
\pgfsetroundjoin%
\definecolor{currentfill}{rgb}{0.631373,0.788235,0.956863}%
\pgfsetfillcolor{currentfill}%
\pgfsetlinewidth{0.481800pt}%
\definecolor{currentstroke}{rgb}{1.000000,1.000000,1.000000}%
\pgfsetstrokecolor{currentstroke}%
\pgfsetdash{}{0pt}%
\pgfpathmoveto{\pgfqpoint{2.771056in}{6.623764in}}%
\pgfpathcurveto{\pgfqpoint{2.782106in}{6.623764in}}{\pgfqpoint{2.792705in}{6.628155in}}{\pgfqpoint{2.800519in}{6.635968in}}%
\pgfpathcurveto{\pgfqpoint{2.808332in}{6.643782in}}{\pgfqpoint{2.812722in}{6.654381in}}{\pgfqpoint{2.812722in}{6.665431in}}%
\pgfpathcurveto{\pgfqpoint{2.812722in}{6.676481in}}{\pgfqpoint{2.808332in}{6.687080in}}{\pgfqpoint{2.800519in}{6.694894in}}%
\pgfpathcurveto{\pgfqpoint{2.792705in}{6.702708in}}{\pgfqpoint{2.782106in}{6.707098in}}{\pgfqpoint{2.771056in}{6.707098in}}%
\pgfpathcurveto{\pgfqpoint{2.760006in}{6.707098in}}{\pgfqpoint{2.749407in}{6.702708in}}{\pgfqpoint{2.741593in}{6.694894in}}%
\pgfpathcurveto{\pgfqpoint{2.733779in}{6.687080in}}{\pgfqpoint{2.729389in}{6.676481in}}{\pgfqpoint{2.729389in}{6.665431in}}%
\pgfpathcurveto{\pgfqpoint{2.729389in}{6.654381in}}{\pgfqpoint{2.733779in}{6.643782in}}{\pgfqpoint{2.741593in}{6.635968in}}%
\pgfpathcurveto{\pgfqpoint{2.749407in}{6.628155in}}{\pgfqpoint{2.760006in}{6.623764in}}{\pgfqpoint{2.771056in}{6.623764in}}%
\pgfpathclose%
\pgfusepath{stroke,fill}%
\end{pgfscope}%
\begin{pgfscope}%
\pgfpathrectangle{\pgfqpoint{0.481978in}{0.331635in}}{\pgfqpoint{9.300000in}{7.700000in}}%
\pgfusepath{clip}%
\pgfsetbuttcap%
\pgfsetroundjoin%
\definecolor{currentfill}{rgb}{0.631373,0.788235,0.956863}%
\pgfsetfillcolor{currentfill}%
\pgfsetlinewidth{0.481800pt}%
\definecolor{currentstroke}{rgb}{1.000000,1.000000,1.000000}%
\pgfsetstrokecolor{currentstroke}%
\pgfsetdash{}{0pt}%
\pgfpathmoveto{\pgfqpoint{4.853047in}{5.880469in}}%
\pgfpathcurveto{\pgfqpoint{4.864097in}{5.880469in}}{\pgfqpoint{4.874696in}{5.884859in}}{\pgfqpoint{4.882510in}{5.892673in}}%
\pgfpathcurveto{\pgfqpoint{4.890324in}{5.900486in}}{\pgfqpoint{4.894714in}{5.911085in}}{\pgfqpoint{4.894714in}{5.922135in}}%
\pgfpathcurveto{\pgfqpoint{4.894714in}{5.933186in}}{\pgfqpoint{4.890324in}{5.943785in}}{\pgfqpoint{4.882510in}{5.951598in}}%
\pgfpathcurveto{\pgfqpoint{4.874696in}{5.959412in}}{\pgfqpoint{4.864097in}{5.963802in}}{\pgfqpoint{4.853047in}{5.963802in}}%
\pgfpathcurveto{\pgfqpoint{4.841997in}{5.963802in}}{\pgfqpoint{4.831398in}{5.959412in}}{\pgfqpoint{4.823584in}{5.951598in}}%
\pgfpathcurveto{\pgfqpoint{4.815771in}{5.943785in}}{\pgfqpoint{4.811381in}{5.933186in}}{\pgfqpoint{4.811381in}{5.922135in}}%
\pgfpathcurveto{\pgfqpoint{4.811381in}{5.911085in}}{\pgfqpoint{4.815771in}{5.900486in}}{\pgfqpoint{4.823584in}{5.892673in}}%
\pgfpathcurveto{\pgfqpoint{4.831398in}{5.884859in}}{\pgfqpoint{4.841997in}{5.880469in}}{\pgfqpoint{4.853047in}{5.880469in}}%
\pgfpathclose%
\pgfusepath{stroke,fill}%
\end{pgfscope}%
\begin{pgfscope}%
\pgfpathrectangle{\pgfqpoint{0.481978in}{0.331635in}}{\pgfqpoint{9.300000in}{7.700000in}}%
\pgfusepath{clip}%
\pgfsetbuttcap%
\pgfsetroundjoin%
\definecolor{currentfill}{rgb}{0.631373,0.788235,0.956863}%
\pgfsetfillcolor{currentfill}%
\pgfsetlinewidth{0.481800pt}%
\definecolor{currentstroke}{rgb}{1.000000,1.000000,1.000000}%
\pgfsetstrokecolor{currentstroke}%
\pgfsetdash{}{0pt}%
\pgfpathmoveto{\pgfqpoint{5.879927in}{0.802391in}}%
\pgfpathcurveto{\pgfqpoint{5.890977in}{0.802391in}}{\pgfqpoint{5.901576in}{0.806781in}}{\pgfqpoint{5.909390in}{0.814595in}}%
\pgfpathcurveto{\pgfqpoint{5.917204in}{0.822409in}}{\pgfqpoint{5.921594in}{0.833008in}}{\pgfqpoint{5.921594in}{0.844058in}}%
\pgfpathcurveto{\pgfqpoint{5.921594in}{0.855108in}}{\pgfqpoint{5.917204in}{0.865707in}}{\pgfqpoint{5.909390in}{0.873521in}}%
\pgfpathcurveto{\pgfqpoint{5.901576in}{0.881334in}}{\pgfqpoint{5.890977in}{0.885724in}}{\pgfqpoint{5.879927in}{0.885724in}}%
\pgfpathcurveto{\pgfqpoint{5.868877in}{0.885724in}}{\pgfqpoint{5.858278in}{0.881334in}}{\pgfqpoint{5.850465in}{0.873521in}}%
\pgfpathcurveto{\pgfqpoint{5.842651in}{0.865707in}}{\pgfqpoint{5.838261in}{0.855108in}}{\pgfqpoint{5.838261in}{0.844058in}}%
\pgfpathcurveto{\pgfqpoint{5.838261in}{0.833008in}}{\pgfqpoint{5.842651in}{0.822409in}}{\pgfqpoint{5.850465in}{0.814595in}}%
\pgfpathcurveto{\pgfqpoint{5.858278in}{0.806781in}}{\pgfqpoint{5.868877in}{0.802391in}}{\pgfqpoint{5.879927in}{0.802391in}}%
\pgfpathclose%
\pgfusepath{stroke,fill}%
\end{pgfscope}%
\begin{pgfscope}%
\pgfpathrectangle{\pgfqpoint{0.481978in}{0.331635in}}{\pgfqpoint{9.300000in}{7.700000in}}%
\pgfusepath{clip}%
\pgfsetbuttcap%
\pgfsetroundjoin%
\definecolor{currentfill}{rgb}{0.631373,0.788235,0.956863}%
\pgfsetfillcolor{currentfill}%
\pgfsetlinewidth{0.481800pt}%
\definecolor{currentstroke}{rgb}{1.000000,1.000000,1.000000}%
\pgfsetstrokecolor{currentstroke}%
\pgfsetdash{}{0pt}%
\pgfpathmoveto{\pgfqpoint{6.834691in}{2.041724in}}%
\pgfpathcurveto{\pgfqpoint{6.845741in}{2.041724in}}{\pgfqpoint{6.856340in}{2.046114in}}{\pgfqpoint{6.864153in}{2.053928in}}%
\pgfpathcurveto{\pgfqpoint{6.871967in}{2.061742in}}{\pgfqpoint{6.876357in}{2.072341in}}{\pgfqpoint{6.876357in}{2.083391in}}%
\pgfpathcurveto{\pgfqpoint{6.876357in}{2.094441in}}{\pgfqpoint{6.871967in}{2.105040in}}{\pgfqpoint{6.864153in}{2.112854in}}%
\pgfpathcurveto{\pgfqpoint{6.856340in}{2.120667in}}{\pgfqpoint{6.845741in}{2.125058in}}{\pgfqpoint{6.834691in}{2.125058in}}%
\pgfpathcurveto{\pgfqpoint{6.823641in}{2.125058in}}{\pgfqpoint{6.813041in}{2.120667in}}{\pgfqpoint{6.805228in}{2.112854in}}%
\pgfpathcurveto{\pgfqpoint{6.797414in}{2.105040in}}{\pgfqpoint{6.793024in}{2.094441in}}{\pgfqpoint{6.793024in}{2.083391in}}%
\pgfpathcurveto{\pgfqpoint{6.793024in}{2.072341in}}{\pgfqpoint{6.797414in}{2.061742in}}{\pgfqpoint{6.805228in}{2.053928in}}%
\pgfpathcurveto{\pgfqpoint{6.813041in}{2.046114in}}{\pgfqpoint{6.823641in}{2.041724in}}{\pgfqpoint{6.834691in}{2.041724in}}%
\pgfpathclose%
\pgfusepath{stroke,fill}%
\end{pgfscope}%
\begin{pgfscope}%
\pgfpathrectangle{\pgfqpoint{0.481978in}{0.331635in}}{\pgfqpoint{9.300000in}{7.700000in}}%
\pgfusepath{clip}%
\pgfsetbuttcap%
\pgfsetroundjoin%
\definecolor{currentfill}{rgb}{0.631373,0.788235,0.956863}%
\pgfsetfillcolor{currentfill}%
\pgfsetlinewidth{0.481800pt}%
\definecolor{currentstroke}{rgb}{1.000000,1.000000,1.000000}%
\pgfsetstrokecolor{currentstroke}%
\pgfsetdash{}{0pt}%
\pgfpathmoveto{\pgfqpoint{5.297141in}{2.737467in}}%
\pgfpathcurveto{\pgfqpoint{5.308191in}{2.737467in}}{\pgfqpoint{5.318790in}{2.741857in}}{\pgfqpoint{5.326604in}{2.749671in}}%
\pgfpathcurveto{\pgfqpoint{5.334417in}{2.757484in}}{\pgfqpoint{5.338808in}{2.768083in}}{\pgfqpoint{5.338808in}{2.779134in}}%
\pgfpathcurveto{\pgfqpoint{5.338808in}{2.790184in}}{\pgfqpoint{5.334417in}{2.800783in}}{\pgfqpoint{5.326604in}{2.808596in}}%
\pgfpathcurveto{\pgfqpoint{5.318790in}{2.816410in}}{\pgfqpoint{5.308191in}{2.820800in}}{\pgfqpoint{5.297141in}{2.820800in}}%
\pgfpathcurveto{\pgfqpoint{5.286091in}{2.820800in}}{\pgfqpoint{5.275492in}{2.816410in}}{\pgfqpoint{5.267678in}{2.808596in}}%
\pgfpathcurveto{\pgfqpoint{5.259864in}{2.800783in}}{\pgfqpoint{5.255474in}{2.790184in}}{\pgfqpoint{5.255474in}{2.779134in}}%
\pgfpathcurveto{\pgfqpoint{5.255474in}{2.768083in}}{\pgfqpoint{5.259864in}{2.757484in}}{\pgfqpoint{5.267678in}{2.749671in}}%
\pgfpathcurveto{\pgfqpoint{5.275492in}{2.741857in}}{\pgfqpoint{5.286091in}{2.737467in}}{\pgfqpoint{5.297141in}{2.737467in}}%
\pgfpathclose%
\pgfusepath{stroke,fill}%
\end{pgfscope}%
\begin{pgfscope}%
\pgfpathrectangle{\pgfqpoint{0.481978in}{0.331635in}}{\pgfqpoint{9.300000in}{7.700000in}}%
\pgfusepath{clip}%
\pgfsetbuttcap%
\pgfsetroundjoin%
\definecolor{currentfill}{rgb}{0.631373,0.788235,0.956863}%
\pgfsetfillcolor{currentfill}%
\pgfsetlinewidth{0.481800pt}%
\definecolor{currentstroke}{rgb}{1.000000,1.000000,1.000000}%
\pgfsetstrokecolor{currentstroke}%
\pgfsetdash{}{0pt}%
\pgfpathmoveto{\pgfqpoint{6.713307in}{5.141056in}}%
\pgfpathcurveto{\pgfqpoint{6.724358in}{5.141056in}}{\pgfqpoint{6.734957in}{5.145446in}}{\pgfqpoint{6.742770in}{5.153260in}}%
\pgfpathcurveto{\pgfqpoint{6.750584in}{5.161073in}}{\pgfqpoint{6.754974in}{5.171673in}}{\pgfqpoint{6.754974in}{5.182723in}}%
\pgfpathcurveto{\pgfqpoint{6.754974in}{5.193773in}}{\pgfqpoint{6.750584in}{5.204372in}}{\pgfqpoint{6.742770in}{5.212185in}}%
\pgfpathcurveto{\pgfqpoint{6.734957in}{5.219999in}}{\pgfqpoint{6.724358in}{5.224389in}}{\pgfqpoint{6.713307in}{5.224389in}}%
\pgfpathcurveto{\pgfqpoint{6.702257in}{5.224389in}}{\pgfqpoint{6.691658in}{5.219999in}}{\pgfqpoint{6.683845in}{5.212185in}}%
\pgfpathcurveto{\pgfqpoint{6.676031in}{5.204372in}}{\pgfqpoint{6.671641in}{5.193773in}}{\pgfqpoint{6.671641in}{5.182723in}}%
\pgfpathcurveto{\pgfqpoint{6.671641in}{5.171673in}}{\pgfqpoint{6.676031in}{5.161073in}}{\pgfqpoint{6.683845in}{5.153260in}}%
\pgfpathcurveto{\pgfqpoint{6.691658in}{5.145446in}}{\pgfqpoint{6.702257in}{5.141056in}}{\pgfqpoint{6.713307in}{5.141056in}}%
\pgfpathclose%
\pgfusepath{stroke,fill}%
\end{pgfscope}%
\begin{pgfscope}%
\pgfpathrectangle{\pgfqpoint{0.481978in}{0.331635in}}{\pgfqpoint{9.300000in}{7.700000in}}%
\pgfusepath{clip}%
\pgfsetbuttcap%
\pgfsetroundjoin%
\definecolor{currentfill}{rgb}{0.631373,0.788235,0.956863}%
\pgfsetfillcolor{currentfill}%
\pgfsetlinewidth{0.481800pt}%
\definecolor{currentstroke}{rgb}{1.000000,1.000000,1.000000}%
\pgfsetstrokecolor{currentstroke}%
\pgfsetdash{}{0pt}%
\pgfpathmoveto{\pgfqpoint{5.117952in}{2.310213in}}%
\pgfpathcurveto{\pgfqpoint{5.129002in}{2.310213in}}{\pgfqpoint{5.139601in}{2.314603in}}{\pgfqpoint{5.147415in}{2.322417in}}%
\pgfpathcurveto{\pgfqpoint{5.155228in}{2.330230in}}{\pgfqpoint{5.159618in}{2.340829in}}{\pgfqpoint{5.159618in}{2.351880in}}%
\pgfpathcurveto{\pgfqpoint{5.159618in}{2.362930in}}{\pgfqpoint{5.155228in}{2.373529in}}{\pgfqpoint{5.147415in}{2.381342in}}%
\pgfpathcurveto{\pgfqpoint{5.139601in}{2.389156in}}{\pgfqpoint{5.129002in}{2.393546in}}{\pgfqpoint{5.117952in}{2.393546in}}%
\pgfpathcurveto{\pgfqpoint{5.106902in}{2.393546in}}{\pgfqpoint{5.096303in}{2.389156in}}{\pgfqpoint{5.088489in}{2.381342in}}%
\pgfpathcurveto{\pgfqpoint{5.080675in}{2.373529in}}{\pgfqpoint{5.076285in}{2.362930in}}{\pgfqpoint{5.076285in}{2.351880in}}%
\pgfpathcurveto{\pgfqpoint{5.076285in}{2.340829in}}{\pgfqpoint{5.080675in}{2.330230in}}{\pgfqpoint{5.088489in}{2.322417in}}%
\pgfpathcurveto{\pgfqpoint{5.096303in}{2.314603in}}{\pgfqpoint{5.106902in}{2.310213in}}{\pgfqpoint{5.117952in}{2.310213in}}%
\pgfpathclose%
\pgfusepath{stroke,fill}%
\end{pgfscope}%
\begin{pgfscope}%
\pgfpathrectangle{\pgfqpoint{0.481978in}{0.331635in}}{\pgfqpoint{9.300000in}{7.700000in}}%
\pgfusepath{clip}%
\pgfsetbuttcap%
\pgfsetroundjoin%
\definecolor{currentfill}{rgb}{0.631373,0.788235,0.956863}%
\pgfsetfillcolor{currentfill}%
\pgfsetlinewidth{0.481800pt}%
\definecolor{currentstroke}{rgb}{1.000000,1.000000,1.000000}%
\pgfsetstrokecolor{currentstroke}%
\pgfsetdash{}{0pt}%
\pgfpathmoveto{\pgfqpoint{7.076892in}{3.052698in}}%
\pgfpathcurveto{\pgfqpoint{7.087942in}{3.052698in}}{\pgfqpoint{7.098541in}{3.057088in}}{\pgfqpoint{7.106355in}{3.064902in}}%
\pgfpathcurveto{\pgfqpoint{7.114168in}{3.072716in}}{\pgfqpoint{7.118559in}{3.083315in}}{\pgfqpoint{7.118559in}{3.094365in}}%
\pgfpathcurveto{\pgfqpoint{7.118559in}{3.105415in}}{\pgfqpoint{7.114168in}{3.116014in}}{\pgfqpoint{7.106355in}{3.123828in}}%
\pgfpathcurveto{\pgfqpoint{7.098541in}{3.131641in}}{\pgfqpoint{7.087942in}{3.136031in}}{\pgfqpoint{7.076892in}{3.136031in}}%
\pgfpathcurveto{\pgfqpoint{7.065842in}{3.136031in}}{\pgfqpoint{7.055243in}{3.131641in}}{\pgfqpoint{7.047429in}{3.123828in}}%
\pgfpathcurveto{\pgfqpoint{7.039616in}{3.116014in}}{\pgfqpoint{7.035225in}{3.105415in}}{\pgfqpoint{7.035225in}{3.094365in}}%
\pgfpathcurveto{\pgfqpoint{7.035225in}{3.083315in}}{\pgfqpoint{7.039616in}{3.072716in}}{\pgfqpoint{7.047429in}{3.064902in}}%
\pgfpathcurveto{\pgfqpoint{7.055243in}{3.057088in}}{\pgfqpoint{7.065842in}{3.052698in}}{\pgfqpoint{7.076892in}{3.052698in}}%
\pgfpathclose%
\pgfusepath{stroke,fill}%
\end{pgfscope}%
\begin{pgfscope}%
\pgfpathrectangle{\pgfqpoint{0.481978in}{0.331635in}}{\pgfqpoint{9.300000in}{7.700000in}}%
\pgfusepath{clip}%
\pgfsetbuttcap%
\pgfsetroundjoin%
\definecolor{currentfill}{rgb}{0.631373,0.788235,0.956863}%
\pgfsetfillcolor{currentfill}%
\pgfsetlinewidth{0.481800pt}%
\definecolor{currentstroke}{rgb}{1.000000,1.000000,1.000000}%
\pgfsetstrokecolor{currentstroke}%
\pgfsetdash{}{0pt}%
\pgfpathmoveto{\pgfqpoint{7.507895in}{1.911864in}}%
\pgfpathcurveto{\pgfqpoint{7.518945in}{1.911864in}}{\pgfqpoint{7.529544in}{1.916254in}}{\pgfqpoint{7.537358in}{1.924068in}}%
\pgfpathcurveto{\pgfqpoint{7.545171in}{1.931881in}}{\pgfqpoint{7.549562in}{1.942480in}}{\pgfqpoint{7.549562in}{1.953530in}}%
\pgfpathcurveto{\pgfqpoint{7.549562in}{1.964581in}}{\pgfqpoint{7.545171in}{1.975180in}}{\pgfqpoint{7.537358in}{1.982993in}}%
\pgfpathcurveto{\pgfqpoint{7.529544in}{1.990807in}}{\pgfqpoint{7.518945in}{1.995197in}}{\pgfqpoint{7.507895in}{1.995197in}}%
\pgfpathcurveto{\pgfqpoint{7.496845in}{1.995197in}}{\pgfqpoint{7.486246in}{1.990807in}}{\pgfqpoint{7.478432in}{1.982993in}}%
\pgfpathcurveto{\pgfqpoint{7.470619in}{1.975180in}}{\pgfqpoint{7.466228in}{1.964581in}}{\pgfqpoint{7.466228in}{1.953530in}}%
\pgfpathcurveto{\pgfqpoint{7.466228in}{1.942480in}}{\pgfqpoint{7.470619in}{1.931881in}}{\pgfqpoint{7.478432in}{1.924068in}}%
\pgfpathcurveto{\pgfqpoint{7.486246in}{1.916254in}}{\pgfqpoint{7.496845in}{1.911864in}}{\pgfqpoint{7.507895in}{1.911864in}}%
\pgfpathclose%
\pgfusepath{stroke,fill}%
\end{pgfscope}%
\begin{pgfscope}%
\pgfpathrectangle{\pgfqpoint{0.481978in}{0.331635in}}{\pgfqpoint{9.300000in}{7.700000in}}%
\pgfusepath{clip}%
\pgfsetbuttcap%
\pgfsetroundjoin%
\definecolor{currentfill}{rgb}{0.631373,0.788235,0.956863}%
\pgfsetfillcolor{currentfill}%
\pgfsetlinewidth{0.481800pt}%
\definecolor{currentstroke}{rgb}{1.000000,1.000000,1.000000}%
\pgfsetstrokecolor{currentstroke}%
\pgfsetdash{}{0pt}%
\pgfpathmoveto{\pgfqpoint{4.102323in}{5.788720in}}%
\pgfpathcurveto{\pgfqpoint{4.113373in}{5.788720in}}{\pgfqpoint{4.123972in}{5.793111in}}{\pgfqpoint{4.131786in}{5.800924in}}%
\pgfpathcurveto{\pgfqpoint{4.139599in}{5.808738in}}{\pgfqpoint{4.143989in}{5.819337in}}{\pgfqpoint{4.143989in}{5.830387in}}%
\pgfpathcurveto{\pgfqpoint{4.143989in}{5.841437in}}{\pgfqpoint{4.139599in}{5.852036in}}{\pgfqpoint{4.131786in}{5.859850in}}%
\pgfpathcurveto{\pgfqpoint{4.123972in}{5.867663in}}{\pgfqpoint{4.113373in}{5.872054in}}{\pgfqpoint{4.102323in}{5.872054in}}%
\pgfpathcurveto{\pgfqpoint{4.091273in}{5.872054in}}{\pgfqpoint{4.080674in}{5.867663in}}{\pgfqpoint{4.072860in}{5.859850in}}%
\pgfpathcurveto{\pgfqpoint{4.065046in}{5.852036in}}{\pgfqpoint{4.060656in}{5.841437in}}{\pgfqpoint{4.060656in}{5.830387in}}%
\pgfpathcurveto{\pgfqpoint{4.060656in}{5.819337in}}{\pgfqpoint{4.065046in}{5.808738in}}{\pgfqpoint{4.072860in}{5.800924in}}%
\pgfpathcurveto{\pgfqpoint{4.080674in}{5.793111in}}{\pgfqpoint{4.091273in}{5.788720in}}{\pgfqpoint{4.102323in}{5.788720in}}%
\pgfpathclose%
\pgfusepath{stroke,fill}%
\end{pgfscope}%
\begin{pgfscope}%
\pgfpathrectangle{\pgfqpoint{0.481978in}{0.331635in}}{\pgfqpoint{9.300000in}{7.700000in}}%
\pgfusepath{clip}%
\pgfsetbuttcap%
\pgfsetroundjoin%
\definecolor{currentfill}{rgb}{0.631373,0.788235,0.956863}%
\pgfsetfillcolor{currentfill}%
\pgfsetlinewidth{0.481800pt}%
\definecolor{currentstroke}{rgb}{1.000000,1.000000,1.000000}%
\pgfsetstrokecolor{currentstroke}%
\pgfsetdash{}{0pt}%
\pgfpathmoveto{\pgfqpoint{6.020143in}{5.274735in}}%
\pgfpathcurveto{\pgfqpoint{6.031193in}{5.274735in}}{\pgfqpoint{6.041792in}{5.279126in}}{\pgfqpoint{6.049606in}{5.286939in}}%
\pgfpathcurveto{\pgfqpoint{6.057419in}{5.294753in}}{\pgfqpoint{6.061810in}{5.305352in}}{\pgfqpoint{6.061810in}{5.316402in}}%
\pgfpathcurveto{\pgfqpoint{6.061810in}{5.327452in}}{\pgfqpoint{6.057419in}{5.338051in}}{\pgfqpoint{6.049606in}{5.345865in}}%
\pgfpathcurveto{\pgfqpoint{6.041792in}{5.353678in}}{\pgfqpoint{6.031193in}{5.358069in}}{\pgfqpoint{6.020143in}{5.358069in}}%
\pgfpathcurveto{\pgfqpoint{6.009093in}{5.358069in}}{\pgfqpoint{5.998494in}{5.353678in}}{\pgfqpoint{5.990680in}{5.345865in}}%
\pgfpathcurveto{\pgfqpoint{5.982867in}{5.338051in}}{\pgfqpoint{5.978476in}{5.327452in}}{\pgfqpoint{5.978476in}{5.316402in}}%
\pgfpathcurveto{\pgfqpoint{5.978476in}{5.305352in}}{\pgfqpoint{5.982867in}{5.294753in}}{\pgfqpoint{5.990680in}{5.286939in}}%
\pgfpathcurveto{\pgfqpoint{5.998494in}{5.279126in}}{\pgfqpoint{6.009093in}{5.274735in}}{\pgfqpoint{6.020143in}{5.274735in}}%
\pgfpathclose%
\pgfusepath{stroke,fill}%
\end{pgfscope}%
\begin{pgfscope}%
\pgfpathrectangle{\pgfqpoint{0.481978in}{0.331635in}}{\pgfqpoint{9.300000in}{7.700000in}}%
\pgfusepath{clip}%
\pgfsetbuttcap%
\pgfsetroundjoin%
\definecolor{currentfill}{rgb}{0.631373,0.788235,0.956863}%
\pgfsetfillcolor{currentfill}%
\pgfsetlinewidth{0.481800pt}%
\definecolor{currentstroke}{rgb}{1.000000,1.000000,1.000000}%
\pgfsetstrokecolor{currentstroke}%
\pgfsetdash{}{0pt}%
\pgfpathmoveto{\pgfqpoint{7.865979in}{5.764555in}}%
\pgfpathcurveto{\pgfqpoint{7.877029in}{5.764555in}}{\pgfqpoint{7.887628in}{5.768945in}}{\pgfqpoint{7.895442in}{5.776759in}}%
\pgfpathcurveto{\pgfqpoint{7.903255in}{5.784573in}}{\pgfqpoint{7.907645in}{5.795172in}}{\pgfqpoint{7.907645in}{5.806222in}}%
\pgfpathcurveto{\pgfqpoint{7.907645in}{5.817272in}}{\pgfqpoint{7.903255in}{5.827871in}}{\pgfqpoint{7.895442in}{5.835685in}}%
\pgfpathcurveto{\pgfqpoint{7.887628in}{5.843498in}}{\pgfqpoint{7.877029in}{5.847888in}}{\pgfqpoint{7.865979in}{5.847888in}}%
\pgfpathcurveto{\pgfqpoint{7.854929in}{5.847888in}}{\pgfqpoint{7.844330in}{5.843498in}}{\pgfqpoint{7.836516in}{5.835685in}}%
\pgfpathcurveto{\pgfqpoint{7.828702in}{5.827871in}}{\pgfqpoint{7.824312in}{5.817272in}}{\pgfqpoint{7.824312in}{5.806222in}}%
\pgfpathcurveto{\pgfqpoint{7.824312in}{5.795172in}}{\pgfqpoint{7.828702in}{5.784573in}}{\pgfqpoint{7.836516in}{5.776759in}}%
\pgfpathcurveto{\pgfqpoint{7.844330in}{5.768945in}}{\pgfqpoint{7.854929in}{5.764555in}}{\pgfqpoint{7.865979in}{5.764555in}}%
\pgfpathclose%
\pgfusepath{stroke,fill}%
\end{pgfscope}%
\begin{pgfscope}%
\pgfpathrectangle{\pgfqpoint{0.481978in}{0.331635in}}{\pgfqpoint{9.300000in}{7.700000in}}%
\pgfusepath{clip}%
\pgfsetbuttcap%
\pgfsetroundjoin%
\definecolor{currentfill}{rgb}{0.631373,0.788235,0.956863}%
\pgfsetfillcolor{currentfill}%
\pgfsetlinewidth{0.481800pt}%
\definecolor{currentstroke}{rgb}{1.000000,1.000000,1.000000}%
\pgfsetstrokecolor{currentstroke}%
\pgfsetdash{}{0pt}%
\pgfpathmoveto{\pgfqpoint{5.392504in}{5.371877in}}%
\pgfpathcurveto{\pgfqpoint{5.403554in}{5.371877in}}{\pgfqpoint{5.414153in}{5.376268in}}{\pgfqpoint{5.421967in}{5.384081in}}%
\pgfpathcurveto{\pgfqpoint{5.429780in}{5.391895in}}{\pgfqpoint{5.434171in}{5.402494in}}{\pgfqpoint{5.434171in}{5.413544in}}%
\pgfpathcurveto{\pgfqpoint{5.434171in}{5.424594in}}{\pgfqpoint{5.429780in}{5.435193in}}{\pgfqpoint{5.421967in}{5.443007in}}%
\pgfpathcurveto{\pgfqpoint{5.414153in}{5.450820in}}{\pgfqpoint{5.403554in}{5.455211in}}{\pgfqpoint{5.392504in}{5.455211in}}%
\pgfpathcurveto{\pgfqpoint{5.381454in}{5.455211in}}{\pgfqpoint{5.370855in}{5.450820in}}{\pgfqpoint{5.363041in}{5.443007in}}%
\pgfpathcurveto{\pgfqpoint{5.355228in}{5.435193in}}{\pgfqpoint{5.350837in}{5.424594in}}{\pgfqpoint{5.350837in}{5.413544in}}%
\pgfpathcurveto{\pgfqpoint{5.350837in}{5.402494in}}{\pgfqpoint{5.355228in}{5.391895in}}{\pgfqpoint{5.363041in}{5.384081in}}%
\pgfpathcurveto{\pgfqpoint{5.370855in}{5.376268in}}{\pgfqpoint{5.381454in}{5.371877in}}{\pgfqpoint{5.392504in}{5.371877in}}%
\pgfpathclose%
\pgfusepath{stroke,fill}%
\end{pgfscope}%
\begin{pgfscope}%
\pgfpathrectangle{\pgfqpoint{0.481978in}{0.331635in}}{\pgfqpoint{9.300000in}{7.700000in}}%
\pgfusepath{clip}%
\pgfsetbuttcap%
\pgfsetroundjoin%
\definecolor{currentfill}{rgb}{0.631373,0.788235,0.956863}%
\pgfsetfillcolor{currentfill}%
\pgfsetlinewidth{0.481800pt}%
\definecolor{currentstroke}{rgb}{1.000000,1.000000,1.000000}%
\pgfsetstrokecolor{currentstroke}%
\pgfsetdash{}{0pt}%
\pgfpathmoveto{\pgfqpoint{2.932058in}{5.382422in}}%
\pgfpathcurveto{\pgfqpoint{2.943108in}{5.382422in}}{\pgfqpoint{2.953707in}{5.386813in}}{\pgfqpoint{2.961521in}{5.394626in}}%
\pgfpathcurveto{\pgfqpoint{2.969335in}{5.402440in}}{\pgfqpoint{2.973725in}{5.413039in}}{\pgfqpoint{2.973725in}{5.424089in}}%
\pgfpathcurveto{\pgfqpoint{2.973725in}{5.435139in}}{\pgfqpoint{2.969335in}{5.445738in}}{\pgfqpoint{2.961521in}{5.453552in}}%
\pgfpathcurveto{\pgfqpoint{2.953707in}{5.461366in}}{\pgfqpoint{2.943108in}{5.465756in}}{\pgfqpoint{2.932058in}{5.465756in}}%
\pgfpathcurveto{\pgfqpoint{2.921008in}{5.465756in}}{\pgfqpoint{2.910409in}{5.461366in}}{\pgfqpoint{2.902595in}{5.453552in}}%
\pgfpathcurveto{\pgfqpoint{2.894782in}{5.445738in}}{\pgfqpoint{2.890391in}{5.435139in}}{\pgfqpoint{2.890391in}{5.424089in}}%
\pgfpathcurveto{\pgfqpoint{2.890391in}{5.413039in}}{\pgfqpoint{2.894782in}{5.402440in}}{\pgfqpoint{2.902595in}{5.394626in}}%
\pgfpathcurveto{\pgfqpoint{2.910409in}{5.386813in}}{\pgfqpoint{2.921008in}{5.382422in}}{\pgfqpoint{2.932058in}{5.382422in}}%
\pgfpathclose%
\pgfusepath{stroke,fill}%
\end{pgfscope}%
\begin{pgfscope}%
\pgfpathrectangle{\pgfqpoint{0.481978in}{0.331635in}}{\pgfqpoint{9.300000in}{7.700000in}}%
\pgfusepath{clip}%
\pgfsetbuttcap%
\pgfsetroundjoin%
\definecolor{currentfill}{rgb}{0.631373,0.788235,0.956863}%
\pgfsetfillcolor{currentfill}%
\pgfsetlinewidth{0.481800pt}%
\definecolor{currentstroke}{rgb}{1.000000,1.000000,1.000000}%
\pgfsetstrokecolor{currentstroke}%
\pgfsetdash{}{0pt}%
\pgfpathmoveto{\pgfqpoint{7.779256in}{5.038032in}}%
\pgfpathcurveto{\pgfqpoint{7.790306in}{5.038032in}}{\pgfqpoint{7.800905in}{5.042422in}}{\pgfqpoint{7.808718in}{5.050236in}}%
\pgfpathcurveto{\pgfqpoint{7.816532in}{5.058050in}}{\pgfqpoint{7.820922in}{5.068649in}}{\pgfqpoint{7.820922in}{5.079699in}}%
\pgfpathcurveto{\pgfqpoint{7.820922in}{5.090749in}}{\pgfqpoint{7.816532in}{5.101348in}}{\pgfqpoint{7.808718in}{5.109162in}}%
\pgfpathcurveto{\pgfqpoint{7.800905in}{5.116975in}}{\pgfqpoint{7.790306in}{5.121365in}}{\pgfqpoint{7.779256in}{5.121365in}}%
\pgfpathcurveto{\pgfqpoint{7.768206in}{5.121365in}}{\pgfqpoint{7.757607in}{5.116975in}}{\pgfqpoint{7.749793in}{5.109162in}}%
\pgfpathcurveto{\pgfqpoint{7.741979in}{5.101348in}}{\pgfqpoint{7.737589in}{5.090749in}}{\pgfqpoint{7.737589in}{5.079699in}}%
\pgfpathcurveto{\pgfqpoint{7.737589in}{5.068649in}}{\pgfqpoint{7.741979in}{5.058050in}}{\pgfqpoint{7.749793in}{5.050236in}}%
\pgfpathcurveto{\pgfqpoint{7.757607in}{5.042422in}}{\pgfqpoint{7.768206in}{5.038032in}}{\pgfqpoint{7.779256in}{5.038032in}}%
\pgfpathclose%
\pgfusepath{stroke,fill}%
\end{pgfscope}%
\begin{pgfscope}%
\pgfpathrectangle{\pgfqpoint{0.481978in}{0.331635in}}{\pgfqpoint{9.300000in}{7.700000in}}%
\pgfusepath{clip}%
\pgfsetbuttcap%
\pgfsetroundjoin%
\definecolor{currentfill}{rgb}{0.631373,0.788235,0.956863}%
\pgfsetfillcolor{currentfill}%
\pgfsetlinewidth{0.481800pt}%
\definecolor{currentstroke}{rgb}{1.000000,1.000000,1.000000}%
\pgfsetstrokecolor{currentstroke}%
\pgfsetdash{}{0pt}%
\pgfpathmoveto{\pgfqpoint{7.196974in}{1.589439in}}%
\pgfpathcurveto{\pgfqpoint{7.208024in}{1.589439in}}{\pgfqpoint{7.218623in}{1.593829in}}{\pgfqpoint{7.226437in}{1.601643in}}%
\pgfpathcurveto{\pgfqpoint{7.234251in}{1.609456in}}{\pgfqpoint{7.238641in}{1.620055in}}{\pgfqpoint{7.238641in}{1.631105in}}%
\pgfpathcurveto{\pgfqpoint{7.238641in}{1.642155in}}{\pgfqpoint{7.234251in}{1.652755in}}{\pgfqpoint{7.226437in}{1.660568in}}%
\pgfpathcurveto{\pgfqpoint{7.218623in}{1.668382in}}{\pgfqpoint{7.208024in}{1.672772in}}{\pgfqpoint{7.196974in}{1.672772in}}%
\pgfpathcurveto{\pgfqpoint{7.185924in}{1.672772in}}{\pgfqpoint{7.175325in}{1.668382in}}{\pgfqpoint{7.167512in}{1.660568in}}%
\pgfpathcurveto{\pgfqpoint{7.159698in}{1.652755in}}{\pgfqpoint{7.155308in}{1.642155in}}{\pgfqpoint{7.155308in}{1.631105in}}%
\pgfpathcurveto{\pgfqpoint{7.155308in}{1.620055in}}{\pgfqpoint{7.159698in}{1.609456in}}{\pgfqpoint{7.167512in}{1.601643in}}%
\pgfpathcurveto{\pgfqpoint{7.175325in}{1.593829in}}{\pgfqpoint{7.185924in}{1.589439in}}{\pgfqpoint{7.196974in}{1.589439in}}%
\pgfpathclose%
\pgfusepath{stroke,fill}%
\end{pgfscope}%
\begin{pgfscope}%
\pgfpathrectangle{\pgfqpoint{0.481978in}{0.331635in}}{\pgfqpoint{9.300000in}{7.700000in}}%
\pgfusepath{clip}%
\pgfsetbuttcap%
\pgfsetroundjoin%
\definecolor{currentfill}{rgb}{0.631373,0.788235,0.956863}%
\pgfsetfillcolor{currentfill}%
\pgfsetlinewidth{0.481800pt}%
\definecolor{currentstroke}{rgb}{1.000000,1.000000,1.000000}%
\pgfsetstrokecolor{currentstroke}%
\pgfsetdash{}{0pt}%
\pgfpathmoveto{\pgfqpoint{3.418933in}{1.305418in}}%
\pgfpathcurveto{\pgfqpoint{3.429983in}{1.305418in}}{\pgfqpoint{3.440582in}{1.309808in}}{\pgfqpoint{3.448396in}{1.317621in}}%
\pgfpathcurveto{\pgfqpoint{3.456210in}{1.325435in}}{\pgfqpoint{3.460600in}{1.336034in}}{\pgfqpoint{3.460600in}{1.347084in}}%
\pgfpathcurveto{\pgfqpoint{3.460600in}{1.358134in}}{\pgfqpoint{3.456210in}{1.368733in}}{\pgfqpoint{3.448396in}{1.376547in}}%
\pgfpathcurveto{\pgfqpoint{3.440582in}{1.384361in}}{\pgfqpoint{3.429983in}{1.388751in}}{\pgfqpoint{3.418933in}{1.388751in}}%
\pgfpathcurveto{\pgfqpoint{3.407883in}{1.388751in}}{\pgfqpoint{3.397284in}{1.384361in}}{\pgfqpoint{3.389471in}{1.376547in}}%
\pgfpathcurveto{\pgfqpoint{3.381657in}{1.368733in}}{\pgfqpoint{3.377267in}{1.358134in}}{\pgfqpoint{3.377267in}{1.347084in}}%
\pgfpathcurveto{\pgfqpoint{3.377267in}{1.336034in}}{\pgfqpoint{3.381657in}{1.325435in}}{\pgfqpoint{3.389471in}{1.317621in}}%
\pgfpathcurveto{\pgfqpoint{3.397284in}{1.309808in}}{\pgfqpoint{3.407883in}{1.305418in}}{\pgfqpoint{3.418933in}{1.305418in}}%
\pgfpathclose%
\pgfusepath{stroke,fill}%
\end{pgfscope}%
\begin{pgfscope}%
\pgfpathrectangle{\pgfqpoint{0.481978in}{0.331635in}}{\pgfqpoint{9.300000in}{7.700000in}}%
\pgfusepath{clip}%
\pgfsetbuttcap%
\pgfsetroundjoin%
\definecolor{currentfill}{rgb}{0.631373,0.788235,0.956863}%
\pgfsetfillcolor{currentfill}%
\pgfsetlinewidth{0.481800pt}%
\definecolor{currentstroke}{rgb}{1.000000,1.000000,1.000000}%
\pgfsetstrokecolor{currentstroke}%
\pgfsetdash{}{0pt}%
\pgfpathmoveto{\pgfqpoint{5.333628in}{3.453590in}}%
\pgfpathcurveto{\pgfqpoint{5.344678in}{3.453590in}}{\pgfqpoint{5.355277in}{3.457980in}}{\pgfqpoint{5.363090in}{3.465794in}}%
\pgfpathcurveto{\pgfqpoint{5.370904in}{3.473608in}}{\pgfqpoint{5.375294in}{3.484207in}}{\pgfqpoint{5.375294in}{3.495257in}}%
\pgfpathcurveto{\pgfqpoint{5.375294in}{3.506307in}}{\pgfqpoint{5.370904in}{3.516906in}}{\pgfqpoint{5.363090in}{3.524719in}}%
\pgfpathcurveto{\pgfqpoint{5.355277in}{3.532533in}}{\pgfqpoint{5.344678in}{3.536923in}}{\pgfqpoint{5.333628in}{3.536923in}}%
\pgfpathcurveto{\pgfqpoint{5.322578in}{3.536923in}}{\pgfqpoint{5.311979in}{3.532533in}}{\pgfqpoint{5.304165in}{3.524719in}}%
\pgfpathcurveto{\pgfqpoint{5.296351in}{3.516906in}}{\pgfqpoint{5.291961in}{3.506307in}}{\pgfqpoint{5.291961in}{3.495257in}}%
\pgfpathcurveto{\pgfqpoint{5.291961in}{3.484207in}}{\pgfqpoint{5.296351in}{3.473608in}}{\pgfqpoint{5.304165in}{3.465794in}}%
\pgfpathcurveto{\pgfqpoint{5.311979in}{3.457980in}}{\pgfqpoint{5.322578in}{3.453590in}}{\pgfqpoint{5.333628in}{3.453590in}}%
\pgfpathclose%
\pgfusepath{stroke,fill}%
\end{pgfscope}%
\begin{pgfscope}%
\pgfpathrectangle{\pgfqpoint{0.481978in}{0.331635in}}{\pgfqpoint{9.300000in}{7.700000in}}%
\pgfusepath{clip}%
\pgfsetbuttcap%
\pgfsetroundjoin%
\definecolor{currentfill}{rgb}{0.631373,0.788235,0.956863}%
\pgfsetfillcolor{currentfill}%
\pgfsetlinewidth{0.481800pt}%
\definecolor{currentstroke}{rgb}{1.000000,1.000000,1.000000}%
\pgfsetstrokecolor{currentstroke}%
\pgfsetdash{}{0pt}%
\pgfpathmoveto{\pgfqpoint{6.946844in}{2.793604in}}%
\pgfpathcurveto{\pgfqpoint{6.957894in}{2.793604in}}{\pgfqpoint{6.968493in}{2.797994in}}{\pgfqpoint{6.976307in}{2.805808in}}%
\pgfpathcurveto{\pgfqpoint{6.984120in}{2.813622in}}{\pgfqpoint{6.988511in}{2.824221in}}{\pgfqpoint{6.988511in}{2.835271in}}%
\pgfpathcurveto{\pgfqpoint{6.988511in}{2.846321in}}{\pgfqpoint{6.984120in}{2.856920in}}{\pgfqpoint{6.976307in}{2.864734in}}%
\pgfpathcurveto{\pgfqpoint{6.968493in}{2.872547in}}{\pgfqpoint{6.957894in}{2.876937in}}{\pgfqpoint{6.946844in}{2.876937in}}%
\pgfpathcurveto{\pgfqpoint{6.935794in}{2.876937in}}{\pgfqpoint{6.925195in}{2.872547in}}{\pgfqpoint{6.917381in}{2.864734in}}%
\pgfpathcurveto{\pgfqpoint{6.909568in}{2.856920in}}{\pgfqpoint{6.905177in}{2.846321in}}{\pgfqpoint{6.905177in}{2.835271in}}%
\pgfpathcurveto{\pgfqpoint{6.905177in}{2.824221in}}{\pgfqpoint{6.909568in}{2.813622in}}{\pgfqpoint{6.917381in}{2.805808in}}%
\pgfpathcurveto{\pgfqpoint{6.925195in}{2.797994in}}{\pgfqpoint{6.935794in}{2.793604in}}{\pgfqpoint{6.946844in}{2.793604in}}%
\pgfpathclose%
\pgfusepath{stroke,fill}%
\end{pgfscope}%
\begin{pgfscope}%
\pgfpathrectangle{\pgfqpoint{0.481978in}{0.331635in}}{\pgfqpoint{9.300000in}{7.700000in}}%
\pgfusepath{clip}%
\pgfsetbuttcap%
\pgfsetroundjoin%
\definecolor{currentfill}{rgb}{0.631373,0.788235,0.956863}%
\pgfsetfillcolor{currentfill}%
\pgfsetlinewidth{0.481800pt}%
\definecolor{currentstroke}{rgb}{1.000000,1.000000,1.000000}%
\pgfsetstrokecolor{currentstroke}%
\pgfsetdash{}{0pt}%
\pgfpathmoveto{\pgfqpoint{8.414129in}{4.979055in}}%
\pgfpathcurveto{\pgfqpoint{8.425179in}{4.979055in}}{\pgfqpoint{8.435778in}{4.983445in}}{\pgfqpoint{8.443592in}{4.991258in}}%
\pgfpathcurveto{\pgfqpoint{8.451406in}{4.999072in}}{\pgfqpoint{8.455796in}{5.009671in}}{\pgfqpoint{8.455796in}{5.020721in}}%
\pgfpathcurveto{\pgfqpoint{8.455796in}{5.031771in}}{\pgfqpoint{8.451406in}{5.042370in}}{\pgfqpoint{8.443592in}{5.050184in}}%
\pgfpathcurveto{\pgfqpoint{8.435778in}{5.057998in}}{\pgfqpoint{8.425179in}{5.062388in}}{\pgfqpoint{8.414129in}{5.062388in}}%
\pgfpathcurveto{\pgfqpoint{8.403079in}{5.062388in}}{\pgfqpoint{8.392480in}{5.057998in}}{\pgfqpoint{8.384666in}{5.050184in}}%
\pgfpathcurveto{\pgfqpoint{8.376853in}{5.042370in}}{\pgfqpoint{8.372463in}{5.031771in}}{\pgfqpoint{8.372463in}{5.020721in}}%
\pgfpathcurveto{\pgfqpoint{8.372463in}{5.009671in}}{\pgfqpoint{8.376853in}{4.999072in}}{\pgfqpoint{8.384666in}{4.991258in}}%
\pgfpathcurveto{\pgfqpoint{8.392480in}{4.983445in}}{\pgfqpoint{8.403079in}{4.979055in}}{\pgfqpoint{8.414129in}{4.979055in}}%
\pgfpathclose%
\pgfusepath{stroke,fill}%
\end{pgfscope}%
\begin{pgfscope}%
\pgfpathrectangle{\pgfqpoint{0.481978in}{0.331635in}}{\pgfqpoint{9.300000in}{7.700000in}}%
\pgfusepath{clip}%
\pgfsetbuttcap%
\pgfsetroundjoin%
\definecolor{currentfill}{rgb}{0.631373,0.788235,0.956863}%
\pgfsetfillcolor{currentfill}%
\pgfsetlinewidth{0.481800pt}%
\definecolor{currentstroke}{rgb}{1.000000,1.000000,1.000000}%
\pgfsetstrokecolor{currentstroke}%
\pgfsetdash{}{0pt}%
\pgfpathmoveto{\pgfqpoint{4.957931in}{2.096496in}}%
\pgfpathcurveto{\pgfqpoint{4.968982in}{2.096496in}}{\pgfqpoint{4.979581in}{2.100886in}}{\pgfqpoint{4.987394in}{2.108699in}}%
\pgfpathcurveto{\pgfqpoint{4.995208in}{2.116513in}}{\pgfqpoint{4.999598in}{2.127112in}}{\pgfqpoint{4.999598in}{2.138162in}}%
\pgfpathcurveto{\pgfqpoint{4.999598in}{2.149212in}}{\pgfqpoint{4.995208in}{2.159811in}}{\pgfqpoint{4.987394in}{2.167625in}}%
\pgfpathcurveto{\pgfqpoint{4.979581in}{2.175439in}}{\pgfqpoint{4.968982in}{2.179829in}}{\pgfqpoint{4.957931in}{2.179829in}}%
\pgfpathcurveto{\pgfqpoint{4.946881in}{2.179829in}}{\pgfqpoint{4.936282in}{2.175439in}}{\pgfqpoint{4.928469in}{2.167625in}}%
\pgfpathcurveto{\pgfqpoint{4.920655in}{2.159811in}}{\pgfqpoint{4.916265in}{2.149212in}}{\pgfqpoint{4.916265in}{2.138162in}}%
\pgfpathcurveto{\pgfqpoint{4.916265in}{2.127112in}}{\pgfqpoint{4.920655in}{2.116513in}}{\pgfqpoint{4.928469in}{2.108699in}}%
\pgfpathcurveto{\pgfqpoint{4.936282in}{2.100886in}}{\pgfqpoint{4.946881in}{2.096496in}}{\pgfqpoint{4.957931in}{2.096496in}}%
\pgfpathclose%
\pgfusepath{stroke,fill}%
\end{pgfscope}%
\begin{pgfscope}%
\pgfpathrectangle{\pgfqpoint{0.481978in}{0.331635in}}{\pgfqpoint{9.300000in}{7.700000in}}%
\pgfusepath{clip}%
\pgfsetbuttcap%
\pgfsetroundjoin%
\definecolor{currentfill}{rgb}{0.631373,0.788235,0.956863}%
\pgfsetfillcolor{currentfill}%
\pgfsetlinewidth{0.481800pt}%
\definecolor{currentstroke}{rgb}{1.000000,1.000000,1.000000}%
\pgfsetstrokecolor{currentstroke}%
\pgfsetdash{}{0pt}%
\pgfpathmoveto{\pgfqpoint{7.947630in}{5.352040in}}%
\pgfpathcurveto{\pgfqpoint{7.958680in}{5.352040in}}{\pgfqpoint{7.969279in}{5.356431in}}{\pgfqpoint{7.977093in}{5.364244in}}%
\pgfpathcurveto{\pgfqpoint{7.984906in}{5.372058in}}{\pgfqpoint{7.989297in}{5.382657in}}{\pgfqpoint{7.989297in}{5.393707in}}%
\pgfpathcurveto{\pgfqpoint{7.989297in}{5.404757in}}{\pgfqpoint{7.984906in}{5.415356in}}{\pgfqpoint{7.977093in}{5.423170in}}%
\pgfpathcurveto{\pgfqpoint{7.969279in}{5.430983in}}{\pgfqpoint{7.958680in}{5.435374in}}{\pgfqpoint{7.947630in}{5.435374in}}%
\pgfpathcurveto{\pgfqpoint{7.936580in}{5.435374in}}{\pgfqpoint{7.925981in}{5.430983in}}{\pgfqpoint{7.918167in}{5.423170in}}%
\pgfpathcurveto{\pgfqpoint{7.910354in}{5.415356in}}{\pgfqpoint{7.905963in}{5.404757in}}{\pgfqpoint{7.905963in}{5.393707in}}%
\pgfpathcurveto{\pgfqpoint{7.905963in}{5.382657in}}{\pgfqpoint{7.910354in}{5.372058in}}{\pgfqpoint{7.918167in}{5.364244in}}%
\pgfpathcurveto{\pgfqpoint{7.925981in}{5.356431in}}{\pgfqpoint{7.936580in}{5.352040in}}{\pgfqpoint{7.947630in}{5.352040in}}%
\pgfpathclose%
\pgfusepath{stroke,fill}%
\end{pgfscope}%
\begin{pgfscope}%
\pgfpathrectangle{\pgfqpoint{0.481978in}{0.331635in}}{\pgfqpoint{9.300000in}{7.700000in}}%
\pgfusepath{clip}%
\pgfsetbuttcap%
\pgfsetroundjoin%
\definecolor{currentfill}{rgb}{0.631373,0.788235,0.956863}%
\pgfsetfillcolor{currentfill}%
\pgfsetlinewidth{0.481800pt}%
\definecolor{currentstroke}{rgb}{1.000000,1.000000,1.000000}%
\pgfsetstrokecolor{currentstroke}%
\pgfsetdash{}{0pt}%
\pgfpathmoveto{\pgfqpoint{3.205045in}{1.529755in}}%
\pgfpathcurveto{\pgfqpoint{3.216095in}{1.529755in}}{\pgfqpoint{3.226694in}{1.534145in}}{\pgfqpoint{3.234508in}{1.541959in}}%
\pgfpathcurveto{\pgfqpoint{3.242321in}{1.549773in}}{\pgfqpoint{3.246711in}{1.560372in}}{\pgfqpoint{3.246711in}{1.571422in}}%
\pgfpathcurveto{\pgfqpoint{3.246711in}{1.582472in}}{\pgfqpoint{3.242321in}{1.593071in}}{\pgfqpoint{3.234508in}{1.600885in}}%
\pgfpathcurveto{\pgfqpoint{3.226694in}{1.608698in}}{\pgfqpoint{3.216095in}{1.613088in}}{\pgfqpoint{3.205045in}{1.613088in}}%
\pgfpathcurveto{\pgfqpoint{3.193995in}{1.613088in}}{\pgfqpoint{3.183396in}{1.608698in}}{\pgfqpoint{3.175582in}{1.600885in}}%
\pgfpathcurveto{\pgfqpoint{3.167768in}{1.593071in}}{\pgfqpoint{3.163378in}{1.582472in}}{\pgfqpoint{3.163378in}{1.571422in}}%
\pgfpathcurveto{\pgfqpoint{3.163378in}{1.560372in}}{\pgfqpoint{3.167768in}{1.549773in}}{\pgfqpoint{3.175582in}{1.541959in}}%
\pgfpathcurveto{\pgfqpoint{3.183396in}{1.534145in}}{\pgfqpoint{3.193995in}{1.529755in}}{\pgfqpoint{3.205045in}{1.529755in}}%
\pgfpathclose%
\pgfusepath{stroke,fill}%
\end{pgfscope}%
\begin{pgfscope}%
\pgfpathrectangle{\pgfqpoint{0.481978in}{0.331635in}}{\pgfqpoint{9.300000in}{7.700000in}}%
\pgfusepath{clip}%
\pgfsetbuttcap%
\pgfsetroundjoin%
\definecolor{currentfill}{rgb}{0.631373,0.788235,0.956863}%
\pgfsetfillcolor{currentfill}%
\pgfsetlinewidth{0.481800pt}%
\definecolor{currentstroke}{rgb}{1.000000,1.000000,1.000000}%
\pgfsetstrokecolor{currentstroke}%
\pgfsetdash{}{0pt}%
\pgfpathmoveto{\pgfqpoint{7.723488in}{5.483998in}}%
\pgfpathcurveto{\pgfqpoint{7.734538in}{5.483998in}}{\pgfqpoint{7.745137in}{5.488389in}}{\pgfqpoint{7.752951in}{5.496202in}}%
\pgfpathcurveto{\pgfqpoint{7.760765in}{5.504016in}}{\pgfqpoint{7.765155in}{5.514615in}}{\pgfqpoint{7.765155in}{5.525665in}}%
\pgfpathcurveto{\pgfqpoint{7.765155in}{5.536715in}}{\pgfqpoint{7.760765in}{5.547314in}}{\pgfqpoint{7.752951in}{5.555128in}}%
\pgfpathcurveto{\pgfqpoint{7.745137in}{5.562942in}}{\pgfqpoint{7.734538in}{5.567332in}}{\pgfqpoint{7.723488in}{5.567332in}}%
\pgfpathcurveto{\pgfqpoint{7.712438in}{5.567332in}}{\pgfqpoint{7.701839in}{5.562942in}}{\pgfqpoint{7.694025in}{5.555128in}}%
\pgfpathcurveto{\pgfqpoint{7.686212in}{5.547314in}}{\pgfqpoint{7.681822in}{5.536715in}}{\pgfqpoint{7.681822in}{5.525665in}}%
\pgfpathcurveto{\pgfqpoint{7.681822in}{5.514615in}}{\pgfqpoint{7.686212in}{5.504016in}}{\pgfqpoint{7.694025in}{5.496202in}}%
\pgfpathcurveto{\pgfqpoint{7.701839in}{5.488389in}}{\pgfqpoint{7.712438in}{5.483998in}}{\pgfqpoint{7.723488in}{5.483998in}}%
\pgfpathclose%
\pgfusepath{stroke,fill}%
\end{pgfscope}%
\begin{pgfscope}%
\pgfpathrectangle{\pgfqpoint{0.481978in}{0.331635in}}{\pgfqpoint{9.300000in}{7.700000in}}%
\pgfusepath{clip}%
\pgfsetbuttcap%
\pgfsetroundjoin%
\definecolor{currentfill}{rgb}{0.631373,0.788235,0.956863}%
\pgfsetfillcolor{currentfill}%
\pgfsetlinewidth{0.481800pt}%
\definecolor{currentstroke}{rgb}{1.000000,1.000000,1.000000}%
\pgfsetstrokecolor{currentstroke}%
\pgfsetdash{}{0pt}%
\pgfpathmoveto{\pgfqpoint{4.705109in}{1.044958in}}%
\pgfpathcurveto{\pgfqpoint{4.716159in}{1.044958in}}{\pgfqpoint{4.726758in}{1.049348in}}{\pgfqpoint{4.734572in}{1.057162in}}%
\pgfpathcurveto{\pgfqpoint{4.742386in}{1.064976in}}{\pgfqpoint{4.746776in}{1.075575in}}{\pgfqpoint{4.746776in}{1.086625in}}%
\pgfpathcurveto{\pgfqpoint{4.746776in}{1.097675in}}{\pgfqpoint{4.742386in}{1.108274in}}{\pgfqpoint{4.734572in}{1.116088in}}%
\pgfpathcurveto{\pgfqpoint{4.726758in}{1.123901in}}{\pgfqpoint{4.716159in}{1.128291in}}{\pgfqpoint{4.705109in}{1.128291in}}%
\pgfpathcurveto{\pgfqpoint{4.694059in}{1.128291in}}{\pgfqpoint{4.683460in}{1.123901in}}{\pgfqpoint{4.675646in}{1.116088in}}%
\pgfpathcurveto{\pgfqpoint{4.667833in}{1.108274in}}{\pgfqpoint{4.663442in}{1.097675in}}{\pgfqpoint{4.663442in}{1.086625in}}%
\pgfpathcurveto{\pgfqpoint{4.663442in}{1.075575in}}{\pgfqpoint{4.667833in}{1.064976in}}{\pgfqpoint{4.675646in}{1.057162in}}%
\pgfpathcurveto{\pgfqpoint{4.683460in}{1.049348in}}{\pgfqpoint{4.694059in}{1.044958in}}{\pgfqpoint{4.705109in}{1.044958in}}%
\pgfpathclose%
\pgfusepath{stroke,fill}%
\end{pgfscope}%
\begin{pgfscope}%
\pgfpathrectangle{\pgfqpoint{0.481978in}{0.331635in}}{\pgfqpoint{9.300000in}{7.700000in}}%
\pgfusepath{clip}%
\pgfsetbuttcap%
\pgfsetroundjoin%
\definecolor{currentfill}{rgb}{0.631373,0.788235,0.956863}%
\pgfsetfillcolor{currentfill}%
\pgfsetlinewidth{0.481800pt}%
\definecolor{currentstroke}{rgb}{1.000000,1.000000,1.000000}%
\pgfsetstrokecolor{currentstroke}%
\pgfsetdash{}{0pt}%
\pgfpathmoveto{\pgfqpoint{3.932688in}{3.871886in}}%
\pgfpathcurveto{\pgfqpoint{3.943738in}{3.871886in}}{\pgfqpoint{3.954337in}{3.876276in}}{\pgfqpoint{3.962150in}{3.884090in}}%
\pgfpathcurveto{\pgfqpoint{3.969964in}{3.891903in}}{\pgfqpoint{3.974354in}{3.902502in}}{\pgfqpoint{3.974354in}{3.913553in}}%
\pgfpathcurveto{\pgfqpoint{3.974354in}{3.924603in}}{\pgfqpoint{3.969964in}{3.935202in}}{\pgfqpoint{3.962150in}{3.943015in}}%
\pgfpathcurveto{\pgfqpoint{3.954337in}{3.950829in}}{\pgfqpoint{3.943738in}{3.955219in}}{\pgfqpoint{3.932688in}{3.955219in}}%
\pgfpathcurveto{\pgfqpoint{3.921637in}{3.955219in}}{\pgfqpoint{3.911038in}{3.950829in}}{\pgfqpoint{3.903225in}{3.943015in}}%
\pgfpathcurveto{\pgfqpoint{3.895411in}{3.935202in}}{\pgfqpoint{3.891021in}{3.924603in}}{\pgfqpoint{3.891021in}{3.913553in}}%
\pgfpathcurveto{\pgfqpoint{3.891021in}{3.902502in}}{\pgfqpoint{3.895411in}{3.891903in}}{\pgfqpoint{3.903225in}{3.884090in}}%
\pgfpathcurveto{\pgfqpoint{3.911038in}{3.876276in}}{\pgfqpoint{3.921637in}{3.871886in}}{\pgfqpoint{3.932688in}{3.871886in}}%
\pgfpathclose%
\pgfusepath{stroke,fill}%
\end{pgfscope}%
\begin{pgfscope}%
\pgfpathrectangle{\pgfqpoint{0.481978in}{0.331635in}}{\pgfqpoint{9.300000in}{7.700000in}}%
\pgfusepath{clip}%
\pgfsetbuttcap%
\pgfsetroundjoin%
\definecolor{currentfill}{rgb}{0.631373,0.788235,0.956863}%
\pgfsetfillcolor{currentfill}%
\pgfsetlinewidth{0.481800pt}%
\definecolor{currentstroke}{rgb}{1.000000,1.000000,1.000000}%
\pgfsetstrokecolor{currentstroke}%
\pgfsetdash{}{0pt}%
\pgfpathmoveto{\pgfqpoint{7.599765in}{5.924449in}}%
\pgfpathcurveto{\pgfqpoint{7.610815in}{5.924449in}}{\pgfqpoint{7.621414in}{5.928840in}}{\pgfqpoint{7.629228in}{5.936653in}}%
\pgfpathcurveto{\pgfqpoint{7.637042in}{5.944467in}}{\pgfqpoint{7.641432in}{5.955066in}}{\pgfqpoint{7.641432in}{5.966116in}}%
\pgfpathcurveto{\pgfqpoint{7.641432in}{5.977166in}}{\pgfqpoint{7.637042in}{5.987765in}}{\pgfqpoint{7.629228in}{5.995579in}}%
\pgfpathcurveto{\pgfqpoint{7.621414in}{6.003392in}}{\pgfqpoint{7.610815in}{6.007783in}}{\pgfqpoint{7.599765in}{6.007783in}}%
\pgfpathcurveto{\pgfqpoint{7.588715in}{6.007783in}}{\pgfqpoint{7.578116in}{6.003392in}}{\pgfqpoint{7.570302in}{5.995579in}}%
\pgfpathcurveto{\pgfqpoint{7.562489in}{5.987765in}}{\pgfqpoint{7.558098in}{5.977166in}}{\pgfqpoint{7.558098in}{5.966116in}}%
\pgfpathcurveto{\pgfqpoint{7.558098in}{5.955066in}}{\pgfqpoint{7.562489in}{5.944467in}}{\pgfqpoint{7.570302in}{5.936653in}}%
\pgfpathcurveto{\pgfqpoint{7.578116in}{5.928840in}}{\pgfqpoint{7.588715in}{5.924449in}}{\pgfqpoint{7.599765in}{5.924449in}}%
\pgfpathclose%
\pgfusepath{stroke,fill}%
\end{pgfscope}%
\begin{pgfscope}%
\pgfpathrectangle{\pgfqpoint{0.481978in}{0.331635in}}{\pgfqpoint{9.300000in}{7.700000in}}%
\pgfusepath{clip}%
\pgfsetbuttcap%
\pgfsetroundjoin%
\definecolor{currentfill}{rgb}{0.631373,0.788235,0.956863}%
\pgfsetfillcolor{currentfill}%
\pgfsetlinewidth{0.481800pt}%
\definecolor{currentstroke}{rgb}{1.000000,1.000000,1.000000}%
\pgfsetstrokecolor{currentstroke}%
\pgfsetdash{}{0pt}%
\pgfpathmoveto{\pgfqpoint{6.870443in}{1.661336in}}%
\pgfpathcurveto{\pgfqpoint{6.881493in}{1.661336in}}{\pgfqpoint{6.892092in}{1.665727in}}{\pgfqpoint{6.899905in}{1.673540in}}%
\pgfpathcurveto{\pgfqpoint{6.907719in}{1.681354in}}{\pgfqpoint{6.912109in}{1.691953in}}{\pgfqpoint{6.912109in}{1.703003in}}%
\pgfpathcurveto{\pgfqpoint{6.912109in}{1.714053in}}{\pgfqpoint{6.907719in}{1.724652in}}{\pgfqpoint{6.899905in}{1.732466in}}%
\pgfpathcurveto{\pgfqpoint{6.892092in}{1.740279in}}{\pgfqpoint{6.881493in}{1.744670in}}{\pgfqpoint{6.870443in}{1.744670in}}%
\pgfpathcurveto{\pgfqpoint{6.859393in}{1.744670in}}{\pgfqpoint{6.848794in}{1.740279in}}{\pgfqpoint{6.840980in}{1.732466in}}%
\pgfpathcurveto{\pgfqpoint{6.833166in}{1.724652in}}{\pgfqpoint{6.828776in}{1.714053in}}{\pgfqpoint{6.828776in}{1.703003in}}%
\pgfpathcurveto{\pgfqpoint{6.828776in}{1.691953in}}{\pgfqpoint{6.833166in}{1.681354in}}{\pgfqpoint{6.840980in}{1.673540in}}%
\pgfpathcurveto{\pgfqpoint{6.848794in}{1.665727in}}{\pgfqpoint{6.859393in}{1.661336in}}{\pgfqpoint{6.870443in}{1.661336in}}%
\pgfpathclose%
\pgfusepath{stroke,fill}%
\end{pgfscope}%
\begin{pgfscope}%
\pgfpathrectangle{\pgfqpoint{0.481978in}{0.331635in}}{\pgfqpoint{9.300000in}{7.700000in}}%
\pgfusepath{clip}%
\pgfsetbuttcap%
\pgfsetroundjoin%
\definecolor{currentfill}{rgb}{0.631373,0.788235,0.956863}%
\pgfsetfillcolor{currentfill}%
\pgfsetlinewidth{0.481800pt}%
\definecolor{currentstroke}{rgb}{1.000000,1.000000,1.000000}%
\pgfsetstrokecolor{currentstroke}%
\pgfsetdash{}{0pt}%
\pgfpathmoveto{\pgfqpoint{6.112529in}{2.110735in}}%
\pgfpathcurveto{\pgfqpoint{6.123579in}{2.110735in}}{\pgfqpoint{6.134178in}{2.115125in}}{\pgfqpoint{6.141991in}{2.122939in}}%
\pgfpathcurveto{\pgfqpoint{6.149805in}{2.130752in}}{\pgfqpoint{6.154195in}{2.141352in}}{\pgfqpoint{6.154195in}{2.152402in}}%
\pgfpathcurveto{\pgfqpoint{6.154195in}{2.163452in}}{\pgfqpoint{6.149805in}{2.174051in}}{\pgfqpoint{6.141991in}{2.181864in}}%
\pgfpathcurveto{\pgfqpoint{6.134178in}{2.189678in}}{\pgfqpoint{6.123579in}{2.194068in}}{\pgfqpoint{6.112529in}{2.194068in}}%
\pgfpathcurveto{\pgfqpoint{6.101478in}{2.194068in}}{\pgfqpoint{6.090879in}{2.189678in}}{\pgfqpoint{6.083066in}{2.181864in}}%
\pgfpathcurveto{\pgfqpoint{6.075252in}{2.174051in}}{\pgfqpoint{6.070862in}{2.163452in}}{\pgfqpoint{6.070862in}{2.152402in}}%
\pgfpathcurveto{\pgfqpoint{6.070862in}{2.141352in}}{\pgfqpoint{6.075252in}{2.130752in}}{\pgfqpoint{6.083066in}{2.122939in}}%
\pgfpathcurveto{\pgfqpoint{6.090879in}{2.115125in}}{\pgfqpoint{6.101478in}{2.110735in}}{\pgfqpoint{6.112529in}{2.110735in}}%
\pgfpathclose%
\pgfusepath{stroke,fill}%
\end{pgfscope}%
\begin{pgfscope}%
\pgfpathrectangle{\pgfqpoint{0.481978in}{0.331635in}}{\pgfqpoint{9.300000in}{7.700000in}}%
\pgfusepath{clip}%
\pgfsetbuttcap%
\pgfsetroundjoin%
\definecolor{currentfill}{rgb}{0.631373,0.788235,0.956863}%
\pgfsetfillcolor{currentfill}%
\pgfsetlinewidth{0.481800pt}%
\definecolor{currentstroke}{rgb}{1.000000,1.000000,1.000000}%
\pgfsetstrokecolor{currentstroke}%
\pgfsetdash{}{0pt}%
\pgfpathmoveto{\pgfqpoint{7.807922in}{5.754210in}}%
\pgfpathcurveto{\pgfqpoint{7.818972in}{5.754210in}}{\pgfqpoint{7.829571in}{5.758600in}}{\pgfqpoint{7.837385in}{5.766413in}}%
\pgfpathcurveto{\pgfqpoint{7.845199in}{5.774227in}}{\pgfqpoint{7.849589in}{5.784826in}}{\pgfqpoint{7.849589in}{5.795876in}}%
\pgfpathcurveto{\pgfqpoint{7.849589in}{5.806926in}}{\pgfqpoint{7.845199in}{5.817525in}}{\pgfqpoint{7.837385in}{5.825339in}}%
\pgfpathcurveto{\pgfqpoint{7.829571in}{5.833153in}}{\pgfqpoint{7.818972in}{5.837543in}}{\pgfqpoint{7.807922in}{5.837543in}}%
\pgfpathcurveto{\pgfqpoint{7.796872in}{5.837543in}}{\pgfqpoint{7.786273in}{5.833153in}}{\pgfqpoint{7.778460in}{5.825339in}}%
\pgfpathcurveto{\pgfqpoint{7.770646in}{5.817525in}}{\pgfqpoint{7.766256in}{5.806926in}}{\pgfqpoint{7.766256in}{5.795876in}}%
\pgfpathcurveto{\pgfqpoint{7.766256in}{5.784826in}}{\pgfqpoint{7.770646in}{5.774227in}}{\pgfqpoint{7.778460in}{5.766413in}}%
\pgfpathcurveto{\pgfqpoint{7.786273in}{5.758600in}}{\pgfqpoint{7.796872in}{5.754210in}}{\pgfqpoint{7.807922in}{5.754210in}}%
\pgfpathclose%
\pgfusepath{stroke,fill}%
\end{pgfscope}%
\begin{pgfscope}%
\pgfpathrectangle{\pgfqpoint{0.481978in}{0.331635in}}{\pgfqpoint{9.300000in}{7.700000in}}%
\pgfusepath{clip}%
\pgfsetbuttcap%
\pgfsetroundjoin%
\definecolor{currentfill}{rgb}{0.631373,0.788235,0.956863}%
\pgfsetfillcolor{currentfill}%
\pgfsetlinewidth{0.481800pt}%
\definecolor{currentstroke}{rgb}{1.000000,1.000000,1.000000}%
\pgfsetstrokecolor{currentstroke}%
\pgfsetdash{}{0pt}%
\pgfpathmoveto{\pgfqpoint{3.326485in}{2.016983in}}%
\pgfpathcurveto{\pgfqpoint{3.337535in}{2.016983in}}{\pgfqpoint{3.348134in}{2.021373in}}{\pgfqpoint{3.355947in}{2.029187in}}%
\pgfpathcurveto{\pgfqpoint{3.363761in}{2.037001in}}{\pgfqpoint{3.368151in}{2.047600in}}{\pgfqpoint{3.368151in}{2.058650in}}%
\pgfpathcurveto{\pgfqpoint{3.368151in}{2.069700in}}{\pgfqpoint{3.363761in}{2.080299in}}{\pgfqpoint{3.355947in}{2.088113in}}%
\pgfpathcurveto{\pgfqpoint{3.348134in}{2.095926in}}{\pgfqpoint{3.337535in}{2.100317in}}{\pgfqpoint{3.326485in}{2.100317in}}%
\pgfpathcurveto{\pgfqpoint{3.315435in}{2.100317in}}{\pgfqpoint{3.304836in}{2.095926in}}{\pgfqpoint{3.297022in}{2.088113in}}%
\pgfpathcurveto{\pgfqpoint{3.289208in}{2.080299in}}{\pgfqpoint{3.284818in}{2.069700in}}{\pgfqpoint{3.284818in}{2.058650in}}%
\pgfpathcurveto{\pgfqpoint{3.284818in}{2.047600in}}{\pgfqpoint{3.289208in}{2.037001in}}{\pgfqpoint{3.297022in}{2.029187in}}%
\pgfpathcurveto{\pgfqpoint{3.304836in}{2.021373in}}{\pgfqpoint{3.315435in}{2.016983in}}{\pgfqpoint{3.326485in}{2.016983in}}%
\pgfpathclose%
\pgfusepath{stroke,fill}%
\end{pgfscope}%
\begin{pgfscope}%
\pgfpathrectangle{\pgfqpoint{0.481978in}{0.331635in}}{\pgfqpoint{9.300000in}{7.700000in}}%
\pgfusepath{clip}%
\pgfsetbuttcap%
\pgfsetroundjoin%
\definecolor{currentfill}{rgb}{0.631373,0.788235,0.956863}%
\pgfsetfillcolor{currentfill}%
\pgfsetlinewidth{0.481800pt}%
\definecolor{currentstroke}{rgb}{1.000000,1.000000,1.000000}%
\pgfsetstrokecolor{currentstroke}%
\pgfsetdash{}{0pt}%
\pgfpathmoveto{\pgfqpoint{1.925994in}{5.466001in}}%
\pgfpathcurveto{\pgfqpoint{1.937044in}{5.466001in}}{\pgfqpoint{1.947643in}{5.470391in}}{\pgfqpoint{1.955457in}{5.478205in}}%
\pgfpathcurveto{\pgfqpoint{1.963270in}{5.486018in}}{\pgfqpoint{1.967660in}{5.496617in}}{\pgfqpoint{1.967660in}{5.507668in}}%
\pgfpathcurveto{\pgfqpoint{1.967660in}{5.518718in}}{\pgfqpoint{1.963270in}{5.529317in}}{\pgfqpoint{1.955457in}{5.537130in}}%
\pgfpathcurveto{\pgfqpoint{1.947643in}{5.544944in}}{\pgfqpoint{1.937044in}{5.549334in}}{\pgfqpoint{1.925994in}{5.549334in}}%
\pgfpathcurveto{\pgfqpoint{1.914944in}{5.549334in}}{\pgfqpoint{1.904345in}{5.544944in}}{\pgfqpoint{1.896531in}{5.537130in}}%
\pgfpathcurveto{\pgfqpoint{1.888717in}{5.529317in}}{\pgfqpoint{1.884327in}{5.518718in}}{\pgfqpoint{1.884327in}{5.507668in}}%
\pgfpathcurveto{\pgfqpoint{1.884327in}{5.496617in}}{\pgfqpoint{1.888717in}{5.486018in}}{\pgfqpoint{1.896531in}{5.478205in}}%
\pgfpathcurveto{\pgfqpoint{1.904345in}{5.470391in}}{\pgfqpoint{1.914944in}{5.466001in}}{\pgfqpoint{1.925994in}{5.466001in}}%
\pgfpathclose%
\pgfusepath{stroke,fill}%
\end{pgfscope}%
\begin{pgfscope}%
\pgfpathrectangle{\pgfqpoint{0.481978in}{0.331635in}}{\pgfqpoint{9.300000in}{7.700000in}}%
\pgfusepath{clip}%
\pgfsetbuttcap%
\pgfsetroundjoin%
\definecolor{currentfill}{rgb}{0.631373,0.788235,0.956863}%
\pgfsetfillcolor{currentfill}%
\pgfsetlinewidth{0.481800pt}%
\definecolor{currentstroke}{rgb}{1.000000,1.000000,1.000000}%
\pgfsetstrokecolor{currentstroke}%
\pgfsetdash{}{0pt}%
\pgfpathmoveto{\pgfqpoint{5.878648in}{0.807236in}}%
\pgfpathcurveto{\pgfqpoint{5.889698in}{0.807236in}}{\pgfqpoint{5.900297in}{0.811627in}}{\pgfqpoint{5.908111in}{0.819440in}}%
\pgfpathcurveto{\pgfqpoint{5.915925in}{0.827254in}}{\pgfqpoint{5.920315in}{0.837853in}}{\pgfqpoint{5.920315in}{0.848903in}}%
\pgfpathcurveto{\pgfqpoint{5.920315in}{0.859953in}}{\pgfqpoint{5.915925in}{0.870552in}}{\pgfqpoint{5.908111in}{0.878366in}}%
\pgfpathcurveto{\pgfqpoint{5.900297in}{0.886179in}}{\pgfqpoint{5.889698in}{0.890570in}}{\pgfqpoint{5.878648in}{0.890570in}}%
\pgfpathcurveto{\pgfqpoint{5.867598in}{0.890570in}}{\pgfqpoint{5.856999in}{0.886179in}}{\pgfqpoint{5.849185in}{0.878366in}}%
\pgfpathcurveto{\pgfqpoint{5.841372in}{0.870552in}}{\pgfqpoint{5.836981in}{0.859953in}}{\pgfqpoint{5.836981in}{0.848903in}}%
\pgfpathcurveto{\pgfqpoint{5.836981in}{0.837853in}}{\pgfqpoint{5.841372in}{0.827254in}}{\pgfqpoint{5.849185in}{0.819440in}}%
\pgfpathcurveto{\pgfqpoint{5.856999in}{0.811627in}}{\pgfqpoint{5.867598in}{0.807236in}}{\pgfqpoint{5.878648in}{0.807236in}}%
\pgfpathclose%
\pgfusepath{stroke,fill}%
\end{pgfscope}%
\begin{pgfscope}%
\pgfpathrectangle{\pgfqpoint{0.481978in}{0.331635in}}{\pgfqpoint{9.300000in}{7.700000in}}%
\pgfusepath{clip}%
\pgfsetbuttcap%
\pgfsetroundjoin%
\definecolor{currentfill}{rgb}{0.631373,0.788235,0.956863}%
\pgfsetfillcolor{currentfill}%
\pgfsetlinewidth{0.481800pt}%
\definecolor{currentstroke}{rgb}{1.000000,1.000000,1.000000}%
\pgfsetstrokecolor{currentstroke}%
\pgfsetdash{}{0pt}%
\pgfpathmoveto{\pgfqpoint{8.178842in}{4.564124in}}%
\pgfpathcurveto{\pgfqpoint{8.189892in}{4.564124in}}{\pgfqpoint{8.200491in}{4.568514in}}{\pgfqpoint{8.208305in}{4.576327in}}%
\pgfpathcurveto{\pgfqpoint{8.216119in}{4.584141in}}{\pgfqpoint{8.220509in}{4.594740in}}{\pgfqpoint{8.220509in}{4.605790in}}%
\pgfpathcurveto{\pgfqpoint{8.220509in}{4.616840in}}{\pgfqpoint{8.216119in}{4.627439in}}{\pgfqpoint{8.208305in}{4.635253in}}%
\pgfpathcurveto{\pgfqpoint{8.200491in}{4.643067in}}{\pgfqpoint{8.189892in}{4.647457in}}{\pgfqpoint{8.178842in}{4.647457in}}%
\pgfpathcurveto{\pgfqpoint{8.167792in}{4.647457in}}{\pgfqpoint{8.157193in}{4.643067in}}{\pgfqpoint{8.149379in}{4.635253in}}%
\pgfpathcurveto{\pgfqpoint{8.141566in}{4.627439in}}{\pgfqpoint{8.137176in}{4.616840in}}{\pgfqpoint{8.137176in}{4.605790in}}%
\pgfpathcurveto{\pgfqpoint{8.137176in}{4.594740in}}{\pgfqpoint{8.141566in}{4.584141in}}{\pgfqpoint{8.149379in}{4.576327in}}%
\pgfpathcurveto{\pgfqpoint{8.157193in}{4.568514in}}{\pgfqpoint{8.167792in}{4.564124in}}{\pgfqpoint{8.178842in}{4.564124in}}%
\pgfpathclose%
\pgfusepath{stroke,fill}%
\end{pgfscope}%
\begin{pgfscope}%
\pgfpathrectangle{\pgfqpoint{0.481978in}{0.331635in}}{\pgfqpoint{9.300000in}{7.700000in}}%
\pgfusepath{clip}%
\pgfsetbuttcap%
\pgfsetroundjoin%
\definecolor{currentfill}{rgb}{0.631373,0.788235,0.956863}%
\pgfsetfillcolor{currentfill}%
\pgfsetlinewidth{0.481800pt}%
\definecolor{currentstroke}{rgb}{1.000000,1.000000,1.000000}%
\pgfsetstrokecolor{currentstroke}%
\pgfsetdash{}{0pt}%
\pgfpathmoveto{\pgfqpoint{6.339836in}{2.841214in}}%
\pgfpathcurveto{\pgfqpoint{6.350886in}{2.841214in}}{\pgfqpoint{6.361485in}{2.845605in}}{\pgfqpoint{6.369299in}{2.853418in}}%
\pgfpathcurveto{\pgfqpoint{6.377112in}{2.861232in}}{\pgfqpoint{6.381503in}{2.871831in}}{\pgfqpoint{6.381503in}{2.882881in}}%
\pgfpathcurveto{\pgfqpoint{6.381503in}{2.893931in}}{\pgfqpoint{6.377112in}{2.904530in}}{\pgfqpoint{6.369299in}{2.912344in}}%
\pgfpathcurveto{\pgfqpoint{6.361485in}{2.920157in}}{\pgfqpoint{6.350886in}{2.924548in}}{\pgfqpoint{6.339836in}{2.924548in}}%
\pgfpathcurveto{\pgfqpoint{6.328786in}{2.924548in}}{\pgfqpoint{6.318187in}{2.920157in}}{\pgfqpoint{6.310373in}{2.912344in}}%
\pgfpathcurveto{\pgfqpoint{6.302560in}{2.904530in}}{\pgfqpoint{6.298169in}{2.893931in}}{\pgfqpoint{6.298169in}{2.882881in}}%
\pgfpathcurveto{\pgfqpoint{6.298169in}{2.871831in}}{\pgfqpoint{6.302560in}{2.861232in}}{\pgfqpoint{6.310373in}{2.853418in}}%
\pgfpathcurveto{\pgfqpoint{6.318187in}{2.845605in}}{\pgfqpoint{6.328786in}{2.841214in}}{\pgfqpoint{6.339836in}{2.841214in}}%
\pgfpathclose%
\pgfusepath{stroke,fill}%
\end{pgfscope}%
\begin{pgfscope}%
\pgfpathrectangle{\pgfqpoint{0.481978in}{0.331635in}}{\pgfqpoint{9.300000in}{7.700000in}}%
\pgfusepath{clip}%
\pgfsetbuttcap%
\pgfsetroundjoin%
\definecolor{currentfill}{rgb}{0.631373,0.788235,0.956863}%
\pgfsetfillcolor{currentfill}%
\pgfsetlinewidth{0.481800pt}%
\definecolor{currentstroke}{rgb}{1.000000,1.000000,1.000000}%
\pgfsetstrokecolor{currentstroke}%
\pgfsetdash{}{0pt}%
\pgfpathmoveto{\pgfqpoint{5.971524in}{2.035399in}}%
\pgfpathcurveto{\pgfqpoint{5.982574in}{2.035399in}}{\pgfqpoint{5.993173in}{2.039789in}}{\pgfqpoint{6.000987in}{2.047603in}}%
\pgfpathcurveto{\pgfqpoint{6.008801in}{2.055416in}}{\pgfqpoint{6.013191in}{2.066015in}}{\pgfqpoint{6.013191in}{2.077066in}}%
\pgfpathcurveto{\pgfqpoint{6.013191in}{2.088116in}}{\pgfqpoint{6.008801in}{2.098715in}}{\pgfqpoint{6.000987in}{2.106528in}}%
\pgfpathcurveto{\pgfqpoint{5.993173in}{2.114342in}}{\pgfqpoint{5.982574in}{2.118732in}}{\pgfqpoint{5.971524in}{2.118732in}}%
\pgfpathcurveto{\pgfqpoint{5.960474in}{2.118732in}}{\pgfqpoint{5.949875in}{2.114342in}}{\pgfqpoint{5.942062in}{2.106528in}}%
\pgfpathcurveto{\pgfqpoint{5.934248in}{2.098715in}}{\pgfqpoint{5.929858in}{2.088116in}}{\pgfqpoint{5.929858in}{2.077066in}}%
\pgfpathcurveto{\pgfqpoint{5.929858in}{2.066015in}}{\pgfqpoint{5.934248in}{2.055416in}}{\pgfqpoint{5.942062in}{2.047603in}}%
\pgfpathcurveto{\pgfqpoint{5.949875in}{2.039789in}}{\pgfqpoint{5.960474in}{2.035399in}}{\pgfqpoint{5.971524in}{2.035399in}}%
\pgfpathclose%
\pgfusepath{stroke,fill}%
\end{pgfscope}%
\begin{pgfscope}%
\pgfpathrectangle{\pgfqpoint{0.481978in}{0.331635in}}{\pgfqpoint{9.300000in}{7.700000in}}%
\pgfusepath{clip}%
\pgfsetbuttcap%
\pgfsetroundjoin%
\definecolor{currentfill}{rgb}{0.631373,0.788235,0.956863}%
\pgfsetfillcolor{currentfill}%
\pgfsetlinewidth{0.481800pt}%
\definecolor{currentstroke}{rgb}{1.000000,1.000000,1.000000}%
\pgfsetstrokecolor{currentstroke}%
\pgfsetdash{}{0pt}%
\pgfpathmoveto{\pgfqpoint{3.002811in}{6.449802in}}%
\pgfpathcurveto{\pgfqpoint{3.013861in}{6.449802in}}{\pgfqpoint{3.024460in}{6.454193in}}{\pgfqpoint{3.032274in}{6.462006in}}%
\pgfpathcurveto{\pgfqpoint{3.040087in}{6.469820in}}{\pgfqpoint{3.044478in}{6.480419in}}{\pgfqpoint{3.044478in}{6.491469in}}%
\pgfpathcurveto{\pgfqpoint{3.044478in}{6.502519in}}{\pgfqpoint{3.040087in}{6.513118in}}{\pgfqpoint{3.032274in}{6.520932in}}%
\pgfpathcurveto{\pgfqpoint{3.024460in}{6.528746in}}{\pgfqpoint{3.013861in}{6.533136in}}{\pgfqpoint{3.002811in}{6.533136in}}%
\pgfpathcurveto{\pgfqpoint{2.991761in}{6.533136in}}{\pgfqpoint{2.981162in}{6.528746in}}{\pgfqpoint{2.973348in}{6.520932in}}%
\pgfpathcurveto{\pgfqpoint{2.965534in}{6.513118in}}{\pgfqpoint{2.961144in}{6.502519in}}{\pgfqpoint{2.961144in}{6.491469in}}%
\pgfpathcurveto{\pgfqpoint{2.961144in}{6.480419in}}{\pgfqpoint{2.965534in}{6.469820in}}{\pgfqpoint{2.973348in}{6.462006in}}%
\pgfpathcurveto{\pgfqpoint{2.981162in}{6.454193in}}{\pgfqpoint{2.991761in}{6.449802in}}{\pgfqpoint{3.002811in}{6.449802in}}%
\pgfpathclose%
\pgfusepath{stroke,fill}%
\end{pgfscope}%
\begin{pgfscope}%
\pgfpathrectangle{\pgfqpoint{0.481978in}{0.331635in}}{\pgfqpoint{9.300000in}{7.700000in}}%
\pgfusepath{clip}%
\pgfsetbuttcap%
\pgfsetroundjoin%
\definecolor{currentfill}{rgb}{0.631373,0.788235,0.956863}%
\pgfsetfillcolor{currentfill}%
\pgfsetlinewidth{0.481800pt}%
\definecolor{currentstroke}{rgb}{1.000000,1.000000,1.000000}%
\pgfsetstrokecolor{currentstroke}%
\pgfsetdash{}{0pt}%
\pgfpathmoveto{\pgfqpoint{5.207999in}{6.849862in}}%
\pgfpathcurveto{\pgfqpoint{5.219049in}{6.849862in}}{\pgfqpoint{5.229649in}{6.854253in}}{\pgfqpoint{5.237462in}{6.862066in}}%
\pgfpathcurveto{\pgfqpoint{5.245276in}{6.869880in}}{\pgfqpoint{5.249666in}{6.880479in}}{\pgfqpoint{5.249666in}{6.891529in}}%
\pgfpathcurveto{\pgfqpoint{5.249666in}{6.902579in}}{\pgfqpoint{5.245276in}{6.913178in}}{\pgfqpoint{5.237462in}{6.920992in}}%
\pgfpathcurveto{\pgfqpoint{5.229649in}{6.928805in}}{\pgfqpoint{5.219049in}{6.933196in}}{\pgfqpoint{5.207999in}{6.933196in}}%
\pgfpathcurveto{\pgfqpoint{5.196949in}{6.933196in}}{\pgfqpoint{5.186350in}{6.928805in}}{\pgfqpoint{5.178537in}{6.920992in}}%
\pgfpathcurveto{\pgfqpoint{5.170723in}{6.913178in}}{\pgfqpoint{5.166333in}{6.902579in}}{\pgfqpoint{5.166333in}{6.891529in}}%
\pgfpathcurveto{\pgfqpoint{5.166333in}{6.880479in}}{\pgfqpoint{5.170723in}{6.869880in}}{\pgfqpoint{5.178537in}{6.862066in}}%
\pgfpathcurveto{\pgfqpoint{5.186350in}{6.854253in}}{\pgfqpoint{5.196949in}{6.849862in}}{\pgfqpoint{5.207999in}{6.849862in}}%
\pgfpathclose%
\pgfusepath{stroke,fill}%
\end{pgfscope}%
\begin{pgfscope}%
\pgfpathrectangle{\pgfqpoint{0.481978in}{0.331635in}}{\pgfqpoint{9.300000in}{7.700000in}}%
\pgfusepath{clip}%
\pgfsetbuttcap%
\pgfsetroundjoin%
\definecolor{currentfill}{rgb}{0.631373,0.788235,0.956863}%
\pgfsetfillcolor{currentfill}%
\pgfsetlinewidth{0.481800pt}%
\definecolor{currentstroke}{rgb}{1.000000,1.000000,1.000000}%
\pgfsetstrokecolor{currentstroke}%
\pgfsetdash{}{0pt}%
\pgfpathmoveto{\pgfqpoint{4.854525in}{3.573118in}}%
\pgfpathcurveto{\pgfqpoint{4.865575in}{3.573118in}}{\pgfqpoint{4.876174in}{3.577508in}}{\pgfqpoint{4.883988in}{3.585322in}}%
\pgfpathcurveto{\pgfqpoint{4.891801in}{3.593135in}}{\pgfqpoint{4.896192in}{3.603734in}}{\pgfqpoint{4.896192in}{3.614784in}}%
\pgfpathcurveto{\pgfqpoint{4.896192in}{3.625834in}}{\pgfqpoint{4.891801in}{3.636433in}}{\pgfqpoint{4.883988in}{3.644247in}}%
\pgfpathcurveto{\pgfqpoint{4.876174in}{3.652061in}}{\pgfqpoint{4.865575in}{3.656451in}}{\pgfqpoint{4.854525in}{3.656451in}}%
\pgfpathcurveto{\pgfqpoint{4.843475in}{3.656451in}}{\pgfqpoint{4.832876in}{3.652061in}}{\pgfqpoint{4.825062in}{3.644247in}}%
\pgfpathcurveto{\pgfqpoint{4.817249in}{3.636433in}}{\pgfqpoint{4.812858in}{3.625834in}}{\pgfqpoint{4.812858in}{3.614784in}}%
\pgfpathcurveto{\pgfqpoint{4.812858in}{3.603734in}}{\pgfqpoint{4.817249in}{3.593135in}}{\pgfqpoint{4.825062in}{3.585322in}}%
\pgfpathcurveto{\pgfqpoint{4.832876in}{3.577508in}}{\pgfqpoint{4.843475in}{3.573118in}}{\pgfqpoint{4.854525in}{3.573118in}}%
\pgfpathclose%
\pgfusepath{stroke,fill}%
\end{pgfscope}%
\begin{pgfscope}%
\pgfpathrectangle{\pgfqpoint{0.481978in}{0.331635in}}{\pgfqpoint{9.300000in}{7.700000in}}%
\pgfusepath{clip}%
\pgfsetbuttcap%
\pgfsetroundjoin%
\definecolor{currentfill}{rgb}{0.631373,0.788235,0.956863}%
\pgfsetfillcolor{currentfill}%
\pgfsetlinewidth{0.481800pt}%
\definecolor{currentstroke}{rgb}{1.000000,1.000000,1.000000}%
\pgfsetstrokecolor{currentstroke}%
\pgfsetdash{}{0pt}%
\pgfpathmoveto{\pgfqpoint{3.734095in}{1.462941in}}%
\pgfpathcurveto{\pgfqpoint{3.745145in}{1.462941in}}{\pgfqpoint{3.755744in}{1.467331in}}{\pgfqpoint{3.763558in}{1.475145in}}%
\pgfpathcurveto{\pgfqpoint{3.771371in}{1.482959in}}{\pgfqpoint{3.775761in}{1.493558in}}{\pgfqpoint{3.775761in}{1.504608in}}%
\pgfpathcurveto{\pgfqpoint{3.775761in}{1.515658in}}{\pgfqpoint{3.771371in}{1.526257in}}{\pgfqpoint{3.763558in}{1.534071in}}%
\pgfpathcurveto{\pgfqpoint{3.755744in}{1.541884in}}{\pgfqpoint{3.745145in}{1.546274in}}{\pgfqpoint{3.734095in}{1.546274in}}%
\pgfpathcurveto{\pgfqpoint{3.723045in}{1.546274in}}{\pgfqpoint{3.712446in}{1.541884in}}{\pgfqpoint{3.704632in}{1.534071in}}%
\pgfpathcurveto{\pgfqpoint{3.696818in}{1.526257in}}{\pgfqpoint{3.692428in}{1.515658in}}{\pgfqpoint{3.692428in}{1.504608in}}%
\pgfpathcurveto{\pgfqpoint{3.692428in}{1.493558in}}{\pgfqpoint{3.696818in}{1.482959in}}{\pgfqpoint{3.704632in}{1.475145in}}%
\pgfpathcurveto{\pgfqpoint{3.712446in}{1.467331in}}{\pgfqpoint{3.723045in}{1.462941in}}{\pgfqpoint{3.734095in}{1.462941in}}%
\pgfpathclose%
\pgfusepath{stroke,fill}%
\end{pgfscope}%
\begin{pgfscope}%
\pgfpathrectangle{\pgfqpoint{0.481978in}{0.331635in}}{\pgfqpoint{9.300000in}{7.700000in}}%
\pgfusepath{clip}%
\pgfsetbuttcap%
\pgfsetroundjoin%
\definecolor{currentfill}{rgb}{0.631373,0.788235,0.956863}%
\pgfsetfillcolor{currentfill}%
\pgfsetlinewidth{0.481800pt}%
\definecolor{currentstroke}{rgb}{1.000000,1.000000,1.000000}%
\pgfsetstrokecolor{currentstroke}%
\pgfsetdash{}{0pt}%
\pgfpathmoveto{\pgfqpoint{3.327053in}{1.390346in}}%
\pgfpathcurveto{\pgfqpoint{3.338103in}{1.390346in}}{\pgfqpoint{3.348702in}{1.394736in}}{\pgfqpoint{3.356516in}{1.402550in}}%
\pgfpathcurveto{\pgfqpoint{3.364329in}{1.410363in}}{\pgfqpoint{3.368720in}{1.420962in}}{\pgfqpoint{3.368720in}{1.432012in}}%
\pgfpathcurveto{\pgfqpoint{3.368720in}{1.443062in}}{\pgfqpoint{3.364329in}{1.453661in}}{\pgfqpoint{3.356516in}{1.461475in}}%
\pgfpathcurveto{\pgfqpoint{3.348702in}{1.469289in}}{\pgfqpoint{3.338103in}{1.473679in}}{\pgfqpoint{3.327053in}{1.473679in}}%
\pgfpathcurveto{\pgfqpoint{3.316003in}{1.473679in}}{\pgfqpoint{3.305404in}{1.469289in}}{\pgfqpoint{3.297590in}{1.461475in}}%
\pgfpathcurveto{\pgfqpoint{3.289777in}{1.453661in}}{\pgfqpoint{3.285386in}{1.443062in}}{\pgfqpoint{3.285386in}{1.432012in}}%
\pgfpathcurveto{\pgfqpoint{3.285386in}{1.420962in}}{\pgfqpoint{3.289777in}{1.410363in}}{\pgfqpoint{3.297590in}{1.402550in}}%
\pgfpathcurveto{\pgfqpoint{3.305404in}{1.394736in}}{\pgfqpoint{3.316003in}{1.390346in}}{\pgfqpoint{3.327053in}{1.390346in}}%
\pgfpathclose%
\pgfusepath{stroke,fill}%
\end{pgfscope}%
\begin{pgfscope}%
\pgfpathrectangle{\pgfqpoint{0.481978in}{0.331635in}}{\pgfqpoint{9.300000in}{7.700000in}}%
\pgfusepath{clip}%
\pgfsetbuttcap%
\pgfsetroundjoin%
\definecolor{currentfill}{rgb}{0.631373,0.788235,0.956863}%
\pgfsetfillcolor{currentfill}%
\pgfsetlinewidth{0.481800pt}%
\definecolor{currentstroke}{rgb}{1.000000,1.000000,1.000000}%
\pgfsetstrokecolor{currentstroke}%
\pgfsetdash{}{0pt}%
\pgfpathmoveto{\pgfqpoint{5.992810in}{4.018090in}}%
\pgfpathcurveto{\pgfqpoint{6.003860in}{4.018090in}}{\pgfqpoint{6.014459in}{4.022480in}}{\pgfqpoint{6.022273in}{4.030294in}}%
\pgfpathcurveto{\pgfqpoint{6.030087in}{4.038108in}}{\pgfqpoint{6.034477in}{4.048707in}}{\pgfqpoint{6.034477in}{4.059757in}}%
\pgfpathcurveto{\pgfqpoint{6.034477in}{4.070807in}}{\pgfqpoint{6.030087in}{4.081406in}}{\pgfqpoint{6.022273in}{4.089220in}}%
\pgfpathcurveto{\pgfqpoint{6.014459in}{4.097033in}}{\pgfqpoint{6.003860in}{4.101423in}}{\pgfqpoint{5.992810in}{4.101423in}}%
\pgfpathcurveto{\pgfqpoint{5.981760in}{4.101423in}}{\pgfqpoint{5.971161in}{4.097033in}}{\pgfqpoint{5.963347in}{4.089220in}}%
\pgfpathcurveto{\pgfqpoint{5.955534in}{4.081406in}}{\pgfqpoint{5.951144in}{4.070807in}}{\pgfqpoint{5.951144in}{4.059757in}}%
\pgfpathcurveto{\pgfqpoint{5.951144in}{4.048707in}}{\pgfqpoint{5.955534in}{4.038108in}}{\pgfqpoint{5.963347in}{4.030294in}}%
\pgfpathcurveto{\pgfqpoint{5.971161in}{4.022480in}}{\pgfqpoint{5.981760in}{4.018090in}}{\pgfqpoint{5.992810in}{4.018090in}}%
\pgfpathclose%
\pgfusepath{stroke,fill}%
\end{pgfscope}%
\begin{pgfscope}%
\pgfpathrectangle{\pgfqpoint{0.481978in}{0.331635in}}{\pgfqpoint{9.300000in}{7.700000in}}%
\pgfusepath{clip}%
\pgfsetbuttcap%
\pgfsetroundjoin%
\definecolor{currentfill}{rgb}{0.631373,0.788235,0.956863}%
\pgfsetfillcolor{currentfill}%
\pgfsetlinewidth{0.481800pt}%
\definecolor{currentstroke}{rgb}{1.000000,1.000000,1.000000}%
\pgfsetstrokecolor{currentstroke}%
\pgfsetdash{}{0pt}%
\pgfpathmoveto{\pgfqpoint{5.648838in}{3.512473in}}%
\pgfpathcurveto{\pgfqpoint{5.659888in}{3.512473in}}{\pgfqpoint{5.670487in}{3.516864in}}{\pgfqpoint{5.678301in}{3.524677in}}%
\pgfpathcurveto{\pgfqpoint{5.686115in}{3.532491in}}{\pgfqpoint{5.690505in}{3.543090in}}{\pgfqpoint{5.690505in}{3.554140in}}%
\pgfpathcurveto{\pgfqpoint{5.690505in}{3.565190in}}{\pgfqpoint{5.686115in}{3.575789in}}{\pgfqpoint{5.678301in}{3.583603in}}%
\pgfpathcurveto{\pgfqpoint{5.670487in}{3.591416in}}{\pgfqpoint{5.659888in}{3.595807in}}{\pgfqpoint{5.648838in}{3.595807in}}%
\pgfpathcurveto{\pgfqpoint{5.637788in}{3.595807in}}{\pgfqpoint{5.627189in}{3.591416in}}{\pgfqpoint{5.619376in}{3.583603in}}%
\pgfpathcurveto{\pgfqpoint{5.611562in}{3.575789in}}{\pgfqpoint{5.607172in}{3.565190in}}{\pgfqpoint{5.607172in}{3.554140in}}%
\pgfpathcurveto{\pgfqpoint{5.607172in}{3.543090in}}{\pgfqpoint{5.611562in}{3.532491in}}{\pgfqpoint{5.619376in}{3.524677in}}%
\pgfpathcurveto{\pgfqpoint{5.627189in}{3.516864in}}{\pgfqpoint{5.637788in}{3.512473in}}{\pgfqpoint{5.648838in}{3.512473in}}%
\pgfpathclose%
\pgfusepath{stroke,fill}%
\end{pgfscope}%
\begin{pgfscope}%
\pgfpathrectangle{\pgfqpoint{0.481978in}{0.331635in}}{\pgfqpoint{9.300000in}{7.700000in}}%
\pgfusepath{clip}%
\pgfsetbuttcap%
\pgfsetroundjoin%
\definecolor{currentfill}{rgb}{0.631373,0.788235,0.956863}%
\pgfsetfillcolor{currentfill}%
\pgfsetlinewidth{0.481800pt}%
\definecolor{currentstroke}{rgb}{1.000000,1.000000,1.000000}%
\pgfsetstrokecolor{currentstroke}%
\pgfsetdash{}{0pt}%
\pgfpathmoveto{\pgfqpoint{6.635439in}{5.352176in}}%
\pgfpathcurveto{\pgfqpoint{6.646489in}{5.352176in}}{\pgfqpoint{6.657088in}{5.356566in}}{\pgfqpoint{6.664902in}{5.364380in}}%
\pgfpathcurveto{\pgfqpoint{6.672715in}{5.372193in}}{\pgfqpoint{6.677105in}{5.382792in}}{\pgfqpoint{6.677105in}{5.393842in}}%
\pgfpathcurveto{\pgfqpoint{6.677105in}{5.404892in}}{\pgfqpoint{6.672715in}{5.415491in}}{\pgfqpoint{6.664902in}{5.423305in}}%
\pgfpathcurveto{\pgfqpoint{6.657088in}{5.431119in}}{\pgfqpoint{6.646489in}{5.435509in}}{\pgfqpoint{6.635439in}{5.435509in}}%
\pgfpathcurveto{\pgfqpoint{6.624389in}{5.435509in}}{\pgfqpoint{6.613790in}{5.431119in}}{\pgfqpoint{6.605976in}{5.423305in}}%
\pgfpathcurveto{\pgfqpoint{6.598162in}{5.415491in}}{\pgfqpoint{6.593772in}{5.404892in}}{\pgfqpoint{6.593772in}{5.393842in}}%
\pgfpathcurveto{\pgfqpoint{6.593772in}{5.382792in}}{\pgfqpoint{6.598162in}{5.372193in}}{\pgfqpoint{6.605976in}{5.364380in}}%
\pgfpathcurveto{\pgfqpoint{6.613790in}{5.356566in}}{\pgfqpoint{6.624389in}{5.352176in}}{\pgfqpoint{6.635439in}{5.352176in}}%
\pgfpathclose%
\pgfusepath{stroke,fill}%
\end{pgfscope}%
\begin{pgfscope}%
\pgfpathrectangle{\pgfqpoint{0.481978in}{0.331635in}}{\pgfqpoint{9.300000in}{7.700000in}}%
\pgfusepath{clip}%
\pgfsetbuttcap%
\pgfsetroundjoin%
\definecolor{currentfill}{rgb}{0.631373,0.788235,0.956863}%
\pgfsetfillcolor{currentfill}%
\pgfsetlinewidth{0.481800pt}%
\definecolor{currentstroke}{rgb}{1.000000,1.000000,1.000000}%
\pgfsetstrokecolor{currentstroke}%
\pgfsetdash{}{0pt}%
\pgfpathmoveto{\pgfqpoint{6.399047in}{2.029599in}}%
\pgfpathcurveto{\pgfqpoint{6.410097in}{2.029599in}}{\pgfqpoint{6.420696in}{2.033990in}}{\pgfqpoint{6.428510in}{2.041803in}}%
\pgfpathcurveto{\pgfqpoint{6.436323in}{2.049617in}}{\pgfqpoint{6.440713in}{2.060216in}}{\pgfqpoint{6.440713in}{2.071266in}}%
\pgfpathcurveto{\pgfqpoint{6.440713in}{2.082316in}}{\pgfqpoint{6.436323in}{2.092915in}}{\pgfqpoint{6.428510in}{2.100729in}}%
\pgfpathcurveto{\pgfqpoint{6.420696in}{2.108542in}}{\pgfqpoint{6.410097in}{2.112933in}}{\pgfqpoint{6.399047in}{2.112933in}}%
\pgfpathcurveto{\pgfqpoint{6.387997in}{2.112933in}}{\pgfqpoint{6.377398in}{2.108542in}}{\pgfqpoint{6.369584in}{2.100729in}}%
\pgfpathcurveto{\pgfqpoint{6.361770in}{2.092915in}}{\pgfqpoint{6.357380in}{2.082316in}}{\pgfqpoint{6.357380in}{2.071266in}}%
\pgfpathcurveto{\pgfqpoint{6.357380in}{2.060216in}}{\pgfqpoint{6.361770in}{2.049617in}}{\pgfqpoint{6.369584in}{2.041803in}}%
\pgfpathcurveto{\pgfqpoint{6.377398in}{2.033990in}}{\pgfqpoint{6.387997in}{2.029599in}}{\pgfqpoint{6.399047in}{2.029599in}}%
\pgfpathclose%
\pgfusepath{stroke,fill}%
\end{pgfscope}%
\begin{pgfscope}%
\pgfpathrectangle{\pgfqpoint{0.481978in}{0.331635in}}{\pgfqpoint{9.300000in}{7.700000in}}%
\pgfusepath{clip}%
\pgfsetbuttcap%
\pgfsetroundjoin%
\definecolor{currentfill}{rgb}{0.631373,0.788235,0.956863}%
\pgfsetfillcolor{currentfill}%
\pgfsetlinewidth{0.481800pt}%
\definecolor{currentstroke}{rgb}{1.000000,1.000000,1.000000}%
\pgfsetstrokecolor{currentstroke}%
\pgfsetdash{}{0pt}%
\pgfpathmoveto{\pgfqpoint{2.913660in}{6.822845in}}%
\pgfpathcurveto{\pgfqpoint{2.924710in}{6.822845in}}{\pgfqpoint{2.935309in}{6.827235in}}{\pgfqpoint{2.943123in}{6.835049in}}%
\pgfpathcurveto{\pgfqpoint{2.950937in}{6.842863in}}{\pgfqpoint{2.955327in}{6.853462in}}{\pgfqpoint{2.955327in}{6.864512in}}%
\pgfpathcurveto{\pgfqpoint{2.955327in}{6.875562in}}{\pgfqpoint{2.950937in}{6.886161in}}{\pgfqpoint{2.943123in}{6.893975in}}%
\pgfpathcurveto{\pgfqpoint{2.935309in}{6.901788in}}{\pgfqpoint{2.924710in}{6.906179in}}{\pgfqpoint{2.913660in}{6.906179in}}%
\pgfpathcurveto{\pgfqpoint{2.902610in}{6.906179in}}{\pgfqpoint{2.892011in}{6.901788in}}{\pgfqpoint{2.884197in}{6.893975in}}%
\pgfpathcurveto{\pgfqpoint{2.876384in}{6.886161in}}{\pgfqpoint{2.871994in}{6.875562in}}{\pgfqpoint{2.871994in}{6.864512in}}%
\pgfpathcurveto{\pgfqpoint{2.871994in}{6.853462in}}{\pgfqpoint{2.876384in}{6.842863in}}{\pgfqpoint{2.884197in}{6.835049in}}%
\pgfpathcurveto{\pgfqpoint{2.892011in}{6.827235in}}{\pgfqpoint{2.902610in}{6.822845in}}{\pgfqpoint{2.913660in}{6.822845in}}%
\pgfpathclose%
\pgfusepath{stroke,fill}%
\end{pgfscope}%
\begin{pgfscope}%
\pgfpathrectangle{\pgfqpoint{0.481978in}{0.331635in}}{\pgfqpoint{9.300000in}{7.700000in}}%
\pgfusepath{clip}%
\pgfsetbuttcap%
\pgfsetroundjoin%
\definecolor{currentfill}{rgb}{0.631373,0.788235,0.956863}%
\pgfsetfillcolor{currentfill}%
\pgfsetlinewidth{0.481800pt}%
\definecolor{currentstroke}{rgb}{1.000000,1.000000,1.000000}%
\pgfsetstrokecolor{currentstroke}%
\pgfsetdash{}{0pt}%
\pgfpathmoveto{\pgfqpoint{4.618112in}{4.334558in}}%
\pgfpathcurveto{\pgfqpoint{4.629162in}{4.334558in}}{\pgfqpoint{4.639761in}{4.338948in}}{\pgfqpoint{4.647575in}{4.346762in}}%
\pgfpathcurveto{\pgfqpoint{4.655388in}{4.354575in}}{\pgfqpoint{4.659778in}{4.365174in}}{\pgfqpoint{4.659778in}{4.376225in}}%
\pgfpathcurveto{\pgfqpoint{4.659778in}{4.387275in}}{\pgfqpoint{4.655388in}{4.397874in}}{\pgfqpoint{4.647575in}{4.405687in}}%
\pgfpathcurveto{\pgfqpoint{4.639761in}{4.413501in}}{\pgfqpoint{4.629162in}{4.417891in}}{\pgfqpoint{4.618112in}{4.417891in}}%
\pgfpathcurveto{\pgfqpoint{4.607062in}{4.417891in}}{\pgfqpoint{4.596463in}{4.413501in}}{\pgfqpoint{4.588649in}{4.405687in}}%
\pgfpathcurveto{\pgfqpoint{4.580835in}{4.397874in}}{\pgfqpoint{4.576445in}{4.387275in}}{\pgfqpoint{4.576445in}{4.376225in}}%
\pgfpathcurveto{\pgfqpoint{4.576445in}{4.365174in}}{\pgfqpoint{4.580835in}{4.354575in}}{\pgfqpoint{4.588649in}{4.346762in}}%
\pgfpathcurveto{\pgfqpoint{4.596463in}{4.338948in}}{\pgfqpoint{4.607062in}{4.334558in}}{\pgfqpoint{4.618112in}{4.334558in}}%
\pgfpathclose%
\pgfusepath{stroke,fill}%
\end{pgfscope}%
\begin{pgfscope}%
\pgfpathrectangle{\pgfqpoint{0.481978in}{0.331635in}}{\pgfqpoint{9.300000in}{7.700000in}}%
\pgfusepath{clip}%
\pgfsetbuttcap%
\pgfsetroundjoin%
\definecolor{currentfill}{rgb}{0.631373,0.788235,0.956863}%
\pgfsetfillcolor{currentfill}%
\pgfsetlinewidth{0.481800pt}%
\definecolor{currentstroke}{rgb}{1.000000,1.000000,1.000000}%
\pgfsetstrokecolor{currentstroke}%
\pgfsetdash{}{0pt}%
\pgfpathmoveto{\pgfqpoint{7.244029in}{2.429882in}}%
\pgfpathcurveto{\pgfqpoint{7.255079in}{2.429882in}}{\pgfqpoint{7.265678in}{2.434272in}}{\pgfqpoint{7.273491in}{2.442085in}}%
\pgfpathcurveto{\pgfqpoint{7.281305in}{2.449899in}}{\pgfqpoint{7.285695in}{2.460498in}}{\pgfqpoint{7.285695in}{2.471548in}}%
\pgfpathcurveto{\pgfqpoint{7.285695in}{2.482598in}}{\pgfqpoint{7.281305in}{2.493197in}}{\pgfqpoint{7.273491in}{2.501011in}}%
\pgfpathcurveto{\pgfqpoint{7.265678in}{2.508825in}}{\pgfqpoint{7.255079in}{2.513215in}}{\pgfqpoint{7.244029in}{2.513215in}}%
\pgfpathcurveto{\pgfqpoint{7.232978in}{2.513215in}}{\pgfqpoint{7.222379in}{2.508825in}}{\pgfqpoint{7.214566in}{2.501011in}}%
\pgfpathcurveto{\pgfqpoint{7.206752in}{2.493197in}}{\pgfqpoint{7.202362in}{2.482598in}}{\pgfqpoint{7.202362in}{2.471548in}}%
\pgfpathcurveto{\pgfqpoint{7.202362in}{2.460498in}}{\pgfqpoint{7.206752in}{2.449899in}}{\pgfqpoint{7.214566in}{2.442085in}}%
\pgfpathcurveto{\pgfqpoint{7.222379in}{2.434272in}}{\pgfqpoint{7.232978in}{2.429882in}}{\pgfqpoint{7.244029in}{2.429882in}}%
\pgfpathclose%
\pgfusepath{stroke,fill}%
\end{pgfscope}%
\begin{pgfscope}%
\pgfpathrectangle{\pgfqpoint{0.481978in}{0.331635in}}{\pgfqpoint{9.300000in}{7.700000in}}%
\pgfusepath{clip}%
\pgfsetbuttcap%
\pgfsetroundjoin%
\definecolor{currentfill}{rgb}{0.631373,0.788235,0.956863}%
\pgfsetfillcolor{currentfill}%
\pgfsetlinewidth{0.481800pt}%
\definecolor{currentstroke}{rgb}{1.000000,1.000000,1.000000}%
\pgfsetstrokecolor{currentstroke}%
\pgfsetdash{}{0pt}%
\pgfpathmoveto{\pgfqpoint{6.714656in}{1.525447in}}%
\pgfpathcurveto{\pgfqpoint{6.725706in}{1.525447in}}{\pgfqpoint{6.736305in}{1.529837in}}{\pgfqpoint{6.744119in}{1.537651in}}%
\pgfpathcurveto{\pgfqpoint{6.751932in}{1.545465in}}{\pgfqpoint{6.756322in}{1.556064in}}{\pgfqpoint{6.756322in}{1.567114in}}%
\pgfpathcurveto{\pgfqpoint{6.756322in}{1.578164in}}{\pgfqpoint{6.751932in}{1.588763in}}{\pgfqpoint{6.744119in}{1.596577in}}%
\pgfpathcurveto{\pgfqpoint{6.736305in}{1.604390in}}{\pgfqpoint{6.725706in}{1.608780in}}{\pgfqpoint{6.714656in}{1.608780in}}%
\pgfpathcurveto{\pgfqpoint{6.703606in}{1.608780in}}{\pgfqpoint{6.693007in}{1.604390in}}{\pgfqpoint{6.685193in}{1.596577in}}%
\pgfpathcurveto{\pgfqpoint{6.677379in}{1.588763in}}{\pgfqpoint{6.672989in}{1.578164in}}{\pgfqpoint{6.672989in}{1.567114in}}%
\pgfpathcurveto{\pgfqpoint{6.672989in}{1.556064in}}{\pgfqpoint{6.677379in}{1.545465in}}{\pgfqpoint{6.685193in}{1.537651in}}%
\pgfpathcurveto{\pgfqpoint{6.693007in}{1.529837in}}{\pgfqpoint{6.703606in}{1.525447in}}{\pgfqpoint{6.714656in}{1.525447in}}%
\pgfpathclose%
\pgfusepath{stroke,fill}%
\end{pgfscope}%
\begin{pgfscope}%
\pgfpathrectangle{\pgfqpoint{0.481978in}{0.331635in}}{\pgfqpoint{9.300000in}{7.700000in}}%
\pgfusepath{clip}%
\pgfsetbuttcap%
\pgfsetroundjoin%
\definecolor{currentfill}{rgb}{0.631373,0.788235,0.956863}%
\pgfsetfillcolor{currentfill}%
\pgfsetlinewidth{0.481800pt}%
\definecolor{currentstroke}{rgb}{1.000000,1.000000,1.000000}%
\pgfsetstrokecolor{currentstroke}%
\pgfsetdash{}{0pt}%
\pgfpathmoveto{\pgfqpoint{3.591337in}{1.267658in}}%
\pgfpathcurveto{\pgfqpoint{3.602388in}{1.267658in}}{\pgfqpoint{3.612987in}{1.272048in}}{\pgfqpoint{3.620800in}{1.279862in}}%
\pgfpathcurveto{\pgfqpoint{3.628614in}{1.287675in}}{\pgfqpoint{3.633004in}{1.298274in}}{\pgfqpoint{3.633004in}{1.309324in}}%
\pgfpathcurveto{\pgfqpoint{3.633004in}{1.320375in}}{\pgfqpoint{3.628614in}{1.330974in}}{\pgfqpoint{3.620800in}{1.338787in}}%
\pgfpathcurveto{\pgfqpoint{3.612987in}{1.346601in}}{\pgfqpoint{3.602388in}{1.350991in}}{\pgfqpoint{3.591337in}{1.350991in}}%
\pgfpathcurveto{\pgfqpoint{3.580287in}{1.350991in}}{\pgfqpoint{3.569688in}{1.346601in}}{\pgfqpoint{3.561875in}{1.338787in}}%
\pgfpathcurveto{\pgfqpoint{3.554061in}{1.330974in}}{\pgfqpoint{3.549671in}{1.320375in}}{\pgfqpoint{3.549671in}{1.309324in}}%
\pgfpathcurveto{\pgfqpoint{3.549671in}{1.298274in}}{\pgfqpoint{3.554061in}{1.287675in}}{\pgfqpoint{3.561875in}{1.279862in}}%
\pgfpathcurveto{\pgfqpoint{3.569688in}{1.272048in}}{\pgfqpoint{3.580287in}{1.267658in}}{\pgfqpoint{3.591337in}{1.267658in}}%
\pgfpathclose%
\pgfusepath{stroke,fill}%
\end{pgfscope}%
\begin{pgfscope}%
\pgfpathrectangle{\pgfqpoint{0.481978in}{0.331635in}}{\pgfqpoint{9.300000in}{7.700000in}}%
\pgfusepath{clip}%
\pgfsetbuttcap%
\pgfsetroundjoin%
\definecolor{currentfill}{rgb}{0.631373,0.788235,0.956863}%
\pgfsetfillcolor{currentfill}%
\pgfsetlinewidth{0.481800pt}%
\definecolor{currentstroke}{rgb}{1.000000,1.000000,1.000000}%
\pgfsetstrokecolor{currentstroke}%
\pgfsetdash{}{0pt}%
\pgfpathmoveto{\pgfqpoint{8.476836in}{5.231722in}}%
\pgfpathcurveto{\pgfqpoint{8.487886in}{5.231722in}}{\pgfqpoint{8.498485in}{5.236112in}}{\pgfqpoint{8.506299in}{5.243926in}}%
\pgfpathcurveto{\pgfqpoint{8.514112in}{5.251740in}}{\pgfqpoint{8.518503in}{5.262339in}}{\pgfqpoint{8.518503in}{5.273389in}}%
\pgfpathcurveto{\pgfqpoint{8.518503in}{5.284439in}}{\pgfqpoint{8.514112in}{5.295038in}}{\pgfqpoint{8.506299in}{5.302852in}}%
\pgfpathcurveto{\pgfqpoint{8.498485in}{5.310665in}}{\pgfqpoint{8.487886in}{5.315055in}}{\pgfqpoint{8.476836in}{5.315055in}}%
\pgfpathcurveto{\pgfqpoint{8.465786in}{5.315055in}}{\pgfqpoint{8.455187in}{5.310665in}}{\pgfqpoint{8.447373in}{5.302852in}}%
\pgfpathcurveto{\pgfqpoint{8.439559in}{5.295038in}}{\pgfqpoint{8.435169in}{5.284439in}}{\pgfqpoint{8.435169in}{5.273389in}}%
\pgfpathcurveto{\pgfqpoint{8.435169in}{5.262339in}}{\pgfqpoint{8.439559in}{5.251740in}}{\pgfqpoint{8.447373in}{5.243926in}}%
\pgfpathcurveto{\pgfqpoint{8.455187in}{5.236112in}}{\pgfqpoint{8.465786in}{5.231722in}}{\pgfqpoint{8.476836in}{5.231722in}}%
\pgfpathclose%
\pgfusepath{stroke,fill}%
\end{pgfscope}%
\begin{pgfscope}%
\pgfpathrectangle{\pgfqpoint{0.481978in}{0.331635in}}{\pgfqpoint{9.300000in}{7.700000in}}%
\pgfusepath{clip}%
\pgfsetbuttcap%
\pgfsetroundjoin%
\definecolor{currentfill}{rgb}{0.631373,0.788235,0.956863}%
\pgfsetfillcolor{currentfill}%
\pgfsetlinewidth{0.481800pt}%
\definecolor{currentstroke}{rgb}{1.000000,1.000000,1.000000}%
\pgfsetstrokecolor{currentstroke}%
\pgfsetdash{}{0pt}%
\pgfpathmoveto{\pgfqpoint{6.886735in}{1.411937in}}%
\pgfpathcurveto{\pgfqpoint{6.897785in}{1.411937in}}{\pgfqpoint{6.908385in}{1.416327in}}{\pgfqpoint{6.916198in}{1.424141in}}%
\pgfpathcurveto{\pgfqpoint{6.924012in}{1.431955in}}{\pgfqpoint{6.928402in}{1.442554in}}{\pgfqpoint{6.928402in}{1.453604in}}%
\pgfpathcurveto{\pgfqpoint{6.928402in}{1.464654in}}{\pgfqpoint{6.924012in}{1.475253in}}{\pgfqpoint{6.916198in}{1.483067in}}%
\pgfpathcurveto{\pgfqpoint{6.908385in}{1.490880in}}{\pgfqpoint{6.897785in}{1.495270in}}{\pgfqpoint{6.886735in}{1.495270in}}%
\pgfpathcurveto{\pgfqpoint{6.875685in}{1.495270in}}{\pgfqpoint{6.865086in}{1.490880in}}{\pgfqpoint{6.857273in}{1.483067in}}%
\pgfpathcurveto{\pgfqpoint{6.849459in}{1.475253in}}{\pgfqpoint{6.845069in}{1.464654in}}{\pgfqpoint{6.845069in}{1.453604in}}%
\pgfpathcurveto{\pgfqpoint{6.845069in}{1.442554in}}{\pgfqpoint{6.849459in}{1.431955in}}{\pgfqpoint{6.857273in}{1.424141in}}%
\pgfpathcurveto{\pgfqpoint{6.865086in}{1.416327in}}{\pgfqpoint{6.875685in}{1.411937in}}{\pgfqpoint{6.886735in}{1.411937in}}%
\pgfpathclose%
\pgfusepath{stroke,fill}%
\end{pgfscope}%
\begin{pgfscope}%
\pgfpathrectangle{\pgfqpoint{0.481978in}{0.331635in}}{\pgfqpoint{9.300000in}{7.700000in}}%
\pgfusepath{clip}%
\pgfsetbuttcap%
\pgfsetroundjoin%
\definecolor{currentfill}{rgb}{0.631373,0.788235,0.956863}%
\pgfsetfillcolor{currentfill}%
\pgfsetlinewidth{0.481800pt}%
\definecolor{currentstroke}{rgb}{1.000000,1.000000,1.000000}%
\pgfsetstrokecolor{currentstroke}%
\pgfsetdash{}{0pt}%
\pgfpathmoveto{\pgfqpoint{5.148908in}{1.630570in}}%
\pgfpathcurveto{\pgfqpoint{5.159959in}{1.630570in}}{\pgfqpoint{5.170558in}{1.634961in}}{\pgfqpoint{5.178371in}{1.642774in}}%
\pgfpathcurveto{\pgfqpoint{5.186185in}{1.650588in}}{\pgfqpoint{5.190575in}{1.661187in}}{\pgfqpoint{5.190575in}{1.672237in}}%
\pgfpathcurveto{\pgfqpoint{5.190575in}{1.683287in}}{\pgfqpoint{5.186185in}{1.693886in}}{\pgfqpoint{5.178371in}{1.701700in}}%
\pgfpathcurveto{\pgfqpoint{5.170558in}{1.709514in}}{\pgfqpoint{5.159959in}{1.713904in}}{\pgfqpoint{5.148908in}{1.713904in}}%
\pgfpathcurveto{\pgfqpoint{5.137858in}{1.713904in}}{\pgfqpoint{5.127259in}{1.709514in}}{\pgfqpoint{5.119446in}{1.701700in}}%
\pgfpathcurveto{\pgfqpoint{5.111632in}{1.693886in}}{\pgfqpoint{5.107242in}{1.683287in}}{\pgfqpoint{5.107242in}{1.672237in}}%
\pgfpathcurveto{\pgfqpoint{5.107242in}{1.661187in}}{\pgfqpoint{5.111632in}{1.650588in}}{\pgfqpoint{5.119446in}{1.642774in}}%
\pgfpathcurveto{\pgfqpoint{5.127259in}{1.634961in}}{\pgfqpoint{5.137858in}{1.630570in}}{\pgfqpoint{5.148908in}{1.630570in}}%
\pgfpathclose%
\pgfusepath{stroke,fill}%
\end{pgfscope}%
\begin{pgfscope}%
\pgfpathrectangle{\pgfqpoint{0.481978in}{0.331635in}}{\pgfqpoint{9.300000in}{7.700000in}}%
\pgfusepath{clip}%
\pgfsetbuttcap%
\pgfsetroundjoin%
\definecolor{currentfill}{rgb}{0.631373,0.788235,0.956863}%
\pgfsetfillcolor{currentfill}%
\pgfsetlinewidth{0.481800pt}%
\definecolor{currentstroke}{rgb}{1.000000,1.000000,1.000000}%
\pgfsetstrokecolor{currentstroke}%
\pgfsetdash{}{0pt}%
\pgfpathmoveto{\pgfqpoint{5.442394in}{6.812671in}}%
\pgfpathcurveto{\pgfqpoint{5.453444in}{6.812671in}}{\pgfqpoint{5.464043in}{6.817061in}}{\pgfqpoint{5.471856in}{6.824874in}}%
\pgfpathcurveto{\pgfqpoint{5.479670in}{6.832688in}}{\pgfqpoint{5.484060in}{6.843287in}}{\pgfqpoint{5.484060in}{6.854337in}}%
\pgfpathcurveto{\pgfqpoint{5.484060in}{6.865387in}}{\pgfqpoint{5.479670in}{6.875986in}}{\pgfqpoint{5.471856in}{6.883800in}}%
\pgfpathcurveto{\pgfqpoint{5.464043in}{6.891614in}}{\pgfqpoint{5.453444in}{6.896004in}}{\pgfqpoint{5.442394in}{6.896004in}}%
\pgfpathcurveto{\pgfqpoint{5.431343in}{6.896004in}}{\pgfqpoint{5.420744in}{6.891614in}}{\pgfqpoint{5.412931in}{6.883800in}}%
\pgfpathcurveto{\pgfqpoint{5.405117in}{6.875986in}}{\pgfqpoint{5.400727in}{6.865387in}}{\pgfqpoint{5.400727in}{6.854337in}}%
\pgfpathcurveto{\pgfqpoint{5.400727in}{6.843287in}}{\pgfqpoint{5.405117in}{6.832688in}}{\pgfqpoint{5.412931in}{6.824874in}}%
\pgfpathcurveto{\pgfqpoint{5.420744in}{6.817061in}}{\pgfqpoint{5.431343in}{6.812671in}}{\pgfqpoint{5.442394in}{6.812671in}}%
\pgfpathclose%
\pgfusepath{stroke,fill}%
\end{pgfscope}%
\begin{pgfscope}%
\pgfpathrectangle{\pgfqpoint{0.481978in}{0.331635in}}{\pgfqpoint{9.300000in}{7.700000in}}%
\pgfusepath{clip}%
\pgfsetbuttcap%
\pgfsetroundjoin%
\definecolor{currentfill}{rgb}{0.631373,0.788235,0.956863}%
\pgfsetfillcolor{currentfill}%
\pgfsetlinewidth{0.481800pt}%
\definecolor{currentstroke}{rgb}{1.000000,1.000000,1.000000}%
\pgfsetstrokecolor{currentstroke}%
\pgfsetdash{}{0pt}%
\pgfpathmoveto{\pgfqpoint{6.133252in}{3.334993in}}%
\pgfpathcurveto{\pgfqpoint{6.144302in}{3.334993in}}{\pgfqpoint{6.154901in}{3.339383in}}{\pgfqpoint{6.162714in}{3.347196in}}%
\pgfpathcurveto{\pgfqpoint{6.170528in}{3.355010in}}{\pgfqpoint{6.174918in}{3.365609in}}{\pgfqpoint{6.174918in}{3.376659in}}%
\pgfpathcurveto{\pgfqpoint{6.174918in}{3.387709in}}{\pgfqpoint{6.170528in}{3.398308in}}{\pgfqpoint{6.162714in}{3.406122in}}%
\pgfpathcurveto{\pgfqpoint{6.154901in}{3.413936in}}{\pgfqpoint{6.144302in}{3.418326in}}{\pgfqpoint{6.133252in}{3.418326in}}%
\pgfpathcurveto{\pgfqpoint{6.122201in}{3.418326in}}{\pgfqpoint{6.111602in}{3.413936in}}{\pgfqpoint{6.103789in}{3.406122in}}%
\pgfpathcurveto{\pgfqpoint{6.095975in}{3.398308in}}{\pgfqpoint{6.091585in}{3.387709in}}{\pgfqpoint{6.091585in}{3.376659in}}%
\pgfpathcurveto{\pgfqpoint{6.091585in}{3.365609in}}{\pgfqpoint{6.095975in}{3.355010in}}{\pgfqpoint{6.103789in}{3.347196in}}%
\pgfpathcurveto{\pgfqpoint{6.111602in}{3.339383in}}{\pgfqpoint{6.122201in}{3.334993in}}{\pgfqpoint{6.133252in}{3.334993in}}%
\pgfpathclose%
\pgfusepath{stroke,fill}%
\end{pgfscope}%
\begin{pgfscope}%
\pgfpathrectangle{\pgfqpoint{0.481978in}{0.331635in}}{\pgfqpoint{9.300000in}{7.700000in}}%
\pgfusepath{clip}%
\pgfsetbuttcap%
\pgfsetroundjoin%
\definecolor{currentfill}{rgb}{0.631373,0.788235,0.956863}%
\pgfsetfillcolor{currentfill}%
\pgfsetlinewidth{0.481800pt}%
\definecolor{currentstroke}{rgb}{1.000000,1.000000,1.000000}%
\pgfsetstrokecolor{currentstroke}%
\pgfsetdash{}{0pt}%
\pgfpathmoveto{\pgfqpoint{6.565040in}{5.025815in}}%
\pgfpathcurveto{\pgfqpoint{6.576090in}{5.025815in}}{\pgfqpoint{6.586689in}{5.030205in}}{\pgfqpoint{6.594503in}{5.038019in}}%
\pgfpathcurveto{\pgfqpoint{6.602316in}{5.045832in}}{\pgfqpoint{6.606707in}{5.056431in}}{\pgfqpoint{6.606707in}{5.067481in}}%
\pgfpathcurveto{\pgfqpoint{6.606707in}{5.078532in}}{\pgfqpoint{6.602316in}{5.089131in}}{\pgfqpoint{6.594503in}{5.096944in}}%
\pgfpathcurveto{\pgfqpoint{6.586689in}{5.104758in}}{\pgfqpoint{6.576090in}{5.109148in}}{\pgfqpoint{6.565040in}{5.109148in}}%
\pgfpathcurveto{\pgfqpoint{6.553990in}{5.109148in}}{\pgfqpoint{6.543391in}{5.104758in}}{\pgfqpoint{6.535577in}{5.096944in}}%
\pgfpathcurveto{\pgfqpoint{6.527763in}{5.089131in}}{\pgfqpoint{6.523373in}{5.078532in}}{\pgfqpoint{6.523373in}{5.067481in}}%
\pgfpathcurveto{\pgfqpoint{6.523373in}{5.056431in}}{\pgfqpoint{6.527763in}{5.045832in}}{\pgfqpoint{6.535577in}{5.038019in}}%
\pgfpathcurveto{\pgfqpoint{6.543391in}{5.030205in}}{\pgfqpoint{6.553990in}{5.025815in}}{\pgfqpoint{6.565040in}{5.025815in}}%
\pgfpathclose%
\pgfusepath{stroke,fill}%
\end{pgfscope}%
\begin{pgfscope}%
\pgfpathrectangle{\pgfqpoint{0.481978in}{0.331635in}}{\pgfqpoint{9.300000in}{7.700000in}}%
\pgfusepath{clip}%
\pgfsetbuttcap%
\pgfsetroundjoin%
\definecolor{currentfill}{rgb}{0.631373,0.788235,0.956863}%
\pgfsetfillcolor{currentfill}%
\pgfsetlinewidth{0.481800pt}%
\definecolor{currentstroke}{rgb}{1.000000,1.000000,1.000000}%
\pgfsetstrokecolor{currentstroke}%
\pgfsetdash{}{0pt}%
\pgfpathmoveto{\pgfqpoint{6.488247in}{4.397656in}}%
\pgfpathcurveto{\pgfqpoint{6.499297in}{4.397656in}}{\pgfqpoint{6.509896in}{4.402046in}}{\pgfqpoint{6.517710in}{4.409860in}}%
\pgfpathcurveto{\pgfqpoint{6.525523in}{4.417673in}}{\pgfqpoint{6.529914in}{4.428272in}}{\pgfqpoint{6.529914in}{4.439323in}}%
\pgfpathcurveto{\pgfqpoint{6.529914in}{4.450373in}}{\pgfqpoint{6.525523in}{4.460972in}}{\pgfqpoint{6.517710in}{4.468785in}}%
\pgfpathcurveto{\pgfqpoint{6.509896in}{4.476599in}}{\pgfqpoint{6.499297in}{4.480989in}}{\pgfqpoint{6.488247in}{4.480989in}}%
\pgfpathcurveto{\pgfqpoint{6.477197in}{4.480989in}}{\pgfqpoint{6.466598in}{4.476599in}}{\pgfqpoint{6.458784in}{4.468785in}}%
\pgfpathcurveto{\pgfqpoint{6.450970in}{4.460972in}}{\pgfqpoint{6.446580in}{4.450373in}}{\pgfqpoint{6.446580in}{4.439323in}}%
\pgfpathcurveto{\pgfqpoint{6.446580in}{4.428272in}}{\pgfqpoint{6.450970in}{4.417673in}}{\pgfqpoint{6.458784in}{4.409860in}}%
\pgfpathcurveto{\pgfqpoint{6.466598in}{4.402046in}}{\pgfqpoint{6.477197in}{4.397656in}}{\pgfqpoint{6.488247in}{4.397656in}}%
\pgfpathclose%
\pgfusepath{stroke,fill}%
\end{pgfscope}%
\begin{pgfscope}%
\pgfpathrectangle{\pgfqpoint{0.481978in}{0.331635in}}{\pgfqpoint{9.300000in}{7.700000in}}%
\pgfusepath{clip}%
\pgfsetbuttcap%
\pgfsetroundjoin%
\definecolor{currentfill}{rgb}{0.631373,0.788235,0.956863}%
\pgfsetfillcolor{currentfill}%
\pgfsetlinewidth{0.481800pt}%
\definecolor{currentstroke}{rgb}{1.000000,1.000000,1.000000}%
\pgfsetstrokecolor{currentstroke}%
\pgfsetdash{}{0pt}%
\pgfpathmoveto{\pgfqpoint{6.070621in}{5.004506in}}%
\pgfpathcurveto{\pgfqpoint{6.081671in}{5.004506in}}{\pgfqpoint{6.092270in}{5.008896in}}{\pgfqpoint{6.100084in}{5.016710in}}%
\pgfpathcurveto{\pgfqpoint{6.107897in}{5.024524in}}{\pgfqpoint{6.112288in}{5.035123in}}{\pgfqpoint{6.112288in}{5.046173in}}%
\pgfpathcurveto{\pgfqpoint{6.112288in}{5.057223in}}{\pgfqpoint{6.107897in}{5.067822in}}{\pgfqpoint{6.100084in}{5.075636in}}%
\pgfpathcurveto{\pgfqpoint{6.092270in}{5.083449in}}{\pgfqpoint{6.081671in}{5.087839in}}{\pgfqpoint{6.070621in}{5.087839in}}%
\pgfpathcurveto{\pgfqpoint{6.059571in}{5.087839in}}{\pgfqpoint{6.048972in}{5.083449in}}{\pgfqpoint{6.041158in}{5.075636in}}%
\pgfpathcurveto{\pgfqpoint{6.033344in}{5.067822in}}{\pgfqpoint{6.028954in}{5.057223in}}{\pgfqpoint{6.028954in}{5.046173in}}%
\pgfpathcurveto{\pgfqpoint{6.028954in}{5.035123in}}{\pgfqpoint{6.033344in}{5.024524in}}{\pgfqpoint{6.041158in}{5.016710in}}%
\pgfpathcurveto{\pgfqpoint{6.048972in}{5.008896in}}{\pgfqpoint{6.059571in}{5.004506in}}{\pgfqpoint{6.070621in}{5.004506in}}%
\pgfpathclose%
\pgfusepath{stroke,fill}%
\end{pgfscope}%
\begin{pgfscope}%
\pgfpathrectangle{\pgfqpoint{0.481978in}{0.331635in}}{\pgfqpoint{9.300000in}{7.700000in}}%
\pgfusepath{clip}%
\pgfsetbuttcap%
\pgfsetroundjoin%
\definecolor{currentfill}{rgb}{0.631373,0.788235,0.956863}%
\pgfsetfillcolor{currentfill}%
\pgfsetlinewidth{0.481800pt}%
\definecolor{currentstroke}{rgb}{1.000000,1.000000,1.000000}%
\pgfsetstrokecolor{currentstroke}%
\pgfsetdash{}{0pt}%
\pgfpathmoveto{\pgfqpoint{4.466261in}{1.283836in}}%
\pgfpathcurveto{\pgfqpoint{4.477311in}{1.283836in}}{\pgfqpoint{4.487910in}{1.288226in}}{\pgfqpoint{4.495724in}{1.296040in}}%
\pgfpathcurveto{\pgfqpoint{4.503537in}{1.303854in}}{\pgfqpoint{4.507928in}{1.314453in}}{\pgfqpoint{4.507928in}{1.325503in}}%
\pgfpathcurveto{\pgfqpoint{4.507928in}{1.336553in}}{\pgfqpoint{4.503537in}{1.347152in}}{\pgfqpoint{4.495724in}{1.354966in}}%
\pgfpathcurveto{\pgfqpoint{4.487910in}{1.362779in}}{\pgfqpoint{4.477311in}{1.367169in}}{\pgfqpoint{4.466261in}{1.367169in}}%
\pgfpathcurveto{\pgfqpoint{4.455211in}{1.367169in}}{\pgfqpoint{4.444612in}{1.362779in}}{\pgfqpoint{4.436798in}{1.354966in}}%
\pgfpathcurveto{\pgfqpoint{4.428984in}{1.347152in}}{\pgfqpoint{4.424594in}{1.336553in}}{\pgfqpoint{4.424594in}{1.325503in}}%
\pgfpathcurveto{\pgfqpoint{4.424594in}{1.314453in}}{\pgfqpoint{4.428984in}{1.303854in}}{\pgfqpoint{4.436798in}{1.296040in}}%
\pgfpathcurveto{\pgfqpoint{4.444612in}{1.288226in}}{\pgfqpoint{4.455211in}{1.283836in}}{\pgfqpoint{4.466261in}{1.283836in}}%
\pgfpathclose%
\pgfusepath{stroke,fill}%
\end{pgfscope}%
\begin{pgfscope}%
\pgfpathrectangle{\pgfqpoint{0.481978in}{0.331635in}}{\pgfqpoint{9.300000in}{7.700000in}}%
\pgfusepath{clip}%
\pgfsetbuttcap%
\pgfsetroundjoin%
\definecolor{currentfill}{rgb}{0.631373,0.788235,0.956863}%
\pgfsetfillcolor{currentfill}%
\pgfsetlinewidth{0.481800pt}%
\definecolor{currentstroke}{rgb}{1.000000,1.000000,1.000000}%
\pgfsetstrokecolor{currentstroke}%
\pgfsetdash{}{0pt}%
\pgfpathmoveto{\pgfqpoint{3.012102in}{4.611721in}}%
\pgfpathcurveto{\pgfqpoint{3.023152in}{4.611721in}}{\pgfqpoint{3.033751in}{4.616111in}}{\pgfqpoint{3.041565in}{4.623925in}}%
\pgfpathcurveto{\pgfqpoint{3.049379in}{4.631739in}}{\pgfqpoint{3.053769in}{4.642338in}}{\pgfqpoint{3.053769in}{4.653388in}}%
\pgfpathcurveto{\pgfqpoint{3.053769in}{4.664438in}}{\pgfqpoint{3.049379in}{4.675037in}}{\pgfqpoint{3.041565in}{4.682850in}}%
\pgfpathcurveto{\pgfqpoint{3.033751in}{4.690664in}}{\pgfqpoint{3.023152in}{4.695054in}}{\pgfqpoint{3.012102in}{4.695054in}}%
\pgfpathcurveto{\pgfqpoint{3.001052in}{4.695054in}}{\pgfqpoint{2.990453in}{4.690664in}}{\pgfqpoint{2.982639in}{4.682850in}}%
\pgfpathcurveto{\pgfqpoint{2.974826in}{4.675037in}}{\pgfqpoint{2.970436in}{4.664438in}}{\pgfqpoint{2.970436in}{4.653388in}}%
\pgfpathcurveto{\pgfqpoint{2.970436in}{4.642338in}}{\pgfqpoint{2.974826in}{4.631739in}}{\pgfqpoint{2.982639in}{4.623925in}}%
\pgfpathcurveto{\pgfqpoint{2.990453in}{4.616111in}}{\pgfqpoint{3.001052in}{4.611721in}}{\pgfqpoint{3.012102in}{4.611721in}}%
\pgfpathclose%
\pgfusepath{stroke,fill}%
\end{pgfscope}%
\begin{pgfscope}%
\pgfpathrectangle{\pgfqpoint{0.481978in}{0.331635in}}{\pgfqpoint{9.300000in}{7.700000in}}%
\pgfusepath{clip}%
\pgfsetbuttcap%
\pgfsetroundjoin%
\definecolor{currentfill}{rgb}{0.631373,0.788235,0.956863}%
\pgfsetfillcolor{currentfill}%
\pgfsetlinewidth{0.481800pt}%
\definecolor{currentstroke}{rgb}{1.000000,1.000000,1.000000}%
\pgfsetstrokecolor{currentstroke}%
\pgfsetdash{}{0pt}%
\pgfpathmoveto{\pgfqpoint{6.293539in}{3.584833in}}%
\pgfpathcurveto{\pgfqpoint{6.304589in}{3.584833in}}{\pgfqpoint{6.315188in}{3.589224in}}{\pgfqpoint{6.323001in}{3.597037in}}%
\pgfpathcurveto{\pgfqpoint{6.330815in}{3.604851in}}{\pgfqpoint{6.335205in}{3.615450in}}{\pgfqpoint{6.335205in}{3.626500in}}%
\pgfpathcurveto{\pgfqpoint{6.335205in}{3.637550in}}{\pgfqpoint{6.330815in}{3.648149in}}{\pgfqpoint{6.323001in}{3.655963in}}%
\pgfpathcurveto{\pgfqpoint{6.315188in}{3.663777in}}{\pgfqpoint{6.304589in}{3.668167in}}{\pgfqpoint{6.293539in}{3.668167in}}%
\pgfpathcurveto{\pgfqpoint{6.282489in}{3.668167in}}{\pgfqpoint{6.271890in}{3.663777in}}{\pgfqpoint{6.264076in}{3.655963in}}%
\pgfpathcurveto{\pgfqpoint{6.256262in}{3.648149in}}{\pgfqpoint{6.251872in}{3.637550in}}{\pgfqpoint{6.251872in}{3.626500in}}%
\pgfpathcurveto{\pgfqpoint{6.251872in}{3.615450in}}{\pgfqpoint{6.256262in}{3.604851in}}{\pgfqpoint{6.264076in}{3.597037in}}%
\pgfpathcurveto{\pgfqpoint{6.271890in}{3.589224in}}{\pgfqpoint{6.282489in}{3.584833in}}{\pgfqpoint{6.293539in}{3.584833in}}%
\pgfpathclose%
\pgfusepath{stroke,fill}%
\end{pgfscope}%
\begin{pgfscope}%
\pgfpathrectangle{\pgfqpoint{0.481978in}{0.331635in}}{\pgfqpoint{9.300000in}{7.700000in}}%
\pgfusepath{clip}%
\pgfsetbuttcap%
\pgfsetroundjoin%
\definecolor{currentfill}{rgb}{0.631373,0.788235,0.956863}%
\pgfsetfillcolor{currentfill}%
\pgfsetlinewidth{0.481800pt}%
\definecolor{currentstroke}{rgb}{1.000000,1.000000,1.000000}%
\pgfsetstrokecolor{currentstroke}%
\pgfsetdash{}{0pt}%
\pgfpathmoveto{\pgfqpoint{2.026106in}{1.607439in}}%
\pgfpathcurveto{\pgfqpoint{2.037156in}{1.607439in}}{\pgfqpoint{2.047755in}{1.611829in}}{\pgfqpoint{2.055569in}{1.619643in}}%
\pgfpathcurveto{\pgfqpoint{2.063382in}{1.627456in}}{\pgfqpoint{2.067772in}{1.638056in}}{\pgfqpoint{2.067772in}{1.649106in}}%
\pgfpathcurveto{\pgfqpoint{2.067772in}{1.660156in}}{\pgfqpoint{2.063382in}{1.670755in}}{\pgfqpoint{2.055569in}{1.678568in}}%
\pgfpathcurveto{\pgfqpoint{2.047755in}{1.686382in}}{\pgfqpoint{2.037156in}{1.690772in}}{\pgfqpoint{2.026106in}{1.690772in}}%
\pgfpathcurveto{\pgfqpoint{2.015056in}{1.690772in}}{\pgfqpoint{2.004457in}{1.686382in}}{\pgfqpoint{1.996643in}{1.678568in}}%
\pgfpathcurveto{\pgfqpoint{1.988829in}{1.670755in}}{\pgfqpoint{1.984439in}{1.660156in}}{\pgfqpoint{1.984439in}{1.649106in}}%
\pgfpathcurveto{\pgfqpoint{1.984439in}{1.638056in}}{\pgfqpoint{1.988829in}{1.627456in}}{\pgfqpoint{1.996643in}{1.619643in}}%
\pgfpathcurveto{\pgfqpoint{2.004457in}{1.611829in}}{\pgfqpoint{2.015056in}{1.607439in}}{\pgfqpoint{2.026106in}{1.607439in}}%
\pgfpathclose%
\pgfusepath{stroke,fill}%
\end{pgfscope}%
\begin{pgfscope}%
\pgfpathrectangle{\pgfqpoint{0.481978in}{0.331635in}}{\pgfqpoint{9.300000in}{7.700000in}}%
\pgfusepath{clip}%
\pgfsetbuttcap%
\pgfsetroundjoin%
\definecolor{currentfill}{rgb}{1.000000,0.705882,0.509804}%
\pgfsetfillcolor{currentfill}%
\pgfsetlinewidth{0.481800pt}%
\definecolor{currentstroke}{rgb}{1.000000,1.000000,1.000000}%
\pgfsetstrokecolor{currentstroke}%
\pgfsetdash{}{0pt}%
\pgfpathmoveto{\pgfqpoint{4.850856in}{2.813025in}}%
\pgfpathcurveto{\pgfqpoint{4.861906in}{2.813025in}}{\pgfqpoint{4.872505in}{2.817415in}}{\pgfqpoint{4.880319in}{2.825229in}}%
\pgfpathcurveto{\pgfqpoint{4.888132in}{2.833043in}}{\pgfqpoint{4.892522in}{2.843642in}}{\pgfqpoint{4.892522in}{2.854692in}}%
\pgfpathcurveto{\pgfqpoint{4.892522in}{2.865742in}}{\pgfqpoint{4.888132in}{2.876341in}}{\pgfqpoint{4.880319in}{2.884155in}}%
\pgfpathcurveto{\pgfqpoint{4.872505in}{2.891968in}}{\pgfqpoint{4.861906in}{2.896358in}}{\pgfqpoint{4.850856in}{2.896358in}}%
\pgfpathcurveto{\pgfqpoint{4.839806in}{2.896358in}}{\pgfqpoint{4.829207in}{2.891968in}}{\pgfqpoint{4.821393in}{2.884155in}}%
\pgfpathcurveto{\pgfqpoint{4.813579in}{2.876341in}}{\pgfqpoint{4.809189in}{2.865742in}}{\pgfqpoint{4.809189in}{2.854692in}}%
\pgfpathcurveto{\pgfqpoint{4.809189in}{2.843642in}}{\pgfqpoint{4.813579in}{2.833043in}}{\pgfqpoint{4.821393in}{2.825229in}}%
\pgfpathcurveto{\pgfqpoint{4.829207in}{2.817415in}}{\pgfqpoint{4.839806in}{2.813025in}}{\pgfqpoint{4.850856in}{2.813025in}}%
\pgfpathclose%
\pgfusepath{stroke,fill}%
\end{pgfscope}%
\begin{pgfscope}%
\pgfpathrectangle{\pgfqpoint{0.481978in}{0.331635in}}{\pgfqpoint{9.300000in}{7.700000in}}%
\pgfusepath{clip}%
\pgfsetbuttcap%
\pgfsetroundjoin%
\definecolor{currentfill}{rgb}{1.000000,0.705882,0.509804}%
\pgfsetfillcolor{currentfill}%
\pgfsetlinewidth{0.481800pt}%
\definecolor{currentstroke}{rgb}{1.000000,1.000000,1.000000}%
\pgfsetstrokecolor{currentstroke}%
\pgfsetdash{}{0pt}%
\pgfpathmoveto{\pgfqpoint{2.916721in}{3.078678in}}%
\pgfpathcurveto{\pgfqpoint{2.927771in}{3.078678in}}{\pgfqpoint{2.938370in}{3.083069in}}{\pgfqpoint{2.946183in}{3.090882in}}%
\pgfpathcurveto{\pgfqpoint{2.953997in}{3.098696in}}{\pgfqpoint{2.958387in}{3.109295in}}{\pgfqpoint{2.958387in}{3.120345in}}%
\pgfpathcurveto{\pgfqpoint{2.958387in}{3.131395in}}{\pgfqpoint{2.953997in}{3.141994in}}{\pgfqpoint{2.946183in}{3.149808in}}%
\pgfpathcurveto{\pgfqpoint{2.938370in}{3.157621in}}{\pgfqpoint{2.927771in}{3.162012in}}{\pgfqpoint{2.916721in}{3.162012in}}%
\pgfpathcurveto{\pgfqpoint{2.905671in}{3.162012in}}{\pgfqpoint{2.895072in}{3.157621in}}{\pgfqpoint{2.887258in}{3.149808in}}%
\pgfpathcurveto{\pgfqpoint{2.879444in}{3.141994in}}{\pgfqpoint{2.875054in}{3.131395in}}{\pgfqpoint{2.875054in}{3.120345in}}%
\pgfpathcurveto{\pgfqpoint{2.875054in}{3.109295in}}{\pgfqpoint{2.879444in}{3.098696in}}{\pgfqpoint{2.887258in}{3.090882in}}%
\pgfpathcurveto{\pgfqpoint{2.895072in}{3.083069in}}{\pgfqpoint{2.905671in}{3.078678in}}{\pgfqpoint{2.916721in}{3.078678in}}%
\pgfpathclose%
\pgfusepath{stroke,fill}%
\end{pgfscope}%
\begin{pgfscope}%
\pgfpathrectangle{\pgfqpoint{0.481978in}{0.331635in}}{\pgfqpoint{9.300000in}{7.700000in}}%
\pgfusepath{clip}%
\pgfsetbuttcap%
\pgfsetroundjoin%
\definecolor{currentfill}{rgb}{1.000000,0.705882,0.509804}%
\pgfsetfillcolor{currentfill}%
\pgfsetlinewidth{0.481800pt}%
\definecolor{currentstroke}{rgb}{1.000000,1.000000,1.000000}%
\pgfsetstrokecolor{currentstroke}%
\pgfsetdash{}{0pt}%
\pgfpathmoveto{\pgfqpoint{3.594827in}{5.661176in}}%
\pgfpathcurveto{\pgfqpoint{3.605878in}{5.661176in}}{\pgfqpoint{3.616477in}{5.665566in}}{\pgfqpoint{3.624290in}{5.673380in}}%
\pgfpathcurveto{\pgfqpoint{3.632104in}{5.681193in}}{\pgfqpoint{3.636494in}{5.691792in}}{\pgfqpoint{3.636494in}{5.702842in}}%
\pgfpathcurveto{\pgfqpoint{3.636494in}{5.713892in}}{\pgfqpoint{3.632104in}{5.724491in}}{\pgfqpoint{3.624290in}{5.732305in}}%
\pgfpathcurveto{\pgfqpoint{3.616477in}{5.740119in}}{\pgfqpoint{3.605878in}{5.744509in}}{\pgfqpoint{3.594827in}{5.744509in}}%
\pgfpathcurveto{\pgfqpoint{3.583777in}{5.744509in}}{\pgfqpoint{3.573178in}{5.740119in}}{\pgfqpoint{3.565365in}{5.732305in}}%
\pgfpathcurveto{\pgfqpoint{3.557551in}{5.724491in}}{\pgfqpoint{3.553161in}{5.713892in}}{\pgfqpoint{3.553161in}{5.702842in}}%
\pgfpathcurveto{\pgfqpoint{3.553161in}{5.691792in}}{\pgfqpoint{3.557551in}{5.681193in}}{\pgfqpoint{3.565365in}{5.673380in}}%
\pgfpathcurveto{\pgfqpoint{3.573178in}{5.665566in}}{\pgfqpoint{3.583777in}{5.661176in}}{\pgfqpoint{3.594827in}{5.661176in}}%
\pgfpathclose%
\pgfusepath{stroke,fill}%
\end{pgfscope}%
\begin{pgfscope}%
\pgfpathrectangle{\pgfqpoint{0.481978in}{0.331635in}}{\pgfqpoint{9.300000in}{7.700000in}}%
\pgfusepath{clip}%
\pgfsetbuttcap%
\pgfsetroundjoin%
\definecolor{currentfill}{rgb}{1.000000,0.705882,0.509804}%
\pgfsetfillcolor{currentfill}%
\pgfsetlinewidth{0.481800pt}%
\definecolor{currentstroke}{rgb}{1.000000,1.000000,1.000000}%
\pgfsetstrokecolor{currentstroke}%
\pgfsetdash{}{0pt}%
\pgfpathmoveto{\pgfqpoint{3.147191in}{3.768081in}}%
\pgfpathcurveto{\pgfqpoint{3.158241in}{3.768081in}}{\pgfqpoint{3.168840in}{3.772471in}}{\pgfqpoint{3.176654in}{3.780285in}}%
\pgfpathcurveto{\pgfqpoint{3.184467in}{3.788099in}}{\pgfqpoint{3.188857in}{3.798698in}}{\pgfqpoint{3.188857in}{3.809748in}}%
\pgfpathcurveto{\pgfqpoint{3.188857in}{3.820798in}}{\pgfqpoint{3.184467in}{3.831397in}}{\pgfqpoint{3.176654in}{3.839211in}}%
\pgfpathcurveto{\pgfqpoint{3.168840in}{3.847024in}}{\pgfqpoint{3.158241in}{3.851415in}}{\pgfqpoint{3.147191in}{3.851415in}}%
\pgfpathcurveto{\pgfqpoint{3.136141in}{3.851415in}}{\pgfqpoint{3.125542in}{3.847024in}}{\pgfqpoint{3.117728in}{3.839211in}}%
\pgfpathcurveto{\pgfqpoint{3.109914in}{3.831397in}}{\pgfqpoint{3.105524in}{3.820798in}}{\pgfqpoint{3.105524in}{3.809748in}}%
\pgfpathcurveto{\pgfqpoint{3.105524in}{3.798698in}}{\pgfqpoint{3.109914in}{3.788099in}}{\pgfqpoint{3.117728in}{3.780285in}}%
\pgfpathcurveto{\pgfqpoint{3.125542in}{3.772471in}}{\pgfqpoint{3.136141in}{3.768081in}}{\pgfqpoint{3.147191in}{3.768081in}}%
\pgfpathclose%
\pgfusepath{stroke,fill}%
\end{pgfscope}%
\begin{pgfscope}%
\pgfpathrectangle{\pgfqpoint{0.481978in}{0.331635in}}{\pgfqpoint{9.300000in}{7.700000in}}%
\pgfusepath{clip}%
\pgfsetbuttcap%
\pgfsetroundjoin%
\definecolor{currentfill}{rgb}{1.000000,0.705882,0.509804}%
\pgfsetfillcolor{currentfill}%
\pgfsetlinewidth{0.481800pt}%
\definecolor{currentstroke}{rgb}{1.000000,1.000000,1.000000}%
\pgfsetstrokecolor{currentstroke}%
\pgfsetdash{}{0pt}%
\pgfpathmoveto{\pgfqpoint{3.148403in}{2.273646in}}%
\pgfpathcurveto{\pgfqpoint{3.159453in}{2.273646in}}{\pgfqpoint{3.170052in}{2.278036in}}{\pgfqpoint{3.177866in}{2.285850in}}%
\pgfpathcurveto{\pgfqpoint{3.185679in}{2.293663in}}{\pgfqpoint{3.190070in}{2.304262in}}{\pgfqpoint{3.190070in}{2.315312in}}%
\pgfpathcurveto{\pgfqpoint{3.190070in}{2.326363in}}{\pgfqpoint{3.185679in}{2.336962in}}{\pgfqpoint{3.177866in}{2.344775in}}%
\pgfpathcurveto{\pgfqpoint{3.170052in}{2.352589in}}{\pgfqpoint{3.159453in}{2.356979in}}{\pgfqpoint{3.148403in}{2.356979in}}%
\pgfpathcurveto{\pgfqpoint{3.137353in}{2.356979in}}{\pgfqpoint{3.126754in}{2.352589in}}{\pgfqpoint{3.118940in}{2.344775in}}%
\pgfpathcurveto{\pgfqpoint{3.111127in}{2.336962in}}{\pgfqpoint{3.106736in}{2.326363in}}{\pgfqpoint{3.106736in}{2.315312in}}%
\pgfpathcurveto{\pgfqpoint{3.106736in}{2.304262in}}{\pgfqpoint{3.111127in}{2.293663in}}{\pgfqpoint{3.118940in}{2.285850in}}%
\pgfpathcurveto{\pgfqpoint{3.126754in}{2.278036in}}{\pgfqpoint{3.137353in}{2.273646in}}{\pgfqpoint{3.148403in}{2.273646in}}%
\pgfpathclose%
\pgfusepath{stroke,fill}%
\end{pgfscope}%
\begin{pgfscope}%
\pgfpathrectangle{\pgfqpoint{0.481978in}{0.331635in}}{\pgfqpoint{9.300000in}{7.700000in}}%
\pgfusepath{clip}%
\pgfsetbuttcap%
\pgfsetroundjoin%
\definecolor{currentfill}{rgb}{1.000000,0.705882,0.509804}%
\pgfsetfillcolor{currentfill}%
\pgfsetlinewidth{0.481800pt}%
\definecolor{currentstroke}{rgb}{1.000000,1.000000,1.000000}%
\pgfsetstrokecolor{currentstroke}%
\pgfsetdash{}{0pt}%
\pgfpathmoveto{\pgfqpoint{3.951193in}{0.913087in}}%
\pgfpathcurveto{\pgfqpoint{3.962243in}{0.913087in}}{\pgfqpoint{3.972842in}{0.917477in}}{\pgfqpoint{3.980656in}{0.925291in}}%
\pgfpathcurveto{\pgfqpoint{3.988470in}{0.933105in}}{\pgfqpoint{3.992860in}{0.943704in}}{\pgfqpoint{3.992860in}{0.954754in}}%
\pgfpathcurveto{\pgfqpoint{3.992860in}{0.965804in}}{\pgfqpoint{3.988470in}{0.976403in}}{\pgfqpoint{3.980656in}{0.984217in}}%
\pgfpathcurveto{\pgfqpoint{3.972842in}{0.992030in}}{\pgfqpoint{3.962243in}{0.996421in}}{\pgfqpoint{3.951193in}{0.996421in}}%
\pgfpathcurveto{\pgfqpoint{3.940143in}{0.996421in}}{\pgfqpoint{3.929544in}{0.992030in}}{\pgfqpoint{3.921731in}{0.984217in}}%
\pgfpathcurveto{\pgfqpoint{3.913917in}{0.976403in}}{\pgfqpoint{3.909527in}{0.965804in}}{\pgfqpoint{3.909527in}{0.954754in}}%
\pgfpathcurveto{\pgfqpoint{3.909527in}{0.943704in}}{\pgfqpoint{3.913917in}{0.933105in}}{\pgfqpoint{3.921731in}{0.925291in}}%
\pgfpathcurveto{\pgfqpoint{3.929544in}{0.917477in}}{\pgfqpoint{3.940143in}{0.913087in}}{\pgfqpoint{3.951193in}{0.913087in}}%
\pgfpathclose%
\pgfusepath{stroke,fill}%
\end{pgfscope}%
\begin{pgfscope}%
\pgfpathrectangle{\pgfqpoint{0.481978in}{0.331635in}}{\pgfqpoint{9.300000in}{7.700000in}}%
\pgfusepath{clip}%
\pgfsetbuttcap%
\pgfsetroundjoin%
\definecolor{currentfill}{rgb}{1.000000,0.705882,0.509804}%
\pgfsetfillcolor{currentfill}%
\pgfsetlinewidth{0.481800pt}%
\definecolor{currentstroke}{rgb}{1.000000,1.000000,1.000000}%
\pgfsetstrokecolor{currentstroke}%
\pgfsetdash{}{0pt}%
\pgfpathmoveto{\pgfqpoint{5.646419in}{4.278341in}}%
\pgfpathcurveto{\pgfqpoint{5.657469in}{4.278341in}}{\pgfqpoint{5.668068in}{4.282731in}}{\pgfqpoint{5.675881in}{4.290545in}}%
\pgfpathcurveto{\pgfqpoint{5.683695in}{4.298359in}}{\pgfqpoint{5.688085in}{4.308958in}}{\pgfqpoint{5.688085in}{4.320008in}}%
\pgfpathcurveto{\pgfqpoint{5.688085in}{4.331058in}}{\pgfqpoint{5.683695in}{4.341657in}}{\pgfqpoint{5.675881in}{4.349471in}}%
\pgfpathcurveto{\pgfqpoint{5.668068in}{4.357284in}}{\pgfqpoint{5.657469in}{4.361675in}}{\pgfqpoint{5.646419in}{4.361675in}}%
\pgfpathcurveto{\pgfqpoint{5.635369in}{4.361675in}}{\pgfqpoint{5.624769in}{4.357284in}}{\pgfqpoint{5.616956in}{4.349471in}}%
\pgfpathcurveto{\pgfqpoint{5.609142in}{4.341657in}}{\pgfqpoint{5.604752in}{4.331058in}}{\pgfqpoint{5.604752in}{4.320008in}}%
\pgfpathcurveto{\pgfqpoint{5.604752in}{4.308958in}}{\pgfqpoint{5.609142in}{4.298359in}}{\pgfqpoint{5.616956in}{4.290545in}}%
\pgfpathcurveto{\pgfqpoint{5.624769in}{4.282731in}}{\pgfqpoint{5.635369in}{4.278341in}}{\pgfqpoint{5.646419in}{4.278341in}}%
\pgfpathclose%
\pgfusepath{stroke,fill}%
\end{pgfscope}%
\begin{pgfscope}%
\pgfpathrectangle{\pgfqpoint{0.481978in}{0.331635in}}{\pgfqpoint{9.300000in}{7.700000in}}%
\pgfusepath{clip}%
\pgfsetbuttcap%
\pgfsetroundjoin%
\definecolor{currentfill}{rgb}{1.000000,0.705882,0.509804}%
\pgfsetfillcolor{currentfill}%
\pgfsetlinewidth{0.481800pt}%
\definecolor{currentstroke}{rgb}{1.000000,1.000000,1.000000}%
\pgfsetstrokecolor{currentstroke}%
\pgfsetdash{}{0pt}%
\pgfpathmoveto{\pgfqpoint{2.847205in}{4.307014in}}%
\pgfpathcurveto{\pgfqpoint{2.858255in}{4.307014in}}{\pgfqpoint{2.868854in}{4.311405in}}{\pgfqpoint{2.876668in}{4.319218in}}%
\pgfpathcurveto{\pgfqpoint{2.884481in}{4.327032in}}{\pgfqpoint{2.888872in}{4.337631in}}{\pgfqpoint{2.888872in}{4.348681in}}%
\pgfpathcurveto{\pgfqpoint{2.888872in}{4.359731in}}{\pgfqpoint{2.884481in}{4.370330in}}{\pgfqpoint{2.876668in}{4.378144in}}%
\pgfpathcurveto{\pgfqpoint{2.868854in}{4.385957in}}{\pgfqpoint{2.858255in}{4.390348in}}{\pgfqpoint{2.847205in}{4.390348in}}%
\pgfpathcurveto{\pgfqpoint{2.836155in}{4.390348in}}{\pgfqpoint{2.825556in}{4.385957in}}{\pgfqpoint{2.817742in}{4.378144in}}%
\pgfpathcurveto{\pgfqpoint{2.809928in}{4.370330in}}{\pgfqpoint{2.805538in}{4.359731in}}{\pgfqpoint{2.805538in}{4.348681in}}%
\pgfpathcurveto{\pgfqpoint{2.805538in}{4.337631in}}{\pgfqpoint{2.809928in}{4.327032in}}{\pgfqpoint{2.817742in}{4.319218in}}%
\pgfpathcurveto{\pgfqpoint{2.825556in}{4.311405in}}{\pgfqpoint{2.836155in}{4.307014in}}{\pgfqpoint{2.847205in}{4.307014in}}%
\pgfpathclose%
\pgfusepath{stroke,fill}%
\end{pgfscope}%
\begin{pgfscope}%
\pgfpathrectangle{\pgfqpoint{0.481978in}{0.331635in}}{\pgfqpoint{9.300000in}{7.700000in}}%
\pgfusepath{clip}%
\pgfsetbuttcap%
\pgfsetroundjoin%
\definecolor{currentfill}{rgb}{1.000000,0.705882,0.509804}%
\pgfsetfillcolor{currentfill}%
\pgfsetlinewidth{0.481800pt}%
\definecolor{currentstroke}{rgb}{1.000000,1.000000,1.000000}%
\pgfsetstrokecolor{currentstroke}%
\pgfsetdash{}{0pt}%
\pgfpathmoveto{\pgfqpoint{3.926575in}{2.335876in}}%
\pgfpathcurveto{\pgfqpoint{3.937625in}{2.335876in}}{\pgfqpoint{3.948224in}{2.340266in}}{\pgfqpoint{3.956038in}{2.348080in}}%
\pgfpathcurveto{\pgfqpoint{3.963852in}{2.355894in}}{\pgfqpoint{3.968242in}{2.366493in}}{\pgfqpoint{3.968242in}{2.377543in}}%
\pgfpathcurveto{\pgfqpoint{3.968242in}{2.388593in}}{\pgfqpoint{3.963852in}{2.399192in}}{\pgfqpoint{3.956038in}{2.407006in}}%
\pgfpathcurveto{\pgfqpoint{3.948224in}{2.414819in}}{\pgfqpoint{3.937625in}{2.419209in}}{\pgfqpoint{3.926575in}{2.419209in}}%
\pgfpathcurveto{\pgfqpoint{3.915525in}{2.419209in}}{\pgfqpoint{3.904926in}{2.414819in}}{\pgfqpoint{3.897112in}{2.407006in}}%
\pgfpathcurveto{\pgfqpoint{3.889299in}{2.399192in}}{\pgfqpoint{3.884908in}{2.388593in}}{\pgfqpoint{3.884908in}{2.377543in}}%
\pgfpathcurveto{\pgfqpoint{3.884908in}{2.366493in}}{\pgfqpoint{3.889299in}{2.355894in}}{\pgfqpoint{3.897112in}{2.348080in}}%
\pgfpathcurveto{\pgfqpoint{3.904926in}{2.340266in}}{\pgfqpoint{3.915525in}{2.335876in}}{\pgfqpoint{3.926575in}{2.335876in}}%
\pgfpathclose%
\pgfusepath{stroke,fill}%
\end{pgfscope}%
\begin{pgfscope}%
\pgfpathrectangle{\pgfqpoint{0.481978in}{0.331635in}}{\pgfqpoint{9.300000in}{7.700000in}}%
\pgfusepath{clip}%
\pgfsetbuttcap%
\pgfsetroundjoin%
\definecolor{currentfill}{rgb}{1.000000,0.705882,0.509804}%
\pgfsetfillcolor{currentfill}%
\pgfsetlinewidth{0.481800pt}%
\definecolor{currentstroke}{rgb}{1.000000,1.000000,1.000000}%
\pgfsetstrokecolor{currentstroke}%
\pgfsetdash{}{0pt}%
\pgfpathmoveto{\pgfqpoint{2.406478in}{5.862103in}}%
\pgfpathcurveto{\pgfqpoint{2.417529in}{5.862103in}}{\pgfqpoint{2.428128in}{5.866493in}}{\pgfqpoint{2.435941in}{5.874307in}}%
\pgfpathcurveto{\pgfqpoint{2.443755in}{5.882121in}}{\pgfqpoint{2.448145in}{5.892720in}}{\pgfqpoint{2.448145in}{5.903770in}}%
\pgfpathcurveto{\pgfqpoint{2.448145in}{5.914820in}}{\pgfqpoint{2.443755in}{5.925419in}}{\pgfqpoint{2.435941in}{5.933233in}}%
\pgfpathcurveto{\pgfqpoint{2.428128in}{5.941046in}}{\pgfqpoint{2.417529in}{5.945436in}}{\pgfqpoint{2.406478in}{5.945436in}}%
\pgfpathcurveto{\pgfqpoint{2.395428in}{5.945436in}}{\pgfqpoint{2.384829in}{5.941046in}}{\pgfqpoint{2.377016in}{5.933233in}}%
\pgfpathcurveto{\pgfqpoint{2.369202in}{5.925419in}}{\pgfqpoint{2.364812in}{5.914820in}}{\pgfqpoint{2.364812in}{5.903770in}}%
\pgfpathcurveto{\pgfqpoint{2.364812in}{5.892720in}}{\pgfqpoint{2.369202in}{5.882121in}}{\pgfqpoint{2.377016in}{5.874307in}}%
\pgfpathcurveto{\pgfqpoint{2.384829in}{5.866493in}}{\pgfqpoint{2.395428in}{5.862103in}}{\pgfqpoint{2.406478in}{5.862103in}}%
\pgfpathclose%
\pgfusepath{stroke,fill}%
\end{pgfscope}%
\begin{pgfscope}%
\pgfpathrectangle{\pgfqpoint{0.481978in}{0.331635in}}{\pgfqpoint{9.300000in}{7.700000in}}%
\pgfusepath{clip}%
\pgfsetbuttcap%
\pgfsetroundjoin%
\definecolor{currentfill}{rgb}{1.000000,0.705882,0.509804}%
\pgfsetfillcolor{currentfill}%
\pgfsetlinewidth{0.481800pt}%
\definecolor{currentstroke}{rgb}{1.000000,1.000000,1.000000}%
\pgfsetstrokecolor{currentstroke}%
\pgfsetdash{}{0pt}%
\pgfpathmoveto{\pgfqpoint{4.065253in}{2.330807in}}%
\pgfpathcurveto{\pgfqpoint{4.076304in}{2.330807in}}{\pgfqpoint{4.086903in}{2.335197in}}{\pgfqpoint{4.094716in}{2.343011in}}%
\pgfpathcurveto{\pgfqpoint{4.102530in}{2.350824in}}{\pgfqpoint{4.106920in}{2.361423in}}{\pgfqpoint{4.106920in}{2.372474in}}%
\pgfpathcurveto{\pgfqpoint{4.106920in}{2.383524in}}{\pgfqpoint{4.102530in}{2.394123in}}{\pgfqpoint{4.094716in}{2.401936in}}%
\pgfpathcurveto{\pgfqpoint{4.086903in}{2.409750in}}{\pgfqpoint{4.076304in}{2.414140in}}{\pgfqpoint{4.065253in}{2.414140in}}%
\pgfpathcurveto{\pgfqpoint{4.054203in}{2.414140in}}{\pgfqpoint{4.043604in}{2.409750in}}{\pgfqpoint{4.035791in}{2.401936in}}%
\pgfpathcurveto{\pgfqpoint{4.027977in}{2.394123in}}{\pgfqpoint{4.023587in}{2.383524in}}{\pgfqpoint{4.023587in}{2.372474in}}%
\pgfpathcurveto{\pgfqpoint{4.023587in}{2.361423in}}{\pgfqpoint{4.027977in}{2.350824in}}{\pgfqpoint{4.035791in}{2.343011in}}%
\pgfpathcurveto{\pgfqpoint{4.043604in}{2.335197in}}{\pgfqpoint{4.054203in}{2.330807in}}{\pgfqpoint{4.065253in}{2.330807in}}%
\pgfpathclose%
\pgfusepath{stroke,fill}%
\end{pgfscope}%
\begin{pgfscope}%
\pgfpathrectangle{\pgfqpoint{0.481978in}{0.331635in}}{\pgfqpoint{9.300000in}{7.700000in}}%
\pgfusepath{clip}%
\pgfsetbuttcap%
\pgfsetroundjoin%
\definecolor{currentfill}{rgb}{1.000000,0.705882,0.509804}%
\pgfsetfillcolor{currentfill}%
\pgfsetlinewidth{0.481800pt}%
\definecolor{currentstroke}{rgb}{1.000000,1.000000,1.000000}%
\pgfsetstrokecolor{currentstroke}%
\pgfsetdash{}{0pt}%
\pgfpathmoveto{\pgfqpoint{4.470303in}{5.164449in}}%
\pgfpathcurveto{\pgfqpoint{4.481353in}{5.164449in}}{\pgfqpoint{4.491952in}{5.168840in}}{\pgfqpoint{4.499765in}{5.176653in}}%
\pgfpathcurveto{\pgfqpoint{4.507579in}{5.184467in}}{\pgfqpoint{4.511969in}{5.195066in}}{\pgfqpoint{4.511969in}{5.206116in}}%
\pgfpathcurveto{\pgfqpoint{4.511969in}{5.217166in}}{\pgfqpoint{4.507579in}{5.227765in}}{\pgfqpoint{4.499765in}{5.235579in}}%
\pgfpathcurveto{\pgfqpoint{4.491952in}{5.243392in}}{\pgfqpoint{4.481353in}{5.247783in}}{\pgfqpoint{4.470303in}{5.247783in}}%
\pgfpathcurveto{\pgfqpoint{4.459253in}{5.247783in}}{\pgfqpoint{4.448653in}{5.243392in}}{\pgfqpoint{4.440840in}{5.235579in}}%
\pgfpathcurveto{\pgfqpoint{4.433026in}{5.227765in}}{\pgfqpoint{4.428636in}{5.217166in}}{\pgfqpoint{4.428636in}{5.206116in}}%
\pgfpathcurveto{\pgfqpoint{4.428636in}{5.195066in}}{\pgfqpoint{4.433026in}{5.184467in}}{\pgfqpoint{4.440840in}{5.176653in}}%
\pgfpathcurveto{\pgfqpoint{4.448653in}{5.168840in}}{\pgfqpoint{4.459253in}{5.164449in}}{\pgfqpoint{4.470303in}{5.164449in}}%
\pgfpathclose%
\pgfusepath{stroke,fill}%
\end{pgfscope}%
\begin{pgfscope}%
\pgfpathrectangle{\pgfqpoint{0.481978in}{0.331635in}}{\pgfqpoint{9.300000in}{7.700000in}}%
\pgfusepath{clip}%
\pgfsetbuttcap%
\pgfsetroundjoin%
\definecolor{currentfill}{rgb}{1.000000,0.705882,0.509804}%
\pgfsetfillcolor{currentfill}%
\pgfsetlinewidth{0.481800pt}%
\definecolor{currentstroke}{rgb}{1.000000,1.000000,1.000000}%
\pgfsetstrokecolor{currentstroke}%
\pgfsetdash{}{0pt}%
\pgfpathmoveto{\pgfqpoint{2.649493in}{4.061155in}}%
\pgfpathcurveto{\pgfqpoint{2.660543in}{4.061155in}}{\pgfqpoint{2.671142in}{4.065545in}}{\pgfqpoint{2.678956in}{4.073359in}}%
\pgfpathcurveto{\pgfqpoint{2.686769in}{4.081172in}}{\pgfqpoint{2.691160in}{4.091771in}}{\pgfqpoint{2.691160in}{4.102821in}}%
\pgfpathcurveto{\pgfqpoint{2.691160in}{4.113871in}}{\pgfqpoint{2.686769in}{4.124470in}}{\pgfqpoint{2.678956in}{4.132284in}}%
\pgfpathcurveto{\pgfqpoint{2.671142in}{4.140098in}}{\pgfqpoint{2.660543in}{4.144488in}}{\pgfqpoint{2.649493in}{4.144488in}}%
\pgfpathcurveto{\pgfqpoint{2.638443in}{4.144488in}}{\pgfqpoint{2.627844in}{4.140098in}}{\pgfqpoint{2.620030in}{4.132284in}}%
\pgfpathcurveto{\pgfqpoint{2.612216in}{4.124470in}}{\pgfqpoint{2.607826in}{4.113871in}}{\pgfqpoint{2.607826in}{4.102821in}}%
\pgfpathcurveto{\pgfqpoint{2.607826in}{4.091771in}}{\pgfqpoint{2.612216in}{4.081172in}}{\pgfqpoint{2.620030in}{4.073359in}}%
\pgfpathcurveto{\pgfqpoint{2.627844in}{4.065545in}}{\pgfqpoint{2.638443in}{4.061155in}}{\pgfqpoint{2.649493in}{4.061155in}}%
\pgfpathclose%
\pgfusepath{stroke,fill}%
\end{pgfscope}%
\begin{pgfscope}%
\pgfpathrectangle{\pgfqpoint{0.481978in}{0.331635in}}{\pgfqpoint{9.300000in}{7.700000in}}%
\pgfusepath{clip}%
\pgfsetbuttcap%
\pgfsetroundjoin%
\definecolor{currentfill}{rgb}{1.000000,0.705882,0.509804}%
\pgfsetfillcolor{currentfill}%
\pgfsetlinewidth{0.481800pt}%
\definecolor{currentstroke}{rgb}{1.000000,1.000000,1.000000}%
\pgfsetstrokecolor{currentstroke}%
\pgfsetdash{}{0pt}%
\pgfpathmoveto{\pgfqpoint{2.136818in}{3.273180in}}%
\pgfpathcurveto{\pgfqpoint{2.147868in}{3.273180in}}{\pgfqpoint{2.158467in}{3.277570in}}{\pgfqpoint{2.166281in}{3.285383in}}%
\pgfpathcurveto{\pgfqpoint{2.174094in}{3.293197in}}{\pgfqpoint{2.178485in}{3.303796in}}{\pgfqpoint{2.178485in}{3.314846in}}%
\pgfpathcurveto{\pgfqpoint{2.178485in}{3.325896in}}{\pgfqpoint{2.174094in}{3.336495in}}{\pgfqpoint{2.166281in}{3.344309in}}%
\pgfpathcurveto{\pgfqpoint{2.158467in}{3.352123in}}{\pgfqpoint{2.147868in}{3.356513in}}{\pgfqpoint{2.136818in}{3.356513in}}%
\pgfpathcurveto{\pgfqpoint{2.125768in}{3.356513in}}{\pgfqpoint{2.115169in}{3.352123in}}{\pgfqpoint{2.107355in}{3.344309in}}%
\pgfpathcurveto{\pgfqpoint{2.099542in}{3.336495in}}{\pgfqpoint{2.095151in}{3.325896in}}{\pgfqpoint{2.095151in}{3.314846in}}%
\pgfpathcurveto{\pgfqpoint{2.095151in}{3.303796in}}{\pgfqpoint{2.099542in}{3.293197in}}{\pgfqpoint{2.107355in}{3.285383in}}%
\pgfpathcurveto{\pgfqpoint{2.115169in}{3.277570in}}{\pgfqpoint{2.125768in}{3.273180in}}{\pgfqpoint{2.136818in}{3.273180in}}%
\pgfpathclose%
\pgfusepath{stroke,fill}%
\end{pgfscope}%
\begin{pgfscope}%
\pgfpathrectangle{\pgfqpoint{0.481978in}{0.331635in}}{\pgfqpoint{9.300000in}{7.700000in}}%
\pgfusepath{clip}%
\pgfsetbuttcap%
\pgfsetroundjoin%
\definecolor{currentfill}{rgb}{1.000000,0.705882,0.509804}%
\pgfsetfillcolor{currentfill}%
\pgfsetlinewidth{0.481800pt}%
\definecolor{currentstroke}{rgb}{1.000000,1.000000,1.000000}%
\pgfsetstrokecolor{currentstroke}%
\pgfsetdash{}{0pt}%
\pgfpathmoveto{\pgfqpoint{3.304744in}{3.404313in}}%
\pgfpathcurveto{\pgfqpoint{3.315794in}{3.404313in}}{\pgfqpoint{3.326393in}{3.408703in}}{\pgfqpoint{3.334207in}{3.416517in}}%
\pgfpathcurveto{\pgfqpoint{3.342021in}{3.424331in}}{\pgfqpoint{3.346411in}{3.434930in}}{\pgfqpoint{3.346411in}{3.445980in}}%
\pgfpathcurveto{\pgfqpoint{3.346411in}{3.457030in}}{\pgfqpoint{3.342021in}{3.467629in}}{\pgfqpoint{3.334207in}{3.475443in}}%
\pgfpathcurveto{\pgfqpoint{3.326393in}{3.483256in}}{\pgfqpoint{3.315794in}{3.487646in}}{\pgfqpoint{3.304744in}{3.487646in}}%
\pgfpathcurveto{\pgfqpoint{3.293694in}{3.487646in}}{\pgfqpoint{3.283095in}{3.483256in}}{\pgfqpoint{3.275282in}{3.475443in}}%
\pgfpathcurveto{\pgfqpoint{3.267468in}{3.467629in}}{\pgfqpoint{3.263078in}{3.457030in}}{\pgfqpoint{3.263078in}{3.445980in}}%
\pgfpathcurveto{\pgfqpoint{3.263078in}{3.434930in}}{\pgfqpoint{3.267468in}{3.424331in}}{\pgfqpoint{3.275282in}{3.416517in}}%
\pgfpathcurveto{\pgfqpoint{3.283095in}{3.408703in}}{\pgfqpoint{3.293694in}{3.404313in}}{\pgfqpoint{3.304744in}{3.404313in}}%
\pgfpathclose%
\pgfusepath{stroke,fill}%
\end{pgfscope}%
\begin{pgfscope}%
\pgfpathrectangle{\pgfqpoint{0.481978in}{0.331635in}}{\pgfqpoint{9.300000in}{7.700000in}}%
\pgfusepath{clip}%
\pgfsetbuttcap%
\pgfsetroundjoin%
\definecolor{currentfill}{rgb}{1.000000,0.705882,0.509804}%
\pgfsetfillcolor{currentfill}%
\pgfsetlinewidth{0.481800pt}%
\definecolor{currentstroke}{rgb}{1.000000,1.000000,1.000000}%
\pgfsetstrokecolor{currentstroke}%
\pgfsetdash{}{0pt}%
\pgfpathmoveto{\pgfqpoint{1.284269in}{4.704795in}}%
\pgfpathcurveto{\pgfqpoint{1.295319in}{4.704795in}}{\pgfqpoint{1.305918in}{4.709185in}}{\pgfqpoint{1.313731in}{4.716999in}}%
\pgfpathcurveto{\pgfqpoint{1.321545in}{4.724812in}}{\pgfqpoint{1.325935in}{4.735411in}}{\pgfqpoint{1.325935in}{4.746461in}}%
\pgfpathcurveto{\pgfqpoint{1.325935in}{4.757511in}}{\pgfqpoint{1.321545in}{4.768110in}}{\pgfqpoint{1.313731in}{4.775924in}}%
\pgfpathcurveto{\pgfqpoint{1.305918in}{4.783738in}}{\pgfqpoint{1.295319in}{4.788128in}}{\pgfqpoint{1.284269in}{4.788128in}}%
\pgfpathcurveto{\pgfqpoint{1.273219in}{4.788128in}}{\pgfqpoint{1.262620in}{4.783738in}}{\pgfqpoint{1.254806in}{4.775924in}}%
\pgfpathcurveto{\pgfqpoint{1.246992in}{4.768110in}}{\pgfqpoint{1.242602in}{4.757511in}}{\pgfqpoint{1.242602in}{4.746461in}}%
\pgfpathcurveto{\pgfqpoint{1.242602in}{4.735411in}}{\pgfqpoint{1.246992in}{4.724812in}}{\pgfqpoint{1.254806in}{4.716999in}}%
\pgfpathcurveto{\pgfqpoint{1.262620in}{4.709185in}}{\pgfqpoint{1.273219in}{4.704795in}}{\pgfqpoint{1.284269in}{4.704795in}}%
\pgfpathclose%
\pgfusepath{stroke,fill}%
\end{pgfscope}%
\begin{pgfscope}%
\pgfpathrectangle{\pgfqpoint{0.481978in}{0.331635in}}{\pgfqpoint{9.300000in}{7.700000in}}%
\pgfusepath{clip}%
\pgfsetbuttcap%
\pgfsetroundjoin%
\definecolor{currentfill}{rgb}{1.000000,0.705882,0.509804}%
\pgfsetfillcolor{currentfill}%
\pgfsetlinewidth{0.481800pt}%
\definecolor{currentstroke}{rgb}{1.000000,1.000000,1.000000}%
\pgfsetstrokecolor{currentstroke}%
\pgfsetdash{}{0pt}%
\pgfpathmoveto{\pgfqpoint{3.921640in}{6.496134in}}%
\pgfpathcurveto{\pgfqpoint{3.932691in}{6.496134in}}{\pgfqpoint{3.943290in}{6.500525in}}{\pgfqpoint{3.951103in}{6.508338in}}%
\pgfpathcurveto{\pgfqpoint{3.958917in}{6.516152in}}{\pgfqpoint{3.963307in}{6.526751in}}{\pgfqpoint{3.963307in}{6.537801in}}%
\pgfpathcurveto{\pgfqpoint{3.963307in}{6.548851in}}{\pgfqpoint{3.958917in}{6.559450in}}{\pgfqpoint{3.951103in}{6.567264in}}%
\pgfpathcurveto{\pgfqpoint{3.943290in}{6.575077in}}{\pgfqpoint{3.932691in}{6.579468in}}{\pgfqpoint{3.921640in}{6.579468in}}%
\pgfpathcurveto{\pgfqpoint{3.910590in}{6.579468in}}{\pgfqpoint{3.899991in}{6.575077in}}{\pgfqpoint{3.892178in}{6.567264in}}%
\pgfpathcurveto{\pgfqpoint{3.884364in}{6.559450in}}{\pgfqpoint{3.879974in}{6.548851in}}{\pgfqpoint{3.879974in}{6.537801in}}%
\pgfpathcurveto{\pgfqpoint{3.879974in}{6.526751in}}{\pgfqpoint{3.884364in}{6.516152in}}{\pgfqpoint{3.892178in}{6.508338in}}%
\pgfpathcurveto{\pgfqpoint{3.899991in}{6.500525in}}{\pgfqpoint{3.910590in}{6.496134in}}{\pgfqpoint{3.921640in}{6.496134in}}%
\pgfpathclose%
\pgfusepath{stroke,fill}%
\end{pgfscope}%
\begin{pgfscope}%
\pgfpathrectangle{\pgfqpoint{0.481978in}{0.331635in}}{\pgfqpoint{9.300000in}{7.700000in}}%
\pgfusepath{clip}%
\pgfsetbuttcap%
\pgfsetroundjoin%
\definecolor{currentfill}{rgb}{1.000000,0.705882,0.509804}%
\pgfsetfillcolor{currentfill}%
\pgfsetlinewidth{0.481800pt}%
\definecolor{currentstroke}{rgb}{1.000000,1.000000,1.000000}%
\pgfsetstrokecolor{currentstroke}%
\pgfsetdash{}{0pt}%
\pgfpathmoveto{\pgfqpoint{4.551669in}{3.847471in}}%
\pgfpathcurveto{\pgfqpoint{4.562719in}{3.847471in}}{\pgfqpoint{4.573318in}{3.851862in}}{\pgfqpoint{4.581131in}{3.859675in}}%
\pgfpathcurveto{\pgfqpoint{4.588945in}{3.867489in}}{\pgfqpoint{4.593335in}{3.878088in}}{\pgfqpoint{4.593335in}{3.889138in}}%
\pgfpathcurveto{\pgfqpoint{4.593335in}{3.900188in}}{\pgfqpoint{4.588945in}{3.910787in}}{\pgfqpoint{4.581131in}{3.918601in}}%
\pgfpathcurveto{\pgfqpoint{4.573318in}{3.926414in}}{\pgfqpoint{4.562719in}{3.930805in}}{\pgfqpoint{4.551669in}{3.930805in}}%
\pgfpathcurveto{\pgfqpoint{4.540618in}{3.930805in}}{\pgfqpoint{4.530019in}{3.926414in}}{\pgfqpoint{4.522206in}{3.918601in}}%
\pgfpathcurveto{\pgfqpoint{4.514392in}{3.910787in}}{\pgfqpoint{4.510002in}{3.900188in}}{\pgfqpoint{4.510002in}{3.889138in}}%
\pgfpathcurveto{\pgfqpoint{4.510002in}{3.878088in}}{\pgfqpoint{4.514392in}{3.867489in}}{\pgfqpoint{4.522206in}{3.859675in}}%
\pgfpathcurveto{\pgfqpoint{4.530019in}{3.851862in}}{\pgfqpoint{4.540618in}{3.847471in}}{\pgfqpoint{4.551669in}{3.847471in}}%
\pgfpathclose%
\pgfusepath{stroke,fill}%
\end{pgfscope}%
\begin{pgfscope}%
\pgfpathrectangle{\pgfqpoint{0.481978in}{0.331635in}}{\pgfqpoint{9.300000in}{7.700000in}}%
\pgfusepath{clip}%
\pgfsetbuttcap%
\pgfsetroundjoin%
\definecolor{currentfill}{rgb}{1.000000,0.705882,0.509804}%
\pgfsetfillcolor{currentfill}%
\pgfsetlinewidth{0.481800pt}%
\definecolor{currentstroke}{rgb}{1.000000,1.000000,1.000000}%
\pgfsetstrokecolor{currentstroke}%
\pgfsetdash{}{0pt}%
\pgfpathmoveto{\pgfqpoint{2.003717in}{3.772999in}}%
\pgfpathcurveto{\pgfqpoint{2.014767in}{3.772999in}}{\pgfqpoint{2.025367in}{3.777389in}}{\pgfqpoint{2.033180in}{3.785202in}}%
\pgfpathcurveto{\pgfqpoint{2.040994in}{3.793016in}}{\pgfqpoint{2.045384in}{3.803615in}}{\pgfqpoint{2.045384in}{3.814665in}}%
\pgfpathcurveto{\pgfqpoint{2.045384in}{3.825715in}}{\pgfqpoint{2.040994in}{3.836314in}}{\pgfqpoint{2.033180in}{3.844128in}}%
\pgfpathcurveto{\pgfqpoint{2.025367in}{3.851942in}}{\pgfqpoint{2.014767in}{3.856332in}}{\pgfqpoint{2.003717in}{3.856332in}}%
\pgfpathcurveto{\pgfqpoint{1.992667in}{3.856332in}}{\pgfqpoint{1.982068in}{3.851942in}}{\pgfqpoint{1.974255in}{3.844128in}}%
\pgfpathcurveto{\pgfqpoint{1.966441in}{3.836314in}}{\pgfqpoint{1.962051in}{3.825715in}}{\pgfqpoint{1.962051in}{3.814665in}}%
\pgfpathcurveto{\pgfqpoint{1.962051in}{3.803615in}}{\pgfqpoint{1.966441in}{3.793016in}}{\pgfqpoint{1.974255in}{3.785202in}}%
\pgfpathcurveto{\pgfqpoint{1.982068in}{3.777389in}}{\pgfqpoint{1.992667in}{3.772999in}}{\pgfqpoint{2.003717in}{3.772999in}}%
\pgfpathclose%
\pgfusepath{stroke,fill}%
\end{pgfscope}%
\begin{pgfscope}%
\pgfpathrectangle{\pgfqpoint{0.481978in}{0.331635in}}{\pgfqpoint{9.300000in}{7.700000in}}%
\pgfusepath{clip}%
\pgfsetbuttcap%
\pgfsetroundjoin%
\definecolor{currentfill}{rgb}{1.000000,0.705882,0.509804}%
\pgfsetfillcolor{currentfill}%
\pgfsetlinewidth{0.481800pt}%
\definecolor{currentstroke}{rgb}{1.000000,1.000000,1.000000}%
\pgfsetstrokecolor{currentstroke}%
\pgfsetdash{}{0pt}%
\pgfpathmoveto{\pgfqpoint{2.821317in}{3.686823in}}%
\pgfpathcurveto{\pgfqpoint{2.832367in}{3.686823in}}{\pgfqpoint{2.842966in}{3.691213in}}{\pgfqpoint{2.850780in}{3.699026in}}%
\pgfpathcurveto{\pgfqpoint{2.858593in}{3.706840in}}{\pgfqpoint{2.862983in}{3.717439in}}{\pgfqpoint{2.862983in}{3.728489in}}%
\pgfpathcurveto{\pgfqpoint{2.862983in}{3.739539in}}{\pgfqpoint{2.858593in}{3.750138in}}{\pgfqpoint{2.850780in}{3.757952in}}%
\pgfpathcurveto{\pgfqpoint{2.842966in}{3.765766in}}{\pgfqpoint{2.832367in}{3.770156in}}{\pgfqpoint{2.821317in}{3.770156in}}%
\pgfpathcurveto{\pgfqpoint{2.810267in}{3.770156in}}{\pgfqpoint{2.799668in}{3.765766in}}{\pgfqpoint{2.791854in}{3.757952in}}%
\pgfpathcurveto{\pgfqpoint{2.784040in}{3.750138in}}{\pgfqpoint{2.779650in}{3.739539in}}{\pgfqpoint{2.779650in}{3.728489in}}%
\pgfpathcurveto{\pgfqpoint{2.779650in}{3.717439in}}{\pgfqpoint{2.784040in}{3.706840in}}{\pgfqpoint{2.791854in}{3.699026in}}%
\pgfpathcurveto{\pgfqpoint{2.799668in}{3.691213in}}{\pgfqpoint{2.810267in}{3.686823in}}{\pgfqpoint{2.821317in}{3.686823in}}%
\pgfpathclose%
\pgfusepath{stroke,fill}%
\end{pgfscope}%
\begin{pgfscope}%
\pgfpathrectangle{\pgfqpoint{0.481978in}{0.331635in}}{\pgfqpoint{9.300000in}{7.700000in}}%
\pgfusepath{clip}%
\pgfsetbuttcap%
\pgfsetroundjoin%
\definecolor{currentfill}{rgb}{1.000000,0.705882,0.509804}%
\pgfsetfillcolor{currentfill}%
\pgfsetlinewidth{0.481800pt}%
\definecolor{currentstroke}{rgb}{1.000000,1.000000,1.000000}%
\pgfsetstrokecolor{currentstroke}%
\pgfsetdash{}{0pt}%
\pgfpathmoveto{\pgfqpoint{2.335285in}{3.839174in}}%
\pgfpathcurveto{\pgfqpoint{2.346335in}{3.839174in}}{\pgfqpoint{2.356934in}{3.843564in}}{\pgfqpoint{2.364748in}{3.851378in}}%
\pgfpathcurveto{\pgfqpoint{2.372561in}{3.859192in}}{\pgfqpoint{2.376952in}{3.869791in}}{\pgfqpoint{2.376952in}{3.880841in}}%
\pgfpathcurveto{\pgfqpoint{2.376952in}{3.891891in}}{\pgfqpoint{2.372561in}{3.902490in}}{\pgfqpoint{2.364748in}{3.910304in}}%
\pgfpathcurveto{\pgfqpoint{2.356934in}{3.918117in}}{\pgfqpoint{2.346335in}{3.922507in}}{\pgfqpoint{2.335285in}{3.922507in}}%
\pgfpathcurveto{\pgfqpoint{2.324235in}{3.922507in}}{\pgfqpoint{2.313636in}{3.918117in}}{\pgfqpoint{2.305822in}{3.910304in}}%
\pgfpathcurveto{\pgfqpoint{2.298009in}{3.902490in}}{\pgfqpoint{2.293618in}{3.891891in}}{\pgfqpoint{2.293618in}{3.880841in}}%
\pgfpathcurveto{\pgfqpoint{2.293618in}{3.869791in}}{\pgfqpoint{2.298009in}{3.859192in}}{\pgfqpoint{2.305822in}{3.851378in}}%
\pgfpathcurveto{\pgfqpoint{2.313636in}{3.843564in}}{\pgfqpoint{2.324235in}{3.839174in}}{\pgfqpoint{2.335285in}{3.839174in}}%
\pgfpathclose%
\pgfusepath{stroke,fill}%
\end{pgfscope}%
\begin{pgfscope}%
\pgfpathrectangle{\pgfqpoint{0.481978in}{0.331635in}}{\pgfqpoint{9.300000in}{7.700000in}}%
\pgfusepath{clip}%
\pgfsetbuttcap%
\pgfsetroundjoin%
\definecolor{currentfill}{rgb}{1.000000,0.705882,0.509804}%
\pgfsetfillcolor{currentfill}%
\pgfsetlinewidth{0.481800pt}%
\definecolor{currentstroke}{rgb}{1.000000,1.000000,1.000000}%
\pgfsetstrokecolor{currentstroke}%
\pgfsetdash{}{0pt}%
\pgfpathmoveto{\pgfqpoint{4.485108in}{3.824317in}}%
\pgfpathcurveto{\pgfqpoint{4.496158in}{3.824317in}}{\pgfqpoint{4.506757in}{3.828707in}}{\pgfqpoint{4.514571in}{3.836521in}}%
\pgfpathcurveto{\pgfqpoint{4.522385in}{3.844334in}}{\pgfqpoint{4.526775in}{3.854933in}}{\pgfqpoint{4.526775in}{3.865983in}}%
\pgfpathcurveto{\pgfqpoint{4.526775in}{3.877033in}}{\pgfqpoint{4.522385in}{3.887632in}}{\pgfqpoint{4.514571in}{3.895446in}}%
\pgfpathcurveto{\pgfqpoint{4.506757in}{3.903260in}}{\pgfqpoint{4.496158in}{3.907650in}}{\pgfqpoint{4.485108in}{3.907650in}}%
\pgfpathcurveto{\pgfqpoint{4.474058in}{3.907650in}}{\pgfqpoint{4.463459in}{3.903260in}}{\pgfqpoint{4.455645in}{3.895446in}}%
\pgfpathcurveto{\pgfqpoint{4.447832in}{3.887632in}}{\pgfqpoint{4.443442in}{3.877033in}}{\pgfqpoint{4.443442in}{3.865983in}}%
\pgfpathcurveto{\pgfqpoint{4.443442in}{3.854933in}}{\pgfqpoint{4.447832in}{3.844334in}}{\pgfqpoint{4.455645in}{3.836521in}}%
\pgfpathcurveto{\pgfqpoint{4.463459in}{3.828707in}}{\pgfqpoint{4.474058in}{3.824317in}}{\pgfqpoint{4.485108in}{3.824317in}}%
\pgfpathclose%
\pgfusepath{stroke,fill}%
\end{pgfscope}%
\begin{pgfscope}%
\pgfpathrectangle{\pgfqpoint{0.481978in}{0.331635in}}{\pgfqpoint{9.300000in}{7.700000in}}%
\pgfusepath{clip}%
\pgfsetbuttcap%
\pgfsetroundjoin%
\definecolor{currentfill}{rgb}{1.000000,0.705882,0.509804}%
\pgfsetfillcolor{currentfill}%
\pgfsetlinewidth{0.481800pt}%
\definecolor{currentstroke}{rgb}{1.000000,1.000000,1.000000}%
\pgfsetstrokecolor{currentstroke}%
\pgfsetdash{}{0pt}%
\pgfpathmoveto{\pgfqpoint{1.703134in}{4.248095in}}%
\pgfpathcurveto{\pgfqpoint{1.714184in}{4.248095in}}{\pgfqpoint{1.724783in}{4.252485in}}{\pgfqpoint{1.732597in}{4.260299in}}%
\pgfpathcurveto{\pgfqpoint{1.740411in}{4.268112in}}{\pgfqpoint{1.744801in}{4.278711in}}{\pgfqpoint{1.744801in}{4.289761in}}%
\pgfpathcurveto{\pgfqpoint{1.744801in}{4.300812in}}{\pgfqpoint{1.740411in}{4.311411in}}{\pgfqpoint{1.732597in}{4.319224in}}%
\pgfpathcurveto{\pgfqpoint{1.724783in}{4.327038in}}{\pgfqpoint{1.714184in}{4.331428in}}{\pgfqpoint{1.703134in}{4.331428in}}%
\pgfpathcurveto{\pgfqpoint{1.692084in}{4.331428in}}{\pgfqpoint{1.681485in}{4.327038in}}{\pgfqpoint{1.673671in}{4.319224in}}%
\pgfpathcurveto{\pgfqpoint{1.665858in}{4.311411in}}{\pgfqpoint{1.661467in}{4.300812in}}{\pgfqpoint{1.661467in}{4.289761in}}%
\pgfpathcurveto{\pgfqpoint{1.661467in}{4.278711in}}{\pgfqpoint{1.665858in}{4.268112in}}{\pgfqpoint{1.673671in}{4.260299in}}%
\pgfpathcurveto{\pgfqpoint{1.681485in}{4.252485in}}{\pgfqpoint{1.692084in}{4.248095in}}{\pgfqpoint{1.703134in}{4.248095in}}%
\pgfpathclose%
\pgfusepath{stroke,fill}%
\end{pgfscope}%
\begin{pgfscope}%
\pgfpathrectangle{\pgfqpoint{0.481978in}{0.331635in}}{\pgfqpoint{9.300000in}{7.700000in}}%
\pgfusepath{clip}%
\pgfsetbuttcap%
\pgfsetroundjoin%
\definecolor{currentfill}{rgb}{1.000000,0.705882,0.509804}%
\pgfsetfillcolor{currentfill}%
\pgfsetlinewidth{0.481800pt}%
\definecolor{currentstroke}{rgb}{1.000000,1.000000,1.000000}%
\pgfsetstrokecolor{currentstroke}%
\pgfsetdash{}{0pt}%
\pgfpathmoveto{\pgfqpoint{4.082596in}{4.738956in}}%
\pgfpathcurveto{\pgfqpoint{4.093646in}{4.738956in}}{\pgfqpoint{4.104245in}{4.743347in}}{\pgfqpoint{4.112059in}{4.751160in}}%
\pgfpathcurveto{\pgfqpoint{4.119873in}{4.758974in}}{\pgfqpoint{4.124263in}{4.769573in}}{\pgfqpoint{4.124263in}{4.780623in}}%
\pgfpathcurveto{\pgfqpoint{4.124263in}{4.791673in}}{\pgfqpoint{4.119873in}{4.802272in}}{\pgfqpoint{4.112059in}{4.810086in}}%
\pgfpathcurveto{\pgfqpoint{4.104245in}{4.817899in}}{\pgfqpoint{4.093646in}{4.822290in}}{\pgfqpoint{4.082596in}{4.822290in}}%
\pgfpathcurveto{\pgfqpoint{4.071546in}{4.822290in}}{\pgfqpoint{4.060947in}{4.817899in}}{\pgfqpoint{4.053133in}{4.810086in}}%
\pgfpathcurveto{\pgfqpoint{4.045320in}{4.802272in}}{\pgfqpoint{4.040929in}{4.791673in}}{\pgfqpoint{4.040929in}{4.780623in}}%
\pgfpathcurveto{\pgfqpoint{4.040929in}{4.769573in}}{\pgfqpoint{4.045320in}{4.758974in}}{\pgfqpoint{4.053133in}{4.751160in}}%
\pgfpathcurveto{\pgfqpoint{4.060947in}{4.743347in}}{\pgfqpoint{4.071546in}{4.738956in}}{\pgfqpoint{4.082596in}{4.738956in}}%
\pgfpathclose%
\pgfusepath{stroke,fill}%
\end{pgfscope}%
\begin{pgfscope}%
\pgfpathrectangle{\pgfqpoint{0.481978in}{0.331635in}}{\pgfqpoint{9.300000in}{7.700000in}}%
\pgfusepath{clip}%
\pgfsetbuttcap%
\pgfsetroundjoin%
\definecolor{currentfill}{rgb}{1.000000,0.705882,0.509804}%
\pgfsetfillcolor{currentfill}%
\pgfsetlinewidth{0.481800pt}%
\definecolor{currentstroke}{rgb}{1.000000,1.000000,1.000000}%
\pgfsetstrokecolor{currentstroke}%
\pgfsetdash{}{0pt}%
\pgfpathmoveto{\pgfqpoint{3.376694in}{3.788825in}}%
\pgfpathcurveto{\pgfqpoint{3.387744in}{3.788825in}}{\pgfqpoint{3.398343in}{3.793215in}}{\pgfqpoint{3.406156in}{3.801029in}}%
\pgfpathcurveto{\pgfqpoint{3.413970in}{3.808843in}}{\pgfqpoint{3.418360in}{3.819442in}}{\pgfqpoint{3.418360in}{3.830492in}}%
\pgfpathcurveto{\pgfqpoint{3.418360in}{3.841542in}}{\pgfqpoint{3.413970in}{3.852141in}}{\pgfqpoint{3.406156in}{3.859955in}}%
\pgfpathcurveto{\pgfqpoint{3.398343in}{3.867768in}}{\pgfqpoint{3.387744in}{3.872158in}}{\pgfqpoint{3.376694in}{3.872158in}}%
\pgfpathcurveto{\pgfqpoint{3.365644in}{3.872158in}}{\pgfqpoint{3.355044in}{3.867768in}}{\pgfqpoint{3.347231in}{3.859955in}}%
\pgfpathcurveto{\pgfqpoint{3.339417in}{3.852141in}}{\pgfqpoint{3.335027in}{3.841542in}}{\pgfqpoint{3.335027in}{3.830492in}}%
\pgfpathcurveto{\pgfqpoint{3.335027in}{3.819442in}}{\pgfqpoint{3.339417in}{3.808843in}}{\pgfqpoint{3.347231in}{3.801029in}}%
\pgfpathcurveto{\pgfqpoint{3.355044in}{3.793215in}}{\pgfqpoint{3.365644in}{3.788825in}}{\pgfqpoint{3.376694in}{3.788825in}}%
\pgfpathclose%
\pgfusepath{stroke,fill}%
\end{pgfscope}%
\begin{pgfscope}%
\pgfpathrectangle{\pgfqpoint{0.481978in}{0.331635in}}{\pgfqpoint{9.300000in}{7.700000in}}%
\pgfusepath{clip}%
\pgfsetbuttcap%
\pgfsetroundjoin%
\definecolor{currentfill}{rgb}{1.000000,0.705882,0.509804}%
\pgfsetfillcolor{currentfill}%
\pgfsetlinewidth{0.481800pt}%
\definecolor{currentstroke}{rgb}{1.000000,1.000000,1.000000}%
\pgfsetstrokecolor{currentstroke}%
\pgfsetdash{}{0pt}%
\pgfpathmoveto{\pgfqpoint{4.094855in}{5.240576in}}%
\pgfpathcurveto{\pgfqpoint{4.105906in}{5.240576in}}{\pgfqpoint{4.116505in}{5.244966in}}{\pgfqpoint{4.124318in}{5.252780in}}%
\pgfpathcurveto{\pgfqpoint{4.132132in}{5.260594in}}{\pgfqpoint{4.136522in}{5.271193in}}{\pgfqpoint{4.136522in}{5.282243in}}%
\pgfpathcurveto{\pgfqpoint{4.136522in}{5.293293in}}{\pgfqpoint{4.132132in}{5.303892in}}{\pgfqpoint{4.124318in}{5.311705in}}%
\pgfpathcurveto{\pgfqpoint{4.116505in}{5.319519in}}{\pgfqpoint{4.105906in}{5.323909in}}{\pgfqpoint{4.094855in}{5.323909in}}%
\pgfpathcurveto{\pgfqpoint{4.083805in}{5.323909in}}{\pgfqpoint{4.073206in}{5.319519in}}{\pgfqpoint{4.065393in}{5.311705in}}%
\pgfpathcurveto{\pgfqpoint{4.057579in}{5.303892in}}{\pgfqpoint{4.053189in}{5.293293in}}{\pgfqpoint{4.053189in}{5.282243in}}%
\pgfpathcurveto{\pgfqpoint{4.053189in}{5.271193in}}{\pgfqpoint{4.057579in}{5.260594in}}{\pgfqpoint{4.065393in}{5.252780in}}%
\pgfpathcurveto{\pgfqpoint{4.073206in}{5.244966in}}{\pgfqpoint{4.083805in}{5.240576in}}{\pgfqpoint{4.094855in}{5.240576in}}%
\pgfpathclose%
\pgfusepath{stroke,fill}%
\end{pgfscope}%
\begin{pgfscope}%
\pgfpathrectangle{\pgfqpoint{0.481978in}{0.331635in}}{\pgfqpoint{9.300000in}{7.700000in}}%
\pgfusepath{clip}%
\pgfsetbuttcap%
\pgfsetroundjoin%
\definecolor{currentfill}{rgb}{1.000000,0.705882,0.509804}%
\pgfsetfillcolor{currentfill}%
\pgfsetlinewidth{0.481800pt}%
\definecolor{currentstroke}{rgb}{1.000000,1.000000,1.000000}%
\pgfsetstrokecolor{currentstroke}%
\pgfsetdash{}{0pt}%
\pgfpathmoveto{\pgfqpoint{4.055791in}{5.307232in}}%
\pgfpathcurveto{\pgfqpoint{4.066841in}{5.307232in}}{\pgfqpoint{4.077440in}{5.311623in}}{\pgfqpoint{4.085254in}{5.319436in}}%
\pgfpathcurveto{\pgfqpoint{4.093068in}{5.327250in}}{\pgfqpoint{4.097458in}{5.337849in}}{\pgfqpoint{4.097458in}{5.348899in}}%
\pgfpathcurveto{\pgfqpoint{4.097458in}{5.359949in}}{\pgfqpoint{4.093068in}{5.370548in}}{\pgfqpoint{4.085254in}{5.378362in}}%
\pgfpathcurveto{\pgfqpoint{4.077440in}{5.386175in}}{\pgfqpoint{4.066841in}{5.390566in}}{\pgfqpoint{4.055791in}{5.390566in}}%
\pgfpathcurveto{\pgfqpoint{4.044741in}{5.390566in}}{\pgfqpoint{4.034142in}{5.386175in}}{\pgfqpoint{4.026328in}{5.378362in}}%
\pgfpathcurveto{\pgfqpoint{4.018515in}{5.370548in}}{\pgfqpoint{4.014125in}{5.359949in}}{\pgfqpoint{4.014125in}{5.348899in}}%
\pgfpathcurveto{\pgfqpoint{4.014125in}{5.337849in}}{\pgfqpoint{4.018515in}{5.327250in}}{\pgfqpoint{4.026328in}{5.319436in}}%
\pgfpathcurveto{\pgfqpoint{4.034142in}{5.311623in}}{\pgfqpoint{4.044741in}{5.307232in}}{\pgfqpoint{4.055791in}{5.307232in}}%
\pgfpathclose%
\pgfusepath{stroke,fill}%
\end{pgfscope}%
\begin{pgfscope}%
\pgfpathrectangle{\pgfqpoint{0.481978in}{0.331635in}}{\pgfqpoint{9.300000in}{7.700000in}}%
\pgfusepath{clip}%
\pgfsetbuttcap%
\pgfsetroundjoin%
\definecolor{currentfill}{rgb}{1.000000,0.705882,0.509804}%
\pgfsetfillcolor{currentfill}%
\pgfsetlinewidth{0.481800pt}%
\definecolor{currentstroke}{rgb}{1.000000,1.000000,1.000000}%
\pgfsetstrokecolor{currentstroke}%
\pgfsetdash{}{0pt}%
\pgfpathmoveto{\pgfqpoint{4.975557in}{3.343781in}}%
\pgfpathcurveto{\pgfqpoint{4.986607in}{3.343781in}}{\pgfqpoint{4.997206in}{3.348171in}}{\pgfqpoint{5.005020in}{3.355985in}}%
\pgfpathcurveto{\pgfqpoint{5.012833in}{3.363799in}}{\pgfqpoint{5.017223in}{3.374398in}}{\pgfqpoint{5.017223in}{3.385448in}}%
\pgfpathcurveto{\pgfqpoint{5.017223in}{3.396498in}}{\pgfqpoint{5.012833in}{3.407097in}}{\pgfqpoint{5.005020in}{3.414910in}}%
\pgfpathcurveto{\pgfqpoint{4.997206in}{3.422724in}}{\pgfqpoint{4.986607in}{3.427114in}}{\pgfqpoint{4.975557in}{3.427114in}}%
\pgfpathcurveto{\pgfqpoint{4.964507in}{3.427114in}}{\pgfqpoint{4.953908in}{3.422724in}}{\pgfqpoint{4.946094in}{3.414910in}}%
\pgfpathcurveto{\pgfqpoint{4.938280in}{3.407097in}}{\pgfqpoint{4.933890in}{3.396498in}}{\pgfqpoint{4.933890in}{3.385448in}}%
\pgfpathcurveto{\pgfqpoint{4.933890in}{3.374398in}}{\pgfqpoint{4.938280in}{3.363799in}}{\pgfqpoint{4.946094in}{3.355985in}}%
\pgfpathcurveto{\pgfqpoint{4.953908in}{3.348171in}}{\pgfqpoint{4.964507in}{3.343781in}}{\pgfqpoint{4.975557in}{3.343781in}}%
\pgfpathclose%
\pgfusepath{stroke,fill}%
\end{pgfscope}%
\begin{pgfscope}%
\pgfpathrectangle{\pgfqpoint{0.481978in}{0.331635in}}{\pgfqpoint{9.300000in}{7.700000in}}%
\pgfusepath{clip}%
\pgfsetbuttcap%
\pgfsetroundjoin%
\definecolor{currentfill}{rgb}{1.000000,0.705882,0.509804}%
\pgfsetfillcolor{currentfill}%
\pgfsetlinewidth{0.481800pt}%
\definecolor{currentstroke}{rgb}{1.000000,1.000000,1.000000}%
\pgfsetstrokecolor{currentstroke}%
\pgfsetdash{}{0pt}%
\pgfpathmoveto{\pgfqpoint{2.986906in}{3.677358in}}%
\pgfpathcurveto{\pgfqpoint{2.997956in}{3.677358in}}{\pgfqpoint{3.008555in}{3.681748in}}{\pgfqpoint{3.016369in}{3.689562in}}%
\pgfpathcurveto{\pgfqpoint{3.024183in}{3.697376in}}{\pgfqpoint{3.028573in}{3.707975in}}{\pgfqpoint{3.028573in}{3.719025in}}%
\pgfpathcurveto{\pgfqpoint{3.028573in}{3.730075in}}{\pgfqpoint{3.024183in}{3.740674in}}{\pgfqpoint{3.016369in}{3.748488in}}%
\pgfpathcurveto{\pgfqpoint{3.008555in}{3.756301in}}{\pgfqpoint{2.997956in}{3.760692in}}{\pgfqpoint{2.986906in}{3.760692in}}%
\pgfpathcurveto{\pgfqpoint{2.975856in}{3.760692in}}{\pgfqpoint{2.965257in}{3.756301in}}{\pgfqpoint{2.957444in}{3.748488in}}%
\pgfpathcurveto{\pgfqpoint{2.949630in}{3.740674in}}{\pgfqpoint{2.945240in}{3.730075in}}{\pgfqpoint{2.945240in}{3.719025in}}%
\pgfpathcurveto{\pgfqpoint{2.945240in}{3.707975in}}{\pgfqpoint{2.949630in}{3.697376in}}{\pgfqpoint{2.957444in}{3.689562in}}%
\pgfpathcurveto{\pgfqpoint{2.965257in}{3.681748in}}{\pgfqpoint{2.975856in}{3.677358in}}{\pgfqpoint{2.986906in}{3.677358in}}%
\pgfpathclose%
\pgfusepath{stroke,fill}%
\end{pgfscope}%
\begin{pgfscope}%
\pgfpathrectangle{\pgfqpoint{0.481978in}{0.331635in}}{\pgfqpoint{9.300000in}{7.700000in}}%
\pgfusepath{clip}%
\pgfsetbuttcap%
\pgfsetroundjoin%
\definecolor{currentfill}{rgb}{1.000000,0.705882,0.509804}%
\pgfsetfillcolor{currentfill}%
\pgfsetlinewidth{0.481800pt}%
\definecolor{currentstroke}{rgb}{1.000000,1.000000,1.000000}%
\pgfsetstrokecolor{currentstroke}%
\pgfsetdash{}{0pt}%
\pgfpathmoveto{\pgfqpoint{3.977136in}{2.674234in}}%
\pgfpathcurveto{\pgfqpoint{3.988186in}{2.674234in}}{\pgfqpoint{3.998785in}{2.678625in}}{\pgfqpoint{4.006599in}{2.686438in}}%
\pgfpathcurveto{\pgfqpoint{4.014412in}{2.694252in}}{\pgfqpoint{4.018802in}{2.704851in}}{\pgfqpoint{4.018802in}{2.715901in}}%
\pgfpathcurveto{\pgfqpoint{4.018802in}{2.726951in}}{\pgfqpoint{4.014412in}{2.737550in}}{\pgfqpoint{4.006599in}{2.745364in}}%
\pgfpathcurveto{\pgfqpoint{3.998785in}{2.753178in}}{\pgfqpoint{3.988186in}{2.757568in}}{\pgfqpoint{3.977136in}{2.757568in}}%
\pgfpathcurveto{\pgfqpoint{3.966086in}{2.757568in}}{\pgfqpoint{3.955487in}{2.753178in}}{\pgfqpoint{3.947673in}{2.745364in}}%
\pgfpathcurveto{\pgfqpoint{3.939859in}{2.737550in}}{\pgfqpoint{3.935469in}{2.726951in}}{\pgfqpoint{3.935469in}{2.715901in}}%
\pgfpathcurveto{\pgfqpoint{3.935469in}{2.704851in}}{\pgfqpoint{3.939859in}{2.694252in}}{\pgfqpoint{3.947673in}{2.686438in}}%
\pgfpathcurveto{\pgfqpoint{3.955487in}{2.678625in}}{\pgfqpoint{3.966086in}{2.674234in}}{\pgfqpoint{3.977136in}{2.674234in}}%
\pgfpathclose%
\pgfusepath{stroke,fill}%
\end{pgfscope}%
\begin{pgfscope}%
\pgfpathrectangle{\pgfqpoint{0.481978in}{0.331635in}}{\pgfqpoint{9.300000in}{7.700000in}}%
\pgfusepath{clip}%
\pgfsetbuttcap%
\pgfsetroundjoin%
\definecolor{currentfill}{rgb}{1.000000,0.705882,0.509804}%
\pgfsetfillcolor{currentfill}%
\pgfsetlinewidth{0.481800pt}%
\definecolor{currentstroke}{rgb}{1.000000,1.000000,1.000000}%
\pgfsetstrokecolor{currentstroke}%
\pgfsetdash{}{0pt}%
\pgfpathmoveto{\pgfqpoint{3.838833in}{6.919703in}}%
\pgfpathcurveto{\pgfqpoint{3.849883in}{6.919703in}}{\pgfqpoint{3.860482in}{6.924093in}}{\pgfqpoint{3.868296in}{6.931906in}}%
\pgfpathcurveto{\pgfqpoint{3.876109in}{6.939720in}}{\pgfqpoint{3.880500in}{6.950319in}}{\pgfqpoint{3.880500in}{6.961369in}}%
\pgfpathcurveto{\pgfqpoint{3.880500in}{6.972419in}}{\pgfqpoint{3.876109in}{6.983018in}}{\pgfqpoint{3.868296in}{6.990832in}}%
\pgfpathcurveto{\pgfqpoint{3.860482in}{6.998646in}}{\pgfqpoint{3.849883in}{7.003036in}}{\pgfqpoint{3.838833in}{7.003036in}}%
\pgfpathcurveto{\pgfqpoint{3.827783in}{7.003036in}}{\pgfqpoint{3.817184in}{6.998646in}}{\pgfqpoint{3.809370in}{6.990832in}}%
\pgfpathcurveto{\pgfqpoint{3.801557in}{6.983018in}}{\pgfqpoint{3.797166in}{6.972419in}}{\pgfqpoint{3.797166in}{6.961369in}}%
\pgfpathcurveto{\pgfqpoint{3.797166in}{6.950319in}}{\pgfqpoint{3.801557in}{6.939720in}}{\pgfqpoint{3.809370in}{6.931906in}}%
\pgfpathcurveto{\pgfqpoint{3.817184in}{6.924093in}}{\pgfqpoint{3.827783in}{6.919703in}}{\pgfqpoint{3.838833in}{6.919703in}}%
\pgfpathclose%
\pgfusepath{stroke,fill}%
\end{pgfscope}%
\begin{pgfscope}%
\pgfpathrectangle{\pgfqpoint{0.481978in}{0.331635in}}{\pgfqpoint{9.300000in}{7.700000in}}%
\pgfusepath{clip}%
\pgfsetbuttcap%
\pgfsetroundjoin%
\definecolor{currentfill}{rgb}{1.000000,0.705882,0.509804}%
\pgfsetfillcolor{currentfill}%
\pgfsetlinewidth{0.481800pt}%
\definecolor{currentstroke}{rgb}{1.000000,1.000000,1.000000}%
\pgfsetstrokecolor{currentstroke}%
\pgfsetdash{}{0pt}%
\pgfpathmoveto{\pgfqpoint{2.868567in}{3.772096in}}%
\pgfpathcurveto{\pgfqpoint{2.879617in}{3.772096in}}{\pgfqpoint{2.890216in}{3.776487in}}{\pgfqpoint{2.898030in}{3.784300in}}%
\pgfpathcurveto{\pgfqpoint{2.905843in}{3.792114in}}{\pgfqpoint{2.910234in}{3.802713in}}{\pgfqpoint{2.910234in}{3.813763in}}%
\pgfpathcurveto{\pgfqpoint{2.910234in}{3.824813in}}{\pgfqpoint{2.905843in}{3.835412in}}{\pgfqpoint{2.898030in}{3.843226in}}%
\pgfpathcurveto{\pgfqpoint{2.890216in}{3.851040in}}{\pgfqpoint{2.879617in}{3.855430in}}{\pgfqpoint{2.868567in}{3.855430in}}%
\pgfpathcurveto{\pgfqpoint{2.857517in}{3.855430in}}{\pgfqpoint{2.846918in}{3.851040in}}{\pgfqpoint{2.839104in}{3.843226in}}%
\pgfpathcurveto{\pgfqpoint{2.831290in}{3.835412in}}{\pgfqpoint{2.826900in}{3.824813in}}{\pgfqpoint{2.826900in}{3.813763in}}%
\pgfpathcurveto{\pgfqpoint{2.826900in}{3.802713in}}{\pgfqpoint{2.831290in}{3.792114in}}{\pgfqpoint{2.839104in}{3.784300in}}%
\pgfpathcurveto{\pgfqpoint{2.846918in}{3.776487in}}{\pgfqpoint{2.857517in}{3.772096in}}{\pgfqpoint{2.868567in}{3.772096in}}%
\pgfpathclose%
\pgfusepath{stroke,fill}%
\end{pgfscope}%
\begin{pgfscope}%
\pgfpathrectangle{\pgfqpoint{0.481978in}{0.331635in}}{\pgfqpoint{9.300000in}{7.700000in}}%
\pgfusepath{clip}%
\pgfsetbuttcap%
\pgfsetroundjoin%
\definecolor{currentfill}{rgb}{1.000000,0.705882,0.509804}%
\pgfsetfillcolor{currentfill}%
\pgfsetlinewidth{0.481800pt}%
\definecolor{currentstroke}{rgb}{1.000000,1.000000,1.000000}%
\pgfsetstrokecolor{currentstroke}%
\pgfsetdash{}{0pt}%
\pgfpathmoveto{\pgfqpoint{5.264259in}{4.891797in}}%
\pgfpathcurveto{\pgfqpoint{5.275309in}{4.891797in}}{\pgfqpoint{5.285908in}{4.896188in}}{\pgfqpoint{5.293722in}{4.904001in}}%
\pgfpathcurveto{\pgfqpoint{5.301536in}{4.911815in}}{\pgfqpoint{5.305926in}{4.922414in}}{\pgfqpoint{5.305926in}{4.933464in}}%
\pgfpathcurveto{\pgfqpoint{5.305926in}{4.944514in}}{\pgfqpoint{5.301536in}{4.955113in}}{\pgfqpoint{5.293722in}{4.962927in}}%
\pgfpathcurveto{\pgfqpoint{5.285908in}{4.970740in}}{\pgfqpoint{5.275309in}{4.975131in}}{\pgfqpoint{5.264259in}{4.975131in}}%
\pgfpathcurveto{\pgfqpoint{5.253209in}{4.975131in}}{\pgfqpoint{5.242610in}{4.970740in}}{\pgfqpoint{5.234796in}{4.962927in}}%
\pgfpathcurveto{\pgfqpoint{5.226983in}{4.955113in}}{\pgfqpoint{5.222593in}{4.944514in}}{\pgfqpoint{5.222593in}{4.933464in}}%
\pgfpathcurveto{\pgfqpoint{5.222593in}{4.922414in}}{\pgfqpoint{5.226983in}{4.911815in}}{\pgfqpoint{5.234796in}{4.904001in}}%
\pgfpathcurveto{\pgfqpoint{5.242610in}{4.896188in}}{\pgfqpoint{5.253209in}{4.891797in}}{\pgfqpoint{5.264259in}{4.891797in}}%
\pgfpathclose%
\pgfusepath{stroke,fill}%
\end{pgfscope}%
\begin{pgfscope}%
\pgfpathrectangle{\pgfqpoint{0.481978in}{0.331635in}}{\pgfqpoint{9.300000in}{7.700000in}}%
\pgfusepath{clip}%
\pgfsetbuttcap%
\pgfsetroundjoin%
\definecolor{currentfill}{rgb}{1.000000,0.705882,0.509804}%
\pgfsetfillcolor{currentfill}%
\pgfsetlinewidth{0.481800pt}%
\definecolor{currentstroke}{rgb}{1.000000,1.000000,1.000000}%
\pgfsetstrokecolor{currentstroke}%
\pgfsetdash{}{0pt}%
\pgfpathmoveto{\pgfqpoint{3.655520in}{5.720597in}}%
\pgfpathcurveto{\pgfqpoint{3.666571in}{5.720597in}}{\pgfqpoint{3.677170in}{5.724987in}}{\pgfqpoint{3.684983in}{5.732801in}}%
\pgfpathcurveto{\pgfqpoint{3.692797in}{5.740614in}}{\pgfqpoint{3.697187in}{5.751213in}}{\pgfqpoint{3.697187in}{5.762264in}}%
\pgfpathcurveto{\pgfqpoint{3.697187in}{5.773314in}}{\pgfqpoint{3.692797in}{5.783913in}}{\pgfqpoint{3.684983in}{5.791726in}}%
\pgfpathcurveto{\pgfqpoint{3.677170in}{5.799540in}}{\pgfqpoint{3.666571in}{5.803930in}}{\pgfqpoint{3.655520in}{5.803930in}}%
\pgfpathcurveto{\pgfqpoint{3.644470in}{5.803930in}}{\pgfqpoint{3.633871in}{5.799540in}}{\pgfqpoint{3.626058in}{5.791726in}}%
\pgfpathcurveto{\pgfqpoint{3.618244in}{5.783913in}}{\pgfqpoint{3.613854in}{5.773314in}}{\pgfqpoint{3.613854in}{5.762264in}}%
\pgfpathcurveto{\pgfqpoint{3.613854in}{5.751213in}}{\pgfqpoint{3.618244in}{5.740614in}}{\pgfqpoint{3.626058in}{5.732801in}}%
\pgfpathcurveto{\pgfqpoint{3.633871in}{5.724987in}}{\pgfqpoint{3.644470in}{5.720597in}}{\pgfqpoint{3.655520in}{5.720597in}}%
\pgfpathclose%
\pgfusepath{stroke,fill}%
\end{pgfscope}%
\begin{pgfscope}%
\pgfpathrectangle{\pgfqpoint{0.481978in}{0.331635in}}{\pgfqpoint{9.300000in}{7.700000in}}%
\pgfusepath{clip}%
\pgfsetbuttcap%
\pgfsetroundjoin%
\definecolor{currentfill}{rgb}{1.000000,0.705882,0.509804}%
\pgfsetfillcolor{currentfill}%
\pgfsetlinewidth{0.481800pt}%
\definecolor{currentstroke}{rgb}{1.000000,1.000000,1.000000}%
\pgfsetstrokecolor{currentstroke}%
\pgfsetdash{}{0pt}%
\pgfpathmoveto{\pgfqpoint{1.251672in}{4.022336in}}%
\pgfpathcurveto{\pgfqpoint{1.262722in}{4.022336in}}{\pgfqpoint{1.273321in}{4.026727in}}{\pgfqpoint{1.281134in}{4.034540in}}%
\pgfpathcurveto{\pgfqpoint{1.288948in}{4.042354in}}{\pgfqpoint{1.293338in}{4.052953in}}{\pgfqpoint{1.293338in}{4.064003in}}%
\pgfpathcurveto{\pgfqpoint{1.293338in}{4.075053in}}{\pgfqpoint{1.288948in}{4.085652in}}{\pgfqpoint{1.281134in}{4.093466in}}%
\pgfpathcurveto{\pgfqpoint{1.273321in}{4.101279in}}{\pgfqpoint{1.262722in}{4.105670in}}{\pgfqpoint{1.251672in}{4.105670in}}%
\pgfpathcurveto{\pgfqpoint{1.240621in}{4.105670in}}{\pgfqpoint{1.230022in}{4.101279in}}{\pgfqpoint{1.222209in}{4.093466in}}%
\pgfpathcurveto{\pgfqpoint{1.214395in}{4.085652in}}{\pgfqpoint{1.210005in}{4.075053in}}{\pgfqpoint{1.210005in}{4.064003in}}%
\pgfpathcurveto{\pgfqpoint{1.210005in}{4.052953in}}{\pgfqpoint{1.214395in}{4.042354in}}{\pgfqpoint{1.222209in}{4.034540in}}%
\pgfpathcurveto{\pgfqpoint{1.230022in}{4.026727in}}{\pgfqpoint{1.240621in}{4.022336in}}{\pgfqpoint{1.251672in}{4.022336in}}%
\pgfpathclose%
\pgfusepath{stroke,fill}%
\end{pgfscope}%
\begin{pgfscope}%
\pgfpathrectangle{\pgfqpoint{0.481978in}{0.331635in}}{\pgfqpoint{9.300000in}{7.700000in}}%
\pgfusepath{clip}%
\pgfsetbuttcap%
\pgfsetroundjoin%
\definecolor{currentfill}{rgb}{1.000000,0.705882,0.509804}%
\pgfsetfillcolor{currentfill}%
\pgfsetlinewidth{0.481800pt}%
\definecolor{currentstroke}{rgb}{1.000000,1.000000,1.000000}%
\pgfsetstrokecolor{currentstroke}%
\pgfsetdash{}{0pt}%
\pgfpathmoveto{\pgfqpoint{1.875951in}{4.156394in}}%
\pgfpathcurveto{\pgfqpoint{1.887001in}{4.156394in}}{\pgfqpoint{1.897601in}{4.160785in}}{\pgfqpoint{1.905414in}{4.168598in}}%
\pgfpathcurveto{\pgfqpoint{1.913228in}{4.176412in}}{\pgfqpoint{1.917618in}{4.187011in}}{\pgfqpoint{1.917618in}{4.198061in}}%
\pgfpathcurveto{\pgfqpoint{1.917618in}{4.209111in}}{\pgfqpoint{1.913228in}{4.219710in}}{\pgfqpoint{1.905414in}{4.227524in}}%
\pgfpathcurveto{\pgfqpoint{1.897601in}{4.235337in}}{\pgfqpoint{1.887001in}{4.239728in}}{\pgfqpoint{1.875951in}{4.239728in}}%
\pgfpathcurveto{\pgfqpoint{1.864901in}{4.239728in}}{\pgfqpoint{1.854302in}{4.235337in}}{\pgfqpoint{1.846489in}{4.227524in}}%
\pgfpathcurveto{\pgfqpoint{1.838675in}{4.219710in}}{\pgfqpoint{1.834285in}{4.209111in}}{\pgfqpoint{1.834285in}{4.198061in}}%
\pgfpathcurveto{\pgfqpoint{1.834285in}{4.187011in}}{\pgfqpoint{1.838675in}{4.176412in}}{\pgfqpoint{1.846489in}{4.168598in}}%
\pgfpathcurveto{\pgfqpoint{1.854302in}{4.160785in}}{\pgfqpoint{1.864901in}{4.156394in}}{\pgfqpoint{1.875951in}{4.156394in}}%
\pgfpathclose%
\pgfusepath{stroke,fill}%
\end{pgfscope}%
\begin{pgfscope}%
\pgfpathrectangle{\pgfqpoint{0.481978in}{0.331635in}}{\pgfqpoint{9.300000in}{7.700000in}}%
\pgfusepath{clip}%
\pgfsetbuttcap%
\pgfsetroundjoin%
\definecolor{currentfill}{rgb}{1.000000,0.705882,0.509804}%
\pgfsetfillcolor{currentfill}%
\pgfsetlinewidth{0.481800pt}%
\definecolor{currentstroke}{rgb}{1.000000,1.000000,1.000000}%
\pgfsetstrokecolor{currentstroke}%
\pgfsetdash{}{0pt}%
\pgfpathmoveto{\pgfqpoint{2.496147in}{3.949319in}}%
\pgfpathcurveto{\pgfqpoint{2.507197in}{3.949319in}}{\pgfqpoint{2.517796in}{3.953709in}}{\pgfqpoint{2.525609in}{3.961522in}}%
\pgfpathcurveto{\pgfqpoint{2.533423in}{3.969336in}}{\pgfqpoint{2.537813in}{3.979935in}}{\pgfqpoint{2.537813in}{3.990985in}}%
\pgfpathcurveto{\pgfqpoint{2.537813in}{4.002035in}}{\pgfqpoint{2.533423in}{4.012634in}}{\pgfqpoint{2.525609in}{4.020448in}}%
\pgfpathcurveto{\pgfqpoint{2.517796in}{4.028262in}}{\pgfqpoint{2.507197in}{4.032652in}}{\pgfqpoint{2.496147in}{4.032652in}}%
\pgfpathcurveto{\pgfqpoint{2.485097in}{4.032652in}}{\pgfqpoint{2.474498in}{4.028262in}}{\pgfqpoint{2.466684in}{4.020448in}}%
\pgfpathcurveto{\pgfqpoint{2.458870in}{4.012634in}}{\pgfqpoint{2.454480in}{4.002035in}}{\pgfqpoint{2.454480in}{3.990985in}}%
\pgfpathcurveto{\pgfqpoint{2.454480in}{3.979935in}}{\pgfqpoint{2.458870in}{3.969336in}}{\pgfqpoint{2.466684in}{3.961522in}}%
\pgfpathcurveto{\pgfqpoint{2.474498in}{3.953709in}}{\pgfqpoint{2.485097in}{3.949319in}}{\pgfqpoint{2.496147in}{3.949319in}}%
\pgfpathclose%
\pgfusepath{stroke,fill}%
\end{pgfscope}%
\begin{pgfscope}%
\pgfpathrectangle{\pgfqpoint{0.481978in}{0.331635in}}{\pgfqpoint{9.300000in}{7.700000in}}%
\pgfusepath{clip}%
\pgfsetbuttcap%
\pgfsetroundjoin%
\definecolor{currentfill}{rgb}{1.000000,0.705882,0.509804}%
\pgfsetfillcolor{currentfill}%
\pgfsetlinewidth{0.481800pt}%
\definecolor{currentstroke}{rgb}{1.000000,1.000000,1.000000}%
\pgfsetstrokecolor{currentstroke}%
\pgfsetdash{}{0pt}%
\pgfpathmoveto{\pgfqpoint{3.452404in}{6.239137in}}%
\pgfpathcurveto{\pgfqpoint{3.463454in}{6.239137in}}{\pgfqpoint{3.474053in}{6.243527in}}{\pgfqpoint{3.481867in}{6.251341in}}%
\pgfpathcurveto{\pgfqpoint{3.489680in}{6.259155in}}{\pgfqpoint{3.494071in}{6.269754in}}{\pgfqpoint{3.494071in}{6.280804in}}%
\pgfpathcurveto{\pgfqpoint{3.494071in}{6.291854in}}{\pgfqpoint{3.489680in}{6.302453in}}{\pgfqpoint{3.481867in}{6.310267in}}%
\pgfpathcurveto{\pgfqpoint{3.474053in}{6.318080in}}{\pgfqpoint{3.463454in}{6.322470in}}{\pgfqpoint{3.452404in}{6.322470in}}%
\pgfpathcurveto{\pgfqpoint{3.441354in}{6.322470in}}{\pgfqpoint{3.430755in}{6.318080in}}{\pgfqpoint{3.422941in}{6.310267in}}%
\pgfpathcurveto{\pgfqpoint{3.415128in}{6.302453in}}{\pgfqpoint{3.410737in}{6.291854in}}{\pgfqpoint{3.410737in}{6.280804in}}%
\pgfpathcurveto{\pgfqpoint{3.410737in}{6.269754in}}{\pgfqpoint{3.415128in}{6.259155in}}{\pgfqpoint{3.422941in}{6.251341in}}%
\pgfpathcurveto{\pgfqpoint{3.430755in}{6.243527in}}{\pgfqpoint{3.441354in}{6.239137in}}{\pgfqpoint{3.452404in}{6.239137in}}%
\pgfpathclose%
\pgfusepath{stroke,fill}%
\end{pgfscope}%
\begin{pgfscope}%
\pgfpathrectangle{\pgfqpoint{0.481978in}{0.331635in}}{\pgfqpoint{9.300000in}{7.700000in}}%
\pgfusepath{clip}%
\pgfsetbuttcap%
\pgfsetroundjoin%
\definecolor{currentfill}{rgb}{1.000000,0.705882,0.509804}%
\pgfsetfillcolor{currentfill}%
\pgfsetlinewidth{0.481800pt}%
\definecolor{currentstroke}{rgb}{1.000000,1.000000,1.000000}%
\pgfsetstrokecolor{currentstroke}%
\pgfsetdash{}{0pt}%
\pgfpathmoveto{\pgfqpoint{1.475053in}{4.543874in}}%
\pgfpathcurveto{\pgfqpoint{1.486103in}{4.543874in}}{\pgfqpoint{1.496702in}{4.548264in}}{\pgfqpoint{1.504515in}{4.556078in}}%
\pgfpathcurveto{\pgfqpoint{1.512329in}{4.563891in}}{\pgfqpoint{1.516719in}{4.574490in}}{\pgfqpoint{1.516719in}{4.585541in}}%
\pgfpathcurveto{\pgfqpoint{1.516719in}{4.596591in}}{\pgfqpoint{1.512329in}{4.607190in}}{\pgfqpoint{1.504515in}{4.615003in}}%
\pgfpathcurveto{\pgfqpoint{1.496702in}{4.622817in}}{\pgfqpoint{1.486103in}{4.627207in}}{\pgfqpoint{1.475053in}{4.627207in}}%
\pgfpathcurveto{\pgfqpoint{1.464002in}{4.627207in}}{\pgfqpoint{1.453403in}{4.622817in}}{\pgfqpoint{1.445590in}{4.615003in}}%
\pgfpathcurveto{\pgfqpoint{1.437776in}{4.607190in}}{\pgfqpoint{1.433386in}{4.596591in}}{\pgfqpoint{1.433386in}{4.585541in}}%
\pgfpathcurveto{\pgfqpoint{1.433386in}{4.574490in}}{\pgfqpoint{1.437776in}{4.563891in}}{\pgfqpoint{1.445590in}{4.556078in}}%
\pgfpathcurveto{\pgfqpoint{1.453403in}{4.548264in}}{\pgfqpoint{1.464002in}{4.543874in}}{\pgfqpoint{1.475053in}{4.543874in}}%
\pgfpathclose%
\pgfusepath{stroke,fill}%
\end{pgfscope}%
\begin{pgfscope}%
\pgfpathrectangle{\pgfqpoint{0.481978in}{0.331635in}}{\pgfqpoint{9.300000in}{7.700000in}}%
\pgfusepath{clip}%
\pgfsetbuttcap%
\pgfsetroundjoin%
\definecolor{currentfill}{rgb}{1.000000,0.705882,0.509804}%
\pgfsetfillcolor{currentfill}%
\pgfsetlinewidth{0.481800pt}%
\definecolor{currentstroke}{rgb}{1.000000,1.000000,1.000000}%
\pgfsetstrokecolor{currentstroke}%
\pgfsetdash{}{0pt}%
\pgfpathmoveto{\pgfqpoint{2.980194in}{4.461586in}}%
\pgfpathcurveto{\pgfqpoint{2.991244in}{4.461586in}}{\pgfqpoint{3.001843in}{4.465976in}}{\pgfqpoint{3.009657in}{4.473789in}}%
\pgfpathcurveto{\pgfqpoint{3.017470in}{4.481603in}}{\pgfqpoint{3.021861in}{4.492202in}}{\pgfqpoint{3.021861in}{4.503252in}}%
\pgfpathcurveto{\pgfqpoint{3.021861in}{4.514302in}}{\pgfqpoint{3.017470in}{4.524901in}}{\pgfqpoint{3.009657in}{4.532715in}}%
\pgfpathcurveto{\pgfqpoint{3.001843in}{4.540529in}}{\pgfqpoint{2.991244in}{4.544919in}}{\pgfqpoint{2.980194in}{4.544919in}}%
\pgfpathcurveto{\pgfqpoint{2.969144in}{4.544919in}}{\pgfqpoint{2.958545in}{4.540529in}}{\pgfqpoint{2.950731in}{4.532715in}}%
\pgfpathcurveto{\pgfqpoint{2.942917in}{4.524901in}}{\pgfqpoint{2.938527in}{4.514302in}}{\pgfqpoint{2.938527in}{4.503252in}}%
\pgfpathcurveto{\pgfqpoint{2.938527in}{4.492202in}}{\pgfqpoint{2.942917in}{4.481603in}}{\pgfqpoint{2.950731in}{4.473789in}}%
\pgfpathcurveto{\pgfqpoint{2.958545in}{4.465976in}}{\pgfqpoint{2.969144in}{4.461586in}}{\pgfqpoint{2.980194in}{4.461586in}}%
\pgfpathclose%
\pgfusepath{stroke,fill}%
\end{pgfscope}%
\begin{pgfscope}%
\pgfpathrectangle{\pgfqpoint{0.481978in}{0.331635in}}{\pgfqpoint{9.300000in}{7.700000in}}%
\pgfusepath{clip}%
\pgfsetbuttcap%
\pgfsetroundjoin%
\definecolor{currentfill}{rgb}{1.000000,0.705882,0.509804}%
\pgfsetfillcolor{currentfill}%
\pgfsetlinewidth{0.481800pt}%
\definecolor{currentstroke}{rgb}{1.000000,1.000000,1.000000}%
\pgfsetstrokecolor{currentstroke}%
\pgfsetdash{}{0pt}%
\pgfpathmoveto{\pgfqpoint{4.618269in}{4.659204in}}%
\pgfpathcurveto{\pgfqpoint{4.629319in}{4.659204in}}{\pgfqpoint{4.639918in}{4.663594in}}{\pgfqpoint{4.647732in}{4.671408in}}%
\pgfpathcurveto{\pgfqpoint{4.655545in}{4.679222in}}{\pgfqpoint{4.659935in}{4.689821in}}{\pgfqpoint{4.659935in}{4.700871in}}%
\pgfpathcurveto{\pgfqpoint{4.659935in}{4.711921in}}{\pgfqpoint{4.655545in}{4.722520in}}{\pgfqpoint{4.647732in}{4.730334in}}%
\pgfpathcurveto{\pgfqpoint{4.639918in}{4.738147in}}{\pgfqpoint{4.629319in}{4.742537in}}{\pgfqpoint{4.618269in}{4.742537in}}%
\pgfpathcurveto{\pgfqpoint{4.607219in}{4.742537in}}{\pgfqpoint{4.596620in}{4.738147in}}{\pgfqpoint{4.588806in}{4.730334in}}%
\pgfpathcurveto{\pgfqpoint{4.580992in}{4.722520in}}{\pgfqpoint{4.576602in}{4.711921in}}{\pgfqpoint{4.576602in}{4.700871in}}%
\pgfpathcurveto{\pgfqpoint{4.576602in}{4.689821in}}{\pgfqpoint{4.580992in}{4.679222in}}{\pgfqpoint{4.588806in}{4.671408in}}%
\pgfpathcurveto{\pgfqpoint{4.596620in}{4.663594in}}{\pgfqpoint{4.607219in}{4.659204in}}{\pgfqpoint{4.618269in}{4.659204in}}%
\pgfpathclose%
\pgfusepath{stroke,fill}%
\end{pgfscope}%
\begin{pgfscope}%
\pgfpathrectangle{\pgfqpoint{0.481978in}{0.331635in}}{\pgfqpoint{9.300000in}{7.700000in}}%
\pgfusepath{clip}%
\pgfsetbuttcap%
\pgfsetroundjoin%
\definecolor{currentfill}{rgb}{1.000000,0.705882,0.509804}%
\pgfsetfillcolor{currentfill}%
\pgfsetlinewidth{0.481800pt}%
\definecolor{currentstroke}{rgb}{1.000000,1.000000,1.000000}%
\pgfsetstrokecolor{currentstroke}%
\pgfsetdash{}{0pt}%
\pgfpathmoveto{\pgfqpoint{3.693817in}{4.042801in}}%
\pgfpathcurveto{\pgfqpoint{3.704867in}{4.042801in}}{\pgfqpoint{3.715466in}{4.047191in}}{\pgfqpoint{3.723279in}{4.055005in}}%
\pgfpathcurveto{\pgfqpoint{3.731093in}{4.062819in}}{\pgfqpoint{3.735483in}{4.073418in}}{\pgfqpoint{3.735483in}{4.084468in}}%
\pgfpathcurveto{\pgfqpoint{3.735483in}{4.095518in}}{\pgfqpoint{3.731093in}{4.106117in}}{\pgfqpoint{3.723279in}{4.113930in}}%
\pgfpathcurveto{\pgfqpoint{3.715466in}{4.121744in}}{\pgfqpoint{3.704867in}{4.126134in}}{\pgfqpoint{3.693817in}{4.126134in}}%
\pgfpathcurveto{\pgfqpoint{3.682766in}{4.126134in}}{\pgfqpoint{3.672167in}{4.121744in}}{\pgfqpoint{3.664354in}{4.113930in}}%
\pgfpathcurveto{\pgfqpoint{3.656540in}{4.106117in}}{\pgfqpoint{3.652150in}{4.095518in}}{\pgfqpoint{3.652150in}{4.084468in}}%
\pgfpathcurveto{\pgfqpoint{3.652150in}{4.073418in}}{\pgfqpoint{3.656540in}{4.062819in}}{\pgfqpoint{3.664354in}{4.055005in}}%
\pgfpathcurveto{\pgfqpoint{3.672167in}{4.047191in}}{\pgfqpoint{3.682766in}{4.042801in}}{\pgfqpoint{3.693817in}{4.042801in}}%
\pgfpathclose%
\pgfusepath{stroke,fill}%
\end{pgfscope}%
\begin{pgfscope}%
\pgfpathrectangle{\pgfqpoint{0.481978in}{0.331635in}}{\pgfqpoint{9.300000in}{7.700000in}}%
\pgfusepath{clip}%
\pgfsetbuttcap%
\pgfsetroundjoin%
\definecolor{currentfill}{rgb}{1.000000,0.705882,0.509804}%
\pgfsetfillcolor{currentfill}%
\pgfsetlinewidth{0.481800pt}%
\definecolor{currentstroke}{rgb}{1.000000,1.000000,1.000000}%
\pgfsetstrokecolor{currentstroke}%
\pgfsetdash{}{0pt}%
\pgfpathmoveto{\pgfqpoint{4.823817in}{3.092458in}}%
\pgfpathcurveto{\pgfqpoint{4.834868in}{3.092458in}}{\pgfqpoint{4.845467in}{3.096848in}}{\pgfqpoint{4.853280in}{3.104662in}}%
\pgfpathcurveto{\pgfqpoint{4.861094in}{3.112475in}}{\pgfqpoint{4.865484in}{3.123074in}}{\pgfqpoint{4.865484in}{3.134124in}}%
\pgfpathcurveto{\pgfqpoint{4.865484in}{3.145175in}}{\pgfqpoint{4.861094in}{3.155774in}}{\pgfqpoint{4.853280in}{3.163587in}}%
\pgfpathcurveto{\pgfqpoint{4.845467in}{3.171401in}}{\pgfqpoint{4.834868in}{3.175791in}}{\pgfqpoint{4.823817in}{3.175791in}}%
\pgfpathcurveto{\pgfqpoint{4.812767in}{3.175791in}}{\pgfqpoint{4.802168in}{3.171401in}}{\pgfqpoint{4.794355in}{3.163587in}}%
\pgfpathcurveto{\pgfqpoint{4.786541in}{3.155774in}}{\pgfqpoint{4.782151in}{3.145175in}}{\pgfqpoint{4.782151in}{3.134124in}}%
\pgfpathcurveto{\pgfqpoint{4.782151in}{3.123074in}}{\pgfqpoint{4.786541in}{3.112475in}}{\pgfqpoint{4.794355in}{3.104662in}}%
\pgfpathcurveto{\pgfqpoint{4.802168in}{3.096848in}}{\pgfqpoint{4.812767in}{3.092458in}}{\pgfqpoint{4.823817in}{3.092458in}}%
\pgfpathclose%
\pgfusepath{stroke,fill}%
\end{pgfscope}%
\begin{pgfscope}%
\pgfpathrectangle{\pgfqpoint{0.481978in}{0.331635in}}{\pgfqpoint{9.300000in}{7.700000in}}%
\pgfusepath{clip}%
\pgfsetbuttcap%
\pgfsetroundjoin%
\definecolor{currentfill}{rgb}{1.000000,0.705882,0.509804}%
\pgfsetfillcolor{currentfill}%
\pgfsetlinewidth{0.481800pt}%
\definecolor{currentstroke}{rgb}{1.000000,1.000000,1.000000}%
\pgfsetstrokecolor{currentstroke}%
\pgfsetdash{}{0pt}%
\pgfpathmoveto{\pgfqpoint{2.627728in}{3.873639in}}%
\pgfpathcurveto{\pgfqpoint{2.638778in}{3.873639in}}{\pgfqpoint{2.649377in}{3.878029in}}{\pgfqpoint{2.657191in}{3.885843in}}%
\pgfpathcurveto{\pgfqpoint{2.665004in}{3.893656in}}{\pgfqpoint{2.669395in}{3.904255in}}{\pgfqpoint{2.669395in}{3.915305in}}%
\pgfpathcurveto{\pgfqpoint{2.669395in}{3.926356in}}{\pgfqpoint{2.665004in}{3.936955in}}{\pgfqpoint{2.657191in}{3.944768in}}%
\pgfpathcurveto{\pgfqpoint{2.649377in}{3.952582in}}{\pgfqpoint{2.638778in}{3.956972in}}{\pgfqpoint{2.627728in}{3.956972in}}%
\pgfpathcurveto{\pgfqpoint{2.616678in}{3.956972in}}{\pgfqpoint{2.606079in}{3.952582in}}{\pgfqpoint{2.598265in}{3.944768in}}%
\pgfpathcurveto{\pgfqpoint{2.590452in}{3.936955in}}{\pgfqpoint{2.586061in}{3.926356in}}{\pgfqpoint{2.586061in}{3.915305in}}%
\pgfpathcurveto{\pgfqpoint{2.586061in}{3.904255in}}{\pgfqpoint{2.590452in}{3.893656in}}{\pgfqpoint{2.598265in}{3.885843in}}%
\pgfpathcurveto{\pgfqpoint{2.606079in}{3.878029in}}{\pgfqpoint{2.616678in}{3.873639in}}{\pgfqpoint{2.627728in}{3.873639in}}%
\pgfpathclose%
\pgfusepath{stroke,fill}%
\end{pgfscope}%
\begin{pgfscope}%
\pgfpathrectangle{\pgfqpoint{0.481978in}{0.331635in}}{\pgfqpoint{9.300000in}{7.700000in}}%
\pgfusepath{clip}%
\pgfsetbuttcap%
\pgfsetroundjoin%
\definecolor{currentfill}{rgb}{1.000000,0.705882,0.509804}%
\pgfsetfillcolor{currentfill}%
\pgfsetlinewidth{0.481800pt}%
\definecolor{currentstroke}{rgb}{1.000000,1.000000,1.000000}%
\pgfsetstrokecolor{currentstroke}%
\pgfsetdash{}{0pt}%
\pgfpathmoveto{\pgfqpoint{2.782878in}{3.328820in}}%
\pgfpathcurveto{\pgfqpoint{2.793928in}{3.328820in}}{\pgfqpoint{2.804527in}{3.333210in}}{\pgfqpoint{2.812341in}{3.341024in}}%
\pgfpathcurveto{\pgfqpoint{2.820155in}{3.348837in}}{\pgfqpoint{2.824545in}{3.359436in}}{\pgfqpoint{2.824545in}{3.370487in}}%
\pgfpathcurveto{\pgfqpoint{2.824545in}{3.381537in}}{\pgfqpoint{2.820155in}{3.392136in}}{\pgfqpoint{2.812341in}{3.399949in}}%
\pgfpathcurveto{\pgfqpoint{2.804527in}{3.407763in}}{\pgfqpoint{2.793928in}{3.412153in}}{\pgfqpoint{2.782878in}{3.412153in}}%
\pgfpathcurveto{\pgfqpoint{2.771828in}{3.412153in}}{\pgfqpoint{2.761229in}{3.407763in}}{\pgfqpoint{2.753415in}{3.399949in}}%
\pgfpathcurveto{\pgfqpoint{2.745602in}{3.392136in}}{\pgfqpoint{2.741212in}{3.381537in}}{\pgfqpoint{2.741212in}{3.370487in}}%
\pgfpathcurveto{\pgfqpoint{2.741212in}{3.359436in}}{\pgfqpoint{2.745602in}{3.348837in}}{\pgfqpoint{2.753415in}{3.341024in}}%
\pgfpathcurveto{\pgfqpoint{2.761229in}{3.333210in}}{\pgfqpoint{2.771828in}{3.328820in}}{\pgfqpoint{2.782878in}{3.328820in}}%
\pgfpathclose%
\pgfusepath{stroke,fill}%
\end{pgfscope}%
\begin{pgfscope}%
\pgfpathrectangle{\pgfqpoint{0.481978in}{0.331635in}}{\pgfqpoint{9.300000in}{7.700000in}}%
\pgfusepath{clip}%
\pgfsetbuttcap%
\pgfsetroundjoin%
\definecolor{currentfill}{rgb}{1.000000,0.705882,0.509804}%
\pgfsetfillcolor{currentfill}%
\pgfsetlinewidth{0.481800pt}%
\definecolor{currentstroke}{rgb}{1.000000,1.000000,1.000000}%
\pgfsetstrokecolor{currentstroke}%
\pgfsetdash{}{0pt}%
\pgfpathmoveto{\pgfqpoint{3.755167in}{4.840302in}}%
\pgfpathcurveto{\pgfqpoint{3.766217in}{4.840302in}}{\pgfqpoint{3.776816in}{4.844693in}}{\pgfqpoint{3.784630in}{4.852506in}}%
\pgfpathcurveto{\pgfqpoint{3.792443in}{4.860320in}}{\pgfqpoint{3.796834in}{4.870919in}}{\pgfqpoint{3.796834in}{4.881969in}}%
\pgfpathcurveto{\pgfqpoint{3.796834in}{4.893019in}}{\pgfqpoint{3.792443in}{4.903618in}}{\pgfqpoint{3.784630in}{4.911432in}}%
\pgfpathcurveto{\pgfqpoint{3.776816in}{4.919245in}}{\pgfqpoint{3.766217in}{4.923636in}}{\pgfqpoint{3.755167in}{4.923636in}}%
\pgfpathcurveto{\pgfqpoint{3.744117in}{4.923636in}}{\pgfqpoint{3.733518in}{4.919245in}}{\pgfqpoint{3.725704in}{4.911432in}}%
\pgfpathcurveto{\pgfqpoint{3.717891in}{4.903618in}}{\pgfqpoint{3.713500in}{4.893019in}}{\pgfqpoint{3.713500in}{4.881969in}}%
\pgfpathcurveto{\pgfqpoint{3.713500in}{4.870919in}}{\pgfqpoint{3.717891in}{4.860320in}}{\pgfqpoint{3.725704in}{4.852506in}}%
\pgfpathcurveto{\pgfqpoint{3.733518in}{4.844693in}}{\pgfqpoint{3.744117in}{4.840302in}}{\pgfqpoint{3.755167in}{4.840302in}}%
\pgfpathclose%
\pgfusepath{stroke,fill}%
\end{pgfscope}%
\begin{pgfscope}%
\pgfpathrectangle{\pgfqpoint{0.481978in}{0.331635in}}{\pgfqpoint{9.300000in}{7.700000in}}%
\pgfusepath{clip}%
\pgfsetbuttcap%
\pgfsetroundjoin%
\definecolor{currentfill}{rgb}{1.000000,0.705882,0.509804}%
\pgfsetfillcolor{currentfill}%
\pgfsetlinewidth{0.481800pt}%
\definecolor{currentstroke}{rgb}{1.000000,1.000000,1.000000}%
\pgfsetstrokecolor{currentstroke}%
\pgfsetdash{}{0pt}%
\pgfpathmoveto{\pgfqpoint{4.616665in}{2.856244in}}%
\pgfpathcurveto{\pgfqpoint{4.627715in}{2.856244in}}{\pgfqpoint{4.638314in}{2.860634in}}{\pgfqpoint{4.646128in}{2.868448in}}%
\pgfpathcurveto{\pgfqpoint{4.653941in}{2.876261in}}{\pgfqpoint{4.658332in}{2.886860in}}{\pgfqpoint{4.658332in}{2.897911in}}%
\pgfpathcurveto{\pgfqpoint{4.658332in}{2.908961in}}{\pgfqpoint{4.653941in}{2.919560in}}{\pgfqpoint{4.646128in}{2.927373in}}%
\pgfpathcurveto{\pgfqpoint{4.638314in}{2.935187in}}{\pgfqpoint{4.627715in}{2.939577in}}{\pgfqpoint{4.616665in}{2.939577in}}%
\pgfpathcurveto{\pgfqpoint{4.605615in}{2.939577in}}{\pgfqpoint{4.595016in}{2.935187in}}{\pgfqpoint{4.587202in}{2.927373in}}%
\pgfpathcurveto{\pgfqpoint{4.579389in}{2.919560in}}{\pgfqpoint{4.574998in}{2.908961in}}{\pgfqpoint{4.574998in}{2.897911in}}%
\pgfpathcurveto{\pgfqpoint{4.574998in}{2.886860in}}{\pgfqpoint{4.579389in}{2.876261in}}{\pgfqpoint{4.587202in}{2.868448in}}%
\pgfpathcurveto{\pgfqpoint{4.595016in}{2.860634in}}{\pgfqpoint{4.605615in}{2.856244in}}{\pgfqpoint{4.616665in}{2.856244in}}%
\pgfpathclose%
\pgfusepath{stroke,fill}%
\end{pgfscope}%
\begin{pgfscope}%
\pgfpathrectangle{\pgfqpoint{0.481978in}{0.331635in}}{\pgfqpoint{9.300000in}{7.700000in}}%
\pgfusepath{clip}%
\pgfsetbuttcap%
\pgfsetroundjoin%
\definecolor{currentfill}{rgb}{1.000000,0.705882,0.509804}%
\pgfsetfillcolor{currentfill}%
\pgfsetlinewidth{0.481800pt}%
\definecolor{currentstroke}{rgb}{1.000000,1.000000,1.000000}%
\pgfsetstrokecolor{currentstroke}%
\pgfsetdash{}{0pt}%
\pgfpathmoveto{\pgfqpoint{2.176301in}{2.880759in}}%
\pgfpathcurveto{\pgfqpoint{2.187351in}{2.880759in}}{\pgfqpoint{2.197950in}{2.885149in}}{\pgfqpoint{2.205763in}{2.892963in}}%
\pgfpathcurveto{\pgfqpoint{2.213577in}{2.900777in}}{\pgfqpoint{2.217967in}{2.911376in}}{\pgfqpoint{2.217967in}{2.922426in}}%
\pgfpathcurveto{\pgfqpoint{2.217967in}{2.933476in}}{\pgfqpoint{2.213577in}{2.944075in}}{\pgfqpoint{2.205763in}{2.951889in}}%
\pgfpathcurveto{\pgfqpoint{2.197950in}{2.959702in}}{\pgfqpoint{2.187351in}{2.964092in}}{\pgfqpoint{2.176301in}{2.964092in}}%
\pgfpathcurveto{\pgfqpoint{2.165250in}{2.964092in}}{\pgfqpoint{2.154651in}{2.959702in}}{\pgfqpoint{2.146838in}{2.951889in}}%
\pgfpathcurveto{\pgfqpoint{2.139024in}{2.944075in}}{\pgfqpoint{2.134634in}{2.933476in}}{\pgfqpoint{2.134634in}{2.922426in}}%
\pgfpathcurveto{\pgfqpoint{2.134634in}{2.911376in}}{\pgfqpoint{2.139024in}{2.900777in}}{\pgfqpoint{2.146838in}{2.892963in}}%
\pgfpathcurveto{\pgfqpoint{2.154651in}{2.885149in}}{\pgfqpoint{2.165250in}{2.880759in}}{\pgfqpoint{2.176301in}{2.880759in}}%
\pgfpathclose%
\pgfusepath{stroke,fill}%
\end{pgfscope}%
\begin{pgfscope}%
\pgfpathrectangle{\pgfqpoint{0.481978in}{0.331635in}}{\pgfqpoint{9.300000in}{7.700000in}}%
\pgfusepath{clip}%
\pgfsetbuttcap%
\pgfsetroundjoin%
\definecolor{currentfill}{rgb}{1.000000,0.705882,0.509804}%
\pgfsetfillcolor{currentfill}%
\pgfsetlinewidth{0.481800pt}%
\definecolor{currentstroke}{rgb}{1.000000,1.000000,1.000000}%
\pgfsetstrokecolor{currentstroke}%
\pgfsetdash{}{0pt}%
\pgfpathmoveto{\pgfqpoint{4.459267in}{4.545856in}}%
\pgfpathcurveto{\pgfqpoint{4.470317in}{4.545856in}}{\pgfqpoint{4.480916in}{4.550246in}}{\pgfqpoint{4.488730in}{4.558060in}}%
\pgfpathcurveto{\pgfqpoint{4.496543in}{4.565874in}}{\pgfqpoint{4.500934in}{4.576473in}}{\pgfqpoint{4.500934in}{4.587523in}}%
\pgfpathcurveto{\pgfqpoint{4.500934in}{4.598573in}}{\pgfqpoint{4.496543in}{4.609172in}}{\pgfqpoint{4.488730in}{4.616986in}}%
\pgfpathcurveto{\pgfqpoint{4.480916in}{4.624799in}}{\pgfqpoint{4.470317in}{4.629190in}}{\pgfqpoint{4.459267in}{4.629190in}}%
\pgfpathcurveto{\pgfqpoint{4.448217in}{4.629190in}}{\pgfqpoint{4.437618in}{4.624799in}}{\pgfqpoint{4.429804in}{4.616986in}}%
\pgfpathcurveto{\pgfqpoint{4.421991in}{4.609172in}}{\pgfqpoint{4.417600in}{4.598573in}}{\pgfqpoint{4.417600in}{4.587523in}}%
\pgfpathcurveto{\pgfqpoint{4.417600in}{4.576473in}}{\pgfqpoint{4.421991in}{4.565874in}}{\pgfqpoint{4.429804in}{4.558060in}}%
\pgfpathcurveto{\pgfqpoint{4.437618in}{4.550246in}}{\pgfqpoint{4.448217in}{4.545856in}}{\pgfqpoint{4.459267in}{4.545856in}}%
\pgfpathclose%
\pgfusepath{stroke,fill}%
\end{pgfscope}%
\begin{pgfscope}%
\pgfpathrectangle{\pgfqpoint{0.481978in}{0.331635in}}{\pgfqpoint{9.300000in}{7.700000in}}%
\pgfusepath{clip}%
\pgfsetbuttcap%
\pgfsetroundjoin%
\definecolor{currentfill}{rgb}{1.000000,0.705882,0.509804}%
\pgfsetfillcolor{currentfill}%
\pgfsetlinewidth{0.481800pt}%
\definecolor{currentstroke}{rgb}{1.000000,1.000000,1.000000}%
\pgfsetstrokecolor{currentstroke}%
\pgfsetdash{}{0pt}%
\pgfpathmoveto{\pgfqpoint{4.335059in}{2.978217in}}%
\pgfpathcurveto{\pgfqpoint{4.346109in}{2.978217in}}{\pgfqpoint{4.356708in}{2.982607in}}{\pgfqpoint{4.364522in}{2.990420in}}%
\pgfpathcurveto{\pgfqpoint{4.372336in}{2.998234in}}{\pgfqpoint{4.376726in}{3.008833in}}{\pgfqpoint{4.376726in}{3.019883in}}%
\pgfpathcurveto{\pgfqpoint{4.376726in}{3.030933in}}{\pgfqpoint{4.372336in}{3.041532in}}{\pgfqpoint{4.364522in}{3.049346in}}%
\pgfpathcurveto{\pgfqpoint{4.356708in}{3.057160in}}{\pgfqpoint{4.346109in}{3.061550in}}{\pgfqpoint{4.335059in}{3.061550in}}%
\pgfpathcurveto{\pgfqpoint{4.324009in}{3.061550in}}{\pgfqpoint{4.313410in}{3.057160in}}{\pgfqpoint{4.305597in}{3.049346in}}%
\pgfpathcurveto{\pgfqpoint{4.297783in}{3.041532in}}{\pgfqpoint{4.293393in}{3.030933in}}{\pgfqpoint{4.293393in}{3.019883in}}%
\pgfpathcurveto{\pgfqpoint{4.293393in}{3.008833in}}{\pgfqpoint{4.297783in}{2.998234in}}{\pgfqpoint{4.305597in}{2.990420in}}%
\pgfpathcurveto{\pgfqpoint{4.313410in}{2.982607in}}{\pgfqpoint{4.324009in}{2.978217in}}{\pgfqpoint{4.335059in}{2.978217in}}%
\pgfpathclose%
\pgfusepath{stroke,fill}%
\end{pgfscope}%
\begin{pgfscope}%
\pgfpathrectangle{\pgfqpoint{0.481978in}{0.331635in}}{\pgfqpoint{9.300000in}{7.700000in}}%
\pgfusepath{clip}%
\pgfsetbuttcap%
\pgfsetroundjoin%
\definecolor{currentfill}{rgb}{1.000000,0.705882,0.509804}%
\pgfsetfillcolor{currentfill}%
\pgfsetlinewidth{0.481800pt}%
\definecolor{currentstroke}{rgb}{1.000000,1.000000,1.000000}%
\pgfsetstrokecolor{currentstroke}%
\pgfsetdash{}{0pt}%
\pgfpathmoveto{\pgfqpoint{4.103001in}{2.947748in}}%
\pgfpathcurveto{\pgfqpoint{4.114051in}{2.947748in}}{\pgfqpoint{4.124650in}{2.952138in}}{\pgfqpoint{4.132463in}{2.959952in}}%
\pgfpathcurveto{\pgfqpoint{4.140277in}{2.967766in}}{\pgfqpoint{4.144667in}{2.978365in}}{\pgfqpoint{4.144667in}{2.989415in}}%
\pgfpathcurveto{\pgfqpoint{4.144667in}{3.000465in}}{\pgfqpoint{4.140277in}{3.011064in}}{\pgfqpoint{4.132463in}{3.018878in}}%
\pgfpathcurveto{\pgfqpoint{4.124650in}{3.026691in}}{\pgfqpoint{4.114051in}{3.031082in}}{\pgfqpoint{4.103001in}{3.031082in}}%
\pgfpathcurveto{\pgfqpoint{4.091950in}{3.031082in}}{\pgfqpoint{4.081351in}{3.026691in}}{\pgfqpoint{4.073538in}{3.018878in}}%
\pgfpathcurveto{\pgfqpoint{4.065724in}{3.011064in}}{\pgfqpoint{4.061334in}{3.000465in}}{\pgfqpoint{4.061334in}{2.989415in}}%
\pgfpathcurveto{\pgfqpoint{4.061334in}{2.978365in}}{\pgfqpoint{4.065724in}{2.967766in}}{\pgfqpoint{4.073538in}{2.959952in}}%
\pgfpathcurveto{\pgfqpoint{4.081351in}{2.952138in}}{\pgfqpoint{4.091950in}{2.947748in}}{\pgfqpoint{4.103001in}{2.947748in}}%
\pgfpathclose%
\pgfusepath{stroke,fill}%
\end{pgfscope}%
\begin{pgfscope}%
\pgfpathrectangle{\pgfqpoint{0.481978in}{0.331635in}}{\pgfqpoint{9.300000in}{7.700000in}}%
\pgfusepath{clip}%
\pgfsetbuttcap%
\pgfsetroundjoin%
\definecolor{currentfill}{rgb}{1.000000,0.705882,0.509804}%
\pgfsetfillcolor{currentfill}%
\pgfsetlinewidth{0.481800pt}%
\definecolor{currentstroke}{rgb}{1.000000,1.000000,1.000000}%
\pgfsetstrokecolor{currentstroke}%
\pgfsetdash{}{0pt}%
\pgfpathmoveto{\pgfqpoint{1.240028in}{3.747940in}}%
\pgfpathcurveto{\pgfqpoint{1.251078in}{3.747940in}}{\pgfqpoint{1.261677in}{3.752330in}}{\pgfqpoint{1.269490in}{3.760144in}}%
\pgfpathcurveto{\pgfqpoint{1.277304in}{3.767957in}}{\pgfqpoint{1.281694in}{3.778556in}}{\pgfqpoint{1.281694in}{3.789606in}}%
\pgfpathcurveto{\pgfqpoint{1.281694in}{3.800657in}}{\pgfqpoint{1.277304in}{3.811256in}}{\pgfqpoint{1.269490in}{3.819069in}}%
\pgfpathcurveto{\pgfqpoint{1.261677in}{3.826883in}}{\pgfqpoint{1.251078in}{3.831273in}}{\pgfqpoint{1.240028in}{3.831273in}}%
\pgfpathcurveto{\pgfqpoint{1.228977in}{3.831273in}}{\pgfqpoint{1.218378in}{3.826883in}}{\pgfqpoint{1.210565in}{3.819069in}}%
\pgfpathcurveto{\pgfqpoint{1.202751in}{3.811256in}}{\pgfqpoint{1.198361in}{3.800657in}}{\pgfqpoint{1.198361in}{3.789606in}}%
\pgfpathcurveto{\pgfqpoint{1.198361in}{3.778556in}}{\pgfqpoint{1.202751in}{3.767957in}}{\pgfqpoint{1.210565in}{3.760144in}}%
\pgfpathcurveto{\pgfqpoint{1.218378in}{3.752330in}}{\pgfqpoint{1.228977in}{3.747940in}}{\pgfqpoint{1.240028in}{3.747940in}}%
\pgfpathclose%
\pgfusepath{stroke,fill}%
\end{pgfscope}%
\begin{pgfscope}%
\pgfpathrectangle{\pgfqpoint{0.481978in}{0.331635in}}{\pgfqpoint{9.300000in}{7.700000in}}%
\pgfusepath{clip}%
\pgfsetbuttcap%
\pgfsetroundjoin%
\definecolor{currentfill}{rgb}{1.000000,0.705882,0.509804}%
\pgfsetfillcolor{currentfill}%
\pgfsetlinewidth{0.481800pt}%
\definecolor{currentstroke}{rgb}{1.000000,1.000000,1.000000}%
\pgfsetstrokecolor{currentstroke}%
\pgfsetdash{}{0pt}%
\pgfpathmoveto{\pgfqpoint{3.665627in}{3.670075in}}%
\pgfpathcurveto{\pgfqpoint{3.676677in}{3.670075in}}{\pgfqpoint{3.687276in}{3.674466in}}{\pgfqpoint{3.695090in}{3.682279in}}%
\pgfpathcurveto{\pgfqpoint{3.702904in}{3.690093in}}{\pgfqpoint{3.707294in}{3.700692in}}{\pgfqpoint{3.707294in}{3.711742in}}%
\pgfpathcurveto{\pgfqpoint{3.707294in}{3.722792in}}{\pgfqpoint{3.702904in}{3.733391in}}{\pgfqpoint{3.695090in}{3.741205in}}%
\pgfpathcurveto{\pgfqpoint{3.687276in}{3.749018in}}{\pgfqpoint{3.676677in}{3.753409in}}{\pgfqpoint{3.665627in}{3.753409in}}%
\pgfpathcurveto{\pgfqpoint{3.654577in}{3.753409in}}{\pgfqpoint{3.643978in}{3.749018in}}{\pgfqpoint{3.636164in}{3.741205in}}%
\pgfpathcurveto{\pgfqpoint{3.628351in}{3.733391in}}{\pgfqpoint{3.623961in}{3.722792in}}{\pgfqpoint{3.623961in}{3.711742in}}%
\pgfpathcurveto{\pgfqpoint{3.623961in}{3.700692in}}{\pgfqpoint{3.628351in}{3.690093in}}{\pgfqpoint{3.636164in}{3.682279in}}%
\pgfpathcurveto{\pgfqpoint{3.643978in}{3.674466in}}{\pgfqpoint{3.654577in}{3.670075in}}{\pgfqpoint{3.665627in}{3.670075in}}%
\pgfpathclose%
\pgfusepath{stroke,fill}%
\end{pgfscope}%
\begin{pgfscope}%
\pgfpathrectangle{\pgfqpoint{0.481978in}{0.331635in}}{\pgfqpoint{9.300000in}{7.700000in}}%
\pgfusepath{clip}%
\pgfsetbuttcap%
\pgfsetroundjoin%
\definecolor{currentfill}{rgb}{1.000000,0.705882,0.509804}%
\pgfsetfillcolor{currentfill}%
\pgfsetlinewidth{0.481800pt}%
\definecolor{currentstroke}{rgb}{1.000000,1.000000,1.000000}%
\pgfsetstrokecolor{currentstroke}%
\pgfsetdash{}{0pt}%
\pgfpathmoveto{\pgfqpoint{3.637232in}{4.380124in}}%
\pgfpathcurveto{\pgfqpoint{3.648282in}{4.380124in}}{\pgfqpoint{3.658881in}{4.384514in}}{\pgfqpoint{3.666694in}{4.392328in}}%
\pgfpathcurveto{\pgfqpoint{3.674508in}{4.400142in}}{\pgfqpoint{3.678898in}{4.410741in}}{\pgfqpoint{3.678898in}{4.421791in}}%
\pgfpathcurveto{\pgfqpoint{3.678898in}{4.432841in}}{\pgfqpoint{3.674508in}{4.443440in}}{\pgfqpoint{3.666694in}{4.451253in}}%
\pgfpathcurveto{\pgfqpoint{3.658881in}{4.459067in}}{\pgfqpoint{3.648282in}{4.463457in}}{\pgfqpoint{3.637232in}{4.463457in}}%
\pgfpathcurveto{\pgfqpoint{3.626181in}{4.463457in}}{\pgfqpoint{3.615582in}{4.459067in}}{\pgfqpoint{3.607769in}{4.451253in}}%
\pgfpathcurveto{\pgfqpoint{3.599955in}{4.443440in}}{\pgfqpoint{3.595565in}{4.432841in}}{\pgfqpoint{3.595565in}{4.421791in}}%
\pgfpathcurveto{\pgfqpoint{3.595565in}{4.410741in}}{\pgfqpoint{3.599955in}{4.400142in}}{\pgfqpoint{3.607769in}{4.392328in}}%
\pgfpathcurveto{\pgfqpoint{3.615582in}{4.384514in}}{\pgfqpoint{3.626181in}{4.380124in}}{\pgfqpoint{3.637232in}{4.380124in}}%
\pgfpathclose%
\pgfusepath{stroke,fill}%
\end{pgfscope}%
\begin{pgfscope}%
\pgfpathrectangle{\pgfqpoint{0.481978in}{0.331635in}}{\pgfqpoint{9.300000in}{7.700000in}}%
\pgfusepath{clip}%
\pgfsetbuttcap%
\pgfsetroundjoin%
\definecolor{currentfill}{rgb}{1.000000,0.705882,0.509804}%
\pgfsetfillcolor{currentfill}%
\pgfsetlinewidth{0.481800pt}%
\definecolor{currentstroke}{rgb}{1.000000,1.000000,1.000000}%
\pgfsetstrokecolor{currentstroke}%
\pgfsetdash{}{0pt}%
\pgfpathmoveto{\pgfqpoint{1.338259in}{3.701359in}}%
\pgfpathcurveto{\pgfqpoint{1.349309in}{3.701359in}}{\pgfqpoint{1.359908in}{3.705749in}}{\pgfqpoint{1.367722in}{3.713562in}}%
\pgfpathcurveto{\pgfqpoint{1.375535in}{3.721376in}}{\pgfqpoint{1.379926in}{3.731975in}}{\pgfqpoint{1.379926in}{3.743025in}}%
\pgfpathcurveto{\pgfqpoint{1.379926in}{3.754075in}}{\pgfqpoint{1.375535in}{3.764674in}}{\pgfqpoint{1.367722in}{3.772488in}}%
\pgfpathcurveto{\pgfqpoint{1.359908in}{3.780302in}}{\pgfqpoint{1.349309in}{3.784692in}}{\pgfqpoint{1.338259in}{3.784692in}}%
\pgfpathcurveto{\pgfqpoint{1.327209in}{3.784692in}}{\pgfqpoint{1.316610in}{3.780302in}}{\pgfqpoint{1.308796in}{3.772488in}}%
\pgfpathcurveto{\pgfqpoint{1.300983in}{3.764674in}}{\pgfqpoint{1.296592in}{3.754075in}}{\pgfqpoint{1.296592in}{3.743025in}}%
\pgfpathcurveto{\pgfqpoint{1.296592in}{3.731975in}}{\pgfqpoint{1.300983in}{3.721376in}}{\pgfqpoint{1.308796in}{3.713562in}}%
\pgfpathcurveto{\pgfqpoint{1.316610in}{3.705749in}}{\pgfqpoint{1.327209in}{3.701359in}}{\pgfqpoint{1.338259in}{3.701359in}}%
\pgfpathclose%
\pgfusepath{stroke,fill}%
\end{pgfscope}%
\begin{pgfscope}%
\pgfpathrectangle{\pgfqpoint{0.481978in}{0.331635in}}{\pgfqpoint{9.300000in}{7.700000in}}%
\pgfusepath{clip}%
\pgfsetbuttcap%
\pgfsetroundjoin%
\definecolor{currentfill}{rgb}{1.000000,0.705882,0.509804}%
\pgfsetfillcolor{currentfill}%
\pgfsetlinewidth{0.481800pt}%
\definecolor{currentstroke}{rgb}{1.000000,1.000000,1.000000}%
\pgfsetstrokecolor{currentstroke}%
\pgfsetdash{}{0pt}%
\pgfpathmoveto{\pgfqpoint{1.995054in}{2.990321in}}%
\pgfpathcurveto{\pgfqpoint{2.006104in}{2.990321in}}{\pgfqpoint{2.016703in}{2.994711in}}{\pgfqpoint{2.024516in}{3.002525in}}%
\pgfpathcurveto{\pgfqpoint{2.032330in}{3.010338in}}{\pgfqpoint{2.036720in}{3.020937in}}{\pgfqpoint{2.036720in}{3.031987in}}%
\pgfpathcurveto{\pgfqpoint{2.036720in}{3.043038in}}{\pgfqpoint{2.032330in}{3.053637in}}{\pgfqpoint{2.024516in}{3.061450in}}%
\pgfpathcurveto{\pgfqpoint{2.016703in}{3.069264in}}{\pgfqpoint{2.006104in}{3.073654in}}{\pgfqpoint{1.995054in}{3.073654in}}%
\pgfpathcurveto{\pgfqpoint{1.984003in}{3.073654in}}{\pgfqpoint{1.973404in}{3.069264in}}{\pgfqpoint{1.965591in}{3.061450in}}%
\pgfpathcurveto{\pgfqpoint{1.957777in}{3.053637in}}{\pgfqpoint{1.953387in}{3.043038in}}{\pgfqpoint{1.953387in}{3.031987in}}%
\pgfpathcurveto{\pgfqpoint{1.953387in}{3.020937in}}{\pgfqpoint{1.957777in}{3.010338in}}{\pgfqpoint{1.965591in}{3.002525in}}%
\pgfpathcurveto{\pgfqpoint{1.973404in}{2.994711in}}{\pgfqpoint{1.984003in}{2.990321in}}{\pgfqpoint{1.995054in}{2.990321in}}%
\pgfpathclose%
\pgfusepath{stroke,fill}%
\end{pgfscope}%
\begin{pgfscope}%
\pgfpathrectangle{\pgfqpoint{0.481978in}{0.331635in}}{\pgfqpoint{9.300000in}{7.700000in}}%
\pgfusepath{clip}%
\pgfsetbuttcap%
\pgfsetroundjoin%
\definecolor{currentfill}{rgb}{1.000000,0.705882,0.509804}%
\pgfsetfillcolor{currentfill}%
\pgfsetlinewidth{0.481800pt}%
\definecolor{currentstroke}{rgb}{1.000000,1.000000,1.000000}%
\pgfsetstrokecolor{currentstroke}%
\pgfsetdash{}{0pt}%
\pgfpathmoveto{\pgfqpoint{1.350306in}{3.961584in}}%
\pgfpathcurveto{\pgfqpoint{1.361356in}{3.961584in}}{\pgfqpoint{1.371955in}{3.965974in}}{\pgfqpoint{1.379769in}{3.973788in}}%
\pgfpathcurveto{\pgfqpoint{1.387582in}{3.981601in}}{\pgfqpoint{1.391973in}{3.992200in}}{\pgfqpoint{1.391973in}{4.003250in}}%
\pgfpathcurveto{\pgfqpoint{1.391973in}{4.014301in}}{\pgfqpoint{1.387582in}{4.024900in}}{\pgfqpoint{1.379769in}{4.032713in}}%
\pgfpathcurveto{\pgfqpoint{1.371955in}{4.040527in}}{\pgfqpoint{1.361356in}{4.044917in}}{\pgfqpoint{1.350306in}{4.044917in}}%
\pgfpathcurveto{\pgfqpoint{1.339256in}{4.044917in}}{\pgfqpoint{1.328657in}{4.040527in}}{\pgfqpoint{1.320843in}{4.032713in}}%
\pgfpathcurveto{\pgfqpoint{1.313029in}{4.024900in}}{\pgfqpoint{1.308639in}{4.014301in}}{\pgfqpoint{1.308639in}{4.003250in}}%
\pgfpathcurveto{\pgfqpoint{1.308639in}{3.992200in}}{\pgfqpoint{1.313029in}{3.981601in}}{\pgfqpoint{1.320843in}{3.973788in}}%
\pgfpathcurveto{\pgfqpoint{1.328657in}{3.965974in}}{\pgfqpoint{1.339256in}{3.961584in}}{\pgfqpoint{1.350306in}{3.961584in}}%
\pgfpathclose%
\pgfusepath{stroke,fill}%
\end{pgfscope}%
\begin{pgfscope}%
\pgfpathrectangle{\pgfqpoint{0.481978in}{0.331635in}}{\pgfqpoint{9.300000in}{7.700000in}}%
\pgfusepath{clip}%
\pgfsetbuttcap%
\pgfsetroundjoin%
\definecolor{currentfill}{rgb}{1.000000,0.705882,0.509804}%
\pgfsetfillcolor{currentfill}%
\pgfsetlinewidth{0.481800pt}%
\definecolor{currentstroke}{rgb}{1.000000,1.000000,1.000000}%
\pgfsetstrokecolor{currentstroke}%
\pgfsetdash{}{0pt}%
\pgfpathmoveto{\pgfqpoint{1.701822in}{4.024914in}}%
\pgfpathcurveto{\pgfqpoint{1.712872in}{4.024914in}}{\pgfqpoint{1.723471in}{4.029304in}}{\pgfqpoint{1.731284in}{4.037117in}}%
\pgfpathcurveto{\pgfqpoint{1.739098in}{4.044931in}}{\pgfqpoint{1.743488in}{4.055530in}}{\pgfqpoint{1.743488in}{4.066580in}}%
\pgfpathcurveto{\pgfqpoint{1.743488in}{4.077630in}}{\pgfqpoint{1.739098in}{4.088229in}}{\pgfqpoint{1.731284in}{4.096043in}}%
\pgfpathcurveto{\pgfqpoint{1.723471in}{4.103857in}}{\pgfqpoint{1.712872in}{4.108247in}}{\pgfqpoint{1.701822in}{4.108247in}}%
\pgfpathcurveto{\pgfqpoint{1.690771in}{4.108247in}}{\pgfqpoint{1.680172in}{4.103857in}}{\pgfqpoint{1.672359in}{4.096043in}}%
\pgfpathcurveto{\pgfqpoint{1.664545in}{4.088229in}}{\pgfqpoint{1.660155in}{4.077630in}}{\pgfqpoint{1.660155in}{4.066580in}}%
\pgfpathcurveto{\pgfqpoint{1.660155in}{4.055530in}}{\pgfqpoint{1.664545in}{4.044931in}}{\pgfqpoint{1.672359in}{4.037117in}}%
\pgfpathcurveto{\pgfqpoint{1.680172in}{4.029304in}}{\pgfqpoint{1.690771in}{4.024914in}}{\pgfqpoint{1.701822in}{4.024914in}}%
\pgfpathclose%
\pgfusepath{stroke,fill}%
\end{pgfscope}%
\begin{pgfscope}%
\pgfpathrectangle{\pgfqpoint{0.481978in}{0.331635in}}{\pgfqpoint{9.300000in}{7.700000in}}%
\pgfusepath{clip}%
\pgfsetbuttcap%
\pgfsetroundjoin%
\definecolor{currentfill}{rgb}{1.000000,0.705882,0.509804}%
\pgfsetfillcolor{currentfill}%
\pgfsetlinewidth{0.481800pt}%
\definecolor{currentstroke}{rgb}{1.000000,1.000000,1.000000}%
\pgfsetstrokecolor{currentstroke}%
\pgfsetdash{}{0pt}%
\pgfpathmoveto{\pgfqpoint{3.815204in}{3.964944in}}%
\pgfpathcurveto{\pgfqpoint{3.826254in}{3.964944in}}{\pgfqpoint{3.836853in}{3.969335in}}{\pgfqpoint{3.844667in}{3.977148in}}%
\pgfpathcurveto{\pgfqpoint{3.852481in}{3.984962in}}{\pgfqpoint{3.856871in}{3.995561in}}{\pgfqpoint{3.856871in}{4.006611in}}%
\pgfpathcurveto{\pgfqpoint{3.856871in}{4.017661in}}{\pgfqpoint{3.852481in}{4.028260in}}{\pgfqpoint{3.844667in}{4.036074in}}%
\pgfpathcurveto{\pgfqpoint{3.836853in}{4.043888in}}{\pgfqpoint{3.826254in}{4.048278in}}{\pgfqpoint{3.815204in}{4.048278in}}%
\pgfpathcurveto{\pgfqpoint{3.804154in}{4.048278in}}{\pgfqpoint{3.793555in}{4.043888in}}{\pgfqpoint{3.785742in}{4.036074in}}%
\pgfpathcurveto{\pgfqpoint{3.777928in}{4.028260in}}{\pgfqpoint{3.773538in}{4.017661in}}{\pgfqpoint{3.773538in}{4.006611in}}%
\pgfpathcurveto{\pgfqpoint{3.773538in}{3.995561in}}{\pgfqpoint{3.777928in}{3.984962in}}{\pgfqpoint{3.785742in}{3.977148in}}%
\pgfpathcurveto{\pgfqpoint{3.793555in}{3.969335in}}{\pgfqpoint{3.804154in}{3.964944in}}{\pgfqpoint{3.815204in}{3.964944in}}%
\pgfpathclose%
\pgfusepath{stroke,fill}%
\end{pgfscope}%
\begin{pgfscope}%
\pgfpathrectangle{\pgfqpoint{0.481978in}{0.331635in}}{\pgfqpoint{9.300000in}{7.700000in}}%
\pgfusepath{clip}%
\pgfsetbuttcap%
\pgfsetroundjoin%
\definecolor{currentfill}{rgb}{1.000000,0.705882,0.509804}%
\pgfsetfillcolor{currentfill}%
\pgfsetlinewidth{0.481800pt}%
\definecolor{currentstroke}{rgb}{1.000000,1.000000,1.000000}%
\pgfsetstrokecolor{currentstroke}%
\pgfsetdash{}{0pt}%
\pgfpathmoveto{\pgfqpoint{3.592234in}{4.931619in}}%
\pgfpathcurveto{\pgfqpoint{3.603285in}{4.931619in}}{\pgfqpoint{3.613884in}{4.936009in}}{\pgfqpoint{3.621697in}{4.943823in}}%
\pgfpathcurveto{\pgfqpoint{3.629511in}{4.951636in}}{\pgfqpoint{3.633901in}{4.962235in}}{\pgfqpoint{3.633901in}{4.973285in}}%
\pgfpathcurveto{\pgfqpoint{3.633901in}{4.984335in}}{\pgfqpoint{3.629511in}{4.994934in}}{\pgfqpoint{3.621697in}{5.002748in}}%
\pgfpathcurveto{\pgfqpoint{3.613884in}{5.010562in}}{\pgfqpoint{3.603285in}{5.014952in}}{\pgfqpoint{3.592234in}{5.014952in}}%
\pgfpathcurveto{\pgfqpoint{3.581184in}{5.014952in}}{\pgfqpoint{3.570585in}{5.010562in}}{\pgfqpoint{3.562772in}{5.002748in}}%
\pgfpathcurveto{\pgfqpoint{3.554958in}{4.994934in}}{\pgfqpoint{3.550568in}{4.984335in}}{\pgfqpoint{3.550568in}{4.973285in}}%
\pgfpathcurveto{\pgfqpoint{3.550568in}{4.962235in}}{\pgfqpoint{3.554958in}{4.951636in}}{\pgfqpoint{3.562772in}{4.943823in}}%
\pgfpathcurveto{\pgfqpoint{3.570585in}{4.936009in}}{\pgfqpoint{3.581184in}{4.931619in}}{\pgfqpoint{3.592234in}{4.931619in}}%
\pgfpathclose%
\pgfusepath{stroke,fill}%
\end{pgfscope}%
\begin{pgfscope}%
\pgfpathrectangle{\pgfqpoint{0.481978in}{0.331635in}}{\pgfqpoint{9.300000in}{7.700000in}}%
\pgfusepath{clip}%
\pgfsetbuttcap%
\pgfsetroundjoin%
\definecolor{currentfill}{rgb}{1.000000,0.705882,0.509804}%
\pgfsetfillcolor{currentfill}%
\pgfsetlinewidth{0.481800pt}%
\definecolor{currentstroke}{rgb}{1.000000,1.000000,1.000000}%
\pgfsetstrokecolor{currentstroke}%
\pgfsetdash{}{0pt}%
\pgfpathmoveto{\pgfqpoint{4.400900in}{2.356598in}}%
\pgfpathcurveto{\pgfqpoint{4.411950in}{2.356598in}}{\pgfqpoint{4.422549in}{2.360988in}}{\pgfqpoint{4.430363in}{2.368802in}}%
\pgfpathcurveto{\pgfqpoint{4.438176in}{2.376615in}}{\pgfqpoint{4.442567in}{2.387214in}}{\pgfqpoint{4.442567in}{2.398264in}}%
\pgfpathcurveto{\pgfqpoint{4.442567in}{2.409315in}}{\pgfqpoint{4.438176in}{2.419914in}}{\pgfqpoint{4.430363in}{2.427727in}}%
\pgfpathcurveto{\pgfqpoint{4.422549in}{2.435541in}}{\pgfqpoint{4.411950in}{2.439931in}}{\pgfqpoint{4.400900in}{2.439931in}}%
\pgfpathcurveto{\pgfqpoint{4.389850in}{2.439931in}}{\pgfqpoint{4.379251in}{2.435541in}}{\pgfqpoint{4.371437in}{2.427727in}}%
\pgfpathcurveto{\pgfqpoint{4.363624in}{2.419914in}}{\pgfqpoint{4.359233in}{2.409315in}}{\pgfqpoint{4.359233in}{2.398264in}}%
\pgfpathcurveto{\pgfqpoint{4.359233in}{2.387214in}}{\pgfqpoint{4.363624in}{2.376615in}}{\pgfqpoint{4.371437in}{2.368802in}}%
\pgfpathcurveto{\pgfqpoint{4.379251in}{2.360988in}}{\pgfqpoint{4.389850in}{2.356598in}}{\pgfqpoint{4.400900in}{2.356598in}}%
\pgfpathclose%
\pgfusepath{stroke,fill}%
\end{pgfscope}%
\begin{pgfscope}%
\pgfpathrectangle{\pgfqpoint{0.481978in}{0.331635in}}{\pgfqpoint{9.300000in}{7.700000in}}%
\pgfusepath{clip}%
\pgfsetbuttcap%
\pgfsetroundjoin%
\definecolor{currentfill}{rgb}{1.000000,0.705882,0.509804}%
\pgfsetfillcolor{currentfill}%
\pgfsetlinewidth{0.481800pt}%
\definecolor{currentstroke}{rgb}{1.000000,1.000000,1.000000}%
\pgfsetstrokecolor{currentstroke}%
\pgfsetdash{}{0pt}%
\pgfpathmoveto{\pgfqpoint{4.677099in}{3.453777in}}%
\pgfpathcurveto{\pgfqpoint{4.688149in}{3.453777in}}{\pgfqpoint{4.698748in}{3.458167in}}{\pgfqpoint{4.706562in}{3.465981in}}%
\pgfpathcurveto{\pgfqpoint{4.714375in}{3.473794in}}{\pgfqpoint{4.718766in}{3.484393in}}{\pgfqpoint{4.718766in}{3.495443in}}%
\pgfpathcurveto{\pgfqpoint{4.718766in}{3.506493in}}{\pgfqpoint{4.714375in}{3.517093in}}{\pgfqpoint{4.706562in}{3.524906in}}%
\pgfpathcurveto{\pgfqpoint{4.698748in}{3.532720in}}{\pgfqpoint{4.688149in}{3.537110in}}{\pgfqpoint{4.677099in}{3.537110in}}%
\pgfpathcurveto{\pgfqpoint{4.666049in}{3.537110in}}{\pgfqpoint{4.655450in}{3.532720in}}{\pgfqpoint{4.647636in}{3.524906in}}%
\pgfpathcurveto{\pgfqpoint{4.639823in}{3.517093in}}{\pgfqpoint{4.635432in}{3.506493in}}{\pgfqpoint{4.635432in}{3.495443in}}%
\pgfpathcurveto{\pgfqpoint{4.635432in}{3.484393in}}{\pgfqpoint{4.639823in}{3.473794in}}{\pgfqpoint{4.647636in}{3.465981in}}%
\pgfpathcurveto{\pgfqpoint{4.655450in}{3.458167in}}{\pgfqpoint{4.666049in}{3.453777in}}{\pgfqpoint{4.677099in}{3.453777in}}%
\pgfpathclose%
\pgfusepath{stroke,fill}%
\end{pgfscope}%
\begin{pgfscope}%
\pgfpathrectangle{\pgfqpoint{0.481978in}{0.331635in}}{\pgfqpoint{9.300000in}{7.700000in}}%
\pgfusepath{clip}%
\pgfsetbuttcap%
\pgfsetroundjoin%
\definecolor{currentfill}{rgb}{1.000000,0.705882,0.509804}%
\pgfsetfillcolor{currentfill}%
\pgfsetlinewidth{0.481800pt}%
\definecolor{currentstroke}{rgb}{1.000000,1.000000,1.000000}%
\pgfsetstrokecolor{currentstroke}%
\pgfsetdash{}{0pt}%
\pgfpathmoveto{\pgfqpoint{2.595817in}{5.461427in}}%
\pgfpathcurveto{\pgfqpoint{2.606867in}{5.461427in}}{\pgfqpoint{2.617466in}{5.465817in}}{\pgfqpoint{2.625280in}{5.473631in}}%
\pgfpathcurveto{\pgfqpoint{2.633094in}{5.481444in}}{\pgfqpoint{2.637484in}{5.492043in}}{\pgfqpoint{2.637484in}{5.503093in}}%
\pgfpathcurveto{\pgfqpoint{2.637484in}{5.514144in}}{\pgfqpoint{2.633094in}{5.524743in}}{\pgfqpoint{2.625280in}{5.532556in}}%
\pgfpathcurveto{\pgfqpoint{2.617466in}{5.540370in}}{\pgfqpoint{2.606867in}{5.544760in}}{\pgfqpoint{2.595817in}{5.544760in}}%
\pgfpathcurveto{\pgfqpoint{2.584767in}{5.544760in}}{\pgfqpoint{2.574168in}{5.540370in}}{\pgfqpoint{2.566354in}{5.532556in}}%
\pgfpathcurveto{\pgfqpoint{2.558541in}{5.524743in}}{\pgfqpoint{2.554150in}{5.514144in}}{\pgfqpoint{2.554150in}{5.503093in}}%
\pgfpathcurveto{\pgfqpoint{2.554150in}{5.492043in}}{\pgfqpoint{2.558541in}{5.481444in}}{\pgfqpoint{2.566354in}{5.473631in}}%
\pgfpathcurveto{\pgfqpoint{2.574168in}{5.465817in}}{\pgfqpoint{2.584767in}{5.461427in}}{\pgfqpoint{2.595817in}{5.461427in}}%
\pgfpathclose%
\pgfusepath{stroke,fill}%
\end{pgfscope}%
\begin{pgfscope}%
\pgfpathrectangle{\pgfqpoint{0.481978in}{0.331635in}}{\pgfqpoint{9.300000in}{7.700000in}}%
\pgfusepath{clip}%
\pgfsetbuttcap%
\pgfsetroundjoin%
\definecolor{currentfill}{rgb}{1.000000,0.705882,0.509804}%
\pgfsetfillcolor{currentfill}%
\pgfsetlinewidth{0.481800pt}%
\definecolor{currentstroke}{rgb}{1.000000,1.000000,1.000000}%
\pgfsetstrokecolor{currentstroke}%
\pgfsetdash{}{0pt}%
\pgfpathmoveto{\pgfqpoint{3.380362in}{4.087253in}}%
\pgfpathcurveto{\pgfqpoint{3.391412in}{4.087253in}}{\pgfqpoint{3.402011in}{4.091643in}}{\pgfqpoint{3.409824in}{4.099457in}}%
\pgfpathcurveto{\pgfqpoint{3.417638in}{4.107270in}}{\pgfqpoint{3.422028in}{4.117869in}}{\pgfqpoint{3.422028in}{4.128920in}}%
\pgfpathcurveto{\pgfqpoint{3.422028in}{4.139970in}}{\pgfqpoint{3.417638in}{4.150569in}}{\pgfqpoint{3.409824in}{4.158382in}}%
\pgfpathcurveto{\pgfqpoint{3.402011in}{4.166196in}}{\pgfqpoint{3.391412in}{4.170586in}}{\pgfqpoint{3.380362in}{4.170586in}}%
\pgfpathcurveto{\pgfqpoint{3.369311in}{4.170586in}}{\pgfqpoint{3.358712in}{4.166196in}}{\pgfqpoint{3.350899in}{4.158382in}}%
\pgfpathcurveto{\pgfqpoint{3.343085in}{4.150569in}}{\pgfqpoint{3.338695in}{4.139970in}}{\pgfqpoint{3.338695in}{4.128920in}}%
\pgfpathcurveto{\pgfqpoint{3.338695in}{4.117869in}}{\pgfqpoint{3.343085in}{4.107270in}}{\pgfqpoint{3.350899in}{4.099457in}}%
\pgfpathcurveto{\pgfqpoint{3.358712in}{4.091643in}}{\pgfqpoint{3.369311in}{4.087253in}}{\pgfqpoint{3.380362in}{4.087253in}}%
\pgfpathclose%
\pgfusepath{stroke,fill}%
\end{pgfscope}%
\begin{pgfscope}%
\pgfpathrectangle{\pgfqpoint{0.481978in}{0.331635in}}{\pgfqpoint{9.300000in}{7.700000in}}%
\pgfusepath{clip}%
\pgfsetbuttcap%
\pgfsetroundjoin%
\definecolor{currentfill}{rgb}{1.000000,0.705882,0.509804}%
\pgfsetfillcolor{currentfill}%
\pgfsetlinewidth{0.481800pt}%
\definecolor{currentstroke}{rgb}{1.000000,1.000000,1.000000}%
\pgfsetstrokecolor{currentstroke}%
\pgfsetdash{}{0pt}%
\pgfpathmoveto{\pgfqpoint{8.094040in}{3.873702in}}%
\pgfpathcurveto{\pgfqpoint{8.105090in}{3.873702in}}{\pgfqpoint{8.115689in}{3.878093in}}{\pgfqpoint{8.123503in}{3.885906in}}%
\pgfpathcurveto{\pgfqpoint{8.131317in}{3.893720in}}{\pgfqpoint{8.135707in}{3.904319in}}{\pgfqpoint{8.135707in}{3.915369in}}%
\pgfpathcurveto{\pgfqpoint{8.135707in}{3.926419in}}{\pgfqpoint{8.131317in}{3.937018in}}{\pgfqpoint{8.123503in}{3.944832in}}%
\pgfpathcurveto{\pgfqpoint{8.115689in}{3.952645in}}{\pgfqpoint{8.105090in}{3.957036in}}{\pgfqpoint{8.094040in}{3.957036in}}%
\pgfpathcurveto{\pgfqpoint{8.082990in}{3.957036in}}{\pgfqpoint{8.072391in}{3.952645in}}{\pgfqpoint{8.064577in}{3.944832in}}%
\pgfpathcurveto{\pgfqpoint{8.056764in}{3.937018in}}{\pgfqpoint{8.052374in}{3.926419in}}{\pgfqpoint{8.052374in}{3.915369in}}%
\pgfpathcurveto{\pgfqpoint{8.052374in}{3.904319in}}{\pgfqpoint{8.056764in}{3.893720in}}{\pgfqpoint{8.064577in}{3.885906in}}%
\pgfpathcurveto{\pgfqpoint{8.072391in}{3.878093in}}{\pgfqpoint{8.082990in}{3.873702in}}{\pgfqpoint{8.094040in}{3.873702in}}%
\pgfpathclose%
\pgfusepath{stroke,fill}%
\end{pgfscope}%
\begin{pgfscope}%
\pgfpathrectangle{\pgfqpoint{0.481978in}{0.331635in}}{\pgfqpoint{9.300000in}{7.700000in}}%
\pgfusepath{clip}%
\pgfsetbuttcap%
\pgfsetroundjoin%
\definecolor{currentfill}{rgb}{1.000000,0.705882,0.509804}%
\pgfsetfillcolor{currentfill}%
\pgfsetlinewidth{0.481800pt}%
\definecolor{currentstroke}{rgb}{1.000000,1.000000,1.000000}%
\pgfsetstrokecolor{currentstroke}%
\pgfsetdash{}{0pt}%
\pgfpathmoveto{\pgfqpoint{3.449664in}{4.246640in}}%
\pgfpathcurveto{\pgfqpoint{3.460714in}{4.246640in}}{\pgfqpoint{3.471313in}{4.251030in}}{\pgfqpoint{3.479127in}{4.258844in}}%
\pgfpathcurveto{\pgfqpoint{3.486941in}{4.266658in}}{\pgfqpoint{3.491331in}{4.277257in}}{\pgfqpoint{3.491331in}{4.288307in}}%
\pgfpathcurveto{\pgfqpoint{3.491331in}{4.299357in}}{\pgfqpoint{3.486941in}{4.309956in}}{\pgfqpoint{3.479127in}{4.317770in}}%
\pgfpathcurveto{\pgfqpoint{3.471313in}{4.325583in}}{\pgfqpoint{3.460714in}{4.329974in}}{\pgfqpoint{3.449664in}{4.329974in}}%
\pgfpathcurveto{\pgfqpoint{3.438614in}{4.329974in}}{\pgfqpoint{3.428015in}{4.325583in}}{\pgfqpoint{3.420202in}{4.317770in}}%
\pgfpathcurveto{\pgfqpoint{3.412388in}{4.309956in}}{\pgfqpoint{3.407998in}{4.299357in}}{\pgfqpoint{3.407998in}{4.288307in}}%
\pgfpathcurveto{\pgfqpoint{3.407998in}{4.277257in}}{\pgfqpoint{3.412388in}{4.266658in}}{\pgfqpoint{3.420202in}{4.258844in}}%
\pgfpathcurveto{\pgfqpoint{3.428015in}{4.251030in}}{\pgfqpoint{3.438614in}{4.246640in}}{\pgfqpoint{3.449664in}{4.246640in}}%
\pgfpathclose%
\pgfusepath{stroke,fill}%
\end{pgfscope}%
\begin{pgfscope}%
\pgfpathrectangle{\pgfqpoint{0.481978in}{0.331635in}}{\pgfqpoint{9.300000in}{7.700000in}}%
\pgfusepath{clip}%
\pgfsetbuttcap%
\pgfsetroundjoin%
\definecolor{currentfill}{rgb}{1.000000,0.705882,0.509804}%
\pgfsetfillcolor{currentfill}%
\pgfsetlinewidth{0.481800pt}%
\definecolor{currentstroke}{rgb}{1.000000,1.000000,1.000000}%
\pgfsetstrokecolor{currentstroke}%
\pgfsetdash{}{0pt}%
\pgfpathmoveto{\pgfqpoint{3.549059in}{7.204392in}}%
\pgfpathcurveto{\pgfqpoint{3.560110in}{7.204392in}}{\pgfqpoint{3.570709in}{7.208782in}}{\pgfqpoint{3.578522in}{7.216596in}}%
\pgfpathcurveto{\pgfqpoint{3.586336in}{7.224409in}}{\pgfqpoint{3.590726in}{7.235008in}}{\pgfqpoint{3.590726in}{7.246059in}}%
\pgfpathcurveto{\pgfqpoint{3.590726in}{7.257109in}}{\pgfqpoint{3.586336in}{7.267708in}}{\pgfqpoint{3.578522in}{7.275521in}}%
\pgfpathcurveto{\pgfqpoint{3.570709in}{7.283335in}}{\pgfqpoint{3.560110in}{7.287725in}}{\pgfqpoint{3.549059in}{7.287725in}}%
\pgfpathcurveto{\pgfqpoint{3.538009in}{7.287725in}}{\pgfqpoint{3.527410in}{7.283335in}}{\pgfqpoint{3.519597in}{7.275521in}}%
\pgfpathcurveto{\pgfqpoint{3.511783in}{7.267708in}}{\pgfqpoint{3.507393in}{7.257109in}}{\pgfqpoint{3.507393in}{7.246059in}}%
\pgfpathcurveto{\pgfqpoint{3.507393in}{7.235008in}}{\pgfqpoint{3.511783in}{7.224409in}}{\pgfqpoint{3.519597in}{7.216596in}}%
\pgfpathcurveto{\pgfqpoint{3.527410in}{7.208782in}}{\pgfqpoint{3.538009in}{7.204392in}}{\pgfqpoint{3.549059in}{7.204392in}}%
\pgfpathclose%
\pgfusepath{stroke,fill}%
\end{pgfscope}%
\begin{pgfscope}%
\pgfpathrectangle{\pgfqpoint{0.481978in}{0.331635in}}{\pgfqpoint{9.300000in}{7.700000in}}%
\pgfusepath{clip}%
\pgfsetbuttcap%
\pgfsetroundjoin%
\definecolor{currentfill}{rgb}{1.000000,0.705882,0.509804}%
\pgfsetfillcolor{currentfill}%
\pgfsetlinewidth{0.481800pt}%
\definecolor{currentstroke}{rgb}{1.000000,1.000000,1.000000}%
\pgfsetstrokecolor{currentstroke}%
\pgfsetdash{}{0pt}%
\pgfpathmoveto{\pgfqpoint{1.737569in}{3.281104in}}%
\pgfpathcurveto{\pgfqpoint{1.748619in}{3.281104in}}{\pgfqpoint{1.759218in}{3.285494in}}{\pgfqpoint{1.767031in}{3.293308in}}%
\pgfpathcurveto{\pgfqpoint{1.774845in}{3.301121in}}{\pgfqpoint{1.779235in}{3.311720in}}{\pgfqpoint{1.779235in}{3.322770in}}%
\pgfpathcurveto{\pgfqpoint{1.779235in}{3.333821in}}{\pgfqpoint{1.774845in}{3.344420in}}{\pgfqpoint{1.767031in}{3.352233in}}%
\pgfpathcurveto{\pgfqpoint{1.759218in}{3.360047in}}{\pgfqpoint{1.748619in}{3.364437in}}{\pgfqpoint{1.737569in}{3.364437in}}%
\pgfpathcurveto{\pgfqpoint{1.726518in}{3.364437in}}{\pgfqpoint{1.715919in}{3.360047in}}{\pgfqpoint{1.708106in}{3.352233in}}%
\pgfpathcurveto{\pgfqpoint{1.700292in}{3.344420in}}{\pgfqpoint{1.695902in}{3.333821in}}{\pgfqpoint{1.695902in}{3.322770in}}%
\pgfpathcurveto{\pgfqpoint{1.695902in}{3.311720in}}{\pgfqpoint{1.700292in}{3.301121in}}{\pgfqpoint{1.708106in}{3.293308in}}%
\pgfpathcurveto{\pgfqpoint{1.715919in}{3.285494in}}{\pgfqpoint{1.726518in}{3.281104in}}{\pgfqpoint{1.737569in}{3.281104in}}%
\pgfpathclose%
\pgfusepath{stroke,fill}%
\end{pgfscope}%
\begin{pgfscope}%
\pgfpathrectangle{\pgfqpoint{0.481978in}{0.331635in}}{\pgfqpoint{9.300000in}{7.700000in}}%
\pgfusepath{clip}%
\pgfsetbuttcap%
\pgfsetroundjoin%
\definecolor{currentfill}{rgb}{1.000000,0.705882,0.509804}%
\pgfsetfillcolor{currentfill}%
\pgfsetlinewidth{0.481800pt}%
\definecolor{currentstroke}{rgb}{1.000000,1.000000,1.000000}%
\pgfsetstrokecolor{currentstroke}%
\pgfsetdash{}{0pt}%
\pgfpathmoveto{\pgfqpoint{3.522445in}{3.104341in}}%
\pgfpathcurveto{\pgfqpoint{3.533495in}{3.104341in}}{\pgfqpoint{3.544094in}{3.108731in}}{\pgfqpoint{3.551907in}{3.116545in}}%
\pgfpathcurveto{\pgfqpoint{3.559721in}{3.124358in}}{\pgfqpoint{3.564111in}{3.134957in}}{\pgfqpoint{3.564111in}{3.146007in}}%
\pgfpathcurveto{\pgfqpoint{3.564111in}{3.157057in}}{\pgfqpoint{3.559721in}{3.167657in}}{\pgfqpoint{3.551907in}{3.175470in}}%
\pgfpathcurveto{\pgfqpoint{3.544094in}{3.183284in}}{\pgfqpoint{3.533495in}{3.187674in}}{\pgfqpoint{3.522445in}{3.187674in}}%
\pgfpathcurveto{\pgfqpoint{3.511395in}{3.187674in}}{\pgfqpoint{3.500795in}{3.183284in}}{\pgfqpoint{3.492982in}{3.175470in}}%
\pgfpathcurveto{\pgfqpoint{3.485168in}{3.167657in}}{\pgfqpoint{3.480778in}{3.157057in}}{\pgfqpoint{3.480778in}{3.146007in}}%
\pgfpathcurveto{\pgfqpoint{3.480778in}{3.134957in}}{\pgfqpoint{3.485168in}{3.124358in}}{\pgfqpoint{3.492982in}{3.116545in}}%
\pgfpathcurveto{\pgfqpoint{3.500795in}{3.108731in}}{\pgfqpoint{3.511395in}{3.104341in}}{\pgfqpoint{3.522445in}{3.104341in}}%
\pgfpathclose%
\pgfusepath{stroke,fill}%
\end{pgfscope}%
\begin{pgfscope}%
\pgfpathrectangle{\pgfqpoint{0.481978in}{0.331635in}}{\pgfqpoint{9.300000in}{7.700000in}}%
\pgfusepath{clip}%
\pgfsetbuttcap%
\pgfsetroundjoin%
\definecolor{currentfill}{rgb}{1.000000,0.705882,0.509804}%
\pgfsetfillcolor{currentfill}%
\pgfsetlinewidth{0.481800pt}%
\definecolor{currentstroke}{rgb}{1.000000,1.000000,1.000000}%
\pgfsetstrokecolor{currentstroke}%
\pgfsetdash{}{0pt}%
\pgfpathmoveto{\pgfqpoint{5.442526in}{4.550181in}}%
\pgfpathcurveto{\pgfqpoint{5.453577in}{4.550181in}}{\pgfqpoint{5.464176in}{4.554571in}}{\pgfqpoint{5.471989in}{4.562385in}}%
\pgfpathcurveto{\pgfqpoint{5.479803in}{4.570198in}}{\pgfqpoint{5.484193in}{4.580797in}}{\pgfqpoint{5.484193in}{4.591847in}}%
\pgfpathcurveto{\pgfqpoint{5.484193in}{4.602898in}}{\pgfqpoint{5.479803in}{4.613497in}}{\pgfqpoint{5.471989in}{4.621310in}}%
\pgfpathcurveto{\pgfqpoint{5.464176in}{4.629124in}}{\pgfqpoint{5.453577in}{4.633514in}}{\pgfqpoint{5.442526in}{4.633514in}}%
\pgfpathcurveto{\pgfqpoint{5.431476in}{4.633514in}}{\pgfqpoint{5.420877in}{4.629124in}}{\pgfqpoint{5.413064in}{4.621310in}}%
\pgfpathcurveto{\pgfqpoint{5.405250in}{4.613497in}}{\pgfqpoint{5.400860in}{4.602898in}}{\pgfqpoint{5.400860in}{4.591847in}}%
\pgfpathcurveto{\pgfqpoint{5.400860in}{4.580797in}}{\pgfqpoint{5.405250in}{4.570198in}}{\pgfqpoint{5.413064in}{4.562385in}}%
\pgfpathcurveto{\pgfqpoint{5.420877in}{4.554571in}}{\pgfqpoint{5.431476in}{4.550181in}}{\pgfqpoint{5.442526in}{4.550181in}}%
\pgfpathclose%
\pgfusepath{stroke,fill}%
\end{pgfscope}%
\begin{pgfscope}%
\pgfpathrectangle{\pgfqpoint{0.481978in}{0.331635in}}{\pgfqpoint{9.300000in}{7.700000in}}%
\pgfusepath{clip}%
\pgfsetbuttcap%
\pgfsetroundjoin%
\definecolor{currentfill}{rgb}{1.000000,0.705882,0.509804}%
\pgfsetfillcolor{currentfill}%
\pgfsetlinewidth{0.481800pt}%
\definecolor{currentstroke}{rgb}{1.000000,1.000000,1.000000}%
\pgfsetstrokecolor{currentstroke}%
\pgfsetdash{}{0pt}%
\pgfpathmoveto{\pgfqpoint{3.182729in}{4.161824in}}%
\pgfpathcurveto{\pgfqpoint{3.193779in}{4.161824in}}{\pgfqpoint{3.204378in}{4.166214in}}{\pgfqpoint{3.212192in}{4.174028in}}%
\pgfpathcurveto{\pgfqpoint{3.220005in}{4.181842in}}{\pgfqpoint{3.224396in}{4.192441in}}{\pgfqpoint{3.224396in}{4.203491in}}%
\pgfpathcurveto{\pgfqpoint{3.224396in}{4.214541in}}{\pgfqpoint{3.220005in}{4.225140in}}{\pgfqpoint{3.212192in}{4.232953in}}%
\pgfpathcurveto{\pgfqpoint{3.204378in}{4.240767in}}{\pgfqpoint{3.193779in}{4.245157in}}{\pgfqpoint{3.182729in}{4.245157in}}%
\pgfpathcurveto{\pgfqpoint{3.171679in}{4.245157in}}{\pgfqpoint{3.161080in}{4.240767in}}{\pgfqpoint{3.153266in}{4.232953in}}%
\pgfpathcurveto{\pgfqpoint{3.145453in}{4.225140in}}{\pgfqpoint{3.141062in}{4.214541in}}{\pgfqpoint{3.141062in}{4.203491in}}%
\pgfpathcurveto{\pgfqpoint{3.141062in}{4.192441in}}{\pgfqpoint{3.145453in}{4.181842in}}{\pgfqpoint{3.153266in}{4.174028in}}%
\pgfpathcurveto{\pgfqpoint{3.161080in}{4.166214in}}{\pgfqpoint{3.171679in}{4.161824in}}{\pgfqpoint{3.182729in}{4.161824in}}%
\pgfpathclose%
\pgfusepath{stroke,fill}%
\end{pgfscope}%
\begin{pgfscope}%
\pgfpathrectangle{\pgfqpoint{0.481978in}{0.331635in}}{\pgfqpoint{9.300000in}{7.700000in}}%
\pgfusepath{clip}%
\pgfsetbuttcap%
\pgfsetroundjoin%
\definecolor{currentfill}{rgb}{1.000000,0.705882,0.509804}%
\pgfsetfillcolor{currentfill}%
\pgfsetlinewidth{0.481800pt}%
\definecolor{currentstroke}{rgb}{1.000000,1.000000,1.000000}%
\pgfsetstrokecolor{currentstroke}%
\pgfsetdash{}{0pt}%
\pgfpathmoveto{\pgfqpoint{5.580105in}{6.284102in}}%
\pgfpathcurveto{\pgfqpoint{5.591155in}{6.284102in}}{\pgfqpoint{5.601754in}{6.288492in}}{\pgfqpoint{5.609568in}{6.296305in}}%
\pgfpathcurveto{\pgfqpoint{5.617382in}{6.304119in}}{\pgfqpoint{5.621772in}{6.314718in}}{\pgfqpoint{5.621772in}{6.325768in}}%
\pgfpathcurveto{\pgfqpoint{5.621772in}{6.336818in}}{\pgfqpoint{5.617382in}{6.347417in}}{\pgfqpoint{5.609568in}{6.355231in}}%
\pgfpathcurveto{\pgfqpoint{5.601754in}{6.363045in}}{\pgfqpoint{5.591155in}{6.367435in}}{\pgfqpoint{5.580105in}{6.367435in}}%
\pgfpathcurveto{\pgfqpoint{5.569055in}{6.367435in}}{\pgfqpoint{5.558456in}{6.363045in}}{\pgfqpoint{5.550642in}{6.355231in}}%
\pgfpathcurveto{\pgfqpoint{5.542829in}{6.347417in}}{\pgfqpoint{5.538438in}{6.336818in}}{\pgfqpoint{5.538438in}{6.325768in}}%
\pgfpathcurveto{\pgfqpoint{5.538438in}{6.314718in}}{\pgfqpoint{5.542829in}{6.304119in}}{\pgfqpoint{5.550642in}{6.296305in}}%
\pgfpathcurveto{\pgfqpoint{5.558456in}{6.288492in}}{\pgfqpoint{5.569055in}{6.284102in}}{\pgfqpoint{5.580105in}{6.284102in}}%
\pgfpathclose%
\pgfusepath{stroke,fill}%
\end{pgfscope}%
\begin{pgfscope}%
\pgfpathrectangle{\pgfqpoint{0.481978in}{0.331635in}}{\pgfqpoint{9.300000in}{7.700000in}}%
\pgfusepath{clip}%
\pgfsetbuttcap%
\pgfsetroundjoin%
\definecolor{currentfill}{rgb}{1.000000,0.705882,0.509804}%
\pgfsetfillcolor{currentfill}%
\pgfsetlinewidth{0.481800pt}%
\definecolor{currentstroke}{rgb}{1.000000,1.000000,1.000000}%
\pgfsetstrokecolor{currentstroke}%
\pgfsetdash{}{0pt}%
\pgfpathmoveto{\pgfqpoint{3.383682in}{4.885031in}}%
\pgfpathcurveto{\pgfqpoint{3.394732in}{4.885031in}}{\pgfqpoint{3.405331in}{4.889422in}}{\pgfqpoint{3.413144in}{4.897235in}}%
\pgfpathcurveto{\pgfqpoint{3.420958in}{4.905049in}}{\pgfqpoint{3.425348in}{4.915648in}}{\pgfqpoint{3.425348in}{4.926698in}}%
\pgfpathcurveto{\pgfqpoint{3.425348in}{4.937748in}}{\pgfqpoint{3.420958in}{4.948347in}}{\pgfqpoint{3.413144in}{4.956161in}}%
\pgfpathcurveto{\pgfqpoint{3.405331in}{4.963975in}}{\pgfqpoint{3.394732in}{4.968365in}}{\pgfqpoint{3.383682in}{4.968365in}}%
\pgfpathcurveto{\pgfqpoint{3.372631in}{4.968365in}}{\pgfqpoint{3.362032in}{4.963975in}}{\pgfqpoint{3.354219in}{4.956161in}}%
\pgfpathcurveto{\pgfqpoint{3.346405in}{4.948347in}}{\pgfqpoint{3.342015in}{4.937748in}}{\pgfqpoint{3.342015in}{4.926698in}}%
\pgfpathcurveto{\pgfqpoint{3.342015in}{4.915648in}}{\pgfqpoint{3.346405in}{4.905049in}}{\pgfqpoint{3.354219in}{4.897235in}}%
\pgfpathcurveto{\pgfqpoint{3.362032in}{4.889422in}}{\pgfqpoint{3.372631in}{4.885031in}}{\pgfqpoint{3.383682in}{4.885031in}}%
\pgfpathclose%
\pgfusepath{stroke,fill}%
\end{pgfscope}%
\begin{pgfscope}%
\pgfpathrectangle{\pgfqpoint{0.481978in}{0.331635in}}{\pgfqpoint{9.300000in}{7.700000in}}%
\pgfusepath{clip}%
\pgfsetbuttcap%
\pgfsetroundjoin%
\definecolor{currentfill}{rgb}{1.000000,0.705882,0.509804}%
\pgfsetfillcolor{currentfill}%
\pgfsetlinewidth{0.481800pt}%
\definecolor{currentstroke}{rgb}{1.000000,1.000000,1.000000}%
\pgfsetstrokecolor{currentstroke}%
\pgfsetdash{}{0pt}%
\pgfpathmoveto{\pgfqpoint{3.010244in}{5.125365in}}%
\pgfpathcurveto{\pgfqpoint{3.021294in}{5.125365in}}{\pgfqpoint{3.031893in}{5.129755in}}{\pgfqpoint{3.039707in}{5.137569in}}%
\pgfpathcurveto{\pgfqpoint{3.047521in}{5.145382in}}{\pgfqpoint{3.051911in}{5.155981in}}{\pgfqpoint{3.051911in}{5.167031in}}%
\pgfpathcurveto{\pgfqpoint{3.051911in}{5.178081in}}{\pgfqpoint{3.047521in}{5.188680in}}{\pgfqpoint{3.039707in}{5.196494in}}%
\pgfpathcurveto{\pgfqpoint{3.031893in}{5.204308in}}{\pgfqpoint{3.021294in}{5.208698in}}{\pgfqpoint{3.010244in}{5.208698in}}%
\pgfpathcurveto{\pgfqpoint{2.999194in}{5.208698in}}{\pgfqpoint{2.988595in}{5.204308in}}{\pgfqpoint{2.980781in}{5.196494in}}%
\pgfpathcurveto{\pgfqpoint{2.972968in}{5.188680in}}{\pgfqpoint{2.968578in}{5.178081in}}{\pgfqpoint{2.968578in}{5.167031in}}%
\pgfpathcurveto{\pgfqpoint{2.968578in}{5.155981in}}{\pgfqpoint{2.972968in}{5.145382in}}{\pgfqpoint{2.980781in}{5.137569in}}%
\pgfpathcurveto{\pgfqpoint{2.988595in}{5.129755in}}{\pgfqpoint{2.999194in}{5.125365in}}{\pgfqpoint{3.010244in}{5.125365in}}%
\pgfpathclose%
\pgfusepath{stroke,fill}%
\end{pgfscope}%
\begin{pgfscope}%
\pgfpathrectangle{\pgfqpoint{0.481978in}{0.331635in}}{\pgfqpoint{9.300000in}{7.700000in}}%
\pgfusepath{clip}%
\pgfsetbuttcap%
\pgfsetroundjoin%
\definecolor{currentfill}{rgb}{1.000000,0.705882,0.509804}%
\pgfsetfillcolor{currentfill}%
\pgfsetlinewidth{0.481800pt}%
\definecolor{currentstroke}{rgb}{1.000000,1.000000,1.000000}%
\pgfsetstrokecolor{currentstroke}%
\pgfsetdash{}{0pt}%
\pgfpathmoveto{\pgfqpoint{4.553603in}{3.997453in}}%
\pgfpathcurveto{\pgfqpoint{4.564653in}{3.997453in}}{\pgfqpoint{4.575252in}{4.001843in}}{\pgfqpoint{4.583066in}{4.009657in}}%
\pgfpathcurveto{\pgfqpoint{4.590880in}{4.017470in}}{\pgfqpoint{4.595270in}{4.028069in}}{\pgfqpoint{4.595270in}{4.039119in}}%
\pgfpathcurveto{\pgfqpoint{4.595270in}{4.050170in}}{\pgfqpoint{4.590880in}{4.060769in}}{\pgfqpoint{4.583066in}{4.068582in}}%
\pgfpathcurveto{\pgfqpoint{4.575252in}{4.076396in}}{\pgfqpoint{4.564653in}{4.080786in}}{\pgfqpoint{4.553603in}{4.080786in}}%
\pgfpathcurveto{\pgfqpoint{4.542553in}{4.080786in}}{\pgfqpoint{4.531954in}{4.076396in}}{\pgfqpoint{4.524140in}{4.068582in}}%
\pgfpathcurveto{\pgfqpoint{4.516327in}{4.060769in}}{\pgfqpoint{4.511936in}{4.050170in}}{\pgfqpoint{4.511936in}{4.039119in}}%
\pgfpathcurveto{\pgfqpoint{4.511936in}{4.028069in}}{\pgfqpoint{4.516327in}{4.017470in}}{\pgfqpoint{4.524140in}{4.009657in}}%
\pgfpathcurveto{\pgfqpoint{4.531954in}{4.001843in}}{\pgfqpoint{4.542553in}{3.997453in}}{\pgfqpoint{4.553603in}{3.997453in}}%
\pgfpathclose%
\pgfusepath{stroke,fill}%
\end{pgfscope}%
\begin{pgfscope}%
\pgfpathrectangle{\pgfqpoint{0.481978in}{0.331635in}}{\pgfqpoint{9.300000in}{7.700000in}}%
\pgfusepath{clip}%
\pgfsetbuttcap%
\pgfsetroundjoin%
\definecolor{currentfill}{rgb}{1.000000,0.705882,0.509804}%
\pgfsetfillcolor{currentfill}%
\pgfsetlinewidth{0.481800pt}%
\definecolor{currentstroke}{rgb}{1.000000,1.000000,1.000000}%
\pgfsetstrokecolor{currentstroke}%
\pgfsetdash{}{0pt}%
\pgfpathmoveto{\pgfqpoint{9.359202in}{1.323697in}}%
\pgfpathcurveto{\pgfqpoint{9.370252in}{1.323697in}}{\pgfqpoint{9.380851in}{1.328088in}}{\pgfqpoint{9.388665in}{1.335901in}}%
\pgfpathcurveto{\pgfqpoint{9.396478in}{1.343715in}}{\pgfqpoint{9.400868in}{1.354314in}}{\pgfqpoint{9.400868in}{1.365364in}}%
\pgfpathcurveto{\pgfqpoint{9.400868in}{1.376414in}}{\pgfqpoint{9.396478in}{1.387013in}}{\pgfqpoint{9.388665in}{1.394827in}}%
\pgfpathcurveto{\pgfqpoint{9.380851in}{1.402640in}}{\pgfqpoint{9.370252in}{1.407031in}}{\pgfqpoint{9.359202in}{1.407031in}}%
\pgfpathcurveto{\pgfqpoint{9.348152in}{1.407031in}}{\pgfqpoint{9.337553in}{1.402640in}}{\pgfqpoint{9.329739in}{1.394827in}}%
\pgfpathcurveto{\pgfqpoint{9.321925in}{1.387013in}}{\pgfqpoint{9.317535in}{1.376414in}}{\pgfqpoint{9.317535in}{1.365364in}}%
\pgfpathcurveto{\pgfqpoint{9.317535in}{1.354314in}}{\pgfqpoint{9.321925in}{1.343715in}}{\pgfqpoint{9.329739in}{1.335901in}}%
\pgfpathcurveto{\pgfqpoint{9.337553in}{1.328088in}}{\pgfqpoint{9.348152in}{1.323697in}}{\pgfqpoint{9.359202in}{1.323697in}}%
\pgfpathclose%
\pgfusepath{stroke,fill}%
\end{pgfscope}%
\begin{pgfscope}%
\pgfpathrectangle{\pgfqpoint{0.481978in}{0.331635in}}{\pgfqpoint{9.300000in}{7.700000in}}%
\pgfusepath{clip}%
\pgfsetbuttcap%
\pgfsetroundjoin%
\definecolor{currentfill}{rgb}{1.000000,0.705882,0.509804}%
\pgfsetfillcolor{currentfill}%
\pgfsetlinewidth{0.481800pt}%
\definecolor{currentstroke}{rgb}{1.000000,1.000000,1.000000}%
\pgfsetstrokecolor{currentstroke}%
\pgfsetdash{}{0pt}%
\pgfpathmoveto{\pgfqpoint{3.953659in}{2.869946in}}%
\pgfpathcurveto{\pgfqpoint{3.964709in}{2.869946in}}{\pgfqpoint{3.975308in}{2.874336in}}{\pgfqpoint{3.983121in}{2.882150in}}%
\pgfpathcurveto{\pgfqpoint{3.990935in}{2.889963in}}{\pgfqpoint{3.995325in}{2.900563in}}{\pgfqpoint{3.995325in}{2.911613in}}%
\pgfpathcurveto{\pgfqpoint{3.995325in}{2.922663in}}{\pgfqpoint{3.990935in}{2.933262in}}{\pgfqpoint{3.983121in}{2.941075in}}%
\pgfpathcurveto{\pgfqpoint{3.975308in}{2.948889in}}{\pgfqpoint{3.964709in}{2.953279in}}{\pgfqpoint{3.953659in}{2.953279in}}%
\pgfpathcurveto{\pgfqpoint{3.942609in}{2.953279in}}{\pgfqpoint{3.932009in}{2.948889in}}{\pgfqpoint{3.924196in}{2.941075in}}%
\pgfpathcurveto{\pgfqpoint{3.916382in}{2.933262in}}{\pgfqpoint{3.911992in}{2.922663in}}{\pgfqpoint{3.911992in}{2.911613in}}%
\pgfpathcurveto{\pgfqpoint{3.911992in}{2.900563in}}{\pgfqpoint{3.916382in}{2.889963in}}{\pgfqpoint{3.924196in}{2.882150in}}%
\pgfpathcurveto{\pgfqpoint{3.932009in}{2.874336in}}{\pgfqpoint{3.942609in}{2.869946in}}{\pgfqpoint{3.953659in}{2.869946in}}%
\pgfpathclose%
\pgfusepath{stroke,fill}%
\end{pgfscope}%
\begin{pgfscope}%
\pgfpathrectangle{\pgfqpoint{0.481978in}{0.331635in}}{\pgfqpoint{9.300000in}{7.700000in}}%
\pgfusepath{clip}%
\pgfsetbuttcap%
\pgfsetroundjoin%
\definecolor{currentfill}{rgb}{1.000000,0.705882,0.509804}%
\pgfsetfillcolor{currentfill}%
\pgfsetlinewidth{0.481800pt}%
\definecolor{currentstroke}{rgb}{1.000000,1.000000,1.000000}%
\pgfsetstrokecolor{currentstroke}%
\pgfsetdash{}{0pt}%
\pgfpathmoveto{\pgfqpoint{4.408411in}{2.598323in}}%
\pgfpathcurveto{\pgfqpoint{4.419461in}{2.598323in}}{\pgfqpoint{4.430060in}{2.602713in}}{\pgfqpoint{4.437874in}{2.610527in}}%
\pgfpathcurveto{\pgfqpoint{4.445688in}{2.618341in}}{\pgfqpoint{4.450078in}{2.628940in}}{\pgfqpoint{4.450078in}{2.639990in}}%
\pgfpathcurveto{\pgfqpoint{4.450078in}{2.651040in}}{\pgfqpoint{4.445688in}{2.661639in}}{\pgfqpoint{4.437874in}{2.669453in}}%
\pgfpathcurveto{\pgfqpoint{4.430060in}{2.677266in}}{\pgfqpoint{4.419461in}{2.681656in}}{\pgfqpoint{4.408411in}{2.681656in}}%
\pgfpathcurveto{\pgfqpoint{4.397361in}{2.681656in}}{\pgfqpoint{4.386762in}{2.677266in}}{\pgfqpoint{4.378948in}{2.669453in}}%
\pgfpathcurveto{\pgfqpoint{4.371135in}{2.661639in}}{\pgfqpoint{4.366745in}{2.651040in}}{\pgfqpoint{4.366745in}{2.639990in}}%
\pgfpathcurveto{\pgfqpoint{4.366745in}{2.628940in}}{\pgfqpoint{4.371135in}{2.618341in}}{\pgfqpoint{4.378948in}{2.610527in}}%
\pgfpathcurveto{\pgfqpoint{4.386762in}{2.602713in}}{\pgfqpoint{4.397361in}{2.598323in}}{\pgfqpoint{4.408411in}{2.598323in}}%
\pgfpathclose%
\pgfusepath{stroke,fill}%
\end{pgfscope}%
\begin{pgfscope}%
\pgfpathrectangle{\pgfqpoint{0.481978in}{0.331635in}}{\pgfqpoint{9.300000in}{7.700000in}}%
\pgfusepath{clip}%
\pgfsetbuttcap%
\pgfsetroundjoin%
\definecolor{currentfill}{rgb}{1.000000,0.705882,0.509804}%
\pgfsetfillcolor{currentfill}%
\pgfsetlinewidth{0.481800pt}%
\definecolor{currentstroke}{rgb}{1.000000,1.000000,1.000000}%
\pgfsetstrokecolor{currentstroke}%
\pgfsetdash{}{0pt}%
\pgfpathmoveto{\pgfqpoint{1.290181in}{5.534021in}}%
\pgfpathcurveto{\pgfqpoint{1.301231in}{5.534021in}}{\pgfqpoint{1.311830in}{5.538412in}}{\pgfqpoint{1.319644in}{5.546225in}}%
\pgfpathcurveto{\pgfqpoint{1.327458in}{5.554039in}}{\pgfqpoint{1.331848in}{5.564638in}}{\pgfqpoint{1.331848in}{5.575688in}}%
\pgfpathcurveto{\pgfqpoint{1.331848in}{5.586738in}}{\pgfqpoint{1.327458in}{5.597337in}}{\pgfqpoint{1.319644in}{5.605151in}}%
\pgfpathcurveto{\pgfqpoint{1.311830in}{5.612964in}}{\pgfqpoint{1.301231in}{5.617355in}}{\pgfqpoint{1.290181in}{5.617355in}}%
\pgfpathcurveto{\pgfqpoint{1.279131in}{5.617355in}}{\pgfqpoint{1.268532in}{5.612964in}}{\pgfqpoint{1.260718in}{5.605151in}}%
\pgfpathcurveto{\pgfqpoint{1.252905in}{5.597337in}}{\pgfqpoint{1.248514in}{5.586738in}}{\pgfqpoint{1.248514in}{5.575688in}}%
\pgfpathcurveto{\pgfqpoint{1.248514in}{5.564638in}}{\pgfqpoint{1.252905in}{5.554039in}}{\pgfqpoint{1.260718in}{5.546225in}}%
\pgfpathcurveto{\pgfqpoint{1.268532in}{5.538412in}}{\pgfqpoint{1.279131in}{5.534021in}}{\pgfqpoint{1.290181in}{5.534021in}}%
\pgfpathclose%
\pgfusepath{stroke,fill}%
\end{pgfscope}%
\begin{pgfscope}%
\pgfpathrectangle{\pgfqpoint{0.481978in}{0.331635in}}{\pgfqpoint{9.300000in}{7.700000in}}%
\pgfusepath{clip}%
\pgfsetbuttcap%
\pgfsetroundjoin%
\definecolor{currentfill}{rgb}{1.000000,0.705882,0.509804}%
\pgfsetfillcolor{currentfill}%
\pgfsetlinewidth{0.481800pt}%
\definecolor{currentstroke}{rgb}{1.000000,1.000000,1.000000}%
\pgfsetstrokecolor{currentstroke}%
\pgfsetdash{}{0pt}%
\pgfpathmoveto{\pgfqpoint{2.624493in}{3.220628in}}%
\pgfpathcurveto{\pgfqpoint{2.635543in}{3.220628in}}{\pgfqpoint{2.646142in}{3.225018in}}{\pgfqpoint{2.653956in}{3.232832in}}%
\pgfpathcurveto{\pgfqpoint{2.661769in}{3.240645in}}{\pgfqpoint{2.666160in}{3.251244in}}{\pgfqpoint{2.666160in}{3.262294in}}%
\pgfpathcurveto{\pgfqpoint{2.666160in}{3.273345in}}{\pgfqpoint{2.661769in}{3.283944in}}{\pgfqpoint{2.653956in}{3.291757in}}%
\pgfpathcurveto{\pgfqpoint{2.646142in}{3.299571in}}{\pgfqpoint{2.635543in}{3.303961in}}{\pgfqpoint{2.624493in}{3.303961in}}%
\pgfpathcurveto{\pgfqpoint{2.613443in}{3.303961in}}{\pgfqpoint{2.602844in}{3.299571in}}{\pgfqpoint{2.595030in}{3.291757in}}%
\pgfpathcurveto{\pgfqpoint{2.587216in}{3.283944in}}{\pgfqpoint{2.582826in}{3.273345in}}{\pgfqpoint{2.582826in}{3.262294in}}%
\pgfpathcurveto{\pgfqpoint{2.582826in}{3.251244in}}{\pgfqpoint{2.587216in}{3.240645in}}{\pgfqpoint{2.595030in}{3.232832in}}%
\pgfpathcurveto{\pgfqpoint{2.602844in}{3.225018in}}{\pgfqpoint{2.613443in}{3.220628in}}{\pgfqpoint{2.624493in}{3.220628in}}%
\pgfpathclose%
\pgfusepath{stroke,fill}%
\end{pgfscope}%
\begin{pgfscope}%
\pgfpathrectangle{\pgfqpoint{0.481978in}{0.331635in}}{\pgfqpoint{9.300000in}{7.700000in}}%
\pgfusepath{clip}%
\pgfsetbuttcap%
\pgfsetroundjoin%
\definecolor{currentfill}{rgb}{1.000000,0.705882,0.509804}%
\pgfsetfillcolor{currentfill}%
\pgfsetlinewidth{0.481800pt}%
\definecolor{currentstroke}{rgb}{1.000000,1.000000,1.000000}%
\pgfsetstrokecolor{currentstroke}%
\pgfsetdash{}{0pt}%
\pgfpathmoveto{\pgfqpoint{5.315161in}{4.120952in}}%
\pgfpathcurveto{\pgfqpoint{5.326211in}{4.120952in}}{\pgfqpoint{5.336810in}{4.125342in}}{\pgfqpoint{5.344623in}{4.133156in}}%
\pgfpathcurveto{\pgfqpoint{5.352437in}{4.140970in}}{\pgfqpoint{5.356827in}{4.151569in}}{\pgfqpoint{5.356827in}{4.162619in}}%
\pgfpathcurveto{\pgfqpoint{5.356827in}{4.173669in}}{\pgfqpoint{5.352437in}{4.184268in}}{\pgfqpoint{5.344623in}{4.192082in}}%
\pgfpathcurveto{\pgfqpoint{5.336810in}{4.199895in}}{\pgfqpoint{5.326211in}{4.204285in}}{\pgfqpoint{5.315161in}{4.204285in}}%
\pgfpathcurveto{\pgfqpoint{5.304111in}{4.204285in}}{\pgfqpoint{5.293512in}{4.199895in}}{\pgfqpoint{5.285698in}{4.192082in}}%
\pgfpathcurveto{\pgfqpoint{5.277884in}{4.184268in}}{\pgfqpoint{5.273494in}{4.173669in}}{\pgfqpoint{5.273494in}{4.162619in}}%
\pgfpathcurveto{\pgfqpoint{5.273494in}{4.151569in}}{\pgfqpoint{5.277884in}{4.140970in}}{\pgfqpoint{5.285698in}{4.133156in}}%
\pgfpathcurveto{\pgfqpoint{5.293512in}{4.125342in}}{\pgfqpoint{5.304111in}{4.120952in}}{\pgfqpoint{5.315161in}{4.120952in}}%
\pgfpathclose%
\pgfusepath{stroke,fill}%
\end{pgfscope}%
\begin{pgfscope}%
\pgfpathrectangle{\pgfqpoint{0.481978in}{0.331635in}}{\pgfqpoint{9.300000in}{7.700000in}}%
\pgfusepath{clip}%
\pgfsetbuttcap%
\pgfsetroundjoin%
\definecolor{currentfill}{rgb}{1.000000,0.705882,0.509804}%
\pgfsetfillcolor{currentfill}%
\pgfsetlinewidth{0.481800pt}%
\definecolor{currentstroke}{rgb}{1.000000,1.000000,1.000000}%
\pgfsetstrokecolor{currentstroke}%
\pgfsetdash{}{0pt}%
\pgfpathmoveto{\pgfqpoint{4.062917in}{3.218584in}}%
\pgfpathcurveto{\pgfqpoint{4.073967in}{3.218584in}}{\pgfqpoint{4.084566in}{3.222974in}}{\pgfqpoint{4.092380in}{3.230788in}}%
\pgfpathcurveto{\pgfqpoint{4.100193in}{3.238601in}}{\pgfqpoint{4.104583in}{3.249200in}}{\pgfqpoint{4.104583in}{3.260250in}}%
\pgfpathcurveto{\pgfqpoint{4.104583in}{3.271300in}}{\pgfqpoint{4.100193in}{3.281899in}}{\pgfqpoint{4.092380in}{3.289713in}}%
\pgfpathcurveto{\pgfqpoint{4.084566in}{3.297527in}}{\pgfqpoint{4.073967in}{3.301917in}}{\pgfqpoint{4.062917in}{3.301917in}}%
\pgfpathcurveto{\pgfqpoint{4.051867in}{3.301917in}}{\pgfqpoint{4.041268in}{3.297527in}}{\pgfqpoint{4.033454in}{3.289713in}}%
\pgfpathcurveto{\pgfqpoint{4.025640in}{3.281899in}}{\pgfqpoint{4.021250in}{3.271300in}}{\pgfqpoint{4.021250in}{3.260250in}}%
\pgfpathcurveto{\pgfqpoint{4.021250in}{3.249200in}}{\pgfqpoint{4.025640in}{3.238601in}}{\pgfqpoint{4.033454in}{3.230788in}}%
\pgfpathcurveto{\pgfqpoint{4.041268in}{3.222974in}}{\pgfqpoint{4.051867in}{3.218584in}}{\pgfqpoint{4.062917in}{3.218584in}}%
\pgfpathclose%
\pgfusepath{stroke,fill}%
\end{pgfscope}%
\begin{pgfscope}%
\pgfpathrectangle{\pgfqpoint{0.481978in}{0.331635in}}{\pgfqpoint{9.300000in}{7.700000in}}%
\pgfusepath{clip}%
\pgfsetbuttcap%
\pgfsetroundjoin%
\definecolor{currentfill}{rgb}{1.000000,0.705882,0.509804}%
\pgfsetfillcolor{currentfill}%
\pgfsetlinewidth{0.481800pt}%
\definecolor{currentstroke}{rgb}{1.000000,1.000000,1.000000}%
\pgfsetstrokecolor{currentstroke}%
\pgfsetdash{}{0pt}%
\pgfpathmoveto{\pgfqpoint{4.828561in}{4.028463in}}%
\pgfpathcurveto{\pgfqpoint{4.839611in}{4.028463in}}{\pgfqpoint{4.850211in}{4.032853in}}{\pgfqpoint{4.858024in}{4.040667in}}%
\pgfpathcurveto{\pgfqpoint{4.865838in}{4.048480in}}{\pgfqpoint{4.870228in}{4.059079in}}{\pgfqpoint{4.870228in}{4.070130in}}%
\pgfpathcurveto{\pgfqpoint{4.870228in}{4.081180in}}{\pgfqpoint{4.865838in}{4.091779in}}{\pgfqpoint{4.858024in}{4.099592in}}%
\pgfpathcurveto{\pgfqpoint{4.850211in}{4.107406in}}{\pgfqpoint{4.839611in}{4.111796in}}{\pgfqpoint{4.828561in}{4.111796in}}%
\pgfpathcurveto{\pgfqpoint{4.817511in}{4.111796in}}{\pgfqpoint{4.806912in}{4.107406in}}{\pgfqpoint{4.799099in}{4.099592in}}%
\pgfpathcurveto{\pgfqpoint{4.791285in}{4.091779in}}{\pgfqpoint{4.786895in}{4.081180in}}{\pgfqpoint{4.786895in}{4.070130in}}%
\pgfpathcurveto{\pgfqpoint{4.786895in}{4.059079in}}{\pgfqpoint{4.791285in}{4.048480in}}{\pgfqpoint{4.799099in}{4.040667in}}%
\pgfpathcurveto{\pgfqpoint{4.806912in}{4.032853in}}{\pgfqpoint{4.817511in}{4.028463in}}{\pgfqpoint{4.828561in}{4.028463in}}%
\pgfpathclose%
\pgfusepath{stroke,fill}%
\end{pgfscope}%
\begin{pgfscope}%
\pgfpathrectangle{\pgfqpoint{0.481978in}{0.331635in}}{\pgfqpoint{9.300000in}{7.700000in}}%
\pgfusepath{clip}%
\pgfsetbuttcap%
\pgfsetroundjoin%
\definecolor{currentfill}{rgb}{1.000000,0.705882,0.509804}%
\pgfsetfillcolor{currentfill}%
\pgfsetlinewidth{0.481800pt}%
\definecolor{currentstroke}{rgb}{1.000000,1.000000,1.000000}%
\pgfsetstrokecolor{currentstroke}%
\pgfsetdash{}{0pt}%
\pgfpathmoveto{\pgfqpoint{2.387650in}{3.484085in}}%
\pgfpathcurveto{\pgfqpoint{2.398701in}{3.484085in}}{\pgfqpoint{2.409300in}{3.488475in}}{\pgfqpoint{2.417113in}{3.496289in}}%
\pgfpathcurveto{\pgfqpoint{2.424927in}{3.504102in}}{\pgfqpoint{2.429317in}{3.514701in}}{\pgfqpoint{2.429317in}{3.525751in}}%
\pgfpathcurveto{\pgfqpoint{2.429317in}{3.536801in}}{\pgfqpoint{2.424927in}{3.547400in}}{\pgfqpoint{2.417113in}{3.555214in}}%
\pgfpathcurveto{\pgfqpoint{2.409300in}{3.563028in}}{\pgfqpoint{2.398701in}{3.567418in}}{\pgfqpoint{2.387650in}{3.567418in}}%
\pgfpathcurveto{\pgfqpoint{2.376600in}{3.567418in}}{\pgfqpoint{2.366001in}{3.563028in}}{\pgfqpoint{2.358188in}{3.555214in}}%
\pgfpathcurveto{\pgfqpoint{2.350374in}{3.547400in}}{\pgfqpoint{2.345984in}{3.536801in}}{\pgfqpoint{2.345984in}{3.525751in}}%
\pgfpathcurveto{\pgfqpoint{2.345984in}{3.514701in}}{\pgfqpoint{2.350374in}{3.504102in}}{\pgfqpoint{2.358188in}{3.496289in}}%
\pgfpathcurveto{\pgfqpoint{2.366001in}{3.488475in}}{\pgfqpoint{2.376600in}{3.484085in}}{\pgfqpoint{2.387650in}{3.484085in}}%
\pgfpathclose%
\pgfusepath{stroke,fill}%
\end{pgfscope}%
\begin{pgfscope}%
\pgfpathrectangle{\pgfqpoint{0.481978in}{0.331635in}}{\pgfqpoint{9.300000in}{7.700000in}}%
\pgfusepath{clip}%
\pgfsetbuttcap%
\pgfsetroundjoin%
\definecolor{currentfill}{rgb}{1.000000,0.705882,0.509804}%
\pgfsetfillcolor{currentfill}%
\pgfsetlinewidth{0.481800pt}%
\definecolor{currentstroke}{rgb}{1.000000,1.000000,1.000000}%
\pgfsetstrokecolor{currentstroke}%
\pgfsetdash{}{0pt}%
\pgfpathmoveto{\pgfqpoint{3.134591in}{4.885871in}}%
\pgfpathcurveto{\pgfqpoint{3.145641in}{4.885871in}}{\pgfqpoint{3.156240in}{4.890261in}}{\pgfqpoint{3.164054in}{4.898075in}}%
\pgfpathcurveto{\pgfqpoint{3.171867in}{4.905888in}}{\pgfqpoint{3.176258in}{4.916487in}}{\pgfqpoint{3.176258in}{4.927538in}}%
\pgfpathcurveto{\pgfqpoint{3.176258in}{4.938588in}}{\pgfqpoint{3.171867in}{4.949187in}}{\pgfqpoint{3.164054in}{4.957000in}}%
\pgfpathcurveto{\pgfqpoint{3.156240in}{4.964814in}}{\pgfqpoint{3.145641in}{4.969204in}}{\pgfqpoint{3.134591in}{4.969204in}}%
\pgfpathcurveto{\pgfqpoint{3.123541in}{4.969204in}}{\pgfqpoint{3.112942in}{4.964814in}}{\pgfqpoint{3.105128in}{4.957000in}}%
\pgfpathcurveto{\pgfqpoint{3.097315in}{4.949187in}}{\pgfqpoint{3.092924in}{4.938588in}}{\pgfqpoint{3.092924in}{4.927538in}}%
\pgfpathcurveto{\pgfqpoint{3.092924in}{4.916487in}}{\pgfqpoint{3.097315in}{4.905888in}}{\pgfqpoint{3.105128in}{4.898075in}}%
\pgfpathcurveto{\pgfqpoint{3.112942in}{4.890261in}}{\pgfqpoint{3.123541in}{4.885871in}}{\pgfqpoint{3.134591in}{4.885871in}}%
\pgfpathclose%
\pgfusepath{stroke,fill}%
\end{pgfscope}%
\begin{pgfscope}%
\pgfpathrectangle{\pgfqpoint{0.481978in}{0.331635in}}{\pgfqpoint{9.300000in}{7.700000in}}%
\pgfusepath{clip}%
\pgfsetbuttcap%
\pgfsetroundjoin%
\definecolor{currentfill}{rgb}{1.000000,0.705882,0.509804}%
\pgfsetfillcolor{currentfill}%
\pgfsetlinewidth{0.481800pt}%
\definecolor{currentstroke}{rgb}{1.000000,1.000000,1.000000}%
\pgfsetstrokecolor{currentstroke}%
\pgfsetdash{}{0pt}%
\pgfpathmoveto{\pgfqpoint{1.638268in}{4.660633in}}%
\pgfpathcurveto{\pgfqpoint{1.649318in}{4.660633in}}{\pgfqpoint{1.659917in}{4.665024in}}{\pgfqpoint{1.667731in}{4.672837in}}%
\pgfpathcurveto{\pgfqpoint{1.675544in}{4.680651in}}{\pgfqpoint{1.679935in}{4.691250in}}{\pgfqpoint{1.679935in}{4.702300in}}%
\pgfpathcurveto{\pgfqpoint{1.679935in}{4.713350in}}{\pgfqpoint{1.675544in}{4.723949in}}{\pgfqpoint{1.667731in}{4.731763in}}%
\pgfpathcurveto{\pgfqpoint{1.659917in}{4.739576in}}{\pgfqpoint{1.649318in}{4.743967in}}{\pgfqpoint{1.638268in}{4.743967in}}%
\pgfpathcurveto{\pgfqpoint{1.627218in}{4.743967in}}{\pgfqpoint{1.616619in}{4.739576in}}{\pgfqpoint{1.608805in}{4.731763in}}%
\pgfpathcurveto{\pgfqpoint{1.600992in}{4.723949in}}{\pgfqpoint{1.596601in}{4.713350in}}{\pgfqpoint{1.596601in}{4.702300in}}%
\pgfpathcurveto{\pgfqpoint{1.596601in}{4.691250in}}{\pgfqpoint{1.600992in}{4.680651in}}{\pgfqpoint{1.608805in}{4.672837in}}%
\pgfpathcurveto{\pgfqpoint{1.616619in}{4.665024in}}{\pgfqpoint{1.627218in}{4.660633in}}{\pgfqpoint{1.638268in}{4.660633in}}%
\pgfpathclose%
\pgfusepath{stroke,fill}%
\end{pgfscope}%
\begin{pgfscope}%
\pgfpathrectangle{\pgfqpoint{0.481978in}{0.331635in}}{\pgfqpoint{9.300000in}{7.700000in}}%
\pgfusepath{clip}%
\pgfsetbuttcap%
\pgfsetroundjoin%
\definecolor{currentfill}{rgb}{1.000000,0.705882,0.509804}%
\pgfsetfillcolor{currentfill}%
\pgfsetlinewidth{0.481800pt}%
\definecolor{currentstroke}{rgb}{1.000000,1.000000,1.000000}%
\pgfsetstrokecolor{currentstroke}%
\pgfsetdash{}{0pt}%
\pgfpathmoveto{\pgfqpoint{4.067183in}{6.795571in}}%
\pgfpathcurveto{\pgfqpoint{4.078233in}{6.795571in}}{\pgfqpoint{4.088832in}{6.799962in}}{\pgfqpoint{4.096645in}{6.807775in}}%
\pgfpathcurveto{\pgfqpoint{4.104459in}{6.815589in}}{\pgfqpoint{4.108849in}{6.826188in}}{\pgfqpoint{4.108849in}{6.837238in}}%
\pgfpathcurveto{\pgfqpoint{4.108849in}{6.848288in}}{\pgfqpoint{4.104459in}{6.858887in}}{\pgfqpoint{4.096645in}{6.866701in}}%
\pgfpathcurveto{\pgfqpoint{4.088832in}{6.874514in}}{\pgfqpoint{4.078233in}{6.878905in}}{\pgfqpoint{4.067183in}{6.878905in}}%
\pgfpathcurveto{\pgfqpoint{4.056132in}{6.878905in}}{\pgfqpoint{4.045533in}{6.874514in}}{\pgfqpoint{4.037720in}{6.866701in}}%
\pgfpathcurveto{\pgfqpoint{4.029906in}{6.858887in}}{\pgfqpoint{4.025516in}{6.848288in}}{\pgfqpoint{4.025516in}{6.837238in}}%
\pgfpathcurveto{\pgfqpoint{4.025516in}{6.826188in}}{\pgfqpoint{4.029906in}{6.815589in}}{\pgfqpoint{4.037720in}{6.807775in}}%
\pgfpathcurveto{\pgfqpoint{4.045533in}{6.799962in}}{\pgfqpoint{4.056132in}{6.795571in}}{\pgfqpoint{4.067183in}{6.795571in}}%
\pgfpathclose%
\pgfusepath{stroke,fill}%
\end{pgfscope}%
\begin{pgfscope}%
\pgfpathrectangle{\pgfqpoint{0.481978in}{0.331635in}}{\pgfqpoint{9.300000in}{7.700000in}}%
\pgfusepath{clip}%
\pgfsetbuttcap%
\pgfsetroundjoin%
\definecolor{currentfill}{rgb}{1.000000,0.705882,0.509804}%
\pgfsetfillcolor{currentfill}%
\pgfsetlinewidth{0.481800pt}%
\definecolor{currentstroke}{rgb}{1.000000,1.000000,1.000000}%
\pgfsetstrokecolor{currentstroke}%
\pgfsetdash{}{0pt}%
\pgfpathmoveto{\pgfqpoint{3.079878in}{3.877328in}}%
\pgfpathcurveto{\pgfqpoint{3.090928in}{3.877328in}}{\pgfqpoint{3.101527in}{3.881718in}}{\pgfqpoint{3.109341in}{3.889532in}}%
\pgfpathcurveto{\pgfqpoint{3.117155in}{3.897345in}}{\pgfqpoint{3.121545in}{3.907944in}}{\pgfqpoint{3.121545in}{3.918994in}}%
\pgfpathcurveto{\pgfqpoint{3.121545in}{3.930045in}}{\pgfqpoint{3.117155in}{3.940644in}}{\pgfqpoint{3.109341in}{3.948457in}}%
\pgfpathcurveto{\pgfqpoint{3.101527in}{3.956271in}}{\pgfqpoint{3.090928in}{3.960661in}}{\pgfqpoint{3.079878in}{3.960661in}}%
\pgfpathcurveto{\pgfqpoint{3.068828in}{3.960661in}}{\pgfqpoint{3.058229in}{3.956271in}}{\pgfqpoint{3.050416in}{3.948457in}}%
\pgfpathcurveto{\pgfqpoint{3.042602in}{3.940644in}}{\pgfqpoint{3.038212in}{3.930045in}}{\pgfqpoint{3.038212in}{3.918994in}}%
\pgfpathcurveto{\pgfqpoint{3.038212in}{3.907944in}}{\pgfqpoint{3.042602in}{3.897345in}}{\pgfqpoint{3.050416in}{3.889532in}}%
\pgfpathcurveto{\pgfqpoint{3.058229in}{3.881718in}}{\pgfqpoint{3.068828in}{3.877328in}}{\pgfqpoint{3.079878in}{3.877328in}}%
\pgfpathclose%
\pgfusepath{stroke,fill}%
\end{pgfscope}%
\begin{pgfscope}%
\pgfpathrectangle{\pgfqpoint{0.481978in}{0.331635in}}{\pgfqpoint{9.300000in}{7.700000in}}%
\pgfusepath{clip}%
\pgfsetbuttcap%
\pgfsetroundjoin%
\definecolor{currentfill}{rgb}{1.000000,0.705882,0.509804}%
\pgfsetfillcolor{currentfill}%
\pgfsetlinewidth{0.481800pt}%
\definecolor{currentstroke}{rgb}{1.000000,1.000000,1.000000}%
\pgfsetstrokecolor{currentstroke}%
\pgfsetdash{}{0pt}%
\pgfpathmoveto{\pgfqpoint{3.762277in}{2.592935in}}%
\pgfpathcurveto{\pgfqpoint{3.773328in}{2.592935in}}{\pgfqpoint{3.783927in}{2.597326in}}{\pgfqpoint{3.791740in}{2.605139in}}%
\pgfpathcurveto{\pgfqpoint{3.799554in}{2.612953in}}{\pgfqpoint{3.803944in}{2.623552in}}{\pgfqpoint{3.803944in}{2.634602in}}%
\pgfpathcurveto{\pgfqpoint{3.803944in}{2.645652in}}{\pgfqpoint{3.799554in}{2.656251in}}{\pgfqpoint{3.791740in}{2.664065in}}%
\pgfpathcurveto{\pgfqpoint{3.783927in}{2.671878in}}{\pgfqpoint{3.773328in}{2.676269in}}{\pgfqpoint{3.762277in}{2.676269in}}%
\pgfpathcurveto{\pgfqpoint{3.751227in}{2.676269in}}{\pgfqpoint{3.740628in}{2.671878in}}{\pgfqpoint{3.732815in}{2.664065in}}%
\pgfpathcurveto{\pgfqpoint{3.725001in}{2.656251in}}{\pgfqpoint{3.720611in}{2.645652in}}{\pgfqpoint{3.720611in}{2.634602in}}%
\pgfpathcurveto{\pgfqpoint{3.720611in}{2.623552in}}{\pgfqpoint{3.725001in}{2.612953in}}{\pgfqpoint{3.732815in}{2.605139in}}%
\pgfpathcurveto{\pgfqpoint{3.740628in}{2.597326in}}{\pgfqpoint{3.751227in}{2.592935in}}{\pgfqpoint{3.762277in}{2.592935in}}%
\pgfpathclose%
\pgfusepath{stroke,fill}%
\end{pgfscope}%
\begin{pgfscope}%
\pgfpathrectangle{\pgfqpoint{0.481978in}{0.331635in}}{\pgfqpoint{9.300000in}{7.700000in}}%
\pgfusepath{clip}%
\pgfsetbuttcap%
\pgfsetroundjoin%
\definecolor{currentfill}{rgb}{1.000000,0.705882,0.509804}%
\pgfsetfillcolor{currentfill}%
\pgfsetlinewidth{0.481800pt}%
\definecolor{currentstroke}{rgb}{1.000000,1.000000,1.000000}%
\pgfsetstrokecolor{currentstroke}%
\pgfsetdash{}{0pt}%
\pgfpathmoveto{\pgfqpoint{3.580814in}{3.862560in}}%
\pgfpathcurveto{\pgfqpoint{3.591864in}{3.862560in}}{\pgfqpoint{3.602463in}{3.866950in}}{\pgfqpoint{3.610277in}{3.874764in}}%
\pgfpathcurveto{\pgfqpoint{3.618091in}{3.882578in}}{\pgfqpoint{3.622481in}{3.893177in}}{\pgfqpoint{3.622481in}{3.904227in}}%
\pgfpathcurveto{\pgfqpoint{3.622481in}{3.915277in}}{\pgfqpoint{3.618091in}{3.925876in}}{\pgfqpoint{3.610277in}{3.933689in}}%
\pgfpathcurveto{\pgfqpoint{3.602463in}{3.941503in}}{\pgfqpoint{3.591864in}{3.945893in}}{\pgfqpoint{3.580814in}{3.945893in}}%
\pgfpathcurveto{\pgfqpoint{3.569764in}{3.945893in}}{\pgfqpoint{3.559165in}{3.941503in}}{\pgfqpoint{3.551352in}{3.933689in}}%
\pgfpathcurveto{\pgfqpoint{3.543538in}{3.925876in}}{\pgfqpoint{3.539148in}{3.915277in}}{\pgfqpoint{3.539148in}{3.904227in}}%
\pgfpathcurveto{\pgfqpoint{3.539148in}{3.893177in}}{\pgfqpoint{3.543538in}{3.882578in}}{\pgfqpoint{3.551352in}{3.874764in}}%
\pgfpathcurveto{\pgfqpoint{3.559165in}{3.866950in}}{\pgfqpoint{3.569764in}{3.862560in}}{\pgfqpoint{3.580814in}{3.862560in}}%
\pgfpathclose%
\pgfusepath{stroke,fill}%
\end{pgfscope}%
\begin{pgfscope}%
\pgfpathrectangle{\pgfqpoint{0.481978in}{0.331635in}}{\pgfqpoint{9.300000in}{7.700000in}}%
\pgfusepath{clip}%
\pgfsetbuttcap%
\pgfsetroundjoin%
\definecolor{currentfill}{rgb}{1.000000,0.705882,0.509804}%
\pgfsetfillcolor{currentfill}%
\pgfsetlinewidth{0.481800pt}%
\definecolor{currentstroke}{rgb}{1.000000,1.000000,1.000000}%
\pgfsetstrokecolor{currentstroke}%
\pgfsetdash{}{0pt}%
\pgfpathmoveto{\pgfqpoint{4.361455in}{2.789942in}}%
\pgfpathcurveto{\pgfqpoint{4.372505in}{2.789942in}}{\pgfqpoint{4.383104in}{2.794332in}}{\pgfqpoint{4.390918in}{2.802146in}}%
\pgfpathcurveto{\pgfqpoint{4.398732in}{2.809960in}}{\pgfqpoint{4.403122in}{2.820559in}}{\pgfqpoint{4.403122in}{2.831609in}}%
\pgfpathcurveto{\pgfqpoint{4.403122in}{2.842659in}}{\pgfqpoint{4.398732in}{2.853258in}}{\pgfqpoint{4.390918in}{2.861072in}}%
\pgfpathcurveto{\pgfqpoint{4.383104in}{2.868885in}}{\pgfqpoint{4.372505in}{2.873276in}}{\pgfqpoint{4.361455in}{2.873276in}}%
\pgfpathcurveto{\pgfqpoint{4.350405in}{2.873276in}}{\pgfqpoint{4.339806in}{2.868885in}}{\pgfqpoint{4.331992in}{2.861072in}}%
\pgfpathcurveto{\pgfqpoint{4.324179in}{2.853258in}}{\pgfqpoint{4.319789in}{2.842659in}}{\pgfqpoint{4.319789in}{2.831609in}}%
\pgfpathcurveto{\pgfqpoint{4.319789in}{2.820559in}}{\pgfqpoint{4.324179in}{2.809960in}}{\pgfqpoint{4.331992in}{2.802146in}}%
\pgfpathcurveto{\pgfqpoint{4.339806in}{2.794332in}}{\pgfqpoint{4.350405in}{2.789942in}}{\pgfqpoint{4.361455in}{2.789942in}}%
\pgfpathclose%
\pgfusepath{stroke,fill}%
\end{pgfscope}%
\begin{pgfscope}%
\pgfpathrectangle{\pgfqpoint{0.481978in}{0.331635in}}{\pgfqpoint{9.300000in}{7.700000in}}%
\pgfusepath{clip}%
\pgfsetbuttcap%
\pgfsetroundjoin%
\definecolor{currentfill}{rgb}{1.000000,0.705882,0.509804}%
\pgfsetfillcolor{currentfill}%
\pgfsetlinewidth{0.481800pt}%
\definecolor{currentstroke}{rgb}{1.000000,1.000000,1.000000}%
\pgfsetstrokecolor{currentstroke}%
\pgfsetdash{}{0pt}%
\pgfpathmoveto{\pgfqpoint{3.821252in}{6.879688in}}%
\pgfpathcurveto{\pgfqpoint{3.832302in}{6.879688in}}{\pgfqpoint{3.842901in}{6.884078in}}{\pgfqpoint{3.850714in}{6.891892in}}%
\pgfpathcurveto{\pgfqpoint{3.858528in}{6.899706in}}{\pgfqpoint{3.862918in}{6.910305in}}{\pgfqpoint{3.862918in}{6.921355in}}%
\pgfpathcurveto{\pgfqpoint{3.862918in}{6.932405in}}{\pgfqpoint{3.858528in}{6.943004in}}{\pgfqpoint{3.850714in}{6.950817in}}%
\pgfpathcurveto{\pgfqpoint{3.842901in}{6.958631in}}{\pgfqpoint{3.832302in}{6.963021in}}{\pgfqpoint{3.821252in}{6.963021in}}%
\pgfpathcurveto{\pgfqpoint{3.810202in}{6.963021in}}{\pgfqpoint{3.799602in}{6.958631in}}{\pgfqpoint{3.791789in}{6.950817in}}%
\pgfpathcurveto{\pgfqpoint{3.783975in}{6.943004in}}{\pgfqpoint{3.779585in}{6.932405in}}{\pgfqpoint{3.779585in}{6.921355in}}%
\pgfpathcurveto{\pgfqpoint{3.779585in}{6.910305in}}{\pgfqpoint{3.783975in}{6.899706in}}{\pgfqpoint{3.791789in}{6.891892in}}%
\pgfpathcurveto{\pgfqpoint{3.799602in}{6.884078in}}{\pgfqpoint{3.810202in}{6.879688in}}{\pgfqpoint{3.821252in}{6.879688in}}%
\pgfpathclose%
\pgfusepath{stroke,fill}%
\end{pgfscope}%
\begin{pgfscope}%
\pgfpathrectangle{\pgfqpoint{0.481978in}{0.331635in}}{\pgfqpoint{9.300000in}{7.700000in}}%
\pgfusepath{clip}%
\pgfsetbuttcap%
\pgfsetroundjoin%
\definecolor{currentfill}{rgb}{1.000000,0.705882,0.509804}%
\pgfsetfillcolor{currentfill}%
\pgfsetlinewidth{0.481800pt}%
\definecolor{currentstroke}{rgb}{1.000000,1.000000,1.000000}%
\pgfsetstrokecolor{currentstroke}%
\pgfsetdash{}{0pt}%
\pgfpathmoveto{\pgfqpoint{2.933293in}{5.599270in}}%
\pgfpathcurveto{\pgfqpoint{2.944344in}{5.599270in}}{\pgfqpoint{2.954943in}{5.603660in}}{\pgfqpoint{2.962756in}{5.611473in}}%
\pgfpathcurveto{\pgfqpoint{2.970570in}{5.619287in}}{\pgfqpoint{2.974960in}{5.629886in}}{\pgfqpoint{2.974960in}{5.640936in}}%
\pgfpathcurveto{\pgfqpoint{2.974960in}{5.651986in}}{\pgfqpoint{2.970570in}{5.662585in}}{\pgfqpoint{2.962756in}{5.670399in}}%
\pgfpathcurveto{\pgfqpoint{2.954943in}{5.678213in}}{\pgfqpoint{2.944344in}{5.682603in}}{\pgfqpoint{2.933293in}{5.682603in}}%
\pgfpathcurveto{\pgfqpoint{2.922243in}{5.682603in}}{\pgfqpoint{2.911644in}{5.678213in}}{\pgfqpoint{2.903831in}{5.670399in}}%
\pgfpathcurveto{\pgfqpoint{2.896017in}{5.662585in}}{\pgfqpoint{2.891627in}{5.651986in}}{\pgfqpoint{2.891627in}{5.640936in}}%
\pgfpathcurveto{\pgfqpoint{2.891627in}{5.629886in}}{\pgfqpoint{2.896017in}{5.619287in}}{\pgfqpoint{2.903831in}{5.611473in}}%
\pgfpathcurveto{\pgfqpoint{2.911644in}{5.603660in}}{\pgfqpoint{2.922243in}{5.599270in}}{\pgfqpoint{2.933293in}{5.599270in}}%
\pgfpathclose%
\pgfusepath{stroke,fill}%
\end{pgfscope}%
\begin{pgfscope}%
\pgfpathrectangle{\pgfqpoint{0.481978in}{0.331635in}}{\pgfqpoint{9.300000in}{7.700000in}}%
\pgfusepath{clip}%
\pgfsetbuttcap%
\pgfsetroundjoin%
\definecolor{currentfill}{rgb}{1.000000,0.705882,0.509804}%
\pgfsetfillcolor{currentfill}%
\pgfsetlinewidth{0.481800pt}%
\definecolor{currentstroke}{rgb}{1.000000,1.000000,1.000000}%
\pgfsetstrokecolor{currentstroke}%
\pgfsetdash{}{0pt}%
\pgfpathmoveto{\pgfqpoint{4.420071in}{4.726582in}}%
\pgfpathcurveto{\pgfqpoint{4.431121in}{4.726582in}}{\pgfqpoint{4.441720in}{4.730973in}}{\pgfqpoint{4.449534in}{4.738786in}}%
\pgfpathcurveto{\pgfqpoint{4.457348in}{4.746600in}}{\pgfqpoint{4.461738in}{4.757199in}}{\pgfqpoint{4.461738in}{4.768249in}}%
\pgfpathcurveto{\pgfqpoint{4.461738in}{4.779299in}}{\pgfqpoint{4.457348in}{4.789898in}}{\pgfqpoint{4.449534in}{4.797712in}}%
\pgfpathcurveto{\pgfqpoint{4.441720in}{4.805525in}}{\pgfqpoint{4.431121in}{4.809916in}}{\pgfqpoint{4.420071in}{4.809916in}}%
\pgfpathcurveto{\pgfqpoint{4.409021in}{4.809916in}}{\pgfqpoint{4.398422in}{4.805525in}}{\pgfqpoint{4.390609in}{4.797712in}}%
\pgfpathcurveto{\pgfqpoint{4.382795in}{4.789898in}}{\pgfqpoint{4.378405in}{4.779299in}}{\pgfqpoint{4.378405in}{4.768249in}}%
\pgfpathcurveto{\pgfqpoint{4.378405in}{4.757199in}}{\pgfqpoint{4.382795in}{4.746600in}}{\pgfqpoint{4.390609in}{4.738786in}}%
\pgfpathcurveto{\pgfqpoint{4.398422in}{4.730973in}}{\pgfqpoint{4.409021in}{4.726582in}}{\pgfqpoint{4.420071in}{4.726582in}}%
\pgfpathclose%
\pgfusepath{stroke,fill}%
\end{pgfscope}%
\begin{pgfscope}%
\pgfpathrectangle{\pgfqpoint{0.481978in}{0.331635in}}{\pgfqpoint{9.300000in}{7.700000in}}%
\pgfusepath{clip}%
\pgfsetbuttcap%
\pgfsetroundjoin%
\definecolor{currentfill}{rgb}{1.000000,0.705882,0.509804}%
\pgfsetfillcolor{currentfill}%
\pgfsetlinewidth{0.481800pt}%
\definecolor{currentstroke}{rgb}{1.000000,1.000000,1.000000}%
\pgfsetstrokecolor{currentstroke}%
\pgfsetdash{}{0pt}%
\pgfpathmoveto{\pgfqpoint{2.246703in}{4.966839in}}%
\pgfpathcurveto{\pgfqpoint{2.257753in}{4.966839in}}{\pgfqpoint{2.268352in}{4.971230in}}{\pgfqpoint{2.276166in}{4.979043in}}%
\pgfpathcurveto{\pgfqpoint{2.283979in}{4.986857in}}{\pgfqpoint{2.288370in}{4.997456in}}{\pgfqpoint{2.288370in}{5.008506in}}%
\pgfpathcurveto{\pgfqpoint{2.288370in}{5.019556in}}{\pgfqpoint{2.283979in}{5.030155in}}{\pgfqpoint{2.276166in}{5.037969in}}%
\pgfpathcurveto{\pgfqpoint{2.268352in}{5.045782in}}{\pgfqpoint{2.257753in}{5.050173in}}{\pgfqpoint{2.246703in}{5.050173in}}%
\pgfpathcurveto{\pgfqpoint{2.235653in}{5.050173in}}{\pgfqpoint{2.225054in}{5.045782in}}{\pgfqpoint{2.217240in}{5.037969in}}%
\pgfpathcurveto{\pgfqpoint{2.209427in}{5.030155in}}{\pgfqpoint{2.205036in}{5.019556in}}{\pgfqpoint{2.205036in}{5.008506in}}%
\pgfpathcurveto{\pgfqpoint{2.205036in}{4.997456in}}{\pgfqpoint{2.209427in}{4.986857in}}{\pgfqpoint{2.217240in}{4.979043in}}%
\pgfpathcurveto{\pgfqpoint{2.225054in}{4.971230in}}{\pgfqpoint{2.235653in}{4.966839in}}{\pgfqpoint{2.246703in}{4.966839in}}%
\pgfpathclose%
\pgfusepath{stroke,fill}%
\end{pgfscope}%
\begin{pgfscope}%
\pgfpathrectangle{\pgfqpoint{0.481978in}{0.331635in}}{\pgfqpoint{9.300000in}{7.700000in}}%
\pgfusepath{clip}%
\pgfsetbuttcap%
\pgfsetroundjoin%
\definecolor{currentfill}{rgb}{1.000000,0.705882,0.509804}%
\pgfsetfillcolor{currentfill}%
\pgfsetlinewidth{0.481800pt}%
\definecolor{currentstroke}{rgb}{1.000000,1.000000,1.000000}%
\pgfsetstrokecolor{currentstroke}%
\pgfsetdash{}{0pt}%
\pgfpathmoveto{\pgfqpoint{4.205087in}{5.595225in}}%
\pgfpathcurveto{\pgfqpoint{4.216137in}{5.595225in}}{\pgfqpoint{4.226736in}{5.599616in}}{\pgfqpoint{4.234550in}{5.607429in}}%
\pgfpathcurveto{\pgfqpoint{4.242363in}{5.615243in}}{\pgfqpoint{4.246753in}{5.625842in}}{\pgfqpoint{4.246753in}{5.636892in}}%
\pgfpathcurveto{\pgfqpoint{4.246753in}{5.647942in}}{\pgfqpoint{4.242363in}{5.658541in}}{\pgfqpoint{4.234550in}{5.666355in}}%
\pgfpathcurveto{\pgfqpoint{4.226736in}{5.674168in}}{\pgfqpoint{4.216137in}{5.678559in}}{\pgfqpoint{4.205087in}{5.678559in}}%
\pgfpathcurveto{\pgfqpoint{4.194037in}{5.678559in}}{\pgfqpoint{4.183438in}{5.674168in}}{\pgfqpoint{4.175624in}{5.666355in}}%
\pgfpathcurveto{\pgfqpoint{4.167810in}{5.658541in}}{\pgfqpoint{4.163420in}{5.647942in}}{\pgfqpoint{4.163420in}{5.636892in}}%
\pgfpathcurveto{\pgfqpoint{4.163420in}{5.625842in}}{\pgfqpoint{4.167810in}{5.615243in}}{\pgfqpoint{4.175624in}{5.607429in}}%
\pgfpathcurveto{\pgfqpoint{4.183438in}{5.599616in}}{\pgfqpoint{4.194037in}{5.595225in}}{\pgfqpoint{4.205087in}{5.595225in}}%
\pgfpathclose%
\pgfusepath{stroke,fill}%
\end{pgfscope}%
\begin{pgfscope}%
\pgfpathrectangle{\pgfqpoint{0.481978in}{0.331635in}}{\pgfqpoint{9.300000in}{7.700000in}}%
\pgfusepath{clip}%
\pgfsetbuttcap%
\pgfsetroundjoin%
\definecolor{currentfill}{rgb}{1.000000,0.705882,0.509804}%
\pgfsetfillcolor{currentfill}%
\pgfsetlinewidth{0.481800pt}%
\definecolor{currentstroke}{rgb}{1.000000,1.000000,1.000000}%
\pgfsetstrokecolor{currentstroke}%
\pgfsetdash{}{0pt}%
\pgfpathmoveto{\pgfqpoint{3.838690in}{4.184095in}}%
\pgfpathcurveto{\pgfqpoint{3.849740in}{4.184095in}}{\pgfqpoint{3.860339in}{4.188485in}}{\pgfqpoint{3.868152in}{4.196298in}}%
\pgfpathcurveto{\pgfqpoint{3.875966in}{4.204112in}}{\pgfqpoint{3.880356in}{4.214711in}}{\pgfqpoint{3.880356in}{4.225761in}}%
\pgfpathcurveto{\pgfqpoint{3.880356in}{4.236811in}}{\pgfqpoint{3.875966in}{4.247410in}}{\pgfqpoint{3.868152in}{4.255224in}}%
\pgfpathcurveto{\pgfqpoint{3.860339in}{4.263038in}}{\pgfqpoint{3.849740in}{4.267428in}}{\pgfqpoint{3.838690in}{4.267428in}}%
\pgfpathcurveto{\pgfqpoint{3.827639in}{4.267428in}}{\pgfqpoint{3.817040in}{4.263038in}}{\pgfqpoint{3.809227in}{4.255224in}}%
\pgfpathcurveto{\pgfqpoint{3.801413in}{4.247410in}}{\pgfqpoint{3.797023in}{4.236811in}}{\pgfqpoint{3.797023in}{4.225761in}}%
\pgfpathcurveto{\pgfqpoint{3.797023in}{4.214711in}}{\pgfqpoint{3.801413in}{4.204112in}}{\pgfqpoint{3.809227in}{4.196298in}}%
\pgfpathcurveto{\pgfqpoint{3.817040in}{4.188485in}}{\pgfqpoint{3.827639in}{4.184095in}}{\pgfqpoint{3.838690in}{4.184095in}}%
\pgfpathclose%
\pgfusepath{stroke,fill}%
\end{pgfscope}%
\begin{pgfscope}%
\pgfpathrectangle{\pgfqpoint{0.481978in}{0.331635in}}{\pgfqpoint{9.300000in}{7.700000in}}%
\pgfusepath{clip}%
\pgfsetbuttcap%
\pgfsetroundjoin%
\definecolor{currentfill}{rgb}{1.000000,0.705882,0.509804}%
\pgfsetfillcolor{currentfill}%
\pgfsetlinewidth{0.481800pt}%
\definecolor{currentstroke}{rgb}{1.000000,1.000000,1.000000}%
\pgfsetstrokecolor{currentstroke}%
\pgfsetdash{}{0pt}%
\pgfpathmoveto{\pgfqpoint{1.293480in}{5.530196in}}%
\pgfpathcurveto{\pgfqpoint{1.304530in}{5.530196in}}{\pgfqpoint{1.315129in}{5.534586in}}{\pgfqpoint{1.322943in}{5.542400in}}%
\pgfpathcurveto{\pgfqpoint{1.330756in}{5.550213in}}{\pgfqpoint{1.335147in}{5.560812in}}{\pgfqpoint{1.335147in}{5.571862in}}%
\pgfpathcurveto{\pgfqpoint{1.335147in}{5.582913in}}{\pgfqpoint{1.330756in}{5.593512in}}{\pgfqpoint{1.322943in}{5.601325in}}%
\pgfpathcurveto{\pgfqpoint{1.315129in}{5.609139in}}{\pgfqpoint{1.304530in}{5.613529in}}{\pgfqpoint{1.293480in}{5.613529in}}%
\pgfpathcurveto{\pgfqpoint{1.282430in}{5.613529in}}{\pgfqpoint{1.271831in}{5.609139in}}{\pgfqpoint{1.264017in}{5.601325in}}%
\pgfpathcurveto{\pgfqpoint{1.256204in}{5.593512in}}{\pgfqpoint{1.251813in}{5.582913in}}{\pgfqpoint{1.251813in}{5.571862in}}%
\pgfpathcurveto{\pgfqpoint{1.251813in}{5.560812in}}{\pgfqpoint{1.256204in}{5.550213in}}{\pgfqpoint{1.264017in}{5.542400in}}%
\pgfpathcurveto{\pgfqpoint{1.271831in}{5.534586in}}{\pgfqpoint{1.282430in}{5.530196in}}{\pgfqpoint{1.293480in}{5.530196in}}%
\pgfpathclose%
\pgfusepath{stroke,fill}%
\end{pgfscope}%
\begin{pgfscope}%
\pgfpathrectangle{\pgfqpoint{0.481978in}{0.331635in}}{\pgfqpoint{9.300000in}{7.700000in}}%
\pgfusepath{clip}%
\pgfsetbuttcap%
\pgfsetroundjoin%
\definecolor{currentfill}{rgb}{1.000000,0.705882,0.509804}%
\pgfsetfillcolor{currentfill}%
\pgfsetlinewidth{0.481800pt}%
\definecolor{currentstroke}{rgb}{1.000000,1.000000,1.000000}%
\pgfsetstrokecolor{currentstroke}%
\pgfsetdash{}{0pt}%
\pgfpathmoveto{\pgfqpoint{4.735330in}{2.499100in}}%
\pgfpathcurveto{\pgfqpoint{4.746380in}{2.499100in}}{\pgfqpoint{4.756979in}{2.503490in}}{\pgfqpoint{4.764793in}{2.511303in}}%
\pgfpathcurveto{\pgfqpoint{4.772606in}{2.519117in}}{\pgfqpoint{4.776997in}{2.529716in}}{\pgfqpoint{4.776997in}{2.540766in}}%
\pgfpathcurveto{\pgfqpoint{4.776997in}{2.551816in}}{\pgfqpoint{4.772606in}{2.562415in}}{\pgfqpoint{4.764793in}{2.570229in}}%
\pgfpathcurveto{\pgfqpoint{4.756979in}{2.578043in}}{\pgfqpoint{4.746380in}{2.582433in}}{\pgfqpoint{4.735330in}{2.582433in}}%
\pgfpathcurveto{\pgfqpoint{4.724280in}{2.582433in}}{\pgfqpoint{4.713681in}{2.578043in}}{\pgfqpoint{4.705867in}{2.570229in}}%
\pgfpathcurveto{\pgfqpoint{4.698054in}{2.562415in}}{\pgfqpoint{4.693663in}{2.551816in}}{\pgfqpoint{4.693663in}{2.540766in}}%
\pgfpathcurveto{\pgfqpoint{4.693663in}{2.529716in}}{\pgfqpoint{4.698054in}{2.519117in}}{\pgfqpoint{4.705867in}{2.511303in}}%
\pgfpathcurveto{\pgfqpoint{4.713681in}{2.503490in}}{\pgfqpoint{4.724280in}{2.499100in}}{\pgfqpoint{4.735330in}{2.499100in}}%
\pgfpathclose%
\pgfusepath{stroke,fill}%
\end{pgfscope}%
\begin{pgfscope}%
\pgfpathrectangle{\pgfqpoint{0.481978in}{0.331635in}}{\pgfqpoint{9.300000in}{7.700000in}}%
\pgfusepath{clip}%
\pgfsetbuttcap%
\pgfsetroundjoin%
\definecolor{currentfill}{rgb}{1.000000,0.705882,0.509804}%
\pgfsetfillcolor{currentfill}%
\pgfsetlinewidth{0.481800pt}%
\definecolor{currentstroke}{rgb}{1.000000,1.000000,1.000000}%
\pgfsetstrokecolor{currentstroke}%
\pgfsetdash{}{0pt}%
\pgfpathmoveto{\pgfqpoint{4.464515in}{4.972136in}}%
\pgfpathcurveto{\pgfqpoint{4.475565in}{4.972136in}}{\pgfqpoint{4.486164in}{4.976526in}}{\pgfqpoint{4.493978in}{4.984340in}}%
\pgfpathcurveto{\pgfqpoint{4.501792in}{4.992153in}}{\pgfqpoint{4.506182in}{5.002752in}}{\pgfqpoint{4.506182in}{5.013803in}}%
\pgfpathcurveto{\pgfqpoint{4.506182in}{5.024853in}}{\pgfqpoint{4.501792in}{5.035452in}}{\pgfqpoint{4.493978in}{5.043265in}}%
\pgfpathcurveto{\pgfqpoint{4.486164in}{5.051079in}}{\pgfqpoint{4.475565in}{5.055469in}}{\pgfqpoint{4.464515in}{5.055469in}}%
\pgfpathcurveto{\pgfqpoint{4.453465in}{5.055469in}}{\pgfqpoint{4.442866in}{5.051079in}}{\pgfqpoint{4.435052in}{5.043265in}}%
\pgfpathcurveto{\pgfqpoint{4.427239in}{5.035452in}}{\pgfqpoint{4.422849in}{5.024853in}}{\pgfqpoint{4.422849in}{5.013803in}}%
\pgfpathcurveto{\pgfqpoint{4.422849in}{5.002752in}}{\pgfqpoint{4.427239in}{4.992153in}}{\pgfqpoint{4.435052in}{4.984340in}}%
\pgfpathcurveto{\pgfqpoint{4.442866in}{4.976526in}}{\pgfqpoint{4.453465in}{4.972136in}}{\pgfqpoint{4.464515in}{4.972136in}}%
\pgfpathclose%
\pgfusepath{stroke,fill}%
\end{pgfscope}%
\begin{pgfscope}%
\pgfpathrectangle{\pgfqpoint{0.481978in}{0.331635in}}{\pgfqpoint{9.300000in}{7.700000in}}%
\pgfusepath{clip}%
\pgfsetbuttcap%
\pgfsetroundjoin%
\definecolor{currentfill}{rgb}{1.000000,0.705882,0.509804}%
\pgfsetfillcolor{currentfill}%
\pgfsetlinewidth{0.481800pt}%
\definecolor{currentstroke}{rgb}{1.000000,1.000000,1.000000}%
\pgfsetstrokecolor{currentstroke}%
\pgfsetdash{}{0pt}%
\pgfpathmoveto{\pgfqpoint{3.760896in}{2.867556in}}%
\pgfpathcurveto{\pgfqpoint{3.771946in}{2.867556in}}{\pgfqpoint{3.782545in}{2.871947in}}{\pgfqpoint{3.790359in}{2.879760in}}%
\pgfpathcurveto{\pgfqpoint{3.798172in}{2.887574in}}{\pgfqpoint{3.802563in}{2.898173in}}{\pgfqpoint{3.802563in}{2.909223in}}%
\pgfpathcurveto{\pgfqpoint{3.802563in}{2.920273in}}{\pgfqpoint{3.798172in}{2.930872in}}{\pgfqpoint{3.790359in}{2.938686in}}%
\pgfpathcurveto{\pgfqpoint{3.782545in}{2.946499in}}{\pgfqpoint{3.771946in}{2.950890in}}{\pgfqpoint{3.760896in}{2.950890in}}%
\pgfpathcurveto{\pgfqpoint{3.749846in}{2.950890in}}{\pgfqpoint{3.739247in}{2.946499in}}{\pgfqpoint{3.731433in}{2.938686in}}%
\pgfpathcurveto{\pgfqpoint{3.723620in}{2.930872in}}{\pgfqpoint{3.719229in}{2.920273in}}{\pgfqpoint{3.719229in}{2.909223in}}%
\pgfpathcurveto{\pgfqpoint{3.719229in}{2.898173in}}{\pgfqpoint{3.723620in}{2.887574in}}{\pgfqpoint{3.731433in}{2.879760in}}%
\pgfpathcurveto{\pgfqpoint{3.739247in}{2.871947in}}{\pgfqpoint{3.749846in}{2.867556in}}{\pgfqpoint{3.760896in}{2.867556in}}%
\pgfpathclose%
\pgfusepath{stroke,fill}%
\end{pgfscope}%
\begin{pgfscope}%
\pgfpathrectangle{\pgfqpoint{0.481978in}{0.331635in}}{\pgfqpoint{9.300000in}{7.700000in}}%
\pgfusepath{clip}%
\pgfsetbuttcap%
\pgfsetroundjoin%
\definecolor{currentfill}{rgb}{1.000000,0.705882,0.509804}%
\pgfsetfillcolor{currentfill}%
\pgfsetlinewidth{0.481800pt}%
\definecolor{currentstroke}{rgb}{1.000000,1.000000,1.000000}%
\pgfsetstrokecolor{currentstroke}%
\pgfsetdash{}{0pt}%
\pgfpathmoveto{\pgfqpoint{2.336586in}{3.476510in}}%
\pgfpathcurveto{\pgfqpoint{2.347636in}{3.476510in}}{\pgfqpoint{2.358235in}{3.480900in}}{\pgfqpoint{2.366049in}{3.488714in}}%
\pgfpathcurveto{\pgfqpoint{2.373863in}{3.496528in}}{\pgfqpoint{2.378253in}{3.507127in}}{\pgfqpoint{2.378253in}{3.518177in}}%
\pgfpathcurveto{\pgfqpoint{2.378253in}{3.529227in}}{\pgfqpoint{2.373863in}{3.539826in}}{\pgfqpoint{2.366049in}{3.547639in}}%
\pgfpathcurveto{\pgfqpoint{2.358235in}{3.555453in}}{\pgfqpoint{2.347636in}{3.559843in}}{\pgfqpoint{2.336586in}{3.559843in}}%
\pgfpathcurveto{\pgfqpoint{2.325536in}{3.559843in}}{\pgfqpoint{2.314937in}{3.555453in}}{\pgfqpoint{2.307123in}{3.547639in}}%
\pgfpathcurveto{\pgfqpoint{2.299310in}{3.539826in}}{\pgfqpoint{2.294919in}{3.529227in}}{\pgfqpoint{2.294919in}{3.518177in}}%
\pgfpathcurveto{\pgfqpoint{2.294919in}{3.507127in}}{\pgfqpoint{2.299310in}{3.496528in}}{\pgfqpoint{2.307123in}{3.488714in}}%
\pgfpathcurveto{\pgfqpoint{2.314937in}{3.480900in}}{\pgfqpoint{2.325536in}{3.476510in}}{\pgfqpoint{2.336586in}{3.476510in}}%
\pgfpathclose%
\pgfusepath{stroke,fill}%
\end{pgfscope}%
\begin{pgfscope}%
\pgfpathrectangle{\pgfqpoint{0.481978in}{0.331635in}}{\pgfqpoint{9.300000in}{7.700000in}}%
\pgfusepath{clip}%
\pgfsetbuttcap%
\pgfsetroundjoin%
\definecolor{currentfill}{rgb}{1.000000,0.705882,0.509804}%
\pgfsetfillcolor{currentfill}%
\pgfsetlinewidth{0.481800pt}%
\definecolor{currentstroke}{rgb}{1.000000,1.000000,1.000000}%
\pgfsetstrokecolor{currentstroke}%
\pgfsetdash{}{0pt}%
\pgfpathmoveto{\pgfqpoint{3.224347in}{6.004169in}}%
\pgfpathcurveto{\pgfqpoint{3.235397in}{6.004169in}}{\pgfqpoint{3.245996in}{6.008560in}}{\pgfqpoint{3.253810in}{6.016373in}}%
\pgfpathcurveto{\pgfqpoint{3.261624in}{6.024187in}}{\pgfqpoint{3.266014in}{6.034786in}}{\pgfqpoint{3.266014in}{6.045836in}}%
\pgfpathcurveto{\pgfqpoint{3.266014in}{6.056886in}}{\pgfqpoint{3.261624in}{6.067485in}}{\pgfqpoint{3.253810in}{6.075299in}}%
\pgfpathcurveto{\pgfqpoint{3.245996in}{6.083112in}}{\pgfqpoint{3.235397in}{6.087503in}}{\pgfqpoint{3.224347in}{6.087503in}}%
\pgfpathcurveto{\pgfqpoint{3.213297in}{6.087503in}}{\pgfqpoint{3.202698in}{6.083112in}}{\pgfqpoint{3.194884in}{6.075299in}}%
\pgfpathcurveto{\pgfqpoint{3.187071in}{6.067485in}}{\pgfqpoint{3.182681in}{6.056886in}}{\pgfqpoint{3.182681in}{6.045836in}}%
\pgfpathcurveto{\pgfqpoint{3.182681in}{6.034786in}}{\pgfqpoint{3.187071in}{6.024187in}}{\pgfqpoint{3.194884in}{6.016373in}}%
\pgfpathcurveto{\pgfqpoint{3.202698in}{6.008560in}}{\pgfqpoint{3.213297in}{6.004169in}}{\pgfqpoint{3.224347in}{6.004169in}}%
\pgfpathclose%
\pgfusepath{stroke,fill}%
\end{pgfscope}%
\begin{pgfscope}%
\pgfpathrectangle{\pgfqpoint{0.481978in}{0.331635in}}{\pgfqpoint{9.300000in}{7.700000in}}%
\pgfusepath{clip}%
\pgfsetbuttcap%
\pgfsetroundjoin%
\definecolor{currentfill}{rgb}{1.000000,0.705882,0.509804}%
\pgfsetfillcolor{currentfill}%
\pgfsetlinewidth{0.481800pt}%
\definecolor{currentstroke}{rgb}{1.000000,1.000000,1.000000}%
\pgfsetstrokecolor{currentstroke}%
\pgfsetdash{}{0pt}%
\pgfpathmoveto{\pgfqpoint{3.710549in}{6.521137in}}%
\pgfpathcurveto{\pgfqpoint{3.721600in}{6.521137in}}{\pgfqpoint{3.732199in}{6.525527in}}{\pgfqpoint{3.740012in}{6.533341in}}%
\pgfpathcurveto{\pgfqpoint{3.747826in}{6.541155in}}{\pgfqpoint{3.752216in}{6.551754in}}{\pgfqpoint{3.752216in}{6.562804in}}%
\pgfpathcurveto{\pgfqpoint{3.752216in}{6.573854in}}{\pgfqpoint{3.747826in}{6.584453in}}{\pgfqpoint{3.740012in}{6.592267in}}%
\pgfpathcurveto{\pgfqpoint{3.732199in}{6.600080in}}{\pgfqpoint{3.721600in}{6.604470in}}{\pgfqpoint{3.710549in}{6.604470in}}%
\pgfpathcurveto{\pgfqpoint{3.699499in}{6.604470in}}{\pgfqpoint{3.688900in}{6.600080in}}{\pgfqpoint{3.681087in}{6.592267in}}%
\pgfpathcurveto{\pgfqpoint{3.673273in}{6.584453in}}{\pgfqpoint{3.668883in}{6.573854in}}{\pgfqpoint{3.668883in}{6.562804in}}%
\pgfpathcurveto{\pgfqpoint{3.668883in}{6.551754in}}{\pgfqpoint{3.673273in}{6.541155in}}{\pgfqpoint{3.681087in}{6.533341in}}%
\pgfpathcurveto{\pgfqpoint{3.688900in}{6.525527in}}{\pgfqpoint{3.699499in}{6.521137in}}{\pgfqpoint{3.710549in}{6.521137in}}%
\pgfpathclose%
\pgfusepath{stroke,fill}%
\end{pgfscope}%
\begin{pgfscope}%
\pgfpathrectangle{\pgfqpoint{0.481978in}{0.331635in}}{\pgfqpoint{9.300000in}{7.700000in}}%
\pgfusepath{clip}%
\pgfsetbuttcap%
\pgfsetroundjoin%
\definecolor{currentfill}{rgb}{1.000000,0.705882,0.509804}%
\pgfsetfillcolor{currentfill}%
\pgfsetlinewidth{0.481800pt}%
\definecolor{currentstroke}{rgb}{1.000000,1.000000,1.000000}%
\pgfsetstrokecolor{currentstroke}%
\pgfsetdash{}{0pt}%
\pgfpathmoveto{\pgfqpoint{3.320504in}{5.247157in}}%
\pgfpathcurveto{\pgfqpoint{3.331554in}{5.247157in}}{\pgfqpoint{3.342153in}{5.251548in}}{\pgfqpoint{3.349966in}{5.259361in}}%
\pgfpathcurveto{\pgfqpoint{3.357780in}{5.267175in}}{\pgfqpoint{3.362170in}{5.277774in}}{\pgfqpoint{3.362170in}{5.288824in}}%
\pgfpathcurveto{\pgfqpoint{3.362170in}{5.299874in}}{\pgfqpoint{3.357780in}{5.310473in}}{\pgfqpoint{3.349966in}{5.318287in}}%
\pgfpathcurveto{\pgfqpoint{3.342153in}{5.326100in}}{\pgfqpoint{3.331554in}{5.330491in}}{\pgfqpoint{3.320504in}{5.330491in}}%
\pgfpathcurveto{\pgfqpoint{3.309454in}{5.330491in}}{\pgfqpoint{3.298855in}{5.326100in}}{\pgfqpoint{3.291041in}{5.318287in}}%
\pgfpathcurveto{\pgfqpoint{3.283227in}{5.310473in}}{\pgfqpoint{3.278837in}{5.299874in}}{\pgfqpoint{3.278837in}{5.288824in}}%
\pgfpathcurveto{\pgfqpoint{3.278837in}{5.277774in}}{\pgfqpoint{3.283227in}{5.267175in}}{\pgfqpoint{3.291041in}{5.259361in}}%
\pgfpathcurveto{\pgfqpoint{3.298855in}{5.251548in}}{\pgfqpoint{3.309454in}{5.247157in}}{\pgfqpoint{3.320504in}{5.247157in}}%
\pgfpathclose%
\pgfusepath{stroke,fill}%
\end{pgfscope}%
\begin{pgfscope}%
\pgfpathrectangle{\pgfqpoint{0.481978in}{0.331635in}}{\pgfqpoint{9.300000in}{7.700000in}}%
\pgfusepath{clip}%
\pgfsetbuttcap%
\pgfsetroundjoin%
\definecolor{currentfill}{rgb}{1.000000,0.705882,0.509804}%
\pgfsetfillcolor{currentfill}%
\pgfsetlinewidth{0.481800pt}%
\definecolor{currentstroke}{rgb}{1.000000,1.000000,1.000000}%
\pgfsetstrokecolor{currentstroke}%
\pgfsetdash{}{0pt}%
\pgfpathmoveto{\pgfqpoint{5.389021in}{1.238745in}}%
\pgfpathcurveto{\pgfqpoint{5.400072in}{1.238745in}}{\pgfqpoint{5.410671in}{1.243135in}}{\pgfqpoint{5.418484in}{1.250948in}}%
\pgfpathcurveto{\pgfqpoint{5.426298in}{1.258762in}}{\pgfqpoint{5.430688in}{1.269361in}}{\pgfqpoint{5.430688in}{1.280411in}}%
\pgfpathcurveto{\pgfqpoint{5.430688in}{1.291461in}}{\pgfqpoint{5.426298in}{1.302060in}}{\pgfqpoint{5.418484in}{1.309874in}}%
\pgfpathcurveto{\pgfqpoint{5.410671in}{1.317688in}}{\pgfqpoint{5.400072in}{1.322078in}}{\pgfqpoint{5.389021in}{1.322078in}}%
\pgfpathcurveto{\pgfqpoint{5.377971in}{1.322078in}}{\pgfqpoint{5.367372in}{1.317688in}}{\pgfqpoint{5.359559in}{1.309874in}}%
\pgfpathcurveto{\pgfqpoint{5.351745in}{1.302060in}}{\pgfqpoint{5.347355in}{1.291461in}}{\pgfqpoint{5.347355in}{1.280411in}}%
\pgfpathcurveto{\pgfqpoint{5.347355in}{1.269361in}}{\pgfqpoint{5.351745in}{1.258762in}}{\pgfqpoint{5.359559in}{1.250948in}}%
\pgfpathcurveto{\pgfqpoint{5.367372in}{1.243135in}}{\pgfqpoint{5.377971in}{1.238745in}}{\pgfqpoint{5.389021in}{1.238745in}}%
\pgfpathclose%
\pgfusepath{stroke,fill}%
\end{pgfscope}%
\begin{pgfscope}%
\pgfpathrectangle{\pgfqpoint{0.481978in}{0.331635in}}{\pgfqpoint{9.300000in}{7.700000in}}%
\pgfusepath{clip}%
\pgfsetbuttcap%
\pgfsetroundjoin%
\definecolor{currentfill}{rgb}{1.000000,0.705882,0.509804}%
\pgfsetfillcolor{currentfill}%
\pgfsetlinewidth{0.481800pt}%
\definecolor{currentstroke}{rgb}{1.000000,1.000000,1.000000}%
\pgfsetstrokecolor{currentstroke}%
\pgfsetdash{}{0pt}%
\pgfpathmoveto{\pgfqpoint{1.225270in}{3.155365in}}%
\pgfpathcurveto{\pgfqpoint{1.236320in}{3.155365in}}{\pgfqpoint{1.246919in}{3.159755in}}{\pgfqpoint{1.254733in}{3.167569in}}%
\pgfpathcurveto{\pgfqpoint{1.262547in}{3.175383in}}{\pgfqpoint{1.266937in}{3.185982in}}{\pgfqpoint{1.266937in}{3.197032in}}%
\pgfpathcurveto{\pgfqpoint{1.266937in}{3.208082in}}{\pgfqpoint{1.262547in}{3.218681in}}{\pgfqpoint{1.254733in}{3.226495in}}%
\pgfpathcurveto{\pgfqpoint{1.246919in}{3.234308in}}{\pgfqpoint{1.236320in}{3.238698in}}{\pgfqpoint{1.225270in}{3.238698in}}%
\pgfpathcurveto{\pgfqpoint{1.214220in}{3.238698in}}{\pgfqpoint{1.203621in}{3.234308in}}{\pgfqpoint{1.195807in}{3.226495in}}%
\pgfpathcurveto{\pgfqpoint{1.187994in}{3.218681in}}{\pgfqpoint{1.183603in}{3.208082in}}{\pgfqpoint{1.183603in}{3.197032in}}%
\pgfpathcurveto{\pgfqpoint{1.183603in}{3.185982in}}{\pgfqpoint{1.187994in}{3.175383in}}{\pgfqpoint{1.195807in}{3.167569in}}%
\pgfpathcurveto{\pgfqpoint{1.203621in}{3.159755in}}{\pgfqpoint{1.214220in}{3.155365in}}{\pgfqpoint{1.225270in}{3.155365in}}%
\pgfpathclose%
\pgfusepath{stroke,fill}%
\end{pgfscope}%
\begin{pgfscope}%
\pgfpathrectangle{\pgfqpoint{0.481978in}{0.331635in}}{\pgfqpoint{9.300000in}{7.700000in}}%
\pgfusepath{clip}%
\pgfsetbuttcap%
\pgfsetroundjoin%
\definecolor{currentfill}{rgb}{1.000000,0.705882,0.509804}%
\pgfsetfillcolor{currentfill}%
\pgfsetlinewidth{0.481800pt}%
\definecolor{currentstroke}{rgb}{1.000000,1.000000,1.000000}%
\pgfsetstrokecolor{currentstroke}%
\pgfsetdash{}{0pt}%
\pgfpathmoveto{\pgfqpoint{4.019367in}{2.930474in}}%
\pgfpathcurveto{\pgfqpoint{4.030418in}{2.930474in}}{\pgfqpoint{4.041017in}{2.934864in}}{\pgfqpoint{4.048830in}{2.942677in}}%
\pgfpathcurveto{\pgfqpoint{4.056644in}{2.950491in}}{\pgfqpoint{4.061034in}{2.961090in}}{\pgfqpoint{4.061034in}{2.972140in}}%
\pgfpathcurveto{\pgfqpoint{4.061034in}{2.983190in}}{\pgfqpoint{4.056644in}{2.993789in}}{\pgfqpoint{4.048830in}{3.001603in}}%
\pgfpathcurveto{\pgfqpoint{4.041017in}{3.009417in}}{\pgfqpoint{4.030418in}{3.013807in}}{\pgfqpoint{4.019367in}{3.013807in}}%
\pgfpathcurveto{\pgfqpoint{4.008317in}{3.013807in}}{\pgfqpoint{3.997718in}{3.009417in}}{\pgfqpoint{3.989905in}{3.001603in}}%
\pgfpathcurveto{\pgfqpoint{3.982091in}{2.993789in}}{\pgfqpoint{3.977701in}{2.983190in}}{\pgfqpoint{3.977701in}{2.972140in}}%
\pgfpathcurveto{\pgfqpoint{3.977701in}{2.961090in}}{\pgfqpoint{3.982091in}{2.950491in}}{\pgfqpoint{3.989905in}{2.942677in}}%
\pgfpathcurveto{\pgfqpoint{3.997718in}{2.934864in}}{\pgfqpoint{4.008317in}{2.930474in}}{\pgfqpoint{4.019367in}{2.930474in}}%
\pgfpathclose%
\pgfusepath{stroke,fill}%
\end{pgfscope}%
\begin{pgfscope}%
\pgfpathrectangle{\pgfqpoint{0.481978in}{0.331635in}}{\pgfqpoint{9.300000in}{7.700000in}}%
\pgfusepath{clip}%
\pgfsetbuttcap%
\pgfsetroundjoin%
\definecolor{currentfill}{rgb}{1.000000,0.705882,0.509804}%
\pgfsetfillcolor{currentfill}%
\pgfsetlinewidth{0.481800pt}%
\definecolor{currentstroke}{rgb}{1.000000,1.000000,1.000000}%
\pgfsetstrokecolor{currentstroke}%
\pgfsetdash{}{0pt}%
\pgfpathmoveto{\pgfqpoint{3.098846in}{4.048792in}}%
\pgfpathcurveto{\pgfqpoint{3.109896in}{4.048792in}}{\pgfqpoint{3.120495in}{4.053182in}}{\pgfqpoint{3.128309in}{4.060996in}}%
\pgfpathcurveto{\pgfqpoint{3.136122in}{4.068809in}}{\pgfqpoint{3.140512in}{4.079408in}}{\pgfqpoint{3.140512in}{4.090458in}}%
\pgfpathcurveto{\pgfqpoint{3.140512in}{4.101509in}}{\pgfqpoint{3.136122in}{4.112108in}}{\pgfqpoint{3.128309in}{4.119921in}}%
\pgfpathcurveto{\pgfqpoint{3.120495in}{4.127735in}}{\pgfqpoint{3.109896in}{4.132125in}}{\pgfqpoint{3.098846in}{4.132125in}}%
\pgfpathcurveto{\pgfqpoint{3.087796in}{4.132125in}}{\pgfqpoint{3.077197in}{4.127735in}}{\pgfqpoint{3.069383in}{4.119921in}}%
\pgfpathcurveto{\pgfqpoint{3.061569in}{4.112108in}}{\pgfqpoint{3.057179in}{4.101509in}}{\pgfqpoint{3.057179in}{4.090458in}}%
\pgfpathcurveto{\pgfqpoint{3.057179in}{4.079408in}}{\pgfqpoint{3.061569in}{4.068809in}}{\pgfqpoint{3.069383in}{4.060996in}}%
\pgfpathcurveto{\pgfqpoint{3.077197in}{4.053182in}}{\pgfqpoint{3.087796in}{4.048792in}}{\pgfqpoint{3.098846in}{4.048792in}}%
\pgfpathclose%
\pgfusepath{stroke,fill}%
\end{pgfscope}%
\begin{pgfscope}%
\pgfpathrectangle{\pgfqpoint{0.481978in}{0.331635in}}{\pgfqpoint{9.300000in}{7.700000in}}%
\pgfusepath{clip}%
\pgfsetbuttcap%
\pgfsetroundjoin%
\definecolor{currentfill}{rgb}{1.000000,0.705882,0.509804}%
\pgfsetfillcolor{currentfill}%
\pgfsetlinewidth{0.481800pt}%
\definecolor{currentstroke}{rgb}{1.000000,1.000000,1.000000}%
\pgfsetstrokecolor{currentstroke}%
\pgfsetdash{}{0pt}%
\pgfpathmoveto{\pgfqpoint{2.491460in}{4.500035in}}%
\pgfpathcurveto{\pgfqpoint{2.502510in}{4.500035in}}{\pgfqpoint{2.513109in}{4.504426in}}{\pgfqpoint{2.520923in}{4.512239in}}%
\pgfpathcurveto{\pgfqpoint{2.528736in}{4.520053in}}{\pgfqpoint{2.533126in}{4.530652in}}{\pgfqpoint{2.533126in}{4.541702in}}%
\pgfpathcurveto{\pgfqpoint{2.533126in}{4.552752in}}{\pgfqpoint{2.528736in}{4.563351in}}{\pgfqpoint{2.520923in}{4.571165in}}%
\pgfpathcurveto{\pgfqpoint{2.513109in}{4.578978in}}{\pgfqpoint{2.502510in}{4.583369in}}{\pgfqpoint{2.491460in}{4.583369in}}%
\pgfpathcurveto{\pgfqpoint{2.480410in}{4.583369in}}{\pgfqpoint{2.469811in}{4.578978in}}{\pgfqpoint{2.461997in}{4.571165in}}%
\pgfpathcurveto{\pgfqpoint{2.454183in}{4.563351in}}{\pgfqpoint{2.449793in}{4.552752in}}{\pgfqpoint{2.449793in}{4.541702in}}%
\pgfpathcurveto{\pgfqpoint{2.449793in}{4.530652in}}{\pgfqpoint{2.454183in}{4.520053in}}{\pgfqpoint{2.461997in}{4.512239in}}%
\pgfpathcurveto{\pgfqpoint{2.469811in}{4.504426in}}{\pgfqpoint{2.480410in}{4.500035in}}{\pgfqpoint{2.491460in}{4.500035in}}%
\pgfpathclose%
\pgfusepath{stroke,fill}%
\end{pgfscope}%
\begin{pgfscope}%
\pgfpathrectangle{\pgfqpoint{0.481978in}{0.331635in}}{\pgfqpoint{9.300000in}{7.700000in}}%
\pgfusepath{clip}%
\pgfsetbuttcap%
\pgfsetroundjoin%
\definecolor{currentfill}{rgb}{1.000000,0.705882,0.509804}%
\pgfsetfillcolor{currentfill}%
\pgfsetlinewidth{0.481800pt}%
\definecolor{currentstroke}{rgb}{1.000000,1.000000,1.000000}%
\pgfsetstrokecolor{currentstroke}%
\pgfsetdash{}{0pt}%
\pgfpathmoveto{\pgfqpoint{4.722330in}{3.350611in}}%
\pgfpathcurveto{\pgfqpoint{4.733380in}{3.350611in}}{\pgfqpoint{4.743979in}{3.355001in}}{\pgfqpoint{4.751792in}{3.362814in}}%
\pgfpathcurveto{\pgfqpoint{4.759606in}{3.370628in}}{\pgfqpoint{4.763996in}{3.381227in}}{\pgfqpoint{4.763996in}{3.392277in}}%
\pgfpathcurveto{\pgfqpoint{4.763996in}{3.403327in}}{\pgfqpoint{4.759606in}{3.413926in}}{\pgfqpoint{4.751792in}{3.421740in}}%
\pgfpathcurveto{\pgfqpoint{4.743979in}{3.429554in}}{\pgfqpoint{4.733380in}{3.433944in}}{\pgfqpoint{4.722330in}{3.433944in}}%
\pgfpathcurveto{\pgfqpoint{4.711279in}{3.433944in}}{\pgfqpoint{4.700680in}{3.429554in}}{\pgfqpoint{4.692867in}{3.421740in}}%
\pgfpathcurveto{\pgfqpoint{4.685053in}{3.413926in}}{\pgfqpoint{4.680663in}{3.403327in}}{\pgfqpoint{4.680663in}{3.392277in}}%
\pgfpathcurveto{\pgfqpoint{4.680663in}{3.381227in}}{\pgfqpoint{4.685053in}{3.370628in}}{\pgfqpoint{4.692867in}{3.362814in}}%
\pgfpathcurveto{\pgfqpoint{4.700680in}{3.355001in}}{\pgfqpoint{4.711279in}{3.350611in}}{\pgfqpoint{4.722330in}{3.350611in}}%
\pgfpathclose%
\pgfusepath{stroke,fill}%
\end{pgfscope}%
\begin{pgfscope}%
\pgfpathrectangle{\pgfqpoint{0.481978in}{0.331635in}}{\pgfqpoint{9.300000in}{7.700000in}}%
\pgfusepath{clip}%
\pgfsetbuttcap%
\pgfsetroundjoin%
\definecolor{currentfill}{rgb}{1.000000,0.705882,0.509804}%
\pgfsetfillcolor{currentfill}%
\pgfsetlinewidth{0.481800pt}%
\definecolor{currentstroke}{rgb}{1.000000,1.000000,1.000000}%
\pgfsetstrokecolor{currentstroke}%
\pgfsetdash{}{0pt}%
\pgfpathmoveto{\pgfqpoint{4.962053in}{4.722304in}}%
\pgfpathcurveto{\pgfqpoint{4.973103in}{4.722304in}}{\pgfqpoint{4.983702in}{4.726694in}}{\pgfqpoint{4.991516in}{4.734507in}}%
\pgfpathcurveto{\pgfqpoint{4.999330in}{4.742321in}}{\pgfqpoint{5.003720in}{4.752920in}}{\pgfqpoint{5.003720in}{4.763970in}}%
\pgfpathcurveto{\pgfqpoint{5.003720in}{4.775020in}}{\pgfqpoint{4.999330in}{4.785619in}}{\pgfqpoint{4.991516in}{4.793433in}}%
\pgfpathcurveto{\pgfqpoint{4.983702in}{4.801247in}}{\pgfqpoint{4.973103in}{4.805637in}}{\pgfqpoint{4.962053in}{4.805637in}}%
\pgfpathcurveto{\pgfqpoint{4.951003in}{4.805637in}}{\pgfqpoint{4.940404in}{4.801247in}}{\pgfqpoint{4.932590in}{4.793433in}}%
\pgfpathcurveto{\pgfqpoint{4.924777in}{4.785619in}}{\pgfqpoint{4.920387in}{4.775020in}}{\pgfqpoint{4.920387in}{4.763970in}}%
\pgfpathcurveto{\pgfqpoint{4.920387in}{4.752920in}}{\pgfqpoint{4.924777in}{4.742321in}}{\pgfqpoint{4.932590in}{4.734507in}}%
\pgfpathcurveto{\pgfqpoint{4.940404in}{4.726694in}}{\pgfqpoint{4.951003in}{4.722304in}}{\pgfqpoint{4.962053in}{4.722304in}}%
\pgfpathclose%
\pgfusepath{stroke,fill}%
\end{pgfscope}%
\begin{pgfscope}%
\pgfpathrectangle{\pgfqpoint{0.481978in}{0.331635in}}{\pgfqpoint{9.300000in}{7.700000in}}%
\pgfusepath{clip}%
\pgfsetbuttcap%
\pgfsetroundjoin%
\definecolor{currentfill}{rgb}{1.000000,0.705882,0.509804}%
\pgfsetfillcolor{currentfill}%
\pgfsetlinewidth{0.481800pt}%
\definecolor{currentstroke}{rgb}{1.000000,1.000000,1.000000}%
\pgfsetstrokecolor{currentstroke}%
\pgfsetdash{}{0pt}%
\pgfpathmoveto{\pgfqpoint{4.756086in}{2.480692in}}%
\pgfpathcurveto{\pgfqpoint{4.767136in}{2.480692in}}{\pgfqpoint{4.777735in}{2.485082in}}{\pgfqpoint{4.785549in}{2.492896in}}%
\pgfpathcurveto{\pgfqpoint{4.793363in}{2.500709in}}{\pgfqpoint{4.797753in}{2.511308in}}{\pgfqpoint{4.797753in}{2.522358in}}%
\pgfpathcurveto{\pgfqpoint{4.797753in}{2.533408in}}{\pgfqpoint{4.793363in}{2.544008in}}{\pgfqpoint{4.785549in}{2.551821in}}%
\pgfpathcurveto{\pgfqpoint{4.777735in}{2.559635in}}{\pgfqpoint{4.767136in}{2.564025in}}{\pgfqpoint{4.756086in}{2.564025in}}%
\pgfpathcurveto{\pgfqpoint{4.745036in}{2.564025in}}{\pgfqpoint{4.734437in}{2.559635in}}{\pgfqpoint{4.726623in}{2.551821in}}%
\pgfpathcurveto{\pgfqpoint{4.718810in}{2.544008in}}{\pgfqpoint{4.714420in}{2.533408in}}{\pgfqpoint{4.714420in}{2.522358in}}%
\pgfpathcurveto{\pgfqpoint{4.714420in}{2.511308in}}{\pgfqpoint{4.718810in}{2.500709in}}{\pgfqpoint{4.726623in}{2.492896in}}%
\pgfpathcurveto{\pgfqpoint{4.734437in}{2.485082in}}{\pgfqpoint{4.745036in}{2.480692in}}{\pgfqpoint{4.756086in}{2.480692in}}%
\pgfpathclose%
\pgfusepath{stroke,fill}%
\end{pgfscope}%
\begin{pgfscope}%
\pgfpathrectangle{\pgfqpoint{0.481978in}{0.331635in}}{\pgfqpoint{9.300000in}{7.700000in}}%
\pgfusepath{clip}%
\pgfsetbuttcap%
\pgfsetroundjoin%
\definecolor{currentfill}{rgb}{1.000000,0.705882,0.509804}%
\pgfsetfillcolor{currentfill}%
\pgfsetlinewidth{0.481800pt}%
\definecolor{currentstroke}{rgb}{1.000000,1.000000,1.000000}%
\pgfsetstrokecolor{currentstroke}%
\pgfsetdash{}{0pt}%
\pgfpathmoveto{\pgfqpoint{2.604559in}{2.316516in}}%
\pgfpathcurveto{\pgfqpoint{2.615609in}{2.316516in}}{\pgfqpoint{2.626208in}{2.320906in}}{\pgfqpoint{2.634022in}{2.328720in}}%
\pgfpathcurveto{\pgfqpoint{2.641836in}{2.336534in}}{\pgfqpoint{2.646226in}{2.347133in}}{\pgfqpoint{2.646226in}{2.358183in}}%
\pgfpathcurveto{\pgfqpoint{2.646226in}{2.369233in}}{\pgfqpoint{2.641836in}{2.379832in}}{\pgfqpoint{2.634022in}{2.387645in}}%
\pgfpathcurveto{\pgfqpoint{2.626208in}{2.395459in}}{\pgfqpoint{2.615609in}{2.399849in}}{\pgfqpoint{2.604559in}{2.399849in}}%
\pgfpathcurveto{\pgfqpoint{2.593509in}{2.399849in}}{\pgfqpoint{2.582910in}{2.395459in}}{\pgfqpoint{2.575096in}{2.387645in}}%
\pgfpathcurveto{\pgfqpoint{2.567283in}{2.379832in}}{\pgfqpoint{2.562892in}{2.369233in}}{\pgfqpoint{2.562892in}{2.358183in}}%
\pgfpathcurveto{\pgfqpoint{2.562892in}{2.347133in}}{\pgfqpoint{2.567283in}{2.336534in}}{\pgfqpoint{2.575096in}{2.328720in}}%
\pgfpathcurveto{\pgfqpoint{2.582910in}{2.320906in}}{\pgfqpoint{2.593509in}{2.316516in}}{\pgfqpoint{2.604559in}{2.316516in}}%
\pgfpathclose%
\pgfusepath{stroke,fill}%
\end{pgfscope}%
\begin{pgfscope}%
\pgfpathrectangle{\pgfqpoint{0.481978in}{0.331635in}}{\pgfqpoint{9.300000in}{7.700000in}}%
\pgfusepath{clip}%
\pgfsetbuttcap%
\pgfsetroundjoin%
\definecolor{currentfill}{rgb}{1.000000,0.705882,0.509804}%
\pgfsetfillcolor{currentfill}%
\pgfsetlinewidth{0.481800pt}%
\definecolor{currentstroke}{rgb}{1.000000,1.000000,1.000000}%
\pgfsetstrokecolor{currentstroke}%
\pgfsetdash{}{0pt}%
\pgfpathmoveto{\pgfqpoint{3.564142in}{5.365515in}}%
\pgfpathcurveto{\pgfqpoint{3.575192in}{5.365515in}}{\pgfqpoint{3.585791in}{5.369905in}}{\pgfqpoint{3.593604in}{5.377719in}}%
\pgfpathcurveto{\pgfqpoint{3.601418in}{5.385533in}}{\pgfqpoint{3.605808in}{5.396132in}}{\pgfqpoint{3.605808in}{5.407182in}}%
\pgfpathcurveto{\pgfqpoint{3.605808in}{5.418232in}}{\pgfqpoint{3.601418in}{5.428831in}}{\pgfqpoint{3.593604in}{5.436645in}}%
\pgfpathcurveto{\pgfqpoint{3.585791in}{5.444458in}}{\pgfqpoint{3.575192in}{5.448849in}}{\pgfqpoint{3.564142in}{5.448849in}}%
\pgfpathcurveto{\pgfqpoint{3.553091in}{5.448849in}}{\pgfqpoint{3.542492in}{5.444458in}}{\pgfqpoint{3.534679in}{5.436645in}}%
\pgfpathcurveto{\pgfqpoint{3.526865in}{5.428831in}}{\pgfqpoint{3.522475in}{5.418232in}}{\pgfqpoint{3.522475in}{5.407182in}}%
\pgfpathcurveto{\pgfqpoint{3.522475in}{5.396132in}}{\pgfqpoint{3.526865in}{5.385533in}}{\pgfqpoint{3.534679in}{5.377719in}}%
\pgfpathcurveto{\pgfqpoint{3.542492in}{5.369905in}}{\pgfqpoint{3.553091in}{5.365515in}}{\pgfqpoint{3.564142in}{5.365515in}}%
\pgfpathclose%
\pgfusepath{stroke,fill}%
\end{pgfscope}%
\begin{pgfscope}%
\pgfpathrectangle{\pgfqpoint{0.481978in}{0.331635in}}{\pgfqpoint{9.300000in}{7.700000in}}%
\pgfusepath{clip}%
\pgfsetbuttcap%
\pgfsetroundjoin%
\definecolor{currentfill}{rgb}{1.000000,0.705882,0.509804}%
\pgfsetfillcolor{currentfill}%
\pgfsetlinewidth{0.481800pt}%
\definecolor{currentstroke}{rgb}{1.000000,1.000000,1.000000}%
\pgfsetstrokecolor{currentstroke}%
\pgfsetdash{}{0pt}%
\pgfpathmoveto{\pgfqpoint{3.840892in}{5.590809in}}%
\pgfpathcurveto{\pgfqpoint{3.851942in}{5.590809in}}{\pgfqpoint{3.862541in}{5.595199in}}{\pgfqpoint{3.870355in}{5.603013in}}%
\pgfpathcurveto{\pgfqpoint{3.878168in}{5.610826in}}{\pgfqpoint{3.882559in}{5.621425in}}{\pgfqpoint{3.882559in}{5.632476in}}%
\pgfpathcurveto{\pgfqpoint{3.882559in}{5.643526in}}{\pgfqpoint{3.878168in}{5.654125in}}{\pgfqpoint{3.870355in}{5.661938in}}%
\pgfpathcurveto{\pgfqpoint{3.862541in}{5.669752in}}{\pgfqpoint{3.851942in}{5.674142in}}{\pgfqpoint{3.840892in}{5.674142in}}%
\pgfpathcurveto{\pgfqpoint{3.829842in}{5.674142in}}{\pgfqpoint{3.819243in}{5.669752in}}{\pgfqpoint{3.811429in}{5.661938in}}%
\pgfpathcurveto{\pgfqpoint{3.803616in}{5.654125in}}{\pgfqpoint{3.799225in}{5.643526in}}{\pgfqpoint{3.799225in}{5.632476in}}%
\pgfpathcurveto{\pgfqpoint{3.799225in}{5.621425in}}{\pgfqpoint{3.803616in}{5.610826in}}{\pgfqpoint{3.811429in}{5.603013in}}%
\pgfpathcurveto{\pgfqpoint{3.819243in}{5.595199in}}{\pgfqpoint{3.829842in}{5.590809in}}{\pgfqpoint{3.840892in}{5.590809in}}%
\pgfpathclose%
\pgfusepath{stroke,fill}%
\end{pgfscope}%
\begin{pgfscope}%
\pgfpathrectangle{\pgfqpoint{0.481978in}{0.331635in}}{\pgfqpoint{9.300000in}{7.700000in}}%
\pgfusepath{clip}%
\pgfsetbuttcap%
\pgfsetroundjoin%
\definecolor{currentfill}{rgb}{1.000000,0.705882,0.509804}%
\pgfsetfillcolor{currentfill}%
\pgfsetlinewidth{0.481800pt}%
\definecolor{currentstroke}{rgb}{1.000000,1.000000,1.000000}%
\pgfsetstrokecolor{currentstroke}%
\pgfsetdash{}{0pt}%
\pgfpathmoveto{\pgfqpoint{5.376976in}{4.611624in}}%
\pgfpathcurveto{\pgfqpoint{5.388026in}{4.611624in}}{\pgfqpoint{5.398625in}{4.616014in}}{\pgfqpoint{5.406439in}{4.623827in}}%
\pgfpathcurveto{\pgfqpoint{5.414252in}{4.631641in}}{\pgfqpoint{5.418643in}{4.642240in}}{\pgfqpoint{5.418643in}{4.653290in}}%
\pgfpathcurveto{\pgfqpoint{5.418643in}{4.664340in}}{\pgfqpoint{5.414252in}{4.674939in}}{\pgfqpoint{5.406439in}{4.682753in}}%
\pgfpathcurveto{\pgfqpoint{5.398625in}{4.690567in}}{\pgfqpoint{5.388026in}{4.694957in}}{\pgfqpoint{5.376976in}{4.694957in}}%
\pgfpathcurveto{\pgfqpoint{5.365926in}{4.694957in}}{\pgfqpoint{5.355327in}{4.690567in}}{\pgfqpoint{5.347513in}{4.682753in}}%
\pgfpathcurveto{\pgfqpoint{5.339699in}{4.674939in}}{\pgfqpoint{5.335309in}{4.664340in}}{\pgfqpoint{5.335309in}{4.653290in}}%
\pgfpathcurveto{\pgfqpoint{5.335309in}{4.642240in}}{\pgfqpoint{5.339699in}{4.631641in}}{\pgfqpoint{5.347513in}{4.623827in}}%
\pgfpathcurveto{\pgfqpoint{5.355327in}{4.616014in}}{\pgfqpoint{5.365926in}{4.611624in}}{\pgfqpoint{5.376976in}{4.611624in}}%
\pgfpathclose%
\pgfusepath{stroke,fill}%
\end{pgfscope}%
\begin{pgfscope}%
\pgfpathrectangle{\pgfqpoint{0.481978in}{0.331635in}}{\pgfqpoint{9.300000in}{7.700000in}}%
\pgfusepath{clip}%
\pgfsetbuttcap%
\pgfsetroundjoin%
\definecolor{currentfill}{rgb}{1.000000,0.705882,0.509804}%
\pgfsetfillcolor{currentfill}%
\pgfsetlinewidth{0.481800pt}%
\definecolor{currentstroke}{rgb}{1.000000,1.000000,1.000000}%
\pgfsetstrokecolor{currentstroke}%
\pgfsetdash{}{0pt}%
\pgfpathmoveto{\pgfqpoint{4.185210in}{4.176863in}}%
\pgfpathcurveto{\pgfqpoint{4.196260in}{4.176863in}}{\pgfqpoint{4.206859in}{4.181253in}}{\pgfqpoint{4.214673in}{4.189066in}}%
\pgfpathcurveto{\pgfqpoint{4.222486in}{4.196880in}}{\pgfqpoint{4.226877in}{4.207479in}}{\pgfqpoint{4.226877in}{4.218529in}}%
\pgfpathcurveto{\pgfqpoint{4.226877in}{4.229579in}}{\pgfqpoint{4.222486in}{4.240178in}}{\pgfqpoint{4.214673in}{4.247992in}}%
\pgfpathcurveto{\pgfqpoint{4.206859in}{4.255806in}}{\pgfqpoint{4.196260in}{4.260196in}}{\pgfqpoint{4.185210in}{4.260196in}}%
\pgfpathcurveto{\pgfqpoint{4.174160in}{4.260196in}}{\pgfqpoint{4.163561in}{4.255806in}}{\pgfqpoint{4.155747in}{4.247992in}}%
\pgfpathcurveto{\pgfqpoint{4.147933in}{4.240178in}}{\pgfqpoint{4.143543in}{4.229579in}}{\pgfqpoint{4.143543in}{4.218529in}}%
\pgfpathcurveto{\pgfqpoint{4.143543in}{4.207479in}}{\pgfqpoint{4.147933in}{4.196880in}}{\pgfqpoint{4.155747in}{4.189066in}}%
\pgfpathcurveto{\pgfqpoint{4.163561in}{4.181253in}}{\pgfqpoint{4.174160in}{4.176863in}}{\pgfqpoint{4.185210in}{4.176863in}}%
\pgfpathclose%
\pgfusepath{stroke,fill}%
\end{pgfscope}%
\begin{pgfscope}%
\pgfpathrectangle{\pgfqpoint{0.481978in}{0.331635in}}{\pgfqpoint{9.300000in}{7.700000in}}%
\pgfusepath{clip}%
\pgfsetbuttcap%
\pgfsetroundjoin%
\definecolor{currentfill}{rgb}{1.000000,0.705882,0.509804}%
\pgfsetfillcolor{currentfill}%
\pgfsetlinewidth{0.481800pt}%
\definecolor{currentstroke}{rgb}{1.000000,1.000000,1.000000}%
\pgfsetstrokecolor{currentstroke}%
\pgfsetdash{}{0pt}%
\pgfpathmoveto{\pgfqpoint{4.500422in}{3.003250in}}%
\pgfpathcurveto{\pgfqpoint{4.511472in}{3.003250in}}{\pgfqpoint{4.522071in}{3.007640in}}{\pgfqpoint{4.529885in}{3.015454in}}%
\pgfpathcurveto{\pgfqpoint{4.537699in}{3.023267in}}{\pgfqpoint{4.542089in}{3.033866in}}{\pgfqpoint{4.542089in}{3.044916in}}%
\pgfpathcurveto{\pgfqpoint{4.542089in}{3.055967in}}{\pgfqpoint{4.537699in}{3.066566in}}{\pgfqpoint{4.529885in}{3.074379in}}%
\pgfpathcurveto{\pgfqpoint{4.522071in}{3.082193in}}{\pgfqpoint{4.511472in}{3.086583in}}{\pgfqpoint{4.500422in}{3.086583in}}%
\pgfpathcurveto{\pgfqpoint{4.489372in}{3.086583in}}{\pgfqpoint{4.478773in}{3.082193in}}{\pgfqpoint{4.470959in}{3.074379in}}%
\pgfpathcurveto{\pgfqpoint{4.463146in}{3.066566in}}{\pgfqpoint{4.458755in}{3.055967in}}{\pgfqpoint{4.458755in}{3.044916in}}%
\pgfpathcurveto{\pgfqpoint{4.458755in}{3.033866in}}{\pgfqpoint{4.463146in}{3.023267in}}{\pgfqpoint{4.470959in}{3.015454in}}%
\pgfpathcurveto{\pgfqpoint{4.478773in}{3.007640in}}{\pgfqpoint{4.489372in}{3.003250in}}{\pgfqpoint{4.500422in}{3.003250in}}%
\pgfpathclose%
\pgfusepath{stroke,fill}%
\end{pgfscope}%
\begin{pgfscope}%
\pgfpathrectangle{\pgfqpoint{0.481978in}{0.331635in}}{\pgfqpoint{9.300000in}{7.700000in}}%
\pgfusepath{clip}%
\pgfsetbuttcap%
\pgfsetroundjoin%
\definecolor{currentfill}{rgb}{1.000000,0.705882,0.509804}%
\pgfsetfillcolor{currentfill}%
\pgfsetlinewidth{0.481800pt}%
\definecolor{currentstroke}{rgb}{1.000000,1.000000,1.000000}%
\pgfsetstrokecolor{currentstroke}%
\pgfsetdash{}{0pt}%
\pgfpathmoveto{\pgfqpoint{4.103119in}{4.252690in}}%
\pgfpathcurveto{\pgfqpoint{4.114169in}{4.252690in}}{\pgfqpoint{4.124768in}{4.257080in}}{\pgfqpoint{4.132582in}{4.264893in}}%
\pgfpathcurveto{\pgfqpoint{4.140396in}{4.272707in}}{\pgfqpoint{4.144786in}{4.283306in}}{\pgfqpoint{4.144786in}{4.294356in}}%
\pgfpathcurveto{\pgfqpoint{4.144786in}{4.305406in}}{\pgfqpoint{4.140396in}{4.316005in}}{\pgfqpoint{4.132582in}{4.323819in}}%
\pgfpathcurveto{\pgfqpoint{4.124768in}{4.331633in}}{\pgfqpoint{4.114169in}{4.336023in}}{\pgfqpoint{4.103119in}{4.336023in}}%
\pgfpathcurveto{\pgfqpoint{4.092069in}{4.336023in}}{\pgfqpoint{4.081470in}{4.331633in}}{\pgfqpoint{4.073656in}{4.323819in}}%
\pgfpathcurveto{\pgfqpoint{4.065843in}{4.316005in}}{\pgfqpoint{4.061452in}{4.305406in}}{\pgfqpoint{4.061452in}{4.294356in}}%
\pgfpathcurveto{\pgfqpoint{4.061452in}{4.283306in}}{\pgfqpoint{4.065843in}{4.272707in}}{\pgfqpoint{4.073656in}{4.264893in}}%
\pgfpathcurveto{\pgfqpoint{4.081470in}{4.257080in}}{\pgfqpoint{4.092069in}{4.252690in}}{\pgfqpoint{4.103119in}{4.252690in}}%
\pgfpathclose%
\pgfusepath{stroke,fill}%
\end{pgfscope}%
\begin{pgfscope}%
\pgfpathrectangle{\pgfqpoint{0.481978in}{0.331635in}}{\pgfqpoint{9.300000in}{7.700000in}}%
\pgfusepath{clip}%
\pgfsetbuttcap%
\pgfsetroundjoin%
\definecolor{currentfill}{rgb}{1.000000,0.705882,0.509804}%
\pgfsetfillcolor{currentfill}%
\pgfsetlinewidth{0.481800pt}%
\definecolor{currentstroke}{rgb}{1.000000,1.000000,1.000000}%
\pgfsetstrokecolor{currentstroke}%
\pgfsetdash{}{0pt}%
\pgfpathmoveto{\pgfqpoint{3.235019in}{4.465313in}}%
\pgfpathcurveto{\pgfqpoint{3.246069in}{4.465313in}}{\pgfqpoint{3.256668in}{4.469703in}}{\pgfqpoint{3.264482in}{4.477517in}}%
\pgfpathcurveto{\pgfqpoint{3.272295in}{4.485331in}}{\pgfqpoint{3.276686in}{4.495930in}}{\pgfqpoint{3.276686in}{4.506980in}}%
\pgfpathcurveto{\pgfqpoint{3.276686in}{4.518030in}}{\pgfqpoint{3.272295in}{4.528629in}}{\pgfqpoint{3.264482in}{4.536443in}}%
\pgfpathcurveto{\pgfqpoint{3.256668in}{4.544256in}}{\pgfqpoint{3.246069in}{4.548647in}}{\pgfqpoint{3.235019in}{4.548647in}}%
\pgfpathcurveto{\pgfqpoint{3.223969in}{4.548647in}}{\pgfqpoint{3.213370in}{4.544256in}}{\pgfqpoint{3.205556in}{4.536443in}}%
\pgfpathcurveto{\pgfqpoint{3.197743in}{4.528629in}}{\pgfqpoint{3.193352in}{4.518030in}}{\pgfqpoint{3.193352in}{4.506980in}}%
\pgfpathcurveto{\pgfqpoint{3.193352in}{4.495930in}}{\pgfqpoint{3.197743in}{4.485331in}}{\pgfqpoint{3.205556in}{4.477517in}}%
\pgfpathcurveto{\pgfqpoint{3.213370in}{4.469703in}}{\pgfqpoint{3.223969in}{4.465313in}}{\pgfqpoint{3.235019in}{4.465313in}}%
\pgfpathclose%
\pgfusepath{stroke,fill}%
\end{pgfscope}%
\begin{pgfscope}%
\pgfpathrectangle{\pgfqpoint{0.481978in}{0.331635in}}{\pgfqpoint{9.300000in}{7.700000in}}%
\pgfusepath{clip}%
\pgfsetbuttcap%
\pgfsetroundjoin%
\definecolor{currentfill}{rgb}{1.000000,0.705882,0.509804}%
\pgfsetfillcolor{currentfill}%
\pgfsetlinewidth{0.481800pt}%
\definecolor{currentstroke}{rgb}{1.000000,1.000000,1.000000}%
\pgfsetstrokecolor{currentstroke}%
\pgfsetdash{}{0pt}%
\pgfpathmoveto{\pgfqpoint{1.466431in}{3.069920in}}%
\pgfpathcurveto{\pgfqpoint{1.477481in}{3.069920in}}{\pgfqpoint{1.488080in}{3.074310in}}{\pgfqpoint{1.495894in}{3.082124in}}%
\pgfpathcurveto{\pgfqpoint{1.503708in}{3.089937in}}{\pgfqpoint{1.508098in}{3.100536in}}{\pgfqpoint{1.508098in}{3.111586in}}%
\pgfpathcurveto{\pgfqpoint{1.508098in}{3.122636in}}{\pgfqpoint{1.503708in}{3.133235in}}{\pgfqpoint{1.495894in}{3.141049in}}%
\pgfpathcurveto{\pgfqpoint{1.488080in}{3.148863in}}{\pgfqpoint{1.477481in}{3.153253in}}{\pgfqpoint{1.466431in}{3.153253in}}%
\pgfpathcurveto{\pgfqpoint{1.455381in}{3.153253in}}{\pgfqpoint{1.444782in}{3.148863in}}{\pgfqpoint{1.436968in}{3.141049in}}%
\pgfpathcurveto{\pgfqpoint{1.429155in}{3.133235in}}{\pgfqpoint{1.424764in}{3.122636in}}{\pgfqpoint{1.424764in}{3.111586in}}%
\pgfpathcurveto{\pgfqpoint{1.424764in}{3.100536in}}{\pgfqpoint{1.429155in}{3.089937in}}{\pgfqpoint{1.436968in}{3.082124in}}%
\pgfpathcurveto{\pgfqpoint{1.444782in}{3.074310in}}{\pgfqpoint{1.455381in}{3.069920in}}{\pgfqpoint{1.466431in}{3.069920in}}%
\pgfpathclose%
\pgfusepath{stroke,fill}%
\end{pgfscope}%
\begin{pgfscope}%
\pgfpathrectangle{\pgfqpoint{0.481978in}{0.331635in}}{\pgfqpoint{9.300000in}{7.700000in}}%
\pgfusepath{clip}%
\pgfsetbuttcap%
\pgfsetroundjoin%
\definecolor{currentfill}{rgb}{1.000000,0.705882,0.509804}%
\pgfsetfillcolor{currentfill}%
\pgfsetlinewidth{0.481800pt}%
\definecolor{currentstroke}{rgb}{1.000000,1.000000,1.000000}%
\pgfsetstrokecolor{currentstroke}%
\pgfsetdash{}{0pt}%
\pgfpathmoveto{\pgfqpoint{5.649061in}{4.441319in}}%
\pgfpathcurveto{\pgfqpoint{5.660112in}{4.441319in}}{\pgfqpoint{5.670711in}{4.445709in}}{\pgfqpoint{5.678524in}{4.453523in}}%
\pgfpathcurveto{\pgfqpoint{5.686338in}{4.461336in}}{\pgfqpoint{5.690728in}{4.471935in}}{\pgfqpoint{5.690728in}{4.482985in}}%
\pgfpathcurveto{\pgfqpoint{5.690728in}{4.494036in}}{\pgfqpoint{5.686338in}{4.504635in}}{\pgfqpoint{5.678524in}{4.512448in}}%
\pgfpathcurveto{\pgfqpoint{5.670711in}{4.520262in}}{\pgfqpoint{5.660112in}{4.524652in}}{\pgfqpoint{5.649061in}{4.524652in}}%
\pgfpathcurveto{\pgfqpoint{5.638011in}{4.524652in}}{\pgfqpoint{5.627412in}{4.520262in}}{\pgfqpoint{5.619599in}{4.512448in}}%
\pgfpathcurveto{\pgfqpoint{5.611785in}{4.504635in}}{\pgfqpoint{5.607395in}{4.494036in}}{\pgfqpoint{5.607395in}{4.482985in}}%
\pgfpathcurveto{\pgfqpoint{5.607395in}{4.471935in}}{\pgfqpoint{5.611785in}{4.461336in}}{\pgfqpoint{5.619599in}{4.453523in}}%
\pgfpathcurveto{\pgfqpoint{5.627412in}{4.445709in}}{\pgfqpoint{5.638011in}{4.441319in}}{\pgfqpoint{5.649061in}{4.441319in}}%
\pgfpathclose%
\pgfusepath{stroke,fill}%
\end{pgfscope}%
\begin{pgfscope}%
\pgfpathrectangle{\pgfqpoint{0.481978in}{0.331635in}}{\pgfqpoint{9.300000in}{7.700000in}}%
\pgfusepath{clip}%
\pgfsetbuttcap%
\pgfsetroundjoin%
\definecolor{currentfill}{rgb}{1.000000,0.705882,0.509804}%
\pgfsetfillcolor{currentfill}%
\pgfsetlinewidth{0.481800pt}%
\definecolor{currentstroke}{rgb}{1.000000,1.000000,1.000000}%
\pgfsetstrokecolor{currentstroke}%
\pgfsetdash{}{0pt}%
\pgfpathmoveto{\pgfqpoint{3.679062in}{2.651300in}}%
\pgfpathcurveto{\pgfqpoint{3.690112in}{2.651300in}}{\pgfqpoint{3.700711in}{2.655690in}}{\pgfqpoint{3.708525in}{2.663503in}}%
\pgfpathcurveto{\pgfqpoint{3.716338in}{2.671317in}}{\pgfqpoint{3.720729in}{2.681916in}}{\pgfqpoint{3.720729in}{2.692966in}}%
\pgfpathcurveto{\pgfqpoint{3.720729in}{2.704016in}}{\pgfqpoint{3.716338in}{2.714615in}}{\pgfqpoint{3.708525in}{2.722429in}}%
\pgfpathcurveto{\pgfqpoint{3.700711in}{2.730243in}}{\pgfqpoint{3.690112in}{2.734633in}}{\pgfqpoint{3.679062in}{2.734633in}}%
\pgfpathcurveto{\pgfqpoint{3.668012in}{2.734633in}}{\pgfqpoint{3.657413in}{2.730243in}}{\pgfqpoint{3.649599in}{2.722429in}}%
\pgfpathcurveto{\pgfqpoint{3.641786in}{2.714615in}}{\pgfqpoint{3.637395in}{2.704016in}}{\pgfqpoint{3.637395in}{2.692966in}}%
\pgfpathcurveto{\pgfqpoint{3.637395in}{2.681916in}}{\pgfqpoint{3.641786in}{2.671317in}}{\pgfqpoint{3.649599in}{2.663503in}}%
\pgfpathcurveto{\pgfqpoint{3.657413in}{2.655690in}}{\pgfqpoint{3.668012in}{2.651300in}}{\pgfqpoint{3.679062in}{2.651300in}}%
\pgfpathclose%
\pgfusepath{stroke,fill}%
\end{pgfscope}%
\begin{pgfscope}%
\pgfpathrectangle{\pgfqpoint{0.481978in}{0.331635in}}{\pgfqpoint{9.300000in}{7.700000in}}%
\pgfusepath{clip}%
\pgfsetbuttcap%
\pgfsetroundjoin%
\definecolor{currentfill}{rgb}{1.000000,0.705882,0.509804}%
\pgfsetfillcolor{currentfill}%
\pgfsetlinewidth{0.481800pt}%
\definecolor{currentstroke}{rgb}{1.000000,1.000000,1.000000}%
\pgfsetstrokecolor{currentstroke}%
\pgfsetdash{}{0pt}%
\pgfpathmoveto{\pgfqpoint{2.616545in}{4.527795in}}%
\pgfpathcurveto{\pgfqpoint{2.627596in}{4.527795in}}{\pgfqpoint{2.638195in}{4.532186in}}{\pgfqpoint{2.646008in}{4.539999in}}%
\pgfpathcurveto{\pgfqpoint{2.653822in}{4.547813in}}{\pgfqpoint{2.658212in}{4.558412in}}{\pgfqpoint{2.658212in}{4.569462in}}%
\pgfpathcurveto{\pgfqpoint{2.658212in}{4.580512in}}{\pgfqpoint{2.653822in}{4.591111in}}{\pgfqpoint{2.646008in}{4.598925in}}%
\pgfpathcurveto{\pgfqpoint{2.638195in}{4.606739in}}{\pgfqpoint{2.627596in}{4.611129in}}{\pgfqpoint{2.616545in}{4.611129in}}%
\pgfpathcurveto{\pgfqpoint{2.605495in}{4.611129in}}{\pgfqpoint{2.594896in}{4.606739in}}{\pgfqpoint{2.587083in}{4.598925in}}%
\pgfpathcurveto{\pgfqpoint{2.579269in}{4.591111in}}{\pgfqpoint{2.574879in}{4.580512in}}{\pgfqpoint{2.574879in}{4.569462in}}%
\pgfpathcurveto{\pgfqpoint{2.574879in}{4.558412in}}{\pgfqpoint{2.579269in}{4.547813in}}{\pgfqpoint{2.587083in}{4.539999in}}%
\pgfpathcurveto{\pgfqpoint{2.594896in}{4.532186in}}{\pgfqpoint{2.605495in}{4.527795in}}{\pgfqpoint{2.616545in}{4.527795in}}%
\pgfpathclose%
\pgfusepath{stroke,fill}%
\end{pgfscope}%
\begin{pgfscope}%
\pgfpathrectangle{\pgfqpoint{0.481978in}{0.331635in}}{\pgfqpoint{9.300000in}{7.700000in}}%
\pgfusepath{clip}%
\pgfsetbuttcap%
\pgfsetroundjoin%
\definecolor{currentfill}{rgb}{1.000000,0.705882,0.509804}%
\pgfsetfillcolor{currentfill}%
\pgfsetlinewidth{0.481800pt}%
\definecolor{currentstroke}{rgb}{1.000000,1.000000,1.000000}%
\pgfsetstrokecolor{currentstroke}%
\pgfsetdash{}{0pt}%
\pgfpathmoveto{\pgfqpoint{5.192724in}{4.988152in}}%
\pgfpathcurveto{\pgfqpoint{5.203774in}{4.988152in}}{\pgfqpoint{5.214373in}{4.992542in}}{\pgfqpoint{5.222187in}{5.000355in}}%
\pgfpathcurveto{\pgfqpoint{5.230000in}{5.008169in}}{\pgfqpoint{5.234391in}{5.018768in}}{\pgfqpoint{5.234391in}{5.029818in}}%
\pgfpathcurveto{\pgfqpoint{5.234391in}{5.040868in}}{\pgfqpoint{5.230000in}{5.051467in}}{\pgfqpoint{5.222187in}{5.059281in}}%
\pgfpathcurveto{\pgfqpoint{5.214373in}{5.067095in}}{\pgfqpoint{5.203774in}{5.071485in}}{\pgfqpoint{5.192724in}{5.071485in}}%
\pgfpathcurveto{\pgfqpoint{5.181674in}{5.071485in}}{\pgfqpoint{5.171075in}{5.067095in}}{\pgfqpoint{5.163261in}{5.059281in}}%
\pgfpathcurveto{\pgfqpoint{5.155447in}{5.051467in}}{\pgfqpoint{5.151057in}{5.040868in}}{\pgfqpoint{5.151057in}{5.029818in}}%
\pgfpathcurveto{\pgfqpoint{5.151057in}{5.018768in}}{\pgfqpoint{5.155447in}{5.008169in}}{\pgfqpoint{5.163261in}{5.000355in}}%
\pgfpathcurveto{\pgfqpoint{5.171075in}{4.992542in}}{\pgfqpoint{5.181674in}{4.988152in}}{\pgfqpoint{5.192724in}{4.988152in}}%
\pgfpathclose%
\pgfusepath{stroke,fill}%
\end{pgfscope}%
\begin{pgfscope}%
\pgfpathrectangle{\pgfqpoint{0.481978in}{0.331635in}}{\pgfqpoint{9.300000in}{7.700000in}}%
\pgfusepath{clip}%
\pgfsetbuttcap%
\pgfsetroundjoin%
\definecolor{currentfill}{rgb}{1.000000,0.705882,0.509804}%
\pgfsetfillcolor{currentfill}%
\pgfsetlinewidth{0.481800pt}%
\definecolor{currentstroke}{rgb}{1.000000,1.000000,1.000000}%
\pgfsetstrokecolor{currentstroke}%
\pgfsetdash{}{0pt}%
\pgfpathmoveto{\pgfqpoint{3.374008in}{2.945401in}}%
\pgfpathcurveto{\pgfqpoint{3.385058in}{2.945401in}}{\pgfqpoint{3.395657in}{2.949791in}}{\pgfqpoint{3.403471in}{2.957605in}}%
\pgfpathcurveto{\pgfqpoint{3.411284in}{2.965419in}}{\pgfqpoint{3.415675in}{2.976018in}}{\pgfqpoint{3.415675in}{2.987068in}}%
\pgfpathcurveto{\pgfqpoint{3.415675in}{2.998118in}}{\pgfqpoint{3.411284in}{3.008717in}}{\pgfqpoint{3.403471in}{3.016531in}}%
\pgfpathcurveto{\pgfqpoint{3.395657in}{3.024344in}}{\pgfqpoint{3.385058in}{3.028734in}}{\pgfqpoint{3.374008in}{3.028734in}}%
\pgfpathcurveto{\pgfqpoint{3.362958in}{3.028734in}}{\pgfqpoint{3.352359in}{3.024344in}}{\pgfqpoint{3.344545in}{3.016531in}}%
\pgfpathcurveto{\pgfqpoint{3.336732in}{3.008717in}}{\pgfqpoint{3.332341in}{2.998118in}}{\pgfqpoint{3.332341in}{2.987068in}}%
\pgfpathcurveto{\pgfqpoint{3.332341in}{2.976018in}}{\pgfqpoint{3.336732in}{2.965419in}}{\pgfqpoint{3.344545in}{2.957605in}}%
\pgfpathcurveto{\pgfqpoint{3.352359in}{2.949791in}}{\pgfqpoint{3.362958in}{2.945401in}}{\pgfqpoint{3.374008in}{2.945401in}}%
\pgfpathclose%
\pgfusepath{stroke,fill}%
\end{pgfscope}%
\begin{pgfscope}%
\pgfpathrectangle{\pgfqpoint{0.481978in}{0.331635in}}{\pgfqpoint{9.300000in}{7.700000in}}%
\pgfusepath{clip}%
\pgfsetbuttcap%
\pgfsetroundjoin%
\definecolor{currentfill}{rgb}{1.000000,0.705882,0.509804}%
\pgfsetfillcolor{currentfill}%
\pgfsetlinewidth{0.481800pt}%
\definecolor{currentstroke}{rgb}{1.000000,1.000000,1.000000}%
\pgfsetstrokecolor{currentstroke}%
\pgfsetdash{}{0pt}%
\pgfpathmoveto{\pgfqpoint{4.257082in}{4.276759in}}%
\pgfpathcurveto{\pgfqpoint{4.268132in}{4.276759in}}{\pgfqpoint{4.278731in}{4.281149in}}{\pgfqpoint{4.286544in}{4.288962in}}%
\pgfpathcurveto{\pgfqpoint{4.294358in}{4.296776in}}{\pgfqpoint{4.298748in}{4.307375in}}{\pgfqpoint{4.298748in}{4.318425in}}%
\pgfpathcurveto{\pgfqpoint{4.298748in}{4.329475in}}{\pgfqpoint{4.294358in}{4.340074in}}{\pgfqpoint{4.286544in}{4.347888in}}%
\pgfpathcurveto{\pgfqpoint{4.278731in}{4.355702in}}{\pgfqpoint{4.268132in}{4.360092in}}{\pgfqpoint{4.257082in}{4.360092in}}%
\pgfpathcurveto{\pgfqpoint{4.246031in}{4.360092in}}{\pgfqpoint{4.235432in}{4.355702in}}{\pgfqpoint{4.227619in}{4.347888in}}%
\pgfpathcurveto{\pgfqpoint{4.219805in}{4.340074in}}{\pgfqpoint{4.215415in}{4.329475in}}{\pgfqpoint{4.215415in}{4.318425in}}%
\pgfpathcurveto{\pgfqpoint{4.215415in}{4.307375in}}{\pgfqpoint{4.219805in}{4.296776in}}{\pgfqpoint{4.227619in}{4.288962in}}%
\pgfpathcurveto{\pgfqpoint{4.235432in}{4.281149in}}{\pgfqpoint{4.246031in}{4.276759in}}{\pgfqpoint{4.257082in}{4.276759in}}%
\pgfpathclose%
\pgfusepath{stroke,fill}%
\end{pgfscope}%
\begin{pgfscope}%
\pgfpathrectangle{\pgfqpoint{0.481978in}{0.331635in}}{\pgfqpoint{9.300000in}{7.700000in}}%
\pgfusepath{clip}%
\pgfsetbuttcap%
\pgfsetroundjoin%
\definecolor{currentfill}{rgb}{1.000000,0.705882,0.509804}%
\pgfsetfillcolor{currentfill}%
\pgfsetlinewidth{0.481800pt}%
\definecolor{currentstroke}{rgb}{1.000000,1.000000,1.000000}%
\pgfsetstrokecolor{currentstroke}%
\pgfsetdash{}{0pt}%
\pgfpathmoveto{\pgfqpoint{1.027576in}{3.744338in}}%
\pgfpathcurveto{\pgfqpoint{1.038626in}{3.744338in}}{\pgfqpoint{1.049225in}{3.748728in}}{\pgfqpoint{1.057038in}{3.756542in}}%
\pgfpathcurveto{\pgfqpoint{1.064852in}{3.764355in}}{\pgfqpoint{1.069242in}{3.774954in}}{\pgfqpoint{1.069242in}{3.786004in}}%
\pgfpathcurveto{\pgfqpoint{1.069242in}{3.797054in}}{\pgfqpoint{1.064852in}{3.807653in}}{\pgfqpoint{1.057038in}{3.815467in}}%
\pgfpathcurveto{\pgfqpoint{1.049225in}{3.823281in}}{\pgfqpoint{1.038626in}{3.827671in}}{\pgfqpoint{1.027576in}{3.827671in}}%
\pgfpathcurveto{\pgfqpoint{1.016525in}{3.827671in}}{\pgfqpoint{1.005926in}{3.823281in}}{\pgfqpoint{0.998113in}{3.815467in}}%
\pgfpathcurveto{\pgfqpoint{0.990299in}{3.807653in}}{\pgfqpoint{0.985909in}{3.797054in}}{\pgfqpoint{0.985909in}{3.786004in}}%
\pgfpathcurveto{\pgfqpoint{0.985909in}{3.774954in}}{\pgfqpoint{0.990299in}{3.764355in}}{\pgfqpoint{0.998113in}{3.756542in}}%
\pgfpathcurveto{\pgfqpoint{1.005926in}{3.748728in}}{\pgfqpoint{1.016525in}{3.744338in}}{\pgfqpoint{1.027576in}{3.744338in}}%
\pgfpathclose%
\pgfusepath{stroke,fill}%
\end{pgfscope}%
\begin{pgfscope}%
\pgfpathrectangle{\pgfqpoint{0.481978in}{0.331635in}}{\pgfqpoint{9.300000in}{7.700000in}}%
\pgfusepath{clip}%
\pgfsetbuttcap%
\pgfsetroundjoin%
\definecolor{currentfill}{rgb}{1.000000,0.705882,0.509804}%
\pgfsetfillcolor{currentfill}%
\pgfsetlinewidth{0.481800pt}%
\definecolor{currentstroke}{rgb}{1.000000,1.000000,1.000000}%
\pgfsetstrokecolor{currentstroke}%
\pgfsetdash{}{0pt}%
\pgfpathmoveto{\pgfqpoint{2.111519in}{4.599199in}}%
\pgfpathcurveto{\pgfqpoint{2.122569in}{4.599199in}}{\pgfqpoint{2.133168in}{4.603589in}}{\pgfqpoint{2.140982in}{4.611403in}}%
\pgfpathcurveto{\pgfqpoint{2.148795in}{4.619216in}}{\pgfqpoint{2.153186in}{4.629815in}}{\pgfqpoint{2.153186in}{4.640866in}}%
\pgfpathcurveto{\pgfqpoint{2.153186in}{4.651916in}}{\pgfqpoint{2.148795in}{4.662515in}}{\pgfqpoint{2.140982in}{4.670328in}}%
\pgfpathcurveto{\pgfqpoint{2.133168in}{4.678142in}}{\pgfqpoint{2.122569in}{4.682532in}}{\pgfqpoint{2.111519in}{4.682532in}}%
\pgfpathcurveto{\pgfqpoint{2.100469in}{4.682532in}}{\pgfqpoint{2.089870in}{4.678142in}}{\pgfqpoint{2.082056in}{4.670328in}}%
\pgfpathcurveto{\pgfqpoint{2.074243in}{4.662515in}}{\pgfqpoint{2.069852in}{4.651916in}}{\pgfqpoint{2.069852in}{4.640866in}}%
\pgfpathcurveto{\pgfqpoint{2.069852in}{4.629815in}}{\pgfqpoint{2.074243in}{4.619216in}}{\pgfqpoint{2.082056in}{4.611403in}}%
\pgfpathcurveto{\pgfqpoint{2.089870in}{4.603589in}}{\pgfqpoint{2.100469in}{4.599199in}}{\pgfqpoint{2.111519in}{4.599199in}}%
\pgfpathclose%
\pgfusepath{stroke,fill}%
\end{pgfscope}%
\begin{pgfscope}%
\pgfpathrectangle{\pgfqpoint{0.481978in}{0.331635in}}{\pgfqpoint{9.300000in}{7.700000in}}%
\pgfusepath{clip}%
\pgfsetbuttcap%
\pgfsetroundjoin%
\definecolor{currentfill}{rgb}{1.000000,0.705882,0.509804}%
\pgfsetfillcolor{currentfill}%
\pgfsetlinewidth{0.481800pt}%
\definecolor{currentstroke}{rgb}{1.000000,1.000000,1.000000}%
\pgfsetstrokecolor{currentstroke}%
\pgfsetdash{}{0pt}%
\pgfpathmoveto{\pgfqpoint{2.540709in}{6.106130in}}%
\pgfpathcurveto{\pgfqpoint{2.551759in}{6.106130in}}{\pgfqpoint{2.562358in}{6.110521in}}{\pgfqpoint{2.570172in}{6.118334in}}%
\pgfpathcurveto{\pgfqpoint{2.577985in}{6.126148in}}{\pgfqpoint{2.582376in}{6.136747in}}{\pgfqpoint{2.582376in}{6.147797in}}%
\pgfpathcurveto{\pgfqpoint{2.582376in}{6.158847in}}{\pgfqpoint{2.577985in}{6.169446in}}{\pgfqpoint{2.570172in}{6.177260in}}%
\pgfpathcurveto{\pgfqpoint{2.562358in}{6.185073in}}{\pgfqpoint{2.551759in}{6.189464in}}{\pgfqpoint{2.540709in}{6.189464in}}%
\pgfpathcurveto{\pgfqpoint{2.529659in}{6.189464in}}{\pgfqpoint{2.519060in}{6.185073in}}{\pgfqpoint{2.511246in}{6.177260in}}%
\pgfpathcurveto{\pgfqpoint{2.503432in}{6.169446in}}{\pgfqpoint{2.499042in}{6.158847in}}{\pgfqpoint{2.499042in}{6.147797in}}%
\pgfpathcurveto{\pgfqpoint{2.499042in}{6.136747in}}{\pgfqpoint{2.503432in}{6.126148in}}{\pgfqpoint{2.511246in}{6.118334in}}%
\pgfpathcurveto{\pgfqpoint{2.519060in}{6.110521in}}{\pgfqpoint{2.529659in}{6.106130in}}{\pgfqpoint{2.540709in}{6.106130in}}%
\pgfpathclose%
\pgfusepath{stroke,fill}%
\end{pgfscope}%
\begin{pgfscope}%
\pgfpathrectangle{\pgfqpoint{0.481978in}{0.331635in}}{\pgfqpoint{9.300000in}{7.700000in}}%
\pgfusepath{clip}%
\pgfsetbuttcap%
\pgfsetroundjoin%
\definecolor{currentfill}{rgb}{1.000000,0.705882,0.509804}%
\pgfsetfillcolor{currentfill}%
\pgfsetlinewidth{0.481800pt}%
\definecolor{currentstroke}{rgb}{1.000000,1.000000,1.000000}%
\pgfsetstrokecolor{currentstroke}%
\pgfsetdash{}{0pt}%
\pgfpathmoveto{\pgfqpoint{2.593367in}{2.323294in}}%
\pgfpathcurveto{\pgfqpoint{2.604417in}{2.323294in}}{\pgfqpoint{2.615016in}{2.327684in}}{\pgfqpoint{2.622829in}{2.335498in}}%
\pgfpathcurveto{\pgfqpoint{2.630643in}{2.343311in}}{\pgfqpoint{2.635033in}{2.353910in}}{\pgfqpoint{2.635033in}{2.364960in}}%
\pgfpathcurveto{\pgfqpoint{2.635033in}{2.376011in}}{\pgfqpoint{2.630643in}{2.386610in}}{\pgfqpoint{2.622829in}{2.394423in}}%
\pgfpathcurveto{\pgfqpoint{2.615016in}{2.402237in}}{\pgfqpoint{2.604417in}{2.406627in}}{\pgfqpoint{2.593367in}{2.406627in}}%
\pgfpathcurveto{\pgfqpoint{2.582317in}{2.406627in}}{\pgfqpoint{2.571717in}{2.402237in}}{\pgfqpoint{2.563904in}{2.394423in}}%
\pgfpathcurveto{\pgfqpoint{2.556090in}{2.386610in}}{\pgfqpoint{2.551700in}{2.376011in}}{\pgfqpoint{2.551700in}{2.364960in}}%
\pgfpathcurveto{\pgfqpoint{2.551700in}{2.353910in}}{\pgfqpoint{2.556090in}{2.343311in}}{\pgfqpoint{2.563904in}{2.335498in}}%
\pgfpathcurveto{\pgfqpoint{2.571717in}{2.327684in}}{\pgfqpoint{2.582317in}{2.323294in}}{\pgfqpoint{2.593367in}{2.323294in}}%
\pgfpathclose%
\pgfusepath{stroke,fill}%
\end{pgfscope}%
\begin{pgfscope}%
\pgfpathrectangle{\pgfqpoint{0.481978in}{0.331635in}}{\pgfqpoint{9.300000in}{7.700000in}}%
\pgfusepath{clip}%
\pgfsetbuttcap%
\pgfsetroundjoin%
\definecolor{currentfill}{rgb}{1.000000,0.705882,0.509804}%
\pgfsetfillcolor{currentfill}%
\pgfsetlinewidth{0.481800pt}%
\definecolor{currentstroke}{rgb}{1.000000,1.000000,1.000000}%
\pgfsetstrokecolor{currentstroke}%
\pgfsetdash{}{0pt}%
\pgfpathmoveto{\pgfqpoint{1.941607in}{3.916070in}}%
\pgfpathcurveto{\pgfqpoint{1.952657in}{3.916070in}}{\pgfqpoint{1.963256in}{3.920460in}}{\pgfqpoint{1.971070in}{3.928274in}}%
\pgfpathcurveto{\pgfqpoint{1.978883in}{3.936087in}}{\pgfqpoint{1.983274in}{3.946686in}}{\pgfqpoint{1.983274in}{3.957736in}}%
\pgfpathcurveto{\pgfqpoint{1.983274in}{3.968787in}}{\pgfqpoint{1.978883in}{3.979386in}}{\pgfqpoint{1.971070in}{3.987199in}}%
\pgfpathcurveto{\pgfqpoint{1.963256in}{3.995013in}}{\pgfqpoint{1.952657in}{3.999403in}}{\pgfqpoint{1.941607in}{3.999403in}}%
\pgfpathcurveto{\pgfqpoint{1.930557in}{3.999403in}}{\pgfqpoint{1.919958in}{3.995013in}}{\pgfqpoint{1.912144in}{3.987199in}}%
\pgfpathcurveto{\pgfqpoint{1.904331in}{3.979386in}}{\pgfqpoint{1.899940in}{3.968787in}}{\pgfqpoint{1.899940in}{3.957736in}}%
\pgfpathcurveto{\pgfqpoint{1.899940in}{3.946686in}}{\pgfqpoint{1.904331in}{3.936087in}}{\pgfqpoint{1.912144in}{3.928274in}}%
\pgfpathcurveto{\pgfqpoint{1.919958in}{3.920460in}}{\pgfqpoint{1.930557in}{3.916070in}}{\pgfqpoint{1.941607in}{3.916070in}}%
\pgfpathclose%
\pgfusepath{stroke,fill}%
\end{pgfscope}%
\begin{pgfscope}%
\pgfpathrectangle{\pgfqpoint{0.481978in}{0.331635in}}{\pgfqpoint{9.300000in}{7.700000in}}%
\pgfusepath{clip}%
\pgfsetbuttcap%
\pgfsetroundjoin%
\definecolor{currentfill}{rgb}{1.000000,0.705882,0.509804}%
\pgfsetfillcolor{currentfill}%
\pgfsetlinewidth{0.481800pt}%
\definecolor{currentstroke}{rgb}{1.000000,1.000000,1.000000}%
\pgfsetstrokecolor{currentstroke}%
\pgfsetdash{}{0pt}%
\pgfpathmoveto{\pgfqpoint{2.880395in}{3.273104in}}%
\pgfpathcurveto{\pgfqpoint{2.891445in}{3.273104in}}{\pgfqpoint{2.902044in}{3.277494in}}{\pgfqpoint{2.909858in}{3.285308in}}%
\pgfpathcurveto{\pgfqpoint{2.917671in}{3.293122in}}{\pgfqpoint{2.922061in}{3.303721in}}{\pgfqpoint{2.922061in}{3.314771in}}%
\pgfpathcurveto{\pgfqpoint{2.922061in}{3.325821in}}{\pgfqpoint{2.917671in}{3.336420in}}{\pgfqpoint{2.909858in}{3.344234in}}%
\pgfpathcurveto{\pgfqpoint{2.902044in}{3.352047in}}{\pgfqpoint{2.891445in}{3.356438in}}{\pgfqpoint{2.880395in}{3.356438in}}%
\pgfpathcurveto{\pgfqpoint{2.869345in}{3.356438in}}{\pgfqpoint{2.858746in}{3.352047in}}{\pgfqpoint{2.850932in}{3.344234in}}%
\pgfpathcurveto{\pgfqpoint{2.843118in}{3.336420in}}{\pgfqpoint{2.838728in}{3.325821in}}{\pgfqpoint{2.838728in}{3.314771in}}%
\pgfpathcurveto{\pgfqpoint{2.838728in}{3.303721in}}{\pgfqpoint{2.843118in}{3.293122in}}{\pgfqpoint{2.850932in}{3.285308in}}%
\pgfpathcurveto{\pgfqpoint{2.858746in}{3.277494in}}{\pgfqpoint{2.869345in}{3.273104in}}{\pgfqpoint{2.880395in}{3.273104in}}%
\pgfpathclose%
\pgfusepath{stroke,fill}%
\end{pgfscope}%
\begin{pgfscope}%
\pgfpathrectangle{\pgfqpoint{0.481978in}{0.331635in}}{\pgfqpoint{9.300000in}{7.700000in}}%
\pgfusepath{clip}%
\pgfsetbuttcap%
\pgfsetroundjoin%
\definecolor{currentfill}{rgb}{1.000000,0.705882,0.509804}%
\pgfsetfillcolor{currentfill}%
\pgfsetlinewidth{0.481800pt}%
\definecolor{currentstroke}{rgb}{1.000000,1.000000,1.000000}%
\pgfsetstrokecolor{currentstroke}%
\pgfsetdash{}{0pt}%
\pgfpathmoveto{\pgfqpoint{3.084100in}{2.910420in}}%
\pgfpathcurveto{\pgfqpoint{3.095150in}{2.910420in}}{\pgfqpoint{3.105749in}{2.914810in}}{\pgfqpoint{3.113563in}{2.922623in}}%
\pgfpathcurveto{\pgfqpoint{3.121377in}{2.930437in}}{\pgfqpoint{3.125767in}{2.941036in}}{\pgfqpoint{3.125767in}{2.952086in}}%
\pgfpathcurveto{\pgfqpoint{3.125767in}{2.963136in}}{\pgfqpoint{3.121377in}{2.973735in}}{\pgfqpoint{3.113563in}{2.981549in}}%
\pgfpathcurveto{\pgfqpoint{3.105749in}{2.989363in}}{\pgfqpoint{3.095150in}{2.993753in}}{\pgfqpoint{3.084100in}{2.993753in}}%
\pgfpathcurveto{\pgfqpoint{3.073050in}{2.993753in}}{\pgfqpoint{3.062451in}{2.989363in}}{\pgfqpoint{3.054637in}{2.981549in}}%
\pgfpathcurveto{\pgfqpoint{3.046824in}{2.973735in}}{\pgfqpoint{3.042434in}{2.963136in}}{\pgfqpoint{3.042434in}{2.952086in}}%
\pgfpathcurveto{\pgfqpoint{3.042434in}{2.941036in}}{\pgfqpoint{3.046824in}{2.930437in}}{\pgfqpoint{3.054637in}{2.922623in}}%
\pgfpathcurveto{\pgfqpoint{3.062451in}{2.914810in}}{\pgfqpoint{3.073050in}{2.910420in}}{\pgfqpoint{3.084100in}{2.910420in}}%
\pgfpathclose%
\pgfusepath{stroke,fill}%
\end{pgfscope}%
\begin{pgfscope}%
\pgfpathrectangle{\pgfqpoint{0.481978in}{0.331635in}}{\pgfqpoint{9.300000in}{7.700000in}}%
\pgfusepath{clip}%
\pgfsetbuttcap%
\pgfsetroundjoin%
\definecolor{currentfill}{rgb}{1.000000,0.705882,0.509804}%
\pgfsetfillcolor{currentfill}%
\pgfsetlinewidth{0.481800pt}%
\definecolor{currentstroke}{rgb}{1.000000,1.000000,1.000000}%
\pgfsetstrokecolor{currentstroke}%
\pgfsetdash{}{0pt}%
\pgfpathmoveto{\pgfqpoint{5.368141in}{2.996376in}}%
\pgfpathcurveto{\pgfqpoint{5.379191in}{2.996376in}}{\pgfqpoint{5.389790in}{3.000766in}}{\pgfqpoint{5.397604in}{3.008580in}}%
\pgfpathcurveto{\pgfqpoint{5.405417in}{3.016394in}}{\pgfqpoint{5.409807in}{3.026993in}}{\pgfqpoint{5.409807in}{3.038043in}}%
\pgfpathcurveto{\pgfqpoint{5.409807in}{3.049093in}}{\pgfqpoint{5.405417in}{3.059692in}}{\pgfqpoint{5.397604in}{3.067506in}}%
\pgfpathcurveto{\pgfqpoint{5.389790in}{3.075319in}}{\pgfqpoint{5.379191in}{3.079709in}}{\pgfqpoint{5.368141in}{3.079709in}}%
\pgfpathcurveto{\pgfqpoint{5.357091in}{3.079709in}}{\pgfqpoint{5.346492in}{3.075319in}}{\pgfqpoint{5.338678in}{3.067506in}}%
\pgfpathcurveto{\pgfqpoint{5.330864in}{3.059692in}}{\pgfqpoint{5.326474in}{3.049093in}}{\pgfqpoint{5.326474in}{3.038043in}}%
\pgfpathcurveto{\pgfqpoint{5.326474in}{3.026993in}}{\pgfqpoint{5.330864in}{3.016394in}}{\pgfqpoint{5.338678in}{3.008580in}}%
\pgfpathcurveto{\pgfqpoint{5.346492in}{3.000766in}}{\pgfqpoint{5.357091in}{2.996376in}}{\pgfqpoint{5.368141in}{2.996376in}}%
\pgfpathclose%
\pgfusepath{stroke,fill}%
\end{pgfscope}%
\begin{pgfscope}%
\pgfpathrectangle{\pgfqpoint{0.481978in}{0.331635in}}{\pgfqpoint{9.300000in}{7.700000in}}%
\pgfusepath{clip}%
\pgfsetbuttcap%
\pgfsetroundjoin%
\definecolor{currentfill}{rgb}{1.000000,0.705882,0.509804}%
\pgfsetfillcolor{currentfill}%
\pgfsetlinewidth{0.481800pt}%
\definecolor{currentstroke}{rgb}{1.000000,1.000000,1.000000}%
\pgfsetstrokecolor{currentstroke}%
\pgfsetdash{}{0pt}%
\pgfpathmoveto{\pgfqpoint{2.886916in}{3.885707in}}%
\pgfpathcurveto{\pgfqpoint{2.897966in}{3.885707in}}{\pgfqpoint{2.908565in}{3.890097in}}{\pgfqpoint{2.916379in}{3.897911in}}%
\pgfpathcurveto{\pgfqpoint{2.924192in}{3.905724in}}{\pgfqpoint{2.928583in}{3.916323in}}{\pgfqpoint{2.928583in}{3.927373in}}%
\pgfpathcurveto{\pgfqpoint{2.928583in}{3.938424in}}{\pgfqpoint{2.924192in}{3.949023in}}{\pgfqpoint{2.916379in}{3.956836in}}%
\pgfpathcurveto{\pgfqpoint{2.908565in}{3.964650in}}{\pgfqpoint{2.897966in}{3.969040in}}{\pgfqpoint{2.886916in}{3.969040in}}%
\pgfpathcurveto{\pgfqpoint{2.875866in}{3.969040in}}{\pgfqpoint{2.865267in}{3.964650in}}{\pgfqpoint{2.857453in}{3.956836in}}%
\pgfpathcurveto{\pgfqpoint{2.849640in}{3.949023in}}{\pgfqpoint{2.845249in}{3.938424in}}{\pgfqpoint{2.845249in}{3.927373in}}%
\pgfpathcurveto{\pgfqpoint{2.845249in}{3.916323in}}{\pgfqpoint{2.849640in}{3.905724in}}{\pgfqpoint{2.857453in}{3.897911in}}%
\pgfpathcurveto{\pgfqpoint{2.865267in}{3.890097in}}{\pgfqpoint{2.875866in}{3.885707in}}{\pgfqpoint{2.886916in}{3.885707in}}%
\pgfpathclose%
\pgfusepath{stroke,fill}%
\end{pgfscope}%
\begin{pgfscope}%
\pgfpathrectangle{\pgfqpoint{0.481978in}{0.331635in}}{\pgfqpoint{9.300000in}{7.700000in}}%
\pgfusepath{clip}%
\pgfsetbuttcap%
\pgfsetroundjoin%
\definecolor{currentfill}{rgb}{1.000000,0.705882,0.509804}%
\pgfsetfillcolor{currentfill}%
\pgfsetlinewidth{0.481800pt}%
\definecolor{currentstroke}{rgb}{1.000000,1.000000,1.000000}%
\pgfsetstrokecolor{currentstroke}%
\pgfsetdash{}{0pt}%
\pgfpathmoveto{\pgfqpoint{4.450031in}{2.732253in}}%
\pgfpathcurveto{\pgfqpoint{4.461082in}{2.732253in}}{\pgfqpoint{4.471681in}{2.736643in}}{\pgfqpoint{4.479494in}{2.744457in}}%
\pgfpathcurveto{\pgfqpoint{4.487308in}{2.752270in}}{\pgfqpoint{4.491698in}{2.762869in}}{\pgfqpoint{4.491698in}{2.773919in}}%
\pgfpathcurveto{\pgfqpoint{4.491698in}{2.784970in}}{\pgfqpoint{4.487308in}{2.795569in}}{\pgfqpoint{4.479494in}{2.803382in}}%
\pgfpathcurveto{\pgfqpoint{4.471681in}{2.811196in}}{\pgfqpoint{4.461082in}{2.815586in}}{\pgfqpoint{4.450031in}{2.815586in}}%
\pgfpathcurveto{\pgfqpoint{4.438981in}{2.815586in}}{\pgfqpoint{4.428382in}{2.811196in}}{\pgfqpoint{4.420569in}{2.803382in}}%
\pgfpathcurveto{\pgfqpoint{4.412755in}{2.795569in}}{\pgfqpoint{4.408365in}{2.784970in}}{\pgfqpoint{4.408365in}{2.773919in}}%
\pgfpathcurveto{\pgfqpoint{4.408365in}{2.762869in}}{\pgfqpoint{4.412755in}{2.752270in}}{\pgfqpoint{4.420569in}{2.744457in}}%
\pgfpathcurveto{\pgfqpoint{4.428382in}{2.736643in}}{\pgfqpoint{4.438981in}{2.732253in}}{\pgfqpoint{4.450031in}{2.732253in}}%
\pgfpathclose%
\pgfusepath{stroke,fill}%
\end{pgfscope}%
\begin{pgfscope}%
\pgfpathrectangle{\pgfqpoint{0.481978in}{0.331635in}}{\pgfqpoint{9.300000in}{7.700000in}}%
\pgfusepath{clip}%
\pgfsetbuttcap%
\pgfsetroundjoin%
\definecolor{currentfill}{rgb}{1.000000,0.705882,0.509804}%
\pgfsetfillcolor{currentfill}%
\pgfsetlinewidth{0.481800pt}%
\definecolor{currentstroke}{rgb}{1.000000,1.000000,1.000000}%
\pgfsetstrokecolor{currentstroke}%
\pgfsetdash{}{0pt}%
\pgfpathmoveto{\pgfqpoint{2.491094in}{5.161273in}}%
\pgfpathcurveto{\pgfqpoint{2.502144in}{5.161273in}}{\pgfqpoint{2.512743in}{5.165663in}}{\pgfqpoint{2.520557in}{5.173476in}}%
\pgfpathcurveto{\pgfqpoint{2.528370in}{5.181290in}}{\pgfqpoint{2.532760in}{5.191889in}}{\pgfqpoint{2.532760in}{5.202939in}}%
\pgfpathcurveto{\pgfqpoint{2.532760in}{5.213989in}}{\pgfqpoint{2.528370in}{5.224588in}}{\pgfqpoint{2.520557in}{5.232402in}}%
\pgfpathcurveto{\pgfqpoint{2.512743in}{5.240216in}}{\pgfqpoint{2.502144in}{5.244606in}}{\pgfqpoint{2.491094in}{5.244606in}}%
\pgfpathcurveto{\pgfqpoint{2.480044in}{5.244606in}}{\pgfqpoint{2.469445in}{5.240216in}}{\pgfqpoint{2.461631in}{5.232402in}}%
\pgfpathcurveto{\pgfqpoint{2.453817in}{5.224588in}}{\pgfqpoint{2.449427in}{5.213989in}}{\pgfqpoint{2.449427in}{5.202939in}}%
\pgfpathcurveto{\pgfqpoint{2.449427in}{5.191889in}}{\pgfqpoint{2.453817in}{5.181290in}}{\pgfqpoint{2.461631in}{5.173476in}}%
\pgfpathcurveto{\pgfqpoint{2.469445in}{5.165663in}}{\pgfqpoint{2.480044in}{5.161273in}}{\pgfqpoint{2.491094in}{5.161273in}}%
\pgfpathclose%
\pgfusepath{stroke,fill}%
\end{pgfscope}%
\begin{pgfscope}%
\pgfpathrectangle{\pgfqpoint{0.481978in}{0.331635in}}{\pgfqpoint{9.300000in}{7.700000in}}%
\pgfusepath{clip}%
\pgfsetbuttcap%
\pgfsetroundjoin%
\definecolor{currentfill}{rgb}{1.000000,0.705882,0.509804}%
\pgfsetfillcolor{currentfill}%
\pgfsetlinewidth{0.481800pt}%
\definecolor{currentstroke}{rgb}{1.000000,1.000000,1.000000}%
\pgfsetstrokecolor{currentstroke}%
\pgfsetdash{}{0pt}%
\pgfpathmoveto{\pgfqpoint{2.662048in}{3.580893in}}%
\pgfpathcurveto{\pgfqpoint{2.673099in}{3.580893in}}{\pgfqpoint{2.683698in}{3.585283in}}{\pgfqpoint{2.691511in}{3.593097in}}%
\pgfpathcurveto{\pgfqpoint{2.699325in}{3.600911in}}{\pgfqpoint{2.703715in}{3.611510in}}{\pgfqpoint{2.703715in}{3.622560in}}%
\pgfpathcurveto{\pgfqpoint{2.703715in}{3.633610in}}{\pgfqpoint{2.699325in}{3.644209in}}{\pgfqpoint{2.691511in}{3.652023in}}%
\pgfpathcurveto{\pgfqpoint{2.683698in}{3.659836in}}{\pgfqpoint{2.673099in}{3.664226in}}{\pgfqpoint{2.662048in}{3.664226in}}%
\pgfpathcurveto{\pgfqpoint{2.650998in}{3.664226in}}{\pgfqpoint{2.640399in}{3.659836in}}{\pgfqpoint{2.632586in}{3.652023in}}%
\pgfpathcurveto{\pgfqpoint{2.624772in}{3.644209in}}{\pgfqpoint{2.620382in}{3.633610in}}{\pgfqpoint{2.620382in}{3.622560in}}%
\pgfpathcurveto{\pgfqpoint{2.620382in}{3.611510in}}{\pgfqpoint{2.624772in}{3.600911in}}{\pgfqpoint{2.632586in}{3.593097in}}%
\pgfpathcurveto{\pgfqpoint{2.640399in}{3.585283in}}{\pgfqpoint{2.650998in}{3.580893in}}{\pgfqpoint{2.662048in}{3.580893in}}%
\pgfpathclose%
\pgfusepath{stroke,fill}%
\end{pgfscope}%
\begin{pgfscope}%
\pgfpathrectangle{\pgfqpoint{0.481978in}{0.331635in}}{\pgfqpoint{9.300000in}{7.700000in}}%
\pgfusepath{clip}%
\pgfsetbuttcap%
\pgfsetroundjoin%
\definecolor{currentfill}{rgb}{1.000000,0.705882,0.509804}%
\pgfsetfillcolor{currentfill}%
\pgfsetlinewidth{0.481800pt}%
\definecolor{currentstroke}{rgb}{1.000000,1.000000,1.000000}%
\pgfsetstrokecolor{currentstroke}%
\pgfsetdash{}{0pt}%
\pgfpathmoveto{\pgfqpoint{4.025822in}{3.570760in}}%
\pgfpathcurveto{\pgfqpoint{4.036872in}{3.570760in}}{\pgfqpoint{4.047471in}{3.575151in}}{\pgfqpoint{4.055285in}{3.582964in}}%
\pgfpathcurveto{\pgfqpoint{4.063098in}{3.590778in}}{\pgfqpoint{4.067489in}{3.601377in}}{\pgfqpoint{4.067489in}{3.612427in}}%
\pgfpathcurveto{\pgfqpoint{4.067489in}{3.623477in}}{\pgfqpoint{4.063098in}{3.634076in}}{\pgfqpoint{4.055285in}{3.641890in}}%
\pgfpathcurveto{\pgfqpoint{4.047471in}{3.649703in}}{\pgfqpoint{4.036872in}{3.654094in}}{\pgfqpoint{4.025822in}{3.654094in}}%
\pgfpathcurveto{\pgfqpoint{4.014772in}{3.654094in}}{\pgfqpoint{4.004173in}{3.649703in}}{\pgfqpoint{3.996359in}{3.641890in}}%
\pgfpathcurveto{\pgfqpoint{3.988546in}{3.634076in}}{\pgfqpoint{3.984155in}{3.623477in}}{\pgfqpoint{3.984155in}{3.612427in}}%
\pgfpathcurveto{\pgfqpoint{3.984155in}{3.601377in}}{\pgfqpoint{3.988546in}{3.590778in}}{\pgfqpoint{3.996359in}{3.582964in}}%
\pgfpathcurveto{\pgfqpoint{4.004173in}{3.575151in}}{\pgfqpoint{4.014772in}{3.570760in}}{\pgfqpoint{4.025822in}{3.570760in}}%
\pgfpathclose%
\pgfusepath{stroke,fill}%
\end{pgfscope}%
\begin{pgfscope}%
\pgfpathrectangle{\pgfqpoint{0.481978in}{0.331635in}}{\pgfqpoint{9.300000in}{7.700000in}}%
\pgfusepath{clip}%
\pgfsetbuttcap%
\pgfsetroundjoin%
\definecolor{currentfill}{rgb}{1.000000,0.705882,0.509804}%
\pgfsetfillcolor{currentfill}%
\pgfsetlinewidth{0.481800pt}%
\definecolor{currentstroke}{rgb}{1.000000,1.000000,1.000000}%
\pgfsetstrokecolor{currentstroke}%
\pgfsetdash{}{0pt}%
\pgfpathmoveto{\pgfqpoint{3.490511in}{5.055258in}}%
\pgfpathcurveto{\pgfqpoint{3.501561in}{5.055258in}}{\pgfqpoint{3.512160in}{5.059649in}}{\pgfqpoint{3.519974in}{5.067462in}}%
\pgfpathcurveto{\pgfqpoint{3.527788in}{5.075276in}}{\pgfqpoint{3.532178in}{5.085875in}}{\pgfqpoint{3.532178in}{5.096925in}}%
\pgfpathcurveto{\pgfqpoint{3.532178in}{5.107975in}}{\pgfqpoint{3.527788in}{5.118574in}}{\pgfqpoint{3.519974in}{5.126388in}}%
\pgfpathcurveto{\pgfqpoint{3.512160in}{5.134202in}}{\pgfqpoint{3.501561in}{5.138592in}}{\pgfqpoint{3.490511in}{5.138592in}}%
\pgfpathcurveto{\pgfqpoint{3.479461in}{5.138592in}}{\pgfqpoint{3.468862in}{5.134202in}}{\pgfqpoint{3.461048in}{5.126388in}}%
\pgfpathcurveto{\pgfqpoint{3.453235in}{5.118574in}}{\pgfqpoint{3.448844in}{5.107975in}}{\pgfqpoint{3.448844in}{5.096925in}}%
\pgfpathcurveto{\pgfqpoint{3.448844in}{5.085875in}}{\pgfqpoint{3.453235in}{5.075276in}}{\pgfqpoint{3.461048in}{5.067462in}}%
\pgfpathcurveto{\pgfqpoint{3.468862in}{5.059649in}}{\pgfqpoint{3.479461in}{5.055258in}}{\pgfqpoint{3.490511in}{5.055258in}}%
\pgfpathclose%
\pgfusepath{stroke,fill}%
\end{pgfscope}%
\begin{pgfscope}%
\pgfpathrectangle{\pgfqpoint{0.481978in}{0.331635in}}{\pgfqpoint{9.300000in}{7.700000in}}%
\pgfusepath{clip}%
\pgfsetbuttcap%
\pgfsetroundjoin%
\definecolor{currentfill}{rgb}{1.000000,0.705882,0.509804}%
\pgfsetfillcolor{currentfill}%
\pgfsetlinewidth{0.481800pt}%
\definecolor{currentstroke}{rgb}{1.000000,1.000000,1.000000}%
\pgfsetstrokecolor{currentstroke}%
\pgfsetdash{}{0pt}%
\pgfpathmoveto{\pgfqpoint{3.394544in}{3.150995in}}%
\pgfpathcurveto{\pgfqpoint{3.405594in}{3.150995in}}{\pgfqpoint{3.416193in}{3.155386in}}{\pgfqpoint{3.424007in}{3.163199in}}%
\pgfpathcurveto{\pgfqpoint{3.431820in}{3.171013in}}{\pgfqpoint{3.436211in}{3.181612in}}{\pgfqpoint{3.436211in}{3.192662in}}%
\pgfpathcurveto{\pgfqpoint{3.436211in}{3.203712in}}{\pgfqpoint{3.431820in}{3.214311in}}{\pgfqpoint{3.424007in}{3.222125in}}%
\pgfpathcurveto{\pgfqpoint{3.416193in}{3.229939in}}{\pgfqpoint{3.405594in}{3.234329in}}{\pgfqpoint{3.394544in}{3.234329in}}%
\pgfpathcurveto{\pgfqpoint{3.383494in}{3.234329in}}{\pgfqpoint{3.372895in}{3.229939in}}{\pgfqpoint{3.365081in}{3.222125in}}%
\pgfpathcurveto{\pgfqpoint{3.357268in}{3.214311in}}{\pgfqpoint{3.352877in}{3.203712in}}{\pgfqpoint{3.352877in}{3.192662in}}%
\pgfpathcurveto{\pgfqpoint{3.352877in}{3.181612in}}{\pgfqpoint{3.357268in}{3.171013in}}{\pgfqpoint{3.365081in}{3.163199in}}%
\pgfpathcurveto{\pgfqpoint{3.372895in}{3.155386in}}{\pgfqpoint{3.383494in}{3.150995in}}{\pgfqpoint{3.394544in}{3.150995in}}%
\pgfpathclose%
\pgfusepath{stroke,fill}%
\end{pgfscope}%
\begin{pgfscope}%
\pgfpathrectangle{\pgfqpoint{0.481978in}{0.331635in}}{\pgfqpoint{9.300000in}{7.700000in}}%
\pgfusepath{clip}%
\pgfsetbuttcap%
\pgfsetroundjoin%
\definecolor{currentfill}{rgb}{1.000000,0.705882,0.509804}%
\pgfsetfillcolor{currentfill}%
\pgfsetlinewidth{0.481800pt}%
\definecolor{currentstroke}{rgb}{1.000000,1.000000,1.000000}%
\pgfsetstrokecolor{currentstroke}%
\pgfsetdash{}{0pt}%
\pgfpathmoveto{\pgfqpoint{3.295196in}{3.631675in}}%
\pgfpathcurveto{\pgfqpoint{3.306246in}{3.631675in}}{\pgfqpoint{3.316845in}{3.636065in}}{\pgfqpoint{3.324659in}{3.643879in}}%
\pgfpathcurveto{\pgfqpoint{3.332473in}{3.651692in}}{\pgfqpoint{3.336863in}{3.662291in}}{\pgfqpoint{3.336863in}{3.673341in}}%
\pgfpathcurveto{\pgfqpoint{3.336863in}{3.684391in}}{\pgfqpoint{3.332473in}{3.694991in}}{\pgfqpoint{3.324659in}{3.702804in}}%
\pgfpathcurveto{\pgfqpoint{3.316845in}{3.710618in}}{\pgfqpoint{3.306246in}{3.715008in}}{\pgfqpoint{3.295196in}{3.715008in}}%
\pgfpathcurveto{\pgfqpoint{3.284146in}{3.715008in}}{\pgfqpoint{3.273547in}{3.710618in}}{\pgfqpoint{3.265733in}{3.702804in}}%
\pgfpathcurveto{\pgfqpoint{3.257920in}{3.694991in}}{\pgfqpoint{3.253529in}{3.684391in}}{\pgfqpoint{3.253529in}{3.673341in}}%
\pgfpathcurveto{\pgfqpoint{3.253529in}{3.662291in}}{\pgfqpoint{3.257920in}{3.651692in}}{\pgfqpoint{3.265733in}{3.643879in}}%
\pgfpathcurveto{\pgfqpoint{3.273547in}{3.636065in}}{\pgfqpoint{3.284146in}{3.631675in}}{\pgfqpoint{3.295196in}{3.631675in}}%
\pgfpathclose%
\pgfusepath{stroke,fill}%
\end{pgfscope}%
\begin{pgfscope}%
\pgfpathrectangle{\pgfqpoint{0.481978in}{0.331635in}}{\pgfqpoint{9.300000in}{7.700000in}}%
\pgfusepath{clip}%
\pgfsetbuttcap%
\pgfsetroundjoin%
\definecolor{currentfill}{rgb}{1.000000,0.705882,0.509804}%
\pgfsetfillcolor{currentfill}%
\pgfsetlinewidth{0.481800pt}%
\definecolor{currentstroke}{rgb}{1.000000,1.000000,1.000000}%
\pgfsetstrokecolor{currentstroke}%
\pgfsetdash{}{0pt}%
\pgfpathmoveto{\pgfqpoint{3.905090in}{4.963082in}}%
\pgfpathcurveto{\pgfqpoint{3.916140in}{4.963082in}}{\pgfqpoint{3.926739in}{4.967472in}}{\pgfqpoint{3.934553in}{4.975286in}}%
\pgfpathcurveto{\pgfqpoint{3.942366in}{4.983099in}}{\pgfqpoint{3.946757in}{4.993698in}}{\pgfqpoint{3.946757in}{5.004749in}}%
\pgfpathcurveto{\pgfqpoint{3.946757in}{5.015799in}}{\pgfqpoint{3.942366in}{5.026398in}}{\pgfqpoint{3.934553in}{5.034211in}}%
\pgfpathcurveto{\pgfqpoint{3.926739in}{5.042025in}}{\pgfqpoint{3.916140in}{5.046415in}}{\pgfqpoint{3.905090in}{5.046415in}}%
\pgfpathcurveto{\pgfqpoint{3.894040in}{5.046415in}}{\pgfqpoint{3.883441in}{5.042025in}}{\pgfqpoint{3.875627in}{5.034211in}}%
\pgfpathcurveto{\pgfqpoint{3.867814in}{5.026398in}}{\pgfqpoint{3.863423in}{5.015799in}}{\pgfqpoint{3.863423in}{5.004749in}}%
\pgfpathcurveto{\pgfqpoint{3.863423in}{4.993698in}}{\pgfqpoint{3.867814in}{4.983099in}}{\pgfqpoint{3.875627in}{4.975286in}}%
\pgfpathcurveto{\pgfqpoint{3.883441in}{4.967472in}}{\pgfqpoint{3.894040in}{4.963082in}}{\pgfqpoint{3.905090in}{4.963082in}}%
\pgfpathclose%
\pgfusepath{stroke,fill}%
\end{pgfscope}%
\begin{pgfscope}%
\pgfpathrectangle{\pgfqpoint{0.481978in}{0.331635in}}{\pgfqpoint{9.300000in}{7.700000in}}%
\pgfusepath{clip}%
\pgfsetbuttcap%
\pgfsetroundjoin%
\definecolor{currentfill}{rgb}{1.000000,0.705882,0.509804}%
\pgfsetfillcolor{currentfill}%
\pgfsetlinewidth{0.481800pt}%
\definecolor{currentstroke}{rgb}{1.000000,1.000000,1.000000}%
\pgfsetstrokecolor{currentstroke}%
\pgfsetdash{}{0pt}%
\pgfpathmoveto{\pgfqpoint{4.067534in}{3.361895in}}%
\pgfpathcurveto{\pgfqpoint{4.078584in}{3.361895in}}{\pgfqpoint{4.089183in}{3.366285in}}{\pgfqpoint{4.096996in}{3.374098in}}%
\pgfpathcurveto{\pgfqpoint{4.104810in}{3.381912in}}{\pgfqpoint{4.109200in}{3.392511in}}{\pgfqpoint{4.109200in}{3.403561in}}%
\pgfpathcurveto{\pgfqpoint{4.109200in}{3.414611in}}{\pgfqpoint{4.104810in}{3.425210in}}{\pgfqpoint{4.096996in}{3.433024in}}%
\pgfpathcurveto{\pgfqpoint{4.089183in}{3.440838in}}{\pgfqpoint{4.078584in}{3.445228in}}{\pgfqpoint{4.067534in}{3.445228in}}%
\pgfpathcurveto{\pgfqpoint{4.056483in}{3.445228in}}{\pgfqpoint{4.045884in}{3.440838in}}{\pgfqpoint{4.038071in}{3.433024in}}%
\pgfpathcurveto{\pgfqpoint{4.030257in}{3.425210in}}{\pgfqpoint{4.025867in}{3.414611in}}{\pgfqpoint{4.025867in}{3.403561in}}%
\pgfpathcurveto{\pgfqpoint{4.025867in}{3.392511in}}{\pgfqpoint{4.030257in}{3.381912in}}{\pgfqpoint{4.038071in}{3.374098in}}%
\pgfpathcurveto{\pgfqpoint{4.045884in}{3.366285in}}{\pgfqpoint{4.056483in}{3.361895in}}{\pgfqpoint{4.067534in}{3.361895in}}%
\pgfpathclose%
\pgfusepath{stroke,fill}%
\end{pgfscope}%
\begin{pgfscope}%
\pgfpathrectangle{\pgfqpoint{0.481978in}{0.331635in}}{\pgfqpoint{9.300000in}{7.700000in}}%
\pgfusepath{clip}%
\pgfsetbuttcap%
\pgfsetroundjoin%
\definecolor{currentfill}{rgb}{1.000000,0.705882,0.509804}%
\pgfsetfillcolor{currentfill}%
\pgfsetlinewidth{0.481800pt}%
\definecolor{currentstroke}{rgb}{1.000000,1.000000,1.000000}%
\pgfsetstrokecolor{currentstroke}%
\pgfsetdash{}{0pt}%
\pgfpathmoveto{\pgfqpoint{1.713787in}{3.836311in}}%
\pgfpathcurveto{\pgfqpoint{1.724837in}{3.836311in}}{\pgfqpoint{1.735436in}{3.840701in}}{\pgfqpoint{1.743250in}{3.848515in}}%
\pgfpathcurveto{\pgfqpoint{1.751064in}{3.856329in}}{\pgfqpoint{1.755454in}{3.866928in}}{\pgfqpoint{1.755454in}{3.877978in}}%
\pgfpathcurveto{\pgfqpoint{1.755454in}{3.889028in}}{\pgfqpoint{1.751064in}{3.899627in}}{\pgfqpoint{1.743250in}{3.907440in}}%
\pgfpathcurveto{\pgfqpoint{1.735436in}{3.915254in}}{\pgfqpoint{1.724837in}{3.919644in}}{\pgfqpoint{1.713787in}{3.919644in}}%
\pgfpathcurveto{\pgfqpoint{1.702737in}{3.919644in}}{\pgfqpoint{1.692138in}{3.915254in}}{\pgfqpoint{1.684324in}{3.907440in}}%
\pgfpathcurveto{\pgfqpoint{1.676511in}{3.899627in}}{\pgfqpoint{1.672121in}{3.889028in}}{\pgfqpoint{1.672121in}{3.877978in}}%
\pgfpathcurveto{\pgfqpoint{1.672121in}{3.866928in}}{\pgfqpoint{1.676511in}{3.856329in}}{\pgfqpoint{1.684324in}{3.848515in}}%
\pgfpathcurveto{\pgfqpoint{1.692138in}{3.840701in}}{\pgfqpoint{1.702737in}{3.836311in}}{\pgfqpoint{1.713787in}{3.836311in}}%
\pgfpathclose%
\pgfusepath{stroke,fill}%
\end{pgfscope}%
\begin{pgfscope}%
\pgfpathrectangle{\pgfqpoint{0.481978in}{0.331635in}}{\pgfqpoint{9.300000in}{7.700000in}}%
\pgfusepath{clip}%
\pgfsetbuttcap%
\pgfsetroundjoin%
\definecolor{currentfill}{rgb}{1.000000,0.705882,0.509804}%
\pgfsetfillcolor{currentfill}%
\pgfsetlinewidth{0.481800pt}%
\definecolor{currentstroke}{rgb}{1.000000,1.000000,1.000000}%
\pgfsetstrokecolor{currentstroke}%
\pgfsetdash{}{0pt}%
\pgfpathmoveto{\pgfqpoint{2.059946in}{4.470058in}}%
\pgfpathcurveto{\pgfqpoint{2.070996in}{4.470058in}}{\pgfqpoint{2.081595in}{4.474448in}}{\pgfqpoint{2.089408in}{4.482262in}}%
\pgfpathcurveto{\pgfqpoint{2.097222in}{4.490075in}}{\pgfqpoint{2.101612in}{4.500674in}}{\pgfqpoint{2.101612in}{4.511724in}}%
\pgfpathcurveto{\pgfqpoint{2.101612in}{4.522775in}}{\pgfqpoint{2.097222in}{4.533374in}}{\pgfqpoint{2.089408in}{4.541187in}}%
\pgfpathcurveto{\pgfqpoint{2.081595in}{4.549001in}}{\pgfqpoint{2.070996in}{4.553391in}}{\pgfqpoint{2.059946in}{4.553391in}}%
\pgfpathcurveto{\pgfqpoint{2.048895in}{4.553391in}}{\pgfqpoint{2.038296in}{4.549001in}}{\pgfqpoint{2.030483in}{4.541187in}}%
\pgfpathcurveto{\pgfqpoint{2.022669in}{4.533374in}}{\pgfqpoint{2.018279in}{4.522775in}}{\pgfqpoint{2.018279in}{4.511724in}}%
\pgfpathcurveto{\pgfqpoint{2.018279in}{4.500674in}}{\pgfqpoint{2.022669in}{4.490075in}}{\pgfqpoint{2.030483in}{4.482262in}}%
\pgfpathcurveto{\pgfqpoint{2.038296in}{4.474448in}}{\pgfqpoint{2.048895in}{4.470058in}}{\pgfqpoint{2.059946in}{4.470058in}}%
\pgfpathclose%
\pgfusepath{stroke,fill}%
\end{pgfscope}%
\begin{pgfscope}%
\pgfpathrectangle{\pgfqpoint{0.481978in}{0.331635in}}{\pgfqpoint{9.300000in}{7.700000in}}%
\pgfusepath{clip}%
\pgfsetbuttcap%
\pgfsetroundjoin%
\definecolor{currentfill}{rgb}{1.000000,0.705882,0.509804}%
\pgfsetfillcolor{currentfill}%
\pgfsetlinewidth{0.481800pt}%
\definecolor{currentstroke}{rgb}{1.000000,1.000000,1.000000}%
\pgfsetstrokecolor{currentstroke}%
\pgfsetdash{}{0pt}%
\pgfpathmoveto{\pgfqpoint{3.097846in}{5.814328in}}%
\pgfpathcurveto{\pgfqpoint{3.108896in}{5.814328in}}{\pgfqpoint{3.119495in}{5.818719in}}{\pgfqpoint{3.127309in}{5.826532in}}%
\pgfpathcurveto{\pgfqpoint{3.135122in}{5.834346in}}{\pgfqpoint{3.139513in}{5.844945in}}{\pgfqpoint{3.139513in}{5.855995in}}%
\pgfpathcurveto{\pgfqpoint{3.139513in}{5.867045in}}{\pgfqpoint{3.135122in}{5.877644in}}{\pgfqpoint{3.127309in}{5.885458in}}%
\pgfpathcurveto{\pgfqpoint{3.119495in}{5.893271in}}{\pgfqpoint{3.108896in}{5.897662in}}{\pgfqpoint{3.097846in}{5.897662in}}%
\pgfpathcurveto{\pgfqpoint{3.086796in}{5.897662in}}{\pgfqpoint{3.076197in}{5.893271in}}{\pgfqpoint{3.068383in}{5.885458in}}%
\pgfpathcurveto{\pgfqpoint{3.060570in}{5.877644in}}{\pgfqpoint{3.056179in}{5.867045in}}{\pgfqpoint{3.056179in}{5.855995in}}%
\pgfpathcurveto{\pgfqpoint{3.056179in}{5.844945in}}{\pgfqpoint{3.060570in}{5.834346in}}{\pgfqpoint{3.068383in}{5.826532in}}%
\pgfpathcurveto{\pgfqpoint{3.076197in}{5.818719in}}{\pgfqpoint{3.086796in}{5.814328in}}{\pgfqpoint{3.097846in}{5.814328in}}%
\pgfpathclose%
\pgfusepath{stroke,fill}%
\end{pgfscope}%
\begin{pgfscope}%
\pgfpathrectangle{\pgfqpoint{0.481978in}{0.331635in}}{\pgfqpoint{9.300000in}{7.700000in}}%
\pgfusepath{clip}%
\pgfsetbuttcap%
\pgfsetroundjoin%
\definecolor{currentfill}{rgb}{1.000000,0.705882,0.509804}%
\pgfsetfillcolor{currentfill}%
\pgfsetlinewidth{0.481800pt}%
\definecolor{currentstroke}{rgb}{1.000000,1.000000,1.000000}%
\pgfsetstrokecolor{currentstroke}%
\pgfsetdash{}{0pt}%
\pgfpathmoveto{\pgfqpoint{2.416512in}{2.668881in}}%
\pgfpathcurveto{\pgfqpoint{2.427562in}{2.668881in}}{\pgfqpoint{2.438161in}{2.673271in}}{\pgfqpoint{2.445975in}{2.681085in}}%
\pgfpathcurveto{\pgfqpoint{2.453789in}{2.688899in}}{\pgfqpoint{2.458179in}{2.699498in}}{\pgfqpoint{2.458179in}{2.710548in}}%
\pgfpathcurveto{\pgfqpoint{2.458179in}{2.721598in}}{\pgfqpoint{2.453789in}{2.732197in}}{\pgfqpoint{2.445975in}{2.740011in}}%
\pgfpathcurveto{\pgfqpoint{2.438161in}{2.747824in}}{\pgfqpoint{2.427562in}{2.752214in}}{\pgfqpoint{2.416512in}{2.752214in}}%
\pgfpathcurveto{\pgfqpoint{2.405462in}{2.752214in}}{\pgfqpoint{2.394863in}{2.747824in}}{\pgfqpoint{2.387049in}{2.740011in}}%
\pgfpathcurveto{\pgfqpoint{2.379236in}{2.732197in}}{\pgfqpoint{2.374846in}{2.721598in}}{\pgfqpoint{2.374846in}{2.710548in}}%
\pgfpathcurveto{\pgfqpoint{2.374846in}{2.699498in}}{\pgfqpoint{2.379236in}{2.688899in}}{\pgfqpoint{2.387049in}{2.681085in}}%
\pgfpathcurveto{\pgfqpoint{2.394863in}{2.673271in}}{\pgfqpoint{2.405462in}{2.668881in}}{\pgfqpoint{2.416512in}{2.668881in}}%
\pgfpathclose%
\pgfusepath{stroke,fill}%
\end{pgfscope}%
\begin{pgfscope}%
\pgfpathrectangle{\pgfqpoint{0.481978in}{0.331635in}}{\pgfqpoint{9.300000in}{7.700000in}}%
\pgfusepath{clip}%
\pgfsetbuttcap%
\pgfsetroundjoin%
\definecolor{currentfill}{rgb}{1.000000,0.705882,0.509804}%
\pgfsetfillcolor{currentfill}%
\pgfsetlinewidth{0.481800pt}%
\definecolor{currentstroke}{rgb}{1.000000,1.000000,1.000000}%
\pgfsetstrokecolor{currentstroke}%
\pgfsetdash{}{0pt}%
\pgfpathmoveto{\pgfqpoint{1.548965in}{4.218503in}}%
\pgfpathcurveto{\pgfqpoint{1.560015in}{4.218503in}}{\pgfqpoint{1.570614in}{4.222894in}}{\pgfqpoint{1.578428in}{4.230707in}}%
\pgfpathcurveto{\pgfqpoint{1.586241in}{4.238521in}}{\pgfqpoint{1.590632in}{4.249120in}}{\pgfqpoint{1.590632in}{4.260170in}}%
\pgfpathcurveto{\pgfqpoint{1.590632in}{4.271220in}}{\pgfqpoint{1.586241in}{4.281819in}}{\pgfqpoint{1.578428in}{4.289633in}}%
\pgfpathcurveto{\pgfqpoint{1.570614in}{4.297447in}}{\pgfqpoint{1.560015in}{4.301837in}}{\pgfqpoint{1.548965in}{4.301837in}}%
\pgfpathcurveto{\pgfqpoint{1.537915in}{4.301837in}}{\pgfqpoint{1.527316in}{4.297447in}}{\pgfqpoint{1.519502in}{4.289633in}}%
\pgfpathcurveto{\pgfqpoint{1.511689in}{4.281819in}}{\pgfqpoint{1.507298in}{4.271220in}}{\pgfqpoint{1.507298in}{4.260170in}}%
\pgfpathcurveto{\pgfqpoint{1.507298in}{4.249120in}}{\pgfqpoint{1.511689in}{4.238521in}}{\pgfqpoint{1.519502in}{4.230707in}}%
\pgfpathcurveto{\pgfqpoint{1.527316in}{4.222894in}}{\pgfqpoint{1.537915in}{4.218503in}}{\pgfqpoint{1.548965in}{4.218503in}}%
\pgfpathclose%
\pgfusepath{stroke,fill}%
\end{pgfscope}%
\begin{pgfscope}%
\pgfpathrectangle{\pgfqpoint{0.481978in}{0.331635in}}{\pgfqpoint{9.300000in}{7.700000in}}%
\pgfusepath{clip}%
\pgfsetbuttcap%
\pgfsetroundjoin%
\definecolor{currentfill}{rgb}{1.000000,0.705882,0.509804}%
\pgfsetfillcolor{currentfill}%
\pgfsetlinewidth{0.481800pt}%
\definecolor{currentstroke}{rgb}{1.000000,1.000000,1.000000}%
\pgfsetstrokecolor{currentstroke}%
\pgfsetdash{}{0pt}%
\pgfpathmoveto{\pgfqpoint{4.223289in}{4.599896in}}%
\pgfpathcurveto{\pgfqpoint{4.234340in}{4.599896in}}{\pgfqpoint{4.244939in}{4.604286in}}{\pgfqpoint{4.252752in}{4.612099in}}%
\pgfpathcurveto{\pgfqpoint{4.260566in}{4.619913in}}{\pgfqpoint{4.264956in}{4.630512in}}{\pgfqpoint{4.264956in}{4.641562in}}%
\pgfpathcurveto{\pgfqpoint{4.264956in}{4.652612in}}{\pgfqpoint{4.260566in}{4.663211in}}{\pgfqpoint{4.252752in}{4.671025in}}%
\pgfpathcurveto{\pgfqpoint{4.244939in}{4.678839in}}{\pgfqpoint{4.234340in}{4.683229in}}{\pgfqpoint{4.223289in}{4.683229in}}%
\pgfpathcurveto{\pgfqpoint{4.212239in}{4.683229in}}{\pgfqpoint{4.201640in}{4.678839in}}{\pgfqpoint{4.193827in}{4.671025in}}%
\pgfpathcurveto{\pgfqpoint{4.186013in}{4.663211in}}{\pgfqpoint{4.181623in}{4.652612in}}{\pgfqpoint{4.181623in}{4.641562in}}%
\pgfpathcurveto{\pgfqpoint{4.181623in}{4.630512in}}{\pgfqpoint{4.186013in}{4.619913in}}{\pgfqpoint{4.193827in}{4.612099in}}%
\pgfpathcurveto{\pgfqpoint{4.201640in}{4.604286in}}{\pgfqpoint{4.212239in}{4.599896in}}{\pgfqpoint{4.223289in}{4.599896in}}%
\pgfpathclose%
\pgfusepath{stroke,fill}%
\end{pgfscope}%
\begin{pgfscope}%
\pgfpathrectangle{\pgfqpoint{0.481978in}{0.331635in}}{\pgfqpoint{9.300000in}{7.700000in}}%
\pgfusepath{clip}%
\pgfsetbuttcap%
\pgfsetroundjoin%
\definecolor{currentfill}{rgb}{1.000000,0.705882,0.509804}%
\pgfsetfillcolor{currentfill}%
\pgfsetlinewidth{0.481800pt}%
\definecolor{currentstroke}{rgb}{1.000000,1.000000,1.000000}%
\pgfsetstrokecolor{currentstroke}%
\pgfsetdash{}{0pt}%
\pgfpathmoveto{\pgfqpoint{3.078663in}{3.560810in}}%
\pgfpathcurveto{\pgfqpoint{3.089713in}{3.560810in}}{\pgfqpoint{3.100312in}{3.565200in}}{\pgfqpoint{3.108126in}{3.573014in}}%
\pgfpathcurveto{\pgfqpoint{3.115940in}{3.580828in}}{\pgfqpoint{3.120330in}{3.591427in}}{\pgfqpoint{3.120330in}{3.602477in}}%
\pgfpathcurveto{\pgfqpoint{3.120330in}{3.613527in}}{\pgfqpoint{3.115940in}{3.624126in}}{\pgfqpoint{3.108126in}{3.631940in}}%
\pgfpathcurveto{\pgfqpoint{3.100312in}{3.639753in}}{\pgfqpoint{3.089713in}{3.644144in}}{\pgfqpoint{3.078663in}{3.644144in}}%
\pgfpathcurveto{\pgfqpoint{3.067613in}{3.644144in}}{\pgfqpoint{3.057014in}{3.639753in}}{\pgfqpoint{3.049200in}{3.631940in}}%
\pgfpathcurveto{\pgfqpoint{3.041387in}{3.624126in}}{\pgfqpoint{3.036997in}{3.613527in}}{\pgfqpoint{3.036997in}{3.602477in}}%
\pgfpathcurveto{\pgfqpoint{3.036997in}{3.591427in}}{\pgfqpoint{3.041387in}{3.580828in}}{\pgfqpoint{3.049200in}{3.573014in}}%
\pgfpathcurveto{\pgfqpoint{3.057014in}{3.565200in}}{\pgfqpoint{3.067613in}{3.560810in}}{\pgfqpoint{3.078663in}{3.560810in}}%
\pgfpathclose%
\pgfusepath{stroke,fill}%
\end{pgfscope}%
\begin{pgfscope}%
\pgfpathrectangle{\pgfqpoint{0.481978in}{0.331635in}}{\pgfqpoint{9.300000in}{7.700000in}}%
\pgfusepath{clip}%
\pgfsetbuttcap%
\pgfsetroundjoin%
\definecolor{currentfill}{rgb}{1.000000,0.705882,0.509804}%
\pgfsetfillcolor{currentfill}%
\pgfsetlinewidth{0.481800pt}%
\definecolor{currentstroke}{rgb}{1.000000,1.000000,1.000000}%
\pgfsetstrokecolor{currentstroke}%
\pgfsetdash{}{0pt}%
\pgfpathmoveto{\pgfqpoint{4.163791in}{4.017675in}}%
\pgfpathcurveto{\pgfqpoint{4.174841in}{4.017675in}}{\pgfqpoint{4.185441in}{4.022066in}}{\pgfqpoint{4.193254in}{4.029879in}}%
\pgfpathcurveto{\pgfqpoint{4.201068in}{4.037693in}}{\pgfqpoint{4.205458in}{4.048292in}}{\pgfqpoint{4.205458in}{4.059342in}}%
\pgfpathcurveto{\pgfqpoint{4.205458in}{4.070392in}}{\pgfqpoint{4.201068in}{4.080991in}}{\pgfqpoint{4.193254in}{4.088805in}}%
\pgfpathcurveto{\pgfqpoint{4.185441in}{4.096618in}}{\pgfqpoint{4.174841in}{4.101009in}}{\pgfqpoint{4.163791in}{4.101009in}}%
\pgfpathcurveto{\pgfqpoint{4.152741in}{4.101009in}}{\pgfqpoint{4.142142in}{4.096618in}}{\pgfqpoint{4.134329in}{4.088805in}}%
\pgfpathcurveto{\pgfqpoint{4.126515in}{4.080991in}}{\pgfqpoint{4.122125in}{4.070392in}}{\pgfqpoint{4.122125in}{4.059342in}}%
\pgfpathcurveto{\pgfqpoint{4.122125in}{4.048292in}}{\pgfqpoint{4.126515in}{4.037693in}}{\pgfqpoint{4.134329in}{4.029879in}}%
\pgfpathcurveto{\pgfqpoint{4.142142in}{4.022066in}}{\pgfqpoint{4.152741in}{4.017675in}}{\pgfqpoint{4.163791in}{4.017675in}}%
\pgfpathclose%
\pgfusepath{stroke,fill}%
\end{pgfscope}%
\begin{pgfscope}%
\pgfpathrectangle{\pgfqpoint{0.481978in}{0.331635in}}{\pgfqpoint{9.300000in}{7.700000in}}%
\pgfusepath{clip}%
\pgfsetbuttcap%
\pgfsetroundjoin%
\definecolor{currentfill}{rgb}{1.000000,0.705882,0.509804}%
\pgfsetfillcolor{currentfill}%
\pgfsetlinewidth{0.481800pt}%
\definecolor{currentstroke}{rgb}{1.000000,1.000000,1.000000}%
\pgfsetstrokecolor{currentstroke}%
\pgfsetdash{}{0pt}%
\pgfpathmoveto{\pgfqpoint{1.998764in}{4.788927in}}%
\pgfpathcurveto{\pgfqpoint{2.009814in}{4.788927in}}{\pgfqpoint{2.020413in}{4.793317in}}{\pgfqpoint{2.028227in}{4.801131in}}%
\pgfpathcurveto{\pgfqpoint{2.036040in}{4.808945in}}{\pgfqpoint{2.040431in}{4.819544in}}{\pgfqpoint{2.040431in}{4.830594in}}%
\pgfpathcurveto{\pgfqpoint{2.040431in}{4.841644in}}{\pgfqpoint{2.036040in}{4.852243in}}{\pgfqpoint{2.028227in}{4.860057in}}%
\pgfpathcurveto{\pgfqpoint{2.020413in}{4.867870in}}{\pgfqpoint{2.009814in}{4.872261in}}{\pgfqpoint{1.998764in}{4.872261in}}%
\pgfpathcurveto{\pgfqpoint{1.987714in}{4.872261in}}{\pgfqpoint{1.977115in}{4.867870in}}{\pgfqpoint{1.969301in}{4.860057in}}%
\pgfpathcurveto{\pgfqpoint{1.961487in}{4.852243in}}{\pgfqpoint{1.957097in}{4.841644in}}{\pgfqpoint{1.957097in}{4.830594in}}%
\pgfpathcurveto{\pgfqpoint{1.957097in}{4.819544in}}{\pgfqpoint{1.961487in}{4.808945in}}{\pgfqpoint{1.969301in}{4.801131in}}%
\pgfpathcurveto{\pgfqpoint{1.977115in}{4.793317in}}{\pgfqpoint{1.987714in}{4.788927in}}{\pgfqpoint{1.998764in}{4.788927in}}%
\pgfpathclose%
\pgfusepath{stroke,fill}%
\end{pgfscope}%
\begin{pgfscope}%
\pgfpathrectangle{\pgfqpoint{0.481978in}{0.331635in}}{\pgfqpoint{9.300000in}{7.700000in}}%
\pgfusepath{clip}%
\pgfsetbuttcap%
\pgfsetroundjoin%
\definecolor{currentfill}{rgb}{1.000000,0.705882,0.509804}%
\pgfsetfillcolor{currentfill}%
\pgfsetlinewidth{0.481800pt}%
\definecolor{currentstroke}{rgb}{1.000000,1.000000,1.000000}%
\pgfsetstrokecolor{currentstroke}%
\pgfsetdash{}{0pt}%
\pgfpathmoveto{\pgfqpoint{4.371217in}{3.333039in}}%
\pgfpathcurveto{\pgfqpoint{4.382267in}{3.333039in}}{\pgfqpoint{4.392866in}{3.337430in}}{\pgfqpoint{4.400680in}{3.345243in}}%
\pgfpathcurveto{\pgfqpoint{4.408494in}{3.353057in}}{\pgfqpoint{4.412884in}{3.363656in}}{\pgfqpoint{4.412884in}{3.374706in}}%
\pgfpathcurveto{\pgfqpoint{4.412884in}{3.385756in}}{\pgfqpoint{4.408494in}{3.396355in}}{\pgfqpoint{4.400680in}{3.404169in}}%
\pgfpathcurveto{\pgfqpoint{4.392866in}{3.411982in}}{\pgfqpoint{4.382267in}{3.416373in}}{\pgfqpoint{4.371217in}{3.416373in}}%
\pgfpathcurveto{\pgfqpoint{4.360167in}{3.416373in}}{\pgfqpoint{4.349568in}{3.411982in}}{\pgfqpoint{4.341755in}{3.404169in}}%
\pgfpathcurveto{\pgfqpoint{4.333941in}{3.396355in}}{\pgfqpoint{4.329551in}{3.385756in}}{\pgfqpoint{4.329551in}{3.374706in}}%
\pgfpathcurveto{\pgfqpoint{4.329551in}{3.363656in}}{\pgfqpoint{4.333941in}{3.353057in}}{\pgfqpoint{4.341755in}{3.345243in}}%
\pgfpathcurveto{\pgfqpoint{4.349568in}{3.337430in}}{\pgfqpoint{4.360167in}{3.333039in}}{\pgfqpoint{4.371217in}{3.333039in}}%
\pgfpathclose%
\pgfusepath{stroke,fill}%
\end{pgfscope}%
\begin{pgfscope}%
\pgfpathrectangle{\pgfqpoint{0.481978in}{0.331635in}}{\pgfqpoint{9.300000in}{7.700000in}}%
\pgfusepath{clip}%
\pgfsetbuttcap%
\pgfsetroundjoin%
\definecolor{currentfill}{rgb}{1.000000,0.705882,0.509804}%
\pgfsetfillcolor{currentfill}%
\pgfsetlinewidth{0.481800pt}%
\definecolor{currentstroke}{rgb}{1.000000,1.000000,1.000000}%
\pgfsetstrokecolor{currentstroke}%
\pgfsetdash{}{0pt}%
\pgfpathmoveto{\pgfqpoint{2.452590in}{4.204860in}}%
\pgfpathcurveto{\pgfqpoint{2.463640in}{4.204860in}}{\pgfqpoint{2.474239in}{4.209250in}}{\pgfqpoint{2.482053in}{4.217064in}}%
\pgfpathcurveto{\pgfqpoint{2.489866in}{4.224877in}}{\pgfqpoint{2.494257in}{4.235476in}}{\pgfqpoint{2.494257in}{4.246526in}}%
\pgfpathcurveto{\pgfqpoint{2.494257in}{4.257577in}}{\pgfqpoint{2.489866in}{4.268176in}}{\pgfqpoint{2.482053in}{4.275989in}}%
\pgfpathcurveto{\pgfqpoint{2.474239in}{4.283803in}}{\pgfqpoint{2.463640in}{4.288193in}}{\pgfqpoint{2.452590in}{4.288193in}}%
\pgfpathcurveto{\pgfqpoint{2.441540in}{4.288193in}}{\pgfqpoint{2.430941in}{4.283803in}}{\pgfqpoint{2.423127in}{4.275989in}}%
\pgfpathcurveto{\pgfqpoint{2.415314in}{4.268176in}}{\pgfqpoint{2.410923in}{4.257577in}}{\pgfqpoint{2.410923in}{4.246526in}}%
\pgfpathcurveto{\pgfqpoint{2.410923in}{4.235476in}}{\pgfqpoint{2.415314in}{4.224877in}}{\pgfqpoint{2.423127in}{4.217064in}}%
\pgfpathcurveto{\pgfqpoint{2.430941in}{4.209250in}}{\pgfqpoint{2.441540in}{4.204860in}}{\pgfqpoint{2.452590in}{4.204860in}}%
\pgfpathclose%
\pgfusepath{stroke,fill}%
\end{pgfscope}%
\begin{pgfscope}%
\pgfpathrectangle{\pgfqpoint{0.481978in}{0.331635in}}{\pgfqpoint{9.300000in}{7.700000in}}%
\pgfusepath{clip}%
\pgfsetbuttcap%
\pgfsetroundjoin%
\definecolor{currentfill}{rgb}{1.000000,0.705882,0.509804}%
\pgfsetfillcolor{currentfill}%
\pgfsetlinewidth{0.481800pt}%
\definecolor{currentstroke}{rgb}{1.000000,1.000000,1.000000}%
\pgfsetstrokecolor{currentstroke}%
\pgfsetdash{}{0pt}%
\pgfpathmoveto{\pgfqpoint{4.281738in}{4.785661in}}%
\pgfpathcurveto{\pgfqpoint{4.292788in}{4.785661in}}{\pgfqpoint{4.303387in}{4.790051in}}{\pgfqpoint{4.311201in}{4.797865in}}%
\pgfpathcurveto{\pgfqpoint{4.319014in}{4.805679in}}{\pgfqpoint{4.323405in}{4.816278in}}{\pgfqpoint{4.323405in}{4.827328in}}%
\pgfpathcurveto{\pgfqpoint{4.323405in}{4.838378in}}{\pgfqpoint{4.319014in}{4.848977in}}{\pgfqpoint{4.311201in}{4.856790in}}%
\pgfpathcurveto{\pgfqpoint{4.303387in}{4.864604in}}{\pgfqpoint{4.292788in}{4.868994in}}{\pgfqpoint{4.281738in}{4.868994in}}%
\pgfpathcurveto{\pgfqpoint{4.270688in}{4.868994in}}{\pgfqpoint{4.260089in}{4.864604in}}{\pgfqpoint{4.252275in}{4.856790in}}%
\pgfpathcurveto{\pgfqpoint{4.244462in}{4.848977in}}{\pgfqpoint{4.240071in}{4.838378in}}{\pgfqpoint{4.240071in}{4.827328in}}%
\pgfpathcurveto{\pgfqpoint{4.240071in}{4.816278in}}{\pgfqpoint{4.244462in}{4.805679in}}{\pgfqpoint{4.252275in}{4.797865in}}%
\pgfpathcurveto{\pgfqpoint{4.260089in}{4.790051in}}{\pgfqpoint{4.270688in}{4.785661in}}{\pgfqpoint{4.281738in}{4.785661in}}%
\pgfpathclose%
\pgfusepath{stroke,fill}%
\end{pgfscope}%
\begin{pgfscope}%
\pgfpathrectangle{\pgfqpoint{0.481978in}{0.331635in}}{\pgfqpoint{9.300000in}{7.700000in}}%
\pgfusepath{clip}%
\pgfsetbuttcap%
\pgfsetroundjoin%
\definecolor{currentfill}{rgb}{1.000000,0.705882,0.509804}%
\pgfsetfillcolor{currentfill}%
\pgfsetlinewidth{0.481800pt}%
\definecolor{currentstroke}{rgb}{1.000000,1.000000,1.000000}%
\pgfsetstrokecolor{currentstroke}%
\pgfsetdash{}{0pt}%
\pgfpathmoveto{\pgfqpoint{3.234311in}{3.396812in}}%
\pgfpathcurveto{\pgfqpoint{3.245361in}{3.396812in}}{\pgfqpoint{3.255960in}{3.401202in}}{\pgfqpoint{3.263774in}{3.409016in}}%
\pgfpathcurveto{\pgfqpoint{3.271587in}{3.416830in}}{\pgfqpoint{3.275977in}{3.427429in}}{\pgfqpoint{3.275977in}{3.438479in}}%
\pgfpathcurveto{\pgfqpoint{3.275977in}{3.449529in}}{\pgfqpoint{3.271587in}{3.460128in}}{\pgfqpoint{3.263774in}{3.467941in}}%
\pgfpathcurveto{\pgfqpoint{3.255960in}{3.475755in}}{\pgfqpoint{3.245361in}{3.480145in}}{\pgfqpoint{3.234311in}{3.480145in}}%
\pgfpathcurveto{\pgfqpoint{3.223261in}{3.480145in}}{\pgfqpoint{3.212662in}{3.475755in}}{\pgfqpoint{3.204848in}{3.467941in}}%
\pgfpathcurveto{\pgfqpoint{3.197034in}{3.460128in}}{\pgfqpoint{3.192644in}{3.449529in}}{\pgfqpoint{3.192644in}{3.438479in}}%
\pgfpathcurveto{\pgfqpoint{3.192644in}{3.427429in}}{\pgfqpoint{3.197034in}{3.416830in}}{\pgfqpoint{3.204848in}{3.409016in}}%
\pgfpathcurveto{\pgfqpoint{3.212662in}{3.401202in}}{\pgfqpoint{3.223261in}{3.396812in}}{\pgfqpoint{3.234311in}{3.396812in}}%
\pgfpathclose%
\pgfusepath{stroke,fill}%
\end{pgfscope}%
\begin{pgfscope}%
\pgfpathrectangle{\pgfqpoint{0.481978in}{0.331635in}}{\pgfqpoint{9.300000in}{7.700000in}}%
\pgfusepath{clip}%
\pgfsetbuttcap%
\pgfsetroundjoin%
\definecolor{currentfill}{rgb}{1.000000,0.705882,0.509804}%
\pgfsetfillcolor{currentfill}%
\pgfsetlinewidth{0.481800pt}%
\definecolor{currentstroke}{rgb}{1.000000,1.000000,1.000000}%
\pgfsetstrokecolor{currentstroke}%
\pgfsetdash{}{0pt}%
\pgfpathmoveto{\pgfqpoint{3.915243in}{6.654693in}}%
\pgfpathcurveto{\pgfqpoint{3.926293in}{6.654693in}}{\pgfqpoint{3.936892in}{6.659084in}}{\pgfqpoint{3.944705in}{6.666897in}}%
\pgfpathcurveto{\pgfqpoint{3.952519in}{6.674711in}}{\pgfqpoint{3.956909in}{6.685310in}}{\pgfqpoint{3.956909in}{6.696360in}}%
\pgfpathcurveto{\pgfqpoint{3.956909in}{6.707410in}}{\pgfqpoint{3.952519in}{6.718009in}}{\pgfqpoint{3.944705in}{6.725823in}}%
\pgfpathcurveto{\pgfqpoint{3.936892in}{6.733637in}}{\pgfqpoint{3.926293in}{6.738027in}}{\pgfqpoint{3.915243in}{6.738027in}}%
\pgfpathcurveto{\pgfqpoint{3.904192in}{6.738027in}}{\pgfqpoint{3.893593in}{6.733637in}}{\pgfqpoint{3.885780in}{6.725823in}}%
\pgfpathcurveto{\pgfqpoint{3.877966in}{6.718009in}}{\pgfqpoint{3.873576in}{6.707410in}}{\pgfqpoint{3.873576in}{6.696360in}}%
\pgfpathcurveto{\pgfqpoint{3.873576in}{6.685310in}}{\pgfqpoint{3.877966in}{6.674711in}}{\pgfqpoint{3.885780in}{6.666897in}}%
\pgfpathcurveto{\pgfqpoint{3.893593in}{6.659084in}}{\pgfqpoint{3.904192in}{6.654693in}}{\pgfqpoint{3.915243in}{6.654693in}}%
\pgfpathclose%
\pgfusepath{stroke,fill}%
\end{pgfscope}%
\begin{pgfscope}%
\pgfpathrectangle{\pgfqpoint{0.481978in}{0.331635in}}{\pgfqpoint{9.300000in}{7.700000in}}%
\pgfusepath{clip}%
\pgfsetbuttcap%
\pgfsetroundjoin%
\definecolor{currentfill}{rgb}{1.000000,0.705882,0.509804}%
\pgfsetfillcolor{currentfill}%
\pgfsetlinewidth{0.481800pt}%
\definecolor{currentstroke}{rgb}{1.000000,1.000000,1.000000}%
\pgfsetstrokecolor{currentstroke}%
\pgfsetdash{}{0pt}%
\pgfpathmoveto{\pgfqpoint{4.358866in}{3.143475in}}%
\pgfpathcurveto{\pgfqpoint{4.369916in}{3.143475in}}{\pgfqpoint{4.380515in}{3.147866in}}{\pgfqpoint{4.388328in}{3.155679in}}%
\pgfpathcurveto{\pgfqpoint{4.396142in}{3.163493in}}{\pgfqpoint{4.400532in}{3.174092in}}{\pgfqpoint{4.400532in}{3.185142in}}%
\pgfpathcurveto{\pgfqpoint{4.400532in}{3.196192in}}{\pgfqpoint{4.396142in}{3.206791in}}{\pgfqpoint{4.388328in}{3.214605in}}%
\pgfpathcurveto{\pgfqpoint{4.380515in}{3.222418in}}{\pgfqpoint{4.369916in}{3.226809in}}{\pgfqpoint{4.358866in}{3.226809in}}%
\pgfpathcurveto{\pgfqpoint{4.347816in}{3.226809in}}{\pgfqpoint{4.337217in}{3.222418in}}{\pgfqpoint{4.329403in}{3.214605in}}%
\pgfpathcurveto{\pgfqpoint{4.321589in}{3.206791in}}{\pgfqpoint{4.317199in}{3.196192in}}{\pgfqpoint{4.317199in}{3.185142in}}%
\pgfpathcurveto{\pgfqpoint{4.317199in}{3.174092in}}{\pgfqpoint{4.321589in}{3.163493in}}{\pgfqpoint{4.329403in}{3.155679in}}%
\pgfpathcurveto{\pgfqpoint{4.337217in}{3.147866in}}{\pgfqpoint{4.347816in}{3.143475in}}{\pgfqpoint{4.358866in}{3.143475in}}%
\pgfpathclose%
\pgfusepath{stroke,fill}%
\end{pgfscope}%
\begin{pgfscope}%
\pgfpathrectangle{\pgfqpoint{0.481978in}{0.331635in}}{\pgfqpoint{9.300000in}{7.700000in}}%
\pgfusepath{clip}%
\pgfsetbuttcap%
\pgfsetroundjoin%
\definecolor{currentfill}{rgb}{1.000000,0.705882,0.509804}%
\pgfsetfillcolor{currentfill}%
\pgfsetlinewidth{0.481800pt}%
\definecolor{currentstroke}{rgb}{1.000000,1.000000,1.000000}%
\pgfsetstrokecolor{currentstroke}%
\pgfsetdash{}{0pt}%
\pgfpathmoveto{\pgfqpoint{4.512704in}{5.547197in}}%
\pgfpathcurveto{\pgfqpoint{4.523754in}{5.547197in}}{\pgfqpoint{4.534353in}{5.551587in}}{\pgfqpoint{4.542167in}{5.559401in}}%
\pgfpathcurveto{\pgfqpoint{4.549981in}{5.567214in}}{\pgfqpoint{4.554371in}{5.577813in}}{\pgfqpoint{4.554371in}{5.588863in}}%
\pgfpathcurveto{\pgfqpoint{4.554371in}{5.599914in}}{\pgfqpoint{4.549981in}{5.610513in}}{\pgfqpoint{4.542167in}{5.618326in}}%
\pgfpathcurveto{\pgfqpoint{4.534353in}{5.626140in}}{\pgfqpoint{4.523754in}{5.630530in}}{\pgfqpoint{4.512704in}{5.630530in}}%
\pgfpathcurveto{\pgfqpoint{4.501654in}{5.630530in}}{\pgfqpoint{4.491055in}{5.626140in}}{\pgfqpoint{4.483242in}{5.618326in}}%
\pgfpathcurveto{\pgfqpoint{4.475428in}{5.610513in}}{\pgfqpoint{4.471038in}{5.599914in}}{\pgfqpoint{4.471038in}{5.588863in}}%
\pgfpathcurveto{\pgfqpoint{4.471038in}{5.577813in}}{\pgfqpoint{4.475428in}{5.567214in}}{\pgfqpoint{4.483242in}{5.559401in}}%
\pgfpathcurveto{\pgfqpoint{4.491055in}{5.551587in}}{\pgfqpoint{4.501654in}{5.547197in}}{\pgfqpoint{4.512704in}{5.547197in}}%
\pgfpathclose%
\pgfusepath{stroke,fill}%
\end{pgfscope}%
\begin{pgfscope}%
\pgfpathrectangle{\pgfqpoint{0.481978in}{0.331635in}}{\pgfqpoint{9.300000in}{7.700000in}}%
\pgfusepath{clip}%
\pgfsetbuttcap%
\pgfsetroundjoin%
\definecolor{currentfill}{rgb}{1.000000,0.705882,0.509804}%
\pgfsetfillcolor{currentfill}%
\pgfsetlinewidth{0.481800pt}%
\definecolor{currentstroke}{rgb}{1.000000,1.000000,1.000000}%
\pgfsetstrokecolor{currentstroke}%
\pgfsetdash{}{0pt}%
\pgfpathmoveto{\pgfqpoint{2.811663in}{4.114795in}}%
\pgfpathcurveto{\pgfqpoint{2.822714in}{4.114795in}}{\pgfqpoint{2.833313in}{4.119185in}}{\pgfqpoint{2.841126in}{4.126999in}}%
\pgfpathcurveto{\pgfqpoint{2.848940in}{4.134813in}}{\pgfqpoint{2.853330in}{4.145412in}}{\pgfqpoint{2.853330in}{4.156462in}}%
\pgfpathcurveto{\pgfqpoint{2.853330in}{4.167512in}}{\pgfqpoint{2.848940in}{4.178111in}}{\pgfqpoint{2.841126in}{4.185925in}}%
\pgfpathcurveto{\pgfqpoint{2.833313in}{4.193738in}}{\pgfqpoint{2.822714in}{4.198128in}}{\pgfqpoint{2.811663in}{4.198128in}}%
\pgfpathcurveto{\pgfqpoint{2.800613in}{4.198128in}}{\pgfqpoint{2.790014in}{4.193738in}}{\pgfqpoint{2.782201in}{4.185925in}}%
\pgfpathcurveto{\pgfqpoint{2.774387in}{4.178111in}}{\pgfqpoint{2.769997in}{4.167512in}}{\pgfqpoint{2.769997in}{4.156462in}}%
\pgfpathcurveto{\pgfqpoint{2.769997in}{4.145412in}}{\pgfqpoint{2.774387in}{4.134813in}}{\pgfqpoint{2.782201in}{4.126999in}}%
\pgfpathcurveto{\pgfqpoint{2.790014in}{4.119185in}}{\pgfqpoint{2.800613in}{4.114795in}}{\pgfqpoint{2.811663in}{4.114795in}}%
\pgfpathclose%
\pgfusepath{stroke,fill}%
\end{pgfscope}%
\begin{pgfscope}%
\pgfpathrectangle{\pgfqpoint{0.481978in}{0.331635in}}{\pgfqpoint{9.300000in}{7.700000in}}%
\pgfusepath{clip}%
\pgfsetbuttcap%
\pgfsetroundjoin%
\definecolor{currentfill}{rgb}{1.000000,0.705882,0.509804}%
\pgfsetfillcolor{currentfill}%
\pgfsetlinewidth{0.481800pt}%
\definecolor{currentstroke}{rgb}{1.000000,1.000000,1.000000}%
\pgfsetstrokecolor{currentstroke}%
\pgfsetdash{}{0pt}%
\pgfpathmoveto{\pgfqpoint{4.849105in}{2.668321in}}%
\pgfpathcurveto{\pgfqpoint{4.860155in}{2.668321in}}{\pgfqpoint{4.870754in}{2.672711in}}{\pgfqpoint{4.878568in}{2.680525in}}%
\pgfpathcurveto{\pgfqpoint{4.886382in}{2.688339in}}{\pgfqpoint{4.890772in}{2.698938in}}{\pgfqpoint{4.890772in}{2.709988in}}%
\pgfpathcurveto{\pgfqpoint{4.890772in}{2.721038in}}{\pgfqpoint{4.886382in}{2.731637in}}{\pgfqpoint{4.878568in}{2.739450in}}%
\pgfpathcurveto{\pgfqpoint{4.870754in}{2.747264in}}{\pgfqpoint{4.860155in}{2.751654in}}{\pgfqpoint{4.849105in}{2.751654in}}%
\pgfpathcurveto{\pgfqpoint{4.838055in}{2.751654in}}{\pgfqpoint{4.827456in}{2.747264in}}{\pgfqpoint{4.819642in}{2.739450in}}%
\pgfpathcurveto{\pgfqpoint{4.811829in}{2.731637in}}{\pgfqpoint{4.807438in}{2.721038in}}{\pgfqpoint{4.807438in}{2.709988in}}%
\pgfpathcurveto{\pgfqpoint{4.807438in}{2.698938in}}{\pgfqpoint{4.811829in}{2.688339in}}{\pgfqpoint{4.819642in}{2.680525in}}%
\pgfpathcurveto{\pgfqpoint{4.827456in}{2.672711in}}{\pgfqpoint{4.838055in}{2.668321in}}{\pgfqpoint{4.849105in}{2.668321in}}%
\pgfpathclose%
\pgfusepath{stroke,fill}%
\end{pgfscope}%
\begin{pgfscope}%
\pgfpathrectangle{\pgfqpoint{0.481978in}{0.331635in}}{\pgfqpoint{9.300000in}{7.700000in}}%
\pgfusepath{clip}%
\pgfsetbuttcap%
\pgfsetroundjoin%
\definecolor{currentfill}{rgb}{1.000000,0.705882,0.509804}%
\pgfsetfillcolor{currentfill}%
\pgfsetlinewidth{0.481800pt}%
\definecolor{currentstroke}{rgb}{1.000000,1.000000,1.000000}%
\pgfsetstrokecolor{currentstroke}%
\pgfsetdash{}{0pt}%
\pgfpathmoveto{\pgfqpoint{3.186199in}{3.136786in}}%
\pgfpathcurveto{\pgfqpoint{3.197249in}{3.136786in}}{\pgfqpoint{3.207848in}{3.141177in}}{\pgfqpoint{3.215662in}{3.148990in}}%
\pgfpathcurveto{\pgfqpoint{3.223475in}{3.156804in}}{\pgfqpoint{3.227866in}{3.167403in}}{\pgfqpoint{3.227866in}{3.178453in}}%
\pgfpathcurveto{\pgfqpoint{3.227866in}{3.189503in}}{\pgfqpoint{3.223475in}{3.200102in}}{\pgfqpoint{3.215662in}{3.207916in}}%
\pgfpathcurveto{\pgfqpoint{3.207848in}{3.215729in}}{\pgfqpoint{3.197249in}{3.220120in}}{\pgfqpoint{3.186199in}{3.220120in}}%
\pgfpathcurveto{\pgfqpoint{3.175149in}{3.220120in}}{\pgfqpoint{3.164550in}{3.215729in}}{\pgfqpoint{3.156736in}{3.207916in}}%
\pgfpathcurveto{\pgfqpoint{3.148922in}{3.200102in}}{\pgfqpoint{3.144532in}{3.189503in}}{\pgfqpoint{3.144532in}{3.178453in}}%
\pgfpathcurveto{\pgfqpoint{3.144532in}{3.167403in}}{\pgfqpoint{3.148922in}{3.156804in}}{\pgfqpoint{3.156736in}{3.148990in}}%
\pgfpathcurveto{\pgfqpoint{3.164550in}{3.141177in}}{\pgfqpoint{3.175149in}{3.136786in}}{\pgfqpoint{3.186199in}{3.136786in}}%
\pgfpathclose%
\pgfusepath{stroke,fill}%
\end{pgfscope}%
\begin{pgfscope}%
\pgfpathrectangle{\pgfqpoint{0.481978in}{0.331635in}}{\pgfqpoint{9.300000in}{7.700000in}}%
\pgfusepath{clip}%
\pgfsetbuttcap%
\pgfsetroundjoin%
\definecolor{currentfill}{rgb}{1.000000,0.705882,0.509804}%
\pgfsetfillcolor{currentfill}%
\pgfsetlinewidth{0.481800pt}%
\definecolor{currentstroke}{rgb}{1.000000,1.000000,1.000000}%
\pgfsetstrokecolor{currentstroke}%
\pgfsetdash{}{0pt}%
\pgfpathmoveto{\pgfqpoint{2.733970in}{2.873017in}}%
\pgfpathcurveto{\pgfqpoint{2.745020in}{2.873017in}}{\pgfqpoint{2.755619in}{2.877407in}}{\pgfqpoint{2.763433in}{2.885221in}}%
\pgfpathcurveto{\pgfqpoint{2.771246in}{2.893034in}}{\pgfqpoint{2.775637in}{2.903633in}}{\pgfqpoint{2.775637in}{2.914683in}}%
\pgfpathcurveto{\pgfqpoint{2.775637in}{2.925734in}}{\pgfqpoint{2.771246in}{2.936333in}}{\pgfqpoint{2.763433in}{2.944146in}}%
\pgfpathcurveto{\pgfqpoint{2.755619in}{2.951960in}}{\pgfqpoint{2.745020in}{2.956350in}}{\pgfqpoint{2.733970in}{2.956350in}}%
\pgfpathcurveto{\pgfqpoint{2.722920in}{2.956350in}}{\pgfqpoint{2.712321in}{2.951960in}}{\pgfqpoint{2.704507in}{2.944146in}}%
\pgfpathcurveto{\pgfqpoint{2.696694in}{2.936333in}}{\pgfqpoint{2.692303in}{2.925734in}}{\pgfqpoint{2.692303in}{2.914683in}}%
\pgfpathcurveto{\pgfqpoint{2.692303in}{2.903633in}}{\pgfqpoint{2.696694in}{2.893034in}}{\pgfqpoint{2.704507in}{2.885221in}}%
\pgfpathcurveto{\pgfqpoint{2.712321in}{2.877407in}}{\pgfqpoint{2.722920in}{2.873017in}}{\pgfqpoint{2.733970in}{2.873017in}}%
\pgfpathclose%
\pgfusepath{stroke,fill}%
\end{pgfscope}%
\begin{pgfscope}%
\pgfpathrectangle{\pgfqpoint{0.481978in}{0.331635in}}{\pgfqpoint{9.300000in}{7.700000in}}%
\pgfusepath{clip}%
\pgfsetbuttcap%
\pgfsetroundjoin%
\definecolor{currentfill}{rgb}{1.000000,0.705882,0.509804}%
\pgfsetfillcolor{currentfill}%
\pgfsetlinewidth{0.481800pt}%
\definecolor{currentstroke}{rgb}{1.000000,1.000000,1.000000}%
\pgfsetstrokecolor{currentstroke}%
\pgfsetdash{}{0pt}%
\pgfpathmoveto{\pgfqpoint{4.054746in}{2.658697in}}%
\pgfpathcurveto{\pgfqpoint{4.065796in}{2.658697in}}{\pgfqpoint{4.076395in}{2.663087in}}{\pgfqpoint{4.084209in}{2.670901in}}%
\pgfpathcurveto{\pgfqpoint{4.092023in}{2.678714in}}{\pgfqpoint{4.096413in}{2.689313in}}{\pgfqpoint{4.096413in}{2.700364in}}%
\pgfpathcurveto{\pgfqpoint{4.096413in}{2.711414in}}{\pgfqpoint{4.092023in}{2.722013in}}{\pgfqpoint{4.084209in}{2.729826in}}%
\pgfpathcurveto{\pgfqpoint{4.076395in}{2.737640in}}{\pgfqpoint{4.065796in}{2.742030in}}{\pgfqpoint{4.054746in}{2.742030in}}%
\pgfpathcurveto{\pgfqpoint{4.043696in}{2.742030in}}{\pgfqpoint{4.033097in}{2.737640in}}{\pgfqpoint{4.025283in}{2.729826in}}%
\pgfpathcurveto{\pgfqpoint{4.017470in}{2.722013in}}{\pgfqpoint{4.013080in}{2.711414in}}{\pgfqpoint{4.013080in}{2.700364in}}%
\pgfpathcurveto{\pgfqpoint{4.013080in}{2.689313in}}{\pgfqpoint{4.017470in}{2.678714in}}{\pgfqpoint{4.025283in}{2.670901in}}%
\pgfpathcurveto{\pgfqpoint{4.033097in}{2.663087in}}{\pgfqpoint{4.043696in}{2.658697in}}{\pgfqpoint{4.054746in}{2.658697in}}%
\pgfpathclose%
\pgfusepath{stroke,fill}%
\end{pgfscope}%
\begin{pgfscope}%
\pgfpathrectangle{\pgfqpoint{0.481978in}{0.331635in}}{\pgfqpoint{9.300000in}{7.700000in}}%
\pgfusepath{clip}%
\pgfsetbuttcap%
\pgfsetroundjoin%
\definecolor{currentfill}{rgb}{1.000000,0.705882,0.509804}%
\pgfsetfillcolor{currentfill}%
\pgfsetlinewidth{0.481800pt}%
\definecolor{currentstroke}{rgb}{1.000000,1.000000,1.000000}%
\pgfsetstrokecolor{currentstroke}%
\pgfsetdash{}{0pt}%
\pgfpathmoveto{\pgfqpoint{3.726283in}{6.087306in}}%
\pgfpathcurveto{\pgfqpoint{3.737334in}{6.087306in}}{\pgfqpoint{3.747933in}{6.091696in}}{\pgfqpoint{3.755746in}{6.099510in}}%
\pgfpathcurveto{\pgfqpoint{3.763560in}{6.107324in}}{\pgfqpoint{3.767950in}{6.117923in}}{\pgfqpoint{3.767950in}{6.128973in}}%
\pgfpathcurveto{\pgfqpoint{3.767950in}{6.140023in}}{\pgfqpoint{3.763560in}{6.150622in}}{\pgfqpoint{3.755746in}{6.158435in}}%
\pgfpathcurveto{\pgfqpoint{3.747933in}{6.166249in}}{\pgfqpoint{3.737334in}{6.170639in}}{\pgfqpoint{3.726283in}{6.170639in}}%
\pgfpathcurveto{\pgfqpoint{3.715233in}{6.170639in}}{\pgfqpoint{3.704634in}{6.166249in}}{\pgfqpoint{3.696821in}{6.158435in}}%
\pgfpathcurveto{\pgfqpoint{3.689007in}{6.150622in}}{\pgfqpoint{3.684617in}{6.140023in}}{\pgfqpoint{3.684617in}{6.128973in}}%
\pgfpathcurveto{\pgfqpoint{3.684617in}{6.117923in}}{\pgfqpoint{3.689007in}{6.107324in}}{\pgfqpoint{3.696821in}{6.099510in}}%
\pgfpathcurveto{\pgfqpoint{3.704634in}{6.091696in}}{\pgfqpoint{3.715233in}{6.087306in}}{\pgfqpoint{3.726283in}{6.087306in}}%
\pgfpathclose%
\pgfusepath{stroke,fill}%
\end{pgfscope}%
\begin{pgfscope}%
\pgfpathrectangle{\pgfqpoint{0.481978in}{0.331635in}}{\pgfqpoint{9.300000in}{7.700000in}}%
\pgfusepath{clip}%
\pgfsetbuttcap%
\pgfsetroundjoin%
\definecolor{currentfill}{rgb}{1.000000,0.705882,0.509804}%
\pgfsetfillcolor{currentfill}%
\pgfsetlinewidth{0.481800pt}%
\definecolor{currentstroke}{rgb}{1.000000,1.000000,1.000000}%
\pgfsetstrokecolor{currentstroke}%
\pgfsetdash{}{0pt}%
\pgfpathmoveto{\pgfqpoint{3.284360in}{6.066090in}}%
\pgfpathcurveto{\pgfqpoint{3.295410in}{6.066090in}}{\pgfqpoint{3.306009in}{6.070480in}}{\pgfqpoint{3.313823in}{6.078294in}}%
\pgfpathcurveto{\pgfqpoint{3.321637in}{6.086108in}}{\pgfqpoint{3.326027in}{6.096707in}}{\pgfqpoint{3.326027in}{6.107757in}}%
\pgfpathcurveto{\pgfqpoint{3.326027in}{6.118807in}}{\pgfqpoint{3.321637in}{6.129406in}}{\pgfqpoint{3.313823in}{6.137220in}}%
\pgfpathcurveto{\pgfqpoint{3.306009in}{6.145033in}}{\pgfqpoint{3.295410in}{6.149423in}}{\pgfqpoint{3.284360in}{6.149423in}}%
\pgfpathcurveto{\pgfqpoint{3.273310in}{6.149423in}}{\pgfqpoint{3.262711in}{6.145033in}}{\pgfqpoint{3.254897in}{6.137220in}}%
\pgfpathcurveto{\pgfqpoint{3.247084in}{6.129406in}}{\pgfqpoint{3.242694in}{6.118807in}}{\pgfqpoint{3.242694in}{6.107757in}}%
\pgfpathcurveto{\pgfqpoint{3.242694in}{6.096707in}}{\pgfqpoint{3.247084in}{6.086108in}}{\pgfqpoint{3.254897in}{6.078294in}}%
\pgfpathcurveto{\pgfqpoint{3.262711in}{6.070480in}}{\pgfqpoint{3.273310in}{6.066090in}}{\pgfqpoint{3.284360in}{6.066090in}}%
\pgfpathclose%
\pgfusepath{stroke,fill}%
\end{pgfscope}%
\begin{pgfscope}%
\pgfpathrectangle{\pgfqpoint{0.481978in}{0.331635in}}{\pgfqpoint{9.300000in}{7.700000in}}%
\pgfusepath{clip}%
\pgfsetbuttcap%
\pgfsetroundjoin%
\definecolor{currentfill}{rgb}{1.000000,0.705882,0.509804}%
\pgfsetfillcolor{currentfill}%
\pgfsetlinewidth{0.481800pt}%
\definecolor{currentstroke}{rgb}{1.000000,1.000000,1.000000}%
\pgfsetstrokecolor{currentstroke}%
\pgfsetdash{}{0pt}%
\pgfpathmoveto{\pgfqpoint{4.144311in}{2.733847in}}%
\pgfpathcurveto{\pgfqpoint{4.155361in}{2.733847in}}{\pgfqpoint{4.165960in}{2.738237in}}{\pgfqpoint{4.173773in}{2.746051in}}%
\pgfpathcurveto{\pgfqpoint{4.181587in}{2.753865in}}{\pgfqpoint{4.185977in}{2.764464in}}{\pgfqpoint{4.185977in}{2.775514in}}%
\pgfpathcurveto{\pgfqpoint{4.185977in}{2.786564in}}{\pgfqpoint{4.181587in}{2.797163in}}{\pgfqpoint{4.173773in}{2.804977in}}%
\pgfpathcurveto{\pgfqpoint{4.165960in}{2.812790in}}{\pgfqpoint{4.155361in}{2.817181in}}{\pgfqpoint{4.144311in}{2.817181in}}%
\pgfpathcurveto{\pgfqpoint{4.133261in}{2.817181in}}{\pgfqpoint{4.122661in}{2.812790in}}{\pgfqpoint{4.114848in}{2.804977in}}%
\pgfpathcurveto{\pgfqpoint{4.107034in}{2.797163in}}{\pgfqpoint{4.102644in}{2.786564in}}{\pgfqpoint{4.102644in}{2.775514in}}%
\pgfpathcurveto{\pgfqpoint{4.102644in}{2.764464in}}{\pgfqpoint{4.107034in}{2.753865in}}{\pgfqpoint{4.114848in}{2.746051in}}%
\pgfpathcurveto{\pgfqpoint{4.122661in}{2.738237in}}{\pgfqpoint{4.133261in}{2.733847in}}{\pgfqpoint{4.144311in}{2.733847in}}%
\pgfpathclose%
\pgfusepath{stroke,fill}%
\end{pgfscope}%
\begin{pgfscope}%
\pgfpathrectangle{\pgfqpoint{0.481978in}{0.331635in}}{\pgfqpoint{9.300000in}{7.700000in}}%
\pgfusepath{clip}%
\pgfsetbuttcap%
\pgfsetroundjoin%
\definecolor{currentfill}{rgb}{1.000000,0.705882,0.509804}%
\pgfsetfillcolor{currentfill}%
\pgfsetlinewidth{0.481800pt}%
\definecolor{currentstroke}{rgb}{1.000000,1.000000,1.000000}%
\pgfsetstrokecolor{currentstroke}%
\pgfsetdash{}{0pt}%
\pgfpathmoveto{\pgfqpoint{2.660738in}{2.575194in}}%
\pgfpathcurveto{\pgfqpoint{2.671788in}{2.575194in}}{\pgfqpoint{2.682387in}{2.579585in}}{\pgfqpoint{2.690201in}{2.587398in}}%
\pgfpathcurveto{\pgfqpoint{2.698014in}{2.595212in}}{\pgfqpoint{2.702404in}{2.605811in}}{\pgfqpoint{2.702404in}{2.616861in}}%
\pgfpathcurveto{\pgfqpoint{2.702404in}{2.627911in}}{\pgfqpoint{2.698014in}{2.638510in}}{\pgfqpoint{2.690201in}{2.646324in}}%
\pgfpathcurveto{\pgfqpoint{2.682387in}{2.654137in}}{\pgfqpoint{2.671788in}{2.658528in}}{\pgfqpoint{2.660738in}{2.658528in}}%
\pgfpathcurveto{\pgfqpoint{2.649688in}{2.658528in}}{\pgfqpoint{2.639089in}{2.654137in}}{\pgfqpoint{2.631275in}{2.646324in}}%
\pgfpathcurveto{\pgfqpoint{2.623461in}{2.638510in}}{\pgfqpoint{2.619071in}{2.627911in}}{\pgfqpoint{2.619071in}{2.616861in}}%
\pgfpathcurveto{\pgfqpoint{2.619071in}{2.605811in}}{\pgfqpoint{2.623461in}{2.595212in}}{\pgfqpoint{2.631275in}{2.587398in}}%
\pgfpathcurveto{\pgfqpoint{2.639089in}{2.579585in}}{\pgfqpoint{2.649688in}{2.575194in}}{\pgfqpoint{2.660738in}{2.575194in}}%
\pgfpathclose%
\pgfusepath{stroke,fill}%
\end{pgfscope}%
\begin{pgfscope}%
\pgfpathrectangle{\pgfqpoint{0.481978in}{0.331635in}}{\pgfqpoint{9.300000in}{7.700000in}}%
\pgfusepath{clip}%
\pgfsetbuttcap%
\pgfsetroundjoin%
\definecolor{currentfill}{rgb}{1.000000,0.705882,0.509804}%
\pgfsetfillcolor{currentfill}%
\pgfsetlinewidth{0.481800pt}%
\definecolor{currentstroke}{rgb}{1.000000,1.000000,1.000000}%
\pgfsetstrokecolor{currentstroke}%
\pgfsetdash{}{0pt}%
\pgfpathmoveto{\pgfqpoint{3.685446in}{2.354103in}}%
\pgfpathcurveto{\pgfqpoint{3.696496in}{2.354103in}}{\pgfqpoint{3.707095in}{2.358493in}}{\pgfqpoint{3.714908in}{2.366307in}}%
\pgfpathcurveto{\pgfqpoint{3.722722in}{2.374121in}}{\pgfqpoint{3.727112in}{2.384720in}}{\pgfqpoint{3.727112in}{2.395770in}}%
\pgfpathcurveto{\pgfqpoint{3.727112in}{2.406820in}}{\pgfqpoint{3.722722in}{2.417419in}}{\pgfqpoint{3.714908in}{2.425233in}}%
\pgfpathcurveto{\pgfqpoint{3.707095in}{2.433046in}}{\pgfqpoint{3.696496in}{2.437436in}}{\pgfqpoint{3.685446in}{2.437436in}}%
\pgfpathcurveto{\pgfqpoint{3.674395in}{2.437436in}}{\pgfqpoint{3.663796in}{2.433046in}}{\pgfqpoint{3.655983in}{2.425233in}}%
\pgfpathcurveto{\pgfqpoint{3.648169in}{2.417419in}}{\pgfqpoint{3.643779in}{2.406820in}}{\pgfqpoint{3.643779in}{2.395770in}}%
\pgfpathcurveto{\pgfqpoint{3.643779in}{2.384720in}}{\pgfqpoint{3.648169in}{2.374121in}}{\pgfqpoint{3.655983in}{2.366307in}}%
\pgfpathcurveto{\pgfqpoint{3.663796in}{2.358493in}}{\pgfqpoint{3.674395in}{2.354103in}}{\pgfqpoint{3.685446in}{2.354103in}}%
\pgfpathclose%
\pgfusepath{stroke,fill}%
\end{pgfscope}%
\begin{pgfscope}%
\pgfpathrectangle{\pgfqpoint{0.481978in}{0.331635in}}{\pgfqpoint{9.300000in}{7.700000in}}%
\pgfusepath{clip}%
\pgfsetbuttcap%
\pgfsetroundjoin%
\definecolor{currentfill}{rgb}{1.000000,0.705882,0.509804}%
\pgfsetfillcolor{currentfill}%
\pgfsetlinewidth{0.481800pt}%
\definecolor{currentstroke}{rgb}{1.000000,1.000000,1.000000}%
\pgfsetstrokecolor{currentstroke}%
\pgfsetdash{}{0pt}%
\pgfpathmoveto{\pgfqpoint{3.030052in}{5.719688in}}%
\pgfpathcurveto{\pgfqpoint{3.041102in}{5.719688in}}{\pgfqpoint{3.051701in}{5.724078in}}{\pgfqpoint{3.059514in}{5.731892in}}%
\pgfpathcurveto{\pgfqpoint{3.067328in}{5.739706in}}{\pgfqpoint{3.071718in}{5.750305in}}{\pgfqpoint{3.071718in}{5.761355in}}%
\pgfpathcurveto{\pgfqpoint{3.071718in}{5.772405in}}{\pgfqpoint{3.067328in}{5.783004in}}{\pgfqpoint{3.059514in}{5.790818in}}%
\pgfpathcurveto{\pgfqpoint{3.051701in}{5.798631in}}{\pgfqpoint{3.041102in}{5.803022in}}{\pgfqpoint{3.030052in}{5.803022in}}%
\pgfpathcurveto{\pgfqpoint{3.019001in}{5.803022in}}{\pgfqpoint{3.008402in}{5.798631in}}{\pgfqpoint{3.000589in}{5.790818in}}%
\pgfpathcurveto{\pgfqpoint{2.992775in}{5.783004in}}{\pgfqpoint{2.988385in}{5.772405in}}{\pgfqpoint{2.988385in}{5.761355in}}%
\pgfpathcurveto{\pgfqpoint{2.988385in}{5.750305in}}{\pgfqpoint{2.992775in}{5.739706in}}{\pgfqpoint{3.000589in}{5.731892in}}%
\pgfpathcurveto{\pgfqpoint{3.008402in}{5.724078in}}{\pgfqpoint{3.019001in}{5.719688in}}{\pgfqpoint{3.030052in}{5.719688in}}%
\pgfpathclose%
\pgfusepath{stroke,fill}%
\end{pgfscope}%
\begin{pgfscope}%
\pgfpathrectangle{\pgfqpoint{0.481978in}{0.331635in}}{\pgfqpoint{9.300000in}{7.700000in}}%
\pgfusepath{clip}%
\pgfsetbuttcap%
\pgfsetroundjoin%
\definecolor{currentfill}{rgb}{1.000000,0.705882,0.509804}%
\pgfsetfillcolor{currentfill}%
\pgfsetlinewidth{0.481800pt}%
\definecolor{currentstroke}{rgb}{1.000000,1.000000,1.000000}%
\pgfsetstrokecolor{currentstroke}%
\pgfsetdash{}{0pt}%
\pgfpathmoveto{\pgfqpoint{1.530784in}{2.492756in}}%
\pgfpathcurveto{\pgfqpoint{1.541834in}{2.492756in}}{\pgfqpoint{1.552434in}{2.497146in}}{\pgfqpoint{1.560247in}{2.504960in}}%
\pgfpathcurveto{\pgfqpoint{1.568061in}{2.512773in}}{\pgfqpoint{1.572451in}{2.523373in}}{\pgfqpoint{1.572451in}{2.534423in}}%
\pgfpathcurveto{\pgfqpoint{1.572451in}{2.545473in}}{\pgfqpoint{1.568061in}{2.556072in}}{\pgfqpoint{1.560247in}{2.563885in}}%
\pgfpathcurveto{\pgfqpoint{1.552434in}{2.571699in}}{\pgfqpoint{1.541834in}{2.576089in}}{\pgfqpoint{1.530784in}{2.576089in}}%
\pgfpathcurveto{\pgfqpoint{1.519734in}{2.576089in}}{\pgfqpoint{1.509135in}{2.571699in}}{\pgfqpoint{1.501322in}{2.563885in}}%
\pgfpathcurveto{\pgfqpoint{1.493508in}{2.556072in}}{\pgfqpoint{1.489118in}{2.545473in}}{\pgfqpoint{1.489118in}{2.534423in}}%
\pgfpathcurveto{\pgfqpoint{1.489118in}{2.523373in}}{\pgfqpoint{1.493508in}{2.512773in}}{\pgfqpoint{1.501322in}{2.504960in}}%
\pgfpathcurveto{\pgfqpoint{1.509135in}{2.497146in}}{\pgfqpoint{1.519734in}{2.492756in}}{\pgfqpoint{1.530784in}{2.492756in}}%
\pgfpathclose%
\pgfusepath{stroke,fill}%
\end{pgfscope}%
\begin{pgfscope}%
\pgfpathrectangle{\pgfqpoint{0.481978in}{0.331635in}}{\pgfqpoint{9.300000in}{7.700000in}}%
\pgfusepath{clip}%
\pgfsetbuttcap%
\pgfsetroundjoin%
\definecolor{currentfill}{rgb}{1.000000,0.705882,0.509804}%
\pgfsetfillcolor{currentfill}%
\pgfsetlinewidth{0.481800pt}%
\definecolor{currentstroke}{rgb}{1.000000,1.000000,1.000000}%
\pgfsetstrokecolor{currentstroke}%
\pgfsetdash{}{0pt}%
\pgfpathmoveto{\pgfqpoint{1.565141in}{5.059110in}}%
\pgfpathcurveto{\pgfqpoint{1.576191in}{5.059110in}}{\pgfqpoint{1.586790in}{5.063500in}}{\pgfqpoint{1.594604in}{5.071313in}}%
\pgfpathcurveto{\pgfqpoint{1.602418in}{5.079127in}}{\pgfqpoint{1.606808in}{5.089726in}}{\pgfqpoint{1.606808in}{5.100776in}}%
\pgfpathcurveto{\pgfqpoint{1.606808in}{5.111826in}}{\pgfqpoint{1.602418in}{5.122425in}}{\pgfqpoint{1.594604in}{5.130239in}}%
\pgfpathcurveto{\pgfqpoint{1.586790in}{5.138053in}}{\pgfqpoint{1.576191in}{5.142443in}}{\pgfqpoint{1.565141in}{5.142443in}}%
\pgfpathcurveto{\pgfqpoint{1.554091in}{5.142443in}}{\pgfqpoint{1.543492in}{5.138053in}}{\pgfqpoint{1.535678in}{5.130239in}}%
\pgfpathcurveto{\pgfqpoint{1.527865in}{5.122425in}}{\pgfqpoint{1.523475in}{5.111826in}}{\pgfqpoint{1.523475in}{5.100776in}}%
\pgfpathcurveto{\pgfqpoint{1.523475in}{5.089726in}}{\pgfqpoint{1.527865in}{5.079127in}}{\pgfqpoint{1.535678in}{5.071313in}}%
\pgfpathcurveto{\pgfqpoint{1.543492in}{5.063500in}}{\pgfqpoint{1.554091in}{5.059110in}}{\pgfqpoint{1.565141in}{5.059110in}}%
\pgfpathclose%
\pgfusepath{stroke,fill}%
\end{pgfscope}%
\begin{pgfscope}%
\pgfpathrectangle{\pgfqpoint{0.481978in}{0.331635in}}{\pgfqpoint{9.300000in}{7.700000in}}%
\pgfusepath{clip}%
\pgfsetbuttcap%
\pgfsetroundjoin%
\definecolor{currentfill}{rgb}{1.000000,0.705882,0.509804}%
\pgfsetfillcolor{currentfill}%
\pgfsetlinewidth{0.481800pt}%
\definecolor{currentstroke}{rgb}{1.000000,1.000000,1.000000}%
\pgfsetstrokecolor{currentstroke}%
\pgfsetdash{}{0pt}%
\pgfpathmoveto{\pgfqpoint{4.138549in}{4.903543in}}%
\pgfpathcurveto{\pgfqpoint{4.149599in}{4.903543in}}{\pgfqpoint{4.160198in}{4.907934in}}{\pgfqpoint{4.168011in}{4.915747in}}%
\pgfpathcurveto{\pgfqpoint{4.175825in}{4.923561in}}{\pgfqpoint{4.180215in}{4.934160in}}{\pgfqpoint{4.180215in}{4.945210in}}%
\pgfpathcurveto{\pgfqpoint{4.180215in}{4.956260in}}{\pgfqpoint{4.175825in}{4.966859in}}{\pgfqpoint{4.168011in}{4.974673in}}%
\pgfpathcurveto{\pgfqpoint{4.160198in}{4.982487in}}{\pgfqpoint{4.149599in}{4.986877in}}{\pgfqpoint{4.138549in}{4.986877in}}%
\pgfpathcurveto{\pgfqpoint{4.127498in}{4.986877in}}{\pgfqpoint{4.116899in}{4.982487in}}{\pgfqpoint{4.109086in}{4.974673in}}%
\pgfpathcurveto{\pgfqpoint{4.101272in}{4.966859in}}{\pgfqpoint{4.096882in}{4.956260in}}{\pgfqpoint{4.096882in}{4.945210in}}%
\pgfpathcurveto{\pgfqpoint{4.096882in}{4.934160in}}{\pgfqpoint{4.101272in}{4.923561in}}{\pgfqpoint{4.109086in}{4.915747in}}%
\pgfpathcurveto{\pgfqpoint{4.116899in}{4.907934in}}{\pgfqpoint{4.127498in}{4.903543in}}{\pgfqpoint{4.138549in}{4.903543in}}%
\pgfpathclose%
\pgfusepath{stroke,fill}%
\end{pgfscope}%
\begin{pgfscope}%
\pgfpathrectangle{\pgfqpoint{0.481978in}{0.331635in}}{\pgfqpoint{9.300000in}{7.700000in}}%
\pgfusepath{clip}%
\pgfsetbuttcap%
\pgfsetroundjoin%
\definecolor{currentfill}{rgb}{1.000000,0.705882,0.509804}%
\pgfsetfillcolor{currentfill}%
\pgfsetlinewidth{0.481800pt}%
\definecolor{currentstroke}{rgb}{1.000000,1.000000,1.000000}%
\pgfsetstrokecolor{currentstroke}%
\pgfsetdash{}{0pt}%
\pgfpathmoveto{\pgfqpoint{4.409605in}{4.724804in}}%
\pgfpathcurveto{\pgfqpoint{4.420655in}{4.724804in}}{\pgfqpoint{4.431254in}{4.729194in}}{\pgfqpoint{4.439068in}{4.737007in}}%
\pgfpathcurveto{\pgfqpoint{4.446882in}{4.744821in}}{\pgfqpoint{4.451272in}{4.755420in}}{\pgfqpoint{4.451272in}{4.766470in}}%
\pgfpathcurveto{\pgfqpoint{4.451272in}{4.777520in}}{\pgfqpoint{4.446882in}{4.788119in}}{\pgfqpoint{4.439068in}{4.795933in}}%
\pgfpathcurveto{\pgfqpoint{4.431254in}{4.803747in}}{\pgfqpoint{4.420655in}{4.808137in}}{\pgfqpoint{4.409605in}{4.808137in}}%
\pgfpathcurveto{\pgfqpoint{4.398555in}{4.808137in}}{\pgfqpoint{4.387956in}{4.803747in}}{\pgfqpoint{4.380142in}{4.795933in}}%
\pgfpathcurveto{\pgfqpoint{4.372329in}{4.788119in}}{\pgfqpoint{4.367939in}{4.777520in}}{\pgfqpoint{4.367939in}{4.766470in}}%
\pgfpathcurveto{\pgfqpoint{4.367939in}{4.755420in}}{\pgfqpoint{4.372329in}{4.744821in}}{\pgfqpoint{4.380142in}{4.737007in}}%
\pgfpathcurveto{\pgfqpoint{4.387956in}{4.729194in}}{\pgfqpoint{4.398555in}{4.724804in}}{\pgfqpoint{4.409605in}{4.724804in}}%
\pgfpathclose%
\pgfusepath{stroke,fill}%
\end{pgfscope}%
\begin{pgfscope}%
\pgfpathrectangle{\pgfqpoint{0.481978in}{0.331635in}}{\pgfqpoint{9.300000in}{7.700000in}}%
\pgfusepath{clip}%
\pgfsetbuttcap%
\pgfsetroundjoin%
\definecolor{currentfill}{rgb}{1.000000,0.705882,0.509804}%
\pgfsetfillcolor{currentfill}%
\pgfsetlinewidth{0.481800pt}%
\definecolor{currentstroke}{rgb}{1.000000,1.000000,1.000000}%
\pgfsetstrokecolor{currentstroke}%
\pgfsetdash{}{0pt}%
\pgfpathmoveto{\pgfqpoint{4.134948in}{3.124159in}}%
\pgfpathcurveto{\pgfqpoint{4.145998in}{3.124159in}}{\pgfqpoint{4.156597in}{3.128549in}}{\pgfqpoint{4.164410in}{3.136363in}}%
\pgfpathcurveto{\pgfqpoint{4.172224in}{3.144177in}}{\pgfqpoint{4.176614in}{3.154776in}}{\pgfqpoint{4.176614in}{3.165826in}}%
\pgfpathcurveto{\pgfqpoint{4.176614in}{3.176876in}}{\pgfqpoint{4.172224in}{3.187475in}}{\pgfqpoint{4.164410in}{3.195289in}}%
\pgfpathcurveto{\pgfqpoint{4.156597in}{3.203102in}}{\pgfqpoint{4.145998in}{3.207493in}}{\pgfqpoint{4.134948in}{3.207493in}}%
\pgfpathcurveto{\pgfqpoint{4.123898in}{3.207493in}}{\pgfqpoint{4.113298in}{3.203102in}}{\pgfqpoint{4.105485in}{3.195289in}}%
\pgfpathcurveto{\pgfqpoint{4.097671in}{3.187475in}}{\pgfqpoint{4.093281in}{3.176876in}}{\pgfqpoint{4.093281in}{3.165826in}}%
\pgfpathcurveto{\pgfqpoint{4.093281in}{3.154776in}}{\pgfqpoint{4.097671in}{3.144177in}}{\pgfqpoint{4.105485in}{3.136363in}}%
\pgfpathcurveto{\pgfqpoint{4.113298in}{3.128549in}}{\pgfqpoint{4.123898in}{3.124159in}}{\pgfqpoint{4.134948in}{3.124159in}}%
\pgfpathclose%
\pgfusepath{stroke,fill}%
\end{pgfscope}%
\begin{pgfscope}%
\pgfpathrectangle{\pgfqpoint{0.481978in}{0.331635in}}{\pgfqpoint{9.300000in}{7.700000in}}%
\pgfusepath{clip}%
\pgfsetbuttcap%
\pgfsetroundjoin%
\definecolor{currentfill}{rgb}{1.000000,0.705882,0.509804}%
\pgfsetfillcolor{currentfill}%
\pgfsetlinewidth{0.481800pt}%
\definecolor{currentstroke}{rgb}{1.000000,1.000000,1.000000}%
\pgfsetstrokecolor{currentstroke}%
\pgfsetdash{}{0pt}%
\pgfpathmoveto{\pgfqpoint{2.632233in}{4.333030in}}%
\pgfpathcurveto{\pgfqpoint{2.643283in}{4.333030in}}{\pgfqpoint{2.653882in}{4.337420in}}{\pgfqpoint{2.661695in}{4.345234in}}%
\pgfpathcurveto{\pgfqpoint{2.669509in}{4.353047in}}{\pgfqpoint{2.673899in}{4.363646in}}{\pgfqpoint{2.673899in}{4.374697in}}%
\pgfpathcurveto{\pgfqpoint{2.673899in}{4.385747in}}{\pgfqpoint{2.669509in}{4.396346in}}{\pgfqpoint{2.661695in}{4.404159in}}%
\pgfpathcurveto{\pgfqpoint{2.653882in}{4.411973in}}{\pgfqpoint{2.643283in}{4.416363in}}{\pgfqpoint{2.632233in}{4.416363in}}%
\pgfpathcurveto{\pgfqpoint{2.621183in}{4.416363in}}{\pgfqpoint{2.610584in}{4.411973in}}{\pgfqpoint{2.602770in}{4.404159in}}%
\pgfpathcurveto{\pgfqpoint{2.594956in}{4.396346in}}{\pgfqpoint{2.590566in}{4.385747in}}{\pgfqpoint{2.590566in}{4.374697in}}%
\pgfpathcurveto{\pgfqpoint{2.590566in}{4.363646in}}{\pgfqpoint{2.594956in}{4.353047in}}{\pgfqpoint{2.602770in}{4.345234in}}%
\pgfpathcurveto{\pgfqpoint{2.610584in}{4.337420in}}{\pgfqpoint{2.621183in}{4.333030in}}{\pgfqpoint{2.632233in}{4.333030in}}%
\pgfpathclose%
\pgfusepath{stroke,fill}%
\end{pgfscope}%
\begin{pgfscope}%
\pgfpathrectangle{\pgfqpoint{0.481978in}{0.331635in}}{\pgfqpoint{9.300000in}{7.700000in}}%
\pgfusepath{clip}%
\pgfsetbuttcap%
\pgfsetroundjoin%
\definecolor{currentfill}{rgb}{1.000000,0.705882,0.509804}%
\pgfsetfillcolor{currentfill}%
\pgfsetlinewidth{0.481800pt}%
\definecolor{currentstroke}{rgb}{1.000000,1.000000,1.000000}%
\pgfsetstrokecolor{currentstroke}%
\pgfsetdash{}{0pt}%
\pgfpathmoveto{\pgfqpoint{3.332817in}{4.260900in}}%
\pgfpathcurveto{\pgfqpoint{3.343867in}{4.260900in}}{\pgfqpoint{3.354466in}{4.265290in}}{\pgfqpoint{3.362280in}{4.273104in}}%
\pgfpathcurveto{\pgfqpoint{3.370094in}{4.280917in}}{\pgfqpoint{3.374484in}{4.291516in}}{\pgfqpoint{3.374484in}{4.302567in}}%
\pgfpathcurveto{\pgfqpoint{3.374484in}{4.313617in}}{\pgfqpoint{3.370094in}{4.324216in}}{\pgfqpoint{3.362280in}{4.332029in}}%
\pgfpathcurveto{\pgfqpoint{3.354466in}{4.339843in}}{\pgfqpoint{3.343867in}{4.344233in}}{\pgfqpoint{3.332817in}{4.344233in}}%
\pgfpathcurveto{\pgfqpoint{3.321767in}{4.344233in}}{\pgfqpoint{3.311168in}{4.339843in}}{\pgfqpoint{3.303354in}{4.332029in}}%
\pgfpathcurveto{\pgfqpoint{3.295541in}{4.324216in}}{\pgfqpoint{3.291150in}{4.313617in}}{\pgfqpoint{3.291150in}{4.302567in}}%
\pgfpathcurveto{\pgfqpoint{3.291150in}{4.291516in}}{\pgfqpoint{3.295541in}{4.280917in}}{\pgfqpoint{3.303354in}{4.273104in}}%
\pgfpathcurveto{\pgfqpoint{3.311168in}{4.265290in}}{\pgfqpoint{3.321767in}{4.260900in}}{\pgfqpoint{3.332817in}{4.260900in}}%
\pgfpathclose%
\pgfusepath{stroke,fill}%
\end{pgfscope}%
\begin{pgfscope}%
\pgfpathrectangle{\pgfqpoint{0.481978in}{0.331635in}}{\pgfqpoint{9.300000in}{7.700000in}}%
\pgfusepath{clip}%
\pgfsetbuttcap%
\pgfsetroundjoin%
\definecolor{currentfill}{rgb}{1.000000,0.705882,0.509804}%
\pgfsetfillcolor{currentfill}%
\pgfsetlinewidth{0.481800pt}%
\definecolor{currentstroke}{rgb}{1.000000,1.000000,1.000000}%
\pgfsetstrokecolor{currentstroke}%
\pgfsetdash{}{0pt}%
\pgfpathmoveto{\pgfqpoint{3.045518in}{4.242593in}}%
\pgfpathcurveto{\pgfqpoint{3.056568in}{4.242593in}}{\pgfqpoint{3.067167in}{4.246983in}}{\pgfqpoint{3.074980in}{4.254797in}}%
\pgfpathcurveto{\pgfqpoint{3.082794in}{4.262610in}}{\pgfqpoint{3.087184in}{4.273209in}}{\pgfqpoint{3.087184in}{4.284260in}}%
\pgfpathcurveto{\pgfqpoint{3.087184in}{4.295310in}}{\pgfqpoint{3.082794in}{4.305909in}}{\pgfqpoint{3.074980in}{4.313722in}}%
\pgfpathcurveto{\pgfqpoint{3.067167in}{4.321536in}}{\pgfqpoint{3.056568in}{4.325926in}}{\pgfqpoint{3.045518in}{4.325926in}}%
\pgfpathcurveto{\pgfqpoint{3.034468in}{4.325926in}}{\pgfqpoint{3.023869in}{4.321536in}}{\pgfqpoint{3.016055in}{4.313722in}}%
\pgfpathcurveto{\pgfqpoint{3.008241in}{4.305909in}}{\pgfqpoint{3.003851in}{4.295310in}}{\pgfqpoint{3.003851in}{4.284260in}}%
\pgfpathcurveto{\pgfqpoint{3.003851in}{4.273209in}}{\pgfqpoint{3.008241in}{4.262610in}}{\pgfqpoint{3.016055in}{4.254797in}}%
\pgfpathcurveto{\pgfqpoint{3.023869in}{4.246983in}}{\pgfqpoint{3.034468in}{4.242593in}}{\pgfqpoint{3.045518in}{4.242593in}}%
\pgfpathclose%
\pgfusepath{stroke,fill}%
\end{pgfscope}%
\begin{pgfscope}%
\pgfpathrectangle{\pgfqpoint{0.481978in}{0.331635in}}{\pgfqpoint{9.300000in}{7.700000in}}%
\pgfusepath{clip}%
\pgfsetbuttcap%
\pgfsetroundjoin%
\definecolor{currentfill}{rgb}{1.000000,0.705882,0.509804}%
\pgfsetfillcolor{currentfill}%
\pgfsetlinewidth{0.481800pt}%
\definecolor{currentstroke}{rgb}{1.000000,1.000000,1.000000}%
\pgfsetstrokecolor{currentstroke}%
\pgfsetdash{}{0pt}%
\pgfpathmoveto{\pgfqpoint{2.948753in}{3.408484in}}%
\pgfpathcurveto{\pgfqpoint{2.959803in}{3.408484in}}{\pgfqpoint{2.970402in}{3.412875in}}{\pgfqpoint{2.978216in}{3.420688in}}%
\pgfpathcurveto{\pgfqpoint{2.986030in}{3.428502in}}{\pgfqpoint{2.990420in}{3.439101in}}{\pgfqpoint{2.990420in}{3.450151in}}%
\pgfpathcurveto{\pgfqpoint{2.990420in}{3.461201in}}{\pgfqpoint{2.986030in}{3.471800in}}{\pgfqpoint{2.978216in}{3.479614in}}%
\pgfpathcurveto{\pgfqpoint{2.970402in}{3.487427in}}{\pgfqpoint{2.959803in}{3.491818in}}{\pgfqpoint{2.948753in}{3.491818in}}%
\pgfpathcurveto{\pgfqpoint{2.937703in}{3.491818in}}{\pgfqpoint{2.927104in}{3.487427in}}{\pgfqpoint{2.919290in}{3.479614in}}%
\pgfpathcurveto{\pgfqpoint{2.911477in}{3.471800in}}{\pgfqpoint{2.907087in}{3.461201in}}{\pgfqpoint{2.907087in}{3.450151in}}%
\pgfpathcurveto{\pgfqpoint{2.907087in}{3.439101in}}{\pgfqpoint{2.911477in}{3.428502in}}{\pgfqpoint{2.919290in}{3.420688in}}%
\pgfpathcurveto{\pgfqpoint{2.927104in}{3.412875in}}{\pgfqpoint{2.937703in}{3.408484in}}{\pgfqpoint{2.948753in}{3.408484in}}%
\pgfpathclose%
\pgfusepath{stroke,fill}%
\end{pgfscope}%
\begin{pgfscope}%
\pgfpathrectangle{\pgfqpoint{0.481978in}{0.331635in}}{\pgfqpoint{9.300000in}{7.700000in}}%
\pgfusepath{clip}%
\pgfsetbuttcap%
\pgfsetroundjoin%
\definecolor{currentfill}{rgb}{1.000000,0.705882,0.509804}%
\pgfsetfillcolor{currentfill}%
\pgfsetlinewidth{0.481800pt}%
\definecolor{currentstroke}{rgb}{1.000000,1.000000,1.000000}%
\pgfsetstrokecolor{currentstroke}%
\pgfsetdash{}{0pt}%
\pgfpathmoveto{\pgfqpoint{3.979638in}{3.574715in}}%
\pgfpathcurveto{\pgfqpoint{3.990688in}{3.574715in}}{\pgfqpoint{4.001287in}{3.579106in}}{\pgfqpoint{4.009101in}{3.586919in}}%
\pgfpathcurveto{\pgfqpoint{4.016914in}{3.594733in}}{\pgfqpoint{4.021304in}{3.605332in}}{\pgfqpoint{4.021304in}{3.616382in}}%
\pgfpathcurveto{\pgfqpoint{4.021304in}{3.627432in}}{\pgfqpoint{4.016914in}{3.638031in}}{\pgfqpoint{4.009101in}{3.645845in}}%
\pgfpathcurveto{\pgfqpoint{4.001287in}{3.653659in}}{\pgfqpoint{3.990688in}{3.658049in}}{\pgfqpoint{3.979638in}{3.658049in}}%
\pgfpathcurveto{\pgfqpoint{3.968588in}{3.658049in}}{\pgfqpoint{3.957989in}{3.653659in}}{\pgfqpoint{3.950175in}{3.645845in}}%
\pgfpathcurveto{\pgfqpoint{3.942361in}{3.638031in}}{\pgfqpoint{3.937971in}{3.627432in}}{\pgfqpoint{3.937971in}{3.616382in}}%
\pgfpathcurveto{\pgfqpoint{3.937971in}{3.605332in}}{\pgfqpoint{3.942361in}{3.594733in}}{\pgfqpoint{3.950175in}{3.586919in}}%
\pgfpathcurveto{\pgfqpoint{3.957989in}{3.579106in}}{\pgfqpoint{3.968588in}{3.574715in}}{\pgfqpoint{3.979638in}{3.574715in}}%
\pgfpathclose%
\pgfusepath{stroke,fill}%
\end{pgfscope}%
\begin{pgfscope}%
\pgfpathrectangle{\pgfqpoint{0.481978in}{0.331635in}}{\pgfqpoint{9.300000in}{7.700000in}}%
\pgfusepath{clip}%
\pgfsetbuttcap%
\pgfsetroundjoin%
\definecolor{currentfill}{rgb}{1.000000,0.705882,0.509804}%
\pgfsetfillcolor{currentfill}%
\pgfsetlinewidth{0.481800pt}%
\definecolor{currentstroke}{rgb}{1.000000,1.000000,1.000000}%
\pgfsetstrokecolor{currentstroke}%
\pgfsetdash{}{0pt}%
\pgfpathmoveto{\pgfqpoint{4.333852in}{2.410081in}}%
\pgfpathcurveto{\pgfqpoint{4.344902in}{2.410081in}}{\pgfqpoint{4.355501in}{2.414471in}}{\pgfqpoint{4.363315in}{2.422284in}}%
\pgfpathcurveto{\pgfqpoint{4.371129in}{2.430098in}}{\pgfqpoint{4.375519in}{2.440697in}}{\pgfqpoint{4.375519in}{2.451747in}}%
\pgfpathcurveto{\pgfqpoint{4.375519in}{2.462797in}}{\pgfqpoint{4.371129in}{2.473396in}}{\pgfqpoint{4.363315in}{2.481210in}}%
\pgfpathcurveto{\pgfqpoint{4.355501in}{2.489024in}}{\pgfqpoint{4.344902in}{2.493414in}}{\pgfqpoint{4.333852in}{2.493414in}}%
\pgfpathcurveto{\pgfqpoint{4.322802in}{2.493414in}}{\pgfqpoint{4.312203in}{2.489024in}}{\pgfqpoint{4.304389in}{2.481210in}}%
\pgfpathcurveto{\pgfqpoint{4.296576in}{2.473396in}}{\pgfqpoint{4.292186in}{2.462797in}}{\pgfqpoint{4.292186in}{2.451747in}}%
\pgfpathcurveto{\pgfqpoint{4.292186in}{2.440697in}}{\pgfqpoint{4.296576in}{2.430098in}}{\pgfqpoint{4.304389in}{2.422284in}}%
\pgfpathcurveto{\pgfqpoint{4.312203in}{2.414471in}}{\pgfqpoint{4.322802in}{2.410081in}}{\pgfqpoint{4.333852in}{2.410081in}}%
\pgfpathclose%
\pgfusepath{stroke,fill}%
\end{pgfscope}%
\begin{pgfscope}%
\pgfpathrectangle{\pgfqpoint{0.481978in}{0.331635in}}{\pgfqpoint{9.300000in}{7.700000in}}%
\pgfusepath{clip}%
\pgfsetbuttcap%
\pgfsetroundjoin%
\definecolor{currentfill}{rgb}{1.000000,0.705882,0.509804}%
\pgfsetfillcolor{currentfill}%
\pgfsetlinewidth{0.481800pt}%
\definecolor{currentstroke}{rgb}{1.000000,1.000000,1.000000}%
\pgfsetstrokecolor{currentstroke}%
\pgfsetdash{}{0pt}%
\pgfpathmoveto{\pgfqpoint{1.939275in}{4.182873in}}%
\pgfpathcurveto{\pgfqpoint{1.950326in}{4.182873in}}{\pgfqpoint{1.960925in}{4.187263in}}{\pgfqpoint{1.968738in}{4.195077in}}%
\pgfpathcurveto{\pgfqpoint{1.976552in}{4.202890in}}{\pgfqpoint{1.980942in}{4.213489in}}{\pgfqpoint{1.980942in}{4.224539in}}%
\pgfpathcurveto{\pgfqpoint{1.980942in}{4.235590in}}{\pgfqpoint{1.976552in}{4.246189in}}{\pgfqpoint{1.968738in}{4.254002in}}%
\pgfpathcurveto{\pgfqpoint{1.960925in}{4.261816in}}{\pgfqpoint{1.950326in}{4.266206in}}{\pgfqpoint{1.939275in}{4.266206in}}%
\pgfpathcurveto{\pgfqpoint{1.928225in}{4.266206in}}{\pgfqpoint{1.917626in}{4.261816in}}{\pgfqpoint{1.909813in}{4.254002in}}%
\pgfpathcurveto{\pgfqpoint{1.901999in}{4.246189in}}{\pgfqpoint{1.897609in}{4.235590in}}{\pgfqpoint{1.897609in}{4.224539in}}%
\pgfpathcurveto{\pgfqpoint{1.897609in}{4.213489in}}{\pgfqpoint{1.901999in}{4.202890in}}{\pgfqpoint{1.909813in}{4.195077in}}%
\pgfpathcurveto{\pgfqpoint{1.917626in}{4.187263in}}{\pgfqpoint{1.928225in}{4.182873in}}{\pgfqpoint{1.939275in}{4.182873in}}%
\pgfpathclose%
\pgfusepath{stroke,fill}%
\end{pgfscope}%
\begin{pgfscope}%
\pgfpathrectangle{\pgfqpoint{0.481978in}{0.331635in}}{\pgfqpoint{9.300000in}{7.700000in}}%
\pgfusepath{clip}%
\pgfsetbuttcap%
\pgfsetroundjoin%
\definecolor{currentfill}{rgb}{1.000000,0.705882,0.509804}%
\pgfsetfillcolor{currentfill}%
\pgfsetlinewidth{0.481800pt}%
\definecolor{currentstroke}{rgb}{1.000000,1.000000,1.000000}%
\pgfsetstrokecolor{currentstroke}%
\pgfsetdash{}{0pt}%
\pgfpathmoveto{\pgfqpoint{4.479280in}{2.869645in}}%
\pgfpathcurveto{\pgfqpoint{4.490330in}{2.869645in}}{\pgfqpoint{4.500929in}{2.874035in}}{\pgfqpoint{4.508743in}{2.881849in}}%
\pgfpathcurveto{\pgfqpoint{4.516557in}{2.889662in}}{\pgfqpoint{4.520947in}{2.900261in}}{\pgfqpoint{4.520947in}{2.911311in}}%
\pgfpathcurveto{\pgfqpoint{4.520947in}{2.922362in}}{\pgfqpoint{4.516557in}{2.932961in}}{\pgfqpoint{4.508743in}{2.940774in}}%
\pgfpathcurveto{\pgfqpoint{4.500929in}{2.948588in}}{\pgfqpoint{4.490330in}{2.952978in}}{\pgfqpoint{4.479280in}{2.952978in}}%
\pgfpathcurveto{\pgfqpoint{4.468230in}{2.952978in}}{\pgfqpoint{4.457631in}{2.948588in}}{\pgfqpoint{4.449817in}{2.940774in}}%
\pgfpathcurveto{\pgfqpoint{4.442004in}{2.932961in}}{\pgfqpoint{4.437614in}{2.922362in}}{\pgfqpoint{4.437614in}{2.911311in}}%
\pgfpathcurveto{\pgfqpoint{4.437614in}{2.900261in}}{\pgfqpoint{4.442004in}{2.889662in}}{\pgfqpoint{4.449817in}{2.881849in}}%
\pgfpathcurveto{\pgfqpoint{4.457631in}{2.874035in}}{\pgfqpoint{4.468230in}{2.869645in}}{\pgfqpoint{4.479280in}{2.869645in}}%
\pgfpathclose%
\pgfusepath{stroke,fill}%
\end{pgfscope}%
\begin{pgfscope}%
\pgfpathrectangle{\pgfqpoint{0.481978in}{0.331635in}}{\pgfqpoint{9.300000in}{7.700000in}}%
\pgfusepath{clip}%
\pgfsetbuttcap%
\pgfsetroundjoin%
\definecolor{currentfill}{rgb}{1.000000,0.705882,0.509804}%
\pgfsetfillcolor{currentfill}%
\pgfsetlinewidth{0.481800pt}%
\definecolor{currentstroke}{rgb}{1.000000,1.000000,1.000000}%
\pgfsetstrokecolor{currentstroke}%
\pgfsetdash{}{0pt}%
\pgfpathmoveto{\pgfqpoint{3.013348in}{2.602675in}}%
\pgfpathcurveto{\pgfqpoint{3.024398in}{2.602675in}}{\pgfqpoint{3.034997in}{2.607065in}}{\pgfqpoint{3.042811in}{2.614879in}}%
\pgfpathcurveto{\pgfqpoint{3.050624in}{2.622692in}}{\pgfqpoint{3.055014in}{2.633291in}}{\pgfqpoint{3.055014in}{2.644342in}}%
\pgfpathcurveto{\pgfqpoint{3.055014in}{2.655392in}}{\pgfqpoint{3.050624in}{2.665991in}}{\pgfqpoint{3.042811in}{2.673804in}}%
\pgfpathcurveto{\pgfqpoint{3.034997in}{2.681618in}}{\pgfqpoint{3.024398in}{2.686008in}}{\pgfqpoint{3.013348in}{2.686008in}}%
\pgfpathcurveto{\pgfqpoint{3.002298in}{2.686008in}}{\pgfqpoint{2.991699in}{2.681618in}}{\pgfqpoint{2.983885in}{2.673804in}}%
\pgfpathcurveto{\pgfqpoint{2.976071in}{2.665991in}}{\pgfqpoint{2.971681in}{2.655392in}}{\pgfqpoint{2.971681in}{2.644342in}}%
\pgfpathcurveto{\pgfqpoint{2.971681in}{2.633291in}}{\pgfqpoint{2.976071in}{2.622692in}}{\pgfqpoint{2.983885in}{2.614879in}}%
\pgfpathcurveto{\pgfqpoint{2.991699in}{2.607065in}}{\pgfqpoint{3.002298in}{2.602675in}}{\pgfqpoint{3.013348in}{2.602675in}}%
\pgfpathclose%
\pgfusepath{stroke,fill}%
\end{pgfscope}%
\begin{pgfscope}%
\pgfpathrectangle{\pgfqpoint{0.481978in}{0.331635in}}{\pgfqpoint{9.300000in}{7.700000in}}%
\pgfusepath{clip}%
\pgfsetbuttcap%
\pgfsetroundjoin%
\definecolor{currentfill}{rgb}{1.000000,0.705882,0.509804}%
\pgfsetfillcolor{currentfill}%
\pgfsetlinewidth{0.481800pt}%
\definecolor{currentstroke}{rgb}{1.000000,1.000000,1.000000}%
\pgfsetstrokecolor{currentstroke}%
\pgfsetdash{}{0pt}%
\pgfpathmoveto{\pgfqpoint{4.165514in}{6.486612in}}%
\pgfpathcurveto{\pgfqpoint{4.176565in}{6.486612in}}{\pgfqpoint{4.187164in}{6.491002in}}{\pgfqpoint{4.194977in}{6.498816in}}%
\pgfpathcurveto{\pgfqpoint{4.202791in}{6.506629in}}{\pgfqpoint{4.207181in}{6.517228in}}{\pgfqpoint{4.207181in}{6.528278in}}%
\pgfpathcurveto{\pgfqpoint{4.207181in}{6.539328in}}{\pgfqpoint{4.202791in}{6.549927in}}{\pgfqpoint{4.194977in}{6.557741in}}%
\pgfpathcurveto{\pgfqpoint{4.187164in}{6.565555in}}{\pgfqpoint{4.176565in}{6.569945in}}{\pgfqpoint{4.165514in}{6.569945in}}%
\pgfpathcurveto{\pgfqpoint{4.154464in}{6.569945in}}{\pgfqpoint{4.143865in}{6.565555in}}{\pgfqpoint{4.136052in}{6.557741in}}%
\pgfpathcurveto{\pgfqpoint{4.128238in}{6.549927in}}{\pgfqpoint{4.123848in}{6.539328in}}{\pgfqpoint{4.123848in}{6.528278in}}%
\pgfpathcurveto{\pgfqpoint{4.123848in}{6.517228in}}{\pgfqpoint{4.128238in}{6.506629in}}{\pgfqpoint{4.136052in}{6.498816in}}%
\pgfpathcurveto{\pgfqpoint{4.143865in}{6.491002in}}{\pgfqpoint{4.154464in}{6.486612in}}{\pgfqpoint{4.165514in}{6.486612in}}%
\pgfpathclose%
\pgfusepath{stroke,fill}%
\end{pgfscope}%
\begin{pgfscope}%
\pgfpathrectangle{\pgfqpoint{0.481978in}{0.331635in}}{\pgfqpoint{9.300000in}{7.700000in}}%
\pgfusepath{clip}%
\pgfsetbuttcap%
\pgfsetroundjoin%
\definecolor{currentfill}{rgb}{1.000000,0.705882,0.509804}%
\pgfsetfillcolor{currentfill}%
\pgfsetlinewidth{0.481800pt}%
\definecolor{currentstroke}{rgb}{1.000000,1.000000,1.000000}%
\pgfsetstrokecolor{currentstroke}%
\pgfsetdash{}{0pt}%
\pgfpathmoveto{\pgfqpoint{2.157737in}{4.110593in}}%
\pgfpathcurveto{\pgfqpoint{2.168787in}{4.110593in}}{\pgfqpoint{2.179387in}{4.114984in}}{\pgfqpoint{2.187200in}{4.122797in}}%
\pgfpathcurveto{\pgfqpoint{2.195014in}{4.130611in}}{\pgfqpoint{2.199404in}{4.141210in}}{\pgfqpoint{2.199404in}{4.152260in}}%
\pgfpathcurveto{\pgfqpoint{2.199404in}{4.163310in}}{\pgfqpoint{2.195014in}{4.173909in}}{\pgfqpoint{2.187200in}{4.181723in}}%
\pgfpathcurveto{\pgfqpoint{2.179387in}{4.189536in}}{\pgfqpoint{2.168787in}{4.193927in}}{\pgfqpoint{2.157737in}{4.193927in}}%
\pgfpathcurveto{\pgfqpoint{2.146687in}{4.193927in}}{\pgfqpoint{2.136088in}{4.189536in}}{\pgfqpoint{2.128275in}{4.181723in}}%
\pgfpathcurveto{\pgfqpoint{2.120461in}{4.173909in}}{\pgfqpoint{2.116071in}{4.163310in}}{\pgfqpoint{2.116071in}{4.152260in}}%
\pgfpathcurveto{\pgfqpoint{2.116071in}{4.141210in}}{\pgfqpoint{2.120461in}{4.130611in}}{\pgfqpoint{2.128275in}{4.122797in}}%
\pgfpathcurveto{\pgfqpoint{2.136088in}{4.114984in}}{\pgfqpoint{2.146687in}{4.110593in}}{\pgfqpoint{2.157737in}{4.110593in}}%
\pgfpathclose%
\pgfusepath{stroke,fill}%
\end{pgfscope}%
\begin{pgfscope}%
\pgfpathrectangle{\pgfqpoint{0.481978in}{0.331635in}}{\pgfqpoint{9.300000in}{7.700000in}}%
\pgfusepath{clip}%
\pgfsetbuttcap%
\pgfsetroundjoin%
\definecolor{currentfill}{rgb}{1.000000,0.705882,0.509804}%
\pgfsetfillcolor{currentfill}%
\pgfsetlinewidth{0.481800pt}%
\definecolor{currentstroke}{rgb}{1.000000,1.000000,1.000000}%
\pgfsetstrokecolor{currentstroke}%
\pgfsetdash{}{0pt}%
\pgfpathmoveto{\pgfqpoint{2.216691in}{5.366568in}}%
\pgfpathcurveto{\pgfqpoint{2.227741in}{5.366568in}}{\pgfqpoint{2.238340in}{5.370958in}}{\pgfqpoint{2.246154in}{5.378772in}}%
\pgfpathcurveto{\pgfqpoint{2.253968in}{5.386586in}}{\pgfqpoint{2.258358in}{5.397185in}}{\pgfqpoint{2.258358in}{5.408235in}}%
\pgfpathcurveto{\pgfqpoint{2.258358in}{5.419285in}}{\pgfqpoint{2.253968in}{5.429884in}}{\pgfqpoint{2.246154in}{5.437697in}}%
\pgfpathcurveto{\pgfqpoint{2.238340in}{5.445511in}}{\pgfqpoint{2.227741in}{5.449901in}}{\pgfqpoint{2.216691in}{5.449901in}}%
\pgfpathcurveto{\pgfqpoint{2.205641in}{5.449901in}}{\pgfqpoint{2.195042in}{5.445511in}}{\pgfqpoint{2.187228in}{5.437697in}}%
\pgfpathcurveto{\pgfqpoint{2.179415in}{5.429884in}}{\pgfqpoint{2.175025in}{5.419285in}}{\pgfqpoint{2.175025in}{5.408235in}}%
\pgfpathcurveto{\pgfqpoint{2.175025in}{5.397185in}}{\pgfqpoint{2.179415in}{5.386586in}}{\pgfqpoint{2.187228in}{5.378772in}}%
\pgfpathcurveto{\pgfqpoint{2.195042in}{5.370958in}}{\pgfqpoint{2.205641in}{5.366568in}}{\pgfqpoint{2.216691in}{5.366568in}}%
\pgfpathclose%
\pgfusepath{stroke,fill}%
\end{pgfscope}%
\begin{pgfscope}%
\pgfpathrectangle{\pgfqpoint{0.481978in}{0.331635in}}{\pgfqpoint{9.300000in}{7.700000in}}%
\pgfusepath{clip}%
\pgfsetbuttcap%
\pgfsetroundjoin%
\definecolor{currentfill}{rgb}{1.000000,0.705882,0.509804}%
\pgfsetfillcolor{currentfill}%
\pgfsetlinewidth{0.481800pt}%
\definecolor{currentstroke}{rgb}{1.000000,1.000000,1.000000}%
\pgfsetstrokecolor{currentstroke}%
\pgfsetdash{}{0pt}%
\pgfpathmoveto{\pgfqpoint{4.655686in}{2.205125in}}%
\pgfpathcurveto{\pgfqpoint{4.666736in}{2.205125in}}{\pgfqpoint{4.677335in}{2.209515in}}{\pgfqpoint{4.685149in}{2.217329in}}%
\pgfpathcurveto{\pgfqpoint{4.692962in}{2.225143in}}{\pgfqpoint{4.697353in}{2.235742in}}{\pgfqpoint{4.697353in}{2.246792in}}%
\pgfpathcurveto{\pgfqpoint{4.697353in}{2.257842in}}{\pgfqpoint{4.692962in}{2.268441in}}{\pgfqpoint{4.685149in}{2.276255in}}%
\pgfpathcurveto{\pgfqpoint{4.677335in}{2.284068in}}{\pgfqpoint{4.666736in}{2.288458in}}{\pgfqpoint{4.655686in}{2.288458in}}%
\pgfpathcurveto{\pgfqpoint{4.644636in}{2.288458in}}{\pgfqpoint{4.634037in}{2.284068in}}{\pgfqpoint{4.626223in}{2.276255in}}%
\pgfpathcurveto{\pgfqpoint{4.618410in}{2.268441in}}{\pgfqpoint{4.614019in}{2.257842in}}{\pgfqpoint{4.614019in}{2.246792in}}%
\pgfpathcurveto{\pgfqpoint{4.614019in}{2.235742in}}{\pgfqpoint{4.618410in}{2.225143in}}{\pgfqpoint{4.626223in}{2.217329in}}%
\pgfpathcurveto{\pgfqpoint{4.634037in}{2.209515in}}{\pgfqpoint{4.644636in}{2.205125in}}{\pgfqpoint{4.655686in}{2.205125in}}%
\pgfpathclose%
\pgfusepath{stroke,fill}%
\end{pgfscope}%
\begin{pgfscope}%
\pgfpathrectangle{\pgfqpoint{0.481978in}{0.331635in}}{\pgfqpoint{9.300000in}{7.700000in}}%
\pgfusepath{clip}%
\pgfsetbuttcap%
\pgfsetroundjoin%
\definecolor{currentfill}{rgb}{1.000000,0.705882,0.509804}%
\pgfsetfillcolor{currentfill}%
\pgfsetlinewidth{0.481800pt}%
\definecolor{currentstroke}{rgb}{1.000000,1.000000,1.000000}%
\pgfsetstrokecolor{currentstroke}%
\pgfsetdash{}{0pt}%
\pgfpathmoveto{\pgfqpoint{3.175062in}{3.307064in}}%
\pgfpathcurveto{\pgfqpoint{3.186112in}{3.307064in}}{\pgfqpoint{3.196711in}{3.311454in}}{\pgfqpoint{3.204525in}{3.319268in}}%
\pgfpathcurveto{\pgfqpoint{3.212339in}{3.327081in}}{\pgfqpoint{3.216729in}{3.337681in}}{\pgfqpoint{3.216729in}{3.348731in}}%
\pgfpathcurveto{\pgfqpoint{3.216729in}{3.359781in}}{\pgfqpoint{3.212339in}{3.370380in}}{\pgfqpoint{3.204525in}{3.378193in}}%
\pgfpathcurveto{\pgfqpoint{3.196711in}{3.386007in}}{\pgfqpoint{3.186112in}{3.390397in}}{\pgfqpoint{3.175062in}{3.390397in}}%
\pgfpathcurveto{\pgfqpoint{3.164012in}{3.390397in}}{\pgfqpoint{3.153413in}{3.386007in}}{\pgfqpoint{3.145599in}{3.378193in}}%
\pgfpathcurveto{\pgfqpoint{3.137786in}{3.370380in}}{\pgfqpoint{3.133396in}{3.359781in}}{\pgfqpoint{3.133396in}{3.348731in}}%
\pgfpathcurveto{\pgfqpoint{3.133396in}{3.337681in}}{\pgfqpoint{3.137786in}{3.327081in}}{\pgfqpoint{3.145599in}{3.319268in}}%
\pgfpathcurveto{\pgfqpoint{3.153413in}{3.311454in}}{\pgfqpoint{3.164012in}{3.307064in}}{\pgfqpoint{3.175062in}{3.307064in}}%
\pgfpathclose%
\pgfusepath{stroke,fill}%
\end{pgfscope}%
\begin{pgfscope}%
\pgfpathrectangle{\pgfqpoint{0.481978in}{0.331635in}}{\pgfqpoint{9.300000in}{7.700000in}}%
\pgfusepath{clip}%
\pgfsetbuttcap%
\pgfsetroundjoin%
\definecolor{currentfill}{rgb}{1.000000,0.705882,0.509804}%
\pgfsetfillcolor{currentfill}%
\pgfsetlinewidth{0.481800pt}%
\definecolor{currentstroke}{rgb}{1.000000,1.000000,1.000000}%
\pgfsetstrokecolor{currentstroke}%
\pgfsetdash{}{0pt}%
\pgfpathmoveto{\pgfqpoint{2.367140in}{4.674797in}}%
\pgfpathcurveto{\pgfqpoint{2.378190in}{4.674797in}}{\pgfqpoint{2.388789in}{4.679187in}}{\pgfqpoint{2.396603in}{4.687001in}}%
\pgfpathcurveto{\pgfqpoint{2.404417in}{4.694815in}}{\pgfqpoint{2.408807in}{4.705414in}}{\pgfqpoint{2.408807in}{4.716464in}}%
\pgfpathcurveto{\pgfqpoint{2.408807in}{4.727514in}}{\pgfqpoint{2.404417in}{4.738113in}}{\pgfqpoint{2.396603in}{4.745927in}}%
\pgfpathcurveto{\pgfqpoint{2.388789in}{4.753740in}}{\pgfqpoint{2.378190in}{4.758131in}}{\pgfqpoint{2.367140in}{4.758131in}}%
\pgfpathcurveto{\pgfqpoint{2.356090in}{4.758131in}}{\pgfqpoint{2.345491in}{4.753740in}}{\pgfqpoint{2.337678in}{4.745927in}}%
\pgfpathcurveto{\pgfqpoint{2.329864in}{4.738113in}}{\pgfqpoint{2.325474in}{4.727514in}}{\pgfqpoint{2.325474in}{4.716464in}}%
\pgfpathcurveto{\pgfqpoint{2.325474in}{4.705414in}}{\pgfqpoint{2.329864in}{4.694815in}}{\pgfqpoint{2.337678in}{4.687001in}}%
\pgfpathcurveto{\pgfqpoint{2.345491in}{4.679187in}}{\pgfqpoint{2.356090in}{4.674797in}}{\pgfqpoint{2.367140in}{4.674797in}}%
\pgfpathclose%
\pgfusepath{stroke,fill}%
\end{pgfscope}%
\begin{pgfscope}%
\pgfpathrectangle{\pgfqpoint{0.481978in}{0.331635in}}{\pgfqpoint{9.300000in}{7.700000in}}%
\pgfusepath{clip}%
\pgfsetbuttcap%
\pgfsetroundjoin%
\definecolor{currentfill}{rgb}{1.000000,0.705882,0.509804}%
\pgfsetfillcolor{currentfill}%
\pgfsetlinewidth{0.481800pt}%
\definecolor{currentstroke}{rgb}{1.000000,1.000000,1.000000}%
\pgfsetstrokecolor{currentstroke}%
\pgfsetdash{}{0pt}%
\pgfpathmoveto{\pgfqpoint{4.838472in}{2.994339in}}%
\pgfpathcurveto{\pgfqpoint{4.849522in}{2.994339in}}{\pgfqpoint{4.860121in}{2.998729in}}{\pgfqpoint{4.867935in}{3.006543in}}%
\pgfpathcurveto{\pgfqpoint{4.875748in}{3.014357in}}{\pgfqpoint{4.880139in}{3.024956in}}{\pgfqpoint{4.880139in}{3.036006in}}%
\pgfpathcurveto{\pgfqpoint{4.880139in}{3.047056in}}{\pgfqpoint{4.875748in}{3.057655in}}{\pgfqpoint{4.867935in}{3.065468in}}%
\pgfpathcurveto{\pgfqpoint{4.860121in}{3.073282in}}{\pgfqpoint{4.849522in}{3.077672in}}{\pgfqpoint{4.838472in}{3.077672in}}%
\pgfpathcurveto{\pgfqpoint{4.827422in}{3.077672in}}{\pgfqpoint{4.816823in}{3.073282in}}{\pgfqpoint{4.809009in}{3.065468in}}%
\pgfpathcurveto{\pgfqpoint{4.801196in}{3.057655in}}{\pgfqpoint{4.796805in}{3.047056in}}{\pgfqpoint{4.796805in}{3.036006in}}%
\pgfpathcurveto{\pgfqpoint{4.796805in}{3.024956in}}{\pgfqpoint{4.801196in}{3.014357in}}{\pgfqpoint{4.809009in}{3.006543in}}%
\pgfpathcurveto{\pgfqpoint{4.816823in}{2.998729in}}{\pgfqpoint{4.827422in}{2.994339in}}{\pgfqpoint{4.838472in}{2.994339in}}%
\pgfpathclose%
\pgfusepath{stroke,fill}%
\end{pgfscope}%
\begin{pgfscope}%
\pgfpathrectangle{\pgfqpoint{0.481978in}{0.331635in}}{\pgfqpoint{9.300000in}{7.700000in}}%
\pgfusepath{clip}%
\pgfsetbuttcap%
\pgfsetroundjoin%
\definecolor{currentfill}{rgb}{1.000000,0.705882,0.509804}%
\pgfsetfillcolor{currentfill}%
\pgfsetlinewidth{0.481800pt}%
\definecolor{currentstroke}{rgb}{1.000000,1.000000,1.000000}%
\pgfsetstrokecolor{currentstroke}%
\pgfsetdash{}{0pt}%
\pgfpathmoveto{\pgfqpoint{4.322216in}{5.005904in}}%
\pgfpathcurveto{\pgfqpoint{4.333266in}{5.005904in}}{\pgfqpoint{4.343866in}{5.010294in}}{\pgfqpoint{4.351679in}{5.018108in}}%
\pgfpathcurveto{\pgfqpoint{4.359493in}{5.025921in}}{\pgfqpoint{4.363883in}{5.036520in}}{\pgfqpoint{4.363883in}{5.047571in}}%
\pgfpathcurveto{\pgfqpoint{4.363883in}{5.058621in}}{\pgfqpoint{4.359493in}{5.069220in}}{\pgfqpoint{4.351679in}{5.077033in}}%
\pgfpathcurveto{\pgfqpoint{4.343866in}{5.084847in}}{\pgfqpoint{4.333266in}{5.089237in}}{\pgfqpoint{4.322216in}{5.089237in}}%
\pgfpathcurveto{\pgfqpoint{4.311166in}{5.089237in}}{\pgfqpoint{4.300567in}{5.084847in}}{\pgfqpoint{4.292754in}{5.077033in}}%
\pgfpathcurveto{\pgfqpoint{4.284940in}{5.069220in}}{\pgfqpoint{4.280550in}{5.058621in}}{\pgfqpoint{4.280550in}{5.047571in}}%
\pgfpathcurveto{\pgfqpoint{4.280550in}{5.036520in}}{\pgfqpoint{4.284940in}{5.025921in}}{\pgfqpoint{4.292754in}{5.018108in}}%
\pgfpathcurveto{\pgfqpoint{4.300567in}{5.010294in}}{\pgfqpoint{4.311166in}{5.005904in}}{\pgfqpoint{4.322216in}{5.005904in}}%
\pgfpathclose%
\pgfusepath{stroke,fill}%
\end{pgfscope}%
\begin{pgfscope}%
\pgfpathrectangle{\pgfqpoint{0.481978in}{0.331635in}}{\pgfqpoint{9.300000in}{7.700000in}}%
\pgfusepath{clip}%
\pgfsetbuttcap%
\pgfsetroundjoin%
\definecolor{currentfill}{rgb}{1.000000,0.705882,0.509804}%
\pgfsetfillcolor{currentfill}%
\pgfsetlinewidth{0.481800pt}%
\definecolor{currentstroke}{rgb}{1.000000,1.000000,1.000000}%
\pgfsetstrokecolor{currentstroke}%
\pgfsetdash{}{0pt}%
\pgfpathmoveto{\pgfqpoint{3.615208in}{3.422279in}}%
\pgfpathcurveto{\pgfqpoint{3.626258in}{3.422279in}}{\pgfqpoint{3.636857in}{3.426669in}}{\pgfqpoint{3.644670in}{3.434483in}}%
\pgfpathcurveto{\pgfqpoint{3.652484in}{3.442296in}}{\pgfqpoint{3.656874in}{3.452895in}}{\pgfqpoint{3.656874in}{3.463945in}}%
\pgfpathcurveto{\pgfqpoint{3.656874in}{3.474995in}}{\pgfqpoint{3.652484in}{3.485595in}}{\pgfqpoint{3.644670in}{3.493408in}}%
\pgfpathcurveto{\pgfqpoint{3.636857in}{3.501222in}}{\pgfqpoint{3.626258in}{3.505612in}}{\pgfqpoint{3.615208in}{3.505612in}}%
\pgfpathcurveto{\pgfqpoint{3.604158in}{3.505612in}}{\pgfqpoint{3.593559in}{3.501222in}}{\pgfqpoint{3.585745in}{3.493408in}}%
\pgfpathcurveto{\pgfqpoint{3.577931in}{3.485595in}}{\pgfqpoint{3.573541in}{3.474995in}}{\pgfqpoint{3.573541in}{3.463945in}}%
\pgfpathcurveto{\pgfqpoint{3.573541in}{3.452895in}}{\pgfqpoint{3.577931in}{3.442296in}}{\pgfqpoint{3.585745in}{3.434483in}}%
\pgfpathcurveto{\pgfqpoint{3.593559in}{3.426669in}}{\pgfqpoint{3.604158in}{3.422279in}}{\pgfqpoint{3.615208in}{3.422279in}}%
\pgfpathclose%
\pgfusepath{stroke,fill}%
\end{pgfscope}%
\begin{pgfscope}%
\pgfpathrectangle{\pgfqpoint{0.481978in}{0.331635in}}{\pgfqpoint{9.300000in}{7.700000in}}%
\pgfusepath{clip}%
\pgfsetbuttcap%
\pgfsetroundjoin%
\definecolor{currentfill}{rgb}{1.000000,0.705882,0.509804}%
\pgfsetfillcolor{currentfill}%
\pgfsetlinewidth{0.481800pt}%
\definecolor{currentstroke}{rgb}{1.000000,1.000000,1.000000}%
\pgfsetstrokecolor{currentstroke}%
\pgfsetdash{}{0pt}%
\pgfpathmoveto{\pgfqpoint{3.424608in}{6.905454in}}%
\pgfpathcurveto{\pgfqpoint{3.435658in}{6.905454in}}{\pgfqpoint{3.446257in}{6.909844in}}{\pgfqpoint{3.454071in}{6.917658in}}%
\pgfpathcurveto{\pgfqpoint{3.461884in}{6.925471in}}{\pgfqpoint{3.466274in}{6.936070in}}{\pgfqpoint{3.466274in}{6.947120in}}%
\pgfpathcurveto{\pgfqpoint{3.466274in}{6.958171in}}{\pgfqpoint{3.461884in}{6.968770in}}{\pgfqpoint{3.454071in}{6.976583in}}%
\pgfpathcurveto{\pgfqpoint{3.446257in}{6.984397in}}{\pgfqpoint{3.435658in}{6.988787in}}{\pgfqpoint{3.424608in}{6.988787in}}%
\pgfpathcurveto{\pgfqpoint{3.413558in}{6.988787in}}{\pgfqpoint{3.402959in}{6.984397in}}{\pgfqpoint{3.395145in}{6.976583in}}%
\pgfpathcurveto{\pgfqpoint{3.387331in}{6.968770in}}{\pgfqpoint{3.382941in}{6.958171in}}{\pgfqpoint{3.382941in}{6.947120in}}%
\pgfpathcurveto{\pgfqpoint{3.382941in}{6.936070in}}{\pgfqpoint{3.387331in}{6.925471in}}{\pgfqpoint{3.395145in}{6.917658in}}%
\pgfpathcurveto{\pgfqpoint{3.402959in}{6.909844in}}{\pgfqpoint{3.413558in}{6.905454in}}{\pgfqpoint{3.424608in}{6.905454in}}%
\pgfpathclose%
\pgfusepath{stroke,fill}%
\end{pgfscope}%
\begin{pgfscope}%
\pgfpathrectangle{\pgfqpoint{0.481978in}{0.331635in}}{\pgfqpoint{9.300000in}{7.700000in}}%
\pgfusepath{clip}%
\pgfsetbuttcap%
\pgfsetroundjoin%
\definecolor{currentfill}{rgb}{1.000000,0.705882,0.509804}%
\pgfsetfillcolor{currentfill}%
\pgfsetlinewidth{0.481800pt}%
\definecolor{currentstroke}{rgb}{1.000000,1.000000,1.000000}%
\pgfsetstrokecolor{currentstroke}%
\pgfsetdash{}{0pt}%
\pgfpathmoveto{\pgfqpoint{5.585402in}{5.198457in}}%
\pgfpathcurveto{\pgfqpoint{5.596452in}{5.198457in}}{\pgfqpoint{5.607051in}{5.202848in}}{\pgfqpoint{5.614865in}{5.210661in}}%
\pgfpathcurveto{\pgfqpoint{5.622678in}{5.218475in}}{\pgfqpoint{5.627069in}{5.229074in}}{\pgfqpoint{5.627069in}{5.240124in}}%
\pgfpathcurveto{\pgfqpoint{5.627069in}{5.251174in}}{\pgfqpoint{5.622678in}{5.261773in}}{\pgfqpoint{5.614865in}{5.269587in}}%
\pgfpathcurveto{\pgfqpoint{5.607051in}{5.277400in}}{\pgfqpoint{5.596452in}{5.281791in}}{\pgfqpoint{5.585402in}{5.281791in}}%
\pgfpathcurveto{\pgfqpoint{5.574352in}{5.281791in}}{\pgfqpoint{5.563753in}{5.277400in}}{\pgfqpoint{5.555939in}{5.269587in}}%
\pgfpathcurveto{\pgfqpoint{5.548125in}{5.261773in}}{\pgfqpoint{5.543735in}{5.251174in}}{\pgfqpoint{5.543735in}{5.240124in}}%
\pgfpathcurveto{\pgfqpoint{5.543735in}{5.229074in}}{\pgfqpoint{5.548125in}{5.218475in}}{\pgfqpoint{5.555939in}{5.210661in}}%
\pgfpathcurveto{\pgfqpoint{5.563753in}{5.202848in}}{\pgfqpoint{5.574352in}{5.198457in}}{\pgfqpoint{5.585402in}{5.198457in}}%
\pgfpathclose%
\pgfusepath{stroke,fill}%
\end{pgfscope}%
\begin{pgfscope}%
\pgfpathrectangle{\pgfqpoint{0.481978in}{0.331635in}}{\pgfqpoint{9.300000in}{7.700000in}}%
\pgfusepath{clip}%
\pgfsetbuttcap%
\pgfsetroundjoin%
\definecolor{currentfill}{rgb}{1.000000,0.705882,0.509804}%
\pgfsetfillcolor{currentfill}%
\pgfsetlinewidth{0.481800pt}%
\definecolor{currentstroke}{rgb}{1.000000,1.000000,1.000000}%
\pgfsetstrokecolor{currentstroke}%
\pgfsetdash{}{0pt}%
\pgfpathmoveto{\pgfqpoint{3.153342in}{5.068040in}}%
\pgfpathcurveto{\pgfqpoint{3.164392in}{5.068040in}}{\pgfqpoint{3.174991in}{5.072431in}}{\pgfqpoint{3.182805in}{5.080244in}}%
\pgfpathcurveto{\pgfqpoint{3.190618in}{5.088058in}}{\pgfqpoint{3.195008in}{5.098657in}}{\pgfqpoint{3.195008in}{5.109707in}}%
\pgfpathcurveto{\pgfqpoint{3.195008in}{5.120757in}}{\pgfqpoint{3.190618in}{5.131356in}}{\pgfqpoint{3.182805in}{5.139170in}}%
\pgfpathcurveto{\pgfqpoint{3.174991in}{5.146983in}}{\pgfqpoint{3.164392in}{5.151374in}}{\pgfqpoint{3.153342in}{5.151374in}}%
\pgfpathcurveto{\pgfqpoint{3.142292in}{5.151374in}}{\pgfqpoint{3.131693in}{5.146983in}}{\pgfqpoint{3.123879in}{5.139170in}}%
\pgfpathcurveto{\pgfqpoint{3.116065in}{5.131356in}}{\pgfqpoint{3.111675in}{5.120757in}}{\pgfqpoint{3.111675in}{5.109707in}}%
\pgfpathcurveto{\pgfqpoint{3.111675in}{5.098657in}}{\pgfqpoint{3.116065in}{5.088058in}}{\pgfqpoint{3.123879in}{5.080244in}}%
\pgfpathcurveto{\pgfqpoint{3.131693in}{5.072431in}}{\pgfqpoint{3.142292in}{5.068040in}}{\pgfqpoint{3.153342in}{5.068040in}}%
\pgfpathclose%
\pgfusepath{stroke,fill}%
\end{pgfscope}%
\begin{pgfscope}%
\pgfpathrectangle{\pgfqpoint{0.481978in}{0.331635in}}{\pgfqpoint{9.300000in}{7.700000in}}%
\pgfusepath{clip}%
\pgfsetbuttcap%
\pgfsetroundjoin%
\definecolor{currentfill}{rgb}{1.000000,0.705882,0.509804}%
\pgfsetfillcolor{currentfill}%
\pgfsetlinewidth{0.481800pt}%
\definecolor{currentstroke}{rgb}{1.000000,1.000000,1.000000}%
\pgfsetstrokecolor{currentstroke}%
\pgfsetdash{}{0pt}%
\pgfpathmoveto{\pgfqpoint{3.297548in}{5.440827in}}%
\pgfpathcurveto{\pgfqpoint{3.308598in}{5.440827in}}{\pgfqpoint{3.319197in}{5.445218in}}{\pgfqpoint{3.327011in}{5.453031in}}%
\pgfpathcurveto{\pgfqpoint{3.334825in}{5.460845in}}{\pgfqpoint{3.339215in}{5.471444in}}{\pgfqpoint{3.339215in}{5.482494in}}%
\pgfpathcurveto{\pgfqpoint{3.339215in}{5.493544in}}{\pgfqpoint{3.334825in}{5.504143in}}{\pgfqpoint{3.327011in}{5.511957in}}%
\pgfpathcurveto{\pgfqpoint{3.319197in}{5.519770in}}{\pgfqpoint{3.308598in}{5.524161in}}{\pgfqpoint{3.297548in}{5.524161in}}%
\pgfpathcurveto{\pgfqpoint{3.286498in}{5.524161in}}{\pgfqpoint{3.275899in}{5.519770in}}{\pgfqpoint{3.268086in}{5.511957in}}%
\pgfpathcurveto{\pgfqpoint{3.260272in}{5.504143in}}{\pgfqpoint{3.255882in}{5.493544in}}{\pgfqpoint{3.255882in}{5.482494in}}%
\pgfpathcurveto{\pgfqpoint{3.255882in}{5.471444in}}{\pgfqpoint{3.260272in}{5.460845in}}{\pgfqpoint{3.268086in}{5.453031in}}%
\pgfpathcurveto{\pgfqpoint{3.275899in}{5.445218in}}{\pgfqpoint{3.286498in}{5.440827in}}{\pgfqpoint{3.297548in}{5.440827in}}%
\pgfpathclose%
\pgfusepath{stroke,fill}%
\end{pgfscope}%
\begin{pgfscope}%
\pgfpathrectangle{\pgfqpoint{0.481978in}{0.331635in}}{\pgfqpoint{9.300000in}{7.700000in}}%
\pgfusepath{clip}%
\pgfsetbuttcap%
\pgfsetroundjoin%
\definecolor{currentfill}{rgb}{1.000000,0.705882,0.509804}%
\pgfsetfillcolor{currentfill}%
\pgfsetlinewidth{0.481800pt}%
\definecolor{currentstroke}{rgb}{1.000000,1.000000,1.000000}%
\pgfsetstrokecolor{currentstroke}%
\pgfsetdash{}{0pt}%
\pgfpathmoveto{\pgfqpoint{4.669793in}{2.217934in}}%
\pgfpathcurveto{\pgfqpoint{4.680844in}{2.217934in}}{\pgfqpoint{4.691443in}{2.222324in}}{\pgfqpoint{4.699256in}{2.230138in}}%
\pgfpathcurveto{\pgfqpoint{4.707070in}{2.237951in}}{\pgfqpoint{4.711460in}{2.248550in}}{\pgfqpoint{4.711460in}{2.259600in}}%
\pgfpathcurveto{\pgfqpoint{4.711460in}{2.270650in}}{\pgfqpoint{4.707070in}{2.281250in}}{\pgfqpoint{4.699256in}{2.289063in}}%
\pgfpathcurveto{\pgfqpoint{4.691443in}{2.296877in}}{\pgfqpoint{4.680844in}{2.301267in}}{\pgfqpoint{4.669793in}{2.301267in}}%
\pgfpathcurveto{\pgfqpoint{4.658743in}{2.301267in}}{\pgfqpoint{4.648144in}{2.296877in}}{\pgfqpoint{4.640331in}{2.289063in}}%
\pgfpathcurveto{\pgfqpoint{4.632517in}{2.281250in}}{\pgfqpoint{4.628127in}{2.270650in}}{\pgfqpoint{4.628127in}{2.259600in}}%
\pgfpathcurveto{\pgfqpoint{4.628127in}{2.248550in}}{\pgfqpoint{4.632517in}{2.237951in}}{\pgfqpoint{4.640331in}{2.230138in}}%
\pgfpathcurveto{\pgfqpoint{4.648144in}{2.222324in}}{\pgfqpoint{4.658743in}{2.217934in}}{\pgfqpoint{4.669793in}{2.217934in}}%
\pgfpathclose%
\pgfusepath{stroke,fill}%
\end{pgfscope}%
\begin{pgfscope}%
\pgfpathrectangle{\pgfqpoint{0.481978in}{0.331635in}}{\pgfqpoint{9.300000in}{7.700000in}}%
\pgfusepath{clip}%
\pgfsetbuttcap%
\pgfsetroundjoin%
\definecolor{currentfill}{rgb}{1.000000,0.705882,0.509804}%
\pgfsetfillcolor{currentfill}%
\pgfsetlinewidth{0.481800pt}%
\definecolor{currentstroke}{rgb}{1.000000,1.000000,1.000000}%
\pgfsetstrokecolor{currentstroke}%
\pgfsetdash{}{0pt}%
\pgfpathmoveto{\pgfqpoint{1.546741in}{5.103666in}}%
\pgfpathcurveto{\pgfqpoint{1.557791in}{5.103666in}}{\pgfqpoint{1.568390in}{5.108056in}}{\pgfqpoint{1.576204in}{5.115870in}}%
\pgfpathcurveto{\pgfqpoint{1.584017in}{5.123683in}}{\pgfqpoint{1.588408in}{5.134282in}}{\pgfqpoint{1.588408in}{5.145333in}}%
\pgfpathcurveto{\pgfqpoint{1.588408in}{5.156383in}}{\pgfqpoint{1.584017in}{5.166982in}}{\pgfqpoint{1.576204in}{5.174795in}}%
\pgfpathcurveto{\pgfqpoint{1.568390in}{5.182609in}}{\pgfqpoint{1.557791in}{5.186999in}}{\pgfqpoint{1.546741in}{5.186999in}}%
\pgfpathcurveto{\pgfqpoint{1.535691in}{5.186999in}}{\pgfqpoint{1.525092in}{5.182609in}}{\pgfqpoint{1.517278in}{5.174795in}}%
\pgfpathcurveto{\pgfqpoint{1.509465in}{5.166982in}}{\pgfqpoint{1.505074in}{5.156383in}}{\pgfqpoint{1.505074in}{5.145333in}}%
\pgfpathcurveto{\pgfqpoint{1.505074in}{5.134282in}}{\pgfqpoint{1.509465in}{5.123683in}}{\pgfqpoint{1.517278in}{5.115870in}}%
\pgfpathcurveto{\pgfqpoint{1.525092in}{5.108056in}}{\pgfqpoint{1.535691in}{5.103666in}}{\pgfqpoint{1.546741in}{5.103666in}}%
\pgfpathclose%
\pgfusepath{stroke,fill}%
\end{pgfscope}%
\begin{pgfscope}%
\pgfpathrectangle{\pgfqpoint{0.481978in}{0.331635in}}{\pgfqpoint{9.300000in}{7.700000in}}%
\pgfusepath{clip}%
\pgfsetbuttcap%
\pgfsetroundjoin%
\definecolor{currentfill}{rgb}{1.000000,0.705882,0.509804}%
\pgfsetfillcolor{currentfill}%
\pgfsetlinewidth{0.481800pt}%
\definecolor{currentstroke}{rgb}{1.000000,1.000000,1.000000}%
\pgfsetstrokecolor{currentstroke}%
\pgfsetdash{}{0pt}%
\pgfpathmoveto{\pgfqpoint{4.426621in}{5.353124in}}%
\pgfpathcurveto{\pgfqpoint{4.437671in}{5.353124in}}{\pgfqpoint{4.448270in}{5.357514in}}{\pgfqpoint{4.456084in}{5.365328in}}%
\pgfpathcurveto{\pgfqpoint{4.463897in}{5.373142in}}{\pgfqpoint{4.468287in}{5.383741in}}{\pgfqpoint{4.468287in}{5.394791in}}%
\pgfpathcurveto{\pgfqpoint{4.468287in}{5.405841in}}{\pgfqpoint{4.463897in}{5.416440in}}{\pgfqpoint{4.456084in}{5.424254in}}%
\pgfpathcurveto{\pgfqpoint{4.448270in}{5.432067in}}{\pgfqpoint{4.437671in}{5.436458in}}{\pgfqpoint{4.426621in}{5.436458in}}%
\pgfpathcurveto{\pgfqpoint{4.415571in}{5.436458in}}{\pgfqpoint{4.404972in}{5.432067in}}{\pgfqpoint{4.397158in}{5.424254in}}%
\pgfpathcurveto{\pgfqpoint{4.389344in}{5.416440in}}{\pgfqpoint{4.384954in}{5.405841in}}{\pgfqpoint{4.384954in}{5.394791in}}%
\pgfpathcurveto{\pgfqpoint{4.384954in}{5.383741in}}{\pgfqpoint{4.389344in}{5.373142in}}{\pgfqpoint{4.397158in}{5.365328in}}%
\pgfpathcurveto{\pgfqpoint{4.404972in}{5.357514in}}{\pgfqpoint{4.415571in}{5.353124in}}{\pgfqpoint{4.426621in}{5.353124in}}%
\pgfpathclose%
\pgfusepath{stroke,fill}%
\end{pgfscope}%
\begin{pgfscope}%
\pgfpathrectangle{\pgfqpoint{0.481978in}{0.331635in}}{\pgfqpoint{9.300000in}{7.700000in}}%
\pgfusepath{clip}%
\pgfsetbuttcap%
\pgfsetroundjoin%
\definecolor{currentfill}{rgb}{1.000000,0.705882,0.509804}%
\pgfsetfillcolor{currentfill}%
\pgfsetlinewidth{0.481800pt}%
\definecolor{currentstroke}{rgb}{1.000000,1.000000,1.000000}%
\pgfsetstrokecolor{currentstroke}%
\pgfsetdash{}{0pt}%
\pgfpathmoveto{\pgfqpoint{2.946822in}{5.311992in}}%
\pgfpathcurveto{\pgfqpoint{2.957872in}{5.311992in}}{\pgfqpoint{2.968471in}{5.316382in}}{\pgfqpoint{2.976285in}{5.324196in}}%
\pgfpathcurveto{\pgfqpoint{2.984098in}{5.332009in}}{\pgfqpoint{2.988489in}{5.342608in}}{\pgfqpoint{2.988489in}{5.353658in}}%
\pgfpathcurveto{\pgfqpoint{2.988489in}{5.364709in}}{\pgfqpoint{2.984098in}{5.375308in}}{\pgfqpoint{2.976285in}{5.383121in}}%
\pgfpathcurveto{\pgfqpoint{2.968471in}{5.390935in}}{\pgfqpoint{2.957872in}{5.395325in}}{\pgfqpoint{2.946822in}{5.395325in}}%
\pgfpathcurveto{\pgfqpoint{2.935772in}{5.395325in}}{\pgfqpoint{2.925173in}{5.390935in}}{\pgfqpoint{2.917359in}{5.383121in}}%
\pgfpathcurveto{\pgfqpoint{2.909546in}{5.375308in}}{\pgfqpoint{2.905155in}{5.364709in}}{\pgfqpoint{2.905155in}{5.353658in}}%
\pgfpathcurveto{\pgfqpoint{2.905155in}{5.342608in}}{\pgfqpoint{2.909546in}{5.332009in}}{\pgfqpoint{2.917359in}{5.324196in}}%
\pgfpathcurveto{\pgfqpoint{2.925173in}{5.316382in}}{\pgfqpoint{2.935772in}{5.311992in}}{\pgfqpoint{2.946822in}{5.311992in}}%
\pgfpathclose%
\pgfusepath{stroke,fill}%
\end{pgfscope}%
\begin{pgfscope}%
\pgfpathrectangle{\pgfqpoint{0.481978in}{0.331635in}}{\pgfqpoint{9.300000in}{7.700000in}}%
\pgfusepath{clip}%
\pgfsetbuttcap%
\pgfsetroundjoin%
\definecolor{currentfill}{rgb}{1.000000,0.705882,0.509804}%
\pgfsetfillcolor{currentfill}%
\pgfsetlinewidth{0.481800pt}%
\definecolor{currentstroke}{rgb}{1.000000,1.000000,1.000000}%
\pgfsetstrokecolor{currentstroke}%
\pgfsetdash{}{0pt}%
\pgfpathmoveto{\pgfqpoint{9.359251in}{1.323480in}}%
\pgfpathcurveto{\pgfqpoint{9.370301in}{1.323480in}}{\pgfqpoint{9.380900in}{1.327870in}}{\pgfqpoint{9.388713in}{1.335684in}}%
\pgfpathcurveto{\pgfqpoint{9.396527in}{1.343497in}}{\pgfqpoint{9.400917in}{1.354096in}}{\pgfqpoint{9.400917in}{1.365147in}}%
\pgfpathcurveto{\pgfqpoint{9.400917in}{1.376197in}}{\pgfqpoint{9.396527in}{1.386796in}}{\pgfqpoint{9.388713in}{1.394609in}}%
\pgfpathcurveto{\pgfqpoint{9.380900in}{1.402423in}}{\pgfqpoint{9.370301in}{1.406813in}}{\pgfqpoint{9.359251in}{1.406813in}}%
\pgfpathcurveto{\pgfqpoint{9.348201in}{1.406813in}}{\pgfqpoint{9.337601in}{1.402423in}}{\pgfqpoint{9.329788in}{1.394609in}}%
\pgfpathcurveto{\pgfqpoint{9.321974in}{1.386796in}}{\pgfqpoint{9.317584in}{1.376197in}}{\pgfqpoint{9.317584in}{1.365147in}}%
\pgfpathcurveto{\pgfqpoint{9.317584in}{1.354096in}}{\pgfqpoint{9.321974in}{1.343497in}}{\pgfqpoint{9.329788in}{1.335684in}}%
\pgfpathcurveto{\pgfqpoint{9.337601in}{1.327870in}}{\pgfqpoint{9.348201in}{1.323480in}}{\pgfqpoint{9.359251in}{1.323480in}}%
\pgfpathclose%
\pgfusepath{stroke,fill}%
\end{pgfscope}%
\begin{pgfscope}%
\pgfpathrectangle{\pgfqpoint{0.481978in}{0.331635in}}{\pgfqpoint{9.300000in}{7.700000in}}%
\pgfusepath{clip}%
\pgfsetbuttcap%
\pgfsetroundjoin%
\definecolor{currentfill}{rgb}{1.000000,0.705882,0.509804}%
\pgfsetfillcolor{currentfill}%
\pgfsetlinewidth{0.481800pt}%
\definecolor{currentstroke}{rgb}{1.000000,1.000000,1.000000}%
\pgfsetstrokecolor{currentstroke}%
\pgfsetdash{}{0pt}%
\pgfpathmoveto{\pgfqpoint{3.311383in}{5.655787in}}%
\pgfpathcurveto{\pgfqpoint{3.322433in}{5.655787in}}{\pgfqpoint{3.333032in}{5.660177in}}{\pgfqpoint{3.340846in}{5.667990in}}%
\pgfpathcurveto{\pgfqpoint{3.348659in}{5.675804in}}{\pgfqpoint{3.353049in}{5.686403in}}{\pgfqpoint{3.353049in}{5.697453in}}%
\pgfpathcurveto{\pgfqpoint{3.353049in}{5.708503in}}{\pgfqpoint{3.348659in}{5.719102in}}{\pgfqpoint{3.340846in}{5.726916in}}%
\pgfpathcurveto{\pgfqpoint{3.333032in}{5.734730in}}{\pgfqpoint{3.322433in}{5.739120in}}{\pgfqpoint{3.311383in}{5.739120in}}%
\pgfpathcurveto{\pgfqpoint{3.300333in}{5.739120in}}{\pgfqpoint{3.289734in}{5.734730in}}{\pgfqpoint{3.281920in}{5.726916in}}%
\pgfpathcurveto{\pgfqpoint{3.274106in}{5.719102in}}{\pgfqpoint{3.269716in}{5.708503in}}{\pgfqpoint{3.269716in}{5.697453in}}%
\pgfpathcurveto{\pgfqpoint{3.269716in}{5.686403in}}{\pgfqpoint{3.274106in}{5.675804in}}{\pgfqpoint{3.281920in}{5.667990in}}%
\pgfpathcurveto{\pgfqpoint{3.289734in}{5.660177in}}{\pgfqpoint{3.300333in}{5.655787in}}{\pgfqpoint{3.311383in}{5.655787in}}%
\pgfpathclose%
\pgfusepath{stroke,fill}%
\end{pgfscope}%
\begin{pgfscope}%
\pgfpathrectangle{\pgfqpoint{0.481978in}{0.331635in}}{\pgfqpoint{9.300000in}{7.700000in}}%
\pgfusepath{clip}%
\pgfsetbuttcap%
\pgfsetroundjoin%
\definecolor{currentfill}{rgb}{1.000000,0.705882,0.509804}%
\pgfsetfillcolor{currentfill}%
\pgfsetlinewidth{0.481800pt}%
\definecolor{currentstroke}{rgb}{1.000000,1.000000,1.000000}%
\pgfsetstrokecolor{currentstroke}%
\pgfsetdash{}{0pt}%
\pgfpathmoveto{\pgfqpoint{3.768020in}{3.816422in}}%
\pgfpathcurveto{\pgfqpoint{3.779070in}{3.816422in}}{\pgfqpoint{3.789669in}{3.820813in}}{\pgfqpoint{3.797483in}{3.828626in}}%
\pgfpathcurveto{\pgfqpoint{3.805296in}{3.836440in}}{\pgfqpoint{3.809687in}{3.847039in}}{\pgfqpoint{3.809687in}{3.858089in}}%
\pgfpathcurveto{\pgfqpoint{3.809687in}{3.869139in}}{\pgfqpoint{3.805296in}{3.879738in}}{\pgfqpoint{3.797483in}{3.887552in}}%
\pgfpathcurveto{\pgfqpoint{3.789669in}{3.895365in}}{\pgfqpoint{3.779070in}{3.899756in}}{\pgfqpoint{3.768020in}{3.899756in}}%
\pgfpathcurveto{\pgfqpoint{3.756970in}{3.899756in}}{\pgfqpoint{3.746371in}{3.895365in}}{\pgfqpoint{3.738557in}{3.887552in}}%
\pgfpathcurveto{\pgfqpoint{3.730744in}{3.879738in}}{\pgfqpoint{3.726353in}{3.869139in}}{\pgfqpoint{3.726353in}{3.858089in}}%
\pgfpathcurveto{\pgfqpoint{3.726353in}{3.847039in}}{\pgfqpoint{3.730744in}{3.836440in}}{\pgfqpoint{3.738557in}{3.828626in}}%
\pgfpathcurveto{\pgfqpoint{3.746371in}{3.820813in}}{\pgfqpoint{3.756970in}{3.816422in}}{\pgfqpoint{3.768020in}{3.816422in}}%
\pgfpathclose%
\pgfusepath{stroke,fill}%
\end{pgfscope}%
\begin{pgfscope}%
\pgfpathrectangle{\pgfqpoint{0.481978in}{0.331635in}}{\pgfqpoint{9.300000in}{7.700000in}}%
\pgfusepath{clip}%
\pgfsetbuttcap%
\pgfsetroundjoin%
\definecolor{currentfill}{rgb}{1.000000,0.705882,0.509804}%
\pgfsetfillcolor{currentfill}%
\pgfsetlinewidth{0.481800pt}%
\definecolor{currentstroke}{rgb}{1.000000,1.000000,1.000000}%
\pgfsetstrokecolor{currentstroke}%
\pgfsetdash{}{0pt}%
\pgfpathmoveto{\pgfqpoint{3.827532in}{5.079630in}}%
\pgfpathcurveto{\pgfqpoint{3.838582in}{5.079630in}}{\pgfqpoint{3.849181in}{5.084020in}}{\pgfqpoint{3.856995in}{5.091834in}}%
\pgfpathcurveto{\pgfqpoint{3.864808in}{5.099647in}}{\pgfqpoint{3.869199in}{5.110246in}}{\pgfqpoint{3.869199in}{5.121296in}}%
\pgfpathcurveto{\pgfqpoint{3.869199in}{5.132346in}}{\pgfqpoint{3.864808in}{5.142945in}}{\pgfqpoint{3.856995in}{5.150759in}}%
\pgfpathcurveto{\pgfqpoint{3.849181in}{5.158573in}}{\pgfqpoint{3.838582in}{5.162963in}}{\pgfqpoint{3.827532in}{5.162963in}}%
\pgfpathcurveto{\pgfqpoint{3.816482in}{5.162963in}}{\pgfqpoint{3.805883in}{5.158573in}}{\pgfqpoint{3.798069in}{5.150759in}}%
\pgfpathcurveto{\pgfqpoint{3.790255in}{5.142945in}}{\pgfqpoint{3.785865in}{5.132346in}}{\pgfqpoint{3.785865in}{5.121296in}}%
\pgfpathcurveto{\pgfqpoint{3.785865in}{5.110246in}}{\pgfqpoint{3.790255in}{5.099647in}}{\pgfqpoint{3.798069in}{5.091834in}}%
\pgfpathcurveto{\pgfqpoint{3.805883in}{5.084020in}}{\pgfqpoint{3.816482in}{5.079630in}}{\pgfqpoint{3.827532in}{5.079630in}}%
\pgfpathclose%
\pgfusepath{stroke,fill}%
\end{pgfscope}%
\begin{pgfscope}%
\pgfpathrectangle{\pgfqpoint{0.481978in}{0.331635in}}{\pgfqpoint{9.300000in}{7.700000in}}%
\pgfusepath{clip}%
\pgfsetbuttcap%
\pgfsetroundjoin%
\definecolor{currentfill}{rgb}{1.000000,0.705882,0.509804}%
\pgfsetfillcolor{currentfill}%
\pgfsetlinewidth{0.481800pt}%
\definecolor{currentstroke}{rgb}{1.000000,1.000000,1.000000}%
\pgfsetstrokecolor{currentstroke}%
\pgfsetdash{}{0pt}%
\pgfpathmoveto{\pgfqpoint{1.059884in}{2.150352in}}%
\pgfpathcurveto{\pgfqpoint{1.070934in}{2.150352in}}{\pgfqpoint{1.081533in}{2.154743in}}{\pgfqpoint{1.089346in}{2.162556in}}%
\pgfpathcurveto{\pgfqpoint{1.097160in}{2.170370in}}{\pgfqpoint{1.101550in}{2.180969in}}{\pgfqpoint{1.101550in}{2.192019in}}%
\pgfpathcurveto{\pgfqpoint{1.101550in}{2.203069in}}{\pgfqpoint{1.097160in}{2.213668in}}{\pgfqpoint{1.089346in}{2.221482in}}%
\pgfpathcurveto{\pgfqpoint{1.081533in}{2.229295in}}{\pgfqpoint{1.070934in}{2.233686in}}{\pgfqpoint{1.059884in}{2.233686in}}%
\pgfpathcurveto{\pgfqpoint{1.048833in}{2.233686in}}{\pgfqpoint{1.038234in}{2.229295in}}{\pgfqpoint{1.030421in}{2.221482in}}%
\pgfpathcurveto{\pgfqpoint{1.022607in}{2.213668in}}{\pgfqpoint{1.018217in}{2.203069in}}{\pgfqpoint{1.018217in}{2.192019in}}%
\pgfpathcurveto{\pgfqpoint{1.018217in}{2.180969in}}{\pgfqpoint{1.022607in}{2.170370in}}{\pgfqpoint{1.030421in}{2.162556in}}%
\pgfpathcurveto{\pgfqpoint{1.038234in}{2.154743in}}{\pgfqpoint{1.048833in}{2.150352in}}{\pgfqpoint{1.059884in}{2.150352in}}%
\pgfpathclose%
\pgfusepath{stroke,fill}%
\end{pgfscope}%
\begin{pgfscope}%
\pgfpathrectangle{\pgfqpoint{0.481978in}{0.331635in}}{\pgfqpoint{9.300000in}{7.700000in}}%
\pgfusepath{clip}%
\pgfsetbuttcap%
\pgfsetroundjoin%
\definecolor{currentfill}{rgb}{1.000000,0.705882,0.509804}%
\pgfsetfillcolor{currentfill}%
\pgfsetlinewidth{0.481800pt}%
\definecolor{currentstroke}{rgb}{1.000000,1.000000,1.000000}%
\pgfsetstrokecolor{currentstroke}%
\pgfsetdash{}{0pt}%
\pgfpathmoveto{\pgfqpoint{4.209058in}{3.207223in}}%
\pgfpathcurveto{\pgfqpoint{4.220108in}{3.207223in}}{\pgfqpoint{4.230707in}{3.211614in}}{\pgfqpoint{4.238521in}{3.219427in}}%
\pgfpathcurveto{\pgfqpoint{4.246334in}{3.227241in}}{\pgfqpoint{4.250725in}{3.237840in}}{\pgfqpoint{4.250725in}{3.248890in}}%
\pgfpathcurveto{\pgfqpoint{4.250725in}{3.259940in}}{\pgfqpoint{4.246334in}{3.270539in}}{\pgfqpoint{4.238521in}{3.278353in}}%
\pgfpathcurveto{\pgfqpoint{4.230707in}{3.286166in}}{\pgfqpoint{4.220108in}{3.290557in}}{\pgfqpoint{4.209058in}{3.290557in}}%
\pgfpathcurveto{\pgfqpoint{4.198008in}{3.290557in}}{\pgfqpoint{4.187409in}{3.286166in}}{\pgfqpoint{4.179595in}{3.278353in}}%
\pgfpathcurveto{\pgfqpoint{4.171781in}{3.270539in}}{\pgfqpoint{4.167391in}{3.259940in}}{\pgfqpoint{4.167391in}{3.248890in}}%
\pgfpathcurveto{\pgfqpoint{4.167391in}{3.237840in}}{\pgfqpoint{4.171781in}{3.227241in}}{\pgfqpoint{4.179595in}{3.219427in}}%
\pgfpathcurveto{\pgfqpoint{4.187409in}{3.211614in}}{\pgfqpoint{4.198008in}{3.207223in}}{\pgfqpoint{4.209058in}{3.207223in}}%
\pgfpathclose%
\pgfusepath{stroke,fill}%
\end{pgfscope}%
\begin{pgfscope}%
\pgfpathrectangle{\pgfqpoint{0.481978in}{0.331635in}}{\pgfqpoint{9.300000in}{7.700000in}}%
\pgfusepath{clip}%
\pgfsetbuttcap%
\pgfsetroundjoin%
\definecolor{currentfill}{rgb}{1.000000,0.705882,0.509804}%
\pgfsetfillcolor{currentfill}%
\pgfsetlinewidth{0.481800pt}%
\definecolor{currentstroke}{rgb}{1.000000,1.000000,1.000000}%
\pgfsetstrokecolor{currentstroke}%
\pgfsetdash{}{0pt}%
\pgfpathmoveto{\pgfqpoint{3.072069in}{2.744087in}}%
\pgfpathcurveto{\pgfqpoint{3.083120in}{2.744087in}}{\pgfqpoint{3.093719in}{2.748477in}}{\pgfqpoint{3.101532in}{2.756291in}}%
\pgfpathcurveto{\pgfqpoint{3.109346in}{2.764104in}}{\pgfqpoint{3.113736in}{2.774703in}}{\pgfqpoint{3.113736in}{2.785754in}}%
\pgfpathcurveto{\pgfqpoint{3.113736in}{2.796804in}}{\pgfqpoint{3.109346in}{2.807403in}}{\pgfqpoint{3.101532in}{2.815216in}}%
\pgfpathcurveto{\pgfqpoint{3.093719in}{2.823030in}}{\pgfqpoint{3.083120in}{2.827420in}}{\pgfqpoint{3.072069in}{2.827420in}}%
\pgfpathcurveto{\pgfqpoint{3.061019in}{2.827420in}}{\pgfqpoint{3.050420in}{2.823030in}}{\pgfqpoint{3.042607in}{2.815216in}}%
\pgfpathcurveto{\pgfqpoint{3.034793in}{2.807403in}}{\pgfqpoint{3.030403in}{2.796804in}}{\pgfqpoint{3.030403in}{2.785754in}}%
\pgfpathcurveto{\pgfqpoint{3.030403in}{2.774703in}}{\pgfqpoint{3.034793in}{2.764104in}}{\pgfqpoint{3.042607in}{2.756291in}}%
\pgfpathcurveto{\pgfqpoint{3.050420in}{2.748477in}}{\pgfqpoint{3.061019in}{2.744087in}}{\pgfqpoint{3.072069in}{2.744087in}}%
\pgfpathclose%
\pgfusepath{stroke,fill}%
\end{pgfscope}%
\begin{pgfscope}%
\pgfpathrectangle{\pgfqpoint{0.481978in}{0.331635in}}{\pgfqpoint{9.300000in}{7.700000in}}%
\pgfusepath{clip}%
\pgfsetbuttcap%
\pgfsetroundjoin%
\definecolor{currentfill}{rgb}{1.000000,0.705882,0.509804}%
\pgfsetfillcolor{currentfill}%
\pgfsetlinewidth{0.481800pt}%
\definecolor{currentstroke}{rgb}{1.000000,1.000000,1.000000}%
\pgfsetstrokecolor{currentstroke}%
\pgfsetdash{}{0pt}%
\pgfpathmoveto{\pgfqpoint{4.167601in}{3.771028in}}%
\pgfpathcurveto{\pgfqpoint{4.178652in}{3.771028in}}{\pgfqpoint{4.189251in}{3.775418in}}{\pgfqpoint{4.197064in}{3.783232in}}%
\pgfpathcurveto{\pgfqpoint{4.204878in}{3.791046in}}{\pgfqpoint{4.209268in}{3.801645in}}{\pgfqpoint{4.209268in}{3.812695in}}%
\pgfpathcurveto{\pgfqpoint{4.209268in}{3.823745in}}{\pgfqpoint{4.204878in}{3.834344in}}{\pgfqpoint{4.197064in}{3.842158in}}%
\pgfpathcurveto{\pgfqpoint{4.189251in}{3.849971in}}{\pgfqpoint{4.178652in}{3.854361in}}{\pgfqpoint{4.167601in}{3.854361in}}%
\pgfpathcurveto{\pgfqpoint{4.156551in}{3.854361in}}{\pgfqpoint{4.145952in}{3.849971in}}{\pgfqpoint{4.138139in}{3.842158in}}%
\pgfpathcurveto{\pgfqpoint{4.130325in}{3.834344in}}{\pgfqpoint{4.125935in}{3.823745in}}{\pgfqpoint{4.125935in}{3.812695in}}%
\pgfpathcurveto{\pgfqpoint{4.125935in}{3.801645in}}{\pgfqpoint{4.130325in}{3.791046in}}{\pgfqpoint{4.138139in}{3.783232in}}%
\pgfpathcurveto{\pgfqpoint{4.145952in}{3.775418in}}{\pgfqpoint{4.156551in}{3.771028in}}{\pgfqpoint{4.167601in}{3.771028in}}%
\pgfpathclose%
\pgfusepath{stroke,fill}%
\end{pgfscope}%
\begin{pgfscope}%
\pgfpathrectangle{\pgfqpoint{0.481978in}{0.331635in}}{\pgfqpoint{9.300000in}{7.700000in}}%
\pgfusepath{clip}%
\pgfsetbuttcap%
\pgfsetroundjoin%
\definecolor{currentfill}{rgb}{1.000000,0.705882,0.509804}%
\pgfsetfillcolor{currentfill}%
\pgfsetlinewidth{0.481800pt}%
\definecolor{currentstroke}{rgb}{1.000000,1.000000,1.000000}%
\pgfsetstrokecolor{currentstroke}%
\pgfsetdash{}{0pt}%
\pgfpathmoveto{\pgfqpoint{4.458251in}{2.508756in}}%
\pgfpathcurveto{\pgfqpoint{4.469301in}{2.508756in}}{\pgfqpoint{4.479900in}{2.513146in}}{\pgfqpoint{4.487713in}{2.520960in}}%
\pgfpathcurveto{\pgfqpoint{4.495527in}{2.528774in}}{\pgfqpoint{4.499917in}{2.539373in}}{\pgfqpoint{4.499917in}{2.550423in}}%
\pgfpathcurveto{\pgfqpoint{4.499917in}{2.561473in}}{\pgfqpoint{4.495527in}{2.572072in}}{\pgfqpoint{4.487713in}{2.579886in}}%
\pgfpathcurveto{\pgfqpoint{4.479900in}{2.587699in}}{\pgfqpoint{4.469301in}{2.592090in}}{\pgfqpoint{4.458251in}{2.592090in}}%
\pgfpathcurveto{\pgfqpoint{4.447201in}{2.592090in}}{\pgfqpoint{4.436602in}{2.587699in}}{\pgfqpoint{4.428788in}{2.579886in}}%
\pgfpathcurveto{\pgfqpoint{4.420974in}{2.572072in}}{\pgfqpoint{4.416584in}{2.561473in}}{\pgfqpoint{4.416584in}{2.550423in}}%
\pgfpathcurveto{\pgfqpoint{4.416584in}{2.539373in}}{\pgfqpoint{4.420974in}{2.528774in}}{\pgfqpoint{4.428788in}{2.520960in}}%
\pgfpathcurveto{\pgfqpoint{4.436602in}{2.513146in}}{\pgfqpoint{4.447201in}{2.508756in}}{\pgfqpoint{4.458251in}{2.508756in}}%
\pgfpathclose%
\pgfusepath{stroke,fill}%
\end{pgfscope}%
\begin{pgfscope}%
\pgfpathrectangle{\pgfqpoint{0.481978in}{0.331635in}}{\pgfqpoint{9.300000in}{7.700000in}}%
\pgfusepath{clip}%
\pgfsetbuttcap%
\pgfsetroundjoin%
\definecolor{currentfill}{rgb}{1.000000,0.705882,0.509804}%
\pgfsetfillcolor{currentfill}%
\pgfsetlinewidth{0.481800pt}%
\definecolor{currentstroke}{rgb}{1.000000,1.000000,1.000000}%
\pgfsetstrokecolor{currentstroke}%
\pgfsetdash{}{0pt}%
\pgfpathmoveto{\pgfqpoint{3.705775in}{4.566842in}}%
\pgfpathcurveto{\pgfqpoint{3.716825in}{4.566842in}}{\pgfqpoint{3.727424in}{4.571233in}}{\pgfqpoint{3.735238in}{4.579046in}}%
\pgfpathcurveto{\pgfqpoint{3.743052in}{4.586860in}}{\pgfqpoint{3.747442in}{4.597459in}}{\pgfqpoint{3.747442in}{4.608509in}}%
\pgfpathcurveto{\pgfqpoint{3.747442in}{4.619559in}}{\pgfqpoint{3.743052in}{4.630158in}}{\pgfqpoint{3.735238in}{4.637972in}}%
\pgfpathcurveto{\pgfqpoint{3.727424in}{4.645785in}}{\pgfqpoint{3.716825in}{4.650176in}}{\pgfqpoint{3.705775in}{4.650176in}}%
\pgfpathcurveto{\pgfqpoint{3.694725in}{4.650176in}}{\pgfqpoint{3.684126in}{4.645785in}}{\pgfqpoint{3.676313in}{4.637972in}}%
\pgfpathcurveto{\pgfqpoint{3.668499in}{4.630158in}}{\pgfqpoint{3.664109in}{4.619559in}}{\pgfqpoint{3.664109in}{4.608509in}}%
\pgfpathcurveto{\pgfqpoint{3.664109in}{4.597459in}}{\pgfqpoint{3.668499in}{4.586860in}}{\pgfqpoint{3.676313in}{4.579046in}}%
\pgfpathcurveto{\pgfqpoint{3.684126in}{4.571233in}}{\pgfqpoint{3.694725in}{4.566842in}}{\pgfqpoint{3.705775in}{4.566842in}}%
\pgfpathclose%
\pgfusepath{stroke,fill}%
\end{pgfscope}%
\begin{pgfscope}%
\pgfpathrectangle{\pgfqpoint{0.481978in}{0.331635in}}{\pgfqpoint{9.300000in}{7.700000in}}%
\pgfusepath{clip}%
\pgfsetbuttcap%
\pgfsetroundjoin%
\definecolor{currentfill}{rgb}{1.000000,0.705882,0.509804}%
\pgfsetfillcolor{currentfill}%
\pgfsetlinewidth{0.481800pt}%
\definecolor{currentstroke}{rgb}{1.000000,1.000000,1.000000}%
\pgfsetstrokecolor{currentstroke}%
\pgfsetdash{}{0pt}%
\pgfpathmoveto{\pgfqpoint{3.840708in}{2.812768in}}%
\pgfpathcurveto{\pgfqpoint{3.851758in}{2.812768in}}{\pgfqpoint{3.862357in}{2.817158in}}{\pgfqpoint{3.870171in}{2.824972in}}%
\pgfpathcurveto{\pgfqpoint{3.877985in}{2.832785in}}{\pgfqpoint{3.882375in}{2.843385in}}{\pgfqpoint{3.882375in}{2.854435in}}%
\pgfpathcurveto{\pgfqpoint{3.882375in}{2.865485in}}{\pgfqpoint{3.877985in}{2.876084in}}{\pgfqpoint{3.870171in}{2.883897in}}%
\pgfpathcurveto{\pgfqpoint{3.862357in}{2.891711in}}{\pgfqpoint{3.851758in}{2.896101in}}{\pgfqpoint{3.840708in}{2.896101in}}%
\pgfpathcurveto{\pgfqpoint{3.829658in}{2.896101in}}{\pgfqpoint{3.819059in}{2.891711in}}{\pgfqpoint{3.811245in}{2.883897in}}%
\pgfpathcurveto{\pgfqpoint{3.803432in}{2.876084in}}{\pgfqpoint{3.799042in}{2.865485in}}{\pgfqpoint{3.799042in}{2.854435in}}%
\pgfpathcurveto{\pgfqpoint{3.799042in}{2.843385in}}{\pgfqpoint{3.803432in}{2.832785in}}{\pgfqpoint{3.811245in}{2.824972in}}%
\pgfpathcurveto{\pgfqpoint{3.819059in}{2.817158in}}{\pgfqpoint{3.829658in}{2.812768in}}{\pgfqpoint{3.840708in}{2.812768in}}%
\pgfpathclose%
\pgfusepath{stroke,fill}%
\end{pgfscope}%
\begin{pgfscope}%
\pgfpathrectangle{\pgfqpoint{0.481978in}{0.331635in}}{\pgfqpoint{9.300000in}{7.700000in}}%
\pgfusepath{clip}%
\pgfsetbuttcap%
\pgfsetroundjoin%
\definecolor{currentfill}{rgb}{1.000000,0.705882,0.509804}%
\pgfsetfillcolor{currentfill}%
\pgfsetlinewidth{0.481800pt}%
\definecolor{currentstroke}{rgb}{1.000000,1.000000,1.000000}%
\pgfsetstrokecolor{currentstroke}%
\pgfsetdash{}{0pt}%
\pgfpathmoveto{\pgfqpoint{2.841012in}{4.126690in}}%
\pgfpathcurveto{\pgfqpoint{2.852062in}{4.126690in}}{\pgfqpoint{2.862661in}{4.131080in}}{\pgfqpoint{2.870475in}{4.138894in}}%
\pgfpathcurveto{\pgfqpoint{2.878288in}{4.146708in}}{\pgfqpoint{2.882679in}{4.157307in}}{\pgfqpoint{2.882679in}{4.168357in}}%
\pgfpathcurveto{\pgfqpoint{2.882679in}{4.179407in}}{\pgfqpoint{2.878288in}{4.190006in}}{\pgfqpoint{2.870475in}{4.197819in}}%
\pgfpathcurveto{\pgfqpoint{2.862661in}{4.205633in}}{\pgfqpoint{2.852062in}{4.210023in}}{\pgfqpoint{2.841012in}{4.210023in}}%
\pgfpathcurveto{\pgfqpoint{2.829962in}{4.210023in}}{\pgfqpoint{2.819363in}{4.205633in}}{\pgfqpoint{2.811549in}{4.197819in}}%
\pgfpathcurveto{\pgfqpoint{2.803736in}{4.190006in}}{\pgfqpoint{2.799345in}{4.179407in}}{\pgfqpoint{2.799345in}{4.168357in}}%
\pgfpathcurveto{\pgfqpoint{2.799345in}{4.157307in}}{\pgfqpoint{2.803736in}{4.146708in}}{\pgfqpoint{2.811549in}{4.138894in}}%
\pgfpathcurveto{\pgfqpoint{2.819363in}{4.131080in}}{\pgfqpoint{2.829962in}{4.126690in}}{\pgfqpoint{2.841012in}{4.126690in}}%
\pgfpathclose%
\pgfusepath{stroke,fill}%
\end{pgfscope}%
\begin{pgfscope}%
\pgfpathrectangle{\pgfqpoint{0.481978in}{0.331635in}}{\pgfqpoint{9.300000in}{7.700000in}}%
\pgfusepath{clip}%
\pgfsetbuttcap%
\pgfsetroundjoin%
\definecolor{currentfill}{rgb}{1.000000,0.705882,0.509804}%
\pgfsetfillcolor{currentfill}%
\pgfsetlinewidth{0.481800pt}%
\definecolor{currentstroke}{rgb}{1.000000,1.000000,1.000000}%
\pgfsetstrokecolor{currentstroke}%
\pgfsetdash{}{0pt}%
\pgfpathmoveto{\pgfqpoint{2.668476in}{1.607002in}}%
\pgfpathcurveto{\pgfqpoint{2.679526in}{1.607002in}}{\pgfqpoint{2.690125in}{1.611392in}}{\pgfqpoint{2.697939in}{1.619206in}}%
\pgfpathcurveto{\pgfqpoint{2.705752in}{1.627020in}}{\pgfqpoint{2.710143in}{1.637619in}}{\pgfqpoint{2.710143in}{1.648669in}}%
\pgfpathcurveto{\pgfqpoint{2.710143in}{1.659719in}}{\pgfqpoint{2.705752in}{1.670318in}}{\pgfqpoint{2.697939in}{1.678131in}}%
\pgfpathcurveto{\pgfqpoint{2.690125in}{1.685945in}}{\pgfqpoint{2.679526in}{1.690335in}}{\pgfqpoint{2.668476in}{1.690335in}}%
\pgfpathcurveto{\pgfqpoint{2.657426in}{1.690335in}}{\pgfqpoint{2.646827in}{1.685945in}}{\pgfqpoint{2.639013in}{1.678131in}}%
\pgfpathcurveto{\pgfqpoint{2.631199in}{1.670318in}}{\pgfqpoint{2.626809in}{1.659719in}}{\pgfqpoint{2.626809in}{1.648669in}}%
\pgfpathcurveto{\pgfqpoint{2.626809in}{1.637619in}}{\pgfqpoint{2.631199in}{1.627020in}}{\pgfqpoint{2.639013in}{1.619206in}}%
\pgfpathcurveto{\pgfqpoint{2.646827in}{1.611392in}}{\pgfqpoint{2.657426in}{1.607002in}}{\pgfqpoint{2.668476in}{1.607002in}}%
\pgfpathclose%
\pgfusepath{stroke,fill}%
\end{pgfscope}%
\begin{pgfscope}%
\pgfpathrectangle{\pgfqpoint{0.481978in}{0.331635in}}{\pgfqpoint{9.300000in}{7.700000in}}%
\pgfusepath{clip}%
\pgfsetbuttcap%
\pgfsetroundjoin%
\definecolor{currentfill}{rgb}{1.000000,0.705882,0.509804}%
\pgfsetfillcolor{currentfill}%
\pgfsetlinewidth{0.481800pt}%
\definecolor{currentstroke}{rgb}{1.000000,1.000000,1.000000}%
\pgfsetstrokecolor{currentstroke}%
\pgfsetdash{}{0pt}%
\pgfpathmoveto{\pgfqpoint{3.108258in}{5.387808in}}%
\pgfpathcurveto{\pgfqpoint{3.119308in}{5.387808in}}{\pgfqpoint{3.129907in}{5.392198in}}{\pgfqpoint{3.137720in}{5.400012in}}%
\pgfpathcurveto{\pgfqpoint{3.145534in}{5.407826in}}{\pgfqpoint{3.149924in}{5.418425in}}{\pgfqpoint{3.149924in}{5.429475in}}%
\pgfpathcurveto{\pgfqpoint{3.149924in}{5.440525in}}{\pgfqpoint{3.145534in}{5.451124in}}{\pgfqpoint{3.137720in}{5.458938in}}%
\pgfpathcurveto{\pgfqpoint{3.129907in}{5.466751in}}{\pgfqpoint{3.119308in}{5.471142in}}{\pgfqpoint{3.108258in}{5.471142in}}%
\pgfpathcurveto{\pgfqpoint{3.097207in}{5.471142in}}{\pgfqpoint{3.086608in}{5.466751in}}{\pgfqpoint{3.078795in}{5.458938in}}%
\pgfpathcurveto{\pgfqpoint{3.070981in}{5.451124in}}{\pgfqpoint{3.066591in}{5.440525in}}{\pgfqpoint{3.066591in}{5.429475in}}%
\pgfpathcurveto{\pgfqpoint{3.066591in}{5.418425in}}{\pgfqpoint{3.070981in}{5.407826in}}{\pgfqpoint{3.078795in}{5.400012in}}%
\pgfpathcurveto{\pgfqpoint{3.086608in}{5.392198in}}{\pgfqpoint{3.097207in}{5.387808in}}{\pgfqpoint{3.108258in}{5.387808in}}%
\pgfpathclose%
\pgfusepath{stroke,fill}%
\end{pgfscope}%
\begin{pgfscope}%
\pgfpathrectangle{\pgfqpoint{0.481978in}{0.331635in}}{\pgfqpoint{9.300000in}{7.700000in}}%
\pgfusepath{clip}%
\pgfsetbuttcap%
\pgfsetroundjoin%
\definecolor{currentfill}{rgb}{1.000000,0.705882,0.509804}%
\pgfsetfillcolor{currentfill}%
\pgfsetlinewidth{0.481800pt}%
\definecolor{currentstroke}{rgb}{1.000000,1.000000,1.000000}%
\pgfsetstrokecolor{currentstroke}%
\pgfsetdash{}{0pt}%
\pgfpathmoveto{\pgfqpoint{1.529852in}{4.755458in}}%
\pgfpathcurveto{\pgfqpoint{1.540902in}{4.755458in}}{\pgfqpoint{1.551501in}{4.759848in}}{\pgfqpoint{1.559315in}{4.767662in}}%
\pgfpathcurveto{\pgfqpoint{1.567128in}{4.775476in}}{\pgfqpoint{1.571519in}{4.786075in}}{\pgfqpoint{1.571519in}{4.797125in}}%
\pgfpathcurveto{\pgfqpoint{1.571519in}{4.808175in}}{\pgfqpoint{1.567128in}{4.818774in}}{\pgfqpoint{1.559315in}{4.826588in}}%
\pgfpathcurveto{\pgfqpoint{1.551501in}{4.834401in}}{\pgfqpoint{1.540902in}{4.838791in}}{\pgfqpoint{1.529852in}{4.838791in}}%
\pgfpathcurveto{\pgfqpoint{1.518802in}{4.838791in}}{\pgfqpoint{1.508203in}{4.834401in}}{\pgfqpoint{1.500389in}{4.826588in}}%
\pgfpathcurveto{\pgfqpoint{1.492576in}{4.818774in}}{\pgfqpoint{1.488185in}{4.808175in}}{\pgfqpoint{1.488185in}{4.797125in}}%
\pgfpathcurveto{\pgfqpoint{1.488185in}{4.786075in}}{\pgfqpoint{1.492576in}{4.775476in}}{\pgfqpoint{1.500389in}{4.767662in}}%
\pgfpathcurveto{\pgfqpoint{1.508203in}{4.759848in}}{\pgfqpoint{1.518802in}{4.755458in}}{\pgfqpoint{1.529852in}{4.755458in}}%
\pgfpathclose%
\pgfusepath{stroke,fill}%
\end{pgfscope}%
\begin{pgfscope}%
\pgfpathrectangle{\pgfqpoint{0.481978in}{0.331635in}}{\pgfqpoint{9.300000in}{7.700000in}}%
\pgfusepath{clip}%
\pgfsetbuttcap%
\pgfsetroundjoin%
\definecolor{currentfill}{rgb}{1.000000,0.705882,0.509804}%
\pgfsetfillcolor{currentfill}%
\pgfsetlinewidth{0.481800pt}%
\definecolor{currentstroke}{rgb}{1.000000,1.000000,1.000000}%
\pgfsetstrokecolor{currentstroke}%
\pgfsetdash{}{0pt}%
\pgfpathmoveto{\pgfqpoint{4.765650in}{4.912183in}}%
\pgfpathcurveto{\pgfqpoint{4.776700in}{4.912183in}}{\pgfqpoint{4.787299in}{4.916573in}}{\pgfqpoint{4.795113in}{4.924387in}}%
\pgfpathcurveto{\pgfqpoint{4.802926in}{4.932201in}}{\pgfqpoint{4.807316in}{4.942800in}}{\pgfqpoint{4.807316in}{4.953850in}}%
\pgfpathcurveto{\pgfqpoint{4.807316in}{4.964900in}}{\pgfqpoint{4.802926in}{4.975499in}}{\pgfqpoint{4.795113in}{4.983313in}}%
\pgfpathcurveto{\pgfqpoint{4.787299in}{4.991126in}}{\pgfqpoint{4.776700in}{4.995516in}}{\pgfqpoint{4.765650in}{4.995516in}}%
\pgfpathcurveto{\pgfqpoint{4.754600in}{4.995516in}}{\pgfqpoint{4.744001in}{4.991126in}}{\pgfqpoint{4.736187in}{4.983313in}}%
\pgfpathcurveto{\pgfqpoint{4.728373in}{4.975499in}}{\pgfqpoint{4.723983in}{4.964900in}}{\pgfqpoint{4.723983in}{4.953850in}}%
\pgfpathcurveto{\pgfqpoint{4.723983in}{4.942800in}}{\pgfqpoint{4.728373in}{4.932201in}}{\pgfqpoint{4.736187in}{4.924387in}}%
\pgfpathcurveto{\pgfqpoint{4.744001in}{4.916573in}}{\pgfqpoint{4.754600in}{4.912183in}}{\pgfqpoint{4.765650in}{4.912183in}}%
\pgfpathclose%
\pgfusepath{stroke,fill}%
\end{pgfscope}%
\begin{pgfscope}%
\pgfpathrectangle{\pgfqpoint{0.481978in}{0.331635in}}{\pgfqpoint{9.300000in}{7.700000in}}%
\pgfusepath{clip}%
\pgfsetbuttcap%
\pgfsetroundjoin%
\definecolor{currentfill}{rgb}{1.000000,0.705882,0.509804}%
\pgfsetfillcolor{currentfill}%
\pgfsetlinewidth{0.481800pt}%
\definecolor{currentstroke}{rgb}{1.000000,1.000000,1.000000}%
\pgfsetstrokecolor{currentstroke}%
\pgfsetdash{}{0pt}%
\pgfpathmoveto{\pgfqpoint{1.272867in}{4.296402in}}%
\pgfpathcurveto{\pgfqpoint{1.283917in}{4.296402in}}{\pgfqpoint{1.294516in}{4.300793in}}{\pgfqpoint{1.302330in}{4.308606in}}%
\pgfpathcurveto{\pgfqpoint{1.310144in}{4.316420in}}{\pgfqpoint{1.314534in}{4.327019in}}{\pgfqpoint{1.314534in}{4.338069in}}%
\pgfpathcurveto{\pgfqpoint{1.314534in}{4.349119in}}{\pgfqpoint{1.310144in}{4.359718in}}{\pgfqpoint{1.302330in}{4.367532in}}%
\pgfpathcurveto{\pgfqpoint{1.294516in}{4.375346in}}{\pgfqpoint{1.283917in}{4.379736in}}{\pgfqpoint{1.272867in}{4.379736in}}%
\pgfpathcurveto{\pgfqpoint{1.261817in}{4.379736in}}{\pgfqpoint{1.251218in}{4.375346in}}{\pgfqpoint{1.243405in}{4.367532in}}%
\pgfpathcurveto{\pgfqpoint{1.235591in}{4.359718in}}{\pgfqpoint{1.231201in}{4.349119in}}{\pgfqpoint{1.231201in}{4.338069in}}%
\pgfpathcurveto{\pgfqpoint{1.231201in}{4.327019in}}{\pgfqpoint{1.235591in}{4.316420in}}{\pgfqpoint{1.243405in}{4.308606in}}%
\pgfpathcurveto{\pgfqpoint{1.251218in}{4.300793in}}{\pgfqpoint{1.261817in}{4.296402in}}{\pgfqpoint{1.272867in}{4.296402in}}%
\pgfpathclose%
\pgfusepath{stroke,fill}%
\end{pgfscope}%
\begin{pgfscope}%
\pgfpathrectangle{\pgfqpoint{0.481978in}{0.331635in}}{\pgfqpoint{9.300000in}{7.700000in}}%
\pgfusepath{clip}%
\pgfsetbuttcap%
\pgfsetroundjoin%
\definecolor{currentfill}{rgb}{1.000000,0.705882,0.509804}%
\pgfsetfillcolor{currentfill}%
\pgfsetlinewidth{0.481800pt}%
\definecolor{currentstroke}{rgb}{1.000000,1.000000,1.000000}%
\pgfsetstrokecolor{currentstroke}%
\pgfsetdash{}{0pt}%
\pgfpathmoveto{\pgfqpoint{3.381281in}{4.075176in}}%
\pgfpathcurveto{\pgfqpoint{3.392331in}{4.075176in}}{\pgfqpoint{3.402930in}{4.079567in}}{\pgfqpoint{3.410743in}{4.087380in}}%
\pgfpathcurveto{\pgfqpoint{3.418557in}{4.095194in}}{\pgfqpoint{3.422947in}{4.105793in}}{\pgfqpoint{3.422947in}{4.116843in}}%
\pgfpathcurveto{\pgfqpoint{3.422947in}{4.127893in}}{\pgfqpoint{3.418557in}{4.138492in}}{\pgfqpoint{3.410743in}{4.146306in}}%
\pgfpathcurveto{\pgfqpoint{3.402930in}{4.154119in}}{\pgfqpoint{3.392331in}{4.158510in}}{\pgfqpoint{3.381281in}{4.158510in}}%
\pgfpathcurveto{\pgfqpoint{3.370230in}{4.158510in}}{\pgfqpoint{3.359631in}{4.154119in}}{\pgfqpoint{3.351818in}{4.146306in}}%
\pgfpathcurveto{\pgfqpoint{3.344004in}{4.138492in}}{\pgfqpoint{3.339614in}{4.127893in}}{\pgfqpoint{3.339614in}{4.116843in}}%
\pgfpathcurveto{\pgfqpoint{3.339614in}{4.105793in}}{\pgfqpoint{3.344004in}{4.095194in}}{\pgfqpoint{3.351818in}{4.087380in}}%
\pgfpathcurveto{\pgfqpoint{3.359631in}{4.079567in}}{\pgfqpoint{3.370230in}{4.075176in}}{\pgfqpoint{3.381281in}{4.075176in}}%
\pgfpathclose%
\pgfusepath{stroke,fill}%
\end{pgfscope}%
\begin{pgfscope}%
\pgfpathrectangle{\pgfqpoint{0.481978in}{0.331635in}}{\pgfqpoint{9.300000in}{7.700000in}}%
\pgfusepath{clip}%
\pgfsetbuttcap%
\pgfsetroundjoin%
\definecolor{currentfill}{rgb}{1.000000,0.705882,0.509804}%
\pgfsetfillcolor{currentfill}%
\pgfsetlinewidth{0.481800pt}%
\definecolor{currentstroke}{rgb}{1.000000,1.000000,1.000000}%
\pgfsetstrokecolor{currentstroke}%
\pgfsetdash{}{0pt}%
\pgfpathmoveto{\pgfqpoint{4.049596in}{2.490832in}}%
\pgfpathcurveto{\pgfqpoint{4.060646in}{2.490832in}}{\pgfqpoint{4.071245in}{2.495222in}}{\pgfqpoint{4.079059in}{2.503036in}}%
\pgfpathcurveto{\pgfqpoint{4.086873in}{2.510850in}}{\pgfqpoint{4.091263in}{2.521449in}}{\pgfqpoint{4.091263in}{2.532499in}}%
\pgfpathcurveto{\pgfqpoint{4.091263in}{2.543549in}}{\pgfqpoint{4.086873in}{2.554148in}}{\pgfqpoint{4.079059in}{2.561961in}}%
\pgfpathcurveto{\pgfqpoint{4.071245in}{2.569775in}}{\pgfqpoint{4.060646in}{2.574165in}}{\pgfqpoint{4.049596in}{2.574165in}}%
\pgfpathcurveto{\pgfqpoint{4.038546in}{2.574165in}}{\pgfqpoint{4.027947in}{2.569775in}}{\pgfqpoint{4.020133in}{2.561961in}}%
\pgfpathcurveto{\pgfqpoint{4.012320in}{2.554148in}}{\pgfqpoint{4.007929in}{2.543549in}}{\pgfqpoint{4.007929in}{2.532499in}}%
\pgfpathcurveto{\pgfqpoint{4.007929in}{2.521449in}}{\pgfqpoint{4.012320in}{2.510850in}}{\pgfqpoint{4.020133in}{2.503036in}}%
\pgfpathcurveto{\pgfqpoint{4.027947in}{2.495222in}}{\pgfqpoint{4.038546in}{2.490832in}}{\pgfqpoint{4.049596in}{2.490832in}}%
\pgfpathclose%
\pgfusepath{stroke,fill}%
\end{pgfscope}%
\begin{pgfscope}%
\pgfpathrectangle{\pgfqpoint{0.481978in}{0.331635in}}{\pgfqpoint{9.300000in}{7.700000in}}%
\pgfusepath{clip}%
\pgfsetbuttcap%
\pgfsetroundjoin%
\definecolor{currentfill}{rgb}{1.000000,0.705882,0.509804}%
\pgfsetfillcolor{currentfill}%
\pgfsetlinewidth{0.481800pt}%
\definecolor{currentstroke}{rgb}{1.000000,1.000000,1.000000}%
\pgfsetstrokecolor{currentstroke}%
\pgfsetdash{}{0pt}%
\pgfpathmoveto{\pgfqpoint{3.917035in}{4.780865in}}%
\pgfpathcurveto{\pgfqpoint{3.928085in}{4.780865in}}{\pgfqpoint{3.938684in}{4.785255in}}{\pgfqpoint{3.946497in}{4.793069in}}%
\pgfpathcurveto{\pgfqpoint{3.954311in}{4.800883in}}{\pgfqpoint{3.958701in}{4.811482in}}{\pgfqpoint{3.958701in}{4.822532in}}%
\pgfpathcurveto{\pgfqpoint{3.958701in}{4.833582in}}{\pgfqpoint{3.954311in}{4.844181in}}{\pgfqpoint{3.946497in}{4.851995in}}%
\pgfpathcurveto{\pgfqpoint{3.938684in}{4.859808in}}{\pgfqpoint{3.928085in}{4.864199in}}{\pgfqpoint{3.917035in}{4.864199in}}%
\pgfpathcurveto{\pgfqpoint{3.905984in}{4.864199in}}{\pgfqpoint{3.895385in}{4.859808in}}{\pgfqpoint{3.887572in}{4.851995in}}%
\pgfpathcurveto{\pgfqpoint{3.879758in}{4.844181in}}{\pgfqpoint{3.875368in}{4.833582in}}{\pgfqpoint{3.875368in}{4.822532in}}%
\pgfpathcurveto{\pgfqpoint{3.875368in}{4.811482in}}{\pgfqpoint{3.879758in}{4.800883in}}{\pgfqpoint{3.887572in}{4.793069in}}%
\pgfpathcurveto{\pgfqpoint{3.895385in}{4.785255in}}{\pgfqpoint{3.905984in}{4.780865in}}{\pgfqpoint{3.917035in}{4.780865in}}%
\pgfpathclose%
\pgfusepath{stroke,fill}%
\end{pgfscope}%
\begin{pgfscope}%
\pgfpathrectangle{\pgfqpoint{0.481978in}{0.331635in}}{\pgfqpoint{9.300000in}{7.700000in}}%
\pgfusepath{clip}%
\pgfsetbuttcap%
\pgfsetroundjoin%
\definecolor{currentfill}{rgb}{1.000000,0.705882,0.509804}%
\pgfsetfillcolor{currentfill}%
\pgfsetlinewidth{0.481800pt}%
\definecolor{currentstroke}{rgb}{1.000000,1.000000,1.000000}%
\pgfsetstrokecolor{currentstroke}%
\pgfsetdash{}{0pt}%
\pgfpathmoveto{\pgfqpoint{2.834471in}{5.928469in}}%
\pgfpathcurveto{\pgfqpoint{2.845521in}{5.928469in}}{\pgfqpoint{2.856120in}{5.932859in}}{\pgfqpoint{2.863933in}{5.940673in}}%
\pgfpathcurveto{\pgfqpoint{2.871747in}{5.948486in}}{\pgfqpoint{2.876137in}{5.959086in}}{\pgfqpoint{2.876137in}{5.970136in}}%
\pgfpathcurveto{\pgfqpoint{2.876137in}{5.981186in}}{\pgfqpoint{2.871747in}{5.991785in}}{\pgfqpoint{2.863933in}{5.999598in}}%
\pgfpathcurveto{\pgfqpoint{2.856120in}{6.007412in}}{\pgfqpoint{2.845521in}{6.011802in}}{\pgfqpoint{2.834471in}{6.011802in}}%
\pgfpathcurveto{\pgfqpoint{2.823420in}{6.011802in}}{\pgfqpoint{2.812821in}{6.007412in}}{\pgfqpoint{2.805008in}{5.999598in}}%
\pgfpathcurveto{\pgfqpoint{2.797194in}{5.991785in}}{\pgfqpoint{2.792804in}{5.981186in}}{\pgfqpoint{2.792804in}{5.970136in}}%
\pgfpathcurveto{\pgfqpoint{2.792804in}{5.959086in}}{\pgfqpoint{2.797194in}{5.948486in}}{\pgfqpoint{2.805008in}{5.940673in}}%
\pgfpathcurveto{\pgfqpoint{2.812821in}{5.932859in}}{\pgfqpoint{2.823420in}{5.928469in}}{\pgfqpoint{2.834471in}{5.928469in}}%
\pgfpathclose%
\pgfusepath{stroke,fill}%
\end{pgfscope}%
\begin{pgfscope}%
\pgfpathrectangle{\pgfqpoint{0.481978in}{0.331635in}}{\pgfqpoint{9.300000in}{7.700000in}}%
\pgfusepath{clip}%
\pgfsetbuttcap%
\pgfsetroundjoin%
\definecolor{currentfill}{rgb}{1.000000,0.705882,0.509804}%
\pgfsetfillcolor{currentfill}%
\pgfsetlinewidth{0.481800pt}%
\definecolor{currentstroke}{rgb}{1.000000,1.000000,1.000000}%
\pgfsetstrokecolor{currentstroke}%
\pgfsetdash{}{0pt}%
\pgfpathmoveto{\pgfqpoint{1.856829in}{3.973348in}}%
\pgfpathcurveto{\pgfqpoint{1.867879in}{3.973348in}}{\pgfqpoint{1.878478in}{3.977738in}}{\pgfqpoint{1.886291in}{3.985552in}}%
\pgfpathcurveto{\pgfqpoint{1.894105in}{3.993366in}}{\pgfqpoint{1.898495in}{4.003965in}}{\pgfqpoint{1.898495in}{4.015015in}}%
\pgfpathcurveto{\pgfqpoint{1.898495in}{4.026065in}}{\pgfqpoint{1.894105in}{4.036664in}}{\pgfqpoint{1.886291in}{4.044478in}}%
\pgfpathcurveto{\pgfqpoint{1.878478in}{4.052291in}}{\pgfqpoint{1.867879in}{4.056681in}}{\pgfqpoint{1.856829in}{4.056681in}}%
\pgfpathcurveto{\pgfqpoint{1.845778in}{4.056681in}}{\pgfqpoint{1.835179in}{4.052291in}}{\pgfqpoint{1.827366in}{4.044478in}}%
\pgfpathcurveto{\pgfqpoint{1.819552in}{4.036664in}}{\pgfqpoint{1.815162in}{4.026065in}}{\pgfqpoint{1.815162in}{4.015015in}}%
\pgfpathcurveto{\pgfqpoint{1.815162in}{4.003965in}}{\pgfqpoint{1.819552in}{3.993366in}}{\pgfqpoint{1.827366in}{3.985552in}}%
\pgfpathcurveto{\pgfqpoint{1.835179in}{3.977738in}}{\pgfqpoint{1.845778in}{3.973348in}}{\pgfqpoint{1.856829in}{3.973348in}}%
\pgfpathclose%
\pgfusepath{stroke,fill}%
\end{pgfscope}%
\begin{pgfscope}%
\pgfpathrectangle{\pgfqpoint{0.481978in}{0.331635in}}{\pgfqpoint{9.300000in}{7.700000in}}%
\pgfusepath{clip}%
\pgfsetbuttcap%
\pgfsetroundjoin%
\definecolor{currentfill}{rgb}{1.000000,0.705882,0.509804}%
\pgfsetfillcolor{currentfill}%
\pgfsetlinewidth{0.481800pt}%
\definecolor{currentstroke}{rgb}{1.000000,1.000000,1.000000}%
\pgfsetstrokecolor{currentstroke}%
\pgfsetdash{}{0pt}%
\pgfpathmoveto{\pgfqpoint{4.069841in}{2.077850in}}%
\pgfpathcurveto{\pgfqpoint{4.080891in}{2.077850in}}{\pgfqpoint{4.091490in}{2.082240in}}{\pgfqpoint{4.099303in}{2.090054in}}%
\pgfpathcurveto{\pgfqpoint{4.107117in}{2.097867in}}{\pgfqpoint{4.111507in}{2.108466in}}{\pgfqpoint{4.111507in}{2.119516in}}%
\pgfpathcurveto{\pgfqpoint{4.111507in}{2.130566in}}{\pgfqpoint{4.107117in}{2.141165in}}{\pgfqpoint{4.099303in}{2.148979in}}%
\pgfpathcurveto{\pgfqpoint{4.091490in}{2.156793in}}{\pgfqpoint{4.080891in}{2.161183in}}{\pgfqpoint{4.069841in}{2.161183in}}%
\pgfpathcurveto{\pgfqpoint{4.058790in}{2.161183in}}{\pgfqpoint{4.048191in}{2.156793in}}{\pgfqpoint{4.040378in}{2.148979in}}%
\pgfpathcurveto{\pgfqpoint{4.032564in}{2.141165in}}{\pgfqpoint{4.028174in}{2.130566in}}{\pgfqpoint{4.028174in}{2.119516in}}%
\pgfpathcurveto{\pgfqpoint{4.028174in}{2.108466in}}{\pgfqpoint{4.032564in}{2.097867in}}{\pgfqpoint{4.040378in}{2.090054in}}%
\pgfpathcurveto{\pgfqpoint{4.048191in}{2.082240in}}{\pgfqpoint{4.058790in}{2.077850in}}{\pgfqpoint{4.069841in}{2.077850in}}%
\pgfpathclose%
\pgfusepath{stroke,fill}%
\end{pgfscope}%
\begin{pgfscope}%
\pgfpathrectangle{\pgfqpoint{0.481978in}{0.331635in}}{\pgfqpoint{9.300000in}{7.700000in}}%
\pgfusepath{clip}%
\pgfsetbuttcap%
\pgfsetroundjoin%
\definecolor{currentfill}{rgb}{1.000000,0.705882,0.509804}%
\pgfsetfillcolor{currentfill}%
\pgfsetlinewidth{0.481800pt}%
\definecolor{currentstroke}{rgb}{1.000000,1.000000,1.000000}%
\pgfsetstrokecolor{currentstroke}%
\pgfsetdash{}{0pt}%
\pgfpathmoveto{\pgfqpoint{3.968076in}{5.367897in}}%
\pgfpathcurveto{\pgfqpoint{3.979127in}{5.367897in}}{\pgfqpoint{3.989726in}{5.372288in}}{\pgfqpoint{3.997539in}{5.380101in}}%
\pgfpathcurveto{\pgfqpoint{4.005353in}{5.387915in}}{\pgfqpoint{4.009743in}{5.398514in}}{\pgfqpoint{4.009743in}{5.409564in}}%
\pgfpathcurveto{\pgfqpoint{4.009743in}{5.420614in}}{\pgfqpoint{4.005353in}{5.431213in}}{\pgfqpoint{3.997539in}{5.439027in}}%
\pgfpathcurveto{\pgfqpoint{3.989726in}{5.446840in}}{\pgfqpoint{3.979127in}{5.451231in}}{\pgfqpoint{3.968076in}{5.451231in}}%
\pgfpathcurveto{\pgfqpoint{3.957026in}{5.451231in}}{\pgfqpoint{3.946427in}{5.446840in}}{\pgfqpoint{3.938614in}{5.439027in}}%
\pgfpathcurveto{\pgfqpoint{3.930800in}{5.431213in}}{\pgfqpoint{3.926410in}{5.420614in}}{\pgfqpoint{3.926410in}{5.409564in}}%
\pgfpathcurveto{\pgfqpoint{3.926410in}{5.398514in}}{\pgfqpoint{3.930800in}{5.387915in}}{\pgfqpoint{3.938614in}{5.380101in}}%
\pgfpathcurveto{\pgfqpoint{3.946427in}{5.372288in}}{\pgfqpoint{3.957026in}{5.367897in}}{\pgfqpoint{3.968076in}{5.367897in}}%
\pgfpathclose%
\pgfusepath{stroke,fill}%
\end{pgfscope}%
\begin{pgfscope}%
\pgfpathrectangle{\pgfqpoint{0.481978in}{0.331635in}}{\pgfqpoint{9.300000in}{7.700000in}}%
\pgfusepath{clip}%
\pgfsetbuttcap%
\pgfsetroundjoin%
\definecolor{currentfill}{rgb}{1.000000,0.705882,0.509804}%
\pgfsetfillcolor{currentfill}%
\pgfsetlinewidth{0.481800pt}%
\definecolor{currentstroke}{rgb}{1.000000,1.000000,1.000000}%
\pgfsetstrokecolor{currentstroke}%
\pgfsetdash{}{0pt}%
\pgfpathmoveto{\pgfqpoint{3.839235in}{2.937156in}}%
\pgfpathcurveto{\pgfqpoint{3.850285in}{2.937156in}}{\pgfqpoint{3.860884in}{2.941546in}}{\pgfqpoint{3.868698in}{2.949360in}}%
\pgfpathcurveto{\pgfqpoint{3.876511in}{2.957173in}}{\pgfqpoint{3.880901in}{2.967772in}}{\pgfqpoint{3.880901in}{2.978822in}}%
\pgfpathcurveto{\pgfqpoint{3.880901in}{2.989872in}}{\pgfqpoint{3.876511in}{3.000471in}}{\pgfqpoint{3.868698in}{3.008285in}}%
\pgfpathcurveto{\pgfqpoint{3.860884in}{3.016099in}}{\pgfqpoint{3.850285in}{3.020489in}}{\pgfqpoint{3.839235in}{3.020489in}}%
\pgfpathcurveto{\pgfqpoint{3.828185in}{3.020489in}}{\pgfqpoint{3.817586in}{3.016099in}}{\pgfqpoint{3.809772in}{3.008285in}}%
\pgfpathcurveto{\pgfqpoint{3.801958in}{3.000471in}}{\pgfqpoint{3.797568in}{2.989872in}}{\pgfqpoint{3.797568in}{2.978822in}}%
\pgfpathcurveto{\pgfqpoint{3.797568in}{2.967772in}}{\pgfqpoint{3.801958in}{2.957173in}}{\pgfqpoint{3.809772in}{2.949360in}}%
\pgfpathcurveto{\pgfqpoint{3.817586in}{2.941546in}}{\pgfqpoint{3.828185in}{2.937156in}}{\pgfqpoint{3.839235in}{2.937156in}}%
\pgfpathclose%
\pgfusepath{stroke,fill}%
\end{pgfscope}%
\begin{pgfscope}%
\pgfpathrectangle{\pgfqpoint{0.481978in}{0.331635in}}{\pgfqpoint{9.300000in}{7.700000in}}%
\pgfusepath{clip}%
\pgfsetbuttcap%
\pgfsetroundjoin%
\definecolor{currentfill}{rgb}{1.000000,0.705882,0.509804}%
\pgfsetfillcolor{currentfill}%
\pgfsetlinewidth{0.481800pt}%
\definecolor{currentstroke}{rgb}{1.000000,1.000000,1.000000}%
\pgfsetstrokecolor{currentstroke}%
\pgfsetdash{}{0pt}%
\pgfpathmoveto{\pgfqpoint{4.658407in}{6.466627in}}%
\pgfpathcurveto{\pgfqpoint{4.669457in}{6.466627in}}{\pgfqpoint{4.680056in}{6.471018in}}{\pgfqpoint{4.687869in}{6.478831in}}%
\pgfpathcurveto{\pgfqpoint{4.695683in}{6.486645in}}{\pgfqpoint{4.700073in}{6.497244in}}{\pgfqpoint{4.700073in}{6.508294in}}%
\pgfpathcurveto{\pgfqpoint{4.700073in}{6.519344in}}{\pgfqpoint{4.695683in}{6.529943in}}{\pgfqpoint{4.687869in}{6.537757in}}%
\pgfpathcurveto{\pgfqpoint{4.680056in}{6.545570in}}{\pgfqpoint{4.669457in}{6.549961in}}{\pgfqpoint{4.658407in}{6.549961in}}%
\pgfpathcurveto{\pgfqpoint{4.647356in}{6.549961in}}{\pgfqpoint{4.636757in}{6.545570in}}{\pgfqpoint{4.628944in}{6.537757in}}%
\pgfpathcurveto{\pgfqpoint{4.621130in}{6.529943in}}{\pgfqpoint{4.616740in}{6.519344in}}{\pgfqpoint{4.616740in}{6.508294in}}%
\pgfpathcurveto{\pgfqpoint{4.616740in}{6.497244in}}{\pgfqpoint{4.621130in}{6.486645in}}{\pgfqpoint{4.628944in}{6.478831in}}%
\pgfpathcurveto{\pgfqpoint{4.636757in}{6.471018in}}{\pgfqpoint{4.647356in}{6.466627in}}{\pgfqpoint{4.658407in}{6.466627in}}%
\pgfpathclose%
\pgfusepath{stroke,fill}%
\end{pgfscope}%
\begin{pgfscope}%
\pgfpathrectangle{\pgfqpoint{0.481978in}{0.331635in}}{\pgfqpoint{9.300000in}{7.700000in}}%
\pgfusepath{clip}%
\pgfsetbuttcap%
\pgfsetroundjoin%
\definecolor{currentfill}{rgb}{1.000000,0.705882,0.509804}%
\pgfsetfillcolor{currentfill}%
\pgfsetlinewidth{0.481800pt}%
\definecolor{currentstroke}{rgb}{1.000000,1.000000,1.000000}%
\pgfsetstrokecolor{currentstroke}%
\pgfsetdash{}{0pt}%
\pgfpathmoveto{\pgfqpoint{4.784824in}{5.879172in}}%
\pgfpathcurveto{\pgfqpoint{4.795874in}{5.879172in}}{\pgfqpoint{4.806473in}{5.883562in}}{\pgfqpoint{4.814287in}{5.891376in}}%
\pgfpathcurveto{\pgfqpoint{4.822100in}{5.899189in}}{\pgfqpoint{4.826491in}{5.909788in}}{\pgfqpoint{4.826491in}{5.920838in}}%
\pgfpathcurveto{\pgfqpoint{4.826491in}{5.931889in}}{\pgfqpoint{4.822100in}{5.942488in}}{\pgfqpoint{4.814287in}{5.950301in}}%
\pgfpathcurveto{\pgfqpoint{4.806473in}{5.958115in}}{\pgfqpoint{4.795874in}{5.962505in}}{\pgfqpoint{4.784824in}{5.962505in}}%
\pgfpathcurveto{\pgfqpoint{4.773774in}{5.962505in}}{\pgfqpoint{4.763175in}{5.958115in}}{\pgfqpoint{4.755361in}{5.950301in}}%
\pgfpathcurveto{\pgfqpoint{4.747548in}{5.942488in}}{\pgfqpoint{4.743157in}{5.931889in}}{\pgfqpoint{4.743157in}{5.920838in}}%
\pgfpathcurveto{\pgfqpoint{4.743157in}{5.909788in}}{\pgfqpoint{4.747548in}{5.899189in}}{\pgfqpoint{4.755361in}{5.891376in}}%
\pgfpathcurveto{\pgfqpoint{4.763175in}{5.883562in}}{\pgfqpoint{4.773774in}{5.879172in}}{\pgfqpoint{4.784824in}{5.879172in}}%
\pgfpathclose%
\pgfusepath{stroke,fill}%
\end{pgfscope}%
\begin{pgfscope}%
\pgfpathrectangle{\pgfqpoint{0.481978in}{0.331635in}}{\pgfqpoint{9.300000in}{7.700000in}}%
\pgfusepath{clip}%
\pgfsetbuttcap%
\pgfsetroundjoin%
\definecolor{currentfill}{rgb}{1.000000,0.705882,0.509804}%
\pgfsetfillcolor{currentfill}%
\pgfsetlinewidth{0.481800pt}%
\definecolor{currentstroke}{rgb}{1.000000,1.000000,1.000000}%
\pgfsetstrokecolor{currentstroke}%
\pgfsetdash{}{0pt}%
\pgfpathmoveto{\pgfqpoint{5.305166in}{4.606714in}}%
\pgfpathcurveto{\pgfqpoint{5.316216in}{4.606714in}}{\pgfqpoint{5.326815in}{4.611104in}}{\pgfqpoint{5.334629in}{4.618918in}}%
\pgfpathcurveto{\pgfqpoint{5.342442in}{4.626731in}}{\pgfqpoint{5.346833in}{4.637330in}}{\pgfqpoint{5.346833in}{4.648380in}}%
\pgfpathcurveto{\pgfqpoint{5.346833in}{4.659431in}}{\pgfqpoint{5.342442in}{4.670030in}}{\pgfqpoint{5.334629in}{4.677843in}}%
\pgfpathcurveto{\pgfqpoint{5.326815in}{4.685657in}}{\pgfqpoint{5.316216in}{4.690047in}}{\pgfqpoint{5.305166in}{4.690047in}}%
\pgfpathcurveto{\pgfqpoint{5.294116in}{4.690047in}}{\pgfqpoint{5.283517in}{4.685657in}}{\pgfqpoint{5.275703in}{4.677843in}}%
\pgfpathcurveto{\pgfqpoint{5.267890in}{4.670030in}}{\pgfqpoint{5.263499in}{4.659431in}}{\pgfqpoint{5.263499in}{4.648380in}}%
\pgfpathcurveto{\pgfqpoint{5.263499in}{4.637330in}}{\pgfqpoint{5.267890in}{4.626731in}}{\pgfqpoint{5.275703in}{4.618918in}}%
\pgfpathcurveto{\pgfqpoint{5.283517in}{4.611104in}}{\pgfqpoint{5.294116in}{4.606714in}}{\pgfqpoint{5.305166in}{4.606714in}}%
\pgfpathclose%
\pgfusepath{stroke,fill}%
\end{pgfscope}%
\begin{pgfscope}%
\pgfpathrectangle{\pgfqpoint{0.481978in}{0.331635in}}{\pgfqpoint{9.300000in}{7.700000in}}%
\pgfusepath{clip}%
\pgfsetbuttcap%
\pgfsetroundjoin%
\definecolor{currentfill}{rgb}{1.000000,0.705882,0.509804}%
\pgfsetfillcolor{currentfill}%
\pgfsetlinewidth{0.481800pt}%
\definecolor{currentstroke}{rgb}{1.000000,1.000000,1.000000}%
\pgfsetstrokecolor{currentstroke}%
\pgfsetdash{}{0pt}%
\pgfpathmoveto{\pgfqpoint{3.719294in}{4.720429in}}%
\pgfpathcurveto{\pgfqpoint{3.730344in}{4.720429in}}{\pgfqpoint{3.740943in}{4.724819in}}{\pgfqpoint{3.748756in}{4.732633in}}%
\pgfpathcurveto{\pgfqpoint{3.756570in}{4.740446in}}{\pgfqpoint{3.760960in}{4.751045in}}{\pgfqpoint{3.760960in}{4.762095in}}%
\pgfpathcurveto{\pgfqpoint{3.760960in}{4.773145in}}{\pgfqpoint{3.756570in}{4.783745in}}{\pgfqpoint{3.748756in}{4.791558in}}%
\pgfpathcurveto{\pgfqpoint{3.740943in}{4.799372in}}{\pgfqpoint{3.730344in}{4.803762in}}{\pgfqpoint{3.719294in}{4.803762in}}%
\pgfpathcurveto{\pgfqpoint{3.708243in}{4.803762in}}{\pgfqpoint{3.697644in}{4.799372in}}{\pgfqpoint{3.689831in}{4.791558in}}%
\pgfpathcurveto{\pgfqpoint{3.682017in}{4.783745in}}{\pgfqpoint{3.677627in}{4.773145in}}{\pgfqpoint{3.677627in}{4.762095in}}%
\pgfpathcurveto{\pgfqpoint{3.677627in}{4.751045in}}{\pgfqpoint{3.682017in}{4.740446in}}{\pgfqpoint{3.689831in}{4.732633in}}%
\pgfpathcurveto{\pgfqpoint{3.697644in}{4.724819in}}{\pgfqpoint{3.708243in}{4.720429in}}{\pgfqpoint{3.719294in}{4.720429in}}%
\pgfpathclose%
\pgfusepath{stroke,fill}%
\end{pgfscope}%
\begin{pgfscope}%
\pgfpathrectangle{\pgfqpoint{0.481978in}{0.331635in}}{\pgfqpoint{9.300000in}{7.700000in}}%
\pgfusepath{clip}%
\pgfsetbuttcap%
\pgfsetroundjoin%
\definecolor{currentfill}{rgb}{1.000000,0.705882,0.509804}%
\pgfsetfillcolor{currentfill}%
\pgfsetlinewidth{0.481800pt}%
\definecolor{currentstroke}{rgb}{1.000000,1.000000,1.000000}%
\pgfsetstrokecolor{currentstroke}%
\pgfsetdash{}{0pt}%
\pgfpathmoveto{\pgfqpoint{6.457903in}{1.796543in}}%
\pgfpathcurveto{\pgfqpoint{6.468953in}{1.796543in}}{\pgfqpoint{6.479552in}{1.800933in}}{\pgfqpoint{6.487366in}{1.808747in}}%
\pgfpathcurveto{\pgfqpoint{6.495179in}{1.816561in}}{\pgfqpoint{6.499570in}{1.827160in}}{\pgfqpoint{6.499570in}{1.838210in}}%
\pgfpathcurveto{\pgfqpoint{6.499570in}{1.849260in}}{\pgfqpoint{6.495179in}{1.859859in}}{\pgfqpoint{6.487366in}{1.867673in}}%
\pgfpathcurveto{\pgfqpoint{6.479552in}{1.875486in}}{\pgfqpoint{6.468953in}{1.879876in}}{\pgfqpoint{6.457903in}{1.879876in}}%
\pgfpathcurveto{\pgfqpoint{6.446853in}{1.879876in}}{\pgfqpoint{6.436254in}{1.875486in}}{\pgfqpoint{6.428440in}{1.867673in}}%
\pgfpathcurveto{\pgfqpoint{6.420627in}{1.859859in}}{\pgfqpoint{6.416236in}{1.849260in}}{\pgfqpoint{6.416236in}{1.838210in}}%
\pgfpathcurveto{\pgfqpoint{6.416236in}{1.827160in}}{\pgfqpoint{6.420627in}{1.816561in}}{\pgfqpoint{6.428440in}{1.808747in}}%
\pgfpathcurveto{\pgfqpoint{6.436254in}{1.800933in}}{\pgfqpoint{6.446853in}{1.796543in}}{\pgfqpoint{6.457903in}{1.796543in}}%
\pgfpathclose%
\pgfusepath{stroke,fill}%
\end{pgfscope}%
\begin{pgfscope}%
\pgfpathrectangle{\pgfqpoint{0.481978in}{0.331635in}}{\pgfqpoint{9.300000in}{7.700000in}}%
\pgfusepath{clip}%
\pgfsetbuttcap%
\pgfsetroundjoin%
\definecolor{currentfill}{rgb}{1.000000,0.705882,0.509804}%
\pgfsetfillcolor{currentfill}%
\pgfsetlinewidth{0.481800pt}%
\definecolor{currentstroke}{rgb}{1.000000,1.000000,1.000000}%
\pgfsetstrokecolor{currentstroke}%
\pgfsetdash{}{0pt}%
\pgfpathmoveto{\pgfqpoint{3.091678in}{5.600061in}}%
\pgfpathcurveto{\pgfqpoint{3.102728in}{5.600061in}}{\pgfqpoint{3.113327in}{5.604452in}}{\pgfqpoint{3.121141in}{5.612265in}}%
\pgfpathcurveto{\pgfqpoint{3.128955in}{5.620079in}}{\pgfqpoint{3.133345in}{5.630678in}}{\pgfqpoint{3.133345in}{5.641728in}}%
\pgfpathcurveto{\pgfqpoint{3.133345in}{5.652778in}}{\pgfqpoint{3.128955in}{5.663377in}}{\pgfqpoint{3.121141in}{5.671191in}}%
\pgfpathcurveto{\pgfqpoint{3.113327in}{5.679004in}}{\pgfqpoint{3.102728in}{5.683395in}}{\pgfqpoint{3.091678in}{5.683395in}}%
\pgfpathcurveto{\pgfqpoint{3.080628in}{5.683395in}}{\pgfqpoint{3.070029in}{5.679004in}}{\pgfqpoint{3.062215in}{5.671191in}}%
\pgfpathcurveto{\pgfqpoint{3.054402in}{5.663377in}}{\pgfqpoint{3.050012in}{5.652778in}}{\pgfqpoint{3.050012in}{5.641728in}}%
\pgfpathcurveto{\pgfqpoint{3.050012in}{5.630678in}}{\pgfqpoint{3.054402in}{5.620079in}}{\pgfqpoint{3.062215in}{5.612265in}}%
\pgfpathcurveto{\pgfqpoint{3.070029in}{5.604452in}}{\pgfqpoint{3.080628in}{5.600061in}}{\pgfqpoint{3.091678in}{5.600061in}}%
\pgfpathclose%
\pgfusepath{stroke,fill}%
\end{pgfscope}%
\begin{pgfscope}%
\pgfpathrectangle{\pgfqpoint{0.481978in}{0.331635in}}{\pgfqpoint{9.300000in}{7.700000in}}%
\pgfusepath{clip}%
\pgfsetbuttcap%
\pgfsetroundjoin%
\definecolor{currentfill}{rgb}{1.000000,0.705882,0.509804}%
\pgfsetfillcolor{currentfill}%
\pgfsetlinewidth{0.481800pt}%
\definecolor{currentstroke}{rgb}{1.000000,1.000000,1.000000}%
\pgfsetstrokecolor{currentstroke}%
\pgfsetdash{}{0pt}%
\pgfpathmoveto{\pgfqpoint{0.904705in}{4.489588in}}%
\pgfpathcurveto{\pgfqpoint{0.915755in}{4.489588in}}{\pgfqpoint{0.926354in}{4.493978in}}{\pgfqpoint{0.934168in}{4.501792in}}%
\pgfpathcurveto{\pgfqpoint{0.941982in}{4.509605in}}{\pgfqpoint{0.946372in}{4.520204in}}{\pgfqpoint{0.946372in}{4.531254in}}%
\pgfpathcurveto{\pgfqpoint{0.946372in}{4.542305in}}{\pgfqpoint{0.941982in}{4.552904in}}{\pgfqpoint{0.934168in}{4.560717in}}%
\pgfpathcurveto{\pgfqpoint{0.926354in}{4.568531in}}{\pgfqpoint{0.915755in}{4.572921in}}{\pgfqpoint{0.904705in}{4.572921in}}%
\pgfpathcurveto{\pgfqpoint{0.893655in}{4.572921in}}{\pgfqpoint{0.883056in}{4.568531in}}{\pgfqpoint{0.875242in}{4.560717in}}%
\pgfpathcurveto{\pgfqpoint{0.867429in}{4.552904in}}{\pgfqpoint{0.863039in}{4.542305in}}{\pgfqpoint{0.863039in}{4.531254in}}%
\pgfpathcurveto{\pgfqpoint{0.863039in}{4.520204in}}{\pgfqpoint{0.867429in}{4.509605in}}{\pgfqpoint{0.875242in}{4.501792in}}%
\pgfpathcurveto{\pgfqpoint{0.883056in}{4.493978in}}{\pgfqpoint{0.893655in}{4.489588in}}{\pgfqpoint{0.904705in}{4.489588in}}%
\pgfpathclose%
\pgfusepath{stroke,fill}%
\end{pgfscope}%
\begin{pgfscope}%
\pgfpathrectangle{\pgfqpoint{0.481978in}{0.331635in}}{\pgfqpoint{9.300000in}{7.700000in}}%
\pgfusepath{clip}%
\pgfsetbuttcap%
\pgfsetroundjoin%
\definecolor{currentfill}{rgb}{1.000000,0.705882,0.509804}%
\pgfsetfillcolor{currentfill}%
\pgfsetlinewidth{0.481800pt}%
\definecolor{currentstroke}{rgb}{1.000000,1.000000,1.000000}%
\pgfsetstrokecolor{currentstroke}%
\pgfsetdash{}{0pt}%
\pgfpathmoveto{\pgfqpoint{5.515150in}{3.272450in}}%
\pgfpathcurveto{\pgfqpoint{5.526200in}{3.272450in}}{\pgfqpoint{5.536799in}{3.276841in}}{\pgfqpoint{5.544612in}{3.284654in}}%
\pgfpathcurveto{\pgfqpoint{5.552426in}{3.292468in}}{\pgfqpoint{5.556816in}{3.303067in}}{\pgfqpoint{5.556816in}{3.314117in}}%
\pgfpathcurveto{\pgfqpoint{5.556816in}{3.325167in}}{\pgfqpoint{5.552426in}{3.335766in}}{\pgfqpoint{5.544612in}{3.343580in}}%
\pgfpathcurveto{\pgfqpoint{5.536799in}{3.351393in}}{\pgfqpoint{5.526200in}{3.355784in}}{\pgfqpoint{5.515150in}{3.355784in}}%
\pgfpathcurveto{\pgfqpoint{5.504099in}{3.355784in}}{\pgfqpoint{5.493500in}{3.351393in}}{\pgfqpoint{5.485687in}{3.343580in}}%
\pgfpathcurveto{\pgfqpoint{5.477873in}{3.335766in}}{\pgfqpoint{5.473483in}{3.325167in}}{\pgfqpoint{5.473483in}{3.314117in}}%
\pgfpathcurveto{\pgfqpoint{5.473483in}{3.303067in}}{\pgfqpoint{5.477873in}{3.292468in}}{\pgfqpoint{5.485687in}{3.284654in}}%
\pgfpathcurveto{\pgfqpoint{5.493500in}{3.276841in}}{\pgfqpoint{5.504099in}{3.272450in}}{\pgfqpoint{5.515150in}{3.272450in}}%
\pgfpathclose%
\pgfusepath{stroke,fill}%
\end{pgfscope}%
\begin{pgfscope}%
\pgfpathrectangle{\pgfqpoint{0.481978in}{0.331635in}}{\pgfqpoint{9.300000in}{7.700000in}}%
\pgfusepath{clip}%
\pgfsetbuttcap%
\pgfsetroundjoin%
\definecolor{currentfill}{rgb}{1.000000,0.705882,0.509804}%
\pgfsetfillcolor{currentfill}%
\pgfsetlinewidth{0.481800pt}%
\definecolor{currentstroke}{rgb}{1.000000,1.000000,1.000000}%
\pgfsetstrokecolor{currentstroke}%
\pgfsetdash{}{0pt}%
\pgfpathmoveto{\pgfqpoint{3.655066in}{5.193021in}}%
\pgfpathcurveto{\pgfqpoint{3.666116in}{5.193021in}}{\pgfqpoint{3.676715in}{5.197411in}}{\pgfqpoint{3.684528in}{5.205225in}}%
\pgfpathcurveto{\pgfqpoint{3.692342in}{5.213038in}}{\pgfqpoint{3.696732in}{5.223637in}}{\pgfqpoint{3.696732in}{5.234687in}}%
\pgfpathcurveto{\pgfqpoint{3.696732in}{5.245738in}}{\pgfqpoint{3.692342in}{5.256337in}}{\pgfqpoint{3.684528in}{5.264150in}}%
\pgfpathcurveto{\pgfqpoint{3.676715in}{5.271964in}}{\pgfqpoint{3.666116in}{5.276354in}}{\pgfqpoint{3.655066in}{5.276354in}}%
\pgfpathcurveto{\pgfqpoint{3.644016in}{5.276354in}}{\pgfqpoint{3.633417in}{5.271964in}}{\pgfqpoint{3.625603in}{5.264150in}}%
\pgfpathcurveto{\pgfqpoint{3.617789in}{5.256337in}}{\pgfqpoint{3.613399in}{5.245738in}}{\pgfqpoint{3.613399in}{5.234687in}}%
\pgfpathcurveto{\pgfqpoint{3.613399in}{5.223637in}}{\pgfqpoint{3.617789in}{5.213038in}}{\pgfqpoint{3.625603in}{5.205225in}}%
\pgfpathcurveto{\pgfqpoint{3.633417in}{5.197411in}}{\pgfqpoint{3.644016in}{5.193021in}}{\pgfqpoint{3.655066in}{5.193021in}}%
\pgfpathclose%
\pgfusepath{stroke,fill}%
\end{pgfscope}%
\begin{pgfscope}%
\pgfpathrectangle{\pgfqpoint{0.481978in}{0.331635in}}{\pgfqpoint{9.300000in}{7.700000in}}%
\pgfusepath{clip}%
\pgfsetbuttcap%
\pgfsetroundjoin%
\definecolor{currentfill}{rgb}{1.000000,0.705882,0.509804}%
\pgfsetfillcolor{currentfill}%
\pgfsetlinewidth{0.481800pt}%
\definecolor{currentstroke}{rgb}{1.000000,1.000000,1.000000}%
\pgfsetstrokecolor{currentstroke}%
\pgfsetdash{}{0pt}%
\pgfpathmoveto{\pgfqpoint{4.606001in}{2.677957in}}%
\pgfpathcurveto{\pgfqpoint{4.617051in}{2.677957in}}{\pgfqpoint{4.627650in}{2.682348in}}{\pgfqpoint{4.635464in}{2.690161in}}%
\pgfpathcurveto{\pgfqpoint{4.643277in}{2.697975in}}{\pgfqpoint{4.647667in}{2.708574in}}{\pgfqpoint{4.647667in}{2.719624in}}%
\pgfpathcurveto{\pgfqpoint{4.647667in}{2.730674in}}{\pgfqpoint{4.643277in}{2.741273in}}{\pgfqpoint{4.635464in}{2.749087in}}%
\pgfpathcurveto{\pgfqpoint{4.627650in}{2.756901in}}{\pgfqpoint{4.617051in}{2.761291in}}{\pgfqpoint{4.606001in}{2.761291in}}%
\pgfpathcurveto{\pgfqpoint{4.594951in}{2.761291in}}{\pgfqpoint{4.584352in}{2.756901in}}{\pgfqpoint{4.576538in}{2.749087in}}%
\pgfpathcurveto{\pgfqpoint{4.568724in}{2.741273in}}{\pgfqpoint{4.564334in}{2.730674in}}{\pgfqpoint{4.564334in}{2.719624in}}%
\pgfpathcurveto{\pgfqpoint{4.564334in}{2.708574in}}{\pgfqpoint{4.568724in}{2.697975in}}{\pgfqpoint{4.576538in}{2.690161in}}%
\pgfpathcurveto{\pgfqpoint{4.584352in}{2.682348in}}{\pgfqpoint{4.594951in}{2.677957in}}{\pgfqpoint{4.606001in}{2.677957in}}%
\pgfpathclose%
\pgfusepath{stroke,fill}%
\end{pgfscope}%
\begin{pgfscope}%
\pgfpathrectangle{\pgfqpoint{0.481978in}{0.331635in}}{\pgfqpoint{9.300000in}{7.700000in}}%
\pgfusepath{clip}%
\pgfsetbuttcap%
\pgfsetroundjoin%
\definecolor{currentfill}{rgb}{1.000000,0.705882,0.509804}%
\pgfsetfillcolor{currentfill}%
\pgfsetlinewidth{0.481800pt}%
\definecolor{currentstroke}{rgb}{1.000000,1.000000,1.000000}%
\pgfsetstrokecolor{currentstroke}%
\pgfsetdash{}{0pt}%
\pgfpathmoveto{\pgfqpoint{2.788909in}{4.513266in}}%
\pgfpathcurveto{\pgfqpoint{2.799959in}{4.513266in}}{\pgfqpoint{2.810558in}{4.517656in}}{\pgfqpoint{2.818372in}{4.525470in}}%
\pgfpathcurveto{\pgfqpoint{2.826185in}{4.533283in}}{\pgfqpoint{2.830575in}{4.543882in}}{\pgfqpoint{2.830575in}{4.554933in}}%
\pgfpathcurveto{\pgfqpoint{2.830575in}{4.565983in}}{\pgfqpoint{2.826185in}{4.576582in}}{\pgfqpoint{2.818372in}{4.584395in}}%
\pgfpathcurveto{\pgfqpoint{2.810558in}{4.592209in}}{\pgfqpoint{2.799959in}{4.596599in}}{\pgfqpoint{2.788909in}{4.596599in}}%
\pgfpathcurveto{\pgfqpoint{2.777859in}{4.596599in}}{\pgfqpoint{2.767260in}{4.592209in}}{\pgfqpoint{2.759446in}{4.584395in}}%
\pgfpathcurveto{\pgfqpoint{2.751632in}{4.576582in}}{\pgfqpoint{2.747242in}{4.565983in}}{\pgfqpoint{2.747242in}{4.554933in}}%
\pgfpathcurveto{\pgfqpoint{2.747242in}{4.543882in}}{\pgfqpoint{2.751632in}{4.533283in}}{\pgfqpoint{2.759446in}{4.525470in}}%
\pgfpathcurveto{\pgfqpoint{2.767260in}{4.517656in}}{\pgfqpoint{2.777859in}{4.513266in}}{\pgfqpoint{2.788909in}{4.513266in}}%
\pgfpathclose%
\pgfusepath{stroke,fill}%
\end{pgfscope}%
\begin{pgfscope}%
\pgfpathrectangle{\pgfqpoint{0.481978in}{0.331635in}}{\pgfqpoint{9.300000in}{7.700000in}}%
\pgfusepath{clip}%
\pgfsetbuttcap%
\pgfsetroundjoin%
\definecolor{currentfill}{rgb}{1.000000,0.705882,0.509804}%
\pgfsetfillcolor{currentfill}%
\pgfsetlinewidth{0.481800pt}%
\definecolor{currentstroke}{rgb}{1.000000,1.000000,1.000000}%
\pgfsetstrokecolor{currentstroke}%
\pgfsetdash{}{0pt}%
\pgfpathmoveto{\pgfqpoint{2.529879in}{3.143449in}}%
\pgfpathcurveto{\pgfqpoint{2.540929in}{3.143449in}}{\pgfqpoint{2.551528in}{3.147839in}}{\pgfqpoint{2.559342in}{3.155653in}}%
\pgfpathcurveto{\pgfqpoint{2.567155in}{3.163466in}}{\pgfqpoint{2.571546in}{3.174065in}}{\pgfqpoint{2.571546in}{3.185116in}}%
\pgfpathcurveto{\pgfqpoint{2.571546in}{3.196166in}}{\pgfqpoint{2.567155in}{3.206765in}}{\pgfqpoint{2.559342in}{3.214578in}}%
\pgfpathcurveto{\pgfqpoint{2.551528in}{3.222392in}}{\pgfqpoint{2.540929in}{3.226782in}}{\pgfqpoint{2.529879in}{3.226782in}}%
\pgfpathcurveto{\pgfqpoint{2.518829in}{3.226782in}}{\pgfqpoint{2.508230in}{3.222392in}}{\pgfqpoint{2.500416in}{3.214578in}}%
\pgfpathcurveto{\pgfqpoint{2.492602in}{3.206765in}}{\pgfqpoint{2.488212in}{3.196166in}}{\pgfqpoint{2.488212in}{3.185116in}}%
\pgfpathcurveto{\pgfqpoint{2.488212in}{3.174065in}}{\pgfqpoint{2.492602in}{3.163466in}}{\pgfqpoint{2.500416in}{3.155653in}}%
\pgfpathcurveto{\pgfqpoint{2.508230in}{3.147839in}}{\pgfqpoint{2.518829in}{3.143449in}}{\pgfqpoint{2.529879in}{3.143449in}}%
\pgfpathclose%
\pgfusepath{stroke,fill}%
\end{pgfscope}%
\begin{pgfscope}%
\pgfpathrectangle{\pgfqpoint{0.481978in}{0.331635in}}{\pgfqpoint{9.300000in}{7.700000in}}%
\pgfusepath{clip}%
\pgfsetbuttcap%
\pgfsetroundjoin%
\definecolor{currentfill}{rgb}{1.000000,0.705882,0.509804}%
\pgfsetfillcolor{currentfill}%
\pgfsetlinewidth{0.481800pt}%
\definecolor{currentstroke}{rgb}{1.000000,1.000000,1.000000}%
\pgfsetstrokecolor{currentstroke}%
\pgfsetdash{}{0pt}%
\pgfpathmoveto{\pgfqpoint{2.960587in}{2.499432in}}%
\pgfpathcurveto{\pgfqpoint{2.971637in}{2.499432in}}{\pgfqpoint{2.982236in}{2.503822in}}{\pgfqpoint{2.990050in}{2.511636in}}%
\pgfpathcurveto{\pgfqpoint{2.997864in}{2.519449in}}{\pgfqpoint{3.002254in}{2.530048in}}{\pgfqpoint{3.002254in}{2.541098in}}%
\pgfpathcurveto{\pgfqpoint{3.002254in}{2.552148in}}{\pgfqpoint{2.997864in}{2.562747in}}{\pgfqpoint{2.990050in}{2.570561in}}%
\pgfpathcurveto{\pgfqpoint{2.982236in}{2.578375in}}{\pgfqpoint{2.971637in}{2.582765in}}{\pgfqpoint{2.960587in}{2.582765in}}%
\pgfpathcurveto{\pgfqpoint{2.949537in}{2.582765in}}{\pgfqpoint{2.938938in}{2.578375in}}{\pgfqpoint{2.931124in}{2.570561in}}%
\pgfpathcurveto{\pgfqpoint{2.923311in}{2.562747in}}{\pgfqpoint{2.918920in}{2.552148in}}{\pgfqpoint{2.918920in}{2.541098in}}%
\pgfpathcurveto{\pgfqpoint{2.918920in}{2.530048in}}{\pgfqpoint{2.923311in}{2.519449in}}{\pgfqpoint{2.931124in}{2.511636in}}%
\pgfpathcurveto{\pgfqpoint{2.938938in}{2.503822in}}{\pgfqpoint{2.949537in}{2.499432in}}{\pgfqpoint{2.960587in}{2.499432in}}%
\pgfpathclose%
\pgfusepath{stroke,fill}%
\end{pgfscope}%
\begin{pgfscope}%
\pgfpathrectangle{\pgfqpoint{0.481978in}{0.331635in}}{\pgfqpoint{9.300000in}{7.700000in}}%
\pgfusepath{clip}%
\pgfsetbuttcap%
\pgfsetroundjoin%
\definecolor{currentfill}{rgb}{1.000000,0.705882,0.509804}%
\pgfsetfillcolor{currentfill}%
\pgfsetlinewidth{0.481800pt}%
\definecolor{currentstroke}{rgb}{1.000000,1.000000,1.000000}%
\pgfsetstrokecolor{currentstroke}%
\pgfsetdash{}{0pt}%
\pgfpathmoveto{\pgfqpoint{8.052568in}{3.832077in}}%
\pgfpathcurveto{\pgfqpoint{8.063618in}{3.832077in}}{\pgfqpoint{8.074217in}{3.836468in}}{\pgfqpoint{8.082031in}{3.844281in}}%
\pgfpathcurveto{\pgfqpoint{8.089844in}{3.852095in}}{\pgfqpoint{8.094234in}{3.862694in}}{\pgfqpoint{8.094234in}{3.873744in}}%
\pgfpathcurveto{\pgfqpoint{8.094234in}{3.884794in}}{\pgfqpoint{8.089844in}{3.895393in}}{\pgfqpoint{8.082031in}{3.903207in}}%
\pgfpathcurveto{\pgfqpoint{8.074217in}{3.911021in}}{\pgfqpoint{8.063618in}{3.915411in}}{\pgfqpoint{8.052568in}{3.915411in}}%
\pgfpathcurveto{\pgfqpoint{8.041518in}{3.915411in}}{\pgfqpoint{8.030919in}{3.911021in}}{\pgfqpoint{8.023105in}{3.903207in}}%
\pgfpathcurveto{\pgfqpoint{8.015291in}{3.895393in}}{\pgfqpoint{8.010901in}{3.884794in}}{\pgfqpoint{8.010901in}{3.873744in}}%
\pgfpathcurveto{\pgfqpoint{8.010901in}{3.862694in}}{\pgfqpoint{8.015291in}{3.852095in}}{\pgfqpoint{8.023105in}{3.844281in}}%
\pgfpathcurveto{\pgfqpoint{8.030919in}{3.836468in}}{\pgfqpoint{8.041518in}{3.832077in}}{\pgfqpoint{8.052568in}{3.832077in}}%
\pgfpathclose%
\pgfusepath{stroke,fill}%
\end{pgfscope}%
\begin{pgfscope}%
\pgfpathrectangle{\pgfqpoint{0.481978in}{0.331635in}}{\pgfqpoint{9.300000in}{7.700000in}}%
\pgfusepath{clip}%
\pgfsetbuttcap%
\pgfsetroundjoin%
\definecolor{currentfill}{rgb}{1.000000,0.705882,0.509804}%
\pgfsetfillcolor{currentfill}%
\pgfsetlinewidth{0.481800pt}%
\definecolor{currentstroke}{rgb}{1.000000,1.000000,1.000000}%
\pgfsetstrokecolor{currentstroke}%
\pgfsetdash{}{0pt}%
\pgfpathmoveto{\pgfqpoint{1.457328in}{2.495820in}}%
\pgfpathcurveto{\pgfqpoint{1.468378in}{2.495820in}}{\pgfqpoint{1.478977in}{2.500210in}}{\pgfqpoint{1.486791in}{2.508024in}}%
\pgfpathcurveto{\pgfqpoint{1.494604in}{2.515837in}}{\pgfqpoint{1.498994in}{2.526436in}}{\pgfqpoint{1.498994in}{2.537487in}}%
\pgfpathcurveto{\pgfqpoint{1.498994in}{2.548537in}}{\pgfqpoint{1.494604in}{2.559136in}}{\pgfqpoint{1.486791in}{2.566949in}}%
\pgfpathcurveto{\pgfqpoint{1.478977in}{2.574763in}}{\pgfqpoint{1.468378in}{2.579153in}}{\pgfqpoint{1.457328in}{2.579153in}}%
\pgfpathcurveto{\pgfqpoint{1.446278in}{2.579153in}}{\pgfqpoint{1.435679in}{2.574763in}}{\pgfqpoint{1.427865in}{2.566949in}}%
\pgfpathcurveto{\pgfqpoint{1.420051in}{2.559136in}}{\pgfqpoint{1.415661in}{2.548537in}}{\pgfqpoint{1.415661in}{2.537487in}}%
\pgfpathcurveto{\pgfqpoint{1.415661in}{2.526436in}}{\pgfqpoint{1.420051in}{2.515837in}}{\pgfqpoint{1.427865in}{2.508024in}}%
\pgfpathcurveto{\pgfqpoint{1.435679in}{2.500210in}}{\pgfqpoint{1.446278in}{2.495820in}}{\pgfqpoint{1.457328in}{2.495820in}}%
\pgfpathclose%
\pgfusepath{stroke,fill}%
\end{pgfscope}%
\begin{pgfscope}%
\pgfpathrectangle{\pgfqpoint{0.481978in}{0.331635in}}{\pgfqpoint{9.300000in}{7.700000in}}%
\pgfusepath{clip}%
\pgfsetbuttcap%
\pgfsetroundjoin%
\definecolor{currentfill}{rgb}{1.000000,0.705882,0.509804}%
\pgfsetfillcolor{currentfill}%
\pgfsetlinewidth{0.481800pt}%
\definecolor{currentstroke}{rgb}{1.000000,1.000000,1.000000}%
\pgfsetstrokecolor{currentstroke}%
\pgfsetdash{}{0pt}%
\pgfpathmoveto{\pgfqpoint{1.959498in}{3.611662in}}%
\pgfpathcurveto{\pgfqpoint{1.970548in}{3.611662in}}{\pgfqpoint{1.981147in}{3.616053in}}{\pgfqpoint{1.988961in}{3.623866in}}%
\pgfpathcurveto{\pgfqpoint{1.996775in}{3.631680in}}{\pgfqpoint{2.001165in}{3.642279in}}{\pgfqpoint{2.001165in}{3.653329in}}%
\pgfpathcurveto{\pgfqpoint{2.001165in}{3.664379in}}{\pgfqpoint{1.996775in}{3.674978in}}{\pgfqpoint{1.988961in}{3.682792in}}%
\pgfpathcurveto{\pgfqpoint{1.981147in}{3.690605in}}{\pgfqpoint{1.970548in}{3.694996in}}{\pgfqpoint{1.959498in}{3.694996in}}%
\pgfpathcurveto{\pgfqpoint{1.948448in}{3.694996in}}{\pgfqpoint{1.937849in}{3.690605in}}{\pgfqpoint{1.930035in}{3.682792in}}%
\pgfpathcurveto{\pgfqpoint{1.922222in}{3.674978in}}{\pgfqpoint{1.917831in}{3.664379in}}{\pgfqpoint{1.917831in}{3.653329in}}%
\pgfpathcurveto{\pgfqpoint{1.917831in}{3.642279in}}{\pgfqpoint{1.922222in}{3.631680in}}{\pgfqpoint{1.930035in}{3.623866in}}%
\pgfpathcurveto{\pgfqpoint{1.937849in}{3.616053in}}{\pgfqpoint{1.948448in}{3.611662in}}{\pgfqpoint{1.959498in}{3.611662in}}%
\pgfpathclose%
\pgfusepath{stroke,fill}%
\end{pgfscope}%
\begin{pgfscope}%
\pgfpathrectangle{\pgfqpoint{0.481978in}{0.331635in}}{\pgfqpoint{9.300000in}{7.700000in}}%
\pgfusepath{clip}%
\pgfsetbuttcap%
\pgfsetroundjoin%
\definecolor{currentfill}{rgb}{1.000000,0.705882,0.509804}%
\pgfsetfillcolor{currentfill}%
\pgfsetlinewidth{0.481800pt}%
\definecolor{currentstroke}{rgb}{1.000000,1.000000,1.000000}%
\pgfsetstrokecolor{currentstroke}%
\pgfsetdash{}{0pt}%
\pgfpathmoveto{\pgfqpoint{5.141137in}{2.853956in}}%
\pgfpathcurveto{\pgfqpoint{5.152187in}{2.853956in}}{\pgfqpoint{5.162786in}{2.858347in}}{\pgfqpoint{5.170600in}{2.866160in}}%
\pgfpathcurveto{\pgfqpoint{5.178414in}{2.873974in}}{\pgfqpoint{5.182804in}{2.884573in}}{\pgfqpoint{5.182804in}{2.895623in}}%
\pgfpathcurveto{\pgfqpoint{5.182804in}{2.906673in}}{\pgfqpoint{5.178414in}{2.917272in}}{\pgfqpoint{5.170600in}{2.925086in}}%
\pgfpathcurveto{\pgfqpoint{5.162786in}{2.932899in}}{\pgfqpoint{5.152187in}{2.937290in}}{\pgfqpoint{5.141137in}{2.937290in}}%
\pgfpathcurveto{\pgfqpoint{5.130087in}{2.937290in}}{\pgfqpoint{5.119488in}{2.932899in}}{\pgfqpoint{5.111674in}{2.925086in}}%
\pgfpathcurveto{\pgfqpoint{5.103861in}{2.917272in}}{\pgfqpoint{5.099470in}{2.906673in}}{\pgfqpoint{5.099470in}{2.895623in}}%
\pgfpathcurveto{\pgfqpoint{5.099470in}{2.884573in}}{\pgfqpoint{5.103861in}{2.873974in}}{\pgfqpoint{5.111674in}{2.866160in}}%
\pgfpathcurveto{\pgfqpoint{5.119488in}{2.858347in}}{\pgfqpoint{5.130087in}{2.853956in}}{\pgfqpoint{5.141137in}{2.853956in}}%
\pgfpathclose%
\pgfusepath{stroke,fill}%
\end{pgfscope}%
\begin{pgfscope}%
\pgfpathrectangle{\pgfqpoint{0.481978in}{0.331635in}}{\pgfqpoint{9.300000in}{7.700000in}}%
\pgfusepath{clip}%
\pgfsetbuttcap%
\pgfsetroundjoin%
\definecolor{currentfill}{rgb}{1.000000,0.705882,0.509804}%
\pgfsetfillcolor{currentfill}%
\pgfsetlinewidth{0.481800pt}%
\definecolor{currentstroke}{rgb}{1.000000,1.000000,1.000000}%
\pgfsetstrokecolor{currentstroke}%
\pgfsetdash{}{0pt}%
\pgfpathmoveto{\pgfqpoint{5.415961in}{3.905778in}}%
\pgfpathcurveto{\pgfqpoint{5.427011in}{3.905778in}}{\pgfqpoint{5.437610in}{3.910169in}}{\pgfqpoint{5.445423in}{3.917982in}}%
\pgfpathcurveto{\pgfqpoint{5.453237in}{3.925796in}}{\pgfqpoint{5.457627in}{3.936395in}}{\pgfqpoint{5.457627in}{3.947445in}}%
\pgfpathcurveto{\pgfqpoint{5.457627in}{3.958495in}}{\pgfqpoint{5.453237in}{3.969094in}}{\pgfqpoint{5.445423in}{3.976908in}}%
\pgfpathcurveto{\pgfqpoint{5.437610in}{3.984721in}}{\pgfqpoint{5.427011in}{3.989112in}}{\pgfqpoint{5.415961in}{3.989112in}}%
\pgfpathcurveto{\pgfqpoint{5.404910in}{3.989112in}}{\pgfqpoint{5.394311in}{3.984721in}}{\pgfqpoint{5.386498in}{3.976908in}}%
\pgfpathcurveto{\pgfqpoint{5.378684in}{3.969094in}}{\pgfqpoint{5.374294in}{3.958495in}}{\pgfqpoint{5.374294in}{3.947445in}}%
\pgfpathcurveto{\pgfqpoint{5.374294in}{3.936395in}}{\pgfqpoint{5.378684in}{3.925796in}}{\pgfqpoint{5.386498in}{3.917982in}}%
\pgfpathcurveto{\pgfqpoint{5.394311in}{3.910169in}}{\pgfqpoint{5.404910in}{3.905778in}}{\pgfqpoint{5.415961in}{3.905778in}}%
\pgfpathclose%
\pgfusepath{stroke,fill}%
\end{pgfscope}%
\begin{pgfscope}%
\pgfpathrectangle{\pgfqpoint{0.481978in}{0.331635in}}{\pgfqpoint{9.300000in}{7.700000in}}%
\pgfusepath{clip}%
\pgfsetbuttcap%
\pgfsetroundjoin%
\definecolor{currentfill}{rgb}{1.000000,0.705882,0.509804}%
\pgfsetfillcolor{currentfill}%
\pgfsetlinewidth{0.481800pt}%
\definecolor{currentstroke}{rgb}{1.000000,1.000000,1.000000}%
\pgfsetstrokecolor{currentstroke}%
\pgfsetdash{}{0pt}%
\pgfpathmoveto{\pgfqpoint{3.418504in}{2.566381in}}%
\pgfpathcurveto{\pgfqpoint{3.429554in}{2.566381in}}{\pgfqpoint{3.440153in}{2.570771in}}{\pgfqpoint{3.447967in}{2.578584in}}%
\pgfpathcurveto{\pgfqpoint{3.455781in}{2.586398in}}{\pgfqpoint{3.460171in}{2.596997in}}{\pgfqpoint{3.460171in}{2.608047in}}%
\pgfpathcurveto{\pgfqpoint{3.460171in}{2.619097in}}{\pgfqpoint{3.455781in}{2.629696in}}{\pgfqpoint{3.447967in}{2.637510in}}%
\pgfpathcurveto{\pgfqpoint{3.440153in}{2.645324in}}{\pgfqpoint{3.429554in}{2.649714in}}{\pgfqpoint{3.418504in}{2.649714in}}%
\pgfpathcurveto{\pgfqpoint{3.407454in}{2.649714in}}{\pgfqpoint{3.396855in}{2.645324in}}{\pgfqpoint{3.389041in}{2.637510in}}%
\pgfpathcurveto{\pgfqpoint{3.381228in}{2.629696in}}{\pgfqpoint{3.376838in}{2.619097in}}{\pgfqpoint{3.376838in}{2.608047in}}%
\pgfpathcurveto{\pgfqpoint{3.376838in}{2.596997in}}{\pgfqpoint{3.381228in}{2.586398in}}{\pgfqpoint{3.389041in}{2.578584in}}%
\pgfpathcurveto{\pgfqpoint{3.396855in}{2.570771in}}{\pgfqpoint{3.407454in}{2.566381in}}{\pgfqpoint{3.418504in}{2.566381in}}%
\pgfpathclose%
\pgfusepath{stroke,fill}%
\end{pgfscope}%
\begin{pgfscope}%
\pgfpathrectangle{\pgfqpoint{0.481978in}{0.331635in}}{\pgfqpoint{9.300000in}{7.700000in}}%
\pgfusepath{clip}%
\pgfsetbuttcap%
\pgfsetroundjoin%
\definecolor{currentfill}{rgb}{1.000000,0.705882,0.509804}%
\pgfsetfillcolor{currentfill}%
\pgfsetlinewidth{0.481800pt}%
\definecolor{currentstroke}{rgb}{1.000000,1.000000,1.000000}%
\pgfsetstrokecolor{currentstroke}%
\pgfsetdash{}{0pt}%
\pgfpathmoveto{\pgfqpoint{1.418868in}{3.681751in}}%
\pgfpathcurveto{\pgfqpoint{1.429918in}{3.681751in}}{\pgfqpoint{1.440517in}{3.686141in}}{\pgfqpoint{1.448331in}{3.693954in}}%
\pgfpathcurveto{\pgfqpoint{1.456145in}{3.701768in}}{\pgfqpoint{1.460535in}{3.712367in}}{\pgfqpoint{1.460535in}{3.723417in}}%
\pgfpathcurveto{\pgfqpoint{1.460535in}{3.734467in}}{\pgfqpoint{1.456145in}{3.745066in}}{\pgfqpoint{1.448331in}{3.752880in}}%
\pgfpathcurveto{\pgfqpoint{1.440517in}{3.760694in}}{\pgfqpoint{1.429918in}{3.765084in}}{\pgfqpoint{1.418868in}{3.765084in}}%
\pgfpathcurveto{\pgfqpoint{1.407818in}{3.765084in}}{\pgfqpoint{1.397219in}{3.760694in}}{\pgfqpoint{1.389405in}{3.752880in}}%
\pgfpathcurveto{\pgfqpoint{1.381592in}{3.745066in}}{\pgfqpoint{1.377202in}{3.734467in}}{\pgfqpoint{1.377202in}{3.723417in}}%
\pgfpathcurveto{\pgfqpoint{1.377202in}{3.712367in}}{\pgfqpoint{1.381592in}{3.701768in}}{\pgfqpoint{1.389405in}{3.693954in}}%
\pgfpathcurveto{\pgfqpoint{1.397219in}{3.686141in}}{\pgfqpoint{1.407818in}{3.681751in}}{\pgfqpoint{1.418868in}{3.681751in}}%
\pgfpathclose%
\pgfusepath{stroke,fill}%
\end{pgfscope}%
\begin{pgfscope}%
\pgfpathrectangle{\pgfqpoint{0.481978in}{0.331635in}}{\pgfqpoint{9.300000in}{7.700000in}}%
\pgfusepath{clip}%
\pgfsetbuttcap%
\pgfsetroundjoin%
\definecolor{currentfill}{rgb}{1.000000,0.705882,0.509804}%
\pgfsetfillcolor{currentfill}%
\pgfsetlinewidth{0.481800pt}%
\definecolor{currentstroke}{rgb}{1.000000,1.000000,1.000000}%
\pgfsetstrokecolor{currentstroke}%
\pgfsetdash{}{0pt}%
\pgfpathmoveto{\pgfqpoint{4.390885in}{4.205008in}}%
\pgfpathcurveto{\pgfqpoint{4.401935in}{4.205008in}}{\pgfqpoint{4.412535in}{4.209399in}}{\pgfqpoint{4.420348in}{4.217212in}}%
\pgfpathcurveto{\pgfqpoint{4.428162in}{4.225026in}}{\pgfqpoint{4.432552in}{4.235625in}}{\pgfqpoint{4.432552in}{4.246675in}}%
\pgfpathcurveto{\pgfqpoint{4.432552in}{4.257725in}}{\pgfqpoint{4.428162in}{4.268324in}}{\pgfqpoint{4.420348in}{4.276138in}}%
\pgfpathcurveto{\pgfqpoint{4.412535in}{4.283951in}}{\pgfqpoint{4.401935in}{4.288342in}}{\pgfqpoint{4.390885in}{4.288342in}}%
\pgfpathcurveto{\pgfqpoint{4.379835in}{4.288342in}}{\pgfqpoint{4.369236in}{4.283951in}}{\pgfqpoint{4.361423in}{4.276138in}}%
\pgfpathcurveto{\pgfqpoint{4.353609in}{4.268324in}}{\pgfqpoint{4.349219in}{4.257725in}}{\pgfqpoint{4.349219in}{4.246675in}}%
\pgfpathcurveto{\pgfqpoint{4.349219in}{4.235625in}}{\pgfqpoint{4.353609in}{4.225026in}}{\pgfqpoint{4.361423in}{4.217212in}}%
\pgfpathcurveto{\pgfqpoint{4.369236in}{4.209399in}}{\pgfqpoint{4.379835in}{4.205008in}}{\pgfqpoint{4.390885in}{4.205008in}}%
\pgfpathclose%
\pgfusepath{stroke,fill}%
\end{pgfscope}%
\begin{pgfscope}%
\pgfpathrectangle{\pgfqpoint{0.481978in}{0.331635in}}{\pgfqpoint{9.300000in}{7.700000in}}%
\pgfusepath{clip}%
\pgfsetbuttcap%
\pgfsetroundjoin%
\definecolor{currentfill}{rgb}{1.000000,0.705882,0.509804}%
\pgfsetfillcolor{currentfill}%
\pgfsetlinewidth{0.481800pt}%
\definecolor{currentstroke}{rgb}{1.000000,1.000000,1.000000}%
\pgfsetstrokecolor{currentstroke}%
\pgfsetdash{}{0pt}%
\pgfpathmoveto{\pgfqpoint{2.103399in}{2.744470in}}%
\pgfpathcurveto{\pgfqpoint{2.114449in}{2.744470in}}{\pgfqpoint{2.125048in}{2.748860in}}{\pgfqpoint{2.132862in}{2.756674in}}%
\pgfpathcurveto{\pgfqpoint{2.140676in}{2.764488in}}{\pgfqpoint{2.145066in}{2.775087in}}{\pgfqpoint{2.145066in}{2.786137in}}%
\pgfpathcurveto{\pgfqpoint{2.145066in}{2.797187in}}{\pgfqpoint{2.140676in}{2.807786in}}{\pgfqpoint{2.132862in}{2.815599in}}%
\pgfpathcurveto{\pgfqpoint{2.125048in}{2.823413in}}{\pgfqpoint{2.114449in}{2.827803in}}{\pgfqpoint{2.103399in}{2.827803in}}%
\pgfpathcurveto{\pgfqpoint{2.092349in}{2.827803in}}{\pgfqpoint{2.081750in}{2.823413in}}{\pgfqpoint{2.073937in}{2.815599in}}%
\pgfpathcurveto{\pgfqpoint{2.066123in}{2.807786in}}{\pgfqpoint{2.061733in}{2.797187in}}{\pgfqpoint{2.061733in}{2.786137in}}%
\pgfpathcurveto{\pgfqpoint{2.061733in}{2.775087in}}{\pgfqpoint{2.066123in}{2.764488in}}{\pgfqpoint{2.073937in}{2.756674in}}%
\pgfpathcurveto{\pgfqpoint{2.081750in}{2.748860in}}{\pgfqpoint{2.092349in}{2.744470in}}{\pgfqpoint{2.103399in}{2.744470in}}%
\pgfpathclose%
\pgfusepath{stroke,fill}%
\end{pgfscope}%
\begin{pgfscope}%
\pgfpathrectangle{\pgfqpoint{0.481978in}{0.331635in}}{\pgfqpoint{9.300000in}{7.700000in}}%
\pgfusepath{clip}%
\pgfsetbuttcap%
\pgfsetroundjoin%
\definecolor{currentfill}{rgb}{1.000000,0.705882,0.509804}%
\pgfsetfillcolor{currentfill}%
\pgfsetlinewidth{0.481800pt}%
\definecolor{currentstroke}{rgb}{1.000000,1.000000,1.000000}%
\pgfsetstrokecolor{currentstroke}%
\pgfsetdash{}{0pt}%
\pgfpathmoveto{\pgfqpoint{3.156352in}{3.653836in}}%
\pgfpathcurveto{\pgfqpoint{3.167402in}{3.653836in}}{\pgfqpoint{3.178001in}{3.658226in}}{\pgfqpoint{3.185815in}{3.666040in}}%
\pgfpathcurveto{\pgfqpoint{3.193628in}{3.673854in}}{\pgfqpoint{3.198018in}{3.684453in}}{\pgfqpoint{3.198018in}{3.695503in}}%
\pgfpathcurveto{\pgfqpoint{3.198018in}{3.706553in}}{\pgfqpoint{3.193628in}{3.717152in}}{\pgfqpoint{3.185815in}{3.724966in}}%
\pgfpathcurveto{\pgfqpoint{3.178001in}{3.732779in}}{\pgfqpoint{3.167402in}{3.737169in}}{\pgfqpoint{3.156352in}{3.737169in}}%
\pgfpathcurveto{\pgfqpoint{3.145302in}{3.737169in}}{\pgfqpoint{3.134703in}{3.732779in}}{\pgfqpoint{3.126889in}{3.724966in}}%
\pgfpathcurveto{\pgfqpoint{3.119075in}{3.717152in}}{\pgfqpoint{3.114685in}{3.706553in}}{\pgfqpoint{3.114685in}{3.695503in}}%
\pgfpathcurveto{\pgfqpoint{3.114685in}{3.684453in}}{\pgfqpoint{3.119075in}{3.673854in}}{\pgfqpoint{3.126889in}{3.666040in}}%
\pgfpathcurveto{\pgfqpoint{3.134703in}{3.658226in}}{\pgfqpoint{3.145302in}{3.653836in}}{\pgfqpoint{3.156352in}{3.653836in}}%
\pgfpathclose%
\pgfusepath{stroke,fill}%
\end{pgfscope}%
\begin{pgfscope}%
\pgfpathrectangle{\pgfqpoint{0.481978in}{0.331635in}}{\pgfqpoint{9.300000in}{7.700000in}}%
\pgfusepath{clip}%
\pgfsetbuttcap%
\pgfsetroundjoin%
\definecolor{currentfill}{rgb}{1.000000,0.705882,0.509804}%
\pgfsetfillcolor{currentfill}%
\pgfsetlinewidth{0.481800pt}%
\definecolor{currentstroke}{rgb}{1.000000,1.000000,1.000000}%
\pgfsetstrokecolor{currentstroke}%
\pgfsetdash{}{0pt}%
\pgfpathmoveto{\pgfqpoint{4.782434in}{5.369392in}}%
\pgfpathcurveto{\pgfqpoint{4.793484in}{5.369392in}}{\pgfqpoint{4.804083in}{5.373782in}}{\pgfqpoint{4.811897in}{5.381596in}}%
\pgfpathcurveto{\pgfqpoint{4.819711in}{5.389410in}}{\pgfqpoint{4.824101in}{5.400009in}}{\pgfqpoint{4.824101in}{5.411059in}}%
\pgfpathcurveto{\pgfqpoint{4.824101in}{5.422109in}}{\pgfqpoint{4.819711in}{5.432708in}}{\pgfqpoint{4.811897in}{5.440522in}}%
\pgfpathcurveto{\pgfqpoint{4.804083in}{5.448335in}}{\pgfqpoint{4.793484in}{5.452725in}}{\pgfqpoint{4.782434in}{5.452725in}}%
\pgfpathcurveto{\pgfqpoint{4.771384in}{5.452725in}}{\pgfqpoint{4.760785in}{5.448335in}}{\pgfqpoint{4.752972in}{5.440522in}}%
\pgfpathcurveto{\pgfqpoint{4.745158in}{5.432708in}}{\pgfqpoint{4.740768in}{5.422109in}}{\pgfqpoint{4.740768in}{5.411059in}}%
\pgfpathcurveto{\pgfqpoint{4.740768in}{5.400009in}}{\pgfqpoint{4.745158in}{5.389410in}}{\pgfqpoint{4.752972in}{5.381596in}}%
\pgfpathcurveto{\pgfqpoint{4.760785in}{5.373782in}}{\pgfqpoint{4.771384in}{5.369392in}}{\pgfqpoint{4.782434in}{5.369392in}}%
\pgfpathclose%
\pgfusepath{stroke,fill}%
\end{pgfscope}%
\begin{pgfscope}%
\pgfpathrectangle{\pgfqpoint{0.481978in}{0.331635in}}{\pgfqpoint{9.300000in}{7.700000in}}%
\pgfusepath{clip}%
\pgfsetbuttcap%
\pgfsetroundjoin%
\definecolor{currentfill}{rgb}{1.000000,0.705882,0.509804}%
\pgfsetfillcolor{currentfill}%
\pgfsetlinewidth{0.481800pt}%
\definecolor{currentstroke}{rgb}{1.000000,1.000000,1.000000}%
\pgfsetstrokecolor{currentstroke}%
\pgfsetdash{}{0pt}%
\pgfpathmoveto{\pgfqpoint{4.313894in}{2.562157in}}%
\pgfpathcurveto{\pgfqpoint{4.324944in}{2.562157in}}{\pgfqpoint{4.335543in}{2.566547in}}{\pgfqpoint{4.343356in}{2.574361in}}%
\pgfpathcurveto{\pgfqpoint{4.351170in}{2.582174in}}{\pgfqpoint{4.355560in}{2.592773in}}{\pgfqpoint{4.355560in}{2.603823in}}%
\pgfpathcurveto{\pgfqpoint{4.355560in}{2.614874in}}{\pgfqpoint{4.351170in}{2.625473in}}{\pgfqpoint{4.343356in}{2.633286in}}%
\pgfpathcurveto{\pgfqpoint{4.335543in}{2.641100in}}{\pgfqpoint{4.324944in}{2.645490in}}{\pgfqpoint{4.313894in}{2.645490in}}%
\pgfpathcurveto{\pgfqpoint{4.302843in}{2.645490in}}{\pgfqpoint{4.292244in}{2.641100in}}{\pgfqpoint{4.284431in}{2.633286in}}%
\pgfpathcurveto{\pgfqpoint{4.276617in}{2.625473in}}{\pgfqpoint{4.272227in}{2.614874in}}{\pgfqpoint{4.272227in}{2.603823in}}%
\pgfpathcurveto{\pgfqpoint{4.272227in}{2.592773in}}{\pgfqpoint{4.276617in}{2.582174in}}{\pgfqpoint{4.284431in}{2.574361in}}%
\pgfpathcurveto{\pgfqpoint{4.292244in}{2.566547in}}{\pgfqpoint{4.302843in}{2.562157in}}{\pgfqpoint{4.313894in}{2.562157in}}%
\pgfpathclose%
\pgfusepath{stroke,fill}%
\end{pgfscope}%
\begin{pgfscope}%
\pgfpathrectangle{\pgfqpoint{0.481978in}{0.331635in}}{\pgfqpoint{9.300000in}{7.700000in}}%
\pgfusepath{clip}%
\pgfsetbuttcap%
\pgfsetroundjoin%
\definecolor{currentfill}{rgb}{1.000000,0.705882,0.509804}%
\pgfsetfillcolor{currentfill}%
\pgfsetlinewidth{0.481800pt}%
\definecolor{currentstroke}{rgb}{1.000000,1.000000,1.000000}%
\pgfsetstrokecolor{currentstroke}%
\pgfsetdash{}{0pt}%
\pgfpathmoveto{\pgfqpoint{3.534223in}{5.881213in}}%
\pgfpathcurveto{\pgfqpoint{3.545273in}{5.881213in}}{\pgfqpoint{3.555872in}{5.885603in}}{\pgfqpoint{3.563686in}{5.893417in}}%
\pgfpathcurveto{\pgfqpoint{3.571500in}{5.901230in}}{\pgfqpoint{3.575890in}{5.911829in}}{\pgfqpoint{3.575890in}{5.922879in}}%
\pgfpathcurveto{\pgfqpoint{3.575890in}{5.933929in}}{\pgfqpoint{3.571500in}{5.944528in}}{\pgfqpoint{3.563686in}{5.952342in}}%
\pgfpathcurveto{\pgfqpoint{3.555872in}{5.960156in}}{\pgfqpoint{3.545273in}{5.964546in}}{\pgfqpoint{3.534223in}{5.964546in}}%
\pgfpathcurveto{\pgfqpoint{3.523173in}{5.964546in}}{\pgfqpoint{3.512574in}{5.960156in}}{\pgfqpoint{3.504760in}{5.952342in}}%
\pgfpathcurveto{\pgfqpoint{3.496947in}{5.944528in}}{\pgfqpoint{3.492556in}{5.933929in}}{\pgfqpoint{3.492556in}{5.922879in}}%
\pgfpathcurveto{\pgfqpoint{3.492556in}{5.911829in}}{\pgfqpoint{3.496947in}{5.901230in}}{\pgfqpoint{3.504760in}{5.893417in}}%
\pgfpathcurveto{\pgfqpoint{3.512574in}{5.885603in}}{\pgfqpoint{3.523173in}{5.881213in}}{\pgfqpoint{3.534223in}{5.881213in}}%
\pgfpathclose%
\pgfusepath{stroke,fill}%
\end{pgfscope}%
\begin{pgfscope}%
\pgfpathrectangle{\pgfqpoint{0.481978in}{0.331635in}}{\pgfqpoint{9.300000in}{7.700000in}}%
\pgfusepath{clip}%
\pgfsetbuttcap%
\pgfsetroundjoin%
\definecolor{currentfill}{rgb}{1.000000,0.705882,0.509804}%
\pgfsetfillcolor{currentfill}%
\pgfsetlinewidth{0.481800pt}%
\definecolor{currentstroke}{rgb}{1.000000,1.000000,1.000000}%
\pgfsetstrokecolor{currentstroke}%
\pgfsetdash{}{0pt}%
\pgfpathmoveto{\pgfqpoint{2.718188in}{5.658172in}}%
\pgfpathcurveto{\pgfqpoint{2.729238in}{5.658172in}}{\pgfqpoint{2.739837in}{5.662562in}}{\pgfqpoint{2.747651in}{5.670376in}}%
\pgfpathcurveto{\pgfqpoint{2.755465in}{5.678190in}}{\pgfqpoint{2.759855in}{5.688789in}}{\pgfqpoint{2.759855in}{5.699839in}}%
\pgfpathcurveto{\pgfqpoint{2.759855in}{5.710889in}}{\pgfqpoint{2.755465in}{5.721488in}}{\pgfqpoint{2.747651in}{5.729302in}}%
\pgfpathcurveto{\pgfqpoint{2.739837in}{5.737115in}}{\pgfqpoint{2.729238in}{5.741506in}}{\pgfqpoint{2.718188in}{5.741506in}}%
\pgfpathcurveto{\pgfqpoint{2.707138in}{5.741506in}}{\pgfqpoint{2.696539in}{5.737115in}}{\pgfqpoint{2.688726in}{5.729302in}}%
\pgfpathcurveto{\pgfqpoint{2.680912in}{5.721488in}}{\pgfqpoint{2.676522in}{5.710889in}}{\pgfqpoint{2.676522in}{5.699839in}}%
\pgfpathcurveto{\pgfqpoint{2.676522in}{5.688789in}}{\pgfqpoint{2.680912in}{5.678190in}}{\pgfqpoint{2.688726in}{5.670376in}}%
\pgfpathcurveto{\pgfqpoint{2.696539in}{5.662562in}}{\pgfqpoint{2.707138in}{5.658172in}}{\pgfqpoint{2.718188in}{5.658172in}}%
\pgfpathclose%
\pgfusepath{stroke,fill}%
\end{pgfscope}%
\begin{pgfscope}%
\pgfpathrectangle{\pgfqpoint{0.481978in}{0.331635in}}{\pgfqpoint{9.300000in}{7.700000in}}%
\pgfusepath{clip}%
\pgfsetbuttcap%
\pgfsetroundjoin%
\definecolor{currentfill}{rgb}{1.000000,0.705882,0.509804}%
\pgfsetfillcolor{currentfill}%
\pgfsetlinewidth{0.481800pt}%
\definecolor{currentstroke}{rgb}{1.000000,1.000000,1.000000}%
\pgfsetstrokecolor{currentstroke}%
\pgfsetdash{}{0pt}%
\pgfpathmoveto{\pgfqpoint{4.798223in}{3.912894in}}%
\pgfpathcurveto{\pgfqpoint{4.809274in}{3.912894in}}{\pgfqpoint{4.819873in}{3.917284in}}{\pgfqpoint{4.827686in}{3.925098in}}%
\pgfpathcurveto{\pgfqpoint{4.835500in}{3.932911in}}{\pgfqpoint{4.839890in}{3.943510in}}{\pgfqpoint{4.839890in}{3.954560in}}%
\pgfpathcurveto{\pgfqpoint{4.839890in}{3.965610in}}{\pgfqpoint{4.835500in}{3.976210in}}{\pgfqpoint{4.827686in}{3.984023in}}%
\pgfpathcurveto{\pgfqpoint{4.819873in}{3.991837in}}{\pgfqpoint{4.809274in}{3.996227in}}{\pgfqpoint{4.798223in}{3.996227in}}%
\pgfpathcurveto{\pgfqpoint{4.787173in}{3.996227in}}{\pgfqpoint{4.776574in}{3.991837in}}{\pgfqpoint{4.768761in}{3.984023in}}%
\pgfpathcurveto{\pgfqpoint{4.760947in}{3.976210in}}{\pgfqpoint{4.756557in}{3.965610in}}{\pgfqpoint{4.756557in}{3.954560in}}%
\pgfpathcurveto{\pgfqpoint{4.756557in}{3.943510in}}{\pgfqpoint{4.760947in}{3.932911in}}{\pgfqpoint{4.768761in}{3.925098in}}%
\pgfpathcurveto{\pgfqpoint{4.776574in}{3.917284in}}{\pgfqpoint{4.787173in}{3.912894in}}{\pgfqpoint{4.798223in}{3.912894in}}%
\pgfpathclose%
\pgfusepath{stroke,fill}%
\end{pgfscope}%
\begin{pgfscope}%
\pgfpathrectangle{\pgfqpoint{0.481978in}{0.331635in}}{\pgfqpoint{9.300000in}{7.700000in}}%
\pgfusepath{clip}%
\pgfsetbuttcap%
\pgfsetroundjoin%
\definecolor{currentfill}{rgb}{1.000000,0.705882,0.509804}%
\pgfsetfillcolor{currentfill}%
\pgfsetlinewidth{0.481800pt}%
\definecolor{currentstroke}{rgb}{1.000000,1.000000,1.000000}%
\pgfsetstrokecolor{currentstroke}%
\pgfsetdash{}{0pt}%
\pgfpathmoveto{\pgfqpoint{2.688115in}{4.275920in}}%
\pgfpathcurveto{\pgfqpoint{2.699165in}{4.275920in}}{\pgfqpoint{2.709764in}{4.280311in}}{\pgfqpoint{2.717578in}{4.288124in}}%
\pgfpathcurveto{\pgfqpoint{2.725392in}{4.295938in}}{\pgfqpoint{2.729782in}{4.306537in}}{\pgfqpoint{2.729782in}{4.317587in}}%
\pgfpathcurveto{\pgfqpoint{2.729782in}{4.328637in}}{\pgfqpoint{2.725392in}{4.339236in}}{\pgfqpoint{2.717578in}{4.347050in}}%
\pgfpathcurveto{\pgfqpoint{2.709764in}{4.354863in}}{\pgfqpoint{2.699165in}{4.359254in}}{\pgfqpoint{2.688115in}{4.359254in}}%
\pgfpathcurveto{\pgfqpoint{2.677065in}{4.359254in}}{\pgfqpoint{2.666466in}{4.354863in}}{\pgfqpoint{2.658652in}{4.347050in}}%
\pgfpathcurveto{\pgfqpoint{2.650839in}{4.339236in}}{\pgfqpoint{2.646449in}{4.328637in}}{\pgfqpoint{2.646449in}{4.317587in}}%
\pgfpathcurveto{\pgfqpoint{2.646449in}{4.306537in}}{\pgfqpoint{2.650839in}{4.295938in}}{\pgfqpoint{2.658652in}{4.288124in}}%
\pgfpathcurveto{\pgfqpoint{2.666466in}{4.280311in}}{\pgfqpoint{2.677065in}{4.275920in}}{\pgfqpoint{2.688115in}{4.275920in}}%
\pgfpathclose%
\pgfusepath{stroke,fill}%
\end{pgfscope}%
\begin{pgfscope}%
\pgfpathrectangle{\pgfqpoint{0.481978in}{0.331635in}}{\pgfqpoint{9.300000in}{7.700000in}}%
\pgfusepath{clip}%
\pgfsetbuttcap%
\pgfsetroundjoin%
\definecolor{currentfill}{rgb}{1.000000,0.705882,0.509804}%
\pgfsetfillcolor{currentfill}%
\pgfsetlinewidth{0.481800pt}%
\definecolor{currentstroke}{rgb}{1.000000,1.000000,1.000000}%
\pgfsetstrokecolor{currentstroke}%
\pgfsetdash{}{0pt}%
\pgfpathmoveto{\pgfqpoint{3.356009in}{3.291155in}}%
\pgfpathcurveto{\pgfqpoint{3.367060in}{3.291155in}}{\pgfqpoint{3.377659in}{3.295545in}}{\pgfqpoint{3.385472in}{3.303359in}}%
\pgfpathcurveto{\pgfqpoint{3.393286in}{3.311173in}}{\pgfqpoint{3.397676in}{3.321772in}}{\pgfqpoint{3.397676in}{3.332822in}}%
\pgfpathcurveto{\pgfqpoint{3.397676in}{3.343872in}}{\pgfqpoint{3.393286in}{3.354471in}}{\pgfqpoint{3.385472in}{3.362285in}}%
\pgfpathcurveto{\pgfqpoint{3.377659in}{3.370098in}}{\pgfqpoint{3.367060in}{3.374488in}}{\pgfqpoint{3.356009in}{3.374488in}}%
\pgfpathcurveto{\pgfqpoint{3.344959in}{3.374488in}}{\pgfqpoint{3.334360in}{3.370098in}}{\pgfqpoint{3.326547in}{3.362285in}}%
\pgfpathcurveto{\pgfqpoint{3.318733in}{3.354471in}}{\pgfqpoint{3.314343in}{3.343872in}}{\pgfqpoint{3.314343in}{3.332822in}}%
\pgfpathcurveto{\pgfqpoint{3.314343in}{3.321772in}}{\pgfqpoint{3.318733in}{3.311173in}}{\pgfqpoint{3.326547in}{3.303359in}}%
\pgfpathcurveto{\pgfqpoint{3.334360in}{3.295545in}}{\pgfqpoint{3.344959in}{3.291155in}}{\pgfqpoint{3.356009in}{3.291155in}}%
\pgfpathclose%
\pgfusepath{stroke,fill}%
\end{pgfscope}%
\begin{pgfscope}%
\pgfpathrectangle{\pgfqpoint{0.481978in}{0.331635in}}{\pgfqpoint{9.300000in}{7.700000in}}%
\pgfusepath{clip}%
\pgfsetbuttcap%
\pgfsetroundjoin%
\definecolor{currentfill}{rgb}{1.000000,0.705882,0.509804}%
\pgfsetfillcolor{currentfill}%
\pgfsetlinewidth{0.481800pt}%
\definecolor{currentstroke}{rgb}{1.000000,1.000000,1.000000}%
\pgfsetstrokecolor{currentstroke}%
\pgfsetdash{}{0pt}%
\pgfpathmoveto{\pgfqpoint{2.750103in}{4.713490in}}%
\pgfpathcurveto{\pgfqpoint{2.761153in}{4.713490in}}{\pgfqpoint{2.771752in}{4.717880in}}{\pgfqpoint{2.779565in}{4.725694in}}%
\pgfpathcurveto{\pgfqpoint{2.787379in}{4.733507in}}{\pgfqpoint{2.791769in}{4.744106in}}{\pgfqpoint{2.791769in}{4.755156in}}%
\pgfpathcurveto{\pgfqpoint{2.791769in}{4.766206in}}{\pgfqpoint{2.787379in}{4.776805in}}{\pgfqpoint{2.779565in}{4.784619in}}%
\pgfpathcurveto{\pgfqpoint{2.771752in}{4.792433in}}{\pgfqpoint{2.761153in}{4.796823in}}{\pgfqpoint{2.750103in}{4.796823in}}%
\pgfpathcurveto{\pgfqpoint{2.739052in}{4.796823in}}{\pgfqpoint{2.728453in}{4.792433in}}{\pgfqpoint{2.720640in}{4.784619in}}%
\pgfpathcurveto{\pgfqpoint{2.712826in}{4.776805in}}{\pgfqpoint{2.708436in}{4.766206in}}{\pgfqpoint{2.708436in}{4.755156in}}%
\pgfpathcurveto{\pgfqpoint{2.708436in}{4.744106in}}{\pgfqpoint{2.712826in}{4.733507in}}{\pgfqpoint{2.720640in}{4.725694in}}%
\pgfpathcurveto{\pgfqpoint{2.728453in}{4.717880in}}{\pgfqpoint{2.739052in}{4.713490in}}{\pgfqpoint{2.750103in}{4.713490in}}%
\pgfpathclose%
\pgfusepath{stroke,fill}%
\end{pgfscope}%
\begin{pgfscope}%
\pgfpathrectangle{\pgfqpoint{0.481978in}{0.331635in}}{\pgfqpoint{9.300000in}{7.700000in}}%
\pgfusepath{clip}%
\pgfsetbuttcap%
\pgfsetroundjoin%
\definecolor{currentfill}{rgb}{1.000000,0.705882,0.509804}%
\pgfsetfillcolor{currentfill}%
\pgfsetlinewidth{0.481800pt}%
\definecolor{currentstroke}{rgb}{1.000000,1.000000,1.000000}%
\pgfsetstrokecolor{currentstroke}%
\pgfsetdash{}{0pt}%
\pgfpathmoveto{\pgfqpoint{3.502659in}{2.513832in}}%
\pgfpathcurveto{\pgfqpoint{3.513709in}{2.513832in}}{\pgfqpoint{3.524308in}{2.518222in}}{\pgfqpoint{3.532122in}{2.526036in}}%
\pgfpathcurveto{\pgfqpoint{3.539935in}{2.533850in}}{\pgfqpoint{3.544326in}{2.544449in}}{\pgfqpoint{3.544326in}{2.555499in}}%
\pgfpathcurveto{\pgfqpoint{3.544326in}{2.566549in}}{\pgfqpoint{3.539935in}{2.577148in}}{\pgfqpoint{3.532122in}{2.584962in}}%
\pgfpathcurveto{\pgfqpoint{3.524308in}{2.592775in}}{\pgfqpoint{3.513709in}{2.597166in}}{\pgfqpoint{3.502659in}{2.597166in}}%
\pgfpathcurveto{\pgfqpoint{3.491609in}{2.597166in}}{\pgfqpoint{3.481010in}{2.592775in}}{\pgfqpoint{3.473196in}{2.584962in}}%
\pgfpathcurveto{\pgfqpoint{3.465383in}{2.577148in}}{\pgfqpoint{3.460992in}{2.566549in}}{\pgfqpoint{3.460992in}{2.555499in}}%
\pgfpathcurveto{\pgfqpoint{3.460992in}{2.544449in}}{\pgfqpoint{3.465383in}{2.533850in}}{\pgfqpoint{3.473196in}{2.526036in}}%
\pgfpathcurveto{\pgfqpoint{3.481010in}{2.518222in}}{\pgfqpoint{3.491609in}{2.513832in}}{\pgfqpoint{3.502659in}{2.513832in}}%
\pgfpathclose%
\pgfusepath{stroke,fill}%
\end{pgfscope}%
\begin{pgfscope}%
\pgfpathrectangle{\pgfqpoint{0.481978in}{0.331635in}}{\pgfqpoint{9.300000in}{7.700000in}}%
\pgfusepath{clip}%
\pgfsetbuttcap%
\pgfsetroundjoin%
\definecolor{currentfill}{rgb}{1.000000,0.705882,0.509804}%
\pgfsetfillcolor{currentfill}%
\pgfsetlinewidth{0.481800pt}%
\definecolor{currentstroke}{rgb}{1.000000,1.000000,1.000000}%
\pgfsetstrokecolor{currentstroke}%
\pgfsetdash{}{0pt}%
\pgfpathmoveto{\pgfqpoint{3.883915in}{4.705845in}}%
\pgfpathcurveto{\pgfqpoint{3.894965in}{4.705845in}}{\pgfqpoint{3.905564in}{4.710236in}}{\pgfqpoint{3.913378in}{4.718049in}}%
\pgfpathcurveto{\pgfqpoint{3.921192in}{4.725863in}}{\pgfqpoint{3.925582in}{4.736462in}}{\pgfqpoint{3.925582in}{4.747512in}}%
\pgfpathcurveto{\pgfqpoint{3.925582in}{4.758562in}}{\pgfqpoint{3.921192in}{4.769161in}}{\pgfqpoint{3.913378in}{4.776975in}}%
\pgfpathcurveto{\pgfqpoint{3.905564in}{4.784788in}}{\pgfqpoint{3.894965in}{4.789179in}}{\pgfqpoint{3.883915in}{4.789179in}}%
\pgfpathcurveto{\pgfqpoint{3.872865in}{4.789179in}}{\pgfqpoint{3.862266in}{4.784788in}}{\pgfqpoint{3.854452in}{4.776975in}}%
\pgfpathcurveto{\pgfqpoint{3.846639in}{4.769161in}}{\pgfqpoint{3.842249in}{4.758562in}}{\pgfqpoint{3.842249in}{4.747512in}}%
\pgfpathcurveto{\pgfqpoint{3.842249in}{4.736462in}}{\pgfqpoint{3.846639in}{4.725863in}}{\pgfqpoint{3.854452in}{4.718049in}}%
\pgfpathcurveto{\pgfqpoint{3.862266in}{4.710236in}}{\pgfqpoint{3.872865in}{4.705845in}}{\pgfqpoint{3.883915in}{4.705845in}}%
\pgfpathclose%
\pgfusepath{stroke,fill}%
\end{pgfscope}%
\begin{pgfscope}%
\pgfpathrectangle{\pgfqpoint{0.481978in}{0.331635in}}{\pgfqpoint{9.300000in}{7.700000in}}%
\pgfusepath{clip}%
\pgfsetbuttcap%
\pgfsetroundjoin%
\definecolor{currentfill}{rgb}{1.000000,0.705882,0.509804}%
\pgfsetfillcolor{currentfill}%
\pgfsetlinewidth{0.481800pt}%
\definecolor{currentstroke}{rgb}{1.000000,1.000000,1.000000}%
\pgfsetstrokecolor{currentstroke}%
\pgfsetdash{}{0pt}%
\pgfpathmoveto{\pgfqpoint{4.261705in}{2.302620in}}%
\pgfpathcurveto{\pgfqpoint{4.272755in}{2.302620in}}{\pgfqpoint{4.283354in}{2.307010in}}{\pgfqpoint{4.291168in}{2.314824in}}%
\pgfpathcurveto{\pgfqpoint{4.298982in}{2.322638in}}{\pgfqpoint{4.303372in}{2.333237in}}{\pgfqpoint{4.303372in}{2.344287in}}%
\pgfpathcurveto{\pgfqpoint{4.303372in}{2.355337in}}{\pgfqpoint{4.298982in}{2.365936in}}{\pgfqpoint{4.291168in}{2.373750in}}%
\pgfpathcurveto{\pgfqpoint{4.283354in}{2.381563in}}{\pgfqpoint{4.272755in}{2.385953in}}{\pgfqpoint{4.261705in}{2.385953in}}%
\pgfpathcurveto{\pgfqpoint{4.250655in}{2.385953in}}{\pgfqpoint{4.240056in}{2.381563in}}{\pgfqpoint{4.232243in}{2.373750in}}%
\pgfpathcurveto{\pgfqpoint{4.224429in}{2.365936in}}{\pgfqpoint{4.220039in}{2.355337in}}{\pgfqpoint{4.220039in}{2.344287in}}%
\pgfpathcurveto{\pgfqpoint{4.220039in}{2.333237in}}{\pgfqpoint{4.224429in}{2.322638in}}{\pgfqpoint{4.232243in}{2.314824in}}%
\pgfpathcurveto{\pgfqpoint{4.240056in}{2.307010in}}{\pgfqpoint{4.250655in}{2.302620in}}{\pgfqpoint{4.261705in}{2.302620in}}%
\pgfpathclose%
\pgfusepath{stroke,fill}%
\end{pgfscope}%
\begin{pgfscope}%
\pgfpathrectangle{\pgfqpoint{0.481978in}{0.331635in}}{\pgfqpoint{9.300000in}{7.700000in}}%
\pgfusepath{clip}%
\pgfsetbuttcap%
\pgfsetroundjoin%
\definecolor{currentfill}{rgb}{1.000000,0.705882,0.509804}%
\pgfsetfillcolor{currentfill}%
\pgfsetlinewidth{0.481800pt}%
\definecolor{currentstroke}{rgb}{1.000000,1.000000,1.000000}%
\pgfsetstrokecolor{currentstroke}%
\pgfsetdash{}{0pt}%
\pgfpathmoveto{\pgfqpoint{1.767331in}{4.627608in}}%
\pgfpathcurveto{\pgfqpoint{1.778382in}{4.627608in}}{\pgfqpoint{1.788981in}{4.631998in}}{\pgfqpoint{1.796794in}{4.639811in}}%
\pgfpathcurveto{\pgfqpoint{1.804608in}{4.647625in}}{\pgfqpoint{1.808998in}{4.658224in}}{\pgfqpoint{1.808998in}{4.669274in}}%
\pgfpathcurveto{\pgfqpoint{1.808998in}{4.680324in}}{\pgfqpoint{1.804608in}{4.690923in}}{\pgfqpoint{1.796794in}{4.698737in}}%
\pgfpathcurveto{\pgfqpoint{1.788981in}{4.706551in}}{\pgfqpoint{1.778382in}{4.710941in}}{\pgfqpoint{1.767331in}{4.710941in}}%
\pgfpathcurveto{\pgfqpoint{1.756281in}{4.710941in}}{\pgfqpoint{1.745682in}{4.706551in}}{\pgfqpoint{1.737869in}{4.698737in}}%
\pgfpathcurveto{\pgfqpoint{1.730055in}{4.690923in}}{\pgfqpoint{1.725665in}{4.680324in}}{\pgfqpoint{1.725665in}{4.669274in}}%
\pgfpathcurveto{\pgfqpoint{1.725665in}{4.658224in}}{\pgfqpoint{1.730055in}{4.647625in}}{\pgfqpoint{1.737869in}{4.639811in}}%
\pgfpathcurveto{\pgfqpoint{1.745682in}{4.631998in}}{\pgfqpoint{1.756281in}{4.627608in}}{\pgfqpoint{1.767331in}{4.627608in}}%
\pgfpathclose%
\pgfusepath{stroke,fill}%
\end{pgfscope}%
\begin{pgfscope}%
\pgfpathrectangle{\pgfqpoint{0.481978in}{0.331635in}}{\pgfqpoint{9.300000in}{7.700000in}}%
\pgfusepath{clip}%
\pgfsetbuttcap%
\pgfsetroundjoin%
\definecolor{currentfill}{rgb}{1.000000,0.705882,0.509804}%
\pgfsetfillcolor{currentfill}%
\pgfsetlinewidth{0.481800pt}%
\definecolor{currentstroke}{rgb}{1.000000,1.000000,1.000000}%
\pgfsetstrokecolor{currentstroke}%
\pgfsetdash{}{0pt}%
\pgfpathmoveto{\pgfqpoint{1.457321in}{4.643687in}}%
\pgfpathcurveto{\pgfqpoint{1.468371in}{4.643687in}}{\pgfqpoint{1.478970in}{4.648077in}}{\pgfqpoint{1.486784in}{4.655891in}}%
\pgfpathcurveto{\pgfqpoint{1.494597in}{4.663705in}}{\pgfqpoint{1.498988in}{4.674304in}}{\pgfqpoint{1.498988in}{4.685354in}}%
\pgfpathcurveto{\pgfqpoint{1.498988in}{4.696404in}}{\pgfqpoint{1.494597in}{4.707003in}}{\pgfqpoint{1.486784in}{4.714817in}}%
\pgfpathcurveto{\pgfqpoint{1.478970in}{4.722630in}}{\pgfqpoint{1.468371in}{4.727020in}}{\pgfqpoint{1.457321in}{4.727020in}}%
\pgfpathcurveto{\pgfqpoint{1.446271in}{4.727020in}}{\pgfqpoint{1.435672in}{4.722630in}}{\pgfqpoint{1.427858in}{4.714817in}}%
\pgfpathcurveto{\pgfqpoint{1.420045in}{4.707003in}}{\pgfqpoint{1.415654in}{4.696404in}}{\pgfqpoint{1.415654in}{4.685354in}}%
\pgfpathcurveto{\pgfqpoint{1.415654in}{4.674304in}}{\pgfqpoint{1.420045in}{4.663705in}}{\pgfqpoint{1.427858in}{4.655891in}}%
\pgfpathcurveto{\pgfqpoint{1.435672in}{4.648077in}}{\pgfqpoint{1.446271in}{4.643687in}}{\pgfqpoint{1.457321in}{4.643687in}}%
\pgfpathclose%
\pgfusepath{stroke,fill}%
\end{pgfscope}%
\begin{pgfscope}%
\pgfpathrectangle{\pgfqpoint{0.481978in}{0.331635in}}{\pgfqpoint{9.300000in}{7.700000in}}%
\pgfusepath{clip}%
\pgfsetbuttcap%
\pgfsetroundjoin%
\definecolor{currentfill}{rgb}{1.000000,0.705882,0.509804}%
\pgfsetfillcolor{currentfill}%
\pgfsetlinewidth{0.481800pt}%
\definecolor{currentstroke}{rgb}{1.000000,1.000000,1.000000}%
\pgfsetstrokecolor{currentstroke}%
\pgfsetdash{}{0pt}%
\pgfpathmoveto{\pgfqpoint{3.872133in}{3.246559in}}%
\pgfpathcurveto{\pgfqpoint{3.883183in}{3.246559in}}{\pgfqpoint{3.893782in}{3.250949in}}{\pgfqpoint{3.901596in}{3.258763in}}%
\pgfpathcurveto{\pgfqpoint{3.909410in}{3.266576in}}{\pgfqpoint{3.913800in}{3.277175in}}{\pgfqpoint{3.913800in}{3.288225in}}%
\pgfpathcurveto{\pgfqpoint{3.913800in}{3.299276in}}{\pgfqpoint{3.909410in}{3.309875in}}{\pgfqpoint{3.901596in}{3.317688in}}%
\pgfpathcurveto{\pgfqpoint{3.893782in}{3.325502in}}{\pgfqpoint{3.883183in}{3.329892in}}{\pgfqpoint{3.872133in}{3.329892in}}%
\pgfpathcurveto{\pgfqpoint{3.861083in}{3.329892in}}{\pgfqpoint{3.850484in}{3.325502in}}{\pgfqpoint{3.842670in}{3.317688in}}%
\pgfpathcurveto{\pgfqpoint{3.834857in}{3.309875in}}{\pgfqpoint{3.830466in}{3.299276in}}{\pgfqpoint{3.830466in}{3.288225in}}%
\pgfpathcurveto{\pgfqpoint{3.830466in}{3.277175in}}{\pgfqpoint{3.834857in}{3.266576in}}{\pgfqpoint{3.842670in}{3.258763in}}%
\pgfpathcurveto{\pgfqpoint{3.850484in}{3.250949in}}{\pgfqpoint{3.861083in}{3.246559in}}{\pgfqpoint{3.872133in}{3.246559in}}%
\pgfpathclose%
\pgfusepath{stroke,fill}%
\end{pgfscope}%
\begin{pgfscope}%
\pgfpathrectangle{\pgfqpoint{0.481978in}{0.331635in}}{\pgfqpoint{9.300000in}{7.700000in}}%
\pgfusepath{clip}%
\pgfsetbuttcap%
\pgfsetroundjoin%
\definecolor{currentfill}{rgb}{1.000000,0.705882,0.509804}%
\pgfsetfillcolor{currentfill}%
\pgfsetlinewidth{0.481800pt}%
\definecolor{currentstroke}{rgb}{1.000000,1.000000,1.000000}%
\pgfsetstrokecolor{currentstroke}%
\pgfsetdash{}{0pt}%
\pgfpathmoveto{\pgfqpoint{3.604393in}{3.059359in}}%
\pgfpathcurveto{\pgfqpoint{3.615443in}{3.059359in}}{\pgfqpoint{3.626042in}{3.063750in}}{\pgfqpoint{3.633855in}{3.071563in}}%
\pgfpathcurveto{\pgfqpoint{3.641669in}{3.079377in}}{\pgfqpoint{3.646059in}{3.089976in}}{\pgfqpoint{3.646059in}{3.101026in}}%
\pgfpathcurveto{\pgfqpoint{3.646059in}{3.112076in}}{\pgfqpoint{3.641669in}{3.122675in}}{\pgfqpoint{3.633855in}{3.130489in}}%
\pgfpathcurveto{\pgfqpoint{3.626042in}{3.138302in}}{\pgfqpoint{3.615443in}{3.142693in}}{\pgfqpoint{3.604393in}{3.142693in}}%
\pgfpathcurveto{\pgfqpoint{3.593343in}{3.142693in}}{\pgfqpoint{3.582744in}{3.138302in}}{\pgfqpoint{3.574930in}{3.130489in}}%
\pgfpathcurveto{\pgfqpoint{3.567116in}{3.122675in}}{\pgfqpoint{3.562726in}{3.112076in}}{\pgfqpoint{3.562726in}{3.101026in}}%
\pgfpathcurveto{\pgfqpoint{3.562726in}{3.089976in}}{\pgfqpoint{3.567116in}{3.079377in}}{\pgfqpoint{3.574930in}{3.071563in}}%
\pgfpathcurveto{\pgfqpoint{3.582744in}{3.063750in}}{\pgfqpoint{3.593343in}{3.059359in}}{\pgfqpoint{3.604393in}{3.059359in}}%
\pgfpathclose%
\pgfusepath{stroke,fill}%
\end{pgfscope}%
\begin{pgfscope}%
\pgfpathrectangle{\pgfqpoint{0.481978in}{0.331635in}}{\pgfqpoint{9.300000in}{7.700000in}}%
\pgfusepath{clip}%
\pgfsetbuttcap%
\pgfsetroundjoin%
\definecolor{currentfill}{rgb}{1.000000,0.705882,0.509804}%
\pgfsetfillcolor{currentfill}%
\pgfsetlinewidth{0.481800pt}%
\definecolor{currentstroke}{rgb}{1.000000,1.000000,1.000000}%
\pgfsetstrokecolor{currentstroke}%
\pgfsetdash{}{0pt}%
\pgfpathmoveto{\pgfqpoint{3.157682in}{4.400131in}}%
\pgfpathcurveto{\pgfqpoint{3.168733in}{4.400131in}}{\pgfqpoint{3.179332in}{4.404521in}}{\pgfqpoint{3.187145in}{4.412334in}}%
\pgfpathcurveto{\pgfqpoint{3.194959in}{4.420148in}}{\pgfqpoint{3.199349in}{4.430747in}}{\pgfqpoint{3.199349in}{4.441797in}}%
\pgfpathcurveto{\pgfqpoint{3.199349in}{4.452847in}}{\pgfqpoint{3.194959in}{4.463446in}}{\pgfqpoint{3.187145in}{4.471260in}}%
\pgfpathcurveto{\pgfqpoint{3.179332in}{4.479074in}}{\pgfqpoint{3.168733in}{4.483464in}}{\pgfqpoint{3.157682in}{4.483464in}}%
\pgfpathcurveto{\pgfqpoint{3.146632in}{4.483464in}}{\pgfqpoint{3.136033in}{4.479074in}}{\pgfqpoint{3.128220in}{4.471260in}}%
\pgfpathcurveto{\pgfqpoint{3.120406in}{4.463446in}}{\pgfqpoint{3.116016in}{4.452847in}}{\pgfqpoint{3.116016in}{4.441797in}}%
\pgfpathcurveto{\pgfqpoint{3.116016in}{4.430747in}}{\pgfqpoint{3.120406in}{4.420148in}}{\pgfqpoint{3.128220in}{4.412334in}}%
\pgfpathcurveto{\pgfqpoint{3.136033in}{4.404521in}}{\pgfqpoint{3.146632in}{4.400131in}}{\pgfqpoint{3.157682in}{4.400131in}}%
\pgfpathclose%
\pgfusepath{stroke,fill}%
\end{pgfscope}%
\begin{pgfscope}%
\pgfpathrectangle{\pgfqpoint{0.481978in}{0.331635in}}{\pgfqpoint{9.300000in}{7.700000in}}%
\pgfusepath{clip}%
\pgfsetbuttcap%
\pgfsetroundjoin%
\definecolor{currentfill}{rgb}{1.000000,0.705882,0.509804}%
\pgfsetfillcolor{currentfill}%
\pgfsetlinewidth{0.481800pt}%
\definecolor{currentstroke}{rgb}{1.000000,1.000000,1.000000}%
\pgfsetstrokecolor{currentstroke}%
\pgfsetdash{}{0pt}%
\pgfpathmoveto{\pgfqpoint{3.462770in}{5.070375in}}%
\pgfpathcurveto{\pgfqpoint{3.473820in}{5.070375in}}{\pgfqpoint{3.484419in}{5.074765in}}{\pgfqpoint{3.492233in}{5.082579in}}%
\pgfpathcurveto{\pgfqpoint{3.500047in}{5.090392in}}{\pgfqpoint{3.504437in}{5.100991in}}{\pgfqpoint{3.504437in}{5.112041in}}%
\pgfpathcurveto{\pgfqpoint{3.504437in}{5.123092in}}{\pgfqpoint{3.500047in}{5.133691in}}{\pgfqpoint{3.492233in}{5.141504in}}%
\pgfpathcurveto{\pgfqpoint{3.484419in}{5.149318in}}{\pgfqpoint{3.473820in}{5.153708in}}{\pgfqpoint{3.462770in}{5.153708in}}%
\pgfpathcurveto{\pgfqpoint{3.451720in}{5.153708in}}{\pgfqpoint{3.441121in}{5.149318in}}{\pgfqpoint{3.433307in}{5.141504in}}%
\pgfpathcurveto{\pgfqpoint{3.425494in}{5.133691in}}{\pgfqpoint{3.421104in}{5.123092in}}{\pgfqpoint{3.421104in}{5.112041in}}%
\pgfpathcurveto{\pgfqpoint{3.421104in}{5.100991in}}{\pgfqpoint{3.425494in}{5.090392in}}{\pgfqpoint{3.433307in}{5.082579in}}%
\pgfpathcurveto{\pgfqpoint{3.441121in}{5.074765in}}{\pgfqpoint{3.451720in}{5.070375in}}{\pgfqpoint{3.462770in}{5.070375in}}%
\pgfpathclose%
\pgfusepath{stroke,fill}%
\end{pgfscope}%
\begin{pgfscope}%
\pgfpathrectangle{\pgfqpoint{0.481978in}{0.331635in}}{\pgfqpoint{9.300000in}{7.700000in}}%
\pgfusepath{clip}%
\pgfsetbuttcap%
\pgfsetroundjoin%
\definecolor{currentfill}{rgb}{1.000000,0.705882,0.509804}%
\pgfsetfillcolor{currentfill}%
\pgfsetlinewidth{0.481800pt}%
\definecolor{currentstroke}{rgb}{1.000000,1.000000,1.000000}%
\pgfsetstrokecolor{currentstroke}%
\pgfsetdash{}{0pt}%
\pgfpathmoveto{\pgfqpoint{4.591082in}{3.522891in}}%
\pgfpathcurveto{\pgfqpoint{4.602132in}{3.522891in}}{\pgfqpoint{4.612731in}{3.527282in}}{\pgfqpoint{4.620545in}{3.535095in}}%
\pgfpathcurveto{\pgfqpoint{4.628358in}{3.542909in}}{\pgfqpoint{4.632749in}{3.553508in}}{\pgfqpoint{4.632749in}{3.564558in}}%
\pgfpathcurveto{\pgfqpoint{4.632749in}{3.575608in}}{\pgfqpoint{4.628358in}{3.586207in}}{\pgfqpoint{4.620545in}{3.594021in}}%
\pgfpathcurveto{\pgfqpoint{4.612731in}{3.601834in}}{\pgfqpoint{4.602132in}{3.606225in}}{\pgfqpoint{4.591082in}{3.606225in}}%
\pgfpathcurveto{\pgfqpoint{4.580032in}{3.606225in}}{\pgfqpoint{4.569433in}{3.601834in}}{\pgfqpoint{4.561619in}{3.594021in}}%
\pgfpathcurveto{\pgfqpoint{4.553806in}{3.586207in}}{\pgfqpoint{4.549415in}{3.575608in}}{\pgfqpoint{4.549415in}{3.564558in}}%
\pgfpathcurveto{\pgfqpoint{4.549415in}{3.553508in}}{\pgfqpoint{4.553806in}{3.542909in}}{\pgfqpoint{4.561619in}{3.535095in}}%
\pgfpathcurveto{\pgfqpoint{4.569433in}{3.527282in}}{\pgfqpoint{4.580032in}{3.522891in}}{\pgfqpoint{4.591082in}{3.522891in}}%
\pgfpathclose%
\pgfusepath{stroke,fill}%
\end{pgfscope}%
\begin{pgfscope}%
\pgfpathrectangle{\pgfqpoint{0.481978in}{0.331635in}}{\pgfqpoint{9.300000in}{7.700000in}}%
\pgfusepath{clip}%
\pgfsetbuttcap%
\pgfsetroundjoin%
\definecolor{currentfill}{rgb}{1.000000,0.705882,0.509804}%
\pgfsetfillcolor{currentfill}%
\pgfsetlinewidth{0.481800pt}%
\definecolor{currentstroke}{rgb}{1.000000,1.000000,1.000000}%
\pgfsetstrokecolor{currentstroke}%
\pgfsetdash{}{0pt}%
\pgfpathmoveto{\pgfqpoint{5.774624in}{4.557577in}}%
\pgfpathcurveto{\pgfqpoint{5.785674in}{4.557577in}}{\pgfqpoint{5.796274in}{4.561967in}}{\pgfqpoint{5.804087in}{4.569781in}}%
\pgfpathcurveto{\pgfqpoint{5.811901in}{4.577595in}}{\pgfqpoint{5.816291in}{4.588194in}}{\pgfqpoint{5.816291in}{4.599244in}}%
\pgfpathcurveto{\pgfqpoint{5.816291in}{4.610294in}}{\pgfqpoint{5.811901in}{4.620893in}}{\pgfqpoint{5.804087in}{4.628707in}}%
\pgfpathcurveto{\pgfqpoint{5.796274in}{4.636520in}}{\pgfqpoint{5.785674in}{4.640911in}}{\pgfqpoint{5.774624in}{4.640911in}}%
\pgfpathcurveto{\pgfqpoint{5.763574in}{4.640911in}}{\pgfqpoint{5.752975in}{4.636520in}}{\pgfqpoint{5.745162in}{4.628707in}}%
\pgfpathcurveto{\pgfqpoint{5.737348in}{4.620893in}}{\pgfqpoint{5.732958in}{4.610294in}}{\pgfqpoint{5.732958in}{4.599244in}}%
\pgfpathcurveto{\pgfqpoint{5.732958in}{4.588194in}}{\pgfqpoint{5.737348in}{4.577595in}}{\pgfqpoint{5.745162in}{4.569781in}}%
\pgfpathcurveto{\pgfqpoint{5.752975in}{4.561967in}}{\pgfqpoint{5.763574in}{4.557577in}}{\pgfqpoint{5.774624in}{4.557577in}}%
\pgfpathclose%
\pgfusepath{stroke,fill}%
\end{pgfscope}%
\begin{pgfscope}%
\pgfpathrectangle{\pgfqpoint{0.481978in}{0.331635in}}{\pgfqpoint{9.300000in}{7.700000in}}%
\pgfusepath{clip}%
\pgfsetbuttcap%
\pgfsetroundjoin%
\definecolor{currentfill}{rgb}{1.000000,0.705882,0.509804}%
\pgfsetfillcolor{currentfill}%
\pgfsetlinewidth{0.481800pt}%
\definecolor{currentstroke}{rgb}{1.000000,1.000000,1.000000}%
\pgfsetstrokecolor{currentstroke}%
\pgfsetdash{}{0pt}%
\pgfpathmoveto{\pgfqpoint{3.755577in}{2.666678in}}%
\pgfpathcurveto{\pgfqpoint{3.766628in}{2.666678in}}{\pgfqpoint{3.777227in}{2.671068in}}{\pgfqpoint{3.785040in}{2.678882in}}%
\pgfpathcurveto{\pgfqpoint{3.792854in}{2.686696in}}{\pgfqpoint{3.797244in}{2.697295in}}{\pgfqpoint{3.797244in}{2.708345in}}%
\pgfpathcurveto{\pgfqpoint{3.797244in}{2.719395in}}{\pgfqpoint{3.792854in}{2.729994in}}{\pgfqpoint{3.785040in}{2.737808in}}%
\pgfpathcurveto{\pgfqpoint{3.777227in}{2.745621in}}{\pgfqpoint{3.766628in}{2.750012in}}{\pgfqpoint{3.755577in}{2.750012in}}%
\pgfpathcurveto{\pgfqpoint{3.744527in}{2.750012in}}{\pgfqpoint{3.733928in}{2.745621in}}{\pgfqpoint{3.726115in}{2.737808in}}%
\pgfpathcurveto{\pgfqpoint{3.718301in}{2.729994in}}{\pgfqpoint{3.713911in}{2.719395in}}{\pgfqpoint{3.713911in}{2.708345in}}%
\pgfpathcurveto{\pgfqpoint{3.713911in}{2.697295in}}{\pgfqpoint{3.718301in}{2.686696in}}{\pgfqpoint{3.726115in}{2.678882in}}%
\pgfpathcurveto{\pgfqpoint{3.733928in}{2.671068in}}{\pgfqpoint{3.744527in}{2.666678in}}{\pgfqpoint{3.755577in}{2.666678in}}%
\pgfpathclose%
\pgfusepath{stroke,fill}%
\end{pgfscope}%
\begin{pgfscope}%
\pgfpathrectangle{\pgfqpoint{0.481978in}{0.331635in}}{\pgfqpoint{9.300000in}{7.700000in}}%
\pgfusepath{clip}%
\pgfsetbuttcap%
\pgfsetroundjoin%
\definecolor{currentfill}{rgb}{1.000000,0.705882,0.509804}%
\pgfsetfillcolor{currentfill}%
\pgfsetlinewidth{0.481800pt}%
\definecolor{currentstroke}{rgb}{1.000000,1.000000,1.000000}%
\pgfsetstrokecolor{currentstroke}%
\pgfsetdash{}{0pt}%
\pgfpathmoveto{\pgfqpoint{2.355130in}{2.560250in}}%
\pgfpathcurveto{\pgfqpoint{2.366180in}{2.560250in}}{\pgfqpoint{2.376779in}{2.564640in}}{\pgfqpoint{2.384593in}{2.572454in}}%
\pgfpathcurveto{\pgfqpoint{2.392406in}{2.580267in}}{\pgfqpoint{2.396797in}{2.590866in}}{\pgfqpoint{2.396797in}{2.601916in}}%
\pgfpathcurveto{\pgfqpoint{2.396797in}{2.612967in}}{\pgfqpoint{2.392406in}{2.623566in}}{\pgfqpoint{2.384593in}{2.631379in}}%
\pgfpathcurveto{\pgfqpoint{2.376779in}{2.639193in}}{\pgfqpoint{2.366180in}{2.643583in}}{\pgfqpoint{2.355130in}{2.643583in}}%
\pgfpathcurveto{\pgfqpoint{2.344080in}{2.643583in}}{\pgfqpoint{2.333481in}{2.639193in}}{\pgfqpoint{2.325667in}{2.631379in}}%
\pgfpathcurveto{\pgfqpoint{2.317854in}{2.623566in}}{\pgfqpoint{2.313463in}{2.612967in}}{\pgfqpoint{2.313463in}{2.601916in}}%
\pgfpathcurveto{\pgfqpoint{2.313463in}{2.590866in}}{\pgfqpoint{2.317854in}{2.580267in}}{\pgfqpoint{2.325667in}{2.572454in}}%
\pgfpathcurveto{\pgfqpoint{2.333481in}{2.564640in}}{\pgfqpoint{2.344080in}{2.560250in}}{\pgfqpoint{2.355130in}{2.560250in}}%
\pgfpathclose%
\pgfusepath{stroke,fill}%
\end{pgfscope}%
\begin{pgfscope}%
\pgfpathrectangle{\pgfqpoint{0.481978in}{0.331635in}}{\pgfqpoint{9.300000in}{7.700000in}}%
\pgfusepath{clip}%
\pgfsetbuttcap%
\pgfsetroundjoin%
\definecolor{currentfill}{rgb}{1.000000,0.705882,0.509804}%
\pgfsetfillcolor{currentfill}%
\pgfsetlinewidth{0.481800pt}%
\definecolor{currentstroke}{rgb}{1.000000,1.000000,1.000000}%
\pgfsetstrokecolor{currentstroke}%
\pgfsetdash{}{0pt}%
\pgfpathmoveto{\pgfqpoint{3.074624in}{3.245468in}}%
\pgfpathcurveto{\pgfqpoint{3.085674in}{3.245468in}}{\pgfqpoint{3.096273in}{3.249858in}}{\pgfqpoint{3.104087in}{3.257672in}}%
\pgfpathcurveto{\pgfqpoint{3.111901in}{3.265485in}}{\pgfqpoint{3.116291in}{3.276084in}}{\pgfqpoint{3.116291in}{3.287134in}}%
\pgfpathcurveto{\pgfqpoint{3.116291in}{3.298185in}}{\pgfqpoint{3.111901in}{3.308784in}}{\pgfqpoint{3.104087in}{3.316597in}}%
\pgfpathcurveto{\pgfqpoint{3.096273in}{3.324411in}}{\pgfqpoint{3.085674in}{3.328801in}}{\pgfqpoint{3.074624in}{3.328801in}}%
\pgfpathcurveto{\pgfqpoint{3.063574in}{3.328801in}}{\pgfqpoint{3.052975in}{3.324411in}}{\pgfqpoint{3.045161in}{3.316597in}}%
\pgfpathcurveto{\pgfqpoint{3.037348in}{3.308784in}}{\pgfqpoint{3.032957in}{3.298185in}}{\pgfqpoint{3.032957in}{3.287134in}}%
\pgfpathcurveto{\pgfqpoint{3.032957in}{3.276084in}}{\pgfqpoint{3.037348in}{3.265485in}}{\pgfqpoint{3.045161in}{3.257672in}}%
\pgfpathcurveto{\pgfqpoint{3.052975in}{3.249858in}}{\pgfqpoint{3.063574in}{3.245468in}}{\pgfqpoint{3.074624in}{3.245468in}}%
\pgfpathclose%
\pgfusepath{stroke,fill}%
\end{pgfscope}%
\begin{pgfscope}%
\pgfpathrectangle{\pgfqpoint{0.481978in}{0.331635in}}{\pgfqpoint{9.300000in}{7.700000in}}%
\pgfusepath{clip}%
\pgfsetbuttcap%
\pgfsetroundjoin%
\definecolor{currentfill}{rgb}{1.000000,0.705882,0.509804}%
\pgfsetfillcolor{currentfill}%
\pgfsetlinewidth{0.481800pt}%
\definecolor{currentstroke}{rgb}{1.000000,1.000000,1.000000}%
\pgfsetstrokecolor{currentstroke}%
\pgfsetdash{}{0pt}%
\pgfpathmoveto{\pgfqpoint{2.232057in}{4.434934in}}%
\pgfpathcurveto{\pgfqpoint{2.243108in}{4.434934in}}{\pgfqpoint{2.253707in}{4.439325in}}{\pgfqpoint{2.261520in}{4.447138in}}%
\pgfpathcurveto{\pgfqpoint{2.269334in}{4.454952in}}{\pgfqpoint{2.273724in}{4.465551in}}{\pgfqpoint{2.273724in}{4.476601in}}%
\pgfpathcurveto{\pgfqpoint{2.273724in}{4.487651in}}{\pgfqpoint{2.269334in}{4.498250in}}{\pgfqpoint{2.261520in}{4.506064in}}%
\pgfpathcurveto{\pgfqpoint{2.253707in}{4.513877in}}{\pgfqpoint{2.243108in}{4.518268in}}{\pgfqpoint{2.232057in}{4.518268in}}%
\pgfpathcurveto{\pgfqpoint{2.221007in}{4.518268in}}{\pgfqpoint{2.210408in}{4.513877in}}{\pgfqpoint{2.202595in}{4.506064in}}%
\pgfpathcurveto{\pgfqpoint{2.194781in}{4.498250in}}{\pgfqpoint{2.190391in}{4.487651in}}{\pgfqpoint{2.190391in}{4.476601in}}%
\pgfpathcurveto{\pgfqpoint{2.190391in}{4.465551in}}{\pgfqpoint{2.194781in}{4.454952in}}{\pgfqpoint{2.202595in}{4.447138in}}%
\pgfpathcurveto{\pgfqpoint{2.210408in}{4.439325in}}{\pgfqpoint{2.221007in}{4.434934in}}{\pgfqpoint{2.232057in}{4.434934in}}%
\pgfpathclose%
\pgfusepath{stroke,fill}%
\end{pgfscope}%
\begin{pgfscope}%
\pgfpathrectangle{\pgfqpoint{0.481978in}{0.331635in}}{\pgfqpoint{9.300000in}{7.700000in}}%
\pgfusepath{clip}%
\pgfsetbuttcap%
\pgfsetroundjoin%
\definecolor{currentfill}{rgb}{1.000000,0.705882,0.509804}%
\pgfsetfillcolor{currentfill}%
\pgfsetlinewidth{0.481800pt}%
\definecolor{currentstroke}{rgb}{1.000000,1.000000,1.000000}%
\pgfsetstrokecolor{currentstroke}%
\pgfsetdash{}{0pt}%
\pgfpathmoveto{\pgfqpoint{3.954754in}{3.700619in}}%
\pgfpathcurveto{\pgfqpoint{3.965804in}{3.700619in}}{\pgfqpoint{3.976403in}{3.705009in}}{\pgfqpoint{3.984217in}{3.712823in}}%
\pgfpathcurveto{\pgfqpoint{3.992031in}{3.720636in}}{\pgfqpoint{3.996421in}{3.731235in}}{\pgfqpoint{3.996421in}{3.742286in}}%
\pgfpathcurveto{\pgfqpoint{3.996421in}{3.753336in}}{\pgfqpoint{3.992031in}{3.763935in}}{\pgfqpoint{3.984217in}{3.771748in}}%
\pgfpathcurveto{\pgfqpoint{3.976403in}{3.779562in}}{\pgfqpoint{3.965804in}{3.783952in}}{\pgfqpoint{3.954754in}{3.783952in}}%
\pgfpathcurveto{\pgfqpoint{3.943704in}{3.783952in}}{\pgfqpoint{3.933105in}{3.779562in}}{\pgfqpoint{3.925291in}{3.771748in}}%
\pgfpathcurveto{\pgfqpoint{3.917478in}{3.763935in}}{\pgfqpoint{3.913088in}{3.753336in}}{\pgfqpoint{3.913088in}{3.742286in}}%
\pgfpathcurveto{\pgfqpoint{3.913088in}{3.731235in}}{\pgfqpoint{3.917478in}{3.720636in}}{\pgfqpoint{3.925291in}{3.712823in}}%
\pgfpathcurveto{\pgfqpoint{3.933105in}{3.705009in}}{\pgfqpoint{3.943704in}{3.700619in}}{\pgfqpoint{3.954754in}{3.700619in}}%
\pgfpathclose%
\pgfusepath{stroke,fill}%
\end{pgfscope}%
\begin{pgfscope}%
\pgfpathrectangle{\pgfqpoint{0.481978in}{0.331635in}}{\pgfqpoint{9.300000in}{7.700000in}}%
\pgfusepath{clip}%
\pgfsetbuttcap%
\pgfsetroundjoin%
\definecolor{currentfill}{rgb}{1.000000,0.705882,0.509804}%
\pgfsetfillcolor{currentfill}%
\pgfsetlinewidth{0.481800pt}%
\definecolor{currentstroke}{rgb}{1.000000,1.000000,1.000000}%
\pgfsetstrokecolor{currentstroke}%
\pgfsetdash{}{0pt}%
\pgfpathmoveto{\pgfqpoint{1.921313in}{5.061256in}}%
\pgfpathcurveto{\pgfqpoint{1.932363in}{5.061256in}}{\pgfqpoint{1.942962in}{5.065646in}}{\pgfqpoint{1.950776in}{5.073460in}}%
\pgfpathcurveto{\pgfqpoint{1.958590in}{5.081274in}}{\pgfqpoint{1.962980in}{5.091873in}}{\pgfqpoint{1.962980in}{5.102923in}}%
\pgfpathcurveto{\pgfqpoint{1.962980in}{5.113973in}}{\pgfqpoint{1.958590in}{5.124572in}}{\pgfqpoint{1.950776in}{5.132386in}}%
\pgfpathcurveto{\pgfqpoint{1.942962in}{5.140199in}}{\pgfqpoint{1.932363in}{5.144590in}}{\pgfqpoint{1.921313in}{5.144590in}}%
\pgfpathcurveto{\pgfqpoint{1.910263in}{5.144590in}}{\pgfqpoint{1.899664in}{5.140199in}}{\pgfqpoint{1.891850in}{5.132386in}}%
\pgfpathcurveto{\pgfqpoint{1.884037in}{5.124572in}}{\pgfqpoint{1.879647in}{5.113973in}}{\pgfqpoint{1.879647in}{5.102923in}}%
\pgfpathcurveto{\pgfqpoint{1.879647in}{5.091873in}}{\pgfqpoint{1.884037in}{5.081274in}}{\pgfqpoint{1.891850in}{5.073460in}}%
\pgfpathcurveto{\pgfqpoint{1.899664in}{5.065646in}}{\pgfqpoint{1.910263in}{5.061256in}}{\pgfqpoint{1.921313in}{5.061256in}}%
\pgfpathclose%
\pgfusepath{stroke,fill}%
\end{pgfscope}%
\begin{pgfscope}%
\pgfpathrectangle{\pgfqpoint{0.481978in}{0.331635in}}{\pgfqpoint{9.300000in}{7.700000in}}%
\pgfusepath{clip}%
\pgfsetbuttcap%
\pgfsetroundjoin%
\definecolor{currentfill}{rgb}{1.000000,0.705882,0.509804}%
\pgfsetfillcolor{currentfill}%
\pgfsetlinewidth{0.481800pt}%
\definecolor{currentstroke}{rgb}{1.000000,1.000000,1.000000}%
\pgfsetstrokecolor{currentstroke}%
\pgfsetdash{}{0pt}%
\pgfpathmoveto{\pgfqpoint{3.918352in}{4.402192in}}%
\pgfpathcurveto{\pgfqpoint{3.929402in}{4.402192in}}{\pgfqpoint{3.940001in}{4.406582in}}{\pgfqpoint{3.947815in}{4.414396in}}%
\pgfpathcurveto{\pgfqpoint{3.955628in}{4.422209in}}{\pgfqpoint{3.960018in}{4.432808in}}{\pgfqpoint{3.960018in}{4.443858in}}%
\pgfpathcurveto{\pgfqpoint{3.960018in}{4.454909in}}{\pgfqpoint{3.955628in}{4.465508in}}{\pgfqpoint{3.947815in}{4.473321in}}%
\pgfpathcurveto{\pgfqpoint{3.940001in}{4.481135in}}{\pgfqpoint{3.929402in}{4.485525in}}{\pgfqpoint{3.918352in}{4.485525in}}%
\pgfpathcurveto{\pgfqpoint{3.907302in}{4.485525in}}{\pgfqpoint{3.896703in}{4.481135in}}{\pgfqpoint{3.888889in}{4.473321in}}%
\pgfpathcurveto{\pgfqpoint{3.881075in}{4.465508in}}{\pgfqpoint{3.876685in}{4.454909in}}{\pgfqpoint{3.876685in}{4.443858in}}%
\pgfpathcurveto{\pgfqpoint{3.876685in}{4.432808in}}{\pgfqpoint{3.881075in}{4.422209in}}{\pgfqpoint{3.888889in}{4.414396in}}%
\pgfpathcurveto{\pgfqpoint{3.896703in}{4.406582in}}{\pgfqpoint{3.907302in}{4.402192in}}{\pgfqpoint{3.918352in}{4.402192in}}%
\pgfpathclose%
\pgfusepath{stroke,fill}%
\end{pgfscope}%
\begin{pgfscope}%
\pgfpathrectangle{\pgfqpoint{0.481978in}{0.331635in}}{\pgfqpoint{9.300000in}{7.700000in}}%
\pgfusepath{clip}%
\pgfsetbuttcap%
\pgfsetroundjoin%
\definecolor{currentfill}{rgb}{1.000000,0.705882,0.509804}%
\pgfsetfillcolor{currentfill}%
\pgfsetlinewidth{0.481800pt}%
\definecolor{currentstroke}{rgb}{1.000000,1.000000,1.000000}%
\pgfsetstrokecolor{currentstroke}%
\pgfsetdash{}{0pt}%
\pgfpathmoveto{\pgfqpoint{1.874082in}{4.402216in}}%
\pgfpathcurveto{\pgfqpoint{1.885132in}{4.402216in}}{\pgfqpoint{1.895731in}{4.406606in}}{\pgfqpoint{1.903545in}{4.414420in}}%
\pgfpathcurveto{\pgfqpoint{1.911358in}{4.422234in}}{\pgfqpoint{1.915749in}{4.432833in}}{\pgfqpoint{1.915749in}{4.443883in}}%
\pgfpathcurveto{\pgfqpoint{1.915749in}{4.454933in}}{\pgfqpoint{1.911358in}{4.465532in}}{\pgfqpoint{1.903545in}{4.473346in}}%
\pgfpathcurveto{\pgfqpoint{1.895731in}{4.481159in}}{\pgfqpoint{1.885132in}{4.485549in}}{\pgfqpoint{1.874082in}{4.485549in}}%
\pgfpathcurveto{\pgfqpoint{1.863032in}{4.485549in}}{\pgfqpoint{1.852433in}{4.481159in}}{\pgfqpoint{1.844619in}{4.473346in}}%
\pgfpathcurveto{\pgfqpoint{1.836806in}{4.465532in}}{\pgfqpoint{1.832415in}{4.454933in}}{\pgfqpoint{1.832415in}{4.443883in}}%
\pgfpathcurveto{\pgfqpoint{1.832415in}{4.432833in}}{\pgfqpoint{1.836806in}{4.422234in}}{\pgfqpoint{1.844619in}{4.414420in}}%
\pgfpathcurveto{\pgfqpoint{1.852433in}{4.406606in}}{\pgfqpoint{1.863032in}{4.402216in}}{\pgfqpoint{1.874082in}{4.402216in}}%
\pgfpathclose%
\pgfusepath{stroke,fill}%
\end{pgfscope}%
\begin{pgfscope}%
\pgfpathrectangle{\pgfqpoint{0.481978in}{0.331635in}}{\pgfqpoint{9.300000in}{7.700000in}}%
\pgfusepath{clip}%
\pgfsetbuttcap%
\pgfsetroundjoin%
\definecolor{currentfill}{rgb}{1.000000,0.705882,0.509804}%
\pgfsetfillcolor{currentfill}%
\pgfsetlinewidth{0.481800pt}%
\definecolor{currentstroke}{rgb}{1.000000,1.000000,1.000000}%
\pgfsetstrokecolor{currentstroke}%
\pgfsetdash{}{0pt}%
\pgfpathmoveto{\pgfqpoint{4.966201in}{4.808664in}}%
\pgfpathcurveto{\pgfqpoint{4.977251in}{4.808664in}}{\pgfqpoint{4.987850in}{4.813054in}}{\pgfqpoint{4.995663in}{4.820868in}}%
\pgfpathcurveto{\pgfqpoint{5.003477in}{4.828681in}}{\pgfqpoint{5.007867in}{4.839280in}}{\pgfqpoint{5.007867in}{4.850330in}}%
\pgfpathcurveto{\pgfqpoint{5.007867in}{4.861380in}}{\pgfqpoint{5.003477in}{4.871979in}}{\pgfqpoint{4.995663in}{4.879793in}}%
\pgfpathcurveto{\pgfqpoint{4.987850in}{4.887607in}}{\pgfqpoint{4.977251in}{4.891997in}}{\pgfqpoint{4.966201in}{4.891997in}}%
\pgfpathcurveto{\pgfqpoint{4.955151in}{4.891997in}}{\pgfqpoint{4.944552in}{4.887607in}}{\pgfqpoint{4.936738in}{4.879793in}}%
\pgfpathcurveto{\pgfqpoint{4.928924in}{4.871979in}}{\pgfqpoint{4.924534in}{4.861380in}}{\pgfqpoint{4.924534in}{4.850330in}}%
\pgfpathcurveto{\pgfqpoint{4.924534in}{4.839280in}}{\pgfqpoint{4.928924in}{4.828681in}}{\pgfqpoint{4.936738in}{4.820868in}}%
\pgfpathcurveto{\pgfqpoint{4.944552in}{4.813054in}}{\pgfqpoint{4.955151in}{4.808664in}}{\pgfqpoint{4.966201in}{4.808664in}}%
\pgfpathclose%
\pgfusepath{stroke,fill}%
\end{pgfscope}%
\begin{pgfscope}%
\pgfpathrectangle{\pgfqpoint{0.481978in}{0.331635in}}{\pgfqpoint{9.300000in}{7.700000in}}%
\pgfusepath{clip}%
\pgfsetbuttcap%
\pgfsetroundjoin%
\definecolor{currentfill}{rgb}{0.631373,0.788235,0.956863}%
\pgfsetfillcolor{currentfill}%
\pgfsetlinewidth{1.003750pt}%
\definecolor{currentstroke}{rgb}{0.631373,0.788235,0.956863}%
\pgfsetstrokecolor{currentstroke}%
\pgfsetdash{}{0pt}%
\pgfsys@defobject{currentmarker}{\pgfqpoint{-0.041667in}{-0.041667in}}{\pgfqpoint{0.041667in}{0.041667in}}{%
\pgfpathmoveto{\pgfqpoint{0.000000in}{-0.041667in}}%
\pgfpathcurveto{\pgfqpoint{0.011050in}{-0.041667in}}{\pgfqpoint{0.021649in}{-0.037276in}}{\pgfqpoint{0.029463in}{-0.029463in}}%
\pgfpathcurveto{\pgfqpoint{0.037276in}{-0.021649in}}{\pgfqpoint{0.041667in}{-0.011050in}}{\pgfqpoint{0.041667in}{0.000000in}}%
\pgfpathcurveto{\pgfqpoint{0.041667in}{0.011050in}}{\pgfqpoint{0.037276in}{0.021649in}}{\pgfqpoint{0.029463in}{0.029463in}}%
\pgfpathcurveto{\pgfqpoint{0.021649in}{0.037276in}}{\pgfqpoint{0.011050in}{0.041667in}}{\pgfqpoint{0.000000in}{0.041667in}}%
\pgfpathcurveto{\pgfqpoint{-0.011050in}{0.041667in}}{\pgfqpoint{-0.021649in}{0.037276in}}{\pgfqpoint{-0.029463in}{0.029463in}}%
\pgfpathcurveto{\pgfqpoint{-0.037276in}{0.021649in}}{\pgfqpoint{-0.041667in}{0.011050in}}{\pgfqpoint{-0.041667in}{0.000000in}}%
\pgfpathcurveto{\pgfqpoint{-0.041667in}{-0.011050in}}{\pgfqpoint{-0.037276in}{-0.021649in}}{\pgfqpoint{-0.029463in}{-0.029463in}}%
\pgfpathcurveto{\pgfqpoint{-0.021649in}{-0.037276in}}{\pgfqpoint{-0.011050in}{-0.041667in}}{\pgfqpoint{0.000000in}{-0.041667in}}%
\pgfpathclose%
\pgfusepath{stroke,fill}%
}%
\end{pgfscope}%
\begin{pgfscope}%
\pgfpathrectangle{\pgfqpoint{0.481978in}{0.331635in}}{\pgfqpoint{9.300000in}{7.700000in}}%
\pgfusepath{clip}%
\pgfsetbuttcap%
\pgfsetroundjoin%
\definecolor{currentfill}{rgb}{1.000000,0.705882,0.509804}%
\pgfsetfillcolor{currentfill}%
\pgfsetlinewidth{1.003750pt}%
\definecolor{currentstroke}{rgb}{1.000000,0.705882,0.509804}%
\pgfsetstrokecolor{currentstroke}%
\pgfsetdash{}{0pt}%
\pgfsys@defobject{currentmarker}{\pgfqpoint{-0.041667in}{-0.041667in}}{\pgfqpoint{0.041667in}{0.041667in}}{%
\pgfpathmoveto{\pgfqpoint{0.000000in}{-0.041667in}}%
\pgfpathcurveto{\pgfqpoint{0.011050in}{-0.041667in}}{\pgfqpoint{0.021649in}{-0.037276in}}{\pgfqpoint{0.029463in}{-0.029463in}}%
\pgfpathcurveto{\pgfqpoint{0.037276in}{-0.021649in}}{\pgfqpoint{0.041667in}{-0.011050in}}{\pgfqpoint{0.041667in}{0.000000in}}%
\pgfpathcurveto{\pgfqpoint{0.041667in}{0.011050in}}{\pgfqpoint{0.037276in}{0.021649in}}{\pgfqpoint{0.029463in}{0.029463in}}%
\pgfpathcurveto{\pgfqpoint{0.021649in}{0.037276in}}{\pgfqpoint{0.011050in}{0.041667in}}{\pgfqpoint{0.000000in}{0.041667in}}%
\pgfpathcurveto{\pgfqpoint{-0.011050in}{0.041667in}}{\pgfqpoint{-0.021649in}{0.037276in}}{\pgfqpoint{-0.029463in}{0.029463in}}%
\pgfpathcurveto{\pgfqpoint{-0.037276in}{0.021649in}}{\pgfqpoint{-0.041667in}{0.011050in}}{\pgfqpoint{-0.041667in}{0.000000in}}%
\pgfpathcurveto{\pgfqpoint{-0.041667in}{-0.011050in}}{\pgfqpoint{-0.037276in}{-0.021649in}}{\pgfqpoint{-0.029463in}{-0.029463in}}%
\pgfpathcurveto{\pgfqpoint{-0.021649in}{-0.037276in}}{\pgfqpoint{-0.011050in}{-0.041667in}}{\pgfqpoint{0.000000in}{-0.041667in}}%
\pgfpathclose%
\pgfusepath{stroke,fill}%
}%
\end{pgfscope}%
\begin{pgfscope}%
\pgfsetbuttcap%
\pgfsetroundjoin%
\definecolor{currentfill}{rgb}{0.000000,0.000000,0.000000}%
\pgfsetfillcolor{currentfill}%
\pgfsetlinewidth{0.803000pt}%
\definecolor{currentstroke}{rgb}{0.000000,0.000000,0.000000}%
\pgfsetstrokecolor{currentstroke}%
\pgfsetdash{}{0pt}%
\pgfsys@defobject{currentmarker}{\pgfqpoint{0.000000in}{-0.048611in}}{\pgfqpoint{0.000000in}{0.000000in}}{%
\pgfpathmoveto{\pgfqpoint{0.000000in}{0.000000in}}%
\pgfpathlineto{\pgfqpoint{0.000000in}{-0.048611in}}%
\pgfusepath{stroke,fill}%
}%
\begin{pgfscope}%
\pgfsys@transformshift{0.539104in}{0.331635in}%
\pgfsys@useobject{currentmarker}{}%
\end{pgfscope}%
\end{pgfscope}%
\begin{pgfscope}%
\definecolor{textcolor}{rgb}{0.000000,0.000000,0.000000}%
\pgfsetstrokecolor{textcolor}%
\pgfsetfillcolor{textcolor}%
\pgftext[x=0.539104in,y=0.234413in,,top]{\color{textcolor}\sffamily\fontsize{10.000000}{12.000000}\selectfont \ensuremath{-}30}%
\end{pgfscope}%
\begin{pgfscope}%
\pgfsetbuttcap%
\pgfsetroundjoin%
\definecolor{currentfill}{rgb}{0.000000,0.000000,0.000000}%
\pgfsetfillcolor{currentfill}%
\pgfsetlinewidth{0.803000pt}%
\definecolor{currentstroke}{rgb}{0.000000,0.000000,0.000000}%
\pgfsetstrokecolor{currentstroke}%
\pgfsetdash{}{0pt}%
\pgfsys@defobject{currentmarker}{\pgfqpoint{0.000000in}{-0.048611in}}{\pgfqpoint{0.000000in}{0.000000in}}{%
\pgfpathmoveto{\pgfqpoint{0.000000in}{0.000000in}}%
\pgfpathlineto{\pgfqpoint{0.000000in}{-0.048611in}}%
\pgfusepath{stroke,fill}%
}%
\begin{pgfscope}%
\pgfsys@transformshift{2.011781in}{0.331635in}%
\pgfsys@useobject{currentmarker}{}%
\end{pgfscope}%
\end{pgfscope}%
\begin{pgfscope}%
\definecolor{textcolor}{rgb}{0.000000,0.000000,0.000000}%
\pgfsetstrokecolor{textcolor}%
\pgfsetfillcolor{textcolor}%
\pgftext[x=2.011781in,y=0.234413in,,top]{\color{textcolor}\sffamily\fontsize{10.000000}{12.000000}\selectfont \ensuremath{-}20}%
\end{pgfscope}%
\begin{pgfscope}%
\pgfsetbuttcap%
\pgfsetroundjoin%
\definecolor{currentfill}{rgb}{0.000000,0.000000,0.000000}%
\pgfsetfillcolor{currentfill}%
\pgfsetlinewidth{0.803000pt}%
\definecolor{currentstroke}{rgb}{0.000000,0.000000,0.000000}%
\pgfsetstrokecolor{currentstroke}%
\pgfsetdash{}{0pt}%
\pgfsys@defobject{currentmarker}{\pgfqpoint{0.000000in}{-0.048611in}}{\pgfqpoint{0.000000in}{0.000000in}}{%
\pgfpathmoveto{\pgfqpoint{0.000000in}{0.000000in}}%
\pgfpathlineto{\pgfqpoint{0.000000in}{-0.048611in}}%
\pgfusepath{stroke,fill}%
}%
\begin{pgfscope}%
\pgfsys@transformshift{3.484457in}{0.331635in}%
\pgfsys@useobject{currentmarker}{}%
\end{pgfscope}%
\end{pgfscope}%
\begin{pgfscope}%
\definecolor{textcolor}{rgb}{0.000000,0.000000,0.000000}%
\pgfsetstrokecolor{textcolor}%
\pgfsetfillcolor{textcolor}%
\pgftext[x=3.484457in,y=0.234413in,,top]{\color{textcolor}\sffamily\fontsize{10.000000}{12.000000}\selectfont \ensuremath{-}10}%
\end{pgfscope}%
\begin{pgfscope}%
\pgfsetbuttcap%
\pgfsetroundjoin%
\definecolor{currentfill}{rgb}{0.000000,0.000000,0.000000}%
\pgfsetfillcolor{currentfill}%
\pgfsetlinewidth{0.803000pt}%
\definecolor{currentstroke}{rgb}{0.000000,0.000000,0.000000}%
\pgfsetstrokecolor{currentstroke}%
\pgfsetdash{}{0pt}%
\pgfsys@defobject{currentmarker}{\pgfqpoint{0.000000in}{-0.048611in}}{\pgfqpoint{0.000000in}{0.000000in}}{%
\pgfpathmoveto{\pgfqpoint{0.000000in}{0.000000in}}%
\pgfpathlineto{\pgfqpoint{0.000000in}{-0.048611in}}%
\pgfusepath{stroke,fill}%
}%
\begin{pgfscope}%
\pgfsys@transformshift{4.957134in}{0.331635in}%
\pgfsys@useobject{currentmarker}{}%
\end{pgfscope}%
\end{pgfscope}%
\begin{pgfscope}%
\definecolor{textcolor}{rgb}{0.000000,0.000000,0.000000}%
\pgfsetstrokecolor{textcolor}%
\pgfsetfillcolor{textcolor}%
\pgftext[x=4.957134in,y=0.234413in,,top]{\color{textcolor}\sffamily\fontsize{10.000000}{12.000000}\selectfont 0}%
\end{pgfscope}%
\begin{pgfscope}%
\pgfsetbuttcap%
\pgfsetroundjoin%
\definecolor{currentfill}{rgb}{0.000000,0.000000,0.000000}%
\pgfsetfillcolor{currentfill}%
\pgfsetlinewidth{0.803000pt}%
\definecolor{currentstroke}{rgb}{0.000000,0.000000,0.000000}%
\pgfsetstrokecolor{currentstroke}%
\pgfsetdash{}{0pt}%
\pgfsys@defobject{currentmarker}{\pgfqpoint{0.000000in}{-0.048611in}}{\pgfqpoint{0.000000in}{0.000000in}}{%
\pgfpathmoveto{\pgfqpoint{0.000000in}{0.000000in}}%
\pgfpathlineto{\pgfqpoint{0.000000in}{-0.048611in}}%
\pgfusepath{stroke,fill}%
}%
\begin{pgfscope}%
\pgfsys@transformshift{6.429811in}{0.331635in}%
\pgfsys@useobject{currentmarker}{}%
\end{pgfscope}%
\end{pgfscope}%
\begin{pgfscope}%
\definecolor{textcolor}{rgb}{0.000000,0.000000,0.000000}%
\pgfsetstrokecolor{textcolor}%
\pgfsetfillcolor{textcolor}%
\pgftext[x=6.429811in,y=0.234413in,,top]{\color{textcolor}\sffamily\fontsize{10.000000}{12.000000}\selectfont 10}%
\end{pgfscope}%
\begin{pgfscope}%
\pgfsetbuttcap%
\pgfsetroundjoin%
\definecolor{currentfill}{rgb}{0.000000,0.000000,0.000000}%
\pgfsetfillcolor{currentfill}%
\pgfsetlinewidth{0.803000pt}%
\definecolor{currentstroke}{rgb}{0.000000,0.000000,0.000000}%
\pgfsetstrokecolor{currentstroke}%
\pgfsetdash{}{0pt}%
\pgfsys@defobject{currentmarker}{\pgfqpoint{0.000000in}{-0.048611in}}{\pgfqpoint{0.000000in}{0.000000in}}{%
\pgfpathmoveto{\pgfqpoint{0.000000in}{0.000000in}}%
\pgfpathlineto{\pgfqpoint{0.000000in}{-0.048611in}}%
\pgfusepath{stroke,fill}%
}%
\begin{pgfscope}%
\pgfsys@transformshift{7.902487in}{0.331635in}%
\pgfsys@useobject{currentmarker}{}%
\end{pgfscope}%
\end{pgfscope}%
\begin{pgfscope}%
\definecolor{textcolor}{rgb}{0.000000,0.000000,0.000000}%
\pgfsetstrokecolor{textcolor}%
\pgfsetfillcolor{textcolor}%
\pgftext[x=7.902487in,y=0.234413in,,top]{\color{textcolor}\sffamily\fontsize{10.000000}{12.000000}\selectfont 20}%
\end{pgfscope}%
\begin{pgfscope}%
\pgfsetbuttcap%
\pgfsetroundjoin%
\definecolor{currentfill}{rgb}{0.000000,0.000000,0.000000}%
\pgfsetfillcolor{currentfill}%
\pgfsetlinewidth{0.803000pt}%
\definecolor{currentstroke}{rgb}{0.000000,0.000000,0.000000}%
\pgfsetstrokecolor{currentstroke}%
\pgfsetdash{}{0pt}%
\pgfsys@defobject{currentmarker}{\pgfqpoint{0.000000in}{-0.048611in}}{\pgfqpoint{0.000000in}{0.000000in}}{%
\pgfpathmoveto{\pgfqpoint{0.000000in}{0.000000in}}%
\pgfpathlineto{\pgfqpoint{0.000000in}{-0.048611in}}%
\pgfusepath{stroke,fill}%
}%
\begin{pgfscope}%
\pgfsys@transformshift{9.375164in}{0.331635in}%
\pgfsys@useobject{currentmarker}{}%
\end{pgfscope}%
\end{pgfscope}%
\begin{pgfscope}%
\definecolor{textcolor}{rgb}{0.000000,0.000000,0.000000}%
\pgfsetstrokecolor{textcolor}%
\pgfsetfillcolor{textcolor}%
\pgftext[x=9.375164in,y=0.234413in,,top]{\color{textcolor}\sffamily\fontsize{10.000000}{12.000000}\selectfont 30}%
\end{pgfscope}%
\begin{pgfscope}%
\pgfsetbuttcap%
\pgfsetroundjoin%
\definecolor{currentfill}{rgb}{0.000000,0.000000,0.000000}%
\pgfsetfillcolor{currentfill}%
\pgfsetlinewidth{0.803000pt}%
\definecolor{currentstroke}{rgb}{0.000000,0.000000,0.000000}%
\pgfsetstrokecolor{currentstroke}%
\pgfsetdash{}{0pt}%
\pgfsys@defobject{currentmarker}{\pgfqpoint{-0.048611in}{0.000000in}}{\pgfqpoint{-0.000000in}{0.000000in}}{%
\pgfpathmoveto{\pgfqpoint{-0.000000in}{0.000000in}}%
\pgfpathlineto{\pgfqpoint{-0.048611in}{0.000000in}}%
\pgfusepath{stroke,fill}%
}%
\begin{pgfscope}%
\pgfsys@transformshift{0.481978in}{1.322252in}%
\pgfsys@useobject{currentmarker}{}%
\end{pgfscope}%
\end{pgfscope}%
\begin{pgfscope}%
\definecolor{textcolor}{rgb}{0.000000,0.000000,0.000000}%
\pgfsetstrokecolor{textcolor}%
\pgfsetfillcolor{textcolor}%
\pgftext[x=0.100000in, y=1.269491in, left, base]{\color{textcolor}\sffamily\fontsize{10.000000}{12.000000}\selectfont \ensuremath{-}20}%
\end{pgfscope}%
\begin{pgfscope}%
\pgfsetbuttcap%
\pgfsetroundjoin%
\definecolor{currentfill}{rgb}{0.000000,0.000000,0.000000}%
\pgfsetfillcolor{currentfill}%
\pgfsetlinewidth{0.803000pt}%
\definecolor{currentstroke}{rgb}{0.000000,0.000000,0.000000}%
\pgfsetstrokecolor{currentstroke}%
\pgfsetdash{}{0pt}%
\pgfsys@defobject{currentmarker}{\pgfqpoint{-0.048611in}{0.000000in}}{\pgfqpoint{-0.000000in}{0.000000in}}{%
\pgfpathmoveto{\pgfqpoint{-0.000000in}{0.000000in}}%
\pgfpathlineto{\pgfqpoint{-0.048611in}{0.000000in}}%
\pgfusepath{stroke,fill}%
}%
\begin{pgfscope}%
\pgfsys@transformshift{0.481978in}{2.613640in}%
\pgfsys@useobject{currentmarker}{}%
\end{pgfscope}%
\end{pgfscope}%
\begin{pgfscope}%
\definecolor{textcolor}{rgb}{0.000000,0.000000,0.000000}%
\pgfsetstrokecolor{textcolor}%
\pgfsetfillcolor{textcolor}%
\pgftext[x=0.100000in, y=2.560879in, left, base]{\color{textcolor}\sffamily\fontsize{10.000000}{12.000000}\selectfont \ensuremath{-}10}%
\end{pgfscope}%
\begin{pgfscope}%
\pgfsetbuttcap%
\pgfsetroundjoin%
\definecolor{currentfill}{rgb}{0.000000,0.000000,0.000000}%
\pgfsetfillcolor{currentfill}%
\pgfsetlinewidth{0.803000pt}%
\definecolor{currentstroke}{rgb}{0.000000,0.000000,0.000000}%
\pgfsetstrokecolor{currentstroke}%
\pgfsetdash{}{0pt}%
\pgfsys@defobject{currentmarker}{\pgfqpoint{-0.048611in}{0.000000in}}{\pgfqpoint{-0.000000in}{0.000000in}}{%
\pgfpathmoveto{\pgfqpoint{-0.000000in}{0.000000in}}%
\pgfpathlineto{\pgfqpoint{-0.048611in}{0.000000in}}%
\pgfusepath{stroke,fill}%
}%
\begin{pgfscope}%
\pgfsys@transformshift{0.481978in}{3.905028in}%
\pgfsys@useobject{currentmarker}{}%
\end{pgfscope}%
\end{pgfscope}%
\begin{pgfscope}%
\definecolor{textcolor}{rgb}{0.000000,0.000000,0.000000}%
\pgfsetstrokecolor{textcolor}%
\pgfsetfillcolor{textcolor}%
\pgftext[x=0.296390in, y=3.852267in, left, base]{\color{textcolor}\sffamily\fontsize{10.000000}{12.000000}\selectfont 0}%
\end{pgfscope}%
\begin{pgfscope}%
\pgfsetbuttcap%
\pgfsetroundjoin%
\definecolor{currentfill}{rgb}{0.000000,0.000000,0.000000}%
\pgfsetfillcolor{currentfill}%
\pgfsetlinewidth{0.803000pt}%
\definecolor{currentstroke}{rgb}{0.000000,0.000000,0.000000}%
\pgfsetstrokecolor{currentstroke}%
\pgfsetdash{}{0pt}%
\pgfsys@defobject{currentmarker}{\pgfqpoint{-0.048611in}{0.000000in}}{\pgfqpoint{-0.000000in}{0.000000in}}{%
\pgfpathmoveto{\pgfqpoint{-0.000000in}{0.000000in}}%
\pgfpathlineto{\pgfqpoint{-0.048611in}{0.000000in}}%
\pgfusepath{stroke,fill}%
}%
\begin{pgfscope}%
\pgfsys@transformshift{0.481978in}{5.196416in}%
\pgfsys@useobject{currentmarker}{}%
\end{pgfscope}%
\end{pgfscope}%
\begin{pgfscope}%
\definecolor{textcolor}{rgb}{0.000000,0.000000,0.000000}%
\pgfsetstrokecolor{textcolor}%
\pgfsetfillcolor{textcolor}%
\pgftext[x=0.208025in, y=5.143654in, left, base]{\color{textcolor}\sffamily\fontsize{10.000000}{12.000000}\selectfont 10}%
\end{pgfscope}%
\begin{pgfscope}%
\pgfsetbuttcap%
\pgfsetroundjoin%
\definecolor{currentfill}{rgb}{0.000000,0.000000,0.000000}%
\pgfsetfillcolor{currentfill}%
\pgfsetlinewidth{0.803000pt}%
\definecolor{currentstroke}{rgb}{0.000000,0.000000,0.000000}%
\pgfsetstrokecolor{currentstroke}%
\pgfsetdash{}{0pt}%
\pgfsys@defobject{currentmarker}{\pgfqpoint{-0.048611in}{0.000000in}}{\pgfqpoint{-0.000000in}{0.000000in}}{%
\pgfpathmoveto{\pgfqpoint{-0.000000in}{0.000000in}}%
\pgfpathlineto{\pgfqpoint{-0.048611in}{0.000000in}}%
\pgfusepath{stroke,fill}%
}%
\begin{pgfscope}%
\pgfsys@transformshift{0.481978in}{6.487804in}%
\pgfsys@useobject{currentmarker}{}%
\end{pgfscope}%
\end{pgfscope}%
\begin{pgfscope}%
\definecolor{textcolor}{rgb}{0.000000,0.000000,0.000000}%
\pgfsetstrokecolor{textcolor}%
\pgfsetfillcolor{textcolor}%
\pgftext[x=0.208025in, y=6.435042in, left, base]{\color{textcolor}\sffamily\fontsize{10.000000}{12.000000}\selectfont 20}%
\end{pgfscope}%
\begin{pgfscope}%
\pgfsetbuttcap%
\pgfsetroundjoin%
\definecolor{currentfill}{rgb}{0.000000,0.000000,0.000000}%
\pgfsetfillcolor{currentfill}%
\pgfsetlinewidth{0.803000pt}%
\definecolor{currentstroke}{rgb}{0.000000,0.000000,0.000000}%
\pgfsetstrokecolor{currentstroke}%
\pgfsetdash{}{0pt}%
\pgfsys@defobject{currentmarker}{\pgfqpoint{-0.048611in}{0.000000in}}{\pgfqpoint{-0.000000in}{0.000000in}}{%
\pgfpathmoveto{\pgfqpoint{-0.000000in}{0.000000in}}%
\pgfpathlineto{\pgfqpoint{-0.048611in}{0.000000in}}%
\pgfusepath{stroke,fill}%
}%
\begin{pgfscope}%
\pgfsys@transformshift{0.481978in}{7.779192in}%
\pgfsys@useobject{currentmarker}{}%
\end{pgfscope}%
\end{pgfscope}%
\begin{pgfscope}%
\definecolor{textcolor}{rgb}{0.000000,0.000000,0.000000}%
\pgfsetstrokecolor{textcolor}%
\pgfsetfillcolor{textcolor}%
\pgftext[x=0.208025in, y=7.726430in, left, base]{\color{textcolor}\sffamily\fontsize{10.000000}{12.000000}\selectfont 30}%
\end{pgfscope}%
\begin{pgfscope}%
\pgfpathrectangle{\pgfqpoint{0.481978in}{0.331635in}}{\pgfqpoint{9.300000in}{7.700000in}}%
\pgfusepath{clip}%
\pgfsetrectcap%
\pgfsetroundjoin%
\pgfsetlinewidth{1.505625pt}%
\definecolor{currentstroke}{rgb}{0.631373,0.788235,0.956863}%
\pgfsetstrokecolor{currentstroke}%
\pgfsetstrokeopacity{0.800000}%
\pgfsetdash{}{0pt}%
\pgfpathmoveto{\pgfqpoint{5.562292in}{2.046179in}}%
\pgfpathlineto{\pgfqpoint{5.711094in}{3.773662in}}%
\pgfusepath{stroke}%
\end{pgfscope}%
\begin{pgfscope}%
\pgfpathrectangle{\pgfqpoint{0.481978in}{0.331635in}}{\pgfqpoint{9.300000in}{7.700000in}}%
\pgfusepath{clip}%
\pgfsetrectcap%
\pgfsetroundjoin%
\pgfsetlinewidth{1.505625pt}%
\definecolor{currentstroke}{rgb}{0.631373,0.788235,0.956863}%
\pgfsetstrokecolor{currentstroke}%
\pgfsetstrokeopacity{0.800000}%
\pgfsetdash{}{0pt}%
\pgfpathmoveto{\pgfqpoint{6.909912in}{4.460102in}}%
\pgfpathlineto{\pgfqpoint{5.711094in}{3.773662in}}%
\pgfusepath{stroke}%
\end{pgfscope}%
\begin{pgfscope}%
\pgfpathrectangle{\pgfqpoint{0.481978in}{0.331635in}}{\pgfqpoint{9.300000in}{7.700000in}}%
\pgfusepath{clip}%
\pgfsetrectcap%
\pgfsetroundjoin%
\pgfsetlinewidth{1.505625pt}%
\definecolor{currentstroke}{rgb}{0.631373,0.788235,0.956863}%
\pgfsetstrokecolor{currentstroke}%
\pgfsetstrokeopacity{0.800000}%
\pgfsetdash{}{0pt}%
\pgfpathmoveto{\pgfqpoint{7.565874in}{5.883101in}}%
\pgfpathlineto{\pgfqpoint{5.711094in}{3.773662in}}%
\pgfusepath{stroke}%
\end{pgfscope}%
\begin{pgfscope}%
\pgfpathrectangle{\pgfqpoint{0.481978in}{0.331635in}}{\pgfqpoint{9.300000in}{7.700000in}}%
\pgfusepath{clip}%
\pgfsetrectcap%
\pgfsetroundjoin%
\pgfsetlinewidth{1.505625pt}%
\definecolor{currentstroke}{rgb}{0.631373,0.788235,0.956863}%
\pgfsetstrokecolor{currentstroke}%
\pgfsetstrokeopacity{0.800000}%
\pgfsetdash{}{0pt}%
\pgfpathmoveto{\pgfqpoint{8.104384in}{5.189132in}}%
\pgfpathlineto{\pgfqpoint{5.711094in}{3.773662in}}%
\pgfusepath{stroke}%
\end{pgfscope}%
\begin{pgfscope}%
\pgfpathrectangle{\pgfqpoint{0.481978in}{0.331635in}}{\pgfqpoint{9.300000in}{7.700000in}}%
\pgfusepath{clip}%
\pgfsetrectcap%
\pgfsetroundjoin%
\pgfsetlinewidth{1.505625pt}%
\definecolor{currentstroke}{rgb}{0.631373,0.788235,0.956863}%
\pgfsetstrokecolor{currentstroke}%
\pgfsetstrokeopacity{0.800000}%
\pgfsetdash{}{0pt}%
\pgfpathmoveto{\pgfqpoint{7.967295in}{4.825787in}}%
\pgfpathlineto{\pgfqpoint{5.711094in}{3.773662in}}%
\pgfusepath{stroke}%
\end{pgfscope}%
\begin{pgfscope}%
\pgfpathrectangle{\pgfqpoint{0.481978in}{0.331635in}}{\pgfqpoint{9.300000in}{7.700000in}}%
\pgfusepath{clip}%
\pgfsetrectcap%
\pgfsetroundjoin%
\pgfsetlinewidth{1.505625pt}%
\definecolor{currentstroke}{rgb}{0.631373,0.788235,0.956863}%
\pgfsetstrokecolor{currentstroke}%
\pgfsetstrokeopacity{0.800000}%
\pgfsetdash{}{0pt}%
\pgfpathmoveto{\pgfqpoint{5.184843in}{3.705537in}}%
\pgfpathlineto{\pgfqpoint{5.711094in}{3.773662in}}%
\pgfusepath{stroke}%
\end{pgfscope}%
\begin{pgfscope}%
\pgfpathrectangle{\pgfqpoint{0.481978in}{0.331635in}}{\pgfqpoint{9.300000in}{7.700000in}}%
\pgfusepath{clip}%
\pgfsetrectcap%
\pgfsetroundjoin%
\pgfsetlinewidth{1.505625pt}%
\definecolor{currentstroke}{rgb}{0.631373,0.788235,0.956863}%
\pgfsetstrokecolor{currentstroke}%
\pgfsetstrokeopacity{0.800000}%
\pgfsetdash{}{0pt}%
\pgfpathmoveto{\pgfqpoint{2.850031in}{2.147835in}}%
\pgfpathlineto{\pgfqpoint{5.711094in}{3.773662in}}%
\pgfusepath{stroke}%
\end{pgfscope}%
\begin{pgfscope}%
\pgfpathrectangle{\pgfqpoint{0.481978in}{0.331635in}}{\pgfqpoint{9.300000in}{7.700000in}}%
\pgfusepath{clip}%
\pgfsetrectcap%
\pgfsetroundjoin%
\pgfsetlinewidth{1.505625pt}%
\definecolor{currentstroke}{rgb}{0.631373,0.788235,0.956863}%
\pgfsetstrokecolor{currentstroke}%
\pgfsetstrokeopacity{0.800000}%
\pgfsetdash{}{0pt}%
\pgfpathmoveto{\pgfqpoint{5.765949in}{2.940193in}}%
\pgfpathlineto{\pgfqpoint{5.711094in}{3.773662in}}%
\pgfusepath{stroke}%
\end{pgfscope}%
\begin{pgfscope}%
\pgfpathrectangle{\pgfqpoint{0.481978in}{0.331635in}}{\pgfqpoint{9.300000in}{7.700000in}}%
\pgfusepath{clip}%
\pgfsetrectcap%
\pgfsetroundjoin%
\pgfsetlinewidth{1.505625pt}%
\definecolor{currentstroke}{rgb}{0.631373,0.788235,0.956863}%
\pgfsetstrokecolor{currentstroke}%
\pgfsetstrokeopacity{0.800000}%
\pgfsetdash{}{0pt}%
\pgfpathmoveto{\pgfqpoint{6.395914in}{3.230225in}}%
\pgfpathlineto{\pgfqpoint{5.711094in}{3.773662in}}%
\pgfusepath{stroke}%
\end{pgfscope}%
\begin{pgfscope}%
\pgfpathrectangle{\pgfqpoint{0.481978in}{0.331635in}}{\pgfqpoint{9.300000in}{7.700000in}}%
\pgfusepath{clip}%
\pgfsetrectcap%
\pgfsetroundjoin%
\pgfsetlinewidth{1.505625pt}%
\definecolor{currentstroke}{rgb}{0.631373,0.788235,0.956863}%
\pgfsetstrokecolor{currentstroke}%
\pgfsetstrokeopacity{0.800000}%
\pgfsetdash{}{0pt}%
\pgfpathmoveto{\pgfqpoint{4.098452in}{6.271448in}}%
\pgfpathlineto{\pgfqpoint{5.711094in}{3.773662in}}%
\pgfusepath{stroke}%
\end{pgfscope}%
\begin{pgfscope}%
\pgfpathrectangle{\pgfqpoint{0.481978in}{0.331635in}}{\pgfqpoint{9.300000in}{7.700000in}}%
\pgfusepath{clip}%
\pgfsetrectcap%
\pgfsetroundjoin%
\pgfsetlinewidth{1.505625pt}%
\definecolor{currentstroke}{rgb}{0.631373,0.788235,0.956863}%
\pgfsetstrokecolor{currentstroke}%
\pgfsetstrokeopacity{0.800000}%
\pgfsetdash{}{0pt}%
\pgfpathmoveto{\pgfqpoint{6.735020in}{4.508870in}}%
\pgfpathlineto{\pgfqpoint{5.711094in}{3.773662in}}%
\pgfusepath{stroke}%
\end{pgfscope}%
\begin{pgfscope}%
\pgfpathrectangle{\pgfqpoint{0.481978in}{0.331635in}}{\pgfqpoint{9.300000in}{7.700000in}}%
\pgfusepath{clip}%
\pgfsetrectcap%
\pgfsetroundjoin%
\pgfsetlinewidth{1.505625pt}%
\definecolor{currentstroke}{rgb}{0.631373,0.788235,0.956863}%
\pgfsetstrokecolor{currentstroke}%
\pgfsetstrokeopacity{0.800000}%
\pgfsetdash{}{0pt}%
\pgfpathmoveto{\pgfqpoint{5.876608in}{2.528955in}}%
\pgfpathlineto{\pgfqpoint{5.711094in}{3.773662in}}%
\pgfusepath{stroke}%
\end{pgfscope}%
\begin{pgfscope}%
\pgfpathrectangle{\pgfqpoint{0.481978in}{0.331635in}}{\pgfqpoint{9.300000in}{7.700000in}}%
\pgfusepath{clip}%
\pgfsetrectcap%
\pgfsetroundjoin%
\pgfsetlinewidth{1.505625pt}%
\definecolor{currentstroke}{rgb}{0.631373,0.788235,0.956863}%
\pgfsetstrokecolor{currentstroke}%
\pgfsetstrokeopacity{0.800000}%
\pgfsetdash{}{0pt}%
\pgfpathmoveto{\pgfqpoint{4.683187in}{4.425480in}}%
\pgfpathlineto{\pgfqpoint{5.711094in}{3.773662in}}%
\pgfusepath{stroke}%
\end{pgfscope}%
\begin{pgfscope}%
\pgfpathrectangle{\pgfqpoint{0.481978in}{0.331635in}}{\pgfqpoint{9.300000in}{7.700000in}}%
\pgfusepath{clip}%
\pgfsetrectcap%
\pgfsetroundjoin%
\pgfsetlinewidth{1.505625pt}%
\definecolor{currentstroke}{rgb}{0.631373,0.788235,0.956863}%
\pgfsetstrokecolor{currentstroke}%
\pgfsetstrokeopacity{0.800000}%
\pgfsetdash{}{0pt}%
\pgfpathmoveto{\pgfqpoint{2.854341in}{2.041662in}}%
\pgfpathlineto{\pgfqpoint{5.711094in}{3.773662in}}%
\pgfusepath{stroke}%
\end{pgfscope}%
\begin{pgfscope}%
\pgfpathrectangle{\pgfqpoint{0.481978in}{0.331635in}}{\pgfqpoint{9.300000in}{7.700000in}}%
\pgfusepath{clip}%
\pgfsetrectcap%
\pgfsetroundjoin%
\pgfsetlinewidth{1.505625pt}%
\definecolor{currentstroke}{rgb}{0.631373,0.788235,0.956863}%
\pgfsetstrokecolor{currentstroke}%
\pgfsetstrokeopacity{0.800000}%
\pgfsetdash{}{0pt}%
\pgfpathmoveto{\pgfqpoint{7.738499in}{4.320193in}}%
\pgfpathlineto{\pgfqpoint{5.711094in}{3.773662in}}%
\pgfusepath{stroke}%
\end{pgfscope}%
\begin{pgfscope}%
\pgfpathrectangle{\pgfqpoint{0.481978in}{0.331635in}}{\pgfqpoint{9.300000in}{7.700000in}}%
\pgfusepath{clip}%
\pgfsetrectcap%
\pgfsetroundjoin%
\pgfsetlinewidth{1.505625pt}%
\definecolor{currentstroke}{rgb}{0.631373,0.788235,0.956863}%
\pgfsetstrokecolor{currentstroke}%
\pgfsetstrokeopacity{0.800000}%
\pgfsetdash{}{0pt}%
\pgfpathmoveto{\pgfqpoint{4.181272in}{7.669721in}}%
\pgfpathlineto{\pgfqpoint{5.711094in}{3.773662in}}%
\pgfusepath{stroke}%
\end{pgfscope}%
\begin{pgfscope}%
\pgfpathrectangle{\pgfqpoint{0.481978in}{0.331635in}}{\pgfqpoint{9.300000in}{7.700000in}}%
\pgfusepath{clip}%
\pgfsetrectcap%
\pgfsetroundjoin%
\pgfsetlinewidth{1.505625pt}%
\definecolor{currentstroke}{rgb}{0.631373,0.788235,0.956863}%
\pgfsetstrokecolor{currentstroke}%
\pgfsetstrokeopacity{0.800000}%
\pgfsetdash{}{0pt}%
\pgfpathmoveto{\pgfqpoint{6.285628in}{4.457219in}}%
\pgfpathlineto{\pgfqpoint{5.711094in}{3.773662in}}%
\pgfusepath{stroke}%
\end{pgfscope}%
\begin{pgfscope}%
\pgfpathrectangle{\pgfqpoint{0.481978in}{0.331635in}}{\pgfqpoint{9.300000in}{7.700000in}}%
\pgfusepath{clip}%
\pgfsetrectcap%
\pgfsetroundjoin%
\pgfsetlinewidth{1.505625pt}%
\definecolor{currentstroke}{rgb}{0.631373,0.788235,0.956863}%
\pgfsetstrokecolor{currentstroke}%
\pgfsetstrokeopacity{0.800000}%
\pgfsetdash{}{0pt}%
\pgfpathmoveto{\pgfqpoint{7.326306in}{5.231850in}}%
\pgfpathlineto{\pgfqpoint{5.711094in}{3.773662in}}%
\pgfusepath{stroke}%
\end{pgfscope}%
\begin{pgfscope}%
\pgfpathrectangle{\pgfqpoint{0.481978in}{0.331635in}}{\pgfqpoint{9.300000in}{7.700000in}}%
\pgfusepath{clip}%
\pgfsetrectcap%
\pgfsetroundjoin%
\pgfsetlinewidth{1.505625pt}%
\definecolor{currentstroke}{rgb}{0.631373,0.788235,0.956863}%
\pgfsetstrokecolor{currentstroke}%
\pgfsetstrokeopacity{0.800000}%
\pgfsetdash{}{0pt}%
\pgfpathmoveto{\pgfqpoint{5.231346in}{2.275215in}}%
\pgfpathlineto{\pgfqpoint{5.711094in}{3.773662in}}%
\pgfusepath{stroke}%
\end{pgfscope}%
\begin{pgfscope}%
\pgfpathrectangle{\pgfqpoint{0.481978in}{0.331635in}}{\pgfqpoint{9.300000in}{7.700000in}}%
\pgfusepath{clip}%
\pgfsetrectcap%
\pgfsetroundjoin%
\pgfsetlinewidth{1.505625pt}%
\definecolor{currentstroke}{rgb}{0.631373,0.788235,0.956863}%
\pgfsetstrokecolor{currentstroke}%
\pgfsetstrokeopacity{0.800000}%
\pgfsetdash{}{0pt}%
\pgfpathmoveto{\pgfqpoint{2.924270in}{1.932016in}}%
\pgfpathlineto{\pgfqpoint{5.711094in}{3.773662in}}%
\pgfusepath{stroke}%
\end{pgfscope}%
\begin{pgfscope}%
\pgfpathrectangle{\pgfqpoint{0.481978in}{0.331635in}}{\pgfqpoint{9.300000in}{7.700000in}}%
\pgfusepath{clip}%
\pgfsetrectcap%
\pgfsetroundjoin%
\pgfsetlinewidth{1.505625pt}%
\definecolor{currentstroke}{rgb}{0.631373,0.788235,0.956863}%
\pgfsetstrokecolor{currentstroke}%
\pgfsetstrokeopacity{0.800000}%
\pgfsetdash{}{0pt}%
\pgfpathmoveto{\pgfqpoint{5.426953in}{7.077580in}}%
\pgfpathlineto{\pgfqpoint{5.711094in}{3.773662in}}%
\pgfusepath{stroke}%
\end{pgfscope}%
\begin{pgfscope}%
\pgfpathrectangle{\pgfqpoint{0.481978in}{0.331635in}}{\pgfqpoint{9.300000in}{7.700000in}}%
\pgfusepath{clip}%
\pgfsetrectcap%
\pgfsetroundjoin%
\pgfsetlinewidth{1.505625pt}%
\definecolor{currentstroke}{rgb}{0.631373,0.788235,0.956863}%
\pgfsetstrokecolor{currentstroke}%
\pgfsetstrokeopacity{0.800000}%
\pgfsetdash{}{0pt}%
\pgfpathmoveto{\pgfqpoint{8.241979in}{5.269247in}}%
\pgfpathlineto{\pgfqpoint{5.711094in}{3.773662in}}%
\pgfusepath{stroke}%
\end{pgfscope}%
\begin{pgfscope}%
\pgfpathrectangle{\pgfqpoint{0.481978in}{0.331635in}}{\pgfqpoint{9.300000in}{7.700000in}}%
\pgfusepath{clip}%
\pgfsetrectcap%
\pgfsetroundjoin%
\pgfsetlinewidth{1.505625pt}%
\definecolor{currentstroke}{rgb}{0.631373,0.788235,0.956863}%
\pgfsetstrokecolor{currentstroke}%
\pgfsetstrokeopacity{0.800000}%
\pgfsetdash{}{0pt}%
\pgfpathmoveto{\pgfqpoint{5.988042in}{2.740024in}}%
\pgfpathlineto{\pgfqpoint{5.711094in}{3.773662in}}%
\pgfusepath{stroke}%
\end{pgfscope}%
\begin{pgfscope}%
\pgfpathrectangle{\pgfqpoint{0.481978in}{0.331635in}}{\pgfqpoint{9.300000in}{7.700000in}}%
\pgfusepath{clip}%
\pgfsetrectcap%
\pgfsetroundjoin%
\pgfsetlinewidth{1.505625pt}%
\definecolor{currentstroke}{rgb}{0.631373,0.788235,0.956863}%
\pgfsetstrokecolor{currentstroke}%
\pgfsetstrokeopacity{0.800000}%
\pgfsetdash{}{0pt}%
\pgfpathmoveto{\pgfqpoint{6.404379in}{5.407426in}}%
\pgfpathlineto{\pgfqpoint{5.711094in}{3.773662in}}%
\pgfusepath{stroke}%
\end{pgfscope}%
\begin{pgfscope}%
\pgfpathrectangle{\pgfqpoint{0.481978in}{0.331635in}}{\pgfqpoint{9.300000in}{7.700000in}}%
\pgfusepath{clip}%
\pgfsetrectcap%
\pgfsetroundjoin%
\pgfsetlinewidth{1.505625pt}%
\definecolor{currentstroke}{rgb}{0.631373,0.788235,0.956863}%
\pgfsetstrokecolor{currentstroke}%
\pgfsetstrokeopacity{0.800000}%
\pgfsetdash{}{0pt}%
\pgfpathmoveto{\pgfqpoint{7.465449in}{4.156782in}}%
\pgfpathlineto{\pgfqpoint{5.711094in}{3.773662in}}%
\pgfusepath{stroke}%
\end{pgfscope}%
\begin{pgfscope}%
\pgfpathrectangle{\pgfqpoint{0.481978in}{0.331635in}}{\pgfqpoint{9.300000in}{7.700000in}}%
\pgfusepath{clip}%
\pgfsetrectcap%
\pgfsetroundjoin%
\pgfsetlinewidth{1.505625pt}%
\definecolor{currentstroke}{rgb}{0.631373,0.788235,0.956863}%
\pgfsetstrokecolor{currentstroke}%
\pgfsetstrokeopacity{0.800000}%
\pgfsetdash{}{0pt}%
\pgfpathmoveto{\pgfqpoint{6.297588in}{4.859032in}}%
\pgfpathlineto{\pgfqpoint{5.711094in}{3.773662in}}%
\pgfusepath{stroke}%
\end{pgfscope}%
\begin{pgfscope}%
\pgfpathrectangle{\pgfqpoint{0.481978in}{0.331635in}}{\pgfqpoint{9.300000in}{7.700000in}}%
\pgfusepath{clip}%
\pgfsetrectcap%
\pgfsetroundjoin%
\pgfsetlinewidth{1.505625pt}%
\definecolor{currentstroke}{rgb}{0.631373,0.788235,0.956863}%
\pgfsetstrokecolor{currentstroke}%
\pgfsetstrokeopacity{0.800000}%
\pgfsetdash{}{0pt}%
\pgfpathmoveto{\pgfqpoint{7.598727in}{4.532028in}}%
\pgfpathlineto{\pgfqpoint{5.711094in}{3.773662in}}%
\pgfusepath{stroke}%
\end{pgfscope}%
\begin{pgfscope}%
\pgfpathrectangle{\pgfqpoint{0.481978in}{0.331635in}}{\pgfqpoint{9.300000in}{7.700000in}}%
\pgfusepath{clip}%
\pgfsetrectcap%
\pgfsetroundjoin%
\pgfsetlinewidth{1.505625pt}%
\definecolor{currentstroke}{rgb}{0.631373,0.788235,0.956863}%
\pgfsetstrokecolor{currentstroke}%
\pgfsetstrokeopacity{0.800000}%
\pgfsetdash{}{0pt}%
\pgfpathmoveto{\pgfqpoint{4.778349in}{5.151613in}}%
\pgfpathlineto{\pgfqpoint{5.711094in}{3.773662in}}%
\pgfusepath{stroke}%
\end{pgfscope}%
\begin{pgfscope}%
\pgfpathrectangle{\pgfqpoint{0.481978in}{0.331635in}}{\pgfqpoint{9.300000in}{7.700000in}}%
\pgfusepath{clip}%
\pgfsetrectcap%
\pgfsetroundjoin%
\pgfsetlinewidth{1.505625pt}%
\definecolor{currentstroke}{rgb}{0.631373,0.788235,0.956863}%
\pgfsetstrokecolor{currentstroke}%
\pgfsetstrokeopacity{0.800000}%
\pgfsetdash{}{0pt}%
\pgfpathmoveto{\pgfqpoint{5.249516in}{5.519220in}}%
\pgfpathlineto{\pgfqpoint{5.711094in}{3.773662in}}%
\pgfusepath{stroke}%
\end{pgfscope}%
\begin{pgfscope}%
\pgfpathrectangle{\pgfqpoint{0.481978in}{0.331635in}}{\pgfqpoint{9.300000in}{7.700000in}}%
\pgfusepath{clip}%
\pgfsetrectcap%
\pgfsetroundjoin%
\pgfsetlinewidth{1.505625pt}%
\definecolor{currentstroke}{rgb}{0.631373,0.788235,0.956863}%
\pgfsetstrokecolor{currentstroke}%
\pgfsetstrokeopacity{0.800000}%
\pgfsetdash{}{0pt}%
\pgfpathmoveto{\pgfqpoint{6.692326in}{1.689848in}}%
\pgfpathlineto{\pgfqpoint{5.711094in}{3.773662in}}%
\pgfusepath{stroke}%
\end{pgfscope}%
\begin{pgfscope}%
\pgfpathrectangle{\pgfqpoint{0.481978in}{0.331635in}}{\pgfqpoint{9.300000in}{7.700000in}}%
\pgfusepath{clip}%
\pgfsetrectcap%
\pgfsetroundjoin%
\pgfsetlinewidth{1.505625pt}%
\definecolor{currentstroke}{rgb}{0.631373,0.788235,0.956863}%
\pgfsetstrokecolor{currentstroke}%
\pgfsetstrokeopacity{0.800000}%
\pgfsetdash{}{0pt}%
\pgfpathmoveto{\pgfqpoint{4.317631in}{1.948751in}}%
\pgfpathlineto{\pgfqpoint{5.711094in}{3.773662in}}%
\pgfusepath{stroke}%
\end{pgfscope}%
\begin{pgfscope}%
\pgfpathrectangle{\pgfqpoint{0.481978in}{0.331635in}}{\pgfqpoint{9.300000in}{7.700000in}}%
\pgfusepath{clip}%
\pgfsetrectcap%
\pgfsetroundjoin%
\pgfsetlinewidth{1.505625pt}%
\definecolor{currentstroke}{rgb}{0.631373,0.788235,0.956863}%
\pgfsetstrokecolor{currentstroke}%
\pgfsetstrokeopacity{0.800000}%
\pgfsetdash{}{0pt}%
\pgfpathmoveto{\pgfqpoint{7.680936in}{6.049583in}}%
\pgfpathlineto{\pgfqpoint{5.711094in}{3.773662in}}%
\pgfusepath{stroke}%
\end{pgfscope}%
\begin{pgfscope}%
\pgfpathrectangle{\pgfqpoint{0.481978in}{0.331635in}}{\pgfqpoint{9.300000in}{7.700000in}}%
\pgfusepath{clip}%
\pgfsetrectcap%
\pgfsetroundjoin%
\pgfsetlinewidth{1.505625pt}%
\definecolor{currentstroke}{rgb}{0.631373,0.788235,0.956863}%
\pgfsetstrokecolor{currentstroke}%
\pgfsetstrokeopacity{0.800000}%
\pgfsetdash{}{0pt}%
\pgfpathmoveto{\pgfqpoint{4.088639in}{6.016536in}}%
\pgfpathlineto{\pgfqpoint{5.711094in}{3.773662in}}%
\pgfusepath{stroke}%
\end{pgfscope}%
\begin{pgfscope}%
\pgfpathrectangle{\pgfqpoint{0.481978in}{0.331635in}}{\pgfqpoint{9.300000in}{7.700000in}}%
\pgfusepath{clip}%
\pgfsetrectcap%
\pgfsetroundjoin%
\pgfsetlinewidth{1.505625pt}%
\definecolor{currentstroke}{rgb}{0.631373,0.788235,0.956863}%
\pgfsetstrokecolor{currentstroke}%
\pgfsetstrokeopacity{0.800000}%
\pgfsetdash{}{0pt}%
\pgfpathmoveto{\pgfqpoint{2.271824in}{3.813978in}}%
\pgfpathlineto{\pgfqpoint{5.711094in}{3.773662in}}%
\pgfusepath{stroke}%
\end{pgfscope}%
\begin{pgfscope}%
\pgfpathrectangle{\pgfqpoint{0.481978in}{0.331635in}}{\pgfqpoint{9.300000in}{7.700000in}}%
\pgfusepath{clip}%
\pgfsetrectcap%
\pgfsetroundjoin%
\pgfsetlinewidth{1.505625pt}%
\definecolor{currentstroke}{rgb}{0.631373,0.788235,0.956863}%
\pgfsetstrokecolor{currentstroke}%
\pgfsetstrokeopacity{0.800000}%
\pgfsetdash{}{0pt}%
\pgfpathmoveto{\pgfqpoint{6.175909in}{2.684328in}}%
\pgfpathlineto{\pgfqpoint{5.711094in}{3.773662in}}%
\pgfusepath{stroke}%
\end{pgfscope}%
\begin{pgfscope}%
\pgfpathrectangle{\pgfqpoint{0.481978in}{0.331635in}}{\pgfqpoint{9.300000in}{7.700000in}}%
\pgfusepath{clip}%
\pgfsetrectcap%
\pgfsetroundjoin%
\pgfsetlinewidth{1.505625pt}%
\definecolor{currentstroke}{rgb}{0.631373,0.788235,0.956863}%
\pgfsetstrokecolor{currentstroke}%
\pgfsetstrokeopacity{0.800000}%
\pgfsetdash{}{0pt}%
\pgfpathmoveto{\pgfqpoint{5.076404in}{1.764299in}}%
\pgfpathlineto{\pgfqpoint{5.711094in}{3.773662in}}%
\pgfusepath{stroke}%
\end{pgfscope}%
\begin{pgfscope}%
\pgfpathrectangle{\pgfqpoint{0.481978in}{0.331635in}}{\pgfqpoint{9.300000in}{7.700000in}}%
\pgfusepath{clip}%
\pgfsetrectcap%
\pgfsetroundjoin%
\pgfsetlinewidth{1.505625pt}%
\definecolor{currentstroke}{rgb}{0.631373,0.788235,0.956863}%
\pgfsetstrokecolor{currentstroke}%
\pgfsetstrokeopacity{0.800000}%
\pgfsetdash{}{0pt}%
\pgfpathmoveto{\pgfqpoint{8.665313in}{4.858730in}}%
\pgfpathlineto{\pgfqpoint{5.711094in}{3.773662in}}%
\pgfusepath{stroke}%
\end{pgfscope}%
\begin{pgfscope}%
\pgfpathrectangle{\pgfqpoint{0.481978in}{0.331635in}}{\pgfqpoint{9.300000in}{7.700000in}}%
\pgfusepath{clip}%
\pgfsetrectcap%
\pgfsetroundjoin%
\pgfsetlinewidth{1.505625pt}%
\definecolor{currentstroke}{rgb}{0.631373,0.788235,0.956863}%
\pgfsetstrokecolor{currentstroke}%
\pgfsetstrokeopacity{0.800000}%
\pgfsetdash{}{0pt}%
\pgfpathmoveto{\pgfqpoint{7.138229in}{2.600769in}}%
\pgfpathlineto{\pgfqpoint{5.711094in}{3.773662in}}%
\pgfusepath{stroke}%
\end{pgfscope}%
\begin{pgfscope}%
\pgfpathrectangle{\pgfqpoint{0.481978in}{0.331635in}}{\pgfqpoint{9.300000in}{7.700000in}}%
\pgfusepath{clip}%
\pgfsetrectcap%
\pgfsetroundjoin%
\pgfsetlinewidth{1.505625pt}%
\definecolor{currentstroke}{rgb}{0.631373,0.788235,0.956863}%
\pgfsetstrokecolor{currentstroke}%
\pgfsetstrokeopacity{0.800000}%
\pgfsetdash{}{0pt}%
\pgfpathmoveto{\pgfqpoint{6.963010in}{1.855391in}}%
\pgfpathlineto{\pgfqpoint{5.711094in}{3.773662in}}%
\pgfusepath{stroke}%
\end{pgfscope}%
\begin{pgfscope}%
\pgfpathrectangle{\pgfqpoint{0.481978in}{0.331635in}}{\pgfqpoint{9.300000in}{7.700000in}}%
\pgfusepath{clip}%
\pgfsetrectcap%
\pgfsetroundjoin%
\pgfsetlinewidth{1.505625pt}%
\definecolor{currentstroke}{rgb}{0.631373,0.788235,0.956863}%
\pgfsetstrokecolor{currentstroke}%
\pgfsetstrokeopacity{0.800000}%
\pgfsetdash{}{0pt}%
\pgfpathmoveto{\pgfqpoint{8.113494in}{5.309463in}}%
\pgfpathlineto{\pgfqpoint{5.711094in}{3.773662in}}%
\pgfusepath{stroke}%
\end{pgfscope}%
\begin{pgfscope}%
\pgfpathrectangle{\pgfqpoint{0.481978in}{0.331635in}}{\pgfqpoint{9.300000in}{7.700000in}}%
\pgfusepath{clip}%
\pgfsetrectcap%
\pgfsetroundjoin%
\pgfsetlinewidth{1.505625pt}%
\definecolor{currentstroke}{rgb}{0.631373,0.788235,0.956863}%
\pgfsetstrokecolor{currentstroke}%
\pgfsetstrokeopacity{0.800000}%
\pgfsetdash{}{0pt}%
\pgfpathmoveto{\pgfqpoint{5.616150in}{1.874161in}}%
\pgfpathlineto{\pgfqpoint{5.711094in}{3.773662in}}%
\pgfusepath{stroke}%
\end{pgfscope}%
\begin{pgfscope}%
\pgfpathrectangle{\pgfqpoint{0.481978in}{0.331635in}}{\pgfqpoint{9.300000in}{7.700000in}}%
\pgfusepath{clip}%
\pgfsetrectcap%
\pgfsetroundjoin%
\pgfsetlinewidth{1.505625pt}%
\definecolor{currentstroke}{rgb}{0.631373,0.788235,0.956863}%
\pgfsetstrokecolor{currentstroke}%
\pgfsetstrokeopacity{0.800000}%
\pgfsetdash{}{0pt}%
\pgfpathmoveto{\pgfqpoint{4.785427in}{1.312445in}}%
\pgfpathlineto{\pgfqpoint{5.711094in}{3.773662in}}%
\pgfusepath{stroke}%
\end{pgfscope}%
\begin{pgfscope}%
\pgfpathrectangle{\pgfqpoint{0.481978in}{0.331635in}}{\pgfqpoint{9.300000in}{7.700000in}}%
\pgfusepath{clip}%
\pgfsetrectcap%
\pgfsetroundjoin%
\pgfsetlinewidth{1.505625pt}%
\definecolor{currentstroke}{rgb}{0.631373,0.788235,0.956863}%
\pgfsetstrokecolor{currentstroke}%
\pgfsetstrokeopacity{0.800000}%
\pgfsetdash{}{0pt}%
\pgfpathmoveto{\pgfqpoint{8.275701in}{5.009576in}}%
\pgfpathlineto{\pgfqpoint{5.711094in}{3.773662in}}%
\pgfusepath{stroke}%
\end{pgfscope}%
\begin{pgfscope}%
\pgfpathrectangle{\pgfqpoint{0.481978in}{0.331635in}}{\pgfqpoint{9.300000in}{7.700000in}}%
\pgfusepath{clip}%
\pgfsetrectcap%
\pgfsetroundjoin%
\pgfsetlinewidth{1.505625pt}%
\definecolor{currentstroke}{rgb}{0.631373,0.788235,0.956863}%
\pgfsetstrokecolor{currentstroke}%
\pgfsetstrokeopacity{0.800000}%
\pgfsetdash{}{0pt}%
\pgfpathmoveto{\pgfqpoint{7.698866in}{5.118187in}}%
\pgfpathlineto{\pgfqpoint{5.711094in}{3.773662in}}%
\pgfusepath{stroke}%
\end{pgfscope}%
\begin{pgfscope}%
\pgfpathrectangle{\pgfqpoint{0.481978in}{0.331635in}}{\pgfqpoint{9.300000in}{7.700000in}}%
\pgfusepath{clip}%
\pgfsetrectcap%
\pgfsetroundjoin%
\pgfsetlinewidth{1.505625pt}%
\definecolor{currentstroke}{rgb}{0.631373,0.788235,0.956863}%
\pgfsetstrokecolor{currentstroke}%
\pgfsetstrokeopacity{0.800000}%
\pgfsetdash{}{0pt}%
\pgfpathmoveto{\pgfqpoint{2.744094in}{5.050535in}}%
\pgfpathlineto{\pgfqpoint{5.711094in}{3.773662in}}%
\pgfusepath{stroke}%
\end{pgfscope}%
\begin{pgfscope}%
\pgfpathrectangle{\pgfqpoint{0.481978in}{0.331635in}}{\pgfqpoint{9.300000in}{7.700000in}}%
\pgfusepath{clip}%
\pgfsetrectcap%
\pgfsetroundjoin%
\pgfsetlinewidth{1.505625pt}%
\definecolor{currentstroke}{rgb}{0.631373,0.788235,0.956863}%
\pgfsetstrokecolor{currentstroke}%
\pgfsetstrokeopacity{0.800000}%
\pgfsetdash{}{0pt}%
\pgfpathmoveto{\pgfqpoint{7.761828in}{4.319095in}}%
\pgfpathlineto{\pgfqpoint{5.711094in}{3.773662in}}%
\pgfusepath{stroke}%
\end{pgfscope}%
\begin{pgfscope}%
\pgfpathrectangle{\pgfqpoint{0.481978in}{0.331635in}}{\pgfqpoint{9.300000in}{7.700000in}}%
\pgfusepath{clip}%
\pgfsetrectcap%
\pgfsetroundjoin%
\pgfsetlinewidth{1.505625pt}%
\definecolor{currentstroke}{rgb}{0.631373,0.788235,0.956863}%
\pgfsetstrokecolor{currentstroke}%
\pgfsetstrokeopacity{0.800000}%
\pgfsetdash{}{0pt}%
\pgfpathmoveto{\pgfqpoint{6.999874in}{2.249399in}}%
\pgfpathlineto{\pgfqpoint{5.711094in}{3.773662in}}%
\pgfusepath{stroke}%
\end{pgfscope}%
\begin{pgfscope}%
\pgfpathrectangle{\pgfqpoint{0.481978in}{0.331635in}}{\pgfqpoint{9.300000in}{7.700000in}}%
\pgfusepath{clip}%
\pgfsetrectcap%
\pgfsetroundjoin%
\pgfsetlinewidth{1.505625pt}%
\definecolor{currentstroke}{rgb}{0.631373,0.788235,0.956863}%
\pgfsetstrokecolor{currentstroke}%
\pgfsetstrokeopacity{0.800000}%
\pgfsetdash{}{0pt}%
\pgfpathmoveto{\pgfqpoint{5.604350in}{5.430750in}}%
\pgfpathlineto{\pgfqpoint{5.711094in}{3.773662in}}%
\pgfusepath{stroke}%
\end{pgfscope}%
\begin{pgfscope}%
\pgfpathrectangle{\pgfqpoint{0.481978in}{0.331635in}}{\pgfqpoint{9.300000in}{7.700000in}}%
\pgfusepath{clip}%
\pgfsetrectcap%
\pgfsetroundjoin%
\pgfsetlinewidth{1.505625pt}%
\definecolor{currentstroke}{rgb}{0.631373,0.788235,0.956863}%
\pgfsetstrokecolor{currentstroke}%
\pgfsetstrokeopacity{0.800000}%
\pgfsetdash{}{0pt}%
\pgfpathmoveto{\pgfqpoint{2.349323in}{2.274882in}}%
\pgfpathlineto{\pgfqpoint{5.711094in}{3.773662in}}%
\pgfusepath{stroke}%
\end{pgfscope}%
\begin{pgfscope}%
\pgfpathrectangle{\pgfqpoint{0.481978in}{0.331635in}}{\pgfqpoint{9.300000in}{7.700000in}}%
\pgfusepath{clip}%
\pgfsetrectcap%
\pgfsetroundjoin%
\pgfsetlinewidth{1.505625pt}%
\definecolor{currentstroke}{rgb}{0.631373,0.788235,0.956863}%
\pgfsetstrokecolor{currentstroke}%
\pgfsetstrokeopacity{0.800000}%
\pgfsetdash{}{0pt}%
\pgfpathmoveto{\pgfqpoint{6.569928in}{3.257001in}}%
\pgfpathlineto{\pgfqpoint{5.711094in}{3.773662in}}%
\pgfusepath{stroke}%
\end{pgfscope}%
\begin{pgfscope}%
\pgfpathrectangle{\pgfqpoint{0.481978in}{0.331635in}}{\pgfqpoint{9.300000in}{7.700000in}}%
\pgfusepath{clip}%
\pgfsetrectcap%
\pgfsetroundjoin%
\pgfsetlinewidth{1.505625pt}%
\definecolor{currentstroke}{rgb}{0.631373,0.788235,0.956863}%
\pgfsetstrokecolor{currentstroke}%
\pgfsetstrokeopacity{0.800000}%
\pgfsetdash{}{0pt}%
\pgfpathmoveto{\pgfqpoint{5.147663in}{2.612815in}}%
\pgfpathlineto{\pgfqpoint{5.711094in}{3.773662in}}%
\pgfusepath{stroke}%
\end{pgfscope}%
\begin{pgfscope}%
\pgfpathrectangle{\pgfqpoint{0.481978in}{0.331635in}}{\pgfqpoint{9.300000in}{7.700000in}}%
\pgfusepath{clip}%
\pgfsetrectcap%
\pgfsetroundjoin%
\pgfsetlinewidth{1.505625pt}%
\definecolor{currentstroke}{rgb}{0.631373,0.788235,0.956863}%
\pgfsetstrokecolor{currentstroke}%
\pgfsetstrokeopacity{0.800000}%
\pgfsetdash{}{0pt}%
\pgfpathmoveto{\pgfqpoint{6.704806in}{2.233579in}}%
\pgfpathlineto{\pgfqpoint{5.711094in}{3.773662in}}%
\pgfusepath{stroke}%
\end{pgfscope}%
\begin{pgfscope}%
\pgfpathrectangle{\pgfqpoint{0.481978in}{0.331635in}}{\pgfqpoint{9.300000in}{7.700000in}}%
\pgfusepath{clip}%
\pgfsetrectcap%
\pgfsetroundjoin%
\pgfsetlinewidth{1.505625pt}%
\definecolor{currentstroke}{rgb}{0.631373,0.788235,0.956863}%
\pgfsetstrokecolor{currentstroke}%
\pgfsetstrokeopacity{0.800000}%
\pgfsetdash{}{0pt}%
\pgfpathmoveto{\pgfqpoint{2.794290in}{5.372826in}}%
\pgfpathlineto{\pgfqpoint{5.711094in}{3.773662in}}%
\pgfusepath{stroke}%
\end{pgfscope}%
\begin{pgfscope}%
\pgfpathrectangle{\pgfqpoint{0.481978in}{0.331635in}}{\pgfqpoint{9.300000in}{7.700000in}}%
\pgfusepath{clip}%
\pgfsetrectcap%
\pgfsetroundjoin%
\pgfsetlinewidth{1.505625pt}%
\definecolor{currentstroke}{rgb}{0.631373,0.788235,0.956863}%
\pgfsetstrokecolor{currentstroke}%
\pgfsetstrokeopacity{0.800000}%
\pgfsetdash{}{0pt}%
\pgfpathmoveto{\pgfqpoint{8.083200in}{5.007083in}}%
\pgfpathlineto{\pgfqpoint{5.711094in}{3.773662in}}%
\pgfusepath{stroke}%
\end{pgfscope}%
\begin{pgfscope}%
\pgfpathrectangle{\pgfqpoint{0.481978in}{0.331635in}}{\pgfqpoint{9.300000in}{7.700000in}}%
\pgfusepath{clip}%
\pgfsetrectcap%
\pgfsetroundjoin%
\pgfsetlinewidth{1.505625pt}%
\definecolor{currentstroke}{rgb}{0.631373,0.788235,0.956863}%
\pgfsetstrokecolor{currentstroke}%
\pgfsetstrokeopacity{0.800000}%
\pgfsetdash{}{0pt}%
\pgfpathmoveto{\pgfqpoint{4.500321in}{1.322703in}}%
\pgfpathlineto{\pgfqpoint{5.711094in}{3.773662in}}%
\pgfusepath{stroke}%
\end{pgfscope}%
\begin{pgfscope}%
\pgfpathrectangle{\pgfqpoint{0.481978in}{0.331635in}}{\pgfqpoint{9.300000in}{7.700000in}}%
\pgfusepath{clip}%
\pgfsetrectcap%
\pgfsetroundjoin%
\pgfsetlinewidth{1.505625pt}%
\definecolor{currentstroke}{rgb}{0.631373,0.788235,0.956863}%
\pgfsetstrokecolor{currentstroke}%
\pgfsetstrokeopacity{0.800000}%
\pgfsetdash{}{0pt}%
\pgfpathmoveto{\pgfqpoint{3.168751in}{1.375498in}}%
\pgfpathlineto{\pgfqpoint{5.711094in}{3.773662in}}%
\pgfusepath{stroke}%
\end{pgfscope}%
\begin{pgfscope}%
\pgfpathrectangle{\pgfqpoint{0.481978in}{0.331635in}}{\pgfqpoint{9.300000in}{7.700000in}}%
\pgfusepath{clip}%
\pgfsetrectcap%
\pgfsetroundjoin%
\pgfsetlinewidth{1.505625pt}%
\definecolor{currentstroke}{rgb}{0.631373,0.788235,0.956863}%
\pgfsetstrokecolor{currentstroke}%
\pgfsetstrokeopacity{0.800000}%
\pgfsetdash{}{0pt}%
\pgfpathmoveto{\pgfqpoint{4.177608in}{7.681635in}}%
\pgfpathlineto{\pgfqpoint{5.711094in}{3.773662in}}%
\pgfusepath{stroke}%
\end{pgfscope}%
\begin{pgfscope}%
\pgfpathrectangle{\pgfqpoint{0.481978in}{0.331635in}}{\pgfqpoint{9.300000in}{7.700000in}}%
\pgfusepath{clip}%
\pgfsetrectcap%
\pgfsetroundjoin%
\pgfsetlinewidth{1.505625pt}%
\definecolor{currentstroke}{rgb}{0.631373,0.788235,0.956863}%
\pgfsetstrokecolor{currentstroke}%
\pgfsetstrokeopacity{0.800000}%
\pgfsetdash{}{0pt}%
\pgfpathmoveto{\pgfqpoint{6.895459in}{3.175574in}}%
\pgfpathlineto{\pgfqpoint{5.711094in}{3.773662in}}%
\pgfusepath{stroke}%
\end{pgfscope}%
\begin{pgfscope}%
\pgfpathrectangle{\pgfqpoint{0.481978in}{0.331635in}}{\pgfqpoint{9.300000in}{7.700000in}}%
\pgfusepath{clip}%
\pgfsetrectcap%
\pgfsetroundjoin%
\pgfsetlinewidth{1.505625pt}%
\definecolor{currentstroke}{rgb}{0.631373,0.788235,0.956863}%
\pgfsetstrokecolor{currentstroke}%
\pgfsetstrokeopacity{0.800000}%
\pgfsetdash{}{0pt}%
\pgfpathmoveto{\pgfqpoint{7.647915in}{1.908788in}}%
\pgfpathlineto{\pgfqpoint{5.711094in}{3.773662in}}%
\pgfusepath{stroke}%
\end{pgfscope}%
\begin{pgfscope}%
\pgfpathrectangle{\pgfqpoint{0.481978in}{0.331635in}}{\pgfqpoint{9.300000in}{7.700000in}}%
\pgfusepath{clip}%
\pgfsetrectcap%
\pgfsetroundjoin%
\pgfsetlinewidth{1.505625pt}%
\definecolor{currentstroke}{rgb}{0.631373,0.788235,0.956863}%
\pgfsetstrokecolor{currentstroke}%
\pgfsetstrokeopacity{0.800000}%
\pgfsetdash{}{0pt}%
\pgfpathmoveto{\pgfqpoint{7.913291in}{4.980495in}}%
\pgfpathlineto{\pgfqpoint{5.711094in}{3.773662in}}%
\pgfusepath{stroke}%
\end{pgfscope}%
\begin{pgfscope}%
\pgfpathrectangle{\pgfqpoint{0.481978in}{0.331635in}}{\pgfqpoint{9.300000in}{7.700000in}}%
\pgfusepath{clip}%
\pgfsetrectcap%
\pgfsetroundjoin%
\pgfsetlinewidth{1.505625pt}%
\definecolor{currentstroke}{rgb}{0.631373,0.788235,0.956863}%
\pgfsetstrokecolor{currentstroke}%
\pgfsetstrokeopacity{0.800000}%
\pgfsetdash{}{0pt}%
\pgfpathmoveto{\pgfqpoint{5.605585in}{2.864452in}}%
\pgfpathlineto{\pgfqpoint{5.711094in}{3.773662in}}%
\pgfusepath{stroke}%
\end{pgfscope}%
\begin{pgfscope}%
\pgfpathrectangle{\pgfqpoint{0.481978in}{0.331635in}}{\pgfqpoint{9.300000in}{7.700000in}}%
\pgfusepath{clip}%
\pgfsetrectcap%
\pgfsetroundjoin%
\pgfsetlinewidth{1.505625pt}%
\definecolor{currentstroke}{rgb}{0.631373,0.788235,0.956863}%
\pgfsetstrokecolor{currentstroke}%
\pgfsetstrokeopacity{0.800000}%
\pgfsetdash{}{0pt}%
\pgfpathmoveto{\pgfqpoint{4.066101in}{5.156275in}}%
\pgfpathlineto{\pgfqpoint{5.711094in}{3.773662in}}%
\pgfusepath{stroke}%
\end{pgfscope}%
\begin{pgfscope}%
\pgfpathrectangle{\pgfqpoint{0.481978in}{0.331635in}}{\pgfqpoint{9.300000in}{7.700000in}}%
\pgfusepath{clip}%
\pgfsetrectcap%
\pgfsetroundjoin%
\pgfsetlinewidth{1.505625pt}%
\definecolor{currentstroke}{rgb}{0.631373,0.788235,0.956863}%
\pgfsetstrokecolor{currentstroke}%
\pgfsetstrokeopacity{0.800000}%
\pgfsetdash{}{0pt}%
\pgfpathmoveto{\pgfqpoint{6.162715in}{2.553984in}}%
\pgfpathlineto{\pgfqpoint{5.711094in}{3.773662in}}%
\pgfusepath{stroke}%
\end{pgfscope}%
\begin{pgfscope}%
\pgfpathrectangle{\pgfqpoint{0.481978in}{0.331635in}}{\pgfqpoint{9.300000in}{7.700000in}}%
\pgfusepath{clip}%
\pgfsetrectcap%
\pgfsetroundjoin%
\pgfsetlinewidth{1.505625pt}%
\definecolor{currentstroke}{rgb}{0.631373,0.788235,0.956863}%
\pgfsetstrokecolor{currentstroke}%
\pgfsetstrokeopacity{0.800000}%
\pgfsetdash{}{0pt}%
\pgfpathmoveto{\pgfqpoint{6.160576in}{3.132339in}}%
\pgfpathlineto{\pgfqpoint{5.711094in}{3.773662in}}%
\pgfusepath{stroke}%
\end{pgfscope}%
\begin{pgfscope}%
\pgfpathrectangle{\pgfqpoint{0.481978in}{0.331635in}}{\pgfqpoint{9.300000in}{7.700000in}}%
\pgfusepath{clip}%
\pgfsetrectcap%
\pgfsetroundjoin%
\pgfsetlinewidth{1.505625pt}%
\definecolor{currentstroke}{rgb}{0.631373,0.788235,0.956863}%
\pgfsetstrokecolor{currentstroke}%
\pgfsetstrokeopacity{0.800000}%
\pgfsetdash{}{0pt}%
\pgfpathmoveto{\pgfqpoint{4.893800in}{6.410049in}}%
\pgfpathlineto{\pgfqpoint{5.711094in}{3.773662in}}%
\pgfusepath{stroke}%
\end{pgfscope}%
\begin{pgfscope}%
\pgfpathrectangle{\pgfqpoint{0.481978in}{0.331635in}}{\pgfqpoint{9.300000in}{7.700000in}}%
\pgfusepath{clip}%
\pgfsetrectcap%
\pgfsetroundjoin%
\pgfsetlinewidth{1.505625pt}%
\definecolor{currentstroke}{rgb}{0.631373,0.788235,0.956863}%
\pgfsetstrokecolor{currentstroke}%
\pgfsetstrokeopacity{0.800000}%
\pgfsetdash{}{0pt}%
\pgfpathmoveto{\pgfqpoint{3.428502in}{1.854777in}}%
\pgfpathlineto{\pgfqpoint{5.711094in}{3.773662in}}%
\pgfusepath{stroke}%
\end{pgfscope}%
\begin{pgfscope}%
\pgfpathrectangle{\pgfqpoint{0.481978in}{0.331635in}}{\pgfqpoint{9.300000in}{7.700000in}}%
\pgfusepath{clip}%
\pgfsetrectcap%
\pgfsetroundjoin%
\pgfsetlinewidth{1.505625pt}%
\definecolor{currentstroke}{rgb}{0.631373,0.788235,0.956863}%
\pgfsetstrokecolor{currentstroke}%
\pgfsetstrokeopacity{0.800000}%
\pgfsetdash{}{0pt}%
\pgfpathmoveto{\pgfqpoint{3.370561in}{7.155344in}}%
\pgfpathlineto{\pgfqpoint{5.711094in}{3.773662in}}%
\pgfusepath{stroke}%
\end{pgfscope}%
\begin{pgfscope}%
\pgfpathrectangle{\pgfqpoint{0.481978in}{0.331635in}}{\pgfqpoint{9.300000in}{7.700000in}}%
\pgfusepath{clip}%
\pgfsetrectcap%
\pgfsetroundjoin%
\pgfsetlinewidth{1.505625pt}%
\definecolor{currentstroke}{rgb}{0.631373,0.788235,0.956863}%
\pgfsetstrokecolor{currentstroke}%
\pgfsetstrokeopacity{0.800000}%
\pgfsetdash{}{0pt}%
\pgfpathmoveto{\pgfqpoint{6.915286in}{4.198455in}}%
\pgfpathlineto{\pgfqpoint{5.711094in}{3.773662in}}%
\pgfusepath{stroke}%
\end{pgfscope}%
\begin{pgfscope}%
\pgfpathrectangle{\pgfqpoint{0.481978in}{0.331635in}}{\pgfqpoint{9.300000in}{7.700000in}}%
\pgfusepath{clip}%
\pgfsetrectcap%
\pgfsetroundjoin%
\pgfsetlinewidth{1.505625pt}%
\definecolor{currentstroke}{rgb}{0.631373,0.788235,0.956863}%
\pgfsetstrokecolor{currentstroke}%
\pgfsetstrokeopacity{0.800000}%
\pgfsetdash{}{0pt}%
\pgfpathmoveto{\pgfqpoint{3.604723in}{5.526537in}}%
\pgfpathlineto{\pgfqpoint{5.711094in}{3.773662in}}%
\pgfusepath{stroke}%
\end{pgfscope}%
\begin{pgfscope}%
\pgfpathrectangle{\pgfqpoint{0.481978in}{0.331635in}}{\pgfqpoint{9.300000in}{7.700000in}}%
\pgfusepath{clip}%
\pgfsetrectcap%
\pgfsetroundjoin%
\pgfsetlinewidth{1.505625pt}%
\definecolor{currentstroke}{rgb}{0.631373,0.788235,0.956863}%
\pgfsetstrokecolor{currentstroke}%
\pgfsetstrokeopacity{0.800000}%
\pgfsetdash{}{0pt}%
\pgfpathmoveto{\pgfqpoint{4.283816in}{3.539087in}}%
\pgfpathlineto{\pgfqpoint{5.711094in}{3.773662in}}%
\pgfusepath{stroke}%
\end{pgfscope}%
\begin{pgfscope}%
\pgfpathrectangle{\pgfqpoint{0.481978in}{0.331635in}}{\pgfqpoint{9.300000in}{7.700000in}}%
\pgfusepath{clip}%
\pgfsetrectcap%
\pgfsetroundjoin%
\pgfsetlinewidth{1.505625pt}%
\definecolor{currentstroke}{rgb}{0.631373,0.788235,0.956863}%
\pgfsetstrokecolor{currentstroke}%
\pgfsetstrokeopacity{0.800000}%
\pgfsetdash{}{0pt}%
\pgfpathmoveto{\pgfqpoint{7.064950in}{1.944367in}}%
\pgfpathlineto{\pgfqpoint{5.711094in}{3.773662in}}%
\pgfusepath{stroke}%
\end{pgfscope}%
\begin{pgfscope}%
\pgfpathrectangle{\pgfqpoint{0.481978in}{0.331635in}}{\pgfqpoint{9.300000in}{7.700000in}}%
\pgfusepath{clip}%
\pgfsetrectcap%
\pgfsetroundjoin%
\pgfsetlinewidth{1.505625pt}%
\definecolor{currentstroke}{rgb}{0.631373,0.788235,0.956863}%
\pgfsetstrokecolor{currentstroke}%
\pgfsetstrokeopacity{0.800000}%
\pgfsetdash{}{0pt}%
\pgfpathmoveto{\pgfqpoint{3.497950in}{6.643419in}}%
\pgfpathlineto{\pgfqpoint{5.711094in}{3.773662in}}%
\pgfusepath{stroke}%
\end{pgfscope}%
\begin{pgfscope}%
\pgfpathrectangle{\pgfqpoint{0.481978in}{0.331635in}}{\pgfqpoint{9.300000in}{7.700000in}}%
\pgfusepath{clip}%
\pgfsetrectcap%
\pgfsetroundjoin%
\pgfsetlinewidth{1.505625pt}%
\definecolor{currentstroke}{rgb}{0.631373,0.788235,0.956863}%
\pgfsetstrokecolor{currentstroke}%
\pgfsetstrokeopacity{0.800000}%
\pgfsetdash{}{0pt}%
\pgfpathmoveto{\pgfqpoint{5.579377in}{3.207767in}}%
\pgfpathlineto{\pgfqpoint{5.711094in}{3.773662in}}%
\pgfusepath{stroke}%
\end{pgfscope}%
\begin{pgfscope}%
\pgfpathrectangle{\pgfqpoint{0.481978in}{0.331635in}}{\pgfqpoint{9.300000in}{7.700000in}}%
\pgfusepath{clip}%
\pgfsetrectcap%
\pgfsetroundjoin%
\pgfsetlinewidth{1.505625pt}%
\definecolor{currentstroke}{rgb}{0.631373,0.788235,0.956863}%
\pgfsetstrokecolor{currentstroke}%
\pgfsetstrokeopacity{0.800000}%
\pgfsetdash{}{0pt}%
\pgfpathmoveto{\pgfqpoint{8.844215in}{5.072545in}}%
\pgfpathlineto{\pgfqpoint{5.711094in}{3.773662in}}%
\pgfusepath{stroke}%
\end{pgfscope}%
\begin{pgfscope}%
\pgfpathrectangle{\pgfqpoint{0.481978in}{0.331635in}}{\pgfqpoint{9.300000in}{7.700000in}}%
\pgfusepath{clip}%
\pgfsetrectcap%
\pgfsetroundjoin%
\pgfsetlinewidth{1.505625pt}%
\definecolor{currentstroke}{rgb}{0.631373,0.788235,0.956863}%
\pgfsetstrokecolor{currentstroke}%
\pgfsetstrokeopacity{0.800000}%
\pgfsetdash{}{0pt}%
\pgfpathmoveto{\pgfqpoint{4.858261in}{1.852850in}}%
\pgfpathlineto{\pgfqpoint{5.711094in}{3.773662in}}%
\pgfusepath{stroke}%
\end{pgfscope}%
\begin{pgfscope}%
\pgfpathrectangle{\pgfqpoint{0.481978in}{0.331635in}}{\pgfqpoint{9.300000in}{7.700000in}}%
\pgfusepath{clip}%
\pgfsetrectcap%
\pgfsetroundjoin%
\pgfsetlinewidth{1.505625pt}%
\definecolor{currentstroke}{rgb}{0.631373,0.788235,0.956863}%
\pgfsetstrokecolor{currentstroke}%
\pgfsetstrokeopacity{0.800000}%
\pgfsetdash{}{0pt}%
\pgfpathmoveto{\pgfqpoint{7.438148in}{1.970168in}}%
\pgfpathlineto{\pgfqpoint{5.711094in}{3.773662in}}%
\pgfusepath{stroke}%
\end{pgfscope}%
\begin{pgfscope}%
\pgfpathrectangle{\pgfqpoint{0.481978in}{0.331635in}}{\pgfqpoint{9.300000in}{7.700000in}}%
\pgfusepath{clip}%
\pgfsetrectcap%
\pgfsetroundjoin%
\pgfsetlinewidth{1.505625pt}%
\definecolor{currentstroke}{rgb}{0.631373,0.788235,0.956863}%
\pgfsetstrokecolor{currentstroke}%
\pgfsetstrokeopacity{0.800000}%
\pgfsetdash{}{0pt}%
\pgfpathmoveto{\pgfqpoint{3.422792in}{1.040261in}}%
\pgfpathlineto{\pgfqpoint{5.711094in}{3.773662in}}%
\pgfusepath{stroke}%
\end{pgfscope}%
\begin{pgfscope}%
\pgfpathrectangle{\pgfqpoint{0.481978in}{0.331635in}}{\pgfqpoint{9.300000in}{7.700000in}}%
\pgfusepath{clip}%
\pgfsetrectcap%
\pgfsetroundjoin%
\pgfsetlinewidth{1.505625pt}%
\definecolor{currentstroke}{rgb}{0.631373,0.788235,0.956863}%
\pgfsetstrokecolor{currentstroke}%
\pgfsetstrokeopacity{0.800000}%
\pgfsetdash{}{0pt}%
\pgfpathmoveto{\pgfqpoint{2.139723in}{2.484932in}}%
\pgfpathlineto{\pgfqpoint{5.711094in}{3.773662in}}%
\pgfusepath{stroke}%
\end{pgfscope}%
\begin{pgfscope}%
\pgfpathrectangle{\pgfqpoint{0.481978in}{0.331635in}}{\pgfqpoint{9.300000in}{7.700000in}}%
\pgfusepath{clip}%
\pgfsetrectcap%
\pgfsetroundjoin%
\pgfsetlinewidth{1.505625pt}%
\definecolor{currentstroke}{rgb}{0.631373,0.788235,0.956863}%
\pgfsetstrokecolor{currentstroke}%
\pgfsetstrokeopacity{0.800000}%
\pgfsetdash{}{0pt}%
\pgfpathmoveto{\pgfqpoint{8.049554in}{5.466103in}}%
\pgfpathlineto{\pgfqpoint{5.711094in}{3.773662in}}%
\pgfusepath{stroke}%
\end{pgfscope}%
\begin{pgfscope}%
\pgfpathrectangle{\pgfqpoint{0.481978in}{0.331635in}}{\pgfqpoint{9.300000in}{7.700000in}}%
\pgfusepath{clip}%
\pgfsetrectcap%
\pgfsetroundjoin%
\pgfsetlinewidth{1.505625pt}%
\definecolor{currentstroke}{rgb}{0.631373,0.788235,0.956863}%
\pgfsetstrokecolor{currentstroke}%
\pgfsetstrokeopacity{0.800000}%
\pgfsetdash{}{0pt}%
\pgfpathmoveto{\pgfqpoint{3.389691in}{1.627458in}}%
\pgfpathlineto{\pgfqpoint{5.711094in}{3.773662in}}%
\pgfusepath{stroke}%
\end{pgfscope}%
\begin{pgfscope}%
\pgfpathrectangle{\pgfqpoint{0.481978in}{0.331635in}}{\pgfqpoint{9.300000in}{7.700000in}}%
\pgfusepath{clip}%
\pgfsetrectcap%
\pgfsetroundjoin%
\pgfsetlinewidth{1.505625pt}%
\definecolor{currentstroke}{rgb}{0.631373,0.788235,0.956863}%
\pgfsetstrokecolor{currentstroke}%
\pgfsetstrokeopacity{0.800000}%
\pgfsetdash{}{0pt}%
\pgfpathmoveto{\pgfqpoint{6.131345in}{5.250597in}}%
\pgfpathlineto{\pgfqpoint{5.711094in}{3.773662in}}%
\pgfusepath{stroke}%
\end{pgfscope}%
\begin{pgfscope}%
\pgfpathrectangle{\pgfqpoint{0.481978in}{0.331635in}}{\pgfqpoint{9.300000in}{7.700000in}}%
\pgfusepath{clip}%
\pgfsetrectcap%
\pgfsetroundjoin%
\pgfsetlinewidth{1.505625pt}%
\definecolor{currentstroke}{rgb}{0.631373,0.788235,0.956863}%
\pgfsetstrokecolor{currentstroke}%
\pgfsetstrokeopacity{0.800000}%
\pgfsetdash{}{0pt}%
\pgfpathmoveto{\pgfqpoint{4.856079in}{6.415647in}}%
\pgfpathlineto{\pgfqpoint{5.711094in}{3.773662in}}%
\pgfusepath{stroke}%
\end{pgfscope}%
\begin{pgfscope}%
\pgfpathrectangle{\pgfqpoint{0.481978in}{0.331635in}}{\pgfqpoint{9.300000in}{7.700000in}}%
\pgfusepath{clip}%
\pgfsetrectcap%
\pgfsetroundjoin%
\pgfsetlinewidth{1.505625pt}%
\definecolor{currentstroke}{rgb}{0.631373,0.788235,0.956863}%
\pgfsetstrokecolor{currentstroke}%
\pgfsetstrokeopacity{0.800000}%
\pgfsetdash{}{0pt}%
\pgfpathmoveto{\pgfqpoint{4.975752in}{4.620168in}}%
\pgfpathlineto{\pgfqpoint{5.711094in}{3.773662in}}%
\pgfusepath{stroke}%
\end{pgfscope}%
\begin{pgfscope}%
\pgfpathrectangle{\pgfqpoint{0.481978in}{0.331635in}}{\pgfqpoint{9.300000in}{7.700000in}}%
\pgfusepath{clip}%
\pgfsetrectcap%
\pgfsetroundjoin%
\pgfsetlinewidth{1.505625pt}%
\definecolor{currentstroke}{rgb}{0.631373,0.788235,0.956863}%
\pgfsetstrokecolor{currentstroke}%
\pgfsetstrokeopacity{0.800000}%
\pgfsetdash{}{0pt}%
\pgfpathmoveto{\pgfqpoint{5.729756in}{5.746283in}}%
\pgfpathlineto{\pgfqpoint{5.711094in}{3.773662in}}%
\pgfusepath{stroke}%
\end{pgfscope}%
\begin{pgfscope}%
\pgfpathrectangle{\pgfqpoint{0.481978in}{0.331635in}}{\pgfqpoint{9.300000in}{7.700000in}}%
\pgfusepath{clip}%
\pgfsetrectcap%
\pgfsetroundjoin%
\pgfsetlinewidth{1.505625pt}%
\definecolor{currentstroke}{rgb}{0.631373,0.788235,0.956863}%
\pgfsetstrokecolor{currentstroke}%
\pgfsetstrokeopacity{0.800000}%
\pgfsetdash{}{0pt}%
\pgfpathmoveto{\pgfqpoint{5.526353in}{1.663412in}}%
\pgfpathlineto{\pgfqpoint{5.711094in}{3.773662in}}%
\pgfusepath{stroke}%
\end{pgfscope}%
\begin{pgfscope}%
\pgfpathrectangle{\pgfqpoint{0.481978in}{0.331635in}}{\pgfqpoint{9.300000in}{7.700000in}}%
\pgfusepath{clip}%
\pgfsetrectcap%
\pgfsetroundjoin%
\pgfsetlinewidth{1.505625pt}%
\definecolor{currentstroke}{rgb}{0.631373,0.788235,0.956863}%
\pgfsetstrokecolor{currentstroke}%
\pgfsetstrokeopacity{0.800000}%
\pgfsetdash{}{0pt}%
\pgfpathmoveto{\pgfqpoint{6.278735in}{2.430417in}}%
\pgfpathlineto{\pgfqpoint{5.711094in}{3.773662in}}%
\pgfusepath{stroke}%
\end{pgfscope}%
\begin{pgfscope}%
\pgfpathrectangle{\pgfqpoint{0.481978in}{0.331635in}}{\pgfqpoint{9.300000in}{7.700000in}}%
\pgfusepath{clip}%
\pgfsetrectcap%
\pgfsetroundjoin%
\pgfsetlinewidth{1.505625pt}%
\definecolor{currentstroke}{rgb}{0.631373,0.788235,0.956863}%
\pgfsetstrokecolor{currentstroke}%
\pgfsetstrokeopacity{0.800000}%
\pgfsetdash{}{0pt}%
\pgfpathmoveto{\pgfqpoint{2.008732in}{2.146558in}}%
\pgfpathlineto{\pgfqpoint{5.711094in}{3.773662in}}%
\pgfusepath{stroke}%
\end{pgfscope}%
\begin{pgfscope}%
\pgfpathrectangle{\pgfqpoint{0.481978in}{0.331635in}}{\pgfqpoint{9.300000in}{7.700000in}}%
\pgfusepath{clip}%
\pgfsetrectcap%
\pgfsetroundjoin%
\pgfsetlinewidth{1.505625pt}%
\definecolor{currentstroke}{rgb}{0.631373,0.788235,0.956863}%
\pgfsetstrokecolor{currentstroke}%
\pgfsetstrokeopacity{0.800000}%
\pgfsetdash{}{0pt}%
\pgfpathmoveto{\pgfqpoint{5.928515in}{1.775836in}}%
\pgfpathlineto{\pgfqpoint{5.711094in}{3.773662in}}%
\pgfusepath{stroke}%
\end{pgfscope}%
\begin{pgfscope}%
\pgfpathrectangle{\pgfqpoint{0.481978in}{0.331635in}}{\pgfqpoint{9.300000in}{7.700000in}}%
\pgfusepath{clip}%
\pgfsetrectcap%
\pgfsetroundjoin%
\pgfsetlinewidth{1.505625pt}%
\definecolor{currentstroke}{rgb}{0.631373,0.788235,0.956863}%
\pgfsetstrokecolor{currentstroke}%
\pgfsetstrokeopacity{0.800000}%
\pgfsetdash{}{0pt}%
\pgfpathmoveto{\pgfqpoint{6.134955in}{2.473624in}}%
\pgfpathlineto{\pgfqpoint{5.711094in}{3.773662in}}%
\pgfusepath{stroke}%
\end{pgfscope}%
\begin{pgfscope}%
\pgfpathrectangle{\pgfqpoint{0.481978in}{0.331635in}}{\pgfqpoint{9.300000in}{7.700000in}}%
\pgfusepath{clip}%
\pgfsetrectcap%
\pgfsetroundjoin%
\pgfsetlinewidth{1.505625pt}%
\definecolor{currentstroke}{rgb}{0.631373,0.788235,0.956863}%
\pgfsetstrokecolor{currentstroke}%
\pgfsetstrokeopacity{0.800000}%
\pgfsetdash{}{0pt}%
\pgfpathmoveto{\pgfqpoint{5.647865in}{3.664159in}}%
\pgfpathlineto{\pgfqpoint{5.711094in}{3.773662in}}%
\pgfusepath{stroke}%
\end{pgfscope}%
\begin{pgfscope}%
\pgfpathrectangle{\pgfqpoint{0.481978in}{0.331635in}}{\pgfqpoint{9.300000in}{7.700000in}}%
\pgfusepath{clip}%
\pgfsetrectcap%
\pgfsetroundjoin%
\pgfsetlinewidth{1.505625pt}%
\definecolor{currentstroke}{rgb}{0.631373,0.788235,0.956863}%
\pgfsetstrokecolor{currentstroke}%
\pgfsetstrokeopacity{0.800000}%
\pgfsetdash{}{0pt}%
\pgfpathmoveto{\pgfqpoint{6.505006in}{5.447387in}}%
\pgfpathlineto{\pgfqpoint{5.711094in}{3.773662in}}%
\pgfusepath{stroke}%
\end{pgfscope}%
\begin{pgfscope}%
\pgfpathrectangle{\pgfqpoint{0.481978in}{0.331635in}}{\pgfqpoint{9.300000in}{7.700000in}}%
\pgfusepath{clip}%
\pgfsetrectcap%
\pgfsetroundjoin%
\pgfsetlinewidth{1.505625pt}%
\definecolor{currentstroke}{rgb}{0.631373,0.788235,0.956863}%
\pgfsetstrokecolor{currentstroke}%
\pgfsetstrokeopacity{0.800000}%
\pgfsetdash{}{0pt}%
\pgfpathmoveto{\pgfqpoint{6.819209in}{3.684591in}}%
\pgfpathlineto{\pgfqpoint{5.711094in}{3.773662in}}%
\pgfusepath{stroke}%
\end{pgfscope}%
\begin{pgfscope}%
\pgfpathrectangle{\pgfqpoint{0.481978in}{0.331635in}}{\pgfqpoint{9.300000in}{7.700000in}}%
\pgfusepath{clip}%
\pgfsetrectcap%
\pgfsetroundjoin%
\pgfsetlinewidth{1.505625pt}%
\definecolor{currentstroke}{rgb}{0.631373,0.788235,0.956863}%
\pgfsetstrokecolor{currentstroke}%
\pgfsetstrokeopacity{0.800000}%
\pgfsetdash{}{0pt}%
\pgfpathmoveto{\pgfqpoint{7.024686in}{3.334185in}}%
\pgfpathlineto{\pgfqpoint{5.711094in}{3.773662in}}%
\pgfusepath{stroke}%
\end{pgfscope}%
\begin{pgfscope}%
\pgfpathrectangle{\pgfqpoint{0.481978in}{0.331635in}}{\pgfqpoint{9.300000in}{7.700000in}}%
\pgfusepath{clip}%
\pgfsetrectcap%
\pgfsetroundjoin%
\pgfsetlinewidth{1.505625pt}%
\definecolor{currentstroke}{rgb}{0.631373,0.788235,0.956863}%
\pgfsetstrokecolor{currentstroke}%
\pgfsetstrokeopacity{0.800000}%
\pgfsetdash{}{0pt}%
\pgfpathmoveto{\pgfqpoint{2.950033in}{6.709299in}}%
\pgfpathlineto{\pgfqpoint{5.711094in}{3.773662in}}%
\pgfusepath{stroke}%
\end{pgfscope}%
\begin{pgfscope}%
\pgfpathrectangle{\pgfqpoint{0.481978in}{0.331635in}}{\pgfqpoint{9.300000in}{7.700000in}}%
\pgfusepath{clip}%
\pgfsetrectcap%
\pgfsetroundjoin%
\pgfsetlinewidth{1.505625pt}%
\definecolor{currentstroke}{rgb}{0.631373,0.788235,0.956863}%
\pgfsetstrokecolor{currentstroke}%
\pgfsetstrokeopacity{0.800000}%
\pgfsetdash{}{0pt}%
\pgfpathmoveto{\pgfqpoint{4.721839in}{7.179361in}}%
\pgfpathlineto{\pgfqpoint{5.711094in}{3.773662in}}%
\pgfusepath{stroke}%
\end{pgfscope}%
\begin{pgfscope}%
\pgfpathrectangle{\pgfqpoint{0.481978in}{0.331635in}}{\pgfqpoint{9.300000in}{7.700000in}}%
\pgfusepath{clip}%
\pgfsetrectcap%
\pgfsetroundjoin%
\pgfsetlinewidth{1.505625pt}%
\definecolor{currentstroke}{rgb}{0.631373,0.788235,0.956863}%
\pgfsetstrokecolor{currentstroke}%
\pgfsetstrokeopacity{0.800000}%
\pgfsetdash{}{0pt}%
\pgfpathmoveto{\pgfqpoint{6.257325in}{5.337340in}}%
\pgfpathlineto{\pgfqpoint{5.711094in}{3.773662in}}%
\pgfusepath{stroke}%
\end{pgfscope}%
\begin{pgfscope}%
\pgfpathrectangle{\pgfqpoint{0.481978in}{0.331635in}}{\pgfqpoint{9.300000in}{7.700000in}}%
\pgfusepath{clip}%
\pgfsetrectcap%
\pgfsetroundjoin%
\pgfsetlinewidth{1.505625pt}%
\definecolor{currentstroke}{rgb}{0.631373,0.788235,0.956863}%
\pgfsetstrokecolor{currentstroke}%
\pgfsetstrokeopacity{0.800000}%
\pgfsetdash{}{0pt}%
\pgfpathmoveto{\pgfqpoint{7.727200in}{4.854891in}}%
\pgfpathlineto{\pgfqpoint{5.711094in}{3.773662in}}%
\pgfusepath{stroke}%
\end{pgfscope}%
\begin{pgfscope}%
\pgfpathrectangle{\pgfqpoint{0.481978in}{0.331635in}}{\pgfqpoint{9.300000in}{7.700000in}}%
\pgfusepath{clip}%
\pgfsetrectcap%
\pgfsetroundjoin%
\pgfsetlinewidth{1.505625pt}%
\definecolor{currentstroke}{rgb}{0.631373,0.788235,0.956863}%
\pgfsetstrokecolor{currentstroke}%
\pgfsetstrokeopacity{0.800000}%
\pgfsetdash{}{0pt}%
\pgfpathmoveto{\pgfqpoint{3.030295in}{1.728271in}}%
\pgfpathlineto{\pgfqpoint{5.711094in}{3.773662in}}%
\pgfusepath{stroke}%
\end{pgfscope}%
\begin{pgfscope}%
\pgfpathrectangle{\pgfqpoint{0.481978in}{0.331635in}}{\pgfqpoint{9.300000in}{7.700000in}}%
\pgfusepath{clip}%
\pgfsetrectcap%
\pgfsetroundjoin%
\pgfsetlinewidth{1.505625pt}%
\definecolor{currentstroke}{rgb}{0.631373,0.788235,0.956863}%
\pgfsetstrokecolor{currentstroke}%
\pgfsetstrokeopacity{0.800000}%
\pgfsetdash{}{0pt}%
\pgfpathmoveto{\pgfqpoint{4.665732in}{6.497651in}}%
\pgfpathlineto{\pgfqpoint{5.711094in}{3.773662in}}%
\pgfusepath{stroke}%
\end{pgfscope}%
\begin{pgfscope}%
\pgfpathrectangle{\pgfqpoint{0.481978in}{0.331635in}}{\pgfqpoint{9.300000in}{7.700000in}}%
\pgfusepath{clip}%
\pgfsetrectcap%
\pgfsetroundjoin%
\pgfsetlinewidth{1.505625pt}%
\definecolor{currentstroke}{rgb}{0.631373,0.788235,0.956863}%
\pgfsetstrokecolor{currentstroke}%
\pgfsetstrokeopacity{0.800000}%
\pgfsetdash{}{0pt}%
\pgfpathmoveto{\pgfqpoint{6.123240in}{2.041304in}}%
\pgfpathlineto{\pgfqpoint{5.711094in}{3.773662in}}%
\pgfusepath{stroke}%
\end{pgfscope}%
\begin{pgfscope}%
\pgfpathrectangle{\pgfqpoint{0.481978in}{0.331635in}}{\pgfqpoint{9.300000in}{7.700000in}}%
\pgfusepath{clip}%
\pgfsetrectcap%
\pgfsetroundjoin%
\pgfsetlinewidth{1.505625pt}%
\definecolor{currentstroke}{rgb}{0.631373,0.788235,0.956863}%
\pgfsetstrokecolor{currentstroke}%
\pgfsetstrokeopacity{0.800000}%
\pgfsetdash{}{0pt}%
\pgfpathmoveto{\pgfqpoint{2.813753in}{6.523043in}}%
\pgfpathlineto{\pgfqpoint{5.711094in}{3.773662in}}%
\pgfusepath{stroke}%
\end{pgfscope}%
\begin{pgfscope}%
\pgfpathrectangle{\pgfqpoint{0.481978in}{0.331635in}}{\pgfqpoint{9.300000in}{7.700000in}}%
\pgfusepath{clip}%
\pgfsetrectcap%
\pgfsetroundjoin%
\pgfsetlinewidth{1.505625pt}%
\definecolor{currentstroke}{rgb}{0.631373,0.788235,0.956863}%
\pgfsetstrokecolor{currentstroke}%
\pgfsetstrokeopacity{0.800000}%
\pgfsetdash{}{0pt}%
\pgfpathmoveto{\pgfqpoint{8.071034in}{4.423951in}}%
\pgfpathlineto{\pgfqpoint{5.711094in}{3.773662in}}%
\pgfusepath{stroke}%
\end{pgfscope}%
\begin{pgfscope}%
\pgfpathrectangle{\pgfqpoint{0.481978in}{0.331635in}}{\pgfqpoint{9.300000in}{7.700000in}}%
\pgfusepath{clip}%
\pgfsetrectcap%
\pgfsetroundjoin%
\pgfsetlinewidth{1.505625pt}%
\definecolor{currentstroke}{rgb}{0.631373,0.788235,0.956863}%
\pgfsetstrokecolor{currentstroke}%
\pgfsetstrokeopacity{0.800000}%
\pgfsetdash{}{0pt}%
\pgfpathmoveto{\pgfqpoint{6.467051in}{2.365045in}}%
\pgfpathlineto{\pgfqpoint{5.711094in}{3.773662in}}%
\pgfusepath{stroke}%
\end{pgfscope}%
\begin{pgfscope}%
\pgfpathrectangle{\pgfqpoint{0.481978in}{0.331635in}}{\pgfqpoint{9.300000in}{7.700000in}}%
\pgfusepath{clip}%
\pgfsetrectcap%
\pgfsetroundjoin%
\pgfsetlinewidth{1.505625pt}%
\definecolor{currentstroke}{rgb}{0.631373,0.788235,0.956863}%
\pgfsetstrokecolor{currentstroke}%
\pgfsetstrokeopacity{0.800000}%
\pgfsetdash{}{0pt}%
\pgfpathmoveto{\pgfqpoint{7.133111in}{4.714051in}}%
\pgfpathlineto{\pgfqpoint{5.711094in}{3.773662in}}%
\pgfusepath{stroke}%
\end{pgfscope}%
\begin{pgfscope}%
\pgfpathrectangle{\pgfqpoint{0.481978in}{0.331635in}}{\pgfqpoint{9.300000in}{7.700000in}}%
\pgfusepath{clip}%
\pgfsetrectcap%
\pgfsetroundjoin%
\pgfsetlinewidth{1.505625pt}%
\definecolor{currentstroke}{rgb}{0.631373,0.788235,0.956863}%
\pgfsetstrokecolor{currentstroke}%
\pgfsetstrokeopacity{0.800000}%
\pgfsetdash{}{0pt}%
\pgfpathmoveto{\pgfqpoint{5.228856in}{2.192608in}}%
\pgfpathlineto{\pgfqpoint{5.711094in}{3.773662in}}%
\pgfusepath{stroke}%
\end{pgfscope}%
\begin{pgfscope}%
\pgfpathrectangle{\pgfqpoint{0.481978in}{0.331635in}}{\pgfqpoint{9.300000in}{7.700000in}}%
\pgfusepath{clip}%
\pgfsetrectcap%
\pgfsetroundjoin%
\pgfsetlinewidth{1.505625pt}%
\definecolor{currentstroke}{rgb}{0.631373,0.788235,0.956863}%
\pgfsetstrokecolor{currentstroke}%
\pgfsetstrokeopacity{0.800000}%
\pgfsetdash{}{0pt}%
\pgfpathmoveto{\pgfqpoint{6.173996in}{3.318811in}}%
\pgfpathlineto{\pgfqpoint{5.711094in}{3.773662in}}%
\pgfusepath{stroke}%
\end{pgfscope}%
\begin{pgfscope}%
\pgfpathrectangle{\pgfqpoint{0.481978in}{0.331635in}}{\pgfqpoint{9.300000in}{7.700000in}}%
\pgfusepath{clip}%
\pgfsetrectcap%
\pgfsetroundjoin%
\pgfsetlinewidth{1.505625pt}%
\definecolor{currentstroke}{rgb}{0.631373,0.788235,0.956863}%
\pgfsetstrokecolor{currentstroke}%
\pgfsetstrokeopacity{0.800000}%
\pgfsetdash{}{0pt}%
\pgfpathmoveto{\pgfqpoint{2.961409in}{1.537320in}}%
\pgfpathlineto{\pgfqpoint{5.711094in}{3.773662in}}%
\pgfusepath{stroke}%
\end{pgfscope}%
\begin{pgfscope}%
\pgfpathrectangle{\pgfqpoint{0.481978in}{0.331635in}}{\pgfqpoint{9.300000in}{7.700000in}}%
\pgfusepath{clip}%
\pgfsetrectcap%
\pgfsetroundjoin%
\pgfsetlinewidth{1.505625pt}%
\definecolor{currentstroke}{rgb}{0.631373,0.788235,0.956863}%
\pgfsetstrokecolor{currentstroke}%
\pgfsetstrokeopacity{0.800000}%
\pgfsetdash{}{0pt}%
\pgfpathmoveto{\pgfqpoint{5.678749in}{4.888232in}}%
\pgfpathlineto{\pgfqpoint{5.711094in}{3.773662in}}%
\pgfusepath{stroke}%
\end{pgfscope}%
\begin{pgfscope}%
\pgfpathrectangle{\pgfqpoint{0.481978in}{0.331635in}}{\pgfqpoint{9.300000in}{7.700000in}}%
\pgfusepath{clip}%
\pgfsetrectcap%
\pgfsetroundjoin%
\pgfsetlinewidth{1.505625pt}%
\definecolor{currentstroke}{rgb}{0.631373,0.788235,0.956863}%
\pgfsetstrokecolor{currentstroke}%
\pgfsetstrokeopacity{0.800000}%
\pgfsetdash{}{0pt}%
\pgfpathmoveto{\pgfqpoint{4.278462in}{1.312611in}}%
\pgfpathlineto{\pgfqpoint{5.711094in}{3.773662in}}%
\pgfusepath{stroke}%
\end{pgfscope}%
\begin{pgfscope}%
\pgfpathrectangle{\pgfqpoint{0.481978in}{0.331635in}}{\pgfqpoint{9.300000in}{7.700000in}}%
\pgfusepath{clip}%
\pgfsetrectcap%
\pgfsetroundjoin%
\pgfsetlinewidth{1.505625pt}%
\definecolor{currentstroke}{rgb}{0.631373,0.788235,0.956863}%
\pgfsetstrokecolor{currentstroke}%
\pgfsetstrokeopacity{0.800000}%
\pgfsetdash{}{0pt}%
\pgfpathmoveto{\pgfqpoint{6.739909in}{2.297981in}}%
\pgfpathlineto{\pgfqpoint{5.711094in}{3.773662in}}%
\pgfusepath{stroke}%
\end{pgfscope}%
\begin{pgfscope}%
\pgfpathrectangle{\pgfqpoint{0.481978in}{0.331635in}}{\pgfqpoint{9.300000in}{7.700000in}}%
\pgfusepath{clip}%
\pgfsetrectcap%
\pgfsetroundjoin%
\pgfsetlinewidth{1.505625pt}%
\definecolor{currentstroke}{rgb}{0.631373,0.788235,0.956863}%
\pgfsetstrokecolor{currentstroke}%
\pgfsetstrokeopacity{0.800000}%
\pgfsetdash{}{0pt}%
\pgfpathmoveto{\pgfqpoint{5.837904in}{2.580252in}}%
\pgfpathlineto{\pgfqpoint{5.711094in}{3.773662in}}%
\pgfusepath{stroke}%
\end{pgfscope}%
\begin{pgfscope}%
\pgfpathrectangle{\pgfqpoint{0.481978in}{0.331635in}}{\pgfqpoint{9.300000in}{7.700000in}}%
\pgfusepath{clip}%
\pgfsetrectcap%
\pgfsetroundjoin%
\pgfsetlinewidth{1.505625pt}%
\definecolor{currentstroke}{rgb}{0.631373,0.788235,0.956863}%
\pgfsetstrokecolor{currentstroke}%
\pgfsetstrokeopacity{0.800000}%
\pgfsetdash{}{0pt}%
\pgfpathmoveto{\pgfqpoint{7.682296in}{2.311061in}}%
\pgfpathlineto{\pgfqpoint{5.711094in}{3.773662in}}%
\pgfusepath{stroke}%
\end{pgfscope}%
\begin{pgfscope}%
\pgfpathrectangle{\pgfqpoint{0.481978in}{0.331635in}}{\pgfqpoint{9.300000in}{7.700000in}}%
\pgfusepath{clip}%
\pgfsetrectcap%
\pgfsetroundjoin%
\pgfsetlinewidth{1.505625pt}%
\definecolor{currentstroke}{rgb}{0.631373,0.788235,0.956863}%
\pgfsetstrokecolor{currentstroke}%
\pgfsetstrokeopacity{0.800000}%
\pgfsetdash{}{0pt}%
\pgfpathmoveto{\pgfqpoint{4.017749in}{1.899805in}}%
\pgfpathlineto{\pgfqpoint{5.711094in}{3.773662in}}%
\pgfusepath{stroke}%
\end{pgfscope}%
\begin{pgfscope}%
\pgfpathrectangle{\pgfqpoint{0.481978in}{0.331635in}}{\pgfqpoint{9.300000in}{7.700000in}}%
\pgfusepath{clip}%
\pgfsetrectcap%
\pgfsetroundjoin%
\pgfsetlinewidth{1.505625pt}%
\definecolor{currentstroke}{rgb}{0.631373,0.788235,0.956863}%
\pgfsetstrokecolor{currentstroke}%
\pgfsetstrokeopacity{0.800000}%
\pgfsetdash{}{0pt}%
\pgfpathmoveto{\pgfqpoint{6.127545in}{4.884465in}}%
\pgfpathlineto{\pgfqpoint{5.711094in}{3.773662in}}%
\pgfusepath{stroke}%
\end{pgfscope}%
\begin{pgfscope}%
\pgfpathrectangle{\pgfqpoint{0.481978in}{0.331635in}}{\pgfqpoint{9.300000in}{7.700000in}}%
\pgfusepath{clip}%
\pgfsetrectcap%
\pgfsetroundjoin%
\pgfsetlinewidth{1.505625pt}%
\definecolor{currentstroke}{rgb}{0.631373,0.788235,0.956863}%
\pgfsetstrokecolor{currentstroke}%
\pgfsetstrokeopacity{0.800000}%
\pgfsetdash{}{0pt}%
\pgfpathmoveto{\pgfqpoint{7.971575in}{5.289470in}}%
\pgfpathlineto{\pgfqpoint{5.711094in}{3.773662in}}%
\pgfusepath{stroke}%
\end{pgfscope}%
\begin{pgfscope}%
\pgfpathrectangle{\pgfqpoint{0.481978in}{0.331635in}}{\pgfqpoint{9.300000in}{7.700000in}}%
\pgfusepath{clip}%
\pgfsetrectcap%
\pgfsetroundjoin%
\pgfsetlinewidth{1.505625pt}%
\definecolor{currentstroke}{rgb}{0.631373,0.788235,0.956863}%
\pgfsetstrokecolor{currentstroke}%
\pgfsetstrokeopacity{0.800000}%
\pgfsetdash{}{0pt}%
\pgfpathmoveto{\pgfqpoint{5.508663in}{2.419584in}}%
\pgfpathlineto{\pgfqpoint{5.711094in}{3.773662in}}%
\pgfusepath{stroke}%
\end{pgfscope}%
\begin{pgfscope}%
\pgfpathrectangle{\pgfqpoint{0.481978in}{0.331635in}}{\pgfqpoint{9.300000in}{7.700000in}}%
\pgfusepath{clip}%
\pgfsetrectcap%
\pgfsetroundjoin%
\pgfsetlinewidth{1.505625pt}%
\definecolor{currentstroke}{rgb}{0.631373,0.788235,0.956863}%
\pgfsetstrokecolor{currentstroke}%
\pgfsetstrokeopacity{0.800000}%
\pgfsetdash{}{0pt}%
\pgfpathmoveto{\pgfqpoint{4.730932in}{7.181326in}}%
\pgfpathlineto{\pgfqpoint{5.711094in}{3.773662in}}%
\pgfusepath{stroke}%
\end{pgfscope}%
\begin{pgfscope}%
\pgfpathrectangle{\pgfqpoint{0.481978in}{0.331635in}}{\pgfqpoint{9.300000in}{7.700000in}}%
\pgfusepath{clip}%
\pgfsetrectcap%
\pgfsetroundjoin%
\pgfsetlinewidth{1.505625pt}%
\definecolor{currentstroke}{rgb}{0.631373,0.788235,0.956863}%
\pgfsetstrokecolor{currentstroke}%
\pgfsetstrokeopacity{0.800000}%
\pgfsetdash{}{0pt}%
\pgfpathmoveto{\pgfqpoint{5.337518in}{5.951535in}}%
\pgfpathlineto{\pgfqpoint{5.711094in}{3.773662in}}%
\pgfusepath{stroke}%
\end{pgfscope}%
\begin{pgfscope}%
\pgfpathrectangle{\pgfqpoint{0.481978in}{0.331635in}}{\pgfqpoint{9.300000in}{7.700000in}}%
\pgfusepath{clip}%
\pgfsetrectcap%
\pgfsetroundjoin%
\pgfsetlinewidth{1.505625pt}%
\definecolor{currentstroke}{rgb}{0.631373,0.788235,0.956863}%
\pgfsetstrokecolor{currentstroke}%
\pgfsetstrokeopacity{0.800000}%
\pgfsetdash{}{0pt}%
\pgfpathmoveto{\pgfqpoint{5.165140in}{3.130223in}}%
\pgfpathlineto{\pgfqpoint{5.711094in}{3.773662in}}%
\pgfusepath{stroke}%
\end{pgfscope}%
\begin{pgfscope}%
\pgfpathrectangle{\pgfqpoint{0.481978in}{0.331635in}}{\pgfqpoint{9.300000in}{7.700000in}}%
\pgfusepath{clip}%
\pgfsetrectcap%
\pgfsetroundjoin%
\pgfsetlinewidth{1.505625pt}%
\definecolor{currentstroke}{rgb}{0.631373,0.788235,0.956863}%
\pgfsetstrokecolor{currentstroke}%
\pgfsetstrokeopacity{0.800000}%
\pgfsetdash{}{0pt}%
\pgfpathmoveto{\pgfqpoint{6.635187in}{2.624141in}}%
\pgfpathlineto{\pgfqpoint{5.711094in}{3.773662in}}%
\pgfusepath{stroke}%
\end{pgfscope}%
\begin{pgfscope}%
\pgfpathrectangle{\pgfqpoint{0.481978in}{0.331635in}}{\pgfqpoint{9.300000in}{7.700000in}}%
\pgfusepath{clip}%
\pgfsetrectcap%
\pgfsetroundjoin%
\pgfsetlinewidth{1.505625pt}%
\definecolor{currentstroke}{rgb}{0.631373,0.788235,0.956863}%
\pgfsetstrokecolor{currentstroke}%
\pgfsetstrokeopacity{0.800000}%
\pgfsetdash{}{0pt}%
\pgfpathmoveto{\pgfqpoint{2.269929in}{6.475456in}}%
\pgfpathlineto{\pgfqpoint{5.711094in}{3.773662in}}%
\pgfusepath{stroke}%
\end{pgfscope}%
\begin{pgfscope}%
\pgfpathrectangle{\pgfqpoint{0.481978in}{0.331635in}}{\pgfqpoint{9.300000in}{7.700000in}}%
\pgfusepath{clip}%
\pgfsetrectcap%
\pgfsetroundjoin%
\pgfsetlinewidth{1.505625pt}%
\definecolor{currentstroke}{rgb}{0.631373,0.788235,0.956863}%
\pgfsetstrokecolor{currentstroke}%
\pgfsetstrokeopacity{0.800000}%
\pgfsetdash{}{0pt}%
\pgfpathmoveto{\pgfqpoint{6.041629in}{4.151018in}}%
\pgfpathlineto{\pgfqpoint{5.711094in}{3.773662in}}%
\pgfusepath{stroke}%
\end{pgfscope}%
\begin{pgfscope}%
\pgfpathrectangle{\pgfqpoint{0.481978in}{0.331635in}}{\pgfqpoint{9.300000in}{7.700000in}}%
\pgfusepath{clip}%
\pgfsetrectcap%
\pgfsetroundjoin%
\pgfsetlinewidth{1.505625pt}%
\definecolor{currentstroke}{rgb}{0.631373,0.788235,0.956863}%
\pgfsetstrokecolor{currentstroke}%
\pgfsetstrokeopacity{0.800000}%
\pgfsetdash{}{0pt}%
\pgfpathmoveto{\pgfqpoint{7.965678in}{4.024809in}}%
\pgfpathlineto{\pgfqpoint{5.711094in}{3.773662in}}%
\pgfusepath{stroke}%
\end{pgfscope}%
\begin{pgfscope}%
\pgfpathrectangle{\pgfqpoint{0.481978in}{0.331635in}}{\pgfqpoint{9.300000in}{7.700000in}}%
\pgfusepath{clip}%
\pgfsetrectcap%
\pgfsetroundjoin%
\pgfsetlinewidth{1.505625pt}%
\definecolor{currentstroke}{rgb}{0.631373,0.788235,0.956863}%
\pgfsetstrokecolor{currentstroke}%
\pgfsetstrokeopacity{0.800000}%
\pgfsetdash{}{0pt}%
\pgfpathmoveto{\pgfqpoint{6.739115in}{1.489329in}}%
\pgfpathlineto{\pgfqpoint{5.711094in}{3.773662in}}%
\pgfusepath{stroke}%
\end{pgfscope}%
\begin{pgfscope}%
\pgfpathrectangle{\pgfqpoint{0.481978in}{0.331635in}}{\pgfqpoint{9.300000in}{7.700000in}}%
\pgfusepath{clip}%
\pgfsetrectcap%
\pgfsetroundjoin%
\pgfsetlinewidth{1.505625pt}%
\definecolor{currentstroke}{rgb}{0.631373,0.788235,0.956863}%
\pgfsetstrokecolor{currentstroke}%
\pgfsetstrokeopacity{0.800000}%
\pgfsetdash{}{0pt}%
\pgfpathmoveto{\pgfqpoint{5.908920in}{1.565271in}}%
\pgfpathlineto{\pgfqpoint{5.711094in}{3.773662in}}%
\pgfusepath{stroke}%
\end{pgfscope}%
\begin{pgfscope}%
\pgfpathrectangle{\pgfqpoint{0.481978in}{0.331635in}}{\pgfqpoint{9.300000in}{7.700000in}}%
\pgfusepath{clip}%
\pgfsetrectcap%
\pgfsetroundjoin%
\pgfsetlinewidth{1.505625pt}%
\definecolor{currentstroke}{rgb}{0.631373,0.788235,0.956863}%
\pgfsetstrokecolor{currentstroke}%
\pgfsetstrokeopacity{0.800000}%
\pgfsetdash{}{0pt}%
\pgfpathmoveto{\pgfqpoint{4.782057in}{1.328628in}}%
\pgfpathlineto{\pgfqpoint{5.711094in}{3.773662in}}%
\pgfusepath{stroke}%
\end{pgfscope}%
\begin{pgfscope}%
\pgfpathrectangle{\pgfqpoint{0.481978in}{0.331635in}}{\pgfqpoint{9.300000in}{7.700000in}}%
\pgfusepath{clip}%
\pgfsetrectcap%
\pgfsetroundjoin%
\pgfsetlinewidth{1.505625pt}%
\definecolor{currentstroke}{rgb}{0.631373,0.788235,0.956863}%
\pgfsetstrokecolor{currentstroke}%
\pgfsetstrokeopacity{0.800000}%
\pgfsetdash{}{0pt}%
\pgfpathmoveto{\pgfqpoint{2.762676in}{1.438822in}}%
\pgfpathlineto{\pgfqpoint{5.711094in}{3.773662in}}%
\pgfusepath{stroke}%
\end{pgfscope}%
\begin{pgfscope}%
\pgfpathrectangle{\pgfqpoint{0.481978in}{0.331635in}}{\pgfqpoint{9.300000in}{7.700000in}}%
\pgfusepath{clip}%
\pgfsetrectcap%
\pgfsetroundjoin%
\pgfsetlinewidth{1.505625pt}%
\definecolor{currentstroke}{rgb}{0.631373,0.788235,0.956863}%
\pgfsetstrokecolor{currentstroke}%
\pgfsetstrokeopacity{0.800000}%
\pgfsetdash{}{0pt}%
\pgfpathmoveto{\pgfqpoint{6.674444in}{1.875958in}}%
\pgfpathlineto{\pgfqpoint{5.711094in}{3.773662in}}%
\pgfusepath{stroke}%
\end{pgfscope}%
\begin{pgfscope}%
\pgfpathrectangle{\pgfqpoint{0.481978in}{0.331635in}}{\pgfqpoint{9.300000in}{7.700000in}}%
\pgfusepath{clip}%
\pgfsetrectcap%
\pgfsetroundjoin%
\pgfsetlinewidth{1.505625pt}%
\definecolor{currentstroke}{rgb}{0.631373,0.788235,0.956863}%
\pgfsetstrokecolor{currentstroke}%
\pgfsetstrokeopacity{0.800000}%
\pgfsetdash{}{0pt}%
\pgfpathmoveto{\pgfqpoint{7.190919in}{2.117314in}}%
\pgfpathlineto{\pgfqpoint{5.711094in}{3.773662in}}%
\pgfusepath{stroke}%
\end{pgfscope}%
\begin{pgfscope}%
\pgfpathrectangle{\pgfqpoint{0.481978in}{0.331635in}}{\pgfqpoint{9.300000in}{7.700000in}}%
\pgfusepath{clip}%
\pgfsetrectcap%
\pgfsetroundjoin%
\pgfsetlinewidth{1.505625pt}%
\definecolor{currentstroke}{rgb}{0.631373,0.788235,0.956863}%
\pgfsetstrokecolor{currentstroke}%
\pgfsetstrokeopacity{0.800000}%
\pgfsetdash{}{0pt}%
\pgfpathmoveto{\pgfqpoint{5.231440in}{6.892073in}}%
\pgfpathlineto{\pgfqpoint{5.711094in}{3.773662in}}%
\pgfusepath{stroke}%
\end{pgfscope}%
\begin{pgfscope}%
\pgfpathrectangle{\pgfqpoint{0.481978in}{0.331635in}}{\pgfqpoint{9.300000in}{7.700000in}}%
\pgfusepath{clip}%
\pgfsetrectcap%
\pgfsetroundjoin%
\pgfsetlinewidth{1.505625pt}%
\definecolor{currentstroke}{rgb}{0.631373,0.788235,0.956863}%
\pgfsetstrokecolor{currentstroke}%
\pgfsetstrokeopacity{0.800000}%
\pgfsetdash{}{0pt}%
\pgfpathmoveto{\pgfqpoint{7.641158in}{4.391700in}}%
\pgfpathlineto{\pgfqpoint{5.711094in}{3.773662in}}%
\pgfusepath{stroke}%
\end{pgfscope}%
\begin{pgfscope}%
\pgfpathrectangle{\pgfqpoint{0.481978in}{0.331635in}}{\pgfqpoint{9.300000in}{7.700000in}}%
\pgfusepath{clip}%
\pgfsetrectcap%
\pgfsetroundjoin%
\pgfsetlinewidth{1.505625pt}%
\definecolor{currentstroke}{rgb}{0.631373,0.788235,0.956863}%
\pgfsetstrokecolor{currentstroke}%
\pgfsetstrokeopacity{0.800000}%
\pgfsetdash{}{0pt}%
\pgfpathmoveto{\pgfqpoint{5.917655in}{4.377356in}}%
\pgfpathlineto{\pgfqpoint{5.711094in}{3.773662in}}%
\pgfusepath{stroke}%
\end{pgfscope}%
\begin{pgfscope}%
\pgfpathrectangle{\pgfqpoint{0.481978in}{0.331635in}}{\pgfqpoint{9.300000in}{7.700000in}}%
\pgfusepath{clip}%
\pgfsetrectcap%
\pgfsetroundjoin%
\pgfsetlinewidth{1.505625pt}%
\definecolor{currentstroke}{rgb}{0.631373,0.788235,0.956863}%
\pgfsetstrokecolor{currentstroke}%
\pgfsetstrokeopacity{0.800000}%
\pgfsetdash{}{0pt}%
\pgfpathmoveto{\pgfqpoint{6.915293in}{3.968555in}}%
\pgfpathlineto{\pgfqpoint{5.711094in}{3.773662in}}%
\pgfusepath{stroke}%
\end{pgfscope}%
\begin{pgfscope}%
\pgfpathrectangle{\pgfqpoint{0.481978in}{0.331635in}}{\pgfqpoint{9.300000in}{7.700000in}}%
\pgfusepath{clip}%
\pgfsetrectcap%
\pgfsetroundjoin%
\pgfsetlinewidth{1.505625pt}%
\definecolor{currentstroke}{rgb}{0.631373,0.788235,0.956863}%
\pgfsetstrokecolor{currentstroke}%
\pgfsetstrokeopacity{0.800000}%
\pgfsetdash{}{0pt}%
\pgfpathmoveto{\pgfqpoint{5.971393in}{2.386439in}}%
\pgfpathlineto{\pgfqpoint{5.711094in}{3.773662in}}%
\pgfusepath{stroke}%
\end{pgfscope}%
\begin{pgfscope}%
\pgfpathrectangle{\pgfqpoint{0.481978in}{0.331635in}}{\pgfqpoint{9.300000in}{7.700000in}}%
\pgfusepath{clip}%
\pgfsetrectcap%
\pgfsetroundjoin%
\pgfsetlinewidth{1.505625pt}%
\definecolor{currentstroke}{rgb}{0.631373,0.788235,0.956863}%
\pgfsetstrokecolor{currentstroke}%
\pgfsetstrokeopacity{0.800000}%
\pgfsetdash{}{0pt}%
\pgfpathmoveto{\pgfqpoint{2.928552in}{6.619267in}}%
\pgfpathlineto{\pgfqpoint{5.711094in}{3.773662in}}%
\pgfusepath{stroke}%
\end{pgfscope}%
\begin{pgfscope}%
\pgfpathrectangle{\pgfqpoint{0.481978in}{0.331635in}}{\pgfqpoint{9.300000in}{7.700000in}}%
\pgfusepath{clip}%
\pgfsetrectcap%
\pgfsetroundjoin%
\pgfsetlinewidth{1.505625pt}%
\definecolor{currentstroke}{rgb}{0.631373,0.788235,0.956863}%
\pgfsetstrokecolor{currentstroke}%
\pgfsetstrokeopacity{0.800000}%
\pgfsetdash{}{0pt}%
\pgfpathmoveto{\pgfqpoint{5.743990in}{2.614242in}}%
\pgfpathlineto{\pgfqpoint{5.711094in}{3.773662in}}%
\pgfusepath{stroke}%
\end{pgfscope}%
\begin{pgfscope}%
\pgfpathrectangle{\pgfqpoint{0.481978in}{0.331635in}}{\pgfqpoint{9.300000in}{7.700000in}}%
\pgfusepath{clip}%
\pgfsetrectcap%
\pgfsetroundjoin%
\pgfsetlinewidth{1.505625pt}%
\definecolor{currentstroke}{rgb}{0.631373,0.788235,0.956863}%
\pgfsetstrokecolor{currentstroke}%
\pgfsetstrokeopacity{0.800000}%
\pgfsetdash{}{0pt}%
\pgfpathmoveto{\pgfqpoint{8.326824in}{4.923794in}}%
\pgfpathlineto{\pgfqpoint{5.711094in}{3.773662in}}%
\pgfusepath{stroke}%
\end{pgfscope}%
\begin{pgfscope}%
\pgfpathrectangle{\pgfqpoint{0.481978in}{0.331635in}}{\pgfqpoint{9.300000in}{7.700000in}}%
\pgfusepath{clip}%
\pgfsetrectcap%
\pgfsetroundjoin%
\pgfsetlinewidth{1.505625pt}%
\definecolor{currentstroke}{rgb}{0.631373,0.788235,0.956863}%
\pgfsetstrokecolor{currentstroke}%
\pgfsetstrokeopacity{0.800000}%
\pgfsetdash{}{0pt}%
\pgfpathmoveto{\pgfqpoint{4.333517in}{5.783751in}}%
\pgfpathlineto{\pgfqpoint{5.711094in}{3.773662in}}%
\pgfusepath{stroke}%
\end{pgfscope}%
\begin{pgfscope}%
\pgfpathrectangle{\pgfqpoint{0.481978in}{0.331635in}}{\pgfqpoint{9.300000in}{7.700000in}}%
\pgfusepath{clip}%
\pgfsetrectcap%
\pgfsetroundjoin%
\pgfsetlinewidth{1.505625pt}%
\definecolor{currentstroke}{rgb}{0.631373,0.788235,0.956863}%
\pgfsetstrokecolor{currentstroke}%
\pgfsetstrokeopacity{0.800000}%
\pgfsetdash{}{0pt}%
\pgfpathmoveto{\pgfqpoint{5.146855in}{2.060560in}}%
\pgfpathlineto{\pgfqpoint{5.711094in}{3.773662in}}%
\pgfusepath{stroke}%
\end{pgfscope}%
\begin{pgfscope}%
\pgfpathrectangle{\pgfqpoint{0.481978in}{0.331635in}}{\pgfqpoint{9.300000in}{7.700000in}}%
\pgfusepath{clip}%
\pgfsetrectcap%
\pgfsetroundjoin%
\pgfsetlinewidth{1.505625pt}%
\definecolor{currentstroke}{rgb}{0.631373,0.788235,0.956863}%
\pgfsetstrokecolor{currentstroke}%
\pgfsetstrokeopacity{0.800000}%
\pgfsetdash{}{0pt}%
\pgfpathmoveto{\pgfqpoint{6.726675in}{5.410343in}}%
\pgfpathlineto{\pgfqpoint{5.711094in}{3.773662in}}%
\pgfusepath{stroke}%
\end{pgfscope}%
\begin{pgfscope}%
\pgfpathrectangle{\pgfqpoint{0.481978in}{0.331635in}}{\pgfqpoint{9.300000in}{7.700000in}}%
\pgfusepath{clip}%
\pgfsetrectcap%
\pgfsetroundjoin%
\pgfsetlinewidth{1.505625pt}%
\definecolor{currentstroke}{rgb}{0.631373,0.788235,0.956863}%
\pgfsetstrokecolor{currentstroke}%
\pgfsetstrokeopacity{0.800000}%
\pgfsetdash{}{0pt}%
\pgfpathmoveto{\pgfqpoint{5.444209in}{2.575038in}}%
\pgfpathlineto{\pgfqpoint{5.711094in}{3.773662in}}%
\pgfusepath{stroke}%
\end{pgfscope}%
\begin{pgfscope}%
\pgfpathrectangle{\pgfqpoint{0.481978in}{0.331635in}}{\pgfqpoint{9.300000in}{7.700000in}}%
\pgfusepath{clip}%
\pgfsetrectcap%
\pgfsetroundjoin%
\pgfsetlinewidth{1.505625pt}%
\definecolor{currentstroke}{rgb}{0.631373,0.788235,0.956863}%
\pgfsetstrokecolor{currentstroke}%
\pgfsetstrokeopacity{0.800000}%
\pgfsetdash{}{0pt}%
\pgfpathmoveto{\pgfqpoint{6.317681in}{2.783974in}}%
\pgfpathlineto{\pgfqpoint{5.711094in}{3.773662in}}%
\pgfusepath{stroke}%
\end{pgfscope}%
\begin{pgfscope}%
\pgfpathrectangle{\pgfqpoint{0.481978in}{0.331635in}}{\pgfqpoint{9.300000in}{7.700000in}}%
\pgfusepath{clip}%
\pgfsetrectcap%
\pgfsetroundjoin%
\pgfsetlinewidth{1.505625pt}%
\definecolor{currentstroke}{rgb}{0.631373,0.788235,0.956863}%
\pgfsetstrokecolor{currentstroke}%
\pgfsetstrokeopacity{0.800000}%
\pgfsetdash{}{0pt}%
\pgfpathmoveto{\pgfqpoint{5.001156in}{3.894549in}}%
\pgfpathlineto{\pgfqpoint{5.711094in}{3.773662in}}%
\pgfusepath{stroke}%
\end{pgfscope}%
\begin{pgfscope}%
\pgfpathrectangle{\pgfqpoint{0.481978in}{0.331635in}}{\pgfqpoint{9.300000in}{7.700000in}}%
\pgfusepath{clip}%
\pgfsetrectcap%
\pgfsetroundjoin%
\pgfsetlinewidth{1.505625pt}%
\definecolor{currentstroke}{rgb}{0.631373,0.788235,0.956863}%
\pgfsetstrokecolor{currentstroke}%
\pgfsetstrokeopacity{0.800000}%
\pgfsetdash{}{0pt}%
\pgfpathmoveto{\pgfqpoint{5.223174in}{3.276248in}}%
\pgfpathlineto{\pgfqpoint{5.711094in}{3.773662in}}%
\pgfusepath{stroke}%
\end{pgfscope}%
\begin{pgfscope}%
\pgfpathrectangle{\pgfqpoint{0.481978in}{0.331635in}}{\pgfqpoint{9.300000in}{7.700000in}}%
\pgfusepath{clip}%
\pgfsetrectcap%
\pgfsetroundjoin%
\pgfsetlinewidth{1.505625pt}%
\definecolor{currentstroke}{rgb}{0.631373,0.788235,0.956863}%
\pgfsetstrokecolor{currentstroke}%
\pgfsetstrokeopacity{0.800000}%
\pgfsetdash{}{0pt}%
\pgfpathmoveto{\pgfqpoint{5.990771in}{2.551896in}}%
\pgfpathlineto{\pgfqpoint{5.711094in}{3.773662in}}%
\pgfusepath{stroke}%
\end{pgfscope}%
\begin{pgfscope}%
\pgfpathrectangle{\pgfqpoint{0.481978in}{0.331635in}}{\pgfqpoint{9.300000in}{7.700000in}}%
\pgfusepath{clip}%
\pgfsetrectcap%
\pgfsetroundjoin%
\pgfsetlinewidth{1.505625pt}%
\definecolor{currentstroke}{rgb}{0.631373,0.788235,0.956863}%
\pgfsetstrokecolor{currentstroke}%
\pgfsetstrokeopacity{0.800000}%
\pgfsetdash{}{0pt}%
\pgfpathmoveto{\pgfqpoint{5.037316in}{4.329956in}}%
\pgfpathlineto{\pgfqpoint{5.711094in}{3.773662in}}%
\pgfusepath{stroke}%
\end{pgfscope}%
\begin{pgfscope}%
\pgfpathrectangle{\pgfqpoint{0.481978in}{0.331635in}}{\pgfqpoint{9.300000in}{7.700000in}}%
\pgfusepath{clip}%
\pgfsetrectcap%
\pgfsetroundjoin%
\pgfsetlinewidth{1.505625pt}%
\definecolor{currentstroke}{rgb}{0.631373,0.788235,0.956863}%
\pgfsetstrokecolor{currentstroke}%
\pgfsetstrokeopacity{0.800000}%
\pgfsetdash{}{0pt}%
\pgfpathmoveto{\pgfqpoint{4.961622in}{0.885246in}}%
\pgfpathlineto{\pgfqpoint{5.711094in}{3.773662in}}%
\pgfusepath{stroke}%
\end{pgfscope}%
\begin{pgfscope}%
\pgfpathrectangle{\pgfqpoint{0.481978in}{0.331635in}}{\pgfqpoint{9.300000in}{7.700000in}}%
\pgfusepath{clip}%
\pgfsetrectcap%
\pgfsetroundjoin%
\pgfsetlinewidth{1.505625pt}%
\definecolor{currentstroke}{rgb}{0.631373,0.788235,0.956863}%
\pgfsetstrokecolor{currentstroke}%
\pgfsetstrokeopacity{0.800000}%
\pgfsetdash{}{0pt}%
\pgfpathmoveto{\pgfqpoint{5.509130in}{2.091290in}}%
\pgfpathlineto{\pgfqpoint{5.711094in}{3.773662in}}%
\pgfusepath{stroke}%
\end{pgfscope}%
\begin{pgfscope}%
\pgfpathrectangle{\pgfqpoint{0.481978in}{0.331635in}}{\pgfqpoint{9.300000in}{7.700000in}}%
\pgfusepath{clip}%
\pgfsetrectcap%
\pgfsetroundjoin%
\pgfsetlinewidth{1.505625pt}%
\definecolor{currentstroke}{rgb}{0.631373,0.788235,0.956863}%
\pgfsetstrokecolor{currentstroke}%
\pgfsetstrokeopacity{0.800000}%
\pgfsetdash{}{0pt}%
\pgfpathmoveto{\pgfqpoint{2.965360in}{6.422179in}}%
\pgfpathlineto{\pgfqpoint{5.711094in}{3.773662in}}%
\pgfusepath{stroke}%
\end{pgfscope}%
\begin{pgfscope}%
\pgfpathrectangle{\pgfqpoint{0.481978in}{0.331635in}}{\pgfqpoint{9.300000in}{7.700000in}}%
\pgfusepath{clip}%
\pgfsetrectcap%
\pgfsetroundjoin%
\pgfsetlinewidth{1.505625pt}%
\definecolor{currentstroke}{rgb}{0.631373,0.788235,0.956863}%
\pgfsetstrokecolor{currentstroke}%
\pgfsetstrokeopacity{0.800000}%
\pgfsetdash{}{0pt}%
\pgfpathmoveto{\pgfqpoint{6.394060in}{2.566866in}}%
\pgfpathlineto{\pgfqpoint{5.711094in}{3.773662in}}%
\pgfusepath{stroke}%
\end{pgfscope}%
\begin{pgfscope}%
\pgfpathrectangle{\pgfqpoint{0.481978in}{0.331635in}}{\pgfqpoint{9.300000in}{7.700000in}}%
\pgfusepath{clip}%
\pgfsetrectcap%
\pgfsetroundjoin%
\pgfsetlinewidth{1.505625pt}%
\definecolor{currentstroke}{rgb}{0.631373,0.788235,0.956863}%
\pgfsetstrokecolor{currentstroke}%
\pgfsetstrokeopacity{0.800000}%
\pgfsetdash{}{0pt}%
\pgfpathmoveto{\pgfqpoint{2.186468in}{6.813277in}}%
\pgfpathlineto{\pgfqpoint{5.711094in}{3.773662in}}%
\pgfusepath{stroke}%
\end{pgfscope}%
\begin{pgfscope}%
\pgfpathrectangle{\pgfqpoint{0.481978in}{0.331635in}}{\pgfqpoint{9.300000in}{7.700000in}}%
\pgfusepath{clip}%
\pgfsetrectcap%
\pgfsetroundjoin%
\pgfsetlinewidth{1.505625pt}%
\definecolor{currentstroke}{rgb}{0.631373,0.788235,0.956863}%
\pgfsetstrokecolor{currentstroke}%
\pgfsetstrokeopacity{0.800000}%
\pgfsetdash{}{0pt}%
\pgfpathmoveto{\pgfqpoint{7.121234in}{2.917579in}}%
\pgfpathlineto{\pgfqpoint{5.711094in}{3.773662in}}%
\pgfusepath{stroke}%
\end{pgfscope}%
\begin{pgfscope}%
\pgfpathrectangle{\pgfqpoint{0.481978in}{0.331635in}}{\pgfqpoint{9.300000in}{7.700000in}}%
\pgfusepath{clip}%
\pgfsetrectcap%
\pgfsetroundjoin%
\pgfsetlinewidth{1.505625pt}%
\definecolor{currentstroke}{rgb}{0.631373,0.788235,0.956863}%
\pgfsetstrokecolor{currentstroke}%
\pgfsetstrokeopacity{0.800000}%
\pgfsetdash{}{0pt}%
\pgfpathmoveto{\pgfqpoint{8.167603in}{5.004497in}}%
\pgfpathlineto{\pgfqpoint{5.711094in}{3.773662in}}%
\pgfusepath{stroke}%
\end{pgfscope}%
\begin{pgfscope}%
\pgfpathrectangle{\pgfqpoint{0.481978in}{0.331635in}}{\pgfqpoint{9.300000in}{7.700000in}}%
\pgfusepath{clip}%
\pgfsetrectcap%
\pgfsetroundjoin%
\pgfsetlinewidth{1.505625pt}%
\definecolor{currentstroke}{rgb}{0.631373,0.788235,0.956863}%
\pgfsetstrokecolor{currentstroke}%
\pgfsetstrokeopacity{0.800000}%
\pgfsetdash{}{0pt}%
\pgfpathmoveto{\pgfqpoint{6.293261in}{4.446775in}}%
\pgfpathlineto{\pgfqpoint{5.711094in}{3.773662in}}%
\pgfusepath{stroke}%
\end{pgfscope}%
\begin{pgfscope}%
\pgfpathrectangle{\pgfqpoint{0.481978in}{0.331635in}}{\pgfqpoint{9.300000in}{7.700000in}}%
\pgfusepath{clip}%
\pgfsetrectcap%
\pgfsetroundjoin%
\pgfsetlinewidth{1.505625pt}%
\definecolor{currentstroke}{rgb}{0.631373,0.788235,0.956863}%
\pgfsetstrokecolor{currentstroke}%
\pgfsetstrokeopacity{0.800000}%
\pgfsetdash{}{0pt}%
\pgfpathmoveto{\pgfqpoint{4.855717in}{7.204103in}}%
\pgfpathlineto{\pgfqpoint{5.711094in}{3.773662in}}%
\pgfusepath{stroke}%
\end{pgfscope}%
\begin{pgfscope}%
\pgfpathrectangle{\pgfqpoint{0.481978in}{0.331635in}}{\pgfqpoint{9.300000in}{7.700000in}}%
\pgfusepath{clip}%
\pgfsetrectcap%
\pgfsetroundjoin%
\pgfsetlinewidth{1.505625pt}%
\definecolor{currentstroke}{rgb}{0.631373,0.788235,0.956863}%
\pgfsetstrokecolor{currentstroke}%
\pgfsetstrokeopacity{0.800000}%
\pgfsetdash{}{0pt}%
\pgfpathmoveto{\pgfqpoint{8.228148in}{4.830010in}}%
\pgfpathlineto{\pgfqpoint{5.711094in}{3.773662in}}%
\pgfusepath{stroke}%
\end{pgfscope}%
\begin{pgfscope}%
\pgfpathrectangle{\pgfqpoint{0.481978in}{0.331635in}}{\pgfqpoint{9.300000in}{7.700000in}}%
\pgfusepath{clip}%
\pgfsetrectcap%
\pgfsetroundjoin%
\pgfsetlinewidth{1.505625pt}%
\definecolor{currentstroke}{rgb}{0.631373,0.788235,0.956863}%
\pgfsetstrokecolor{currentstroke}%
\pgfsetstrokeopacity{0.800000}%
\pgfsetdash{}{0pt}%
\pgfpathmoveto{\pgfqpoint{7.375231in}{1.952235in}}%
\pgfpathlineto{\pgfqpoint{5.711094in}{3.773662in}}%
\pgfusepath{stroke}%
\end{pgfscope}%
\begin{pgfscope}%
\pgfpathrectangle{\pgfqpoint{0.481978in}{0.331635in}}{\pgfqpoint{9.300000in}{7.700000in}}%
\pgfusepath{clip}%
\pgfsetrectcap%
\pgfsetroundjoin%
\pgfsetlinewidth{1.505625pt}%
\definecolor{currentstroke}{rgb}{0.631373,0.788235,0.956863}%
\pgfsetstrokecolor{currentstroke}%
\pgfsetstrokeopacity{0.800000}%
\pgfsetdash{}{0pt}%
\pgfpathmoveto{\pgfqpoint{6.949155in}{1.649974in}}%
\pgfpathlineto{\pgfqpoint{5.711094in}{3.773662in}}%
\pgfusepath{stroke}%
\end{pgfscope}%
\begin{pgfscope}%
\pgfpathrectangle{\pgfqpoint{0.481978in}{0.331635in}}{\pgfqpoint{9.300000in}{7.700000in}}%
\pgfusepath{clip}%
\pgfsetrectcap%
\pgfsetroundjoin%
\pgfsetlinewidth{1.505625pt}%
\definecolor{currentstroke}{rgb}{0.631373,0.788235,0.956863}%
\pgfsetstrokecolor{currentstroke}%
\pgfsetstrokeopacity{0.800000}%
\pgfsetdash{}{0pt}%
\pgfpathmoveto{\pgfqpoint{7.251466in}{2.520062in}}%
\pgfpathlineto{\pgfqpoint{5.711094in}{3.773662in}}%
\pgfusepath{stroke}%
\end{pgfscope}%
\begin{pgfscope}%
\pgfpathrectangle{\pgfqpoint{0.481978in}{0.331635in}}{\pgfqpoint{9.300000in}{7.700000in}}%
\pgfusepath{clip}%
\pgfsetrectcap%
\pgfsetroundjoin%
\pgfsetlinewidth{1.505625pt}%
\definecolor{currentstroke}{rgb}{0.631373,0.788235,0.956863}%
\pgfsetstrokecolor{currentstroke}%
\pgfsetstrokeopacity{0.800000}%
\pgfsetdash{}{0pt}%
\pgfpathmoveto{\pgfqpoint{5.500453in}{2.173795in}}%
\pgfpathlineto{\pgfqpoint{5.711094in}{3.773662in}}%
\pgfusepath{stroke}%
\end{pgfscope}%
\begin{pgfscope}%
\pgfpathrectangle{\pgfqpoint{0.481978in}{0.331635in}}{\pgfqpoint{9.300000in}{7.700000in}}%
\pgfusepath{clip}%
\pgfsetrectcap%
\pgfsetroundjoin%
\pgfsetlinewidth{1.505625pt}%
\definecolor{currentstroke}{rgb}{0.631373,0.788235,0.956863}%
\pgfsetstrokecolor{currentstroke}%
\pgfsetstrokeopacity{0.800000}%
\pgfsetdash{}{0pt}%
\pgfpathmoveto{\pgfqpoint{6.149200in}{2.319484in}}%
\pgfpathlineto{\pgfqpoint{5.711094in}{3.773662in}}%
\pgfusepath{stroke}%
\end{pgfscope}%
\begin{pgfscope}%
\pgfpathrectangle{\pgfqpoint{0.481978in}{0.331635in}}{\pgfqpoint{9.300000in}{7.700000in}}%
\pgfusepath{clip}%
\pgfsetrectcap%
\pgfsetroundjoin%
\pgfsetlinewidth{1.505625pt}%
\definecolor{currentstroke}{rgb}{0.631373,0.788235,0.956863}%
\pgfsetstrokecolor{currentstroke}%
\pgfsetstrokeopacity{0.800000}%
\pgfsetdash{}{0pt}%
\pgfpathmoveto{\pgfqpoint{2.951748in}{1.154804in}}%
\pgfpathlineto{\pgfqpoint{5.711094in}{3.773662in}}%
\pgfusepath{stroke}%
\end{pgfscope}%
\begin{pgfscope}%
\pgfpathrectangle{\pgfqpoint{0.481978in}{0.331635in}}{\pgfqpoint{9.300000in}{7.700000in}}%
\pgfusepath{clip}%
\pgfsetrectcap%
\pgfsetroundjoin%
\pgfsetlinewidth{1.505625pt}%
\definecolor{currentstroke}{rgb}{0.631373,0.788235,0.956863}%
\pgfsetstrokecolor{currentstroke}%
\pgfsetstrokeopacity{0.800000}%
\pgfsetdash{}{0pt}%
\pgfpathmoveto{\pgfqpoint{6.416375in}{3.235325in}}%
\pgfpathlineto{\pgfqpoint{5.711094in}{3.773662in}}%
\pgfusepath{stroke}%
\end{pgfscope}%
\begin{pgfscope}%
\pgfpathrectangle{\pgfqpoint{0.481978in}{0.331635in}}{\pgfqpoint{9.300000in}{7.700000in}}%
\pgfusepath{clip}%
\pgfsetrectcap%
\pgfsetroundjoin%
\pgfsetlinewidth{1.505625pt}%
\definecolor{currentstroke}{rgb}{0.631373,0.788235,0.956863}%
\pgfsetstrokecolor{currentstroke}%
\pgfsetstrokeopacity{0.800000}%
\pgfsetdash{}{0pt}%
\pgfpathmoveto{\pgfqpoint{2.918703in}{1.265777in}}%
\pgfpathlineto{\pgfqpoint{5.711094in}{3.773662in}}%
\pgfusepath{stroke}%
\end{pgfscope}%
\begin{pgfscope}%
\pgfpathrectangle{\pgfqpoint{0.481978in}{0.331635in}}{\pgfqpoint{9.300000in}{7.700000in}}%
\pgfusepath{clip}%
\pgfsetrectcap%
\pgfsetroundjoin%
\pgfsetlinewidth{1.505625pt}%
\definecolor{currentstroke}{rgb}{0.631373,0.788235,0.956863}%
\pgfsetstrokecolor{currentstroke}%
\pgfsetstrokeopacity{0.800000}%
\pgfsetdash{}{0pt}%
\pgfpathmoveto{\pgfqpoint{2.176114in}{5.871183in}}%
\pgfpathlineto{\pgfqpoint{5.711094in}{3.773662in}}%
\pgfusepath{stroke}%
\end{pgfscope}%
\begin{pgfscope}%
\pgfpathrectangle{\pgfqpoint{0.481978in}{0.331635in}}{\pgfqpoint{9.300000in}{7.700000in}}%
\pgfusepath{clip}%
\pgfsetrectcap%
\pgfsetroundjoin%
\pgfsetlinewidth{1.505625pt}%
\definecolor{currentstroke}{rgb}{0.631373,0.788235,0.956863}%
\pgfsetstrokecolor{currentstroke}%
\pgfsetstrokeopacity{0.800000}%
\pgfsetdash{}{0pt}%
\pgfpathmoveto{\pgfqpoint{6.830655in}{1.576246in}}%
\pgfpathlineto{\pgfqpoint{5.711094in}{3.773662in}}%
\pgfusepath{stroke}%
\end{pgfscope}%
\begin{pgfscope}%
\pgfpathrectangle{\pgfqpoint{0.481978in}{0.331635in}}{\pgfqpoint{9.300000in}{7.700000in}}%
\pgfusepath{clip}%
\pgfsetrectcap%
\pgfsetroundjoin%
\pgfsetlinewidth{1.505625pt}%
\definecolor{currentstroke}{rgb}{0.631373,0.788235,0.956863}%
\pgfsetstrokecolor{currentstroke}%
\pgfsetstrokeopacity{0.800000}%
\pgfsetdash{}{0pt}%
\pgfpathmoveto{\pgfqpoint{6.368029in}{1.379909in}}%
\pgfpathlineto{\pgfqpoint{5.711094in}{3.773662in}}%
\pgfusepath{stroke}%
\end{pgfscope}%
\begin{pgfscope}%
\pgfpathrectangle{\pgfqpoint{0.481978in}{0.331635in}}{\pgfqpoint{9.300000in}{7.700000in}}%
\pgfusepath{clip}%
\pgfsetrectcap%
\pgfsetroundjoin%
\pgfsetlinewidth{1.505625pt}%
\definecolor{currentstroke}{rgb}{0.631373,0.788235,0.956863}%
\pgfsetstrokecolor{currentstroke}%
\pgfsetstrokeopacity{0.800000}%
\pgfsetdash{}{0pt}%
\pgfpathmoveto{\pgfqpoint{7.095890in}{2.699343in}}%
\pgfpathlineto{\pgfqpoint{5.711094in}{3.773662in}}%
\pgfusepath{stroke}%
\end{pgfscope}%
\begin{pgfscope}%
\pgfpathrectangle{\pgfqpoint{0.481978in}{0.331635in}}{\pgfqpoint{9.300000in}{7.700000in}}%
\pgfusepath{clip}%
\pgfsetrectcap%
\pgfsetroundjoin%
\pgfsetlinewidth{1.505625pt}%
\definecolor{currentstroke}{rgb}{0.631373,0.788235,0.956863}%
\pgfsetstrokecolor{currentstroke}%
\pgfsetstrokeopacity{0.800000}%
\pgfsetdash{}{0pt}%
\pgfpathmoveto{\pgfqpoint{5.900690in}{3.673565in}}%
\pgfpathlineto{\pgfqpoint{5.711094in}{3.773662in}}%
\pgfusepath{stroke}%
\end{pgfscope}%
\begin{pgfscope}%
\pgfpathrectangle{\pgfqpoint{0.481978in}{0.331635in}}{\pgfqpoint{9.300000in}{7.700000in}}%
\pgfusepath{clip}%
\pgfsetrectcap%
\pgfsetroundjoin%
\pgfsetlinewidth{1.505625pt}%
\definecolor{currentstroke}{rgb}{0.631373,0.788235,0.956863}%
\pgfsetstrokecolor{currentstroke}%
\pgfsetstrokeopacity{0.800000}%
\pgfsetdash{}{0pt}%
\pgfpathmoveto{\pgfqpoint{6.724845in}{4.610887in}}%
\pgfpathlineto{\pgfqpoint{5.711094in}{3.773662in}}%
\pgfusepath{stroke}%
\end{pgfscope}%
\begin{pgfscope}%
\pgfpathrectangle{\pgfqpoint{0.481978in}{0.331635in}}{\pgfqpoint{9.300000in}{7.700000in}}%
\pgfusepath{clip}%
\pgfsetrectcap%
\pgfsetroundjoin%
\pgfsetlinewidth{1.505625pt}%
\definecolor{currentstroke}{rgb}{0.631373,0.788235,0.956863}%
\pgfsetstrokecolor{currentstroke}%
\pgfsetstrokeopacity{0.800000}%
\pgfsetdash{}{0pt}%
\pgfpathmoveto{\pgfqpoint{8.573001in}{5.152498in}}%
\pgfpathlineto{\pgfqpoint{5.711094in}{3.773662in}}%
\pgfusepath{stroke}%
\end{pgfscope}%
\begin{pgfscope}%
\pgfpathrectangle{\pgfqpoint{0.481978in}{0.331635in}}{\pgfqpoint{9.300000in}{7.700000in}}%
\pgfusepath{clip}%
\pgfsetrectcap%
\pgfsetroundjoin%
\pgfsetlinewidth{1.505625pt}%
\definecolor{currentstroke}{rgb}{0.631373,0.788235,0.956863}%
\pgfsetstrokecolor{currentstroke}%
\pgfsetstrokeopacity{0.800000}%
\pgfsetdash{}{0pt}%
\pgfpathmoveto{\pgfqpoint{5.469587in}{7.033042in}}%
\pgfpathlineto{\pgfqpoint{5.711094in}{3.773662in}}%
\pgfusepath{stroke}%
\end{pgfscope}%
\begin{pgfscope}%
\pgfpathrectangle{\pgfqpoint{0.481978in}{0.331635in}}{\pgfqpoint{9.300000in}{7.700000in}}%
\pgfusepath{clip}%
\pgfsetrectcap%
\pgfsetroundjoin%
\pgfsetlinewidth{1.505625pt}%
\definecolor{currentstroke}{rgb}{0.631373,0.788235,0.956863}%
\pgfsetstrokecolor{currentstroke}%
\pgfsetstrokeopacity{0.800000}%
\pgfsetdash{}{0pt}%
\pgfpathmoveto{\pgfqpoint{4.990883in}{4.358746in}}%
\pgfpathlineto{\pgfqpoint{5.711094in}{3.773662in}}%
\pgfusepath{stroke}%
\end{pgfscope}%
\begin{pgfscope}%
\pgfpathrectangle{\pgfqpoint{0.481978in}{0.331635in}}{\pgfqpoint{9.300000in}{7.700000in}}%
\pgfusepath{clip}%
\pgfsetrectcap%
\pgfsetroundjoin%
\pgfsetlinewidth{1.505625pt}%
\definecolor{currentstroke}{rgb}{0.631373,0.788235,0.956863}%
\pgfsetstrokecolor{currentstroke}%
\pgfsetstrokeopacity{0.800000}%
\pgfsetdash{}{0pt}%
\pgfpathmoveto{\pgfqpoint{6.589980in}{1.687564in}}%
\pgfpathlineto{\pgfqpoint{5.711094in}{3.773662in}}%
\pgfusepath{stroke}%
\end{pgfscope}%
\begin{pgfscope}%
\pgfpathrectangle{\pgfqpoint{0.481978in}{0.331635in}}{\pgfqpoint{9.300000in}{7.700000in}}%
\pgfusepath{clip}%
\pgfsetrectcap%
\pgfsetroundjoin%
\pgfsetlinewidth{1.505625pt}%
\definecolor{currentstroke}{rgb}{0.631373,0.788235,0.956863}%
\pgfsetstrokecolor{currentstroke}%
\pgfsetstrokeopacity{0.800000}%
\pgfsetdash{}{0pt}%
\pgfpathmoveto{\pgfqpoint{5.765884in}{2.327113in}}%
\pgfpathlineto{\pgfqpoint{5.711094in}{3.773662in}}%
\pgfusepath{stroke}%
\end{pgfscope}%
\begin{pgfscope}%
\pgfpathrectangle{\pgfqpoint{0.481978in}{0.331635in}}{\pgfqpoint{9.300000in}{7.700000in}}%
\pgfusepath{clip}%
\pgfsetrectcap%
\pgfsetroundjoin%
\pgfsetlinewidth{1.505625pt}%
\definecolor{currentstroke}{rgb}{0.631373,0.788235,0.956863}%
\pgfsetstrokecolor{currentstroke}%
\pgfsetstrokeopacity{0.800000}%
\pgfsetdash{}{0pt}%
\pgfpathmoveto{\pgfqpoint{4.097733in}{6.308336in}}%
\pgfpathlineto{\pgfqpoint{5.711094in}{3.773662in}}%
\pgfusepath{stroke}%
\end{pgfscope}%
\begin{pgfscope}%
\pgfpathrectangle{\pgfqpoint{0.481978in}{0.331635in}}{\pgfqpoint{9.300000in}{7.700000in}}%
\pgfusepath{clip}%
\pgfsetrectcap%
\pgfsetroundjoin%
\pgfsetlinewidth{1.505625pt}%
\definecolor{currentstroke}{rgb}{0.631373,0.788235,0.956863}%
\pgfsetstrokecolor{currentstroke}%
\pgfsetstrokeopacity{0.800000}%
\pgfsetdash{}{0pt}%
\pgfpathmoveto{\pgfqpoint{7.842322in}{5.306029in}}%
\pgfpathlineto{\pgfqpoint{5.711094in}{3.773662in}}%
\pgfusepath{stroke}%
\end{pgfscope}%
\begin{pgfscope}%
\pgfpathrectangle{\pgfqpoint{0.481978in}{0.331635in}}{\pgfqpoint{9.300000in}{7.700000in}}%
\pgfusepath{clip}%
\pgfsetrectcap%
\pgfsetroundjoin%
\pgfsetlinewidth{1.505625pt}%
\definecolor{currentstroke}{rgb}{0.631373,0.788235,0.956863}%
\pgfsetstrokecolor{currentstroke}%
\pgfsetstrokeopacity{0.800000}%
\pgfsetdash{}{0pt}%
\pgfpathmoveto{\pgfqpoint{3.960820in}{4.623230in}}%
\pgfpathlineto{\pgfqpoint{5.711094in}{3.773662in}}%
\pgfusepath{stroke}%
\end{pgfscope}%
\begin{pgfscope}%
\pgfpathrectangle{\pgfqpoint{0.481978in}{0.331635in}}{\pgfqpoint{9.300000in}{7.700000in}}%
\pgfusepath{clip}%
\pgfsetrectcap%
\pgfsetroundjoin%
\pgfsetlinewidth{1.505625pt}%
\definecolor{currentstroke}{rgb}{0.631373,0.788235,0.956863}%
\pgfsetstrokecolor{currentstroke}%
\pgfsetstrokeopacity{0.800000}%
\pgfsetdash{}{0pt}%
\pgfpathmoveto{\pgfqpoint{3.426685in}{4.665956in}}%
\pgfpathlineto{\pgfqpoint{5.711094in}{3.773662in}}%
\pgfusepath{stroke}%
\end{pgfscope}%
\begin{pgfscope}%
\pgfpathrectangle{\pgfqpoint{0.481978in}{0.331635in}}{\pgfqpoint{9.300000in}{7.700000in}}%
\pgfusepath{clip}%
\pgfsetrectcap%
\pgfsetroundjoin%
\pgfsetlinewidth{1.505625pt}%
\definecolor{currentstroke}{rgb}{0.631373,0.788235,0.956863}%
\pgfsetstrokecolor{currentstroke}%
\pgfsetstrokeopacity{0.800000}%
\pgfsetdash{}{0pt}%
\pgfpathmoveto{\pgfqpoint{6.445366in}{2.091249in}}%
\pgfpathlineto{\pgfqpoint{5.711094in}{3.773662in}}%
\pgfusepath{stroke}%
\end{pgfscope}%
\begin{pgfscope}%
\pgfpathrectangle{\pgfqpoint{0.481978in}{0.331635in}}{\pgfqpoint{9.300000in}{7.700000in}}%
\pgfusepath{clip}%
\pgfsetrectcap%
\pgfsetroundjoin%
\pgfsetlinewidth{1.505625pt}%
\definecolor{currentstroke}{rgb}{0.631373,0.788235,0.956863}%
\pgfsetstrokecolor{currentstroke}%
\pgfsetstrokeopacity{0.800000}%
\pgfsetdash{}{0pt}%
\pgfpathmoveto{\pgfqpoint{8.216289in}{4.738244in}}%
\pgfpathlineto{\pgfqpoint{5.711094in}{3.773662in}}%
\pgfusepath{stroke}%
\end{pgfscope}%
\begin{pgfscope}%
\pgfpathrectangle{\pgfqpoint{0.481978in}{0.331635in}}{\pgfqpoint{9.300000in}{7.700000in}}%
\pgfusepath{clip}%
\pgfsetrectcap%
\pgfsetroundjoin%
\pgfsetlinewidth{1.505625pt}%
\definecolor{currentstroke}{rgb}{0.631373,0.788235,0.956863}%
\pgfsetstrokecolor{currentstroke}%
\pgfsetstrokeopacity{0.800000}%
\pgfsetdash{}{0pt}%
\pgfpathmoveto{\pgfqpoint{8.236668in}{4.419102in}}%
\pgfpathlineto{\pgfqpoint{5.711094in}{3.773662in}}%
\pgfusepath{stroke}%
\end{pgfscope}%
\begin{pgfscope}%
\pgfpathrectangle{\pgfqpoint{0.481978in}{0.331635in}}{\pgfqpoint{9.300000in}{7.700000in}}%
\pgfusepath{clip}%
\pgfsetrectcap%
\pgfsetroundjoin%
\pgfsetlinewidth{1.505625pt}%
\definecolor{currentstroke}{rgb}{0.631373,0.788235,0.956863}%
\pgfsetstrokecolor{currentstroke}%
\pgfsetstrokeopacity{0.800000}%
\pgfsetdash{}{0pt}%
\pgfpathmoveto{\pgfqpoint{7.008316in}{2.365537in}}%
\pgfpathlineto{\pgfqpoint{5.711094in}{3.773662in}}%
\pgfusepath{stroke}%
\end{pgfscope}%
\begin{pgfscope}%
\pgfpathrectangle{\pgfqpoint{0.481978in}{0.331635in}}{\pgfqpoint{9.300000in}{7.700000in}}%
\pgfusepath{clip}%
\pgfsetrectcap%
\pgfsetroundjoin%
\pgfsetlinewidth{1.505625pt}%
\definecolor{currentstroke}{rgb}{0.631373,0.788235,0.956863}%
\pgfsetstrokecolor{currentstroke}%
\pgfsetstrokeopacity{0.800000}%
\pgfsetdash{}{0pt}%
\pgfpathmoveto{\pgfqpoint{8.317641in}{5.164385in}}%
\pgfpathlineto{\pgfqpoint{5.711094in}{3.773662in}}%
\pgfusepath{stroke}%
\end{pgfscope}%
\begin{pgfscope}%
\pgfpathrectangle{\pgfqpoint{0.481978in}{0.331635in}}{\pgfqpoint{9.300000in}{7.700000in}}%
\pgfusepath{clip}%
\pgfsetrectcap%
\pgfsetroundjoin%
\pgfsetlinewidth{1.505625pt}%
\definecolor{currentstroke}{rgb}{0.631373,0.788235,0.956863}%
\pgfsetstrokecolor{currentstroke}%
\pgfsetstrokeopacity{0.800000}%
\pgfsetdash{}{0pt}%
\pgfpathmoveto{\pgfqpoint{7.082145in}{1.740727in}}%
\pgfpathlineto{\pgfqpoint{5.711094in}{3.773662in}}%
\pgfusepath{stroke}%
\end{pgfscope}%
\begin{pgfscope}%
\pgfpathrectangle{\pgfqpoint{0.481978in}{0.331635in}}{\pgfqpoint{9.300000in}{7.700000in}}%
\pgfusepath{clip}%
\pgfsetrectcap%
\pgfsetroundjoin%
\pgfsetlinewidth{1.505625pt}%
\definecolor{currentstroke}{rgb}{0.631373,0.788235,0.956863}%
\pgfsetstrokecolor{currentstroke}%
\pgfsetstrokeopacity{0.800000}%
\pgfsetdash{}{0pt}%
\pgfpathmoveto{\pgfqpoint{5.724298in}{1.804236in}}%
\pgfpathlineto{\pgfqpoint{5.711094in}{3.773662in}}%
\pgfusepath{stroke}%
\end{pgfscope}%
\begin{pgfscope}%
\pgfpathrectangle{\pgfqpoint{0.481978in}{0.331635in}}{\pgfqpoint{9.300000in}{7.700000in}}%
\pgfusepath{clip}%
\pgfsetrectcap%
\pgfsetroundjoin%
\pgfsetlinewidth{1.505625pt}%
\definecolor{currentstroke}{rgb}{0.631373,0.788235,0.956863}%
\pgfsetstrokecolor{currentstroke}%
\pgfsetstrokeopacity{0.800000}%
\pgfsetdash{}{0pt}%
\pgfpathmoveto{\pgfqpoint{2.561461in}{2.024440in}}%
\pgfpathlineto{\pgfqpoint{5.711094in}{3.773662in}}%
\pgfusepath{stroke}%
\end{pgfscope}%
\begin{pgfscope}%
\pgfpathrectangle{\pgfqpoint{0.481978in}{0.331635in}}{\pgfqpoint{9.300000in}{7.700000in}}%
\pgfusepath{clip}%
\pgfsetrectcap%
\pgfsetroundjoin%
\pgfsetlinewidth{1.505625pt}%
\definecolor{currentstroke}{rgb}{0.631373,0.788235,0.956863}%
\pgfsetstrokecolor{currentstroke}%
\pgfsetstrokeopacity{0.800000}%
\pgfsetdash{}{0pt}%
\pgfpathmoveto{\pgfqpoint{6.177807in}{1.877434in}}%
\pgfpathlineto{\pgfqpoint{5.711094in}{3.773662in}}%
\pgfusepath{stroke}%
\end{pgfscope}%
\begin{pgfscope}%
\pgfpathrectangle{\pgfqpoint{0.481978in}{0.331635in}}{\pgfqpoint{9.300000in}{7.700000in}}%
\pgfusepath{clip}%
\pgfsetrectcap%
\pgfsetroundjoin%
\pgfsetlinewidth{1.505625pt}%
\definecolor{currentstroke}{rgb}{0.631373,0.788235,0.956863}%
\pgfsetstrokecolor{currentstroke}%
\pgfsetstrokeopacity{0.800000}%
\pgfsetdash{}{0pt}%
\pgfpathmoveto{\pgfqpoint{6.336424in}{0.681635in}}%
\pgfpathlineto{\pgfqpoint{5.711094in}{3.773662in}}%
\pgfusepath{stroke}%
\end{pgfscope}%
\begin{pgfscope}%
\pgfpathrectangle{\pgfqpoint{0.481978in}{0.331635in}}{\pgfqpoint{9.300000in}{7.700000in}}%
\pgfusepath{clip}%
\pgfsetrectcap%
\pgfsetroundjoin%
\pgfsetlinewidth{1.505625pt}%
\definecolor{currentstroke}{rgb}{0.631373,0.788235,0.956863}%
\pgfsetstrokecolor{currentstroke}%
\pgfsetstrokeopacity{0.800000}%
\pgfsetdash{}{0pt}%
\pgfpathmoveto{\pgfqpoint{7.682062in}{5.953421in}}%
\pgfpathlineto{\pgfqpoint{5.711094in}{3.773662in}}%
\pgfusepath{stroke}%
\end{pgfscope}%
\begin{pgfscope}%
\pgfpathrectangle{\pgfqpoint{0.481978in}{0.331635in}}{\pgfqpoint{9.300000in}{7.700000in}}%
\pgfusepath{clip}%
\pgfsetrectcap%
\pgfsetroundjoin%
\pgfsetlinewidth{1.505625pt}%
\definecolor{currentstroke}{rgb}{0.631373,0.788235,0.956863}%
\pgfsetstrokecolor{currentstroke}%
\pgfsetstrokeopacity{0.800000}%
\pgfsetdash{}{0pt}%
\pgfpathmoveto{\pgfqpoint{6.477636in}{3.880545in}}%
\pgfpathlineto{\pgfqpoint{5.711094in}{3.773662in}}%
\pgfusepath{stroke}%
\end{pgfscope}%
\begin{pgfscope}%
\pgfpathrectangle{\pgfqpoint{0.481978in}{0.331635in}}{\pgfqpoint{9.300000in}{7.700000in}}%
\pgfusepath{clip}%
\pgfsetrectcap%
\pgfsetroundjoin%
\pgfsetlinewidth{1.505625pt}%
\definecolor{currentstroke}{rgb}{0.631373,0.788235,0.956863}%
\pgfsetstrokecolor{currentstroke}%
\pgfsetstrokeopacity{0.800000}%
\pgfsetdash{}{0pt}%
\pgfpathmoveto{\pgfqpoint{6.604622in}{2.950776in}}%
\pgfpathlineto{\pgfqpoint{5.711094in}{3.773662in}}%
\pgfusepath{stroke}%
\end{pgfscope}%
\begin{pgfscope}%
\pgfpathrectangle{\pgfqpoint{0.481978in}{0.331635in}}{\pgfqpoint{9.300000in}{7.700000in}}%
\pgfusepath{clip}%
\pgfsetrectcap%
\pgfsetroundjoin%
\pgfsetlinewidth{1.505625pt}%
\definecolor{currentstroke}{rgb}{0.631373,0.788235,0.956863}%
\pgfsetstrokecolor{currentstroke}%
\pgfsetstrokeopacity{0.800000}%
\pgfsetdash{}{0pt}%
\pgfpathmoveto{\pgfqpoint{7.441282in}{2.708973in}}%
\pgfpathlineto{\pgfqpoint{5.711094in}{3.773662in}}%
\pgfusepath{stroke}%
\end{pgfscope}%
\begin{pgfscope}%
\pgfpathrectangle{\pgfqpoint{0.481978in}{0.331635in}}{\pgfqpoint{9.300000in}{7.700000in}}%
\pgfusepath{clip}%
\pgfsetrectcap%
\pgfsetroundjoin%
\pgfsetlinewidth{1.505625pt}%
\definecolor{currentstroke}{rgb}{0.631373,0.788235,0.956863}%
\pgfsetstrokecolor{currentstroke}%
\pgfsetstrokeopacity{0.800000}%
\pgfsetdash{}{0pt}%
\pgfpathmoveto{\pgfqpoint{5.021448in}{5.598038in}}%
\pgfpathlineto{\pgfqpoint{5.711094in}{3.773662in}}%
\pgfusepath{stroke}%
\end{pgfscope}%
\begin{pgfscope}%
\pgfpathrectangle{\pgfqpoint{0.481978in}{0.331635in}}{\pgfqpoint{9.300000in}{7.700000in}}%
\pgfusepath{clip}%
\pgfsetrectcap%
\pgfsetroundjoin%
\pgfsetlinewidth{1.505625pt}%
\definecolor{currentstroke}{rgb}{0.631373,0.788235,0.956863}%
\pgfsetstrokecolor{currentstroke}%
\pgfsetstrokeopacity{0.800000}%
\pgfsetdash{}{0pt}%
\pgfpathmoveto{\pgfqpoint{6.840343in}{4.443893in}}%
\pgfpathlineto{\pgfqpoint{5.711094in}{3.773662in}}%
\pgfusepath{stroke}%
\end{pgfscope}%
\begin{pgfscope}%
\pgfpathrectangle{\pgfqpoint{0.481978in}{0.331635in}}{\pgfqpoint{9.300000in}{7.700000in}}%
\pgfusepath{clip}%
\pgfsetrectcap%
\pgfsetroundjoin%
\pgfsetlinewidth{1.505625pt}%
\definecolor{currentstroke}{rgb}{0.631373,0.788235,0.956863}%
\pgfsetstrokecolor{currentstroke}%
\pgfsetstrokeopacity{0.800000}%
\pgfsetdash{}{0pt}%
\pgfpathmoveto{\pgfqpoint{6.128623in}{6.002066in}}%
\pgfpathlineto{\pgfqpoint{5.711094in}{3.773662in}}%
\pgfusepath{stroke}%
\end{pgfscope}%
\begin{pgfscope}%
\pgfpathrectangle{\pgfqpoint{0.481978in}{0.331635in}}{\pgfqpoint{9.300000in}{7.700000in}}%
\pgfusepath{clip}%
\pgfsetrectcap%
\pgfsetroundjoin%
\pgfsetlinewidth{1.505625pt}%
\definecolor{currentstroke}{rgb}{0.631373,0.788235,0.956863}%
\pgfsetstrokecolor{currentstroke}%
\pgfsetstrokeopacity{0.800000}%
\pgfsetdash{}{0pt}%
\pgfpathmoveto{\pgfqpoint{6.130665in}{5.997020in}}%
\pgfpathlineto{\pgfqpoint{5.711094in}{3.773662in}}%
\pgfusepath{stroke}%
\end{pgfscope}%
\begin{pgfscope}%
\pgfpathrectangle{\pgfqpoint{0.481978in}{0.331635in}}{\pgfqpoint{9.300000in}{7.700000in}}%
\pgfusepath{clip}%
\pgfsetrectcap%
\pgfsetroundjoin%
\pgfsetlinewidth{1.505625pt}%
\definecolor{currentstroke}{rgb}{0.631373,0.788235,0.956863}%
\pgfsetstrokecolor{currentstroke}%
\pgfsetstrokeopacity{0.800000}%
\pgfsetdash{}{0pt}%
\pgfpathmoveto{\pgfqpoint{5.250013in}{5.339556in}}%
\pgfpathlineto{\pgfqpoint{5.711094in}{3.773662in}}%
\pgfusepath{stroke}%
\end{pgfscope}%
\begin{pgfscope}%
\pgfpathrectangle{\pgfqpoint{0.481978in}{0.331635in}}{\pgfqpoint{9.300000in}{7.700000in}}%
\pgfusepath{clip}%
\pgfsetrectcap%
\pgfsetroundjoin%
\pgfsetlinewidth{1.505625pt}%
\definecolor{currentstroke}{rgb}{0.631373,0.788235,0.956863}%
\pgfsetstrokecolor{currentstroke}%
\pgfsetstrokeopacity{0.800000}%
\pgfsetdash{}{0pt}%
\pgfpathmoveto{\pgfqpoint{7.725554in}{5.826615in}}%
\pgfpathlineto{\pgfqpoint{5.711094in}{3.773662in}}%
\pgfusepath{stroke}%
\end{pgfscope}%
\begin{pgfscope}%
\pgfpathrectangle{\pgfqpoint{0.481978in}{0.331635in}}{\pgfqpoint{9.300000in}{7.700000in}}%
\pgfusepath{clip}%
\pgfsetrectcap%
\pgfsetroundjoin%
\pgfsetlinewidth{1.505625pt}%
\definecolor{currentstroke}{rgb}{0.631373,0.788235,0.956863}%
\pgfsetstrokecolor{currentstroke}%
\pgfsetstrokeopacity{0.800000}%
\pgfsetdash{}{0pt}%
\pgfpathmoveto{\pgfqpoint{5.804196in}{2.934150in}}%
\pgfpathlineto{\pgfqpoint{5.711094in}{3.773662in}}%
\pgfusepath{stroke}%
\end{pgfscope}%
\begin{pgfscope}%
\pgfpathrectangle{\pgfqpoint{0.481978in}{0.331635in}}{\pgfqpoint{9.300000in}{7.700000in}}%
\pgfusepath{clip}%
\pgfsetrectcap%
\pgfsetroundjoin%
\pgfsetlinewidth{1.505625pt}%
\definecolor{currentstroke}{rgb}{0.631373,0.788235,0.956863}%
\pgfsetstrokecolor{currentstroke}%
\pgfsetstrokeopacity{0.800000}%
\pgfsetdash{}{0pt}%
\pgfpathmoveto{\pgfqpoint{4.361412in}{6.142763in}}%
\pgfpathlineto{\pgfqpoint{5.711094in}{3.773662in}}%
\pgfusepath{stroke}%
\end{pgfscope}%
\begin{pgfscope}%
\pgfpathrectangle{\pgfqpoint{0.481978in}{0.331635in}}{\pgfqpoint{9.300000in}{7.700000in}}%
\pgfusepath{clip}%
\pgfsetrectcap%
\pgfsetroundjoin%
\pgfsetlinewidth{1.505625pt}%
\definecolor{currentstroke}{rgb}{0.631373,0.788235,0.956863}%
\pgfsetstrokecolor{currentstroke}%
\pgfsetstrokeopacity{0.800000}%
\pgfsetdash{}{0pt}%
\pgfpathmoveto{\pgfqpoint{7.515857in}{5.150238in}}%
\pgfpathlineto{\pgfqpoint{5.711094in}{3.773662in}}%
\pgfusepath{stroke}%
\end{pgfscope}%
\begin{pgfscope}%
\pgfpathrectangle{\pgfqpoint{0.481978in}{0.331635in}}{\pgfqpoint{9.300000in}{7.700000in}}%
\pgfusepath{clip}%
\pgfsetrectcap%
\pgfsetroundjoin%
\pgfsetlinewidth{1.505625pt}%
\definecolor{currentstroke}{rgb}{0.631373,0.788235,0.956863}%
\pgfsetstrokecolor{currentstroke}%
\pgfsetstrokeopacity{0.800000}%
\pgfsetdash{}{0pt}%
\pgfpathmoveto{\pgfqpoint{8.605266in}{4.835347in}}%
\pgfpathlineto{\pgfqpoint{5.711094in}{3.773662in}}%
\pgfusepath{stroke}%
\end{pgfscope}%
\begin{pgfscope}%
\pgfpathrectangle{\pgfqpoint{0.481978in}{0.331635in}}{\pgfqpoint{9.300000in}{7.700000in}}%
\pgfusepath{clip}%
\pgfsetrectcap%
\pgfsetroundjoin%
\pgfsetlinewidth{1.505625pt}%
\definecolor{currentstroke}{rgb}{0.631373,0.788235,0.956863}%
\pgfsetstrokecolor{currentstroke}%
\pgfsetstrokeopacity{0.800000}%
\pgfsetdash{}{0pt}%
\pgfpathmoveto{\pgfqpoint{6.196847in}{1.949591in}}%
\pgfpathlineto{\pgfqpoint{5.711094in}{3.773662in}}%
\pgfusepath{stroke}%
\end{pgfscope}%
\begin{pgfscope}%
\pgfpathrectangle{\pgfqpoint{0.481978in}{0.331635in}}{\pgfqpoint{9.300000in}{7.700000in}}%
\pgfusepath{clip}%
\pgfsetrectcap%
\pgfsetroundjoin%
\pgfsetlinewidth{1.505625pt}%
\definecolor{currentstroke}{rgb}{0.631373,0.788235,0.956863}%
\pgfsetstrokecolor{currentstroke}%
\pgfsetstrokeopacity{0.800000}%
\pgfsetdash{}{0pt}%
\pgfpathmoveto{\pgfqpoint{4.336260in}{2.062934in}}%
\pgfpathlineto{\pgfqpoint{5.711094in}{3.773662in}}%
\pgfusepath{stroke}%
\end{pgfscope}%
\begin{pgfscope}%
\pgfpathrectangle{\pgfqpoint{0.481978in}{0.331635in}}{\pgfqpoint{9.300000in}{7.700000in}}%
\pgfusepath{clip}%
\pgfsetrectcap%
\pgfsetroundjoin%
\pgfsetlinewidth{1.505625pt}%
\definecolor{currentstroke}{rgb}{0.631373,0.788235,0.956863}%
\pgfsetstrokecolor{currentstroke}%
\pgfsetstrokeopacity{0.800000}%
\pgfsetdash{}{0pt}%
\pgfpathmoveto{\pgfqpoint{4.441169in}{5.411759in}}%
\pgfpathlineto{\pgfqpoint{5.711094in}{3.773662in}}%
\pgfusepath{stroke}%
\end{pgfscope}%
\begin{pgfscope}%
\pgfpathrectangle{\pgfqpoint{0.481978in}{0.331635in}}{\pgfqpoint{9.300000in}{7.700000in}}%
\pgfusepath{clip}%
\pgfsetrectcap%
\pgfsetroundjoin%
\pgfsetlinewidth{1.505625pt}%
\definecolor{currentstroke}{rgb}{0.631373,0.788235,0.956863}%
\pgfsetstrokecolor{currentstroke}%
\pgfsetstrokeopacity{0.800000}%
\pgfsetdash{}{0pt}%
\pgfpathmoveto{\pgfqpoint{6.227427in}{1.697900in}}%
\pgfpathlineto{\pgfqpoint{5.711094in}{3.773662in}}%
\pgfusepath{stroke}%
\end{pgfscope}%
\begin{pgfscope}%
\pgfpathrectangle{\pgfqpoint{0.481978in}{0.331635in}}{\pgfqpoint{9.300000in}{7.700000in}}%
\pgfusepath{clip}%
\pgfsetrectcap%
\pgfsetroundjoin%
\pgfsetlinewidth{1.505625pt}%
\definecolor{currentstroke}{rgb}{0.631373,0.788235,0.956863}%
\pgfsetstrokecolor{currentstroke}%
\pgfsetstrokeopacity{0.800000}%
\pgfsetdash{}{0pt}%
\pgfpathmoveto{\pgfqpoint{7.013858in}{2.259947in}}%
\pgfpathlineto{\pgfqpoint{5.711094in}{3.773662in}}%
\pgfusepath{stroke}%
\end{pgfscope}%
\begin{pgfscope}%
\pgfpathrectangle{\pgfqpoint{0.481978in}{0.331635in}}{\pgfqpoint{9.300000in}{7.700000in}}%
\pgfusepath{clip}%
\pgfsetrectcap%
\pgfsetroundjoin%
\pgfsetlinewidth{1.505625pt}%
\definecolor{currentstroke}{rgb}{0.631373,0.788235,0.956863}%
\pgfsetstrokecolor{currentstroke}%
\pgfsetstrokeopacity{0.800000}%
\pgfsetdash{}{0pt}%
\pgfpathmoveto{\pgfqpoint{3.180339in}{6.512223in}}%
\pgfpathlineto{\pgfqpoint{5.711094in}{3.773662in}}%
\pgfusepath{stroke}%
\end{pgfscope}%
\begin{pgfscope}%
\pgfpathrectangle{\pgfqpoint{0.481978in}{0.331635in}}{\pgfqpoint{9.300000in}{7.700000in}}%
\pgfusepath{clip}%
\pgfsetrectcap%
\pgfsetroundjoin%
\pgfsetlinewidth{1.505625pt}%
\definecolor{currentstroke}{rgb}{0.631373,0.788235,0.956863}%
\pgfsetstrokecolor{currentstroke}%
\pgfsetstrokeopacity{0.800000}%
\pgfsetdash{}{0pt}%
\pgfpathmoveto{\pgfqpoint{4.845738in}{0.865290in}}%
\pgfpathlineto{\pgfqpoint{5.711094in}{3.773662in}}%
\pgfusepath{stroke}%
\end{pgfscope}%
\begin{pgfscope}%
\pgfpathrectangle{\pgfqpoint{0.481978in}{0.331635in}}{\pgfqpoint{9.300000in}{7.700000in}}%
\pgfusepath{clip}%
\pgfsetrectcap%
\pgfsetroundjoin%
\pgfsetlinewidth{1.505625pt}%
\definecolor{currentstroke}{rgb}{0.631373,0.788235,0.956863}%
\pgfsetstrokecolor{currentstroke}%
\pgfsetstrokeopacity{0.800000}%
\pgfsetdash{}{0pt}%
\pgfpathmoveto{\pgfqpoint{4.394203in}{1.848846in}}%
\pgfpathlineto{\pgfqpoint{5.711094in}{3.773662in}}%
\pgfusepath{stroke}%
\end{pgfscope}%
\begin{pgfscope}%
\pgfpathrectangle{\pgfqpoint{0.481978in}{0.331635in}}{\pgfqpoint{9.300000in}{7.700000in}}%
\pgfusepath{clip}%
\pgfsetrectcap%
\pgfsetroundjoin%
\pgfsetlinewidth{1.505625pt}%
\definecolor{currentstroke}{rgb}{0.631373,0.788235,0.956863}%
\pgfsetstrokecolor{currentstroke}%
\pgfsetstrokeopacity{0.800000}%
\pgfsetdash{}{0pt}%
\pgfpathmoveto{\pgfqpoint{5.179568in}{5.355810in}}%
\pgfpathlineto{\pgfqpoint{5.711094in}{3.773662in}}%
\pgfusepath{stroke}%
\end{pgfscope}%
\begin{pgfscope}%
\pgfpathrectangle{\pgfqpoint{0.481978in}{0.331635in}}{\pgfqpoint{9.300000in}{7.700000in}}%
\pgfusepath{clip}%
\pgfsetrectcap%
\pgfsetroundjoin%
\pgfsetlinewidth{1.505625pt}%
\definecolor{currentstroke}{rgb}{0.631373,0.788235,0.956863}%
\pgfsetstrokecolor{currentstroke}%
\pgfsetstrokeopacity{0.800000}%
\pgfsetdash{}{0pt}%
\pgfpathmoveto{\pgfqpoint{7.941055in}{4.986937in}}%
\pgfpathlineto{\pgfqpoint{5.711094in}{3.773662in}}%
\pgfusepath{stroke}%
\end{pgfscope}%
\begin{pgfscope}%
\pgfpathrectangle{\pgfqpoint{0.481978in}{0.331635in}}{\pgfqpoint{9.300000in}{7.700000in}}%
\pgfusepath{clip}%
\pgfsetrectcap%
\pgfsetroundjoin%
\pgfsetlinewidth{1.505625pt}%
\definecolor{currentstroke}{rgb}{0.631373,0.788235,0.956863}%
\pgfsetstrokecolor{currentstroke}%
\pgfsetstrokeopacity{0.800000}%
\pgfsetdash{}{0pt}%
\pgfpathmoveto{\pgfqpoint{3.294139in}{1.577402in}}%
\pgfpathlineto{\pgfqpoint{5.711094in}{3.773662in}}%
\pgfusepath{stroke}%
\end{pgfscope}%
\begin{pgfscope}%
\pgfpathrectangle{\pgfqpoint{0.481978in}{0.331635in}}{\pgfqpoint{9.300000in}{7.700000in}}%
\pgfusepath{clip}%
\pgfsetrectcap%
\pgfsetroundjoin%
\pgfsetlinewidth{1.505625pt}%
\definecolor{currentstroke}{rgb}{0.631373,0.788235,0.956863}%
\pgfsetstrokecolor{currentstroke}%
\pgfsetstrokeopacity{0.800000}%
\pgfsetdash{}{0pt}%
\pgfpathmoveto{\pgfqpoint{7.359994in}{4.031512in}}%
\pgfpathlineto{\pgfqpoint{5.711094in}{3.773662in}}%
\pgfusepath{stroke}%
\end{pgfscope}%
\begin{pgfscope}%
\pgfpathrectangle{\pgfqpoint{0.481978in}{0.331635in}}{\pgfqpoint{9.300000in}{7.700000in}}%
\pgfusepath{clip}%
\pgfsetrectcap%
\pgfsetroundjoin%
\pgfsetlinewidth{1.505625pt}%
\definecolor{currentstroke}{rgb}{0.631373,0.788235,0.956863}%
\pgfsetstrokecolor{currentstroke}%
\pgfsetstrokeopacity{0.800000}%
\pgfsetdash{}{0pt}%
\pgfpathmoveto{\pgfqpoint{7.314751in}{4.915613in}}%
\pgfpathlineto{\pgfqpoint{5.711094in}{3.773662in}}%
\pgfusepath{stroke}%
\end{pgfscope}%
\begin{pgfscope}%
\pgfpathrectangle{\pgfqpoint{0.481978in}{0.331635in}}{\pgfqpoint{9.300000in}{7.700000in}}%
\pgfusepath{clip}%
\pgfsetrectcap%
\pgfsetroundjoin%
\pgfsetlinewidth{1.505625pt}%
\definecolor{currentstroke}{rgb}{0.631373,0.788235,0.956863}%
\pgfsetstrokecolor{currentstroke}%
\pgfsetstrokeopacity{0.800000}%
\pgfsetdash{}{0pt}%
\pgfpathmoveto{\pgfqpoint{3.066979in}{6.682028in}}%
\pgfpathlineto{\pgfqpoint{5.711094in}{3.773662in}}%
\pgfusepath{stroke}%
\end{pgfscope}%
\begin{pgfscope}%
\pgfpathrectangle{\pgfqpoint{0.481978in}{0.331635in}}{\pgfqpoint{9.300000in}{7.700000in}}%
\pgfusepath{clip}%
\pgfsetrectcap%
\pgfsetroundjoin%
\pgfsetlinewidth{1.505625pt}%
\definecolor{currentstroke}{rgb}{0.631373,0.788235,0.956863}%
\pgfsetstrokecolor{currentstroke}%
\pgfsetstrokeopacity{0.800000}%
\pgfsetdash{}{0pt}%
\pgfpathmoveto{\pgfqpoint{2.849648in}{6.910141in}}%
\pgfpathlineto{\pgfqpoint{5.711094in}{3.773662in}}%
\pgfusepath{stroke}%
\end{pgfscope}%
\begin{pgfscope}%
\pgfpathrectangle{\pgfqpoint{0.481978in}{0.331635in}}{\pgfqpoint{9.300000in}{7.700000in}}%
\pgfusepath{clip}%
\pgfsetrectcap%
\pgfsetroundjoin%
\pgfsetlinewidth{1.505625pt}%
\definecolor{currentstroke}{rgb}{0.631373,0.788235,0.956863}%
\pgfsetstrokecolor{currentstroke}%
\pgfsetstrokeopacity{0.800000}%
\pgfsetdash{}{0pt}%
\pgfpathmoveto{\pgfqpoint{2.771056in}{6.665431in}}%
\pgfpathlineto{\pgfqpoint{5.711094in}{3.773662in}}%
\pgfusepath{stroke}%
\end{pgfscope}%
\begin{pgfscope}%
\pgfpathrectangle{\pgfqpoint{0.481978in}{0.331635in}}{\pgfqpoint{9.300000in}{7.700000in}}%
\pgfusepath{clip}%
\pgfsetrectcap%
\pgfsetroundjoin%
\pgfsetlinewidth{1.505625pt}%
\definecolor{currentstroke}{rgb}{0.631373,0.788235,0.956863}%
\pgfsetstrokecolor{currentstroke}%
\pgfsetstrokeopacity{0.800000}%
\pgfsetdash{}{0pt}%
\pgfpathmoveto{\pgfqpoint{4.853047in}{5.922135in}}%
\pgfpathlineto{\pgfqpoint{5.711094in}{3.773662in}}%
\pgfusepath{stroke}%
\end{pgfscope}%
\begin{pgfscope}%
\pgfpathrectangle{\pgfqpoint{0.481978in}{0.331635in}}{\pgfqpoint{9.300000in}{7.700000in}}%
\pgfusepath{clip}%
\pgfsetrectcap%
\pgfsetroundjoin%
\pgfsetlinewidth{1.505625pt}%
\definecolor{currentstroke}{rgb}{0.631373,0.788235,0.956863}%
\pgfsetstrokecolor{currentstroke}%
\pgfsetstrokeopacity{0.800000}%
\pgfsetdash{}{0pt}%
\pgfpathmoveto{\pgfqpoint{5.879927in}{0.844058in}}%
\pgfpathlineto{\pgfqpoint{5.711094in}{3.773662in}}%
\pgfusepath{stroke}%
\end{pgfscope}%
\begin{pgfscope}%
\pgfpathrectangle{\pgfqpoint{0.481978in}{0.331635in}}{\pgfqpoint{9.300000in}{7.700000in}}%
\pgfusepath{clip}%
\pgfsetrectcap%
\pgfsetroundjoin%
\pgfsetlinewidth{1.505625pt}%
\definecolor{currentstroke}{rgb}{0.631373,0.788235,0.956863}%
\pgfsetstrokecolor{currentstroke}%
\pgfsetstrokeopacity{0.800000}%
\pgfsetdash{}{0pt}%
\pgfpathmoveto{\pgfqpoint{6.834691in}{2.083391in}}%
\pgfpathlineto{\pgfqpoint{5.711094in}{3.773662in}}%
\pgfusepath{stroke}%
\end{pgfscope}%
\begin{pgfscope}%
\pgfpathrectangle{\pgfqpoint{0.481978in}{0.331635in}}{\pgfqpoint{9.300000in}{7.700000in}}%
\pgfusepath{clip}%
\pgfsetrectcap%
\pgfsetroundjoin%
\pgfsetlinewidth{1.505625pt}%
\definecolor{currentstroke}{rgb}{0.631373,0.788235,0.956863}%
\pgfsetstrokecolor{currentstroke}%
\pgfsetstrokeopacity{0.800000}%
\pgfsetdash{}{0pt}%
\pgfpathmoveto{\pgfqpoint{5.297141in}{2.779134in}}%
\pgfpathlineto{\pgfqpoint{5.711094in}{3.773662in}}%
\pgfusepath{stroke}%
\end{pgfscope}%
\begin{pgfscope}%
\pgfpathrectangle{\pgfqpoint{0.481978in}{0.331635in}}{\pgfqpoint{9.300000in}{7.700000in}}%
\pgfusepath{clip}%
\pgfsetrectcap%
\pgfsetroundjoin%
\pgfsetlinewidth{1.505625pt}%
\definecolor{currentstroke}{rgb}{0.631373,0.788235,0.956863}%
\pgfsetstrokecolor{currentstroke}%
\pgfsetstrokeopacity{0.800000}%
\pgfsetdash{}{0pt}%
\pgfpathmoveto{\pgfqpoint{6.713307in}{5.182723in}}%
\pgfpathlineto{\pgfqpoint{5.711094in}{3.773662in}}%
\pgfusepath{stroke}%
\end{pgfscope}%
\begin{pgfscope}%
\pgfpathrectangle{\pgfqpoint{0.481978in}{0.331635in}}{\pgfqpoint{9.300000in}{7.700000in}}%
\pgfusepath{clip}%
\pgfsetrectcap%
\pgfsetroundjoin%
\pgfsetlinewidth{1.505625pt}%
\definecolor{currentstroke}{rgb}{0.631373,0.788235,0.956863}%
\pgfsetstrokecolor{currentstroke}%
\pgfsetstrokeopacity{0.800000}%
\pgfsetdash{}{0pt}%
\pgfpathmoveto{\pgfqpoint{5.117952in}{2.351880in}}%
\pgfpathlineto{\pgfqpoint{5.711094in}{3.773662in}}%
\pgfusepath{stroke}%
\end{pgfscope}%
\begin{pgfscope}%
\pgfpathrectangle{\pgfqpoint{0.481978in}{0.331635in}}{\pgfqpoint{9.300000in}{7.700000in}}%
\pgfusepath{clip}%
\pgfsetrectcap%
\pgfsetroundjoin%
\pgfsetlinewidth{1.505625pt}%
\definecolor{currentstroke}{rgb}{0.631373,0.788235,0.956863}%
\pgfsetstrokecolor{currentstroke}%
\pgfsetstrokeopacity{0.800000}%
\pgfsetdash{}{0pt}%
\pgfpathmoveto{\pgfqpoint{7.076892in}{3.094365in}}%
\pgfpathlineto{\pgfqpoint{5.711094in}{3.773662in}}%
\pgfusepath{stroke}%
\end{pgfscope}%
\begin{pgfscope}%
\pgfpathrectangle{\pgfqpoint{0.481978in}{0.331635in}}{\pgfqpoint{9.300000in}{7.700000in}}%
\pgfusepath{clip}%
\pgfsetrectcap%
\pgfsetroundjoin%
\pgfsetlinewidth{1.505625pt}%
\definecolor{currentstroke}{rgb}{0.631373,0.788235,0.956863}%
\pgfsetstrokecolor{currentstroke}%
\pgfsetstrokeopacity{0.800000}%
\pgfsetdash{}{0pt}%
\pgfpathmoveto{\pgfqpoint{7.507895in}{1.953530in}}%
\pgfpathlineto{\pgfqpoint{5.711094in}{3.773662in}}%
\pgfusepath{stroke}%
\end{pgfscope}%
\begin{pgfscope}%
\pgfpathrectangle{\pgfqpoint{0.481978in}{0.331635in}}{\pgfqpoint{9.300000in}{7.700000in}}%
\pgfusepath{clip}%
\pgfsetrectcap%
\pgfsetroundjoin%
\pgfsetlinewidth{1.505625pt}%
\definecolor{currentstroke}{rgb}{0.631373,0.788235,0.956863}%
\pgfsetstrokecolor{currentstroke}%
\pgfsetstrokeopacity{0.800000}%
\pgfsetdash{}{0pt}%
\pgfpathmoveto{\pgfqpoint{4.102323in}{5.830387in}}%
\pgfpathlineto{\pgfqpoint{5.711094in}{3.773662in}}%
\pgfusepath{stroke}%
\end{pgfscope}%
\begin{pgfscope}%
\pgfpathrectangle{\pgfqpoint{0.481978in}{0.331635in}}{\pgfqpoint{9.300000in}{7.700000in}}%
\pgfusepath{clip}%
\pgfsetrectcap%
\pgfsetroundjoin%
\pgfsetlinewidth{1.505625pt}%
\definecolor{currentstroke}{rgb}{0.631373,0.788235,0.956863}%
\pgfsetstrokecolor{currentstroke}%
\pgfsetstrokeopacity{0.800000}%
\pgfsetdash{}{0pt}%
\pgfpathmoveto{\pgfqpoint{6.020143in}{5.316402in}}%
\pgfpathlineto{\pgfqpoint{5.711094in}{3.773662in}}%
\pgfusepath{stroke}%
\end{pgfscope}%
\begin{pgfscope}%
\pgfpathrectangle{\pgfqpoint{0.481978in}{0.331635in}}{\pgfqpoint{9.300000in}{7.700000in}}%
\pgfusepath{clip}%
\pgfsetrectcap%
\pgfsetroundjoin%
\pgfsetlinewidth{1.505625pt}%
\definecolor{currentstroke}{rgb}{0.631373,0.788235,0.956863}%
\pgfsetstrokecolor{currentstroke}%
\pgfsetstrokeopacity{0.800000}%
\pgfsetdash{}{0pt}%
\pgfpathmoveto{\pgfqpoint{7.865979in}{5.806222in}}%
\pgfpathlineto{\pgfqpoint{5.711094in}{3.773662in}}%
\pgfusepath{stroke}%
\end{pgfscope}%
\begin{pgfscope}%
\pgfpathrectangle{\pgfqpoint{0.481978in}{0.331635in}}{\pgfqpoint{9.300000in}{7.700000in}}%
\pgfusepath{clip}%
\pgfsetrectcap%
\pgfsetroundjoin%
\pgfsetlinewidth{1.505625pt}%
\definecolor{currentstroke}{rgb}{0.631373,0.788235,0.956863}%
\pgfsetstrokecolor{currentstroke}%
\pgfsetstrokeopacity{0.800000}%
\pgfsetdash{}{0pt}%
\pgfpathmoveto{\pgfqpoint{5.392504in}{5.413544in}}%
\pgfpathlineto{\pgfqpoint{5.711094in}{3.773662in}}%
\pgfusepath{stroke}%
\end{pgfscope}%
\begin{pgfscope}%
\pgfpathrectangle{\pgfqpoint{0.481978in}{0.331635in}}{\pgfqpoint{9.300000in}{7.700000in}}%
\pgfusepath{clip}%
\pgfsetrectcap%
\pgfsetroundjoin%
\pgfsetlinewidth{1.505625pt}%
\definecolor{currentstroke}{rgb}{0.631373,0.788235,0.956863}%
\pgfsetstrokecolor{currentstroke}%
\pgfsetstrokeopacity{0.800000}%
\pgfsetdash{}{0pt}%
\pgfpathmoveto{\pgfqpoint{2.932058in}{5.424089in}}%
\pgfpathlineto{\pgfqpoint{5.711094in}{3.773662in}}%
\pgfusepath{stroke}%
\end{pgfscope}%
\begin{pgfscope}%
\pgfpathrectangle{\pgfqpoint{0.481978in}{0.331635in}}{\pgfqpoint{9.300000in}{7.700000in}}%
\pgfusepath{clip}%
\pgfsetrectcap%
\pgfsetroundjoin%
\pgfsetlinewidth{1.505625pt}%
\definecolor{currentstroke}{rgb}{0.631373,0.788235,0.956863}%
\pgfsetstrokecolor{currentstroke}%
\pgfsetstrokeopacity{0.800000}%
\pgfsetdash{}{0pt}%
\pgfpathmoveto{\pgfqpoint{7.779256in}{5.079699in}}%
\pgfpathlineto{\pgfqpoint{5.711094in}{3.773662in}}%
\pgfusepath{stroke}%
\end{pgfscope}%
\begin{pgfscope}%
\pgfpathrectangle{\pgfqpoint{0.481978in}{0.331635in}}{\pgfqpoint{9.300000in}{7.700000in}}%
\pgfusepath{clip}%
\pgfsetrectcap%
\pgfsetroundjoin%
\pgfsetlinewidth{1.505625pt}%
\definecolor{currentstroke}{rgb}{0.631373,0.788235,0.956863}%
\pgfsetstrokecolor{currentstroke}%
\pgfsetstrokeopacity{0.800000}%
\pgfsetdash{}{0pt}%
\pgfpathmoveto{\pgfqpoint{7.196974in}{1.631105in}}%
\pgfpathlineto{\pgfqpoint{5.711094in}{3.773662in}}%
\pgfusepath{stroke}%
\end{pgfscope}%
\begin{pgfscope}%
\pgfpathrectangle{\pgfqpoint{0.481978in}{0.331635in}}{\pgfqpoint{9.300000in}{7.700000in}}%
\pgfusepath{clip}%
\pgfsetrectcap%
\pgfsetroundjoin%
\pgfsetlinewidth{1.505625pt}%
\definecolor{currentstroke}{rgb}{0.631373,0.788235,0.956863}%
\pgfsetstrokecolor{currentstroke}%
\pgfsetstrokeopacity{0.800000}%
\pgfsetdash{}{0pt}%
\pgfpathmoveto{\pgfqpoint{3.418933in}{1.347084in}}%
\pgfpathlineto{\pgfqpoint{5.711094in}{3.773662in}}%
\pgfusepath{stroke}%
\end{pgfscope}%
\begin{pgfscope}%
\pgfpathrectangle{\pgfqpoint{0.481978in}{0.331635in}}{\pgfqpoint{9.300000in}{7.700000in}}%
\pgfusepath{clip}%
\pgfsetrectcap%
\pgfsetroundjoin%
\pgfsetlinewidth{1.505625pt}%
\definecolor{currentstroke}{rgb}{0.631373,0.788235,0.956863}%
\pgfsetstrokecolor{currentstroke}%
\pgfsetstrokeopacity{0.800000}%
\pgfsetdash{}{0pt}%
\pgfpathmoveto{\pgfqpoint{5.333628in}{3.495257in}}%
\pgfpathlineto{\pgfqpoint{5.711094in}{3.773662in}}%
\pgfusepath{stroke}%
\end{pgfscope}%
\begin{pgfscope}%
\pgfpathrectangle{\pgfqpoint{0.481978in}{0.331635in}}{\pgfqpoint{9.300000in}{7.700000in}}%
\pgfusepath{clip}%
\pgfsetrectcap%
\pgfsetroundjoin%
\pgfsetlinewidth{1.505625pt}%
\definecolor{currentstroke}{rgb}{0.631373,0.788235,0.956863}%
\pgfsetstrokecolor{currentstroke}%
\pgfsetstrokeopacity{0.800000}%
\pgfsetdash{}{0pt}%
\pgfpathmoveto{\pgfqpoint{6.946844in}{2.835271in}}%
\pgfpathlineto{\pgfqpoint{5.711094in}{3.773662in}}%
\pgfusepath{stroke}%
\end{pgfscope}%
\begin{pgfscope}%
\pgfpathrectangle{\pgfqpoint{0.481978in}{0.331635in}}{\pgfqpoint{9.300000in}{7.700000in}}%
\pgfusepath{clip}%
\pgfsetrectcap%
\pgfsetroundjoin%
\pgfsetlinewidth{1.505625pt}%
\definecolor{currentstroke}{rgb}{0.631373,0.788235,0.956863}%
\pgfsetstrokecolor{currentstroke}%
\pgfsetstrokeopacity{0.800000}%
\pgfsetdash{}{0pt}%
\pgfpathmoveto{\pgfqpoint{8.414129in}{5.020721in}}%
\pgfpathlineto{\pgfqpoint{5.711094in}{3.773662in}}%
\pgfusepath{stroke}%
\end{pgfscope}%
\begin{pgfscope}%
\pgfpathrectangle{\pgfqpoint{0.481978in}{0.331635in}}{\pgfqpoint{9.300000in}{7.700000in}}%
\pgfusepath{clip}%
\pgfsetrectcap%
\pgfsetroundjoin%
\pgfsetlinewidth{1.505625pt}%
\definecolor{currentstroke}{rgb}{0.631373,0.788235,0.956863}%
\pgfsetstrokecolor{currentstroke}%
\pgfsetstrokeopacity{0.800000}%
\pgfsetdash{}{0pt}%
\pgfpathmoveto{\pgfqpoint{4.957931in}{2.138162in}}%
\pgfpathlineto{\pgfqpoint{5.711094in}{3.773662in}}%
\pgfusepath{stroke}%
\end{pgfscope}%
\begin{pgfscope}%
\pgfpathrectangle{\pgfqpoint{0.481978in}{0.331635in}}{\pgfqpoint{9.300000in}{7.700000in}}%
\pgfusepath{clip}%
\pgfsetrectcap%
\pgfsetroundjoin%
\pgfsetlinewidth{1.505625pt}%
\definecolor{currentstroke}{rgb}{0.631373,0.788235,0.956863}%
\pgfsetstrokecolor{currentstroke}%
\pgfsetstrokeopacity{0.800000}%
\pgfsetdash{}{0pt}%
\pgfpathmoveto{\pgfqpoint{7.947630in}{5.393707in}}%
\pgfpathlineto{\pgfqpoint{5.711094in}{3.773662in}}%
\pgfusepath{stroke}%
\end{pgfscope}%
\begin{pgfscope}%
\pgfpathrectangle{\pgfqpoint{0.481978in}{0.331635in}}{\pgfqpoint{9.300000in}{7.700000in}}%
\pgfusepath{clip}%
\pgfsetrectcap%
\pgfsetroundjoin%
\pgfsetlinewidth{1.505625pt}%
\definecolor{currentstroke}{rgb}{0.631373,0.788235,0.956863}%
\pgfsetstrokecolor{currentstroke}%
\pgfsetstrokeopacity{0.800000}%
\pgfsetdash{}{0pt}%
\pgfpathmoveto{\pgfqpoint{3.205045in}{1.571422in}}%
\pgfpathlineto{\pgfqpoint{5.711094in}{3.773662in}}%
\pgfusepath{stroke}%
\end{pgfscope}%
\begin{pgfscope}%
\pgfpathrectangle{\pgfqpoint{0.481978in}{0.331635in}}{\pgfqpoint{9.300000in}{7.700000in}}%
\pgfusepath{clip}%
\pgfsetrectcap%
\pgfsetroundjoin%
\pgfsetlinewidth{1.505625pt}%
\definecolor{currentstroke}{rgb}{0.631373,0.788235,0.956863}%
\pgfsetstrokecolor{currentstroke}%
\pgfsetstrokeopacity{0.800000}%
\pgfsetdash{}{0pt}%
\pgfpathmoveto{\pgfqpoint{7.723488in}{5.525665in}}%
\pgfpathlineto{\pgfqpoint{5.711094in}{3.773662in}}%
\pgfusepath{stroke}%
\end{pgfscope}%
\begin{pgfscope}%
\pgfpathrectangle{\pgfqpoint{0.481978in}{0.331635in}}{\pgfqpoint{9.300000in}{7.700000in}}%
\pgfusepath{clip}%
\pgfsetrectcap%
\pgfsetroundjoin%
\pgfsetlinewidth{1.505625pt}%
\definecolor{currentstroke}{rgb}{0.631373,0.788235,0.956863}%
\pgfsetstrokecolor{currentstroke}%
\pgfsetstrokeopacity{0.800000}%
\pgfsetdash{}{0pt}%
\pgfpathmoveto{\pgfqpoint{4.705109in}{1.086625in}}%
\pgfpathlineto{\pgfqpoint{5.711094in}{3.773662in}}%
\pgfusepath{stroke}%
\end{pgfscope}%
\begin{pgfscope}%
\pgfpathrectangle{\pgfqpoint{0.481978in}{0.331635in}}{\pgfqpoint{9.300000in}{7.700000in}}%
\pgfusepath{clip}%
\pgfsetrectcap%
\pgfsetroundjoin%
\pgfsetlinewidth{1.505625pt}%
\definecolor{currentstroke}{rgb}{0.631373,0.788235,0.956863}%
\pgfsetstrokecolor{currentstroke}%
\pgfsetstrokeopacity{0.800000}%
\pgfsetdash{}{0pt}%
\pgfpathmoveto{\pgfqpoint{3.932688in}{3.913553in}}%
\pgfpathlineto{\pgfqpoint{5.711094in}{3.773662in}}%
\pgfusepath{stroke}%
\end{pgfscope}%
\begin{pgfscope}%
\pgfpathrectangle{\pgfqpoint{0.481978in}{0.331635in}}{\pgfqpoint{9.300000in}{7.700000in}}%
\pgfusepath{clip}%
\pgfsetrectcap%
\pgfsetroundjoin%
\pgfsetlinewidth{1.505625pt}%
\definecolor{currentstroke}{rgb}{0.631373,0.788235,0.956863}%
\pgfsetstrokecolor{currentstroke}%
\pgfsetstrokeopacity{0.800000}%
\pgfsetdash{}{0pt}%
\pgfpathmoveto{\pgfqpoint{7.599765in}{5.966116in}}%
\pgfpathlineto{\pgfqpoint{5.711094in}{3.773662in}}%
\pgfusepath{stroke}%
\end{pgfscope}%
\begin{pgfscope}%
\pgfpathrectangle{\pgfqpoint{0.481978in}{0.331635in}}{\pgfqpoint{9.300000in}{7.700000in}}%
\pgfusepath{clip}%
\pgfsetrectcap%
\pgfsetroundjoin%
\pgfsetlinewidth{1.505625pt}%
\definecolor{currentstroke}{rgb}{0.631373,0.788235,0.956863}%
\pgfsetstrokecolor{currentstroke}%
\pgfsetstrokeopacity{0.800000}%
\pgfsetdash{}{0pt}%
\pgfpathmoveto{\pgfqpoint{6.870443in}{1.703003in}}%
\pgfpathlineto{\pgfqpoint{5.711094in}{3.773662in}}%
\pgfusepath{stroke}%
\end{pgfscope}%
\begin{pgfscope}%
\pgfpathrectangle{\pgfqpoint{0.481978in}{0.331635in}}{\pgfqpoint{9.300000in}{7.700000in}}%
\pgfusepath{clip}%
\pgfsetrectcap%
\pgfsetroundjoin%
\pgfsetlinewidth{1.505625pt}%
\definecolor{currentstroke}{rgb}{0.631373,0.788235,0.956863}%
\pgfsetstrokecolor{currentstroke}%
\pgfsetstrokeopacity{0.800000}%
\pgfsetdash{}{0pt}%
\pgfpathmoveto{\pgfqpoint{6.112529in}{2.152402in}}%
\pgfpathlineto{\pgfqpoint{5.711094in}{3.773662in}}%
\pgfusepath{stroke}%
\end{pgfscope}%
\begin{pgfscope}%
\pgfpathrectangle{\pgfqpoint{0.481978in}{0.331635in}}{\pgfqpoint{9.300000in}{7.700000in}}%
\pgfusepath{clip}%
\pgfsetrectcap%
\pgfsetroundjoin%
\pgfsetlinewidth{1.505625pt}%
\definecolor{currentstroke}{rgb}{0.631373,0.788235,0.956863}%
\pgfsetstrokecolor{currentstroke}%
\pgfsetstrokeopacity{0.800000}%
\pgfsetdash{}{0pt}%
\pgfpathmoveto{\pgfqpoint{7.807922in}{5.795876in}}%
\pgfpathlineto{\pgfqpoint{5.711094in}{3.773662in}}%
\pgfusepath{stroke}%
\end{pgfscope}%
\begin{pgfscope}%
\pgfpathrectangle{\pgfqpoint{0.481978in}{0.331635in}}{\pgfqpoint{9.300000in}{7.700000in}}%
\pgfusepath{clip}%
\pgfsetrectcap%
\pgfsetroundjoin%
\pgfsetlinewidth{1.505625pt}%
\definecolor{currentstroke}{rgb}{0.631373,0.788235,0.956863}%
\pgfsetstrokecolor{currentstroke}%
\pgfsetstrokeopacity{0.800000}%
\pgfsetdash{}{0pt}%
\pgfpathmoveto{\pgfqpoint{3.326485in}{2.058650in}}%
\pgfpathlineto{\pgfqpoint{5.711094in}{3.773662in}}%
\pgfusepath{stroke}%
\end{pgfscope}%
\begin{pgfscope}%
\pgfpathrectangle{\pgfqpoint{0.481978in}{0.331635in}}{\pgfqpoint{9.300000in}{7.700000in}}%
\pgfusepath{clip}%
\pgfsetrectcap%
\pgfsetroundjoin%
\pgfsetlinewidth{1.505625pt}%
\definecolor{currentstroke}{rgb}{0.631373,0.788235,0.956863}%
\pgfsetstrokecolor{currentstroke}%
\pgfsetstrokeopacity{0.800000}%
\pgfsetdash{}{0pt}%
\pgfpathmoveto{\pgfqpoint{1.925994in}{5.507668in}}%
\pgfpathlineto{\pgfqpoint{5.711094in}{3.773662in}}%
\pgfusepath{stroke}%
\end{pgfscope}%
\begin{pgfscope}%
\pgfpathrectangle{\pgfqpoint{0.481978in}{0.331635in}}{\pgfqpoint{9.300000in}{7.700000in}}%
\pgfusepath{clip}%
\pgfsetrectcap%
\pgfsetroundjoin%
\pgfsetlinewidth{1.505625pt}%
\definecolor{currentstroke}{rgb}{0.631373,0.788235,0.956863}%
\pgfsetstrokecolor{currentstroke}%
\pgfsetstrokeopacity{0.800000}%
\pgfsetdash{}{0pt}%
\pgfpathmoveto{\pgfqpoint{5.878648in}{0.848903in}}%
\pgfpathlineto{\pgfqpoint{5.711094in}{3.773662in}}%
\pgfusepath{stroke}%
\end{pgfscope}%
\begin{pgfscope}%
\pgfpathrectangle{\pgfqpoint{0.481978in}{0.331635in}}{\pgfqpoint{9.300000in}{7.700000in}}%
\pgfusepath{clip}%
\pgfsetrectcap%
\pgfsetroundjoin%
\pgfsetlinewidth{1.505625pt}%
\definecolor{currentstroke}{rgb}{0.631373,0.788235,0.956863}%
\pgfsetstrokecolor{currentstroke}%
\pgfsetstrokeopacity{0.800000}%
\pgfsetdash{}{0pt}%
\pgfpathmoveto{\pgfqpoint{8.178842in}{4.605790in}}%
\pgfpathlineto{\pgfqpoint{5.711094in}{3.773662in}}%
\pgfusepath{stroke}%
\end{pgfscope}%
\begin{pgfscope}%
\pgfpathrectangle{\pgfqpoint{0.481978in}{0.331635in}}{\pgfqpoint{9.300000in}{7.700000in}}%
\pgfusepath{clip}%
\pgfsetrectcap%
\pgfsetroundjoin%
\pgfsetlinewidth{1.505625pt}%
\definecolor{currentstroke}{rgb}{0.631373,0.788235,0.956863}%
\pgfsetstrokecolor{currentstroke}%
\pgfsetstrokeopacity{0.800000}%
\pgfsetdash{}{0pt}%
\pgfpathmoveto{\pgfqpoint{6.339836in}{2.882881in}}%
\pgfpathlineto{\pgfqpoint{5.711094in}{3.773662in}}%
\pgfusepath{stroke}%
\end{pgfscope}%
\begin{pgfscope}%
\pgfpathrectangle{\pgfqpoint{0.481978in}{0.331635in}}{\pgfqpoint{9.300000in}{7.700000in}}%
\pgfusepath{clip}%
\pgfsetrectcap%
\pgfsetroundjoin%
\pgfsetlinewidth{1.505625pt}%
\definecolor{currentstroke}{rgb}{0.631373,0.788235,0.956863}%
\pgfsetstrokecolor{currentstroke}%
\pgfsetstrokeopacity{0.800000}%
\pgfsetdash{}{0pt}%
\pgfpathmoveto{\pgfqpoint{5.971524in}{2.077066in}}%
\pgfpathlineto{\pgfqpoint{5.711094in}{3.773662in}}%
\pgfusepath{stroke}%
\end{pgfscope}%
\begin{pgfscope}%
\pgfpathrectangle{\pgfqpoint{0.481978in}{0.331635in}}{\pgfqpoint{9.300000in}{7.700000in}}%
\pgfusepath{clip}%
\pgfsetrectcap%
\pgfsetroundjoin%
\pgfsetlinewidth{1.505625pt}%
\definecolor{currentstroke}{rgb}{0.631373,0.788235,0.956863}%
\pgfsetstrokecolor{currentstroke}%
\pgfsetstrokeopacity{0.800000}%
\pgfsetdash{}{0pt}%
\pgfpathmoveto{\pgfqpoint{3.002811in}{6.491469in}}%
\pgfpathlineto{\pgfqpoint{5.711094in}{3.773662in}}%
\pgfusepath{stroke}%
\end{pgfscope}%
\begin{pgfscope}%
\pgfpathrectangle{\pgfqpoint{0.481978in}{0.331635in}}{\pgfqpoint{9.300000in}{7.700000in}}%
\pgfusepath{clip}%
\pgfsetrectcap%
\pgfsetroundjoin%
\pgfsetlinewidth{1.505625pt}%
\definecolor{currentstroke}{rgb}{0.631373,0.788235,0.956863}%
\pgfsetstrokecolor{currentstroke}%
\pgfsetstrokeopacity{0.800000}%
\pgfsetdash{}{0pt}%
\pgfpathmoveto{\pgfqpoint{5.207999in}{6.891529in}}%
\pgfpathlineto{\pgfqpoint{5.711094in}{3.773662in}}%
\pgfusepath{stroke}%
\end{pgfscope}%
\begin{pgfscope}%
\pgfpathrectangle{\pgfqpoint{0.481978in}{0.331635in}}{\pgfqpoint{9.300000in}{7.700000in}}%
\pgfusepath{clip}%
\pgfsetrectcap%
\pgfsetroundjoin%
\pgfsetlinewidth{1.505625pt}%
\definecolor{currentstroke}{rgb}{0.631373,0.788235,0.956863}%
\pgfsetstrokecolor{currentstroke}%
\pgfsetstrokeopacity{0.800000}%
\pgfsetdash{}{0pt}%
\pgfpathmoveto{\pgfqpoint{4.854525in}{3.614784in}}%
\pgfpathlineto{\pgfqpoint{5.711094in}{3.773662in}}%
\pgfusepath{stroke}%
\end{pgfscope}%
\begin{pgfscope}%
\pgfpathrectangle{\pgfqpoint{0.481978in}{0.331635in}}{\pgfqpoint{9.300000in}{7.700000in}}%
\pgfusepath{clip}%
\pgfsetrectcap%
\pgfsetroundjoin%
\pgfsetlinewidth{1.505625pt}%
\definecolor{currentstroke}{rgb}{0.631373,0.788235,0.956863}%
\pgfsetstrokecolor{currentstroke}%
\pgfsetstrokeopacity{0.800000}%
\pgfsetdash{}{0pt}%
\pgfpathmoveto{\pgfqpoint{3.734095in}{1.504608in}}%
\pgfpathlineto{\pgfqpoint{5.711094in}{3.773662in}}%
\pgfusepath{stroke}%
\end{pgfscope}%
\begin{pgfscope}%
\pgfpathrectangle{\pgfqpoint{0.481978in}{0.331635in}}{\pgfqpoint{9.300000in}{7.700000in}}%
\pgfusepath{clip}%
\pgfsetrectcap%
\pgfsetroundjoin%
\pgfsetlinewidth{1.505625pt}%
\definecolor{currentstroke}{rgb}{0.631373,0.788235,0.956863}%
\pgfsetstrokecolor{currentstroke}%
\pgfsetstrokeopacity{0.800000}%
\pgfsetdash{}{0pt}%
\pgfpathmoveto{\pgfqpoint{3.327053in}{1.432012in}}%
\pgfpathlineto{\pgfqpoint{5.711094in}{3.773662in}}%
\pgfusepath{stroke}%
\end{pgfscope}%
\begin{pgfscope}%
\pgfpathrectangle{\pgfqpoint{0.481978in}{0.331635in}}{\pgfqpoint{9.300000in}{7.700000in}}%
\pgfusepath{clip}%
\pgfsetrectcap%
\pgfsetroundjoin%
\pgfsetlinewidth{1.505625pt}%
\definecolor{currentstroke}{rgb}{0.631373,0.788235,0.956863}%
\pgfsetstrokecolor{currentstroke}%
\pgfsetstrokeopacity{0.800000}%
\pgfsetdash{}{0pt}%
\pgfpathmoveto{\pgfqpoint{5.992810in}{4.059757in}}%
\pgfpathlineto{\pgfqpoint{5.711094in}{3.773662in}}%
\pgfusepath{stroke}%
\end{pgfscope}%
\begin{pgfscope}%
\pgfpathrectangle{\pgfqpoint{0.481978in}{0.331635in}}{\pgfqpoint{9.300000in}{7.700000in}}%
\pgfusepath{clip}%
\pgfsetrectcap%
\pgfsetroundjoin%
\pgfsetlinewidth{1.505625pt}%
\definecolor{currentstroke}{rgb}{0.631373,0.788235,0.956863}%
\pgfsetstrokecolor{currentstroke}%
\pgfsetstrokeopacity{0.800000}%
\pgfsetdash{}{0pt}%
\pgfpathmoveto{\pgfqpoint{5.648838in}{3.554140in}}%
\pgfpathlineto{\pgfqpoint{5.711094in}{3.773662in}}%
\pgfusepath{stroke}%
\end{pgfscope}%
\begin{pgfscope}%
\pgfpathrectangle{\pgfqpoint{0.481978in}{0.331635in}}{\pgfqpoint{9.300000in}{7.700000in}}%
\pgfusepath{clip}%
\pgfsetrectcap%
\pgfsetroundjoin%
\pgfsetlinewidth{1.505625pt}%
\definecolor{currentstroke}{rgb}{0.631373,0.788235,0.956863}%
\pgfsetstrokecolor{currentstroke}%
\pgfsetstrokeopacity{0.800000}%
\pgfsetdash{}{0pt}%
\pgfpathmoveto{\pgfqpoint{6.635439in}{5.393842in}}%
\pgfpathlineto{\pgfqpoint{5.711094in}{3.773662in}}%
\pgfusepath{stroke}%
\end{pgfscope}%
\begin{pgfscope}%
\pgfpathrectangle{\pgfqpoint{0.481978in}{0.331635in}}{\pgfqpoint{9.300000in}{7.700000in}}%
\pgfusepath{clip}%
\pgfsetrectcap%
\pgfsetroundjoin%
\pgfsetlinewidth{1.505625pt}%
\definecolor{currentstroke}{rgb}{0.631373,0.788235,0.956863}%
\pgfsetstrokecolor{currentstroke}%
\pgfsetstrokeopacity{0.800000}%
\pgfsetdash{}{0pt}%
\pgfpathmoveto{\pgfqpoint{6.399047in}{2.071266in}}%
\pgfpathlineto{\pgfqpoint{5.711094in}{3.773662in}}%
\pgfusepath{stroke}%
\end{pgfscope}%
\begin{pgfscope}%
\pgfpathrectangle{\pgfqpoint{0.481978in}{0.331635in}}{\pgfqpoint{9.300000in}{7.700000in}}%
\pgfusepath{clip}%
\pgfsetrectcap%
\pgfsetroundjoin%
\pgfsetlinewidth{1.505625pt}%
\definecolor{currentstroke}{rgb}{0.631373,0.788235,0.956863}%
\pgfsetstrokecolor{currentstroke}%
\pgfsetstrokeopacity{0.800000}%
\pgfsetdash{}{0pt}%
\pgfpathmoveto{\pgfqpoint{2.913660in}{6.864512in}}%
\pgfpathlineto{\pgfqpoint{5.711094in}{3.773662in}}%
\pgfusepath{stroke}%
\end{pgfscope}%
\begin{pgfscope}%
\pgfpathrectangle{\pgfqpoint{0.481978in}{0.331635in}}{\pgfqpoint{9.300000in}{7.700000in}}%
\pgfusepath{clip}%
\pgfsetrectcap%
\pgfsetroundjoin%
\pgfsetlinewidth{1.505625pt}%
\definecolor{currentstroke}{rgb}{0.631373,0.788235,0.956863}%
\pgfsetstrokecolor{currentstroke}%
\pgfsetstrokeopacity{0.800000}%
\pgfsetdash{}{0pt}%
\pgfpathmoveto{\pgfqpoint{4.618112in}{4.376225in}}%
\pgfpathlineto{\pgfqpoint{5.711094in}{3.773662in}}%
\pgfusepath{stroke}%
\end{pgfscope}%
\begin{pgfscope}%
\pgfpathrectangle{\pgfqpoint{0.481978in}{0.331635in}}{\pgfqpoint{9.300000in}{7.700000in}}%
\pgfusepath{clip}%
\pgfsetrectcap%
\pgfsetroundjoin%
\pgfsetlinewidth{1.505625pt}%
\definecolor{currentstroke}{rgb}{0.631373,0.788235,0.956863}%
\pgfsetstrokecolor{currentstroke}%
\pgfsetstrokeopacity{0.800000}%
\pgfsetdash{}{0pt}%
\pgfpathmoveto{\pgfqpoint{7.244029in}{2.471548in}}%
\pgfpathlineto{\pgfqpoint{5.711094in}{3.773662in}}%
\pgfusepath{stroke}%
\end{pgfscope}%
\begin{pgfscope}%
\pgfpathrectangle{\pgfqpoint{0.481978in}{0.331635in}}{\pgfqpoint{9.300000in}{7.700000in}}%
\pgfusepath{clip}%
\pgfsetrectcap%
\pgfsetroundjoin%
\pgfsetlinewidth{1.505625pt}%
\definecolor{currentstroke}{rgb}{0.631373,0.788235,0.956863}%
\pgfsetstrokecolor{currentstroke}%
\pgfsetstrokeopacity{0.800000}%
\pgfsetdash{}{0pt}%
\pgfpathmoveto{\pgfqpoint{6.714656in}{1.567114in}}%
\pgfpathlineto{\pgfqpoint{5.711094in}{3.773662in}}%
\pgfusepath{stroke}%
\end{pgfscope}%
\begin{pgfscope}%
\pgfpathrectangle{\pgfqpoint{0.481978in}{0.331635in}}{\pgfqpoint{9.300000in}{7.700000in}}%
\pgfusepath{clip}%
\pgfsetrectcap%
\pgfsetroundjoin%
\pgfsetlinewidth{1.505625pt}%
\definecolor{currentstroke}{rgb}{0.631373,0.788235,0.956863}%
\pgfsetstrokecolor{currentstroke}%
\pgfsetstrokeopacity{0.800000}%
\pgfsetdash{}{0pt}%
\pgfpathmoveto{\pgfqpoint{3.591337in}{1.309324in}}%
\pgfpathlineto{\pgfqpoint{5.711094in}{3.773662in}}%
\pgfusepath{stroke}%
\end{pgfscope}%
\begin{pgfscope}%
\pgfpathrectangle{\pgfqpoint{0.481978in}{0.331635in}}{\pgfqpoint{9.300000in}{7.700000in}}%
\pgfusepath{clip}%
\pgfsetrectcap%
\pgfsetroundjoin%
\pgfsetlinewidth{1.505625pt}%
\definecolor{currentstroke}{rgb}{0.631373,0.788235,0.956863}%
\pgfsetstrokecolor{currentstroke}%
\pgfsetstrokeopacity{0.800000}%
\pgfsetdash{}{0pt}%
\pgfpathmoveto{\pgfqpoint{8.476836in}{5.273389in}}%
\pgfpathlineto{\pgfqpoint{5.711094in}{3.773662in}}%
\pgfusepath{stroke}%
\end{pgfscope}%
\begin{pgfscope}%
\pgfpathrectangle{\pgfqpoint{0.481978in}{0.331635in}}{\pgfqpoint{9.300000in}{7.700000in}}%
\pgfusepath{clip}%
\pgfsetrectcap%
\pgfsetroundjoin%
\pgfsetlinewidth{1.505625pt}%
\definecolor{currentstroke}{rgb}{0.631373,0.788235,0.956863}%
\pgfsetstrokecolor{currentstroke}%
\pgfsetstrokeopacity{0.800000}%
\pgfsetdash{}{0pt}%
\pgfpathmoveto{\pgfqpoint{6.886735in}{1.453604in}}%
\pgfpathlineto{\pgfqpoint{5.711094in}{3.773662in}}%
\pgfusepath{stroke}%
\end{pgfscope}%
\begin{pgfscope}%
\pgfpathrectangle{\pgfqpoint{0.481978in}{0.331635in}}{\pgfqpoint{9.300000in}{7.700000in}}%
\pgfusepath{clip}%
\pgfsetrectcap%
\pgfsetroundjoin%
\pgfsetlinewidth{1.505625pt}%
\definecolor{currentstroke}{rgb}{0.631373,0.788235,0.956863}%
\pgfsetstrokecolor{currentstroke}%
\pgfsetstrokeopacity{0.800000}%
\pgfsetdash{}{0pt}%
\pgfpathmoveto{\pgfqpoint{5.148908in}{1.672237in}}%
\pgfpathlineto{\pgfqpoint{5.711094in}{3.773662in}}%
\pgfusepath{stroke}%
\end{pgfscope}%
\begin{pgfscope}%
\pgfpathrectangle{\pgfqpoint{0.481978in}{0.331635in}}{\pgfqpoint{9.300000in}{7.700000in}}%
\pgfusepath{clip}%
\pgfsetrectcap%
\pgfsetroundjoin%
\pgfsetlinewidth{1.505625pt}%
\definecolor{currentstroke}{rgb}{0.631373,0.788235,0.956863}%
\pgfsetstrokecolor{currentstroke}%
\pgfsetstrokeopacity{0.800000}%
\pgfsetdash{}{0pt}%
\pgfpathmoveto{\pgfqpoint{5.442394in}{6.854337in}}%
\pgfpathlineto{\pgfqpoint{5.711094in}{3.773662in}}%
\pgfusepath{stroke}%
\end{pgfscope}%
\begin{pgfscope}%
\pgfpathrectangle{\pgfqpoint{0.481978in}{0.331635in}}{\pgfqpoint{9.300000in}{7.700000in}}%
\pgfusepath{clip}%
\pgfsetrectcap%
\pgfsetroundjoin%
\pgfsetlinewidth{1.505625pt}%
\definecolor{currentstroke}{rgb}{0.631373,0.788235,0.956863}%
\pgfsetstrokecolor{currentstroke}%
\pgfsetstrokeopacity{0.800000}%
\pgfsetdash{}{0pt}%
\pgfpathmoveto{\pgfqpoint{6.133252in}{3.376659in}}%
\pgfpathlineto{\pgfqpoint{5.711094in}{3.773662in}}%
\pgfusepath{stroke}%
\end{pgfscope}%
\begin{pgfscope}%
\pgfpathrectangle{\pgfqpoint{0.481978in}{0.331635in}}{\pgfqpoint{9.300000in}{7.700000in}}%
\pgfusepath{clip}%
\pgfsetrectcap%
\pgfsetroundjoin%
\pgfsetlinewidth{1.505625pt}%
\definecolor{currentstroke}{rgb}{0.631373,0.788235,0.956863}%
\pgfsetstrokecolor{currentstroke}%
\pgfsetstrokeopacity{0.800000}%
\pgfsetdash{}{0pt}%
\pgfpathmoveto{\pgfqpoint{6.565040in}{5.067481in}}%
\pgfpathlineto{\pgfqpoint{5.711094in}{3.773662in}}%
\pgfusepath{stroke}%
\end{pgfscope}%
\begin{pgfscope}%
\pgfpathrectangle{\pgfqpoint{0.481978in}{0.331635in}}{\pgfqpoint{9.300000in}{7.700000in}}%
\pgfusepath{clip}%
\pgfsetrectcap%
\pgfsetroundjoin%
\pgfsetlinewidth{1.505625pt}%
\definecolor{currentstroke}{rgb}{0.631373,0.788235,0.956863}%
\pgfsetstrokecolor{currentstroke}%
\pgfsetstrokeopacity{0.800000}%
\pgfsetdash{}{0pt}%
\pgfpathmoveto{\pgfqpoint{6.488247in}{4.439323in}}%
\pgfpathlineto{\pgfqpoint{5.711094in}{3.773662in}}%
\pgfusepath{stroke}%
\end{pgfscope}%
\begin{pgfscope}%
\pgfpathrectangle{\pgfqpoint{0.481978in}{0.331635in}}{\pgfqpoint{9.300000in}{7.700000in}}%
\pgfusepath{clip}%
\pgfsetrectcap%
\pgfsetroundjoin%
\pgfsetlinewidth{1.505625pt}%
\definecolor{currentstroke}{rgb}{0.631373,0.788235,0.956863}%
\pgfsetstrokecolor{currentstroke}%
\pgfsetstrokeopacity{0.800000}%
\pgfsetdash{}{0pt}%
\pgfpathmoveto{\pgfqpoint{6.070621in}{5.046173in}}%
\pgfpathlineto{\pgfqpoint{5.711094in}{3.773662in}}%
\pgfusepath{stroke}%
\end{pgfscope}%
\begin{pgfscope}%
\pgfpathrectangle{\pgfqpoint{0.481978in}{0.331635in}}{\pgfqpoint{9.300000in}{7.700000in}}%
\pgfusepath{clip}%
\pgfsetrectcap%
\pgfsetroundjoin%
\pgfsetlinewidth{1.505625pt}%
\definecolor{currentstroke}{rgb}{0.631373,0.788235,0.956863}%
\pgfsetstrokecolor{currentstroke}%
\pgfsetstrokeopacity{0.800000}%
\pgfsetdash{}{0pt}%
\pgfpathmoveto{\pgfqpoint{4.466261in}{1.325503in}}%
\pgfpathlineto{\pgfqpoint{5.711094in}{3.773662in}}%
\pgfusepath{stroke}%
\end{pgfscope}%
\begin{pgfscope}%
\pgfpathrectangle{\pgfqpoint{0.481978in}{0.331635in}}{\pgfqpoint{9.300000in}{7.700000in}}%
\pgfusepath{clip}%
\pgfsetrectcap%
\pgfsetroundjoin%
\pgfsetlinewidth{1.505625pt}%
\definecolor{currentstroke}{rgb}{0.631373,0.788235,0.956863}%
\pgfsetstrokecolor{currentstroke}%
\pgfsetstrokeopacity{0.800000}%
\pgfsetdash{}{0pt}%
\pgfpathmoveto{\pgfqpoint{3.012102in}{4.653388in}}%
\pgfpathlineto{\pgfqpoint{5.711094in}{3.773662in}}%
\pgfusepath{stroke}%
\end{pgfscope}%
\begin{pgfscope}%
\pgfpathrectangle{\pgfqpoint{0.481978in}{0.331635in}}{\pgfqpoint{9.300000in}{7.700000in}}%
\pgfusepath{clip}%
\pgfsetrectcap%
\pgfsetroundjoin%
\pgfsetlinewidth{1.505625pt}%
\definecolor{currentstroke}{rgb}{0.631373,0.788235,0.956863}%
\pgfsetstrokecolor{currentstroke}%
\pgfsetstrokeopacity{0.800000}%
\pgfsetdash{}{0pt}%
\pgfpathmoveto{\pgfqpoint{6.293539in}{3.626500in}}%
\pgfpathlineto{\pgfqpoint{5.711094in}{3.773662in}}%
\pgfusepath{stroke}%
\end{pgfscope}%
\begin{pgfscope}%
\pgfpathrectangle{\pgfqpoint{0.481978in}{0.331635in}}{\pgfqpoint{9.300000in}{7.700000in}}%
\pgfusepath{clip}%
\pgfsetrectcap%
\pgfsetroundjoin%
\pgfsetlinewidth{1.505625pt}%
\definecolor{currentstroke}{rgb}{0.631373,0.788235,0.956863}%
\pgfsetstrokecolor{currentstroke}%
\pgfsetstrokeopacity{0.800000}%
\pgfsetdash{}{0pt}%
\pgfpathmoveto{\pgfqpoint{2.026106in}{1.649106in}}%
\pgfpathlineto{\pgfqpoint{5.711094in}{3.773662in}}%
\pgfusepath{stroke}%
\end{pgfscope}%
\begin{pgfscope}%
\pgfpathrectangle{\pgfqpoint{0.481978in}{0.331635in}}{\pgfqpoint{9.300000in}{7.700000in}}%
\pgfusepath{clip}%
\pgfsetrectcap%
\pgfsetroundjoin%
\pgfsetlinewidth{1.505625pt}%
\definecolor{currentstroke}{rgb}{1.000000,0.705882,0.509804}%
\pgfsetstrokecolor{currentstroke}%
\pgfsetstrokeopacity{0.800000}%
\pgfsetdash{}{0pt}%
\pgfpathmoveto{\pgfqpoint{4.850856in}{2.854692in}}%
\pgfpathlineto{\pgfqpoint{3.502867in}{4.072531in}}%
\pgfusepath{stroke}%
\end{pgfscope}%
\begin{pgfscope}%
\pgfpathrectangle{\pgfqpoint{0.481978in}{0.331635in}}{\pgfqpoint{9.300000in}{7.700000in}}%
\pgfusepath{clip}%
\pgfsetrectcap%
\pgfsetroundjoin%
\pgfsetlinewidth{1.505625pt}%
\definecolor{currentstroke}{rgb}{1.000000,0.705882,0.509804}%
\pgfsetstrokecolor{currentstroke}%
\pgfsetstrokeopacity{0.800000}%
\pgfsetdash{}{0pt}%
\pgfpathmoveto{\pgfqpoint{2.916721in}{3.120345in}}%
\pgfpathlineto{\pgfqpoint{3.502867in}{4.072531in}}%
\pgfusepath{stroke}%
\end{pgfscope}%
\begin{pgfscope}%
\pgfpathrectangle{\pgfqpoint{0.481978in}{0.331635in}}{\pgfqpoint{9.300000in}{7.700000in}}%
\pgfusepath{clip}%
\pgfsetrectcap%
\pgfsetroundjoin%
\pgfsetlinewidth{1.505625pt}%
\definecolor{currentstroke}{rgb}{1.000000,0.705882,0.509804}%
\pgfsetstrokecolor{currentstroke}%
\pgfsetstrokeopacity{0.800000}%
\pgfsetdash{}{0pt}%
\pgfpathmoveto{\pgfqpoint{3.594827in}{5.702842in}}%
\pgfpathlineto{\pgfqpoint{3.502867in}{4.072531in}}%
\pgfusepath{stroke}%
\end{pgfscope}%
\begin{pgfscope}%
\pgfpathrectangle{\pgfqpoint{0.481978in}{0.331635in}}{\pgfqpoint{9.300000in}{7.700000in}}%
\pgfusepath{clip}%
\pgfsetrectcap%
\pgfsetroundjoin%
\pgfsetlinewidth{1.505625pt}%
\definecolor{currentstroke}{rgb}{1.000000,0.705882,0.509804}%
\pgfsetstrokecolor{currentstroke}%
\pgfsetstrokeopacity{0.800000}%
\pgfsetdash{}{0pt}%
\pgfpathmoveto{\pgfqpoint{3.147191in}{3.809748in}}%
\pgfpathlineto{\pgfqpoint{3.502867in}{4.072531in}}%
\pgfusepath{stroke}%
\end{pgfscope}%
\begin{pgfscope}%
\pgfpathrectangle{\pgfqpoint{0.481978in}{0.331635in}}{\pgfqpoint{9.300000in}{7.700000in}}%
\pgfusepath{clip}%
\pgfsetrectcap%
\pgfsetroundjoin%
\pgfsetlinewidth{1.505625pt}%
\definecolor{currentstroke}{rgb}{1.000000,0.705882,0.509804}%
\pgfsetstrokecolor{currentstroke}%
\pgfsetstrokeopacity{0.800000}%
\pgfsetdash{}{0pt}%
\pgfpathmoveto{\pgfqpoint{3.148403in}{2.315312in}}%
\pgfpathlineto{\pgfqpoint{3.502867in}{4.072531in}}%
\pgfusepath{stroke}%
\end{pgfscope}%
\begin{pgfscope}%
\pgfpathrectangle{\pgfqpoint{0.481978in}{0.331635in}}{\pgfqpoint{9.300000in}{7.700000in}}%
\pgfusepath{clip}%
\pgfsetrectcap%
\pgfsetroundjoin%
\pgfsetlinewidth{1.505625pt}%
\definecolor{currentstroke}{rgb}{1.000000,0.705882,0.509804}%
\pgfsetstrokecolor{currentstroke}%
\pgfsetstrokeopacity{0.800000}%
\pgfsetdash{}{0pt}%
\pgfpathmoveto{\pgfqpoint{3.951193in}{0.954754in}}%
\pgfpathlineto{\pgfqpoint{3.502867in}{4.072531in}}%
\pgfusepath{stroke}%
\end{pgfscope}%
\begin{pgfscope}%
\pgfpathrectangle{\pgfqpoint{0.481978in}{0.331635in}}{\pgfqpoint{9.300000in}{7.700000in}}%
\pgfusepath{clip}%
\pgfsetrectcap%
\pgfsetroundjoin%
\pgfsetlinewidth{1.505625pt}%
\definecolor{currentstroke}{rgb}{1.000000,0.705882,0.509804}%
\pgfsetstrokecolor{currentstroke}%
\pgfsetstrokeopacity{0.800000}%
\pgfsetdash{}{0pt}%
\pgfpathmoveto{\pgfqpoint{5.646419in}{4.320008in}}%
\pgfpathlineto{\pgfqpoint{3.502867in}{4.072531in}}%
\pgfusepath{stroke}%
\end{pgfscope}%
\begin{pgfscope}%
\pgfpathrectangle{\pgfqpoint{0.481978in}{0.331635in}}{\pgfqpoint{9.300000in}{7.700000in}}%
\pgfusepath{clip}%
\pgfsetrectcap%
\pgfsetroundjoin%
\pgfsetlinewidth{1.505625pt}%
\definecolor{currentstroke}{rgb}{1.000000,0.705882,0.509804}%
\pgfsetstrokecolor{currentstroke}%
\pgfsetstrokeopacity{0.800000}%
\pgfsetdash{}{0pt}%
\pgfpathmoveto{\pgfqpoint{2.847205in}{4.348681in}}%
\pgfpathlineto{\pgfqpoint{3.502867in}{4.072531in}}%
\pgfusepath{stroke}%
\end{pgfscope}%
\begin{pgfscope}%
\pgfpathrectangle{\pgfqpoint{0.481978in}{0.331635in}}{\pgfqpoint{9.300000in}{7.700000in}}%
\pgfusepath{clip}%
\pgfsetrectcap%
\pgfsetroundjoin%
\pgfsetlinewidth{1.505625pt}%
\definecolor{currentstroke}{rgb}{1.000000,0.705882,0.509804}%
\pgfsetstrokecolor{currentstroke}%
\pgfsetstrokeopacity{0.800000}%
\pgfsetdash{}{0pt}%
\pgfpathmoveto{\pgfqpoint{3.926575in}{2.377543in}}%
\pgfpathlineto{\pgfqpoint{3.502867in}{4.072531in}}%
\pgfusepath{stroke}%
\end{pgfscope}%
\begin{pgfscope}%
\pgfpathrectangle{\pgfqpoint{0.481978in}{0.331635in}}{\pgfqpoint{9.300000in}{7.700000in}}%
\pgfusepath{clip}%
\pgfsetrectcap%
\pgfsetroundjoin%
\pgfsetlinewidth{1.505625pt}%
\definecolor{currentstroke}{rgb}{1.000000,0.705882,0.509804}%
\pgfsetstrokecolor{currentstroke}%
\pgfsetstrokeopacity{0.800000}%
\pgfsetdash{}{0pt}%
\pgfpathmoveto{\pgfqpoint{2.406478in}{5.903770in}}%
\pgfpathlineto{\pgfqpoint{3.502867in}{4.072531in}}%
\pgfusepath{stroke}%
\end{pgfscope}%
\begin{pgfscope}%
\pgfpathrectangle{\pgfqpoint{0.481978in}{0.331635in}}{\pgfqpoint{9.300000in}{7.700000in}}%
\pgfusepath{clip}%
\pgfsetrectcap%
\pgfsetroundjoin%
\pgfsetlinewidth{1.505625pt}%
\definecolor{currentstroke}{rgb}{1.000000,0.705882,0.509804}%
\pgfsetstrokecolor{currentstroke}%
\pgfsetstrokeopacity{0.800000}%
\pgfsetdash{}{0pt}%
\pgfpathmoveto{\pgfqpoint{4.065253in}{2.372474in}}%
\pgfpathlineto{\pgfqpoint{3.502867in}{4.072531in}}%
\pgfusepath{stroke}%
\end{pgfscope}%
\begin{pgfscope}%
\pgfpathrectangle{\pgfqpoint{0.481978in}{0.331635in}}{\pgfqpoint{9.300000in}{7.700000in}}%
\pgfusepath{clip}%
\pgfsetrectcap%
\pgfsetroundjoin%
\pgfsetlinewidth{1.505625pt}%
\definecolor{currentstroke}{rgb}{1.000000,0.705882,0.509804}%
\pgfsetstrokecolor{currentstroke}%
\pgfsetstrokeopacity{0.800000}%
\pgfsetdash{}{0pt}%
\pgfpathmoveto{\pgfqpoint{4.470303in}{5.206116in}}%
\pgfpathlineto{\pgfqpoint{3.502867in}{4.072531in}}%
\pgfusepath{stroke}%
\end{pgfscope}%
\begin{pgfscope}%
\pgfpathrectangle{\pgfqpoint{0.481978in}{0.331635in}}{\pgfqpoint{9.300000in}{7.700000in}}%
\pgfusepath{clip}%
\pgfsetrectcap%
\pgfsetroundjoin%
\pgfsetlinewidth{1.505625pt}%
\definecolor{currentstroke}{rgb}{1.000000,0.705882,0.509804}%
\pgfsetstrokecolor{currentstroke}%
\pgfsetstrokeopacity{0.800000}%
\pgfsetdash{}{0pt}%
\pgfpathmoveto{\pgfqpoint{2.649493in}{4.102821in}}%
\pgfpathlineto{\pgfqpoint{3.502867in}{4.072531in}}%
\pgfusepath{stroke}%
\end{pgfscope}%
\begin{pgfscope}%
\pgfpathrectangle{\pgfqpoint{0.481978in}{0.331635in}}{\pgfqpoint{9.300000in}{7.700000in}}%
\pgfusepath{clip}%
\pgfsetrectcap%
\pgfsetroundjoin%
\pgfsetlinewidth{1.505625pt}%
\definecolor{currentstroke}{rgb}{1.000000,0.705882,0.509804}%
\pgfsetstrokecolor{currentstroke}%
\pgfsetstrokeopacity{0.800000}%
\pgfsetdash{}{0pt}%
\pgfpathmoveto{\pgfqpoint{2.136818in}{3.314846in}}%
\pgfpathlineto{\pgfqpoint{3.502867in}{4.072531in}}%
\pgfusepath{stroke}%
\end{pgfscope}%
\begin{pgfscope}%
\pgfpathrectangle{\pgfqpoint{0.481978in}{0.331635in}}{\pgfqpoint{9.300000in}{7.700000in}}%
\pgfusepath{clip}%
\pgfsetrectcap%
\pgfsetroundjoin%
\pgfsetlinewidth{1.505625pt}%
\definecolor{currentstroke}{rgb}{1.000000,0.705882,0.509804}%
\pgfsetstrokecolor{currentstroke}%
\pgfsetstrokeopacity{0.800000}%
\pgfsetdash{}{0pt}%
\pgfpathmoveto{\pgfqpoint{3.304744in}{3.445980in}}%
\pgfpathlineto{\pgfqpoint{3.502867in}{4.072531in}}%
\pgfusepath{stroke}%
\end{pgfscope}%
\begin{pgfscope}%
\pgfpathrectangle{\pgfqpoint{0.481978in}{0.331635in}}{\pgfqpoint{9.300000in}{7.700000in}}%
\pgfusepath{clip}%
\pgfsetrectcap%
\pgfsetroundjoin%
\pgfsetlinewidth{1.505625pt}%
\definecolor{currentstroke}{rgb}{1.000000,0.705882,0.509804}%
\pgfsetstrokecolor{currentstroke}%
\pgfsetstrokeopacity{0.800000}%
\pgfsetdash{}{0pt}%
\pgfpathmoveto{\pgfqpoint{1.284269in}{4.746461in}}%
\pgfpathlineto{\pgfqpoint{3.502867in}{4.072531in}}%
\pgfusepath{stroke}%
\end{pgfscope}%
\begin{pgfscope}%
\pgfpathrectangle{\pgfqpoint{0.481978in}{0.331635in}}{\pgfqpoint{9.300000in}{7.700000in}}%
\pgfusepath{clip}%
\pgfsetrectcap%
\pgfsetroundjoin%
\pgfsetlinewidth{1.505625pt}%
\definecolor{currentstroke}{rgb}{1.000000,0.705882,0.509804}%
\pgfsetstrokecolor{currentstroke}%
\pgfsetstrokeopacity{0.800000}%
\pgfsetdash{}{0pt}%
\pgfpathmoveto{\pgfqpoint{3.921640in}{6.537801in}}%
\pgfpathlineto{\pgfqpoint{3.502867in}{4.072531in}}%
\pgfusepath{stroke}%
\end{pgfscope}%
\begin{pgfscope}%
\pgfpathrectangle{\pgfqpoint{0.481978in}{0.331635in}}{\pgfqpoint{9.300000in}{7.700000in}}%
\pgfusepath{clip}%
\pgfsetrectcap%
\pgfsetroundjoin%
\pgfsetlinewidth{1.505625pt}%
\definecolor{currentstroke}{rgb}{1.000000,0.705882,0.509804}%
\pgfsetstrokecolor{currentstroke}%
\pgfsetstrokeopacity{0.800000}%
\pgfsetdash{}{0pt}%
\pgfpathmoveto{\pgfqpoint{4.551669in}{3.889138in}}%
\pgfpathlineto{\pgfqpoint{3.502867in}{4.072531in}}%
\pgfusepath{stroke}%
\end{pgfscope}%
\begin{pgfscope}%
\pgfpathrectangle{\pgfqpoint{0.481978in}{0.331635in}}{\pgfqpoint{9.300000in}{7.700000in}}%
\pgfusepath{clip}%
\pgfsetrectcap%
\pgfsetroundjoin%
\pgfsetlinewidth{1.505625pt}%
\definecolor{currentstroke}{rgb}{1.000000,0.705882,0.509804}%
\pgfsetstrokecolor{currentstroke}%
\pgfsetstrokeopacity{0.800000}%
\pgfsetdash{}{0pt}%
\pgfpathmoveto{\pgfqpoint{2.003717in}{3.814665in}}%
\pgfpathlineto{\pgfqpoint{3.502867in}{4.072531in}}%
\pgfusepath{stroke}%
\end{pgfscope}%
\begin{pgfscope}%
\pgfpathrectangle{\pgfqpoint{0.481978in}{0.331635in}}{\pgfqpoint{9.300000in}{7.700000in}}%
\pgfusepath{clip}%
\pgfsetrectcap%
\pgfsetroundjoin%
\pgfsetlinewidth{1.505625pt}%
\definecolor{currentstroke}{rgb}{1.000000,0.705882,0.509804}%
\pgfsetstrokecolor{currentstroke}%
\pgfsetstrokeopacity{0.800000}%
\pgfsetdash{}{0pt}%
\pgfpathmoveto{\pgfqpoint{2.821317in}{3.728489in}}%
\pgfpathlineto{\pgfqpoint{3.502867in}{4.072531in}}%
\pgfusepath{stroke}%
\end{pgfscope}%
\begin{pgfscope}%
\pgfpathrectangle{\pgfqpoint{0.481978in}{0.331635in}}{\pgfqpoint{9.300000in}{7.700000in}}%
\pgfusepath{clip}%
\pgfsetrectcap%
\pgfsetroundjoin%
\pgfsetlinewidth{1.505625pt}%
\definecolor{currentstroke}{rgb}{1.000000,0.705882,0.509804}%
\pgfsetstrokecolor{currentstroke}%
\pgfsetstrokeopacity{0.800000}%
\pgfsetdash{}{0pt}%
\pgfpathmoveto{\pgfqpoint{2.335285in}{3.880841in}}%
\pgfpathlineto{\pgfqpoint{3.502867in}{4.072531in}}%
\pgfusepath{stroke}%
\end{pgfscope}%
\begin{pgfscope}%
\pgfpathrectangle{\pgfqpoint{0.481978in}{0.331635in}}{\pgfqpoint{9.300000in}{7.700000in}}%
\pgfusepath{clip}%
\pgfsetrectcap%
\pgfsetroundjoin%
\pgfsetlinewidth{1.505625pt}%
\definecolor{currentstroke}{rgb}{1.000000,0.705882,0.509804}%
\pgfsetstrokecolor{currentstroke}%
\pgfsetstrokeopacity{0.800000}%
\pgfsetdash{}{0pt}%
\pgfpathmoveto{\pgfqpoint{4.485108in}{3.865983in}}%
\pgfpathlineto{\pgfqpoint{3.502867in}{4.072531in}}%
\pgfusepath{stroke}%
\end{pgfscope}%
\begin{pgfscope}%
\pgfpathrectangle{\pgfqpoint{0.481978in}{0.331635in}}{\pgfqpoint{9.300000in}{7.700000in}}%
\pgfusepath{clip}%
\pgfsetrectcap%
\pgfsetroundjoin%
\pgfsetlinewidth{1.505625pt}%
\definecolor{currentstroke}{rgb}{1.000000,0.705882,0.509804}%
\pgfsetstrokecolor{currentstroke}%
\pgfsetstrokeopacity{0.800000}%
\pgfsetdash{}{0pt}%
\pgfpathmoveto{\pgfqpoint{1.703134in}{4.289761in}}%
\pgfpathlineto{\pgfqpoint{3.502867in}{4.072531in}}%
\pgfusepath{stroke}%
\end{pgfscope}%
\begin{pgfscope}%
\pgfpathrectangle{\pgfqpoint{0.481978in}{0.331635in}}{\pgfqpoint{9.300000in}{7.700000in}}%
\pgfusepath{clip}%
\pgfsetrectcap%
\pgfsetroundjoin%
\pgfsetlinewidth{1.505625pt}%
\definecolor{currentstroke}{rgb}{1.000000,0.705882,0.509804}%
\pgfsetstrokecolor{currentstroke}%
\pgfsetstrokeopacity{0.800000}%
\pgfsetdash{}{0pt}%
\pgfpathmoveto{\pgfqpoint{4.082596in}{4.780623in}}%
\pgfpathlineto{\pgfqpoint{3.502867in}{4.072531in}}%
\pgfusepath{stroke}%
\end{pgfscope}%
\begin{pgfscope}%
\pgfpathrectangle{\pgfqpoint{0.481978in}{0.331635in}}{\pgfqpoint{9.300000in}{7.700000in}}%
\pgfusepath{clip}%
\pgfsetrectcap%
\pgfsetroundjoin%
\pgfsetlinewidth{1.505625pt}%
\definecolor{currentstroke}{rgb}{1.000000,0.705882,0.509804}%
\pgfsetstrokecolor{currentstroke}%
\pgfsetstrokeopacity{0.800000}%
\pgfsetdash{}{0pt}%
\pgfpathmoveto{\pgfqpoint{3.376694in}{3.830492in}}%
\pgfpathlineto{\pgfqpoint{3.502867in}{4.072531in}}%
\pgfusepath{stroke}%
\end{pgfscope}%
\begin{pgfscope}%
\pgfpathrectangle{\pgfqpoint{0.481978in}{0.331635in}}{\pgfqpoint{9.300000in}{7.700000in}}%
\pgfusepath{clip}%
\pgfsetrectcap%
\pgfsetroundjoin%
\pgfsetlinewidth{1.505625pt}%
\definecolor{currentstroke}{rgb}{1.000000,0.705882,0.509804}%
\pgfsetstrokecolor{currentstroke}%
\pgfsetstrokeopacity{0.800000}%
\pgfsetdash{}{0pt}%
\pgfpathmoveto{\pgfqpoint{4.094855in}{5.282243in}}%
\pgfpathlineto{\pgfqpoint{3.502867in}{4.072531in}}%
\pgfusepath{stroke}%
\end{pgfscope}%
\begin{pgfscope}%
\pgfpathrectangle{\pgfqpoint{0.481978in}{0.331635in}}{\pgfqpoint{9.300000in}{7.700000in}}%
\pgfusepath{clip}%
\pgfsetrectcap%
\pgfsetroundjoin%
\pgfsetlinewidth{1.505625pt}%
\definecolor{currentstroke}{rgb}{1.000000,0.705882,0.509804}%
\pgfsetstrokecolor{currentstroke}%
\pgfsetstrokeopacity{0.800000}%
\pgfsetdash{}{0pt}%
\pgfpathmoveto{\pgfqpoint{4.055791in}{5.348899in}}%
\pgfpathlineto{\pgfqpoint{3.502867in}{4.072531in}}%
\pgfusepath{stroke}%
\end{pgfscope}%
\begin{pgfscope}%
\pgfpathrectangle{\pgfqpoint{0.481978in}{0.331635in}}{\pgfqpoint{9.300000in}{7.700000in}}%
\pgfusepath{clip}%
\pgfsetrectcap%
\pgfsetroundjoin%
\pgfsetlinewidth{1.505625pt}%
\definecolor{currentstroke}{rgb}{1.000000,0.705882,0.509804}%
\pgfsetstrokecolor{currentstroke}%
\pgfsetstrokeopacity{0.800000}%
\pgfsetdash{}{0pt}%
\pgfpathmoveto{\pgfqpoint{4.975557in}{3.385448in}}%
\pgfpathlineto{\pgfqpoint{3.502867in}{4.072531in}}%
\pgfusepath{stroke}%
\end{pgfscope}%
\begin{pgfscope}%
\pgfpathrectangle{\pgfqpoint{0.481978in}{0.331635in}}{\pgfqpoint{9.300000in}{7.700000in}}%
\pgfusepath{clip}%
\pgfsetrectcap%
\pgfsetroundjoin%
\pgfsetlinewidth{1.505625pt}%
\definecolor{currentstroke}{rgb}{1.000000,0.705882,0.509804}%
\pgfsetstrokecolor{currentstroke}%
\pgfsetstrokeopacity{0.800000}%
\pgfsetdash{}{0pt}%
\pgfpathmoveto{\pgfqpoint{2.986906in}{3.719025in}}%
\pgfpathlineto{\pgfqpoint{3.502867in}{4.072531in}}%
\pgfusepath{stroke}%
\end{pgfscope}%
\begin{pgfscope}%
\pgfpathrectangle{\pgfqpoint{0.481978in}{0.331635in}}{\pgfqpoint{9.300000in}{7.700000in}}%
\pgfusepath{clip}%
\pgfsetrectcap%
\pgfsetroundjoin%
\pgfsetlinewidth{1.505625pt}%
\definecolor{currentstroke}{rgb}{1.000000,0.705882,0.509804}%
\pgfsetstrokecolor{currentstroke}%
\pgfsetstrokeopacity{0.800000}%
\pgfsetdash{}{0pt}%
\pgfpathmoveto{\pgfqpoint{3.977136in}{2.715901in}}%
\pgfpathlineto{\pgfqpoint{3.502867in}{4.072531in}}%
\pgfusepath{stroke}%
\end{pgfscope}%
\begin{pgfscope}%
\pgfpathrectangle{\pgfqpoint{0.481978in}{0.331635in}}{\pgfqpoint{9.300000in}{7.700000in}}%
\pgfusepath{clip}%
\pgfsetrectcap%
\pgfsetroundjoin%
\pgfsetlinewidth{1.505625pt}%
\definecolor{currentstroke}{rgb}{1.000000,0.705882,0.509804}%
\pgfsetstrokecolor{currentstroke}%
\pgfsetstrokeopacity{0.800000}%
\pgfsetdash{}{0pt}%
\pgfpathmoveto{\pgfqpoint{3.838833in}{6.961369in}}%
\pgfpathlineto{\pgfqpoint{3.502867in}{4.072531in}}%
\pgfusepath{stroke}%
\end{pgfscope}%
\begin{pgfscope}%
\pgfpathrectangle{\pgfqpoint{0.481978in}{0.331635in}}{\pgfqpoint{9.300000in}{7.700000in}}%
\pgfusepath{clip}%
\pgfsetrectcap%
\pgfsetroundjoin%
\pgfsetlinewidth{1.505625pt}%
\definecolor{currentstroke}{rgb}{1.000000,0.705882,0.509804}%
\pgfsetstrokecolor{currentstroke}%
\pgfsetstrokeopacity{0.800000}%
\pgfsetdash{}{0pt}%
\pgfpathmoveto{\pgfqpoint{2.868567in}{3.813763in}}%
\pgfpathlineto{\pgfqpoint{3.502867in}{4.072531in}}%
\pgfusepath{stroke}%
\end{pgfscope}%
\begin{pgfscope}%
\pgfpathrectangle{\pgfqpoint{0.481978in}{0.331635in}}{\pgfqpoint{9.300000in}{7.700000in}}%
\pgfusepath{clip}%
\pgfsetrectcap%
\pgfsetroundjoin%
\pgfsetlinewidth{1.505625pt}%
\definecolor{currentstroke}{rgb}{1.000000,0.705882,0.509804}%
\pgfsetstrokecolor{currentstroke}%
\pgfsetstrokeopacity{0.800000}%
\pgfsetdash{}{0pt}%
\pgfpathmoveto{\pgfqpoint{5.264259in}{4.933464in}}%
\pgfpathlineto{\pgfqpoint{3.502867in}{4.072531in}}%
\pgfusepath{stroke}%
\end{pgfscope}%
\begin{pgfscope}%
\pgfpathrectangle{\pgfqpoint{0.481978in}{0.331635in}}{\pgfqpoint{9.300000in}{7.700000in}}%
\pgfusepath{clip}%
\pgfsetrectcap%
\pgfsetroundjoin%
\pgfsetlinewidth{1.505625pt}%
\definecolor{currentstroke}{rgb}{1.000000,0.705882,0.509804}%
\pgfsetstrokecolor{currentstroke}%
\pgfsetstrokeopacity{0.800000}%
\pgfsetdash{}{0pt}%
\pgfpathmoveto{\pgfqpoint{3.655520in}{5.762264in}}%
\pgfpathlineto{\pgfqpoint{3.502867in}{4.072531in}}%
\pgfusepath{stroke}%
\end{pgfscope}%
\begin{pgfscope}%
\pgfpathrectangle{\pgfqpoint{0.481978in}{0.331635in}}{\pgfqpoint{9.300000in}{7.700000in}}%
\pgfusepath{clip}%
\pgfsetrectcap%
\pgfsetroundjoin%
\pgfsetlinewidth{1.505625pt}%
\definecolor{currentstroke}{rgb}{1.000000,0.705882,0.509804}%
\pgfsetstrokecolor{currentstroke}%
\pgfsetstrokeopacity{0.800000}%
\pgfsetdash{}{0pt}%
\pgfpathmoveto{\pgfqpoint{1.251672in}{4.064003in}}%
\pgfpathlineto{\pgfqpoint{3.502867in}{4.072531in}}%
\pgfusepath{stroke}%
\end{pgfscope}%
\begin{pgfscope}%
\pgfpathrectangle{\pgfqpoint{0.481978in}{0.331635in}}{\pgfqpoint{9.300000in}{7.700000in}}%
\pgfusepath{clip}%
\pgfsetrectcap%
\pgfsetroundjoin%
\pgfsetlinewidth{1.505625pt}%
\definecolor{currentstroke}{rgb}{1.000000,0.705882,0.509804}%
\pgfsetstrokecolor{currentstroke}%
\pgfsetstrokeopacity{0.800000}%
\pgfsetdash{}{0pt}%
\pgfpathmoveto{\pgfqpoint{1.875951in}{4.198061in}}%
\pgfpathlineto{\pgfqpoint{3.502867in}{4.072531in}}%
\pgfusepath{stroke}%
\end{pgfscope}%
\begin{pgfscope}%
\pgfpathrectangle{\pgfqpoint{0.481978in}{0.331635in}}{\pgfqpoint{9.300000in}{7.700000in}}%
\pgfusepath{clip}%
\pgfsetrectcap%
\pgfsetroundjoin%
\pgfsetlinewidth{1.505625pt}%
\definecolor{currentstroke}{rgb}{1.000000,0.705882,0.509804}%
\pgfsetstrokecolor{currentstroke}%
\pgfsetstrokeopacity{0.800000}%
\pgfsetdash{}{0pt}%
\pgfpathmoveto{\pgfqpoint{2.496147in}{3.990985in}}%
\pgfpathlineto{\pgfqpoint{3.502867in}{4.072531in}}%
\pgfusepath{stroke}%
\end{pgfscope}%
\begin{pgfscope}%
\pgfpathrectangle{\pgfqpoint{0.481978in}{0.331635in}}{\pgfqpoint{9.300000in}{7.700000in}}%
\pgfusepath{clip}%
\pgfsetrectcap%
\pgfsetroundjoin%
\pgfsetlinewidth{1.505625pt}%
\definecolor{currentstroke}{rgb}{1.000000,0.705882,0.509804}%
\pgfsetstrokecolor{currentstroke}%
\pgfsetstrokeopacity{0.800000}%
\pgfsetdash{}{0pt}%
\pgfpathmoveto{\pgfqpoint{3.452404in}{6.280804in}}%
\pgfpathlineto{\pgfqpoint{3.502867in}{4.072531in}}%
\pgfusepath{stroke}%
\end{pgfscope}%
\begin{pgfscope}%
\pgfpathrectangle{\pgfqpoint{0.481978in}{0.331635in}}{\pgfqpoint{9.300000in}{7.700000in}}%
\pgfusepath{clip}%
\pgfsetrectcap%
\pgfsetroundjoin%
\pgfsetlinewidth{1.505625pt}%
\definecolor{currentstroke}{rgb}{1.000000,0.705882,0.509804}%
\pgfsetstrokecolor{currentstroke}%
\pgfsetstrokeopacity{0.800000}%
\pgfsetdash{}{0pt}%
\pgfpathmoveto{\pgfqpoint{1.475053in}{4.585541in}}%
\pgfpathlineto{\pgfqpoint{3.502867in}{4.072531in}}%
\pgfusepath{stroke}%
\end{pgfscope}%
\begin{pgfscope}%
\pgfpathrectangle{\pgfqpoint{0.481978in}{0.331635in}}{\pgfqpoint{9.300000in}{7.700000in}}%
\pgfusepath{clip}%
\pgfsetrectcap%
\pgfsetroundjoin%
\pgfsetlinewidth{1.505625pt}%
\definecolor{currentstroke}{rgb}{1.000000,0.705882,0.509804}%
\pgfsetstrokecolor{currentstroke}%
\pgfsetstrokeopacity{0.800000}%
\pgfsetdash{}{0pt}%
\pgfpathmoveto{\pgfqpoint{2.980194in}{4.503252in}}%
\pgfpathlineto{\pgfqpoint{3.502867in}{4.072531in}}%
\pgfusepath{stroke}%
\end{pgfscope}%
\begin{pgfscope}%
\pgfpathrectangle{\pgfqpoint{0.481978in}{0.331635in}}{\pgfqpoint{9.300000in}{7.700000in}}%
\pgfusepath{clip}%
\pgfsetrectcap%
\pgfsetroundjoin%
\pgfsetlinewidth{1.505625pt}%
\definecolor{currentstroke}{rgb}{1.000000,0.705882,0.509804}%
\pgfsetstrokecolor{currentstroke}%
\pgfsetstrokeopacity{0.800000}%
\pgfsetdash{}{0pt}%
\pgfpathmoveto{\pgfqpoint{4.618269in}{4.700871in}}%
\pgfpathlineto{\pgfqpoint{3.502867in}{4.072531in}}%
\pgfusepath{stroke}%
\end{pgfscope}%
\begin{pgfscope}%
\pgfpathrectangle{\pgfqpoint{0.481978in}{0.331635in}}{\pgfqpoint{9.300000in}{7.700000in}}%
\pgfusepath{clip}%
\pgfsetrectcap%
\pgfsetroundjoin%
\pgfsetlinewidth{1.505625pt}%
\definecolor{currentstroke}{rgb}{1.000000,0.705882,0.509804}%
\pgfsetstrokecolor{currentstroke}%
\pgfsetstrokeopacity{0.800000}%
\pgfsetdash{}{0pt}%
\pgfpathmoveto{\pgfqpoint{3.693817in}{4.084468in}}%
\pgfpathlineto{\pgfqpoint{3.502867in}{4.072531in}}%
\pgfusepath{stroke}%
\end{pgfscope}%
\begin{pgfscope}%
\pgfpathrectangle{\pgfqpoint{0.481978in}{0.331635in}}{\pgfqpoint{9.300000in}{7.700000in}}%
\pgfusepath{clip}%
\pgfsetrectcap%
\pgfsetroundjoin%
\pgfsetlinewidth{1.505625pt}%
\definecolor{currentstroke}{rgb}{1.000000,0.705882,0.509804}%
\pgfsetstrokecolor{currentstroke}%
\pgfsetstrokeopacity{0.800000}%
\pgfsetdash{}{0pt}%
\pgfpathmoveto{\pgfqpoint{4.823817in}{3.134124in}}%
\pgfpathlineto{\pgfqpoint{3.502867in}{4.072531in}}%
\pgfusepath{stroke}%
\end{pgfscope}%
\begin{pgfscope}%
\pgfpathrectangle{\pgfqpoint{0.481978in}{0.331635in}}{\pgfqpoint{9.300000in}{7.700000in}}%
\pgfusepath{clip}%
\pgfsetrectcap%
\pgfsetroundjoin%
\pgfsetlinewidth{1.505625pt}%
\definecolor{currentstroke}{rgb}{1.000000,0.705882,0.509804}%
\pgfsetstrokecolor{currentstroke}%
\pgfsetstrokeopacity{0.800000}%
\pgfsetdash{}{0pt}%
\pgfpathmoveto{\pgfqpoint{2.627728in}{3.915305in}}%
\pgfpathlineto{\pgfqpoint{3.502867in}{4.072531in}}%
\pgfusepath{stroke}%
\end{pgfscope}%
\begin{pgfscope}%
\pgfpathrectangle{\pgfqpoint{0.481978in}{0.331635in}}{\pgfqpoint{9.300000in}{7.700000in}}%
\pgfusepath{clip}%
\pgfsetrectcap%
\pgfsetroundjoin%
\pgfsetlinewidth{1.505625pt}%
\definecolor{currentstroke}{rgb}{1.000000,0.705882,0.509804}%
\pgfsetstrokecolor{currentstroke}%
\pgfsetstrokeopacity{0.800000}%
\pgfsetdash{}{0pt}%
\pgfpathmoveto{\pgfqpoint{2.782878in}{3.370487in}}%
\pgfpathlineto{\pgfqpoint{3.502867in}{4.072531in}}%
\pgfusepath{stroke}%
\end{pgfscope}%
\begin{pgfscope}%
\pgfpathrectangle{\pgfqpoint{0.481978in}{0.331635in}}{\pgfqpoint{9.300000in}{7.700000in}}%
\pgfusepath{clip}%
\pgfsetrectcap%
\pgfsetroundjoin%
\pgfsetlinewidth{1.505625pt}%
\definecolor{currentstroke}{rgb}{1.000000,0.705882,0.509804}%
\pgfsetstrokecolor{currentstroke}%
\pgfsetstrokeopacity{0.800000}%
\pgfsetdash{}{0pt}%
\pgfpathmoveto{\pgfqpoint{3.755167in}{4.881969in}}%
\pgfpathlineto{\pgfqpoint{3.502867in}{4.072531in}}%
\pgfusepath{stroke}%
\end{pgfscope}%
\begin{pgfscope}%
\pgfpathrectangle{\pgfqpoint{0.481978in}{0.331635in}}{\pgfqpoint{9.300000in}{7.700000in}}%
\pgfusepath{clip}%
\pgfsetrectcap%
\pgfsetroundjoin%
\pgfsetlinewidth{1.505625pt}%
\definecolor{currentstroke}{rgb}{1.000000,0.705882,0.509804}%
\pgfsetstrokecolor{currentstroke}%
\pgfsetstrokeopacity{0.800000}%
\pgfsetdash{}{0pt}%
\pgfpathmoveto{\pgfqpoint{4.616665in}{2.897911in}}%
\pgfpathlineto{\pgfqpoint{3.502867in}{4.072531in}}%
\pgfusepath{stroke}%
\end{pgfscope}%
\begin{pgfscope}%
\pgfpathrectangle{\pgfqpoint{0.481978in}{0.331635in}}{\pgfqpoint{9.300000in}{7.700000in}}%
\pgfusepath{clip}%
\pgfsetrectcap%
\pgfsetroundjoin%
\pgfsetlinewidth{1.505625pt}%
\definecolor{currentstroke}{rgb}{1.000000,0.705882,0.509804}%
\pgfsetstrokecolor{currentstroke}%
\pgfsetstrokeopacity{0.800000}%
\pgfsetdash{}{0pt}%
\pgfpathmoveto{\pgfqpoint{2.176301in}{2.922426in}}%
\pgfpathlineto{\pgfqpoint{3.502867in}{4.072531in}}%
\pgfusepath{stroke}%
\end{pgfscope}%
\begin{pgfscope}%
\pgfpathrectangle{\pgfqpoint{0.481978in}{0.331635in}}{\pgfqpoint{9.300000in}{7.700000in}}%
\pgfusepath{clip}%
\pgfsetrectcap%
\pgfsetroundjoin%
\pgfsetlinewidth{1.505625pt}%
\definecolor{currentstroke}{rgb}{1.000000,0.705882,0.509804}%
\pgfsetstrokecolor{currentstroke}%
\pgfsetstrokeopacity{0.800000}%
\pgfsetdash{}{0pt}%
\pgfpathmoveto{\pgfqpoint{4.459267in}{4.587523in}}%
\pgfpathlineto{\pgfqpoint{3.502867in}{4.072531in}}%
\pgfusepath{stroke}%
\end{pgfscope}%
\begin{pgfscope}%
\pgfpathrectangle{\pgfqpoint{0.481978in}{0.331635in}}{\pgfqpoint{9.300000in}{7.700000in}}%
\pgfusepath{clip}%
\pgfsetrectcap%
\pgfsetroundjoin%
\pgfsetlinewidth{1.505625pt}%
\definecolor{currentstroke}{rgb}{1.000000,0.705882,0.509804}%
\pgfsetstrokecolor{currentstroke}%
\pgfsetstrokeopacity{0.800000}%
\pgfsetdash{}{0pt}%
\pgfpathmoveto{\pgfqpoint{4.335059in}{3.019883in}}%
\pgfpathlineto{\pgfqpoint{3.502867in}{4.072531in}}%
\pgfusepath{stroke}%
\end{pgfscope}%
\begin{pgfscope}%
\pgfpathrectangle{\pgfqpoint{0.481978in}{0.331635in}}{\pgfqpoint{9.300000in}{7.700000in}}%
\pgfusepath{clip}%
\pgfsetrectcap%
\pgfsetroundjoin%
\pgfsetlinewidth{1.505625pt}%
\definecolor{currentstroke}{rgb}{1.000000,0.705882,0.509804}%
\pgfsetstrokecolor{currentstroke}%
\pgfsetstrokeopacity{0.800000}%
\pgfsetdash{}{0pt}%
\pgfpathmoveto{\pgfqpoint{4.103001in}{2.989415in}}%
\pgfpathlineto{\pgfqpoint{3.502867in}{4.072531in}}%
\pgfusepath{stroke}%
\end{pgfscope}%
\begin{pgfscope}%
\pgfpathrectangle{\pgfqpoint{0.481978in}{0.331635in}}{\pgfqpoint{9.300000in}{7.700000in}}%
\pgfusepath{clip}%
\pgfsetrectcap%
\pgfsetroundjoin%
\pgfsetlinewidth{1.505625pt}%
\definecolor{currentstroke}{rgb}{1.000000,0.705882,0.509804}%
\pgfsetstrokecolor{currentstroke}%
\pgfsetstrokeopacity{0.800000}%
\pgfsetdash{}{0pt}%
\pgfpathmoveto{\pgfqpoint{1.240028in}{3.789606in}}%
\pgfpathlineto{\pgfqpoint{3.502867in}{4.072531in}}%
\pgfusepath{stroke}%
\end{pgfscope}%
\begin{pgfscope}%
\pgfpathrectangle{\pgfqpoint{0.481978in}{0.331635in}}{\pgfqpoint{9.300000in}{7.700000in}}%
\pgfusepath{clip}%
\pgfsetrectcap%
\pgfsetroundjoin%
\pgfsetlinewidth{1.505625pt}%
\definecolor{currentstroke}{rgb}{1.000000,0.705882,0.509804}%
\pgfsetstrokecolor{currentstroke}%
\pgfsetstrokeopacity{0.800000}%
\pgfsetdash{}{0pt}%
\pgfpathmoveto{\pgfqpoint{3.665627in}{3.711742in}}%
\pgfpathlineto{\pgfqpoint{3.502867in}{4.072531in}}%
\pgfusepath{stroke}%
\end{pgfscope}%
\begin{pgfscope}%
\pgfpathrectangle{\pgfqpoint{0.481978in}{0.331635in}}{\pgfqpoint{9.300000in}{7.700000in}}%
\pgfusepath{clip}%
\pgfsetrectcap%
\pgfsetroundjoin%
\pgfsetlinewidth{1.505625pt}%
\definecolor{currentstroke}{rgb}{1.000000,0.705882,0.509804}%
\pgfsetstrokecolor{currentstroke}%
\pgfsetstrokeopacity{0.800000}%
\pgfsetdash{}{0pt}%
\pgfpathmoveto{\pgfqpoint{3.637232in}{4.421791in}}%
\pgfpathlineto{\pgfqpoint{3.502867in}{4.072531in}}%
\pgfusepath{stroke}%
\end{pgfscope}%
\begin{pgfscope}%
\pgfpathrectangle{\pgfqpoint{0.481978in}{0.331635in}}{\pgfqpoint{9.300000in}{7.700000in}}%
\pgfusepath{clip}%
\pgfsetrectcap%
\pgfsetroundjoin%
\pgfsetlinewidth{1.505625pt}%
\definecolor{currentstroke}{rgb}{1.000000,0.705882,0.509804}%
\pgfsetstrokecolor{currentstroke}%
\pgfsetstrokeopacity{0.800000}%
\pgfsetdash{}{0pt}%
\pgfpathmoveto{\pgfqpoint{1.338259in}{3.743025in}}%
\pgfpathlineto{\pgfqpoint{3.502867in}{4.072531in}}%
\pgfusepath{stroke}%
\end{pgfscope}%
\begin{pgfscope}%
\pgfpathrectangle{\pgfqpoint{0.481978in}{0.331635in}}{\pgfqpoint{9.300000in}{7.700000in}}%
\pgfusepath{clip}%
\pgfsetrectcap%
\pgfsetroundjoin%
\pgfsetlinewidth{1.505625pt}%
\definecolor{currentstroke}{rgb}{1.000000,0.705882,0.509804}%
\pgfsetstrokecolor{currentstroke}%
\pgfsetstrokeopacity{0.800000}%
\pgfsetdash{}{0pt}%
\pgfpathmoveto{\pgfqpoint{1.995054in}{3.031987in}}%
\pgfpathlineto{\pgfqpoint{3.502867in}{4.072531in}}%
\pgfusepath{stroke}%
\end{pgfscope}%
\begin{pgfscope}%
\pgfpathrectangle{\pgfqpoint{0.481978in}{0.331635in}}{\pgfqpoint{9.300000in}{7.700000in}}%
\pgfusepath{clip}%
\pgfsetrectcap%
\pgfsetroundjoin%
\pgfsetlinewidth{1.505625pt}%
\definecolor{currentstroke}{rgb}{1.000000,0.705882,0.509804}%
\pgfsetstrokecolor{currentstroke}%
\pgfsetstrokeopacity{0.800000}%
\pgfsetdash{}{0pt}%
\pgfpathmoveto{\pgfqpoint{1.350306in}{4.003250in}}%
\pgfpathlineto{\pgfqpoint{3.502867in}{4.072531in}}%
\pgfusepath{stroke}%
\end{pgfscope}%
\begin{pgfscope}%
\pgfpathrectangle{\pgfqpoint{0.481978in}{0.331635in}}{\pgfqpoint{9.300000in}{7.700000in}}%
\pgfusepath{clip}%
\pgfsetrectcap%
\pgfsetroundjoin%
\pgfsetlinewidth{1.505625pt}%
\definecolor{currentstroke}{rgb}{1.000000,0.705882,0.509804}%
\pgfsetstrokecolor{currentstroke}%
\pgfsetstrokeopacity{0.800000}%
\pgfsetdash{}{0pt}%
\pgfpathmoveto{\pgfqpoint{1.701822in}{4.066580in}}%
\pgfpathlineto{\pgfqpoint{3.502867in}{4.072531in}}%
\pgfusepath{stroke}%
\end{pgfscope}%
\begin{pgfscope}%
\pgfpathrectangle{\pgfqpoint{0.481978in}{0.331635in}}{\pgfqpoint{9.300000in}{7.700000in}}%
\pgfusepath{clip}%
\pgfsetrectcap%
\pgfsetroundjoin%
\pgfsetlinewidth{1.505625pt}%
\definecolor{currentstroke}{rgb}{1.000000,0.705882,0.509804}%
\pgfsetstrokecolor{currentstroke}%
\pgfsetstrokeopacity{0.800000}%
\pgfsetdash{}{0pt}%
\pgfpathmoveto{\pgfqpoint{3.815204in}{4.006611in}}%
\pgfpathlineto{\pgfqpoint{3.502867in}{4.072531in}}%
\pgfusepath{stroke}%
\end{pgfscope}%
\begin{pgfscope}%
\pgfpathrectangle{\pgfqpoint{0.481978in}{0.331635in}}{\pgfqpoint{9.300000in}{7.700000in}}%
\pgfusepath{clip}%
\pgfsetrectcap%
\pgfsetroundjoin%
\pgfsetlinewidth{1.505625pt}%
\definecolor{currentstroke}{rgb}{1.000000,0.705882,0.509804}%
\pgfsetstrokecolor{currentstroke}%
\pgfsetstrokeopacity{0.800000}%
\pgfsetdash{}{0pt}%
\pgfpathmoveto{\pgfqpoint{3.592234in}{4.973285in}}%
\pgfpathlineto{\pgfqpoint{3.502867in}{4.072531in}}%
\pgfusepath{stroke}%
\end{pgfscope}%
\begin{pgfscope}%
\pgfpathrectangle{\pgfqpoint{0.481978in}{0.331635in}}{\pgfqpoint{9.300000in}{7.700000in}}%
\pgfusepath{clip}%
\pgfsetrectcap%
\pgfsetroundjoin%
\pgfsetlinewidth{1.505625pt}%
\definecolor{currentstroke}{rgb}{1.000000,0.705882,0.509804}%
\pgfsetstrokecolor{currentstroke}%
\pgfsetstrokeopacity{0.800000}%
\pgfsetdash{}{0pt}%
\pgfpathmoveto{\pgfqpoint{4.400900in}{2.398264in}}%
\pgfpathlineto{\pgfqpoint{3.502867in}{4.072531in}}%
\pgfusepath{stroke}%
\end{pgfscope}%
\begin{pgfscope}%
\pgfpathrectangle{\pgfqpoint{0.481978in}{0.331635in}}{\pgfqpoint{9.300000in}{7.700000in}}%
\pgfusepath{clip}%
\pgfsetrectcap%
\pgfsetroundjoin%
\pgfsetlinewidth{1.505625pt}%
\definecolor{currentstroke}{rgb}{1.000000,0.705882,0.509804}%
\pgfsetstrokecolor{currentstroke}%
\pgfsetstrokeopacity{0.800000}%
\pgfsetdash{}{0pt}%
\pgfpathmoveto{\pgfqpoint{4.677099in}{3.495443in}}%
\pgfpathlineto{\pgfqpoint{3.502867in}{4.072531in}}%
\pgfusepath{stroke}%
\end{pgfscope}%
\begin{pgfscope}%
\pgfpathrectangle{\pgfqpoint{0.481978in}{0.331635in}}{\pgfqpoint{9.300000in}{7.700000in}}%
\pgfusepath{clip}%
\pgfsetrectcap%
\pgfsetroundjoin%
\pgfsetlinewidth{1.505625pt}%
\definecolor{currentstroke}{rgb}{1.000000,0.705882,0.509804}%
\pgfsetstrokecolor{currentstroke}%
\pgfsetstrokeopacity{0.800000}%
\pgfsetdash{}{0pt}%
\pgfpathmoveto{\pgfqpoint{2.595817in}{5.503093in}}%
\pgfpathlineto{\pgfqpoint{3.502867in}{4.072531in}}%
\pgfusepath{stroke}%
\end{pgfscope}%
\begin{pgfscope}%
\pgfpathrectangle{\pgfqpoint{0.481978in}{0.331635in}}{\pgfqpoint{9.300000in}{7.700000in}}%
\pgfusepath{clip}%
\pgfsetrectcap%
\pgfsetroundjoin%
\pgfsetlinewidth{1.505625pt}%
\definecolor{currentstroke}{rgb}{1.000000,0.705882,0.509804}%
\pgfsetstrokecolor{currentstroke}%
\pgfsetstrokeopacity{0.800000}%
\pgfsetdash{}{0pt}%
\pgfpathmoveto{\pgfqpoint{3.380362in}{4.128920in}}%
\pgfpathlineto{\pgfqpoint{3.502867in}{4.072531in}}%
\pgfusepath{stroke}%
\end{pgfscope}%
\begin{pgfscope}%
\pgfpathrectangle{\pgfqpoint{0.481978in}{0.331635in}}{\pgfqpoint{9.300000in}{7.700000in}}%
\pgfusepath{clip}%
\pgfsetrectcap%
\pgfsetroundjoin%
\pgfsetlinewidth{1.505625pt}%
\definecolor{currentstroke}{rgb}{1.000000,0.705882,0.509804}%
\pgfsetstrokecolor{currentstroke}%
\pgfsetstrokeopacity{0.800000}%
\pgfsetdash{}{0pt}%
\pgfpathmoveto{\pgfqpoint{8.094040in}{3.915369in}}%
\pgfpathlineto{\pgfqpoint{3.502867in}{4.072531in}}%
\pgfusepath{stroke}%
\end{pgfscope}%
\begin{pgfscope}%
\pgfpathrectangle{\pgfqpoint{0.481978in}{0.331635in}}{\pgfqpoint{9.300000in}{7.700000in}}%
\pgfusepath{clip}%
\pgfsetrectcap%
\pgfsetroundjoin%
\pgfsetlinewidth{1.505625pt}%
\definecolor{currentstroke}{rgb}{1.000000,0.705882,0.509804}%
\pgfsetstrokecolor{currentstroke}%
\pgfsetstrokeopacity{0.800000}%
\pgfsetdash{}{0pt}%
\pgfpathmoveto{\pgfqpoint{3.449664in}{4.288307in}}%
\pgfpathlineto{\pgfqpoint{3.502867in}{4.072531in}}%
\pgfusepath{stroke}%
\end{pgfscope}%
\begin{pgfscope}%
\pgfpathrectangle{\pgfqpoint{0.481978in}{0.331635in}}{\pgfqpoint{9.300000in}{7.700000in}}%
\pgfusepath{clip}%
\pgfsetrectcap%
\pgfsetroundjoin%
\pgfsetlinewidth{1.505625pt}%
\definecolor{currentstroke}{rgb}{1.000000,0.705882,0.509804}%
\pgfsetstrokecolor{currentstroke}%
\pgfsetstrokeopacity{0.800000}%
\pgfsetdash{}{0pt}%
\pgfpathmoveto{\pgfqpoint{3.549059in}{7.246059in}}%
\pgfpathlineto{\pgfqpoint{3.502867in}{4.072531in}}%
\pgfusepath{stroke}%
\end{pgfscope}%
\begin{pgfscope}%
\pgfpathrectangle{\pgfqpoint{0.481978in}{0.331635in}}{\pgfqpoint{9.300000in}{7.700000in}}%
\pgfusepath{clip}%
\pgfsetrectcap%
\pgfsetroundjoin%
\pgfsetlinewidth{1.505625pt}%
\definecolor{currentstroke}{rgb}{1.000000,0.705882,0.509804}%
\pgfsetstrokecolor{currentstroke}%
\pgfsetstrokeopacity{0.800000}%
\pgfsetdash{}{0pt}%
\pgfpathmoveto{\pgfqpoint{1.737569in}{3.322770in}}%
\pgfpathlineto{\pgfqpoint{3.502867in}{4.072531in}}%
\pgfusepath{stroke}%
\end{pgfscope}%
\begin{pgfscope}%
\pgfpathrectangle{\pgfqpoint{0.481978in}{0.331635in}}{\pgfqpoint{9.300000in}{7.700000in}}%
\pgfusepath{clip}%
\pgfsetrectcap%
\pgfsetroundjoin%
\pgfsetlinewidth{1.505625pt}%
\definecolor{currentstroke}{rgb}{1.000000,0.705882,0.509804}%
\pgfsetstrokecolor{currentstroke}%
\pgfsetstrokeopacity{0.800000}%
\pgfsetdash{}{0pt}%
\pgfpathmoveto{\pgfqpoint{3.522445in}{3.146007in}}%
\pgfpathlineto{\pgfqpoint{3.502867in}{4.072531in}}%
\pgfusepath{stroke}%
\end{pgfscope}%
\begin{pgfscope}%
\pgfpathrectangle{\pgfqpoint{0.481978in}{0.331635in}}{\pgfqpoint{9.300000in}{7.700000in}}%
\pgfusepath{clip}%
\pgfsetrectcap%
\pgfsetroundjoin%
\pgfsetlinewidth{1.505625pt}%
\definecolor{currentstroke}{rgb}{1.000000,0.705882,0.509804}%
\pgfsetstrokecolor{currentstroke}%
\pgfsetstrokeopacity{0.800000}%
\pgfsetdash{}{0pt}%
\pgfpathmoveto{\pgfqpoint{5.442526in}{4.591847in}}%
\pgfpathlineto{\pgfqpoint{3.502867in}{4.072531in}}%
\pgfusepath{stroke}%
\end{pgfscope}%
\begin{pgfscope}%
\pgfpathrectangle{\pgfqpoint{0.481978in}{0.331635in}}{\pgfqpoint{9.300000in}{7.700000in}}%
\pgfusepath{clip}%
\pgfsetrectcap%
\pgfsetroundjoin%
\pgfsetlinewidth{1.505625pt}%
\definecolor{currentstroke}{rgb}{1.000000,0.705882,0.509804}%
\pgfsetstrokecolor{currentstroke}%
\pgfsetstrokeopacity{0.800000}%
\pgfsetdash{}{0pt}%
\pgfpathmoveto{\pgfqpoint{3.182729in}{4.203491in}}%
\pgfpathlineto{\pgfqpoint{3.502867in}{4.072531in}}%
\pgfusepath{stroke}%
\end{pgfscope}%
\begin{pgfscope}%
\pgfpathrectangle{\pgfqpoint{0.481978in}{0.331635in}}{\pgfqpoint{9.300000in}{7.700000in}}%
\pgfusepath{clip}%
\pgfsetrectcap%
\pgfsetroundjoin%
\pgfsetlinewidth{1.505625pt}%
\definecolor{currentstroke}{rgb}{1.000000,0.705882,0.509804}%
\pgfsetstrokecolor{currentstroke}%
\pgfsetstrokeopacity{0.800000}%
\pgfsetdash{}{0pt}%
\pgfpathmoveto{\pgfqpoint{5.580105in}{6.325768in}}%
\pgfpathlineto{\pgfqpoint{3.502867in}{4.072531in}}%
\pgfusepath{stroke}%
\end{pgfscope}%
\begin{pgfscope}%
\pgfpathrectangle{\pgfqpoint{0.481978in}{0.331635in}}{\pgfqpoint{9.300000in}{7.700000in}}%
\pgfusepath{clip}%
\pgfsetrectcap%
\pgfsetroundjoin%
\pgfsetlinewidth{1.505625pt}%
\definecolor{currentstroke}{rgb}{1.000000,0.705882,0.509804}%
\pgfsetstrokecolor{currentstroke}%
\pgfsetstrokeopacity{0.800000}%
\pgfsetdash{}{0pt}%
\pgfpathmoveto{\pgfqpoint{3.383682in}{4.926698in}}%
\pgfpathlineto{\pgfqpoint{3.502867in}{4.072531in}}%
\pgfusepath{stroke}%
\end{pgfscope}%
\begin{pgfscope}%
\pgfpathrectangle{\pgfqpoint{0.481978in}{0.331635in}}{\pgfqpoint{9.300000in}{7.700000in}}%
\pgfusepath{clip}%
\pgfsetrectcap%
\pgfsetroundjoin%
\pgfsetlinewidth{1.505625pt}%
\definecolor{currentstroke}{rgb}{1.000000,0.705882,0.509804}%
\pgfsetstrokecolor{currentstroke}%
\pgfsetstrokeopacity{0.800000}%
\pgfsetdash{}{0pt}%
\pgfpathmoveto{\pgfqpoint{3.010244in}{5.167031in}}%
\pgfpathlineto{\pgfqpoint{3.502867in}{4.072531in}}%
\pgfusepath{stroke}%
\end{pgfscope}%
\begin{pgfscope}%
\pgfpathrectangle{\pgfqpoint{0.481978in}{0.331635in}}{\pgfqpoint{9.300000in}{7.700000in}}%
\pgfusepath{clip}%
\pgfsetrectcap%
\pgfsetroundjoin%
\pgfsetlinewidth{1.505625pt}%
\definecolor{currentstroke}{rgb}{1.000000,0.705882,0.509804}%
\pgfsetstrokecolor{currentstroke}%
\pgfsetstrokeopacity{0.800000}%
\pgfsetdash{}{0pt}%
\pgfpathmoveto{\pgfqpoint{4.553603in}{4.039119in}}%
\pgfpathlineto{\pgfqpoint{3.502867in}{4.072531in}}%
\pgfusepath{stroke}%
\end{pgfscope}%
\begin{pgfscope}%
\pgfpathrectangle{\pgfqpoint{0.481978in}{0.331635in}}{\pgfqpoint{9.300000in}{7.700000in}}%
\pgfusepath{clip}%
\pgfsetrectcap%
\pgfsetroundjoin%
\pgfsetlinewidth{1.505625pt}%
\definecolor{currentstroke}{rgb}{1.000000,0.705882,0.509804}%
\pgfsetstrokecolor{currentstroke}%
\pgfsetstrokeopacity{0.800000}%
\pgfsetdash{}{0pt}%
\pgfpathmoveto{\pgfqpoint{9.359202in}{1.365364in}}%
\pgfpathlineto{\pgfqpoint{3.502867in}{4.072531in}}%
\pgfusepath{stroke}%
\end{pgfscope}%
\begin{pgfscope}%
\pgfpathrectangle{\pgfqpoint{0.481978in}{0.331635in}}{\pgfqpoint{9.300000in}{7.700000in}}%
\pgfusepath{clip}%
\pgfsetrectcap%
\pgfsetroundjoin%
\pgfsetlinewidth{1.505625pt}%
\definecolor{currentstroke}{rgb}{1.000000,0.705882,0.509804}%
\pgfsetstrokecolor{currentstroke}%
\pgfsetstrokeopacity{0.800000}%
\pgfsetdash{}{0pt}%
\pgfpathmoveto{\pgfqpoint{3.953659in}{2.911613in}}%
\pgfpathlineto{\pgfqpoint{3.502867in}{4.072531in}}%
\pgfusepath{stroke}%
\end{pgfscope}%
\begin{pgfscope}%
\pgfpathrectangle{\pgfqpoint{0.481978in}{0.331635in}}{\pgfqpoint{9.300000in}{7.700000in}}%
\pgfusepath{clip}%
\pgfsetrectcap%
\pgfsetroundjoin%
\pgfsetlinewidth{1.505625pt}%
\definecolor{currentstroke}{rgb}{1.000000,0.705882,0.509804}%
\pgfsetstrokecolor{currentstroke}%
\pgfsetstrokeopacity{0.800000}%
\pgfsetdash{}{0pt}%
\pgfpathmoveto{\pgfqpoint{4.408411in}{2.639990in}}%
\pgfpathlineto{\pgfqpoint{3.502867in}{4.072531in}}%
\pgfusepath{stroke}%
\end{pgfscope}%
\begin{pgfscope}%
\pgfpathrectangle{\pgfqpoint{0.481978in}{0.331635in}}{\pgfqpoint{9.300000in}{7.700000in}}%
\pgfusepath{clip}%
\pgfsetrectcap%
\pgfsetroundjoin%
\pgfsetlinewidth{1.505625pt}%
\definecolor{currentstroke}{rgb}{1.000000,0.705882,0.509804}%
\pgfsetstrokecolor{currentstroke}%
\pgfsetstrokeopacity{0.800000}%
\pgfsetdash{}{0pt}%
\pgfpathmoveto{\pgfqpoint{1.290181in}{5.575688in}}%
\pgfpathlineto{\pgfqpoint{3.502867in}{4.072531in}}%
\pgfusepath{stroke}%
\end{pgfscope}%
\begin{pgfscope}%
\pgfpathrectangle{\pgfqpoint{0.481978in}{0.331635in}}{\pgfqpoint{9.300000in}{7.700000in}}%
\pgfusepath{clip}%
\pgfsetrectcap%
\pgfsetroundjoin%
\pgfsetlinewidth{1.505625pt}%
\definecolor{currentstroke}{rgb}{1.000000,0.705882,0.509804}%
\pgfsetstrokecolor{currentstroke}%
\pgfsetstrokeopacity{0.800000}%
\pgfsetdash{}{0pt}%
\pgfpathmoveto{\pgfqpoint{2.624493in}{3.262294in}}%
\pgfpathlineto{\pgfqpoint{3.502867in}{4.072531in}}%
\pgfusepath{stroke}%
\end{pgfscope}%
\begin{pgfscope}%
\pgfpathrectangle{\pgfqpoint{0.481978in}{0.331635in}}{\pgfqpoint{9.300000in}{7.700000in}}%
\pgfusepath{clip}%
\pgfsetrectcap%
\pgfsetroundjoin%
\pgfsetlinewidth{1.505625pt}%
\definecolor{currentstroke}{rgb}{1.000000,0.705882,0.509804}%
\pgfsetstrokecolor{currentstroke}%
\pgfsetstrokeopacity{0.800000}%
\pgfsetdash{}{0pt}%
\pgfpathmoveto{\pgfqpoint{5.315161in}{4.162619in}}%
\pgfpathlineto{\pgfqpoint{3.502867in}{4.072531in}}%
\pgfusepath{stroke}%
\end{pgfscope}%
\begin{pgfscope}%
\pgfpathrectangle{\pgfqpoint{0.481978in}{0.331635in}}{\pgfqpoint{9.300000in}{7.700000in}}%
\pgfusepath{clip}%
\pgfsetrectcap%
\pgfsetroundjoin%
\pgfsetlinewidth{1.505625pt}%
\definecolor{currentstroke}{rgb}{1.000000,0.705882,0.509804}%
\pgfsetstrokecolor{currentstroke}%
\pgfsetstrokeopacity{0.800000}%
\pgfsetdash{}{0pt}%
\pgfpathmoveto{\pgfqpoint{4.062917in}{3.260250in}}%
\pgfpathlineto{\pgfqpoint{3.502867in}{4.072531in}}%
\pgfusepath{stroke}%
\end{pgfscope}%
\begin{pgfscope}%
\pgfpathrectangle{\pgfqpoint{0.481978in}{0.331635in}}{\pgfqpoint{9.300000in}{7.700000in}}%
\pgfusepath{clip}%
\pgfsetrectcap%
\pgfsetroundjoin%
\pgfsetlinewidth{1.505625pt}%
\definecolor{currentstroke}{rgb}{1.000000,0.705882,0.509804}%
\pgfsetstrokecolor{currentstroke}%
\pgfsetstrokeopacity{0.800000}%
\pgfsetdash{}{0pt}%
\pgfpathmoveto{\pgfqpoint{4.828561in}{4.070130in}}%
\pgfpathlineto{\pgfqpoint{3.502867in}{4.072531in}}%
\pgfusepath{stroke}%
\end{pgfscope}%
\begin{pgfscope}%
\pgfpathrectangle{\pgfqpoint{0.481978in}{0.331635in}}{\pgfqpoint{9.300000in}{7.700000in}}%
\pgfusepath{clip}%
\pgfsetrectcap%
\pgfsetroundjoin%
\pgfsetlinewidth{1.505625pt}%
\definecolor{currentstroke}{rgb}{1.000000,0.705882,0.509804}%
\pgfsetstrokecolor{currentstroke}%
\pgfsetstrokeopacity{0.800000}%
\pgfsetdash{}{0pt}%
\pgfpathmoveto{\pgfqpoint{2.387650in}{3.525751in}}%
\pgfpathlineto{\pgfqpoint{3.502867in}{4.072531in}}%
\pgfusepath{stroke}%
\end{pgfscope}%
\begin{pgfscope}%
\pgfpathrectangle{\pgfqpoint{0.481978in}{0.331635in}}{\pgfqpoint{9.300000in}{7.700000in}}%
\pgfusepath{clip}%
\pgfsetrectcap%
\pgfsetroundjoin%
\pgfsetlinewidth{1.505625pt}%
\definecolor{currentstroke}{rgb}{1.000000,0.705882,0.509804}%
\pgfsetstrokecolor{currentstroke}%
\pgfsetstrokeopacity{0.800000}%
\pgfsetdash{}{0pt}%
\pgfpathmoveto{\pgfqpoint{3.134591in}{4.927538in}}%
\pgfpathlineto{\pgfqpoint{3.502867in}{4.072531in}}%
\pgfusepath{stroke}%
\end{pgfscope}%
\begin{pgfscope}%
\pgfpathrectangle{\pgfqpoint{0.481978in}{0.331635in}}{\pgfqpoint{9.300000in}{7.700000in}}%
\pgfusepath{clip}%
\pgfsetrectcap%
\pgfsetroundjoin%
\pgfsetlinewidth{1.505625pt}%
\definecolor{currentstroke}{rgb}{1.000000,0.705882,0.509804}%
\pgfsetstrokecolor{currentstroke}%
\pgfsetstrokeopacity{0.800000}%
\pgfsetdash{}{0pt}%
\pgfpathmoveto{\pgfqpoint{1.638268in}{4.702300in}}%
\pgfpathlineto{\pgfqpoint{3.502867in}{4.072531in}}%
\pgfusepath{stroke}%
\end{pgfscope}%
\begin{pgfscope}%
\pgfpathrectangle{\pgfqpoint{0.481978in}{0.331635in}}{\pgfqpoint{9.300000in}{7.700000in}}%
\pgfusepath{clip}%
\pgfsetrectcap%
\pgfsetroundjoin%
\pgfsetlinewidth{1.505625pt}%
\definecolor{currentstroke}{rgb}{1.000000,0.705882,0.509804}%
\pgfsetstrokecolor{currentstroke}%
\pgfsetstrokeopacity{0.800000}%
\pgfsetdash{}{0pt}%
\pgfpathmoveto{\pgfqpoint{4.067183in}{6.837238in}}%
\pgfpathlineto{\pgfqpoint{3.502867in}{4.072531in}}%
\pgfusepath{stroke}%
\end{pgfscope}%
\begin{pgfscope}%
\pgfpathrectangle{\pgfqpoint{0.481978in}{0.331635in}}{\pgfqpoint{9.300000in}{7.700000in}}%
\pgfusepath{clip}%
\pgfsetrectcap%
\pgfsetroundjoin%
\pgfsetlinewidth{1.505625pt}%
\definecolor{currentstroke}{rgb}{1.000000,0.705882,0.509804}%
\pgfsetstrokecolor{currentstroke}%
\pgfsetstrokeopacity{0.800000}%
\pgfsetdash{}{0pt}%
\pgfpathmoveto{\pgfqpoint{3.079878in}{3.918994in}}%
\pgfpathlineto{\pgfqpoint{3.502867in}{4.072531in}}%
\pgfusepath{stroke}%
\end{pgfscope}%
\begin{pgfscope}%
\pgfpathrectangle{\pgfqpoint{0.481978in}{0.331635in}}{\pgfqpoint{9.300000in}{7.700000in}}%
\pgfusepath{clip}%
\pgfsetrectcap%
\pgfsetroundjoin%
\pgfsetlinewidth{1.505625pt}%
\definecolor{currentstroke}{rgb}{1.000000,0.705882,0.509804}%
\pgfsetstrokecolor{currentstroke}%
\pgfsetstrokeopacity{0.800000}%
\pgfsetdash{}{0pt}%
\pgfpathmoveto{\pgfqpoint{3.762277in}{2.634602in}}%
\pgfpathlineto{\pgfqpoint{3.502867in}{4.072531in}}%
\pgfusepath{stroke}%
\end{pgfscope}%
\begin{pgfscope}%
\pgfpathrectangle{\pgfqpoint{0.481978in}{0.331635in}}{\pgfqpoint{9.300000in}{7.700000in}}%
\pgfusepath{clip}%
\pgfsetrectcap%
\pgfsetroundjoin%
\pgfsetlinewidth{1.505625pt}%
\definecolor{currentstroke}{rgb}{1.000000,0.705882,0.509804}%
\pgfsetstrokecolor{currentstroke}%
\pgfsetstrokeopacity{0.800000}%
\pgfsetdash{}{0pt}%
\pgfpathmoveto{\pgfqpoint{3.580814in}{3.904227in}}%
\pgfpathlineto{\pgfqpoint{3.502867in}{4.072531in}}%
\pgfusepath{stroke}%
\end{pgfscope}%
\begin{pgfscope}%
\pgfpathrectangle{\pgfqpoint{0.481978in}{0.331635in}}{\pgfqpoint{9.300000in}{7.700000in}}%
\pgfusepath{clip}%
\pgfsetrectcap%
\pgfsetroundjoin%
\pgfsetlinewidth{1.505625pt}%
\definecolor{currentstroke}{rgb}{1.000000,0.705882,0.509804}%
\pgfsetstrokecolor{currentstroke}%
\pgfsetstrokeopacity{0.800000}%
\pgfsetdash{}{0pt}%
\pgfpathmoveto{\pgfqpoint{4.361455in}{2.831609in}}%
\pgfpathlineto{\pgfqpoint{3.502867in}{4.072531in}}%
\pgfusepath{stroke}%
\end{pgfscope}%
\begin{pgfscope}%
\pgfpathrectangle{\pgfqpoint{0.481978in}{0.331635in}}{\pgfqpoint{9.300000in}{7.700000in}}%
\pgfusepath{clip}%
\pgfsetrectcap%
\pgfsetroundjoin%
\pgfsetlinewidth{1.505625pt}%
\definecolor{currentstroke}{rgb}{1.000000,0.705882,0.509804}%
\pgfsetstrokecolor{currentstroke}%
\pgfsetstrokeopacity{0.800000}%
\pgfsetdash{}{0pt}%
\pgfpathmoveto{\pgfqpoint{3.821252in}{6.921355in}}%
\pgfpathlineto{\pgfqpoint{3.502867in}{4.072531in}}%
\pgfusepath{stroke}%
\end{pgfscope}%
\begin{pgfscope}%
\pgfpathrectangle{\pgfqpoint{0.481978in}{0.331635in}}{\pgfqpoint{9.300000in}{7.700000in}}%
\pgfusepath{clip}%
\pgfsetrectcap%
\pgfsetroundjoin%
\pgfsetlinewidth{1.505625pt}%
\definecolor{currentstroke}{rgb}{1.000000,0.705882,0.509804}%
\pgfsetstrokecolor{currentstroke}%
\pgfsetstrokeopacity{0.800000}%
\pgfsetdash{}{0pt}%
\pgfpathmoveto{\pgfqpoint{2.933293in}{5.640936in}}%
\pgfpathlineto{\pgfqpoint{3.502867in}{4.072531in}}%
\pgfusepath{stroke}%
\end{pgfscope}%
\begin{pgfscope}%
\pgfpathrectangle{\pgfqpoint{0.481978in}{0.331635in}}{\pgfqpoint{9.300000in}{7.700000in}}%
\pgfusepath{clip}%
\pgfsetrectcap%
\pgfsetroundjoin%
\pgfsetlinewidth{1.505625pt}%
\definecolor{currentstroke}{rgb}{1.000000,0.705882,0.509804}%
\pgfsetstrokecolor{currentstroke}%
\pgfsetstrokeopacity{0.800000}%
\pgfsetdash{}{0pt}%
\pgfpathmoveto{\pgfqpoint{4.420071in}{4.768249in}}%
\pgfpathlineto{\pgfqpoint{3.502867in}{4.072531in}}%
\pgfusepath{stroke}%
\end{pgfscope}%
\begin{pgfscope}%
\pgfpathrectangle{\pgfqpoint{0.481978in}{0.331635in}}{\pgfqpoint{9.300000in}{7.700000in}}%
\pgfusepath{clip}%
\pgfsetrectcap%
\pgfsetroundjoin%
\pgfsetlinewidth{1.505625pt}%
\definecolor{currentstroke}{rgb}{1.000000,0.705882,0.509804}%
\pgfsetstrokecolor{currentstroke}%
\pgfsetstrokeopacity{0.800000}%
\pgfsetdash{}{0pt}%
\pgfpathmoveto{\pgfqpoint{2.246703in}{5.008506in}}%
\pgfpathlineto{\pgfqpoint{3.502867in}{4.072531in}}%
\pgfusepath{stroke}%
\end{pgfscope}%
\begin{pgfscope}%
\pgfpathrectangle{\pgfqpoint{0.481978in}{0.331635in}}{\pgfqpoint{9.300000in}{7.700000in}}%
\pgfusepath{clip}%
\pgfsetrectcap%
\pgfsetroundjoin%
\pgfsetlinewidth{1.505625pt}%
\definecolor{currentstroke}{rgb}{1.000000,0.705882,0.509804}%
\pgfsetstrokecolor{currentstroke}%
\pgfsetstrokeopacity{0.800000}%
\pgfsetdash{}{0pt}%
\pgfpathmoveto{\pgfqpoint{4.205087in}{5.636892in}}%
\pgfpathlineto{\pgfqpoint{3.502867in}{4.072531in}}%
\pgfusepath{stroke}%
\end{pgfscope}%
\begin{pgfscope}%
\pgfpathrectangle{\pgfqpoint{0.481978in}{0.331635in}}{\pgfqpoint{9.300000in}{7.700000in}}%
\pgfusepath{clip}%
\pgfsetrectcap%
\pgfsetroundjoin%
\pgfsetlinewidth{1.505625pt}%
\definecolor{currentstroke}{rgb}{1.000000,0.705882,0.509804}%
\pgfsetstrokecolor{currentstroke}%
\pgfsetstrokeopacity{0.800000}%
\pgfsetdash{}{0pt}%
\pgfpathmoveto{\pgfqpoint{3.838690in}{4.225761in}}%
\pgfpathlineto{\pgfqpoint{3.502867in}{4.072531in}}%
\pgfusepath{stroke}%
\end{pgfscope}%
\begin{pgfscope}%
\pgfpathrectangle{\pgfqpoint{0.481978in}{0.331635in}}{\pgfqpoint{9.300000in}{7.700000in}}%
\pgfusepath{clip}%
\pgfsetrectcap%
\pgfsetroundjoin%
\pgfsetlinewidth{1.505625pt}%
\definecolor{currentstroke}{rgb}{1.000000,0.705882,0.509804}%
\pgfsetstrokecolor{currentstroke}%
\pgfsetstrokeopacity{0.800000}%
\pgfsetdash{}{0pt}%
\pgfpathmoveto{\pgfqpoint{1.293480in}{5.571862in}}%
\pgfpathlineto{\pgfqpoint{3.502867in}{4.072531in}}%
\pgfusepath{stroke}%
\end{pgfscope}%
\begin{pgfscope}%
\pgfpathrectangle{\pgfqpoint{0.481978in}{0.331635in}}{\pgfqpoint{9.300000in}{7.700000in}}%
\pgfusepath{clip}%
\pgfsetrectcap%
\pgfsetroundjoin%
\pgfsetlinewidth{1.505625pt}%
\definecolor{currentstroke}{rgb}{1.000000,0.705882,0.509804}%
\pgfsetstrokecolor{currentstroke}%
\pgfsetstrokeopacity{0.800000}%
\pgfsetdash{}{0pt}%
\pgfpathmoveto{\pgfqpoint{4.735330in}{2.540766in}}%
\pgfpathlineto{\pgfqpoint{3.502867in}{4.072531in}}%
\pgfusepath{stroke}%
\end{pgfscope}%
\begin{pgfscope}%
\pgfpathrectangle{\pgfqpoint{0.481978in}{0.331635in}}{\pgfqpoint{9.300000in}{7.700000in}}%
\pgfusepath{clip}%
\pgfsetrectcap%
\pgfsetroundjoin%
\pgfsetlinewidth{1.505625pt}%
\definecolor{currentstroke}{rgb}{1.000000,0.705882,0.509804}%
\pgfsetstrokecolor{currentstroke}%
\pgfsetstrokeopacity{0.800000}%
\pgfsetdash{}{0pt}%
\pgfpathmoveto{\pgfqpoint{4.464515in}{5.013803in}}%
\pgfpathlineto{\pgfqpoint{3.502867in}{4.072531in}}%
\pgfusepath{stroke}%
\end{pgfscope}%
\begin{pgfscope}%
\pgfpathrectangle{\pgfqpoint{0.481978in}{0.331635in}}{\pgfqpoint{9.300000in}{7.700000in}}%
\pgfusepath{clip}%
\pgfsetrectcap%
\pgfsetroundjoin%
\pgfsetlinewidth{1.505625pt}%
\definecolor{currentstroke}{rgb}{1.000000,0.705882,0.509804}%
\pgfsetstrokecolor{currentstroke}%
\pgfsetstrokeopacity{0.800000}%
\pgfsetdash{}{0pt}%
\pgfpathmoveto{\pgfqpoint{3.760896in}{2.909223in}}%
\pgfpathlineto{\pgfqpoint{3.502867in}{4.072531in}}%
\pgfusepath{stroke}%
\end{pgfscope}%
\begin{pgfscope}%
\pgfpathrectangle{\pgfqpoint{0.481978in}{0.331635in}}{\pgfqpoint{9.300000in}{7.700000in}}%
\pgfusepath{clip}%
\pgfsetrectcap%
\pgfsetroundjoin%
\pgfsetlinewidth{1.505625pt}%
\definecolor{currentstroke}{rgb}{1.000000,0.705882,0.509804}%
\pgfsetstrokecolor{currentstroke}%
\pgfsetstrokeopacity{0.800000}%
\pgfsetdash{}{0pt}%
\pgfpathmoveto{\pgfqpoint{2.336586in}{3.518177in}}%
\pgfpathlineto{\pgfqpoint{3.502867in}{4.072531in}}%
\pgfusepath{stroke}%
\end{pgfscope}%
\begin{pgfscope}%
\pgfpathrectangle{\pgfqpoint{0.481978in}{0.331635in}}{\pgfqpoint{9.300000in}{7.700000in}}%
\pgfusepath{clip}%
\pgfsetrectcap%
\pgfsetroundjoin%
\pgfsetlinewidth{1.505625pt}%
\definecolor{currentstroke}{rgb}{1.000000,0.705882,0.509804}%
\pgfsetstrokecolor{currentstroke}%
\pgfsetstrokeopacity{0.800000}%
\pgfsetdash{}{0pt}%
\pgfpathmoveto{\pgfqpoint{3.224347in}{6.045836in}}%
\pgfpathlineto{\pgfqpoint{3.502867in}{4.072531in}}%
\pgfusepath{stroke}%
\end{pgfscope}%
\begin{pgfscope}%
\pgfpathrectangle{\pgfqpoint{0.481978in}{0.331635in}}{\pgfqpoint{9.300000in}{7.700000in}}%
\pgfusepath{clip}%
\pgfsetrectcap%
\pgfsetroundjoin%
\pgfsetlinewidth{1.505625pt}%
\definecolor{currentstroke}{rgb}{1.000000,0.705882,0.509804}%
\pgfsetstrokecolor{currentstroke}%
\pgfsetstrokeopacity{0.800000}%
\pgfsetdash{}{0pt}%
\pgfpathmoveto{\pgfqpoint{3.710549in}{6.562804in}}%
\pgfpathlineto{\pgfqpoint{3.502867in}{4.072531in}}%
\pgfusepath{stroke}%
\end{pgfscope}%
\begin{pgfscope}%
\pgfpathrectangle{\pgfqpoint{0.481978in}{0.331635in}}{\pgfqpoint{9.300000in}{7.700000in}}%
\pgfusepath{clip}%
\pgfsetrectcap%
\pgfsetroundjoin%
\pgfsetlinewidth{1.505625pt}%
\definecolor{currentstroke}{rgb}{1.000000,0.705882,0.509804}%
\pgfsetstrokecolor{currentstroke}%
\pgfsetstrokeopacity{0.800000}%
\pgfsetdash{}{0pt}%
\pgfpathmoveto{\pgfqpoint{3.320504in}{5.288824in}}%
\pgfpathlineto{\pgfqpoint{3.502867in}{4.072531in}}%
\pgfusepath{stroke}%
\end{pgfscope}%
\begin{pgfscope}%
\pgfpathrectangle{\pgfqpoint{0.481978in}{0.331635in}}{\pgfqpoint{9.300000in}{7.700000in}}%
\pgfusepath{clip}%
\pgfsetrectcap%
\pgfsetroundjoin%
\pgfsetlinewidth{1.505625pt}%
\definecolor{currentstroke}{rgb}{1.000000,0.705882,0.509804}%
\pgfsetstrokecolor{currentstroke}%
\pgfsetstrokeopacity{0.800000}%
\pgfsetdash{}{0pt}%
\pgfpathmoveto{\pgfqpoint{5.389021in}{1.280411in}}%
\pgfpathlineto{\pgfqpoint{3.502867in}{4.072531in}}%
\pgfusepath{stroke}%
\end{pgfscope}%
\begin{pgfscope}%
\pgfpathrectangle{\pgfqpoint{0.481978in}{0.331635in}}{\pgfqpoint{9.300000in}{7.700000in}}%
\pgfusepath{clip}%
\pgfsetrectcap%
\pgfsetroundjoin%
\pgfsetlinewidth{1.505625pt}%
\definecolor{currentstroke}{rgb}{1.000000,0.705882,0.509804}%
\pgfsetstrokecolor{currentstroke}%
\pgfsetstrokeopacity{0.800000}%
\pgfsetdash{}{0pt}%
\pgfpathmoveto{\pgfqpoint{1.225270in}{3.197032in}}%
\pgfpathlineto{\pgfqpoint{3.502867in}{4.072531in}}%
\pgfusepath{stroke}%
\end{pgfscope}%
\begin{pgfscope}%
\pgfpathrectangle{\pgfqpoint{0.481978in}{0.331635in}}{\pgfqpoint{9.300000in}{7.700000in}}%
\pgfusepath{clip}%
\pgfsetrectcap%
\pgfsetroundjoin%
\pgfsetlinewidth{1.505625pt}%
\definecolor{currentstroke}{rgb}{1.000000,0.705882,0.509804}%
\pgfsetstrokecolor{currentstroke}%
\pgfsetstrokeopacity{0.800000}%
\pgfsetdash{}{0pt}%
\pgfpathmoveto{\pgfqpoint{4.019367in}{2.972140in}}%
\pgfpathlineto{\pgfqpoint{3.502867in}{4.072531in}}%
\pgfusepath{stroke}%
\end{pgfscope}%
\begin{pgfscope}%
\pgfpathrectangle{\pgfqpoint{0.481978in}{0.331635in}}{\pgfqpoint{9.300000in}{7.700000in}}%
\pgfusepath{clip}%
\pgfsetrectcap%
\pgfsetroundjoin%
\pgfsetlinewidth{1.505625pt}%
\definecolor{currentstroke}{rgb}{1.000000,0.705882,0.509804}%
\pgfsetstrokecolor{currentstroke}%
\pgfsetstrokeopacity{0.800000}%
\pgfsetdash{}{0pt}%
\pgfpathmoveto{\pgfqpoint{3.098846in}{4.090458in}}%
\pgfpathlineto{\pgfqpoint{3.502867in}{4.072531in}}%
\pgfusepath{stroke}%
\end{pgfscope}%
\begin{pgfscope}%
\pgfpathrectangle{\pgfqpoint{0.481978in}{0.331635in}}{\pgfqpoint{9.300000in}{7.700000in}}%
\pgfusepath{clip}%
\pgfsetrectcap%
\pgfsetroundjoin%
\pgfsetlinewidth{1.505625pt}%
\definecolor{currentstroke}{rgb}{1.000000,0.705882,0.509804}%
\pgfsetstrokecolor{currentstroke}%
\pgfsetstrokeopacity{0.800000}%
\pgfsetdash{}{0pt}%
\pgfpathmoveto{\pgfqpoint{2.491460in}{4.541702in}}%
\pgfpathlineto{\pgfqpoint{3.502867in}{4.072531in}}%
\pgfusepath{stroke}%
\end{pgfscope}%
\begin{pgfscope}%
\pgfpathrectangle{\pgfqpoint{0.481978in}{0.331635in}}{\pgfqpoint{9.300000in}{7.700000in}}%
\pgfusepath{clip}%
\pgfsetrectcap%
\pgfsetroundjoin%
\pgfsetlinewidth{1.505625pt}%
\definecolor{currentstroke}{rgb}{1.000000,0.705882,0.509804}%
\pgfsetstrokecolor{currentstroke}%
\pgfsetstrokeopacity{0.800000}%
\pgfsetdash{}{0pt}%
\pgfpathmoveto{\pgfqpoint{4.722330in}{3.392277in}}%
\pgfpathlineto{\pgfqpoint{3.502867in}{4.072531in}}%
\pgfusepath{stroke}%
\end{pgfscope}%
\begin{pgfscope}%
\pgfpathrectangle{\pgfqpoint{0.481978in}{0.331635in}}{\pgfqpoint{9.300000in}{7.700000in}}%
\pgfusepath{clip}%
\pgfsetrectcap%
\pgfsetroundjoin%
\pgfsetlinewidth{1.505625pt}%
\definecolor{currentstroke}{rgb}{1.000000,0.705882,0.509804}%
\pgfsetstrokecolor{currentstroke}%
\pgfsetstrokeopacity{0.800000}%
\pgfsetdash{}{0pt}%
\pgfpathmoveto{\pgfqpoint{4.962053in}{4.763970in}}%
\pgfpathlineto{\pgfqpoint{3.502867in}{4.072531in}}%
\pgfusepath{stroke}%
\end{pgfscope}%
\begin{pgfscope}%
\pgfpathrectangle{\pgfqpoint{0.481978in}{0.331635in}}{\pgfqpoint{9.300000in}{7.700000in}}%
\pgfusepath{clip}%
\pgfsetrectcap%
\pgfsetroundjoin%
\pgfsetlinewidth{1.505625pt}%
\definecolor{currentstroke}{rgb}{1.000000,0.705882,0.509804}%
\pgfsetstrokecolor{currentstroke}%
\pgfsetstrokeopacity{0.800000}%
\pgfsetdash{}{0pt}%
\pgfpathmoveto{\pgfqpoint{4.756086in}{2.522358in}}%
\pgfpathlineto{\pgfqpoint{3.502867in}{4.072531in}}%
\pgfusepath{stroke}%
\end{pgfscope}%
\begin{pgfscope}%
\pgfpathrectangle{\pgfqpoint{0.481978in}{0.331635in}}{\pgfqpoint{9.300000in}{7.700000in}}%
\pgfusepath{clip}%
\pgfsetrectcap%
\pgfsetroundjoin%
\pgfsetlinewidth{1.505625pt}%
\definecolor{currentstroke}{rgb}{1.000000,0.705882,0.509804}%
\pgfsetstrokecolor{currentstroke}%
\pgfsetstrokeopacity{0.800000}%
\pgfsetdash{}{0pt}%
\pgfpathmoveto{\pgfqpoint{2.604559in}{2.358183in}}%
\pgfpathlineto{\pgfqpoint{3.502867in}{4.072531in}}%
\pgfusepath{stroke}%
\end{pgfscope}%
\begin{pgfscope}%
\pgfpathrectangle{\pgfqpoint{0.481978in}{0.331635in}}{\pgfqpoint{9.300000in}{7.700000in}}%
\pgfusepath{clip}%
\pgfsetrectcap%
\pgfsetroundjoin%
\pgfsetlinewidth{1.505625pt}%
\definecolor{currentstroke}{rgb}{1.000000,0.705882,0.509804}%
\pgfsetstrokecolor{currentstroke}%
\pgfsetstrokeopacity{0.800000}%
\pgfsetdash{}{0pt}%
\pgfpathmoveto{\pgfqpoint{3.564142in}{5.407182in}}%
\pgfpathlineto{\pgfqpoint{3.502867in}{4.072531in}}%
\pgfusepath{stroke}%
\end{pgfscope}%
\begin{pgfscope}%
\pgfpathrectangle{\pgfqpoint{0.481978in}{0.331635in}}{\pgfqpoint{9.300000in}{7.700000in}}%
\pgfusepath{clip}%
\pgfsetrectcap%
\pgfsetroundjoin%
\pgfsetlinewidth{1.505625pt}%
\definecolor{currentstroke}{rgb}{1.000000,0.705882,0.509804}%
\pgfsetstrokecolor{currentstroke}%
\pgfsetstrokeopacity{0.800000}%
\pgfsetdash{}{0pt}%
\pgfpathmoveto{\pgfqpoint{3.840892in}{5.632476in}}%
\pgfpathlineto{\pgfqpoint{3.502867in}{4.072531in}}%
\pgfusepath{stroke}%
\end{pgfscope}%
\begin{pgfscope}%
\pgfpathrectangle{\pgfqpoint{0.481978in}{0.331635in}}{\pgfqpoint{9.300000in}{7.700000in}}%
\pgfusepath{clip}%
\pgfsetrectcap%
\pgfsetroundjoin%
\pgfsetlinewidth{1.505625pt}%
\definecolor{currentstroke}{rgb}{1.000000,0.705882,0.509804}%
\pgfsetstrokecolor{currentstroke}%
\pgfsetstrokeopacity{0.800000}%
\pgfsetdash{}{0pt}%
\pgfpathmoveto{\pgfqpoint{5.376976in}{4.653290in}}%
\pgfpathlineto{\pgfqpoint{3.502867in}{4.072531in}}%
\pgfusepath{stroke}%
\end{pgfscope}%
\begin{pgfscope}%
\pgfpathrectangle{\pgfqpoint{0.481978in}{0.331635in}}{\pgfqpoint{9.300000in}{7.700000in}}%
\pgfusepath{clip}%
\pgfsetrectcap%
\pgfsetroundjoin%
\pgfsetlinewidth{1.505625pt}%
\definecolor{currentstroke}{rgb}{1.000000,0.705882,0.509804}%
\pgfsetstrokecolor{currentstroke}%
\pgfsetstrokeopacity{0.800000}%
\pgfsetdash{}{0pt}%
\pgfpathmoveto{\pgfqpoint{4.185210in}{4.218529in}}%
\pgfpathlineto{\pgfqpoint{3.502867in}{4.072531in}}%
\pgfusepath{stroke}%
\end{pgfscope}%
\begin{pgfscope}%
\pgfpathrectangle{\pgfqpoint{0.481978in}{0.331635in}}{\pgfqpoint{9.300000in}{7.700000in}}%
\pgfusepath{clip}%
\pgfsetrectcap%
\pgfsetroundjoin%
\pgfsetlinewidth{1.505625pt}%
\definecolor{currentstroke}{rgb}{1.000000,0.705882,0.509804}%
\pgfsetstrokecolor{currentstroke}%
\pgfsetstrokeopacity{0.800000}%
\pgfsetdash{}{0pt}%
\pgfpathmoveto{\pgfqpoint{4.500422in}{3.044916in}}%
\pgfpathlineto{\pgfqpoint{3.502867in}{4.072531in}}%
\pgfusepath{stroke}%
\end{pgfscope}%
\begin{pgfscope}%
\pgfpathrectangle{\pgfqpoint{0.481978in}{0.331635in}}{\pgfqpoint{9.300000in}{7.700000in}}%
\pgfusepath{clip}%
\pgfsetrectcap%
\pgfsetroundjoin%
\pgfsetlinewidth{1.505625pt}%
\definecolor{currentstroke}{rgb}{1.000000,0.705882,0.509804}%
\pgfsetstrokecolor{currentstroke}%
\pgfsetstrokeopacity{0.800000}%
\pgfsetdash{}{0pt}%
\pgfpathmoveto{\pgfqpoint{4.103119in}{4.294356in}}%
\pgfpathlineto{\pgfqpoint{3.502867in}{4.072531in}}%
\pgfusepath{stroke}%
\end{pgfscope}%
\begin{pgfscope}%
\pgfpathrectangle{\pgfqpoint{0.481978in}{0.331635in}}{\pgfqpoint{9.300000in}{7.700000in}}%
\pgfusepath{clip}%
\pgfsetrectcap%
\pgfsetroundjoin%
\pgfsetlinewidth{1.505625pt}%
\definecolor{currentstroke}{rgb}{1.000000,0.705882,0.509804}%
\pgfsetstrokecolor{currentstroke}%
\pgfsetstrokeopacity{0.800000}%
\pgfsetdash{}{0pt}%
\pgfpathmoveto{\pgfqpoint{3.235019in}{4.506980in}}%
\pgfpathlineto{\pgfqpoint{3.502867in}{4.072531in}}%
\pgfusepath{stroke}%
\end{pgfscope}%
\begin{pgfscope}%
\pgfpathrectangle{\pgfqpoint{0.481978in}{0.331635in}}{\pgfqpoint{9.300000in}{7.700000in}}%
\pgfusepath{clip}%
\pgfsetrectcap%
\pgfsetroundjoin%
\pgfsetlinewidth{1.505625pt}%
\definecolor{currentstroke}{rgb}{1.000000,0.705882,0.509804}%
\pgfsetstrokecolor{currentstroke}%
\pgfsetstrokeopacity{0.800000}%
\pgfsetdash{}{0pt}%
\pgfpathmoveto{\pgfqpoint{1.466431in}{3.111586in}}%
\pgfpathlineto{\pgfqpoint{3.502867in}{4.072531in}}%
\pgfusepath{stroke}%
\end{pgfscope}%
\begin{pgfscope}%
\pgfpathrectangle{\pgfqpoint{0.481978in}{0.331635in}}{\pgfqpoint{9.300000in}{7.700000in}}%
\pgfusepath{clip}%
\pgfsetrectcap%
\pgfsetroundjoin%
\pgfsetlinewidth{1.505625pt}%
\definecolor{currentstroke}{rgb}{1.000000,0.705882,0.509804}%
\pgfsetstrokecolor{currentstroke}%
\pgfsetstrokeopacity{0.800000}%
\pgfsetdash{}{0pt}%
\pgfpathmoveto{\pgfqpoint{5.649061in}{4.482985in}}%
\pgfpathlineto{\pgfqpoint{3.502867in}{4.072531in}}%
\pgfusepath{stroke}%
\end{pgfscope}%
\begin{pgfscope}%
\pgfpathrectangle{\pgfqpoint{0.481978in}{0.331635in}}{\pgfqpoint{9.300000in}{7.700000in}}%
\pgfusepath{clip}%
\pgfsetrectcap%
\pgfsetroundjoin%
\pgfsetlinewidth{1.505625pt}%
\definecolor{currentstroke}{rgb}{1.000000,0.705882,0.509804}%
\pgfsetstrokecolor{currentstroke}%
\pgfsetstrokeopacity{0.800000}%
\pgfsetdash{}{0pt}%
\pgfpathmoveto{\pgfqpoint{3.679062in}{2.692966in}}%
\pgfpathlineto{\pgfqpoint{3.502867in}{4.072531in}}%
\pgfusepath{stroke}%
\end{pgfscope}%
\begin{pgfscope}%
\pgfpathrectangle{\pgfqpoint{0.481978in}{0.331635in}}{\pgfqpoint{9.300000in}{7.700000in}}%
\pgfusepath{clip}%
\pgfsetrectcap%
\pgfsetroundjoin%
\pgfsetlinewidth{1.505625pt}%
\definecolor{currentstroke}{rgb}{1.000000,0.705882,0.509804}%
\pgfsetstrokecolor{currentstroke}%
\pgfsetstrokeopacity{0.800000}%
\pgfsetdash{}{0pt}%
\pgfpathmoveto{\pgfqpoint{2.616545in}{4.569462in}}%
\pgfpathlineto{\pgfqpoint{3.502867in}{4.072531in}}%
\pgfusepath{stroke}%
\end{pgfscope}%
\begin{pgfscope}%
\pgfpathrectangle{\pgfqpoint{0.481978in}{0.331635in}}{\pgfqpoint{9.300000in}{7.700000in}}%
\pgfusepath{clip}%
\pgfsetrectcap%
\pgfsetroundjoin%
\pgfsetlinewidth{1.505625pt}%
\definecolor{currentstroke}{rgb}{1.000000,0.705882,0.509804}%
\pgfsetstrokecolor{currentstroke}%
\pgfsetstrokeopacity{0.800000}%
\pgfsetdash{}{0pt}%
\pgfpathmoveto{\pgfqpoint{5.192724in}{5.029818in}}%
\pgfpathlineto{\pgfqpoint{3.502867in}{4.072531in}}%
\pgfusepath{stroke}%
\end{pgfscope}%
\begin{pgfscope}%
\pgfpathrectangle{\pgfqpoint{0.481978in}{0.331635in}}{\pgfqpoint{9.300000in}{7.700000in}}%
\pgfusepath{clip}%
\pgfsetrectcap%
\pgfsetroundjoin%
\pgfsetlinewidth{1.505625pt}%
\definecolor{currentstroke}{rgb}{1.000000,0.705882,0.509804}%
\pgfsetstrokecolor{currentstroke}%
\pgfsetstrokeopacity{0.800000}%
\pgfsetdash{}{0pt}%
\pgfpathmoveto{\pgfqpoint{3.374008in}{2.987068in}}%
\pgfpathlineto{\pgfqpoint{3.502867in}{4.072531in}}%
\pgfusepath{stroke}%
\end{pgfscope}%
\begin{pgfscope}%
\pgfpathrectangle{\pgfqpoint{0.481978in}{0.331635in}}{\pgfqpoint{9.300000in}{7.700000in}}%
\pgfusepath{clip}%
\pgfsetrectcap%
\pgfsetroundjoin%
\pgfsetlinewidth{1.505625pt}%
\definecolor{currentstroke}{rgb}{1.000000,0.705882,0.509804}%
\pgfsetstrokecolor{currentstroke}%
\pgfsetstrokeopacity{0.800000}%
\pgfsetdash{}{0pt}%
\pgfpathmoveto{\pgfqpoint{4.257082in}{4.318425in}}%
\pgfpathlineto{\pgfqpoint{3.502867in}{4.072531in}}%
\pgfusepath{stroke}%
\end{pgfscope}%
\begin{pgfscope}%
\pgfpathrectangle{\pgfqpoint{0.481978in}{0.331635in}}{\pgfqpoint{9.300000in}{7.700000in}}%
\pgfusepath{clip}%
\pgfsetrectcap%
\pgfsetroundjoin%
\pgfsetlinewidth{1.505625pt}%
\definecolor{currentstroke}{rgb}{1.000000,0.705882,0.509804}%
\pgfsetstrokecolor{currentstroke}%
\pgfsetstrokeopacity{0.800000}%
\pgfsetdash{}{0pt}%
\pgfpathmoveto{\pgfqpoint{1.027576in}{3.786004in}}%
\pgfpathlineto{\pgfqpoint{3.502867in}{4.072531in}}%
\pgfusepath{stroke}%
\end{pgfscope}%
\begin{pgfscope}%
\pgfpathrectangle{\pgfqpoint{0.481978in}{0.331635in}}{\pgfqpoint{9.300000in}{7.700000in}}%
\pgfusepath{clip}%
\pgfsetrectcap%
\pgfsetroundjoin%
\pgfsetlinewidth{1.505625pt}%
\definecolor{currentstroke}{rgb}{1.000000,0.705882,0.509804}%
\pgfsetstrokecolor{currentstroke}%
\pgfsetstrokeopacity{0.800000}%
\pgfsetdash{}{0pt}%
\pgfpathmoveto{\pgfqpoint{2.111519in}{4.640866in}}%
\pgfpathlineto{\pgfqpoint{3.502867in}{4.072531in}}%
\pgfusepath{stroke}%
\end{pgfscope}%
\begin{pgfscope}%
\pgfpathrectangle{\pgfqpoint{0.481978in}{0.331635in}}{\pgfqpoint{9.300000in}{7.700000in}}%
\pgfusepath{clip}%
\pgfsetrectcap%
\pgfsetroundjoin%
\pgfsetlinewidth{1.505625pt}%
\definecolor{currentstroke}{rgb}{1.000000,0.705882,0.509804}%
\pgfsetstrokecolor{currentstroke}%
\pgfsetstrokeopacity{0.800000}%
\pgfsetdash{}{0pt}%
\pgfpathmoveto{\pgfqpoint{2.540709in}{6.147797in}}%
\pgfpathlineto{\pgfqpoint{3.502867in}{4.072531in}}%
\pgfusepath{stroke}%
\end{pgfscope}%
\begin{pgfscope}%
\pgfpathrectangle{\pgfqpoint{0.481978in}{0.331635in}}{\pgfqpoint{9.300000in}{7.700000in}}%
\pgfusepath{clip}%
\pgfsetrectcap%
\pgfsetroundjoin%
\pgfsetlinewidth{1.505625pt}%
\definecolor{currentstroke}{rgb}{1.000000,0.705882,0.509804}%
\pgfsetstrokecolor{currentstroke}%
\pgfsetstrokeopacity{0.800000}%
\pgfsetdash{}{0pt}%
\pgfpathmoveto{\pgfqpoint{2.593367in}{2.364960in}}%
\pgfpathlineto{\pgfqpoint{3.502867in}{4.072531in}}%
\pgfusepath{stroke}%
\end{pgfscope}%
\begin{pgfscope}%
\pgfpathrectangle{\pgfqpoint{0.481978in}{0.331635in}}{\pgfqpoint{9.300000in}{7.700000in}}%
\pgfusepath{clip}%
\pgfsetrectcap%
\pgfsetroundjoin%
\pgfsetlinewidth{1.505625pt}%
\definecolor{currentstroke}{rgb}{1.000000,0.705882,0.509804}%
\pgfsetstrokecolor{currentstroke}%
\pgfsetstrokeopacity{0.800000}%
\pgfsetdash{}{0pt}%
\pgfpathmoveto{\pgfqpoint{1.941607in}{3.957736in}}%
\pgfpathlineto{\pgfqpoint{3.502867in}{4.072531in}}%
\pgfusepath{stroke}%
\end{pgfscope}%
\begin{pgfscope}%
\pgfpathrectangle{\pgfqpoint{0.481978in}{0.331635in}}{\pgfqpoint{9.300000in}{7.700000in}}%
\pgfusepath{clip}%
\pgfsetrectcap%
\pgfsetroundjoin%
\pgfsetlinewidth{1.505625pt}%
\definecolor{currentstroke}{rgb}{1.000000,0.705882,0.509804}%
\pgfsetstrokecolor{currentstroke}%
\pgfsetstrokeopacity{0.800000}%
\pgfsetdash{}{0pt}%
\pgfpathmoveto{\pgfqpoint{2.880395in}{3.314771in}}%
\pgfpathlineto{\pgfqpoint{3.502867in}{4.072531in}}%
\pgfusepath{stroke}%
\end{pgfscope}%
\begin{pgfscope}%
\pgfpathrectangle{\pgfqpoint{0.481978in}{0.331635in}}{\pgfqpoint{9.300000in}{7.700000in}}%
\pgfusepath{clip}%
\pgfsetrectcap%
\pgfsetroundjoin%
\pgfsetlinewidth{1.505625pt}%
\definecolor{currentstroke}{rgb}{1.000000,0.705882,0.509804}%
\pgfsetstrokecolor{currentstroke}%
\pgfsetstrokeopacity{0.800000}%
\pgfsetdash{}{0pt}%
\pgfpathmoveto{\pgfqpoint{3.084100in}{2.952086in}}%
\pgfpathlineto{\pgfqpoint{3.502867in}{4.072531in}}%
\pgfusepath{stroke}%
\end{pgfscope}%
\begin{pgfscope}%
\pgfpathrectangle{\pgfqpoint{0.481978in}{0.331635in}}{\pgfqpoint{9.300000in}{7.700000in}}%
\pgfusepath{clip}%
\pgfsetrectcap%
\pgfsetroundjoin%
\pgfsetlinewidth{1.505625pt}%
\definecolor{currentstroke}{rgb}{1.000000,0.705882,0.509804}%
\pgfsetstrokecolor{currentstroke}%
\pgfsetstrokeopacity{0.800000}%
\pgfsetdash{}{0pt}%
\pgfpathmoveto{\pgfqpoint{5.368141in}{3.038043in}}%
\pgfpathlineto{\pgfqpoint{3.502867in}{4.072531in}}%
\pgfusepath{stroke}%
\end{pgfscope}%
\begin{pgfscope}%
\pgfpathrectangle{\pgfqpoint{0.481978in}{0.331635in}}{\pgfqpoint{9.300000in}{7.700000in}}%
\pgfusepath{clip}%
\pgfsetrectcap%
\pgfsetroundjoin%
\pgfsetlinewidth{1.505625pt}%
\definecolor{currentstroke}{rgb}{1.000000,0.705882,0.509804}%
\pgfsetstrokecolor{currentstroke}%
\pgfsetstrokeopacity{0.800000}%
\pgfsetdash{}{0pt}%
\pgfpathmoveto{\pgfqpoint{2.886916in}{3.927373in}}%
\pgfpathlineto{\pgfqpoint{3.502867in}{4.072531in}}%
\pgfusepath{stroke}%
\end{pgfscope}%
\begin{pgfscope}%
\pgfpathrectangle{\pgfqpoint{0.481978in}{0.331635in}}{\pgfqpoint{9.300000in}{7.700000in}}%
\pgfusepath{clip}%
\pgfsetrectcap%
\pgfsetroundjoin%
\pgfsetlinewidth{1.505625pt}%
\definecolor{currentstroke}{rgb}{1.000000,0.705882,0.509804}%
\pgfsetstrokecolor{currentstroke}%
\pgfsetstrokeopacity{0.800000}%
\pgfsetdash{}{0pt}%
\pgfpathmoveto{\pgfqpoint{4.450031in}{2.773919in}}%
\pgfpathlineto{\pgfqpoint{3.502867in}{4.072531in}}%
\pgfusepath{stroke}%
\end{pgfscope}%
\begin{pgfscope}%
\pgfpathrectangle{\pgfqpoint{0.481978in}{0.331635in}}{\pgfqpoint{9.300000in}{7.700000in}}%
\pgfusepath{clip}%
\pgfsetrectcap%
\pgfsetroundjoin%
\pgfsetlinewidth{1.505625pt}%
\definecolor{currentstroke}{rgb}{1.000000,0.705882,0.509804}%
\pgfsetstrokecolor{currentstroke}%
\pgfsetstrokeopacity{0.800000}%
\pgfsetdash{}{0pt}%
\pgfpathmoveto{\pgfqpoint{2.491094in}{5.202939in}}%
\pgfpathlineto{\pgfqpoint{3.502867in}{4.072531in}}%
\pgfusepath{stroke}%
\end{pgfscope}%
\begin{pgfscope}%
\pgfpathrectangle{\pgfqpoint{0.481978in}{0.331635in}}{\pgfqpoint{9.300000in}{7.700000in}}%
\pgfusepath{clip}%
\pgfsetrectcap%
\pgfsetroundjoin%
\pgfsetlinewidth{1.505625pt}%
\definecolor{currentstroke}{rgb}{1.000000,0.705882,0.509804}%
\pgfsetstrokecolor{currentstroke}%
\pgfsetstrokeopacity{0.800000}%
\pgfsetdash{}{0pt}%
\pgfpathmoveto{\pgfqpoint{2.662048in}{3.622560in}}%
\pgfpathlineto{\pgfqpoint{3.502867in}{4.072531in}}%
\pgfusepath{stroke}%
\end{pgfscope}%
\begin{pgfscope}%
\pgfpathrectangle{\pgfqpoint{0.481978in}{0.331635in}}{\pgfqpoint{9.300000in}{7.700000in}}%
\pgfusepath{clip}%
\pgfsetrectcap%
\pgfsetroundjoin%
\pgfsetlinewidth{1.505625pt}%
\definecolor{currentstroke}{rgb}{1.000000,0.705882,0.509804}%
\pgfsetstrokecolor{currentstroke}%
\pgfsetstrokeopacity{0.800000}%
\pgfsetdash{}{0pt}%
\pgfpathmoveto{\pgfqpoint{4.025822in}{3.612427in}}%
\pgfpathlineto{\pgfqpoint{3.502867in}{4.072531in}}%
\pgfusepath{stroke}%
\end{pgfscope}%
\begin{pgfscope}%
\pgfpathrectangle{\pgfqpoint{0.481978in}{0.331635in}}{\pgfqpoint{9.300000in}{7.700000in}}%
\pgfusepath{clip}%
\pgfsetrectcap%
\pgfsetroundjoin%
\pgfsetlinewidth{1.505625pt}%
\definecolor{currentstroke}{rgb}{1.000000,0.705882,0.509804}%
\pgfsetstrokecolor{currentstroke}%
\pgfsetstrokeopacity{0.800000}%
\pgfsetdash{}{0pt}%
\pgfpathmoveto{\pgfqpoint{3.490511in}{5.096925in}}%
\pgfpathlineto{\pgfqpoint{3.502867in}{4.072531in}}%
\pgfusepath{stroke}%
\end{pgfscope}%
\begin{pgfscope}%
\pgfpathrectangle{\pgfqpoint{0.481978in}{0.331635in}}{\pgfqpoint{9.300000in}{7.700000in}}%
\pgfusepath{clip}%
\pgfsetrectcap%
\pgfsetroundjoin%
\pgfsetlinewidth{1.505625pt}%
\definecolor{currentstroke}{rgb}{1.000000,0.705882,0.509804}%
\pgfsetstrokecolor{currentstroke}%
\pgfsetstrokeopacity{0.800000}%
\pgfsetdash{}{0pt}%
\pgfpathmoveto{\pgfqpoint{3.394544in}{3.192662in}}%
\pgfpathlineto{\pgfqpoint{3.502867in}{4.072531in}}%
\pgfusepath{stroke}%
\end{pgfscope}%
\begin{pgfscope}%
\pgfpathrectangle{\pgfqpoint{0.481978in}{0.331635in}}{\pgfqpoint{9.300000in}{7.700000in}}%
\pgfusepath{clip}%
\pgfsetrectcap%
\pgfsetroundjoin%
\pgfsetlinewidth{1.505625pt}%
\definecolor{currentstroke}{rgb}{1.000000,0.705882,0.509804}%
\pgfsetstrokecolor{currentstroke}%
\pgfsetstrokeopacity{0.800000}%
\pgfsetdash{}{0pt}%
\pgfpathmoveto{\pgfqpoint{3.295196in}{3.673341in}}%
\pgfpathlineto{\pgfqpoint{3.502867in}{4.072531in}}%
\pgfusepath{stroke}%
\end{pgfscope}%
\begin{pgfscope}%
\pgfpathrectangle{\pgfqpoint{0.481978in}{0.331635in}}{\pgfqpoint{9.300000in}{7.700000in}}%
\pgfusepath{clip}%
\pgfsetrectcap%
\pgfsetroundjoin%
\pgfsetlinewidth{1.505625pt}%
\definecolor{currentstroke}{rgb}{1.000000,0.705882,0.509804}%
\pgfsetstrokecolor{currentstroke}%
\pgfsetstrokeopacity{0.800000}%
\pgfsetdash{}{0pt}%
\pgfpathmoveto{\pgfqpoint{3.905090in}{5.004749in}}%
\pgfpathlineto{\pgfqpoint{3.502867in}{4.072531in}}%
\pgfusepath{stroke}%
\end{pgfscope}%
\begin{pgfscope}%
\pgfpathrectangle{\pgfqpoint{0.481978in}{0.331635in}}{\pgfqpoint{9.300000in}{7.700000in}}%
\pgfusepath{clip}%
\pgfsetrectcap%
\pgfsetroundjoin%
\pgfsetlinewidth{1.505625pt}%
\definecolor{currentstroke}{rgb}{1.000000,0.705882,0.509804}%
\pgfsetstrokecolor{currentstroke}%
\pgfsetstrokeopacity{0.800000}%
\pgfsetdash{}{0pt}%
\pgfpathmoveto{\pgfqpoint{4.067534in}{3.403561in}}%
\pgfpathlineto{\pgfqpoint{3.502867in}{4.072531in}}%
\pgfusepath{stroke}%
\end{pgfscope}%
\begin{pgfscope}%
\pgfpathrectangle{\pgfqpoint{0.481978in}{0.331635in}}{\pgfqpoint{9.300000in}{7.700000in}}%
\pgfusepath{clip}%
\pgfsetrectcap%
\pgfsetroundjoin%
\pgfsetlinewidth{1.505625pt}%
\definecolor{currentstroke}{rgb}{1.000000,0.705882,0.509804}%
\pgfsetstrokecolor{currentstroke}%
\pgfsetstrokeopacity{0.800000}%
\pgfsetdash{}{0pt}%
\pgfpathmoveto{\pgfqpoint{1.713787in}{3.877978in}}%
\pgfpathlineto{\pgfqpoint{3.502867in}{4.072531in}}%
\pgfusepath{stroke}%
\end{pgfscope}%
\begin{pgfscope}%
\pgfpathrectangle{\pgfqpoint{0.481978in}{0.331635in}}{\pgfqpoint{9.300000in}{7.700000in}}%
\pgfusepath{clip}%
\pgfsetrectcap%
\pgfsetroundjoin%
\pgfsetlinewidth{1.505625pt}%
\definecolor{currentstroke}{rgb}{1.000000,0.705882,0.509804}%
\pgfsetstrokecolor{currentstroke}%
\pgfsetstrokeopacity{0.800000}%
\pgfsetdash{}{0pt}%
\pgfpathmoveto{\pgfqpoint{2.059946in}{4.511724in}}%
\pgfpathlineto{\pgfqpoint{3.502867in}{4.072531in}}%
\pgfusepath{stroke}%
\end{pgfscope}%
\begin{pgfscope}%
\pgfpathrectangle{\pgfqpoint{0.481978in}{0.331635in}}{\pgfqpoint{9.300000in}{7.700000in}}%
\pgfusepath{clip}%
\pgfsetrectcap%
\pgfsetroundjoin%
\pgfsetlinewidth{1.505625pt}%
\definecolor{currentstroke}{rgb}{1.000000,0.705882,0.509804}%
\pgfsetstrokecolor{currentstroke}%
\pgfsetstrokeopacity{0.800000}%
\pgfsetdash{}{0pt}%
\pgfpathmoveto{\pgfqpoint{3.097846in}{5.855995in}}%
\pgfpathlineto{\pgfqpoint{3.502867in}{4.072531in}}%
\pgfusepath{stroke}%
\end{pgfscope}%
\begin{pgfscope}%
\pgfpathrectangle{\pgfqpoint{0.481978in}{0.331635in}}{\pgfqpoint{9.300000in}{7.700000in}}%
\pgfusepath{clip}%
\pgfsetrectcap%
\pgfsetroundjoin%
\pgfsetlinewidth{1.505625pt}%
\definecolor{currentstroke}{rgb}{1.000000,0.705882,0.509804}%
\pgfsetstrokecolor{currentstroke}%
\pgfsetstrokeopacity{0.800000}%
\pgfsetdash{}{0pt}%
\pgfpathmoveto{\pgfqpoint{2.416512in}{2.710548in}}%
\pgfpathlineto{\pgfqpoint{3.502867in}{4.072531in}}%
\pgfusepath{stroke}%
\end{pgfscope}%
\begin{pgfscope}%
\pgfpathrectangle{\pgfqpoint{0.481978in}{0.331635in}}{\pgfqpoint{9.300000in}{7.700000in}}%
\pgfusepath{clip}%
\pgfsetrectcap%
\pgfsetroundjoin%
\pgfsetlinewidth{1.505625pt}%
\definecolor{currentstroke}{rgb}{1.000000,0.705882,0.509804}%
\pgfsetstrokecolor{currentstroke}%
\pgfsetstrokeopacity{0.800000}%
\pgfsetdash{}{0pt}%
\pgfpathmoveto{\pgfqpoint{1.548965in}{4.260170in}}%
\pgfpathlineto{\pgfqpoint{3.502867in}{4.072531in}}%
\pgfusepath{stroke}%
\end{pgfscope}%
\begin{pgfscope}%
\pgfpathrectangle{\pgfqpoint{0.481978in}{0.331635in}}{\pgfqpoint{9.300000in}{7.700000in}}%
\pgfusepath{clip}%
\pgfsetrectcap%
\pgfsetroundjoin%
\pgfsetlinewidth{1.505625pt}%
\definecolor{currentstroke}{rgb}{1.000000,0.705882,0.509804}%
\pgfsetstrokecolor{currentstroke}%
\pgfsetstrokeopacity{0.800000}%
\pgfsetdash{}{0pt}%
\pgfpathmoveto{\pgfqpoint{4.223289in}{4.641562in}}%
\pgfpathlineto{\pgfqpoint{3.502867in}{4.072531in}}%
\pgfusepath{stroke}%
\end{pgfscope}%
\begin{pgfscope}%
\pgfpathrectangle{\pgfqpoint{0.481978in}{0.331635in}}{\pgfqpoint{9.300000in}{7.700000in}}%
\pgfusepath{clip}%
\pgfsetrectcap%
\pgfsetroundjoin%
\pgfsetlinewidth{1.505625pt}%
\definecolor{currentstroke}{rgb}{1.000000,0.705882,0.509804}%
\pgfsetstrokecolor{currentstroke}%
\pgfsetstrokeopacity{0.800000}%
\pgfsetdash{}{0pt}%
\pgfpathmoveto{\pgfqpoint{3.078663in}{3.602477in}}%
\pgfpathlineto{\pgfqpoint{3.502867in}{4.072531in}}%
\pgfusepath{stroke}%
\end{pgfscope}%
\begin{pgfscope}%
\pgfpathrectangle{\pgfqpoint{0.481978in}{0.331635in}}{\pgfqpoint{9.300000in}{7.700000in}}%
\pgfusepath{clip}%
\pgfsetrectcap%
\pgfsetroundjoin%
\pgfsetlinewidth{1.505625pt}%
\definecolor{currentstroke}{rgb}{1.000000,0.705882,0.509804}%
\pgfsetstrokecolor{currentstroke}%
\pgfsetstrokeopacity{0.800000}%
\pgfsetdash{}{0pt}%
\pgfpathmoveto{\pgfqpoint{4.163791in}{4.059342in}}%
\pgfpathlineto{\pgfqpoint{3.502867in}{4.072531in}}%
\pgfusepath{stroke}%
\end{pgfscope}%
\begin{pgfscope}%
\pgfpathrectangle{\pgfqpoint{0.481978in}{0.331635in}}{\pgfqpoint{9.300000in}{7.700000in}}%
\pgfusepath{clip}%
\pgfsetrectcap%
\pgfsetroundjoin%
\pgfsetlinewidth{1.505625pt}%
\definecolor{currentstroke}{rgb}{1.000000,0.705882,0.509804}%
\pgfsetstrokecolor{currentstroke}%
\pgfsetstrokeopacity{0.800000}%
\pgfsetdash{}{0pt}%
\pgfpathmoveto{\pgfqpoint{1.998764in}{4.830594in}}%
\pgfpathlineto{\pgfqpoint{3.502867in}{4.072531in}}%
\pgfusepath{stroke}%
\end{pgfscope}%
\begin{pgfscope}%
\pgfpathrectangle{\pgfqpoint{0.481978in}{0.331635in}}{\pgfqpoint{9.300000in}{7.700000in}}%
\pgfusepath{clip}%
\pgfsetrectcap%
\pgfsetroundjoin%
\pgfsetlinewidth{1.505625pt}%
\definecolor{currentstroke}{rgb}{1.000000,0.705882,0.509804}%
\pgfsetstrokecolor{currentstroke}%
\pgfsetstrokeopacity{0.800000}%
\pgfsetdash{}{0pt}%
\pgfpathmoveto{\pgfqpoint{4.371217in}{3.374706in}}%
\pgfpathlineto{\pgfqpoint{3.502867in}{4.072531in}}%
\pgfusepath{stroke}%
\end{pgfscope}%
\begin{pgfscope}%
\pgfpathrectangle{\pgfqpoint{0.481978in}{0.331635in}}{\pgfqpoint{9.300000in}{7.700000in}}%
\pgfusepath{clip}%
\pgfsetrectcap%
\pgfsetroundjoin%
\pgfsetlinewidth{1.505625pt}%
\definecolor{currentstroke}{rgb}{1.000000,0.705882,0.509804}%
\pgfsetstrokecolor{currentstroke}%
\pgfsetstrokeopacity{0.800000}%
\pgfsetdash{}{0pt}%
\pgfpathmoveto{\pgfqpoint{2.452590in}{4.246526in}}%
\pgfpathlineto{\pgfqpoint{3.502867in}{4.072531in}}%
\pgfusepath{stroke}%
\end{pgfscope}%
\begin{pgfscope}%
\pgfpathrectangle{\pgfqpoint{0.481978in}{0.331635in}}{\pgfqpoint{9.300000in}{7.700000in}}%
\pgfusepath{clip}%
\pgfsetrectcap%
\pgfsetroundjoin%
\pgfsetlinewidth{1.505625pt}%
\definecolor{currentstroke}{rgb}{1.000000,0.705882,0.509804}%
\pgfsetstrokecolor{currentstroke}%
\pgfsetstrokeopacity{0.800000}%
\pgfsetdash{}{0pt}%
\pgfpathmoveto{\pgfqpoint{4.281738in}{4.827328in}}%
\pgfpathlineto{\pgfqpoint{3.502867in}{4.072531in}}%
\pgfusepath{stroke}%
\end{pgfscope}%
\begin{pgfscope}%
\pgfpathrectangle{\pgfqpoint{0.481978in}{0.331635in}}{\pgfqpoint{9.300000in}{7.700000in}}%
\pgfusepath{clip}%
\pgfsetrectcap%
\pgfsetroundjoin%
\pgfsetlinewidth{1.505625pt}%
\definecolor{currentstroke}{rgb}{1.000000,0.705882,0.509804}%
\pgfsetstrokecolor{currentstroke}%
\pgfsetstrokeopacity{0.800000}%
\pgfsetdash{}{0pt}%
\pgfpathmoveto{\pgfqpoint{3.234311in}{3.438479in}}%
\pgfpathlineto{\pgfqpoint{3.502867in}{4.072531in}}%
\pgfusepath{stroke}%
\end{pgfscope}%
\begin{pgfscope}%
\pgfpathrectangle{\pgfqpoint{0.481978in}{0.331635in}}{\pgfqpoint{9.300000in}{7.700000in}}%
\pgfusepath{clip}%
\pgfsetrectcap%
\pgfsetroundjoin%
\pgfsetlinewidth{1.505625pt}%
\definecolor{currentstroke}{rgb}{1.000000,0.705882,0.509804}%
\pgfsetstrokecolor{currentstroke}%
\pgfsetstrokeopacity{0.800000}%
\pgfsetdash{}{0pt}%
\pgfpathmoveto{\pgfqpoint{3.915243in}{6.696360in}}%
\pgfpathlineto{\pgfqpoint{3.502867in}{4.072531in}}%
\pgfusepath{stroke}%
\end{pgfscope}%
\begin{pgfscope}%
\pgfpathrectangle{\pgfqpoint{0.481978in}{0.331635in}}{\pgfqpoint{9.300000in}{7.700000in}}%
\pgfusepath{clip}%
\pgfsetrectcap%
\pgfsetroundjoin%
\pgfsetlinewidth{1.505625pt}%
\definecolor{currentstroke}{rgb}{1.000000,0.705882,0.509804}%
\pgfsetstrokecolor{currentstroke}%
\pgfsetstrokeopacity{0.800000}%
\pgfsetdash{}{0pt}%
\pgfpathmoveto{\pgfqpoint{4.358866in}{3.185142in}}%
\pgfpathlineto{\pgfqpoint{3.502867in}{4.072531in}}%
\pgfusepath{stroke}%
\end{pgfscope}%
\begin{pgfscope}%
\pgfpathrectangle{\pgfqpoint{0.481978in}{0.331635in}}{\pgfqpoint{9.300000in}{7.700000in}}%
\pgfusepath{clip}%
\pgfsetrectcap%
\pgfsetroundjoin%
\pgfsetlinewidth{1.505625pt}%
\definecolor{currentstroke}{rgb}{1.000000,0.705882,0.509804}%
\pgfsetstrokecolor{currentstroke}%
\pgfsetstrokeopacity{0.800000}%
\pgfsetdash{}{0pt}%
\pgfpathmoveto{\pgfqpoint{4.512704in}{5.588863in}}%
\pgfpathlineto{\pgfqpoint{3.502867in}{4.072531in}}%
\pgfusepath{stroke}%
\end{pgfscope}%
\begin{pgfscope}%
\pgfpathrectangle{\pgfqpoint{0.481978in}{0.331635in}}{\pgfqpoint{9.300000in}{7.700000in}}%
\pgfusepath{clip}%
\pgfsetrectcap%
\pgfsetroundjoin%
\pgfsetlinewidth{1.505625pt}%
\definecolor{currentstroke}{rgb}{1.000000,0.705882,0.509804}%
\pgfsetstrokecolor{currentstroke}%
\pgfsetstrokeopacity{0.800000}%
\pgfsetdash{}{0pt}%
\pgfpathmoveto{\pgfqpoint{2.811663in}{4.156462in}}%
\pgfpathlineto{\pgfqpoint{3.502867in}{4.072531in}}%
\pgfusepath{stroke}%
\end{pgfscope}%
\begin{pgfscope}%
\pgfpathrectangle{\pgfqpoint{0.481978in}{0.331635in}}{\pgfqpoint{9.300000in}{7.700000in}}%
\pgfusepath{clip}%
\pgfsetrectcap%
\pgfsetroundjoin%
\pgfsetlinewidth{1.505625pt}%
\definecolor{currentstroke}{rgb}{1.000000,0.705882,0.509804}%
\pgfsetstrokecolor{currentstroke}%
\pgfsetstrokeopacity{0.800000}%
\pgfsetdash{}{0pt}%
\pgfpathmoveto{\pgfqpoint{4.849105in}{2.709988in}}%
\pgfpathlineto{\pgfqpoint{3.502867in}{4.072531in}}%
\pgfusepath{stroke}%
\end{pgfscope}%
\begin{pgfscope}%
\pgfpathrectangle{\pgfqpoint{0.481978in}{0.331635in}}{\pgfqpoint{9.300000in}{7.700000in}}%
\pgfusepath{clip}%
\pgfsetrectcap%
\pgfsetroundjoin%
\pgfsetlinewidth{1.505625pt}%
\definecolor{currentstroke}{rgb}{1.000000,0.705882,0.509804}%
\pgfsetstrokecolor{currentstroke}%
\pgfsetstrokeopacity{0.800000}%
\pgfsetdash{}{0pt}%
\pgfpathmoveto{\pgfqpoint{3.186199in}{3.178453in}}%
\pgfpathlineto{\pgfqpoint{3.502867in}{4.072531in}}%
\pgfusepath{stroke}%
\end{pgfscope}%
\begin{pgfscope}%
\pgfpathrectangle{\pgfqpoint{0.481978in}{0.331635in}}{\pgfqpoint{9.300000in}{7.700000in}}%
\pgfusepath{clip}%
\pgfsetrectcap%
\pgfsetroundjoin%
\pgfsetlinewidth{1.505625pt}%
\definecolor{currentstroke}{rgb}{1.000000,0.705882,0.509804}%
\pgfsetstrokecolor{currentstroke}%
\pgfsetstrokeopacity{0.800000}%
\pgfsetdash{}{0pt}%
\pgfpathmoveto{\pgfqpoint{2.733970in}{2.914683in}}%
\pgfpathlineto{\pgfqpoint{3.502867in}{4.072531in}}%
\pgfusepath{stroke}%
\end{pgfscope}%
\begin{pgfscope}%
\pgfpathrectangle{\pgfqpoint{0.481978in}{0.331635in}}{\pgfqpoint{9.300000in}{7.700000in}}%
\pgfusepath{clip}%
\pgfsetrectcap%
\pgfsetroundjoin%
\pgfsetlinewidth{1.505625pt}%
\definecolor{currentstroke}{rgb}{1.000000,0.705882,0.509804}%
\pgfsetstrokecolor{currentstroke}%
\pgfsetstrokeopacity{0.800000}%
\pgfsetdash{}{0pt}%
\pgfpathmoveto{\pgfqpoint{4.054746in}{2.700364in}}%
\pgfpathlineto{\pgfqpoint{3.502867in}{4.072531in}}%
\pgfusepath{stroke}%
\end{pgfscope}%
\begin{pgfscope}%
\pgfpathrectangle{\pgfqpoint{0.481978in}{0.331635in}}{\pgfqpoint{9.300000in}{7.700000in}}%
\pgfusepath{clip}%
\pgfsetrectcap%
\pgfsetroundjoin%
\pgfsetlinewidth{1.505625pt}%
\definecolor{currentstroke}{rgb}{1.000000,0.705882,0.509804}%
\pgfsetstrokecolor{currentstroke}%
\pgfsetstrokeopacity{0.800000}%
\pgfsetdash{}{0pt}%
\pgfpathmoveto{\pgfqpoint{3.726283in}{6.128973in}}%
\pgfpathlineto{\pgfqpoint{3.502867in}{4.072531in}}%
\pgfusepath{stroke}%
\end{pgfscope}%
\begin{pgfscope}%
\pgfpathrectangle{\pgfqpoint{0.481978in}{0.331635in}}{\pgfqpoint{9.300000in}{7.700000in}}%
\pgfusepath{clip}%
\pgfsetrectcap%
\pgfsetroundjoin%
\pgfsetlinewidth{1.505625pt}%
\definecolor{currentstroke}{rgb}{1.000000,0.705882,0.509804}%
\pgfsetstrokecolor{currentstroke}%
\pgfsetstrokeopacity{0.800000}%
\pgfsetdash{}{0pt}%
\pgfpathmoveto{\pgfqpoint{3.284360in}{6.107757in}}%
\pgfpathlineto{\pgfqpoint{3.502867in}{4.072531in}}%
\pgfusepath{stroke}%
\end{pgfscope}%
\begin{pgfscope}%
\pgfpathrectangle{\pgfqpoint{0.481978in}{0.331635in}}{\pgfqpoint{9.300000in}{7.700000in}}%
\pgfusepath{clip}%
\pgfsetrectcap%
\pgfsetroundjoin%
\pgfsetlinewidth{1.505625pt}%
\definecolor{currentstroke}{rgb}{1.000000,0.705882,0.509804}%
\pgfsetstrokecolor{currentstroke}%
\pgfsetstrokeopacity{0.800000}%
\pgfsetdash{}{0pt}%
\pgfpathmoveto{\pgfqpoint{4.144311in}{2.775514in}}%
\pgfpathlineto{\pgfqpoint{3.502867in}{4.072531in}}%
\pgfusepath{stroke}%
\end{pgfscope}%
\begin{pgfscope}%
\pgfpathrectangle{\pgfqpoint{0.481978in}{0.331635in}}{\pgfqpoint{9.300000in}{7.700000in}}%
\pgfusepath{clip}%
\pgfsetrectcap%
\pgfsetroundjoin%
\pgfsetlinewidth{1.505625pt}%
\definecolor{currentstroke}{rgb}{1.000000,0.705882,0.509804}%
\pgfsetstrokecolor{currentstroke}%
\pgfsetstrokeopacity{0.800000}%
\pgfsetdash{}{0pt}%
\pgfpathmoveto{\pgfqpoint{2.660738in}{2.616861in}}%
\pgfpathlineto{\pgfqpoint{3.502867in}{4.072531in}}%
\pgfusepath{stroke}%
\end{pgfscope}%
\begin{pgfscope}%
\pgfpathrectangle{\pgfqpoint{0.481978in}{0.331635in}}{\pgfqpoint{9.300000in}{7.700000in}}%
\pgfusepath{clip}%
\pgfsetrectcap%
\pgfsetroundjoin%
\pgfsetlinewidth{1.505625pt}%
\definecolor{currentstroke}{rgb}{1.000000,0.705882,0.509804}%
\pgfsetstrokecolor{currentstroke}%
\pgfsetstrokeopacity{0.800000}%
\pgfsetdash{}{0pt}%
\pgfpathmoveto{\pgfqpoint{3.685446in}{2.395770in}}%
\pgfpathlineto{\pgfqpoint{3.502867in}{4.072531in}}%
\pgfusepath{stroke}%
\end{pgfscope}%
\begin{pgfscope}%
\pgfpathrectangle{\pgfqpoint{0.481978in}{0.331635in}}{\pgfqpoint{9.300000in}{7.700000in}}%
\pgfusepath{clip}%
\pgfsetrectcap%
\pgfsetroundjoin%
\pgfsetlinewidth{1.505625pt}%
\definecolor{currentstroke}{rgb}{1.000000,0.705882,0.509804}%
\pgfsetstrokecolor{currentstroke}%
\pgfsetstrokeopacity{0.800000}%
\pgfsetdash{}{0pt}%
\pgfpathmoveto{\pgfqpoint{3.030052in}{5.761355in}}%
\pgfpathlineto{\pgfqpoint{3.502867in}{4.072531in}}%
\pgfusepath{stroke}%
\end{pgfscope}%
\begin{pgfscope}%
\pgfpathrectangle{\pgfqpoint{0.481978in}{0.331635in}}{\pgfqpoint{9.300000in}{7.700000in}}%
\pgfusepath{clip}%
\pgfsetrectcap%
\pgfsetroundjoin%
\pgfsetlinewidth{1.505625pt}%
\definecolor{currentstroke}{rgb}{1.000000,0.705882,0.509804}%
\pgfsetstrokecolor{currentstroke}%
\pgfsetstrokeopacity{0.800000}%
\pgfsetdash{}{0pt}%
\pgfpathmoveto{\pgfqpoint{1.530784in}{2.534423in}}%
\pgfpathlineto{\pgfqpoint{3.502867in}{4.072531in}}%
\pgfusepath{stroke}%
\end{pgfscope}%
\begin{pgfscope}%
\pgfpathrectangle{\pgfqpoint{0.481978in}{0.331635in}}{\pgfqpoint{9.300000in}{7.700000in}}%
\pgfusepath{clip}%
\pgfsetrectcap%
\pgfsetroundjoin%
\pgfsetlinewidth{1.505625pt}%
\definecolor{currentstroke}{rgb}{1.000000,0.705882,0.509804}%
\pgfsetstrokecolor{currentstroke}%
\pgfsetstrokeopacity{0.800000}%
\pgfsetdash{}{0pt}%
\pgfpathmoveto{\pgfqpoint{1.565141in}{5.100776in}}%
\pgfpathlineto{\pgfqpoint{3.502867in}{4.072531in}}%
\pgfusepath{stroke}%
\end{pgfscope}%
\begin{pgfscope}%
\pgfpathrectangle{\pgfqpoint{0.481978in}{0.331635in}}{\pgfqpoint{9.300000in}{7.700000in}}%
\pgfusepath{clip}%
\pgfsetrectcap%
\pgfsetroundjoin%
\pgfsetlinewidth{1.505625pt}%
\definecolor{currentstroke}{rgb}{1.000000,0.705882,0.509804}%
\pgfsetstrokecolor{currentstroke}%
\pgfsetstrokeopacity{0.800000}%
\pgfsetdash{}{0pt}%
\pgfpathmoveto{\pgfqpoint{4.138549in}{4.945210in}}%
\pgfpathlineto{\pgfqpoint{3.502867in}{4.072531in}}%
\pgfusepath{stroke}%
\end{pgfscope}%
\begin{pgfscope}%
\pgfpathrectangle{\pgfqpoint{0.481978in}{0.331635in}}{\pgfqpoint{9.300000in}{7.700000in}}%
\pgfusepath{clip}%
\pgfsetrectcap%
\pgfsetroundjoin%
\pgfsetlinewidth{1.505625pt}%
\definecolor{currentstroke}{rgb}{1.000000,0.705882,0.509804}%
\pgfsetstrokecolor{currentstroke}%
\pgfsetstrokeopacity{0.800000}%
\pgfsetdash{}{0pt}%
\pgfpathmoveto{\pgfqpoint{4.409605in}{4.766470in}}%
\pgfpathlineto{\pgfqpoint{3.502867in}{4.072531in}}%
\pgfusepath{stroke}%
\end{pgfscope}%
\begin{pgfscope}%
\pgfpathrectangle{\pgfqpoint{0.481978in}{0.331635in}}{\pgfqpoint{9.300000in}{7.700000in}}%
\pgfusepath{clip}%
\pgfsetrectcap%
\pgfsetroundjoin%
\pgfsetlinewidth{1.505625pt}%
\definecolor{currentstroke}{rgb}{1.000000,0.705882,0.509804}%
\pgfsetstrokecolor{currentstroke}%
\pgfsetstrokeopacity{0.800000}%
\pgfsetdash{}{0pt}%
\pgfpathmoveto{\pgfqpoint{4.134948in}{3.165826in}}%
\pgfpathlineto{\pgfqpoint{3.502867in}{4.072531in}}%
\pgfusepath{stroke}%
\end{pgfscope}%
\begin{pgfscope}%
\pgfpathrectangle{\pgfqpoint{0.481978in}{0.331635in}}{\pgfqpoint{9.300000in}{7.700000in}}%
\pgfusepath{clip}%
\pgfsetrectcap%
\pgfsetroundjoin%
\pgfsetlinewidth{1.505625pt}%
\definecolor{currentstroke}{rgb}{1.000000,0.705882,0.509804}%
\pgfsetstrokecolor{currentstroke}%
\pgfsetstrokeopacity{0.800000}%
\pgfsetdash{}{0pt}%
\pgfpathmoveto{\pgfqpoint{2.632233in}{4.374697in}}%
\pgfpathlineto{\pgfqpoint{3.502867in}{4.072531in}}%
\pgfusepath{stroke}%
\end{pgfscope}%
\begin{pgfscope}%
\pgfpathrectangle{\pgfqpoint{0.481978in}{0.331635in}}{\pgfqpoint{9.300000in}{7.700000in}}%
\pgfusepath{clip}%
\pgfsetrectcap%
\pgfsetroundjoin%
\pgfsetlinewidth{1.505625pt}%
\definecolor{currentstroke}{rgb}{1.000000,0.705882,0.509804}%
\pgfsetstrokecolor{currentstroke}%
\pgfsetstrokeopacity{0.800000}%
\pgfsetdash{}{0pt}%
\pgfpathmoveto{\pgfqpoint{3.332817in}{4.302567in}}%
\pgfpathlineto{\pgfqpoint{3.502867in}{4.072531in}}%
\pgfusepath{stroke}%
\end{pgfscope}%
\begin{pgfscope}%
\pgfpathrectangle{\pgfqpoint{0.481978in}{0.331635in}}{\pgfqpoint{9.300000in}{7.700000in}}%
\pgfusepath{clip}%
\pgfsetrectcap%
\pgfsetroundjoin%
\pgfsetlinewidth{1.505625pt}%
\definecolor{currentstroke}{rgb}{1.000000,0.705882,0.509804}%
\pgfsetstrokecolor{currentstroke}%
\pgfsetstrokeopacity{0.800000}%
\pgfsetdash{}{0pt}%
\pgfpathmoveto{\pgfqpoint{3.045518in}{4.284260in}}%
\pgfpathlineto{\pgfqpoint{3.502867in}{4.072531in}}%
\pgfusepath{stroke}%
\end{pgfscope}%
\begin{pgfscope}%
\pgfpathrectangle{\pgfqpoint{0.481978in}{0.331635in}}{\pgfqpoint{9.300000in}{7.700000in}}%
\pgfusepath{clip}%
\pgfsetrectcap%
\pgfsetroundjoin%
\pgfsetlinewidth{1.505625pt}%
\definecolor{currentstroke}{rgb}{1.000000,0.705882,0.509804}%
\pgfsetstrokecolor{currentstroke}%
\pgfsetstrokeopacity{0.800000}%
\pgfsetdash{}{0pt}%
\pgfpathmoveto{\pgfqpoint{2.948753in}{3.450151in}}%
\pgfpathlineto{\pgfqpoint{3.502867in}{4.072531in}}%
\pgfusepath{stroke}%
\end{pgfscope}%
\begin{pgfscope}%
\pgfpathrectangle{\pgfqpoint{0.481978in}{0.331635in}}{\pgfqpoint{9.300000in}{7.700000in}}%
\pgfusepath{clip}%
\pgfsetrectcap%
\pgfsetroundjoin%
\pgfsetlinewidth{1.505625pt}%
\definecolor{currentstroke}{rgb}{1.000000,0.705882,0.509804}%
\pgfsetstrokecolor{currentstroke}%
\pgfsetstrokeopacity{0.800000}%
\pgfsetdash{}{0pt}%
\pgfpathmoveto{\pgfqpoint{3.979638in}{3.616382in}}%
\pgfpathlineto{\pgfqpoint{3.502867in}{4.072531in}}%
\pgfusepath{stroke}%
\end{pgfscope}%
\begin{pgfscope}%
\pgfpathrectangle{\pgfqpoint{0.481978in}{0.331635in}}{\pgfqpoint{9.300000in}{7.700000in}}%
\pgfusepath{clip}%
\pgfsetrectcap%
\pgfsetroundjoin%
\pgfsetlinewidth{1.505625pt}%
\definecolor{currentstroke}{rgb}{1.000000,0.705882,0.509804}%
\pgfsetstrokecolor{currentstroke}%
\pgfsetstrokeopacity{0.800000}%
\pgfsetdash{}{0pt}%
\pgfpathmoveto{\pgfqpoint{4.333852in}{2.451747in}}%
\pgfpathlineto{\pgfqpoint{3.502867in}{4.072531in}}%
\pgfusepath{stroke}%
\end{pgfscope}%
\begin{pgfscope}%
\pgfpathrectangle{\pgfqpoint{0.481978in}{0.331635in}}{\pgfqpoint{9.300000in}{7.700000in}}%
\pgfusepath{clip}%
\pgfsetrectcap%
\pgfsetroundjoin%
\pgfsetlinewidth{1.505625pt}%
\definecolor{currentstroke}{rgb}{1.000000,0.705882,0.509804}%
\pgfsetstrokecolor{currentstroke}%
\pgfsetstrokeopacity{0.800000}%
\pgfsetdash{}{0pt}%
\pgfpathmoveto{\pgfqpoint{1.939275in}{4.224539in}}%
\pgfpathlineto{\pgfqpoint{3.502867in}{4.072531in}}%
\pgfusepath{stroke}%
\end{pgfscope}%
\begin{pgfscope}%
\pgfpathrectangle{\pgfqpoint{0.481978in}{0.331635in}}{\pgfqpoint{9.300000in}{7.700000in}}%
\pgfusepath{clip}%
\pgfsetrectcap%
\pgfsetroundjoin%
\pgfsetlinewidth{1.505625pt}%
\definecolor{currentstroke}{rgb}{1.000000,0.705882,0.509804}%
\pgfsetstrokecolor{currentstroke}%
\pgfsetstrokeopacity{0.800000}%
\pgfsetdash{}{0pt}%
\pgfpathmoveto{\pgfqpoint{4.479280in}{2.911311in}}%
\pgfpathlineto{\pgfqpoint{3.502867in}{4.072531in}}%
\pgfusepath{stroke}%
\end{pgfscope}%
\begin{pgfscope}%
\pgfpathrectangle{\pgfqpoint{0.481978in}{0.331635in}}{\pgfqpoint{9.300000in}{7.700000in}}%
\pgfusepath{clip}%
\pgfsetrectcap%
\pgfsetroundjoin%
\pgfsetlinewidth{1.505625pt}%
\definecolor{currentstroke}{rgb}{1.000000,0.705882,0.509804}%
\pgfsetstrokecolor{currentstroke}%
\pgfsetstrokeopacity{0.800000}%
\pgfsetdash{}{0pt}%
\pgfpathmoveto{\pgfqpoint{3.013348in}{2.644342in}}%
\pgfpathlineto{\pgfqpoint{3.502867in}{4.072531in}}%
\pgfusepath{stroke}%
\end{pgfscope}%
\begin{pgfscope}%
\pgfpathrectangle{\pgfqpoint{0.481978in}{0.331635in}}{\pgfqpoint{9.300000in}{7.700000in}}%
\pgfusepath{clip}%
\pgfsetrectcap%
\pgfsetroundjoin%
\pgfsetlinewidth{1.505625pt}%
\definecolor{currentstroke}{rgb}{1.000000,0.705882,0.509804}%
\pgfsetstrokecolor{currentstroke}%
\pgfsetstrokeopacity{0.800000}%
\pgfsetdash{}{0pt}%
\pgfpathmoveto{\pgfqpoint{4.165514in}{6.528278in}}%
\pgfpathlineto{\pgfqpoint{3.502867in}{4.072531in}}%
\pgfusepath{stroke}%
\end{pgfscope}%
\begin{pgfscope}%
\pgfpathrectangle{\pgfqpoint{0.481978in}{0.331635in}}{\pgfqpoint{9.300000in}{7.700000in}}%
\pgfusepath{clip}%
\pgfsetrectcap%
\pgfsetroundjoin%
\pgfsetlinewidth{1.505625pt}%
\definecolor{currentstroke}{rgb}{1.000000,0.705882,0.509804}%
\pgfsetstrokecolor{currentstroke}%
\pgfsetstrokeopacity{0.800000}%
\pgfsetdash{}{0pt}%
\pgfpathmoveto{\pgfqpoint{2.157737in}{4.152260in}}%
\pgfpathlineto{\pgfqpoint{3.502867in}{4.072531in}}%
\pgfusepath{stroke}%
\end{pgfscope}%
\begin{pgfscope}%
\pgfpathrectangle{\pgfqpoint{0.481978in}{0.331635in}}{\pgfqpoint{9.300000in}{7.700000in}}%
\pgfusepath{clip}%
\pgfsetrectcap%
\pgfsetroundjoin%
\pgfsetlinewidth{1.505625pt}%
\definecolor{currentstroke}{rgb}{1.000000,0.705882,0.509804}%
\pgfsetstrokecolor{currentstroke}%
\pgfsetstrokeopacity{0.800000}%
\pgfsetdash{}{0pt}%
\pgfpathmoveto{\pgfqpoint{2.216691in}{5.408235in}}%
\pgfpathlineto{\pgfqpoint{3.502867in}{4.072531in}}%
\pgfusepath{stroke}%
\end{pgfscope}%
\begin{pgfscope}%
\pgfpathrectangle{\pgfqpoint{0.481978in}{0.331635in}}{\pgfqpoint{9.300000in}{7.700000in}}%
\pgfusepath{clip}%
\pgfsetrectcap%
\pgfsetroundjoin%
\pgfsetlinewidth{1.505625pt}%
\definecolor{currentstroke}{rgb}{1.000000,0.705882,0.509804}%
\pgfsetstrokecolor{currentstroke}%
\pgfsetstrokeopacity{0.800000}%
\pgfsetdash{}{0pt}%
\pgfpathmoveto{\pgfqpoint{4.655686in}{2.246792in}}%
\pgfpathlineto{\pgfqpoint{3.502867in}{4.072531in}}%
\pgfusepath{stroke}%
\end{pgfscope}%
\begin{pgfscope}%
\pgfpathrectangle{\pgfqpoint{0.481978in}{0.331635in}}{\pgfqpoint{9.300000in}{7.700000in}}%
\pgfusepath{clip}%
\pgfsetrectcap%
\pgfsetroundjoin%
\pgfsetlinewidth{1.505625pt}%
\definecolor{currentstroke}{rgb}{1.000000,0.705882,0.509804}%
\pgfsetstrokecolor{currentstroke}%
\pgfsetstrokeopacity{0.800000}%
\pgfsetdash{}{0pt}%
\pgfpathmoveto{\pgfqpoint{3.175062in}{3.348731in}}%
\pgfpathlineto{\pgfqpoint{3.502867in}{4.072531in}}%
\pgfusepath{stroke}%
\end{pgfscope}%
\begin{pgfscope}%
\pgfpathrectangle{\pgfqpoint{0.481978in}{0.331635in}}{\pgfqpoint{9.300000in}{7.700000in}}%
\pgfusepath{clip}%
\pgfsetrectcap%
\pgfsetroundjoin%
\pgfsetlinewidth{1.505625pt}%
\definecolor{currentstroke}{rgb}{1.000000,0.705882,0.509804}%
\pgfsetstrokecolor{currentstroke}%
\pgfsetstrokeopacity{0.800000}%
\pgfsetdash{}{0pt}%
\pgfpathmoveto{\pgfqpoint{2.367140in}{4.716464in}}%
\pgfpathlineto{\pgfqpoint{3.502867in}{4.072531in}}%
\pgfusepath{stroke}%
\end{pgfscope}%
\begin{pgfscope}%
\pgfpathrectangle{\pgfqpoint{0.481978in}{0.331635in}}{\pgfqpoint{9.300000in}{7.700000in}}%
\pgfusepath{clip}%
\pgfsetrectcap%
\pgfsetroundjoin%
\pgfsetlinewidth{1.505625pt}%
\definecolor{currentstroke}{rgb}{1.000000,0.705882,0.509804}%
\pgfsetstrokecolor{currentstroke}%
\pgfsetstrokeopacity{0.800000}%
\pgfsetdash{}{0pt}%
\pgfpathmoveto{\pgfqpoint{4.838472in}{3.036006in}}%
\pgfpathlineto{\pgfqpoint{3.502867in}{4.072531in}}%
\pgfusepath{stroke}%
\end{pgfscope}%
\begin{pgfscope}%
\pgfpathrectangle{\pgfqpoint{0.481978in}{0.331635in}}{\pgfqpoint{9.300000in}{7.700000in}}%
\pgfusepath{clip}%
\pgfsetrectcap%
\pgfsetroundjoin%
\pgfsetlinewidth{1.505625pt}%
\definecolor{currentstroke}{rgb}{1.000000,0.705882,0.509804}%
\pgfsetstrokecolor{currentstroke}%
\pgfsetstrokeopacity{0.800000}%
\pgfsetdash{}{0pt}%
\pgfpathmoveto{\pgfqpoint{4.322216in}{5.047571in}}%
\pgfpathlineto{\pgfqpoint{3.502867in}{4.072531in}}%
\pgfusepath{stroke}%
\end{pgfscope}%
\begin{pgfscope}%
\pgfpathrectangle{\pgfqpoint{0.481978in}{0.331635in}}{\pgfqpoint{9.300000in}{7.700000in}}%
\pgfusepath{clip}%
\pgfsetrectcap%
\pgfsetroundjoin%
\pgfsetlinewidth{1.505625pt}%
\definecolor{currentstroke}{rgb}{1.000000,0.705882,0.509804}%
\pgfsetstrokecolor{currentstroke}%
\pgfsetstrokeopacity{0.800000}%
\pgfsetdash{}{0pt}%
\pgfpathmoveto{\pgfqpoint{3.615208in}{3.463945in}}%
\pgfpathlineto{\pgfqpoint{3.502867in}{4.072531in}}%
\pgfusepath{stroke}%
\end{pgfscope}%
\begin{pgfscope}%
\pgfpathrectangle{\pgfqpoint{0.481978in}{0.331635in}}{\pgfqpoint{9.300000in}{7.700000in}}%
\pgfusepath{clip}%
\pgfsetrectcap%
\pgfsetroundjoin%
\pgfsetlinewidth{1.505625pt}%
\definecolor{currentstroke}{rgb}{1.000000,0.705882,0.509804}%
\pgfsetstrokecolor{currentstroke}%
\pgfsetstrokeopacity{0.800000}%
\pgfsetdash{}{0pt}%
\pgfpathmoveto{\pgfqpoint{3.424608in}{6.947120in}}%
\pgfpathlineto{\pgfqpoint{3.502867in}{4.072531in}}%
\pgfusepath{stroke}%
\end{pgfscope}%
\begin{pgfscope}%
\pgfpathrectangle{\pgfqpoint{0.481978in}{0.331635in}}{\pgfqpoint{9.300000in}{7.700000in}}%
\pgfusepath{clip}%
\pgfsetrectcap%
\pgfsetroundjoin%
\pgfsetlinewidth{1.505625pt}%
\definecolor{currentstroke}{rgb}{1.000000,0.705882,0.509804}%
\pgfsetstrokecolor{currentstroke}%
\pgfsetstrokeopacity{0.800000}%
\pgfsetdash{}{0pt}%
\pgfpathmoveto{\pgfqpoint{5.585402in}{5.240124in}}%
\pgfpathlineto{\pgfqpoint{3.502867in}{4.072531in}}%
\pgfusepath{stroke}%
\end{pgfscope}%
\begin{pgfscope}%
\pgfpathrectangle{\pgfqpoint{0.481978in}{0.331635in}}{\pgfqpoint{9.300000in}{7.700000in}}%
\pgfusepath{clip}%
\pgfsetrectcap%
\pgfsetroundjoin%
\pgfsetlinewidth{1.505625pt}%
\definecolor{currentstroke}{rgb}{1.000000,0.705882,0.509804}%
\pgfsetstrokecolor{currentstroke}%
\pgfsetstrokeopacity{0.800000}%
\pgfsetdash{}{0pt}%
\pgfpathmoveto{\pgfqpoint{3.153342in}{5.109707in}}%
\pgfpathlineto{\pgfqpoint{3.502867in}{4.072531in}}%
\pgfusepath{stroke}%
\end{pgfscope}%
\begin{pgfscope}%
\pgfpathrectangle{\pgfqpoint{0.481978in}{0.331635in}}{\pgfqpoint{9.300000in}{7.700000in}}%
\pgfusepath{clip}%
\pgfsetrectcap%
\pgfsetroundjoin%
\pgfsetlinewidth{1.505625pt}%
\definecolor{currentstroke}{rgb}{1.000000,0.705882,0.509804}%
\pgfsetstrokecolor{currentstroke}%
\pgfsetstrokeopacity{0.800000}%
\pgfsetdash{}{0pt}%
\pgfpathmoveto{\pgfqpoint{3.297548in}{5.482494in}}%
\pgfpathlineto{\pgfqpoint{3.502867in}{4.072531in}}%
\pgfusepath{stroke}%
\end{pgfscope}%
\begin{pgfscope}%
\pgfpathrectangle{\pgfqpoint{0.481978in}{0.331635in}}{\pgfqpoint{9.300000in}{7.700000in}}%
\pgfusepath{clip}%
\pgfsetrectcap%
\pgfsetroundjoin%
\pgfsetlinewidth{1.505625pt}%
\definecolor{currentstroke}{rgb}{1.000000,0.705882,0.509804}%
\pgfsetstrokecolor{currentstroke}%
\pgfsetstrokeopacity{0.800000}%
\pgfsetdash{}{0pt}%
\pgfpathmoveto{\pgfqpoint{4.669793in}{2.259600in}}%
\pgfpathlineto{\pgfqpoint{3.502867in}{4.072531in}}%
\pgfusepath{stroke}%
\end{pgfscope}%
\begin{pgfscope}%
\pgfpathrectangle{\pgfqpoint{0.481978in}{0.331635in}}{\pgfqpoint{9.300000in}{7.700000in}}%
\pgfusepath{clip}%
\pgfsetrectcap%
\pgfsetroundjoin%
\pgfsetlinewidth{1.505625pt}%
\definecolor{currentstroke}{rgb}{1.000000,0.705882,0.509804}%
\pgfsetstrokecolor{currentstroke}%
\pgfsetstrokeopacity{0.800000}%
\pgfsetdash{}{0pt}%
\pgfpathmoveto{\pgfqpoint{1.546741in}{5.145333in}}%
\pgfpathlineto{\pgfqpoint{3.502867in}{4.072531in}}%
\pgfusepath{stroke}%
\end{pgfscope}%
\begin{pgfscope}%
\pgfpathrectangle{\pgfqpoint{0.481978in}{0.331635in}}{\pgfqpoint{9.300000in}{7.700000in}}%
\pgfusepath{clip}%
\pgfsetrectcap%
\pgfsetroundjoin%
\pgfsetlinewidth{1.505625pt}%
\definecolor{currentstroke}{rgb}{1.000000,0.705882,0.509804}%
\pgfsetstrokecolor{currentstroke}%
\pgfsetstrokeopacity{0.800000}%
\pgfsetdash{}{0pt}%
\pgfpathmoveto{\pgfqpoint{4.426621in}{5.394791in}}%
\pgfpathlineto{\pgfqpoint{3.502867in}{4.072531in}}%
\pgfusepath{stroke}%
\end{pgfscope}%
\begin{pgfscope}%
\pgfpathrectangle{\pgfqpoint{0.481978in}{0.331635in}}{\pgfqpoint{9.300000in}{7.700000in}}%
\pgfusepath{clip}%
\pgfsetrectcap%
\pgfsetroundjoin%
\pgfsetlinewidth{1.505625pt}%
\definecolor{currentstroke}{rgb}{1.000000,0.705882,0.509804}%
\pgfsetstrokecolor{currentstroke}%
\pgfsetstrokeopacity{0.800000}%
\pgfsetdash{}{0pt}%
\pgfpathmoveto{\pgfqpoint{2.946822in}{5.353658in}}%
\pgfpathlineto{\pgfqpoint{3.502867in}{4.072531in}}%
\pgfusepath{stroke}%
\end{pgfscope}%
\begin{pgfscope}%
\pgfpathrectangle{\pgfqpoint{0.481978in}{0.331635in}}{\pgfqpoint{9.300000in}{7.700000in}}%
\pgfusepath{clip}%
\pgfsetrectcap%
\pgfsetroundjoin%
\pgfsetlinewidth{1.505625pt}%
\definecolor{currentstroke}{rgb}{1.000000,0.705882,0.509804}%
\pgfsetstrokecolor{currentstroke}%
\pgfsetstrokeopacity{0.800000}%
\pgfsetdash{}{0pt}%
\pgfpathmoveto{\pgfqpoint{9.359251in}{1.365147in}}%
\pgfpathlineto{\pgfqpoint{3.502867in}{4.072531in}}%
\pgfusepath{stroke}%
\end{pgfscope}%
\begin{pgfscope}%
\pgfpathrectangle{\pgfqpoint{0.481978in}{0.331635in}}{\pgfqpoint{9.300000in}{7.700000in}}%
\pgfusepath{clip}%
\pgfsetrectcap%
\pgfsetroundjoin%
\pgfsetlinewidth{1.505625pt}%
\definecolor{currentstroke}{rgb}{1.000000,0.705882,0.509804}%
\pgfsetstrokecolor{currentstroke}%
\pgfsetstrokeopacity{0.800000}%
\pgfsetdash{}{0pt}%
\pgfpathmoveto{\pgfqpoint{3.311383in}{5.697453in}}%
\pgfpathlineto{\pgfqpoint{3.502867in}{4.072531in}}%
\pgfusepath{stroke}%
\end{pgfscope}%
\begin{pgfscope}%
\pgfpathrectangle{\pgfqpoint{0.481978in}{0.331635in}}{\pgfqpoint{9.300000in}{7.700000in}}%
\pgfusepath{clip}%
\pgfsetrectcap%
\pgfsetroundjoin%
\pgfsetlinewidth{1.505625pt}%
\definecolor{currentstroke}{rgb}{1.000000,0.705882,0.509804}%
\pgfsetstrokecolor{currentstroke}%
\pgfsetstrokeopacity{0.800000}%
\pgfsetdash{}{0pt}%
\pgfpathmoveto{\pgfqpoint{3.768020in}{3.858089in}}%
\pgfpathlineto{\pgfqpoint{3.502867in}{4.072531in}}%
\pgfusepath{stroke}%
\end{pgfscope}%
\begin{pgfscope}%
\pgfpathrectangle{\pgfqpoint{0.481978in}{0.331635in}}{\pgfqpoint{9.300000in}{7.700000in}}%
\pgfusepath{clip}%
\pgfsetrectcap%
\pgfsetroundjoin%
\pgfsetlinewidth{1.505625pt}%
\definecolor{currentstroke}{rgb}{1.000000,0.705882,0.509804}%
\pgfsetstrokecolor{currentstroke}%
\pgfsetstrokeopacity{0.800000}%
\pgfsetdash{}{0pt}%
\pgfpathmoveto{\pgfqpoint{3.827532in}{5.121296in}}%
\pgfpathlineto{\pgfqpoint{3.502867in}{4.072531in}}%
\pgfusepath{stroke}%
\end{pgfscope}%
\begin{pgfscope}%
\pgfpathrectangle{\pgfqpoint{0.481978in}{0.331635in}}{\pgfqpoint{9.300000in}{7.700000in}}%
\pgfusepath{clip}%
\pgfsetrectcap%
\pgfsetroundjoin%
\pgfsetlinewidth{1.505625pt}%
\definecolor{currentstroke}{rgb}{1.000000,0.705882,0.509804}%
\pgfsetstrokecolor{currentstroke}%
\pgfsetstrokeopacity{0.800000}%
\pgfsetdash{}{0pt}%
\pgfpathmoveto{\pgfqpoint{1.059884in}{2.192019in}}%
\pgfpathlineto{\pgfqpoint{3.502867in}{4.072531in}}%
\pgfusepath{stroke}%
\end{pgfscope}%
\begin{pgfscope}%
\pgfpathrectangle{\pgfqpoint{0.481978in}{0.331635in}}{\pgfqpoint{9.300000in}{7.700000in}}%
\pgfusepath{clip}%
\pgfsetrectcap%
\pgfsetroundjoin%
\pgfsetlinewidth{1.505625pt}%
\definecolor{currentstroke}{rgb}{1.000000,0.705882,0.509804}%
\pgfsetstrokecolor{currentstroke}%
\pgfsetstrokeopacity{0.800000}%
\pgfsetdash{}{0pt}%
\pgfpathmoveto{\pgfqpoint{4.209058in}{3.248890in}}%
\pgfpathlineto{\pgfqpoint{3.502867in}{4.072531in}}%
\pgfusepath{stroke}%
\end{pgfscope}%
\begin{pgfscope}%
\pgfpathrectangle{\pgfqpoint{0.481978in}{0.331635in}}{\pgfqpoint{9.300000in}{7.700000in}}%
\pgfusepath{clip}%
\pgfsetrectcap%
\pgfsetroundjoin%
\pgfsetlinewidth{1.505625pt}%
\definecolor{currentstroke}{rgb}{1.000000,0.705882,0.509804}%
\pgfsetstrokecolor{currentstroke}%
\pgfsetstrokeopacity{0.800000}%
\pgfsetdash{}{0pt}%
\pgfpathmoveto{\pgfqpoint{3.072069in}{2.785754in}}%
\pgfpathlineto{\pgfqpoint{3.502867in}{4.072531in}}%
\pgfusepath{stroke}%
\end{pgfscope}%
\begin{pgfscope}%
\pgfpathrectangle{\pgfqpoint{0.481978in}{0.331635in}}{\pgfqpoint{9.300000in}{7.700000in}}%
\pgfusepath{clip}%
\pgfsetrectcap%
\pgfsetroundjoin%
\pgfsetlinewidth{1.505625pt}%
\definecolor{currentstroke}{rgb}{1.000000,0.705882,0.509804}%
\pgfsetstrokecolor{currentstroke}%
\pgfsetstrokeopacity{0.800000}%
\pgfsetdash{}{0pt}%
\pgfpathmoveto{\pgfqpoint{4.167601in}{3.812695in}}%
\pgfpathlineto{\pgfqpoint{3.502867in}{4.072531in}}%
\pgfusepath{stroke}%
\end{pgfscope}%
\begin{pgfscope}%
\pgfpathrectangle{\pgfqpoint{0.481978in}{0.331635in}}{\pgfqpoint{9.300000in}{7.700000in}}%
\pgfusepath{clip}%
\pgfsetrectcap%
\pgfsetroundjoin%
\pgfsetlinewidth{1.505625pt}%
\definecolor{currentstroke}{rgb}{1.000000,0.705882,0.509804}%
\pgfsetstrokecolor{currentstroke}%
\pgfsetstrokeopacity{0.800000}%
\pgfsetdash{}{0pt}%
\pgfpathmoveto{\pgfqpoint{4.458251in}{2.550423in}}%
\pgfpathlineto{\pgfqpoint{3.502867in}{4.072531in}}%
\pgfusepath{stroke}%
\end{pgfscope}%
\begin{pgfscope}%
\pgfpathrectangle{\pgfqpoint{0.481978in}{0.331635in}}{\pgfqpoint{9.300000in}{7.700000in}}%
\pgfusepath{clip}%
\pgfsetrectcap%
\pgfsetroundjoin%
\pgfsetlinewidth{1.505625pt}%
\definecolor{currentstroke}{rgb}{1.000000,0.705882,0.509804}%
\pgfsetstrokecolor{currentstroke}%
\pgfsetstrokeopacity{0.800000}%
\pgfsetdash{}{0pt}%
\pgfpathmoveto{\pgfqpoint{3.705775in}{4.608509in}}%
\pgfpathlineto{\pgfqpoint{3.502867in}{4.072531in}}%
\pgfusepath{stroke}%
\end{pgfscope}%
\begin{pgfscope}%
\pgfpathrectangle{\pgfqpoint{0.481978in}{0.331635in}}{\pgfqpoint{9.300000in}{7.700000in}}%
\pgfusepath{clip}%
\pgfsetrectcap%
\pgfsetroundjoin%
\pgfsetlinewidth{1.505625pt}%
\definecolor{currentstroke}{rgb}{1.000000,0.705882,0.509804}%
\pgfsetstrokecolor{currentstroke}%
\pgfsetstrokeopacity{0.800000}%
\pgfsetdash{}{0pt}%
\pgfpathmoveto{\pgfqpoint{3.840708in}{2.854435in}}%
\pgfpathlineto{\pgfqpoint{3.502867in}{4.072531in}}%
\pgfusepath{stroke}%
\end{pgfscope}%
\begin{pgfscope}%
\pgfpathrectangle{\pgfqpoint{0.481978in}{0.331635in}}{\pgfqpoint{9.300000in}{7.700000in}}%
\pgfusepath{clip}%
\pgfsetrectcap%
\pgfsetroundjoin%
\pgfsetlinewidth{1.505625pt}%
\definecolor{currentstroke}{rgb}{1.000000,0.705882,0.509804}%
\pgfsetstrokecolor{currentstroke}%
\pgfsetstrokeopacity{0.800000}%
\pgfsetdash{}{0pt}%
\pgfpathmoveto{\pgfqpoint{2.841012in}{4.168357in}}%
\pgfpathlineto{\pgfqpoint{3.502867in}{4.072531in}}%
\pgfusepath{stroke}%
\end{pgfscope}%
\begin{pgfscope}%
\pgfpathrectangle{\pgfqpoint{0.481978in}{0.331635in}}{\pgfqpoint{9.300000in}{7.700000in}}%
\pgfusepath{clip}%
\pgfsetrectcap%
\pgfsetroundjoin%
\pgfsetlinewidth{1.505625pt}%
\definecolor{currentstroke}{rgb}{1.000000,0.705882,0.509804}%
\pgfsetstrokecolor{currentstroke}%
\pgfsetstrokeopacity{0.800000}%
\pgfsetdash{}{0pt}%
\pgfpathmoveto{\pgfqpoint{2.668476in}{1.648669in}}%
\pgfpathlineto{\pgfqpoint{3.502867in}{4.072531in}}%
\pgfusepath{stroke}%
\end{pgfscope}%
\begin{pgfscope}%
\pgfpathrectangle{\pgfqpoint{0.481978in}{0.331635in}}{\pgfqpoint{9.300000in}{7.700000in}}%
\pgfusepath{clip}%
\pgfsetrectcap%
\pgfsetroundjoin%
\pgfsetlinewidth{1.505625pt}%
\definecolor{currentstroke}{rgb}{1.000000,0.705882,0.509804}%
\pgfsetstrokecolor{currentstroke}%
\pgfsetstrokeopacity{0.800000}%
\pgfsetdash{}{0pt}%
\pgfpathmoveto{\pgfqpoint{3.108258in}{5.429475in}}%
\pgfpathlineto{\pgfqpoint{3.502867in}{4.072531in}}%
\pgfusepath{stroke}%
\end{pgfscope}%
\begin{pgfscope}%
\pgfpathrectangle{\pgfqpoint{0.481978in}{0.331635in}}{\pgfqpoint{9.300000in}{7.700000in}}%
\pgfusepath{clip}%
\pgfsetrectcap%
\pgfsetroundjoin%
\pgfsetlinewidth{1.505625pt}%
\definecolor{currentstroke}{rgb}{1.000000,0.705882,0.509804}%
\pgfsetstrokecolor{currentstroke}%
\pgfsetstrokeopacity{0.800000}%
\pgfsetdash{}{0pt}%
\pgfpathmoveto{\pgfqpoint{1.529852in}{4.797125in}}%
\pgfpathlineto{\pgfqpoint{3.502867in}{4.072531in}}%
\pgfusepath{stroke}%
\end{pgfscope}%
\begin{pgfscope}%
\pgfpathrectangle{\pgfqpoint{0.481978in}{0.331635in}}{\pgfqpoint{9.300000in}{7.700000in}}%
\pgfusepath{clip}%
\pgfsetrectcap%
\pgfsetroundjoin%
\pgfsetlinewidth{1.505625pt}%
\definecolor{currentstroke}{rgb}{1.000000,0.705882,0.509804}%
\pgfsetstrokecolor{currentstroke}%
\pgfsetstrokeopacity{0.800000}%
\pgfsetdash{}{0pt}%
\pgfpathmoveto{\pgfqpoint{4.765650in}{4.953850in}}%
\pgfpathlineto{\pgfqpoint{3.502867in}{4.072531in}}%
\pgfusepath{stroke}%
\end{pgfscope}%
\begin{pgfscope}%
\pgfpathrectangle{\pgfqpoint{0.481978in}{0.331635in}}{\pgfqpoint{9.300000in}{7.700000in}}%
\pgfusepath{clip}%
\pgfsetrectcap%
\pgfsetroundjoin%
\pgfsetlinewidth{1.505625pt}%
\definecolor{currentstroke}{rgb}{1.000000,0.705882,0.509804}%
\pgfsetstrokecolor{currentstroke}%
\pgfsetstrokeopacity{0.800000}%
\pgfsetdash{}{0pt}%
\pgfpathmoveto{\pgfqpoint{1.272867in}{4.338069in}}%
\pgfpathlineto{\pgfqpoint{3.502867in}{4.072531in}}%
\pgfusepath{stroke}%
\end{pgfscope}%
\begin{pgfscope}%
\pgfpathrectangle{\pgfqpoint{0.481978in}{0.331635in}}{\pgfqpoint{9.300000in}{7.700000in}}%
\pgfusepath{clip}%
\pgfsetrectcap%
\pgfsetroundjoin%
\pgfsetlinewidth{1.505625pt}%
\definecolor{currentstroke}{rgb}{1.000000,0.705882,0.509804}%
\pgfsetstrokecolor{currentstroke}%
\pgfsetstrokeopacity{0.800000}%
\pgfsetdash{}{0pt}%
\pgfpathmoveto{\pgfqpoint{3.381281in}{4.116843in}}%
\pgfpathlineto{\pgfqpoint{3.502867in}{4.072531in}}%
\pgfusepath{stroke}%
\end{pgfscope}%
\begin{pgfscope}%
\pgfpathrectangle{\pgfqpoint{0.481978in}{0.331635in}}{\pgfqpoint{9.300000in}{7.700000in}}%
\pgfusepath{clip}%
\pgfsetrectcap%
\pgfsetroundjoin%
\pgfsetlinewidth{1.505625pt}%
\definecolor{currentstroke}{rgb}{1.000000,0.705882,0.509804}%
\pgfsetstrokecolor{currentstroke}%
\pgfsetstrokeopacity{0.800000}%
\pgfsetdash{}{0pt}%
\pgfpathmoveto{\pgfqpoint{4.049596in}{2.532499in}}%
\pgfpathlineto{\pgfqpoint{3.502867in}{4.072531in}}%
\pgfusepath{stroke}%
\end{pgfscope}%
\begin{pgfscope}%
\pgfpathrectangle{\pgfqpoint{0.481978in}{0.331635in}}{\pgfqpoint{9.300000in}{7.700000in}}%
\pgfusepath{clip}%
\pgfsetrectcap%
\pgfsetroundjoin%
\pgfsetlinewidth{1.505625pt}%
\definecolor{currentstroke}{rgb}{1.000000,0.705882,0.509804}%
\pgfsetstrokecolor{currentstroke}%
\pgfsetstrokeopacity{0.800000}%
\pgfsetdash{}{0pt}%
\pgfpathmoveto{\pgfqpoint{3.917035in}{4.822532in}}%
\pgfpathlineto{\pgfqpoint{3.502867in}{4.072531in}}%
\pgfusepath{stroke}%
\end{pgfscope}%
\begin{pgfscope}%
\pgfpathrectangle{\pgfqpoint{0.481978in}{0.331635in}}{\pgfqpoint{9.300000in}{7.700000in}}%
\pgfusepath{clip}%
\pgfsetrectcap%
\pgfsetroundjoin%
\pgfsetlinewidth{1.505625pt}%
\definecolor{currentstroke}{rgb}{1.000000,0.705882,0.509804}%
\pgfsetstrokecolor{currentstroke}%
\pgfsetstrokeopacity{0.800000}%
\pgfsetdash{}{0pt}%
\pgfpathmoveto{\pgfqpoint{2.834471in}{5.970136in}}%
\pgfpathlineto{\pgfqpoint{3.502867in}{4.072531in}}%
\pgfusepath{stroke}%
\end{pgfscope}%
\begin{pgfscope}%
\pgfpathrectangle{\pgfqpoint{0.481978in}{0.331635in}}{\pgfqpoint{9.300000in}{7.700000in}}%
\pgfusepath{clip}%
\pgfsetrectcap%
\pgfsetroundjoin%
\pgfsetlinewidth{1.505625pt}%
\definecolor{currentstroke}{rgb}{1.000000,0.705882,0.509804}%
\pgfsetstrokecolor{currentstroke}%
\pgfsetstrokeopacity{0.800000}%
\pgfsetdash{}{0pt}%
\pgfpathmoveto{\pgfqpoint{1.856829in}{4.015015in}}%
\pgfpathlineto{\pgfqpoint{3.502867in}{4.072531in}}%
\pgfusepath{stroke}%
\end{pgfscope}%
\begin{pgfscope}%
\pgfpathrectangle{\pgfqpoint{0.481978in}{0.331635in}}{\pgfqpoint{9.300000in}{7.700000in}}%
\pgfusepath{clip}%
\pgfsetrectcap%
\pgfsetroundjoin%
\pgfsetlinewidth{1.505625pt}%
\definecolor{currentstroke}{rgb}{1.000000,0.705882,0.509804}%
\pgfsetstrokecolor{currentstroke}%
\pgfsetstrokeopacity{0.800000}%
\pgfsetdash{}{0pt}%
\pgfpathmoveto{\pgfqpoint{4.069841in}{2.119516in}}%
\pgfpathlineto{\pgfqpoint{3.502867in}{4.072531in}}%
\pgfusepath{stroke}%
\end{pgfscope}%
\begin{pgfscope}%
\pgfpathrectangle{\pgfqpoint{0.481978in}{0.331635in}}{\pgfqpoint{9.300000in}{7.700000in}}%
\pgfusepath{clip}%
\pgfsetrectcap%
\pgfsetroundjoin%
\pgfsetlinewidth{1.505625pt}%
\definecolor{currentstroke}{rgb}{1.000000,0.705882,0.509804}%
\pgfsetstrokecolor{currentstroke}%
\pgfsetstrokeopacity{0.800000}%
\pgfsetdash{}{0pt}%
\pgfpathmoveto{\pgfqpoint{3.968076in}{5.409564in}}%
\pgfpathlineto{\pgfqpoint{3.502867in}{4.072531in}}%
\pgfusepath{stroke}%
\end{pgfscope}%
\begin{pgfscope}%
\pgfpathrectangle{\pgfqpoint{0.481978in}{0.331635in}}{\pgfqpoint{9.300000in}{7.700000in}}%
\pgfusepath{clip}%
\pgfsetrectcap%
\pgfsetroundjoin%
\pgfsetlinewidth{1.505625pt}%
\definecolor{currentstroke}{rgb}{1.000000,0.705882,0.509804}%
\pgfsetstrokecolor{currentstroke}%
\pgfsetstrokeopacity{0.800000}%
\pgfsetdash{}{0pt}%
\pgfpathmoveto{\pgfqpoint{3.839235in}{2.978822in}}%
\pgfpathlineto{\pgfqpoint{3.502867in}{4.072531in}}%
\pgfusepath{stroke}%
\end{pgfscope}%
\begin{pgfscope}%
\pgfpathrectangle{\pgfqpoint{0.481978in}{0.331635in}}{\pgfqpoint{9.300000in}{7.700000in}}%
\pgfusepath{clip}%
\pgfsetrectcap%
\pgfsetroundjoin%
\pgfsetlinewidth{1.505625pt}%
\definecolor{currentstroke}{rgb}{1.000000,0.705882,0.509804}%
\pgfsetstrokecolor{currentstroke}%
\pgfsetstrokeopacity{0.800000}%
\pgfsetdash{}{0pt}%
\pgfpathmoveto{\pgfqpoint{4.658407in}{6.508294in}}%
\pgfpathlineto{\pgfqpoint{3.502867in}{4.072531in}}%
\pgfusepath{stroke}%
\end{pgfscope}%
\begin{pgfscope}%
\pgfpathrectangle{\pgfqpoint{0.481978in}{0.331635in}}{\pgfqpoint{9.300000in}{7.700000in}}%
\pgfusepath{clip}%
\pgfsetrectcap%
\pgfsetroundjoin%
\pgfsetlinewidth{1.505625pt}%
\definecolor{currentstroke}{rgb}{1.000000,0.705882,0.509804}%
\pgfsetstrokecolor{currentstroke}%
\pgfsetstrokeopacity{0.800000}%
\pgfsetdash{}{0pt}%
\pgfpathmoveto{\pgfqpoint{4.784824in}{5.920838in}}%
\pgfpathlineto{\pgfqpoint{3.502867in}{4.072531in}}%
\pgfusepath{stroke}%
\end{pgfscope}%
\begin{pgfscope}%
\pgfpathrectangle{\pgfqpoint{0.481978in}{0.331635in}}{\pgfqpoint{9.300000in}{7.700000in}}%
\pgfusepath{clip}%
\pgfsetrectcap%
\pgfsetroundjoin%
\pgfsetlinewidth{1.505625pt}%
\definecolor{currentstroke}{rgb}{1.000000,0.705882,0.509804}%
\pgfsetstrokecolor{currentstroke}%
\pgfsetstrokeopacity{0.800000}%
\pgfsetdash{}{0pt}%
\pgfpathmoveto{\pgfqpoint{5.305166in}{4.648380in}}%
\pgfpathlineto{\pgfqpoint{3.502867in}{4.072531in}}%
\pgfusepath{stroke}%
\end{pgfscope}%
\begin{pgfscope}%
\pgfpathrectangle{\pgfqpoint{0.481978in}{0.331635in}}{\pgfqpoint{9.300000in}{7.700000in}}%
\pgfusepath{clip}%
\pgfsetrectcap%
\pgfsetroundjoin%
\pgfsetlinewidth{1.505625pt}%
\definecolor{currentstroke}{rgb}{1.000000,0.705882,0.509804}%
\pgfsetstrokecolor{currentstroke}%
\pgfsetstrokeopacity{0.800000}%
\pgfsetdash{}{0pt}%
\pgfpathmoveto{\pgfqpoint{3.719294in}{4.762095in}}%
\pgfpathlineto{\pgfqpoint{3.502867in}{4.072531in}}%
\pgfusepath{stroke}%
\end{pgfscope}%
\begin{pgfscope}%
\pgfpathrectangle{\pgfqpoint{0.481978in}{0.331635in}}{\pgfqpoint{9.300000in}{7.700000in}}%
\pgfusepath{clip}%
\pgfsetrectcap%
\pgfsetroundjoin%
\pgfsetlinewidth{1.505625pt}%
\definecolor{currentstroke}{rgb}{1.000000,0.705882,0.509804}%
\pgfsetstrokecolor{currentstroke}%
\pgfsetstrokeopacity{0.800000}%
\pgfsetdash{}{0pt}%
\pgfpathmoveto{\pgfqpoint{6.457903in}{1.838210in}}%
\pgfpathlineto{\pgfqpoint{3.502867in}{4.072531in}}%
\pgfusepath{stroke}%
\end{pgfscope}%
\begin{pgfscope}%
\pgfpathrectangle{\pgfqpoint{0.481978in}{0.331635in}}{\pgfqpoint{9.300000in}{7.700000in}}%
\pgfusepath{clip}%
\pgfsetrectcap%
\pgfsetroundjoin%
\pgfsetlinewidth{1.505625pt}%
\definecolor{currentstroke}{rgb}{1.000000,0.705882,0.509804}%
\pgfsetstrokecolor{currentstroke}%
\pgfsetstrokeopacity{0.800000}%
\pgfsetdash{}{0pt}%
\pgfpathmoveto{\pgfqpoint{3.091678in}{5.641728in}}%
\pgfpathlineto{\pgfqpoint{3.502867in}{4.072531in}}%
\pgfusepath{stroke}%
\end{pgfscope}%
\begin{pgfscope}%
\pgfpathrectangle{\pgfqpoint{0.481978in}{0.331635in}}{\pgfqpoint{9.300000in}{7.700000in}}%
\pgfusepath{clip}%
\pgfsetrectcap%
\pgfsetroundjoin%
\pgfsetlinewidth{1.505625pt}%
\definecolor{currentstroke}{rgb}{1.000000,0.705882,0.509804}%
\pgfsetstrokecolor{currentstroke}%
\pgfsetstrokeopacity{0.800000}%
\pgfsetdash{}{0pt}%
\pgfpathmoveto{\pgfqpoint{0.904705in}{4.531254in}}%
\pgfpathlineto{\pgfqpoint{3.502867in}{4.072531in}}%
\pgfusepath{stroke}%
\end{pgfscope}%
\begin{pgfscope}%
\pgfpathrectangle{\pgfqpoint{0.481978in}{0.331635in}}{\pgfqpoint{9.300000in}{7.700000in}}%
\pgfusepath{clip}%
\pgfsetrectcap%
\pgfsetroundjoin%
\pgfsetlinewidth{1.505625pt}%
\definecolor{currentstroke}{rgb}{1.000000,0.705882,0.509804}%
\pgfsetstrokecolor{currentstroke}%
\pgfsetstrokeopacity{0.800000}%
\pgfsetdash{}{0pt}%
\pgfpathmoveto{\pgfqpoint{5.515150in}{3.314117in}}%
\pgfpathlineto{\pgfqpoint{3.502867in}{4.072531in}}%
\pgfusepath{stroke}%
\end{pgfscope}%
\begin{pgfscope}%
\pgfpathrectangle{\pgfqpoint{0.481978in}{0.331635in}}{\pgfqpoint{9.300000in}{7.700000in}}%
\pgfusepath{clip}%
\pgfsetrectcap%
\pgfsetroundjoin%
\pgfsetlinewidth{1.505625pt}%
\definecolor{currentstroke}{rgb}{1.000000,0.705882,0.509804}%
\pgfsetstrokecolor{currentstroke}%
\pgfsetstrokeopacity{0.800000}%
\pgfsetdash{}{0pt}%
\pgfpathmoveto{\pgfqpoint{3.655066in}{5.234687in}}%
\pgfpathlineto{\pgfqpoint{3.502867in}{4.072531in}}%
\pgfusepath{stroke}%
\end{pgfscope}%
\begin{pgfscope}%
\pgfpathrectangle{\pgfqpoint{0.481978in}{0.331635in}}{\pgfqpoint{9.300000in}{7.700000in}}%
\pgfusepath{clip}%
\pgfsetrectcap%
\pgfsetroundjoin%
\pgfsetlinewidth{1.505625pt}%
\definecolor{currentstroke}{rgb}{1.000000,0.705882,0.509804}%
\pgfsetstrokecolor{currentstroke}%
\pgfsetstrokeopacity{0.800000}%
\pgfsetdash{}{0pt}%
\pgfpathmoveto{\pgfqpoint{4.606001in}{2.719624in}}%
\pgfpathlineto{\pgfqpoint{3.502867in}{4.072531in}}%
\pgfusepath{stroke}%
\end{pgfscope}%
\begin{pgfscope}%
\pgfpathrectangle{\pgfqpoint{0.481978in}{0.331635in}}{\pgfqpoint{9.300000in}{7.700000in}}%
\pgfusepath{clip}%
\pgfsetrectcap%
\pgfsetroundjoin%
\pgfsetlinewidth{1.505625pt}%
\definecolor{currentstroke}{rgb}{1.000000,0.705882,0.509804}%
\pgfsetstrokecolor{currentstroke}%
\pgfsetstrokeopacity{0.800000}%
\pgfsetdash{}{0pt}%
\pgfpathmoveto{\pgfqpoint{2.788909in}{4.554933in}}%
\pgfpathlineto{\pgfqpoint{3.502867in}{4.072531in}}%
\pgfusepath{stroke}%
\end{pgfscope}%
\begin{pgfscope}%
\pgfpathrectangle{\pgfqpoint{0.481978in}{0.331635in}}{\pgfqpoint{9.300000in}{7.700000in}}%
\pgfusepath{clip}%
\pgfsetrectcap%
\pgfsetroundjoin%
\pgfsetlinewidth{1.505625pt}%
\definecolor{currentstroke}{rgb}{1.000000,0.705882,0.509804}%
\pgfsetstrokecolor{currentstroke}%
\pgfsetstrokeopacity{0.800000}%
\pgfsetdash{}{0pt}%
\pgfpathmoveto{\pgfqpoint{2.529879in}{3.185116in}}%
\pgfpathlineto{\pgfqpoint{3.502867in}{4.072531in}}%
\pgfusepath{stroke}%
\end{pgfscope}%
\begin{pgfscope}%
\pgfpathrectangle{\pgfqpoint{0.481978in}{0.331635in}}{\pgfqpoint{9.300000in}{7.700000in}}%
\pgfusepath{clip}%
\pgfsetrectcap%
\pgfsetroundjoin%
\pgfsetlinewidth{1.505625pt}%
\definecolor{currentstroke}{rgb}{1.000000,0.705882,0.509804}%
\pgfsetstrokecolor{currentstroke}%
\pgfsetstrokeopacity{0.800000}%
\pgfsetdash{}{0pt}%
\pgfpathmoveto{\pgfqpoint{2.960587in}{2.541098in}}%
\pgfpathlineto{\pgfqpoint{3.502867in}{4.072531in}}%
\pgfusepath{stroke}%
\end{pgfscope}%
\begin{pgfscope}%
\pgfpathrectangle{\pgfqpoint{0.481978in}{0.331635in}}{\pgfqpoint{9.300000in}{7.700000in}}%
\pgfusepath{clip}%
\pgfsetrectcap%
\pgfsetroundjoin%
\pgfsetlinewidth{1.505625pt}%
\definecolor{currentstroke}{rgb}{1.000000,0.705882,0.509804}%
\pgfsetstrokecolor{currentstroke}%
\pgfsetstrokeopacity{0.800000}%
\pgfsetdash{}{0pt}%
\pgfpathmoveto{\pgfqpoint{8.052568in}{3.873744in}}%
\pgfpathlineto{\pgfqpoint{3.502867in}{4.072531in}}%
\pgfusepath{stroke}%
\end{pgfscope}%
\begin{pgfscope}%
\pgfpathrectangle{\pgfqpoint{0.481978in}{0.331635in}}{\pgfqpoint{9.300000in}{7.700000in}}%
\pgfusepath{clip}%
\pgfsetrectcap%
\pgfsetroundjoin%
\pgfsetlinewidth{1.505625pt}%
\definecolor{currentstroke}{rgb}{1.000000,0.705882,0.509804}%
\pgfsetstrokecolor{currentstroke}%
\pgfsetstrokeopacity{0.800000}%
\pgfsetdash{}{0pt}%
\pgfpathmoveto{\pgfqpoint{1.457328in}{2.537487in}}%
\pgfpathlineto{\pgfqpoint{3.502867in}{4.072531in}}%
\pgfusepath{stroke}%
\end{pgfscope}%
\begin{pgfscope}%
\pgfpathrectangle{\pgfqpoint{0.481978in}{0.331635in}}{\pgfqpoint{9.300000in}{7.700000in}}%
\pgfusepath{clip}%
\pgfsetrectcap%
\pgfsetroundjoin%
\pgfsetlinewidth{1.505625pt}%
\definecolor{currentstroke}{rgb}{1.000000,0.705882,0.509804}%
\pgfsetstrokecolor{currentstroke}%
\pgfsetstrokeopacity{0.800000}%
\pgfsetdash{}{0pt}%
\pgfpathmoveto{\pgfqpoint{1.959498in}{3.653329in}}%
\pgfpathlineto{\pgfqpoint{3.502867in}{4.072531in}}%
\pgfusepath{stroke}%
\end{pgfscope}%
\begin{pgfscope}%
\pgfpathrectangle{\pgfqpoint{0.481978in}{0.331635in}}{\pgfqpoint{9.300000in}{7.700000in}}%
\pgfusepath{clip}%
\pgfsetrectcap%
\pgfsetroundjoin%
\pgfsetlinewidth{1.505625pt}%
\definecolor{currentstroke}{rgb}{1.000000,0.705882,0.509804}%
\pgfsetstrokecolor{currentstroke}%
\pgfsetstrokeopacity{0.800000}%
\pgfsetdash{}{0pt}%
\pgfpathmoveto{\pgfqpoint{5.141137in}{2.895623in}}%
\pgfpathlineto{\pgfqpoint{3.502867in}{4.072531in}}%
\pgfusepath{stroke}%
\end{pgfscope}%
\begin{pgfscope}%
\pgfpathrectangle{\pgfqpoint{0.481978in}{0.331635in}}{\pgfqpoint{9.300000in}{7.700000in}}%
\pgfusepath{clip}%
\pgfsetrectcap%
\pgfsetroundjoin%
\pgfsetlinewidth{1.505625pt}%
\definecolor{currentstroke}{rgb}{1.000000,0.705882,0.509804}%
\pgfsetstrokecolor{currentstroke}%
\pgfsetstrokeopacity{0.800000}%
\pgfsetdash{}{0pt}%
\pgfpathmoveto{\pgfqpoint{5.415961in}{3.947445in}}%
\pgfpathlineto{\pgfqpoint{3.502867in}{4.072531in}}%
\pgfusepath{stroke}%
\end{pgfscope}%
\begin{pgfscope}%
\pgfpathrectangle{\pgfqpoint{0.481978in}{0.331635in}}{\pgfqpoint{9.300000in}{7.700000in}}%
\pgfusepath{clip}%
\pgfsetrectcap%
\pgfsetroundjoin%
\pgfsetlinewidth{1.505625pt}%
\definecolor{currentstroke}{rgb}{1.000000,0.705882,0.509804}%
\pgfsetstrokecolor{currentstroke}%
\pgfsetstrokeopacity{0.800000}%
\pgfsetdash{}{0pt}%
\pgfpathmoveto{\pgfqpoint{3.418504in}{2.608047in}}%
\pgfpathlineto{\pgfqpoint{3.502867in}{4.072531in}}%
\pgfusepath{stroke}%
\end{pgfscope}%
\begin{pgfscope}%
\pgfpathrectangle{\pgfqpoint{0.481978in}{0.331635in}}{\pgfqpoint{9.300000in}{7.700000in}}%
\pgfusepath{clip}%
\pgfsetrectcap%
\pgfsetroundjoin%
\pgfsetlinewidth{1.505625pt}%
\definecolor{currentstroke}{rgb}{1.000000,0.705882,0.509804}%
\pgfsetstrokecolor{currentstroke}%
\pgfsetstrokeopacity{0.800000}%
\pgfsetdash{}{0pt}%
\pgfpathmoveto{\pgfqpoint{1.418868in}{3.723417in}}%
\pgfpathlineto{\pgfqpoint{3.502867in}{4.072531in}}%
\pgfusepath{stroke}%
\end{pgfscope}%
\begin{pgfscope}%
\pgfpathrectangle{\pgfqpoint{0.481978in}{0.331635in}}{\pgfqpoint{9.300000in}{7.700000in}}%
\pgfusepath{clip}%
\pgfsetrectcap%
\pgfsetroundjoin%
\pgfsetlinewidth{1.505625pt}%
\definecolor{currentstroke}{rgb}{1.000000,0.705882,0.509804}%
\pgfsetstrokecolor{currentstroke}%
\pgfsetstrokeopacity{0.800000}%
\pgfsetdash{}{0pt}%
\pgfpathmoveto{\pgfqpoint{4.390885in}{4.246675in}}%
\pgfpathlineto{\pgfqpoint{3.502867in}{4.072531in}}%
\pgfusepath{stroke}%
\end{pgfscope}%
\begin{pgfscope}%
\pgfpathrectangle{\pgfqpoint{0.481978in}{0.331635in}}{\pgfqpoint{9.300000in}{7.700000in}}%
\pgfusepath{clip}%
\pgfsetrectcap%
\pgfsetroundjoin%
\pgfsetlinewidth{1.505625pt}%
\definecolor{currentstroke}{rgb}{1.000000,0.705882,0.509804}%
\pgfsetstrokecolor{currentstroke}%
\pgfsetstrokeopacity{0.800000}%
\pgfsetdash{}{0pt}%
\pgfpathmoveto{\pgfqpoint{2.103399in}{2.786137in}}%
\pgfpathlineto{\pgfqpoint{3.502867in}{4.072531in}}%
\pgfusepath{stroke}%
\end{pgfscope}%
\begin{pgfscope}%
\pgfpathrectangle{\pgfqpoint{0.481978in}{0.331635in}}{\pgfqpoint{9.300000in}{7.700000in}}%
\pgfusepath{clip}%
\pgfsetrectcap%
\pgfsetroundjoin%
\pgfsetlinewidth{1.505625pt}%
\definecolor{currentstroke}{rgb}{1.000000,0.705882,0.509804}%
\pgfsetstrokecolor{currentstroke}%
\pgfsetstrokeopacity{0.800000}%
\pgfsetdash{}{0pt}%
\pgfpathmoveto{\pgfqpoint{3.156352in}{3.695503in}}%
\pgfpathlineto{\pgfqpoint{3.502867in}{4.072531in}}%
\pgfusepath{stroke}%
\end{pgfscope}%
\begin{pgfscope}%
\pgfpathrectangle{\pgfqpoint{0.481978in}{0.331635in}}{\pgfqpoint{9.300000in}{7.700000in}}%
\pgfusepath{clip}%
\pgfsetrectcap%
\pgfsetroundjoin%
\pgfsetlinewidth{1.505625pt}%
\definecolor{currentstroke}{rgb}{1.000000,0.705882,0.509804}%
\pgfsetstrokecolor{currentstroke}%
\pgfsetstrokeopacity{0.800000}%
\pgfsetdash{}{0pt}%
\pgfpathmoveto{\pgfqpoint{4.782434in}{5.411059in}}%
\pgfpathlineto{\pgfqpoint{3.502867in}{4.072531in}}%
\pgfusepath{stroke}%
\end{pgfscope}%
\begin{pgfscope}%
\pgfpathrectangle{\pgfqpoint{0.481978in}{0.331635in}}{\pgfqpoint{9.300000in}{7.700000in}}%
\pgfusepath{clip}%
\pgfsetrectcap%
\pgfsetroundjoin%
\pgfsetlinewidth{1.505625pt}%
\definecolor{currentstroke}{rgb}{1.000000,0.705882,0.509804}%
\pgfsetstrokecolor{currentstroke}%
\pgfsetstrokeopacity{0.800000}%
\pgfsetdash{}{0pt}%
\pgfpathmoveto{\pgfqpoint{4.313894in}{2.603823in}}%
\pgfpathlineto{\pgfqpoint{3.502867in}{4.072531in}}%
\pgfusepath{stroke}%
\end{pgfscope}%
\begin{pgfscope}%
\pgfpathrectangle{\pgfqpoint{0.481978in}{0.331635in}}{\pgfqpoint{9.300000in}{7.700000in}}%
\pgfusepath{clip}%
\pgfsetrectcap%
\pgfsetroundjoin%
\pgfsetlinewidth{1.505625pt}%
\definecolor{currentstroke}{rgb}{1.000000,0.705882,0.509804}%
\pgfsetstrokecolor{currentstroke}%
\pgfsetstrokeopacity{0.800000}%
\pgfsetdash{}{0pt}%
\pgfpathmoveto{\pgfqpoint{3.534223in}{5.922879in}}%
\pgfpathlineto{\pgfqpoint{3.502867in}{4.072531in}}%
\pgfusepath{stroke}%
\end{pgfscope}%
\begin{pgfscope}%
\pgfpathrectangle{\pgfqpoint{0.481978in}{0.331635in}}{\pgfqpoint{9.300000in}{7.700000in}}%
\pgfusepath{clip}%
\pgfsetrectcap%
\pgfsetroundjoin%
\pgfsetlinewidth{1.505625pt}%
\definecolor{currentstroke}{rgb}{1.000000,0.705882,0.509804}%
\pgfsetstrokecolor{currentstroke}%
\pgfsetstrokeopacity{0.800000}%
\pgfsetdash{}{0pt}%
\pgfpathmoveto{\pgfqpoint{2.718188in}{5.699839in}}%
\pgfpathlineto{\pgfqpoint{3.502867in}{4.072531in}}%
\pgfusepath{stroke}%
\end{pgfscope}%
\begin{pgfscope}%
\pgfpathrectangle{\pgfqpoint{0.481978in}{0.331635in}}{\pgfqpoint{9.300000in}{7.700000in}}%
\pgfusepath{clip}%
\pgfsetrectcap%
\pgfsetroundjoin%
\pgfsetlinewidth{1.505625pt}%
\definecolor{currentstroke}{rgb}{1.000000,0.705882,0.509804}%
\pgfsetstrokecolor{currentstroke}%
\pgfsetstrokeopacity{0.800000}%
\pgfsetdash{}{0pt}%
\pgfpathmoveto{\pgfqpoint{4.798223in}{3.954560in}}%
\pgfpathlineto{\pgfqpoint{3.502867in}{4.072531in}}%
\pgfusepath{stroke}%
\end{pgfscope}%
\begin{pgfscope}%
\pgfpathrectangle{\pgfqpoint{0.481978in}{0.331635in}}{\pgfqpoint{9.300000in}{7.700000in}}%
\pgfusepath{clip}%
\pgfsetrectcap%
\pgfsetroundjoin%
\pgfsetlinewidth{1.505625pt}%
\definecolor{currentstroke}{rgb}{1.000000,0.705882,0.509804}%
\pgfsetstrokecolor{currentstroke}%
\pgfsetstrokeopacity{0.800000}%
\pgfsetdash{}{0pt}%
\pgfpathmoveto{\pgfqpoint{2.688115in}{4.317587in}}%
\pgfpathlineto{\pgfqpoint{3.502867in}{4.072531in}}%
\pgfusepath{stroke}%
\end{pgfscope}%
\begin{pgfscope}%
\pgfpathrectangle{\pgfqpoint{0.481978in}{0.331635in}}{\pgfqpoint{9.300000in}{7.700000in}}%
\pgfusepath{clip}%
\pgfsetrectcap%
\pgfsetroundjoin%
\pgfsetlinewidth{1.505625pt}%
\definecolor{currentstroke}{rgb}{1.000000,0.705882,0.509804}%
\pgfsetstrokecolor{currentstroke}%
\pgfsetstrokeopacity{0.800000}%
\pgfsetdash{}{0pt}%
\pgfpathmoveto{\pgfqpoint{3.356009in}{3.332822in}}%
\pgfpathlineto{\pgfqpoint{3.502867in}{4.072531in}}%
\pgfusepath{stroke}%
\end{pgfscope}%
\begin{pgfscope}%
\pgfpathrectangle{\pgfqpoint{0.481978in}{0.331635in}}{\pgfqpoint{9.300000in}{7.700000in}}%
\pgfusepath{clip}%
\pgfsetrectcap%
\pgfsetroundjoin%
\pgfsetlinewidth{1.505625pt}%
\definecolor{currentstroke}{rgb}{1.000000,0.705882,0.509804}%
\pgfsetstrokecolor{currentstroke}%
\pgfsetstrokeopacity{0.800000}%
\pgfsetdash{}{0pt}%
\pgfpathmoveto{\pgfqpoint{2.750103in}{4.755156in}}%
\pgfpathlineto{\pgfqpoint{3.502867in}{4.072531in}}%
\pgfusepath{stroke}%
\end{pgfscope}%
\begin{pgfscope}%
\pgfpathrectangle{\pgfqpoint{0.481978in}{0.331635in}}{\pgfqpoint{9.300000in}{7.700000in}}%
\pgfusepath{clip}%
\pgfsetrectcap%
\pgfsetroundjoin%
\pgfsetlinewidth{1.505625pt}%
\definecolor{currentstroke}{rgb}{1.000000,0.705882,0.509804}%
\pgfsetstrokecolor{currentstroke}%
\pgfsetstrokeopacity{0.800000}%
\pgfsetdash{}{0pt}%
\pgfpathmoveto{\pgfqpoint{3.502659in}{2.555499in}}%
\pgfpathlineto{\pgfqpoint{3.502867in}{4.072531in}}%
\pgfusepath{stroke}%
\end{pgfscope}%
\begin{pgfscope}%
\pgfpathrectangle{\pgfqpoint{0.481978in}{0.331635in}}{\pgfqpoint{9.300000in}{7.700000in}}%
\pgfusepath{clip}%
\pgfsetrectcap%
\pgfsetroundjoin%
\pgfsetlinewidth{1.505625pt}%
\definecolor{currentstroke}{rgb}{1.000000,0.705882,0.509804}%
\pgfsetstrokecolor{currentstroke}%
\pgfsetstrokeopacity{0.800000}%
\pgfsetdash{}{0pt}%
\pgfpathmoveto{\pgfqpoint{3.883915in}{4.747512in}}%
\pgfpathlineto{\pgfqpoint{3.502867in}{4.072531in}}%
\pgfusepath{stroke}%
\end{pgfscope}%
\begin{pgfscope}%
\pgfpathrectangle{\pgfqpoint{0.481978in}{0.331635in}}{\pgfqpoint{9.300000in}{7.700000in}}%
\pgfusepath{clip}%
\pgfsetrectcap%
\pgfsetroundjoin%
\pgfsetlinewidth{1.505625pt}%
\definecolor{currentstroke}{rgb}{1.000000,0.705882,0.509804}%
\pgfsetstrokecolor{currentstroke}%
\pgfsetstrokeopacity{0.800000}%
\pgfsetdash{}{0pt}%
\pgfpathmoveto{\pgfqpoint{4.261705in}{2.344287in}}%
\pgfpathlineto{\pgfqpoint{3.502867in}{4.072531in}}%
\pgfusepath{stroke}%
\end{pgfscope}%
\begin{pgfscope}%
\pgfpathrectangle{\pgfqpoint{0.481978in}{0.331635in}}{\pgfqpoint{9.300000in}{7.700000in}}%
\pgfusepath{clip}%
\pgfsetrectcap%
\pgfsetroundjoin%
\pgfsetlinewidth{1.505625pt}%
\definecolor{currentstroke}{rgb}{1.000000,0.705882,0.509804}%
\pgfsetstrokecolor{currentstroke}%
\pgfsetstrokeopacity{0.800000}%
\pgfsetdash{}{0pt}%
\pgfpathmoveto{\pgfqpoint{1.767331in}{4.669274in}}%
\pgfpathlineto{\pgfqpoint{3.502867in}{4.072531in}}%
\pgfusepath{stroke}%
\end{pgfscope}%
\begin{pgfscope}%
\pgfpathrectangle{\pgfqpoint{0.481978in}{0.331635in}}{\pgfqpoint{9.300000in}{7.700000in}}%
\pgfusepath{clip}%
\pgfsetrectcap%
\pgfsetroundjoin%
\pgfsetlinewidth{1.505625pt}%
\definecolor{currentstroke}{rgb}{1.000000,0.705882,0.509804}%
\pgfsetstrokecolor{currentstroke}%
\pgfsetstrokeopacity{0.800000}%
\pgfsetdash{}{0pt}%
\pgfpathmoveto{\pgfqpoint{1.457321in}{4.685354in}}%
\pgfpathlineto{\pgfqpoint{3.502867in}{4.072531in}}%
\pgfusepath{stroke}%
\end{pgfscope}%
\begin{pgfscope}%
\pgfpathrectangle{\pgfqpoint{0.481978in}{0.331635in}}{\pgfqpoint{9.300000in}{7.700000in}}%
\pgfusepath{clip}%
\pgfsetrectcap%
\pgfsetroundjoin%
\pgfsetlinewidth{1.505625pt}%
\definecolor{currentstroke}{rgb}{1.000000,0.705882,0.509804}%
\pgfsetstrokecolor{currentstroke}%
\pgfsetstrokeopacity{0.800000}%
\pgfsetdash{}{0pt}%
\pgfpathmoveto{\pgfqpoint{3.872133in}{3.288225in}}%
\pgfpathlineto{\pgfqpoint{3.502867in}{4.072531in}}%
\pgfusepath{stroke}%
\end{pgfscope}%
\begin{pgfscope}%
\pgfpathrectangle{\pgfqpoint{0.481978in}{0.331635in}}{\pgfqpoint{9.300000in}{7.700000in}}%
\pgfusepath{clip}%
\pgfsetrectcap%
\pgfsetroundjoin%
\pgfsetlinewidth{1.505625pt}%
\definecolor{currentstroke}{rgb}{1.000000,0.705882,0.509804}%
\pgfsetstrokecolor{currentstroke}%
\pgfsetstrokeopacity{0.800000}%
\pgfsetdash{}{0pt}%
\pgfpathmoveto{\pgfqpoint{3.604393in}{3.101026in}}%
\pgfpathlineto{\pgfqpoint{3.502867in}{4.072531in}}%
\pgfusepath{stroke}%
\end{pgfscope}%
\begin{pgfscope}%
\pgfpathrectangle{\pgfqpoint{0.481978in}{0.331635in}}{\pgfqpoint{9.300000in}{7.700000in}}%
\pgfusepath{clip}%
\pgfsetrectcap%
\pgfsetroundjoin%
\pgfsetlinewidth{1.505625pt}%
\definecolor{currentstroke}{rgb}{1.000000,0.705882,0.509804}%
\pgfsetstrokecolor{currentstroke}%
\pgfsetstrokeopacity{0.800000}%
\pgfsetdash{}{0pt}%
\pgfpathmoveto{\pgfqpoint{3.157682in}{4.441797in}}%
\pgfpathlineto{\pgfqpoint{3.502867in}{4.072531in}}%
\pgfusepath{stroke}%
\end{pgfscope}%
\begin{pgfscope}%
\pgfpathrectangle{\pgfqpoint{0.481978in}{0.331635in}}{\pgfqpoint{9.300000in}{7.700000in}}%
\pgfusepath{clip}%
\pgfsetrectcap%
\pgfsetroundjoin%
\pgfsetlinewidth{1.505625pt}%
\definecolor{currentstroke}{rgb}{1.000000,0.705882,0.509804}%
\pgfsetstrokecolor{currentstroke}%
\pgfsetstrokeopacity{0.800000}%
\pgfsetdash{}{0pt}%
\pgfpathmoveto{\pgfqpoint{3.462770in}{5.112041in}}%
\pgfpathlineto{\pgfqpoint{3.502867in}{4.072531in}}%
\pgfusepath{stroke}%
\end{pgfscope}%
\begin{pgfscope}%
\pgfpathrectangle{\pgfqpoint{0.481978in}{0.331635in}}{\pgfqpoint{9.300000in}{7.700000in}}%
\pgfusepath{clip}%
\pgfsetrectcap%
\pgfsetroundjoin%
\pgfsetlinewidth{1.505625pt}%
\definecolor{currentstroke}{rgb}{1.000000,0.705882,0.509804}%
\pgfsetstrokecolor{currentstroke}%
\pgfsetstrokeopacity{0.800000}%
\pgfsetdash{}{0pt}%
\pgfpathmoveto{\pgfqpoint{4.591082in}{3.564558in}}%
\pgfpathlineto{\pgfqpoint{3.502867in}{4.072531in}}%
\pgfusepath{stroke}%
\end{pgfscope}%
\begin{pgfscope}%
\pgfpathrectangle{\pgfqpoint{0.481978in}{0.331635in}}{\pgfqpoint{9.300000in}{7.700000in}}%
\pgfusepath{clip}%
\pgfsetrectcap%
\pgfsetroundjoin%
\pgfsetlinewidth{1.505625pt}%
\definecolor{currentstroke}{rgb}{1.000000,0.705882,0.509804}%
\pgfsetstrokecolor{currentstroke}%
\pgfsetstrokeopacity{0.800000}%
\pgfsetdash{}{0pt}%
\pgfpathmoveto{\pgfqpoint{5.774624in}{4.599244in}}%
\pgfpathlineto{\pgfqpoint{3.502867in}{4.072531in}}%
\pgfusepath{stroke}%
\end{pgfscope}%
\begin{pgfscope}%
\pgfpathrectangle{\pgfqpoint{0.481978in}{0.331635in}}{\pgfqpoint{9.300000in}{7.700000in}}%
\pgfusepath{clip}%
\pgfsetrectcap%
\pgfsetroundjoin%
\pgfsetlinewidth{1.505625pt}%
\definecolor{currentstroke}{rgb}{1.000000,0.705882,0.509804}%
\pgfsetstrokecolor{currentstroke}%
\pgfsetstrokeopacity{0.800000}%
\pgfsetdash{}{0pt}%
\pgfpathmoveto{\pgfqpoint{3.755577in}{2.708345in}}%
\pgfpathlineto{\pgfqpoint{3.502867in}{4.072531in}}%
\pgfusepath{stroke}%
\end{pgfscope}%
\begin{pgfscope}%
\pgfpathrectangle{\pgfqpoint{0.481978in}{0.331635in}}{\pgfqpoint{9.300000in}{7.700000in}}%
\pgfusepath{clip}%
\pgfsetrectcap%
\pgfsetroundjoin%
\pgfsetlinewidth{1.505625pt}%
\definecolor{currentstroke}{rgb}{1.000000,0.705882,0.509804}%
\pgfsetstrokecolor{currentstroke}%
\pgfsetstrokeopacity{0.800000}%
\pgfsetdash{}{0pt}%
\pgfpathmoveto{\pgfqpoint{2.355130in}{2.601916in}}%
\pgfpathlineto{\pgfqpoint{3.502867in}{4.072531in}}%
\pgfusepath{stroke}%
\end{pgfscope}%
\begin{pgfscope}%
\pgfpathrectangle{\pgfqpoint{0.481978in}{0.331635in}}{\pgfqpoint{9.300000in}{7.700000in}}%
\pgfusepath{clip}%
\pgfsetrectcap%
\pgfsetroundjoin%
\pgfsetlinewidth{1.505625pt}%
\definecolor{currentstroke}{rgb}{1.000000,0.705882,0.509804}%
\pgfsetstrokecolor{currentstroke}%
\pgfsetstrokeopacity{0.800000}%
\pgfsetdash{}{0pt}%
\pgfpathmoveto{\pgfqpoint{3.074624in}{3.287134in}}%
\pgfpathlineto{\pgfqpoint{3.502867in}{4.072531in}}%
\pgfusepath{stroke}%
\end{pgfscope}%
\begin{pgfscope}%
\pgfpathrectangle{\pgfqpoint{0.481978in}{0.331635in}}{\pgfqpoint{9.300000in}{7.700000in}}%
\pgfusepath{clip}%
\pgfsetrectcap%
\pgfsetroundjoin%
\pgfsetlinewidth{1.505625pt}%
\definecolor{currentstroke}{rgb}{1.000000,0.705882,0.509804}%
\pgfsetstrokecolor{currentstroke}%
\pgfsetstrokeopacity{0.800000}%
\pgfsetdash{}{0pt}%
\pgfpathmoveto{\pgfqpoint{2.232057in}{4.476601in}}%
\pgfpathlineto{\pgfqpoint{3.502867in}{4.072531in}}%
\pgfusepath{stroke}%
\end{pgfscope}%
\begin{pgfscope}%
\pgfpathrectangle{\pgfqpoint{0.481978in}{0.331635in}}{\pgfqpoint{9.300000in}{7.700000in}}%
\pgfusepath{clip}%
\pgfsetrectcap%
\pgfsetroundjoin%
\pgfsetlinewidth{1.505625pt}%
\definecolor{currentstroke}{rgb}{1.000000,0.705882,0.509804}%
\pgfsetstrokecolor{currentstroke}%
\pgfsetstrokeopacity{0.800000}%
\pgfsetdash{}{0pt}%
\pgfpathmoveto{\pgfqpoint{3.954754in}{3.742286in}}%
\pgfpathlineto{\pgfqpoint{3.502867in}{4.072531in}}%
\pgfusepath{stroke}%
\end{pgfscope}%
\begin{pgfscope}%
\pgfpathrectangle{\pgfqpoint{0.481978in}{0.331635in}}{\pgfqpoint{9.300000in}{7.700000in}}%
\pgfusepath{clip}%
\pgfsetrectcap%
\pgfsetroundjoin%
\pgfsetlinewidth{1.505625pt}%
\definecolor{currentstroke}{rgb}{1.000000,0.705882,0.509804}%
\pgfsetstrokecolor{currentstroke}%
\pgfsetstrokeopacity{0.800000}%
\pgfsetdash{}{0pt}%
\pgfpathmoveto{\pgfqpoint{1.921313in}{5.102923in}}%
\pgfpathlineto{\pgfqpoint{3.502867in}{4.072531in}}%
\pgfusepath{stroke}%
\end{pgfscope}%
\begin{pgfscope}%
\pgfpathrectangle{\pgfqpoint{0.481978in}{0.331635in}}{\pgfqpoint{9.300000in}{7.700000in}}%
\pgfusepath{clip}%
\pgfsetrectcap%
\pgfsetroundjoin%
\pgfsetlinewidth{1.505625pt}%
\definecolor{currentstroke}{rgb}{1.000000,0.705882,0.509804}%
\pgfsetstrokecolor{currentstroke}%
\pgfsetstrokeopacity{0.800000}%
\pgfsetdash{}{0pt}%
\pgfpathmoveto{\pgfqpoint{3.918352in}{4.443858in}}%
\pgfpathlineto{\pgfqpoint{3.502867in}{4.072531in}}%
\pgfusepath{stroke}%
\end{pgfscope}%
\begin{pgfscope}%
\pgfpathrectangle{\pgfqpoint{0.481978in}{0.331635in}}{\pgfqpoint{9.300000in}{7.700000in}}%
\pgfusepath{clip}%
\pgfsetrectcap%
\pgfsetroundjoin%
\pgfsetlinewidth{1.505625pt}%
\definecolor{currentstroke}{rgb}{1.000000,0.705882,0.509804}%
\pgfsetstrokecolor{currentstroke}%
\pgfsetstrokeopacity{0.800000}%
\pgfsetdash{}{0pt}%
\pgfpathmoveto{\pgfqpoint{1.874082in}{4.443883in}}%
\pgfpathlineto{\pgfqpoint{3.502867in}{4.072531in}}%
\pgfusepath{stroke}%
\end{pgfscope}%
\begin{pgfscope}%
\pgfpathrectangle{\pgfqpoint{0.481978in}{0.331635in}}{\pgfqpoint{9.300000in}{7.700000in}}%
\pgfusepath{clip}%
\pgfsetrectcap%
\pgfsetroundjoin%
\pgfsetlinewidth{1.505625pt}%
\definecolor{currentstroke}{rgb}{1.000000,0.705882,0.509804}%
\pgfsetstrokecolor{currentstroke}%
\pgfsetstrokeopacity{0.800000}%
\pgfsetdash{}{0pt}%
\pgfpathmoveto{\pgfqpoint{4.966201in}{4.850330in}}%
\pgfpathlineto{\pgfqpoint{3.502867in}{4.072531in}}%
\pgfusepath{stroke}%
\end{pgfscope}%
\begin{pgfscope}%
\pgfsetrectcap%
\pgfsetmiterjoin%
\pgfsetlinewidth{0.803000pt}%
\definecolor{currentstroke}{rgb}{0.000000,0.000000,0.000000}%
\pgfsetstrokecolor{currentstroke}%
\pgfsetdash{}{0pt}%
\pgfpathmoveto{\pgfqpoint{0.481978in}{0.331635in}}%
\pgfpathlineto{\pgfqpoint{0.481978in}{8.031635in}}%
\pgfusepath{stroke}%
\end{pgfscope}%
\begin{pgfscope}%
\pgfsetrectcap%
\pgfsetmiterjoin%
\pgfsetlinewidth{0.803000pt}%
\definecolor{currentstroke}{rgb}{0.000000,0.000000,0.000000}%
\pgfsetstrokecolor{currentstroke}%
\pgfsetdash{}{0pt}%
\pgfpathmoveto{\pgfqpoint{9.781978in}{0.331635in}}%
\pgfpathlineto{\pgfqpoint{9.781978in}{8.031635in}}%
\pgfusepath{stroke}%
\end{pgfscope}%
\begin{pgfscope}%
\pgfsetrectcap%
\pgfsetmiterjoin%
\pgfsetlinewidth{0.803000pt}%
\definecolor{currentstroke}{rgb}{0.000000,0.000000,0.000000}%
\pgfsetstrokecolor{currentstroke}%
\pgfsetdash{}{0pt}%
\pgfpathmoveto{\pgfqpoint{0.481978in}{0.331635in}}%
\pgfpathlineto{\pgfqpoint{9.781978in}{0.331635in}}%
\pgfusepath{stroke}%
\end{pgfscope}%
\begin{pgfscope}%
\pgfsetrectcap%
\pgfsetmiterjoin%
\pgfsetlinewidth{0.803000pt}%
\definecolor{currentstroke}{rgb}{0.000000,0.000000,0.000000}%
\pgfsetstrokecolor{currentstroke}%
\pgfsetdash{}{0pt}%
\pgfpathmoveto{\pgfqpoint{0.481978in}{8.031635in}}%
\pgfpathlineto{\pgfqpoint{9.781978in}{8.031635in}}%
\pgfusepath{stroke}%
\end{pgfscope}%
\begin{pgfscope}%
\definecolor{textcolor}{rgb}{0.000000,0.000000,0.000000}%
\pgfsetstrokecolor{textcolor}%
\pgfsetfillcolor{textcolor}%
\pgftext[x=5.131978in,y=8.114968in,,base]{\color{textcolor}\sffamily\fontsize{12.000000}{14.400000}\selectfont T-SNE for Pix3D and S2R:3DFREE}%
\end{pgfscope}%
\begin{pgfscope}%
\pgfsetbuttcap%
\pgfsetmiterjoin%
\definecolor{currentfill}{rgb}{1.000000,1.000000,1.000000}%
\pgfsetfillcolor{currentfill}%
\pgfsetfillopacity{0.800000}%
\pgfsetlinewidth{1.003750pt}%
\definecolor{currentstroke}{rgb}{0.800000,0.800000,0.800000}%
\pgfsetstrokecolor{currentstroke}%
\pgfsetstrokeopacity{0.800000}%
\pgfsetdash{}{0pt}%
\pgfpathmoveto{\pgfqpoint{9.879200in}{3.956944in}}%
\pgfpathlineto{\pgfqpoint{11.186410in}{3.956944in}}%
\pgfpathquadraticcurveto{\pgfqpoint{11.214188in}{3.956944in}}{\pgfqpoint{11.214188in}{3.984722in}}%
\pgfpathlineto{\pgfqpoint{11.214188in}{4.378548in}}%
\pgfpathquadraticcurveto{\pgfqpoint{11.214188in}{4.406326in}}{\pgfqpoint{11.186410in}{4.406326in}}%
\pgfpathlineto{\pgfqpoint{9.879200in}{4.406326in}}%
\pgfpathquadraticcurveto{\pgfqpoint{9.851422in}{4.406326in}}{\pgfqpoint{9.851422in}{4.378548in}}%
\pgfpathlineto{\pgfqpoint{9.851422in}{3.984722in}}%
\pgfpathquadraticcurveto{\pgfqpoint{9.851422in}{3.956944in}}{\pgfqpoint{9.879200in}{3.956944in}}%
\pgfpathclose%
\pgfusepath{stroke,fill}%
\end{pgfscope}%
\begin{pgfscope}%
\pgfsetbuttcap%
\pgfsetroundjoin%
\definecolor{currentfill}{rgb}{0.631373,0.788235,0.956863}%
\pgfsetfillcolor{currentfill}%
\pgfsetlinewidth{1.003750pt}%
\definecolor{currentstroke}{rgb}{0.631373,0.788235,0.956863}%
\pgfsetstrokecolor{currentstroke}%
\pgfsetdash{}{0pt}%
\pgfsys@defobject{currentmarker}{\pgfqpoint{-0.041667in}{-0.041667in}}{\pgfqpoint{0.041667in}{0.041667in}}{%
\pgfpathmoveto{\pgfqpoint{0.000000in}{-0.041667in}}%
\pgfpathcurveto{\pgfqpoint{0.011050in}{-0.041667in}}{\pgfqpoint{0.021649in}{-0.037276in}}{\pgfqpoint{0.029463in}{-0.029463in}}%
\pgfpathcurveto{\pgfqpoint{0.037276in}{-0.021649in}}{\pgfqpoint{0.041667in}{-0.011050in}}{\pgfqpoint{0.041667in}{0.000000in}}%
\pgfpathcurveto{\pgfqpoint{0.041667in}{0.011050in}}{\pgfqpoint{0.037276in}{0.021649in}}{\pgfqpoint{0.029463in}{0.029463in}}%
\pgfpathcurveto{\pgfqpoint{0.021649in}{0.037276in}}{\pgfqpoint{0.011050in}{0.041667in}}{\pgfqpoint{0.000000in}{0.041667in}}%
\pgfpathcurveto{\pgfqpoint{-0.011050in}{0.041667in}}{\pgfqpoint{-0.021649in}{0.037276in}}{\pgfqpoint{-0.029463in}{0.029463in}}%
\pgfpathcurveto{\pgfqpoint{-0.037276in}{0.021649in}}{\pgfqpoint{-0.041667in}{0.011050in}}{\pgfqpoint{-0.041667in}{0.000000in}}%
\pgfpathcurveto{\pgfqpoint{-0.041667in}{-0.011050in}}{\pgfqpoint{-0.037276in}{-0.021649in}}{\pgfqpoint{-0.029463in}{-0.029463in}}%
\pgfpathcurveto{\pgfqpoint{-0.021649in}{-0.037276in}}{\pgfqpoint{-0.011050in}{-0.041667in}}{\pgfqpoint{0.000000in}{-0.041667in}}%
\pgfpathclose%
\pgfusepath{stroke,fill}%
}%
\begin{pgfscope}%
\pgfsys@transformshift{10.045867in}{4.281705in}%
\pgfsys@useobject{currentmarker}{}%
\end{pgfscope}%
\end{pgfscope}%
\begin{pgfscope}%
\definecolor{textcolor}{rgb}{0.000000,0.000000,0.000000}%
\pgfsetstrokecolor{textcolor}%
\pgfsetfillcolor{textcolor}%
\pgftext[x=10.295867in,y=4.245247in,left,base]{\color{textcolor}\sffamily\fontsize{10.000000}{12.000000}\selectfont Pix3D}%
\end{pgfscope}%
\begin{pgfscope}%
\pgfsetbuttcap%
\pgfsetroundjoin%
\definecolor{currentfill}{rgb}{1.000000,0.705882,0.509804}%
\pgfsetfillcolor{currentfill}%
\pgfsetlinewidth{1.003750pt}%
\definecolor{currentstroke}{rgb}{1.000000,0.705882,0.509804}%
\pgfsetstrokecolor{currentstroke}%
\pgfsetdash{}{0pt}%
\pgfsys@defobject{currentmarker}{\pgfqpoint{-0.041667in}{-0.041667in}}{\pgfqpoint{0.041667in}{0.041667in}}{%
\pgfpathmoveto{\pgfqpoint{0.000000in}{-0.041667in}}%
\pgfpathcurveto{\pgfqpoint{0.011050in}{-0.041667in}}{\pgfqpoint{0.021649in}{-0.037276in}}{\pgfqpoint{0.029463in}{-0.029463in}}%
\pgfpathcurveto{\pgfqpoint{0.037276in}{-0.021649in}}{\pgfqpoint{0.041667in}{-0.011050in}}{\pgfqpoint{0.041667in}{0.000000in}}%
\pgfpathcurveto{\pgfqpoint{0.041667in}{0.011050in}}{\pgfqpoint{0.037276in}{0.021649in}}{\pgfqpoint{0.029463in}{0.029463in}}%
\pgfpathcurveto{\pgfqpoint{0.021649in}{0.037276in}}{\pgfqpoint{0.011050in}{0.041667in}}{\pgfqpoint{0.000000in}{0.041667in}}%
\pgfpathcurveto{\pgfqpoint{-0.011050in}{0.041667in}}{\pgfqpoint{-0.021649in}{0.037276in}}{\pgfqpoint{-0.029463in}{0.029463in}}%
\pgfpathcurveto{\pgfqpoint{-0.037276in}{0.021649in}}{\pgfqpoint{-0.041667in}{0.011050in}}{\pgfqpoint{-0.041667in}{0.000000in}}%
\pgfpathcurveto{\pgfqpoint{-0.041667in}{-0.011050in}}{\pgfqpoint{-0.037276in}{-0.021649in}}{\pgfqpoint{-0.029463in}{-0.029463in}}%
\pgfpathcurveto{\pgfqpoint{-0.021649in}{-0.037276in}}{\pgfqpoint{-0.011050in}{-0.041667in}}{\pgfqpoint{0.000000in}{-0.041667in}}%
\pgfpathclose%
\pgfusepath{stroke,fill}%
}%
\begin{pgfscope}%
\pgfsys@transformshift{10.045867in}{4.077848in}%
\pgfsys@useobject{currentmarker}{}%
\end{pgfscope}%
\end{pgfscope}%
\begin{pgfscope}%
\definecolor{textcolor}{rgb}{0.000000,0.000000,0.000000}%
\pgfsetstrokecolor{textcolor}%
\pgfsetfillcolor{textcolor}%
\pgftext[x=10.295867in,y=4.041390in,left,base]{\color{textcolor}\sffamily\fontsize{10.000000}{12.000000}\selectfont S2R:3DFREE}%
\end{pgfscope}%
\end{pgfpicture}%
\makeatother%
\endgroup%
}
    \caption[\gls{tsne} for Pix3D and \gls{free}.]{\gls{tsne} visualization for images from Pix3D and \gls{free} dataset.We observe that \gls{free} still doesnt encapsulate the embedding space like Pix3D.}
    \label{fig:pix3d_s2r3dfree}
\end{figure}


We plot the visualization for each dataset individually with the real dataset in \autoref{fig:tsne per dataset}.
From this figure, we can observe that the latent space of the real dataset and \gls{free} are overlapping and not clustered separately.
We can also see that \gls{ai2thor} has the least common latent space, while Blenderproc has the maximum.
Openrooms and SceneNet also have latent space significantly apart from the real dataset, while Hyperism and \gls{front} have a widespread space that intersects the real data.

To summarize, all the proclaimed photorealistic datasets used in \autoref{sec:a-survey-on-photorealism}, have atleast some latent space common with the real dataset.
Hence, we could see a discrepancy in the user survey wherein the participants had a confused perspective of photorealism.

\subsection{Quantitative}\label{subsec:quantitative}

%We visualized the embedding space using \gls{tsne} in above \autoref{subsec:qualitative}.
%All the datasets seem to have a very close relationship with the real dataset, as we see at least some points being in the intersection with the latent space of the real dataset.
%This section compares the datasets with quantitative assessment using \gls{mse} and \gls{fid}.
%As seen in \autoref{tab:quantitative-dataset-comparison}, BlenderProc has the least MSE of 5.53 to Pix3D, while Openrooms have the highest MES of 7.63.
%\gls{free} has a MSE of 7.075, which is below Openrooms and Hyperism.
%Interestingly, \gls{ai2thor} seems to perform better in quantitative assessment of photorealism,
%where both MSE and \gls{fid} are considerably lesser than other datasets,
%establishing that \gls{ai2thor} is closer to the real dataset than what was visualized in \autoref{fig:photorealistic tsne} and \autoref{fig:tsne per dataset}.
%\gls{free} has an \gls{fid} of 178.83, which is lesser than Openrooms, Hyperism, and SceneNet.
%
%\begin{table}[ht]
%    \centering
%    \begin{tabular}{|c |c |c |c|}
%        \hline
%        Dataset & \gls{mse} & \gls{fid} \\ [0.5ex]
%        \hline\hline
%        Openrooms & 7.63 & 189.43 \\
%        \hline
%        \gls{ai2thor} & 6.94 & 164.61 \\
%        \hline
%        BlenderProc & 5.53 & 173.37 \\
%        \hline
%        Hyperism & 7.12 & 186.57 \\
%        \hline
%        \gls{front} & 6.93 & 167.65 \\
%        \hline
%        InteriorNet & 6.5828 & 160.87 \\
%        \hline
%        SceneNet & 6.7553 & 185.49 \\
%        \hline
%        \gls{free} & 7.0750 & 178.83 \\[1ex]
%        \hline
%    \end{tabular}
%    \caption{Table represents quantitative measure to compare synthetic dataset distribution with the real dataset(Pix3D)}
%    \label{tab:quantitative-dataset-comparison}
%\end{table}

We visualized the embedding space using \gls{tsne} in above \autoref{subsec:qualitative}.
All the datasets seem to have a very close relationship with the real dataset, as we see at least some points being in the intersection with the latent space of the real dataset.
This section compares the datasets with quantitative assessment using \gls{fid}.
\gls{fid} is calculated as explained in \autoref{subsec:fr'echet-inception-distance)}.
For any two distributions or datasets to be similar we expect the \gls{fid} to be diminutive.

As seen in \autoref{tab:quantitative-dataset-comparison}, InteriorNet has the least \gls{fid} of 160.87 to Pix3D, while Openrooms have the highest \gls{fid} of 189.43.
Interestingly, \gls{ai2thor} seems to perform better in quantitative assessment of photorealism,
where \gls{fid} are considerably lesser than other datasets,
establishing that \gls{ai2thor} is closer to the real dataset, contrary to what the survey results in \autoref{sec:a-survey-on-photorealism} concluded.
\gls{free} has an \gls{fid} of 178.83, which is lesser than Openrooms, Hyperism, and SceneNet.

\begin{table}[ht]
    \centering
    \begin{tabular}{|c |c |}
        \hline
        Dataset & \gls{fid} \\ [0.5ex]
        \hline\hline
        InteriorNet  & 160.87 \\
        \hline
        \gls{ai2thor} & 164.61 \\
        \hline
        \gls{front}  & 167.65 \\
        \hline
        BlenderProc  & 173.37 \\
        \hline
        \gls{free} & 178.83 \\
        \hline
        SceneNet & 185.49 \\
        \hline
        Hyperism  & 186.57 \\
        \hline
        Openrooms & 189.43 \\[1ex]
        \hline
    \end{tabular}
    \caption[\gls{fid} Comparison for Synthetic Datasets with Real Dataset.]{Table represents quantitative - \gls{fid} measure to compare synthetic dataset distribution with the real dataset(Pix3D).
    InteriorNet has the least \gls{fid} and hence the most similarity to real dataset distribution.}
    \label{tab:quantitative-dataset-comparison}
\end{table}

\subsection{Domain Gap for \gls{free} Ablation Datasets}


\begin{figure}[!ht]
    \centering
    \resizebox{0.49\linewidth}{6cm}{%% Creator: Matplotlib, PGF backend
%%
%% To include the figure in your LaTeX document, write
%%   \input{<filename>.pgf}
%%
%% Make sure the required packages are loaded in your preamble
%%   \usepackage{pgf}
%%
%% Figures using additional raster images can only be included by \input if
%% they are in the same directory as the main LaTeX file. For loading figures
%% from other directories you can use the `import` package
%%   \usepackage{import}
%%
%% and then include the figures with
%%   \import{<path to file>}{<filename>.pgf}
%%
%% Matplotlib used the following preamble
%%   \usepackage{fontspec}
%%   \setmainfont{DejaVuSerif.ttf}[Path=\detokenize{/Users/apple/opt/anaconda3/envs/kaolin/lib/python3.7/site-packages/matplotlib/mpl-data/fonts/ttf/}]
%%   \setsansfont{DejaVuSans.ttf}[Path=\detokenize{/Users/apple/opt/anaconda3/envs/kaolin/lib/python3.7/site-packages/matplotlib/mpl-data/fonts/ttf/}]
%%   \setmonofont{DejaVuSansMono.ttf}[Path=\detokenize{/Users/apple/opt/anaconda3/envs/kaolin/lib/python3.7/site-packages/matplotlib/mpl-data/fonts/ttf/}]
%%
\begingroup%
\makeatletter%
\begin{pgfpicture}%
\pgfpathrectangle{\pgfpointorigin}{\pgfqpoint{11.959465in}{8.341596in}}%
\pgfusepath{use as bounding box, clip}%
\begin{pgfscope}%
\pgfsetbuttcap%
\pgfsetmiterjoin%
\definecolor{currentfill}{rgb}{1.000000,1.000000,1.000000}%
\pgfsetfillcolor{currentfill}%
\pgfsetlinewidth{0.000000pt}%
\definecolor{currentstroke}{rgb}{1.000000,1.000000,1.000000}%
\pgfsetstrokecolor{currentstroke}%
\pgfsetdash{}{0pt}%
\pgfpathmoveto{\pgfqpoint{0.000000in}{0.000000in}}%
\pgfpathlineto{\pgfqpoint{11.959465in}{0.000000in}}%
\pgfpathlineto{\pgfqpoint{11.959465in}{8.341596in}}%
\pgfpathlineto{\pgfqpoint{0.000000in}{8.341596in}}%
\pgfpathclose%
\pgfusepath{fill}%
\end{pgfscope}%
\begin{pgfscope}%
\pgfsetbuttcap%
\pgfsetmiterjoin%
\definecolor{currentfill}{rgb}{1.000000,1.000000,1.000000}%
\pgfsetfillcolor{currentfill}%
\pgfsetlinewidth{0.000000pt}%
\definecolor{currentstroke}{rgb}{0.000000,0.000000,0.000000}%
\pgfsetstrokecolor{currentstroke}%
\pgfsetstrokeopacity{0.000000}%
\pgfsetdash{}{0pt}%
\pgfpathmoveto{\pgfqpoint{0.481978in}{0.331635in}}%
\pgfpathlineto{\pgfqpoint{9.781978in}{0.331635in}}%
\pgfpathlineto{\pgfqpoint{9.781978in}{8.031635in}}%
\pgfpathlineto{\pgfqpoint{0.481978in}{8.031635in}}%
\pgfpathclose%
\pgfusepath{fill}%
\end{pgfscope}%
\begin{pgfscope}%
\pgfpathrectangle{\pgfqpoint{0.481978in}{0.331635in}}{\pgfqpoint{9.300000in}{7.700000in}}%
\pgfusepath{clip}%
\pgfsetbuttcap%
\pgfsetroundjoin%
\definecolor{currentfill}{rgb}{0.631373,0.788235,0.956863}%
\pgfsetfillcolor{currentfill}%
\pgfsetlinewidth{0.481800pt}%
\definecolor{currentstroke}{rgb}{1.000000,1.000000,1.000000}%
\pgfsetstrokecolor{currentstroke}%
\pgfsetdash{}{0pt}%
\pgfpathmoveto{\pgfqpoint{8.052863in}{1.510426in}}%
\pgfpathcurveto{\pgfqpoint{8.063914in}{1.510426in}}{\pgfqpoint{8.074513in}{1.514817in}}{\pgfqpoint{8.082326in}{1.522630in}}%
\pgfpathcurveto{\pgfqpoint{8.090140in}{1.530444in}}{\pgfqpoint{8.094530in}{1.541043in}}{\pgfqpoint{8.094530in}{1.552093in}}%
\pgfpathcurveto{\pgfqpoint{8.094530in}{1.563143in}}{\pgfqpoint{8.090140in}{1.573742in}}{\pgfqpoint{8.082326in}{1.581556in}}%
\pgfpathcurveto{\pgfqpoint{8.074513in}{1.589369in}}{\pgfqpoint{8.063914in}{1.593760in}}{\pgfqpoint{8.052863in}{1.593760in}}%
\pgfpathcurveto{\pgfqpoint{8.041813in}{1.593760in}}{\pgfqpoint{8.031214in}{1.589369in}}{\pgfqpoint{8.023401in}{1.581556in}}%
\pgfpathcurveto{\pgfqpoint{8.015587in}{1.573742in}}{\pgfqpoint{8.011197in}{1.563143in}}{\pgfqpoint{8.011197in}{1.552093in}}%
\pgfpathcurveto{\pgfqpoint{8.011197in}{1.541043in}}{\pgfqpoint{8.015587in}{1.530444in}}{\pgfqpoint{8.023401in}{1.522630in}}%
\pgfpathcurveto{\pgfqpoint{8.031214in}{1.514817in}}{\pgfqpoint{8.041813in}{1.510426in}}{\pgfqpoint{8.052863in}{1.510426in}}%
\pgfpathclose%
\pgfusepath{stroke,fill}%
\end{pgfscope}%
\begin{pgfscope}%
\pgfpathrectangle{\pgfqpoint{0.481978in}{0.331635in}}{\pgfqpoint{9.300000in}{7.700000in}}%
\pgfusepath{clip}%
\pgfsetbuttcap%
\pgfsetroundjoin%
\definecolor{currentfill}{rgb}{0.631373,0.788235,0.956863}%
\pgfsetfillcolor{currentfill}%
\pgfsetlinewidth{0.481800pt}%
\definecolor{currentstroke}{rgb}{1.000000,1.000000,1.000000}%
\pgfsetstrokecolor{currentstroke}%
\pgfsetdash{}{0pt}%
\pgfpathmoveto{\pgfqpoint{6.390490in}{4.831804in}}%
\pgfpathcurveto{\pgfqpoint{6.401540in}{4.831804in}}{\pgfqpoint{6.412139in}{4.836194in}}{\pgfqpoint{6.419952in}{4.844008in}}%
\pgfpathcurveto{\pgfqpoint{6.427766in}{4.851821in}}{\pgfqpoint{6.432156in}{4.862420in}}{\pgfqpoint{6.432156in}{4.873471in}}%
\pgfpathcurveto{\pgfqpoint{6.432156in}{4.884521in}}{\pgfqpoint{6.427766in}{4.895120in}}{\pgfqpoint{6.419952in}{4.902933in}}%
\pgfpathcurveto{\pgfqpoint{6.412139in}{4.910747in}}{\pgfqpoint{6.401540in}{4.915137in}}{\pgfqpoint{6.390490in}{4.915137in}}%
\pgfpathcurveto{\pgfqpoint{6.379439in}{4.915137in}}{\pgfqpoint{6.368840in}{4.910747in}}{\pgfqpoint{6.361027in}{4.902933in}}%
\pgfpathcurveto{\pgfqpoint{6.353213in}{4.895120in}}{\pgfqpoint{6.348823in}{4.884521in}}{\pgfqpoint{6.348823in}{4.873471in}}%
\pgfpathcurveto{\pgfqpoint{6.348823in}{4.862420in}}{\pgfqpoint{6.353213in}{4.851821in}}{\pgfqpoint{6.361027in}{4.844008in}}%
\pgfpathcurveto{\pgfqpoint{6.368840in}{4.836194in}}{\pgfqpoint{6.379439in}{4.831804in}}{\pgfqpoint{6.390490in}{4.831804in}}%
\pgfpathclose%
\pgfusepath{stroke,fill}%
\end{pgfscope}%
\begin{pgfscope}%
\pgfpathrectangle{\pgfqpoint{0.481978in}{0.331635in}}{\pgfqpoint{9.300000in}{7.700000in}}%
\pgfusepath{clip}%
\pgfsetbuttcap%
\pgfsetroundjoin%
\definecolor{currentfill}{rgb}{0.631373,0.788235,0.956863}%
\pgfsetfillcolor{currentfill}%
\pgfsetlinewidth{0.481800pt}%
\definecolor{currentstroke}{rgb}{1.000000,1.000000,1.000000}%
\pgfsetstrokecolor{currentstroke}%
\pgfsetdash{}{0pt}%
\pgfpathmoveto{\pgfqpoint{4.724066in}{3.560981in}}%
\pgfpathcurveto{\pgfqpoint{4.735116in}{3.560981in}}{\pgfqpoint{4.745715in}{3.565371in}}{\pgfqpoint{4.753529in}{3.573185in}}%
\pgfpathcurveto{\pgfqpoint{4.761342in}{3.580998in}}{\pgfqpoint{4.765733in}{3.591597in}}{\pgfqpoint{4.765733in}{3.602648in}}%
\pgfpathcurveto{\pgfqpoint{4.765733in}{3.613698in}}{\pgfqpoint{4.761342in}{3.624297in}}{\pgfqpoint{4.753529in}{3.632110in}}%
\pgfpathcurveto{\pgfqpoint{4.745715in}{3.639924in}}{\pgfqpoint{4.735116in}{3.644314in}}{\pgfqpoint{4.724066in}{3.644314in}}%
\pgfpathcurveto{\pgfqpoint{4.713016in}{3.644314in}}{\pgfqpoint{4.702417in}{3.639924in}}{\pgfqpoint{4.694603in}{3.632110in}}%
\pgfpathcurveto{\pgfqpoint{4.686790in}{3.624297in}}{\pgfqpoint{4.682399in}{3.613698in}}{\pgfqpoint{4.682399in}{3.602648in}}%
\pgfpathcurveto{\pgfqpoint{4.682399in}{3.591597in}}{\pgfqpoint{4.686790in}{3.580998in}}{\pgfqpoint{4.694603in}{3.573185in}}%
\pgfpathcurveto{\pgfqpoint{4.702417in}{3.565371in}}{\pgfqpoint{4.713016in}{3.560981in}}{\pgfqpoint{4.724066in}{3.560981in}}%
\pgfpathclose%
\pgfusepath{stroke,fill}%
\end{pgfscope}%
\begin{pgfscope}%
\pgfpathrectangle{\pgfqpoint{0.481978in}{0.331635in}}{\pgfqpoint{9.300000in}{7.700000in}}%
\pgfusepath{clip}%
\pgfsetbuttcap%
\pgfsetroundjoin%
\definecolor{currentfill}{rgb}{0.631373,0.788235,0.956863}%
\pgfsetfillcolor{currentfill}%
\pgfsetlinewidth{0.481800pt}%
\definecolor{currentstroke}{rgb}{1.000000,1.000000,1.000000}%
\pgfsetstrokecolor{currentstroke}%
\pgfsetdash{}{0pt}%
\pgfpathmoveto{\pgfqpoint{3.262933in}{3.133757in}}%
\pgfpathcurveto{\pgfqpoint{3.273983in}{3.133757in}}{\pgfqpoint{3.284582in}{3.138147in}}{\pgfqpoint{3.292396in}{3.145961in}}%
\pgfpathcurveto{\pgfqpoint{3.300209in}{3.153774in}}{\pgfqpoint{3.304600in}{3.164373in}}{\pgfqpoint{3.304600in}{3.175424in}}%
\pgfpathcurveto{\pgfqpoint{3.304600in}{3.186474in}}{\pgfqpoint{3.300209in}{3.197073in}}{\pgfqpoint{3.292396in}{3.204886in}}%
\pgfpathcurveto{\pgfqpoint{3.284582in}{3.212700in}}{\pgfqpoint{3.273983in}{3.217090in}}{\pgfqpoint{3.262933in}{3.217090in}}%
\pgfpathcurveto{\pgfqpoint{3.251883in}{3.217090in}}{\pgfqpoint{3.241284in}{3.212700in}}{\pgfqpoint{3.233470in}{3.204886in}}%
\pgfpathcurveto{\pgfqpoint{3.225657in}{3.197073in}}{\pgfqpoint{3.221266in}{3.186474in}}{\pgfqpoint{3.221266in}{3.175424in}}%
\pgfpathcurveto{\pgfqpoint{3.221266in}{3.164373in}}{\pgfqpoint{3.225657in}{3.153774in}}{\pgfqpoint{3.233470in}{3.145961in}}%
\pgfpathcurveto{\pgfqpoint{3.241284in}{3.138147in}}{\pgfqpoint{3.251883in}{3.133757in}}{\pgfqpoint{3.262933in}{3.133757in}}%
\pgfpathclose%
\pgfusepath{stroke,fill}%
\end{pgfscope}%
\begin{pgfscope}%
\pgfpathrectangle{\pgfqpoint{0.481978in}{0.331635in}}{\pgfqpoint{9.300000in}{7.700000in}}%
\pgfusepath{clip}%
\pgfsetbuttcap%
\pgfsetroundjoin%
\definecolor{currentfill}{rgb}{0.631373,0.788235,0.956863}%
\pgfsetfillcolor{currentfill}%
\pgfsetlinewidth{0.481800pt}%
\definecolor{currentstroke}{rgb}{1.000000,1.000000,1.000000}%
\pgfsetstrokecolor{currentstroke}%
\pgfsetdash{}{0pt}%
\pgfpathmoveto{\pgfqpoint{8.262238in}{1.069352in}}%
\pgfpathcurveto{\pgfqpoint{8.273288in}{1.069352in}}{\pgfqpoint{8.283887in}{1.073742in}}{\pgfqpoint{8.291700in}{1.081556in}}%
\pgfpathcurveto{\pgfqpoint{8.299514in}{1.089369in}}{\pgfqpoint{8.303904in}{1.099968in}}{\pgfqpoint{8.303904in}{1.111019in}}%
\pgfpathcurveto{\pgfqpoint{8.303904in}{1.122069in}}{\pgfqpoint{8.299514in}{1.132668in}}{\pgfqpoint{8.291700in}{1.140481in}}%
\pgfpathcurveto{\pgfqpoint{8.283887in}{1.148295in}}{\pgfqpoint{8.273288in}{1.152685in}}{\pgfqpoint{8.262238in}{1.152685in}}%
\pgfpathcurveto{\pgfqpoint{8.251188in}{1.152685in}}{\pgfqpoint{8.240589in}{1.148295in}}{\pgfqpoint{8.232775in}{1.140481in}}%
\pgfpathcurveto{\pgfqpoint{8.224961in}{1.132668in}}{\pgfqpoint{8.220571in}{1.122069in}}{\pgfqpoint{8.220571in}{1.111019in}}%
\pgfpathcurveto{\pgfqpoint{8.220571in}{1.099968in}}{\pgfqpoint{8.224961in}{1.089369in}}{\pgfqpoint{8.232775in}{1.081556in}}%
\pgfpathcurveto{\pgfqpoint{8.240589in}{1.073742in}}{\pgfqpoint{8.251188in}{1.069352in}}{\pgfqpoint{8.262238in}{1.069352in}}%
\pgfpathclose%
\pgfusepath{stroke,fill}%
\end{pgfscope}%
\begin{pgfscope}%
\pgfpathrectangle{\pgfqpoint{0.481978in}{0.331635in}}{\pgfqpoint{9.300000in}{7.700000in}}%
\pgfusepath{clip}%
\pgfsetbuttcap%
\pgfsetroundjoin%
\definecolor{currentfill}{rgb}{0.631373,0.788235,0.956863}%
\pgfsetfillcolor{currentfill}%
\pgfsetlinewidth{0.481800pt}%
\definecolor{currentstroke}{rgb}{1.000000,1.000000,1.000000}%
\pgfsetstrokecolor{currentstroke}%
\pgfsetdash{}{0pt}%
\pgfpathmoveto{\pgfqpoint{7.580160in}{5.017168in}}%
\pgfpathcurveto{\pgfqpoint{7.591210in}{5.017168in}}{\pgfqpoint{7.601809in}{5.021558in}}{\pgfqpoint{7.609622in}{5.029372in}}%
\pgfpathcurveto{\pgfqpoint{7.617436in}{5.037185in}}{\pgfqpoint{7.621826in}{5.047784in}}{\pgfqpoint{7.621826in}{5.058834in}}%
\pgfpathcurveto{\pgfqpoint{7.621826in}{5.069885in}}{\pgfqpoint{7.617436in}{5.080484in}}{\pgfqpoint{7.609622in}{5.088297in}}%
\pgfpathcurveto{\pgfqpoint{7.601809in}{5.096111in}}{\pgfqpoint{7.591210in}{5.100501in}}{\pgfqpoint{7.580160in}{5.100501in}}%
\pgfpathcurveto{\pgfqpoint{7.569110in}{5.100501in}}{\pgfqpoint{7.558511in}{5.096111in}}{\pgfqpoint{7.550697in}{5.088297in}}%
\pgfpathcurveto{\pgfqpoint{7.542883in}{5.080484in}}{\pgfqpoint{7.538493in}{5.069885in}}{\pgfqpoint{7.538493in}{5.058834in}}%
\pgfpathcurveto{\pgfqpoint{7.538493in}{5.047784in}}{\pgfqpoint{7.542883in}{5.037185in}}{\pgfqpoint{7.550697in}{5.029372in}}%
\pgfpathcurveto{\pgfqpoint{7.558511in}{5.021558in}}{\pgfqpoint{7.569110in}{5.017168in}}{\pgfqpoint{7.580160in}{5.017168in}}%
\pgfpathclose%
\pgfusepath{stroke,fill}%
\end{pgfscope}%
\begin{pgfscope}%
\pgfpathrectangle{\pgfqpoint{0.481978in}{0.331635in}}{\pgfqpoint{9.300000in}{7.700000in}}%
\pgfusepath{clip}%
\pgfsetbuttcap%
\pgfsetroundjoin%
\definecolor{currentfill}{rgb}{0.631373,0.788235,0.956863}%
\pgfsetfillcolor{currentfill}%
\pgfsetlinewidth{0.481800pt}%
\definecolor{currentstroke}{rgb}{1.000000,1.000000,1.000000}%
\pgfsetstrokecolor{currentstroke}%
\pgfsetdash{}{0pt}%
\pgfpathmoveto{\pgfqpoint{7.496625in}{2.019733in}}%
\pgfpathcurveto{\pgfqpoint{7.507675in}{2.019733in}}{\pgfqpoint{7.518274in}{2.024123in}}{\pgfqpoint{7.526088in}{2.031937in}}%
\pgfpathcurveto{\pgfqpoint{7.533901in}{2.039751in}}{\pgfqpoint{7.538292in}{2.050350in}}{\pgfqpoint{7.538292in}{2.061400in}}%
\pgfpathcurveto{\pgfqpoint{7.538292in}{2.072450in}}{\pgfqpoint{7.533901in}{2.083049in}}{\pgfqpoint{7.526088in}{2.090862in}}%
\pgfpathcurveto{\pgfqpoint{7.518274in}{2.098676in}}{\pgfqpoint{7.507675in}{2.103066in}}{\pgfqpoint{7.496625in}{2.103066in}}%
\pgfpathcurveto{\pgfqpoint{7.485575in}{2.103066in}}{\pgfqpoint{7.474976in}{2.098676in}}{\pgfqpoint{7.467162in}{2.090862in}}%
\pgfpathcurveto{\pgfqpoint{7.459349in}{2.083049in}}{\pgfqpoint{7.454958in}{2.072450in}}{\pgfqpoint{7.454958in}{2.061400in}}%
\pgfpathcurveto{\pgfqpoint{7.454958in}{2.050350in}}{\pgfqpoint{7.459349in}{2.039751in}}{\pgfqpoint{7.467162in}{2.031937in}}%
\pgfpathcurveto{\pgfqpoint{7.474976in}{2.024123in}}{\pgfqpoint{7.485575in}{2.019733in}}{\pgfqpoint{7.496625in}{2.019733in}}%
\pgfpathclose%
\pgfusepath{stroke,fill}%
\end{pgfscope}%
\begin{pgfscope}%
\pgfpathrectangle{\pgfqpoint{0.481978in}{0.331635in}}{\pgfqpoint{9.300000in}{7.700000in}}%
\pgfusepath{clip}%
\pgfsetbuttcap%
\pgfsetroundjoin%
\definecolor{currentfill}{rgb}{0.631373,0.788235,0.956863}%
\pgfsetfillcolor{currentfill}%
\pgfsetlinewidth{0.481800pt}%
\definecolor{currentstroke}{rgb}{1.000000,1.000000,1.000000}%
\pgfsetstrokecolor{currentstroke}%
\pgfsetdash{}{0pt}%
\pgfpathmoveto{\pgfqpoint{3.509030in}{2.716762in}}%
\pgfpathcurveto{\pgfqpoint{3.520080in}{2.716762in}}{\pgfqpoint{3.530679in}{2.721152in}}{\pgfqpoint{3.538493in}{2.728966in}}%
\pgfpathcurveto{\pgfqpoint{3.546307in}{2.736780in}}{\pgfqpoint{3.550697in}{2.747379in}}{\pgfqpoint{3.550697in}{2.758429in}}%
\pgfpathcurveto{\pgfqpoint{3.550697in}{2.769479in}}{\pgfqpoint{3.546307in}{2.780078in}}{\pgfqpoint{3.538493in}{2.787892in}}%
\pgfpathcurveto{\pgfqpoint{3.530679in}{2.795705in}}{\pgfqpoint{3.520080in}{2.800095in}}{\pgfqpoint{3.509030in}{2.800095in}}%
\pgfpathcurveto{\pgfqpoint{3.497980in}{2.800095in}}{\pgfqpoint{3.487381in}{2.795705in}}{\pgfqpoint{3.479567in}{2.787892in}}%
\pgfpathcurveto{\pgfqpoint{3.471754in}{2.780078in}}{\pgfqpoint{3.467364in}{2.769479in}}{\pgfqpoint{3.467364in}{2.758429in}}%
\pgfpathcurveto{\pgfqpoint{3.467364in}{2.747379in}}{\pgfqpoint{3.471754in}{2.736780in}}{\pgfqpoint{3.479567in}{2.728966in}}%
\pgfpathcurveto{\pgfqpoint{3.487381in}{2.721152in}}{\pgfqpoint{3.497980in}{2.716762in}}{\pgfqpoint{3.509030in}{2.716762in}}%
\pgfpathclose%
\pgfusepath{stroke,fill}%
\end{pgfscope}%
\begin{pgfscope}%
\pgfpathrectangle{\pgfqpoint{0.481978in}{0.331635in}}{\pgfqpoint{9.300000in}{7.700000in}}%
\pgfusepath{clip}%
\pgfsetbuttcap%
\pgfsetroundjoin%
\definecolor{currentfill}{rgb}{0.631373,0.788235,0.956863}%
\pgfsetfillcolor{currentfill}%
\pgfsetlinewidth{0.481800pt}%
\definecolor{currentstroke}{rgb}{1.000000,1.000000,1.000000}%
\pgfsetstrokecolor{currentstroke}%
\pgfsetdash{}{0pt}%
\pgfpathmoveto{\pgfqpoint{6.905988in}{4.369888in}}%
\pgfpathcurveto{\pgfqpoint{6.917038in}{4.369888in}}{\pgfqpoint{6.927637in}{4.374278in}}{\pgfqpoint{6.935451in}{4.382092in}}%
\pgfpathcurveto{\pgfqpoint{6.943265in}{4.389906in}}{\pgfqpoint{6.947655in}{4.400505in}}{\pgfqpoint{6.947655in}{4.411555in}}%
\pgfpathcurveto{\pgfqpoint{6.947655in}{4.422605in}}{\pgfqpoint{6.943265in}{4.433204in}}{\pgfqpoint{6.935451in}{4.441018in}}%
\pgfpathcurveto{\pgfqpoint{6.927637in}{4.448831in}}{\pgfqpoint{6.917038in}{4.453221in}}{\pgfqpoint{6.905988in}{4.453221in}}%
\pgfpathcurveto{\pgfqpoint{6.894938in}{4.453221in}}{\pgfqpoint{6.884339in}{4.448831in}}{\pgfqpoint{6.876525in}{4.441018in}}%
\pgfpathcurveto{\pgfqpoint{6.868712in}{4.433204in}}{\pgfqpoint{6.864321in}{4.422605in}}{\pgfqpoint{6.864321in}{4.411555in}}%
\pgfpathcurveto{\pgfqpoint{6.864321in}{4.400505in}}{\pgfqpoint{6.868712in}{4.389906in}}{\pgfqpoint{6.876525in}{4.382092in}}%
\pgfpathcurveto{\pgfqpoint{6.884339in}{4.374278in}}{\pgfqpoint{6.894938in}{4.369888in}}{\pgfqpoint{6.905988in}{4.369888in}}%
\pgfpathclose%
\pgfusepath{stroke,fill}%
\end{pgfscope}%
\begin{pgfscope}%
\pgfpathrectangle{\pgfqpoint{0.481978in}{0.331635in}}{\pgfqpoint{9.300000in}{7.700000in}}%
\pgfusepath{clip}%
\pgfsetbuttcap%
\pgfsetroundjoin%
\definecolor{currentfill}{rgb}{0.631373,0.788235,0.956863}%
\pgfsetfillcolor{currentfill}%
\pgfsetlinewidth{0.481800pt}%
\definecolor{currentstroke}{rgb}{1.000000,1.000000,1.000000}%
\pgfsetstrokecolor{currentstroke}%
\pgfsetdash{}{0pt}%
\pgfpathmoveto{\pgfqpoint{2.394843in}{4.902575in}}%
\pgfpathcurveto{\pgfqpoint{2.405893in}{4.902575in}}{\pgfqpoint{2.416492in}{4.906966in}}{\pgfqpoint{2.424306in}{4.914779in}}%
\pgfpathcurveto{\pgfqpoint{2.432120in}{4.922593in}}{\pgfqpoint{2.436510in}{4.933192in}}{\pgfqpoint{2.436510in}{4.944242in}}%
\pgfpathcurveto{\pgfqpoint{2.436510in}{4.955292in}}{\pgfqpoint{2.432120in}{4.965891in}}{\pgfqpoint{2.424306in}{4.973705in}}%
\pgfpathcurveto{\pgfqpoint{2.416492in}{4.981518in}}{\pgfqpoint{2.405893in}{4.985909in}}{\pgfqpoint{2.394843in}{4.985909in}}%
\pgfpathcurveto{\pgfqpoint{2.383793in}{4.985909in}}{\pgfqpoint{2.373194in}{4.981518in}}{\pgfqpoint{2.365380in}{4.973705in}}%
\pgfpathcurveto{\pgfqpoint{2.357567in}{4.965891in}}{\pgfqpoint{2.353176in}{4.955292in}}{\pgfqpoint{2.353176in}{4.944242in}}%
\pgfpathcurveto{\pgfqpoint{2.353176in}{4.933192in}}{\pgfqpoint{2.357567in}{4.922593in}}{\pgfqpoint{2.365380in}{4.914779in}}%
\pgfpathcurveto{\pgfqpoint{2.373194in}{4.906966in}}{\pgfqpoint{2.383793in}{4.902575in}}{\pgfqpoint{2.394843in}{4.902575in}}%
\pgfpathclose%
\pgfusepath{stroke,fill}%
\end{pgfscope}%
\begin{pgfscope}%
\pgfpathrectangle{\pgfqpoint{0.481978in}{0.331635in}}{\pgfqpoint{9.300000in}{7.700000in}}%
\pgfusepath{clip}%
\pgfsetbuttcap%
\pgfsetroundjoin%
\definecolor{currentfill}{rgb}{0.631373,0.788235,0.956863}%
\pgfsetfillcolor{currentfill}%
\pgfsetlinewidth{0.481800pt}%
\definecolor{currentstroke}{rgb}{1.000000,1.000000,1.000000}%
\pgfsetstrokecolor{currentstroke}%
\pgfsetdash{}{0pt}%
\pgfpathmoveto{\pgfqpoint{4.587397in}{5.212299in}}%
\pgfpathcurveto{\pgfqpoint{4.598447in}{5.212299in}}{\pgfqpoint{4.609046in}{5.216689in}}{\pgfqpoint{4.616859in}{5.224503in}}%
\pgfpathcurveto{\pgfqpoint{4.624673in}{5.232316in}}{\pgfqpoint{4.629063in}{5.242915in}}{\pgfqpoint{4.629063in}{5.253965in}}%
\pgfpathcurveto{\pgfqpoint{4.629063in}{5.265016in}}{\pgfqpoint{4.624673in}{5.275615in}}{\pgfqpoint{4.616859in}{5.283428in}}%
\pgfpathcurveto{\pgfqpoint{4.609046in}{5.291242in}}{\pgfqpoint{4.598447in}{5.295632in}}{\pgfqpoint{4.587397in}{5.295632in}}%
\pgfpathcurveto{\pgfqpoint{4.576346in}{5.295632in}}{\pgfqpoint{4.565747in}{5.291242in}}{\pgfqpoint{4.557934in}{5.283428in}}%
\pgfpathcurveto{\pgfqpoint{4.550120in}{5.275615in}}{\pgfqpoint{4.545730in}{5.265016in}}{\pgfqpoint{4.545730in}{5.253965in}}%
\pgfpathcurveto{\pgfqpoint{4.545730in}{5.242915in}}{\pgfqpoint{4.550120in}{5.232316in}}{\pgfqpoint{4.557934in}{5.224503in}}%
\pgfpathcurveto{\pgfqpoint{4.565747in}{5.216689in}}{\pgfqpoint{4.576346in}{5.212299in}}{\pgfqpoint{4.587397in}{5.212299in}}%
\pgfpathclose%
\pgfusepath{stroke,fill}%
\end{pgfscope}%
\begin{pgfscope}%
\pgfpathrectangle{\pgfqpoint{0.481978in}{0.331635in}}{\pgfqpoint{9.300000in}{7.700000in}}%
\pgfusepath{clip}%
\pgfsetbuttcap%
\pgfsetroundjoin%
\definecolor{currentfill}{rgb}{0.631373,0.788235,0.956863}%
\pgfsetfillcolor{currentfill}%
\pgfsetlinewidth{0.481800pt}%
\definecolor{currentstroke}{rgb}{1.000000,1.000000,1.000000}%
\pgfsetstrokecolor{currentstroke}%
\pgfsetdash{}{0pt}%
\pgfpathmoveto{\pgfqpoint{5.606626in}{4.828749in}}%
\pgfpathcurveto{\pgfqpoint{5.617676in}{4.828749in}}{\pgfqpoint{5.628275in}{4.833140in}}{\pgfqpoint{5.636089in}{4.840953in}}%
\pgfpathcurveto{\pgfqpoint{5.643902in}{4.848767in}}{\pgfqpoint{5.648293in}{4.859366in}}{\pgfqpoint{5.648293in}{4.870416in}}%
\pgfpathcurveto{\pgfqpoint{5.648293in}{4.881466in}}{\pgfqpoint{5.643902in}{4.892065in}}{\pgfqpoint{5.636089in}{4.899879in}}%
\pgfpathcurveto{\pgfqpoint{5.628275in}{4.907693in}}{\pgfqpoint{5.617676in}{4.912083in}}{\pgfqpoint{5.606626in}{4.912083in}}%
\pgfpathcurveto{\pgfqpoint{5.595576in}{4.912083in}}{\pgfqpoint{5.584977in}{4.907693in}}{\pgfqpoint{5.577163in}{4.899879in}}%
\pgfpathcurveto{\pgfqpoint{5.569349in}{4.892065in}}{\pgfqpoint{5.564959in}{4.881466in}}{\pgfqpoint{5.564959in}{4.870416in}}%
\pgfpathcurveto{\pgfqpoint{5.564959in}{4.859366in}}{\pgfqpoint{5.569349in}{4.848767in}}{\pgfqpoint{5.577163in}{4.840953in}}%
\pgfpathcurveto{\pgfqpoint{5.584977in}{4.833140in}}{\pgfqpoint{5.595576in}{4.828749in}}{\pgfqpoint{5.606626in}{4.828749in}}%
\pgfpathclose%
\pgfusepath{stroke,fill}%
\end{pgfscope}%
\begin{pgfscope}%
\pgfpathrectangle{\pgfqpoint{0.481978in}{0.331635in}}{\pgfqpoint{9.300000in}{7.700000in}}%
\pgfusepath{clip}%
\pgfsetbuttcap%
\pgfsetroundjoin%
\definecolor{currentfill}{rgb}{0.631373,0.788235,0.956863}%
\pgfsetfillcolor{currentfill}%
\pgfsetlinewidth{0.481800pt}%
\definecolor{currentstroke}{rgb}{1.000000,1.000000,1.000000}%
\pgfsetstrokecolor{currentstroke}%
\pgfsetdash{}{0pt}%
\pgfpathmoveto{\pgfqpoint{5.093085in}{5.604051in}}%
\pgfpathcurveto{\pgfqpoint{5.104136in}{5.604051in}}{\pgfqpoint{5.114735in}{5.608441in}}{\pgfqpoint{5.122548in}{5.616255in}}%
\pgfpathcurveto{\pgfqpoint{5.130362in}{5.624068in}}{\pgfqpoint{5.134752in}{5.634667in}}{\pgfqpoint{5.134752in}{5.645718in}}%
\pgfpathcurveto{\pgfqpoint{5.134752in}{5.656768in}}{\pgfqpoint{5.130362in}{5.667367in}}{\pgfqpoint{5.122548in}{5.675180in}}%
\pgfpathcurveto{\pgfqpoint{5.114735in}{5.682994in}}{\pgfqpoint{5.104136in}{5.687384in}}{\pgfqpoint{5.093085in}{5.687384in}}%
\pgfpathcurveto{\pgfqpoint{5.082035in}{5.687384in}}{\pgfqpoint{5.071436in}{5.682994in}}{\pgfqpoint{5.063623in}{5.675180in}}%
\pgfpathcurveto{\pgfqpoint{5.055809in}{5.667367in}}{\pgfqpoint{5.051419in}{5.656768in}}{\pgfqpoint{5.051419in}{5.645718in}}%
\pgfpathcurveto{\pgfqpoint{5.051419in}{5.634667in}}{\pgfqpoint{5.055809in}{5.624068in}}{\pgfqpoint{5.063623in}{5.616255in}}%
\pgfpathcurveto{\pgfqpoint{5.071436in}{5.608441in}}{\pgfqpoint{5.082035in}{5.604051in}}{\pgfqpoint{5.093085in}{5.604051in}}%
\pgfpathclose%
\pgfusepath{stroke,fill}%
\end{pgfscope}%
\begin{pgfscope}%
\pgfpathrectangle{\pgfqpoint{0.481978in}{0.331635in}}{\pgfqpoint{9.300000in}{7.700000in}}%
\pgfusepath{clip}%
\pgfsetbuttcap%
\pgfsetroundjoin%
\definecolor{currentfill}{rgb}{0.631373,0.788235,0.956863}%
\pgfsetfillcolor{currentfill}%
\pgfsetlinewidth{0.481800pt}%
\definecolor{currentstroke}{rgb}{1.000000,1.000000,1.000000}%
\pgfsetstrokecolor{currentstroke}%
\pgfsetdash{}{0pt}%
\pgfpathmoveto{\pgfqpoint{7.877868in}{4.769526in}}%
\pgfpathcurveto{\pgfqpoint{7.888918in}{4.769526in}}{\pgfqpoint{7.899517in}{4.773916in}}{\pgfqpoint{7.907331in}{4.781730in}}%
\pgfpathcurveto{\pgfqpoint{7.915144in}{4.789543in}}{\pgfqpoint{7.919535in}{4.800143in}}{\pgfqpoint{7.919535in}{4.811193in}}%
\pgfpathcurveto{\pgfqpoint{7.919535in}{4.822243in}}{\pgfqpoint{7.915144in}{4.832842in}}{\pgfqpoint{7.907331in}{4.840655in}}%
\pgfpathcurveto{\pgfqpoint{7.899517in}{4.848469in}}{\pgfqpoint{7.888918in}{4.852859in}}{\pgfqpoint{7.877868in}{4.852859in}}%
\pgfpathcurveto{\pgfqpoint{7.866818in}{4.852859in}}{\pgfqpoint{7.856219in}{4.848469in}}{\pgfqpoint{7.848405in}{4.840655in}}%
\pgfpathcurveto{\pgfqpoint{7.840591in}{4.832842in}}{\pgfqpoint{7.836201in}{4.822243in}}{\pgfqpoint{7.836201in}{4.811193in}}%
\pgfpathcurveto{\pgfqpoint{7.836201in}{4.800143in}}{\pgfqpoint{7.840591in}{4.789543in}}{\pgfqpoint{7.848405in}{4.781730in}}%
\pgfpathcurveto{\pgfqpoint{7.856219in}{4.773916in}}{\pgfqpoint{7.866818in}{4.769526in}}{\pgfqpoint{7.877868in}{4.769526in}}%
\pgfpathclose%
\pgfusepath{stroke,fill}%
\end{pgfscope}%
\begin{pgfscope}%
\pgfpathrectangle{\pgfqpoint{0.481978in}{0.331635in}}{\pgfqpoint{9.300000in}{7.700000in}}%
\pgfusepath{clip}%
\pgfsetbuttcap%
\pgfsetroundjoin%
\definecolor{currentfill}{rgb}{0.631373,0.788235,0.956863}%
\pgfsetfillcolor{currentfill}%
\pgfsetlinewidth{0.481800pt}%
\definecolor{currentstroke}{rgb}{1.000000,1.000000,1.000000}%
\pgfsetstrokecolor{currentstroke}%
\pgfsetdash{}{0pt}%
\pgfpathmoveto{\pgfqpoint{2.375730in}{2.359798in}}%
\pgfpathcurveto{\pgfqpoint{2.386780in}{2.359798in}}{\pgfqpoint{2.397379in}{2.364188in}}{\pgfqpoint{2.405192in}{2.372002in}}%
\pgfpathcurveto{\pgfqpoint{2.413006in}{2.379815in}}{\pgfqpoint{2.417396in}{2.390414in}}{\pgfqpoint{2.417396in}{2.401464in}}%
\pgfpathcurveto{\pgfqpoint{2.417396in}{2.412514in}}{\pgfqpoint{2.413006in}{2.423114in}}{\pgfqpoint{2.405192in}{2.430927in}}%
\pgfpathcurveto{\pgfqpoint{2.397379in}{2.438741in}}{\pgfqpoint{2.386780in}{2.443131in}}{\pgfqpoint{2.375730in}{2.443131in}}%
\pgfpathcurveto{\pgfqpoint{2.364680in}{2.443131in}}{\pgfqpoint{2.354081in}{2.438741in}}{\pgfqpoint{2.346267in}{2.430927in}}%
\pgfpathcurveto{\pgfqpoint{2.338453in}{2.423114in}}{\pgfqpoint{2.334063in}{2.412514in}}{\pgfqpoint{2.334063in}{2.401464in}}%
\pgfpathcurveto{\pgfqpoint{2.334063in}{2.390414in}}{\pgfqpoint{2.338453in}{2.379815in}}{\pgfqpoint{2.346267in}{2.372002in}}%
\pgfpathcurveto{\pgfqpoint{2.354081in}{2.364188in}}{\pgfqpoint{2.364680in}{2.359798in}}{\pgfqpoint{2.375730in}{2.359798in}}%
\pgfpathclose%
\pgfusepath{stroke,fill}%
\end{pgfscope}%
\begin{pgfscope}%
\pgfpathrectangle{\pgfqpoint{0.481978in}{0.331635in}}{\pgfqpoint{9.300000in}{7.700000in}}%
\pgfusepath{clip}%
\pgfsetbuttcap%
\pgfsetroundjoin%
\definecolor{currentfill}{rgb}{0.631373,0.788235,0.956863}%
\pgfsetfillcolor{currentfill}%
\pgfsetlinewidth{0.481800pt}%
\definecolor{currentstroke}{rgb}{1.000000,1.000000,1.000000}%
\pgfsetstrokecolor{currentstroke}%
\pgfsetdash{}{0pt}%
\pgfpathmoveto{\pgfqpoint{9.359251in}{4.500176in}}%
\pgfpathcurveto{\pgfqpoint{9.370301in}{4.500176in}}{\pgfqpoint{9.380900in}{4.504566in}}{\pgfqpoint{9.388713in}{4.512380in}}%
\pgfpathcurveto{\pgfqpoint{9.396527in}{4.520193in}}{\pgfqpoint{9.400917in}{4.530792in}}{\pgfqpoint{9.400917in}{4.541842in}}%
\pgfpathcurveto{\pgfqpoint{9.400917in}{4.552893in}}{\pgfqpoint{9.396527in}{4.563492in}}{\pgfqpoint{9.388713in}{4.571305in}}%
\pgfpathcurveto{\pgfqpoint{9.380900in}{4.579119in}}{\pgfqpoint{9.370301in}{4.583509in}}{\pgfqpoint{9.359251in}{4.583509in}}%
\pgfpathcurveto{\pgfqpoint{9.348201in}{4.583509in}}{\pgfqpoint{9.337601in}{4.579119in}}{\pgfqpoint{9.329788in}{4.571305in}}%
\pgfpathcurveto{\pgfqpoint{9.321974in}{4.563492in}}{\pgfqpoint{9.317584in}{4.552893in}}{\pgfqpoint{9.317584in}{4.541842in}}%
\pgfpathcurveto{\pgfqpoint{9.317584in}{4.530792in}}{\pgfqpoint{9.321974in}{4.520193in}}{\pgfqpoint{9.329788in}{4.512380in}}%
\pgfpathcurveto{\pgfqpoint{9.337601in}{4.504566in}}{\pgfqpoint{9.348201in}{4.500176in}}{\pgfqpoint{9.359251in}{4.500176in}}%
\pgfpathclose%
\pgfusepath{stroke,fill}%
\end{pgfscope}%
\begin{pgfscope}%
\pgfpathrectangle{\pgfqpoint{0.481978in}{0.331635in}}{\pgfqpoint{9.300000in}{7.700000in}}%
\pgfusepath{clip}%
\pgfsetbuttcap%
\pgfsetroundjoin%
\definecolor{currentfill}{rgb}{0.631373,0.788235,0.956863}%
\pgfsetfillcolor{currentfill}%
\pgfsetlinewidth{0.481800pt}%
\definecolor{currentstroke}{rgb}{1.000000,1.000000,1.000000}%
\pgfsetstrokecolor{currentstroke}%
\pgfsetdash{}{0pt}%
\pgfpathmoveto{\pgfqpoint{6.013955in}{5.344683in}}%
\pgfpathcurveto{\pgfqpoint{6.025005in}{5.344683in}}{\pgfqpoint{6.035604in}{5.349074in}}{\pgfqpoint{6.043417in}{5.356887in}}%
\pgfpathcurveto{\pgfqpoint{6.051231in}{5.364701in}}{\pgfqpoint{6.055621in}{5.375300in}}{\pgfqpoint{6.055621in}{5.386350in}}%
\pgfpathcurveto{\pgfqpoint{6.055621in}{5.397400in}}{\pgfqpoint{6.051231in}{5.407999in}}{\pgfqpoint{6.043417in}{5.415813in}}%
\pgfpathcurveto{\pgfqpoint{6.035604in}{5.423626in}}{\pgfqpoint{6.025005in}{5.428017in}}{\pgfqpoint{6.013955in}{5.428017in}}%
\pgfpathcurveto{\pgfqpoint{6.002904in}{5.428017in}}{\pgfqpoint{5.992305in}{5.423626in}}{\pgfqpoint{5.984492in}{5.415813in}}%
\pgfpathcurveto{\pgfqpoint{5.976678in}{5.407999in}}{\pgfqpoint{5.972288in}{5.397400in}}{\pgfqpoint{5.972288in}{5.386350in}}%
\pgfpathcurveto{\pgfqpoint{5.972288in}{5.375300in}}{\pgfqpoint{5.976678in}{5.364701in}}{\pgfqpoint{5.984492in}{5.356887in}}%
\pgfpathcurveto{\pgfqpoint{5.992305in}{5.349074in}}{\pgfqpoint{6.002904in}{5.344683in}}{\pgfqpoint{6.013955in}{5.344683in}}%
\pgfpathclose%
\pgfusepath{stroke,fill}%
\end{pgfscope}%
\begin{pgfscope}%
\pgfpathrectangle{\pgfqpoint{0.481978in}{0.331635in}}{\pgfqpoint{9.300000in}{7.700000in}}%
\pgfusepath{clip}%
\pgfsetbuttcap%
\pgfsetroundjoin%
\definecolor{currentfill}{rgb}{0.631373,0.788235,0.956863}%
\pgfsetfillcolor{currentfill}%
\pgfsetlinewidth{0.481800pt}%
\definecolor{currentstroke}{rgb}{1.000000,1.000000,1.000000}%
\pgfsetstrokecolor{currentstroke}%
\pgfsetdash{}{0pt}%
\pgfpathmoveto{\pgfqpoint{3.830222in}{6.778917in}}%
\pgfpathcurveto{\pgfqpoint{3.841272in}{6.778917in}}{\pgfqpoint{3.851871in}{6.783308in}}{\pgfqpoint{3.859685in}{6.791121in}}%
\pgfpathcurveto{\pgfqpoint{3.867499in}{6.798935in}}{\pgfqpoint{3.871889in}{6.809534in}}{\pgfqpoint{3.871889in}{6.820584in}}%
\pgfpathcurveto{\pgfqpoint{3.871889in}{6.831634in}}{\pgfqpoint{3.867499in}{6.842233in}}{\pgfqpoint{3.859685in}{6.850047in}}%
\pgfpathcurveto{\pgfqpoint{3.851871in}{6.857860in}}{\pgfqpoint{3.841272in}{6.862251in}}{\pgfqpoint{3.830222in}{6.862251in}}%
\pgfpathcurveto{\pgfqpoint{3.819172in}{6.862251in}}{\pgfqpoint{3.808573in}{6.857860in}}{\pgfqpoint{3.800759in}{6.850047in}}%
\pgfpathcurveto{\pgfqpoint{3.792946in}{6.842233in}}{\pgfqpoint{3.788556in}{6.831634in}}{\pgfqpoint{3.788556in}{6.820584in}}%
\pgfpathcurveto{\pgfqpoint{3.788556in}{6.809534in}}{\pgfqpoint{3.792946in}{6.798935in}}{\pgfqpoint{3.800759in}{6.791121in}}%
\pgfpathcurveto{\pgfqpoint{3.808573in}{6.783308in}}{\pgfqpoint{3.819172in}{6.778917in}}{\pgfqpoint{3.830222in}{6.778917in}}%
\pgfpathclose%
\pgfusepath{stroke,fill}%
\end{pgfscope}%
\begin{pgfscope}%
\pgfpathrectangle{\pgfqpoint{0.481978in}{0.331635in}}{\pgfqpoint{9.300000in}{7.700000in}}%
\pgfusepath{clip}%
\pgfsetbuttcap%
\pgfsetroundjoin%
\definecolor{currentfill}{rgb}{0.631373,0.788235,0.956863}%
\pgfsetfillcolor{currentfill}%
\pgfsetlinewidth{0.481800pt}%
\definecolor{currentstroke}{rgb}{1.000000,1.000000,1.000000}%
\pgfsetstrokecolor{currentstroke}%
\pgfsetdash{}{0pt}%
\pgfpathmoveto{\pgfqpoint{8.786314in}{1.252936in}}%
\pgfpathcurveto{\pgfqpoint{8.797364in}{1.252936in}}{\pgfqpoint{8.807963in}{1.257326in}}{\pgfqpoint{8.815777in}{1.265140in}}%
\pgfpathcurveto{\pgfqpoint{8.823591in}{1.272953in}}{\pgfqpoint{8.827981in}{1.283552in}}{\pgfqpoint{8.827981in}{1.294603in}}%
\pgfpathcurveto{\pgfqpoint{8.827981in}{1.305653in}}{\pgfqpoint{8.823591in}{1.316252in}}{\pgfqpoint{8.815777in}{1.324065in}}%
\pgfpathcurveto{\pgfqpoint{8.807963in}{1.331879in}}{\pgfqpoint{8.797364in}{1.336269in}}{\pgfqpoint{8.786314in}{1.336269in}}%
\pgfpathcurveto{\pgfqpoint{8.775264in}{1.336269in}}{\pgfqpoint{8.764665in}{1.331879in}}{\pgfqpoint{8.756851in}{1.324065in}}%
\pgfpathcurveto{\pgfqpoint{8.749038in}{1.316252in}}{\pgfqpoint{8.744648in}{1.305653in}}{\pgfqpoint{8.744648in}{1.294603in}}%
\pgfpathcurveto{\pgfqpoint{8.744648in}{1.283552in}}{\pgfqpoint{8.749038in}{1.272953in}}{\pgfqpoint{8.756851in}{1.265140in}}%
\pgfpathcurveto{\pgfqpoint{8.764665in}{1.257326in}}{\pgfqpoint{8.775264in}{1.252936in}}{\pgfqpoint{8.786314in}{1.252936in}}%
\pgfpathclose%
\pgfusepath{stroke,fill}%
\end{pgfscope}%
\begin{pgfscope}%
\pgfpathrectangle{\pgfqpoint{0.481978in}{0.331635in}}{\pgfqpoint{9.300000in}{7.700000in}}%
\pgfusepath{clip}%
\pgfsetbuttcap%
\pgfsetroundjoin%
\definecolor{currentfill}{rgb}{0.631373,0.788235,0.956863}%
\pgfsetfillcolor{currentfill}%
\pgfsetlinewidth{0.481800pt}%
\definecolor{currentstroke}{rgb}{1.000000,1.000000,1.000000}%
\pgfsetstrokecolor{currentstroke}%
\pgfsetdash{}{0pt}%
\pgfpathmoveto{\pgfqpoint{1.235540in}{2.404404in}}%
\pgfpathcurveto{\pgfqpoint{1.246590in}{2.404404in}}{\pgfqpoint{1.257190in}{2.408794in}}{\pgfqpoint{1.265003in}{2.416608in}}%
\pgfpathcurveto{\pgfqpoint{1.272817in}{2.424422in}}{\pgfqpoint{1.277207in}{2.435021in}}{\pgfqpoint{1.277207in}{2.446071in}}%
\pgfpathcurveto{\pgfqpoint{1.277207in}{2.457121in}}{\pgfqpoint{1.272817in}{2.467720in}}{\pgfqpoint{1.265003in}{2.475534in}}%
\pgfpathcurveto{\pgfqpoint{1.257190in}{2.483347in}}{\pgfqpoint{1.246590in}{2.487737in}}{\pgfqpoint{1.235540in}{2.487737in}}%
\pgfpathcurveto{\pgfqpoint{1.224490in}{2.487737in}}{\pgfqpoint{1.213891in}{2.483347in}}{\pgfqpoint{1.206078in}{2.475534in}}%
\pgfpathcurveto{\pgfqpoint{1.198264in}{2.467720in}}{\pgfqpoint{1.193874in}{2.457121in}}{\pgfqpoint{1.193874in}{2.446071in}}%
\pgfpathcurveto{\pgfqpoint{1.193874in}{2.435021in}}{\pgfqpoint{1.198264in}{2.424422in}}{\pgfqpoint{1.206078in}{2.416608in}}%
\pgfpathcurveto{\pgfqpoint{1.213891in}{2.408794in}}{\pgfqpoint{1.224490in}{2.404404in}}{\pgfqpoint{1.235540in}{2.404404in}}%
\pgfpathclose%
\pgfusepath{stroke,fill}%
\end{pgfscope}%
\begin{pgfscope}%
\pgfpathrectangle{\pgfqpoint{0.481978in}{0.331635in}}{\pgfqpoint{9.300000in}{7.700000in}}%
\pgfusepath{clip}%
\pgfsetbuttcap%
\pgfsetroundjoin%
\definecolor{currentfill}{rgb}{0.631373,0.788235,0.956863}%
\pgfsetfillcolor{currentfill}%
\pgfsetlinewidth{0.481800pt}%
\definecolor{currentstroke}{rgb}{1.000000,1.000000,1.000000}%
\pgfsetstrokecolor{currentstroke}%
\pgfsetdash{}{0pt}%
\pgfpathmoveto{\pgfqpoint{2.423803in}{3.142163in}}%
\pgfpathcurveto{\pgfqpoint{2.434853in}{3.142163in}}{\pgfqpoint{2.445452in}{3.146553in}}{\pgfqpoint{2.453265in}{3.154366in}}%
\pgfpathcurveto{\pgfqpoint{2.461079in}{3.162180in}}{\pgfqpoint{2.465469in}{3.172779in}}{\pgfqpoint{2.465469in}{3.183829in}}%
\pgfpathcurveto{\pgfqpoint{2.465469in}{3.194879in}}{\pgfqpoint{2.461079in}{3.205478in}}{\pgfqpoint{2.453265in}{3.213292in}}%
\pgfpathcurveto{\pgfqpoint{2.445452in}{3.221106in}}{\pgfqpoint{2.434853in}{3.225496in}}{\pgfqpoint{2.423803in}{3.225496in}}%
\pgfpathcurveto{\pgfqpoint{2.412752in}{3.225496in}}{\pgfqpoint{2.402153in}{3.221106in}}{\pgfqpoint{2.394340in}{3.213292in}}%
\pgfpathcurveto{\pgfqpoint{2.386526in}{3.205478in}}{\pgfqpoint{2.382136in}{3.194879in}}{\pgfqpoint{2.382136in}{3.183829in}}%
\pgfpathcurveto{\pgfqpoint{2.382136in}{3.172779in}}{\pgfqpoint{2.386526in}{3.162180in}}{\pgfqpoint{2.394340in}{3.154366in}}%
\pgfpathcurveto{\pgfqpoint{2.402153in}{3.146553in}}{\pgfqpoint{2.412752in}{3.142163in}}{\pgfqpoint{2.423803in}{3.142163in}}%
\pgfpathclose%
\pgfusepath{stroke,fill}%
\end{pgfscope}%
\begin{pgfscope}%
\pgfpathrectangle{\pgfqpoint{0.481978in}{0.331635in}}{\pgfqpoint{9.300000in}{7.700000in}}%
\pgfusepath{clip}%
\pgfsetbuttcap%
\pgfsetroundjoin%
\definecolor{currentfill}{rgb}{0.631373,0.788235,0.956863}%
\pgfsetfillcolor{currentfill}%
\pgfsetlinewidth{0.481800pt}%
\definecolor{currentstroke}{rgb}{1.000000,1.000000,1.000000}%
\pgfsetstrokecolor{currentstroke}%
\pgfsetdash{}{0pt}%
\pgfpathmoveto{\pgfqpoint{3.099413in}{4.257579in}}%
\pgfpathcurveto{\pgfqpoint{3.110463in}{4.257579in}}{\pgfqpoint{3.121062in}{4.261969in}}{\pgfqpoint{3.128876in}{4.269782in}}%
\pgfpathcurveto{\pgfqpoint{3.136690in}{4.277596in}}{\pgfqpoint{3.141080in}{4.288195in}}{\pgfqpoint{3.141080in}{4.299245in}}%
\pgfpathcurveto{\pgfqpoint{3.141080in}{4.310295in}}{\pgfqpoint{3.136690in}{4.320894in}}{\pgfqpoint{3.128876in}{4.328708in}}%
\pgfpathcurveto{\pgfqpoint{3.121062in}{4.336522in}}{\pgfqpoint{3.110463in}{4.340912in}}{\pgfqpoint{3.099413in}{4.340912in}}%
\pgfpathcurveto{\pgfqpoint{3.088363in}{4.340912in}}{\pgfqpoint{3.077764in}{4.336522in}}{\pgfqpoint{3.069950in}{4.328708in}}%
\pgfpathcurveto{\pgfqpoint{3.062137in}{4.320894in}}{\pgfqpoint{3.057746in}{4.310295in}}{\pgfqpoint{3.057746in}{4.299245in}}%
\pgfpathcurveto{\pgfqpoint{3.057746in}{4.288195in}}{\pgfqpoint{3.062137in}{4.277596in}}{\pgfqpoint{3.069950in}{4.269782in}}%
\pgfpathcurveto{\pgfqpoint{3.077764in}{4.261969in}}{\pgfqpoint{3.088363in}{4.257579in}}{\pgfqpoint{3.099413in}{4.257579in}}%
\pgfpathclose%
\pgfusepath{stroke,fill}%
\end{pgfscope}%
\begin{pgfscope}%
\pgfpathrectangle{\pgfqpoint{0.481978in}{0.331635in}}{\pgfqpoint{9.300000in}{7.700000in}}%
\pgfusepath{clip}%
\pgfsetbuttcap%
\pgfsetroundjoin%
\definecolor{currentfill}{rgb}{0.631373,0.788235,0.956863}%
\pgfsetfillcolor{currentfill}%
\pgfsetlinewidth{0.481800pt}%
\definecolor{currentstroke}{rgb}{1.000000,1.000000,1.000000}%
\pgfsetstrokecolor{currentstroke}%
\pgfsetdash{}{0pt}%
\pgfpathmoveto{\pgfqpoint{3.438982in}{4.455762in}}%
\pgfpathcurveto{\pgfqpoint{3.450032in}{4.455762in}}{\pgfqpoint{3.460631in}{4.460152in}}{\pgfqpoint{3.468445in}{4.467966in}}%
\pgfpathcurveto{\pgfqpoint{3.476258in}{4.475780in}}{\pgfqpoint{3.480648in}{4.486379in}}{\pgfqpoint{3.480648in}{4.497429in}}%
\pgfpathcurveto{\pgfqpoint{3.480648in}{4.508479in}}{\pgfqpoint{3.476258in}{4.519078in}}{\pgfqpoint{3.468445in}{4.526892in}}%
\pgfpathcurveto{\pgfqpoint{3.460631in}{4.534705in}}{\pgfqpoint{3.450032in}{4.539096in}}{\pgfqpoint{3.438982in}{4.539096in}}%
\pgfpathcurveto{\pgfqpoint{3.427932in}{4.539096in}}{\pgfqpoint{3.417333in}{4.534705in}}{\pgfqpoint{3.409519in}{4.526892in}}%
\pgfpathcurveto{\pgfqpoint{3.401705in}{4.519078in}}{\pgfqpoint{3.397315in}{4.508479in}}{\pgfqpoint{3.397315in}{4.497429in}}%
\pgfpathcurveto{\pgfqpoint{3.397315in}{4.486379in}}{\pgfqpoint{3.401705in}{4.475780in}}{\pgfqpoint{3.409519in}{4.467966in}}%
\pgfpathcurveto{\pgfqpoint{3.417333in}{4.460152in}}{\pgfqpoint{3.427932in}{4.455762in}}{\pgfqpoint{3.438982in}{4.455762in}}%
\pgfpathclose%
\pgfusepath{stroke,fill}%
\end{pgfscope}%
\begin{pgfscope}%
\pgfpathrectangle{\pgfqpoint{0.481978in}{0.331635in}}{\pgfqpoint{9.300000in}{7.700000in}}%
\pgfusepath{clip}%
\pgfsetbuttcap%
\pgfsetroundjoin%
\definecolor{currentfill}{rgb}{0.631373,0.788235,0.956863}%
\pgfsetfillcolor{currentfill}%
\pgfsetlinewidth{0.481800pt}%
\definecolor{currentstroke}{rgb}{1.000000,1.000000,1.000000}%
\pgfsetstrokecolor{currentstroke}%
\pgfsetdash{}{0pt}%
\pgfpathmoveto{\pgfqpoint{6.239241in}{3.949736in}}%
\pgfpathcurveto{\pgfqpoint{6.250292in}{3.949736in}}{\pgfqpoint{6.260891in}{3.954126in}}{\pgfqpoint{6.268704in}{3.961940in}}%
\pgfpathcurveto{\pgfqpoint{6.276518in}{3.969754in}}{\pgfqpoint{6.280908in}{3.980353in}}{\pgfqpoint{6.280908in}{3.991403in}}%
\pgfpathcurveto{\pgfqpoint{6.280908in}{4.002453in}}{\pgfqpoint{6.276518in}{4.013052in}}{\pgfqpoint{6.268704in}{4.020866in}}%
\pgfpathcurveto{\pgfqpoint{6.260891in}{4.028679in}}{\pgfqpoint{6.250292in}{4.033069in}}{\pgfqpoint{6.239241in}{4.033069in}}%
\pgfpathcurveto{\pgfqpoint{6.228191in}{4.033069in}}{\pgfqpoint{6.217592in}{4.028679in}}{\pgfqpoint{6.209779in}{4.020866in}}%
\pgfpathcurveto{\pgfqpoint{6.201965in}{4.013052in}}{\pgfqpoint{6.197575in}{4.002453in}}{\pgfqpoint{6.197575in}{3.991403in}}%
\pgfpathcurveto{\pgfqpoint{6.197575in}{3.980353in}}{\pgfqpoint{6.201965in}{3.969754in}}{\pgfqpoint{6.209779in}{3.961940in}}%
\pgfpathcurveto{\pgfqpoint{6.217592in}{3.954126in}}{\pgfqpoint{6.228191in}{3.949736in}}{\pgfqpoint{6.239241in}{3.949736in}}%
\pgfpathclose%
\pgfusepath{stroke,fill}%
\end{pgfscope}%
\begin{pgfscope}%
\pgfpathrectangle{\pgfqpoint{0.481978in}{0.331635in}}{\pgfqpoint{9.300000in}{7.700000in}}%
\pgfusepath{clip}%
\pgfsetbuttcap%
\pgfsetroundjoin%
\definecolor{currentfill}{rgb}{0.631373,0.788235,0.956863}%
\pgfsetfillcolor{currentfill}%
\pgfsetlinewidth{0.481800pt}%
\definecolor{currentstroke}{rgb}{1.000000,1.000000,1.000000}%
\pgfsetstrokecolor{currentstroke}%
\pgfsetdash{}{0pt}%
\pgfpathmoveto{\pgfqpoint{7.439054in}{1.158266in}}%
\pgfpathcurveto{\pgfqpoint{7.450104in}{1.158266in}}{\pgfqpoint{7.460703in}{1.162656in}}{\pgfqpoint{7.468516in}{1.170470in}}%
\pgfpathcurveto{\pgfqpoint{7.476330in}{1.178284in}}{\pgfqpoint{7.480720in}{1.188883in}}{\pgfqpoint{7.480720in}{1.199933in}}%
\pgfpathcurveto{\pgfqpoint{7.480720in}{1.210983in}}{\pgfqpoint{7.476330in}{1.221582in}}{\pgfqpoint{7.468516in}{1.229396in}}%
\pgfpathcurveto{\pgfqpoint{7.460703in}{1.237209in}}{\pgfqpoint{7.450104in}{1.241599in}}{\pgfqpoint{7.439054in}{1.241599in}}%
\pgfpathcurveto{\pgfqpoint{7.428003in}{1.241599in}}{\pgfqpoint{7.417404in}{1.237209in}}{\pgfqpoint{7.409591in}{1.229396in}}%
\pgfpathcurveto{\pgfqpoint{7.401777in}{1.221582in}}{\pgfqpoint{7.397387in}{1.210983in}}{\pgfqpoint{7.397387in}{1.199933in}}%
\pgfpathcurveto{\pgfqpoint{7.397387in}{1.188883in}}{\pgfqpoint{7.401777in}{1.178284in}}{\pgfqpoint{7.409591in}{1.170470in}}%
\pgfpathcurveto{\pgfqpoint{7.417404in}{1.162656in}}{\pgfqpoint{7.428003in}{1.158266in}}{\pgfqpoint{7.439054in}{1.158266in}}%
\pgfpathclose%
\pgfusepath{stroke,fill}%
\end{pgfscope}%
\begin{pgfscope}%
\pgfpathrectangle{\pgfqpoint{0.481978in}{0.331635in}}{\pgfqpoint{9.300000in}{7.700000in}}%
\pgfusepath{clip}%
\pgfsetbuttcap%
\pgfsetroundjoin%
\definecolor{currentfill}{rgb}{0.631373,0.788235,0.956863}%
\pgfsetfillcolor{currentfill}%
\pgfsetlinewidth{0.481800pt}%
\definecolor{currentstroke}{rgb}{1.000000,1.000000,1.000000}%
\pgfsetstrokecolor{currentstroke}%
\pgfsetdash{}{0pt}%
\pgfpathmoveto{\pgfqpoint{7.510101in}{1.642964in}}%
\pgfpathcurveto{\pgfqpoint{7.521151in}{1.642964in}}{\pgfqpoint{7.531750in}{1.647355in}}{\pgfqpoint{7.539564in}{1.655168in}}%
\pgfpathcurveto{\pgfqpoint{7.547377in}{1.662982in}}{\pgfqpoint{7.551768in}{1.673581in}}{\pgfqpoint{7.551768in}{1.684631in}}%
\pgfpathcurveto{\pgfqpoint{7.551768in}{1.695681in}}{\pgfqpoint{7.547377in}{1.706280in}}{\pgfqpoint{7.539564in}{1.714094in}}%
\pgfpathcurveto{\pgfqpoint{7.531750in}{1.721907in}}{\pgfqpoint{7.521151in}{1.726298in}}{\pgfqpoint{7.510101in}{1.726298in}}%
\pgfpathcurveto{\pgfqpoint{7.499051in}{1.726298in}}{\pgfqpoint{7.488452in}{1.721907in}}{\pgfqpoint{7.480638in}{1.714094in}}%
\pgfpathcurveto{\pgfqpoint{7.472824in}{1.706280in}}{\pgfqpoint{7.468434in}{1.695681in}}{\pgfqpoint{7.468434in}{1.684631in}}%
\pgfpathcurveto{\pgfqpoint{7.468434in}{1.673581in}}{\pgfqpoint{7.472824in}{1.662982in}}{\pgfqpoint{7.480638in}{1.655168in}}%
\pgfpathcurveto{\pgfqpoint{7.488452in}{1.647355in}}{\pgfqpoint{7.499051in}{1.642964in}}{\pgfqpoint{7.510101in}{1.642964in}}%
\pgfpathclose%
\pgfusepath{stroke,fill}%
\end{pgfscope}%
\begin{pgfscope}%
\pgfpathrectangle{\pgfqpoint{0.481978in}{0.331635in}}{\pgfqpoint{9.300000in}{7.700000in}}%
\pgfusepath{clip}%
\pgfsetbuttcap%
\pgfsetroundjoin%
\definecolor{currentfill}{rgb}{0.631373,0.788235,0.956863}%
\pgfsetfillcolor{currentfill}%
\pgfsetlinewidth{0.481800pt}%
\definecolor{currentstroke}{rgb}{1.000000,1.000000,1.000000}%
\pgfsetstrokecolor{currentstroke}%
\pgfsetdash{}{0pt}%
\pgfpathmoveto{\pgfqpoint{2.315914in}{4.192320in}}%
\pgfpathcurveto{\pgfqpoint{2.326964in}{4.192320in}}{\pgfqpoint{2.337563in}{4.196710in}}{\pgfqpoint{2.345376in}{4.204524in}}%
\pgfpathcurveto{\pgfqpoint{2.353190in}{4.212337in}}{\pgfqpoint{2.357580in}{4.222936in}}{\pgfqpoint{2.357580in}{4.233986in}}%
\pgfpathcurveto{\pgfqpoint{2.357580in}{4.245036in}}{\pgfqpoint{2.353190in}{4.255635in}}{\pgfqpoint{2.345376in}{4.263449in}}%
\pgfpathcurveto{\pgfqpoint{2.337563in}{4.271263in}}{\pgfqpoint{2.326964in}{4.275653in}}{\pgfqpoint{2.315914in}{4.275653in}}%
\pgfpathcurveto{\pgfqpoint{2.304863in}{4.275653in}}{\pgfqpoint{2.294264in}{4.271263in}}{\pgfqpoint{2.286451in}{4.263449in}}%
\pgfpathcurveto{\pgfqpoint{2.278637in}{4.255635in}}{\pgfqpoint{2.274247in}{4.245036in}}{\pgfqpoint{2.274247in}{4.233986in}}%
\pgfpathcurveto{\pgfqpoint{2.274247in}{4.222936in}}{\pgfqpoint{2.278637in}{4.212337in}}{\pgfqpoint{2.286451in}{4.204524in}}%
\pgfpathcurveto{\pgfqpoint{2.294264in}{4.196710in}}{\pgfqpoint{2.304863in}{4.192320in}}{\pgfqpoint{2.315914in}{4.192320in}}%
\pgfpathclose%
\pgfusepath{stroke,fill}%
\end{pgfscope}%
\begin{pgfscope}%
\pgfpathrectangle{\pgfqpoint{0.481978in}{0.331635in}}{\pgfqpoint{9.300000in}{7.700000in}}%
\pgfusepath{clip}%
\pgfsetbuttcap%
\pgfsetroundjoin%
\definecolor{currentfill}{rgb}{0.631373,0.788235,0.956863}%
\pgfsetfillcolor{currentfill}%
\pgfsetlinewidth{0.481800pt}%
\definecolor{currentstroke}{rgb}{1.000000,1.000000,1.000000}%
\pgfsetstrokecolor{currentstroke}%
\pgfsetdash{}{0pt}%
\pgfpathmoveto{\pgfqpoint{6.878899in}{1.071438in}}%
\pgfpathcurveto{\pgfqpoint{6.889949in}{1.071438in}}{\pgfqpoint{6.900548in}{1.075828in}}{\pgfqpoint{6.908362in}{1.083641in}}%
\pgfpathcurveto{\pgfqpoint{6.916175in}{1.091455in}}{\pgfqpoint{6.920566in}{1.102054in}}{\pgfqpoint{6.920566in}{1.113104in}}%
\pgfpathcurveto{\pgfqpoint{6.920566in}{1.124154in}}{\pgfqpoint{6.916175in}{1.134753in}}{\pgfqpoint{6.908362in}{1.142567in}}%
\pgfpathcurveto{\pgfqpoint{6.900548in}{1.150381in}}{\pgfqpoint{6.889949in}{1.154771in}}{\pgfqpoint{6.878899in}{1.154771in}}%
\pgfpathcurveto{\pgfqpoint{6.867849in}{1.154771in}}{\pgfqpoint{6.857250in}{1.150381in}}{\pgfqpoint{6.849436in}{1.142567in}}%
\pgfpathcurveto{\pgfqpoint{6.841623in}{1.134753in}}{\pgfqpoint{6.837232in}{1.124154in}}{\pgfqpoint{6.837232in}{1.113104in}}%
\pgfpathcurveto{\pgfqpoint{6.837232in}{1.102054in}}{\pgfqpoint{6.841623in}{1.091455in}}{\pgfqpoint{6.849436in}{1.083641in}}%
\pgfpathcurveto{\pgfqpoint{6.857250in}{1.075828in}}{\pgfqpoint{6.867849in}{1.071438in}}{\pgfqpoint{6.878899in}{1.071438in}}%
\pgfpathclose%
\pgfusepath{stroke,fill}%
\end{pgfscope}%
\begin{pgfscope}%
\pgfpathrectangle{\pgfqpoint{0.481978in}{0.331635in}}{\pgfqpoint{9.300000in}{7.700000in}}%
\pgfusepath{clip}%
\pgfsetbuttcap%
\pgfsetroundjoin%
\definecolor{currentfill}{rgb}{0.631373,0.788235,0.956863}%
\pgfsetfillcolor{currentfill}%
\pgfsetlinewidth{0.481800pt}%
\definecolor{currentstroke}{rgb}{1.000000,1.000000,1.000000}%
\pgfsetstrokecolor{currentstroke}%
\pgfsetdash{}{0pt}%
\pgfpathmoveto{\pgfqpoint{3.274514in}{1.534614in}}%
\pgfpathcurveto{\pgfqpoint{3.285565in}{1.534614in}}{\pgfqpoint{3.296164in}{1.539004in}}{\pgfqpoint{3.303977in}{1.546818in}}%
\pgfpathcurveto{\pgfqpoint{3.311791in}{1.554631in}}{\pgfqpoint{3.316181in}{1.565230in}}{\pgfqpoint{3.316181in}{1.576281in}}%
\pgfpathcurveto{\pgfqpoint{3.316181in}{1.587331in}}{\pgfqpoint{3.311791in}{1.597930in}}{\pgfqpoint{3.303977in}{1.605743in}}%
\pgfpathcurveto{\pgfqpoint{3.296164in}{1.613557in}}{\pgfqpoint{3.285565in}{1.617947in}}{\pgfqpoint{3.274514in}{1.617947in}}%
\pgfpathcurveto{\pgfqpoint{3.263464in}{1.617947in}}{\pgfqpoint{3.252865in}{1.613557in}}{\pgfqpoint{3.245052in}{1.605743in}}%
\pgfpathcurveto{\pgfqpoint{3.237238in}{1.597930in}}{\pgfqpoint{3.232848in}{1.587331in}}{\pgfqpoint{3.232848in}{1.576281in}}%
\pgfpathcurveto{\pgfqpoint{3.232848in}{1.565230in}}{\pgfqpoint{3.237238in}{1.554631in}}{\pgfqpoint{3.245052in}{1.546818in}}%
\pgfpathcurveto{\pgfqpoint{3.252865in}{1.539004in}}{\pgfqpoint{3.263464in}{1.534614in}}{\pgfqpoint{3.274514in}{1.534614in}}%
\pgfpathclose%
\pgfusepath{stroke,fill}%
\end{pgfscope}%
\begin{pgfscope}%
\pgfpathrectangle{\pgfqpoint{0.481978in}{0.331635in}}{\pgfqpoint{9.300000in}{7.700000in}}%
\pgfusepath{clip}%
\pgfsetbuttcap%
\pgfsetroundjoin%
\definecolor{currentfill}{rgb}{0.631373,0.788235,0.956863}%
\pgfsetfillcolor{currentfill}%
\pgfsetlinewidth{0.481800pt}%
\definecolor{currentstroke}{rgb}{1.000000,1.000000,1.000000}%
\pgfsetstrokecolor{currentstroke}%
\pgfsetdash{}{0pt}%
\pgfpathmoveto{\pgfqpoint{6.005151in}{4.440552in}}%
\pgfpathcurveto{\pgfqpoint{6.016201in}{4.440552in}}{\pgfqpoint{6.026800in}{4.444943in}}{\pgfqpoint{6.034613in}{4.452756in}}%
\pgfpathcurveto{\pgfqpoint{6.042427in}{4.460570in}}{\pgfqpoint{6.046817in}{4.471169in}}{\pgfqpoint{6.046817in}{4.482219in}}%
\pgfpathcurveto{\pgfqpoint{6.046817in}{4.493269in}}{\pgfqpoint{6.042427in}{4.503868in}}{\pgfqpoint{6.034613in}{4.511682in}}%
\pgfpathcurveto{\pgfqpoint{6.026800in}{4.519495in}}{\pgfqpoint{6.016201in}{4.523886in}}{\pgfqpoint{6.005151in}{4.523886in}}%
\pgfpathcurveto{\pgfqpoint{5.994100in}{4.523886in}}{\pgfqpoint{5.983501in}{4.519495in}}{\pgfqpoint{5.975688in}{4.511682in}}%
\pgfpathcurveto{\pgfqpoint{5.967874in}{4.503868in}}{\pgfqpoint{5.963484in}{4.493269in}}{\pgfqpoint{5.963484in}{4.482219in}}%
\pgfpathcurveto{\pgfqpoint{5.963484in}{4.471169in}}{\pgfqpoint{5.967874in}{4.460570in}}{\pgfqpoint{5.975688in}{4.452756in}}%
\pgfpathcurveto{\pgfqpoint{5.983501in}{4.444943in}}{\pgfqpoint{5.994100in}{4.440552in}}{\pgfqpoint{6.005151in}{4.440552in}}%
\pgfpathclose%
\pgfusepath{stroke,fill}%
\end{pgfscope}%
\begin{pgfscope}%
\pgfpathrectangle{\pgfqpoint{0.481978in}{0.331635in}}{\pgfqpoint{9.300000in}{7.700000in}}%
\pgfusepath{clip}%
\pgfsetbuttcap%
\pgfsetroundjoin%
\definecolor{currentfill}{rgb}{0.631373,0.788235,0.956863}%
\pgfsetfillcolor{currentfill}%
\pgfsetlinewidth{0.481800pt}%
\definecolor{currentstroke}{rgb}{1.000000,1.000000,1.000000}%
\pgfsetstrokecolor{currentstroke}%
\pgfsetdash{}{0pt}%
\pgfpathmoveto{\pgfqpoint{7.064826in}{4.790265in}}%
\pgfpathcurveto{\pgfqpoint{7.075876in}{4.790265in}}{\pgfqpoint{7.086475in}{4.794656in}}{\pgfqpoint{7.094289in}{4.802469in}}%
\pgfpathcurveto{\pgfqpoint{7.102102in}{4.810283in}}{\pgfqpoint{7.106493in}{4.820882in}}{\pgfqpoint{7.106493in}{4.831932in}}%
\pgfpathcurveto{\pgfqpoint{7.106493in}{4.842982in}}{\pgfqpoint{7.102102in}{4.853581in}}{\pgfqpoint{7.094289in}{4.861395in}}%
\pgfpathcurveto{\pgfqpoint{7.086475in}{4.869208in}}{\pgfqpoint{7.075876in}{4.873599in}}{\pgfqpoint{7.064826in}{4.873599in}}%
\pgfpathcurveto{\pgfqpoint{7.053776in}{4.873599in}}{\pgfqpoint{7.043177in}{4.869208in}}{\pgfqpoint{7.035363in}{4.861395in}}%
\pgfpathcurveto{\pgfqpoint{7.027550in}{4.853581in}}{\pgfqpoint{7.023159in}{4.842982in}}{\pgfqpoint{7.023159in}{4.831932in}}%
\pgfpathcurveto{\pgfqpoint{7.023159in}{4.820882in}}{\pgfqpoint{7.027550in}{4.810283in}}{\pgfqpoint{7.035363in}{4.802469in}}%
\pgfpathcurveto{\pgfqpoint{7.043177in}{4.794656in}}{\pgfqpoint{7.053776in}{4.790265in}}{\pgfqpoint{7.064826in}{4.790265in}}%
\pgfpathclose%
\pgfusepath{stroke,fill}%
\end{pgfscope}%
\begin{pgfscope}%
\pgfpathrectangle{\pgfqpoint{0.481978in}{0.331635in}}{\pgfqpoint{9.300000in}{7.700000in}}%
\pgfusepath{clip}%
\pgfsetbuttcap%
\pgfsetroundjoin%
\definecolor{currentfill}{rgb}{0.631373,0.788235,0.956863}%
\pgfsetfillcolor{currentfill}%
\pgfsetlinewidth{0.481800pt}%
\definecolor{currentstroke}{rgb}{1.000000,1.000000,1.000000}%
\pgfsetstrokecolor{currentstroke}%
\pgfsetdash{}{0pt}%
\pgfpathmoveto{\pgfqpoint{6.961902in}{1.575435in}}%
\pgfpathcurveto{\pgfqpoint{6.972953in}{1.575435in}}{\pgfqpoint{6.983552in}{1.579825in}}{\pgfqpoint{6.991365in}{1.587638in}}%
\pgfpathcurveto{\pgfqpoint{6.999179in}{1.595452in}}{\pgfqpoint{7.003569in}{1.606051in}}{\pgfqpoint{7.003569in}{1.617101in}}%
\pgfpathcurveto{\pgfqpoint{7.003569in}{1.628151in}}{\pgfqpoint{6.999179in}{1.638750in}}{\pgfqpoint{6.991365in}{1.646564in}}%
\pgfpathcurveto{\pgfqpoint{6.983552in}{1.654378in}}{\pgfqpoint{6.972953in}{1.658768in}}{\pgfqpoint{6.961902in}{1.658768in}}%
\pgfpathcurveto{\pgfqpoint{6.950852in}{1.658768in}}{\pgfqpoint{6.940253in}{1.654378in}}{\pgfqpoint{6.932440in}{1.646564in}}%
\pgfpathcurveto{\pgfqpoint{6.924626in}{1.638750in}}{\pgfqpoint{6.920236in}{1.628151in}}{\pgfqpoint{6.920236in}{1.617101in}}%
\pgfpathcurveto{\pgfqpoint{6.920236in}{1.606051in}}{\pgfqpoint{6.924626in}{1.595452in}}{\pgfqpoint{6.932440in}{1.587638in}}%
\pgfpathcurveto{\pgfqpoint{6.940253in}{1.579825in}}{\pgfqpoint{6.950852in}{1.575435in}}{\pgfqpoint{6.961902in}{1.575435in}}%
\pgfpathclose%
\pgfusepath{stroke,fill}%
\end{pgfscope}%
\begin{pgfscope}%
\pgfpathrectangle{\pgfqpoint{0.481978in}{0.331635in}}{\pgfqpoint{9.300000in}{7.700000in}}%
\pgfusepath{clip}%
\pgfsetbuttcap%
\pgfsetroundjoin%
\definecolor{currentfill}{rgb}{0.631373,0.788235,0.956863}%
\pgfsetfillcolor{currentfill}%
\pgfsetlinewidth{0.481800pt}%
\definecolor{currentstroke}{rgb}{1.000000,1.000000,1.000000}%
\pgfsetstrokecolor{currentstroke}%
\pgfsetdash{}{0pt}%
\pgfpathmoveto{\pgfqpoint{3.800404in}{4.844822in}}%
\pgfpathcurveto{\pgfqpoint{3.811455in}{4.844822in}}{\pgfqpoint{3.822054in}{4.849212in}}{\pgfqpoint{3.829867in}{4.857026in}}%
\pgfpathcurveto{\pgfqpoint{3.837681in}{4.864840in}}{\pgfqpoint{3.842071in}{4.875439in}}{\pgfqpoint{3.842071in}{4.886489in}}%
\pgfpathcurveto{\pgfqpoint{3.842071in}{4.897539in}}{\pgfqpoint{3.837681in}{4.908138in}}{\pgfqpoint{3.829867in}{4.915952in}}%
\pgfpathcurveto{\pgfqpoint{3.822054in}{4.923765in}}{\pgfqpoint{3.811455in}{4.928155in}}{\pgfqpoint{3.800404in}{4.928155in}}%
\pgfpathcurveto{\pgfqpoint{3.789354in}{4.928155in}}{\pgfqpoint{3.778755in}{4.923765in}}{\pgfqpoint{3.770942in}{4.915952in}}%
\pgfpathcurveto{\pgfqpoint{3.763128in}{4.908138in}}{\pgfqpoint{3.758738in}{4.897539in}}{\pgfqpoint{3.758738in}{4.886489in}}%
\pgfpathcurveto{\pgfqpoint{3.758738in}{4.875439in}}{\pgfqpoint{3.763128in}{4.864840in}}{\pgfqpoint{3.770942in}{4.857026in}}%
\pgfpathcurveto{\pgfqpoint{3.778755in}{4.849212in}}{\pgfqpoint{3.789354in}{4.844822in}}{\pgfqpoint{3.800404in}{4.844822in}}%
\pgfpathclose%
\pgfusepath{stroke,fill}%
\end{pgfscope}%
\begin{pgfscope}%
\pgfpathrectangle{\pgfqpoint{0.481978in}{0.331635in}}{\pgfqpoint{9.300000in}{7.700000in}}%
\pgfusepath{clip}%
\pgfsetbuttcap%
\pgfsetroundjoin%
\definecolor{currentfill}{rgb}{0.631373,0.788235,0.956863}%
\pgfsetfillcolor{currentfill}%
\pgfsetlinewidth{0.481800pt}%
\definecolor{currentstroke}{rgb}{1.000000,1.000000,1.000000}%
\pgfsetstrokecolor{currentstroke}%
\pgfsetdash{}{0pt}%
\pgfpathmoveto{\pgfqpoint{6.058849in}{1.448720in}}%
\pgfpathcurveto{\pgfqpoint{6.069899in}{1.448720in}}{\pgfqpoint{6.080498in}{1.453110in}}{\pgfqpoint{6.088312in}{1.460924in}}%
\pgfpathcurveto{\pgfqpoint{6.096125in}{1.468738in}}{\pgfqpoint{6.100516in}{1.479337in}}{\pgfqpoint{6.100516in}{1.490387in}}%
\pgfpathcurveto{\pgfqpoint{6.100516in}{1.501437in}}{\pgfqpoint{6.096125in}{1.512036in}}{\pgfqpoint{6.088312in}{1.519850in}}%
\pgfpathcurveto{\pgfqpoint{6.080498in}{1.527663in}}{\pgfqpoint{6.069899in}{1.532054in}}{\pgfqpoint{6.058849in}{1.532054in}}%
\pgfpathcurveto{\pgfqpoint{6.047799in}{1.532054in}}{\pgfqpoint{6.037200in}{1.527663in}}{\pgfqpoint{6.029386in}{1.519850in}}%
\pgfpathcurveto{\pgfqpoint{6.021573in}{1.512036in}}{\pgfqpoint{6.017182in}{1.501437in}}{\pgfqpoint{6.017182in}{1.490387in}}%
\pgfpathcurveto{\pgfqpoint{6.017182in}{1.479337in}}{\pgfqpoint{6.021573in}{1.468738in}}{\pgfqpoint{6.029386in}{1.460924in}}%
\pgfpathcurveto{\pgfqpoint{6.037200in}{1.453110in}}{\pgfqpoint{6.047799in}{1.448720in}}{\pgfqpoint{6.058849in}{1.448720in}}%
\pgfpathclose%
\pgfusepath{stroke,fill}%
\end{pgfscope}%
\begin{pgfscope}%
\pgfpathrectangle{\pgfqpoint{0.481978in}{0.331635in}}{\pgfqpoint{9.300000in}{7.700000in}}%
\pgfusepath{clip}%
\pgfsetbuttcap%
\pgfsetroundjoin%
\definecolor{currentfill}{rgb}{0.631373,0.788235,0.956863}%
\pgfsetfillcolor{currentfill}%
\pgfsetlinewidth{0.481800pt}%
\definecolor{currentstroke}{rgb}{1.000000,1.000000,1.000000}%
\pgfsetstrokecolor{currentstroke}%
\pgfsetdash{}{0pt}%
\pgfpathmoveto{\pgfqpoint{5.307606in}{5.921468in}}%
\pgfpathcurveto{\pgfqpoint{5.318656in}{5.921468in}}{\pgfqpoint{5.329255in}{5.925859in}}{\pgfqpoint{5.337068in}{5.933672in}}%
\pgfpathcurveto{\pgfqpoint{5.344882in}{5.941486in}}{\pgfqpoint{5.349272in}{5.952085in}}{\pgfqpoint{5.349272in}{5.963135in}}%
\pgfpathcurveto{\pgfqpoint{5.349272in}{5.974185in}}{\pgfqpoint{5.344882in}{5.984784in}}{\pgfqpoint{5.337068in}{5.992598in}}%
\pgfpathcurveto{\pgfqpoint{5.329255in}{6.000411in}}{\pgfqpoint{5.318656in}{6.004802in}}{\pgfqpoint{5.307606in}{6.004802in}}%
\pgfpathcurveto{\pgfqpoint{5.296555in}{6.004802in}}{\pgfqpoint{5.285956in}{6.000411in}}{\pgfqpoint{5.278143in}{5.992598in}}%
\pgfpathcurveto{\pgfqpoint{5.270329in}{5.984784in}}{\pgfqpoint{5.265939in}{5.974185in}}{\pgfqpoint{5.265939in}{5.963135in}}%
\pgfpathcurveto{\pgfqpoint{5.265939in}{5.952085in}}{\pgfqpoint{5.270329in}{5.941486in}}{\pgfqpoint{5.278143in}{5.933672in}}%
\pgfpathcurveto{\pgfqpoint{5.285956in}{5.925859in}}{\pgfqpoint{5.296555in}{5.921468in}}{\pgfqpoint{5.307606in}{5.921468in}}%
\pgfpathclose%
\pgfusepath{stroke,fill}%
\end{pgfscope}%
\begin{pgfscope}%
\pgfpathrectangle{\pgfqpoint{0.481978in}{0.331635in}}{\pgfqpoint{9.300000in}{7.700000in}}%
\pgfusepath{clip}%
\pgfsetbuttcap%
\pgfsetroundjoin%
\definecolor{currentfill}{rgb}{0.631373,0.788235,0.956863}%
\pgfsetfillcolor{currentfill}%
\pgfsetlinewidth{0.481800pt}%
\definecolor{currentstroke}{rgb}{1.000000,1.000000,1.000000}%
\pgfsetstrokecolor{currentstroke}%
\pgfsetdash{}{0pt}%
\pgfpathmoveto{\pgfqpoint{7.033678in}{3.796632in}}%
\pgfpathcurveto{\pgfqpoint{7.044728in}{3.796632in}}{\pgfqpoint{7.055327in}{3.801023in}}{\pgfqpoint{7.063140in}{3.808836in}}%
\pgfpathcurveto{\pgfqpoint{7.070954in}{3.816650in}}{\pgfqpoint{7.075344in}{3.827249in}}{\pgfqpoint{7.075344in}{3.838299in}}%
\pgfpathcurveto{\pgfqpoint{7.075344in}{3.849349in}}{\pgfqpoint{7.070954in}{3.859948in}}{\pgfqpoint{7.063140in}{3.867762in}}%
\pgfpathcurveto{\pgfqpoint{7.055327in}{3.875576in}}{\pgfqpoint{7.044728in}{3.879966in}}{\pgfqpoint{7.033678in}{3.879966in}}%
\pgfpathcurveto{\pgfqpoint{7.022627in}{3.879966in}}{\pgfqpoint{7.012028in}{3.875576in}}{\pgfqpoint{7.004215in}{3.867762in}}%
\pgfpathcurveto{\pgfqpoint{6.996401in}{3.859948in}}{\pgfqpoint{6.992011in}{3.849349in}}{\pgfqpoint{6.992011in}{3.838299in}}%
\pgfpathcurveto{\pgfqpoint{6.992011in}{3.827249in}}{\pgfqpoint{6.996401in}{3.816650in}}{\pgfqpoint{7.004215in}{3.808836in}}%
\pgfpathcurveto{\pgfqpoint{7.012028in}{3.801023in}}{\pgfqpoint{7.022627in}{3.796632in}}{\pgfqpoint{7.033678in}{3.796632in}}%
\pgfpathclose%
\pgfusepath{stroke,fill}%
\end{pgfscope}%
\begin{pgfscope}%
\pgfpathrectangle{\pgfqpoint{0.481978in}{0.331635in}}{\pgfqpoint{9.300000in}{7.700000in}}%
\pgfusepath{clip}%
\pgfsetbuttcap%
\pgfsetroundjoin%
\definecolor{currentfill}{rgb}{0.631373,0.788235,0.956863}%
\pgfsetfillcolor{currentfill}%
\pgfsetlinewidth{0.481800pt}%
\definecolor{currentstroke}{rgb}{1.000000,1.000000,1.000000}%
\pgfsetstrokecolor{currentstroke}%
\pgfsetdash{}{0pt}%
\pgfpathmoveto{\pgfqpoint{7.805428in}{0.806753in}}%
\pgfpathcurveto{\pgfqpoint{7.816478in}{0.806753in}}{\pgfqpoint{7.827077in}{0.811143in}}{\pgfqpoint{7.834891in}{0.818957in}}%
\pgfpathcurveto{\pgfqpoint{7.842704in}{0.826770in}}{\pgfqpoint{7.847095in}{0.837369in}}{\pgfqpoint{7.847095in}{0.848419in}}%
\pgfpathcurveto{\pgfqpoint{7.847095in}{0.859470in}}{\pgfqpoint{7.842704in}{0.870069in}}{\pgfqpoint{7.834891in}{0.877882in}}%
\pgfpathcurveto{\pgfqpoint{7.827077in}{0.885696in}}{\pgfqpoint{7.816478in}{0.890086in}}{\pgfqpoint{7.805428in}{0.890086in}}%
\pgfpathcurveto{\pgfqpoint{7.794378in}{0.890086in}}{\pgfqpoint{7.783779in}{0.885696in}}{\pgfqpoint{7.775965in}{0.877882in}}%
\pgfpathcurveto{\pgfqpoint{7.768151in}{0.870069in}}{\pgfqpoint{7.763761in}{0.859470in}}{\pgfqpoint{7.763761in}{0.848419in}}%
\pgfpathcurveto{\pgfqpoint{7.763761in}{0.837369in}}{\pgfqpoint{7.768151in}{0.826770in}}{\pgfqpoint{7.775965in}{0.818957in}}%
\pgfpathcurveto{\pgfqpoint{7.783779in}{0.811143in}}{\pgfqpoint{7.794378in}{0.806753in}}{\pgfqpoint{7.805428in}{0.806753in}}%
\pgfpathclose%
\pgfusepath{stroke,fill}%
\end{pgfscope}%
\begin{pgfscope}%
\pgfpathrectangle{\pgfqpoint{0.481978in}{0.331635in}}{\pgfqpoint{9.300000in}{7.700000in}}%
\pgfusepath{clip}%
\pgfsetbuttcap%
\pgfsetroundjoin%
\definecolor{currentfill}{rgb}{0.631373,0.788235,0.956863}%
\pgfsetfillcolor{currentfill}%
\pgfsetlinewidth{0.481800pt}%
\definecolor{currentstroke}{rgb}{1.000000,1.000000,1.000000}%
\pgfsetstrokecolor{currentstroke}%
\pgfsetdash{}{0pt}%
\pgfpathmoveto{\pgfqpoint{3.104471in}{4.739766in}}%
\pgfpathcurveto{\pgfqpoint{3.115521in}{4.739766in}}{\pgfqpoint{3.126120in}{4.744156in}}{\pgfqpoint{3.133933in}{4.751969in}}%
\pgfpathcurveto{\pgfqpoint{3.141747in}{4.759783in}}{\pgfqpoint{3.146137in}{4.770382in}}{\pgfqpoint{3.146137in}{4.781432in}}%
\pgfpathcurveto{\pgfqpoint{3.146137in}{4.792482in}}{\pgfqpoint{3.141747in}{4.803081in}}{\pgfqpoint{3.133933in}{4.810895in}}%
\pgfpathcurveto{\pgfqpoint{3.126120in}{4.818709in}}{\pgfqpoint{3.115521in}{4.823099in}}{\pgfqpoint{3.104471in}{4.823099in}}%
\pgfpathcurveto{\pgfqpoint{3.093421in}{4.823099in}}{\pgfqpoint{3.082822in}{4.818709in}}{\pgfqpoint{3.075008in}{4.810895in}}%
\pgfpathcurveto{\pgfqpoint{3.067194in}{4.803081in}}{\pgfqpoint{3.062804in}{4.792482in}}{\pgfqpoint{3.062804in}{4.781432in}}%
\pgfpathcurveto{\pgfqpoint{3.062804in}{4.770382in}}{\pgfqpoint{3.067194in}{4.759783in}}{\pgfqpoint{3.075008in}{4.751969in}}%
\pgfpathcurveto{\pgfqpoint{3.082822in}{4.744156in}}{\pgfqpoint{3.093421in}{4.739766in}}{\pgfqpoint{3.104471in}{4.739766in}}%
\pgfpathclose%
\pgfusepath{stroke,fill}%
\end{pgfscope}%
\begin{pgfscope}%
\pgfpathrectangle{\pgfqpoint{0.481978in}{0.331635in}}{\pgfqpoint{9.300000in}{7.700000in}}%
\pgfusepath{clip}%
\pgfsetbuttcap%
\pgfsetroundjoin%
\definecolor{currentfill}{rgb}{0.631373,0.788235,0.956863}%
\pgfsetfillcolor{currentfill}%
\pgfsetlinewidth{0.481800pt}%
\definecolor{currentstroke}{rgb}{1.000000,1.000000,1.000000}%
\pgfsetstrokecolor{currentstroke}%
\pgfsetdash{}{0pt}%
\pgfpathmoveto{\pgfqpoint{1.582673in}{3.689374in}}%
\pgfpathcurveto{\pgfqpoint{1.593723in}{3.689374in}}{\pgfqpoint{1.604322in}{3.693764in}}{\pgfqpoint{1.612135in}{3.701578in}}%
\pgfpathcurveto{\pgfqpoint{1.619949in}{3.709391in}}{\pgfqpoint{1.624339in}{3.719990in}}{\pgfqpoint{1.624339in}{3.731040in}}%
\pgfpathcurveto{\pgfqpoint{1.624339in}{3.742091in}}{\pgfqpoint{1.619949in}{3.752690in}}{\pgfqpoint{1.612135in}{3.760503in}}%
\pgfpathcurveto{\pgfqpoint{1.604322in}{3.768317in}}{\pgfqpoint{1.593723in}{3.772707in}}{\pgfqpoint{1.582673in}{3.772707in}}%
\pgfpathcurveto{\pgfqpoint{1.571622in}{3.772707in}}{\pgfqpoint{1.561023in}{3.768317in}}{\pgfqpoint{1.553210in}{3.760503in}}%
\pgfpathcurveto{\pgfqpoint{1.545396in}{3.752690in}}{\pgfqpoint{1.541006in}{3.742091in}}{\pgfqpoint{1.541006in}{3.731040in}}%
\pgfpathcurveto{\pgfqpoint{1.541006in}{3.719990in}}{\pgfqpoint{1.545396in}{3.709391in}}{\pgfqpoint{1.553210in}{3.701578in}}%
\pgfpathcurveto{\pgfqpoint{1.561023in}{3.693764in}}{\pgfqpoint{1.571622in}{3.689374in}}{\pgfqpoint{1.582673in}{3.689374in}}%
\pgfpathclose%
\pgfusepath{stroke,fill}%
\end{pgfscope}%
\begin{pgfscope}%
\pgfpathrectangle{\pgfqpoint{0.481978in}{0.331635in}}{\pgfqpoint{9.300000in}{7.700000in}}%
\pgfusepath{clip}%
\pgfsetbuttcap%
\pgfsetroundjoin%
\definecolor{currentfill}{rgb}{0.631373,0.788235,0.956863}%
\pgfsetfillcolor{currentfill}%
\pgfsetlinewidth{0.481800pt}%
\definecolor{currentstroke}{rgb}{1.000000,1.000000,1.000000}%
\pgfsetstrokecolor{currentstroke}%
\pgfsetdash{}{0pt}%
\pgfpathmoveto{\pgfqpoint{7.082906in}{5.476502in}}%
\pgfpathcurveto{\pgfqpoint{7.093957in}{5.476502in}}{\pgfqpoint{7.104556in}{5.480892in}}{\pgfqpoint{7.112369in}{5.488706in}}%
\pgfpathcurveto{\pgfqpoint{7.120183in}{5.496519in}}{\pgfqpoint{7.124573in}{5.507118in}}{\pgfqpoint{7.124573in}{5.518169in}}%
\pgfpathcurveto{\pgfqpoint{7.124573in}{5.529219in}}{\pgfqpoint{7.120183in}{5.539818in}}{\pgfqpoint{7.112369in}{5.547631in}}%
\pgfpathcurveto{\pgfqpoint{7.104556in}{5.555445in}}{\pgfqpoint{7.093957in}{5.559835in}}{\pgfqpoint{7.082906in}{5.559835in}}%
\pgfpathcurveto{\pgfqpoint{7.071856in}{5.559835in}}{\pgfqpoint{7.061257in}{5.555445in}}{\pgfqpoint{7.053444in}{5.547631in}}%
\pgfpathcurveto{\pgfqpoint{7.045630in}{5.539818in}}{\pgfqpoint{7.041240in}{5.529219in}}{\pgfqpoint{7.041240in}{5.518169in}}%
\pgfpathcurveto{\pgfqpoint{7.041240in}{5.507118in}}{\pgfqpoint{7.045630in}{5.496519in}}{\pgfqpoint{7.053444in}{5.488706in}}%
\pgfpathcurveto{\pgfqpoint{7.061257in}{5.480892in}}{\pgfqpoint{7.071856in}{5.476502in}}{\pgfqpoint{7.082906in}{5.476502in}}%
\pgfpathclose%
\pgfusepath{stroke,fill}%
\end{pgfscope}%
\begin{pgfscope}%
\pgfpathrectangle{\pgfqpoint{0.481978in}{0.331635in}}{\pgfqpoint{9.300000in}{7.700000in}}%
\pgfusepath{clip}%
\pgfsetbuttcap%
\pgfsetroundjoin%
\definecolor{currentfill}{rgb}{0.631373,0.788235,0.956863}%
\pgfsetfillcolor{currentfill}%
\pgfsetlinewidth{0.481800pt}%
\definecolor{currentstroke}{rgb}{1.000000,1.000000,1.000000}%
\pgfsetstrokecolor{currentstroke}%
\pgfsetdash{}{0pt}%
\pgfpathmoveto{\pgfqpoint{2.863295in}{2.519612in}}%
\pgfpathcurveto{\pgfqpoint{2.874345in}{2.519612in}}{\pgfqpoint{2.884944in}{2.524002in}}{\pgfqpoint{2.892758in}{2.531816in}}%
\pgfpathcurveto{\pgfqpoint{2.900572in}{2.539629in}}{\pgfqpoint{2.904962in}{2.550228in}}{\pgfqpoint{2.904962in}{2.561278in}}%
\pgfpathcurveto{\pgfqpoint{2.904962in}{2.572329in}}{\pgfqpoint{2.900572in}{2.582928in}}{\pgfqpoint{2.892758in}{2.590741in}}%
\pgfpathcurveto{\pgfqpoint{2.884944in}{2.598555in}}{\pgfqpoint{2.874345in}{2.602945in}}{\pgfqpoint{2.863295in}{2.602945in}}%
\pgfpathcurveto{\pgfqpoint{2.852245in}{2.602945in}}{\pgfqpoint{2.841646in}{2.598555in}}{\pgfqpoint{2.833833in}{2.590741in}}%
\pgfpathcurveto{\pgfqpoint{2.826019in}{2.582928in}}{\pgfqpoint{2.821629in}{2.572329in}}{\pgfqpoint{2.821629in}{2.561278in}}%
\pgfpathcurveto{\pgfqpoint{2.821629in}{2.550228in}}{\pgfqpoint{2.826019in}{2.539629in}}{\pgfqpoint{2.833833in}{2.531816in}}%
\pgfpathcurveto{\pgfqpoint{2.841646in}{2.524002in}}{\pgfqpoint{2.852245in}{2.519612in}}{\pgfqpoint{2.863295in}{2.519612in}}%
\pgfpathclose%
\pgfusepath{stroke,fill}%
\end{pgfscope}%
\begin{pgfscope}%
\pgfpathrectangle{\pgfqpoint{0.481978in}{0.331635in}}{\pgfqpoint{9.300000in}{7.700000in}}%
\pgfusepath{clip}%
\pgfsetbuttcap%
\pgfsetroundjoin%
\definecolor{currentfill}{rgb}{0.631373,0.788235,0.956863}%
\pgfsetfillcolor{currentfill}%
\pgfsetlinewidth{0.481800pt}%
\definecolor{currentstroke}{rgb}{1.000000,1.000000,1.000000}%
\pgfsetstrokecolor{currentstroke}%
\pgfsetdash{}{0pt}%
\pgfpathmoveto{\pgfqpoint{7.718781in}{2.730738in}}%
\pgfpathcurveto{\pgfqpoint{7.729831in}{2.730738in}}{\pgfqpoint{7.740431in}{2.735128in}}{\pgfqpoint{7.748244in}{2.742942in}}%
\pgfpathcurveto{\pgfqpoint{7.756058in}{2.750756in}}{\pgfqpoint{7.760448in}{2.761355in}}{\pgfqpoint{7.760448in}{2.772405in}}%
\pgfpathcurveto{\pgfqpoint{7.760448in}{2.783455in}}{\pgfqpoint{7.756058in}{2.794054in}}{\pgfqpoint{7.748244in}{2.801867in}}%
\pgfpathcurveto{\pgfqpoint{7.740431in}{2.809681in}}{\pgfqpoint{7.729831in}{2.814071in}}{\pgfqpoint{7.718781in}{2.814071in}}%
\pgfpathcurveto{\pgfqpoint{7.707731in}{2.814071in}}{\pgfqpoint{7.697132in}{2.809681in}}{\pgfqpoint{7.689319in}{2.801867in}}%
\pgfpathcurveto{\pgfqpoint{7.681505in}{2.794054in}}{\pgfqpoint{7.677115in}{2.783455in}}{\pgfqpoint{7.677115in}{2.772405in}}%
\pgfpathcurveto{\pgfqpoint{7.677115in}{2.761355in}}{\pgfqpoint{7.681505in}{2.750756in}}{\pgfqpoint{7.689319in}{2.742942in}}%
\pgfpathcurveto{\pgfqpoint{7.697132in}{2.735128in}}{\pgfqpoint{7.707731in}{2.730738in}}{\pgfqpoint{7.718781in}{2.730738in}}%
\pgfpathclose%
\pgfusepath{stroke,fill}%
\end{pgfscope}%
\begin{pgfscope}%
\pgfpathrectangle{\pgfqpoint{0.481978in}{0.331635in}}{\pgfqpoint{9.300000in}{7.700000in}}%
\pgfusepath{clip}%
\pgfsetbuttcap%
\pgfsetroundjoin%
\definecolor{currentfill}{rgb}{0.631373,0.788235,0.956863}%
\pgfsetfillcolor{currentfill}%
\pgfsetlinewidth{0.481800pt}%
\definecolor{currentstroke}{rgb}{1.000000,1.000000,1.000000}%
\pgfsetstrokecolor{currentstroke}%
\pgfsetdash{}{0pt}%
\pgfpathmoveto{\pgfqpoint{3.697003in}{3.898380in}}%
\pgfpathcurveto{\pgfqpoint{3.708053in}{3.898380in}}{\pgfqpoint{3.718652in}{3.902770in}}{\pgfqpoint{3.726465in}{3.910584in}}%
\pgfpathcurveto{\pgfqpoint{3.734279in}{3.918397in}}{\pgfqpoint{3.738669in}{3.928996in}}{\pgfqpoint{3.738669in}{3.940047in}}%
\pgfpathcurveto{\pgfqpoint{3.738669in}{3.951097in}}{\pgfqpoint{3.734279in}{3.961696in}}{\pgfqpoint{3.726465in}{3.969509in}}%
\pgfpathcurveto{\pgfqpoint{3.718652in}{3.977323in}}{\pgfqpoint{3.708053in}{3.981713in}}{\pgfqpoint{3.697003in}{3.981713in}}%
\pgfpathcurveto{\pgfqpoint{3.685953in}{3.981713in}}{\pgfqpoint{3.675354in}{3.977323in}}{\pgfqpoint{3.667540in}{3.969509in}}%
\pgfpathcurveto{\pgfqpoint{3.659726in}{3.961696in}}{\pgfqpoint{3.655336in}{3.951097in}}{\pgfqpoint{3.655336in}{3.940047in}}%
\pgfpathcurveto{\pgfqpoint{3.655336in}{3.928996in}}{\pgfqpoint{3.659726in}{3.918397in}}{\pgfqpoint{3.667540in}{3.910584in}}%
\pgfpathcurveto{\pgfqpoint{3.675354in}{3.902770in}}{\pgfqpoint{3.685953in}{3.898380in}}{\pgfqpoint{3.697003in}{3.898380in}}%
\pgfpathclose%
\pgfusepath{stroke,fill}%
\end{pgfscope}%
\begin{pgfscope}%
\pgfpathrectangle{\pgfqpoint{0.481978in}{0.331635in}}{\pgfqpoint{9.300000in}{7.700000in}}%
\pgfusepath{clip}%
\pgfsetbuttcap%
\pgfsetroundjoin%
\definecolor{currentfill}{rgb}{0.631373,0.788235,0.956863}%
\pgfsetfillcolor{currentfill}%
\pgfsetlinewidth{0.481800pt}%
\definecolor{currentstroke}{rgb}{1.000000,1.000000,1.000000}%
\pgfsetstrokecolor{currentstroke}%
\pgfsetdash{}{0pt}%
\pgfpathmoveto{\pgfqpoint{6.644743in}{0.639968in}}%
\pgfpathcurveto{\pgfqpoint{6.655793in}{0.639968in}}{\pgfqpoint{6.666392in}{0.644359in}}{\pgfqpoint{6.674206in}{0.652172in}}%
\pgfpathcurveto{\pgfqpoint{6.682019in}{0.659986in}}{\pgfqpoint{6.686409in}{0.670585in}}{\pgfqpoint{6.686409in}{0.681635in}}%
\pgfpathcurveto{\pgfqpoint{6.686409in}{0.692685in}}{\pgfqpoint{6.682019in}{0.703284in}}{\pgfqpoint{6.674206in}{0.711098in}}%
\pgfpathcurveto{\pgfqpoint{6.666392in}{0.718911in}}{\pgfqpoint{6.655793in}{0.723302in}}{\pgfqpoint{6.644743in}{0.723302in}}%
\pgfpathcurveto{\pgfqpoint{6.633693in}{0.723302in}}{\pgfqpoint{6.623094in}{0.718911in}}{\pgfqpoint{6.615280in}{0.711098in}}%
\pgfpathcurveto{\pgfqpoint{6.607466in}{0.703284in}}{\pgfqpoint{6.603076in}{0.692685in}}{\pgfqpoint{6.603076in}{0.681635in}}%
\pgfpathcurveto{\pgfqpoint{6.603076in}{0.670585in}}{\pgfqpoint{6.607466in}{0.659986in}}{\pgfqpoint{6.615280in}{0.652172in}}%
\pgfpathcurveto{\pgfqpoint{6.623094in}{0.644359in}}{\pgfqpoint{6.633693in}{0.639968in}}{\pgfqpoint{6.644743in}{0.639968in}}%
\pgfpathclose%
\pgfusepath{stroke,fill}%
\end{pgfscope}%
\begin{pgfscope}%
\pgfpathrectangle{\pgfqpoint{0.481978in}{0.331635in}}{\pgfqpoint{9.300000in}{7.700000in}}%
\pgfusepath{clip}%
\pgfsetbuttcap%
\pgfsetroundjoin%
\definecolor{currentfill}{rgb}{0.631373,0.788235,0.956863}%
\pgfsetfillcolor{currentfill}%
\pgfsetlinewidth{0.481800pt}%
\definecolor{currentstroke}{rgb}{1.000000,1.000000,1.000000}%
\pgfsetstrokecolor{currentstroke}%
\pgfsetdash{}{0pt}%
\pgfpathmoveto{\pgfqpoint{2.360953in}{1.944615in}}%
\pgfpathcurveto{\pgfqpoint{2.372003in}{1.944615in}}{\pgfqpoint{2.382602in}{1.949006in}}{\pgfqpoint{2.390416in}{1.956819in}}%
\pgfpathcurveto{\pgfqpoint{2.398230in}{1.964633in}}{\pgfqpoint{2.402620in}{1.975232in}}{\pgfqpoint{2.402620in}{1.986282in}}%
\pgfpathcurveto{\pgfqpoint{2.402620in}{1.997332in}}{\pgfqpoint{2.398230in}{2.007931in}}{\pgfqpoint{2.390416in}{2.015745in}}%
\pgfpathcurveto{\pgfqpoint{2.382602in}{2.023559in}}{\pgfqpoint{2.372003in}{2.027949in}}{\pgfqpoint{2.360953in}{2.027949in}}%
\pgfpathcurveto{\pgfqpoint{2.349903in}{2.027949in}}{\pgfqpoint{2.339304in}{2.023559in}}{\pgfqpoint{2.331490in}{2.015745in}}%
\pgfpathcurveto{\pgfqpoint{2.323677in}{2.007931in}}{\pgfqpoint{2.319286in}{1.997332in}}{\pgfqpoint{2.319286in}{1.986282in}}%
\pgfpathcurveto{\pgfqpoint{2.319286in}{1.975232in}}{\pgfqpoint{2.323677in}{1.964633in}}{\pgfqpoint{2.331490in}{1.956819in}}%
\pgfpathcurveto{\pgfqpoint{2.339304in}{1.949006in}}{\pgfqpoint{2.349903in}{1.944615in}}{\pgfqpoint{2.360953in}{1.944615in}}%
\pgfpathclose%
\pgfusepath{stroke,fill}%
\end{pgfscope}%
\begin{pgfscope}%
\pgfpathrectangle{\pgfqpoint{0.481978in}{0.331635in}}{\pgfqpoint{9.300000in}{7.700000in}}%
\pgfusepath{clip}%
\pgfsetbuttcap%
\pgfsetroundjoin%
\definecolor{currentfill}{rgb}{0.631373,0.788235,0.956863}%
\pgfsetfillcolor{currentfill}%
\pgfsetlinewidth{0.481800pt}%
\definecolor{currentstroke}{rgb}{1.000000,1.000000,1.000000}%
\pgfsetstrokecolor{currentstroke}%
\pgfsetdash{}{0pt}%
\pgfpathmoveto{\pgfqpoint{3.463144in}{2.136651in}}%
\pgfpathcurveto{\pgfqpoint{3.474194in}{2.136651in}}{\pgfqpoint{3.484793in}{2.141041in}}{\pgfqpoint{3.492606in}{2.148855in}}%
\pgfpathcurveto{\pgfqpoint{3.500420in}{2.156668in}}{\pgfqpoint{3.504810in}{2.167267in}}{\pgfqpoint{3.504810in}{2.178317in}}%
\pgfpathcurveto{\pgfqpoint{3.504810in}{2.189368in}}{\pgfqpoint{3.500420in}{2.199967in}}{\pgfqpoint{3.492606in}{2.207780in}}%
\pgfpathcurveto{\pgfqpoint{3.484793in}{2.215594in}}{\pgfqpoint{3.474194in}{2.219984in}}{\pgfqpoint{3.463144in}{2.219984in}}%
\pgfpathcurveto{\pgfqpoint{3.452094in}{2.219984in}}{\pgfqpoint{3.441495in}{2.215594in}}{\pgfqpoint{3.433681in}{2.207780in}}%
\pgfpathcurveto{\pgfqpoint{3.425867in}{2.199967in}}{\pgfqpoint{3.421477in}{2.189368in}}{\pgfqpoint{3.421477in}{2.178317in}}%
\pgfpathcurveto{\pgfqpoint{3.421477in}{2.167267in}}{\pgfqpoint{3.425867in}{2.156668in}}{\pgfqpoint{3.433681in}{2.148855in}}%
\pgfpathcurveto{\pgfqpoint{3.441495in}{2.141041in}}{\pgfqpoint{3.452094in}{2.136651in}}{\pgfqpoint{3.463144in}{2.136651in}}%
\pgfpathclose%
\pgfusepath{stroke,fill}%
\end{pgfscope}%
\begin{pgfscope}%
\pgfpathrectangle{\pgfqpoint{0.481978in}{0.331635in}}{\pgfqpoint{9.300000in}{7.700000in}}%
\pgfusepath{clip}%
\pgfsetbuttcap%
\pgfsetroundjoin%
\definecolor{currentfill}{rgb}{0.631373,0.788235,0.956863}%
\pgfsetfillcolor{currentfill}%
\pgfsetlinewidth{0.481800pt}%
\definecolor{currentstroke}{rgb}{1.000000,1.000000,1.000000}%
\pgfsetstrokecolor{currentstroke}%
\pgfsetdash{}{0pt}%
\pgfpathmoveto{\pgfqpoint{8.681814in}{2.091340in}}%
\pgfpathcurveto{\pgfqpoint{8.692865in}{2.091340in}}{\pgfqpoint{8.703464in}{2.095730in}}{\pgfqpoint{8.711277in}{2.103544in}}%
\pgfpathcurveto{\pgfqpoint{8.719091in}{2.111357in}}{\pgfqpoint{8.723481in}{2.121956in}}{\pgfqpoint{8.723481in}{2.133007in}}%
\pgfpathcurveto{\pgfqpoint{8.723481in}{2.144057in}}{\pgfqpoint{8.719091in}{2.154656in}}{\pgfqpoint{8.711277in}{2.162469in}}%
\pgfpathcurveto{\pgfqpoint{8.703464in}{2.170283in}}{\pgfqpoint{8.692865in}{2.174673in}}{\pgfqpoint{8.681814in}{2.174673in}}%
\pgfpathcurveto{\pgfqpoint{8.670764in}{2.174673in}}{\pgfqpoint{8.660165in}{2.170283in}}{\pgfqpoint{8.652352in}{2.162469in}}%
\pgfpathcurveto{\pgfqpoint{8.644538in}{2.154656in}}{\pgfqpoint{8.640148in}{2.144057in}}{\pgfqpoint{8.640148in}{2.133007in}}%
\pgfpathcurveto{\pgfqpoint{8.640148in}{2.121956in}}{\pgfqpoint{8.644538in}{2.111357in}}{\pgfqpoint{8.652352in}{2.103544in}}%
\pgfpathcurveto{\pgfqpoint{8.660165in}{2.095730in}}{\pgfqpoint{8.670764in}{2.091340in}}{\pgfqpoint{8.681814in}{2.091340in}}%
\pgfpathclose%
\pgfusepath{stroke,fill}%
\end{pgfscope}%
\begin{pgfscope}%
\pgfpathrectangle{\pgfqpoint{0.481978in}{0.331635in}}{\pgfqpoint{9.300000in}{7.700000in}}%
\pgfusepath{clip}%
\pgfsetbuttcap%
\pgfsetroundjoin%
\definecolor{currentfill}{rgb}{0.631373,0.788235,0.956863}%
\pgfsetfillcolor{currentfill}%
\pgfsetlinewidth{0.481800pt}%
\definecolor{currentstroke}{rgb}{1.000000,1.000000,1.000000}%
\pgfsetstrokecolor{currentstroke}%
\pgfsetdash{}{0pt}%
\pgfpathmoveto{\pgfqpoint{5.188038in}{5.085722in}}%
\pgfpathcurveto{\pgfqpoint{5.199089in}{5.085722in}}{\pgfqpoint{5.209688in}{5.090113in}}{\pgfqpoint{5.217501in}{5.097926in}}%
\pgfpathcurveto{\pgfqpoint{5.225315in}{5.105740in}}{\pgfqpoint{5.229705in}{5.116339in}}{\pgfqpoint{5.229705in}{5.127389in}}%
\pgfpathcurveto{\pgfqpoint{5.229705in}{5.138439in}}{\pgfqpoint{5.225315in}{5.149038in}}{\pgfqpoint{5.217501in}{5.156852in}}%
\pgfpathcurveto{\pgfqpoint{5.209688in}{5.164666in}}{\pgfqpoint{5.199089in}{5.169056in}}{\pgfqpoint{5.188038in}{5.169056in}}%
\pgfpathcurveto{\pgfqpoint{5.176988in}{5.169056in}}{\pgfqpoint{5.166389in}{5.164666in}}{\pgfqpoint{5.158576in}{5.156852in}}%
\pgfpathcurveto{\pgfqpoint{5.150762in}{5.149038in}}{\pgfqpoint{5.146372in}{5.138439in}}{\pgfqpoint{5.146372in}{5.127389in}}%
\pgfpathcurveto{\pgfqpoint{5.146372in}{5.116339in}}{\pgfqpoint{5.150762in}{5.105740in}}{\pgfqpoint{5.158576in}{5.097926in}}%
\pgfpathcurveto{\pgfqpoint{5.166389in}{5.090113in}}{\pgfqpoint{5.176988in}{5.085722in}}{\pgfqpoint{5.188038in}{5.085722in}}%
\pgfpathclose%
\pgfusepath{stroke,fill}%
\end{pgfscope}%
\begin{pgfscope}%
\pgfpathrectangle{\pgfqpoint{0.481978in}{0.331635in}}{\pgfqpoint{9.300000in}{7.700000in}}%
\pgfusepath{clip}%
\pgfsetbuttcap%
\pgfsetroundjoin%
\definecolor{currentfill}{rgb}{0.631373,0.788235,0.956863}%
\pgfsetfillcolor{currentfill}%
\pgfsetlinewidth{0.481800pt}%
\definecolor{currentstroke}{rgb}{1.000000,1.000000,1.000000}%
\pgfsetstrokecolor{currentstroke}%
\pgfsetdash{}{0pt}%
\pgfpathmoveto{\pgfqpoint{2.109676in}{2.935946in}}%
\pgfpathcurveto{\pgfqpoint{2.120726in}{2.935946in}}{\pgfqpoint{2.131325in}{2.940336in}}{\pgfqpoint{2.139139in}{2.948150in}}%
\pgfpathcurveto{\pgfqpoint{2.146952in}{2.955964in}}{\pgfqpoint{2.151342in}{2.966563in}}{\pgfqpoint{2.151342in}{2.977613in}}%
\pgfpathcurveto{\pgfqpoint{2.151342in}{2.988663in}}{\pgfqpoint{2.146952in}{2.999262in}}{\pgfqpoint{2.139139in}{3.007076in}}%
\pgfpathcurveto{\pgfqpoint{2.131325in}{3.014889in}}{\pgfqpoint{2.120726in}{3.019279in}}{\pgfqpoint{2.109676in}{3.019279in}}%
\pgfpathcurveto{\pgfqpoint{2.098626in}{3.019279in}}{\pgfqpoint{2.088027in}{3.014889in}}{\pgfqpoint{2.080213in}{3.007076in}}%
\pgfpathcurveto{\pgfqpoint{2.072399in}{2.999262in}}{\pgfqpoint{2.068009in}{2.988663in}}{\pgfqpoint{2.068009in}{2.977613in}}%
\pgfpathcurveto{\pgfqpoint{2.068009in}{2.966563in}}{\pgfqpoint{2.072399in}{2.955964in}}{\pgfqpoint{2.080213in}{2.948150in}}%
\pgfpathcurveto{\pgfqpoint{2.088027in}{2.940336in}}{\pgfqpoint{2.098626in}{2.935946in}}{\pgfqpoint{2.109676in}{2.935946in}}%
\pgfpathclose%
\pgfusepath{stroke,fill}%
\end{pgfscope}%
\begin{pgfscope}%
\pgfpathrectangle{\pgfqpoint{0.481978in}{0.331635in}}{\pgfqpoint{9.300000in}{7.700000in}}%
\pgfusepath{clip}%
\pgfsetbuttcap%
\pgfsetroundjoin%
\definecolor{currentfill}{rgb}{0.631373,0.788235,0.956863}%
\pgfsetfillcolor{currentfill}%
\pgfsetlinewidth{0.481800pt}%
\definecolor{currentstroke}{rgb}{1.000000,1.000000,1.000000}%
\pgfsetstrokecolor{currentstroke}%
\pgfsetdash{}{0pt}%
\pgfpathmoveto{\pgfqpoint{0.984432in}{4.164418in}}%
\pgfpathcurveto{\pgfqpoint{0.995482in}{4.164418in}}{\pgfqpoint{1.006081in}{4.168808in}}{\pgfqpoint{1.013895in}{4.176622in}}%
\pgfpathcurveto{\pgfqpoint{1.021709in}{4.184436in}}{\pgfqpoint{1.026099in}{4.195035in}}{\pgfqpoint{1.026099in}{4.206085in}}%
\pgfpathcurveto{\pgfqpoint{1.026099in}{4.217135in}}{\pgfqpoint{1.021709in}{4.227734in}}{\pgfqpoint{1.013895in}{4.235548in}}%
\pgfpathcurveto{\pgfqpoint{1.006081in}{4.243361in}}{\pgfqpoint{0.995482in}{4.247751in}}{\pgfqpoint{0.984432in}{4.247751in}}%
\pgfpathcurveto{\pgfqpoint{0.973382in}{4.247751in}}{\pgfqpoint{0.962783in}{4.243361in}}{\pgfqpoint{0.954970in}{4.235548in}}%
\pgfpathcurveto{\pgfqpoint{0.947156in}{4.227734in}}{\pgfqpoint{0.942766in}{4.217135in}}{\pgfqpoint{0.942766in}{4.206085in}}%
\pgfpathcurveto{\pgfqpoint{0.942766in}{4.195035in}}{\pgfqpoint{0.947156in}{4.184436in}}{\pgfqpoint{0.954970in}{4.176622in}}%
\pgfpathcurveto{\pgfqpoint{0.962783in}{4.168808in}}{\pgfqpoint{0.973382in}{4.164418in}}{\pgfqpoint{0.984432in}{4.164418in}}%
\pgfpathclose%
\pgfusepath{stroke,fill}%
\end{pgfscope}%
\begin{pgfscope}%
\pgfpathrectangle{\pgfqpoint{0.481978in}{0.331635in}}{\pgfqpoint{9.300000in}{7.700000in}}%
\pgfusepath{clip}%
\pgfsetbuttcap%
\pgfsetroundjoin%
\definecolor{currentfill}{rgb}{1.000000,0.705882,0.509804}%
\pgfsetfillcolor{currentfill}%
\pgfsetlinewidth{0.481800pt}%
\definecolor{currentstroke}{rgb}{1.000000,1.000000,1.000000}%
\pgfsetstrokecolor{currentstroke}%
\pgfsetdash{}{0pt}%
\pgfpathmoveto{\pgfqpoint{6.044554in}{2.136058in}}%
\pgfpathcurveto{\pgfqpoint{6.055604in}{2.136058in}}{\pgfqpoint{6.066203in}{2.140448in}}{\pgfqpoint{6.074017in}{2.148261in}}%
\pgfpathcurveto{\pgfqpoint{6.081831in}{2.156075in}}{\pgfqpoint{6.086221in}{2.166674in}}{\pgfqpoint{6.086221in}{2.177724in}}%
\pgfpathcurveto{\pgfqpoint{6.086221in}{2.188774in}}{\pgfqpoint{6.081831in}{2.199373in}}{\pgfqpoint{6.074017in}{2.207187in}}%
\pgfpathcurveto{\pgfqpoint{6.066203in}{2.215001in}}{\pgfqpoint{6.055604in}{2.219391in}}{\pgfqpoint{6.044554in}{2.219391in}}%
\pgfpathcurveto{\pgfqpoint{6.033504in}{2.219391in}}{\pgfqpoint{6.022905in}{2.215001in}}{\pgfqpoint{6.015091in}{2.207187in}}%
\pgfpathcurveto{\pgfqpoint{6.007278in}{2.199373in}}{\pgfqpoint{6.002887in}{2.188774in}}{\pgfqpoint{6.002887in}{2.177724in}}%
\pgfpathcurveto{\pgfqpoint{6.002887in}{2.166674in}}{\pgfqpoint{6.007278in}{2.156075in}}{\pgfqpoint{6.015091in}{2.148261in}}%
\pgfpathcurveto{\pgfqpoint{6.022905in}{2.140448in}}{\pgfqpoint{6.033504in}{2.136058in}}{\pgfqpoint{6.044554in}{2.136058in}}%
\pgfpathclose%
\pgfusepath{stroke,fill}%
\end{pgfscope}%
\begin{pgfscope}%
\pgfpathrectangle{\pgfqpoint{0.481978in}{0.331635in}}{\pgfqpoint{9.300000in}{7.700000in}}%
\pgfusepath{clip}%
\pgfsetbuttcap%
\pgfsetroundjoin%
\definecolor{currentfill}{rgb}{1.000000,0.705882,0.509804}%
\pgfsetfillcolor{currentfill}%
\pgfsetlinewidth{0.481800pt}%
\definecolor{currentstroke}{rgb}{1.000000,1.000000,1.000000}%
\pgfsetstrokecolor{currentstroke}%
\pgfsetdash{}{0pt}%
\pgfpathmoveto{\pgfqpoint{4.376435in}{1.830640in}}%
\pgfpathcurveto{\pgfqpoint{4.387485in}{1.830640in}}{\pgfqpoint{4.398084in}{1.835030in}}{\pgfqpoint{4.405897in}{1.842844in}}%
\pgfpathcurveto{\pgfqpoint{4.413711in}{1.850658in}}{\pgfqpoint{4.418101in}{1.861257in}}{\pgfqpoint{4.418101in}{1.872307in}}%
\pgfpathcurveto{\pgfqpoint{4.418101in}{1.883357in}}{\pgfqpoint{4.413711in}{1.893956in}}{\pgfqpoint{4.405897in}{1.901770in}}%
\pgfpathcurveto{\pgfqpoint{4.398084in}{1.909583in}}{\pgfqpoint{4.387485in}{1.913974in}}{\pgfqpoint{4.376435in}{1.913974in}}%
\pgfpathcurveto{\pgfqpoint{4.365385in}{1.913974in}}{\pgfqpoint{4.354785in}{1.909583in}}{\pgfqpoint{4.346972in}{1.901770in}}%
\pgfpathcurveto{\pgfqpoint{4.339158in}{1.893956in}}{\pgfqpoint{4.334768in}{1.883357in}}{\pgfqpoint{4.334768in}{1.872307in}}%
\pgfpathcurveto{\pgfqpoint{4.334768in}{1.861257in}}{\pgfqpoint{4.339158in}{1.850658in}}{\pgfqpoint{4.346972in}{1.842844in}}%
\pgfpathcurveto{\pgfqpoint{4.354785in}{1.835030in}}{\pgfqpoint{4.365385in}{1.830640in}}{\pgfqpoint{4.376435in}{1.830640in}}%
\pgfpathclose%
\pgfusepath{stroke,fill}%
\end{pgfscope}%
\begin{pgfscope}%
\pgfpathrectangle{\pgfqpoint{0.481978in}{0.331635in}}{\pgfqpoint{9.300000in}{7.700000in}}%
\pgfusepath{clip}%
\pgfsetbuttcap%
\pgfsetroundjoin%
\definecolor{currentfill}{rgb}{1.000000,0.705882,0.509804}%
\pgfsetfillcolor{currentfill}%
\pgfsetlinewidth{0.481800pt}%
\definecolor{currentstroke}{rgb}{1.000000,1.000000,1.000000}%
\pgfsetstrokecolor{currentstroke}%
\pgfsetdash{}{0pt}%
\pgfpathmoveto{\pgfqpoint{8.943163in}{5.720288in}}%
\pgfpathcurveto{\pgfqpoint{8.954213in}{5.720288in}}{\pgfqpoint{8.964812in}{5.724678in}}{\pgfqpoint{8.972626in}{5.732492in}}%
\pgfpathcurveto{\pgfqpoint{8.980439in}{5.740305in}}{\pgfqpoint{8.984830in}{5.750904in}}{\pgfqpoint{8.984830in}{5.761954in}}%
\pgfpathcurveto{\pgfqpoint{8.984830in}{5.773005in}}{\pgfqpoint{8.980439in}{5.783604in}}{\pgfqpoint{8.972626in}{5.791417in}}%
\pgfpathcurveto{\pgfqpoint{8.964812in}{5.799231in}}{\pgfqpoint{8.954213in}{5.803621in}}{\pgfqpoint{8.943163in}{5.803621in}}%
\pgfpathcurveto{\pgfqpoint{8.932113in}{5.803621in}}{\pgfqpoint{8.921514in}{5.799231in}}{\pgfqpoint{8.913700in}{5.791417in}}%
\pgfpathcurveto{\pgfqpoint{8.905887in}{5.783604in}}{\pgfqpoint{8.901496in}{5.773005in}}{\pgfqpoint{8.901496in}{5.761954in}}%
\pgfpathcurveto{\pgfqpoint{8.901496in}{5.750904in}}{\pgfqpoint{8.905887in}{5.740305in}}{\pgfqpoint{8.913700in}{5.732492in}}%
\pgfpathcurveto{\pgfqpoint{8.921514in}{5.724678in}}{\pgfqpoint{8.932113in}{5.720288in}}{\pgfqpoint{8.943163in}{5.720288in}}%
\pgfpathclose%
\pgfusepath{stroke,fill}%
\end{pgfscope}%
\begin{pgfscope}%
\pgfpathrectangle{\pgfqpoint{0.481978in}{0.331635in}}{\pgfqpoint{9.300000in}{7.700000in}}%
\pgfusepath{clip}%
\pgfsetbuttcap%
\pgfsetroundjoin%
\definecolor{currentfill}{rgb}{1.000000,0.705882,0.509804}%
\pgfsetfillcolor{currentfill}%
\pgfsetlinewidth{0.481800pt}%
\definecolor{currentstroke}{rgb}{1.000000,1.000000,1.000000}%
\pgfsetstrokecolor{currentstroke}%
\pgfsetdash{}{0pt}%
\pgfpathmoveto{\pgfqpoint{5.919865in}{2.990099in}}%
\pgfpathcurveto{\pgfqpoint{5.930915in}{2.990099in}}{\pgfqpoint{5.941514in}{2.994489in}}{\pgfqpoint{5.949328in}{3.002303in}}%
\pgfpathcurveto{\pgfqpoint{5.957141in}{3.010117in}}{\pgfqpoint{5.961532in}{3.020716in}}{\pgfqpoint{5.961532in}{3.031766in}}%
\pgfpathcurveto{\pgfqpoint{5.961532in}{3.042816in}}{\pgfqpoint{5.957141in}{3.053415in}}{\pgfqpoint{5.949328in}{3.061229in}}%
\pgfpathcurveto{\pgfqpoint{5.941514in}{3.069042in}}{\pgfqpoint{5.930915in}{3.073432in}}{\pgfqpoint{5.919865in}{3.073432in}}%
\pgfpathcurveto{\pgfqpoint{5.908815in}{3.073432in}}{\pgfqpoint{5.898216in}{3.069042in}}{\pgfqpoint{5.890402in}{3.061229in}}%
\pgfpathcurveto{\pgfqpoint{5.882589in}{3.053415in}}{\pgfqpoint{5.878198in}{3.042816in}}{\pgfqpoint{5.878198in}{3.031766in}}%
\pgfpathcurveto{\pgfqpoint{5.878198in}{3.020716in}}{\pgfqpoint{5.882589in}{3.010117in}}{\pgfqpoint{5.890402in}{3.002303in}}%
\pgfpathcurveto{\pgfqpoint{5.898216in}{2.994489in}}{\pgfqpoint{5.908815in}{2.990099in}}{\pgfqpoint{5.919865in}{2.990099in}}%
\pgfpathclose%
\pgfusepath{stroke,fill}%
\end{pgfscope}%
\begin{pgfscope}%
\pgfpathrectangle{\pgfqpoint{0.481978in}{0.331635in}}{\pgfqpoint{9.300000in}{7.700000in}}%
\pgfusepath{clip}%
\pgfsetbuttcap%
\pgfsetroundjoin%
\definecolor{currentfill}{rgb}{1.000000,0.705882,0.509804}%
\pgfsetfillcolor{currentfill}%
\pgfsetlinewidth{0.481800pt}%
\definecolor{currentstroke}{rgb}{1.000000,1.000000,1.000000}%
\pgfsetstrokecolor{currentstroke}%
\pgfsetdash{}{0pt}%
\pgfpathmoveto{\pgfqpoint{5.035065in}{3.870271in}}%
\pgfpathcurveto{\pgfqpoint{5.046115in}{3.870271in}}{\pgfqpoint{5.056714in}{3.874661in}}{\pgfqpoint{5.064528in}{3.882475in}}%
\pgfpathcurveto{\pgfqpoint{5.072341in}{3.890289in}}{\pgfqpoint{5.076731in}{3.900888in}}{\pgfqpoint{5.076731in}{3.911938in}}%
\pgfpathcurveto{\pgfqpoint{5.076731in}{3.922988in}}{\pgfqpoint{5.072341in}{3.933587in}}{\pgfqpoint{5.064528in}{3.941401in}}%
\pgfpathcurveto{\pgfqpoint{5.056714in}{3.949214in}}{\pgfqpoint{5.046115in}{3.953605in}}{\pgfqpoint{5.035065in}{3.953605in}}%
\pgfpathcurveto{\pgfqpoint{5.024015in}{3.953605in}}{\pgfqpoint{5.013416in}{3.949214in}}{\pgfqpoint{5.005602in}{3.941401in}}%
\pgfpathcurveto{\pgfqpoint{4.997788in}{3.933587in}}{\pgfqpoint{4.993398in}{3.922988in}}{\pgfqpoint{4.993398in}{3.911938in}}%
\pgfpathcurveto{\pgfqpoint{4.993398in}{3.900888in}}{\pgfqpoint{4.997788in}{3.890289in}}{\pgfqpoint{5.005602in}{3.882475in}}%
\pgfpathcurveto{\pgfqpoint{5.013416in}{3.874661in}}{\pgfqpoint{5.024015in}{3.870271in}}{\pgfqpoint{5.035065in}{3.870271in}}%
\pgfpathclose%
\pgfusepath{stroke,fill}%
\end{pgfscope}%
\begin{pgfscope}%
\pgfpathrectangle{\pgfqpoint{0.481978in}{0.331635in}}{\pgfqpoint{9.300000in}{7.700000in}}%
\pgfusepath{clip}%
\pgfsetbuttcap%
\pgfsetroundjoin%
\definecolor{currentfill}{rgb}{1.000000,0.705882,0.509804}%
\pgfsetfillcolor{currentfill}%
\pgfsetlinewidth{0.481800pt}%
\definecolor{currentstroke}{rgb}{1.000000,1.000000,1.000000}%
\pgfsetstrokecolor{currentstroke}%
\pgfsetdash{}{0pt}%
\pgfpathmoveto{\pgfqpoint{7.526077in}{7.039392in}}%
\pgfpathcurveto{\pgfqpoint{7.537127in}{7.039392in}}{\pgfqpoint{7.547726in}{7.043782in}}{\pgfqpoint{7.555540in}{7.051596in}}%
\pgfpathcurveto{\pgfqpoint{7.563353in}{7.059409in}}{\pgfqpoint{7.567744in}{7.070008in}}{\pgfqpoint{7.567744in}{7.081058in}}%
\pgfpathcurveto{\pgfqpoint{7.567744in}{7.092109in}}{\pgfqpoint{7.563353in}{7.102708in}}{\pgfqpoint{7.555540in}{7.110521in}}%
\pgfpathcurveto{\pgfqpoint{7.547726in}{7.118335in}}{\pgfqpoint{7.537127in}{7.122725in}}{\pgfqpoint{7.526077in}{7.122725in}}%
\pgfpathcurveto{\pgfqpoint{7.515027in}{7.122725in}}{\pgfqpoint{7.504428in}{7.118335in}}{\pgfqpoint{7.496614in}{7.110521in}}%
\pgfpathcurveto{\pgfqpoint{7.488800in}{7.102708in}}{\pgfqpoint{7.484410in}{7.092109in}}{\pgfqpoint{7.484410in}{7.081058in}}%
\pgfpathcurveto{\pgfqpoint{7.484410in}{7.070008in}}{\pgfqpoint{7.488800in}{7.059409in}}{\pgfqpoint{7.496614in}{7.051596in}}%
\pgfpathcurveto{\pgfqpoint{7.504428in}{7.043782in}}{\pgfqpoint{7.515027in}{7.039392in}}{\pgfqpoint{7.526077in}{7.039392in}}%
\pgfpathclose%
\pgfusepath{stroke,fill}%
\end{pgfscope}%
\begin{pgfscope}%
\pgfpathrectangle{\pgfqpoint{0.481978in}{0.331635in}}{\pgfqpoint{9.300000in}{7.700000in}}%
\pgfusepath{clip}%
\pgfsetbuttcap%
\pgfsetroundjoin%
\definecolor{currentfill}{rgb}{1.000000,0.705882,0.509804}%
\pgfsetfillcolor{currentfill}%
\pgfsetlinewidth{0.481800pt}%
\definecolor{currentstroke}{rgb}{1.000000,1.000000,1.000000}%
\pgfsetstrokecolor{currentstroke}%
\pgfsetdash{}{0pt}%
\pgfpathmoveto{\pgfqpoint{6.779267in}{7.330326in}}%
\pgfpathcurveto{\pgfqpoint{6.790317in}{7.330326in}}{\pgfqpoint{6.800916in}{7.334717in}}{\pgfqpoint{6.808730in}{7.342530in}}%
\pgfpathcurveto{\pgfqpoint{6.816544in}{7.350344in}}{\pgfqpoint{6.820934in}{7.360943in}}{\pgfqpoint{6.820934in}{7.371993in}}%
\pgfpathcurveto{\pgfqpoint{6.820934in}{7.383043in}}{\pgfqpoint{6.816544in}{7.393642in}}{\pgfqpoint{6.808730in}{7.401456in}}%
\pgfpathcurveto{\pgfqpoint{6.800916in}{7.409269in}}{\pgfqpoint{6.790317in}{7.413660in}}{\pgfqpoint{6.779267in}{7.413660in}}%
\pgfpathcurveto{\pgfqpoint{6.768217in}{7.413660in}}{\pgfqpoint{6.757618in}{7.409269in}}{\pgfqpoint{6.749804in}{7.401456in}}%
\pgfpathcurveto{\pgfqpoint{6.741991in}{7.393642in}}{\pgfqpoint{6.737600in}{7.383043in}}{\pgfqpoint{6.737600in}{7.371993in}}%
\pgfpathcurveto{\pgfqpoint{6.737600in}{7.360943in}}{\pgfqpoint{6.741991in}{7.350344in}}{\pgfqpoint{6.749804in}{7.342530in}}%
\pgfpathcurveto{\pgfqpoint{6.757618in}{7.334717in}}{\pgfqpoint{6.768217in}{7.330326in}}{\pgfqpoint{6.779267in}{7.330326in}}%
\pgfpathclose%
\pgfusepath{stroke,fill}%
\end{pgfscope}%
\begin{pgfscope}%
\pgfpathrectangle{\pgfqpoint{0.481978in}{0.331635in}}{\pgfqpoint{9.300000in}{7.700000in}}%
\pgfusepath{clip}%
\pgfsetbuttcap%
\pgfsetroundjoin%
\definecolor{currentfill}{rgb}{1.000000,0.705882,0.509804}%
\pgfsetfillcolor{currentfill}%
\pgfsetlinewidth{0.481800pt}%
\definecolor{currentstroke}{rgb}{1.000000,1.000000,1.000000}%
\pgfsetstrokecolor{currentstroke}%
\pgfsetdash{}{0pt}%
\pgfpathmoveto{\pgfqpoint{2.475590in}{6.035520in}}%
\pgfpathcurveto{\pgfqpoint{2.486640in}{6.035520in}}{\pgfqpoint{2.497239in}{6.039910in}}{\pgfqpoint{2.505053in}{6.047723in}}%
\pgfpathcurveto{\pgfqpoint{2.512866in}{6.055537in}}{\pgfqpoint{2.517257in}{6.066136in}}{\pgfqpoint{2.517257in}{6.077186in}}%
\pgfpathcurveto{\pgfqpoint{2.517257in}{6.088236in}}{\pgfqpoint{2.512866in}{6.098835in}}{\pgfqpoint{2.505053in}{6.106649in}}%
\pgfpathcurveto{\pgfqpoint{2.497239in}{6.114463in}}{\pgfqpoint{2.486640in}{6.118853in}}{\pgfqpoint{2.475590in}{6.118853in}}%
\pgfpathcurveto{\pgfqpoint{2.464540in}{6.118853in}}{\pgfqpoint{2.453941in}{6.114463in}}{\pgfqpoint{2.446127in}{6.106649in}}%
\pgfpathcurveto{\pgfqpoint{2.438314in}{6.098835in}}{\pgfqpoint{2.433923in}{6.088236in}}{\pgfqpoint{2.433923in}{6.077186in}}%
\pgfpathcurveto{\pgfqpoint{2.433923in}{6.066136in}}{\pgfqpoint{2.438314in}{6.055537in}}{\pgfqpoint{2.446127in}{6.047723in}}%
\pgfpathcurveto{\pgfqpoint{2.453941in}{6.039910in}}{\pgfqpoint{2.464540in}{6.035520in}}{\pgfqpoint{2.475590in}{6.035520in}}%
\pgfpathclose%
\pgfusepath{stroke,fill}%
\end{pgfscope}%
\begin{pgfscope}%
\pgfpathrectangle{\pgfqpoint{0.481978in}{0.331635in}}{\pgfqpoint{9.300000in}{7.700000in}}%
\pgfusepath{clip}%
\pgfsetbuttcap%
\pgfsetroundjoin%
\definecolor{currentfill}{rgb}{1.000000,0.705882,0.509804}%
\pgfsetfillcolor{currentfill}%
\pgfsetlinewidth{0.481800pt}%
\definecolor{currentstroke}{rgb}{1.000000,1.000000,1.000000}%
\pgfsetstrokecolor{currentstroke}%
\pgfsetdash{}{0pt}%
\pgfpathmoveto{\pgfqpoint{3.242777in}{3.542174in}}%
\pgfpathcurveto{\pgfqpoint{3.253827in}{3.542174in}}{\pgfqpoint{3.264426in}{3.546564in}}{\pgfqpoint{3.272240in}{3.554378in}}%
\pgfpathcurveto{\pgfqpoint{3.280054in}{3.562192in}}{\pgfqpoint{3.284444in}{3.572791in}}{\pgfqpoint{3.284444in}{3.583841in}}%
\pgfpathcurveto{\pgfqpoint{3.284444in}{3.594891in}}{\pgfqpoint{3.280054in}{3.605490in}}{\pgfqpoint{3.272240in}{3.613304in}}%
\pgfpathcurveto{\pgfqpoint{3.264426in}{3.621117in}}{\pgfqpoint{3.253827in}{3.625507in}}{\pgfqpoint{3.242777in}{3.625507in}}%
\pgfpathcurveto{\pgfqpoint{3.231727in}{3.625507in}}{\pgfqpoint{3.221128in}{3.621117in}}{\pgfqpoint{3.213314in}{3.613304in}}%
\pgfpathcurveto{\pgfqpoint{3.205501in}{3.605490in}}{\pgfqpoint{3.201111in}{3.594891in}}{\pgfqpoint{3.201111in}{3.583841in}}%
\pgfpathcurveto{\pgfqpoint{3.201111in}{3.572791in}}{\pgfqpoint{3.205501in}{3.562192in}}{\pgfqpoint{3.213314in}{3.554378in}}%
\pgfpathcurveto{\pgfqpoint{3.221128in}{3.546564in}}{\pgfqpoint{3.231727in}{3.542174in}}{\pgfqpoint{3.242777in}{3.542174in}}%
\pgfpathclose%
\pgfusepath{stroke,fill}%
\end{pgfscope}%
\begin{pgfscope}%
\pgfpathrectangle{\pgfqpoint{0.481978in}{0.331635in}}{\pgfqpoint{9.300000in}{7.700000in}}%
\pgfusepath{clip}%
\pgfsetbuttcap%
\pgfsetroundjoin%
\definecolor{currentfill}{rgb}{1.000000,0.705882,0.509804}%
\pgfsetfillcolor{currentfill}%
\pgfsetlinewidth{0.481800pt}%
\definecolor{currentstroke}{rgb}{1.000000,1.000000,1.000000}%
\pgfsetstrokecolor{currentstroke}%
\pgfsetdash{}{0pt}%
\pgfpathmoveto{\pgfqpoint{4.227999in}{4.356170in}}%
\pgfpathcurveto{\pgfqpoint{4.239049in}{4.356170in}}{\pgfqpoint{4.249648in}{4.360560in}}{\pgfqpoint{4.257461in}{4.368374in}}%
\pgfpathcurveto{\pgfqpoint{4.265275in}{4.376187in}}{\pgfqpoint{4.269665in}{4.386786in}}{\pgfqpoint{4.269665in}{4.397836in}}%
\pgfpathcurveto{\pgfqpoint{4.269665in}{4.408886in}}{\pgfqpoint{4.265275in}{4.419486in}}{\pgfqpoint{4.257461in}{4.427299in}}%
\pgfpathcurveto{\pgfqpoint{4.249648in}{4.435113in}}{\pgfqpoint{4.239049in}{4.439503in}}{\pgfqpoint{4.227999in}{4.439503in}}%
\pgfpathcurveto{\pgfqpoint{4.216948in}{4.439503in}}{\pgfqpoint{4.206349in}{4.435113in}}{\pgfqpoint{4.198536in}{4.427299in}}%
\pgfpathcurveto{\pgfqpoint{4.190722in}{4.419486in}}{\pgfqpoint{4.186332in}{4.408886in}}{\pgfqpoint{4.186332in}{4.397836in}}%
\pgfpathcurveto{\pgfqpoint{4.186332in}{4.386786in}}{\pgfqpoint{4.190722in}{4.376187in}}{\pgfqpoint{4.198536in}{4.368374in}}%
\pgfpathcurveto{\pgfqpoint{4.206349in}{4.360560in}}{\pgfqpoint{4.216948in}{4.356170in}}{\pgfqpoint{4.227999in}{4.356170in}}%
\pgfpathclose%
\pgfusepath{stroke,fill}%
\end{pgfscope}%
\begin{pgfscope}%
\pgfpathrectangle{\pgfqpoint{0.481978in}{0.331635in}}{\pgfqpoint{9.300000in}{7.700000in}}%
\pgfusepath{clip}%
\pgfsetbuttcap%
\pgfsetroundjoin%
\definecolor{currentfill}{rgb}{1.000000,0.705882,0.509804}%
\pgfsetfillcolor{currentfill}%
\pgfsetlinewidth{0.481800pt}%
\definecolor{currentstroke}{rgb}{1.000000,1.000000,1.000000}%
\pgfsetstrokecolor{currentstroke}%
\pgfsetdash{}{0pt}%
\pgfpathmoveto{\pgfqpoint{3.413419in}{5.316706in}}%
\pgfpathcurveto{\pgfqpoint{3.424469in}{5.316706in}}{\pgfqpoint{3.435068in}{5.321096in}}{\pgfqpoint{3.442882in}{5.328909in}}%
\pgfpathcurveto{\pgfqpoint{3.450695in}{5.336723in}}{\pgfqpoint{3.455086in}{5.347322in}}{\pgfqpoint{3.455086in}{5.358372in}}%
\pgfpathcurveto{\pgfqpoint{3.455086in}{5.369422in}}{\pgfqpoint{3.450695in}{5.380021in}}{\pgfqpoint{3.442882in}{5.387835in}}%
\pgfpathcurveto{\pgfqpoint{3.435068in}{5.395649in}}{\pgfqpoint{3.424469in}{5.400039in}}{\pgfqpoint{3.413419in}{5.400039in}}%
\pgfpathcurveto{\pgfqpoint{3.402369in}{5.400039in}}{\pgfqpoint{3.391770in}{5.395649in}}{\pgfqpoint{3.383956in}{5.387835in}}%
\pgfpathcurveto{\pgfqpoint{3.376142in}{5.380021in}}{\pgfqpoint{3.371752in}{5.369422in}}{\pgfqpoint{3.371752in}{5.358372in}}%
\pgfpathcurveto{\pgfqpoint{3.371752in}{5.347322in}}{\pgfqpoint{3.376142in}{5.336723in}}{\pgfqpoint{3.383956in}{5.328909in}}%
\pgfpathcurveto{\pgfqpoint{3.391770in}{5.321096in}}{\pgfqpoint{3.402369in}{5.316706in}}{\pgfqpoint{3.413419in}{5.316706in}}%
\pgfpathclose%
\pgfusepath{stroke,fill}%
\end{pgfscope}%
\begin{pgfscope}%
\pgfpathrectangle{\pgfqpoint{0.481978in}{0.331635in}}{\pgfqpoint{9.300000in}{7.700000in}}%
\pgfusepath{clip}%
\pgfsetbuttcap%
\pgfsetroundjoin%
\definecolor{currentfill}{rgb}{1.000000,0.705882,0.509804}%
\pgfsetfillcolor{currentfill}%
\pgfsetlinewidth{0.481800pt}%
\definecolor{currentstroke}{rgb}{1.000000,1.000000,1.000000}%
\pgfsetstrokecolor{currentstroke}%
\pgfsetdash{}{0pt}%
\pgfpathmoveto{\pgfqpoint{3.930822in}{5.610835in}}%
\pgfpathcurveto{\pgfqpoint{3.941872in}{5.610835in}}{\pgfqpoint{3.952471in}{5.615226in}}{\pgfqpoint{3.960285in}{5.623039in}}%
\pgfpathcurveto{\pgfqpoint{3.968099in}{5.630853in}}{\pgfqpoint{3.972489in}{5.641452in}}{\pgfqpoint{3.972489in}{5.652502in}}%
\pgfpathcurveto{\pgfqpoint{3.972489in}{5.663552in}}{\pgfqpoint{3.968099in}{5.674151in}}{\pgfqpoint{3.960285in}{5.681965in}}%
\pgfpathcurveto{\pgfqpoint{3.952471in}{5.689778in}}{\pgfqpoint{3.941872in}{5.694169in}}{\pgfqpoint{3.930822in}{5.694169in}}%
\pgfpathcurveto{\pgfqpoint{3.919772in}{5.694169in}}{\pgfqpoint{3.909173in}{5.689778in}}{\pgfqpoint{3.901359in}{5.681965in}}%
\pgfpathcurveto{\pgfqpoint{3.893546in}{5.674151in}}{\pgfqpoint{3.889156in}{5.663552in}}{\pgfqpoint{3.889156in}{5.652502in}}%
\pgfpathcurveto{\pgfqpoint{3.889156in}{5.641452in}}{\pgfqpoint{3.893546in}{5.630853in}}{\pgfqpoint{3.901359in}{5.623039in}}%
\pgfpathcurveto{\pgfqpoint{3.909173in}{5.615226in}}{\pgfqpoint{3.919772in}{5.610835in}}{\pgfqpoint{3.930822in}{5.610835in}}%
\pgfpathclose%
\pgfusepath{stroke,fill}%
\end{pgfscope}%
\begin{pgfscope}%
\pgfpathrectangle{\pgfqpoint{0.481978in}{0.331635in}}{\pgfqpoint{9.300000in}{7.700000in}}%
\pgfusepath{clip}%
\pgfsetbuttcap%
\pgfsetroundjoin%
\definecolor{currentfill}{rgb}{1.000000,0.705882,0.509804}%
\pgfsetfillcolor{currentfill}%
\pgfsetlinewidth{0.481800pt}%
\definecolor{currentstroke}{rgb}{1.000000,1.000000,1.000000}%
\pgfsetstrokecolor{currentstroke}%
\pgfsetdash{}{0pt}%
\pgfpathmoveto{\pgfqpoint{3.950735in}{3.042884in}}%
\pgfpathcurveto{\pgfqpoint{3.961785in}{3.042884in}}{\pgfqpoint{3.972384in}{3.047275in}}{\pgfqpoint{3.980198in}{3.055088in}}%
\pgfpathcurveto{\pgfqpoint{3.988011in}{3.062902in}}{\pgfqpoint{3.992402in}{3.073501in}}{\pgfqpoint{3.992402in}{3.084551in}}%
\pgfpathcurveto{\pgfqpoint{3.992402in}{3.095601in}}{\pgfqpoint{3.988011in}{3.106200in}}{\pgfqpoint{3.980198in}{3.114014in}}%
\pgfpathcurveto{\pgfqpoint{3.972384in}{3.121827in}}{\pgfqpoint{3.961785in}{3.126218in}}{\pgfqpoint{3.950735in}{3.126218in}}%
\pgfpathcurveto{\pgfqpoint{3.939685in}{3.126218in}}{\pgfqpoint{3.929086in}{3.121827in}}{\pgfqpoint{3.921272in}{3.114014in}}%
\pgfpathcurveto{\pgfqpoint{3.913459in}{3.106200in}}{\pgfqpoint{3.909068in}{3.095601in}}{\pgfqpoint{3.909068in}{3.084551in}}%
\pgfpathcurveto{\pgfqpoint{3.909068in}{3.073501in}}{\pgfqpoint{3.913459in}{3.062902in}}{\pgfqpoint{3.921272in}{3.055088in}}%
\pgfpathcurveto{\pgfqpoint{3.929086in}{3.047275in}}{\pgfqpoint{3.939685in}{3.042884in}}{\pgfqpoint{3.950735in}{3.042884in}}%
\pgfpathclose%
\pgfusepath{stroke,fill}%
\end{pgfscope}%
\begin{pgfscope}%
\pgfpathrectangle{\pgfqpoint{0.481978in}{0.331635in}}{\pgfqpoint{9.300000in}{7.700000in}}%
\pgfusepath{clip}%
\pgfsetbuttcap%
\pgfsetroundjoin%
\definecolor{currentfill}{rgb}{1.000000,0.705882,0.509804}%
\pgfsetfillcolor{currentfill}%
\pgfsetlinewidth{0.481800pt}%
\definecolor{currentstroke}{rgb}{1.000000,1.000000,1.000000}%
\pgfsetstrokecolor{currentstroke}%
\pgfsetdash{}{0pt}%
\pgfpathmoveto{\pgfqpoint{4.062075in}{3.437490in}}%
\pgfpathcurveto{\pgfqpoint{4.073125in}{3.437490in}}{\pgfqpoint{4.083724in}{3.441880in}}{\pgfqpoint{4.091537in}{3.449694in}}%
\pgfpathcurveto{\pgfqpoint{4.099351in}{3.457507in}}{\pgfqpoint{4.103741in}{3.468106in}}{\pgfqpoint{4.103741in}{3.479157in}}%
\pgfpathcurveto{\pgfqpoint{4.103741in}{3.490207in}}{\pgfqpoint{4.099351in}{3.500806in}}{\pgfqpoint{4.091537in}{3.508619in}}%
\pgfpathcurveto{\pgfqpoint{4.083724in}{3.516433in}}{\pgfqpoint{4.073125in}{3.520823in}}{\pgfqpoint{4.062075in}{3.520823in}}%
\pgfpathcurveto{\pgfqpoint{4.051024in}{3.520823in}}{\pgfqpoint{4.040425in}{3.516433in}}{\pgfqpoint{4.032612in}{3.508619in}}%
\pgfpathcurveto{\pgfqpoint{4.024798in}{3.500806in}}{\pgfqpoint{4.020408in}{3.490207in}}{\pgfqpoint{4.020408in}{3.479157in}}%
\pgfpathcurveto{\pgfqpoint{4.020408in}{3.468106in}}{\pgfqpoint{4.024798in}{3.457507in}}{\pgfqpoint{4.032612in}{3.449694in}}%
\pgfpathcurveto{\pgfqpoint{4.040425in}{3.441880in}}{\pgfqpoint{4.051024in}{3.437490in}}{\pgfqpoint{4.062075in}{3.437490in}}%
\pgfpathclose%
\pgfusepath{stroke,fill}%
\end{pgfscope}%
\begin{pgfscope}%
\pgfpathrectangle{\pgfqpoint{0.481978in}{0.331635in}}{\pgfqpoint{9.300000in}{7.700000in}}%
\pgfusepath{clip}%
\pgfsetbuttcap%
\pgfsetroundjoin%
\definecolor{currentfill}{rgb}{1.000000,0.705882,0.509804}%
\pgfsetfillcolor{currentfill}%
\pgfsetlinewidth{0.481800pt}%
\definecolor{currentstroke}{rgb}{1.000000,1.000000,1.000000}%
\pgfsetstrokecolor{currentstroke}%
\pgfsetdash{}{0pt}%
\pgfpathmoveto{\pgfqpoint{5.989908in}{3.404674in}}%
\pgfpathcurveto{\pgfqpoint{6.000958in}{3.404674in}}{\pgfqpoint{6.011557in}{3.409064in}}{\pgfqpoint{6.019370in}{3.416878in}}%
\pgfpathcurveto{\pgfqpoint{6.027184in}{3.424692in}}{\pgfqpoint{6.031574in}{3.435291in}}{\pgfqpoint{6.031574in}{3.446341in}}%
\pgfpathcurveto{\pgfqpoint{6.031574in}{3.457391in}}{\pgfqpoint{6.027184in}{3.467990in}}{\pgfqpoint{6.019370in}{3.475804in}}%
\pgfpathcurveto{\pgfqpoint{6.011557in}{3.483617in}}{\pgfqpoint{6.000958in}{3.488007in}}{\pgfqpoint{5.989908in}{3.488007in}}%
\pgfpathcurveto{\pgfqpoint{5.978857in}{3.488007in}}{\pgfqpoint{5.968258in}{3.483617in}}{\pgfqpoint{5.960445in}{3.475804in}}%
\pgfpathcurveto{\pgfqpoint{5.952631in}{3.467990in}}{\pgfqpoint{5.948241in}{3.457391in}}{\pgfqpoint{5.948241in}{3.446341in}}%
\pgfpathcurveto{\pgfqpoint{5.948241in}{3.435291in}}{\pgfqpoint{5.952631in}{3.424692in}}{\pgfqpoint{5.960445in}{3.416878in}}%
\pgfpathcurveto{\pgfqpoint{5.968258in}{3.409064in}}{\pgfqpoint{5.978857in}{3.404674in}}{\pgfqpoint{5.989908in}{3.404674in}}%
\pgfpathclose%
\pgfusepath{stroke,fill}%
\end{pgfscope}%
\begin{pgfscope}%
\pgfpathrectangle{\pgfqpoint{0.481978in}{0.331635in}}{\pgfqpoint{9.300000in}{7.700000in}}%
\pgfusepath{clip}%
\pgfsetbuttcap%
\pgfsetroundjoin%
\definecolor{currentfill}{rgb}{1.000000,0.705882,0.509804}%
\pgfsetfillcolor{currentfill}%
\pgfsetlinewidth{0.481800pt}%
\definecolor{currentstroke}{rgb}{1.000000,1.000000,1.000000}%
\pgfsetstrokecolor{currentstroke}%
\pgfsetdash{}{0pt}%
\pgfpathmoveto{\pgfqpoint{8.313309in}{4.090148in}}%
\pgfpathcurveto{\pgfqpoint{8.324359in}{4.090148in}}{\pgfqpoint{8.334958in}{4.094538in}}{\pgfqpoint{8.342772in}{4.102351in}}%
\pgfpathcurveto{\pgfqpoint{8.350586in}{4.110165in}}{\pgfqpoint{8.354976in}{4.120764in}}{\pgfqpoint{8.354976in}{4.131814in}}%
\pgfpathcurveto{\pgfqpoint{8.354976in}{4.142864in}}{\pgfqpoint{8.350586in}{4.153463in}}{\pgfqpoint{8.342772in}{4.161277in}}%
\pgfpathcurveto{\pgfqpoint{8.334958in}{4.169091in}}{\pgfqpoint{8.324359in}{4.173481in}}{\pgfqpoint{8.313309in}{4.173481in}}%
\pgfpathcurveto{\pgfqpoint{8.302259in}{4.173481in}}{\pgfqpoint{8.291660in}{4.169091in}}{\pgfqpoint{8.283846in}{4.161277in}}%
\pgfpathcurveto{\pgfqpoint{8.276033in}{4.153463in}}{\pgfqpoint{8.271642in}{4.142864in}}{\pgfqpoint{8.271642in}{4.131814in}}%
\pgfpathcurveto{\pgfqpoint{8.271642in}{4.120764in}}{\pgfqpoint{8.276033in}{4.110165in}}{\pgfqpoint{8.283846in}{4.102351in}}%
\pgfpathcurveto{\pgfqpoint{8.291660in}{4.094538in}}{\pgfqpoint{8.302259in}{4.090148in}}{\pgfqpoint{8.313309in}{4.090148in}}%
\pgfpathclose%
\pgfusepath{stroke,fill}%
\end{pgfscope}%
\begin{pgfscope}%
\pgfpathrectangle{\pgfqpoint{0.481978in}{0.331635in}}{\pgfqpoint{9.300000in}{7.700000in}}%
\pgfusepath{clip}%
\pgfsetbuttcap%
\pgfsetroundjoin%
\definecolor{currentfill}{rgb}{1.000000,0.705882,0.509804}%
\pgfsetfillcolor{currentfill}%
\pgfsetlinewidth{0.481800pt}%
\definecolor{currentstroke}{rgb}{1.000000,1.000000,1.000000}%
\pgfsetstrokecolor{currentstroke}%
\pgfsetdash{}{0pt}%
\pgfpathmoveto{\pgfqpoint{4.541721in}{3.238983in}}%
\pgfpathcurveto{\pgfqpoint{4.552771in}{3.238983in}}{\pgfqpoint{4.563371in}{3.243373in}}{\pgfqpoint{4.571184in}{3.251187in}}%
\pgfpathcurveto{\pgfqpoint{4.578998in}{3.259000in}}{\pgfqpoint{4.583388in}{3.269599in}}{\pgfqpoint{4.583388in}{3.280649in}}%
\pgfpathcurveto{\pgfqpoint{4.583388in}{3.291700in}}{\pgfqpoint{4.578998in}{3.302299in}}{\pgfqpoint{4.571184in}{3.310112in}}%
\pgfpathcurveto{\pgfqpoint{4.563371in}{3.317926in}}{\pgfqpoint{4.552771in}{3.322316in}}{\pgfqpoint{4.541721in}{3.322316in}}%
\pgfpathcurveto{\pgfqpoint{4.530671in}{3.322316in}}{\pgfqpoint{4.520072in}{3.317926in}}{\pgfqpoint{4.512259in}{3.310112in}}%
\pgfpathcurveto{\pgfqpoint{4.504445in}{3.302299in}}{\pgfqpoint{4.500055in}{3.291700in}}{\pgfqpoint{4.500055in}{3.280649in}}%
\pgfpathcurveto{\pgfqpoint{4.500055in}{3.269599in}}{\pgfqpoint{4.504445in}{3.259000in}}{\pgfqpoint{4.512259in}{3.251187in}}%
\pgfpathcurveto{\pgfqpoint{4.520072in}{3.243373in}}{\pgfqpoint{4.530671in}{3.238983in}}{\pgfqpoint{4.541721in}{3.238983in}}%
\pgfpathclose%
\pgfusepath{stroke,fill}%
\end{pgfscope}%
\begin{pgfscope}%
\pgfpathrectangle{\pgfqpoint{0.481978in}{0.331635in}}{\pgfqpoint{9.300000in}{7.700000in}}%
\pgfusepath{clip}%
\pgfsetbuttcap%
\pgfsetroundjoin%
\definecolor{currentfill}{rgb}{1.000000,0.705882,0.509804}%
\pgfsetfillcolor{currentfill}%
\pgfsetlinewidth{0.481800pt}%
\definecolor{currentstroke}{rgb}{1.000000,1.000000,1.000000}%
\pgfsetstrokecolor{currentstroke}%
\pgfsetdash{}{0pt}%
\pgfpathmoveto{\pgfqpoint{8.281939in}{6.665290in}}%
\pgfpathcurveto{\pgfqpoint{8.292989in}{6.665290in}}{\pgfqpoint{8.303588in}{6.669681in}}{\pgfqpoint{8.311402in}{6.677494in}}%
\pgfpathcurveto{\pgfqpoint{8.319216in}{6.685308in}}{\pgfqpoint{8.323606in}{6.695907in}}{\pgfqpoint{8.323606in}{6.706957in}}%
\pgfpathcurveto{\pgfqpoint{8.323606in}{6.718007in}}{\pgfqpoint{8.319216in}{6.728606in}}{\pgfqpoint{8.311402in}{6.736420in}}%
\pgfpathcurveto{\pgfqpoint{8.303588in}{6.744234in}}{\pgfqpoint{8.292989in}{6.748624in}}{\pgfqpoint{8.281939in}{6.748624in}}%
\pgfpathcurveto{\pgfqpoint{8.270889in}{6.748624in}}{\pgfqpoint{8.260290in}{6.744234in}}{\pgfqpoint{8.252476in}{6.736420in}}%
\pgfpathcurveto{\pgfqpoint{8.244663in}{6.728606in}}{\pgfqpoint{8.240273in}{6.718007in}}{\pgfqpoint{8.240273in}{6.706957in}}%
\pgfpathcurveto{\pgfqpoint{8.240273in}{6.695907in}}{\pgfqpoint{8.244663in}{6.685308in}}{\pgfqpoint{8.252476in}{6.677494in}}%
\pgfpathcurveto{\pgfqpoint{8.260290in}{6.669681in}}{\pgfqpoint{8.270889in}{6.665290in}}{\pgfqpoint{8.281939in}{6.665290in}}%
\pgfpathclose%
\pgfusepath{stroke,fill}%
\end{pgfscope}%
\begin{pgfscope}%
\pgfpathrectangle{\pgfqpoint{0.481978in}{0.331635in}}{\pgfqpoint{9.300000in}{7.700000in}}%
\pgfusepath{clip}%
\pgfsetbuttcap%
\pgfsetroundjoin%
\definecolor{currentfill}{rgb}{1.000000,0.705882,0.509804}%
\pgfsetfillcolor{currentfill}%
\pgfsetlinewidth{0.481800pt}%
\definecolor{currentstroke}{rgb}{1.000000,1.000000,1.000000}%
\pgfsetstrokecolor{currentstroke}%
\pgfsetdash{}{0pt}%
\pgfpathmoveto{\pgfqpoint{7.611554in}{7.512902in}}%
\pgfpathcurveto{\pgfqpoint{7.622605in}{7.512902in}}{\pgfqpoint{7.633204in}{7.517292in}}{\pgfqpoint{7.641017in}{7.525106in}}%
\pgfpathcurveto{\pgfqpoint{7.648831in}{7.532920in}}{\pgfqpoint{7.653221in}{7.543519in}}{\pgfqpoint{7.653221in}{7.554569in}}%
\pgfpathcurveto{\pgfqpoint{7.653221in}{7.565619in}}{\pgfqpoint{7.648831in}{7.576218in}}{\pgfqpoint{7.641017in}{7.584032in}}%
\pgfpathcurveto{\pgfqpoint{7.633204in}{7.591845in}}{\pgfqpoint{7.622605in}{7.596236in}}{\pgfqpoint{7.611554in}{7.596236in}}%
\pgfpathcurveto{\pgfqpoint{7.600504in}{7.596236in}}{\pgfqpoint{7.589905in}{7.591845in}}{\pgfqpoint{7.582092in}{7.584032in}}%
\pgfpathcurveto{\pgfqpoint{7.574278in}{7.576218in}}{\pgfqpoint{7.569888in}{7.565619in}}{\pgfqpoint{7.569888in}{7.554569in}}%
\pgfpathcurveto{\pgfqpoint{7.569888in}{7.543519in}}{\pgfqpoint{7.574278in}{7.532920in}}{\pgfqpoint{7.582092in}{7.525106in}}%
\pgfpathcurveto{\pgfqpoint{7.589905in}{7.517292in}}{\pgfqpoint{7.600504in}{7.512902in}}{\pgfqpoint{7.611554in}{7.512902in}}%
\pgfpathclose%
\pgfusepath{stroke,fill}%
\end{pgfscope}%
\begin{pgfscope}%
\pgfpathrectangle{\pgfqpoint{0.481978in}{0.331635in}}{\pgfqpoint{9.300000in}{7.700000in}}%
\pgfusepath{clip}%
\pgfsetbuttcap%
\pgfsetroundjoin%
\definecolor{currentfill}{rgb}{1.000000,0.705882,0.509804}%
\pgfsetfillcolor{currentfill}%
\pgfsetlinewidth{0.481800pt}%
\definecolor{currentstroke}{rgb}{1.000000,1.000000,1.000000}%
\pgfsetstrokecolor{currentstroke}%
\pgfsetdash{}{0pt}%
\pgfpathmoveto{\pgfqpoint{2.721741in}{3.669095in}}%
\pgfpathcurveto{\pgfqpoint{2.732792in}{3.669095in}}{\pgfqpoint{2.743391in}{3.673485in}}{\pgfqpoint{2.751204in}{3.681299in}}%
\pgfpathcurveto{\pgfqpoint{2.759018in}{3.689112in}}{\pgfqpoint{2.763408in}{3.699711in}}{\pgfqpoint{2.763408in}{3.710762in}}%
\pgfpathcurveto{\pgfqpoint{2.763408in}{3.721812in}}{\pgfqpoint{2.759018in}{3.732411in}}{\pgfqpoint{2.751204in}{3.740224in}}%
\pgfpathcurveto{\pgfqpoint{2.743391in}{3.748038in}}{\pgfqpoint{2.732792in}{3.752428in}}{\pgfqpoint{2.721741in}{3.752428in}}%
\pgfpathcurveto{\pgfqpoint{2.710691in}{3.752428in}}{\pgfqpoint{2.700092in}{3.748038in}}{\pgfqpoint{2.692279in}{3.740224in}}%
\pgfpathcurveto{\pgfqpoint{2.684465in}{3.732411in}}{\pgfqpoint{2.680075in}{3.721812in}}{\pgfqpoint{2.680075in}{3.710762in}}%
\pgfpathcurveto{\pgfqpoint{2.680075in}{3.699711in}}{\pgfqpoint{2.684465in}{3.689112in}}{\pgfqpoint{2.692279in}{3.681299in}}%
\pgfpathcurveto{\pgfqpoint{2.700092in}{3.673485in}}{\pgfqpoint{2.710691in}{3.669095in}}{\pgfqpoint{2.721741in}{3.669095in}}%
\pgfpathclose%
\pgfusepath{stroke,fill}%
\end{pgfscope}%
\begin{pgfscope}%
\pgfpathrectangle{\pgfqpoint{0.481978in}{0.331635in}}{\pgfqpoint{9.300000in}{7.700000in}}%
\pgfusepath{clip}%
\pgfsetbuttcap%
\pgfsetroundjoin%
\definecolor{currentfill}{rgb}{1.000000,0.705882,0.509804}%
\pgfsetfillcolor{currentfill}%
\pgfsetlinewidth{0.481800pt}%
\definecolor{currentstroke}{rgb}{1.000000,1.000000,1.000000}%
\pgfsetstrokecolor{currentstroke}%
\pgfsetdash{}{0pt}%
\pgfpathmoveto{\pgfqpoint{6.813584in}{2.399944in}}%
\pgfpathcurveto{\pgfqpoint{6.824634in}{2.399944in}}{\pgfqpoint{6.835233in}{2.404334in}}{\pgfqpoint{6.843047in}{2.412148in}}%
\pgfpathcurveto{\pgfqpoint{6.850860in}{2.419962in}}{\pgfqpoint{6.855251in}{2.430561in}}{\pgfqpoint{6.855251in}{2.441611in}}%
\pgfpathcurveto{\pgfqpoint{6.855251in}{2.452661in}}{\pgfqpoint{6.850860in}{2.463260in}}{\pgfqpoint{6.843047in}{2.471074in}}%
\pgfpathcurveto{\pgfqpoint{6.835233in}{2.478887in}}{\pgfqpoint{6.824634in}{2.483278in}}{\pgfqpoint{6.813584in}{2.483278in}}%
\pgfpathcurveto{\pgfqpoint{6.802534in}{2.483278in}}{\pgfqpoint{6.791935in}{2.478887in}}{\pgfqpoint{6.784121in}{2.471074in}}%
\pgfpathcurveto{\pgfqpoint{6.776308in}{2.463260in}}{\pgfqpoint{6.771917in}{2.452661in}}{\pgfqpoint{6.771917in}{2.441611in}}%
\pgfpathcurveto{\pgfqpoint{6.771917in}{2.430561in}}{\pgfqpoint{6.776308in}{2.419962in}}{\pgfqpoint{6.784121in}{2.412148in}}%
\pgfpathcurveto{\pgfqpoint{6.791935in}{2.404334in}}{\pgfqpoint{6.802534in}{2.399944in}}{\pgfqpoint{6.813584in}{2.399944in}}%
\pgfpathclose%
\pgfusepath{stroke,fill}%
\end{pgfscope}%
\begin{pgfscope}%
\pgfpathrectangle{\pgfqpoint{0.481978in}{0.331635in}}{\pgfqpoint{9.300000in}{7.700000in}}%
\pgfusepath{clip}%
\pgfsetbuttcap%
\pgfsetroundjoin%
\definecolor{currentfill}{rgb}{1.000000,0.705882,0.509804}%
\pgfsetfillcolor{currentfill}%
\pgfsetlinewidth{0.481800pt}%
\definecolor{currentstroke}{rgb}{1.000000,1.000000,1.000000}%
\pgfsetstrokecolor{currentstroke}%
\pgfsetdash{}{0pt}%
\pgfpathmoveto{\pgfqpoint{7.019489in}{7.639968in}}%
\pgfpathcurveto{\pgfqpoint{7.030540in}{7.639968in}}{\pgfqpoint{7.041139in}{7.644359in}}{\pgfqpoint{7.048952in}{7.652172in}}%
\pgfpathcurveto{\pgfqpoint{7.056766in}{7.659986in}}{\pgfqpoint{7.061156in}{7.670585in}}{\pgfqpoint{7.061156in}{7.681635in}}%
\pgfpathcurveto{\pgfqpoint{7.061156in}{7.692685in}}{\pgfqpoint{7.056766in}{7.703284in}}{\pgfqpoint{7.048952in}{7.711098in}}%
\pgfpathcurveto{\pgfqpoint{7.041139in}{7.718911in}}{\pgfqpoint{7.030540in}{7.723302in}}{\pgfqpoint{7.019489in}{7.723302in}}%
\pgfpathcurveto{\pgfqpoint{7.008439in}{7.723302in}}{\pgfqpoint{6.997840in}{7.718911in}}{\pgfqpoint{6.990027in}{7.711098in}}%
\pgfpathcurveto{\pgfqpoint{6.982213in}{7.703284in}}{\pgfqpoint{6.977823in}{7.692685in}}{\pgfqpoint{6.977823in}{7.681635in}}%
\pgfpathcurveto{\pgfqpoint{6.977823in}{7.670585in}}{\pgfqpoint{6.982213in}{7.659986in}}{\pgfqpoint{6.990027in}{7.652172in}}%
\pgfpathcurveto{\pgfqpoint{6.997840in}{7.644359in}}{\pgfqpoint{7.008439in}{7.639968in}}{\pgfqpoint{7.019489in}{7.639968in}}%
\pgfpathclose%
\pgfusepath{stroke,fill}%
\end{pgfscope}%
\begin{pgfscope}%
\pgfpathrectangle{\pgfqpoint{0.481978in}{0.331635in}}{\pgfqpoint{9.300000in}{7.700000in}}%
\pgfusepath{clip}%
\pgfsetbuttcap%
\pgfsetroundjoin%
\definecolor{currentfill}{rgb}{1.000000,0.705882,0.509804}%
\pgfsetfillcolor{currentfill}%
\pgfsetlinewidth{0.481800pt}%
\definecolor{currentstroke}{rgb}{1.000000,1.000000,1.000000}%
\pgfsetstrokecolor{currentstroke}%
\pgfsetdash{}{0pt}%
\pgfpathmoveto{\pgfqpoint{6.102927in}{7.240561in}}%
\pgfpathcurveto{\pgfqpoint{6.113977in}{7.240561in}}{\pgfqpoint{6.124576in}{7.244951in}}{\pgfqpoint{6.132389in}{7.252764in}}%
\pgfpathcurveto{\pgfqpoint{6.140203in}{7.260578in}}{\pgfqpoint{6.144593in}{7.271177in}}{\pgfqpoint{6.144593in}{7.282227in}}%
\pgfpathcurveto{\pgfqpoint{6.144593in}{7.293277in}}{\pgfqpoint{6.140203in}{7.303876in}}{\pgfqpoint{6.132389in}{7.311690in}}%
\pgfpathcurveto{\pgfqpoint{6.124576in}{7.319504in}}{\pgfqpoint{6.113977in}{7.323894in}}{\pgfqpoint{6.102927in}{7.323894in}}%
\pgfpathcurveto{\pgfqpoint{6.091876in}{7.323894in}}{\pgfqpoint{6.081277in}{7.319504in}}{\pgfqpoint{6.073464in}{7.311690in}}%
\pgfpathcurveto{\pgfqpoint{6.065650in}{7.303876in}}{\pgfqpoint{6.061260in}{7.293277in}}{\pgfqpoint{6.061260in}{7.282227in}}%
\pgfpathcurveto{\pgfqpoint{6.061260in}{7.271177in}}{\pgfqpoint{6.065650in}{7.260578in}}{\pgfqpoint{6.073464in}{7.252764in}}%
\pgfpathcurveto{\pgfqpoint{6.081277in}{7.244951in}}{\pgfqpoint{6.091876in}{7.240561in}}{\pgfqpoint{6.102927in}{7.240561in}}%
\pgfpathclose%
\pgfusepath{stroke,fill}%
\end{pgfscope}%
\begin{pgfscope}%
\pgfpathrectangle{\pgfqpoint{0.481978in}{0.331635in}}{\pgfqpoint{9.300000in}{7.700000in}}%
\pgfusepath{clip}%
\pgfsetbuttcap%
\pgfsetroundjoin%
\definecolor{currentfill}{rgb}{1.000000,0.705882,0.509804}%
\pgfsetfillcolor{currentfill}%
\pgfsetlinewidth{0.481800pt}%
\definecolor{currentstroke}{rgb}{1.000000,1.000000,1.000000}%
\pgfsetstrokecolor{currentstroke}%
\pgfsetdash{}{0pt}%
\pgfpathmoveto{\pgfqpoint{5.548025in}{3.709019in}}%
\pgfpathcurveto{\pgfqpoint{5.559075in}{3.709019in}}{\pgfqpoint{5.569674in}{3.713409in}}{\pgfqpoint{5.577488in}{3.721223in}}%
\pgfpathcurveto{\pgfqpoint{5.585301in}{3.729036in}}{\pgfqpoint{5.589692in}{3.739635in}}{\pgfqpoint{5.589692in}{3.750685in}}%
\pgfpathcurveto{\pgfqpoint{5.589692in}{3.761736in}}{\pgfqpoint{5.585301in}{3.772335in}}{\pgfqpoint{5.577488in}{3.780148in}}%
\pgfpathcurveto{\pgfqpoint{5.569674in}{3.787962in}}{\pgfqpoint{5.559075in}{3.792352in}}{\pgfqpoint{5.548025in}{3.792352in}}%
\pgfpathcurveto{\pgfqpoint{5.536975in}{3.792352in}}{\pgfqpoint{5.526376in}{3.787962in}}{\pgfqpoint{5.518562in}{3.780148in}}%
\pgfpathcurveto{\pgfqpoint{5.510748in}{3.772335in}}{\pgfqpoint{5.506358in}{3.761736in}}{\pgfqpoint{5.506358in}{3.750685in}}%
\pgfpathcurveto{\pgfqpoint{5.506358in}{3.739635in}}{\pgfqpoint{5.510748in}{3.729036in}}{\pgfqpoint{5.518562in}{3.721223in}}%
\pgfpathcurveto{\pgfqpoint{5.526376in}{3.713409in}}{\pgfqpoint{5.536975in}{3.709019in}}{\pgfqpoint{5.548025in}{3.709019in}}%
\pgfpathclose%
\pgfusepath{stroke,fill}%
\end{pgfscope}%
\begin{pgfscope}%
\pgfpathrectangle{\pgfqpoint{0.481978in}{0.331635in}}{\pgfqpoint{9.300000in}{7.700000in}}%
\pgfusepath{clip}%
\pgfsetbuttcap%
\pgfsetroundjoin%
\definecolor{currentfill}{rgb}{1.000000,0.705882,0.509804}%
\pgfsetfillcolor{currentfill}%
\pgfsetlinewidth{0.481800pt}%
\definecolor{currentstroke}{rgb}{1.000000,1.000000,1.000000}%
\pgfsetstrokecolor{currentstroke}%
\pgfsetdash{}{0pt}%
\pgfpathmoveto{\pgfqpoint{4.725147in}{4.229750in}}%
\pgfpathcurveto{\pgfqpoint{4.736197in}{4.229750in}}{\pgfqpoint{4.746796in}{4.234140in}}{\pgfqpoint{4.754610in}{4.241953in}}%
\pgfpathcurveto{\pgfqpoint{4.762423in}{4.249767in}}{\pgfqpoint{4.766814in}{4.260366in}}{\pgfqpoint{4.766814in}{4.271416in}}%
\pgfpathcurveto{\pgfqpoint{4.766814in}{4.282466in}}{\pgfqpoint{4.762423in}{4.293065in}}{\pgfqpoint{4.754610in}{4.300879in}}%
\pgfpathcurveto{\pgfqpoint{4.746796in}{4.308693in}}{\pgfqpoint{4.736197in}{4.313083in}}{\pgfqpoint{4.725147in}{4.313083in}}%
\pgfpathcurveto{\pgfqpoint{4.714097in}{4.313083in}}{\pgfqpoint{4.703498in}{4.308693in}}{\pgfqpoint{4.695684in}{4.300879in}}%
\pgfpathcurveto{\pgfqpoint{4.687870in}{4.293065in}}{\pgfqpoint{4.683480in}{4.282466in}}{\pgfqpoint{4.683480in}{4.271416in}}%
\pgfpathcurveto{\pgfqpoint{4.683480in}{4.260366in}}{\pgfqpoint{4.687870in}{4.249767in}}{\pgfqpoint{4.695684in}{4.241953in}}%
\pgfpathcurveto{\pgfqpoint{4.703498in}{4.234140in}}{\pgfqpoint{4.714097in}{4.229750in}}{\pgfqpoint{4.725147in}{4.229750in}}%
\pgfpathclose%
\pgfusepath{stroke,fill}%
\end{pgfscope}%
\begin{pgfscope}%
\pgfpathrectangle{\pgfqpoint{0.481978in}{0.331635in}}{\pgfqpoint{9.300000in}{7.700000in}}%
\pgfusepath{clip}%
\pgfsetbuttcap%
\pgfsetroundjoin%
\definecolor{currentfill}{rgb}{1.000000,0.705882,0.509804}%
\pgfsetfillcolor{currentfill}%
\pgfsetlinewidth{0.481800pt}%
\definecolor{currentstroke}{rgb}{1.000000,1.000000,1.000000}%
\pgfsetstrokecolor{currentstroke}%
\pgfsetdash{}{0pt}%
\pgfpathmoveto{\pgfqpoint{7.998254in}{3.604984in}}%
\pgfpathcurveto{\pgfqpoint{8.009304in}{3.604984in}}{\pgfqpoint{8.019903in}{3.609374in}}{\pgfqpoint{8.027716in}{3.617188in}}%
\pgfpathcurveto{\pgfqpoint{8.035530in}{3.625002in}}{\pgfqpoint{8.039920in}{3.635601in}}{\pgfqpoint{8.039920in}{3.646651in}}%
\pgfpathcurveto{\pgfqpoint{8.039920in}{3.657701in}}{\pgfqpoint{8.035530in}{3.668300in}}{\pgfqpoint{8.027716in}{3.676114in}}%
\pgfpathcurveto{\pgfqpoint{8.019903in}{3.683927in}}{\pgfqpoint{8.009304in}{3.688317in}}{\pgfqpoint{7.998254in}{3.688317in}}%
\pgfpathcurveto{\pgfqpoint{7.987203in}{3.688317in}}{\pgfqpoint{7.976604in}{3.683927in}}{\pgfqpoint{7.968791in}{3.676114in}}%
\pgfpathcurveto{\pgfqpoint{7.960977in}{3.668300in}}{\pgfqpoint{7.956587in}{3.657701in}}{\pgfqpoint{7.956587in}{3.646651in}}%
\pgfpathcurveto{\pgfqpoint{7.956587in}{3.635601in}}{\pgfqpoint{7.960977in}{3.625002in}}{\pgfqpoint{7.968791in}{3.617188in}}%
\pgfpathcurveto{\pgfqpoint{7.976604in}{3.609374in}}{\pgfqpoint{7.987203in}{3.604984in}}{\pgfqpoint{7.998254in}{3.604984in}}%
\pgfpathclose%
\pgfusepath{stroke,fill}%
\end{pgfscope}%
\begin{pgfscope}%
\pgfpathrectangle{\pgfqpoint{0.481978in}{0.331635in}}{\pgfqpoint{9.300000in}{7.700000in}}%
\pgfusepath{clip}%
\pgfsetbuttcap%
\pgfsetroundjoin%
\definecolor{currentfill}{rgb}{1.000000,0.705882,0.509804}%
\pgfsetfillcolor{currentfill}%
\pgfsetlinewidth{0.481800pt}%
\definecolor{currentstroke}{rgb}{1.000000,1.000000,1.000000}%
\pgfsetstrokecolor{currentstroke}%
\pgfsetdash{}{0pt}%
\pgfpathmoveto{\pgfqpoint{5.102532in}{4.485893in}}%
\pgfpathcurveto{\pgfqpoint{5.113582in}{4.485893in}}{\pgfqpoint{5.124181in}{4.490283in}}{\pgfqpoint{5.131995in}{4.498097in}}%
\pgfpathcurveto{\pgfqpoint{5.139808in}{4.505910in}}{\pgfqpoint{5.144199in}{4.516509in}}{\pgfqpoint{5.144199in}{4.527559in}}%
\pgfpathcurveto{\pgfqpoint{5.144199in}{4.538609in}}{\pgfqpoint{5.139808in}{4.549208in}}{\pgfqpoint{5.131995in}{4.557022in}}%
\pgfpathcurveto{\pgfqpoint{5.124181in}{4.564836in}}{\pgfqpoint{5.113582in}{4.569226in}}{\pgfqpoint{5.102532in}{4.569226in}}%
\pgfpathcurveto{\pgfqpoint{5.091482in}{4.569226in}}{\pgfqpoint{5.080883in}{4.564836in}}{\pgfqpoint{5.073069in}{4.557022in}}%
\pgfpathcurveto{\pgfqpoint{5.065256in}{4.549208in}}{\pgfqpoint{5.060865in}{4.538609in}}{\pgfqpoint{5.060865in}{4.527559in}}%
\pgfpathcurveto{\pgfqpoint{5.060865in}{4.516509in}}{\pgfqpoint{5.065256in}{4.505910in}}{\pgfqpoint{5.073069in}{4.498097in}}%
\pgfpathcurveto{\pgfqpoint{5.080883in}{4.490283in}}{\pgfqpoint{5.091482in}{4.485893in}}{\pgfqpoint{5.102532in}{4.485893in}}%
\pgfpathclose%
\pgfusepath{stroke,fill}%
\end{pgfscope}%
\begin{pgfscope}%
\pgfpathrectangle{\pgfqpoint{0.481978in}{0.331635in}}{\pgfqpoint{9.300000in}{7.700000in}}%
\pgfusepath{clip}%
\pgfsetbuttcap%
\pgfsetroundjoin%
\definecolor{currentfill}{rgb}{1.000000,0.705882,0.509804}%
\pgfsetfillcolor{currentfill}%
\pgfsetlinewidth{0.481800pt}%
\definecolor{currentstroke}{rgb}{1.000000,1.000000,1.000000}%
\pgfsetstrokecolor{currentstroke}%
\pgfsetdash{}{0pt}%
\pgfpathmoveto{\pgfqpoint{4.679104in}{4.702735in}}%
\pgfpathcurveto{\pgfqpoint{4.690154in}{4.702735in}}{\pgfqpoint{4.700753in}{4.707125in}}{\pgfqpoint{4.708567in}{4.714939in}}%
\pgfpathcurveto{\pgfqpoint{4.716381in}{4.722752in}}{\pgfqpoint{4.720771in}{4.733351in}}{\pgfqpoint{4.720771in}{4.744402in}}%
\pgfpathcurveto{\pgfqpoint{4.720771in}{4.755452in}}{\pgfqpoint{4.716381in}{4.766051in}}{\pgfqpoint{4.708567in}{4.773864in}}%
\pgfpathcurveto{\pgfqpoint{4.700753in}{4.781678in}}{\pgfqpoint{4.690154in}{4.786068in}}{\pgfqpoint{4.679104in}{4.786068in}}%
\pgfpathcurveto{\pgfqpoint{4.668054in}{4.786068in}}{\pgfqpoint{4.657455in}{4.781678in}}{\pgfqpoint{4.649641in}{4.773864in}}%
\pgfpathcurveto{\pgfqpoint{4.641828in}{4.766051in}}{\pgfqpoint{4.637437in}{4.755452in}}{\pgfqpoint{4.637437in}{4.744402in}}%
\pgfpathcurveto{\pgfqpoint{4.637437in}{4.733351in}}{\pgfqpoint{4.641828in}{4.722752in}}{\pgfqpoint{4.649641in}{4.714939in}}%
\pgfpathcurveto{\pgfqpoint{4.657455in}{4.707125in}}{\pgfqpoint{4.668054in}{4.702735in}}{\pgfqpoint{4.679104in}{4.702735in}}%
\pgfpathclose%
\pgfusepath{stroke,fill}%
\end{pgfscope}%
\begin{pgfscope}%
\pgfpathrectangle{\pgfqpoint{0.481978in}{0.331635in}}{\pgfqpoint{9.300000in}{7.700000in}}%
\pgfusepath{clip}%
\pgfsetbuttcap%
\pgfsetroundjoin%
\definecolor{currentfill}{rgb}{1.000000,0.705882,0.509804}%
\pgfsetfillcolor{currentfill}%
\pgfsetlinewidth{0.481800pt}%
\definecolor{currentstroke}{rgb}{1.000000,1.000000,1.000000}%
\pgfsetstrokecolor{currentstroke}%
\pgfsetdash{}{0pt}%
\pgfpathmoveto{\pgfqpoint{4.103990in}{2.254849in}}%
\pgfpathcurveto{\pgfqpoint{4.115040in}{2.254849in}}{\pgfqpoint{4.125639in}{2.259240in}}{\pgfqpoint{4.133453in}{2.267053in}}%
\pgfpathcurveto{\pgfqpoint{4.141266in}{2.274867in}}{\pgfqpoint{4.145657in}{2.285466in}}{\pgfqpoint{4.145657in}{2.296516in}}%
\pgfpathcurveto{\pgfqpoint{4.145657in}{2.307566in}}{\pgfqpoint{4.141266in}{2.318165in}}{\pgfqpoint{4.133453in}{2.325979in}}%
\pgfpathcurveto{\pgfqpoint{4.125639in}{2.333792in}}{\pgfqpoint{4.115040in}{2.338183in}}{\pgfqpoint{4.103990in}{2.338183in}}%
\pgfpathcurveto{\pgfqpoint{4.092940in}{2.338183in}}{\pgfqpoint{4.082341in}{2.333792in}}{\pgfqpoint{4.074527in}{2.325979in}}%
\pgfpathcurveto{\pgfqpoint{4.066714in}{2.318165in}}{\pgfqpoint{4.062323in}{2.307566in}}{\pgfqpoint{4.062323in}{2.296516in}}%
\pgfpathcurveto{\pgfqpoint{4.062323in}{2.285466in}}{\pgfqpoint{4.066714in}{2.274867in}}{\pgfqpoint{4.074527in}{2.267053in}}%
\pgfpathcurveto{\pgfqpoint{4.082341in}{2.259240in}}{\pgfqpoint{4.092940in}{2.254849in}}{\pgfqpoint{4.103990in}{2.254849in}}%
\pgfpathclose%
\pgfusepath{stroke,fill}%
\end{pgfscope}%
\begin{pgfscope}%
\pgfpathrectangle{\pgfqpoint{0.481978in}{0.331635in}}{\pgfqpoint{9.300000in}{7.700000in}}%
\pgfusepath{clip}%
\pgfsetbuttcap%
\pgfsetroundjoin%
\definecolor{currentfill}{rgb}{1.000000,0.705882,0.509804}%
\pgfsetfillcolor{currentfill}%
\pgfsetlinewidth{0.481800pt}%
\definecolor{currentstroke}{rgb}{1.000000,1.000000,1.000000}%
\pgfsetstrokecolor{currentstroke}%
\pgfsetdash{}{0pt}%
\pgfpathmoveto{\pgfqpoint{4.530179in}{2.861464in}}%
\pgfpathcurveto{\pgfqpoint{4.541229in}{2.861464in}}{\pgfqpoint{4.551828in}{2.865854in}}{\pgfqpoint{4.559642in}{2.873667in}}%
\pgfpathcurveto{\pgfqpoint{4.567455in}{2.881481in}}{\pgfqpoint{4.571846in}{2.892080in}}{\pgfqpoint{4.571846in}{2.903130in}}%
\pgfpathcurveto{\pgfqpoint{4.571846in}{2.914180in}}{\pgfqpoint{4.567455in}{2.924779in}}{\pgfqpoint{4.559642in}{2.932593in}}%
\pgfpathcurveto{\pgfqpoint{4.551828in}{2.940407in}}{\pgfqpoint{4.541229in}{2.944797in}}{\pgfqpoint{4.530179in}{2.944797in}}%
\pgfpathcurveto{\pgfqpoint{4.519129in}{2.944797in}}{\pgfqpoint{4.508530in}{2.940407in}}{\pgfqpoint{4.500716in}{2.932593in}}%
\pgfpathcurveto{\pgfqpoint{4.492902in}{2.924779in}}{\pgfqpoint{4.488512in}{2.914180in}}{\pgfqpoint{4.488512in}{2.903130in}}%
\pgfpathcurveto{\pgfqpoint{4.488512in}{2.892080in}}{\pgfqpoint{4.492902in}{2.881481in}}{\pgfqpoint{4.500716in}{2.873667in}}%
\pgfpathcurveto{\pgfqpoint{4.508530in}{2.865854in}}{\pgfqpoint{4.519129in}{2.861464in}}{\pgfqpoint{4.530179in}{2.861464in}}%
\pgfpathclose%
\pgfusepath{stroke,fill}%
\end{pgfscope}%
\begin{pgfscope}%
\pgfpathrectangle{\pgfqpoint{0.481978in}{0.331635in}}{\pgfqpoint{9.300000in}{7.700000in}}%
\pgfusepath{clip}%
\pgfsetbuttcap%
\pgfsetroundjoin%
\definecolor{currentfill}{rgb}{1.000000,0.705882,0.509804}%
\pgfsetfillcolor{currentfill}%
\pgfsetlinewidth{0.481800pt}%
\definecolor{currentstroke}{rgb}{1.000000,1.000000,1.000000}%
\pgfsetstrokecolor{currentstroke}%
\pgfsetdash{}{0pt}%
\pgfpathmoveto{\pgfqpoint{3.358304in}{5.994925in}}%
\pgfpathcurveto{\pgfqpoint{3.369354in}{5.994925in}}{\pgfqpoint{3.379953in}{5.999315in}}{\pgfqpoint{3.387767in}{6.007129in}}%
\pgfpathcurveto{\pgfqpoint{3.395581in}{6.014942in}}{\pgfqpoint{3.399971in}{6.025541in}}{\pgfqpoint{3.399971in}{6.036591in}}%
\pgfpathcurveto{\pgfqpoint{3.399971in}{6.047642in}}{\pgfqpoint{3.395581in}{6.058241in}}{\pgfqpoint{3.387767in}{6.066054in}}%
\pgfpathcurveto{\pgfqpoint{3.379953in}{6.073868in}}{\pgfqpoint{3.369354in}{6.078258in}}{\pgfqpoint{3.358304in}{6.078258in}}%
\pgfpathcurveto{\pgfqpoint{3.347254in}{6.078258in}}{\pgfqpoint{3.336655in}{6.073868in}}{\pgfqpoint{3.328841in}{6.066054in}}%
\pgfpathcurveto{\pgfqpoint{3.321028in}{6.058241in}}{\pgfqpoint{3.316638in}{6.047642in}}{\pgfqpoint{3.316638in}{6.036591in}}%
\pgfpathcurveto{\pgfqpoint{3.316638in}{6.025541in}}{\pgfqpoint{3.321028in}{6.014942in}}{\pgfqpoint{3.328841in}{6.007129in}}%
\pgfpathcurveto{\pgfqpoint{3.336655in}{5.999315in}}{\pgfqpoint{3.347254in}{5.994925in}}{\pgfqpoint{3.358304in}{5.994925in}}%
\pgfpathclose%
\pgfusepath{stroke,fill}%
\end{pgfscope}%
\begin{pgfscope}%
\pgfpathrectangle{\pgfqpoint{0.481978in}{0.331635in}}{\pgfqpoint{9.300000in}{7.700000in}}%
\pgfusepath{clip}%
\pgfsetbuttcap%
\pgfsetroundjoin%
\definecolor{currentfill}{rgb}{1.000000,0.705882,0.509804}%
\pgfsetfillcolor{currentfill}%
\pgfsetlinewidth{0.481800pt}%
\definecolor{currentstroke}{rgb}{1.000000,1.000000,1.000000}%
\pgfsetstrokecolor{currentstroke}%
\pgfsetdash{}{0pt}%
\pgfpathmoveto{\pgfqpoint{5.398084in}{2.590748in}}%
\pgfpathcurveto{\pgfqpoint{5.409134in}{2.590748in}}{\pgfqpoint{5.419733in}{2.595138in}}{\pgfqpoint{5.427547in}{2.602952in}}%
\pgfpathcurveto{\pgfqpoint{5.435360in}{2.610765in}}{\pgfqpoint{5.439750in}{2.621364in}}{\pgfqpoint{5.439750in}{2.632414in}}%
\pgfpathcurveto{\pgfqpoint{5.439750in}{2.643464in}}{\pgfqpoint{5.435360in}{2.654063in}}{\pgfqpoint{5.427547in}{2.661877in}}%
\pgfpathcurveto{\pgfqpoint{5.419733in}{2.669691in}}{\pgfqpoint{5.409134in}{2.674081in}}{\pgfqpoint{5.398084in}{2.674081in}}%
\pgfpathcurveto{\pgfqpoint{5.387034in}{2.674081in}}{\pgfqpoint{5.376435in}{2.669691in}}{\pgfqpoint{5.368621in}{2.661877in}}%
\pgfpathcurveto{\pgfqpoint{5.360807in}{2.654063in}}{\pgfqpoint{5.356417in}{2.643464in}}{\pgfqpoint{5.356417in}{2.632414in}}%
\pgfpathcurveto{\pgfqpoint{5.356417in}{2.621364in}}{\pgfqpoint{5.360807in}{2.610765in}}{\pgfqpoint{5.368621in}{2.602952in}}%
\pgfpathcurveto{\pgfqpoint{5.376435in}{2.595138in}}{\pgfqpoint{5.387034in}{2.590748in}}{\pgfqpoint{5.398084in}{2.590748in}}%
\pgfpathclose%
\pgfusepath{stroke,fill}%
\end{pgfscope}%
\begin{pgfscope}%
\pgfpathrectangle{\pgfqpoint{0.481978in}{0.331635in}}{\pgfqpoint{9.300000in}{7.700000in}}%
\pgfusepath{clip}%
\pgfsetbuttcap%
\pgfsetroundjoin%
\definecolor{currentfill}{rgb}{1.000000,0.705882,0.509804}%
\pgfsetfillcolor{currentfill}%
\pgfsetlinewidth{0.481800pt}%
\definecolor{currentstroke}{rgb}{1.000000,1.000000,1.000000}%
\pgfsetstrokecolor{currentstroke}%
\pgfsetdash{}{0pt}%
\pgfpathmoveto{\pgfqpoint{8.487669in}{6.396251in}}%
\pgfpathcurveto{\pgfqpoint{8.498719in}{6.396251in}}{\pgfqpoint{8.509318in}{6.400642in}}{\pgfqpoint{8.517131in}{6.408455in}}%
\pgfpathcurveto{\pgfqpoint{8.524945in}{6.416269in}}{\pgfqpoint{8.529335in}{6.426868in}}{\pgfqpoint{8.529335in}{6.437918in}}%
\pgfpathcurveto{\pgfqpoint{8.529335in}{6.448968in}}{\pgfqpoint{8.524945in}{6.459567in}}{\pgfqpoint{8.517131in}{6.467381in}}%
\pgfpathcurveto{\pgfqpoint{8.509318in}{6.475194in}}{\pgfqpoint{8.498719in}{6.479585in}}{\pgfqpoint{8.487669in}{6.479585in}}%
\pgfpathcurveto{\pgfqpoint{8.476619in}{6.479585in}}{\pgfqpoint{8.466020in}{6.475194in}}{\pgfqpoint{8.458206in}{6.467381in}}%
\pgfpathcurveto{\pgfqpoint{8.450392in}{6.459567in}}{\pgfqpoint{8.446002in}{6.448968in}}{\pgfqpoint{8.446002in}{6.437918in}}%
\pgfpathcurveto{\pgfqpoint{8.446002in}{6.426868in}}{\pgfqpoint{8.450392in}{6.416269in}}{\pgfqpoint{8.458206in}{6.408455in}}%
\pgfpathcurveto{\pgfqpoint{8.466020in}{6.400642in}}{\pgfqpoint{8.476619in}{6.396251in}}{\pgfqpoint{8.487669in}{6.396251in}}%
\pgfpathclose%
\pgfusepath{stroke,fill}%
\end{pgfscope}%
\begin{pgfscope}%
\pgfpathrectangle{\pgfqpoint{0.481978in}{0.331635in}}{\pgfqpoint{9.300000in}{7.700000in}}%
\pgfusepath{clip}%
\pgfsetbuttcap%
\pgfsetroundjoin%
\definecolor{currentfill}{rgb}{1.000000,0.705882,0.509804}%
\pgfsetfillcolor{currentfill}%
\pgfsetlinewidth{0.481800pt}%
\definecolor{currentstroke}{rgb}{1.000000,1.000000,1.000000}%
\pgfsetstrokecolor{currentstroke}%
\pgfsetdash{}{0pt}%
\pgfpathmoveto{\pgfqpoint{4.319461in}{3.830119in}}%
\pgfpathcurveto{\pgfqpoint{4.330511in}{3.830119in}}{\pgfqpoint{4.341110in}{3.834510in}}{\pgfqpoint{4.348923in}{3.842323in}}%
\pgfpathcurveto{\pgfqpoint{4.356737in}{3.850137in}}{\pgfqpoint{4.361127in}{3.860736in}}{\pgfqpoint{4.361127in}{3.871786in}}%
\pgfpathcurveto{\pgfqpoint{4.361127in}{3.882836in}}{\pgfqpoint{4.356737in}{3.893435in}}{\pgfqpoint{4.348923in}{3.901249in}}%
\pgfpathcurveto{\pgfqpoint{4.341110in}{3.909062in}}{\pgfqpoint{4.330511in}{3.913453in}}{\pgfqpoint{4.319461in}{3.913453in}}%
\pgfpathcurveto{\pgfqpoint{4.308410in}{3.913453in}}{\pgfqpoint{4.297811in}{3.909062in}}{\pgfqpoint{4.289998in}{3.901249in}}%
\pgfpathcurveto{\pgfqpoint{4.282184in}{3.893435in}}{\pgfqpoint{4.277794in}{3.882836in}}{\pgfqpoint{4.277794in}{3.871786in}}%
\pgfpathcurveto{\pgfqpoint{4.277794in}{3.860736in}}{\pgfqpoint{4.282184in}{3.850137in}}{\pgfqpoint{4.289998in}{3.842323in}}%
\pgfpathcurveto{\pgfqpoint{4.297811in}{3.834510in}}{\pgfqpoint{4.308410in}{3.830119in}}{\pgfqpoint{4.319461in}{3.830119in}}%
\pgfpathclose%
\pgfusepath{stroke,fill}%
\end{pgfscope}%
\begin{pgfscope}%
\pgfpathrectangle{\pgfqpoint{0.481978in}{0.331635in}}{\pgfqpoint{9.300000in}{7.700000in}}%
\pgfusepath{clip}%
\pgfsetbuttcap%
\pgfsetroundjoin%
\definecolor{currentfill}{rgb}{1.000000,0.705882,0.509804}%
\pgfsetfillcolor{currentfill}%
\pgfsetlinewidth{0.481800pt}%
\definecolor{currentstroke}{rgb}{1.000000,1.000000,1.000000}%
\pgfsetstrokecolor{currentstroke}%
\pgfsetdash{}{0pt}%
\pgfpathmoveto{\pgfqpoint{5.195031in}{3.280936in}}%
\pgfpathcurveto{\pgfqpoint{5.206081in}{3.280936in}}{\pgfqpoint{5.216680in}{3.285326in}}{\pgfqpoint{5.224494in}{3.293140in}}%
\pgfpathcurveto{\pgfqpoint{5.232307in}{3.300953in}}{\pgfqpoint{5.236698in}{3.311552in}}{\pgfqpoint{5.236698in}{3.322603in}}%
\pgfpathcurveto{\pgfqpoint{5.236698in}{3.333653in}}{\pgfqpoint{5.232307in}{3.344252in}}{\pgfqpoint{5.224494in}{3.352065in}}%
\pgfpathcurveto{\pgfqpoint{5.216680in}{3.359879in}}{\pgfqpoint{5.206081in}{3.364269in}}{\pgfqpoint{5.195031in}{3.364269in}}%
\pgfpathcurveto{\pgfqpoint{5.183981in}{3.364269in}}{\pgfqpoint{5.173382in}{3.359879in}}{\pgfqpoint{5.165568in}{3.352065in}}%
\pgfpathcurveto{\pgfqpoint{5.157755in}{3.344252in}}{\pgfqpoint{5.153364in}{3.333653in}}{\pgfqpoint{5.153364in}{3.322603in}}%
\pgfpathcurveto{\pgfqpoint{5.153364in}{3.311552in}}{\pgfqpoint{5.157755in}{3.300953in}}{\pgfqpoint{5.165568in}{3.293140in}}%
\pgfpathcurveto{\pgfqpoint{5.173382in}{3.285326in}}{\pgfqpoint{5.183981in}{3.280936in}}{\pgfqpoint{5.195031in}{3.280936in}}%
\pgfpathclose%
\pgfusepath{stroke,fill}%
\end{pgfscope}%
\begin{pgfscope}%
\pgfpathrectangle{\pgfqpoint{0.481978in}{0.331635in}}{\pgfqpoint{9.300000in}{7.700000in}}%
\pgfusepath{clip}%
\pgfsetbuttcap%
\pgfsetroundjoin%
\definecolor{currentfill}{rgb}{1.000000,0.705882,0.509804}%
\pgfsetfillcolor{currentfill}%
\pgfsetlinewidth{0.481800pt}%
\definecolor{currentstroke}{rgb}{1.000000,1.000000,1.000000}%
\pgfsetstrokecolor{currentstroke}%
\pgfsetdash{}{0pt}%
\pgfpathmoveto{\pgfqpoint{6.327559in}{7.566476in}}%
\pgfpathcurveto{\pgfqpoint{6.338609in}{7.566476in}}{\pgfqpoint{6.349208in}{7.570866in}}{\pgfqpoint{6.357022in}{7.578680in}}%
\pgfpathcurveto{\pgfqpoint{6.364836in}{7.586493in}}{\pgfqpoint{6.369226in}{7.597092in}}{\pgfqpoint{6.369226in}{7.608142in}}%
\pgfpathcurveto{\pgfqpoint{6.369226in}{7.619192in}}{\pgfqpoint{6.364836in}{7.629792in}}{\pgfqpoint{6.357022in}{7.637605in}}%
\pgfpathcurveto{\pgfqpoint{6.349208in}{7.645419in}}{\pgfqpoint{6.338609in}{7.649809in}}{\pgfqpoint{6.327559in}{7.649809in}}%
\pgfpathcurveto{\pgfqpoint{6.316509in}{7.649809in}}{\pgfqpoint{6.305910in}{7.645419in}}{\pgfqpoint{6.298096in}{7.637605in}}%
\pgfpathcurveto{\pgfqpoint{6.290283in}{7.629792in}}{\pgfqpoint{6.285892in}{7.619192in}}{\pgfqpoint{6.285892in}{7.608142in}}%
\pgfpathcurveto{\pgfqpoint{6.285892in}{7.597092in}}{\pgfqpoint{6.290283in}{7.586493in}}{\pgfqpoint{6.298096in}{7.578680in}}%
\pgfpathcurveto{\pgfqpoint{6.305910in}{7.570866in}}{\pgfqpoint{6.316509in}{7.566476in}}{\pgfqpoint{6.327559in}{7.566476in}}%
\pgfpathclose%
\pgfusepath{stroke,fill}%
\end{pgfscope}%
\begin{pgfscope}%
\pgfpathrectangle{\pgfqpoint{0.481978in}{0.331635in}}{\pgfqpoint{9.300000in}{7.700000in}}%
\pgfusepath{clip}%
\pgfsetbuttcap%
\pgfsetroundjoin%
\definecolor{currentfill}{rgb}{1.000000,0.705882,0.509804}%
\pgfsetfillcolor{currentfill}%
\pgfsetlinewidth{0.481800pt}%
\definecolor{currentstroke}{rgb}{1.000000,1.000000,1.000000}%
\pgfsetstrokecolor{currentstroke}%
\pgfsetdash{}{0pt}%
\pgfpathmoveto{\pgfqpoint{6.512708in}{3.228299in}}%
\pgfpathcurveto{\pgfqpoint{6.523758in}{3.228299in}}{\pgfqpoint{6.534357in}{3.232690in}}{\pgfqpoint{6.542171in}{3.240503in}}%
\pgfpathcurveto{\pgfqpoint{6.549984in}{3.248317in}}{\pgfqpoint{6.554375in}{3.258916in}}{\pgfqpoint{6.554375in}{3.269966in}}%
\pgfpathcurveto{\pgfqpoint{6.554375in}{3.281016in}}{\pgfqpoint{6.549984in}{3.291615in}}{\pgfqpoint{6.542171in}{3.299429in}}%
\pgfpathcurveto{\pgfqpoint{6.534357in}{3.307242in}}{\pgfqpoint{6.523758in}{3.311633in}}{\pgfqpoint{6.512708in}{3.311633in}}%
\pgfpathcurveto{\pgfqpoint{6.501658in}{3.311633in}}{\pgfqpoint{6.491059in}{3.307242in}}{\pgfqpoint{6.483245in}{3.299429in}}%
\pgfpathcurveto{\pgfqpoint{6.475432in}{3.291615in}}{\pgfqpoint{6.471041in}{3.281016in}}{\pgfqpoint{6.471041in}{3.269966in}}%
\pgfpathcurveto{\pgfqpoint{6.471041in}{3.258916in}}{\pgfqpoint{6.475432in}{3.248317in}}{\pgfqpoint{6.483245in}{3.240503in}}%
\pgfpathcurveto{\pgfqpoint{6.491059in}{3.232690in}}{\pgfqpoint{6.501658in}{3.228299in}}{\pgfqpoint{6.512708in}{3.228299in}}%
\pgfpathclose%
\pgfusepath{stroke,fill}%
\end{pgfscope}%
\begin{pgfscope}%
\pgfpathrectangle{\pgfqpoint{0.481978in}{0.331635in}}{\pgfqpoint{9.300000in}{7.700000in}}%
\pgfusepath{clip}%
\pgfsetbuttcap%
\pgfsetroundjoin%
\definecolor{currentfill}{rgb}{1.000000,0.705882,0.509804}%
\pgfsetfillcolor{currentfill}%
\pgfsetlinewidth{0.481800pt}%
\definecolor{currentstroke}{rgb}{1.000000,1.000000,1.000000}%
\pgfsetstrokecolor{currentstroke}%
\pgfsetdash{}{0pt}%
\pgfpathmoveto{\pgfqpoint{5.258302in}{2.755878in}}%
\pgfpathcurveto{\pgfqpoint{5.269352in}{2.755878in}}{\pgfqpoint{5.279951in}{2.760268in}}{\pgfqpoint{5.287765in}{2.768082in}}%
\pgfpathcurveto{\pgfqpoint{5.295578in}{2.775896in}}{\pgfqpoint{5.299968in}{2.786495in}}{\pgfqpoint{5.299968in}{2.797545in}}%
\pgfpathcurveto{\pgfqpoint{5.299968in}{2.808595in}}{\pgfqpoint{5.295578in}{2.819194in}}{\pgfqpoint{5.287765in}{2.827008in}}%
\pgfpathcurveto{\pgfqpoint{5.279951in}{2.834821in}}{\pgfqpoint{5.269352in}{2.839211in}}{\pgfqpoint{5.258302in}{2.839211in}}%
\pgfpathcurveto{\pgfqpoint{5.247252in}{2.839211in}}{\pgfqpoint{5.236653in}{2.834821in}}{\pgfqpoint{5.228839in}{2.827008in}}%
\pgfpathcurveto{\pgfqpoint{5.221025in}{2.819194in}}{\pgfqpoint{5.216635in}{2.808595in}}{\pgfqpoint{5.216635in}{2.797545in}}%
\pgfpathcurveto{\pgfqpoint{5.216635in}{2.786495in}}{\pgfqpoint{5.221025in}{2.775896in}}{\pgfqpoint{5.228839in}{2.768082in}}%
\pgfpathcurveto{\pgfqpoint{5.236653in}{2.760268in}}{\pgfqpoint{5.247252in}{2.755878in}}{\pgfqpoint{5.258302in}{2.755878in}}%
\pgfpathclose%
\pgfusepath{stroke,fill}%
\end{pgfscope}%
\begin{pgfscope}%
\pgfpathrectangle{\pgfqpoint{0.481978in}{0.331635in}}{\pgfqpoint{9.300000in}{7.700000in}}%
\pgfusepath{clip}%
\pgfsetbuttcap%
\pgfsetroundjoin%
\definecolor{currentfill}{rgb}{1.000000,0.705882,0.509804}%
\pgfsetfillcolor{currentfill}%
\pgfsetlinewidth{0.481800pt}%
\definecolor{currentstroke}{rgb}{1.000000,1.000000,1.000000}%
\pgfsetstrokecolor{currentstroke}%
\pgfsetdash{}{0pt}%
\pgfpathmoveto{\pgfqpoint{8.656102in}{6.052204in}}%
\pgfpathcurveto{\pgfqpoint{8.667152in}{6.052204in}}{\pgfqpoint{8.677751in}{6.056594in}}{\pgfqpoint{8.685565in}{6.064408in}}%
\pgfpathcurveto{\pgfqpoint{8.693379in}{6.072222in}}{\pgfqpoint{8.697769in}{6.082821in}}{\pgfqpoint{8.697769in}{6.093871in}}%
\pgfpathcurveto{\pgfqpoint{8.697769in}{6.104921in}}{\pgfqpoint{8.693379in}{6.115520in}}{\pgfqpoint{8.685565in}{6.123334in}}%
\pgfpathcurveto{\pgfqpoint{8.677751in}{6.131147in}}{\pgfqpoint{8.667152in}{6.135538in}}{\pgfqpoint{8.656102in}{6.135538in}}%
\pgfpathcurveto{\pgfqpoint{8.645052in}{6.135538in}}{\pgfqpoint{8.634453in}{6.131147in}}{\pgfqpoint{8.626639in}{6.123334in}}%
\pgfpathcurveto{\pgfqpoint{8.618826in}{6.115520in}}{\pgfqpoint{8.614435in}{6.104921in}}{\pgfqpoint{8.614435in}{6.093871in}}%
\pgfpathcurveto{\pgfqpoint{8.614435in}{6.082821in}}{\pgfqpoint{8.618826in}{6.072222in}}{\pgfqpoint{8.626639in}{6.064408in}}%
\pgfpathcurveto{\pgfqpoint{8.634453in}{6.056594in}}{\pgfqpoint{8.645052in}{6.052204in}}{\pgfqpoint{8.656102in}{6.052204in}}%
\pgfpathclose%
\pgfusepath{stroke,fill}%
\end{pgfscope}%
\begin{pgfscope}%
\pgfpathrectangle{\pgfqpoint{0.481978in}{0.331635in}}{\pgfqpoint{9.300000in}{7.700000in}}%
\pgfusepath{clip}%
\pgfsetbuttcap%
\pgfsetroundjoin%
\definecolor{currentfill}{rgb}{1.000000,0.705882,0.509804}%
\pgfsetfillcolor{currentfill}%
\pgfsetlinewidth{0.481800pt}%
\definecolor{currentstroke}{rgb}{1.000000,1.000000,1.000000}%
\pgfsetstrokecolor{currentstroke}%
\pgfsetdash{}{0pt}%
\pgfpathmoveto{\pgfqpoint{5.476507in}{4.174809in}}%
\pgfpathcurveto{\pgfqpoint{5.487557in}{4.174809in}}{\pgfqpoint{5.498156in}{4.179200in}}{\pgfqpoint{5.505970in}{4.187013in}}%
\pgfpathcurveto{\pgfqpoint{5.513783in}{4.194827in}}{\pgfqpoint{5.518173in}{4.205426in}}{\pgfqpoint{5.518173in}{4.216476in}}%
\pgfpathcurveto{\pgfqpoint{5.518173in}{4.227526in}}{\pgfqpoint{5.513783in}{4.238125in}}{\pgfqpoint{5.505970in}{4.245939in}}%
\pgfpathcurveto{\pgfqpoint{5.498156in}{4.253752in}}{\pgfqpoint{5.487557in}{4.258143in}}{\pgfqpoint{5.476507in}{4.258143in}}%
\pgfpathcurveto{\pgfqpoint{5.465457in}{4.258143in}}{\pgfqpoint{5.454858in}{4.253752in}}{\pgfqpoint{5.447044in}{4.245939in}}%
\pgfpathcurveto{\pgfqpoint{5.439230in}{4.238125in}}{\pgfqpoint{5.434840in}{4.227526in}}{\pgfqpoint{5.434840in}{4.216476in}}%
\pgfpathcurveto{\pgfqpoint{5.434840in}{4.205426in}}{\pgfqpoint{5.439230in}{4.194827in}}{\pgfqpoint{5.447044in}{4.187013in}}%
\pgfpathcurveto{\pgfqpoint{5.454858in}{4.179200in}}{\pgfqpoint{5.465457in}{4.174809in}}{\pgfqpoint{5.476507in}{4.174809in}}%
\pgfpathclose%
\pgfusepath{stroke,fill}%
\end{pgfscope}%
\begin{pgfscope}%
\pgfpathrectangle{\pgfqpoint{0.481978in}{0.331635in}}{\pgfqpoint{9.300000in}{7.700000in}}%
\pgfusepath{clip}%
\pgfsetbuttcap%
\pgfsetroundjoin%
\definecolor{currentfill}{rgb}{1.000000,0.705882,0.509804}%
\pgfsetfillcolor{currentfill}%
\pgfsetlinewidth{0.481800pt}%
\definecolor{currentstroke}{rgb}{1.000000,1.000000,1.000000}%
\pgfsetstrokecolor{currentstroke}%
\pgfsetdash{}{0pt}%
\pgfpathmoveto{\pgfqpoint{7.379869in}{3.247695in}}%
\pgfpathcurveto{\pgfqpoint{7.390920in}{3.247695in}}{\pgfqpoint{7.401519in}{3.252085in}}{\pgfqpoint{7.409332in}{3.259899in}}%
\pgfpathcurveto{\pgfqpoint{7.417146in}{3.267712in}}{\pgfqpoint{7.421536in}{3.278311in}}{\pgfqpoint{7.421536in}{3.289361in}}%
\pgfpathcurveto{\pgfqpoint{7.421536in}{3.300412in}}{\pgfqpoint{7.417146in}{3.311011in}}{\pgfqpoint{7.409332in}{3.318824in}}%
\pgfpathcurveto{\pgfqpoint{7.401519in}{3.326638in}}{\pgfqpoint{7.390920in}{3.331028in}}{\pgfqpoint{7.379869in}{3.331028in}}%
\pgfpathcurveto{\pgfqpoint{7.368819in}{3.331028in}}{\pgfqpoint{7.358220in}{3.326638in}}{\pgfqpoint{7.350407in}{3.318824in}}%
\pgfpathcurveto{\pgfqpoint{7.342593in}{3.311011in}}{\pgfqpoint{7.338203in}{3.300412in}}{\pgfqpoint{7.338203in}{3.289361in}}%
\pgfpathcurveto{\pgfqpoint{7.338203in}{3.278311in}}{\pgfqpoint{7.342593in}{3.267712in}}{\pgfqpoint{7.350407in}{3.259899in}}%
\pgfpathcurveto{\pgfqpoint{7.358220in}{3.252085in}}{\pgfqpoint{7.368819in}{3.247695in}}{\pgfqpoint{7.379869in}{3.247695in}}%
\pgfpathclose%
\pgfusepath{stroke,fill}%
\end{pgfscope}%
\begin{pgfscope}%
\pgfpathrectangle{\pgfqpoint{0.481978in}{0.331635in}}{\pgfqpoint{9.300000in}{7.700000in}}%
\pgfusepath{clip}%
\pgfsetbuttcap%
\pgfsetroundjoin%
\definecolor{currentfill}{rgb}{1.000000,0.705882,0.509804}%
\pgfsetfillcolor{currentfill}%
\pgfsetlinewidth{0.481800pt}%
\definecolor{currentstroke}{rgb}{1.000000,1.000000,1.000000}%
\pgfsetstrokecolor{currentstroke}%
\pgfsetdash{}{0pt}%
\pgfpathmoveto{\pgfqpoint{6.352024in}{2.733115in}}%
\pgfpathcurveto{\pgfqpoint{6.363074in}{2.733115in}}{\pgfqpoint{6.373673in}{2.737505in}}{\pgfqpoint{6.381486in}{2.745318in}}%
\pgfpathcurveto{\pgfqpoint{6.389300in}{2.753132in}}{\pgfqpoint{6.393690in}{2.763731in}}{\pgfqpoint{6.393690in}{2.774781in}}%
\pgfpathcurveto{\pgfqpoint{6.393690in}{2.785831in}}{\pgfqpoint{6.389300in}{2.796430in}}{\pgfqpoint{6.381486in}{2.804244in}}%
\pgfpathcurveto{\pgfqpoint{6.373673in}{2.812058in}}{\pgfqpoint{6.363074in}{2.816448in}}{\pgfqpoint{6.352024in}{2.816448in}}%
\pgfpathcurveto{\pgfqpoint{6.340974in}{2.816448in}}{\pgfqpoint{6.330375in}{2.812058in}}{\pgfqpoint{6.322561in}{2.804244in}}%
\pgfpathcurveto{\pgfqpoint{6.314747in}{2.796430in}}{\pgfqpoint{6.310357in}{2.785831in}}{\pgfqpoint{6.310357in}{2.774781in}}%
\pgfpathcurveto{\pgfqpoint{6.310357in}{2.763731in}}{\pgfqpoint{6.314747in}{2.753132in}}{\pgfqpoint{6.322561in}{2.745318in}}%
\pgfpathcurveto{\pgfqpoint{6.330375in}{2.737505in}}{\pgfqpoint{6.340974in}{2.733115in}}{\pgfqpoint{6.352024in}{2.733115in}}%
\pgfpathclose%
\pgfusepath{stroke,fill}%
\end{pgfscope}%
\begin{pgfscope}%
\pgfpathrectangle{\pgfqpoint{0.481978in}{0.331635in}}{\pgfqpoint{9.300000in}{7.700000in}}%
\pgfusepath{clip}%
\pgfsetbuttcap%
\pgfsetroundjoin%
\definecolor{currentfill}{rgb}{1.000000,0.705882,0.509804}%
\pgfsetfillcolor{currentfill}%
\pgfsetlinewidth{0.481800pt}%
\definecolor{currentstroke}{rgb}{1.000000,1.000000,1.000000}%
\pgfsetstrokecolor{currentstroke}%
\pgfsetdash{}{0pt}%
\pgfpathmoveto{\pgfqpoint{7.010012in}{6.613365in}}%
\pgfpathcurveto{\pgfqpoint{7.021062in}{6.613365in}}{\pgfqpoint{7.031662in}{6.617755in}}{\pgfqpoint{7.039475in}{6.625569in}}%
\pgfpathcurveto{\pgfqpoint{7.047289in}{6.633382in}}{\pgfqpoint{7.051679in}{6.643981in}}{\pgfqpoint{7.051679in}{6.655032in}}%
\pgfpathcurveto{\pgfqpoint{7.051679in}{6.666082in}}{\pgfqpoint{7.047289in}{6.676681in}}{\pgfqpoint{7.039475in}{6.684494in}}%
\pgfpathcurveto{\pgfqpoint{7.031662in}{6.692308in}}{\pgfqpoint{7.021062in}{6.696698in}}{\pgfqpoint{7.010012in}{6.696698in}}%
\pgfpathcurveto{\pgfqpoint{6.998962in}{6.696698in}}{\pgfqpoint{6.988363in}{6.692308in}}{\pgfqpoint{6.980550in}{6.684494in}}%
\pgfpathcurveto{\pgfqpoint{6.972736in}{6.676681in}}{\pgfqpoint{6.968346in}{6.666082in}}{\pgfqpoint{6.968346in}{6.655032in}}%
\pgfpathcurveto{\pgfqpoint{6.968346in}{6.643981in}}{\pgfqpoint{6.972736in}{6.633382in}}{\pgfqpoint{6.980550in}{6.625569in}}%
\pgfpathcurveto{\pgfqpoint{6.988363in}{6.617755in}}{\pgfqpoint{6.998962in}{6.613365in}}{\pgfqpoint{7.010012in}{6.613365in}}%
\pgfpathclose%
\pgfusepath{stroke,fill}%
\end{pgfscope}%
\begin{pgfscope}%
\pgfpathrectangle{\pgfqpoint{0.481978in}{0.331635in}}{\pgfqpoint{9.300000in}{7.700000in}}%
\pgfusepath{clip}%
\pgfsetbuttcap%
\pgfsetroundjoin%
\definecolor{currentfill}{rgb}{1.000000,0.705882,0.509804}%
\pgfsetfillcolor{currentfill}%
\pgfsetlinewidth{0.481800pt}%
\definecolor{currentstroke}{rgb}{1.000000,1.000000,1.000000}%
\pgfsetstrokecolor{currentstroke}%
\pgfsetdash{}{0pt}%
\pgfpathmoveto{\pgfqpoint{4.957439in}{2.224376in}}%
\pgfpathcurveto{\pgfqpoint{4.968489in}{2.224376in}}{\pgfqpoint{4.979088in}{2.228766in}}{\pgfqpoint{4.986902in}{2.236580in}}%
\pgfpathcurveto{\pgfqpoint{4.994715in}{2.244394in}}{\pgfqpoint{4.999106in}{2.254993in}}{\pgfqpoint{4.999106in}{2.266043in}}%
\pgfpathcurveto{\pgfqpoint{4.999106in}{2.277093in}}{\pgfqpoint{4.994715in}{2.287692in}}{\pgfqpoint{4.986902in}{2.295506in}}%
\pgfpathcurveto{\pgfqpoint{4.979088in}{2.303319in}}{\pgfqpoint{4.968489in}{2.307710in}}{\pgfqpoint{4.957439in}{2.307710in}}%
\pgfpathcurveto{\pgfqpoint{4.946389in}{2.307710in}}{\pgfqpoint{4.935790in}{2.303319in}}{\pgfqpoint{4.927976in}{2.295506in}}%
\pgfpathcurveto{\pgfqpoint{4.920163in}{2.287692in}}{\pgfqpoint{4.915772in}{2.277093in}}{\pgfqpoint{4.915772in}{2.266043in}}%
\pgfpathcurveto{\pgfqpoint{4.915772in}{2.254993in}}{\pgfqpoint{4.920163in}{2.244394in}}{\pgfqpoint{4.927976in}{2.236580in}}%
\pgfpathcurveto{\pgfqpoint{4.935790in}{2.228766in}}{\pgfqpoint{4.946389in}{2.224376in}}{\pgfqpoint{4.957439in}{2.224376in}}%
\pgfpathclose%
\pgfusepath{stroke,fill}%
\end{pgfscope}%
\begin{pgfscope}%
\pgfpathrectangle{\pgfqpoint{0.481978in}{0.331635in}}{\pgfqpoint{9.300000in}{7.700000in}}%
\pgfusepath{clip}%
\pgfsetbuttcap%
\pgfsetroundjoin%
\definecolor{currentfill}{rgb}{1.000000,0.705882,0.509804}%
\pgfsetfillcolor{currentfill}%
\pgfsetlinewidth{0.481800pt}%
\definecolor{currentstroke}{rgb}{1.000000,1.000000,1.000000}%
\pgfsetstrokecolor{currentstroke}%
\pgfsetdash{}{0pt}%
\pgfpathmoveto{\pgfqpoint{1.938901in}{5.922513in}}%
\pgfpathcurveto{\pgfqpoint{1.949952in}{5.922513in}}{\pgfqpoint{1.960551in}{5.926903in}}{\pgfqpoint{1.968364in}{5.934716in}}%
\pgfpathcurveto{\pgfqpoint{1.976178in}{5.942530in}}{\pgfqpoint{1.980568in}{5.953129in}}{\pgfqpoint{1.980568in}{5.964179in}}%
\pgfpathcurveto{\pgfqpoint{1.980568in}{5.975229in}}{\pgfqpoint{1.976178in}{5.985828in}}{\pgfqpoint{1.968364in}{5.993642in}}%
\pgfpathcurveto{\pgfqpoint{1.960551in}{6.001456in}}{\pgfqpoint{1.949952in}{6.005846in}}{\pgfqpoint{1.938901in}{6.005846in}}%
\pgfpathcurveto{\pgfqpoint{1.927851in}{6.005846in}}{\pgfqpoint{1.917252in}{6.001456in}}{\pgfqpoint{1.909439in}{5.993642in}}%
\pgfpathcurveto{\pgfqpoint{1.901625in}{5.985828in}}{\pgfqpoint{1.897235in}{5.975229in}}{\pgfqpoint{1.897235in}{5.964179in}}%
\pgfpathcurveto{\pgfqpoint{1.897235in}{5.953129in}}{\pgfqpoint{1.901625in}{5.942530in}}{\pgfqpoint{1.909439in}{5.934716in}}%
\pgfpathcurveto{\pgfqpoint{1.917252in}{5.926903in}}{\pgfqpoint{1.927851in}{5.922513in}}{\pgfqpoint{1.938901in}{5.922513in}}%
\pgfpathclose%
\pgfusepath{stroke,fill}%
\end{pgfscope}%
\begin{pgfscope}%
\pgfpathrectangle{\pgfqpoint{0.481978in}{0.331635in}}{\pgfqpoint{9.300000in}{7.700000in}}%
\pgfusepath{clip}%
\pgfsetbuttcap%
\pgfsetroundjoin%
\definecolor{currentfill}{rgb}{1.000000,0.705882,0.509804}%
\pgfsetfillcolor{currentfill}%
\pgfsetlinewidth{0.481800pt}%
\definecolor{currentstroke}{rgb}{1.000000,1.000000,1.000000}%
\pgfsetstrokecolor{currentstroke}%
\pgfsetdash{}{0pt}%
\pgfpathmoveto{\pgfqpoint{6.653857in}{6.971094in}}%
\pgfpathcurveto{\pgfqpoint{6.664907in}{6.971094in}}{\pgfqpoint{6.675506in}{6.975485in}}{\pgfqpoint{6.683320in}{6.983298in}}%
\pgfpathcurveto{\pgfqpoint{6.691133in}{6.991112in}}{\pgfqpoint{6.695524in}{7.001711in}}{\pgfqpoint{6.695524in}{7.012761in}}%
\pgfpathcurveto{\pgfqpoint{6.695524in}{7.023811in}}{\pgfqpoint{6.691133in}{7.034410in}}{\pgfqpoint{6.683320in}{7.042224in}}%
\pgfpathcurveto{\pgfqpoint{6.675506in}{7.050037in}}{\pgfqpoint{6.664907in}{7.054428in}}{\pgfqpoint{6.653857in}{7.054428in}}%
\pgfpathcurveto{\pgfqpoint{6.642807in}{7.054428in}}{\pgfqpoint{6.632208in}{7.050037in}}{\pgfqpoint{6.624394in}{7.042224in}}%
\pgfpathcurveto{\pgfqpoint{6.616581in}{7.034410in}}{\pgfqpoint{6.612190in}{7.023811in}}{\pgfqpoint{6.612190in}{7.012761in}}%
\pgfpathcurveto{\pgfqpoint{6.612190in}{7.001711in}}{\pgfqpoint{6.616581in}{6.991112in}}{\pgfqpoint{6.624394in}{6.983298in}}%
\pgfpathcurveto{\pgfqpoint{6.632208in}{6.975485in}}{\pgfqpoint{6.642807in}{6.971094in}}{\pgfqpoint{6.653857in}{6.971094in}}%
\pgfpathclose%
\pgfusepath{stroke,fill}%
\end{pgfscope}%
\begin{pgfscope}%
\pgfpathrectangle{\pgfqpoint{0.481978in}{0.331635in}}{\pgfqpoint{9.300000in}{7.700000in}}%
\pgfusepath{clip}%
\pgfsetbuttcap%
\pgfsetroundjoin%
\definecolor{currentfill}{rgb}{1.000000,0.705882,0.509804}%
\pgfsetfillcolor{currentfill}%
\pgfsetlinewidth{0.481800pt}%
\definecolor{currentstroke}{rgb}{1.000000,1.000000,1.000000}%
\pgfsetstrokecolor{currentstroke}%
\pgfsetdash{}{0pt}%
\pgfpathmoveto{\pgfqpoint{7.825213in}{7.117074in}}%
\pgfpathcurveto{\pgfqpoint{7.836263in}{7.117074in}}{\pgfqpoint{7.846862in}{7.121465in}}{\pgfqpoint{7.854676in}{7.129278in}}%
\pgfpathcurveto{\pgfqpoint{7.862489in}{7.137092in}}{\pgfqpoint{7.866880in}{7.147691in}}{\pgfqpoint{7.866880in}{7.158741in}}%
\pgfpathcurveto{\pgfqpoint{7.866880in}{7.169791in}}{\pgfqpoint{7.862489in}{7.180390in}}{\pgfqpoint{7.854676in}{7.188204in}}%
\pgfpathcurveto{\pgfqpoint{7.846862in}{7.196017in}}{\pgfqpoint{7.836263in}{7.200408in}}{\pgfqpoint{7.825213in}{7.200408in}}%
\pgfpathcurveto{\pgfqpoint{7.814163in}{7.200408in}}{\pgfqpoint{7.803564in}{7.196017in}}{\pgfqpoint{7.795750in}{7.188204in}}%
\pgfpathcurveto{\pgfqpoint{7.787937in}{7.180390in}}{\pgfqpoint{7.783546in}{7.169791in}}{\pgfqpoint{7.783546in}{7.158741in}}%
\pgfpathcurveto{\pgfqpoint{7.783546in}{7.147691in}}{\pgfqpoint{7.787937in}{7.137092in}}{\pgfqpoint{7.795750in}{7.129278in}}%
\pgfpathcurveto{\pgfqpoint{7.803564in}{7.121465in}}{\pgfqpoint{7.814163in}{7.117074in}}{\pgfqpoint{7.825213in}{7.117074in}}%
\pgfpathclose%
\pgfusepath{stroke,fill}%
\end{pgfscope}%
\begin{pgfscope}%
\pgfpathrectangle{\pgfqpoint{0.481978in}{0.331635in}}{\pgfqpoint{9.300000in}{7.700000in}}%
\pgfusepath{clip}%
\pgfsetbuttcap%
\pgfsetroundjoin%
\definecolor{currentfill}{rgb}{1.000000,0.705882,0.509804}%
\pgfsetfillcolor{currentfill}%
\pgfsetlinewidth{0.481800pt}%
\definecolor{currentstroke}{rgb}{1.000000,1.000000,1.000000}%
\pgfsetstrokecolor{currentstroke}%
\pgfsetdash{}{0pt}%
\pgfpathmoveto{\pgfqpoint{4.408915in}{2.505647in}}%
\pgfpathcurveto{\pgfqpoint{4.419965in}{2.505647in}}{\pgfqpoint{4.430564in}{2.510037in}}{\pgfqpoint{4.438378in}{2.517850in}}%
\pgfpathcurveto{\pgfqpoint{4.446191in}{2.525664in}}{\pgfqpoint{4.450582in}{2.536263in}}{\pgfqpoint{4.450582in}{2.547313in}}%
\pgfpathcurveto{\pgfqpoint{4.450582in}{2.558363in}}{\pgfqpoint{4.446191in}{2.568962in}}{\pgfqpoint{4.438378in}{2.576776in}}%
\pgfpathcurveto{\pgfqpoint{4.430564in}{2.584590in}}{\pgfqpoint{4.419965in}{2.588980in}}{\pgfqpoint{4.408915in}{2.588980in}}%
\pgfpathcurveto{\pgfqpoint{4.397865in}{2.588980in}}{\pgfqpoint{4.387266in}{2.584590in}}{\pgfqpoint{4.379452in}{2.576776in}}%
\pgfpathcurveto{\pgfqpoint{4.371638in}{2.568962in}}{\pgfqpoint{4.367248in}{2.558363in}}{\pgfqpoint{4.367248in}{2.547313in}}%
\pgfpathcurveto{\pgfqpoint{4.367248in}{2.536263in}}{\pgfqpoint{4.371638in}{2.525664in}}{\pgfqpoint{4.379452in}{2.517850in}}%
\pgfpathcurveto{\pgfqpoint{4.387266in}{2.510037in}}{\pgfqpoint{4.397865in}{2.505647in}}{\pgfqpoint{4.408915in}{2.505647in}}%
\pgfpathclose%
\pgfusepath{stroke,fill}%
\end{pgfscope}%
\begin{pgfscope}%
\pgfpathrectangle{\pgfqpoint{0.481978in}{0.331635in}}{\pgfqpoint{9.300000in}{7.700000in}}%
\pgfusepath{clip}%
\pgfsetbuttcap%
\pgfsetroundjoin%
\definecolor{currentfill}{rgb}{1.000000,0.705882,0.509804}%
\pgfsetfillcolor{currentfill}%
\pgfsetlinewidth{0.481800pt}%
\definecolor{currentstroke}{rgb}{1.000000,1.000000,1.000000}%
\pgfsetstrokecolor{currentstroke}%
\pgfsetdash{}{0pt}%
\pgfpathmoveto{\pgfqpoint{0.904705in}{3.119494in}}%
\pgfpathcurveto{\pgfqpoint{0.915755in}{3.119494in}}{\pgfqpoint{0.926354in}{3.123885in}}{\pgfqpoint{0.934168in}{3.131698in}}%
\pgfpathcurveto{\pgfqpoint{0.941982in}{3.139512in}}{\pgfqpoint{0.946372in}{3.150111in}}{\pgfqpoint{0.946372in}{3.161161in}}%
\pgfpathcurveto{\pgfqpoint{0.946372in}{3.172211in}}{\pgfqpoint{0.941982in}{3.182810in}}{\pgfqpoint{0.934168in}{3.190624in}}%
\pgfpathcurveto{\pgfqpoint{0.926354in}{3.198437in}}{\pgfqpoint{0.915755in}{3.202828in}}{\pgfqpoint{0.904705in}{3.202828in}}%
\pgfpathcurveto{\pgfqpoint{0.893655in}{3.202828in}}{\pgfqpoint{0.883056in}{3.198437in}}{\pgfqpoint{0.875242in}{3.190624in}}%
\pgfpathcurveto{\pgfqpoint{0.867429in}{3.182810in}}{\pgfqpoint{0.863039in}{3.172211in}}{\pgfqpoint{0.863039in}{3.161161in}}%
\pgfpathcurveto{\pgfqpoint{0.863039in}{3.150111in}}{\pgfqpoint{0.867429in}{3.139512in}}{\pgfqpoint{0.875242in}{3.131698in}}%
\pgfpathcurveto{\pgfqpoint{0.883056in}{3.123885in}}{\pgfqpoint{0.893655in}{3.119494in}}{\pgfqpoint{0.904705in}{3.119494in}}%
\pgfpathclose%
\pgfusepath{stroke,fill}%
\end{pgfscope}%
\begin{pgfscope}%
\pgfpathrectangle{\pgfqpoint{0.481978in}{0.331635in}}{\pgfqpoint{9.300000in}{7.700000in}}%
\pgfusepath{clip}%
\pgfsetbuttcap%
\pgfsetroundjoin%
\definecolor{currentfill}{rgb}{1.000000,0.705882,0.509804}%
\pgfsetfillcolor{currentfill}%
\pgfsetlinewidth{0.481800pt}%
\definecolor{currentstroke}{rgb}{1.000000,1.000000,1.000000}%
\pgfsetstrokecolor{currentstroke}%
\pgfsetdash{}{0pt}%
\pgfpathmoveto{\pgfqpoint{8.224572in}{3.304265in}}%
\pgfpathcurveto{\pgfqpoint{8.235623in}{3.304265in}}{\pgfqpoint{8.246222in}{3.308655in}}{\pgfqpoint{8.254035in}{3.316469in}}%
\pgfpathcurveto{\pgfqpoint{8.261849in}{3.324282in}}{\pgfqpoint{8.266239in}{3.334881in}}{\pgfqpoint{8.266239in}{3.345931in}}%
\pgfpathcurveto{\pgfqpoint{8.266239in}{3.356981in}}{\pgfqpoint{8.261849in}{3.367580in}}{\pgfqpoint{8.254035in}{3.375394in}}%
\pgfpathcurveto{\pgfqpoint{8.246222in}{3.383208in}}{\pgfqpoint{8.235623in}{3.387598in}}{\pgfqpoint{8.224572in}{3.387598in}}%
\pgfpathcurveto{\pgfqpoint{8.213522in}{3.387598in}}{\pgfqpoint{8.202923in}{3.383208in}}{\pgfqpoint{8.195110in}{3.375394in}}%
\pgfpathcurveto{\pgfqpoint{8.187296in}{3.367580in}}{\pgfqpoint{8.182906in}{3.356981in}}{\pgfqpoint{8.182906in}{3.345931in}}%
\pgfpathcurveto{\pgfqpoint{8.182906in}{3.334881in}}{\pgfqpoint{8.187296in}{3.324282in}}{\pgfqpoint{8.195110in}{3.316469in}}%
\pgfpathcurveto{\pgfqpoint{8.202923in}{3.308655in}}{\pgfqpoint{8.213522in}{3.304265in}}{\pgfqpoint{8.224572in}{3.304265in}}%
\pgfpathclose%
\pgfusepath{stroke,fill}%
\end{pgfscope}%
\begin{pgfscope}%
\pgfpathrectangle{\pgfqpoint{0.481978in}{0.331635in}}{\pgfqpoint{9.300000in}{7.700000in}}%
\pgfusepath{clip}%
\pgfsetbuttcap%
\pgfsetroundjoin%
\definecolor{currentfill}{rgb}{0.631373,0.788235,0.956863}%
\pgfsetfillcolor{currentfill}%
\pgfsetlinewidth{1.003750pt}%
\definecolor{currentstroke}{rgb}{0.631373,0.788235,0.956863}%
\pgfsetstrokecolor{currentstroke}%
\pgfsetdash{}{0pt}%
\pgfsys@defobject{currentmarker}{\pgfqpoint{-0.041667in}{-0.041667in}}{\pgfqpoint{0.041667in}{0.041667in}}{%
\pgfpathmoveto{\pgfqpoint{0.000000in}{-0.041667in}}%
\pgfpathcurveto{\pgfqpoint{0.011050in}{-0.041667in}}{\pgfqpoint{0.021649in}{-0.037276in}}{\pgfqpoint{0.029463in}{-0.029463in}}%
\pgfpathcurveto{\pgfqpoint{0.037276in}{-0.021649in}}{\pgfqpoint{0.041667in}{-0.011050in}}{\pgfqpoint{0.041667in}{0.000000in}}%
\pgfpathcurveto{\pgfqpoint{0.041667in}{0.011050in}}{\pgfqpoint{0.037276in}{0.021649in}}{\pgfqpoint{0.029463in}{0.029463in}}%
\pgfpathcurveto{\pgfqpoint{0.021649in}{0.037276in}}{\pgfqpoint{0.011050in}{0.041667in}}{\pgfqpoint{0.000000in}{0.041667in}}%
\pgfpathcurveto{\pgfqpoint{-0.011050in}{0.041667in}}{\pgfqpoint{-0.021649in}{0.037276in}}{\pgfqpoint{-0.029463in}{0.029463in}}%
\pgfpathcurveto{\pgfqpoint{-0.037276in}{0.021649in}}{\pgfqpoint{-0.041667in}{0.011050in}}{\pgfqpoint{-0.041667in}{0.000000in}}%
\pgfpathcurveto{\pgfqpoint{-0.041667in}{-0.011050in}}{\pgfqpoint{-0.037276in}{-0.021649in}}{\pgfqpoint{-0.029463in}{-0.029463in}}%
\pgfpathcurveto{\pgfqpoint{-0.021649in}{-0.037276in}}{\pgfqpoint{-0.011050in}{-0.041667in}}{\pgfqpoint{0.000000in}{-0.041667in}}%
\pgfpathclose%
\pgfusepath{stroke,fill}%
}%
\end{pgfscope}%
\begin{pgfscope}%
\pgfpathrectangle{\pgfqpoint{0.481978in}{0.331635in}}{\pgfqpoint{9.300000in}{7.700000in}}%
\pgfusepath{clip}%
\pgfsetbuttcap%
\pgfsetroundjoin%
\definecolor{currentfill}{rgb}{1.000000,0.705882,0.509804}%
\pgfsetfillcolor{currentfill}%
\pgfsetlinewidth{1.003750pt}%
\definecolor{currentstroke}{rgb}{1.000000,0.705882,0.509804}%
\pgfsetstrokecolor{currentstroke}%
\pgfsetdash{}{0pt}%
\pgfsys@defobject{currentmarker}{\pgfqpoint{-0.041667in}{-0.041667in}}{\pgfqpoint{0.041667in}{0.041667in}}{%
\pgfpathmoveto{\pgfqpoint{0.000000in}{-0.041667in}}%
\pgfpathcurveto{\pgfqpoint{0.011050in}{-0.041667in}}{\pgfqpoint{0.021649in}{-0.037276in}}{\pgfqpoint{0.029463in}{-0.029463in}}%
\pgfpathcurveto{\pgfqpoint{0.037276in}{-0.021649in}}{\pgfqpoint{0.041667in}{-0.011050in}}{\pgfqpoint{0.041667in}{0.000000in}}%
\pgfpathcurveto{\pgfqpoint{0.041667in}{0.011050in}}{\pgfqpoint{0.037276in}{0.021649in}}{\pgfqpoint{0.029463in}{0.029463in}}%
\pgfpathcurveto{\pgfqpoint{0.021649in}{0.037276in}}{\pgfqpoint{0.011050in}{0.041667in}}{\pgfqpoint{0.000000in}{0.041667in}}%
\pgfpathcurveto{\pgfqpoint{-0.011050in}{0.041667in}}{\pgfqpoint{-0.021649in}{0.037276in}}{\pgfqpoint{-0.029463in}{0.029463in}}%
\pgfpathcurveto{\pgfqpoint{-0.037276in}{0.021649in}}{\pgfqpoint{-0.041667in}{0.011050in}}{\pgfqpoint{-0.041667in}{0.000000in}}%
\pgfpathcurveto{\pgfqpoint{-0.041667in}{-0.011050in}}{\pgfqpoint{-0.037276in}{-0.021649in}}{\pgfqpoint{-0.029463in}{-0.029463in}}%
\pgfpathcurveto{\pgfqpoint{-0.021649in}{-0.037276in}}{\pgfqpoint{-0.011050in}{-0.041667in}}{\pgfqpoint{0.000000in}{-0.041667in}}%
\pgfpathclose%
\pgfusepath{stroke,fill}%
}%
\end{pgfscope}%
\begin{pgfscope}%
\pgfsetbuttcap%
\pgfsetroundjoin%
\definecolor{currentfill}{rgb}{0.000000,0.000000,0.000000}%
\pgfsetfillcolor{currentfill}%
\pgfsetlinewidth{0.803000pt}%
\definecolor{currentstroke}{rgb}{0.000000,0.000000,0.000000}%
\pgfsetstrokecolor{currentstroke}%
\pgfsetdash{}{0pt}%
\pgfsys@defobject{currentmarker}{\pgfqpoint{0.000000in}{-0.048611in}}{\pgfqpoint{0.000000in}{0.000000in}}{%
\pgfpathmoveto{\pgfqpoint{0.000000in}{0.000000in}}%
\pgfpathlineto{\pgfqpoint{0.000000in}{-0.048611in}}%
\pgfusepath{stroke,fill}%
}%
\begin{pgfscope}%
\pgfsys@transformshift{0.741718in}{0.331635in}%
\pgfsys@useobject{currentmarker}{}%
\end{pgfscope}%
\end{pgfscope}%
\begin{pgfscope}%
\definecolor{textcolor}{rgb}{0.000000,0.000000,0.000000}%
\pgfsetstrokecolor{textcolor}%
\pgfsetfillcolor{textcolor}%
\pgftext[x=0.741718in,y=0.234413in,,top]{\color{textcolor}\sffamily\fontsize{10.000000}{12.000000}\selectfont \ensuremath{-}60}%
\end{pgfscope}%
\begin{pgfscope}%
\pgfsetbuttcap%
\pgfsetroundjoin%
\definecolor{currentfill}{rgb}{0.000000,0.000000,0.000000}%
\pgfsetfillcolor{currentfill}%
\pgfsetlinewidth{0.803000pt}%
\definecolor{currentstroke}{rgb}{0.000000,0.000000,0.000000}%
\pgfsetstrokecolor{currentstroke}%
\pgfsetdash{}{0pt}%
\pgfsys@defobject{currentmarker}{\pgfqpoint{0.000000in}{-0.048611in}}{\pgfqpoint{0.000000in}{0.000000in}}{%
\pgfpathmoveto{\pgfqpoint{0.000000in}{0.000000in}}%
\pgfpathlineto{\pgfqpoint{0.000000in}{-0.048611in}}%
\pgfusepath{stroke,fill}%
}%
\begin{pgfscope}%
\pgfsys@transformshift{2.361715in}{0.331635in}%
\pgfsys@useobject{currentmarker}{}%
\end{pgfscope}%
\end{pgfscope}%
\begin{pgfscope}%
\definecolor{textcolor}{rgb}{0.000000,0.000000,0.000000}%
\pgfsetstrokecolor{textcolor}%
\pgfsetfillcolor{textcolor}%
\pgftext[x=2.361715in,y=0.234413in,,top]{\color{textcolor}\sffamily\fontsize{10.000000}{12.000000}\selectfont \ensuremath{-}40}%
\end{pgfscope}%
\begin{pgfscope}%
\pgfsetbuttcap%
\pgfsetroundjoin%
\definecolor{currentfill}{rgb}{0.000000,0.000000,0.000000}%
\pgfsetfillcolor{currentfill}%
\pgfsetlinewidth{0.803000pt}%
\definecolor{currentstroke}{rgb}{0.000000,0.000000,0.000000}%
\pgfsetstrokecolor{currentstroke}%
\pgfsetdash{}{0pt}%
\pgfsys@defobject{currentmarker}{\pgfqpoint{0.000000in}{-0.048611in}}{\pgfqpoint{0.000000in}{0.000000in}}{%
\pgfpathmoveto{\pgfqpoint{0.000000in}{0.000000in}}%
\pgfpathlineto{\pgfqpoint{0.000000in}{-0.048611in}}%
\pgfusepath{stroke,fill}%
}%
\begin{pgfscope}%
\pgfsys@transformshift{3.981712in}{0.331635in}%
\pgfsys@useobject{currentmarker}{}%
\end{pgfscope}%
\end{pgfscope}%
\begin{pgfscope}%
\definecolor{textcolor}{rgb}{0.000000,0.000000,0.000000}%
\pgfsetstrokecolor{textcolor}%
\pgfsetfillcolor{textcolor}%
\pgftext[x=3.981712in,y=0.234413in,,top]{\color{textcolor}\sffamily\fontsize{10.000000}{12.000000}\selectfont \ensuremath{-}20}%
\end{pgfscope}%
\begin{pgfscope}%
\pgfsetbuttcap%
\pgfsetroundjoin%
\definecolor{currentfill}{rgb}{0.000000,0.000000,0.000000}%
\pgfsetfillcolor{currentfill}%
\pgfsetlinewidth{0.803000pt}%
\definecolor{currentstroke}{rgb}{0.000000,0.000000,0.000000}%
\pgfsetstrokecolor{currentstroke}%
\pgfsetdash{}{0pt}%
\pgfsys@defobject{currentmarker}{\pgfqpoint{0.000000in}{-0.048611in}}{\pgfqpoint{0.000000in}{0.000000in}}{%
\pgfpathmoveto{\pgfqpoint{0.000000in}{0.000000in}}%
\pgfpathlineto{\pgfqpoint{0.000000in}{-0.048611in}}%
\pgfusepath{stroke,fill}%
}%
\begin{pgfscope}%
\pgfsys@transformshift{5.601708in}{0.331635in}%
\pgfsys@useobject{currentmarker}{}%
\end{pgfscope}%
\end{pgfscope}%
\begin{pgfscope}%
\definecolor{textcolor}{rgb}{0.000000,0.000000,0.000000}%
\pgfsetstrokecolor{textcolor}%
\pgfsetfillcolor{textcolor}%
\pgftext[x=5.601708in,y=0.234413in,,top]{\color{textcolor}\sffamily\fontsize{10.000000}{12.000000}\selectfont 0}%
\end{pgfscope}%
\begin{pgfscope}%
\pgfsetbuttcap%
\pgfsetroundjoin%
\definecolor{currentfill}{rgb}{0.000000,0.000000,0.000000}%
\pgfsetfillcolor{currentfill}%
\pgfsetlinewidth{0.803000pt}%
\definecolor{currentstroke}{rgb}{0.000000,0.000000,0.000000}%
\pgfsetstrokecolor{currentstroke}%
\pgfsetdash{}{0pt}%
\pgfsys@defobject{currentmarker}{\pgfqpoint{0.000000in}{-0.048611in}}{\pgfqpoint{0.000000in}{0.000000in}}{%
\pgfpathmoveto{\pgfqpoint{0.000000in}{0.000000in}}%
\pgfpathlineto{\pgfqpoint{0.000000in}{-0.048611in}}%
\pgfusepath{stroke,fill}%
}%
\begin{pgfscope}%
\pgfsys@transformshift{7.221705in}{0.331635in}%
\pgfsys@useobject{currentmarker}{}%
\end{pgfscope}%
\end{pgfscope}%
\begin{pgfscope}%
\definecolor{textcolor}{rgb}{0.000000,0.000000,0.000000}%
\pgfsetstrokecolor{textcolor}%
\pgfsetfillcolor{textcolor}%
\pgftext[x=7.221705in,y=0.234413in,,top]{\color{textcolor}\sffamily\fontsize{10.000000}{12.000000}\selectfont 20}%
\end{pgfscope}%
\begin{pgfscope}%
\pgfsetbuttcap%
\pgfsetroundjoin%
\definecolor{currentfill}{rgb}{0.000000,0.000000,0.000000}%
\pgfsetfillcolor{currentfill}%
\pgfsetlinewidth{0.803000pt}%
\definecolor{currentstroke}{rgb}{0.000000,0.000000,0.000000}%
\pgfsetstrokecolor{currentstroke}%
\pgfsetdash{}{0pt}%
\pgfsys@defobject{currentmarker}{\pgfqpoint{0.000000in}{-0.048611in}}{\pgfqpoint{0.000000in}{0.000000in}}{%
\pgfpathmoveto{\pgfqpoint{0.000000in}{0.000000in}}%
\pgfpathlineto{\pgfqpoint{0.000000in}{-0.048611in}}%
\pgfusepath{stroke,fill}%
}%
\begin{pgfscope}%
\pgfsys@transformshift{8.841702in}{0.331635in}%
\pgfsys@useobject{currentmarker}{}%
\end{pgfscope}%
\end{pgfscope}%
\begin{pgfscope}%
\definecolor{textcolor}{rgb}{0.000000,0.000000,0.000000}%
\pgfsetstrokecolor{textcolor}%
\pgfsetfillcolor{textcolor}%
\pgftext[x=8.841702in,y=0.234413in,,top]{\color{textcolor}\sffamily\fontsize{10.000000}{12.000000}\selectfont 40}%
\end{pgfscope}%
\begin{pgfscope}%
\pgfsetbuttcap%
\pgfsetroundjoin%
\definecolor{currentfill}{rgb}{0.000000,0.000000,0.000000}%
\pgfsetfillcolor{currentfill}%
\pgfsetlinewidth{0.803000pt}%
\definecolor{currentstroke}{rgb}{0.000000,0.000000,0.000000}%
\pgfsetstrokecolor{currentstroke}%
\pgfsetdash{}{0pt}%
\pgfsys@defobject{currentmarker}{\pgfqpoint{-0.048611in}{0.000000in}}{\pgfqpoint{-0.000000in}{0.000000in}}{%
\pgfpathmoveto{\pgfqpoint{-0.000000in}{0.000000in}}%
\pgfpathlineto{\pgfqpoint{-0.048611in}{0.000000in}}%
\pgfusepath{stroke,fill}%
}%
\begin{pgfscope}%
\pgfsys@transformshift{0.481978in}{0.879543in}%
\pgfsys@useobject{currentmarker}{}%
\end{pgfscope}%
\end{pgfscope}%
\begin{pgfscope}%
\definecolor{textcolor}{rgb}{0.000000,0.000000,0.000000}%
\pgfsetstrokecolor{textcolor}%
\pgfsetfillcolor{textcolor}%
\pgftext[x=0.100000in, y=0.826781in, left, base]{\color{textcolor}\sffamily\fontsize{10.000000}{12.000000}\selectfont \ensuremath{-}75}%
\end{pgfscope}%
\begin{pgfscope}%
\pgfsetbuttcap%
\pgfsetroundjoin%
\definecolor{currentfill}{rgb}{0.000000,0.000000,0.000000}%
\pgfsetfillcolor{currentfill}%
\pgfsetlinewidth{0.803000pt}%
\definecolor{currentstroke}{rgb}{0.000000,0.000000,0.000000}%
\pgfsetstrokecolor{currentstroke}%
\pgfsetdash{}{0pt}%
\pgfsys@defobject{currentmarker}{\pgfqpoint{-0.048611in}{0.000000in}}{\pgfqpoint{-0.000000in}{0.000000in}}{%
\pgfpathmoveto{\pgfqpoint{-0.000000in}{0.000000in}}%
\pgfpathlineto{\pgfqpoint{-0.048611in}{0.000000in}}%
\pgfusepath{stroke,fill}%
}%
\begin{pgfscope}%
\pgfsys@transformshift{0.481978in}{1.925520in}%
\pgfsys@useobject{currentmarker}{}%
\end{pgfscope}%
\end{pgfscope}%
\begin{pgfscope}%
\definecolor{textcolor}{rgb}{0.000000,0.000000,0.000000}%
\pgfsetstrokecolor{textcolor}%
\pgfsetfillcolor{textcolor}%
\pgftext[x=0.100000in, y=1.872759in, left, base]{\color{textcolor}\sffamily\fontsize{10.000000}{12.000000}\selectfont \ensuremath{-}50}%
\end{pgfscope}%
\begin{pgfscope}%
\pgfsetbuttcap%
\pgfsetroundjoin%
\definecolor{currentfill}{rgb}{0.000000,0.000000,0.000000}%
\pgfsetfillcolor{currentfill}%
\pgfsetlinewidth{0.803000pt}%
\definecolor{currentstroke}{rgb}{0.000000,0.000000,0.000000}%
\pgfsetstrokecolor{currentstroke}%
\pgfsetdash{}{0pt}%
\pgfsys@defobject{currentmarker}{\pgfqpoint{-0.048611in}{0.000000in}}{\pgfqpoint{-0.000000in}{0.000000in}}{%
\pgfpathmoveto{\pgfqpoint{-0.000000in}{0.000000in}}%
\pgfpathlineto{\pgfqpoint{-0.048611in}{0.000000in}}%
\pgfusepath{stroke,fill}%
}%
\begin{pgfscope}%
\pgfsys@transformshift{0.481978in}{2.971498in}%
\pgfsys@useobject{currentmarker}{}%
\end{pgfscope}%
\end{pgfscope}%
\begin{pgfscope}%
\definecolor{textcolor}{rgb}{0.000000,0.000000,0.000000}%
\pgfsetstrokecolor{textcolor}%
\pgfsetfillcolor{textcolor}%
\pgftext[x=0.100000in, y=2.918736in, left, base]{\color{textcolor}\sffamily\fontsize{10.000000}{12.000000}\selectfont \ensuremath{-}25}%
\end{pgfscope}%
\begin{pgfscope}%
\pgfsetbuttcap%
\pgfsetroundjoin%
\definecolor{currentfill}{rgb}{0.000000,0.000000,0.000000}%
\pgfsetfillcolor{currentfill}%
\pgfsetlinewidth{0.803000pt}%
\definecolor{currentstroke}{rgb}{0.000000,0.000000,0.000000}%
\pgfsetstrokecolor{currentstroke}%
\pgfsetdash{}{0pt}%
\pgfsys@defobject{currentmarker}{\pgfqpoint{-0.048611in}{0.000000in}}{\pgfqpoint{-0.000000in}{0.000000in}}{%
\pgfpathmoveto{\pgfqpoint{-0.000000in}{0.000000in}}%
\pgfpathlineto{\pgfqpoint{-0.048611in}{0.000000in}}%
\pgfusepath{stroke,fill}%
}%
\begin{pgfscope}%
\pgfsys@transformshift{0.481978in}{4.017475in}%
\pgfsys@useobject{currentmarker}{}%
\end{pgfscope}%
\end{pgfscope}%
\begin{pgfscope}%
\definecolor{textcolor}{rgb}{0.000000,0.000000,0.000000}%
\pgfsetstrokecolor{textcolor}%
\pgfsetfillcolor{textcolor}%
\pgftext[x=0.296390in, y=3.964714in, left, base]{\color{textcolor}\sffamily\fontsize{10.000000}{12.000000}\selectfont 0}%
\end{pgfscope}%
\begin{pgfscope}%
\pgfsetbuttcap%
\pgfsetroundjoin%
\definecolor{currentfill}{rgb}{0.000000,0.000000,0.000000}%
\pgfsetfillcolor{currentfill}%
\pgfsetlinewidth{0.803000pt}%
\definecolor{currentstroke}{rgb}{0.000000,0.000000,0.000000}%
\pgfsetstrokecolor{currentstroke}%
\pgfsetdash{}{0pt}%
\pgfsys@defobject{currentmarker}{\pgfqpoint{-0.048611in}{0.000000in}}{\pgfqpoint{-0.000000in}{0.000000in}}{%
\pgfpathmoveto{\pgfqpoint{-0.000000in}{0.000000in}}%
\pgfpathlineto{\pgfqpoint{-0.048611in}{0.000000in}}%
\pgfusepath{stroke,fill}%
}%
\begin{pgfscope}%
\pgfsys@transformshift{0.481978in}{5.063453in}%
\pgfsys@useobject{currentmarker}{}%
\end{pgfscope}%
\end{pgfscope}%
\begin{pgfscope}%
\definecolor{textcolor}{rgb}{0.000000,0.000000,0.000000}%
\pgfsetstrokecolor{textcolor}%
\pgfsetfillcolor{textcolor}%
\pgftext[x=0.208025in, y=5.010691in, left, base]{\color{textcolor}\sffamily\fontsize{10.000000}{12.000000}\selectfont 25}%
\end{pgfscope}%
\begin{pgfscope}%
\pgfsetbuttcap%
\pgfsetroundjoin%
\definecolor{currentfill}{rgb}{0.000000,0.000000,0.000000}%
\pgfsetfillcolor{currentfill}%
\pgfsetlinewidth{0.803000pt}%
\definecolor{currentstroke}{rgb}{0.000000,0.000000,0.000000}%
\pgfsetstrokecolor{currentstroke}%
\pgfsetdash{}{0pt}%
\pgfsys@defobject{currentmarker}{\pgfqpoint{-0.048611in}{0.000000in}}{\pgfqpoint{-0.000000in}{0.000000in}}{%
\pgfpathmoveto{\pgfqpoint{-0.000000in}{0.000000in}}%
\pgfpathlineto{\pgfqpoint{-0.048611in}{0.000000in}}%
\pgfusepath{stroke,fill}%
}%
\begin{pgfscope}%
\pgfsys@transformshift{0.481978in}{6.109431in}%
\pgfsys@useobject{currentmarker}{}%
\end{pgfscope}%
\end{pgfscope}%
\begin{pgfscope}%
\definecolor{textcolor}{rgb}{0.000000,0.000000,0.000000}%
\pgfsetstrokecolor{textcolor}%
\pgfsetfillcolor{textcolor}%
\pgftext[x=0.208025in, y=6.056669in, left, base]{\color{textcolor}\sffamily\fontsize{10.000000}{12.000000}\selectfont 50}%
\end{pgfscope}%
\begin{pgfscope}%
\pgfsetbuttcap%
\pgfsetroundjoin%
\definecolor{currentfill}{rgb}{0.000000,0.000000,0.000000}%
\pgfsetfillcolor{currentfill}%
\pgfsetlinewidth{0.803000pt}%
\definecolor{currentstroke}{rgb}{0.000000,0.000000,0.000000}%
\pgfsetstrokecolor{currentstroke}%
\pgfsetdash{}{0pt}%
\pgfsys@defobject{currentmarker}{\pgfqpoint{-0.048611in}{0.000000in}}{\pgfqpoint{-0.000000in}{0.000000in}}{%
\pgfpathmoveto{\pgfqpoint{-0.000000in}{0.000000in}}%
\pgfpathlineto{\pgfqpoint{-0.048611in}{0.000000in}}%
\pgfusepath{stroke,fill}%
}%
\begin{pgfscope}%
\pgfsys@transformshift{0.481978in}{7.155408in}%
\pgfsys@useobject{currentmarker}{}%
\end{pgfscope}%
\end{pgfscope}%
\begin{pgfscope}%
\definecolor{textcolor}{rgb}{0.000000,0.000000,0.000000}%
\pgfsetstrokecolor{textcolor}%
\pgfsetfillcolor{textcolor}%
\pgftext[x=0.208025in, y=7.102647in, left, base]{\color{textcolor}\sffamily\fontsize{10.000000}{12.000000}\selectfont 75}%
\end{pgfscope}%
\begin{pgfscope}%
\pgfpathrectangle{\pgfqpoint{0.481978in}{0.331635in}}{\pgfqpoint{9.300000in}{7.700000in}}%
\pgfusepath{clip}%
\pgfsetrectcap%
\pgfsetroundjoin%
\pgfsetlinewidth{1.505625pt}%
\definecolor{currentstroke}{rgb}{0.631373,0.788235,0.956863}%
\pgfsetstrokecolor{currentstroke}%
\pgfsetstrokeopacity{0.800000}%
\pgfsetdash{}{0pt}%
\pgfpathmoveto{\pgfqpoint{8.052863in}{1.552093in}}%
\pgfpathlineto{\pgfqpoint{5.229698in}{3.467077in}}%
\pgfusepath{stroke}%
\end{pgfscope}%
\begin{pgfscope}%
\pgfpathrectangle{\pgfqpoint{0.481978in}{0.331635in}}{\pgfqpoint{9.300000in}{7.700000in}}%
\pgfusepath{clip}%
\pgfsetrectcap%
\pgfsetroundjoin%
\pgfsetlinewidth{1.505625pt}%
\definecolor{currentstroke}{rgb}{0.631373,0.788235,0.956863}%
\pgfsetstrokecolor{currentstroke}%
\pgfsetstrokeopacity{0.800000}%
\pgfsetdash{}{0pt}%
\pgfpathmoveto{\pgfqpoint{6.390490in}{4.873471in}}%
\pgfpathlineto{\pgfqpoint{5.229698in}{3.467077in}}%
\pgfusepath{stroke}%
\end{pgfscope}%
\begin{pgfscope}%
\pgfpathrectangle{\pgfqpoint{0.481978in}{0.331635in}}{\pgfqpoint{9.300000in}{7.700000in}}%
\pgfusepath{clip}%
\pgfsetrectcap%
\pgfsetroundjoin%
\pgfsetlinewidth{1.505625pt}%
\definecolor{currentstroke}{rgb}{0.631373,0.788235,0.956863}%
\pgfsetstrokecolor{currentstroke}%
\pgfsetstrokeopacity{0.800000}%
\pgfsetdash{}{0pt}%
\pgfpathmoveto{\pgfqpoint{4.724066in}{3.602648in}}%
\pgfpathlineto{\pgfqpoint{5.229698in}{3.467077in}}%
\pgfusepath{stroke}%
\end{pgfscope}%
\begin{pgfscope}%
\pgfpathrectangle{\pgfqpoint{0.481978in}{0.331635in}}{\pgfqpoint{9.300000in}{7.700000in}}%
\pgfusepath{clip}%
\pgfsetrectcap%
\pgfsetroundjoin%
\pgfsetlinewidth{1.505625pt}%
\definecolor{currentstroke}{rgb}{0.631373,0.788235,0.956863}%
\pgfsetstrokecolor{currentstroke}%
\pgfsetstrokeopacity{0.800000}%
\pgfsetdash{}{0pt}%
\pgfpathmoveto{\pgfqpoint{3.262933in}{3.175424in}}%
\pgfpathlineto{\pgfqpoint{5.229698in}{3.467077in}}%
\pgfusepath{stroke}%
\end{pgfscope}%
\begin{pgfscope}%
\pgfpathrectangle{\pgfqpoint{0.481978in}{0.331635in}}{\pgfqpoint{9.300000in}{7.700000in}}%
\pgfusepath{clip}%
\pgfsetrectcap%
\pgfsetroundjoin%
\pgfsetlinewidth{1.505625pt}%
\definecolor{currentstroke}{rgb}{0.631373,0.788235,0.956863}%
\pgfsetstrokecolor{currentstroke}%
\pgfsetstrokeopacity{0.800000}%
\pgfsetdash{}{0pt}%
\pgfpathmoveto{\pgfqpoint{8.262238in}{1.111019in}}%
\pgfpathlineto{\pgfqpoint{5.229698in}{3.467077in}}%
\pgfusepath{stroke}%
\end{pgfscope}%
\begin{pgfscope}%
\pgfpathrectangle{\pgfqpoint{0.481978in}{0.331635in}}{\pgfqpoint{9.300000in}{7.700000in}}%
\pgfusepath{clip}%
\pgfsetrectcap%
\pgfsetroundjoin%
\pgfsetlinewidth{1.505625pt}%
\definecolor{currentstroke}{rgb}{0.631373,0.788235,0.956863}%
\pgfsetstrokecolor{currentstroke}%
\pgfsetstrokeopacity{0.800000}%
\pgfsetdash{}{0pt}%
\pgfpathmoveto{\pgfqpoint{7.580160in}{5.058834in}}%
\pgfpathlineto{\pgfqpoint{5.229698in}{3.467077in}}%
\pgfusepath{stroke}%
\end{pgfscope}%
\begin{pgfscope}%
\pgfpathrectangle{\pgfqpoint{0.481978in}{0.331635in}}{\pgfqpoint{9.300000in}{7.700000in}}%
\pgfusepath{clip}%
\pgfsetrectcap%
\pgfsetroundjoin%
\pgfsetlinewidth{1.505625pt}%
\definecolor{currentstroke}{rgb}{0.631373,0.788235,0.956863}%
\pgfsetstrokecolor{currentstroke}%
\pgfsetstrokeopacity{0.800000}%
\pgfsetdash{}{0pt}%
\pgfpathmoveto{\pgfqpoint{7.496625in}{2.061400in}}%
\pgfpathlineto{\pgfqpoint{5.229698in}{3.467077in}}%
\pgfusepath{stroke}%
\end{pgfscope}%
\begin{pgfscope}%
\pgfpathrectangle{\pgfqpoint{0.481978in}{0.331635in}}{\pgfqpoint{9.300000in}{7.700000in}}%
\pgfusepath{clip}%
\pgfsetrectcap%
\pgfsetroundjoin%
\pgfsetlinewidth{1.505625pt}%
\definecolor{currentstroke}{rgb}{0.631373,0.788235,0.956863}%
\pgfsetstrokecolor{currentstroke}%
\pgfsetstrokeopacity{0.800000}%
\pgfsetdash{}{0pt}%
\pgfpathmoveto{\pgfqpoint{3.509030in}{2.758429in}}%
\pgfpathlineto{\pgfqpoint{5.229698in}{3.467077in}}%
\pgfusepath{stroke}%
\end{pgfscope}%
\begin{pgfscope}%
\pgfpathrectangle{\pgfqpoint{0.481978in}{0.331635in}}{\pgfqpoint{9.300000in}{7.700000in}}%
\pgfusepath{clip}%
\pgfsetrectcap%
\pgfsetroundjoin%
\pgfsetlinewidth{1.505625pt}%
\definecolor{currentstroke}{rgb}{0.631373,0.788235,0.956863}%
\pgfsetstrokecolor{currentstroke}%
\pgfsetstrokeopacity{0.800000}%
\pgfsetdash{}{0pt}%
\pgfpathmoveto{\pgfqpoint{6.905988in}{4.411555in}}%
\pgfpathlineto{\pgfqpoint{5.229698in}{3.467077in}}%
\pgfusepath{stroke}%
\end{pgfscope}%
\begin{pgfscope}%
\pgfpathrectangle{\pgfqpoint{0.481978in}{0.331635in}}{\pgfqpoint{9.300000in}{7.700000in}}%
\pgfusepath{clip}%
\pgfsetrectcap%
\pgfsetroundjoin%
\pgfsetlinewidth{1.505625pt}%
\definecolor{currentstroke}{rgb}{0.631373,0.788235,0.956863}%
\pgfsetstrokecolor{currentstroke}%
\pgfsetstrokeopacity{0.800000}%
\pgfsetdash{}{0pt}%
\pgfpathmoveto{\pgfqpoint{2.394843in}{4.944242in}}%
\pgfpathlineto{\pgfqpoint{5.229698in}{3.467077in}}%
\pgfusepath{stroke}%
\end{pgfscope}%
\begin{pgfscope}%
\pgfpathrectangle{\pgfqpoint{0.481978in}{0.331635in}}{\pgfqpoint{9.300000in}{7.700000in}}%
\pgfusepath{clip}%
\pgfsetrectcap%
\pgfsetroundjoin%
\pgfsetlinewidth{1.505625pt}%
\definecolor{currentstroke}{rgb}{0.631373,0.788235,0.956863}%
\pgfsetstrokecolor{currentstroke}%
\pgfsetstrokeopacity{0.800000}%
\pgfsetdash{}{0pt}%
\pgfpathmoveto{\pgfqpoint{4.587397in}{5.253965in}}%
\pgfpathlineto{\pgfqpoint{5.229698in}{3.467077in}}%
\pgfusepath{stroke}%
\end{pgfscope}%
\begin{pgfscope}%
\pgfpathrectangle{\pgfqpoint{0.481978in}{0.331635in}}{\pgfqpoint{9.300000in}{7.700000in}}%
\pgfusepath{clip}%
\pgfsetrectcap%
\pgfsetroundjoin%
\pgfsetlinewidth{1.505625pt}%
\definecolor{currentstroke}{rgb}{0.631373,0.788235,0.956863}%
\pgfsetstrokecolor{currentstroke}%
\pgfsetstrokeopacity{0.800000}%
\pgfsetdash{}{0pt}%
\pgfpathmoveto{\pgfqpoint{5.606626in}{4.870416in}}%
\pgfpathlineto{\pgfqpoint{5.229698in}{3.467077in}}%
\pgfusepath{stroke}%
\end{pgfscope}%
\begin{pgfscope}%
\pgfpathrectangle{\pgfqpoint{0.481978in}{0.331635in}}{\pgfqpoint{9.300000in}{7.700000in}}%
\pgfusepath{clip}%
\pgfsetrectcap%
\pgfsetroundjoin%
\pgfsetlinewidth{1.505625pt}%
\definecolor{currentstroke}{rgb}{0.631373,0.788235,0.956863}%
\pgfsetstrokecolor{currentstroke}%
\pgfsetstrokeopacity{0.800000}%
\pgfsetdash{}{0pt}%
\pgfpathmoveto{\pgfqpoint{5.093085in}{5.645718in}}%
\pgfpathlineto{\pgfqpoint{5.229698in}{3.467077in}}%
\pgfusepath{stroke}%
\end{pgfscope}%
\begin{pgfscope}%
\pgfpathrectangle{\pgfqpoint{0.481978in}{0.331635in}}{\pgfqpoint{9.300000in}{7.700000in}}%
\pgfusepath{clip}%
\pgfsetrectcap%
\pgfsetroundjoin%
\pgfsetlinewidth{1.505625pt}%
\definecolor{currentstroke}{rgb}{0.631373,0.788235,0.956863}%
\pgfsetstrokecolor{currentstroke}%
\pgfsetstrokeopacity{0.800000}%
\pgfsetdash{}{0pt}%
\pgfpathmoveto{\pgfqpoint{7.877868in}{4.811193in}}%
\pgfpathlineto{\pgfqpoint{5.229698in}{3.467077in}}%
\pgfusepath{stroke}%
\end{pgfscope}%
\begin{pgfscope}%
\pgfpathrectangle{\pgfqpoint{0.481978in}{0.331635in}}{\pgfqpoint{9.300000in}{7.700000in}}%
\pgfusepath{clip}%
\pgfsetrectcap%
\pgfsetroundjoin%
\pgfsetlinewidth{1.505625pt}%
\definecolor{currentstroke}{rgb}{0.631373,0.788235,0.956863}%
\pgfsetstrokecolor{currentstroke}%
\pgfsetstrokeopacity{0.800000}%
\pgfsetdash{}{0pt}%
\pgfpathmoveto{\pgfqpoint{2.375730in}{2.401464in}}%
\pgfpathlineto{\pgfqpoint{5.229698in}{3.467077in}}%
\pgfusepath{stroke}%
\end{pgfscope}%
\begin{pgfscope}%
\pgfpathrectangle{\pgfqpoint{0.481978in}{0.331635in}}{\pgfqpoint{9.300000in}{7.700000in}}%
\pgfusepath{clip}%
\pgfsetrectcap%
\pgfsetroundjoin%
\pgfsetlinewidth{1.505625pt}%
\definecolor{currentstroke}{rgb}{0.631373,0.788235,0.956863}%
\pgfsetstrokecolor{currentstroke}%
\pgfsetstrokeopacity{0.800000}%
\pgfsetdash{}{0pt}%
\pgfpathmoveto{\pgfqpoint{9.359251in}{4.541842in}}%
\pgfpathlineto{\pgfqpoint{5.229698in}{3.467077in}}%
\pgfusepath{stroke}%
\end{pgfscope}%
\begin{pgfscope}%
\pgfpathrectangle{\pgfqpoint{0.481978in}{0.331635in}}{\pgfqpoint{9.300000in}{7.700000in}}%
\pgfusepath{clip}%
\pgfsetrectcap%
\pgfsetroundjoin%
\pgfsetlinewidth{1.505625pt}%
\definecolor{currentstroke}{rgb}{0.631373,0.788235,0.956863}%
\pgfsetstrokecolor{currentstroke}%
\pgfsetstrokeopacity{0.800000}%
\pgfsetdash{}{0pt}%
\pgfpathmoveto{\pgfqpoint{6.013955in}{5.386350in}}%
\pgfpathlineto{\pgfqpoint{5.229698in}{3.467077in}}%
\pgfusepath{stroke}%
\end{pgfscope}%
\begin{pgfscope}%
\pgfpathrectangle{\pgfqpoint{0.481978in}{0.331635in}}{\pgfqpoint{9.300000in}{7.700000in}}%
\pgfusepath{clip}%
\pgfsetrectcap%
\pgfsetroundjoin%
\pgfsetlinewidth{1.505625pt}%
\definecolor{currentstroke}{rgb}{0.631373,0.788235,0.956863}%
\pgfsetstrokecolor{currentstroke}%
\pgfsetstrokeopacity{0.800000}%
\pgfsetdash{}{0pt}%
\pgfpathmoveto{\pgfqpoint{3.830222in}{6.820584in}}%
\pgfpathlineto{\pgfqpoint{5.229698in}{3.467077in}}%
\pgfusepath{stroke}%
\end{pgfscope}%
\begin{pgfscope}%
\pgfpathrectangle{\pgfqpoint{0.481978in}{0.331635in}}{\pgfqpoint{9.300000in}{7.700000in}}%
\pgfusepath{clip}%
\pgfsetrectcap%
\pgfsetroundjoin%
\pgfsetlinewidth{1.505625pt}%
\definecolor{currentstroke}{rgb}{0.631373,0.788235,0.956863}%
\pgfsetstrokecolor{currentstroke}%
\pgfsetstrokeopacity{0.800000}%
\pgfsetdash{}{0pt}%
\pgfpathmoveto{\pgfqpoint{8.786314in}{1.294603in}}%
\pgfpathlineto{\pgfqpoint{5.229698in}{3.467077in}}%
\pgfusepath{stroke}%
\end{pgfscope}%
\begin{pgfscope}%
\pgfpathrectangle{\pgfqpoint{0.481978in}{0.331635in}}{\pgfqpoint{9.300000in}{7.700000in}}%
\pgfusepath{clip}%
\pgfsetrectcap%
\pgfsetroundjoin%
\pgfsetlinewidth{1.505625pt}%
\definecolor{currentstroke}{rgb}{0.631373,0.788235,0.956863}%
\pgfsetstrokecolor{currentstroke}%
\pgfsetstrokeopacity{0.800000}%
\pgfsetdash{}{0pt}%
\pgfpathmoveto{\pgfqpoint{1.235540in}{2.446071in}}%
\pgfpathlineto{\pgfqpoint{5.229698in}{3.467077in}}%
\pgfusepath{stroke}%
\end{pgfscope}%
\begin{pgfscope}%
\pgfpathrectangle{\pgfqpoint{0.481978in}{0.331635in}}{\pgfqpoint{9.300000in}{7.700000in}}%
\pgfusepath{clip}%
\pgfsetrectcap%
\pgfsetroundjoin%
\pgfsetlinewidth{1.505625pt}%
\definecolor{currentstroke}{rgb}{0.631373,0.788235,0.956863}%
\pgfsetstrokecolor{currentstroke}%
\pgfsetstrokeopacity{0.800000}%
\pgfsetdash{}{0pt}%
\pgfpathmoveto{\pgfqpoint{2.423803in}{3.183829in}}%
\pgfpathlineto{\pgfqpoint{5.229698in}{3.467077in}}%
\pgfusepath{stroke}%
\end{pgfscope}%
\begin{pgfscope}%
\pgfpathrectangle{\pgfqpoint{0.481978in}{0.331635in}}{\pgfqpoint{9.300000in}{7.700000in}}%
\pgfusepath{clip}%
\pgfsetrectcap%
\pgfsetroundjoin%
\pgfsetlinewidth{1.505625pt}%
\definecolor{currentstroke}{rgb}{0.631373,0.788235,0.956863}%
\pgfsetstrokecolor{currentstroke}%
\pgfsetstrokeopacity{0.800000}%
\pgfsetdash{}{0pt}%
\pgfpathmoveto{\pgfqpoint{3.099413in}{4.299245in}}%
\pgfpathlineto{\pgfqpoint{5.229698in}{3.467077in}}%
\pgfusepath{stroke}%
\end{pgfscope}%
\begin{pgfscope}%
\pgfpathrectangle{\pgfqpoint{0.481978in}{0.331635in}}{\pgfqpoint{9.300000in}{7.700000in}}%
\pgfusepath{clip}%
\pgfsetrectcap%
\pgfsetroundjoin%
\pgfsetlinewidth{1.505625pt}%
\definecolor{currentstroke}{rgb}{0.631373,0.788235,0.956863}%
\pgfsetstrokecolor{currentstroke}%
\pgfsetstrokeopacity{0.800000}%
\pgfsetdash{}{0pt}%
\pgfpathmoveto{\pgfqpoint{3.438982in}{4.497429in}}%
\pgfpathlineto{\pgfqpoint{5.229698in}{3.467077in}}%
\pgfusepath{stroke}%
\end{pgfscope}%
\begin{pgfscope}%
\pgfpathrectangle{\pgfqpoint{0.481978in}{0.331635in}}{\pgfqpoint{9.300000in}{7.700000in}}%
\pgfusepath{clip}%
\pgfsetrectcap%
\pgfsetroundjoin%
\pgfsetlinewidth{1.505625pt}%
\definecolor{currentstroke}{rgb}{0.631373,0.788235,0.956863}%
\pgfsetstrokecolor{currentstroke}%
\pgfsetstrokeopacity{0.800000}%
\pgfsetdash{}{0pt}%
\pgfpathmoveto{\pgfqpoint{6.239241in}{3.991403in}}%
\pgfpathlineto{\pgfqpoint{5.229698in}{3.467077in}}%
\pgfusepath{stroke}%
\end{pgfscope}%
\begin{pgfscope}%
\pgfpathrectangle{\pgfqpoint{0.481978in}{0.331635in}}{\pgfqpoint{9.300000in}{7.700000in}}%
\pgfusepath{clip}%
\pgfsetrectcap%
\pgfsetroundjoin%
\pgfsetlinewidth{1.505625pt}%
\definecolor{currentstroke}{rgb}{0.631373,0.788235,0.956863}%
\pgfsetstrokecolor{currentstroke}%
\pgfsetstrokeopacity{0.800000}%
\pgfsetdash{}{0pt}%
\pgfpathmoveto{\pgfqpoint{7.439054in}{1.199933in}}%
\pgfpathlineto{\pgfqpoint{5.229698in}{3.467077in}}%
\pgfusepath{stroke}%
\end{pgfscope}%
\begin{pgfscope}%
\pgfpathrectangle{\pgfqpoint{0.481978in}{0.331635in}}{\pgfqpoint{9.300000in}{7.700000in}}%
\pgfusepath{clip}%
\pgfsetrectcap%
\pgfsetroundjoin%
\pgfsetlinewidth{1.505625pt}%
\definecolor{currentstroke}{rgb}{0.631373,0.788235,0.956863}%
\pgfsetstrokecolor{currentstroke}%
\pgfsetstrokeopacity{0.800000}%
\pgfsetdash{}{0pt}%
\pgfpathmoveto{\pgfqpoint{7.510101in}{1.684631in}}%
\pgfpathlineto{\pgfqpoint{5.229698in}{3.467077in}}%
\pgfusepath{stroke}%
\end{pgfscope}%
\begin{pgfscope}%
\pgfpathrectangle{\pgfqpoint{0.481978in}{0.331635in}}{\pgfqpoint{9.300000in}{7.700000in}}%
\pgfusepath{clip}%
\pgfsetrectcap%
\pgfsetroundjoin%
\pgfsetlinewidth{1.505625pt}%
\definecolor{currentstroke}{rgb}{0.631373,0.788235,0.956863}%
\pgfsetstrokecolor{currentstroke}%
\pgfsetstrokeopacity{0.800000}%
\pgfsetdash{}{0pt}%
\pgfpathmoveto{\pgfqpoint{2.315914in}{4.233986in}}%
\pgfpathlineto{\pgfqpoint{5.229698in}{3.467077in}}%
\pgfusepath{stroke}%
\end{pgfscope}%
\begin{pgfscope}%
\pgfpathrectangle{\pgfqpoint{0.481978in}{0.331635in}}{\pgfqpoint{9.300000in}{7.700000in}}%
\pgfusepath{clip}%
\pgfsetrectcap%
\pgfsetroundjoin%
\pgfsetlinewidth{1.505625pt}%
\definecolor{currentstroke}{rgb}{0.631373,0.788235,0.956863}%
\pgfsetstrokecolor{currentstroke}%
\pgfsetstrokeopacity{0.800000}%
\pgfsetdash{}{0pt}%
\pgfpathmoveto{\pgfqpoint{6.878899in}{1.113104in}}%
\pgfpathlineto{\pgfqpoint{5.229698in}{3.467077in}}%
\pgfusepath{stroke}%
\end{pgfscope}%
\begin{pgfscope}%
\pgfpathrectangle{\pgfqpoint{0.481978in}{0.331635in}}{\pgfqpoint{9.300000in}{7.700000in}}%
\pgfusepath{clip}%
\pgfsetrectcap%
\pgfsetroundjoin%
\pgfsetlinewidth{1.505625pt}%
\definecolor{currentstroke}{rgb}{0.631373,0.788235,0.956863}%
\pgfsetstrokecolor{currentstroke}%
\pgfsetstrokeopacity{0.800000}%
\pgfsetdash{}{0pt}%
\pgfpathmoveto{\pgfqpoint{3.274514in}{1.576281in}}%
\pgfpathlineto{\pgfqpoint{5.229698in}{3.467077in}}%
\pgfusepath{stroke}%
\end{pgfscope}%
\begin{pgfscope}%
\pgfpathrectangle{\pgfqpoint{0.481978in}{0.331635in}}{\pgfqpoint{9.300000in}{7.700000in}}%
\pgfusepath{clip}%
\pgfsetrectcap%
\pgfsetroundjoin%
\pgfsetlinewidth{1.505625pt}%
\definecolor{currentstroke}{rgb}{0.631373,0.788235,0.956863}%
\pgfsetstrokecolor{currentstroke}%
\pgfsetstrokeopacity{0.800000}%
\pgfsetdash{}{0pt}%
\pgfpathmoveto{\pgfqpoint{6.005151in}{4.482219in}}%
\pgfpathlineto{\pgfqpoint{5.229698in}{3.467077in}}%
\pgfusepath{stroke}%
\end{pgfscope}%
\begin{pgfscope}%
\pgfpathrectangle{\pgfqpoint{0.481978in}{0.331635in}}{\pgfqpoint{9.300000in}{7.700000in}}%
\pgfusepath{clip}%
\pgfsetrectcap%
\pgfsetroundjoin%
\pgfsetlinewidth{1.505625pt}%
\definecolor{currentstroke}{rgb}{0.631373,0.788235,0.956863}%
\pgfsetstrokecolor{currentstroke}%
\pgfsetstrokeopacity{0.800000}%
\pgfsetdash{}{0pt}%
\pgfpathmoveto{\pgfqpoint{7.064826in}{4.831932in}}%
\pgfpathlineto{\pgfqpoint{5.229698in}{3.467077in}}%
\pgfusepath{stroke}%
\end{pgfscope}%
\begin{pgfscope}%
\pgfpathrectangle{\pgfqpoint{0.481978in}{0.331635in}}{\pgfqpoint{9.300000in}{7.700000in}}%
\pgfusepath{clip}%
\pgfsetrectcap%
\pgfsetroundjoin%
\pgfsetlinewidth{1.505625pt}%
\definecolor{currentstroke}{rgb}{0.631373,0.788235,0.956863}%
\pgfsetstrokecolor{currentstroke}%
\pgfsetstrokeopacity{0.800000}%
\pgfsetdash{}{0pt}%
\pgfpathmoveto{\pgfqpoint{6.961902in}{1.617101in}}%
\pgfpathlineto{\pgfqpoint{5.229698in}{3.467077in}}%
\pgfusepath{stroke}%
\end{pgfscope}%
\begin{pgfscope}%
\pgfpathrectangle{\pgfqpoint{0.481978in}{0.331635in}}{\pgfqpoint{9.300000in}{7.700000in}}%
\pgfusepath{clip}%
\pgfsetrectcap%
\pgfsetroundjoin%
\pgfsetlinewidth{1.505625pt}%
\definecolor{currentstroke}{rgb}{0.631373,0.788235,0.956863}%
\pgfsetstrokecolor{currentstroke}%
\pgfsetstrokeopacity{0.800000}%
\pgfsetdash{}{0pt}%
\pgfpathmoveto{\pgfqpoint{3.800404in}{4.886489in}}%
\pgfpathlineto{\pgfqpoint{5.229698in}{3.467077in}}%
\pgfusepath{stroke}%
\end{pgfscope}%
\begin{pgfscope}%
\pgfpathrectangle{\pgfqpoint{0.481978in}{0.331635in}}{\pgfqpoint{9.300000in}{7.700000in}}%
\pgfusepath{clip}%
\pgfsetrectcap%
\pgfsetroundjoin%
\pgfsetlinewidth{1.505625pt}%
\definecolor{currentstroke}{rgb}{0.631373,0.788235,0.956863}%
\pgfsetstrokecolor{currentstroke}%
\pgfsetstrokeopacity{0.800000}%
\pgfsetdash{}{0pt}%
\pgfpathmoveto{\pgfqpoint{6.058849in}{1.490387in}}%
\pgfpathlineto{\pgfqpoint{5.229698in}{3.467077in}}%
\pgfusepath{stroke}%
\end{pgfscope}%
\begin{pgfscope}%
\pgfpathrectangle{\pgfqpoint{0.481978in}{0.331635in}}{\pgfqpoint{9.300000in}{7.700000in}}%
\pgfusepath{clip}%
\pgfsetrectcap%
\pgfsetroundjoin%
\pgfsetlinewidth{1.505625pt}%
\definecolor{currentstroke}{rgb}{0.631373,0.788235,0.956863}%
\pgfsetstrokecolor{currentstroke}%
\pgfsetstrokeopacity{0.800000}%
\pgfsetdash{}{0pt}%
\pgfpathmoveto{\pgfqpoint{5.307606in}{5.963135in}}%
\pgfpathlineto{\pgfqpoint{5.229698in}{3.467077in}}%
\pgfusepath{stroke}%
\end{pgfscope}%
\begin{pgfscope}%
\pgfpathrectangle{\pgfqpoint{0.481978in}{0.331635in}}{\pgfqpoint{9.300000in}{7.700000in}}%
\pgfusepath{clip}%
\pgfsetrectcap%
\pgfsetroundjoin%
\pgfsetlinewidth{1.505625pt}%
\definecolor{currentstroke}{rgb}{0.631373,0.788235,0.956863}%
\pgfsetstrokecolor{currentstroke}%
\pgfsetstrokeopacity{0.800000}%
\pgfsetdash{}{0pt}%
\pgfpathmoveto{\pgfqpoint{7.033678in}{3.838299in}}%
\pgfpathlineto{\pgfqpoint{5.229698in}{3.467077in}}%
\pgfusepath{stroke}%
\end{pgfscope}%
\begin{pgfscope}%
\pgfpathrectangle{\pgfqpoint{0.481978in}{0.331635in}}{\pgfqpoint{9.300000in}{7.700000in}}%
\pgfusepath{clip}%
\pgfsetrectcap%
\pgfsetroundjoin%
\pgfsetlinewidth{1.505625pt}%
\definecolor{currentstroke}{rgb}{0.631373,0.788235,0.956863}%
\pgfsetstrokecolor{currentstroke}%
\pgfsetstrokeopacity{0.800000}%
\pgfsetdash{}{0pt}%
\pgfpathmoveto{\pgfqpoint{7.805428in}{0.848419in}}%
\pgfpathlineto{\pgfqpoint{5.229698in}{3.467077in}}%
\pgfusepath{stroke}%
\end{pgfscope}%
\begin{pgfscope}%
\pgfpathrectangle{\pgfqpoint{0.481978in}{0.331635in}}{\pgfqpoint{9.300000in}{7.700000in}}%
\pgfusepath{clip}%
\pgfsetrectcap%
\pgfsetroundjoin%
\pgfsetlinewidth{1.505625pt}%
\definecolor{currentstroke}{rgb}{0.631373,0.788235,0.956863}%
\pgfsetstrokecolor{currentstroke}%
\pgfsetstrokeopacity{0.800000}%
\pgfsetdash{}{0pt}%
\pgfpathmoveto{\pgfqpoint{3.104471in}{4.781432in}}%
\pgfpathlineto{\pgfqpoint{5.229698in}{3.467077in}}%
\pgfusepath{stroke}%
\end{pgfscope}%
\begin{pgfscope}%
\pgfpathrectangle{\pgfqpoint{0.481978in}{0.331635in}}{\pgfqpoint{9.300000in}{7.700000in}}%
\pgfusepath{clip}%
\pgfsetrectcap%
\pgfsetroundjoin%
\pgfsetlinewidth{1.505625pt}%
\definecolor{currentstroke}{rgb}{0.631373,0.788235,0.956863}%
\pgfsetstrokecolor{currentstroke}%
\pgfsetstrokeopacity{0.800000}%
\pgfsetdash{}{0pt}%
\pgfpathmoveto{\pgfqpoint{1.582673in}{3.731040in}}%
\pgfpathlineto{\pgfqpoint{5.229698in}{3.467077in}}%
\pgfusepath{stroke}%
\end{pgfscope}%
\begin{pgfscope}%
\pgfpathrectangle{\pgfqpoint{0.481978in}{0.331635in}}{\pgfqpoint{9.300000in}{7.700000in}}%
\pgfusepath{clip}%
\pgfsetrectcap%
\pgfsetroundjoin%
\pgfsetlinewidth{1.505625pt}%
\definecolor{currentstroke}{rgb}{0.631373,0.788235,0.956863}%
\pgfsetstrokecolor{currentstroke}%
\pgfsetstrokeopacity{0.800000}%
\pgfsetdash{}{0pt}%
\pgfpathmoveto{\pgfqpoint{7.082906in}{5.518169in}}%
\pgfpathlineto{\pgfqpoint{5.229698in}{3.467077in}}%
\pgfusepath{stroke}%
\end{pgfscope}%
\begin{pgfscope}%
\pgfpathrectangle{\pgfqpoint{0.481978in}{0.331635in}}{\pgfqpoint{9.300000in}{7.700000in}}%
\pgfusepath{clip}%
\pgfsetrectcap%
\pgfsetroundjoin%
\pgfsetlinewidth{1.505625pt}%
\definecolor{currentstroke}{rgb}{0.631373,0.788235,0.956863}%
\pgfsetstrokecolor{currentstroke}%
\pgfsetstrokeopacity{0.800000}%
\pgfsetdash{}{0pt}%
\pgfpathmoveto{\pgfqpoint{2.863295in}{2.561278in}}%
\pgfpathlineto{\pgfqpoint{5.229698in}{3.467077in}}%
\pgfusepath{stroke}%
\end{pgfscope}%
\begin{pgfscope}%
\pgfpathrectangle{\pgfqpoint{0.481978in}{0.331635in}}{\pgfqpoint{9.300000in}{7.700000in}}%
\pgfusepath{clip}%
\pgfsetrectcap%
\pgfsetroundjoin%
\pgfsetlinewidth{1.505625pt}%
\definecolor{currentstroke}{rgb}{0.631373,0.788235,0.956863}%
\pgfsetstrokecolor{currentstroke}%
\pgfsetstrokeopacity{0.800000}%
\pgfsetdash{}{0pt}%
\pgfpathmoveto{\pgfqpoint{7.718781in}{2.772405in}}%
\pgfpathlineto{\pgfqpoint{5.229698in}{3.467077in}}%
\pgfusepath{stroke}%
\end{pgfscope}%
\begin{pgfscope}%
\pgfpathrectangle{\pgfqpoint{0.481978in}{0.331635in}}{\pgfqpoint{9.300000in}{7.700000in}}%
\pgfusepath{clip}%
\pgfsetrectcap%
\pgfsetroundjoin%
\pgfsetlinewidth{1.505625pt}%
\definecolor{currentstroke}{rgb}{0.631373,0.788235,0.956863}%
\pgfsetstrokecolor{currentstroke}%
\pgfsetstrokeopacity{0.800000}%
\pgfsetdash{}{0pt}%
\pgfpathmoveto{\pgfqpoint{3.697003in}{3.940047in}}%
\pgfpathlineto{\pgfqpoint{5.229698in}{3.467077in}}%
\pgfusepath{stroke}%
\end{pgfscope}%
\begin{pgfscope}%
\pgfpathrectangle{\pgfqpoint{0.481978in}{0.331635in}}{\pgfqpoint{9.300000in}{7.700000in}}%
\pgfusepath{clip}%
\pgfsetrectcap%
\pgfsetroundjoin%
\pgfsetlinewidth{1.505625pt}%
\definecolor{currentstroke}{rgb}{0.631373,0.788235,0.956863}%
\pgfsetstrokecolor{currentstroke}%
\pgfsetstrokeopacity{0.800000}%
\pgfsetdash{}{0pt}%
\pgfpathmoveto{\pgfqpoint{6.644743in}{0.681635in}}%
\pgfpathlineto{\pgfqpoint{5.229698in}{3.467077in}}%
\pgfusepath{stroke}%
\end{pgfscope}%
\begin{pgfscope}%
\pgfpathrectangle{\pgfqpoint{0.481978in}{0.331635in}}{\pgfqpoint{9.300000in}{7.700000in}}%
\pgfusepath{clip}%
\pgfsetrectcap%
\pgfsetroundjoin%
\pgfsetlinewidth{1.505625pt}%
\definecolor{currentstroke}{rgb}{0.631373,0.788235,0.956863}%
\pgfsetstrokecolor{currentstroke}%
\pgfsetstrokeopacity{0.800000}%
\pgfsetdash{}{0pt}%
\pgfpathmoveto{\pgfqpoint{2.360953in}{1.986282in}}%
\pgfpathlineto{\pgfqpoint{5.229698in}{3.467077in}}%
\pgfusepath{stroke}%
\end{pgfscope}%
\begin{pgfscope}%
\pgfpathrectangle{\pgfqpoint{0.481978in}{0.331635in}}{\pgfqpoint{9.300000in}{7.700000in}}%
\pgfusepath{clip}%
\pgfsetrectcap%
\pgfsetroundjoin%
\pgfsetlinewidth{1.505625pt}%
\definecolor{currentstroke}{rgb}{0.631373,0.788235,0.956863}%
\pgfsetstrokecolor{currentstroke}%
\pgfsetstrokeopacity{0.800000}%
\pgfsetdash{}{0pt}%
\pgfpathmoveto{\pgfqpoint{3.463144in}{2.178317in}}%
\pgfpathlineto{\pgfqpoint{5.229698in}{3.467077in}}%
\pgfusepath{stroke}%
\end{pgfscope}%
\begin{pgfscope}%
\pgfpathrectangle{\pgfqpoint{0.481978in}{0.331635in}}{\pgfqpoint{9.300000in}{7.700000in}}%
\pgfusepath{clip}%
\pgfsetrectcap%
\pgfsetroundjoin%
\pgfsetlinewidth{1.505625pt}%
\definecolor{currentstroke}{rgb}{0.631373,0.788235,0.956863}%
\pgfsetstrokecolor{currentstroke}%
\pgfsetstrokeopacity{0.800000}%
\pgfsetdash{}{0pt}%
\pgfpathmoveto{\pgfqpoint{8.681814in}{2.133007in}}%
\pgfpathlineto{\pgfqpoint{5.229698in}{3.467077in}}%
\pgfusepath{stroke}%
\end{pgfscope}%
\begin{pgfscope}%
\pgfpathrectangle{\pgfqpoint{0.481978in}{0.331635in}}{\pgfqpoint{9.300000in}{7.700000in}}%
\pgfusepath{clip}%
\pgfsetrectcap%
\pgfsetroundjoin%
\pgfsetlinewidth{1.505625pt}%
\definecolor{currentstroke}{rgb}{0.631373,0.788235,0.956863}%
\pgfsetstrokecolor{currentstroke}%
\pgfsetstrokeopacity{0.800000}%
\pgfsetdash{}{0pt}%
\pgfpathmoveto{\pgfqpoint{5.188038in}{5.127389in}}%
\pgfpathlineto{\pgfqpoint{5.229698in}{3.467077in}}%
\pgfusepath{stroke}%
\end{pgfscope}%
\begin{pgfscope}%
\pgfpathrectangle{\pgfqpoint{0.481978in}{0.331635in}}{\pgfqpoint{9.300000in}{7.700000in}}%
\pgfusepath{clip}%
\pgfsetrectcap%
\pgfsetroundjoin%
\pgfsetlinewidth{1.505625pt}%
\definecolor{currentstroke}{rgb}{0.631373,0.788235,0.956863}%
\pgfsetstrokecolor{currentstroke}%
\pgfsetstrokeopacity{0.800000}%
\pgfsetdash{}{0pt}%
\pgfpathmoveto{\pgfqpoint{2.109676in}{2.977613in}}%
\pgfpathlineto{\pgfqpoint{5.229698in}{3.467077in}}%
\pgfusepath{stroke}%
\end{pgfscope}%
\begin{pgfscope}%
\pgfpathrectangle{\pgfqpoint{0.481978in}{0.331635in}}{\pgfqpoint{9.300000in}{7.700000in}}%
\pgfusepath{clip}%
\pgfsetrectcap%
\pgfsetroundjoin%
\pgfsetlinewidth{1.505625pt}%
\definecolor{currentstroke}{rgb}{0.631373,0.788235,0.956863}%
\pgfsetstrokecolor{currentstroke}%
\pgfsetstrokeopacity{0.800000}%
\pgfsetdash{}{0pt}%
\pgfpathmoveto{\pgfqpoint{0.984432in}{4.206085in}}%
\pgfpathlineto{\pgfqpoint{5.229698in}{3.467077in}}%
\pgfusepath{stroke}%
\end{pgfscope}%
\begin{pgfscope}%
\pgfpathrectangle{\pgfqpoint{0.481978in}{0.331635in}}{\pgfqpoint{9.300000in}{7.700000in}}%
\pgfusepath{clip}%
\pgfsetrectcap%
\pgfsetroundjoin%
\pgfsetlinewidth{1.505625pt}%
\definecolor{currentstroke}{rgb}{1.000000,0.705882,0.509804}%
\pgfsetstrokecolor{currentstroke}%
\pgfsetstrokeopacity{0.800000}%
\pgfsetdash{}{0pt}%
\pgfpathmoveto{\pgfqpoint{6.044554in}{2.177724in}}%
\pgfpathlineto{\pgfqpoint{5.573130in}{4.513635in}}%
\pgfusepath{stroke}%
\end{pgfscope}%
\begin{pgfscope}%
\pgfpathrectangle{\pgfqpoint{0.481978in}{0.331635in}}{\pgfqpoint{9.300000in}{7.700000in}}%
\pgfusepath{clip}%
\pgfsetrectcap%
\pgfsetroundjoin%
\pgfsetlinewidth{1.505625pt}%
\definecolor{currentstroke}{rgb}{1.000000,0.705882,0.509804}%
\pgfsetstrokecolor{currentstroke}%
\pgfsetstrokeopacity{0.800000}%
\pgfsetdash{}{0pt}%
\pgfpathmoveto{\pgfqpoint{4.376435in}{1.872307in}}%
\pgfpathlineto{\pgfqpoint{5.573130in}{4.513635in}}%
\pgfusepath{stroke}%
\end{pgfscope}%
\begin{pgfscope}%
\pgfpathrectangle{\pgfqpoint{0.481978in}{0.331635in}}{\pgfqpoint{9.300000in}{7.700000in}}%
\pgfusepath{clip}%
\pgfsetrectcap%
\pgfsetroundjoin%
\pgfsetlinewidth{1.505625pt}%
\definecolor{currentstroke}{rgb}{1.000000,0.705882,0.509804}%
\pgfsetstrokecolor{currentstroke}%
\pgfsetstrokeopacity{0.800000}%
\pgfsetdash{}{0pt}%
\pgfpathmoveto{\pgfqpoint{8.943163in}{5.761954in}}%
\pgfpathlineto{\pgfqpoint{5.573130in}{4.513635in}}%
\pgfusepath{stroke}%
\end{pgfscope}%
\begin{pgfscope}%
\pgfpathrectangle{\pgfqpoint{0.481978in}{0.331635in}}{\pgfqpoint{9.300000in}{7.700000in}}%
\pgfusepath{clip}%
\pgfsetrectcap%
\pgfsetroundjoin%
\pgfsetlinewidth{1.505625pt}%
\definecolor{currentstroke}{rgb}{1.000000,0.705882,0.509804}%
\pgfsetstrokecolor{currentstroke}%
\pgfsetstrokeopacity{0.800000}%
\pgfsetdash{}{0pt}%
\pgfpathmoveto{\pgfqpoint{5.919865in}{3.031766in}}%
\pgfpathlineto{\pgfqpoint{5.573130in}{4.513635in}}%
\pgfusepath{stroke}%
\end{pgfscope}%
\begin{pgfscope}%
\pgfpathrectangle{\pgfqpoint{0.481978in}{0.331635in}}{\pgfqpoint{9.300000in}{7.700000in}}%
\pgfusepath{clip}%
\pgfsetrectcap%
\pgfsetroundjoin%
\pgfsetlinewidth{1.505625pt}%
\definecolor{currentstroke}{rgb}{1.000000,0.705882,0.509804}%
\pgfsetstrokecolor{currentstroke}%
\pgfsetstrokeopacity{0.800000}%
\pgfsetdash{}{0pt}%
\pgfpathmoveto{\pgfqpoint{5.035065in}{3.911938in}}%
\pgfpathlineto{\pgfqpoint{5.573130in}{4.513635in}}%
\pgfusepath{stroke}%
\end{pgfscope}%
\begin{pgfscope}%
\pgfpathrectangle{\pgfqpoint{0.481978in}{0.331635in}}{\pgfqpoint{9.300000in}{7.700000in}}%
\pgfusepath{clip}%
\pgfsetrectcap%
\pgfsetroundjoin%
\pgfsetlinewidth{1.505625pt}%
\definecolor{currentstroke}{rgb}{1.000000,0.705882,0.509804}%
\pgfsetstrokecolor{currentstroke}%
\pgfsetstrokeopacity{0.800000}%
\pgfsetdash{}{0pt}%
\pgfpathmoveto{\pgfqpoint{7.526077in}{7.081058in}}%
\pgfpathlineto{\pgfqpoint{5.573130in}{4.513635in}}%
\pgfusepath{stroke}%
\end{pgfscope}%
\begin{pgfscope}%
\pgfpathrectangle{\pgfqpoint{0.481978in}{0.331635in}}{\pgfqpoint{9.300000in}{7.700000in}}%
\pgfusepath{clip}%
\pgfsetrectcap%
\pgfsetroundjoin%
\pgfsetlinewidth{1.505625pt}%
\definecolor{currentstroke}{rgb}{1.000000,0.705882,0.509804}%
\pgfsetstrokecolor{currentstroke}%
\pgfsetstrokeopacity{0.800000}%
\pgfsetdash{}{0pt}%
\pgfpathmoveto{\pgfqpoint{6.779267in}{7.371993in}}%
\pgfpathlineto{\pgfqpoint{5.573130in}{4.513635in}}%
\pgfusepath{stroke}%
\end{pgfscope}%
\begin{pgfscope}%
\pgfpathrectangle{\pgfqpoint{0.481978in}{0.331635in}}{\pgfqpoint{9.300000in}{7.700000in}}%
\pgfusepath{clip}%
\pgfsetrectcap%
\pgfsetroundjoin%
\pgfsetlinewidth{1.505625pt}%
\definecolor{currentstroke}{rgb}{1.000000,0.705882,0.509804}%
\pgfsetstrokecolor{currentstroke}%
\pgfsetstrokeopacity{0.800000}%
\pgfsetdash{}{0pt}%
\pgfpathmoveto{\pgfqpoint{2.475590in}{6.077186in}}%
\pgfpathlineto{\pgfqpoint{5.573130in}{4.513635in}}%
\pgfusepath{stroke}%
\end{pgfscope}%
\begin{pgfscope}%
\pgfpathrectangle{\pgfqpoint{0.481978in}{0.331635in}}{\pgfqpoint{9.300000in}{7.700000in}}%
\pgfusepath{clip}%
\pgfsetrectcap%
\pgfsetroundjoin%
\pgfsetlinewidth{1.505625pt}%
\definecolor{currentstroke}{rgb}{1.000000,0.705882,0.509804}%
\pgfsetstrokecolor{currentstroke}%
\pgfsetstrokeopacity{0.800000}%
\pgfsetdash{}{0pt}%
\pgfpathmoveto{\pgfqpoint{3.242777in}{3.583841in}}%
\pgfpathlineto{\pgfqpoint{5.573130in}{4.513635in}}%
\pgfusepath{stroke}%
\end{pgfscope}%
\begin{pgfscope}%
\pgfpathrectangle{\pgfqpoint{0.481978in}{0.331635in}}{\pgfqpoint{9.300000in}{7.700000in}}%
\pgfusepath{clip}%
\pgfsetrectcap%
\pgfsetroundjoin%
\pgfsetlinewidth{1.505625pt}%
\definecolor{currentstroke}{rgb}{1.000000,0.705882,0.509804}%
\pgfsetstrokecolor{currentstroke}%
\pgfsetstrokeopacity{0.800000}%
\pgfsetdash{}{0pt}%
\pgfpathmoveto{\pgfqpoint{4.227999in}{4.397836in}}%
\pgfpathlineto{\pgfqpoint{5.573130in}{4.513635in}}%
\pgfusepath{stroke}%
\end{pgfscope}%
\begin{pgfscope}%
\pgfpathrectangle{\pgfqpoint{0.481978in}{0.331635in}}{\pgfqpoint{9.300000in}{7.700000in}}%
\pgfusepath{clip}%
\pgfsetrectcap%
\pgfsetroundjoin%
\pgfsetlinewidth{1.505625pt}%
\definecolor{currentstroke}{rgb}{1.000000,0.705882,0.509804}%
\pgfsetstrokecolor{currentstroke}%
\pgfsetstrokeopacity{0.800000}%
\pgfsetdash{}{0pt}%
\pgfpathmoveto{\pgfqpoint{3.413419in}{5.358372in}}%
\pgfpathlineto{\pgfqpoint{5.573130in}{4.513635in}}%
\pgfusepath{stroke}%
\end{pgfscope}%
\begin{pgfscope}%
\pgfpathrectangle{\pgfqpoint{0.481978in}{0.331635in}}{\pgfqpoint{9.300000in}{7.700000in}}%
\pgfusepath{clip}%
\pgfsetrectcap%
\pgfsetroundjoin%
\pgfsetlinewidth{1.505625pt}%
\definecolor{currentstroke}{rgb}{1.000000,0.705882,0.509804}%
\pgfsetstrokecolor{currentstroke}%
\pgfsetstrokeopacity{0.800000}%
\pgfsetdash{}{0pt}%
\pgfpathmoveto{\pgfqpoint{3.930822in}{5.652502in}}%
\pgfpathlineto{\pgfqpoint{5.573130in}{4.513635in}}%
\pgfusepath{stroke}%
\end{pgfscope}%
\begin{pgfscope}%
\pgfpathrectangle{\pgfqpoint{0.481978in}{0.331635in}}{\pgfqpoint{9.300000in}{7.700000in}}%
\pgfusepath{clip}%
\pgfsetrectcap%
\pgfsetroundjoin%
\pgfsetlinewidth{1.505625pt}%
\definecolor{currentstroke}{rgb}{1.000000,0.705882,0.509804}%
\pgfsetstrokecolor{currentstroke}%
\pgfsetstrokeopacity{0.800000}%
\pgfsetdash{}{0pt}%
\pgfpathmoveto{\pgfqpoint{3.950735in}{3.084551in}}%
\pgfpathlineto{\pgfqpoint{5.573130in}{4.513635in}}%
\pgfusepath{stroke}%
\end{pgfscope}%
\begin{pgfscope}%
\pgfpathrectangle{\pgfqpoint{0.481978in}{0.331635in}}{\pgfqpoint{9.300000in}{7.700000in}}%
\pgfusepath{clip}%
\pgfsetrectcap%
\pgfsetroundjoin%
\pgfsetlinewidth{1.505625pt}%
\definecolor{currentstroke}{rgb}{1.000000,0.705882,0.509804}%
\pgfsetstrokecolor{currentstroke}%
\pgfsetstrokeopacity{0.800000}%
\pgfsetdash{}{0pt}%
\pgfpathmoveto{\pgfqpoint{4.062075in}{3.479157in}}%
\pgfpathlineto{\pgfqpoint{5.573130in}{4.513635in}}%
\pgfusepath{stroke}%
\end{pgfscope}%
\begin{pgfscope}%
\pgfpathrectangle{\pgfqpoint{0.481978in}{0.331635in}}{\pgfqpoint{9.300000in}{7.700000in}}%
\pgfusepath{clip}%
\pgfsetrectcap%
\pgfsetroundjoin%
\pgfsetlinewidth{1.505625pt}%
\definecolor{currentstroke}{rgb}{1.000000,0.705882,0.509804}%
\pgfsetstrokecolor{currentstroke}%
\pgfsetstrokeopacity{0.800000}%
\pgfsetdash{}{0pt}%
\pgfpathmoveto{\pgfqpoint{5.989908in}{3.446341in}}%
\pgfpathlineto{\pgfqpoint{5.573130in}{4.513635in}}%
\pgfusepath{stroke}%
\end{pgfscope}%
\begin{pgfscope}%
\pgfpathrectangle{\pgfqpoint{0.481978in}{0.331635in}}{\pgfqpoint{9.300000in}{7.700000in}}%
\pgfusepath{clip}%
\pgfsetrectcap%
\pgfsetroundjoin%
\pgfsetlinewidth{1.505625pt}%
\definecolor{currentstroke}{rgb}{1.000000,0.705882,0.509804}%
\pgfsetstrokecolor{currentstroke}%
\pgfsetstrokeopacity{0.800000}%
\pgfsetdash{}{0pt}%
\pgfpathmoveto{\pgfqpoint{8.313309in}{4.131814in}}%
\pgfpathlineto{\pgfqpoint{5.573130in}{4.513635in}}%
\pgfusepath{stroke}%
\end{pgfscope}%
\begin{pgfscope}%
\pgfpathrectangle{\pgfqpoint{0.481978in}{0.331635in}}{\pgfqpoint{9.300000in}{7.700000in}}%
\pgfusepath{clip}%
\pgfsetrectcap%
\pgfsetroundjoin%
\pgfsetlinewidth{1.505625pt}%
\definecolor{currentstroke}{rgb}{1.000000,0.705882,0.509804}%
\pgfsetstrokecolor{currentstroke}%
\pgfsetstrokeopacity{0.800000}%
\pgfsetdash{}{0pt}%
\pgfpathmoveto{\pgfqpoint{4.541721in}{3.280649in}}%
\pgfpathlineto{\pgfqpoint{5.573130in}{4.513635in}}%
\pgfusepath{stroke}%
\end{pgfscope}%
\begin{pgfscope}%
\pgfpathrectangle{\pgfqpoint{0.481978in}{0.331635in}}{\pgfqpoint{9.300000in}{7.700000in}}%
\pgfusepath{clip}%
\pgfsetrectcap%
\pgfsetroundjoin%
\pgfsetlinewidth{1.505625pt}%
\definecolor{currentstroke}{rgb}{1.000000,0.705882,0.509804}%
\pgfsetstrokecolor{currentstroke}%
\pgfsetstrokeopacity{0.800000}%
\pgfsetdash{}{0pt}%
\pgfpathmoveto{\pgfqpoint{8.281939in}{6.706957in}}%
\pgfpathlineto{\pgfqpoint{5.573130in}{4.513635in}}%
\pgfusepath{stroke}%
\end{pgfscope}%
\begin{pgfscope}%
\pgfpathrectangle{\pgfqpoint{0.481978in}{0.331635in}}{\pgfqpoint{9.300000in}{7.700000in}}%
\pgfusepath{clip}%
\pgfsetrectcap%
\pgfsetroundjoin%
\pgfsetlinewidth{1.505625pt}%
\definecolor{currentstroke}{rgb}{1.000000,0.705882,0.509804}%
\pgfsetstrokecolor{currentstroke}%
\pgfsetstrokeopacity{0.800000}%
\pgfsetdash{}{0pt}%
\pgfpathmoveto{\pgfqpoint{7.611554in}{7.554569in}}%
\pgfpathlineto{\pgfqpoint{5.573130in}{4.513635in}}%
\pgfusepath{stroke}%
\end{pgfscope}%
\begin{pgfscope}%
\pgfpathrectangle{\pgfqpoint{0.481978in}{0.331635in}}{\pgfqpoint{9.300000in}{7.700000in}}%
\pgfusepath{clip}%
\pgfsetrectcap%
\pgfsetroundjoin%
\pgfsetlinewidth{1.505625pt}%
\definecolor{currentstroke}{rgb}{1.000000,0.705882,0.509804}%
\pgfsetstrokecolor{currentstroke}%
\pgfsetstrokeopacity{0.800000}%
\pgfsetdash{}{0pt}%
\pgfpathmoveto{\pgfqpoint{2.721741in}{3.710762in}}%
\pgfpathlineto{\pgfqpoint{5.573130in}{4.513635in}}%
\pgfusepath{stroke}%
\end{pgfscope}%
\begin{pgfscope}%
\pgfpathrectangle{\pgfqpoint{0.481978in}{0.331635in}}{\pgfqpoint{9.300000in}{7.700000in}}%
\pgfusepath{clip}%
\pgfsetrectcap%
\pgfsetroundjoin%
\pgfsetlinewidth{1.505625pt}%
\definecolor{currentstroke}{rgb}{1.000000,0.705882,0.509804}%
\pgfsetstrokecolor{currentstroke}%
\pgfsetstrokeopacity{0.800000}%
\pgfsetdash{}{0pt}%
\pgfpathmoveto{\pgfqpoint{6.813584in}{2.441611in}}%
\pgfpathlineto{\pgfqpoint{5.573130in}{4.513635in}}%
\pgfusepath{stroke}%
\end{pgfscope}%
\begin{pgfscope}%
\pgfpathrectangle{\pgfqpoint{0.481978in}{0.331635in}}{\pgfqpoint{9.300000in}{7.700000in}}%
\pgfusepath{clip}%
\pgfsetrectcap%
\pgfsetroundjoin%
\pgfsetlinewidth{1.505625pt}%
\definecolor{currentstroke}{rgb}{1.000000,0.705882,0.509804}%
\pgfsetstrokecolor{currentstroke}%
\pgfsetstrokeopacity{0.800000}%
\pgfsetdash{}{0pt}%
\pgfpathmoveto{\pgfqpoint{7.019489in}{7.681635in}}%
\pgfpathlineto{\pgfqpoint{5.573130in}{4.513635in}}%
\pgfusepath{stroke}%
\end{pgfscope}%
\begin{pgfscope}%
\pgfpathrectangle{\pgfqpoint{0.481978in}{0.331635in}}{\pgfqpoint{9.300000in}{7.700000in}}%
\pgfusepath{clip}%
\pgfsetrectcap%
\pgfsetroundjoin%
\pgfsetlinewidth{1.505625pt}%
\definecolor{currentstroke}{rgb}{1.000000,0.705882,0.509804}%
\pgfsetstrokecolor{currentstroke}%
\pgfsetstrokeopacity{0.800000}%
\pgfsetdash{}{0pt}%
\pgfpathmoveto{\pgfqpoint{6.102927in}{7.282227in}}%
\pgfpathlineto{\pgfqpoint{5.573130in}{4.513635in}}%
\pgfusepath{stroke}%
\end{pgfscope}%
\begin{pgfscope}%
\pgfpathrectangle{\pgfqpoint{0.481978in}{0.331635in}}{\pgfqpoint{9.300000in}{7.700000in}}%
\pgfusepath{clip}%
\pgfsetrectcap%
\pgfsetroundjoin%
\pgfsetlinewidth{1.505625pt}%
\definecolor{currentstroke}{rgb}{1.000000,0.705882,0.509804}%
\pgfsetstrokecolor{currentstroke}%
\pgfsetstrokeopacity{0.800000}%
\pgfsetdash{}{0pt}%
\pgfpathmoveto{\pgfqpoint{5.548025in}{3.750685in}}%
\pgfpathlineto{\pgfqpoint{5.573130in}{4.513635in}}%
\pgfusepath{stroke}%
\end{pgfscope}%
\begin{pgfscope}%
\pgfpathrectangle{\pgfqpoint{0.481978in}{0.331635in}}{\pgfqpoint{9.300000in}{7.700000in}}%
\pgfusepath{clip}%
\pgfsetrectcap%
\pgfsetroundjoin%
\pgfsetlinewidth{1.505625pt}%
\definecolor{currentstroke}{rgb}{1.000000,0.705882,0.509804}%
\pgfsetstrokecolor{currentstroke}%
\pgfsetstrokeopacity{0.800000}%
\pgfsetdash{}{0pt}%
\pgfpathmoveto{\pgfqpoint{4.725147in}{4.271416in}}%
\pgfpathlineto{\pgfqpoint{5.573130in}{4.513635in}}%
\pgfusepath{stroke}%
\end{pgfscope}%
\begin{pgfscope}%
\pgfpathrectangle{\pgfqpoint{0.481978in}{0.331635in}}{\pgfqpoint{9.300000in}{7.700000in}}%
\pgfusepath{clip}%
\pgfsetrectcap%
\pgfsetroundjoin%
\pgfsetlinewidth{1.505625pt}%
\definecolor{currentstroke}{rgb}{1.000000,0.705882,0.509804}%
\pgfsetstrokecolor{currentstroke}%
\pgfsetstrokeopacity{0.800000}%
\pgfsetdash{}{0pt}%
\pgfpathmoveto{\pgfqpoint{7.998254in}{3.646651in}}%
\pgfpathlineto{\pgfqpoint{5.573130in}{4.513635in}}%
\pgfusepath{stroke}%
\end{pgfscope}%
\begin{pgfscope}%
\pgfpathrectangle{\pgfqpoint{0.481978in}{0.331635in}}{\pgfqpoint{9.300000in}{7.700000in}}%
\pgfusepath{clip}%
\pgfsetrectcap%
\pgfsetroundjoin%
\pgfsetlinewidth{1.505625pt}%
\definecolor{currentstroke}{rgb}{1.000000,0.705882,0.509804}%
\pgfsetstrokecolor{currentstroke}%
\pgfsetstrokeopacity{0.800000}%
\pgfsetdash{}{0pt}%
\pgfpathmoveto{\pgfqpoint{5.102532in}{4.527559in}}%
\pgfpathlineto{\pgfqpoint{5.573130in}{4.513635in}}%
\pgfusepath{stroke}%
\end{pgfscope}%
\begin{pgfscope}%
\pgfpathrectangle{\pgfqpoint{0.481978in}{0.331635in}}{\pgfqpoint{9.300000in}{7.700000in}}%
\pgfusepath{clip}%
\pgfsetrectcap%
\pgfsetroundjoin%
\pgfsetlinewidth{1.505625pt}%
\definecolor{currentstroke}{rgb}{1.000000,0.705882,0.509804}%
\pgfsetstrokecolor{currentstroke}%
\pgfsetstrokeopacity{0.800000}%
\pgfsetdash{}{0pt}%
\pgfpathmoveto{\pgfqpoint{4.679104in}{4.744402in}}%
\pgfpathlineto{\pgfqpoint{5.573130in}{4.513635in}}%
\pgfusepath{stroke}%
\end{pgfscope}%
\begin{pgfscope}%
\pgfpathrectangle{\pgfqpoint{0.481978in}{0.331635in}}{\pgfqpoint{9.300000in}{7.700000in}}%
\pgfusepath{clip}%
\pgfsetrectcap%
\pgfsetroundjoin%
\pgfsetlinewidth{1.505625pt}%
\definecolor{currentstroke}{rgb}{1.000000,0.705882,0.509804}%
\pgfsetstrokecolor{currentstroke}%
\pgfsetstrokeopacity{0.800000}%
\pgfsetdash{}{0pt}%
\pgfpathmoveto{\pgfqpoint{4.103990in}{2.296516in}}%
\pgfpathlineto{\pgfqpoint{5.573130in}{4.513635in}}%
\pgfusepath{stroke}%
\end{pgfscope}%
\begin{pgfscope}%
\pgfpathrectangle{\pgfqpoint{0.481978in}{0.331635in}}{\pgfqpoint{9.300000in}{7.700000in}}%
\pgfusepath{clip}%
\pgfsetrectcap%
\pgfsetroundjoin%
\pgfsetlinewidth{1.505625pt}%
\definecolor{currentstroke}{rgb}{1.000000,0.705882,0.509804}%
\pgfsetstrokecolor{currentstroke}%
\pgfsetstrokeopacity{0.800000}%
\pgfsetdash{}{0pt}%
\pgfpathmoveto{\pgfqpoint{4.530179in}{2.903130in}}%
\pgfpathlineto{\pgfqpoint{5.573130in}{4.513635in}}%
\pgfusepath{stroke}%
\end{pgfscope}%
\begin{pgfscope}%
\pgfpathrectangle{\pgfqpoint{0.481978in}{0.331635in}}{\pgfqpoint{9.300000in}{7.700000in}}%
\pgfusepath{clip}%
\pgfsetrectcap%
\pgfsetroundjoin%
\pgfsetlinewidth{1.505625pt}%
\definecolor{currentstroke}{rgb}{1.000000,0.705882,0.509804}%
\pgfsetstrokecolor{currentstroke}%
\pgfsetstrokeopacity{0.800000}%
\pgfsetdash{}{0pt}%
\pgfpathmoveto{\pgfqpoint{3.358304in}{6.036591in}}%
\pgfpathlineto{\pgfqpoint{5.573130in}{4.513635in}}%
\pgfusepath{stroke}%
\end{pgfscope}%
\begin{pgfscope}%
\pgfpathrectangle{\pgfqpoint{0.481978in}{0.331635in}}{\pgfqpoint{9.300000in}{7.700000in}}%
\pgfusepath{clip}%
\pgfsetrectcap%
\pgfsetroundjoin%
\pgfsetlinewidth{1.505625pt}%
\definecolor{currentstroke}{rgb}{1.000000,0.705882,0.509804}%
\pgfsetstrokecolor{currentstroke}%
\pgfsetstrokeopacity{0.800000}%
\pgfsetdash{}{0pt}%
\pgfpathmoveto{\pgfqpoint{5.398084in}{2.632414in}}%
\pgfpathlineto{\pgfqpoint{5.573130in}{4.513635in}}%
\pgfusepath{stroke}%
\end{pgfscope}%
\begin{pgfscope}%
\pgfpathrectangle{\pgfqpoint{0.481978in}{0.331635in}}{\pgfqpoint{9.300000in}{7.700000in}}%
\pgfusepath{clip}%
\pgfsetrectcap%
\pgfsetroundjoin%
\pgfsetlinewidth{1.505625pt}%
\definecolor{currentstroke}{rgb}{1.000000,0.705882,0.509804}%
\pgfsetstrokecolor{currentstroke}%
\pgfsetstrokeopacity{0.800000}%
\pgfsetdash{}{0pt}%
\pgfpathmoveto{\pgfqpoint{8.487669in}{6.437918in}}%
\pgfpathlineto{\pgfqpoint{5.573130in}{4.513635in}}%
\pgfusepath{stroke}%
\end{pgfscope}%
\begin{pgfscope}%
\pgfpathrectangle{\pgfqpoint{0.481978in}{0.331635in}}{\pgfqpoint{9.300000in}{7.700000in}}%
\pgfusepath{clip}%
\pgfsetrectcap%
\pgfsetroundjoin%
\pgfsetlinewidth{1.505625pt}%
\definecolor{currentstroke}{rgb}{1.000000,0.705882,0.509804}%
\pgfsetstrokecolor{currentstroke}%
\pgfsetstrokeopacity{0.800000}%
\pgfsetdash{}{0pt}%
\pgfpathmoveto{\pgfqpoint{4.319461in}{3.871786in}}%
\pgfpathlineto{\pgfqpoint{5.573130in}{4.513635in}}%
\pgfusepath{stroke}%
\end{pgfscope}%
\begin{pgfscope}%
\pgfpathrectangle{\pgfqpoint{0.481978in}{0.331635in}}{\pgfqpoint{9.300000in}{7.700000in}}%
\pgfusepath{clip}%
\pgfsetrectcap%
\pgfsetroundjoin%
\pgfsetlinewidth{1.505625pt}%
\definecolor{currentstroke}{rgb}{1.000000,0.705882,0.509804}%
\pgfsetstrokecolor{currentstroke}%
\pgfsetstrokeopacity{0.800000}%
\pgfsetdash{}{0pt}%
\pgfpathmoveto{\pgfqpoint{5.195031in}{3.322603in}}%
\pgfpathlineto{\pgfqpoint{5.573130in}{4.513635in}}%
\pgfusepath{stroke}%
\end{pgfscope}%
\begin{pgfscope}%
\pgfpathrectangle{\pgfqpoint{0.481978in}{0.331635in}}{\pgfqpoint{9.300000in}{7.700000in}}%
\pgfusepath{clip}%
\pgfsetrectcap%
\pgfsetroundjoin%
\pgfsetlinewidth{1.505625pt}%
\definecolor{currentstroke}{rgb}{1.000000,0.705882,0.509804}%
\pgfsetstrokecolor{currentstroke}%
\pgfsetstrokeopacity{0.800000}%
\pgfsetdash{}{0pt}%
\pgfpathmoveto{\pgfqpoint{6.327559in}{7.608142in}}%
\pgfpathlineto{\pgfqpoint{5.573130in}{4.513635in}}%
\pgfusepath{stroke}%
\end{pgfscope}%
\begin{pgfscope}%
\pgfpathrectangle{\pgfqpoint{0.481978in}{0.331635in}}{\pgfqpoint{9.300000in}{7.700000in}}%
\pgfusepath{clip}%
\pgfsetrectcap%
\pgfsetroundjoin%
\pgfsetlinewidth{1.505625pt}%
\definecolor{currentstroke}{rgb}{1.000000,0.705882,0.509804}%
\pgfsetstrokecolor{currentstroke}%
\pgfsetstrokeopacity{0.800000}%
\pgfsetdash{}{0pt}%
\pgfpathmoveto{\pgfqpoint{6.512708in}{3.269966in}}%
\pgfpathlineto{\pgfqpoint{5.573130in}{4.513635in}}%
\pgfusepath{stroke}%
\end{pgfscope}%
\begin{pgfscope}%
\pgfpathrectangle{\pgfqpoint{0.481978in}{0.331635in}}{\pgfqpoint{9.300000in}{7.700000in}}%
\pgfusepath{clip}%
\pgfsetrectcap%
\pgfsetroundjoin%
\pgfsetlinewidth{1.505625pt}%
\definecolor{currentstroke}{rgb}{1.000000,0.705882,0.509804}%
\pgfsetstrokecolor{currentstroke}%
\pgfsetstrokeopacity{0.800000}%
\pgfsetdash{}{0pt}%
\pgfpathmoveto{\pgfqpoint{5.258302in}{2.797545in}}%
\pgfpathlineto{\pgfqpoint{5.573130in}{4.513635in}}%
\pgfusepath{stroke}%
\end{pgfscope}%
\begin{pgfscope}%
\pgfpathrectangle{\pgfqpoint{0.481978in}{0.331635in}}{\pgfqpoint{9.300000in}{7.700000in}}%
\pgfusepath{clip}%
\pgfsetrectcap%
\pgfsetroundjoin%
\pgfsetlinewidth{1.505625pt}%
\definecolor{currentstroke}{rgb}{1.000000,0.705882,0.509804}%
\pgfsetstrokecolor{currentstroke}%
\pgfsetstrokeopacity{0.800000}%
\pgfsetdash{}{0pt}%
\pgfpathmoveto{\pgfqpoint{8.656102in}{6.093871in}}%
\pgfpathlineto{\pgfqpoint{5.573130in}{4.513635in}}%
\pgfusepath{stroke}%
\end{pgfscope}%
\begin{pgfscope}%
\pgfpathrectangle{\pgfqpoint{0.481978in}{0.331635in}}{\pgfqpoint{9.300000in}{7.700000in}}%
\pgfusepath{clip}%
\pgfsetrectcap%
\pgfsetroundjoin%
\pgfsetlinewidth{1.505625pt}%
\definecolor{currentstroke}{rgb}{1.000000,0.705882,0.509804}%
\pgfsetstrokecolor{currentstroke}%
\pgfsetstrokeopacity{0.800000}%
\pgfsetdash{}{0pt}%
\pgfpathmoveto{\pgfqpoint{5.476507in}{4.216476in}}%
\pgfpathlineto{\pgfqpoint{5.573130in}{4.513635in}}%
\pgfusepath{stroke}%
\end{pgfscope}%
\begin{pgfscope}%
\pgfpathrectangle{\pgfqpoint{0.481978in}{0.331635in}}{\pgfqpoint{9.300000in}{7.700000in}}%
\pgfusepath{clip}%
\pgfsetrectcap%
\pgfsetroundjoin%
\pgfsetlinewidth{1.505625pt}%
\definecolor{currentstroke}{rgb}{1.000000,0.705882,0.509804}%
\pgfsetstrokecolor{currentstroke}%
\pgfsetstrokeopacity{0.800000}%
\pgfsetdash{}{0pt}%
\pgfpathmoveto{\pgfqpoint{7.379869in}{3.289361in}}%
\pgfpathlineto{\pgfqpoint{5.573130in}{4.513635in}}%
\pgfusepath{stroke}%
\end{pgfscope}%
\begin{pgfscope}%
\pgfpathrectangle{\pgfqpoint{0.481978in}{0.331635in}}{\pgfqpoint{9.300000in}{7.700000in}}%
\pgfusepath{clip}%
\pgfsetrectcap%
\pgfsetroundjoin%
\pgfsetlinewidth{1.505625pt}%
\definecolor{currentstroke}{rgb}{1.000000,0.705882,0.509804}%
\pgfsetstrokecolor{currentstroke}%
\pgfsetstrokeopacity{0.800000}%
\pgfsetdash{}{0pt}%
\pgfpathmoveto{\pgfqpoint{6.352024in}{2.774781in}}%
\pgfpathlineto{\pgfqpoint{5.573130in}{4.513635in}}%
\pgfusepath{stroke}%
\end{pgfscope}%
\begin{pgfscope}%
\pgfpathrectangle{\pgfqpoint{0.481978in}{0.331635in}}{\pgfqpoint{9.300000in}{7.700000in}}%
\pgfusepath{clip}%
\pgfsetrectcap%
\pgfsetroundjoin%
\pgfsetlinewidth{1.505625pt}%
\definecolor{currentstroke}{rgb}{1.000000,0.705882,0.509804}%
\pgfsetstrokecolor{currentstroke}%
\pgfsetstrokeopacity{0.800000}%
\pgfsetdash{}{0pt}%
\pgfpathmoveto{\pgfqpoint{7.010012in}{6.655032in}}%
\pgfpathlineto{\pgfqpoint{5.573130in}{4.513635in}}%
\pgfusepath{stroke}%
\end{pgfscope}%
\begin{pgfscope}%
\pgfpathrectangle{\pgfqpoint{0.481978in}{0.331635in}}{\pgfqpoint{9.300000in}{7.700000in}}%
\pgfusepath{clip}%
\pgfsetrectcap%
\pgfsetroundjoin%
\pgfsetlinewidth{1.505625pt}%
\definecolor{currentstroke}{rgb}{1.000000,0.705882,0.509804}%
\pgfsetstrokecolor{currentstroke}%
\pgfsetstrokeopacity{0.800000}%
\pgfsetdash{}{0pt}%
\pgfpathmoveto{\pgfqpoint{4.957439in}{2.266043in}}%
\pgfpathlineto{\pgfqpoint{5.573130in}{4.513635in}}%
\pgfusepath{stroke}%
\end{pgfscope}%
\begin{pgfscope}%
\pgfpathrectangle{\pgfqpoint{0.481978in}{0.331635in}}{\pgfqpoint{9.300000in}{7.700000in}}%
\pgfusepath{clip}%
\pgfsetrectcap%
\pgfsetroundjoin%
\pgfsetlinewidth{1.505625pt}%
\definecolor{currentstroke}{rgb}{1.000000,0.705882,0.509804}%
\pgfsetstrokecolor{currentstroke}%
\pgfsetstrokeopacity{0.800000}%
\pgfsetdash{}{0pt}%
\pgfpathmoveto{\pgfqpoint{1.938901in}{5.964179in}}%
\pgfpathlineto{\pgfqpoint{5.573130in}{4.513635in}}%
\pgfusepath{stroke}%
\end{pgfscope}%
\begin{pgfscope}%
\pgfpathrectangle{\pgfqpoint{0.481978in}{0.331635in}}{\pgfqpoint{9.300000in}{7.700000in}}%
\pgfusepath{clip}%
\pgfsetrectcap%
\pgfsetroundjoin%
\pgfsetlinewidth{1.505625pt}%
\definecolor{currentstroke}{rgb}{1.000000,0.705882,0.509804}%
\pgfsetstrokecolor{currentstroke}%
\pgfsetstrokeopacity{0.800000}%
\pgfsetdash{}{0pt}%
\pgfpathmoveto{\pgfqpoint{6.653857in}{7.012761in}}%
\pgfpathlineto{\pgfqpoint{5.573130in}{4.513635in}}%
\pgfusepath{stroke}%
\end{pgfscope}%
\begin{pgfscope}%
\pgfpathrectangle{\pgfqpoint{0.481978in}{0.331635in}}{\pgfqpoint{9.300000in}{7.700000in}}%
\pgfusepath{clip}%
\pgfsetrectcap%
\pgfsetroundjoin%
\pgfsetlinewidth{1.505625pt}%
\definecolor{currentstroke}{rgb}{1.000000,0.705882,0.509804}%
\pgfsetstrokecolor{currentstroke}%
\pgfsetstrokeopacity{0.800000}%
\pgfsetdash{}{0pt}%
\pgfpathmoveto{\pgfqpoint{7.825213in}{7.158741in}}%
\pgfpathlineto{\pgfqpoint{5.573130in}{4.513635in}}%
\pgfusepath{stroke}%
\end{pgfscope}%
\begin{pgfscope}%
\pgfpathrectangle{\pgfqpoint{0.481978in}{0.331635in}}{\pgfqpoint{9.300000in}{7.700000in}}%
\pgfusepath{clip}%
\pgfsetrectcap%
\pgfsetroundjoin%
\pgfsetlinewidth{1.505625pt}%
\definecolor{currentstroke}{rgb}{1.000000,0.705882,0.509804}%
\pgfsetstrokecolor{currentstroke}%
\pgfsetstrokeopacity{0.800000}%
\pgfsetdash{}{0pt}%
\pgfpathmoveto{\pgfqpoint{4.408915in}{2.547313in}}%
\pgfpathlineto{\pgfqpoint{5.573130in}{4.513635in}}%
\pgfusepath{stroke}%
\end{pgfscope}%
\begin{pgfscope}%
\pgfpathrectangle{\pgfqpoint{0.481978in}{0.331635in}}{\pgfqpoint{9.300000in}{7.700000in}}%
\pgfusepath{clip}%
\pgfsetrectcap%
\pgfsetroundjoin%
\pgfsetlinewidth{1.505625pt}%
\definecolor{currentstroke}{rgb}{1.000000,0.705882,0.509804}%
\pgfsetstrokecolor{currentstroke}%
\pgfsetstrokeopacity{0.800000}%
\pgfsetdash{}{0pt}%
\pgfpathmoveto{\pgfqpoint{0.904705in}{3.161161in}}%
\pgfpathlineto{\pgfqpoint{5.573130in}{4.513635in}}%
\pgfusepath{stroke}%
\end{pgfscope}%
\begin{pgfscope}%
\pgfpathrectangle{\pgfqpoint{0.481978in}{0.331635in}}{\pgfqpoint{9.300000in}{7.700000in}}%
\pgfusepath{clip}%
\pgfsetrectcap%
\pgfsetroundjoin%
\pgfsetlinewidth{1.505625pt}%
\definecolor{currentstroke}{rgb}{1.000000,0.705882,0.509804}%
\pgfsetstrokecolor{currentstroke}%
\pgfsetstrokeopacity{0.800000}%
\pgfsetdash{}{0pt}%
\pgfpathmoveto{\pgfqpoint{8.224572in}{3.345931in}}%
\pgfpathlineto{\pgfqpoint{5.573130in}{4.513635in}}%
\pgfusepath{stroke}%
\end{pgfscope}%
\begin{pgfscope}%
\pgfsetrectcap%
\pgfsetmiterjoin%
\pgfsetlinewidth{0.803000pt}%
\definecolor{currentstroke}{rgb}{0.000000,0.000000,0.000000}%
\pgfsetstrokecolor{currentstroke}%
\pgfsetdash{}{0pt}%
\pgfpathmoveto{\pgfqpoint{0.481978in}{0.331635in}}%
\pgfpathlineto{\pgfqpoint{0.481978in}{8.031635in}}%
\pgfusepath{stroke}%
\end{pgfscope}%
\begin{pgfscope}%
\pgfsetrectcap%
\pgfsetmiterjoin%
\pgfsetlinewidth{0.803000pt}%
\definecolor{currentstroke}{rgb}{0.000000,0.000000,0.000000}%
\pgfsetstrokecolor{currentstroke}%
\pgfsetdash{}{0pt}%
\pgfpathmoveto{\pgfqpoint{9.781978in}{0.331635in}}%
\pgfpathlineto{\pgfqpoint{9.781978in}{8.031635in}}%
\pgfusepath{stroke}%
\end{pgfscope}%
\begin{pgfscope}%
\pgfsetrectcap%
\pgfsetmiterjoin%
\pgfsetlinewidth{0.803000pt}%
\definecolor{currentstroke}{rgb}{0.000000,0.000000,0.000000}%
\pgfsetstrokecolor{currentstroke}%
\pgfsetdash{}{0pt}%
\pgfpathmoveto{\pgfqpoint{0.481978in}{0.331635in}}%
\pgfpathlineto{\pgfqpoint{9.781978in}{0.331635in}}%
\pgfusepath{stroke}%
\end{pgfscope}%
\begin{pgfscope}%
\pgfsetrectcap%
\pgfsetmiterjoin%
\pgfsetlinewidth{0.803000pt}%
\definecolor{currentstroke}{rgb}{0.000000,0.000000,0.000000}%
\pgfsetstrokecolor{currentstroke}%
\pgfsetdash{}{0pt}%
\pgfpathmoveto{\pgfqpoint{0.481978in}{8.031635in}}%
\pgfpathlineto{\pgfqpoint{9.781978in}{8.031635in}}%
\pgfusepath{stroke}%
\end{pgfscope}%
\begin{pgfscope}%
\definecolor{textcolor}{rgb}{0.000000,0.000000,0.000000}%
\pgfsetstrokecolor{textcolor}%
\pgfsetfillcolor{textcolor}%
\pgftext[x=5.131978in,y=8.114968in,,base]{\color{textcolor}\sffamily\fontsize{12.000000}{14.400000}\selectfont T-SNE for chair images (s2r3dfree\_textureless)}%
\end{pgfscope}%
\begin{pgfscope}%
\pgfsetbuttcap%
\pgfsetmiterjoin%
\definecolor{currentfill}{rgb}{1.000000,1.000000,1.000000}%
\pgfsetfillcolor{currentfill}%
\pgfsetfillopacity{0.800000}%
\pgfsetlinewidth{1.003750pt}%
\definecolor{currentstroke}{rgb}{0.800000,0.800000,0.800000}%
\pgfsetstrokecolor{currentstroke}%
\pgfsetstrokeopacity{0.800000}%
\pgfsetdash{}{0pt}%
\pgfpathmoveto{\pgfqpoint{9.879200in}{3.955012in}}%
\pgfpathlineto{\pgfqpoint{11.831688in}{3.955012in}}%
\pgfpathquadraticcurveto{\pgfqpoint{11.859465in}{3.955012in}}{\pgfqpoint{11.859465in}{3.982789in}}%
\pgfpathlineto{\pgfqpoint{11.859465in}{4.380481in}}%
\pgfpathquadraticcurveto{\pgfqpoint{11.859465in}{4.408258in}}{\pgfqpoint{11.831688in}{4.408258in}}%
\pgfpathlineto{\pgfqpoint{9.879200in}{4.408258in}}%
\pgfpathquadraticcurveto{\pgfqpoint{9.851422in}{4.408258in}}{\pgfqpoint{9.851422in}{4.380481in}}%
\pgfpathlineto{\pgfqpoint{9.851422in}{3.982789in}}%
\pgfpathquadraticcurveto{\pgfqpoint{9.851422in}{3.955012in}}{\pgfqpoint{9.879200in}{3.955012in}}%
\pgfpathclose%
\pgfusepath{stroke,fill}%
\end{pgfscope}%
\begin{pgfscope}%
\pgfsetbuttcap%
\pgfsetroundjoin%
\definecolor{currentfill}{rgb}{0.631373,0.788235,0.956863}%
\pgfsetfillcolor{currentfill}%
\pgfsetlinewidth{1.003750pt}%
\definecolor{currentstroke}{rgb}{0.631373,0.788235,0.956863}%
\pgfsetstrokecolor{currentstroke}%
\pgfsetdash{}{0pt}%
\pgfsys@defobject{currentmarker}{\pgfqpoint{-0.041667in}{-0.041667in}}{\pgfqpoint{0.041667in}{0.041667in}}{%
\pgfpathmoveto{\pgfqpoint{0.000000in}{-0.041667in}}%
\pgfpathcurveto{\pgfqpoint{0.011050in}{-0.041667in}}{\pgfqpoint{0.021649in}{-0.037276in}}{\pgfqpoint{0.029463in}{-0.029463in}}%
\pgfpathcurveto{\pgfqpoint{0.037276in}{-0.021649in}}{\pgfqpoint{0.041667in}{-0.011050in}}{\pgfqpoint{0.041667in}{0.000000in}}%
\pgfpathcurveto{\pgfqpoint{0.041667in}{0.011050in}}{\pgfqpoint{0.037276in}{0.021649in}}{\pgfqpoint{0.029463in}{0.029463in}}%
\pgfpathcurveto{\pgfqpoint{0.021649in}{0.037276in}}{\pgfqpoint{0.011050in}{0.041667in}}{\pgfqpoint{0.000000in}{0.041667in}}%
\pgfpathcurveto{\pgfqpoint{-0.011050in}{0.041667in}}{\pgfqpoint{-0.021649in}{0.037276in}}{\pgfqpoint{-0.029463in}{0.029463in}}%
\pgfpathcurveto{\pgfqpoint{-0.037276in}{0.021649in}}{\pgfqpoint{-0.041667in}{0.011050in}}{\pgfqpoint{-0.041667in}{0.000000in}}%
\pgfpathcurveto{\pgfqpoint{-0.041667in}{-0.011050in}}{\pgfqpoint{-0.037276in}{-0.021649in}}{\pgfqpoint{-0.029463in}{-0.029463in}}%
\pgfpathcurveto{\pgfqpoint{-0.021649in}{-0.037276in}}{\pgfqpoint{-0.011050in}{-0.041667in}}{\pgfqpoint{0.000000in}{-0.041667in}}%
\pgfpathclose%
\pgfusepath{stroke,fill}%
}%
\begin{pgfscope}%
\pgfsys@transformshift{10.045867in}{4.283638in}%
\pgfsys@useobject{currentmarker}{}%
\end{pgfscope}%
\end{pgfscope}%
\begin{pgfscope}%
\definecolor{textcolor}{rgb}{0.000000,0.000000,0.000000}%
\pgfsetstrokecolor{textcolor}%
\pgfsetfillcolor{textcolor}%
\pgftext[x=10.295867in,y=4.247180in,left,base]{\color{textcolor}\sffamily\fontsize{10.000000}{12.000000}\selectfont Pix3D}%
\end{pgfscope}%
\begin{pgfscope}%
\pgfsetbuttcap%
\pgfsetroundjoin%
\definecolor{currentfill}{rgb}{1.000000,0.705882,0.509804}%
\pgfsetfillcolor{currentfill}%
\pgfsetlinewidth{1.003750pt}%
\definecolor{currentstroke}{rgb}{1.000000,0.705882,0.509804}%
\pgfsetstrokecolor{currentstroke}%
\pgfsetdash{}{0pt}%
\pgfsys@defobject{currentmarker}{\pgfqpoint{-0.041667in}{-0.041667in}}{\pgfqpoint{0.041667in}{0.041667in}}{%
\pgfpathmoveto{\pgfqpoint{0.000000in}{-0.041667in}}%
\pgfpathcurveto{\pgfqpoint{0.011050in}{-0.041667in}}{\pgfqpoint{0.021649in}{-0.037276in}}{\pgfqpoint{0.029463in}{-0.029463in}}%
\pgfpathcurveto{\pgfqpoint{0.037276in}{-0.021649in}}{\pgfqpoint{0.041667in}{-0.011050in}}{\pgfqpoint{0.041667in}{0.000000in}}%
\pgfpathcurveto{\pgfqpoint{0.041667in}{0.011050in}}{\pgfqpoint{0.037276in}{0.021649in}}{\pgfqpoint{0.029463in}{0.029463in}}%
\pgfpathcurveto{\pgfqpoint{0.021649in}{0.037276in}}{\pgfqpoint{0.011050in}{0.041667in}}{\pgfqpoint{0.000000in}{0.041667in}}%
\pgfpathcurveto{\pgfqpoint{-0.011050in}{0.041667in}}{\pgfqpoint{-0.021649in}{0.037276in}}{\pgfqpoint{-0.029463in}{0.029463in}}%
\pgfpathcurveto{\pgfqpoint{-0.037276in}{0.021649in}}{\pgfqpoint{-0.041667in}{0.011050in}}{\pgfqpoint{-0.041667in}{0.000000in}}%
\pgfpathcurveto{\pgfqpoint{-0.041667in}{-0.011050in}}{\pgfqpoint{-0.037276in}{-0.021649in}}{\pgfqpoint{-0.029463in}{-0.029463in}}%
\pgfpathcurveto{\pgfqpoint{-0.021649in}{-0.037276in}}{\pgfqpoint{-0.011050in}{-0.041667in}}{\pgfqpoint{0.000000in}{-0.041667in}}%
\pgfpathclose%
\pgfusepath{stroke,fill}%
}%
\begin{pgfscope}%
\pgfsys@transformshift{10.045867in}{4.079781in}%
\pgfsys@useobject{currentmarker}{}%
\end{pgfscope}%
\end{pgfscope}%
\begin{pgfscope}%
\definecolor{textcolor}{rgb}{0.000000,0.000000,0.000000}%
\pgfsetstrokecolor{textcolor}%
\pgfsetfillcolor{textcolor}%
\pgftext[x=10.295867in,y=4.043322in,left,base]{\color{textcolor}\sffamily\fontsize{10.000000}{12.000000}\selectfont s2r3dfree\_textureless}%
\end{pgfscope}%
\end{pgfpicture}%
\makeatother%
\endgroup%
}
    \resizebox{0.49\linewidth}{6cm}{%% Creator: Matplotlib, PGF backend
%%
%% To include the figure in your LaTeX document, write
%%   \input{<filename>.pgf}
%%
%% Make sure the required packages are loaded in your preamble
%%   \usepackage{pgf}
%%
%% Figures using additional raster images can only be included by \input if
%% they are in the same directory as the main LaTeX file. For loading figures
%% from other directories you can use the `import` package
%%   \usepackage{import}
%%
%% and then include the figures with
%%   \import{<path to file>}{<filename>.pgf}
%%
%% Matplotlib used the following preamble
%%   \usepackage{fontspec}
%%   \setmainfont{DejaVuSerif.ttf}[Path=\detokenize{/Users/apple/opt/anaconda3/envs/kaolin/lib/python3.7/site-packages/matplotlib/mpl-data/fonts/ttf/}]
%%   \setsansfont{DejaVuSans.ttf}[Path=\detokenize{/Users/apple/opt/anaconda3/envs/kaolin/lib/python3.7/site-packages/matplotlib/mpl-data/fonts/ttf/}]
%%   \setmonofont{DejaVuSansMono.ttf}[Path=\detokenize{/Users/apple/opt/anaconda3/envs/kaolin/lib/python3.7/site-packages/matplotlib/mpl-data/fonts/ttf/}]
%%
\begingroup%
\makeatletter%
\begin{pgfpicture}%
\pgfpathrectangle{\pgfpointorigin}{\pgfqpoint{12.248365in}{8.341596in}}%
\pgfusepath{use as bounding box, clip}%
\begin{pgfscope}%
\pgfsetbuttcap%
\pgfsetmiterjoin%
\definecolor{currentfill}{rgb}{1.000000,1.000000,1.000000}%
\pgfsetfillcolor{currentfill}%
\pgfsetlinewidth{0.000000pt}%
\definecolor{currentstroke}{rgb}{1.000000,1.000000,1.000000}%
\pgfsetstrokecolor{currentstroke}%
\pgfsetdash{}{0pt}%
\pgfpathmoveto{\pgfqpoint{-0.000000in}{0.000000in}}%
\pgfpathlineto{\pgfqpoint{12.248365in}{0.000000in}}%
\pgfpathlineto{\pgfqpoint{12.248365in}{8.341596in}}%
\pgfpathlineto{\pgfqpoint{-0.000000in}{8.341596in}}%
\pgfpathclose%
\pgfusepath{fill}%
\end{pgfscope}%
\begin{pgfscope}%
\pgfsetbuttcap%
\pgfsetmiterjoin%
\definecolor{currentfill}{rgb}{1.000000,1.000000,1.000000}%
\pgfsetfillcolor{currentfill}%
\pgfsetlinewidth{0.000000pt}%
\definecolor{currentstroke}{rgb}{0.000000,0.000000,0.000000}%
\pgfsetstrokecolor{currentstroke}%
\pgfsetstrokeopacity{0.000000}%
\pgfsetdash{}{0pt}%
\pgfpathmoveto{\pgfqpoint{0.393613in}{0.331635in}}%
\pgfpathlineto{\pgfqpoint{9.693613in}{0.331635in}}%
\pgfpathlineto{\pgfqpoint{9.693613in}{8.031635in}}%
\pgfpathlineto{\pgfqpoint{0.393613in}{8.031635in}}%
\pgfpathclose%
\pgfusepath{fill}%
\end{pgfscope}%
\begin{pgfscope}%
\pgfpathrectangle{\pgfqpoint{0.393613in}{0.331635in}}{\pgfqpoint{9.300000in}{7.700000in}}%
\pgfusepath{clip}%
\pgfsetbuttcap%
\pgfsetroundjoin%
\definecolor{currentfill}{rgb}{0.631373,0.788235,0.956863}%
\pgfsetfillcolor{currentfill}%
\pgfsetlinewidth{0.481800pt}%
\definecolor{currentstroke}{rgb}{1.000000,1.000000,1.000000}%
\pgfsetstrokecolor{currentstroke}%
\pgfsetdash{}{0pt}%
\pgfpathmoveto{\pgfqpoint{2.878164in}{1.516475in}}%
\pgfpathcurveto{\pgfqpoint{2.889214in}{1.516475in}}{\pgfqpoint{2.899813in}{1.520865in}}{\pgfqpoint{2.907626in}{1.528679in}}%
\pgfpathcurveto{\pgfqpoint{2.915440in}{1.536492in}}{\pgfqpoint{2.919830in}{1.547091in}}{\pgfqpoint{2.919830in}{1.558141in}}%
\pgfpathcurveto{\pgfqpoint{2.919830in}{1.569191in}}{\pgfqpoint{2.915440in}{1.579790in}}{\pgfqpoint{2.907626in}{1.587604in}}%
\pgfpathcurveto{\pgfqpoint{2.899813in}{1.595418in}}{\pgfqpoint{2.889214in}{1.599808in}}{\pgfqpoint{2.878164in}{1.599808in}}%
\pgfpathcurveto{\pgfqpoint{2.867113in}{1.599808in}}{\pgfqpoint{2.856514in}{1.595418in}}{\pgfqpoint{2.848701in}{1.587604in}}%
\pgfpathcurveto{\pgfqpoint{2.840887in}{1.579790in}}{\pgfqpoint{2.836497in}{1.569191in}}{\pgfqpoint{2.836497in}{1.558141in}}%
\pgfpathcurveto{\pgfqpoint{2.836497in}{1.547091in}}{\pgfqpoint{2.840887in}{1.536492in}}{\pgfqpoint{2.848701in}{1.528679in}}%
\pgfpathcurveto{\pgfqpoint{2.856514in}{1.520865in}}{\pgfqpoint{2.867113in}{1.516475in}}{\pgfqpoint{2.878164in}{1.516475in}}%
\pgfpathclose%
\pgfusepath{stroke,fill}%
\end{pgfscope}%
\begin{pgfscope}%
\pgfpathrectangle{\pgfqpoint{0.393613in}{0.331635in}}{\pgfqpoint{9.300000in}{7.700000in}}%
\pgfusepath{clip}%
\pgfsetbuttcap%
\pgfsetroundjoin%
\definecolor{currentfill}{rgb}{0.631373,0.788235,0.956863}%
\pgfsetfillcolor{currentfill}%
\pgfsetlinewidth{0.481800pt}%
\definecolor{currentstroke}{rgb}{1.000000,1.000000,1.000000}%
\pgfsetstrokecolor{currentstroke}%
\pgfsetdash{}{0pt}%
\pgfpathmoveto{\pgfqpoint{1.610944in}{5.101202in}}%
\pgfpathcurveto{\pgfqpoint{1.621994in}{5.101202in}}{\pgfqpoint{1.632593in}{5.105593in}}{\pgfqpoint{1.640407in}{5.113406in}}%
\pgfpathcurveto{\pgfqpoint{1.648220in}{5.121220in}}{\pgfqpoint{1.652611in}{5.131819in}}{\pgfqpoint{1.652611in}{5.142869in}}%
\pgfpathcurveto{\pgfqpoint{1.652611in}{5.153919in}}{\pgfqpoint{1.648220in}{5.164518in}}{\pgfqpoint{1.640407in}{5.172332in}}%
\pgfpathcurveto{\pgfqpoint{1.632593in}{5.180145in}}{\pgfqpoint{1.621994in}{5.184536in}}{\pgfqpoint{1.610944in}{5.184536in}}%
\pgfpathcurveto{\pgfqpoint{1.599894in}{5.184536in}}{\pgfqpoint{1.589295in}{5.180145in}}{\pgfqpoint{1.581481in}{5.172332in}}%
\pgfpathcurveto{\pgfqpoint{1.573667in}{5.164518in}}{\pgfqpoint{1.569277in}{5.153919in}}{\pgfqpoint{1.569277in}{5.142869in}}%
\pgfpathcurveto{\pgfqpoint{1.569277in}{5.131819in}}{\pgfqpoint{1.573667in}{5.121220in}}{\pgfqpoint{1.581481in}{5.113406in}}%
\pgfpathcurveto{\pgfqpoint{1.589295in}{5.105593in}}{\pgfqpoint{1.599894in}{5.101202in}}{\pgfqpoint{1.610944in}{5.101202in}}%
\pgfpathclose%
\pgfusepath{stroke,fill}%
\end{pgfscope}%
\begin{pgfscope}%
\pgfpathrectangle{\pgfqpoint{0.393613in}{0.331635in}}{\pgfqpoint{9.300000in}{7.700000in}}%
\pgfusepath{clip}%
\pgfsetbuttcap%
\pgfsetroundjoin%
\definecolor{currentfill}{rgb}{0.631373,0.788235,0.956863}%
\pgfsetfillcolor{currentfill}%
\pgfsetlinewidth{0.481800pt}%
\definecolor{currentstroke}{rgb}{1.000000,1.000000,1.000000}%
\pgfsetstrokecolor{currentstroke}%
\pgfsetdash{}{0pt}%
\pgfpathmoveto{\pgfqpoint{4.136764in}{5.848270in}}%
\pgfpathcurveto{\pgfqpoint{4.147814in}{5.848270in}}{\pgfqpoint{4.158413in}{5.852660in}}{\pgfqpoint{4.166227in}{5.860474in}}%
\pgfpathcurveto{\pgfqpoint{4.174040in}{5.868288in}}{\pgfqpoint{4.178431in}{5.878887in}}{\pgfqpoint{4.178431in}{5.889937in}}%
\pgfpathcurveto{\pgfqpoint{4.178431in}{5.900987in}}{\pgfqpoint{4.174040in}{5.911586in}}{\pgfqpoint{4.166227in}{5.919399in}}%
\pgfpathcurveto{\pgfqpoint{4.158413in}{5.927213in}}{\pgfqpoint{4.147814in}{5.931603in}}{\pgfqpoint{4.136764in}{5.931603in}}%
\pgfpathcurveto{\pgfqpoint{4.125714in}{5.931603in}}{\pgfqpoint{4.115115in}{5.927213in}}{\pgfqpoint{4.107301in}{5.919399in}}%
\pgfpathcurveto{\pgfqpoint{4.099488in}{5.911586in}}{\pgfqpoint{4.095097in}{5.900987in}}{\pgfqpoint{4.095097in}{5.889937in}}%
\pgfpathcurveto{\pgfqpoint{4.095097in}{5.878887in}}{\pgfqpoint{4.099488in}{5.868288in}}{\pgfqpoint{4.107301in}{5.860474in}}%
\pgfpathcurveto{\pgfqpoint{4.115115in}{5.852660in}}{\pgfqpoint{4.125714in}{5.848270in}}{\pgfqpoint{4.136764in}{5.848270in}}%
\pgfpathclose%
\pgfusepath{stroke,fill}%
\end{pgfscope}%
\begin{pgfscope}%
\pgfpathrectangle{\pgfqpoint{0.393613in}{0.331635in}}{\pgfqpoint{9.300000in}{7.700000in}}%
\pgfusepath{clip}%
\pgfsetbuttcap%
\pgfsetroundjoin%
\definecolor{currentfill}{rgb}{0.631373,0.788235,0.956863}%
\pgfsetfillcolor{currentfill}%
\pgfsetlinewidth{0.481800pt}%
\definecolor{currentstroke}{rgb}{1.000000,1.000000,1.000000}%
\pgfsetstrokecolor{currentstroke}%
\pgfsetdash{}{0pt}%
\pgfpathmoveto{\pgfqpoint{3.118402in}{4.346395in}}%
\pgfpathcurveto{\pgfqpoint{3.129452in}{4.346395in}}{\pgfqpoint{3.140051in}{4.350785in}}{\pgfqpoint{3.147865in}{4.358599in}}%
\pgfpathcurveto{\pgfqpoint{3.155678in}{4.366412in}}{\pgfqpoint{3.160068in}{4.377011in}}{\pgfqpoint{3.160068in}{4.388061in}}%
\pgfpathcurveto{\pgfqpoint{3.160068in}{4.399112in}}{\pgfqpoint{3.155678in}{4.409711in}}{\pgfqpoint{3.147865in}{4.417524in}}%
\pgfpathcurveto{\pgfqpoint{3.140051in}{4.425338in}}{\pgfqpoint{3.129452in}{4.429728in}}{\pgfqpoint{3.118402in}{4.429728in}}%
\pgfpathcurveto{\pgfqpoint{3.107352in}{4.429728in}}{\pgfqpoint{3.096753in}{4.425338in}}{\pgfqpoint{3.088939in}{4.417524in}}%
\pgfpathcurveto{\pgfqpoint{3.081125in}{4.409711in}}{\pgfqpoint{3.076735in}{4.399112in}}{\pgfqpoint{3.076735in}{4.388061in}}%
\pgfpathcurveto{\pgfqpoint{3.076735in}{4.377011in}}{\pgfqpoint{3.081125in}{4.366412in}}{\pgfqpoint{3.088939in}{4.358599in}}%
\pgfpathcurveto{\pgfqpoint{3.096753in}{4.350785in}}{\pgfqpoint{3.107352in}{4.346395in}}{\pgfqpoint{3.118402in}{4.346395in}}%
\pgfpathclose%
\pgfusepath{stroke,fill}%
\end{pgfscope}%
\begin{pgfscope}%
\pgfpathrectangle{\pgfqpoint{0.393613in}{0.331635in}}{\pgfqpoint{9.300000in}{7.700000in}}%
\pgfusepath{clip}%
\pgfsetbuttcap%
\pgfsetroundjoin%
\definecolor{currentfill}{rgb}{0.631373,0.788235,0.956863}%
\pgfsetfillcolor{currentfill}%
\pgfsetlinewidth{0.481800pt}%
\definecolor{currentstroke}{rgb}{1.000000,1.000000,1.000000}%
\pgfsetstrokecolor{currentstroke}%
\pgfsetdash{}{0pt}%
\pgfpathmoveto{\pgfqpoint{3.792263in}{1.975665in}}%
\pgfpathcurveto{\pgfqpoint{3.803313in}{1.975665in}}{\pgfqpoint{3.813912in}{1.980055in}}{\pgfqpoint{3.821725in}{1.987869in}}%
\pgfpathcurveto{\pgfqpoint{3.829539in}{1.995682in}}{\pgfqpoint{3.833929in}{2.006282in}}{\pgfqpoint{3.833929in}{2.017332in}}%
\pgfpathcurveto{\pgfqpoint{3.833929in}{2.028382in}}{\pgfqpoint{3.829539in}{2.038981in}}{\pgfqpoint{3.821725in}{2.046794in}}%
\pgfpathcurveto{\pgfqpoint{3.813912in}{2.054608in}}{\pgfqpoint{3.803313in}{2.058998in}}{\pgfqpoint{3.792263in}{2.058998in}}%
\pgfpathcurveto{\pgfqpoint{3.781212in}{2.058998in}}{\pgfqpoint{3.770613in}{2.054608in}}{\pgfqpoint{3.762800in}{2.046794in}}%
\pgfpathcurveto{\pgfqpoint{3.754986in}{2.038981in}}{\pgfqpoint{3.750596in}{2.028382in}}{\pgfqpoint{3.750596in}{2.017332in}}%
\pgfpathcurveto{\pgfqpoint{3.750596in}{2.006282in}}{\pgfqpoint{3.754986in}{1.995682in}}{\pgfqpoint{3.762800in}{1.987869in}}%
\pgfpathcurveto{\pgfqpoint{3.770613in}{1.980055in}}{\pgfqpoint{3.781212in}{1.975665in}}{\pgfqpoint{3.792263in}{1.975665in}}%
\pgfpathclose%
\pgfusepath{stroke,fill}%
\end{pgfscope}%
\begin{pgfscope}%
\pgfpathrectangle{\pgfqpoint{0.393613in}{0.331635in}}{\pgfqpoint{9.300000in}{7.700000in}}%
\pgfusepath{clip}%
\pgfsetbuttcap%
\pgfsetroundjoin%
\definecolor{currentfill}{rgb}{0.631373,0.788235,0.956863}%
\pgfsetfillcolor{currentfill}%
\pgfsetlinewidth{0.481800pt}%
\definecolor{currentstroke}{rgb}{1.000000,1.000000,1.000000}%
\pgfsetstrokecolor{currentstroke}%
\pgfsetdash{}{0pt}%
\pgfpathmoveto{\pgfqpoint{1.145785in}{5.805643in}}%
\pgfpathcurveto{\pgfqpoint{1.156835in}{5.805643in}}{\pgfqpoint{1.167434in}{5.810034in}}{\pgfqpoint{1.175247in}{5.817847in}}%
\pgfpathcurveto{\pgfqpoint{1.183061in}{5.825661in}}{\pgfqpoint{1.187451in}{5.836260in}}{\pgfqpoint{1.187451in}{5.847310in}}%
\pgfpathcurveto{\pgfqpoint{1.187451in}{5.858360in}}{\pgfqpoint{1.183061in}{5.868959in}}{\pgfqpoint{1.175247in}{5.876773in}}%
\pgfpathcurveto{\pgfqpoint{1.167434in}{5.884586in}}{\pgfqpoint{1.156835in}{5.888977in}}{\pgfqpoint{1.145785in}{5.888977in}}%
\pgfpathcurveto{\pgfqpoint{1.134734in}{5.888977in}}{\pgfqpoint{1.124135in}{5.884586in}}{\pgfqpoint{1.116322in}{5.876773in}}%
\pgfpathcurveto{\pgfqpoint{1.108508in}{5.868959in}}{\pgfqpoint{1.104118in}{5.858360in}}{\pgfqpoint{1.104118in}{5.847310in}}%
\pgfpathcurveto{\pgfqpoint{1.104118in}{5.836260in}}{\pgfqpoint{1.108508in}{5.825661in}}{\pgfqpoint{1.116322in}{5.817847in}}%
\pgfpathcurveto{\pgfqpoint{1.124135in}{5.810034in}}{\pgfqpoint{1.134734in}{5.805643in}}{\pgfqpoint{1.145785in}{5.805643in}}%
\pgfpathclose%
\pgfusepath{stroke,fill}%
\end{pgfscope}%
\begin{pgfscope}%
\pgfpathrectangle{\pgfqpoint{0.393613in}{0.331635in}}{\pgfqpoint{9.300000in}{7.700000in}}%
\pgfusepath{clip}%
\pgfsetbuttcap%
\pgfsetroundjoin%
\definecolor{currentfill}{rgb}{0.631373,0.788235,0.956863}%
\pgfsetfillcolor{currentfill}%
\pgfsetlinewidth{0.481800pt}%
\definecolor{currentstroke}{rgb}{1.000000,1.000000,1.000000}%
\pgfsetstrokecolor{currentstroke}%
\pgfsetdash{}{0pt}%
\pgfpathmoveto{\pgfqpoint{2.952892in}{1.892306in}}%
\pgfpathcurveto{\pgfqpoint{2.963942in}{1.892306in}}{\pgfqpoint{2.974541in}{1.896697in}}{\pgfqpoint{2.982355in}{1.904510in}}%
\pgfpathcurveto{\pgfqpoint{2.990168in}{1.912324in}}{\pgfqpoint{2.994559in}{1.922923in}}{\pgfqpoint{2.994559in}{1.933973in}}%
\pgfpathcurveto{\pgfqpoint{2.994559in}{1.945023in}}{\pgfqpoint{2.990168in}{1.955622in}}{\pgfqpoint{2.982355in}{1.963436in}}%
\pgfpathcurveto{\pgfqpoint{2.974541in}{1.971249in}}{\pgfqpoint{2.963942in}{1.975640in}}{\pgfqpoint{2.952892in}{1.975640in}}%
\pgfpathcurveto{\pgfqpoint{2.941842in}{1.975640in}}{\pgfqpoint{2.931243in}{1.971249in}}{\pgfqpoint{2.923429in}{1.963436in}}%
\pgfpathcurveto{\pgfqpoint{2.915616in}{1.955622in}}{\pgfqpoint{2.911225in}{1.945023in}}{\pgfqpoint{2.911225in}{1.933973in}}%
\pgfpathcurveto{\pgfqpoint{2.911225in}{1.922923in}}{\pgfqpoint{2.915616in}{1.912324in}}{\pgfqpoint{2.923429in}{1.904510in}}%
\pgfpathcurveto{\pgfqpoint{2.931243in}{1.896697in}}{\pgfqpoint{2.941842in}{1.892306in}}{\pgfqpoint{2.952892in}{1.892306in}}%
\pgfpathclose%
\pgfusepath{stroke,fill}%
\end{pgfscope}%
\begin{pgfscope}%
\pgfpathrectangle{\pgfqpoint{0.393613in}{0.331635in}}{\pgfqpoint{9.300000in}{7.700000in}}%
\pgfusepath{clip}%
\pgfsetbuttcap%
\pgfsetroundjoin%
\definecolor{currentfill}{rgb}{0.631373,0.788235,0.956863}%
\pgfsetfillcolor{currentfill}%
\pgfsetlinewidth{0.481800pt}%
\definecolor{currentstroke}{rgb}{1.000000,1.000000,1.000000}%
\pgfsetstrokecolor{currentstroke}%
\pgfsetdash{}{0pt}%
\pgfpathmoveto{\pgfqpoint{4.435471in}{5.526080in}}%
\pgfpathcurveto{\pgfqpoint{4.446521in}{5.526080in}}{\pgfqpoint{4.457120in}{5.530471in}}{\pgfqpoint{4.464933in}{5.538284in}}%
\pgfpathcurveto{\pgfqpoint{4.472747in}{5.546098in}}{\pgfqpoint{4.477137in}{5.556697in}}{\pgfqpoint{4.477137in}{5.567747in}}%
\pgfpathcurveto{\pgfqpoint{4.477137in}{5.578797in}}{\pgfqpoint{4.472747in}{5.589396in}}{\pgfqpoint{4.464933in}{5.597210in}}%
\pgfpathcurveto{\pgfqpoint{4.457120in}{5.605023in}}{\pgfqpoint{4.446521in}{5.609414in}}{\pgfqpoint{4.435471in}{5.609414in}}%
\pgfpathcurveto{\pgfqpoint{4.424421in}{5.609414in}}{\pgfqpoint{4.413822in}{5.605023in}}{\pgfqpoint{4.406008in}{5.597210in}}%
\pgfpathcurveto{\pgfqpoint{4.398194in}{5.589396in}}{\pgfqpoint{4.393804in}{5.578797in}}{\pgfqpoint{4.393804in}{5.567747in}}%
\pgfpathcurveto{\pgfqpoint{4.393804in}{5.556697in}}{\pgfqpoint{4.398194in}{5.546098in}}{\pgfqpoint{4.406008in}{5.538284in}}%
\pgfpathcurveto{\pgfqpoint{4.413822in}{5.530471in}}{\pgfqpoint{4.424421in}{5.526080in}}{\pgfqpoint{4.435471in}{5.526080in}}%
\pgfpathclose%
\pgfusepath{stroke,fill}%
\end{pgfscope}%
\begin{pgfscope}%
\pgfpathrectangle{\pgfqpoint{0.393613in}{0.331635in}}{\pgfqpoint{9.300000in}{7.700000in}}%
\pgfusepath{clip}%
\pgfsetbuttcap%
\pgfsetroundjoin%
\definecolor{currentfill}{rgb}{0.631373,0.788235,0.956863}%
\pgfsetfillcolor{currentfill}%
\pgfsetlinewidth{0.481800pt}%
\definecolor{currentstroke}{rgb}{1.000000,1.000000,1.000000}%
\pgfsetstrokecolor{currentstroke}%
\pgfsetdash{}{0pt}%
\pgfpathmoveto{\pgfqpoint{1.974238in}{5.767666in}}%
\pgfpathcurveto{\pgfqpoint{1.985288in}{5.767666in}}{\pgfqpoint{1.995887in}{5.772056in}}{\pgfqpoint{2.003700in}{5.779869in}}%
\pgfpathcurveto{\pgfqpoint{2.011514in}{5.787683in}}{\pgfqpoint{2.015904in}{5.798282in}}{\pgfqpoint{2.015904in}{5.809332in}}%
\pgfpathcurveto{\pgfqpoint{2.015904in}{5.820382in}}{\pgfqpoint{2.011514in}{5.830981in}}{\pgfqpoint{2.003700in}{5.838795in}}%
\pgfpathcurveto{\pgfqpoint{1.995887in}{5.846609in}}{\pgfqpoint{1.985288in}{5.850999in}}{\pgfqpoint{1.974238in}{5.850999in}}%
\pgfpathcurveto{\pgfqpoint{1.963187in}{5.850999in}}{\pgfqpoint{1.952588in}{5.846609in}}{\pgfqpoint{1.944775in}{5.838795in}}%
\pgfpathcurveto{\pgfqpoint{1.936961in}{5.830981in}}{\pgfqpoint{1.932571in}{5.820382in}}{\pgfqpoint{1.932571in}{5.809332in}}%
\pgfpathcurveto{\pgfqpoint{1.932571in}{5.798282in}}{\pgfqpoint{1.936961in}{5.787683in}}{\pgfqpoint{1.944775in}{5.779869in}}%
\pgfpathcurveto{\pgfqpoint{1.952588in}{5.772056in}}{\pgfqpoint{1.963187in}{5.767666in}}{\pgfqpoint{1.974238in}{5.767666in}}%
\pgfpathclose%
\pgfusepath{stroke,fill}%
\end{pgfscope}%
\begin{pgfscope}%
\pgfpathrectangle{\pgfqpoint{0.393613in}{0.331635in}}{\pgfqpoint{9.300000in}{7.700000in}}%
\pgfusepath{clip}%
\pgfsetbuttcap%
\pgfsetroundjoin%
\definecolor{currentfill}{rgb}{0.631373,0.788235,0.956863}%
\pgfsetfillcolor{currentfill}%
\pgfsetlinewidth{0.481800pt}%
\definecolor{currentstroke}{rgb}{1.000000,1.000000,1.000000}%
\pgfsetstrokecolor{currentstroke}%
\pgfsetdash{}{0pt}%
\pgfpathmoveto{\pgfqpoint{2.379728in}{6.669063in}}%
\pgfpathcurveto{\pgfqpoint{2.390778in}{6.669063in}}{\pgfqpoint{2.401377in}{6.673453in}}{\pgfqpoint{2.409191in}{6.681267in}}%
\pgfpathcurveto{\pgfqpoint{2.417004in}{6.689081in}}{\pgfqpoint{2.421395in}{6.699680in}}{\pgfqpoint{2.421395in}{6.710730in}}%
\pgfpathcurveto{\pgfqpoint{2.421395in}{6.721780in}}{\pgfqpoint{2.417004in}{6.732379in}}{\pgfqpoint{2.409191in}{6.740192in}}%
\pgfpathcurveto{\pgfqpoint{2.401377in}{6.748006in}}{\pgfqpoint{2.390778in}{6.752396in}}{\pgfqpoint{2.379728in}{6.752396in}}%
\pgfpathcurveto{\pgfqpoint{2.368678in}{6.752396in}}{\pgfqpoint{2.358079in}{6.748006in}}{\pgfqpoint{2.350265in}{6.740192in}}%
\pgfpathcurveto{\pgfqpoint{2.342451in}{6.732379in}}{\pgfqpoint{2.338061in}{6.721780in}}{\pgfqpoint{2.338061in}{6.710730in}}%
\pgfpathcurveto{\pgfqpoint{2.338061in}{6.699680in}}{\pgfqpoint{2.342451in}{6.689081in}}{\pgfqpoint{2.350265in}{6.681267in}}%
\pgfpathcurveto{\pgfqpoint{2.358079in}{6.673453in}}{\pgfqpoint{2.368678in}{6.669063in}}{\pgfqpoint{2.379728in}{6.669063in}}%
\pgfpathclose%
\pgfusepath{stroke,fill}%
\end{pgfscope}%
\begin{pgfscope}%
\pgfpathrectangle{\pgfqpoint{0.393613in}{0.331635in}}{\pgfqpoint{9.300000in}{7.700000in}}%
\pgfusepath{clip}%
\pgfsetbuttcap%
\pgfsetroundjoin%
\definecolor{currentfill}{rgb}{0.631373,0.788235,0.956863}%
\pgfsetfillcolor{currentfill}%
\pgfsetlinewidth{0.481800pt}%
\definecolor{currentstroke}{rgb}{1.000000,1.000000,1.000000}%
\pgfsetstrokecolor{currentstroke}%
\pgfsetdash{}{0pt}%
\pgfpathmoveto{\pgfqpoint{2.765242in}{4.850114in}}%
\pgfpathcurveto{\pgfqpoint{2.776292in}{4.850114in}}{\pgfqpoint{2.786891in}{4.854505in}}{\pgfqpoint{2.794705in}{4.862318in}}%
\pgfpathcurveto{\pgfqpoint{2.802518in}{4.870132in}}{\pgfqpoint{2.806908in}{4.880731in}}{\pgfqpoint{2.806908in}{4.891781in}}%
\pgfpathcurveto{\pgfqpoint{2.806908in}{4.902831in}}{\pgfqpoint{2.802518in}{4.913430in}}{\pgfqpoint{2.794705in}{4.921244in}}%
\pgfpathcurveto{\pgfqpoint{2.786891in}{4.929057in}}{\pgfqpoint{2.776292in}{4.933448in}}{\pgfqpoint{2.765242in}{4.933448in}}%
\pgfpathcurveto{\pgfqpoint{2.754192in}{4.933448in}}{\pgfqpoint{2.743593in}{4.929057in}}{\pgfqpoint{2.735779in}{4.921244in}}%
\pgfpathcurveto{\pgfqpoint{2.727965in}{4.913430in}}{\pgfqpoint{2.723575in}{4.902831in}}{\pgfqpoint{2.723575in}{4.891781in}}%
\pgfpathcurveto{\pgfqpoint{2.723575in}{4.880731in}}{\pgfqpoint{2.727965in}{4.870132in}}{\pgfqpoint{2.735779in}{4.862318in}}%
\pgfpathcurveto{\pgfqpoint{2.743593in}{4.854505in}}{\pgfqpoint{2.754192in}{4.850114in}}{\pgfqpoint{2.765242in}{4.850114in}}%
\pgfpathclose%
\pgfusepath{stroke,fill}%
\end{pgfscope}%
\begin{pgfscope}%
\pgfpathrectangle{\pgfqpoint{0.393613in}{0.331635in}}{\pgfqpoint{9.300000in}{7.700000in}}%
\pgfusepath{clip}%
\pgfsetbuttcap%
\pgfsetroundjoin%
\definecolor{currentfill}{rgb}{0.631373,0.788235,0.956863}%
\pgfsetfillcolor{currentfill}%
\pgfsetlinewidth{0.481800pt}%
\definecolor{currentstroke}{rgb}{1.000000,1.000000,1.000000}%
\pgfsetstrokecolor{currentstroke}%
\pgfsetdash{}{0pt}%
\pgfpathmoveto{\pgfqpoint{2.626630in}{6.501501in}}%
\pgfpathcurveto{\pgfqpoint{2.637680in}{6.501501in}}{\pgfqpoint{2.648279in}{6.505892in}}{\pgfqpoint{2.656093in}{6.513705in}}%
\pgfpathcurveto{\pgfqpoint{2.663906in}{6.521519in}}{\pgfqpoint{2.668296in}{6.532118in}}{\pgfqpoint{2.668296in}{6.543168in}}%
\pgfpathcurveto{\pgfqpoint{2.668296in}{6.554218in}}{\pgfqpoint{2.663906in}{6.564817in}}{\pgfqpoint{2.656093in}{6.572631in}}%
\pgfpathcurveto{\pgfqpoint{2.648279in}{6.580445in}}{\pgfqpoint{2.637680in}{6.584835in}}{\pgfqpoint{2.626630in}{6.584835in}}%
\pgfpathcurveto{\pgfqpoint{2.615580in}{6.584835in}}{\pgfqpoint{2.604981in}{6.580445in}}{\pgfqpoint{2.597167in}{6.572631in}}%
\pgfpathcurveto{\pgfqpoint{2.589353in}{6.564817in}}{\pgfqpoint{2.584963in}{6.554218in}}{\pgfqpoint{2.584963in}{6.543168in}}%
\pgfpathcurveto{\pgfqpoint{2.584963in}{6.532118in}}{\pgfqpoint{2.589353in}{6.521519in}}{\pgfqpoint{2.597167in}{6.513705in}}%
\pgfpathcurveto{\pgfqpoint{2.604981in}{6.505892in}}{\pgfqpoint{2.615580in}{6.501501in}}{\pgfqpoint{2.626630in}{6.501501in}}%
\pgfpathclose%
\pgfusepath{stroke,fill}%
\end{pgfscope}%
\begin{pgfscope}%
\pgfpathrectangle{\pgfqpoint{0.393613in}{0.331635in}}{\pgfqpoint{9.300000in}{7.700000in}}%
\pgfusepath{clip}%
\pgfsetbuttcap%
\pgfsetroundjoin%
\definecolor{currentfill}{rgb}{0.631373,0.788235,0.956863}%
\pgfsetfillcolor{currentfill}%
\pgfsetlinewidth{0.481800pt}%
\definecolor{currentstroke}{rgb}{1.000000,1.000000,1.000000}%
\pgfsetstrokecolor{currentstroke}%
\pgfsetdash{}{0pt}%
\pgfpathmoveto{\pgfqpoint{3.301695in}{5.285888in}}%
\pgfpathcurveto{\pgfqpoint{3.312745in}{5.285888in}}{\pgfqpoint{3.323344in}{5.290279in}}{\pgfqpoint{3.331157in}{5.298092in}}%
\pgfpathcurveto{\pgfqpoint{3.338971in}{5.305906in}}{\pgfqpoint{3.343361in}{5.316505in}}{\pgfqpoint{3.343361in}{5.327555in}}%
\pgfpathcurveto{\pgfqpoint{3.343361in}{5.338605in}}{\pgfqpoint{3.338971in}{5.349204in}}{\pgfqpoint{3.331157in}{5.357018in}}%
\pgfpathcurveto{\pgfqpoint{3.323344in}{5.364831in}}{\pgfqpoint{3.312745in}{5.369222in}}{\pgfqpoint{3.301695in}{5.369222in}}%
\pgfpathcurveto{\pgfqpoint{3.290644in}{5.369222in}}{\pgfqpoint{3.280045in}{5.364831in}}{\pgfqpoint{3.272232in}{5.357018in}}%
\pgfpathcurveto{\pgfqpoint{3.264418in}{5.349204in}}{\pgfqpoint{3.260028in}{5.338605in}}{\pgfqpoint{3.260028in}{5.327555in}}%
\pgfpathcurveto{\pgfqpoint{3.260028in}{5.316505in}}{\pgfqpoint{3.264418in}{5.305906in}}{\pgfqpoint{3.272232in}{5.298092in}}%
\pgfpathcurveto{\pgfqpoint{3.280045in}{5.290279in}}{\pgfqpoint{3.290644in}{5.285888in}}{\pgfqpoint{3.301695in}{5.285888in}}%
\pgfpathclose%
\pgfusepath{stroke,fill}%
\end{pgfscope}%
\begin{pgfscope}%
\pgfpathrectangle{\pgfqpoint{0.393613in}{0.331635in}}{\pgfqpoint{9.300000in}{7.700000in}}%
\pgfusepath{clip}%
\pgfsetbuttcap%
\pgfsetroundjoin%
\definecolor{currentfill}{rgb}{0.631373,0.788235,0.956863}%
\pgfsetfillcolor{currentfill}%
\pgfsetlinewidth{0.481800pt}%
\definecolor{currentstroke}{rgb}{1.000000,1.000000,1.000000}%
\pgfsetstrokecolor{currentstroke}%
\pgfsetdash{}{0pt}%
\pgfpathmoveto{\pgfqpoint{0.968107in}{5.855717in}}%
\pgfpathcurveto{\pgfqpoint{0.979157in}{5.855717in}}{\pgfqpoint{0.989756in}{5.860108in}}{\pgfqpoint{0.997570in}{5.867921in}}%
\pgfpathcurveto{\pgfqpoint{1.005384in}{5.875735in}}{\pgfqpoint{1.009774in}{5.886334in}}{\pgfqpoint{1.009774in}{5.897384in}}%
\pgfpathcurveto{\pgfqpoint{1.009774in}{5.908434in}}{\pgfqpoint{1.005384in}{5.919033in}}{\pgfqpoint{0.997570in}{5.926847in}}%
\pgfpathcurveto{\pgfqpoint{0.989756in}{5.934660in}}{\pgfqpoint{0.979157in}{5.939051in}}{\pgfqpoint{0.968107in}{5.939051in}}%
\pgfpathcurveto{\pgfqpoint{0.957057in}{5.939051in}}{\pgfqpoint{0.946458in}{5.934660in}}{\pgfqpoint{0.938644in}{5.926847in}}%
\pgfpathcurveto{\pgfqpoint{0.930831in}{5.919033in}}{\pgfqpoint{0.926441in}{5.908434in}}{\pgfqpoint{0.926441in}{5.897384in}}%
\pgfpathcurveto{\pgfqpoint{0.926441in}{5.886334in}}{\pgfqpoint{0.930831in}{5.875735in}}{\pgfqpoint{0.938644in}{5.867921in}}%
\pgfpathcurveto{\pgfqpoint{0.946458in}{5.860108in}}{\pgfqpoint{0.957057in}{5.855717in}}{\pgfqpoint{0.968107in}{5.855717in}}%
\pgfpathclose%
\pgfusepath{stroke,fill}%
\end{pgfscope}%
\begin{pgfscope}%
\pgfpathrectangle{\pgfqpoint{0.393613in}{0.331635in}}{\pgfqpoint{9.300000in}{7.700000in}}%
\pgfusepath{clip}%
\pgfsetbuttcap%
\pgfsetroundjoin%
\definecolor{currentfill}{rgb}{0.631373,0.788235,0.956863}%
\pgfsetfillcolor{currentfill}%
\pgfsetlinewidth{0.481800pt}%
\definecolor{currentstroke}{rgb}{1.000000,1.000000,1.000000}%
\pgfsetstrokecolor{currentstroke}%
\pgfsetdash{}{0pt}%
\pgfpathmoveto{\pgfqpoint{4.228775in}{6.130862in}}%
\pgfpathcurveto{\pgfqpoint{4.239826in}{6.130862in}}{\pgfqpoint{4.250425in}{6.135252in}}{\pgfqpoint{4.258238in}{6.143066in}}%
\pgfpathcurveto{\pgfqpoint{4.266052in}{6.150879in}}{\pgfqpoint{4.270442in}{6.161478in}}{\pgfqpoint{4.270442in}{6.172529in}}%
\pgfpathcurveto{\pgfqpoint{4.270442in}{6.183579in}}{\pgfqpoint{4.266052in}{6.194178in}}{\pgfqpoint{4.258238in}{6.201991in}}%
\pgfpathcurveto{\pgfqpoint{4.250425in}{6.209805in}}{\pgfqpoint{4.239826in}{6.214195in}}{\pgfqpoint{4.228775in}{6.214195in}}%
\pgfpathcurveto{\pgfqpoint{4.217725in}{6.214195in}}{\pgfqpoint{4.207126in}{6.209805in}}{\pgfqpoint{4.199313in}{6.201991in}}%
\pgfpathcurveto{\pgfqpoint{4.191499in}{6.194178in}}{\pgfqpoint{4.187109in}{6.183579in}}{\pgfqpoint{4.187109in}{6.172529in}}%
\pgfpathcurveto{\pgfqpoint{4.187109in}{6.161478in}}{\pgfqpoint{4.191499in}{6.150879in}}{\pgfqpoint{4.199313in}{6.143066in}}%
\pgfpathcurveto{\pgfqpoint{4.207126in}{6.135252in}}{\pgfqpoint{4.217725in}{6.130862in}}{\pgfqpoint{4.228775in}{6.130862in}}%
\pgfpathclose%
\pgfusepath{stroke,fill}%
\end{pgfscope}%
\begin{pgfscope}%
\pgfpathrectangle{\pgfqpoint{0.393613in}{0.331635in}}{\pgfqpoint{9.300000in}{7.700000in}}%
\pgfusepath{clip}%
\pgfsetbuttcap%
\pgfsetroundjoin%
\definecolor{currentfill}{rgb}{0.631373,0.788235,0.956863}%
\pgfsetfillcolor{currentfill}%
\pgfsetlinewidth{0.481800pt}%
\definecolor{currentstroke}{rgb}{1.000000,1.000000,1.000000}%
\pgfsetstrokecolor{currentstroke}%
\pgfsetdash{}{0pt}%
\pgfpathmoveto{\pgfqpoint{2.425460in}{6.990714in}}%
\pgfpathcurveto{\pgfqpoint{2.436510in}{6.990714in}}{\pgfqpoint{2.447109in}{6.995104in}}{\pgfqpoint{2.454922in}{7.002918in}}%
\pgfpathcurveto{\pgfqpoint{2.462736in}{7.010731in}}{\pgfqpoint{2.467126in}{7.021330in}}{\pgfqpoint{2.467126in}{7.032380in}}%
\pgfpathcurveto{\pgfqpoint{2.467126in}{7.043431in}}{\pgfqpoint{2.462736in}{7.054030in}}{\pgfqpoint{2.454922in}{7.061843in}}%
\pgfpathcurveto{\pgfqpoint{2.447109in}{7.069657in}}{\pgfqpoint{2.436510in}{7.074047in}}{\pgfqpoint{2.425460in}{7.074047in}}%
\pgfpathcurveto{\pgfqpoint{2.414409in}{7.074047in}}{\pgfqpoint{2.403810in}{7.069657in}}{\pgfqpoint{2.395997in}{7.061843in}}%
\pgfpathcurveto{\pgfqpoint{2.388183in}{7.054030in}}{\pgfqpoint{2.383793in}{7.043431in}}{\pgfqpoint{2.383793in}{7.032380in}}%
\pgfpathcurveto{\pgfqpoint{2.383793in}{7.021330in}}{\pgfqpoint{2.388183in}{7.010731in}}{\pgfqpoint{2.395997in}{7.002918in}}%
\pgfpathcurveto{\pgfqpoint{2.403810in}{6.995104in}}{\pgfqpoint{2.414409in}{6.990714in}}{\pgfqpoint{2.425460in}{6.990714in}}%
\pgfpathclose%
\pgfusepath{stroke,fill}%
\end{pgfscope}%
\begin{pgfscope}%
\pgfpathrectangle{\pgfqpoint{0.393613in}{0.331635in}}{\pgfqpoint{9.300000in}{7.700000in}}%
\pgfusepath{clip}%
\pgfsetbuttcap%
\pgfsetroundjoin%
\definecolor{currentfill}{rgb}{0.631373,0.788235,0.956863}%
\pgfsetfillcolor{currentfill}%
\pgfsetlinewidth{0.481800pt}%
\definecolor{currentstroke}{rgb}{1.000000,1.000000,1.000000}%
\pgfsetstrokecolor{currentstroke}%
\pgfsetdash{}{0pt}%
\pgfpathmoveto{\pgfqpoint{2.491424in}{5.665706in}}%
\pgfpathcurveto{\pgfqpoint{2.502474in}{5.665706in}}{\pgfqpoint{2.513073in}{5.670096in}}{\pgfqpoint{2.520887in}{5.677910in}}%
\pgfpathcurveto{\pgfqpoint{2.528700in}{5.685723in}}{\pgfqpoint{2.533091in}{5.696322in}}{\pgfqpoint{2.533091in}{5.707373in}}%
\pgfpathcurveto{\pgfqpoint{2.533091in}{5.718423in}}{\pgfqpoint{2.528700in}{5.729022in}}{\pgfqpoint{2.520887in}{5.736835in}}%
\pgfpathcurveto{\pgfqpoint{2.513073in}{5.744649in}}{\pgfqpoint{2.502474in}{5.749039in}}{\pgfqpoint{2.491424in}{5.749039in}}%
\pgfpathcurveto{\pgfqpoint{2.480374in}{5.749039in}}{\pgfqpoint{2.469775in}{5.744649in}}{\pgfqpoint{2.461961in}{5.736835in}}%
\pgfpathcurveto{\pgfqpoint{2.454148in}{5.729022in}}{\pgfqpoint{2.449757in}{5.718423in}}{\pgfqpoint{2.449757in}{5.707373in}}%
\pgfpathcurveto{\pgfqpoint{2.449757in}{5.696322in}}{\pgfqpoint{2.454148in}{5.685723in}}{\pgfqpoint{2.461961in}{5.677910in}}%
\pgfpathcurveto{\pgfqpoint{2.469775in}{5.670096in}}{\pgfqpoint{2.480374in}{5.665706in}}{\pgfqpoint{2.491424in}{5.665706in}}%
\pgfpathclose%
\pgfusepath{stroke,fill}%
\end{pgfscope}%
\begin{pgfscope}%
\pgfpathrectangle{\pgfqpoint{0.393613in}{0.331635in}}{\pgfqpoint{9.300000in}{7.700000in}}%
\pgfusepath{clip}%
\pgfsetbuttcap%
\pgfsetroundjoin%
\definecolor{currentfill}{rgb}{0.631373,0.788235,0.956863}%
\pgfsetfillcolor{currentfill}%
\pgfsetlinewidth{0.481800pt}%
\definecolor{currentstroke}{rgb}{1.000000,1.000000,1.000000}%
\pgfsetstrokecolor{currentstroke}%
\pgfsetdash{}{0pt}%
\pgfpathmoveto{\pgfqpoint{5.405752in}{7.030029in}}%
\pgfpathcurveto{\pgfqpoint{5.416802in}{7.030029in}}{\pgfqpoint{5.427401in}{7.034419in}}{\pgfqpoint{5.435215in}{7.042233in}}%
\pgfpathcurveto{\pgfqpoint{5.443029in}{7.050046in}}{\pgfqpoint{5.447419in}{7.060645in}}{\pgfqpoint{5.447419in}{7.071695in}}%
\pgfpathcurveto{\pgfqpoint{5.447419in}{7.082746in}}{\pgfqpoint{5.443029in}{7.093345in}}{\pgfqpoint{5.435215in}{7.101158in}}%
\pgfpathcurveto{\pgfqpoint{5.427401in}{7.108972in}}{\pgfqpoint{5.416802in}{7.113362in}}{\pgfqpoint{5.405752in}{7.113362in}}%
\pgfpathcurveto{\pgfqpoint{5.394702in}{7.113362in}}{\pgfqpoint{5.384103in}{7.108972in}}{\pgfqpoint{5.376289in}{7.101158in}}%
\pgfpathcurveto{\pgfqpoint{5.368476in}{7.093345in}}{\pgfqpoint{5.364086in}{7.082746in}}{\pgfqpoint{5.364086in}{7.071695in}}%
\pgfpathcurveto{\pgfqpoint{5.364086in}{7.060645in}}{\pgfqpoint{5.368476in}{7.050046in}}{\pgfqpoint{5.376289in}{7.042233in}}%
\pgfpathcurveto{\pgfqpoint{5.384103in}{7.034419in}}{\pgfqpoint{5.394702in}{7.030029in}}{\pgfqpoint{5.405752in}{7.030029in}}%
\pgfpathclose%
\pgfusepath{stroke,fill}%
\end{pgfscope}%
\begin{pgfscope}%
\pgfpathrectangle{\pgfqpoint{0.393613in}{0.331635in}}{\pgfqpoint{9.300000in}{7.700000in}}%
\pgfusepath{clip}%
\pgfsetbuttcap%
\pgfsetroundjoin%
\definecolor{currentfill}{rgb}{0.631373,0.788235,0.956863}%
\pgfsetfillcolor{currentfill}%
\pgfsetlinewidth{0.481800pt}%
\definecolor{currentstroke}{rgb}{1.000000,1.000000,1.000000}%
\pgfsetstrokecolor{currentstroke}%
\pgfsetdash{}{0pt}%
\pgfpathmoveto{\pgfqpoint{4.029243in}{1.937457in}}%
\pgfpathcurveto{\pgfqpoint{4.040293in}{1.937457in}}{\pgfqpoint{4.050892in}{1.941847in}}{\pgfqpoint{4.058706in}{1.949661in}}%
\pgfpathcurveto{\pgfqpoint{4.066520in}{1.957475in}}{\pgfqpoint{4.070910in}{1.968074in}}{\pgfqpoint{4.070910in}{1.979124in}}%
\pgfpathcurveto{\pgfqpoint{4.070910in}{1.990174in}}{\pgfqpoint{4.066520in}{2.000773in}}{\pgfqpoint{4.058706in}{2.008587in}}%
\pgfpathcurveto{\pgfqpoint{4.050892in}{2.016400in}}{\pgfqpoint{4.040293in}{2.020790in}}{\pgfqpoint{4.029243in}{2.020790in}}%
\pgfpathcurveto{\pgfqpoint{4.018193in}{2.020790in}}{\pgfqpoint{4.007594in}{2.016400in}}{\pgfqpoint{3.999780in}{2.008587in}}%
\pgfpathcurveto{\pgfqpoint{3.991967in}{2.000773in}}{\pgfqpoint{3.987576in}{1.990174in}}{\pgfqpoint{3.987576in}{1.979124in}}%
\pgfpathcurveto{\pgfqpoint{3.987576in}{1.968074in}}{\pgfqpoint{3.991967in}{1.957475in}}{\pgfqpoint{3.999780in}{1.949661in}}%
\pgfpathcurveto{\pgfqpoint{4.007594in}{1.941847in}}{\pgfqpoint{4.018193in}{1.937457in}}{\pgfqpoint{4.029243in}{1.937457in}}%
\pgfpathclose%
\pgfusepath{stroke,fill}%
\end{pgfscope}%
\begin{pgfscope}%
\pgfpathrectangle{\pgfqpoint{0.393613in}{0.331635in}}{\pgfqpoint{9.300000in}{7.700000in}}%
\pgfusepath{clip}%
\pgfsetbuttcap%
\pgfsetroundjoin%
\definecolor{currentfill}{rgb}{0.631373,0.788235,0.956863}%
\pgfsetfillcolor{currentfill}%
\pgfsetlinewidth{0.481800pt}%
\definecolor{currentstroke}{rgb}{1.000000,1.000000,1.000000}%
\pgfsetstrokecolor{currentstroke}%
\pgfsetdash{}{0pt}%
\pgfpathmoveto{\pgfqpoint{3.912187in}{3.956741in}}%
\pgfpathcurveto{\pgfqpoint{3.923237in}{3.956741in}}{\pgfqpoint{3.933836in}{3.961131in}}{\pgfqpoint{3.941650in}{3.968945in}}%
\pgfpathcurveto{\pgfqpoint{3.949463in}{3.976759in}}{\pgfqpoint{3.953853in}{3.987358in}}{\pgfqpoint{3.953853in}{3.998408in}}%
\pgfpathcurveto{\pgfqpoint{3.953853in}{4.009458in}}{\pgfqpoint{3.949463in}{4.020057in}}{\pgfqpoint{3.941650in}{4.027871in}}%
\pgfpathcurveto{\pgfqpoint{3.933836in}{4.035684in}}{\pgfqpoint{3.923237in}{4.040074in}}{\pgfqpoint{3.912187in}{4.040074in}}%
\pgfpathcurveto{\pgfqpoint{3.901137in}{4.040074in}}{\pgfqpoint{3.890538in}{4.035684in}}{\pgfqpoint{3.882724in}{4.027871in}}%
\pgfpathcurveto{\pgfqpoint{3.874910in}{4.020057in}}{\pgfqpoint{3.870520in}{4.009458in}}{\pgfqpoint{3.870520in}{3.998408in}}%
\pgfpathcurveto{\pgfqpoint{3.870520in}{3.987358in}}{\pgfqpoint{3.874910in}{3.976759in}}{\pgfqpoint{3.882724in}{3.968945in}}%
\pgfpathcurveto{\pgfqpoint{3.890538in}{3.961131in}}{\pgfqpoint{3.901137in}{3.956741in}}{\pgfqpoint{3.912187in}{3.956741in}}%
\pgfpathclose%
\pgfusepath{stroke,fill}%
\end{pgfscope}%
\begin{pgfscope}%
\pgfpathrectangle{\pgfqpoint{0.393613in}{0.331635in}}{\pgfqpoint{9.300000in}{7.700000in}}%
\pgfusepath{clip}%
\pgfsetbuttcap%
\pgfsetroundjoin%
\definecolor{currentfill}{rgb}{0.631373,0.788235,0.956863}%
\pgfsetfillcolor{currentfill}%
\pgfsetlinewidth{0.481800pt}%
\definecolor{currentstroke}{rgb}{1.000000,1.000000,1.000000}%
\pgfsetstrokecolor{currentstroke}%
\pgfsetdash{}{0pt}%
\pgfpathmoveto{\pgfqpoint{3.189809in}{3.244936in}}%
\pgfpathcurveto{\pgfqpoint{3.200859in}{3.244936in}}{\pgfqpoint{3.211458in}{3.249326in}}{\pgfqpoint{3.219271in}{3.257140in}}%
\pgfpathcurveto{\pgfqpoint{3.227085in}{3.264954in}}{\pgfqpoint{3.231475in}{3.275553in}}{\pgfqpoint{3.231475in}{3.286603in}}%
\pgfpathcurveto{\pgfqpoint{3.231475in}{3.297653in}}{\pgfqpoint{3.227085in}{3.308252in}}{\pgfqpoint{3.219271in}{3.316065in}}%
\pgfpathcurveto{\pgfqpoint{3.211458in}{3.323879in}}{\pgfqpoint{3.200859in}{3.328269in}}{\pgfqpoint{3.189809in}{3.328269in}}%
\pgfpathcurveto{\pgfqpoint{3.178758in}{3.328269in}}{\pgfqpoint{3.168159in}{3.323879in}}{\pgfqpoint{3.160346in}{3.316065in}}%
\pgfpathcurveto{\pgfqpoint{3.152532in}{3.308252in}}{\pgfqpoint{3.148142in}{3.297653in}}{\pgfqpoint{3.148142in}{3.286603in}}%
\pgfpathcurveto{\pgfqpoint{3.148142in}{3.275553in}}{\pgfqpoint{3.152532in}{3.264954in}}{\pgfqpoint{3.160346in}{3.257140in}}%
\pgfpathcurveto{\pgfqpoint{3.168159in}{3.249326in}}{\pgfqpoint{3.178758in}{3.244936in}}{\pgfqpoint{3.189809in}{3.244936in}}%
\pgfpathclose%
\pgfusepath{stroke,fill}%
\end{pgfscope}%
\begin{pgfscope}%
\pgfpathrectangle{\pgfqpoint{0.393613in}{0.331635in}}{\pgfqpoint{9.300000in}{7.700000in}}%
\pgfusepath{clip}%
\pgfsetbuttcap%
\pgfsetroundjoin%
\definecolor{currentfill}{rgb}{0.631373,0.788235,0.956863}%
\pgfsetfillcolor{currentfill}%
\pgfsetlinewidth{0.481800pt}%
\definecolor{currentstroke}{rgb}{1.000000,1.000000,1.000000}%
\pgfsetstrokecolor{currentstroke}%
\pgfsetdash{}{0pt}%
\pgfpathmoveto{\pgfqpoint{2.615024in}{4.448849in}}%
\pgfpathcurveto{\pgfqpoint{2.626074in}{4.448849in}}{\pgfqpoint{2.636673in}{4.453239in}}{\pgfqpoint{2.644487in}{4.461053in}}%
\pgfpathcurveto{\pgfqpoint{2.652300in}{4.468867in}}{\pgfqpoint{2.656691in}{4.479466in}}{\pgfqpoint{2.656691in}{4.490516in}}%
\pgfpathcurveto{\pgfqpoint{2.656691in}{4.501566in}}{\pgfqpoint{2.652300in}{4.512165in}}{\pgfqpoint{2.644487in}{4.519979in}}%
\pgfpathcurveto{\pgfqpoint{2.636673in}{4.527792in}}{\pgfqpoint{2.626074in}{4.532182in}}{\pgfqpoint{2.615024in}{4.532182in}}%
\pgfpathcurveto{\pgfqpoint{2.603974in}{4.532182in}}{\pgfqpoint{2.593375in}{4.527792in}}{\pgfqpoint{2.585561in}{4.519979in}}%
\pgfpathcurveto{\pgfqpoint{2.577747in}{4.512165in}}{\pgfqpoint{2.573357in}{4.501566in}}{\pgfqpoint{2.573357in}{4.490516in}}%
\pgfpathcurveto{\pgfqpoint{2.573357in}{4.479466in}}{\pgfqpoint{2.577747in}{4.468867in}}{\pgfqpoint{2.585561in}{4.461053in}}%
\pgfpathcurveto{\pgfqpoint{2.593375in}{4.453239in}}{\pgfqpoint{2.603974in}{4.448849in}}{\pgfqpoint{2.615024in}{4.448849in}}%
\pgfpathclose%
\pgfusepath{stroke,fill}%
\end{pgfscope}%
\begin{pgfscope}%
\pgfpathrectangle{\pgfqpoint{0.393613in}{0.331635in}}{\pgfqpoint{9.300000in}{7.700000in}}%
\pgfusepath{clip}%
\pgfsetbuttcap%
\pgfsetroundjoin%
\definecolor{currentfill}{rgb}{0.631373,0.788235,0.956863}%
\pgfsetfillcolor{currentfill}%
\pgfsetlinewidth{0.481800pt}%
\definecolor{currentstroke}{rgb}{1.000000,1.000000,1.000000}%
\pgfsetstrokecolor{currentstroke}%
\pgfsetdash{}{0pt}%
\pgfpathmoveto{\pgfqpoint{2.589699in}{5.150005in}}%
\pgfpathcurveto{\pgfqpoint{2.600749in}{5.150005in}}{\pgfqpoint{2.611348in}{5.154395in}}{\pgfqpoint{2.619161in}{5.162209in}}%
\pgfpathcurveto{\pgfqpoint{2.626975in}{5.170022in}}{\pgfqpoint{2.631365in}{5.180621in}}{\pgfqpoint{2.631365in}{5.191672in}}%
\pgfpathcurveto{\pgfqpoint{2.631365in}{5.202722in}}{\pgfqpoint{2.626975in}{5.213321in}}{\pgfqpoint{2.619161in}{5.221134in}}%
\pgfpathcurveto{\pgfqpoint{2.611348in}{5.228948in}}{\pgfqpoint{2.600749in}{5.233338in}}{\pgfqpoint{2.589699in}{5.233338in}}%
\pgfpathcurveto{\pgfqpoint{2.578649in}{5.233338in}}{\pgfqpoint{2.568050in}{5.228948in}}{\pgfqpoint{2.560236in}{5.221134in}}%
\pgfpathcurveto{\pgfqpoint{2.552422in}{5.213321in}}{\pgfqpoint{2.548032in}{5.202722in}}{\pgfqpoint{2.548032in}{5.191672in}}%
\pgfpathcurveto{\pgfqpoint{2.548032in}{5.180621in}}{\pgfqpoint{2.552422in}{5.170022in}}{\pgfqpoint{2.560236in}{5.162209in}}%
\pgfpathcurveto{\pgfqpoint{2.568050in}{5.154395in}}{\pgfqpoint{2.578649in}{5.150005in}}{\pgfqpoint{2.589699in}{5.150005in}}%
\pgfpathclose%
\pgfusepath{stroke,fill}%
\end{pgfscope}%
\begin{pgfscope}%
\pgfpathrectangle{\pgfqpoint{0.393613in}{0.331635in}}{\pgfqpoint{9.300000in}{7.700000in}}%
\pgfusepath{clip}%
\pgfsetbuttcap%
\pgfsetroundjoin%
\definecolor{currentfill}{rgb}{0.631373,0.788235,0.956863}%
\pgfsetfillcolor{currentfill}%
\pgfsetlinewidth{0.481800pt}%
\definecolor{currentstroke}{rgb}{1.000000,1.000000,1.000000}%
\pgfsetstrokecolor{currentstroke}%
\pgfsetdash{}{0pt}%
\pgfpathmoveto{\pgfqpoint{1.233235in}{3.927163in}}%
\pgfpathcurveto{\pgfqpoint{1.244285in}{3.927163in}}{\pgfqpoint{1.254884in}{3.931553in}}{\pgfqpoint{1.262698in}{3.939367in}}%
\pgfpathcurveto{\pgfqpoint{1.270512in}{3.947181in}}{\pgfqpoint{1.274902in}{3.957780in}}{\pgfqpoint{1.274902in}{3.968830in}}%
\pgfpathcurveto{\pgfqpoint{1.274902in}{3.979880in}}{\pgfqpoint{1.270512in}{3.990479in}}{\pgfqpoint{1.262698in}{3.998292in}}%
\pgfpathcurveto{\pgfqpoint{1.254884in}{4.006106in}}{\pgfqpoint{1.244285in}{4.010496in}}{\pgfqpoint{1.233235in}{4.010496in}}%
\pgfpathcurveto{\pgfqpoint{1.222185in}{4.010496in}}{\pgfqpoint{1.211586in}{4.006106in}}{\pgfqpoint{1.203772in}{3.998292in}}%
\pgfpathcurveto{\pgfqpoint{1.195959in}{3.990479in}}{\pgfqpoint{1.191569in}{3.979880in}}{\pgfqpoint{1.191569in}{3.968830in}}%
\pgfpathcurveto{\pgfqpoint{1.191569in}{3.957780in}}{\pgfqpoint{1.195959in}{3.947181in}}{\pgfqpoint{1.203772in}{3.939367in}}%
\pgfpathcurveto{\pgfqpoint{1.211586in}{3.931553in}}{\pgfqpoint{1.222185in}{3.927163in}}{\pgfqpoint{1.233235in}{3.927163in}}%
\pgfpathclose%
\pgfusepath{stroke,fill}%
\end{pgfscope}%
\begin{pgfscope}%
\pgfpathrectangle{\pgfqpoint{0.393613in}{0.331635in}}{\pgfqpoint{9.300000in}{7.700000in}}%
\pgfusepath{clip}%
\pgfsetbuttcap%
\pgfsetroundjoin%
\definecolor{currentfill}{rgb}{0.631373,0.788235,0.956863}%
\pgfsetfillcolor{currentfill}%
\pgfsetlinewidth{0.481800pt}%
\definecolor{currentstroke}{rgb}{1.000000,1.000000,1.000000}%
\pgfsetstrokecolor{currentstroke}%
\pgfsetdash{}{0pt}%
\pgfpathmoveto{\pgfqpoint{3.810785in}{1.904817in}}%
\pgfpathcurveto{\pgfqpoint{3.821835in}{1.904817in}}{\pgfqpoint{3.832434in}{1.909207in}}{\pgfqpoint{3.840248in}{1.917021in}}%
\pgfpathcurveto{\pgfqpoint{3.848061in}{1.924835in}}{\pgfqpoint{3.852452in}{1.935434in}}{\pgfqpoint{3.852452in}{1.946484in}}%
\pgfpathcurveto{\pgfqpoint{3.852452in}{1.957534in}}{\pgfqpoint{3.848061in}{1.968133in}}{\pgfqpoint{3.840248in}{1.975946in}}%
\pgfpathcurveto{\pgfqpoint{3.832434in}{1.983760in}}{\pgfqpoint{3.821835in}{1.988150in}}{\pgfqpoint{3.810785in}{1.988150in}}%
\pgfpathcurveto{\pgfqpoint{3.799735in}{1.988150in}}{\pgfqpoint{3.789136in}{1.983760in}}{\pgfqpoint{3.781322in}{1.975946in}}%
\pgfpathcurveto{\pgfqpoint{3.773509in}{1.968133in}}{\pgfqpoint{3.769118in}{1.957534in}}{\pgfqpoint{3.769118in}{1.946484in}}%
\pgfpathcurveto{\pgfqpoint{3.769118in}{1.935434in}}{\pgfqpoint{3.773509in}{1.924835in}}{\pgfqpoint{3.781322in}{1.917021in}}%
\pgfpathcurveto{\pgfqpoint{3.789136in}{1.909207in}}{\pgfqpoint{3.799735in}{1.904817in}}{\pgfqpoint{3.810785in}{1.904817in}}%
\pgfpathclose%
\pgfusepath{stroke,fill}%
\end{pgfscope}%
\begin{pgfscope}%
\pgfpathrectangle{\pgfqpoint{0.393613in}{0.331635in}}{\pgfqpoint{9.300000in}{7.700000in}}%
\pgfusepath{clip}%
\pgfsetbuttcap%
\pgfsetroundjoin%
\definecolor{currentfill}{rgb}{0.631373,0.788235,0.956863}%
\pgfsetfillcolor{currentfill}%
\pgfsetlinewidth{0.481800pt}%
\definecolor{currentstroke}{rgb}{1.000000,1.000000,1.000000}%
\pgfsetstrokecolor{currentstroke}%
\pgfsetdash{}{0pt}%
\pgfpathmoveto{\pgfqpoint{2.870620in}{1.733520in}}%
\pgfpathcurveto{\pgfqpoint{2.881670in}{1.733520in}}{\pgfqpoint{2.892269in}{1.737911in}}{\pgfqpoint{2.900083in}{1.745724in}}%
\pgfpathcurveto{\pgfqpoint{2.907897in}{1.753538in}}{\pgfqpoint{2.912287in}{1.764137in}}{\pgfqpoint{2.912287in}{1.775187in}}%
\pgfpathcurveto{\pgfqpoint{2.912287in}{1.786237in}}{\pgfqpoint{2.907897in}{1.796836in}}{\pgfqpoint{2.900083in}{1.804650in}}%
\pgfpathcurveto{\pgfqpoint{2.892269in}{1.812463in}}{\pgfqpoint{2.881670in}{1.816854in}}{\pgfqpoint{2.870620in}{1.816854in}}%
\pgfpathcurveto{\pgfqpoint{2.859570in}{1.816854in}}{\pgfqpoint{2.848971in}{1.812463in}}{\pgfqpoint{2.841158in}{1.804650in}}%
\pgfpathcurveto{\pgfqpoint{2.833344in}{1.796836in}}{\pgfqpoint{2.828954in}{1.786237in}}{\pgfqpoint{2.828954in}{1.775187in}}%
\pgfpathcurveto{\pgfqpoint{2.828954in}{1.764137in}}{\pgfqpoint{2.833344in}{1.753538in}}{\pgfqpoint{2.841158in}{1.745724in}}%
\pgfpathcurveto{\pgfqpoint{2.848971in}{1.737911in}}{\pgfqpoint{2.859570in}{1.733520in}}{\pgfqpoint{2.870620in}{1.733520in}}%
\pgfpathclose%
\pgfusepath{stroke,fill}%
\end{pgfscope}%
\begin{pgfscope}%
\pgfpathrectangle{\pgfqpoint{0.393613in}{0.331635in}}{\pgfqpoint{9.300000in}{7.700000in}}%
\pgfusepath{clip}%
\pgfsetbuttcap%
\pgfsetroundjoin%
\definecolor{currentfill}{rgb}{0.631373,0.788235,0.956863}%
\pgfsetfillcolor{currentfill}%
\pgfsetlinewidth{0.481800pt}%
\definecolor{currentstroke}{rgb}{1.000000,1.000000,1.000000}%
\pgfsetstrokecolor{currentstroke}%
\pgfsetdash{}{0pt}%
\pgfpathmoveto{\pgfqpoint{2.247275in}{4.426248in}}%
\pgfpathcurveto{\pgfqpoint{2.258325in}{4.426248in}}{\pgfqpoint{2.268924in}{4.430639in}}{\pgfqpoint{2.276738in}{4.438452in}}%
\pgfpathcurveto{\pgfqpoint{2.284552in}{4.446266in}}{\pgfqpoint{2.288942in}{4.456865in}}{\pgfqpoint{2.288942in}{4.467915in}}%
\pgfpathcurveto{\pgfqpoint{2.288942in}{4.478965in}}{\pgfqpoint{2.284552in}{4.489564in}}{\pgfqpoint{2.276738in}{4.497378in}}%
\pgfpathcurveto{\pgfqpoint{2.268924in}{4.505191in}}{\pgfqpoint{2.258325in}{4.509582in}}{\pgfqpoint{2.247275in}{4.509582in}}%
\pgfpathcurveto{\pgfqpoint{2.236225in}{4.509582in}}{\pgfqpoint{2.225626in}{4.505191in}}{\pgfqpoint{2.217812in}{4.497378in}}%
\pgfpathcurveto{\pgfqpoint{2.209999in}{4.489564in}}{\pgfqpoint{2.205609in}{4.478965in}}{\pgfqpoint{2.205609in}{4.467915in}}%
\pgfpathcurveto{\pgfqpoint{2.205609in}{4.456865in}}{\pgfqpoint{2.209999in}{4.446266in}}{\pgfqpoint{2.217812in}{4.438452in}}%
\pgfpathcurveto{\pgfqpoint{2.225626in}{4.430639in}}{\pgfqpoint{2.236225in}{4.426248in}}{\pgfqpoint{2.247275in}{4.426248in}}%
\pgfpathclose%
\pgfusepath{stroke,fill}%
\end{pgfscope}%
\begin{pgfscope}%
\pgfpathrectangle{\pgfqpoint{0.393613in}{0.331635in}}{\pgfqpoint{9.300000in}{7.700000in}}%
\pgfusepath{clip}%
\pgfsetbuttcap%
\pgfsetroundjoin%
\definecolor{currentfill}{rgb}{0.631373,0.788235,0.956863}%
\pgfsetfillcolor{currentfill}%
\pgfsetlinewidth{0.481800pt}%
\definecolor{currentstroke}{rgb}{1.000000,1.000000,1.000000}%
\pgfsetstrokecolor{currentstroke}%
\pgfsetdash{}{0pt}%
\pgfpathmoveto{\pgfqpoint{3.278707in}{2.466573in}}%
\pgfpathcurveto{\pgfqpoint{3.289757in}{2.466573in}}{\pgfqpoint{3.300356in}{2.470964in}}{\pgfqpoint{3.308169in}{2.478777in}}%
\pgfpathcurveto{\pgfqpoint{3.315983in}{2.486591in}}{\pgfqpoint{3.320373in}{2.497190in}}{\pgfqpoint{3.320373in}{2.508240in}}%
\pgfpathcurveto{\pgfqpoint{3.320373in}{2.519290in}}{\pgfqpoint{3.315983in}{2.529889in}}{\pgfqpoint{3.308169in}{2.537703in}}%
\pgfpathcurveto{\pgfqpoint{3.300356in}{2.545516in}}{\pgfqpoint{3.289757in}{2.549907in}}{\pgfqpoint{3.278707in}{2.549907in}}%
\pgfpathcurveto{\pgfqpoint{3.267656in}{2.549907in}}{\pgfqpoint{3.257057in}{2.545516in}}{\pgfqpoint{3.249244in}{2.537703in}}%
\pgfpathcurveto{\pgfqpoint{3.241430in}{2.529889in}}{\pgfqpoint{3.237040in}{2.519290in}}{\pgfqpoint{3.237040in}{2.508240in}}%
\pgfpathcurveto{\pgfqpoint{3.237040in}{2.497190in}}{\pgfqpoint{3.241430in}{2.486591in}}{\pgfqpoint{3.249244in}{2.478777in}}%
\pgfpathcurveto{\pgfqpoint{3.257057in}{2.470964in}}{\pgfqpoint{3.267656in}{2.466573in}}{\pgfqpoint{3.278707in}{2.466573in}}%
\pgfpathclose%
\pgfusepath{stroke,fill}%
\end{pgfscope}%
\begin{pgfscope}%
\pgfpathrectangle{\pgfqpoint{0.393613in}{0.331635in}}{\pgfqpoint{9.300000in}{7.700000in}}%
\pgfusepath{clip}%
\pgfsetbuttcap%
\pgfsetroundjoin%
\definecolor{currentfill}{rgb}{0.631373,0.788235,0.956863}%
\pgfsetfillcolor{currentfill}%
\pgfsetlinewidth{0.481800pt}%
\definecolor{currentstroke}{rgb}{1.000000,1.000000,1.000000}%
\pgfsetstrokecolor{currentstroke}%
\pgfsetdash{}{0pt}%
\pgfpathmoveto{\pgfqpoint{4.507490in}{0.639968in}}%
\pgfpathcurveto{\pgfqpoint{4.518540in}{0.639968in}}{\pgfqpoint{4.529139in}{0.644359in}}{\pgfqpoint{4.536953in}{0.652172in}}%
\pgfpathcurveto{\pgfqpoint{4.544766in}{0.659986in}}{\pgfqpoint{4.549157in}{0.670585in}}{\pgfqpoint{4.549157in}{0.681635in}}%
\pgfpathcurveto{\pgfqpoint{4.549157in}{0.692685in}}{\pgfqpoint{4.544766in}{0.703284in}}{\pgfqpoint{4.536953in}{0.711098in}}%
\pgfpathcurveto{\pgfqpoint{4.529139in}{0.718911in}}{\pgfqpoint{4.518540in}{0.723302in}}{\pgfqpoint{4.507490in}{0.723302in}}%
\pgfpathcurveto{\pgfqpoint{4.496440in}{0.723302in}}{\pgfqpoint{4.485841in}{0.718911in}}{\pgfqpoint{4.478027in}{0.711098in}}%
\pgfpathcurveto{\pgfqpoint{4.470214in}{0.703284in}}{\pgfqpoint{4.465823in}{0.692685in}}{\pgfqpoint{4.465823in}{0.681635in}}%
\pgfpathcurveto{\pgfqpoint{4.465823in}{0.670585in}}{\pgfqpoint{4.470214in}{0.659986in}}{\pgfqpoint{4.478027in}{0.652172in}}%
\pgfpathcurveto{\pgfqpoint{4.485841in}{0.644359in}}{\pgfqpoint{4.496440in}{0.639968in}}{\pgfqpoint{4.507490in}{0.639968in}}%
\pgfpathclose%
\pgfusepath{stroke,fill}%
\end{pgfscope}%
\begin{pgfscope}%
\pgfpathrectangle{\pgfqpoint{0.393613in}{0.331635in}}{\pgfqpoint{9.300000in}{7.700000in}}%
\pgfusepath{clip}%
\pgfsetbuttcap%
\pgfsetroundjoin%
\definecolor{currentfill}{rgb}{0.631373,0.788235,0.956863}%
\pgfsetfillcolor{currentfill}%
\pgfsetlinewidth{0.481800pt}%
\definecolor{currentstroke}{rgb}{1.000000,1.000000,1.000000}%
\pgfsetstrokecolor{currentstroke}%
\pgfsetdash{}{0pt}%
\pgfpathmoveto{\pgfqpoint{2.850278in}{7.132769in}}%
\pgfpathcurveto{\pgfqpoint{2.861328in}{7.132769in}}{\pgfqpoint{2.871927in}{7.137159in}}{\pgfqpoint{2.879740in}{7.144973in}}%
\pgfpathcurveto{\pgfqpoint{2.887554in}{7.152787in}}{\pgfqpoint{2.891944in}{7.163386in}}{\pgfqpoint{2.891944in}{7.174436in}}%
\pgfpathcurveto{\pgfqpoint{2.891944in}{7.185486in}}{\pgfqpoint{2.887554in}{7.196085in}}{\pgfqpoint{2.879740in}{7.203899in}}%
\pgfpathcurveto{\pgfqpoint{2.871927in}{7.211712in}}{\pgfqpoint{2.861328in}{7.216103in}}{\pgfqpoint{2.850278in}{7.216103in}}%
\pgfpathcurveto{\pgfqpoint{2.839227in}{7.216103in}}{\pgfqpoint{2.828628in}{7.211712in}}{\pgfqpoint{2.820815in}{7.203899in}}%
\pgfpathcurveto{\pgfqpoint{2.813001in}{7.196085in}}{\pgfqpoint{2.808611in}{7.185486in}}{\pgfqpoint{2.808611in}{7.174436in}}%
\pgfpathcurveto{\pgfqpoint{2.808611in}{7.163386in}}{\pgfqpoint{2.813001in}{7.152787in}}{\pgfqpoint{2.820815in}{7.144973in}}%
\pgfpathcurveto{\pgfqpoint{2.828628in}{7.137159in}}{\pgfqpoint{2.839227in}{7.132769in}}{\pgfqpoint{2.850278in}{7.132769in}}%
\pgfpathclose%
\pgfusepath{stroke,fill}%
\end{pgfscope}%
\begin{pgfscope}%
\pgfpathrectangle{\pgfqpoint{0.393613in}{0.331635in}}{\pgfqpoint{9.300000in}{7.700000in}}%
\pgfusepath{clip}%
\pgfsetbuttcap%
\pgfsetroundjoin%
\definecolor{currentfill}{rgb}{0.631373,0.788235,0.956863}%
\pgfsetfillcolor{currentfill}%
\pgfsetlinewidth{0.481800pt}%
\definecolor{currentstroke}{rgb}{1.000000,1.000000,1.000000}%
\pgfsetstrokecolor{currentstroke}%
\pgfsetdash{}{0pt}%
\pgfpathmoveto{\pgfqpoint{1.938431in}{4.996291in}}%
\pgfpathcurveto{\pgfqpoint{1.949481in}{4.996291in}}{\pgfqpoint{1.960080in}{5.000682in}}{\pgfqpoint{1.967894in}{5.008495in}}%
\pgfpathcurveto{\pgfqpoint{1.975708in}{5.016309in}}{\pgfqpoint{1.980098in}{5.026908in}}{\pgfqpoint{1.980098in}{5.037958in}}%
\pgfpathcurveto{\pgfqpoint{1.980098in}{5.049008in}}{\pgfqpoint{1.975708in}{5.059607in}}{\pgfqpoint{1.967894in}{5.067421in}}%
\pgfpathcurveto{\pgfqpoint{1.960080in}{5.075234in}}{\pgfqpoint{1.949481in}{5.079625in}}{\pgfqpoint{1.938431in}{5.079625in}}%
\pgfpathcurveto{\pgfqpoint{1.927381in}{5.079625in}}{\pgfqpoint{1.916782in}{5.075234in}}{\pgfqpoint{1.908968in}{5.067421in}}%
\pgfpathcurveto{\pgfqpoint{1.901155in}{5.059607in}}{\pgfqpoint{1.896764in}{5.049008in}}{\pgfqpoint{1.896764in}{5.037958in}}%
\pgfpathcurveto{\pgfqpoint{1.896764in}{5.026908in}}{\pgfqpoint{1.901155in}{5.016309in}}{\pgfqpoint{1.908968in}{5.008495in}}%
\pgfpathcurveto{\pgfqpoint{1.916782in}{5.000682in}}{\pgfqpoint{1.927381in}{4.996291in}}{\pgfqpoint{1.938431in}{4.996291in}}%
\pgfpathclose%
\pgfusepath{stroke,fill}%
\end{pgfscope}%
\begin{pgfscope}%
\pgfpathrectangle{\pgfqpoint{0.393613in}{0.331635in}}{\pgfqpoint{9.300000in}{7.700000in}}%
\pgfusepath{clip}%
\pgfsetbuttcap%
\pgfsetroundjoin%
\definecolor{currentfill}{rgb}{0.631373,0.788235,0.956863}%
\pgfsetfillcolor{currentfill}%
\pgfsetlinewidth{0.481800pt}%
\definecolor{currentstroke}{rgb}{1.000000,1.000000,1.000000}%
\pgfsetstrokecolor{currentstroke}%
\pgfsetdash{}{0pt}%
\pgfpathmoveto{\pgfqpoint{2.919129in}{0.889402in}}%
\pgfpathcurveto{\pgfqpoint{2.930179in}{0.889402in}}{\pgfqpoint{2.940778in}{0.893792in}}{\pgfqpoint{2.948592in}{0.901606in}}%
\pgfpathcurveto{\pgfqpoint{2.956406in}{0.909420in}}{\pgfqpoint{2.960796in}{0.920019in}}{\pgfqpoint{2.960796in}{0.931069in}}%
\pgfpathcurveto{\pgfqpoint{2.960796in}{0.942119in}}{\pgfqpoint{2.956406in}{0.952718in}}{\pgfqpoint{2.948592in}{0.960532in}}%
\pgfpathcurveto{\pgfqpoint{2.940778in}{0.968345in}}{\pgfqpoint{2.930179in}{0.972735in}}{\pgfqpoint{2.919129in}{0.972735in}}%
\pgfpathcurveto{\pgfqpoint{2.908079in}{0.972735in}}{\pgfqpoint{2.897480in}{0.968345in}}{\pgfqpoint{2.889666in}{0.960532in}}%
\pgfpathcurveto{\pgfqpoint{2.881853in}{0.952718in}}{\pgfqpoint{2.877462in}{0.942119in}}{\pgfqpoint{2.877462in}{0.931069in}}%
\pgfpathcurveto{\pgfqpoint{2.877462in}{0.920019in}}{\pgfqpoint{2.881853in}{0.909420in}}{\pgfqpoint{2.889666in}{0.901606in}}%
\pgfpathcurveto{\pgfqpoint{2.897480in}{0.893792in}}{\pgfqpoint{2.908079in}{0.889402in}}{\pgfqpoint{2.919129in}{0.889402in}}%
\pgfpathclose%
\pgfusepath{stroke,fill}%
\end{pgfscope}%
\begin{pgfscope}%
\pgfpathrectangle{\pgfqpoint{0.393613in}{0.331635in}}{\pgfqpoint{9.300000in}{7.700000in}}%
\pgfusepath{clip}%
\pgfsetbuttcap%
\pgfsetroundjoin%
\definecolor{currentfill}{rgb}{0.631373,0.788235,0.956863}%
\pgfsetfillcolor{currentfill}%
\pgfsetlinewidth{0.481800pt}%
\definecolor{currentstroke}{rgb}{1.000000,1.000000,1.000000}%
\pgfsetstrokecolor{currentstroke}%
\pgfsetdash{}{0pt}%
\pgfpathmoveto{\pgfqpoint{2.553778in}{6.288010in}}%
\pgfpathcurveto{\pgfqpoint{2.564829in}{6.288010in}}{\pgfqpoint{2.575428in}{6.292401in}}{\pgfqpoint{2.583241in}{6.300214in}}%
\pgfpathcurveto{\pgfqpoint{2.591055in}{6.308028in}}{\pgfqpoint{2.595445in}{6.318627in}}{\pgfqpoint{2.595445in}{6.329677in}}%
\pgfpathcurveto{\pgfqpoint{2.595445in}{6.340727in}}{\pgfqpoint{2.591055in}{6.351326in}}{\pgfqpoint{2.583241in}{6.359140in}}%
\pgfpathcurveto{\pgfqpoint{2.575428in}{6.366953in}}{\pgfqpoint{2.564829in}{6.371344in}}{\pgfqpoint{2.553778in}{6.371344in}}%
\pgfpathcurveto{\pgfqpoint{2.542728in}{6.371344in}}{\pgfqpoint{2.532129in}{6.366953in}}{\pgfqpoint{2.524316in}{6.359140in}}%
\pgfpathcurveto{\pgfqpoint{2.516502in}{6.351326in}}{\pgfqpoint{2.512112in}{6.340727in}}{\pgfqpoint{2.512112in}{6.329677in}}%
\pgfpathcurveto{\pgfqpoint{2.512112in}{6.318627in}}{\pgfqpoint{2.516502in}{6.308028in}}{\pgfqpoint{2.524316in}{6.300214in}}%
\pgfpathcurveto{\pgfqpoint{2.532129in}{6.292401in}}{\pgfqpoint{2.542728in}{6.288010in}}{\pgfqpoint{2.553778in}{6.288010in}}%
\pgfpathclose%
\pgfusepath{stroke,fill}%
\end{pgfscope}%
\begin{pgfscope}%
\pgfpathrectangle{\pgfqpoint{0.393613in}{0.331635in}}{\pgfqpoint{9.300000in}{7.700000in}}%
\pgfusepath{clip}%
\pgfsetbuttcap%
\pgfsetroundjoin%
\definecolor{currentfill}{rgb}{0.631373,0.788235,0.956863}%
\pgfsetfillcolor{currentfill}%
\pgfsetlinewidth{0.481800pt}%
\definecolor{currentstroke}{rgb}{1.000000,1.000000,1.000000}%
\pgfsetstrokecolor{currentstroke}%
\pgfsetdash{}{0pt}%
\pgfpathmoveto{\pgfqpoint{2.490513in}{3.733398in}}%
\pgfpathcurveto{\pgfqpoint{2.501563in}{3.733398in}}{\pgfqpoint{2.512162in}{3.737788in}}{\pgfqpoint{2.519975in}{3.745602in}}%
\pgfpathcurveto{\pgfqpoint{2.527789in}{3.753416in}}{\pgfqpoint{2.532179in}{3.764015in}}{\pgfqpoint{2.532179in}{3.775065in}}%
\pgfpathcurveto{\pgfqpoint{2.532179in}{3.786115in}}{\pgfqpoint{2.527789in}{3.796714in}}{\pgfqpoint{2.519975in}{3.804528in}}%
\pgfpathcurveto{\pgfqpoint{2.512162in}{3.812341in}}{\pgfqpoint{2.501563in}{3.816731in}}{\pgfqpoint{2.490513in}{3.816731in}}%
\pgfpathcurveto{\pgfqpoint{2.479463in}{3.816731in}}{\pgfqpoint{2.468863in}{3.812341in}}{\pgfqpoint{2.461050in}{3.804528in}}%
\pgfpathcurveto{\pgfqpoint{2.453236in}{3.796714in}}{\pgfqpoint{2.448846in}{3.786115in}}{\pgfqpoint{2.448846in}{3.775065in}}%
\pgfpathcurveto{\pgfqpoint{2.448846in}{3.764015in}}{\pgfqpoint{2.453236in}{3.753416in}}{\pgfqpoint{2.461050in}{3.745602in}}%
\pgfpathcurveto{\pgfqpoint{2.468863in}{3.737788in}}{\pgfqpoint{2.479463in}{3.733398in}}{\pgfqpoint{2.490513in}{3.733398in}}%
\pgfpathclose%
\pgfusepath{stroke,fill}%
\end{pgfscope}%
\begin{pgfscope}%
\pgfpathrectangle{\pgfqpoint{0.393613in}{0.331635in}}{\pgfqpoint{9.300000in}{7.700000in}}%
\pgfusepath{clip}%
\pgfsetbuttcap%
\pgfsetroundjoin%
\definecolor{currentfill}{rgb}{0.631373,0.788235,0.956863}%
\pgfsetfillcolor{currentfill}%
\pgfsetlinewidth{0.481800pt}%
\definecolor{currentstroke}{rgb}{1.000000,1.000000,1.000000}%
\pgfsetstrokecolor{currentstroke}%
\pgfsetdash{}{0pt}%
\pgfpathmoveto{\pgfqpoint{0.816340in}{4.715929in}}%
\pgfpathcurveto{\pgfqpoint{0.827390in}{4.715929in}}{\pgfqpoint{0.837989in}{4.720319in}}{\pgfqpoint{0.845803in}{4.728133in}}%
\pgfpathcurveto{\pgfqpoint{0.853616in}{4.735946in}}{\pgfqpoint{0.858006in}{4.746545in}}{\pgfqpoint{0.858006in}{4.757595in}}%
\pgfpathcurveto{\pgfqpoint{0.858006in}{4.768646in}}{\pgfqpoint{0.853616in}{4.779245in}}{\pgfqpoint{0.845803in}{4.787058in}}%
\pgfpathcurveto{\pgfqpoint{0.837989in}{4.794872in}}{\pgfqpoint{0.827390in}{4.799262in}}{\pgfqpoint{0.816340in}{4.799262in}}%
\pgfpathcurveto{\pgfqpoint{0.805290in}{4.799262in}}{\pgfqpoint{0.794691in}{4.794872in}}{\pgfqpoint{0.786877in}{4.787058in}}%
\pgfpathcurveto{\pgfqpoint{0.779063in}{4.779245in}}{\pgfqpoint{0.774673in}{4.768646in}}{\pgfqpoint{0.774673in}{4.757595in}}%
\pgfpathcurveto{\pgfqpoint{0.774673in}{4.746545in}}{\pgfqpoint{0.779063in}{4.735946in}}{\pgfqpoint{0.786877in}{4.728133in}}%
\pgfpathcurveto{\pgfqpoint{0.794691in}{4.720319in}}{\pgfqpoint{0.805290in}{4.715929in}}{\pgfqpoint{0.816340in}{4.715929in}}%
\pgfpathclose%
\pgfusepath{stroke,fill}%
\end{pgfscope}%
\begin{pgfscope}%
\pgfpathrectangle{\pgfqpoint{0.393613in}{0.331635in}}{\pgfqpoint{9.300000in}{7.700000in}}%
\pgfusepath{clip}%
\pgfsetbuttcap%
\pgfsetroundjoin%
\definecolor{currentfill}{rgb}{0.631373,0.788235,0.956863}%
\pgfsetfillcolor{currentfill}%
\pgfsetlinewidth{0.481800pt}%
\definecolor{currentstroke}{rgb}{1.000000,1.000000,1.000000}%
\pgfsetstrokecolor{currentstroke}%
\pgfsetdash{}{0pt}%
\pgfpathmoveto{\pgfqpoint{1.972243in}{3.973088in}}%
\pgfpathcurveto{\pgfqpoint{1.983293in}{3.973088in}}{\pgfqpoint{1.993892in}{3.977478in}}{\pgfqpoint{2.001706in}{3.985292in}}%
\pgfpathcurveto{\pgfqpoint{2.009519in}{3.993105in}}{\pgfqpoint{2.013910in}{4.003704in}}{\pgfqpoint{2.013910in}{4.014754in}}%
\pgfpathcurveto{\pgfqpoint{2.013910in}{4.025805in}}{\pgfqpoint{2.009519in}{4.036404in}}{\pgfqpoint{2.001706in}{4.044217in}}%
\pgfpathcurveto{\pgfqpoint{1.993892in}{4.052031in}}{\pgfqpoint{1.983293in}{4.056421in}}{\pgfqpoint{1.972243in}{4.056421in}}%
\pgfpathcurveto{\pgfqpoint{1.961193in}{4.056421in}}{\pgfqpoint{1.950594in}{4.052031in}}{\pgfqpoint{1.942780in}{4.044217in}}%
\pgfpathcurveto{\pgfqpoint{1.934967in}{4.036404in}}{\pgfqpoint{1.930576in}{4.025805in}}{\pgfqpoint{1.930576in}{4.014754in}}%
\pgfpathcurveto{\pgfqpoint{1.930576in}{4.003704in}}{\pgfqpoint{1.934967in}{3.993105in}}{\pgfqpoint{1.942780in}{3.985292in}}%
\pgfpathcurveto{\pgfqpoint{1.950594in}{3.977478in}}{\pgfqpoint{1.961193in}{3.973088in}}{\pgfqpoint{1.972243in}{3.973088in}}%
\pgfpathclose%
\pgfusepath{stroke,fill}%
\end{pgfscope}%
\begin{pgfscope}%
\pgfpathrectangle{\pgfqpoint{0.393613in}{0.331635in}}{\pgfqpoint{9.300000in}{7.700000in}}%
\pgfusepath{clip}%
\pgfsetbuttcap%
\pgfsetroundjoin%
\definecolor{currentfill}{rgb}{0.631373,0.788235,0.956863}%
\pgfsetfillcolor{currentfill}%
\pgfsetlinewidth{0.481800pt}%
\definecolor{currentstroke}{rgb}{1.000000,1.000000,1.000000}%
\pgfsetstrokecolor{currentstroke}%
\pgfsetdash{}{0pt}%
\pgfpathmoveto{\pgfqpoint{3.271288in}{1.418894in}}%
\pgfpathcurveto{\pgfqpoint{3.282338in}{1.418894in}}{\pgfqpoint{3.292937in}{1.423284in}}{\pgfqpoint{3.300751in}{1.431098in}}%
\pgfpathcurveto{\pgfqpoint{3.308564in}{1.438911in}}{\pgfqpoint{3.312955in}{1.449510in}}{\pgfqpoint{3.312955in}{1.460560in}}%
\pgfpathcurveto{\pgfqpoint{3.312955in}{1.471611in}}{\pgfqpoint{3.308564in}{1.482210in}}{\pgfqpoint{3.300751in}{1.490023in}}%
\pgfpathcurveto{\pgfqpoint{3.292937in}{1.497837in}}{\pgfqpoint{3.282338in}{1.502227in}}{\pgfqpoint{3.271288in}{1.502227in}}%
\pgfpathcurveto{\pgfqpoint{3.260238in}{1.502227in}}{\pgfqpoint{3.249639in}{1.497837in}}{\pgfqpoint{3.241825in}{1.490023in}}%
\pgfpathcurveto{\pgfqpoint{3.234011in}{1.482210in}}{\pgfqpoint{3.229621in}{1.471611in}}{\pgfqpoint{3.229621in}{1.460560in}}%
\pgfpathcurveto{\pgfqpoint{3.229621in}{1.449510in}}{\pgfqpoint{3.234011in}{1.438911in}}{\pgfqpoint{3.241825in}{1.431098in}}%
\pgfpathcurveto{\pgfqpoint{3.249639in}{1.423284in}}{\pgfqpoint{3.260238in}{1.418894in}}{\pgfqpoint{3.271288in}{1.418894in}}%
\pgfpathclose%
\pgfusepath{stroke,fill}%
\end{pgfscope}%
\begin{pgfscope}%
\pgfpathrectangle{\pgfqpoint{0.393613in}{0.331635in}}{\pgfqpoint{9.300000in}{7.700000in}}%
\pgfusepath{clip}%
\pgfsetbuttcap%
\pgfsetroundjoin%
\definecolor{currentfill}{rgb}{0.631373,0.788235,0.956863}%
\pgfsetfillcolor{currentfill}%
\pgfsetlinewidth{0.481800pt}%
\definecolor{currentstroke}{rgb}{1.000000,1.000000,1.000000}%
\pgfsetstrokecolor{currentstroke}%
\pgfsetdash{}{0pt}%
\pgfpathmoveto{\pgfqpoint{2.443533in}{5.210649in}}%
\pgfpathcurveto{\pgfqpoint{2.454583in}{5.210649in}}{\pgfqpoint{2.465182in}{5.215039in}}{\pgfqpoint{2.472996in}{5.222853in}}%
\pgfpathcurveto{\pgfqpoint{2.480809in}{5.230666in}}{\pgfqpoint{2.485200in}{5.241265in}}{\pgfqpoint{2.485200in}{5.252316in}}%
\pgfpathcurveto{\pgfqpoint{2.485200in}{5.263366in}}{\pgfqpoint{2.480809in}{5.273965in}}{\pgfqpoint{2.472996in}{5.281778in}}%
\pgfpathcurveto{\pgfqpoint{2.465182in}{5.289592in}}{\pgfqpoint{2.454583in}{5.293982in}}{\pgfqpoint{2.443533in}{5.293982in}}%
\pgfpathcurveto{\pgfqpoint{2.432483in}{5.293982in}}{\pgfqpoint{2.421884in}{5.289592in}}{\pgfqpoint{2.414070in}{5.281778in}}%
\pgfpathcurveto{\pgfqpoint{2.406257in}{5.273965in}}{\pgfqpoint{2.401866in}{5.263366in}}{\pgfqpoint{2.401866in}{5.252316in}}%
\pgfpathcurveto{\pgfqpoint{2.401866in}{5.241265in}}{\pgfqpoint{2.406257in}{5.230666in}}{\pgfqpoint{2.414070in}{5.222853in}}%
\pgfpathcurveto{\pgfqpoint{2.421884in}{5.215039in}}{\pgfqpoint{2.432483in}{5.210649in}}{\pgfqpoint{2.443533in}{5.210649in}}%
\pgfpathclose%
\pgfusepath{stroke,fill}%
\end{pgfscope}%
\begin{pgfscope}%
\pgfpathrectangle{\pgfqpoint{0.393613in}{0.331635in}}{\pgfqpoint{9.300000in}{7.700000in}}%
\pgfusepath{clip}%
\pgfsetbuttcap%
\pgfsetroundjoin%
\definecolor{currentfill}{rgb}{0.631373,0.788235,0.956863}%
\pgfsetfillcolor{currentfill}%
\pgfsetlinewidth{0.481800pt}%
\definecolor{currentstroke}{rgb}{1.000000,1.000000,1.000000}%
\pgfsetstrokecolor{currentstroke}%
\pgfsetdash{}{0pt}%
\pgfpathmoveto{\pgfqpoint{3.808024in}{7.639968in}}%
\pgfpathcurveto{\pgfqpoint{3.819074in}{7.639968in}}{\pgfqpoint{3.829673in}{7.644359in}}{\pgfqpoint{3.837487in}{7.652172in}}%
\pgfpathcurveto{\pgfqpoint{3.845300in}{7.659986in}}{\pgfqpoint{3.849690in}{7.670585in}}{\pgfqpoint{3.849690in}{7.681635in}}%
\pgfpathcurveto{\pgfqpoint{3.849690in}{7.692685in}}{\pgfqpoint{3.845300in}{7.703284in}}{\pgfqpoint{3.837487in}{7.711098in}}%
\pgfpathcurveto{\pgfqpoint{3.829673in}{7.718911in}}{\pgfqpoint{3.819074in}{7.723302in}}{\pgfqpoint{3.808024in}{7.723302in}}%
\pgfpathcurveto{\pgfqpoint{3.796974in}{7.723302in}}{\pgfqpoint{3.786375in}{7.718911in}}{\pgfqpoint{3.778561in}{7.711098in}}%
\pgfpathcurveto{\pgfqpoint{3.770747in}{7.703284in}}{\pgfqpoint{3.766357in}{7.692685in}}{\pgfqpoint{3.766357in}{7.681635in}}%
\pgfpathcurveto{\pgfqpoint{3.766357in}{7.670585in}}{\pgfqpoint{3.770747in}{7.659986in}}{\pgfqpoint{3.778561in}{7.652172in}}%
\pgfpathcurveto{\pgfqpoint{3.786375in}{7.644359in}}{\pgfqpoint{3.796974in}{7.639968in}}{\pgfqpoint{3.808024in}{7.639968in}}%
\pgfpathclose%
\pgfusepath{stroke,fill}%
\end{pgfscope}%
\begin{pgfscope}%
\pgfpathrectangle{\pgfqpoint{0.393613in}{0.331635in}}{\pgfqpoint{9.300000in}{7.700000in}}%
\pgfusepath{clip}%
\pgfsetbuttcap%
\pgfsetroundjoin%
\definecolor{currentfill}{rgb}{0.631373,0.788235,0.956863}%
\pgfsetfillcolor{currentfill}%
\pgfsetlinewidth{0.481800pt}%
\definecolor{currentstroke}{rgb}{1.000000,1.000000,1.000000}%
\pgfsetstrokecolor{currentstroke}%
\pgfsetdash{}{0pt}%
\pgfpathmoveto{\pgfqpoint{1.545120in}{6.799076in}}%
\pgfpathcurveto{\pgfqpoint{1.556170in}{6.799076in}}{\pgfqpoint{1.566769in}{6.803467in}}{\pgfqpoint{1.574582in}{6.811280in}}%
\pgfpathcurveto{\pgfqpoint{1.582396in}{6.819094in}}{\pgfqpoint{1.586786in}{6.829693in}}{\pgfqpoint{1.586786in}{6.840743in}}%
\pgfpathcurveto{\pgfqpoint{1.586786in}{6.851793in}}{\pgfqpoint{1.582396in}{6.862392in}}{\pgfqpoint{1.574582in}{6.870206in}}%
\pgfpathcurveto{\pgfqpoint{1.566769in}{6.878020in}}{\pgfqpoint{1.556170in}{6.882410in}}{\pgfqpoint{1.545120in}{6.882410in}}%
\pgfpathcurveto{\pgfqpoint{1.534069in}{6.882410in}}{\pgfqpoint{1.523470in}{6.878020in}}{\pgfqpoint{1.515657in}{6.870206in}}%
\pgfpathcurveto{\pgfqpoint{1.507843in}{6.862392in}}{\pgfqpoint{1.503453in}{6.851793in}}{\pgfqpoint{1.503453in}{6.840743in}}%
\pgfpathcurveto{\pgfqpoint{1.503453in}{6.829693in}}{\pgfqpoint{1.507843in}{6.819094in}}{\pgfqpoint{1.515657in}{6.811280in}}%
\pgfpathcurveto{\pgfqpoint{1.523470in}{6.803467in}}{\pgfqpoint{1.534069in}{6.799076in}}{\pgfqpoint{1.545120in}{6.799076in}}%
\pgfpathclose%
\pgfusepath{stroke,fill}%
\end{pgfscope}%
\begin{pgfscope}%
\pgfpathrectangle{\pgfqpoint{0.393613in}{0.331635in}}{\pgfqpoint{9.300000in}{7.700000in}}%
\pgfusepath{clip}%
\pgfsetbuttcap%
\pgfsetroundjoin%
\definecolor{currentfill}{rgb}{0.631373,0.788235,0.956863}%
\pgfsetfillcolor{currentfill}%
\pgfsetlinewidth{0.481800pt}%
\definecolor{currentstroke}{rgb}{1.000000,1.000000,1.000000}%
\pgfsetstrokecolor{currentstroke}%
\pgfsetdash{}{0pt}%
\pgfpathmoveto{\pgfqpoint{4.495356in}{6.292143in}}%
\pgfpathcurveto{\pgfqpoint{4.506406in}{6.292143in}}{\pgfqpoint{4.517005in}{6.296534in}}{\pgfqpoint{4.524819in}{6.304347in}}%
\pgfpathcurveto{\pgfqpoint{4.532632in}{6.312161in}}{\pgfqpoint{4.537023in}{6.322760in}}{\pgfqpoint{4.537023in}{6.333810in}}%
\pgfpathcurveto{\pgfqpoint{4.537023in}{6.344860in}}{\pgfqpoint{4.532632in}{6.355459in}}{\pgfqpoint{4.524819in}{6.363273in}}%
\pgfpathcurveto{\pgfqpoint{4.517005in}{6.371087in}}{\pgfqpoint{4.506406in}{6.375477in}}{\pgfqpoint{4.495356in}{6.375477in}}%
\pgfpathcurveto{\pgfqpoint{4.484306in}{6.375477in}}{\pgfqpoint{4.473707in}{6.371087in}}{\pgfqpoint{4.465893in}{6.363273in}}%
\pgfpathcurveto{\pgfqpoint{4.458080in}{6.355459in}}{\pgfqpoint{4.453689in}{6.344860in}}{\pgfqpoint{4.453689in}{6.333810in}}%
\pgfpathcurveto{\pgfqpoint{4.453689in}{6.322760in}}{\pgfqpoint{4.458080in}{6.312161in}}{\pgfqpoint{4.465893in}{6.304347in}}%
\pgfpathcurveto{\pgfqpoint{4.473707in}{6.296534in}}{\pgfqpoint{4.484306in}{6.292143in}}{\pgfqpoint{4.495356in}{6.292143in}}%
\pgfpathclose%
\pgfusepath{stroke,fill}%
\end{pgfscope}%
\begin{pgfscope}%
\pgfpathrectangle{\pgfqpoint{0.393613in}{0.331635in}}{\pgfqpoint{9.300000in}{7.700000in}}%
\pgfusepath{clip}%
\pgfsetbuttcap%
\pgfsetroundjoin%
\definecolor{currentfill}{rgb}{0.631373,0.788235,0.956863}%
\pgfsetfillcolor{currentfill}%
\pgfsetlinewidth{0.481800pt}%
\definecolor{currentstroke}{rgb}{1.000000,1.000000,1.000000}%
\pgfsetstrokecolor{currentstroke}%
\pgfsetdash{}{0pt}%
\pgfpathmoveto{\pgfqpoint{2.111783in}{3.157576in}}%
\pgfpathcurveto{\pgfqpoint{2.122833in}{3.157576in}}{\pgfqpoint{2.133432in}{3.161966in}}{\pgfqpoint{2.141245in}{3.169780in}}%
\pgfpathcurveto{\pgfqpoint{2.149059in}{3.177593in}}{\pgfqpoint{2.153449in}{3.188192in}}{\pgfqpoint{2.153449in}{3.199242in}}%
\pgfpathcurveto{\pgfqpoint{2.153449in}{3.210292in}}{\pgfqpoint{2.149059in}{3.220891in}}{\pgfqpoint{2.141245in}{3.228705in}}%
\pgfpathcurveto{\pgfqpoint{2.133432in}{3.236519in}}{\pgfqpoint{2.122833in}{3.240909in}}{\pgfqpoint{2.111783in}{3.240909in}}%
\pgfpathcurveto{\pgfqpoint{2.100732in}{3.240909in}}{\pgfqpoint{2.090133in}{3.236519in}}{\pgfqpoint{2.082320in}{3.228705in}}%
\pgfpathcurveto{\pgfqpoint{2.074506in}{3.220891in}}{\pgfqpoint{2.070116in}{3.210292in}}{\pgfqpoint{2.070116in}{3.199242in}}%
\pgfpathcurveto{\pgfqpoint{2.070116in}{3.188192in}}{\pgfqpoint{2.074506in}{3.177593in}}{\pgfqpoint{2.082320in}{3.169780in}}%
\pgfpathcurveto{\pgfqpoint{2.090133in}{3.161966in}}{\pgfqpoint{2.100732in}{3.157576in}}{\pgfqpoint{2.111783in}{3.157576in}}%
\pgfpathclose%
\pgfusepath{stroke,fill}%
\end{pgfscope}%
\begin{pgfscope}%
\pgfpathrectangle{\pgfqpoint{0.393613in}{0.331635in}}{\pgfqpoint{9.300000in}{7.700000in}}%
\pgfusepath{clip}%
\pgfsetbuttcap%
\pgfsetroundjoin%
\definecolor{currentfill}{rgb}{0.631373,0.788235,0.956863}%
\pgfsetfillcolor{currentfill}%
\pgfsetlinewidth{0.481800pt}%
\definecolor{currentstroke}{rgb}{1.000000,1.000000,1.000000}%
\pgfsetstrokecolor{currentstroke}%
\pgfsetdash{}{0pt}%
\pgfpathmoveto{\pgfqpoint{2.968097in}{5.645878in}}%
\pgfpathcurveto{\pgfqpoint{2.979147in}{5.645878in}}{\pgfqpoint{2.989746in}{5.650268in}}{\pgfqpoint{2.997560in}{5.658082in}}%
\pgfpathcurveto{\pgfqpoint{3.005373in}{5.665896in}}{\pgfqpoint{3.009764in}{5.676495in}}{\pgfqpoint{3.009764in}{5.687545in}}%
\pgfpathcurveto{\pgfqpoint{3.009764in}{5.698595in}}{\pgfqpoint{3.005373in}{5.709194in}}{\pgfqpoint{2.997560in}{5.717008in}}%
\pgfpathcurveto{\pgfqpoint{2.989746in}{5.724821in}}{\pgfqpoint{2.979147in}{5.729212in}}{\pgfqpoint{2.968097in}{5.729212in}}%
\pgfpathcurveto{\pgfqpoint{2.957047in}{5.729212in}}{\pgfqpoint{2.946448in}{5.724821in}}{\pgfqpoint{2.938634in}{5.717008in}}%
\pgfpathcurveto{\pgfqpoint{2.930821in}{5.709194in}}{\pgfqpoint{2.926430in}{5.698595in}}{\pgfqpoint{2.926430in}{5.687545in}}%
\pgfpathcurveto{\pgfqpoint{2.926430in}{5.676495in}}{\pgfqpoint{2.930821in}{5.665896in}}{\pgfqpoint{2.938634in}{5.658082in}}%
\pgfpathcurveto{\pgfqpoint{2.946448in}{5.650268in}}{\pgfqpoint{2.957047in}{5.645878in}}{\pgfqpoint{2.968097in}{5.645878in}}%
\pgfpathclose%
\pgfusepath{stroke,fill}%
\end{pgfscope}%
\begin{pgfscope}%
\pgfpathrectangle{\pgfqpoint{0.393613in}{0.331635in}}{\pgfqpoint{9.300000in}{7.700000in}}%
\pgfusepath{clip}%
\pgfsetbuttcap%
\pgfsetroundjoin%
\definecolor{currentfill}{rgb}{0.631373,0.788235,0.956863}%
\pgfsetfillcolor{currentfill}%
\pgfsetlinewidth{0.481800pt}%
\definecolor{currentstroke}{rgb}{1.000000,1.000000,1.000000}%
\pgfsetstrokecolor{currentstroke}%
\pgfsetdash{}{0pt}%
\pgfpathmoveto{\pgfqpoint{2.017579in}{2.370733in}}%
\pgfpathcurveto{\pgfqpoint{2.028629in}{2.370733in}}{\pgfqpoint{2.039228in}{2.375123in}}{\pgfqpoint{2.047042in}{2.382937in}}%
\pgfpathcurveto{\pgfqpoint{2.054856in}{2.390750in}}{\pgfqpoint{2.059246in}{2.401350in}}{\pgfqpoint{2.059246in}{2.412400in}}%
\pgfpathcurveto{\pgfqpoint{2.059246in}{2.423450in}}{\pgfqpoint{2.054856in}{2.434049in}}{\pgfqpoint{2.047042in}{2.441862in}}%
\pgfpathcurveto{\pgfqpoint{2.039228in}{2.449676in}}{\pgfqpoint{2.028629in}{2.454066in}}{\pgfqpoint{2.017579in}{2.454066in}}%
\pgfpathcurveto{\pgfqpoint{2.006529in}{2.454066in}}{\pgfqpoint{1.995930in}{2.449676in}}{\pgfqpoint{1.988116in}{2.441862in}}%
\pgfpathcurveto{\pgfqpoint{1.980303in}{2.434049in}}{\pgfqpoint{1.975912in}{2.423450in}}{\pgfqpoint{1.975912in}{2.412400in}}%
\pgfpathcurveto{\pgfqpoint{1.975912in}{2.401350in}}{\pgfqpoint{1.980303in}{2.390750in}}{\pgfqpoint{1.988116in}{2.382937in}}%
\pgfpathcurveto{\pgfqpoint{1.995930in}{2.375123in}}{\pgfqpoint{2.006529in}{2.370733in}}{\pgfqpoint{2.017579in}{2.370733in}}%
\pgfpathclose%
\pgfusepath{stroke,fill}%
\end{pgfscope}%
\begin{pgfscope}%
\pgfpathrectangle{\pgfqpoint{0.393613in}{0.331635in}}{\pgfqpoint{9.300000in}{7.700000in}}%
\pgfusepath{clip}%
\pgfsetbuttcap%
\pgfsetroundjoin%
\definecolor{currentfill}{rgb}{0.631373,0.788235,0.956863}%
\pgfsetfillcolor{currentfill}%
\pgfsetlinewidth{0.481800pt}%
\definecolor{currentstroke}{rgb}{1.000000,1.000000,1.000000}%
\pgfsetstrokecolor{currentstroke}%
\pgfsetdash{}{0pt}%
\pgfpathmoveto{\pgfqpoint{4.875280in}{6.831971in}}%
\pgfpathcurveto{\pgfqpoint{4.886330in}{6.831971in}}{\pgfqpoint{4.896929in}{6.836361in}}{\pgfqpoint{4.904743in}{6.844175in}}%
\pgfpathcurveto{\pgfqpoint{4.912556in}{6.851989in}}{\pgfqpoint{4.916947in}{6.862588in}}{\pgfqpoint{4.916947in}{6.873638in}}%
\pgfpathcurveto{\pgfqpoint{4.916947in}{6.884688in}}{\pgfqpoint{4.912556in}{6.895287in}}{\pgfqpoint{4.904743in}{6.903100in}}%
\pgfpathcurveto{\pgfqpoint{4.896929in}{6.910914in}}{\pgfqpoint{4.886330in}{6.915304in}}{\pgfqpoint{4.875280in}{6.915304in}}%
\pgfpathcurveto{\pgfqpoint{4.864230in}{6.915304in}}{\pgfqpoint{4.853631in}{6.910914in}}{\pgfqpoint{4.845817in}{6.903100in}}%
\pgfpathcurveto{\pgfqpoint{4.838004in}{6.895287in}}{\pgfqpoint{4.833613in}{6.884688in}}{\pgfqpoint{4.833613in}{6.873638in}}%
\pgfpathcurveto{\pgfqpoint{4.833613in}{6.862588in}}{\pgfqpoint{4.838004in}{6.851989in}}{\pgfqpoint{4.845817in}{6.844175in}}%
\pgfpathcurveto{\pgfqpoint{4.853631in}{6.836361in}}{\pgfqpoint{4.864230in}{6.831971in}}{\pgfqpoint{4.875280in}{6.831971in}}%
\pgfpathclose%
\pgfusepath{stroke,fill}%
\end{pgfscope}%
\begin{pgfscope}%
\pgfpathrectangle{\pgfqpoint{0.393613in}{0.331635in}}{\pgfqpoint{9.300000in}{7.700000in}}%
\pgfusepath{clip}%
\pgfsetbuttcap%
\pgfsetroundjoin%
\definecolor{currentfill}{rgb}{0.631373,0.788235,0.956863}%
\pgfsetfillcolor{currentfill}%
\pgfsetlinewidth{0.481800pt}%
\definecolor{currentstroke}{rgb}{1.000000,1.000000,1.000000}%
\pgfsetstrokecolor{currentstroke}%
\pgfsetdash{}{0pt}%
\pgfpathmoveto{\pgfqpoint{4.821671in}{6.252549in}}%
\pgfpathcurveto{\pgfqpoint{4.832721in}{6.252549in}}{\pgfqpoint{4.843320in}{6.256939in}}{\pgfqpoint{4.851134in}{6.264753in}}%
\pgfpathcurveto{\pgfqpoint{4.858947in}{6.272566in}}{\pgfqpoint{4.863338in}{6.283165in}}{\pgfqpoint{4.863338in}{6.294216in}}%
\pgfpathcurveto{\pgfqpoint{4.863338in}{6.305266in}}{\pgfqpoint{4.858947in}{6.315865in}}{\pgfqpoint{4.851134in}{6.323678in}}%
\pgfpathcurveto{\pgfqpoint{4.843320in}{6.331492in}}{\pgfqpoint{4.832721in}{6.335882in}}{\pgfqpoint{4.821671in}{6.335882in}}%
\pgfpathcurveto{\pgfqpoint{4.810621in}{6.335882in}}{\pgfqpoint{4.800022in}{6.331492in}}{\pgfqpoint{4.792208in}{6.323678in}}%
\pgfpathcurveto{\pgfqpoint{4.784395in}{6.315865in}}{\pgfqpoint{4.780004in}{6.305266in}}{\pgfqpoint{4.780004in}{6.294216in}}%
\pgfpathcurveto{\pgfqpoint{4.780004in}{6.283165in}}{\pgfqpoint{4.784395in}{6.272566in}}{\pgfqpoint{4.792208in}{6.264753in}}%
\pgfpathcurveto{\pgfqpoint{4.800022in}{6.256939in}}{\pgfqpoint{4.810621in}{6.252549in}}{\pgfqpoint{4.821671in}{6.252549in}}%
\pgfpathclose%
\pgfusepath{stroke,fill}%
\end{pgfscope}%
\begin{pgfscope}%
\pgfpathrectangle{\pgfqpoint{0.393613in}{0.331635in}}{\pgfqpoint{9.300000in}{7.700000in}}%
\pgfusepath{clip}%
\pgfsetbuttcap%
\pgfsetroundjoin%
\definecolor{currentfill}{rgb}{0.631373,0.788235,0.956863}%
\pgfsetfillcolor{currentfill}%
\pgfsetlinewidth{0.481800pt}%
\definecolor{currentstroke}{rgb}{1.000000,1.000000,1.000000}%
\pgfsetstrokecolor{currentstroke}%
\pgfsetdash{}{0pt}%
\pgfpathmoveto{\pgfqpoint{2.038680in}{2.394137in}}%
\pgfpathcurveto{\pgfqpoint{2.049730in}{2.394137in}}{\pgfqpoint{2.060329in}{2.398527in}}{\pgfqpoint{2.068142in}{2.406340in}}%
\pgfpathcurveto{\pgfqpoint{2.075956in}{2.414154in}}{\pgfqpoint{2.080346in}{2.424753in}}{\pgfqpoint{2.080346in}{2.435803in}}%
\pgfpathcurveto{\pgfqpoint{2.080346in}{2.446853in}}{\pgfqpoint{2.075956in}{2.457452in}}{\pgfqpoint{2.068142in}{2.465266in}}%
\pgfpathcurveto{\pgfqpoint{2.060329in}{2.473080in}}{\pgfqpoint{2.049730in}{2.477470in}}{\pgfqpoint{2.038680in}{2.477470in}}%
\pgfpathcurveto{\pgfqpoint{2.027629in}{2.477470in}}{\pgfqpoint{2.017030in}{2.473080in}}{\pgfqpoint{2.009217in}{2.465266in}}%
\pgfpathcurveto{\pgfqpoint{2.001403in}{2.457452in}}{\pgfqpoint{1.997013in}{2.446853in}}{\pgfqpoint{1.997013in}{2.435803in}}%
\pgfpathcurveto{\pgfqpoint{1.997013in}{2.424753in}}{\pgfqpoint{2.001403in}{2.414154in}}{\pgfqpoint{2.009217in}{2.406340in}}%
\pgfpathcurveto{\pgfqpoint{2.017030in}{2.398527in}}{\pgfqpoint{2.027629in}{2.394137in}}{\pgfqpoint{2.038680in}{2.394137in}}%
\pgfpathclose%
\pgfusepath{stroke,fill}%
\end{pgfscope}%
\begin{pgfscope}%
\pgfpathrectangle{\pgfqpoint{0.393613in}{0.331635in}}{\pgfqpoint{9.300000in}{7.700000in}}%
\pgfusepath{clip}%
\pgfsetbuttcap%
\pgfsetroundjoin%
\definecolor{currentfill}{rgb}{0.631373,0.788235,0.956863}%
\pgfsetfillcolor{currentfill}%
\pgfsetlinewidth{0.481800pt}%
\definecolor{currentstroke}{rgb}{1.000000,1.000000,1.000000}%
\pgfsetstrokecolor{currentstroke}%
\pgfsetdash{}{0pt}%
\pgfpathmoveto{\pgfqpoint{3.158987in}{6.265025in}}%
\pgfpathcurveto{\pgfqpoint{3.170037in}{6.265025in}}{\pgfqpoint{3.180636in}{6.269415in}}{\pgfqpoint{3.188449in}{6.277229in}}%
\pgfpathcurveto{\pgfqpoint{3.196263in}{6.285043in}}{\pgfqpoint{3.200653in}{6.295642in}}{\pgfqpoint{3.200653in}{6.306692in}}%
\pgfpathcurveto{\pgfqpoint{3.200653in}{6.317742in}}{\pgfqpoint{3.196263in}{6.328341in}}{\pgfqpoint{3.188449in}{6.336155in}}%
\pgfpathcurveto{\pgfqpoint{3.180636in}{6.343968in}}{\pgfqpoint{3.170037in}{6.348358in}}{\pgfqpoint{3.158987in}{6.348358in}}%
\pgfpathcurveto{\pgfqpoint{3.147936in}{6.348358in}}{\pgfqpoint{3.137337in}{6.343968in}}{\pgfqpoint{3.129524in}{6.336155in}}%
\pgfpathcurveto{\pgfqpoint{3.121710in}{6.328341in}}{\pgfqpoint{3.117320in}{6.317742in}}{\pgfqpoint{3.117320in}{6.306692in}}%
\pgfpathcurveto{\pgfqpoint{3.117320in}{6.295642in}}{\pgfqpoint{3.121710in}{6.285043in}}{\pgfqpoint{3.129524in}{6.277229in}}%
\pgfpathcurveto{\pgfqpoint{3.137337in}{6.269415in}}{\pgfqpoint{3.147936in}{6.265025in}}{\pgfqpoint{3.158987in}{6.265025in}}%
\pgfpathclose%
\pgfusepath{stroke,fill}%
\end{pgfscope}%
\begin{pgfscope}%
\pgfpathrectangle{\pgfqpoint{0.393613in}{0.331635in}}{\pgfqpoint{9.300000in}{7.700000in}}%
\pgfusepath{clip}%
\pgfsetbuttcap%
\pgfsetroundjoin%
\definecolor{currentfill}{rgb}{0.631373,0.788235,0.956863}%
\pgfsetfillcolor{currentfill}%
\pgfsetlinewidth{0.481800pt}%
\definecolor{currentstroke}{rgb}{1.000000,1.000000,1.000000}%
\pgfsetstrokecolor{currentstroke}%
\pgfsetdash{}{0pt}%
\pgfpathmoveto{\pgfqpoint{3.163926in}{3.153448in}}%
\pgfpathcurveto{\pgfqpoint{3.174976in}{3.153448in}}{\pgfqpoint{3.185575in}{3.157838in}}{\pgfqpoint{3.193389in}{3.165652in}}%
\pgfpathcurveto{\pgfqpoint{3.201203in}{3.173465in}}{\pgfqpoint{3.205593in}{3.184064in}}{\pgfqpoint{3.205593in}{3.195114in}}%
\pgfpathcurveto{\pgfqpoint{3.205593in}{3.206165in}}{\pgfqpoint{3.201203in}{3.216764in}}{\pgfqpoint{3.193389in}{3.224577in}}%
\pgfpathcurveto{\pgfqpoint{3.185575in}{3.232391in}}{\pgfqpoint{3.174976in}{3.236781in}}{\pgfqpoint{3.163926in}{3.236781in}}%
\pgfpathcurveto{\pgfqpoint{3.152876in}{3.236781in}}{\pgfqpoint{3.142277in}{3.232391in}}{\pgfqpoint{3.134463in}{3.224577in}}%
\pgfpathcurveto{\pgfqpoint{3.126650in}{3.216764in}}{\pgfqpoint{3.122259in}{3.206165in}}{\pgfqpoint{3.122259in}{3.195114in}}%
\pgfpathcurveto{\pgfqpoint{3.122259in}{3.184064in}}{\pgfqpoint{3.126650in}{3.173465in}}{\pgfqpoint{3.134463in}{3.165652in}}%
\pgfpathcurveto{\pgfqpoint{3.142277in}{3.157838in}}{\pgfqpoint{3.152876in}{3.153448in}}{\pgfqpoint{3.163926in}{3.153448in}}%
\pgfpathclose%
\pgfusepath{stroke,fill}%
\end{pgfscope}%
\begin{pgfscope}%
\pgfpathrectangle{\pgfqpoint{0.393613in}{0.331635in}}{\pgfqpoint{9.300000in}{7.700000in}}%
\pgfusepath{clip}%
\pgfsetbuttcap%
\pgfsetroundjoin%
\definecolor{currentfill}{rgb}{0.631373,0.788235,0.956863}%
\pgfsetfillcolor{currentfill}%
\pgfsetlinewidth{0.481800pt}%
\definecolor{currentstroke}{rgb}{1.000000,1.000000,1.000000}%
\pgfsetstrokecolor{currentstroke}%
\pgfsetdash{}{0pt}%
\pgfpathmoveto{\pgfqpoint{3.868734in}{6.834554in}}%
\pgfpathcurveto{\pgfqpoint{3.879784in}{6.834554in}}{\pgfqpoint{3.890383in}{6.838945in}}{\pgfqpoint{3.898197in}{6.846758in}}%
\pgfpathcurveto{\pgfqpoint{3.906010in}{6.854572in}}{\pgfqpoint{3.910401in}{6.865171in}}{\pgfqpoint{3.910401in}{6.876221in}}%
\pgfpathcurveto{\pgfqpoint{3.910401in}{6.887271in}}{\pgfqpoint{3.906010in}{6.897870in}}{\pgfqpoint{3.898197in}{6.905684in}}%
\pgfpathcurveto{\pgfqpoint{3.890383in}{6.913497in}}{\pgfqpoint{3.879784in}{6.917888in}}{\pgfqpoint{3.868734in}{6.917888in}}%
\pgfpathcurveto{\pgfqpoint{3.857684in}{6.917888in}}{\pgfqpoint{3.847085in}{6.913497in}}{\pgfqpoint{3.839271in}{6.905684in}}%
\pgfpathcurveto{\pgfqpoint{3.831458in}{6.897870in}}{\pgfqpoint{3.827067in}{6.887271in}}{\pgfqpoint{3.827067in}{6.876221in}}%
\pgfpathcurveto{\pgfqpoint{3.827067in}{6.865171in}}{\pgfqpoint{3.831458in}{6.854572in}}{\pgfqpoint{3.839271in}{6.846758in}}%
\pgfpathcurveto{\pgfqpoint{3.847085in}{6.838945in}}{\pgfqpoint{3.857684in}{6.834554in}}{\pgfqpoint{3.868734in}{6.834554in}}%
\pgfpathclose%
\pgfusepath{stroke,fill}%
\end{pgfscope}%
\begin{pgfscope}%
\pgfpathrectangle{\pgfqpoint{0.393613in}{0.331635in}}{\pgfqpoint{9.300000in}{7.700000in}}%
\pgfusepath{clip}%
\pgfsetbuttcap%
\pgfsetroundjoin%
\definecolor{currentfill}{rgb}{1.000000,0.705882,0.509804}%
\pgfsetfillcolor{currentfill}%
\pgfsetlinewidth{0.481800pt}%
\definecolor{currentstroke}{rgb}{1.000000,1.000000,1.000000}%
\pgfsetstrokecolor{currentstroke}%
\pgfsetdash{}{0pt}%
\pgfpathmoveto{\pgfqpoint{8.729535in}{4.279064in}}%
\pgfpathcurveto{\pgfqpoint{8.740586in}{4.279064in}}{\pgfqpoint{8.751185in}{4.283454in}}{\pgfqpoint{8.758998in}{4.291268in}}%
\pgfpathcurveto{\pgfqpoint{8.766812in}{4.299081in}}{\pgfqpoint{8.771202in}{4.309680in}}{\pgfqpoint{8.771202in}{4.320731in}}%
\pgfpathcurveto{\pgfqpoint{8.771202in}{4.331781in}}{\pgfqpoint{8.766812in}{4.342380in}}{\pgfqpoint{8.758998in}{4.350193in}}%
\pgfpathcurveto{\pgfqpoint{8.751185in}{4.358007in}}{\pgfqpoint{8.740586in}{4.362397in}}{\pgfqpoint{8.729535in}{4.362397in}}%
\pgfpathcurveto{\pgfqpoint{8.718485in}{4.362397in}}{\pgfqpoint{8.707886in}{4.358007in}}{\pgfqpoint{8.700073in}{4.350193in}}%
\pgfpathcurveto{\pgfqpoint{8.692259in}{4.342380in}}{\pgfqpoint{8.687869in}{4.331781in}}{\pgfqpoint{8.687869in}{4.320731in}}%
\pgfpathcurveto{\pgfqpoint{8.687869in}{4.309680in}}{\pgfqpoint{8.692259in}{4.299081in}}{\pgfqpoint{8.700073in}{4.291268in}}%
\pgfpathcurveto{\pgfqpoint{8.707886in}{4.283454in}}{\pgfqpoint{8.718485in}{4.279064in}}{\pgfqpoint{8.729535in}{4.279064in}}%
\pgfpathclose%
\pgfusepath{stroke,fill}%
\end{pgfscope}%
\begin{pgfscope}%
\pgfpathrectangle{\pgfqpoint{0.393613in}{0.331635in}}{\pgfqpoint{9.300000in}{7.700000in}}%
\pgfusepath{clip}%
\pgfsetbuttcap%
\pgfsetroundjoin%
\definecolor{currentfill}{rgb}{1.000000,0.705882,0.509804}%
\pgfsetfillcolor{currentfill}%
\pgfsetlinewidth{0.481800pt}%
\definecolor{currentstroke}{rgb}{1.000000,1.000000,1.000000}%
\pgfsetstrokecolor{currentstroke}%
\pgfsetdash{}{0pt}%
\pgfpathmoveto{\pgfqpoint{8.008246in}{2.574721in}}%
\pgfpathcurveto{\pgfqpoint{8.019296in}{2.574721in}}{\pgfqpoint{8.029895in}{2.579111in}}{\pgfqpoint{8.037709in}{2.586924in}}%
\pgfpathcurveto{\pgfqpoint{8.045522in}{2.594738in}}{\pgfqpoint{8.049913in}{2.605337in}}{\pgfqpoint{8.049913in}{2.616387in}}%
\pgfpathcurveto{\pgfqpoint{8.049913in}{2.627437in}}{\pgfqpoint{8.045522in}{2.638036in}}{\pgfqpoint{8.037709in}{2.645850in}}%
\pgfpathcurveto{\pgfqpoint{8.029895in}{2.653664in}}{\pgfqpoint{8.019296in}{2.658054in}}{\pgfqpoint{8.008246in}{2.658054in}}%
\pgfpathcurveto{\pgfqpoint{7.997196in}{2.658054in}}{\pgfqpoint{7.986597in}{2.653664in}}{\pgfqpoint{7.978783in}{2.645850in}}%
\pgfpathcurveto{\pgfqpoint{7.970970in}{2.638036in}}{\pgfqpoint{7.966579in}{2.627437in}}{\pgfqpoint{7.966579in}{2.616387in}}%
\pgfpathcurveto{\pgfqpoint{7.966579in}{2.605337in}}{\pgfqpoint{7.970970in}{2.594738in}}{\pgfqpoint{7.978783in}{2.586924in}}%
\pgfpathcurveto{\pgfqpoint{7.986597in}{2.579111in}}{\pgfqpoint{7.997196in}{2.574721in}}{\pgfqpoint{8.008246in}{2.574721in}}%
\pgfpathclose%
\pgfusepath{stroke,fill}%
\end{pgfscope}%
\begin{pgfscope}%
\pgfpathrectangle{\pgfqpoint{0.393613in}{0.331635in}}{\pgfqpoint{9.300000in}{7.700000in}}%
\pgfusepath{clip}%
\pgfsetbuttcap%
\pgfsetroundjoin%
\definecolor{currentfill}{rgb}{1.000000,0.705882,0.509804}%
\pgfsetfillcolor{currentfill}%
\pgfsetlinewidth{0.481800pt}%
\definecolor{currentstroke}{rgb}{1.000000,1.000000,1.000000}%
\pgfsetstrokecolor{currentstroke}%
\pgfsetdash{}{0pt}%
\pgfpathmoveto{\pgfqpoint{6.843430in}{2.204420in}}%
\pgfpathcurveto{\pgfqpoint{6.854480in}{2.204420in}}{\pgfqpoint{6.865079in}{2.208810in}}{\pgfqpoint{6.872893in}{2.216624in}}%
\pgfpathcurveto{\pgfqpoint{6.880706in}{2.224437in}}{\pgfqpoint{6.885097in}{2.235036in}}{\pgfqpoint{6.885097in}{2.246087in}}%
\pgfpathcurveto{\pgfqpoint{6.885097in}{2.257137in}}{\pgfqpoint{6.880706in}{2.267736in}}{\pgfqpoint{6.872893in}{2.275549in}}%
\pgfpathcurveto{\pgfqpoint{6.865079in}{2.283363in}}{\pgfqpoint{6.854480in}{2.287753in}}{\pgfqpoint{6.843430in}{2.287753in}}%
\pgfpathcurveto{\pgfqpoint{6.832380in}{2.287753in}}{\pgfqpoint{6.821781in}{2.283363in}}{\pgfqpoint{6.813967in}{2.275549in}}%
\pgfpathcurveto{\pgfqpoint{6.806153in}{2.267736in}}{\pgfqpoint{6.801763in}{2.257137in}}{\pgfqpoint{6.801763in}{2.246087in}}%
\pgfpathcurveto{\pgfqpoint{6.801763in}{2.235036in}}{\pgfqpoint{6.806153in}{2.224437in}}{\pgfqpoint{6.813967in}{2.216624in}}%
\pgfpathcurveto{\pgfqpoint{6.821781in}{2.208810in}}{\pgfqpoint{6.832380in}{2.204420in}}{\pgfqpoint{6.843430in}{2.204420in}}%
\pgfpathclose%
\pgfusepath{stroke,fill}%
\end{pgfscope}%
\begin{pgfscope}%
\pgfpathrectangle{\pgfqpoint{0.393613in}{0.331635in}}{\pgfqpoint{9.300000in}{7.700000in}}%
\pgfusepath{clip}%
\pgfsetbuttcap%
\pgfsetroundjoin%
\definecolor{currentfill}{rgb}{1.000000,0.705882,0.509804}%
\pgfsetfillcolor{currentfill}%
\pgfsetlinewidth{0.481800pt}%
\definecolor{currentstroke}{rgb}{1.000000,1.000000,1.000000}%
\pgfsetstrokecolor{currentstroke}%
\pgfsetdash{}{0pt}%
\pgfpathmoveto{\pgfqpoint{6.884025in}{6.061158in}}%
\pgfpathcurveto{\pgfqpoint{6.895076in}{6.061158in}}{\pgfqpoint{6.905675in}{6.065548in}}{\pgfqpoint{6.913488in}{6.073362in}}%
\pgfpathcurveto{\pgfqpoint{6.921302in}{6.081176in}}{\pgfqpoint{6.925692in}{6.091775in}}{\pgfqpoint{6.925692in}{6.102825in}}%
\pgfpathcurveto{\pgfqpoint{6.925692in}{6.113875in}}{\pgfqpoint{6.921302in}{6.124474in}}{\pgfqpoint{6.913488in}{6.132288in}}%
\pgfpathcurveto{\pgfqpoint{6.905675in}{6.140101in}}{\pgfqpoint{6.895076in}{6.144492in}}{\pgfqpoint{6.884025in}{6.144492in}}%
\pgfpathcurveto{\pgfqpoint{6.872975in}{6.144492in}}{\pgfqpoint{6.862376in}{6.140101in}}{\pgfqpoint{6.854563in}{6.132288in}}%
\pgfpathcurveto{\pgfqpoint{6.846749in}{6.124474in}}{\pgfqpoint{6.842359in}{6.113875in}}{\pgfqpoint{6.842359in}{6.102825in}}%
\pgfpathcurveto{\pgfqpoint{6.842359in}{6.091775in}}{\pgfqpoint{6.846749in}{6.081176in}}{\pgfqpoint{6.854563in}{6.073362in}}%
\pgfpathcurveto{\pgfqpoint{6.862376in}{6.065548in}}{\pgfqpoint{6.872975in}{6.061158in}}{\pgfqpoint{6.884025in}{6.061158in}}%
\pgfpathclose%
\pgfusepath{stroke,fill}%
\end{pgfscope}%
\begin{pgfscope}%
\pgfpathrectangle{\pgfqpoint{0.393613in}{0.331635in}}{\pgfqpoint{9.300000in}{7.700000in}}%
\pgfusepath{clip}%
\pgfsetbuttcap%
\pgfsetroundjoin%
\definecolor{currentfill}{rgb}{1.000000,0.705882,0.509804}%
\pgfsetfillcolor{currentfill}%
\pgfsetlinewidth{0.481800pt}%
\definecolor{currentstroke}{rgb}{1.000000,1.000000,1.000000}%
\pgfsetstrokecolor{currentstroke}%
\pgfsetdash{}{0pt}%
\pgfpathmoveto{\pgfqpoint{8.217971in}{5.450756in}}%
\pgfpathcurveto{\pgfqpoint{8.229022in}{5.450756in}}{\pgfqpoint{8.239621in}{5.455146in}}{\pgfqpoint{8.247434in}{5.462960in}}%
\pgfpathcurveto{\pgfqpoint{8.255248in}{5.470774in}}{\pgfqpoint{8.259638in}{5.481373in}}{\pgfqpoint{8.259638in}{5.492423in}}%
\pgfpathcurveto{\pgfqpoint{8.259638in}{5.503473in}}{\pgfqpoint{8.255248in}{5.514072in}}{\pgfqpoint{8.247434in}{5.521886in}}%
\pgfpathcurveto{\pgfqpoint{8.239621in}{5.529699in}}{\pgfqpoint{8.229022in}{5.534089in}}{\pgfqpoint{8.217971in}{5.534089in}}%
\pgfpathcurveto{\pgfqpoint{8.206921in}{5.534089in}}{\pgfqpoint{8.196322in}{5.529699in}}{\pgfqpoint{8.188509in}{5.521886in}}%
\pgfpathcurveto{\pgfqpoint{8.180695in}{5.514072in}}{\pgfqpoint{8.176305in}{5.503473in}}{\pgfqpoint{8.176305in}{5.492423in}}%
\pgfpathcurveto{\pgfqpoint{8.176305in}{5.481373in}}{\pgfqpoint{8.180695in}{5.470774in}}{\pgfqpoint{8.188509in}{5.462960in}}%
\pgfpathcurveto{\pgfqpoint{8.196322in}{5.455146in}}{\pgfqpoint{8.206921in}{5.450756in}}{\pgfqpoint{8.217971in}{5.450756in}}%
\pgfpathclose%
\pgfusepath{stroke,fill}%
\end{pgfscope}%
\begin{pgfscope}%
\pgfpathrectangle{\pgfqpoint{0.393613in}{0.331635in}}{\pgfqpoint{9.300000in}{7.700000in}}%
\pgfusepath{clip}%
\pgfsetbuttcap%
\pgfsetroundjoin%
\definecolor{currentfill}{rgb}{1.000000,0.705882,0.509804}%
\pgfsetfillcolor{currentfill}%
\pgfsetlinewidth{0.481800pt}%
\definecolor{currentstroke}{rgb}{1.000000,1.000000,1.000000}%
\pgfsetstrokecolor{currentstroke}%
\pgfsetdash{}{0pt}%
\pgfpathmoveto{\pgfqpoint{7.449270in}{4.145806in}}%
\pgfpathcurveto{\pgfqpoint{7.460320in}{4.145806in}}{\pgfqpoint{7.470919in}{4.150197in}}{\pgfqpoint{7.478733in}{4.158010in}}%
\pgfpathcurveto{\pgfqpoint{7.486546in}{4.165824in}}{\pgfqpoint{7.490937in}{4.176423in}}{\pgfqpoint{7.490937in}{4.187473in}}%
\pgfpathcurveto{\pgfqpoint{7.490937in}{4.198523in}}{\pgfqpoint{7.486546in}{4.209122in}}{\pgfqpoint{7.478733in}{4.216936in}}%
\pgfpathcurveto{\pgfqpoint{7.470919in}{4.224750in}}{\pgfqpoint{7.460320in}{4.229140in}}{\pgfqpoint{7.449270in}{4.229140in}}%
\pgfpathcurveto{\pgfqpoint{7.438220in}{4.229140in}}{\pgfqpoint{7.427621in}{4.224750in}}{\pgfqpoint{7.419807in}{4.216936in}}%
\pgfpathcurveto{\pgfqpoint{7.411994in}{4.209122in}}{\pgfqpoint{7.407603in}{4.198523in}}{\pgfqpoint{7.407603in}{4.187473in}}%
\pgfpathcurveto{\pgfqpoint{7.407603in}{4.176423in}}{\pgfqpoint{7.411994in}{4.165824in}}{\pgfqpoint{7.419807in}{4.158010in}}%
\pgfpathcurveto{\pgfqpoint{7.427621in}{4.150197in}}{\pgfqpoint{7.438220in}{4.145806in}}{\pgfqpoint{7.449270in}{4.145806in}}%
\pgfpathclose%
\pgfusepath{stroke,fill}%
\end{pgfscope}%
\begin{pgfscope}%
\pgfpathrectangle{\pgfqpoint{0.393613in}{0.331635in}}{\pgfqpoint{9.300000in}{7.700000in}}%
\pgfusepath{clip}%
\pgfsetbuttcap%
\pgfsetroundjoin%
\definecolor{currentfill}{rgb}{1.000000,0.705882,0.509804}%
\pgfsetfillcolor{currentfill}%
\pgfsetlinewidth{0.481800pt}%
\definecolor{currentstroke}{rgb}{1.000000,1.000000,1.000000}%
\pgfsetstrokecolor{currentstroke}%
\pgfsetdash{}{0pt}%
\pgfpathmoveto{\pgfqpoint{7.634735in}{5.895591in}}%
\pgfpathcurveto{\pgfqpoint{7.645786in}{5.895591in}}{\pgfqpoint{7.656385in}{5.899981in}}{\pgfqpoint{7.664198in}{5.907795in}}%
\pgfpathcurveto{\pgfqpoint{7.672012in}{5.915609in}}{\pgfqpoint{7.676402in}{5.926208in}}{\pgfqpoint{7.676402in}{5.937258in}}%
\pgfpathcurveto{\pgfqpoint{7.676402in}{5.948308in}}{\pgfqpoint{7.672012in}{5.958907in}}{\pgfqpoint{7.664198in}{5.966721in}}%
\pgfpathcurveto{\pgfqpoint{7.656385in}{5.974534in}}{\pgfqpoint{7.645786in}{5.978925in}}{\pgfqpoint{7.634735in}{5.978925in}}%
\pgfpathcurveto{\pgfqpoint{7.623685in}{5.978925in}}{\pgfqpoint{7.613086in}{5.974534in}}{\pgfqpoint{7.605273in}{5.966721in}}%
\pgfpathcurveto{\pgfqpoint{7.597459in}{5.958907in}}{\pgfqpoint{7.593069in}{5.948308in}}{\pgfqpoint{7.593069in}{5.937258in}}%
\pgfpathcurveto{\pgfqpoint{7.593069in}{5.926208in}}{\pgfqpoint{7.597459in}{5.915609in}}{\pgfqpoint{7.605273in}{5.907795in}}%
\pgfpathcurveto{\pgfqpoint{7.613086in}{5.899981in}}{\pgfqpoint{7.623685in}{5.895591in}}{\pgfqpoint{7.634735in}{5.895591in}}%
\pgfpathclose%
\pgfusepath{stroke,fill}%
\end{pgfscope}%
\begin{pgfscope}%
\pgfpathrectangle{\pgfqpoint{0.393613in}{0.331635in}}{\pgfqpoint{9.300000in}{7.700000in}}%
\pgfusepath{clip}%
\pgfsetbuttcap%
\pgfsetroundjoin%
\definecolor{currentfill}{rgb}{1.000000,0.705882,0.509804}%
\pgfsetfillcolor{currentfill}%
\pgfsetlinewidth{0.481800pt}%
\definecolor{currentstroke}{rgb}{1.000000,1.000000,1.000000}%
\pgfsetstrokecolor{currentstroke}%
\pgfsetdash{}{0pt}%
\pgfpathmoveto{\pgfqpoint{7.737786in}{3.811293in}}%
\pgfpathcurveto{\pgfqpoint{7.748836in}{3.811293in}}{\pgfqpoint{7.759435in}{3.815683in}}{\pgfqpoint{7.767249in}{3.823497in}}%
\pgfpathcurveto{\pgfqpoint{7.775062in}{3.831310in}}{\pgfqpoint{7.779453in}{3.841909in}}{\pgfqpoint{7.779453in}{3.852959in}}%
\pgfpathcurveto{\pgfqpoint{7.779453in}{3.864009in}}{\pgfqpoint{7.775062in}{3.874608in}}{\pgfqpoint{7.767249in}{3.882422in}}%
\pgfpathcurveto{\pgfqpoint{7.759435in}{3.890236in}}{\pgfqpoint{7.748836in}{3.894626in}}{\pgfqpoint{7.737786in}{3.894626in}}%
\pgfpathcurveto{\pgfqpoint{7.726736in}{3.894626in}}{\pgfqpoint{7.716137in}{3.890236in}}{\pgfqpoint{7.708323in}{3.882422in}}%
\pgfpathcurveto{\pgfqpoint{7.700510in}{3.874608in}}{\pgfqpoint{7.696119in}{3.864009in}}{\pgfqpoint{7.696119in}{3.852959in}}%
\pgfpathcurveto{\pgfqpoint{7.696119in}{3.841909in}}{\pgfqpoint{7.700510in}{3.831310in}}{\pgfqpoint{7.708323in}{3.823497in}}%
\pgfpathcurveto{\pgfqpoint{7.716137in}{3.815683in}}{\pgfqpoint{7.726736in}{3.811293in}}{\pgfqpoint{7.737786in}{3.811293in}}%
\pgfpathclose%
\pgfusepath{stroke,fill}%
\end{pgfscope}%
\begin{pgfscope}%
\pgfpathrectangle{\pgfqpoint{0.393613in}{0.331635in}}{\pgfqpoint{9.300000in}{7.700000in}}%
\pgfusepath{clip}%
\pgfsetbuttcap%
\pgfsetroundjoin%
\definecolor{currentfill}{rgb}{1.000000,0.705882,0.509804}%
\pgfsetfillcolor{currentfill}%
\pgfsetlinewidth{0.481800pt}%
\definecolor{currentstroke}{rgb}{1.000000,1.000000,1.000000}%
\pgfsetstrokecolor{currentstroke}%
\pgfsetdash{}{0pt}%
\pgfpathmoveto{\pgfqpoint{6.985227in}{5.946910in}}%
\pgfpathcurveto{\pgfqpoint{6.996277in}{5.946910in}}{\pgfqpoint{7.006876in}{5.951301in}}{\pgfqpoint{7.014689in}{5.959114in}}%
\pgfpathcurveto{\pgfqpoint{7.022503in}{5.966928in}}{\pgfqpoint{7.026893in}{5.977527in}}{\pgfqpoint{7.026893in}{5.988577in}}%
\pgfpathcurveto{\pgfqpoint{7.026893in}{5.999627in}}{\pgfqpoint{7.022503in}{6.010226in}}{\pgfqpoint{7.014689in}{6.018040in}}%
\pgfpathcurveto{\pgfqpoint{7.006876in}{6.025853in}}{\pgfqpoint{6.996277in}{6.030244in}}{\pgfqpoint{6.985227in}{6.030244in}}%
\pgfpathcurveto{\pgfqpoint{6.974177in}{6.030244in}}{\pgfqpoint{6.963578in}{6.025853in}}{\pgfqpoint{6.955764in}{6.018040in}}%
\pgfpathcurveto{\pgfqpoint{6.947950in}{6.010226in}}{\pgfqpoint{6.943560in}{5.999627in}}{\pgfqpoint{6.943560in}{5.988577in}}%
\pgfpathcurveto{\pgfqpoint{6.943560in}{5.977527in}}{\pgfqpoint{6.947950in}{5.966928in}}{\pgfqpoint{6.955764in}{5.959114in}}%
\pgfpathcurveto{\pgfqpoint{6.963578in}{5.951301in}}{\pgfqpoint{6.974177in}{5.946910in}}{\pgfqpoint{6.985227in}{5.946910in}}%
\pgfpathclose%
\pgfusepath{stroke,fill}%
\end{pgfscope}%
\begin{pgfscope}%
\pgfpathrectangle{\pgfqpoint{0.393613in}{0.331635in}}{\pgfqpoint{9.300000in}{7.700000in}}%
\pgfusepath{clip}%
\pgfsetbuttcap%
\pgfsetroundjoin%
\definecolor{currentfill}{rgb}{1.000000,0.705882,0.509804}%
\pgfsetfillcolor{currentfill}%
\pgfsetlinewidth{0.481800pt}%
\definecolor{currentstroke}{rgb}{1.000000,1.000000,1.000000}%
\pgfsetstrokecolor{currentstroke}%
\pgfsetdash{}{0pt}%
\pgfpathmoveto{\pgfqpoint{8.526151in}{3.449595in}}%
\pgfpathcurveto{\pgfqpoint{8.537202in}{3.449595in}}{\pgfqpoint{8.547801in}{3.453986in}}{\pgfqpoint{8.555614in}{3.461799in}}%
\pgfpathcurveto{\pgfqpoint{8.563428in}{3.469613in}}{\pgfqpoint{8.567818in}{3.480212in}}{\pgfqpoint{8.567818in}{3.491262in}}%
\pgfpathcurveto{\pgfqpoint{8.567818in}{3.502312in}}{\pgfqpoint{8.563428in}{3.512911in}}{\pgfqpoint{8.555614in}{3.520725in}}%
\pgfpathcurveto{\pgfqpoint{8.547801in}{3.528538in}}{\pgfqpoint{8.537202in}{3.532929in}}{\pgfqpoint{8.526151in}{3.532929in}}%
\pgfpathcurveto{\pgfqpoint{8.515101in}{3.532929in}}{\pgfqpoint{8.504502in}{3.528538in}}{\pgfqpoint{8.496689in}{3.520725in}}%
\pgfpathcurveto{\pgfqpoint{8.488875in}{3.512911in}}{\pgfqpoint{8.484485in}{3.502312in}}{\pgfqpoint{8.484485in}{3.491262in}}%
\pgfpathcurveto{\pgfqpoint{8.484485in}{3.480212in}}{\pgfqpoint{8.488875in}{3.469613in}}{\pgfqpoint{8.496689in}{3.461799in}}%
\pgfpathcurveto{\pgfqpoint{8.504502in}{3.453986in}}{\pgfqpoint{8.515101in}{3.449595in}}{\pgfqpoint{8.526151in}{3.449595in}}%
\pgfpathclose%
\pgfusepath{stroke,fill}%
\end{pgfscope}%
\begin{pgfscope}%
\pgfpathrectangle{\pgfqpoint{0.393613in}{0.331635in}}{\pgfqpoint{9.300000in}{7.700000in}}%
\pgfusepath{clip}%
\pgfsetbuttcap%
\pgfsetroundjoin%
\definecolor{currentfill}{rgb}{1.000000,0.705882,0.509804}%
\pgfsetfillcolor{currentfill}%
\pgfsetlinewidth{0.481800pt}%
\definecolor{currentstroke}{rgb}{1.000000,1.000000,1.000000}%
\pgfsetstrokecolor{currentstroke}%
\pgfsetdash{}{0pt}%
\pgfpathmoveto{\pgfqpoint{4.970295in}{4.640782in}}%
\pgfpathcurveto{\pgfqpoint{4.981345in}{4.640782in}}{\pgfqpoint{4.991944in}{4.645172in}}{\pgfqpoint{4.999758in}{4.652985in}}%
\pgfpathcurveto{\pgfqpoint{5.007571in}{4.660799in}}{\pgfqpoint{5.011962in}{4.671398in}}{\pgfqpoint{5.011962in}{4.682448in}}%
\pgfpathcurveto{\pgfqpoint{5.011962in}{4.693498in}}{\pgfqpoint{5.007571in}{4.704097in}}{\pgfqpoint{4.999758in}{4.711911in}}%
\pgfpathcurveto{\pgfqpoint{4.991944in}{4.719725in}}{\pgfqpoint{4.981345in}{4.724115in}}{\pgfqpoint{4.970295in}{4.724115in}}%
\pgfpathcurveto{\pgfqpoint{4.959245in}{4.724115in}}{\pgfqpoint{4.948646in}{4.719725in}}{\pgfqpoint{4.940832in}{4.711911in}}%
\pgfpathcurveto{\pgfqpoint{4.933018in}{4.704097in}}{\pgfqpoint{4.928628in}{4.693498in}}{\pgfqpoint{4.928628in}{4.682448in}}%
\pgfpathcurveto{\pgfqpoint{4.928628in}{4.671398in}}{\pgfqpoint{4.933018in}{4.660799in}}{\pgfqpoint{4.940832in}{4.652985in}}%
\pgfpathcurveto{\pgfqpoint{4.948646in}{4.645172in}}{\pgfqpoint{4.959245in}{4.640782in}}{\pgfqpoint{4.970295in}{4.640782in}}%
\pgfpathclose%
\pgfusepath{stroke,fill}%
\end{pgfscope}%
\begin{pgfscope}%
\pgfpathrectangle{\pgfqpoint{0.393613in}{0.331635in}}{\pgfqpoint{9.300000in}{7.700000in}}%
\pgfusepath{clip}%
\pgfsetbuttcap%
\pgfsetroundjoin%
\definecolor{currentfill}{rgb}{1.000000,0.705882,0.509804}%
\pgfsetfillcolor{currentfill}%
\pgfsetlinewidth{0.481800pt}%
\definecolor{currentstroke}{rgb}{1.000000,1.000000,1.000000}%
\pgfsetstrokecolor{currentstroke}%
\pgfsetdash{}{0pt}%
\pgfpathmoveto{\pgfqpoint{4.066721in}{5.258266in}}%
\pgfpathcurveto{\pgfqpoint{4.077771in}{5.258266in}}{\pgfqpoint{4.088370in}{5.262656in}}{\pgfqpoint{4.096184in}{5.270470in}}%
\pgfpathcurveto{\pgfqpoint{4.103997in}{5.278284in}}{\pgfqpoint{4.108388in}{5.288883in}}{\pgfqpoint{4.108388in}{5.299933in}}%
\pgfpathcurveto{\pgfqpoint{4.108388in}{5.310983in}}{\pgfqpoint{4.103997in}{5.321582in}}{\pgfqpoint{4.096184in}{5.329396in}}%
\pgfpathcurveto{\pgfqpoint{4.088370in}{5.337209in}}{\pgfqpoint{4.077771in}{5.341600in}}{\pgfqpoint{4.066721in}{5.341600in}}%
\pgfpathcurveto{\pgfqpoint{4.055671in}{5.341600in}}{\pgfqpoint{4.045072in}{5.337209in}}{\pgfqpoint{4.037258in}{5.329396in}}%
\pgfpathcurveto{\pgfqpoint{4.029444in}{5.321582in}}{\pgfqpoint{4.025054in}{5.310983in}}{\pgfqpoint{4.025054in}{5.299933in}}%
\pgfpathcurveto{\pgfqpoint{4.025054in}{5.288883in}}{\pgfqpoint{4.029444in}{5.278284in}}{\pgfqpoint{4.037258in}{5.270470in}}%
\pgfpathcurveto{\pgfqpoint{4.045072in}{5.262656in}}{\pgfqpoint{4.055671in}{5.258266in}}{\pgfqpoint{4.066721in}{5.258266in}}%
\pgfpathclose%
\pgfusepath{stroke,fill}%
\end{pgfscope}%
\begin{pgfscope}%
\pgfpathrectangle{\pgfqpoint{0.393613in}{0.331635in}}{\pgfqpoint{9.300000in}{7.700000in}}%
\pgfusepath{clip}%
\pgfsetbuttcap%
\pgfsetroundjoin%
\definecolor{currentfill}{rgb}{1.000000,0.705882,0.509804}%
\pgfsetfillcolor{currentfill}%
\pgfsetlinewidth{0.481800pt}%
\definecolor{currentstroke}{rgb}{1.000000,1.000000,1.000000}%
\pgfsetstrokecolor{currentstroke}%
\pgfsetdash{}{0pt}%
\pgfpathmoveto{\pgfqpoint{5.047977in}{5.087951in}}%
\pgfpathcurveto{\pgfqpoint{5.059027in}{5.087951in}}{\pgfqpoint{5.069626in}{5.092341in}}{\pgfqpoint{5.077439in}{5.100155in}}%
\pgfpathcurveto{\pgfqpoint{5.085253in}{5.107968in}}{\pgfqpoint{5.089643in}{5.118567in}}{\pgfqpoint{5.089643in}{5.129617in}}%
\pgfpathcurveto{\pgfqpoint{5.089643in}{5.140668in}}{\pgfqpoint{5.085253in}{5.151267in}}{\pgfqpoint{5.077439in}{5.159080in}}%
\pgfpathcurveto{\pgfqpoint{5.069626in}{5.166894in}}{\pgfqpoint{5.059027in}{5.171284in}}{\pgfqpoint{5.047977in}{5.171284in}}%
\pgfpathcurveto{\pgfqpoint{5.036927in}{5.171284in}}{\pgfqpoint{5.026328in}{5.166894in}}{\pgfqpoint{5.018514in}{5.159080in}}%
\pgfpathcurveto{\pgfqpoint{5.010700in}{5.151267in}}{\pgfqpoint{5.006310in}{5.140668in}}{\pgfqpoint{5.006310in}{5.129617in}}%
\pgfpathcurveto{\pgfqpoint{5.006310in}{5.118567in}}{\pgfqpoint{5.010700in}{5.107968in}}{\pgfqpoint{5.018514in}{5.100155in}}%
\pgfpathcurveto{\pgfqpoint{5.026328in}{5.092341in}}{\pgfqpoint{5.036927in}{5.087951in}}{\pgfqpoint{5.047977in}{5.087951in}}%
\pgfpathclose%
\pgfusepath{stroke,fill}%
\end{pgfscope}%
\begin{pgfscope}%
\pgfpathrectangle{\pgfqpoint{0.393613in}{0.331635in}}{\pgfqpoint{9.300000in}{7.700000in}}%
\pgfusepath{clip}%
\pgfsetbuttcap%
\pgfsetroundjoin%
\definecolor{currentfill}{rgb}{1.000000,0.705882,0.509804}%
\pgfsetfillcolor{currentfill}%
\pgfsetlinewidth{0.481800pt}%
\definecolor{currentstroke}{rgb}{1.000000,1.000000,1.000000}%
\pgfsetstrokecolor{currentstroke}%
\pgfsetdash{}{0pt}%
\pgfpathmoveto{\pgfqpoint{8.708537in}{3.107892in}}%
\pgfpathcurveto{\pgfqpoint{8.719587in}{3.107892in}}{\pgfqpoint{8.730186in}{3.112282in}}{\pgfqpoint{8.738000in}{3.120096in}}%
\pgfpathcurveto{\pgfqpoint{8.745813in}{3.127909in}}{\pgfqpoint{8.750204in}{3.138508in}}{\pgfqpoint{8.750204in}{3.149559in}}%
\pgfpathcurveto{\pgfqpoint{8.750204in}{3.160609in}}{\pgfqpoint{8.745813in}{3.171208in}}{\pgfqpoint{8.738000in}{3.179021in}}%
\pgfpathcurveto{\pgfqpoint{8.730186in}{3.186835in}}{\pgfqpoint{8.719587in}{3.191225in}}{\pgfqpoint{8.708537in}{3.191225in}}%
\pgfpathcurveto{\pgfqpoint{8.697487in}{3.191225in}}{\pgfqpoint{8.686888in}{3.186835in}}{\pgfqpoint{8.679074in}{3.179021in}}%
\pgfpathcurveto{\pgfqpoint{8.671261in}{3.171208in}}{\pgfqpoint{8.666870in}{3.160609in}}{\pgfqpoint{8.666870in}{3.149559in}}%
\pgfpathcurveto{\pgfqpoint{8.666870in}{3.138508in}}{\pgfqpoint{8.671261in}{3.127909in}}{\pgfqpoint{8.679074in}{3.120096in}}%
\pgfpathcurveto{\pgfqpoint{8.686888in}{3.112282in}}{\pgfqpoint{8.697487in}{3.107892in}}{\pgfqpoint{8.708537in}{3.107892in}}%
\pgfpathclose%
\pgfusepath{stroke,fill}%
\end{pgfscope}%
\begin{pgfscope}%
\pgfpathrectangle{\pgfqpoint{0.393613in}{0.331635in}}{\pgfqpoint{9.300000in}{7.700000in}}%
\pgfusepath{clip}%
\pgfsetbuttcap%
\pgfsetroundjoin%
\definecolor{currentfill}{rgb}{1.000000,0.705882,0.509804}%
\pgfsetfillcolor{currentfill}%
\pgfsetlinewidth{0.481800pt}%
\definecolor{currentstroke}{rgb}{1.000000,1.000000,1.000000}%
\pgfsetstrokecolor{currentstroke}%
\pgfsetdash{}{0pt}%
\pgfpathmoveto{\pgfqpoint{6.611184in}{2.557264in}}%
\pgfpathcurveto{\pgfqpoint{6.622235in}{2.557264in}}{\pgfqpoint{6.632834in}{2.561655in}}{\pgfqpoint{6.640647in}{2.569468in}}%
\pgfpathcurveto{\pgfqpoint{6.648461in}{2.577282in}}{\pgfqpoint{6.652851in}{2.587881in}}{\pgfqpoint{6.652851in}{2.598931in}}%
\pgfpathcurveto{\pgfqpoint{6.652851in}{2.609981in}}{\pgfqpoint{6.648461in}{2.620580in}}{\pgfqpoint{6.640647in}{2.628394in}}%
\pgfpathcurveto{\pgfqpoint{6.632834in}{2.636208in}}{\pgfqpoint{6.622235in}{2.640598in}}{\pgfqpoint{6.611184in}{2.640598in}}%
\pgfpathcurveto{\pgfqpoint{6.600134in}{2.640598in}}{\pgfqpoint{6.589535in}{2.636208in}}{\pgfqpoint{6.581722in}{2.628394in}}%
\pgfpathcurveto{\pgfqpoint{6.573908in}{2.620580in}}{\pgfqpoint{6.569518in}{2.609981in}}{\pgfqpoint{6.569518in}{2.598931in}}%
\pgfpathcurveto{\pgfqpoint{6.569518in}{2.587881in}}{\pgfqpoint{6.573908in}{2.577282in}}{\pgfqpoint{6.581722in}{2.569468in}}%
\pgfpathcurveto{\pgfqpoint{6.589535in}{2.561655in}}{\pgfqpoint{6.600134in}{2.557264in}}{\pgfqpoint{6.611184in}{2.557264in}}%
\pgfpathclose%
\pgfusepath{stroke,fill}%
\end{pgfscope}%
\begin{pgfscope}%
\pgfpathrectangle{\pgfqpoint{0.393613in}{0.331635in}}{\pgfqpoint{9.300000in}{7.700000in}}%
\pgfusepath{clip}%
\pgfsetbuttcap%
\pgfsetroundjoin%
\definecolor{currentfill}{rgb}{1.000000,0.705882,0.509804}%
\pgfsetfillcolor{currentfill}%
\pgfsetlinewidth{0.481800pt}%
\definecolor{currentstroke}{rgb}{1.000000,1.000000,1.000000}%
\pgfsetstrokecolor{currentstroke}%
\pgfsetdash{}{0pt}%
\pgfpathmoveto{\pgfqpoint{5.696576in}{3.888908in}}%
\pgfpathcurveto{\pgfqpoint{5.707627in}{3.888908in}}{\pgfqpoint{5.718226in}{3.893298in}}{\pgfqpoint{5.726039in}{3.901112in}}%
\pgfpathcurveto{\pgfqpoint{5.733853in}{3.908925in}}{\pgfqpoint{5.738243in}{3.919525in}}{\pgfqpoint{5.738243in}{3.930575in}}%
\pgfpathcurveto{\pgfqpoint{5.738243in}{3.941625in}}{\pgfqpoint{5.733853in}{3.952224in}}{\pgfqpoint{5.726039in}{3.960037in}}%
\pgfpathcurveto{\pgfqpoint{5.718226in}{3.967851in}}{\pgfqpoint{5.707627in}{3.972241in}}{\pgfqpoint{5.696576in}{3.972241in}}%
\pgfpathcurveto{\pgfqpoint{5.685526in}{3.972241in}}{\pgfqpoint{5.674927in}{3.967851in}}{\pgfqpoint{5.667114in}{3.960037in}}%
\pgfpathcurveto{\pgfqpoint{5.659300in}{3.952224in}}{\pgfqpoint{5.654910in}{3.941625in}}{\pgfqpoint{5.654910in}{3.930575in}}%
\pgfpathcurveto{\pgfqpoint{5.654910in}{3.919525in}}{\pgfqpoint{5.659300in}{3.908925in}}{\pgfqpoint{5.667114in}{3.901112in}}%
\pgfpathcurveto{\pgfqpoint{5.674927in}{3.893298in}}{\pgfqpoint{5.685526in}{3.888908in}}{\pgfqpoint{5.696576in}{3.888908in}}%
\pgfpathclose%
\pgfusepath{stroke,fill}%
\end{pgfscope}%
\begin{pgfscope}%
\pgfpathrectangle{\pgfqpoint{0.393613in}{0.331635in}}{\pgfqpoint{9.300000in}{7.700000in}}%
\pgfusepath{clip}%
\pgfsetbuttcap%
\pgfsetroundjoin%
\definecolor{currentfill}{rgb}{1.000000,0.705882,0.509804}%
\pgfsetfillcolor{currentfill}%
\pgfsetlinewidth{0.481800pt}%
\definecolor{currentstroke}{rgb}{1.000000,1.000000,1.000000}%
\pgfsetstrokecolor{currentstroke}%
\pgfsetdash{}{0pt}%
\pgfpathmoveto{\pgfqpoint{4.929999in}{4.103506in}}%
\pgfpathcurveto{\pgfqpoint{4.941049in}{4.103506in}}{\pgfqpoint{4.951648in}{4.107896in}}{\pgfqpoint{4.959462in}{4.115710in}}%
\pgfpathcurveto{\pgfqpoint{4.967275in}{4.123524in}}{\pgfqpoint{4.971666in}{4.134123in}}{\pgfqpoint{4.971666in}{4.145173in}}%
\pgfpathcurveto{\pgfqpoint{4.971666in}{4.156223in}}{\pgfqpoint{4.967275in}{4.166822in}}{\pgfqpoint{4.959462in}{4.174636in}}%
\pgfpathcurveto{\pgfqpoint{4.951648in}{4.182449in}}{\pgfqpoint{4.941049in}{4.186840in}}{\pgfqpoint{4.929999in}{4.186840in}}%
\pgfpathcurveto{\pgfqpoint{4.918949in}{4.186840in}}{\pgfqpoint{4.908350in}{4.182449in}}{\pgfqpoint{4.900536in}{4.174636in}}%
\pgfpathcurveto{\pgfqpoint{4.892722in}{4.166822in}}{\pgfqpoint{4.888332in}{4.156223in}}{\pgfqpoint{4.888332in}{4.145173in}}%
\pgfpathcurveto{\pgfqpoint{4.888332in}{4.134123in}}{\pgfqpoint{4.892722in}{4.123524in}}{\pgfqpoint{4.900536in}{4.115710in}}%
\pgfpathcurveto{\pgfqpoint{4.908350in}{4.107896in}}{\pgfqpoint{4.918949in}{4.103506in}}{\pgfqpoint{4.929999in}{4.103506in}}%
\pgfpathclose%
\pgfusepath{stroke,fill}%
\end{pgfscope}%
\begin{pgfscope}%
\pgfpathrectangle{\pgfqpoint{0.393613in}{0.331635in}}{\pgfqpoint{9.300000in}{7.700000in}}%
\pgfusepath{clip}%
\pgfsetbuttcap%
\pgfsetroundjoin%
\definecolor{currentfill}{rgb}{1.000000,0.705882,0.509804}%
\pgfsetfillcolor{currentfill}%
\pgfsetlinewidth{0.481800pt}%
\definecolor{currentstroke}{rgb}{1.000000,1.000000,1.000000}%
\pgfsetstrokecolor{currentstroke}%
\pgfsetdash{}{0pt}%
\pgfpathmoveto{\pgfqpoint{4.410117in}{4.593834in}}%
\pgfpathcurveto{\pgfqpoint{4.421167in}{4.593834in}}{\pgfqpoint{4.431766in}{4.598224in}}{\pgfqpoint{4.439580in}{4.606038in}}%
\pgfpathcurveto{\pgfqpoint{4.447393in}{4.613851in}}{\pgfqpoint{4.451784in}{4.624450in}}{\pgfqpoint{4.451784in}{4.635500in}}%
\pgfpathcurveto{\pgfqpoint{4.451784in}{4.646551in}}{\pgfqpoint{4.447393in}{4.657150in}}{\pgfqpoint{4.439580in}{4.664963in}}%
\pgfpathcurveto{\pgfqpoint{4.431766in}{4.672777in}}{\pgfqpoint{4.421167in}{4.677167in}}{\pgfqpoint{4.410117in}{4.677167in}}%
\pgfpathcurveto{\pgfqpoint{4.399067in}{4.677167in}}{\pgfqpoint{4.388468in}{4.672777in}}{\pgfqpoint{4.380654in}{4.664963in}}%
\pgfpathcurveto{\pgfqpoint{4.372841in}{4.657150in}}{\pgfqpoint{4.368450in}{4.646551in}}{\pgfqpoint{4.368450in}{4.635500in}}%
\pgfpathcurveto{\pgfqpoint{4.368450in}{4.624450in}}{\pgfqpoint{4.372841in}{4.613851in}}{\pgfqpoint{4.380654in}{4.606038in}}%
\pgfpathcurveto{\pgfqpoint{4.388468in}{4.598224in}}{\pgfqpoint{4.399067in}{4.593834in}}{\pgfqpoint{4.410117in}{4.593834in}}%
\pgfpathclose%
\pgfusepath{stroke,fill}%
\end{pgfscope}%
\begin{pgfscope}%
\pgfpathrectangle{\pgfqpoint{0.393613in}{0.331635in}}{\pgfqpoint{9.300000in}{7.700000in}}%
\pgfusepath{clip}%
\pgfsetbuttcap%
\pgfsetroundjoin%
\definecolor{currentfill}{rgb}{1.000000,0.705882,0.509804}%
\pgfsetfillcolor{currentfill}%
\pgfsetlinewidth{0.481800pt}%
\definecolor{currentstroke}{rgb}{1.000000,1.000000,1.000000}%
\pgfsetstrokecolor{currentstroke}%
\pgfsetdash{}{0pt}%
\pgfpathmoveto{\pgfqpoint{8.226679in}{3.197436in}}%
\pgfpathcurveto{\pgfqpoint{8.237730in}{3.197436in}}{\pgfqpoint{8.248329in}{3.201826in}}{\pgfqpoint{8.256142in}{3.209640in}}%
\pgfpathcurveto{\pgfqpoint{8.263956in}{3.217453in}}{\pgfqpoint{8.268346in}{3.228052in}}{\pgfqpoint{8.268346in}{3.239102in}}%
\pgfpathcurveto{\pgfqpoint{8.268346in}{3.250153in}}{\pgfqpoint{8.263956in}{3.260752in}}{\pgfqpoint{8.256142in}{3.268565in}}%
\pgfpathcurveto{\pgfqpoint{8.248329in}{3.276379in}}{\pgfqpoint{8.237730in}{3.280769in}}{\pgfqpoint{8.226679in}{3.280769in}}%
\pgfpathcurveto{\pgfqpoint{8.215629in}{3.280769in}}{\pgfqpoint{8.205030in}{3.276379in}}{\pgfqpoint{8.197217in}{3.268565in}}%
\pgfpathcurveto{\pgfqpoint{8.189403in}{3.260752in}}{\pgfqpoint{8.185013in}{3.250153in}}{\pgfqpoint{8.185013in}{3.239102in}}%
\pgfpathcurveto{\pgfqpoint{8.185013in}{3.228052in}}{\pgfqpoint{8.189403in}{3.217453in}}{\pgfqpoint{8.197217in}{3.209640in}}%
\pgfpathcurveto{\pgfqpoint{8.205030in}{3.201826in}}{\pgfqpoint{8.215629in}{3.197436in}}{\pgfqpoint{8.226679in}{3.197436in}}%
\pgfpathclose%
\pgfusepath{stroke,fill}%
\end{pgfscope}%
\begin{pgfscope}%
\pgfpathrectangle{\pgfqpoint{0.393613in}{0.331635in}}{\pgfqpoint{9.300000in}{7.700000in}}%
\pgfusepath{clip}%
\pgfsetbuttcap%
\pgfsetroundjoin%
\definecolor{currentfill}{rgb}{1.000000,0.705882,0.509804}%
\pgfsetfillcolor{currentfill}%
\pgfsetlinewidth{0.481800pt}%
\definecolor{currentstroke}{rgb}{1.000000,1.000000,1.000000}%
\pgfsetstrokecolor{currentstroke}%
\pgfsetdash{}{0pt}%
\pgfpathmoveto{\pgfqpoint{8.462150in}{4.732949in}}%
\pgfpathcurveto{\pgfqpoint{8.473200in}{4.732949in}}{\pgfqpoint{8.483799in}{4.737339in}}{\pgfqpoint{8.491612in}{4.745152in}}%
\pgfpathcurveto{\pgfqpoint{8.499426in}{4.752966in}}{\pgfqpoint{8.503816in}{4.763565in}}{\pgfqpoint{8.503816in}{4.774615in}}%
\pgfpathcurveto{\pgfqpoint{8.503816in}{4.785665in}}{\pgfqpoint{8.499426in}{4.796264in}}{\pgfqpoint{8.491612in}{4.804078in}}%
\pgfpathcurveto{\pgfqpoint{8.483799in}{4.811892in}}{\pgfqpoint{8.473200in}{4.816282in}}{\pgfqpoint{8.462150in}{4.816282in}}%
\pgfpathcurveto{\pgfqpoint{8.451099in}{4.816282in}}{\pgfqpoint{8.440500in}{4.811892in}}{\pgfqpoint{8.432687in}{4.804078in}}%
\pgfpathcurveto{\pgfqpoint{8.424873in}{4.796264in}}{\pgfqpoint{8.420483in}{4.785665in}}{\pgfqpoint{8.420483in}{4.774615in}}%
\pgfpathcurveto{\pgfqpoint{8.420483in}{4.763565in}}{\pgfqpoint{8.424873in}{4.752966in}}{\pgfqpoint{8.432687in}{4.745152in}}%
\pgfpathcurveto{\pgfqpoint{8.440500in}{4.737339in}}{\pgfqpoint{8.451099in}{4.732949in}}{\pgfqpoint{8.462150in}{4.732949in}}%
\pgfpathclose%
\pgfusepath{stroke,fill}%
\end{pgfscope}%
\begin{pgfscope}%
\pgfpathrectangle{\pgfqpoint{0.393613in}{0.331635in}}{\pgfqpoint{9.300000in}{7.700000in}}%
\pgfusepath{clip}%
\pgfsetbuttcap%
\pgfsetroundjoin%
\definecolor{currentfill}{rgb}{1.000000,0.705882,0.509804}%
\pgfsetfillcolor{currentfill}%
\pgfsetlinewidth{0.481800pt}%
\definecolor{currentstroke}{rgb}{1.000000,1.000000,1.000000}%
\pgfsetstrokecolor{currentstroke}%
\pgfsetdash{}{0pt}%
\pgfpathmoveto{\pgfqpoint{5.762899in}{4.672439in}}%
\pgfpathcurveto{\pgfqpoint{5.773949in}{4.672439in}}{\pgfqpoint{5.784548in}{4.676829in}}{\pgfqpoint{5.792362in}{4.684643in}}%
\pgfpathcurveto{\pgfqpoint{5.800176in}{4.692456in}}{\pgfqpoint{5.804566in}{4.703055in}}{\pgfqpoint{5.804566in}{4.714105in}}%
\pgfpathcurveto{\pgfqpoint{5.804566in}{4.725156in}}{\pgfqpoint{5.800176in}{4.735755in}}{\pgfqpoint{5.792362in}{4.743568in}}%
\pgfpathcurveto{\pgfqpoint{5.784548in}{4.751382in}}{\pgfqpoint{5.773949in}{4.755772in}}{\pgfqpoint{5.762899in}{4.755772in}}%
\pgfpathcurveto{\pgfqpoint{5.751849in}{4.755772in}}{\pgfqpoint{5.741250in}{4.751382in}}{\pgfqpoint{5.733436in}{4.743568in}}%
\pgfpathcurveto{\pgfqpoint{5.725623in}{4.735755in}}{\pgfqpoint{5.721233in}{4.725156in}}{\pgfqpoint{5.721233in}{4.714105in}}%
\pgfpathcurveto{\pgfqpoint{5.721233in}{4.703055in}}{\pgfqpoint{5.725623in}{4.692456in}}{\pgfqpoint{5.733436in}{4.684643in}}%
\pgfpathcurveto{\pgfqpoint{5.741250in}{4.676829in}}{\pgfqpoint{5.751849in}{4.672439in}}{\pgfqpoint{5.762899in}{4.672439in}}%
\pgfpathclose%
\pgfusepath{stroke,fill}%
\end{pgfscope}%
\begin{pgfscope}%
\pgfpathrectangle{\pgfqpoint{0.393613in}{0.331635in}}{\pgfqpoint{9.300000in}{7.700000in}}%
\pgfusepath{clip}%
\pgfsetbuttcap%
\pgfsetroundjoin%
\definecolor{currentfill}{rgb}{1.000000,0.705882,0.509804}%
\pgfsetfillcolor{currentfill}%
\pgfsetlinewidth{0.481800pt}%
\definecolor{currentstroke}{rgb}{1.000000,1.000000,1.000000}%
\pgfsetstrokecolor{currentstroke}%
\pgfsetdash{}{0pt}%
\pgfpathmoveto{\pgfqpoint{8.593098in}{3.906325in}}%
\pgfpathcurveto{\pgfqpoint{8.604148in}{3.906325in}}{\pgfqpoint{8.614747in}{3.910715in}}{\pgfqpoint{8.622561in}{3.918529in}}%
\pgfpathcurveto{\pgfqpoint{8.630374in}{3.926342in}}{\pgfqpoint{8.634765in}{3.936941in}}{\pgfqpoint{8.634765in}{3.947991in}}%
\pgfpathcurveto{\pgfqpoint{8.634765in}{3.959042in}}{\pgfqpoint{8.630374in}{3.969641in}}{\pgfqpoint{8.622561in}{3.977454in}}%
\pgfpathcurveto{\pgfqpoint{8.614747in}{3.985268in}}{\pgfqpoint{8.604148in}{3.989658in}}{\pgfqpoint{8.593098in}{3.989658in}}%
\pgfpathcurveto{\pgfqpoint{8.582048in}{3.989658in}}{\pgfqpoint{8.571449in}{3.985268in}}{\pgfqpoint{8.563635in}{3.977454in}}%
\pgfpathcurveto{\pgfqpoint{8.555822in}{3.969641in}}{\pgfqpoint{8.551431in}{3.959042in}}{\pgfqpoint{8.551431in}{3.947991in}}%
\pgfpathcurveto{\pgfqpoint{8.551431in}{3.936941in}}{\pgfqpoint{8.555822in}{3.926342in}}{\pgfqpoint{8.563635in}{3.918529in}}%
\pgfpathcurveto{\pgfqpoint{8.571449in}{3.910715in}}{\pgfqpoint{8.582048in}{3.906325in}}{\pgfqpoint{8.593098in}{3.906325in}}%
\pgfpathclose%
\pgfusepath{stroke,fill}%
\end{pgfscope}%
\begin{pgfscope}%
\pgfpathrectangle{\pgfqpoint{0.393613in}{0.331635in}}{\pgfqpoint{9.300000in}{7.700000in}}%
\pgfusepath{clip}%
\pgfsetbuttcap%
\pgfsetroundjoin%
\definecolor{currentfill}{rgb}{1.000000,0.705882,0.509804}%
\pgfsetfillcolor{currentfill}%
\pgfsetlinewidth{0.481800pt}%
\definecolor{currentstroke}{rgb}{1.000000,1.000000,1.000000}%
\pgfsetstrokecolor{currentstroke}%
\pgfsetdash{}{0pt}%
\pgfpathmoveto{\pgfqpoint{5.266097in}{2.274799in}}%
\pgfpathcurveto{\pgfqpoint{5.277148in}{2.274799in}}{\pgfqpoint{5.287747in}{2.279189in}}{\pgfqpoint{5.295560in}{2.287003in}}%
\pgfpathcurveto{\pgfqpoint{5.303374in}{2.294816in}}{\pgfqpoint{5.307764in}{2.305415in}}{\pgfqpoint{5.307764in}{2.316466in}}%
\pgfpathcurveto{\pgfqpoint{5.307764in}{2.327516in}}{\pgfqpoint{5.303374in}{2.338115in}}{\pgfqpoint{5.295560in}{2.345928in}}%
\pgfpathcurveto{\pgfqpoint{5.287747in}{2.353742in}}{\pgfqpoint{5.277148in}{2.358132in}}{\pgfqpoint{5.266097in}{2.358132in}}%
\pgfpathcurveto{\pgfqpoint{5.255047in}{2.358132in}}{\pgfqpoint{5.244448in}{2.353742in}}{\pgfqpoint{5.236635in}{2.345928in}}%
\pgfpathcurveto{\pgfqpoint{5.228821in}{2.338115in}}{\pgfqpoint{5.224431in}{2.327516in}}{\pgfqpoint{5.224431in}{2.316466in}}%
\pgfpathcurveto{\pgfqpoint{5.224431in}{2.305415in}}{\pgfqpoint{5.228821in}{2.294816in}}{\pgfqpoint{5.236635in}{2.287003in}}%
\pgfpathcurveto{\pgfqpoint{5.244448in}{2.279189in}}{\pgfqpoint{5.255047in}{2.274799in}}{\pgfqpoint{5.266097in}{2.274799in}}%
\pgfpathclose%
\pgfusepath{stroke,fill}%
\end{pgfscope}%
\begin{pgfscope}%
\pgfpathrectangle{\pgfqpoint{0.393613in}{0.331635in}}{\pgfqpoint{9.300000in}{7.700000in}}%
\pgfusepath{clip}%
\pgfsetbuttcap%
\pgfsetroundjoin%
\definecolor{currentfill}{rgb}{1.000000,0.705882,0.509804}%
\pgfsetfillcolor{currentfill}%
\pgfsetlinewidth{0.481800pt}%
\definecolor{currentstroke}{rgb}{1.000000,1.000000,1.000000}%
\pgfsetstrokecolor{currentstroke}%
\pgfsetdash{}{0pt}%
\pgfpathmoveto{\pgfqpoint{7.489133in}{2.337951in}}%
\pgfpathcurveto{\pgfqpoint{7.500183in}{2.337951in}}{\pgfqpoint{7.510782in}{2.342341in}}{\pgfqpoint{7.518596in}{2.350155in}}%
\pgfpathcurveto{\pgfqpoint{7.526410in}{2.357968in}}{\pgfqpoint{7.530800in}{2.368567in}}{\pgfqpoint{7.530800in}{2.379617in}}%
\pgfpathcurveto{\pgfqpoint{7.530800in}{2.390668in}}{\pgfqpoint{7.526410in}{2.401267in}}{\pgfqpoint{7.518596in}{2.409080in}}%
\pgfpathcurveto{\pgfqpoint{7.510782in}{2.416894in}}{\pgfqpoint{7.500183in}{2.421284in}}{\pgfqpoint{7.489133in}{2.421284in}}%
\pgfpathcurveto{\pgfqpoint{7.478083in}{2.421284in}}{\pgfqpoint{7.467484in}{2.416894in}}{\pgfqpoint{7.459670in}{2.409080in}}%
\pgfpathcurveto{\pgfqpoint{7.451857in}{2.401267in}}{\pgfqpoint{7.447467in}{2.390668in}}{\pgfqpoint{7.447467in}{2.379617in}}%
\pgfpathcurveto{\pgfqpoint{7.447467in}{2.368567in}}{\pgfqpoint{7.451857in}{2.357968in}}{\pgfqpoint{7.459670in}{2.350155in}}%
\pgfpathcurveto{\pgfqpoint{7.467484in}{2.342341in}}{\pgfqpoint{7.478083in}{2.337951in}}{\pgfqpoint{7.489133in}{2.337951in}}%
\pgfpathclose%
\pgfusepath{stroke,fill}%
\end{pgfscope}%
\begin{pgfscope}%
\pgfpathrectangle{\pgfqpoint{0.393613in}{0.331635in}}{\pgfqpoint{9.300000in}{7.700000in}}%
\pgfusepath{clip}%
\pgfsetbuttcap%
\pgfsetroundjoin%
\definecolor{currentfill}{rgb}{1.000000,0.705882,0.509804}%
\pgfsetfillcolor{currentfill}%
\pgfsetlinewidth{0.481800pt}%
\definecolor{currentstroke}{rgb}{1.000000,1.000000,1.000000}%
\pgfsetstrokecolor{currentstroke}%
\pgfsetdash{}{0pt}%
\pgfpathmoveto{\pgfqpoint{7.375591in}{4.687021in}}%
\pgfpathcurveto{\pgfqpoint{7.386641in}{4.687021in}}{\pgfqpoint{7.397240in}{4.691411in}}{\pgfqpoint{7.405054in}{4.699225in}}%
\pgfpathcurveto{\pgfqpoint{7.412868in}{4.707038in}}{\pgfqpoint{7.417258in}{4.717637in}}{\pgfqpoint{7.417258in}{4.728687in}}%
\pgfpathcurveto{\pgfqpoint{7.417258in}{4.739737in}}{\pgfqpoint{7.412868in}{4.750337in}}{\pgfqpoint{7.405054in}{4.758150in}}%
\pgfpathcurveto{\pgfqpoint{7.397240in}{4.765964in}}{\pgfqpoint{7.386641in}{4.770354in}}{\pgfqpoint{7.375591in}{4.770354in}}%
\pgfpathcurveto{\pgfqpoint{7.364541in}{4.770354in}}{\pgfqpoint{7.353942in}{4.765964in}}{\pgfqpoint{7.346128in}{4.758150in}}%
\pgfpathcurveto{\pgfqpoint{7.338315in}{4.750337in}}{\pgfqpoint{7.333924in}{4.739737in}}{\pgfqpoint{7.333924in}{4.728687in}}%
\pgfpathcurveto{\pgfqpoint{7.333924in}{4.717637in}}{\pgfqpoint{7.338315in}{4.707038in}}{\pgfqpoint{7.346128in}{4.699225in}}%
\pgfpathcurveto{\pgfqpoint{7.353942in}{4.691411in}}{\pgfqpoint{7.364541in}{4.687021in}}{\pgfqpoint{7.375591in}{4.687021in}}%
\pgfpathclose%
\pgfusepath{stroke,fill}%
\end{pgfscope}%
\begin{pgfscope}%
\pgfpathrectangle{\pgfqpoint{0.393613in}{0.331635in}}{\pgfqpoint{9.300000in}{7.700000in}}%
\pgfusepath{clip}%
\pgfsetbuttcap%
\pgfsetroundjoin%
\definecolor{currentfill}{rgb}{1.000000,0.705882,0.509804}%
\pgfsetfillcolor{currentfill}%
\pgfsetlinewidth{0.481800pt}%
\definecolor{currentstroke}{rgb}{1.000000,1.000000,1.000000}%
\pgfsetstrokecolor{currentstroke}%
\pgfsetdash{}{0pt}%
\pgfpathmoveto{\pgfqpoint{5.281132in}{3.971102in}}%
\pgfpathcurveto{\pgfqpoint{5.292182in}{3.971102in}}{\pgfqpoint{5.302781in}{3.975492in}}{\pgfqpoint{5.310594in}{3.983305in}}%
\pgfpathcurveto{\pgfqpoint{5.318408in}{3.991119in}}{\pgfqpoint{5.322798in}{4.001718in}}{\pgfqpoint{5.322798in}{4.012768in}}%
\pgfpathcurveto{\pgfqpoint{5.322798in}{4.023818in}}{\pgfqpoint{5.318408in}{4.034417in}}{\pgfqpoint{5.310594in}{4.042231in}}%
\pgfpathcurveto{\pgfqpoint{5.302781in}{4.050045in}}{\pgfqpoint{5.292182in}{4.054435in}}{\pgfqpoint{5.281132in}{4.054435in}}%
\pgfpathcurveto{\pgfqpoint{5.270082in}{4.054435in}}{\pgfqpoint{5.259483in}{4.050045in}}{\pgfqpoint{5.251669in}{4.042231in}}%
\pgfpathcurveto{\pgfqpoint{5.243855in}{4.034417in}}{\pgfqpoint{5.239465in}{4.023818in}}{\pgfqpoint{5.239465in}{4.012768in}}%
\pgfpathcurveto{\pgfqpoint{5.239465in}{4.001718in}}{\pgfqpoint{5.243855in}{3.991119in}}{\pgfqpoint{5.251669in}{3.983305in}}%
\pgfpathcurveto{\pgfqpoint{5.259483in}{3.975492in}}{\pgfqpoint{5.270082in}{3.971102in}}{\pgfqpoint{5.281132in}{3.971102in}}%
\pgfpathclose%
\pgfusepath{stroke,fill}%
\end{pgfscope}%
\begin{pgfscope}%
\pgfpathrectangle{\pgfqpoint{0.393613in}{0.331635in}}{\pgfqpoint{9.300000in}{7.700000in}}%
\pgfusepath{clip}%
\pgfsetbuttcap%
\pgfsetroundjoin%
\definecolor{currentfill}{rgb}{1.000000,0.705882,0.509804}%
\pgfsetfillcolor{currentfill}%
\pgfsetlinewidth{0.481800pt}%
\definecolor{currentstroke}{rgb}{1.000000,1.000000,1.000000}%
\pgfsetstrokecolor{currentstroke}%
\pgfsetdash{}{0pt}%
\pgfpathmoveto{\pgfqpoint{8.775963in}{3.764502in}}%
\pgfpathcurveto{\pgfqpoint{8.787013in}{3.764502in}}{\pgfqpoint{8.797612in}{3.768892in}}{\pgfqpoint{8.805425in}{3.776706in}}%
\pgfpathcurveto{\pgfqpoint{8.813239in}{3.784519in}}{\pgfqpoint{8.817629in}{3.795118in}}{\pgfqpoint{8.817629in}{3.806168in}}%
\pgfpathcurveto{\pgfqpoint{8.817629in}{3.817219in}}{\pgfqpoint{8.813239in}{3.827818in}}{\pgfqpoint{8.805425in}{3.835631in}}%
\pgfpathcurveto{\pgfqpoint{8.797612in}{3.843445in}}{\pgfqpoint{8.787013in}{3.847835in}}{\pgfqpoint{8.775963in}{3.847835in}}%
\pgfpathcurveto{\pgfqpoint{8.764912in}{3.847835in}}{\pgfqpoint{8.754313in}{3.843445in}}{\pgfqpoint{8.746500in}{3.835631in}}%
\pgfpathcurveto{\pgfqpoint{8.738686in}{3.827818in}}{\pgfqpoint{8.734296in}{3.817219in}}{\pgfqpoint{8.734296in}{3.806168in}}%
\pgfpathcurveto{\pgfqpoint{8.734296in}{3.795118in}}{\pgfqpoint{8.738686in}{3.784519in}}{\pgfqpoint{8.746500in}{3.776706in}}%
\pgfpathcurveto{\pgfqpoint{8.754313in}{3.768892in}}{\pgfqpoint{8.764912in}{3.764502in}}{\pgfqpoint{8.775963in}{3.764502in}}%
\pgfpathclose%
\pgfusepath{stroke,fill}%
\end{pgfscope}%
\begin{pgfscope}%
\pgfpathrectangle{\pgfqpoint{0.393613in}{0.331635in}}{\pgfqpoint{9.300000in}{7.700000in}}%
\pgfusepath{clip}%
\pgfsetbuttcap%
\pgfsetroundjoin%
\definecolor{currentfill}{rgb}{1.000000,0.705882,0.509804}%
\pgfsetfillcolor{currentfill}%
\pgfsetlinewidth{0.481800pt}%
\definecolor{currentstroke}{rgb}{1.000000,1.000000,1.000000}%
\pgfsetstrokecolor{currentstroke}%
\pgfsetdash{}{0pt}%
\pgfpathmoveto{\pgfqpoint{9.270885in}{4.190667in}}%
\pgfpathcurveto{\pgfqpoint{9.281935in}{4.190667in}}{\pgfqpoint{9.292534in}{4.195057in}}{\pgfqpoint{9.300348in}{4.202871in}}%
\pgfpathcurveto{\pgfqpoint{9.308162in}{4.210684in}}{\pgfqpoint{9.312552in}{4.221283in}}{\pgfqpoint{9.312552in}{4.232334in}}%
\pgfpathcurveto{\pgfqpoint{9.312552in}{4.243384in}}{\pgfqpoint{9.308162in}{4.253983in}}{\pgfqpoint{9.300348in}{4.261796in}}%
\pgfpathcurveto{\pgfqpoint{9.292534in}{4.269610in}}{\pgfqpoint{9.281935in}{4.274000in}}{\pgfqpoint{9.270885in}{4.274000in}}%
\pgfpathcurveto{\pgfqpoint{9.259835in}{4.274000in}}{\pgfqpoint{9.249236in}{4.269610in}}{\pgfqpoint{9.241422in}{4.261796in}}%
\pgfpathcurveto{\pgfqpoint{9.233609in}{4.253983in}}{\pgfqpoint{9.229219in}{4.243384in}}{\pgfqpoint{9.229219in}{4.232334in}}%
\pgfpathcurveto{\pgfqpoint{9.229219in}{4.221283in}}{\pgfqpoint{9.233609in}{4.210684in}}{\pgfqpoint{9.241422in}{4.202871in}}%
\pgfpathcurveto{\pgfqpoint{9.249236in}{4.195057in}}{\pgfqpoint{9.259835in}{4.190667in}}{\pgfqpoint{9.270885in}{4.190667in}}%
\pgfpathclose%
\pgfusepath{stroke,fill}%
\end{pgfscope}%
\begin{pgfscope}%
\pgfpathrectangle{\pgfqpoint{0.393613in}{0.331635in}}{\pgfqpoint{9.300000in}{7.700000in}}%
\pgfusepath{clip}%
\pgfsetbuttcap%
\pgfsetroundjoin%
\definecolor{currentfill}{rgb}{1.000000,0.705882,0.509804}%
\pgfsetfillcolor{currentfill}%
\pgfsetlinewidth{0.481800pt}%
\definecolor{currentstroke}{rgb}{1.000000,1.000000,1.000000}%
\pgfsetstrokecolor{currentstroke}%
\pgfsetdash{}{0pt}%
\pgfpathmoveto{\pgfqpoint{5.221194in}{3.827244in}}%
\pgfpathcurveto{\pgfqpoint{5.232244in}{3.827244in}}{\pgfqpoint{5.242843in}{3.831635in}}{\pgfqpoint{5.250657in}{3.839448in}}%
\pgfpathcurveto{\pgfqpoint{5.258470in}{3.847262in}}{\pgfqpoint{5.262861in}{3.857861in}}{\pgfqpoint{5.262861in}{3.868911in}}%
\pgfpathcurveto{\pgfqpoint{5.262861in}{3.879961in}}{\pgfqpoint{5.258470in}{3.890560in}}{\pgfqpoint{5.250657in}{3.898374in}}%
\pgfpathcurveto{\pgfqpoint{5.242843in}{3.906187in}}{\pgfqpoint{5.232244in}{3.910578in}}{\pgfqpoint{5.221194in}{3.910578in}}%
\pgfpathcurveto{\pgfqpoint{5.210144in}{3.910578in}}{\pgfqpoint{5.199545in}{3.906187in}}{\pgfqpoint{5.191731in}{3.898374in}}%
\pgfpathcurveto{\pgfqpoint{5.183917in}{3.890560in}}{\pgfqpoint{5.179527in}{3.879961in}}{\pgfqpoint{5.179527in}{3.868911in}}%
\pgfpathcurveto{\pgfqpoint{5.179527in}{3.857861in}}{\pgfqpoint{5.183917in}{3.847262in}}{\pgfqpoint{5.191731in}{3.839448in}}%
\pgfpathcurveto{\pgfqpoint{5.199545in}{3.831635in}}{\pgfqpoint{5.210144in}{3.827244in}}{\pgfqpoint{5.221194in}{3.827244in}}%
\pgfpathclose%
\pgfusepath{stroke,fill}%
\end{pgfscope}%
\begin{pgfscope}%
\pgfpathrectangle{\pgfqpoint{0.393613in}{0.331635in}}{\pgfqpoint{9.300000in}{7.700000in}}%
\pgfusepath{clip}%
\pgfsetbuttcap%
\pgfsetroundjoin%
\definecolor{currentfill}{rgb}{1.000000,0.705882,0.509804}%
\pgfsetfillcolor{currentfill}%
\pgfsetlinewidth{0.481800pt}%
\definecolor{currentstroke}{rgb}{1.000000,1.000000,1.000000}%
\pgfsetstrokecolor{currentstroke}%
\pgfsetdash{}{0pt}%
\pgfpathmoveto{\pgfqpoint{4.726563in}{5.063919in}}%
\pgfpathcurveto{\pgfqpoint{4.737613in}{5.063919in}}{\pgfqpoint{4.748212in}{5.068309in}}{\pgfqpoint{4.756026in}{5.076123in}}%
\pgfpathcurveto{\pgfqpoint{4.763839in}{5.083936in}}{\pgfqpoint{4.768230in}{5.094535in}}{\pgfqpoint{4.768230in}{5.105585in}}%
\pgfpathcurveto{\pgfqpoint{4.768230in}{5.116636in}}{\pgfqpoint{4.763839in}{5.127235in}}{\pgfqpoint{4.756026in}{5.135048in}}%
\pgfpathcurveto{\pgfqpoint{4.748212in}{5.142862in}}{\pgfqpoint{4.737613in}{5.147252in}}{\pgfqpoint{4.726563in}{5.147252in}}%
\pgfpathcurveto{\pgfqpoint{4.715513in}{5.147252in}}{\pgfqpoint{4.704914in}{5.142862in}}{\pgfqpoint{4.697100in}{5.135048in}}%
\pgfpathcurveto{\pgfqpoint{4.689286in}{5.127235in}}{\pgfqpoint{4.684896in}{5.116636in}}{\pgfqpoint{4.684896in}{5.105585in}}%
\pgfpathcurveto{\pgfqpoint{4.684896in}{5.094535in}}{\pgfqpoint{4.689286in}{5.083936in}}{\pgfqpoint{4.697100in}{5.076123in}}%
\pgfpathcurveto{\pgfqpoint{4.704914in}{5.068309in}}{\pgfqpoint{4.715513in}{5.063919in}}{\pgfqpoint{4.726563in}{5.063919in}}%
\pgfpathclose%
\pgfusepath{stroke,fill}%
\end{pgfscope}%
\begin{pgfscope}%
\pgfpathrectangle{\pgfqpoint{0.393613in}{0.331635in}}{\pgfqpoint{9.300000in}{7.700000in}}%
\pgfusepath{clip}%
\pgfsetbuttcap%
\pgfsetroundjoin%
\definecolor{currentfill}{rgb}{1.000000,0.705882,0.509804}%
\pgfsetfillcolor{currentfill}%
\pgfsetlinewidth{0.481800pt}%
\definecolor{currentstroke}{rgb}{1.000000,1.000000,1.000000}%
\pgfsetstrokecolor{currentstroke}%
\pgfsetdash{}{0pt}%
\pgfpathmoveto{\pgfqpoint{5.210620in}{2.259081in}}%
\pgfpathcurveto{\pgfqpoint{5.221670in}{2.259081in}}{\pgfqpoint{5.232269in}{2.263471in}}{\pgfqpoint{5.240083in}{2.271285in}}%
\pgfpathcurveto{\pgfqpoint{5.247897in}{2.279098in}}{\pgfqpoint{5.252287in}{2.289698in}}{\pgfqpoint{5.252287in}{2.300748in}}%
\pgfpathcurveto{\pgfqpoint{5.252287in}{2.311798in}}{\pgfqpoint{5.247897in}{2.322397in}}{\pgfqpoint{5.240083in}{2.330210in}}%
\pgfpathcurveto{\pgfqpoint{5.232269in}{2.338024in}}{\pgfqpoint{5.221670in}{2.342414in}}{\pgfqpoint{5.210620in}{2.342414in}}%
\pgfpathcurveto{\pgfqpoint{5.199570in}{2.342414in}}{\pgfqpoint{5.188971in}{2.338024in}}{\pgfqpoint{5.181158in}{2.330210in}}%
\pgfpathcurveto{\pgfqpoint{5.173344in}{2.322397in}}{\pgfqpoint{5.168954in}{2.311798in}}{\pgfqpoint{5.168954in}{2.300748in}}%
\pgfpathcurveto{\pgfqpoint{5.168954in}{2.289698in}}{\pgfqpoint{5.173344in}{2.279098in}}{\pgfqpoint{5.181158in}{2.271285in}}%
\pgfpathcurveto{\pgfqpoint{5.188971in}{2.263471in}}{\pgfqpoint{5.199570in}{2.259081in}}{\pgfqpoint{5.210620in}{2.259081in}}%
\pgfpathclose%
\pgfusepath{stroke,fill}%
\end{pgfscope}%
\begin{pgfscope}%
\pgfpathrectangle{\pgfqpoint{0.393613in}{0.331635in}}{\pgfqpoint{9.300000in}{7.700000in}}%
\pgfusepath{clip}%
\pgfsetbuttcap%
\pgfsetroundjoin%
\definecolor{currentfill}{rgb}{1.000000,0.705882,0.509804}%
\pgfsetfillcolor{currentfill}%
\pgfsetlinewidth{0.481800pt}%
\definecolor{currentstroke}{rgb}{1.000000,1.000000,1.000000}%
\pgfsetstrokecolor{currentstroke}%
\pgfsetdash{}{0pt}%
\pgfpathmoveto{\pgfqpoint{6.611736in}{2.781108in}}%
\pgfpathcurveto{\pgfqpoint{6.622786in}{2.781108in}}{\pgfqpoint{6.633385in}{2.785498in}}{\pgfqpoint{6.641199in}{2.793312in}}%
\pgfpathcurveto{\pgfqpoint{6.649013in}{2.801125in}}{\pgfqpoint{6.653403in}{2.811724in}}{\pgfqpoint{6.653403in}{2.822774in}}%
\pgfpathcurveto{\pgfqpoint{6.653403in}{2.833824in}}{\pgfqpoint{6.649013in}{2.844423in}}{\pgfqpoint{6.641199in}{2.852237in}}%
\pgfpathcurveto{\pgfqpoint{6.633385in}{2.860051in}}{\pgfqpoint{6.622786in}{2.864441in}}{\pgfqpoint{6.611736in}{2.864441in}}%
\pgfpathcurveto{\pgfqpoint{6.600686in}{2.864441in}}{\pgfqpoint{6.590087in}{2.860051in}}{\pgfqpoint{6.582273in}{2.852237in}}%
\pgfpathcurveto{\pgfqpoint{6.574460in}{2.844423in}}{\pgfqpoint{6.570069in}{2.833824in}}{\pgfqpoint{6.570069in}{2.822774in}}%
\pgfpathcurveto{\pgfqpoint{6.570069in}{2.811724in}}{\pgfqpoint{6.574460in}{2.801125in}}{\pgfqpoint{6.582273in}{2.793312in}}%
\pgfpathcurveto{\pgfqpoint{6.590087in}{2.785498in}}{\pgfqpoint{6.600686in}{2.781108in}}{\pgfqpoint{6.611736in}{2.781108in}}%
\pgfpathclose%
\pgfusepath{stroke,fill}%
\end{pgfscope}%
\begin{pgfscope}%
\pgfpathrectangle{\pgfqpoint{0.393613in}{0.331635in}}{\pgfqpoint{9.300000in}{7.700000in}}%
\pgfusepath{clip}%
\pgfsetbuttcap%
\pgfsetroundjoin%
\definecolor{currentfill}{rgb}{1.000000,0.705882,0.509804}%
\pgfsetfillcolor{currentfill}%
\pgfsetlinewidth{0.481800pt}%
\definecolor{currentstroke}{rgb}{1.000000,1.000000,1.000000}%
\pgfsetstrokecolor{currentstroke}%
\pgfsetdash{}{0pt}%
\pgfpathmoveto{\pgfqpoint{4.838508in}{5.640517in}}%
\pgfpathcurveto{\pgfqpoint{4.849558in}{5.640517in}}{\pgfqpoint{4.860157in}{5.644907in}}{\pgfqpoint{4.867971in}{5.652721in}}%
\pgfpathcurveto{\pgfqpoint{4.875784in}{5.660535in}}{\pgfqpoint{4.880175in}{5.671134in}}{\pgfqpoint{4.880175in}{5.682184in}}%
\pgfpathcurveto{\pgfqpoint{4.880175in}{5.693234in}}{\pgfqpoint{4.875784in}{5.703833in}}{\pgfqpoint{4.867971in}{5.711647in}}%
\pgfpathcurveto{\pgfqpoint{4.860157in}{5.719460in}}{\pgfqpoint{4.849558in}{5.723850in}}{\pgfqpoint{4.838508in}{5.723850in}}%
\pgfpathcurveto{\pgfqpoint{4.827458in}{5.723850in}}{\pgfqpoint{4.816859in}{5.719460in}}{\pgfqpoint{4.809045in}{5.711647in}}%
\pgfpathcurveto{\pgfqpoint{4.801232in}{5.703833in}}{\pgfqpoint{4.796841in}{5.693234in}}{\pgfqpoint{4.796841in}{5.682184in}}%
\pgfpathcurveto{\pgfqpoint{4.796841in}{5.671134in}}{\pgfqpoint{4.801232in}{5.660535in}}{\pgfqpoint{4.809045in}{5.652721in}}%
\pgfpathcurveto{\pgfqpoint{4.816859in}{5.644907in}}{\pgfqpoint{4.827458in}{5.640517in}}{\pgfqpoint{4.838508in}{5.640517in}}%
\pgfpathclose%
\pgfusepath{stroke,fill}%
\end{pgfscope}%
\begin{pgfscope}%
\pgfpathrectangle{\pgfqpoint{0.393613in}{0.331635in}}{\pgfqpoint{9.300000in}{7.700000in}}%
\pgfusepath{clip}%
\pgfsetbuttcap%
\pgfsetroundjoin%
\definecolor{currentfill}{rgb}{1.000000,0.705882,0.509804}%
\pgfsetfillcolor{currentfill}%
\pgfsetlinewidth{0.481800pt}%
\definecolor{currentstroke}{rgb}{1.000000,1.000000,1.000000}%
\pgfsetstrokecolor{currentstroke}%
\pgfsetdash{}{0pt}%
\pgfpathmoveto{\pgfqpoint{7.784989in}{5.637904in}}%
\pgfpathcurveto{\pgfqpoint{7.796039in}{5.637904in}}{\pgfqpoint{7.806638in}{5.642294in}}{\pgfqpoint{7.814452in}{5.650108in}}%
\pgfpathcurveto{\pgfqpoint{7.822265in}{5.657921in}}{\pgfqpoint{7.826656in}{5.668520in}}{\pgfqpoint{7.826656in}{5.679570in}}%
\pgfpathcurveto{\pgfqpoint{7.826656in}{5.690621in}}{\pgfqpoint{7.822265in}{5.701220in}}{\pgfqpoint{7.814452in}{5.709033in}}%
\pgfpathcurveto{\pgfqpoint{7.806638in}{5.716847in}}{\pgfqpoint{7.796039in}{5.721237in}}{\pgfqpoint{7.784989in}{5.721237in}}%
\pgfpathcurveto{\pgfqpoint{7.773939in}{5.721237in}}{\pgfqpoint{7.763340in}{5.716847in}}{\pgfqpoint{7.755526in}{5.709033in}}%
\pgfpathcurveto{\pgfqpoint{7.747713in}{5.701220in}}{\pgfqpoint{7.743322in}{5.690621in}}{\pgfqpoint{7.743322in}{5.679570in}}%
\pgfpathcurveto{\pgfqpoint{7.743322in}{5.668520in}}{\pgfqpoint{7.747713in}{5.657921in}}{\pgfqpoint{7.755526in}{5.650108in}}%
\pgfpathcurveto{\pgfqpoint{7.763340in}{5.642294in}}{\pgfqpoint{7.773939in}{5.637904in}}{\pgfqpoint{7.784989in}{5.637904in}}%
\pgfpathclose%
\pgfusepath{stroke,fill}%
\end{pgfscope}%
\begin{pgfscope}%
\pgfpathrectangle{\pgfqpoint{0.393613in}{0.331635in}}{\pgfqpoint{9.300000in}{7.700000in}}%
\pgfusepath{clip}%
\pgfsetbuttcap%
\pgfsetroundjoin%
\definecolor{currentfill}{rgb}{1.000000,0.705882,0.509804}%
\pgfsetfillcolor{currentfill}%
\pgfsetlinewidth{0.481800pt}%
\definecolor{currentstroke}{rgb}{1.000000,1.000000,1.000000}%
\pgfsetstrokecolor{currentstroke}%
\pgfsetdash{}{0pt}%
\pgfpathmoveto{\pgfqpoint{9.033656in}{3.832510in}}%
\pgfpathcurveto{\pgfqpoint{9.044706in}{3.832510in}}{\pgfqpoint{9.055305in}{3.836901in}}{\pgfqpoint{9.063119in}{3.844714in}}%
\pgfpathcurveto{\pgfqpoint{9.070933in}{3.852528in}}{\pgfqpoint{9.075323in}{3.863127in}}{\pgfqpoint{9.075323in}{3.874177in}}%
\pgfpathcurveto{\pgfqpoint{9.075323in}{3.885227in}}{\pgfqpoint{9.070933in}{3.895826in}}{\pgfqpoint{9.063119in}{3.903640in}}%
\pgfpathcurveto{\pgfqpoint{9.055305in}{3.911454in}}{\pgfqpoint{9.044706in}{3.915844in}}{\pgfqpoint{9.033656in}{3.915844in}}%
\pgfpathcurveto{\pgfqpoint{9.022606in}{3.915844in}}{\pgfqpoint{9.012007in}{3.911454in}}{\pgfqpoint{9.004193in}{3.903640in}}%
\pgfpathcurveto{\pgfqpoint{8.996380in}{3.895826in}}{\pgfqpoint{8.991990in}{3.885227in}}{\pgfqpoint{8.991990in}{3.874177in}}%
\pgfpathcurveto{\pgfqpoint{8.991990in}{3.863127in}}{\pgfqpoint{8.996380in}{3.852528in}}{\pgfqpoint{9.004193in}{3.844714in}}%
\pgfpathcurveto{\pgfqpoint{9.012007in}{3.836901in}}{\pgfqpoint{9.022606in}{3.832510in}}{\pgfqpoint{9.033656in}{3.832510in}}%
\pgfpathclose%
\pgfusepath{stroke,fill}%
\end{pgfscope}%
\begin{pgfscope}%
\pgfpathrectangle{\pgfqpoint{0.393613in}{0.331635in}}{\pgfqpoint{9.300000in}{7.700000in}}%
\pgfusepath{clip}%
\pgfsetbuttcap%
\pgfsetroundjoin%
\definecolor{currentfill}{rgb}{1.000000,0.705882,0.509804}%
\pgfsetfillcolor{currentfill}%
\pgfsetlinewidth{0.481800pt}%
\definecolor{currentstroke}{rgb}{1.000000,1.000000,1.000000}%
\pgfsetstrokecolor{currentstroke}%
\pgfsetdash{}{0pt}%
\pgfpathmoveto{\pgfqpoint{4.512348in}{4.999893in}}%
\pgfpathcurveto{\pgfqpoint{4.523398in}{4.999893in}}{\pgfqpoint{4.533997in}{5.004283in}}{\pgfqpoint{4.541810in}{5.012097in}}%
\pgfpathcurveto{\pgfqpoint{4.549624in}{5.019910in}}{\pgfqpoint{4.554014in}{5.030509in}}{\pgfqpoint{4.554014in}{5.041560in}}%
\pgfpathcurveto{\pgfqpoint{4.554014in}{5.052610in}}{\pgfqpoint{4.549624in}{5.063209in}}{\pgfqpoint{4.541810in}{5.071022in}}%
\pgfpathcurveto{\pgfqpoint{4.533997in}{5.078836in}}{\pgfqpoint{4.523398in}{5.083226in}}{\pgfqpoint{4.512348in}{5.083226in}}%
\pgfpathcurveto{\pgfqpoint{4.501297in}{5.083226in}}{\pgfqpoint{4.490698in}{5.078836in}}{\pgfqpoint{4.482885in}{5.071022in}}%
\pgfpathcurveto{\pgfqpoint{4.475071in}{5.063209in}}{\pgfqpoint{4.470681in}{5.052610in}}{\pgfqpoint{4.470681in}{5.041560in}}%
\pgfpathcurveto{\pgfqpoint{4.470681in}{5.030509in}}{\pgfqpoint{4.475071in}{5.019910in}}{\pgfqpoint{4.482885in}{5.012097in}}%
\pgfpathcurveto{\pgfqpoint{4.490698in}{5.004283in}}{\pgfqpoint{4.501297in}{4.999893in}}{\pgfqpoint{4.512348in}{4.999893in}}%
\pgfpathclose%
\pgfusepath{stroke,fill}%
\end{pgfscope}%
\begin{pgfscope}%
\pgfpathrectangle{\pgfqpoint{0.393613in}{0.331635in}}{\pgfqpoint{9.300000in}{7.700000in}}%
\pgfusepath{clip}%
\pgfsetbuttcap%
\pgfsetroundjoin%
\definecolor{currentfill}{rgb}{1.000000,0.705882,0.509804}%
\pgfsetfillcolor{currentfill}%
\pgfsetlinewidth{0.481800pt}%
\definecolor{currentstroke}{rgb}{1.000000,1.000000,1.000000}%
\pgfsetstrokecolor{currentstroke}%
\pgfsetdash{}{0pt}%
\pgfpathmoveto{\pgfqpoint{5.861358in}{5.827605in}}%
\pgfpathcurveto{\pgfqpoint{5.872408in}{5.827605in}}{\pgfqpoint{5.883007in}{5.831995in}}{\pgfqpoint{5.890820in}{5.839809in}}%
\pgfpathcurveto{\pgfqpoint{5.898634in}{5.847623in}}{\pgfqpoint{5.903024in}{5.858222in}}{\pgfqpoint{5.903024in}{5.869272in}}%
\pgfpathcurveto{\pgfqpoint{5.903024in}{5.880322in}}{\pgfqpoint{5.898634in}{5.890921in}}{\pgfqpoint{5.890820in}{5.898734in}}%
\pgfpathcurveto{\pgfqpoint{5.883007in}{5.906548in}}{\pgfqpoint{5.872408in}{5.910938in}}{\pgfqpoint{5.861358in}{5.910938in}}%
\pgfpathcurveto{\pgfqpoint{5.850307in}{5.910938in}}{\pgfqpoint{5.839708in}{5.906548in}}{\pgfqpoint{5.831895in}{5.898734in}}%
\pgfpathcurveto{\pgfqpoint{5.824081in}{5.890921in}}{\pgfqpoint{5.819691in}{5.880322in}}{\pgfqpoint{5.819691in}{5.869272in}}%
\pgfpathcurveto{\pgfqpoint{5.819691in}{5.858222in}}{\pgfqpoint{5.824081in}{5.847623in}}{\pgfqpoint{5.831895in}{5.839809in}}%
\pgfpathcurveto{\pgfqpoint{5.839708in}{5.831995in}}{\pgfqpoint{5.850307in}{5.827605in}}{\pgfqpoint{5.861358in}{5.827605in}}%
\pgfpathclose%
\pgfusepath{stroke,fill}%
\end{pgfscope}%
\begin{pgfscope}%
\pgfpathrectangle{\pgfqpoint{0.393613in}{0.331635in}}{\pgfqpoint{9.300000in}{7.700000in}}%
\pgfusepath{clip}%
\pgfsetbuttcap%
\pgfsetroundjoin%
\definecolor{currentfill}{rgb}{1.000000,0.705882,0.509804}%
\pgfsetfillcolor{currentfill}%
\pgfsetlinewidth{0.481800pt}%
\definecolor{currentstroke}{rgb}{1.000000,1.000000,1.000000}%
\pgfsetstrokecolor{currentstroke}%
\pgfsetdash{}{0pt}%
\pgfpathmoveto{\pgfqpoint{5.626156in}{2.786395in}}%
\pgfpathcurveto{\pgfqpoint{5.637207in}{2.786395in}}{\pgfqpoint{5.647806in}{2.790785in}}{\pgfqpoint{5.655619in}{2.798599in}}%
\pgfpathcurveto{\pgfqpoint{5.663433in}{2.806413in}}{\pgfqpoint{5.667823in}{2.817012in}}{\pgfqpoint{5.667823in}{2.828062in}}%
\pgfpathcurveto{\pgfqpoint{5.667823in}{2.839112in}}{\pgfqpoint{5.663433in}{2.849711in}}{\pgfqpoint{5.655619in}{2.857525in}}%
\pgfpathcurveto{\pgfqpoint{5.647806in}{2.865338in}}{\pgfqpoint{5.637207in}{2.869729in}}{\pgfqpoint{5.626156in}{2.869729in}}%
\pgfpathcurveto{\pgfqpoint{5.615106in}{2.869729in}}{\pgfqpoint{5.604507in}{2.865338in}}{\pgfqpoint{5.596694in}{2.857525in}}%
\pgfpathcurveto{\pgfqpoint{5.588880in}{2.849711in}}{\pgfqpoint{5.584490in}{2.839112in}}{\pgfqpoint{5.584490in}{2.828062in}}%
\pgfpathcurveto{\pgfqpoint{5.584490in}{2.817012in}}{\pgfqpoint{5.588880in}{2.806413in}}{\pgfqpoint{5.596694in}{2.798599in}}%
\pgfpathcurveto{\pgfqpoint{5.604507in}{2.790785in}}{\pgfqpoint{5.615106in}{2.786395in}}{\pgfqpoint{5.626156in}{2.786395in}}%
\pgfpathclose%
\pgfusepath{stroke,fill}%
\end{pgfscope}%
\begin{pgfscope}%
\pgfpathrectangle{\pgfqpoint{0.393613in}{0.331635in}}{\pgfqpoint{9.300000in}{7.700000in}}%
\pgfusepath{clip}%
\pgfsetbuttcap%
\pgfsetroundjoin%
\definecolor{currentfill}{rgb}{1.000000,0.705882,0.509804}%
\pgfsetfillcolor{currentfill}%
\pgfsetlinewidth{0.481800pt}%
\definecolor{currentstroke}{rgb}{1.000000,1.000000,1.000000}%
\pgfsetstrokecolor{currentstroke}%
\pgfsetdash{}{0pt}%
\pgfpathmoveto{\pgfqpoint{4.868691in}{3.435194in}}%
\pgfpathcurveto{\pgfqpoint{4.879741in}{3.435194in}}{\pgfqpoint{4.890340in}{3.439584in}}{\pgfqpoint{4.898154in}{3.447397in}}%
\pgfpathcurveto{\pgfqpoint{4.905967in}{3.455211in}}{\pgfqpoint{4.910358in}{3.465810in}}{\pgfqpoint{4.910358in}{3.476860in}}%
\pgfpathcurveto{\pgfqpoint{4.910358in}{3.487910in}}{\pgfqpoint{4.905967in}{3.498509in}}{\pgfqpoint{4.898154in}{3.506323in}}%
\pgfpathcurveto{\pgfqpoint{4.890340in}{3.514137in}}{\pgfqpoint{4.879741in}{3.518527in}}{\pgfqpoint{4.868691in}{3.518527in}}%
\pgfpathcurveto{\pgfqpoint{4.857641in}{3.518527in}}{\pgfqpoint{4.847042in}{3.514137in}}{\pgfqpoint{4.839228in}{3.506323in}}%
\pgfpathcurveto{\pgfqpoint{4.831414in}{3.498509in}}{\pgfqpoint{4.827024in}{3.487910in}}{\pgfqpoint{4.827024in}{3.476860in}}%
\pgfpathcurveto{\pgfqpoint{4.827024in}{3.465810in}}{\pgfqpoint{4.831414in}{3.455211in}}{\pgfqpoint{4.839228in}{3.447397in}}%
\pgfpathcurveto{\pgfqpoint{4.847042in}{3.439584in}}{\pgfqpoint{4.857641in}{3.435194in}}{\pgfqpoint{4.868691in}{3.435194in}}%
\pgfpathclose%
\pgfusepath{stroke,fill}%
\end{pgfscope}%
\begin{pgfscope}%
\pgfpathrectangle{\pgfqpoint{0.393613in}{0.331635in}}{\pgfqpoint{9.300000in}{7.700000in}}%
\pgfusepath{clip}%
\pgfsetbuttcap%
\pgfsetroundjoin%
\definecolor{currentfill}{rgb}{1.000000,0.705882,0.509804}%
\pgfsetfillcolor{currentfill}%
\pgfsetlinewidth{0.481800pt}%
\definecolor{currentstroke}{rgb}{1.000000,1.000000,1.000000}%
\pgfsetstrokecolor{currentstroke}%
\pgfsetdash{}{0pt}%
\pgfpathmoveto{\pgfqpoint{7.939732in}{4.099002in}}%
\pgfpathcurveto{\pgfqpoint{7.950782in}{4.099002in}}{\pgfqpoint{7.961381in}{4.103392in}}{\pgfqpoint{7.969195in}{4.111205in}}%
\pgfpathcurveto{\pgfqpoint{7.977009in}{4.119019in}}{\pgfqpoint{7.981399in}{4.129618in}}{\pgfqpoint{7.981399in}{4.140668in}}%
\pgfpathcurveto{\pgfqpoint{7.981399in}{4.151718in}}{\pgfqpoint{7.977009in}{4.162317in}}{\pgfqpoint{7.969195in}{4.170131in}}%
\pgfpathcurveto{\pgfqpoint{7.961381in}{4.177945in}}{\pgfqpoint{7.950782in}{4.182335in}}{\pgfqpoint{7.939732in}{4.182335in}}%
\pgfpathcurveto{\pgfqpoint{7.928682in}{4.182335in}}{\pgfqpoint{7.918083in}{4.177945in}}{\pgfqpoint{7.910269in}{4.170131in}}%
\pgfpathcurveto{\pgfqpoint{7.902456in}{4.162317in}}{\pgfqpoint{7.898065in}{4.151718in}}{\pgfqpoint{7.898065in}{4.140668in}}%
\pgfpathcurveto{\pgfqpoint{7.898065in}{4.129618in}}{\pgfqpoint{7.902456in}{4.119019in}}{\pgfqpoint{7.910269in}{4.111205in}}%
\pgfpathcurveto{\pgfqpoint{7.918083in}{4.103392in}}{\pgfqpoint{7.928682in}{4.099002in}}{\pgfqpoint{7.939732in}{4.099002in}}%
\pgfpathclose%
\pgfusepath{stroke,fill}%
\end{pgfscope}%
\begin{pgfscope}%
\pgfpathrectangle{\pgfqpoint{0.393613in}{0.331635in}}{\pgfqpoint{9.300000in}{7.700000in}}%
\pgfusepath{clip}%
\pgfsetbuttcap%
\pgfsetroundjoin%
\definecolor{currentfill}{rgb}{1.000000,0.705882,0.509804}%
\pgfsetfillcolor{currentfill}%
\pgfsetlinewidth{0.481800pt}%
\definecolor{currentstroke}{rgb}{1.000000,1.000000,1.000000}%
\pgfsetstrokecolor{currentstroke}%
\pgfsetdash{}{0pt}%
\pgfpathmoveto{\pgfqpoint{5.975649in}{5.859945in}}%
\pgfpathcurveto{\pgfqpoint{5.986700in}{5.859945in}}{\pgfqpoint{5.997299in}{5.864336in}}{\pgfqpoint{6.005112in}{5.872149in}}%
\pgfpathcurveto{\pgfqpoint{6.012926in}{5.879963in}}{\pgfqpoint{6.017316in}{5.890562in}}{\pgfqpoint{6.017316in}{5.901612in}}%
\pgfpathcurveto{\pgfqpoint{6.017316in}{5.912662in}}{\pgfqpoint{6.012926in}{5.923261in}}{\pgfqpoint{6.005112in}{5.931075in}}%
\pgfpathcurveto{\pgfqpoint{5.997299in}{5.938888in}}{\pgfqpoint{5.986700in}{5.943279in}}{\pgfqpoint{5.975649in}{5.943279in}}%
\pgfpathcurveto{\pgfqpoint{5.964599in}{5.943279in}}{\pgfqpoint{5.954000in}{5.938888in}}{\pgfqpoint{5.946187in}{5.931075in}}%
\pgfpathcurveto{\pgfqpoint{5.938373in}{5.923261in}}{\pgfqpoint{5.933983in}{5.912662in}}{\pgfqpoint{5.933983in}{5.901612in}}%
\pgfpathcurveto{\pgfqpoint{5.933983in}{5.890562in}}{\pgfqpoint{5.938373in}{5.879963in}}{\pgfqpoint{5.946187in}{5.872149in}}%
\pgfpathcurveto{\pgfqpoint{5.954000in}{5.864336in}}{\pgfqpoint{5.964599in}{5.859945in}}{\pgfqpoint{5.975649in}{5.859945in}}%
\pgfpathclose%
\pgfusepath{stroke,fill}%
\end{pgfscope}%
\begin{pgfscope}%
\pgfpathrectangle{\pgfqpoint{0.393613in}{0.331635in}}{\pgfqpoint{9.300000in}{7.700000in}}%
\pgfusepath{clip}%
\pgfsetbuttcap%
\pgfsetroundjoin%
\definecolor{currentfill}{rgb}{1.000000,0.705882,0.509804}%
\pgfsetfillcolor{currentfill}%
\pgfsetlinewidth{0.481800pt}%
\definecolor{currentstroke}{rgb}{1.000000,1.000000,1.000000}%
\pgfsetstrokecolor{currentstroke}%
\pgfsetdash{}{0pt}%
\pgfpathmoveto{\pgfqpoint{5.533563in}{3.205376in}}%
\pgfpathcurveto{\pgfqpoint{5.544613in}{3.205376in}}{\pgfqpoint{5.555212in}{3.209767in}}{\pgfqpoint{5.563026in}{3.217580in}}%
\pgfpathcurveto{\pgfqpoint{5.570840in}{3.225394in}}{\pgfqpoint{5.575230in}{3.235993in}}{\pgfqpoint{5.575230in}{3.247043in}}%
\pgfpathcurveto{\pgfqpoint{5.575230in}{3.258093in}}{\pgfqpoint{5.570840in}{3.268692in}}{\pgfqpoint{5.563026in}{3.276506in}}%
\pgfpathcurveto{\pgfqpoint{5.555212in}{3.284320in}}{\pgfqpoint{5.544613in}{3.288710in}}{\pgfqpoint{5.533563in}{3.288710in}}%
\pgfpathcurveto{\pgfqpoint{5.522513in}{3.288710in}}{\pgfqpoint{5.511914in}{3.284320in}}{\pgfqpoint{5.504100in}{3.276506in}}%
\pgfpathcurveto{\pgfqpoint{5.496287in}{3.268692in}}{\pgfqpoint{5.491897in}{3.258093in}}{\pgfqpoint{5.491897in}{3.247043in}}%
\pgfpathcurveto{\pgfqpoint{5.491897in}{3.235993in}}{\pgfqpoint{5.496287in}{3.225394in}}{\pgfqpoint{5.504100in}{3.217580in}}%
\pgfpathcurveto{\pgfqpoint{5.511914in}{3.209767in}}{\pgfqpoint{5.522513in}{3.205376in}}{\pgfqpoint{5.533563in}{3.205376in}}%
\pgfpathclose%
\pgfusepath{stroke,fill}%
\end{pgfscope}%
\begin{pgfscope}%
\pgfpathrectangle{\pgfqpoint{0.393613in}{0.331635in}}{\pgfqpoint{9.300000in}{7.700000in}}%
\pgfusepath{clip}%
\pgfsetbuttcap%
\pgfsetroundjoin%
\definecolor{currentfill}{rgb}{1.000000,0.705882,0.509804}%
\pgfsetfillcolor{currentfill}%
\pgfsetlinewidth{0.481800pt}%
\definecolor{currentstroke}{rgb}{1.000000,1.000000,1.000000}%
\pgfsetstrokecolor{currentstroke}%
\pgfsetdash{}{0pt}%
\pgfpathmoveto{\pgfqpoint{8.879060in}{4.485403in}}%
\pgfpathcurveto{\pgfqpoint{8.890110in}{4.485403in}}{\pgfqpoint{8.900709in}{4.489794in}}{\pgfqpoint{8.908523in}{4.497607in}}%
\pgfpathcurveto{\pgfqpoint{8.916336in}{4.505421in}}{\pgfqpoint{8.920726in}{4.516020in}}{\pgfqpoint{8.920726in}{4.527070in}}%
\pgfpathcurveto{\pgfqpoint{8.920726in}{4.538120in}}{\pgfqpoint{8.916336in}{4.548719in}}{\pgfqpoint{8.908523in}{4.556533in}}%
\pgfpathcurveto{\pgfqpoint{8.900709in}{4.564347in}}{\pgfqpoint{8.890110in}{4.568737in}}{\pgfqpoint{8.879060in}{4.568737in}}%
\pgfpathcurveto{\pgfqpoint{8.868010in}{4.568737in}}{\pgfqpoint{8.857411in}{4.564347in}}{\pgfqpoint{8.849597in}{4.556533in}}%
\pgfpathcurveto{\pgfqpoint{8.841783in}{4.548719in}}{\pgfqpoint{8.837393in}{4.538120in}}{\pgfqpoint{8.837393in}{4.527070in}}%
\pgfpathcurveto{\pgfqpoint{8.837393in}{4.516020in}}{\pgfqpoint{8.841783in}{4.505421in}}{\pgfqpoint{8.849597in}{4.497607in}}%
\pgfpathcurveto{\pgfqpoint{8.857411in}{4.489794in}}{\pgfqpoint{8.868010in}{4.485403in}}{\pgfqpoint{8.879060in}{4.485403in}}%
\pgfpathclose%
\pgfusepath{stroke,fill}%
\end{pgfscope}%
\begin{pgfscope}%
\pgfpathrectangle{\pgfqpoint{0.393613in}{0.331635in}}{\pgfqpoint{9.300000in}{7.700000in}}%
\pgfusepath{clip}%
\pgfsetbuttcap%
\pgfsetroundjoin%
\definecolor{currentfill}{rgb}{1.000000,0.705882,0.509804}%
\pgfsetfillcolor{currentfill}%
\pgfsetlinewidth{0.481800pt}%
\definecolor{currentstroke}{rgb}{1.000000,1.000000,1.000000}%
\pgfsetstrokecolor{currentstroke}%
\pgfsetdash{}{0pt}%
\pgfpathmoveto{\pgfqpoint{4.426686in}{4.638474in}}%
\pgfpathcurveto{\pgfqpoint{4.437736in}{4.638474in}}{\pgfqpoint{4.448335in}{4.642864in}}{\pgfqpoint{4.456149in}{4.650678in}}%
\pgfpathcurveto{\pgfqpoint{4.463963in}{4.658492in}}{\pgfqpoint{4.468353in}{4.669091in}}{\pgfqpoint{4.468353in}{4.680141in}}%
\pgfpathcurveto{\pgfqpoint{4.468353in}{4.691191in}}{\pgfqpoint{4.463963in}{4.701790in}}{\pgfqpoint{4.456149in}{4.709603in}}%
\pgfpathcurveto{\pgfqpoint{4.448335in}{4.717417in}}{\pgfqpoint{4.437736in}{4.721807in}}{\pgfqpoint{4.426686in}{4.721807in}}%
\pgfpathcurveto{\pgfqpoint{4.415636in}{4.721807in}}{\pgfqpoint{4.405037in}{4.717417in}}{\pgfqpoint{4.397223in}{4.709603in}}%
\pgfpathcurveto{\pgfqpoint{4.389410in}{4.701790in}}{\pgfqpoint{4.385020in}{4.691191in}}{\pgfqpoint{4.385020in}{4.680141in}}%
\pgfpathcurveto{\pgfqpoint{4.385020in}{4.669091in}}{\pgfqpoint{4.389410in}{4.658492in}}{\pgfqpoint{4.397223in}{4.650678in}}%
\pgfpathcurveto{\pgfqpoint{4.405037in}{4.642864in}}{\pgfqpoint{4.415636in}{4.638474in}}{\pgfqpoint{4.426686in}{4.638474in}}%
\pgfpathclose%
\pgfusepath{stroke,fill}%
\end{pgfscope}%
\begin{pgfscope}%
\pgfpathrectangle{\pgfqpoint{0.393613in}{0.331635in}}{\pgfqpoint{9.300000in}{7.700000in}}%
\pgfusepath{clip}%
\pgfsetbuttcap%
\pgfsetroundjoin%
\definecolor{currentfill}{rgb}{1.000000,0.705882,0.509804}%
\pgfsetfillcolor{currentfill}%
\pgfsetlinewidth{0.481800pt}%
\definecolor{currentstroke}{rgb}{1.000000,1.000000,1.000000}%
\pgfsetstrokecolor{currentstroke}%
\pgfsetdash{}{0pt}%
\pgfpathmoveto{\pgfqpoint{6.115172in}{3.487692in}}%
\pgfpathcurveto{\pgfqpoint{6.126222in}{3.487692in}}{\pgfqpoint{6.136821in}{3.492083in}}{\pgfqpoint{6.144635in}{3.499896in}}%
\pgfpathcurveto{\pgfqpoint{6.152449in}{3.507710in}}{\pgfqpoint{6.156839in}{3.518309in}}{\pgfqpoint{6.156839in}{3.529359in}}%
\pgfpathcurveto{\pgfqpoint{6.156839in}{3.540409in}}{\pgfqpoint{6.152449in}{3.551008in}}{\pgfqpoint{6.144635in}{3.558822in}}%
\pgfpathcurveto{\pgfqpoint{6.136821in}{3.566636in}}{\pgfqpoint{6.126222in}{3.571026in}}{\pgfqpoint{6.115172in}{3.571026in}}%
\pgfpathcurveto{\pgfqpoint{6.104122in}{3.571026in}}{\pgfqpoint{6.093523in}{3.566636in}}{\pgfqpoint{6.085709in}{3.558822in}}%
\pgfpathcurveto{\pgfqpoint{6.077896in}{3.551008in}}{\pgfqpoint{6.073506in}{3.540409in}}{\pgfqpoint{6.073506in}{3.529359in}}%
\pgfpathcurveto{\pgfqpoint{6.073506in}{3.518309in}}{\pgfqpoint{6.077896in}{3.507710in}}{\pgfqpoint{6.085709in}{3.499896in}}%
\pgfpathcurveto{\pgfqpoint{6.093523in}{3.492083in}}{\pgfqpoint{6.104122in}{3.487692in}}{\pgfqpoint{6.115172in}{3.487692in}}%
\pgfpathclose%
\pgfusepath{stroke,fill}%
\end{pgfscope}%
\begin{pgfscope}%
\pgfpathrectangle{\pgfqpoint{0.393613in}{0.331635in}}{\pgfqpoint{9.300000in}{7.700000in}}%
\pgfusepath{clip}%
\pgfsetbuttcap%
\pgfsetroundjoin%
\definecolor{currentfill}{rgb}{1.000000,0.705882,0.509804}%
\pgfsetfillcolor{currentfill}%
\pgfsetlinewidth{0.481800pt}%
\definecolor{currentstroke}{rgb}{1.000000,1.000000,1.000000}%
\pgfsetstrokecolor{currentstroke}%
\pgfsetdash{}{0pt}%
\pgfpathmoveto{\pgfqpoint{5.118229in}{2.365376in}}%
\pgfpathcurveto{\pgfqpoint{5.129279in}{2.365376in}}{\pgfqpoint{5.139878in}{2.369767in}}{\pgfqpoint{5.147691in}{2.377580in}}%
\pgfpathcurveto{\pgfqpoint{5.155505in}{2.385394in}}{\pgfqpoint{5.159895in}{2.395993in}}{\pgfqpoint{5.159895in}{2.407043in}}%
\pgfpathcurveto{\pgfqpoint{5.159895in}{2.418093in}}{\pgfqpoint{5.155505in}{2.428692in}}{\pgfqpoint{5.147691in}{2.436506in}}%
\pgfpathcurveto{\pgfqpoint{5.139878in}{2.444319in}}{\pgfqpoint{5.129279in}{2.448710in}}{\pgfqpoint{5.118229in}{2.448710in}}%
\pgfpathcurveto{\pgfqpoint{5.107178in}{2.448710in}}{\pgfqpoint{5.096579in}{2.444319in}}{\pgfqpoint{5.088766in}{2.436506in}}%
\pgfpathcurveto{\pgfqpoint{5.080952in}{2.428692in}}{\pgfqpoint{5.076562in}{2.418093in}}{\pgfqpoint{5.076562in}{2.407043in}}%
\pgfpathcurveto{\pgfqpoint{5.076562in}{2.395993in}}{\pgfqpoint{5.080952in}{2.385394in}}{\pgfqpoint{5.088766in}{2.377580in}}%
\pgfpathcurveto{\pgfqpoint{5.096579in}{2.369767in}}{\pgfqpoint{5.107178in}{2.365376in}}{\pgfqpoint{5.118229in}{2.365376in}}%
\pgfpathclose%
\pgfusepath{stroke,fill}%
\end{pgfscope}%
\begin{pgfscope}%
\pgfpathrectangle{\pgfqpoint{0.393613in}{0.331635in}}{\pgfqpoint{9.300000in}{7.700000in}}%
\pgfusepath{clip}%
\pgfsetbuttcap%
\pgfsetroundjoin%
\definecolor{currentfill}{rgb}{1.000000,0.705882,0.509804}%
\pgfsetfillcolor{currentfill}%
\pgfsetlinewidth{0.481800pt}%
\definecolor{currentstroke}{rgb}{1.000000,1.000000,1.000000}%
\pgfsetstrokecolor{currentstroke}%
\pgfsetdash{}{0pt}%
\pgfpathmoveto{\pgfqpoint{6.785951in}{6.307098in}}%
\pgfpathcurveto{\pgfqpoint{6.797001in}{6.307098in}}{\pgfqpoint{6.807600in}{6.311489in}}{\pgfqpoint{6.815414in}{6.319302in}}%
\pgfpathcurveto{\pgfqpoint{6.823227in}{6.327116in}}{\pgfqpoint{6.827617in}{6.337715in}}{\pgfqpoint{6.827617in}{6.348765in}}%
\pgfpathcurveto{\pgfqpoint{6.827617in}{6.359815in}}{\pgfqpoint{6.823227in}{6.370414in}}{\pgfqpoint{6.815414in}{6.378228in}}%
\pgfpathcurveto{\pgfqpoint{6.807600in}{6.386042in}}{\pgfqpoint{6.797001in}{6.390432in}}{\pgfqpoint{6.785951in}{6.390432in}}%
\pgfpathcurveto{\pgfqpoint{6.774901in}{6.390432in}}{\pgfqpoint{6.764302in}{6.386042in}}{\pgfqpoint{6.756488in}{6.378228in}}%
\pgfpathcurveto{\pgfqpoint{6.748674in}{6.370414in}}{\pgfqpoint{6.744284in}{6.359815in}}{\pgfqpoint{6.744284in}{6.348765in}}%
\pgfpathcurveto{\pgfqpoint{6.744284in}{6.337715in}}{\pgfqpoint{6.748674in}{6.327116in}}{\pgfqpoint{6.756488in}{6.319302in}}%
\pgfpathcurveto{\pgfqpoint{6.764302in}{6.311489in}}{\pgfqpoint{6.774901in}{6.307098in}}{\pgfqpoint{6.785951in}{6.307098in}}%
\pgfpathclose%
\pgfusepath{stroke,fill}%
\end{pgfscope}%
\begin{pgfscope}%
\pgfpathrectangle{\pgfqpoint{0.393613in}{0.331635in}}{\pgfqpoint{9.300000in}{7.700000in}}%
\pgfusepath{clip}%
\pgfsetbuttcap%
\pgfsetroundjoin%
\definecolor{currentfill}{rgb}{1.000000,0.705882,0.509804}%
\pgfsetfillcolor{currentfill}%
\pgfsetlinewidth{0.481800pt}%
\definecolor{currentstroke}{rgb}{1.000000,1.000000,1.000000}%
\pgfsetstrokecolor{currentstroke}%
\pgfsetdash{}{0pt}%
\pgfpathmoveto{\pgfqpoint{4.549543in}{2.828118in}}%
\pgfpathcurveto{\pgfqpoint{4.560593in}{2.828118in}}{\pgfqpoint{4.571192in}{2.832508in}}{\pgfqpoint{4.579006in}{2.840322in}}%
\pgfpathcurveto{\pgfqpoint{4.586820in}{2.848135in}}{\pgfqpoint{4.591210in}{2.858734in}}{\pgfqpoint{4.591210in}{2.869784in}}%
\pgfpathcurveto{\pgfqpoint{4.591210in}{2.880835in}}{\pgfqpoint{4.586820in}{2.891434in}}{\pgfqpoint{4.579006in}{2.899247in}}%
\pgfpathcurveto{\pgfqpoint{4.571192in}{2.907061in}}{\pgfqpoint{4.560593in}{2.911451in}}{\pgfqpoint{4.549543in}{2.911451in}}%
\pgfpathcurveto{\pgfqpoint{4.538493in}{2.911451in}}{\pgfqpoint{4.527894in}{2.907061in}}{\pgfqpoint{4.520080in}{2.899247in}}%
\pgfpathcurveto{\pgfqpoint{4.512267in}{2.891434in}}{\pgfqpoint{4.507876in}{2.880835in}}{\pgfqpoint{4.507876in}{2.869784in}}%
\pgfpathcurveto{\pgfqpoint{4.507876in}{2.858734in}}{\pgfqpoint{4.512267in}{2.848135in}}{\pgfqpoint{4.520080in}{2.840322in}}%
\pgfpathcurveto{\pgfqpoint{4.527894in}{2.832508in}}{\pgfqpoint{4.538493in}{2.828118in}}{\pgfqpoint{4.549543in}{2.828118in}}%
\pgfpathclose%
\pgfusepath{stroke,fill}%
\end{pgfscope}%
\begin{pgfscope}%
\pgfpathrectangle{\pgfqpoint{0.393613in}{0.331635in}}{\pgfqpoint{9.300000in}{7.700000in}}%
\pgfusepath{clip}%
\pgfsetbuttcap%
\pgfsetroundjoin%
\definecolor{currentfill}{rgb}{1.000000,0.705882,0.509804}%
\pgfsetfillcolor{currentfill}%
\pgfsetlinewidth{0.481800pt}%
\definecolor{currentstroke}{rgb}{1.000000,1.000000,1.000000}%
\pgfsetstrokecolor{currentstroke}%
\pgfsetdash{}{0pt}%
\pgfpathmoveto{\pgfqpoint{7.362784in}{4.414752in}}%
\pgfpathcurveto{\pgfqpoint{7.373834in}{4.414752in}}{\pgfqpoint{7.384433in}{4.419142in}}{\pgfqpoint{7.392247in}{4.426956in}}%
\pgfpathcurveto{\pgfqpoint{7.400060in}{4.434770in}}{\pgfqpoint{7.404451in}{4.445369in}}{\pgfqpoint{7.404451in}{4.456419in}}%
\pgfpathcurveto{\pgfqpoint{7.404451in}{4.467469in}}{\pgfqpoint{7.400060in}{4.478068in}}{\pgfqpoint{7.392247in}{4.485881in}}%
\pgfpathcurveto{\pgfqpoint{7.384433in}{4.493695in}}{\pgfqpoint{7.373834in}{4.498085in}}{\pgfqpoint{7.362784in}{4.498085in}}%
\pgfpathcurveto{\pgfqpoint{7.351734in}{4.498085in}}{\pgfqpoint{7.341135in}{4.493695in}}{\pgfqpoint{7.333321in}{4.485881in}}%
\pgfpathcurveto{\pgfqpoint{7.325508in}{4.478068in}}{\pgfqpoint{7.321117in}{4.467469in}}{\pgfqpoint{7.321117in}{4.456419in}}%
\pgfpathcurveto{\pgfqpoint{7.321117in}{4.445369in}}{\pgfqpoint{7.325508in}{4.434770in}}{\pgfqpoint{7.333321in}{4.426956in}}%
\pgfpathcurveto{\pgfqpoint{7.341135in}{4.419142in}}{\pgfqpoint{7.351734in}{4.414752in}}{\pgfqpoint{7.362784in}{4.414752in}}%
\pgfpathclose%
\pgfusepath{stroke,fill}%
\end{pgfscope}%
\begin{pgfscope}%
\pgfpathrectangle{\pgfqpoint{0.393613in}{0.331635in}}{\pgfqpoint{9.300000in}{7.700000in}}%
\pgfusepath{clip}%
\pgfsetbuttcap%
\pgfsetroundjoin%
\definecolor{currentfill}{rgb}{1.000000,0.705882,0.509804}%
\pgfsetfillcolor{currentfill}%
\pgfsetlinewidth{0.481800pt}%
\definecolor{currentstroke}{rgb}{1.000000,1.000000,1.000000}%
\pgfsetstrokecolor{currentstroke}%
\pgfsetdash{}{0pt}%
\pgfpathmoveto{\pgfqpoint{3.789706in}{4.934223in}}%
\pgfpathcurveto{\pgfqpoint{3.800756in}{4.934223in}}{\pgfqpoint{3.811355in}{4.938613in}}{\pgfqpoint{3.819168in}{4.946427in}}%
\pgfpathcurveto{\pgfqpoint{3.826982in}{4.954240in}}{\pgfqpoint{3.831372in}{4.964839in}}{\pgfqpoint{3.831372in}{4.975890in}}%
\pgfpathcurveto{\pgfqpoint{3.831372in}{4.986940in}}{\pgfqpoint{3.826982in}{4.997539in}}{\pgfqpoint{3.819168in}{5.005352in}}%
\pgfpathcurveto{\pgfqpoint{3.811355in}{5.013166in}}{\pgfqpoint{3.800756in}{5.017556in}}{\pgfqpoint{3.789706in}{5.017556in}}%
\pgfpathcurveto{\pgfqpoint{3.778655in}{5.017556in}}{\pgfqpoint{3.768056in}{5.013166in}}{\pgfqpoint{3.760243in}{5.005352in}}%
\pgfpathcurveto{\pgfqpoint{3.752429in}{4.997539in}}{\pgfqpoint{3.748039in}{4.986940in}}{\pgfqpoint{3.748039in}{4.975890in}}%
\pgfpathcurveto{\pgfqpoint{3.748039in}{4.964839in}}{\pgfqpoint{3.752429in}{4.954240in}}{\pgfqpoint{3.760243in}{4.946427in}}%
\pgfpathcurveto{\pgfqpoint{3.768056in}{4.938613in}}{\pgfqpoint{3.778655in}{4.934223in}}{\pgfqpoint{3.789706in}{4.934223in}}%
\pgfpathclose%
\pgfusepath{stroke,fill}%
\end{pgfscope}%
\begin{pgfscope}%
\pgfpathrectangle{\pgfqpoint{0.393613in}{0.331635in}}{\pgfqpoint{9.300000in}{7.700000in}}%
\pgfusepath{clip}%
\pgfsetbuttcap%
\pgfsetroundjoin%
\definecolor{currentfill}{rgb}{0.631373,0.788235,0.956863}%
\pgfsetfillcolor{currentfill}%
\pgfsetlinewidth{1.003750pt}%
\definecolor{currentstroke}{rgb}{0.631373,0.788235,0.956863}%
\pgfsetstrokecolor{currentstroke}%
\pgfsetdash{}{0pt}%
\pgfsys@defobject{currentmarker}{\pgfqpoint{-0.041667in}{-0.041667in}}{\pgfqpoint{0.041667in}{0.041667in}}{%
\pgfpathmoveto{\pgfqpoint{0.000000in}{-0.041667in}}%
\pgfpathcurveto{\pgfqpoint{0.011050in}{-0.041667in}}{\pgfqpoint{0.021649in}{-0.037276in}}{\pgfqpoint{0.029463in}{-0.029463in}}%
\pgfpathcurveto{\pgfqpoint{0.037276in}{-0.021649in}}{\pgfqpoint{0.041667in}{-0.011050in}}{\pgfqpoint{0.041667in}{0.000000in}}%
\pgfpathcurveto{\pgfqpoint{0.041667in}{0.011050in}}{\pgfqpoint{0.037276in}{0.021649in}}{\pgfqpoint{0.029463in}{0.029463in}}%
\pgfpathcurveto{\pgfqpoint{0.021649in}{0.037276in}}{\pgfqpoint{0.011050in}{0.041667in}}{\pgfqpoint{0.000000in}{0.041667in}}%
\pgfpathcurveto{\pgfqpoint{-0.011050in}{0.041667in}}{\pgfqpoint{-0.021649in}{0.037276in}}{\pgfqpoint{-0.029463in}{0.029463in}}%
\pgfpathcurveto{\pgfqpoint{-0.037276in}{0.021649in}}{\pgfqpoint{-0.041667in}{0.011050in}}{\pgfqpoint{-0.041667in}{0.000000in}}%
\pgfpathcurveto{\pgfqpoint{-0.041667in}{-0.011050in}}{\pgfqpoint{-0.037276in}{-0.021649in}}{\pgfqpoint{-0.029463in}{-0.029463in}}%
\pgfpathcurveto{\pgfqpoint{-0.021649in}{-0.037276in}}{\pgfqpoint{-0.011050in}{-0.041667in}}{\pgfqpoint{0.000000in}{-0.041667in}}%
\pgfpathclose%
\pgfusepath{stroke,fill}%
}%
\end{pgfscope}%
\begin{pgfscope}%
\pgfpathrectangle{\pgfqpoint{0.393613in}{0.331635in}}{\pgfqpoint{9.300000in}{7.700000in}}%
\pgfusepath{clip}%
\pgfsetbuttcap%
\pgfsetroundjoin%
\definecolor{currentfill}{rgb}{1.000000,0.705882,0.509804}%
\pgfsetfillcolor{currentfill}%
\pgfsetlinewidth{1.003750pt}%
\definecolor{currentstroke}{rgb}{1.000000,0.705882,0.509804}%
\pgfsetstrokecolor{currentstroke}%
\pgfsetdash{}{0pt}%
\pgfsys@defobject{currentmarker}{\pgfqpoint{-0.041667in}{-0.041667in}}{\pgfqpoint{0.041667in}{0.041667in}}{%
\pgfpathmoveto{\pgfqpoint{0.000000in}{-0.041667in}}%
\pgfpathcurveto{\pgfqpoint{0.011050in}{-0.041667in}}{\pgfqpoint{0.021649in}{-0.037276in}}{\pgfqpoint{0.029463in}{-0.029463in}}%
\pgfpathcurveto{\pgfqpoint{0.037276in}{-0.021649in}}{\pgfqpoint{0.041667in}{-0.011050in}}{\pgfqpoint{0.041667in}{0.000000in}}%
\pgfpathcurveto{\pgfqpoint{0.041667in}{0.011050in}}{\pgfqpoint{0.037276in}{0.021649in}}{\pgfqpoint{0.029463in}{0.029463in}}%
\pgfpathcurveto{\pgfqpoint{0.021649in}{0.037276in}}{\pgfqpoint{0.011050in}{0.041667in}}{\pgfqpoint{0.000000in}{0.041667in}}%
\pgfpathcurveto{\pgfqpoint{-0.011050in}{0.041667in}}{\pgfqpoint{-0.021649in}{0.037276in}}{\pgfqpoint{-0.029463in}{0.029463in}}%
\pgfpathcurveto{\pgfqpoint{-0.037276in}{0.021649in}}{\pgfqpoint{-0.041667in}{0.011050in}}{\pgfqpoint{-0.041667in}{0.000000in}}%
\pgfpathcurveto{\pgfqpoint{-0.041667in}{-0.011050in}}{\pgfqpoint{-0.037276in}{-0.021649in}}{\pgfqpoint{-0.029463in}{-0.029463in}}%
\pgfpathcurveto{\pgfqpoint{-0.021649in}{-0.037276in}}{\pgfqpoint{-0.011050in}{-0.041667in}}{\pgfqpoint{0.000000in}{-0.041667in}}%
\pgfpathclose%
\pgfusepath{stroke,fill}%
}%
\end{pgfscope}%
\begin{pgfscope}%
\pgfsetbuttcap%
\pgfsetroundjoin%
\definecolor{currentfill}{rgb}{0.000000,0.000000,0.000000}%
\pgfsetfillcolor{currentfill}%
\pgfsetlinewidth{0.803000pt}%
\definecolor{currentstroke}{rgb}{0.000000,0.000000,0.000000}%
\pgfsetstrokecolor{currentstroke}%
\pgfsetdash{}{0pt}%
\pgfsys@defobject{currentmarker}{\pgfqpoint{0.000000in}{-0.048611in}}{\pgfqpoint{0.000000in}{0.000000in}}{%
\pgfpathmoveto{\pgfqpoint{0.000000in}{0.000000in}}%
\pgfpathlineto{\pgfqpoint{0.000000in}{-0.048611in}}%
\pgfusepath{stroke,fill}%
}%
\begin{pgfscope}%
\pgfsys@transformshift{1.503282in}{0.331635in}%
\pgfsys@useobject{currentmarker}{}%
\end{pgfscope}%
\end{pgfscope}%
\begin{pgfscope}%
\definecolor{textcolor}{rgb}{0.000000,0.000000,0.000000}%
\pgfsetstrokecolor{textcolor}%
\pgfsetfillcolor{textcolor}%
\pgftext[x=1.503282in,y=0.234413in,,top]{\color{textcolor}\sffamily\fontsize{10.000000}{12.000000}\selectfont \ensuremath{-}10}%
\end{pgfscope}%
\begin{pgfscope}%
\pgfsetbuttcap%
\pgfsetroundjoin%
\definecolor{currentfill}{rgb}{0.000000,0.000000,0.000000}%
\pgfsetfillcolor{currentfill}%
\pgfsetlinewidth{0.803000pt}%
\definecolor{currentstroke}{rgb}{0.000000,0.000000,0.000000}%
\pgfsetstrokecolor{currentstroke}%
\pgfsetdash{}{0pt}%
\pgfsys@defobject{currentmarker}{\pgfqpoint{0.000000in}{-0.048611in}}{\pgfqpoint{0.000000in}{0.000000in}}{%
\pgfpathmoveto{\pgfqpoint{0.000000in}{0.000000in}}%
\pgfpathlineto{\pgfqpoint{0.000000in}{-0.048611in}}%
\pgfusepath{stroke,fill}%
}%
\begin{pgfscope}%
\pgfsys@transformshift{2.759833in}{0.331635in}%
\pgfsys@useobject{currentmarker}{}%
\end{pgfscope}%
\end{pgfscope}%
\begin{pgfscope}%
\definecolor{textcolor}{rgb}{0.000000,0.000000,0.000000}%
\pgfsetstrokecolor{textcolor}%
\pgfsetfillcolor{textcolor}%
\pgftext[x=2.759833in,y=0.234413in,,top]{\color{textcolor}\sffamily\fontsize{10.000000}{12.000000}\selectfont \ensuremath{-}8}%
\end{pgfscope}%
\begin{pgfscope}%
\pgfsetbuttcap%
\pgfsetroundjoin%
\definecolor{currentfill}{rgb}{0.000000,0.000000,0.000000}%
\pgfsetfillcolor{currentfill}%
\pgfsetlinewidth{0.803000pt}%
\definecolor{currentstroke}{rgb}{0.000000,0.000000,0.000000}%
\pgfsetstrokecolor{currentstroke}%
\pgfsetdash{}{0pt}%
\pgfsys@defobject{currentmarker}{\pgfqpoint{0.000000in}{-0.048611in}}{\pgfqpoint{0.000000in}{0.000000in}}{%
\pgfpathmoveto{\pgfqpoint{0.000000in}{0.000000in}}%
\pgfpathlineto{\pgfqpoint{0.000000in}{-0.048611in}}%
\pgfusepath{stroke,fill}%
}%
\begin{pgfscope}%
\pgfsys@transformshift{4.016384in}{0.331635in}%
\pgfsys@useobject{currentmarker}{}%
\end{pgfscope}%
\end{pgfscope}%
\begin{pgfscope}%
\definecolor{textcolor}{rgb}{0.000000,0.000000,0.000000}%
\pgfsetstrokecolor{textcolor}%
\pgfsetfillcolor{textcolor}%
\pgftext[x=4.016384in,y=0.234413in,,top]{\color{textcolor}\sffamily\fontsize{10.000000}{12.000000}\selectfont \ensuremath{-}6}%
\end{pgfscope}%
\begin{pgfscope}%
\pgfsetbuttcap%
\pgfsetroundjoin%
\definecolor{currentfill}{rgb}{0.000000,0.000000,0.000000}%
\pgfsetfillcolor{currentfill}%
\pgfsetlinewidth{0.803000pt}%
\definecolor{currentstroke}{rgb}{0.000000,0.000000,0.000000}%
\pgfsetstrokecolor{currentstroke}%
\pgfsetdash{}{0pt}%
\pgfsys@defobject{currentmarker}{\pgfqpoint{0.000000in}{-0.048611in}}{\pgfqpoint{0.000000in}{0.000000in}}{%
\pgfpathmoveto{\pgfqpoint{0.000000in}{0.000000in}}%
\pgfpathlineto{\pgfqpoint{0.000000in}{-0.048611in}}%
\pgfusepath{stroke,fill}%
}%
\begin{pgfscope}%
\pgfsys@transformshift{5.272935in}{0.331635in}%
\pgfsys@useobject{currentmarker}{}%
\end{pgfscope}%
\end{pgfscope}%
\begin{pgfscope}%
\definecolor{textcolor}{rgb}{0.000000,0.000000,0.000000}%
\pgfsetstrokecolor{textcolor}%
\pgfsetfillcolor{textcolor}%
\pgftext[x=5.272935in,y=0.234413in,,top]{\color{textcolor}\sffamily\fontsize{10.000000}{12.000000}\selectfont \ensuremath{-}4}%
\end{pgfscope}%
\begin{pgfscope}%
\pgfsetbuttcap%
\pgfsetroundjoin%
\definecolor{currentfill}{rgb}{0.000000,0.000000,0.000000}%
\pgfsetfillcolor{currentfill}%
\pgfsetlinewidth{0.803000pt}%
\definecolor{currentstroke}{rgb}{0.000000,0.000000,0.000000}%
\pgfsetstrokecolor{currentstroke}%
\pgfsetdash{}{0pt}%
\pgfsys@defobject{currentmarker}{\pgfqpoint{0.000000in}{-0.048611in}}{\pgfqpoint{0.000000in}{0.000000in}}{%
\pgfpathmoveto{\pgfqpoint{0.000000in}{0.000000in}}%
\pgfpathlineto{\pgfqpoint{0.000000in}{-0.048611in}}%
\pgfusepath{stroke,fill}%
}%
\begin{pgfscope}%
\pgfsys@transformshift{6.529487in}{0.331635in}%
\pgfsys@useobject{currentmarker}{}%
\end{pgfscope}%
\end{pgfscope}%
\begin{pgfscope}%
\definecolor{textcolor}{rgb}{0.000000,0.000000,0.000000}%
\pgfsetstrokecolor{textcolor}%
\pgfsetfillcolor{textcolor}%
\pgftext[x=6.529487in,y=0.234413in,,top]{\color{textcolor}\sffamily\fontsize{10.000000}{12.000000}\selectfont \ensuremath{-}2}%
\end{pgfscope}%
\begin{pgfscope}%
\pgfsetbuttcap%
\pgfsetroundjoin%
\definecolor{currentfill}{rgb}{0.000000,0.000000,0.000000}%
\pgfsetfillcolor{currentfill}%
\pgfsetlinewidth{0.803000pt}%
\definecolor{currentstroke}{rgb}{0.000000,0.000000,0.000000}%
\pgfsetstrokecolor{currentstroke}%
\pgfsetdash{}{0pt}%
\pgfsys@defobject{currentmarker}{\pgfqpoint{0.000000in}{-0.048611in}}{\pgfqpoint{0.000000in}{0.000000in}}{%
\pgfpathmoveto{\pgfqpoint{0.000000in}{0.000000in}}%
\pgfpathlineto{\pgfqpoint{0.000000in}{-0.048611in}}%
\pgfusepath{stroke,fill}%
}%
\begin{pgfscope}%
\pgfsys@transformshift{7.786038in}{0.331635in}%
\pgfsys@useobject{currentmarker}{}%
\end{pgfscope}%
\end{pgfscope}%
\begin{pgfscope}%
\definecolor{textcolor}{rgb}{0.000000,0.000000,0.000000}%
\pgfsetstrokecolor{textcolor}%
\pgfsetfillcolor{textcolor}%
\pgftext[x=7.786038in,y=0.234413in,,top]{\color{textcolor}\sffamily\fontsize{10.000000}{12.000000}\selectfont 0}%
\end{pgfscope}%
\begin{pgfscope}%
\pgfsetbuttcap%
\pgfsetroundjoin%
\definecolor{currentfill}{rgb}{0.000000,0.000000,0.000000}%
\pgfsetfillcolor{currentfill}%
\pgfsetlinewidth{0.803000pt}%
\definecolor{currentstroke}{rgb}{0.000000,0.000000,0.000000}%
\pgfsetstrokecolor{currentstroke}%
\pgfsetdash{}{0pt}%
\pgfsys@defobject{currentmarker}{\pgfqpoint{0.000000in}{-0.048611in}}{\pgfqpoint{0.000000in}{0.000000in}}{%
\pgfpathmoveto{\pgfqpoint{0.000000in}{0.000000in}}%
\pgfpathlineto{\pgfqpoint{0.000000in}{-0.048611in}}%
\pgfusepath{stroke,fill}%
}%
\begin{pgfscope}%
\pgfsys@transformshift{9.042589in}{0.331635in}%
\pgfsys@useobject{currentmarker}{}%
\end{pgfscope}%
\end{pgfscope}%
\begin{pgfscope}%
\definecolor{textcolor}{rgb}{0.000000,0.000000,0.000000}%
\pgfsetstrokecolor{textcolor}%
\pgfsetfillcolor{textcolor}%
\pgftext[x=9.042589in,y=0.234413in,,top]{\color{textcolor}\sffamily\fontsize{10.000000}{12.000000}\selectfont 2}%
\end{pgfscope}%
\begin{pgfscope}%
\pgfsetbuttcap%
\pgfsetroundjoin%
\definecolor{currentfill}{rgb}{0.000000,0.000000,0.000000}%
\pgfsetfillcolor{currentfill}%
\pgfsetlinewidth{0.803000pt}%
\definecolor{currentstroke}{rgb}{0.000000,0.000000,0.000000}%
\pgfsetstrokecolor{currentstroke}%
\pgfsetdash{}{0pt}%
\pgfsys@defobject{currentmarker}{\pgfqpoint{-0.048611in}{0.000000in}}{\pgfqpoint{-0.000000in}{0.000000in}}{%
\pgfpathmoveto{\pgfqpoint{-0.000000in}{0.000000in}}%
\pgfpathlineto{\pgfqpoint{-0.048611in}{0.000000in}}%
\pgfusepath{stroke,fill}%
}%
\begin{pgfscope}%
\pgfsys@transformshift{0.393613in}{0.678543in}%
\pgfsys@useobject{currentmarker}{}%
\end{pgfscope}%
\end{pgfscope}%
\begin{pgfscope}%
\definecolor{textcolor}{rgb}{0.000000,0.000000,0.000000}%
\pgfsetstrokecolor{textcolor}%
\pgfsetfillcolor{textcolor}%
\pgftext[x=0.100000in, y=0.625781in, left, base]{\color{textcolor}\sffamily\fontsize{10.000000}{12.000000}\selectfont \ensuremath{-}2}%
\end{pgfscope}%
\begin{pgfscope}%
\pgfsetbuttcap%
\pgfsetroundjoin%
\definecolor{currentfill}{rgb}{0.000000,0.000000,0.000000}%
\pgfsetfillcolor{currentfill}%
\pgfsetlinewidth{0.803000pt}%
\definecolor{currentstroke}{rgb}{0.000000,0.000000,0.000000}%
\pgfsetstrokecolor{currentstroke}%
\pgfsetdash{}{0pt}%
\pgfsys@defobject{currentmarker}{\pgfqpoint{-0.048611in}{0.000000in}}{\pgfqpoint{-0.000000in}{0.000000in}}{%
\pgfpathmoveto{\pgfqpoint{-0.000000in}{0.000000in}}%
\pgfpathlineto{\pgfqpoint{-0.048611in}{0.000000in}}%
\pgfusepath{stroke,fill}%
}%
\begin{pgfscope}%
\pgfsys@transformshift{0.393613in}{2.018524in}%
\pgfsys@useobject{currentmarker}{}%
\end{pgfscope}%
\end{pgfscope}%
\begin{pgfscope}%
\definecolor{textcolor}{rgb}{0.000000,0.000000,0.000000}%
\pgfsetstrokecolor{textcolor}%
\pgfsetfillcolor{textcolor}%
\pgftext[x=0.208025in, y=1.965762in, left, base]{\color{textcolor}\sffamily\fontsize{10.000000}{12.000000}\selectfont 0}%
\end{pgfscope}%
\begin{pgfscope}%
\pgfsetbuttcap%
\pgfsetroundjoin%
\definecolor{currentfill}{rgb}{0.000000,0.000000,0.000000}%
\pgfsetfillcolor{currentfill}%
\pgfsetlinewidth{0.803000pt}%
\definecolor{currentstroke}{rgb}{0.000000,0.000000,0.000000}%
\pgfsetstrokecolor{currentstroke}%
\pgfsetdash{}{0pt}%
\pgfsys@defobject{currentmarker}{\pgfqpoint{-0.048611in}{0.000000in}}{\pgfqpoint{-0.000000in}{0.000000in}}{%
\pgfpathmoveto{\pgfqpoint{-0.000000in}{0.000000in}}%
\pgfpathlineto{\pgfqpoint{-0.048611in}{0.000000in}}%
\pgfusepath{stroke,fill}%
}%
\begin{pgfscope}%
\pgfsys@transformshift{0.393613in}{3.358505in}%
\pgfsys@useobject{currentmarker}{}%
\end{pgfscope}%
\end{pgfscope}%
\begin{pgfscope}%
\definecolor{textcolor}{rgb}{0.000000,0.000000,0.000000}%
\pgfsetstrokecolor{textcolor}%
\pgfsetfillcolor{textcolor}%
\pgftext[x=0.208025in, y=3.305743in, left, base]{\color{textcolor}\sffamily\fontsize{10.000000}{12.000000}\selectfont 2}%
\end{pgfscope}%
\begin{pgfscope}%
\pgfsetbuttcap%
\pgfsetroundjoin%
\definecolor{currentfill}{rgb}{0.000000,0.000000,0.000000}%
\pgfsetfillcolor{currentfill}%
\pgfsetlinewidth{0.803000pt}%
\definecolor{currentstroke}{rgb}{0.000000,0.000000,0.000000}%
\pgfsetstrokecolor{currentstroke}%
\pgfsetdash{}{0pt}%
\pgfsys@defobject{currentmarker}{\pgfqpoint{-0.048611in}{0.000000in}}{\pgfqpoint{-0.000000in}{0.000000in}}{%
\pgfpathmoveto{\pgfqpoint{-0.000000in}{0.000000in}}%
\pgfpathlineto{\pgfqpoint{-0.048611in}{0.000000in}}%
\pgfusepath{stroke,fill}%
}%
\begin{pgfscope}%
\pgfsys@transformshift{0.393613in}{4.698486in}%
\pgfsys@useobject{currentmarker}{}%
\end{pgfscope}%
\end{pgfscope}%
\begin{pgfscope}%
\definecolor{textcolor}{rgb}{0.000000,0.000000,0.000000}%
\pgfsetstrokecolor{textcolor}%
\pgfsetfillcolor{textcolor}%
\pgftext[x=0.208025in, y=4.645724in, left, base]{\color{textcolor}\sffamily\fontsize{10.000000}{12.000000}\selectfont 4}%
\end{pgfscope}%
\begin{pgfscope}%
\pgfsetbuttcap%
\pgfsetroundjoin%
\definecolor{currentfill}{rgb}{0.000000,0.000000,0.000000}%
\pgfsetfillcolor{currentfill}%
\pgfsetlinewidth{0.803000pt}%
\definecolor{currentstroke}{rgb}{0.000000,0.000000,0.000000}%
\pgfsetstrokecolor{currentstroke}%
\pgfsetdash{}{0pt}%
\pgfsys@defobject{currentmarker}{\pgfqpoint{-0.048611in}{0.000000in}}{\pgfqpoint{-0.000000in}{0.000000in}}{%
\pgfpathmoveto{\pgfqpoint{-0.000000in}{0.000000in}}%
\pgfpathlineto{\pgfqpoint{-0.048611in}{0.000000in}}%
\pgfusepath{stroke,fill}%
}%
\begin{pgfscope}%
\pgfsys@transformshift{0.393613in}{6.038467in}%
\pgfsys@useobject{currentmarker}{}%
\end{pgfscope}%
\end{pgfscope}%
\begin{pgfscope}%
\definecolor{textcolor}{rgb}{0.000000,0.000000,0.000000}%
\pgfsetstrokecolor{textcolor}%
\pgfsetfillcolor{textcolor}%
\pgftext[x=0.208025in, y=5.985705in, left, base]{\color{textcolor}\sffamily\fontsize{10.000000}{12.000000}\selectfont 6}%
\end{pgfscope}%
\begin{pgfscope}%
\pgfsetbuttcap%
\pgfsetroundjoin%
\definecolor{currentfill}{rgb}{0.000000,0.000000,0.000000}%
\pgfsetfillcolor{currentfill}%
\pgfsetlinewidth{0.803000pt}%
\definecolor{currentstroke}{rgb}{0.000000,0.000000,0.000000}%
\pgfsetstrokecolor{currentstroke}%
\pgfsetdash{}{0pt}%
\pgfsys@defobject{currentmarker}{\pgfqpoint{-0.048611in}{0.000000in}}{\pgfqpoint{-0.000000in}{0.000000in}}{%
\pgfpathmoveto{\pgfqpoint{-0.000000in}{0.000000in}}%
\pgfpathlineto{\pgfqpoint{-0.048611in}{0.000000in}}%
\pgfusepath{stroke,fill}%
}%
\begin{pgfscope}%
\pgfsys@transformshift{0.393613in}{7.378448in}%
\pgfsys@useobject{currentmarker}{}%
\end{pgfscope}%
\end{pgfscope}%
\begin{pgfscope}%
\definecolor{textcolor}{rgb}{0.000000,0.000000,0.000000}%
\pgfsetstrokecolor{textcolor}%
\pgfsetfillcolor{textcolor}%
\pgftext[x=0.208025in, y=7.325687in, left, base]{\color{textcolor}\sffamily\fontsize{10.000000}{12.000000}\selectfont 8}%
\end{pgfscope}%
\begin{pgfscope}%
\pgfpathrectangle{\pgfqpoint{0.393613in}{0.331635in}}{\pgfqpoint{9.300000in}{7.700000in}}%
\pgfusepath{clip}%
\pgfsetrectcap%
\pgfsetroundjoin%
\pgfsetlinewidth{1.505625pt}%
\definecolor{currentstroke}{rgb}{0.631373,0.788235,0.956863}%
\pgfsetstrokecolor{currentstroke}%
\pgfsetstrokeopacity{0.800000}%
\pgfsetdash{}{0pt}%
\pgfpathmoveto{\pgfqpoint{2.878164in}{1.558141in}}%
\pgfpathlineto{\pgfqpoint{2.941011in}{4.613488in}}%
\pgfusepath{stroke}%
\end{pgfscope}%
\begin{pgfscope}%
\pgfpathrectangle{\pgfqpoint{0.393613in}{0.331635in}}{\pgfqpoint{9.300000in}{7.700000in}}%
\pgfusepath{clip}%
\pgfsetrectcap%
\pgfsetroundjoin%
\pgfsetlinewidth{1.505625pt}%
\definecolor{currentstroke}{rgb}{0.631373,0.788235,0.956863}%
\pgfsetstrokecolor{currentstroke}%
\pgfsetstrokeopacity{0.800000}%
\pgfsetdash{}{0pt}%
\pgfpathmoveto{\pgfqpoint{1.610944in}{5.142869in}}%
\pgfpathlineto{\pgfqpoint{2.941011in}{4.613488in}}%
\pgfusepath{stroke}%
\end{pgfscope}%
\begin{pgfscope}%
\pgfpathrectangle{\pgfqpoint{0.393613in}{0.331635in}}{\pgfqpoint{9.300000in}{7.700000in}}%
\pgfusepath{clip}%
\pgfsetrectcap%
\pgfsetroundjoin%
\pgfsetlinewidth{1.505625pt}%
\definecolor{currentstroke}{rgb}{0.631373,0.788235,0.956863}%
\pgfsetstrokecolor{currentstroke}%
\pgfsetstrokeopacity{0.800000}%
\pgfsetdash{}{0pt}%
\pgfpathmoveto{\pgfqpoint{4.136764in}{5.889937in}}%
\pgfpathlineto{\pgfqpoint{2.941011in}{4.613488in}}%
\pgfusepath{stroke}%
\end{pgfscope}%
\begin{pgfscope}%
\pgfpathrectangle{\pgfqpoint{0.393613in}{0.331635in}}{\pgfqpoint{9.300000in}{7.700000in}}%
\pgfusepath{clip}%
\pgfsetrectcap%
\pgfsetroundjoin%
\pgfsetlinewidth{1.505625pt}%
\definecolor{currentstroke}{rgb}{0.631373,0.788235,0.956863}%
\pgfsetstrokecolor{currentstroke}%
\pgfsetstrokeopacity{0.800000}%
\pgfsetdash{}{0pt}%
\pgfpathmoveto{\pgfqpoint{3.118402in}{4.388061in}}%
\pgfpathlineto{\pgfqpoint{2.941011in}{4.613488in}}%
\pgfusepath{stroke}%
\end{pgfscope}%
\begin{pgfscope}%
\pgfpathrectangle{\pgfqpoint{0.393613in}{0.331635in}}{\pgfqpoint{9.300000in}{7.700000in}}%
\pgfusepath{clip}%
\pgfsetrectcap%
\pgfsetroundjoin%
\pgfsetlinewidth{1.505625pt}%
\definecolor{currentstroke}{rgb}{0.631373,0.788235,0.956863}%
\pgfsetstrokecolor{currentstroke}%
\pgfsetstrokeopacity{0.800000}%
\pgfsetdash{}{0pt}%
\pgfpathmoveto{\pgfqpoint{3.792263in}{2.017332in}}%
\pgfpathlineto{\pgfqpoint{2.941011in}{4.613488in}}%
\pgfusepath{stroke}%
\end{pgfscope}%
\begin{pgfscope}%
\pgfpathrectangle{\pgfqpoint{0.393613in}{0.331635in}}{\pgfqpoint{9.300000in}{7.700000in}}%
\pgfusepath{clip}%
\pgfsetrectcap%
\pgfsetroundjoin%
\pgfsetlinewidth{1.505625pt}%
\definecolor{currentstroke}{rgb}{0.631373,0.788235,0.956863}%
\pgfsetstrokecolor{currentstroke}%
\pgfsetstrokeopacity{0.800000}%
\pgfsetdash{}{0pt}%
\pgfpathmoveto{\pgfqpoint{1.145785in}{5.847310in}}%
\pgfpathlineto{\pgfqpoint{2.941011in}{4.613488in}}%
\pgfusepath{stroke}%
\end{pgfscope}%
\begin{pgfscope}%
\pgfpathrectangle{\pgfqpoint{0.393613in}{0.331635in}}{\pgfqpoint{9.300000in}{7.700000in}}%
\pgfusepath{clip}%
\pgfsetrectcap%
\pgfsetroundjoin%
\pgfsetlinewidth{1.505625pt}%
\definecolor{currentstroke}{rgb}{0.631373,0.788235,0.956863}%
\pgfsetstrokecolor{currentstroke}%
\pgfsetstrokeopacity{0.800000}%
\pgfsetdash{}{0pt}%
\pgfpathmoveto{\pgfqpoint{2.952892in}{1.933973in}}%
\pgfpathlineto{\pgfqpoint{2.941011in}{4.613488in}}%
\pgfusepath{stroke}%
\end{pgfscope}%
\begin{pgfscope}%
\pgfpathrectangle{\pgfqpoint{0.393613in}{0.331635in}}{\pgfqpoint{9.300000in}{7.700000in}}%
\pgfusepath{clip}%
\pgfsetrectcap%
\pgfsetroundjoin%
\pgfsetlinewidth{1.505625pt}%
\definecolor{currentstroke}{rgb}{0.631373,0.788235,0.956863}%
\pgfsetstrokecolor{currentstroke}%
\pgfsetstrokeopacity{0.800000}%
\pgfsetdash{}{0pt}%
\pgfpathmoveto{\pgfqpoint{4.435471in}{5.567747in}}%
\pgfpathlineto{\pgfqpoint{2.941011in}{4.613488in}}%
\pgfusepath{stroke}%
\end{pgfscope}%
\begin{pgfscope}%
\pgfpathrectangle{\pgfqpoint{0.393613in}{0.331635in}}{\pgfqpoint{9.300000in}{7.700000in}}%
\pgfusepath{clip}%
\pgfsetrectcap%
\pgfsetroundjoin%
\pgfsetlinewidth{1.505625pt}%
\definecolor{currentstroke}{rgb}{0.631373,0.788235,0.956863}%
\pgfsetstrokecolor{currentstroke}%
\pgfsetstrokeopacity{0.800000}%
\pgfsetdash{}{0pt}%
\pgfpathmoveto{\pgfqpoint{1.974238in}{5.809332in}}%
\pgfpathlineto{\pgfqpoint{2.941011in}{4.613488in}}%
\pgfusepath{stroke}%
\end{pgfscope}%
\begin{pgfscope}%
\pgfpathrectangle{\pgfqpoint{0.393613in}{0.331635in}}{\pgfqpoint{9.300000in}{7.700000in}}%
\pgfusepath{clip}%
\pgfsetrectcap%
\pgfsetroundjoin%
\pgfsetlinewidth{1.505625pt}%
\definecolor{currentstroke}{rgb}{0.631373,0.788235,0.956863}%
\pgfsetstrokecolor{currentstroke}%
\pgfsetstrokeopacity{0.800000}%
\pgfsetdash{}{0pt}%
\pgfpathmoveto{\pgfqpoint{2.379728in}{6.710730in}}%
\pgfpathlineto{\pgfqpoint{2.941011in}{4.613488in}}%
\pgfusepath{stroke}%
\end{pgfscope}%
\begin{pgfscope}%
\pgfpathrectangle{\pgfqpoint{0.393613in}{0.331635in}}{\pgfqpoint{9.300000in}{7.700000in}}%
\pgfusepath{clip}%
\pgfsetrectcap%
\pgfsetroundjoin%
\pgfsetlinewidth{1.505625pt}%
\definecolor{currentstroke}{rgb}{0.631373,0.788235,0.956863}%
\pgfsetstrokecolor{currentstroke}%
\pgfsetstrokeopacity{0.800000}%
\pgfsetdash{}{0pt}%
\pgfpathmoveto{\pgfqpoint{2.765242in}{4.891781in}}%
\pgfpathlineto{\pgfqpoint{2.941011in}{4.613488in}}%
\pgfusepath{stroke}%
\end{pgfscope}%
\begin{pgfscope}%
\pgfpathrectangle{\pgfqpoint{0.393613in}{0.331635in}}{\pgfqpoint{9.300000in}{7.700000in}}%
\pgfusepath{clip}%
\pgfsetrectcap%
\pgfsetroundjoin%
\pgfsetlinewidth{1.505625pt}%
\definecolor{currentstroke}{rgb}{0.631373,0.788235,0.956863}%
\pgfsetstrokecolor{currentstroke}%
\pgfsetstrokeopacity{0.800000}%
\pgfsetdash{}{0pt}%
\pgfpathmoveto{\pgfqpoint{2.626630in}{6.543168in}}%
\pgfpathlineto{\pgfqpoint{2.941011in}{4.613488in}}%
\pgfusepath{stroke}%
\end{pgfscope}%
\begin{pgfscope}%
\pgfpathrectangle{\pgfqpoint{0.393613in}{0.331635in}}{\pgfqpoint{9.300000in}{7.700000in}}%
\pgfusepath{clip}%
\pgfsetrectcap%
\pgfsetroundjoin%
\pgfsetlinewidth{1.505625pt}%
\definecolor{currentstroke}{rgb}{0.631373,0.788235,0.956863}%
\pgfsetstrokecolor{currentstroke}%
\pgfsetstrokeopacity{0.800000}%
\pgfsetdash{}{0pt}%
\pgfpathmoveto{\pgfqpoint{3.301695in}{5.327555in}}%
\pgfpathlineto{\pgfqpoint{2.941011in}{4.613488in}}%
\pgfusepath{stroke}%
\end{pgfscope}%
\begin{pgfscope}%
\pgfpathrectangle{\pgfqpoint{0.393613in}{0.331635in}}{\pgfqpoint{9.300000in}{7.700000in}}%
\pgfusepath{clip}%
\pgfsetrectcap%
\pgfsetroundjoin%
\pgfsetlinewidth{1.505625pt}%
\definecolor{currentstroke}{rgb}{0.631373,0.788235,0.956863}%
\pgfsetstrokecolor{currentstroke}%
\pgfsetstrokeopacity{0.800000}%
\pgfsetdash{}{0pt}%
\pgfpathmoveto{\pgfqpoint{0.968107in}{5.897384in}}%
\pgfpathlineto{\pgfqpoint{2.941011in}{4.613488in}}%
\pgfusepath{stroke}%
\end{pgfscope}%
\begin{pgfscope}%
\pgfpathrectangle{\pgfqpoint{0.393613in}{0.331635in}}{\pgfqpoint{9.300000in}{7.700000in}}%
\pgfusepath{clip}%
\pgfsetrectcap%
\pgfsetroundjoin%
\pgfsetlinewidth{1.505625pt}%
\definecolor{currentstroke}{rgb}{0.631373,0.788235,0.956863}%
\pgfsetstrokecolor{currentstroke}%
\pgfsetstrokeopacity{0.800000}%
\pgfsetdash{}{0pt}%
\pgfpathmoveto{\pgfqpoint{4.228775in}{6.172529in}}%
\pgfpathlineto{\pgfqpoint{2.941011in}{4.613488in}}%
\pgfusepath{stroke}%
\end{pgfscope}%
\begin{pgfscope}%
\pgfpathrectangle{\pgfqpoint{0.393613in}{0.331635in}}{\pgfqpoint{9.300000in}{7.700000in}}%
\pgfusepath{clip}%
\pgfsetrectcap%
\pgfsetroundjoin%
\pgfsetlinewidth{1.505625pt}%
\definecolor{currentstroke}{rgb}{0.631373,0.788235,0.956863}%
\pgfsetstrokecolor{currentstroke}%
\pgfsetstrokeopacity{0.800000}%
\pgfsetdash{}{0pt}%
\pgfpathmoveto{\pgfqpoint{2.425460in}{7.032380in}}%
\pgfpathlineto{\pgfqpoint{2.941011in}{4.613488in}}%
\pgfusepath{stroke}%
\end{pgfscope}%
\begin{pgfscope}%
\pgfpathrectangle{\pgfqpoint{0.393613in}{0.331635in}}{\pgfqpoint{9.300000in}{7.700000in}}%
\pgfusepath{clip}%
\pgfsetrectcap%
\pgfsetroundjoin%
\pgfsetlinewidth{1.505625pt}%
\definecolor{currentstroke}{rgb}{0.631373,0.788235,0.956863}%
\pgfsetstrokecolor{currentstroke}%
\pgfsetstrokeopacity{0.800000}%
\pgfsetdash{}{0pt}%
\pgfpathmoveto{\pgfqpoint{2.491424in}{5.707373in}}%
\pgfpathlineto{\pgfqpoint{2.941011in}{4.613488in}}%
\pgfusepath{stroke}%
\end{pgfscope}%
\begin{pgfscope}%
\pgfpathrectangle{\pgfqpoint{0.393613in}{0.331635in}}{\pgfqpoint{9.300000in}{7.700000in}}%
\pgfusepath{clip}%
\pgfsetrectcap%
\pgfsetroundjoin%
\pgfsetlinewidth{1.505625pt}%
\definecolor{currentstroke}{rgb}{0.631373,0.788235,0.956863}%
\pgfsetstrokecolor{currentstroke}%
\pgfsetstrokeopacity{0.800000}%
\pgfsetdash{}{0pt}%
\pgfpathmoveto{\pgfqpoint{5.405752in}{7.071695in}}%
\pgfpathlineto{\pgfqpoint{2.941011in}{4.613488in}}%
\pgfusepath{stroke}%
\end{pgfscope}%
\begin{pgfscope}%
\pgfpathrectangle{\pgfqpoint{0.393613in}{0.331635in}}{\pgfqpoint{9.300000in}{7.700000in}}%
\pgfusepath{clip}%
\pgfsetrectcap%
\pgfsetroundjoin%
\pgfsetlinewidth{1.505625pt}%
\definecolor{currentstroke}{rgb}{0.631373,0.788235,0.956863}%
\pgfsetstrokecolor{currentstroke}%
\pgfsetstrokeopacity{0.800000}%
\pgfsetdash{}{0pt}%
\pgfpathmoveto{\pgfqpoint{4.029243in}{1.979124in}}%
\pgfpathlineto{\pgfqpoint{2.941011in}{4.613488in}}%
\pgfusepath{stroke}%
\end{pgfscope}%
\begin{pgfscope}%
\pgfpathrectangle{\pgfqpoint{0.393613in}{0.331635in}}{\pgfqpoint{9.300000in}{7.700000in}}%
\pgfusepath{clip}%
\pgfsetrectcap%
\pgfsetroundjoin%
\pgfsetlinewidth{1.505625pt}%
\definecolor{currentstroke}{rgb}{0.631373,0.788235,0.956863}%
\pgfsetstrokecolor{currentstroke}%
\pgfsetstrokeopacity{0.800000}%
\pgfsetdash{}{0pt}%
\pgfpathmoveto{\pgfqpoint{3.912187in}{3.998408in}}%
\pgfpathlineto{\pgfqpoint{2.941011in}{4.613488in}}%
\pgfusepath{stroke}%
\end{pgfscope}%
\begin{pgfscope}%
\pgfpathrectangle{\pgfqpoint{0.393613in}{0.331635in}}{\pgfqpoint{9.300000in}{7.700000in}}%
\pgfusepath{clip}%
\pgfsetrectcap%
\pgfsetroundjoin%
\pgfsetlinewidth{1.505625pt}%
\definecolor{currentstroke}{rgb}{0.631373,0.788235,0.956863}%
\pgfsetstrokecolor{currentstroke}%
\pgfsetstrokeopacity{0.800000}%
\pgfsetdash{}{0pt}%
\pgfpathmoveto{\pgfqpoint{3.189809in}{3.286603in}}%
\pgfpathlineto{\pgfqpoint{2.941011in}{4.613488in}}%
\pgfusepath{stroke}%
\end{pgfscope}%
\begin{pgfscope}%
\pgfpathrectangle{\pgfqpoint{0.393613in}{0.331635in}}{\pgfqpoint{9.300000in}{7.700000in}}%
\pgfusepath{clip}%
\pgfsetrectcap%
\pgfsetroundjoin%
\pgfsetlinewidth{1.505625pt}%
\definecolor{currentstroke}{rgb}{0.631373,0.788235,0.956863}%
\pgfsetstrokecolor{currentstroke}%
\pgfsetstrokeopacity{0.800000}%
\pgfsetdash{}{0pt}%
\pgfpathmoveto{\pgfqpoint{2.615024in}{4.490516in}}%
\pgfpathlineto{\pgfqpoint{2.941011in}{4.613488in}}%
\pgfusepath{stroke}%
\end{pgfscope}%
\begin{pgfscope}%
\pgfpathrectangle{\pgfqpoint{0.393613in}{0.331635in}}{\pgfqpoint{9.300000in}{7.700000in}}%
\pgfusepath{clip}%
\pgfsetrectcap%
\pgfsetroundjoin%
\pgfsetlinewidth{1.505625pt}%
\definecolor{currentstroke}{rgb}{0.631373,0.788235,0.956863}%
\pgfsetstrokecolor{currentstroke}%
\pgfsetstrokeopacity{0.800000}%
\pgfsetdash{}{0pt}%
\pgfpathmoveto{\pgfqpoint{2.589699in}{5.191672in}}%
\pgfpathlineto{\pgfqpoint{2.941011in}{4.613488in}}%
\pgfusepath{stroke}%
\end{pgfscope}%
\begin{pgfscope}%
\pgfpathrectangle{\pgfqpoint{0.393613in}{0.331635in}}{\pgfqpoint{9.300000in}{7.700000in}}%
\pgfusepath{clip}%
\pgfsetrectcap%
\pgfsetroundjoin%
\pgfsetlinewidth{1.505625pt}%
\definecolor{currentstroke}{rgb}{0.631373,0.788235,0.956863}%
\pgfsetstrokecolor{currentstroke}%
\pgfsetstrokeopacity{0.800000}%
\pgfsetdash{}{0pt}%
\pgfpathmoveto{\pgfqpoint{1.233235in}{3.968830in}}%
\pgfpathlineto{\pgfqpoint{2.941011in}{4.613488in}}%
\pgfusepath{stroke}%
\end{pgfscope}%
\begin{pgfscope}%
\pgfpathrectangle{\pgfqpoint{0.393613in}{0.331635in}}{\pgfqpoint{9.300000in}{7.700000in}}%
\pgfusepath{clip}%
\pgfsetrectcap%
\pgfsetroundjoin%
\pgfsetlinewidth{1.505625pt}%
\definecolor{currentstroke}{rgb}{0.631373,0.788235,0.956863}%
\pgfsetstrokecolor{currentstroke}%
\pgfsetstrokeopacity{0.800000}%
\pgfsetdash{}{0pt}%
\pgfpathmoveto{\pgfqpoint{3.810785in}{1.946484in}}%
\pgfpathlineto{\pgfqpoint{2.941011in}{4.613488in}}%
\pgfusepath{stroke}%
\end{pgfscope}%
\begin{pgfscope}%
\pgfpathrectangle{\pgfqpoint{0.393613in}{0.331635in}}{\pgfqpoint{9.300000in}{7.700000in}}%
\pgfusepath{clip}%
\pgfsetrectcap%
\pgfsetroundjoin%
\pgfsetlinewidth{1.505625pt}%
\definecolor{currentstroke}{rgb}{0.631373,0.788235,0.956863}%
\pgfsetstrokecolor{currentstroke}%
\pgfsetstrokeopacity{0.800000}%
\pgfsetdash{}{0pt}%
\pgfpathmoveto{\pgfqpoint{2.870620in}{1.775187in}}%
\pgfpathlineto{\pgfqpoint{2.941011in}{4.613488in}}%
\pgfusepath{stroke}%
\end{pgfscope}%
\begin{pgfscope}%
\pgfpathrectangle{\pgfqpoint{0.393613in}{0.331635in}}{\pgfqpoint{9.300000in}{7.700000in}}%
\pgfusepath{clip}%
\pgfsetrectcap%
\pgfsetroundjoin%
\pgfsetlinewidth{1.505625pt}%
\definecolor{currentstroke}{rgb}{0.631373,0.788235,0.956863}%
\pgfsetstrokecolor{currentstroke}%
\pgfsetstrokeopacity{0.800000}%
\pgfsetdash{}{0pt}%
\pgfpathmoveto{\pgfqpoint{2.247275in}{4.467915in}}%
\pgfpathlineto{\pgfqpoint{2.941011in}{4.613488in}}%
\pgfusepath{stroke}%
\end{pgfscope}%
\begin{pgfscope}%
\pgfpathrectangle{\pgfqpoint{0.393613in}{0.331635in}}{\pgfqpoint{9.300000in}{7.700000in}}%
\pgfusepath{clip}%
\pgfsetrectcap%
\pgfsetroundjoin%
\pgfsetlinewidth{1.505625pt}%
\definecolor{currentstroke}{rgb}{0.631373,0.788235,0.956863}%
\pgfsetstrokecolor{currentstroke}%
\pgfsetstrokeopacity{0.800000}%
\pgfsetdash{}{0pt}%
\pgfpathmoveto{\pgfqpoint{3.278707in}{2.508240in}}%
\pgfpathlineto{\pgfqpoint{2.941011in}{4.613488in}}%
\pgfusepath{stroke}%
\end{pgfscope}%
\begin{pgfscope}%
\pgfpathrectangle{\pgfqpoint{0.393613in}{0.331635in}}{\pgfqpoint{9.300000in}{7.700000in}}%
\pgfusepath{clip}%
\pgfsetrectcap%
\pgfsetroundjoin%
\pgfsetlinewidth{1.505625pt}%
\definecolor{currentstroke}{rgb}{0.631373,0.788235,0.956863}%
\pgfsetstrokecolor{currentstroke}%
\pgfsetstrokeopacity{0.800000}%
\pgfsetdash{}{0pt}%
\pgfpathmoveto{\pgfqpoint{4.507490in}{0.681635in}}%
\pgfpathlineto{\pgfqpoint{2.941011in}{4.613488in}}%
\pgfusepath{stroke}%
\end{pgfscope}%
\begin{pgfscope}%
\pgfpathrectangle{\pgfqpoint{0.393613in}{0.331635in}}{\pgfqpoint{9.300000in}{7.700000in}}%
\pgfusepath{clip}%
\pgfsetrectcap%
\pgfsetroundjoin%
\pgfsetlinewidth{1.505625pt}%
\definecolor{currentstroke}{rgb}{0.631373,0.788235,0.956863}%
\pgfsetstrokecolor{currentstroke}%
\pgfsetstrokeopacity{0.800000}%
\pgfsetdash{}{0pt}%
\pgfpathmoveto{\pgfqpoint{2.850278in}{7.174436in}}%
\pgfpathlineto{\pgfqpoint{2.941011in}{4.613488in}}%
\pgfusepath{stroke}%
\end{pgfscope}%
\begin{pgfscope}%
\pgfpathrectangle{\pgfqpoint{0.393613in}{0.331635in}}{\pgfqpoint{9.300000in}{7.700000in}}%
\pgfusepath{clip}%
\pgfsetrectcap%
\pgfsetroundjoin%
\pgfsetlinewidth{1.505625pt}%
\definecolor{currentstroke}{rgb}{0.631373,0.788235,0.956863}%
\pgfsetstrokecolor{currentstroke}%
\pgfsetstrokeopacity{0.800000}%
\pgfsetdash{}{0pt}%
\pgfpathmoveto{\pgfqpoint{1.938431in}{5.037958in}}%
\pgfpathlineto{\pgfqpoint{2.941011in}{4.613488in}}%
\pgfusepath{stroke}%
\end{pgfscope}%
\begin{pgfscope}%
\pgfpathrectangle{\pgfqpoint{0.393613in}{0.331635in}}{\pgfqpoint{9.300000in}{7.700000in}}%
\pgfusepath{clip}%
\pgfsetrectcap%
\pgfsetroundjoin%
\pgfsetlinewidth{1.505625pt}%
\definecolor{currentstroke}{rgb}{0.631373,0.788235,0.956863}%
\pgfsetstrokecolor{currentstroke}%
\pgfsetstrokeopacity{0.800000}%
\pgfsetdash{}{0pt}%
\pgfpathmoveto{\pgfqpoint{2.919129in}{0.931069in}}%
\pgfpathlineto{\pgfqpoint{2.941011in}{4.613488in}}%
\pgfusepath{stroke}%
\end{pgfscope}%
\begin{pgfscope}%
\pgfpathrectangle{\pgfqpoint{0.393613in}{0.331635in}}{\pgfqpoint{9.300000in}{7.700000in}}%
\pgfusepath{clip}%
\pgfsetrectcap%
\pgfsetroundjoin%
\pgfsetlinewidth{1.505625pt}%
\definecolor{currentstroke}{rgb}{0.631373,0.788235,0.956863}%
\pgfsetstrokecolor{currentstroke}%
\pgfsetstrokeopacity{0.800000}%
\pgfsetdash{}{0pt}%
\pgfpathmoveto{\pgfqpoint{2.553778in}{6.329677in}}%
\pgfpathlineto{\pgfqpoint{2.941011in}{4.613488in}}%
\pgfusepath{stroke}%
\end{pgfscope}%
\begin{pgfscope}%
\pgfpathrectangle{\pgfqpoint{0.393613in}{0.331635in}}{\pgfqpoint{9.300000in}{7.700000in}}%
\pgfusepath{clip}%
\pgfsetrectcap%
\pgfsetroundjoin%
\pgfsetlinewidth{1.505625pt}%
\definecolor{currentstroke}{rgb}{0.631373,0.788235,0.956863}%
\pgfsetstrokecolor{currentstroke}%
\pgfsetstrokeopacity{0.800000}%
\pgfsetdash{}{0pt}%
\pgfpathmoveto{\pgfqpoint{2.490513in}{3.775065in}}%
\pgfpathlineto{\pgfqpoint{2.941011in}{4.613488in}}%
\pgfusepath{stroke}%
\end{pgfscope}%
\begin{pgfscope}%
\pgfpathrectangle{\pgfqpoint{0.393613in}{0.331635in}}{\pgfqpoint{9.300000in}{7.700000in}}%
\pgfusepath{clip}%
\pgfsetrectcap%
\pgfsetroundjoin%
\pgfsetlinewidth{1.505625pt}%
\definecolor{currentstroke}{rgb}{0.631373,0.788235,0.956863}%
\pgfsetstrokecolor{currentstroke}%
\pgfsetstrokeopacity{0.800000}%
\pgfsetdash{}{0pt}%
\pgfpathmoveto{\pgfqpoint{0.816340in}{4.757595in}}%
\pgfpathlineto{\pgfqpoint{2.941011in}{4.613488in}}%
\pgfusepath{stroke}%
\end{pgfscope}%
\begin{pgfscope}%
\pgfpathrectangle{\pgfqpoint{0.393613in}{0.331635in}}{\pgfqpoint{9.300000in}{7.700000in}}%
\pgfusepath{clip}%
\pgfsetrectcap%
\pgfsetroundjoin%
\pgfsetlinewidth{1.505625pt}%
\definecolor{currentstroke}{rgb}{0.631373,0.788235,0.956863}%
\pgfsetstrokecolor{currentstroke}%
\pgfsetstrokeopacity{0.800000}%
\pgfsetdash{}{0pt}%
\pgfpathmoveto{\pgfqpoint{1.972243in}{4.014754in}}%
\pgfpathlineto{\pgfqpoint{2.941011in}{4.613488in}}%
\pgfusepath{stroke}%
\end{pgfscope}%
\begin{pgfscope}%
\pgfpathrectangle{\pgfqpoint{0.393613in}{0.331635in}}{\pgfqpoint{9.300000in}{7.700000in}}%
\pgfusepath{clip}%
\pgfsetrectcap%
\pgfsetroundjoin%
\pgfsetlinewidth{1.505625pt}%
\definecolor{currentstroke}{rgb}{0.631373,0.788235,0.956863}%
\pgfsetstrokecolor{currentstroke}%
\pgfsetstrokeopacity{0.800000}%
\pgfsetdash{}{0pt}%
\pgfpathmoveto{\pgfqpoint{3.271288in}{1.460560in}}%
\pgfpathlineto{\pgfqpoint{2.941011in}{4.613488in}}%
\pgfusepath{stroke}%
\end{pgfscope}%
\begin{pgfscope}%
\pgfpathrectangle{\pgfqpoint{0.393613in}{0.331635in}}{\pgfqpoint{9.300000in}{7.700000in}}%
\pgfusepath{clip}%
\pgfsetrectcap%
\pgfsetroundjoin%
\pgfsetlinewidth{1.505625pt}%
\definecolor{currentstroke}{rgb}{0.631373,0.788235,0.956863}%
\pgfsetstrokecolor{currentstroke}%
\pgfsetstrokeopacity{0.800000}%
\pgfsetdash{}{0pt}%
\pgfpathmoveto{\pgfqpoint{2.443533in}{5.252316in}}%
\pgfpathlineto{\pgfqpoint{2.941011in}{4.613488in}}%
\pgfusepath{stroke}%
\end{pgfscope}%
\begin{pgfscope}%
\pgfpathrectangle{\pgfqpoint{0.393613in}{0.331635in}}{\pgfqpoint{9.300000in}{7.700000in}}%
\pgfusepath{clip}%
\pgfsetrectcap%
\pgfsetroundjoin%
\pgfsetlinewidth{1.505625pt}%
\definecolor{currentstroke}{rgb}{0.631373,0.788235,0.956863}%
\pgfsetstrokecolor{currentstroke}%
\pgfsetstrokeopacity{0.800000}%
\pgfsetdash{}{0pt}%
\pgfpathmoveto{\pgfqpoint{3.808024in}{7.681635in}}%
\pgfpathlineto{\pgfqpoint{2.941011in}{4.613488in}}%
\pgfusepath{stroke}%
\end{pgfscope}%
\begin{pgfscope}%
\pgfpathrectangle{\pgfqpoint{0.393613in}{0.331635in}}{\pgfqpoint{9.300000in}{7.700000in}}%
\pgfusepath{clip}%
\pgfsetrectcap%
\pgfsetroundjoin%
\pgfsetlinewidth{1.505625pt}%
\definecolor{currentstroke}{rgb}{0.631373,0.788235,0.956863}%
\pgfsetstrokecolor{currentstroke}%
\pgfsetstrokeopacity{0.800000}%
\pgfsetdash{}{0pt}%
\pgfpathmoveto{\pgfqpoint{1.545120in}{6.840743in}}%
\pgfpathlineto{\pgfqpoint{2.941011in}{4.613488in}}%
\pgfusepath{stroke}%
\end{pgfscope}%
\begin{pgfscope}%
\pgfpathrectangle{\pgfqpoint{0.393613in}{0.331635in}}{\pgfqpoint{9.300000in}{7.700000in}}%
\pgfusepath{clip}%
\pgfsetrectcap%
\pgfsetroundjoin%
\pgfsetlinewidth{1.505625pt}%
\definecolor{currentstroke}{rgb}{0.631373,0.788235,0.956863}%
\pgfsetstrokecolor{currentstroke}%
\pgfsetstrokeopacity{0.800000}%
\pgfsetdash{}{0pt}%
\pgfpathmoveto{\pgfqpoint{4.495356in}{6.333810in}}%
\pgfpathlineto{\pgfqpoint{2.941011in}{4.613488in}}%
\pgfusepath{stroke}%
\end{pgfscope}%
\begin{pgfscope}%
\pgfpathrectangle{\pgfqpoint{0.393613in}{0.331635in}}{\pgfqpoint{9.300000in}{7.700000in}}%
\pgfusepath{clip}%
\pgfsetrectcap%
\pgfsetroundjoin%
\pgfsetlinewidth{1.505625pt}%
\definecolor{currentstroke}{rgb}{0.631373,0.788235,0.956863}%
\pgfsetstrokecolor{currentstroke}%
\pgfsetstrokeopacity{0.800000}%
\pgfsetdash{}{0pt}%
\pgfpathmoveto{\pgfqpoint{2.111783in}{3.199242in}}%
\pgfpathlineto{\pgfqpoint{2.941011in}{4.613488in}}%
\pgfusepath{stroke}%
\end{pgfscope}%
\begin{pgfscope}%
\pgfpathrectangle{\pgfqpoint{0.393613in}{0.331635in}}{\pgfqpoint{9.300000in}{7.700000in}}%
\pgfusepath{clip}%
\pgfsetrectcap%
\pgfsetroundjoin%
\pgfsetlinewidth{1.505625pt}%
\definecolor{currentstroke}{rgb}{0.631373,0.788235,0.956863}%
\pgfsetstrokecolor{currentstroke}%
\pgfsetstrokeopacity{0.800000}%
\pgfsetdash{}{0pt}%
\pgfpathmoveto{\pgfqpoint{2.968097in}{5.687545in}}%
\pgfpathlineto{\pgfqpoint{2.941011in}{4.613488in}}%
\pgfusepath{stroke}%
\end{pgfscope}%
\begin{pgfscope}%
\pgfpathrectangle{\pgfqpoint{0.393613in}{0.331635in}}{\pgfqpoint{9.300000in}{7.700000in}}%
\pgfusepath{clip}%
\pgfsetrectcap%
\pgfsetroundjoin%
\pgfsetlinewidth{1.505625pt}%
\definecolor{currentstroke}{rgb}{0.631373,0.788235,0.956863}%
\pgfsetstrokecolor{currentstroke}%
\pgfsetstrokeopacity{0.800000}%
\pgfsetdash{}{0pt}%
\pgfpathmoveto{\pgfqpoint{2.017579in}{2.412400in}}%
\pgfpathlineto{\pgfqpoint{2.941011in}{4.613488in}}%
\pgfusepath{stroke}%
\end{pgfscope}%
\begin{pgfscope}%
\pgfpathrectangle{\pgfqpoint{0.393613in}{0.331635in}}{\pgfqpoint{9.300000in}{7.700000in}}%
\pgfusepath{clip}%
\pgfsetrectcap%
\pgfsetroundjoin%
\pgfsetlinewidth{1.505625pt}%
\definecolor{currentstroke}{rgb}{0.631373,0.788235,0.956863}%
\pgfsetstrokecolor{currentstroke}%
\pgfsetstrokeopacity{0.800000}%
\pgfsetdash{}{0pt}%
\pgfpathmoveto{\pgfqpoint{4.875280in}{6.873638in}}%
\pgfpathlineto{\pgfqpoint{2.941011in}{4.613488in}}%
\pgfusepath{stroke}%
\end{pgfscope}%
\begin{pgfscope}%
\pgfpathrectangle{\pgfqpoint{0.393613in}{0.331635in}}{\pgfqpoint{9.300000in}{7.700000in}}%
\pgfusepath{clip}%
\pgfsetrectcap%
\pgfsetroundjoin%
\pgfsetlinewidth{1.505625pt}%
\definecolor{currentstroke}{rgb}{0.631373,0.788235,0.956863}%
\pgfsetstrokecolor{currentstroke}%
\pgfsetstrokeopacity{0.800000}%
\pgfsetdash{}{0pt}%
\pgfpathmoveto{\pgfqpoint{4.821671in}{6.294216in}}%
\pgfpathlineto{\pgfqpoint{2.941011in}{4.613488in}}%
\pgfusepath{stroke}%
\end{pgfscope}%
\begin{pgfscope}%
\pgfpathrectangle{\pgfqpoint{0.393613in}{0.331635in}}{\pgfqpoint{9.300000in}{7.700000in}}%
\pgfusepath{clip}%
\pgfsetrectcap%
\pgfsetroundjoin%
\pgfsetlinewidth{1.505625pt}%
\definecolor{currentstroke}{rgb}{0.631373,0.788235,0.956863}%
\pgfsetstrokecolor{currentstroke}%
\pgfsetstrokeopacity{0.800000}%
\pgfsetdash{}{0pt}%
\pgfpathmoveto{\pgfqpoint{2.038680in}{2.435803in}}%
\pgfpathlineto{\pgfqpoint{2.941011in}{4.613488in}}%
\pgfusepath{stroke}%
\end{pgfscope}%
\begin{pgfscope}%
\pgfpathrectangle{\pgfqpoint{0.393613in}{0.331635in}}{\pgfqpoint{9.300000in}{7.700000in}}%
\pgfusepath{clip}%
\pgfsetrectcap%
\pgfsetroundjoin%
\pgfsetlinewidth{1.505625pt}%
\definecolor{currentstroke}{rgb}{0.631373,0.788235,0.956863}%
\pgfsetstrokecolor{currentstroke}%
\pgfsetstrokeopacity{0.800000}%
\pgfsetdash{}{0pt}%
\pgfpathmoveto{\pgfqpoint{3.158987in}{6.306692in}}%
\pgfpathlineto{\pgfqpoint{2.941011in}{4.613488in}}%
\pgfusepath{stroke}%
\end{pgfscope}%
\begin{pgfscope}%
\pgfpathrectangle{\pgfqpoint{0.393613in}{0.331635in}}{\pgfqpoint{9.300000in}{7.700000in}}%
\pgfusepath{clip}%
\pgfsetrectcap%
\pgfsetroundjoin%
\pgfsetlinewidth{1.505625pt}%
\definecolor{currentstroke}{rgb}{0.631373,0.788235,0.956863}%
\pgfsetstrokecolor{currentstroke}%
\pgfsetstrokeopacity{0.800000}%
\pgfsetdash{}{0pt}%
\pgfpathmoveto{\pgfqpoint{3.163926in}{3.195114in}}%
\pgfpathlineto{\pgfqpoint{2.941011in}{4.613488in}}%
\pgfusepath{stroke}%
\end{pgfscope}%
\begin{pgfscope}%
\pgfpathrectangle{\pgfqpoint{0.393613in}{0.331635in}}{\pgfqpoint{9.300000in}{7.700000in}}%
\pgfusepath{clip}%
\pgfsetrectcap%
\pgfsetroundjoin%
\pgfsetlinewidth{1.505625pt}%
\definecolor{currentstroke}{rgb}{0.631373,0.788235,0.956863}%
\pgfsetstrokecolor{currentstroke}%
\pgfsetstrokeopacity{0.800000}%
\pgfsetdash{}{0pt}%
\pgfpathmoveto{\pgfqpoint{3.868734in}{6.876221in}}%
\pgfpathlineto{\pgfqpoint{2.941011in}{4.613488in}}%
\pgfusepath{stroke}%
\end{pgfscope}%
\begin{pgfscope}%
\pgfpathrectangle{\pgfqpoint{0.393613in}{0.331635in}}{\pgfqpoint{9.300000in}{7.700000in}}%
\pgfusepath{clip}%
\pgfsetrectcap%
\pgfsetroundjoin%
\pgfsetlinewidth{1.505625pt}%
\definecolor{currentstroke}{rgb}{1.000000,0.705882,0.509804}%
\pgfsetstrokecolor{currentstroke}%
\pgfsetstrokeopacity{0.800000}%
\pgfsetdash{}{0pt}%
\pgfpathmoveto{\pgfqpoint{8.729535in}{4.320731in}}%
\pgfpathlineto{\pgfqpoint{6.534666in}{4.191861in}}%
\pgfusepath{stroke}%
\end{pgfscope}%
\begin{pgfscope}%
\pgfpathrectangle{\pgfqpoint{0.393613in}{0.331635in}}{\pgfqpoint{9.300000in}{7.700000in}}%
\pgfusepath{clip}%
\pgfsetrectcap%
\pgfsetroundjoin%
\pgfsetlinewidth{1.505625pt}%
\definecolor{currentstroke}{rgb}{1.000000,0.705882,0.509804}%
\pgfsetstrokecolor{currentstroke}%
\pgfsetstrokeopacity{0.800000}%
\pgfsetdash{}{0pt}%
\pgfpathmoveto{\pgfqpoint{8.008246in}{2.616387in}}%
\pgfpathlineto{\pgfqpoint{6.534666in}{4.191861in}}%
\pgfusepath{stroke}%
\end{pgfscope}%
\begin{pgfscope}%
\pgfpathrectangle{\pgfqpoint{0.393613in}{0.331635in}}{\pgfqpoint{9.300000in}{7.700000in}}%
\pgfusepath{clip}%
\pgfsetrectcap%
\pgfsetroundjoin%
\pgfsetlinewidth{1.505625pt}%
\definecolor{currentstroke}{rgb}{1.000000,0.705882,0.509804}%
\pgfsetstrokecolor{currentstroke}%
\pgfsetstrokeopacity{0.800000}%
\pgfsetdash{}{0pt}%
\pgfpathmoveto{\pgfqpoint{6.843430in}{2.246087in}}%
\pgfpathlineto{\pgfqpoint{6.534666in}{4.191861in}}%
\pgfusepath{stroke}%
\end{pgfscope}%
\begin{pgfscope}%
\pgfpathrectangle{\pgfqpoint{0.393613in}{0.331635in}}{\pgfqpoint{9.300000in}{7.700000in}}%
\pgfusepath{clip}%
\pgfsetrectcap%
\pgfsetroundjoin%
\pgfsetlinewidth{1.505625pt}%
\definecolor{currentstroke}{rgb}{1.000000,0.705882,0.509804}%
\pgfsetstrokecolor{currentstroke}%
\pgfsetstrokeopacity{0.800000}%
\pgfsetdash{}{0pt}%
\pgfpathmoveto{\pgfqpoint{6.884025in}{6.102825in}}%
\pgfpathlineto{\pgfqpoint{6.534666in}{4.191861in}}%
\pgfusepath{stroke}%
\end{pgfscope}%
\begin{pgfscope}%
\pgfpathrectangle{\pgfqpoint{0.393613in}{0.331635in}}{\pgfqpoint{9.300000in}{7.700000in}}%
\pgfusepath{clip}%
\pgfsetrectcap%
\pgfsetroundjoin%
\pgfsetlinewidth{1.505625pt}%
\definecolor{currentstroke}{rgb}{1.000000,0.705882,0.509804}%
\pgfsetstrokecolor{currentstroke}%
\pgfsetstrokeopacity{0.800000}%
\pgfsetdash{}{0pt}%
\pgfpathmoveto{\pgfqpoint{8.217971in}{5.492423in}}%
\pgfpathlineto{\pgfqpoint{6.534666in}{4.191861in}}%
\pgfusepath{stroke}%
\end{pgfscope}%
\begin{pgfscope}%
\pgfpathrectangle{\pgfqpoint{0.393613in}{0.331635in}}{\pgfqpoint{9.300000in}{7.700000in}}%
\pgfusepath{clip}%
\pgfsetrectcap%
\pgfsetroundjoin%
\pgfsetlinewidth{1.505625pt}%
\definecolor{currentstroke}{rgb}{1.000000,0.705882,0.509804}%
\pgfsetstrokecolor{currentstroke}%
\pgfsetstrokeopacity{0.800000}%
\pgfsetdash{}{0pt}%
\pgfpathmoveto{\pgfqpoint{7.449270in}{4.187473in}}%
\pgfpathlineto{\pgfqpoint{6.534666in}{4.191861in}}%
\pgfusepath{stroke}%
\end{pgfscope}%
\begin{pgfscope}%
\pgfpathrectangle{\pgfqpoint{0.393613in}{0.331635in}}{\pgfqpoint{9.300000in}{7.700000in}}%
\pgfusepath{clip}%
\pgfsetrectcap%
\pgfsetroundjoin%
\pgfsetlinewidth{1.505625pt}%
\definecolor{currentstroke}{rgb}{1.000000,0.705882,0.509804}%
\pgfsetstrokecolor{currentstroke}%
\pgfsetstrokeopacity{0.800000}%
\pgfsetdash{}{0pt}%
\pgfpathmoveto{\pgfqpoint{7.634735in}{5.937258in}}%
\pgfpathlineto{\pgfqpoint{6.534666in}{4.191861in}}%
\pgfusepath{stroke}%
\end{pgfscope}%
\begin{pgfscope}%
\pgfpathrectangle{\pgfqpoint{0.393613in}{0.331635in}}{\pgfqpoint{9.300000in}{7.700000in}}%
\pgfusepath{clip}%
\pgfsetrectcap%
\pgfsetroundjoin%
\pgfsetlinewidth{1.505625pt}%
\definecolor{currentstroke}{rgb}{1.000000,0.705882,0.509804}%
\pgfsetstrokecolor{currentstroke}%
\pgfsetstrokeopacity{0.800000}%
\pgfsetdash{}{0pt}%
\pgfpathmoveto{\pgfqpoint{7.737786in}{3.852959in}}%
\pgfpathlineto{\pgfqpoint{6.534666in}{4.191861in}}%
\pgfusepath{stroke}%
\end{pgfscope}%
\begin{pgfscope}%
\pgfpathrectangle{\pgfqpoint{0.393613in}{0.331635in}}{\pgfqpoint{9.300000in}{7.700000in}}%
\pgfusepath{clip}%
\pgfsetrectcap%
\pgfsetroundjoin%
\pgfsetlinewidth{1.505625pt}%
\definecolor{currentstroke}{rgb}{1.000000,0.705882,0.509804}%
\pgfsetstrokecolor{currentstroke}%
\pgfsetstrokeopacity{0.800000}%
\pgfsetdash{}{0pt}%
\pgfpathmoveto{\pgfqpoint{6.985227in}{5.988577in}}%
\pgfpathlineto{\pgfqpoint{6.534666in}{4.191861in}}%
\pgfusepath{stroke}%
\end{pgfscope}%
\begin{pgfscope}%
\pgfpathrectangle{\pgfqpoint{0.393613in}{0.331635in}}{\pgfqpoint{9.300000in}{7.700000in}}%
\pgfusepath{clip}%
\pgfsetrectcap%
\pgfsetroundjoin%
\pgfsetlinewidth{1.505625pt}%
\definecolor{currentstroke}{rgb}{1.000000,0.705882,0.509804}%
\pgfsetstrokecolor{currentstroke}%
\pgfsetstrokeopacity{0.800000}%
\pgfsetdash{}{0pt}%
\pgfpathmoveto{\pgfqpoint{8.526151in}{3.491262in}}%
\pgfpathlineto{\pgfqpoint{6.534666in}{4.191861in}}%
\pgfusepath{stroke}%
\end{pgfscope}%
\begin{pgfscope}%
\pgfpathrectangle{\pgfqpoint{0.393613in}{0.331635in}}{\pgfqpoint{9.300000in}{7.700000in}}%
\pgfusepath{clip}%
\pgfsetrectcap%
\pgfsetroundjoin%
\pgfsetlinewidth{1.505625pt}%
\definecolor{currentstroke}{rgb}{1.000000,0.705882,0.509804}%
\pgfsetstrokecolor{currentstroke}%
\pgfsetstrokeopacity{0.800000}%
\pgfsetdash{}{0pt}%
\pgfpathmoveto{\pgfqpoint{4.970295in}{4.682448in}}%
\pgfpathlineto{\pgfqpoint{6.534666in}{4.191861in}}%
\pgfusepath{stroke}%
\end{pgfscope}%
\begin{pgfscope}%
\pgfpathrectangle{\pgfqpoint{0.393613in}{0.331635in}}{\pgfqpoint{9.300000in}{7.700000in}}%
\pgfusepath{clip}%
\pgfsetrectcap%
\pgfsetroundjoin%
\pgfsetlinewidth{1.505625pt}%
\definecolor{currentstroke}{rgb}{1.000000,0.705882,0.509804}%
\pgfsetstrokecolor{currentstroke}%
\pgfsetstrokeopacity{0.800000}%
\pgfsetdash{}{0pt}%
\pgfpathmoveto{\pgfqpoint{4.066721in}{5.299933in}}%
\pgfpathlineto{\pgfqpoint{6.534666in}{4.191861in}}%
\pgfusepath{stroke}%
\end{pgfscope}%
\begin{pgfscope}%
\pgfpathrectangle{\pgfqpoint{0.393613in}{0.331635in}}{\pgfqpoint{9.300000in}{7.700000in}}%
\pgfusepath{clip}%
\pgfsetrectcap%
\pgfsetroundjoin%
\pgfsetlinewidth{1.505625pt}%
\definecolor{currentstroke}{rgb}{1.000000,0.705882,0.509804}%
\pgfsetstrokecolor{currentstroke}%
\pgfsetstrokeopacity{0.800000}%
\pgfsetdash{}{0pt}%
\pgfpathmoveto{\pgfqpoint{5.047977in}{5.129617in}}%
\pgfpathlineto{\pgfqpoint{6.534666in}{4.191861in}}%
\pgfusepath{stroke}%
\end{pgfscope}%
\begin{pgfscope}%
\pgfpathrectangle{\pgfqpoint{0.393613in}{0.331635in}}{\pgfqpoint{9.300000in}{7.700000in}}%
\pgfusepath{clip}%
\pgfsetrectcap%
\pgfsetroundjoin%
\pgfsetlinewidth{1.505625pt}%
\definecolor{currentstroke}{rgb}{1.000000,0.705882,0.509804}%
\pgfsetstrokecolor{currentstroke}%
\pgfsetstrokeopacity{0.800000}%
\pgfsetdash{}{0pt}%
\pgfpathmoveto{\pgfqpoint{8.708537in}{3.149559in}}%
\pgfpathlineto{\pgfqpoint{6.534666in}{4.191861in}}%
\pgfusepath{stroke}%
\end{pgfscope}%
\begin{pgfscope}%
\pgfpathrectangle{\pgfqpoint{0.393613in}{0.331635in}}{\pgfqpoint{9.300000in}{7.700000in}}%
\pgfusepath{clip}%
\pgfsetrectcap%
\pgfsetroundjoin%
\pgfsetlinewidth{1.505625pt}%
\definecolor{currentstroke}{rgb}{1.000000,0.705882,0.509804}%
\pgfsetstrokecolor{currentstroke}%
\pgfsetstrokeopacity{0.800000}%
\pgfsetdash{}{0pt}%
\pgfpathmoveto{\pgfqpoint{6.611184in}{2.598931in}}%
\pgfpathlineto{\pgfqpoint{6.534666in}{4.191861in}}%
\pgfusepath{stroke}%
\end{pgfscope}%
\begin{pgfscope}%
\pgfpathrectangle{\pgfqpoint{0.393613in}{0.331635in}}{\pgfqpoint{9.300000in}{7.700000in}}%
\pgfusepath{clip}%
\pgfsetrectcap%
\pgfsetroundjoin%
\pgfsetlinewidth{1.505625pt}%
\definecolor{currentstroke}{rgb}{1.000000,0.705882,0.509804}%
\pgfsetstrokecolor{currentstroke}%
\pgfsetstrokeopacity{0.800000}%
\pgfsetdash{}{0pt}%
\pgfpathmoveto{\pgfqpoint{5.696576in}{3.930575in}}%
\pgfpathlineto{\pgfqpoint{6.534666in}{4.191861in}}%
\pgfusepath{stroke}%
\end{pgfscope}%
\begin{pgfscope}%
\pgfpathrectangle{\pgfqpoint{0.393613in}{0.331635in}}{\pgfqpoint{9.300000in}{7.700000in}}%
\pgfusepath{clip}%
\pgfsetrectcap%
\pgfsetroundjoin%
\pgfsetlinewidth{1.505625pt}%
\definecolor{currentstroke}{rgb}{1.000000,0.705882,0.509804}%
\pgfsetstrokecolor{currentstroke}%
\pgfsetstrokeopacity{0.800000}%
\pgfsetdash{}{0pt}%
\pgfpathmoveto{\pgfqpoint{4.929999in}{4.145173in}}%
\pgfpathlineto{\pgfqpoint{6.534666in}{4.191861in}}%
\pgfusepath{stroke}%
\end{pgfscope}%
\begin{pgfscope}%
\pgfpathrectangle{\pgfqpoint{0.393613in}{0.331635in}}{\pgfqpoint{9.300000in}{7.700000in}}%
\pgfusepath{clip}%
\pgfsetrectcap%
\pgfsetroundjoin%
\pgfsetlinewidth{1.505625pt}%
\definecolor{currentstroke}{rgb}{1.000000,0.705882,0.509804}%
\pgfsetstrokecolor{currentstroke}%
\pgfsetstrokeopacity{0.800000}%
\pgfsetdash{}{0pt}%
\pgfpathmoveto{\pgfqpoint{4.410117in}{4.635500in}}%
\pgfpathlineto{\pgfqpoint{6.534666in}{4.191861in}}%
\pgfusepath{stroke}%
\end{pgfscope}%
\begin{pgfscope}%
\pgfpathrectangle{\pgfqpoint{0.393613in}{0.331635in}}{\pgfqpoint{9.300000in}{7.700000in}}%
\pgfusepath{clip}%
\pgfsetrectcap%
\pgfsetroundjoin%
\pgfsetlinewidth{1.505625pt}%
\definecolor{currentstroke}{rgb}{1.000000,0.705882,0.509804}%
\pgfsetstrokecolor{currentstroke}%
\pgfsetstrokeopacity{0.800000}%
\pgfsetdash{}{0pt}%
\pgfpathmoveto{\pgfqpoint{8.226679in}{3.239102in}}%
\pgfpathlineto{\pgfqpoint{6.534666in}{4.191861in}}%
\pgfusepath{stroke}%
\end{pgfscope}%
\begin{pgfscope}%
\pgfpathrectangle{\pgfqpoint{0.393613in}{0.331635in}}{\pgfqpoint{9.300000in}{7.700000in}}%
\pgfusepath{clip}%
\pgfsetrectcap%
\pgfsetroundjoin%
\pgfsetlinewidth{1.505625pt}%
\definecolor{currentstroke}{rgb}{1.000000,0.705882,0.509804}%
\pgfsetstrokecolor{currentstroke}%
\pgfsetstrokeopacity{0.800000}%
\pgfsetdash{}{0pt}%
\pgfpathmoveto{\pgfqpoint{8.462150in}{4.774615in}}%
\pgfpathlineto{\pgfqpoint{6.534666in}{4.191861in}}%
\pgfusepath{stroke}%
\end{pgfscope}%
\begin{pgfscope}%
\pgfpathrectangle{\pgfqpoint{0.393613in}{0.331635in}}{\pgfqpoint{9.300000in}{7.700000in}}%
\pgfusepath{clip}%
\pgfsetrectcap%
\pgfsetroundjoin%
\pgfsetlinewidth{1.505625pt}%
\definecolor{currentstroke}{rgb}{1.000000,0.705882,0.509804}%
\pgfsetstrokecolor{currentstroke}%
\pgfsetstrokeopacity{0.800000}%
\pgfsetdash{}{0pt}%
\pgfpathmoveto{\pgfqpoint{5.762899in}{4.714105in}}%
\pgfpathlineto{\pgfqpoint{6.534666in}{4.191861in}}%
\pgfusepath{stroke}%
\end{pgfscope}%
\begin{pgfscope}%
\pgfpathrectangle{\pgfqpoint{0.393613in}{0.331635in}}{\pgfqpoint{9.300000in}{7.700000in}}%
\pgfusepath{clip}%
\pgfsetrectcap%
\pgfsetroundjoin%
\pgfsetlinewidth{1.505625pt}%
\definecolor{currentstroke}{rgb}{1.000000,0.705882,0.509804}%
\pgfsetstrokecolor{currentstroke}%
\pgfsetstrokeopacity{0.800000}%
\pgfsetdash{}{0pt}%
\pgfpathmoveto{\pgfqpoint{8.593098in}{3.947991in}}%
\pgfpathlineto{\pgfqpoint{6.534666in}{4.191861in}}%
\pgfusepath{stroke}%
\end{pgfscope}%
\begin{pgfscope}%
\pgfpathrectangle{\pgfqpoint{0.393613in}{0.331635in}}{\pgfqpoint{9.300000in}{7.700000in}}%
\pgfusepath{clip}%
\pgfsetrectcap%
\pgfsetroundjoin%
\pgfsetlinewidth{1.505625pt}%
\definecolor{currentstroke}{rgb}{1.000000,0.705882,0.509804}%
\pgfsetstrokecolor{currentstroke}%
\pgfsetstrokeopacity{0.800000}%
\pgfsetdash{}{0pt}%
\pgfpathmoveto{\pgfqpoint{5.266097in}{2.316466in}}%
\pgfpathlineto{\pgfqpoint{6.534666in}{4.191861in}}%
\pgfusepath{stroke}%
\end{pgfscope}%
\begin{pgfscope}%
\pgfpathrectangle{\pgfqpoint{0.393613in}{0.331635in}}{\pgfqpoint{9.300000in}{7.700000in}}%
\pgfusepath{clip}%
\pgfsetrectcap%
\pgfsetroundjoin%
\pgfsetlinewidth{1.505625pt}%
\definecolor{currentstroke}{rgb}{1.000000,0.705882,0.509804}%
\pgfsetstrokecolor{currentstroke}%
\pgfsetstrokeopacity{0.800000}%
\pgfsetdash{}{0pt}%
\pgfpathmoveto{\pgfqpoint{7.489133in}{2.379617in}}%
\pgfpathlineto{\pgfqpoint{6.534666in}{4.191861in}}%
\pgfusepath{stroke}%
\end{pgfscope}%
\begin{pgfscope}%
\pgfpathrectangle{\pgfqpoint{0.393613in}{0.331635in}}{\pgfqpoint{9.300000in}{7.700000in}}%
\pgfusepath{clip}%
\pgfsetrectcap%
\pgfsetroundjoin%
\pgfsetlinewidth{1.505625pt}%
\definecolor{currentstroke}{rgb}{1.000000,0.705882,0.509804}%
\pgfsetstrokecolor{currentstroke}%
\pgfsetstrokeopacity{0.800000}%
\pgfsetdash{}{0pt}%
\pgfpathmoveto{\pgfqpoint{7.375591in}{4.728687in}}%
\pgfpathlineto{\pgfqpoint{6.534666in}{4.191861in}}%
\pgfusepath{stroke}%
\end{pgfscope}%
\begin{pgfscope}%
\pgfpathrectangle{\pgfqpoint{0.393613in}{0.331635in}}{\pgfqpoint{9.300000in}{7.700000in}}%
\pgfusepath{clip}%
\pgfsetrectcap%
\pgfsetroundjoin%
\pgfsetlinewidth{1.505625pt}%
\definecolor{currentstroke}{rgb}{1.000000,0.705882,0.509804}%
\pgfsetstrokecolor{currentstroke}%
\pgfsetstrokeopacity{0.800000}%
\pgfsetdash{}{0pt}%
\pgfpathmoveto{\pgfqpoint{5.281132in}{4.012768in}}%
\pgfpathlineto{\pgfqpoint{6.534666in}{4.191861in}}%
\pgfusepath{stroke}%
\end{pgfscope}%
\begin{pgfscope}%
\pgfpathrectangle{\pgfqpoint{0.393613in}{0.331635in}}{\pgfqpoint{9.300000in}{7.700000in}}%
\pgfusepath{clip}%
\pgfsetrectcap%
\pgfsetroundjoin%
\pgfsetlinewidth{1.505625pt}%
\definecolor{currentstroke}{rgb}{1.000000,0.705882,0.509804}%
\pgfsetstrokecolor{currentstroke}%
\pgfsetstrokeopacity{0.800000}%
\pgfsetdash{}{0pt}%
\pgfpathmoveto{\pgfqpoint{8.775963in}{3.806168in}}%
\pgfpathlineto{\pgfqpoint{6.534666in}{4.191861in}}%
\pgfusepath{stroke}%
\end{pgfscope}%
\begin{pgfscope}%
\pgfpathrectangle{\pgfqpoint{0.393613in}{0.331635in}}{\pgfqpoint{9.300000in}{7.700000in}}%
\pgfusepath{clip}%
\pgfsetrectcap%
\pgfsetroundjoin%
\pgfsetlinewidth{1.505625pt}%
\definecolor{currentstroke}{rgb}{1.000000,0.705882,0.509804}%
\pgfsetstrokecolor{currentstroke}%
\pgfsetstrokeopacity{0.800000}%
\pgfsetdash{}{0pt}%
\pgfpathmoveto{\pgfqpoint{9.270885in}{4.232334in}}%
\pgfpathlineto{\pgfqpoint{6.534666in}{4.191861in}}%
\pgfusepath{stroke}%
\end{pgfscope}%
\begin{pgfscope}%
\pgfpathrectangle{\pgfqpoint{0.393613in}{0.331635in}}{\pgfqpoint{9.300000in}{7.700000in}}%
\pgfusepath{clip}%
\pgfsetrectcap%
\pgfsetroundjoin%
\pgfsetlinewidth{1.505625pt}%
\definecolor{currentstroke}{rgb}{1.000000,0.705882,0.509804}%
\pgfsetstrokecolor{currentstroke}%
\pgfsetstrokeopacity{0.800000}%
\pgfsetdash{}{0pt}%
\pgfpathmoveto{\pgfqpoint{5.221194in}{3.868911in}}%
\pgfpathlineto{\pgfqpoint{6.534666in}{4.191861in}}%
\pgfusepath{stroke}%
\end{pgfscope}%
\begin{pgfscope}%
\pgfpathrectangle{\pgfqpoint{0.393613in}{0.331635in}}{\pgfqpoint{9.300000in}{7.700000in}}%
\pgfusepath{clip}%
\pgfsetrectcap%
\pgfsetroundjoin%
\pgfsetlinewidth{1.505625pt}%
\definecolor{currentstroke}{rgb}{1.000000,0.705882,0.509804}%
\pgfsetstrokecolor{currentstroke}%
\pgfsetstrokeopacity{0.800000}%
\pgfsetdash{}{0pt}%
\pgfpathmoveto{\pgfqpoint{4.726563in}{5.105585in}}%
\pgfpathlineto{\pgfqpoint{6.534666in}{4.191861in}}%
\pgfusepath{stroke}%
\end{pgfscope}%
\begin{pgfscope}%
\pgfpathrectangle{\pgfqpoint{0.393613in}{0.331635in}}{\pgfqpoint{9.300000in}{7.700000in}}%
\pgfusepath{clip}%
\pgfsetrectcap%
\pgfsetroundjoin%
\pgfsetlinewidth{1.505625pt}%
\definecolor{currentstroke}{rgb}{1.000000,0.705882,0.509804}%
\pgfsetstrokecolor{currentstroke}%
\pgfsetstrokeopacity{0.800000}%
\pgfsetdash{}{0pt}%
\pgfpathmoveto{\pgfqpoint{5.210620in}{2.300748in}}%
\pgfpathlineto{\pgfqpoint{6.534666in}{4.191861in}}%
\pgfusepath{stroke}%
\end{pgfscope}%
\begin{pgfscope}%
\pgfpathrectangle{\pgfqpoint{0.393613in}{0.331635in}}{\pgfqpoint{9.300000in}{7.700000in}}%
\pgfusepath{clip}%
\pgfsetrectcap%
\pgfsetroundjoin%
\pgfsetlinewidth{1.505625pt}%
\definecolor{currentstroke}{rgb}{1.000000,0.705882,0.509804}%
\pgfsetstrokecolor{currentstroke}%
\pgfsetstrokeopacity{0.800000}%
\pgfsetdash{}{0pt}%
\pgfpathmoveto{\pgfqpoint{6.611736in}{2.822774in}}%
\pgfpathlineto{\pgfqpoint{6.534666in}{4.191861in}}%
\pgfusepath{stroke}%
\end{pgfscope}%
\begin{pgfscope}%
\pgfpathrectangle{\pgfqpoint{0.393613in}{0.331635in}}{\pgfqpoint{9.300000in}{7.700000in}}%
\pgfusepath{clip}%
\pgfsetrectcap%
\pgfsetroundjoin%
\pgfsetlinewidth{1.505625pt}%
\definecolor{currentstroke}{rgb}{1.000000,0.705882,0.509804}%
\pgfsetstrokecolor{currentstroke}%
\pgfsetstrokeopacity{0.800000}%
\pgfsetdash{}{0pt}%
\pgfpathmoveto{\pgfqpoint{4.838508in}{5.682184in}}%
\pgfpathlineto{\pgfqpoint{6.534666in}{4.191861in}}%
\pgfusepath{stroke}%
\end{pgfscope}%
\begin{pgfscope}%
\pgfpathrectangle{\pgfqpoint{0.393613in}{0.331635in}}{\pgfqpoint{9.300000in}{7.700000in}}%
\pgfusepath{clip}%
\pgfsetrectcap%
\pgfsetroundjoin%
\pgfsetlinewidth{1.505625pt}%
\definecolor{currentstroke}{rgb}{1.000000,0.705882,0.509804}%
\pgfsetstrokecolor{currentstroke}%
\pgfsetstrokeopacity{0.800000}%
\pgfsetdash{}{0pt}%
\pgfpathmoveto{\pgfqpoint{7.784989in}{5.679570in}}%
\pgfpathlineto{\pgfqpoint{6.534666in}{4.191861in}}%
\pgfusepath{stroke}%
\end{pgfscope}%
\begin{pgfscope}%
\pgfpathrectangle{\pgfqpoint{0.393613in}{0.331635in}}{\pgfqpoint{9.300000in}{7.700000in}}%
\pgfusepath{clip}%
\pgfsetrectcap%
\pgfsetroundjoin%
\pgfsetlinewidth{1.505625pt}%
\definecolor{currentstroke}{rgb}{1.000000,0.705882,0.509804}%
\pgfsetstrokecolor{currentstroke}%
\pgfsetstrokeopacity{0.800000}%
\pgfsetdash{}{0pt}%
\pgfpathmoveto{\pgfqpoint{9.033656in}{3.874177in}}%
\pgfpathlineto{\pgfqpoint{6.534666in}{4.191861in}}%
\pgfusepath{stroke}%
\end{pgfscope}%
\begin{pgfscope}%
\pgfpathrectangle{\pgfqpoint{0.393613in}{0.331635in}}{\pgfqpoint{9.300000in}{7.700000in}}%
\pgfusepath{clip}%
\pgfsetrectcap%
\pgfsetroundjoin%
\pgfsetlinewidth{1.505625pt}%
\definecolor{currentstroke}{rgb}{1.000000,0.705882,0.509804}%
\pgfsetstrokecolor{currentstroke}%
\pgfsetstrokeopacity{0.800000}%
\pgfsetdash{}{0pt}%
\pgfpathmoveto{\pgfqpoint{4.512348in}{5.041560in}}%
\pgfpathlineto{\pgfqpoint{6.534666in}{4.191861in}}%
\pgfusepath{stroke}%
\end{pgfscope}%
\begin{pgfscope}%
\pgfpathrectangle{\pgfqpoint{0.393613in}{0.331635in}}{\pgfqpoint{9.300000in}{7.700000in}}%
\pgfusepath{clip}%
\pgfsetrectcap%
\pgfsetroundjoin%
\pgfsetlinewidth{1.505625pt}%
\definecolor{currentstroke}{rgb}{1.000000,0.705882,0.509804}%
\pgfsetstrokecolor{currentstroke}%
\pgfsetstrokeopacity{0.800000}%
\pgfsetdash{}{0pt}%
\pgfpathmoveto{\pgfqpoint{5.861358in}{5.869272in}}%
\pgfpathlineto{\pgfqpoint{6.534666in}{4.191861in}}%
\pgfusepath{stroke}%
\end{pgfscope}%
\begin{pgfscope}%
\pgfpathrectangle{\pgfqpoint{0.393613in}{0.331635in}}{\pgfqpoint{9.300000in}{7.700000in}}%
\pgfusepath{clip}%
\pgfsetrectcap%
\pgfsetroundjoin%
\pgfsetlinewidth{1.505625pt}%
\definecolor{currentstroke}{rgb}{1.000000,0.705882,0.509804}%
\pgfsetstrokecolor{currentstroke}%
\pgfsetstrokeopacity{0.800000}%
\pgfsetdash{}{0pt}%
\pgfpathmoveto{\pgfqpoint{5.626156in}{2.828062in}}%
\pgfpathlineto{\pgfqpoint{6.534666in}{4.191861in}}%
\pgfusepath{stroke}%
\end{pgfscope}%
\begin{pgfscope}%
\pgfpathrectangle{\pgfqpoint{0.393613in}{0.331635in}}{\pgfqpoint{9.300000in}{7.700000in}}%
\pgfusepath{clip}%
\pgfsetrectcap%
\pgfsetroundjoin%
\pgfsetlinewidth{1.505625pt}%
\definecolor{currentstroke}{rgb}{1.000000,0.705882,0.509804}%
\pgfsetstrokecolor{currentstroke}%
\pgfsetstrokeopacity{0.800000}%
\pgfsetdash{}{0pt}%
\pgfpathmoveto{\pgfqpoint{4.868691in}{3.476860in}}%
\pgfpathlineto{\pgfqpoint{6.534666in}{4.191861in}}%
\pgfusepath{stroke}%
\end{pgfscope}%
\begin{pgfscope}%
\pgfpathrectangle{\pgfqpoint{0.393613in}{0.331635in}}{\pgfqpoint{9.300000in}{7.700000in}}%
\pgfusepath{clip}%
\pgfsetrectcap%
\pgfsetroundjoin%
\pgfsetlinewidth{1.505625pt}%
\definecolor{currentstroke}{rgb}{1.000000,0.705882,0.509804}%
\pgfsetstrokecolor{currentstroke}%
\pgfsetstrokeopacity{0.800000}%
\pgfsetdash{}{0pt}%
\pgfpathmoveto{\pgfqpoint{7.939732in}{4.140668in}}%
\pgfpathlineto{\pgfqpoint{6.534666in}{4.191861in}}%
\pgfusepath{stroke}%
\end{pgfscope}%
\begin{pgfscope}%
\pgfpathrectangle{\pgfqpoint{0.393613in}{0.331635in}}{\pgfqpoint{9.300000in}{7.700000in}}%
\pgfusepath{clip}%
\pgfsetrectcap%
\pgfsetroundjoin%
\pgfsetlinewidth{1.505625pt}%
\definecolor{currentstroke}{rgb}{1.000000,0.705882,0.509804}%
\pgfsetstrokecolor{currentstroke}%
\pgfsetstrokeopacity{0.800000}%
\pgfsetdash{}{0pt}%
\pgfpathmoveto{\pgfqpoint{5.975649in}{5.901612in}}%
\pgfpathlineto{\pgfqpoint{6.534666in}{4.191861in}}%
\pgfusepath{stroke}%
\end{pgfscope}%
\begin{pgfscope}%
\pgfpathrectangle{\pgfqpoint{0.393613in}{0.331635in}}{\pgfqpoint{9.300000in}{7.700000in}}%
\pgfusepath{clip}%
\pgfsetrectcap%
\pgfsetroundjoin%
\pgfsetlinewidth{1.505625pt}%
\definecolor{currentstroke}{rgb}{1.000000,0.705882,0.509804}%
\pgfsetstrokecolor{currentstroke}%
\pgfsetstrokeopacity{0.800000}%
\pgfsetdash{}{0pt}%
\pgfpathmoveto{\pgfqpoint{5.533563in}{3.247043in}}%
\pgfpathlineto{\pgfqpoint{6.534666in}{4.191861in}}%
\pgfusepath{stroke}%
\end{pgfscope}%
\begin{pgfscope}%
\pgfpathrectangle{\pgfqpoint{0.393613in}{0.331635in}}{\pgfqpoint{9.300000in}{7.700000in}}%
\pgfusepath{clip}%
\pgfsetrectcap%
\pgfsetroundjoin%
\pgfsetlinewidth{1.505625pt}%
\definecolor{currentstroke}{rgb}{1.000000,0.705882,0.509804}%
\pgfsetstrokecolor{currentstroke}%
\pgfsetstrokeopacity{0.800000}%
\pgfsetdash{}{0pt}%
\pgfpathmoveto{\pgfqpoint{8.879060in}{4.527070in}}%
\pgfpathlineto{\pgfqpoint{6.534666in}{4.191861in}}%
\pgfusepath{stroke}%
\end{pgfscope}%
\begin{pgfscope}%
\pgfpathrectangle{\pgfqpoint{0.393613in}{0.331635in}}{\pgfqpoint{9.300000in}{7.700000in}}%
\pgfusepath{clip}%
\pgfsetrectcap%
\pgfsetroundjoin%
\pgfsetlinewidth{1.505625pt}%
\definecolor{currentstroke}{rgb}{1.000000,0.705882,0.509804}%
\pgfsetstrokecolor{currentstroke}%
\pgfsetstrokeopacity{0.800000}%
\pgfsetdash{}{0pt}%
\pgfpathmoveto{\pgfqpoint{4.426686in}{4.680141in}}%
\pgfpathlineto{\pgfqpoint{6.534666in}{4.191861in}}%
\pgfusepath{stroke}%
\end{pgfscope}%
\begin{pgfscope}%
\pgfpathrectangle{\pgfqpoint{0.393613in}{0.331635in}}{\pgfqpoint{9.300000in}{7.700000in}}%
\pgfusepath{clip}%
\pgfsetrectcap%
\pgfsetroundjoin%
\pgfsetlinewidth{1.505625pt}%
\definecolor{currentstroke}{rgb}{1.000000,0.705882,0.509804}%
\pgfsetstrokecolor{currentstroke}%
\pgfsetstrokeopacity{0.800000}%
\pgfsetdash{}{0pt}%
\pgfpathmoveto{\pgfqpoint{6.115172in}{3.529359in}}%
\pgfpathlineto{\pgfqpoint{6.534666in}{4.191861in}}%
\pgfusepath{stroke}%
\end{pgfscope}%
\begin{pgfscope}%
\pgfpathrectangle{\pgfqpoint{0.393613in}{0.331635in}}{\pgfqpoint{9.300000in}{7.700000in}}%
\pgfusepath{clip}%
\pgfsetrectcap%
\pgfsetroundjoin%
\pgfsetlinewidth{1.505625pt}%
\definecolor{currentstroke}{rgb}{1.000000,0.705882,0.509804}%
\pgfsetstrokecolor{currentstroke}%
\pgfsetstrokeopacity{0.800000}%
\pgfsetdash{}{0pt}%
\pgfpathmoveto{\pgfqpoint{5.118229in}{2.407043in}}%
\pgfpathlineto{\pgfqpoint{6.534666in}{4.191861in}}%
\pgfusepath{stroke}%
\end{pgfscope}%
\begin{pgfscope}%
\pgfpathrectangle{\pgfqpoint{0.393613in}{0.331635in}}{\pgfqpoint{9.300000in}{7.700000in}}%
\pgfusepath{clip}%
\pgfsetrectcap%
\pgfsetroundjoin%
\pgfsetlinewidth{1.505625pt}%
\definecolor{currentstroke}{rgb}{1.000000,0.705882,0.509804}%
\pgfsetstrokecolor{currentstroke}%
\pgfsetstrokeopacity{0.800000}%
\pgfsetdash{}{0pt}%
\pgfpathmoveto{\pgfqpoint{6.785951in}{6.348765in}}%
\pgfpathlineto{\pgfqpoint{6.534666in}{4.191861in}}%
\pgfusepath{stroke}%
\end{pgfscope}%
\begin{pgfscope}%
\pgfpathrectangle{\pgfqpoint{0.393613in}{0.331635in}}{\pgfqpoint{9.300000in}{7.700000in}}%
\pgfusepath{clip}%
\pgfsetrectcap%
\pgfsetroundjoin%
\pgfsetlinewidth{1.505625pt}%
\definecolor{currentstroke}{rgb}{1.000000,0.705882,0.509804}%
\pgfsetstrokecolor{currentstroke}%
\pgfsetstrokeopacity{0.800000}%
\pgfsetdash{}{0pt}%
\pgfpathmoveto{\pgfqpoint{4.549543in}{2.869784in}}%
\pgfpathlineto{\pgfqpoint{6.534666in}{4.191861in}}%
\pgfusepath{stroke}%
\end{pgfscope}%
\begin{pgfscope}%
\pgfpathrectangle{\pgfqpoint{0.393613in}{0.331635in}}{\pgfqpoint{9.300000in}{7.700000in}}%
\pgfusepath{clip}%
\pgfsetrectcap%
\pgfsetroundjoin%
\pgfsetlinewidth{1.505625pt}%
\definecolor{currentstroke}{rgb}{1.000000,0.705882,0.509804}%
\pgfsetstrokecolor{currentstroke}%
\pgfsetstrokeopacity{0.800000}%
\pgfsetdash{}{0pt}%
\pgfpathmoveto{\pgfqpoint{7.362784in}{4.456419in}}%
\pgfpathlineto{\pgfqpoint{6.534666in}{4.191861in}}%
\pgfusepath{stroke}%
\end{pgfscope}%
\begin{pgfscope}%
\pgfpathrectangle{\pgfqpoint{0.393613in}{0.331635in}}{\pgfqpoint{9.300000in}{7.700000in}}%
\pgfusepath{clip}%
\pgfsetrectcap%
\pgfsetroundjoin%
\pgfsetlinewidth{1.505625pt}%
\definecolor{currentstroke}{rgb}{1.000000,0.705882,0.509804}%
\pgfsetstrokecolor{currentstroke}%
\pgfsetstrokeopacity{0.800000}%
\pgfsetdash{}{0pt}%
\pgfpathmoveto{\pgfqpoint{3.789706in}{4.975890in}}%
\pgfpathlineto{\pgfqpoint{6.534666in}{4.191861in}}%
\pgfusepath{stroke}%
\end{pgfscope}%
\begin{pgfscope}%
\pgfsetrectcap%
\pgfsetmiterjoin%
\pgfsetlinewidth{0.803000pt}%
\definecolor{currentstroke}{rgb}{0.000000,0.000000,0.000000}%
\pgfsetstrokecolor{currentstroke}%
\pgfsetdash{}{0pt}%
\pgfpathmoveto{\pgfqpoint{0.393613in}{0.331635in}}%
\pgfpathlineto{\pgfqpoint{0.393613in}{8.031635in}}%
\pgfusepath{stroke}%
\end{pgfscope}%
\begin{pgfscope}%
\pgfsetrectcap%
\pgfsetmiterjoin%
\pgfsetlinewidth{0.803000pt}%
\definecolor{currentstroke}{rgb}{0.000000,0.000000,0.000000}%
\pgfsetstrokecolor{currentstroke}%
\pgfsetdash{}{0pt}%
\pgfpathmoveto{\pgfqpoint{9.693613in}{0.331635in}}%
\pgfpathlineto{\pgfqpoint{9.693613in}{8.031635in}}%
\pgfusepath{stroke}%
\end{pgfscope}%
\begin{pgfscope}%
\pgfsetrectcap%
\pgfsetmiterjoin%
\pgfsetlinewidth{0.803000pt}%
\definecolor{currentstroke}{rgb}{0.000000,0.000000,0.000000}%
\pgfsetstrokecolor{currentstroke}%
\pgfsetdash{}{0pt}%
\pgfpathmoveto{\pgfqpoint{0.393613in}{0.331635in}}%
\pgfpathlineto{\pgfqpoint{9.693612in}{0.331635in}}%
\pgfusepath{stroke}%
\end{pgfscope}%
\begin{pgfscope}%
\pgfsetrectcap%
\pgfsetmiterjoin%
\pgfsetlinewidth{0.803000pt}%
\definecolor{currentstroke}{rgb}{0.000000,0.000000,0.000000}%
\pgfsetstrokecolor{currentstroke}%
\pgfsetdash{}{0pt}%
\pgfpathmoveto{\pgfqpoint{0.393613in}{8.031635in}}%
\pgfpathlineto{\pgfqpoint{9.693612in}{8.031635in}}%
\pgfusepath{stroke}%
\end{pgfscope}%
\begin{pgfscope}%
\definecolor{textcolor}{rgb}{0.000000,0.000000,0.000000}%
\pgfsetstrokecolor{textcolor}%
\pgfsetfillcolor{textcolor}%
\pgftext[x=5.043613in,y=8.114968in,,base]{\color{textcolor}\sffamily\fontsize{12.000000}{14.400000}\selectfont T-SNE for chair images (s2r3dfree\_textureless\_light)}%
\end{pgfscope}%
\begin{pgfscope}%
\pgfsetbuttcap%
\pgfsetmiterjoin%
\definecolor{currentfill}{rgb}{1.000000,1.000000,1.000000}%
\pgfsetfillcolor{currentfill}%
\pgfsetfillopacity{0.800000}%
\pgfsetlinewidth{1.003750pt}%
\definecolor{currentstroke}{rgb}{0.800000,0.800000,0.800000}%
\pgfsetstrokecolor{currentstroke}%
\pgfsetstrokeopacity{0.800000}%
\pgfsetdash{}{0pt}%
\pgfpathmoveto{\pgfqpoint{9.790835in}{3.955012in}}%
\pgfpathlineto{\pgfqpoint{12.120587in}{3.955012in}}%
\pgfpathquadraticcurveto{\pgfqpoint{12.148365in}{3.955012in}}{\pgfqpoint{12.148365in}{3.982789in}}%
\pgfpathlineto{\pgfqpoint{12.148365in}{4.380481in}}%
\pgfpathquadraticcurveto{\pgfqpoint{12.148365in}{4.408258in}}{\pgfqpoint{12.120587in}{4.408258in}}%
\pgfpathlineto{\pgfqpoint{9.790835in}{4.408258in}}%
\pgfpathquadraticcurveto{\pgfqpoint{9.763057in}{4.408258in}}{\pgfqpoint{9.763057in}{4.380481in}}%
\pgfpathlineto{\pgfqpoint{9.763057in}{3.982789in}}%
\pgfpathquadraticcurveto{\pgfqpoint{9.763057in}{3.955012in}}{\pgfqpoint{9.790835in}{3.955012in}}%
\pgfpathclose%
\pgfusepath{stroke,fill}%
\end{pgfscope}%
\begin{pgfscope}%
\pgfsetbuttcap%
\pgfsetroundjoin%
\definecolor{currentfill}{rgb}{0.631373,0.788235,0.956863}%
\pgfsetfillcolor{currentfill}%
\pgfsetlinewidth{1.003750pt}%
\definecolor{currentstroke}{rgb}{0.631373,0.788235,0.956863}%
\pgfsetstrokecolor{currentstroke}%
\pgfsetdash{}{0pt}%
\pgfsys@defobject{currentmarker}{\pgfqpoint{-0.041667in}{-0.041667in}}{\pgfqpoint{0.041667in}{0.041667in}}{%
\pgfpathmoveto{\pgfqpoint{0.000000in}{-0.041667in}}%
\pgfpathcurveto{\pgfqpoint{0.011050in}{-0.041667in}}{\pgfqpoint{0.021649in}{-0.037276in}}{\pgfqpoint{0.029463in}{-0.029463in}}%
\pgfpathcurveto{\pgfqpoint{0.037276in}{-0.021649in}}{\pgfqpoint{0.041667in}{-0.011050in}}{\pgfqpoint{0.041667in}{0.000000in}}%
\pgfpathcurveto{\pgfqpoint{0.041667in}{0.011050in}}{\pgfqpoint{0.037276in}{0.021649in}}{\pgfqpoint{0.029463in}{0.029463in}}%
\pgfpathcurveto{\pgfqpoint{0.021649in}{0.037276in}}{\pgfqpoint{0.011050in}{0.041667in}}{\pgfqpoint{0.000000in}{0.041667in}}%
\pgfpathcurveto{\pgfqpoint{-0.011050in}{0.041667in}}{\pgfqpoint{-0.021649in}{0.037276in}}{\pgfqpoint{-0.029463in}{0.029463in}}%
\pgfpathcurveto{\pgfqpoint{-0.037276in}{0.021649in}}{\pgfqpoint{-0.041667in}{0.011050in}}{\pgfqpoint{-0.041667in}{0.000000in}}%
\pgfpathcurveto{\pgfqpoint{-0.041667in}{-0.011050in}}{\pgfqpoint{-0.037276in}{-0.021649in}}{\pgfqpoint{-0.029463in}{-0.029463in}}%
\pgfpathcurveto{\pgfqpoint{-0.021649in}{-0.037276in}}{\pgfqpoint{-0.011050in}{-0.041667in}}{\pgfqpoint{0.000000in}{-0.041667in}}%
\pgfpathclose%
\pgfusepath{stroke,fill}%
}%
\begin{pgfscope}%
\pgfsys@transformshift{9.957501in}{4.283638in}%
\pgfsys@useobject{currentmarker}{}%
\end{pgfscope}%
\end{pgfscope}%
\begin{pgfscope}%
\definecolor{textcolor}{rgb}{0.000000,0.000000,0.000000}%
\pgfsetstrokecolor{textcolor}%
\pgfsetfillcolor{textcolor}%
\pgftext[x=10.207501in,y=4.247180in,left,base]{\color{textcolor}\sffamily\fontsize{10.000000}{12.000000}\selectfont Pix3D}%
\end{pgfscope}%
\begin{pgfscope}%
\pgfsetbuttcap%
\pgfsetroundjoin%
\definecolor{currentfill}{rgb}{1.000000,0.705882,0.509804}%
\pgfsetfillcolor{currentfill}%
\pgfsetlinewidth{1.003750pt}%
\definecolor{currentstroke}{rgb}{1.000000,0.705882,0.509804}%
\pgfsetstrokecolor{currentstroke}%
\pgfsetdash{}{0pt}%
\pgfsys@defobject{currentmarker}{\pgfqpoint{-0.041667in}{-0.041667in}}{\pgfqpoint{0.041667in}{0.041667in}}{%
\pgfpathmoveto{\pgfqpoint{0.000000in}{-0.041667in}}%
\pgfpathcurveto{\pgfqpoint{0.011050in}{-0.041667in}}{\pgfqpoint{0.021649in}{-0.037276in}}{\pgfqpoint{0.029463in}{-0.029463in}}%
\pgfpathcurveto{\pgfqpoint{0.037276in}{-0.021649in}}{\pgfqpoint{0.041667in}{-0.011050in}}{\pgfqpoint{0.041667in}{0.000000in}}%
\pgfpathcurveto{\pgfqpoint{0.041667in}{0.011050in}}{\pgfqpoint{0.037276in}{0.021649in}}{\pgfqpoint{0.029463in}{0.029463in}}%
\pgfpathcurveto{\pgfqpoint{0.021649in}{0.037276in}}{\pgfqpoint{0.011050in}{0.041667in}}{\pgfqpoint{0.000000in}{0.041667in}}%
\pgfpathcurveto{\pgfqpoint{-0.011050in}{0.041667in}}{\pgfqpoint{-0.021649in}{0.037276in}}{\pgfqpoint{-0.029463in}{0.029463in}}%
\pgfpathcurveto{\pgfqpoint{-0.037276in}{0.021649in}}{\pgfqpoint{-0.041667in}{0.011050in}}{\pgfqpoint{-0.041667in}{0.000000in}}%
\pgfpathcurveto{\pgfqpoint{-0.041667in}{-0.011050in}}{\pgfqpoint{-0.037276in}{-0.021649in}}{\pgfqpoint{-0.029463in}{-0.029463in}}%
\pgfpathcurveto{\pgfqpoint{-0.021649in}{-0.037276in}}{\pgfqpoint{-0.011050in}{-0.041667in}}{\pgfqpoint{0.000000in}{-0.041667in}}%
\pgfpathclose%
\pgfusepath{stroke,fill}%
}%
\begin{pgfscope}%
\pgfsys@transformshift{9.957501in}{4.079781in}%
\pgfsys@useobject{currentmarker}{}%
\end{pgfscope}%
\end{pgfscope}%
\begin{pgfscope}%
\definecolor{textcolor}{rgb}{0.000000,0.000000,0.000000}%
\pgfsetstrokecolor{textcolor}%
\pgfsetfillcolor{textcolor}%
\pgftext[x=10.207501in,y=4.043322in,left,base]{\color{textcolor}\sffamily\fontsize{10.000000}{12.000000}\selectfont s2r3dfree\_textureless\_light}%
\end{pgfscope}%
\end{pgfpicture}%
\makeatother%
\endgroup%
}\\
    \resizebox{0.49\linewidth}{6cm}{%% Creator: Matplotlib, PGF backend
%%
%% To include the figure in your LaTeX document, write
%%   \input{<filename>.pgf}
%%
%% Make sure the required packages are loaded in your preamble
%%   \usepackage{pgf}
%%
%% Figures using additional raster images can only be included by \input if
%% they are in the same directory as the main LaTeX file. For loading figures
%% from other directories you can use the `import` package
%%   \usepackage{import}
%%
%% and then include the figures with
%%   \import{<path to file>}{<filename>.pgf}
%%
%% Matplotlib used the following preamble
%%   \usepackage{fontspec}
%%   \setmainfont{DejaVuSerif.ttf}[Path=\detokenize{/Users/apple/opt/anaconda3/envs/kaolin/lib/python3.7/site-packages/matplotlib/mpl-data/fonts/ttf/}]
%%   \setsansfont{DejaVuSans.ttf}[Path=\detokenize{/Users/apple/opt/anaconda3/envs/kaolin/lib/python3.7/site-packages/matplotlib/mpl-data/fonts/ttf/}]
%%   \setmonofont{DejaVuSansMono.ttf}[Path=\detokenize{/Users/apple/opt/anaconda3/envs/kaolin/lib/python3.7/site-packages/matplotlib/mpl-data/fonts/ttf/}]
%%
\begingroup%
\makeatletter%
\begin{pgfpicture}%
\pgfpathrectangle{\pgfpointorigin}{\pgfqpoint{5.541978in}{4.337596in}}%
\pgfusepath{use as bounding box, clip}%
\begin{pgfscope}%
\pgfsetbuttcap%
\pgfsetmiterjoin%
\definecolor{currentfill}{rgb}{1.000000,1.000000,1.000000}%
\pgfsetfillcolor{currentfill}%
\pgfsetlinewidth{0.000000pt}%
\definecolor{currentstroke}{rgb}{1.000000,1.000000,1.000000}%
\pgfsetstrokecolor{currentstroke}%
\pgfsetdash{}{0pt}%
\pgfpathmoveto{\pgfqpoint{0.000000in}{0.000000in}}%
\pgfpathlineto{\pgfqpoint{5.541978in}{0.000000in}}%
\pgfpathlineto{\pgfqpoint{5.541978in}{4.337596in}}%
\pgfpathlineto{\pgfqpoint{0.000000in}{4.337596in}}%
\pgfpathclose%
\pgfusepath{fill}%
\end{pgfscope}%
\begin{pgfscope}%
\pgfsetbuttcap%
\pgfsetmiterjoin%
\definecolor{currentfill}{rgb}{1.000000,1.000000,1.000000}%
\pgfsetfillcolor{currentfill}%
\pgfsetlinewidth{0.000000pt}%
\definecolor{currentstroke}{rgb}{0.000000,0.000000,0.000000}%
\pgfsetstrokecolor{currentstroke}%
\pgfsetstrokeopacity{0.000000}%
\pgfsetdash{}{0pt}%
\pgfpathmoveto{\pgfqpoint{0.481978in}{0.331635in}}%
\pgfpathlineto{\pgfqpoint{5.441978in}{0.331635in}}%
\pgfpathlineto{\pgfqpoint{5.441978in}{4.027635in}}%
\pgfpathlineto{\pgfqpoint{0.481978in}{4.027635in}}%
\pgfpathclose%
\pgfusepath{fill}%
\end{pgfscope}%
\begin{pgfscope}%
\pgfpathrectangle{\pgfqpoint{0.481978in}{0.331635in}}{\pgfqpoint{4.960000in}{3.696000in}}%
\pgfusepath{clip}%
\pgfsetbuttcap%
\pgfsetroundjoin%
\definecolor{currentfill}{rgb}{1.000000,0.705882,0.509804}%
\pgfsetfillcolor{currentfill}%
\pgfsetlinewidth{0.481800pt}%
\definecolor{currentstroke}{rgb}{1.000000,1.000000,1.000000}%
\pgfsetstrokecolor{currentstroke}%
\pgfsetdash{}{0pt}%
\pgfpathmoveto{\pgfqpoint{1.904809in}{1.341693in}}%
\pgfpathcurveto{\pgfqpoint{1.915859in}{1.341693in}}{\pgfqpoint{1.926458in}{1.346083in}}{\pgfqpoint{1.934272in}{1.353897in}}%
\pgfpathcurveto{\pgfqpoint{1.942086in}{1.361711in}}{\pgfqpoint{1.946476in}{1.372310in}}{\pgfqpoint{1.946476in}{1.383360in}}%
\pgfpathcurveto{\pgfqpoint{1.946476in}{1.394410in}}{\pgfqpoint{1.942086in}{1.405009in}}{\pgfqpoint{1.934272in}{1.412822in}}%
\pgfpathcurveto{\pgfqpoint{1.926458in}{1.420636in}}{\pgfqpoint{1.915859in}{1.425026in}}{\pgfqpoint{1.904809in}{1.425026in}}%
\pgfpathcurveto{\pgfqpoint{1.893759in}{1.425026in}}{\pgfqpoint{1.883160in}{1.420636in}}{\pgfqpoint{1.875346in}{1.412822in}}%
\pgfpathcurveto{\pgfqpoint{1.867533in}{1.405009in}}{\pgfqpoint{1.863143in}{1.394410in}}{\pgfqpoint{1.863143in}{1.383360in}}%
\pgfpathcurveto{\pgfqpoint{1.863143in}{1.372310in}}{\pgfqpoint{1.867533in}{1.361711in}}{\pgfqpoint{1.875346in}{1.353897in}}%
\pgfpathcurveto{\pgfqpoint{1.883160in}{1.346083in}}{\pgfqpoint{1.893759in}{1.341693in}}{\pgfqpoint{1.904809in}{1.341693in}}%
\pgfpathclose%
\pgfusepath{stroke,fill}%
\end{pgfscope}%
\begin{pgfscope}%
\pgfpathrectangle{\pgfqpoint{0.481978in}{0.331635in}}{\pgfqpoint{4.960000in}{3.696000in}}%
\pgfusepath{clip}%
\pgfsetbuttcap%
\pgfsetroundjoin%
\definecolor{currentfill}{rgb}{1.000000,0.705882,0.509804}%
\pgfsetfillcolor{currentfill}%
\pgfsetlinewidth{0.481800pt}%
\definecolor{currentstroke}{rgb}{1.000000,1.000000,1.000000}%
\pgfsetstrokecolor{currentstroke}%
\pgfsetdash{}{0pt}%
\pgfpathmoveto{\pgfqpoint{3.073685in}{3.469191in}}%
\pgfpathcurveto{\pgfqpoint{3.084735in}{3.469191in}}{\pgfqpoint{3.095334in}{3.473582in}}{\pgfqpoint{3.103148in}{3.481395in}}%
\pgfpathcurveto{\pgfqpoint{3.110962in}{3.489209in}}{\pgfqpoint{3.115352in}{3.499808in}}{\pgfqpoint{3.115352in}{3.510858in}}%
\pgfpathcurveto{\pgfqpoint{3.115352in}{3.521908in}}{\pgfqpoint{3.110962in}{3.532507in}}{\pgfqpoint{3.103148in}{3.540321in}}%
\pgfpathcurveto{\pgfqpoint{3.095334in}{3.548134in}}{\pgfqpoint{3.084735in}{3.552525in}}{\pgfqpoint{3.073685in}{3.552525in}}%
\pgfpathcurveto{\pgfqpoint{3.062635in}{3.552525in}}{\pgfqpoint{3.052036in}{3.548134in}}{\pgfqpoint{3.044222in}{3.540321in}}%
\pgfpathcurveto{\pgfqpoint{3.036409in}{3.532507in}}{\pgfqpoint{3.032018in}{3.521908in}}{\pgfqpoint{3.032018in}{3.510858in}}%
\pgfpathcurveto{\pgfqpoint{3.032018in}{3.499808in}}{\pgfqpoint{3.036409in}{3.489209in}}{\pgfqpoint{3.044222in}{3.481395in}}%
\pgfpathcurveto{\pgfqpoint{3.052036in}{3.473582in}}{\pgfqpoint{3.062635in}{3.469191in}}{\pgfqpoint{3.073685in}{3.469191in}}%
\pgfpathclose%
\pgfusepath{stroke,fill}%
\end{pgfscope}%
\begin{pgfscope}%
\pgfpathrectangle{\pgfqpoint{0.481978in}{0.331635in}}{\pgfqpoint{4.960000in}{3.696000in}}%
\pgfusepath{clip}%
\pgfsetbuttcap%
\pgfsetroundjoin%
\definecolor{currentfill}{rgb}{1.000000,0.705882,0.509804}%
\pgfsetfillcolor{currentfill}%
\pgfsetlinewidth{0.481800pt}%
\definecolor{currentstroke}{rgb}{1.000000,1.000000,1.000000}%
\pgfsetstrokecolor{currentstroke}%
\pgfsetdash{}{0pt}%
\pgfpathmoveto{\pgfqpoint{3.119479in}{3.433540in}}%
\pgfpathcurveto{\pgfqpoint{3.130530in}{3.433540in}}{\pgfqpoint{3.141129in}{3.437930in}}{\pgfqpoint{3.148942in}{3.445743in}}%
\pgfpathcurveto{\pgfqpoint{3.156756in}{3.453557in}}{\pgfqpoint{3.161146in}{3.464156in}}{\pgfqpoint{3.161146in}{3.475206in}}%
\pgfpathcurveto{\pgfqpoint{3.161146in}{3.486256in}}{\pgfqpoint{3.156756in}{3.496855in}}{\pgfqpoint{3.148942in}{3.504669in}}%
\pgfpathcurveto{\pgfqpoint{3.141129in}{3.512483in}}{\pgfqpoint{3.130530in}{3.516873in}}{\pgfqpoint{3.119479in}{3.516873in}}%
\pgfpathcurveto{\pgfqpoint{3.108429in}{3.516873in}}{\pgfqpoint{3.097830in}{3.512483in}}{\pgfqpoint{3.090017in}{3.504669in}}%
\pgfpathcurveto{\pgfqpoint{3.082203in}{3.496855in}}{\pgfqpoint{3.077813in}{3.486256in}}{\pgfqpoint{3.077813in}{3.475206in}}%
\pgfpathcurveto{\pgfqpoint{3.077813in}{3.464156in}}{\pgfqpoint{3.082203in}{3.453557in}}{\pgfqpoint{3.090017in}{3.445743in}}%
\pgfpathcurveto{\pgfqpoint{3.097830in}{3.437930in}}{\pgfqpoint{3.108429in}{3.433540in}}{\pgfqpoint{3.119479in}{3.433540in}}%
\pgfpathclose%
\pgfusepath{stroke,fill}%
\end{pgfscope}%
\begin{pgfscope}%
\pgfpathrectangle{\pgfqpoint{0.481978in}{0.331635in}}{\pgfqpoint{4.960000in}{3.696000in}}%
\pgfusepath{clip}%
\pgfsetbuttcap%
\pgfsetroundjoin%
\definecolor{currentfill}{rgb}{1.000000,0.705882,0.509804}%
\pgfsetfillcolor{currentfill}%
\pgfsetlinewidth{0.481800pt}%
\definecolor{currentstroke}{rgb}{1.000000,1.000000,1.000000}%
\pgfsetstrokecolor{currentstroke}%
\pgfsetdash{}{0pt}%
\pgfpathmoveto{\pgfqpoint{2.700059in}{2.049552in}}%
\pgfpathcurveto{\pgfqpoint{2.711109in}{2.049552in}}{\pgfqpoint{2.721708in}{2.053943in}}{\pgfqpoint{2.729522in}{2.061756in}}%
\pgfpathcurveto{\pgfqpoint{2.737336in}{2.069570in}}{\pgfqpoint{2.741726in}{2.080169in}}{\pgfqpoint{2.741726in}{2.091219in}}%
\pgfpathcurveto{\pgfqpoint{2.741726in}{2.102269in}}{\pgfqpoint{2.737336in}{2.112868in}}{\pgfqpoint{2.729522in}{2.120682in}}%
\pgfpathcurveto{\pgfqpoint{2.721708in}{2.128495in}}{\pgfqpoint{2.711109in}{2.132886in}}{\pgfqpoint{2.700059in}{2.132886in}}%
\pgfpathcurveto{\pgfqpoint{2.689009in}{2.132886in}}{\pgfqpoint{2.678410in}{2.128495in}}{\pgfqpoint{2.670597in}{2.120682in}}%
\pgfpathcurveto{\pgfqpoint{2.662783in}{2.112868in}}{\pgfqpoint{2.658393in}{2.102269in}}{\pgfqpoint{2.658393in}{2.091219in}}%
\pgfpathcurveto{\pgfqpoint{2.658393in}{2.080169in}}{\pgfqpoint{2.662783in}{2.069570in}}{\pgfqpoint{2.670597in}{2.061756in}}%
\pgfpathcurveto{\pgfqpoint{2.678410in}{2.053943in}}{\pgfqpoint{2.689009in}{2.049552in}}{\pgfqpoint{2.700059in}{2.049552in}}%
\pgfpathclose%
\pgfusepath{stroke,fill}%
\end{pgfscope}%
\begin{pgfscope}%
\pgfpathrectangle{\pgfqpoint{0.481978in}{0.331635in}}{\pgfqpoint{4.960000in}{3.696000in}}%
\pgfusepath{clip}%
\pgfsetbuttcap%
\pgfsetroundjoin%
\definecolor{currentfill}{rgb}{1.000000,0.705882,0.509804}%
\pgfsetfillcolor{currentfill}%
\pgfsetlinewidth{0.481800pt}%
\definecolor{currentstroke}{rgb}{1.000000,1.000000,1.000000}%
\pgfsetstrokecolor{currentstroke}%
\pgfsetdash{}{0pt}%
\pgfpathmoveto{\pgfqpoint{1.891481in}{2.941651in}}%
\pgfpathcurveto{\pgfqpoint{1.902531in}{2.941651in}}{\pgfqpoint{1.913130in}{2.946041in}}{\pgfqpoint{1.920944in}{2.953855in}}%
\pgfpathcurveto{\pgfqpoint{1.928757in}{2.961669in}}{\pgfqpoint{1.933147in}{2.972268in}}{\pgfqpoint{1.933147in}{2.983318in}}%
\pgfpathcurveto{\pgfqpoint{1.933147in}{2.994368in}}{\pgfqpoint{1.928757in}{3.004967in}}{\pgfqpoint{1.920944in}{3.012780in}}%
\pgfpathcurveto{\pgfqpoint{1.913130in}{3.020594in}}{\pgfqpoint{1.902531in}{3.024984in}}{\pgfqpoint{1.891481in}{3.024984in}}%
\pgfpathcurveto{\pgfqpoint{1.880431in}{3.024984in}}{\pgfqpoint{1.869832in}{3.020594in}}{\pgfqpoint{1.862018in}{3.012780in}}%
\pgfpathcurveto{\pgfqpoint{1.854204in}{3.004967in}}{\pgfqpoint{1.849814in}{2.994368in}}{\pgfqpoint{1.849814in}{2.983318in}}%
\pgfpathcurveto{\pgfqpoint{1.849814in}{2.972268in}}{\pgfqpoint{1.854204in}{2.961669in}}{\pgfqpoint{1.862018in}{2.953855in}}%
\pgfpathcurveto{\pgfqpoint{1.869832in}{2.946041in}}{\pgfqpoint{1.880431in}{2.941651in}}{\pgfqpoint{1.891481in}{2.941651in}}%
\pgfpathclose%
\pgfusepath{stroke,fill}%
\end{pgfscope}%
\begin{pgfscope}%
\pgfpathrectangle{\pgfqpoint{0.481978in}{0.331635in}}{\pgfqpoint{4.960000in}{3.696000in}}%
\pgfusepath{clip}%
\pgfsetbuttcap%
\pgfsetroundjoin%
\definecolor{currentfill}{rgb}{1.000000,0.705882,0.509804}%
\pgfsetfillcolor{currentfill}%
\pgfsetlinewidth{0.481800pt}%
\definecolor{currentstroke}{rgb}{1.000000,1.000000,1.000000}%
\pgfsetstrokecolor{currentstroke}%
\pgfsetdash{}{0pt}%
\pgfpathmoveto{\pgfqpoint{2.666389in}{2.796602in}}%
\pgfpathcurveto{\pgfqpoint{2.677439in}{2.796602in}}{\pgfqpoint{2.688038in}{2.800993in}}{\pgfqpoint{2.695852in}{2.808806in}}%
\pgfpathcurveto{\pgfqpoint{2.703665in}{2.816620in}}{\pgfqpoint{2.708056in}{2.827219in}}{\pgfqpoint{2.708056in}{2.838269in}}%
\pgfpathcurveto{\pgfqpoint{2.708056in}{2.849319in}}{\pgfqpoint{2.703665in}{2.859918in}}{\pgfqpoint{2.695852in}{2.867732in}}%
\pgfpathcurveto{\pgfqpoint{2.688038in}{2.875546in}}{\pgfqpoint{2.677439in}{2.879936in}}{\pgfqpoint{2.666389in}{2.879936in}}%
\pgfpathcurveto{\pgfqpoint{2.655339in}{2.879936in}}{\pgfqpoint{2.644740in}{2.875546in}}{\pgfqpoint{2.636926in}{2.867732in}}%
\pgfpathcurveto{\pgfqpoint{2.629113in}{2.859918in}}{\pgfqpoint{2.624722in}{2.849319in}}{\pgfqpoint{2.624722in}{2.838269in}}%
\pgfpathcurveto{\pgfqpoint{2.624722in}{2.827219in}}{\pgfqpoint{2.629113in}{2.816620in}}{\pgfqpoint{2.636926in}{2.808806in}}%
\pgfpathcurveto{\pgfqpoint{2.644740in}{2.800993in}}{\pgfqpoint{2.655339in}{2.796602in}}{\pgfqpoint{2.666389in}{2.796602in}}%
\pgfpathclose%
\pgfusepath{stroke,fill}%
\end{pgfscope}%
\begin{pgfscope}%
\pgfpathrectangle{\pgfqpoint{0.481978in}{0.331635in}}{\pgfqpoint{4.960000in}{3.696000in}}%
\pgfusepath{clip}%
\pgfsetbuttcap%
\pgfsetroundjoin%
\definecolor{currentfill}{rgb}{1.000000,0.705882,0.509804}%
\pgfsetfillcolor{currentfill}%
\pgfsetlinewidth{0.481800pt}%
\definecolor{currentstroke}{rgb}{1.000000,1.000000,1.000000}%
\pgfsetstrokecolor{currentstroke}%
\pgfsetdash{}{0pt}%
\pgfpathmoveto{\pgfqpoint{2.524989in}{2.887274in}}%
\pgfpathcurveto{\pgfqpoint{2.536039in}{2.887274in}}{\pgfqpoint{2.546638in}{2.891664in}}{\pgfqpoint{2.554452in}{2.899478in}}%
\pgfpathcurveto{\pgfqpoint{2.562266in}{2.907292in}}{\pgfqpoint{2.566656in}{2.917891in}}{\pgfqpoint{2.566656in}{2.928941in}}%
\pgfpathcurveto{\pgfqpoint{2.566656in}{2.939991in}}{\pgfqpoint{2.562266in}{2.950590in}}{\pgfqpoint{2.554452in}{2.958404in}}%
\pgfpathcurveto{\pgfqpoint{2.546638in}{2.966217in}}{\pgfqpoint{2.536039in}{2.970607in}}{\pgfqpoint{2.524989in}{2.970607in}}%
\pgfpathcurveto{\pgfqpoint{2.513939in}{2.970607in}}{\pgfqpoint{2.503340in}{2.966217in}}{\pgfqpoint{2.495526in}{2.958404in}}%
\pgfpathcurveto{\pgfqpoint{2.487713in}{2.950590in}}{\pgfqpoint{2.483322in}{2.939991in}}{\pgfqpoint{2.483322in}{2.928941in}}%
\pgfpathcurveto{\pgfqpoint{2.483322in}{2.917891in}}{\pgfqpoint{2.487713in}{2.907292in}}{\pgfqpoint{2.495526in}{2.899478in}}%
\pgfpathcurveto{\pgfqpoint{2.503340in}{2.891664in}}{\pgfqpoint{2.513939in}{2.887274in}}{\pgfqpoint{2.524989in}{2.887274in}}%
\pgfpathclose%
\pgfusepath{stroke,fill}%
\end{pgfscope}%
\begin{pgfscope}%
\pgfpathrectangle{\pgfqpoint{0.481978in}{0.331635in}}{\pgfqpoint{4.960000in}{3.696000in}}%
\pgfusepath{clip}%
\pgfsetbuttcap%
\pgfsetroundjoin%
\definecolor{currentfill}{rgb}{1.000000,0.705882,0.509804}%
\pgfsetfillcolor{currentfill}%
\pgfsetlinewidth{0.481800pt}%
\definecolor{currentstroke}{rgb}{1.000000,1.000000,1.000000}%
\pgfsetstrokecolor{currentstroke}%
\pgfsetdash{}{0pt}%
\pgfpathmoveto{\pgfqpoint{2.128863in}{2.799654in}}%
\pgfpathcurveto{\pgfqpoint{2.139914in}{2.799654in}}{\pgfqpoint{2.150513in}{2.804044in}}{\pgfqpoint{2.158326in}{2.811858in}}%
\pgfpathcurveto{\pgfqpoint{2.166140in}{2.819671in}}{\pgfqpoint{2.170530in}{2.830270in}}{\pgfqpoint{2.170530in}{2.841321in}}%
\pgfpathcurveto{\pgfqpoint{2.170530in}{2.852371in}}{\pgfqpoint{2.166140in}{2.862970in}}{\pgfqpoint{2.158326in}{2.870783in}}%
\pgfpathcurveto{\pgfqpoint{2.150513in}{2.878597in}}{\pgfqpoint{2.139914in}{2.882987in}}{\pgfqpoint{2.128863in}{2.882987in}}%
\pgfpathcurveto{\pgfqpoint{2.117813in}{2.882987in}}{\pgfqpoint{2.107214in}{2.878597in}}{\pgfqpoint{2.099401in}{2.870783in}}%
\pgfpathcurveto{\pgfqpoint{2.091587in}{2.862970in}}{\pgfqpoint{2.087197in}{2.852371in}}{\pgfqpoint{2.087197in}{2.841321in}}%
\pgfpathcurveto{\pgfqpoint{2.087197in}{2.830270in}}{\pgfqpoint{2.091587in}{2.819671in}}{\pgfqpoint{2.099401in}{2.811858in}}%
\pgfpathcurveto{\pgfqpoint{2.107214in}{2.804044in}}{\pgfqpoint{2.117813in}{2.799654in}}{\pgfqpoint{2.128863in}{2.799654in}}%
\pgfpathclose%
\pgfusepath{stroke,fill}%
\end{pgfscope}%
\begin{pgfscope}%
\pgfpathrectangle{\pgfqpoint{0.481978in}{0.331635in}}{\pgfqpoint{4.960000in}{3.696000in}}%
\pgfusepath{clip}%
\pgfsetbuttcap%
\pgfsetroundjoin%
\definecolor{currentfill}{rgb}{1.000000,0.705882,0.509804}%
\pgfsetfillcolor{currentfill}%
\pgfsetlinewidth{0.481800pt}%
\definecolor{currentstroke}{rgb}{1.000000,1.000000,1.000000}%
\pgfsetstrokecolor{currentstroke}%
\pgfsetdash{}{0pt}%
\pgfpathmoveto{\pgfqpoint{4.812092in}{2.858201in}}%
\pgfpathcurveto{\pgfqpoint{4.823142in}{2.858201in}}{\pgfqpoint{4.833741in}{2.862592in}}{\pgfqpoint{4.841555in}{2.870405in}}%
\pgfpathcurveto{\pgfqpoint{4.849368in}{2.878219in}}{\pgfqpoint{4.853759in}{2.888818in}}{\pgfqpoint{4.853759in}{2.899868in}}%
\pgfpathcurveto{\pgfqpoint{4.853759in}{2.910918in}}{\pgfqpoint{4.849368in}{2.921517in}}{\pgfqpoint{4.841555in}{2.929331in}}%
\pgfpathcurveto{\pgfqpoint{4.833741in}{2.937145in}}{\pgfqpoint{4.823142in}{2.941535in}}{\pgfqpoint{4.812092in}{2.941535in}}%
\pgfpathcurveto{\pgfqpoint{4.801042in}{2.941535in}}{\pgfqpoint{4.790443in}{2.937145in}}{\pgfqpoint{4.782629in}{2.929331in}}%
\pgfpathcurveto{\pgfqpoint{4.774815in}{2.921517in}}{\pgfqpoint{4.770425in}{2.910918in}}{\pgfqpoint{4.770425in}{2.899868in}}%
\pgfpathcurveto{\pgfqpoint{4.770425in}{2.888818in}}{\pgfqpoint{4.774815in}{2.878219in}}{\pgfqpoint{4.782629in}{2.870405in}}%
\pgfpathcurveto{\pgfqpoint{4.790443in}{2.862592in}}{\pgfqpoint{4.801042in}{2.858201in}}{\pgfqpoint{4.812092in}{2.858201in}}%
\pgfpathclose%
\pgfusepath{stroke,fill}%
\end{pgfscope}%
\begin{pgfscope}%
\pgfpathrectangle{\pgfqpoint{0.481978in}{0.331635in}}{\pgfqpoint{4.960000in}{3.696000in}}%
\pgfusepath{clip}%
\pgfsetbuttcap%
\pgfsetroundjoin%
\definecolor{currentfill}{rgb}{1.000000,0.705882,0.509804}%
\pgfsetfillcolor{currentfill}%
\pgfsetlinewidth{0.481800pt}%
\definecolor{currentstroke}{rgb}{1.000000,1.000000,1.000000}%
\pgfsetstrokecolor{currentstroke}%
\pgfsetdash{}{0pt}%
\pgfpathmoveto{\pgfqpoint{3.248808in}{2.856948in}}%
\pgfpathcurveto{\pgfqpoint{3.259858in}{2.856948in}}{\pgfqpoint{3.270457in}{2.861338in}}{\pgfqpoint{3.278271in}{2.869152in}}%
\pgfpathcurveto{\pgfqpoint{3.286084in}{2.876965in}}{\pgfqpoint{3.290474in}{2.887564in}}{\pgfqpoint{3.290474in}{2.898614in}}%
\pgfpathcurveto{\pgfqpoint{3.290474in}{2.909665in}}{\pgfqpoint{3.286084in}{2.920264in}}{\pgfqpoint{3.278271in}{2.928077in}}%
\pgfpathcurveto{\pgfqpoint{3.270457in}{2.935891in}}{\pgfqpoint{3.259858in}{2.940281in}}{\pgfqpoint{3.248808in}{2.940281in}}%
\pgfpathcurveto{\pgfqpoint{3.237758in}{2.940281in}}{\pgfqpoint{3.227159in}{2.935891in}}{\pgfqpoint{3.219345in}{2.928077in}}%
\pgfpathcurveto{\pgfqpoint{3.211531in}{2.920264in}}{\pgfqpoint{3.207141in}{2.909665in}}{\pgfqpoint{3.207141in}{2.898614in}}%
\pgfpathcurveto{\pgfqpoint{3.207141in}{2.887564in}}{\pgfqpoint{3.211531in}{2.876965in}}{\pgfqpoint{3.219345in}{2.869152in}}%
\pgfpathcurveto{\pgfqpoint{3.227159in}{2.861338in}}{\pgfqpoint{3.237758in}{2.856948in}}{\pgfqpoint{3.248808in}{2.856948in}}%
\pgfpathclose%
\pgfusepath{stroke,fill}%
\end{pgfscope}%
\begin{pgfscope}%
\pgfpathrectangle{\pgfqpoint{0.481978in}{0.331635in}}{\pgfqpoint{4.960000in}{3.696000in}}%
\pgfusepath{clip}%
\pgfsetbuttcap%
\pgfsetroundjoin%
\definecolor{currentfill}{rgb}{1.000000,0.705882,0.509804}%
\pgfsetfillcolor{currentfill}%
\pgfsetlinewidth{0.481800pt}%
\definecolor{currentstroke}{rgb}{1.000000,1.000000,1.000000}%
\pgfsetstrokecolor{currentstroke}%
\pgfsetdash{}{0pt}%
\pgfpathmoveto{\pgfqpoint{2.522551in}{3.622458in}}%
\pgfpathcurveto{\pgfqpoint{2.533602in}{3.622458in}}{\pgfqpoint{2.544201in}{3.626849in}}{\pgfqpoint{2.552014in}{3.634662in}}%
\pgfpathcurveto{\pgfqpoint{2.559828in}{3.642476in}}{\pgfqpoint{2.564218in}{3.653075in}}{\pgfqpoint{2.564218in}{3.664125in}}%
\pgfpathcurveto{\pgfqpoint{2.564218in}{3.675175in}}{\pgfqpoint{2.559828in}{3.685774in}}{\pgfqpoint{2.552014in}{3.693588in}}%
\pgfpathcurveto{\pgfqpoint{2.544201in}{3.701401in}}{\pgfqpoint{2.533602in}{3.705792in}}{\pgfqpoint{2.522551in}{3.705792in}}%
\pgfpathcurveto{\pgfqpoint{2.511501in}{3.705792in}}{\pgfqpoint{2.500902in}{3.701401in}}{\pgfqpoint{2.493089in}{3.693588in}}%
\pgfpathcurveto{\pgfqpoint{2.485275in}{3.685774in}}{\pgfqpoint{2.480885in}{3.675175in}}{\pgfqpoint{2.480885in}{3.664125in}}%
\pgfpathcurveto{\pgfqpoint{2.480885in}{3.653075in}}{\pgfqpoint{2.485275in}{3.642476in}}{\pgfqpoint{2.493089in}{3.634662in}}%
\pgfpathcurveto{\pgfqpoint{2.500902in}{3.626849in}}{\pgfqpoint{2.511501in}{3.622458in}}{\pgfqpoint{2.522551in}{3.622458in}}%
\pgfpathclose%
\pgfusepath{stroke,fill}%
\end{pgfscope}%
\begin{pgfscope}%
\pgfpathrectangle{\pgfqpoint{0.481978in}{0.331635in}}{\pgfqpoint{4.960000in}{3.696000in}}%
\pgfusepath{clip}%
\pgfsetbuttcap%
\pgfsetroundjoin%
\definecolor{currentfill}{rgb}{1.000000,0.705882,0.509804}%
\pgfsetfillcolor{currentfill}%
\pgfsetlinewidth{0.481800pt}%
\definecolor{currentstroke}{rgb}{1.000000,1.000000,1.000000}%
\pgfsetstrokecolor{currentstroke}%
\pgfsetdash{}{0pt}%
\pgfpathmoveto{\pgfqpoint{0.846816in}{2.641664in}}%
\pgfpathcurveto{\pgfqpoint{0.857866in}{2.641664in}}{\pgfqpoint{0.868465in}{2.646055in}}{\pgfqpoint{0.876279in}{2.653868in}}%
\pgfpathcurveto{\pgfqpoint{0.884093in}{2.661682in}}{\pgfqpoint{0.888483in}{2.672281in}}{\pgfqpoint{0.888483in}{2.683331in}}%
\pgfpathcurveto{\pgfqpoint{0.888483in}{2.694381in}}{\pgfqpoint{0.884093in}{2.704980in}}{\pgfqpoint{0.876279in}{2.712794in}}%
\pgfpathcurveto{\pgfqpoint{0.868465in}{2.720607in}}{\pgfqpoint{0.857866in}{2.724998in}}{\pgfqpoint{0.846816in}{2.724998in}}%
\pgfpathcurveto{\pgfqpoint{0.835766in}{2.724998in}}{\pgfqpoint{0.825167in}{2.720607in}}{\pgfqpoint{0.817354in}{2.712794in}}%
\pgfpathcurveto{\pgfqpoint{0.809540in}{2.704980in}}{\pgfqpoint{0.805150in}{2.694381in}}{\pgfqpoint{0.805150in}{2.683331in}}%
\pgfpathcurveto{\pgfqpoint{0.805150in}{2.672281in}}{\pgfqpoint{0.809540in}{2.661682in}}{\pgfqpoint{0.817354in}{2.653868in}}%
\pgfpathcurveto{\pgfqpoint{0.825167in}{2.646055in}}{\pgfqpoint{0.835766in}{2.641664in}}{\pgfqpoint{0.846816in}{2.641664in}}%
\pgfpathclose%
\pgfusepath{stroke,fill}%
\end{pgfscope}%
\begin{pgfscope}%
\pgfpathrectangle{\pgfqpoint{0.481978in}{0.331635in}}{\pgfqpoint{4.960000in}{3.696000in}}%
\pgfusepath{clip}%
\pgfsetbuttcap%
\pgfsetroundjoin%
\definecolor{currentfill}{rgb}{1.000000,0.705882,0.509804}%
\pgfsetfillcolor{currentfill}%
\pgfsetlinewidth{0.481800pt}%
\definecolor{currentstroke}{rgb}{1.000000,1.000000,1.000000}%
\pgfsetstrokecolor{currentstroke}%
\pgfsetdash{}{0pt}%
\pgfpathmoveto{\pgfqpoint{1.393074in}{2.404771in}}%
\pgfpathcurveto{\pgfqpoint{1.404125in}{2.404771in}}{\pgfqpoint{1.414724in}{2.409161in}}{\pgfqpoint{1.422537in}{2.416974in}}%
\pgfpathcurveto{\pgfqpoint{1.430351in}{2.424788in}}{\pgfqpoint{1.434741in}{2.435387in}}{\pgfqpoint{1.434741in}{2.446437in}}%
\pgfpathcurveto{\pgfqpoint{1.434741in}{2.457487in}}{\pgfqpoint{1.430351in}{2.468086in}}{\pgfqpoint{1.422537in}{2.475900in}}%
\pgfpathcurveto{\pgfqpoint{1.414724in}{2.483714in}}{\pgfqpoint{1.404125in}{2.488104in}}{\pgfqpoint{1.393074in}{2.488104in}}%
\pgfpathcurveto{\pgfqpoint{1.382024in}{2.488104in}}{\pgfqpoint{1.371425in}{2.483714in}}{\pgfqpoint{1.363612in}{2.475900in}}%
\pgfpathcurveto{\pgfqpoint{1.355798in}{2.468086in}}{\pgfqpoint{1.351408in}{2.457487in}}{\pgfqpoint{1.351408in}{2.446437in}}%
\pgfpathcurveto{\pgfqpoint{1.351408in}{2.435387in}}{\pgfqpoint{1.355798in}{2.424788in}}{\pgfqpoint{1.363612in}{2.416974in}}%
\pgfpathcurveto{\pgfqpoint{1.371425in}{2.409161in}}{\pgfqpoint{1.382024in}{2.404771in}}{\pgfqpoint{1.393074in}{2.404771in}}%
\pgfpathclose%
\pgfusepath{stroke,fill}%
\end{pgfscope}%
\begin{pgfscope}%
\pgfpathrectangle{\pgfqpoint{0.481978in}{0.331635in}}{\pgfqpoint{4.960000in}{3.696000in}}%
\pgfusepath{clip}%
\pgfsetbuttcap%
\pgfsetroundjoin%
\definecolor{currentfill}{rgb}{1.000000,0.705882,0.509804}%
\pgfsetfillcolor{currentfill}%
\pgfsetlinewidth{0.481800pt}%
\definecolor{currentstroke}{rgb}{1.000000,1.000000,1.000000}%
\pgfsetstrokecolor{currentstroke}%
\pgfsetdash{}{0pt}%
\pgfpathmoveto{\pgfqpoint{3.040652in}{2.123928in}}%
\pgfpathcurveto{\pgfqpoint{3.051702in}{2.123928in}}{\pgfqpoint{3.062301in}{2.128318in}}{\pgfqpoint{3.070115in}{2.136132in}}%
\pgfpathcurveto{\pgfqpoint{3.077928in}{2.143946in}}{\pgfqpoint{3.082318in}{2.154545in}}{\pgfqpoint{3.082318in}{2.165595in}}%
\pgfpathcurveto{\pgfqpoint{3.082318in}{2.176645in}}{\pgfqpoint{3.077928in}{2.187244in}}{\pgfqpoint{3.070115in}{2.195057in}}%
\pgfpathcurveto{\pgfqpoint{3.062301in}{2.202871in}}{\pgfqpoint{3.051702in}{2.207261in}}{\pgfqpoint{3.040652in}{2.207261in}}%
\pgfpathcurveto{\pgfqpoint{3.029602in}{2.207261in}}{\pgfqpoint{3.019003in}{2.202871in}}{\pgfqpoint{3.011189in}{2.195057in}}%
\pgfpathcurveto{\pgfqpoint{3.003375in}{2.187244in}}{\pgfqpoint{2.998985in}{2.176645in}}{\pgfqpoint{2.998985in}{2.165595in}}%
\pgfpathcurveto{\pgfqpoint{2.998985in}{2.154545in}}{\pgfqpoint{3.003375in}{2.143946in}}{\pgfqpoint{3.011189in}{2.136132in}}%
\pgfpathcurveto{\pgfqpoint{3.019003in}{2.128318in}}{\pgfqpoint{3.029602in}{2.123928in}}{\pgfqpoint{3.040652in}{2.123928in}}%
\pgfpathclose%
\pgfusepath{stroke,fill}%
\end{pgfscope}%
\begin{pgfscope}%
\pgfpathrectangle{\pgfqpoint{0.481978in}{0.331635in}}{\pgfqpoint{4.960000in}{3.696000in}}%
\pgfusepath{clip}%
\pgfsetbuttcap%
\pgfsetroundjoin%
\definecolor{currentfill}{rgb}{1.000000,0.705882,0.509804}%
\pgfsetfillcolor{currentfill}%
\pgfsetlinewidth{0.481800pt}%
\definecolor{currentstroke}{rgb}{1.000000,1.000000,1.000000}%
\pgfsetstrokecolor{currentstroke}%
\pgfsetdash{}{0pt}%
\pgfpathmoveto{\pgfqpoint{3.072022in}{2.304356in}}%
\pgfpathcurveto{\pgfqpoint{3.083073in}{2.304356in}}{\pgfqpoint{3.093672in}{2.308746in}}{\pgfqpoint{3.101485in}{2.316560in}}%
\pgfpathcurveto{\pgfqpoint{3.109299in}{2.324373in}}{\pgfqpoint{3.113689in}{2.334973in}}{\pgfqpoint{3.113689in}{2.346023in}}%
\pgfpathcurveto{\pgfqpoint{3.113689in}{2.357073in}}{\pgfqpoint{3.109299in}{2.367672in}}{\pgfqpoint{3.101485in}{2.375485in}}%
\pgfpathcurveto{\pgfqpoint{3.093672in}{2.383299in}}{\pgfqpoint{3.083073in}{2.387689in}}{\pgfqpoint{3.072022in}{2.387689in}}%
\pgfpathcurveto{\pgfqpoint{3.060972in}{2.387689in}}{\pgfqpoint{3.050373in}{2.383299in}}{\pgfqpoint{3.042560in}{2.375485in}}%
\pgfpathcurveto{\pgfqpoint{3.034746in}{2.367672in}}{\pgfqpoint{3.030356in}{2.357073in}}{\pgfqpoint{3.030356in}{2.346023in}}%
\pgfpathcurveto{\pgfqpoint{3.030356in}{2.334973in}}{\pgfqpoint{3.034746in}{2.324373in}}{\pgfqpoint{3.042560in}{2.316560in}}%
\pgfpathcurveto{\pgfqpoint{3.050373in}{2.308746in}}{\pgfqpoint{3.060972in}{2.304356in}}{\pgfqpoint{3.072022in}{2.304356in}}%
\pgfpathclose%
\pgfusepath{stroke,fill}%
\end{pgfscope}%
\begin{pgfscope}%
\pgfpathrectangle{\pgfqpoint{0.481978in}{0.331635in}}{\pgfqpoint{4.960000in}{3.696000in}}%
\pgfusepath{clip}%
\pgfsetbuttcap%
\pgfsetroundjoin%
\definecolor{currentfill}{rgb}{1.000000,0.705882,0.509804}%
\pgfsetfillcolor{currentfill}%
\pgfsetlinewidth{0.481800pt}%
\definecolor{currentstroke}{rgb}{1.000000,1.000000,1.000000}%
\pgfsetstrokecolor{currentstroke}%
\pgfsetdash{}{0pt}%
\pgfpathmoveto{\pgfqpoint{0.985638in}{2.118117in}}%
\pgfpathcurveto{\pgfqpoint{0.996688in}{2.118117in}}{\pgfqpoint{1.007287in}{2.122508in}}{\pgfqpoint{1.015101in}{2.130321in}}%
\pgfpathcurveto{\pgfqpoint{1.022914in}{2.138135in}}{\pgfqpoint{1.027305in}{2.148734in}}{\pgfqpoint{1.027305in}{2.159784in}}%
\pgfpathcurveto{\pgfqpoint{1.027305in}{2.170834in}}{\pgfqpoint{1.022914in}{2.181433in}}{\pgfqpoint{1.015101in}{2.189247in}}%
\pgfpathcurveto{\pgfqpoint{1.007287in}{2.197061in}}{\pgfqpoint{0.996688in}{2.201451in}}{\pgfqpoint{0.985638in}{2.201451in}}%
\pgfpathcurveto{\pgfqpoint{0.974588in}{2.201451in}}{\pgfqpoint{0.963989in}{2.197061in}}{\pgfqpoint{0.956175in}{2.189247in}}%
\pgfpathcurveto{\pgfqpoint{0.948362in}{2.181433in}}{\pgfqpoint{0.943971in}{2.170834in}}{\pgfqpoint{0.943971in}{2.159784in}}%
\pgfpathcurveto{\pgfqpoint{0.943971in}{2.148734in}}{\pgfqpoint{0.948362in}{2.138135in}}{\pgfqpoint{0.956175in}{2.130321in}}%
\pgfpathcurveto{\pgfqpoint{0.963989in}{2.122508in}}{\pgfqpoint{0.974588in}{2.118117in}}{\pgfqpoint{0.985638in}{2.118117in}}%
\pgfpathclose%
\pgfusepath{stroke,fill}%
\end{pgfscope}%
\begin{pgfscope}%
\pgfpathrectangle{\pgfqpoint{0.481978in}{0.331635in}}{\pgfqpoint{4.960000in}{3.696000in}}%
\pgfusepath{clip}%
\pgfsetbuttcap%
\pgfsetroundjoin%
\definecolor{currentfill}{rgb}{1.000000,0.705882,0.509804}%
\pgfsetfillcolor{currentfill}%
\pgfsetlinewidth{0.481800pt}%
\definecolor{currentstroke}{rgb}{1.000000,1.000000,1.000000}%
\pgfsetstrokecolor{currentstroke}%
\pgfsetdash{}{0pt}%
\pgfpathmoveto{\pgfqpoint{3.472787in}{3.492523in}}%
\pgfpathcurveto{\pgfqpoint{3.483838in}{3.492523in}}{\pgfqpoint{3.494437in}{3.496914in}}{\pgfqpoint{3.502250in}{3.504727in}}%
\pgfpathcurveto{\pgfqpoint{3.510064in}{3.512541in}}{\pgfqpoint{3.514454in}{3.523140in}}{\pgfqpoint{3.514454in}{3.534190in}}%
\pgfpathcurveto{\pgfqpoint{3.514454in}{3.545240in}}{\pgfqpoint{3.510064in}{3.555839in}}{\pgfqpoint{3.502250in}{3.563653in}}%
\pgfpathcurveto{\pgfqpoint{3.494437in}{3.571466in}}{\pgfqpoint{3.483838in}{3.575857in}}{\pgfqpoint{3.472787in}{3.575857in}}%
\pgfpathcurveto{\pgfqpoint{3.461737in}{3.575857in}}{\pgfqpoint{3.451138in}{3.571466in}}{\pgfqpoint{3.443325in}{3.563653in}}%
\pgfpathcurveto{\pgfqpoint{3.435511in}{3.555839in}}{\pgfqpoint{3.431121in}{3.545240in}}{\pgfqpoint{3.431121in}{3.534190in}}%
\pgfpathcurveto{\pgfqpoint{3.431121in}{3.523140in}}{\pgfqpoint{3.435511in}{3.512541in}}{\pgfqpoint{3.443325in}{3.504727in}}%
\pgfpathcurveto{\pgfqpoint{3.451138in}{3.496914in}}{\pgfqpoint{3.461737in}{3.492523in}}{\pgfqpoint{3.472787in}{3.492523in}}%
\pgfpathclose%
\pgfusepath{stroke,fill}%
\end{pgfscope}%
\begin{pgfscope}%
\pgfpathrectangle{\pgfqpoint{0.481978in}{0.331635in}}{\pgfqpoint{4.960000in}{3.696000in}}%
\pgfusepath{clip}%
\pgfsetbuttcap%
\pgfsetroundjoin%
\definecolor{currentfill}{rgb}{1.000000,0.705882,0.509804}%
\pgfsetfillcolor{currentfill}%
\pgfsetlinewidth{0.481800pt}%
\definecolor{currentstroke}{rgb}{1.000000,1.000000,1.000000}%
\pgfsetstrokecolor{currentstroke}%
\pgfsetdash{}{0pt}%
\pgfpathmoveto{\pgfqpoint{4.278318in}{0.868558in}}%
\pgfpathcurveto{\pgfqpoint{4.289369in}{0.868558in}}{\pgfqpoint{4.299968in}{0.872948in}}{\pgfqpoint{4.307781in}{0.880762in}}%
\pgfpathcurveto{\pgfqpoint{4.315595in}{0.888576in}}{\pgfqpoint{4.319985in}{0.899175in}}{\pgfqpoint{4.319985in}{0.910225in}}%
\pgfpathcurveto{\pgfqpoint{4.319985in}{0.921275in}}{\pgfqpoint{4.315595in}{0.931874in}}{\pgfqpoint{4.307781in}{0.939688in}}%
\pgfpathcurveto{\pgfqpoint{4.299968in}{0.947501in}}{\pgfqpoint{4.289369in}{0.951891in}}{\pgfqpoint{4.278318in}{0.951891in}}%
\pgfpathcurveto{\pgfqpoint{4.267268in}{0.951891in}}{\pgfqpoint{4.256669in}{0.947501in}}{\pgfqpoint{4.248856in}{0.939688in}}%
\pgfpathcurveto{\pgfqpoint{4.241042in}{0.931874in}}{\pgfqpoint{4.236652in}{0.921275in}}{\pgfqpoint{4.236652in}{0.910225in}}%
\pgfpathcurveto{\pgfqpoint{4.236652in}{0.899175in}}{\pgfqpoint{4.241042in}{0.888576in}}{\pgfqpoint{4.248856in}{0.880762in}}%
\pgfpathcurveto{\pgfqpoint{4.256669in}{0.872948in}}{\pgfqpoint{4.267268in}{0.868558in}}{\pgfqpoint{4.278318in}{0.868558in}}%
\pgfpathclose%
\pgfusepath{stroke,fill}%
\end{pgfscope}%
\begin{pgfscope}%
\pgfpathrectangle{\pgfqpoint{0.481978in}{0.331635in}}{\pgfqpoint{4.960000in}{3.696000in}}%
\pgfusepath{clip}%
\pgfsetbuttcap%
\pgfsetroundjoin%
\definecolor{currentfill}{rgb}{1.000000,0.705882,0.509804}%
\pgfsetfillcolor{currentfill}%
\pgfsetlinewidth{0.481800pt}%
\definecolor{currentstroke}{rgb}{1.000000,1.000000,1.000000}%
\pgfsetstrokecolor{currentstroke}%
\pgfsetdash{}{0pt}%
\pgfpathmoveto{\pgfqpoint{2.356273in}{1.996820in}}%
\pgfpathcurveto{\pgfqpoint{2.367324in}{1.996820in}}{\pgfqpoint{2.377923in}{2.001210in}}{\pgfqpoint{2.385736in}{2.009024in}}%
\pgfpathcurveto{\pgfqpoint{2.393550in}{2.016838in}}{\pgfqpoint{2.397940in}{2.027437in}}{\pgfqpoint{2.397940in}{2.038487in}}%
\pgfpathcurveto{\pgfqpoint{2.397940in}{2.049537in}}{\pgfqpoint{2.393550in}{2.060136in}}{\pgfqpoint{2.385736in}{2.067950in}}%
\pgfpathcurveto{\pgfqpoint{2.377923in}{2.075763in}}{\pgfqpoint{2.367324in}{2.080153in}}{\pgfqpoint{2.356273in}{2.080153in}}%
\pgfpathcurveto{\pgfqpoint{2.345223in}{2.080153in}}{\pgfqpoint{2.334624in}{2.075763in}}{\pgfqpoint{2.326811in}{2.067950in}}%
\pgfpathcurveto{\pgfqpoint{2.318997in}{2.060136in}}{\pgfqpoint{2.314607in}{2.049537in}}{\pgfqpoint{2.314607in}{2.038487in}}%
\pgfpathcurveto{\pgfqpoint{2.314607in}{2.027437in}}{\pgfqpoint{2.318997in}{2.016838in}}{\pgfqpoint{2.326811in}{2.009024in}}%
\pgfpathcurveto{\pgfqpoint{2.334624in}{2.001210in}}{\pgfqpoint{2.345223in}{1.996820in}}{\pgfqpoint{2.356273in}{1.996820in}}%
\pgfpathclose%
\pgfusepath{stroke,fill}%
\end{pgfscope}%
\begin{pgfscope}%
\pgfpathrectangle{\pgfqpoint{0.481978in}{0.331635in}}{\pgfqpoint{4.960000in}{3.696000in}}%
\pgfusepath{clip}%
\pgfsetbuttcap%
\pgfsetroundjoin%
\definecolor{currentfill}{rgb}{1.000000,0.705882,0.509804}%
\pgfsetfillcolor{currentfill}%
\pgfsetlinewidth{0.481800pt}%
\definecolor{currentstroke}{rgb}{1.000000,1.000000,1.000000}%
\pgfsetstrokecolor{currentstroke}%
\pgfsetdash{}{0pt}%
\pgfpathmoveto{\pgfqpoint{2.481436in}{2.368085in}}%
\pgfpathcurveto{\pgfqpoint{2.492487in}{2.368085in}}{\pgfqpoint{2.503086in}{2.372475in}}{\pgfqpoint{2.510899in}{2.380289in}}%
\pgfpathcurveto{\pgfqpoint{2.518713in}{2.388102in}}{\pgfqpoint{2.523103in}{2.398701in}}{\pgfqpoint{2.523103in}{2.409751in}}%
\pgfpathcurveto{\pgfqpoint{2.523103in}{2.420802in}}{\pgfqpoint{2.518713in}{2.431401in}}{\pgfqpoint{2.510899in}{2.439214in}}%
\pgfpathcurveto{\pgfqpoint{2.503086in}{2.447028in}}{\pgfqpoint{2.492487in}{2.451418in}}{\pgfqpoint{2.481436in}{2.451418in}}%
\pgfpathcurveto{\pgfqpoint{2.470386in}{2.451418in}}{\pgfqpoint{2.459787in}{2.447028in}}{\pgfqpoint{2.451974in}{2.439214in}}%
\pgfpathcurveto{\pgfqpoint{2.444160in}{2.431401in}}{\pgfqpoint{2.439770in}{2.420802in}}{\pgfqpoint{2.439770in}{2.409751in}}%
\pgfpathcurveto{\pgfqpoint{2.439770in}{2.398701in}}{\pgfqpoint{2.444160in}{2.388102in}}{\pgfqpoint{2.451974in}{2.380289in}}%
\pgfpathcurveto{\pgfqpoint{2.459787in}{2.372475in}}{\pgfqpoint{2.470386in}{2.368085in}}{\pgfqpoint{2.481436in}{2.368085in}}%
\pgfpathclose%
\pgfusepath{stroke,fill}%
\end{pgfscope}%
\begin{pgfscope}%
\pgfpathrectangle{\pgfqpoint{0.481978in}{0.331635in}}{\pgfqpoint{4.960000in}{3.696000in}}%
\pgfusepath{clip}%
\pgfsetbuttcap%
\pgfsetroundjoin%
\definecolor{currentfill}{rgb}{1.000000,0.705882,0.509804}%
\pgfsetfillcolor{currentfill}%
\pgfsetlinewidth{0.481800pt}%
\definecolor{currentstroke}{rgb}{1.000000,1.000000,1.000000}%
\pgfsetstrokecolor{currentstroke}%
\pgfsetdash{}{0pt}%
\pgfpathmoveto{\pgfqpoint{1.981089in}{2.637495in}}%
\pgfpathcurveto{\pgfqpoint{1.992139in}{2.637495in}}{\pgfqpoint{2.002738in}{2.641885in}}{\pgfqpoint{2.010552in}{2.649698in}}%
\pgfpathcurveto{\pgfqpoint{2.018365in}{2.657512in}}{\pgfqpoint{2.022755in}{2.668111in}}{\pgfqpoint{2.022755in}{2.679161in}}%
\pgfpathcurveto{\pgfqpoint{2.022755in}{2.690211in}}{\pgfqpoint{2.018365in}{2.700810in}}{\pgfqpoint{2.010552in}{2.708624in}}%
\pgfpathcurveto{\pgfqpoint{2.002738in}{2.716438in}}{\pgfqpoint{1.992139in}{2.720828in}}{\pgfqpoint{1.981089in}{2.720828in}}%
\pgfpathcurveto{\pgfqpoint{1.970039in}{2.720828in}}{\pgfqpoint{1.959440in}{2.716438in}}{\pgfqpoint{1.951626in}{2.708624in}}%
\pgfpathcurveto{\pgfqpoint{1.943812in}{2.700810in}}{\pgfqpoint{1.939422in}{2.690211in}}{\pgfqpoint{1.939422in}{2.679161in}}%
\pgfpathcurveto{\pgfqpoint{1.939422in}{2.668111in}}{\pgfqpoint{1.943812in}{2.657512in}}{\pgfqpoint{1.951626in}{2.649698in}}%
\pgfpathcurveto{\pgfqpoint{1.959440in}{2.641885in}}{\pgfqpoint{1.970039in}{2.637495in}}{\pgfqpoint{1.981089in}{2.637495in}}%
\pgfpathclose%
\pgfusepath{stroke,fill}%
\end{pgfscope}%
\begin{pgfscope}%
\pgfpathrectangle{\pgfqpoint{0.481978in}{0.331635in}}{\pgfqpoint{4.960000in}{3.696000in}}%
\pgfusepath{clip}%
\pgfsetbuttcap%
\pgfsetroundjoin%
\definecolor{currentfill}{rgb}{1.000000,0.705882,0.509804}%
\pgfsetfillcolor{currentfill}%
\pgfsetlinewidth{0.481800pt}%
\definecolor{currentstroke}{rgb}{1.000000,1.000000,1.000000}%
\pgfsetstrokecolor{currentstroke}%
\pgfsetdash{}{0pt}%
\pgfpathmoveto{\pgfqpoint{3.009334in}{2.675152in}}%
\pgfpathcurveto{\pgfqpoint{3.020384in}{2.675152in}}{\pgfqpoint{3.030983in}{2.679542in}}{\pgfqpoint{3.038797in}{2.687356in}}%
\pgfpathcurveto{\pgfqpoint{3.046610in}{2.695169in}}{\pgfqpoint{3.051000in}{2.705768in}}{\pgfqpoint{3.051000in}{2.716818in}}%
\pgfpathcurveto{\pgfqpoint{3.051000in}{2.727868in}}{\pgfqpoint{3.046610in}{2.738467in}}{\pgfqpoint{3.038797in}{2.746281in}}%
\pgfpathcurveto{\pgfqpoint{3.030983in}{2.754095in}}{\pgfqpoint{3.020384in}{2.758485in}}{\pgfqpoint{3.009334in}{2.758485in}}%
\pgfpathcurveto{\pgfqpoint{2.998284in}{2.758485in}}{\pgfqpoint{2.987685in}{2.754095in}}{\pgfqpoint{2.979871in}{2.746281in}}%
\pgfpathcurveto{\pgfqpoint{2.972057in}{2.738467in}}{\pgfqpoint{2.967667in}{2.727868in}}{\pgfqpoint{2.967667in}{2.716818in}}%
\pgfpathcurveto{\pgfqpoint{2.967667in}{2.705768in}}{\pgfqpoint{2.972057in}{2.695169in}}{\pgfqpoint{2.979871in}{2.687356in}}%
\pgfpathcurveto{\pgfqpoint{2.987685in}{2.679542in}}{\pgfqpoint{2.998284in}{2.675152in}}{\pgfqpoint{3.009334in}{2.675152in}}%
\pgfpathclose%
\pgfusepath{stroke,fill}%
\end{pgfscope}%
\begin{pgfscope}%
\pgfpathrectangle{\pgfqpoint{0.481978in}{0.331635in}}{\pgfqpoint{4.960000in}{3.696000in}}%
\pgfusepath{clip}%
\pgfsetbuttcap%
\pgfsetroundjoin%
\definecolor{currentfill}{rgb}{1.000000,0.705882,0.509804}%
\pgfsetfillcolor{currentfill}%
\pgfsetlinewidth{0.481800pt}%
\definecolor{currentstroke}{rgb}{1.000000,1.000000,1.000000}%
\pgfsetstrokecolor{currentstroke}%
\pgfsetdash{}{0pt}%
\pgfpathmoveto{\pgfqpoint{2.838079in}{2.732561in}}%
\pgfpathcurveto{\pgfqpoint{2.849129in}{2.732561in}}{\pgfqpoint{2.859728in}{2.736952in}}{\pgfqpoint{2.867542in}{2.744765in}}%
\pgfpathcurveto{\pgfqpoint{2.875355in}{2.752579in}}{\pgfqpoint{2.879746in}{2.763178in}}{\pgfqpoint{2.879746in}{2.774228in}}%
\pgfpathcurveto{\pgfqpoint{2.879746in}{2.785278in}}{\pgfqpoint{2.875355in}{2.795877in}}{\pgfqpoint{2.867542in}{2.803691in}}%
\pgfpathcurveto{\pgfqpoint{2.859728in}{2.811504in}}{\pgfqpoint{2.849129in}{2.815895in}}{\pgfqpoint{2.838079in}{2.815895in}}%
\pgfpathcurveto{\pgfqpoint{2.827029in}{2.815895in}}{\pgfqpoint{2.816430in}{2.811504in}}{\pgfqpoint{2.808616in}{2.803691in}}%
\pgfpathcurveto{\pgfqpoint{2.800803in}{2.795877in}}{\pgfqpoint{2.796412in}{2.785278in}}{\pgfqpoint{2.796412in}{2.774228in}}%
\pgfpathcurveto{\pgfqpoint{2.796412in}{2.763178in}}{\pgfqpoint{2.800803in}{2.752579in}}{\pgfqpoint{2.808616in}{2.744765in}}%
\pgfpathcurveto{\pgfqpoint{2.816430in}{2.736952in}}{\pgfqpoint{2.827029in}{2.732561in}}{\pgfqpoint{2.838079in}{2.732561in}}%
\pgfpathclose%
\pgfusepath{stroke,fill}%
\end{pgfscope}%
\begin{pgfscope}%
\pgfpathrectangle{\pgfqpoint{0.481978in}{0.331635in}}{\pgfqpoint{4.960000in}{3.696000in}}%
\pgfusepath{clip}%
\pgfsetbuttcap%
\pgfsetroundjoin%
\definecolor{currentfill}{rgb}{1.000000,0.705882,0.509804}%
\pgfsetfillcolor{currentfill}%
\pgfsetlinewidth{0.481800pt}%
\definecolor{currentstroke}{rgb}{1.000000,1.000000,1.000000}%
\pgfsetstrokecolor{currentstroke}%
\pgfsetdash{}{0pt}%
\pgfpathmoveto{\pgfqpoint{1.984051in}{2.205297in}}%
\pgfpathcurveto{\pgfqpoint{1.995101in}{2.205297in}}{\pgfqpoint{2.005700in}{2.209687in}}{\pgfqpoint{2.013514in}{2.217501in}}%
\pgfpathcurveto{\pgfqpoint{2.021328in}{2.225315in}}{\pgfqpoint{2.025718in}{2.235914in}}{\pgfqpoint{2.025718in}{2.246964in}}%
\pgfpathcurveto{\pgfqpoint{2.025718in}{2.258014in}}{\pgfqpoint{2.021328in}{2.268613in}}{\pgfqpoint{2.013514in}{2.276427in}}%
\pgfpathcurveto{\pgfqpoint{2.005700in}{2.284240in}}{\pgfqpoint{1.995101in}{2.288631in}}{\pgfqpoint{1.984051in}{2.288631in}}%
\pgfpathcurveto{\pgfqpoint{1.973001in}{2.288631in}}{\pgfqpoint{1.962402in}{2.284240in}}{\pgfqpoint{1.954588in}{2.276427in}}%
\pgfpathcurveto{\pgfqpoint{1.946775in}{2.268613in}}{\pgfqpoint{1.942385in}{2.258014in}}{\pgfqpoint{1.942385in}{2.246964in}}%
\pgfpathcurveto{\pgfqpoint{1.942385in}{2.235914in}}{\pgfqpoint{1.946775in}{2.225315in}}{\pgfqpoint{1.954588in}{2.217501in}}%
\pgfpathcurveto{\pgfqpoint{1.962402in}{2.209687in}}{\pgfqpoint{1.973001in}{2.205297in}}{\pgfqpoint{1.984051in}{2.205297in}}%
\pgfpathclose%
\pgfusepath{stroke,fill}%
\end{pgfscope}%
\begin{pgfscope}%
\pgfpathrectangle{\pgfqpoint{0.481978in}{0.331635in}}{\pgfqpoint{4.960000in}{3.696000in}}%
\pgfusepath{clip}%
\pgfsetbuttcap%
\pgfsetroundjoin%
\definecolor{currentfill}{rgb}{1.000000,0.705882,0.509804}%
\pgfsetfillcolor{currentfill}%
\pgfsetlinewidth{0.481800pt}%
\definecolor{currentstroke}{rgb}{1.000000,1.000000,1.000000}%
\pgfsetstrokecolor{currentstroke}%
\pgfsetdash{}{0pt}%
\pgfpathmoveto{\pgfqpoint{2.727962in}{2.221618in}}%
\pgfpathcurveto{\pgfqpoint{2.739012in}{2.221618in}}{\pgfqpoint{2.749611in}{2.226008in}}{\pgfqpoint{2.757424in}{2.233822in}}%
\pgfpathcurveto{\pgfqpoint{2.765238in}{2.241635in}}{\pgfqpoint{2.769628in}{2.252234in}}{\pgfqpoint{2.769628in}{2.263285in}}%
\pgfpathcurveto{\pgfqpoint{2.769628in}{2.274335in}}{\pgfqpoint{2.765238in}{2.284934in}}{\pgfqpoint{2.757424in}{2.292747in}}%
\pgfpathcurveto{\pgfqpoint{2.749611in}{2.300561in}}{\pgfqpoint{2.739012in}{2.304951in}}{\pgfqpoint{2.727962in}{2.304951in}}%
\pgfpathcurveto{\pgfqpoint{2.716911in}{2.304951in}}{\pgfqpoint{2.706312in}{2.300561in}}{\pgfqpoint{2.698499in}{2.292747in}}%
\pgfpathcurveto{\pgfqpoint{2.690685in}{2.284934in}}{\pgfqpoint{2.686295in}{2.274335in}}{\pgfqpoint{2.686295in}{2.263285in}}%
\pgfpathcurveto{\pgfqpoint{2.686295in}{2.252234in}}{\pgfqpoint{2.690685in}{2.241635in}}{\pgfqpoint{2.698499in}{2.233822in}}%
\pgfpathcurveto{\pgfqpoint{2.706312in}{2.226008in}}{\pgfqpoint{2.716911in}{2.221618in}}{\pgfqpoint{2.727962in}{2.221618in}}%
\pgfpathclose%
\pgfusepath{stroke,fill}%
\end{pgfscope}%
\begin{pgfscope}%
\pgfpathrectangle{\pgfqpoint{0.481978in}{0.331635in}}{\pgfqpoint{4.960000in}{3.696000in}}%
\pgfusepath{clip}%
\pgfsetbuttcap%
\pgfsetroundjoin%
\definecolor{currentfill}{rgb}{1.000000,0.705882,0.509804}%
\pgfsetfillcolor{currentfill}%
\pgfsetlinewidth{0.481800pt}%
\definecolor{currentstroke}{rgb}{1.000000,1.000000,1.000000}%
\pgfsetstrokecolor{currentstroke}%
\pgfsetdash{}{0pt}%
\pgfpathmoveto{\pgfqpoint{2.390293in}{3.625370in}}%
\pgfpathcurveto{\pgfqpoint{2.401343in}{3.625370in}}{\pgfqpoint{2.411942in}{3.629760in}}{\pgfqpoint{2.419756in}{3.637573in}}%
\pgfpathcurveto{\pgfqpoint{2.427569in}{3.645387in}}{\pgfqpoint{2.431960in}{3.655986in}}{\pgfqpoint{2.431960in}{3.667036in}}%
\pgfpathcurveto{\pgfqpoint{2.431960in}{3.678086in}}{\pgfqpoint{2.427569in}{3.688685in}}{\pgfqpoint{2.419756in}{3.696499in}}%
\pgfpathcurveto{\pgfqpoint{2.411942in}{3.704313in}}{\pgfqpoint{2.401343in}{3.708703in}}{\pgfqpoint{2.390293in}{3.708703in}}%
\pgfpathcurveto{\pgfqpoint{2.379243in}{3.708703in}}{\pgfqpoint{2.368644in}{3.704313in}}{\pgfqpoint{2.360830in}{3.696499in}}%
\pgfpathcurveto{\pgfqpoint{2.353017in}{3.688685in}}{\pgfqpoint{2.348626in}{3.678086in}}{\pgfqpoint{2.348626in}{3.667036in}}%
\pgfpathcurveto{\pgfqpoint{2.348626in}{3.655986in}}{\pgfqpoint{2.353017in}{3.645387in}}{\pgfqpoint{2.360830in}{3.637573in}}%
\pgfpathcurveto{\pgfqpoint{2.368644in}{3.629760in}}{\pgfqpoint{2.379243in}{3.625370in}}{\pgfqpoint{2.390293in}{3.625370in}}%
\pgfpathclose%
\pgfusepath{stroke,fill}%
\end{pgfscope}%
\begin{pgfscope}%
\pgfpathrectangle{\pgfqpoint{0.481978in}{0.331635in}}{\pgfqpoint{4.960000in}{3.696000in}}%
\pgfusepath{clip}%
\pgfsetbuttcap%
\pgfsetroundjoin%
\definecolor{currentfill}{rgb}{1.000000,0.705882,0.509804}%
\pgfsetfillcolor{currentfill}%
\pgfsetlinewidth{0.481800pt}%
\definecolor{currentstroke}{rgb}{1.000000,1.000000,1.000000}%
\pgfsetstrokecolor{currentstroke}%
\pgfsetdash{}{0pt}%
\pgfpathmoveto{\pgfqpoint{1.379840in}{2.608602in}}%
\pgfpathcurveto{\pgfqpoint{1.390890in}{2.608602in}}{\pgfqpoint{1.401489in}{2.612992in}}{\pgfqpoint{1.409302in}{2.620806in}}%
\pgfpathcurveto{\pgfqpoint{1.417116in}{2.628619in}}{\pgfqpoint{1.421506in}{2.639218in}}{\pgfqpoint{1.421506in}{2.650268in}}%
\pgfpathcurveto{\pgfqpoint{1.421506in}{2.661318in}}{\pgfqpoint{1.417116in}{2.671917in}}{\pgfqpoint{1.409302in}{2.679731in}}%
\pgfpathcurveto{\pgfqpoint{1.401489in}{2.687545in}}{\pgfqpoint{1.390890in}{2.691935in}}{\pgfqpoint{1.379840in}{2.691935in}}%
\pgfpathcurveto{\pgfqpoint{1.368789in}{2.691935in}}{\pgfqpoint{1.358190in}{2.687545in}}{\pgfqpoint{1.350377in}{2.679731in}}%
\pgfpathcurveto{\pgfqpoint{1.342563in}{2.671917in}}{\pgfqpoint{1.338173in}{2.661318in}}{\pgfqpoint{1.338173in}{2.650268in}}%
\pgfpathcurveto{\pgfqpoint{1.338173in}{2.639218in}}{\pgfqpoint{1.342563in}{2.628619in}}{\pgfqpoint{1.350377in}{2.620806in}}%
\pgfpathcurveto{\pgfqpoint{1.358190in}{2.612992in}}{\pgfqpoint{1.368789in}{2.608602in}}{\pgfqpoint{1.379840in}{2.608602in}}%
\pgfpathclose%
\pgfusepath{stroke,fill}%
\end{pgfscope}%
\begin{pgfscope}%
\pgfpathrectangle{\pgfqpoint{0.481978in}{0.331635in}}{\pgfqpoint{4.960000in}{3.696000in}}%
\pgfusepath{clip}%
\pgfsetbuttcap%
\pgfsetroundjoin%
\definecolor{currentfill}{rgb}{1.000000,0.705882,0.509804}%
\pgfsetfillcolor{currentfill}%
\pgfsetlinewidth{0.481800pt}%
\definecolor{currentstroke}{rgb}{1.000000,1.000000,1.000000}%
\pgfsetstrokecolor{currentstroke}%
\pgfsetdash{}{0pt}%
\pgfpathmoveto{\pgfqpoint{4.883091in}{2.877656in}}%
\pgfpathcurveto{\pgfqpoint{4.894141in}{2.877656in}}{\pgfqpoint{4.904740in}{2.882046in}}{\pgfqpoint{4.912553in}{2.889860in}}%
\pgfpathcurveto{\pgfqpoint{4.920367in}{2.897674in}}{\pgfqpoint{4.924757in}{2.908273in}}{\pgfqpoint{4.924757in}{2.919323in}}%
\pgfpathcurveto{\pgfqpoint{4.924757in}{2.930373in}}{\pgfqpoint{4.920367in}{2.940972in}}{\pgfqpoint{4.912553in}{2.948786in}}%
\pgfpathcurveto{\pgfqpoint{4.904740in}{2.956599in}}{\pgfqpoint{4.894141in}{2.960989in}}{\pgfqpoint{4.883091in}{2.960989in}}%
\pgfpathcurveto{\pgfqpoint{4.872041in}{2.960989in}}{\pgfqpoint{4.861442in}{2.956599in}}{\pgfqpoint{4.853628in}{2.948786in}}%
\pgfpathcurveto{\pgfqpoint{4.845814in}{2.940972in}}{\pgfqpoint{4.841424in}{2.930373in}}{\pgfqpoint{4.841424in}{2.919323in}}%
\pgfpathcurveto{\pgfqpoint{4.841424in}{2.908273in}}{\pgfqpoint{4.845814in}{2.897674in}}{\pgfqpoint{4.853628in}{2.889860in}}%
\pgfpathcurveto{\pgfqpoint{4.861442in}{2.882046in}}{\pgfqpoint{4.872041in}{2.877656in}}{\pgfqpoint{4.883091in}{2.877656in}}%
\pgfpathclose%
\pgfusepath{stroke,fill}%
\end{pgfscope}%
\begin{pgfscope}%
\pgfpathrectangle{\pgfqpoint{0.481978in}{0.331635in}}{\pgfqpoint{4.960000in}{3.696000in}}%
\pgfusepath{clip}%
\pgfsetbuttcap%
\pgfsetroundjoin%
\definecolor{currentfill}{rgb}{1.000000,0.705882,0.509804}%
\pgfsetfillcolor{currentfill}%
\pgfsetlinewidth{0.481800pt}%
\definecolor{currentstroke}{rgb}{1.000000,1.000000,1.000000}%
\pgfsetstrokecolor{currentstroke}%
\pgfsetdash{}{0pt}%
\pgfpathmoveto{\pgfqpoint{4.909759in}{1.600865in}}%
\pgfpathcurveto{\pgfqpoint{4.920809in}{1.600865in}}{\pgfqpoint{4.931408in}{1.605256in}}{\pgfqpoint{4.939222in}{1.613069in}}%
\pgfpathcurveto{\pgfqpoint{4.947035in}{1.620883in}}{\pgfqpoint{4.951426in}{1.631482in}}{\pgfqpoint{4.951426in}{1.642532in}}%
\pgfpathcurveto{\pgfqpoint{4.951426in}{1.653582in}}{\pgfqpoint{4.947035in}{1.664181in}}{\pgfqpoint{4.939222in}{1.671995in}}%
\pgfpathcurveto{\pgfqpoint{4.931408in}{1.679808in}}{\pgfqpoint{4.920809in}{1.684199in}}{\pgfqpoint{4.909759in}{1.684199in}}%
\pgfpathcurveto{\pgfqpoint{4.898709in}{1.684199in}}{\pgfqpoint{4.888110in}{1.679808in}}{\pgfqpoint{4.880296in}{1.671995in}}%
\pgfpathcurveto{\pgfqpoint{4.872482in}{1.664181in}}{\pgfqpoint{4.868092in}{1.653582in}}{\pgfqpoint{4.868092in}{1.642532in}}%
\pgfpathcurveto{\pgfqpoint{4.868092in}{1.631482in}}{\pgfqpoint{4.872482in}{1.620883in}}{\pgfqpoint{4.880296in}{1.613069in}}%
\pgfpathcurveto{\pgfqpoint{4.888110in}{1.605256in}}{\pgfqpoint{4.898709in}{1.600865in}}{\pgfqpoint{4.909759in}{1.600865in}}%
\pgfpathclose%
\pgfusepath{stroke,fill}%
\end{pgfscope}%
\begin{pgfscope}%
\pgfpathrectangle{\pgfqpoint{0.481978in}{0.331635in}}{\pgfqpoint{4.960000in}{3.696000in}}%
\pgfusepath{clip}%
\pgfsetbuttcap%
\pgfsetroundjoin%
\definecolor{currentfill}{rgb}{1.000000,0.705882,0.509804}%
\pgfsetfillcolor{currentfill}%
\pgfsetlinewidth{0.481800pt}%
\definecolor{currentstroke}{rgb}{1.000000,1.000000,1.000000}%
\pgfsetstrokecolor{currentstroke}%
\pgfsetdash{}{0pt}%
\pgfpathmoveto{\pgfqpoint{1.974653in}{2.425424in}}%
\pgfpathcurveto{\pgfqpoint{1.985703in}{2.425424in}}{\pgfqpoint{1.996302in}{2.429814in}}{\pgfqpoint{2.004116in}{2.437628in}}%
\pgfpathcurveto{\pgfqpoint{2.011930in}{2.445442in}}{\pgfqpoint{2.016320in}{2.456041in}}{\pgfqpoint{2.016320in}{2.467091in}}%
\pgfpathcurveto{\pgfqpoint{2.016320in}{2.478141in}}{\pgfqpoint{2.011930in}{2.488740in}}{\pgfqpoint{2.004116in}{2.496554in}}%
\pgfpathcurveto{\pgfqpoint{1.996302in}{2.504367in}}{\pgfqpoint{1.985703in}{2.508757in}}{\pgfqpoint{1.974653in}{2.508757in}}%
\pgfpathcurveto{\pgfqpoint{1.963603in}{2.508757in}}{\pgfqpoint{1.953004in}{2.504367in}}{\pgfqpoint{1.945190in}{2.496554in}}%
\pgfpathcurveto{\pgfqpoint{1.937377in}{2.488740in}}{\pgfqpoint{1.932987in}{2.478141in}}{\pgfqpoint{1.932987in}{2.467091in}}%
\pgfpathcurveto{\pgfqpoint{1.932987in}{2.456041in}}{\pgfqpoint{1.937377in}{2.445442in}}{\pgfqpoint{1.945190in}{2.437628in}}%
\pgfpathcurveto{\pgfqpoint{1.953004in}{2.429814in}}{\pgfqpoint{1.963603in}{2.425424in}}{\pgfqpoint{1.974653in}{2.425424in}}%
\pgfpathclose%
\pgfusepath{stroke,fill}%
\end{pgfscope}%
\begin{pgfscope}%
\pgfpathrectangle{\pgfqpoint{0.481978in}{0.331635in}}{\pgfqpoint{4.960000in}{3.696000in}}%
\pgfusepath{clip}%
\pgfsetbuttcap%
\pgfsetroundjoin%
\definecolor{currentfill}{rgb}{1.000000,0.705882,0.509804}%
\pgfsetfillcolor{currentfill}%
\pgfsetlinewidth{0.481800pt}%
\definecolor{currentstroke}{rgb}{1.000000,1.000000,1.000000}%
\pgfsetstrokecolor{currentstroke}%
\pgfsetdash{}{0pt}%
\pgfpathmoveto{\pgfqpoint{0.942652in}{2.698212in}}%
\pgfpathcurveto{\pgfqpoint{0.953702in}{2.698212in}}{\pgfqpoint{0.964301in}{2.702602in}}{\pgfqpoint{0.972115in}{2.710416in}}%
\pgfpathcurveto{\pgfqpoint{0.979929in}{2.718230in}}{\pgfqpoint{0.984319in}{2.728829in}}{\pgfqpoint{0.984319in}{2.739879in}}%
\pgfpathcurveto{\pgfqpoint{0.984319in}{2.750929in}}{\pgfqpoint{0.979929in}{2.761528in}}{\pgfqpoint{0.972115in}{2.769341in}}%
\pgfpathcurveto{\pgfqpoint{0.964301in}{2.777155in}}{\pgfqpoint{0.953702in}{2.781545in}}{\pgfqpoint{0.942652in}{2.781545in}}%
\pgfpathcurveto{\pgfqpoint{0.931602in}{2.781545in}}{\pgfqpoint{0.921003in}{2.777155in}}{\pgfqpoint{0.913189in}{2.769341in}}%
\pgfpathcurveto{\pgfqpoint{0.905376in}{2.761528in}}{\pgfqpoint{0.900986in}{2.750929in}}{\pgfqpoint{0.900986in}{2.739879in}}%
\pgfpathcurveto{\pgfqpoint{0.900986in}{2.728829in}}{\pgfqpoint{0.905376in}{2.718230in}}{\pgfqpoint{0.913189in}{2.710416in}}%
\pgfpathcurveto{\pgfqpoint{0.921003in}{2.702602in}}{\pgfqpoint{0.931602in}{2.698212in}}{\pgfqpoint{0.942652in}{2.698212in}}%
\pgfpathclose%
\pgfusepath{stroke,fill}%
\end{pgfscope}%
\begin{pgfscope}%
\pgfpathrectangle{\pgfqpoint{0.481978in}{0.331635in}}{\pgfqpoint{4.960000in}{3.696000in}}%
\pgfusepath{clip}%
\pgfsetbuttcap%
\pgfsetroundjoin%
\definecolor{currentfill}{rgb}{1.000000,0.705882,0.509804}%
\pgfsetfillcolor{currentfill}%
\pgfsetlinewidth{0.481800pt}%
\definecolor{currentstroke}{rgb}{1.000000,1.000000,1.000000}%
\pgfsetstrokecolor{currentstroke}%
\pgfsetdash{}{0pt}%
\pgfpathmoveto{\pgfqpoint{2.801723in}{1.852351in}}%
\pgfpathcurveto{\pgfqpoint{2.812773in}{1.852351in}}{\pgfqpoint{2.823372in}{1.856741in}}{\pgfqpoint{2.831185in}{1.864555in}}%
\pgfpathcurveto{\pgfqpoint{2.838999in}{1.872368in}}{\pgfqpoint{2.843389in}{1.882967in}}{\pgfqpoint{2.843389in}{1.894017in}}%
\pgfpathcurveto{\pgfqpoint{2.843389in}{1.905067in}}{\pgfqpoint{2.838999in}{1.915667in}}{\pgfqpoint{2.831185in}{1.923480in}}%
\pgfpathcurveto{\pgfqpoint{2.823372in}{1.931294in}}{\pgfqpoint{2.812773in}{1.935684in}}{\pgfqpoint{2.801723in}{1.935684in}}%
\pgfpathcurveto{\pgfqpoint{2.790672in}{1.935684in}}{\pgfqpoint{2.780073in}{1.931294in}}{\pgfqpoint{2.772260in}{1.923480in}}%
\pgfpathcurveto{\pgfqpoint{2.764446in}{1.915667in}}{\pgfqpoint{2.760056in}{1.905067in}}{\pgfqpoint{2.760056in}{1.894017in}}%
\pgfpathcurveto{\pgfqpoint{2.760056in}{1.882967in}}{\pgfqpoint{2.764446in}{1.872368in}}{\pgfqpoint{2.772260in}{1.864555in}}%
\pgfpathcurveto{\pgfqpoint{2.780073in}{1.856741in}}{\pgfqpoint{2.790672in}{1.852351in}}{\pgfqpoint{2.801723in}{1.852351in}}%
\pgfpathclose%
\pgfusepath{stroke,fill}%
\end{pgfscope}%
\begin{pgfscope}%
\pgfpathrectangle{\pgfqpoint{0.481978in}{0.331635in}}{\pgfqpoint{4.960000in}{3.696000in}}%
\pgfusepath{clip}%
\pgfsetbuttcap%
\pgfsetroundjoin%
\definecolor{currentfill}{rgb}{1.000000,0.705882,0.509804}%
\pgfsetfillcolor{currentfill}%
\pgfsetlinewidth{0.481800pt}%
\definecolor{currentstroke}{rgb}{1.000000,1.000000,1.000000}%
\pgfsetstrokecolor{currentstroke}%
\pgfsetdash{}{0pt}%
\pgfpathmoveto{\pgfqpoint{3.038069in}{3.519751in}}%
\pgfpathcurveto{\pgfqpoint{3.049119in}{3.519751in}}{\pgfqpoint{3.059718in}{3.524142in}}{\pgfqpoint{3.067532in}{3.531955in}}%
\pgfpathcurveto{\pgfqpoint{3.075346in}{3.539769in}}{\pgfqpoint{3.079736in}{3.550368in}}{\pgfqpoint{3.079736in}{3.561418in}}%
\pgfpathcurveto{\pgfqpoint{3.079736in}{3.572468in}}{\pgfqpoint{3.075346in}{3.583067in}}{\pgfqpoint{3.067532in}{3.590881in}}%
\pgfpathcurveto{\pgfqpoint{3.059718in}{3.598694in}}{\pgfqpoint{3.049119in}{3.603085in}}{\pgfqpoint{3.038069in}{3.603085in}}%
\pgfpathcurveto{\pgfqpoint{3.027019in}{3.603085in}}{\pgfqpoint{3.016420in}{3.598694in}}{\pgfqpoint{3.008606in}{3.590881in}}%
\pgfpathcurveto{\pgfqpoint{3.000793in}{3.583067in}}{\pgfqpoint{2.996403in}{3.572468in}}{\pgfqpoint{2.996403in}{3.561418in}}%
\pgfpathcurveto{\pgfqpoint{2.996403in}{3.550368in}}{\pgfqpoint{3.000793in}{3.539769in}}{\pgfqpoint{3.008606in}{3.531955in}}%
\pgfpathcurveto{\pgfqpoint{3.016420in}{3.524142in}}{\pgfqpoint{3.027019in}{3.519751in}}{\pgfqpoint{3.038069in}{3.519751in}}%
\pgfpathclose%
\pgfusepath{stroke,fill}%
\end{pgfscope}%
\begin{pgfscope}%
\pgfpathrectangle{\pgfqpoint{0.481978in}{0.331635in}}{\pgfqpoint{4.960000in}{3.696000in}}%
\pgfusepath{clip}%
\pgfsetbuttcap%
\pgfsetroundjoin%
\definecolor{currentfill}{rgb}{1.000000,0.705882,0.509804}%
\pgfsetfillcolor{currentfill}%
\pgfsetlinewidth{0.481800pt}%
\definecolor{currentstroke}{rgb}{1.000000,1.000000,1.000000}%
\pgfsetstrokecolor{currentstroke}%
\pgfsetdash{}{0pt}%
\pgfpathmoveto{\pgfqpoint{2.772742in}{2.703442in}}%
\pgfpathcurveto{\pgfqpoint{2.783792in}{2.703442in}}{\pgfqpoint{2.794391in}{2.707832in}}{\pgfqpoint{2.802205in}{2.715646in}}%
\pgfpathcurveto{\pgfqpoint{2.810019in}{2.723459in}}{\pgfqpoint{2.814409in}{2.734058in}}{\pgfqpoint{2.814409in}{2.745109in}}%
\pgfpathcurveto{\pgfqpoint{2.814409in}{2.756159in}}{\pgfqpoint{2.810019in}{2.766758in}}{\pgfqpoint{2.802205in}{2.774571in}}%
\pgfpathcurveto{\pgfqpoint{2.794391in}{2.782385in}}{\pgfqpoint{2.783792in}{2.786775in}}{\pgfqpoint{2.772742in}{2.786775in}}%
\pgfpathcurveto{\pgfqpoint{2.761692in}{2.786775in}}{\pgfqpoint{2.751093in}{2.782385in}}{\pgfqpoint{2.743279in}{2.774571in}}%
\pgfpathcurveto{\pgfqpoint{2.735466in}{2.766758in}}{\pgfqpoint{2.731076in}{2.756159in}}{\pgfqpoint{2.731076in}{2.745109in}}%
\pgfpathcurveto{\pgfqpoint{2.731076in}{2.734058in}}{\pgfqpoint{2.735466in}{2.723459in}}{\pgfqpoint{2.743279in}{2.715646in}}%
\pgfpathcurveto{\pgfqpoint{2.751093in}{2.707832in}}{\pgfqpoint{2.761692in}{2.703442in}}{\pgfqpoint{2.772742in}{2.703442in}}%
\pgfpathclose%
\pgfusepath{stroke,fill}%
\end{pgfscope}%
\begin{pgfscope}%
\pgfpathrectangle{\pgfqpoint{0.481978in}{0.331635in}}{\pgfqpoint{4.960000in}{3.696000in}}%
\pgfusepath{clip}%
\pgfsetbuttcap%
\pgfsetroundjoin%
\definecolor{currentfill}{rgb}{1.000000,0.705882,0.509804}%
\pgfsetfillcolor{currentfill}%
\pgfsetlinewidth{0.481800pt}%
\definecolor{currentstroke}{rgb}{1.000000,1.000000,1.000000}%
\pgfsetstrokecolor{currentstroke}%
\pgfsetdash{}{0pt}%
\pgfpathmoveto{\pgfqpoint{4.487558in}{1.831470in}}%
\pgfpathcurveto{\pgfqpoint{4.498608in}{1.831470in}}{\pgfqpoint{4.509207in}{1.835860in}}{\pgfqpoint{4.517020in}{1.843673in}}%
\pgfpathcurveto{\pgfqpoint{4.524834in}{1.851487in}}{\pgfqpoint{4.529224in}{1.862086in}}{\pgfqpoint{4.529224in}{1.873136in}}%
\pgfpathcurveto{\pgfqpoint{4.529224in}{1.884186in}}{\pgfqpoint{4.524834in}{1.894785in}}{\pgfqpoint{4.517020in}{1.902599in}}%
\pgfpathcurveto{\pgfqpoint{4.509207in}{1.910413in}}{\pgfqpoint{4.498608in}{1.914803in}}{\pgfqpoint{4.487558in}{1.914803in}}%
\pgfpathcurveto{\pgfqpoint{4.476508in}{1.914803in}}{\pgfqpoint{4.465908in}{1.910413in}}{\pgfqpoint{4.458095in}{1.902599in}}%
\pgfpathcurveto{\pgfqpoint{4.450281in}{1.894785in}}{\pgfqpoint{4.445891in}{1.884186in}}{\pgfqpoint{4.445891in}{1.873136in}}%
\pgfpathcurveto{\pgfqpoint{4.445891in}{1.862086in}}{\pgfqpoint{4.450281in}{1.851487in}}{\pgfqpoint{4.458095in}{1.843673in}}%
\pgfpathcurveto{\pgfqpoint{4.465908in}{1.835860in}}{\pgfqpoint{4.476508in}{1.831470in}}{\pgfqpoint{4.487558in}{1.831470in}}%
\pgfpathclose%
\pgfusepath{stroke,fill}%
\end{pgfscope}%
\begin{pgfscope}%
\pgfpathrectangle{\pgfqpoint{0.481978in}{0.331635in}}{\pgfqpoint{4.960000in}{3.696000in}}%
\pgfusepath{clip}%
\pgfsetbuttcap%
\pgfsetroundjoin%
\definecolor{currentfill}{rgb}{1.000000,0.705882,0.509804}%
\pgfsetfillcolor{currentfill}%
\pgfsetlinewidth{0.481800pt}%
\definecolor{currentstroke}{rgb}{1.000000,1.000000,1.000000}%
\pgfsetstrokecolor{currentstroke}%
\pgfsetdash{}{0pt}%
\pgfpathmoveto{\pgfqpoint{2.930081in}{2.313420in}}%
\pgfpathcurveto{\pgfqpoint{2.941131in}{2.313420in}}{\pgfqpoint{2.951730in}{2.317810in}}{\pgfqpoint{2.959544in}{2.325624in}}%
\pgfpathcurveto{\pgfqpoint{2.967357in}{2.333437in}}{\pgfqpoint{2.971748in}{2.344036in}}{\pgfqpoint{2.971748in}{2.355086in}}%
\pgfpathcurveto{\pgfqpoint{2.971748in}{2.366137in}}{\pgfqpoint{2.967357in}{2.376736in}}{\pgfqpoint{2.959544in}{2.384549in}}%
\pgfpathcurveto{\pgfqpoint{2.951730in}{2.392363in}}{\pgfqpoint{2.941131in}{2.396753in}}{\pgfqpoint{2.930081in}{2.396753in}}%
\pgfpathcurveto{\pgfqpoint{2.919031in}{2.396753in}}{\pgfqpoint{2.908432in}{2.392363in}}{\pgfqpoint{2.900618in}{2.384549in}}%
\pgfpathcurveto{\pgfqpoint{2.892805in}{2.376736in}}{\pgfqpoint{2.888414in}{2.366137in}}{\pgfqpoint{2.888414in}{2.355086in}}%
\pgfpathcurveto{\pgfqpoint{2.888414in}{2.344036in}}{\pgfqpoint{2.892805in}{2.333437in}}{\pgfqpoint{2.900618in}{2.325624in}}%
\pgfpathcurveto{\pgfqpoint{2.908432in}{2.317810in}}{\pgfqpoint{2.919031in}{2.313420in}}{\pgfqpoint{2.930081in}{2.313420in}}%
\pgfpathclose%
\pgfusepath{stroke,fill}%
\end{pgfscope}%
\begin{pgfscope}%
\pgfpathrectangle{\pgfqpoint{0.481978in}{0.331635in}}{\pgfqpoint{4.960000in}{3.696000in}}%
\pgfusepath{clip}%
\pgfsetbuttcap%
\pgfsetroundjoin%
\definecolor{currentfill}{rgb}{1.000000,0.705882,0.509804}%
\pgfsetfillcolor{currentfill}%
\pgfsetlinewidth{0.481800pt}%
\definecolor{currentstroke}{rgb}{1.000000,1.000000,1.000000}%
\pgfsetstrokecolor{currentstroke}%
\pgfsetdash{}{0pt}%
\pgfpathmoveto{\pgfqpoint{2.178558in}{1.943600in}}%
\pgfpathcurveto{\pgfqpoint{2.189608in}{1.943600in}}{\pgfqpoint{2.200207in}{1.947990in}}{\pgfqpoint{2.208021in}{1.955804in}}%
\pgfpathcurveto{\pgfqpoint{2.215835in}{1.963617in}}{\pgfqpoint{2.220225in}{1.974216in}}{\pgfqpoint{2.220225in}{1.985266in}}%
\pgfpathcurveto{\pgfqpoint{2.220225in}{1.996317in}}{\pgfqpoint{2.215835in}{2.006916in}}{\pgfqpoint{2.208021in}{2.014729in}}%
\pgfpathcurveto{\pgfqpoint{2.200207in}{2.022543in}}{\pgfqpoint{2.189608in}{2.026933in}}{\pgfqpoint{2.178558in}{2.026933in}}%
\pgfpathcurveto{\pgfqpoint{2.167508in}{2.026933in}}{\pgfqpoint{2.156909in}{2.022543in}}{\pgfqpoint{2.149095in}{2.014729in}}%
\pgfpathcurveto{\pgfqpoint{2.141282in}{2.006916in}}{\pgfqpoint{2.136892in}{1.996317in}}{\pgfqpoint{2.136892in}{1.985266in}}%
\pgfpathcurveto{\pgfqpoint{2.136892in}{1.974216in}}{\pgfqpoint{2.141282in}{1.963617in}}{\pgfqpoint{2.149095in}{1.955804in}}%
\pgfpathcurveto{\pgfqpoint{2.156909in}{1.947990in}}{\pgfqpoint{2.167508in}{1.943600in}}{\pgfqpoint{2.178558in}{1.943600in}}%
\pgfpathclose%
\pgfusepath{stroke,fill}%
\end{pgfscope}%
\begin{pgfscope}%
\pgfpathrectangle{\pgfqpoint{0.481978in}{0.331635in}}{\pgfqpoint{4.960000in}{3.696000in}}%
\pgfusepath{clip}%
\pgfsetbuttcap%
\pgfsetroundjoin%
\definecolor{currentfill}{rgb}{1.000000,0.705882,0.509804}%
\pgfsetfillcolor{currentfill}%
\pgfsetlinewidth{0.481800pt}%
\definecolor{currentstroke}{rgb}{1.000000,1.000000,1.000000}%
\pgfsetstrokecolor{currentstroke}%
\pgfsetdash{}{0pt}%
\pgfpathmoveto{\pgfqpoint{4.180528in}{2.531945in}}%
\pgfpathcurveto{\pgfqpoint{4.191578in}{2.531945in}}{\pgfqpoint{4.202177in}{2.536335in}}{\pgfqpoint{4.209991in}{2.544149in}}%
\pgfpathcurveto{\pgfqpoint{4.217804in}{2.551963in}}{\pgfqpoint{4.222195in}{2.562562in}}{\pgfqpoint{4.222195in}{2.573612in}}%
\pgfpathcurveto{\pgfqpoint{4.222195in}{2.584662in}}{\pgfqpoint{4.217804in}{2.595261in}}{\pgfqpoint{4.209991in}{2.603075in}}%
\pgfpathcurveto{\pgfqpoint{4.202177in}{2.610888in}}{\pgfqpoint{4.191578in}{2.615279in}}{\pgfqpoint{4.180528in}{2.615279in}}%
\pgfpathcurveto{\pgfqpoint{4.169478in}{2.615279in}}{\pgfqpoint{4.158879in}{2.610888in}}{\pgfqpoint{4.151065in}{2.603075in}}%
\pgfpathcurveto{\pgfqpoint{4.143252in}{2.595261in}}{\pgfqpoint{4.138861in}{2.584662in}}{\pgfqpoint{4.138861in}{2.573612in}}%
\pgfpathcurveto{\pgfqpoint{4.138861in}{2.562562in}}{\pgfqpoint{4.143252in}{2.551963in}}{\pgfqpoint{4.151065in}{2.544149in}}%
\pgfpathcurveto{\pgfqpoint{4.158879in}{2.536335in}}{\pgfqpoint{4.169478in}{2.531945in}}{\pgfqpoint{4.180528in}{2.531945in}}%
\pgfpathclose%
\pgfusepath{stroke,fill}%
\end{pgfscope}%
\begin{pgfscope}%
\pgfpathrectangle{\pgfqpoint{0.481978in}{0.331635in}}{\pgfqpoint{4.960000in}{3.696000in}}%
\pgfusepath{clip}%
\pgfsetbuttcap%
\pgfsetroundjoin%
\definecolor{currentfill}{rgb}{1.000000,0.705882,0.509804}%
\pgfsetfillcolor{currentfill}%
\pgfsetlinewidth{0.481800pt}%
\definecolor{currentstroke}{rgb}{1.000000,1.000000,1.000000}%
\pgfsetstrokecolor{currentstroke}%
\pgfsetdash{}{0pt}%
\pgfpathmoveto{\pgfqpoint{1.477002in}{3.218647in}}%
\pgfpathcurveto{\pgfqpoint{1.488052in}{3.218647in}}{\pgfqpoint{1.498651in}{3.223037in}}{\pgfqpoint{1.506464in}{3.230851in}}%
\pgfpathcurveto{\pgfqpoint{1.514278in}{3.238665in}}{\pgfqpoint{1.518668in}{3.249264in}}{\pgfqpoint{1.518668in}{3.260314in}}%
\pgfpathcurveto{\pgfqpoint{1.518668in}{3.271364in}}{\pgfqpoint{1.514278in}{3.281963in}}{\pgfqpoint{1.506464in}{3.289777in}}%
\pgfpathcurveto{\pgfqpoint{1.498651in}{3.297590in}}{\pgfqpoint{1.488052in}{3.301981in}}{\pgfqpoint{1.477002in}{3.301981in}}%
\pgfpathcurveto{\pgfqpoint{1.465952in}{3.301981in}}{\pgfqpoint{1.455352in}{3.297590in}}{\pgfqpoint{1.447539in}{3.289777in}}%
\pgfpathcurveto{\pgfqpoint{1.439725in}{3.281963in}}{\pgfqpoint{1.435335in}{3.271364in}}{\pgfqpoint{1.435335in}{3.260314in}}%
\pgfpathcurveto{\pgfqpoint{1.435335in}{3.249264in}}{\pgfqpoint{1.439725in}{3.238665in}}{\pgfqpoint{1.447539in}{3.230851in}}%
\pgfpathcurveto{\pgfqpoint{1.455352in}{3.223037in}}{\pgfqpoint{1.465952in}{3.218647in}}{\pgfqpoint{1.477002in}{3.218647in}}%
\pgfpathclose%
\pgfusepath{stroke,fill}%
\end{pgfscope}%
\begin{pgfscope}%
\pgfpathrectangle{\pgfqpoint{0.481978in}{0.331635in}}{\pgfqpoint{4.960000in}{3.696000in}}%
\pgfusepath{clip}%
\pgfsetbuttcap%
\pgfsetroundjoin%
\definecolor{currentfill}{rgb}{1.000000,0.705882,0.509804}%
\pgfsetfillcolor{currentfill}%
\pgfsetlinewidth{0.481800pt}%
\definecolor{currentstroke}{rgb}{1.000000,1.000000,1.000000}%
\pgfsetstrokecolor{currentstroke}%
\pgfsetdash{}{0pt}%
\pgfpathmoveto{\pgfqpoint{2.910482in}{1.986001in}}%
\pgfpathcurveto{\pgfqpoint{2.921532in}{1.986001in}}{\pgfqpoint{2.932131in}{1.990391in}}{\pgfqpoint{2.939945in}{1.998205in}}%
\pgfpathcurveto{\pgfqpoint{2.947759in}{2.006019in}}{\pgfqpoint{2.952149in}{2.016618in}}{\pgfqpoint{2.952149in}{2.027668in}}%
\pgfpathcurveto{\pgfqpoint{2.952149in}{2.038718in}}{\pgfqpoint{2.947759in}{2.049317in}}{\pgfqpoint{2.939945in}{2.057131in}}%
\pgfpathcurveto{\pgfqpoint{2.932131in}{2.064944in}}{\pgfqpoint{2.921532in}{2.069334in}}{\pgfqpoint{2.910482in}{2.069334in}}%
\pgfpathcurveto{\pgfqpoint{2.899432in}{2.069334in}}{\pgfqpoint{2.888833in}{2.064944in}}{\pgfqpoint{2.881019in}{2.057131in}}%
\pgfpathcurveto{\pgfqpoint{2.873206in}{2.049317in}}{\pgfqpoint{2.868815in}{2.038718in}}{\pgfqpoint{2.868815in}{2.027668in}}%
\pgfpathcurveto{\pgfqpoint{2.868815in}{2.016618in}}{\pgfqpoint{2.873206in}{2.006019in}}{\pgfqpoint{2.881019in}{1.998205in}}%
\pgfpathcurveto{\pgfqpoint{2.888833in}{1.990391in}}{\pgfqpoint{2.899432in}{1.986001in}}{\pgfqpoint{2.910482in}{1.986001in}}%
\pgfpathclose%
\pgfusepath{stroke,fill}%
\end{pgfscope}%
\begin{pgfscope}%
\pgfpathrectangle{\pgfqpoint{0.481978in}{0.331635in}}{\pgfqpoint{4.960000in}{3.696000in}}%
\pgfusepath{clip}%
\pgfsetbuttcap%
\pgfsetroundjoin%
\definecolor{currentfill}{rgb}{1.000000,0.705882,0.509804}%
\pgfsetfillcolor{currentfill}%
\pgfsetlinewidth{0.481800pt}%
\definecolor{currentstroke}{rgb}{1.000000,1.000000,1.000000}%
\pgfsetstrokecolor{currentstroke}%
\pgfsetdash{}{0pt}%
\pgfpathmoveto{\pgfqpoint{2.467737in}{3.029811in}}%
\pgfpathcurveto{\pgfqpoint{2.478787in}{3.029811in}}{\pgfqpoint{2.489387in}{3.034201in}}{\pgfqpoint{2.497200in}{3.042015in}}%
\pgfpathcurveto{\pgfqpoint{2.505014in}{3.049828in}}{\pgfqpoint{2.509404in}{3.060427in}}{\pgfqpoint{2.509404in}{3.071477in}}%
\pgfpathcurveto{\pgfqpoint{2.509404in}{3.082528in}}{\pgfqpoint{2.505014in}{3.093127in}}{\pgfqpoint{2.497200in}{3.100940in}}%
\pgfpathcurveto{\pgfqpoint{2.489387in}{3.108754in}}{\pgfqpoint{2.478787in}{3.113144in}}{\pgfqpoint{2.467737in}{3.113144in}}%
\pgfpathcurveto{\pgfqpoint{2.456687in}{3.113144in}}{\pgfqpoint{2.446088in}{3.108754in}}{\pgfqpoint{2.438275in}{3.100940in}}%
\pgfpathcurveto{\pgfqpoint{2.430461in}{3.093127in}}{\pgfqpoint{2.426071in}{3.082528in}}{\pgfqpoint{2.426071in}{3.071477in}}%
\pgfpathcurveto{\pgfqpoint{2.426071in}{3.060427in}}{\pgfqpoint{2.430461in}{3.049828in}}{\pgfqpoint{2.438275in}{3.042015in}}%
\pgfpathcurveto{\pgfqpoint{2.446088in}{3.034201in}}{\pgfqpoint{2.456687in}{3.029811in}}{\pgfqpoint{2.467737in}{3.029811in}}%
\pgfpathclose%
\pgfusepath{stroke,fill}%
\end{pgfscope}%
\begin{pgfscope}%
\pgfpathrectangle{\pgfqpoint{0.481978in}{0.331635in}}{\pgfqpoint{4.960000in}{3.696000in}}%
\pgfusepath{clip}%
\pgfsetbuttcap%
\pgfsetroundjoin%
\definecolor{currentfill}{rgb}{1.000000,0.705882,0.509804}%
\pgfsetfillcolor{currentfill}%
\pgfsetlinewidth{0.481800pt}%
\definecolor{currentstroke}{rgb}{1.000000,1.000000,1.000000}%
\pgfsetstrokecolor{currentstroke}%
\pgfsetdash{}{0pt}%
\pgfpathmoveto{\pgfqpoint{1.626172in}{1.871654in}}%
\pgfpathcurveto{\pgfqpoint{1.637222in}{1.871654in}}{\pgfqpoint{1.647821in}{1.876044in}}{\pgfqpoint{1.655635in}{1.883858in}}%
\pgfpathcurveto{\pgfqpoint{1.663448in}{1.891671in}}{\pgfqpoint{1.667839in}{1.902270in}}{\pgfqpoint{1.667839in}{1.913320in}}%
\pgfpathcurveto{\pgfqpoint{1.667839in}{1.924370in}}{\pgfqpoint{1.663448in}{1.934969in}}{\pgfqpoint{1.655635in}{1.942783in}}%
\pgfpathcurveto{\pgfqpoint{1.647821in}{1.950597in}}{\pgfqpoint{1.637222in}{1.954987in}}{\pgfqpoint{1.626172in}{1.954987in}}%
\pgfpathcurveto{\pgfqpoint{1.615122in}{1.954987in}}{\pgfqpoint{1.604523in}{1.950597in}}{\pgfqpoint{1.596709in}{1.942783in}}%
\pgfpathcurveto{\pgfqpoint{1.588896in}{1.934969in}}{\pgfqpoint{1.584505in}{1.924370in}}{\pgfqpoint{1.584505in}{1.913320in}}%
\pgfpathcurveto{\pgfqpoint{1.584505in}{1.902270in}}{\pgfqpoint{1.588896in}{1.891671in}}{\pgfqpoint{1.596709in}{1.883858in}}%
\pgfpathcurveto{\pgfqpoint{1.604523in}{1.876044in}}{\pgfqpoint{1.615122in}{1.871654in}}{\pgfqpoint{1.626172in}{1.871654in}}%
\pgfpathclose%
\pgfusepath{stroke,fill}%
\end{pgfscope}%
\begin{pgfscope}%
\pgfpathrectangle{\pgfqpoint{0.481978in}{0.331635in}}{\pgfqpoint{4.960000in}{3.696000in}}%
\pgfusepath{clip}%
\pgfsetbuttcap%
\pgfsetroundjoin%
\definecolor{currentfill}{rgb}{1.000000,0.705882,0.509804}%
\pgfsetfillcolor{currentfill}%
\pgfsetlinewidth{0.481800pt}%
\definecolor{currentstroke}{rgb}{1.000000,1.000000,1.000000}%
\pgfsetstrokecolor{currentstroke}%
\pgfsetdash{}{0pt}%
\pgfpathmoveto{\pgfqpoint{3.095619in}{2.423142in}}%
\pgfpathcurveto{\pgfqpoint{3.106669in}{2.423142in}}{\pgfqpoint{3.117268in}{2.427532in}}{\pgfqpoint{3.125081in}{2.435346in}}%
\pgfpathcurveto{\pgfqpoint{3.132895in}{2.443159in}}{\pgfqpoint{3.137285in}{2.453759in}}{\pgfqpoint{3.137285in}{2.464809in}}%
\pgfpathcurveto{\pgfqpoint{3.137285in}{2.475859in}}{\pgfqpoint{3.132895in}{2.486458in}}{\pgfqpoint{3.125081in}{2.494271in}}%
\pgfpathcurveto{\pgfqpoint{3.117268in}{2.502085in}}{\pgfqpoint{3.106669in}{2.506475in}}{\pgfqpoint{3.095619in}{2.506475in}}%
\pgfpathcurveto{\pgfqpoint{3.084568in}{2.506475in}}{\pgfqpoint{3.073969in}{2.502085in}}{\pgfqpoint{3.066156in}{2.494271in}}%
\pgfpathcurveto{\pgfqpoint{3.058342in}{2.486458in}}{\pgfqpoint{3.053952in}{2.475859in}}{\pgfqpoint{3.053952in}{2.464809in}}%
\pgfpathcurveto{\pgfqpoint{3.053952in}{2.453759in}}{\pgfqpoint{3.058342in}{2.443159in}}{\pgfqpoint{3.066156in}{2.435346in}}%
\pgfpathcurveto{\pgfqpoint{3.073969in}{2.427532in}}{\pgfqpoint{3.084568in}{2.423142in}}{\pgfqpoint{3.095619in}{2.423142in}}%
\pgfpathclose%
\pgfusepath{stroke,fill}%
\end{pgfscope}%
\begin{pgfscope}%
\pgfpathrectangle{\pgfqpoint{0.481978in}{0.331635in}}{\pgfqpoint{4.960000in}{3.696000in}}%
\pgfusepath{clip}%
\pgfsetbuttcap%
\pgfsetroundjoin%
\definecolor{currentfill}{rgb}{1.000000,0.705882,0.509804}%
\pgfsetfillcolor{currentfill}%
\pgfsetlinewidth{0.481800pt}%
\definecolor{currentstroke}{rgb}{1.000000,1.000000,1.000000}%
\pgfsetstrokecolor{currentstroke}%
\pgfsetdash{}{0pt}%
\pgfpathmoveto{\pgfqpoint{2.196010in}{2.486005in}}%
\pgfpathcurveto{\pgfqpoint{2.207060in}{2.486005in}}{\pgfqpoint{2.217659in}{2.490396in}}{\pgfqpoint{2.225473in}{2.498209in}}%
\pgfpathcurveto{\pgfqpoint{2.233286in}{2.506023in}}{\pgfqpoint{2.237677in}{2.516622in}}{\pgfqpoint{2.237677in}{2.527672in}}%
\pgfpathcurveto{\pgfqpoint{2.237677in}{2.538722in}}{\pgfqpoint{2.233286in}{2.549321in}}{\pgfqpoint{2.225473in}{2.557135in}}%
\pgfpathcurveto{\pgfqpoint{2.217659in}{2.564948in}}{\pgfqpoint{2.207060in}{2.569339in}}{\pgfqpoint{2.196010in}{2.569339in}}%
\pgfpathcurveto{\pgfqpoint{2.184960in}{2.569339in}}{\pgfqpoint{2.174361in}{2.564948in}}{\pgfqpoint{2.166547in}{2.557135in}}%
\pgfpathcurveto{\pgfqpoint{2.158733in}{2.549321in}}{\pgfqpoint{2.154343in}{2.538722in}}{\pgfqpoint{2.154343in}{2.527672in}}%
\pgfpathcurveto{\pgfqpoint{2.154343in}{2.516622in}}{\pgfqpoint{2.158733in}{2.506023in}}{\pgfqpoint{2.166547in}{2.498209in}}%
\pgfpathcurveto{\pgfqpoint{2.174361in}{2.490396in}}{\pgfqpoint{2.184960in}{2.486005in}}{\pgfqpoint{2.196010in}{2.486005in}}%
\pgfpathclose%
\pgfusepath{stroke,fill}%
\end{pgfscope}%
\begin{pgfscope}%
\pgfpathrectangle{\pgfqpoint{0.481978in}{0.331635in}}{\pgfqpoint{4.960000in}{3.696000in}}%
\pgfusepath{clip}%
\pgfsetbuttcap%
\pgfsetroundjoin%
\definecolor{currentfill}{rgb}{1.000000,0.705882,0.509804}%
\pgfsetfillcolor{currentfill}%
\pgfsetlinewidth{0.481800pt}%
\definecolor{currentstroke}{rgb}{1.000000,1.000000,1.000000}%
\pgfsetstrokecolor{currentstroke}%
\pgfsetdash{}{0pt}%
\pgfpathmoveto{\pgfqpoint{2.852264in}{2.599407in}}%
\pgfpathcurveto{\pgfqpoint{2.863314in}{2.599407in}}{\pgfqpoint{2.873913in}{2.603797in}}{\pgfqpoint{2.881726in}{2.611611in}}%
\pgfpathcurveto{\pgfqpoint{2.889540in}{2.619424in}}{\pgfqpoint{2.893930in}{2.630023in}}{\pgfqpoint{2.893930in}{2.641074in}}%
\pgfpathcurveto{\pgfqpoint{2.893930in}{2.652124in}}{\pgfqpoint{2.889540in}{2.662723in}}{\pgfqpoint{2.881726in}{2.670536in}}%
\pgfpathcurveto{\pgfqpoint{2.873913in}{2.678350in}}{\pgfqpoint{2.863314in}{2.682740in}}{\pgfqpoint{2.852264in}{2.682740in}}%
\pgfpathcurveto{\pgfqpoint{2.841214in}{2.682740in}}{\pgfqpoint{2.830615in}{2.678350in}}{\pgfqpoint{2.822801in}{2.670536in}}%
\pgfpathcurveto{\pgfqpoint{2.814987in}{2.662723in}}{\pgfqpoint{2.810597in}{2.652124in}}{\pgfqpoint{2.810597in}{2.641074in}}%
\pgfpathcurveto{\pgfqpoint{2.810597in}{2.630023in}}{\pgfqpoint{2.814987in}{2.619424in}}{\pgfqpoint{2.822801in}{2.611611in}}%
\pgfpathcurveto{\pgfqpoint{2.830615in}{2.603797in}}{\pgfqpoint{2.841214in}{2.599407in}}{\pgfqpoint{2.852264in}{2.599407in}}%
\pgfpathclose%
\pgfusepath{stroke,fill}%
\end{pgfscope}%
\begin{pgfscope}%
\pgfpathrectangle{\pgfqpoint{0.481978in}{0.331635in}}{\pgfqpoint{4.960000in}{3.696000in}}%
\pgfusepath{clip}%
\pgfsetbuttcap%
\pgfsetroundjoin%
\definecolor{currentfill}{rgb}{1.000000,0.705882,0.509804}%
\pgfsetfillcolor{currentfill}%
\pgfsetlinewidth{0.481800pt}%
\definecolor{currentstroke}{rgb}{1.000000,1.000000,1.000000}%
\pgfsetstrokecolor{currentstroke}%
\pgfsetdash{}{0pt}%
\pgfpathmoveto{\pgfqpoint{1.848058in}{1.961414in}}%
\pgfpathcurveto{\pgfqpoint{1.859108in}{1.961414in}}{\pgfqpoint{1.869707in}{1.965804in}}{\pgfqpoint{1.877520in}{1.973618in}}%
\pgfpathcurveto{\pgfqpoint{1.885334in}{1.981431in}}{\pgfqpoint{1.889724in}{1.992030in}}{\pgfqpoint{1.889724in}{2.003081in}}%
\pgfpathcurveto{\pgfqpoint{1.889724in}{2.014131in}}{\pgfqpoint{1.885334in}{2.024730in}}{\pgfqpoint{1.877520in}{2.032543in}}%
\pgfpathcurveto{\pgfqpoint{1.869707in}{2.040357in}}{\pgfqpoint{1.859108in}{2.044747in}}{\pgfqpoint{1.848058in}{2.044747in}}%
\pgfpathcurveto{\pgfqpoint{1.837007in}{2.044747in}}{\pgfqpoint{1.826408in}{2.040357in}}{\pgfqpoint{1.818595in}{2.032543in}}%
\pgfpathcurveto{\pgfqpoint{1.810781in}{2.024730in}}{\pgfqpoint{1.806391in}{2.014131in}}{\pgfqpoint{1.806391in}{2.003081in}}%
\pgfpathcurveto{\pgfqpoint{1.806391in}{1.992030in}}{\pgfqpoint{1.810781in}{1.981431in}}{\pgfqpoint{1.818595in}{1.973618in}}%
\pgfpathcurveto{\pgfqpoint{1.826408in}{1.965804in}}{\pgfqpoint{1.837007in}{1.961414in}}{\pgfqpoint{1.848058in}{1.961414in}}%
\pgfpathclose%
\pgfusepath{stroke,fill}%
\end{pgfscope}%
\begin{pgfscope}%
\pgfpathrectangle{\pgfqpoint{0.481978in}{0.331635in}}{\pgfqpoint{4.960000in}{3.696000in}}%
\pgfusepath{clip}%
\pgfsetbuttcap%
\pgfsetroundjoin%
\definecolor{currentfill}{rgb}{1.000000,0.705882,0.509804}%
\pgfsetfillcolor{currentfill}%
\pgfsetlinewidth{0.481800pt}%
\definecolor{currentstroke}{rgb}{1.000000,1.000000,1.000000}%
\pgfsetstrokecolor{currentstroke}%
\pgfsetdash{}{0pt}%
\pgfpathmoveto{\pgfqpoint{2.153651in}{1.961360in}}%
\pgfpathcurveto{\pgfqpoint{2.164701in}{1.961360in}}{\pgfqpoint{2.175300in}{1.965750in}}{\pgfqpoint{2.183114in}{1.973564in}}%
\pgfpathcurveto{\pgfqpoint{2.190928in}{1.981378in}}{\pgfqpoint{2.195318in}{1.991977in}}{\pgfqpoint{2.195318in}{2.003027in}}%
\pgfpathcurveto{\pgfqpoint{2.195318in}{2.014077in}}{\pgfqpoint{2.190928in}{2.024676in}}{\pgfqpoint{2.183114in}{2.032490in}}%
\pgfpathcurveto{\pgfqpoint{2.175300in}{2.040303in}}{\pgfqpoint{2.164701in}{2.044693in}}{\pgfqpoint{2.153651in}{2.044693in}}%
\pgfpathcurveto{\pgfqpoint{2.142601in}{2.044693in}}{\pgfqpoint{2.132002in}{2.040303in}}{\pgfqpoint{2.124188in}{2.032490in}}%
\pgfpathcurveto{\pgfqpoint{2.116375in}{2.024676in}}{\pgfqpoint{2.111985in}{2.014077in}}{\pgfqpoint{2.111985in}{2.003027in}}%
\pgfpathcurveto{\pgfqpoint{2.111985in}{1.991977in}}{\pgfqpoint{2.116375in}{1.981378in}}{\pgfqpoint{2.124188in}{1.973564in}}%
\pgfpathcurveto{\pgfqpoint{2.132002in}{1.965750in}}{\pgfqpoint{2.142601in}{1.961360in}}{\pgfqpoint{2.153651in}{1.961360in}}%
\pgfpathclose%
\pgfusepath{stroke,fill}%
\end{pgfscope}%
\begin{pgfscope}%
\pgfpathrectangle{\pgfqpoint{0.481978in}{0.331635in}}{\pgfqpoint{4.960000in}{3.696000in}}%
\pgfusepath{clip}%
\pgfsetbuttcap%
\pgfsetroundjoin%
\definecolor{currentfill}{rgb}{1.000000,0.705882,0.509804}%
\pgfsetfillcolor{currentfill}%
\pgfsetlinewidth{0.481800pt}%
\definecolor{currentstroke}{rgb}{1.000000,1.000000,1.000000}%
\pgfsetstrokecolor{currentstroke}%
\pgfsetdash{}{0pt}%
\pgfpathmoveto{\pgfqpoint{3.013735in}{2.535902in}}%
\pgfpathcurveto{\pgfqpoint{3.024785in}{2.535902in}}{\pgfqpoint{3.035384in}{2.540292in}}{\pgfqpoint{3.043197in}{2.548106in}}%
\pgfpathcurveto{\pgfqpoint{3.051011in}{2.555920in}}{\pgfqpoint{3.055401in}{2.566519in}}{\pgfqpoint{3.055401in}{2.577569in}}%
\pgfpathcurveto{\pgfqpoint{3.055401in}{2.588619in}}{\pgfqpoint{3.051011in}{2.599218in}}{\pgfqpoint{3.043197in}{2.607032in}}%
\pgfpathcurveto{\pgfqpoint{3.035384in}{2.614845in}}{\pgfqpoint{3.024785in}{2.619236in}}{\pgfqpoint{3.013735in}{2.619236in}}%
\pgfpathcurveto{\pgfqpoint{3.002685in}{2.619236in}}{\pgfqpoint{2.992086in}{2.614845in}}{\pgfqpoint{2.984272in}{2.607032in}}%
\pgfpathcurveto{\pgfqpoint{2.976458in}{2.599218in}}{\pgfqpoint{2.972068in}{2.588619in}}{\pgfqpoint{2.972068in}{2.577569in}}%
\pgfpathcurveto{\pgfqpoint{2.972068in}{2.566519in}}{\pgfqpoint{2.976458in}{2.555920in}}{\pgfqpoint{2.984272in}{2.548106in}}%
\pgfpathcurveto{\pgfqpoint{2.992086in}{2.540292in}}{\pgfqpoint{3.002685in}{2.535902in}}{\pgfqpoint{3.013735in}{2.535902in}}%
\pgfpathclose%
\pgfusepath{stroke,fill}%
\end{pgfscope}%
\begin{pgfscope}%
\pgfpathrectangle{\pgfqpoint{0.481978in}{0.331635in}}{\pgfqpoint{4.960000in}{3.696000in}}%
\pgfusepath{clip}%
\pgfsetbuttcap%
\pgfsetroundjoin%
\definecolor{currentfill}{rgb}{1.000000,0.705882,0.509804}%
\pgfsetfillcolor{currentfill}%
\pgfsetlinewidth{0.481800pt}%
\definecolor{currentstroke}{rgb}{1.000000,1.000000,1.000000}%
\pgfsetstrokecolor{currentstroke}%
\pgfsetdash{}{0pt}%
\pgfpathmoveto{\pgfqpoint{2.891152in}{1.854074in}}%
\pgfpathcurveto{\pgfqpoint{2.902202in}{1.854074in}}{\pgfqpoint{2.912801in}{1.858464in}}{\pgfqpoint{2.920615in}{1.866278in}}%
\pgfpathcurveto{\pgfqpoint{2.928428in}{1.874091in}}{\pgfqpoint{2.932819in}{1.884690in}}{\pgfqpoint{2.932819in}{1.895740in}}%
\pgfpathcurveto{\pgfqpoint{2.932819in}{1.906791in}}{\pgfqpoint{2.928428in}{1.917390in}}{\pgfqpoint{2.920615in}{1.925203in}}%
\pgfpathcurveto{\pgfqpoint{2.912801in}{1.933017in}}{\pgfqpoint{2.902202in}{1.937407in}}{\pgfqpoint{2.891152in}{1.937407in}}%
\pgfpathcurveto{\pgfqpoint{2.880102in}{1.937407in}}{\pgfqpoint{2.869503in}{1.933017in}}{\pgfqpoint{2.861689in}{1.925203in}}%
\pgfpathcurveto{\pgfqpoint{2.853875in}{1.917390in}}{\pgfqpoint{2.849485in}{1.906791in}}{\pgfqpoint{2.849485in}{1.895740in}}%
\pgfpathcurveto{\pgfqpoint{2.849485in}{1.884690in}}{\pgfqpoint{2.853875in}{1.874091in}}{\pgfqpoint{2.861689in}{1.866278in}}%
\pgfpathcurveto{\pgfqpoint{2.869503in}{1.858464in}}{\pgfqpoint{2.880102in}{1.854074in}}{\pgfqpoint{2.891152in}{1.854074in}}%
\pgfpathclose%
\pgfusepath{stroke,fill}%
\end{pgfscope}%
\begin{pgfscope}%
\pgfpathrectangle{\pgfqpoint{0.481978in}{0.331635in}}{\pgfqpoint{4.960000in}{3.696000in}}%
\pgfusepath{clip}%
\pgfsetbuttcap%
\pgfsetroundjoin%
\definecolor{currentfill}{rgb}{1.000000,0.705882,0.509804}%
\pgfsetfillcolor{currentfill}%
\pgfsetlinewidth{0.481800pt}%
\definecolor{currentstroke}{rgb}{1.000000,1.000000,1.000000}%
\pgfsetstrokecolor{currentstroke}%
\pgfsetdash{}{0pt}%
\pgfpathmoveto{\pgfqpoint{2.132533in}{1.757733in}}%
\pgfpathcurveto{\pgfqpoint{2.143583in}{1.757733in}}{\pgfqpoint{2.154182in}{1.762123in}}{\pgfqpoint{2.161996in}{1.769937in}}%
\pgfpathcurveto{\pgfqpoint{2.169810in}{1.777750in}}{\pgfqpoint{2.174200in}{1.788349in}}{\pgfqpoint{2.174200in}{1.799399in}}%
\pgfpathcurveto{\pgfqpoint{2.174200in}{1.810449in}}{\pgfqpoint{2.169810in}{1.821049in}}{\pgfqpoint{2.161996in}{1.828862in}}%
\pgfpathcurveto{\pgfqpoint{2.154182in}{1.836676in}}{\pgfqpoint{2.143583in}{1.841066in}}{\pgfqpoint{2.132533in}{1.841066in}}%
\pgfpathcurveto{\pgfqpoint{2.121483in}{1.841066in}}{\pgfqpoint{2.110884in}{1.836676in}}{\pgfqpoint{2.103070in}{1.828862in}}%
\pgfpathcurveto{\pgfqpoint{2.095257in}{1.821049in}}{\pgfqpoint{2.090866in}{1.810449in}}{\pgfqpoint{2.090866in}{1.799399in}}%
\pgfpathcurveto{\pgfqpoint{2.090866in}{1.788349in}}{\pgfqpoint{2.095257in}{1.777750in}}{\pgfqpoint{2.103070in}{1.769937in}}%
\pgfpathcurveto{\pgfqpoint{2.110884in}{1.762123in}}{\pgfqpoint{2.121483in}{1.757733in}}{\pgfqpoint{2.132533in}{1.757733in}}%
\pgfpathclose%
\pgfusepath{stroke,fill}%
\end{pgfscope}%
\begin{pgfscope}%
\pgfpathrectangle{\pgfqpoint{0.481978in}{0.331635in}}{\pgfqpoint{4.960000in}{3.696000in}}%
\pgfusepath{clip}%
\pgfsetbuttcap%
\pgfsetroundjoin%
\definecolor{currentfill}{rgb}{1.000000,0.705882,0.509804}%
\pgfsetfillcolor{currentfill}%
\pgfsetlinewidth{0.481800pt}%
\definecolor{currentstroke}{rgb}{1.000000,1.000000,1.000000}%
\pgfsetstrokecolor{currentstroke}%
\pgfsetdash{}{0pt}%
\pgfpathmoveto{\pgfqpoint{1.029745in}{1.864572in}}%
\pgfpathcurveto{\pgfqpoint{1.040795in}{1.864572in}}{\pgfqpoint{1.051394in}{1.868963in}}{\pgfqpoint{1.059207in}{1.876776in}}%
\pgfpathcurveto{\pgfqpoint{1.067021in}{1.884590in}}{\pgfqpoint{1.071411in}{1.895189in}}{\pgfqpoint{1.071411in}{1.906239in}}%
\pgfpathcurveto{\pgfqpoint{1.071411in}{1.917289in}}{\pgfqpoint{1.067021in}{1.927888in}}{\pgfqpoint{1.059207in}{1.935702in}}%
\pgfpathcurveto{\pgfqpoint{1.051394in}{1.943516in}}{\pgfqpoint{1.040795in}{1.947906in}}{\pgfqpoint{1.029745in}{1.947906in}}%
\pgfpathcurveto{\pgfqpoint{1.018694in}{1.947906in}}{\pgfqpoint{1.008095in}{1.943516in}}{\pgfqpoint{1.000282in}{1.935702in}}%
\pgfpathcurveto{\pgfqpoint{0.992468in}{1.927888in}}{\pgfqpoint{0.988078in}{1.917289in}}{\pgfqpoint{0.988078in}{1.906239in}}%
\pgfpathcurveto{\pgfqpoint{0.988078in}{1.895189in}}{\pgfqpoint{0.992468in}{1.884590in}}{\pgfqpoint{1.000282in}{1.876776in}}%
\pgfpathcurveto{\pgfqpoint{1.008095in}{1.868963in}}{\pgfqpoint{1.018694in}{1.864572in}}{\pgfqpoint{1.029745in}{1.864572in}}%
\pgfpathclose%
\pgfusepath{stroke,fill}%
\end{pgfscope}%
\begin{pgfscope}%
\pgfpathrectangle{\pgfqpoint{0.481978in}{0.331635in}}{\pgfqpoint{4.960000in}{3.696000in}}%
\pgfusepath{clip}%
\pgfsetbuttcap%
\pgfsetroundjoin%
\definecolor{currentfill}{rgb}{1.000000,0.705882,0.509804}%
\pgfsetfillcolor{currentfill}%
\pgfsetlinewidth{0.481800pt}%
\definecolor{currentstroke}{rgb}{1.000000,1.000000,1.000000}%
\pgfsetstrokecolor{currentstroke}%
\pgfsetdash{}{0pt}%
\pgfpathmoveto{\pgfqpoint{1.708883in}{2.455552in}}%
\pgfpathcurveto{\pgfqpoint{1.719933in}{2.455552in}}{\pgfqpoint{1.730532in}{2.459942in}}{\pgfqpoint{1.738346in}{2.467756in}}%
\pgfpathcurveto{\pgfqpoint{1.746160in}{2.475569in}}{\pgfqpoint{1.750550in}{2.486168in}}{\pgfqpoint{1.750550in}{2.497218in}}%
\pgfpathcurveto{\pgfqpoint{1.750550in}{2.508269in}}{\pgfqpoint{1.746160in}{2.518868in}}{\pgfqpoint{1.738346in}{2.526681in}}%
\pgfpathcurveto{\pgfqpoint{1.730532in}{2.534495in}}{\pgfqpoint{1.719933in}{2.538885in}}{\pgfqpoint{1.708883in}{2.538885in}}%
\pgfpathcurveto{\pgfqpoint{1.697833in}{2.538885in}}{\pgfqpoint{1.687234in}{2.534495in}}{\pgfqpoint{1.679420in}{2.526681in}}%
\pgfpathcurveto{\pgfqpoint{1.671607in}{2.518868in}}{\pgfqpoint{1.667217in}{2.508269in}}{\pgfqpoint{1.667217in}{2.497218in}}%
\pgfpathcurveto{\pgfqpoint{1.667217in}{2.486168in}}{\pgfqpoint{1.671607in}{2.475569in}}{\pgfqpoint{1.679420in}{2.467756in}}%
\pgfpathcurveto{\pgfqpoint{1.687234in}{2.459942in}}{\pgfqpoint{1.697833in}{2.455552in}}{\pgfqpoint{1.708883in}{2.455552in}}%
\pgfpathclose%
\pgfusepath{stroke,fill}%
\end{pgfscope}%
\begin{pgfscope}%
\pgfpathrectangle{\pgfqpoint{0.481978in}{0.331635in}}{\pgfqpoint{4.960000in}{3.696000in}}%
\pgfusepath{clip}%
\pgfsetbuttcap%
\pgfsetroundjoin%
\definecolor{currentfill}{rgb}{1.000000,0.705882,0.509804}%
\pgfsetfillcolor{currentfill}%
\pgfsetlinewidth{0.481800pt}%
\definecolor{currentstroke}{rgb}{1.000000,1.000000,1.000000}%
\pgfsetstrokecolor{currentstroke}%
\pgfsetdash{}{0pt}%
\pgfpathmoveto{\pgfqpoint{1.090142in}{2.240694in}}%
\pgfpathcurveto{\pgfqpoint{1.101192in}{2.240694in}}{\pgfqpoint{1.111791in}{2.245085in}}{\pgfqpoint{1.119605in}{2.252898in}}%
\pgfpathcurveto{\pgfqpoint{1.127418in}{2.260712in}}{\pgfqpoint{1.131809in}{2.271311in}}{\pgfqpoint{1.131809in}{2.282361in}}%
\pgfpathcurveto{\pgfqpoint{1.131809in}{2.293411in}}{\pgfqpoint{1.127418in}{2.304010in}}{\pgfqpoint{1.119605in}{2.311824in}}%
\pgfpathcurveto{\pgfqpoint{1.111791in}{2.319638in}}{\pgfqpoint{1.101192in}{2.324028in}}{\pgfqpoint{1.090142in}{2.324028in}}%
\pgfpathcurveto{\pgfqpoint{1.079092in}{2.324028in}}{\pgfqpoint{1.068493in}{2.319638in}}{\pgfqpoint{1.060679in}{2.311824in}}%
\pgfpathcurveto{\pgfqpoint{1.052866in}{2.304010in}}{\pgfqpoint{1.048475in}{2.293411in}}{\pgfqpoint{1.048475in}{2.282361in}}%
\pgfpathcurveto{\pgfqpoint{1.048475in}{2.271311in}}{\pgfqpoint{1.052866in}{2.260712in}}{\pgfqpoint{1.060679in}{2.252898in}}%
\pgfpathcurveto{\pgfqpoint{1.068493in}{2.245085in}}{\pgfqpoint{1.079092in}{2.240694in}}{\pgfqpoint{1.090142in}{2.240694in}}%
\pgfpathclose%
\pgfusepath{stroke,fill}%
\end{pgfscope}%
\begin{pgfscope}%
\pgfpathrectangle{\pgfqpoint{0.481978in}{0.331635in}}{\pgfqpoint{4.960000in}{3.696000in}}%
\pgfusepath{clip}%
\pgfsetbuttcap%
\pgfsetroundjoin%
\definecolor{currentfill}{rgb}{1.000000,0.705882,0.509804}%
\pgfsetfillcolor{currentfill}%
\pgfsetlinewidth{0.481800pt}%
\definecolor{currentstroke}{rgb}{1.000000,1.000000,1.000000}%
\pgfsetstrokecolor{currentstroke}%
\pgfsetdash{}{0pt}%
\pgfpathmoveto{\pgfqpoint{1.678355in}{3.220825in}}%
\pgfpathcurveto{\pgfqpoint{1.689405in}{3.220825in}}{\pgfqpoint{1.700004in}{3.225215in}}{\pgfqpoint{1.707818in}{3.233029in}}%
\pgfpathcurveto{\pgfqpoint{1.715632in}{3.240842in}}{\pgfqpoint{1.720022in}{3.251441in}}{\pgfqpoint{1.720022in}{3.262492in}}%
\pgfpathcurveto{\pgfqpoint{1.720022in}{3.273542in}}{\pgfqpoint{1.715632in}{3.284141in}}{\pgfqpoint{1.707818in}{3.291954in}}%
\pgfpathcurveto{\pgfqpoint{1.700004in}{3.299768in}}{\pgfqpoint{1.689405in}{3.304158in}}{\pgfqpoint{1.678355in}{3.304158in}}%
\pgfpathcurveto{\pgfqpoint{1.667305in}{3.304158in}}{\pgfqpoint{1.656706in}{3.299768in}}{\pgfqpoint{1.648892in}{3.291954in}}%
\pgfpathcurveto{\pgfqpoint{1.641079in}{3.284141in}}{\pgfqpoint{1.636688in}{3.273542in}}{\pgfqpoint{1.636688in}{3.262492in}}%
\pgfpathcurveto{\pgfqpoint{1.636688in}{3.251441in}}{\pgfqpoint{1.641079in}{3.240842in}}{\pgfqpoint{1.648892in}{3.233029in}}%
\pgfpathcurveto{\pgfqpoint{1.656706in}{3.225215in}}{\pgfqpoint{1.667305in}{3.220825in}}{\pgfqpoint{1.678355in}{3.220825in}}%
\pgfpathclose%
\pgfusepath{stroke,fill}%
\end{pgfscope}%
\begin{pgfscope}%
\pgfpathrectangle{\pgfqpoint{0.481978in}{0.331635in}}{\pgfqpoint{4.960000in}{3.696000in}}%
\pgfusepath{clip}%
\pgfsetbuttcap%
\pgfsetroundjoin%
\definecolor{currentfill}{rgb}{1.000000,0.705882,0.509804}%
\pgfsetfillcolor{currentfill}%
\pgfsetlinewidth{0.481800pt}%
\definecolor{currentstroke}{rgb}{1.000000,1.000000,1.000000}%
\pgfsetstrokecolor{currentstroke}%
\pgfsetdash{}{0pt}%
\pgfpathmoveto{\pgfqpoint{1.865099in}{2.625195in}}%
\pgfpathcurveto{\pgfqpoint{1.876149in}{2.625195in}}{\pgfqpoint{1.886748in}{2.629586in}}{\pgfqpoint{1.894562in}{2.637399in}}%
\pgfpathcurveto{\pgfqpoint{1.902376in}{2.645213in}}{\pgfqpoint{1.906766in}{2.655812in}}{\pgfqpoint{1.906766in}{2.666862in}}%
\pgfpathcurveto{\pgfqpoint{1.906766in}{2.677912in}}{\pgfqpoint{1.902376in}{2.688511in}}{\pgfqpoint{1.894562in}{2.696325in}}%
\pgfpathcurveto{\pgfqpoint{1.886748in}{2.704138in}}{\pgfqpoint{1.876149in}{2.708529in}}{\pgfqpoint{1.865099in}{2.708529in}}%
\pgfpathcurveto{\pgfqpoint{1.854049in}{2.708529in}}{\pgfqpoint{1.843450in}{2.704138in}}{\pgfqpoint{1.835636in}{2.696325in}}%
\pgfpathcurveto{\pgfqpoint{1.827823in}{2.688511in}}{\pgfqpoint{1.823432in}{2.677912in}}{\pgfqpoint{1.823432in}{2.666862in}}%
\pgfpathcurveto{\pgfqpoint{1.823432in}{2.655812in}}{\pgfqpoint{1.827823in}{2.645213in}}{\pgfqpoint{1.835636in}{2.637399in}}%
\pgfpathcurveto{\pgfqpoint{1.843450in}{2.629586in}}{\pgfqpoint{1.854049in}{2.625195in}}{\pgfqpoint{1.865099in}{2.625195in}}%
\pgfpathclose%
\pgfusepath{stroke,fill}%
\end{pgfscope}%
\begin{pgfscope}%
\pgfpathrectangle{\pgfqpoint{0.481978in}{0.331635in}}{\pgfqpoint{4.960000in}{3.696000in}}%
\pgfusepath{clip}%
\pgfsetbuttcap%
\pgfsetroundjoin%
\definecolor{currentfill}{rgb}{1.000000,0.705882,0.509804}%
\pgfsetfillcolor{currentfill}%
\pgfsetlinewidth{0.481800pt}%
\definecolor{currentstroke}{rgb}{1.000000,1.000000,1.000000}%
\pgfsetstrokecolor{currentstroke}%
\pgfsetdash{}{0pt}%
\pgfpathmoveto{\pgfqpoint{1.669235in}{2.251310in}}%
\pgfpathcurveto{\pgfqpoint{1.680285in}{2.251310in}}{\pgfqpoint{1.690884in}{2.255700in}}{\pgfqpoint{1.698697in}{2.263514in}}%
\pgfpathcurveto{\pgfqpoint{1.706511in}{2.271328in}}{\pgfqpoint{1.710901in}{2.281927in}}{\pgfqpoint{1.710901in}{2.292977in}}%
\pgfpathcurveto{\pgfqpoint{1.710901in}{2.304027in}}{\pgfqpoint{1.706511in}{2.314626in}}{\pgfqpoint{1.698697in}{2.322440in}}%
\pgfpathcurveto{\pgfqpoint{1.690884in}{2.330253in}}{\pgfqpoint{1.680285in}{2.334643in}}{\pgfqpoint{1.669235in}{2.334643in}}%
\pgfpathcurveto{\pgfqpoint{1.658185in}{2.334643in}}{\pgfqpoint{1.647586in}{2.330253in}}{\pgfqpoint{1.639772in}{2.322440in}}%
\pgfpathcurveto{\pgfqpoint{1.631958in}{2.314626in}}{\pgfqpoint{1.627568in}{2.304027in}}{\pgfqpoint{1.627568in}{2.292977in}}%
\pgfpathcurveto{\pgfqpoint{1.627568in}{2.281927in}}{\pgfqpoint{1.631958in}{2.271328in}}{\pgfqpoint{1.639772in}{2.263514in}}%
\pgfpathcurveto{\pgfqpoint{1.647586in}{2.255700in}}{\pgfqpoint{1.658185in}{2.251310in}}{\pgfqpoint{1.669235in}{2.251310in}}%
\pgfpathclose%
\pgfusepath{stroke,fill}%
\end{pgfscope}%
\begin{pgfscope}%
\pgfpathrectangle{\pgfqpoint{0.481978in}{0.331635in}}{\pgfqpoint{4.960000in}{3.696000in}}%
\pgfusepath{clip}%
\pgfsetbuttcap%
\pgfsetroundjoin%
\definecolor{currentfill}{rgb}{1.000000,0.705882,0.509804}%
\pgfsetfillcolor{currentfill}%
\pgfsetlinewidth{0.481800pt}%
\definecolor{currentstroke}{rgb}{1.000000,1.000000,1.000000}%
\pgfsetstrokecolor{currentstroke}%
\pgfsetdash{}{0pt}%
\pgfpathmoveto{\pgfqpoint{2.275735in}{3.724441in}}%
\pgfpathcurveto{\pgfqpoint{2.286785in}{3.724441in}}{\pgfqpoint{2.297384in}{3.728831in}}{\pgfqpoint{2.305198in}{3.736645in}}%
\pgfpathcurveto{\pgfqpoint{2.313012in}{3.744459in}}{\pgfqpoint{2.317402in}{3.755058in}}{\pgfqpoint{2.317402in}{3.766108in}}%
\pgfpathcurveto{\pgfqpoint{2.317402in}{3.777158in}}{\pgfqpoint{2.313012in}{3.787757in}}{\pgfqpoint{2.305198in}{3.795570in}}%
\pgfpathcurveto{\pgfqpoint{2.297384in}{3.803384in}}{\pgfqpoint{2.286785in}{3.807774in}}{\pgfqpoint{2.275735in}{3.807774in}}%
\pgfpathcurveto{\pgfqpoint{2.264685in}{3.807774in}}{\pgfqpoint{2.254086in}{3.803384in}}{\pgfqpoint{2.246272in}{3.795570in}}%
\pgfpathcurveto{\pgfqpoint{2.238459in}{3.787757in}}{\pgfqpoint{2.234069in}{3.777158in}}{\pgfqpoint{2.234069in}{3.766108in}}%
\pgfpathcurveto{\pgfqpoint{2.234069in}{3.755058in}}{\pgfqpoint{2.238459in}{3.744459in}}{\pgfqpoint{2.246272in}{3.736645in}}%
\pgfpathcurveto{\pgfqpoint{2.254086in}{3.728831in}}{\pgfqpoint{2.264685in}{3.724441in}}{\pgfqpoint{2.275735in}{3.724441in}}%
\pgfpathclose%
\pgfusepath{stroke,fill}%
\end{pgfscope}%
\begin{pgfscope}%
\pgfpathrectangle{\pgfqpoint{0.481978in}{0.331635in}}{\pgfqpoint{4.960000in}{3.696000in}}%
\pgfusepath{clip}%
\pgfsetbuttcap%
\pgfsetroundjoin%
\definecolor{currentfill}{rgb}{1.000000,0.705882,0.509804}%
\pgfsetfillcolor{currentfill}%
\pgfsetlinewidth{0.481800pt}%
\definecolor{currentstroke}{rgb}{1.000000,1.000000,1.000000}%
\pgfsetstrokecolor{currentstroke}%
\pgfsetdash{}{0pt}%
\pgfpathmoveto{\pgfqpoint{1.698708in}{2.079799in}}%
\pgfpathcurveto{\pgfqpoint{1.709758in}{2.079799in}}{\pgfqpoint{1.720357in}{2.084189in}}{\pgfqpoint{1.728171in}{2.092003in}}%
\pgfpathcurveto{\pgfqpoint{1.735984in}{2.099816in}}{\pgfqpoint{1.740375in}{2.110415in}}{\pgfqpoint{1.740375in}{2.121465in}}%
\pgfpathcurveto{\pgfqpoint{1.740375in}{2.132515in}}{\pgfqpoint{1.735984in}{2.143115in}}{\pgfqpoint{1.728171in}{2.150928in}}%
\pgfpathcurveto{\pgfqpoint{1.720357in}{2.158742in}}{\pgfqpoint{1.709758in}{2.163132in}}{\pgfqpoint{1.698708in}{2.163132in}}%
\pgfpathcurveto{\pgfqpoint{1.687658in}{2.163132in}}{\pgfqpoint{1.677059in}{2.158742in}}{\pgfqpoint{1.669245in}{2.150928in}}%
\pgfpathcurveto{\pgfqpoint{1.661432in}{2.143115in}}{\pgfqpoint{1.657041in}{2.132515in}}{\pgfqpoint{1.657041in}{2.121465in}}%
\pgfpathcurveto{\pgfqpoint{1.657041in}{2.110415in}}{\pgfqpoint{1.661432in}{2.099816in}}{\pgfqpoint{1.669245in}{2.092003in}}%
\pgfpathcurveto{\pgfqpoint{1.677059in}{2.084189in}}{\pgfqpoint{1.687658in}{2.079799in}}{\pgfqpoint{1.698708in}{2.079799in}}%
\pgfpathclose%
\pgfusepath{stroke,fill}%
\end{pgfscope}%
\begin{pgfscope}%
\pgfpathrectangle{\pgfqpoint{0.481978in}{0.331635in}}{\pgfqpoint{4.960000in}{3.696000in}}%
\pgfusepath{clip}%
\pgfsetbuttcap%
\pgfsetroundjoin%
\definecolor{currentfill}{rgb}{1.000000,0.705882,0.509804}%
\pgfsetfillcolor{currentfill}%
\pgfsetlinewidth{0.481800pt}%
\definecolor{currentstroke}{rgb}{1.000000,1.000000,1.000000}%
\pgfsetstrokecolor{currentstroke}%
\pgfsetdash{}{0pt}%
\pgfpathmoveto{\pgfqpoint{1.895840in}{2.202496in}}%
\pgfpathcurveto{\pgfqpoint{1.906890in}{2.202496in}}{\pgfqpoint{1.917489in}{2.206887in}}{\pgfqpoint{1.925303in}{2.214700in}}%
\pgfpathcurveto{\pgfqpoint{1.933116in}{2.222514in}}{\pgfqpoint{1.937507in}{2.233113in}}{\pgfqpoint{1.937507in}{2.244163in}}%
\pgfpathcurveto{\pgfqpoint{1.937507in}{2.255213in}}{\pgfqpoint{1.933116in}{2.265812in}}{\pgfqpoint{1.925303in}{2.273626in}}%
\pgfpathcurveto{\pgfqpoint{1.917489in}{2.281439in}}{\pgfqpoint{1.906890in}{2.285830in}}{\pgfqpoint{1.895840in}{2.285830in}}%
\pgfpathcurveto{\pgfqpoint{1.884790in}{2.285830in}}{\pgfqpoint{1.874191in}{2.281439in}}{\pgfqpoint{1.866377in}{2.273626in}}%
\pgfpathcurveto{\pgfqpoint{1.858563in}{2.265812in}}{\pgfqpoint{1.854173in}{2.255213in}}{\pgfqpoint{1.854173in}{2.244163in}}%
\pgfpathcurveto{\pgfqpoint{1.854173in}{2.233113in}}{\pgfqpoint{1.858563in}{2.222514in}}{\pgfqpoint{1.866377in}{2.214700in}}%
\pgfpathcurveto{\pgfqpoint{1.874191in}{2.206887in}}{\pgfqpoint{1.884790in}{2.202496in}}{\pgfqpoint{1.895840in}{2.202496in}}%
\pgfpathclose%
\pgfusepath{stroke,fill}%
\end{pgfscope}%
\begin{pgfscope}%
\pgfpathrectangle{\pgfqpoint{0.481978in}{0.331635in}}{\pgfqpoint{4.960000in}{3.696000in}}%
\pgfusepath{clip}%
\pgfsetbuttcap%
\pgfsetroundjoin%
\definecolor{currentfill}{rgb}{1.000000,0.705882,0.509804}%
\pgfsetfillcolor{currentfill}%
\pgfsetlinewidth{0.481800pt}%
\definecolor{currentstroke}{rgb}{1.000000,1.000000,1.000000}%
\pgfsetstrokecolor{currentstroke}%
\pgfsetdash{}{0pt}%
\pgfpathmoveto{\pgfqpoint{1.504175in}{2.162414in}}%
\pgfpathcurveto{\pgfqpoint{1.515225in}{2.162414in}}{\pgfqpoint{1.525824in}{2.166804in}}{\pgfqpoint{1.533637in}{2.174618in}}%
\pgfpathcurveto{\pgfqpoint{1.541451in}{2.182432in}}{\pgfqpoint{1.545841in}{2.193031in}}{\pgfqpoint{1.545841in}{2.204081in}}%
\pgfpathcurveto{\pgfqpoint{1.545841in}{2.215131in}}{\pgfqpoint{1.541451in}{2.225730in}}{\pgfqpoint{1.533637in}{2.233543in}}%
\pgfpathcurveto{\pgfqpoint{1.525824in}{2.241357in}}{\pgfqpoint{1.515225in}{2.245747in}}{\pgfqpoint{1.504175in}{2.245747in}}%
\pgfpathcurveto{\pgfqpoint{1.493125in}{2.245747in}}{\pgfqpoint{1.482526in}{2.241357in}}{\pgfqpoint{1.474712in}{2.233543in}}%
\pgfpathcurveto{\pgfqpoint{1.466898in}{2.225730in}}{\pgfqpoint{1.462508in}{2.215131in}}{\pgfqpoint{1.462508in}{2.204081in}}%
\pgfpathcurveto{\pgfqpoint{1.462508in}{2.193031in}}{\pgfqpoint{1.466898in}{2.182432in}}{\pgfqpoint{1.474712in}{2.174618in}}%
\pgfpathcurveto{\pgfqpoint{1.482526in}{2.166804in}}{\pgfqpoint{1.493125in}{2.162414in}}{\pgfqpoint{1.504175in}{2.162414in}}%
\pgfpathclose%
\pgfusepath{stroke,fill}%
\end{pgfscope}%
\begin{pgfscope}%
\pgfpathrectangle{\pgfqpoint{0.481978in}{0.331635in}}{\pgfqpoint{4.960000in}{3.696000in}}%
\pgfusepath{clip}%
\pgfsetbuttcap%
\pgfsetroundjoin%
\definecolor{currentfill}{rgb}{1.000000,0.705882,0.509804}%
\pgfsetfillcolor{currentfill}%
\pgfsetlinewidth{0.481800pt}%
\definecolor{currentstroke}{rgb}{1.000000,1.000000,1.000000}%
\pgfsetstrokecolor{currentstroke}%
\pgfsetdash{}{0pt}%
\pgfpathmoveto{\pgfqpoint{2.448513in}{2.312803in}}%
\pgfpathcurveto{\pgfqpoint{2.459563in}{2.312803in}}{\pgfqpoint{2.470162in}{2.317194in}}{\pgfqpoint{2.477975in}{2.325007in}}%
\pgfpathcurveto{\pgfqpoint{2.485789in}{2.332821in}}{\pgfqpoint{2.490179in}{2.343420in}}{\pgfqpoint{2.490179in}{2.354470in}}%
\pgfpathcurveto{\pgfqpoint{2.490179in}{2.365520in}}{\pgfqpoint{2.485789in}{2.376119in}}{\pgfqpoint{2.477975in}{2.383933in}}%
\pgfpathcurveto{\pgfqpoint{2.470162in}{2.391746in}}{\pgfqpoint{2.459563in}{2.396137in}}{\pgfqpoint{2.448513in}{2.396137in}}%
\pgfpathcurveto{\pgfqpoint{2.437462in}{2.396137in}}{\pgfqpoint{2.426863in}{2.391746in}}{\pgfqpoint{2.419050in}{2.383933in}}%
\pgfpathcurveto{\pgfqpoint{2.411236in}{2.376119in}}{\pgfqpoint{2.406846in}{2.365520in}}{\pgfqpoint{2.406846in}{2.354470in}}%
\pgfpathcurveto{\pgfqpoint{2.406846in}{2.343420in}}{\pgfqpoint{2.411236in}{2.332821in}}{\pgfqpoint{2.419050in}{2.325007in}}%
\pgfpathcurveto{\pgfqpoint{2.426863in}{2.317194in}}{\pgfqpoint{2.437462in}{2.312803in}}{\pgfqpoint{2.448513in}{2.312803in}}%
\pgfpathclose%
\pgfusepath{stroke,fill}%
\end{pgfscope}%
\begin{pgfscope}%
\pgfpathrectangle{\pgfqpoint{0.481978in}{0.331635in}}{\pgfqpoint{4.960000in}{3.696000in}}%
\pgfusepath{clip}%
\pgfsetbuttcap%
\pgfsetroundjoin%
\definecolor{currentfill}{rgb}{1.000000,0.705882,0.509804}%
\pgfsetfillcolor{currentfill}%
\pgfsetlinewidth{0.481800pt}%
\definecolor{currentstroke}{rgb}{1.000000,1.000000,1.000000}%
\pgfsetstrokecolor{currentstroke}%
\pgfsetdash{}{0pt}%
\pgfpathmoveto{\pgfqpoint{1.491354in}{3.311396in}}%
\pgfpathcurveto{\pgfqpoint{1.502405in}{3.311396in}}{\pgfqpoint{1.513004in}{3.315787in}}{\pgfqpoint{1.520817in}{3.323600in}}%
\pgfpathcurveto{\pgfqpoint{1.528631in}{3.331414in}}{\pgfqpoint{1.533021in}{3.342013in}}{\pgfqpoint{1.533021in}{3.353063in}}%
\pgfpathcurveto{\pgfqpoint{1.533021in}{3.364113in}}{\pgfqpoint{1.528631in}{3.374712in}}{\pgfqpoint{1.520817in}{3.382526in}}%
\pgfpathcurveto{\pgfqpoint{1.513004in}{3.390340in}}{\pgfqpoint{1.502405in}{3.394730in}}{\pgfqpoint{1.491354in}{3.394730in}}%
\pgfpathcurveto{\pgfqpoint{1.480304in}{3.394730in}}{\pgfqpoint{1.469705in}{3.390340in}}{\pgfqpoint{1.461892in}{3.382526in}}%
\pgfpathcurveto{\pgfqpoint{1.454078in}{3.374712in}}{\pgfqpoint{1.449688in}{3.364113in}}{\pgfqpoint{1.449688in}{3.353063in}}%
\pgfpathcurveto{\pgfqpoint{1.449688in}{3.342013in}}{\pgfqpoint{1.454078in}{3.331414in}}{\pgfqpoint{1.461892in}{3.323600in}}%
\pgfpathcurveto{\pgfqpoint{1.469705in}{3.315787in}}{\pgfqpoint{1.480304in}{3.311396in}}{\pgfqpoint{1.491354in}{3.311396in}}%
\pgfpathclose%
\pgfusepath{stroke,fill}%
\end{pgfscope}%
\begin{pgfscope}%
\pgfpathrectangle{\pgfqpoint{0.481978in}{0.331635in}}{\pgfqpoint{4.960000in}{3.696000in}}%
\pgfusepath{clip}%
\pgfsetbuttcap%
\pgfsetroundjoin%
\definecolor{currentfill}{rgb}{1.000000,0.705882,0.509804}%
\pgfsetfillcolor{currentfill}%
\pgfsetlinewidth{0.481800pt}%
\definecolor{currentstroke}{rgb}{1.000000,1.000000,1.000000}%
\pgfsetstrokecolor{currentstroke}%
\pgfsetdash{}{0pt}%
\pgfpathmoveto{\pgfqpoint{2.562202in}{2.853413in}}%
\pgfpathcurveto{\pgfqpoint{2.573252in}{2.853413in}}{\pgfqpoint{2.583851in}{2.857803in}}{\pgfqpoint{2.591665in}{2.865617in}}%
\pgfpathcurveto{\pgfqpoint{2.599478in}{2.873431in}}{\pgfqpoint{2.603869in}{2.884030in}}{\pgfqpoint{2.603869in}{2.895080in}}%
\pgfpathcurveto{\pgfqpoint{2.603869in}{2.906130in}}{\pgfqpoint{2.599478in}{2.916729in}}{\pgfqpoint{2.591665in}{2.924543in}}%
\pgfpathcurveto{\pgfqpoint{2.583851in}{2.932356in}}{\pgfqpoint{2.573252in}{2.936747in}}{\pgfqpoint{2.562202in}{2.936747in}}%
\pgfpathcurveto{\pgfqpoint{2.551152in}{2.936747in}}{\pgfqpoint{2.540553in}{2.932356in}}{\pgfqpoint{2.532739in}{2.924543in}}%
\pgfpathcurveto{\pgfqpoint{2.524925in}{2.916729in}}{\pgfqpoint{2.520535in}{2.906130in}}{\pgfqpoint{2.520535in}{2.895080in}}%
\pgfpathcurveto{\pgfqpoint{2.520535in}{2.884030in}}{\pgfqpoint{2.524925in}{2.873431in}}{\pgfqpoint{2.532739in}{2.865617in}}%
\pgfpathcurveto{\pgfqpoint{2.540553in}{2.857803in}}{\pgfqpoint{2.551152in}{2.853413in}}{\pgfqpoint{2.562202in}{2.853413in}}%
\pgfpathclose%
\pgfusepath{stroke,fill}%
\end{pgfscope}%
\begin{pgfscope}%
\pgfpathrectangle{\pgfqpoint{0.481978in}{0.331635in}}{\pgfqpoint{4.960000in}{3.696000in}}%
\pgfusepath{clip}%
\pgfsetbuttcap%
\pgfsetroundjoin%
\definecolor{currentfill}{rgb}{1.000000,0.705882,0.509804}%
\pgfsetfillcolor{currentfill}%
\pgfsetlinewidth{0.481800pt}%
\definecolor{currentstroke}{rgb}{1.000000,1.000000,1.000000}%
\pgfsetstrokecolor{currentstroke}%
\pgfsetdash{}{0pt}%
\pgfpathmoveto{\pgfqpoint{3.825003in}{2.911916in}}%
\pgfpathcurveto{\pgfqpoint{3.836053in}{2.911916in}}{\pgfqpoint{3.846652in}{2.916306in}}{\pgfqpoint{3.854466in}{2.924119in}}%
\pgfpathcurveto{\pgfqpoint{3.862279in}{2.931933in}}{\pgfqpoint{3.866670in}{2.942532in}}{\pgfqpoint{3.866670in}{2.953582in}}%
\pgfpathcurveto{\pgfqpoint{3.866670in}{2.964632in}}{\pgfqpoint{3.862279in}{2.975231in}}{\pgfqpoint{3.854466in}{2.983045in}}%
\pgfpathcurveto{\pgfqpoint{3.846652in}{2.990859in}}{\pgfqpoint{3.836053in}{2.995249in}}{\pgfqpoint{3.825003in}{2.995249in}}%
\pgfpathcurveto{\pgfqpoint{3.813953in}{2.995249in}}{\pgfqpoint{3.803354in}{2.990859in}}{\pgfqpoint{3.795540in}{2.983045in}}%
\pgfpathcurveto{\pgfqpoint{3.787726in}{2.975231in}}{\pgfqpoint{3.783336in}{2.964632in}}{\pgfqpoint{3.783336in}{2.953582in}}%
\pgfpathcurveto{\pgfqpoint{3.783336in}{2.942532in}}{\pgfqpoint{3.787726in}{2.931933in}}{\pgfqpoint{3.795540in}{2.924119in}}%
\pgfpathcurveto{\pgfqpoint{3.803354in}{2.916306in}}{\pgfqpoint{3.813953in}{2.911916in}}{\pgfqpoint{3.825003in}{2.911916in}}%
\pgfpathclose%
\pgfusepath{stroke,fill}%
\end{pgfscope}%
\begin{pgfscope}%
\pgfpathrectangle{\pgfqpoint{0.481978in}{0.331635in}}{\pgfqpoint{4.960000in}{3.696000in}}%
\pgfusepath{clip}%
\pgfsetbuttcap%
\pgfsetroundjoin%
\definecolor{currentfill}{rgb}{1.000000,0.705882,0.509804}%
\pgfsetfillcolor{currentfill}%
\pgfsetlinewidth{0.481800pt}%
\definecolor{currentstroke}{rgb}{1.000000,1.000000,1.000000}%
\pgfsetstrokecolor{currentstroke}%
\pgfsetdash{}{0pt}%
\pgfpathmoveto{\pgfqpoint{2.213684in}{3.120645in}}%
\pgfpathcurveto{\pgfqpoint{2.224734in}{3.120645in}}{\pgfqpoint{2.235333in}{3.125035in}}{\pgfqpoint{2.243147in}{3.132849in}}%
\pgfpathcurveto{\pgfqpoint{2.250960in}{3.140662in}}{\pgfqpoint{2.255350in}{3.151261in}}{\pgfqpoint{2.255350in}{3.162311in}}%
\pgfpathcurveto{\pgfqpoint{2.255350in}{3.173362in}}{\pgfqpoint{2.250960in}{3.183961in}}{\pgfqpoint{2.243147in}{3.191774in}}%
\pgfpathcurveto{\pgfqpoint{2.235333in}{3.199588in}}{\pgfqpoint{2.224734in}{3.203978in}}{\pgfqpoint{2.213684in}{3.203978in}}%
\pgfpathcurveto{\pgfqpoint{2.202634in}{3.203978in}}{\pgfqpoint{2.192035in}{3.199588in}}{\pgfqpoint{2.184221in}{3.191774in}}%
\pgfpathcurveto{\pgfqpoint{2.176407in}{3.183961in}}{\pgfqpoint{2.172017in}{3.173362in}}{\pgfqpoint{2.172017in}{3.162311in}}%
\pgfpathcurveto{\pgfqpoint{2.172017in}{3.151261in}}{\pgfqpoint{2.176407in}{3.140662in}}{\pgfqpoint{2.184221in}{3.132849in}}%
\pgfpathcurveto{\pgfqpoint{2.192035in}{3.125035in}}{\pgfqpoint{2.202634in}{3.120645in}}{\pgfqpoint{2.213684in}{3.120645in}}%
\pgfpathclose%
\pgfusepath{stroke,fill}%
\end{pgfscope}%
\begin{pgfscope}%
\pgfpathrectangle{\pgfqpoint{0.481978in}{0.331635in}}{\pgfqpoint{4.960000in}{3.696000in}}%
\pgfusepath{clip}%
\pgfsetbuttcap%
\pgfsetroundjoin%
\definecolor{currentfill}{rgb}{1.000000,0.705882,0.509804}%
\pgfsetfillcolor{currentfill}%
\pgfsetlinewidth{0.481800pt}%
\definecolor{currentstroke}{rgb}{1.000000,1.000000,1.000000}%
\pgfsetstrokecolor{currentstroke}%
\pgfsetdash{}{0pt}%
\pgfpathmoveto{\pgfqpoint{2.808638in}{2.474572in}}%
\pgfpathcurveto{\pgfqpoint{2.819689in}{2.474572in}}{\pgfqpoint{2.830288in}{2.478962in}}{\pgfqpoint{2.838101in}{2.486775in}}%
\pgfpathcurveto{\pgfqpoint{2.845915in}{2.494589in}}{\pgfqpoint{2.850305in}{2.505188in}}{\pgfqpoint{2.850305in}{2.516238in}}%
\pgfpathcurveto{\pgfqpoint{2.850305in}{2.527288in}}{\pgfqpoint{2.845915in}{2.537887in}}{\pgfqpoint{2.838101in}{2.545701in}}%
\pgfpathcurveto{\pgfqpoint{2.830288in}{2.553515in}}{\pgfqpoint{2.819689in}{2.557905in}}{\pgfqpoint{2.808638in}{2.557905in}}%
\pgfpathcurveto{\pgfqpoint{2.797588in}{2.557905in}}{\pgfqpoint{2.786989in}{2.553515in}}{\pgfqpoint{2.779176in}{2.545701in}}%
\pgfpathcurveto{\pgfqpoint{2.771362in}{2.537887in}}{\pgfqpoint{2.766972in}{2.527288in}}{\pgfqpoint{2.766972in}{2.516238in}}%
\pgfpathcurveto{\pgfqpoint{2.766972in}{2.505188in}}{\pgfqpoint{2.771362in}{2.494589in}}{\pgfqpoint{2.779176in}{2.486775in}}%
\pgfpathcurveto{\pgfqpoint{2.786989in}{2.478962in}}{\pgfqpoint{2.797588in}{2.474572in}}{\pgfqpoint{2.808638in}{2.474572in}}%
\pgfpathclose%
\pgfusepath{stroke,fill}%
\end{pgfscope}%
\begin{pgfscope}%
\pgfpathrectangle{\pgfqpoint{0.481978in}{0.331635in}}{\pgfqpoint{4.960000in}{3.696000in}}%
\pgfusepath{clip}%
\pgfsetbuttcap%
\pgfsetroundjoin%
\definecolor{currentfill}{rgb}{1.000000,0.705882,0.509804}%
\pgfsetfillcolor{currentfill}%
\pgfsetlinewidth{0.481800pt}%
\definecolor{currentstroke}{rgb}{1.000000,1.000000,1.000000}%
\pgfsetstrokecolor{currentstroke}%
\pgfsetdash{}{0pt}%
\pgfpathmoveto{\pgfqpoint{3.433772in}{3.051637in}}%
\pgfpathcurveto{\pgfqpoint{3.444822in}{3.051637in}}{\pgfqpoint{3.455421in}{3.056027in}}{\pgfqpoint{3.463234in}{3.063841in}}%
\pgfpathcurveto{\pgfqpoint{3.471048in}{3.071654in}}{\pgfqpoint{3.475438in}{3.082253in}}{\pgfqpoint{3.475438in}{3.093303in}}%
\pgfpathcurveto{\pgfqpoint{3.475438in}{3.104353in}}{\pgfqpoint{3.471048in}{3.114952in}}{\pgfqpoint{3.463234in}{3.122766in}}%
\pgfpathcurveto{\pgfqpoint{3.455421in}{3.130580in}}{\pgfqpoint{3.444822in}{3.134970in}}{\pgfqpoint{3.433772in}{3.134970in}}%
\pgfpathcurveto{\pgfqpoint{3.422722in}{3.134970in}}{\pgfqpoint{3.412122in}{3.130580in}}{\pgfqpoint{3.404309in}{3.122766in}}%
\pgfpathcurveto{\pgfqpoint{3.396495in}{3.114952in}}{\pgfqpoint{3.392105in}{3.104353in}}{\pgfqpoint{3.392105in}{3.093303in}}%
\pgfpathcurveto{\pgfqpoint{3.392105in}{3.082253in}}{\pgfqpoint{3.396495in}{3.071654in}}{\pgfqpoint{3.404309in}{3.063841in}}%
\pgfpathcurveto{\pgfqpoint{3.412122in}{3.056027in}}{\pgfqpoint{3.422722in}{3.051637in}}{\pgfqpoint{3.433772in}{3.051637in}}%
\pgfpathclose%
\pgfusepath{stroke,fill}%
\end{pgfscope}%
\begin{pgfscope}%
\pgfpathrectangle{\pgfqpoint{0.481978in}{0.331635in}}{\pgfqpoint{4.960000in}{3.696000in}}%
\pgfusepath{clip}%
\pgfsetbuttcap%
\pgfsetroundjoin%
\definecolor{currentfill}{rgb}{1.000000,0.705882,0.509804}%
\pgfsetfillcolor{currentfill}%
\pgfsetlinewidth{0.481800pt}%
\definecolor{currentstroke}{rgb}{1.000000,1.000000,1.000000}%
\pgfsetstrokecolor{currentstroke}%
\pgfsetdash{}{0pt}%
\pgfpathmoveto{\pgfqpoint{3.258010in}{2.112862in}}%
\pgfpathcurveto{\pgfqpoint{3.269060in}{2.112862in}}{\pgfqpoint{3.279659in}{2.117252in}}{\pgfqpoint{3.287473in}{2.125066in}}%
\pgfpathcurveto{\pgfqpoint{3.295287in}{2.132879in}}{\pgfqpoint{3.299677in}{2.143478in}}{\pgfqpoint{3.299677in}{2.154528in}}%
\pgfpathcurveto{\pgfqpoint{3.299677in}{2.165579in}}{\pgfqpoint{3.295287in}{2.176178in}}{\pgfqpoint{3.287473in}{2.183991in}}%
\pgfpathcurveto{\pgfqpoint{3.279659in}{2.191805in}}{\pgfqpoint{3.269060in}{2.196195in}}{\pgfqpoint{3.258010in}{2.196195in}}%
\pgfpathcurveto{\pgfqpoint{3.246960in}{2.196195in}}{\pgfqpoint{3.236361in}{2.191805in}}{\pgfqpoint{3.228547in}{2.183991in}}%
\pgfpathcurveto{\pgfqpoint{3.220734in}{2.176178in}}{\pgfqpoint{3.216343in}{2.165579in}}{\pgfqpoint{3.216343in}{2.154528in}}%
\pgfpathcurveto{\pgfqpoint{3.216343in}{2.143478in}}{\pgfqpoint{3.220734in}{2.132879in}}{\pgfqpoint{3.228547in}{2.125066in}}%
\pgfpathcurveto{\pgfqpoint{3.236361in}{2.117252in}}{\pgfqpoint{3.246960in}{2.112862in}}{\pgfqpoint{3.258010in}{2.112862in}}%
\pgfpathclose%
\pgfusepath{stroke,fill}%
\end{pgfscope}%
\begin{pgfscope}%
\pgfpathrectangle{\pgfqpoint{0.481978in}{0.331635in}}{\pgfqpoint{4.960000in}{3.696000in}}%
\pgfusepath{clip}%
\pgfsetbuttcap%
\pgfsetroundjoin%
\definecolor{currentfill}{rgb}{1.000000,0.705882,0.509804}%
\pgfsetfillcolor{currentfill}%
\pgfsetlinewidth{0.481800pt}%
\definecolor{currentstroke}{rgb}{1.000000,1.000000,1.000000}%
\pgfsetstrokecolor{currentstroke}%
\pgfsetdash{}{0pt}%
\pgfpathmoveto{\pgfqpoint{3.144336in}{2.889550in}}%
\pgfpathcurveto{\pgfqpoint{3.155386in}{2.889550in}}{\pgfqpoint{3.165985in}{2.893940in}}{\pgfqpoint{3.173799in}{2.901754in}}%
\pgfpathcurveto{\pgfqpoint{3.181612in}{2.909567in}}{\pgfqpoint{3.186003in}{2.920166in}}{\pgfqpoint{3.186003in}{2.931216in}}%
\pgfpathcurveto{\pgfqpoint{3.186003in}{2.942267in}}{\pgfqpoint{3.181612in}{2.952866in}}{\pgfqpoint{3.173799in}{2.960679in}}%
\pgfpathcurveto{\pgfqpoint{3.165985in}{2.968493in}}{\pgfqpoint{3.155386in}{2.972883in}}{\pgfqpoint{3.144336in}{2.972883in}}%
\pgfpathcurveto{\pgfqpoint{3.133286in}{2.972883in}}{\pgfqpoint{3.122687in}{2.968493in}}{\pgfqpoint{3.114873in}{2.960679in}}%
\pgfpathcurveto{\pgfqpoint{3.107060in}{2.952866in}}{\pgfqpoint{3.102669in}{2.942267in}}{\pgfqpoint{3.102669in}{2.931216in}}%
\pgfpathcurveto{\pgfqpoint{3.102669in}{2.920166in}}{\pgfqpoint{3.107060in}{2.909567in}}{\pgfqpoint{3.114873in}{2.901754in}}%
\pgfpathcurveto{\pgfqpoint{3.122687in}{2.893940in}}{\pgfqpoint{3.133286in}{2.889550in}}{\pgfqpoint{3.144336in}{2.889550in}}%
\pgfpathclose%
\pgfusepath{stroke,fill}%
\end{pgfscope}%
\begin{pgfscope}%
\pgfpathrectangle{\pgfqpoint{0.481978in}{0.331635in}}{\pgfqpoint{4.960000in}{3.696000in}}%
\pgfusepath{clip}%
\pgfsetbuttcap%
\pgfsetroundjoin%
\definecolor{currentfill}{rgb}{1.000000,0.705882,0.509804}%
\pgfsetfillcolor{currentfill}%
\pgfsetlinewidth{0.481800pt}%
\definecolor{currentstroke}{rgb}{1.000000,1.000000,1.000000}%
\pgfsetstrokecolor{currentstroke}%
\pgfsetdash{}{0pt}%
\pgfpathmoveto{\pgfqpoint{1.083523in}{2.452217in}}%
\pgfpathcurveto{\pgfqpoint{1.094573in}{2.452217in}}{\pgfqpoint{1.105172in}{2.456607in}}{\pgfqpoint{1.112986in}{2.464421in}}%
\pgfpathcurveto{\pgfqpoint{1.120800in}{2.472235in}}{\pgfqpoint{1.125190in}{2.482834in}}{\pgfqpoint{1.125190in}{2.493884in}}%
\pgfpathcurveto{\pgfqpoint{1.125190in}{2.504934in}}{\pgfqpoint{1.120800in}{2.515533in}}{\pgfqpoint{1.112986in}{2.523347in}}%
\pgfpathcurveto{\pgfqpoint{1.105172in}{2.531160in}}{\pgfqpoint{1.094573in}{2.535550in}}{\pgfqpoint{1.083523in}{2.535550in}}%
\pgfpathcurveto{\pgfqpoint{1.072473in}{2.535550in}}{\pgfqpoint{1.061874in}{2.531160in}}{\pgfqpoint{1.054060in}{2.523347in}}%
\pgfpathcurveto{\pgfqpoint{1.046247in}{2.515533in}}{\pgfqpoint{1.041857in}{2.504934in}}{\pgfqpoint{1.041857in}{2.493884in}}%
\pgfpathcurveto{\pgfqpoint{1.041857in}{2.482834in}}{\pgfqpoint{1.046247in}{2.472235in}}{\pgfqpoint{1.054060in}{2.464421in}}%
\pgfpathcurveto{\pgfqpoint{1.061874in}{2.456607in}}{\pgfqpoint{1.072473in}{2.452217in}}{\pgfqpoint{1.083523in}{2.452217in}}%
\pgfpathclose%
\pgfusepath{stroke,fill}%
\end{pgfscope}%
\begin{pgfscope}%
\pgfpathrectangle{\pgfqpoint{0.481978in}{0.331635in}}{\pgfqpoint{4.960000in}{3.696000in}}%
\pgfusepath{clip}%
\pgfsetbuttcap%
\pgfsetroundjoin%
\definecolor{currentfill}{rgb}{1.000000,0.705882,0.509804}%
\pgfsetfillcolor{currentfill}%
\pgfsetlinewidth{0.481800pt}%
\definecolor{currentstroke}{rgb}{1.000000,1.000000,1.000000}%
\pgfsetstrokecolor{currentstroke}%
\pgfsetdash{}{0pt}%
\pgfpathmoveto{\pgfqpoint{1.082227in}{2.612065in}}%
\pgfpathcurveto{\pgfqpoint{1.093277in}{2.612065in}}{\pgfqpoint{1.103876in}{2.616455in}}{\pgfqpoint{1.111690in}{2.624269in}}%
\pgfpathcurveto{\pgfqpoint{1.119503in}{2.632083in}}{\pgfqpoint{1.123893in}{2.642682in}}{\pgfqpoint{1.123893in}{2.653732in}}%
\pgfpathcurveto{\pgfqpoint{1.123893in}{2.664782in}}{\pgfqpoint{1.119503in}{2.675381in}}{\pgfqpoint{1.111690in}{2.683195in}}%
\pgfpathcurveto{\pgfqpoint{1.103876in}{2.691008in}}{\pgfqpoint{1.093277in}{2.695398in}}{\pgfqpoint{1.082227in}{2.695398in}}%
\pgfpathcurveto{\pgfqpoint{1.071177in}{2.695398in}}{\pgfqpoint{1.060578in}{2.691008in}}{\pgfqpoint{1.052764in}{2.683195in}}%
\pgfpathcurveto{\pgfqpoint{1.044950in}{2.675381in}}{\pgfqpoint{1.040560in}{2.664782in}}{\pgfqpoint{1.040560in}{2.653732in}}%
\pgfpathcurveto{\pgfqpoint{1.040560in}{2.642682in}}{\pgfqpoint{1.044950in}{2.632083in}}{\pgfqpoint{1.052764in}{2.624269in}}%
\pgfpathcurveto{\pgfqpoint{1.060578in}{2.616455in}}{\pgfqpoint{1.071177in}{2.612065in}}{\pgfqpoint{1.082227in}{2.612065in}}%
\pgfpathclose%
\pgfusepath{stroke,fill}%
\end{pgfscope}%
\begin{pgfscope}%
\pgfpathrectangle{\pgfqpoint{0.481978in}{0.331635in}}{\pgfqpoint{4.960000in}{3.696000in}}%
\pgfusepath{clip}%
\pgfsetbuttcap%
\pgfsetroundjoin%
\definecolor{currentfill}{rgb}{1.000000,0.705882,0.509804}%
\pgfsetfillcolor{currentfill}%
\pgfsetlinewidth{0.481800pt}%
\definecolor{currentstroke}{rgb}{1.000000,1.000000,1.000000}%
\pgfsetstrokecolor{currentstroke}%
\pgfsetdash{}{0pt}%
\pgfpathmoveto{\pgfqpoint{2.917483in}{2.860539in}}%
\pgfpathcurveto{\pgfqpoint{2.928534in}{2.860539in}}{\pgfqpoint{2.939133in}{2.864929in}}{\pgfqpoint{2.946946in}{2.872743in}}%
\pgfpathcurveto{\pgfqpoint{2.954760in}{2.880556in}}{\pgfqpoint{2.959150in}{2.891155in}}{\pgfqpoint{2.959150in}{2.902205in}}%
\pgfpathcurveto{\pgfqpoint{2.959150in}{2.913255in}}{\pgfqpoint{2.954760in}{2.923855in}}{\pgfqpoint{2.946946in}{2.931668in}}%
\pgfpathcurveto{\pgfqpoint{2.939133in}{2.939482in}}{\pgfqpoint{2.928534in}{2.943872in}}{\pgfqpoint{2.917483in}{2.943872in}}%
\pgfpathcurveto{\pgfqpoint{2.906433in}{2.943872in}}{\pgfqpoint{2.895834in}{2.939482in}}{\pgfqpoint{2.888021in}{2.931668in}}%
\pgfpathcurveto{\pgfqpoint{2.880207in}{2.923855in}}{\pgfqpoint{2.875817in}{2.913255in}}{\pgfqpoint{2.875817in}{2.902205in}}%
\pgfpathcurveto{\pgfqpoint{2.875817in}{2.891155in}}{\pgfqpoint{2.880207in}{2.880556in}}{\pgfqpoint{2.888021in}{2.872743in}}%
\pgfpathcurveto{\pgfqpoint{2.895834in}{2.864929in}}{\pgfqpoint{2.906433in}{2.860539in}}{\pgfqpoint{2.917483in}{2.860539in}}%
\pgfpathclose%
\pgfusepath{stroke,fill}%
\end{pgfscope}%
\begin{pgfscope}%
\pgfpathrectangle{\pgfqpoint{0.481978in}{0.331635in}}{\pgfqpoint{4.960000in}{3.696000in}}%
\pgfusepath{clip}%
\pgfsetbuttcap%
\pgfsetroundjoin%
\definecolor{currentfill}{rgb}{1.000000,0.705882,0.509804}%
\pgfsetfillcolor{currentfill}%
\pgfsetlinewidth{0.481800pt}%
\definecolor{currentstroke}{rgb}{1.000000,1.000000,1.000000}%
\pgfsetstrokecolor{currentstroke}%
\pgfsetdash{}{0pt}%
\pgfpathmoveto{\pgfqpoint{1.958529in}{3.746360in}}%
\pgfpathcurveto{\pgfqpoint{1.969579in}{3.746360in}}{\pgfqpoint{1.980178in}{3.750750in}}{\pgfqpoint{1.987992in}{3.758564in}}%
\pgfpathcurveto{\pgfqpoint{1.995806in}{3.766378in}}{\pgfqpoint{2.000196in}{3.776977in}}{\pgfqpoint{2.000196in}{3.788027in}}%
\pgfpathcurveto{\pgfqpoint{2.000196in}{3.799077in}}{\pgfqpoint{1.995806in}{3.809676in}}{\pgfqpoint{1.987992in}{3.817490in}}%
\pgfpathcurveto{\pgfqpoint{1.980178in}{3.825303in}}{\pgfqpoint{1.969579in}{3.829693in}}{\pgfqpoint{1.958529in}{3.829693in}}%
\pgfpathcurveto{\pgfqpoint{1.947479in}{3.829693in}}{\pgfqpoint{1.936880in}{3.825303in}}{\pgfqpoint{1.929067in}{3.817490in}}%
\pgfpathcurveto{\pgfqpoint{1.921253in}{3.809676in}}{\pgfqpoint{1.916863in}{3.799077in}}{\pgfqpoint{1.916863in}{3.788027in}}%
\pgfpathcurveto{\pgfqpoint{1.916863in}{3.776977in}}{\pgfqpoint{1.921253in}{3.766378in}}{\pgfqpoint{1.929067in}{3.758564in}}%
\pgfpathcurveto{\pgfqpoint{1.936880in}{3.750750in}}{\pgfqpoint{1.947479in}{3.746360in}}{\pgfqpoint{1.958529in}{3.746360in}}%
\pgfpathclose%
\pgfusepath{stroke,fill}%
\end{pgfscope}%
\begin{pgfscope}%
\pgfpathrectangle{\pgfqpoint{0.481978in}{0.331635in}}{\pgfqpoint{4.960000in}{3.696000in}}%
\pgfusepath{clip}%
\pgfsetbuttcap%
\pgfsetroundjoin%
\definecolor{currentfill}{rgb}{1.000000,0.705882,0.509804}%
\pgfsetfillcolor{currentfill}%
\pgfsetlinewidth{0.481800pt}%
\definecolor{currentstroke}{rgb}{1.000000,1.000000,1.000000}%
\pgfsetstrokecolor{currentstroke}%
\pgfsetdash{}{0pt}%
\pgfpathmoveto{\pgfqpoint{3.064570in}{1.987569in}}%
\pgfpathcurveto{\pgfqpoint{3.075620in}{1.987569in}}{\pgfqpoint{3.086219in}{1.991959in}}{\pgfqpoint{3.094033in}{1.999773in}}%
\pgfpathcurveto{\pgfqpoint{3.101846in}{2.007586in}}{\pgfqpoint{3.106237in}{2.018185in}}{\pgfqpoint{3.106237in}{2.029235in}}%
\pgfpathcurveto{\pgfqpoint{3.106237in}{2.040286in}}{\pgfqpoint{3.101846in}{2.050885in}}{\pgfqpoint{3.094033in}{2.058698in}}%
\pgfpathcurveto{\pgfqpoint{3.086219in}{2.066512in}}{\pgfqpoint{3.075620in}{2.070902in}}{\pgfqpoint{3.064570in}{2.070902in}}%
\pgfpathcurveto{\pgfqpoint{3.053520in}{2.070902in}}{\pgfqpoint{3.042921in}{2.066512in}}{\pgfqpoint{3.035107in}{2.058698in}}%
\pgfpathcurveto{\pgfqpoint{3.027294in}{2.050885in}}{\pgfqpoint{3.022903in}{2.040286in}}{\pgfqpoint{3.022903in}{2.029235in}}%
\pgfpathcurveto{\pgfqpoint{3.022903in}{2.018185in}}{\pgfqpoint{3.027294in}{2.007586in}}{\pgfqpoint{3.035107in}{1.999773in}}%
\pgfpathcurveto{\pgfqpoint{3.042921in}{1.991959in}}{\pgfqpoint{3.053520in}{1.987569in}}{\pgfqpoint{3.064570in}{1.987569in}}%
\pgfpathclose%
\pgfusepath{stroke,fill}%
\end{pgfscope}%
\begin{pgfscope}%
\pgfpathrectangle{\pgfqpoint{0.481978in}{0.331635in}}{\pgfqpoint{4.960000in}{3.696000in}}%
\pgfusepath{clip}%
\pgfsetbuttcap%
\pgfsetroundjoin%
\definecolor{currentfill}{rgb}{1.000000,0.705882,0.509804}%
\pgfsetfillcolor{currentfill}%
\pgfsetlinewidth{0.481800pt}%
\definecolor{currentstroke}{rgb}{1.000000,1.000000,1.000000}%
\pgfsetstrokecolor{currentstroke}%
\pgfsetdash{}{0pt}%
\pgfpathmoveto{\pgfqpoint{2.863689in}{2.294514in}}%
\pgfpathcurveto{\pgfqpoint{2.874739in}{2.294514in}}{\pgfqpoint{2.885338in}{2.298904in}}{\pgfqpoint{2.893152in}{2.306718in}}%
\pgfpathcurveto{\pgfqpoint{2.900965in}{2.314531in}}{\pgfqpoint{2.905356in}{2.325130in}}{\pgfqpoint{2.905356in}{2.336180in}}%
\pgfpathcurveto{\pgfqpoint{2.905356in}{2.347231in}}{\pgfqpoint{2.900965in}{2.357830in}}{\pgfqpoint{2.893152in}{2.365643in}}%
\pgfpathcurveto{\pgfqpoint{2.885338in}{2.373457in}}{\pgfqpoint{2.874739in}{2.377847in}}{\pgfqpoint{2.863689in}{2.377847in}}%
\pgfpathcurveto{\pgfqpoint{2.852639in}{2.377847in}}{\pgfqpoint{2.842040in}{2.373457in}}{\pgfqpoint{2.834226in}{2.365643in}}%
\pgfpathcurveto{\pgfqpoint{2.826413in}{2.357830in}}{\pgfqpoint{2.822022in}{2.347231in}}{\pgfqpoint{2.822022in}{2.336180in}}%
\pgfpathcurveto{\pgfqpoint{2.822022in}{2.325130in}}{\pgfqpoint{2.826413in}{2.314531in}}{\pgfqpoint{2.834226in}{2.306718in}}%
\pgfpathcurveto{\pgfqpoint{2.842040in}{2.298904in}}{\pgfqpoint{2.852639in}{2.294514in}}{\pgfqpoint{2.863689in}{2.294514in}}%
\pgfpathclose%
\pgfusepath{stroke,fill}%
\end{pgfscope}%
\begin{pgfscope}%
\pgfpathrectangle{\pgfqpoint{0.481978in}{0.331635in}}{\pgfqpoint{4.960000in}{3.696000in}}%
\pgfusepath{clip}%
\pgfsetbuttcap%
\pgfsetroundjoin%
\definecolor{currentfill}{rgb}{1.000000,0.705882,0.509804}%
\pgfsetfillcolor{currentfill}%
\pgfsetlinewidth{0.481800pt}%
\definecolor{currentstroke}{rgb}{1.000000,1.000000,1.000000}%
\pgfsetstrokecolor{currentstroke}%
\pgfsetdash{}{0pt}%
\pgfpathmoveto{\pgfqpoint{1.553667in}{2.031264in}}%
\pgfpathcurveto{\pgfqpoint{1.564717in}{2.031264in}}{\pgfqpoint{1.575316in}{2.035654in}}{\pgfqpoint{1.583129in}{2.043468in}}%
\pgfpathcurveto{\pgfqpoint{1.590943in}{2.051281in}}{\pgfqpoint{1.595333in}{2.061880in}}{\pgfqpoint{1.595333in}{2.072930in}}%
\pgfpathcurveto{\pgfqpoint{1.595333in}{2.083980in}}{\pgfqpoint{1.590943in}{2.094579in}}{\pgfqpoint{1.583129in}{2.102393in}}%
\pgfpathcurveto{\pgfqpoint{1.575316in}{2.110207in}}{\pgfqpoint{1.564717in}{2.114597in}}{\pgfqpoint{1.553667in}{2.114597in}}%
\pgfpathcurveto{\pgfqpoint{1.542617in}{2.114597in}}{\pgfqpoint{1.532018in}{2.110207in}}{\pgfqpoint{1.524204in}{2.102393in}}%
\pgfpathcurveto{\pgfqpoint{1.516390in}{2.094579in}}{\pgfqpoint{1.512000in}{2.083980in}}{\pgfqpoint{1.512000in}{2.072930in}}%
\pgfpathcurveto{\pgfqpoint{1.512000in}{2.061880in}}{\pgfqpoint{1.516390in}{2.051281in}}{\pgfqpoint{1.524204in}{2.043468in}}%
\pgfpathcurveto{\pgfqpoint{1.532018in}{2.035654in}}{\pgfqpoint{1.542617in}{2.031264in}}{\pgfqpoint{1.553667in}{2.031264in}}%
\pgfpathclose%
\pgfusepath{stroke,fill}%
\end{pgfscope}%
\begin{pgfscope}%
\pgfpathrectangle{\pgfqpoint{0.481978in}{0.331635in}}{\pgfqpoint{4.960000in}{3.696000in}}%
\pgfusepath{clip}%
\pgfsetbuttcap%
\pgfsetroundjoin%
\definecolor{currentfill}{rgb}{1.000000,0.705882,0.509804}%
\pgfsetfillcolor{currentfill}%
\pgfsetlinewidth{0.481800pt}%
\definecolor{currentstroke}{rgb}{1.000000,1.000000,1.000000}%
\pgfsetstrokecolor{currentstroke}%
\pgfsetdash{}{0pt}%
\pgfpathmoveto{\pgfqpoint{0.967300in}{2.685380in}}%
\pgfpathcurveto{\pgfqpoint{0.978350in}{2.685380in}}{\pgfqpoint{0.988949in}{2.689770in}}{\pgfqpoint{0.996762in}{2.697584in}}%
\pgfpathcurveto{\pgfqpoint{1.004576in}{2.705397in}}{\pgfqpoint{1.008966in}{2.715996in}}{\pgfqpoint{1.008966in}{2.727046in}}%
\pgfpathcurveto{\pgfqpoint{1.008966in}{2.738096in}}{\pgfqpoint{1.004576in}{2.748696in}}{\pgfqpoint{0.996762in}{2.756509in}}%
\pgfpathcurveto{\pgfqpoint{0.988949in}{2.764323in}}{\pgfqpoint{0.978350in}{2.768713in}}{\pgfqpoint{0.967300in}{2.768713in}}%
\pgfpathcurveto{\pgfqpoint{0.956249in}{2.768713in}}{\pgfqpoint{0.945650in}{2.764323in}}{\pgfqpoint{0.937837in}{2.756509in}}%
\pgfpathcurveto{\pgfqpoint{0.930023in}{2.748696in}}{\pgfqpoint{0.925633in}{2.738096in}}{\pgfqpoint{0.925633in}{2.727046in}}%
\pgfpathcurveto{\pgfqpoint{0.925633in}{2.715996in}}{\pgfqpoint{0.930023in}{2.705397in}}{\pgfqpoint{0.937837in}{2.697584in}}%
\pgfpathcurveto{\pgfqpoint{0.945650in}{2.689770in}}{\pgfqpoint{0.956249in}{2.685380in}}{\pgfqpoint{0.967300in}{2.685380in}}%
\pgfpathclose%
\pgfusepath{stroke,fill}%
\end{pgfscope}%
\begin{pgfscope}%
\pgfpathrectangle{\pgfqpoint{0.481978in}{0.331635in}}{\pgfqpoint{4.960000in}{3.696000in}}%
\pgfusepath{clip}%
\pgfsetbuttcap%
\pgfsetroundjoin%
\definecolor{currentfill}{rgb}{1.000000,0.705882,0.509804}%
\pgfsetfillcolor{currentfill}%
\pgfsetlinewidth{0.481800pt}%
\definecolor{currentstroke}{rgb}{1.000000,1.000000,1.000000}%
\pgfsetstrokecolor{currentstroke}%
\pgfsetdash{}{0pt}%
\pgfpathmoveto{\pgfqpoint{3.183693in}{2.811174in}}%
\pgfpathcurveto{\pgfqpoint{3.194743in}{2.811174in}}{\pgfqpoint{3.205342in}{2.815564in}}{\pgfqpoint{3.213155in}{2.823378in}}%
\pgfpathcurveto{\pgfqpoint{3.220969in}{2.831192in}}{\pgfqpoint{3.225359in}{2.841791in}}{\pgfqpoint{3.225359in}{2.852841in}}%
\pgfpathcurveto{\pgfqpoint{3.225359in}{2.863891in}}{\pgfqpoint{3.220969in}{2.874490in}}{\pgfqpoint{3.213155in}{2.882304in}}%
\pgfpathcurveto{\pgfqpoint{3.205342in}{2.890117in}}{\pgfqpoint{3.194743in}{2.894507in}}{\pgfqpoint{3.183693in}{2.894507in}}%
\pgfpathcurveto{\pgfqpoint{3.172642in}{2.894507in}}{\pgfqpoint{3.162043in}{2.890117in}}{\pgfqpoint{3.154230in}{2.882304in}}%
\pgfpathcurveto{\pgfqpoint{3.146416in}{2.874490in}}{\pgfqpoint{3.142026in}{2.863891in}}{\pgfqpoint{3.142026in}{2.852841in}}%
\pgfpathcurveto{\pgfqpoint{3.142026in}{2.841791in}}{\pgfqpoint{3.146416in}{2.831192in}}{\pgfqpoint{3.154230in}{2.823378in}}%
\pgfpathcurveto{\pgfqpoint{3.162043in}{2.815564in}}{\pgfqpoint{3.172642in}{2.811174in}}{\pgfqpoint{3.183693in}{2.811174in}}%
\pgfpathclose%
\pgfusepath{stroke,fill}%
\end{pgfscope}%
\begin{pgfscope}%
\pgfpathrectangle{\pgfqpoint{0.481978in}{0.331635in}}{\pgfqpoint{4.960000in}{3.696000in}}%
\pgfusepath{clip}%
\pgfsetbuttcap%
\pgfsetroundjoin%
\definecolor{currentfill}{rgb}{1.000000,0.705882,0.509804}%
\pgfsetfillcolor{currentfill}%
\pgfsetlinewidth{0.481800pt}%
\definecolor{currentstroke}{rgb}{1.000000,1.000000,1.000000}%
\pgfsetstrokecolor{currentstroke}%
\pgfsetdash{}{0pt}%
\pgfpathmoveto{\pgfqpoint{1.396213in}{1.533246in}}%
\pgfpathcurveto{\pgfqpoint{1.407263in}{1.533246in}}{\pgfqpoint{1.417862in}{1.537636in}}{\pgfqpoint{1.425675in}{1.545450in}}%
\pgfpathcurveto{\pgfqpoint{1.433489in}{1.553263in}}{\pgfqpoint{1.437879in}{1.563862in}}{\pgfqpoint{1.437879in}{1.574913in}}%
\pgfpathcurveto{\pgfqpoint{1.437879in}{1.585963in}}{\pgfqpoint{1.433489in}{1.596562in}}{\pgfqpoint{1.425675in}{1.604375in}}%
\pgfpathcurveto{\pgfqpoint{1.417862in}{1.612189in}}{\pgfqpoint{1.407263in}{1.616579in}}{\pgfqpoint{1.396213in}{1.616579in}}%
\pgfpathcurveto{\pgfqpoint{1.385162in}{1.616579in}}{\pgfqpoint{1.374563in}{1.612189in}}{\pgfqpoint{1.366750in}{1.604375in}}%
\pgfpathcurveto{\pgfqpoint{1.358936in}{1.596562in}}{\pgfqpoint{1.354546in}{1.585963in}}{\pgfqpoint{1.354546in}{1.574913in}}%
\pgfpathcurveto{\pgfqpoint{1.354546in}{1.563862in}}{\pgfqpoint{1.358936in}{1.553263in}}{\pgfqpoint{1.366750in}{1.545450in}}%
\pgfpathcurveto{\pgfqpoint{1.374563in}{1.537636in}}{\pgfqpoint{1.385162in}{1.533246in}}{\pgfqpoint{1.396213in}{1.533246in}}%
\pgfpathclose%
\pgfusepath{stroke,fill}%
\end{pgfscope}%
\begin{pgfscope}%
\pgfpathrectangle{\pgfqpoint{0.481978in}{0.331635in}}{\pgfqpoint{4.960000in}{3.696000in}}%
\pgfusepath{clip}%
\pgfsetbuttcap%
\pgfsetroundjoin%
\definecolor{currentfill}{rgb}{1.000000,0.705882,0.509804}%
\pgfsetfillcolor{currentfill}%
\pgfsetlinewidth{0.481800pt}%
\definecolor{currentstroke}{rgb}{1.000000,1.000000,1.000000}%
\pgfsetstrokecolor{currentstroke}%
\pgfsetdash{}{0pt}%
\pgfpathmoveto{\pgfqpoint{1.259005in}{2.310098in}}%
\pgfpathcurveto{\pgfqpoint{1.270055in}{2.310098in}}{\pgfqpoint{1.280654in}{2.314488in}}{\pgfqpoint{1.288468in}{2.322301in}}%
\pgfpathcurveto{\pgfqpoint{1.296281in}{2.330115in}}{\pgfqpoint{1.300672in}{2.340714in}}{\pgfqpoint{1.300672in}{2.351764in}}%
\pgfpathcurveto{\pgfqpoint{1.300672in}{2.362814in}}{\pgfqpoint{1.296281in}{2.373413in}}{\pgfqpoint{1.288468in}{2.381227in}}%
\pgfpathcurveto{\pgfqpoint{1.280654in}{2.389041in}}{\pgfqpoint{1.270055in}{2.393431in}}{\pgfqpoint{1.259005in}{2.393431in}}%
\pgfpathcurveto{\pgfqpoint{1.247955in}{2.393431in}}{\pgfqpoint{1.237356in}{2.389041in}}{\pgfqpoint{1.229542in}{2.381227in}}%
\pgfpathcurveto{\pgfqpoint{1.221729in}{2.373413in}}{\pgfqpoint{1.217338in}{2.362814in}}{\pgfqpoint{1.217338in}{2.351764in}}%
\pgfpathcurveto{\pgfqpoint{1.217338in}{2.340714in}}{\pgfqpoint{1.221729in}{2.330115in}}{\pgfqpoint{1.229542in}{2.322301in}}%
\pgfpathcurveto{\pgfqpoint{1.237356in}{2.314488in}}{\pgfqpoint{1.247955in}{2.310098in}}{\pgfqpoint{1.259005in}{2.310098in}}%
\pgfpathclose%
\pgfusepath{stroke,fill}%
\end{pgfscope}%
\begin{pgfscope}%
\pgfpathrectangle{\pgfqpoint{0.481978in}{0.331635in}}{\pgfqpoint{4.960000in}{3.696000in}}%
\pgfusepath{clip}%
\pgfsetbuttcap%
\pgfsetroundjoin%
\definecolor{currentfill}{rgb}{1.000000,0.705882,0.509804}%
\pgfsetfillcolor{currentfill}%
\pgfsetlinewidth{0.481800pt}%
\definecolor{currentstroke}{rgb}{1.000000,1.000000,1.000000}%
\pgfsetstrokecolor{currentstroke}%
\pgfsetdash{}{0pt}%
\pgfpathmoveto{\pgfqpoint{1.303487in}{2.087407in}}%
\pgfpathcurveto{\pgfqpoint{1.314537in}{2.087407in}}{\pgfqpoint{1.325136in}{2.091798in}}{\pgfqpoint{1.332950in}{2.099611in}}%
\pgfpathcurveto{\pgfqpoint{1.340763in}{2.107425in}}{\pgfqpoint{1.345154in}{2.118024in}}{\pgfqpoint{1.345154in}{2.129074in}}%
\pgfpathcurveto{\pgfqpoint{1.345154in}{2.140124in}}{\pgfqpoint{1.340763in}{2.150723in}}{\pgfqpoint{1.332950in}{2.158537in}}%
\pgfpathcurveto{\pgfqpoint{1.325136in}{2.166350in}}{\pgfqpoint{1.314537in}{2.170741in}}{\pgfqpoint{1.303487in}{2.170741in}}%
\pgfpathcurveto{\pgfqpoint{1.292437in}{2.170741in}}{\pgfqpoint{1.281838in}{2.166350in}}{\pgfqpoint{1.274024in}{2.158537in}}%
\pgfpathcurveto{\pgfqpoint{1.266211in}{2.150723in}}{\pgfqpoint{1.261820in}{2.140124in}}{\pgfqpoint{1.261820in}{2.129074in}}%
\pgfpathcurveto{\pgfqpoint{1.261820in}{2.118024in}}{\pgfqpoint{1.266211in}{2.107425in}}{\pgfqpoint{1.274024in}{2.099611in}}%
\pgfpathcurveto{\pgfqpoint{1.281838in}{2.091798in}}{\pgfqpoint{1.292437in}{2.087407in}}{\pgfqpoint{1.303487in}{2.087407in}}%
\pgfpathclose%
\pgfusepath{stroke,fill}%
\end{pgfscope}%
\begin{pgfscope}%
\pgfpathrectangle{\pgfqpoint{0.481978in}{0.331635in}}{\pgfqpoint{4.960000in}{3.696000in}}%
\pgfusepath{clip}%
\pgfsetbuttcap%
\pgfsetroundjoin%
\definecolor{currentfill}{rgb}{1.000000,0.705882,0.509804}%
\pgfsetfillcolor{currentfill}%
\pgfsetlinewidth{0.481800pt}%
\definecolor{currentstroke}{rgb}{1.000000,1.000000,1.000000}%
\pgfsetstrokecolor{currentstroke}%
\pgfsetdash{}{0pt}%
\pgfpathmoveto{\pgfqpoint{2.177859in}{2.904327in}}%
\pgfpathcurveto{\pgfqpoint{2.188909in}{2.904327in}}{\pgfqpoint{2.199508in}{2.908717in}}{\pgfqpoint{2.207322in}{2.916531in}}%
\pgfpathcurveto{\pgfqpoint{2.215136in}{2.924344in}}{\pgfqpoint{2.219526in}{2.934943in}}{\pgfqpoint{2.219526in}{2.945994in}}%
\pgfpathcurveto{\pgfqpoint{2.219526in}{2.957044in}}{\pgfqpoint{2.215136in}{2.967643in}}{\pgfqpoint{2.207322in}{2.975456in}}%
\pgfpathcurveto{\pgfqpoint{2.199508in}{2.983270in}}{\pgfqpoint{2.188909in}{2.987660in}}{\pgfqpoint{2.177859in}{2.987660in}}%
\pgfpathcurveto{\pgfqpoint{2.166809in}{2.987660in}}{\pgfqpoint{2.156210in}{2.983270in}}{\pgfqpoint{2.148396in}{2.975456in}}%
\pgfpathcurveto{\pgfqpoint{2.140583in}{2.967643in}}{\pgfqpoint{2.136193in}{2.957044in}}{\pgfqpoint{2.136193in}{2.945994in}}%
\pgfpathcurveto{\pgfqpoint{2.136193in}{2.934943in}}{\pgfqpoint{2.140583in}{2.924344in}}{\pgfqpoint{2.148396in}{2.916531in}}%
\pgfpathcurveto{\pgfqpoint{2.156210in}{2.908717in}}{\pgfqpoint{2.166809in}{2.904327in}}{\pgfqpoint{2.177859in}{2.904327in}}%
\pgfpathclose%
\pgfusepath{stroke,fill}%
\end{pgfscope}%
\begin{pgfscope}%
\pgfpathrectangle{\pgfqpoint{0.481978in}{0.331635in}}{\pgfqpoint{4.960000in}{3.696000in}}%
\pgfusepath{clip}%
\pgfsetbuttcap%
\pgfsetroundjoin%
\definecolor{currentfill}{rgb}{1.000000,0.705882,0.509804}%
\pgfsetfillcolor{currentfill}%
\pgfsetlinewidth{0.481800pt}%
\definecolor{currentstroke}{rgb}{1.000000,1.000000,1.000000}%
\pgfsetstrokecolor{currentstroke}%
\pgfsetdash{}{0pt}%
\pgfpathmoveto{\pgfqpoint{1.960427in}{2.079203in}}%
\pgfpathcurveto{\pgfqpoint{1.971477in}{2.079203in}}{\pgfqpoint{1.982076in}{2.083593in}}{\pgfqpoint{1.989889in}{2.091407in}}%
\pgfpathcurveto{\pgfqpoint{1.997703in}{2.099220in}}{\pgfqpoint{2.002093in}{2.109819in}}{\pgfqpoint{2.002093in}{2.120870in}}%
\pgfpathcurveto{\pgfqpoint{2.002093in}{2.131920in}}{\pgfqpoint{1.997703in}{2.142519in}}{\pgfqpoint{1.989889in}{2.150332in}}%
\pgfpathcurveto{\pgfqpoint{1.982076in}{2.158146in}}{\pgfqpoint{1.971477in}{2.162536in}}{\pgfqpoint{1.960427in}{2.162536in}}%
\pgfpathcurveto{\pgfqpoint{1.949377in}{2.162536in}}{\pgfqpoint{1.938777in}{2.158146in}}{\pgfqpoint{1.930964in}{2.150332in}}%
\pgfpathcurveto{\pgfqpoint{1.923150in}{2.142519in}}{\pgfqpoint{1.918760in}{2.131920in}}{\pgfqpoint{1.918760in}{2.120870in}}%
\pgfpathcurveto{\pgfqpoint{1.918760in}{2.109819in}}{\pgfqpoint{1.923150in}{2.099220in}}{\pgfqpoint{1.930964in}{2.091407in}}%
\pgfpathcurveto{\pgfqpoint{1.938777in}{2.083593in}}{\pgfqpoint{1.949377in}{2.079203in}}{\pgfqpoint{1.960427in}{2.079203in}}%
\pgfpathclose%
\pgfusepath{stroke,fill}%
\end{pgfscope}%
\begin{pgfscope}%
\pgfpathrectangle{\pgfqpoint{0.481978in}{0.331635in}}{\pgfqpoint{4.960000in}{3.696000in}}%
\pgfusepath{clip}%
\pgfsetbuttcap%
\pgfsetroundjoin%
\definecolor{currentfill}{rgb}{1.000000,0.705882,0.509804}%
\pgfsetfillcolor{currentfill}%
\pgfsetlinewidth{0.481800pt}%
\definecolor{currentstroke}{rgb}{1.000000,1.000000,1.000000}%
\pgfsetstrokecolor{currentstroke}%
\pgfsetdash{}{0pt}%
\pgfpathmoveto{\pgfqpoint{1.507051in}{3.235251in}}%
\pgfpathcurveto{\pgfqpoint{1.518102in}{3.235251in}}{\pgfqpoint{1.528701in}{3.239641in}}{\pgfqpoint{1.536514in}{3.247455in}}%
\pgfpathcurveto{\pgfqpoint{1.544328in}{3.255268in}}{\pgfqpoint{1.548718in}{3.265867in}}{\pgfqpoint{1.548718in}{3.276917in}}%
\pgfpathcurveto{\pgfqpoint{1.548718in}{3.287967in}}{\pgfqpoint{1.544328in}{3.298567in}}{\pgfqpoint{1.536514in}{3.306380in}}%
\pgfpathcurveto{\pgfqpoint{1.528701in}{3.314194in}}{\pgfqpoint{1.518102in}{3.318584in}}{\pgfqpoint{1.507051in}{3.318584in}}%
\pgfpathcurveto{\pgfqpoint{1.496001in}{3.318584in}}{\pgfqpoint{1.485402in}{3.314194in}}{\pgfqpoint{1.477589in}{3.306380in}}%
\pgfpathcurveto{\pgfqpoint{1.469775in}{3.298567in}}{\pgfqpoint{1.465385in}{3.287967in}}{\pgfqpoint{1.465385in}{3.276917in}}%
\pgfpathcurveto{\pgfqpoint{1.465385in}{3.265867in}}{\pgfqpoint{1.469775in}{3.255268in}}{\pgfqpoint{1.477589in}{3.247455in}}%
\pgfpathcurveto{\pgfqpoint{1.485402in}{3.239641in}}{\pgfqpoint{1.496001in}{3.235251in}}{\pgfqpoint{1.507051in}{3.235251in}}%
\pgfpathclose%
\pgfusepath{stroke,fill}%
\end{pgfscope}%
\begin{pgfscope}%
\pgfpathrectangle{\pgfqpoint{0.481978in}{0.331635in}}{\pgfqpoint{4.960000in}{3.696000in}}%
\pgfusepath{clip}%
\pgfsetbuttcap%
\pgfsetroundjoin%
\definecolor{currentfill}{rgb}{1.000000,0.705882,0.509804}%
\pgfsetfillcolor{currentfill}%
\pgfsetlinewidth{0.481800pt}%
\definecolor{currentstroke}{rgb}{1.000000,1.000000,1.000000}%
\pgfsetstrokecolor{currentstroke}%
\pgfsetdash{}{0pt}%
\pgfpathmoveto{\pgfqpoint{1.882966in}{2.792050in}}%
\pgfpathcurveto{\pgfqpoint{1.894016in}{2.792050in}}{\pgfqpoint{1.904615in}{2.796440in}}{\pgfqpoint{1.912428in}{2.804254in}}%
\pgfpathcurveto{\pgfqpoint{1.920242in}{2.812067in}}{\pgfqpoint{1.924632in}{2.822666in}}{\pgfqpoint{1.924632in}{2.833717in}}%
\pgfpathcurveto{\pgfqpoint{1.924632in}{2.844767in}}{\pgfqpoint{1.920242in}{2.855366in}}{\pgfqpoint{1.912428in}{2.863179in}}%
\pgfpathcurveto{\pgfqpoint{1.904615in}{2.870993in}}{\pgfqpoint{1.894016in}{2.875383in}}{\pgfqpoint{1.882966in}{2.875383in}}%
\pgfpathcurveto{\pgfqpoint{1.871915in}{2.875383in}}{\pgfqpoint{1.861316in}{2.870993in}}{\pgfqpoint{1.853503in}{2.863179in}}%
\pgfpathcurveto{\pgfqpoint{1.845689in}{2.855366in}}{\pgfqpoint{1.841299in}{2.844767in}}{\pgfqpoint{1.841299in}{2.833717in}}%
\pgfpathcurveto{\pgfqpoint{1.841299in}{2.822666in}}{\pgfqpoint{1.845689in}{2.812067in}}{\pgfqpoint{1.853503in}{2.804254in}}%
\pgfpathcurveto{\pgfqpoint{1.861316in}{2.796440in}}{\pgfqpoint{1.871915in}{2.792050in}}{\pgfqpoint{1.882966in}{2.792050in}}%
\pgfpathclose%
\pgfusepath{stroke,fill}%
\end{pgfscope}%
\begin{pgfscope}%
\pgfpathrectangle{\pgfqpoint{0.481978in}{0.331635in}}{\pgfqpoint{4.960000in}{3.696000in}}%
\pgfusepath{clip}%
\pgfsetbuttcap%
\pgfsetroundjoin%
\definecolor{currentfill}{rgb}{1.000000,0.705882,0.509804}%
\pgfsetfillcolor{currentfill}%
\pgfsetlinewidth{0.481800pt}%
\definecolor{currentstroke}{rgb}{1.000000,1.000000,1.000000}%
\pgfsetstrokecolor{currentstroke}%
\pgfsetdash{}{0pt}%
\pgfpathmoveto{\pgfqpoint{1.444969in}{2.606947in}}%
\pgfpathcurveto{\pgfqpoint{1.456019in}{2.606947in}}{\pgfqpoint{1.466618in}{2.611337in}}{\pgfqpoint{1.474431in}{2.619151in}}%
\pgfpathcurveto{\pgfqpoint{1.482245in}{2.626964in}}{\pgfqpoint{1.486635in}{2.637563in}}{\pgfqpoint{1.486635in}{2.648614in}}%
\pgfpathcurveto{\pgfqpoint{1.486635in}{2.659664in}}{\pgfqpoint{1.482245in}{2.670263in}}{\pgfqpoint{1.474431in}{2.678076in}}%
\pgfpathcurveto{\pgfqpoint{1.466618in}{2.685890in}}{\pgfqpoint{1.456019in}{2.690280in}}{\pgfqpoint{1.444969in}{2.690280in}}%
\pgfpathcurveto{\pgfqpoint{1.433919in}{2.690280in}}{\pgfqpoint{1.423320in}{2.685890in}}{\pgfqpoint{1.415506in}{2.678076in}}%
\pgfpathcurveto{\pgfqpoint{1.407692in}{2.670263in}}{\pgfqpoint{1.403302in}{2.659664in}}{\pgfqpoint{1.403302in}{2.648614in}}%
\pgfpathcurveto{\pgfqpoint{1.403302in}{2.637563in}}{\pgfqpoint{1.407692in}{2.626964in}}{\pgfqpoint{1.415506in}{2.619151in}}%
\pgfpathcurveto{\pgfqpoint{1.423320in}{2.611337in}}{\pgfqpoint{1.433919in}{2.606947in}}{\pgfqpoint{1.444969in}{2.606947in}}%
\pgfpathclose%
\pgfusepath{stroke,fill}%
\end{pgfscope}%
\begin{pgfscope}%
\pgfpathrectangle{\pgfqpoint{0.481978in}{0.331635in}}{\pgfqpoint{4.960000in}{3.696000in}}%
\pgfusepath{clip}%
\pgfsetbuttcap%
\pgfsetroundjoin%
\definecolor{currentfill}{rgb}{1.000000,0.705882,0.509804}%
\pgfsetfillcolor{currentfill}%
\pgfsetlinewidth{0.481800pt}%
\definecolor{currentstroke}{rgb}{1.000000,1.000000,1.000000}%
\pgfsetstrokecolor{currentstroke}%
\pgfsetdash{}{0pt}%
\pgfpathmoveto{\pgfqpoint{3.242283in}{2.289266in}}%
\pgfpathcurveto{\pgfqpoint{3.253333in}{2.289266in}}{\pgfqpoint{3.263932in}{2.293656in}}{\pgfqpoint{3.271746in}{2.301470in}}%
\pgfpathcurveto{\pgfqpoint{3.279560in}{2.309283in}}{\pgfqpoint{3.283950in}{2.319882in}}{\pgfqpoint{3.283950in}{2.330932in}}%
\pgfpathcurveto{\pgfqpoint{3.283950in}{2.341983in}}{\pgfqpoint{3.279560in}{2.352582in}}{\pgfqpoint{3.271746in}{2.360395in}}%
\pgfpathcurveto{\pgfqpoint{3.263932in}{2.368209in}}{\pgfqpoint{3.253333in}{2.372599in}}{\pgfqpoint{3.242283in}{2.372599in}}%
\pgfpathcurveto{\pgfqpoint{3.231233in}{2.372599in}}{\pgfqpoint{3.220634in}{2.368209in}}{\pgfqpoint{3.212820in}{2.360395in}}%
\pgfpathcurveto{\pgfqpoint{3.205007in}{2.352582in}}{\pgfqpoint{3.200616in}{2.341983in}}{\pgfqpoint{3.200616in}{2.330932in}}%
\pgfpathcurveto{\pgfqpoint{3.200616in}{2.319882in}}{\pgfqpoint{3.205007in}{2.309283in}}{\pgfqpoint{3.212820in}{2.301470in}}%
\pgfpathcurveto{\pgfqpoint{3.220634in}{2.293656in}}{\pgfqpoint{3.231233in}{2.289266in}}{\pgfqpoint{3.242283in}{2.289266in}}%
\pgfpathclose%
\pgfusepath{stroke,fill}%
\end{pgfscope}%
\begin{pgfscope}%
\pgfpathrectangle{\pgfqpoint{0.481978in}{0.331635in}}{\pgfqpoint{4.960000in}{3.696000in}}%
\pgfusepath{clip}%
\pgfsetbuttcap%
\pgfsetroundjoin%
\definecolor{currentfill}{rgb}{1.000000,0.705882,0.509804}%
\pgfsetfillcolor{currentfill}%
\pgfsetlinewidth{0.481800pt}%
\definecolor{currentstroke}{rgb}{1.000000,1.000000,1.000000}%
\pgfsetstrokecolor{currentstroke}%
\pgfsetdash{}{0pt}%
\pgfpathmoveto{\pgfqpoint{2.875424in}{2.222340in}}%
\pgfpathcurveto{\pgfqpoint{2.886474in}{2.222340in}}{\pgfqpoint{2.897073in}{2.226730in}}{\pgfqpoint{2.904887in}{2.234544in}}%
\pgfpathcurveto{\pgfqpoint{2.912700in}{2.242357in}}{\pgfqpoint{2.917091in}{2.252956in}}{\pgfqpoint{2.917091in}{2.264007in}}%
\pgfpathcurveto{\pgfqpoint{2.917091in}{2.275057in}}{\pgfqpoint{2.912700in}{2.285656in}}{\pgfqpoint{2.904887in}{2.293469in}}%
\pgfpathcurveto{\pgfqpoint{2.897073in}{2.301283in}}{\pgfqpoint{2.886474in}{2.305673in}}{\pgfqpoint{2.875424in}{2.305673in}}%
\pgfpathcurveto{\pgfqpoint{2.864374in}{2.305673in}}{\pgfqpoint{2.853775in}{2.301283in}}{\pgfqpoint{2.845961in}{2.293469in}}%
\pgfpathcurveto{\pgfqpoint{2.838147in}{2.285656in}}{\pgfqpoint{2.833757in}{2.275057in}}{\pgfqpoint{2.833757in}{2.264007in}}%
\pgfpathcurveto{\pgfqpoint{2.833757in}{2.252956in}}{\pgfqpoint{2.838147in}{2.242357in}}{\pgfqpoint{2.845961in}{2.234544in}}%
\pgfpathcurveto{\pgfqpoint{2.853775in}{2.226730in}}{\pgfqpoint{2.864374in}{2.222340in}}{\pgfqpoint{2.875424in}{2.222340in}}%
\pgfpathclose%
\pgfusepath{stroke,fill}%
\end{pgfscope}%
\begin{pgfscope}%
\pgfpathrectangle{\pgfqpoint{0.481978in}{0.331635in}}{\pgfqpoint{4.960000in}{3.696000in}}%
\pgfusepath{clip}%
\pgfsetbuttcap%
\pgfsetroundjoin%
\definecolor{currentfill}{rgb}{1.000000,0.705882,0.509804}%
\pgfsetfillcolor{currentfill}%
\pgfsetlinewidth{0.481800pt}%
\definecolor{currentstroke}{rgb}{1.000000,1.000000,1.000000}%
\pgfsetstrokecolor{currentstroke}%
\pgfsetdash{}{0pt}%
\pgfpathmoveto{\pgfqpoint{4.924148in}{2.896114in}}%
\pgfpathcurveto{\pgfqpoint{4.935198in}{2.896114in}}{\pgfqpoint{4.945798in}{2.900504in}}{\pgfqpoint{4.953611in}{2.908318in}}%
\pgfpathcurveto{\pgfqpoint{4.961425in}{2.916132in}}{\pgfqpoint{4.965815in}{2.926731in}}{\pgfqpoint{4.965815in}{2.937781in}}%
\pgfpathcurveto{\pgfqpoint{4.965815in}{2.948831in}}{\pgfqpoint{4.961425in}{2.959430in}}{\pgfqpoint{4.953611in}{2.967244in}}%
\pgfpathcurveto{\pgfqpoint{4.945798in}{2.975057in}}{\pgfqpoint{4.935198in}{2.979447in}}{\pgfqpoint{4.924148in}{2.979447in}}%
\pgfpathcurveto{\pgfqpoint{4.913098in}{2.979447in}}{\pgfqpoint{4.902499in}{2.975057in}}{\pgfqpoint{4.894686in}{2.967244in}}%
\pgfpathcurveto{\pgfqpoint{4.886872in}{2.959430in}}{\pgfqpoint{4.882482in}{2.948831in}}{\pgfqpoint{4.882482in}{2.937781in}}%
\pgfpathcurveto{\pgfqpoint{4.882482in}{2.926731in}}{\pgfqpoint{4.886872in}{2.916132in}}{\pgfqpoint{4.894686in}{2.908318in}}%
\pgfpathcurveto{\pgfqpoint{4.902499in}{2.900504in}}{\pgfqpoint{4.913098in}{2.896114in}}{\pgfqpoint{4.924148in}{2.896114in}}%
\pgfpathclose%
\pgfusepath{stroke,fill}%
\end{pgfscope}%
\begin{pgfscope}%
\pgfpathrectangle{\pgfqpoint{0.481978in}{0.331635in}}{\pgfqpoint{4.960000in}{3.696000in}}%
\pgfusepath{clip}%
\pgfsetbuttcap%
\pgfsetroundjoin%
\definecolor{currentfill}{rgb}{1.000000,0.705882,0.509804}%
\pgfsetfillcolor{currentfill}%
\pgfsetlinewidth{0.481800pt}%
\definecolor{currentstroke}{rgb}{1.000000,1.000000,1.000000}%
\pgfsetstrokecolor{currentstroke}%
\pgfsetdash{}{0pt}%
\pgfpathmoveto{\pgfqpoint{1.214870in}{2.290083in}}%
\pgfpathcurveto{\pgfqpoint{1.225921in}{2.290083in}}{\pgfqpoint{1.236520in}{2.294473in}}{\pgfqpoint{1.244333in}{2.302287in}}%
\pgfpathcurveto{\pgfqpoint{1.252147in}{2.310101in}}{\pgfqpoint{1.256537in}{2.320700in}}{\pgfqpoint{1.256537in}{2.331750in}}%
\pgfpathcurveto{\pgfqpoint{1.256537in}{2.342800in}}{\pgfqpoint{1.252147in}{2.353399in}}{\pgfqpoint{1.244333in}{2.361213in}}%
\pgfpathcurveto{\pgfqpoint{1.236520in}{2.369026in}}{\pgfqpoint{1.225921in}{2.373417in}}{\pgfqpoint{1.214870in}{2.373417in}}%
\pgfpathcurveto{\pgfqpoint{1.203820in}{2.373417in}}{\pgfqpoint{1.193221in}{2.369026in}}{\pgfqpoint{1.185408in}{2.361213in}}%
\pgfpathcurveto{\pgfqpoint{1.177594in}{2.353399in}}{\pgfqpoint{1.173204in}{2.342800in}}{\pgfqpoint{1.173204in}{2.331750in}}%
\pgfpathcurveto{\pgfqpoint{1.173204in}{2.320700in}}{\pgfqpoint{1.177594in}{2.310101in}}{\pgfqpoint{1.185408in}{2.302287in}}%
\pgfpathcurveto{\pgfqpoint{1.193221in}{2.294473in}}{\pgfqpoint{1.203820in}{2.290083in}}{\pgfqpoint{1.214870in}{2.290083in}}%
\pgfpathclose%
\pgfusepath{stroke,fill}%
\end{pgfscope}%
\begin{pgfscope}%
\pgfpathrectangle{\pgfqpoint{0.481978in}{0.331635in}}{\pgfqpoint{4.960000in}{3.696000in}}%
\pgfusepath{clip}%
\pgfsetbuttcap%
\pgfsetroundjoin%
\definecolor{currentfill}{rgb}{1.000000,0.705882,0.509804}%
\pgfsetfillcolor{currentfill}%
\pgfsetlinewidth{0.481800pt}%
\definecolor{currentstroke}{rgb}{1.000000,1.000000,1.000000}%
\pgfsetstrokecolor{currentstroke}%
\pgfsetdash{}{0pt}%
\pgfpathmoveto{\pgfqpoint{2.226211in}{2.939857in}}%
\pgfpathcurveto{\pgfqpoint{2.237261in}{2.939857in}}{\pgfqpoint{2.247860in}{2.944248in}}{\pgfqpoint{2.255674in}{2.952061in}}%
\pgfpathcurveto{\pgfqpoint{2.263487in}{2.959875in}}{\pgfqpoint{2.267878in}{2.970474in}}{\pgfqpoint{2.267878in}{2.981524in}}%
\pgfpathcurveto{\pgfqpoint{2.267878in}{2.992574in}}{\pgfqpoint{2.263487in}{3.003173in}}{\pgfqpoint{2.255674in}{3.010987in}}%
\pgfpathcurveto{\pgfqpoint{2.247860in}{3.018800in}}{\pgfqpoint{2.237261in}{3.023191in}}{\pgfqpoint{2.226211in}{3.023191in}}%
\pgfpathcurveto{\pgfqpoint{2.215161in}{3.023191in}}{\pgfqpoint{2.204562in}{3.018800in}}{\pgfqpoint{2.196748in}{3.010987in}}%
\pgfpathcurveto{\pgfqpoint{2.188934in}{3.003173in}}{\pgfqpoint{2.184544in}{2.992574in}}{\pgfqpoint{2.184544in}{2.981524in}}%
\pgfpathcurveto{\pgfqpoint{2.184544in}{2.970474in}}{\pgfqpoint{2.188934in}{2.959875in}}{\pgfqpoint{2.196748in}{2.952061in}}%
\pgfpathcurveto{\pgfqpoint{2.204562in}{2.944248in}}{\pgfqpoint{2.215161in}{2.939857in}}{\pgfqpoint{2.226211in}{2.939857in}}%
\pgfpathclose%
\pgfusepath{stroke,fill}%
\end{pgfscope}%
\begin{pgfscope}%
\pgfpathrectangle{\pgfqpoint{0.481978in}{0.331635in}}{\pgfqpoint{4.960000in}{3.696000in}}%
\pgfusepath{clip}%
\pgfsetbuttcap%
\pgfsetroundjoin%
\definecolor{currentfill}{rgb}{1.000000,0.705882,0.509804}%
\pgfsetfillcolor{currentfill}%
\pgfsetlinewidth{0.481800pt}%
\definecolor{currentstroke}{rgb}{1.000000,1.000000,1.000000}%
\pgfsetstrokecolor{currentstroke}%
\pgfsetdash{}{0pt}%
\pgfpathmoveto{\pgfqpoint{2.628549in}{2.405451in}}%
\pgfpathcurveto{\pgfqpoint{2.639599in}{2.405451in}}{\pgfqpoint{2.650198in}{2.409841in}}{\pgfqpoint{2.658012in}{2.417655in}}%
\pgfpathcurveto{\pgfqpoint{2.665825in}{2.425469in}}{\pgfqpoint{2.670216in}{2.436068in}}{\pgfqpoint{2.670216in}{2.447118in}}%
\pgfpathcurveto{\pgfqpoint{2.670216in}{2.458168in}}{\pgfqpoint{2.665825in}{2.468767in}}{\pgfqpoint{2.658012in}{2.476581in}}%
\pgfpathcurveto{\pgfqpoint{2.650198in}{2.484394in}}{\pgfqpoint{2.639599in}{2.488784in}}{\pgfqpoint{2.628549in}{2.488784in}}%
\pgfpathcurveto{\pgfqpoint{2.617499in}{2.488784in}}{\pgfqpoint{2.606900in}{2.484394in}}{\pgfqpoint{2.599086in}{2.476581in}}%
\pgfpathcurveto{\pgfqpoint{2.591272in}{2.468767in}}{\pgfqpoint{2.586882in}{2.458168in}}{\pgfqpoint{2.586882in}{2.447118in}}%
\pgfpathcurveto{\pgfqpoint{2.586882in}{2.436068in}}{\pgfqpoint{2.591272in}{2.425469in}}{\pgfqpoint{2.599086in}{2.417655in}}%
\pgfpathcurveto{\pgfqpoint{2.606900in}{2.409841in}}{\pgfqpoint{2.617499in}{2.405451in}}{\pgfqpoint{2.628549in}{2.405451in}}%
\pgfpathclose%
\pgfusepath{stroke,fill}%
\end{pgfscope}%
\begin{pgfscope}%
\pgfpathrectangle{\pgfqpoint{0.481978in}{0.331635in}}{\pgfqpoint{4.960000in}{3.696000in}}%
\pgfusepath{clip}%
\pgfsetbuttcap%
\pgfsetroundjoin%
\definecolor{currentfill}{rgb}{1.000000,0.705882,0.509804}%
\pgfsetfillcolor{currentfill}%
\pgfsetlinewidth{0.481800pt}%
\definecolor{currentstroke}{rgb}{1.000000,1.000000,1.000000}%
\pgfsetstrokecolor{currentstroke}%
\pgfsetdash{}{0pt}%
\pgfpathmoveto{\pgfqpoint{1.823780in}{3.491677in}}%
\pgfpathcurveto{\pgfqpoint{1.834830in}{3.491677in}}{\pgfqpoint{1.845429in}{3.496067in}}{\pgfqpoint{1.853243in}{3.503881in}}%
\pgfpathcurveto{\pgfqpoint{1.861057in}{3.511695in}}{\pgfqpoint{1.865447in}{3.522294in}}{\pgfqpoint{1.865447in}{3.533344in}}%
\pgfpathcurveto{\pgfqpoint{1.865447in}{3.544394in}}{\pgfqpoint{1.861057in}{3.554993in}}{\pgfqpoint{1.853243in}{3.562806in}}%
\pgfpathcurveto{\pgfqpoint{1.845429in}{3.570620in}}{\pgfqpoint{1.834830in}{3.575010in}}{\pgfqpoint{1.823780in}{3.575010in}}%
\pgfpathcurveto{\pgfqpoint{1.812730in}{3.575010in}}{\pgfqpoint{1.802131in}{3.570620in}}{\pgfqpoint{1.794317in}{3.562806in}}%
\pgfpathcurveto{\pgfqpoint{1.786504in}{3.554993in}}{\pgfqpoint{1.782113in}{3.544394in}}{\pgfqpoint{1.782113in}{3.533344in}}%
\pgfpathcurveto{\pgfqpoint{1.782113in}{3.522294in}}{\pgfqpoint{1.786504in}{3.511695in}}{\pgfqpoint{1.794317in}{3.503881in}}%
\pgfpathcurveto{\pgfqpoint{1.802131in}{3.496067in}}{\pgfqpoint{1.812730in}{3.491677in}}{\pgfqpoint{1.823780in}{3.491677in}}%
\pgfpathclose%
\pgfusepath{stroke,fill}%
\end{pgfscope}%
\begin{pgfscope}%
\pgfpathrectangle{\pgfqpoint{0.481978in}{0.331635in}}{\pgfqpoint{4.960000in}{3.696000in}}%
\pgfusepath{clip}%
\pgfsetbuttcap%
\pgfsetroundjoin%
\definecolor{currentfill}{rgb}{1.000000,0.705882,0.509804}%
\pgfsetfillcolor{currentfill}%
\pgfsetlinewidth{0.481800pt}%
\definecolor{currentstroke}{rgb}{1.000000,1.000000,1.000000}%
\pgfsetstrokecolor{currentstroke}%
\pgfsetdash{}{0pt}%
\pgfpathmoveto{\pgfqpoint{2.205390in}{2.709138in}}%
\pgfpathcurveto{\pgfqpoint{2.216440in}{2.709138in}}{\pgfqpoint{2.227039in}{2.713528in}}{\pgfqpoint{2.234853in}{2.721342in}}%
\pgfpathcurveto{\pgfqpoint{2.242666in}{2.729155in}}{\pgfqpoint{2.247057in}{2.739754in}}{\pgfqpoint{2.247057in}{2.750804in}}%
\pgfpathcurveto{\pgfqpoint{2.247057in}{2.761854in}}{\pgfqpoint{2.242666in}{2.772454in}}{\pgfqpoint{2.234853in}{2.780267in}}%
\pgfpathcurveto{\pgfqpoint{2.227039in}{2.788081in}}{\pgfqpoint{2.216440in}{2.792471in}}{\pgfqpoint{2.205390in}{2.792471in}}%
\pgfpathcurveto{\pgfqpoint{2.194340in}{2.792471in}}{\pgfqpoint{2.183741in}{2.788081in}}{\pgfqpoint{2.175927in}{2.780267in}}%
\pgfpathcurveto{\pgfqpoint{2.168114in}{2.772454in}}{\pgfqpoint{2.163723in}{2.761854in}}{\pgfqpoint{2.163723in}{2.750804in}}%
\pgfpathcurveto{\pgfqpoint{2.163723in}{2.739754in}}{\pgfqpoint{2.168114in}{2.729155in}}{\pgfqpoint{2.175927in}{2.721342in}}%
\pgfpathcurveto{\pgfqpoint{2.183741in}{2.713528in}}{\pgfqpoint{2.194340in}{2.709138in}}{\pgfqpoint{2.205390in}{2.709138in}}%
\pgfpathclose%
\pgfusepath{stroke,fill}%
\end{pgfscope}%
\begin{pgfscope}%
\pgfpathrectangle{\pgfqpoint{0.481978in}{0.331635in}}{\pgfqpoint{4.960000in}{3.696000in}}%
\pgfusepath{clip}%
\pgfsetbuttcap%
\pgfsetroundjoin%
\definecolor{currentfill}{rgb}{1.000000,0.705882,0.509804}%
\pgfsetfillcolor{currentfill}%
\pgfsetlinewidth{0.481800pt}%
\definecolor{currentstroke}{rgb}{1.000000,1.000000,1.000000}%
\pgfsetstrokecolor{currentstroke}%
\pgfsetdash{}{0pt}%
\pgfpathmoveto{\pgfqpoint{1.850038in}{1.666057in}}%
\pgfpathcurveto{\pgfqpoint{1.861088in}{1.666057in}}{\pgfqpoint{1.871688in}{1.670447in}}{\pgfqpoint{1.879501in}{1.678261in}}%
\pgfpathcurveto{\pgfqpoint{1.887315in}{1.686074in}}{\pgfqpoint{1.891705in}{1.696673in}}{\pgfqpoint{1.891705in}{1.707723in}}%
\pgfpathcurveto{\pgfqpoint{1.891705in}{1.718774in}}{\pgfqpoint{1.887315in}{1.729373in}}{\pgfqpoint{1.879501in}{1.737186in}}%
\pgfpathcurveto{\pgfqpoint{1.871688in}{1.745000in}}{\pgfqpoint{1.861088in}{1.749390in}}{\pgfqpoint{1.850038in}{1.749390in}}%
\pgfpathcurveto{\pgfqpoint{1.838988in}{1.749390in}}{\pgfqpoint{1.828389in}{1.745000in}}{\pgfqpoint{1.820576in}{1.737186in}}%
\pgfpathcurveto{\pgfqpoint{1.812762in}{1.729373in}}{\pgfqpoint{1.808372in}{1.718774in}}{\pgfqpoint{1.808372in}{1.707723in}}%
\pgfpathcurveto{\pgfqpoint{1.808372in}{1.696673in}}{\pgfqpoint{1.812762in}{1.686074in}}{\pgfqpoint{1.820576in}{1.678261in}}%
\pgfpathcurveto{\pgfqpoint{1.828389in}{1.670447in}}{\pgfqpoint{1.838988in}{1.666057in}}{\pgfqpoint{1.850038in}{1.666057in}}%
\pgfpathclose%
\pgfusepath{stroke,fill}%
\end{pgfscope}%
\begin{pgfscope}%
\pgfpathrectangle{\pgfqpoint{0.481978in}{0.331635in}}{\pgfqpoint{4.960000in}{3.696000in}}%
\pgfusepath{clip}%
\pgfsetbuttcap%
\pgfsetroundjoin%
\definecolor{currentfill}{rgb}{1.000000,0.705882,0.509804}%
\pgfsetfillcolor{currentfill}%
\pgfsetlinewidth{0.481800pt}%
\definecolor{currentstroke}{rgb}{1.000000,1.000000,1.000000}%
\pgfsetstrokecolor{currentstroke}%
\pgfsetdash{}{0pt}%
\pgfpathmoveto{\pgfqpoint{2.915131in}{1.073777in}}%
\pgfpathcurveto{\pgfqpoint{2.926181in}{1.073777in}}{\pgfqpoint{2.936780in}{1.078168in}}{\pgfqpoint{2.944593in}{1.085981in}}%
\pgfpathcurveto{\pgfqpoint{2.952407in}{1.093795in}}{\pgfqpoint{2.956797in}{1.104394in}}{\pgfqpoint{2.956797in}{1.115444in}}%
\pgfpathcurveto{\pgfqpoint{2.956797in}{1.126494in}}{\pgfqpoint{2.952407in}{1.137093in}}{\pgfqpoint{2.944593in}{1.144907in}}%
\pgfpathcurveto{\pgfqpoint{2.936780in}{1.152720in}}{\pgfqpoint{2.926181in}{1.157111in}}{\pgfqpoint{2.915131in}{1.157111in}}%
\pgfpathcurveto{\pgfqpoint{2.904081in}{1.157111in}}{\pgfqpoint{2.893482in}{1.152720in}}{\pgfqpoint{2.885668in}{1.144907in}}%
\pgfpathcurveto{\pgfqpoint{2.877854in}{1.137093in}}{\pgfqpoint{2.873464in}{1.126494in}}{\pgfqpoint{2.873464in}{1.115444in}}%
\pgfpathcurveto{\pgfqpoint{2.873464in}{1.104394in}}{\pgfqpoint{2.877854in}{1.093795in}}{\pgfqpoint{2.885668in}{1.085981in}}%
\pgfpathcurveto{\pgfqpoint{2.893482in}{1.078168in}}{\pgfqpoint{2.904081in}{1.073777in}}{\pgfqpoint{2.915131in}{1.073777in}}%
\pgfpathclose%
\pgfusepath{stroke,fill}%
\end{pgfscope}%
\begin{pgfscope}%
\pgfpathrectangle{\pgfqpoint{0.481978in}{0.331635in}}{\pgfqpoint{4.960000in}{3.696000in}}%
\pgfusepath{clip}%
\pgfsetbuttcap%
\pgfsetroundjoin%
\definecolor{currentfill}{rgb}{1.000000,0.705882,0.509804}%
\pgfsetfillcolor{currentfill}%
\pgfsetlinewidth{0.481800pt}%
\definecolor{currentstroke}{rgb}{1.000000,1.000000,1.000000}%
\pgfsetstrokecolor{currentstroke}%
\pgfsetdash{}{0pt}%
\pgfpathmoveto{\pgfqpoint{2.616748in}{2.639832in}}%
\pgfpathcurveto{\pgfqpoint{2.627798in}{2.639832in}}{\pgfqpoint{2.638397in}{2.644222in}}{\pgfqpoint{2.646211in}{2.652036in}}%
\pgfpathcurveto{\pgfqpoint{2.654025in}{2.659850in}}{\pgfqpoint{2.658415in}{2.670449in}}{\pgfqpoint{2.658415in}{2.681499in}}%
\pgfpathcurveto{\pgfqpoint{2.658415in}{2.692549in}}{\pgfqpoint{2.654025in}{2.703148in}}{\pgfqpoint{2.646211in}{2.710962in}}%
\pgfpathcurveto{\pgfqpoint{2.638397in}{2.718775in}}{\pgfqpoint{2.627798in}{2.723166in}}{\pgfqpoint{2.616748in}{2.723166in}}%
\pgfpathcurveto{\pgfqpoint{2.605698in}{2.723166in}}{\pgfqpoint{2.595099in}{2.718775in}}{\pgfqpoint{2.587285in}{2.710962in}}%
\pgfpathcurveto{\pgfqpoint{2.579472in}{2.703148in}}{\pgfqpoint{2.575082in}{2.692549in}}{\pgfqpoint{2.575082in}{2.681499in}}%
\pgfpathcurveto{\pgfqpoint{2.575082in}{2.670449in}}{\pgfqpoint{2.579472in}{2.659850in}}{\pgfqpoint{2.587285in}{2.652036in}}%
\pgfpathcurveto{\pgfqpoint{2.595099in}{2.644222in}}{\pgfqpoint{2.605698in}{2.639832in}}{\pgfqpoint{2.616748in}{2.639832in}}%
\pgfpathclose%
\pgfusepath{stroke,fill}%
\end{pgfscope}%
\begin{pgfscope}%
\pgfpathrectangle{\pgfqpoint{0.481978in}{0.331635in}}{\pgfqpoint{4.960000in}{3.696000in}}%
\pgfusepath{clip}%
\pgfsetbuttcap%
\pgfsetroundjoin%
\definecolor{currentfill}{rgb}{1.000000,0.705882,0.509804}%
\pgfsetfillcolor{currentfill}%
\pgfsetlinewidth{0.481800pt}%
\definecolor{currentstroke}{rgb}{1.000000,1.000000,1.000000}%
\pgfsetstrokecolor{currentstroke}%
\pgfsetdash{}{0pt}%
\pgfpathmoveto{\pgfqpoint{3.272083in}{3.092743in}}%
\pgfpathcurveto{\pgfqpoint{3.283133in}{3.092743in}}{\pgfqpoint{3.293732in}{3.097134in}}{\pgfqpoint{3.301546in}{3.104947in}}%
\pgfpathcurveto{\pgfqpoint{3.309359in}{3.112761in}}{\pgfqpoint{3.313750in}{3.123360in}}{\pgfqpoint{3.313750in}{3.134410in}}%
\pgfpathcurveto{\pgfqpoint{3.313750in}{3.145460in}}{\pgfqpoint{3.309359in}{3.156059in}}{\pgfqpoint{3.301546in}{3.163873in}}%
\pgfpathcurveto{\pgfqpoint{3.293732in}{3.171687in}}{\pgfqpoint{3.283133in}{3.176077in}}{\pgfqpoint{3.272083in}{3.176077in}}%
\pgfpathcurveto{\pgfqpoint{3.261033in}{3.176077in}}{\pgfqpoint{3.250434in}{3.171687in}}{\pgfqpoint{3.242620in}{3.163873in}}%
\pgfpathcurveto{\pgfqpoint{3.234807in}{3.156059in}}{\pgfqpoint{3.230416in}{3.145460in}}{\pgfqpoint{3.230416in}{3.134410in}}%
\pgfpathcurveto{\pgfqpoint{3.230416in}{3.123360in}}{\pgfqpoint{3.234807in}{3.112761in}}{\pgfqpoint{3.242620in}{3.104947in}}%
\pgfpathcurveto{\pgfqpoint{3.250434in}{3.097134in}}{\pgfqpoint{3.261033in}{3.092743in}}{\pgfqpoint{3.272083in}{3.092743in}}%
\pgfpathclose%
\pgfusepath{stroke,fill}%
\end{pgfscope}%
\begin{pgfscope}%
\pgfpathrectangle{\pgfqpoint{0.481978in}{0.331635in}}{\pgfqpoint{4.960000in}{3.696000in}}%
\pgfusepath{clip}%
\pgfsetbuttcap%
\pgfsetroundjoin%
\definecolor{currentfill}{rgb}{1.000000,0.705882,0.509804}%
\pgfsetfillcolor{currentfill}%
\pgfsetlinewidth{0.481800pt}%
\definecolor{currentstroke}{rgb}{1.000000,1.000000,1.000000}%
\pgfsetstrokecolor{currentstroke}%
\pgfsetdash{}{0pt}%
\pgfpathmoveto{\pgfqpoint{1.683594in}{2.601619in}}%
\pgfpathcurveto{\pgfqpoint{1.694644in}{2.601619in}}{\pgfqpoint{1.705243in}{2.606009in}}{\pgfqpoint{1.713056in}{2.613823in}}%
\pgfpathcurveto{\pgfqpoint{1.720870in}{2.621636in}}{\pgfqpoint{1.725260in}{2.632235in}}{\pgfqpoint{1.725260in}{2.643285in}}%
\pgfpathcurveto{\pgfqpoint{1.725260in}{2.654336in}}{\pgfqpoint{1.720870in}{2.664935in}}{\pgfqpoint{1.713056in}{2.672748in}}%
\pgfpathcurveto{\pgfqpoint{1.705243in}{2.680562in}}{\pgfqpoint{1.694644in}{2.684952in}}{\pgfqpoint{1.683594in}{2.684952in}}%
\pgfpathcurveto{\pgfqpoint{1.672544in}{2.684952in}}{\pgfqpoint{1.661945in}{2.680562in}}{\pgfqpoint{1.654131in}{2.672748in}}%
\pgfpathcurveto{\pgfqpoint{1.646317in}{2.664935in}}{\pgfqpoint{1.641927in}{2.654336in}}{\pgfqpoint{1.641927in}{2.643285in}}%
\pgfpathcurveto{\pgfqpoint{1.641927in}{2.632235in}}{\pgfqpoint{1.646317in}{2.621636in}}{\pgfqpoint{1.654131in}{2.613823in}}%
\pgfpathcurveto{\pgfqpoint{1.661945in}{2.606009in}}{\pgfqpoint{1.672544in}{2.601619in}}{\pgfqpoint{1.683594in}{2.601619in}}%
\pgfpathclose%
\pgfusepath{stroke,fill}%
\end{pgfscope}%
\begin{pgfscope}%
\pgfpathrectangle{\pgfqpoint{0.481978in}{0.331635in}}{\pgfqpoint{4.960000in}{3.696000in}}%
\pgfusepath{clip}%
\pgfsetbuttcap%
\pgfsetroundjoin%
\definecolor{currentfill}{rgb}{1.000000,0.705882,0.509804}%
\pgfsetfillcolor{currentfill}%
\pgfsetlinewidth{0.481800pt}%
\definecolor{currentstroke}{rgb}{1.000000,1.000000,1.000000}%
\pgfsetstrokecolor{currentstroke}%
\pgfsetdash{}{0pt}%
\pgfpathmoveto{\pgfqpoint{2.146764in}{2.855906in}}%
\pgfpathcurveto{\pgfqpoint{2.157814in}{2.855906in}}{\pgfqpoint{2.168413in}{2.860296in}}{\pgfqpoint{2.176227in}{2.868109in}}%
\pgfpathcurveto{\pgfqpoint{2.184041in}{2.875923in}}{\pgfqpoint{2.188431in}{2.886522in}}{\pgfqpoint{2.188431in}{2.897572in}}%
\pgfpathcurveto{\pgfqpoint{2.188431in}{2.908622in}}{\pgfqpoint{2.184041in}{2.919221in}}{\pgfqpoint{2.176227in}{2.927035in}}%
\pgfpathcurveto{\pgfqpoint{2.168413in}{2.934849in}}{\pgfqpoint{2.157814in}{2.939239in}}{\pgfqpoint{2.146764in}{2.939239in}}%
\pgfpathcurveto{\pgfqpoint{2.135714in}{2.939239in}}{\pgfqpoint{2.125115in}{2.934849in}}{\pgfqpoint{2.117302in}{2.927035in}}%
\pgfpathcurveto{\pgfqpoint{2.109488in}{2.919221in}}{\pgfqpoint{2.105098in}{2.908622in}}{\pgfqpoint{2.105098in}{2.897572in}}%
\pgfpathcurveto{\pgfqpoint{2.105098in}{2.886522in}}{\pgfqpoint{2.109488in}{2.875923in}}{\pgfqpoint{2.117302in}{2.868109in}}%
\pgfpathcurveto{\pgfqpoint{2.125115in}{2.860296in}}{\pgfqpoint{2.135714in}{2.855906in}}{\pgfqpoint{2.146764in}{2.855906in}}%
\pgfpathclose%
\pgfusepath{stroke,fill}%
\end{pgfscope}%
\begin{pgfscope}%
\pgfpathrectangle{\pgfqpoint{0.481978in}{0.331635in}}{\pgfqpoint{4.960000in}{3.696000in}}%
\pgfusepath{clip}%
\pgfsetbuttcap%
\pgfsetroundjoin%
\definecolor{currentfill}{rgb}{1.000000,0.705882,0.509804}%
\pgfsetfillcolor{currentfill}%
\pgfsetlinewidth{0.481800pt}%
\definecolor{currentstroke}{rgb}{1.000000,1.000000,1.000000}%
\pgfsetstrokecolor{currentstroke}%
\pgfsetdash{}{0pt}%
\pgfpathmoveto{\pgfqpoint{1.635619in}{2.815618in}}%
\pgfpathcurveto{\pgfqpoint{1.646670in}{2.815618in}}{\pgfqpoint{1.657269in}{2.820008in}}{\pgfqpoint{1.665082in}{2.827821in}}%
\pgfpathcurveto{\pgfqpoint{1.672896in}{2.835635in}}{\pgfqpoint{1.677286in}{2.846234in}}{\pgfqpoint{1.677286in}{2.857284in}}%
\pgfpathcurveto{\pgfqpoint{1.677286in}{2.868334in}}{\pgfqpoint{1.672896in}{2.878933in}}{\pgfqpoint{1.665082in}{2.886747in}}%
\pgfpathcurveto{\pgfqpoint{1.657269in}{2.894561in}}{\pgfqpoint{1.646670in}{2.898951in}}{\pgfqpoint{1.635619in}{2.898951in}}%
\pgfpathcurveto{\pgfqpoint{1.624569in}{2.898951in}}{\pgfqpoint{1.613970in}{2.894561in}}{\pgfqpoint{1.606157in}{2.886747in}}%
\pgfpathcurveto{\pgfqpoint{1.598343in}{2.878933in}}{\pgfqpoint{1.593953in}{2.868334in}}{\pgfqpoint{1.593953in}{2.857284in}}%
\pgfpathcurveto{\pgfqpoint{1.593953in}{2.846234in}}{\pgfqpoint{1.598343in}{2.835635in}}{\pgfqpoint{1.606157in}{2.827821in}}%
\pgfpathcurveto{\pgfqpoint{1.613970in}{2.820008in}}{\pgfqpoint{1.624569in}{2.815618in}}{\pgfqpoint{1.635619in}{2.815618in}}%
\pgfpathclose%
\pgfusepath{stroke,fill}%
\end{pgfscope}%
\begin{pgfscope}%
\pgfpathrectangle{\pgfqpoint{0.481978in}{0.331635in}}{\pgfqpoint{4.960000in}{3.696000in}}%
\pgfusepath{clip}%
\pgfsetbuttcap%
\pgfsetroundjoin%
\definecolor{currentfill}{rgb}{1.000000,0.705882,0.509804}%
\pgfsetfillcolor{currentfill}%
\pgfsetlinewidth{0.481800pt}%
\definecolor{currentstroke}{rgb}{1.000000,1.000000,1.000000}%
\pgfsetstrokecolor{currentstroke}%
\pgfsetdash{}{0pt}%
\pgfpathmoveto{\pgfqpoint{2.315751in}{3.604051in}}%
\pgfpathcurveto{\pgfqpoint{2.326801in}{3.604051in}}{\pgfqpoint{2.337400in}{3.608441in}}{\pgfqpoint{2.345214in}{3.616255in}}%
\pgfpathcurveto{\pgfqpoint{2.353027in}{3.624068in}}{\pgfqpoint{2.357418in}{3.634667in}}{\pgfqpoint{2.357418in}{3.645717in}}%
\pgfpathcurveto{\pgfqpoint{2.357418in}{3.656768in}}{\pgfqpoint{2.353027in}{3.667367in}}{\pgfqpoint{2.345214in}{3.675180in}}%
\pgfpathcurveto{\pgfqpoint{2.337400in}{3.682994in}}{\pgfqpoint{2.326801in}{3.687384in}}{\pgfqpoint{2.315751in}{3.687384in}}%
\pgfpathcurveto{\pgfqpoint{2.304701in}{3.687384in}}{\pgfqpoint{2.294102in}{3.682994in}}{\pgfqpoint{2.286288in}{3.675180in}}%
\pgfpathcurveto{\pgfqpoint{2.278474in}{3.667367in}}{\pgfqpoint{2.274084in}{3.656768in}}{\pgfqpoint{2.274084in}{3.645717in}}%
\pgfpathcurveto{\pgfqpoint{2.274084in}{3.634667in}}{\pgfqpoint{2.278474in}{3.624068in}}{\pgfqpoint{2.286288in}{3.616255in}}%
\pgfpathcurveto{\pgfqpoint{2.294102in}{3.608441in}}{\pgfqpoint{2.304701in}{3.604051in}}{\pgfqpoint{2.315751in}{3.604051in}}%
\pgfpathclose%
\pgfusepath{stroke,fill}%
\end{pgfscope}%
\begin{pgfscope}%
\pgfpathrectangle{\pgfqpoint{0.481978in}{0.331635in}}{\pgfqpoint{4.960000in}{3.696000in}}%
\pgfusepath{clip}%
\pgfsetbuttcap%
\pgfsetroundjoin%
\definecolor{currentfill}{rgb}{1.000000,0.705882,0.509804}%
\pgfsetfillcolor{currentfill}%
\pgfsetlinewidth{0.481800pt}%
\definecolor{currentstroke}{rgb}{1.000000,1.000000,1.000000}%
\pgfsetstrokecolor{currentstroke}%
\pgfsetdash{}{0pt}%
\pgfpathmoveto{\pgfqpoint{2.612288in}{2.847324in}}%
\pgfpathcurveto{\pgfqpoint{2.623338in}{2.847324in}}{\pgfqpoint{2.633937in}{2.851714in}}{\pgfqpoint{2.641750in}{2.859528in}}%
\pgfpathcurveto{\pgfqpoint{2.649564in}{2.867342in}}{\pgfqpoint{2.653954in}{2.877941in}}{\pgfqpoint{2.653954in}{2.888991in}}%
\pgfpathcurveto{\pgfqpoint{2.653954in}{2.900041in}}{\pgfqpoint{2.649564in}{2.910640in}}{\pgfqpoint{2.641750in}{2.918453in}}%
\pgfpathcurveto{\pgfqpoint{2.633937in}{2.926267in}}{\pgfqpoint{2.623338in}{2.930657in}}{\pgfqpoint{2.612288in}{2.930657in}}%
\pgfpathcurveto{\pgfqpoint{2.601237in}{2.930657in}}{\pgfqpoint{2.590638in}{2.926267in}}{\pgfqpoint{2.582825in}{2.918453in}}%
\pgfpathcurveto{\pgfqpoint{2.575011in}{2.910640in}}{\pgfqpoint{2.570621in}{2.900041in}}{\pgfqpoint{2.570621in}{2.888991in}}%
\pgfpathcurveto{\pgfqpoint{2.570621in}{2.877941in}}{\pgfqpoint{2.575011in}{2.867342in}}{\pgfqpoint{2.582825in}{2.859528in}}%
\pgfpathcurveto{\pgfqpoint{2.590638in}{2.851714in}}{\pgfqpoint{2.601237in}{2.847324in}}{\pgfqpoint{2.612288in}{2.847324in}}%
\pgfpathclose%
\pgfusepath{stroke,fill}%
\end{pgfscope}%
\begin{pgfscope}%
\pgfpathrectangle{\pgfqpoint{0.481978in}{0.331635in}}{\pgfqpoint{4.960000in}{3.696000in}}%
\pgfusepath{clip}%
\pgfsetbuttcap%
\pgfsetroundjoin%
\definecolor{currentfill}{rgb}{1.000000,0.705882,0.509804}%
\pgfsetfillcolor{currentfill}%
\pgfsetlinewidth{0.481800pt}%
\definecolor{currentstroke}{rgb}{1.000000,1.000000,1.000000}%
\pgfsetstrokecolor{currentstroke}%
\pgfsetdash{}{0pt}%
\pgfpathmoveto{\pgfqpoint{2.791676in}{2.084321in}}%
\pgfpathcurveto{\pgfqpoint{2.802726in}{2.084321in}}{\pgfqpoint{2.813325in}{2.088712in}}{\pgfqpoint{2.821139in}{2.096525in}}%
\pgfpathcurveto{\pgfqpoint{2.828953in}{2.104339in}}{\pgfqpoint{2.833343in}{2.114938in}}{\pgfqpoint{2.833343in}{2.125988in}}%
\pgfpathcurveto{\pgfqpoint{2.833343in}{2.137038in}}{\pgfqpoint{2.828953in}{2.147637in}}{\pgfqpoint{2.821139in}{2.155451in}}%
\pgfpathcurveto{\pgfqpoint{2.813325in}{2.163264in}}{\pgfqpoint{2.802726in}{2.167655in}}{\pgfqpoint{2.791676in}{2.167655in}}%
\pgfpathcurveto{\pgfqpoint{2.780626in}{2.167655in}}{\pgfqpoint{2.770027in}{2.163264in}}{\pgfqpoint{2.762213in}{2.155451in}}%
\pgfpathcurveto{\pgfqpoint{2.754400in}{2.147637in}}{\pgfqpoint{2.750009in}{2.137038in}}{\pgfqpoint{2.750009in}{2.125988in}}%
\pgfpathcurveto{\pgfqpoint{2.750009in}{2.114938in}}{\pgfqpoint{2.754400in}{2.104339in}}{\pgfqpoint{2.762213in}{2.096525in}}%
\pgfpathcurveto{\pgfqpoint{2.770027in}{2.088712in}}{\pgfqpoint{2.780626in}{2.084321in}}{\pgfqpoint{2.791676in}{2.084321in}}%
\pgfpathclose%
\pgfusepath{stroke,fill}%
\end{pgfscope}%
\begin{pgfscope}%
\pgfpathrectangle{\pgfqpoint{0.481978in}{0.331635in}}{\pgfqpoint{4.960000in}{3.696000in}}%
\pgfusepath{clip}%
\pgfsetbuttcap%
\pgfsetroundjoin%
\definecolor{currentfill}{rgb}{1.000000,0.705882,0.509804}%
\pgfsetfillcolor{currentfill}%
\pgfsetlinewidth{0.481800pt}%
\definecolor{currentstroke}{rgb}{1.000000,1.000000,1.000000}%
\pgfsetstrokecolor{currentstroke}%
\pgfsetdash{}{0pt}%
\pgfpathmoveto{\pgfqpoint{2.324073in}{2.666229in}}%
\pgfpathcurveto{\pgfqpoint{2.335123in}{2.666229in}}{\pgfqpoint{2.345722in}{2.670619in}}{\pgfqpoint{2.353536in}{2.678433in}}%
\pgfpathcurveto{\pgfqpoint{2.361349in}{2.686247in}}{\pgfqpoint{2.365740in}{2.696846in}}{\pgfqpoint{2.365740in}{2.707896in}}%
\pgfpathcurveto{\pgfqpoint{2.365740in}{2.718946in}}{\pgfqpoint{2.361349in}{2.729545in}}{\pgfqpoint{2.353536in}{2.737359in}}%
\pgfpathcurveto{\pgfqpoint{2.345722in}{2.745172in}}{\pgfqpoint{2.335123in}{2.749563in}}{\pgfqpoint{2.324073in}{2.749563in}}%
\pgfpathcurveto{\pgfqpoint{2.313023in}{2.749563in}}{\pgfqpoint{2.302424in}{2.745172in}}{\pgfqpoint{2.294610in}{2.737359in}}%
\pgfpathcurveto{\pgfqpoint{2.286797in}{2.729545in}}{\pgfqpoint{2.282406in}{2.718946in}}{\pgfqpoint{2.282406in}{2.707896in}}%
\pgfpathcurveto{\pgfqpoint{2.282406in}{2.696846in}}{\pgfqpoint{2.286797in}{2.686247in}}{\pgfqpoint{2.294610in}{2.678433in}}%
\pgfpathcurveto{\pgfqpoint{2.302424in}{2.670619in}}{\pgfqpoint{2.313023in}{2.666229in}}{\pgfqpoint{2.324073in}{2.666229in}}%
\pgfpathclose%
\pgfusepath{stroke,fill}%
\end{pgfscope}%
\begin{pgfscope}%
\pgfpathrectangle{\pgfqpoint{0.481978in}{0.331635in}}{\pgfqpoint{4.960000in}{3.696000in}}%
\pgfusepath{clip}%
\pgfsetbuttcap%
\pgfsetroundjoin%
\definecolor{currentfill}{rgb}{1.000000,0.705882,0.509804}%
\pgfsetfillcolor{currentfill}%
\pgfsetlinewidth{0.481800pt}%
\definecolor{currentstroke}{rgb}{1.000000,1.000000,1.000000}%
\pgfsetstrokecolor{currentstroke}%
\pgfsetdash{}{0pt}%
\pgfpathmoveto{\pgfqpoint{2.338731in}{2.717499in}}%
\pgfpathcurveto{\pgfqpoint{2.349781in}{2.717499in}}{\pgfqpoint{2.360380in}{2.721889in}}{\pgfqpoint{2.368194in}{2.729703in}}%
\pgfpathcurveto{\pgfqpoint{2.376007in}{2.737516in}}{\pgfqpoint{2.380398in}{2.748115in}}{\pgfqpoint{2.380398in}{2.759166in}}%
\pgfpathcurveto{\pgfqpoint{2.380398in}{2.770216in}}{\pgfqpoint{2.376007in}{2.780815in}}{\pgfqpoint{2.368194in}{2.788628in}}%
\pgfpathcurveto{\pgfqpoint{2.360380in}{2.796442in}}{\pgfqpoint{2.349781in}{2.800832in}}{\pgfqpoint{2.338731in}{2.800832in}}%
\pgfpathcurveto{\pgfqpoint{2.327681in}{2.800832in}}{\pgfqpoint{2.317082in}{2.796442in}}{\pgfqpoint{2.309268in}{2.788628in}}%
\pgfpathcurveto{\pgfqpoint{2.301454in}{2.780815in}}{\pgfqpoint{2.297064in}{2.770216in}}{\pgfqpoint{2.297064in}{2.759166in}}%
\pgfpathcurveto{\pgfqpoint{2.297064in}{2.748115in}}{\pgfqpoint{2.301454in}{2.737516in}}{\pgfqpoint{2.309268in}{2.729703in}}%
\pgfpathcurveto{\pgfqpoint{2.317082in}{2.721889in}}{\pgfqpoint{2.327681in}{2.717499in}}{\pgfqpoint{2.338731in}{2.717499in}}%
\pgfpathclose%
\pgfusepath{stroke,fill}%
\end{pgfscope}%
\begin{pgfscope}%
\pgfpathrectangle{\pgfqpoint{0.481978in}{0.331635in}}{\pgfqpoint{4.960000in}{3.696000in}}%
\pgfusepath{clip}%
\pgfsetbuttcap%
\pgfsetroundjoin%
\definecolor{currentfill}{rgb}{1.000000,0.705882,0.509804}%
\pgfsetfillcolor{currentfill}%
\pgfsetlinewidth{0.481800pt}%
\definecolor{currentstroke}{rgb}{1.000000,1.000000,1.000000}%
\pgfsetstrokecolor{currentstroke}%
\pgfsetdash{}{0pt}%
\pgfpathmoveto{\pgfqpoint{1.786524in}{1.590166in}}%
\pgfpathcurveto{\pgfqpoint{1.797574in}{1.590166in}}{\pgfqpoint{1.808173in}{1.594556in}}{\pgfqpoint{1.815987in}{1.602369in}}%
\pgfpathcurveto{\pgfqpoint{1.823800in}{1.610183in}}{\pgfqpoint{1.828191in}{1.620782in}}{\pgfqpoint{1.828191in}{1.631832in}}%
\pgfpathcurveto{\pgfqpoint{1.828191in}{1.642882in}}{\pgfqpoint{1.823800in}{1.653481in}}{\pgfqpoint{1.815987in}{1.661295in}}%
\pgfpathcurveto{\pgfqpoint{1.808173in}{1.669109in}}{\pgfqpoint{1.797574in}{1.673499in}}{\pgfqpoint{1.786524in}{1.673499in}}%
\pgfpathcurveto{\pgfqpoint{1.775474in}{1.673499in}}{\pgfqpoint{1.764875in}{1.669109in}}{\pgfqpoint{1.757061in}{1.661295in}}%
\pgfpathcurveto{\pgfqpoint{1.749247in}{1.653481in}}{\pgfqpoint{1.744857in}{1.642882in}}{\pgfqpoint{1.744857in}{1.631832in}}%
\pgfpathcurveto{\pgfqpoint{1.744857in}{1.620782in}}{\pgfqpoint{1.749247in}{1.610183in}}{\pgfqpoint{1.757061in}{1.602369in}}%
\pgfpathcurveto{\pgfqpoint{1.764875in}{1.594556in}}{\pgfqpoint{1.775474in}{1.590166in}}{\pgfqpoint{1.786524in}{1.590166in}}%
\pgfpathclose%
\pgfusepath{stroke,fill}%
\end{pgfscope}%
\begin{pgfscope}%
\pgfpathrectangle{\pgfqpoint{0.481978in}{0.331635in}}{\pgfqpoint{4.960000in}{3.696000in}}%
\pgfusepath{clip}%
\pgfsetbuttcap%
\pgfsetroundjoin%
\definecolor{currentfill}{rgb}{1.000000,0.705882,0.509804}%
\pgfsetfillcolor{currentfill}%
\pgfsetlinewidth{0.481800pt}%
\definecolor{currentstroke}{rgb}{1.000000,1.000000,1.000000}%
\pgfsetstrokecolor{currentstroke}%
\pgfsetdash{}{0pt}%
\pgfpathmoveto{\pgfqpoint{4.353277in}{2.988658in}}%
\pgfpathcurveto{\pgfqpoint{4.364327in}{2.988658in}}{\pgfqpoint{4.374926in}{2.993048in}}{\pgfqpoint{4.382740in}{3.000862in}}%
\pgfpathcurveto{\pgfqpoint{4.390554in}{3.008675in}}{\pgfqpoint{4.394944in}{3.019274in}}{\pgfqpoint{4.394944in}{3.030325in}}%
\pgfpathcurveto{\pgfqpoint{4.394944in}{3.041375in}}{\pgfqpoint{4.390554in}{3.051974in}}{\pgfqpoint{4.382740in}{3.059787in}}%
\pgfpathcurveto{\pgfqpoint{4.374926in}{3.067601in}}{\pgfqpoint{4.364327in}{3.071991in}}{\pgfqpoint{4.353277in}{3.071991in}}%
\pgfpathcurveto{\pgfqpoint{4.342227in}{3.071991in}}{\pgfqpoint{4.331628in}{3.067601in}}{\pgfqpoint{4.323815in}{3.059787in}}%
\pgfpathcurveto{\pgfqpoint{4.316001in}{3.051974in}}{\pgfqpoint{4.311611in}{3.041375in}}{\pgfqpoint{4.311611in}{3.030325in}}%
\pgfpathcurveto{\pgfqpoint{4.311611in}{3.019274in}}{\pgfqpoint{4.316001in}{3.008675in}}{\pgfqpoint{4.323815in}{3.000862in}}%
\pgfpathcurveto{\pgfqpoint{4.331628in}{2.993048in}}{\pgfqpoint{4.342227in}{2.988658in}}{\pgfqpoint{4.353277in}{2.988658in}}%
\pgfpathclose%
\pgfusepath{stroke,fill}%
\end{pgfscope}%
\begin{pgfscope}%
\pgfpathrectangle{\pgfqpoint{0.481978in}{0.331635in}}{\pgfqpoint{4.960000in}{3.696000in}}%
\pgfusepath{clip}%
\pgfsetbuttcap%
\pgfsetroundjoin%
\definecolor{currentfill}{rgb}{1.000000,0.705882,0.509804}%
\pgfsetfillcolor{currentfill}%
\pgfsetlinewidth{0.481800pt}%
\definecolor{currentstroke}{rgb}{1.000000,1.000000,1.000000}%
\pgfsetstrokecolor{currentstroke}%
\pgfsetdash{}{0pt}%
\pgfpathmoveto{\pgfqpoint{2.688812in}{2.775134in}}%
\pgfpathcurveto{\pgfqpoint{2.699862in}{2.775134in}}{\pgfqpoint{2.710461in}{2.779524in}}{\pgfqpoint{2.718274in}{2.787338in}}%
\pgfpathcurveto{\pgfqpoint{2.726088in}{2.795152in}}{\pgfqpoint{2.730478in}{2.805751in}}{\pgfqpoint{2.730478in}{2.816801in}}%
\pgfpathcurveto{\pgfqpoint{2.730478in}{2.827851in}}{\pgfqpoint{2.726088in}{2.838450in}}{\pgfqpoint{2.718274in}{2.846263in}}%
\pgfpathcurveto{\pgfqpoint{2.710461in}{2.854077in}}{\pgfqpoint{2.699862in}{2.858467in}}{\pgfqpoint{2.688812in}{2.858467in}}%
\pgfpathcurveto{\pgfqpoint{2.677761in}{2.858467in}}{\pgfqpoint{2.667162in}{2.854077in}}{\pgfqpoint{2.659349in}{2.846263in}}%
\pgfpathcurveto{\pgfqpoint{2.651535in}{2.838450in}}{\pgfqpoint{2.647145in}{2.827851in}}{\pgfqpoint{2.647145in}{2.816801in}}%
\pgfpathcurveto{\pgfqpoint{2.647145in}{2.805751in}}{\pgfqpoint{2.651535in}{2.795152in}}{\pgfqpoint{2.659349in}{2.787338in}}%
\pgfpathcurveto{\pgfqpoint{2.667162in}{2.779524in}}{\pgfqpoint{2.677761in}{2.775134in}}{\pgfqpoint{2.688812in}{2.775134in}}%
\pgfpathclose%
\pgfusepath{stroke,fill}%
\end{pgfscope}%
\begin{pgfscope}%
\pgfpathrectangle{\pgfqpoint{0.481978in}{0.331635in}}{\pgfqpoint{4.960000in}{3.696000in}}%
\pgfusepath{clip}%
\pgfsetbuttcap%
\pgfsetroundjoin%
\definecolor{currentfill}{rgb}{1.000000,0.705882,0.509804}%
\pgfsetfillcolor{currentfill}%
\pgfsetlinewidth{0.481800pt}%
\definecolor{currentstroke}{rgb}{1.000000,1.000000,1.000000}%
\pgfsetstrokecolor{currentstroke}%
\pgfsetdash{}{0pt}%
\pgfpathmoveto{\pgfqpoint{1.990886in}{3.081333in}}%
\pgfpathcurveto{\pgfqpoint{2.001936in}{3.081333in}}{\pgfqpoint{2.012535in}{3.085723in}}{\pgfqpoint{2.020349in}{3.093536in}}%
\pgfpathcurveto{\pgfqpoint{2.028162in}{3.101350in}}{\pgfqpoint{2.032553in}{3.111949in}}{\pgfqpoint{2.032553in}{3.122999in}}%
\pgfpathcurveto{\pgfqpoint{2.032553in}{3.134049in}}{\pgfqpoint{2.028162in}{3.144648in}}{\pgfqpoint{2.020349in}{3.152462in}}%
\pgfpathcurveto{\pgfqpoint{2.012535in}{3.160276in}}{\pgfqpoint{2.001936in}{3.164666in}}{\pgfqpoint{1.990886in}{3.164666in}}%
\pgfpathcurveto{\pgfqpoint{1.979836in}{3.164666in}}{\pgfqpoint{1.969237in}{3.160276in}}{\pgfqpoint{1.961423in}{3.152462in}}%
\pgfpathcurveto{\pgfqpoint{1.953610in}{3.144648in}}{\pgfqpoint{1.949219in}{3.134049in}}{\pgfqpoint{1.949219in}{3.122999in}}%
\pgfpathcurveto{\pgfqpoint{1.949219in}{3.111949in}}{\pgfqpoint{1.953610in}{3.101350in}}{\pgfqpoint{1.961423in}{3.093536in}}%
\pgfpathcurveto{\pgfqpoint{1.969237in}{3.085723in}}{\pgfqpoint{1.979836in}{3.081333in}}{\pgfqpoint{1.990886in}{3.081333in}}%
\pgfpathclose%
\pgfusepath{stroke,fill}%
\end{pgfscope}%
\begin{pgfscope}%
\pgfpathrectangle{\pgfqpoint{0.481978in}{0.331635in}}{\pgfqpoint{4.960000in}{3.696000in}}%
\pgfusepath{clip}%
\pgfsetbuttcap%
\pgfsetroundjoin%
\definecolor{currentfill}{rgb}{1.000000,0.705882,0.509804}%
\pgfsetfillcolor{currentfill}%
\pgfsetlinewidth{0.481800pt}%
\definecolor{currentstroke}{rgb}{1.000000,1.000000,1.000000}%
\pgfsetstrokecolor{currentstroke}%
\pgfsetdash{}{0pt}%
\pgfpathmoveto{\pgfqpoint{2.134851in}{2.955310in}}%
\pgfpathcurveto{\pgfqpoint{2.145901in}{2.955310in}}{\pgfqpoint{2.156500in}{2.959700in}}{\pgfqpoint{2.164313in}{2.967514in}}%
\pgfpathcurveto{\pgfqpoint{2.172127in}{2.975327in}}{\pgfqpoint{2.176517in}{2.985926in}}{\pgfqpoint{2.176517in}{2.996977in}}%
\pgfpathcurveto{\pgfqpoint{2.176517in}{3.008027in}}{\pgfqpoint{2.172127in}{3.018626in}}{\pgfqpoint{2.164313in}{3.026439in}}%
\pgfpathcurveto{\pgfqpoint{2.156500in}{3.034253in}}{\pgfqpoint{2.145901in}{3.038643in}}{\pgfqpoint{2.134851in}{3.038643in}}%
\pgfpathcurveto{\pgfqpoint{2.123801in}{3.038643in}}{\pgfqpoint{2.113202in}{3.034253in}}{\pgfqpoint{2.105388in}{3.026439in}}%
\pgfpathcurveto{\pgfqpoint{2.097574in}{3.018626in}}{\pgfqpoint{2.093184in}{3.008027in}}{\pgfqpoint{2.093184in}{2.996977in}}%
\pgfpathcurveto{\pgfqpoint{2.093184in}{2.985926in}}{\pgfqpoint{2.097574in}{2.975327in}}{\pgfqpoint{2.105388in}{2.967514in}}%
\pgfpathcurveto{\pgfqpoint{2.113202in}{2.959700in}}{\pgfqpoint{2.123801in}{2.955310in}}{\pgfqpoint{2.134851in}{2.955310in}}%
\pgfpathclose%
\pgfusepath{stroke,fill}%
\end{pgfscope}%
\begin{pgfscope}%
\pgfpathrectangle{\pgfqpoint{0.481978in}{0.331635in}}{\pgfqpoint{4.960000in}{3.696000in}}%
\pgfusepath{clip}%
\pgfsetbuttcap%
\pgfsetroundjoin%
\definecolor{currentfill}{rgb}{1.000000,0.705882,0.509804}%
\pgfsetfillcolor{currentfill}%
\pgfsetlinewidth{0.481800pt}%
\definecolor{currentstroke}{rgb}{1.000000,1.000000,1.000000}%
\pgfsetstrokecolor{currentstroke}%
\pgfsetdash{}{0pt}%
\pgfpathmoveto{\pgfqpoint{0.966075in}{2.682244in}}%
\pgfpathcurveto{\pgfqpoint{0.977125in}{2.682244in}}{\pgfqpoint{0.987724in}{2.686634in}}{\pgfqpoint{0.995538in}{2.694448in}}%
\pgfpathcurveto{\pgfqpoint{1.003352in}{2.702261in}}{\pgfqpoint{1.007742in}{2.712861in}}{\pgfqpoint{1.007742in}{2.723911in}}%
\pgfpathcurveto{\pgfqpoint{1.007742in}{2.734961in}}{\pgfqpoint{1.003352in}{2.745560in}}{\pgfqpoint{0.995538in}{2.753373in}}%
\pgfpathcurveto{\pgfqpoint{0.987724in}{2.761187in}}{\pgfqpoint{0.977125in}{2.765577in}}{\pgfqpoint{0.966075in}{2.765577in}}%
\pgfpathcurveto{\pgfqpoint{0.955025in}{2.765577in}}{\pgfqpoint{0.944426in}{2.761187in}}{\pgfqpoint{0.936613in}{2.753373in}}%
\pgfpathcurveto{\pgfqpoint{0.928799in}{2.745560in}}{\pgfqpoint{0.924409in}{2.734961in}}{\pgfqpoint{0.924409in}{2.723911in}}%
\pgfpathcurveto{\pgfqpoint{0.924409in}{2.712861in}}{\pgfqpoint{0.928799in}{2.702261in}}{\pgfqpoint{0.936613in}{2.694448in}}%
\pgfpathcurveto{\pgfqpoint{0.944426in}{2.686634in}}{\pgfqpoint{0.955025in}{2.682244in}}{\pgfqpoint{0.966075in}{2.682244in}}%
\pgfpathclose%
\pgfusepath{stroke,fill}%
\end{pgfscope}%
\begin{pgfscope}%
\pgfpathrectangle{\pgfqpoint{0.481978in}{0.331635in}}{\pgfqpoint{4.960000in}{3.696000in}}%
\pgfusepath{clip}%
\pgfsetbuttcap%
\pgfsetroundjoin%
\definecolor{currentfill}{rgb}{1.000000,0.705882,0.509804}%
\pgfsetfillcolor{currentfill}%
\pgfsetlinewidth{0.481800pt}%
\definecolor{currentstroke}{rgb}{1.000000,1.000000,1.000000}%
\pgfsetstrokecolor{currentstroke}%
\pgfsetdash{}{0pt}%
\pgfpathmoveto{\pgfqpoint{2.829475in}{2.765083in}}%
\pgfpathcurveto{\pgfqpoint{2.840525in}{2.765083in}}{\pgfqpoint{2.851124in}{2.769473in}}{\pgfqpoint{2.858938in}{2.777287in}}%
\pgfpathcurveto{\pgfqpoint{2.866751in}{2.785100in}}{\pgfqpoint{2.871142in}{2.795699in}}{\pgfqpoint{2.871142in}{2.806749in}}%
\pgfpathcurveto{\pgfqpoint{2.871142in}{2.817800in}}{\pgfqpoint{2.866751in}{2.828399in}}{\pgfqpoint{2.858938in}{2.836212in}}%
\pgfpathcurveto{\pgfqpoint{2.851124in}{2.844026in}}{\pgfqpoint{2.840525in}{2.848416in}}{\pgfqpoint{2.829475in}{2.848416in}}%
\pgfpathcurveto{\pgfqpoint{2.818425in}{2.848416in}}{\pgfqpoint{2.807826in}{2.844026in}}{\pgfqpoint{2.800012in}{2.836212in}}%
\pgfpathcurveto{\pgfqpoint{2.792199in}{2.828399in}}{\pgfqpoint{2.787808in}{2.817800in}}{\pgfqpoint{2.787808in}{2.806749in}}%
\pgfpathcurveto{\pgfqpoint{2.787808in}{2.795699in}}{\pgfqpoint{2.792199in}{2.785100in}}{\pgfqpoint{2.800012in}{2.777287in}}%
\pgfpathcurveto{\pgfqpoint{2.807826in}{2.769473in}}{\pgfqpoint{2.818425in}{2.765083in}}{\pgfqpoint{2.829475in}{2.765083in}}%
\pgfpathclose%
\pgfusepath{stroke,fill}%
\end{pgfscope}%
\begin{pgfscope}%
\pgfpathrectangle{\pgfqpoint{0.481978in}{0.331635in}}{\pgfqpoint{4.960000in}{3.696000in}}%
\pgfusepath{clip}%
\pgfsetbuttcap%
\pgfsetroundjoin%
\definecolor{currentfill}{rgb}{1.000000,0.705882,0.509804}%
\pgfsetfillcolor{currentfill}%
\pgfsetlinewidth{0.481800pt}%
\definecolor{currentstroke}{rgb}{1.000000,1.000000,1.000000}%
\pgfsetstrokecolor{currentstroke}%
\pgfsetdash{}{0pt}%
\pgfpathmoveto{\pgfqpoint{2.536496in}{1.737923in}}%
\pgfpathcurveto{\pgfqpoint{2.547546in}{1.737923in}}{\pgfqpoint{2.558145in}{1.742313in}}{\pgfqpoint{2.565959in}{1.750127in}}%
\pgfpathcurveto{\pgfqpoint{2.573772in}{1.757941in}}{\pgfqpoint{2.578163in}{1.768540in}}{\pgfqpoint{2.578163in}{1.779590in}}%
\pgfpathcurveto{\pgfqpoint{2.578163in}{1.790640in}}{\pgfqpoint{2.573772in}{1.801239in}}{\pgfqpoint{2.565959in}{1.809053in}}%
\pgfpathcurveto{\pgfqpoint{2.558145in}{1.816866in}}{\pgfqpoint{2.547546in}{1.821257in}}{\pgfqpoint{2.536496in}{1.821257in}}%
\pgfpathcurveto{\pgfqpoint{2.525446in}{1.821257in}}{\pgfqpoint{2.514847in}{1.816866in}}{\pgfqpoint{2.507033in}{1.809053in}}%
\pgfpathcurveto{\pgfqpoint{2.499220in}{1.801239in}}{\pgfqpoint{2.494829in}{1.790640in}}{\pgfqpoint{2.494829in}{1.779590in}}%
\pgfpathcurveto{\pgfqpoint{2.494829in}{1.768540in}}{\pgfqpoint{2.499220in}{1.757941in}}{\pgfqpoint{2.507033in}{1.750127in}}%
\pgfpathcurveto{\pgfqpoint{2.514847in}{1.742313in}}{\pgfqpoint{2.525446in}{1.737923in}}{\pgfqpoint{2.536496in}{1.737923in}}%
\pgfpathclose%
\pgfusepath{stroke,fill}%
\end{pgfscope}%
\begin{pgfscope}%
\pgfpathrectangle{\pgfqpoint{0.481978in}{0.331635in}}{\pgfqpoint{4.960000in}{3.696000in}}%
\pgfusepath{clip}%
\pgfsetbuttcap%
\pgfsetroundjoin%
\definecolor{currentfill}{rgb}{1.000000,0.705882,0.509804}%
\pgfsetfillcolor{currentfill}%
\pgfsetlinewidth{0.481800pt}%
\definecolor{currentstroke}{rgb}{1.000000,1.000000,1.000000}%
\pgfsetstrokecolor{currentstroke}%
\pgfsetdash{}{0pt}%
\pgfpathmoveto{\pgfqpoint{1.702979in}{2.584298in}}%
\pgfpathcurveto{\pgfqpoint{1.714029in}{2.584298in}}{\pgfqpoint{1.724628in}{2.588688in}}{\pgfqpoint{1.732442in}{2.596502in}}%
\pgfpathcurveto{\pgfqpoint{1.740256in}{2.604315in}}{\pgfqpoint{1.744646in}{2.614914in}}{\pgfqpoint{1.744646in}{2.625964in}}%
\pgfpathcurveto{\pgfqpoint{1.744646in}{2.637015in}}{\pgfqpoint{1.740256in}{2.647614in}}{\pgfqpoint{1.732442in}{2.655427in}}%
\pgfpathcurveto{\pgfqpoint{1.724628in}{2.663241in}}{\pgfqpoint{1.714029in}{2.667631in}}{\pgfqpoint{1.702979in}{2.667631in}}%
\pgfpathcurveto{\pgfqpoint{1.691929in}{2.667631in}}{\pgfqpoint{1.681330in}{2.663241in}}{\pgfqpoint{1.673516in}{2.655427in}}%
\pgfpathcurveto{\pgfqpoint{1.665703in}{2.647614in}}{\pgfqpoint{1.661312in}{2.637015in}}{\pgfqpoint{1.661312in}{2.625964in}}%
\pgfpathcurveto{\pgfqpoint{1.661312in}{2.614914in}}{\pgfqpoint{1.665703in}{2.604315in}}{\pgfqpoint{1.673516in}{2.596502in}}%
\pgfpathcurveto{\pgfqpoint{1.681330in}{2.588688in}}{\pgfqpoint{1.691929in}{2.584298in}}{\pgfqpoint{1.702979in}{2.584298in}}%
\pgfpathclose%
\pgfusepath{stroke,fill}%
\end{pgfscope}%
\begin{pgfscope}%
\pgfpathrectangle{\pgfqpoint{0.481978in}{0.331635in}}{\pgfqpoint{4.960000in}{3.696000in}}%
\pgfusepath{clip}%
\pgfsetbuttcap%
\pgfsetroundjoin%
\definecolor{currentfill}{rgb}{1.000000,0.705882,0.509804}%
\pgfsetfillcolor{currentfill}%
\pgfsetlinewidth{0.481800pt}%
\definecolor{currentstroke}{rgb}{1.000000,1.000000,1.000000}%
\pgfsetstrokecolor{currentstroke}%
\pgfsetdash{}{0pt}%
\pgfpathmoveto{\pgfqpoint{1.756552in}{2.154457in}}%
\pgfpathcurveto{\pgfqpoint{1.767602in}{2.154457in}}{\pgfqpoint{1.778201in}{2.158847in}}{\pgfqpoint{1.786014in}{2.166661in}}%
\pgfpathcurveto{\pgfqpoint{1.793828in}{2.174474in}}{\pgfqpoint{1.798218in}{2.185073in}}{\pgfqpoint{1.798218in}{2.196124in}}%
\pgfpathcurveto{\pgfqpoint{1.798218in}{2.207174in}}{\pgfqpoint{1.793828in}{2.217773in}}{\pgfqpoint{1.786014in}{2.225586in}}%
\pgfpathcurveto{\pgfqpoint{1.778201in}{2.233400in}}{\pgfqpoint{1.767602in}{2.237790in}}{\pgfqpoint{1.756552in}{2.237790in}}%
\pgfpathcurveto{\pgfqpoint{1.745501in}{2.237790in}}{\pgfqpoint{1.734902in}{2.233400in}}{\pgfqpoint{1.727089in}{2.225586in}}%
\pgfpathcurveto{\pgfqpoint{1.719275in}{2.217773in}}{\pgfqpoint{1.714885in}{2.207174in}}{\pgfqpoint{1.714885in}{2.196124in}}%
\pgfpathcurveto{\pgfqpoint{1.714885in}{2.185073in}}{\pgfqpoint{1.719275in}{2.174474in}}{\pgfqpoint{1.727089in}{2.166661in}}%
\pgfpathcurveto{\pgfqpoint{1.734902in}{2.158847in}}{\pgfqpoint{1.745501in}{2.154457in}}{\pgfqpoint{1.756552in}{2.154457in}}%
\pgfpathclose%
\pgfusepath{stroke,fill}%
\end{pgfscope}%
\begin{pgfscope}%
\pgfpathrectangle{\pgfqpoint{0.481978in}{0.331635in}}{\pgfqpoint{4.960000in}{3.696000in}}%
\pgfusepath{clip}%
\pgfsetbuttcap%
\pgfsetroundjoin%
\definecolor{currentfill}{rgb}{1.000000,0.705882,0.509804}%
\pgfsetfillcolor{currentfill}%
\pgfsetlinewidth{0.481800pt}%
\definecolor{currentstroke}{rgb}{1.000000,1.000000,1.000000}%
\pgfsetstrokecolor{currentstroke}%
\pgfsetdash{}{0pt}%
\pgfpathmoveto{\pgfqpoint{1.568211in}{3.075530in}}%
\pgfpathcurveto{\pgfqpoint{1.579261in}{3.075530in}}{\pgfqpoint{1.589860in}{3.079920in}}{\pgfqpoint{1.597673in}{3.087734in}}%
\pgfpathcurveto{\pgfqpoint{1.605487in}{3.095548in}}{\pgfqpoint{1.609877in}{3.106147in}}{\pgfqpoint{1.609877in}{3.117197in}}%
\pgfpathcurveto{\pgfqpoint{1.609877in}{3.128247in}}{\pgfqpoint{1.605487in}{3.138846in}}{\pgfqpoint{1.597673in}{3.146660in}}%
\pgfpathcurveto{\pgfqpoint{1.589860in}{3.154473in}}{\pgfqpoint{1.579261in}{3.158863in}}{\pgfqpoint{1.568211in}{3.158863in}}%
\pgfpathcurveto{\pgfqpoint{1.557160in}{3.158863in}}{\pgfqpoint{1.546561in}{3.154473in}}{\pgfqpoint{1.538748in}{3.146660in}}%
\pgfpathcurveto{\pgfqpoint{1.530934in}{3.138846in}}{\pgfqpoint{1.526544in}{3.128247in}}{\pgfqpoint{1.526544in}{3.117197in}}%
\pgfpathcurveto{\pgfqpoint{1.526544in}{3.106147in}}{\pgfqpoint{1.530934in}{3.095548in}}{\pgfqpoint{1.538748in}{3.087734in}}%
\pgfpathcurveto{\pgfqpoint{1.546561in}{3.079920in}}{\pgfqpoint{1.557160in}{3.075530in}}{\pgfqpoint{1.568211in}{3.075530in}}%
\pgfpathclose%
\pgfusepath{stroke,fill}%
\end{pgfscope}%
\begin{pgfscope}%
\pgfpathrectangle{\pgfqpoint{0.481978in}{0.331635in}}{\pgfqpoint{4.960000in}{3.696000in}}%
\pgfusepath{clip}%
\pgfsetbuttcap%
\pgfsetroundjoin%
\definecolor{currentfill}{rgb}{1.000000,0.705882,0.509804}%
\pgfsetfillcolor{currentfill}%
\pgfsetlinewidth{0.481800pt}%
\definecolor{currentstroke}{rgb}{1.000000,1.000000,1.000000}%
\pgfsetstrokecolor{currentstroke}%
\pgfsetdash{}{0pt}%
\pgfpathmoveto{\pgfqpoint{1.805413in}{3.110777in}}%
\pgfpathcurveto{\pgfqpoint{1.816463in}{3.110777in}}{\pgfqpoint{1.827063in}{3.115167in}}{\pgfqpoint{1.834876in}{3.122981in}}%
\pgfpathcurveto{\pgfqpoint{1.842690in}{3.130794in}}{\pgfqpoint{1.847080in}{3.141393in}}{\pgfqpoint{1.847080in}{3.152443in}}%
\pgfpathcurveto{\pgfqpoint{1.847080in}{3.163493in}}{\pgfqpoint{1.842690in}{3.174093in}}{\pgfqpoint{1.834876in}{3.181906in}}%
\pgfpathcurveto{\pgfqpoint{1.827063in}{3.189720in}}{\pgfqpoint{1.816463in}{3.194110in}}{\pgfqpoint{1.805413in}{3.194110in}}%
\pgfpathcurveto{\pgfqpoint{1.794363in}{3.194110in}}{\pgfqpoint{1.783764in}{3.189720in}}{\pgfqpoint{1.775951in}{3.181906in}}%
\pgfpathcurveto{\pgfqpoint{1.768137in}{3.174093in}}{\pgfqpoint{1.763747in}{3.163493in}}{\pgfqpoint{1.763747in}{3.152443in}}%
\pgfpathcurveto{\pgfqpoint{1.763747in}{3.141393in}}{\pgfqpoint{1.768137in}{3.130794in}}{\pgfqpoint{1.775951in}{3.122981in}}%
\pgfpathcurveto{\pgfqpoint{1.783764in}{3.115167in}}{\pgfqpoint{1.794363in}{3.110777in}}{\pgfqpoint{1.805413in}{3.110777in}}%
\pgfpathclose%
\pgfusepath{stroke,fill}%
\end{pgfscope}%
\begin{pgfscope}%
\pgfpathrectangle{\pgfqpoint{0.481978in}{0.331635in}}{\pgfqpoint{4.960000in}{3.696000in}}%
\pgfusepath{clip}%
\pgfsetbuttcap%
\pgfsetroundjoin%
\definecolor{currentfill}{rgb}{1.000000,0.705882,0.509804}%
\pgfsetfillcolor{currentfill}%
\pgfsetlinewidth{0.481800pt}%
\definecolor{currentstroke}{rgb}{1.000000,1.000000,1.000000}%
\pgfsetstrokecolor{currentstroke}%
\pgfsetdash{}{0pt}%
\pgfpathmoveto{\pgfqpoint{2.744902in}{3.398776in}}%
\pgfpathcurveto{\pgfqpoint{2.755953in}{3.398776in}}{\pgfqpoint{2.766552in}{3.403166in}}{\pgfqpoint{2.774365in}{3.410980in}}%
\pgfpathcurveto{\pgfqpoint{2.782179in}{3.418794in}}{\pgfqpoint{2.786569in}{3.429393in}}{\pgfqpoint{2.786569in}{3.440443in}}%
\pgfpathcurveto{\pgfqpoint{2.786569in}{3.451493in}}{\pgfqpoint{2.782179in}{3.462092in}}{\pgfqpoint{2.774365in}{3.469906in}}%
\pgfpathcurveto{\pgfqpoint{2.766552in}{3.477719in}}{\pgfqpoint{2.755953in}{3.482110in}}{\pgfqpoint{2.744902in}{3.482110in}}%
\pgfpathcurveto{\pgfqpoint{2.733852in}{3.482110in}}{\pgfqpoint{2.723253in}{3.477719in}}{\pgfqpoint{2.715440in}{3.469906in}}%
\pgfpathcurveto{\pgfqpoint{2.707626in}{3.462092in}}{\pgfqpoint{2.703236in}{3.451493in}}{\pgfqpoint{2.703236in}{3.440443in}}%
\pgfpathcurveto{\pgfqpoint{2.703236in}{3.429393in}}{\pgfqpoint{2.707626in}{3.418794in}}{\pgfqpoint{2.715440in}{3.410980in}}%
\pgfpathcurveto{\pgfqpoint{2.723253in}{3.403166in}}{\pgfqpoint{2.733852in}{3.398776in}}{\pgfqpoint{2.744902in}{3.398776in}}%
\pgfpathclose%
\pgfusepath{stroke,fill}%
\end{pgfscope}%
\begin{pgfscope}%
\pgfpathrectangle{\pgfqpoint{0.481978in}{0.331635in}}{\pgfqpoint{4.960000in}{3.696000in}}%
\pgfusepath{clip}%
\pgfsetbuttcap%
\pgfsetroundjoin%
\definecolor{currentfill}{rgb}{1.000000,0.705882,0.509804}%
\pgfsetfillcolor{currentfill}%
\pgfsetlinewidth{0.481800pt}%
\definecolor{currentstroke}{rgb}{1.000000,1.000000,1.000000}%
\pgfsetstrokecolor{currentstroke}%
\pgfsetdash{}{0pt}%
\pgfpathmoveto{\pgfqpoint{1.466717in}{2.735697in}}%
\pgfpathcurveto{\pgfqpoint{1.477767in}{2.735697in}}{\pgfqpoint{1.488366in}{2.740087in}}{\pgfqpoint{1.496180in}{2.747901in}}%
\pgfpathcurveto{\pgfqpoint{1.503993in}{2.755714in}}{\pgfqpoint{1.508384in}{2.766313in}}{\pgfqpoint{1.508384in}{2.777363in}}%
\pgfpathcurveto{\pgfqpoint{1.508384in}{2.788414in}}{\pgfqpoint{1.503993in}{2.799013in}}{\pgfqpoint{1.496180in}{2.806826in}}%
\pgfpathcurveto{\pgfqpoint{1.488366in}{2.814640in}}{\pgfqpoint{1.477767in}{2.819030in}}{\pgfqpoint{1.466717in}{2.819030in}}%
\pgfpathcurveto{\pgfqpoint{1.455667in}{2.819030in}}{\pgfqpoint{1.445068in}{2.814640in}}{\pgfqpoint{1.437254in}{2.806826in}}%
\pgfpathcurveto{\pgfqpoint{1.429441in}{2.799013in}}{\pgfqpoint{1.425050in}{2.788414in}}{\pgfqpoint{1.425050in}{2.777363in}}%
\pgfpathcurveto{\pgfqpoint{1.425050in}{2.766313in}}{\pgfqpoint{1.429441in}{2.755714in}}{\pgfqpoint{1.437254in}{2.747901in}}%
\pgfpathcurveto{\pgfqpoint{1.445068in}{2.740087in}}{\pgfqpoint{1.455667in}{2.735697in}}{\pgfqpoint{1.466717in}{2.735697in}}%
\pgfpathclose%
\pgfusepath{stroke,fill}%
\end{pgfscope}%
\begin{pgfscope}%
\pgfpathrectangle{\pgfqpoint{0.481978in}{0.331635in}}{\pgfqpoint{4.960000in}{3.696000in}}%
\pgfusepath{clip}%
\pgfsetbuttcap%
\pgfsetroundjoin%
\definecolor{currentfill}{rgb}{1.000000,0.705882,0.509804}%
\pgfsetfillcolor{currentfill}%
\pgfsetlinewidth{0.481800pt}%
\definecolor{currentstroke}{rgb}{1.000000,1.000000,1.000000}%
\pgfsetstrokecolor{currentstroke}%
\pgfsetdash{}{0pt}%
\pgfpathmoveto{\pgfqpoint{2.898260in}{2.922451in}}%
\pgfpathcurveto{\pgfqpoint{2.909310in}{2.922451in}}{\pgfqpoint{2.919909in}{2.926842in}}{\pgfqpoint{2.927723in}{2.934655in}}%
\pgfpathcurveto{\pgfqpoint{2.935537in}{2.942469in}}{\pgfqpoint{2.939927in}{2.953068in}}{\pgfqpoint{2.939927in}{2.964118in}}%
\pgfpathcurveto{\pgfqpoint{2.939927in}{2.975168in}}{\pgfqpoint{2.935537in}{2.985767in}}{\pgfqpoint{2.927723in}{2.993581in}}%
\pgfpathcurveto{\pgfqpoint{2.919909in}{3.001395in}}{\pgfqpoint{2.909310in}{3.005785in}}{\pgfqpoint{2.898260in}{3.005785in}}%
\pgfpathcurveto{\pgfqpoint{2.887210in}{3.005785in}}{\pgfqpoint{2.876611in}{3.001395in}}{\pgfqpoint{2.868797in}{2.993581in}}%
\pgfpathcurveto{\pgfqpoint{2.860984in}{2.985767in}}{\pgfqpoint{2.856594in}{2.975168in}}{\pgfqpoint{2.856594in}{2.964118in}}%
\pgfpathcurveto{\pgfqpoint{2.856594in}{2.953068in}}{\pgfqpoint{2.860984in}{2.942469in}}{\pgfqpoint{2.868797in}{2.934655in}}%
\pgfpathcurveto{\pgfqpoint{2.876611in}{2.926842in}}{\pgfqpoint{2.887210in}{2.922451in}}{\pgfqpoint{2.898260in}{2.922451in}}%
\pgfpathclose%
\pgfusepath{stroke,fill}%
\end{pgfscope}%
\begin{pgfscope}%
\pgfpathrectangle{\pgfqpoint{0.481978in}{0.331635in}}{\pgfqpoint{4.960000in}{3.696000in}}%
\pgfusepath{clip}%
\pgfsetbuttcap%
\pgfsetroundjoin%
\definecolor{currentfill}{rgb}{1.000000,0.705882,0.509804}%
\pgfsetfillcolor{currentfill}%
\pgfsetlinewidth{0.481800pt}%
\definecolor{currentstroke}{rgb}{1.000000,1.000000,1.000000}%
\pgfsetstrokecolor{currentstroke}%
\pgfsetdash{}{0pt}%
\pgfpathmoveto{\pgfqpoint{2.382543in}{2.287145in}}%
\pgfpathcurveto{\pgfqpoint{2.393593in}{2.287145in}}{\pgfqpoint{2.404192in}{2.291535in}}{\pgfqpoint{2.412006in}{2.299349in}}%
\pgfpathcurveto{\pgfqpoint{2.419820in}{2.307163in}}{\pgfqpoint{2.424210in}{2.317762in}}{\pgfqpoint{2.424210in}{2.328812in}}%
\pgfpathcurveto{\pgfqpoint{2.424210in}{2.339862in}}{\pgfqpoint{2.419820in}{2.350461in}}{\pgfqpoint{2.412006in}{2.358275in}}%
\pgfpathcurveto{\pgfqpoint{2.404192in}{2.366088in}}{\pgfqpoint{2.393593in}{2.370478in}}{\pgfqpoint{2.382543in}{2.370478in}}%
\pgfpathcurveto{\pgfqpoint{2.371493in}{2.370478in}}{\pgfqpoint{2.360894in}{2.366088in}}{\pgfqpoint{2.353081in}{2.358275in}}%
\pgfpathcurveto{\pgfqpoint{2.345267in}{2.350461in}}{\pgfqpoint{2.340877in}{2.339862in}}{\pgfqpoint{2.340877in}{2.328812in}}%
\pgfpathcurveto{\pgfqpoint{2.340877in}{2.317762in}}{\pgfqpoint{2.345267in}{2.307163in}}{\pgfqpoint{2.353081in}{2.299349in}}%
\pgfpathcurveto{\pgfqpoint{2.360894in}{2.291535in}}{\pgfqpoint{2.371493in}{2.287145in}}{\pgfqpoint{2.382543in}{2.287145in}}%
\pgfpathclose%
\pgfusepath{stroke,fill}%
\end{pgfscope}%
\begin{pgfscope}%
\pgfpathrectangle{\pgfqpoint{0.481978in}{0.331635in}}{\pgfqpoint{4.960000in}{3.696000in}}%
\pgfusepath{clip}%
\pgfsetbuttcap%
\pgfsetroundjoin%
\definecolor{currentfill}{rgb}{1.000000,0.705882,0.509804}%
\pgfsetfillcolor{currentfill}%
\pgfsetlinewidth{0.481800pt}%
\definecolor{currentstroke}{rgb}{1.000000,1.000000,1.000000}%
\pgfsetstrokecolor{currentstroke}%
\pgfsetdash{}{0pt}%
\pgfpathmoveto{\pgfqpoint{2.770316in}{2.521856in}}%
\pgfpathcurveto{\pgfqpoint{2.781366in}{2.521856in}}{\pgfqpoint{2.791965in}{2.526246in}}{\pgfqpoint{2.799778in}{2.534060in}}%
\pgfpathcurveto{\pgfqpoint{2.807592in}{2.541873in}}{\pgfqpoint{2.811982in}{2.552472in}}{\pgfqpoint{2.811982in}{2.563522in}}%
\pgfpathcurveto{\pgfqpoint{2.811982in}{2.574573in}}{\pgfqpoint{2.807592in}{2.585172in}}{\pgfqpoint{2.799778in}{2.592985in}}%
\pgfpathcurveto{\pgfqpoint{2.791965in}{2.600799in}}{\pgfqpoint{2.781366in}{2.605189in}}{\pgfqpoint{2.770316in}{2.605189in}}%
\pgfpathcurveto{\pgfqpoint{2.759266in}{2.605189in}}{\pgfqpoint{2.748667in}{2.600799in}}{\pgfqpoint{2.740853in}{2.592985in}}%
\pgfpathcurveto{\pgfqpoint{2.733039in}{2.585172in}}{\pgfqpoint{2.728649in}{2.574573in}}{\pgfqpoint{2.728649in}{2.563522in}}%
\pgfpathcurveto{\pgfqpoint{2.728649in}{2.552472in}}{\pgfqpoint{2.733039in}{2.541873in}}{\pgfqpoint{2.740853in}{2.534060in}}%
\pgfpathcurveto{\pgfqpoint{2.748667in}{2.526246in}}{\pgfqpoint{2.759266in}{2.521856in}}{\pgfqpoint{2.770316in}{2.521856in}}%
\pgfpathclose%
\pgfusepath{stroke,fill}%
\end{pgfscope}%
\begin{pgfscope}%
\pgfpathrectangle{\pgfqpoint{0.481978in}{0.331635in}}{\pgfqpoint{4.960000in}{3.696000in}}%
\pgfusepath{clip}%
\pgfsetbuttcap%
\pgfsetroundjoin%
\definecolor{currentfill}{rgb}{1.000000,0.705882,0.509804}%
\pgfsetfillcolor{currentfill}%
\pgfsetlinewidth{0.481800pt}%
\definecolor{currentstroke}{rgb}{1.000000,1.000000,1.000000}%
\pgfsetstrokecolor{currentstroke}%
\pgfsetdash{}{0pt}%
\pgfpathmoveto{\pgfqpoint{1.391784in}{1.893301in}}%
\pgfpathcurveto{\pgfqpoint{1.402834in}{1.893301in}}{\pgfqpoint{1.413433in}{1.897691in}}{\pgfqpoint{1.421246in}{1.905505in}}%
\pgfpathcurveto{\pgfqpoint{1.429060in}{1.913318in}}{\pgfqpoint{1.433450in}{1.923917in}}{\pgfqpoint{1.433450in}{1.934968in}}%
\pgfpathcurveto{\pgfqpoint{1.433450in}{1.946018in}}{\pgfqpoint{1.429060in}{1.956617in}}{\pgfqpoint{1.421246in}{1.964430in}}%
\pgfpathcurveto{\pgfqpoint{1.413433in}{1.972244in}}{\pgfqpoint{1.402834in}{1.976634in}}{\pgfqpoint{1.391784in}{1.976634in}}%
\pgfpathcurveto{\pgfqpoint{1.380733in}{1.976634in}}{\pgfqpoint{1.370134in}{1.972244in}}{\pgfqpoint{1.362321in}{1.964430in}}%
\pgfpathcurveto{\pgfqpoint{1.354507in}{1.956617in}}{\pgfqpoint{1.350117in}{1.946018in}}{\pgfqpoint{1.350117in}{1.934968in}}%
\pgfpathcurveto{\pgfqpoint{1.350117in}{1.923917in}}{\pgfqpoint{1.354507in}{1.913318in}}{\pgfqpoint{1.362321in}{1.905505in}}%
\pgfpathcurveto{\pgfqpoint{1.370134in}{1.897691in}}{\pgfqpoint{1.380733in}{1.893301in}}{\pgfqpoint{1.391784in}{1.893301in}}%
\pgfpathclose%
\pgfusepath{stroke,fill}%
\end{pgfscope}%
\begin{pgfscope}%
\pgfpathrectangle{\pgfqpoint{0.481978in}{0.331635in}}{\pgfqpoint{4.960000in}{3.696000in}}%
\pgfusepath{clip}%
\pgfsetbuttcap%
\pgfsetroundjoin%
\definecolor{currentfill}{rgb}{1.000000,0.705882,0.509804}%
\pgfsetfillcolor{currentfill}%
\pgfsetlinewidth{0.481800pt}%
\definecolor{currentstroke}{rgb}{1.000000,1.000000,1.000000}%
\pgfsetstrokecolor{currentstroke}%
\pgfsetdash{}{0pt}%
\pgfpathmoveto{\pgfqpoint{1.870023in}{2.189897in}}%
\pgfpathcurveto{\pgfqpoint{1.881073in}{2.189897in}}{\pgfqpoint{1.891672in}{2.194288in}}{\pgfqpoint{1.899485in}{2.202101in}}%
\pgfpathcurveto{\pgfqpoint{1.907299in}{2.209915in}}{\pgfqpoint{1.911689in}{2.220514in}}{\pgfqpoint{1.911689in}{2.231564in}}%
\pgfpathcurveto{\pgfqpoint{1.911689in}{2.242614in}}{\pgfqpoint{1.907299in}{2.253213in}}{\pgfqpoint{1.899485in}{2.261027in}}%
\pgfpathcurveto{\pgfqpoint{1.891672in}{2.268841in}}{\pgfqpoint{1.881073in}{2.273231in}}{\pgfqpoint{1.870023in}{2.273231in}}%
\pgfpathcurveto{\pgfqpoint{1.858972in}{2.273231in}}{\pgfqpoint{1.848373in}{2.268841in}}{\pgfqpoint{1.840560in}{2.261027in}}%
\pgfpathcurveto{\pgfqpoint{1.832746in}{2.253213in}}{\pgfqpoint{1.828356in}{2.242614in}}{\pgfqpoint{1.828356in}{2.231564in}}%
\pgfpathcurveto{\pgfqpoint{1.828356in}{2.220514in}}{\pgfqpoint{1.832746in}{2.209915in}}{\pgfqpoint{1.840560in}{2.202101in}}%
\pgfpathcurveto{\pgfqpoint{1.848373in}{2.194288in}}{\pgfqpoint{1.858972in}{2.189897in}}{\pgfqpoint{1.870023in}{2.189897in}}%
\pgfpathclose%
\pgfusepath{stroke,fill}%
\end{pgfscope}%
\begin{pgfscope}%
\pgfpathrectangle{\pgfqpoint{0.481978in}{0.331635in}}{\pgfqpoint{4.960000in}{3.696000in}}%
\pgfusepath{clip}%
\pgfsetbuttcap%
\pgfsetroundjoin%
\definecolor{currentfill}{rgb}{1.000000,0.705882,0.509804}%
\pgfsetfillcolor{currentfill}%
\pgfsetlinewidth{0.481800pt}%
\definecolor{currentstroke}{rgb}{1.000000,1.000000,1.000000}%
\pgfsetstrokecolor{currentstroke}%
\pgfsetdash{}{0pt}%
\pgfpathmoveto{\pgfqpoint{3.278587in}{2.913480in}}%
\pgfpathcurveto{\pgfqpoint{3.289637in}{2.913480in}}{\pgfqpoint{3.300236in}{2.917870in}}{\pgfqpoint{3.308049in}{2.925683in}}%
\pgfpathcurveto{\pgfqpoint{3.315863in}{2.933497in}}{\pgfqpoint{3.320253in}{2.944096in}}{\pgfqpoint{3.320253in}{2.955146in}}%
\pgfpathcurveto{\pgfqpoint{3.320253in}{2.966196in}}{\pgfqpoint{3.315863in}{2.976795in}}{\pgfqpoint{3.308049in}{2.984609in}}%
\pgfpathcurveto{\pgfqpoint{3.300236in}{2.992423in}}{\pgfqpoint{3.289637in}{2.996813in}}{\pgfqpoint{3.278587in}{2.996813in}}%
\pgfpathcurveto{\pgfqpoint{3.267536in}{2.996813in}}{\pgfqpoint{3.256937in}{2.992423in}}{\pgfqpoint{3.249124in}{2.984609in}}%
\pgfpathcurveto{\pgfqpoint{3.241310in}{2.976795in}}{\pgfqpoint{3.236920in}{2.966196in}}{\pgfqpoint{3.236920in}{2.955146in}}%
\pgfpathcurveto{\pgfqpoint{3.236920in}{2.944096in}}{\pgfqpoint{3.241310in}{2.933497in}}{\pgfqpoint{3.249124in}{2.925683in}}%
\pgfpathcurveto{\pgfqpoint{3.256937in}{2.917870in}}{\pgfqpoint{3.267536in}{2.913480in}}{\pgfqpoint{3.278587in}{2.913480in}}%
\pgfpathclose%
\pgfusepath{stroke,fill}%
\end{pgfscope}%
\begin{pgfscope}%
\pgfpathrectangle{\pgfqpoint{0.481978in}{0.331635in}}{\pgfqpoint{4.960000in}{3.696000in}}%
\pgfusepath{clip}%
\pgfsetbuttcap%
\pgfsetroundjoin%
\definecolor{currentfill}{rgb}{1.000000,0.705882,0.509804}%
\pgfsetfillcolor{currentfill}%
\pgfsetlinewidth{0.481800pt}%
\definecolor{currentstroke}{rgb}{1.000000,1.000000,1.000000}%
\pgfsetstrokecolor{currentstroke}%
\pgfsetdash{}{0pt}%
\pgfpathmoveto{\pgfqpoint{2.052985in}{3.295011in}}%
\pgfpathcurveto{\pgfqpoint{2.064035in}{3.295011in}}{\pgfqpoint{2.074634in}{3.299401in}}{\pgfqpoint{2.082448in}{3.307214in}}%
\pgfpathcurveto{\pgfqpoint{2.090261in}{3.315028in}}{\pgfqpoint{2.094652in}{3.325627in}}{\pgfqpoint{2.094652in}{3.336677in}}%
\pgfpathcurveto{\pgfqpoint{2.094652in}{3.347727in}}{\pgfqpoint{2.090261in}{3.358326in}}{\pgfqpoint{2.082448in}{3.366140in}}%
\pgfpathcurveto{\pgfqpoint{2.074634in}{3.373954in}}{\pgfqpoint{2.064035in}{3.378344in}}{\pgfqpoint{2.052985in}{3.378344in}}%
\pgfpathcurveto{\pgfqpoint{2.041935in}{3.378344in}}{\pgfqpoint{2.031336in}{3.373954in}}{\pgfqpoint{2.023522in}{3.366140in}}%
\pgfpathcurveto{\pgfqpoint{2.015708in}{3.358326in}}{\pgfqpoint{2.011318in}{3.347727in}}{\pgfqpoint{2.011318in}{3.336677in}}%
\pgfpathcurveto{\pgfqpoint{2.011318in}{3.325627in}}{\pgfqpoint{2.015708in}{3.315028in}}{\pgfqpoint{2.023522in}{3.307214in}}%
\pgfpathcurveto{\pgfqpoint{2.031336in}{3.299401in}}{\pgfqpoint{2.041935in}{3.295011in}}{\pgfqpoint{2.052985in}{3.295011in}}%
\pgfpathclose%
\pgfusepath{stroke,fill}%
\end{pgfscope}%
\begin{pgfscope}%
\pgfpathrectangle{\pgfqpoint{0.481978in}{0.331635in}}{\pgfqpoint{4.960000in}{3.696000in}}%
\pgfusepath{clip}%
\pgfsetbuttcap%
\pgfsetroundjoin%
\definecolor{currentfill}{rgb}{1.000000,0.705882,0.509804}%
\pgfsetfillcolor{currentfill}%
\pgfsetlinewidth{0.481800pt}%
\definecolor{currentstroke}{rgb}{1.000000,1.000000,1.000000}%
\pgfsetstrokecolor{currentstroke}%
\pgfsetdash{}{0pt}%
\pgfpathmoveto{\pgfqpoint{2.446981in}{3.332705in}}%
\pgfpathcurveto{\pgfqpoint{2.458032in}{3.332705in}}{\pgfqpoint{2.468631in}{3.337095in}}{\pgfqpoint{2.476444in}{3.344908in}}%
\pgfpathcurveto{\pgfqpoint{2.484258in}{3.352722in}}{\pgfqpoint{2.488648in}{3.363321in}}{\pgfqpoint{2.488648in}{3.374371in}}%
\pgfpathcurveto{\pgfqpoint{2.488648in}{3.385421in}}{\pgfqpoint{2.484258in}{3.396020in}}{\pgfqpoint{2.476444in}{3.403834in}}%
\pgfpathcurveto{\pgfqpoint{2.468631in}{3.411648in}}{\pgfqpoint{2.458032in}{3.416038in}}{\pgfqpoint{2.446981in}{3.416038in}}%
\pgfpathcurveto{\pgfqpoint{2.435931in}{3.416038in}}{\pgfqpoint{2.425332in}{3.411648in}}{\pgfqpoint{2.417519in}{3.403834in}}%
\pgfpathcurveto{\pgfqpoint{2.409705in}{3.396020in}}{\pgfqpoint{2.405315in}{3.385421in}}{\pgfqpoint{2.405315in}{3.374371in}}%
\pgfpathcurveto{\pgfqpoint{2.405315in}{3.363321in}}{\pgfqpoint{2.409705in}{3.352722in}}{\pgfqpoint{2.417519in}{3.344908in}}%
\pgfpathcurveto{\pgfqpoint{2.425332in}{3.337095in}}{\pgfqpoint{2.435931in}{3.332705in}}{\pgfqpoint{2.446981in}{3.332705in}}%
\pgfpathclose%
\pgfusepath{stroke,fill}%
\end{pgfscope}%
\begin{pgfscope}%
\pgfpathrectangle{\pgfqpoint{0.481978in}{0.331635in}}{\pgfqpoint{4.960000in}{3.696000in}}%
\pgfusepath{clip}%
\pgfsetbuttcap%
\pgfsetroundjoin%
\definecolor{currentfill}{rgb}{1.000000,0.705882,0.509804}%
\pgfsetfillcolor{currentfill}%
\pgfsetlinewidth{0.481800pt}%
\definecolor{currentstroke}{rgb}{1.000000,1.000000,1.000000}%
\pgfsetstrokecolor{currentstroke}%
\pgfsetdash{}{0pt}%
\pgfpathmoveto{\pgfqpoint{0.935858in}{2.277294in}}%
\pgfpathcurveto{\pgfqpoint{0.946908in}{2.277294in}}{\pgfqpoint{0.957507in}{2.281684in}}{\pgfqpoint{0.965321in}{2.289497in}}%
\pgfpathcurveto{\pgfqpoint{0.973134in}{2.297311in}}{\pgfqpoint{0.977525in}{2.307910in}}{\pgfqpoint{0.977525in}{2.318960in}}%
\pgfpathcurveto{\pgfqpoint{0.977525in}{2.330010in}}{\pgfqpoint{0.973134in}{2.340609in}}{\pgfqpoint{0.965321in}{2.348423in}}%
\pgfpathcurveto{\pgfqpoint{0.957507in}{2.356237in}}{\pgfqpoint{0.946908in}{2.360627in}}{\pgfqpoint{0.935858in}{2.360627in}}%
\pgfpathcurveto{\pgfqpoint{0.924808in}{2.360627in}}{\pgfqpoint{0.914209in}{2.356237in}}{\pgfqpoint{0.906395in}{2.348423in}}%
\pgfpathcurveto{\pgfqpoint{0.898582in}{2.340609in}}{\pgfqpoint{0.894191in}{2.330010in}}{\pgfqpoint{0.894191in}{2.318960in}}%
\pgfpathcurveto{\pgfqpoint{0.894191in}{2.307910in}}{\pgfqpoint{0.898582in}{2.297311in}}{\pgfqpoint{0.906395in}{2.289497in}}%
\pgfpathcurveto{\pgfqpoint{0.914209in}{2.281684in}}{\pgfqpoint{0.924808in}{2.277294in}}{\pgfqpoint{0.935858in}{2.277294in}}%
\pgfpathclose%
\pgfusepath{stroke,fill}%
\end{pgfscope}%
\begin{pgfscope}%
\pgfpathrectangle{\pgfqpoint{0.481978in}{0.331635in}}{\pgfqpoint{4.960000in}{3.696000in}}%
\pgfusepath{clip}%
\pgfsetbuttcap%
\pgfsetroundjoin%
\definecolor{currentfill}{rgb}{1.000000,0.705882,0.509804}%
\pgfsetfillcolor{currentfill}%
\pgfsetlinewidth{0.481800pt}%
\definecolor{currentstroke}{rgb}{1.000000,1.000000,1.000000}%
\pgfsetstrokecolor{currentstroke}%
\pgfsetdash{}{0pt}%
\pgfpathmoveto{\pgfqpoint{4.276696in}{0.864714in}}%
\pgfpathcurveto{\pgfqpoint{4.287746in}{0.864714in}}{\pgfqpoint{4.298345in}{0.869104in}}{\pgfqpoint{4.306159in}{0.876918in}}%
\pgfpathcurveto{\pgfqpoint{4.313972in}{0.884731in}}{\pgfqpoint{4.318363in}{0.895330in}}{\pgfqpoint{4.318363in}{0.906380in}}%
\pgfpathcurveto{\pgfqpoint{4.318363in}{0.917431in}}{\pgfqpoint{4.313972in}{0.928030in}}{\pgfqpoint{4.306159in}{0.935843in}}%
\pgfpathcurveto{\pgfqpoint{4.298345in}{0.943657in}}{\pgfqpoint{4.287746in}{0.948047in}}{\pgfqpoint{4.276696in}{0.948047in}}%
\pgfpathcurveto{\pgfqpoint{4.265646in}{0.948047in}}{\pgfqpoint{4.255047in}{0.943657in}}{\pgfqpoint{4.247233in}{0.935843in}}%
\pgfpathcurveto{\pgfqpoint{4.239420in}{0.928030in}}{\pgfqpoint{4.235029in}{0.917431in}}{\pgfqpoint{4.235029in}{0.906380in}}%
\pgfpathcurveto{\pgfqpoint{4.235029in}{0.895330in}}{\pgfqpoint{4.239420in}{0.884731in}}{\pgfqpoint{4.247233in}{0.876918in}}%
\pgfpathcurveto{\pgfqpoint{4.255047in}{0.869104in}}{\pgfqpoint{4.265646in}{0.864714in}}{\pgfqpoint{4.276696in}{0.864714in}}%
\pgfpathclose%
\pgfusepath{stroke,fill}%
\end{pgfscope}%
\begin{pgfscope}%
\pgfpathrectangle{\pgfqpoint{0.481978in}{0.331635in}}{\pgfqpoint{4.960000in}{3.696000in}}%
\pgfusepath{clip}%
\pgfsetbuttcap%
\pgfsetroundjoin%
\definecolor{currentfill}{rgb}{1.000000,0.705882,0.509804}%
\pgfsetfillcolor{currentfill}%
\pgfsetlinewidth{0.481800pt}%
\definecolor{currentstroke}{rgb}{1.000000,1.000000,1.000000}%
\pgfsetstrokecolor{currentstroke}%
\pgfsetdash{}{0pt}%
\pgfpathmoveto{\pgfqpoint{2.575287in}{2.629923in}}%
\pgfpathcurveto{\pgfqpoint{2.586337in}{2.629923in}}{\pgfqpoint{2.596936in}{2.634313in}}{\pgfqpoint{2.604750in}{2.642127in}}%
\pgfpathcurveto{\pgfqpoint{2.612564in}{2.649940in}}{\pgfqpoint{2.616954in}{2.660539in}}{\pgfqpoint{2.616954in}{2.671589in}}%
\pgfpathcurveto{\pgfqpoint{2.616954in}{2.682639in}}{\pgfqpoint{2.612564in}{2.693238in}}{\pgfqpoint{2.604750in}{2.701052in}}%
\pgfpathcurveto{\pgfqpoint{2.596936in}{2.708866in}}{\pgfqpoint{2.586337in}{2.713256in}}{\pgfqpoint{2.575287in}{2.713256in}}%
\pgfpathcurveto{\pgfqpoint{2.564237in}{2.713256in}}{\pgfqpoint{2.553638in}{2.708866in}}{\pgfqpoint{2.545824in}{2.701052in}}%
\pgfpathcurveto{\pgfqpoint{2.538011in}{2.693238in}}{\pgfqpoint{2.533620in}{2.682639in}}{\pgfqpoint{2.533620in}{2.671589in}}%
\pgfpathcurveto{\pgfqpoint{2.533620in}{2.660539in}}{\pgfqpoint{2.538011in}{2.649940in}}{\pgfqpoint{2.545824in}{2.642127in}}%
\pgfpathcurveto{\pgfqpoint{2.553638in}{2.634313in}}{\pgfqpoint{2.564237in}{2.629923in}}{\pgfqpoint{2.575287in}{2.629923in}}%
\pgfpathclose%
\pgfusepath{stroke,fill}%
\end{pgfscope}%
\begin{pgfscope}%
\pgfpathrectangle{\pgfqpoint{0.481978in}{0.331635in}}{\pgfqpoint{4.960000in}{3.696000in}}%
\pgfusepath{clip}%
\pgfsetbuttcap%
\pgfsetroundjoin%
\definecolor{currentfill}{rgb}{1.000000,0.705882,0.509804}%
\pgfsetfillcolor{currentfill}%
\pgfsetlinewidth{0.481800pt}%
\definecolor{currentstroke}{rgb}{1.000000,1.000000,1.000000}%
\pgfsetstrokecolor{currentstroke}%
\pgfsetdash{}{0pt}%
\pgfpathmoveto{\pgfqpoint{3.924469in}{3.435968in}}%
\pgfpathcurveto{\pgfqpoint{3.935519in}{3.435968in}}{\pgfqpoint{3.946118in}{3.440359in}}{\pgfqpoint{3.953932in}{3.448172in}}%
\pgfpathcurveto{\pgfqpoint{3.961746in}{3.455986in}}{\pgfqpoint{3.966136in}{3.466585in}}{\pgfqpoint{3.966136in}{3.477635in}}%
\pgfpathcurveto{\pgfqpoint{3.966136in}{3.488685in}}{\pgfqpoint{3.961746in}{3.499284in}}{\pgfqpoint{3.953932in}{3.507098in}}%
\pgfpathcurveto{\pgfqpoint{3.946118in}{3.514912in}}{\pgfqpoint{3.935519in}{3.519302in}}{\pgfqpoint{3.924469in}{3.519302in}}%
\pgfpathcurveto{\pgfqpoint{3.913419in}{3.519302in}}{\pgfqpoint{3.902820in}{3.514912in}}{\pgfqpoint{3.895006in}{3.507098in}}%
\pgfpathcurveto{\pgfqpoint{3.887193in}{3.499284in}}{\pgfqpoint{3.882803in}{3.488685in}}{\pgfqpoint{3.882803in}{3.477635in}}%
\pgfpathcurveto{\pgfqpoint{3.882803in}{3.466585in}}{\pgfqpoint{3.887193in}{3.455986in}}{\pgfqpoint{3.895006in}{3.448172in}}%
\pgfpathcurveto{\pgfqpoint{3.902820in}{3.440359in}}{\pgfqpoint{3.913419in}{3.435968in}}{\pgfqpoint{3.924469in}{3.435968in}}%
\pgfpathclose%
\pgfusepath{stroke,fill}%
\end{pgfscope}%
\begin{pgfscope}%
\pgfpathrectangle{\pgfqpoint{0.481978in}{0.331635in}}{\pgfqpoint{4.960000in}{3.696000in}}%
\pgfusepath{clip}%
\pgfsetbuttcap%
\pgfsetroundjoin%
\definecolor{currentfill}{rgb}{1.000000,0.705882,0.509804}%
\pgfsetfillcolor{currentfill}%
\pgfsetlinewidth{0.481800pt}%
\definecolor{currentstroke}{rgb}{1.000000,1.000000,1.000000}%
\pgfsetstrokecolor{currentstroke}%
\pgfsetdash{}{0pt}%
\pgfpathmoveto{\pgfqpoint{1.226711in}{3.171012in}}%
\pgfpathcurveto{\pgfqpoint{1.237761in}{3.171012in}}{\pgfqpoint{1.248360in}{3.175402in}}{\pgfqpoint{1.256174in}{3.183216in}}%
\pgfpathcurveto{\pgfqpoint{1.263987in}{3.191029in}}{\pgfqpoint{1.268378in}{3.201628in}}{\pgfqpoint{1.268378in}{3.212679in}}%
\pgfpathcurveto{\pgfqpoint{1.268378in}{3.223729in}}{\pgfqpoint{1.263987in}{3.234328in}}{\pgfqpoint{1.256174in}{3.242141in}}%
\pgfpathcurveto{\pgfqpoint{1.248360in}{3.249955in}}{\pgfqpoint{1.237761in}{3.254345in}}{\pgfqpoint{1.226711in}{3.254345in}}%
\pgfpathcurveto{\pgfqpoint{1.215661in}{3.254345in}}{\pgfqpoint{1.205062in}{3.249955in}}{\pgfqpoint{1.197248in}{3.242141in}}%
\pgfpathcurveto{\pgfqpoint{1.189434in}{3.234328in}}{\pgfqpoint{1.185044in}{3.223729in}}{\pgfqpoint{1.185044in}{3.212679in}}%
\pgfpathcurveto{\pgfqpoint{1.185044in}{3.201628in}}{\pgfqpoint{1.189434in}{3.191029in}}{\pgfqpoint{1.197248in}{3.183216in}}%
\pgfpathcurveto{\pgfqpoint{1.205062in}{3.175402in}}{\pgfqpoint{1.215661in}{3.171012in}}{\pgfqpoint{1.226711in}{3.171012in}}%
\pgfpathclose%
\pgfusepath{stroke,fill}%
\end{pgfscope}%
\begin{pgfscope}%
\pgfpathrectangle{\pgfqpoint{0.481978in}{0.331635in}}{\pgfqpoint{4.960000in}{3.696000in}}%
\pgfusepath{clip}%
\pgfsetbuttcap%
\pgfsetroundjoin%
\definecolor{currentfill}{rgb}{1.000000,0.705882,0.509804}%
\pgfsetfillcolor{currentfill}%
\pgfsetlinewidth{0.481800pt}%
\definecolor{currentstroke}{rgb}{1.000000,1.000000,1.000000}%
\pgfsetstrokecolor{currentstroke}%
\pgfsetdash{}{0pt}%
\pgfpathmoveto{\pgfqpoint{1.663711in}{3.010898in}}%
\pgfpathcurveto{\pgfqpoint{1.674761in}{3.010898in}}{\pgfqpoint{1.685360in}{3.015288in}}{\pgfqpoint{1.693173in}{3.023101in}}%
\pgfpathcurveto{\pgfqpoint{1.700987in}{3.030915in}}{\pgfqpoint{1.705377in}{3.041514in}}{\pgfqpoint{1.705377in}{3.052564in}}%
\pgfpathcurveto{\pgfqpoint{1.705377in}{3.063614in}}{\pgfqpoint{1.700987in}{3.074213in}}{\pgfqpoint{1.693173in}{3.082027in}}%
\pgfpathcurveto{\pgfqpoint{1.685360in}{3.089841in}}{\pgfqpoint{1.674761in}{3.094231in}}{\pgfqpoint{1.663711in}{3.094231in}}%
\pgfpathcurveto{\pgfqpoint{1.652661in}{3.094231in}}{\pgfqpoint{1.642062in}{3.089841in}}{\pgfqpoint{1.634248in}{3.082027in}}%
\pgfpathcurveto{\pgfqpoint{1.626434in}{3.074213in}}{\pgfqpoint{1.622044in}{3.063614in}}{\pgfqpoint{1.622044in}{3.052564in}}%
\pgfpathcurveto{\pgfqpoint{1.622044in}{3.041514in}}{\pgfqpoint{1.626434in}{3.030915in}}{\pgfqpoint{1.634248in}{3.023101in}}%
\pgfpathcurveto{\pgfqpoint{1.642062in}{3.015288in}}{\pgfqpoint{1.652661in}{3.010898in}}{\pgfqpoint{1.663711in}{3.010898in}}%
\pgfpathclose%
\pgfusepath{stroke,fill}%
\end{pgfscope}%
\begin{pgfscope}%
\pgfpathrectangle{\pgfqpoint{0.481978in}{0.331635in}}{\pgfqpoint{4.960000in}{3.696000in}}%
\pgfusepath{clip}%
\pgfsetbuttcap%
\pgfsetroundjoin%
\definecolor{currentfill}{rgb}{1.000000,0.705882,0.509804}%
\pgfsetfillcolor{currentfill}%
\pgfsetlinewidth{0.481800pt}%
\definecolor{currentstroke}{rgb}{1.000000,1.000000,1.000000}%
\pgfsetstrokecolor{currentstroke}%
\pgfsetdash{}{0pt}%
\pgfpathmoveto{\pgfqpoint{1.438649in}{2.325618in}}%
\pgfpathcurveto{\pgfqpoint{1.449699in}{2.325618in}}{\pgfqpoint{1.460298in}{2.330008in}}{\pgfqpoint{1.468111in}{2.337822in}}%
\pgfpathcurveto{\pgfqpoint{1.475925in}{2.345635in}}{\pgfqpoint{1.480315in}{2.356234in}}{\pgfqpoint{1.480315in}{2.367284in}}%
\pgfpathcurveto{\pgfqpoint{1.480315in}{2.378334in}}{\pgfqpoint{1.475925in}{2.388933in}}{\pgfqpoint{1.468111in}{2.396747in}}%
\pgfpathcurveto{\pgfqpoint{1.460298in}{2.404561in}}{\pgfqpoint{1.449699in}{2.408951in}}{\pgfqpoint{1.438649in}{2.408951in}}%
\pgfpathcurveto{\pgfqpoint{1.427598in}{2.408951in}}{\pgfqpoint{1.416999in}{2.404561in}}{\pgfqpoint{1.409186in}{2.396747in}}%
\pgfpathcurveto{\pgfqpoint{1.401372in}{2.388933in}}{\pgfqpoint{1.396982in}{2.378334in}}{\pgfqpoint{1.396982in}{2.367284in}}%
\pgfpathcurveto{\pgfqpoint{1.396982in}{2.356234in}}{\pgfqpoint{1.401372in}{2.345635in}}{\pgfqpoint{1.409186in}{2.337822in}}%
\pgfpathcurveto{\pgfqpoint{1.416999in}{2.330008in}}{\pgfqpoint{1.427598in}{2.325618in}}{\pgfqpoint{1.438649in}{2.325618in}}%
\pgfpathclose%
\pgfusepath{stroke,fill}%
\end{pgfscope}%
\begin{pgfscope}%
\pgfpathrectangle{\pgfqpoint{0.481978in}{0.331635in}}{\pgfqpoint{4.960000in}{3.696000in}}%
\pgfusepath{clip}%
\pgfsetbuttcap%
\pgfsetroundjoin%
\definecolor{currentfill}{rgb}{1.000000,0.705882,0.509804}%
\pgfsetfillcolor{currentfill}%
\pgfsetlinewidth{0.481800pt}%
\definecolor{currentstroke}{rgb}{1.000000,1.000000,1.000000}%
\pgfsetstrokecolor{currentstroke}%
\pgfsetdash{}{0pt}%
\pgfpathmoveto{\pgfqpoint{2.409489in}{3.320710in}}%
\pgfpathcurveto{\pgfqpoint{2.420539in}{3.320710in}}{\pgfqpoint{2.431138in}{3.325100in}}{\pgfqpoint{2.438952in}{3.332914in}}%
\pgfpathcurveto{\pgfqpoint{2.446765in}{3.340728in}}{\pgfqpoint{2.451156in}{3.351327in}}{\pgfqpoint{2.451156in}{3.362377in}}%
\pgfpathcurveto{\pgfqpoint{2.451156in}{3.373427in}}{\pgfqpoint{2.446765in}{3.384026in}}{\pgfqpoint{2.438952in}{3.391840in}}%
\pgfpathcurveto{\pgfqpoint{2.431138in}{3.399653in}}{\pgfqpoint{2.420539in}{3.404043in}}{\pgfqpoint{2.409489in}{3.404043in}}%
\pgfpathcurveto{\pgfqpoint{2.398439in}{3.404043in}}{\pgfqpoint{2.387840in}{3.399653in}}{\pgfqpoint{2.380026in}{3.391840in}}%
\pgfpathcurveto{\pgfqpoint{2.372212in}{3.384026in}}{\pgfqpoint{2.367822in}{3.373427in}}{\pgfqpoint{2.367822in}{3.362377in}}%
\pgfpathcurveto{\pgfqpoint{2.367822in}{3.351327in}}{\pgfqpoint{2.372212in}{3.340728in}}{\pgfqpoint{2.380026in}{3.332914in}}%
\pgfpathcurveto{\pgfqpoint{2.387840in}{3.325100in}}{\pgfqpoint{2.398439in}{3.320710in}}{\pgfqpoint{2.409489in}{3.320710in}}%
\pgfpathclose%
\pgfusepath{stroke,fill}%
\end{pgfscope}%
\begin{pgfscope}%
\pgfpathrectangle{\pgfqpoint{0.481978in}{0.331635in}}{\pgfqpoint{4.960000in}{3.696000in}}%
\pgfusepath{clip}%
\pgfsetbuttcap%
\pgfsetroundjoin%
\definecolor{currentfill}{rgb}{1.000000,0.705882,0.509804}%
\pgfsetfillcolor{currentfill}%
\pgfsetlinewidth{0.481800pt}%
\definecolor{currentstroke}{rgb}{1.000000,1.000000,1.000000}%
\pgfsetstrokecolor{currentstroke}%
\pgfsetdash{}{0pt}%
\pgfpathmoveto{\pgfqpoint{3.659565in}{3.043762in}}%
\pgfpathcurveto{\pgfqpoint{3.670615in}{3.043762in}}{\pgfqpoint{3.681214in}{3.048153in}}{\pgfqpoint{3.689027in}{3.055966in}}%
\pgfpathcurveto{\pgfqpoint{3.696841in}{3.063780in}}{\pgfqpoint{3.701231in}{3.074379in}}{\pgfqpoint{3.701231in}{3.085429in}}%
\pgfpathcurveto{\pgfqpoint{3.701231in}{3.096479in}}{\pgfqpoint{3.696841in}{3.107078in}}{\pgfqpoint{3.689027in}{3.114892in}}%
\pgfpathcurveto{\pgfqpoint{3.681214in}{3.122705in}}{\pgfqpoint{3.670615in}{3.127096in}}{\pgfqpoint{3.659565in}{3.127096in}}%
\pgfpathcurveto{\pgfqpoint{3.648515in}{3.127096in}}{\pgfqpoint{3.637915in}{3.122705in}}{\pgfqpoint{3.630102in}{3.114892in}}%
\pgfpathcurveto{\pgfqpoint{3.622288in}{3.107078in}}{\pgfqpoint{3.617898in}{3.096479in}}{\pgfqpoint{3.617898in}{3.085429in}}%
\pgfpathcurveto{\pgfqpoint{3.617898in}{3.074379in}}{\pgfqpoint{3.622288in}{3.063780in}}{\pgfqpoint{3.630102in}{3.055966in}}%
\pgfpathcurveto{\pgfqpoint{3.637915in}{3.048153in}}{\pgfqpoint{3.648515in}{3.043762in}}{\pgfqpoint{3.659565in}{3.043762in}}%
\pgfpathclose%
\pgfusepath{stroke,fill}%
\end{pgfscope}%
\begin{pgfscope}%
\pgfpathrectangle{\pgfqpoint{0.481978in}{0.331635in}}{\pgfqpoint{4.960000in}{3.696000in}}%
\pgfusepath{clip}%
\pgfsetbuttcap%
\pgfsetroundjoin%
\definecolor{currentfill}{rgb}{1.000000,0.705882,0.509804}%
\pgfsetfillcolor{currentfill}%
\pgfsetlinewidth{0.481800pt}%
\definecolor{currentstroke}{rgb}{1.000000,1.000000,1.000000}%
\pgfsetstrokecolor{currentstroke}%
\pgfsetdash{}{0pt}%
\pgfpathmoveto{\pgfqpoint{1.402680in}{1.189449in}}%
\pgfpathcurveto{\pgfqpoint{1.413730in}{1.189449in}}{\pgfqpoint{1.424329in}{1.193840in}}{\pgfqpoint{1.432143in}{1.201653in}}%
\pgfpathcurveto{\pgfqpoint{1.439956in}{1.209467in}}{\pgfqpoint{1.444346in}{1.220066in}}{\pgfqpoint{1.444346in}{1.231116in}}%
\pgfpathcurveto{\pgfqpoint{1.444346in}{1.242166in}}{\pgfqpoint{1.439956in}{1.252765in}}{\pgfqpoint{1.432143in}{1.260579in}}%
\pgfpathcurveto{\pgfqpoint{1.424329in}{1.268392in}}{\pgfqpoint{1.413730in}{1.272783in}}{\pgfqpoint{1.402680in}{1.272783in}}%
\pgfpathcurveto{\pgfqpoint{1.391630in}{1.272783in}}{\pgfqpoint{1.381031in}{1.268392in}}{\pgfqpoint{1.373217in}{1.260579in}}%
\pgfpathcurveto{\pgfqpoint{1.365403in}{1.252765in}}{\pgfqpoint{1.361013in}{1.242166in}}{\pgfqpoint{1.361013in}{1.231116in}}%
\pgfpathcurveto{\pgfqpoint{1.361013in}{1.220066in}}{\pgfqpoint{1.365403in}{1.209467in}}{\pgfqpoint{1.373217in}{1.201653in}}%
\pgfpathcurveto{\pgfqpoint{1.381031in}{1.193840in}}{\pgfqpoint{1.391630in}{1.189449in}}{\pgfqpoint{1.402680in}{1.189449in}}%
\pgfpathclose%
\pgfusepath{stroke,fill}%
\end{pgfscope}%
\begin{pgfscope}%
\pgfpathrectangle{\pgfqpoint{0.481978in}{0.331635in}}{\pgfqpoint{4.960000in}{3.696000in}}%
\pgfusepath{clip}%
\pgfsetbuttcap%
\pgfsetroundjoin%
\definecolor{currentfill}{rgb}{1.000000,0.705882,0.509804}%
\pgfsetfillcolor{currentfill}%
\pgfsetlinewidth{0.481800pt}%
\definecolor{currentstroke}{rgb}{1.000000,1.000000,1.000000}%
\pgfsetstrokecolor{currentstroke}%
\pgfsetdash{}{0pt}%
\pgfpathmoveto{\pgfqpoint{3.170837in}{2.429995in}}%
\pgfpathcurveto{\pgfqpoint{3.181887in}{2.429995in}}{\pgfqpoint{3.192486in}{2.434385in}}{\pgfqpoint{3.200300in}{2.442198in}}%
\pgfpathcurveto{\pgfqpoint{3.208113in}{2.450012in}}{\pgfqpoint{3.212504in}{2.460611in}}{\pgfqpoint{3.212504in}{2.471661in}}%
\pgfpathcurveto{\pgfqpoint{3.212504in}{2.482711in}}{\pgfqpoint{3.208113in}{2.493310in}}{\pgfqpoint{3.200300in}{2.501124in}}%
\pgfpathcurveto{\pgfqpoint{3.192486in}{2.508938in}}{\pgfqpoint{3.181887in}{2.513328in}}{\pgfqpoint{3.170837in}{2.513328in}}%
\pgfpathcurveto{\pgfqpoint{3.159787in}{2.513328in}}{\pgfqpoint{3.149188in}{2.508938in}}{\pgfqpoint{3.141374in}{2.501124in}}%
\pgfpathcurveto{\pgfqpoint{3.133561in}{2.493310in}}{\pgfqpoint{3.129170in}{2.482711in}}{\pgfqpoint{3.129170in}{2.471661in}}%
\pgfpathcurveto{\pgfqpoint{3.129170in}{2.460611in}}{\pgfqpoint{3.133561in}{2.450012in}}{\pgfqpoint{3.141374in}{2.442198in}}%
\pgfpathcurveto{\pgfqpoint{3.149188in}{2.434385in}}{\pgfqpoint{3.159787in}{2.429995in}}{\pgfqpoint{3.170837in}{2.429995in}}%
\pgfpathclose%
\pgfusepath{stroke,fill}%
\end{pgfscope}%
\begin{pgfscope}%
\pgfpathrectangle{\pgfqpoint{0.481978in}{0.331635in}}{\pgfqpoint{4.960000in}{3.696000in}}%
\pgfusepath{clip}%
\pgfsetbuttcap%
\pgfsetroundjoin%
\definecolor{currentfill}{rgb}{1.000000,0.705882,0.509804}%
\pgfsetfillcolor{currentfill}%
\pgfsetlinewidth{0.481800pt}%
\definecolor{currentstroke}{rgb}{1.000000,1.000000,1.000000}%
\pgfsetstrokecolor{currentstroke}%
\pgfsetdash{}{0pt}%
\pgfpathmoveto{\pgfqpoint{1.834442in}{1.651663in}}%
\pgfpathcurveto{\pgfqpoint{1.845492in}{1.651663in}}{\pgfqpoint{1.856091in}{1.656053in}}{\pgfqpoint{1.863905in}{1.663867in}}%
\pgfpathcurveto{\pgfqpoint{1.871718in}{1.671680in}}{\pgfqpoint{1.876109in}{1.682279in}}{\pgfqpoint{1.876109in}{1.693329in}}%
\pgfpathcurveto{\pgfqpoint{1.876109in}{1.704380in}}{\pgfqpoint{1.871718in}{1.714979in}}{\pgfqpoint{1.863905in}{1.722792in}}%
\pgfpathcurveto{\pgfqpoint{1.856091in}{1.730606in}}{\pgfqpoint{1.845492in}{1.734996in}}{\pgfqpoint{1.834442in}{1.734996in}}%
\pgfpathcurveto{\pgfqpoint{1.823392in}{1.734996in}}{\pgfqpoint{1.812793in}{1.730606in}}{\pgfqpoint{1.804979in}{1.722792in}}%
\pgfpathcurveto{\pgfqpoint{1.797166in}{1.714979in}}{\pgfqpoint{1.792775in}{1.704380in}}{\pgfqpoint{1.792775in}{1.693329in}}%
\pgfpathcurveto{\pgfqpoint{1.792775in}{1.682279in}}{\pgfqpoint{1.797166in}{1.671680in}}{\pgfqpoint{1.804979in}{1.663867in}}%
\pgfpathcurveto{\pgfqpoint{1.812793in}{1.656053in}}{\pgfqpoint{1.823392in}{1.651663in}}{\pgfqpoint{1.834442in}{1.651663in}}%
\pgfpathclose%
\pgfusepath{stroke,fill}%
\end{pgfscope}%
\begin{pgfscope}%
\pgfpathrectangle{\pgfqpoint{0.481978in}{0.331635in}}{\pgfqpoint{4.960000in}{3.696000in}}%
\pgfusepath{clip}%
\pgfsetbuttcap%
\pgfsetroundjoin%
\definecolor{currentfill}{rgb}{1.000000,0.705882,0.509804}%
\pgfsetfillcolor{currentfill}%
\pgfsetlinewidth{0.481800pt}%
\definecolor{currentstroke}{rgb}{1.000000,1.000000,1.000000}%
\pgfsetstrokecolor{currentstroke}%
\pgfsetdash{}{0pt}%
\pgfpathmoveto{\pgfqpoint{1.780214in}{2.800691in}}%
\pgfpathcurveto{\pgfqpoint{1.791264in}{2.800691in}}{\pgfqpoint{1.801863in}{2.805081in}}{\pgfqpoint{1.809677in}{2.812895in}}%
\pgfpathcurveto{\pgfqpoint{1.817491in}{2.820709in}}{\pgfqpoint{1.821881in}{2.831308in}}{\pgfqpoint{1.821881in}{2.842358in}}%
\pgfpathcurveto{\pgfqpoint{1.821881in}{2.853408in}}{\pgfqpoint{1.817491in}{2.864007in}}{\pgfqpoint{1.809677in}{2.871820in}}%
\pgfpathcurveto{\pgfqpoint{1.801863in}{2.879634in}}{\pgfqpoint{1.791264in}{2.884024in}}{\pgfqpoint{1.780214in}{2.884024in}}%
\pgfpathcurveto{\pgfqpoint{1.769164in}{2.884024in}}{\pgfqpoint{1.758565in}{2.879634in}}{\pgfqpoint{1.750751in}{2.871820in}}%
\pgfpathcurveto{\pgfqpoint{1.742938in}{2.864007in}}{\pgfqpoint{1.738547in}{2.853408in}}{\pgfqpoint{1.738547in}{2.842358in}}%
\pgfpathcurveto{\pgfqpoint{1.738547in}{2.831308in}}{\pgfqpoint{1.742938in}{2.820709in}}{\pgfqpoint{1.750751in}{2.812895in}}%
\pgfpathcurveto{\pgfqpoint{1.758565in}{2.805081in}}{\pgfqpoint{1.769164in}{2.800691in}}{\pgfqpoint{1.780214in}{2.800691in}}%
\pgfpathclose%
\pgfusepath{stroke,fill}%
\end{pgfscope}%
\begin{pgfscope}%
\pgfpathrectangle{\pgfqpoint{0.481978in}{0.331635in}}{\pgfqpoint{4.960000in}{3.696000in}}%
\pgfusepath{clip}%
\pgfsetbuttcap%
\pgfsetroundjoin%
\definecolor{currentfill}{rgb}{1.000000,0.705882,0.509804}%
\pgfsetfillcolor{currentfill}%
\pgfsetlinewidth{0.481800pt}%
\definecolor{currentstroke}{rgb}{1.000000,1.000000,1.000000}%
\pgfsetstrokecolor{currentstroke}%
\pgfsetdash{}{0pt}%
\pgfpathmoveto{\pgfqpoint{2.618562in}{3.134648in}}%
\pgfpathcurveto{\pgfqpoint{2.629612in}{3.134648in}}{\pgfqpoint{2.640211in}{3.139039in}}{\pgfqpoint{2.648025in}{3.146852in}}%
\pgfpathcurveto{\pgfqpoint{2.655838in}{3.154666in}}{\pgfqpoint{2.660229in}{3.165265in}}{\pgfqpoint{2.660229in}{3.176315in}}%
\pgfpathcurveto{\pgfqpoint{2.660229in}{3.187365in}}{\pgfqpoint{2.655838in}{3.197964in}}{\pgfqpoint{2.648025in}{3.205778in}}%
\pgfpathcurveto{\pgfqpoint{2.640211in}{3.213591in}}{\pgfqpoint{2.629612in}{3.217982in}}{\pgfqpoint{2.618562in}{3.217982in}}%
\pgfpathcurveto{\pgfqpoint{2.607512in}{3.217982in}}{\pgfqpoint{2.596913in}{3.213591in}}{\pgfqpoint{2.589099in}{3.205778in}}%
\pgfpathcurveto{\pgfqpoint{2.581285in}{3.197964in}}{\pgfqpoint{2.576895in}{3.187365in}}{\pgfqpoint{2.576895in}{3.176315in}}%
\pgfpathcurveto{\pgfqpoint{2.576895in}{3.165265in}}{\pgfqpoint{2.581285in}{3.154666in}}{\pgfqpoint{2.589099in}{3.146852in}}%
\pgfpathcurveto{\pgfqpoint{2.596913in}{3.139039in}}{\pgfqpoint{2.607512in}{3.134648in}}{\pgfqpoint{2.618562in}{3.134648in}}%
\pgfpathclose%
\pgfusepath{stroke,fill}%
\end{pgfscope}%
\begin{pgfscope}%
\pgfpathrectangle{\pgfqpoint{0.481978in}{0.331635in}}{\pgfqpoint{4.960000in}{3.696000in}}%
\pgfusepath{clip}%
\pgfsetbuttcap%
\pgfsetroundjoin%
\definecolor{currentfill}{rgb}{1.000000,0.705882,0.509804}%
\pgfsetfillcolor{currentfill}%
\pgfsetlinewidth{0.481800pt}%
\definecolor{currentstroke}{rgb}{1.000000,1.000000,1.000000}%
\pgfsetstrokecolor{currentstroke}%
\pgfsetdash{}{0pt}%
\pgfpathmoveto{\pgfqpoint{1.816511in}{1.914868in}}%
\pgfpathcurveto{\pgfqpoint{1.827561in}{1.914868in}}{\pgfqpoint{1.838160in}{1.919258in}}{\pgfqpoint{1.845973in}{1.927072in}}%
\pgfpathcurveto{\pgfqpoint{1.853787in}{1.934886in}}{\pgfqpoint{1.858177in}{1.945485in}}{\pgfqpoint{1.858177in}{1.956535in}}%
\pgfpathcurveto{\pgfqpoint{1.858177in}{1.967585in}}{\pgfqpoint{1.853787in}{1.978184in}}{\pgfqpoint{1.845973in}{1.985998in}}%
\pgfpathcurveto{\pgfqpoint{1.838160in}{1.993811in}}{\pgfqpoint{1.827561in}{1.998201in}}{\pgfqpoint{1.816511in}{1.998201in}}%
\pgfpathcurveto{\pgfqpoint{1.805461in}{1.998201in}}{\pgfqpoint{1.794861in}{1.993811in}}{\pgfqpoint{1.787048in}{1.985998in}}%
\pgfpathcurveto{\pgfqpoint{1.779234in}{1.978184in}}{\pgfqpoint{1.774844in}{1.967585in}}{\pgfqpoint{1.774844in}{1.956535in}}%
\pgfpathcurveto{\pgfqpoint{1.774844in}{1.945485in}}{\pgfqpoint{1.779234in}{1.934886in}}{\pgfqpoint{1.787048in}{1.927072in}}%
\pgfpathcurveto{\pgfqpoint{1.794861in}{1.919258in}}{\pgfqpoint{1.805461in}{1.914868in}}{\pgfqpoint{1.816511in}{1.914868in}}%
\pgfpathclose%
\pgfusepath{stroke,fill}%
\end{pgfscope}%
\begin{pgfscope}%
\pgfpathrectangle{\pgfqpoint{0.481978in}{0.331635in}}{\pgfqpoint{4.960000in}{3.696000in}}%
\pgfusepath{clip}%
\pgfsetbuttcap%
\pgfsetroundjoin%
\definecolor{currentfill}{rgb}{1.000000,0.705882,0.509804}%
\pgfsetfillcolor{currentfill}%
\pgfsetlinewidth{0.481800pt}%
\definecolor{currentstroke}{rgb}{1.000000,1.000000,1.000000}%
\pgfsetstrokecolor{currentstroke}%
\pgfsetdash{}{0pt}%
\pgfpathmoveto{\pgfqpoint{2.749742in}{3.193312in}}%
\pgfpathcurveto{\pgfqpoint{2.760792in}{3.193312in}}{\pgfqpoint{2.771391in}{3.197703in}}{\pgfqpoint{2.779205in}{3.205516in}}%
\pgfpathcurveto{\pgfqpoint{2.787019in}{3.213330in}}{\pgfqpoint{2.791409in}{3.223929in}}{\pgfqpoint{2.791409in}{3.234979in}}%
\pgfpathcurveto{\pgfqpoint{2.791409in}{3.246029in}}{\pgfqpoint{2.787019in}{3.256628in}}{\pgfqpoint{2.779205in}{3.264442in}}%
\pgfpathcurveto{\pgfqpoint{2.771391in}{3.272255in}}{\pgfqpoint{2.760792in}{3.276646in}}{\pgfqpoint{2.749742in}{3.276646in}}%
\pgfpathcurveto{\pgfqpoint{2.738692in}{3.276646in}}{\pgfqpoint{2.728093in}{3.272255in}}{\pgfqpoint{2.720279in}{3.264442in}}%
\pgfpathcurveto{\pgfqpoint{2.712466in}{3.256628in}}{\pgfqpoint{2.708076in}{3.246029in}}{\pgfqpoint{2.708076in}{3.234979in}}%
\pgfpathcurveto{\pgfqpoint{2.708076in}{3.223929in}}{\pgfqpoint{2.712466in}{3.213330in}}{\pgfqpoint{2.720279in}{3.205516in}}%
\pgfpathcurveto{\pgfqpoint{2.728093in}{3.197703in}}{\pgfqpoint{2.738692in}{3.193312in}}{\pgfqpoint{2.749742in}{3.193312in}}%
\pgfpathclose%
\pgfusepath{stroke,fill}%
\end{pgfscope}%
\begin{pgfscope}%
\pgfpathrectangle{\pgfqpoint{0.481978in}{0.331635in}}{\pgfqpoint{4.960000in}{3.696000in}}%
\pgfusepath{clip}%
\pgfsetbuttcap%
\pgfsetroundjoin%
\definecolor{currentfill}{rgb}{1.000000,0.705882,0.509804}%
\pgfsetfillcolor{currentfill}%
\pgfsetlinewidth{0.481800pt}%
\definecolor{currentstroke}{rgb}{1.000000,1.000000,1.000000}%
\pgfsetstrokecolor{currentstroke}%
\pgfsetdash{}{0pt}%
\pgfpathmoveto{\pgfqpoint{1.855378in}{1.681475in}}%
\pgfpathcurveto{\pgfqpoint{1.866429in}{1.681475in}}{\pgfqpoint{1.877028in}{1.685865in}}{\pgfqpoint{1.884841in}{1.693679in}}%
\pgfpathcurveto{\pgfqpoint{1.892655in}{1.701492in}}{\pgfqpoint{1.897045in}{1.712092in}}{\pgfqpoint{1.897045in}{1.723142in}}%
\pgfpathcurveto{\pgfqpoint{1.897045in}{1.734192in}}{\pgfqpoint{1.892655in}{1.744791in}}{\pgfqpoint{1.884841in}{1.752604in}}%
\pgfpathcurveto{\pgfqpoint{1.877028in}{1.760418in}}{\pgfqpoint{1.866429in}{1.764808in}}{\pgfqpoint{1.855378in}{1.764808in}}%
\pgfpathcurveto{\pgfqpoint{1.844328in}{1.764808in}}{\pgfqpoint{1.833729in}{1.760418in}}{\pgfqpoint{1.825916in}{1.752604in}}%
\pgfpathcurveto{\pgfqpoint{1.818102in}{1.744791in}}{\pgfqpoint{1.813712in}{1.734192in}}{\pgfqpoint{1.813712in}{1.723142in}}%
\pgfpathcurveto{\pgfqpoint{1.813712in}{1.712092in}}{\pgfqpoint{1.818102in}{1.701492in}}{\pgfqpoint{1.825916in}{1.693679in}}%
\pgfpathcurveto{\pgfqpoint{1.833729in}{1.685865in}}{\pgfqpoint{1.844328in}{1.681475in}}{\pgfqpoint{1.855378in}{1.681475in}}%
\pgfpathclose%
\pgfusepath{stroke,fill}%
\end{pgfscope}%
\begin{pgfscope}%
\pgfpathrectangle{\pgfqpoint{0.481978in}{0.331635in}}{\pgfqpoint{4.960000in}{3.696000in}}%
\pgfusepath{clip}%
\pgfsetbuttcap%
\pgfsetroundjoin%
\definecolor{currentfill}{rgb}{1.000000,0.705882,0.509804}%
\pgfsetfillcolor{currentfill}%
\pgfsetlinewidth{0.481800pt}%
\definecolor{currentstroke}{rgb}{1.000000,1.000000,1.000000}%
\pgfsetstrokecolor{currentstroke}%
\pgfsetdash{}{0pt}%
\pgfpathmoveto{\pgfqpoint{0.830402in}{2.287986in}}%
\pgfpathcurveto{\pgfqpoint{0.841452in}{2.287986in}}{\pgfqpoint{0.852051in}{2.292376in}}{\pgfqpoint{0.859865in}{2.300190in}}%
\pgfpathcurveto{\pgfqpoint{0.867678in}{2.308004in}}{\pgfqpoint{0.872069in}{2.318603in}}{\pgfqpoint{0.872069in}{2.329653in}}%
\pgfpathcurveto{\pgfqpoint{0.872069in}{2.340703in}}{\pgfqpoint{0.867678in}{2.351302in}}{\pgfqpoint{0.859865in}{2.359116in}}%
\pgfpathcurveto{\pgfqpoint{0.852051in}{2.366929in}}{\pgfqpoint{0.841452in}{2.371319in}}{\pgfqpoint{0.830402in}{2.371319in}}%
\pgfpathcurveto{\pgfqpoint{0.819352in}{2.371319in}}{\pgfqpoint{0.808753in}{2.366929in}}{\pgfqpoint{0.800939in}{2.359116in}}%
\pgfpathcurveto{\pgfqpoint{0.793126in}{2.351302in}}{\pgfqpoint{0.788735in}{2.340703in}}{\pgfqpoint{0.788735in}{2.329653in}}%
\pgfpathcurveto{\pgfqpoint{0.788735in}{2.318603in}}{\pgfqpoint{0.793126in}{2.308004in}}{\pgfqpoint{0.800939in}{2.300190in}}%
\pgfpathcurveto{\pgfqpoint{0.808753in}{2.292376in}}{\pgfqpoint{0.819352in}{2.287986in}}{\pgfqpoint{0.830402in}{2.287986in}}%
\pgfpathclose%
\pgfusepath{stroke,fill}%
\end{pgfscope}%
\begin{pgfscope}%
\pgfpathrectangle{\pgfqpoint{0.481978in}{0.331635in}}{\pgfqpoint{4.960000in}{3.696000in}}%
\pgfusepath{clip}%
\pgfsetbuttcap%
\pgfsetroundjoin%
\definecolor{currentfill}{rgb}{1.000000,0.705882,0.509804}%
\pgfsetfillcolor{currentfill}%
\pgfsetlinewidth{0.481800pt}%
\definecolor{currentstroke}{rgb}{1.000000,1.000000,1.000000}%
\pgfsetstrokecolor{currentstroke}%
\pgfsetdash{}{0pt}%
\pgfpathmoveto{\pgfqpoint{2.483847in}{2.521586in}}%
\pgfpathcurveto{\pgfqpoint{2.494897in}{2.521586in}}{\pgfqpoint{2.505496in}{2.525977in}}{\pgfqpoint{2.513310in}{2.533790in}}%
\pgfpathcurveto{\pgfqpoint{2.521123in}{2.541604in}}{\pgfqpoint{2.525513in}{2.552203in}}{\pgfqpoint{2.525513in}{2.563253in}}%
\pgfpathcurveto{\pgfqpoint{2.525513in}{2.574303in}}{\pgfqpoint{2.521123in}{2.584902in}}{\pgfqpoint{2.513310in}{2.592716in}}%
\pgfpathcurveto{\pgfqpoint{2.505496in}{2.600529in}}{\pgfqpoint{2.494897in}{2.604920in}}{\pgfqpoint{2.483847in}{2.604920in}}%
\pgfpathcurveto{\pgfqpoint{2.472797in}{2.604920in}}{\pgfqpoint{2.462198in}{2.600529in}}{\pgfqpoint{2.454384in}{2.592716in}}%
\pgfpathcurveto{\pgfqpoint{2.446570in}{2.584902in}}{\pgfqpoint{2.442180in}{2.574303in}}{\pgfqpoint{2.442180in}{2.563253in}}%
\pgfpathcurveto{\pgfqpoint{2.442180in}{2.552203in}}{\pgfqpoint{2.446570in}{2.541604in}}{\pgfqpoint{2.454384in}{2.533790in}}%
\pgfpathcurveto{\pgfqpoint{2.462198in}{2.525977in}}{\pgfqpoint{2.472797in}{2.521586in}}{\pgfqpoint{2.483847in}{2.521586in}}%
\pgfpathclose%
\pgfusepath{stroke,fill}%
\end{pgfscope}%
\begin{pgfscope}%
\pgfpathrectangle{\pgfqpoint{0.481978in}{0.331635in}}{\pgfqpoint{4.960000in}{3.696000in}}%
\pgfusepath{clip}%
\pgfsetbuttcap%
\pgfsetroundjoin%
\definecolor{currentfill}{rgb}{1.000000,0.705882,0.509804}%
\pgfsetfillcolor{currentfill}%
\pgfsetlinewidth{0.481800pt}%
\definecolor{currentstroke}{rgb}{1.000000,1.000000,1.000000}%
\pgfsetstrokecolor{currentstroke}%
\pgfsetdash{}{0pt}%
\pgfpathmoveto{\pgfqpoint{2.443354in}{2.788369in}}%
\pgfpathcurveto{\pgfqpoint{2.454404in}{2.788369in}}{\pgfqpoint{2.465003in}{2.792759in}}{\pgfqpoint{2.472817in}{2.800573in}}%
\pgfpathcurveto{\pgfqpoint{2.480630in}{2.808386in}}{\pgfqpoint{2.485020in}{2.818985in}}{\pgfqpoint{2.485020in}{2.830036in}}%
\pgfpathcurveto{\pgfqpoint{2.485020in}{2.841086in}}{\pgfqpoint{2.480630in}{2.851685in}}{\pgfqpoint{2.472817in}{2.859498in}}%
\pgfpathcurveto{\pgfqpoint{2.465003in}{2.867312in}}{\pgfqpoint{2.454404in}{2.871702in}}{\pgfqpoint{2.443354in}{2.871702in}}%
\pgfpathcurveto{\pgfqpoint{2.432304in}{2.871702in}}{\pgfqpoint{2.421705in}{2.867312in}}{\pgfqpoint{2.413891in}{2.859498in}}%
\pgfpathcurveto{\pgfqpoint{2.406077in}{2.851685in}}{\pgfqpoint{2.401687in}{2.841086in}}{\pgfqpoint{2.401687in}{2.830036in}}%
\pgfpathcurveto{\pgfqpoint{2.401687in}{2.818985in}}{\pgfqpoint{2.406077in}{2.808386in}}{\pgfqpoint{2.413891in}{2.800573in}}%
\pgfpathcurveto{\pgfqpoint{2.421705in}{2.792759in}}{\pgfqpoint{2.432304in}{2.788369in}}{\pgfqpoint{2.443354in}{2.788369in}}%
\pgfpathclose%
\pgfusepath{stroke,fill}%
\end{pgfscope}%
\begin{pgfscope}%
\pgfpathrectangle{\pgfqpoint{0.481978in}{0.331635in}}{\pgfqpoint{4.960000in}{3.696000in}}%
\pgfusepath{clip}%
\pgfsetbuttcap%
\pgfsetroundjoin%
\definecolor{currentfill}{rgb}{1.000000,0.705882,0.509804}%
\pgfsetfillcolor{currentfill}%
\pgfsetlinewidth{0.481800pt}%
\definecolor{currentstroke}{rgb}{1.000000,1.000000,1.000000}%
\pgfsetstrokecolor{currentstroke}%
\pgfsetdash{}{0pt}%
\pgfpathmoveto{\pgfqpoint{4.243695in}{0.713542in}}%
\pgfpathcurveto{\pgfqpoint{4.254746in}{0.713542in}}{\pgfqpoint{4.265345in}{0.717932in}}{\pgfqpoint{4.273158in}{0.725746in}}%
\pgfpathcurveto{\pgfqpoint{4.280972in}{0.733559in}}{\pgfqpoint{4.285362in}{0.744159in}}{\pgfqpoint{4.285362in}{0.755209in}}%
\pgfpathcurveto{\pgfqpoint{4.285362in}{0.766259in}}{\pgfqpoint{4.280972in}{0.776858in}}{\pgfqpoint{4.273158in}{0.784671in}}%
\pgfpathcurveto{\pgfqpoint{4.265345in}{0.792485in}}{\pgfqpoint{4.254746in}{0.796875in}}{\pgfqpoint{4.243695in}{0.796875in}}%
\pgfpathcurveto{\pgfqpoint{4.232645in}{0.796875in}}{\pgfqpoint{4.222046in}{0.792485in}}{\pgfqpoint{4.214233in}{0.784671in}}%
\pgfpathcurveto{\pgfqpoint{4.206419in}{0.776858in}}{\pgfqpoint{4.202029in}{0.766259in}}{\pgfqpoint{4.202029in}{0.755209in}}%
\pgfpathcurveto{\pgfqpoint{4.202029in}{0.744159in}}{\pgfqpoint{4.206419in}{0.733559in}}{\pgfqpoint{4.214233in}{0.725746in}}%
\pgfpathcurveto{\pgfqpoint{4.222046in}{0.717932in}}{\pgfqpoint{4.232645in}{0.713542in}}{\pgfqpoint{4.243695in}{0.713542in}}%
\pgfpathclose%
\pgfusepath{stroke,fill}%
\end{pgfscope}%
\begin{pgfscope}%
\pgfpathrectangle{\pgfqpoint{0.481978in}{0.331635in}}{\pgfqpoint{4.960000in}{3.696000in}}%
\pgfusepath{clip}%
\pgfsetbuttcap%
\pgfsetroundjoin%
\definecolor{currentfill}{rgb}{1.000000,0.705882,0.509804}%
\pgfsetfillcolor{currentfill}%
\pgfsetlinewidth{0.481800pt}%
\definecolor{currentstroke}{rgb}{1.000000,1.000000,1.000000}%
\pgfsetstrokecolor{currentstroke}%
\pgfsetdash{}{0pt}%
\pgfpathmoveto{\pgfqpoint{1.558818in}{1.218460in}}%
\pgfpathcurveto{\pgfqpoint{1.569868in}{1.218460in}}{\pgfqpoint{1.580467in}{1.222851in}}{\pgfqpoint{1.588281in}{1.230664in}}%
\pgfpathcurveto{\pgfqpoint{1.596095in}{1.238478in}}{\pgfqpoint{1.600485in}{1.249077in}}{\pgfqpoint{1.600485in}{1.260127in}}%
\pgfpathcurveto{\pgfqpoint{1.600485in}{1.271177in}}{\pgfqpoint{1.596095in}{1.281776in}}{\pgfqpoint{1.588281in}{1.289590in}}%
\pgfpathcurveto{\pgfqpoint{1.580467in}{1.297404in}}{\pgfqpoint{1.569868in}{1.301794in}}{\pgfqpoint{1.558818in}{1.301794in}}%
\pgfpathcurveto{\pgfqpoint{1.547768in}{1.301794in}}{\pgfqpoint{1.537169in}{1.297404in}}{\pgfqpoint{1.529355in}{1.289590in}}%
\pgfpathcurveto{\pgfqpoint{1.521542in}{1.281776in}}{\pgfqpoint{1.517151in}{1.271177in}}{\pgfqpoint{1.517151in}{1.260127in}}%
\pgfpathcurveto{\pgfqpoint{1.517151in}{1.249077in}}{\pgfqpoint{1.521542in}{1.238478in}}{\pgfqpoint{1.529355in}{1.230664in}}%
\pgfpathcurveto{\pgfqpoint{1.537169in}{1.222851in}}{\pgfqpoint{1.547768in}{1.218460in}}{\pgfqpoint{1.558818in}{1.218460in}}%
\pgfpathclose%
\pgfusepath{stroke,fill}%
\end{pgfscope}%
\begin{pgfscope}%
\pgfpathrectangle{\pgfqpoint{0.481978in}{0.331635in}}{\pgfqpoint{4.960000in}{3.696000in}}%
\pgfusepath{clip}%
\pgfsetbuttcap%
\pgfsetroundjoin%
\definecolor{currentfill}{rgb}{1.000000,0.705882,0.509804}%
\pgfsetfillcolor{currentfill}%
\pgfsetlinewidth{0.481800pt}%
\definecolor{currentstroke}{rgb}{1.000000,1.000000,1.000000}%
\pgfsetstrokecolor{currentstroke}%
\pgfsetdash{}{0pt}%
\pgfpathmoveto{\pgfqpoint{1.915714in}{3.758330in}}%
\pgfpathcurveto{\pgfqpoint{1.926764in}{3.758330in}}{\pgfqpoint{1.937363in}{3.762721in}}{\pgfqpoint{1.945177in}{3.770534in}}%
\pgfpathcurveto{\pgfqpoint{1.952990in}{3.778348in}}{\pgfqpoint{1.957381in}{3.788947in}}{\pgfqpoint{1.957381in}{3.799997in}}%
\pgfpathcurveto{\pgfqpoint{1.957381in}{3.811047in}}{\pgfqpoint{1.952990in}{3.821646in}}{\pgfqpoint{1.945177in}{3.829460in}}%
\pgfpathcurveto{\pgfqpoint{1.937363in}{3.837273in}}{\pgfqpoint{1.926764in}{3.841664in}}{\pgfqpoint{1.915714in}{3.841664in}}%
\pgfpathcurveto{\pgfqpoint{1.904664in}{3.841664in}}{\pgfqpoint{1.894065in}{3.837273in}}{\pgfqpoint{1.886251in}{3.829460in}}%
\pgfpathcurveto{\pgfqpoint{1.878437in}{3.821646in}}{\pgfqpoint{1.874047in}{3.811047in}}{\pgfqpoint{1.874047in}{3.799997in}}%
\pgfpathcurveto{\pgfqpoint{1.874047in}{3.788947in}}{\pgfqpoint{1.878437in}{3.778348in}}{\pgfqpoint{1.886251in}{3.770534in}}%
\pgfpathcurveto{\pgfqpoint{1.894065in}{3.762721in}}{\pgfqpoint{1.904664in}{3.758330in}}{\pgfqpoint{1.915714in}{3.758330in}}%
\pgfpathclose%
\pgfusepath{stroke,fill}%
\end{pgfscope}%
\begin{pgfscope}%
\pgfpathrectangle{\pgfqpoint{0.481978in}{0.331635in}}{\pgfqpoint{4.960000in}{3.696000in}}%
\pgfusepath{clip}%
\pgfsetbuttcap%
\pgfsetroundjoin%
\definecolor{currentfill}{rgb}{1.000000,0.705882,0.509804}%
\pgfsetfillcolor{currentfill}%
\pgfsetlinewidth{0.481800pt}%
\definecolor{currentstroke}{rgb}{1.000000,1.000000,1.000000}%
\pgfsetstrokecolor{currentstroke}%
\pgfsetdash{}{0pt}%
\pgfpathmoveto{\pgfqpoint{2.053323in}{3.283689in}}%
\pgfpathcurveto{\pgfqpoint{2.064373in}{3.283689in}}{\pgfqpoint{2.074972in}{3.288079in}}{\pgfqpoint{2.082786in}{3.295892in}}%
\pgfpathcurveto{\pgfqpoint{2.090599in}{3.303706in}}{\pgfqpoint{2.094990in}{3.314305in}}{\pgfqpoint{2.094990in}{3.325355in}}%
\pgfpathcurveto{\pgfqpoint{2.094990in}{3.336405in}}{\pgfqpoint{2.090599in}{3.347004in}}{\pgfqpoint{2.082786in}{3.354818in}}%
\pgfpathcurveto{\pgfqpoint{2.074972in}{3.362632in}}{\pgfqpoint{2.064373in}{3.367022in}}{\pgfqpoint{2.053323in}{3.367022in}}%
\pgfpathcurveto{\pgfqpoint{2.042273in}{3.367022in}}{\pgfqpoint{2.031674in}{3.362632in}}{\pgfqpoint{2.023860in}{3.354818in}}%
\pgfpathcurveto{\pgfqpoint{2.016046in}{3.347004in}}{\pgfqpoint{2.011656in}{3.336405in}}{\pgfqpoint{2.011656in}{3.325355in}}%
\pgfpathcurveto{\pgfqpoint{2.011656in}{3.314305in}}{\pgfqpoint{2.016046in}{3.303706in}}{\pgfqpoint{2.023860in}{3.295892in}}%
\pgfpathcurveto{\pgfqpoint{2.031674in}{3.288079in}}{\pgfqpoint{2.042273in}{3.283689in}}{\pgfqpoint{2.053323in}{3.283689in}}%
\pgfpathclose%
\pgfusepath{stroke,fill}%
\end{pgfscope}%
\begin{pgfscope}%
\pgfpathrectangle{\pgfqpoint{0.481978in}{0.331635in}}{\pgfqpoint{4.960000in}{3.696000in}}%
\pgfusepath{clip}%
\pgfsetbuttcap%
\pgfsetroundjoin%
\definecolor{currentfill}{rgb}{1.000000,0.705882,0.509804}%
\pgfsetfillcolor{currentfill}%
\pgfsetlinewidth{0.481800pt}%
\definecolor{currentstroke}{rgb}{1.000000,1.000000,1.000000}%
\pgfsetstrokecolor{currentstroke}%
\pgfsetdash{}{0pt}%
\pgfpathmoveto{\pgfqpoint{2.331227in}{1.972977in}}%
\pgfpathcurveto{\pgfqpoint{2.342278in}{1.972977in}}{\pgfqpoint{2.352877in}{1.977367in}}{\pgfqpoint{2.360690in}{1.985181in}}%
\pgfpathcurveto{\pgfqpoint{2.368504in}{1.992995in}}{\pgfqpoint{2.372894in}{2.003594in}}{\pgfqpoint{2.372894in}{2.014644in}}%
\pgfpathcurveto{\pgfqpoint{2.372894in}{2.025694in}}{\pgfqpoint{2.368504in}{2.036293in}}{\pgfqpoint{2.360690in}{2.044106in}}%
\pgfpathcurveto{\pgfqpoint{2.352877in}{2.051920in}}{\pgfqpoint{2.342278in}{2.056310in}}{\pgfqpoint{2.331227in}{2.056310in}}%
\pgfpathcurveto{\pgfqpoint{2.320177in}{2.056310in}}{\pgfqpoint{2.309578in}{2.051920in}}{\pgfqpoint{2.301765in}{2.044106in}}%
\pgfpathcurveto{\pgfqpoint{2.293951in}{2.036293in}}{\pgfqpoint{2.289561in}{2.025694in}}{\pgfqpoint{2.289561in}{2.014644in}}%
\pgfpathcurveto{\pgfqpoint{2.289561in}{2.003594in}}{\pgfqpoint{2.293951in}{1.992995in}}{\pgfqpoint{2.301765in}{1.985181in}}%
\pgfpathcurveto{\pgfqpoint{2.309578in}{1.977367in}}{\pgfqpoint{2.320177in}{1.972977in}}{\pgfqpoint{2.331227in}{1.972977in}}%
\pgfpathclose%
\pgfusepath{stroke,fill}%
\end{pgfscope}%
\begin{pgfscope}%
\pgfpathrectangle{\pgfqpoint{0.481978in}{0.331635in}}{\pgfqpoint{4.960000in}{3.696000in}}%
\pgfusepath{clip}%
\pgfsetbuttcap%
\pgfsetroundjoin%
\definecolor{currentfill}{rgb}{1.000000,0.705882,0.509804}%
\pgfsetfillcolor{currentfill}%
\pgfsetlinewidth{0.481800pt}%
\definecolor{currentstroke}{rgb}{1.000000,1.000000,1.000000}%
\pgfsetstrokecolor{currentstroke}%
\pgfsetdash{}{0pt}%
\pgfpathmoveto{\pgfqpoint{2.069873in}{2.646758in}}%
\pgfpathcurveto{\pgfqpoint{2.080924in}{2.646758in}}{\pgfqpoint{2.091523in}{2.651148in}}{\pgfqpoint{2.099336in}{2.658962in}}%
\pgfpathcurveto{\pgfqpoint{2.107150in}{2.666775in}}{\pgfqpoint{2.111540in}{2.677374in}}{\pgfqpoint{2.111540in}{2.688424in}}%
\pgfpathcurveto{\pgfqpoint{2.111540in}{2.699475in}}{\pgfqpoint{2.107150in}{2.710074in}}{\pgfqpoint{2.099336in}{2.717887in}}%
\pgfpathcurveto{\pgfqpoint{2.091523in}{2.725701in}}{\pgfqpoint{2.080924in}{2.730091in}}{\pgfqpoint{2.069873in}{2.730091in}}%
\pgfpathcurveto{\pgfqpoint{2.058823in}{2.730091in}}{\pgfqpoint{2.048224in}{2.725701in}}{\pgfqpoint{2.040411in}{2.717887in}}%
\pgfpathcurveto{\pgfqpoint{2.032597in}{2.710074in}}{\pgfqpoint{2.028207in}{2.699475in}}{\pgfqpoint{2.028207in}{2.688424in}}%
\pgfpathcurveto{\pgfqpoint{2.028207in}{2.677374in}}{\pgfqpoint{2.032597in}{2.666775in}}{\pgfqpoint{2.040411in}{2.658962in}}%
\pgfpathcurveto{\pgfqpoint{2.048224in}{2.651148in}}{\pgfqpoint{2.058823in}{2.646758in}}{\pgfqpoint{2.069873in}{2.646758in}}%
\pgfpathclose%
\pgfusepath{stroke,fill}%
\end{pgfscope}%
\begin{pgfscope}%
\pgfpathrectangle{\pgfqpoint{0.481978in}{0.331635in}}{\pgfqpoint{4.960000in}{3.696000in}}%
\pgfusepath{clip}%
\pgfsetbuttcap%
\pgfsetroundjoin%
\definecolor{currentfill}{rgb}{1.000000,0.705882,0.509804}%
\pgfsetfillcolor{currentfill}%
\pgfsetlinewidth{0.481800pt}%
\definecolor{currentstroke}{rgb}{1.000000,1.000000,1.000000}%
\pgfsetstrokecolor{currentstroke}%
\pgfsetdash{}{0pt}%
\pgfpathmoveto{\pgfqpoint{3.381662in}{2.402497in}}%
\pgfpathcurveto{\pgfqpoint{3.392712in}{2.402497in}}{\pgfqpoint{3.403311in}{2.406887in}}{\pgfqpoint{3.411125in}{2.414700in}}%
\pgfpathcurveto{\pgfqpoint{3.418938in}{2.422514in}}{\pgfqpoint{3.423329in}{2.433113in}}{\pgfqpoint{3.423329in}{2.444163in}}%
\pgfpathcurveto{\pgfqpoint{3.423329in}{2.455213in}}{\pgfqpoint{3.418938in}{2.465812in}}{\pgfqpoint{3.411125in}{2.473626in}}%
\pgfpathcurveto{\pgfqpoint{3.403311in}{2.481440in}}{\pgfqpoint{3.392712in}{2.485830in}}{\pgfqpoint{3.381662in}{2.485830in}}%
\pgfpathcurveto{\pgfqpoint{3.370612in}{2.485830in}}{\pgfqpoint{3.360013in}{2.481440in}}{\pgfqpoint{3.352199in}{2.473626in}}%
\pgfpathcurveto{\pgfqpoint{3.344385in}{2.465812in}}{\pgfqpoint{3.339995in}{2.455213in}}{\pgfqpoint{3.339995in}{2.444163in}}%
\pgfpathcurveto{\pgfqpoint{3.339995in}{2.433113in}}{\pgfqpoint{3.344385in}{2.422514in}}{\pgfqpoint{3.352199in}{2.414700in}}%
\pgfpathcurveto{\pgfqpoint{3.360013in}{2.406887in}}{\pgfqpoint{3.370612in}{2.402497in}}{\pgfqpoint{3.381662in}{2.402497in}}%
\pgfpathclose%
\pgfusepath{stroke,fill}%
\end{pgfscope}%
\begin{pgfscope}%
\pgfpathrectangle{\pgfqpoint{0.481978in}{0.331635in}}{\pgfqpoint{4.960000in}{3.696000in}}%
\pgfusepath{clip}%
\pgfsetbuttcap%
\pgfsetroundjoin%
\definecolor{currentfill}{rgb}{1.000000,0.705882,0.509804}%
\pgfsetfillcolor{currentfill}%
\pgfsetlinewidth{0.481800pt}%
\definecolor{currentstroke}{rgb}{1.000000,1.000000,1.000000}%
\pgfsetstrokecolor{currentstroke}%
\pgfsetdash{}{0pt}%
\pgfpathmoveto{\pgfqpoint{2.324071in}{2.570939in}}%
\pgfpathcurveto{\pgfqpoint{2.335121in}{2.570939in}}{\pgfqpoint{2.345720in}{2.575329in}}{\pgfqpoint{2.353533in}{2.583143in}}%
\pgfpathcurveto{\pgfqpoint{2.361347in}{2.590956in}}{\pgfqpoint{2.365737in}{2.601555in}}{\pgfqpoint{2.365737in}{2.612606in}}%
\pgfpathcurveto{\pgfqpoint{2.365737in}{2.623656in}}{\pgfqpoint{2.361347in}{2.634255in}}{\pgfqpoint{2.353533in}{2.642068in}}%
\pgfpathcurveto{\pgfqpoint{2.345720in}{2.649882in}}{\pgfqpoint{2.335121in}{2.654272in}}{\pgfqpoint{2.324071in}{2.654272in}}%
\pgfpathcurveto{\pgfqpoint{2.313020in}{2.654272in}}{\pgfqpoint{2.302421in}{2.649882in}}{\pgfqpoint{2.294608in}{2.642068in}}%
\pgfpathcurveto{\pgfqpoint{2.286794in}{2.634255in}}{\pgfqpoint{2.282404in}{2.623656in}}{\pgfqpoint{2.282404in}{2.612606in}}%
\pgfpathcurveto{\pgfqpoint{2.282404in}{2.601555in}}{\pgfqpoint{2.286794in}{2.590956in}}{\pgfqpoint{2.294608in}{2.583143in}}%
\pgfpathcurveto{\pgfqpoint{2.302421in}{2.575329in}}{\pgfqpoint{2.313020in}{2.570939in}}{\pgfqpoint{2.324071in}{2.570939in}}%
\pgfpathclose%
\pgfusepath{stroke,fill}%
\end{pgfscope}%
\begin{pgfscope}%
\pgfpathrectangle{\pgfqpoint{0.481978in}{0.331635in}}{\pgfqpoint{4.960000in}{3.696000in}}%
\pgfusepath{clip}%
\pgfsetbuttcap%
\pgfsetroundjoin%
\definecolor{currentfill}{rgb}{1.000000,0.705882,0.509804}%
\pgfsetfillcolor{currentfill}%
\pgfsetlinewidth{0.481800pt}%
\definecolor{currentstroke}{rgb}{1.000000,1.000000,1.000000}%
\pgfsetstrokecolor{currentstroke}%
\pgfsetdash{}{0pt}%
\pgfpathmoveto{\pgfqpoint{1.297191in}{1.891188in}}%
\pgfpathcurveto{\pgfqpoint{1.308241in}{1.891188in}}{\pgfqpoint{1.318840in}{1.895578in}}{\pgfqpoint{1.326654in}{1.903391in}}%
\pgfpathcurveto{\pgfqpoint{1.334467in}{1.911205in}}{\pgfqpoint{1.338858in}{1.921804in}}{\pgfqpoint{1.338858in}{1.932854in}}%
\pgfpathcurveto{\pgfqpoint{1.338858in}{1.943904in}}{\pgfqpoint{1.334467in}{1.954503in}}{\pgfqpoint{1.326654in}{1.962317in}}%
\pgfpathcurveto{\pgfqpoint{1.318840in}{1.970131in}}{\pgfqpoint{1.308241in}{1.974521in}}{\pgfqpoint{1.297191in}{1.974521in}}%
\pgfpathcurveto{\pgfqpoint{1.286141in}{1.974521in}}{\pgfqpoint{1.275542in}{1.970131in}}{\pgfqpoint{1.267728in}{1.962317in}}%
\pgfpathcurveto{\pgfqpoint{1.259915in}{1.954503in}}{\pgfqpoint{1.255524in}{1.943904in}}{\pgfqpoint{1.255524in}{1.932854in}}%
\pgfpathcurveto{\pgfqpoint{1.255524in}{1.921804in}}{\pgfqpoint{1.259915in}{1.911205in}}{\pgfqpoint{1.267728in}{1.903391in}}%
\pgfpathcurveto{\pgfqpoint{1.275542in}{1.895578in}}{\pgfqpoint{1.286141in}{1.891188in}}{\pgfqpoint{1.297191in}{1.891188in}}%
\pgfpathclose%
\pgfusepath{stroke,fill}%
\end{pgfscope}%
\begin{pgfscope}%
\pgfpathrectangle{\pgfqpoint{0.481978in}{0.331635in}}{\pgfqpoint{4.960000in}{3.696000in}}%
\pgfusepath{clip}%
\pgfsetbuttcap%
\pgfsetroundjoin%
\definecolor{currentfill}{rgb}{1.000000,0.705882,0.509804}%
\pgfsetfillcolor{currentfill}%
\pgfsetlinewidth{0.481800pt}%
\definecolor{currentstroke}{rgb}{1.000000,1.000000,1.000000}%
\pgfsetstrokecolor{currentstroke}%
\pgfsetdash{}{0pt}%
\pgfpathmoveto{\pgfqpoint{4.801011in}{1.220565in}}%
\pgfpathcurveto{\pgfqpoint{4.812061in}{1.220565in}}{\pgfqpoint{4.822660in}{1.224956in}}{\pgfqpoint{4.830474in}{1.232769in}}%
\pgfpathcurveto{\pgfqpoint{4.838288in}{1.240583in}}{\pgfqpoint{4.842678in}{1.251182in}}{\pgfqpoint{4.842678in}{1.262232in}}%
\pgfpathcurveto{\pgfqpoint{4.842678in}{1.273282in}}{\pgfqpoint{4.838288in}{1.283881in}}{\pgfqpoint{4.830474in}{1.291695in}}%
\pgfpathcurveto{\pgfqpoint{4.822660in}{1.299508in}}{\pgfqpoint{4.812061in}{1.303899in}}{\pgfqpoint{4.801011in}{1.303899in}}%
\pgfpathcurveto{\pgfqpoint{4.789961in}{1.303899in}}{\pgfqpoint{4.779362in}{1.299508in}}{\pgfqpoint{4.771548in}{1.291695in}}%
\pgfpathcurveto{\pgfqpoint{4.763735in}{1.283881in}}{\pgfqpoint{4.759345in}{1.273282in}}{\pgfqpoint{4.759345in}{1.262232in}}%
\pgfpathcurveto{\pgfqpoint{4.759345in}{1.251182in}}{\pgfqpoint{4.763735in}{1.240583in}}{\pgfqpoint{4.771548in}{1.232769in}}%
\pgfpathcurveto{\pgfqpoint{4.779362in}{1.224956in}}{\pgfqpoint{4.789961in}{1.220565in}}{\pgfqpoint{4.801011in}{1.220565in}}%
\pgfpathclose%
\pgfusepath{stroke,fill}%
\end{pgfscope}%
\begin{pgfscope}%
\pgfpathrectangle{\pgfqpoint{0.481978in}{0.331635in}}{\pgfqpoint{4.960000in}{3.696000in}}%
\pgfusepath{clip}%
\pgfsetbuttcap%
\pgfsetroundjoin%
\definecolor{currentfill}{rgb}{1.000000,0.705882,0.509804}%
\pgfsetfillcolor{currentfill}%
\pgfsetlinewidth{0.481800pt}%
\definecolor{currentstroke}{rgb}{1.000000,1.000000,1.000000}%
\pgfsetstrokecolor{currentstroke}%
\pgfsetdash{}{0pt}%
\pgfpathmoveto{\pgfqpoint{1.381868in}{1.910915in}}%
\pgfpathcurveto{\pgfqpoint{1.392918in}{1.910915in}}{\pgfqpoint{1.403517in}{1.915306in}}{\pgfqpoint{1.411330in}{1.923119in}}%
\pgfpathcurveto{\pgfqpoint{1.419144in}{1.930933in}}{\pgfqpoint{1.423534in}{1.941532in}}{\pgfqpoint{1.423534in}{1.952582in}}%
\pgfpathcurveto{\pgfqpoint{1.423534in}{1.963632in}}{\pgfqpoint{1.419144in}{1.974231in}}{\pgfqpoint{1.411330in}{1.982045in}}%
\pgfpathcurveto{\pgfqpoint{1.403517in}{1.989858in}}{\pgfqpoint{1.392918in}{1.994249in}}{\pgfqpoint{1.381868in}{1.994249in}}%
\pgfpathcurveto{\pgfqpoint{1.370817in}{1.994249in}}{\pgfqpoint{1.360218in}{1.989858in}}{\pgfqpoint{1.352405in}{1.982045in}}%
\pgfpathcurveto{\pgfqpoint{1.344591in}{1.974231in}}{\pgfqpoint{1.340201in}{1.963632in}}{\pgfqpoint{1.340201in}{1.952582in}}%
\pgfpathcurveto{\pgfqpoint{1.340201in}{1.941532in}}{\pgfqpoint{1.344591in}{1.930933in}}{\pgfqpoint{1.352405in}{1.923119in}}%
\pgfpathcurveto{\pgfqpoint{1.360218in}{1.915306in}}{\pgfqpoint{1.370817in}{1.910915in}}{\pgfqpoint{1.381868in}{1.910915in}}%
\pgfpathclose%
\pgfusepath{stroke,fill}%
\end{pgfscope}%
\begin{pgfscope}%
\pgfpathrectangle{\pgfqpoint{0.481978in}{0.331635in}}{\pgfqpoint{4.960000in}{3.696000in}}%
\pgfusepath{clip}%
\pgfsetbuttcap%
\pgfsetroundjoin%
\definecolor{currentfill}{rgb}{1.000000,0.705882,0.509804}%
\pgfsetfillcolor{currentfill}%
\pgfsetlinewidth{0.481800pt}%
\definecolor{currentstroke}{rgb}{1.000000,1.000000,1.000000}%
\pgfsetstrokecolor{currentstroke}%
\pgfsetdash{}{0pt}%
\pgfpathmoveto{\pgfqpoint{3.112049in}{2.184139in}}%
\pgfpathcurveto{\pgfqpoint{3.123100in}{2.184139in}}{\pgfqpoint{3.133699in}{2.188529in}}{\pgfqpoint{3.141512in}{2.196342in}}%
\pgfpathcurveto{\pgfqpoint{3.149326in}{2.204156in}}{\pgfqpoint{3.153716in}{2.214755in}}{\pgfqpoint{3.153716in}{2.225805in}}%
\pgfpathcurveto{\pgfqpoint{3.153716in}{2.236855in}}{\pgfqpoint{3.149326in}{2.247454in}}{\pgfqpoint{3.141512in}{2.255268in}}%
\pgfpathcurveto{\pgfqpoint{3.133699in}{2.263082in}}{\pgfqpoint{3.123100in}{2.267472in}}{\pgfqpoint{3.112049in}{2.267472in}}%
\pgfpathcurveto{\pgfqpoint{3.100999in}{2.267472in}}{\pgfqpoint{3.090400in}{2.263082in}}{\pgfqpoint{3.082587in}{2.255268in}}%
\pgfpathcurveto{\pgfqpoint{3.074773in}{2.247454in}}{\pgfqpoint{3.070383in}{2.236855in}}{\pgfqpoint{3.070383in}{2.225805in}}%
\pgfpathcurveto{\pgfqpoint{3.070383in}{2.214755in}}{\pgfqpoint{3.074773in}{2.204156in}}{\pgfqpoint{3.082587in}{2.196342in}}%
\pgfpathcurveto{\pgfqpoint{3.090400in}{2.188529in}}{\pgfqpoint{3.100999in}{2.184139in}}{\pgfqpoint{3.112049in}{2.184139in}}%
\pgfpathclose%
\pgfusepath{stroke,fill}%
\end{pgfscope}%
\begin{pgfscope}%
\pgfpathrectangle{\pgfqpoint{0.481978in}{0.331635in}}{\pgfqpoint{4.960000in}{3.696000in}}%
\pgfusepath{clip}%
\pgfsetbuttcap%
\pgfsetroundjoin%
\definecolor{currentfill}{rgb}{1.000000,0.705882,0.509804}%
\pgfsetfillcolor{currentfill}%
\pgfsetlinewidth{0.481800pt}%
\definecolor{currentstroke}{rgb}{1.000000,1.000000,1.000000}%
\pgfsetstrokecolor{currentstroke}%
\pgfsetdash{}{0pt}%
\pgfpathmoveto{\pgfqpoint{2.263955in}{2.359116in}}%
\pgfpathcurveto{\pgfqpoint{2.275005in}{2.359116in}}{\pgfqpoint{2.285604in}{2.363506in}}{\pgfqpoint{2.293417in}{2.371320in}}%
\pgfpathcurveto{\pgfqpoint{2.301231in}{2.379134in}}{\pgfqpoint{2.305621in}{2.389733in}}{\pgfqpoint{2.305621in}{2.400783in}}%
\pgfpathcurveto{\pgfqpoint{2.305621in}{2.411833in}}{\pgfqpoint{2.301231in}{2.422432in}}{\pgfqpoint{2.293417in}{2.430245in}}%
\pgfpathcurveto{\pgfqpoint{2.285604in}{2.438059in}}{\pgfqpoint{2.275005in}{2.442449in}}{\pgfqpoint{2.263955in}{2.442449in}}%
\pgfpathcurveto{\pgfqpoint{2.252904in}{2.442449in}}{\pgfqpoint{2.242305in}{2.438059in}}{\pgfqpoint{2.234492in}{2.430245in}}%
\pgfpathcurveto{\pgfqpoint{2.226678in}{2.422432in}}{\pgfqpoint{2.222288in}{2.411833in}}{\pgfqpoint{2.222288in}{2.400783in}}%
\pgfpathcurveto{\pgfqpoint{2.222288in}{2.389733in}}{\pgfqpoint{2.226678in}{2.379134in}}{\pgfqpoint{2.234492in}{2.371320in}}%
\pgfpathcurveto{\pgfqpoint{2.242305in}{2.363506in}}{\pgfqpoint{2.252904in}{2.359116in}}{\pgfqpoint{2.263955in}{2.359116in}}%
\pgfpathclose%
\pgfusepath{stroke,fill}%
\end{pgfscope}%
\begin{pgfscope}%
\pgfpathrectangle{\pgfqpoint{0.481978in}{0.331635in}}{\pgfqpoint{4.960000in}{3.696000in}}%
\pgfusepath{clip}%
\pgfsetbuttcap%
\pgfsetroundjoin%
\definecolor{currentfill}{rgb}{1.000000,0.705882,0.509804}%
\pgfsetfillcolor{currentfill}%
\pgfsetlinewidth{0.481800pt}%
\definecolor{currentstroke}{rgb}{1.000000,1.000000,1.000000}%
\pgfsetstrokecolor{currentstroke}%
\pgfsetdash{}{0pt}%
\pgfpathmoveto{\pgfqpoint{1.430664in}{1.728753in}}%
\pgfpathcurveto{\pgfqpoint{1.441714in}{1.728753in}}{\pgfqpoint{1.452313in}{1.733144in}}{\pgfqpoint{1.460126in}{1.740957in}}%
\pgfpathcurveto{\pgfqpoint{1.467940in}{1.748771in}}{\pgfqpoint{1.472330in}{1.759370in}}{\pgfqpoint{1.472330in}{1.770420in}}%
\pgfpathcurveto{\pgfqpoint{1.472330in}{1.781470in}}{\pgfqpoint{1.467940in}{1.792069in}}{\pgfqpoint{1.460126in}{1.799883in}}%
\pgfpathcurveto{\pgfqpoint{1.452313in}{1.807697in}}{\pgfqpoint{1.441714in}{1.812087in}}{\pgfqpoint{1.430664in}{1.812087in}}%
\pgfpathcurveto{\pgfqpoint{1.419614in}{1.812087in}}{\pgfqpoint{1.409015in}{1.807697in}}{\pgfqpoint{1.401201in}{1.799883in}}%
\pgfpathcurveto{\pgfqpoint{1.393387in}{1.792069in}}{\pgfqpoint{1.388997in}{1.781470in}}{\pgfqpoint{1.388997in}{1.770420in}}%
\pgfpathcurveto{\pgfqpoint{1.388997in}{1.759370in}}{\pgfqpoint{1.393387in}{1.748771in}}{\pgfqpoint{1.401201in}{1.740957in}}%
\pgfpathcurveto{\pgfqpoint{1.409015in}{1.733144in}}{\pgfqpoint{1.419614in}{1.728753in}}{\pgfqpoint{1.430664in}{1.728753in}}%
\pgfpathclose%
\pgfusepath{stroke,fill}%
\end{pgfscope}%
\begin{pgfscope}%
\pgfpathrectangle{\pgfqpoint{0.481978in}{0.331635in}}{\pgfqpoint{4.960000in}{3.696000in}}%
\pgfusepath{clip}%
\pgfsetbuttcap%
\pgfsetroundjoin%
\definecolor{currentfill}{rgb}{1.000000,0.705882,0.509804}%
\pgfsetfillcolor{currentfill}%
\pgfsetlinewidth{0.481800pt}%
\definecolor{currentstroke}{rgb}{1.000000,1.000000,1.000000}%
\pgfsetstrokecolor{currentstroke}%
\pgfsetdash{}{0pt}%
\pgfpathmoveto{\pgfqpoint{1.786845in}{2.954604in}}%
\pgfpathcurveto{\pgfqpoint{1.797895in}{2.954604in}}{\pgfqpoint{1.808494in}{2.958995in}}{\pgfqpoint{1.816308in}{2.966808in}}%
\pgfpathcurveto{\pgfqpoint{1.824121in}{2.974622in}}{\pgfqpoint{1.828511in}{2.985221in}}{\pgfqpoint{1.828511in}{2.996271in}}%
\pgfpathcurveto{\pgfqpoint{1.828511in}{3.007321in}}{\pgfqpoint{1.824121in}{3.017920in}}{\pgfqpoint{1.816308in}{3.025734in}}%
\pgfpathcurveto{\pgfqpoint{1.808494in}{3.033548in}}{\pgfqpoint{1.797895in}{3.037938in}}{\pgfqpoint{1.786845in}{3.037938in}}%
\pgfpathcurveto{\pgfqpoint{1.775795in}{3.037938in}}{\pgfqpoint{1.765196in}{3.033548in}}{\pgfqpoint{1.757382in}{3.025734in}}%
\pgfpathcurveto{\pgfqpoint{1.749568in}{3.017920in}}{\pgfqpoint{1.745178in}{3.007321in}}{\pgfqpoint{1.745178in}{2.996271in}}%
\pgfpathcurveto{\pgfqpoint{1.745178in}{2.985221in}}{\pgfqpoint{1.749568in}{2.974622in}}{\pgfqpoint{1.757382in}{2.966808in}}%
\pgfpathcurveto{\pgfqpoint{1.765196in}{2.958995in}}{\pgfqpoint{1.775795in}{2.954604in}}{\pgfqpoint{1.786845in}{2.954604in}}%
\pgfpathclose%
\pgfusepath{stroke,fill}%
\end{pgfscope}%
\begin{pgfscope}%
\pgfpathrectangle{\pgfqpoint{0.481978in}{0.331635in}}{\pgfqpoint{4.960000in}{3.696000in}}%
\pgfusepath{clip}%
\pgfsetbuttcap%
\pgfsetroundjoin%
\definecolor{currentfill}{rgb}{1.000000,0.705882,0.509804}%
\pgfsetfillcolor{currentfill}%
\pgfsetlinewidth{0.481800pt}%
\definecolor{currentstroke}{rgb}{1.000000,1.000000,1.000000}%
\pgfsetstrokecolor{currentstroke}%
\pgfsetdash{}{0pt}%
\pgfpathmoveto{\pgfqpoint{2.553548in}{2.936038in}}%
\pgfpathcurveto{\pgfqpoint{2.564599in}{2.936038in}}{\pgfqpoint{2.575198in}{2.940428in}}{\pgfqpoint{2.583011in}{2.948241in}}%
\pgfpathcurveto{\pgfqpoint{2.590825in}{2.956055in}}{\pgfqpoint{2.595215in}{2.966654in}}{\pgfqpoint{2.595215in}{2.977704in}}%
\pgfpathcurveto{\pgfqpoint{2.595215in}{2.988754in}}{\pgfqpoint{2.590825in}{2.999353in}}{\pgfqpoint{2.583011in}{3.007167in}}%
\pgfpathcurveto{\pgfqpoint{2.575198in}{3.014981in}}{\pgfqpoint{2.564599in}{3.019371in}}{\pgfqpoint{2.553548in}{3.019371in}}%
\pgfpathcurveto{\pgfqpoint{2.542498in}{3.019371in}}{\pgfqpoint{2.531899in}{3.014981in}}{\pgfqpoint{2.524086in}{3.007167in}}%
\pgfpathcurveto{\pgfqpoint{2.516272in}{2.999353in}}{\pgfqpoint{2.511882in}{2.988754in}}{\pgfqpoint{2.511882in}{2.977704in}}%
\pgfpathcurveto{\pgfqpoint{2.511882in}{2.966654in}}{\pgfqpoint{2.516272in}{2.956055in}}{\pgfqpoint{2.524086in}{2.948241in}}%
\pgfpathcurveto{\pgfqpoint{2.531899in}{2.940428in}}{\pgfqpoint{2.542498in}{2.936038in}}{\pgfqpoint{2.553548in}{2.936038in}}%
\pgfpathclose%
\pgfusepath{stroke,fill}%
\end{pgfscope}%
\begin{pgfscope}%
\pgfpathrectangle{\pgfqpoint{0.481978in}{0.331635in}}{\pgfqpoint{4.960000in}{3.696000in}}%
\pgfusepath{clip}%
\pgfsetbuttcap%
\pgfsetroundjoin%
\definecolor{currentfill}{rgb}{1.000000,0.705882,0.509804}%
\pgfsetfillcolor{currentfill}%
\pgfsetlinewidth{0.481800pt}%
\definecolor{currentstroke}{rgb}{1.000000,1.000000,1.000000}%
\pgfsetstrokecolor{currentstroke}%
\pgfsetdash{}{0pt}%
\pgfpathmoveto{\pgfqpoint{3.143878in}{2.055368in}}%
\pgfpathcurveto{\pgfqpoint{3.154928in}{2.055368in}}{\pgfqpoint{3.165527in}{2.059758in}}{\pgfqpoint{3.173341in}{2.067572in}}%
\pgfpathcurveto{\pgfqpoint{3.181155in}{2.075386in}}{\pgfqpoint{3.185545in}{2.085985in}}{\pgfqpoint{3.185545in}{2.097035in}}%
\pgfpathcurveto{\pgfqpoint{3.185545in}{2.108085in}}{\pgfqpoint{3.181155in}{2.118684in}}{\pgfqpoint{3.173341in}{2.126498in}}%
\pgfpathcurveto{\pgfqpoint{3.165527in}{2.134311in}}{\pgfqpoint{3.154928in}{2.138701in}}{\pgfqpoint{3.143878in}{2.138701in}}%
\pgfpathcurveto{\pgfqpoint{3.132828in}{2.138701in}}{\pgfqpoint{3.122229in}{2.134311in}}{\pgfqpoint{3.114415in}{2.126498in}}%
\pgfpathcurveto{\pgfqpoint{3.106602in}{2.118684in}}{\pgfqpoint{3.102211in}{2.108085in}}{\pgfqpoint{3.102211in}{2.097035in}}%
\pgfpathcurveto{\pgfqpoint{3.102211in}{2.085985in}}{\pgfqpoint{3.106602in}{2.075386in}}{\pgfqpoint{3.114415in}{2.067572in}}%
\pgfpathcurveto{\pgfqpoint{3.122229in}{2.059758in}}{\pgfqpoint{3.132828in}{2.055368in}}{\pgfqpoint{3.143878in}{2.055368in}}%
\pgfpathclose%
\pgfusepath{stroke,fill}%
\end{pgfscope}%
\begin{pgfscope}%
\pgfpathrectangle{\pgfqpoint{0.481978in}{0.331635in}}{\pgfqpoint{4.960000in}{3.696000in}}%
\pgfusepath{clip}%
\pgfsetbuttcap%
\pgfsetroundjoin%
\definecolor{currentfill}{rgb}{1.000000,0.705882,0.509804}%
\pgfsetfillcolor{currentfill}%
\pgfsetlinewidth{0.481800pt}%
\definecolor{currentstroke}{rgb}{1.000000,1.000000,1.000000}%
\pgfsetstrokecolor{currentstroke}%
\pgfsetdash{}{0pt}%
\pgfpathmoveto{\pgfqpoint{1.510599in}{1.598822in}}%
\pgfpathcurveto{\pgfqpoint{1.521649in}{1.598822in}}{\pgfqpoint{1.532249in}{1.603213in}}{\pgfqpoint{1.540062in}{1.611026in}}%
\pgfpathcurveto{\pgfqpoint{1.547876in}{1.618840in}}{\pgfqpoint{1.552266in}{1.629439in}}{\pgfqpoint{1.552266in}{1.640489in}}%
\pgfpathcurveto{\pgfqpoint{1.552266in}{1.651539in}}{\pgfqpoint{1.547876in}{1.662138in}}{\pgfqpoint{1.540062in}{1.669952in}}%
\pgfpathcurveto{\pgfqpoint{1.532249in}{1.677765in}}{\pgfqpoint{1.521649in}{1.682156in}}{\pgfqpoint{1.510599in}{1.682156in}}%
\pgfpathcurveto{\pgfqpoint{1.499549in}{1.682156in}}{\pgfqpoint{1.488950in}{1.677765in}}{\pgfqpoint{1.481137in}{1.669952in}}%
\pgfpathcurveto{\pgfqpoint{1.473323in}{1.662138in}}{\pgfqpoint{1.468933in}{1.651539in}}{\pgfqpoint{1.468933in}{1.640489in}}%
\pgfpathcurveto{\pgfqpoint{1.468933in}{1.629439in}}{\pgfqpoint{1.473323in}{1.618840in}}{\pgfqpoint{1.481137in}{1.611026in}}%
\pgfpathcurveto{\pgfqpoint{1.488950in}{1.603213in}}{\pgfqpoint{1.499549in}{1.598822in}}{\pgfqpoint{1.510599in}{1.598822in}}%
\pgfpathclose%
\pgfusepath{stroke,fill}%
\end{pgfscope}%
\begin{pgfscope}%
\pgfpathrectangle{\pgfqpoint{0.481978in}{0.331635in}}{\pgfqpoint{4.960000in}{3.696000in}}%
\pgfusepath{clip}%
\pgfsetbuttcap%
\pgfsetroundjoin%
\definecolor{currentfill}{rgb}{1.000000,0.705882,0.509804}%
\pgfsetfillcolor{currentfill}%
\pgfsetlinewidth{0.481800pt}%
\definecolor{currentstroke}{rgb}{1.000000,1.000000,1.000000}%
\pgfsetstrokecolor{currentstroke}%
\pgfsetdash{}{0pt}%
\pgfpathmoveto{\pgfqpoint{1.412036in}{2.452994in}}%
\pgfpathcurveto{\pgfqpoint{1.423087in}{2.452994in}}{\pgfqpoint{1.433686in}{2.457384in}}{\pgfqpoint{1.441499in}{2.465198in}}%
\pgfpathcurveto{\pgfqpoint{1.449313in}{2.473012in}}{\pgfqpoint{1.453703in}{2.483611in}}{\pgfqpoint{1.453703in}{2.494661in}}%
\pgfpathcurveto{\pgfqpoint{1.453703in}{2.505711in}}{\pgfqpoint{1.449313in}{2.516310in}}{\pgfqpoint{1.441499in}{2.524124in}}%
\pgfpathcurveto{\pgfqpoint{1.433686in}{2.531937in}}{\pgfqpoint{1.423087in}{2.536328in}}{\pgfqpoint{1.412036in}{2.536328in}}%
\pgfpathcurveto{\pgfqpoint{1.400986in}{2.536328in}}{\pgfqpoint{1.390387in}{2.531937in}}{\pgfqpoint{1.382574in}{2.524124in}}%
\pgfpathcurveto{\pgfqpoint{1.374760in}{2.516310in}}{\pgfqpoint{1.370370in}{2.505711in}}{\pgfqpoint{1.370370in}{2.494661in}}%
\pgfpathcurveto{\pgfqpoint{1.370370in}{2.483611in}}{\pgfqpoint{1.374760in}{2.473012in}}{\pgfqpoint{1.382574in}{2.465198in}}%
\pgfpathcurveto{\pgfqpoint{1.390387in}{2.457384in}}{\pgfqpoint{1.400986in}{2.452994in}}{\pgfqpoint{1.412036in}{2.452994in}}%
\pgfpathclose%
\pgfusepath{stroke,fill}%
\end{pgfscope}%
\begin{pgfscope}%
\pgfpathrectangle{\pgfqpoint{0.481978in}{0.331635in}}{\pgfqpoint{4.960000in}{3.696000in}}%
\pgfusepath{clip}%
\pgfsetbuttcap%
\pgfsetroundjoin%
\definecolor{currentfill}{rgb}{1.000000,0.705882,0.509804}%
\pgfsetfillcolor{currentfill}%
\pgfsetlinewidth{0.481800pt}%
\definecolor{currentstroke}{rgb}{1.000000,1.000000,1.000000}%
\pgfsetstrokecolor{currentstroke}%
\pgfsetdash{}{0pt}%
\pgfpathmoveto{\pgfqpoint{2.580200in}{3.617569in}}%
\pgfpathcurveto{\pgfqpoint{2.591250in}{3.617569in}}{\pgfqpoint{2.601849in}{3.621959in}}{\pgfqpoint{2.609663in}{3.629773in}}%
\pgfpathcurveto{\pgfqpoint{2.617476in}{3.637586in}}{\pgfqpoint{2.621867in}{3.648185in}}{\pgfqpoint{2.621867in}{3.659236in}}%
\pgfpathcurveto{\pgfqpoint{2.621867in}{3.670286in}}{\pgfqpoint{2.617476in}{3.680885in}}{\pgfqpoint{2.609663in}{3.688698in}}%
\pgfpathcurveto{\pgfqpoint{2.601849in}{3.696512in}}{\pgfqpoint{2.591250in}{3.700902in}}{\pgfqpoint{2.580200in}{3.700902in}}%
\pgfpathcurveto{\pgfqpoint{2.569150in}{3.700902in}}{\pgfqpoint{2.558551in}{3.696512in}}{\pgfqpoint{2.550737in}{3.688698in}}%
\pgfpathcurveto{\pgfqpoint{2.542924in}{3.680885in}}{\pgfqpoint{2.538533in}{3.670286in}}{\pgfqpoint{2.538533in}{3.659236in}}%
\pgfpathcurveto{\pgfqpoint{2.538533in}{3.648185in}}{\pgfqpoint{2.542924in}{3.637586in}}{\pgfqpoint{2.550737in}{3.629773in}}%
\pgfpathcurveto{\pgfqpoint{2.558551in}{3.621959in}}{\pgfqpoint{2.569150in}{3.617569in}}{\pgfqpoint{2.580200in}{3.617569in}}%
\pgfpathclose%
\pgfusepath{stroke,fill}%
\end{pgfscope}%
\begin{pgfscope}%
\pgfpathrectangle{\pgfqpoint{0.481978in}{0.331635in}}{\pgfqpoint{4.960000in}{3.696000in}}%
\pgfusepath{clip}%
\pgfsetbuttcap%
\pgfsetroundjoin%
\definecolor{currentfill}{rgb}{1.000000,0.705882,0.509804}%
\pgfsetfillcolor{currentfill}%
\pgfsetlinewidth{0.481800pt}%
\definecolor{currentstroke}{rgb}{1.000000,1.000000,1.000000}%
\pgfsetstrokecolor{currentstroke}%
\pgfsetdash{}{0pt}%
\pgfpathmoveto{\pgfqpoint{1.212889in}{1.643731in}}%
\pgfpathcurveto{\pgfqpoint{1.223939in}{1.643731in}}{\pgfqpoint{1.234538in}{1.648121in}}{\pgfqpoint{1.242352in}{1.655935in}}%
\pgfpathcurveto{\pgfqpoint{1.250165in}{1.663749in}}{\pgfqpoint{1.254556in}{1.674348in}}{\pgfqpoint{1.254556in}{1.685398in}}%
\pgfpathcurveto{\pgfqpoint{1.254556in}{1.696448in}}{\pgfqpoint{1.250165in}{1.707047in}}{\pgfqpoint{1.242352in}{1.714860in}}%
\pgfpathcurveto{\pgfqpoint{1.234538in}{1.722674in}}{\pgfqpoint{1.223939in}{1.727064in}}{\pgfqpoint{1.212889in}{1.727064in}}%
\pgfpathcurveto{\pgfqpoint{1.201839in}{1.727064in}}{\pgfqpoint{1.191240in}{1.722674in}}{\pgfqpoint{1.183426in}{1.714860in}}%
\pgfpathcurveto{\pgfqpoint{1.175613in}{1.707047in}}{\pgfqpoint{1.171222in}{1.696448in}}{\pgfqpoint{1.171222in}{1.685398in}}%
\pgfpathcurveto{\pgfqpoint{1.171222in}{1.674348in}}{\pgfqpoint{1.175613in}{1.663749in}}{\pgfqpoint{1.183426in}{1.655935in}}%
\pgfpathcurveto{\pgfqpoint{1.191240in}{1.648121in}}{\pgfqpoint{1.201839in}{1.643731in}}{\pgfqpoint{1.212889in}{1.643731in}}%
\pgfpathclose%
\pgfusepath{stroke,fill}%
\end{pgfscope}%
\begin{pgfscope}%
\pgfpathrectangle{\pgfqpoint{0.481978in}{0.331635in}}{\pgfqpoint{4.960000in}{3.696000in}}%
\pgfusepath{clip}%
\pgfsetbuttcap%
\pgfsetroundjoin%
\definecolor{currentfill}{rgb}{1.000000,0.705882,0.509804}%
\pgfsetfillcolor{currentfill}%
\pgfsetlinewidth{0.481800pt}%
\definecolor{currentstroke}{rgb}{1.000000,1.000000,1.000000}%
\pgfsetstrokecolor{currentstroke}%
\pgfsetdash{}{0pt}%
\pgfpathmoveto{\pgfqpoint{1.333944in}{3.011283in}}%
\pgfpathcurveto{\pgfqpoint{1.344994in}{3.011283in}}{\pgfqpoint{1.355593in}{3.015674in}}{\pgfqpoint{1.363407in}{3.023487in}}%
\pgfpathcurveto{\pgfqpoint{1.371221in}{3.031301in}}{\pgfqpoint{1.375611in}{3.041900in}}{\pgfqpoint{1.375611in}{3.052950in}}%
\pgfpathcurveto{\pgfqpoint{1.375611in}{3.064000in}}{\pgfqpoint{1.371221in}{3.074599in}}{\pgfqpoint{1.363407in}{3.082413in}}%
\pgfpathcurveto{\pgfqpoint{1.355593in}{3.090226in}}{\pgfqpoint{1.344994in}{3.094617in}}{\pgfqpoint{1.333944in}{3.094617in}}%
\pgfpathcurveto{\pgfqpoint{1.322894in}{3.094617in}}{\pgfqpoint{1.312295in}{3.090226in}}{\pgfqpoint{1.304481in}{3.082413in}}%
\pgfpathcurveto{\pgfqpoint{1.296668in}{3.074599in}}{\pgfqpoint{1.292277in}{3.064000in}}{\pgfqpoint{1.292277in}{3.052950in}}%
\pgfpathcurveto{\pgfqpoint{1.292277in}{3.041900in}}{\pgfqpoint{1.296668in}{3.031301in}}{\pgfqpoint{1.304481in}{3.023487in}}%
\pgfpathcurveto{\pgfqpoint{1.312295in}{3.015674in}}{\pgfqpoint{1.322894in}{3.011283in}}{\pgfqpoint{1.333944in}{3.011283in}}%
\pgfpathclose%
\pgfusepath{stroke,fill}%
\end{pgfscope}%
\begin{pgfscope}%
\pgfpathrectangle{\pgfqpoint{0.481978in}{0.331635in}}{\pgfqpoint{4.960000in}{3.696000in}}%
\pgfusepath{clip}%
\pgfsetbuttcap%
\pgfsetroundjoin%
\definecolor{currentfill}{rgb}{1.000000,0.705882,0.509804}%
\pgfsetfillcolor{currentfill}%
\pgfsetlinewidth{0.481800pt}%
\definecolor{currentstroke}{rgb}{1.000000,1.000000,1.000000}%
\pgfsetstrokecolor{currentstroke}%
\pgfsetdash{}{0pt}%
\pgfpathmoveto{\pgfqpoint{1.910611in}{2.938280in}}%
\pgfpathcurveto{\pgfqpoint{1.921662in}{2.938280in}}{\pgfqpoint{1.932261in}{2.942671in}}{\pgfqpoint{1.940074in}{2.950484in}}%
\pgfpathcurveto{\pgfqpoint{1.947888in}{2.958298in}}{\pgfqpoint{1.952278in}{2.968897in}}{\pgfqpoint{1.952278in}{2.979947in}}%
\pgfpathcurveto{\pgfqpoint{1.952278in}{2.990997in}}{\pgfqpoint{1.947888in}{3.001596in}}{\pgfqpoint{1.940074in}{3.009410in}}%
\pgfpathcurveto{\pgfqpoint{1.932261in}{3.017223in}}{\pgfqpoint{1.921662in}{3.021614in}}{\pgfqpoint{1.910611in}{3.021614in}}%
\pgfpathcurveto{\pgfqpoint{1.899561in}{3.021614in}}{\pgfqpoint{1.888962in}{3.017223in}}{\pgfqpoint{1.881149in}{3.009410in}}%
\pgfpathcurveto{\pgfqpoint{1.873335in}{3.001596in}}{\pgfqpoint{1.868945in}{2.990997in}}{\pgfqpoint{1.868945in}{2.979947in}}%
\pgfpathcurveto{\pgfqpoint{1.868945in}{2.968897in}}{\pgfqpoint{1.873335in}{2.958298in}}{\pgfqpoint{1.881149in}{2.950484in}}%
\pgfpathcurveto{\pgfqpoint{1.888962in}{2.942671in}}{\pgfqpoint{1.899561in}{2.938280in}}{\pgfqpoint{1.910611in}{2.938280in}}%
\pgfpathclose%
\pgfusepath{stroke,fill}%
\end{pgfscope}%
\begin{pgfscope}%
\pgfpathrectangle{\pgfqpoint{0.481978in}{0.331635in}}{\pgfqpoint{4.960000in}{3.696000in}}%
\pgfusepath{clip}%
\pgfsetbuttcap%
\pgfsetroundjoin%
\definecolor{currentfill}{rgb}{1.000000,0.705882,0.509804}%
\pgfsetfillcolor{currentfill}%
\pgfsetlinewidth{0.481800pt}%
\definecolor{currentstroke}{rgb}{1.000000,1.000000,1.000000}%
\pgfsetstrokecolor{currentstroke}%
\pgfsetdash{}{0pt}%
\pgfpathmoveto{\pgfqpoint{1.546438in}{2.944111in}}%
\pgfpathcurveto{\pgfqpoint{1.557488in}{2.944111in}}{\pgfqpoint{1.568087in}{2.948501in}}{\pgfqpoint{1.575901in}{2.956314in}}%
\pgfpathcurveto{\pgfqpoint{1.583714in}{2.964128in}}{\pgfqpoint{1.588105in}{2.974727in}}{\pgfqpoint{1.588105in}{2.985777in}}%
\pgfpathcurveto{\pgfqpoint{1.588105in}{2.996827in}}{\pgfqpoint{1.583714in}{3.007426in}}{\pgfqpoint{1.575901in}{3.015240in}}%
\pgfpathcurveto{\pgfqpoint{1.568087in}{3.023054in}}{\pgfqpoint{1.557488in}{3.027444in}}{\pgfqpoint{1.546438in}{3.027444in}}%
\pgfpathcurveto{\pgfqpoint{1.535388in}{3.027444in}}{\pgfqpoint{1.524789in}{3.023054in}}{\pgfqpoint{1.516975in}{3.015240in}}%
\pgfpathcurveto{\pgfqpoint{1.509162in}{3.007426in}}{\pgfqpoint{1.504771in}{2.996827in}}{\pgfqpoint{1.504771in}{2.985777in}}%
\pgfpathcurveto{\pgfqpoint{1.504771in}{2.974727in}}{\pgfqpoint{1.509162in}{2.964128in}}{\pgfqpoint{1.516975in}{2.956314in}}%
\pgfpathcurveto{\pgfqpoint{1.524789in}{2.948501in}}{\pgfqpoint{1.535388in}{2.944111in}}{\pgfqpoint{1.546438in}{2.944111in}}%
\pgfpathclose%
\pgfusepath{stroke,fill}%
\end{pgfscope}%
\begin{pgfscope}%
\pgfpathrectangle{\pgfqpoint{0.481978in}{0.331635in}}{\pgfqpoint{4.960000in}{3.696000in}}%
\pgfusepath{clip}%
\pgfsetbuttcap%
\pgfsetroundjoin%
\definecolor{currentfill}{rgb}{1.000000,0.705882,0.509804}%
\pgfsetfillcolor{currentfill}%
\pgfsetlinewidth{0.481800pt}%
\definecolor{currentstroke}{rgb}{1.000000,1.000000,1.000000}%
\pgfsetstrokecolor{currentstroke}%
\pgfsetdash{}{0pt}%
\pgfpathmoveto{\pgfqpoint{2.026229in}{2.541848in}}%
\pgfpathcurveto{\pgfqpoint{2.037279in}{2.541848in}}{\pgfqpoint{2.047878in}{2.546238in}}{\pgfqpoint{2.055692in}{2.554051in}}%
\pgfpathcurveto{\pgfqpoint{2.063506in}{2.561865in}}{\pgfqpoint{2.067896in}{2.572464in}}{\pgfqpoint{2.067896in}{2.583514in}}%
\pgfpathcurveto{\pgfqpoint{2.067896in}{2.594564in}}{\pgfqpoint{2.063506in}{2.605163in}}{\pgfqpoint{2.055692in}{2.612977in}}%
\pgfpathcurveto{\pgfqpoint{2.047878in}{2.620791in}}{\pgfqpoint{2.037279in}{2.625181in}}{\pgfqpoint{2.026229in}{2.625181in}}%
\pgfpathcurveto{\pgfqpoint{2.015179in}{2.625181in}}{\pgfqpoint{2.004580in}{2.620791in}}{\pgfqpoint{1.996766in}{2.612977in}}%
\pgfpathcurveto{\pgfqpoint{1.988953in}{2.605163in}}{\pgfqpoint{1.984562in}{2.594564in}}{\pgfqpoint{1.984562in}{2.583514in}}%
\pgfpathcurveto{\pgfqpoint{1.984562in}{2.572464in}}{\pgfqpoint{1.988953in}{2.561865in}}{\pgfqpoint{1.996766in}{2.554051in}}%
\pgfpathcurveto{\pgfqpoint{2.004580in}{2.546238in}}{\pgfqpoint{2.015179in}{2.541848in}}{\pgfqpoint{2.026229in}{2.541848in}}%
\pgfpathclose%
\pgfusepath{stroke,fill}%
\end{pgfscope}%
\begin{pgfscope}%
\pgfpathrectangle{\pgfqpoint{0.481978in}{0.331635in}}{\pgfqpoint{4.960000in}{3.696000in}}%
\pgfusepath{clip}%
\pgfsetbuttcap%
\pgfsetroundjoin%
\definecolor{currentfill}{rgb}{1.000000,0.705882,0.509804}%
\pgfsetfillcolor{currentfill}%
\pgfsetlinewidth{0.481800pt}%
\definecolor{currentstroke}{rgb}{1.000000,1.000000,1.000000}%
\pgfsetstrokecolor{currentstroke}%
\pgfsetdash{}{0pt}%
\pgfpathmoveto{\pgfqpoint{2.497138in}{1.829984in}}%
\pgfpathcurveto{\pgfqpoint{2.508188in}{1.829984in}}{\pgfqpoint{2.518787in}{1.834374in}}{\pgfqpoint{2.526601in}{1.842188in}}%
\pgfpathcurveto{\pgfqpoint{2.534415in}{1.850002in}}{\pgfqpoint{2.538805in}{1.860601in}}{\pgfqpoint{2.538805in}{1.871651in}}%
\pgfpathcurveto{\pgfqpoint{2.538805in}{1.882701in}}{\pgfqpoint{2.534415in}{1.893300in}}{\pgfqpoint{2.526601in}{1.901114in}}%
\pgfpathcurveto{\pgfqpoint{2.518787in}{1.908927in}}{\pgfqpoint{2.508188in}{1.913317in}}{\pgfqpoint{2.497138in}{1.913317in}}%
\pgfpathcurveto{\pgfqpoint{2.486088in}{1.913317in}}{\pgfqpoint{2.475489in}{1.908927in}}{\pgfqpoint{2.467676in}{1.901114in}}%
\pgfpathcurveto{\pgfqpoint{2.459862in}{1.893300in}}{\pgfqpoint{2.455472in}{1.882701in}}{\pgfqpoint{2.455472in}{1.871651in}}%
\pgfpathcurveto{\pgfqpoint{2.455472in}{1.860601in}}{\pgfqpoint{2.459862in}{1.850002in}}{\pgfqpoint{2.467676in}{1.842188in}}%
\pgfpathcurveto{\pgfqpoint{2.475489in}{1.834374in}}{\pgfqpoint{2.486088in}{1.829984in}}{\pgfqpoint{2.497138in}{1.829984in}}%
\pgfpathclose%
\pgfusepath{stroke,fill}%
\end{pgfscope}%
\begin{pgfscope}%
\pgfpathrectangle{\pgfqpoint{0.481978in}{0.331635in}}{\pgfqpoint{4.960000in}{3.696000in}}%
\pgfusepath{clip}%
\pgfsetbuttcap%
\pgfsetroundjoin%
\definecolor{currentfill}{rgb}{1.000000,0.705882,0.509804}%
\pgfsetfillcolor{currentfill}%
\pgfsetlinewidth{0.481800pt}%
\definecolor{currentstroke}{rgb}{1.000000,1.000000,1.000000}%
\pgfsetstrokecolor{currentstroke}%
\pgfsetdash{}{0pt}%
\pgfpathmoveto{\pgfqpoint{2.724487in}{2.926406in}}%
\pgfpathcurveto{\pgfqpoint{2.735537in}{2.926406in}}{\pgfqpoint{2.746136in}{2.930797in}}{\pgfqpoint{2.753950in}{2.938610in}}%
\pgfpathcurveto{\pgfqpoint{2.761763in}{2.946424in}}{\pgfqpoint{2.766154in}{2.957023in}}{\pgfqpoint{2.766154in}{2.968073in}}%
\pgfpathcurveto{\pgfqpoint{2.766154in}{2.979123in}}{\pgfqpoint{2.761763in}{2.989722in}}{\pgfqpoint{2.753950in}{2.997536in}}%
\pgfpathcurveto{\pgfqpoint{2.746136in}{3.005349in}}{\pgfqpoint{2.735537in}{3.009740in}}{\pgfqpoint{2.724487in}{3.009740in}}%
\pgfpathcurveto{\pgfqpoint{2.713437in}{3.009740in}}{\pgfqpoint{2.702838in}{3.005349in}}{\pgfqpoint{2.695024in}{2.997536in}}%
\pgfpathcurveto{\pgfqpoint{2.687210in}{2.989722in}}{\pgfqpoint{2.682820in}{2.979123in}}{\pgfqpoint{2.682820in}{2.968073in}}%
\pgfpathcurveto{\pgfqpoint{2.682820in}{2.957023in}}{\pgfqpoint{2.687210in}{2.946424in}}{\pgfqpoint{2.695024in}{2.938610in}}%
\pgfpathcurveto{\pgfqpoint{2.702838in}{2.930797in}}{\pgfqpoint{2.713437in}{2.926406in}}{\pgfqpoint{2.724487in}{2.926406in}}%
\pgfpathclose%
\pgfusepath{stroke,fill}%
\end{pgfscope}%
\begin{pgfscope}%
\pgfpathrectangle{\pgfqpoint{0.481978in}{0.331635in}}{\pgfqpoint{4.960000in}{3.696000in}}%
\pgfusepath{clip}%
\pgfsetbuttcap%
\pgfsetroundjoin%
\definecolor{currentfill}{rgb}{1.000000,0.705882,0.509804}%
\pgfsetfillcolor{currentfill}%
\pgfsetlinewidth{0.481800pt}%
\definecolor{currentstroke}{rgb}{1.000000,1.000000,1.000000}%
\pgfsetstrokecolor{currentstroke}%
\pgfsetdash{}{0pt}%
\pgfpathmoveto{\pgfqpoint{1.592176in}{3.036537in}}%
\pgfpathcurveto{\pgfqpoint{1.603226in}{3.036537in}}{\pgfqpoint{1.613825in}{3.040927in}}{\pgfqpoint{1.621638in}{3.048741in}}%
\pgfpathcurveto{\pgfqpoint{1.629452in}{3.056555in}}{\pgfqpoint{1.633842in}{3.067154in}}{\pgfqpoint{1.633842in}{3.078204in}}%
\pgfpathcurveto{\pgfqpoint{1.633842in}{3.089254in}}{\pgfqpoint{1.629452in}{3.099853in}}{\pgfqpoint{1.621638in}{3.107667in}}%
\pgfpathcurveto{\pgfqpoint{1.613825in}{3.115480in}}{\pgfqpoint{1.603226in}{3.119871in}}{\pgfqpoint{1.592176in}{3.119871in}}%
\pgfpathcurveto{\pgfqpoint{1.581125in}{3.119871in}}{\pgfqpoint{1.570526in}{3.115480in}}{\pgfqpoint{1.562713in}{3.107667in}}%
\pgfpathcurveto{\pgfqpoint{1.554899in}{3.099853in}}{\pgfqpoint{1.550509in}{3.089254in}}{\pgfqpoint{1.550509in}{3.078204in}}%
\pgfpathcurveto{\pgfqpoint{1.550509in}{3.067154in}}{\pgfqpoint{1.554899in}{3.056555in}}{\pgfqpoint{1.562713in}{3.048741in}}%
\pgfpathcurveto{\pgfqpoint{1.570526in}{3.040927in}}{\pgfqpoint{1.581125in}{3.036537in}}{\pgfqpoint{1.592176in}{3.036537in}}%
\pgfpathclose%
\pgfusepath{stroke,fill}%
\end{pgfscope}%
\begin{pgfscope}%
\pgfpathrectangle{\pgfqpoint{0.481978in}{0.331635in}}{\pgfqpoint{4.960000in}{3.696000in}}%
\pgfusepath{clip}%
\pgfsetbuttcap%
\pgfsetroundjoin%
\definecolor{currentfill}{rgb}{1.000000,0.705882,0.509804}%
\pgfsetfillcolor{currentfill}%
\pgfsetlinewidth{0.481800pt}%
\definecolor{currentstroke}{rgb}{1.000000,1.000000,1.000000}%
\pgfsetstrokecolor{currentstroke}%
\pgfsetdash{}{0pt}%
\pgfpathmoveto{\pgfqpoint{1.230798in}{1.438306in}}%
\pgfpathcurveto{\pgfqpoint{1.241848in}{1.438306in}}{\pgfqpoint{1.252448in}{1.442696in}}{\pgfqpoint{1.260261in}{1.450510in}}%
\pgfpathcurveto{\pgfqpoint{1.268075in}{1.458323in}}{\pgfqpoint{1.272465in}{1.468922in}}{\pgfqpoint{1.272465in}{1.479973in}}%
\pgfpathcurveto{\pgfqpoint{1.272465in}{1.491023in}}{\pgfqpoint{1.268075in}{1.501622in}}{\pgfqpoint{1.260261in}{1.509435in}}%
\pgfpathcurveto{\pgfqpoint{1.252448in}{1.517249in}}{\pgfqpoint{1.241848in}{1.521639in}}{\pgfqpoint{1.230798in}{1.521639in}}%
\pgfpathcurveto{\pgfqpoint{1.219748in}{1.521639in}}{\pgfqpoint{1.209149in}{1.517249in}}{\pgfqpoint{1.201336in}{1.509435in}}%
\pgfpathcurveto{\pgfqpoint{1.193522in}{1.501622in}}{\pgfqpoint{1.189132in}{1.491023in}}{\pgfqpoint{1.189132in}{1.479973in}}%
\pgfpathcurveto{\pgfqpoint{1.189132in}{1.468922in}}{\pgfqpoint{1.193522in}{1.458323in}}{\pgfqpoint{1.201336in}{1.450510in}}%
\pgfpathcurveto{\pgfqpoint{1.209149in}{1.442696in}}{\pgfqpoint{1.219748in}{1.438306in}}{\pgfqpoint{1.230798in}{1.438306in}}%
\pgfpathclose%
\pgfusepath{stroke,fill}%
\end{pgfscope}%
\begin{pgfscope}%
\pgfpathrectangle{\pgfqpoint{0.481978in}{0.331635in}}{\pgfqpoint{4.960000in}{3.696000in}}%
\pgfusepath{clip}%
\pgfsetbuttcap%
\pgfsetroundjoin%
\definecolor{currentfill}{rgb}{1.000000,0.705882,0.509804}%
\pgfsetfillcolor{currentfill}%
\pgfsetlinewidth{0.481800pt}%
\definecolor{currentstroke}{rgb}{1.000000,1.000000,1.000000}%
\pgfsetstrokecolor{currentstroke}%
\pgfsetdash{}{0pt}%
\pgfpathmoveto{\pgfqpoint{1.742469in}{2.440415in}}%
\pgfpathcurveto{\pgfqpoint{1.753519in}{2.440415in}}{\pgfqpoint{1.764118in}{2.444805in}}{\pgfqpoint{1.771931in}{2.452619in}}%
\pgfpathcurveto{\pgfqpoint{1.779745in}{2.460433in}}{\pgfqpoint{1.784135in}{2.471032in}}{\pgfqpoint{1.784135in}{2.482082in}}%
\pgfpathcurveto{\pgfqpoint{1.784135in}{2.493132in}}{\pgfqpoint{1.779745in}{2.503731in}}{\pgfqpoint{1.771931in}{2.511545in}}%
\pgfpathcurveto{\pgfqpoint{1.764118in}{2.519358in}}{\pgfqpoint{1.753519in}{2.523748in}}{\pgfqpoint{1.742469in}{2.523748in}}%
\pgfpathcurveto{\pgfqpoint{1.731418in}{2.523748in}}{\pgfqpoint{1.720819in}{2.519358in}}{\pgfqpoint{1.713006in}{2.511545in}}%
\pgfpathcurveto{\pgfqpoint{1.705192in}{2.503731in}}{\pgfqpoint{1.700802in}{2.493132in}}{\pgfqpoint{1.700802in}{2.482082in}}%
\pgfpathcurveto{\pgfqpoint{1.700802in}{2.471032in}}{\pgfqpoint{1.705192in}{2.460433in}}{\pgfqpoint{1.713006in}{2.452619in}}%
\pgfpathcurveto{\pgfqpoint{1.720819in}{2.444805in}}{\pgfqpoint{1.731418in}{2.440415in}}{\pgfqpoint{1.742469in}{2.440415in}}%
\pgfpathclose%
\pgfusepath{stroke,fill}%
\end{pgfscope}%
\begin{pgfscope}%
\pgfpathrectangle{\pgfqpoint{0.481978in}{0.331635in}}{\pgfqpoint{4.960000in}{3.696000in}}%
\pgfusepath{clip}%
\pgfsetbuttcap%
\pgfsetroundjoin%
\definecolor{currentfill}{rgb}{1.000000,0.705882,0.509804}%
\pgfsetfillcolor{currentfill}%
\pgfsetlinewidth{0.481800pt}%
\definecolor{currentstroke}{rgb}{1.000000,1.000000,1.000000}%
\pgfsetstrokecolor{currentstroke}%
\pgfsetdash{}{0pt}%
\pgfpathmoveto{\pgfqpoint{1.926850in}{3.755036in}}%
\pgfpathcurveto{\pgfqpoint{1.937901in}{3.755036in}}{\pgfqpoint{1.948500in}{3.759426in}}{\pgfqpoint{1.956313in}{3.767240in}}%
\pgfpathcurveto{\pgfqpoint{1.964127in}{3.775053in}}{\pgfqpoint{1.968517in}{3.785652in}}{\pgfqpoint{1.968517in}{3.796702in}}%
\pgfpathcurveto{\pgfqpoint{1.968517in}{3.807753in}}{\pgfqpoint{1.964127in}{3.818352in}}{\pgfqpoint{1.956313in}{3.826165in}}%
\pgfpathcurveto{\pgfqpoint{1.948500in}{3.833979in}}{\pgfqpoint{1.937901in}{3.838369in}}{\pgfqpoint{1.926850in}{3.838369in}}%
\pgfpathcurveto{\pgfqpoint{1.915800in}{3.838369in}}{\pgfqpoint{1.905201in}{3.833979in}}{\pgfqpoint{1.897388in}{3.826165in}}%
\pgfpathcurveto{\pgfqpoint{1.889574in}{3.818352in}}{\pgfqpoint{1.885184in}{3.807753in}}{\pgfqpoint{1.885184in}{3.796702in}}%
\pgfpathcurveto{\pgfqpoint{1.885184in}{3.785652in}}{\pgfqpoint{1.889574in}{3.775053in}}{\pgfqpoint{1.897388in}{3.767240in}}%
\pgfpathcurveto{\pgfqpoint{1.905201in}{3.759426in}}{\pgfqpoint{1.915800in}{3.755036in}}{\pgfqpoint{1.926850in}{3.755036in}}%
\pgfpathclose%
\pgfusepath{stroke,fill}%
\end{pgfscope}%
\begin{pgfscope}%
\pgfpathrectangle{\pgfqpoint{0.481978in}{0.331635in}}{\pgfqpoint{4.960000in}{3.696000in}}%
\pgfusepath{clip}%
\pgfsetbuttcap%
\pgfsetroundjoin%
\definecolor{currentfill}{rgb}{1.000000,0.705882,0.509804}%
\pgfsetfillcolor{currentfill}%
\pgfsetlinewidth{0.481800pt}%
\definecolor{currentstroke}{rgb}{1.000000,1.000000,1.000000}%
\pgfsetstrokecolor{currentstroke}%
\pgfsetdash{}{0pt}%
\pgfpathmoveto{\pgfqpoint{3.551106in}{3.404239in}}%
\pgfpathcurveto{\pgfqpoint{3.562156in}{3.404239in}}{\pgfqpoint{3.572755in}{3.408629in}}{\pgfqpoint{3.580569in}{3.416443in}}%
\pgfpathcurveto{\pgfqpoint{3.588382in}{3.424256in}}{\pgfqpoint{3.592773in}{3.434855in}}{\pgfqpoint{3.592773in}{3.445906in}}%
\pgfpathcurveto{\pgfqpoint{3.592773in}{3.456956in}}{\pgfqpoint{3.588382in}{3.467555in}}{\pgfqpoint{3.580569in}{3.475368in}}%
\pgfpathcurveto{\pgfqpoint{3.572755in}{3.483182in}}{\pgfqpoint{3.562156in}{3.487572in}}{\pgfqpoint{3.551106in}{3.487572in}}%
\pgfpathcurveto{\pgfqpoint{3.540056in}{3.487572in}}{\pgfqpoint{3.529457in}{3.483182in}}{\pgfqpoint{3.521643in}{3.475368in}}%
\pgfpathcurveto{\pgfqpoint{3.513830in}{3.467555in}}{\pgfqpoint{3.509439in}{3.456956in}}{\pgfqpoint{3.509439in}{3.445906in}}%
\pgfpathcurveto{\pgfqpoint{3.509439in}{3.434855in}}{\pgfqpoint{3.513830in}{3.424256in}}{\pgfqpoint{3.521643in}{3.416443in}}%
\pgfpathcurveto{\pgfqpoint{3.529457in}{3.408629in}}{\pgfqpoint{3.540056in}{3.404239in}}{\pgfqpoint{3.551106in}{3.404239in}}%
\pgfpathclose%
\pgfusepath{stroke,fill}%
\end{pgfscope}%
\begin{pgfscope}%
\pgfpathrectangle{\pgfqpoint{0.481978in}{0.331635in}}{\pgfqpoint{4.960000in}{3.696000in}}%
\pgfusepath{clip}%
\pgfsetbuttcap%
\pgfsetroundjoin%
\definecolor{currentfill}{rgb}{1.000000,0.705882,0.509804}%
\pgfsetfillcolor{currentfill}%
\pgfsetlinewidth{0.481800pt}%
\definecolor{currentstroke}{rgb}{1.000000,1.000000,1.000000}%
\pgfsetstrokecolor{currentstroke}%
\pgfsetdash{}{0pt}%
\pgfpathmoveto{\pgfqpoint{1.414391in}{1.541468in}}%
\pgfpathcurveto{\pgfqpoint{1.425441in}{1.541468in}}{\pgfqpoint{1.436040in}{1.545858in}}{\pgfqpoint{1.443853in}{1.553672in}}%
\pgfpathcurveto{\pgfqpoint{1.451667in}{1.561485in}}{\pgfqpoint{1.456057in}{1.572084in}}{\pgfqpoint{1.456057in}{1.583135in}}%
\pgfpathcurveto{\pgfqpoint{1.456057in}{1.594185in}}{\pgfqpoint{1.451667in}{1.604784in}}{\pgfqpoint{1.443853in}{1.612597in}}%
\pgfpathcurveto{\pgfqpoint{1.436040in}{1.620411in}}{\pgfqpoint{1.425441in}{1.624801in}}{\pgfqpoint{1.414391in}{1.624801in}}%
\pgfpathcurveto{\pgfqpoint{1.403341in}{1.624801in}}{\pgfqpoint{1.392742in}{1.620411in}}{\pgfqpoint{1.384928in}{1.612597in}}%
\pgfpathcurveto{\pgfqpoint{1.377114in}{1.604784in}}{\pgfqpoint{1.372724in}{1.594185in}}{\pgfqpoint{1.372724in}{1.583135in}}%
\pgfpathcurveto{\pgfqpoint{1.372724in}{1.572084in}}{\pgfqpoint{1.377114in}{1.561485in}}{\pgfqpoint{1.384928in}{1.553672in}}%
\pgfpathcurveto{\pgfqpoint{1.392742in}{1.545858in}}{\pgfqpoint{1.403341in}{1.541468in}}{\pgfqpoint{1.414391in}{1.541468in}}%
\pgfpathclose%
\pgfusepath{stroke,fill}%
\end{pgfscope}%
\begin{pgfscope}%
\pgfpathrectangle{\pgfqpoint{0.481978in}{0.331635in}}{\pgfqpoint{4.960000in}{3.696000in}}%
\pgfusepath{clip}%
\pgfsetbuttcap%
\pgfsetroundjoin%
\definecolor{currentfill}{rgb}{1.000000,0.705882,0.509804}%
\pgfsetfillcolor{currentfill}%
\pgfsetlinewidth{0.481800pt}%
\definecolor{currentstroke}{rgb}{1.000000,1.000000,1.000000}%
\pgfsetstrokecolor{currentstroke}%
\pgfsetdash{}{0pt}%
\pgfpathmoveto{\pgfqpoint{2.597999in}{3.752992in}}%
\pgfpathcurveto{\pgfqpoint{2.609049in}{3.752992in}}{\pgfqpoint{2.619648in}{3.757382in}}{\pgfqpoint{2.627462in}{3.765196in}}%
\pgfpathcurveto{\pgfqpoint{2.635275in}{3.773009in}}{\pgfqpoint{2.639666in}{3.783608in}}{\pgfqpoint{2.639666in}{3.794659in}}%
\pgfpathcurveto{\pgfqpoint{2.639666in}{3.805709in}}{\pgfqpoint{2.635275in}{3.816308in}}{\pgfqpoint{2.627462in}{3.824121in}}%
\pgfpathcurveto{\pgfqpoint{2.619648in}{3.831935in}}{\pgfqpoint{2.609049in}{3.836325in}}{\pgfqpoint{2.597999in}{3.836325in}}%
\pgfpathcurveto{\pgfqpoint{2.586949in}{3.836325in}}{\pgfqpoint{2.576350in}{3.831935in}}{\pgfqpoint{2.568536in}{3.824121in}}%
\pgfpathcurveto{\pgfqpoint{2.560722in}{3.816308in}}{\pgfqpoint{2.556332in}{3.805709in}}{\pgfqpoint{2.556332in}{3.794659in}}%
\pgfpathcurveto{\pgfqpoint{2.556332in}{3.783608in}}{\pgfqpoint{2.560722in}{3.773009in}}{\pgfqpoint{2.568536in}{3.765196in}}%
\pgfpathcurveto{\pgfqpoint{2.576350in}{3.757382in}}{\pgfqpoint{2.586949in}{3.752992in}}{\pgfqpoint{2.597999in}{3.752992in}}%
\pgfpathclose%
\pgfusepath{stroke,fill}%
\end{pgfscope}%
\begin{pgfscope}%
\pgfpathrectangle{\pgfqpoint{0.481978in}{0.331635in}}{\pgfqpoint{4.960000in}{3.696000in}}%
\pgfusepath{clip}%
\pgfsetbuttcap%
\pgfsetroundjoin%
\definecolor{currentfill}{rgb}{1.000000,0.705882,0.509804}%
\pgfsetfillcolor{currentfill}%
\pgfsetlinewidth{0.481800pt}%
\definecolor{currentstroke}{rgb}{1.000000,1.000000,1.000000}%
\pgfsetstrokecolor{currentstroke}%
\pgfsetdash{}{0pt}%
\pgfpathmoveto{\pgfqpoint{1.145042in}{3.190937in}}%
\pgfpathcurveto{\pgfqpoint{1.156092in}{3.190937in}}{\pgfqpoint{1.166691in}{3.195327in}}{\pgfqpoint{1.174505in}{3.203141in}}%
\pgfpathcurveto{\pgfqpoint{1.182319in}{3.210955in}}{\pgfqpoint{1.186709in}{3.221554in}}{\pgfqpoint{1.186709in}{3.232604in}}%
\pgfpathcurveto{\pgfqpoint{1.186709in}{3.243654in}}{\pgfqpoint{1.182319in}{3.254253in}}{\pgfqpoint{1.174505in}{3.262067in}}%
\pgfpathcurveto{\pgfqpoint{1.166691in}{3.269880in}}{\pgfqpoint{1.156092in}{3.274271in}}{\pgfqpoint{1.145042in}{3.274271in}}%
\pgfpathcurveto{\pgfqpoint{1.133992in}{3.274271in}}{\pgfqpoint{1.123393in}{3.269880in}}{\pgfqpoint{1.115579in}{3.262067in}}%
\pgfpathcurveto{\pgfqpoint{1.107766in}{3.254253in}}{\pgfqpoint{1.103375in}{3.243654in}}{\pgfqpoint{1.103375in}{3.232604in}}%
\pgfpathcurveto{\pgfqpoint{1.103375in}{3.221554in}}{\pgfqpoint{1.107766in}{3.210955in}}{\pgfqpoint{1.115579in}{3.203141in}}%
\pgfpathcurveto{\pgfqpoint{1.123393in}{3.195327in}}{\pgfqpoint{1.133992in}{3.190937in}}{\pgfqpoint{1.145042in}{3.190937in}}%
\pgfpathclose%
\pgfusepath{stroke,fill}%
\end{pgfscope}%
\begin{pgfscope}%
\pgfpathrectangle{\pgfqpoint{0.481978in}{0.331635in}}{\pgfqpoint{4.960000in}{3.696000in}}%
\pgfusepath{clip}%
\pgfsetbuttcap%
\pgfsetroundjoin%
\definecolor{currentfill}{rgb}{1.000000,0.705882,0.509804}%
\pgfsetfillcolor{currentfill}%
\pgfsetlinewidth{0.481800pt}%
\definecolor{currentstroke}{rgb}{1.000000,1.000000,1.000000}%
\pgfsetstrokecolor{currentstroke}%
\pgfsetdash{}{0pt}%
\pgfpathmoveto{\pgfqpoint{2.661184in}{1.975855in}}%
\pgfpathcurveto{\pgfqpoint{2.672234in}{1.975855in}}{\pgfqpoint{2.682833in}{1.980245in}}{\pgfqpoint{2.690647in}{1.988059in}}%
\pgfpathcurveto{\pgfqpoint{2.698460in}{1.995872in}}{\pgfqpoint{2.702851in}{2.006471in}}{\pgfqpoint{2.702851in}{2.017521in}}%
\pgfpathcurveto{\pgfqpoint{2.702851in}{2.028572in}}{\pgfqpoint{2.698460in}{2.039171in}}{\pgfqpoint{2.690647in}{2.046984in}}%
\pgfpathcurveto{\pgfqpoint{2.682833in}{2.054798in}}{\pgfqpoint{2.672234in}{2.059188in}}{\pgfqpoint{2.661184in}{2.059188in}}%
\pgfpathcurveto{\pgfqpoint{2.650134in}{2.059188in}}{\pgfqpoint{2.639535in}{2.054798in}}{\pgfqpoint{2.631721in}{2.046984in}}%
\pgfpathcurveto{\pgfqpoint{2.623908in}{2.039171in}}{\pgfqpoint{2.619517in}{2.028572in}}{\pgfqpoint{2.619517in}{2.017521in}}%
\pgfpathcurveto{\pgfqpoint{2.619517in}{2.006471in}}{\pgfqpoint{2.623908in}{1.995872in}}{\pgfqpoint{2.631721in}{1.988059in}}%
\pgfpathcurveto{\pgfqpoint{2.639535in}{1.980245in}}{\pgfqpoint{2.650134in}{1.975855in}}{\pgfqpoint{2.661184in}{1.975855in}}%
\pgfpathclose%
\pgfusepath{stroke,fill}%
\end{pgfscope}%
\begin{pgfscope}%
\pgfpathrectangle{\pgfqpoint{0.481978in}{0.331635in}}{\pgfqpoint{4.960000in}{3.696000in}}%
\pgfusepath{clip}%
\pgfsetbuttcap%
\pgfsetroundjoin%
\definecolor{currentfill}{rgb}{1.000000,0.705882,0.509804}%
\pgfsetfillcolor{currentfill}%
\pgfsetlinewidth{0.481800pt}%
\definecolor{currentstroke}{rgb}{1.000000,1.000000,1.000000}%
\pgfsetstrokecolor{currentstroke}%
\pgfsetdash{}{0pt}%
\pgfpathmoveto{\pgfqpoint{1.054739in}{2.466714in}}%
\pgfpathcurveto{\pgfqpoint{1.065789in}{2.466714in}}{\pgfqpoint{1.076388in}{2.471104in}}{\pgfqpoint{1.084201in}{2.478918in}}%
\pgfpathcurveto{\pgfqpoint{1.092015in}{2.486732in}}{\pgfqpoint{1.096405in}{2.497331in}}{\pgfqpoint{1.096405in}{2.508381in}}%
\pgfpathcurveto{\pgfqpoint{1.096405in}{2.519431in}}{\pgfqpoint{1.092015in}{2.530030in}}{\pgfqpoint{1.084201in}{2.537844in}}%
\pgfpathcurveto{\pgfqpoint{1.076388in}{2.545657in}}{\pgfqpoint{1.065789in}{2.550047in}}{\pgfqpoint{1.054739in}{2.550047in}}%
\pgfpathcurveto{\pgfqpoint{1.043688in}{2.550047in}}{\pgfqpoint{1.033089in}{2.545657in}}{\pgfqpoint{1.025276in}{2.537844in}}%
\pgfpathcurveto{\pgfqpoint{1.017462in}{2.530030in}}{\pgfqpoint{1.013072in}{2.519431in}}{\pgfqpoint{1.013072in}{2.508381in}}%
\pgfpathcurveto{\pgfqpoint{1.013072in}{2.497331in}}{\pgfqpoint{1.017462in}{2.486732in}}{\pgfqpoint{1.025276in}{2.478918in}}%
\pgfpathcurveto{\pgfqpoint{1.033089in}{2.471104in}}{\pgfqpoint{1.043688in}{2.466714in}}{\pgfqpoint{1.054739in}{2.466714in}}%
\pgfpathclose%
\pgfusepath{stroke,fill}%
\end{pgfscope}%
\begin{pgfscope}%
\pgfpathrectangle{\pgfqpoint{0.481978in}{0.331635in}}{\pgfqpoint{4.960000in}{3.696000in}}%
\pgfusepath{clip}%
\pgfsetbuttcap%
\pgfsetroundjoin%
\definecolor{currentfill}{rgb}{1.000000,0.705882,0.509804}%
\pgfsetfillcolor{currentfill}%
\pgfsetlinewidth{0.481800pt}%
\definecolor{currentstroke}{rgb}{1.000000,1.000000,1.000000}%
\pgfsetstrokecolor{currentstroke}%
\pgfsetdash{}{0pt}%
\pgfpathmoveto{\pgfqpoint{1.362990in}{3.363684in}}%
\pgfpathcurveto{\pgfqpoint{1.374040in}{3.363684in}}{\pgfqpoint{1.384639in}{3.368074in}}{\pgfqpoint{1.392453in}{3.375888in}}%
\pgfpathcurveto{\pgfqpoint{1.400266in}{3.383702in}}{\pgfqpoint{1.404657in}{3.394301in}}{\pgfqpoint{1.404657in}{3.405351in}}%
\pgfpathcurveto{\pgfqpoint{1.404657in}{3.416401in}}{\pgfqpoint{1.400266in}{3.427000in}}{\pgfqpoint{1.392453in}{3.434814in}}%
\pgfpathcurveto{\pgfqpoint{1.384639in}{3.442627in}}{\pgfqpoint{1.374040in}{3.447018in}}{\pgfqpoint{1.362990in}{3.447018in}}%
\pgfpathcurveto{\pgfqpoint{1.351940in}{3.447018in}}{\pgfqpoint{1.341341in}{3.442627in}}{\pgfqpoint{1.333527in}{3.434814in}}%
\pgfpathcurveto{\pgfqpoint{1.325714in}{3.427000in}}{\pgfqpoint{1.321323in}{3.416401in}}{\pgfqpoint{1.321323in}{3.405351in}}%
\pgfpathcurveto{\pgfqpoint{1.321323in}{3.394301in}}{\pgfqpoint{1.325714in}{3.383702in}}{\pgfqpoint{1.333527in}{3.375888in}}%
\pgfpathcurveto{\pgfqpoint{1.341341in}{3.368074in}}{\pgfqpoint{1.351940in}{3.363684in}}{\pgfqpoint{1.362990in}{3.363684in}}%
\pgfpathclose%
\pgfusepath{stroke,fill}%
\end{pgfscope}%
\begin{pgfscope}%
\pgfpathrectangle{\pgfqpoint{0.481978in}{0.331635in}}{\pgfqpoint{4.960000in}{3.696000in}}%
\pgfusepath{clip}%
\pgfsetbuttcap%
\pgfsetroundjoin%
\definecolor{currentfill}{rgb}{1.000000,0.705882,0.509804}%
\pgfsetfillcolor{currentfill}%
\pgfsetlinewidth{0.481800pt}%
\definecolor{currentstroke}{rgb}{1.000000,1.000000,1.000000}%
\pgfsetstrokecolor{currentstroke}%
\pgfsetdash{}{0pt}%
\pgfpathmoveto{\pgfqpoint{1.473426in}{3.230182in}}%
\pgfpathcurveto{\pgfqpoint{1.484476in}{3.230182in}}{\pgfqpoint{1.495075in}{3.234573in}}{\pgfqpoint{1.502889in}{3.242386in}}%
\pgfpathcurveto{\pgfqpoint{1.510702in}{3.250200in}}{\pgfqpoint{1.515093in}{3.260799in}}{\pgfqpoint{1.515093in}{3.271849in}}%
\pgfpathcurveto{\pgfqpoint{1.515093in}{3.282899in}}{\pgfqpoint{1.510702in}{3.293498in}}{\pgfqpoint{1.502889in}{3.301312in}}%
\pgfpathcurveto{\pgfqpoint{1.495075in}{3.309125in}}{\pgfqpoint{1.484476in}{3.313516in}}{\pgfqpoint{1.473426in}{3.313516in}}%
\pgfpathcurveto{\pgfqpoint{1.462376in}{3.313516in}}{\pgfqpoint{1.451777in}{3.309125in}}{\pgfqpoint{1.443963in}{3.301312in}}%
\pgfpathcurveto{\pgfqpoint{1.436150in}{3.293498in}}{\pgfqpoint{1.431759in}{3.282899in}}{\pgfqpoint{1.431759in}{3.271849in}}%
\pgfpathcurveto{\pgfqpoint{1.431759in}{3.260799in}}{\pgfqpoint{1.436150in}{3.250200in}}{\pgfqpoint{1.443963in}{3.242386in}}%
\pgfpathcurveto{\pgfqpoint{1.451777in}{3.234573in}}{\pgfqpoint{1.462376in}{3.230182in}}{\pgfqpoint{1.473426in}{3.230182in}}%
\pgfpathclose%
\pgfusepath{stroke,fill}%
\end{pgfscope}%
\begin{pgfscope}%
\pgfpathrectangle{\pgfqpoint{0.481978in}{0.331635in}}{\pgfqpoint{4.960000in}{3.696000in}}%
\pgfusepath{clip}%
\pgfsetbuttcap%
\pgfsetroundjoin%
\definecolor{currentfill}{rgb}{1.000000,0.705882,0.509804}%
\pgfsetfillcolor{currentfill}%
\pgfsetlinewidth{0.481800pt}%
\definecolor{currentstroke}{rgb}{1.000000,1.000000,1.000000}%
\pgfsetstrokecolor{currentstroke}%
\pgfsetdash{}{0pt}%
\pgfpathmoveto{\pgfqpoint{3.352810in}{2.015282in}}%
\pgfpathcurveto{\pgfqpoint{3.363860in}{2.015282in}}{\pgfqpoint{3.374459in}{2.019672in}}{\pgfqpoint{3.382273in}{2.027486in}}%
\pgfpathcurveto{\pgfqpoint{3.390086in}{2.035299in}}{\pgfqpoint{3.394477in}{2.045898in}}{\pgfqpoint{3.394477in}{2.056948in}}%
\pgfpathcurveto{\pgfqpoint{3.394477in}{2.067999in}}{\pgfqpoint{3.390086in}{2.078598in}}{\pgfqpoint{3.382273in}{2.086411in}}%
\pgfpathcurveto{\pgfqpoint{3.374459in}{2.094225in}}{\pgfqpoint{3.363860in}{2.098615in}}{\pgfqpoint{3.352810in}{2.098615in}}%
\pgfpathcurveto{\pgfqpoint{3.341760in}{2.098615in}}{\pgfqpoint{3.331161in}{2.094225in}}{\pgfqpoint{3.323347in}{2.086411in}}%
\pgfpathcurveto{\pgfqpoint{3.315534in}{2.078598in}}{\pgfqpoint{3.311143in}{2.067999in}}{\pgfqpoint{3.311143in}{2.056948in}}%
\pgfpathcurveto{\pgfqpoint{3.311143in}{2.045898in}}{\pgfqpoint{3.315534in}{2.035299in}}{\pgfqpoint{3.323347in}{2.027486in}}%
\pgfpathcurveto{\pgfqpoint{3.331161in}{2.019672in}}{\pgfqpoint{3.341760in}{2.015282in}}{\pgfqpoint{3.352810in}{2.015282in}}%
\pgfpathclose%
\pgfusepath{stroke,fill}%
\end{pgfscope}%
\begin{pgfscope}%
\pgfpathrectangle{\pgfqpoint{0.481978in}{0.331635in}}{\pgfqpoint{4.960000in}{3.696000in}}%
\pgfusepath{clip}%
\pgfsetbuttcap%
\pgfsetroundjoin%
\definecolor{currentfill}{rgb}{1.000000,0.705882,0.509804}%
\pgfsetfillcolor{currentfill}%
\pgfsetlinewidth{0.481800pt}%
\definecolor{currentstroke}{rgb}{1.000000,1.000000,1.000000}%
\pgfsetstrokecolor{currentstroke}%
\pgfsetdash{}{0pt}%
\pgfpathmoveto{\pgfqpoint{5.216523in}{2.545803in}}%
\pgfpathcurveto{\pgfqpoint{5.227574in}{2.545803in}}{\pgfqpoint{5.238173in}{2.550193in}}{\pgfqpoint{5.245986in}{2.558007in}}%
\pgfpathcurveto{\pgfqpoint{5.253800in}{2.565820in}}{\pgfqpoint{5.258190in}{2.576419in}}{\pgfqpoint{5.258190in}{2.587469in}}%
\pgfpathcurveto{\pgfqpoint{5.258190in}{2.598520in}}{\pgfqpoint{5.253800in}{2.609119in}}{\pgfqpoint{5.245986in}{2.616932in}}%
\pgfpathcurveto{\pgfqpoint{5.238173in}{2.624746in}}{\pgfqpoint{5.227574in}{2.629136in}}{\pgfqpoint{5.216523in}{2.629136in}}%
\pgfpathcurveto{\pgfqpoint{5.205473in}{2.629136in}}{\pgfqpoint{5.194874in}{2.624746in}}{\pgfqpoint{5.187061in}{2.616932in}}%
\pgfpathcurveto{\pgfqpoint{5.179247in}{2.609119in}}{\pgfqpoint{5.174857in}{2.598520in}}{\pgfqpoint{5.174857in}{2.587469in}}%
\pgfpathcurveto{\pgfqpoint{5.174857in}{2.576419in}}{\pgfqpoint{5.179247in}{2.565820in}}{\pgfqpoint{5.187061in}{2.558007in}}%
\pgfpathcurveto{\pgfqpoint{5.194874in}{2.550193in}}{\pgfqpoint{5.205473in}{2.545803in}}{\pgfqpoint{5.216523in}{2.545803in}}%
\pgfpathclose%
\pgfusepath{stroke,fill}%
\end{pgfscope}%
\begin{pgfscope}%
\pgfpathrectangle{\pgfqpoint{0.481978in}{0.331635in}}{\pgfqpoint{4.960000in}{3.696000in}}%
\pgfusepath{clip}%
\pgfsetbuttcap%
\pgfsetroundjoin%
\definecolor{currentfill}{rgb}{1.000000,0.705882,0.509804}%
\pgfsetfillcolor{currentfill}%
\pgfsetlinewidth{0.481800pt}%
\definecolor{currentstroke}{rgb}{1.000000,1.000000,1.000000}%
\pgfsetstrokecolor{currentstroke}%
\pgfsetdash{}{0pt}%
\pgfpathmoveto{\pgfqpoint{2.975880in}{2.195140in}}%
\pgfpathcurveto{\pgfqpoint{2.986930in}{2.195140in}}{\pgfqpoint{2.997529in}{2.199531in}}{\pgfqpoint{3.005343in}{2.207344in}}%
\pgfpathcurveto{\pgfqpoint{3.013156in}{2.215158in}}{\pgfqpoint{3.017547in}{2.225757in}}{\pgfqpoint{3.017547in}{2.236807in}}%
\pgfpathcurveto{\pgfqpoint{3.017547in}{2.247857in}}{\pgfqpoint{3.013156in}{2.258456in}}{\pgfqpoint{3.005343in}{2.266270in}}%
\pgfpathcurveto{\pgfqpoint{2.997529in}{2.274083in}}{\pgfqpoint{2.986930in}{2.278474in}}{\pgfqpoint{2.975880in}{2.278474in}}%
\pgfpathcurveto{\pgfqpoint{2.964830in}{2.278474in}}{\pgfqpoint{2.954231in}{2.274083in}}{\pgfqpoint{2.946417in}{2.266270in}}%
\pgfpathcurveto{\pgfqpoint{2.938604in}{2.258456in}}{\pgfqpoint{2.934213in}{2.247857in}}{\pgfqpoint{2.934213in}{2.236807in}}%
\pgfpathcurveto{\pgfqpoint{2.934213in}{2.225757in}}{\pgfqpoint{2.938604in}{2.215158in}}{\pgfqpoint{2.946417in}{2.207344in}}%
\pgfpathcurveto{\pgfqpoint{2.954231in}{2.199531in}}{\pgfqpoint{2.964830in}{2.195140in}}{\pgfqpoint{2.975880in}{2.195140in}}%
\pgfpathclose%
\pgfusepath{stroke,fill}%
\end{pgfscope}%
\begin{pgfscope}%
\pgfpathrectangle{\pgfqpoint{0.481978in}{0.331635in}}{\pgfqpoint{4.960000in}{3.696000in}}%
\pgfusepath{clip}%
\pgfsetbuttcap%
\pgfsetroundjoin%
\definecolor{currentfill}{rgb}{1.000000,0.705882,0.509804}%
\pgfsetfillcolor{currentfill}%
\pgfsetlinewidth{0.481800pt}%
\definecolor{currentstroke}{rgb}{1.000000,1.000000,1.000000}%
\pgfsetstrokecolor{currentstroke}%
\pgfsetdash{}{0pt}%
\pgfpathmoveto{\pgfqpoint{5.097947in}{1.897786in}}%
\pgfpathcurveto{\pgfqpoint{5.108998in}{1.897786in}}{\pgfqpoint{5.119597in}{1.902177in}}{\pgfqpoint{5.127410in}{1.909990in}}%
\pgfpathcurveto{\pgfqpoint{5.135224in}{1.917804in}}{\pgfqpoint{5.139614in}{1.928403in}}{\pgfqpoint{5.139614in}{1.939453in}}%
\pgfpathcurveto{\pgfqpoint{5.139614in}{1.950503in}}{\pgfqpoint{5.135224in}{1.961102in}}{\pgfqpoint{5.127410in}{1.968916in}}%
\pgfpathcurveto{\pgfqpoint{5.119597in}{1.976729in}}{\pgfqpoint{5.108998in}{1.981120in}}{\pgfqpoint{5.097947in}{1.981120in}}%
\pgfpathcurveto{\pgfqpoint{5.086897in}{1.981120in}}{\pgfqpoint{5.076298in}{1.976729in}}{\pgfqpoint{5.068485in}{1.968916in}}%
\pgfpathcurveto{\pgfqpoint{5.060671in}{1.961102in}}{\pgfqpoint{5.056281in}{1.950503in}}{\pgfqpoint{5.056281in}{1.939453in}}%
\pgfpathcurveto{\pgfqpoint{5.056281in}{1.928403in}}{\pgfqpoint{5.060671in}{1.917804in}}{\pgfqpoint{5.068485in}{1.909990in}}%
\pgfpathcurveto{\pgfqpoint{5.076298in}{1.902177in}}{\pgfqpoint{5.086897in}{1.897786in}}{\pgfqpoint{5.097947in}{1.897786in}}%
\pgfpathclose%
\pgfusepath{stroke,fill}%
\end{pgfscope}%
\begin{pgfscope}%
\pgfpathrectangle{\pgfqpoint{0.481978in}{0.331635in}}{\pgfqpoint{4.960000in}{3.696000in}}%
\pgfusepath{clip}%
\pgfsetbuttcap%
\pgfsetroundjoin%
\definecolor{currentfill}{rgb}{1.000000,0.705882,0.509804}%
\pgfsetfillcolor{currentfill}%
\pgfsetlinewidth{0.481800pt}%
\definecolor{currentstroke}{rgb}{1.000000,1.000000,1.000000}%
\pgfsetstrokecolor{currentstroke}%
\pgfsetdash{}{0pt}%
\pgfpathmoveto{\pgfqpoint{2.202056in}{1.597496in}}%
\pgfpathcurveto{\pgfqpoint{2.213106in}{1.597496in}}{\pgfqpoint{2.223705in}{1.601887in}}{\pgfqpoint{2.231519in}{1.609700in}}%
\pgfpathcurveto{\pgfqpoint{2.239333in}{1.617514in}}{\pgfqpoint{2.243723in}{1.628113in}}{\pgfqpoint{2.243723in}{1.639163in}}%
\pgfpathcurveto{\pgfqpoint{2.243723in}{1.650213in}}{\pgfqpoint{2.239333in}{1.660812in}}{\pgfqpoint{2.231519in}{1.668626in}}%
\pgfpathcurveto{\pgfqpoint{2.223705in}{1.676439in}}{\pgfqpoint{2.213106in}{1.680830in}}{\pgfqpoint{2.202056in}{1.680830in}}%
\pgfpathcurveto{\pgfqpoint{2.191006in}{1.680830in}}{\pgfqpoint{2.180407in}{1.676439in}}{\pgfqpoint{2.172593in}{1.668626in}}%
\pgfpathcurveto{\pgfqpoint{2.164780in}{1.660812in}}{\pgfqpoint{2.160389in}{1.650213in}}{\pgfqpoint{2.160389in}{1.639163in}}%
\pgfpathcurveto{\pgfqpoint{2.160389in}{1.628113in}}{\pgfqpoint{2.164780in}{1.617514in}}{\pgfqpoint{2.172593in}{1.609700in}}%
\pgfpathcurveto{\pgfqpoint{2.180407in}{1.601887in}}{\pgfqpoint{2.191006in}{1.597496in}}{\pgfqpoint{2.202056in}{1.597496in}}%
\pgfpathclose%
\pgfusepath{stroke,fill}%
\end{pgfscope}%
\begin{pgfscope}%
\pgfpathrectangle{\pgfqpoint{0.481978in}{0.331635in}}{\pgfqpoint{4.960000in}{3.696000in}}%
\pgfusepath{clip}%
\pgfsetbuttcap%
\pgfsetroundjoin%
\definecolor{currentfill}{rgb}{1.000000,0.705882,0.509804}%
\pgfsetfillcolor{currentfill}%
\pgfsetlinewidth{0.481800pt}%
\definecolor{currentstroke}{rgb}{1.000000,1.000000,1.000000}%
\pgfsetstrokecolor{currentstroke}%
\pgfsetdash{}{0pt}%
\pgfpathmoveto{\pgfqpoint{2.355362in}{3.316389in}}%
\pgfpathcurveto{\pgfqpoint{2.366412in}{3.316389in}}{\pgfqpoint{2.377011in}{3.320779in}}{\pgfqpoint{2.384824in}{3.328592in}}%
\pgfpathcurveto{\pgfqpoint{2.392638in}{3.336406in}}{\pgfqpoint{2.397028in}{3.347005in}}{\pgfqpoint{2.397028in}{3.358055in}}%
\pgfpathcurveto{\pgfqpoint{2.397028in}{3.369105in}}{\pgfqpoint{2.392638in}{3.379704in}}{\pgfqpoint{2.384824in}{3.387518in}}%
\pgfpathcurveto{\pgfqpoint{2.377011in}{3.395332in}}{\pgfqpoint{2.366412in}{3.399722in}}{\pgfqpoint{2.355362in}{3.399722in}}%
\pgfpathcurveto{\pgfqpoint{2.344311in}{3.399722in}}{\pgfqpoint{2.333712in}{3.395332in}}{\pgfqpoint{2.325899in}{3.387518in}}%
\pgfpathcurveto{\pgfqpoint{2.318085in}{3.379704in}}{\pgfqpoint{2.313695in}{3.369105in}}{\pgfqpoint{2.313695in}{3.358055in}}%
\pgfpathcurveto{\pgfqpoint{2.313695in}{3.347005in}}{\pgfqpoint{2.318085in}{3.336406in}}{\pgfqpoint{2.325899in}{3.328592in}}%
\pgfpathcurveto{\pgfqpoint{2.333712in}{3.320779in}}{\pgfqpoint{2.344311in}{3.316389in}}{\pgfqpoint{2.355362in}{3.316389in}}%
\pgfpathclose%
\pgfusepath{stroke,fill}%
\end{pgfscope}%
\begin{pgfscope}%
\pgfpathrectangle{\pgfqpoint{0.481978in}{0.331635in}}{\pgfqpoint{4.960000in}{3.696000in}}%
\pgfusepath{clip}%
\pgfsetbuttcap%
\pgfsetroundjoin%
\definecolor{currentfill}{rgb}{1.000000,0.705882,0.509804}%
\pgfsetfillcolor{currentfill}%
\pgfsetlinewidth{0.481800pt}%
\definecolor{currentstroke}{rgb}{1.000000,1.000000,1.000000}%
\pgfsetstrokecolor{currentstroke}%
\pgfsetdash{}{0pt}%
\pgfpathmoveto{\pgfqpoint{1.182318in}{2.858811in}}%
\pgfpathcurveto{\pgfqpoint{1.193368in}{2.858811in}}{\pgfqpoint{1.203967in}{2.863202in}}{\pgfqpoint{1.211781in}{2.871015in}}%
\pgfpathcurveto{\pgfqpoint{1.219595in}{2.878829in}}{\pgfqpoint{1.223985in}{2.889428in}}{\pgfqpoint{1.223985in}{2.900478in}}%
\pgfpathcurveto{\pgfqpoint{1.223985in}{2.911528in}}{\pgfqpoint{1.219595in}{2.922127in}}{\pgfqpoint{1.211781in}{2.929941in}}%
\pgfpathcurveto{\pgfqpoint{1.203967in}{2.937754in}}{\pgfqpoint{1.193368in}{2.942145in}}{\pgfqpoint{1.182318in}{2.942145in}}%
\pgfpathcurveto{\pgfqpoint{1.171268in}{2.942145in}}{\pgfqpoint{1.160669in}{2.937754in}}{\pgfqpoint{1.152855in}{2.929941in}}%
\pgfpathcurveto{\pgfqpoint{1.145042in}{2.922127in}}{\pgfqpoint{1.140652in}{2.911528in}}{\pgfqpoint{1.140652in}{2.900478in}}%
\pgfpathcurveto{\pgfqpoint{1.140652in}{2.889428in}}{\pgfqpoint{1.145042in}{2.878829in}}{\pgfqpoint{1.152855in}{2.871015in}}%
\pgfpathcurveto{\pgfqpoint{1.160669in}{2.863202in}}{\pgfqpoint{1.171268in}{2.858811in}}{\pgfqpoint{1.182318in}{2.858811in}}%
\pgfpathclose%
\pgfusepath{stroke,fill}%
\end{pgfscope}%
\begin{pgfscope}%
\pgfpathrectangle{\pgfqpoint{0.481978in}{0.331635in}}{\pgfqpoint{4.960000in}{3.696000in}}%
\pgfusepath{clip}%
\pgfsetbuttcap%
\pgfsetroundjoin%
\definecolor{currentfill}{rgb}{1.000000,0.705882,0.509804}%
\pgfsetfillcolor{currentfill}%
\pgfsetlinewidth{0.481800pt}%
\definecolor{currentstroke}{rgb}{1.000000,1.000000,1.000000}%
\pgfsetstrokecolor{currentstroke}%
\pgfsetdash{}{0pt}%
\pgfpathmoveto{\pgfqpoint{1.453190in}{1.718586in}}%
\pgfpathcurveto{\pgfqpoint{1.464240in}{1.718586in}}{\pgfqpoint{1.474839in}{1.722976in}}{\pgfqpoint{1.482653in}{1.730790in}}%
\pgfpathcurveto{\pgfqpoint{1.490466in}{1.738603in}}{\pgfqpoint{1.494856in}{1.749202in}}{\pgfqpoint{1.494856in}{1.760252in}}%
\pgfpathcurveto{\pgfqpoint{1.494856in}{1.771303in}}{\pgfqpoint{1.490466in}{1.781902in}}{\pgfqpoint{1.482653in}{1.789715in}}%
\pgfpathcurveto{\pgfqpoint{1.474839in}{1.797529in}}{\pgfqpoint{1.464240in}{1.801919in}}{\pgfqpoint{1.453190in}{1.801919in}}%
\pgfpathcurveto{\pgfqpoint{1.442140in}{1.801919in}}{\pgfqpoint{1.431541in}{1.797529in}}{\pgfqpoint{1.423727in}{1.789715in}}%
\pgfpathcurveto{\pgfqpoint{1.415913in}{1.781902in}}{\pgfqpoint{1.411523in}{1.771303in}}{\pgfqpoint{1.411523in}{1.760252in}}%
\pgfpathcurveto{\pgfqpoint{1.411523in}{1.749202in}}{\pgfqpoint{1.415913in}{1.738603in}}{\pgfqpoint{1.423727in}{1.730790in}}%
\pgfpathcurveto{\pgfqpoint{1.431541in}{1.722976in}}{\pgfqpoint{1.442140in}{1.718586in}}{\pgfqpoint{1.453190in}{1.718586in}}%
\pgfpathclose%
\pgfusepath{stroke,fill}%
\end{pgfscope}%
\begin{pgfscope}%
\pgfpathrectangle{\pgfqpoint{0.481978in}{0.331635in}}{\pgfqpoint{4.960000in}{3.696000in}}%
\pgfusepath{clip}%
\pgfsetbuttcap%
\pgfsetroundjoin%
\definecolor{currentfill}{rgb}{1.000000,0.705882,0.509804}%
\pgfsetfillcolor{currentfill}%
\pgfsetlinewidth{0.481800pt}%
\definecolor{currentstroke}{rgb}{1.000000,1.000000,1.000000}%
\pgfsetstrokecolor{currentstroke}%
\pgfsetdash{}{0pt}%
\pgfpathmoveto{\pgfqpoint{1.815578in}{2.574505in}}%
\pgfpathcurveto{\pgfqpoint{1.826628in}{2.574505in}}{\pgfqpoint{1.837228in}{2.578895in}}{\pgfqpoint{1.845041in}{2.586708in}}%
\pgfpathcurveto{\pgfqpoint{1.852855in}{2.594522in}}{\pgfqpoint{1.857245in}{2.605121in}}{\pgfqpoint{1.857245in}{2.616171in}}%
\pgfpathcurveto{\pgfqpoint{1.857245in}{2.627221in}}{\pgfqpoint{1.852855in}{2.637820in}}{\pgfqpoint{1.845041in}{2.645634in}}%
\pgfpathcurveto{\pgfqpoint{1.837228in}{2.653448in}}{\pgfqpoint{1.826628in}{2.657838in}}{\pgfqpoint{1.815578in}{2.657838in}}%
\pgfpathcurveto{\pgfqpoint{1.804528in}{2.657838in}}{\pgfqpoint{1.793929in}{2.653448in}}{\pgfqpoint{1.786116in}{2.645634in}}%
\pgfpathcurveto{\pgfqpoint{1.778302in}{2.637820in}}{\pgfqpoint{1.773912in}{2.627221in}}{\pgfqpoint{1.773912in}{2.616171in}}%
\pgfpathcurveto{\pgfqpoint{1.773912in}{2.605121in}}{\pgfqpoint{1.778302in}{2.594522in}}{\pgfqpoint{1.786116in}{2.586708in}}%
\pgfpathcurveto{\pgfqpoint{1.793929in}{2.578895in}}{\pgfqpoint{1.804528in}{2.574505in}}{\pgfqpoint{1.815578in}{2.574505in}}%
\pgfpathclose%
\pgfusepath{stroke,fill}%
\end{pgfscope}%
\begin{pgfscope}%
\pgfpathrectangle{\pgfqpoint{0.481978in}{0.331635in}}{\pgfqpoint{4.960000in}{3.696000in}}%
\pgfusepath{clip}%
\pgfsetbuttcap%
\pgfsetroundjoin%
\definecolor{currentfill}{rgb}{1.000000,0.705882,0.509804}%
\pgfsetfillcolor{currentfill}%
\pgfsetlinewidth{0.481800pt}%
\definecolor{currentstroke}{rgb}{1.000000,1.000000,1.000000}%
\pgfsetstrokecolor{currentstroke}%
\pgfsetdash{}{0pt}%
\pgfpathmoveto{\pgfqpoint{3.041251in}{2.521304in}}%
\pgfpathcurveto{\pgfqpoint{3.052301in}{2.521304in}}{\pgfqpoint{3.062900in}{2.525694in}}{\pgfqpoint{3.070714in}{2.533507in}}%
\pgfpathcurveto{\pgfqpoint{3.078527in}{2.541321in}}{\pgfqpoint{3.082917in}{2.551920in}}{\pgfqpoint{3.082917in}{2.562970in}}%
\pgfpathcurveto{\pgfqpoint{3.082917in}{2.574020in}}{\pgfqpoint{3.078527in}{2.584619in}}{\pgfqpoint{3.070714in}{2.592433in}}%
\pgfpathcurveto{\pgfqpoint{3.062900in}{2.600247in}}{\pgfqpoint{3.052301in}{2.604637in}}{\pgfqpoint{3.041251in}{2.604637in}}%
\pgfpathcurveto{\pgfqpoint{3.030201in}{2.604637in}}{\pgfqpoint{3.019602in}{2.600247in}}{\pgfqpoint{3.011788in}{2.592433in}}%
\pgfpathcurveto{\pgfqpoint{3.003974in}{2.584619in}}{\pgfqpoint{2.999584in}{2.574020in}}{\pgfqpoint{2.999584in}{2.562970in}}%
\pgfpathcurveto{\pgfqpoint{2.999584in}{2.551920in}}{\pgfqpoint{3.003974in}{2.541321in}}{\pgfqpoint{3.011788in}{2.533507in}}%
\pgfpathcurveto{\pgfqpoint{3.019602in}{2.525694in}}{\pgfqpoint{3.030201in}{2.521304in}}{\pgfqpoint{3.041251in}{2.521304in}}%
\pgfpathclose%
\pgfusepath{stroke,fill}%
\end{pgfscope}%
\begin{pgfscope}%
\pgfpathrectangle{\pgfqpoint{0.481978in}{0.331635in}}{\pgfqpoint{4.960000in}{3.696000in}}%
\pgfusepath{clip}%
\pgfsetbuttcap%
\pgfsetroundjoin%
\definecolor{currentfill}{rgb}{1.000000,0.705882,0.509804}%
\pgfsetfillcolor{currentfill}%
\pgfsetlinewidth{0.481800pt}%
\definecolor{currentstroke}{rgb}{1.000000,1.000000,1.000000}%
\pgfsetstrokecolor{currentstroke}%
\pgfsetdash{}{0pt}%
\pgfpathmoveto{\pgfqpoint{2.709888in}{2.322828in}}%
\pgfpathcurveto{\pgfqpoint{2.720938in}{2.322828in}}{\pgfqpoint{2.731537in}{2.327219in}}{\pgfqpoint{2.739351in}{2.335032in}}%
\pgfpathcurveto{\pgfqpoint{2.747165in}{2.342846in}}{\pgfqpoint{2.751555in}{2.353445in}}{\pgfqpoint{2.751555in}{2.364495in}}%
\pgfpathcurveto{\pgfqpoint{2.751555in}{2.375545in}}{\pgfqpoint{2.747165in}{2.386144in}}{\pgfqpoint{2.739351in}{2.393958in}}%
\pgfpathcurveto{\pgfqpoint{2.731537in}{2.401772in}}{\pgfqpoint{2.720938in}{2.406162in}}{\pgfqpoint{2.709888in}{2.406162in}}%
\pgfpathcurveto{\pgfqpoint{2.698838in}{2.406162in}}{\pgfqpoint{2.688239in}{2.401772in}}{\pgfqpoint{2.680425in}{2.393958in}}%
\pgfpathcurveto{\pgfqpoint{2.672612in}{2.386144in}}{\pgfqpoint{2.668222in}{2.375545in}}{\pgfqpoint{2.668222in}{2.364495in}}%
\pgfpathcurveto{\pgfqpoint{2.668222in}{2.353445in}}{\pgfqpoint{2.672612in}{2.342846in}}{\pgfqpoint{2.680425in}{2.335032in}}%
\pgfpathcurveto{\pgfqpoint{2.688239in}{2.327219in}}{\pgfqpoint{2.698838in}{2.322828in}}{\pgfqpoint{2.709888in}{2.322828in}}%
\pgfpathclose%
\pgfusepath{stroke,fill}%
\end{pgfscope}%
\begin{pgfscope}%
\pgfpathrectangle{\pgfqpoint{0.481978in}{0.331635in}}{\pgfqpoint{4.960000in}{3.696000in}}%
\pgfusepath{clip}%
\pgfsetbuttcap%
\pgfsetroundjoin%
\definecolor{currentfill}{rgb}{1.000000,0.705882,0.509804}%
\pgfsetfillcolor{currentfill}%
\pgfsetlinewidth{0.481800pt}%
\definecolor{currentstroke}{rgb}{1.000000,1.000000,1.000000}%
\pgfsetstrokecolor{currentstroke}%
\pgfsetdash{}{0pt}%
\pgfpathmoveto{\pgfqpoint{2.262261in}{2.547118in}}%
\pgfpathcurveto{\pgfqpoint{2.273312in}{2.547118in}}{\pgfqpoint{2.283911in}{2.551508in}}{\pgfqpoint{2.291724in}{2.559322in}}%
\pgfpathcurveto{\pgfqpoint{2.299538in}{2.567135in}}{\pgfqpoint{2.303928in}{2.577734in}}{\pgfqpoint{2.303928in}{2.588785in}}%
\pgfpathcurveto{\pgfqpoint{2.303928in}{2.599835in}}{\pgfqpoint{2.299538in}{2.610434in}}{\pgfqpoint{2.291724in}{2.618247in}}%
\pgfpathcurveto{\pgfqpoint{2.283911in}{2.626061in}}{\pgfqpoint{2.273312in}{2.630451in}}{\pgfqpoint{2.262261in}{2.630451in}}%
\pgfpathcurveto{\pgfqpoint{2.251211in}{2.630451in}}{\pgfqpoint{2.240612in}{2.626061in}}{\pgfqpoint{2.232799in}{2.618247in}}%
\pgfpathcurveto{\pgfqpoint{2.224985in}{2.610434in}}{\pgfqpoint{2.220595in}{2.599835in}}{\pgfqpoint{2.220595in}{2.588785in}}%
\pgfpathcurveto{\pgfqpoint{2.220595in}{2.577734in}}{\pgfqpoint{2.224985in}{2.567135in}}{\pgfqpoint{2.232799in}{2.559322in}}%
\pgfpathcurveto{\pgfqpoint{2.240612in}{2.551508in}}{\pgfqpoint{2.251211in}{2.547118in}}{\pgfqpoint{2.262261in}{2.547118in}}%
\pgfpathclose%
\pgfusepath{stroke,fill}%
\end{pgfscope}%
\begin{pgfscope}%
\pgfpathrectangle{\pgfqpoint{0.481978in}{0.331635in}}{\pgfqpoint{4.960000in}{3.696000in}}%
\pgfusepath{clip}%
\pgfsetbuttcap%
\pgfsetroundjoin%
\definecolor{currentfill}{rgb}{1.000000,0.705882,0.509804}%
\pgfsetfillcolor{currentfill}%
\pgfsetlinewidth{0.481800pt}%
\definecolor{currentstroke}{rgb}{1.000000,1.000000,1.000000}%
\pgfsetstrokecolor{currentstroke}%
\pgfsetdash{}{0pt}%
\pgfpathmoveto{\pgfqpoint{1.812312in}{2.147414in}}%
\pgfpathcurveto{\pgfqpoint{1.823363in}{2.147414in}}{\pgfqpoint{1.833962in}{2.151804in}}{\pgfqpoint{1.841775in}{2.159617in}}%
\pgfpathcurveto{\pgfqpoint{1.849589in}{2.167431in}}{\pgfqpoint{1.853979in}{2.178030in}}{\pgfqpoint{1.853979in}{2.189080in}}%
\pgfpathcurveto{\pgfqpoint{1.853979in}{2.200130in}}{\pgfqpoint{1.849589in}{2.210729in}}{\pgfqpoint{1.841775in}{2.218543in}}%
\pgfpathcurveto{\pgfqpoint{1.833962in}{2.226357in}}{\pgfqpoint{1.823363in}{2.230747in}}{\pgfqpoint{1.812312in}{2.230747in}}%
\pgfpathcurveto{\pgfqpoint{1.801262in}{2.230747in}}{\pgfqpoint{1.790663in}{2.226357in}}{\pgfqpoint{1.782850in}{2.218543in}}%
\pgfpathcurveto{\pgfqpoint{1.775036in}{2.210729in}}{\pgfqpoint{1.770646in}{2.200130in}}{\pgfqpoint{1.770646in}{2.189080in}}%
\pgfpathcurveto{\pgfqpoint{1.770646in}{2.178030in}}{\pgfqpoint{1.775036in}{2.167431in}}{\pgfqpoint{1.782850in}{2.159617in}}%
\pgfpathcurveto{\pgfqpoint{1.790663in}{2.151804in}}{\pgfqpoint{1.801262in}{2.147414in}}{\pgfqpoint{1.812312in}{2.147414in}}%
\pgfpathclose%
\pgfusepath{stroke,fill}%
\end{pgfscope}%
\begin{pgfscope}%
\pgfpathrectangle{\pgfqpoint{0.481978in}{0.331635in}}{\pgfqpoint{4.960000in}{3.696000in}}%
\pgfusepath{clip}%
\pgfsetbuttcap%
\pgfsetroundjoin%
\definecolor{currentfill}{rgb}{0.631373,0.788235,0.956863}%
\pgfsetfillcolor{currentfill}%
\pgfsetlinewidth{0.481800pt}%
\definecolor{currentstroke}{rgb}{1.000000,1.000000,1.000000}%
\pgfsetstrokecolor{currentstroke}%
\pgfsetdash{}{0pt}%
\pgfpathmoveto{\pgfqpoint{2.202225in}{0.986356in}}%
\pgfpathcurveto{\pgfqpoint{2.213275in}{0.986356in}}{\pgfqpoint{2.223874in}{0.990746in}}{\pgfqpoint{2.231688in}{0.998560in}}%
\pgfpathcurveto{\pgfqpoint{2.239502in}{1.006374in}}{\pgfqpoint{2.243892in}{1.016973in}}{\pgfqpoint{2.243892in}{1.028023in}}%
\pgfpathcurveto{\pgfqpoint{2.243892in}{1.039073in}}{\pgfqpoint{2.239502in}{1.049672in}}{\pgfqpoint{2.231688in}{1.057486in}}%
\pgfpathcurveto{\pgfqpoint{2.223874in}{1.065299in}}{\pgfqpoint{2.213275in}{1.069690in}}{\pgfqpoint{2.202225in}{1.069690in}}%
\pgfpathcurveto{\pgfqpoint{2.191175in}{1.069690in}}{\pgfqpoint{2.180576in}{1.065299in}}{\pgfqpoint{2.172762in}{1.057486in}}%
\pgfpathcurveto{\pgfqpoint{2.164949in}{1.049672in}}{\pgfqpoint{2.160559in}{1.039073in}}{\pgfqpoint{2.160559in}{1.028023in}}%
\pgfpathcurveto{\pgfqpoint{2.160559in}{1.016973in}}{\pgfqpoint{2.164949in}{1.006374in}}{\pgfqpoint{2.172762in}{0.998560in}}%
\pgfpathcurveto{\pgfqpoint{2.180576in}{0.990746in}}{\pgfqpoint{2.191175in}{0.986356in}}{\pgfqpoint{2.202225in}{0.986356in}}%
\pgfpathclose%
\pgfusepath{stroke,fill}%
\end{pgfscope}%
\begin{pgfscope}%
\pgfpathrectangle{\pgfqpoint{0.481978in}{0.331635in}}{\pgfqpoint{4.960000in}{3.696000in}}%
\pgfusepath{clip}%
\pgfsetbuttcap%
\pgfsetroundjoin%
\definecolor{currentfill}{rgb}{0.631373,0.788235,0.956863}%
\pgfsetfillcolor{currentfill}%
\pgfsetlinewidth{0.481800pt}%
\definecolor{currentstroke}{rgb}{1.000000,1.000000,1.000000}%
\pgfsetstrokecolor{currentstroke}%
\pgfsetdash{}{0pt}%
\pgfpathmoveto{\pgfqpoint{2.596676in}{2.170337in}}%
\pgfpathcurveto{\pgfqpoint{2.607726in}{2.170337in}}{\pgfqpoint{2.618325in}{2.174727in}}{\pgfqpoint{2.626139in}{2.182541in}}%
\pgfpathcurveto{\pgfqpoint{2.633953in}{2.190354in}}{\pgfqpoint{2.638343in}{2.200953in}}{\pgfqpoint{2.638343in}{2.212003in}}%
\pgfpathcurveto{\pgfqpoint{2.638343in}{2.223054in}}{\pgfqpoint{2.633953in}{2.233653in}}{\pgfqpoint{2.626139in}{2.241466in}}%
\pgfpathcurveto{\pgfqpoint{2.618325in}{2.249280in}}{\pgfqpoint{2.607726in}{2.253670in}}{\pgfqpoint{2.596676in}{2.253670in}}%
\pgfpathcurveto{\pgfqpoint{2.585626in}{2.253670in}}{\pgfqpoint{2.575027in}{2.249280in}}{\pgfqpoint{2.567214in}{2.241466in}}%
\pgfpathcurveto{\pgfqpoint{2.559400in}{2.233653in}}{\pgfqpoint{2.555010in}{2.223054in}}{\pgfqpoint{2.555010in}{2.212003in}}%
\pgfpathcurveto{\pgfqpoint{2.555010in}{2.200953in}}{\pgfqpoint{2.559400in}{2.190354in}}{\pgfqpoint{2.567214in}{2.182541in}}%
\pgfpathcurveto{\pgfqpoint{2.575027in}{2.174727in}}{\pgfqpoint{2.585626in}{2.170337in}}{\pgfqpoint{2.596676in}{2.170337in}}%
\pgfpathclose%
\pgfusepath{stroke,fill}%
\end{pgfscope}%
\begin{pgfscope}%
\pgfpathrectangle{\pgfqpoint{0.481978in}{0.331635in}}{\pgfqpoint{4.960000in}{3.696000in}}%
\pgfusepath{clip}%
\pgfsetbuttcap%
\pgfsetroundjoin%
\definecolor{currentfill}{rgb}{0.631373,0.788235,0.956863}%
\pgfsetfillcolor{currentfill}%
\pgfsetlinewidth{0.481800pt}%
\definecolor{currentstroke}{rgb}{1.000000,1.000000,1.000000}%
\pgfsetstrokecolor{currentstroke}%
\pgfsetdash{}{0pt}%
\pgfpathmoveto{\pgfqpoint{3.321086in}{1.680657in}}%
\pgfpathcurveto{\pgfqpoint{3.332136in}{1.680657in}}{\pgfqpoint{3.342735in}{1.685047in}}{\pgfqpoint{3.350549in}{1.692861in}}%
\pgfpathcurveto{\pgfqpoint{3.358363in}{1.700674in}}{\pgfqpoint{3.362753in}{1.711274in}}{\pgfqpoint{3.362753in}{1.722324in}}%
\pgfpathcurveto{\pgfqpoint{3.362753in}{1.733374in}}{\pgfqpoint{3.358363in}{1.743973in}}{\pgfqpoint{3.350549in}{1.751786in}}%
\pgfpathcurveto{\pgfqpoint{3.342735in}{1.759600in}}{\pgfqpoint{3.332136in}{1.763990in}}{\pgfqpoint{3.321086in}{1.763990in}}%
\pgfpathcurveto{\pgfqpoint{3.310036in}{1.763990in}}{\pgfqpoint{3.299437in}{1.759600in}}{\pgfqpoint{3.291623in}{1.751786in}}%
\pgfpathcurveto{\pgfqpoint{3.283810in}{1.743973in}}{\pgfqpoint{3.279419in}{1.733374in}}{\pgfqpoint{3.279419in}{1.722324in}}%
\pgfpathcurveto{\pgfqpoint{3.279419in}{1.711274in}}{\pgfqpoint{3.283810in}{1.700674in}}{\pgfqpoint{3.291623in}{1.692861in}}%
\pgfpathcurveto{\pgfqpoint{3.299437in}{1.685047in}}{\pgfqpoint{3.310036in}{1.680657in}}{\pgfqpoint{3.321086in}{1.680657in}}%
\pgfpathclose%
\pgfusepath{stroke,fill}%
\end{pgfscope}%
\begin{pgfscope}%
\pgfpathrectangle{\pgfqpoint{0.481978in}{0.331635in}}{\pgfqpoint{4.960000in}{3.696000in}}%
\pgfusepath{clip}%
\pgfsetbuttcap%
\pgfsetroundjoin%
\definecolor{currentfill}{rgb}{0.631373,0.788235,0.956863}%
\pgfsetfillcolor{currentfill}%
\pgfsetlinewidth{0.481800pt}%
\definecolor{currentstroke}{rgb}{1.000000,1.000000,1.000000}%
\pgfsetstrokecolor{currentstroke}%
\pgfsetdash{}{0pt}%
\pgfpathmoveto{\pgfqpoint{3.502753in}{0.711093in}}%
\pgfpathcurveto{\pgfqpoint{3.513803in}{0.711093in}}{\pgfqpoint{3.524403in}{0.715484in}}{\pgfqpoint{3.532216in}{0.723297in}}%
\pgfpathcurveto{\pgfqpoint{3.540030in}{0.731111in}}{\pgfqpoint{3.544420in}{0.741710in}}{\pgfqpoint{3.544420in}{0.752760in}}%
\pgfpathcurveto{\pgfqpoint{3.544420in}{0.763810in}}{\pgfqpoint{3.540030in}{0.774409in}}{\pgfqpoint{3.532216in}{0.782223in}}%
\pgfpathcurveto{\pgfqpoint{3.524403in}{0.790037in}}{\pgfqpoint{3.513803in}{0.794427in}}{\pgfqpoint{3.502753in}{0.794427in}}%
\pgfpathcurveto{\pgfqpoint{3.491703in}{0.794427in}}{\pgfqpoint{3.481104in}{0.790037in}}{\pgfqpoint{3.473291in}{0.782223in}}%
\pgfpathcurveto{\pgfqpoint{3.465477in}{0.774409in}}{\pgfqpoint{3.461087in}{0.763810in}}{\pgfqpoint{3.461087in}{0.752760in}}%
\pgfpathcurveto{\pgfqpoint{3.461087in}{0.741710in}}{\pgfqpoint{3.465477in}{0.731111in}}{\pgfqpoint{3.473291in}{0.723297in}}%
\pgfpathcurveto{\pgfqpoint{3.481104in}{0.715484in}}{\pgfqpoint{3.491703in}{0.711093in}}{\pgfqpoint{3.502753in}{0.711093in}}%
\pgfpathclose%
\pgfusepath{stroke,fill}%
\end{pgfscope}%
\begin{pgfscope}%
\pgfpathrectangle{\pgfqpoint{0.481978in}{0.331635in}}{\pgfqpoint{4.960000in}{3.696000in}}%
\pgfusepath{clip}%
\pgfsetbuttcap%
\pgfsetroundjoin%
\definecolor{currentfill}{rgb}{0.631373,0.788235,0.956863}%
\pgfsetfillcolor{currentfill}%
\pgfsetlinewidth{0.481800pt}%
\definecolor{currentstroke}{rgb}{1.000000,1.000000,1.000000}%
\pgfsetstrokecolor{currentstroke}%
\pgfsetdash{}{0pt}%
\pgfpathmoveto{\pgfqpoint{2.771539in}{1.152596in}}%
\pgfpathcurveto{\pgfqpoint{2.782589in}{1.152596in}}{\pgfqpoint{2.793188in}{1.156986in}}{\pgfqpoint{2.801001in}{1.164800in}}%
\pgfpathcurveto{\pgfqpoint{2.808815in}{1.172614in}}{\pgfqpoint{2.813205in}{1.183213in}}{\pgfqpoint{2.813205in}{1.194263in}}%
\pgfpathcurveto{\pgfqpoint{2.813205in}{1.205313in}}{\pgfqpoint{2.808815in}{1.215912in}}{\pgfqpoint{2.801001in}{1.223725in}}%
\pgfpathcurveto{\pgfqpoint{2.793188in}{1.231539in}}{\pgfqpoint{2.782589in}{1.235929in}}{\pgfqpoint{2.771539in}{1.235929in}}%
\pgfpathcurveto{\pgfqpoint{2.760489in}{1.235929in}}{\pgfqpoint{2.749890in}{1.231539in}}{\pgfqpoint{2.742076in}{1.223725in}}%
\pgfpathcurveto{\pgfqpoint{2.734262in}{1.215912in}}{\pgfqpoint{2.729872in}{1.205313in}}{\pgfqpoint{2.729872in}{1.194263in}}%
\pgfpathcurveto{\pgfqpoint{2.729872in}{1.183213in}}{\pgfqpoint{2.734262in}{1.172614in}}{\pgfqpoint{2.742076in}{1.164800in}}%
\pgfpathcurveto{\pgfqpoint{2.749890in}{1.156986in}}{\pgfqpoint{2.760489in}{1.152596in}}{\pgfqpoint{2.771539in}{1.152596in}}%
\pgfpathclose%
\pgfusepath{stroke,fill}%
\end{pgfscope}%
\begin{pgfscope}%
\pgfpathrectangle{\pgfqpoint{0.481978in}{0.331635in}}{\pgfqpoint{4.960000in}{3.696000in}}%
\pgfusepath{clip}%
\pgfsetbuttcap%
\pgfsetroundjoin%
\definecolor{currentfill}{rgb}{0.631373,0.788235,0.956863}%
\pgfsetfillcolor{currentfill}%
\pgfsetlinewidth{0.481800pt}%
\definecolor{currentstroke}{rgb}{1.000000,1.000000,1.000000}%
\pgfsetstrokecolor{currentstroke}%
\pgfsetdash{}{0pt}%
\pgfpathmoveto{\pgfqpoint{3.917552in}{1.129093in}}%
\pgfpathcurveto{\pgfqpoint{3.928602in}{1.129093in}}{\pgfqpoint{3.939201in}{1.133483in}}{\pgfqpoint{3.947015in}{1.141297in}}%
\pgfpathcurveto{\pgfqpoint{3.954828in}{1.149110in}}{\pgfqpoint{3.959218in}{1.159709in}}{\pgfqpoint{3.959218in}{1.170759in}}%
\pgfpathcurveto{\pgfqpoint{3.959218in}{1.181809in}}{\pgfqpoint{3.954828in}{1.192408in}}{\pgfqpoint{3.947015in}{1.200222in}}%
\pgfpathcurveto{\pgfqpoint{3.939201in}{1.208036in}}{\pgfqpoint{3.928602in}{1.212426in}}{\pgfqpoint{3.917552in}{1.212426in}}%
\pgfpathcurveto{\pgfqpoint{3.906502in}{1.212426in}}{\pgfqpoint{3.895903in}{1.208036in}}{\pgfqpoint{3.888089in}{1.200222in}}%
\pgfpathcurveto{\pgfqpoint{3.880275in}{1.192408in}}{\pgfqpoint{3.875885in}{1.181809in}}{\pgfqpoint{3.875885in}{1.170759in}}%
\pgfpathcurveto{\pgfqpoint{3.875885in}{1.159709in}}{\pgfqpoint{3.880275in}{1.149110in}}{\pgfqpoint{3.888089in}{1.141297in}}%
\pgfpathcurveto{\pgfqpoint{3.895903in}{1.133483in}}{\pgfqpoint{3.906502in}{1.129093in}}{\pgfqpoint{3.917552in}{1.129093in}}%
\pgfpathclose%
\pgfusepath{stroke,fill}%
\end{pgfscope}%
\begin{pgfscope}%
\pgfpathrectangle{\pgfqpoint{0.481978in}{0.331635in}}{\pgfqpoint{4.960000in}{3.696000in}}%
\pgfusepath{clip}%
\pgfsetbuttcap%
\pgfsetroundjoin%
\definecolor{currentfill}{rgb}{0.631373,0.788235,0.956863}%
\pgfsetfillcolor{currentfill}%
\pgfsetlinewidth{0.481800pt}%
\definecolor{currentstroke}{rgb}{1.000000,1.000000,1.000000}%
\pgfsetstrokecolor{currentstroke}%
\pgfsetdash{}{0pt}%
\pgfpathmoveto{\pgfqpoint{3.801010in}{2.432484in}}%
\pgfpathcurveto{\pgfqpoint{3.812061in}{2.432484in}}{\pgfqpoint{3.822660in}{2.436874in}}{\pgfqpoint{3.830473in}{2.444688in}}%
\pgfpathcurveto{\pgfqpoint{3.838287in}{2.452502in}}{\pgfqpoint{3.842677in}{2.463101in}}{\pgfqpoint{3.842677in}{2.474151in}}%
\pgfpathcurveto{\pgfqpoint{3.842677in}{2.485201in}}{\pgfqpoint{3.838287in}{2.495800in}}{\pgfqpoint{3.830473in}{2.503614in}}%
\pgfpathcurveto{\pgfqpoint{3.822660in}{2.511427in}}{\pgfqpoint{3.812061in}{2.515817in}}{\pgfqpoint{3.801010in}{2.515817in}}%
\pgfpathcurveto{\pgfqpoint{3.789960in}{2.515817in}}{\pgfqpoint{3.779361in}{2.511427in}}{\pgfqpoint{3.771548in}{2.503614in}}%
\pgfpathcurveto{\pgfqpoint{3.763734in}{2.495800in}}{\pgfqpoint{3.759344in}{2.485201in}}{\pgfqpoint{3.759344in}{2.474151in}}%
\pgfpathcurveto{\pgfqpoint{3.759344in}{2.463101in}}{\pgfqpoint{3.763734in}{2.452502in}}{\pgfqpoint{3.771548in}{2.444688in}}%
\pgfpathcurveto{\pgfqpoint{3.779361in}{2.436874in}}{\pgfqpoint{3.789960in}{2.432484in}}{\pgfqpoint{3.801010in}{2.432484in}}%
\pgfpathclose%
\pgfusepath{stroke,fill}%
\end{pgfscope}%
\begin{pgfscope}%
\pgfpathrectangle{\pgfqpoint{0.481978in}{0.331635in}}{\pgfqpoint{4.960000in}{3.696000in}}%
\pgfusepath{clip}%
\pgfsetbuttcap%
\pgfsetroundjoin%
\definecolor{currentfill}{rgb}{0.631373,0.788235,0.956863}%
\pgfsetfillcolor{currentfill}%
\pgfsetlinewidth{0.481800pt}%
\definecolor{currentstroke}{rgb}{1.000000,1.000000,1.000000}%
\pgfsetstrokecolor{currentstroke}%
\pgfsetdash{}{0pt}%
\pgfpathmoveto{\pgfqpoint{3.750240in}{2.652287in}}%
\pgfpathcurveto{\pgfqpoint{3.761290in}{2.652287in}}{\pgfqpoint{3.771889in}{2.656677in}}{\pgfqpoint{3.779703in}{2.664491in}}%
\pgfpathcurveto{\pgfqpoint{3.787516in}{2.672305in}}{\pgfqpoint{3.791907in}{2.682904in}}{\pgfqpoint{3.791907in}{2.693954in}}%
\pgfpathcurveto{\pgfqpoint{3.791907in}{2.705004in}}{\pgfqpoint{3.787516in}{2.715603in}}{\pgfqpoint{3.779703in}{2.723417in}}%
\pgfpathcurveto{\pgfqpoint{3.771889in}{2.731230in}}{\pgfqpoint{3.761290in}{2.735621in}}{\pgfqpoint{3.750240in}{2.735621in}}%
\pgfpathcurveto{\pgfqpoint{3.739190in}{2.735621in}}{\pgfqpoint{3.728591in}{2.731230in}}{\pgfqpoint{3.720777in}{2.723417in}}%
\pgfpathcurveto{\pgfqpoint{3.712964in}{2.715603in}}{\pgfqpoint{3.708573in}{2.705004in}}{\pgfqpoint{3.708573in}{2.693954in}}%
\pgfpathcurveto{\pgfqpoint{3.708573in}{2.682904in}}{\pgfqpoint{3.712964in}{2.672305in}}{\pgfqpoint{3.720777in}{2.664491in}}%
\pgfpathcurveto{\pgfqpoint{3.728591in}{2.656677in}}{\pgfqpoint{3.739190in}{2.652287in}}{\pgfqpoint{3.750240in}{2.652287in}}%
\pgfpathclose%
\pgfusepath{stroke,fill}%
\end{pgfscope}%
\begin{pgfscope}%
\pgfpathrectangle{\pgfqpoint{0.481978in}{0.331635in}}{\pgfqpoint{4.960000in}{3.696000in}}%
\pgfusepath{clip}%
\pgfsetbuttcap%
\pgfsetroundjoin%
\definecolor{currentfill}{rgb}{0.631373,0.788235,0.956863}%
\pgfsetfillcolor{currentfill}%
\pgfsetlinewidth{0.481800pt}%
\definecolor{currentstroke}{rgb}{1.000000,1.000000,1.000000}%
\pgfsetstrokecolor{currentstroke}%
\pgfsetdash{}{0pt}%
\pgfpathmoveto{\pgfqpoint{2.725316in}{0.718863in}}%
\pgfpathcurveto{\pgfqpoint{2.736366in}{0.718863in}}{\pgfqpoint{2.746965in}{0.723254in}}{\pgfqpoint{2.754779in}{0.731067in}}%
\pgfpathcurveto{\pgfqpoint{2.762592in}{0.738881in}}{\pgfqpoint{2.766982in}{0.749480in}}{\pgfqpoint{2.766982in}{0.760530in}}%
\pgfpathcurveto{\pgfqpoint{2.766982in}{0.771580in}}{\pgfqpoint{2.762592in}{0.782179in}}{\pgfqpoint{2.754779in}{0.789993in}}%
\pgfpathcurveto{\pgfqpoint{2.746965in}{0.797806in}}{\pgfqpoint{2.736366in}{0.802197in}}{\pgfqpoint{2.725316in}{0.802197in}}%
\pgfpathcurveto{\pgfqpoint{2.714266in}{0.802197in}}{\pgfqpoint{2.703667in}{0.797806in}}{\pgfqpoint{2.695853in}{0.789993in}}%
\pgfpathcurveto{\pgfqpoint{2.688039in}{0.782179in}}{\pgfqpoint{2.683649in}{0.771580in}}{\pgfqpoint{2.683649in}{0.760530in}}%
\pgfpathcurveto{\pgfqpoint{2.683649in}{0.749480in}}{\pgfqpoint{2.688039in}{0.738881in}}{\pgfqpoint{2.695853in}{0.731067in}}%
\pgfpathcurveto{\pgfqpoint{2.703667in}{0.723254in}}{\pgfqpoint{2.714266in}{0.718863in}}{\pgfqpoint{2.725316in}{0.718863in}}%
\pgfpathclose%
\pgfusepath{stroke,fill}%
\end{pgfscope}%
\begin{pgfscope}%
\pgfpathrectangle{\pgfqpoint{0.481978in}{0.331635in}}{\pgfqpoint{4.960000in}{3.696000in}}%
\pgfusepath{clip}%
\pgfsetbuttcap%
\pgfsetroundjoin%
\definecolor{currentfill}{rgb}{0.631373,0.788235,0.956863}%
\pgfsetfillcolor{currentfill}%
\pgfsetlinewidth{0.481800pt}%
\definecolor{currentstroke}{rgb}{1.000000,1.000000,1.000000}%
\pgfsetstrokecolor{currentstroke}%
\pgfsetdash{}{0pt}%
\pgfpathmoveto{\pgfqpoint{2.695779in}{3.052038in}}%
\pgfpathcurveto{\pgfqpoint{2.706829in}{3.052038in}}{\pgfqpoint{2.717428in}{3.056428in}}{\pgfqpoint{2.725242in}{3.064242in}}%
\pgfpathcurveto{\pgfqpoint{2.733055in}{3.072055in}}{\pgfqpoint{2.737445in}{3.082654in}}{\pgfqpoint{2.737445in}{3.093704in}}%
\pgfpathcurveto{\pgfqpoint{2.737445in}{3.104754in}}{\pgfqpoint{2.733055in}{3.115353in}}{\pgfqpoint{2.725242in}{3.123167in}}%
\pgfpathcurveto{\pgfqpoint{2.717428in}{3.130981in}}{\pgfqpoint{2.706829in}{3.135371in}}{\pgfqpoint{2.695779in}{3.135371in}}%
\pgfpathcurveto{\pgfqpoint{2.684729in}{3.135371in}}{\pgfqpoint{2.674130in}{3.130981in}}{\pgfqpoint{2.666316in}{3.123167in}}%
\pgfpathcurveto{\pgfqpoint{2.658502in}{3.115353in}}{\pgfqpoint{2.654112in}{3.104754in}}{\pgfqpoint{2.654112in}{3.093704in}}%
\pgfpathcurveto{\pgfqpoint{2.654112in}{3.082654in}}{\pgfqpoint{2.658502in}{3.072055in}}{\pgfqpoint{2.666316in}{3.064242in}}%
\pgfpathcurveto{\pgfqpoint{2.674130in}{3.056428in}}{\pgfqpoint{2.684729in}{3.052038in}}{\pgfqpoint{2.695779in}{3.052038in}}%
\pgfpathclose%
\pgfusepath{stroke,fill}%
\end{pgfscope}%
\begin{pgfscope}%
\pgfpathrectangle{\pgfqpoint{0.481978in}{0.331635in}}{\pgfqpoint{4.960000in}{3.696000in}}%
\pgfusepath{clip}%
\pgfsetbuttcap%
\pgfsetroundjoin%
\definecolor{currentfill}{rgb}{0.631373,0.788235,0.956863}%
\pgfsetfillcolor{currentfill}%
\pgfsetlinewidth{0.481800pt}%
\definecolor{currentstroke}{rgb}{1.000000,1.000000,1.000000}%
\pgfsetstrokecolor{currentstroke}%
\pgfsetdash{}{0pt}%
\pgfpathmoveto{\pgfqpoint{4.183138in}{1.303316in}}%
\pgfpathcurveto{\pgfqpoint{4.194188in}{1.303316in}}{\pgfqpoint{4.204788in}{1.307706in}}{\pgfqpoint{4.212601in}{1.315520in}}%
\pgfpathcurveto{\pgfqpoint{4.220415in}{1.323333in}}{\pgfqpoint{4.224805in}{1.333933in}}{\pgfqpoint{4.224805in}{1.344983in}}%
\pgfpathcurveto{\pgfqpoint{4.224805in}{1.356033in}}{\pgfqpoint{4.220415in}{1.366632in}}{\pgfqpoint{4.212601in}{1.374445in}}%
\pgfpathcurveto{\pgfqpoint{4.204788in}{1.382259in}}{\pgfqpoint{4.194188in}{1.386649in}}{\pgfqpoint{4.183138in}{1.386649in}}%
\pgfpathcurveto{\pgfqpoint{4.172088in}{1.386649in}}{\pgfqpoint{4.161489in}{1.382259in}}{\pgfqpoint{4.153676in}{1.374445in}}%
\pgfpathcurveto{\pgfqpoint{4.145862in}{1.366632in}}{\pgfqpoint{4.141472in}{1.356033in}}{\pgfqpoint{4.141472in}{1.344983in}}%
\pgfpathcurveto{\pgfqpoint{4.141472in}{1.333933in}}{\pgfqpoint{4.145862in}{1.323333in}}{\pgfqpoint{4.153676in}{1.315520in}}%
\pgfpathcurveto{\pgfqpoint{4.161489in}{1.307706in}}{\pgfqpoint{4.172088in}{1.303316in}}{\pgfqpoint{4.183138in}{1.303316in}}%
\pgfpathclose%
\pgfusepath{stroke,fill}%
\end{pgfscope}%
\begin{pgfscope}%
\pgfpathrectangle{\pgfqpoint{0.481978in}{0.331635in}}{\pgfqpoint{4.960000in}{3.696000in}}%
\pgfusepath{clip}%
\pgfsetbuttcap%
\pgfsetroundjoin%
\definecolor{currentfill}{rgb}{0.631373,0.788235,0.956863}%
\pgfsetfillcolor{currentfill}%
\pgfsetlinewidth{0.481800pt}%
\definecolor{currentstroke}{rgb}{1.000000,1.000000,1.000000}%
\pgfsetstrokecolor{currentstroke}%
\pgfsetdash{}{0pt}%
\pgfpathmoveto{\pgfqpoint{4.833897in}{3.293210in}}%
\pgfpathcurveto{\pgfqpoint{4.844947in}{3.293210in}}{\pgfqpoint{4.855546in}{3.297600in}}{\pgfqpoint{4.863360in}{3.305413in}}%
\pgfpathcurveto{\pgfqpoint{4.871174in}{3.313227in}}{\pgfqpoint{4.875564in}{3.323826in}}{\pgfqpoint{4.875564in}{3.334876in}}%
\pgfpathcurveto{\pgfqpoint{4.875564in}{3.345926in}}{\pgfqpoint{4.871174in}{3.356525in}}{\pgfqpoint{4.863360in}{3.364339in}}%
\pgfpathcurveto{\pgfqpoint{4.855546in}{3.372153in}}{\pgfqpoint{4.844947in}{3.376543in}}{\pgfqpoint{4.833897in}{3.376543in}}%
\pgfpathcurveto{\pgfqpoint{4.822847in}{3.376543in}}{\pgfqpoint{4.812248in}{3.372153in}}{\pgfqpoint{4.804434in}{3.364339in}}%
\pgfpathcurveto{\pgfqpoint{4.796621in}{3.356525in}}{\pgfqpoint{4.792231in}{3.345926in}}{\pgfqpoint{4.792231in}{3.334876in}}%
\pgfpathcurveto{\pgfqpoint{4.792231in}{3.323826in}}{\pgfqpoint{4.796621in}{3.313227in}}{\pgfqpoint{4.804434in}{3.305413in}}%
\pgfpathcurveto{\pgfqpoint{4.812248in}{3.297600in}}{\pgfqpoint{4.822847in}{3.293210in}}{\pgfqpoint{4.833897in}{3.293210in}}%
\pgfpathclose%
\pgfusepath{stroke,fill}%
\end{pgfscope}%
\begin{pgfscope}%
\pgfpathrectangle{\pgfqpoint{0.481978in}{0.331635in}}{\pgfqpoint{4.960000in}{3.696000in}}%
\pgfusepath{clip}%
\pgfsetbuttcap%
\pgfsetroundjoin%
\definecolor{currentfill}{rgb}{0.631373,0.788235,0.956863}%
\pgfsetfillcolor{currentfill}%
\pgfsetlinewidth{0.481800pt}%
\definecolor{currentstroke}{rgb}{1.000000,1.000000,1.000000}%
\pgfsetstrokecolor{currentstroke}%
\pgfsetdash{}{0pt}%
\pgfpathmoveto{\pgfqpoint{3.853456in}{1.574124in}}%
\pgfpathcurveto{\pgfqpoint{3.864506in}{1.574124in}}{\pgfqpoint{3.875105in}{1.578514in}}{\pgfqpoint{3.882919in}{1.586328in}}%
\pgfpathcurveto{\pgfqpoint{3.890732in}{1.594142in}}{\pgfqpoint{3.895123in}{1.604741in}}{\pgfqpoint{3.895123in}{1.615791in}}%
\pgfpathcurveto{\pgfqpoint{3.895123in}{1.626841in}}{\pgfqpoint{3.890732in}{1.637440in}}{\pgfqpoint{3.882919in}{1.645253in}}%
\pgfpathcurveto{\pgfqpoint{3.875105in}{1.653067in}}{\pgfqpoint{3.864506in}{1.657457in}}{\pgfqpoint{3.853456in}{1.657457in}}%
\pgfpathcurveto{\pgfqpoint{3.842406in}{1.657457in}}{\pgfqpoint{3.831807in}{1.653067in}}{\pgfqpoint{3.823993in}{1.645253in}}%
\pgfpathcurveto{\pgfqpoint{3.816180in}{1.637440in}}{\pgfqpoint{3.811789in}{1.626841in}}{\pgfqpoint{3.811789in}{1.615791in}}%
\pgfpathcurveto{\pgfqpoint{3.811789in}{1.604741in}}{\pgfqpoint{3.816180in}{1.594142in}}{\pgfqpoint{3.823993in}{1.586328in}}%
\pgfpathcurveto{\pgfqpoint{3.831807in}{1.578514in}}{\pgfqpoint{3.842406in}{1.574124in}}{\pgfqpoint{3.853456in}{1.574124in}}%
\pgfpathclose%
\pgfusepath{stroke,fill}%
\end{pgfscope}%
\begin{pgfscope}%
\pgfpathrectangle{\pgfqpoint{0.481978in}{0.331635in}}{\pgfqpoint{4.960000in}{3.696000in}}%
\pgfusepath{clip}%
\pgfsetbuttcap%
\pgfsetroundjoin%
\definecolor{currentfill}{rgb}{0.631373,0.788235,0.956863}%
\pgfsetfillcolor{currentfill}%
\pgfsetlinewidth{0.481800pt}%
\definecolor{currentstroke}{rgb}{1.000000,1.000000,1.000000}%
\pgfsetstrokecolor{currentstroke}%
\pgfsetdash{}{0pt}%
\pgfpathmoveto{\pgfqpoint{3.890909in}{2.100145in}}%
\pgfpathcurveto{\pgfqpoint{3.901959in}{2.100145in}}{\pgfqpoint{3.912558in}{2.104536in}}{\pgfqpoint{3.920372in}{2.112349in}}%
\pgfpathcurveto{\pgfqpoint{3.928186in}{2.120163in}}{\pgfqpoint{3.932576in}{2.130762in}}{\pgfqpoint{3.932576in}{2.141812in}}%
\pgfpathcurveto{\pgfqpoint{3.932576in}{2.152862in}}{\pgfqpoint{3.928186in}{2.163461in}}{\pgfqpoint{3.920372in}{2.171275in}}%
\pgfpathcurveto{\pgfqpoint{3.912558in}{2.179088in}}{\pgfqpoint{3.901959in}{2.183479in}}{\pgfqpoint{3.890909in}{2.183479in}}%
\pgfpathcurveto{\pgfqpoint{3.879859in}{2.183479in}}{\pgfqpoint{3.869260in}{2.179088in}}{\pgfqpoint{3.861446in}{2.171275in}}%
\pgfpathcurveto{\pgfqpoint{3.853633in}{2.163461in}}{\pgfqpoint{3.849243in}{2.152862in}}{\pgfqpoint{3.849243in}{2.141812in}}%
\pgfpathcurveto{\pgfqpoint{3.849243in}{2.130762in}}{\pgfqpoint{3.853633in}{2.120163in}}{\pgfqpoint{3.861446in}{2.112349in}}%
\pgfpathcurveto{\pgfqpoint{3.869260in}{2.104536in}}{\pgfqpoint{3.879859in}{2.100145in}}{\pgfqpoint{3.890909in}{2.100145in}}%
\pgfpathclose%
\pgfusepath{stroke,fill}%
\end{pgfscope}%
\begin{pgfscope}%
\pgfpathrectangle{\pgfqpoint{0.481978in}{0.331635in}}{\pgfqpoint{4.960000in}{3.696000in}}%
\pgfusepath{clip}%
\pgfsetbuttcap%
\pgfsetroundjoin%
\definecolor{currentfill}{rgb}{0.631373,0.788235,0.956863}%
\pgfsetfillcolor{currentfill}%
\pgfsetlinewidth{0.481800pt}%
\definecolor{currentstroke}{rgb}{1.000000,1.000000,1.000000}%
\pgfsetstrokecolor{currentstroke}%
\pgfsetdash{}{0pt}%
\pgfpathmoveto{\pgfqpoint{4.121377in}{2.164028in}}%
\pgfpathcurveto{\pgfqpoint{4.132427in}{2.164028in}}{\pgfqpoint{4.143026in}{2.168419in}}{\pgfqpoint{4.150839in}{2.176232in}}%
\pgfpathcurveto{\pgfqpoint{4.158653in}{2.184046in}}{\pgfqpoint{4.163043in}{2.194645in}}{\pgfqpoint{4.163043in}{2.205695in}}%
\pgfpathcurveto{\pgfqpoint{4.163043in}{2.216745in}}{\pgfqpoint{4.158653in}{2.227344in}}{\pgfqpoint{4.150839in}{2.235158in}}%
\pgfpathcurveto{\pgfqpoint{4.143026in}{2.242971in}}{\pgfqpoint{4.132427in}{2.247362in}}{\pgfqpoint{4.121377in}{2.247362in}}%
\pgfpathcurveto{\pgfqpoint{4.110326in}{2.247362in}}{\pgfqpoint{4.099727in}{2.242971in}}{\pgfqpoint{4.091914in}{2.235158in}}%
\pgfpathcurveto{\pgfqpoint{4.084100in}{2.227344in}}{\pgfqpoint{4.079710in}{2.216745in}}{\pgfqpoint{4.079710in}{2.205695in}}%
\pgfpathcurveto{\pgfqpoint{4.079710in}{2.194645in}}{\pgfqpoint{4.084100in}{2.184046in}}{\pgfqpoint{4.091914in}{2.176232in}}%
\pgfpathcurveto{\pgfqpoint{4.099727in}{2.168419in}}{\pgfqpoint{4.110326in}{2.164028in}}{\pgfqpoint{4.121377in}{2.164028in}}%
\pgfpathclose%
\pgfusepath{stroke,fill}%
\end{pgfscope}%
\begin{pgfscope}%
\pgfpathrectangle{\pgfqpoint{0.481978in}{0.331635in}}{\pgfqpoint{4.960000in}{3.696000in}}%
\pgfusepath{clip}%
\pgfsetbuttcap%
\pgfsetroundjoin%
\definecolor{currentfill}{rgb}{0.631373,0.788235,0.956863}%
\pgfsetfillcolor{currentfill}%
\pgfsetlinewidth{0.481800pt}%
\definecolor{currentstroke}{rgb}{1.000000,1.000000,1.000000}%
\pgfsetstrokecolor{currentstroke}%
\pgfsetdash{}{0pt}%
\pgfpathmoveto{\pgfqpoint{2.445872in}{0.797591in}}%
\pgfpathcurveto{\pgfqpoint{2.456922in}{0.797591in}}{\pgfqpoint{2.467521in}{0.801982in}}{\pgfqpoint{2.475335in}{0.809795in}}%
\pgfpathcurveto{\pgfqpoint{2.483148in}{0.817609in}}{\pgfqpoint{2.487539in}{0.828208in}}{\pgfqpoint{2.487539in}{0.839258in}}%
\pgfpathcurveto{\pgfqpoint{2.487539in}{0.850308in}}{\pgfqpoint{2.483148in}{0.860907in}}{\pgfqpoint{2.475335in}{0.868721in}}%
\pgfpathcurveto{\pgfqpoint{2.467521in}{0.876534in}}{\pgfqpoint{2.456922in}{0.880925in}}{\pgfqpoint{2.445872in}{0.880925in}}%
\pgfpathcurveto{\pgfqpoint{2.434822in}{0.880925in}}{\pgfqpoint{2.424223in}{0.876534in}}{\pgfqpoint{2.416409in}{0.868721in}}%
\pgfpathcurveto{\pgfqpoint{2.408596in}{0.860907in}}{\pgfqpoint{2.404205in}{0.850308in}}{\pgfqpoint{2.404205in}{0.839258in}}%
\pgfpathcurveto{\pgfqpoint{2.404205in}{0.828208in}}{\pgfqpoint{2.408596in}{0.817609in}}{\pgfqpoint{2.416409in}{0.809795in}}%
\pgfpathcurveto{\pgfqpoint{2.424223in}{0.801982in}}{\pgfqpoint{2.434822in}{0.797591in}}{\pgfqpoint{2.445872in}{0.797591in}}%
\pgfpathclose%
\pgfusepath{stroke,fill}%
\end{pgfscope}%
\begin{pgfscope}%
\pgfpathrectangle{\pgfqpoint{0.481978in}{0.331635in}}{\pgfqpoint{4.960000in}{3.696000in}}%
\pgfusepath{clip}%
\pgfsetbuttcap%
\pgfsetroundjoin%
\definecolor{currentfill}{rgb}{0.631373,0.788235,0.956863}%
\pgfsetfillcolor{currentfill}%
\pgfsetlinewidth{0.481800pt}%
\definecolor{currentstroke}{rgb}{1.000000,1.000000,1.000000}%
\pgfsetstrokecolor{currentstroke}%
\pgfsetdash{}{0pt}%
\pgfpathmoveto{\pgfqpoint{3.678569in}{1.404556in}}%
\pgfpathcurveto{\pgfqpoint{3.689619in}{1.404556in}}{\pgfqpoint{3.700219in}{1.408946in}}{\pgfqpoint{3.708032in}{1.416760in}}%
\pgfpathcurveto{\pgfqpoint{3.715846in}{1.424574in}}{\pgfqpoint{3.720236in}{1.435173in}}{\pgfqpoint{3.720236in}{1.446223in}}%
\pgfpathcurveto{\pgfqpoint{3.720236in}{1.457273in}}{\pgfqpoint{3.715846in}{1.467872in}}{\pgfqpoint{3.708032in}{1.475686in}}%
\pgfpathcurveto{\pgfqpoint{3.700219in}{1.483499in}}{\pgfqpoint{3.689619in}{1.487890in}}{\pgfqpoint{3.678569in}{1.487890in}}%
\pgfpathcurveto{\pgfqpoint{3.667519in}{1.487890in}}{\pgfqpoint{3.656920in}{1.483499in}}{\pgfqpoint{3.649107in}{1.475686in}}%
\pgfpathcurveto{\pgfqpoint{3.641293in}{1.467872in}}{\pgfqpoint{3.636903in}{1.457273in}}{\pgfqpoint{3.636903in}{1.446223in}}%
\pgfpathcurveto{\pgfqpoint{3.636903in}{1.435173in}}{\pgfqpoint{3.641293in}{1.424574in}}{\pgfqpoint{3.649107in}{1.416760in}}%
\pgfpathcurveto{\pgfqpoint{3.656920in}{1.408946in}}{\pgfqpoint{3.667519in}{1.404556in}}{\pgfqpoint{3.678569in}{1.404556in}}%
\pgfpathclose%
\pgfusepath{stroke,fill}%
\end{pgfscope}%
\begin{pgfscope}%
\pgfpathrectangle{\pgfqpoint{0.481978in}{0.331635in}}{\pgfqpoint{4.960000in}{3.696000in}}%
\pgfusepath{clip}%
\pgfsetbuttcap%
\pgfsetroundjoin%
\definecolor{currentfill}{rgb}{0.631373,0.788235,0.956863}%
\pgfsetfillcolor{currentfill}%
\pgfsetlinewidth{0.481800pt}%
\definecolor{currentstroke}{rgb}{1.000000,1.000000,1.000000}%
\pgfsetstrokecolor{currentstroke}%
\pgfsetdash{}{0pt}%
\pgfpathmoveto{\pgfqpoint{3.739727in}{2.721524in}}%
\pgfpathcurveto{\pgfqpoint{3.750777in}{2.721524in}}{\pgfqpoint{3.761376in}{2.725915in}}{\pgfqpoint{3.769190in}{2.733728in}}%
\pgfpathcurveto{\pgfqpoint{3.777004in}{2.741542in}}{\pgfqpoint{3.781394in}{2.752141in}}{\pgfqpoint{3.781394in}{2.763191in}}%
\pgfpathcurveto{\pgfqpoint{3.781394in}{2.774241in}}{\pgfqpoint{3.777004in}{2.784840in}}{\pgfqpoint{3.769190in}{2.792654in}}%
\pgfpathcurveto{\pgfqpoint{3.761376in}{2.800467in}}{\pgfqpoint{3.750777in}{2.804858in}}{\pgfqpoint{3.739727in}{2.804858in}}%
\pgfpathcurveto{\pgfqpoint{3.728677in}{2.804858in}}{\pgfqpoint{3.718078in}{2.800467in}}{\pgfqpoint{3.710264in}{2.792654in}}%
\pgfpathcurveto{\pgfqpoint{3.702451in}{2.784840in}}{\pgfqpoint{3.698061in}{2.774241in}}{\pgfqpoint{3.698061in}{2.763191in}}%
\pgfpathcurveto{\pgfqpoint{3.698061in}{2.752141in}}{\pgfqpoint{3.702451in}{2.741542in}}{\pgfqpoint{3.710264in}{2.733728in}}%
\pgfpathcurveto{\pgfqpoint{3.718078in}{2.725915in}}{\pgfqpoint{3.728677in}{2.721524in}}{\pgfqpoint{3.739727in}{2.721524in}}%
\pgfpathclose%
\pgfusepath{stroke,fill}%
\end{pgfscope}%
\begin{pgfscope}%
\pgfpathrectangle{\pgfqpoint{0.481978in}{0.331635in}}{\pgfqpoint{4.960000in}{3.696000in}}%
\pgfusepath{clip}%
\pgfsetbuttcap%
\pgfsetroundjoin%
\definecolor{currentfill}{rgb}{0.631373,0.788235,0.956863}%
\pgfsetfillcolor{currentfill}%
\pgfsetlinewidth{0.481800pt}%
\definecolor{currentstroke}{rgb}{1.000000,1.000000,1.000000}%
\pgfsetstrokecolor{currentstroke}%
\pgfsetdash{}{0pt}%
\pgfpathmoveto{\pgfqpoint{3.304052in}{3.326283in}}%
\pgfpathcurveto{\pgfqpoint{3.315102in}{3.326283in}}{\pgfqpoint{3.325701in}{3.330673in}}{\pgfqpoint{3.333514in}{3.338487in}}%
\pgfpathcurveto{\pgfqpoint{3.341328in}{3.346300in}}{\pgfqpoint{3.345718in}{3.356899in}}{\pgfqpoint{3.345718in}{3.367949in}}%
\pgfpathcurveto{\pgfqpoint{3.345718in}{3.379000in}}{\pgfqpoint{3.341328in}{3.389599in}}{\pgfqpoint{3.333514in}{3.397412in}}%
\pgfpathcurveto{\pgfqpoint{3.325701in}{3.405226in}}{\pgfqpoint{3.315102in}{3.409616in}}{\pgfqpoint{3.304052in}{3.409616in}}%
\pgfpathcurveto{\pgfqpoint{3.293001in}{3.409616in}}{\pgfqpoint{3.282402in}{3.405226in}}{\pgfqpoint{3.274589in}{3.397412in}}%
\pgfpathcurveto{\pgfqpoint{3.266775in}{3.389599in}}{\pgfqpoint{3.262385in}{3.379000in}}{\pgfqpoint{3.262385in}{3.367949in}}%
\pgfpathcurveto{\pgfqpoint{3.262385in}{3.356899in}}{\pgfqpoint{3.266775in}{3.346300in}}{\pgfqpoint{3.274589in}{3.338487in}}%
\pgfpathcurveto{\pgfqpoint{3.282402in}{3.330673in}}{\pgfqpoint{3.293001in}{3.326283in}}{\pgfqpoint{3.304052in}{3.326283in}}%
\pgfpathclose%
\pgfusepath{stroke,fill}%
\end{pgfscope}%
\begin{pgfscope}%
\pgfpathrectangle{\pgfqpoint{0.481978in}{0.331635in}}{\pgfqpoint{4.960000in}{3.696000in}}%
\pgfusepath{clip}%
\pgfsetbuttcap%
\pgfsetroundjoin%
\definecolor{currentfill}{rgb}{0.631373,0.788235,0.956863}%
\pgfsetfillcolor{currentfill}%
\pgfsetlinewidth{0.481800pt}%
\definecolor{currentstroke}{rgb}{1.000000,1.000000,1.000000}%
\pgfsetstrokecolor{currentstroke}%
\pgfsetdash{}{0pt}%
\pgfpathmoveto{\pgfqpoint{3.656864in}{2.516222in}}%
\pgfpathcurveto{\pgfqpoint{3.667914in}{2.516222in}}{\pgfqpoint{3.678513in}{2.520612in}}{\pgfqpoint{3.686327in}{2.528426in}}%
\pgfpathcurveto{\pgfqpoint{3.694140in}{2.536239in}}{\pgfqpoint{3.698531in}{2.546838in}}{\pgfqpoint{3.698531in}{2.557888in}}%
\pgfpathcurveto{\pgfqpoint{3.698531in}{2.568939in}}{\pgfqpoint{3.694140in}{2.579538in}}{\pgfqpoint{3.686327in}{2.587351in}}%
\pgfpathcurveto{\pgfqpoint{3.678513in}{2.595165in}}{\pgfqpoint{3.667914in}{2.599555in}}{\pgfqpoint{3.656864in}{2.599555in}}%
\pgfpathcurveto{\pgfqpoint{3.645814in}{2.599555in}}{\pgfqpoint{3.635215in}{2.595165in}}{\pgfqpoint{3.627401in}{2.587351in}}%
\pgfpathcurveto{\pgfqpoint{3.619588in}{2.579538in}}{\pgfqpoint{3.615197in}{2.568939in}}{\pgfqpoint{3.615197in}{2.557888in}}%
\pgfpathcurveto{\pgfqpoint{3.615197in}{2.546838in}}{\pgfqpoint{3.619588in}{2.536239in}}{\pgfqpoint{3.627401in}{2.528426in}}%
\pgfpathcurveto{\pgfqpoint{3.635215in}{2.520612in}}{\pgfqpoint{3.645814in}{2.516222in}}{\pgfqpoint{3.656864in}{2.516222in}}%
\pgfpathclose%
\pgfusepath{stroke,fill}%
\end{pgfscope}%
\begin{pgfscope}%
\pgfpathrectangle{\pgfqpoint{0.481978in}{0.331635in}}{\pgfqpoint{4.960000in}{3.696000in}}%
\pgfusepath{clip}%
\pgfsetbuttcap%
\pgfsetroundjoin%
\definecolor{currentfill}{rgb}{0.631373,0.788235,0.956863}%
\pgfsetfillcolor{currentfill}%
\pgfsetlinewidth{0.481800pt}%
\definecolor{currentstroke}{rgb}{1.000000,1.000000,1.000000}%
\pgfsetstrokecolor{currentstroke}%
\pgfsetdash{}{0pt}%
\pgfpathmoveto{\pgfqpoint{2.945780in}{0.542107in}}%
\pgfpathcurveto{\pgfqpoint{2.956830in}{0.542107in}}{\pgfqpoint{2.967429in}{0.546497in}}{\pgfqpoint{2.975243in}{0.554311in}}%
\pgfpathcurveto{\pgfqpoint{2.983056in}{0.562124in}}{\pgfqpoint{2.987447in}{0.572724in}}{\pgfqpoint{2.987447in}{0.583774in}}%
\pgfpathcurveto{\pgfqpoint{2.987447in}{0.594824in}}{\pgfqpoint{2.983056in}{0.605423in}}{\pgfqpoint{2.975243in}{0.613236in}}%
\pgfpathcurveto{\pgfqpoint{2.967429in}{0.621050in}}{\pgfqpoint{2.956830in}{0.625440in}}{\pgfqpoint{2.945780in}{0.625440in}}%
\pgfpathcurveto{\pgfqpoint{2.934730in}{0.625440in}}{\pgfqpoint{2.924131in}{0.621050in}}{\pgfqpoint{2.916317in}{0.613236in}}%
\pgfpathcurveto{\pgfqpoint{2.908503in}{0.605423in}}{\pgfqpoint{2.904113in}{0.594824in}}{\pgfqpoint{2.904113in}{0.583774in}}%
\pgfpathcurveto{\pgfqpoint{2.904113in}{0.572724in}}{\pgfqpoint{2.908503in}{0.562124in}}{\pgfqpoint{2.916317in}{0.554311in}}%
\pgfpathcurveto{\pgfqpoint{2.924131in}{0.546497in}}{\pgfqpoint{2.934730in}{0.542107in}}{\pgfqpoint{2.945780in}{0.542107in}}%
\pgfpathclose%
\pgfusepath{stroke,fill}%
\end{pgfscope}%
\begin{pgfscope}%
\pgfpathrectangle{\pgfqpoint{0.481978in}{0.331635in}}{\pgfqpoint{4.960000in}{3.696000in}}%
\pgfusepath{clip}%
\pgfsetbuttcap%
\pgfsetroundjoin%
\definecolor{currentfill}{rgb}{0.631373,0.788235,0.956863}%
\pgfsetfillcolor{currentfill}%
\pgfsetlinewidth{0.481800pt}%
\definecolor{currentstroke}{rgb}{1.000000,1.000000,1.000000}%
\pgfsetstrokecolor{currentstroke}%
\pgfsetdash{}{0pt}%
\pgfpathmoveto{\pgfqpoint{3.781439in}{2.759142in}}%
\pgfpathcurveto{\pgfqpoint{3.792489in}{2.759142in}}{\pgfqpoint{3.803088in}{2.763532in}}{\pgfqpoint{3.810902in}{2.771346in}}%
\pgfpathcurveto{\pgfqpoint{3.818715in}{2.779159in}}{\pgfqpoint{3.823105in}{2.789758in}}{\pgfqpoint{3.823105in}{2.800808in}}%
\pgfpathcurveto{\pgfqpoint{3.823105in}{2.811859in}}{\pgfqpoint{3.818715in}{2.822458in}}{\pgfqpoint{3.810902in}{2.830271in}}%
\pgfpathcurveto{\pgfqpoint{3.803088in}{2.838085in}}{\pgfqpoint{3.792489in}{2.842475in}}{\pgfqpoint{3.781439in}{2.842475in}}%
\pgfpathcurveto{\pgfqpoint{3.770389in}{2.842475in}}{\pgfqpoint{3.759790in}{2.838085in}}{\pgfqpoint{3.751976in}{2.830271in}}%
\pgfpathcurveto{\pgfqpoint{3.744162in}{2.822458in}}{\pgfqpoint{3.739772in}{2.811859in}}{\pgfqpoint{3.739772in}{2.800808in}}%
\pgfpathcurveto{\pgfqpoint{3.739772in}{2.789758in}}{\pgfqpoint{3.744162in}{2.779159in}}{\pgfqpoint{3.751976in}{2.771346in}}%
\pgfpathcurveto{\pgfqpoint{3.759790in}{2.763532in}}{\pgfqpoint{3.770389in}{2.759142in}}{\pgfqpoint{3.781439in}{2.759142in}}%
\pgfpathclose%
\pgfusepath{stroke,fill}%
\end{pgfscope}%
\begin{pgfscope}%
\pgfpathrectangle{\pgfqpoint{0.481978in}{0.331635in}}{\pgfqpoint{4.960000in}{3.696000in}}%
\pgfusepath{clip}%
\pgfsetbuttcap%
\pgfsetroundjoin%
\definecolor{currentfill}{rgb}{0.631373,0.788235,0.956863}%
\pgfsetfillcolor{currentfill}%
\pgfsetlinewidth{0.481800pt}%
\definecolor{currentstroke}{rgb}{1.000000,1.000000,1.000000}%
\pgfsetstrokecolor{currentstroke}%
\pgfsetdash{}{0pt}%
\pgfpathmoveto{\pgfqpoint{3.557003in}{0.711282in}}%
\pgfpathcurveto{\pgfqpoint{3.568053in}{0.711282in}}{\pgfqpoint{3.578652in}{0.715672in}}{\pgfqpoint{3.586466in}{0.723486in}}%
\pgfpathcurveto{\pgfqpoint{3.594279in}{0.731300in}}{\pgfqpoint{3.598670in}{0.741899in}}{\pgfqpoint{3.598670in}{0.752949in}}%
\pgfpathcurveto{\pgfqpoint{3.598670in}{0.763999in}}{\pgfqpoint{3.594279in}{0.774598in}}{\pgfqpoint{3.586466in}{0.782412in}}%
\pgfpathcurveto{\pgfqpoint{3.578652in}{0.790225in}}{\pgfqpoint{3.568053in}{0.794615in}}{\pgfqpoint{3.557003in}{0.794615in}}%
\pgfpathcurveto{\pgfqpoint{3.545953in}{0.794615in}}{\pgfqpoint{3.535354in}{0.790225in}}{\pgfqpoint{3.527540in}{0.782412in}}%
\pgfpathcurveto{\pgfqpoint{3.519726in}{0.774598in}}{\pgfqpoint{3.515336in}{0.763999in}}{\pgfqpoint{3.515336in}{0.752949in}}%
\pgfpathcurveto{\pgfqpoint{3.515336in}{0.741899in}}{\pgfqpoint{3.519726in}{0.731300in}}{\pgfqpoint{3.527540in}{0.723486in}}%
\pgfpathcurveto{\pgfqpoint{3.535354in}{0.715672in}}{\pgfqpoint{3.545953in}{0.711282in}}{\pgfqpoint{3.557003in}{0.711282in}}%
\pgfpathclose%
\pgfusepath{stroke,fill}%
\end{pgfscope}%
\begin{pgfscope}%
\pgfpathrectangle{\pgfqpoint{0.481978in}{0.331635in}}{\pgfqpoint{4.960000in}{3.696000in}}%
\pgfusepath{clip}%
\pgfsetbuttcap%
\pgfsetroundjoin%
\definecolor{currentfill}{rgb}{0.631373,0.788235,0.956863}%
\pgfsetfillcolor{currentfill}%
\pgfsetlinewidth{0.481800pt}%
\definecolor{currentstroke}{rgb}{1.000000,1.000000,1.000000}%
\pgfsetstrokecolor{currentstroke}%
\pgfsetdash{}{0pt}%
\pgfpathmoveto{\pgfqpoint{2.792035in}{0.972681in}}%
\pgfpathcurveto{\pgfqpoint{2.803085in}{0.972681in}}{\pgfqpoint{2.813684in}{0.977071in}}{\pgfqpoint{2.821498in}{0.984885in}}%
\pgfpathcurveto{\pgfqpoint{2.829311in}{0.992699in}}{\pgfqpoint{2.833702in}{1.003298in}}{\pgfqpoint{2.833702in}{1.014348in}}%
\pgfpathcurveto{\pgfqpoint{2.833702in}{1.025398in}}{\pgfqpoint{2.829311in}{1.035997in}}{\pgfqpoint{2.821498in}{1.043811in}}%
\pgfpathcurveto{\pgfqpoint{2.813684in}{1.051624in}}{\pgfqpoint{2.803085in}{1.056015in}}{\pgfqpoint{2.792035in}{1.056015in}}%
\pgfpathcurveto{\pgfqpoint{2.780985in}{1.056015in}}{\pgfqpoint{2.770386in}{1.051624in}}{\pgfqpoint{2.762572in}{1.043811in}}%
\pgfpathcurveto{\pgfqpoint{2.754759in}{1.035997in}}{\pgfqpoint{2.750368in}{1.025398in}}{\pgfqpoint{2.750368in}{1.014348in}}%
\pgfpathcurveto{\pgfqpoint{2.750368in}{1.003298in}}{\pgfqpoint{2.754759in}{0.992699in}}{\pgfqpoint{2.762572in}{0.984885in}}%
\pgfpathcurveto{\pgfqpoint{2.770386in}{0.977071in}}{\pgfqpoint{2.780985in}{0.972681in}}{\pgfqpoint{2.792035in}{0.972681in}}%
\pgfpathclose%
\pgfusepath{stroke,fill}%
\end{pgfscope}%
\begin{pgfscope}%
\pgfpathrectangle{\pgfqpoint{0.481978in}{0.331635in}}{\pgfqpoint{4.960000in}{3.696000in}}%
\pgfusepath{clip}%
\pgfsetbuttcap%
\pgfsetroundjoin%
\definecolor{currentfill}{rgb}{0.631373,0.788235,0.956863}%
\pgfsetfillcolor{currentfill}%
\pgfsetlinewidth{0.481800pt}%
\definecolor{currentstroke}{rgb}{1.000000,1.000000,1.000000}%
\pgfsetstrokecolor{currentstroke}%
\pgfsetdash{}{0pt}%
\pgfpathmoveto{\pgfqpoint{4.082567in}{1.699852in}}%
\pgfpathcurveto{\pgfqpoint{4.093617in}{1.699852in}}{\pgfqpoint{4.104216in}{1.704242in}}{\pgfqpoint{4.112030in}{1.712056in}}%
\pgfpathcurveto{\pgfqpoint{4.119844in}{1.719869in}}{\pgfqpoint{4.124234in}{1.730468in}}{\pgfqpoint{4.124234in}{1.741519in}}%
\pgfpathcurveto{\pgfqpoint{4.124234in}{1.752569in}}{\pgfqpoint{4.119844in}{1.763168in}}{\pgfqpoint{4.112030in}{1.770981in}}%
\pgfpathcurveto{\pgfqpoint{4.104216in}{1.778795in}}{\pgfqpoint{4.093617in}{1.783185in}}{\pgfqpoint{4.082567in}{1.783185in}}%
\pgfpathcurveto{\pgfqpoint{4.071517in}{1.783185in}}{\pgfqpoint{4.060918in}{1.778795in}}{\pgfqpoint{4.053104in}{1.770981in}}%
\pgfpathcurveto{\pgfqpoint{4.045291in}{1.763168in}}{\pgfqpoint{4.040901in}{1.752569in}}{\pgfqpoint{4.040901in}{1.741519in}}%
\pgfpathcurveto{\pgfqpoint{4.040901in}{1.730468in}}{\pgfqpoint{4.045291in}{1.719869in}}{\pgfqpoint{4.053104in}{1.712056in}}%
\pgfpathcurveto{\pgfqpoint{4.060918in}{1.704242in}}{\pgfqpoint{4.071517in}{1.699852in}}{\pgfqpoint{4.082567in}{1.699852in}}%
\pgfpathclose%
\pgfusepath{stroke,fill}%
\end{pgfscope}%
\begin{pgfscope}%
\pgfpathrectangle{\pgfqpoint{0.481978in}{0.331635in}}{\pgfqpoint{4.960000in}{3.696000in}}%
\pgfusepath{clip}%
\pgfsetbuttcap%
\pgfsetroundjoin%
\definecolor{currentfill}{rgb}{0.631373,0.788235,0.956863}%
\pgfsetfillcolor{currentfill}%
\pgfsetlinewidth{0.481800pt}%
\definecolor{currentstroke}{rgb}{1.000000,1.000000,1.000000}%
\pgfsetstrokecolor{currentstroke}%
\pgfsetdash{}{0pt}%
\pgfpathmoveto{\pgfqpoint{3.232867in}{0.875373in}}%
\pgfpathcurveto{\pgfqpoint{3.243917in}{0.875373in}}{\pgfqpoint{3.254516in}{0.879763in}}{\pgfqpoint{3.262329in}{0.887577in}}%
\pgfpathcurveto{\pgfqpoint{3.270143in}{0.895391in}}{\pgfqpoint{3.274533in}{0.905990in}}{\pgfqpoint{3.274533in}{0.917040in}}%
\pgfpathcurveto{\pgfqpoint{3.274533in}{0.928090in}}{\pgfqpoint{3.270143in}{0.938689in}}{\pgfqpoint{3.262329in}{0.946503in}}%
\pgfpathcurveto{\pgfqpoint{3.254516in}{0.954316in}}{\pgfqpoint{3.243917in}{0.958707in}}{\pgfqpoint{3.232867in}{0.958707in}}%
\pgfpathcurveto{\pgfqpoint{3.221816in}{0.958707in}}{\pgfqpoint{3.211217in}{0.954316in}}{\pgfqpoint{3.203404in}{0.946503in}}%
\pgfpathcurveto{\pgfqpoint{3.195590in}{0.938689in}}{\pgfqpoint{3.191200in}{0.928090in}}{\pgfqpoint{3.191200in}{0.917040in}}%
\pgfpathcurveto{\pgfqpoint{3.191200in}{0.905990in}}{\pgfqpoint{3.195590in}{0.895391in}}{\pgfqpoint{3.203404in}{0.887577in}}%
\pgfpathcurveto{\pgfqpoint{3.211217in}{0.879763in}}{\pgfqpoint{3.221816in}{0.875373in}}{\pgfqpoint{3.232867in}{0.875373in}}%
\pgfpathclose%
\pgfusepath{stroke,fill}%
\end{pgfscope}%
\begin{pgfscope}%
\pgfpathrectangle{\pgfqpoint{0.481978in}{0.331635in}}{\pgfqpoint{4.960000in}{3.696000in}}%
\pgfusepath{clip}%
\pgfsetbuttcap%
\pgfsetroundjoin%
\definecolor{currentfill}{rgb}{0.631373,0.788235,0.956863}%
\pgfsetfillcolor{currentfill}%
\pgfsetlinewidth{0.481800pt}%
\definecolor{currentstroke}{rgb}{1.000000,1.000000,1.000000}%
\pgfsetstrokecolor{currentstroke}%
\pgfsetdash{}{0pt}%
\pgfpathmoveto{\pgfqpoint{3.719174in}{2.440999in}}%
\pgfpathcurveto{\pgfqpoint{3.730225in}{2.440999in}}{\pgfqpoint{3.740824in}{2.445389in}}{\pgfqpoint{3.748637in}{2.453203in}}%
\pgfpathcurveto{\pgfqpoint{3.756451in}{2.461016in}}{\pgfqpoint{3.760841in}{2.471615in}}{\pgfqpoint{3.760841in}{2.482665in}}%
\pgfpathcurveto{\pgfqpoint{3.760841in}{2.493715in}}{\pgfqpoint{3.756451in}{2.504315in}}{\pgfqpoint{3.748637in}{2.512128in}}%
\pgfpathcurveto{\pgfqpoint{3.740824in}{2.519942in}}{\pgfqpoint{3.730225in}{2.524332in}}{\pgfqpoint{3.719174in}{2.524332in}}%
\pgfpathcurveto{\pgfqpoint{3.708124in}{2.524332in}}{\pgfqpoint{3.697525in}{2.519942in}}{\pgfqpoint{3.689712in}{2.512128in}}%
\pgfpathcurveto{\pgfqpoint{3.681898in}{2.504315in}}{\pgfqpoint{3.677508in}{2.493715in}}{\pgfqpoint{3.677508in}{2.482665in}}%
\pgfpathcurveto{\pgfqpoint{3.677508in}{2.471615in}}{\pgfqpoint{3.681898in}{2.461016in}}{\pgfqpoint{3.689712in}{2.453203in}}%
\pgfpathcurveto{\pgfqpoint{3.697525in}{2.445389in}}{\pgfqpoint{3.708124in}{2.440999in}}{\pgfqpoint{3.719174in}{2.440999in}}%
\pgfpathclose%
\pgfusepath{stroke,fill}%
\end{pgfscope}%
\begin{pgfscope}%
\pgfpathrectangle{\pgfqpoint{0.481978in}{0.331635in}}{\pgfqpoint{4.960000in}{3.696000in}}%
\pgfusepath{clip}%
\pgfsetbuttcap%
\pgfsetroundjoin%
\definecolor{currentfill}{rgb}{0.631373,0.788235,0.956863}%
\pgfsetfillcolor{currentfill}%
\pgfsetlinewidth{0.481800pt}%
\definecolor{currentstroke}{rgb}{1.000000,1.000000,1.000000}%
\pgfsetstrokecolor{currentstroke}%
\pgfsetdash{}{0pt}%
\pgfpathmoveto{\pgfqpoint{2.663289in}{2.510470in}}%
\pgfpathcurveto{\pgfqpoint{2.674339in}{2.510470in}}{\pgfqpoint{2.684939in}{2.514860in}}{\pgfqpoint{2.692752in}{2.522674in}}%
\pgfpathcurveto{\pgfqpoint{2.700566in}{2.530488in}}{\pgfqpoint{2.704956in}{2.541087in}}{\pgfqpoint{2.704956in}{2.552137in}}%
\pgfpathcurveto{\pgfqpoint{2.704956in}{2.563187in}}{\pgfqpoint{2.700566in}{2.573786in}}{\pgfqpoint{2.692752in}{2.581600in}}%
\pgfpathcurveto{\pgfqpoint{2.684939in}{2.589413in}}{\pgfqpoint{2.674339in}{2.593803in}}{\pgfqpoint{2.663289in}{2.593803in}}%
\pgfpathcurveto{\pgfqpoint{2.652239in}{2.593803in}}{\pgfqpoint{2.641640in}{2.589413in}}{\pgfqpoint{2.633827in}{2.581600in}}%
\pgfpathcurveto{\pgfqpoint{2.626013in}{2.573786in}}{\pgfqpoint{2.621623in}{2.563187in}}{\pgfqpoint{2.621623in}{2.552137in}}%
\pgfpathcurveto{\pgfqpoint{2.621623in}{2.541087in}}{\pgfqpoint{2.626013in}{2.530488in}}{\pgfqpoint{2.633827in}{2.522674in}}%
\pgfpathcurveto{\pgfqpoint{2.641640in}{2.514860in}}{\pgfqpoint{2.652239in}{2.510470in}}{\pgfqpoint{2.663289in}{2.510470in}}%
\pgfpathclose%
\pgfusepath{stroke,fill}%
\end{pgfscope}%
\begin{pgfscope}%
\pgfpathrectangle{\pgfqpoint{0.481978in}{0.331635in}}{\pgfqpoint{4.960000in}{3.696000in}}%
\pgfusepath{clip}%
\pgfsetbuttcap%
\pgfsetroundjoin%
\definecolor{currentfill}{rgb}{0.631373,0.788235,0.956863}%
\pgfsetfillcolor{currentfill}%
\pgfsetlinewidth{0.481800pt}%
\definecolor{currentstroke}{rgb}{1.000000,1.000000,1.000000}%
\pgfsetstrokecolor{currentstroke}%
\pgfsetdash{}{0pt}%
\pgfpathmoveto{\pgfqpoint{3.485653in}{1.198535in}}%
\pgfpathcurveto{\pgfqpoint{3.496703in}{1.198535in}}{\pgfqpoint{3.507302in}{1.202925in}}{\pgfqpoint{3.515115in}{1.210739in}}%
\pgfpathcurveto{\pgfqpoint{3.522929in}{1.218553in}}{\pgfqpoint{3.527319in}{1.229152in}}{\pgfqpoint{3.527319in}{1.240202in}}%
\pgfpathcurveto{\pgfqpoint{3.527319in}{1.251252in}}{\pgfqpoint{3.522929in}{1.261851in}}{\pgfqpoint{3.515115in}{1.269664in}}%
\pgfpathcurveto{\pgfqpoint{3.507302in}{1.277478in}}{\pgfqpoint{3.496703in}{1.281868in}}{\pgfqpoint{3.485653in}{1.281868in}}%
\pgfpathcurveto{\pgfqpoint{3.474602in}{1.281868in}}{\pgfqpoint{3.464003in}{1.277478in}}{\pgfqpoint{3.456190in}{1.269664in}}%
\pgfpathcurveto{\pgfqpoint{3.448376in}{1.261851in}}{\pgfqpoint{3.443986in}{1.251252in}}{\pgfqpoint{3.443986in}{1.240202in}}%
\pgfpathcurveto{\pgfqpoint{3.443986in}{1.229152in}}{\pgfqpoint{3.448376in}{1.218553in}}{\pgfqpoint{3.456190in}{1.210739in}}%
\pgfpathcurveto{\pgfqpoint{3.464003in}{1.202925in}}{\pgfqpoint{3.474602in}{1.198535in}}{\pgfqpoint{3.485653in}{1.198535in}}%
\pgfpathclose%
\pgfusepath{stroke,fill}%
\end{pgfscope}%
\begin{pgfscope}%
\pgfpathrectangle{\pgfqpoint{0.481978in}{0.331635in}}{\pgfqpoint{4.960000in}{3.696000in}}%
\pgfusepath{clip}%
\pgfsetbuttcap%
\pgfsetroundjoin%
\definecolor{currentfill}{rgb}{0.631373,0.788235,0.956863}%
\pgfsetfillcolor{currentfill}%
\pgfsetlinewidth{0.481800pt}%
\definecolor{currentstroke}{rgb}{1.000000,1.000000,1.000000}%
\pgfsetstrokecolor{currentstroke}%
\pgfsetdash{}{0pt}%
\pgfpathmoveto{\pgfqpoint{2.981550in}{0.472357in}}%
\pgfpathcurveto{\pgfqpoint{2.992601in}{0.472357in}}{\pgfqpoint{3.003200in}{0.476747in}}{\pgfqpoint{3.011013in}{0.484561in}}%
\pgfpathcurveto{\pgfqpoint{3.018827in}{0.492374in}}{\pgfqpoint{3.023217in}{0.502973in}}{\pgfqpoint{3.023217in}{0.514023in}}%
\pgfpathcurveto{\pgfqpoint{3.023217in}{0.525074in}}{\pgfqpoint{3.018827in}{0.535673in}}{\pgfqpoint{3.011013in}{0.543486in}}%
\pgfpathcurveto{\pgfqpoint{3.003200in}{0.551300in}}{\pgfqpoint{2.992601in}{0.555690in}}{\pgfqpoint{2.981550in}{0.555690in}}%
\pgfpathcurveto{\pgfqpoint{2.970500in}{0.555690in}}{\pgfqpoint{2.959901in}{0.551300in}}{\pgfqpoint{2.952088in}{0.543486in}}%
\pgfpathcurveto{\pgfqpoint{2.944274in}{0.535673in}}{\pgfqpoint{2.939884in}{0.525074in}}{\pgfqpoint{2.939884in}{0.514023in}}%
\pgfpathcurveto{\pgfqpoint{2.939884in}{0.502973in}}{\pgfqpoint{2.944274in}{0.492374in}}{\pgfqpoint{2.952088in}{0.484561in}}%
\pgfpathcurveto{\pgfqpoint{2.959901in}{0.476747in}}{\pgfqpoint{2.970500in}{0.472357in}}{\pgfqpoint{2.981550in}{0.472357in}}%
\pgfpathclose%
\pgfusepath{stroke,fill}%
\end{pgfscope}%
\begin{pgfscope}%
\pgfpathrectangle{\pgfqpoint{0.481978in}{0.331635in}}{\pgfqpoint{4.960000in}{3.696000in}}%
\pgfusepath{clip}%
\pgfsetbuttcap%
\pgfsetroundjoin%
\definecolor{currentfill}{rgb}{0.631373,0.788235,0.956863}%
\pgfsetfillcolor{currentfill}%
\pgfsetlinewidth{0.481800pt}%
\definecolor{currentstroke}{rgb}{1.000000,1.000000,1.000000}%
\pgfsetstrokecolor{currentstroke}%
\pgfsetdash{}{0pt}%
\pgfpathmoveto{\pgfqpoint{3.622294in}{2.006996in}}%
\pgfpathcurveto{\pgfqpoint{3.633344in}{2.006996in}}{\pgfqpoint{3.643943in}{2.011386in}}{\pgfqpoint{3.651757in}{2.019200in}}%
\pgfpathcurveto{\pgfqpoint{3.659571in}{2.027013in}}{\pgfqpoint{3.663961in}{2.037612in}}{\pgfqpoint{3.663961in}{2.048663in}}%
\pgfpathcurveto{\pgfqpoint{3.663961in}{2.059713in}}{\pgfqpoint{3.659571in}{2.070312in}}{\pgfqpoint{3.651757in}{2.078125in}}%
\pgfpathcurveto{\pgfqpoint{3.643943in}{2.085939in}}{\pgfqpoint{3.633344in}{2.090329in}}{\pgfqpoint{3.622294in}{2.090329in}}%
\pgfpathcurveto{\pgfqpoint{3.611244in}{2.090329in}}{\pgfqpoint{3.600645in}{2.085939in}}{\pgfqpoint{3.592831in}{2.078125in}}%
\pgfpathcurveto{\pgfqpoint{3.585018in}{2.070312in}}{\pgfqpoint{3.580627in}{2.059713in}}{\pgfqpoint{3.580627in}{2.048663in}}%
\pgfpathcurveto{\pgfqpoint{3.580627in}{2.037612in}}{\pgfqpoint{3.585018in}{2.027013in}}{\pgfqpoint{3.592831in}{2.019200in}}%
\pgfpathcurveto{\pgfqpoint{3.600645in}{2.011386in}}{\pgfqpoint{3.611244in}{2.006996in}}{\pgfqpoint{3.622294in}{2.006996in}}%
\pgfpathclose%
\pgfusepath{stroke,fill}%
\end{pgfscope}%
\begin{pgfscope}%
\pgfpathrectangle{\pgfqpoint{0.481978in}{0.331635in}}{\pgfqpoint{4.960000in}{3.696000in}}%
\pgfusepath{clip}%
\pgfsetbuttcap%
\pgfsetroundjoin%
\definecolor{currentfill}{rgb}{0.631373,0.788235,0.956863}%
\pgfsetfillcolor{currentfill}%
\pgfsetlinewidth{0.481800pt}%
\definecolor{currentstroke}{rgb}{1.000000,1.000000,1.000000}%
\pgfsetstrokecolor{currentstroke}%
\pgfsetdash{}{0pt}%
\pgfpathmoveto{\pgfqpoint{4.139050in}{1.317252in}}%
\pgfpathcurveto{\pgfqpoint{4.150100in}{1.317252in}}{\pgfqpoint{4.160699in}{1.321643in}}{\pgfqpoint{4.168512in}{1.329456in}}%
\pgfpathcurveto{\pgfqpoint{4.176326in}{1.337270in}}{\pgfqpoint{4.180716in}{1.347869in}}{\pgfqpoint{4.180716in}{1.358919in}}%
\pgfpathcurveto{\pgfqpoint{4.180716in}{1.369969in}}{\pgfqpoint{4.176326in}{1.380568in}}{\pgfqpoint{4.168512in}{1.388382in}}%
\pgfpathcurveto{\pgfqpoint{4.160699in}{1.396195in}}{\pgfqpoint{4.150100in}{1.400586in}}{\pgfqpoint{4.139050in}{1.400586in}}%
\pgfpathcurveto{\pgfqpoint{4.127999in}{1.400586in}}{\pgfqpoint{4.117400in}{1.396195in}}{\pgfqpoint{4.109587in}{1.388382in}}%
\pgfpathcurveto{\pgfqpoint{4.101773in}{1.380568in}}{\pgfqpoint{4.097383in}{1.369969in}}{\pgfqpoint{4.097383in}{1.358919in}}%
\pgfpathcurveto{\pgfqpoint{4.097383in}{1.347869in}}{\pgfqpoint{4.101773in}{1.337270in}}{\pgfqpoint{4.109587in}{1.329456in}}%
\pgfpathcurveto{\pgfqpoint{4.117400in}{1.321643in}}{\pgfqpoint{4.127999in}{1.317252in}}{\pgfqpoint{4.139050in}{1.317252in}}%
\pgfpathclose%
\pgfusepath{stroke,fill}%
\end{pgfscope}%
\begin{pgfscope}%
\pgfpathrectangle{\pgfqpoint{0.481978in}{0.331635in}}{\pgfqpoint{4.960000in}{3.696000in}}%
\pgfusepath{clip}%
\pgfsetbuttcap%
\pgfsetroundjoin%
\definecolor{currentfill}{rgb}{0.631373,0.788235,0.956863}%
\pgfsetfillcolor{currentfill}%
\pgfsetlinewidth{0.481800pt}%
\definecolor{currentstroke}{rgb}{1.000000,1.000000,1.000000}%
\pgfsetstrokecolor{currentstroke}%
\pgfsetdash{}{0pt}%
\pgfpathmoveto{\pgfqpoint{2.205469in}{0.979014in}}%
\pgfpathcurveto{\pgfqpoint{2.216519in}{0.979014in}}{\pgfqpoint{2.227118in}{0.983404in}}{\pgfqpoint{2.234931in}{0.991218in}}%
\pgfpathcurveto{\pgfqpoint{2.242745in}{0.999031in}}{\pgfqpoint{2.247135in}{1.009630in}}{\pgfqpoint{2.247135in}{1.020681in}}%
\pgfpathcurveto{\pgfqpoint{2.247135in}{1.031731in}}{\pgfqpoint{2.242745in}{1.042330in}}{\pgfqpoint{2.234931in}{1.050143in}}%
\pgfpathcurveto{\pgfqpoint{2.227118in}{1.057957in}}{\pgfqpoint{2.216519in}{1.062347in}}{\pgfqpoint{2.205469in}{1.062347in}}%
\pgfpathcurveto{\pgfqpoint{2.194418in}{1.062347in}}{\pgfqpoint{2.183819in}{1.057957in}}{\pgfqpoint{2.176006in}{1.050143in}}%
\pgfpathcurveto{\pgfqpoint{2.168192in}{1.042330in}}{\pgfqpoint{2.163802in}{1.031731in}}{\pgfqpoint{2.163802in}{1.020681in}}%
\pgfpathcurveto{\pgfqpoint{2.163802in}{1.009630in}}{\pgfqpoint{2.168192in}{0.999031in}}{\pgfqpoint{2.176006in}{0.991218in}}%
\pgfpathcurveto{\pgfqpoint{2.183819in}{0.983404in}}{\pgfqpoint{2.194418in}{0.979014in}}{\pgfqpoint{2.205469in}{0.979014in}}%
\pgfpathclose%
\pgfusepath{stroke,fill}%
\end{pgfscope}%
\begin{pgfscope}%
\pgfpathrectangle{\pgfqpoint{0.481978in}{0.331635in}}{\pgfqpoint{4.960000in}{3.696000in}}%
\pgfusepath{clip}%
\pgfsetbuttcap%
\pgfsetroundjoin%
\definecolor{currentfill}{rgb}{0.631373,0.788235,0.956863}%
\pgfsetfillcolor{currentfill}%
\pgfsetlinewidth{0.481800pt}%
\definecolor{currentstroke}{rgb}{1.000000,1.000000,1.000000}%
\pgfsetstrokecolor{currentstroke}%
\pgfsetdash{}{0pt}%
\pgfpathmoveto{\pgfqpoint{2.990475in}{3.238523in}}%
\pgfpathcurveto{\pgfqpoint{3.001525in}{3.238523in}}{\pgfqpoint{3.012124in}{3.242913in}}{\pgfqpoint{3.019937in}{3.250727in}}%
\pgfpathcurveto{\pgfqpoint{3.027751in}{3.258541in}}{\pgfqpoint{3.032141in}{3.269140in}}{\pgfqpoint{3.032141in}{3.280190in}}%
\pgfpathcurveto{\pgfqpoint{3.032141in}{3.291240in}}{\pgfqpoint{3.027751in}{3.301839in}}{\pgfqpoint{3.019937in}{3.309653in}}%
\pgfpathcurveto{\pgfqpoint{3.012124in}{3.317466in}}{\pgfqpoint{3.001525in}{3.321857in}}{\pgfqpoint{2.990475in}{3.321857in}}%
\pgfpathcurveto{\pgfqpoint{2.979424in}{3.321857in}}{\pgfqpoint{2.968825in}{3.317466in}}{\pgfqpoint{2.961012in}{3.309653in}}%
\pgfpathcurveto{\pgfqpoint{2.953198in}{3.301839in}}{\pgfqpoint{2.948808in}{3.291240in}}{\pgfqpoint{2.948808in}{3.280190in}}%
\pgfpathcurveto{\pgfqpoint{2.948808in}{3.269140in}}{\pgfqpoint{2.953198in}{3.258541in}}{\pgfqpoint{2.961012in}{3.250727in}}%
\pgfpathcurveto{\pgfqpoint{2.968825in}{3.242913in}}{\pgfqpoint{2.979424in}{3.238523in}}{\pgfqpoint{2.990475in}{3.238523in}}%
\pgfpathclose%
\pgfusepath{stroke,fill}%
\end{pgfscope}%
\begin{pgfscope}%
\pgfpathrectangle{\pgfqpoint{0.481978in}{0.331635in}}{\pgfqpoint{4.960000in}{3.696000in}}%
\pgfusepath{clip}%
\pgfsetbuttcap%
\pgfsetroundjoin%
\definecolor{currentfill}{rgb}{0.631373,0.788235,0.956863}%
\pgfsetfillcolor{currentfill}%
\pgfsetlinewidth{0.481800pt}%
\definecolor{currentstroke}{rgb}{1.000000,1.000000,1.000000}%
\pgfsetstrokecolor{currentstroke}%
\pgfsetdash{}{0pt}%
\pgfpathmoveto{\pgfqpoint{2.666430in}{0.810118in}}%
\pgfpathcurveto{\pgfqpoint{2.677480in}{0.810118in}}{\pgfqpoint{2.688079in}{0.814508in}}{\pgfqpoint{2.695893in}{0.822322in}}%
\pgfpathcurveto{\pgfqpoint{2.703707in}{0.830136in}}{\pgfqpoint{2.708097in}{0.840735in}}{\pgfqpoint{2.708097in}{0.851785in}}%
\pgfpathcurveto{\pgfqpoint{2.708097in}{0.862835in}}{\pgfqpoint{2.703707in}{0.873434in}}{\pgfqpoint{2.695893in}{0.881247in}}%
\pgfpathcurveto{\pgfqpoint{2.688079in}{0.889061in}}{\pgfqpoint{2.677480in}{0.893451in}}{\pgfqpoint{2.666430in}{0.893451in}}%
\pgfpathcurveto{\pgfqpoint{2.655380in}{0.893451in}}{\pgfqpoint{2.644781in}{0.889061in}}{\pgfqpoint{2.636967in}{0.881247in}}%
\pgfpathcurveto{\pgfqpoint{2.629154in}{0.873434in}}{\pgfqpoint{2.624764in}{0.862835in}}{\pgfqpoint{2.624764in}{0.851785in}}%
\pgfpathcurveto{\pgfqpoint{2.624764in}{0.840735in}}{\pgfqpoint{2.629154in}{0.830136in}}{\pgfqpoint{2.636967in}{0.822322in}}%
\pgfpathcurveto{\pgfqpoint{2.644781in}{0.814508in}}{\pgfqpoint{2.655380in}{0.810118in}}{\pgfqpoint{2.666430in}{0.810118in}}%
\pgfpathclose%
\pgfusepath{stroke,fill}%
\end{pgfscope}%
\begin{pgfscope}%
\pgfpathrectangle{\pgfqpoint{0.481978in}{0.331635in}}{\pgfqpoint{4.960000in}{3.696000in}}%
\pgfusepath{clip}%
\pgfsetbuttcap%
\pgfsetroundjoin%
\definecolor{currentfill}{rgb}{0.631373,0.788235,0.956863}%
\pgfsetfillcolor{currentfill}%
\pgfsetlinewidth{0.481800pt}%
\definecolor{currentstroke}{rgb}{1.000000,1.000000,1.000000}%
\pgfsetstrokecolor{currentstroke}%
\pgfsetdash{}{0pt}%
\pgfpathmoveto{\pgfqpoint{2.845022in}{2.136089in}}%
\pgfpathcurveto{\pgfqpoint{2.856072in}{2.136089in}}{\pgfqpoint{2.866671in}{2.140479in}}{\pgfqpoint{2.874484in}{2.148293in}}%
\pgfpathcurveto{\pgfqpoint{2.882298in}{2.156106in}}{\pgfqpoint{2.886688in}{2.166705in}}{\pgfqpoint{2.886688in}{2.177755in}}%
\pgfpathcurveto{\pgfqpoint{2.886688in}{2.188806in}}{\pgfqpoint{2.882298in}{2.199405in}}{\pgfqpoint{2.874484in}{2.207218in}}%
\pgfpathcurveto{\pgfqpoint{2.866671in}{2.215032in}}{\pgfqpoint{2.856072in}{2.219422in}}{\pgfqpoint{2.845022in}{2.219422in}}%
\pgfpathcurveto{\pgfqpoint{2.833972in}{2.219422in}}{\pgfqpoint{2.823373in}{2.215032in}}{\pgfqpoint{2.815559in}{2.207218in}}%
\pgfpathcurveto{\pgfqpoint{2.807745in}{2.199405in}}{\pgfqpoint{2.803355in}{2.188806in}}{\pgfqpoint{2.803355in}{2.177755in}}%
\pgfpathcurveto{\pgfqpoint{2.803355in}{2.166705in}}{\pgfqpoint{2.807745in}{2.156106in}}{\pgfqpoint{2.815559in}{2.148293in}}%
\pgfpathcurveto{\pgfqpoint{2.823373in}{2.140479in}}{\pgfqpoint{2.833972in}{2.136089in}}{\pgfqpoint{2.845022in}{2.136089in}}%
\pgfpathclose%
\pgfusepath{stroke,fill}%
\end{pgfscope}%
\begin{pgfscope}%
\pgfpathrectangle{\pgfqpoint{0.481978in}{0.331635in}}{\pgfqpoint{4.960000in}{3.696000in}}%
\pgfusepath{clip}%
\pgfsetbuttcap%
\pgfsetroundjoin%
\definecolor{currentfill}{rgb}{0.631373,0.788235,0.956863}%
\pgfsetfillcolor{currentfill}%
\pgfsetlinewidth{0.481800pt}%
\definecolor{currentstroke}{rgb}{1.000000,1.000000,1.000000}%
\pgfsetstrokecolor{currentstroke}%
\pgfsetdash{}{0pt}%
\pgfpathmoveto{\pgfqpoint{4.914434in}{2.275002in}}%
\pgfpathcurveto{\pgfqpoint{4.925484in}{2.275002in}}{\pgfqpoint{4.936083in}{2.279392in}}{\pgfqpoint{4.943896in}{2.287205in}}%
\pgfpathcurveto{\pgfqpoint{4.951710in}{2.295019in}}{\pgfqpoint{4.956100in}{2.305618in}}{\pgfqpoint{4.956100in}{2.316668in}}%
\pgfpathcurveto{\pgfqpoint{4.956100in}{2.327718in}}{\pgfqpoint{4.951710in}{2.338317in}}{\pgfqpoint{4.943896in}{2.346131in}}%
\pgfpathcurveto{\pgfqpoint{4.936083in}{2.353945in}}{\pgfqpoint{4.925484in}{2.358335in}}{\pgfqpoint{4.914434in}{2.358335in}}%
\pgfpathcurveto{\pgfqpoint{4.903384in}{2.358335in}}{\pgfqpoint{4.892785in}{2.353945in}}{\pgfqpoint{4.884971in}{2.346131in}}%
\pgfpathcurveto{\pgfqpoint{4.877157in}{2.338317in}}{\pgfqpoint{4.872767in}{2.327718in}}{\pgfqpoint{4.872767in}{2.316668in}}%
\pgfpathcurveto{\pgfqpoint{4.872767in}{2.305618in}}{\pgfqpoint{4.877157in}{2.295019in}}{\pgfqpoint{4.884971in}{2.287205in}}%
\pgfpathcurveto{\pgfqpoint{4.892785in}{2.279392in}}{\pgfqpoint{4.903384in}{2.275002in}}{\pgfqpoint{4.914434in}{2.275002in}}%
\pgfpathclose%
\pgfusepath{stroke,fill}%
\end{pgfscope}%
\begin{pgfscope}%
\pgfpathrectangle{\pgfqpoint{0.481978in}{0.331635in}}{\pgfqpoint{4.960000in}{3.696000in}}%
\pgfusepath{clip}%
\pgfsetbuttcap%
\pgfsetroundjoin%
\definecolor{currentfill}{rgb}{0.631373,0.788235,0.956863}%
\pgfsetfillcolor{currentfill}%
\pgfsetlinewidth{0.481800pt}%
\definecolor{currentstroke}{rgb}{1.000000,1.000000,1.000000}%
\pgfsetstrokecolor{currentstroke}%
\pgfsetdash{}{0pt}%
\pgfpathmoveto{\pgfqpoint{3.626594in}{1.673133in}}%
\pgfpathcurveto{\pgfqpoint{3.637644in}{1.673133in}}{\pgfqpoint{3.648243in}{1.677523in}}{\pgfqpoint{3.656056in}{1.685337in}}%
\pgfpathcurveto{\pgfqpoint{3.663870in}{1.693150in}}{\pgfqpoint{3.668260in}{1.703750in}}{\pgfqpoint{3.668260in}{1.714800in}}%
\pgfpathcurveto{\pgfqpoint{3.668260in}{1.725850in}}{\pgfqpoint{3.663870in}{1.736449in}}{\pgfqpoint{3.656056in}{1.744262in}}%
\pgfpathcurveto{\pgfqpoint{3.648243in}{1.752076in}}{\pgfqpoint{3.637644in}{1.756466in}}{\pgfqpoint{3.626594in}{1.756466in}}%
\pgfpathcurveto{\pgfqpoint{3.615543in}{1.756466in}}{\pgfqpoint{3.604944in}{1.752076in}}{\pgfqpoint{3.597131in}{1.744262in}}%
\pgfpathcurveto{\pgfqpoint{3.589317in}{1.736449in}}{\pgfqpoint{3.584927in}{1.725850in}}{\pgfqpoint{3.584927in}{1.714800in}}%
\pgfpathcurveto{\pgfqpoint{3.584927in}{1.703750in}}{\pgfqpoint{3.589317in}{1.693150in}}{\pgfqpoint{3.597131in}{1.685337in}}%
\pgfpathcurveto{\pgfqpoint{3.604944in}{1.677523in}}{\pgfqpoint{3.615543in}{1.673133in}}{\pgfqpoint{3.626594in}{1.673133in}}%
\pgfpathclose%
\pgfusepath{stroke,fill}%
\end{pgfscope}%
\begin{pgfscope}%
\pgfpathrectangle{\pgfqpoint{0.481978in}{0.331635in}}{\pgfqpoint{4.960000in}{3.696000in}}%
\pgfusepath{clip}%
\pgfsetbuttcap%
\pgfsetroundjoin%
\definecolor{currentfill}{rgb}{0.631373,0.788235,0.956863}%
\pgfsetfillcolor{currentfill}%
\pgfsetlinewidth{0.481800pt}%
\definecolor{currentstroke}{rgb}{1.000000,1.000000,1.000000}%
\pgfsetstrokecolor{currentstroke}%
\pgfsetdash{}{0pt}%
\pgfpathmoveto{\pgfqpoint{2.797292in}{1.331510in}}%
\pgfpathcurveto{\pgfqpoint{2.808342in}{1.331510in}}{\pgfqpoint{2.818941in}{1.335901in}}{\pgfqpoint{2.826754in}{1.343714in}}%
\pgfpathcurveto{\pgfqpoint{2.834568in}{1.351528in}}{\pgfqpoint{2.838958in}{1.362127in}}{\pgfqpoint{2.838958in}{1.373177in}}%
\pgfpathcurveto{\pgfqpoint{2.838958in}{1.384227in}}{\pgfqpoint{2.834568in}{1.394826in}}{\pgfqpoint{2.826754in}{1.402640in}}%
\pgfpathcurveto{\pgfqpoint{2.818941in}{1.410453in}}{\pgfqpoint{2.808342in}{1.414844in}}{\pgfqpoint{2.797292in}{1.414844in}}%
\pgfpathcurveto{\pgfqpoint{2.786241in}{1.414844in}}{\pgfqpoint{2.775642in}{1.410453in}}{\pgfqpoint{2.767829in}{1.402640in}}%
\pgfpathcurveto{\pgfqpoint{2.760015in}{1.394826in}}{\pgfqpoint{2.755625in}{1.384227in}}{\pgfqpoint{2.755625in}{1.373177in}}%
\pgfpathcurveto{\pgfqpoint{2.755625in}{1.362127in}}{\pgfqpoint{2.760015in}{1.351528in}}{\pgfqpoint{2.767829in}{1.343714in}}%
\pgfpathcurveto{\pgfqpoint{2.775642in}{1.335901in}}{\pgfqpoint{2.786241in}{1.331510in}}{\pgfqpoint{2.797292in}{1.331510in}}%
\pgfpathclose%
\pgfusepath{stroke,fill}%
\end{pgfscope}%
\begin{pgfscope}%
\pgfpathrectangle{\pgfqpoint{0.481978in}{0.331635in}}{\pgfqpoint{4.960000in}{3.696000in}}%
\pgfusepath{clip}%
\pgfsetbuttcap%
\pgfsetroundjoin%
\definecolor{currentfill}{rgb}{0.631373,0.788235,0.956863}%
\pgfsetfillcolor{currentfill}%
\pgfsetlinewidth{0.481800pt}%
\definecolor{currentstroke}{rgb}{1.000000,1.000000,1.000000}%
\pgfsetstrokecolor{currentstroke}%
\pgfsetdash{}{0pt}%
\pgfpathmoveto{\pgfqpoint{2.640193in}{0.519607in}}%
\pgfpathcurveto{\pgfqpoint{2.651243in}{0.519607in}}{\pgfqpoint{2.661842in}{0.523998in}}{\pgfqpoint{2.669656in}{0.531811in}}%
\pgfpathcurveto{\pgfqpoint{2.677469in}{0.539625in}}{\pgfqpoint{2.681860in}{0.550224in}}{\pgfqpoint{2.681860in}{0.561274in}}%
\pgfpathcurveto{\pgfqpoint{2.681860in}{0.572324in}}{\pgfqpoint{2.677469in}{0.582923in}}{\pgfqpoint{2.669656in}{0.590737in}}%
\pgfpathcurveto{\pgfqpoint{2.661842in}{0.598550in}}{\pgfqpoint{2.651243in}{0.602941in}}{\pgfqpoint{2.640193in}{0.602941in}}%
\pgfpathcurveto{\pgfqpoint{2.629143in}{0.602941in}}{\pgfqpoint{2.618544in}{0.598550in}}{\pgfqpoint{2.610730in}{0.590737in}}%
\pgfpathcurveto{\pgfqpoint{2.602917in}{0.582923in}}{\pgfqpoint{2.598526in}{0.572324in}}{\pgfqpoint{2.598526in}{0.561274in}}%
\pgfpathcurveto{\pgfqpoint{2.598526in}{0.550224in}}{\pgfqpoint{2.602917in}{0.539625in}}{\pgfqpoint{2.610730in}{0.531811in}}%
\pgfpathcurveto{\pgfqpoint{2.618544in}{0.523998in}}{\pgfqpoint{2.629143in}{0.519607in}}{\pgfqpoint{2.640193in}{0.519607in}}%
\pgfpathclose%
\pgfusepath{stroke,fill}%
\end{pgfscope}%
\begin{pgfscope}%
\pgfpathrectangle{\pgfqpoint{0.481978in}{0.331635in}}{\pgfqpoint{4.960000in}{3.696000in}}%
\pgfusepath{clip}%
\pgfsetbuttcap%
\pgfsetroundjoin%
\definecolor{currentfill}{rgb}{0.631373,0.788235,0.956863}%
\pgfsetfillcolor{currentfill}%
\pgfsetlinewidth{0.481800pt}%
\definecolor{currentstroke}{rgb}{1.000000,1.000000,1.000000}%
\pgfsetstrokecolor{currentstroke}%
\pgfsetdash{}{0pt}%
\pgfpathmoveto{\pgfqpoint{3.001265in}{1.164889in}}%
\pgfpathcurveto{\pgfqpoint{3.012315in}{1.164889in}}{\pgfqpoint{3.022914in}{1.169279in}}{\pgfqpoint{3.030728in}{1.177092in}}%
\pgfpathcurveto{\pgfqpoint{3.038542in}{1.184906in}}{\pgfqpoint{3.042932in}{1.195505in}}{\pgfqpoint{3.042932in}{1.206555in}}%
\pgfpathcurveto{\pgfqpoint{3.042932in}{1.217605in}}{\pgfqpoint{3.038542in}{1.228204in}}{\pgfqpoint{3.030728in}{1.236018in}}%
\pgfpathcurveto{\pgfqpoint{3.022914in}{1.243832in}}{\pgfqpoint{3.012315in}{1.248222in}}{\pgfqpoint{3.001265in}{1.248222in}}%
\pgfpathcurveto{\pgfqpoint{2.990215in}{1.248222in}}{\pgfqpoint{2.979616in}{1.243832in}}{\pgfqpoint{2.971802in}{1.236018in}}%
\pgfpathcurveto{\pgfqpoint{2.963989in}{1.228204in}}{\pgfqpoint{2.959598in}{1.217605in}}{\pgfqpoint{2.959598in}{1.206555in}}%
\pgfpathcurveto{\pgfqpoint{2.959598in}{1.195505in}}{\pgfqpoint{2.963989in}{1.184906in}}{\pgfqpoint{2.971802in}{1.177092in}}%
\pgfpathcurveto{\pgfqpoint{2.979616in}{1.169279in}}{\pgfqpoint{2.990215in}{1.164889in}}{\pgfqpoint{3.001265in}{1.164889in}}%
\pgfpathclose%
\pgfusepath{stroke,fill}%
\end{pgfscope}%
\begin{pgfscope}%
\pgfpathrectangle{\pgfqpoint{0.481978in}{0.331635in}}{\pgfqpoint{4.960000in}{3.696000in}}%
\pgfusepath{clip}%
\pgfsetbuttcap%
\pgfsetroundjoin%
\definecolor{currentfill}{rgb}{0.631373,0.788235,0.956863}%
\pgfsetfillcolor{currentfill}%
\pgfsetlinewidth{0.481800pt}%
\definecolor{currentstroke}{rgb}{1.000000,1.000000,1.000000}%
\pgfsetstrokecolor{currentstroke}%
\pgfsetdash{}{0pt}%
\pgfpathmoveto{\pgfqpoint{2.921042in}{1.008513in}}%
\pgfpathcurveto{\pgfqpoint{2.932093in}{1.008513in}}{\pgfqpoint{2.942692in}{1.012904in}}{\pgfqpoint{2.950505in}{1.020717in}}%
\pgfpathcurveto{\pgfqpoint{2.958319in}{1.028531in}}{\pgfqpoint{2.962709in}{1.039130in}}{\pgfqpoint{2.962709in}{1.050180in}}%
\pgfpathcurveto{\pgfqpoint{2.962709in}{1.061230in}}{\pgfqpoint{2.958319in}{1.071829in}}{\pgfqpoint{2.950505in}{1.079643in}}%
\pgfpathcurveto{\pgfqpoint{2.942692in}{1.087456in}}{\pgfqpoint{2.932093in}{1.091847in}}{\pgfqpoint{2.921042in}{1.091847in}}%
\pgfpathcurveto{\pgfqpoint{2.909992in}{1.091847in}}{\pgfqpoint{2.899393in}{1.087456in}}{\pgfqpoint{2.891580in}{1.079643in}}%
\pgfpathcurveto{\pgfqpoint{2.883766in}{1.071829in}}{\pgfqpoint{2.879376in}{1.061230in}}{\pgfqpoint{2.879376in}{1.050180in}}%
\pgfpathcurveto{\pgfqpoint{2.879376in}{1.039130in}}{\pgfqpoint{2.883766in}{1.028531in}}{\pgfqpoint{2.891580in}{1.020717in}}%
\pgfpathcurveto{\pgfqpoint{2.899393in}{1.012904in}}{\pgfqpoint{2.909992in}{1.008513in}}{\pgfqpoint{2.921042in}{1.008513in}}%
\pgfpathclose%
\pgfusepath{stroke,fill}%
\end{pgfscope}%
\begin{pgfscope}%
\pgfpathrectangle{\pgfqpoint{0.481978in}{0.331635in}}{\pgfqpoint{4.960000in}{3.696000in}}%
\pgfusepath{clip}%
\pgfsetbuttcap%
\pgfsetroundjoin%
\definecolor{currentfill}{rgb}{0.631373,0.788235,0.956863}%
\pgfsetfillcolor{currentfill}%
\pgfsetlinewidth{0.481800pt}%
\definecolor{currentstroke}{rgb}{1.000000,1.000000,1.000000}%
\pgfsetstrokecolor{currentstroke}%
\pgfsetdash{}{0pt}%
\pgfpathmoveto{\pgfqpoint{2.393692in}{1.239028in}}%
\pgfpathcurveto{\pgfqpoint{2.404742in}{1.239028in}}{\pgfqpoint{2.415341in}{1.243419in}}{\pgfqpoint{2.423155in}{1.251232in}}%
\pgfpathcurveto{\pgfqpoint{2.430969in}{1.259046in}}{\pgfqpoint{2.435359in}{1.269645in}}{\pgfqpoint{2.435359in}{1.280695in}}%
\pgfpathcurveto{\pgfqpoint{2.435359in}{1.291745in}}{\pgfqpoint{2.430969in}{1.302344in}}{\pgfqpoint{2.423155in}{1.310158in}}%
\pgfpathcurveto{\pgfqpoint{2.415341in}{1.317971in}}{\pgfqpoint{2.404742in}{1.322362in}}{\pgfqpoint{2.393692in}{1.322362in}}%
\pgfpathcurveto{\pgfqpoint{2.382642in}{1.322362in}}{\pgfqpoint{2.372043in}{1.317971in}}{\pgfqpoint{2.364229in}{1.310158in}}%
\pgfpathcurveto{\pgfqpoint{2.356416in}{1.302344in}}{\pgfqpoint{2.352026in}{1.291745in}}{\pgfqpoint{2.352026in}{1.280695in}}%
\pgfpathcurveto{\pgfqpoint{2.352026in}{1.269645in}}{\pgfqpoint{2.356416in}{1.259046in}}{\pgfqpoint{2.364229in}{1.251232in}}%
\pgfpathcurveto{\pgfqpoint{2.372043in}{1.243419in}}{\pgfqpoint{2.382642in}{1.239028in}}{\pgfqpoint{2.393692in}{1.239028in}}%
\pgfpathclose%
\pgfusepath{stroke,fill}%
\end{pgfscope}%
\begin{pgfscope}%
\pgfpathrectangle{\pgfqpoint{0.481978in}{0.331635in}}{\pgfqpoint{4.960000in}{3.696000in}}%
\pgfusepath{clip}%
\pgfsetbuttcap%
\pgfsetroundjoin%
\definecolor{currentfill}{rgb}{0.631373,0.788235,0.956863}%
\pgfsetfillcolor{currentfill}%
\pgfsetlinewidth{0.481800pt}%
\definecolor{currentstroke}{rgb}{1.000000,1.000000,1.000000}%
\pgfsetstrokecolor{currentstroke}%
\pgfsetdash{}{0pt}%
\pgfpathmoveto{\pgfqpoint{3.120031in}{2.719885in}}%
\pgfpathcurveto{\pgfqpoint{3.131081in}{2.719885in}}{\pgfqpoint{3.141680in}{2.724275in}}{\pgfqpoint{3.149494in}{2.732089in}}%
\pgfpathcurveto{\pgfqpoint{3.157308in}{2.739902in}}{\pgfqpoint{3.161698in}{2.750501in}}{\pgfqpoint{3.161698in}{2.761551in}}%
\pgfpathcurveto{\pgfqpoint{3.161698in}{2.772602in}}{\pgfqpoint{3.157308in}{2.783201in}}{\pgfqpoint{3.149494in}{2.791014in}}%
\pgfpathcurveto{\pgfqpoint{3.141680in}{2.798828in}}{\pgfqpoint{3.131081in}{2.803218in}}{\pgfqpoint{3.120031in}{2.803218in}}%
\pgfpathcurveto{\pgfqpoint{3.108981in}{2.803218in}}{\pgfqpoint{3.098382in}{2.798828in}}{\pgfqpoint{3.090568in}{2.791014in}}%
\pgfpathcurveto{\pgfqpoint{3.082755in}{2.783201in}}{\pgfqpoint{3.078364in}{2.772602in}}{\pgfqpoint{3.078364in}{2.761551in}}%
\pgfpathcurveto{\pgfqpoint{3.078364in}{2.750501in}}{\pgfqpoint{3.082755in}{2.739902in}}{\pgfqpoint{3.090568in}{2.732089in}}%
\pgfpathcurveto{\pgfqpoint{3.098382in}{2.724275in}}{\pgfqpoint{3.108981in}{2.719885in}}{\pgfqpoint{3.120031in}{2.719885in}}%
\pgfpathclose%
\pgfusepath{stroke,fill}%
\end{pgfscope}%
\begin{pgfscope}%
\pgfpathrectangle{\pgfqpoint{0.481978in}{0.331635in}}{\pgfqpoint{4.960000in}{3.696000in}}%
\pgfusepath{clip}%
\pgfsetbuttcap%
\pgfsetroundjoin%
\definecolor{currentfill}{rgb}{0.631373,0.788235,0.956863}%
\pgfsetfillcolor{currentfill}%
\pgfsetlinewidth{0.481800pt}%
\definecolor{currentstroke}{rgb}{1.000000,1.000000,1.000000}%
\pgfsetstrokecolor{currentstroke}%
\pgfsetdash{}{0pt}%
\pgfpathmoveto{\pgfqpoint{4.510528in}{2.365559in}}%
\pgfpathcurveto{\pgfqpoint{4.521578in}{2.365559in}}{\pgfqpoint{4.532177in}{2.369949in}}{\pgfqpoint{4.539991in}{2.377763in}}%
\pgfpathcurveto{\pgfqpoint{4.547804in}{2.385577in}}{\pgfqpoint{4.552195in}{2.396176in}}{\pgfqpoint{4.552195in}{2.407226in}}%
\pgfpathcurveto{\pgfqpoint{4.552195in}{2.418276in}}{\pgfqpoint{4.547804in}{2.428875in}}{\pgfqpoint{4.539991in}{2.436689in}}%
\pgfpathcurveto{\pgfqpoint{4.532177in}{2.444502in}}{\pgfqpoint{4.521578in}{2.448892in}}{\pgfqpoint{4.510528in}{2.448892in}}%
\pgfpathcurveto{\pgfqpoint{4.499478in}{2.448892in}}{\pgfqpoint{4.488879in}{2.444502in}}{\pgfqpoint{4.481065in}{2.436689in}}%
\pgfpathcurveto{\pgfqpoint{4.473251in}{2.428875in}}{\pgfqpoint{4.468861in}{2.418276in}}{\pgfqpoint{4.468861in}{2.407226in}}%
\pgfpathcurveto{\pgfqpoint{4.468861in}{2.396176in}}{\pgfqpoint{4.473251in}{2.385577in}}{\pgfqpoint{4.481065in}{2.377763in}}%
\pgfpathcurveto{\pgfqpoint{4.488879in}{2.369949in}}{\pgfqpoint{4.499478in}{2.365559in}}{\pgfqpoint{4.510528in}{2.365559in}}%
\pgfpathclose%
\pgfusepath{stroke,fill}%
\end{pgfscope}%
\begin{pgfscope}%
\pgfpathrectangle{\pgfqpoint{0.481978in}{0.331635in}}{\pgfqpoint{4.960000in}{3.696000in}}%
\pgfusepath{clip}%
\pgfsetbuttcap%
\pgfsetroundjoin%
\definecolor{currentfill}{rgb}{0.631373,0.788235,0.956863}%
\pgfsetfillcolor{currentfill}%
\pgfsetlinewidth{0.481800pt}%
\definecolor{currentstroke}{rgb}{1.000000,1.000000,1.000000}%
\pgfsetstrokecolor{currentstroke}%
\pgfsetdash{}{0pt}%
\pgfpathmoveto{\pgfqpoint{3.942534in}{3.211603in}}%
\pgfpathcurveto{\pgfqpoint{3.953584in}{3.211603in}}{\pgfqpoint{3.964183in}{3.215994in}}{\pgfqpoint{3.971997in}{3.223807in}}%
\pgfpathcurveto{\pgfqpoint{3.979810in}{3.231621in}}{\pgfqpoint{3.984200in}{3.242220in}}{\pgfqpoint{3.984200in}{3.253270in}}%
\pgfpathcurveto{\pgfqpoint{3.984200in}{3.264320in}}{\pgfqpoint{3.979810in}{3.274919in}}{\pgfqpoint{3.971997in}{3.282733in}}%
\pgfpathcurveto{\pgfqpoint{3.964183in}{3.290546in}}{\pgfqpoint{3.953584in}{3.294937in}}{\pgfqpoint{3.942534in}{3.294937in}}%
\pgfpathcurveto{\pgfqpoint{3.931484in}{3.294937in}}{\pgfqpoint{3.920885in}{3.290546in}}{\pgfqpoint{3.913071in}{3.282733in}}%
\pgfpathcurveto{\pgfqpoint{3.905257in}{3.274919in}}{\pgfqpoint{3.900867in}{3.264320in}}{\pgfqpoint{3.900867in}{3.253270in}}%
\pgfpathcurveto{\pgfqpoint{3.900867in}{3.242220in}}{\pgfqpoint{3.905257in}{3.231621in}}{\pgfqpoint{3.913071in}{3.223807in}}%
\pgfpathcurveto{\pgfqpoint{3.920885in}{3.215994in}}{\pgfqpoint{3.931484in}{3.211603in}}{\pgfqpoint{3.942534in}{3.211603in}}%
\pgfpathclose%
\pgfusepath{stroke,fill}%
\end{pgfscope}%
\begin{pgfscope}%
\pgfpathrectangle{\pgfqpoint{0.481978in}{0.331635in}}{\pgfqpoint{4.960000in}{3.696000in}}%
\pgfusepath{clip}%
\pgfsetbuttcap%
\pgfsetroundjoin%
\definecolor{currentfill}{rgb}{0.631373,0.788235,0.956863}%
\pgfsetfillcolor{currentfill}%
\pgfsetlinewidth{0.481800pt}%
\definecolor{currentstroke}{rgb}{1.000000,1.000000,1.000000}%
\pgfsetstrokecolor{currentstroke}%
\pgfsetdash{}{0pt}%
\pgfpathmoveto{\pgfqpoint{4.138057in}{3.232985in}}%
\pgfpathcurveto{\pgfqpoint{4.149108in}{3.232985in}}{\pgfqpoint{4.159707in}{3.237375in}}{\pgfqpoint{4.167520in}{3.245189in}}%
\pgfpathcurveto{\pgfqpoint{4.175334in}{3.253002in}}{\pgfqpoint{4.179724in}{3.263601in}}{\pgfqpoint{4.179724in}{3.274652in}}%
\pgfpathcurveto{\pgfqpoint{4.179724in}{3.285702in}}{\pgfqpoint{4.175334in}{3.296301in}}{\pgfqpoint{4.167520in}{3.304114in}}%
\pgfpathcurveto{\pgfqpoint{4.159707in}{3.311928in}}{\pgfqpoint{4.149108in}{3.316318in}}{\pgfqpoint{4.138057in}{3.316318in}}%
\pgfpathcurveto{\pgfqpoint{4.127007in}{3.316318in}}{\pgfqpoint{4.116408in}{3.311928in}}{\pgfqpoint{4.108595in}{3.304114in}}%
\pgfpathcurveto{\pgfqpoint{4.100781in}{3.296301in}}{\pgfqpoint{4.096391in}{3.285702in}}{\pgfqpoint{4.096391in}{3.274652in}}%
\pgfpathcurveto{\pgfqpoint{4.096391in}{3.263601in}}{\pgfqpoint{4.100781in}{3.253002in}}{\pgfqpoint{4.108595in}{3.245189in}}%
\pgfpathcurveto{\pgfqpoint{4.116408in}{3.237375in}}{\pgfqpoint{4.127007in}{3.232985in}}{\pgfqpoint{4.138057in}{3.232985in}}%
\pgfpathclose%
\pgfusepath{stroke,fill}%
\end{pgfscope}%
\begin{pgfscope}%
\pgfpathrectangle{\pgfqpoint{0.481978in}{0.331635in}}{\pgfqpoint{4.960000in}{3.696000in}}%
\pgfusepath{clip}%
\pgfsetbuttcap%
\pgfsetroundjoin%
\definecolor{currentfill}{rgb}{0.631373,0.788235,0.956863}%
\pgfsetfillcolor{currentfill}%
\pgfsetlinewidth{0.481800pt}%
\definecolor{currentstroke}{rgb}{1.000000,1.000000,1.000000}%
\pgfsetstrokecolor{currentstroke}%
\pgfsetdash{}{0pt}%
\pgfpathmoveto{\pgfqpoint{2.830135in}{0.951790in}}%
\pgfpathcurveto{\pgfqpoint{2.841185in}{0.951790in}}{\pgfqpoint{2.851784in}{0.956180in}}{\pgfqpoint{2.859598in}{0.963994in}}%
\pgfpathcurveto{\pgfqpoint{2.867411in}{0.971807in}}{\pgfqpoint{2.871802in}{0.982406in}}{\pgfqpoint{2.871802in}{0.993456in}}%
\pgfpathcurveto{\pgfqpoint{2.871802in}{1.004507in}}{\pgfqpoint{2.867411in}{1.015106in}}{\pgfqpoint{2.859598in}{1.022919in}}%
\pgfpathcurveto{\pgfqpoint{2.851784in}{1.030733in}}{\pgfqpoint{2.841185in}{1.035123in}}{\pgfqpoint{2.830135in}{1.035123in}}%
\pgfpathcurveto{\pgfqpoint{2.819085in}{1.035123in}}{\pgfqpoint{2.808486in}{1.030733in}}{\pgfqpoint{2.800672in}{1.022919in}}%
\pgfpathcurveto{\pgfqpoint{2.792859in}{1.015106in}}{\pgfqpoint{2.788468in}{1.004507in}}{\pgfqpoint{2.788468in}{0.993456in}}%
\pgfpathcurveto{\pgfqpoint{2.788468in}{0.982406in}}{\pgfqpoint{2.792859in}{0.971807in}}{\pgfqpoint{2.800672in}{0.963994in}}%
\pgfpathcurveto{\pgfqpoint{2.808486in}{0.956180in}}{\pgfqpoint{2.819085in}{0.951790in}}{\pgfqpoint{2.830135in}{0.951790in}}%
\pgfpathclose%
\pgfusepath{stroke,fill}%
\end{pgfscope}%
\begin{pgfscope}%
\pgfpathrectangle{\pgfqpoint{0.481978in}{0.331635in}}{\pgfqpoint{4.960000in}{3.696000in}}%
\pgfusepath{clip}%
\pgfsetbuttcap%
\pgfsetroundjoin%
\definecolor{currentfill}{rgb}{0.631373,0.788235,0.956863}%
\pgfsetfillcolor{currentfill}%
\pgfsetlinewidth{0.481800pt}%
\definecolor{currentstroke}{rgb}{1.000000,1.000000,1.000000}%
\pgfsetstrokecolor{currentstroke}%
\pgfsetdash{}{0pt}%
\pgfpathmoveto{\pgfqpoint{2.773539in}{1.732697in}}%
\pgfpathcurveto{\pgfqpoint{2.784589in}{1.732697in}}{\pgfqpoint{2.795188in}{1.737087in}}{\pgfqpoint{2.803002in}{1.744901in}}%
\pgfpathcurveto{\pgfqpoint{2.810816in}{1.752714in}}{\pgfqpoint{2.815206in}{1.763313in}}{\pgfqpoint{2.815206in}{1.774363in}}%
\pgfpathcurveto{\pgfqpoint{2.815206in}{1.785413in}}{\pgfqpoint{2.810816in}{1.796012in}}{\pgfqpoint{2.803002in}{1.803826in}}%
\pgfpathcurveto{\pgfqpoint{2.795188in}{1.811640in}}{\pgfqpoint{2.784589in}{1.816030in}}{\pgfqpoint{2.773539in}{1.816030in}}%
\pgfpathcurveto{\pgfqpoint{2.762489in}{1.816030in}}{\pgfqpoint{2.751890in}{1.811640in}}{\pgfqpoint{2.744076in}{1.803826in}}%
\pgfpathcurveto{\pgfqpoint{2.736263in}{1.796012in}}{\pgfqpoint{2.731873in}{1.785413in}}{\pgfqpoint{2.731873in}{1.774363in}}%
\pgfpathcurveto{\pgfqpoint{2.731873in}{1.763313in}}{\pgfqpoint{2.736263in}{1.752714in}}{\pgfqpoint{2.744076in}{1.744901in}}%
\pgfpathcurveto{\pgfqpoint{2.751890in}{1.737087in}}{\pgfqpoint{2.762489in}{1.732697in}}{\pgfqpoint{2.773539in}{1.732697in}}%
\pgfpathclose%
\pgfusepath{stroke,fill}%
\end{pgfscope}%
\begin{pgfscope}%
\pgfpathrectangle{\pgfqpoint{0.481978in}{0.331635in}}{\pgfqpoint{4.960000in}{3.696000in}}%
\pgfusepath{clip}%
\pgfsetbuttcap%
\pgfsetroundjoin%
\definecolor{currentfill}{rgb}{0.631373,0.788235,0.956863}%
\pgfsetfillcolor{currentfill}%
\pgfsetlinewidth{0.481800pt}%
\definecolor{currentstroke}{rgb}{1.000000,1.000000,1.000000}%
\pgfsetstrokecolor{currentstroke}%
\pgfsetdash{}{0pt}%
\pgfpathmoveto{\pgfqpoint{3.305756in}{1.492066in}}%
\pgfpathcurveto{\pgfqpoint{3.316806in}{1.492066in}}{\pgfqpoint{3.327405in}{1.496456in}}{\pgfqpoint{3.335219in}{1.504270in}}%
\pgfpathcurveto{\pgfqpoint{3.343032in}{1.512083in}}{\pgfqpoint{3.347423in}{1.522682in}}{\pgfqpoint{3.347423in}{1.533732in}}%
\pgfpathcurveto{\pgfqpoint{3.347423in}{1.544783in}}{\pgfqpoint{3.343032in}{1.555382in}}{\pgfqpoint{3.335219in}{1.563195in}}%
\pgfpathcurveto{\pgfqpoint{3.327405in}{1.571009in}}{\pgfqpoint{3.316806in}{1.575399in}}{\pgfqpoint{3.305756in}{1.575399in}}%
\pgfpathcurveto{\pgfqpoint{3.294706in}{1.575399in}}{\pgfqpoint{3.284107in}{1.571009in}}{\pgfqpoint{3.276293in}{1.563195in}}%
\pgfpathcurveto{\pgfqpoint{3.268480in}{1.555382in}}{\pgfqpoint{3.264089in}{1.544783in}}{\pgfqpoint{3.264089in}{1.533732in}}%
\pgfpathcurveto{\pgfqpoint{3.264089in}{1.522682in}}{\pgfqpoint{3.268480in}{1.512083in}}{\pgfqpoint{3.276293in}{1.504270in}}%
\pgfpathcurveto{\pgfqpoint{3.284107in}{1.496456in}}{\pgfqpoint{3.294706in}{1.492066in}}{\pgfqpoint{3.305756in}{1.492066in}}%
\pgfpathclose%
\pgfusepath{stroke,fill}%
\end{pgfscope}%
\begin{pgfscope}%
\pgfpathrectangle{\pgfqpoint{0.481978in}{0.331635in}}{\pgfqpoint{4.960000in}{3.696000in}}%
\pgfusepath{clip}%
\pgfsetbuttcap%
\pgfsetroundjoin%
\definecolor{currentfill}{rgb}{0.631373,0.788235,0.956863}%
\pgfsetfillcolor{currentfill}%
\pgfsetlinewidth{0.481800pt}%
\definecolor{currentstroke}{rgb}{1.000000,1.000000,1.000000}%
\pgfsetstrokecolor{currentstroke}%
\pgfsetdash{}{0pt}%
\pgfpathmoveto{\pgfqpoint{3.729492in}{2.067915in}}%
\pgfpathcurveto{\pgfqpoint{3.740542in}{2.067915in}}{\pgfqpoint{3.751141in}{2.072305in}}{\pgfqpoint{3.758954in}{2.080119in}}%
\pgfpathcurveto{\pgfqpoint{3.766768in}{2.087933in}}{\pgfqpoint{3.771158in}{2.098532in}}{\pgfqpoint{3.771158in}{2.109582in}}%
\pgfpathcurveto{\pgfqpoint{3.771158in}{2.120632in}}{\pgfqpoint{3.766768in}{2.131231in}}{\pgfqpoint{3.758954in}{2.139045in}}%
\pgfpathcurveto{\pgfqpoint{3.751141in}{2.146858in}}{\pgfqpoint{3.740542in}{2.151248in}}{\pgfqpoint{3.729492in}{2.151248in}}%
\pgfpathcurveto{\pgfqpoint{3.718441in}{2.151248in}}{\pgfqpoint{3.707842in}{2.146858in}}{\pgfqpoint{3.700029in}{2.139045in}}%
\pgfpathcurveto{\pgfqpoint{3.692215in}{2.131231in}}{\pgfqpoint{3.687825in}{2.120632in}}{\pgfqpoint{3.687825in}{2.109582in}}%
\pgfpathcurveto{\pgfqpoint{3.687825in}{2.098532in}}{\pgfqpoint{3.692215in}{2.087933in}}{\pgfqpoint{3.700029in}{2.080119in}}%
\pgfpathcurveto{\pgfqpoint{3.707842in}{2.072305in}}{\pgfqpoint{3.718441in}{2.067915in}}{\pgfqpoint{3.729492in}{2.067915in}}%
\pgfpathclose%
\pgfusepath{stroke,fill}%
\end{pgfscope}%
\begin{pgfscope}%
\pgfpathrectangle{\pgfqpoint{0.481978in}{0.331635in}}{\pgfqpoint{4.960000in}{3.696000in}}%
\pgfusepath{clip}%
\pgfsetbuttcap%
\pgfsetroundjoin%
\definecolor{currentfill}{rgb}{0.631373,0.788235,0.956863}%
\pgfsetfillcolor{currentfill}%
\pgfsetlinewidth{0.481800pt}%
\definecolor{currentstroke}{rgb}{1.000000,1.000000,1.000000}%
\pgfsetstrokecolor{currentstroke}%
\pgfsetdash{}{0pt}%
\pgfpathmoveto{\pgfqpoint{3.105792in}{0.645253in}}%
\pgfpathcurveto{\pgfqpoint{3.116842in}{0.645253in}}{\pgfqpoint{3.127441in}{0.649643in}}{\pgfqpoint{3.135254in}{0.657457in}}%
\pgfpathcurveto{\pgfqpoint{3.143068in}{0.665270in}}{\pgfqpoint{3.147458in}{0.675869in}}{\pgfqpoint{3.147458in}{0.686919in}}%
\pgfpathcurveto{\pgfqpoint{3.147458in}{0.697970in}}{\pgfqpoint{3.143068in}{0.708569in}}{\pgfqpoint{3.135254in}{0.716382in}}%
\pgfpathcurveto{\pgfqpoint{3.127441in}{0.724196in}}{\pgfqpoint{3.116842in}{0.728586in}}{\pgfqpoint{3.105792in}{0.728586in}}%
\pgfpathcurveto{\pgfqpoint{3.094742in}{0.728586in}}{\pgfqpoint{3.084142in}{0.724196in}}{\pgfqpoint{3.076329in}{0.716382in}}%
\pgfpathcurveto{\pgfqpoint{3.068515in}{0.708569in}}{\pgfqpoint{3.064125in}{0.697970in}}{\pgfqpoint{3.064125in}{0.686919in}}%
\pgfpathcurveto{\pgfqpoint{3.064125in}{0.675869in}}{\pgfqpoint{3.068515in}{0.665270in}}{\pgfqpoint{3.076329in}{0.657457in}}%
\pgfpathcurveto{\pgfqpoint{3.084142in}{0.649643in}}{\pgfqpoint{3.094742in}{0.645253in}}{\pgfqpoint{3.105792in}{0.645253in}}%
\pgfpathclose%
\pgfusepath{stroke,fill}%
\end{pgfscope}%
\begin{pgfscope}%
\pgfpathrectangle{\pgfqpoint{0.481978in}{0.331635in}}{\pgfqpoint{4.960000in}{3.696000in}}%
\pgfusepath{clip}%
\pgfsetbuttcap%
\pgfsetroundjoin%
\definecolor{currentfill}{rgb}{0.631373,0.788235,0.956863}%
\pgfsetfillcolor{currentfill}%
\pgfsetlinewidth{0.481800pt}%
\definecolor{currentstroke}{rgb}{1.000000,1.000000,1.000000}%
\pgfsetstrokecolor{currentstroke}%
\pgfsetdash{}{0pt}%
\pgfpathmoveto{\pgfqpoint{3.126698in}{1.718317in}}%
\pgfpathcurveto{\pgfqpoint{3.137749in}{1.718317in}}{\pgfqpoint{3.148348in}{1.722707in}}{\pgfqpoint{3.156161in}{1.730520in}}%
\pgfpathcurveto{\pgfqpoint{3.163975in}{1.738334in}}{\pgfqpoint{3.168365in}{1.748933in}}{\pgfqpoint{3.168365in}{1.759983in}}%
\pgfpathcurveto{\pgfqpoint{3.168365in}{1.771033in}}{\pgfqpoint{3.163975in}{1.781632in}}{\pgfqpoint{3.156161in}{1.789446in}}%
\pgfpathcurveto{\pgfqpoint{3.148348in}{1.797260in}}{\pgfqpoint{3.137749in}{1.801650in}}{\pgfqpoint{3.126698in}{1.801650in}}%
\pgfpathcurveto{\pgfqpoint{3.115648in}{1.801650in}}{\pgfqpoint{3.105049in}{1.797260in}}{\pgfqpoint{3.097236in}{1.789446in}}%
\pgfpathcurveto{\pgfqpoint{3.089422in}{1.781632in}}{\pgfqpoint{3.085032in}{1.771033in}}{\pgfqpoint{3.085032in}{1.759983in}}%
\pgfpathcurveto{\pgfqpoint{3.085032in}{1.748933in}}{\pgfqpoint{3.089422in}{1.738334in}}{\pgfqpoint{3.097236in}{1.730520in}}%
\pgfpathcurveto{\pgfqpoint{3.105049in}{1.722707in}}{\pgfqpoint{3.115648in}{1.718317in}}{\pgfqpoint{3.126698in}{1.718317in}}%
\pgfpathclose%
\pgfusepath{stroke,fill}%
\end{pgfscope}%
\begin{pgfscope}%
\pgfpathrectangle{\pgfqpoint{0.481978in}{0.331635in}}{\pgfqpoint{4.960000in}{3.696000in}}%
\pgfusepath{clip}%
\pgfsetbuttcap%
\pgfsetroundjoin%
\definecolor{currentfill}{rgb}{0.631373,0.788235,0.956863}%
\pgfsetfillcolor{currentfill}%
\pgfsetlinewidth{0.481800pt}%
\definecolor{currentstroke}{rgb}{1.000000,1.000000,1.000000}%
\pgfsetstrokecolor{currentstroke}%
\pgfsetdash{}{0pt}%
\pgfpathmoveto{\pgfqpoint{3.394486in}{1.129447in}}%
\pgfpathcurveto{\pgfqpoint{3.405536in}{1.129447in}}{\pgfqpoint{3.416135in}{1.133838in}}{\pgfqpoint{3.423949in}{1.141651in}}%
\pgfpathcurveto{\pgfqpoint{3.431762in}{1.149465in}}{\pgfqpoint{3.436153in}{1.160064in}}{\pgfqpoint{3.436153in}{1.171114in}}%
\pgfpathcurveto{\pgfqpoint{3.436153in}{1.182164in}}{\pgfqpoint{3.431762in}{1.192763in}}{\pgfqpoint{3.423949in}{1.200577in}}%
\pgfpathcurveto{\pgfqpoint{3.416135in}{1.208391in}}{\pgfqpoint{3.405536in}{1.212781in}}{\pgfqpoint{3.394486in}{1.212781in}}%
\pgfpathcurveto{\pgfqpoint{3.383436in}{1.212781in}}{\pgfqpoint{3.372837in}{1.208391in}}{\pgfqpoint{3.365023in}{1.200577in}}%
\pgfpathcurveto{\pgfqpoint{3.357210in}{1.192763in}}{\pgfqpoint{3.352819in}{1.182164in}}{\pgfqpoint{3.352819in}{1.171114in}}%
\pgfpathcurveto{\pgfqpoint{3.352819in}{1.160064in}}{\pgfqpoint{3.357210in}{1.149465in}}{\pgfqpoint{3.365023in}{1.141651in}}%
\pgfpathcurveto{\pgfqpoint{3.372837in}{1.133838in}}{\pgfqpoint{3.383436in}{1.129447in}}{\pgfqpoint{3.394486in}{1.129447in}}%
\pgfpathclose%
\pgfusepath{stroke,fill}%
\end{pgfscope}%
\begin{pgfscope}%
\pgfpathrectangle{\pgfqpoint{0.481978in}{0.331635in}}{\pgfqpoint{4.960000in}{3.696000in}}%
\pgfusepath{clip}%
\pgfsetbuttcap%
\pgfsetroundjoin%
\definecolor{currentfill}{rgb}{0.631373,0.788235,0.956863}%
\pgfsetfillcolor{currentfill}%
\pgfsetlinewidth{0.481800pt}%
\definecolor{currentstroke}{rgb}{1.000000,1.000000,1.000000}%
\pgfsetstrokecolor{currentstroke}%
\pgfsetdash{}{0pt}%
\pgfpathmoveto{\pgfqpoint{3.272709in}{2.051417in}}%
\pgfpathcurveto{\pgfqpoint{3.283759in}{2.051417in}}{\pgfqpoint{3.294358in}{2.055807in}}{\pgfqpoint{3.302172in}{2.063620in}}%
\pgfpathcurveto{\pgfqpoint{3.309986in}{2.071434in}}{\pgfqpoint{3.314376in}{2.082033in}}{\pgfqpoint{3.314376in}{2.093083in}}%
\pgfpathcurveto{\pgfqpoint{3.314376in}{2.104133in}}{\pgfqpoint{3.309986in}{2.114732in}}{\pgfqpoint{3.302172in}{2.122546in}}%
\pgfpathcurveto{\pgfqpoint{3.294358in}{2.130360in}}{\pgfqpoint{3.283759in}{2.134750in}}{\pgfqpoint{3.272709in}{2.134750in}}%
\pgfpathcurveto{\pgfqpoint{3.261659in}{2.134750in}}{\pgfqpoint{3.251060in}{2.130360in}}{\pgfqpoint{3.243246in}{2.122546in}}%
\pgfpathcurveto{\pgfqpoint{3.235433in}{2.114732in}}{\pgfqpoint{3.231043in}{2.104133in}}{\pgfqpoint{3.231043in}{2.093083in}}%
\pgfpathcurveto{\pgfqpoint{3.231043in}{2.082033in}}{\pgfqpoint{3.235433in}{2.071434in}}{\pgfqpoint{3.243246in}{2.063620in}}%
\pgfpathcurveto{\pgfqpoint{3.251060in}{2.055807in}}{\pgfqpoint{3.261659in}{2.051417in}}{\pgfqpoint{3.272709in}{2.051417in}}%
\pgfpathclose%
\pgfusepath{stroke,fill}%
\end{pgfscope}%
\begin{pgfscope}%
\pgfpathrectangle{\pgfqpoint{0.481978in}{0.331635in}}{\pgfqpoint{4.960000in}{3.696000in}}%
\pgfusepath{clip}%
\pgfsetbuttcap%
\pgfsetroundjoin%
\definecolor{currentfill}{rgb}{0.631373,0.788235,0.956863}%
\pgfsetfillcolor{currentfill}%
\pgfsetlinewidth{0.481800pt}%
\definecolor{currentstroke}{rgb}{1.000000,1.000000,1.000000}%
\pgfsetstrokecolor{currentstroke}%
\pgfsetdash{}{0pt}%
\pgfpathmoveto{\pgfqpoint{1.830270in}{0.951695in}}%
\pgfpathcurveto{\pgfqpoint{1.841320in}{0.951695in}}{\pgfqpoint{1.851919in}{0.956086in}}{\pgfqpoint{1.859733in}{0.963899in}}%
\pgfpathcurveto{\pgfqpoint{1.867547in}{0.971713in}}{\pgfqpoint{1.871937in}{0.982312in}}{\pgfqpoint{1.871937in}{0.993362in}}%
\pgfpathcurveto{\pgfqpoint{1.871937in}{1.004412in}}{\pgfqpoint{1.867547in}{1.015011in}}{\pgfqpoint{1.859733in}{1.022825in}}%
\pgfpathcurveto{\pgfqpoint{1.851919in}{1.030638in}}{\pgfqpoint{1.841320in}{1.035029in}}{\pgfqpoint{1.830270in}{1.035029in}}%
\pgfpathcurveto{\pgfqpoint{1.819220in}{1.035029in}}{\pgfqpoint{1.808621in}{1.030638in}}{\pgfqpoint{1.800807in}{1.022825in}}%
\pgfpathcurveto{\pgfqpoint{1.792994in}{1.015011in}}{\pgfqpoint{1.788604in}{1.004412in}}{\pgfqpoint{1.788604in}{0.993362in}}%
\pgfpathcurveto{\pgfqpoint{1.788604in}{0.982312in}}{\pgfqpoint{1.792994in}{0.971713in}}{\pgfqpoint{1.800807in}{0.963899in}}%
\pgfpathcurveto{\pgfqpoint{1.808621in}{0.956086in}}{\pgfqpoint{1.819220in}{0.951695in}}{\pgfqpoint{1.830270in}{0.951695in}}%
\pgfpathclose%
\pgfusepath{stroke,fill}%
\end{pgfscope}%
\begin{pgfscope}%
\pgfpathrectangle{\pgfqpoint{0.481978in}{0.331635in}}{\pgfqpoint{4.960000in}{3.696000in}}%
\pgfusepath{clip}%
\pgfsetbuttcap%
\pgfsetroundjoin%
\definecolor{currentfill}{rgb}{0.631373,0.788235,0.956863}%
\pgfsetfillcolor{currentfill}%
\pgfsetlinewidth{0.481800pt}%
\definecolor{currentstroke}{rgb}{1.000000,1.000000,1.000000}%
\pgfsetstrokecolor{currentstroke}%
\pgfsetdash{}{0pt}%
\pgfpathmoveto{\pgfqpoint{2.450737in}{0.948434in}}%
\pgfpathcurveto{\pgfqpoint{2.461787in}{0.948434in}}{\pgfqpoint{2.472386in}{0.952824in}}{\pgfqpoint{2.480200in}{0.960638in}}%
\pgfpathcurveto{\pgfqpoint{2.488013in}{0.968452in}}{\pgfqpoint{2.492404in}{0.979051in}}{\pgfqpoint{2.492404in}{0.990101in}}%
\pgfpathcurveto{\pgfqpoint{2.492404in}{1.001151in}}{\pgfqpoint{2.488013in}{1.011750in}}{\pgfqpoint{2.480200in}{1.019564in}}%
\pgfpathcurveto{\pgfqpoint{2.472386in}{1.027377in}}{\pgfqpoint{2.461787in}{1.031767in}}{\pgfqpoint{2.450737in}{1.031767in}}%
\pgfpathcurveto{\pgfqpoint{2.439687in}{1.031767in}}{\pgfqpoint{2.429088in}{1.027377in}}{\pgfqpoint{2.421274in}{1.019564in}}%
\pgfpathcurveto{\pgfqpoint{2.413460in}{1.011750in}}{\pgfqpoint{2.409070in}{1.001151in}}{\pgfqpoint{2.409070in}{0.990101in}}%
\pgfpathcurveto{\pgfqpoint{2.409070in}{0.979051in}}{\pgfqpoint{2.413460in}{0.968452in}}{\pgfqpoint{2.421274in}{0.960638in}}%
\pgfpathcurveto{\pgfqpoint{2.429088in}{0.952824in}}{\pgfqpoint{2.439687in}{0.948434in}}{\pgfqpoint{2.450737in}{0.948434in}}%
\pgfpathclose%
\pgfusepath{stroke,fill}%
\end{pgfscope}%
\begin{pgfscope}%
\pgfpathrectangle{\pgfqpoint{0.481978in}{0.331635in}}{\pgfqpoint{4.960000in}{3.696000in}}%
\pgfusepath{clip}%
\pgfsetbuttcap%
\pgfsetroundjoin%
\definecolor{currentfill}{rgb}{0.631373,0.788235,0.956863}%
\pgfsetfillcolor{currentfill}%
\pgfsetlinewidth{0.481800pt}%
\definecolor{currentstroke}{rgb}{1.000000,1.000000,1.000000}%
\pgfsetstrokecolor{currentstroke}%
\pgfsetdash{}{0pt}%
\pgfpathmoveto{\pgfqpoint{3.452671in}{0.634619in}}%
\pgfpathcurveto{\pgfqpoint{3.463721in}{0.634619in}}{\pgfqpoint{3.474320in}{0.639009in}}{\pgfqpoint{3.482134in}{0.646823in}}%
\pgfpathcurveto{\pgfqpoint{3.489948in}{0.654636in}}{\pgfqpoint{3.494338in}{0.665235in}}{\pgfqpoint{3.494338in}{0.676285in}}%
\pgfpathcurveto{\pgfqpoint{3.494338in}{0.687336in}}{\pgfqpoint{3.489948in}{0.697935in}}{\pgfqpoint{3.482134in}{0.705748in}}%
\pgfpathcurveto{\pgfqpoint{3.474320in}{0.713562in}}{\pgfqpoint{3.463721in}{0.717952in}}{\pgfqpoint{3.452671in}{0.717952in}}%
\pgfpathcurveto{\pgfqpoint{3.441621in}{0.717952in}}{\pgfqpoint{3.431022in}{0.713562in}}{\pgfqpoint{3.423208in}{0.705748in}}%
\pgfpathcurveto{\pgfqpoint{3.415395in}{0.697935in}}{\pgfqpoint{3.411005in}{0.687336in}}{\pgfqpoint{3.411005in}{0.676285in}}%
\pgfpathcurveto{\pgfqpoint{3.411005in}{0.665235in}}{\pgfqpoint{3.415395in}{0.654636in}}{\pgfqpoint{3.423208in}{0.646823in}}%
\pgfpathcurveto{\pgfqpoint{3.431022in}{0.639009in}}{\pgfqpoint{3.441621in}{0.634619in}}{\pgfqpoint{3.452671in}{0.634619in}}%
\pgfpathclose%
\pgfusepath{stroke,fill}%
\end{pgfscope}%
\begin{pgfscope}%
\pgfpathrectangle{\pgfqpoint{0.481978in}{0.331635in}}{\pgfqpoint{4.960000in}{3.696000in}}%
\pgfusepath{clip}%
\pgfsetbuttcap%
\pgfsetroundjoin%
\definecolor{currentfill}{rgb}{0.631373,0.788235,0.956863}%
\pgfsetfillcolor{currentfill}%
\pgfsetlinewidth{0.481800pt}%
\definecolor{currentstroke}{rgb}{1.000000,1.000000,1.000000}%
\pgfsetstrokecolor{currentstroke}%
\pgfsetdash{}{0pt}%
\pgfpathmoveto{\pgfqpoint{2.908589in}{0.611543in}}%
\pgfpathcurveto{\pgfqpoint{2.919639in}{0.611543in}}{\pgfqpoint{2.930238in}{0.615933in}}{\pgfqpoint{2.938052in}{0.623746in}}%
\pgfpathcurveto{\pgfqpoint{2.945866in}{0.631560in}}{\pgfqpoint{2.950256in}{0.642159in}}{\pgfqpoint{2.950256in}{0.653209in}}%
\pgfpathcurveto{\pgfqpoint{2.950256in}{0.664259in}}{\pgfqpoint{2.945866in}{0.674858in}}{\pgfqpoint{2.938052in}{0.682672in}}%
\pgfpathcurveto{\pgfqpoint{2.930238in}{0.690486in}}{\pgfqpoint{2.919639in}{0.694876in}}{\pgfqpoint{2.908589in}{0.694876in}}%
\pgfpathcurveto{\pgfqpoint{2.897539in}{0.694876in}}{\pgfqpoint{2.886940in}{0.690486in}}{\pgfqpoint{2.879127in}{0.682672in}}%
\pgfpathcurveto{\pgfqpoint{2.871313in}{0.674858in}}{\pgfqpoint{2.866923in}{0.664259in}}{\pgfqpoint{2.866923in}{0.653209in}}%
\pgfpathcurveto{\pgfqpoint{2.866923in}{0.642159in}}{\pgfqpoint{2.871313in}{0.631560in}}{\pgfqpoint{2.879127in}{0.623746in}}%
\pgfpathcurveto{\pgfqpoint{2.886940in}{0.615933in}}{\pgfqpoint{2.897539in}{0.611543in}}{\pgfqpoint{2.908589in}{0.611543in}}%
\pgfpathclose%
\pgfusepath{stroke,fill}%
\end{pgfscope}%
\begin{pgfscope}%
\pgfpathrectangle{\pgfqpoint{0.481978in}{0.331635in}}{\pgfqpoint{4.960000in}{3.696000in}}%
\pgfusepath{clip}%
\pgfsetbuttcap%
\pgfsetroundjoin%
\definecolor{currentfill}{rgb}{0.631373,0.788235,0.956863}%
\pgfsetfillcolor{currentfill}%
\pgfsetlinewidth{0.481800pt}%
\definecolor{currentstroke}{rgb}{1.000000,1.000000,1.000000}%
\pgfsetstrokecolor{currentstroke}%
\pgfsetdash{}{0pt}%
\pgfpathmoveto{\pgfqpoint{3.525361in}{2.795083in}}%
\pgfpathcurveto{\pgfqpoint{3.536411in}{2.795083in}}{\pgfqpoint{3.547010in}{2.799474in}}{\pgfqpoint{3.554824in}{2.807287in}}%
\pgfpathcurveto{\pgfqpoint{3.562638in}{2.815101in}}{\pgfqpoint{3.567028in}{2.825700in}}{\pgfqpoint{3.567028in}{2.836750in}}%
\pgfpathcurveto{\pgfqpoint{3.567028in}{2.847800in}}{\pgfqpoint{3.562638in}{2.858399in}}{\pgfqpoint{3.554824in}{2.866213in}}%
\pgfpathcurveto{\pgfqpoint{3.547010in}{2.874026in}}{\pgfqpoint{3.536411in}{2.878417in}}{\pgfqpoint{3.525361in}{2.878417in}}%
\pgfpathcurveto{\pgfqpoint{3.514311in}{2.878417in}}{\pgfqpoint{3.503712in}{2.874026in}}{\pgfqpoint{3.495898in}{2.866213in}}%
\pgfpathcurveto{\pgfqpoint{3.488085in}{2.858399in}}{\pgfqpoint{3.483695in}{2.847800in}}{\pgfqpoint{3.483695in}{2.836750in}}%
\pgfpathcurveto{\pgfqpoint{3.483695in}{2.825700in}}{\pgfqpoint{3.488085in}{2.815101in}}{\pgfqpoint{3.495898in}{2.807287in}}%
\pgfpathcurveto{\pgfqpoint{3.503712in}{2.799474in}}{\pgfqpoint{3.514311in}{2.795083in}}{\pgfqpoint{3.525361in}{2.795083in}}%
\pgfpathclose%
\pgfusepath{stroke,fill}%
\end{pgfscope}%
\begin{pgfscope}%
\pgfpathrectangle{\pgfqpoint{0.481978in}{0.331635in}}{\pgfqpoint{4.960000in}{3.696000in}}%
\pgfusepath{clip}%
\pgfsetbuttcap%
\pgfsetroundjoin%
\definecolor{currentfill}{rgb}{0.631373,0.788235,0.956863}%
\pgfsetfillcolor{currentfill}%
\pgfsetlinewidth{0.481800pt}%
\definecolor{currentstroke}{rgb}{1.000000,1.000000,1.000000}%
\pgfsetstrokecolor{currentstroke}%
\pgfsetdash{}{0pt}%
\pgfpathmoveto{\pgfqpoint{3.422212in}{2.670977in}}%
\pgfpathcurveto{\pgfqpoint{3.433262in}{2.670977in}}{\pgfqpoint{3.443861in}{2.675368in}}{\pgfqpoint{3.451675in}{2.683181in}}%
\pgfpathcurveto{\pgfqpoint{3.459489in}{2.690995in}}{\pgfqpoint{3.463879in}{2.701594in}}{\pgfqpoint{3.463879in}{2.712644in}}%
\pgfpathcurveto{\pgfqpoint{3.463879in}{2.723694in}}{\pgfqpoint{3.459489in}{2.734293in}}{\pgfqpoint{3.451675in}{2.742107in}}%
\pgfpathcurveto{\pgfqpoint{3.443861in}{2.749920in}}{\pgfqpoint{3.433262in}{2.754311in}}{\pgfqpoint{3.422212in}{2.754311in}}%
\pgfpathcurveto{\pgfqpoint{3.411162in}{2.754311in}}{\pgfqpoint{3.400563in}{2.749920in}}{\pgfqpoint{3.392749in}{2.742107in}}%
\pgfpathcurveto{\pgfqpoint{3.384936in}{2.734293in}}{\pgfqpoint{3.380546in}{2.723694in}}{\pgfqpoint{3.380546in}{2.712644in}}%
\pgfpathcurveto{\pgfqpoint{3.380546in}{2.701594in}}{\pgfqpoint{3.384936in}{2.690995in}}{\pgfqpoint{3.392749in}{2.683181in}}%
\pgfpathcurveto{\pgfqpoint{3.400563in}{2.675368in}}{\pgfqpoint{3.411162in}{2.670977in}}{\pgfqpoint{3.422212in}{2.670977in}}%
\pgfpathclose%
\pgfusepath{stroke,fill}%
\end{pgfscope}%
\begin{pgfscope}%
\pgfpathrectangle{\pgfqpoint{0.481978in}{0.331635in}}{\pgfqpoint{4.960000in}{3.696000in}}%
\pgfusepath{clip}%
\pgfsetbuttcap%
\pgfsetroundjoin%
\definecolor{currentfill}{rgb}{0.631373,0.788235,0.956863}%
\pgfsetfillcolor{currentfill}%
\pgfsetlinewidth{0.481800pt}%
\definecolor{currentstroke}{rgb}{1.000000,1.000000,1.000000}%
\pgfsetstrokecolor{currentstroke}%
\pgfsetdash{}{0pt}%
\pgfpathmoveto{\pgfqpoint{3.435148in}{2.457987in}}%
\pgfpathcurveto{\pgfqpoint{3.446198in}{2.457987in}}{\pgfqpoint{3.456797in}{2.462378in}}{\pgfqpoint{3.464611in}{2.470191in}}%
\pgfpathcurveto{\pgfqpoint{3.472424in}{2.478005in}}{\pgfqpoint{3.476814in}{2.488604in}}{\pgfqpoint{3.476814in}{2.499654in}}%
\pgfpathcurveto{\pgfqpoint{3.476814in}{2.510704in}}{\pgfqpoint{3.472424in}{2.521303in}}{\pgfqpoint{3.464611in}{2.529117in}}%
\pgfpathcurveto{\pgfqpoint{3.456797in}{2.536931in}}{\pgfqpoint{3.446198in}{2.541321in}}{\pgfqpoint{3.435148in}{2.541321in}}%
\pgfpathcurveto{\pgfqpoint{3.424098in}{2.541321in}}{\pgfqpoint{3.413499in}{2.536931in}}{\pgfqpoint{3.405685in}{2.529117in}}%
\pgfpathcurveto{\pgfqpoint{3.397871in}{2.521303in}}{\pgfqpoint{3.393481in}{2.510704in}}{\pgfqpoint{3.393481in}{2.499654in}}%
\pgfpathcurveto{\pgfqpoint{3.393481in}{2.488604in}}{\pgfqpoint{3.397871in}{2.478005in}}{\pgfqpoint{3.405685in}{2.470191in}}%
\pgfpathcurveto{\pgfqpoint{3.413499in}{2.462378in}}{\pgfqpoint{3.424098in}{2.457987in}}{\pgfqpoint{3.435148in}{2.457987in}}%
\pgfpathclose%
\pgfusepath{stroke,fill}%
\end{pgfscope}%
\begin{pgfscope}%
\pgfpathrectangle{\pgfqpoint{0.481978in}{0.331635in}}{\pgfqpoint{4.960000in}{3.696000in}}%
\pgfusepath{clip}%
\pgfsetbuttcap%
\pgfsetroundjoin%
\definecolor{currentfill}{rgb}{0.631373,0.788235,0.956863}%
\pgfsetfillcolor{currentfill}%
\pgfsetlinewidth{0.481800pt}%
\definecolor{currentstroke}{rgb}{1.000000,1.000000,1.000000}%
\pgfsetstrokecolor{currentstroke}%
\pgfsetdash{}{0pt}%
\pgfpathmoveto{\pgfqpoint{3.319694in}{2.314890in}}%
\pgfpathcurveto{\pgfqpoint{3.330744in}{2.314890in}}{\pgfqpoint{3.341343in}{2.319280in}}{\pgfqpoint{3.349156in}{2.327094in}}%
\pgfpathcurveto{\pgfqpoint{3.356970in}{2.334907in}}{\pgfqpoint{3.361360in}{2.345506in}}{\pgfqpoint{3.361360in}{2.356556in}}%
\pgfpathcurveto{\pgfqpoint{3.361360in}{2.367607in}}{\pgfqpoint{3.356970in}{2.378206in}}{\pgfqpoint{3.349156in}{2.386019in}}%
\pgfpathcurveto{\pgfqpoint{3.341343in}{2.393833in}}{\pgfqpoint{3.330744in}{2.398223in}}{\pgfqpoint{3.319694in}{2.398223in}}%
\pgfpathcurveto{\pgfqpoint{3.308644in}{2.398223in}}{\pgfqpoint{3.298044in}{2.393833in}}{\pgfqpoint{3.290231in}{2.386019in}}%
\pgfpathcurveto{\pgfqpoint{3.282417in}{2.378206in}}{\pgfqpoint{3.278027in}{2.367607in}}{\pgfqpoint{3.278027in}{2.356556in}}%
\pgfpathcurveto{\pgfqpoint{3.278027in}{2.345506in}}{\pgfqpoint{3.282417in}{2.334907in}}{\pgfqpoint{3.290231in}{2.327094in}}%
\pgfpathcurveto{\pgfqpoint{3.298044in}{2.319280in}}{\pgfqpoint{3.308644in}{2.314890in}}{\pgfqpoint{3.319694in}{2.314890in}}%
\pgfpathclose%
\pgfusepath{stroke,fill}%
\end{pgfscope}%
\begin{pgfscope}%
\pgfpathrectangle{\pgfqpoint{0.481978in}{0.331635in}}{\pgfqpoint{4.960000in}{3.696000in}}%
\pgfusepath{clip}%
\pgfsetbuttcap%
\pgfsetroundjoin%
\definecolor{currentfill}{rgb}{0.631373,0.788235,0.956863}%
\pgfsetfillcolor{currentfill}%
\pgfsetlinewidth{0.481800pt}%
\definecolor{currentstroke}{rgb}{1.000000,1.000000,1.000000}%
\pgfsetstrokecolor{currentstroke}%
\pgfsetdash{}{0pt}%
\pgfpathmoveto{\pgfqpoint{1.843632in}{2.309534in}}%
\pgfpathcurveto{\pgfqpoint{1.854682in}{2.309534in}}{\pgfqpoint{1.865281in}{2.313924in}}{\pgfqpoint{1.873095in}{2.321738in}}%
\pgfpathcurveto{\pgfqpoint{1.880908in}{2.329551in}}{\pgfqpoint{1.885299in}{2.340150in}}{\pgfqpoint{1.885299in}{2.351200in}}%
\pgfpathcurveto{\pgfqpoint{1.885299in}{2.362250in}}{\pgfqpoint{1.880908in}{2.372849in}}{\pgfqpoint{1.873095in}{2.380663in}}%
\pgfpathcurveto{\pgfqpoint{1.865281in}{2.388477in}}{\pgfqpoint{1.854682in}{2.392867in}}{\pgfqpoint{1.843632in}{2.392867in}}%
\pgfpathcurveto{\pgfqpoint{1.832582in}{2.392867in}}{\pgfqpoint{1.821983in}{2.388477in}}{\pgfqpoint{1.814169in}{2.380663in}}%
\pgfpathcurveto{\pgfqpoint{1.806355in}{2.372849in}}{\pgfqpoint{1.801965in}{2.362250in}}{\pgfqpoint{1.801965in}{2.351200in}}%
\pgfpathcurveto{\pgfqpoint{1.801965in}{2.340150in}}{\pgfqpoint{1.806355in}{2.329551in}}{\pgfqpoint{1.814169in}{2.321738in}}%
\pgfpathcurveto{\pgfqpoint{1.821983in}{2.313924in}}{\pgfqpoint{1.832582in}{2.309534in}}{\pgfqpoint{1.843632in}{2.309534in}}%
\pgfpathclose%
\pgfusepath{stroke,fill}%
\end{pgfscope}%
\begin{pgfscope}%
\pgfpathrectangle{\pgfqpoint{0.481978in}{0.331635in}}{\pgfqpoint{4.960000in}{3.696000in}}%
\pgfusepath{clip}%
\pgfsetbuttcap%
\pgfsetroundjoin%
\definecolor{currentfill}{rgb}{0.631373,0.788235,0.956863}%
\pgfsetfillcolor{currentfill}%
\pgfsetlinewidth{0.481800pt}%
\definecolor{currentstroke}{rgb}{1.000000,1.000000,1.000000}%
\pgfsetstrokecolor{currentstroke}%
\pgfsetdash{}{0pt}%
\pgfpathmoveto{\pgfqpoint{3.396710in}{2.194753in}}%
\pgfpathcurveto{\pgfqpoint{3.407760in}{2.194753in}}{\pgfqpoint{3.418359in}{2.199143in}}{\pgfqpoint{3.426173in}{2.206957in}}%
\pgfpathcurveto{\pgfqpoint{3.433986in}{2.214770in}}{\pgfqpoint{3.438376in}{2.225369in}}{\pgfqpoint{3.438376in}{2.236419in}}%
\pgfpathcurveto{\pgfqpoint{3.438376in}{2.247470in}}{\pgfqpoint{3.433986in}{2.258069in}}{\pgfqpoint{3.426173in}{2.265882in}}%
\pgfpathcurveto{\pgfqpoint{3.418359in}{2.273696in}}{\pgfqpoint{3.407760in}{2.278086in}}{\pgfqpoint{3.396710in}{2.278086in}}%
\pgfpathcurveto{\pgfqpoint{3.385660in}{2.278086in}}{\pgfqpoint{3.375061in}{2.273696in}}{\pgfqpoint{3.367247in}{2.265882in}}%
\pgfpathcurveto{\pgfqpoint{3.359433in}{2.258069in}}{\pgfqpoint{3.355043in}{2.247470in}}{\pgfqpoint{3.355043in}{2.236419in}}%
\pgfpathcurveto{\pgfqpoint{3.355043in}{2.225369in}}{\pgfqpoint{3.359433in}{2.214770in}}{\pgfqpoint{3.367247in}{2.206957in}}%
\pgfpathcurveto{\pgfqpoint{3.375061in}{2.199143in}}{\pgfqpoint{3.385660in}{2.194753in}}{\pgfqpoint{3.396710in}{2.194753in}}%
\pgfpathclose%
\pgfusepath{stroke,fill}%
\end{pgfscope}%
\begin{pgfscope}%
\pgfpathrectangle{\pgfqpoint{0.481978in}{0.331635in}}{\pgfqpoint{4.960000in}{3.696000in}}%
\pgfusepath{clip}%
\pgfsetbuttcap%
\pgfsetroundjoin%
\definecolor{currentfill}{rgb}{0.631373,0.788235,0.956863}%
\pgfsetfillcolor{currentfill}%
\pgfsetlinewidth{0.481800pt}%
\definecolor{currentstroke}{rgb}{1.000000,1.000000,1.000000}%
\pgfsetstrokecolor{currentstroke}%
\pgfsetdash{}{0pt}%
\pgfpathmoveto{\pgfqpoint{2.212263in}{2.147174in}}%
\pgfpathcurveto{\pgfqpoint{2.223313in}{2.147174in}}{\pgfqpoint{2.233912in}{2.151565in}}{\pgfqpoint{2.241726in}{2.159378in}}%
\pgfpathcurveto{\pgfqpoint{2.249539in}{2.167192in}}{\pgfqpoint{2.253929in}{2.177791in}}{\pgfqpoint{2.253929in}{2.188841in}}%
\pgfpathcurveto{\pgfqpoint{2.253929in}{2.199891in}}{\pgfqpoint{2.249539in}{2.210490in}}{\pgfqpoint{2.241726in}{2.218304in}}%
\pgfpathcurveto{\pgfqpoint{2.233912in}{2.226117in}}{\pgfqpoint{2.223313in}{2.230508in}}{\pgfqpoint{2.212263in}{2.230508in}}%
\pgfpathcurveto{\pgfqpoint{2.201213in}{2.230508in}}{\pgfqpoint{2.190614in}{2.226117in}}{\pgfqpoint{2.182800in}{2.218304in}}%
\pgfpathcurveto{\pgfqpoint{2.174986in}{2.210490in}}{\pgfqpoint{2.170596in}{2.199891in}}{\pgfqpoint{2.170596in}{2.188841in}}%
\pgfpathcurveto{\pgfqpoint{2.170596in}{2.177791in}}{\pgfqpoint{2.174986in}{2.167192in}}{\pgfqpoint{2.182800in}{2.159378in}}%
\pgfpathcurveto{\pgfqpoint{2.190614in}{2.151565in}}{\pgfqpoint{2.201213in}{2.147174in}}{\pgfqpoint{2.212263in}{2.147174in}}%
\pgfpathclose%
\pgfusepath{stroke,fill}%
\end{pgfscope}%
\begin{pgfscope}%
\pgfpathrectangle{\pgfqpoint{0.481978in}{0.331635in}}{\pgfqpoint{4.960000in}{3.696000in}}%
\pgfusepath{clip}%
\pgfsetbuttcap%
\pgfsetroundjoin%
\definecolor{currentfill}{rgb}{0.631373,0.788235,0.956863}%
\pgfsetfillcolor{currentfill}%
\pgfsetlinewidth{0.481800pt}%
\definecolor{currentstroke}{rgb}{1.000000,1.000000,1.000000}%
\pgfsetstrokecolor{currentstroke}%
\pgfsetdash{}{0pt}%
\pgfpathmoveto{\pgfqpoint{4.964209in}{2.278605in}}%
\pgfpathcurveto{\pgfqpoint{4.975259in}{2.278605in}}{\pgfqpoint{4.985858in}{2.282995in}}{\pgfqpoint{4.993672in}{2.290809in}}%
\pgfpathcurveto{\pgfqpoint{5.001485in}{2.298622in}}{\pgfqpoint{5.005875in}{2.309221in}}{\pgfqpoint{5.005875in}{2.320271in}}%
\pgfpathcurveto{\pgfqpoint{5.005875in}{2.331322in}}{\pgfqpoint{5.001485in}{2.341921in}}{\pgfqpoint{4.993672in}{2.349734in}}%
\pgfpathcurveto{\pgfqpoint{4.985858in}{2.357548in}}{\pgfqpoint{4.975259in}{2.361938in}}{\pgfqpoint{4.964209in}{2.361938in}}%
\pgfpathcurveto{\pgfqpoint{4.953159in}{2.361938in}}{\pgfqpoint{4.942560in}{2.357548in}}{\pgfqpoint{4.934746in}{2.349734in}}%
\pgfpathcurveto{\pgfqpoint{4.926932in}{2.341921in}}{\pgfqpoint{4.922542in}{2.331322in}}{\pgfqpoint{4.922542in}{2.320271in}}%
\pgfpathcurveto{\pgfqpoint{4.922542in}{2.309221in}}{\pgfqpoint{4.926932in}{2.298622in}}{\pgfqpoint{4.934746in}{2.290809in}}%
\pgfpathcurveto{\pgfqpoint{4.942560in}{2.282995in}}{\pgfqpoint{4.953159in}{2.278605in}}{\pgfqpoint{4.964209in}{2.278605in}}%
\pgfpathclose%
\pgfusepath{stroke,fill}%
\end{pgfscope}%
\begin{pgfscope}%
\pgfpathrectangle{\pgfqpoint{0.481978in}{0.331635in}}{\pgfqpoint{4.960000in}{3.696000in}}%
\pgfusepath{clip}%
\pgfsetbuttcap%
\pgfsetroundjoin%
\definecolor{currentfill}{rgb}{0.631373,0.788235,0.956863}%
\pgfsetfillcolor{currentfill}%
\pgfsetlinewidth{0.481800pt}%
\definecolor{currentstroke}{rgb}{1.000000,1.000000,1.000000}%
\pgfsetstrokecolor{currentstroke}%
\pgfsetdash{}{0pt}%
\pgfpathmoveto{\pgfqpoint{3.279740in}{0.537133in}}%
\pgfpathcurveto{\pgfqpoint{3.290790in}{0.537133in}}{\pgfqpoint{3.301389in}{0.541524in}}{\pgfqpoint{3.309203in}{0.549337in}}%
\pgfpathcurveto{\pgfqpoint{3.317016in}{0.557151in}}{\pgfqpoint{3.321407in}{0.567750in}}{\pgfqpoint{3.321407in}{0.578800in}}%
\pgfpathcurveto{\pgfqpoint{3.321407in}{0.589850in}}{\pgfqpoint{3.317016in}{0.600449in}}{\pgfqpoint{3.309203in}{0.608263in}}%
\pgfpathcurveto{\pgfqpoint{3.301389in}{0.616076in}}{\pgfqpoint{3.290790in}{0.620467in}}{\pgfqpoint{3.279740in}{0.620467in}}%
\pgfpathcurveto{\pgfqpoint{3.268690in}{0.620467in}}{\pgfqpoint{3.258091in}{0.616076in}}{\pgfqpoint{3.250277in}{0.608263in}}%
\pgfpathcurveto{\pgfqpoint{3.242464in}{0.600449in}}{\pgfqpoint{3.238073in}{0.589850in}}{\pgfqpoint{3.238073in}{0.578800in}}%
\pgfpathcurveto{\pgfqpoint{3.238073in}{0.567750in}}{\pgfqpoint{3.242464in}{0.557151in}}{\pgfqpoint{3.250277in}{0.549337in}}%
\pgfpathcurveto{\pgfqpoint{3.258091in}{0.541524in}}{\pgfqpoint{3.268690in}{0.537133in}}{\pgfqpoint{3.279740in}{0.537133in}}%
\pgfpathclose%
\pgfusepath{stroke,fill}%
\end{pgfscope}%
\begin{pgfscope}%
\pgfpathrectangle{\pgfqpoint{0.481978in}{0.331635in}}{\pgfqpoint{4.960000in}{3.696000in}}%
\pgfusepath{clip}%
\pgfsetbuttcap%
\pgfsetroundjoin%
\definecolor{currentfill}{rgb}{0.631373,0.788235,0.956863}%
\pgfsetfillcolor{currentfill}%
\pgfsetlinewidth{0.481800pt}%
\definecolor{currentstroke}{rgb}{1.000000,1.000000,1.000000}%
\pgfsetstrokecolor{currentstroke}%
\pgfsetdash{}{0pt}%
\pgfpathmoveto{\pgfqpoint{4.204936in}{2.179389in}}%
\pgfpathcurveto{\pgfqpoint{4.215986in}{2.179389in}}{\pgfqpoint{4.226585in}{2.183780in}}{\pgfqpoint{4.234399in}{2.191593in}}%
\pgfpathcurveto{\pgfqpoint{4.242213in}{2.199407in}}{\pgfqpoint{4.246603in}{2.210006in}}{\pgfqpoint{4.246603in}{2.221056in}}%
\pgfpathcurveto{\pgfqpoint{4.246603in}{2.232106in}}{\pgfqpoint{4.242213in}{2.242705in}}{\pgfqpoint{4.234399in}{2.250519in}}%
\pgfpathcurveto{\pgfqpoint{4.226585in}{2.258332in}}{\pgfqpoint{4.215986in}{2.262723in}}{\pgfqpoint{4.204936in}{2.262723in}}%
\pgfpathcurveto{\pgfqpoint{4.193886in}{2.262723in}}{\pgfqpoint{4.183287in}{2.258332in}}{\pgfqpoint{4.175473in}{2.250519in}}%
\pgfpathcurveto{\pgfqpoint{4.167660in}{2.242705in}}{\pgfqpoint{4.163269in}{2.232106in}}{\pgfqpoint{4.163269in}{2.221056in}}%
\pgfpathcurveto{\pgfqpoint{4.163269in}{2.210006in}}{\pgfqpoint{4.167660in}{2.199407in}}{\pgfqpoint{4.175473in}{2.191593in}}%
\pgfpathcurveto{\pgfqpoint{4.183287in}{2.183780in}}{\pgfqpoint{4.193886in}{2.179389in}}{\pgfqpoint{4.204936in}{2.179389in}}%
\pgfpathclose%
\pgfusepath{stroke,fill}%
\end{pgfscope}%
\begin{pgfscope}%
\pgfpathrectangle{\pgfqpoint{0.481978in}{0.331635in}}{\pgfqpoint{4.960000in}{3.696000in}}%
\pgfusepath{clip}%
\pgfsetbuttcap%
\pgfsetroundjoin%
\definecolor{currentfill}{rgb}{0.631373,0.788235,0.956863}%
\pgfsetfillcolor{currentfill}%
\pgfsetlinewidth{0.481800pt}%
\definecolor{currentstroke}{rgb}{1.000000,1.000000,1.000000}%
\pgfsetstrokecolor{currentstroke}%
\pgfsetdash{}{0pt}%
\pgfpathmoveto{\pgfqpoint{4.440145in}{1.578864in}}%
\pgfpathcurveto{\pgfqpoint{4.451195in}{1.578864in}}{\pgfqpoint{4.461794in}{1.583254in}}{\pgfqpoint{4.469608in}{1.591068in}}%
\pgfpathcurveto{\pgfqpoint{4.477422in}{1.598881in}}{\pgfqpoint{4.481812in}{1.609480in}}{\pgfqpoint{4.481812in}{1.620531in}}%
\pgfpathcurveto{\pgfqpoint{4.481812in}{1.631581in}}{\pgfqpoint{4.477422in}{1.642180in}}{\pgfqpoint{4.469608in}{1.649993in}}%
\pgfpathcurveto{\pgfqpoint{4.461794in}{1.657807in}}{\pgfqpoint{4.451195in}{1.662197in}}{\pgfqpoint{4.440145in}{1.662197in}}%
\pgfpathcurveto{\pgfqpoint{4.429095in}{1.662197in}}{\pgfqpoint{4.418496in}{1.657807in}}{\pgfqpoint{4.410683in}{1.649993in}}%
\pgfpathcurveto{\pgfqpoint{4.402869in}{1.642180in}}{\pgfqpoint{4.398479in}{1.631581in}}{\pgfqpoint{4.398479in}{1.620531in}}%
\pgfpathcurveto{\pgfqpoint{4.398479in}{1.609480in}}{\pgfqpoint{4.402869in}{1.598881in}}{\pgfqpoint{4.410683in}{1.591068in}}%
\pgfpathcurveto{\pgfqpoint{4.418496in}{1.583254in}}{\pgfqpoint{4.429095in}{1.578864in}}{\pgfqpoint{4.440145in}{1.578864in}}%
\pgfpathclose%
\pgfusepath{stroke,fill}%
\end{pgfscope}%
\begin{pgfscope}%
\pgfpathrectangle{\pgfqpoint{0.481978in}{0.331635in}}{\pgfqpoint{4.960000in}{3.696000in}}%
\pgfusepath{clip}%
\pgfsetbuttcap%
\pgfsetroundjoin%
\definecolor{currentfill}{rgb}{0.631373,0.788235,0.956863}%
\pgfsetfillcolor{currentfill}%
\pgfsetlinewidth{0.481800pt}%
\definecolor{currentstroke}{rgb}{1.000000,1.000000,1.000000}%
\pgfsetstrokecolor{currentstroke}%
\pgfsetdash{}{0pt}%
\pgfpathmoveto{\pgfqpoint{1.851390in}{0.942459in}}%
\pgfpathcurveto{\pgfqpoint{1.862440in}{0.942459in}}{\pgfqpoint{1.873039in}{0.946849in}}{\pgfqpoint{1.880853in}{0.954663in}}%
\pgfpathcurveto{\pgfqpoint{1.888666in}{0.962476in}}{\pgfqpoint{1.893056in}{0.973075in}}{\pgfqpoint{1.893056in}{0.984126in}}%
\pgfpathcurveto{\pgfqpoint{1.893056in}{0.995176in}}{\pgfqpoint{1.888666in}{1.005775in}}{\pgfqpoint{1.880853in}{1.013588in}}%
\pgfpathcurveto{\pgfqpoint{1.873039in}{1.021402in}}{\pgfqpoint{1.862440in}{1.025792in}}{\pgfqpoint{1.851390in}{1.025792in}}%
\pgfpathcurveto{\pgfqpoint{1.840340in}{1.025792in}}{\pgfqpoint{1.829741in}{1.021402in}}{\pgfqpoint{1.821927in}{1.013588in}}%
\pgfpathcurveto{\pgfqpoint{1.814113in}{1.005775in}}{\pgfqpoint{1.809723in}{0.995176in}}{\pgfqpoint{1.809723in}{0.984126in}}%
\pgfpathcurveto{\pgfqpoint{1.809723in}{0.973075in}}{\pgfqpoint{1.814113in}{0.962476in}}{\pgfqpoint{1.821927in}{0.954663in}}%
\pgfpathcurveto{\pgfqpoint{1.829741in}{0.946849in}}{\pgfqpoint{1.840340in}{0.942459in}}{\pgfqpoint{1.851390in}{0.942459in}}%
\pgfpathclose%
\pgfusepath{stroke,fill}%
\end{pgfscope}%
\begin{pgfscope}%
\pgfpathrectangle{\pgfqpoint{0.481978in}{0.331635in}}{\pgfqpoint{4.960000in}{3.696000in}}%
\pgfusepath{clip}%
\pgfsetbuttcap%
\pgfsetroundjoin%
\definecolor{currentfill}{rgb}{0.631373,0.788235,0.956863}%
\pgfsetfillcolor{currentfill}%
\pgfsetlinewidth{0.481800pt}%
\definecolor{currentstroke}{rgb}{1.000000,1.000000,1.000000}%
\pgfsetstrokecolor{currentstroke}%
\pgfsetdash{}{0pt}%
\pgfpathmoveto{\pgfqpoint{3.151439in}{1.064241in}}%
\pgfpathcurveto{\pgfqpoint{3.162489in}{1.064241in}}{\pgfqpoint{3.173088in}{1.068631in}}{\pgfqpoint{3.180902in}{1.076445in}}%
\pgfpathcurveto{\pgfqpoint{3.188715in}{1.084258in}}{\pgfqpoint{3.193105in}{1.094857in}}{\pgfqpoint{3.193105in}{1.105908in}}%
\pgfpathcurveto{\pgfqpoint{3.193105in}{1.116958in}}{\pgfqpoint{3.188715in}{1.127557in}}{\pgfqpoint{3.180902in}{1.135370in}}%
\pgfpathcurveto{\pgfqpoint{3.173088in}{1.143184in}}{\pgfqpoint{3.162489in}{1.147574in}}{\pgfqpoint{3.151439in}{1.147574in}}%
\pgfpathcurveto{\pgfqpoint{3.140389in}{1.147574in}}{\pgfqpoint{3.129790in}{1.143184in}}{\pgfqpoint{3.121976in}{1.135370in}}%
\pgfpathcurveto{\pgfqpoint{3.114162in}{1.127557in}}{\pgfqpoint{3.109772in}{1.116958in}}{\pgfqpoint{3.109772in}{1.105908in}}%
\pgfpathcurveto{\pgfqpoint{3.109772in}{1.094857in}}{\pgfqpoint{3.114162in}{1.084258in}}{\pgfqpoint{3.121976in}{1.076445in}}%
\pgfpathcurveto{\pgfqpoint{3.129790in}{1.068631in}}{\pgfqpoint{3.140389in}{1.064241in}}{\pgfqpoint{3.151439in}{1.064241in}}%
\pgfpathclose%
\pgfusepath{stroke,fill}%
\end{pgfscope}%
\begin{pgfscope}%
\pgfpathrectangle{\pgfqpoint{0.481978in}{0.331635in}}{\pgfqpoint{4.960000in}{3.696000in}}%
\pgfusepath{clip}%
\pgfsetbuttcap%
\pgfsetroundjoin%
\definecolor{currentfill}{rgb}{0.631373,0.788235,0.956863}%
\pgfsetfillcolor{currentfill}%
\pgfsetlinewidth{0.481800pt}%
\definecolor{currentstroke}{rgb}{1.000000,1.000000,1.000000}%
\pgfsetstrokecolor{currentstroke}%
\pgfsetdash{}{0pt}%
\pgfpathmoveto{\pgfqpoint{2.882809in}{2.979908in}}%
\pgfpathcurveto{\pgfqpoint{2.893859in}{2.979908in}}{\pgfqpoint{2.904458in}{2.984298in}}{\pgfqpoint{2.912272in}{2.992112in}}%
\pgfpathcurveto{\pgfqpoint{2.920086in}{2.999925in}}{\pgfqpoint{2.924476in}{3.010524in}}{\pgfqpoint{2.924476in}{3.021574in}}%
\pgfpathcurveto{\pgfqpoint{2.924476in}{3.032625in}}{\pgfqpoint{2.920086in}{3.043224in}}{\pgfqpoint{2.912272in}{3.051037in}}%
\pgfpathcurveto{\pgfqpoint{2.904458in}{3.058851in}}{\pgfqpoint{2.893859in}{3.063241in}}{\pgfqpoint{2.882809in}{3.063241in}}%
\pgfpathcurveto{\pgfqpoint{2.871759in}{3.063241in}}{\pgfqpoint{2.861160in}{3.058851in}}{\pgfqpoint{2.853346in}{3.051037in}}%
\pgfpathcurveto{\pgfqpoint{2.845533in}{3.043224in}}{\pgfqpoint{2.841143in}{3.032625in}}{\pgfqpoint{2.841143in}{3.021574in}}%
\pgfpathcurveto{\pgfqpoint{2.841143in}{3.010524in}}{\pgfqpoint{2.845533in}{2.999925in}}{\pgfqpoint{2.853346in}{2.992112in}}%
\pgfpathcurveto{\pgfqpoint{2.861160in}{2.984298in}}{\pgfqpoint{2.871759in}{2.979908in}}{\pgfqpoint{2.882809in}{2.979908in}}%
\pgfpathclose%
\pgfusepath{stroke,fill}%
\end{pgfscope}%
\begin{pgfscope}%
\pgfpathrectangle{\pgfqpoint{0.481978in}{0.331635in}}{\pgfqpoint{4.960000in}{3.696000in}}%
\pgfusepath{clip}%
\pgfsetbuttcap%
\pgfsetroundjoin%
\definecolor{currentfill}{rgb}{0.631373,0.788235,0.956863}%
\pgfsetfillcolor{currentfill}%
\pgfsetlinewidth{0.481800pt}%
\definecolor{currentstroke}{rgb}{1.000000,1.000000,1.000000}%
\pgfsetstrokecolor{currentstroke}%
\pgfsetdash{}{0pt}%
\pgfpathmoveto{\pgfqpoint{3.349813in}{1.647052in}}%
\pgfpathcurveto{\pgfqpoint{3.360863in}{1.647052in}}{\pgfqpoint{3.371462in}{1.651442in}}{\pgfqpoint{3.379276in}{1.659255in}}%
\pgfpathcurveto{\pgfqpoint{3.387090in}{1.667069in}}{\pgfqpoint{3.391480in}{1.677668in}}{\pgfqpoint{3.391480in}{1.688718in}}%
\pgfpathcurveto{\pgfqpoint{3.391480in}{1.699768in}}{\pgfqpoint{3.387090in}{1.710367in}}{\pgfqpoint{3.379276in}{1.718181in}}%
\pgfpathcurveto{\pgfqpoint{3.371462in}{1.725995in}}{\pgfqpoint{3.360863in}{1.730385in}}{\pgfqpoint{3.349813in}{1.730385in}}%
\pgfpathcurveto{\pgfqpoint{3.338763in}{1.730385in}}{\pgfqpoint{3.328164in}{1.725995in}}{\pgfqpoint{3.320350in}{1.718181in}}%
\pgfpathcurveto{\pgfqpoint{3.312537in}{1.710367in}}{\pgfqpoint{3.308147in}{1.699768in}}{\pgfqpoint{3.308147in}{1.688718in}}%
\pgfpathcurveto{\pgfqpoint{3.308147in}{1.677668in}}{\pgfqpoint{3.312537in}{1.667069in}}{\pgfqpoint{3.320350in}{1.659255in}}%
\pgfpathcurveto{\pgfqpoint{3.328164in}{1.651442in}}{\pgfqpoint{3.338763in}{1.647052in}}{\pgfqpoint{3.349813in}{1.647052in}}%
\pgfpathclose%
\pgfusepath{stroke,fill}%
\end{pgfscope}%
\begin{pgfscope}%
\pgfpathrectangle{\pgfqpoint{0.481978in}{0.331635in}}{\pgfqpoint{4.960000in}{3.696000in}}%
\pgfusepath{clip}%
\pgfsetbuttcap%
\pgfsetroundjoin%
\definecolor{currentfill}{rgb}{0.631373,0.788235,0.956863}%
\pgfsetfillcolor{currentfill}%
\pgfsetlinewidth{0.481800pt}%
\definecolor{currentstroke}{rgb}{1.000000,1.000000,1.000000}%
\pgfsetstrokecolor{currentstroke}%
\pgfsetdash{}{0pt}%
\pgfpathmoveto{\pgfqpoint{2.941881in}{0.870356in}}%
\pgfpathcurveto{\pgfqpoint{2.952931in}{0.870356in}}{\pgfqpoint{2.963530in}{0.874747in}}{\pgfqpoint{2.971344in}{0.882560in}}%
\pgfpathcurveto{\pgfqpoint{2.979157in}{0.890374in}}{\pgfqpoint{2.983548in}{0.900973in}}{\pgfqpoint{2.983548in}{0.912023in}}%
\pgfpathcurveto{\pgfqpoint{2.983548in}{0.923073in}}{\pgfqpoint{2.979157in}{0.933672in}}{\pgfqpoint{2.971344in}{0.941486in}}%
\pgfpathcurveto{\pgfqpoint{2.963530in}{0.949299in}}{\pgfqpoint{2.952931in}{0.953690in}}{\pgfqpoint{2.941881in}{0.953690in}}%
\pgfpathcurveto{\pgfqpoint{2.930831in}{0.953690in}}{\pgfqpoint{2.920232in}{0.949299in}}{\pgfqpoint{2.912418in}{0.941486in}}%
\pgfpathcurveto{\pgfqpoint{2.904605in}{0.933672in}}{\pgfqpoint{2.900214in}{0.923073in}}{\pgfqpoint{2.900214in}{0.912023in}}%
\pgfpathcurveto{\pgfqpoint{2.900214in}{0.900973in}}{\pgfqpoint{2.904605in}{0.890374in}}{\pgfqpoint{2.912418in}{0.882560in}}%
\pgfpathcurveto{\pgfqpoint{2.920232in}{0.874747in}}{\pgfqpoint{2.930831in}{0.870356in}}{\pgfqpoint{2.941881in}{0.870356in}}%
\pgfpathclose%
\pgfusepath{stroke,fill}%
\end{pgfscope}%
\begin{pgfscope}%
\pgfpathrectangle{\pgfqpoint{0.481978in}{0.331635in}}{\pgfqpoint{4.960000in}{3.696000in}}%
\pgfusepath{clip}%
\pgfsetbuttcap%
\pgfsetroundjoin%
\definecolor{currentfill}{rgb}{0.631373,0.788235,0.956863}%
\pgfsetfillcolor{currentfill}%
\pgfsetlinewidth{0.481800pt}%
\definecolor{currentstroke}{rgb}{1.000000,1.000000,1.000000}%
\pgfsetstrokecolor{currentstroke}%
\pgfsetdash{}{0pt}%
\pgfpathmoveto{\pgfqpoint{3.692146in}{2.340438in}}%
\pgfpathcurveto{\pgfqpoint{3.703196in}{2.340438in}}{\pgfqpoint{3.713795in}{2.344828in}}{\pgfqpoint{3.721609in}{2.352642in}}%
\pgfpathcurveto{\pgfqpoint{3.729422in}{2.360455in}}{\pgfqpoint{3.733812in}{2.371054in}}{\pgfqpoint{3.733812in}{2.382105in}}%
\pgfpathcurveto{\pgfqpoint{3.733812in}{2.393155in}}{\pgfqpoint{3.729422in}{2.403754in}}{\pgfqpoint{3.721609in}{2.411567in}}%
\pgfpathcurveto{\pgfqpoint{3.713795in}{2.419381in}}{\pgfqpoint{3.703196in}{2.423771in}}{\pgfqpoint{3.692146in}{2.423771in}}%
\pgfpathcurveto{\pgfqpoint{3.681096in}{2.423771in}}{\pgfqpoint{3.670497in}{2.419381in}}{\pgfqpoint{3.662683in}{2.411567in}}%
\pgfpathcurveto{\pgfqpoint{3.654869in}{2.403754in}}{\pgfqpoint{3.650479in}{2.393155in}}{\pgfqpoint{3.650479in}{2.382105in}}%
\pgfpathcurveto{\pgfqpoint{3.650479in}{2.371054in}}{\pgfqpoint{3.654869in}{2.360455in}}{\pgfqpoint{3.662683in}{2.352642in}}%
\pgfpathcurveto{\pgfqpoint{3.670497in}{2.344828in}}{\pgfqpoint{3.681096in}{2.340438in}}{\pgfqpoint{3.692146in}{2.340438in}}%
\pgfpathclose%
\pgfusepath{stroke,fill}%
\end{pgfscope}%
\begin{pgfscope}%
\pgfpathrectangle{\pgfqpoint{0.481978in}{0.331635in}}{\pgfqpoint{4.960000in}{3.696000in}}%
\pgfusepath{clip}%
\pgfsetbuttcap%
\pgfsetroundjoin%
\definecolor{currentfill}{rgb}{0.631373,0.788235,0.956863}%
\pgfsetfillcolor{currentfill}%
\pgfsetlinewidth{0.481800pt}%
\definecolor{currentstroke}{rgb}{1.000000,1.000000,1.000000}%
\pgfsetstrokecolor{currentstroke}%
\pgfsetdash{}{0pt}%
\pgfpathmoveto{\pgfqpoint{2.675899in}{1.401777in}}%
\pgfpathcurveto{\pgfqpoint{2.686949in}{1.401777in}}{\pgfqpoint{2.697548in}{1.406167in}}{\pgfqpoint{2.705362in}{1.413981in}}%
\pgfpathcurveto{\pgfqpoint{2.713176in}{1.421795in}}{\pgfqpoint{2.717566in}{1.432394in}}{\pgfqpoint{2.717566in}{1.443444in}}%
\pgfpathcurveto{\pgfqpoint{2.717566in}{1.454494in}}{\pgfqpoint{2.713176in}{1.465093in}}{\pgfqpoint{2.705362in}{1.472907in}}%
\pgfpathcurveto{\pgfqpoint{2.697548in}{1.480720in}}{\pgfqpoint{2.686949in}{1.485111in}}{\pgfqpoint{2.675899in}{1.485111in}}%
\pgfpathcurveto{\pgfqpoint{2.664849in}{1.485111in}}{\pgfqpoint{2.654250in}{1.480720in}}{\pgfqpoint{2.646436in}{1.472907in}}%
\pgfpathcurveto{\pgfqpoint{2.638623in}{1.465093in}}{\pgfqpoint{2.634232in}{1.454494in}}{\pgfqpoint{2.634232in}{1.443444in}}%
\pgfpathcurveto{\pgfqpoint{2.634232in}{1.432394in}}{\pgfqpoint{2.638623in}{1.421795in}}{\pgfqpoint{2.646436in}{1.413981in}}%
\pgfpathcurveto{\pgfqpoint{2.654250in}{1.406167in}}{\pgfqpoint{2.664849in}{1.401777in}}{\pgfqpoint{2.675899in}{1.401777in}}%
\pgfpathclose%
\pgfusepath{stroke,fill}%
\end{pgfscope}%
\begin{pgfscope}%
\pgfpathrectangle{\pgfqpoint{0.481978in}{0.331635in}}{\pgfqpoint{4.960000in}{3.696000in}}%
\pgfusepath{clip}%
\pgfsetbuttcap%
\pgfsetroundjoin%
\definecolor{currentfill}{rgb}{0.631373,0.788235,0.956863}%
\pgfsetfillcolor{currentfill}%
\pgfsetlinewidth{0.481800pt}%
\definecolor{currentstroke}{rgb}{1.000000,1.000000,1.000000}%
\pgfsetstrokecolor{currentstroke}%
\pgfsetdash{}{0pt}%
\pgfpathmoveto{\pgfqpoint{1.662123in}{1.378604in}}%
\pgfpathcurveto{\pgfqpoint{1.673173in}{1.378604in}}{\pgfqpoint{1.683772in}{1.382994in}}{\pgfqpoint{1.691585in}{1.390807in}}%
\pgfpathcurveto{\pgfqpoint{1.699399in}{1.398621in}}{\pgfqpoint{1.703789in}{1.409220in}}{\pgfqpoint{1.703789in}{1.420270in}}%
\pgfpathcurveto{\pgfqpoint{1.703789in}{1.431320in}}{\pgfqpoint{1.699399in}{1.441919in}}{\pgfqpoint{1.691585in}{1.449733in}}%
\pgfpathcurveto{\pgfqpoint{1.683772in}{1.457547in}}{\pgfqpoint{1.673173in}{1.461937in}}{\pgfqpoint{1.662123in}{1.461937in}}%
\pgfpathcurveto{\pgfqpoint{1.651072in}{1.461937in}}{\pgfqpoint{1.640473in}{1.457547in}}{\pgfqpoint{1.632660in}{1.449733in}}%
\pgfpathcurveto{\pgfqpoint{1.624846in}{1.441919in}}{\pgfqpoint{1.620456in}{1.431320in}}{\pgfqpoint{1.620456in}{1.420270in}}%
\pgfpathcurveto{\pgfqpoint{1.620456in}{1.409220in}}{\pgfqpoint{1.624846in}{1.398621in}}{\pgfqpoint{1.632660in}{1.390807in}}%
\pgfpathcurveto{\pgfqpoint{1.640473in}{1.382994in}}{\pgfqpoint{1.651072in}{1.378604in}}{\pgfqpoint{1.662123in}{1.378604in}}%
\pgfpathclose%
\pgfusepath{stroke,fill}%
\end{pgfscope}%
\begin{pgfscope}%
\pgfpathrectangle{\pgfqpoint{0.481978in}{0.331635in}}{\pgfqpoint{4.960000in}{3.696000in}}%
\pgfusepath{clip}%
\pgfsetbuttcap%
\pgfsetroundjoin%
\definecolor{currentfill}{rgb}{0.631373,0.788235,0.956863}%
\pgfsetfillcolor{currentfill}%
\pgfsetlinewidth{0.481800pt}%
\definecolor{currentstroke}{rgb}{1.000000,1.000000,1.000000}%
\pgfsetstrokecolor{currentstroke}%
\pgfsetdash{}{0pt}%
\pgfpathmoveto{\pgfqpoint{3.493611in}{1.905461in}}%
\pgfpathcurveto{\pgfqpoint{3.504661in}{1.905461in}}{\pgfqpoint{3.515260in}{1.909851in}}{\pgfqpoint{3.523073in}{1.917665in}}%
\pgfpathcurveto{\pgfqpoint{3.530887in}{1.925478in}}{\pgfqpoint{3.535277in}{1.936077in}}{\pgfqpoint{3.535277in}{1.947127in}}%
\pgfpathcurveto{\pgfqpoint{3.535277in}{1.958178in}}{\pgfqpoint{3.530887in}{1.968777in}}{\pgfqpoint{3.523073in}{1.976590in}}%
\pgfpathcurveto{\pgfqpoint{3.515260in}{1.984404in}}{\pgfqpoint{3.504661in}{1.988794in}}{\pgfqpoint{3.493611in}{1.988794in}}%
\pgfpathcurveto{\pgfqpoint{3.482561in}{1.988794in}}{\pgfqpoint{3.471961in}{1.984404in}}{\pgfqpoint{3.464148in}{1.976590in}}%
\pgfpathcurveto{\pgfqpoint{3.456334in}{1.968777in}}{\pgfqpoint{3.451944in}{1.958178in}}{\pgfqpoint{3.451944in}{1.947127in}}%
\pgfpathcurveto{\pgfqpoint{3.451944in}{1.936077in}}{\pgfqpoint{3.456334in}{1.925478in}}{\pgfqpoint{3.464148in}{1.917665in}}%
\pgfpathcurveto{\pgfqpoint{3.471961in}{1.909851in}}{\pgfqpoint{3.482561in}{1.905461in}}{\pgfqpoint{3.493611in}{1.905461in}}%
\pgfpathclose%
\pgfusepath{stroke,fill}%
\end{pgfscope}%
\begin{pgfscope}%
\pgfpathrectangle{\pgfqpoint{0.481978in}{0.331635in}}{\pgfqpoint{4.960000in}{3.696000in}}%
\pgfusepath{clip}%
\pgfsetbuttcap%
\pgfsetroundjoin%
\definecolor{currentfill}{rgb}{0.631373,0.788235,0.956863}%
\pgfsetfillcolor{currentfill}%
\pgfsetlinewidth{0.481800pt}%
\definecolor{currentstroke}{rgb}{1.000000,1.000000,1.000000}%
\pgfsetstrokecolor{currentstroke}%
\pgfsetdash{}{0pt}%
\pgfpathmoveto{\pgfqpoint{2.998121in}{0.967970in}}%
\pgfpathcurveto{\pgfqpoint{3.009171in}{0.967970in}}{\pgfqpoint{3.019770in}{0.972360in}}{\pgfqpoint{3.027584in}{0.980174in}}%
\pgfpathcurveto{\pgfqpoint{3.035397in}{0.987987in}}{\pgfqpoint{3.039788in}{0.998586in}}{\pgfqpoint{3.039788in}{1.009636in}}%
\pgfpathcurveto{\pgfqpoint{3.039788in}{1.020686in}}{\pgfqpoint{3.035397in}{1.031285in}}{\pgfqpoint{3.027584in}{1.039099in}}%
\pgfpathcurveto{\pgfqpoint{3.019770in}{1.046913in}}{\pgfqpoint{3.009171in}{1.051303in}}{\pgfqpoint{2.998121in}{1.051303in}}%
\pgfpathcurveto{\pgfqpoint{2.987071in}{1.051303in}}{\pgfqpoint{2.976472in}{1.046913in}}{\pgfqpoint{2.968658in}{1.039099in}}%
\pgfpathcurveto{\pgfqpoint{2.960845in}{1.031285in}}{\pgfqpoint{2.956454in}{1.020686in}}{\pgfqpoint{2.956454in}{1.009636in}}%
\pgfpathcurveto{\pgfqpoint{2.956454in}{0.998586in}}{\pgfqpoint{2.960845in}{0.987987in}}{\pgfqpoint{2.968658in}{0.980174in}}%
\pgfpathcurveto{\pgfqpoint{2.976472in}{0.972360in}}{\pgfqpoint{2.987071in}{0.967970in}}{\pgfqpoint{2.998121in}{0.967970in}}%
\pgfpathclose%
\pgfusepath{stroke,fill}%
\end{pgfscope}%
\begin{pgfscope}%
\pgfpathrectangle{\pgfqpoint{0.481978in}{0.331635in}}{\pgfqpoint{4.960000in}{3.696000in}}%
\pgfusepath{clip}%
\pgfsetbuttcap%
\pgfsetroundjoin%
\definecolor{currentfill}{rgb}{0.631373,0.788235,0.956863}%
\pgfsetfillcolor{currentfill}%
\pgfsetlinewidth{0.481800pt}%
\definecolor{currentstroke}{rgb}{1.000000,1.000000,1.000000}%
\pgfsetstrokecolor{currentstroke}%
\pgfsetdash{}{0pt}%
\pgfpathmoveto{\pgfqpoint{2.017138in}{0.858290in}}%
\pgfpathcurveto{\pgfqpoint{2.028188in}{0.858290in}}{\pgfqpoint{2.038787in}{0.862680in}}{\pgfqpoint{2.046601in}{0.870494in}}%
\pgfpathcurveto{\pgfqpoint{2.054415in}{0.878307in}}{\pgfqpoint{2.058805in}{0.888906in}}{\pgfqpoint{2.058805in}{0.899956in}}%
\pgfpathcurveto{\pgfqpoint{2.058805in}{0.911006in}}{\pgfqpoint{2.054415in}{0.921605in}}{\pgfqpoint{2.046601in}{0.929419in}}%
\pgfpathcurveto{\pgfqpoint{2.038787in}{0.937233in}}{\pgfqpoint{2.028188in}{0.941623in}}{\pgfqpoint{2.017138in}{0.941623in}}%
\pgfpathcurveto{\pgfqpoint{2.006088in}{0.941623in}}{\pgfqpoint{1.995489in}{0.937233in}}{\pgfqpoint{1.987676in}{0.929419in}}%
\pgfpathcurveto{\pgfqpoint{1.979862in}{0.921605in}}{\pgfqpoint{1.975472in}{0.911006in}}{\pgfqpoint{1.975472in}{0.899956in}}%
\pgfpathcurveto{\pgfqpoint{1.975472in}{0.888906in}}{\pgfqpoint{1.979862in}{0.878307in}}{\pgfqpoint{1.987676in}{0.870494in}}%
\pgfpathcurveto{\pgfqpoint{1.995489in}{0.862680in}}{\pgfqpoint{2.006088in}{0.858290in}}{\pgfqpoint{2.017138in}{0.858290in}}%
\pgfpathclose%
\pgfusepath{stroke,fill}%
\end{pgfscope}%
\begin{pgfscope}%
\pgfpathrectangle{\pgfqpoint{0.481978in}{0.331635in}}{\pgfqpoint{4.960000in}{3.696000in}}%
\pgfusepath{clip}%
\pgfsetbuttcap%
\pgfsetroundjoin%
\definecolor{currentfill}{rgb}{0.631373,0.788235,0.956863}%
\pgfsetfillcolor{currentfill}%
\pgfsetlinewidth{0.481800pt}%
\definecolor{currentstroke}{rgb}{1.000000,1.000000,1.000000}%
\pgfsetstrokecolor{currentstroke}%
\pgfsetdash{}{0pt}%
\pgfpathmoveto{\pgfqpoint{2.584736in}{1.041417in}}%
\pgfpathcurveto{\pgfqpoint{2.595786in}{1.041417in}}{\pgfqpoint{2.606385in}{1.045807in}}{\pgfqpoint{2.614199in}{1.053620in}}%
\pgfpathcurveto{\pgfqpoint{2.622012in}{1.061434in}}{\pgfqpoint{2.626402in}{1.072033in}}{\pgfqpoint{2.626402in}{1.083083in}}%
\pgfpathcurveto{\pgfqpoint{2.626402in}{1.094133in}}{\pgfqpoint{2.622012in}{1.104732in}}{\pgfqpoint{2.614199in}{1.112546in}}%
\pgfpathcurveto{\pgfqpoint{2.606385in}{1.120360in}}{\pgfqpoint{2.595786in}{1.124750in}}{\pgfqpoint{2.584736in}{1.124750in}}%
\pgfpathcurveto{\pgfqpoint{2.573686in}{1.124750in}}{\pgfqpoint{2.563087in}{1.120360in}}{\pgfqpoint{2.555273in}{1.112546in}}%
\pgfpathcurveto{\pgfqpoint{2.547459in}{1.104732in}}{\pgfqpoint{2.543069in}{1.094133in}}{\pgfqpoint{2.543069in}{1.083083in}}%
\pgfpathcurveto{\pgfqpoint{2.543069in}{1.072033in}}{\pgfqpoint{2.547459in}{1.061434in}}{\pgfqpoint{2.555273in}{1.053620in}}%
\pgfpathcurveto{\pgfqpoint{2.563087in}{1.045807in}}{\pgfqpoint{2.573686in}{1.041417in}}{\pgfqpoint{2.584736in}{1.041417in}}%
\pgfpathclose%
\pgfusepath{stroke,fill}%
\end{pgfscope}%
\begin{pgfscope}%
\pgfpathrectangle{\pgfqpoint{0.481978in}{0.331635in}}{\pgfqpoint{4.960000in}{3.696000in}}%
\pgfusepath{clip}%
\pgfsetbuttcap%
\pgfsetroundjoin%
\definecolor{currentfill}{rgb}{0.631373,0.788235,0.956863}%
\pgfsetfillcolor{currentfill}%
\pgfsetlinewidth{0.481800pt}%
\definecolor{currentstroke}{rgb}{1.000000,1.000000,1.000000}%
\pgfsetstrokecolor{currentstroke}%
\pgfsetdash{}{0pt}%
\pgfpathmoveto{\pgfqpoint{2.772380in}{0.909361in}}%
\pgfpathcurveto{\pgfqpoint{2.783430in}{0.909361in}}{\pgfqpoint{2.794029in}{0.913751in}}{\pgfqpoint{2.801843in}{0.921564in}}%
\pgfpathcurveto{\pgfqpoint{2.809657in}{0.929378in}}{\pgfqpoint{2.814047in}{0.939977in}}{\pgfqpoint{2.814047in}{0.951027in}}%
\pgfpathcurveto{\pgfqpoint{2.814047in}{0.962077in}}{\pgfqpoint{2.809657in}{0.972676in}}{\pgfqpoint{2.801843in}{0.980490in}}%
\pgfpathcurveto{\pgfqpoint{2.794029in}{0.988304in}}{\pgfqpoint{2.783430in}{0.992694in}}{\pgfqpoint{2.772380in}{0.992694in}}%
\pgfpathcurveto{\pgfqpoint{2.761330in}{0.992694in}}{\pgfqpoint{2.750731in}{0.988304in}}{\pgfqpoint{2.742917in}{0.980490in}}%
\pgfpathcurveto{\pgfqpoint{2.735104in}{0.972676in}}{\pgfqpoint{2.730713in}{0.962077in}}{\pgfqpoint{2.730713in}{0.951027in}}%
\pgfpathcurveto{\pgfqpoint{2.730713in}{0.939977in}}{\pgfqpoint{2.735104in}{0.929378in}}{\pgfqpoint{2.742917in}{0.921564in}}%
\pgfpathcurveto{\pgfqpoint{2.750731in}{0.913751in}}{\pgfqpoint{2.761330in}{0.909361in}}{\pgfqpoint{2.772380in}{0.909361in}}%
\pgfpathclose%
\pgfusepath{stroke,fill}%
\end{pgfscope}%
\begin{pgfscope}%
\pgfpathrectangle{\pgfqpoint{0.481978in}{0.331635in}}{\pgfqpoint{4.960000in}{3.696000in}}%
\pgfusepath{clip}%
\pgfsetbuttcap%
\pgfsetroundjoin%
\definecolor{currentfill}{rgb}{0.631373,0.788235,0.956863}%
\pgfsetfillcolor{currentfill}%
\pgfsetlinewidth{0.481800pt}%
\definecolor{currentstroke}{rgb}{1.000000,1.000000,1.000000}%
\pgfsetstrokecolor{currentstroke}%
\pgfsetdash{}{0pt}%
\pgfpathmoveto{\pgfqpoint{3.839237in}{1.765180in}}%
\pgfpathcurveto{\pgfqpoint{3.850287in}{1.765180in}}{\pgfqpoint{3.860886in}{1.769571in}}{\pgfqpoint{3.868699in}{1.777384in}}%
\pgfpathcurveto{\pgfqpoint{3.876513in}{1.785198in}}{\pgfqpoint{3.880903in}{1.795797in}}{\pgfqpoint{3.880903in}{1.806847in}}%
\pgfpathcurveto{\pgfqpoint{3.880903in}{1.817897in}}{\pgfqpoint{3.876513in}{1.828496in}}{\pgfqpoint{3.868699in}{1.836310in}}%
\pgfpathcurveto{\pgfqpoint{3.860886in}{1.844123in}}{\pgfqpoint{3.850287in}{1.848514in}}{\pgfqpoint{3.839237in}{1.848514in}}%
\pgfpathcurveto{\pgfqpoint{3.828186in}{1.848514in}}{\pgfqpoint{3.817587in}{1.844123in}}{\pgfqpoint{3.809774in}{1.836310in}}%
\pgfpathcurveto{\pgfqpoint{3.801960in}{1.828496in}}{\pgfqpoint{3.797570in}{1.817897in}}{\pgfqpoint{3.797570in}{1.806847in}}%
\pgfpathcurveto{\pgfqpoint{3.797570in}{1.795797in}}{\pgfqpoint{3.801960in}{1.785198in}}{\pgfqpoint{3.809774in}{1.777384in}}%
\pgfpathcurveto{\pgfqpoint{3.817587in}{1.769571in}}{\pgfqpoint{3.828186in}{1.765180in}}{\pgfqpoint{3.839237in}{1.765180in}}%
\pgfpathclose%
\pgfusepath{stroke,fill}%
\end{pgfscope}%
\begin{pgfscope}%
\pgfpathrectangle{\pgfqpoint{0.481978in}{0.331635in}}{\pgfqpoint{4.960000in}{3.696000in}}%
\pgfusepath{clip}%
\pgfsetbuttcap%
\pgfsetroundjoin%
\definecolor{currentfill}{rgb}{0.631373,0.788235,0.956863}%
\pgfsetfillcolor{currentfill}%
\pgfsetlinewidth{0.481800pt}%
\definecolor{currentstroke}{rgb}{1.000000,1.000000,1.000000}%
\pgfsetstrokecolor{currentstroke}%
\pgfsetdash{}{0pt}%
\pgfpathmoveto{\pgfqpoint{3.923632in}{3.422372in}}%
\pgfpathcurveto{\pgfqpoint{3.934682in}{3.422372in}}{\pgfqpoint{3.945281in}{3.426762in}}{\pgfqpoint{3.953094in}{3.434576in}}%
\pgfpathcurveto{\pgfqpoint{3.960908in}{3.442389in}}{\pgfqpoint{3.965298in}{3.452988in}}{\pgfqpoint{3.965298in}{3.464039in}}%
\pgfpathcurveto{\pgfqpoint{3.965298in}{3.475089in}}{\pgfqpoint{3.960908in}{3.485688in}}{\pgfqpoint{3.953094in}{3.493501in}}%
\pgfpathcurveto{\pgfqpoint{3.945281in}{3.501315in}}{\pgfqpoint{3.934682in}{3.505705in}}{\pgfqpoint{3.923632in}{3.505705in}}%
\pgfpathcurveto{\pgfqpoint{3.912581in}{3.505705in}}{\pgfqpoint{3.901982in}{3.501315in}}{\pgfqpoint{3.894169in}{3.493501in}}%
\pgfpathcurveto{\pgfqpoint{3.886355in}{3.485688in}}{\pgfqpoint{3.881965in}{3.475089in}}{\pgfqpoint{3.881965in}{3.464039in}}%
\pgfpathcurveto{\pgfqpoint{3.881965in}{3.452988in}}{\pgfqpoint{3.886355in}{3.442389in}}{\pgfqpoint{3.894169in}{3.434576in}}%
\pgfpathcurveto{\pgfqpoint{3.901982in}{3.426762in}}{\pgfqpoint{3.912581in}{3.422372in}}{\pgfqpoint{3.923632in}{3.422372in}}%
\pgfpathclose%
\pgfusepath{stroke,fill}%
\end{pgfscope}%
\begin{pgfscope}%
\pgfpathrectangle{\pgfqpoint{0.481978in}{0.331635in}}{\pgfqpoint{4.960000in}{3.696000in}}%
\pgfusepath{clip}%
\pgfsetbuttcap%
\pgfsetroundjoin%
\definecolor{currentfill}{rgb}{0.631373,0.788235,0.956863}%
\pgfsetfillcolor{currentfill}%
\pgfsetlinewidth{0.481800pt}%
\definecolor{currentstroke}{rgb}{1.000000,1.000000,1.000000}%
\pgfsetstrokecolor{currentstroke}%
\pgfsetdash{}{0pt}%
\pgfpathmoveto{\pgfqpoint{3.682114in}{1.828641in}}%
\pgfpathcurveto{\pgfqpoint{3.693164in}{1.828641in}}{\pgfqpoint{3.703763in}{1.833032in}}{\pgfqpoint{3.711577in}{1.840845in}}%
\pgfpathcurveto{\pgfqpoint{3.719390in}{1.848659in}}{\pgfqpoint{3.723780in}{1.859258in}}{\pgfqpoint{3.723780in}{1.870308in}}%
\pgfpathcurveto{\pgfqpoint{3.723780in}{1.881358in}}{\pgfqpoint{3.719390in}{1.891957in}}{\pgfqpoint{3.711577in}{1.899771in}}%
\pgfpathcurveto{\pgfqpoint{3.703763in}{1.907584in}}{\pgfqpoint{3.693164in}{1.911975in}}{\pgfqpoint{3.682114in}{1.911975in}}%
\pgfpathcurveto{\pgfqpoint{3.671064in}{1.911975in}}{\pgfqpoint{3.660465in}{1.907584in}}{\pgfqpoint{3.652651in}{1.899771in}}%
\pgfpathcurveto{\pgfqpoint{3.644837in}{1.891957in}}{\pgfqpoint{3.640447in}{1.881358in}}{\pgfqpoint{3.640447in}{1.870308in}}%
\pgfpathcurveto{\pgfqpoint{3.640447in}{1.859258in}}{\pgfqpoint{3.644837in}{1.848659in}}{\pgfqpoint{3.652651in}{1.840845in}}%
\pgfpathcurveto{\pgfqpoint{3.660465in}{1.833032in}}{\pgfqpoint{3.671064in}{1.828641in}}{\pgfqpoint{3.682114in}{1.828641in}}%
\pgfpathclose%
\pgfusepath{stroke,fill}%
\end{pgfscope}%
\begin{pgfscope}%
\pgfpathrectangle{\pgfqpoint{0.481978in}{0.331635in}}{\pgfqpoint{4.960000in}{3.696000in}}%
\pgfusepath{clip}%
\pgfsetbuttcap%
\pgfsetroundjoin%
\definecolor{currentfill}{rgb}{0.631373,0.788235,0.956863}%
\pgfsetfillcolor{currentfill}%
\pgfsetlinewidth{0.481800pt}%
\definecolor{currentstroke}{rgb}{1.000000,1.000000,1.000000}%
\pgfsetstrokecolor{currentstroke}%
\pgfsetdash{}{0pt}%
\pgfpathmoveto{\pgfqpoint{3.060682in}{1.630962in}}%
\pgfpathcurveto{\pgfqpoint{3.071732in}{1.630962in}}{\pgfqpoint{3.082331in}{1.635353in}}{\pgfqpoint{3.090145in}{1.643166in}}%
\pgfpathcurveto{\pgfqpoint{3.097959in}{1.650980in}}{\pgfqpoint{3.102349in}{1.661579in}}{\pgfqpoint{3.102349in}{1.672629in}}%
\pgfpathcurveto{\pgfqpoint{3.102349in}{1.683679in}}{\pgfqpoint{3.097959in}{1.694278in}}{\pgfqpoint{3.090145in}{1.702092in}}%
\pgfpathcurveto{\pgfqpoint{3.082331in}{1.709906in}}{\pgfqpoint{3.071732in}{1.714296in}}{\pgfqpoint{3.060682in}{1.714296in}}%
\pgfpathcurveto{\pgfqpoint{3.049632in}{1.714296in}}{\pgfqpoint{3.039033in}{1.709906in}}{\pgfqpoint{3.031220in}{1.702092in}}%
\pgfpathcurveto{\pgfqpoint{3.023406in}{1.694278in}}{\pgfqpoint{3.019016in}{1.683679in}}{\pgfqpoint{3.019016in}{1.672629in}}%
\pgfpathcurveto{\pgfqpoint{3.019016in}{1.661579in}}{\pgfqpoint{3.023406in}{1.650980in}}{\pgfqpoint{3.031220in}{1.643166in}}%
\pgfpathcurveto{\pgfqpoint{3.039033in}{1.635353in}}{\pgfqpoint{3.049632in}{1.630962in}}{\pgfqpoint{3.060682in}{1.630962in}}%
\pgfpathclose%
\pgfusepath{stroke,fill}%
\end{pgfscope}%
\begin{pgfscope}%
\pgfpathrectangle{\pgfqpoint{0.481978in}{0.331635in}}{\pgfqpoint{4.960000in}{3.696000in}}%
\pgfusepath{clip}%
\pgfsetbuttcap%
\pgfsetroundjoin%
\definecolor{currentfill}{rgb}{0.631373,0.788235,0.956863}%
\pgfsetfillcolor{currentfill}%
\pgfsetlinewidth{0.481800pt}%
\definecolor{currentstroke}{rgb}{1.000000,1.000000,1.000000}%
\pgfsetstrokecolor{currentstroke}%
\pgfsetdash{}{0pt}%
\pgfpathmoveto{\pgfqpoint{4.145626in}{2.788656in}}%
\pgfpathcurveto{\pgfqpoint{4.156676in}{2.788656in}}{\pgfqpoint{4.167275in}{2.793046in}}{\pgfqpoint{4.175089in}{2.800860in}}%
\pgfpathcurveto{\pgfqpoint{4.182902in}{2.808673in}}{\pgfqpoint{4.187292in}{2.819272in}}{\pgfqpoint{4.187292in}{2.830323in}}%
\pgfpathcurveto{\pgfqpoint{4.187292in}{2.841373in}}{\pgfqpoint{4.182902in}{2.851972in}}{\pgfqpoint{4.175089in}{2.859785in}}%
\pgfpathcurveto{\pgfqpoint{4.167275in}{2.867599in}}{\pgfqpoint{4.156676in}{2.871989in}}{\pgfqpoint{4.145626in}{2.871989in}}%
\pgfpathcurveto{\pgfqpoint{4.134576in}{2.871989in}}{\pgfqpoint{4.123977in}{2.867599in}}{\pgfqpoint{4.116163in}{2.859785in}}%
\pgfpathcurveto{\pgfqpoint{4.108349in}{2.851972in}}{\pgfqpoint{4.103959in}{2.841373in}}{\pgfqpoint{4.103959in}{2.830323in}}%
\pgfpathcurveto{\pgfqpoint{4.103959in}{2.819272in}}{\pgfqpoint{4.108349in}{2.808673in}}{\pgfqpoint{4.116163in}{2.800860in}}%
\pgfpathcurveto{\pgfqpoint{4.123977in}{2.793046in}}{\pgfqpoint{4.134576in}{2.788656in}}{\pgfqpoint{4.145626in}{2.788656in}}%
\pgfpathclose%
\pgfusepath{stroke,fill}%
\end{pgfscope}%
\begin{pgfscope}%
\pgfpathrectangle{\pgfqpoint{0.481978in}{0.331635in}}{\pgfqpoint{4.960000in}{3.696000in}}%
\pgfusepath{clip}%
\pgfsetbuttcap%
\pgfsetroundjoin%
\definecolor{currentfill}{rgb}{0.631373,0.788235,0.956863}%
\pgfsetfillcolor{currentfill}%
\pgfsetlinewidth{0.481800pt}%
\definecolor{currentstroke}{rgb}{1.000000,1.000000,1.000000}%
\pgfsetstrokecolor{currentstroke}%
\pgfsetdash{}{0pt}%
\pgfpathmoveto{\pgfqpoint{2.477464in}{0.728304in}}%
\pgfpathcurveto{\pgfqpoint{2.488514in}{0.728304in}}{\pgfqpoint{2.499113in}{0.732694in}}{\pgfqpoint{2.506926in}{0.740508in}}%
\pgfpathcurveto{\pgfqpoint{2.514740in}{0.748321in}}{\pgfqpoint{2.519130in}{0.758920in}}{\pgfqpoint{2.519130in}{0.769971in}}%
\pgfpathcurveto{\pgfqpoint{2.519130in}{0.781021in}}{\pgfqpoint{2.514740in}{0.791620in}}{\pgfqpoint{2.506926in}{0.799433in}}%
\pgfpathcurveto{\pgfqpoint{2.499113in}{0.807247in}}{\pgfqpoint{2.488514in}{0.811637in}}{\pgfqpoint{2.477464in}{0.811637in}}%
\pgfpathcurveto{\pgfqpoint{2.466414in}{0.811637in}}{\pgfqpoint{2.455815in}{0.807247in}}{\pgfqpoint{2.448001in}{0.799433in}}%
\pgfpathcurveto{\pgfqpoint{2.440187in}{0.791620in}}{\pgfqpoint{2.435797in}{0.781021in}}{\pgfqpoint{2.435797in}{0.769971in}}%
\pgfpathcurveto{\pgfqpoint{2.435797in}{0.758920in}}{\pgfqpoint{2.440187in}{0.748321in}}{\pgfqpoint{2.448001in}{0.740508in}}%
\pgfpathcurveto{\pgfqpoint{2.455815in}{0.732694in}}{\pgfqpoint{2.466414in}{0.728304in}}{\pgfqpoint{2.477464in}{0.728304in}}%
\pgfpathclose%
\pgfusepath{stroke,fill}%
\end{pgfscope}%
\begin{pgfscope}%
\pgfpathrectangle{\pgfqpoint{0.481978in}{0.331635in}}{\pgfqpoint{4.960000in}{3.696000in}}%
\pgfusepath{clip}%
\pgfsetbuttcap%
\pgfsetroundjoin%
\definecolor{currentfill}{rgb}{0.631373,0.788235,0.956863}%
\pgfsetfillcolor{currentfill}%
\pgfsetlinewidth{0.481800pt}%
\definecolor{currentstroke}{rgb}{1.000000,1.000000,1.000000}%
\pgfsetstrokecolor{currentstroke}%
\pgfsetdash{}{0pt}%
\pgfpathmoveto{\pgfqpoint{3.293279in}{2.570576in}}%
\pgfpathcurveto{\pgfqpoint{3.304329in}{2.570576in}}{\pgfqpoint{3.314928in}{2.574966in}}{\pgfqpoint{3.322741in}{2.582779in}}%
\pgfpathcurveto{\pgfqpoint{3.330555in}{2.590593in}}{\pgfqpoint{3.334945in}{2.601192in}}{\pgfqpoint{3.334945in}{2.612242in}}%
\pgfpathcurveto{\pgfqpoint{3.334945in}{2.623292in}}{\pgfqpoint{3.330555in}{2.633891in}}{\pgfqpoint{3.322741in}{2.641705in}}%
\pgfpathcurveto{\pgfqpoint{3.314928in}{2.649519in}}{\pgfqpoint{3.304329in}{2.653909in}}{\pgfqpoint{3.293279in}{2.653909in}}%
\pgfpathcurveto{\pgfqpoint{3.282229in}{2.653909in}}{\pgfqpoint{3.271629in}{2.649519in}}{\pgfqpoint{3.263816in}{2.641705in}}%
\pgfpathcurveto{\pgfqpoint{3.256002in}{2.633891in}}{\pgfqpoint{3.251612in}{2.623292in}}{\pgfqpoint{3.251612in}{2.612242in}}%
\pgfpathcurveto{\pgfqpoint{3.251612in}{2.601192in}}{\pgfqpoint{3.256002in}{2.590593in}}{\pgfqpoint{3.263816in}{2.582779in}}%
\pgfpathcurveto{\pgfqpoint{3.271629in}{2.574966in}}{\pgfqpoint{3.282229in}{2.570576in}}{\pgfqpoint{3.293279in}{2.570576in}}%
\pgfpathclose%
\pgfusepath{stroke,fill}%
\end{pgfscope}%
\begin{pgfscope}%
\pgfpathrectangle{\pgfqpoint{0.481978in}{0.331635in}}{\pgfqpoint{4.960000in}{3.696000in}}%
\pgfusepath{clip}%
\pgfsetbuttcap%
\pgfsetroundjoin%
\definecolor{currentfill}{rgb}{0.631373,0.788235,0.956863}%
\pgfsetfillcolor{currentfill}%
\pgfsetlinewidth{0.481800pt}%
\definecolor{currentstroke}{rgb}{1.000000,1.000000,1.000000}%
\pgfsetstrokecolor{currentstroke}%
\pgfsetdash{}{0pt}%
\pgfpathmoveto{\pgfqpoint{2.485627in}{0.663880in}}%
\pgfpathcurveto{\pgfqpoint{2.496677in}{0.663880in}}{\pgfqpoint{2.507276in}{0.668270in}}{\pgfqpoint{2.515089in}{0.676084in}}%
\pgfpathcurveto{\pgfqpoint{2.522903in}{0.683898in}}{\pgfqpoint{2.527293in}{0.694497in}}{\pgfqpoint{2.527293in}{0.705547in}}%
\pgfpathcurveto{\pgfqpoint{2.527293in}{0.716597in}}{\pgfqpoint{2.522903in}{0.727196in}}{\pgfqpoint{2.515089in}{0.735010in}}%
\pgfpathcurveto{\pgfqpoint{2.507276in}{0.742823in}}{\pgfqpoint{2.496677in}{0.747213in}}{\pgfqpoint{2.485627in}{0.747213in}}%
\pgfpathcurveto{\pgfqpoint{2.474576in}{0.747213in}}{\pgfqpoint{2.463977in}{0.742823in}}{\pgfqpoint{2.456164in}{0.735010in}}%
\pgfpathcurveto{\pgfqpoint{2.448350in}{0.727196in}}{\pgfqpoint{2.443960in}{0.716597in}}{\pgfqpoint{2.443960in}{0.705547in}}%
\pgfpathcurveto{\pgfqpoint{2.443960in}{0.694497in}}{\pgfqpoint{2.448350in}{0.683898in}}{\pgfqpoint{2.456164in}{0.676084in}}%
\pgfpathcurveto{\pgfqpoint{2.463977in}{0.668270in}}{\pgfqpoint{2.474576in}{0.663880in}}{\pgfqpoint{2.485627in}{0.663880in}}%
\pgfpathclose%
\pgfusepath{stroke,fill}%
\end{pgfscope}%
\begin{pgfscope}%
\pgfpathrectangle{\pgfqpoint{0.481978in}{0.331635in}}{\pgfqpoint{4.960000in}{3.696000in}}%
\pgfusepath{clip}%
\pgfsetbuttcap%
\pgfsetroundjoin%
\definecolor{currentfill}{rgb}{0.631373,0.788235,0.956863}%
\pgfsetfillcolor{currentfill}%
\pgfsetlinewidth{0.481800pt}%
\definecolor{currentstroke}{rgb}{1.000000,1.000000,1.000000}%
\pgfsetstrokecolor{currentstroke}%
\pgfsetdash{}{0pt}%
\pgfpathmoveto{\pgfqpoint{5.020912in}{2.273540in}}%
\pgfpathcurveto{\pgfqpoint{5.031962in}{2.273540in}}{\pgfqpoint{5.042562in}{2.277930in}}{\pgfqpoint{5.050375in}{2.285743in}}%
\pgfpathcurveto{\pgfqpoint{5.058189in}{2.293557in}}{\pgfqpoint{5.062579in}{2.304156in}}{\pgfqpoint{5.062579in}{2.315206in}}%
\pgfpathcurveto{\pgfqpoint{5.062579in}{2.326256in}}{\pgfqpoint{5.058189in}{2.336855in}}{\pgfqpoint{5.050375in}{2.344669in}}%
\pgfpathcurveto{\pgfqpoint{5.042562in}{2.352483in}}{\pgfqpoint{5.031962in}{2.356873in}}{\pgfqpoint{5.020912in}{2.356873in}}%
\pgfpathcurveto{\pgfqpoint{5.009862in}{2.356873in}}{\pgfqpoint{4.999263in}{2.352483in}}{\pgfqpoint{4.991450in}{2.344669in}}%
\pgfpathcurveto{\pgfqpoint{4.983636in}{2.336855in}}{\pgfqpoint{4.979246in}{2.326256in}}{\pgfqpoint{4.979246in}{2.315206in}}%
\pgfpathcurveto{\pgfqpoint{4.979246in}{2.304156in}}{\pgfqpoint{4.983636in}{2.293557in}}{\pgfqpoint{4.991450in}{2.285743in}}%
\pgfpathcurveto{\pgfqpoint{4.999263in}{2.277930in}}{\pgfqpoint{5.009862in}{2.273540in}}{\pgfqpoint{5.020912in}{2.273540in}}%
\pgfpathclose%
\pgfusepath{stroke,fill}%
\end{pgfscope}%
\begin{pgfscope}%
\pgfpathrectangle{\pgfqpoint{0.481978in}{0.331635in}}{\pgfqpoint{4.960000in}{3.696000in}}%
\pgfusepath{clip}%
\pgfsetbuttcap%
\pgfsetroundjoin%
\definecolor{currentfill}{rgb}{0.631373,0.788235,0.956863}%
\pgfsetfillcolor{currentfill}%
\pgfsetlinewidth{0.481800pt}%
\definecolor{currentstroke}{rgb}{1.000000,1.000000,1.000000}%
\pgfsetstrokecolor{currentstroke}%
\pgfsetdash{}{0pt}%
\pgfpathmoveto{\pgfqpoint{4.054103in}{1.933070in}}%
\pgfpathcurveto{\pgfqpoint{4.065153in}{1.933070in}}{\pgfqpoint{4.075752in}{1.937461in}}{\pgfqpoint{4.083566in}{1.945274in}}%
\pgfpathcurveto{\pgfqpoint{4.091380in}{1.953088in}}{\pgfqpoint{4.095770in}{1.963687in}}{\pgfqpoint{4.095770in}{1.974737in}}%
\pgfpathcurveto{\pgfqpoint{4.095770in}{1.985787in}}{\pgfqpoint{4.091380in}{1.996386in}}{\pgfqpoint{4.083566in}{2.004200in}}%
\pgfpathcurveto{\pgfqpoint{4.075752in}{2.012013in}}{\pgfqpoint{4.065153in}{2.016404in}}{\pgfqpoint{4.054103in}{2.016404in}}%
\pgfpathcurveto{\pgfqpoint{4.043053in}{2.016404in}}{\pgfqpoint{4.032454in}{2.012013in}}{\pgfqpoint{4.024640in}{2.004200in}}%
\pgfpathcurveto{\pgfqpoint{4.016827in}{1.996386in}}{\pgfqpoint{4.012437in}{1.985787in}}{\pgfqpoint{4.012437in}{1.974737in}}%
\pgfpathcurveto{\pgfqpoint{4.012437in}{1.963687in}}{\pgfqpoint{4.016827in}{1.953088in}}{\pgfqpoint{4.024640in}{1.945274in}}%
\pgfpathcurveto{\pgfqpoint{4.032454in}{1.937461in}}{\pgfqpoint{4.043053in}{1.933070in}}{\pgfqpoint{4.054103in}{1.933070in}}%
\pgfpathclose%
\pgfusepath{stroke,fill}%
\end{pgfscope}%
\begin{pgfscope}%
\pgfpathrectangle{\pgfqpoint{0.481978in}{0.331635in}}{\pgfqpoint{4.960000in}{3.696000in}}%
\pgfusepath{clip}%
\pgfsetbuttcap%
\pgfsetroundjoin%
\definecolor{currentfill}{rgb}{0.631373,0.788235,0.956863}%
\pgfsetfillcolor{currentfill}%
\pgfsetlinewidth{0.481800pt}%
\definecolor{currentstroke}{rgb}{1.000000,1.000000,1.000000}%
\pgfsetstrokecolor{currentstroke}%
\pgfsetdash{}{0pt}%
\pgfpathmoveto{\pgfqpoint{3.589725in}{1.523129in}}%
\pgfpathcurveto{\pgfqpoint{3.600775in}{1.523129in}}{\pgfqpoint{3.611374in}{1.527519in}}{\pgfqpoint{3.619188in}{1.535333in}}%
\pgfpathcurveto{\pgfqpoint{3.627002in}{1.543146in}}{\pgfqpoint{3.631392in}{1.553745in}}{\pgfqpoint{3.631392in}{1.564796in}}%
\pgfpathcurveto{\pgfqpoint{3.631392in}{1.575846in}}{\pgfqpoint{3.627002in}{1.586445in}}{\pgfqpoint{3.619188in}{1.594258in}}%
\pgfpathcurveto{\pgfqpoint{3.611374in}{1.602072in}}{\pgfqpoint{3.600775in}{1.606462in}}{\pgfqpoint{3.589725in}{1.606462in}}%
\pgfpathcurveto{\pgfqpoint{3.578675in}{1.606462in}}{\pgfqpoint{3.568076in}{1.602072in}}{\pgfqpoint{3.560262in}{1.594258in}}%
\pgfpathcurveto{\pgfqpoint{3.552449in}{1.586445in}}{\pgfqpoint{3.548059in}{1.575846in}}{\pgfqpoint{3.548059in}{1.564796in}}%
\pgfpathcurveto{\pgfqpoint{3.548059in}{1.553745in}}{\pgfqpoint{3.552449in}{1.543146in}}{\pgfqpoint{3.560262in}{1.535333in}}%
\pgfpathcurveto{\pgfqpoint{3.568076in}{1.527519in}}{\pgfqpoint{3.578675in}{1.523129in}}{\pgfqpoint{3.589725in}{1.523129in}}%
\pgfpathclose%
\pgfusepath{stroke,fill}%
\end{pgfscope}%
\begin{pgfscope}%
\pgfpathrectangle{\pgfqpoint{0.481978in}{0.331635in}}{\pgfqpoint{4.960000in}{3.696000in}}%
\pgfusepath{clip}%
\pgfsetbuttcap%
\pgfsetroundjoin%
\definecolor{currentfill}{rgb}{0.631373,0.788235,0.956863}%
\pgfsetfillcolor{currentfill}%
\pgfsetlinewidth{0.481800pt}%
\definecolor{currentstroke}{rgb}{1.000000,1.000000,1.000000}%
\pgfsetstrokecolor{currentstroke}%
\pgfsetdash{}{0pt}%
\pgfpathmoveto{\pgfqpoint{2.685295in}{1.196274in}}%
\pgfpathcurveto{\pgfqpoint{2.696345in}{1.196274in}}{\pgfqpoint{2.706944in}{1.200664in}}{\pgfqpoint{2.714757in}{1.208478in}}%
\pgfpathcurveto{\pgfqpoint{2.722571in}{1.216291in}}{\pgfqpoint{2.726961in}{1.226890in}}{\pgfqpoint{2.726961in}{1.237940in}}%
\pgfpathcurveto{\pgfqpoint{2.726961in}{1.248990in}}{\pgfqpoint{2.722571in}{1.259590in}}{\pgfqpoint{2.714757in}{1.267403in}}%
\pgfpathcurveto{\pgfqpoint{2.706944in}{1.275217in}}{\pgfqpoint{2.696345in}{1.279607in}}{\pgfqpoint{2.685295in}{1.279607in}}%
\pgfpathcurveto{\pgfqpoint{2.674245in}{1.279607in}}{\pgfqpoint{2.663646in}{1.275217in}}{\pgfqpoint{2.655832in}{1.267403in}}%
\pgfpathcurveto{\pgfqpoint{2.648018in}{1.259590in}}{\pgfqpoint{2.643628in}{1.248990in}}{\pgfqpoint{2.643628in}{1.237940in}}%
\pgfpathcurveto{\pgfqpoint{2.643628in}{1.226890in}}{\pgfqpoint{2.648018in}{1.216291in}}{\pgfqpoint{2.655832in}{1.208478in}}%
\pgfpathcurveto{\pgfqpoint{2.663646in}{1.200664in}}{\pgfqpoint{2.674245in}{1.196274in}}{\pgfqpoint{2.685295in}{1.196274in}}%
\pgfpathclose%
\pgfusepath{stroke,fill}%
\end{pgfscope}%
\begin{pgfscope}%
\pgfpathrectangle{\pgfqpoint{0.481978in}{0.331635in}}{\pgfqpoint{4.960000in}{3.696000in}}%
\pgfusepath{clip}%
\pgfsetbuttcap%
\pgfsetroundjoin%
\definecolor{currentfill}{rgb}{0.631373,0.788235,0.956863}%
\pgfsetfillcolor{currentfill}%
\pgfsetlinewidth{0.481800pt}%
\definecolor{currentstroke}{rgb}{1.000000,1.000000,1.000000}%
\pgfsetstrokecolor{currentstroke}%
\pgfsetdash{}{0pt}%
\pgfpathmoveto{\pgfqpoint{4.076072in}{3.064135in}}%
\pgfpathcurveto{\pgfqpoint{4.087123in}{3.064135in}}{\pgfqpoint{4.097722in}{3.068525in}}{\pgfqpoint{4.105535in}{3.076339in}}%
\pgfpathcurveto{\pgfqpoint{4.113349in}{3.084152in}}{\pgfqpoint{4.117739in}{3.094751in}}{\pgfqpoint{4.117739in}{3.105802in}}%
\pgfpathcurveto{\pgfqpoint{4.117739in}{3.116852in}}{\pgfqpoint{4.113349in}{3.127451in}}{\pgfqpoint{4.105535in}{3.135264in}}%
\pgfpathcurveto{\pgfqpoint{4.097722in}{3.143078in}}{\pgfqpoint{4.087123in}{3.147468in}}{\pgfqpoint{4.076072in}{3.147468in}}%
\pgfpathcurveto{\pgfqpoint{4.065022in}{3.147468in}}{\pgfqpoint{4.054423in}{3.143078in}}{\pgfqpoint{4.046610in}{3.135264in}}%
\pgfpathcurveto{\pgfqpoint{4.038796in}{3.127451in}}{\pgfqpoint{4.034406in}{3.116852in}}{\pgfqpoint{4.034406in}{3.105802in}}%
\pgfpathcurveto{\pgfqpoint{4.034406in}{3.094751in}}{\pgfqpoint{4.038796in}{3.084152in}}{\pgfqpoint{4.046610in}{3.076339in}}%
\pgfpathcurveto{\pgfqpoint{4.054423in}{3.068525in}}{\pgfqpoint{4.065022in}{3.064135in}}{\pgfqpoint{4.076072in}{3.064135in}}%
\pgfpathclose%
\pgfusepath{stroke,fill}%
\end{pgfscope}%
\begin{pgfscope}%
\pgfpathrectangle{\pgfqpoint{0.481978in}{0.331635in}}{\pgfqpoint{4.960000in}{3.696000in}}%
\pgfusepath{clip}%
\pgfsetbuttcap%
\pgfsetroundjoin%
\definecolor{currentfill}{rgb}{0.631373,0.788235,0.956863}%
\pgfsetfillcolor{currentfill}%
\pgfsetlinewidth{0.481800pt}%
\definecolor{currentstroke}{rgb}{1.000000,1.000000,1.000000}%
\pgfsetstrokecolor{currentstroke}%
\pgfsetdash{}{0pt}%
\pgfpathmoveto{\pgfqpoint{4.014806in}{2.122909in}}%
\pgfpathcurveto{\pgfqpoint{4.025857in}{2.122909in}}{\pgfqpoint{4.036456in}{2.127299in}}{\pgfqpoint{4.044269in}{2.135112in}}%
\pgfpathcurveto{\pgfqpoint{4.052083in}{2.142926in}}{\pgfqpoint{4.056473in}{2.153525in}}{\pgfqpoint{4.056473in}{2.164575in}}%
\pgfpathcurveto{\pgfqpoint{4.056473in}{2.175625in}}{\pgfqpoint{4.052083in}{2.186224in}}{\pgfqpoint{4.044269in}{2.194038in}}%
\pgfpathcurveto{\pgfqpoint{4.036456in}{2.201852in}}{\pgfqpoint{4.025857in}{2.206242in}}{\pgfqpoint{4.014806in}{2.206242in}}%
\pgfpathcurveto{\pgfqpoint{4.003756in}{2.206242in}}{\pgfqpoint{3.993157in}{2.201852in}}{\pgfqpoint{3.985344in}{2.194038in}}%
\pgfpathcurveto{\pgfqpoint{3.977530in}{2.186224in}}{\pgfqpoint{3.973140in}{2.175625in}}{\pgfqpoint{3.973140in}{2.164575in}}%
\pgfpathcurveto{\pgfqpoint{3.973140in}{2.153525in}}{\pgfqpoint{3.977530in}{2.142926in}}{\pgfqpoint{3.985344in}{2.135112in}}%
\pgfpathcurveto{\pgfqpoint{3.993157in}{2.127299in}}{\pgfqpoint{4.003756in}{2.122909in}}{\pgfqpoint{4.014806in}{2.122909in}}%
\pgfpathclose%
\pgfusepath{stroke,fill}%
\end{pgfscope}%
\begin{pgfscope}%
\pgfpathrectangle{\pgfqpoint{0.481978in}{0.331635in}}{\pgfqpoint{4.960000in}{3.696000in}}%
\pgfusepath{clip}%
\pgfsetbuttcap%
\pgfsetroundjoin%
\definecolor{currentfill}{rgb}{0.631373,0.788235,0.956863}%
\pgfsetfillcolor{currentfill}%
\pgfsetlinewidth{0.481800pt}%
\definecolor{currentstroke}{rgb}{1.000000,1.000000,1.000000}%
\pgfsetstrokecolor{currentstroke}%
\pgfsetdash{}{0pt}%
\pgfpathmoveto{\pgfqpoint{3.669689in}{2.288306in}}%
\pgfpathcurveto{\pgfqpoint{3.680740in}{2.288306in}}{\pgfqpoint{3.691339in}{2.292696in}}{\pgfqpoint{3.699152in}{2.300509in}}%
\pgfpathcurveto{\pgfqpoint{3.706966in}{2.308323in}}{\pgfqpoint{3.711356in}{2.318922in}}{\pgfqpoint{3.711356in}{2.329972in}}%
\pgfpathcurveto{\pgfqpoint{3.711356in}{2.341022in}}{\pgfqpoint{3.706966in}{2.351621in}}{\pgfqpoint{3.699152in}{2.359435in}}%
\pgfpathcurveto{\pgfqpoint{3.691339in}{2.367249in}}{\pgfqpoint{3.680740in}{2.371639in}}{\pgfqpoint{3.669689in}{2.371639in}}%
\pgfpathcurveto{\pgfqpoint{3.658639in}{2.371639in}}{\pgfqpoint{3.648040in}{2.367249in}}{\pgfqpoint{3.640227in}{2.359435in}}%
\pgfpathcurveto{\pgfqpoint{3.632413in}{2.351621in}}{\pgfqpoint{3.628023in}{2.341022in}}{\pgfqpoint{3.628023in}{2.329972in}}%
\pgfpathcurveto{\pgfqpoint{3.628023in}{2.318922in}}{\pgfqpoint{3.632413in}{2.308323in}}{\pgfqpoint{3.640227in}{2.300509in}}%
\pgfpathcurveto{\pgfqpoint{3.648040in}{2.292696in}}{\pgfqpoint{3.658639in}{2.288306in}}{\pgfqpoint{3.669689in}{2.288306in}}%
\pgfpathclose%
\pgfusepath{stroke,fill}%
\end{pgfscope}%
\begin{pgfscope}%
\pgfpathrectangle{\pgfqpoint{0.481978in}{0.331635in}}{\pgfqpoint{4.960000in}{3.696000in}}%
\pgfusepath{clip}%
\pgfsetbuttcap%
\pgfsetroundjoin%
\definecolor{currentfill}{rgb}{0.631373,0.788235,0.956863}%
\pgfsetfillcolor{currentfill}%
\pgfsetlinewidth{0.481800pt}%
\definecolor{currentstroke}{rgb}{1.000000,1.000000,1.000000}%
\pgfsetstrokecolor{currentstroke}%
\pgfsetdash{}{0pt}%
\pgfpathmoveto{\pgfqpoint{3.792597in}{2.209995in}}%
\pgfpathcurveto{\pgfqpoint{3.803647in}{2.209995in}}{\pgfqpoint{3.814247in}{2.214385in}}{\pgfqpoint{3.822060in}{2.222199in}}%
\pgfpathcurveto{\pgfqpoint{3.829874in}{2.230013in}}{\pgfqpoint{3.834264in}{2.240612in}}{\pgfqpoint{3.834264in}{2.251662in}}%
\pgfpathcurveto{\pgfqpoint{3.834264in}{2.262712in}}{\pgfqpoint{3.829874in}{2.273311in}}{\pgfqpoint{3.822060in}{2.281124in}}%
\pgfpathcurveto{\pgfqpoint{3.814247in}{2.288938in}}{\pgfqpoint{3.803647in}{2.293328in}}{\pgfqpoint{3.792597in}{2.293328in}}%
\pgfpathcurveto{\pgfqpoint{3.781547in}{2.293328in}}{\pgfqpoint{3.770948in}{2.288938in}}{\pgfqpoint{3.763135in}{2.281124in}}%
\pgfpathcurveto{\pgfqpoint{3.755321in}{2.273311in}}{\pgfqpoint{3.750931in}{2.262712in}}{\pgfqpoint{3.750931in}{2.251662in}}%
\pgfpathcurveto{\pgfqpoint{3.750931in}{2.240612in}}{\pgfqpoint{3.755321in}{2.230013in}}{\pgfqpoint{3.763135in}{2.222199in}}%
\pgfpathcurveto{\pgfqpoint{3.770948in}{2.214385in}}{\pgfqpoint{3.781547in}{2.209995in}}{\pgfqpoint{3.792597in}{2.209995in}}%
\pgfpathclose%
\pgfusepath{stroke,fill}%
\end{pgfscope}%
\begin{pgfscope}%
\pgfpathrectangle{\pgfqpoint{0.481978in}{0.331635in}}{\pgfqpoint{4.960000in}{3.696000in}}%
\pgfusepath{clip}%
\pgfsetbuttcap%
\pgfsetroundjoin%
\definecolor{currentfill}{rgb}{0.631373,0.788235,0.956863}%
\pgfsetfillcolor{currentfill}%
\pgfsetlinewidth{0.481800pt}%
\definecolor{currentstroke}{rgb}{1.000000,1.000000,1.000000}%
\pgfsetstrokecolor{currentstroke}%
\pgfsetdash{}{0pt}%
\pgfpathmoveto{\pgfqpoint{3.517960in}{2.867277in}}%
\pgfpathcurveto{\pgfqpoint{3.529010in}{2.867277in}}{\pgfqpoint{3.539609in}{2.871667in}}{\pgfqpoint{3.547423in}{2.879481in}}%
\pgfpathcurveto{\pgfqpoint{3.555236in}{2.887294in}}{\pgfqpoint{3.559626in}{2.897894in}}{\pgfqpoint{3.559626in}{2.908944in}}%
\pgfpathcurveto{\pgfqpoint{3.559626in}{2.919994in}}{\pgfqpoint{3.555236in}{2.930593in}}{\pgfqpoint{3.547423in}{2.938406in}}%
\pgfpathcurveto{\pgfqpoint{3.539609in}{2.946220in}}{\pgfqpoint{3.529010in}{2.950610in}}{\pgfqpoint{3.517960in}{2.950610in}}%
\pgfpathcurveto{\pgfqpoint{3.506910in}{2.950610in}}{\pgfqpoint{3.496311in}{2.946220in}}{\pgfqpoint{3.488497in}{2.938406in}}%
\pgfpathcurveto{\pgfqpoint{3.480683in}{2.930593in}}{\pgfqpoint{3.476293in}{2.919994in}}{\pgfqpoint{3.476293in}{2.908944in}}%
\pgfpathcurveto{\pgfqpoint{3.476293in}{2.897894in}}{\pgfqpoint{3.480683in}{2.887294in}}{\pgfqpoint{3.488497in}{2.879481in}}%
\pgfpathcurveto{\pgfqpoint{3.496311in}{2.871667in}}{\pgfqpoint{3.506910in}{2.867277in}}{\pgfqpoint{3.517960in}{2.867277in}}%
\pgfpathclose%
\pgfusepath{stroke,fill}%
\end{pgfscope}%
\begin{pgfscope}%
\pgfpathrectangle{\pgfqpoint{0.481978in}{0.331635in}}{\pgfqpoint{4.960000in}{3.696000in}}%
\pgfusepath{clip}%
\pgfsetbuttcap%
\pgfsetroundjoin%
\definecolor{currentfill}{rgb}{0.631373,0.788235,0.956863}%
\pgfsetfillcolor{currentfill}%
\pgfsetlinewidth{0.481800pt}%
\definecolor{currentstroke}{rgb}{1.000000,1.000000,1.000000}%
\pgfsetstrokecolor{currentstroke}%
\pgfsetdash{}{0pt}%
\pgfpathmoveto{\pgfqpoint{2.684754in}{0.691976in}}%
\pgfpathcurveto{\pgfqpoint{2.695804in}{0.691976in}}{\pgfqpoint{2.706403in}{0.696366in}}{\pgfqpoint{2.714217in}{0.704179in}}%
\pgfpathcurveto{\pgfqpoint{2.722030in}{0.711993in}}{\pgfqpoint{2.726421in}{0.722592in}}{\pgfqpoint{2.726421in}{0.733642in}}%
\pgfpathcurveto{\pgfqpoint{2.726421in}{0.744692in}}{\pgfqpoint{2.722030in}{0.755291in}}{\pgfqpoint{2.714217in}{0.763105in}}%
\pgfpathcurveto{\pgfqpoint{2.706403in}{0.770919in}}{\pgfqpoint{2.695804in}{0.775309in}}{\pgfqpoint{2.684754in}{0.775309in}}%
\pgfpathcurveto{\pgfqpoint{2.673704in}{0.775309in}}{\pgfqpoint{2.663105in}{0.770919in}}{\pgfqpoint{2.655291in}{0.763105in}}%
\pgfpathcurveto{\pgfqpoint{2.647478in}{0.755291in}}{\pgfqpoint{2.643087in}{0.744692in}}{\pgfqpoint{2.643087in}{0.733642in}}%
\pgfpathcurveto{\pgfqpoint{2.643087in}{0.722592in}}{\pgfqpoint{2.647478in}{0.711993in}}{\pgfqpoint{2.655291in}{0.704179in}}%
\pgfpathcurveto{\pgfqpoint{2.663105in}{0.696366in}}{\pgfqpoint{2.673704in}{0.691976in}}{\pgfqpoint{2.684754in}{0.691976in}}%
\pgfpathclose%
\pgfusepath{stroke,fill}%
\end{pgfscope}%
\begin{pgfscope}%
\pgfpathrectangle{\pgfqpoint{0.481978in}{0.331635in}}{\pgfqpoint{4.960000in}{3.696000in}}%
\pgfusepath{clip}%
\pgfsetbuttcap%
\pgfsetroundjoin%
\definecolor{currentfill}{rgb}{0.631373,0.788235,0.956863}%
\pgfsetfillcolor{currentfill}%
\pgfsetlinewidth{0.481800pt}%
\definecolor{currentstroke}{rgb}{1.000000,1.000000,1.000000}%
\pgfsetstrokecolor{currentstroke}%
\pgfsetdash{}{0pt}%
\pgfpathmoveto{\pgfqpoint{3.654791in}{2.739540in}}%
\pgfpathcurveto{\pgfqpoint{3.665841in}{2.739540in}}{\pgfqpoint{3.676440in}{2.743930in}}{\pgfqpoint{3.684254in}{2.751744in}}%
\pgfpathcurveto{\pgfqpoint{3.692068in}{2.759557in}}{\pgfqpoint{3.696458in}{2.770156in}}{\pgfqpoint{3.696458in}{2.781206in}}%
\pgfpathcurveto{\pgfqpoint{3.696458in}{2.792257in}}{\pgfqpoint{3.692068in}{2.802856in}}{\pgfqpoint{3.684254in}{2.810669in}}%
\pgfpathcurveto{\pgfqpoint{3.676440in}{2.818483in}}{\pgfqpoint{3.665841in}{2.822873in}}{\pgfqpoint{3.654791in}{2.822873in}}%
\pgfpathcurveto{\pgfqpoint{3.643741in}{2.822873in}}{\pgfqpoint{3.633142in}{2.818483in}}{\pgfqpoint{3.625329in}{2.810669in}}%
\pgfpathcurveto{\pgfqpoint{3.617515in}{2.802856in}}{\pgfqpoint{3.613125in}{2.792257in}}{\pgfqpoint{3.613125in}{2.781206in}}%
\pgfpathcurveto{\pgfqpoint{3.613125in}{2.770156in}}{\pgfqpoint{3.617515in}{2.759557in}}{\pgfqpoint{3.625329in}{2.751744in}}%
\pgfpathcurveto{\pgfqpoint{3.633142in}{2.743930in}}{\pgfqpoint{3.643741in}{2.739540in}}{\pgfqpoint{3.654791in}{2.739540in}}%
\pgfpathclose%
\pgfusepath{stroke,fill}%
\end{pgfscope}%
\begin{pgfscope}%
\pgfpathrectangle{\pgfqpoint{0.481978in}{0.331635in}}{\pgfqpoint{4.960000in}{3.696000in}}%
\pgfusepath{clip}%
\pgfsetbuttcap%
\pgfsetroundjoin%
\definecolor{currentfill}{rgb}{0.631373,0.788235,0.956863}%
\pgfsetfillcolor{currentfill}%
\pgfsetlinewidth{0.481800pt}%
\definecolor{currentstroke}{rgb}{1.000000,1.000000,1.000000}%
\pgfsetstrokecolor{currentstroke}%
\pgfsetdash{}{0pt}%
\pgfpathmoveto{\pgfqpoint{3.969303in}{1.606428in}}%
\pgfpathcurveto{\pgfqpoint{3.980353in}{1.606428in}}{\pgfqpoint{3.990952in}{1.610818in}}{\pgfqpoint{3.998765in}{1.618632in}}%
\pgfpathcurveto{\pgfqpoint{4.006579in}{1.626445in}}{\pgfqpoint{4.010969in}{1.637044in}}{\pgfqpoint{4.010969in}{1.648094in}}%
\pgfpathcurveto{\pgfqpoint{4.010969in}{1.659145in}}{\pgfqpoint{4.006579in}{1.669744in}}{\pgfqpoint{3.998765in}{1.677557in}}%
\pgfpathcurveto{\pgfqpoint{3.990952in}{1.685371in}}{\pgfqpoint{3.980353in}{1.689761in}}{\pgfqpoint{3.969303in}{1.689761in}}%
\pgfpathcurveto{\pgfqpoint{3.958253in}{1.689761in}}{\pgfqpoint{3.947654in}{1.685371in}}{\pgfqpoint{3.939840in}{1.677557in}}%
\pgfpathcurveto{\pgfqpoint{3.932026in}{1.669744in}}{\pgfqpoint{3.927636in}{1.659145in}}{\pgfqpoint{3.927636in}{1.648094in}}%
\pgfpathcurveto{\pgfqpoint{3.927636in}{1.637044in}}{\pgfqpoint{3.932026in}{1.626445in}}{\pgfqpoint{3.939840in}{1.618632in}}%
\pgfpathcurveto{\pgfqpoint{3.947654in}{1.610818in}}{\pgfqpoint{3.958253in}{1.606428in}}{\pgfqpoint{3.969303in}{1.606428in}}%
\pgfpathclose%
\pgfusepath{stroke,fill}%
\end{pgfscope}%
\begin{pgfscope}%
\pgfpathrectangle{\pgfqpoint{0.481978in}{0.331635in}}{\pgfqpoint{4.960000in}{3.696000in}}%
\pgfusepath{clip}%
\pgfsetbuttcap%
\pgfsetroundjoin%
\definecolor{currentfill}{rgb}{0.631373,0.788235,0.956863}%
\pgfsetfillcolor{currentfill}%
\pgfsetlinewidth{0.481800pt}%
\definecolor{currentstroke}{rgb}{1.000000,1.000000,1.000000}%
\pgfsetstrokecolor{currentstroke}%
\pgfsetdash{}{0pt}%
\pgfpathmoveto{\pgfqpoint{3.236827in}{2.192574in}}%
\pgfpathcurveto{\pgfqpoint{3.247878in}{2.192574in}}{\pgfqpoint{3.258477in}{2.196964in}}{\pgfqpoint{3.266290in}{2.204778in}}%
\pgfpathcurveto{\pgfqpoint{3.274104in}{2.212592in}}{\pgfqpoint{3.278494in}{2.223191in}}{\pgfqpoint{3.278494in}{2.234241in}}%
\pgfpathcurveto{\pgfqpoint{3.278494in}{2.245291in}}{\pgfqpoint{3.274104in}{2.255890in}}{\pgfqpoint{3.266290in}{2.263704in}}%
\pgfpathcurveto{\pgfqpoint{3.258477in}{2.271517in}}{\pgfqpoint{3.247878in}{2.275908in}}{\pgfqpoint{3.236827in}{2.275908in}}%
\pgfpathcurveto{\pgfqpoint{3.225777in}{2.275908in}}{\pgfqpoint{3.215178in}{2.271517in}}{\pgfqpoint{3.207365in}{2.263704in}}%
\pgfpathcurveto{\pgfqpoint{3.199551in}{2.255890in}}{\pgfqpoint{3.195161in}{2.245291in}}{\pgfqpoint{3.195161in}{2.234241in}}%
\pgfpathcurveto{\pgfqpoint{3.195161in}{2.223191in}}{\pgfqpoint{3.199551in}{2.212592in}}{\pgfqpoint{3.207365in}{2.204778in}}%
\pgfpathcurveto{\pgfqpoint{3.215178in}{2.196964in}}{\pgfqpoint{3.225777in}{2.192574in}}{\pgfqpoint{3.236827in}{2.192574in}}%
\pgfpathclose%
\pgfusepath{stroke,fill}%
\end{pgfscope}%
\begin{pgfscope}%
\pgfpathrectangle{\pgfqpoint{0.481978in}{0.331635in}}{\pgfqpoint{4.960000in}{3.696000in}}%
\pgfusepath{clip}%
\pgfsetbuttcap%
\pgfsetroundjoin%
\definecolor{currentfill}{rgb}{0.631373,0.788235,0.956863}%
\pgfsetfillcolor{currentfill}%
\pgfsetlinewidth{0.481800pt}%
\definecolor{currentstroke}{rgb}{1.000000,1.000000,1.000000}%
\pgfsetstrokecolor{currentstroke}%
\pgfsetdash{}{0pt}%
\pgfpathmoveto{\pgfqpoint{4.281359in}{1.973416in}}%
\pgfpathcurveto{\pgfqpoint{4.292409in}{1.973416in}}{\pgfqpoint{4.303008in}{1.977806in}}{\pgfqpoint{4.310821in}{1.985620in}}%
\pgfpathcurveto{\pgfqpoint{4.318635in}{1.993433in}}{\pgfqpoint{4.323025in}{2.004032in}}{\pgfqpoint{4.323025in}{2.015083in}}%
\pgfpathcurveto{\pgfqpoint{4.323025in}{2.026133in}}{\pgfqpoint{4.318635in}{2.036732in}}{\pgfqpoint{4.310821in}{2.044545in}}%
\pgfpathcurveto{\pgfqpoint{4.303008in}{2.052359in}}{\pgfqpoint{4.292409in}{2.056749in}}{\pgfqpoint{4.281359in}{2.056749in}}%
\pgfpathcurveto{\pgfqpoint{4.270308in}{2.056749in}}{\pgfqpoint{4.259709in}{2.052359in}}{\pgfqpoint{4.251896in}{2.044545in}}%
\pgfpathcurveto{\pgfqpoint{4.244082in}{2.036732in}}{\pgfqpoint{4.239692in}{2.026133in}}{\pgfqpoint{4.239692in}{2.015083in}}%
\pgfpathcurveto{\pgfqpoint{4.239692in}{2.004032in}}{\pgfqpoint{4.244082in}{1.993433in}}{\pgfqpoint{4.251896in}{1.985620in}}%
\pgfpathcurveto{\pgfqpoint{4.259709in}{1.977806in}}{\pgfqpoint{4.270308in}{1.973416in}}{\pgfqpoint{4.281359in}{1.973416in}}%
\pgfpathclose%
\pgfusepath{stroke,fill}%
\end{pgfscope}%
\begin{pgfscope}%
\pgfpathrectangle{\pgfqpoint{0.481978in}{0.331635in}}{\pgfqpoint{4.960000in}{3.696000in}}%
\pgfusepath{clip}%
\pgfsetbuttcap%
\pgfsetroundjoin%
\definecolor{currentfill}{rgb}{0.631373,0.788235,0.956863}%
\pgfsetfillcolor{currentfill}%
\pgfsetlinewidth{0.481800pt}%
\definecolor{currentstroke}{rgb}{1.000000,1.000000,1.000000}%
\pgfsetstrokecolor{currentstroke}%
\pgfsetdash{}{0pt}%
\pgfpathmoveto{\pgfqpoint{4.179788in}{1.761329in}}%
\pgfpathcurveto{\pgfqpoint{4.190838in}{1.761329in}}{\pgfqpoint{4.201437in}{1.765719in}}{\pgfqpoint{4.209250in}{1.773533in}}%
\pgfpathcurveto{\pgfqpoint{4.217064in}{1.781346in}}{\pgfqpoint{4.221454in}{1.791945in}}{\pgfqpoint{4.221454in}{1.802996in}}%
\pgfpathcurveto{\pgfqpoint{4.221454in}{1.814046in}}{\pgfqpoint{4.217064in}{1.824645in}}{\pgfqpoint{4.209250in}{1.832458in}}%
\pgfpathcurveto{\pgfqpoint{4.201437in}{1.840272in}}{\pgfqpoint{4.190838in}{1.844662in}}{\pgfqpoint{4.179788in}{1.844662in}}%
\pgfpathcurveto{\pgfqpoint{4.168737in}{1.844662in}}{\pgfqpoint{4.158138in}{1.840272in}}{\pgfqpoint{4.150325in}{1.832458in}}%
\pgfpathcurveto{\pgfqpoint{4.142511in}{1.824645in}}{\pgfqpoint{4.138121in}{1.814046in}}{\pgfqpoint{4.138121in}{1.802996in}}%
\pgfpathcurveto{\pgfqpoint{4.138121in}{1.791945in}}{\pgfqpoint{4.142511in}{1.781346in}}{\pgfqpoint{4.150325in}{1.773533in}}%
\pgfpathcurveto{\pgfqpoint{4.158138in}{1.765719in}}{\pgfqpoint{4.168737in}{1.761329in}}{\pgfqpoint{4.179788in}{1.761329in}}%
\pgfpathclose%
\pgfusepath{stroke,fill}%
\end{pgfscope}%
\begin{pgfscope}%
\pgfpathrectangle{\pgfqpoint{0.481978in}{0.331635in}}{\pgfqpoint{4.960000in}{3.696000in}}%
\pgfusepath{clip}%
\pgfsetbuttcap%
\pgfsetroundjoin%
\definecolor{currentfill}{rgb}{0.631373,0.788235,0.956863}%
\pgfsetfillcolor{currentfill}%
\pgfsetlinewidth{0.481800pt}%
\definecolor{currentstroke}{rgb}{1.000000,1.000000,1.000000}%
\pgfsetstrokecolor{currentstroke}%
\pgfsetdash{}{0pt}%
\pgfpathmoveto{\pgfqpoint{4.022983in}{2.268716in}}%
\pgfpathcurveto{\pgfqpoint{4.034033in}{2.268716in}}{\pgfqpoint{4.044632in}{2.273106in}}{\pgfqpoint{4.052445in}{2.280920in}}%
\pgfpathcurveto{\pgfqpoint{4.060259in}{2.288733in}}{\pgfqpoint{4.064649in}{2.299332in}}{\pgfqpoint{4.064649in}{2.310383in}}%
\pgfpathcurveto{\pgfqpoint{4.064649in}{2.321433in}}{\pgfqpoint{4.060259in}{2.332032in}}{\pgfqpoint{4.052445in}{2.339845in}}%
\pgfpathcurveto{\pgfqpoint{4.044632in}{2.347659in}}{\pgfqpoint{4.034033in}{2.352049in}}{\pgfqpoint{4.022983in}{2.352049in}}%
\pgfpathcurveto{\pgfqpoint{4.011933in}{2.352049in}}{\pgfqpoint{4.001334in}{2.347659in}}{\pgfqpoint{3.993520in}{2.339845in}}%
\pgfpathcurveto{\pgfqpoint{3.985706in}{2.332032in}}{\pgfqpoint{3.981316in}{2.321433in}}{\pgfqpoint{3.981316in}{2.310383in}}%
\pgfpathcurveto{\pgfqpoint{3.981316in}{2.299332in}}{\pgfqpoint{3.985706in}{2.288733in}}{\pgfqpoint{3.993520in}{2.280920in}}%
\pgfpathcurveto{\pgfqpoint{4.001334in}{2.273106in}}{\pgfqpoint{4.011933in}{2.268716in}}{\pgfqpoint{4.022983in}{2.268716in}}%
\pgfpathclose%
\pgfusepath{stroke,fill}%
\end{pgfscope}%
\begin{pgfscope}%
\pgfpathrectangle{\pgfqpoint{0.481978in}{0.331635in}}{\pgfqpoint{4.960000in}{3.696000in}}%
\pgfusepath{clip}%
\pgfsetbuttcap%
\pgfsetroundjoin%
\definecolor{currentfill}{rgb}{0.631373,0.788235,0.956863}%
\pgfsetfillcolor{currentfill}%
\pgfsetlinewidth{0.481800pt}%
\definecolor{currentstroke}{rgb}{1.000000,1.000000,1.000000}%
\pgfsetstrokecolor{currentstroke}%
\pgfsetdash{}{0pt}%
\pgfpathmoveto{\pgfqpoint{2.058588in}{0.792358in}}%
\pgfpathcurveto{\pgfqpoint{2.069638in}{0.792358in}}{\pgfqpoint{2.080237in}{0.796749in}}{\pgfqpoint{2.088050in}{0.804562in}}%
\pgfpathcurveto{\pgfqpoint{2.095864in}{0.812376in}}{\pgfqpoint{2.100254in}{0.822975in}}{\pgfqpoint{2.100254in}{0.834025in}}%
\pgfpathcurveto{\pgfqpoint{2.100254in}{0.845075in}}{\pgfqpoint{2.095864in}{0.855674in}}{\pgfqpoint{2.088050in}{0.863488in}}%
\pgfpathcurveto{\pgfqpoint{2.080237in}{0.871301in}}{\pgfqpoint{2.069638in}{0.875692in}}{\pgfqpoint{2.058588in}{0.875692in}}%
\pgfpathcurveto{\pgfqpoint{2.047537in}{0.875692in}}{\pgfqpoint{2.036938in}{0.871301in}}{\pgfqpoint{2.029125in}{0.863488in}}%
\pgfpathcurveto{\pgfqpoint{2.021311in}{0.855674in}}{\pgfqpoint{2.016921in}{0.845075in}}{\pgfqpoint{2.016921in}{0.834025in}}%
\pgfpathcurveto{\pgfqpoint{2.016921in}{0.822975in}}{\pgfqpoint{2.021311in}{0.812376in}}{\pgfqpoint{2.029125in}{0.804562in}}%
\pgfpathcurveto{\pgfqpoint{2.036938in}{0.796749in}}{\pgfqpoint{2.047537in}{0.792358in}}{\pgfqpoint{2.058588in}{0.792358in}}%
\pgfpathclose%
\pgfusepath{stroke,fill}%
\end{pgfscope}%
\begin{pgfscope}%
\pgfpathrectangle{\pgfqpoint{0.481978in}{0.331635in}}{\pgfqpoint{4.960000in}{3.696000in}}%
\pgfusepath{clip}%
\pgfsetbuttcap%
\pgfsetroundjoin%
\definecolor{currentfill}{rgb}{0.631373,0.788235,0.956863}%
\pgfsetfillcolor{currentfill}%
\pgfsetlinewidth{0.481800pt}%
\definecolor{currentstroke}{rgb}{1.000000,1.000000,1.000000}%
\pgfsetstrokecolor{currentstroke}%
\pgfsetdash{}{0pt}%
\pgfpathmoveto{\pgfqpoint{4.242206in}{3.491272in}}%
\pgfpathcurveto{\pgfqpoint{4.253256in}{3.491272in}}{\pgfqpoint{4.263855in}{3.495663in}}{\pgfqpoint{4.271668in}{3.503476in}}%
\pgfpathcurveto{\pgfqpoint{4.279482in}{3.511290in}}{\pgfqpoint{4.283872in}{3.521889in}}{\pgfqpoint{4.283872in}{3.532939in}}%
\pgfpathcurveto{\pgfqpoint{4.283872in}{3.543989in}}{\pgfqpoint{4.279482in}{3.554588in}}{\pgfqpoint{4.271668in}{3.562402in}}%
\pgfpathcurveto{\pgfqpoint{4.263855in}{3.570215in}}{\pgfqpoint{4.253256in}{3.574606in}}{\pgfqpoint{4.242206in}{3.574606in}}%
\pgfpathcurveto{\pgfqpoint{4.231156in}{3.574606in}}{\pgfqpoint{4.220557in}{3.570215in}}{\pgfqpoint{4.212743in}{3.562402in}}%
\pgfpathcurveto{\pgfqpoint{4.204929in}{3.554588in}}{\pgfqpoint{4.200539in}{3.543989in}}{\pgfqpoint{4.200539in}{3.532939in}}%
\pgfpathcurveto{\pgfqpoint{4.200539in}{3.521889in}}{\pgfqpoint{4.204929in}{3.511290in}}{\pgfqpoint{4.212743in}{3.503476in}}%
\pgfpathcurveto{\pgfqpoint{4.220557in}{3.495663in}}{\pgfqpoint{4.231156in}{3.491272in}}{\pgfqpoint{4.242206in}{3.491272in}}%
\pgfpathclose%
\pgfusepath{stroke,fill}%
\end{pgfscope}%
\begin{pgfscope}%
\pgfpathrectangle{\pgfqpoint{0.481978in}{0.331635in}}{\pgfqpoint{4.960000in}{3.696000in}}%
\pgfusepath{clip}%
\pgfsetbuttcap%
\pgfsetroundjoin%
\definecolor{currentfill}{rgb}{0.631373,0.788235,0.956863}%
\pgfsetfillcolor{currentfill}%
\pgfsetlinewidth{0.481800pt}%
\definecolor{currentstroke}{rgb}{1.000000,1.000000,1.000000}%
\pgfsetstrokecolor{currentstroke}%
\pgfsetdash{}{0pt}%
\pgfpathmoveto{\pgfqpoint{3.917773in}{2.669296in}}%
\pgfpathcurveto{\pgfqpoint{3.928823in}{2.669296in}}{\pgfqpoint{3.939422in}{2.673686in}}{\pgfqpoint{3.947235in}{2.681500in}}%
\pgfpathcurveto{\pgfqpoint{3.955049in}{2.689313in}}{\pgfqpoint{3.959439in}{2.699912in}}{\pgfqpoint{3.959439in}{2.710962in}}%
\pgfpathcurveto{\pgfqpoint{3.959439in}{2.722012in}}{\pgfqpoint{3.955049in}{2.732612in}}{\pgfqpoint{3.947235in}{2.740425in}}%
\pgfpathcurveto{\pgfqpoint{3.939422in}{2.748239in}}{\pgfqpoint{3.928823in}{2.752629in}}{\pgfqpoint{3.917773in}{2.752629in}}%
\pgfpathcurveto{\pgfqpoint{3.906722in}{2.752629in}}{\pgfqpoint{3.896123in}{2.748239in}}{\pgfqpoint{3.888310in}{2.740425in}}%
\pgfpathcurveto{\pgfqpoint{3.880496in}{2.732612in}}{\pgfqpoint{3.876106in}{2.722012in}}{\pgfqpoint{3.876106in}{2.710962in}}%
\pgfpathcurveto{\pgfqpoint{3.876106in}{2.699912in}}{\pgfqpoint{3.880496in}{2.689313in}}{\pgfqpoint{3.888310in}{2.681500in}}%
\pgfpathcurveto{\pgfqpoint{3.896123in}{2.673686in}}{\pgfqpoint{3.906722in}{2.669296in}}{\pgfqpoint{3.917773in}{2.669296in}}%
\pgfpathclose%
\pgfusepath{stroke,fill}%
\end{pgfscope}%
\begin{pgfscope}%
\pgfpathrectangle{\pgfqpoint{0.481978in}{0.331635in}}{\pgfqpoint{4.960000in}{3.696000in}}%
\pgfusepath{clip}%
\pgfsetbuttcap%
\pgfsetroundjoin%
\definecolor{currentfill}{rgb}{0.631373,0.788235,0.956863}%
\pgfsetfillcolor{currentfill}%
\pgfsetlinewidth{0.481800pt}%
\definecolor{currentstroke}{rgb}{1.000000,1.000000,1.000000}%
\pgfsetstrokecolor{currentstroke}%
\pgfsetdash{}{0pt}%
\pgfpathmoveto{\pgfqpoint{3.910278in}{2.072164in}}%
\pgfpathcurveto{\pgfqpoint{3.921328in}{2.072164in}}{\pgfqpoint{3.931927in}{2.076555in}}{\pgfqpoint{3.939741in}{2.084368in}}%
\pgfpathcurveto{\pgfqpoint{3.947555in}{2.092182in}}{\pgfqpoint{3.951945in}{2.102781in}}{\pgfqpoint{3.951945in}{2.113831in}}%
\pgfpathcurveto{\pgfqpoint{3.951945in}{2.124881in}}{\pgfqpoint{3.947555in}{2.135480in}}{\pgfqpoint{3.939741in}{2.143294in}}%
\pgfpathcurveto{\pgfqpoint{3.931927in}{2.151107in}}{\pgfqpoint{3.921328in}{2.155498in}}{\pgfqpoint{3.910278in}{2.155498in}}%
\pgfpathcurveto{\pgfqpoint{3.899228in}{2.155498in}}{\pgfqpoint{3.888629in}{2.151107in}}{\pgfqpoint{3.880815in}{2.143294in}}%
\pgfpathcurveto{\pgfqpoint{3.873002in}{2.135480in}}{\pgfqpoint{3.868612in}{2.124881in}}{\pgfqpoint{3.868612in}{2.113831in}}%
\pgfpathcurveto{\pgfqpoint{3.868612in}{2.102781in}}{\pgfqpoint{3.873002in}{2.092182in}}{\pgfqpoint{3.880815in}{2.084368in}}%
\pgfpathcurveto{\pgfqpoint{3.888629in}{2.076555in}}{\pgfqpoint{3.899228in}{2.072164in}}{\pgfqpoint{3.910278in}{2.072164in}}%
\pgfpathclose%
\pgfusepath{stroke,fill}%
\end{pgfscope}%
\begin{pgfscope}%
\pgfpathrectangle{\pgfqpoint{0.481978in}{0.331635in}}{\pgfqpoint{4.960000in}{3.696000in}}%
\pgfusepath{clip}%
\pgfsetbuttcap%
\pgfsetroundjoin%
\definecolor{currentfill}{rgb}{0.631373,0.788235,0.956863}%
\pgfsetfillcolor{currentfill}%
\pgfsetlinewidth{0.481800pt}%
\definecolor{currentstroke}{rgb}{1.000000,1.000000,1.000000}%
\pgfsetstrokecolor{currentstroke}%
\pgfsetdash{}{0pt}%
\pgfpathmoveto{\pgfqpoint{2.808566in}{1.240721in}}%
\pgfpathcurveto{\pgfqpoint{2.819616in}{1.240721in}}{\pgfqpoint{2.830215in}{1.245111in}}{\pgfqpoint{2.838029in}{1.252925in}}%
\pgfpathcurveto{\pgfqpoint{2.845843in}{1.260738in}}{\pgfqpoint{2.850233in}{1.271337in}}{\pgfqpoint{2.850233in}{1.282388in}}%
\pgfpathcurveto{\pgfqpoint{2.850233in}{1.293438in}}{\pgfqpoint{2.845843in}{1.304037in}}{\pgfqpoint{2.838029in}{1.311850in}}%
\pgfpathcurveto{\pgfqpoint{2.830215in}{1.319664in}}{\pgfqpoint{2.819616in}{1.324054in}}{\pgfqpoint{2.808566in}{1.324054in}}%
\pgfpathcurveto{\pgfqpoint{2.797516in}{1.324054in}}{\pgfqpoint{2.786917in}{1.319664in}}{\pgfqpoint{2.779103in}{1.311850in}}%
\pgfpathcurveto{\pgfqpoint{2.771290in}{1.304037in}}{\pgfqpoint{2.766899in}{1.293438in}}{\pgfqpoint{2.766899in}{1.282388in}}%
\pgfpathcurveto{\pgfqpoint{2.766899in}{1.271337in}}{\pgfqpoint{2.771290in}{1.260738in}}{\pgfqpoint{2.779103in}{1.252925in}}%
\pgfpathcurveto{\pgfqpoint{2.786917in}{1.245111in}}{\pgfqpoint{2.797516in}{1.240721in}}{\pgfqpoint{2.808566in}{1.240721in}}%
\pgfpathclose%
\pgfusepath{stroke,fill}%
\end{pgfscope}%
\begin{pgfscope}%
\pgfpathrectangle{\pgfqpoint{0.481978in}{0.331635in}}{\pgfqpoint{4.960000in}{3.696000in}}%
\pgfusepath{clip}%
\pgfsetbuttcap%
\pgfsetroundjoin%
\definecolor{currentfill}{rgb}{0.631373,0.788235,0.956863}%
\pgfsetfillcolor{currentfill}%
\pgfsetlinewidth{0.481800pt}%
\definecolor{currentstroke}{rgb}{1.000000,1.000000,1.000000}%
\pgfsetstrokecolor{currentstroke}%
\pgfsetdash{}{0pt}%
\pgfpathmoveto{\pgfqpoint{3.721687in}{1.298047in}}%
\pgfpathcurveto{\pgfqpoint{3.732737in}{1.298047in}}{\pgfqpoint{3.743336in}{1.302438in}}{\pgfqpoint{3.751149in}{1.310251in}}%
\pgfpathcurveto{\pgfqpoint{3.758963in}{1.318065in}}{\pgfqpoint{3.763353in}{1.328664in}}{\pgfqpoint{3.763353in}{1.339714in}}%
\pgfpathcurveto{\pgfqpoint{3.763353in}{1.350764in}}{\pgfqpoint{3.758963in}{1.361363in}}{\pgfqpoint{3.751149in}{1.369177in}}%
\pgfpathcurveto{\pgfqpoint{3.743336in}{1.376990in}}{\pgfqpoint{3.732737in}{1.381381in}}{\pgfqpoint{3.721687in}{1.381381in}}%
\pgfpathcurveto{\pgfqpoint{3.710636in}{1.381381in}}{\pgfqpoint{3.700037in}{1.376990in}}{\pgfqpoint{3.692224in}{1.369177in}}%
\pgfpathcurveto{\pgfqpoint{3.684410in}{1.361363in}}{\pgfqpoint{3.680020in}{1.350764in}}{\pgfqpoint{3.680020in}{1.339714in}}%
\pgfpathcurveto{\pgfqpoint{3.680020in}{1.328664in}}{\pgfqpoint{3.684410in}{1.318065in}}{\pgfqpoint{3.692224in}{1.310251in}}%
\pgfpathcurveto{\pgfqpoint{3.700037in}{1.302438in}}{\pgfqpoint{3.710636in}{1.298047in}}{\pgfqpoint{3.721687in}{1.298047in}}%
\pgfpathclose%
\pgfusepath{stroke,fill}%
\end{pgfscope}%
\begin{pgfscope}%
\pgfpathrectangle{\pgfqpoint{0.481978in}{0.331635in}}{\pgfqpoint{4.960000in}{3.696000in}}%
\pgfusepath{clip}%
\pgfsetbuttcap%
\pgfsetroundjoin%
\definecolor{currentfill}{rgb}{0.631373,0.788235,0.956863}%
\pgfsetfillcolor{currentfill}%
\pgfsetlinewidth{0.481800pt}%
\definecolor{currentstroke}{rgb}{1.000000,1.000000,1.000000}%
\pgfsetstrokecolor{currentstroke}%
\pgfsetdash{}{0pt}%
\pgfpathmoveto{\pgfqpoint{4.249049in}{3.498397in}}%
\pgfpathcurveto{\pgfqpoint{4.260100in}{3.498397in}}{\pgfqpoint{4.270699in}{3.502787in}}{\pgfqpoint{4.278512in}{3.510601in}}%
\pgfpathcurveto{\pgfqpoint{4.286326in}{3.518414in}}{\pgfqpoint{4.290716in}{3.529013in}}{\pgfqpoint{4.290716in}{3.540064in}}%
\pgfpathcurveto{\pgfqpoint{4.290716in}{3.551114in}}{\pgfqpoint{4.286326in}{3.561713in}}{\pgfqpoint{4.278512in}{3.569526in}}%
\pgfpathcurveto{\pgfqpoint{4.270699in}{3.577340in}}{\pgfqpoint{4.260100in}{3.581730in}}{\pgfqpoint{4.249049in}{3.581730in}}%
\pgfpathcurveto{\pgfqpoint{4.237999in}{3.581730in}}{\pgfqpoint{4.227400in}{3.577340in}}{\pgfqpoint{4.219587in}{3.569526in}}%
\pgfpathcurveto{\pgfqpoint{4.211773in}{3.561713in}}{\pgfqpoint{4.207383in}{3.551114in}}{\pgfqpoint{4.207383in}{3.540064in}}%
\pgfpathcurveto{\pgfqpoint{4.207383in}{3.529013in}}{\pgfqpoint{4.211773in}{3.518414in}}{\pgfqpoint{4.219587in}{3.510601in}}%
\pgfpathcurveto{\pgfqpoint{4.227400in}{3.502787in}}{\pgfqpoint{4.237999in}{3.498397in}}{\pgfqpoint{4.249049in}{3.498397in}}%
\pgfpathclose%
\pgfusepath{stroke,fill}%
\end{pgfscope}%
\begin{pgfscope}%
\pgfpathrectangle{\pgfqpoint{0.481978in}{0.331635in}}{\pgfqpoint{4.960000in}{3.696000in}}%
\pgfusepath{clip}%
\pgfsetbuttcap%
\pgfsetroundjoin%
\definecolor{currentfill}{rgb}{0.631373,0.788235,0.956863}%
\pgfsetfillcolor{currentfill}%
\pgfsetlinewidth{0.481800pt}%
\definecolor{currentstroke}{rgb}{1.000000,1.000000,1.000000}%
\pgfsetstrokecolor{currentstroke}%
\pgfsetdash{}{0pt}%
\pgfpathmoveto{\pgfqpoint{4.318155in}{2.432343in}}%
\pgfpathcurveto{\pgfqpoint{4.329205in}{2.432343in}}{\pgfqpoint{4.339804in}{2.436733in}}{\pgfqpoint{4.347618in}{2.444546in}}%
\pgfpathcurveto{\pgfqpoint{4.355432in}{2.452360in}}{\pgfqpoint{4.359822in}{2.462959in}}{\pgfqpoint{4.359822in}{2.474009in}}%
\pgfpathcurveto{\pgfqpoint{4.359822in}{2.485059in}}{\pgfqpoint{4.355432in}{2.495658in}}{\pgfqpoint{4.347618in}{2.503472in}}%
\pgfpathcurveto{\pgfqpoint{4.339804in}{2.511286in}}{\pgfqpoint{4.329205in}{2.515676in}}{\pgfqpoint{4.318155in}{2.515676in}}%
\pgfpathcurveto{\pgfqpoint{4.307105in}{2.515676in}}{\pgfqpoint{4.296506in}{2.511286in}}{\pgfqpoint{4.288692in}{2.503472in}}%
\pgfpathcurveto{\pgfqpoint{4.280879in}{2.495658in}}{\pgfqpoint{4.276489in}{2.485059in}}{\pgfqpoint{4.276489in}{2.474009in}}%
\pgfpathcurveto{\pgfqpoint{4.276489in}{2.462959in}}{\pgfqpoint{4.280879in}{2.452360in}}{\pgfqpoint{4.288692in}{2.444546in}}%
\pgfpathcurveto{\pgfqpoint{4.296506in}{2.436733in}}{\pgfqpoint{4.307105in}{2.432343in}}{\pgfqpoint{4.318155in}{2.432343in}}%
\pgfpathclose%
\pgfusepath{stroke,fill}%
\end{pgfscope}%
\begin{pgfscope}%
\pgfpathrectangle{\pgfqpoint{0.481978in}{0.331635in}}{\pgfqpoint{4.960000in}{3.696000in}}%
\pgfusepath{clip}%
\pgfsetbuttcap%
\pgfsetroundjoin%
\definecolor{currentfill}{rgb}{0.631373,0.788235,0.956863}%
\pgfsetfillcolor{currentfill}%
\pgfsetlinewidth{0.481800pt}%
\definecolor{currentstroke}{rgb}{1.000000,1.000000,1.000000}%
\pgfsetstrokecolor{currentstroke}%
\pgfsetdash{}{0pt}%
\pgfpathmoveto{\pgfqpoint{2.648273in}{0.804161in}}%
\pgfpathcurveto{\pgfqpoint{2.659323in}{0.804161in}}{\pgfqpoint{2.669922in}{0.808551in}}{\pgfqpoint{2.677735in}{0.816365in}}%
\pgfpathcurveto{\pgfqpoint{2.685549in}{0.824179in}}{\pgfqpoint{2.689939in}{0.834778in}}{\pgfqpoint{2.689939in}{0.845828in}}%
\pgfpathcurveto{\pgfqpoint{2.689939in}{0.856878in}}{\pgfqpoint{2.685549in}{0.867477in}}{\pgfqpoint{2.677735in}{0.875290in}}%
\pgfpathcurveto{\pgfqpoint{2.669922in}{0.883104in}}{\pgfqpoint{2.659323in}{0.887494in}}{\pgfqpoint{2.648273in}{0.887494in}}%
\pgfpathcurveto{\pgfqpoint{2.637222in}{0.887494in}}{\pgfqpoint{2.626623in}{0.883104in}}{\pgfqpoint{2.618810in}{0.875290in}}%
\pgfpathcurveto{\pgfqpoint{2.610996in}{0.867477in}}{\pgfqpoint{2.606606in}{0.856878in}}{\pgfqpoint{2.606606in}{0.845828in}}%
\pgfpathcurveto{\pgfqpoint{2.606606in}{0.834778in}}{\pgfqpoint{2.610996in}{0.824179in}}{\pgfqpoint{2.618810in}{0.816365in}}%
\pgfpathcurveto{\pgfqpoint{2.626623in}{0.808551in}}{\pgfqpoint{2.637222in}{0.804161in}}{\pgfqpoint{2.648273in}{0.804161in}}%
\pgfpathclose%
\pgfusepath{stroke,fill}%
\end{pgfscope}%
\begin{pgfscope}%
\pgfpathrectangle{\pgfqpoint{0.481978in}{0.331635in}}{\pgfqpoint{4.960000in}{3.696000in}}%
\pgfusepath{clip}%
\pgfsetbuttcap%
\pgfsetroundjoin%
\definecolor{currentfill}{rgb}{0.631373,0.788235,0.956863}%
\pgfsetfillcolor{currentfill}%
\pgfsetlinewidth{0.481800pt}%
\definecolor{currentstroke}{rgb}{1.000000,1.000000,1.000000}%
\pgfsetstrokecolor{currentstroke}%
\pgfsetdash{}{0pt}%
\pgfpathmoveto{\pgfqpoint{3.960912in}{3.178460in}}%
\pgfpathcurveto{\pgfqpoint{3.971962in}{3.178460in}}{\pgfqpoint{3.982561in}{3.182851in}}{\pgfqpoint{3.990375in}{3.190664in}}%
\pgfpathcurveto{\pgfqpoint{3.998189in}{3.198478in}}{\pgfqpoint{4.002579in}{3.209077in}}{\pgfqpoint{4.002579in}{3.220127in}}%
\pgfpathcurveto{\pgfqpoint{4.002579in}{3.231177in}}{\pgfqpoint{3.998189in}{3.241776in}}{\pgfqpoint{3.990375in}{3.249590in}}%
\pgfpathcurveto{\pgfqpoint{3.982561in}{3.257404in}}{\pgfqpoint{3.971962in}{3.261794in}}{\pgfqpoint{3.960912in}{3.261794in}}%
\pgfpathcurveto{\pgfqpoint{3.949862in}{3.261794in}}{\pgfqpoint{3.939263in}{3.257404in}}{\pgfqpoint{3.931449in}{3.249590in}}%
\pgfpathcurveto{\pgfqpoint{3.923636in}{3.241776in}}{\pgfqpoint{3.919246in}{3.231177in}}{\pgfqpoint{3.919246in}{3.220127in}}%
\pgfpathcurveto{\pgfqpoint{3.919246in}{3.209077in}}{\pgfqpoint{3.923636in}{3.198478in}}{\pgfqpoint{3.931449in}{3.190664in}}%
\pgfpathcurveto{\pgfqpoint{3.939263in}{3.182851in}}{\pgfqpoint{3.949862in}{3.178460in}}{\pgfqpoint{3.960912in}{3.178460in}}%
\pgfpathclose%
\pgfusepath{stroke,fill}%
\end{pgfscope}%
\begin{pgfscope}%
\pgfpathrectangle{\pgfqpoint{0.481978in}{0.331635in}}{\pgfqpoint{4.960000in}{3.696000in}}%
\pgfusepath{clip}%
\pgfsetbuttcap%
\pgfsetroundjoin%
\definecolor{currentfill}{rgb}{0.631373,0.788235,0.956863}%
\pgfsetfillcolor{currentfill}%
\pgfsetlinewidth{0.481800pt}%
\definecolor{currentstroke}{rgb}{1.000000,1.000000,1.000000}%
\pgfsetstrokecolor{currentstroke}%
\pgfsetdash{}{0pt}%
\pgfpathmoveto{\pgfqpoint{3.165961in}{0.615495in}}%
\pgfpathcurveto{\pgfqpoint{3.177011in}{0.615495in}}{\pgfqpoint{3.187610in}{0.619885in}}{\pgfqpoint{3.195424in}{0.627699in}}%
\pgfpathcurveto{\pgfqpoint{3.203237in}{0.635512in}}{\pgfqpoint{3.207627in}{0.646111in}}{\pgfqpoint{3.207627in}{0.657161in}}%
\pgfpathcurveto{\pgfqpoint{3.207627in}{0.668211in}}{\pgfqpoint{3.203237in}{0.678811in}}{\pgfqpoint{3.195424in}{0.686624in}}%
\pgfpathcurveto{\pgfqpoint{3.187610in}{0.694438in}}{\pgfqpoint{3.177011in}{0.698828in}}{\pgfqpoint{3.165961in}{0.698828in}}%
\pgfpathcurveto{\pgfqpoint{3.154911in}{0.698828in}}{\pgfqpoint{3.144312in}{0.694438in}}{\pgfqpoint{3.136498in}{0.686624in}}%
\pgfpathcurveto{\pgfqpoint{3.128684in}{0.678811in}}{\pgfqpoint{3.124294in}{0.668211in}}{\pgfqpoint{3.124294in}{0.657161in}}%
\pgfpathcurveto{\pgfqpoint{3.124294in}{0.646111in}}{\pgfqpoint{3.128684in}{0.635512in}}{\pgfqpoint{3.136498in}{0.627699in}}%
\pgfpathcurveto{\pgfqpoint{3.144312in}{0.619885in}}{\pgfqpoint{3.154911in}{0.615495in}}{\pgfqpoint{3.165961in}{0.615495in}}%
\pgfpathclose%
\pgfusepath{stroke,fill}%
\end{pgfscope}%
\begin{pgfscope}%
\pgfpathrectangle{\pgfqpoint{0.481978in}{0.331635in}}{\pgfqpoint{4.960000in}{3.696000in}}%
\pgfusepath{clip}%
\pgfsetbuttcap%
\pgfsetroundjoin%
\definecolor{currentfill}{rgb}{0.631373,0.788235,0.956863}%
\pgfsetfillcolor{currentfill}%
\pgfsetlinewidth{0.481800pt}%
\definecolor{currentstroke}{rgb}{1.000000,1.000000,1.000000}%
\pgfsetstrokecolor{currentstroke}%
\pgfsetdash{}{0pt}%
\pgfpathmoveto{\pgfqpoint{3.531854in}{3.248160in}}%
\pgfpathcurveto{\pgfqpoint{3.542904in}{3.248160in}}{\pgfqpoint{3.553503in}{3.252550in}}{\pgfqpoint{3.561317in}{3.260364in}}%
\pgfpathcurveto{\pgfqpoint{3.569130in}{3.268178in}}{\pgfqpoint{3.573520in}{3.278777in}}{\pgfqpoint{3.573520in}{3.289827in}}%
\pgfpathcurveto{\pgfqpoint{3.573520in}{3.300877in}}{\pgfqpoint{3.569130in}{3.311476in}}{\pgfqpoint{3.561317in}{3.319290in}}%
\pgfpathcurveto{\pgfqpoint{3.553503in}{3.327103in}}{\pgfqpoint{3.542904in}{3.331493in}}{\pgfqpoint{3.531854in}{3.331493in}}%
\pgfpathcurveto{\pgfqpoint{3.520804in}{3.331493in}}{\pgfqpoint{3.510205in}{3.327103in}}{\pgfqpoint{3.502391in}{3.319290in}}%
\pgfpathcurveto{\pgfqpoint{3.494577in}{3.311476in}}{\pgfqpoint{3.490187in}{3.300877in}}{\pgfqpoint{3.490187in}{3.289827in}}%
\pgfpathcurveto{\pgfqpoint{3.490187in}{3.278777in}}{\pgfqpoint{3.494577in}{3.268178in}}{\pgfqpoint{3.502391in}{3.260364in}}%
\pgfpathcurveto{\pgfqpoint{3.510205in}{3.252550in}}{\pgfqpoint{3.520804in}{3.248160in}}{\pgfqpoint{3.531854in}{3.248160in}}%
\pgfpathclose%
\pgfusepath{stroke,fill}%
\end{pgfscope}%
\begin{pgfscope}%
\pgfpathrectangle{\pgfqpoint{0.481978in}{0.331635in}}{\pgfqpoint{4.960000in}{3.696000in}}%
\pgfusepath{clip}%
\pgfsetbuttcap%
\pgfsetroundjoin%
\definecolor{currentfill}{rgb}{0.631373,0.788235,0.956863}%
\pgfsetfillcolor{currentfill}%
\pgfsetlinewidth{0.481800pt}%
\definecolor{currentstroke}{rgb}{1.000000,1.000000,1.000000}%
\pgfsetstrokecolor{currentstroke}%
\pgfsetdash{}{0pt}%
\pgfpathmoveto{\pgfqpoint{3.193034in}{0.457968in}}%
\pgfpathcurveto{\pgfqpoint{3.204084in}{0.457968in}}{\pgfqpoint{3.214683in}{0.462359in}}{\pgfqpoint{3.222497in}{0.470172in}}%
\pgfpathcurveto{\pgfqpoint{3.230310in}{0.477986in}}{\pgfqpoint{3.234700in}{0.488585in}}{\pgfqpoint{3.234700in}{0.499635in}}%
\pgfpathcurveto{\pgfqpoint{3.234700in}{0.510685in}}{\pgfqpoint{3.230310in}{0.521284in}}{\pgfqpoint{3.222497in}{0.529098in}}%
\pgfpathcurveto{\pgfqpoint{3.214683in}{0.536911in}}{\pgfqpoint{3.204084in}{0.541302in}}{\pgfqpoint{3.193034in}{0.541302in}}%
\pgfpathcurveto{\pgfqpoint{3.181984in}{0.541302in}}{\pgfqpoint{3.171385in}{0.536911in}}{\pgfqpoint{3.163571in}{0.529098in}}%
\pgfpathcurveto{\pgfqpoint{3.155757in}{0.521284in}}{\pgfqpoint{3.151367in}{0.510685in}}{\pgfqpoint{3.151367in}{0.499635in}}%
\pgfpathcurveto{\pgfqpoint{3.151367in}{0.488585in}}{\pgfqpoint{3.155757in}{0.477986in}}{\pgfqpoint{3.163571in}{0.470172in}}%
\pgfpathcurveto{\pgfqpoint{3.171385in}{0.462359in}}{\pgfqpoint{3.181984in}{0.457968in}}{\pgfqpoint{3.193034in}{0.457968in}}%
\pgfpathclose%
\pgfusepath{stroke,fill}%
\end{pgfscope}%
\begin{pgfscope}%
\pgfpathrectangle{\pgfqpoint{0.481978in}{0.331635in}}{\pgfqpoint{4.960000in}{3.696000in}}%
\pgfusepath{clip}%
\pgfsetbuttcap%
\pgfsetroundjoin%
\definecolor{currentfill}{rgb}{0.631373,0.788235,0.956863}%
\pgfsetfillcolor{currentfill}%
\pgfsetlinewidth{0.481800pt}%
\definecolor{currentstroke}{rgb}{1.000000,1.000000,1.000000}%
\pgfsetstrokecolor{currentstroke}%
\pgfsetdash{}{0pt}%
\pgfpathmoveto{\pgfqpoint{2.811954in}{0.755979in}}%
\pgfpathcurveto{\pgfqpoint{2.823005in}{0.755979in}}{\pgfqpoint{2.833604in}{0.760370in}}{\pgfqpoint{2.841417in}{0.768183in}}%
\pgfpathcurveto{\pgfqpoint{2.849231in}{0.775997in}}{\pgfqpoint{2.853621in}{0.786596in}}{\pgfqpoint{2.853621in}{0.797646in}}%
\pgfpathcurveto{\pgfqpoint{2.853621in}{0.808696in}}{\pgfqpoint{2.849231in}{0.819295in}}{\pgfqpoint{2.841417in}{0.827109in}}%
\pgfpathcurveto{\pgfqpoint{2.833604in}{0.834923in}}{\pgfqpoint{2.823005in}{0.839313in}}{\pgfqpoint{2.811954in}{0.839313in}}%
\pgfpathcurveto{\pgfqpoint{2.800904in}{0.839313in}}{\pgfqpoint{2.790305in}{0.834923in}}{\pgfqpoint{2.782492in}{0.827109in}}%
\pgfpathcurveto{\pgfqpoint{2.774678in}{0.819295in}}{\pgfqpoint{2.770288in}{0.808696in}}{\pgfqpoint{2.770288in}{0.797646in}}%
\pgfpathcurveto{\pgfqpoint{2.770288in}{0.786596in}}{\pgfqpoint{2.774678in}{0.775997in}}{\pgfqpoint{2.782492in}{0.768183in}}%
\pgfpathcurveto{\pgfqpoint{2.790305in}{0.760370in}}{\pgfqpoint{2.800904in}{0.755979in}}{\pgfqpoint{2.811954in}{0.755979in}}%
\pgfpathclose%
\pgfusepath{stroke,fill}%
\end{pgfscope}%
\begin{pgfscope}%
\pgfpathrectangle{\pgfqpoint{0.481978in}{0.331635in}}{\pgfqpoint{4.960000in}{3.696000in}}%
\pgfusepath{clip}%
\pgfsetbuttcap%
\pgfsetroundjoin%
\definecolor{currentfill}{rgb}{0.631373,0.788235,0.956863}%
\pgfsetfillcolor{currentfill}%
\pgfsetlinewidth{0.481800pt}%
\definecolor{currentstroke}{rgb}{1.000000,1.000000,1.000000}%
\pgfsetstrokecolor{currentstroke}%
\pgfsetdash{}{0pt}%
\pgfpathmoveto{\pgfqpoint{4.845268in}{3.298852in}}%
\pgfpathcurveto{\pgfqpoint{4.856318in}{3.298852in}}{\pgfqpoint{4.866917in}{3.303243in}}{\pgfqpoint{4.874730in}{3.311056in}}%
\pgfpathcurveto{\pgfqpoint{4.882544in}{3.318870in}}{\pgfqpoint{4.886934in}{3.329469in}}{\pgfqpoint{4.886934in}{3.340519in}}%
\pgfpathcurveto{\pgfqpoint{4.886934in}{3.351569in}}{\pgfqpoint{4.882544in}{3.362168in}}{\pgfqpoint{4.874730in}{3.369982in}}%
\pgfpathcurveto{\pgfqpoint{4.866917in}{3.377795in}}{\pgfqpoint{4.856318in}{3.382186in}}{\pgfqpoint{4.845268in}{3.382186in}}%
\pgfpathcurveto{\pgfqpoint{4.834218in}{3.382186in}}{\pgfqpoint{4.823618in}{3.377795in}}{\pgfqpoint{4.815805in}{3.369982in}}%
\pgfpathcurveto{\pgfqpoint{4.807991in}{3.362168in}}{\pgfqpoint{4.803601in}{3.351569in}}{\pgfqpoint{4.803601in}{3.340519in}}%
\pgfpathcurveto{\pgfqpoint{4.803601in}{3.329469in}}{\pgfqpoint{4.807991in}{3.318870in}}{\pgfqpoint{4.815805in}{3.311056in}}%
\pgfpathcurveto{\pgfqpoint{4.823618in}{3.303243in}}{\pgfqpoint{4.834218in}{3.298852in}}{\pgfqpoint{4.845268in}{3.298852in}}%
\pgfpathclose%
\pgfusepath{stroke,fill}%
\end{pgfscope}%
\begin{pgfscope}%
\pgfpathrectangle{\pgfqpoint{0.481978in}{0.331635in}}{\pgfqpoint{4.960000in}{3.696000in}}%
\pgfusepath{clip}%
\pgfsetbuttcap%
\pgfsetroundjoin%
\definecolor{currentfill}{rgb}{0.631373,0.788235,0.956863}%
\pgfsetfillcolor{currentfill}%
\pgfsetlinewidth{0.481800pt}%
\definecolor{currentstroke}{rgb}{1.000000,1.000000,1.000000}%
\pgfsetstrokecolor{currentstroke}%
\pgfsetdash{}{0pt}%
\pgfpathmoveto{\pgfqpoint{4.682828in}{2.588308in}}%
\pgfpathcurveto{\pgfqpoint{4.693878in}{2.588308in}}{\pgfqpoint{4.704477in}{2.592699in}}{\pgfqpoint{4.712291in}{2.600512in}}%
\pgfpathcurveto{\pgfqpoint{4.720104in}{2.608326in}}{\pgfqpoint{4.724495in}{2.618925in}}{\pgfqpoint{4.724495in}{2.629975in}}%
\pgfpathcurveto{\pgfqpoint{4.724495in}{2.641025in}}{\pgfqpoint{4.720104in}{2.651624in}}{\pgfqpoint{4.712291in}{2.659438in}}%
\pgfpathcurveto{\pgfqpoint{4.704477in}{2.667251in}}{\pgfqpoint{4.693878in}{2.671642in}}{\pgfqpoint{4.682828in}{2.671642in}}%
\pgfpathcurveto{\pgfqpoint{4.671778in}{2.671642in}}{\pgfqpoint{4.661179in}{2.667251in}}{\pgfqpoint{4.653365in}{2.659438in}}%
\pgfpathcurveto{\pgfqpoint{4.645551in}{2.651624in}}{\pgfqpoint{4.641161in}{2.641025in}}{\pgfqpoint{4.641161in}{2.629975in}}%
\pgfpathcurveto{\pgfqpoint{4.641161in}{2.618925in}}{\pgfqpoint{4.645551in}{2.608326in}}{\pgfqpoint{4.653365in}{2.600512in}}%
\pgfpathcurveto{\pgfqpoint{4.661179in}{2.592699in}}{\pgfqpoint{4.671778in}{2.588308in}}{\pgfqpoint{4.682828in}{2.588308in}}%
\pgfpathclose%
\pgfusepath{stroke,fill}%
\end{pgfscope}%
\begin{pgfscope}%
\pgfpathrectangle{\pgfqpoint{0.481978in}{0.331635in}}{\pgfqpoint{4.960000in}{3.696000in}}%
\pgfusepath{clip}%
\pgfsetbuttcap%
\pgfsetroundjoin%
\definecolor{currentfill}{rgb}{0.631373,0.788235,0.956863}%
\pgfsetfillcolor{currentfill}%
\pgfsetlinewidth{0.481800pt}%
\definecolor{currentstroke}{rgb}{1.000000,1.000000,1.000000}%
\pgfsetstrokecolor{currentstroke}%
\pgfsetdash{}{0pt}%
\pgfpathmoveto{\pgfqpoint{4.592334in}{2.008638in}}%
\pgfpathcurveto{\pgfqpoint{4.603384in}{2.008638in}}{\pgfqpoint{4.613983in}{2.013028in}}{\pgfqpoint{4.621797in}{2.020842in}}%
\pgfpathcurveto{\pgfqpoint{4.629611in}{2.028655in}}{\pgfqpoint{4.634001in}{2.039254in}}{\pgfqpoint{4.634001in}{2.050304in}}%
\pgfpathcurveto{\pgfqpoint{4.634001in}{2.061354in}}{\pgfqpoint{4.629611in}{2.071953in}}{\pgfqpoint{4.621797in}{2.079767in}}%
\pgfpathcurveto{\pgfqpoint{4.613983in}{2.087581in}}{\pgfqpoint{4.603384in}{2.091971in}}{\pgfqpoint{4.592334in}{2.091971in}}%
\pgfpathcurveto{\pgfqpoint{4.581284in}{2.091971in}}{\pgfqpoint{4.570685in}{2.087581in}}{\pgfqpoint{4.562871in}{2.079767in}}%
\pgfpathcurveto{\pgfqpoint{4.555058in}{2.071953in}}{\pgfqpoint{4.550668in}{2.061354in}}{\pgfqpoint{4.550668in}{2.050304in}}%
\pgfpathcurveto{\pgfqpoint{4.550668in}{2.039254in}}{\pgfqpoint{4.555058in}{2.028655in}}{\pgfqpoint{4.562871in}{2.020842in}}%
\pgfpathcurveto{\pgfqpoint{4.570685in}{2.013028in}}{\pgfqpoint{4.581284in}{2.008638in}}{\pgfqpoint{4.592334in}{2.008638in}}%
\pgfpathclose%
\pgfusepath{stroke,fill}%
\end{pgfscope}%
\begin{pgfscope}%
\pgfpathrectangle{\pgfqpoint{0.481978in}{0.331635in}}{\pgfqpoint{4.960000in}{3.696000in}}%
\pgfusepath{clip}%
\pgfsetbuttcap%
\pgfsetroundjoin%
\definecolor{currentfill}{rgb}{0.631373,0.788235,0.956863}%
\pgfsetfillcolor{currentfill}%
\pgfsetlinewidth{0.481800pt}%
\definecolor{currentstroke}{rgb}{1.000000,1.000000,1.000000}%
\pgfsetstrokecolor{currentstroke}%
\pgfsetdash{}{0pt}%
\pgfpathmoveto{\pgfqpoint{3.044931in}{3.040522in}}%
\pgfpathcurveto{\pgfqpoint{3.055981in}{3.040522in}}{\pgfqpoint{3.066580in}{3.044913in}}{\pgfqpoint{3.074394in}{3.052726in}}%
\pgfpathcurveto{\pgfqpoint{3.082207in}{3.060540in}}{\pgfqpoint{3.086598in}{3.071139in}}{\pgfqpoint{3.086598in}{3.082189in}}%
\pgfpathcurveto{\pgfqpoint{3.086598in}{3.093239in}}{\pgfqpoint{3.082207in}{3.103838in}}{\pgfqpoint{3.074394in}{3.111652in}}%
\pgfpathcurveto{\pgfqpoint{3.066580in}{3.119465in}}{\pgfqpoint{3.055981in}{3.123856in}}{\pgfqpoint{3.044931in}{3.123856in}}%
\pgfpathcurveto{\pgfqpoint{3.033881in}{3.123856in}}{\pgfqpoint{3.023282in}{3.119465in}}{\pgfqpoint{3.015468in}{3.111652in}}%
\pgfpathcurveto{\pgfqpoint{3.007654in}{3.103838in}}{\pgfqpoint{3.003264in}{3.093239in}}{\pgfqpoint{3.003264in}{3.082189in}}%
\pgfpathcurveto{\pgfqpoint{3.003264in}{3.071139in}}{\pgfqpoint{3.007654in}{3.060540in}}{\pgfqpoint{3.015468in}{3.052726in}}%
\pgfpathcurveto{\pgfqpoint{3.023282in}{3.044913in}}{\pgfqpoint{3.033881in}{3.040522in}}{\pgfqpoint{3.044931in}{3.040522in}}%
\pgfpathclose%
\pgfusepath{stroke,fill}%
\end{pgfscope}%
\begin{pgfscope}%
\pgfpathrectangle{\pgfqpoint{0.481978in}{0.331635in}}{\pgfqpoint{4.960000in}{3.696000in}}%
\pgfusepath{clip}%
\pgfsetbuttcap%
\pgfsetroundjoin%
\definecolor{currentfill}{rgb}{0.631373,0.788235,0.956863}%
\pgfsetfillcolor{currentfill}%
\pgfsetlinewidth{0.481800pt}%
\definecolor{currentstroke}{rgb}{1.000000,1.000000,1.000000}%
\pgfsetstrokecolor{currentstroke}%
\pgfsetdash{}{0pt}%
\pgfpathmoveto{\pgfqpoint{3.543143in}{1.954217in}}%
\pgfpathcurveto{\pgfqpoint{3.554194in}{1.954217in}}{\pgfqpoint{3.564793in}{1.958607in}}{\pgfqpoint{3.572606in}{1.966421in}}%
\pgfpathcurveto{\pgfqpoint{3.580420in}{1.974234in}}{\pgfqpoint{3.584810in}{1.984834in}}{\pgfqpoint{3.584810in}{1.995884in}}%
\pgfpathcurveto{\pgfqpoint{3.584810in}{2.006934in}}{\pgfqpoint{3.580420in}{2.017533in}}{\pgfqpoint{3.572606in}{2.025346in}}%
\pgfpathcurveto{\pgfqpoint{3.564793in}{2.033160in}}{\pgfqpoint{3.554194in}{2.037550in}}{\pgfqpoint{3.543143in}{2.037550in}}%
\pgfpathcurveto{\pgfqpoint{3.532093in}{2.037550in}}{\pgfqpoint{3.521494in}{2.033160in}}{\pgfqpoint{3.513681in}{2.025346in}}%
\pgfpathcurveto{\pgfqpoint{3.505867in}{2.017533in}}{\pgfqpoint{3.501477in}{2.006934in}}{\pgfqpoint{3.501477in}{1.995884in}}%
\pgfpathcurveto{\pgfqpoint{3.501477in}{1.984834in}}{\pgfqpoint{3.505867in}{1.974234in}}{\pgfqpoint{3.513681in}{1.966421in}}%
\pgfpathcurveto{\pgfqpoint{3.521494in}{1.958607in}}{\pgfqpoint{3.532093in}{1.954217in}}{\pgfqpoint{3.543143in}{1.954217in}}%
\pgfpathclose%
\pgfusepath{stroke,fill}%
\end{pgfscope}%
\begin{pgfscope}%
\pgfpathrectangle{\pgfqpoint{0.481978in}{0.331635in}}{\pgfqpoint{4.960000in}{3.696000in}}%
\pgfusepath{clip}%
\pgfsetbuttcap%
\pgfsetroundjoin%
\definecolor{currentfill}{rgb}{0.631373,0.788235,0.956863}%
\pgfsetfillcolor{currentfill}%
\pgfsetlinewidth{0.481800pt}%
\definecolor{currentstroke}{rgb}{1.000000,1.000000,1.000000}%
\pgfsetstrokecolor{currentstroke}%
\pgfsetdash{}{0pt}%
\pgfpathmoveto{\pgfqpoint{3.150587in}{1.313110in}}%
\pgfpathcurveto{\pgfqpoint{3.161638in}{1.313110in}}{\pgfqpoint{3.172237in}{1.317500in}}{\pgfqpoint{3.180050in}{1.325314in}}%
\pgfpathcurveto{\pgfqpoint{3.187864in}{1.333128in}}{\pgfqpoint{3.192254in}{1.343727in}}{\pgfqpoint{3.192254in}{1.354777in}}%
\pgfpathcurveto{\pgfqpoint{3.192254in}{1.365827in}}{\pgfqpoint{3.187864in}{1.376426in}}{\pgfqpoint{3.180050in}{1.384239in}}%
\pgfpathcurveto{\pgfqpoint{3.172237in}{1.392053in}}{\pgfqpoint{3.161638in}{1.396443in}}{\pgfqpoint{3.150587in}{1.396443in}}%
\pgfpathcurveto{\pgfqpoint{3.139537in}{1.396443in}}{\pgfqpoint{3.128938in}{1.392053in}}{\pgfqpoint{3.121125in}{1.384239in}}%
\pgfpathcurveto{\pgfqpoint{3.113311in}{1.376426in}}{\pgfqpoint{3.108921in}{1.365827in}}{\pgfqpoint{3.108921in}{1.354777in}}%
\pgfpathcurveto{\pgfqpoint{3.108921in}{1.343727in}}{\pgfqpoint{3.113311in}{1.333128in}}{\pgfqpoint{3.121125in}{1.325314in}}%
\pgfpathcurveto{\pgfqpoint{3.128938in}{1.317500in}}{\pgfqpoint{3.139537in}{1.313110in}}{\pgfqpoint{3.150587in}{1.313110in}}%
\pgfpathclose%
\pgfusepath{stroke,fill}%
\end{pgfscope}%
\begin{pgfscope}%
\pgfpathrectangle{\pgfqpoint{0.481978in}{0.331635in}}{\pgfqpoint{4.960000in}{3.696000in}}%
\pgfusepath{clip}%
\pgfsetbuttcap%
\pgfsetroundjoin%
\definecolor{currentfill}{rgb}{0.631373,0.788235,0.956863}%
\pgfsetfillcolor{currentfill}%
\pgfsetlinewidth{0.481800pt}%
\definecolor{currentstroke}{rgb}{1.000000,1.000000,1.000000}%
\pgfsetstrokecolor{currentstroke}%
\pgfsetdash{}{0pt}%
\pgfpathmoveto{\pgfqpoint{3.468441in}{1.802520in}}%
\pgfpathcurveto{\pgfqpoint{3.479491in}{1.802520in}}{\pgfqpoint{3.490090in}{1.806910in}}{\pgfqpoint{3.497904in}{1.814724in}}%
\pgfpathcurveto{\pgfqpoint{3.505718in}{1.822538in}}{\pgfqpoint{3.510108in}{1.833137in}}{\pgfqpoint{3.510108in}{1.844187in}}%
\pgfpathcurveto{\pgfqpoint{3.510108in}{1.855237in}}{\pgfqpoint{3.505718in}{1.865836in}}{\pgfqpoint{3.497904in}{1.873650in}}%
\pgfpathcurveto{\pgfqpoint{3.490090in}{1.881463in}}{\pgfqpoint{3.479491in}{1.885853in}}{\pgfqpoint{3.468441in}{1.885853in}}%
\pgfpathcurveto{\pgfqpoint{3.457391in}{1.885853in}}{\pgfqpoint{3.446792in}{1.881463in}}{\pgfqpoint{3.438978in}{1.873650in}}%
\pgfpathcurveto{\pgfqpoint{3.431165in}{1.865836in}}{\pgfqpoint{3.426775in}{1.855237in}}{\pgfqpoint{3.426775in}{1.844187in}}%
\pgfpathcurveto{\pgfqpoint{3.426775in}{1.833137in}}{\pgfqpoint{3.431165in}{1.822538in}}{\pgfqpoint{3.438978in}{1.814724in}}%
\pgfpathcurveto{\pgfqpoint{3.446792in}{1.806910in}}{\pgfqpoint{3.457391in}{1.802520in}}{\pgfqpoint{3.468441in}{1.802520in}}%
\pgfpathclose%
\pgfusepath{stroke,fill}%
\end{pgfscope}%
\begin{pgfscope}%
\pgfpathrectangle{\pgfqpoint{0.481978in}{0.331635in}}{\pgfqpoint{4.960000in}{3.696000in}}%
\pgfusepath{clip}%
\pgfsetbuttcap%
\pgfsetroundjoin%
\definecolor{currentfill}{rgb}{0.631373,0.788235,0.956863}%
\pgfsetfillcolor{currentfill}%
\pgfsetlinewidth{0.481800pt}%
\definecolor{currentstroke}{rgb}{1.000000,1.000000,1.000000}%
\pgfsetstrokecolor{currentstroke}%
\pgfsetdash{}{0pt}%
\pgfpathmoveto{\pgfqpoint{4.101947in}{2.169723in}}%
\pgfpathcurveto{\pgfqpoint{4.112997in}{2.169723in}}{\pgfqpoint{4.123597in}{2.174113in}}{\pgfqpoint{4.131410in}{2.181927in}}%
\pgfpathcurveto{\pgfqpoint{4.139224in}{2.189741in}}{\pgfqpoint{4.143614in}{2.200340in}}{\pgfqpoint{4.143614in}{2.211390in}}%
\pgfpathcurveto{\pgfqpoint{4.143614in}{2.222440in}}{\pgfqpoint{4.139224in}{2.233039in}}{\pgfqpoint{4.131410in}{2.240853in}}%
\pgfpathcurveto{\pgfqpoint{4.123597in}{2.248666in}}{\pgfqpoint{4.112997in}{2.253057in}}{\pgfqpoint{4.101947in}{2.253057in}}%
\pgfpathcurveto{\pgfqpoint{4.090897in}{2.253057in}}{\pgfqpoint{4.080298in}{2.248666in}}{\pgfqpoint{4.072485in}{2.240853in}}%
\pgfpathcurveto{\pgfqpoint{4.064671in}{2.233039in}}{\pgfqpoint{4.060281in}{2.222440in}}{\pgfqpoint{4.060281in}{2.211390in}}%
\pgfpathcurveto{\pgfqpoint{4.060281in}{2.200340in}}{\pgfqpoint{4.064671in}{2.189741in}}{\pgfqpoint{4.072485in}{2.181927in}}%
\pgfpathcurveto{\pgfqpoint{4.080298in}{2.174113in}}{\pgfqpoint{4.090897in}{2.169723in}}{\pgfqpoint{4.101947in}{2.169723in}}%
\pgfpathclose%
\pgfusepath{stroke,fill}%
\end{pgfscope}%
\begin{pgfscope}%
\pgfpathrectangle{\pgfqpoint{0.481978in}{0.331635in}}{\pgfqpoint{4.960000in}{3.696000in}}%
\pgfusepath{clip}%
\pgfsetbuttcap%
\pgfsetroundjoin%
\definecolor{currentfill}{rgb}{0.631373,0.788235,0.956863}%
\pgfsetfillcolor{currentfill}%
\pgfsetlinewidth{0.481800pt}%
\definecolor{currentstroke}{rgb}{1.000000,1.000000,1.000000}%
\pgfsetstrokecolor{currentstroke}%
\pgfsetdash{}{0pt}%
\pgfpathmoveto{\pgfqpoint{3.201064in}{2.591445in}}%
\pgfpathcurveto{\pgfqpoint{3.212114in}{2.591445in}}{\pgfqpoint{3.222713in}{2.595835in}}{\pgfqpoint{3.230527in}{2.603649in}}%
\pgfpathcurveto{\pgfqpoint{3.238341in}{2.611462in}}{\pgfqpoint{3.242731in}{2.622061in}}{\pgfqpoint{3.242731in}{2.633111in}}%
\pgfpathcurveto{\pgfqpoint{3.242731in}{2.644162in}}{\pgfqpoint{3.238341in}{2.654761in}}{\pgfqpoint{3.230527in}{2.662574in}}%
\pgfpathcurveto{\pgfqpoint{3.222713in}{2.670388in}}{\pgfqpoint{3.212114in}{2.674778in}}{\pgfqpoint{3.201064in}{2.674778in}}%
\pgfpathcurveto{\pgfqpoint{3.190014in}{2.674778in}}{\pgfqpoint{3.179415in}{2.670388in}}{\pgfqpoint{3.171601in}{2.662574in}}%
\pgfpathcurveto{\pgfqpoint{3.163788in}{2.654761in}}{\pgfqpoint{3.159398in}{2.644162in}}{\pgfqpoint{3.159398in}{2.633111in}}%
\pgfpathcurveto{\pgfqpoint{3.159398in}{2.622061in}}{\pgfqpoint{3.163788in}{2.611462in}}{\pgfqpoint{3.171601in}{2.603649in}}%
\pgfpathcurveto{\pgfqpoint{3.179415in}{2.595835in}}{\pgfqpoint{3.190014in}{2.591445in}}{\pgfqpoint{3.201064in}{2.591445in}}%
\pgfpathclose%
\pgfusepath{stroke,fill}%
\end{pgfscope}%
\begin{pgfscope}%
\pgfpathrectangle{\pgfqpoint{0.481978in}{0.331635in}}{\pgfqpoint{4.960000in}{3.696000in}}%
\pgfusepath{clip}%
\pgfsetbuttcap%
\pgfsetroundjoin%
\definecolor{currentfill}{rgb}{0.631373,0.788235,0.956863}%
\pgfsetfillcolor{currentfill}%
\pgfsetlinewidth{0.481800pt}%
\definecolor{currentstroke}{rgb}{1.000000,1.000000,1.000000}%
\pgfsetstrokecolor{currentstroke}%
\pgfsetdash{}{0pt}%
\pgfpathmoveto{\pgfqpoint{3.278230in}{0.555711in}}%
\pgfpathcurveto{\pgfqpoint{3.289280in}{0.555711in}}{\pgfqpoint{3.299879in}{0.560101in}}{\pgfqpoint{3.307693in}{0.567914in}}%
\pgfpathcurveto{\pgfqpoint{3.315507in}{0.575728in}}{\pgfqpoint{3.319897in}{0.586327in}}{\pgfqpoint{3.319897in}{0.597377in}}%
\pgfpathcurveto{\pgfqpoint{3.319897in}{0.608427in}}{\pgfqpoint{3.315507in}{0.619026in}}{\pgfqpoint{3.307693in}{0.626840in}}%
\pgfpathcurveto{\pgfqpoint{3.299879in}{0.634654in}}{\pgfqpoint{3.289280in}{0.639044in}}{\pgfqpoint{3.278230in}{0.639044in}}%
\pgfpathcurveto{\pgfqpoint{3.267180in}{0.639044in}}{\pgfqpoint{3.256581in}{0.634654in}}{\pgfqpoint{3.248768in}{0.626840in}}%
\pgfpathcurveto{\pgfqpoint{3.240954in}{0.619026in}}{\pgfqpoint{3.236564in}{0.608427in}}{\pgfqpoint{3.236564in}{0.597377in}}%
\pgfpathcurveto{\pgfqpoint{3.236564in}{0.586327in}}{\pgfqpoint{3.240954in}{0.575728in}}{\pgfqpoint{3.248768in}{0.567914in}}%
\pgfpathcurveto{\pgfqpoint{3.256581in}{0.560101in}}{\pgfqpoint{3.267180in}{0.555711in}}{\pgfqpoint{3.278230in}{0.555711in}}%
\pgfpathclose%
\pgfusepath{stroke,fill}%
\end{pgfscope}%
\begin{pgfscope}%
\pgfpathrectangle{\pgfqpoint{0.481978in}{0.331635in}}{\pgfqpoint{4.960000in}{3.696000in}}%
\pgfusepath{clip}%
\pgfsetbuttcap%
\pgfsetroundjoin%
\definecolor{currentfill}{rgb}{0.631373,0.788235,0.956863}%
\pgfsetfillcolor{currentfill}%
\pgfsetlinewidth{0.481800pt}%
\definecolor{currentstroke}{rgb}{1.000000,1.000000,1.000000}%
\pgfsetstrokecolor{currentstroke}%
\pgfsetdash{}{0pt}%
\pgfpathmoveto{\pgfqpoint{2.957637in}{0.783386in}}%
\pgfpathcurveto{\pgfqpoint{2.968687in}{0.783386in}}{\pgfqpoint{2.979286in}{0.787776in}}{\pgfqpoint{2.987100in}{0.795589in}}%
\pgfpathcurveto{\pgfqpoint{2.994913in}{0.803403in}}{\pgfqpoint{2.999304in}{0.814002in}}{\pgfqpoint{2.999304in}{0.825052in}}%
\pgfpathcurveto{\pgfqpoint{2.999304in}{0.836102in}}{\pgfqpoint{2.994913in}{0.846701in}}{\pgfqpoint{2.987100in}{0.854515in}}%
\pgfpathcurveto{\pgfqpoint{2.979286in}{0.862329in}}{\pgfqpoint{2.968687in}{0.866719in}}{\pgfqpoint{2.957637in}{0.866719in}}%
\pgfpathcurveto{\pgfqpoint{2.946587in}{0.866719in}}{\pgfqpoint{2.935988in}{0.862329in}}{\pgfqpoint{2.928174in}{0.854515in}}%
\pgfpathcurveto{\pgfqpoint{2.920361in}{0.846701in}}{\pgfqpoint{2.915970in}{0.836102in}}{\pgfqpoint{2.915970in}{0.825052in}}%
\pgfpathcurveto{\pgfqpoint{2.915970in}{0.814002in}}{\pgfqpoint{2.920361in}{0.803403in}}{\pgfqpoint{2.928174in}{0.795589in}}%
\pgfpathcurveto{\pgfqpoint{2.935988in}{0.787776in}}{\pgfqpoint{2.946587in}{0.783386in}}{\pgfqpoint{2.957637in}{0.783386in}}%
\pgfpathclose%
\pgfusepath{stroke,fill}%
\end{pgfscope}%
\begin{pgfscope}%
\pgfpathrectangle{\pgfqpoint{0.481978in}{0.331635in}}{\pgfqpoint{4.960000in}{3.696000in}}%
\pgfusepath{clip}%
\pgfsetbuttcap%
\pgfsetroundjoin%
\definecolor{currentfill}{rgb}{0.631373,0.788235,0.956863}%
\pgfsetfillcolor{currentfill}%
\pgfsetlinewidth{0.481800pt}%
\definecolor{currentstroke}{rgb}{1.000000,1.000000,1.000000}%
\pgfsetstrokecolor{currentstroke}%
\pgfsetdash{}{0pt}%
\pgfpathmoveto{\pgfqpoint{3.039657in}{2.875901in}}%
\pgfpathcurveto{\pgfqpoint{3.050708in}{2.875901in}}{\pgfqpoint{3.061307in}{2.880291in}}{\pgfqpoint{3.069120in}{2.888105in}}%
\pgfpathcurveto{\pgfqpoint{3.076934in}{2.895918in}}{\pgfqpoint{3.081324in}{2.906517in}}{\pgfqpoint{3.081324in}{2.917567in}}%
\pgfpathcurveto{\pgfqpoint{3.081324in}{2.928618in}}{\pgfqpoint{3.076934in}{2.939217in}}{\pgfqpoint{3.069120in}{2.947030in}}%
\pgfpathcurveto{\pgfqpoint{3.061307in}{2.954844in}}{\pgfqpoint{3.050708in}{2.959234in}}{\pgfqpoint{3.039657in}{2.959234in}}%
\pgfpathcurveto{\pgfqpoint{3.028607in}{2.959234in}}{\pgfqpoint{3.018008in}{2.954844in}}{\pgfqpoint{3.010195in}{2.947030in}}%
\pgfpathcurveto{\pgfqpoint{3.002381in}{2.939217in}}{\pgfqpoint{2.997991in}{2.928618in}}{\pgfqpoint{2.997991in}{2.917567in}}%
\pgfpathcurveto{\pgfqpoint{2.997991in}{2.906517in}}{\pgfqpoint{3.002381in}{2.895918in}}{\pgfqpoint{3.010195in}{2.888105in}}%
\pgfpathcurveto{\pgfqpoint{3.018008in}{2.880291in}}{\pgfqpoint{3.028607in}{2.875901in}}{\pgfqpoint{3.039657in}{2.875901in}}%
\pgfpathclose%
\pgfusepath{stroke,fill}%
\end{pgfscope}%
\begin{pgfscope}%
\pgfpathrectangle{\pgfqpoint{0.481978in}{0.331635in}}{\pgfqpoint{4.960000in}{3.696000in}}%
\pgfusepath{clip}%
\pgfsetbuttcap%
\pgfsetroundjoin%
\definecolor{currentfill}{rgb}{0.631373,0.788235,0.956863}%
\pgfsetfillcolor{currentfill}%
\pgfsetlinewidth{0.481800pt}%
\definecolor{currentstroke}{rgb}{1.000000,1.000000,1.000000}%
\pgfsetstrokecolor{currentstroke}%
\pgfsetdash{}{0pt}%
\pgfpathmoveto{\pgfqpoint{4.156124in}{2.452996in}}%
\pgfpathcurveto{\pgfqpoint{4.167174in}{2.452996in}}{\pgfqpoint{4.177773in}{2.457387in}}{\pgfqpoint{4.185587in}{2.465200in}}%
\pgfpathcurveto{\pgfqpoint{4.193400in}{2.473014in}}{\pgfqpoint{4.197791in}{2.483613in}}{\pgfqpoint{4.197791in}{2.494663in}}%
\pgfpathcurveto{\pgfqpoint{4.197791in}{2.505713in}}{\pgfqpoint{4.193400in}{2.516312in}}{\pgfqpoint{4.185587in}{2.524126in}}%
\pgfpathcurveto{\pgfqpoint{4.177773in}{2.531939in}}{\pgfqpoint{4.167174in}{2.536330in}}{\pgfqpoint{4.156124in}{2.536330in}}%
\pgfpathcurveto{\pgfqpoint{4.145074in}{2.536330in}}{\pgfqpoint{4.134475in}{2.531939in}}{\pgfqpoint{4.126661in}{2.524126in}}%
\pgfpathcurveto{\pgfqpoint{4.118848in}{2.516312in}}{\pgfqpoint{4.114457in}{2.505713in}}{\pgfqpoint{4.114457in}{2.494663in}}%
\pgfpathcurveto{\pgfqpoint{4.114457in}{2.483613in}}{\pgfqpoint{4.118848in}{2.473014in}}{\pgfqpoint{4.126661in}{2.465200in}}%
\pgfpathcurveto{\pgfqpoint{4.134475in}{2.457387in}}{\pgfqpoint{4.145074in}{2.452996in}}{\pgfqpoint{4.156124in}{2.452996in}}%
\pgfpathclose%
\pgfusepath{stroke,fill}%
\end{pgfscope}%
\begin{pgfscope}%
\pgfpathrectangle{\pgfqpoint{0.481978in}{0.331635in}}{\pgfqpoint{4.960000in}{3.696000in}}%
\pgfusepath{clip}%
\pgfsetbuttcap%
\pgfsetroundjoin%
\definecolor{currentfill}{rgb}{0.631373,0.788235,0.956863}%
\pgfsetfillcolor{currentfill}%
\pgfsetlinewidth{0.481800pt}%
\definecolor{currentstroke}{rgb}{1.000000,1.000000,1.000000}%
\pgfsetstrokecolor{currentstroke}%
\pgfsetdash{}{0pt}%
\pgfpathmoveto{\pgfqpoint{2.647467in}{0.511185in}}%
\pgfpathcurveto{\pgfqpoint{2.658517in}{0.511185in}}{\pgfqpoint{2.669116in}{0.515575in}}{\pgfqpoint{2.676930in}{0.523389in}}%
\pgfpathcurveto{\pgfqpoint{2.684743in}{0.531203in}}{\pgfqpoint{2.689134in}{0.541802in}}{\pgfqpoint{2.689134in}{0.552852in}}%
\pgfpathcurveto{\pgfqpoint{2.689134in}{0.563902in}}{\pgfqpoint{2.684743in}{0.574501in}}{\pgfqpoint{2.676930in}{0.582314in}}%
\pgfpathcurveto{\pgfqpoint{2.669116in}{0.590128in}}{\pgfqpoint{2.658517in}{0.594518in}}{\pgfqpoint{2.647467in}{0.594518in}}%
\pgfpathcurveto{\pgfqpoint{2.636417in}{0.594518in}}{\pgfqpoint{2.625818in}{0.590128in}}{\pgfqpoint{2.618004in}{0.582314in}}%
\pgfpathcurveto{\pgfqpoint{2.610190in}{0.574501in}}{\pgfqpoint{2.605800in}{0.563902in}}{\pgfqpoint{2.605800in}{0.552852in}}%
\pgfpathcurveto{\pgfqpoint{2.605800in}{0.541802in}}{\pgfqpoint{2.610190in}{0.531203in}}{\pgfqpoint{2.618004in}{0.523389in}}%
\pgfpathcurveto{\pgfqpoint{2.625818in}{0.515575in}}{\pgfqpoint{2.636417in}{0.511185in}}{\pgfqpoint{2.647467in}{0.511185in}}%
\pgfpathclose%
\pgfusepath{stroke,fill}%
\end{pgfscope}%
\begin{pgfscope}%
\pgfpathrectangle{\pgfqpoint{0.481978in}{0.331635in}}{\pgfqpoint{4.960000in}{3.696000in}}%
\pgfusepath{clip}%
\pgfsetbuttcap%
\pgfsetroundjoin%
\definecolor{currentfill}{rgb}{0.631373,0.788235,0.956863}%
\pgfsetfillcolor{currentfill}%
\pgfsetlinewidth{0.481800pt}%
\definecolor{currentstroke}{rgb}{1.000000,1.000000,1.000000}%
\pgfsetstrokecolor{currentstroke}%
\pgfsetdash{}{0pt}%
\pgfpathmoveto{\pgfqpoint{3.643574in}{3.816638in}}%
\pgfpathcurveto{\pgfqpoint{3.654624in}{3.816638in}}{\pgfqpoint{3.665223in}{3.821028in}}{\pgfqpoint{3.673037in}{3.828842in}}%
\pgfpathcurveto{\pgfqpoint{3.680850in}{3.836655in}}{\pgfqpoint{3.685241in}{3.847254in}}{\pgfqpoint{3.685241in}{3.858304in}}%
\pgfpathcurveto{\pgfqpoint{3.685241in}{3.869354in}}{\pgfqpoint{3.680850in}{3.879954in}}{\pgfqpoint{3.673037in}{3.887767in}}%
\pgfpathcurveto{\pgfqpoint{3.665223in}{3.895581in}}{\pgfqpoint{3.654624in}{3.899971in}}{\pgfqpoint{3.643574in}{3.899971in}}%
\pgfpathcurveto{\pgfqpoint{3.632524in}{3.899971in}}{\pgfqpoint{3.621925in}{3.895581in}}{\pgfqpoint{3.614111in}{3.887767in}}%
\pgfpathcurveto{\pgfqpoint{3.606298in}{3.879954in}}{\pgfqpoint{3.601907in}{3.869354in}}{\pgfqpoint{3.601907in}{3.858304in}}%
\pgfpathcurveto{\pgfqpoint{3.601907in}{3.847254in}}{\pgfqpoint{3.606298in}{3.836655in}}{\pgfqpoint{3.614111in}{3.828842in}}%
\pgfpathcurveto{\pgfqpoint{3.621925in}{3.821028in}}{\pgfqpoint{3.632524in}{3.816638in}}{\pgfqpoint{3.643574in}{3.816638in}}%
\pgfpathclose%
\pgfusepath{stroke,fill}%
\end{pgfscope}%
\begin{pgfscope}%
\pgfpathrectangle{\pgfqpoint{0.481978in}{0.331635in}}{\pgfqpoint{4.960000in}{3.696000in}}%
\pgfusepath{clip}%
\pgfsetbuttcap%
\pgfsetroundjoin%
\definecolor{currentfill}{rgb}{0.631373,0.788235,0.956863}%
\pgfsetfillcolor{currentfill}%
\pgfsetlinewidth{0.481800pt}%
\definecolor{currentstroke}{rgb}{1.000000,1.000000,1.000000}%
\pgfsetstrokecolor{currentstroke}%
\pgfsetdash{}{0pt}%
\pgfpathmoveto{\pgfqpoint{3.610736in}{2.596234in}}%
\pgfpathcurveto{\pgfqpoint{3.621786in}{2.596234in}}{\pgfqpoint{3.632385in}{2.600624in}}{\pgfqpoint{3.640198in}{2.608438in}}%
\pgfpathcurveto{\pgfqpoint{3.648012in}{2.616252in}}{\pgfqpoint{3.652402in}{2.626851in}}{\pgfqpoint{3.652402in}{2.637901in}}%
\pgfpathcurveto{\pgfqpoint{3.652402in}{2.648951in}}{\pgfqpoint{3.648012in}{2.659550in}}{\pgfqpoint{3.640198in}{2.667364in}}%
\pgfpathcurveto{\pgfqpoint{3.632385in}{2.675177in}}{\pgfqpoint{3.621786in}{2.679567in}}{\pgfqpoint{3.610736in}{2.679567in}}%
\pgfpathcurveto{\pgfqpoint{3.599685in}{2.679567in}}{\pgfqpoint{3.589086in}{2.675177in}}{\pgfqpoint{3.581273in}{2.667364in}}%
\pgfpathcurveto{\pgfqpoint{3.573459in}{2.659550in}}{\pgfqpoint{3.569069in}{2.648951in}}{\pgfqpoint{3.569069in}{2.637901in}}%
\pgfpathcurveto{\pgfqpoint{3.569069in}{2.626851in}}{\pgfqpoint{3.573459in}{2.616252in}}{\pgfqpoint{3.581273in}{2.608438in}}%
\pgfpathcurveto{\pgfqpoint{3.589086in}{2.600624in}}{\pgfqpoint{3.599685in}{2.596234in}}{\pgfqpoint{3.610736in}{2.596234in}}%
\pgfpathclose%
\pgfusepath{stroke,fill}%
\end{pgfscope}%
\begin{pgfscope}%
\pgfpathrectangle{\pgfqpoint{0.481978in}{0.331635in}}{\pgfqpoint{4.960000in}{3.696000in}}%
\pgfusepath{clip}%
\pgfsetbuttcap%
\pgfsetroundjoin%
\definecolor{currentfill}{rgb}{0.631373,0.788235,0.956863}%
\pgfsetfillcolor{currentfill}%
\pgfsetlinewidth{0.481800pt}%
\definecolor{currentstroke}{rgb}{1.000000,1.000000,1.000000}%
\pgfsetstrokecolor{currentstroke}%
\pgfsetdash{}{0pt}%
\pgfpathmoveto{\pgfqpoint{3.986586in}{2.328037in}}%
\pgfpathcurveto{\pgfqpoint{3.997636in}{2.328037in}}{\pgfqpoint{4.008235in}{2.332427in}}{\pgfqpoint{4.016049in}{2.340241in}}%
\pgfpathcurveto{\pgfqpoint{4.023862in}{2.348054in}}{\pgfqpoint{4.028252in}{2.358653in}}{\pgfqpoint{4.028252in}{2.369703in}}%
\pgfpathcurveto{\pgfqpoint{4.028252in}{2.380754in}}{\pgfqpoint{4.023862in}{2.391353in}}{\pgfqpoint{4.016049in}{2.399166in}}%
\pgfpathcurveto{\pgfqpoint{4.008235in}{2.406980in}}{\pgfqpoint{3.997636in}{2.411370in}}{\pgfqpoint{3.986586in}{2.411370in}}%
\pgfpathcurveto{\pgfqpoint{3.975536in}{2.411370in}}{\pgfqpoint{3.964937in}{2.406980in}}{\pgfqpoint{3.957123in}{2.399166in}}%
\pgfpathcurveto{\pgfqpoint{3.949309in}{2.391353in}}{\pgfqpoint{3.944919in}{2.380754in}}{\pgfqpoint{3.944919in}{2.369703in}}%
\pgfpathcurveto{\pgfqpoint{3.944919in}{2.358653in}}{\pgfqpoint{3.949309in}{2.348054in}}{\pgfqpoint{3.957123in}{2.340241in}}%
\pgfpathcurveto{\pgfqpoint{3.964937in}{2.332427in}}{\pgfqpoint{3.975536in}{2.328037in}}{\pgfqpoint{3.986586in}{2.328037in}}%
\pgfpathclose%
\pgfusepath{stroke,fill}%
\end{pgfscope}%
\begin{pgfscope}%
\pgfpathrectangle{\pgfqpoint{0.481978in}{0.331635in}}{\pgfqpoint{4.960000in}{3.696000in}}%
\pgfusepath{clip}%
\pgfsetbuttcap%
\pgfsetroundjoin%
\definecolor{currentfill}{rgb}{0.631373,0.788235,0.956863}%
\pgfsetfillcolor{currentfill}%
\pgfsetlinewidth{0.481800pt}%
\definecolor{currentstroke}{rgb}{1.000000,1.000000,1.000000}%
\pgfsetstrokecolor{currentstroke}%
\pgfsetdash{}{0pt}%
\pgfpathmoveto{\pgfqpoint{3.408961in}{2.730285in}}%
\pgfpathcurveto{\pgfqpoint{3.420011in}{2.730285in}}{\pgfqpoint{3.430610in}{2.734675in}}{\pgfqpoint{3.438424in}{2.742489in}}%
\pgfpathcurveto{\pgfqpoint{3.446238in}{2.750302in}}{\pgfqpoint{3.450628in}{2.760901in}}{\pgfqpoint{3.450628in}{2.771952in}}%
\pgfpathcurveto{\pgfqpoint{3.450628in}{2.783002in}}{\pgfqpoint{3.446238in}{2.793601in}}{\pgfqpoint{3.438424in}{2.801414in}}%
\pgfpathcurveto{\pgfqpoint{3.430610in}{2.809228in}}{\pgfqpoint{3.420011in}{2.813618in}}{\pgfqpoint{3.408961in}{2.813618in}}%
\pgfpathcurveto{\pgfqpoint{3.397911in}{2.813618in}}{\pgfqpoint{3.387312in}{2.809228in}}{\pgfqpoint{3.379498in}{2.801414in}}%
\pgfpathcurveto{\pgfqpoint{3.371685in}{2.793601in}}{\pgfqpoint{3.367294in}{2.783002in}}{\pgfqpoint{3.367294in}{2.771952in}}%
\pgfpathcurveto{\pgfqpoint{3.367294in}{2.760901in}}{\pgfqpoint{3.371685in}{2.750302in}}{\pgfqpoint{3.379498in}{2.742489in}}%
\pgfpathcurveto{\pgfqpoint{3.387312in}{2.734675in}}{\pgfqpoint{3.397911in}{2.730285in}}{\pgfqpoint{3.408961in}{2.730285in}}%
\pgfpathclose%
\pgfusepath{stroke,fill}%
\end{pgfscope}%
\begin{pgfscope}%
\pgfpathrectangle{\pgfqpoint{0.481978in}{0.331635in}}{\pgfqpoint{4.960000in}{3.696000in}}%
\pgfusepath{clip}%
\pgfsetbuttcap%
\pgfsetroundjoin%
\definecolor{currentfill}{rgb}{0.631373,0.788235,0.956863}%
\pgfsetfillcolor{currentfill}%
\pgfsetlinewidth{0.481800pt}%
\definecolor{currentstroke}{rgb}{1.000000,1.000000,1.000000}%
\pgfsetstrokecolor{currentstroke}%
\pgfsetdash{}{0pt}%
\pgfpathmoveto{\pgfqpoint{2.884018in}{1.258956in}}%
\pgfpathcurveto{\pgfqpoint{2.895068in}{1.258956in}}{\pgfqpoint{2.905667in}{1.263346in}}{\pgfqpoint{2.913480in}{1.271160in}}%
\pgfpathcurveto{\pgfqpoint{2.921294in}{1.278973in}}{\pgfqpoint{2.925684in}{1.289572in}}{\pgfqpoint{2.925684in}{1.300623in}}%
\pgfpathcurveto{\pgfqpoint{2.925684in}{1.311673in}}{\pgfqpoint{2.921294in}{1.322272in}}{\pgfqpoint{2.913480in}{1.330085in}}%
\pgfpathcurveto{\pgfqpoint{2.905667in}{1.337899in}}{\pgfqpoint{2.895068in}{1.342289in}}{\pgfqpoint{2.884018in}{1.342289in}}%
\pgfpathcurveto{\pgfqpoint{2.872967in}{1.342289in}}{\pgfqpoint{2.862368in}{1.337899in}}{\pgfqpoint{2.854555in}{1.330085in}}%
\pgfpathcurveto{\pgfqpoint{2.846741in}{1.322272in}}{\pgfqpoint{2.842351in}{1.311673in}}{\pgfqpoint{2.842351in}{1.300623in}}%
\pgfpathcurveto{\pgfqpoint{2.842351in}{1.289572in}}{\pgfqpoint{2.846741in}{1.278973in}}{\pgfqpoint{2.854555in}{1.271160in}}%
\pgfpathcurveto{\pgfqpoint{2.862368in}{1.263346in}}{\pgfqpoint{2.872967in}{1.258956in}}{\pgfqpoint{2.884018in}{1.258956in}}%
\pgfpathclose%
\pgfusepath{stroke,fill}%
\end{pgfscope}%
\begin{pgfscope}%
\pgfpathrectangle{\pgfqpoint{0.481978in}{0.331635in}}{\pgfqpoint{4.960000in}{3.696000in}}%
\pgfusepath{clip}%
\pgfsetbuttcap%
\pgfsetroundjoin%
\definecolor{currentfill}{rgb}{0.631373,0.788235,0.956863}%
\pgfsetfillcolor{currentfill}%
\pgfsetlinewidth{0.481800pt}%
\definecolor{currentstroke}{rgb}{1.000000,1.000000,1.000000}%
\pgfsetstrokecolor{currentstroke}%
\pgfsetdash{}{0pt}%
\pgfpathmoveto{\pgfqpoint{2.310399in}{0.965389in}}%
\pgfpathcurveto{\pgfqpoint{2.321449in}{0.965389in}}{\pgfqpoint{2.332048in}{0.969779in}}{\pgfqpoint{2.339862in}{0.977593in}}%
\pgfpathcurveto{\pgfqpoint{2.347675in}{0.985407in}}{\pgfqpoint{2.352066in}{0.996006in}}{\pgfqpoint{2.352066in}{1.007056in}}%
\pgfpathcurveto{\pgfqpoint{2.352066in}{1.018106in}}{\pgfqpoint{2.347675in}{1.028705in}}{\pgfqpoint{2.339862in}{1.036519in}}%
\pgfpathcurveto{\pgfqpoint{2.332048in}{1.044332in}}{\pgfqpoint{2.321449in}{1.048723in}}{\pgfqpoint{2.310399in}{1.048723in}}%
\pgfpathcurveto{\pgfqpoint{2.299349in}{1.048723in}}{\pgfqpoint{2.288750in}{1.044332in}}{\pgfqpoint{2.280936in}{1.036519in}}%
\pgfpathcurveto{\pgfqpoint{2.273123in}{1.028705in}}{\pgfqpoint{2.268732in}{1.018106in}}{\pgfqpoint{2.268732in}{1.007056in}}%
\pgfpathcurveto{\pgfqpoint{2.268732in}{0.996006in}}{\pgfqpoint{2.273123in}{0.985407in}}{\pgfqpoint{2.280936in}{0.977593in}}%
\pgfpathcurveto{\pgfqpoint{2.288750in}{0.969779in}}{\pgfqpoint{2.299349in}{0.965389in}}{\pgfqpoint{2.310399in}{0.965389in}}%
\pgfpathclose%
\pgfusepath{stroke,fill}%
\end{pgfscope}%
\begin{pgfscope}%
\pgfpathrectangle{\pgfqpoint{0.481978in}{0.331635in}}{\pgfqpoint{4.960000in}{3.696000in}}%
\pgfusepath{clip}%
\pgfsetbuttcap%
\pgfsetroundjoin%
\definecolor{currentfill}{rgb}{0.631373,0.788235,0.956863}%
\pgfsetfillcolor{currentfill}%
\pgfsetlinewidth{0.481800pt}%
\definecolor{currentstroke}{rgb}{1.000000,1.000000,1.000000}%
\pgfsetstrokecolor{currentstroke}%
\pgfsetdash{}{0pt}%
\pgfpathmoveto{\pgfqpoint{3.865457in}{2.242701in}}%
\pgfpathcurveto{\pgfqpoint{3.876507in}{2.242701in}}{\pgfqpoint{3.887106in}{2.247091in}}{\pgfqpoint{3.894920in}{2.254905in}}%
\pgfpathcurveto{\pgfqpoint{3.902733in}{2.262718in}}{\pgfqpoint{3.907124in}{2.273317in}}{\pgfqpoint{3.907124in}{2.284367in}}%
\pgfpathcurveto{\pgfqpoint{3.907124in}{2.295418in}}{\pgfqpoint{3.902733in}{2.306017in}}{\pgfqpoint{3.894920in}{2.313830in}}%
\pgfpathcurveto{\pgfqpoint{3.887106in}{2.321644in}}{\pgfqpoint{3.876507in}{2.326034in}}{\pgfqpoint{3.865457in}{2.326034in}}%
\pgfpathcurveto{\pgfqpoint{3.854407in}{2.326034in}}{\pgfqpoint{3.843808in}{2.321644in}}{\pgfqpoint{3.835994in}{2.313830in}}%
\pgfpathcurveto{\pgfqpoint{3.828181in}{2.306017in}}{\pgfqpoint{3.823790in}{2.295418in}}{\pgfqpoint{3.823790in}{2.284367in}}%
\pgfpathcurveto{\pgfqpoint{3.823790in}{2.273317in}}{\pgfqpoint{3.828181in}{2.262718in}}{\pgfqpoint{3.835994in}{2.254905in}}%
\pgfpathcurveto{\pgfqpoint{3.843808in}{2.247091in}}{\pgfqpoint{3.854407in}{2.242701in}}{\pgfqpoint{3.865457in}{2.242701in}}%
\pgfpathclose%
\pgfusepath{stroke,fill}%
\end{pgfscope}%
\begin{pgfscope}%
\pgfpathrectangle{\pgfqpoint{0.481978in}{0.331635in}}{\pgfqpoint{4.960000in}{3.696000in}}%
\pgfusepath{clip}%
\pgfsetbuttcap%
\pgfsetroundjoin%
\definecolor{currentfill}{rgb}{0.631373,0.788235,0.956863}%
\pgfsetfillcolor{currentfill}%
\pgfsetlinewidth{0.481800pt}%
\definecolor{currentstroke}{rgb}{1.000000,1.000000,1.000000}%
\pgfsetstrokecolor{currentstroke}%
\pgfsetdash{}{0pt}%
\pgfpathmoveto{\pgfqpoint{2.629656in}{1.034137in}}%
\pgfpathcurveto{\pgfqpoint{2.640706in}{1.034137in}}{\pgfqpoint{2.651305in}{1.038528in}}{\pgfqpoint{2.659119in}{1.046341in}}%
\pgfpathcurveto{\pgfqpoint{2.666932in}{1.054155in}}{\pgfqpoint{2.671323in}{1.064754in}}{\pgfqpoint{2.671323in}{1.075804in}}%
\pgfpathcurveto{\pgfqpoint{2.671323in}{1.086854in}}{\pgfqpoint{2.666932in}{1.097453in}}{\pgfqpoint{2.659119in}{1.105267in}}%
\pgfpathcurveto{\pgfqpoint{2.651305in}{1.113081in}}{\pgfqpoint{2.640706in}{1.117471in}}{\pgfqpoint{2.629656in}{1.117471in}}%
\pgfpathcurveto{\pgfqpoint{2.618606in}{1.117471in}}{\pgfqpoint{2.608007in}{1.113081in}}{\pgfqpoint{2.600193in}{1.105267in}}%
\pgfpathcurveto{\pgfqpoint{2.592380in}{1.097453in}}{\pgfqpoint{2.587989in}{1.086854in}}{\pgfqpoint{2.587989in}{1.075804in}}%
\pgfpathcurveto{\pgfqpoint{2.587989in}{1.064754in}}{\pgfqpoint{2.592380in}{1.054155in}}{\pgfqpoint{2.600193in}{1.046341in}}%
\pgfpathcurveto{\pgfqpoint{2.608007in}{1.038528in}}{\pgfqpoint{2.618606in}{1.034137in}}{\pgfqpoint{2.629656in}{1.034137in}}%
\pgfpathclose%
\pgfusepath{stroke,fill}%
\end{pgfscope}%
\begin{pgfscope}%
\pgfpathrectangle{\pgfqpoint{0.481978in}{0.331635in}}{\pgfqpoint{4.960000in}{3.696000in}}%
\pgfusepath{clip}%
\pgfsetbuttcap%
\pgfsetroundjoin%
\definecolor{currentfill}{rgb}{0.631373,0.788235,0.956863}%
\pgfsetfillcolor{currentfill}%
\pgfsetlinewidth{0.481800pt}%
\definecolor{currentstroke}{rgb}{1.000000,1.000000,1.000000}%
\pgfsetstrokecolor{currentstroke}%
\pgfsetdash{}{0pt}%
\pgfpathmoveto{\pgfqpoint{4.210235in}{2.089423in}}%
\pgfpathcurveto{\pgfqpoint{4.221285in}{2.089423in}}{\pgfqpoint{4.231884in}{2.093813in}}{\pgfqpoint{4.239697in}{2.101627in}}%
\pgfpathcurveto{\pgfqpoint{4.247511in}{2.109440in}}{\pgfqpoint{4.251901in}{2.120039in}}{\pgfqpoint{4.251901in}{2.131089in}}%
\pgfpathcurveto{\pgfqpoint{4.251901in}{2.142140in}}{\pgfqpoint{4.247511in}{2.152739in}}{\pgfqpoint{4.239697in}{2.160552in}}%
\pgfpathcurveto{\pgfqpoint{4.231884in}{2.168366in}}{\pgfqpoint{4.221285in}{2.172756in}}{\pgfqpoint{4.210235in}{2.172756in}}%
\pgfpathcurveto{\pgfqpoint{4.199185in}{2.172756in}}{\pgfqpoint{4.188585in}{2.168366in}}{\pgfqpoint{4.180772in}{2.160552in}}%
\pgfpathcurveto{\pgfqpoint{4.172958in}{2.152739in}}{\pgfqpoint{4.168568in}{2.142140in}}{\pgfqpoint{4.168568in}{2.131089in}}%
\pgfpathcurveto{\pgfqpoint{4.168568in}{2.120039in}}{\pgfqpoint{4.172958in}{2.109440in}}{\pgfqpoint{4.180772in}{2.101627in}}%
\pgfpathcurveto{\pgfqpoint{4.188585in}{2.093813in}}{\pgfqpoint{4.199185in}{2.089423in}}{\pgfqpoint{4.210235in}{2.089423in}}%
\pgfpathclose%
\pgfusepath{stroke,fill}%
\end{pgfscope}%
\begin{pgfscope}%
\pgfpathrectangle{\pgfqpoint{0.481978in}{0.331635in}}{\pgfqpoint{4.960000in}{3.696000in}}%
\pgfusepath{clip}%
\pgfsetbuttcap%
\pgfsetroundjoin%
\definecolor{currentfill}{rgb}{0.631373,0.788235,0.956863}%
\pgfsetfillcolor{currentfill}%
\pgfsetlinewidth{0.481800pt}%
\definecolor{currentstroke}{rgb}{1.000000,1.000000,1.000000}%
\pgfsetstrokecolor{currentstroke}%
\pgfsetdash{}{0pt}%
\pgfpathmoveto{\pgfqpoint{4.728662in}{1.963251in}}%
\pgfpathcurveto{\pgfqpoint{4.739712in}{1.963251in}}{\pgfqpoint{4.750311in}{1.967641in}}{\pgfqpoint{4.758124in}{1.975455in}}%
\pgfpathcurveto{\pgfqpoint{4.765938in}{1.983269in}}{\pgfqpoint{4.770328in}{1.993868in}}{\pgfqpoint{4.770328in}{2.004918in}}%
\pgfpathcurveto{\pgfqpoint{4.770328in}{2.015968in}}{\pgfqpoint{4.765938in}{2.026567in}}{\pgfqpoint{4.758124in}{2.034381in}}%
\pgfpathcurveto{\pgfqpoint{4.750311in}{2.042194in}}{\pgfqpoint{4.739712in}{2.046584in}}{\pgfqpoint{4.728662in}{2.046584in}}%
\pgfpathcurveto{\pgfqpoint{4.717612in}{2.046584in}}{\pgfqpoint{4.707013in}{2.042194in}}{\pgfqpoint{4.699199in}{2.034381in}}%
\pgfpathcurveto{\pgfqpoint{4.691385in}{2.026567in}}{\pgfqpoint{4.686995in}{2.015968in}}{\pgfqpoint{4.686995in}{2.004918in}}%
\pgfpathcurveto{\pgfqpoint{4.686995in}{1.993868in}}{\pgfqpoint{4.691385in}{1.983269in}}{\pgfqpoint{4.699199in}{1.975455in}}%
\pgfpathcurveto{\pgfqpoint{4.707013in}{1.967641in}}{\pgfqpoint{4.717612in}{1.963251in}}{\pgfqpoint{4.728662in}{1.963251in}}%
\pgfpathclose%
\pgfusepath{stroke,fill}%
\end{pgfscope}%
\begin{pgfscope}%
\pgfpathrectangle{\pgfqpoint{0.481978in}{0.331635in}}{\pgfqpoint{4.960000in}{3.696000in}}%
\pgfusepath{clip}%
\pgfsetbuttcap%
\pgfsetroundjoin%
\definecolor{currentfill}{rgb}{0.631373,0.788235,0.956863}%
\pgfsetfillcolor{currentfill}%
\pgfsetlinewidth{0.481800pt}%
\definecolor{currentstroke}{rgb}{1.000000,1.000000,1.000000}%
\pgfsetstrokecolor{currentstroke}%
\pgfsetdash{}{0pt}%
\pgfpathmoveto{\pgfqpoint{4.043870in}{2.018350in}}%
\pgfpathcurveto{\pgfqpoint{4.054920in}{2.018350in}}{\pgfqpoint{4.065519in}{2.022740in}}{\pgfqpoint{4.073332in}{2.030553in}}%
\pgfpathcurveto{\pgfqpoint{4.081146in}{2.038367in}}{\pgfqpoint{4.085536in}{2.048966in}}{\pgfqpoint{4.085536in}{2.060016in}}%
\pgfpathcurveto{\pgfqpoint{4.085536in}{2.071066in}}{\pgfqpoint{4.081146in}{2.081665in}}{\pgfqpoint{4.073332in}{2.089479in}}%
\pgfpathcurveto{\pgfqpoint{4.065519in}{2.097293in}}{\pgfqpoint{4.054920in}{2.101683in}}{\pgfqpoint{4.043870in}{2.101683in}}%
\pgfpathcurveto{\pgfqpoint{4.032819in}{2.101683in}}{\pgfqpoint{4.022220in}{2.097293in}}{\pgfqpoint{4.014407in}{2.089479in}}%
\pgfpathcurveto{\pgfqpoint{4.006593in}{2.081665in}}{\pgfqpoint{4.002203in}{2.071066in}}{\pgfqpoint{4.002203in}{2.060016in}}%
\pgfpathcurveto{\pgfqpoint{4.002203in}{2.048966in}}{\pgfqpoint{4.006593in}{2.038367in}}{\pgfqpoint{4.014407in}{2.030553in}}%
\pgfpathcurveto{\pgfqpoint{4.022220in}{2.022740in}}{\pgfqpoint{4.032819in}{2.018350in}}{\pgfqpoint{4.043870in}{2.018350in}}%
\pgfpathclose%
\pgfusepath{stroke,fill}%
\end{pgfscope}%
\begin{pgfscope}%
\pgfpathrectangle{\pgfqpoint{0.481978in}{0.331635in}}{\pgfqpoint{4.960000in}{3.696000in}}%
\pgfusepath{clip}%
\pgfsetbuttcap%
\pgfsetroundjoin%
\definecolor{currentfill}{rgb}{0.631373,0.788235,0.956863}%
\pgfsetfillcolor{currentfill}%
\pgfsetlinewidth{0.481800pt}%
\definecolor{currentstroke}{rgb}{1.000000,1.000000,1.000000}%
\pgfsetstrokecolor{currentstroke}%
\pgfsetdash{}{0pt}%
\pgfpathmoveto{\pgfqpoint{3.893572in}{1.859486in}}%
\pgfpathcurveto{\pgfqpoint{3.904622in}{1.859486in}}{\pgfqpoint{3.915221in}{1.863877in}}{\pgfqpoint{3.923035in}{1.871690in}}%
\pgfpathcurveto{\pgfqpoint{3.930848in}{1.879504in}}{\pgfqpoint{3.935239in}{1.890103in}}{\pgfqpoint{3.935239in}{1.901153in}}%
\pgfpathcurveto{\pgfqpoint{3.935239in}{1.912203in}}{\pgfqpoint{3.930848in}{1.922802in}}{\pgfqpoint{3.923035in}{1.930616in}}%
\pgfpathcurveto{\pgfqpoint{3.915221in}{1.938429in}}{\pgfqpoint{3.904622in}{1.942820in}}{\pgfqpoint{3.893572in}{1.942820in}}%
\pgfpathcurveto{\pgfqpoint{3.882522in}{1.942820in}}{\pgfqpoint{3.871923in}{1.938429in}}{\pgfqpoint{3.864109in}{1.930616in}}%
\pgfpathcurveto{\pgfqpoint{3.856295in}{1.922802in}}{\pgfqpoint{3.851905in}{1.912203in}}{\pgfqpoint{3.851905in}{1.901153in}}%
\pgfpathcurveto{\pgfqpoint{3.851905in}{1.890103in}}{\pgfqpoint{3.856295in}{1.879504in}}{\pgfqpoint{3.864109in}{1.871690in}}%
\pgfpathcurveto{\pgfqpoint{3.871923in}{1.863877in}}{\pgfqpoint{3.882522in}{1.859486in}}{\pgfqpoint{3.893572in}{1.859486in}}%
\pgfpathclose%
\pgfusepath{stroke,fill}%
\end{pgfscope}%
\begin{pgfscope}%
\pgfpathrectangle{\pgfqpoint{0.481978in}{0.331635in}}{\pgfqpoint{4.960000in}{3.696000in}}%
\pgfusepath{clip}%
\pgfsetbuttcap%
\pgfsetroundjoin%
\definecolor{currentfill}{rgb}{0.631373,0.788235,0.956863}%
\pgfsetfillcolor{currentfill}%
\pgfsetlinewidth{0.481800pt}%
\definecolor{currentstroke}{rgb}{1.000000,1.000000,1.000000}%
\pgfsetstrokecolor{currentstroke}%
\pgfsetdash{}{0pt}%
\pgfpathmoveto{\pgfqpoint{3.653090in}{0.792507in}}%
\pgfpathcurveto{\pgfqpoint{3.664140in}{0.792507in}}{\pgfqpoint{3.674739in}{0.796898in}}{\pgfqpoint{3.682553in}{0.804711in}}%
\pgfpathcurveto{\pgfqpoint{3.690366in}{0.812525in}}{\pgfqpoint{3.694756in}{0.823124in}}{\pgfqpoint{3.694756in}{0.834174in}}%
\pgfpathcurveto{\pgfqpoint{3.694756in}{0.845224in}}{\pgfqpoint{3.690366in}{0.855823in}}{\pgfqpoint{3.682553in}{0.863637in}}%
\pgfpathcurveto{\pgfqpoint{3.674739in}{0.871451in}}{\pgfqpoint{3.664140in}{0.875841in}}{\pgfqpoint{3.653090in}{0.875841in}}%
\pgfpathcurveto{\pgfqpoint{3.642040in}{0.875841in}}{\pgfqpoint{3.631441in}{0.871451in}}{\pgfqpoint{3.623627in}{0.863637in}}%
\pgfpathcurveto{\pgfqpoint{3.615813in}{0.855823in}}{\pgfqpoint{3.611423in}{0.845224in}}{\pgfqpoint{3.611423in}{0.834174in}}%
\pgfpathcurveto{\pgfqpoint{3.611423in}{0.823124in}}{\pgfqpoint{3.615813in}{0.812525in}}{\pgfqpoint{3.623627in}{0.804711in}}%
\pgfpathcurveto{\pgfqpoint{3.631441in}{0.796898in}}{\pgfqpoint{3.642040in}{0.792507in}}{\pgfqpoint{3.653090in}{0.792507in}}%
\pgfpathclose%
\pgfusepath{stroke,fill}%
\end{pgfscope}%
\begin{pgfscope}%
\pgfpathrectangle{\pgfqpoint{0.481978in}{0.331635in}}{\pgfqpoint{4.960000in}{3.696000in}}%
\pgfusepath{clip}%
\pgfsetbuttcap%
\pgfsetroundjoin%
\definecolor{currentfill}{rgb}{0.631373,0.788235,0.956863}%
\pgfsetfillcolor{currentfill}%
\pgfsetlinewidth{0.481800pt}%
\definecolor{currentstroke}{rgb}{1.000000,1.000000,1.000000}%
\pgfsetstrokecolor{currentstroke}%
\pgfsetdash{}{0pt}%
\pgfpathmoveto{\pgfqpoint{3.631133in}{1.943361in}}%
\pgfpathcurveto{\pgfqpoint{3.642183in}{1.943361in}}{\pgfqpoint{3.652782in}{1.947751in}}{\pgfqpoint{3.660596in}{1.955565in}}%
\pgfpathcurveto{\pgfqpoint{3.668409in}{1.963379in}}{\pgfqpoint{3.672800in}{1.973978in}}{\pgfqpoint{3.672800in}{1.985028in}}%
\pgfpathcurveto{\pgfqpoint{3.672800in}{1.996078in}}{\pgfqpoint{3.668409in}{2.006677in}}{\pgfqpoint{3.660596in}{2.014491in}}%
\pgfpathcurveto{\pgfqpoint{3.652782in}{2.022304in}}{\pgfqpoint{3.642183in}{2.026695in}}{\pgfqpoint{3.631133in}{2.026695in}}%
\pgfpathcurveto{\pgfqpoint{3.620083in}{2.026695in}}{\pgfqpoint{3.609484in}{2.022304in}}{\pgfqpoint{3.601670in}{2.014491in}}%
\pgfpathcurveto{\pgfqpoint{3.593857in}{2.006677in}}{\pgfqpoint{3.589466in}{1.996078in}}{\pgfqpoint{3.589466in}{1.985028in}}%
\pgfpathcurveto{\pgfqpoint{3.589466in}{1.973978in}}{\pgfqpoint{3.593857in}{1.963379in}}{\pgfqpoint{3.601670in}{1.955565in}}%
\pgfpathcurveto{\pgfqpoint{3.609484in}{1.947751in}}{\pgfqpoint{3.620083in}{1.943361in}}{\pgfqpoint{3.631133in}{1.943361in}}%
\pgfpathclose%
\pgfusepath{stroke,fill}%
\end{pgfscope}%
\begin{pgfscope}%
\pgfpathrectangle{\pgfqpoint{0.481978in}{0.331635in}}{\pgfqpoint{4.960000in}{3.696000in}}%
\pgfusepath{clip}%
\pgfsetbuttcap%
\pgfsetroundjoin%
\definecolor{currentfill}{rgb}{0.631373,0.788235,0.956863}%
\pgfsetfillcolor{currentfill}%
\pgfsetlinewidth{0.481800pt}%
\definecolor{currentstroke}{rgb}{1.000000,1.000000,1.000000}%
\pgfsetstrokecolor{currentstroke}%
\pgfsetdash{}{0pt}%
\pgfpathmoveto{\pgfqpoint{3.297302in}{2.507360in}}%
\pgfpathcurveto{\pgfqpoint{3.308352in}{2.507360in}}{\pgfqpoint{3.318951in}{2.511750in}}{\pgfqpoint{3.326765in}{2.519564in}}%
\pgfpathcurveto{\pgfqpoint{3.334578in}{2.527377in}}{\pgfqpoint{3.338968in}{2.537976in}}{\pgfqpoint{3.338968in}{2.549026in}}%
\pgfpathcurveto{\pgfqpoint{3.338968in}{2.560076in}}{\pgfqpoint{3.334578in}{2.570675in}}{\pgfqpoint{3.326765in}{2.578489in}}%
\pgfpathcurveto{\pgfqpoint{3.318951in}{2.586303in}}{\pgfqpoint{3.308352in}{2.590693in}}{\pgfqpoint{3.297302in}{2.590693in}}%
\pgfpathcurveto{\pgfqpoint{3.286252in}{2.590693in}}{\pgfqpoint{3.275653in}{2.586303in}}{\pgfqpoint{3.267839in}{2.578489in}}%
\pgfpathcurveto{\pgfqpoint{3.260025in}{2.570675in}}{\pgfqpoint{3.255635in}{2.560076in}}{\pgfqpoint{3.255635in}{2.549026in}}%
\pgfpathcurveto{\pgfqpoint{3.255635in}{2.537976in}}{\pgfqpoint{3.260025in}{2.527377in}}{\pgfqpoint{3.267839in}{2.519564in}}%
\pgfpathcurveto{\pgfqpoint{3.275653in}{2.511750in}}{\pgfqpoint{3.286252in}{2.507360in}}{\pgfqpoint{3.297302in}{2.507360in}}%
\pgfpathclose%
\pgfusepath{stroke,fill}%
\end{pgfscope}%
\begin{pgfscope}%
\pgfpathrectangle{\pgfqpoint{0.481978in}{0.331635in}}{\pgfqpoint{4.960000in}{3.696000in}}%
\pgfusepath{clip}%
\pgfsetbuttcap%
\pgfsetroundjoin%
\definecolor{currentfill}{rgb}{0.631373,0.788235,0.956863}%
\pgfsetfillcolor{currentfill}%
\pgfsetlinewidth{0.481800pt}%
\definecolor{currentstroke}{rgb}{1.000000,1.000000,1.000000}%
\pgfsetstrokecolor{currentstroke}%
\pgfsetdash{}{0pt}%
\pgfpathmoveto{\pgfqpoint{4.495691in}{2.514056in}}%
\pgfpathcurveto{\pgfqpoint{4.506742in}{2.514056in}}{\pgfqpoint{4.517341in}{2.518446in}}{\pgfqpoint{4.525154in}{2.526260in}}%
\pgfpathcurveto{\pgfqpoint{4.532968in}{2.534074in}}{\pgfqpoint{4.537358in}{2.544673in}}{\pgfqpoint{4.537358in}{2.555723in}}%
\pgfpathcurveto{\pgfqpoint{4.537358in}{2.566773in}}{\pgfqpoint{4.532968in}{2.577372in}}{\pgfqpoint{4.525154in}{2.585186in}}%
\pgfpathcurveto{\pgfqpoint{4.517341in}{2.592999in}}{\pgfqpoint{4.506742in}{2.597389in}}{\pgfqpoint{4.495691in}{2.597389in}}%
\pgfpathcurveto{\pgfqpoint{4.484641in}{2.597389in}}{\pgfqpoint{4.474042in}{2.592999in}}{\pgfqpoint{4.466229in}{2.585186in}}%
\pgfpathcurveto{\pgfqpoint{4.458415in}{2.577372in}}{\pgfqpoint{4.454025in}{2.566773in}}{\pgfqpoint{4.454025in}{2.555723in}}%
\pgfpathcurveto{\pgfqpoint{4.454025in}{2.544673in}}{\pgfqpoint{4.458415in}{2.534074in}}{\pgfqpoint{4.466229in}{2.526260in}}%
\pgfpathcurveto{\pgfqpoint{4.474042in}{2.518446in}}{\pgfqpoint{4.484641in}{2.514056in}}{\pgfqpoint{4.495691in}{2.514056in}}%
\pgfpathclose%
\pgfusepath{stroke,fill}%
\end{pgfscope}%
\begin{pgfscope}%
\pgfpathrectangle{\pgfqpoint{0.481978in}{0.331635in}}{\pgfqpoint{4.960000in}{3.696000in}}%
\pgfusepath{clip}%
\pgfsetbuttcap%
\pgfsetroundjoin%
\definecolor{currentfill}{rgb}{0.631373,0.788235,0.956863}%
\pgfsetfillcolor{currentfill}%
\pgfsetlinewidth{0.481800pt}%
\definecolor{currentstroke}{rgb}{1.000000,1.000000,1.000000}%
\pgfsetstrokecolor{currentstroke}%
\pgfsetdash{}{0pt}%
\pgfpathmoveto{\pgfqpoint{2.851486in}{2.977629in}}%
\pgfpathcurveto{\pgfqpoint{2.862536in}{2.977629in}}{\pgfqpoint{2.873135in}{2.982019in}}{\pgfqpoint{2.880948in}{2.989833in}}%
\pgfpathcurveto{\pgfqpoint{2.888762in}{2.997647in}}{\pgfqpoint{2.893152in}{3.008246in}}{\pgfqpoint{2.893152in}{3.019296in}}%
\pgfpathcurveto{\pgfqpoint{2.893152in}{3.030346in}}{\pgfqpoint{2.888762in}{3.040945in}}{\pgfqpoint{2.880948in}{3.048759in}}%
\pgfpathcurveto{\pgfqpoint{2.873135in}{3.056572in}}{\pgfqpoint{2.862536in}{3.060962in}}{\pgfqpoint{2.851486in}{3.060962in}}%
\pgfpathcurveto{\pgfqpoint{2.840435in}{3.060962in}}{\pgfqpoint{2.829836in}{3.056572in}}{\pgfqpoint{2.822023in}{3.048759in}}%
\pgfpathcurveto{\pgfqpoint{2.814209in}{3.040945in}}{\pgfqpoint{2.809819in}{3.030346in}}{\pgfqpoint{2.809819in}{3.019296in}}%
\pgfpathcurveto{\pgfqpoint{2.809819in}{3.008246in}}{\pgfqpoint{2.814209in}{2.997647in}}{\pgfqpoint{2.822023in}{2.989833in}}%
\pgfpathcurveto{\pgfqpoint{2.829836in}{2.982019in}}{\pgfqpoint{2.840435in}{2.977629in}}{\pgfqpoint{2.851486in}{2.977629in}}%
\pgfpathclose%
\pgfusepath{stroke,fill}%
\end{pgfscope}%
\begin{pgfscope}%
\pgfpathrectangle{\pgfqpoint{0.481978in}{0.331635in}}{\pgfqpoint{4.960000in}{3.696000in}}%
\pgfusepath{clip}%
\pgfsetbuttcap%
\pgfsetroundjoin%
\definecolor{currentfill}{rgb}{0.631373,0.788235,0.956863}%
\pgfsetfillcolor{currentfill}%
\pgfsetlinewidth{0.481800pt}%
\definecolor{currentstroke}{rgb}{1.000000,1.000000,1.000000}%
\pgfsetstrokecolor{currentstroke}%
\pgfsetdash{}{0pt}%
\pgfpathmoveto{\pgfqpoint{3.014216in}{1.155345in}}%
\pgfpathcurveto{\pgfqpoint{3.025266in}{1.155345in}}{\pgfqpoint{3.035865in}{1.159735in}}{\pgfqpoint{3.043679in}{1.167549in}}%
\pgfpathcurveto{\pgfqpoint{3.051493in}{1.175363in}}{\pgfqpoint{3.055883in}{1.185962in}}{\pgfqpoint{3.055883in}{1.197012in}}%
\pgfpathcurveto{\pgfqpoint{3.055883in}{1.208062in}}{\pgfqpoint{3.051493in}{1.218661in}}{\pgfqpoint{3.043679in}{1.226475in}}%
\pgfpathcurveto{\pgfqpoint{3.035865in}{1.234288in}}{\pgfqpoint{3.025266in}{1.238679in}}{\pgfqpoint{3.014216in}{1.238679in}}%
\pgfpathcurveto{\pgfqpoint{3.003166in}{1.238679in}}{\pgfqpoint{2.992567in}{1.234288in}}{\pgfqpoint{2.984753in}{1.226475in}}%
\pgfpathcurveto{\pgfqpoint{2.976940in}{1.218661in}}{\pgfqpoint{2.972550in}{1.208062in}}{\pgfqpoint{2.972550in}{1.197012in}}%
\pgfpathcurveto{\pgfqpoint{2.972550in}{1.185962in}}{\pgfqpoint{2.976940in}{1.175363in}}{\pgfqpoint{2.984753in}{1.167549in}}%
\pgfpathcurveto{\pgfqpoint{2.992567in}{1.159735in}}{\pgfqpoint{3.003166in}{1.155345in}}{\pgfqpoint{3.014216in}{1.155345in}}%
\pgfpathclose%
\pgfusepath{stroke,fill}%
\end{pgfscope}%
\begin{pgfscope}%
\pgfpathrectangle{\pgfqpoint{0.481978in}{0.331635in}}{\pgfqpoint{4.960000in}{3.696000in}}%
\pgfusepath{clip}%
\pgfsetbuttcap%
\pgfsetroundjoin%
\definecolor{currentfill}{rgb}{0.631373,0.788235,0.956863}%
\pgfsetfillcolor{currentfill}%
\pgfsetlinewidth{0.481800pt}%
\definecolor{currentstroke}{rgb}{1.000000,1.000000,1.000000}%
\pgfsetstrokecolor{currentstroke}%
\pgfsetdash{}{0pt}%
\pgfpathmoveto{\pgfqpoint{2.085729in}{2.242098in}}%
\pgfpathcurveto{\pgfqpoint{2.096780in}{2.242098in}}{\pgfqpoint{2.107379in}{2.246489in}}{\pgfqpoint{2.115192in}{2.254302in}}%
\pgfpathcurveto{\pgfqpoint{2.123006in}{2.262116in}}{\pgfqpoint{2.127396in}{2.272715in}}{\pgfqpoint{2.127396in}{2.283765in}}%
\pgfpathcurveto{\pgfqpoint{2.127396in}{2.294815in}}{\pgfqpoint{2.123006in}{2.305414in}}{\pgfqpoint{2.115192in}{2.313228in}}%
\pgfpathcurveto{\pgfqpoint{2.107379in}{2.321042in}}{\pgfqpoint{2.096780in}{2.325432in}}{\pgfqpoint{2.085729in}{2.325432in}}%
\pgfpathcurveto{\pgfqpoint{2.074679in}{2.325432in}}{\pgfqpoint{2.064080in}{2.321042in}}{\pgfqpoint{2.056267in}{2.313228in}}%
\pgfpathcurveto{\pgfqpoint{2.048453in}{2.305414in}}{\pgfqpoint{2.044063in}{2.294815in}}{\pgfqpoint{2.044063in}{2.283765in}}%
\pgfpathcurveto{\pgfqpoint{2.044063in}{2.272715in}}{\pgfqpoint{2.048453in}{2.262116in}}{\pgfqpoint{2.056267in}{2.254302in}}%
\pgfpathcurveto{\pgfqpoint{2.064080in}{2.246489in}}{\pgfqpoint{2.074679in}{2.242098in}}{\pgfqpoint{2.085729in}{2.242098in}}%
\pgfpathclose%
\pgfusepath{stroke,fill}%
\end{pgfscope}%
\begin{pgfscope}%
\pgfpathrectangle{\pgfqpoint{0.481978in}{0.331635in}}{\pgfqpoint{4.960000in}{3.696000in}}%
\pgfusepath{clip}%
\pgfsetbuttcap%
\pgfsetroundjoin%
\definecolor{currentfill}{rgb}{0.631373,0.788235,0.956863}%
\pgfsetfillcolor{currentfill}%
\pgfsetlinewidth{0.481800pt}%
\definecolor{currentstroke}{rgb}{1.000000,1.000000,1.000000}%
\pgfsetstrokecolor{currentstroke}%
\pgfsetdash{}{0pt}%
\pgfpathmoveto{\pgfqpoint{4.776797in}{2.654053in}}%
\pgfpathcurveto{\pgfqpoint{4.787847in}{2.654053in}}{\pgfqpoint{4.798446in}{2.658444in}}{\pgfqpoint{4.806260in}{2.666257in}}%
\pgfpathcurveto{\pgfqpoint{4.814074in}{2.674071in}}{\pgfqpoint{4.818464in}{2.684670in}}{\pgfqpoint{4.818464in}{2.695720in}}%
\pgfpathcurveto{\pgfqpoint{4.818464in}{2.706770in}}{\pgfqpoint{4.814074in}{2.717369in}}{\pgfqpoint{4.806260in}{2.725183in}}%
\pgfpathcurveto{\pgfqpoint{4.798446in}{2.732997in}}{\pgfqpoint{4.787847in}{2.737387in}}{\pgfqpoint{4.776797in}{2.737387in}}%
\pgfpathcurveto{\pgfqpoint{4.765747in}{2.737387in}}{\pgfqpoint{4.755148in}{2.732997in}}{\pgfqpoint{4.747334in}{2.725183in}}%
\pgfpathcurveto{\pgfqpoint{4.739521in}{2.717369in}}{\pgfqpoint{4.735130in}{2.706770in}}{\pgfqpoint{4.735130in}{2.695720in}}%
\pgfpathcurveto{\pgfqpoint{4.735130in}{2.684670in}}{\pgfqpoint{4.739521in}{2.674071in}}{\pgfqpoint{4.747334in}{2.666257in}}%
\pgfpathcurveto{\pgfqpoint{4.755148in}{2.658444in}}{\pgfqpoint{4.765747in}{2.654053in}}{\pgfqpoint{4.776797in}{2.654053in}}%
\pgfpathclose%
\pgfusepath{stroke,fill}%
\end{pgfscope}%
\begin{pgfscope}%
\pgfpathrectangle{\pgfqpoint{0.481978in}{0.331635in}}{\pgfqpoint{4.960000in}{3.696000in}}%
\pgfusepath{clip}%
\pgfsetbuttcap%
\pgfsetroundjoin%
\definecolor{currentfill}{rgb}{0.631373,0.788235,0.956863}%
\pgfsetfillcolor{currentfill}%
\pgfsetlinewidth{0.481800pt}%
\definecolor{currentstroke}{rgb}{1.000000,1.000000,1.000000}%
\pgfsetstrokecolor{currentstroke}%
\pgfsetdash{}{0pt}%
\pgfpathmoveto{\pgfqpoint{4.252880in}{2.310642in}}%
\pgfpathcurveto{\pgfqpoint{4.263930in}{2.310642in}}{\pgfqpoint{4.274529in}{2.315032in}}{\pgfqpoint{4.282343in}{2.322846in}}%
\pgfpathcurveto{\pgfqpoint{4.290157in}{2.330659in}}{\pgfqpoint{4.294547in}{2.341258in}}{\pgfqpoint{4.294547in}{2.352308in}}%
\pgfpathcurveto{\pgfqpoint{4.294547in}{2.363358in}}{\pgfqpoint{4.290157in}{2.373957in}}{\pgfqpoint{4.282343in}{2.381771in}}%
\pgfpathcurveto{\pgfqpoint{4.274529in}{2.389585in}}{\pgfqpoint{4.263930in}{2.393975in}}{\pgfqpoint{4.252880in}{2.393975in}}%
\pgfpathcurveto{\pgfqpoint{4.241830in}{2.393975in}}{\pgfqpoint{4.231231in}{2.389585in}}{\pgfqpoint{4.223417in}{2.381771in}}%
\pgfpathcurveto{\pgfqpoint{4.215604in}{2.373957in}}{\pgfqpoint{4.211214in}{2.363358in}}{\pgfqpoint{4.211214in}{2.352308in}}%
\pgfpathcurveto{\pgfqpoint{4.211214in}{2.341258in}}{\pgfqpoint{4.215604in}{2.330659in}}{\pgfqpoint{4.223417in}{2.322846in}}%
\pgfpathcurveto{\pgfqpoint{4.231231in}{2.315032in}}{\pgfqpoint{4.241830in}{2.310642in}}{\pgfqpoint{4.252880in}{2.310642in}}%
\pgfpathclose%
\pgfusepath{stroke,fill}%
\end{pgfscope}%
\begin{pgfscope}%
\pgfpathrectangle{\pgfqpoint{0.481978in}{0.331635in}}{\pgfqpoint{4.960000in}{3.696000in}}%
\pgfusepath{clip}%
\pgfsetbuttcap%
\pgfsetroundjoin%
\definecolor{currentfill}{rgb}{0.631373,0.788235,0.956863}%
\pgfsetfillcolor{currentfill}%
\pgfsetlinewidth{0.481800pt}%
\definecolor{currentstroke}{rgb}{1.000000,1.000000,1.000000}%
\pgfsetstrokecolor{currentstroke}%
\pgfsetdash{}{0pt}%
\pgfpathmoveto{\pgfqpoint{3.598013in}{2.112726in}}%
\pgfpathcurveto{\pgfqpoint{3.609063in}{2.112726in}}{\pgfqpoint{3.619662in}{2.117116in}}{\pgfqpoint{3.627476in}{2.124930in}}%
\pgfpathcurveto{\pgfqpoint{3.635289in}{2.132743in}}{\pgfqpoint{3.639680in}{2.143343in}}{\pgfqpoint{3.639680in}{2.154393in}}%
\pgfpathcurveto{\pgfqpoint{3.639680in}{2.165443in}}{\pgfqpoint{3.635289in}{2.176042in}}{\pgfqpoint{3.627476in}{2.183855in}}%
\pgfpathcurveto{\pgfqpoint{3.619662in}{2.191669in}}{\pgfqpoint{3.609063in}{2.196059in}}{\pgfqpoint{3.598013in}{2.196059in}}%
\pgfpathcurveto{\pgfqpoint{3.586963in}{2.196059in}}{\pgfqpoint{3.576364in}{2.191669in}}{\pgfqpoint{3.568550in}{2.183855in}}%
\pgfpathcurveto{\pgfqpoint{3.560736in}{2.176042in}}{\pgfqpoint{3.556346in}{2.165443in}}{\pgfqpoint{3.556346in}{2.154393in}}%
\pgfpathcurveto{\pgfqpoint{3.556346in}{2.143343in}}{\pgfqpoint{3.560736in}{2.132743in}}{\pgfqpoint{3.568550in}{2.124930in}}%
\pgfpathcurveto{\pgfqpoint{3.576364in}{2.117116in}}{\pgfqpoint{3.586963in}{2.112726in}}{\pgfqpoint{3.598013in}{2.112726in}}%
\pgfpathclose%
\pgfusepath{stroke,fill}%
\end{pgfscope}%
\begin{pgfscope}%
\pgfpathrectangle{\pgfqpoint{0.481978in}{0.331635in}}{\pgfqpoint{4.960000in}{3.696000in}}%
\pgfusepath{clip}%
\pgfsetbuttcap%
\pgfsetroundjoin%
\definecolor{currentfill}{rgb}{0.631373,0.788235,0.956863}%
\pgfsetfillcolor{currentfill}%
\pgfsetlinewidth{0.481800pt}%
\definecolor{currentstroke}{rgb}{1.000000,1.000000,1.000000}%
\pgfsetstrokecolor{currentstroke}%
\pgfsetdash{}{0pt}%
\pgfpathmoveto{\pgfqpoint{2.636586in}{1.134382in}}%
\pgfpathcurveto{\pgfqpoint{2.647637in}{1.134382in}}{\pgfqpoint{2.658236in}{1.138772in}}{\pgfqpoint{2.666049in}{1.146586in}}%
\pgfpathcurveto{\pgfqpoint{2.673863in}{1.154399in}}{\pgfqpoint{2.678253in}{1.164998in}}{\pgfqpoint{2.678253in}{1.176048in}}%
\pgfpathcurveto{\pgfqpoint{2.678253in}{1.187099in}}{\pgfqpoint{2.673863in}{1.197698in}}{\pgfqpoint{2.666049in}{1.205511in}}%
\pgfpathcurveto{\pgfqpoint{2.658236in}{1.213325in}}{\pgfqpoint{2.647637in}{1.217715in}}{\pgfqpoint{2.636586in}{1.217715in}}%
\pgfpathcurveto{\pgfqpoint{2.625536in}{1.217715in}}{\pgfqpoint{2.614937in}{1.213325in}}{\pgfqpoint{2.607124in}{1.205511in}}%
\pgfpathcurveto{\pgfqpoint{2.599310in}{1.197698in}}{\pgfqpoint{2.594920in}{1.187099in}}{\pgfqpoint{2.594920in}{1.176048in}}%
\pgfpathcurveto{\pgfqpoint{2.594920in}{1.164998in}}{\pgfqpoint{2.599310in}{1.154399in}}{\pgfqpoint{2.607124in}{1.146586in}}%
\pgfpathcurveto{\pgfqpoint{2.614937in}{1.138772in}}{\pgfqpoint{2.625536in}{1.134382in}}{\pgfqpoint{2.636586in}{1.134382in}}%
\pgfpathclose%
\pgfusepath{stroke,fill}%
\end{pgfscope}%
\begin{pgfscope}%
\pgfpathrectangle{\pgfqpoint{0.481978in}{0.331635in}}{\pgfqpoint{4.960000in}{3.696000in}}%
\pgfusepath{clip}%
\pgfsetbuttcap%
\pgfsetroundjoin%
\definecolor{currentfill}{rgb}{0.631373,0.788235,0.956863}%
\pgfsetfillcolor{currentfill}%
\pgfsetlinewidth{0.481800pt}%
\definecolor{currentstroke}{rgb}{1.000000,1.000000,1.000000}%
\pgfsetstrokecolor{currentstroke}%
\pgfsetdash{}{0pt}%
\pgfpathmoveto{\pgfqpoint{3.161179in}{1.257620in}}%
\pgfpathcurveto{\pgfqpoint{3.172229in}{1.257620in}}{\pgfqpoint{3.182828in}{1.262010in}}{\pgfqpoint{3.190642in}{1.269823in}}%
\pgfpathcurveto{\pgfqpoint{3.198455in}{1.277637in}}{\pgfqpoint{3.202845in}{1.288236in}}{\pgfqpoint{3.202845in}{1.299286in}}%
\pgfpathcurveto{\pgfqpoint{3.202845in}{1.310336in}}{\pgfqpoint{3.198455in}{1.320935in}}{\pgfqpoint{3.190642in}{1.328749in}}%
\pgfpathcurveto{\pgfqpoint{3.182828in}{1.336563in}}{\pgfqpoint{3.172229in}{1.340953in}}{\pgfqpoint{3.161179in}{1.340953in}}%
\pgfpathcurveto{\pgfqpoint{3.150129in}{1.340953in}}{\pgfqpoint{3.139530in}{1.336563in}}{\pgfqpoint{3.131716in}{1.328749in}}%
\pgfpathcurveto{\pgfqpoint{3.123902in}{1.320935in}}{\pgfqpoint{3.119512in}{1.310336in}}{\pgfqpoint{3.119512in}{1.299286in}}%
\pgfpathcurveto{\pgfqpoint{3.119512in}{1.288236in}}{\pgfqpoint{3.123902in}{1.277637in}}{\pgfqpoint{3.131716in}{1.269823in}}%
\pgfpathcurveto{\pgfqpoint{3.139530in}{1.262010in}}{\pgfqpoint{3.150129in}{1.257620in}}{\pgfqpoint{3.161179in}{1.257620in}}%
\pgfpathclose%
\pgfusepath{stroke,fill}%
\end{pgfscope}%
\begin{pgfscope}%
\pgfpathrectangle{\pgfqpoint{0.481978in}{0.331635in}}{\pgfqpoint{4.960000in}{3.696000in}}%
\pgfusepath{clip}%
\pgfsetbuttcap%
\pgfsetroundjoin%
\definecolor{currentfill}{rgb}{0.631373,0.788235,0.956863}%
\pgfsetfillcolor{currentfill}%
\pgfsetlinewidth{0.481800pt}%
\definecolor{currentstroke}{rgb}{1.000000,1.000000,1.000000}%
\pgfsetstrokecolor{currentstroke}%
\pgfsetdash{}{0pt}%
\pgfpathmoveto{\pgfqpoint{3.463201in}{2.381625in}}%
\pgfpathcurveto{\pgfqpoint{3.474251in}{2.381625in}}{\pgfqpoint{3.484850in}{2.386015in}}{\pgfqpoint{3.492664in}{2.393829in}}%
\pgfpathcurveto{\pgfqpoint{3.500477in}{2.401643in}}{\pgfqpoint{3.504868in}{2.412242in}}{\pgfqpoint{3.504868in}{2.423292in}}%
\pgfpathcurveto{\pgfqpoint{3.504868in}{2.434342in}}{\pgfqpoint{3.500477in}{2.444941in}}{\pgfqpoint{3.492664in}{2.452755in}}%
\pgfpathcurveto{\pgfqpoint{3.484850in}{2.460568in}}{\pgfqpoint{3.474251in}{2.464959in}}{\pgfqpoint{3.463201in}{2.464959in}}%
\pgfpathcurveto{\pgfqpoint{3.452151in}{2.464959in}}{\pgfqpoint{3.441552in}{2.460568in}}{\pgfqpoint{3.433738in}{2.452755in}}%
\pgfpathcurveto{\pgfqpoint{3.425925in}{2.444941in}}{\pgfqpoint{3.421534in}{2.434342in}}{\pgfqpoint{3.421534in}{2.423292in}}%
\pgfpathcurveto{\pgfqpoint{3.421534in}{2.412242in}}{\pgfqpoint{3.425925in}{2.401643in}}{\pgfqpoint{3.433738in}{2.393829in}}%
\pgfpathcurveto{\pgfqpoint{3.441552in}{2.386015in}}{\pgfqpoint{3.452151in}{2.381625in}}{\pgfqpoint{3.463201in}{2.381625in}}%
\pgfpathclose%
\pgfusepath{stroke,fill}%
\end{pgfscope}%
\begin{pgfscope}%
\pgfpathrectangle{\pgfqpoint{0.481978in}{0.331635in}}{\pgfqpoint{4.960000in}{3.696000in}}%
\pgfusepath{clip}%
\pgfsetbuttcap%
\pgfsetroundjoin%
\definecolor{currentfill}{rgb}{0.631373,0.788235,0.956863}%
\pgfsetfillcolor{currentfill}%
\pgfsetlinewidth{0.481800pt}%
\definecolor{currentstroke}{rgb}{1.000000,1.000000,1.000000}%
\pgfsetstrokecolor{currentstroke}%
\pgfsetdash{}{0pt}%
\pgfpathmoveto{\pgfqpoint{3.775497in}{2.720975in}}%
\pgfpathcurveto{\pgfqpoint{3.786547in}{2.720975in}}{\pgfqpoint{3.797146in}{2.725366in}}{\pgfqpoint{3.804960in}{2.733179in}}%
\pgfpathcurveto{\pgfqpoint{3.812774in}{2.740993in}}{\pgfqpoint{3.817164in}{2.751592in}}{\pgfqpoint{3.817164in}{2.762642in}}%
\pgfpathcurveto{\pgfqpoint{3.817164in}{2.773692in}}{\pgfqpoint{3.812774in}{2.784291in}}{\pgfqpoint{3.804960in}{2.792105in}}%
\pgfpathcurveto{\pgfqpoint{3.797146in}{2.799919in}}{\pgfqpoint{3.786547in}{2.804309in}}{\pgfqpoint{3.775497in}{2.804309in}}%
\pgfpathcurveto{\pgfqpoint{3.764447in}{2.804309in}}{\pgfqpoint{3.753848in}{2.799919in}}{\pgfqpoint{3.746034in}{2.792105in}}%
\pgfpathcurveto{\pgfqpoint{3.738221in}{2.784291in}}{\pgfqpoint{3.733830in}{2.773692in}}{\pgfqpoint{3.733830in}{2.762642in}}%
\pgfpathcurveto{\pgfqpoint{3.733830in}{2.751592in}}{\pgfqpoint{3.738221in}{2.740993in}}{\pgfqpoint{3.746034in}{2.733179in}}%
\pgfpathcurveto{\pgfqpoint{3.753848in}{2.725366in}}{\pgfqpoint{3.764447in}{2.720975in}}{\pgfqpoint{3.775497in}{2.720975in}}%
\pgfpathclose%
\pgfusepath{stroke,fill}%
\end{pgfscope}%
\begin{pgfscope}%
\pgfpathrectangle{\pgfqpoint{0.481978in}{0.331635in}}{\pgfqpoint{4.960000in}{3.696000in}}%
\pgfusepath{clip}%
\pgfsetbuttcap%
\pgfsetroundjoin%
\definecolor{currentfill}{rgb}{0.631373,0.788235,0.956863}%
\pgfsetfillcolor{currentfill}%
\pgfsetlinewidth{0.481800pt}%
\definecolor{currentstroke}{rgb}{1.000000,1.000000,1.000000}%
\pgfsetstrokecolor{currentstroke}%
\pgfsetdash{}{0pt}%
\pgfpathmoveto{\pgfqpoint{4.634580in}{2.789198in}}%
\pgfpathcurveto{\pgfqpoint{4.645631in}{2.789198in}}{\pgfqpoint{4.656230in}{2.793588in}}{\pgfqpoint{4.664043in}{2.801401in}}%
\pgfpathcurveto{\pgfqpoint{4.671857in}{2.809215in}}{\pgfqpoint{4.676247in}{2.819814in}}{\pgfqpoint{4.676247in}{2.830864in}}%
\pgfpathcurveto{\pgfqpoint{4.676247in}{2.841914in}}{\pgfqpoint{4.671857in}{2.852513in}}{\pgfqpoint{4.664043in}{2.860327in}}%
\pgfpathcurveto{\pgfqpoint{4.656230in}{2.868141in}}{\pgfqpoint{4.645631in}{2.872531in}}{\pgfqpoint{4.634580in}{2.872531in}}%
\pgfpathcurveto{\pgfqpoint{4.623530in}{2.872531in}}{\pgfqpoint{4.612931in}{2.868141in}}{\pgfqpoint{4.605118in}{2.860327in}}%
\pgfpathcurveto{\pgfqpoint{4.597304in}{2.852513in}}{\pgfqpoint{4.592914in}{2.841914in}}{\pgfqpoint{4.592914in}{2.830864in}}%
\pgfpathcurveto{\pgfqpoint{4.592914in}{2.819814in}}{\pgfqpoint{4.597304in}{2.809215in}}{\pgfqpoint{4.605118in}{2.801401in}}%
\pgfpathcurveto{\pgfqpoint{4.612931in}{2.793588in}}{\pgfqpoint{4.623530in}{2.789198in}}{\pgfqpoint{4.634580in}{2.789198in}}%
\pgfpathclose%
\pgfusepath{stroke,fill}%
\end{pgfscope}%
\begin{pgfscope}%
\pgfpathrectangle{\pgfqpoint{0.481978in}{0.331635in}}{\pgfqpoint{4.960000in}{3.696000in}}%
\pgfusepath{clip}%
\pgfsetbuttcap%
\pgfsetroundjoin%
\definecolor{currentfill}{rgb}{0.631373,0.788235,0.956863}%
\pgfsetfillcolor{currentfill}%
\pgfsetlinewidth{0.481800pt}%
\definecolor{currentstroke}{rgb}{1.000000,1.000000,1.000000}%
\pgfsetstrokecolor{currentstroke}%
\pgfsetdash{}{0pt}%
\pgfpathmoveto{\pgfqpoint{4.538728in}{2.308986in}}%
\pgfpathcurveto{\pgfqpoint{4.549778in}{2.308986in}}{\pgfqpoint{4.560377in}{2.313376in}}{\pgfqpoint{4.568191in}{2.321190in}}%
\pgfpathcurveto{\pgfqpoint{4.576004in}{2.329004in}}{\pgfqpoint{4.580394in}{2.339603in}}{\pgfqpoint{4.580394in}{2.350653in}}%
\pgfpathcurveto{\pgfqpoint{4.580394in}{2.361703in}}{\pgfqpoint{4.576004in}{2.372302in}}{\pgfqpoint{4.568191in}{2.380116in}}%
\pgfpathcurveto{\pgfqpoint{4.560377in}{2.387929in}}{\pgfqpoint{4.549778in}{2.392319in}}{\pgfqpoint{4.538728in}{2.392319in}}%
\pgfpathcurveto{\pgfqpoint{4.527678in}{2.392319in}}{\pgfqpoint{4.517079in}{2.387929in}}{\pgfqpoint{4.509265in}{2.380116in}}%
\pgfpathcurveto{\pgfqpoint{4.501451in}{2.372302in}}{\pgfqpoint{4.497061in}{2.361703in}}{\pgfqpoint{4.497061in}{2.350653in}}%
\pgfpathcurveto{\pgfqpoint{4.497061in}{2.339603in}}{\pgfqpoint{4.501451in}{2.329004in}}{\pgfqpoint{4.509265in}{2.321190in}}%
\pgfpathcurveto{\pgfqpoint{4.517079in}{2.313376in}}{\pgfqpoint{4.527678in}{2.308986in}}{\pgfqpoint{4.538728in}{2.308986in}}%
\pgfpathclose%
\pgfusepath{stroke,fill}%
\end{pgfscope}%
\begin{pgfscope}%
\pgfpathrectangle{\pgfqpoint{0.481978in}{0.331635in}}{\pgfqpoint{4.960000in}{3.696000in}}%
\pgfusepath{clip}%
\pgfsetbuttcap%
\pgfsetroundjoin%
\definecolor{currentfill}{rgb}{0.631373,0.788235,0.956863}%
\pgfsetfillcolor{currentfill}%
\pgfsetlinewidth{0.481800pt}%
\definecolor{currentstroke}{rgb}{1.000000,1.000000,1.000000}%
\pgfsetstrokecolor{currentstroke}%
\pgfsetdash{}{0pt}%
\pgfpathmoveto{\pgfqpoint{4.604558in}{2.461894in}}%
\pgfpathcurveto{\pgfqpoint{4.615608in}{2.461894in}}{\pgfqpoint{4.626207in}{2.466284in}}{\pgfqpoint{4.634020in}{2.474098in}}%
\pgfpathcurveto{\pgfqpoint{4.641834in}{2.481912in}}{\pgfqpoint{4.646224in}{2.492511in}}{\pgfqpoint{4.646224in}{2.503561in}}%
\pgfpathcurveto{\pgfqpoint{4.646224in}{2.514611in}}{\pgfqpoint{4.641834in}{2.525210in}}{\pgfqpoint{4.634020in}{2.533023in}}%
\pgfpathcurveto{\pgfqpoint{4.626207in}{2.540837in}}{\pgfqpoint{4.615608in}{2.545227in}}{\pgfqpoint{4.604558in}{2.545227in}}%
\pgfpathcurveto{\pgfqpoint{4.593507in}{2.545227in}}{\pgfqpoint{4.582908in}{2.540837in}}{\pgfqpoint{4.575095in}{2.533023in}}%
\pgfpathcurveto{\pgfqpoint{4.567281in}{2.525210in}}{\pgfqpoint{4.562891in}{2.514611in}}{\pgfqpoint{4.562891in}{2.503561in}}%
\pgfpathcurveto{\pgfqpoint{4.562891in}{2.492511in}}{\pgfqpoint{4.567281in}{2.481912in}}{\pgfqpoint{4.575095in}{2.474098in}}%
\pgfpathcurveto{\pgfqpoint{4.582908in}{2.466284in}}{\pgfqpoint{4.593507in}{2.461894in}}{\pgfqpoint{4.604558in}{2.461894in}}%
\pgfpathclose%
\pgfusepath{stroke,fill}%
\end{pgfscope}%
\begin{pgfscope}%
\pgfpathrectangle{\pgfqpoint{0.481978in}{0.331635in}}{\pgfqpoint{4.960000in}{3.696000in}}%
\pgfusepath{clip}%
\pgfsetbuttcap%
\pgfsetroundjoin%
\definecolor{currentfill}{rgb}{0.631373,0.788235,0.956863}%
\pgfsetfillcolor{currentfill}%
\pgfsetlinewidth{0.481800pt}%
\definecolor{currentstroke}{rgb}{1.000000,1.000000,1.000000}%
\pgfsetstrokecolor{currentstroke}%
\pgfsetdash{}{0pt}%
\pgfpathmoveto{\pgfqpoint{3.551524in}{2.205325in}}%
\pgfpathcurveto{\pgfqpoint{3.562575in}{2.205325in}}{\pgfqpoint{3.573174in}{2.209715in}}{\pgfqpoint{3.580987in}{2.217529in}}%
\pgfpathcurveto{\pgfqpoint{3.588801in}{2.225343in}}{\pgfqpoint{3.593191in}{2.235942in}}{\pgfqpoint{3.593191in}{2.246992in}}%
\pgfpathcurveto{\pgfqpoint{3.593191in}{2.258042in}}{\pgfqpoint{3.588801in}{2.268641in}}{\pgfqpoint{3.580987in}{2.276455in}}%
\pgfpathcurveto{\pgfqpoint{3.573174in}{2.284268in}}{\pgfqpoint{3.562575in}{2.288658in}}{\pgfqpoint{3.551524in}{2.288658in}}%
\pgfpathcurveto{\pgfqpoint{3.540474in}{2.288658in}}{\pgfqpoint{3.529875in}{2.284268in}}{\pgfqpoint{3.522062in}{2.276455in}}%
\pgfpathcurveto{\pgfqpoint{3.514248in}{2.268641in}}{\pgfqpoint{3.509858in}{2.258042in}}{\pgfqpoint{3.509858in}{2.246992in}}%
\pgfpathcurveto{\pgfqpoint{3.509858in}{2.235942in}}{\pgfqpoint{3.514248in}{2.225343in}}{\pgfqpoint{3.522062in}{2.217529in}}%
\pgfpathcurveto{\pgfqpoint{3.529875in}{2.209715in}}{\pgfqpoint{3.540474in}{2.205325in}}{\pgfqpoint{3.551524in}{2.205325in}}%
\pgfpathclose%
\pgfusepath{stroke,fill}%
\end{pgfscope}%
\begin{pgfscope}%
\pgfpathrectangle{\pgfqpoint{0.481978in}{0.331635in}}{\pgfqpoint{4.960000in}{3.696000in}}%
\pgfusepath{clip}%
\pgfsetbuttcap%
\pgfsetroundjoin%
\definecolor{currentfill}{rgb}{0.631373,0.788235,0.956863}%
\pgfsetfillcolor{currentfill}%
\pgfsetlinewidth{0.481800pt}%
\definecolor{currentstroke}{rgb}{1.000000,1.000000,1.000000}%
\pgfsetstrokecolor{currentstroke}%
\pgfsetdash{}{0pt}%
\pgfpathmoveto{\pgfqpoint{3.932814in}{2.457392in}}%
\pgfpathcurveto{\pgfqpoint{3.943864in}{2.457392in}}{\pgfqpoint{3.954463in}{2.461783in}}{\pgfqpoint{3.962276in}{2.469596in}}%
\pgfpathcurveto{\pgfqpoint{3.970090in}{2.477410in}}{\pgfqpoint{3.974480in}{2.488009in}}{\pgfqpoint{3.974480in}{2.499059in}}%
\pgfpathcurveto{\pgfqpoint{3.974480in}{2.510109in}}{\pgfqpoint{3.970090in}{2.520708in}}{\pgfqpoint{3.962276in}{2.528522in}}%
\pgfpathcurveto{\pgfqpoint{3.954463in}{2.536335in}}{\pgfqpoint{3.943864in}{2.540726in}}{\pgfqpoint{3.932814in}{2.540726in}}%
\pgfpathcurveto{\pgfqpoint{3.921764in}{2.540726in}}{\pgfqpoint{3.911165in}{2.536335in}}{\pgfqpoint{3.903351in}{2.528522in}}%
\pgfpathcurveto{\pgfqpoint{3.895537in}{2.520708in}}{\pgfqpoint{3.891147in}{2.510109in}}{\pgfqpoint{3.891147in}{2.499059in}}%
\pgfpathcurveto{\pgfqpoint{3.891147in}{2.488009in}}{\pgfqpoint{3.895537in}{2.477410in}}{\pgfqpoint{3.903351in}{2.469596in}}%
\pgfpathcurveto{\pgfqpoint{3.911165in}{2.461783in}}{\pgfqpoint{3.921764in}{2.457392in}}{\pgfqpoint{3.932814in}{2.457392in}}%
\pgfpathclose%
\pgfusepath{stroke,fill}%
\end{pgfscope}%
\begin{pgfscope}%
\pgfpathrectangle{\pgfqpoint{0.481978in}{0.331635in}}{\pgfqpoint{4.960000in}{3.696000in}}%
\pgfusepath{clip}%
\pgfsetbuttcap%
\pgfsetroundjoin%
\definecolor{currentfill}{rgb}{0.631373,0.788235,0.956863}%
\pgfsetfillcolor{currentfill}%
\pgfsetlinewidth{0.481800pt}%
\definecolor{currentstroke}{rgb}{1.000000,1.000000,1.000000}%
\pgfsetstrokecolor{currentstroke}%
\pgfsetdash{}{0pt}%
\pgfpathmoveto{\pgfqpoint{0.707432in}{1.951129in}}%
\pgfpathcurveto{\pgfqpoint{0.718483in}{1.951129in}}{\pgfqpoint{0.729082in}{1.955519in}}{\pgfqpoint{0.736895in}{1.963333in}}%
\pgfpathcurveto{\pgfqpoint{0.744709in}{1.971146in}}{\pgfqpoint{0.749099in}{1.981745in}}{\pgfqpoint{0.749099in}{1.992795in}}%
\pgfpathcurveto{\pgfqpoint{0.749099in}{2.003845in}}{\pgfqpoint{0.744709in}{2.014444in}}{\pgfqpoint{0.736895in}{2.022258in}}%
\pgfpathcurveto{\pgfqpoint{0.729082in}{2.030072in}}{\pgfqpoint{0.718483in}{2.034462in}}{\pgfqpoint{0.707432in}{2.034462in}}%
\pgfpathcurveto{\pgfqpoint{0.696382in}{2.034462in}}{\pgfqpoint{0.685783in}{2.030072in}}{\pgfqpoint{0.677970in}{2.022258in}}%
\pgfpathcurveto{\pgfqpoint{0.670156in}{2.014444in}}{\pgfqpoint{0.665766in}{2.003845in}}{\pgfqpoint{0.665766in}{1.992795in}}%
\pgfpathcurveto{\pgfqpoint{0.665766in}{1.981745in}}{\pgfqpoint{0.670156in}{1.971146in}}{\pgfqpoint{0.677970in}{1.963333in}}%
\pgfpathcurveto{\pgfqpoint{0.685783in}{1.955519in}}{\pgfqpoint{0.696382in}{1.951129in}}{\pgfqpoint{0.707432in}{1.951129in}}%
\pgfpathclose%
\pgfusepath{stroke,fill}%
\end{pgfscope}%
\begin{pgfscope}%
\pgfpathrectangle{\pgfqpoint{0.481978in}{0.331635in}}{\pgfqpoint{4.960000in}{3.696000in}}%
\pgfusepath{clip}%
\pgfsetbuttcap%
\pgfsetroundjoin%
\definecolor{currentfill}{rgb}{0.631373,0.788235,0.956863}%
\pgfsetfillcolor{currentfill}%
\pgfsetlinewidth{0.481800pt}%
\definecolor{currentstroke}{rgb}{1.000000,1.000000,1.000000}%
\pgfsetstrokecolor{currentstroke}%
\pgfsetdash{}{0pt}%
\pgfpathmoveto{\pgfqpoint{2.217985in}{1.005542in}}%
\pgfpathcurveto{\pgfqpoint{2.229035in}{1.005542in}}{\pgfqpoint{2.239634in}{1.009933in}}{\pgfqpoint{2.247448in}{1.017746in}}%
\pgfpathcurveto{\pgfqpoint{2.255262in}{1.025560in}}{\pgfqpoint{2.259652in}{1.036159in}}{\pgfqpoint{2.259652in}{1.047209in}}%
\pgfpathcurveto{\pgfqpoint{2.259652in}{1.058259in}}{\pgfqpoint{2.255262in}{1.068858in}}{\pgfqpoint{2.247448in}{1.076672in}}%
\pgfpathcurveto{\pgfqpoint{2.239634in}{1.084486in}}{\pgfqpoint{2.229035in}{1.088876in}}{\pgfqpoint{2.217985in}{1.088876in}}%
\pgfpathcurveto{\pgfqpoint{2.206935in}{1.088876in}}{\pgfqpoint{2.196336in}{1.084486in}}{\pgfqpoint{2.188523in}{1.076672in}}%
\pgfpathcurveto{\pgfqpoint{2.180709in}{1.068858in}}{\pgfqpoint{2.176319in}{1.058259in}}{\pgfqpoint{2.176319in}{1.047209in}}%
\pgfpathcurveto{\pgfqpoint{2.176319in}{1.036159in}}{\pgfqpoint{2.180709in}{1.025560in}}{\pgfqpoint{2.188523in}{1.017746in}}%
\pgfpathcurveto{\pgfqpoint{2.196336in}{1.009933in}}{\pgfqpoint{2.206935in}{1.005542in}}{\pgfqpoint{2.217985in}{1.005542in}}%
\pgfpathclose%
\pgfusepath{stroke,fill}%
\end{pgfscope}%
\begin{pgfscope}%
\pgfpathrectangle{\pgfqpoint{0.481978in}{0.331635in}}{\pgfqpoint{4.960000in}{3.696000in}}%
\pgfusepath{clip}%
\pgfsetbuttcap%
\pgfsetroundjoin%
\definecolor{currentfill}{rgb}{0.631373,0.788235,0.956863}%
\pgfsetfillcolor{currentfill}%
\pgfsetlinewidth{0.481800pt}%
\definecolor{currentstroke}{rgb}{1.000000,1.000000,1.000000}%
\pgfsetstrokecolor{currentstroke}%
\pgfsetdash{}{0pt}%
\pgfpathmoveto{\pgfqpoint{2.428376in}{1.586034in}}%
\pgfpathcurveto{\pgfqpoint{2.439427in}{1.586034in}}{\pgfqpoint{2.450026in}{1.590425in}}{\pgfqpoint{2.457839in}{1.598238in}}%
\pgfpathcurveto{\pgfqpoint{2.465653in}{1.606052in}}{\pgfqpoint{2.470043in}{1.616651in}}{\pgfqpoint{2.470043in}{1.627701in}}%
\pgfpathcurveto{\pgfqpoint{2.470043in}{1.638751in}}{\pgfqpoint{2.465653in}{1.649350in}}{\pgfqpoint{2.457839in}{1.657164in}}%
\pgfpathcurveto{\pgfqpoint{2.450026in}{1.664977in}}{\pgfqpoint{2.439427in}{1.669368in}}{\pgfqpoint{2.428376in}{1.669368in}}%
\pgfpathcurveto{\pgfqpoint{2.417326in}{1.669368in}}{\pgfqpoint{2.406727in}{1.664977in}}{\pgfqpoint{2.398914in}{1.657164in}}%
\pgfpathcurveto{\pgfqpoint{2.391100in}{1.649350in}}{\pgfqpoint{2.386710in}{1.638751in}}{\pgfqpoint{2.386710in}{1.627701in}}%
\pgfpathcurveto{\pgfqpoint{2.386710in}{1.616651in}}{\pgfqpoint{2.391100in}{1.606052in}}{\pgfqpoint{2.398914in}{1.598238in}}%
\pgfpathcurveto{\pgfqpoint{2.406727in}{1.590425in}}{\pgfqpoint{2.417326in}{1.586034in}}{\pgfqpoint{2.428376in}{1.586034in}}%
\pgfpathclose%
\pgfusepath{stroke,fill}%
\end{pgfscope}%
\begin{pgfscope}%
\pgfpathrectangle{\pgfqpoint{0.481978in}{0.331635in}}{\pgfqpoint{4.960000in}{3.696000in}}%
\pgfusepath{clip}%
\pgfsetbuttcap%
\pgfsetroundjoin%
\definecolor{currentfill}{rgb}{0.631373,0.788235,0.956863}%
\pgfsetfillcolor{currentfill}%
\pgfsetlinewidth{0.481800pt}%
\definecolor{currentstroke}{rgb}{1.000000,1.000000,1.000000}%
\pgfsetstrokecolor{currentstroke}%
\pgfsetdash{}{0pt}%
\pgfpathmoveto{\pgfqpoint{4.595189in}{3.165924in}}%
\pgfpathcurveto{\pgfqpoint{4.606239in}{3.165924in}}{\pgfqpoint{4.616838in}{3.170315in}}{\pgfqpoint{4.624652in}{3.178128in}}%
\pgfpathcurveto{\pgfqpoint{4.632466in}{3.185942in}}{\pgfqpoint{4.636856in}{3.196541in}}{\pgfqpoint{4.636856in}{3.207591in}}%
\pgfpathcurveto{\pgfqpoint{4.636856in}{3.218641in}}{\pgfqpoint{4.632466in}{3.229240in}}{\pgfqpoint{4.624652in}{3.237054in}}%
\pgfpathcurveto{\pgfqpoint{4.616838in}{3.244867in}}{\pgfqpoint{4.606239in}{3.249258in}}{\pgfqpoint{4.595189in}{3.249258in}}%
\pgfpathcurveto{\pgfqpoint{4.584139in}{3.249258in}}{\pgfqpoint{4.573540in}{3.244867in}}{\pgfqpoint{4.565726in}{3.237054in}}%
\pgfpathcurveto{\pgfqpoint{4.557913in}{3.229240in}}{\pgfqpoint{4.553522in}{3.218641in}}{\pgfqpoint{4.553522in}{3.207591in}}%
\pgfpathcurveto{\pgfqpoint{4.553522in}{3.196541in}}{\pgfqpoint{4.557913in}{3.185942in}}{\pgfqpoint{4.565726in}{3.178128in}}%
\pgfpathcurveto{\pgfqpoint{4.573540in}{3.170315in}}{\pgfqpoint{4.584139in}{3.165924in}}{\pgfqpoint{4.595189in}{3.165924in}}%
\pgfpathclose%
\pgfusepath{stroke,fill}%
\end{pgfscope}%
\begin{pgfscope}%
\pgfpathrectangle{\pgfqpoint{0.481978in}{0.331635in}}{\pgfqpoint{4.960000in}{3.696000in}}%
\pgfusepath{clip}%
\pgfsetbuttcap%
\pgfsetroundjoin%
\definecolor{currentfill}{rgb}{0.631373,0.788235,0.956863}%
\pgfsetfillcolor{currentfill}%
\pgfsetlinewidth{0.481800pt}%
\definecolor{currentstroke}{rgb}{1.000000,1.000000,1.000000}%
\pgfsetstrokecolor{currentstroke}%
\pgfsetdash{}{0pt}%
\pgfpathmoveto{\pgfqpoint{4.555071in}{2.034074in}}%
\pgfpathcurveto{\pgfqpoint{4.566121in}{2.034074in}}{\pgfqpoint{4.576720in}{2.038464in}}{\pgfqpoint{4.584534in}{2.046277in}}%
\pgfpathcurveto{\pgfqpoint{4.592347in}{2.054091in}}{\pgfqpoint{4.596738in}{2.064690in}}{\pgfqpoint{4.596738in}{2.075740in}}%
\pgfpathcurveto{\pgfqpoint{4.596738in}{2.086790in}}{\pgfqpoint{4.592347in}{2.097389in}}{\pgfqpoint{4.584534in}{2.105203in}}%
\pgfpathcurveto{\pgfqpoint{4.576720in}{2.113017in}}{\pgfqpoint{4.566121in}{2.117407in}}{\pgfqpoint{4.555071in}{2.117407in}}%
\pgfpathcurveto{\pgfqpoint{4.544021in}{2.117407in}}{\pgfqpoint{4.533422in}{2.113017in}}{\pgfqpoint{4.525608in}{2.105203in}}%
\pgfpathcurveto{\pgfqpoint{4.517794in}{2.097389in}}{\pgfqpoint{4.513404in}{2.086790in}}{\pgfqpoint{4.513404in}{2.075740in}}%
\pgfpathcurveto{\pgfqpoint{4.513404in}{2.064690in}}{\pgfqpoint{4.517794in}{2.054091in}}{\pgfqpoint{4.525608in}{2.046277in}}%
\pgfpathcurveto{\pgfqpoint{4.533422in}{2.038464in}}{\pgfqpoint{4.544021in}{2.034074in}}{\pgfqpoint{4.555071in}{2.034074in}}%
\pgfpathclose%
\pgfusepath{stroke,fill}%
\end{pgfscope}%
\begin{pgfscope}%
\pgfpathrectangle{\pgfqpoint{0.481978in}{0.331635in}}{\pgfqpoint{4.960000in}{3.696000in}}%
\pgfusepath{clip}%
\pgfsetbuttcap%
\pgfsetroundjoin%
\definecolor{currentfill}{rgb}{0.631373,0.788235,0.956863}%
\pgfsetfillcolor{currentfill}%
\pgfsetlinewidth{0.481800pt}%
\definecolor{currentstroke}{rgb}{1.000000,1.000000,1.000000}%
\pgfsetstrokecolor{currentstroke}%
\pgfsetdash{}{0pt}%
\pgfpathmoveto{\pgfqpoint{2.495370in}{2.119939in}}%
\pgfpathcurveto{\pgfqpoint{2.506420in}{2.119939in}}{\pgfqpoint{2.517019in}{2.124330in}}{\pgfqpoint{2.524833in}{2.132143in}}%
\pgfpathcurveto{\pgfqpoint{2.532647in}{2.139957in}}{\pgfqpoint{2.537037in}{2.150556in}}{\pgfqpoint{2.537037in}{2.161606in}}%
\pgfpathcurveto{\pgfqpoint{2.537037in}{2.172656in}}{\pgfqpoint{2.532647in}{2.183255in}}{\pgfqpoint{2.524833in}{2.191069in}}%
\pgfpathcurveto{\pgfqpoint{2.517019in}{2.198882in}}{\pgfqpoint{2.506420in}{2.203273in}}{\pgfqpoint{2.495370in}{2.203273in}}%
\pgfpathcurveto{\pgfqpoint{2.484320in}{2.203273in}}{\pgfqpoint{2.473721in}{2.198882in}}{\pgfqpoint{2.465907in}{2.191069in}}%
\pgfpathcurveto{\pgfqpoint{2.458094in}{2.183255in}}{\pgfqpoint{2.453703in}{2.172656in}}{\pgfqpoint{2.453703in}{2.161606in}}%
\pgfpathcurveto{\pgfqpoint{2.453703in}{2.150556in}}{\pgfqpoint{2.458094in}{2.139957in}}{\pgfqpoint{2.465907in}{2.132143in}}%
\pgfpathcurveto{\pgfqpoint{2.473721in}{2.124330in}}{\pgfqpoint{2.484320in}{2.119939in}}{\pgfqpoint{2.495370in}{2.119939in}}%
\pgfpathclose%
\pgfusepath{stroke,fill}%
\end{pgfscope}%
\begin{pgfscope}%
\pgfpathrectangle{\pgfqpoint{0.481978in}{0.331635in}}{\pgfqpoint{4.960000in}{3.696000in}}%
\pgfusepath{clip}%
\pgfsetbuttcap%
\pgfsetroundjoin%
\definecolor{currentfill}{rgb}{0.631373,0.788235,0.956863}%
\pgfsetfillcolor{currentfill}%
\pgfsetlinewidth{0.481800pt}%
\definecolor{currentstroke}{rgb}{1.000000,1.000000,1.000000}%
\pgfsetstrokecolor{currentstroke}%
\pgfsetdash{}{0pt}%
\pgfpathmoveto{\pgfqpoint{3.272209in}{1.899227in}}%
\pgfpathcurveto{\pgfqpoint{3.283259in}{1.899227in}}{\pgfqpoint{3.293858in}{1.903617in}}{\pgfqpoint{3.301672in}{1.911431in}}%
\pgfpathcurveto{\pgfqpoint{3.309486in}{1.919244in}}{\pgfqpoint{3.313876in}{1.929843in}}{\pgfqpoint{3.313876in}{1.940893in}}%
\pgfpathcurveto{\pgfqpoint{3.313876in}{1.951944in}}{\pgfqpoint{3.309486in}{1.962543in}}{\pgfqpoint{3.301672in}{1.970356in}}%
\pgfpathcurveto{\pgfqpoint{3.293858in}{1.978170in}}{\pgfqpoint{3.283259in}{1.982560in}}{\pgfqpoint{3.272209in}{1.982560in}}%
\pgfpathcurveto{\pgfqpoint{3.261159in}{1.982560in}}{\pgfqpoint{3.250560in}{1.978170in}}{\pgfqpoint{3.242746in}{1.970356in}}%
\pgfpathcurveto{\pgfqpoint{3.234933in}{1.962543in}}{\pgfqpoint{3.230543in}{1.951944in}}{\pgfqpoint{3.230543in}{1.940893in}}%
\pgfpathcurveto{\pgfqpoint{3.230543in}{1.929843in}}{\pgfqpoint{3.234933in}{1.919244in}}{\pgfqpoint{3.242746in}{1.911431in}}%
\pgfpathcurveto{\pgfqpoint{3.250560in}{1.903617in}}{\pgfqpoint{3.261159in}{1.899227in}}{\pgfqpoint{3.272209in}{1.899227in}}%
\pgfpathclose%
\pgfusepath{stroke,fill}%
\end{pgfscope}%
\begin{pgfscope}%
\pgfpathrectangle{\pgfqpoint{0.481978in}{0.331635in}}{\pgfqpoint{4.960000in}{3.696000in}}%
\pgfusepath{clip}%
\pgfsetbuttcap%
\pgfsetroundjoin%
\definecolor{currentfill}{rgb}{0.631373,0.788235,0.956863}%
\pgfsetfillcolor{currentfill}%
\pgfsetlinewidth{0.481800pt}%
\definecolor{currentstroke}{rgb}{1.000000,1.000000,1.000000}%
\pgfsetstrokecolor{currentstroke}%
\pgfsetdash{}{0pt}%
\pgfpathmoveto{\pgfqpoint{3.691598in}{1.665844in}}%
\pgfpathcurveto{\pgfqpoint{3.702648in}{1.665844in}}{\pgfqpoint{3.713247in}{1.670234in}}{\pgfqpoint{3.721060in}{1.678048in}}%
\pgfpathcurveto{\pgfqpoint{3.728874in}{1.685862in}}{\pgfqpoint{3.733264in}{1.696461in}}{\pgfqpoint{3.733264in}{1.707511in}}%
\pgfpathcurveto{\pgfqpoint{3.733264in}{1.718561in}}{\pgfqpoint{3.728874in}{1.729160in}}{\pgfqpoint{3.721060in}{1.736974in}}%
\pgfpathcurveto{\pgfqpoint{3.713247in}{1.744787in}}{\pgfqpoint{3.702648in}{1.749178in}}{\pgfqpoint{3.691598in}{1.749178in}}%
\pgfpathcurveto{\pgfqpoint{3.680547in}{1.749178in}}{\pgfqpoint{3.669948in}{1.744787in}}{\pgfqpoint{3.662135in}{1.736974in}}%
\pgfpathcurveto{\pgfqpoint{3.654321in}{1.729160in}}{\pgfqpoint{3.649931in}{1.718561in}}{\pgfqpoint{3.649931in}{1.707511in}}%
\pgfpathcurveto{\pgfqpoint{3.649931in}{1.696461in}}{\pgfqpoint{3.654321in}{1.685862in}}{\pgfqpoint{3.662135in}{1.678048in}}%
\pgfpathcurveto{\pgfqpoint{3.669948in}{1.670234in}}{\pgfqpoint{3.680547in}{1.665844in}}{\pgfqpoint{3.691598in}{1.665844in}}%
\pgfpathclose%
\pgfusepath{stroke,fill}%
\end{pgfscope}%
\begin{pgfscope}%
\pgfpathrectangle{\pgfqpoint{0.481978in}{0.331635in}}{\pgfqpoint{4.960000in}{3.696000in}}%
\pgfusepath{clip}%
\pgfsetbuttcap%
\pgfsetroundjoin%
\definecolor{currentfill}{rgb}{0.631373,0.788235,0.956863}%
\pgfsetfillcolor{currentfill}%
\pgfsetlinewidth{0.481800pt}%
\definecolor{currentstroke}{rgb}{1.000000,1.000000,1.000000}%
\pgfsetstrokecolor{currentstroke}%
\pgfsetdash{}{0pt}%
\pgfpathmoveto{\pgfqpoint{3.738955in}{1.514866in}}%
\pgfpathcurveto{\pgfqpoint{3.750005in}{1.514866in}}{\pgfqpoint{3.760604in}{1.519256in}}{\pgfqpoint{3.768417in}{1.527070in}}%
\pgfpathcurveto{\pgfqpoint{3.776231in}{1.534883in}}{\pgfqpoint{3.780621in}{1.545482in}}{\pgfqpoint{3.780621in}{1.556533in}}%
\pgfpathcurveto{\pgfqpoint{3.780621in}{1.567583in}}{\pgfqpoint{3.776231in}{1.578182in}}{\pgfqpoint{3.768417in}{1.585995in}}%
\pgfpathcurveto{\pgfqpoint{3.760604in}{1.593809in}}{\pgfqpoint{3.750005in}{1.598199in}}{\pgfqpoint{3.738955in}{1.598199in}}%
\pgfpathcurveto{\pgfqpoint{3.727904in}{1.598199in}}{\pgfqpoint{3.717305in}{1.593809in}}{\pgfqpoint{3.709492in}{1.585995in}}%
\pgfpathcurveto{\pgfqpoint{3.701678in}{1.578182in}}{\pgfqpoint{3.697288in}{1.567583in}}{\pgfqpoint{3.697288in}{1.556533in}}%
\pgfpathcurveto{\pgfqpoint{3.697288in}{1.545482in}}{\pgfqpoint{3.701678in}{1.534883in}}{\pgfqpoint{3.709492in}{1.527070in}}%
\pgfpathcurveto{\pgfqpoint{3.717305in}{1.519256in}}{\pgfqpoint{3.727904in}{1.514866in}}{\pgfqpoint{3.738955in}{1.514866in}}%
\pgfpathclose%
\pgfusepath{stroke,fill}%
\end{pgfscope}%
\begin{pgfscope}%
\pgfpathrectangle{\pgfqpoint{0.481978in}{0.331635in}}{\pgfqpoint{4.960000in}{3.696000in}}%
\pgfusepath{clip}%
\pgfsetbuttcap%
\pgfsetroundjoin%
\definecolor{currentfill}{rgb}{0.631373,0.788235,0.956863}%
\pgfsetfillcolor{currentfill}%
\pgfsetlinewidth{0.481800pt}%
\definecolor{currentstroke}{rgb}{1.000000,1.000000,1.000000}%
\pgfsetstrokecolor{currentstroke}%
\pgfsetdash{}{0pt}%
\pgfpathmoveto{\pgfqpoint{3.644297in}{3.817968in}}%
\pgfpathcurveto{\pgfqpoint{3.655347in}{3.817968in}}{\pgfqpoint{3.665946in}{3.822359in}}{\pgfqpoint{3.673760in}{3.830172in}}%
\pgfpathcurveto{\pgfqpoint{3.681574in}{3.837986in}}{\pgfqpoint{3.685964in}{3.848585in}}{\pgfqpoint{3.685964in}{3.859635in}}%
\pgfpathcurveto{\pgfqpoint{3.685964in}{3.870685in}}{\pgfqpoint{3.681574in}{3.881284in}}{\pgfqpoint{3.673760in}{3.889098in}}%
\pgfpathcurveto{\pgfqpoint{3.665946in}{3.896911in}}{\pgfqpoint{3.655347in}{3.901302in}}{\pgfqpoint{3.644297in}{3.901302in}}%
\pgfpathcurveto{\pgfqpoint{3.633247in}{3.901302in}}{\pgfqpoint{3.622648in}{3.896911in}}{\pgfqpoint{3.614834in}{3.889098in}}%
\pgfpathcurveto{\pgfqpoint{3.607021in}{3.881284in}}{\pgfqpoint{3.602630in}{3.870685in}}{\pgfqpoint{3.602630in}{3.859635in}}%
\pgfpathcurveto{\pgfqpoint{3.602630in}{3.848585in}}{\pgfqpoint{3.607021in}{3.837986in}}{\pgfqpoint{3.614834in}{3.830172in}}%
\pgfpathcurveto{\pgfqpoint{3.622648in}{3.822359in}}{\pgfqpoint{3.633247in}{3.817968in}}{\pgfqpoint{3.644297in}{3.817968in}}%
\pgfpathclose%
\pgfusepath{stroke,fill}%
\end{pgfscope}%
\begin{pgfscope}%
\pgfpathrectangle{\pgfqpoint{0.481978in}{0.331635in}}{\pgfqpoint{4.960000in}{3.696000in}}%
\pgfusepath{clip}%
\pgfsetbuttcap%
\pgfsetroundjoin%
\definecolor{currentfill}{rgb}{0.631373,0.788235,0.956863}%
\pgfsetfillcolor{currentfill}%
\pgfsetlinewidth{0.481800pt}%
\definecolor{currentstroke}{rgb}{1.000000,1.000000,1.000000}%
\pgfsetstrokecolor{currentstroke}%
\pgfsetdash{}{0pt}%
\pgfpathmoveto{\pgfqpoint{4.157988in}{1.603208in}}%
\pgfpathcurveto{\pgfqpoint{4.169038in}{1.603208in}}{\pgfqpoint{4.179637in}{1.607598in}}{\pgfqpoint{4.187451in}{1.615412in}}%
\pgfpathcurveto{\pgfqpoint{4.195264in}{1.623225in}}{\pgfqpoint{4.199655in}{1.633824in}}{\pgfqpoint{4.199655in}{1.644874in}}%
\pgfpathcurveto{\pgfqpoint{4.199655in}{1.655925in}}{\pgfqpoint{4.195264in}{1.666524in}}{\pgfqpoint{4.187451in}{1.674337in}}%
\pgfpathcurveto{\pgfqpoint{4.179637in}{1.682151in}}{\pgfqpoint{4.169038in}{1.686541in}}{\pgfqpoint{4.157988in}{1.686541in}}%
\pgfpathcurveto{\pgfqpoint{4.146938in}{1.686541in}}{\pgfqpoint{4.136339in}{1.682151in}}{\pgfqpoint{4.128525in}{1.674337in}}%
\pgfpathcurveto{\pgfqpoint{4.120712in}{1.666524in}}{\pgfqpoint{4.116321in}{1.655925in}}{\pgfqpoint{4.116321in}{1.644874in}}%
\pgfpathcurveto{\pgfqpoint{4.116321in}{1.633824in}}{\pgfqpoint{4.120712in}{1.623225in}}{\pgfqpoint{4.128525in}{1.615412in}}%
\pgfpathcurveto{\pgfqpoint{4.136339in}{1.607598in}}{\pgfqpoint{4.146938in}{1.603208in}}{\pgfqpoint{4.157988in}{1.603208in}}%
\pgfpathclose%
\pgfusepath{stroke,fill}%
\end{pgfscope}%
\begin{pgfscope}%
\pgfpathrectangle{\pgfqpoint{0.481978in}{0.331635in}}{\pgfqpoint{4.960000in}{3.696000in}}%
\pgfusepath{clip}%
\pgfsetbuttcap%
\pgfsetroundjoin%
\definecolor{currentfill}{rgb}{0.631373,0.788235,0.956863}%
\pgfsetfillcolor{currentfill}%
\pgfsetlinewidth{0.481800pt}%
\definecolor{currentstroke}{rgb}{1.000000,1.000000,1.000000}%
\pgfsetstrokecolor{currentstroke}%
\pgfsetdash{}{0pt}%
\pgfpathmoveto{\pgfqpoint{3.933198in}{1.294366in}}%
\pgfpathcurveto{\pgfqpoint{3.944248in}{1.294366in}}{\pgfqpoint{3.954847in}{1.298756in}}{\pgfqpoint{3.962661in}{1.306570in}}%
\pgfpathcurveto{\pgfqpoint{3.970475in}{1.314383in}}{\pgfqpoint{3.974865in}{1.324982in}}{\pgfqpoint{3.974865in}{1.336032in}}%
\pgfpathcurveto{\pgfqpoint{3.974865in}{1.347083in}}{\pgfqpoint{3.970475in}{1.357682in}}{\pgfqpoint{3.962661in}{1.365495in}}%
\pgfpathcurveto{\pgfqpoint{3.954847in}{1.373309in}}{\pgfqpoint{3.944248in}{1.377699in}}{\pgfqpoint{3.933198in}{1.377699in}}%
\pgfpathcurveto{\pgfqpoint{3.922148in}{1.377699in}}{\pgfqpoint{3.911549in}{1.373309in}}{\pgfqpoint{3.903735in}{1.365495in}}%
\pgfpathcurveto{\pgfqpoint{3.895922in}{1.357682in}}{\pgfqpoint{3.891532in}{1.347083in}}{\pgfqpoint{3.891532in}{1.336032in}}%
\pgfpathcurveto{\pgfqpoint{3.891532in}{1.324982in}}{\pgfqpoint{3.895922in}{1.314383in}}{\pgfqpoint{3.903735in}{1.306570in}}%
\pgfpathcurveto{\pgfqpoint{3.911549in}{1.298756in}}{\pgfqpoint{3.922148in}{1.294366in}}{\pgfqpoint{3.933198in}{1.294366in}}%
\pgfpathclose%
\pgfusepath{stroke,fill}%
\end{pgfscope}%
\begin{pgfscope}%
\pgfpathrectangle{\pgfqpoint{0.481978in}{0.331635in}}{\pgfqpoint{4.960000in}{3.696000in}}%
\pgfusepath{clip}%
\pgfsetbuttcap%
\pgfsetroundjoin%
\definecolor{currentfill}{rgb}{0.631373,0.788235,0.956863}%
\pgfsetfillcolor{currentfill}%
\pgfsetlinewidth{0.481800pt}%
\definecolor{currentstroke}{rgb}{1.000000,1.000000,1.000000}%
\pgfsetstrokecolor{currentstroke}%
\pgfsetdash{}{0pt}%
\pgfpathmoveto{\pgfqpoint{3.435789in}{1.163854in}}%
\pgfpathcurveto{\pgfqpoint{3.446839in}{1.163854in}}{\pgfqpoint{3.457438in}{1.168245in}}{\pgfqpoint{3.465251in}{1.176058in}}%
\pgfpathcurveto{\pgfqpoint{3.473065in}{1.183872in}}{\pgfqpoint{3.477455in}{1.194471in}}{\pgfqpoint{3.477455in}{1.205521in}}%
\pgfpathcurveto{\pgfqpoint{3.477455in}{1.216571in}}{\pgfqpoint{3.473065in}{1.227170in}}{\pgfqpoint{3.465251in}{1.234984in}}%
\pgfpathcurveto{\pgfqpoint{3.457438in}{1.242797in}}{\pgfqpoint{3.446839in}{1.247188in}}{\pgfqpoint{3.435789in}{1.247188in}}%
\pgfpathcurveto{\pgfqpoint{3.424738in}{1.247188in}}{\pgfqpoint{3.414139in}{1.242797in}}{\pgfqpoint{3.406326in}{1.234984in}}%
\pgfpathcurveto{\pgfqpoint{3.398512in}{1.227170in}}{\pgfqpoint{3.394122in}{1.216571in}}{\pgfqpoint{3.394122in}{1.205521in}}%
\pgfpathcurveto{\pgfqpoint{3.394122in}{1.194471in}}{\pgfqpoint{3.398512in}{1.183872in}}{\pgfqpoint{3.406326in}{1.176058in}}%
\pgfpathcurveto{\pgfqpoint{3.414139in}{1.168245in}}{\pgfqpoint{3.424738in}{1.163854in}}{\pgfqpoint{3.435789in}{1.163854in}}%
\pgfpathclose%
\pgfusepath{stroke,fill}%
\end{pgfscope}%
\begin{pgfscope}%
\pgfpathrectangle{\pgfqpoint{0.481978in}{0.331635in}}{\pgfqpoint{4.960000in}{3.696000in}}%
\pgfusepath{clip}%
\pgfsetbuttcap%
\pgfsetroundjoin%
\definecolor{currentfill}{rgb}{0.631373,0.788235,0.956863}%
\pgfsetfillcolor{currentfill}%
\pgfsetlinewidth{0.481800pt}%
\definecolor{currentstroke}{rgb}{1.000000,1.000000,1.000000}%
\pgfsetstrokecolor{currentstroke}%
\pgfsetdash{}{0pt}%
\pgfpathmoveto{\pgfqpoint{3.463887in}{2.573598in}}%
\pgfpathcurveto{\pgfqpoint{3.474937in}{2.573598in}}{\pgfqpoint{3.485536in}{2.577989in}}{\pgfqpoint{3.493350in}{2.585802in}}%
\pgfpathcurveto{\pgfqpoint{3.501163in}{2.593616in}}{\pgfqpoint{3.505554in}{2.604215in}}{\pgfqpoint{3.505554in}{2.615265in}}%
\pgfpathcurveto{\pgfqpoint{3.505554in}{2.626315in}}{\pgfqpoint{3.501163in}{2.636914in}}{\pgfqpoint{3.493350in}{2.644728in}}%
\pgfpathcurveto{\pgfqpoint{3.485536in}{2.652541in}}{\pgfqpoint{3.474937in}{2.656932in}}{\pgfqpoint{3.463887in}{2.656932in}}%
\pgfpathcurveto{\pgfqpoint{3.452837in}{2.656932in}}{\pgfqpoint{3.442238in}{2.652541in}}{\pgfqpoint{3.434424in}{2.644728in}}%
\pgfpathcurveto{\pgfqpoint{3.426611in}{2.636914in}}{\pgfqpoint{3.422220in}{2.626315in}}{\pgfqpoint{3.422220in}{2.615265in}}%
\pgfpathcurveto{\pgfqpoint{3.422220in}{2.604215in}}{\pgfqpoint{3.426611in}{2.593616in}}{\pgfqpoint{3.434424in}{2.585802in}}%
\pgfpathcurveto{\pgfqpoint{3.442238in}{2.577989in}}{\pgfqpoint{3.452837in}{2.573598in}}{\pgfqpoint{3.463887in}{2.573598in}}%
\pgfpathclose%
\pgfusepath{stroke,fill}%
\end{pgfscope}%
\begin{pgfscope}%
\pgfpathrectangle{\pgfqpoint{0.481978in}{0.331635in}}{\pgfqpoint{4.960000in}{3.696000in}}%
\pgfusepath{clip}%
\pgfsetbuttcap%
\pgfsetroundjoin%
\definecolor{currentfill}{rgb}{0.631373,0.788235,0.956863}%
\pgfsetfillcolor{currentfill}%
\pgfsetlinewidth{0.481800pt}%
\definecolor{currentstroke}{rgb}{1.000000,1.000000,1.000000}%
\pgfsetstrokecolor{currentstroke}%
\pgfsetdash{}{0pt}%
\pgfpathmoveto{\pgfqpoint{3.504958in}{2.274782in}}%
\pgfpathcurveto{\pgfqpoint{3.516009in}{2.274782in}}{\pgfqpoint{3.526608in}{2.279172in}}{\pgfqpoint{3.534421in}{2.286985in}}%
\pgfpathcurveto{\pgfqpoint{3.542235in}{2.294799in}}{\pgfqpoint{3.546625in}{2.305398in}}{\pgfqpoint{3.546625in}{2.316448in}}%
\pgfpathcurveto{\pgfqpoint{3.546625in}{2.327498in}}{\pgfqpoint{3.542235in}{2.338097in}}{\pgfqpoint{3.534421in}{2.345911in}}%
\pgfpathcurveto{\pgfqpoint{3.526608in}{2.353725in}}{\pgfqpoint{3.516009in}{2.358115in}}{\pgfqpoint{3.504958in}{2.358115in}}%
\pgfpathcurveto{\pgfqpoint{3.493908in}{2.358115in}}{\pgfqpoint{3.483309in}{2.353725in}}{\pgfqpoint{3.475496in}{2.345911in}}%
\pgfpathcurveto{\pgfqpoint{3.467682in}{2.338097in}}{\pgfqpoint{3.463292in}{2.327498in}}{\pgfqpoint{3.463292in}{2.316448in}}%
\pgfpathcurveto{\pgfqpoint{3.463292in}{2.305398in}}{\pgfqpoint{3.467682in}{2.294799in}}{\pgfqpoint{3.475496in}{2.286985in}}%
\pgfpathcurveto{\pgfqpoint{3.483309in}{2.279172in}}{\pgfqpoint{3.493908in}{2.274782in}}{\pgfqpoint{3.504958in}{2.274782in}}%
\pgfpathclose%
\pgfusepath{stroke,fill}%
\end{pgfscope}%
\begin{pgfscope}%
\pgfpathrectangle{\pgfqpoint{0.481978in}{0.331635in}}{\pgfqpoint{4.960000in}{3.696000in}}%
\pgfusepath{clip}%
\pgfsetbuttcap%
\pgfsetroundjoin%
\definecolor{currentfill}{rgb}{0.631373,0.788235,0.956863}%
\pgfsetfillcolor{currentfill}%
\pgfsetlinewidth{0.481800pt}%
\definecolor{currentstroke}{rgb}{1.000000,1.000000,1.000000}%
\pgfsetstrokecolor{currentstroke}%
\pgfsetdash{}{0pt}%
\pgfpathmoveto{\pgfqpoint{3.831055in}{1.112292in}}%
\pgfpathcurveto{\pgfqpoint{3.842105in}{1.112292in}}{\pgfqpoint{3.852704in}{1.116682in}}{\pgfqpoint{3.860517in}{1.124496in}}%
\pgfpathcurveto{\pgfqpoint{3.868331in}{1.132310in}}{\pgfqpoint{3.872721in}{1.142909in}}{\pgfqpoint{3.872721in}{1.153959in}}%
\pgfpathcurveto{\pgfqpoint{3.872721in}{1.165009in}}{\pgfqpoint{3.868331in}{1.175608in}}{\pgfqpoint{3.860517in}{1.183421in}}%
\pgfpathcurveto{\pgfqpoint{3.852704in}{1.191235in}}{\pgfqpoint{3.842105in}{1.195625in}}{\pgfqpoint{3.831055in}{1.195625in}}%
\pgfpathcurveto{\pgfqpoint{3.820004in}{1.195625in}}{\pgfqpoint{3.809405in}{1.191235in}}{\pgfqpoint{3.801592in}{1.183421in}}%
\pgfpathcurveto{\pgfqpoint{3.793778in}{1.175608in}}{\pgfqpoint{3.789388in}{1.165009in}}{\pgfqpoint{3.789388in}{1.153959in}}%
\pgfpathcurveto{\pgfqpoint{3.789388in}{1.142909in}}{\pgfqpoint{3.793778in}{1.132310in}}{\pgfqpoint{3.801592in}{1.124496in}}%
\pgfpathcurveto{\pgfqpoint{3.809405in}{1.116682in}}{\pgfqpoint{3.820004in}{1.112292in}}{\pgfqpoint{3.831055in}{1.112292in}}%
\pgfpathclose%
\pgfusepath{stroke,fill}%
\end{pgfscope}%
\begin{pgfscope}%
\pgfpathrectangle{\pgfqpoint{0.481978in}{0.331635in}}{\pgfqpoint{4.960000in}{3.696000in}}%
\pgfusepath{clip}%
\pgfsetbuttcap%
\pgfsetroundjoin%
\definecolor{currentfill}{rgb}{0.631373,0.788235,0.956863}%
\pgfsetfillcolor{currentfill}%
\pgfsetlinewidth{0.481800pt}%
\definecolor{currentstroke}{rgb}{1.000000,1.000000,1.000000}%
\pgfsetstrokecolor{currentstroke}%
\pgfsetdash{}{0pt}%
\pgfpathmoveto{\pgfqpoint{3.307965in}{1.852676in}}%
\pgfpathcurveto{\pgfqpoint{3.319015in}{1.852676in}}{\pgfqpoint{3.329614in}{1.857067in}}{\pgfqpoint{3.337428in}{1.864880in}}%
\pgfpathcurveto{\pgfqpoint{3.345242in}{1.872694in}}{\pgfqpoint{3.349632in}{1.883293in}}{\pgfqpoint{3.349632in}{1.894343in}}%
\pgfpathcurveto{\pgfqpoint{3.349632in}{1.905393in}}{\pgfqpoint{3.345242in}{1.915992in}}{\pgfqpoint{3.337428in}{1.923806in}}%
\pgfpathcurveto{\pgfqpoint{3.329614in}{1.931620in}}{\pgfqpoint{3.319015in}{1.936010in}}{\pgfqpoint{3.307965in}{1.936010in}}%
\pgfpathcurveto{\pgfqpoint{3.296915in}{1.936010in}}{\pgfqpoint{3.286316in}{1.931620in}}{\pgfqpoint{3.278502in}{1.923806in}}%
\pgfpathcurveto{\pgfqpoint{3.270689in}{1.915992in}}{\pgfqpoint{3.266299in}{1.905393in}}{\pgfqpoint{3.266299in}{1.894343in}}%
\pgfpathcurveto{\pgfqpoint{3.266299in}{1.883293in}}{\pgfqpoint{3.270689in}{1.872694in}}{\pgfqpoint{3.278502in}{1.864880in}}%
\pgfpathcurveto{\pgfqpoint{3.286316in}{1.857067in}}{\pgfqpoint{3.296915in}{1.852676in}}{\pgfqpoint{3.307965in}{1.852676in}}%
\pgfpathclose%
\pgfusepath{stroke,fill}%
\end{pgfscope}%
\begin{pgfscope}%
\pgfpathrectangle{\pgfqpoint{0.481978in}{0.331635in}}{\pgfqpoint{4.960000in}{3.696000in}}%
\pgfusepath{clip}%
\pgfsetbuttcap%
\pgfsetroundjoin%
\definecolor{currentfill}{rgb}{0.631373,0.788235,0.956863}%
\pgfsetfillcolor{currentfill}%
\pgfsetlinewidth{0.481800pt}%
\definecolor{currentstroke}{rgb}{1.000000,1.000000,1.000000}%
\pgfsetstrokecolor{currentstroke}%
\pgfsetdash{}{0pt}%
\pgfpathmoveto{\pgfqpoint{3.139271in}{1.852811in}}%
\pgfpathcurveto{\pgfqpoint{3.150321in}{1.852811in}}{\pgfqpoint{3.160920in}{1.857202in}}{\pgfqpoint{3.168734in}{1.865015in}}%
\pgfpathcurveto{\pgfqpoint{3.176547in}{1.872829in}}{\pgfqpoint{3.180937in}{1.883428in}}{\pgfqpoint{3.180937in}{1.894478in}}%
\pgfpathcurveto{\pgfqpoint{3.180937in}{1.905528in}}{\pgfqpoint{3.176547in}{1.916127in}}{\pgfqpoint{3.168734in}{1.923941in}}%
\pgfpathcurveto{\pgfqpoint{3.160920in}{1.931755in}}{\pgfqpoint{3.150321in}{1.936145in}}{\pgfqpoint{3.139271in}{1.936145in}}%
\pgfpathcurveto{\pgfqpoint{3.128221in}{1.936145in}}{\pgfqpoint{3.117622in}{1.931755in}}{\pgfqpoint{3.109808in}{1.923941in}}%
\pgfpathcurveto{\pgfqpoint{3.101994in}{1.916127in}}{\pgfqpoint{3.097604in}{1.905528in}}{\pgfqpoint{3.097604in}{1.894478in}}%
\pgfpathcurveto{\pgfqpoint{3.097604in}{1.883428in}}{\pgfqpoint{3.101994in}{1.872829in}}{\pgfqpoint{3.109808in}{1.865015in}}%
\pgfpathcurveto{\pgfqpoint{3.117622in}{1.857202in}}{\pgfqpoint{3.128221in}{1.852811in}}{\pgfqpoint{3.139271in}{1.852811in}}%
\pgfpathclose%
\pgfusepath{stroke,fill}%
\end{pgfscope}%
\begin{pgfscope}%
\pgfpathrectangle{\pgfqpoint{0.481978in}{0.331635in}}{\pgfqpoint{4.960000in}{3.696000in}}%
\pgfusepath{clip}%
\pgfsetbuttcap%
\pgfsetroundjoin%
\definecolor{currentfill}{rgb}{0.631373,0.788235,0.956863}%
\pgfsetfillcolor{currentfill}%
\pgfsetlinewidth{0.481800pt}%
\definecolor{currentstroke}{rgb}{1.000000,1.000000,1.000000}%
\pgfsetstrokecolor{currentstroke}%
\pgfsetdash{}{0pt}%
\pgfpathmoveto{\pgfqpoint{2.687418in}{1.222758in}}%
\pgfpathcurveto{\pgfqpoint{2.698469in}{1.222758in}}{\pgfqpoint{2.709068in}{1.227148in}}{\pgfqpoint{2.716881in}{1.234962in}}%
\pgfpathcurveto{\pgfqpoint{2.724695in}{1.242775in}}{\pgfqpoint{2.729085in}{1.253374in}}{\pgfqpoint{2.729085in}{1.264424in}}%
\pgfpathcurveto{\pgfqpoint{2.729085in}{1.275475in}}{\pgfqpoint{2.724695in}{1.286074in}}{\pgfqpoint{2.716881in}{1.293887in}}%
\pgfpathcurveto{\pgfqpoint{2.709068in}{1.301701in}}{\pgfqpoint{2.698469in}{1.306091in}}{\pgfqpoint{2.687418in}{1.306091in}}%
\pgfpathcurveto{\pgfqpoint{2.676368in}{1.306091in}}{\pgfqpoint{2.665769in}{1.301701in}}{\pgfqpoint{2.657956in}{1.293887in}}%
\pgfpathcurveto{\pgfqpoint{2.650142in}{1.286074in}}{\pgfqpoint{2.645752in}{1.275475in}}{\pgfqpoint{2.645752in}{1.264424in}}%
\pgfpathcurveto{\pgfqpoint{2.645752in}{1.253374in}}{\pgfqpoint{2.650142in}{1.242775in}}{\pgfqpoint{2.657956in}{1.234962in}}%
\pgfpathcurveto{\pgfqpoint{2.665769in}{1.227148in}}{\pgfqpoint{2.676368in}{1.222758in}}{\pgfqpoint{2.687418in}{1.222758in}}%
\pgfpathclose%
\pgfusepath{stroke,fill}%
\end{pgfscope}%
\begin{pgfscope}%
\pgfpathrectangle{\pgfqpoint{0.481978in}{0.331635in}}{\pgfqpoint{4.960000in}{3.696000in}}%
\pgfusepath{clip}%
\pgfsetbuttcap%
\pgfsetroundjoin%
\definecolor{currentfill}{rgb}{0.631373,0.788235,0.956863}%
\pgfsetfillcolor{currentfill}%
\pgfsetlinewidth{0.481800pt}%
\definecolor{currentstroke}{rgb}{1.000000,1.000000,1.000000}%
\pgfsetstrokecolor{currentstroke}%
\pgfsetdash{}{0pt}%
\pgfpathmoveto{\pgfqpoint{3.342745in}{2.270740in}}%
\pgfpathcurveto{\pgfqpoint{3.353795in}{2.270740in}}{\pgfqpoint{3.364394in}{2.275130in}}{\pgfqpoint{3.372208in}{2.282944in}}%
\pgfpathcurveto{\pgfqpoint{3.380021in}{2.290757in}}{\pgfqpoint{3.384412in}{2.301356in}}{\pgfqpoint{3.384412in}{2.312406in}}%
\pgfpathcurveto{\pgfqpoint{3.384412in}{2.323457in}}{\pgfqpoint{3.380021in}{2.334056in}}{\pgfqpoint{3.372208in}{2.341869in}}%
\pgfpathcurveto{\pgfqpoint{3.364394in}{2.349683in}}{\pgfqpoint{3.353795in}{2.354073in}}{\pgfqpoint{3.342745in}{2.354073in}}%
\pgfpathcurveto{\pgfqpoint{3.331695in}{2.354073in}}{\pgfqpoint{3.321096in}{2.349683in}}{\pgfqpoint{3.313282in}{2.341869in}}%
\pgfpathcurveto{\pgfqpoint{3.305468in}{2.334056in}}{\pgfqpoint{3.301078in}{2.323457in}}{\pgfqpoint{3.301078in}{2.312406in}}%
\pgfpathcurveto{\pgfqpoint{3.301078in}{2.301356in}}{\pgfqpoint{3.305468in}{2.290757in}}{\pgfqpoint{3.313282in}{2.282944in}}%
\pgfpathcurveto{\pgfqpoint{3.321096in}{2.275130in}}{\pgfqpoint{3.331695in}{2.270740in}}{\pgfqpoint{3.342745in}{2.270740in}}%
\pgfpathclose%
\pgfusepath{stroke,fill}%
\end{pgfscope}%
\begin{pgfscope}%
\pgfpathrectangle{\pgfqpoint{0.481978in}{0.331635in}}{\pgfqpoint{4.960000in}{3.696000in}}%
\pgfusepath{clip}%
\pgfsetbuttcap%
\pgfsetroundjoin%
\definecolor{currentfill}{rgb}{0.631373,0.788235,0.956863}%
\pgfsetfillcolor{currentfill}%
\pgfsetlinewidth{0.481800pt}%
\definecolor{currentstroke}{rgb}{1.000000,1.000000,1.000000}%
\pgfsetstrokecolor{currentstroke}%
\pgfsetdash{}{0pt}%
\pgfpathmoveto{\pgfqpoint{2.957882in}{0.803375in}}%
\pgfpathcurveto{\pgfqpoint{2.968932in}{0.803375in}}{\pgfqpoint{2.979531in}{0.807766in}}{\pgfqpoint{2.987345in}{0.815579in}}%
\pgfpathcurveto{\pgfqpoint{2.995158in}{0.823393in}}{\pgfqpoint{2.999549in}{0.833992in}}{\pgfqpoint{2.999549in}{0.845042in}}%
\pgfpathcurveto{\pgfqpoint{2.999549in}{0.856092in}}{\pgfqpoint{2.995158in}{0.866691in}}{\pgfqpoint{2.987345in}{0.874505in}}%
\pgfpathcurveto{\pgfqpoint{2.979531in}{0.882318in}}{\pgfqpoint{2.968932in}{0.886709in}}{\pgfqpoint{2.957882in}{0.886709in}}%
\pgfpathcurveto{\pgfqpoint{2.946832in}{0.886709in}}{\pgfqpoint{2.936233in}{0.882318in}}{\pgfqpoint{2.928419in}{0.874505in}}%
\pgfpathcurveto{\pgfqpoint{2.920606in}{0.866691in}}{\pgfqpoint{2.916215in}{0.856092in}}{\pgfqpoint{2.916215in}{0.845042in}}%
\pgfpathcurveto{\pgfqpoint{2.916215in}{0.833992in}}{\pgfqpoint{2.920606in}{0.823393in}}{\pgfqpoint{2.928419in}{0.815579in}}%
\pgfpathcurveto{\pgfqpoint{2.936233in}{0.807766in}}{\pgfqpoint{2.946832in}{0.803375in}}{\pgfqpoint{2.957882in}{0.803375in}}%
\pgfpathclose%
\pgfusepath{stroke,fill}%
\end{pgfscope}%
\begin{pgfscope}%
\pgfpathrectangle{\pgfqpoint{0.481978in}{0.331635in}}{\pgfqpoint{4.960000in}{3.696000in}}%
\pgfusepath{clip}%
\pgfsetbuttcap%
\pgfsetroundjoin%
\definecolor{currentfill}{rgb}{0.631373,0.788235,0.956863}%
\pgfsetfillcolor{currentfill}%
\pgfsetlinewidth{0.481800pt}%
\definecolor{currentstroke}{rgb}{1.000000,1.000000,1.000000}%
\pgfsetstrokecolor{currentstroke}%
\pgfsetdash{}{0pt}%
\pgfpathmoveto{\pgfqpoint{2.998166in}{1.563891in}}%
\pgfpathcurveto{\pgfqpoint{3.009216in}{1.563891in}}{\pgfqpoint{3.019815in}{1.568281in}}{\pgfqpoint{3.027629in}{1.576095in}}%
\pgfpathcurveto{\pgfqpoint{3.035442in}{1.583908in}}{\pgfqpoint{3.039833in}{1.594507in}}{\pgfqpoint{3.039833in}{1.605557in}}%
\pgfpathcurveto{\pgfqpoint{3.039833in}{1.616607in}}{\pgfqpoint{3.035442in}{1.627206in}}{\pgfqpoint{3.027629in}{1.635020in}}%
\pgfpathcurveto{\pgfqpoint{3.019815in}{1.642834in}}{\pgfqpoint{3.009216in}{1.647224in}}{\pgfqpoint{2.998166in}{1.647224in}}%
\pgfpathcurveto{\pgfqpoint{2.987116in}{1.647224in}}{\pgfqpoint{2.976517in}{1.642834in}}{\pgfqpoint{2.968703in}{1.635020in}}%
\pgfpathcurveto{\pgfqpoint{2.960890in}{1.627206in}}{\pgfqpoint{2.956499in}{1.616607in}}{\pgfqpoint{2.956499in}{1.605557in}}%
\pgfpathcurveto{\pgfqpoint{2.956499in}{1.594507in}}{\pgfqpoint{2.960890in}{1.583908in}}{\pgfqpoint{2.968703in}{1.576095in}}%
\pgfpathcurveto{\pgfqpoint{2.976517in}{1.568281in}}{\pgfqpoint{2.987116in}{1.563891in}}{\pgfqpoint{2.998166in}{1.563891in}}%
\pgfpathclose%
\pgfusepath{stroke,fill}%
\end{pgfscope}%
\begin{pgfscope}%
\pgfpathrectangle{\pgfqpoint{0.481978in}{0.331635in}}{\pgfqpoint{4.960000in}{3.696000in}}%
\pgfusepath{clip}%
\pgfsetbuttcap%
\pgfsetroundjoin%
\definecolor{currentfill}{rgb}{0.631373,0.788235,0.956863}%
\pgfsetfillcolor{currentfill}%
\pgfsetlinewidth{0.481800pt}%
\definecolor{currentstroke}{rgb}{1.000000,1.000000,1.000000}%
\pgfsetstrokecolor{currentstroke}%
\pgfsetdash{}{0pt}%
\pgfpathmoveto{\pgfqpoint{3.419497in}{1.472546in}}%
\pgfpathcurveto{\pgfqpoint{3.430547in}{1.472546in}}{\pgfqpoint{3.441146in}{1.476936in}}{\pgfqpoint{3.448959in}{1.484750in}}%
\pgfpathcurveto{\pgfqpoint{3.456773in}{1.492563in}}{\pgfqpoint{3.461163in}{1.503162in}}{\pgfqpoint{3.461163in}{1.514212in}}%
\pgfpathcurveto{\pgfqpoint{3.461163in}{1.525263in}}{\pgfqpoint{3.456773in}{1.535862in}}{\pgfqpoint{3.448959in}{1.543675in}}%
\pgfpathcurveto{\pgfqpoint{3.441146in}{1.551489in}}{\pgfqpoint{3.430547in}{1.555879in}}{\pgfqpoint{3.419497in}{1.555879in}}%
\pgfpathcurveto{\pgfqpoint{3.408446in}{1.555879in}}{\pgfqpoint{3.397847in}{1.551489in}}{\pgfqpoint{3.390034in}{1.543675in}}%
\pgfpathcurveto{\pgfqpoint{3.382220in}{1.535862in}}{\pgfqpoint{3.377830in}{1.525263in}}{\pgfqpoint{3.377830in}{1.514212in}}%
\pgfpathcurveto{\pgfqpoint{3.377830in}{1.503162in}}{\pgfqpoint{3.382220in}{1.492563in}}{\pgfqpoint{3.390034in}{1.484750in}}%
\pgfpathcurveto{\pgfqpoint{3.397847in}{1.476936in}}{\pgfqpoint{3.408446in}{1.472546in}}{\pgfqpoint{3.419497in}{1.472546in}}%
\pgfpathclose%
\pgfusepath{stroke,fill}%
\end{pgfscope}%
\begin{pgfscope}%
\pgfpathrectangle{\pgfqpoint{0.481978in}{0.331635in}}{\pgfqpoint{4.960000in}{3.696000in}}%
\pgfusepath{clip}%
\pgfsetbuttcap%
\pgfsetroundjoin%
\definecolor{currentfill}{rgb}{0.631373,0.788235,0.956863}%
\pgfsetfillcolor{currentfill}%
\pgfsetlinewidth{0.481800pt}%
\definecolor{currentstroke}{rgb}{1.000000,1.000000,1.000000}%
\pgfsetstrokecolor{currentstroke}%
\pgfsetdash{}{0pt}%
\pgfpathmoveto{\pgfqpoint{3.105860in}{0.711660in}}%
\pgfpathcurveto{\pgfqpoint{3.116910in}{0.711660in}}{\pgfqpoint{3.127509in}{0.716051in}}{\pgfqpoint{3.135322in}{0.723864in}}%
\pgfpathcurveto{\pgfqpoint{3.143136in}{0.731678in}}{\pgfqpoint{3.147526in}{0.742277in}}{\pgfqpoint{3.147526in}{0.753327in}}%
\pgfpathcurveto{\pgfqpoint{3.147526in}{0.764377in}}{\pgfqpoint{3.143136in}{0.774976in}}{\pgfqpoint{3.135322in}{0.782790in}}%
\pgfpathcurveto{\pgfqpoint{3.127509in}{0.790603in}}{\pgfqpoint{3.116910in}{0.794994in}}{\pgfqpoint{3.105860in}{0.794994in}}%
\pgfpathcurveto{\pgfqpoint{3.094809in}{0.794994in}}{\pgfqpoint{3.084210in}{0.790603in}}{\pgfqpoint{3.076397in}{0.782790in}}%
\pgfpathcurveto{\pgfqpoint{3.068583in}{0.774976in}}{\pgfqpoint{3.064193in}{0.764377in}}{\pgfqpoint{3.064193in}{0.753327in}}%
\pgfpathcurveto{\pgfqpoint{3.064193in}{0.742277in}}{\pgfqpoint{3.068583in}{0.731678in}}{\pgfqpoint{3.076397in}{0.723864in}}%
\pgfpathcurveto{\pgfqpoint{3.084210in}{0.716051in}}{\pgfqpoint{3.094809in}{0.711660in}}{\pgfqpoint{3.105860in}{0.711660in}}%
\pgfpathclose%
\pgfusepath{stroke,fill}%
\end{pgfscope}%
\begin{pgfscope}%
\pgfpathrectangle{\pgfqpoint{0.481978in}{0.331635in}}{\pgfqpoint{4.960000in}{3.696000in}}%
\pgfusepath{clip}%
\pgfsetbuttcap%
\pgfsetroundjoin%
\definecolor{currentfill}{rgb}{0.631373,0.788235,0.956863}%
\pgfsetfillcolor{currentfill}%
\pgfsetlinewidth{0.481800pt}%
\definecolor{currentstroke}{rgb}{1.000000,1.000000,1.000000}%
\pgfsetstrokecolor{currentstroke}%
\pgfsetdash{}{0pt}%
\pgfpathmoveto{\pgfqpoint{3.366070in}{2.377728in}}%
\pgfpathcurveto{\pgfqpoint{3.377120in}{2.377728in}}{\pgfqpoint{3.387719in}{2.382118in}}{\pgfqpoint{3.395533in}{2.389931in}}%
\pgfpathcurveto{\pgfqpoint{3.403346in}{2.397745in}}{\pgfqpoint{3.407737in}{2.408344in}}{\pgfqpoint{3.407737in}{2.419394in}}%
\pgfpathcurveto{\pgfqpoint{3.407737in}{2.430444in}}{\pgfqpoint{3.403346in}{2.441043in}}{\pgfqpoint{3.395533in}{2.448857in}}%
\pgfpathcurveto{\pgfqpoint{3.387719in}{2.456671in}}{\pgfqpoint{3.377120in}{2.461061in}}{\pgfqpoint{3.366070in}{2.461061in}}%
\pgfpathcurveto{\pgfqpoint{3.355020in}{2.461061in}}{\pgfqpoint{3.344421in}{2.456671in}}{\pgfqpoint{3.336607in}{2.448857in}}%
\pgfpathcurveto{\pgfqpoint{3.328793in}{2.441043in}}{\pgfqpoint{3.324403in}{2.430444in}}{\pgfqpoint{3.324403in}{2.419394in}}%
\pgfpathcurveto{\pgfqpoint{3.324403in}{2.408344in}}{\pgfqpoint{3.328793in}{2.397745in}}{\pgfqpoint{3.336607in}{2.389931in}}%
\pgfpathcurveto{\pgfqpoint{3.344421in}{2.382118in}}{\pgfqpoint{3.355020in}{2.377728in}}{\pgfqpoint{3.366070in}{2.377728in}}%
\pgfpathclose%
\pgfusepath{stroke,fill}%
\end{pgfscope}%
\begin{pgfscope}%
\pgfpathrectangle{\pgfqpoint{0.481978in}{0.331635in}}{\pgfqpoint{4.960000in}{3.696000in}}%
\pgfusepath{clip}%
\pgfsetbuttcap%
\pgfsetroundjoin%
\definecolor{currentfill}{rgb}{0.631373,0.788235,0.956863}%
\pgfsetfillcolor{currentfill}%
\pgfsetlinewidth{0.481800pt}%
\definecolor{currentstroke}{rgb}{1.000000,1.000000,1.000000}%
\pgfsetstrokecolor{currentstroke}%
\pgfsetdash{}{0pt}%
\pgfpathmoveto{\pgfqpoint{4.220389in}{2.682242in}}%
\pgfpathcurveto{\pgfqpoint{4.231439in}{2.682242in}}{\pgfqpoint{4.242038in}{2.686632in}}{\pgfqpoint{4.249852in}{2.694446in}}%
\pgfpathcurveto{\pgfqpoint{4.257665in}{2.702259in}}{\pgfqpoint{4.262055in}{2.712858in}}{\pgfqpoint{4.262055in}{2.723908in}}%
\pgfpathcurveto{\pgfqpoint{4.262055in}{2.734958in}}{\pgfqpoint{4.257665in}{2.745558in}}{\pgfqpoint{4.249852in}{2.753371in}}%
\pgfpathcurveto{\pgfqpoint{4.242038in}{2.761185in}}{\pgfqpoint{4.231439in}{2.765575in}}{\pgfqpoint{4.220389in}{2.765575in}}%
\pgfpathcurveto{\pgfqpoint{4.209339in}{2.765575in}}{\pgfqpoint{4.198740in}{2.761185in}}{\pgfqpoint{4.190926in}{2.753371in}}%
\pgfpathcurveto{\pgfqpoint{4.183112in}{2.745558in}}{\pgfqpoint{4.178722in}{2.734958in}}{\pgfqpoint{4.178722in}{2.723908in}}%
\pgfpathcurveto{\pgfqpoint{4.178722in}{2.712858in}}{\pgfqpoint{4.183112in}{2.702259in}}{\pgfqpoint{4.190926in}{2.694446in}}%
\pgfpathcurveto{\pgfqpoint{4.198740in}{2.686632in}}{\pgfqpoint{4.209339in}{2.682242in}}{\pgfqpoint{4.220389in}{2.682242in}}%
\pgfpathclose%
\pgfusepath{stroke,fill}%
\end{pgfscope}%
\begin{pgfscope}%
\pgfpathrectangle{\pgfqpoint{0.481978in}{0.331635in}}{\pgfqpoint{4.960000in}{3.696000in}}%
\pgfusepath{clip}%
\pgfsetbuttcap%
\pgfsetroundjoin%
\definecolor{currentfill}{rgb}{0.631373,0.788235,0.956863}%
\pgfsetfillcolor{currentfill}%
\pgfsetlinewidth{0.481800pt}%
\definecolor{currentstroke}{rgb}{1.000000,1.000000,1.000000}%
\pgfsetstrokecolor{currentstroke}%
\pgfsetdash{}{0pt}%
\pgfpathmoveto{\pgfqpoint{2.413294in}{0.910235in}}%
\pgfpathcurveto{\pgfqpoint{2.424344in}{0.910235in}}{\pgfqpoint{2.434943in}{0.914625in}}{\pgfqpoint{2.442757in}{0.922439in}}%
\pgfpathcurveto{\pgfqpoint{2.450570in}{0.930252in}}{\pgfqpoint{2.454961in}{0.940851in}}{\pgfqpoint{2.454961in}{0.951901in}}%
\pgfpathcurveto{\pgfqpoint{2.454961in}{0.962952in}}{\pgfqpoint{2.450570in}{0.973551in}}{\pgfqpoint{2.442757in}{0.981364in}}%
\pgfpathcurveto{\pgfqpoint{2.434943in}{0.989178in}}{\pgfqpoint{2.424344in}{0.993568in}}{\pgfqpoint{2.413294in}{0.993568in}}%
\pgfpathcurveto{\pgfqpoint{2.402244in}{0.993568in}}{\pgfqpoint{2.391645in}{0.989178in}}{\pgfqpoint{2.383831in}{0.981364in}}%
\pgfpathcurveto{\pgfqpoint{2.376018in}{0.973551in}}{\pgfqpoint{2.371627in}{0.962952in}}{\pgfqpoint{2.371627in}{0.951901in}}%
\pgfpathcurveto{\pgfqpoint{2.371627in}{0.940851in}}{\pgfqpoint{2.376018in}{0.930252in}}{\pgfqpoint{2.383831in}{0.922439in}}%
\pgfpathcurveto{\pgfqpoint{2.391645in}{0.914625in}}{\pgfqpoint{2.402244in}{0.910235in}}{\pgfqpoint{2.413294in}{0.910235in}}%
\pgfpathclose%
\pgfusepath{stroke,fill}%
\end{pgfscope}%
\begin{pgfscope}%
\pgfpathrectangle{\pgfqpoint{0.481978in}{0.331635in}}{\pgfqpoint{4.960000in}{3.696000in}}%
\pgfusepath{clip}%
\pgfsetbuttcap%
\pgfsetroundjoin%
\definecolor{currentfill}{rgb}{0.631373,0.788235,0.956863}%
\pgfsetfillcolor{currentfill}%
\pgfsetlinewidth{0.481800pt}%
\definecolor{currentstroke}{rgb}{1.000000,1.000000,1.000000}%
\pgfsetstrokecolor{currentstroke}%
\pgfsetdash{}{0pt}%
\pgfpathmoveto{\pgfqpoint{2.058354in}{0.761218in}}%
\pgfpathcurveto{\pgfqpoint{2.069404in}{0.761218in}}{\pgfqpoint{2.080003in}{0.765609in}}{\pgfqpoint{2.087817in}{0.773422in}}%
\pgfpathcurveto{\pgfqpoint{2.095630in}{0.781236in}}{\pgfqpoint{2.100021in}{0.791835in}}{\pgfqpoint{2.100021in}{0.802885in}}%
\pgfpathcurveto{\pgfqpoint{2.100021in}{0.813935in}}{\pgfqpoint{2.095630in}{0.824534in}}{\pgfqpoint{2.087817in}{0.832348in}}%
\pgfpathcurveto{\pgfqpoint{2.080003in}{0.840162in}}{\pgfqpoint{2.069404in}{0.844552in}}{\pgfqpoint{2.058354in}{0.844552in}}%
\pgfpathcurveto{\pgfqpoint{2.047304in}{0.844552in}}{\pgfqpoint{2.036705in}{0.840162in}}{\pgfqpoint{2.028891in}{0.832348in}}%
\pgfpathcurveto{\pgfqpoint{2.021078in}{0.824534in}}{\pgfqpoint{2.016687in}{0.813935in}}{\pgfqpoint{2.016687in}{0.802885in}}%
\pgfpathcurveto{\pgfqpoint{2.016687in}{0.791835in}}{\pgfqpoint{2.021078in}{0.781236in}}{\pgfqpoint{2.028891in}{0.773422in}}%
\pgfpathcurveto{\pgfqpoint{2.036705in}{0.765609in}}{\pgfqpoint{2.047304in}{0.761218in}}{\pgfqpoint{2.058354in}{0.761218in}}%
\pgfpathclose%
\pgfusepath{stroke,fill}%
\end{pgfscope}%
\begin{pgfscope}%
\pgfpathrectangle{\pgfqpoint{0.481978in}{0.331635in}}{\pgfqpoint{4.960000in}{3.696000in}}%
\pgfusepath{clip}%
\pgfsetbuttcap%
\pgfsetroundjoin%
\definecolor{currentfill}{rgb}{0.631373,0.788235,0.956863}%
\pgfsetfillcolor{currentfill}%
\pgfsetlinewidth{0.481800pt}%
\definecolor{currentstroke}{rgb}{1.000000,1.000000,1.000000}%
\pgfsetstrokecolor{currentstroke}%
\pgfsetdash{}{0pt}%
\pgfpathmoveto{\pgfqpoint{3.476728in}{2.169910in}}%
\pgfpathcurveto{\pgfqpoint{3.487778in}{2.169910in}}{\pgfqpoint{3.498377in}{2.174300in}}{\pgfqpoint{3.506191in}{2.182114in}}%
\pgfpathcurveto{\pgfqpoint{3.514004in}{2.189927in}}{\pgfqpoint{3.518395in}{2.200527in}}{\pgfqpoint{3.518395in}{2.211577in}}%
\pgfpathcurveto{\pgfqpoint{3.518395in}{2.222627in}}{\pgfqpoint{3.514004in}{2.233226in}}{\pgfqpoint{3.506191in}{2.241039in}}%
\pgfpathcurveto{\pgfqpoint{3.498377in}{2.248853in}}{\pgfqpoint{3.487778in}{2.253243in}}{\pgfqpoint{3.476728in}{2.253243in}}%
\pgfpathcurveto{\pgfqpoint{3.465678in}{2.253243in}}{\pgfqpoint{3.455079in}{2.248853in}}{\pgfqpoint{3.447265in}{2.241039in}}%
\pgfpathcurveto{\pgfqpoint{3.439452in}{2.233226in}}{\pgfqpoint{3.435061in}{2.222627in}}{\pgfqpoint{3.435061in}{2.211577in}}%
\pgfpathcurveto{\pgfqpoint{3.435061in}{2.200527in}}{\pgfqpoint{3.439452in}{2.189927in}}{\pgfqpoint{3.447265in}{2.182114in}}%
\pgfpathcurveto{\pgfqpoint{3.455079in}{2.174300in}}{\pgfqpoint{3.465678in}{2.169910in}}{\pgfqpoint{3.476728in}{2.169910in}}%
\pgfpathclose%
\pgfusepath{stroke,fill}%
\end{pgfscope}%
\begin{pgfscope}%
\pgfpathrectangle{\pgfqpoint{0.481978in}{0.331635in}}{\pgfqpoint{4.960000in}{3.696000in}}%
\pgfusepath{clip}%
\pgfsetbuttcap%
\pgfsetroundjoin%
\definecolor{currentfill}{rgb}{0.631373,0.788235,0.956863}%
\pgfsetfillcolor{currentfill}%
\pgfsetlinewidth{0.481800pt}%
\definecolor{currentstroke}{rgb}{1.000000,1.000000,1.000000}%
\pgfsetstrokecolor{currentstroke}%
\pgfsetdash{}{0pt}%
\pgfpathmoveto{\pgfqpoint{4.137524in}{2.435310in}}%
\pgfpathcurveto{\pgfqpoint{4.148574in}{2.435310in}}{\pgfqpoint{4.159173in}{2.439700in}}{\pgfqpoint{4.166987in}{2.447513in}}%
\pgfpathcurveto{\pgfqpoint{4.174800in}{2.455327in}}{\pgfqpoint{4.179191in}{2.465926in}}{\pgfqpoint{4.179191in}{2.476976in}}%
\pgfpathcurveto{\pgfqpoint{4.179191in}{2.488026in}}{\pgfqpoint{4.174800in}{2.498625in}}{\pgfqpoint{4.166987in}{2.506439in}}%
\pgfpathcurveto{\pgfqpoint{4.159173in}{2.514253in}}{\pgfqpoint{4.148574in}{2.518643in}}{\pgfqpoint{4.137524in}{2.518643in}}%
\pgfpathcurveto{\pgfqpoint{4.126474in}{2.518643in}}{\pgfqpoint{4.115875in}{2.514253in}}{\pgfqpoint{4.108061in}{2.506439in}}%
\pgfpathcurveto{\pgfqpoint{4.100247in}{2.498625in}}{\pgfqpoint{4.095857in}{2.488026in}}{\pgfqpoint{4.095857in}{2.476976in}}%
\pgfpathcurveto{\pgfqpoint{4.095857in}{2.465926in}}{\pgfqpoint{4.100247in}{2.455327in}}{\pgfqpoint{4.108061in}{2.447513in}}%
\pgfpathcurveto{\pgfqpoint{4.115875in}{2.439700in}}{\pgfqpoint{4.126474in}{2.435310in}}{\pgfqpoint{4.137524in}{2.435310in}}%
\pgfpathclose%
\pgfusepath{stroke,fill}%
\end{pgfscope}%
\begin{pgfscope}%
\pgfpathrectangle{\pgfqpoint{0.481978in}{0.331635in}}{\pgfqpoint{4.960000in}{3.696000in}}%
\pgfusepath{clip}%
\pgfsetbuttcap%
\pgfsetroundjoin%
\definecolor{currentfill}{rgb}{0.631373,0.788235,0.956863}%
\pgfsetfillcolor{currentfill}%
\pgfsetlinewidth{0.481800pt}%
\definecolor{currentstroke}{rgb}{1.000000,1.000000,1.000000}%
\pgfsetstrokecolor{currentstroke}%
\pgfsetdash{}{0pt}%
\pgfpathmoveto{\pgfqpoint{4.244000in}{2.719951in}}%
\pgfpathcurveto{\pgfqpoint{4.255050in}{2.719951in}}{\pgfqpoint{4.265649in}{2.724341in}}{\pgfqpoint{4.273463in}{2.732155in}}%
\pgfpathcurveto{\pgfqpoint{4.281277in}{2.739969in}}{\pgfqpoint{4.285667in}{2.750568in}}{\pgfqpoint{4.285667in}{2.761618in}}%
\pgfpathcurveto{\pgfqpoint{4.285667in}{2.772668in}}{\pgfqpoint{4.281277in}{2.783267in}}{\pgfqpoint{4.273463in}{2.791081in}}%
\pgfpathcurveto{\pgfqpoint{4.265649in}{2.798894in}}{\pgfqpoint{4.255050in}{2.803284in}}{\pgfqpoint{4.244000in}{2.803284in}}%
\pgfpathcurveto{\pgfqpoint{4.232950in}{2.803284in}}{\pgfqpoint{4.222351in}{2.798894in}}{\pgfqpoint{4.214537in}{2.791081in}}%
\pgfpathcurveto{\pgfqpoint{4.206724in}{2.783267in}}{\pgfqpoint{4.202333in}{2.772668in}}{\pgfqpoint{4.202333in}{2.761618in}}%
\pgfpathcurveto{\pgfqpoint{4.202333in}{2.750568in}}{\pgfqpoint{4.206724in}{2.739969in}}{\pgfqpoint{4.214537in}{2.732155in}}%
\pgfpathcurveto{\pgfqpoint{4.222351in}{2.724341in}}{\pgfqpoint{4.232950in}{2.719951in}}{\pgfqpoint{4.244000in}{2.719951in}}%
\pgfpathclose%
\pgfusepath{stroke,fill}%
\end{pgfscope}%
\begin{pgfscope}%
\pgfpathrectangle{\pgfqpoint{0.481978in}{0.331635in}}{\pgfqpoint{4.960000in}{3.696000in}}%
\pgfusepath{clip}%
\pgfsetbuttcap%
\pgfsetroundjoin%
\definecolor{currentfill}{rgb}{0.631373,0.788235,0.956863}%
\pgfsetfillcolor{currentfill}%
\pgfsetlinewidth{0.481800pt}%
\definecolor{currentstroke}{rgb}{1.000000,1.000000,1.000000}%
\pgfsetstrokecolor{currentstroke}%
\pgfsetdash{}{0pt}%
\pgfpathmoveto{\pgfqpoint{1.590748in}{1.232333in}}%
\pgfpathcurveto{\pgfqpoint{1.601798in}{1.232333in}}{\pgfqpoint{1.612397in}{1.236724in}}{\pgfqpoint{1.620211in}{1.244537in}}%
\pgfpathcurveto{\pgfqpoint{1.628025in}{1.252351in}}{\pgfqpoint{1.632415in}{1.262950in}}{\pgfqpoint{1.632415in}{1.274000in}}%
\pgfpathcurveto{\pgfqpoint{1.632415in}{1.285050in}}{\pgfqpoint{1.628025in}{1.295649in}}{\pgfqpoint{1.620211in}{1.303463in}}%
\pgfpathcurveto{\pgfqpoint{1.612397in}{1.311277in}}{\pgfqpoint{1.601798in}{1.315667in}}{\pgfqpoint{1.590748in}{1.315667in}}%
\pgfpathcurveto{\pgfqpoint{1.579698in}{1.315667in}}{\pgfqpoint{1.569099in}{1.311277in}}{\pgfqpoint{1.561285in}{1.303463in}}%
\pgfpathcurveto{\pgfqpoint{1.553472in}{1.295649in}}{\pgfqpoint{1.549082in}{1.285050in}}{\pgfqpoint{1.549082in}{1.274000in}}%
\pgfpathcurveto{\pgfqpoint{1.549082in}{1.262950in}}{\pgfqpoint{1.553472in}{1.252351in}}{\pgfqpoint{1.561285in}{1.244537in}}%
\pgfpathcurveto{\pgfqpoint{1.569099in}{1.236724in}}{\pgfqpoint{1.579698in}{1.232333in}}{\pgfqpoint{1.590748in}{1.232333in}}%
\pgfpathclose%
\pgfusepath{stroke,fill}%
\end{pgfscope}%
\begin{pgfscope}%
\pgfpathrectangle{\pgfqpoint{0.481978in}{0.331635in}}{\pgfqpoint{4.960000in}{3.696000in}}%
\pgfusepath{clip}%
\pgfsetbuttcap%
\pgfsetroundjoin%
\definecolor{currentfill}{rgb}{0.631373,0.788235,0.956863}%
\pgfsetfillcolor{currentfill}%
\pgfsetlinewidth{0.481800pt}%
\definecolor{currentstroke}{rgb}{1.000000,1.000000,1.000000}%
\pgfsetstrokecolor{currentstroke}%
\pgfsetdash{}{0pt}%
\pgfpathmoveto{\pgfqpoint{4.029524in}{1.336909in}}%
\pgfpathcurveto{\pgfqpoint{4.040574in}{1.336909in}}{\pgfqpoint{4.051173in}{1.341299in}}{\pgfqpoint{4.058987in}{1.349113in}}%
\pgfpathcurveto{\pgfqpoint{4.066800in}{1.356927in}}{\pgfqpoint{4.071191in}{1.367526in}}{\pgfqpoint{4.071191in}{1.378576in}}%
\pgfpathcurveto{\pgfqpoint{4.071191in}{1.389626in}}{\pgfqpoint{4.066800in}{1.400225in}}{\pgfqpoint{4.058987in}{1.408039in}}%
\pgfpathcurveto{\pgfqpoint{4.051173in}{1.415852in}}{\pgfqpoint{4.040574in}{1.420243in}}{\pgfqpoint{4.029524in}{1.420243in}}%
\pgfpathcurveto{\pgfqpoint{4.018474in}{1.420243in}}{\pgfqpoint{4.007875in}{1.415852in}}{\pgfqpoint{4.000061in}{1.408039in}}%
\pgfpathcurveto{\pgfqpoint{3.992247in}{1.400225in}}{\pgfqpoint{3.987857in}{1.389626in}}{\pgfqpoint{3.987857in}{1.378576in}}%
\pgfpathcurveto{\pgfqpoint{3.987857in}{1.367526in}}{\pgfqpoint{3.992247in}{1.356927in}}{\pgfqpoint{4.000061in}{1.349113in}}%
\pgfpathcurveto{\pgfqpoint{4.007875in}{1.341299in}}{\pgfqpoint{4.018474in}{1.336909in}}{\pgfqpoint{4.029524in}{1.336909in}}%
\pgfpathclose%
\pgfusepath{stroke,fill}%
\end{pgfscope}%
\begin{pgfscope}%
\pgfpathrectangle{\pgfqpoint{0.481978in}{0.331635in}}{\pgfqpoint{4.960000in}{3.696000in}}%
\pgfusepath{clip}%
\pgfsetbuttcap%
\pgfsetroundjoin%
\definecolor{currentfill}{rgb}{0.631373,0.788235,0.956863}%
\pgfsetfillcolor{currentfill}%
\pgfsetlinewidth{0.481800pt}%
\definecolor{currentstroke}{rgb}{1.000000,1.000000,1.000000}%
\pgfsetstrokecolor{currentstroke}%
\pgfsetdash{}{0pt}%
\pgfpathmoveto{\pgfqpoint{3.280459in}{2.179369in}}%
\pgfpathcurveto{\pgfqpoint{3.291509in}{2.179369in}}{\pgfqpoint{3.302108in}{2.183759in}}{\pgfqpoint{3.309922in}{2.191572in}}%
\pgfpathcurveto{\pgfqpoint{3.317735in}{2.199386in}}{\pgfqpoint{3.322126in}{2.209985in}}{\pgfqpoint{3.322126in}{2.221035in}}%
\pgfpathcurveto{\pgfqpoint{3.322126in}{2.232085in}}{\pgfqpoint{3.317735in}{2.242684in}}{\pgfqpoint{3.309922in}{2.250498in}}%
\pgfpathcurveto{\pgfqpoint{3.302108in}{2.258312in}}{\pgfqpoint{3.291509in}{2.262702in}}{\pgfqpoint{3.280459in}{2.262702in}}%
\pgfpathcurveto{\pgfqpoint{3.269409in}{2.262702in}}{\pgfqpoint{3.258810in}{2.258312in}}{\pgfqpoint{3.250996in}{2.250498in}}%
\pgfpathcurveto{\pgfqpoint{3.243183in}{2.242684in}}{\pgfqpoint{3.238792in}{2.232085in}}{\pgfqpoint{3.238792in}{2.221035in}}%
\pgfpathcurveto{\pgfqpoint{3.238792in}{2.209985in}}{\pgfqpoint{3.243183in}{2.199386in}}{\pgfqpoint{3.250996in}{2.191572in}}%
\pgfpathcurveto{\pgfqpoint{3.258810in}{2.183759in}}{\pgfqpoint{3.269409in}{2.179369in}}{\pgfqpoint{3.280459in}{2.179369in}}%
\pgfpathclose%
\pgfusepath{stroke,fill}%
\end{pgfscope}%
\begin{pgfscope}%
\pgfpathrectangle{\pgfqpoint{0.481978in}{0.331635in}}{\pgfqpoint{4.960000in}{3.696000in}}%
\pgfusepath{clip}%
\pgfsetbuttcap%
\pgfsetroundjoin%
\definecolor{currentfill}{rgb}{1.000000,0.705882,0.509804}%
\pgfsetfillcolor{currentfill}%
\pgfsetlinewidth{1.003750pt}%
\definecolor{currentstroke}{rgb}{1.000000,0.705882,0.509804}%
\pgfsetstrokecolor{currentstroke}%
\pgfsetdash{}{0pt}%
\pgfsys@defobject{currentmarker}{\pgfqpoint{-0.041667in}{-0.041667in}}{\pgfqpoint{0.041667in}{0.041667in}}{%
\pgfpathmoveto{\pgfqpoint{0.000000in}{-0.041667in}}%
\pgfpathcurveto{\pgfqpoint{0.011050in}{-0.041667in}}{\pgfqpoint{0.021649in}{-0.037276in}}{\pgfqpoint{0.029463in}{-0.029463in}}%
\pgfpathcurveto{\pgfqpoint{0.037276in}{-0.021649in}}{\pgfqpoint{0.041667in}{-0.011050in}}{\pgfqpoint{0.041667in}{0.000000in}}%
\pgfpathcurveto{\pgfqpoint{0.041667in}{0.011050in}}{\pgfqpoint{0.037276in}{0.021649in}}{\pgfqpoint{0.029463in}{0.029463in}}%
\pgfpathcurveto{\pgfqpoint{0.021649in}{0.037276in}}{\pgfqpoint{0.011050in}{0.041667in}}{\pgfqpoint{0.000000in}{0.041667in}}%
\pgfpathcurveto{\pgfqpoint{-0.011050in}{0.041667in}}{\pgfqpoint{-0.021649in}{0.037276in}}{\pgfqpoint{-0.029463in}{0.029463in}}%
\pgfpathcurveto{\pgfqpoint{-0.037276in}{0.021649in}}{\pgfqpoint{-0.041667in}{0.011050in}}{\pgfqpoint{-0.041667in}{0.000000in}}%
\pgfpathcurveto{\pgfqpoint{-0.041667in}{-0.011050in}}{\pgfqpoint{-0.037276in}{-0.021649in}}{\pgfqpoint{-0.029463in}{-0.029463in}}%
\pgfpathcurveto{\pgfqpoint{-0.021649in}{-0.037276in}}{\pgfqpoint{-0.011050in}{-0.041667in}}{\pgfqpoint{0.000000in}{-0.041667in}}%
\pgfpathclose%
\pgfusepath{stroke,fill}%
}%
\end{pgfscope}%
\begin{pgfscope}%
\pgfpathrectangle{\pgfqpoint{0.481978in}{0.331635in}}{\pgfqpoint{4.960000in}{3.696000in}}%
\pgfusepath{clip}%
\pgfsetbuttcap%
\pgfsetroundjoin%
\definecolor{currentfill}{rgb}{0.631373,0.788235,0.956863}%
\pgfsetfillcolor{currentfill}%
\pgfsetlinewidth{1.003750pt}%
\definecolor{currentstroke}{rgb}{0.631373,0.788235,0.956863}%
\pgfsetstrokecolor{currentstroke}%
\pgfsetdash{}{0pt}%
\pgfsys@defobject{currentmarker}{\pgfqpoint{-0.041667in}{-0.041667in}}{\pgfqpoint{0.041667in}{0.041667in}}{%
\pgfpathmoveto{\pgfqpoint{0.000000in}{-0.041667in}}%
\pgfpathcurveto{\pgfqpoint{0.011050in}{-0.041667in}}{\pgfqpoint{0.021649in}{-0.037276in}}{\pgfqpoint{0.029463in}{-0.029463in}}%
\pgfpathcurveto{\pgfqpoint{0.037276in}{-0.021649in}}{\pgfqpoint{0.041667in}{-0.011050in}}{\pgfqpoint{0.041667in}{0.000000in}}%
\pgfpathcurveto{\pgfqpoint{0.041667in}{0.011050in}}{\pgfqpoint{0.037276in}{0.021649in}}{\pgfqpoint{0.029463in}{0.029463in}}%
\pgfpathcurveto{\pgfqpoint{0.021649in}{0.037276in}}{\pgfqpoint{0.011050in}{0.041667in}}{\pgfqpoint{0.000000in}{0.041667in}}%
\pgfpathcurveto{\pgfqpoint{-0.011050in}{0.041667in}}{\pgfqpoint{-0.021649in}{0.037276in}}{\pgfqpoint{-0.029463in}{0.029463in}}%
\pgfpathcurveto{\pgfqpoint{-0.037276in}{0.021649in}}{\pgfqpoint{-0.041667in}{0.011050in}}{\pgfqpoint{-0.041667in}{0.000000in}}%
\pgfpathcurveto{\pgfqpoint{-0.041667in}{-0.011050in}}{\pgfqpoint{-0.037276in}{-0.021649in}}{\pgfqpoint{-0.029463in}{-0.029463in}}%
\pgfpathcurveto{\pgfqpoint{-0.021649in}{-0.037276in}}{\pgfqpoint{-0.011050in}{-0.041667in}}{\pgfqpoint{0.000000in}{-0.041667in}}%
\pgfpathclose%
\pgfusepath{stroke,fill}%
}%
\end{pgfscope}%
\begin{pgfscope}%
\pgfsetbuttcap%
\pgfsetroundjoin%
\definecolor{currentfill}{rgb}{0.000000,0.000000,0.000000}%
\pgfsetfillcolor{currentfill}%
\pgfsetlinewidth{0.803000pt}%
\definecolor{currentstroke}{rgb}{0.000000,0.000000,0.000000}%
\pgfsetstrokecolor{currentstroke}%
\pgfsetdash{}{0pt}%
\pgfsys@defobject{currentmarker}{\pgfqpoint{0.000000in}{-0.048611in}}{\pgfqpoint{0.000000in}{0.000000in}}{%
\pgfpathmoveto{\pgfqpoint{0.000000in}{0.000000in}}%
\pgfpathlineto{\pgfqpoint{0.000000in}{-0.048611in}}%
\pgfusepath{stroke,fill}%
}%
\begin{pgfscope}%
\pgfsys@transformshift{0.599430in}{0.331635in}%
\pgfsys@useobject{currentmarker}{}%
\end{pgfscope}%
\end{pgfscope}%
\begin{pgfscope}%
\definecolor{textcolor}{rgb}{0.000000,0.000000,0.000000}%
\pgfsetstrokecolor{textcolor}%
\pgfsetfillcolor{textcolor}%
\pgftext[x=0.599430in,y=0.234413in,,top]{\color{textcolor}\sffamily\fontsize{10.000000}{12.000000}\selectfont \ensuremath{-}15}%
\end{pgfscope}%
\begin{pgfscope}%
\pgfsetbuttcap%
\pgfsetroundjoin%
\definecolor{currentfill}{rgb}{0.000000,0.000000,0.000000}%
\pgfsetfillcolor{currentfill}%
\pgfsetlinewidth{0.803000pt}%
\definecolor{currentstroke}{rgb}{0.000000,0.000000,0.000000}%
\pgfsetstrokecolor{currentstroke}%
\pgfsetdash{}{0pt}%
\pgfsys@defobject{currentmarker}{\pgfqpoint{0.000000in}{-0.048611in}}{\pgfqpoint{0.000000in}{0.000000in}}{%
\pgfpathmoveto{\pgfqpoint{0.000000in}{0.000000in}}%
\pgfpathlineto{\pgfqpoint{0.000000in}{-0.048611in}}%
\pgfusepath{stroke,fill}%
}%
\begin{pgfscope}%
\pgfsys@transformshift{1.481736in}{0.331635in}%
\pgfsys@useobject{currentmarker}{}%
\end{pgfscope}%
\end{pgfscope}%
\begin{pgfscope}%
\definecolor{textcolor}{rgb}{0.000000,0.000000,0.000000}%
\pgfsetstrokecolor{textcolor}%
\pgfsetfillcolor{textcolor}%
\pgftext[x=1.481736in,y=0.234413in,,top]{\color{textcolor}\sffamily\fontsize{10.000000}{12.000000}\selectfont \ensuremath{-}10}%
\end{pgfscope}%
\begin{pgfscope}%
\pgfsetbuttcap%
\pgfsetroundjoin%
\definecolor{currentfill}{rgb}{0.000000,0.000000,0.000000}%
\pgfsetfillcolor{currentfill}%
\pgfsetlinewidth{0.803000pt}%
\definecolor{currentstroke}{rgb}{0.000000,0.000000,0.000000}%
\pgfsetstrokecolor{currentstroke}%
\pgfsetdash{}{0pt}%
\pgfsys@defobject{currentmarker}{\pgfqpoint{0.000000in}{-0.048611in}}{\pgfqpoint{0.000000in}{0.000000in}}{%
\pgfpathmoveto{\pgfqpoint{0.000000in}{0.000000in}}%
\pgfpathlineto{\pgfqpoint{0.000000in}{-0.048611in}}%
\pgfusepath{stroke,fill}%
}%
\begin{pgfscope}%
\pgfsys@transformshift{2.364041in}{0.331635in}%
\pgfsys@useobject{currentmarker}{}%
\end{pgfscope}%
\end{pgfscope}%
\begin{pgfscope}%
\definecolor{textcolor}{rgb}{0.000000,0.000000,0.000000}%
\pgfsetstrokecolor{textcolor}%
\pgfsetfillcolor{textcolor}%
\pgftext[x=2.364041in,y=0.234413in,,top]{\color{textcolor}\sffamily\fontsize{10.000000}{12.000000}\selectfont \ensuremath{-}5}%
\end{pgfscope}%
\begin{pgfscope}%
\pgfsetbuttcap%
\pgfsetroundjoin%
\definecolor{currentfill}{rgb}{0.000000,0.000000,0.000000}%
\pgfsetfillcolor{currentfill}%
\pgfsetlinewidth{0.803000pt}%
\definecolor{currentstroke}{rgb}{0.000000,0.000000,0.000000}%
\pgfsetstrokecolor{currentstroke}%
\pgfsetdash{}{0pt}%
\pgfsys@defobject{currentmarker}{\pgfqpoint{0.000000in}{-0.048611in}}{\pgfqpoint{0.000000in}{0.000000in}}{%
\pgfpathmoveto{\pgfqpoint{0.000000in}{0.000000in}}%
\pgfpathlineto{\pgfqpoint{0.000000in}{-0.048611in}}%
\pgfusepath{stroke,fill}%
}%
\begin{pgfscope}%
\pgfsys@transformshift{3.246346in}{0.331635in}%
\pgfsys@useobject{currentmarker}{}%
\end{pgfscope}%
\end{pgfscope}%
\begin{pgfscope}%
\definecolor{textcolor}{rgb}{0.000000,0.000000,0.000000}%
\pgfsetstrokecolor{textcolor}%
\pgfsetfillcolor{textcolor}%
\pgftext[x=3.246346in,y=0.234413in,,top]{\color{textcolor}\sffamily\fontsize{10.000000}{12.000000}\selectfont 0}%
\end{pgfscope}%
\begin{pgfscope}%
\pgfsetbuttcap%
\pgfsetroundjoin%
\definecolor{currentfill}{rgb}{0.000000,0.000000,0.000000}%
\pgfsetfillcolor{currentfill}%
\pgfsetlinewidth{0.803000pt}%
\definecolor{currentstroke}{rgb}{0.000000,0.000000,0.000000}%
\pgfsetstrokecolor{currentstroke}%
\pgfsetdash{}{0pt}%
\pgfsys@defobject{currentmarker}{\pgfqpoint{0.000000in}{-0.048611in}}{\pgfqpoint{0.000000in}{0.000000in}}{%
\pgfpathmoveto{\pgfqpoint{0.000000in}{0.000000in}}%
\pgfpathlineto{\pgfqpoint{0.000000in}{-0.048611in}}%
\pgfusepath{stroke,fill}%
}%
\begin{pgfscope}%
\pgfsys@transformshift{4.128651in}{0.331635in}%
\pgfsys@useobject{currentmarker}{}%
\end{pgfscope}%
\end{pgfscope}%
\begin{pgfscope}%
\definecolor{textcolor}{rgb}{0.000000,0.000000,0.000000}%
\pgfsetstrokecolor{textcolor}%
\pgfsetfillcolor{textcolor}%
\pgftext[x=4.128651in,y=0.234413in,,top]{\color{textcolor}\sffamily\fontsize{10.000000}{12.000000}\selectfont 5}%
\end{pgfscope}%
\begin{pgfscope}%
\pgfsetbuttcap%
\pgfsetroundjoin%
\definecolor{currentfill}{rgb}{0.000000,0.000000,0.000000}%
\pgfsetfillcolor{currentfill}%
\pgfsetlinewidth{0.803000pt}%
\definecolor{currentstroke}{rgb}{0.000000,0.000000,0.000000}%
\pgfsetstrokecolor{currentstroke}%
\pgfsetdash{}{0pt}%
\pgfsys@defobject{currentmarker}{\pgfqpoint{0.000000in}{-0.048611in}}{\pgfqpoint{0.000000in}{0.000000in}}{%
\pgfpathmoveto{\pgfqpoint{0.000000in}{0.000000in}}%
\pgfpathlineto{\pgfqpoint{0.000000in}{-0.048611in}}%
\pgfusepath{stroke,fill}%
}%
\begin{pgfscope}%
\pgfsys@transformshift{5.010957in}{0.331635in}%
\pgfsys@useobject{currentmarker}{}%
\end{pgfscope}%
\end{pgfscope}%
\begin{pgfscope}%
\definecolor{textcolor}{rgb}{0.000000,0.000000,0.000000}%
\pgfsetstrokecolor{textcolor}%
\pgfsetfillcolor{textcolor}%
\pgftext[x=5.010957in,y=0.234413in,,top]{\color{textcolor}\sffamily\fontsize{10.000000}{12.000000}\selectfont 10}%
\end{pgfscope}%
\begin{pgfscope}%
\pgfsetbuttcap%
\pgfsetroundjoin%
\definecolor{currentfill}{rgb}{0.000000,0.000000,0.000000}%
\pgfsetfillcolor{currentfill}%
\pgfsetlinewidth{0.803000pt}%
\definecolor{currentstroke}{rgb}{0.000000,0.000000,0.000000}%
\pgfsetstrokecolor{currentstroke}%
\pgfsetdash{}{0pt}%
\pgfsys@defobject{currentmarker}{\pgfqpoint{-0.048611in}{0.000000in}}{\pgfqpoint{-0.000000in}{0.000000in}}{%
\pgfpathmoveto{\pgfqpoint{-0.000000in}{0.000000in}}%
\pgfpathlineto{\pgfqpoint{-0.048611in}{0.000000in}}%
\pgfusepath{stroke,fill}%
}%
\begin{pgfscope}%
\pgfsys@transformshift{0.481978in}{0.903863in}%
\pgfsys@useobject{currentmarker}{}%
\end{pgfscope}%
\end{pgfscope}%
\begin{pgfscope}%
\definecolor{textcolor}{rgb}{0.000000,0.000000,0.000000}%
\pgfsetstrokecolor{textcolor}%
\pgfsetfillcolor{textcolor}%
\pgftext[x=0.100000in, y=0.851101in, left, base]{\color{textcolor}\sffamily\fontsize{10.000000}{12.000000}\selectfont \ensuremath{-}15}%
\end{pgfscope}%
\begin{pgfscope}%
\pgfsetbuttcap%
\pgfsetroundjoin%
\definecolor{currentfill}{rgb}{0.000000,0.000000,0.000000}%
\pgfsetfillcolor{currentfill}%
\pgfsetlinewidth{0.803000pt}%
\definecolor{currentstroke}{rgb}{0.000000,0.000000,0.000000}%
\pgfsetstrokecolor{currentstroke}%
\pgfsetdash{}{0pt}%
\pgfsys@defobject{currentmarker}{\pgfqpoint{-0.048611in}{0.000000in}}{\pgfqpoint{-0.000000in}{0.000000in}}{%
\pgfpathmoveto{\pgfqpoint{-0.000000in}{0.000000in}}%
\pgfpathlineto{\pgfqpoint{-0.048611in}{0.000000in}}%
\pgfusepath{stroke,fill}%
}%
\begin{pgfscope}%
\pgfsys@transformshift{0.481978in}{1.532148in}%
\pgfsys@useobject{currentmarker}{}%
\end{pgfscope}%
\end{pgfscope}%
\begin{pgfscope}%
\definecolor{textcolor}{rgb}{0.000000,0.000000,0.000000}%
\pgfsetstrokecolor{textcolor}%
\pgfsetfillcolor{textcolor}%
\pgftext[x=0.100000in, y=1.479386in, left, base]{\color{textcolor}\sffamily\fontsize{10.000000}{12.000000}\selectfont \ensuremath{-}10}%
\end{pgfscope}%
\begin{pgfscope}%
\pgfsetbuttcap%
\pgfsetroundjoin%
\definecolor{currentfill}{rgb}{0.000000,0.000000,0.000000}%
\pgfsetfillcolor{currentfill}%
\pgfsetlinewidth{0.803000pt}%
\definecolor{currentstroke}{rgb}{0.000000,0.000000,0.000000}%
\pgfsetstrokecolor{currentstroke}%
\pgfsetdash{}{0pt}%
\pgfsys@defobject{currentmarker}{\pgfqpoint{-0.048611in}{0.000000in}}{\pgfqpoint{-0.000000in}{0.000000in}}{%
\pgfpathmoveto{\pgfqpoint{-0.000000in}{0.000000in}}%
\pgfpathlineto{\pgfqpoint{-0.048611in}{0.000000in}}%
\pgfusepath{stroke,fill}%
}%
\begin{pgfscope}%
\pgfsys@transformshift{0.481978in}{2.160432in}%
\pgfsys@useobject{currentmarker}{}%
\end{pgfscope}%
\end{pgfscope}%
\begin{pgfscope}%
\definecolor{textcolor}{rgb}{0.000000,0.000000,0.000000}%
\pgfsetstrokecolor{textcolor}%
\pgfsetfillcolor{textcolor}%
\pgftext[x=0.188365in, y=2.107671in, left, base]{\color{textcolor}\sffamily\fontsize{10.000000}{12.000000}\selectfont \ensuremath{-}5}%
\end{pgfscope}%
\begin{pgfscope}%
\pgfsetbuttcap%
\pgfsetroundjoin%
\definecolor{currentfill}{rgb}{0.000000,0.000000,0.000000}%
\pgfsetfillcolor{currentfill}%
\pgfsetlinewidth{0.803000pt}%
\definecolor{currentstroke}{rgb}{0.000000,0.000000,0.000000}%
\pgfsetstrokecolor{currentstroke}%
\pgfsetdash{}{0pt}%
\pgfsys@defobject{currentmarker}{\pgfqpoint{-0.048611in}{0.000000in}}{\pgfqpoint{-0.000000in}{0.000000in}}{%
\pgfpathmoveto{\pgfqpoint{-0.000000in}{0.000000in}}%
\pgfpathlineto{\pgfqpoint{-0.048611in}{0.000000in}}%
\pgfusepath{stroke,fill}%
}%
\begin{pgfscope}%
\pgfsys@transformshift{0.481978in}{2.788717in}%
\pgfsys@useobject{currentmarker}{}%
\end{pgfscope}%
\end{pgfscope}%
\begin{pgfscope}%
\definecolor{textcolor}{rgb}{0.000000,0.000000,0.000000}%
\pgfsetstrokecolor{textcolor}%
\pgfsetfillcolor{textcolor}%
\pgftext[x=0.296390in, y=2.735956in, left, base]{\color{textcolor}\sffamily\fontsize{10.000000}{12.000000}\selectfont 0}%
\end{pgfscope}%
\begin{pgfscope}%
\pgfsetbuttcap%
\pgfsetroundjoin%
\definecolor{currentfill}{rgb}{0.000000,0.000000,0.000000}%
\pgfsetfillcolor{currentfill}%
\pgfsetlinewidth{0.803000pt}%
\definecolor{currentstroke}{rgb}{0.000000,0.000000,0.000000}%
\pgfsetstrokecolor{currentstroke}%
\pgfsetdash{}{0pt}%
\pgfsys@defobject{currentmarker}{\pgfqpoint{-0.048611in}{0.000000in}}{\pgfqpoint{-0.000000in}{0.000000in}}{%
\pgfpathmoveto{\pgfqpoint{-0.000000in}{0.000000in}}%
\pgfpathlineto{\pgfqpoint{-0.048611in}{0.000000in}}%
\pgfusepath{stroke,fill}%
}%
\begin{pgfscope}%
\pgfsys@transformshift{0.481978in}{3.417002in}%
\pgfsys@useobject{currentmarker}{}%
\end{pgfscope}%
\end{pgfscope}%
\begin{pgfscope}%
\definecolor{textcolor}{rgb}{0.000000,0.000000,0.000000}%
\pgfsetstrokecolor{textcolor}%
\pgfsetfillcolor{textcolor}%
\pgftext[x=0.296390in, y=3.364241in, left, base]{\color{textcolor}\sffamily\fontsize{10.000000}{12.000000}\selectfont 5}%
\end{pgfscope}%
\begin{pgfscope}%
\pgfpathrectangle{\pgfqpoint{0.481978in}{0.331635in}}{\pgfqpoint{4.960000in}{3.696000in}}%
\pgfusepath{clip}%
\pgfsetrectcap%
\pgfsetroundjoin%
\pgfsetlinewidth{1.505625pt}%
\definecolor{currentstroke}{rgb}{1.000000,0.705882,0.509804}%
\pgfsetstrokecolor{currentstroke}%
\pgfsetstrokeopacity{0.800000}%
\pgfsetdash{}{0pt}%
\pgfpathmoveto{\pgfqpoint{1.904809in}{1.383360in}}%
\pgfpathlineto{\pgfqpoint{2.337997in}{2.565179in}}%
\pgfusepath{stroke}%
\end{pgfscope}%
\begin{pgfscope}%
\pgfpathrectangle{\pgfqpoint{0.481978in}{0.331635in}}{\pgfqpoint{4.960000in}{3.696000in}}%
\pgfusepath{clip}%
\pgfsetrectcap%
\pgfsetroundjoin%
\pgfsetlinewidth{1.505625pt}%
\definecolor{currentstroke}{rgb}{1.000000,0.705882,0.509804}%
\pgfsetstrokecolor{currentstroke}%
\pgfsetstrokeopacity{0.800000}%
\pgfsetdash{}{0pt}%
\pgfpathmoveto{\pgfqpoint{3.073685in}{3.510858in}}%
\pgfpathlineto{\pgfqpoint{2.337997in}{2.565179in}}%
\pgfusepath{stroke}%
\end{pgfscope}%
\begin{pgfscope}%
\pgfpathrectangle{\pgfqpoint{0.481978in}{0.331635in}}{\pgfqpoint{4.960000in}{3.696000in}}%
\pgfusepath{clip}%
\pgfsetrectcap%
\pgfsetroundjoin%
\pgfsetlinewidth{1.505625pt}%
\definecolor{currentstroke}{rgb}{1.000000,0.705882,0.509804}%
\pgfsetstrokecolor{currentstroke}%
\pgfsetstrokeopacity{0.800000}%
\pgfsetdash{}{0pt}%
\pgfpathmoveto{\pgfqpoint{3.119479in}{3.475206in}}%
\pgfpathlineto{\pgfqpoint{2.337997in}{2.565179in}}%
\pgfusepath{stroke}%
\end{pgfscope}%
\begin{pgfscope}%
\pgfpathrectangle{\pgfqpoint{0.481978in}{0.331635in}}{\pgfqpoint{4.960000in}{3.696000in}}%
\pgfusepath{clip}%
\pgfsetrectcap%
\pgfsetroundjoin%
\pgfsetlinewidth{1.505625pt}%
\definecolor{currentstroke}{rgb}{1.000000,0.705882,0.509804}%
\pgfsetstrokecolor{currentstroke}%
\pgfsetstrokeopacity{0.800000}%
\pgfsetdash{}{0pt}%
\pgfpathmoveto{\pgfqpoint{2.700059in}{2.091219in}}%
\pgfpathlineto{\pgfqpoint{2.337997in}{2.565179in}}%
\pgfusepath{stroke}%
\end{pgfscope}%
\begin{pgfscope}%
\pgfpathrectangle{\pgfqpoint{0.481978in}{0.331635in}}{\pgfqpoint{4.960000in}{3.696000in}}%
\pgfusepath{clip}%
\pgfsetrectcap%
\pgfsetroundjoin%
\pgfsetlinewidth{1.505625pt}%
\definecolor{currentstroke}{rgb}{1.000000,0.705882,0.509804}%
\pgfsetstrokecolor{currentstroke}%
\pgfsetstrokeopacity{0.800000}%
\pgfsetdash{}{0pt}%
\pgfpathmoveto{\pgfqpoint{1.891481in}{2.983318in}}%
\pgfpathlineto{\pgfqpoint{2.337997in}{2.565179in}}%
\pgfusepath{stroke}%
\end{pgfscope}%
\begin{pgfscope}%
\pgfpathrectangle{\pgfqpoint{0.481978in}{0.331635in}}{\pgfqpoint{4.960000in}{3.696000in}}%
\pgfusepath{clip}%
\pgfsetrectcap%
\pgfsetroundjoin%
\pgfsetlinewidth{1.505625pt}%
\definecolor{currentstroke}{rgb}{1.000000,0.705882,0.509804}%
\pgfsetstrokecolor{currentstroke}%
\pgfsetstrokeopacity{0.800000}%
\pgfsetdash{}{0pt}%
\pgfpathmoveto{\pgfqpoint{2.666389in}{2.838269in}}%
\pgfpathlineto{\pgfqpoint{2.337997in}{2.565179in}}%
\pgfusepath{stroke}%
\end{pgfscope}%
\begin{pgfscope}%
\pgfpathrectangle{\pgfqpoint{0.481978in}{0.331635in}}{\pgfqpoint{4.960000in}{3.696000in}}%
\pgfusepath{clip}%
\pgfsetrectcap%
\pgfsetroundjoin%
\pgfsetlinewidth{1.505625pt}%
\definecolor{currentstroke}{rgb}{1.000000,0.705882,0.509804}%
\pgfsetstrokecolor{currentstroke}%
\pgfsetstrokeopacity{0.800000}%
\pgfsetdash{}{0pt}%
\pgfpathmoveto{\pgfqpoint{2.524989in}{2.928941in}}%
\pgfpathlineto{\pgfqpoint{2.337997in}{2.565179in}}%
\pgfusepath{stroke}%
\end{pgfscope}%
\begin{pgfscope}%
\pgfpathrectangle{\pgfqpoint{0.481978in}{0.331635in}}{\pgfqpoint{4.960000in}{3.696000in}}%
\pgfusepath{clip}%
\pgfsetrectcap%
\pgfsetroundjoin%
\pgfsetlinewidth{1.505625pt}%
\definecolor{currentstroke}{rgb}{1.000000,0.705882,0.509804}%
\pgfsetstrokecolor{currentstroke}%
\pgfsetstrokeopacity{0.800000}%
\pgfsetdash{}{0pt}%
\pgfpathmoveto{\pgfqpoint{2.128863in}{2.841321in}}%
\pgfpathlineto{\pgfqpoint{2.337997in}{2.565179in}}%
\pgfusepath{stroke}%
\end{pgfscope}%
\begin{pgfscope}%
\pgfpathrectangle{\pgfqpoint{0.481978in}{0.331635in}}{\pgfqpoint{4.960000in}{3.696000in}}%
\pgfusepath{clip}%
\pgfsetrectcap%
\pgfsetroundjoin%
\pgfsetlinewidth{1.505625pt}%
\definecolor{currentstroke}{rgb}{1.000000,0.705882,0.509804}%
\pgfsetstrokecolor{currentstroke}%
\pgfsetstrokeopacity{0.800000}%
\pgfsetdash{}{0pt}%
\pgfpathmoveto{\pgfqpoint{4.812092in}{2.899868in}}%
\pgfpathlineto{\pgfqpoint{2.337997in}{2.565179in}}%
\pgfusepath{stroke}%
\end{pgfscope}%
\begin{pgfscope}%
\pgfpathrectangle{\pgfqpoint{0.481978in}{0.331635in}}{\pgfqpoint{4.960000in}{3.696000in}}%
\pgfusepath{clip}%
\pgfsetrectcap%
\pgfsetroundjoin%
\pgfsetlinewidth{1.505625pt}%
\definecolor{currentstroke}{rgb}{1.000000,0.705882,0.509804}%
\pgfsetstrokecolor{currentstroke}%
\pgfsetstrokeopacity{0.800000}%
\pgfsetdash{}{0pt}%
\pgfpathmoveto{\pgfqpoint{3.248808in}{2.898614in}}%
\pgfpathlineto{\pgfqpoint{2.337997in}{2.565179in}}%
\pgfusepath{stroke}%
\end{pgfscope}%
\begin{pgfscope}%
\pgfpathrectangle{\pgfqpoint{0.481978in}{0.331635in}}{\pgfqpoint{4.960000in}{3.696000in}}%
\pgfusepath{clip}%
\pgfsetrectcap%
\pgfsetroundjoin%
\pgfsetlinewidth{1.505625pt}%
\definecolor{currentstroke}{rgb}{1.000000,0.705882,0.509804}%
\pgfsetstrokecolor{currentstroke}%
\pgfsetstrokeopacity{0.800000}%
\pgfsetdash{}{0pt}%
\pgfpathmoveto{\pgfqpoint{2.522551in}{3.664125in}}%
\pgfpathlineto{\pgfqpoint{2.337997in}{2.565179in}}%
\pgfusepath{stroke}%
\end{pgfscope}%
\begin{pgfscope}%
\pgfpathrectangle{\pgfqpoint{0.481978in}{0.331635in}}{\pgfqpoint{4.960000in}{3.696000in}}%
\pgfusepath{clip}%
\pgfsetrectcap%
\pgfsetroundjoin%
\pgfsetlinewidth{1.505625pt}%
\definecolor{currentstroke}{rgb}{1.000000,0.705882,0.509804}%
\pgfsetstrokecolor{currentstroke}%
\pgfsetstrokeopacity{0.800000}%
\pgfsetdash{}{0pt}%
\pgfpathmoveto{\pgfqpoint{0.846816in}{2.683331in}}%
\pgfpathlineto{\pgfqpoint{2.337997in}{2.565179in}}%
\pgfusepath{stroke}%
\end{pgfscope}%
\begin{pgfscope}%
\pgfpathrectangle{\pgfqpoint{0.481978in}{0.331635in}}{\pgfqpoint{4.960000in}{3.696000in}}%
\pgfusepath{clip}%
\pgfsetrectcap%
\pgfsetroundjoin%
\pgfsetlinewidth{1.505625pt}%
\definecolor{currentstroke}{rgb}{1.000000,0.705882,0.509804}%
\pgfsetstrokecolor{currentstroke}%
\pgfsetstrokeopacity{0.800000}%
\pgfsetdash{}{0pt}%
\pgfpathmoveto{\pgfqpoint{1.393074in}{2.446437in}}%
\pgfpathlineto{\pgfqpoint{2.337997in}{2.565179in}}%
\pgfusepath{stroke}%
\end{pgfscope}%
\begin{pgfscope}%
\pgfpathrectangle{\pgfqpoint{0.481978in}{0.331635in}}{\pgfqpoint{4.960000in}{3.696000in}}%
\pgfusepath{clip}%
\pgfsetrectcap%
\pgfsetroundjoin%
\pgfsetlinewidth{1.505625pt}%
\definecolor{currentstroke}{rgb}{1.000000,0.705882,0.509804}%
\pgfsetstrokecolor{currentstroke}%
\pgfsetstrokeopacity{0.800000}%
\pgfsetdash{}{0pt}%
\pgfpathmoveto{\pgfqpoint{3.040652in}{2.165595in}}%
\pgfpathlineto{\pgfqpoint{2.337997in}{2.565179in}}%
\pgfusepath{stroke}%
\end{pgfscope}%
\begin{pgfscope}%
\pgfpathrectangle{\pgfqpoint{0.481978in}{0.331635in}}{\pgfqpoint{4.960000in}{3.696000in}}%
\pgfusepath{clip}%
\pgfsetrectcap%
\pgfsetroundjoin%
\pgfsetlinewidth{1.505625pt}%
\definecolor{currentstroke}{rgb}{1.000000,0.705882,0.509804}%
\pgfsetstrokecolor{currentstroke}%
\pgfsetstrokeopacity{0.800000}%
\pgfsetdash{}{0pt}%
\pgfpathmoveto{\pgfqpoint{3.072022in}{2.346023in}}%
\pgfpathlineto{\pgfqpoint{2.337997in}{2.565179in}}%
\pgfusepath{stroke}%
\end{pgfscope}%
\begin{pgfscope}%
\pgfpathrectangle{\pgfqpoint{0.481978in}{0.331635in}}{\pgfqpoint{4.960000in}{3.696000in}}%
\pgfusepath{clip}%
\pgfsetrectcap%
\pgfsetroundjoin%
\pgfsetlinewidth{1.505625pt}%
\definecolor{currentstroke}{rgb}{1.000000,0.705882,0.509804}%
\pgfsetstrokecolor{currentstroke}%
\pgfsetstrokeopacity{0.800000}%
\pgfsetdash{}{0pt}%
\pgfpathmoveto{\pgfqpoint{0.985638in}{2.159784in}}%
\pgfpathlineto{\pgfqpoint{2.337997in}{2.565179in}}%
\pgfusepath{stroke}%
\end{pgfscope}%
\begin{pgfscope}%
\pgfpathrectangle{\pgfqpoint{0.481978in}{0.331635in}}{\pgfqpoint{4.960000in}{3.696000in}}%
\pgfusepath{clip}%
\pgfsetrectcap%
\pgfsetroundjoin%
\pgfsetlinewidth{1.505625pt}%
\definecolor{currentstroke}{rgb}{1.000000,0.705882,0.509804}%
\pgfsetstrokecolor{currentstroke}%
\pgfsetstrokeopacity{0.800000}%
\pgfsetdash{}{0pt}%
\pgfpathmoveto{\pgfqpoint{3.472787in}{3.534190in}}%
\pgfpathlineto{\pgfqpoint{2.337997in}{2.565179in}}%
\pgfusepath{stroke}%
\end{pgfscope}%
\begin{pgfscope}%
\pgfpathrectangle{\pgfqpoint{0.481978in}{0.331635in}}{\pgfqpoint{4.960000in}{3.696000in}}%
\pgfusepath{clip}%
\pgfsetrectcap%
\pgfsetroundjoin%
\pgfsetlinewidth{1.505625pt}%
\definecolor{currentstroke}{rgb}{1.000000,0.705882,0.509804}%
\pgfsetstrokecolor{currentstroke}%
\pgfsetstrokeopacity{0.800000}%
\pgfsetdash{}{0pt}%
\pgfpathmoveto{\pgfqpoint{4.278318in}{0.910225in}}%
\pgfpathlineto{\pgfqpoint{2.337997in}{2.565179in}}%
\pgfusepath{stroke}%
\end{pgfscope}%
\begin{pgfscope}%
\pgfpathrectangle{\pgfqpoint{0.481978in}{0.331635in}}{\pgfqpoint{4.960000in}{3.696000in}}%
\pgfusepath{clip}%
\pgfsetrectcap%
\pgfsetroundjoin%
\pgfsetlinewidth{1.505625pt}%
\definecolor{currentstroke}{rgb}{1.000000,0.705882,0.509804}%
\pgfsetstrokecolor{currentstroke}%
\pgfsetstrokeopacity{0.800000}%
\pgfsetdash{}{0pt}%
\pgfpathmoveto{\pgfqpoint{2.356273in}{2.038487in}}%
\pgfpathlineto{\pgfqpoint{2.337997in}{2.565179in}}%
\pgfusepath{stroke}%
\end{pgfscope}%
\begin{pgfscope}%
\pgfpathrectangle{\pgfqpoint{0.481978in}{0.331635in}}{\pgfqpoint{4.960000in}{3.696000in}}%
\pgfusepath{clip}%
\pgfsetrectcap%
\pgfsetroundjoin%
\pgfsetlinewidth{1.505625pt}%
\definecolor{currentstroke}{rgb}{1.000000,0.705882,0.509804}%
\pgfsetstrokecolor{currentstroke}%
\pgfsetstrokeopacity{0.800000}%
\pgfsetdash{}{0pt}%
\pgfpathmoveto{\pgfqpoint{2.481436in}{2.409751in}}%
\pgfpathlineto{\pgfqpoint{2.337997in}{2.565179in}}%
\pgfusepath{stroke}%
\end{pgfscope}%
\begin{pgfscope}%
\pgfpathrectangle{\pgfqpoint{0.481978in}{0.331635in}}{\pgfqpoint{4.960000in}{3.696000in}}%
\pgfusepath{clip}%
\pgfsetrectcap%
\pgfsetroundjoin%
\pgfsetlinewidth{1.505625pt}%
\definecolor{currentstroke}{rgb}{1.000000,0.705882,0.509804}%
\pgfsetstrokecolor{currentstroke}%
\pgfsetstrokeopacity{0.800000}%
\pgfsetdash{}{0pt}%
\pgfpathmoveto{\pgfqpoint{1.981089in}{2.679161in}}%
\pgfpathlineto{\pgfqpoint{2.337997in}{2.565179in}}%
\pgfusepath{stroke}%
\end{pgfscope}%
\begin{pgfscope}%
\pgfpathrectangle{\pgfqpoint{0.481978in}{0.331635in}}{\pgfqpoint{4.960000in}{3.696000in}}%
\pgfusepath{clip}%
\pgfsetrectcap%
\pgfsetroundjoin%
\pgfsetlinewidth{1.505625pt}%
\definecolor{currentstroke}{rgb}{1.000000,0.705882,0.509804}%
\pgfsetstrokecolor{currentstroke}%
\pgfsetstrokeopacity{0.800000}%
\pgfsetdash{}{0pt}%
\pgfpathmoveto{\pgfqpoint{3.009334in}{2.716818in}}%
\pgfpathlineto{\pgfqpoint{2.337997in}{2.565179in}}%
\pgfusepath{stroke}%
\end{pgfscope}%
\begin{pgfscope}%
\pgfpathrectangle{\pgfqpoint{0.481978in}{0.331635in}}{\pgfqpoint{4.960000in}{3.696000in}}%
\pgfusepath{clip}%
\pgfsetrectcap%
\pgfsetroundjoin%
\pgfsetlinewidth{1.505625pt}%
\definecolor{currentstroke}{rgb}{1.000000,0.705882,0.509804}%
\pgfsetstrokecolor{currentstroke}%
\pgfsetstrokeopacity{0.800000}%
\pgfsetdash{}{0pt}%
\pgfpathmoveto{\pgfqpoint{2.838079in}{2.774228in}}%
\pgfpathlineto{\pgfqpoint{2.337997in}{2.565179in}}%
\pgfusepath{stroke}%
\end{pgfscope}%
\begin{pgfscope}%
\pgfpathrectangle{\pgfqpoint{0.481978in}{0.331635in}}{\pgfqpoint{4.960000in}{3.696000in}}%
\pgfusepath{clip}%
\pgfsetrectcap%
\pgfsetroundjoin%
\pgfsetlinewidth{1.505625pt}%
\definecolor{currentstroke}{rgb}{1.000000,0.705882,0.509804}%
\pgfsetstrokecolor{currentstroke}%
\pgfsetstrokeopacity{0.800000}%
\pgfsetdash{}{0pt}%
\pgfpathmoveto{\pgfqpoint{1.984051in}{2.246964in}}%
\pgfpathlineto{\pgfqpoint{2.337997in}{2.565179in}}%
\pgfusepath{stroke}%
\end{pgfscope}%
\begin{pgfscope}%
\pgfpathrectangle{\pgfqpoint{0.481978in}{0.331635in}}{\pgfqpoint{4.960000in}{3.696000in}}%
\pgfusepath{clip}%
\pgfsetrectcap%
\pgfsetroundjoin%
\pgfsetlinewidth{1.505625pt}%
\definecolor{currentstroke}{rgb}{1.000000,0.705882,0.509804}%
\pgfsetstrokecolor{currentstroke}%
\pgfsetstrokeopacity{0.800000}%
\pgfsetdash{}{0pt}%
\pgfpathmoveto{\pgfqpoint{2.727962in}{2.263285in}}%
\pgfpathlineto{\pgfqpoint{2.337997in}{2.565179in}}%
\pgfusepath{stroke}%
\end{pgfscope}%
\begin{pgfscope}%
\pgfpathrectangle{\pgfqpoint{0.481978in}{0.331635in}}{\pgfqpoint{4.960000in}{3.696000in}}%
\pgfusepath{clip}%
\pgfsetrectcap%
\pgfsetroundjoin%
\pgfsetlinewidth{1.505625pt}%
\definecolor{currentstroke}{rgb}{1.000000,0.705882,0.509804}%
\pgfsetstrokecolor{currentstroke}%
\pgfsetstrokeopacity{0.800000}%
\pgfsetdash{}{0pt}%
\pgfpathmoveto{\pgfqpoint{2.390293in}{3.667036in}}%
\pgfpathlineto{\pgfqpoint{2.337997in}{2.565179in}}%
\pgfusepath{stroke}%
\end{pgfscope}%
\begin{pgfscope}%
\pgfpathrectangle{\pgfqpoint{0.481978in}{0.331635in}}{\pgfqpoint{4.960000in}{3.696000in}}%
\pgfusepath{clip}%
\pgfsetrectcap%
\pgfsetroundjoin%
\pgfsetlinewidth{1.505625pt}%
\definecolor{currentstroke}{rgb}{1.000000,0.705882,0.509804}%
\pgfsetstrokecolor{currentstroke}%
\pgfsetstrokeopacity{0.800000}%
\pgfsetdash{}{0pt}%
\pgfpathmoveto{\pgfqpoint{1.379840in}{2.650268in}}%
\pgfpathlineto{\pgfqpoint{2.337997in}{2.565179in}}%
\pgfusepath{stroke}%
\end{pgfscope}%
\begin{pgfscope}%
\pgfpathrectangle{\pgfqpoint{0.481978in}{0.331635in}}{\pgfqpoint{4.960000in}{3.696000in}}%
\pgfusepath{clip}%
\pgfsetrectcap%
\pgfsetroundjoin%
\pgfsetlinewidth{1.505625pt}%
\definecolor{currentstroke}{rgb}{1.000000,0.705882,0.509804}%
\pgfsetstrokecolor{currentstroke}%
\pgfsetstrokeopacity{0.800000}%
\pgfsetdash{}{0pt}%
\pgfpathmoveto{\pgfqpoint{4.883091in}{2.919323in}}%
\pgfpathlineto{\pgfqpoint{2.337997in}{2.565179in}}%
\pgfusepath{stroke}%
\end{pgfscope}%
\begin{pgfscope}%
\pgfpathrectangle{\pgfqpoint{0.481978in}{0.331635in}}{\pgfqpoint{4.960000in}{3.696000in}}%
\pgfusepath{clip}%
\pgfsetrectcap%
\pgfsetroundjoin%
\pgfsetlinewidth{1.505625pt}%
\definecolor{currentstroke}{rgb}{1.000000,0.705882,0.509804}%
\pgfsetstrokecolor{currentstroke}%
\pgfsetstrokeopacity{0.800000}%
\pgfsetdash{}{0pt}%
\pgfpathmoveto{\pgfqpoint{4.909759in}{1.642532in}}%
\pgfpathlineto{\pgfqpoint{2.337997in}{2.565179in}}%
\pgfusepath{stroke}%
\end{pgfscope}%
\begin{pgfscope}%
\pgfpathrectangle{\pgfqpoint{0.481978in}{0.331635in}}{\pgfqpoint{4.960000in}{3.696000in}}%
\pgfusepath{clip}%
\pgfsetrectcap%
\pgfsetroundjoin%
\pgfsetlinewidth{1.505625pt}%
\definecolor{currentstroke}{rgb}{1.000000,0.705882,0.509804}%
\pgfsetstrokecolor{currentstroke}%
\pgfsetstrokeopacity{0.800000}%
\pgfsetdash{}{0pt}%
\pgfpathmoveto{\pgfqpoint{1.974653in}{2.467091in}}%
\pgfpathlineto{\pgfqpoint{2.337997in}{2.565179in}}%
\pgfusepath{stroke}%
\end{pgfscope}%
\begin{pgfscope}%
\pgfpathrectangle{\pgfqpoint{0.481978in}{0.331635in}}{\pgfqpoint{4.960000in}{3.696000in}}%
\pgfusepath{clip}%
\pgfsetrectcap%
\pgfsetroundjoin%
\pgfsetlinewidth{1.505625pt}%
\definecolor{currentstroke}{rgb}{1.000000,0.705882,0.509804}%
\pgfsetstrokecolor{currentstroke}%
\pgfsetstrokeopacity{0.800000}%
\pgfsetdash{}{0pt}%
\pgfpathmoveto{\pgfqpoint{0.942652in}{2.739879in}}%
\pgfpathlineto{\pgfqpoint{2.337997in}{2.565179in}}%
\pgfusepath{stroke}%
\end{pgfscope}%
\begin{pgfscope}%
\pgfpathrectangle{\pgfqpoint{0.481978in}{0.331635in}}{\pgfqpoint{4.960000in}{3.696000in}}%
\pgfusepath{clip}%
\pgfsetrectcap%
\pgfsetroundjoin%
\pgfsetlinewidth{1.505625pt}%
\definecolor{currentstroke}{rgb}{1.000000,0.705882,0.509804}%
\pgfsetstrokecolor{currentstroke}%
\pgfsetstrokeopacity{0.800000}%
\pgfsetdash{}{0pt}%
\pgfpathmoveto{\pgfqpoint{2.801723in}{1.894017in}}%
\pgfpathlineto{\pgfqpoint{2.337997in}{2.565179in}}%
\pgfusepath{stroke}%
\end{pgfscope}%
\begin{pgfscope}%
\pgfpathrectangle{\pgfqpoint{0.481978in}{0.331635in}}{\pgfqpoint{4.960000in}{3.696000in}}%
\pgfusepath{clip}%
\pgfsetrectcap%
\pgfsetroundjoin%
\pgfsetlinewidth{1.505625pt}%
\definecolor{currentstroke}{rgb}{1.000000,0.705882,0.509804}%
\pgfsetstrokecolor{currentstroke}%
\pgfsetstrokeopacity{0.800000}%
\pgfsetdash{}{0pt}%
\pgfpathmoveto{\pgfqpoint{3.038069in}{3.561418in}}%
\pgfpathlineto{\pgfqpoint{2.337997in}{2.565179in}}%
\pgfusepath{stroke}%
\end{pgfscope}%
\begin{pgfscope}%
\pgfpathrectangle{\pgfqpoint{0.481978in}{0.331635in}}{\pgfqpoint{4.960000in}{3.696000in}}%
\pgfusepath{clip}%
\pgfsetrectcap%
\pgfsetroundjoin%
\pgfsetlinewidth{1.505625pt}%
\definecolor{currentstroke}{rgb}{1.000000,0.705882,0.509804}%
\pgfsetstrokecolor{currentstroke}%
\pgfsetstrokeopacity{0.800000}%
\pgfsetdash{}{0pt}%
\pgfpathmoveto{\pgfqpoint{2.772742in}{2.745109in}}%
\pgfpathlineto{\pgfqpoint{2.337997in}{2.565179in}}%
\pgfusepath{stroke}%
\end{pgfscope}%
\begin{pgfscope}%
\pgfpathrectangle{\pgfqpoint{0.481978in}{0.331635in}}{\pgfqpoint{4.960000in}{3.696000in}}%
\pgfusepath{clip}%
\pgfsetrectcap%
\pgfsetroundjoin%
\pgfsetlinewidth{1.505625pt}%
\definecolor{currentstroke}{rgb}{1.000000,0.705882,0.509804}%
\pgfsetstrokecolor{currentstroke}%
\pgfsetstrokeopacity{0.800000}%
\pgfsetdash{}{0pt}%
\pgfpathmoveto{\pgfqpoint{4.487558in}{1.873136in}}%
\pgfpathlineto{\pgfqpoint{2.337997in}{2.565179in}}%
\pgfusepath{stroke}%
\end{pgfscope}%
\begin{pgfscope}%
\pgfpathrectangle{\pgfqpoint{0.481978in}{0.331635in}}{\pgfqpoint{4.960000in}{3.696000in}}%
\pgfusepath{clip}%
\pgfsetrectcap%
\pgfsetroundjoin%
\pgfsetlinewidth{1.505625pt}%
\definecolor{currentstroke}{rgb}{1.000000,0.705882,0.509804}%
\pgfsetstrokecolor{currentstroke}%
\pgfsetstrokeopacity{0.800000}%
\pgfsetdash{}{0pt}%
\pgfpathmoveto{\pgfqpoint{2.930081in}{2.355086in}}%
\pgfpathlineto{\pgfqpoint{2.337997in}{2.565179in}}%
\pgfusepath{stroke}%
\end{pgfscope}%
\begin{pgfscope}%
\pgfpathrectangle{\pgfqpoint{0.481978in}{0.331635in}}{\pgfqpoint{4.960000in}{3.696000in}}%
\pgfusepath{clip}%
\pgfsetrectcap%
\pgfsetroundjoin%
\pgfsetlinewidth{1.505625pt}%
\definecolor{currentstroke}{rgb}{1.000000,0.705882,0.509804}%
\pgfsetstrokecolor{currentstroke}%
\pgfsetstrokeopacity{0.800000}%
\pgfsetdash{}{0pt}%
\pgfpathmoveto{\pgfqpoint{2.178558in}{1.985266in}}%
\pgfpathlineto{\pgfqpoint{2.337997in}{2.565179in}}%
\pgfusepath{stroke}%
\end{pgfscope}%
\begin{pgfscope}%
\pgfpathrectangle{\pgfqpoint{0.481978in}{0.331635in}}{\pgfqpoint{4.960000in}{3.696000in}}%
\pgfusepath{clip}%
\pgfsetrectcap%
\pgfsetroundjoin%
\pgfsetlinewidth{1.505625pt}%
\definecolor{currentstroke}{rgb}{1.000000,0.705882,0.509804}%
\pgfsetstrokecolor{currentstroke}%
\pgfsetstrokeopacity{0.800000}%
\pgfsetdash{}{0pt}%
\pgfpathmoveto{\pgfqpoint{4.180528in}{2.573612in}}%
\pgfpathlineto{\pgfqpoint{2.337997in}{2.565179in}}%
\pgfusepath{stroke}%
\end{pgfscope}%
\begin{pgfscope}%
\pgfpathrectangle{\pgfqpoint{0.481978in}{0.331635in}}{\pgfqpoint{4.960000in}{3.696000in}}%
\pgfusepath{clip}%
\pgfsetrectcap%
\pgfsetroundjoin%
\pgfsetlinewidth{1.505625pt}%
\definecolor{currentstroke}{rgb}{1.000000,0.705882,0.509804}%
\pgfsetstrokecolor{currentstroke}%
\pgfsetstrokeopacity{0.800000}%
\pgfsetdash{}{0pt}%
\pgfpathmoveto{\pgfqpoint{1.477002in}{3.260314in}}%
\pgfpathlineto{\pgfqpoint{2.337997in}{2.565179in}}%
\pgfusepath{stroke}%
\end{pgfscope}%
\begin{pgfscope}%
\pgfpathrectangle{\pgfqpoint{0.481978in}{0.331635in}}{\pgfqpoint{4.960000in}{3.696000in}}%
\pgfusepath{clip}%
\pgfsetrectcap%
\pgfsetroundjoin%
\pgfsetlinewidth{1.505625pt}%
\definecolor{currentstroke}{rgb}{1.000000,0.705882,0.509804}%
\pgfsetstrokecolor{currentstroke}%
\pgfsetstrokeopacity{0.800000}%
\pgfsetdash{}{0pt}%
\pgfpathmoveto{\pgfqpoint{2.910482in}{2.027668in}}%
\pgfpathlineto{\pgfqpoint{2.337997in}{2.565179in}}%
\pgfusepath{stroke}%
\end{pgfscope}%
\begin{pgfscope}%
\pgfpathrectangle{\pgfqpoint{0.481978in}{0.331635in}}{\pgfqpoint{4.960000in}{3.696000in}}%
\pgfusepath{clip}%
\pgfsetrectcap%
\pgfsetroundjoin%
\pgfsetlinewidth{1.505625pt}%
\definecolor{currentstroke}{rgb}{1.000000,0.705882,0.509804}%
\pgfsetstrokecolor{currentstroke}%
\pgfsetstrokeopacity{0.800000}%
\pgfsetdash{}{0pt}%
\pgfpathmoveto{\pgfqpoint{2.467737in}{3.071477in}}%
\pgfpathlineto{\pgfqpoint{2.337997in}{2.565179in}}%
\pgfusepath{stroke}%
\end{pgfscope}%
\begin{pgfscope}%
\pgfpathrectangle{\pgfqpoint{0.481978in}{0.331635in}}{\pgfqpoint{4.960000in}{3.696000in}}%
\pgfusepath{clip}%
\pgfsetrectcap%
\pgfsetroundjoin%
\pgfsetlinewidth{1.505625pt}%
\definecolor{currentstroke}{rgb}{1.000000,0.705882,0.509804}%
\pgfsetstrokecolor{currentstroke}%
\pgfsetstrokeopacity{0.800000}%
\pgfsetdash{}{0pt}%
\pgfpathmoveto{\pgfqpoint{1.626172in}{1.913320in}}%
\pgfpathlineto{\pgfqpoint{2.337997in}{2.565179in}}%
\pgfusepath{stroke}%
\end{pgfscope}%
\begin{pgfscope}%
\pgfpathrectangle{\pgfqpoint{0.481978in}{0.331635in}}{\pgfqpoint{4.960000in}{3.696000in}}%
\pgfusepath{clip}%
\pgfsetrectcap%
\pgfsetroundjoin%
\pgfsetlinewidth{1.505625pt}%
\definecolor{currentstroke}{rgb}{1.000000,0.705882,0.509804}%
\pgfsetstrokecolor{currentstroke}%
\pgfsetstrokeopacity{0.800000}%
\pgfsetdash{}{0pt}%
\pgfpathmoveto{\pgfqpoint{3.095619in}{2.464809in}}%
\pgfpathlineto{\pgfqpoint{2.337997in}{2.565179in}}%
\pgfusepath{stroke}%
\end{pgfscope}%
\begin{pgfscope}%
\pgfpathrectangle{\pgfqpoint{0.481978in}{0.331635in}}{\pgfqpoint{4.960000in}{3.696000in}}%
\pgfusepath{clip}%
\pgfsetrectcap%
\pgfsetroundjoin%
\pgfsetlinewidth{1.505625pt}%
\definecolor{currentstroke}{rgb}{1.000000,0.705882,0.509804}%
\pgfsetstrokecolor{currentstroke}%
\pgfsetstrokeopacity{0.800000}%
\pgfsetdash{}{0pt}%
\pgfpathmoveto{\pgfqpoint{2.196010in}{2.527672in}}%
\pgfpathlineto{\pgfqpoint{2.337997in}{2.565179in}}%
\pgfusepath{stroke}%
\end{pgfscope}%
\begin{pgfscope}%
\pgfpathrectangle{\pgfqpoint{0.481978in}{0.331635in}}{\pgfqpoint{4.960000in}{3.696000in}}%
\pgfusepath{clip}%
\pgfsetrectcap%
\pgfsetroundjoin%
\pgfsetlinewidth{1.505625pt}%
\definecolor{currentstroke}{rgb}{1.000000,0.705882,0.509804}%
\pgfsetstrokecolor{currentstroke}%
\pgfsetstrokeopacity{0.800000}%
\pgfsetdash{}{0pt}%
\pgfpathmoveto{\pgfqpoint{2.852264in}{2.641074in}}%
\pgfpathlineto{\pgfqpoint{2.337997in}{2.565179in}}%
\pgfusepath{stroke}%
\end{pgfscope}%
\begin{pgfscope}%
\pgfpathrectangle{\pgfqpoint{0.481978in}{0.331635in}}{\pgfqpoint{4.960000in}{3.696000in}}%
\pgfusepath{clip}%
\pgfsetrectcap%
\pgfsetroundjoin%
\pgfsetlinewidth{1.505625pt}%
\definecolor{currentstroke}{rgb}{1.000000,0.705882,0.509804}%
\pgfsetstrokecolor{currentstroke}%
\pgfsetstrokeopacity{0.800000}%
\pgfsetdash{}{0pt}%
\pgfpathmoveto{\pgfqpoint{1.848058in}{2.003081in}}%
\pgfpathlineto{\pgfqpoint{2.337997in}{2.565179in}}%
\pgfusepath{stroke}%
\end{pgfscope}%
\begin{pgfscope}%
\pgfpathrectangle{\pgfqpoint{0.481978in}{0.331635in}}{\pgfqpoint{4.960000in}{3.696000in}}%
\pgfusepath{clip}%
\pgfsetrectcap%
\pgfsetroundjoin%
\pgfsetlinewidth{1.505625pt}%
\definecolor{currentstroke}{rgb}{1.000000,0.705882,0.509804}%
\pgfsetstrokecolor{currentstroke}%
\pgfsetstrokeopacity{0.800000}%
\pgfsetdash{}{0pt}%
\pgfpathmoveto{\pgfqpoint{2.153651in}{2.003027in}}%
\pgfpathlineto{\pgfqpoint{2.337997in}{2.565179in}}%
\pgfusepath{stroke}%
\end{pgfscope}%
\begin{pgfscope}%
\pgfpathrectangle{\pgfqpoint{0.481978in}{0.331635in}}{\pgfqpoint{4.960000in}{3.696000in}}%
\pgfusepath{clip}%
\pgfsetrectcap%
\pgfsetroundjoin%
\pgfsetlinewidth{1.505625pt}%
\definecolor{currentstroke}{rgb}{1.000000,0.705882,0.509804}%
\pgfsetstrokecolor{currentstroke}%
\pgfsetstrokeopacity{0.800000}%
\pgfsetdash{}{0pt}%
\pgfpathmoveto{\pgfqpoint{3.013735in}{2.577569in}}%
\pgfpathlineto{\pgfqpoint{2.337997in}{2.565179in}}%
\pgfusepath{stroke}%
\end{pgfscope}%
\begin{pgfscope}%
\pgfpathrectangle{\pgfqpoint{0.481978in}{0.331635in}}{\pgfqpoint{4.960000in}{3.696000in}}%
\pgfusepath{clip}%
\pgfsetrectcap%
\pgfsetroundjoin%
\pgfsetlinewidth{1.505625pt}%
\definecolor{currentstroke}{rgb}{1.000000,0.705882,0.509804}%
\pgfsetstrokecolor{currentstroke}%
\pgfsetstrokeopacity{0.800000}%
\pgfsetdash{}{0pt}%
\pgfpathmoveto{\pgfqpoint{2.891152in}{1.895740in}}%
\pgfpathlineto{\pgfqpoint{2.337997in}{2.565179in}}%
\pgfusepath{stroke}%
\end{pgfscope}%
\begin{pgfscope}%
\pgfpathrectangle{\pgfqpoint{0.481978in}{0.331635in}}{\pgfqpoint{4.960000in}{3.696000in}}%
\pgfusepath{clip}%
\pgfsetrectcap%
\pgfsetroundjoin%
\pgfsetlinewidth{1.505625pt}%
\definecolor{currentstroke}{rgb}{1.000000,0.705882,0.509804}%
\pgfsetstrokecolor{currentstroke}%
\pgfsetstrokeopacity{0.800000}%
\pgfsetdash{}{0pt}%
\pgfpathmoveto{\pgfqpoint{2.132533in}{1.799399in}}%
\pgfpathlineto{\pgfqpoint{2.337997in}{2.565179in}}%
\pgfusepath{stroke}%
\end{pgfscope}%
\begin{pgfscope}%
\pgfpathrectangle{\pgfqpoint{0.481978in}{0.331635in}}{\pgfqpoint{4.960000in}{3.696000in}}%
\pgfusepath{clip}%
\pgfsetrectcap%
\pgfsetroundjoin%
\pgfsetlinewidth{1.505625pt}%
\definecolor{currentstroke}{rgb}{1.000000,0.705882,0.509804}%
\pgfsetstrokecolor{currentstroke}%
\pgfsetstrokeopacity{0.800000}%
\pgfsetdash{}{0pt}%
\pgfpathmoveto{\pgfqpoint{1.029745in}{1.906239in}}%
\pgfpathlineto{\pgfqpoint{2.337997in}{2.565179in}}%
\pgfusepath{stroke}%
\end{pgfscope}%
\begin{pgfscope}%
\pgfpathrectangle{\pgfqpoint{0.481978in}{0.331635in}}{\pgfqpoint{4.960000in}{3.696000in}}%
\pgfusepath{clip}%
\pgfsetrectcap%
\pgfsetroundjoin%
\pgfsetlinewidth{1.505625pt}%
\definecolor{currentstroke}{rgb}{1.000000,0.705882,0.509804}%
\pgfsetstrokecolor{currentstroke}%
\pgfsetstrokeopacity{0.800000}%
\pgfsetdash{}{0pt}%
\pgfpathmoveto{\pgfqpoint{1.708883in}{2.497218in}}%
\pgfpathlineto{\pgfqpoint{2.337997in}{2.565179in}}%
\pgfusepath{stroke}%
\end{pgfscope}%
\begin{pgfscope}%
\pgfpathrectangle{\pgfqpoint{0.481978in}{0.331635in}}{\pgfqpoint{4.960000in}{3.696000in}}%
\pgfusepath{clip}%
\pgfsetrectcap%
\pgfsetroundjoin%
\pgfsetlinewidth{1.505625pt}%
\definecolor{currentstroke}{rgb}{1.000000,0.705882,0.509804}%
\pgfsetstrokecolor{currentstroke}%
\pgfsetstrokeopacity{0.800000}%
\pgfsetdash{}{0pt}%
\pgfpathmoveto{\pgfqpoint{1.090142in}{2.282361in}}%
\pgfpathlineto{\pgfqpoint{2.337997in}{2.565179in}}%
\pgfusepath{stroke}%
\end{pgfscope}%
\begin{pgfscope}%
\pgfpathrectangle{\pgfqpoint{0.481978in}{0.331635in}}{\pgfqpoint{4.960000in}{3.696000in}}%
\pgfusepath{clip}%
\pgfsetrectcap%
\pgfsetroundjoin%
\pgfsetlinewidth{1.505625pt}%
\definecolor{currentstroke}{rgb}{1.000000,0.705882,0.509804}%
\pgfsetstrokecolor{currentstroke}%
\pgfsetstrokeopacity{0.800000}%
\pgfsetdash{}{0pt}%
\pgfpathmoveto{\pgfqpoint{1.678355in}{3.262492in}}%
\pgfpathlineto{\pgfqpoint{2.337997in}{2.565179in}}%
\pgfusepath{stroke}%
\end{pgfscope}%
\begin{pgfscope}%
\pgfpathrectangle{\pgfqpoint{0.481978in}{0.331635in}}{\pgfqpoint{4.960000in}{3.696000in}}%
\pgfusepath{clip}%
\pgfsetrectcap%
\pgfsetroundjoin%
\pgfsetlinewidth{1.505625pt}%
\definecolor{currentstroke}{rgb}{1.000000,0.705882,0.509804}%
\pgfsetstrokecolor{currentstroke}%
\pgfsetstrokeopacity{0.800000}%
\pgfsetdash{}{0pt}%
\pgfpathmoveto{\pgfqpoint{1.865099in}{2.666862in}}%
\pgfpathlineto{\pgfqpoint{2.337997in}{2.565179in}}%
\pgfusepath{stroke}%
\end{pgfscope}%
\begin{pgfscope}%
\pgfpathrectangle{\pgfqpoint{0.481978in}{0.331635in}}{\pgfqpoint{4.960000in}{3.696000in}}%
\pgfusepath{clip}%
\pgfsetrectcap%
\pgfsetroundjoin%
\pgfsetlinewidth{1.505625pt}%
\definecolor{currentstroke}{rgb}{1.000000,0.705882,0.509804}%
\pgfsetstrokecolor{currentstroke}%
\pgfsetstrokeopacity{0.800000}%
\pgfsetdash{}{0pt}%
\pgfpathmoveto{\pgfqpoint{1.669235in}{2.292977in}}%
\pgfpathlineto{\pgfqpoint{2.337997in}{2.565179in}}%
\pgfusepath{stroke}%
\end{pgfscope}%
\begin{pgfscope}%
\pgfpathrectangle{\pgfqpoint{0.481978in}{0.331635in}}{\pgfqpoint{4.960000in}{3.696000in}}%
\pgfusepath{clip}%
\pgfsetrectcap%
\pgfsetroundjoin%
\pgfsetlinewidth{1.505625pt}%
\definecolor{currentstroke}{rgb}{1.000000,0.705882,0.509804}%
\pgfsetstrokecolor{currentstroke}%
\pgfsetstrokeopacity{0.800000}%
\pgfsetdash{}{0pt}%
\pgfpathmoveto{\pgfqpoint{2.275735in}{3.766108in}}%
\pgfpathlineto{\pgfqpoint{2.337997in}{2.565179in}}%
\pgfusepath{stroke}%
\end{pgfscope}%
\begin{pgfscope}%
\pgfpathrectangle{\pgfqpoint{0.481978in}{0.331635in}}{\pgfqpoint{4.960000in}{3.696000in}}%
\pgfusepath{clip}%
\pgfsetrectcap%
\pgfsetroundjoin%
\pgfsetlinewidth{1.505625pt}%
\definecolor{currentstroke}{rgb}{1.000000,0.705882,0.509804}%
\pgfsetstrokecolor{currentstroke}%
\pgfsetstrokeopacity{0.800000}%
\pgfsetdash{}{0pt}%
\pgfpathmoveto{\pgfqpoint{1.698708in}{2.121465in}}%
\pgfpathlineto{\pgfqpoint{2.337997in}{2.565179in}}%
\pgfusepath{stroke}%
\end{pgfscope}%
\begin{pgfscope}%
\pgfpathrectangle{\pgfqpoint{0.481978in}{0.331635in}}{\pgfqpoint{4.960000in}{3.696000in}}%
\pgfusepath{clip}%
\pgfsetrectcap%
\pgfsetroundjoin%
\pgfsetlinewidth{1.505625pt}%
\definecolor{currentstroke}{rgb}{1.000000,0.705882,0.509804}%
\pgfsetstrokecolor{currentstroke}%
\pgfsetstrokeopacity{0.800000}%
\pgfsetdash{}{0pt}%
\pgfpathmoveto{\pgfqpoint{1.895840in}{2.244163in}}%
\pgfpathlineto{\pgfqpoint{2.337997in}{2.565179in}}%
\pgfusepath{stroke}%
\end{pgfscope}%
\begin{pgfscope}%
\pgfpathrectangle{\pgfqpoint{0.481978in}{0.331635in}}{\pgfqpoint{4.960000in}{3.696000in}}%
\pgfusepath{clip}%
\pgfsetrectcap%
\pgfsetroundjoin%
\pgfsetlinewidth{1.505625pt}%
\definecolor{currentstroke}{rgb}{1.000000,0.705882,0.509804}%
\pgfsetstrokecolor{currentstroke}%
\pgfsetstrokeopacity{0.800000}%
\pgfsetdash{}{0pt}%
\pgfpathmoveto{\pgfqpoint{1.504175in}{2.204081in}}%
\pgfpathlineto{\pgfqpoint{2.337997in}{2.565179in}}%
\pgfusepath{stroke}%
\end{pgfscope}%
\begin{pgfscope}%
\pgfpathrectangle{\pgfqpoint{0.481978in}{0.331635in}}{\pgfqpoint{4.960000in}{3.696000in}}%
\pgfusepath{clip}%
\pgfsetrectcap%
\pgfsetroundjoin%
\pgfsetlinewidth{1.505625pt}%
\definecolor{currentstroke}{rgb}{1.000000,0.705882,0.509804}%
\pgfsetstrokecolor{currentstroke}%
\pgfsetstrokeopacity{0.800000}%
\pgfsetdash{}{0pt}%
\pgfpathmoveto{\pgfqpoint{2.448513in}{2.354470in}}%
\pgfpathlineto{\pgfqpoint{2.337997in}{2.565179in}}%
\pgfusepath{stroke}%
\end{pgfscope}%
\begin{pgfscope}%
\pgfpathrectangle{\pgfqpoint{0.481978in}{0.331635in}}{\pgfqpoint{4.960000in}{3.696000in}}%
\pgfusepath{clip}%
\pgfsetrectcap%
\pgfsetroundjoin%
\pgfsetlinewidth{1.505625pt}%
\definecolor{currentstroke}{rgb}{1.000000,0.705882,0.509804}%
\pgfsetstrokecolor{currentstroke}%
\pgfsetstrokeopacity{0.800000}%
\pgfsetdash{}{0pt}%
\pgfpathmoveto{\pgfqpoint{1.491354in}{3.353063in}}%
\pgfpathlineto{\pgfqpoint{2.337997in}{2.565179in}}%
\pgfusepath{stroke}%
\end{pgfscope}%
\begin{pgfscope}%
\pgfpathrectangle{\pgfqpoint{0.481978in}{0.331635in}}{\pgfqpoint{4.960000in}{3.696000in}}%
\pgfusepath{clip}%
\pgfsetrectcap%
\pgfsetroundjoin%
\pgfsetlinewidth{1.505625pt}%
\definecolor{currentstroke}{rgb}{1.000000,0.705882,0.509804}%
\pgfsetstrokecolor{currentstroke}%
\pgfsetstrokeopacity{0.800000}%
\pgfsetdash{}{0pt}%
\pgfpathmoveto{\pgfqpoint{2.562202in}{2.895080in}}%
\pgfpathlineto{\pgfqpoint{2.337997in}{2.565179in}}%
\pgfusepath{stroke}%
\end{pgfscope}%
\begin{pgfscope}%
\pgfpathrectangle{\pgfqpoint{0.481978in}{0.331635in}}{\pgfqpoint{4.960000in}{3.696000in}}%
\pgfusepath{clip}%
\pgfsetrectcap%
\pgfsetroundjoin%
\pgfsetlinewidth{1.505625pt}%
\definecolor{currentstroke}{rgb}{1.000000,0.705882,0.509804}%
\pgfsetstrokecolor{currentstroke}%
\pgfsetstrokeopacity{0.800000}%
\pgfsetdash{}{0pt}%
\pgfpathmoveto{\pgfqpoint{3.825003in}{2.953582in}}%
\pgfpathlineto{\pgfqpoint{2.337997in}{2.565179in}}%
\pgfusepath{stroke}%
\end{pgfscope}%
\begin{pgfscope}%
\pgfpathrectangle{\pgfqpoint{0.481978in}{0.331635in}}{\pgfqpoint{4.960000in}{3.696000in}}%
\pgfusepath{clip}%
\pgfsetrectcap%
\pgfsetroundjoin%
\pgfsetlinewidth{1.505625pt}%
\definecolor{currentstroke}{rgb}{1.000000,0.705882,0.509804}%
\pgfsetstrokecolor{currentstroke}%
\pgfsetstrokeopacity{0.800000}%
\pgfsetdash{}{0pt}%
\pgfpathmoveto{\pgfqpoint{2.213684in}{3.162311in}}%
\pgfpathlineto{\pgfqpoint{2.337997in}{2.565179in}}%
\pgfusepath{stroke}%
\end{pgfscope}%
\begin{pgfscope}%
\pgfpathrectangle{\pgfqpoint{0.481978in}{0.331635in}}{\pgfqpoint{4.960000in}{3.696000in}}%
\pgfusepath{clip}%
\pgfsetrectcap%
\pgfsetroundjoin%
\pgfsetlinewidth{1.505625pt}%
\definecolor{currentstroke}{rgb}{1.000000,0.705882,0.509804}%
\pgfsetstrokecolor{currentstroke}%
\pgfsetstrokeopacity{0.800000}%
\pgfsetdash{}{0pt}%
\pgfpathmoveto{\pgfqpoint{2.808638in}{2.516238in}}%
\pgfpathlineto{\pgfqpoint{2.337997in}{2.565179in}}%
\pgfusepath{stroke}%
\end{pgfscope}%
\begin{pgfscope}%
\pgfpathrectangle{\pgfqpoint{0.481978in}{0.331635in}}{\pgfqpoint{4.960000in}{3.696000in}}%
\pgfusepath{clip}%
\pgfsetrectcap%
\pgfsetroundjoin%
\pgfsetlinewidth{1.505625pt}%
\definecolor{currentstroke}{rgb}{1.000000,0.705882,0.509804}%
\pgfsetstrokecolor{currentstroke}%
\pgfsetstrokeopacity{0.800000}%
\pgfsetdash{}{0pt}%
\pgfpathmoveto{\pgfqpoint{3.433772in}{3.093303in}}%
\pgfpathlineto{\pgfqpoint{2.337997in}{2.565179in}}%
\pgfusepath{stroke}%
\end{pgfscope}%
\begin{pgfscope}%
\pgfpathrectangle{\pgfqpoint{0.481978in}{0.331635in}}{\pgfqpoint{4.960000in}{3.696000in}}%
\pgfusepath{clip}%
\pgfsetrectcap%
\pgfsetroundjoin%
\pgfsetlinewidth{1.505625pt}%
\definecolor{currentstroke}{rgb}{1.000000,0.705882,0.509804}%
\pgfsetstrokecolor{currentstroke}%
\pgfsetstrokeopacity{0.800000}%
\pgfsetdash{}{0pt}%
\pgfpathmoveto{\pgfqpoint{3.258010in}{2.154528in}}%
\pgfpathlineto{\pgfqpoint{2.337997in}{2.565179in}}%
\pgfusepath{stroke}%
\end{pgfscope}%
\begin{pgfscope}%
\pgfpathrectangle{\pgfqpoint{0.481978in}{0.331635in}}{\pgfqpoint{4.960000in}{3.696000in}}%
\pgfusepath{clip}%
\pgfsetrectcap%
\pgfsetroundjoin%
\pgfsetlinewidth{1.505625pt}%
\definecolor{currentstroke}{rgb}{1.000000,0.705882,0.509804}%
\pgfsetstrokecolor{currentstroke}%
\pgfsetstrokeopacity{0.800000}%
\pgfsetdash{}{0pt}%
\pgfpathmoveto{\pgfqpoint{3.144336in}{2.931216in}}%
\pgfpathlineto{\pgfqpoint{2.337997in}{2.565179in}}%
\pgfusepath{stroke}%
\end{pgfscope}%
\begin{pgfscope}%
\pgfpathrectangle{\pgfqpoint{0.481978in}{0.331635in}}{\pgfqpoint{4.960000in}{3.696000in}}%
\pgfusepath{clip}%
\pgfsetrectcap%
\pgfsetroundjoin%
\pgfsetlinewidth{1.505625pt}%
\definecolor{currentstroke}{rgb}{1.000000,0.705882,0.509804}%
\pgfsetstrokecolor{currentstroke}%
\pgfsetstrokeopacity{0.800000}%
\pgfsetdash{}{0pt}%
\pgfpathmoveto{\pgfqpoint{1.083523in}{2.493884in}}%
\pgfpathlineto{\pgfqpoint{2.337997in}{2.565179in}}%
\pgfusepath{stroke}%
\end{pgfscope}%
\begin{pgfscope}%
\pgfpathrectangle{\pgfqpoint{0.481978in}{0.331635in}}{\pgfqpoint{4.960000in}{3.696000in}}%
\pgfusepath{clip}%
\pgfsetrectcap%
\pgfsetroundjoin%
\pgfsetlinewidth{1.505625pt}%
\definecolor{currentstroke}{rgb}{1.000000,0.705882,0.509804}%
\pgfsetstrokecolor{currentstroke}%
\pgfsetstrokeopacity{0.800000}%
\pgfsetdash{}{0pt}%
\pgfpathmoveto{\pgfqpoint{1.082227in}{2.653732in}}%
\pgfpathlineto{\pgfqpoint{2.337997in}{2.565179in}}%
\pgfusepath{stroke}%
\end{pgfscope}%
\begin{pgfscope}%
\pgfpathrectangle{\pgfqpoint{0.481978in}{0.331635in}}{\pgfqpoint{4.960000in}{3.696000in}}%
\pgfusepath{clip}%
\pgfsetrectcap%
\pgfsetroundjoin%
\pgfsetlinewidth{1.505625pt}%
\definecolor{currentstroke}{rgb}{1.000000,0.705882,0.509804}%
\pgfsetstrokecolor{currentstroke}%
\pgfsetstrokeopacity{0.800000}%
\pgfsetdash{}{0pt}%
\pgfpathmoveto{\pgfqpoint{2.917483in}{2.902205in}}%
\pgfpathlineto{\pgfqpoint{2.337997in}{2.565179in}}%
\pgfusepath{stroke}%
\end{pgfscope}%
\begin{pgfscope}%
\pgfpathrectangle{\pgfqpoint{0.481978in}{0.331635in}}{\pgfqpoint{4.960000in}{3.696000in}}%
\pgfusepath{clip}%
\pgfsetrectcap%
\pgfsetroundjoin%
\pgfsetlinewidth{1.505625pt}%
\definecolor{currentstroke}{rgb}{1.000000,0.705882,0.509804}%
\pgfsetstrokecolor{currentstroke}%
\pgfsetstrokeopacity{0.800000}%
\pgfsetdash{}{0pt}%
\pgfpathmoveto{\pgfqpoint{1.958529in}{3.788027in}}%
\pgfpathlineto{\pgfqpoint{2.337997in}{2.565179in}}%
\pgfusepath{stroke}%
\end{pgfscope}%
\begin{pgfscope}%
\pgfpathrectangle{\pgfqpoint{0.481978in}{0.331635in}}{\pgfqpoint{4.960000in}{3.696000in}}%
\pgfusepath{clip}%
\pgfsetrectcap%
\pgfsetroundjoin%
\pgfsetlinewidth{1.505625pt}%
\definecolor{currentstroke}{rgb}{1.000000,0.705882,0.509804}%
\pgfsetstrokecolor{currentstroke}%
\pgfsetstrokeopacity{0.800000}%
\pgfsetdash{}{0pt}%
\pgfpathmoveto{\pgfqpoint{3.064570in}{2.029235in}}%
\pgfpathlineto{\pgfqpoint{2.337997in}{2.565179in}}%
\pgfusepath{stroke}%
\end{pgfscope}%
\begin{pgfscope}%
\pgfpathrectangle{\pgfqpoint{0.481978in}{0.331635in}}{\pgfqpoint{4.960000in}{3.696000in}}%
\pgfusepath{clip}%
\pgfsetrectcap%
\pgfsetroundjoin%
\pgfsetlinewidth{1.505625pt}%
\definecolor{currentstroke}{rgb}{1.000000,0.705882,0.509804}%
\pgfsetstrokecolor{currentstroke}%
\pgfsetstrokeopacity{0.800000}%
\pgfsetdash{}{0pt}%
\pgfpathmoveto{\pgfqpoint{2.863689in}{2.336180in}}%
\pgfpathlineto{\pgfqpoint{2.337997in}{2.565179in}}%
\pgfusepath{stroke}%
\end{pgfscope}%
\begin{pgfscope}%
\pgfpathrectangle{\pgfqpoint{0.481978in}{0.331635in}}{\pgfqpoint{4.960000in}{3.696000in}}%
\pgfusepath{clip}%
\pgfsetrectcap%
\pgfsetroundjoin%
\pgfsetlinewidth{1.505625pt}%
\definecolor{currentstroke}{rgb}{1.000000,0.705882,0.509804}%
\pgfsetstrokecolor{currentstroke}%
\pgfsetstrokeopacity{0.800000}%
\pgfsetdash{}{0pt}%
\pgfpathmoveto{\pgfqpoint{1.553667in}{2.072930in}}%
\pgfpathlineto{\pgfqpoint{2.337997in}{2.565179in}}%
\pgfusepath{stroke}%
\end{pgfscope}%
\begin{pgfscope}%
\pgfpathrectangle{\pgfqpoint{0.481978in}{0.331635in}}{\pgfqpoint{4.960000in}{3.696000in}}%
\pgfusepath{clip}%
\pgfsetrectcap%
\pgfsetroundjoin%
\pgfsetlinewidth{1.505625pt}%
\definecolor{currentstroke}{rgb}{1.000000,0.705882,0.509804}%
\pgfsetstrokecolor{currentstroke}%
\pgfsetstrokeopacity{0.800000}%
\pgfsetdash{}{0pt}%
\pgfpathmoveto{\pgfqpoint{0.967300in}{2.727046in}}%
\pgfpathlineto{\pgfqpoint{2.337997in}{2.565179in}}%
\pgfusepath{stroke}%
\end{pgfscope}%
\begin{pgfscope}%
\pgfpathrectangle{\pgfqpoint{0.481978in}{0.331635in}}{\pgfqpoint{4.960000in}{3.696000in}}%
\pgfusepath{clip}%
\pgfsetrectcap%
\pgfsetroundjoin%
\pgfsetlinewidth{1.505625pt}%
\definecolor{currentstroke}{rgb}{1.000000,0.705882,0.509804}%
\pgfsetstrokecolor{currentstroke}%
\pgfsetstrokeopacity{0.800000}%
\pgfsetdash{}{0pt}%
\pgfpathmoveto{\pgfqpoint{3.183693in}{2.852841in}}%
\pgfpathlineto{\pgfqpoint{2.337997in}{2.565179in}}%
\pgfusepath{stroke}%
\end{pgfscope}%
\begin{pgfscope}%
\pgfpathrectangle{\pgfqpoint{0.481978in}{0.331635in}}{\pgfqpoint{4.960000in}{3.696000in}}%
\pgfusepath{clip}%
\pgfsetrectcap%
\pgfsetroundjoin%
\pgfsetlinewidth{1.505625pt}%
\definecolor{currentstroke}{rgb}{1.000000,0.705882,0.509804}%
\pgfsetstrokecolor{currentstroke}%
\pgfsetstrokeopacity{0.800000}%
\pgfsetdash{}{0pt}%
\pgfpathmoveto{\pgfqpoint{1.396213in}{1.574913in}}%
\pgfpathlineto{\pgfqpoint{2.337997in}{2.565179in}}%
\pgfusepath{stroke}%
\end{pgfscope}%
\begin{pgfscope}%
\pgfpathrectangle{\pgfqpoint{0.481978in}{0.331635in}}{\pgfqpoint{4.960000in}{3.696000in}}%
\pgfusepath{clip}%
\pgfsetrectcap%
\pgfsetroundjoin%
\pgfsetlinewidth{1.505625pt}%
\definecolor{currentstroke}{rgb}{1.000000,0.705882,0.509804}%
\pgfsetstrokecolor{currentstroke}%
\pgfsetstrokeopacity{0.800000}%
\pgfsetdash{}{0pt}%
\pgfpathmoveto{\pgfqpoint{1.259005in}{2.351764in}}%
\pgfpathlineto{\pgfqpoint{2.337997in}{2.565179in}}%
\pgfusepath{stroke}%
\end{pgfscope}%
\begin{pgfscope}%
\pgfpathrectangle{\pgfqpoint{0.481978in}{0.331635in}}{\pgfqpoint{4.960000in}{3.696000in}}%
\pgfusepath{clip}%
\pgfsetrectcap%
\pgfsetroundjoin%
\pgfsetlinewidth{1.505625pt}%
\definecolor{currentstroke}{rgb}{1.000000,0.705882,0.509804}%
\pgfsetstrokecolor{currentstroke}%
\pgfsetstrokeopacity{0.800000}%
\pgfsetdash{}{0pt}%
\pgfpathmoveto{\pgfqpoint{1.303487in}{2.129074in}}%
\pgfpathlineto{\pgfqpoint{2.337997in}{2.565179in}}%
\pgfusepath{stroke}%
\end{pgfscope}%
\begin{pgfscope}%
\pgfpathrectangle{\pgfqpoint{0.481978in}{0.331635in}}{\pgfqpoint{4.960000in}{3.696000in}}%
\pgfusepath{clip}%
\pgfsetrectcap%
\pgfsetroundjoin%
\pgfsetlinewidth{1.505625pt}%
\definecolor{currentstroke}{rgb}{1.000000,0.705882,0.509804}%
\pgfsetstrokecolor{currentstroke}%
\pgfsetstrokeopacity{0.800000}%
\pgfsetdash{}{0pt}%
\pgfpathmoveto{\pgfqpoint{2.177859in}{2.945994in}}%
\pgfpathlineto{\pgfqpoint{2.337997in}{2.565179in}}%
\pgfusepath{stroke}%
\end{pgfscope}%
\begin{pgfscope}%
\pgfpathrectangle{\pgfqpoint{0.481978in}{0.331635in}}{\pgfqpoint{4.960000in}{3.696000in}}%
\pgfusepath{clip}%
\pgfsetrectcap%
\pgfsetroundjoin%
\pgfsetlinewidth{1.505625pt}%
\definecolor{currentstroke}{rgb}{1.000000,0.705882,0.509804}%
\pgfsetstrokecolor{currentstroke}%
\pgfsetstrokeopacity{0.800000}%
\pgfsetdash{}{0pt}%
\pgfpathmoveto{\pgfqpoint{1.960427in}{2.120870in}}%
\pgfpathlineto{\pgfqpoint{2.337997in}{2.565179in}}%
\pgfusepath{stroke}%
\end{pgfscope}%
\begin{pgfscope}%
\pgfpathrectangle{\pgfqpoint{0.481978in}{0.331635in}}{\pgfqpoint{4.960000in}{3.696000in}}%
\pgfusepath{clip}%
\pgfsetrectcap%
\pgfsetroundjoin%
\pgfsetlinewidth{1.505625pt}%
\definecolor{currentstroke}{rgb}{1.000000,0.705882,0.509804}%
\pgfsetstrokecolor{currentstroke}%
\pgfsetstrokeopacity{0.800000}%
\pgfsetdash{}{0pt}%
\pgfpathmoveto{\pgfqpoint{1.507051in}{3.276917in}}%
\pgfpathlineto{\pgfqpoint{2.337997in}{2.565179in}}%
\pgfusepath{stroke}%
\end{pgfscope}%
\begin{pgfscope}%
\pgfpathrectangle{\pgfqpoint{0.481978in}{0.331635in}}{\pgfqpoint{4.960000in}{3.696000in}}%
\pgfusepath{clip}%
\pgfsetrectcap%
\pgfsetroundjoin%
\pgfsetlinewidth{1.505625pt}%
\definecolor{currentstroke}{rgb}{1.000000,0.705882,0.509804}%
\pgfsetstrokecolor{currentstroke}%
\pgfsetstrokeopacity{0.800000}%
\pgfsetdash{}{0pt}%
\pgfpathmoveto{\pgfqpoint{1.882966in}{2.833717in}}%
\pgfpathlineto{\pgfqpoint{2.337997in}{2.565179in}}%
\pgfusepath{stroke}%
\end{pgfscope}%
\begin{pgfscope}%
\pgfpathrectangle{\pgfqpoint{0.481978in}{0.331635in}}{\pgfqpoint{4.960000in}{3.696000in}}%
\pgfusepath{clip}%
\pgfsetrectcap%
\pgfsetroundjoin%
\pgfsetlinewidth{1.505625pt}%
\definecolor{currentstroke}{rgb}{1.000000,0.705882,0.509804}%
\pgfsetstrokecolor{currentstroke}%
\pgfsetstrokeopacity{0.800000}%
\pgfsetdash{}{0pt}%
\pgfpathmoveto{\pgfqpoint{1.444969in}{2.648614in}}%
\pgfpathlineto{\pgfqpoint{2.337997in}{2.565179in}}%
\pgfusepath{stroke}%
\end{pgfscope}%
\begin{pgfscope}%
\pgfpathrectangle{\pgfqpoint{0.481978in}{0.331635in}}{\pgfqpoint{4.960000in}{3.696000in}}%
\pgfusepath{clip}%
\pgfsetrectcap%
\pgfsetroundjoin%
\pgfsetlinewidth{1.505625pt}%
\definecolor{currentstroke}{rgb}{1.000000,0.705882,0.509804}%
\pgfsetstrokecolor{currentstroke}%
\pgfsetstrokeopacity{0.800000}%
\pgfsetdash{}{0pt}%
\pgfpathmoveto{\pgfqpoint{3.242283in}{2.330932in}}%
\pgfpathlineto{\pgfqpoint{2.337997in}{2.565179in}}%
\pgfusepath{stroke}%
\end{pgfscope}%
\begin{pgfscope}%
\pgfpathrectangle{\pgfqpoint{0.481978in}{0.331635in}}{\pgfqpoint{4.960000in}{3.696000in}}%
\pgfusepath{clip}%
\pgfsetrectcap%
\pgfsetroundjoin%
\pgfsetlinewidth{1.505625pt}%
\definecolor{currentstroke}{rgb}{1.000000,0.705882,0.509804}%
\pgfsetstrokecolor{currentstroke}%
\pgfsetstrokeopacity{0.800000}%
\pgfsetdash{}{0pt}%
\pgfpathmoveto{\pgfqpoint{2.875424in}{2.264007in}}%
\pgfpathlineto{\pgfqpoint{2.337997in}{2.565179in}}%
\pgfusepath{stroke}%
\end{pgfscope}%
\begin{pgfscope}%
\pgfpathrectangle{\pgfqpoint{0.481978in}{0.331635in}}{\pgfqpoint{4.960000in}{3.696000in}}%
\pgfusepath{clip}%
\pgfsetrectcap%
\pgfsetroundjoin%
\pgfsetlinewidth{1.505625pt}%
\definecolor{currentstroke}{rgb}{1.000000,0.705882,0.509804}%
\pgfsetstrokecolor{currentstroke}%
\pgfsetstrokeopacity{0.800000}%
\pgfsetdash{}{0pt}%
\pgfpathmoveto{\pgfqpoint{4.924148in}{2.937781in}}%
\pgfpathlineto{\pgfqpoint{2.337997in}{2.565179in}}%
\pgfusepath{stroke}%
\end{pgfscope}%
\begin{pgfscope}%
\pgfpathrectangle{\pgfqpoint{0.481978in}{0.331635in}}{\pgfqpoint{4.960000in}{3.696000in}}%
\pgfusepath{clip}%
\pgfsetrectcap%
\pgfsetroundjoin%
\pgfsetlinewidth{1.505625pt}%
\definecolor{currentstroke}{rgb}{1.000000,0.705882,0.509804}%
\pgfsetstrokecolor{currentstroke}%
\pgfsetstrokeopacity{0.800000}%
\pgfsetdash{}{0pt}%
\pgfpathmoveto{\pgfqpoint{1.214870in}{2.331750in}}%
\pgfpathlineto{\pgfqpoint{2.337997in}{2.565179in}}%
\pgfusepath{stroke}%
\end{pgfscope}%
\begin{pgfscope}%
\pgfpathrectangle{\pgfqpoint{0.481978in}{0.331635in}}{\pgfqpoint{4.960000in}{3.696000in}}%
\pgfusepath{clip}%
\pgfsetrectcap%
\pgfsetroundjoin%
\pgfsetlinewidth{1.505625pt}%
\definecolor{currentstroke}{rgb}{1.000000,0.705882,0.509804}%
\pgfsetstrokecolor{currentstroke}%
\pgfsetstrokeopacity{0.800000}%
\pgfsetdash{}{0pt}%
\pgfpathmoveto{\pgfqpoint{2.226211in}{2.981524in}}%
\pgfpathlineto{\pgfqpoint{2.337997in}{2.565179in}}%
\pgfusepath{stroke}%
\end{pgfscope}%
\begin{pgfscope}%
\pgfpathrectangle{\pgfqpoint{0.481978in}{0.331635in}}{\pgfqpoint{4.960000in}{3.696000in}}%
\pgfusepath{clip}%
\pgfsetrectcap%
\pgfsetroundjoin%
\pgfsetlinewidth{1.505625pt}%
\definecolor{currentstroke}{rgb}{1.000000,0.705882,0.509804}%
\pgfsetstrokecolor{currentstroke}%
\pgfsetstrokeopacity{0.800000}%
\pgfsetdash{}{0pt}%
\pgfpathmoveto{\pgfqpoint{2.628549in}{2.447118in}}%
\pgfpathlineto{\pgfqpoint{2.337997in}{2.565179in}}%
\pgfusepath{stroke}%
\end{pgfscope}%
\begin{pgfscope}%
\pgfpathrectangle{\pgfqpoint{0.481978in}{0.331635in}}{\pgfqpoint{4.960000in}{3.696000in}}%
\pgfusepath{clip}%
\pgfsetrectcap%
\pgfsetroundjoin%
\pgfsetlinewidth{1.505625pt}%
\definecolor{currentstroke}{rgb}{1.000000,0.705882,0.509804}%
\pgfsetstrokecolor{currentstroke}%
\pgfsetstrokeopacity{0.800000}%
\pgfsetdash{}{0pt}%
\pgfpathmoveto{\pgfqpoint{1.823780in}{3.533344in}}%
\pgfpathlineto{\pgfqpoint{2.337997in}{2.565179in}}%
\pgfusepath{stroke}%
\end{pgfscope}%
\begin{pgfscope}%
\pgfpathrectangle{\pgfqpoint{0.481978in}{0.331635in}}{\pgfqpoint{4.960000in}{3.696000in}}%
\pgfusepath{clip}%
\pgfsetrectcap%
\pgfsetroundjoin%
\pgfsetlinewidth{1.505625pt}%
\definecolor{currentstroke}{rgb}{1.000000,0.705882,0.509804}%
\pgfsetstrokecolor{currentstroke}%
\pgfsetstrokeopacity{0.800000}%
\pgfsetdash{}{0pt}%
\pgfpathmoveto{\pgfqpoint{2.205390in}{2.750804in}}%
\pgfpathlineto{\pgfqpoint{2.337997in}{2.565179in}}%
\pgfusepath{stroke}%
\end{pgfscope}%
\begin{pgfscope}%
\pgfpathrectangle{\pgfqpoint{0.481978in}{0.331635in}}{\pgfqpoint{4.960000in}{3.696000in}}%
\pgfusepath{clip}%
\pgfsetrectcap%
\pgfsetroundjoin%
\pgfsetlinewidth{1.505625pt}%
\definecolor{currentstroke}{rgb}{1.000000,0.705882,0.509804}%
\pgfsetstrokecolor{currentstroke}%
\pgfsetstrokeopacity{0.800000}%
\pgfsetdash{}{0pt}%
\pgfpathmoveto{\pgfqpoint{1.850038in}{1.707723in}}%
\pgfpathlineto{\pgfqpoint{2.337997in}{2.565179in}}%
\pgfusepath{stroke}%
\end{pgfscope}%
\begin{pgfscope}%
\pgfpathrectangle{\pgfqpoint{0.481978in}{0.331635in}}{\pgfqpoint{4.960000in}{3.696000in}}%
\pgfusepath{clip}%
\pgfsetrectcap%
\pgfsetroundjoin%
\pgfsetlinewidth{1.505625pt}%
\definecolor{currentstroke}{rgb}{1.000000,0.705882,0.509804}%
\pgfsetstrokecolor{currentstroke}%
\pgfsetstrokeopacity{0.800000}%
\pgfsetdash{}{0pt}%
\pgfpathmoveto{\pgfqpoint{2.915131in}{1.115444in}}%
\pgfpathlineto{\pgfqpoint{2.337997in}{2.565179in}}%
\pgfusepath{stroke}%
\end{pgfscope}%
\begin{pgfscope}%
\pgfpathrectangle{\pgfqpoint{0.481978in}{0.331635in}}{\pgfqpoint{4.960000in}{3.696000in}}%
\pgfusepath{clip}%
\pgfsetrectcap%
\pgfsetroundjoin%
\pgfsetlinewidth{1.505625pt}%
\definecolor{currentstroke}{rgb}{1.000000,0.705882,0.509804}%
\pgfsetstrokecolor{currentstroke}%
\pgfsetstrokeopacity{0.800000}%
\pgfsetdash{}{0pt}%
\pgfpathmoveto{\pgfqpoint{2.616748in}{2.681499in}}%
\pgfpathlineto{\pgfqpoint{2.337997in}{2.565179in}}%
\pgfusepath{stroke}%
\end{pgfscope}%
\begin{pgfscope}%
\pgfpathrectangle{\pgfqpoint{0.481978in}{0.331635in}}{\pgfqpoint{4.960000in}{3.696000in}}%
\pgfusepath{clip}%
\pgfsetrectcap%
\pgfsetroundjoin%
\pgfsetlinewidth{1.505625pt}%
\definecolor{currentstroke}{rgb}{1.000000,0.705882,0.509804}%
\pgfsetstrokecolor{currentstroke}%
\pgfsetstrokeopacity{0.800000}%
\pgfsetdash{}{0pt}%
\pgfpathmoveto{\pgfqpoint{3.272083in}{3.134410in}}%
\pgfpathlineto{\pgfqpoint{2.337997in}{2.565179in}}%
\pgfusepath{stroke}%
\end{pgfscope}%
\begin{pgfscope}%
\pgfpathrectangle{\pgfqpoint{0.481978in}{0.331635in}}{\pgfqpoint{4.960000in}{3.696000in}}%
\pgfusepath{clip}%
\pgfsetrectcap%
\pgfsetroundjoin%
\pgfsetlinewidth{1.505625pt}%
\definecolor{currentstroke}{rgb}{1.000000,0.705882,0.509804}%
\pgfsetstrokecolor{currentstroke}%
\pgfsetstrokeopacity{0.800000}%
\pgfsetdash{}{0pt}%
\pgfpathmoveto{\pgfqpoint{1.683594in}{2.643285in}}%
\pgfpathlineto{\pgfqpoint{2.337997in}{2.565179in}}%
\pgfusepath{stroke}%
\end{pgfscope}%
\begin{pgfscope}%
\pgfpathrectangle{\pgfqpoint{0.481978in}{0.331635in}}{\pgfqpoint{4.960000in}{3.696000in}}%
\pgfusepath{clip}%
\pgfsetrectcap%
\pgfsetroundjoin%
\pgfsetlinewidth{1.505625pt}%
\definecolor{currentstroke}{rgb}{1.000000,0.705882,0.509804}%
\pgfsetstrokecolor{currentstroke}%
\pgfsetstrokeopacity{0.800000}%
\pgfsetdash{}{0pt}%
\pgfpathmoveto{\pgfqpoint{2.146764in}{2.897572in}}%
\pgfpathlineto{\pgfqpoint{2.337997in}{2.565179in}}%
\pgfusepath{stroke}%
\end{pgfscope}%
\begin{pgfscope}%
\pgfpathrectangle{\pgfqpoint{0.481978in}{0.331635in}}{\pgfqpoint{4.960000in}{3.696000in}}%
\pgfusepath{clip}%
\pgfsetrectcap%
\pgfsetroundjoin%
\pgfsetlinewidth{1.505625pt}%
\definecolor{currentstroke}{rgb}{1.000000,0.705882,0.509804}%
\pgfsetstrokecolor{currentstroke}%
\pgfsetstrokeopacity{0.800000}%
\pgfsetdash{}{0pt}%
\pgfpathmoveto{\pgfqpoint{1.635619in}{2.857284in}}%
\pgfpathlineto{\pgfqpoint{2.337997in}{2.565179in}}%
\pgfusepath{stroke}%
\end{pgfscope}%
\begin{pgfscope}%
\pgfpathrectangle{\pgfqpoint{0.481978in}{0.331635in}}{\pgfqpoint{4.960000in}{3.696000in}}%
\pgfusepath{clip}%
\pgfsetrectcap%
\pgfsetroundjoin%
\pgfsetlinewidth{1.505625pt}%
\definecolor{currentstroke}{rgb}{1.000000,0.705882,0.509804}%
\pgfsetstrokecolor{currentstroke}%
\pgfsetstrokeopacity{0.800000}%
\pgfsetdash{}{0pt}%
\pgfpathmoveto{\pgfqpoint{2.315751in}{3.645717in}}%
\pgfpathlineto{\pgfqpoint{2.337997in}{2.565179in}}%
\pgfusepath{stroke}%
\end{pgfscope}%
\begin{pgfscope}%
\pgfpathrectangle{\pgfqpoint{0.481978in}{0.331635in}}{\pgfqpoint{4.960000in}{3.696000in}}%
\pgfusepath{clip}%
\pgfsetrectcap%
\pgfsetroundjoin%
\pgfsetlinewidth{1.505625pt}%
\definecolor{currentstroke}{rgb}{1.000000,0.705882,0.509804}%
\pgfsetstrokecolor{currentstroke}%
\pgfsetstrokeopacity{0.800000}%
\pgfsetdash{}{0pt}%
\pgfpathmoveto{\pgfqpoint{2.612288in}{2.888991in}}%
\pgfpathlineto{\pgfqpoint{2.337997in}{2.565179in}}%
\pgfusepath{stroke}%
\end{pgfscope}%
\begin{pgfscope}%
\pgfpathrectangle{\pgfqpoint{0.481978in}{0.331635in}}{\pgfqpoint{4.960000in}{3.696000in}}%
\pgfusepath{clip}%
\pgfsetrectcap%
\pgfsetroundjoin%
\pgfsetlinewidth{1.505625pt}%
\definecolor{currentstroke}{rgb}{1.000000,0.705882,0.509804}%
\pgfsetstrokecolor{currentstroke}%
\pgfsetstrokeopacity{0.800000}%
\pgfsetdash{}{0pt}%
\pgfpathmoveto{\pgfqpoint{2.791676in}{2.125988in}}%
\pgfpathlineto{\pgfqpoint{2.337997in}{2.565179in}}%
\pgfusepath{stroke}%
\end{pgfscope}%
\begin{pgfscope}%
\pgfpathrectangle{\pgfqpoint{0.481978in}{0.331635in}}{\pgfqpoint{4.960000in}{3.696000in}}%
\pgfusepath{clip}%
\pgfsetrectcap%
\pgfsetroundjoin%
\pgfsetlinewidth{1.505625pt}%
\definecolor{currentstroke}{rgb}{1.000000,0.705882,0.509804}%
\pgfsetstrokecolor{currentstroke}%
\pgfsetstrokeopacity{0.800000}%
\pgfsetdash{}{0pt}%
\pgfpathmoveto{\pgfqpoint{2.324073in}{2.707896in}}%
\pgfpathlineto{\pgfqpoint{2.337997in}{2.565179in}}%
\pgfusepath{stroke}%
\end{pgfscope}%
\begin{pgfscope}%
\pgfpathrectangle{\pgfqpoint{0.481978in}{0.331635in}}{\pgfqpoint{4.960000in}{3.696000in}}%
\pgfusepath{clip}%
\pgfsetrectcap%
\pgfsetroundjoin%
\pgfsetlinewidth{1.505625pt}%
\definecolor{currentstroke}{rgb}{1.000000,0.705882,0.509804}%
\pgfsetstrokecolor{currentstroke}%
\pgfsetstrokeopacity{0.800000}%
\pgfsetdash{}{0pt}%
\pgfpathmoveto{\pgfqpoint{2.338731in}{2.759166in}}%
\pgfpathlineto{\pgfqpoint{2.337997in}{2.565179in}}%
\pgfusepath{stroke}%
\end{pgfscope}%
\begin{pgfscope}%
\pgfpathrectangle{\pgfqpoint{0.481978in}{0.331635in}}{\pgfqpoint{4.960000in}{3.696000in}}%
\pgfusepath{clip}%
\pgfsetrectcap%
\pgfsetroundjoin%
\pgfsetlinewidth{1.505625pt}%
\definecolor{currentstroke}{rgb}{1.000000,0.705882,0.509804}%
\pgfsetstrokecolor{currentstroke}%
\pgfsetstrokeopacity{0.800000}%
\pgfsetdash{}{0pt}%
\pgfpathmoveto{\pgfqpoint{1.786524in}{1.631832in}}%
\pgfpathlineto{\pgfqpoint{2.337997in}{2.565179in}}%
\pgfusepath{stroke}%
\end{pgfscope}%
\begin{pgfscope}%
\pgfpathrectangle{\pgfqpoint{0.481978in}{0.331635in}}{\pgfqpoint{4.960000in}{3.696000in}}%
\pgfusepath{clip}%
\pgfsetrectcap%
\pgfsetroundjoin%
\pgfsetlinewidth{1.505625pt}%
\definecolor{currentstroke}{rgb}{1.000000,0.705882,0.509804}%
\pgfsetstrokecolor{currentstroke}%
\pgfsetstrokeopacity{0.800000}%
\pgfsetdash{}{0pt}%
\pgfpathmoveto{\pgfqpoint{4.353277in}{3.030325in}}%
\pgfpathlineto{\pgfqpoint{2.337997in}{2.565179in}}%
\pgfusepath{stroke}%
\end{pgfscope}%
\begin{pgfscope}%
\pgfpathrectangle{\pgfqpoint{0.481978in}{0.331635in}}{\pgfqpoint{4.960000in}{3.696000in}}%
\pgfusepath{clip}%
\pgfsetrectcap%
\pgfsetroundjoin%
\pgfsetlinewidth{1.505625pt}%
\definecolor{currentstroke}{rgb}{1.000000,0.705882,0.509804}%
\pgfsetstrokecolor{currentstroke}%
\pgfsetstrokeopacity{0.800000}%
\pgfsetdash{}{0pt}%
\pgfpathmoveto{\pgfqpoint{2.688812in}{2.816801in}}%
\pgfpathlineto{\pgfqpoint{2.337997in}{2.565179in}}%
\pgfusepath{stroke}%
\end{pgfscope}%
\begin{pgfscope}%
\pgfpathrectangle{\pgfqpoint{0.481978in}{0.331635in}}{\pgfqpoint{4.960000in}{3.696000in}}%
\pgfusepath{clip}%
\pgfsetrectcap%
\pgfsetroundjoin%
\pgfsetlinewidth{1.505625pt}%
\definecolor{currentstroke}{rgb}{1.000000,0.705882,0.509804}%
\pgfsetstrokecolor{currentstroke}%
\pgfsetstrokeopacity{0.800000}%
\pgfsetdash{}{0pt}%
\pgfpathmoveto{\pgfqpoint{1.990886in}{3.122999in}}%
\pgfpathlineto{\pgfqpoint{2.337997in}{2.565179in}}%
\pgfusepath{stroke}%
\end{pgfscope}%
\begin{pgfscope}%
\pgfpathrectangle{\pgfqpoint{0.481978in}{0.331635in}}{\pgfqpoint{4.960000in}{3.696000in}}%
\pgfusepath{clip}%
\pgfsetrectcap%
\pgfsetroundjoin%
\pgfsetlinewidth{1.505625pt}%
\definecolor{currentstroke}{rgb}{1.000000,0.705882,0.509804}%
\pgfsetstrokecolor{currentstroke}%
\pgfsetstrokeopacity{0.800000}%
\pgfsetdash{}{0pt}%
\pgfpathmoveto{\pgfqpoint{2.134851in}{2.996977in}}%
\pgfpathlineto{\pgfqpoint{2.337997in}{2.565179in}}%
\pgfusepath{stroke}%
\end{pgfscope}%
\begin{pgfscope}%
\pgfpathrectangle{\pgfqpoint{0.481978in}{0.331635in}}{\pgfqpoint{4.960000in}{3.696000in}}%
\pgfusepath{clip}%
\pgfsetrectcap%
\pgfsetroundjoin%
\pgfsetlinewidth{1.505625pt}%
\definecolor{currentstroke}{rgb}{1.000000,0.705882,0.509804}%
\pgfsetstrokecolor{currentstroke}%
\pgfsetstrokeopacity{0.800000}%
\pgfsetdash{}{0pt}%
\pgfpathmoveto{\pgfqpoint{0.966075in}{2.723911in}}%
\pgfpathlineto{\pgfqpoint{2.337997in}{2.565179in}}%
\pgfusepath{stroke}%
\end{pgfscope}%
\begin{pgfscope}%
\pgfpathrectangle{\pgfqpoint{0.481978in}{0.331635in}}{\pgfqpoint{4.960000in}{3.696000in}}%
\pgfusepath{clip}%
\pgfsetrectcap%
\pgfsetroundjoin%
\pgfsetlinewidth{1.505625pt}%
\definecolor{currentstroke}{rgb}{1.000000,0.705882,0.509804}%
\pgfsetstrokecolor{currentstroke}%
\pgfsetstrokeopacity{0.800000}%
\pgfsetdash{}{0pt}%
\pgfpathmoveto{\pgfqpoint{2.829475in}{2.806749in}}%
\pgfpathlineto{\pgfqpoint{2.337997in}{2.565179in}}%
\pgfusepath{stroke}%
\end{pgfscope}%
\begin{pgfscope}%
\pgfpathrectangle{\pgfqpoint{0.481978in}{0.331635in}}{\pgfqpoint{4.960000in}{3.696000in}}%
\pgfusepath{clip}%
\pgfsetrectcap%
\pgfsetroundjoin%
\pgfsetlinewidth{1.505625pt}%
\definecolor{currentstroke}{rgb}{1.000000,0.705882,0.509804}%
\pgfsetstrokecolor{currentstroke}%
\pgfsetstrokeopacity{0.800000}%
\pgfsetdash{}{0pt}%
\pgfpathmoveto{\pgfqpoint{2.536496in}{1.779590in}}%
\pgfpathlineto{\pgfqpoint{2.337997in}{2.565179in}}%
\pgfusepath{stroke}%
\end{pgfscope}%
\begin{pgfscope}%
\pgfpathrectangle{\pgfqpoint{0.481978in}{0.331635in}}{\pgfqpoint{4.960000in}{3.696000in}}%
\pgfusepath{clip}%
\pgfsetrectcap%
\pgfsetroundjoin%
\pgfsetlinewidth{1.505625pt}%
\definecolor{currentstroke}{rgb}{1.000000,0.705882,0.509804}%
\pgfsetstrokecolor{currentstroke}%
\pgfsetstrokeopacity{0.800000}%
\pgfsetdash{}{0pt}%
\pgfpathmoveto{\pgfqpoint{1.702979in}{2.625964in}}%
\pgfpathlineto{\pgfqpoint{2.337997in}{2.565179in}}%
\pgfusepath{stroke}%
\end{pgfscope}%
\begin{pgfscope}%
\pgfpathrectangle{\pgfqpoint{0.481978in}{0.331635in}}{\pgfqpoint{4.960000in}{3.696000in}}%
\pgfusepath{clip}%
\pgfsetrectcap%
\pgfsetroundjoin%
\pgfsetlinewidth{1.505625pt}%
\definecolor{currentstroke}{rgb}{1.000000,0.705882,0.509804}%
\pgfsetstrokecolor{currentstroke}%
\pgfsetstrokeopacity{0.800000}%
\pgfsetdash{}{0pt}%
\pgfpathmoveto{\pgfqpoint{1.756552in}{2.196124in}}%
\pgfpathlineto{\pgfqpoint{2.337997in}{2.565179in}}%
\pgfusepath{stroke}%
\end{pgfscope}%
\begin{pgfscope}%
\pgfpathrectangle{\pgfqpoint{0.481978in}{0.331635in}}{\pgfqpoint{4.960000in}{3.696000in}}%
\pgfusepath{clip}%
\pgfsetrectcap%
\pgfsetroundjoin%
\pgfsetlinewidth{1.505625pt}%
\definecolor{currentstroke}{rgb}{1.000000,0.705882,0.509804}%
\pgfsetstrokecolor{currentstroke}%
\pgfsetstrokeopacity{0.800000}%
\pgfsetdash{}{0pt}%
\pgfpathmoveto{\pgfqpoint{1.568211in}{3.117197in}}%
\pgfpathlineto{\pgfqpoint{2.337997in}{2.565179in}}%
\pgfusepath{stroke}%
\end{pgfscope}%
\begin{pgfscope}%
\pgfpathrectangle{\pgfqpoint{0.481978in}{0.331635in}}{\pgfqpoint{4.960000in}{3.696000in}}%
\pgfusepath{clip}%
\pgfsetrectcap%
\pgfsetroundjoin%
\pgfsetlinewidth{1.505625pt}%
\definecolor{currentstroke}{rgb}{1.000000,0.705882,0.509804}%
\pgfsetstrokecolor{currentstroke}%
\pgfsetstrokeopacity{0.800000}%
\pgfsetdash{}{0pt}%
\pgfpathmoveto{\pgfqpoint{1.805413in}{3.152443in}}%
\pgfpathlineto{\pgfqpoint{2.337997in}{2.565179in}}%
\pgfusepath{stroke}%
\end{pgfscope}%
\begin{pgfscope}%
\pgfpathrectangle{\pgfqpoint{0.481978in}{0.331635in}}{\pgfqpoint{4.960000in}{3.696000in}}%
\pgfusepath{clip}%
\pgfsetrectcap%
\pgfsetroundjoin%
\pgfsetlinewidth{1.505625pt}%
\definecolor{currentstroke}{rgb}{1.000000,0.705882,0.509804}%
\pgfsetstrokecolor{currentstroke}%
\pgfsetstrokeopacity{0.800000}%
\pgfsetdash{}{0pt}%
\pgfpathmoveto{\pgfqpoint{2.744902in}{3.440443in}}%
\pgfpathlineto{\pgfqpoint{2.337997in}{2.565179in}}%
\pgfusepath{stroke}%
\end{pgfscope}%
\begin{pgfscope}%
\pgfpathrectangle{\pgfqpoint{0.481978in}{0.331635in}}{\pgfqpoint{4.960000in}{3.696000in}}%
\pgfusepath{clip}%
\pgfsetrectcap%
\pgfsetroundjoin%
\pgfsetlinewidth{1.505625pt}%
\definecolor{currentstroke}{rgb}{1.000000,0.705882,0.509804}%
\pgfsetstrokecolor{currentstroke}%
\pgfsetstrokeopacity{0.800000}%
\pgfsetdash{}{0pt}%
\pgfpathmoveto{\pgfqpoint{1.466717in}{2.777363in}}%
\pgfpathlineto{\pgfqpoint{2.337997in}{2.565179in}}%
\pgfusepath{stroke}%
\end{pgfscope}%
\begin{pgfscope}%
\pgfpathrectangle{\pgfqpoint{0.481978in}{0.331635in}}{\pgfqpoint{4.960000in}{3.696000in}}%
\pgfusepath{clip}%
\pgfsetrectcap%
\pgfsetroundjoin%
\pgfsetlinewidth{1.505625pt}%
\definecolor{currentstroke}{rgb}{1.000000,0.705882,0.509804}%
\pgfsetstrokecolor{currentstroke}%
\pgfsetstrokeopacity{0.800000}%
\pgfsetdash{}{0pt}%
\pgfpathmoveto{\pgfqpoint{2.898260in}{2.964118in}}%
\pgfpathlineto{\pgfqpoint{2.337997in}{2.565179in}}%
\pgfusepath{stroke}%
\end{pgfscope}%
\begin{pgfscope}%
\pgfpathrectangle{\pgfqpoint{0.481978in}{0.331635in}}{\pgfqpoint{4.960000in}{3.696000in}}%
\pgfusepath{clip}%
\pgfsetrectcap%
\pgfsetroundjoin%
\pgfsetlinewidth{1.505625pt}%
\definecolor{currentstroke}{rgb}{1.000000,0.705882,0.509804}%
\pgfsetstrokecolor{currentstroke}%
\pgfsetstrokeopacity{0.800000}%
\pgfsetdash{}{0pt}%
\pgfpathmoveto{\pgfqpoint{2.382543in}{2.328812in}}%
\pgfpathlineto{\pgfqpoint{2.337997in}{2.565179in}}%
\pgfusepath{stroke}%
\end{pgfscope}%
\begin{pgfscope}%
\pgfpathrectangle{\pgfqpoint{0.481978in}{0.331635in}}{\pgfqpoint{4.960000in}{3.696000in}}%
\pgfusepath{clip}%
\pgfsetrectcap%
\pgfsetroundjoin%
\pgfsetlinewidth{1.505625pt}%
\definecolor{currentstroke}{rgb}{1.000000,0.705882,0.509804}%
\pgfsetstrokecolor{currentstroke}%
\pgfsetstrokeopacity{0.800000}%
\pgfsetdash{}{0pt}%
\pgfpathmoveto{\pgfqpoint{2.770316in}{2.563522in}}%
\pgfpathlineto{\pgfqpoint{2.337997in}{2.565179in}}%
\pgfusepath{stroke}%
\end{pgfscope}%
\begin{pgfscope}%
\pgfpathrectangle{\pgfqpoint{0.481978in}{0.331635in}}{\pgfqpoint{4.960000in}{3.696000in}}%
\pgfusepath{clip}%
\pgfsetrectcap%
\pgfsetroundjoin%
\pgfsetlinewidth{1.505625pt}%
\definecolor{currentstroke}{rgb}{1.000000,0.705882,0.509804}%
\pgfsetstrokecolor{currentstroke}%
\pgfsetstrokeopacity{0.800000}%
\pgfsetdash{}{0pt}%
\pgfpathmoveto{\pgfqpoint{1.391784in}{1.934968in}}%
\pgfpathlineto{\pgfqpoint{2.337997in}{2.565179in}}%
\pgfusepath{stroke}%
\end{pgfscope}%
\begin{pgfscope}%
\pgfpathrectangle{\pgfqpoint{0.481978in}{0.331635in}}{\pgfqpoint{4.960000in}{3.696000in}}%
\pgfusepath{clip}%
\pgfsetrectcap%
\pgfsetroundjoin%
\pgfsetlinewidth{1.505625pt}%
\definecolor{currentstroke}{rgb}{1.000000,0.705882,0.509804}%
\pgfsetstrokecolor{currentstroke}%
\pgfsetstrokeopacity{0.800000}%
\pgfsetdash{}{0pt}%
\pgfpathmoveto{\pgfqpoint{1.870023in}{2.231564in}}%
\pgfpathlineto{\pgfqpoint{2.337997in}{2.565179in}}%
\pgfusepath{stroke}%
\end{pgfscope}%
\begin{pgfscope}%
\pgfpathrectangle{\pgfqpoint{0.481978in}{0.331635in}}{\pgfqpoint{4.960000in}{3.696000in}}%
\pgfusepath{clip}%
\pgfsetrectcap%
\pgfsetroundjoin%
\pgfsetlinewidth{1.505625pt}%
\definecolor{currentstroke}{rgb}{1.000000,0.705882,0.509804}%
\pgfsetstrokecolor{currentstroke}%
\pgfsetstrokeopacity{0.800000}%
\pgfsetdash{}{0pt}%
\pgfpathmoveto{\pgfqpoint{3.278587in}{2.955146in}}%
\pgfpathlineto{\pgfqpoint{2.337997in}{2.565179in}}%
\pgfusepath{stroke}%
\end{pgfscope}%
\begin{pgfscope}%
\pgfpathrectangle{\pgfqpoint{0.481978in}{0.331635in}}{\pgfqpoint{4.960000in}{3.696000in}}%
\pgfusepath{clip}%
\pgfsetrectcap%
\pgfsetroundjoin%
\pgfsetlinewidth{1.505625pt}%
\definecolor{currentstroke}{rgb}{1.000000,0.705882,0.509804}%
\pgfsetstrokecolor{currentstroke}%
\pgfsetstrokeopacity{0.800000}%
\pgfsetdash{}{0pt}%
\pgfpathmoveto{\pgfqpoint{2.052985in}{3.336677in}}%
\pgfpathlineto{\pgfqpoint{2.337997in}{2.565179in}}%
\pgfusepath{stroke}%
\end{pgfscope}%
\begin{pgfscope}%
\pgfpathrectangle{\pgfqpoint{0.481978in}{0.331635in}}{\pgfqpoint{4.960000in}{3.696000in}}%
\pgfusepath{clip}%
\pgfsetrectcap%
\pgfsetroundjoin%
\pgfsetlinewidth{1.505625pt}%
\definecolor{currentstroke}{rgb}{1.000000,0.705882,0.509804}%
\pgfsetstrokecolor{currentstroke}%
\pgfsetstrokeopacity{0.800000}%
\pgfsetdash{}{0pt}%
\pgfpathmoveto{\pgfqpoint{2.446981in}{3.374371in}}%
\pgfpathlineto{\pgfqpoint{2.337997in}{2.565179in}}%
\pgfusepath{stroke}%
\end{pgfscope}%
\begin{pgfscope}%
\pgfpathrectangle{\pgfqpoint{0.481978in}{0.331635in}}{\pgfqpoint{4.960000in}{3.696000in}}%
\pgfusepath{clip}%
\pgfsetrectcap%
\pgfsetroundjoin%
\pgfsetlinewidth{1.505625pt}%
\definecolor{currentstroke}{rgb}{1.000000,0.705882,0.509804}%
\pgfsetstrokecolor{currentstroke}%
\pgfsetstrokeopacity{0.800000}%
\pgfsetdash{}{0pt}%
\pgfpathmoveto{\pgfqpoint{0.935858in}{2.318960in}}%
\pgfpathlineto{\pgfqpoint{2.337997in}{2.565179in}}%
\pgfusepath{stroke}%
\end{pgfscope}%
\begin{pgfscope}%
\pgfpathrectangle{\pgfqpoint{0.481978in}{0.331635in}}{\pgfqpoint{4.960000in}{3.696000in}}%
\pgfusepath{clip}%
\pgfsetrectcap%
\pgfsetroundjoin%
\pgfsetlinewidth{1.505625pt}%
\definecolor{currentstroke}{rgb}{1.000000,0.705882,0.509804}%
\pgfsetstrokecolor{currentstroke}%
\pgfsetstrokeopacity{0.800000}%
\pgfsetdash{}{0pt}%
\pgfpathmoveto{\pgfqpoint{4.276696in}{0.906380in}}%
\pgfpathlineto{\pgfqpoint{2.337997in}{2.565179in}}%
\pgfusepath{stroke}%
\end{pgfscope}%
\begin{pgfscope}%
\pgfpathrectangle{\pgfqpoint{0.481978in}{0.331635in}}{\pgfqpoint{4.960000in}{3.696000in}}%
\pgfusepath{clip}%
\pgfsetrectcap%
\pgfsetroundjoin%
\pgfsetlinewidth{1.505625pt}%
\definecolor{currentstroke}{rgb}{1.000000,0.705882,0.509804}%
\pgfsetstrokecolor{currentstroke}%
\pgfsetstrokeopacity{0.800000}%
\pgfsetdash{}{0pt}%
\pgfpathmoveto{\pgfqpoint{2.575287in}{2.671589in}}%
\pgfpathlineto{\pgfqpoint{2.337997in}{2.565179in}}%
\pgfusepath{stroke}%
\end{pgfscope}%
\begin{pgfscope}%
\pgfpathrectangle{\pgfqpoint{0.481978in}{0.331635in}}{\pgfqpoint{4.960000in}{3.696000in}}%
\pgfusepath{clip}%
\pgfsetrectcap%
\pgfsetroundjoin%
\pgfsetlinewidth{1.505625pt}%
\definecolor{currentstroke}{rgb}{1.000000,0.705882,0.509804}%
\pgfsetstrokecolor{currentstroke}%
\pgfsetstrokeopacity{0.800000}%
\pgfsetdash{}{0pt}%
\pgfpathmoveto{\pgfqpoint{3.924469in}{3.477635in}}%
\pgfpathlineto{\pgfqpoint{2.337997in}{2.565179in}}%
\pgfusepath{stroke}%
\end{pgfscope}%
\begin{pgfscope}%
\pgfpathrectangle{\pgfqpoint{0.481978in}{0.331635in}}{\pgfqpoint{4.960000in}{3.696000in}}%
\pgfusepath{clip}%
\pgfsetrectcap%
\pgfsetroundjoin%
\pgfsetlinewidth{1.505625pt}%
\definecolor{currentstroke}{rgb}{1.000000,0.705882,0.509804}%
\pgfsetstrokecolor{currentstroke}%
\pgfsetstrokeopacity{0.800000}%
\pgfsetdash{}{0pt}%
\pgfpathmoveto{\pgfqpoint{1.226711in}{3.212679in}}%
\pgfpathlineto{\pgfqpoint{2.337997in}{2.565179in}}%
\pgfusepath{stroke}%
\end{pgfscope}%
\begin{pgfscope}%
\pgfpathrectangle{\pgfqpoint{0.481978in}{0.331635in}}{\pgfqpoint{4.960000in}{3.696000in}}%
\pgfusepath{clip}%
\pgfsetrectcap%
\pgfsetroundjoin%
\pgfsetlinewidth{1.505625pt}%
\definecolor{currentstroke}{rgb}{1.000000,0.705882,0.509804}%
\pgfsetstrokecolor{currentstroke}%
\pgfsetstrokeopacity{0.800000}%
\pgfsetdash{}{0pt}%
\pgfpathmoveto{\pgfqpoint{1.663711in}{3.052564in}}%
\pgfpathlineto{\pgfqpoint{2.337997in}{2.565179in}}%
\pgfusepath{stroke}%
\end{pgfscope}%
\begin{pgfscope}%
\pgfpathrectangle{\pgfqpoint{0.481978in}{0.331635in}}{\pgfqpoint{4.960000in}{3.696000in}}%
\pgfusepath{clip}%
\pgfsetrectcap%
\pgfsetroundjoin%
\pgfsetlinewidth{1.505625pt}%
\definecolor{currentstroke}{rgb}{1.000000,0.705882,0.509804}%
\pgfsetstrokecolor{currentstroke}%
\pgfsetstrokeopacity{0.800000}%
\pgfsetdash{}{0pt}%
\pgfpathmoveto{\pgfqpoint{1.438649in}{2.367284in}}%
\pgfpathlineto{\pgfqpoint{2.337997in}{2.565179in}}%
\pgfusepath{stroke}%
\end{pgfscope}%
\begin{pgfscope}%
\pgfpathrectangle{\pgfqpoint{0.481978in}{0.331635in}}{\pgfqpoint{4.960000in}{3.696000in}}%
\pgfusepath{clip}%
\pgfsetrectcap%
\pgfsetroundjoin%
\pgfsetlinewidth{1.505625pt}%
\definecolor{currentstroke}{rgb}{1.000000,0.705882,0.509804}%
\pgfsetstrokecolor{currentstroke}%
\pgfsetstrokeopacity{0.800000}%
\pgfsetdash{}{0pt}%
\pgfpathmoveto{\pgfqpoint{2.409489in}{3.362377in}}%
\pgfpathlineto{\pgfqpoint{2.337997in}{2.565179in}}%
\pgfusepath{stroke}%
\end{pgfscope}%
\begin{pgfscope}%
\pgfpathrectangle{\pgfqpoint{0.481978in}{0.331635in}}{\pgfqpoint{4.960000in}{3.696000in}}%
\pgfusepath{clip}%
\pgfsetrectcap%
\pgfsetroundjoin%
\pgfsetlinewidth{1.505625pt}%
\definecolor{currentstroke}{rgb}{1.000000,0.705882,0.509804}%
\pgfsetstrokecolor{currentstroke}%
\pgfsetstrokeopacity{0.800000}%
\pgfsetdash{}{0pt}%
\pgfpathmoveto{\pgfqpoint{3.659565in}{3.085429in}}%
\pgfpathlineto{\pgfqpoint{2.337997in}{2.565179in}}%
\pgfusepath{stroke}%
\end{pgfscope}%
\begin{pgfscope}%
\pgfpathrectangle{\pgfqpoint{0.481978in}{0.331635in}}{\pgfqpoint{4.960000in}{3.696000in}}%
\pgfusepath{clip}%
\pgfsetrectcap%
\pgfsetroundjoin%
\pgfsetlinewidth{1.505625pt}%
\definecolor{currentstroke}{rgb}{1.000000,0.705882,0.509804}%
\pgfsetstrokecolor{currentstroke}%
\pgfsetstrokeopacity{0.800000}%
\pgfsetdash{}{0pt}%
\pgfpathmoveto{\pgfqpoint{1.402680in}{1.231116in}}%
\pgfpathlineto{\pgfqpoint{2.337997in}{2.565179in}}%
\pgfusepath{stroke}%
\end{pgfscope}%
\begin{pgfscope}%
\pgfpathrectangle{\pgfqpoint{0.481978in}{0.331635in}}{\pgfqpoint{4.960000in}{3.696000in}}%
\pgfusepath{clip}%
\pgfsetrectcap%
\pgfsetroundjoin%
\pgfsetlinewidth{1.505625pt}%
\definecolor{currentstroke}{rgb}{1.000000,0.705882,0.509804}%
\pgfsetstrokecolor{currentstroke}%
\pgfsetstrokeopacity{0.800000}%
\pgfsetdash{}{0pt}%
\pgfpathmoveto{\pgfqpoint{3.170837in}{2.471661in}}%
\pgfpathlineto{\pgfqpoint{2.337997in}{2.565179in}}%
\pgfusepath{stroke}%
\end{pgfscope}%
\begin{pgfscope}%
\pgfpathrectangle{\pgfqpoint{0.481978in}{0.331635in}}{\pgfqpoint{4.960000in}{3.696000in}}%
\pgfusepath{clip}%
\pgfsetrectcap%
\pgfsetroundjoin%
\pgfsetlinewidth{1.505625pt}%
\definecolor{currentstroke}{rgb}{1.000000,0.705882,0.509804}%
\pgfsetstrokecolor{currentstroke}%
\pgfsetstrokeopacity{0.800000}%
\pgfsetdash{}{0pt}%
\pgfpathmoveto{\pgfqpoint{1.834442in}{1.693329in}}%
\pgfpathlineto{\pgfqpoint{2.337997in}{2.565179in}}%
\pgfusepath{stroke}%
\end{pgfscope}%
\begin{pgfscope}%
\pgfpathrectangle{\pgfqpoint{0.481978in}{0.331635in}}{\pgfqpoint{4.960000in}{3.696000in}}%
\pgfusepath{clip}%
\pgfsetrectcap%
\pgfsetroundjoin%
\pgfsetlinewidth{1.505625pt}%
\definecolor{currentstroke}{rgb}{1.000000,0.705882,0.509804}%
\pgfsetstrokecolor{currentstroke}%
\pgfsetstrokeopacity{0.800000}%
\pgfsetdash{}{0pt}%
\pgfpathmoveto{\pgfqpoint{1.780214in}{2.842358in}}%
\pgfpathlineto{\pgfqpoint{2.337997in}{2.565179in}}%
\pgfusepath{stroke}%
\end{pgfscope}%
\begin{pgfscope}%
\pgfpathrectangle{\pgfqpoint{0.481978in}{0.331635in}}{\pgfqpoint{4.960000in}{3.696000in}}%
\pgfusepath{clip}%
\pgfsetrectcap%
\pgfsetroundjoin%
\pgfsetlinewidth{1.505625pt}%
\definecolor{currentstroke}{rgb}{1.000000,0.705882,0.509804}%
\pgfsetstrokecolor{currentstroke}%
\pgfsetstrokeopacity{0.800000}%
\pgfsetdash{}{0pt}%
\pgfpathmoveto{\pgfqpoint{2.618562in}{3.176315in}}%
\pgfpathlineto{\pgfqpoint{2.337997in}{2.565179in}}%
\pgfusepath{stroke}%
\end{pgfscope}%
\begin{pgfscope}%
\pgfpathrectangle{\pgfqpoint{0.481978in}{0.331635in}}{\pgfqpoint{4.960000in}{3.696000in}}%
\pgfusepath{clip}%
\pgfsetrectcap%
\pgfsetroundjoin%
\pgfsetlinewidth{1.505625pt}%
\definecolor{currentstroke}{rgb}{1.000000,0.705882,0.509804}%
\pgfsetstrokecolor{currentstroke}%
\pgfsetstrokeopacity{0.800000}%
\pgfsetdash{}{0pt}%
\pgfpathmoveto{\pgfqpoint{1.816511in}{1.956535in}}%
\pgfpathlineto{\pgfqpoint{2.337997in}{2.565179in}}%
\pgfusepath{stroke}%
\end{pgfscope}%
\begin{pgfscope}%
\pgfpathrectangle{\pgfqpoint{0.481978in}{0.331635in}}{\pgfqpoint{4.960000in}{3.696000in}}%
\pgfusepath{clip}%
\pgfsetrectcap%
\pgfsetroundjoin%
\pgfsetlinewidth{1.505625pt}%
\definecolor{currentstroke}{rgb}{1.000000,0.705882,0.509804}%
\pgfsetstrokecolor{currentstroke}%
\pgfsetstrokeopacity{0.800000}%
\pgfsetdash{}{0pt}%
\pgfpathmoveto{\pgfqpoint{2.749742in}{3.234979in}}%
\pgfpathlineto{\pgfqpoint{2.337997in}{2.565179in}}%
\pgfusepath{stroke}%
\end{pgfscope}%
\begin{pgfscope}%
\pgfpathrectangle{\pgfqpoint{0.481978in}{0.331635in}}{\pgfqpoint{4.960000in}{3.696000in}}%
\pgfusepath{clip}%
\pgfsetrectcap%
\pgfsetroundjoin%
\pgfsetlinewidth{1.505625pt}%
\definecolor{currentstroke}{rgb}{1.000000,0.705882,0.509804}%
\pgfsetstrokecolor{currentstroke}%
\pgfsetstrokeopacity{0.800000}%
\pgfsetdash{}{0pt}%
\pgfpathmoveto{\pgfqpoint{1.855378in}{1.723142in}}%
\pgfpathlineto{\pgfqpoint{2.337997in}{2.565179in}}%
\pgfusepath{stroke}%
\end{pgfscope}%
\begin{pgfscope}%
\pgfpathrectangle{\pgfqpoint{0.481978in}{0.331635in}}{\pgfqpoint{4.960000in}{3.696000in}}%
\pgfusepath{clip}%
\pgfsetrectcap%
\pgfsetroundjoin%
\pgfsetlinewidth{1.505625pt}%
\definecolor{currentstroke}{rgb}{1.000000,0.705882,0.509804}%
\pgfsetstrokecolor{currentstroke}%
\pgfsetstrokeopacity{0.800000}%
\pgfsetdash{}{0pt}%
\pgfpathmoveto{\pgfqpoint{0.830402in}{2.329653in}}%
\pgfpathlineto{\pgfqpoint{2.337997in}{2.565179in}}%
\pgfusepath{stroke}%
\end{pgfscope}%
\begin{pgfscope}%
\pgfpathrectangle{\pgfqpoint{0.481978in}{0.331635in}}{\pgfqpoint{4.960000in}{3.696000in}}%
\pgfusepath{clip}%
\pgfsetrectcap%
\pgfsetroundjoin%
\pgfsetlinewidth{1.505625pt}%
\definecolor{currentstroke}{rgb}{1.000000,0.705882,0.509804}%
\pgfsetstrokecolor{currentstroke}%
\pgfsetstrokeopacity{0.800000}%
\pgfsetdash{}{0pt}%
\pgfpathmoveto{\pgfqpoint{2.483847in}{2.563253in}}%
\pgfpathlineto{\pgfqpoint{2.337997in}{2.565179in}}%
\pgfusepath{stroke}%
\end{pgfscope}%
\begin{pgfscope}%
\pgfpathrectangle{\pgfqpoint{0.481978in}{0.331635in}}{\pgfqpoint{4.960000in}{3.696000in}}%
\pgfusepath{clip}%
\pgfsetrectcap%
\pgfsetroundjoin%
\pgfsetlinewidth{1.505625pt}%
\definecolor{currentstroke}{rgb}{1.000000,0.705882,0.509804}%
\pgfsetstrokecolor{currentstroke}%
\pgfsetstrokeopacity{0.800000}%
\pgfsetdash{}{0pt}%
\pgfpathmoveto{\pgfqpoint{2.443354in}{2.830036in}}%
\pgfpathlineto{\pgfqpoint{2.337997in}{2.565179in}}%
\pgfusepath{stroke}%
\end{pgfscope}%
\begin{pgfscope}%
\pgfpathrectangle{\pgfqpoint{0.481978in}{0.331635in}}{\pgfqpoint{4.960000in}{3.696000in}}%
\pgfusepath{clip}%
\pgfsetrectcap%
\pgfsetroundjoin%
\pgfsetlinewidth{1.505625pt}%
\definecolor{currentstroke}{rgb}{1.000000,0.705882,0.509804}%
\pgfsetstrokecolor{currentstroke}%
\pgfsetstrokeopacity{0.800000}%
\pgfsetdash{}{0pt}%
\pgfpathmoveto{\pgfqpoint{4.243695in}{0.755209in}}%
\pgfpathlineto{\pgfqpoint{2.337997in}{2.565179in}}%
\pgfusepath{stroke}%
\end{pgfscope}%
\begin{pgfscope}%
\pgfpathrectangle{\pgfqpoint{0.481978in}{0.331635in}}{\pgfqpoint{4.960000in}{3.696000in}}%
\pgfusepath{clip}%
\pgfsetrectcap%
\pgfsetroundjoin%
\pgfsetlinewidth{1.505625pt}%
\definecolor{currentstroke}{rgb}{1.000000,0.705882,0.509804}%
\pgfsetstrokecolor{currentstroke}%
\pgfsetstrokeopacity{0.800000}%
\pgfsetdash{}{0pt}%
\pgfpathmoveto{\pgfqpoint{1.558818in}{1.260127in}}%
\pgfpathlineto{\pgfqpoint{2.337997in}{2.565179in}}%
\pgfusepath{stroke}%
\end{pgfscope}%
\begin{pgfscope}%
\pgfpathrectangle{\pgfqpoint{0.481978in}{0.331635in}}{\pgfqpoint{4.960000in}{3.696000in}}%
\pgfusepath{clip}%
\pgfsetrectcap%
\pgfsetroundjoin%
\pgfsetlinewidth{1.505625pt}%
\definecolor{currentstroke}{rgb}{1.000000,0.705882,0.509804}%
\pgfsetstrokecolor{currentstroke}%
\pgfsetstrokeopacity{0.800000}%
\pgfsetdash{}{0pt}%
\pgfpathmoveto{\pgfqpoint{1.915714in}{3.799997in}}%
\pgfpathlineto{\pgfqpoint{2.337997in}{2.565179in}}%
\pgfusepath{stroke}%
\end{pgfscope}%
\begin{pgfscope}%
\pgfpathrectangle{\pgfqpoint{0.481978in}{0.331635in}}{\pgfqpoint{4.960000in}{3.696000in}}%
\pgfusepath{clip}%
\pgfsetrectcap%
\pgfsetroundjoin%
\pgfsetlinewidth{1.505625pt}%
\definecolor{currentstroke}{rgb}{1.000000,0.705882,0.509804}%
\pgfsetstrokecolor{currentstroke}%
\pgfsetstrokeopacity{0.800000}%
\pgfsetdash{}{0pt}%
\pgfpathmoveto{\pgfqpoint{2.053323in}{3.325355in}}%
\pgfpathlineto{\pgfqpoint{2.337997in}{2.565179in}}%
\pgfusepath{stroke}%
\end{pgfscope}%
\begin{pgfscope}%
\pgfpathrectangle{\pgfqpoint{0.481978in}{0.331635in}}{\pgfqpoint{4.960000in}{3.696000in}}%
\pgfusepath{clip}%
\pgfsetrectcap%
\pgfsetroundjoin%
\pgfsetlinewidth{1.505625pt}%
\definecolor{currentstroke}{rgb}{1.000000,0.705882,0.509804}%
\pgfsetstrokecolor{currentstroke}%
\pgfsetstrokeopacity{0.800000}%
\pgfsetdash{}{0pt}%
\pgfpathmoveto{\pgfqpoint{2.331227in}{2.014644in}}%
\pgfpathlineto{\pgfqpoint{2.337997in}{2.565179in}}%
\pgfusepath{stroke}%
\end{pgfscope}%
\begin{pgfscope}%
\pgfpathrectangle{\pgfqpoint{0.481978in}{0.331635in}}{\pgfqpoint{4.960000in}{3.696000in}}%
\pgfusepath{clip}%
\pgfsetrectcap%
\pgfsetroundjoin%
\pgfsetlinewidth{1.505625pt}%
\definecolor{currentstroke}{rgb}{1.000000,0.705882,0.509804}%
\pgfsetstrokecolor{currentstroke}%
\pgfsetstrokeopacity{0.800000}%
\pgfsetdash{}{0pt}%
\pgfpathmoveto{\pgfqpoint{2.069873in}{2.688424in}}%
\pgfpathlineto{\pgfqpoint{2.337997in}{2.565179in}}%
\pgfusepath{stroke}%
\end{pgfscope}%
\begin{pgfscope}%
\pgfpathrectangle{\pgfqpoint{0.481978in}{0.331635in}}{\pgfqpoint{4.960000in}{3.696000in}}%
\pgfusepath{clip}%
\pgfsetrectcap%
\pgfsetroundjoin%
\pgfsetlinewidth{1.505625pt}%
\definecolor{currentstroke}{rgb}{1.000000,0.705882,0.509804}%
\pgfsetstrokecolor{currentstroke}%
\pgfsetstrokeopacity{0.800000}%
\pgfsetdash{}{0pt}%
\pgfpathmoveto{\pgfqpoint{3.381662in}{2.444163in}}%
\pgfpathlineto{\pgfqpoint{2.337997in}{2.565179in}}%
\pgfusepath{stroke}%
\end{pgfscope}%
\begin{pgfscope}%
\pgfpathrectangle{\pgfqpoint{0.481978in}{0.331635in}}{\pgfqpoint{4.960000in}{3.696000in}}%
\pgfusepath{clip}%
\pgfsetrectcap%
\pgfsetroundjoin%
\pgfsetlinewidth{1.505625pt}%
\definecolor{currentstroke}{rgb}{1.000000,0.705882,0.509804}%
\pgfsetstrokecolor{currentstroke}%
\pgfsetstrokeopacity{0.800000}%
\pgfsetdash{}{0pt}%
\pgfpathmoveto{\pgfqpoint{2.324071in}{2.612606in}}%
\pgfpathlineto{\pgfqpoint{2.337997in}{2.565179in}}%
\pgfusepath{stroke}%
\end{pgfscope}%
\begin{pgfscope}%
\pgfpathrectangle{\pgfqpoint{0.481978in}{0.331635in}}{\pgfqpoint{4.960000in}{3.696000in}}%
\pgfusepath{clip}%
\pgfsetrectcap%
\pgfsetroundjoin%
\pgfsetlinewidth{1.505625pt}%
\definecolor{currentstroke}{rgb}{1.000000,0.705882,0.509804}%
\pgfsetstrokecolor{currentstroke}%
\pgfsetstrokeopacity{0.800000}%
\pgfsetdash{}{0pt}%
\pgfpathmoveto{\pgfqpoint{1.297191in}{1.932854in}}%
\pgfpathlineto{\pgfqpoint{2.337997in}{2.565179in}}%
\pgfusepath{stroke}%
\end{pgfscope}%
\begin{pgfscope}%
\pgfpathrectangle{\pgfqpoint{0.481978in}{0.331635in}}{\pgfqpoint{4.960000in}{3.696000in}}%
\pgfusepath{clip}%
\pgfsetrectcap%
\pgfsetroundjoin%
\pgfsetlinewidth{1.505625pt}%
\definecolor{currentstroke}{rgb}{1.000000,0.705882,0.509804}%
\pgfsetstrokecolor{currentstroke}%
\pgfsetstrokeopacity{0.800000}%
\pgfsetdash{}{0pt}%
\pgfpathmoveto{\pgfqpoint{4.801011in}{1.262232in}}%
\pgfpathlineto{\pgfqpoint{2.337997in}{2.565179in}}%
\pgfusepath{stroke}%
\end{pgfscope}%
\begin{pgfscope}%
\pgfpathrectangle{\pgfqpoint{0.481978in}{0.331635in}}{\pgfqpoint{4.960000in}{3.696000in}}%
\pgfusepath{clip}%
\pgfsetrectcap%
\pgfsetroundjoin%
\pgfsetlinewidth{1.505625pt}%
\definecolor{currentstroke}{rgb}{1.000000,0.705882,0.509804}%
\pgfsetstrokecolor{currentstroke}%
\pgfsetstrokeopacity{0.800000}%
\pgfsetdash{}{0pt}%
\pgfpathmoveto{\pgfqpoint{1.381868in}{1.952582in}}%
\pgfpathlineto{\pgfqpoint{2.337997in}{2.565179in}}%
\pgfusepath{stroke}%
\end{pgfscope}%
\begin{pgfscope}%
\pgfpathrectangle{\pgfqpoint{0.481978in}{0.331635in}}{\pgfqpoint{4.960000in}{3.696000in}}%
\pgfusepath{clip}%
\pgfsetrectcap%
\pgfsetroundjoin%
\pgfsetlinewidth{1.505625pt}%
\definecolor{currentstroke}{rgb}{1.000000,0.705882,0.509804}%
\pgfsetstrokecolor{currentstroke}%
\pgfsetstrokeopacity{0.800000}%
\pgfsetdash{}{0pt}%
\pgfpathmoveto{\pgfqpoint{3.112049in}{2.225805in}}%
\pgfpathlineto{\pgfqpoint{2.337997in}{2.565179in}}%
\pgfusepath{stroke}%
\end{pgfscope}%
\begin{pgfscope}%
\pgfpathrectangle{\pgfqpoint{0.481978in}{0.331635in}}{\pgfqpoint{4.960000in}{3.696000in}}%
\pgfusepath{clip}%
\pgfsetrectcap%
\pgfsetroundjoin%
\pgfsetlinewidth{1.505625pt}%
\definecolor{currentstroke}{rgb}{1.000000,0.705882,0.509804}%
\pgfsetstrokecolor{currentstroke}%
\pgfsetstrokeopacity{0.800000}%
\pgfsetdash{}{0pt}%
\pgfpathmoveto{\pgfqpoint{2.263955in}{2.400783in}}%
\pgfpathlineto{\pgfqpoint{2.337997in}{2.565179in}}%
\pgfusepath{stroke}%
\end{pgfscope}%
\begin{pgfscope}%
\pgfpathrectangle{\pgfqpoint{0.481978in}{0.331635in}}{\pgfqpoint{4.960000in}{3.696000in}}%
\pgfusepath{clip}%
\pgfsetrectcap%
\pgfsetroundjoin%
\pgfsetlinewidth{1.505625pt}%
\definecolor{currentstroke}{rgb}{1.000000,0.705882,0.509804}%
\pgfsetstrokecolor{currentstroke}%
\pgfsetstrokeopacity{0.800000}%
\pgfsetdash{}{0pt}%
\pgfpathmoveto{\pgfqpoint{1.430664in}{1.770420in}}%
\pgfpathlineto{\pgfqpoint{2.337997in}{2.565179in}}%
\pgfusepath{stroke}%
\end{pgfscope}%
\begin{pgfscope}%
\pgfpathrectangle{\pgfqpoint{0.481978in}{0.331635in}}{\pgfqpoint{4.960000in}{3.696000in}}%
\pgfusepath{clip}%
\pgfsetrectcap%
\pgfsetroundjoin%
\pgfsetlinewidth{1.505625pt}%
\definecolor{currentstroke}{rgb}{1.000000,0.705882,0.509804}%
\pgfsetstrokecolor{currentstroke}%
\pgfsetstrokeopacity{0.800000}%
\pgfsetdash{}{0pt}%
\pgfpathmoveto{\pgfqpoint{1.786845in}{2.996271in}}%
\pgfpathlineto{\pgfqpoint{2.337997in}{2.565179in}}%
\pgfusepath{stroke}%
\end{pgfscope}%
\begin{pgfscope}%
\pgfpathrectangle{\pgfqpoint{0.481978in}{0.331635in}}{\pgfqpoint{4.960000in}{3.696000in}}%
\pgfusepath{clip}%
\pgfsetrectcap%
\pgfsetroundjoin%
\pgfsetlinewidth{1.505625pt}%
\definecolor{currentstroke}{rgb}{1.000000,0.705882,0.509804}%
\pgfsetstrokecolor{currentstroke}%
\pgfsetstrokeopacity{0.800000}%
\pgfsetdash{}{0pt}%
\pgfpathmoveto{\pgfqpoint{2.553548in}{2.977704in}}%
\pgfpathlineto{\pgfqpoint{2.337997in}{2.565179in}}%
\pgfusepath{stroke}%
\end{pgfscope}%
\begin{pgfscope}%
\pgfpathrectangle{\pgfqpoint{0.481978in}{0.331635in}}{\pgfqpoint{4.960000in}{3.696000in}}%
\pgfusepath{clip}%
\pgfsetrectcap%
\pgfsetroundjoin%
\pgfsetlinewidth{1.505625pt}%
\definecolor{currentstroke}{rgb}{1.000000,0.705882,0.509804}%
\pgfsetstrokecolor{currentstroke}%
\pgfsetstrokeopacity{0.800000}%
\pgfsetdash{}{0pt}%
\pgfpathmoveto{\pgfqpoint{3.143878in}{2.097035in}}%
\pgfpathlineto{\pgfqpoint{2.337997in}{2.565179in}}%
\pgfusepath{stroke}%
\end{pgfscope}%
\begin{pgfscope}%
\pgfpathrectangle{\pgfqpoint{0.481978in}{0.331635in}}{\pgfqpoint{4.960000in}{3.696000in}}%
\pgfusepath{clip}%
\pgfsetrectcap%
\pgfsetroundjoin%
\pgfsetlinewidth{1.505625pt}%
\definecolor{currentstroke}{rgb}{1.000000,0.705882,0.509804}%
\pgfsetstrokecolor{currentstroke}%
\pgfsetstrokeopacity{0.800000}%
\pgfsetdash{}{0pt}%
\pgfpathmoveto{\pgfqpoint{1.510599in}{1.640489in}}%
\pgfpathlineto{\pgfqpoint{2.337997in}{2.565179in}}%
\pgfusepath{stroke}%
\end{pgfscope}%
\begin{pgfscope}%
\pgfpathrectangle{\pgfqpoint{0.481978in}{0.331635in}}{\pgfqpoint{4.960000in}{3.696000in}}%
\pgfusepath{clip}%
\pgfsetrectcap%
\pgfsetroundjoin%
\pgfsetlinewidth{1.505625pt}%
\definecolor{currentstroke}{rgb}{1.000000,0.705882,0.509804}%
\pgfsetstrokecolor{currentstroke}%
\pgfsetstrokeopacity{0.800000}%
\pgfsetdash{}{0pt}%
\pgfpathmoveto{\pgfqpoint{1.412036in}{2.494661in}}%
\pgfpathlineto{\pgfqpoint{2.337997in}{2.565179in}}%
\pgfusepath{stroke}%
\end{pgfscope}%
\begin{pgfscope}%
\pgfpathrectangle{\pgfqpoint{0.481978in}{0.331635in}}{\pgfqpoint{4.960000in}{3.696000in}}%
\pgfusepath{clip}%
\pgfsetrectcap%
\pgfsetroundjoin%
\pgfsetlinewidth{1.505625pt}%
\definecolor{currentstroke}{rgb}{1.000000,0.705882,0.509804}%
\pgfsetstrokecolor{currentstroke}%
\pgfsetstrokeopacity{0.800000}%
\pgfsetdash{}{0pt}%
\pgfpathmoveto{\pgfqpoint{2.580200in}{3.659236in}}%
\pgfpathlineto{\pgfqpoint{2.337997in}{2.565179in}}%
\pgfusepath{stroke}%
\end{pgfscope}%
\begin{pgfscope}%
\pgfpathrectangle{\pgfqpoint{0.481978in}{0.331635in}}{\pgfqpoint{4.960000in}{3.696000in}}%
\pgfusepath{clip}%
\pgfsetrectcap%
\pgfsetroundjoin%
\pgfsetlinewidth{1.505625pt}%
\definecolor{currentstroke}{rgb}{1.000000,0.705882,0.509804}%
\pgfsetstrokecolor{currentstroke}%
\pgfsetstrokeopacity{0.800000}%
\pgfsetdash{}{0pt}%
\pgfpathmoveto{\pgfqpoint{1.212889in}{1.685398in}}%
\pgfpathlineto{\pgfqpoint{2.337997in}{2.565179in}}%
\pgfusepath{stroke}%
\end{pgfscope}%
\begin{pgfscope}%
\pgfpathrectangle{\pgfqpoint{0.481978in}{0.331635in}}{\pgfqpoint{4.960000in}{3.696000in}}%
\pgfusepath{clip}%
\pgfsetrectcap%
\pgfsetroundjoin%
\pgfsetlinewidth{1.505625pt}%
\definecolor{currentstroke}{rgb}{1.000000,0.705882,0.509804}%
\pgfsetstrokecolor{currentstroke}%
\pgfsetstrokeopacity{0.800000}%
\pgfsetdash{}{0pt}%
\pgfpathmoveto{\pgfqpoint{1.333944in}{3.052950in}}%
\pgfpathlineto{\pgfqpoint{2.337997in}{2.565179in}}%
\pgfusepath{stroke}%
\end{pgfscope}%
\begin{pgfscope}%
\pgfpathrectangle{\pgfqpoint{0.481978in}{0.331635in}}{\pgfqpoint{4.960000in}{3.696000in}}%
\pgfusepath{clip}%
\pgfsetrectcap%
\pgfsetroundjoin%
\pgfsetlinewidth{1.505625pt}%
\definecolor{currentstroke}{rgb}{1.000000,0.705882,0.509804}%
\pgfsetstrokecolor{currentstroke}%
\pgfsetstrokeopacity{0.800000}%
\pgfsetdash{}{0pt}%
\pgfpathmoveto{\pgfqpoint{1.910611in}{2.979947in}}%
\pgfpathlineto{\pgfqpoint{2.337997in}{2.565179in}}%
\pgfusepath{stroke}%
\end{pgfscope}%
\begin{pgfscope}%
\pgfpathrectangle{\pgfqpoint{0.481978in}{0.331635in}}{\pgfqpoint{4.960000in}{3.696000in}}%
\pgfusepath{clip}%
\pgfsetrectcap%
\pgfsetroundjoin%
\pgfsetlinewidth{1.505625pt}%
\definecolor{currentstroke}{rgb}{1.000000,0.705882,0.509804}%
\pgfsetstrokecolor{currentstroke}%
\pgfsetstrokeopacity{0.800000}%
\pgfsetdash{}{0pt}%
\pgfpathmoveto{\pgfqpoint{1.546438in}{2.985777in}}%
\pgfpathlineto{\pgfqpoint{2.337997in}{2.565179in}}%
\pgfusepath{stroke}%
\end{pgfscope}%
\begin{pgfscope}%
\pgfpathrectangle{\pgfqpoint{0.481978in}{0.331635in}}{\pgfqpoint{4.960000in}{3.696000in}}%
\pgfusepath{clip}%
\pgfsetrectcap%
\pgfsetroundjoin%
\pgfsetlinewidth{1.505625pt}%
\definecolor{currentstroke}{rgb}{1.000000,0.705882,0.509804}%
\pgfsetstrokecolor{currentstroke}%
\pgfsetstrokeopacity{0.800000}%
\pgfsetdash{}{0pt}%
\pgfpathmoveto{\pgfqpoint{2.026229in}{2.583514in}}%
\pgfpathlineto{\pgfqpoint{2.337997in}{2.565179in}}%
\pgfusepath{stroke}%
\end{pgfscope}%
\begin{pgfscope}%
\pgfpathrectangle{\pgfqpoint{0.481978in}{0.331635in}}{\pgfqpoint{4.960000in}{3.696000in}}%
\pgfusepath{clip}%
\pgfsetrectcap%
\pgfsetroundjoin%
\pgfsetlinewidth{1.505625pt}%
\definecolor{currentstroke}{rgb}{1.000000,0.705882,0.509804}%
\pgfsetstrokecolor{currentstroke}%
\pgfsetstrokeopacity{0.800000}%
\pgfsetdash{}{0pt}%
\pgfpathmoveto{\pgfqpoint{2.497138in}{1.871651in}}%
\pgfpathlineto{\pgfqpoint{2.337997in}{2.565179in}}%
\pgfusepath{stroke}%
\end{pgfscope}%
\begin{pgfscope}%
\pgfpathrectangle{\pgfqpoint{0.481978in}{0.331635in}}{\pgfqpoint{4.960000in}{3.696000in}}%
\pgfusepath{clip}%
\pgfsetrectcap%
\pgfsetroundjoin%
\pgfsetlinewidth{1.505625pt}%
\definecolor{currentstroke}{rgb}{1.000000,0.705882,0.509804}%
\pgfsetstrokecolor{currentstroke}%
\pgfsetstrokeopacity{0.800000}%
\pgfsetdash{}{0pt}%
\pgfpathmoveto{\pgfqpoint{2.724487in}{2.968073in}}%
\pgfpathlineto{\pgfqpoint{2.337997in}{2.565179in}}%
\pgfusepath{stroke}%
\end{pgfscope}%
\begin{pgfscope}%
\pgfpathrectangle{\pgfqpoint{0.481978in}{0.331635in}}{\pgfqpoint{4.960000in}{3.696000in}}%
\pgfusepath{clip}%
\pgfsetrectcap%
\pgfsetroundjoin%
\pgfsetlinewidth{1.505625pt}%
\definecolor{currentstroke}{rgb}{1.000000,0.705882,0.509804}%
\pgfsetstrokecolor{currentstroke}%
\pgfsetstrokeopacity{0.800000}%
\pgfsetdash{}{0pt}%
\pgfpathmoveto{\pgfqpoint{1.592176in}{3.078204in}}%
\pgfpathlineto{\pgfqpoint{2.337997in}{2.565179in}}%
\pgfusepath{stroke}%
\end{pgfscope}%
\begin{pgfscope}%
\pgfpathrectangle{\pgfqpoint{0.481978in}{0.331635in}}{\pgfqpoint{4.960000in}{3.696000in}}%
\pgfusepath{clip}%
\pgfsetrectcap%
\pgfsetroundjoin%
\pgfsetlinewidth{1.505625pt}%
\definecolor{currentstroke}{rgb}{1.000000,0.705882,0.509804}%
\pgfsetstrokecolor{currentstroke}%
\pgfsetstrokeopacity{0.800000}%
\pgfsetdash{}{0pt}%
\pgfpathmoveto{\pgfqpoint{1.230798in}{1.479973in}}%
\pgfpathlineto{\pgfqpoint{2.337997in}{2.565179in}}%
\pgfusepath{stroke}%
\end{pgfscope}%
\begin{pgfscope}%
\pgfpathrectangle{\pgfqpoint{0.481978in}{0.331635in}}{\pgfqpoint{4.960000in}{3.696000in}}%
\pgfusepath{clip}%
\pgfsetrectcap%
\pgfsetroundjoin%
\pgfsetlinewidth{1.505625pt}%
\definecolor{currentstroke}{rgb}{1.000000,0.705882,0.509804}%
\pgfsetstrokecolor{currentstroke}%
\pgfsetstrokeopacity{0.800000}%
\pgfsetdash{}{0pt}%
\pgfpathmoveto{\pgfqpoint{1.742469in}{2.482082in}}%
\pgfpathlineto{\pgfqpoint{2.337997in}{2.565179in}}%
\pgfusepath{stroke}%
\end{pgfscope}%
\begin{pgfscope}%
\pgfpathrectangle{\pgfqpoint{0.481978in}{0.331635in}}{\pgfqpoint{4.960000in}{3.696000in}}%
\pgfusepath{clip}%
\pgfsetrectcap%
\pgfsetroundjoin%
\pgfsetlinewidth{1.505625pt}%
\definecolor{currentstroke}{rgb}{1.000000,0.705882,0.509804}%
\pgfsetstrokecolor{currentstroke}%
\pgfsetstrokeopacity{0.800000}%
\pgfsetdash{}{0pt}%
\pgfpathmoveto{\pgfqpoint{1.926850in}{3.796702in}}%
\pgfpathlineto{\pgfqpoint{2.337997in}{2.565179in}}%
\pgfusepath{stroke}%
\end{pgfscope}%
\begin{pgfscope}%
\pgfpathrectangle{\pgfqpoint{0.481978in}{0.331635in}}{\pgfqpoint{4.960000in}{3.696000in}}%
\pgfusepath{clip}%
\pgfsetrectcap%
\pgfsetroundjoin%
\pgfsetlinewidth{1.505625pt}%
\definecolor{currentstroke}{rgb}{1.000000,0.705882,0.509804}%
\pgfsetstrokecolor{currentstroke}%
\pgfsetstrokeopacity{0.800000}%
\pgfsetdash{}{0pt}%
\pgfpathmoveto{\pgfqpoint{3.551106in}{3.445906in}}%
\pgfpathlineto{\pgfqpoint{2.337997in}{2.565179in}}%
\pgfusepath{stroke}%
\end{pgfscope}%
\begin{pgfscope}%
\pgfpathrectangle{\pgfqpoint{0.481978in}{0.331635in}}{\pgfqpoint{4.960000in}{3.696000in}}%
\pgfusepath{clip}%
\pgfsetrectcap%
\pgfsetroundjoin%
\pgfsetlinewidth{1.505625pt}%
\definecolor{currentstroke}{rgb}{1.000000,0.705882,0.509804}%
\pgfsetstrokecolor{currentstroke}%
\pgfsetstrokeopacity{0.800000}%
\pgfsetdash{}{0pt}%
\pgfpathmoveto{\pgfqpoint{1.414391in}{1.583135in}}%
\pgfpathlineto{\pgfqpoint{2.337997in}{2.565179in}}%
\pgfusepath{stroke}%
\end{pgfscope}%
\begin{pgfscope}%
\pgfpathrectangle{\pgfqpoint{0.481978in}{0.331635in}}{\pgfqpoint{4.960000in}{3.696000in}}%
\pgfusepath{clip}%
\pgfsetrectcap%
\pgfsetroundjoin%
\pgfsetlinewidth{1.505625pt}%
\definecolor{currentstroke}{rgb}{1.000000,0.705882,0.509804}%
\pgfsetstrokecolor{currentstroke}%
\pgfsetstrokeopacity{0.800000}%
\pgfsetdash{}{0pt}%
\pgfpathmoveto{\pgfqpoint{2.597999in}{3.794659in}}%
\pgfpathlineto{\pgfqpoint{2.337997in}{2.565179in}}%
\pgfusepath{stroke}%
\end{pgfscope}%
\begin{pgfscope}%
\pgfpathrectangle{\pgfqpoint{0.481978in}{0.331635in}}{\pgfqpoint{4.960000in}{3.696000in}}%
\pgfusepath{clip}%
\pgfsetrectcap%
\pgfsetroundjoin%
\pgfsetlinewidth{1.505625pt}%
\definecolor{currentstroke}{rgb}{1.000000,0.705882,0.509804}%
\pgfsetstrokecolor{currentstroke}%
\pgfsetstrokeopacity{0.800000}%
\pgfsetdash{}{0pt}%
\pgfpathmoveto{\pgfqpoint{1.145042in}{3.232604in}}%
\pgfpathlineto{\pgfqpoint{2.337997in}{2.565179in}}%
\pgfusepath{stroke}%
\end{pgfscope}%
\begin{pgfscope}%
\pgfpathrectangle{\pgfqpoint{0.481978in}{0.331635in}}{\pgfqpoint{4.960000in}{3.696000in}}%
\pgfusepath{clip}%
\pgfsetrectcap%
\pgfsetroundjoin%
\pgfsetlinewidth{1.505625pt}%
\definecolor{currentstroke}{rgb}{1.000000,0.705882,0.509804}%
\pgfsetstrokecolor{currentstroke}%
\pgfsetstrokeopacity{0.800000}%
\pgfsetdash{}{0pt}%
\pgfpathmoveto{\pgfqpoint{2.661184in}{2.017521in}}%
\pgfpathlineto{\pgfqpoint{2.337997in}{2.565179in}}%
\pgfusepath{stroke}%
\end{pgfscope}%
\begin{pgfscope}%
\pgfpathrectangle{\pgfqpoint{0.481978in}{0.331635in}}{\pgfqpoint{4.960000in}{3.696000in}}%
\pgfusepath{clip}%
\pgfsetrectcap%
\pgfsetroundjoin%
\pgfsetlinewidth{1.505625pt}%
\definecolor{currentstroke}{rgb}{1.000000,0.705882,0.509804}%
\pgfsetstrokecolor{currentstroke}%
\pgfsetstrokeopacity{0.800000}%
\pgfsetdash{}{0pt}%
\pgfpathmoveto{\pgfqpoint{1.054739in}{2.508381in}}%
\pgfpathlineto{\pgfqpoint{2.337997in}{2.565179in}}%
\pgfusepath{stroke}%
\end{pgfscope}%
\begin{pgfscope}%
\pgfpathrectangle{\pgfqpoint{0.481978in}{0.331635in}}{\pgfqpoint{4.960000in}{3.696000in}}%
\pgfusepath{clip}%
\pgfsetrectcap%
\pgfsetroundjoin%
\pgfsetlinewidth{1.505625pt}%
\definecolor{currentstroke}{rgb}{1.000000,0.705882,0.509804}%
\pgfsetstrokecolor{currentstroke}%
\pgfsetstrokeopacity{0.800000}%
\pgfsetdash{}{0pt}%
\pgfpathmoveto{\pgfqpoint{1.362990in}{3.405351in}}%
\pgfpathlineto{\pgfqpoint{2.337997in}{2.565179in}}%
\pgfusepath{stroke}%
\end{pgfscope}%
\begin{pgfscope}%
\pgfpathrectangle{\pgfqpoint{0.481978in}{0.331635in}}{\pgfqpoint{4.960000in}{3.696000in}}%
\pgfusepath{clip}%
\pgfsetrectcap%
\pgfsetroundjoin%
\pgfsetlinewidth{1.505625pt}%
\definecolor{currentstroke}{rgb}{1.000000,0.705882,0.509804}%
\pgfsetstrokecolor{currentstroke}%
\pgfsetstrokeopacity{0.800000}%
\pgfsetdash{}{0pt}%
\pgfpathmoveto{\pgfqpoint{1.473426in}{3.271849in}}%
\pgfpathlineto{\pgfqpoint{2.337997in}{2.565179in}}%
\pgfusepath{stroke}%
\end{pgfscope}%
\begin{pgfscope}%
\pgfpathrectangle{\pgfqpoint{0.481978in}{0.331635in}}{\pgfqpoint{4.960000in}{3.696000in}}%
\pgfusepath{clip}%
\pgfsetrectcap%
\pgfsetroundjoin%
\pgfsetlinewidth{1.505625pt}%
\definecolor{currentstroke}{rgb}{1.000000,0.705882,0.509804}%
\pgfsetstrokecolor{currentstroke}%
\pgfsetstrokeopacity{0.800000}%
\pgfsetdash{}{0pt}%
\pgfpathmoveto{\pgfqpoint{3.352810in}{2.056948in}}%
\pgfpathlineto{\pgfqpoint{2.337997in}{2.565179in}}%
\pgfusepath{stroke}%
\end{pgfscope}%
\begin{pgfscope}%
\pgfpathrectangle{\pgfqpoint{0.481978in}{0.331635in}}{\pgfqpoint{4.960000in}{3.696000in}}%
\pgfusepath{clip}%
\pgfsetrectcap%
\pgfsetroundjoin%
\pgfsetlinewidth{1.505625pt}%
\definecolor{currentstroke}{rgb}{1.000000,0.705882,0.509804}%
\pgfsetstrokecolor{currentstroke}%
\pgfsetstrokeopacity{0.800000}%
\pgfsetdash{}{0pt}%
\pgfpathmoveto{\pgfqpoint{5.216523in}{2.587469in}}%
\pgfpathlineto{\pgfqpoint{2.337997in}{2.565179in}}%
\pgfusepath{stroke}%
\end{pgfscope}%
\begin{pgfscope}%
\pgfpathrectangle{\pgfqpoint{0.481978in}{0.331635in}}{\pgfqpoint{4.960000in}{3.696000in}}%
\pgfusepath{clip}%
\pgfsetrectcap%
\pgfsetroundjoin%
\pgfsetlinewidth{1.505625pt}%
\definecolor{currentstroke}{rgb}{1.000000,0.705882,0.509804}%
\pgfsetstrokecolor{currentstroke}%
\pgfsetstrokeopacity{0.800000}%
\pgfsetdash{}{0pt}%
\pgfpathmoveto{\pgfqpoint{2.975880in}{2.236807in}}%
\pgfpathlineto{\pgfqpoint{2.337997in}{2.565179in}}%
\pgfusepath{stroke}%
\end{pgfscope}%
\begin{pgfscope}%
\pgfpathrectangle{\pgfqpoint{0.481978in}{0.331635in}}{\pgfqpoint{4.960000in}{3.696000in}}%
\pgfusepath{clip}%
\pgfsetrectcap%
\pgfsetroundjoin%
\pgfsetlinewidth{1.505625pt}%
\definecolor{currentstroke}{rgb}{1.000000,0.705882,0.509804}%
\pgfsetstrokecolor{currentstroke}%
\pgfsetstrokeopacity{0.800000}%
\pgfsetdash{}{0pt}%
\pgfpathmoveto{\pgfqpoint{5.097947in}{1.939453in}}%
\pgfpathlineto{\pgfqpoint{2.337997in}{2.565179in}}%
\pgfusepath{stroke}%
\end{pgfscope}%
\begin{pgfscope}%
\pgfpathrectangle{\pgfqpoint{0.481978in}{0.331635in}}{\pgfqpoint{4.960000in}{3.696000in}}%
\pgfusepath{clip}%
\pgfsetrectcap%
\pgfsetroundjoin%
\pgfsetlinewidth{1.505625pt}%
\definecolor{currentstroke}{rgb}{1.000000,0.705882,0.509804}%
\pgfsetstrokecolor{currentstroke}%
\pgfsetstrokeopacity{0.800000}%
\pgfsetdash{}{0pt}%
\pgfpathmoveto{\pgfqpoint{2.202056in}{1.639163in}}%
\pgfpathlineto{\pgfqpoint{2.337997in}{2.565179in}}%
\pgfusepath{stroke}%
\end{pgfscope}%
\begin{pgfscope}%
\pgfpathrectangle{\pgfqpoint{0.481978in}{0.331635in}}{\pgfqpoint{4.960000in}{3.696000in}}%
\pgfusepath{clip}%
\pgfsetrectcap%
\pgfsetroundjoin%
\pgfsetlinewidth{1.505625pt}%
\definecolor{currentstroke}{rgb}{1.000000,0.705882,0.509804}%
\pgfsetstrokecolor{currentstroke}%
\pgfsetstrokeopacity{0.800000}%
\pgfsetdash{}{0pt}%
\pgfpathmoveto{\pgfqpoint{2.355362in}{3.358055in}}%
\pgfpathlineto{\pgfqpoint{2.337997in}{2.565179in}}%
\pgfusepath{stroke}%
\end{pgfscope}%
\begin{pgfscope}%
\pgfpathrectangle{\pgfqpoint{0.481978in}{0.331635in}}{\pgfqpoint{4.960000in}{3.696000in}}%
\pgfusepath{clip}%
\pgfsetrectcap%
\pgfsetroundjoin%
\pgfsetlinewidth{1.505625pt}%
\definecolor{currentstroke}{rgb}{1.000000,0.705882,0.509804}%
\pgfsetstrokecolor{currentstroke}%
\pgfsetstrokeopacity{0.800000}%
\pgfsetdash{}{0pt}%
\pgfpathmoveto{\pgfqpoint{1.182318in}{2.900478in}}%
\pgfpathlineto{\pgfqpoint{2.337997in}{2.565179in}}%
\pgfusepath{stroke}%
\end{pgfscope}%
\begin{pgfscope}%
\pgfpathrectangle{\pgfqpoint{0.481978in}{0.331635in}}{\pgfqpoint{4.960000in}{3.696000in}}%
\pgfusepath{clip}%
\pgfsetrectcap%
\pgfsetroundjoin%
\pgfsetlinewidth{1.505625pt}%
\definecolor{currentstroke}{rgb}{1.000000,0.705882,0.509804}%
\pgfsetstrokecolor{currentstroke}%
\pgfsetstrokeopacity{0.800000}%
\pgfsetdash{}{0pt}%
\pgfpathmoveto{\pgfqpoint{1.453190in}{1.760252in}}%
\pgfpathlineto{\pgfqpoint{2.337997in}{2.565179in}}%
\pgfusepath{stroke}%
\end{pgfscope}%
\begin{pgfscope}%
\pgfpathrectangle{\pgfqpoint{0.481978in}{0.331635in}}{\pgfqpoint{4.960000in}{3.696000in}}%
\pgfusepath{clip}%
\pgfsetrectcap%
\pgfsetroundjoin%
\pgfsetlinewidth{1.505625pt}%
\definecolor{currentstroke}{rgb}{1.000000,0.705882,0.509804}%
\pgfsetstrokecolor{currentstroke}%
\pgfsetstrokeopacity{0.800000}%
\pgfsetdash{}{0pt}%
\pgfpathmoveto{\pgfqpoint{1.815578in}{2.616171in}}%
\pgfpathlineto{\pgfqpoint{2.337997in}{2.565179in}}%
\pgfusepath{stroke}%
\end{pgfscope}%
\begin{pgfscope}%
\pgfpathrectangle{\pgfqpoint{0.481978in}{0.331635in}}{\pgfqpoint{4.960000in}{3.696000in}}%
\pgfusepath{clip}%
\pgfsetrectcap%
\pgfsetroundjoin%
\pgfsetlinewidth{1.505625pt}%
\definecolor{currentstroke}{rgb}{1.000000,0.705882,0.509804}%
\pgfsetstrokecolor{currentstroke}%
\pgfsetstrokeopacity{0.800000}%
\pgfsetdash{}{0pt}%
\pgfpathmoveto{\pgfqpoint{3.041251in}{2.562970in}}%
\pgfpathlineto{\pgfqpoint{2.337997in}{2.565179in}}%
\pgfusepath{stroke}%
\end{pgfscope}%
\begin{pgfscope}%
\pgfpathrectangle{\pgfqpoint{0.481978in}{0.331635in}}{\pgfqpoint{4.960000in}{3.696000in}}%
\pgfusepath{clip}%
\pgfsetrectcap%
\pgfsetroundjoin%
\pgfsetlinewidth{1.505625pt}%
\definecolor{currentstroke}{rgb}{1.000000,0.705882,0.509804}%
\pgfsetstrokecolor{currentstroke}%
\pgfsetstrokeopacity{0.800000}%
\pgfsetdash{}{0pt}%
\pgfpathmoveto{\pgfqpoint{2.709888in}{2.364495in}}%
\pgfpathlineto{\pgfqpoint{2.337997in}{2.565179in}}%
\pgfusepath{stroke}%
\end{pgfscope}%
\begin{pgfscope}%
\pgfpathrectangle{\pgfqpoint{0.481978in}{0.331635in}}{\pgfqpoint{4.960000in}{3.696000in}}%
\pgfusepath{clip}%
\pgfsetrectcap%
\pgfsetroundjoin%
\pgfsetlinewidth{1.505625pt}%
\definecolor{currentstroke}{rgb}{1.000000,0.705882,0.509804}%
\pgfsetstrokecolor{currentstroke}%
\pgfsetstrokeopacity{0.800000}%
\pgfsetdash{}{0pt}%
\pgfpathmoveto{\pgfqpoint{2.262261in}{2.588785in}}%
\pgfpathlineto{\pgfqpoint{2.337997in}{2.565179in}}%
\pgfusepath{stroke}%
\end{pgfscope}%
\begin{pgfscope}%
\pgfpathrectangle{\pgfqpoint{0.481978in}{0.331635in}}{\pgfqpoint{4.960000in}{3.696000in}}%
\pgfusepath{clip}%
\pgfsetrectcap%
\pgfsetroundjoin%
\pgfsetlinewidth{1.505625pt}%
\definecolor{currentstroke}{rgb}{1.000000,0.705882,0.509804}%
\pgfsetstrokecolor{currentstroke}%
\pgfsetstrokeopacity{0.800000}%
\pgfsetdash{}{0pt}%
\pgfpathmoveto{\pgfqpoint{1.812312in}{2.189080in}}%
\pgfpathlineto{\pgfqpoint{2.337997in}{2.565179in}}%
\pgfusepath{stroke}%
\end{pgfscope}%
\begin{pgfscope}%
\pgfpathrectangle{\pgfqpoint{0.481978in}{0.331635in}}{\pgfqpoint{4.960000in}{3.696000in}}%
\pgfusepath{clip}%
\pgfsetrectcap%
\pgfsetroundjoin%
\pgfsetlinewidth{1.505625pt}%
\definecolor{currentstroke}{rgb}{0.631373,0.788235,0.956863}%
\pgfsetstrokecolor{currentstroke}%
\pgfsetstrokeopacity{0.200000}%
\pgfsetdash{}{0pt}%
\pgfpathmoveto{\pgfqpoint{2.202225in}{1.028023in}}%
\pgfpathlineto{\pgfqpoint{3.401164in}{1.890602in}}%
\pgfusepath{stroke}%
\end{pgfscope}%
\begin{pgfscope}%
\pgfpathrectangle{\pgfqpoint{0.481978in}{0.331635in}}{\pgfqpoint{4.960000in}{3.696000in}}%
\pgfusepath{clip}%
\pgfsetrectcap%
\pgfsetroundjoin%
\pgfsetlinewidth{1.505625pt}%
\definecolor{currentstroke}{rgb}{0.631373,0.788235,0.956863}%
\pgfsetstrokecolor{currentstroke}%
\pgfsetstrokeopacity{0.200000}%
\pgfsetdash{}{0pt}%
\pgfpathmoveto{\pgfqpoint{2.596676in}{2.212003in}}%
\pgfpathlineto{\pgfqpoint{3.401164in}{1.890602in}}%
\pgfusepath{stroke}%
\end{pgfscope}%
\begin{pgfscope}%
\pgfpathrectangle{\pgfqpoint{0.481978in}{0.331635in}}{\pgfqpoint{4.960000in}{3.696000in}}%
\pgfusepath{clip}%
\pgfsetrectcap%
\pgfsetroundjoin%
\pgfsetlinewidth{1.505625pt}%
\definecolor{currentstroke}{rgb}{0.631373,0.788235,0.956863}%
\pgfsetstrokecolor{currentstroke}%
\pgfsetstrokeopacity{0.200000}%
\pgfsetdash{}{0pt}%
\pgfpathmoveto{\pgfqpoint{3.321086in}{1.722324in}}%
\pgfpathlineto{\pgfqpoint{3.401164in}{1.890602in}}%
\pgfusepath{stroke}%
\end{pgfscope}%
\begin{pgfscope}%
\pgfpathrectangle{\pgfqpoint{0.481978in}{0.331635in}}{\pgfqpoint{4.960000in}{3.696000in}}%
\pgfusepath{clip}%
\pgfsetrectcap%
\pgfsetroundjoin%
\pgfsetlinewidth{1.505625pt}%
\definecolor{currentstroke}{rgb}{0.631373,0.788235,0.956863}%
\pgfsetstrokecolor{currentstroke}%
\pgfsetstrokeopacity{0.200000}%
\pgfsetdash{}{0pt}%
\pgfpathmoveto{\pgfqpoint{3.502753in}{0.752760in}}%
\pgfpathlineto{\pgfqpoint{3.401164in}{1.890602in}}%
\pgfusepath{stroke}%
\end{pgfscope}%
\begin{pgfscope}%
\pgfpathrectangle{\pgfqpoint{0.481978in}{0.331635in}}{\pgfqpoint{4.960000in}{3.696000in}}%
\pgfusepath{clip}%
\pgfsetrectcap%
\pgfsetroundjoin%
\pgfsetlinewidth{1.505625pt}%
\definecolor{currentstroke}{rgb}{0.631373,0.788235,0.956863}%
\pgfsetstrokecolor{currentstroke}%
\pgfsetstrokeopacity{0.200000}%
\pgfsetdash{}{0pt}%
\pgfpathmoveto{\pgfqpoint{2.771539in}{1.194263in}}%
\pgfpathlineto{\pgfqpoint{3.401164in}{1.890602in}}%
\pgfusepath{stroke}%
\end{pgfscope}%
\begin{pgfscope}%
\pgfpathrectangle{\pgfqpoint{0.481978in}{0.331635in}}{\pgfqpoint{4.960000in}{3.696000in}}%
\pgfusepath{clip}%
\pgfsetrectcap%
\pgfsetroundjoin%
\pgfsetlinewidth{1.505625pt}%
\definecolor{currentstroke}{rgb}{0.631373,0.788235,0.956863}%
\pgfsetstrokecolor{currentstroke}%
\pgfsetstrokeopacity{0.200000}%
\pgfsetdash{}{0pt}%
\pgfpathmoveto{\pgfqpoint{3.917552in}{1.170759in}}%
\pgfpathlineto{\pgfqpoint{3.401164in}{1.890602in}}%
\pgfusepath{stroke}%
\end{pgfscope}%
\begin{pgfscope}%
\pgfpathrectangle{\pgfqpoint{0.481978in}{0.331635in}}{\pgfqpoint{4.960000in}{3.696000in}}%
\pgfusepath{clip}%
\pgfsetrectcap%
\pgfsetroundjoin%
\pgfsetlinewidth{1.505625pt}%
\definecolor{currentstroke}{rgb}{0.631373,0.788235,0.956863}%
\pgfsetstrokecolor{currentstroke}%
\pgfsetstrokeopacity{0.200000}%
\pgfsetdash{}{0pt}%
\pgfpathmoveto{\pgfqpoint{3.801010in}{2.474151in}}%
\pgfpathlineto{\pgfqpoint{3.401164in}{1.890602in}}%
\pgfusepath{stroke}%
\end{pgfscope}%
\begin{pgfscope}%
\pgfpathrectangle{\pgfqpoint{0.481978in}{0.331635in}}{\pgfqpoint{4.960000in}{3.696000in}}%
\pgfusepath{clip}%
\pgfsetrectcap%
\pgfsetroundjoin%
\pgfsetlinewidth{1.505625pt}%
\definecolor{currentstroke}{rgb}{0.631373,0.788235,0.956863}%
\pgfsetstrokecolor{currentstroke}%
\pgfsetstrokeopacity{0.200000}%
\pgfsetdash{}{0pt}%
\pgfpathmoveto{\pgfqpoint{3.750240in}{2.693954in}}%
\pgfpathlineto{\pgfqpoint{3.401164in}{1.890602in}}%
\pgfusepath{stroke}%
\end{pgfscope}%
\begin{pgfscope}%
\pgfpathrectangle{\pgfqpoint{0.481978in}{0.331635in}}{\pgfqpoint{4.960000in}{3.696000in}}%
\pgfusepath{clip}%
\pgfsetrectcap%
\pgfsetroundjoin%
\pgfsetlinewidth{1.505625pt}%
\definecolor{currentstroke}{rgb}{0.631373,0.788235,0.956863}%
\pgfsetstrokecolor{currentstroke}%
\pgfsetstrokeopacity{0.200000}%
\pgfsetdash{}{0pt}%
\pgfpathmoveto{\pgfqpoint{2.725316in}{0.760530in}}%
\pgfpathlineto{\pgfqpoint{3.401164in}{1.890602in}}%
\pgfusepath{stroke}%
\end{pgfscope}%
\begin{pgfscope}%
\pgfpathrectangle{\pgfqpoint{0.481978in}{0.331635in}}{\pgfqpoint{4.960000in}{3.696000in}}%
\pgfusepath{clip}%
\pgfsetrectcap%
\pgfsetroundjoin%
\pgfsetlinewidth{1.505625pt}%
\definecolor{currentstroke}{rgb}{0.631373,0.788235,0.956863}%
\pgfsetstrokecolor{currentstroke}%
\pgfsetstrokeopacity{0.200000}%
\pgfsetdash{}{0pt}%
\pgfpathmoveto{\pgfqpoint{2.695779in}{3.093704in}}%
\pgfpathlineto{\pgfqpoint{3.401164in}{1.890602in}}%
\pgfusepath{stroke}%
\end{pgfscope}%
\begin{pgfscope}%
\pgfpathrectangle{\pgfqpoint{0.481978in}{0.331635in}}{\pgfqpoint{4.960000in}{3.696000in}}%
\pgfusepath{clip}%
\pgfsetrectcap%
\pgfsetroundjoin%
\pgfsetlinewidth{1.505625pt}%
\definecolor{currentstroke}{rgb}{0.631373,0.788235,0.956863}%
\pgfsetstrokecolor{currentstroke}%
\pgfsetstrokeopacity{0.200000}%
\pgfsetdash{}{0pt}%
\pgfpathmoveto{\pgfqpoint{4.183138in}{1.344983in}}%
\pgfpathlineto{\pgfqpoint{3.401164in}{1.890602in}}%
\pgfusepath{stroke}%
\end{pgfscope}%
\begin{pgfscope}%
\pgfpathrectangle{\pgfqpoint{0.481978in}{0.331635in}}{\pgfqpoint{4.960000in}{3.696000in}}%
\pgfusepath{clip}%
\pgfsetrectcap%
\pgfsetroundjoin%
\pgfsetlinewidth{1.505625pt}%
\definecolor{currentstroke}{rgb}{0.631373,0.788235,0.956863}%
\pgfsetstrokecolor{currentstroke}%
\pgfsetstrokeopacity{0.200000}%
\pgfsetdash{}{0pt}%
\pgfpathmoveto{\pgfqpoint{4.833897in}{3.334876in}}%
\pgfpathlineto{\pgfqpoint{3.401164in}{1.890602in}}%
\pgfusepath{stroke}%
\end{pgfscope}%
\begin{pgfscope}%
\pgfpathrectangle{\pgfqpoint{0.481978in}{0.331635in}}{\pgfqpoint{4.960000in}{3.696000in}}%
\pgfusepath{clip}%
\pgfsetrectcap%
\pgfsetroundjoin%
\pgfsetlinewidth{1.505625pt}%
\definecolor{currentstroke}{rgb}{0.631373,0.788235,0.956863}%
\pgfsetstrokecolor{currentstroke}%
\pgfsetstrokeopacity{0.200000}%
\pgfsetdash{}{0pt}%
\pgfpathmoveto{\pgfqpoint{3.853456in}{1.615791in}}%
\pgfpathlineto{\pgfqpoint{3.401164in}{1.890602in}}%
\pgfusepath{stroke}%
\end{pgfscope}%
\begin{pgfscope}%
\pgfpathrectangle{\pgfqpoint{0.481978in}{0.331635in}}{\pgfqpoint{4.960000in}{3.696000in}}%
\pgfusepath{clip}%
\pgfsetrectcap%
\pgfsetroundjoin%
\pgfsetlinewidth{1.505625pt}%
\definecolor{currentstroke}{rgb}{0.631373,0.788235,0.956863}%
\pgfsetstrokecolor{currentstroke}%
\pgfsetstrokeopacity{0.200000}%
\pgfsetdash{}{0pt}%
\pgfpathmoveto{\pgfqpoint{3.890909in}{2.141812in}}%
\pgfpathlineto{\pgfqpoint{3.401164in}{1.890602in}}%
\pgfusepath{stroke}%
\end{pgfscope}%
\begin{pgfscope}%
\pgfpathrectangle{\pgfqpoint{0.481978in}{0.331635in}}{\pgfqpoint{4.960000in}{3.696000in}}%
\pgfusepath{clip}%
\pgfsetrectcap%
\pgfsetroundjoin%
\pgfsetlinewidth{1.505625pt}%
\definecolor{currentstroke}{rgb}{0.631373,0.788235,0.956863}%
\pgfsetstrokecolor{currentstroke}%
\pgfsetstrokeopacity{0.200000}%
\pgfsetdash{}{0pt}%
\pgfpathmoveto{\pgfqpoint{4.121377in}{2.205695in}}%
\pgfpathlineto{\pgfqpoint{3.401164in}{1.890602in}}%
\pgfusepath{stroke}%
\end{pgfscope}%
\begin{pgfscope}%
\pgfpathrectangle{\pgfqpoint{0.481978in}{0.331635in}}{\pgfqpoint{4.960000in}{3.696000in}}%
\pgfusepath{clip}%
\pgfsetrectcap%
\pgfsetroundjoin%
\pgfsetlinewidth{1.505625pt}%
\definecolor{currentstroke}{rgb}{0.631373,0.788235,0.956863}%
\pgfsetstrokecolor{currentstroke}%
\pgfsetstrokeopacity{0.200000}%
\pgfsetdash{}{0pt}%
\pgfpathmoveto{\pgfqpoint{2.445872in}{0.839258in}}%
\pgfpathlineto{\pgfqpoint{3.401164in}{1.890602in}}%
\pgfusepath{stroke}%
\end{pgfscope}%
\begin{pgfscope}%
\pgfpathrectangle{\pgfqpoint{0.481978in}{0.331635in}}{\pgfqpoint{4.960000in}{3.696000in}}%
\pgfusepath{clip}%
\pgfsetrectcap%
\pgfsetroundjoin%
\pgfsetlinewidth{1.505625pt}%
\definecolor{currentstroke}{rgb}{0.631373,0.788235,0.956863}%
\pgfsetstrokecolor{currentstroke}%
\pgfsetstrokeopacity{0.200000}%
\pgfsetdash{}{0pt}%
\pgfpathmoveto{\pgfqpoint{3.678569in}{1.446223in}}%
\pgfpathlineto{\pgfqpoint{3.401164in}{1.890602in}}%
\pgfusepath{stroke}%
\end{pgfscope}%
\begin{pgfscope}%
\pgfpathrectangle{\pgfqpoint{0.481978in}{0.331635in}}{\pgfqpoint{4.960000in}{3.696000in}}%
\pgfusepath{clip}%
\pgfsetrectcap%
\pgfsetroundjoin%
\pgfsetlinewidth{1.505625pt}%
\definecolor{currentstroke}{rgb}{0.631373,0.788235,0.956863}%
\pgfsetstrokecolor{currentstroke}%
\pgfsetstrokeopacity{0.200000}%
\pgfsetdash{}{0pt}%
\pgfpathmoveto{\pgfqpoint{3.739727in}{2.763191in}}%
\pgfpathlineto{\pgfqpoint{3.401164in}{1.890602in}}%
\pgfusepath{stroke}%
\end{pgfscope}%
\begin{pgfscope}%
\pgfpathrectangle{\pgfqpoint{0.481978in}{0.331635in}}{\pgfqpoint{4.960000in}{3.696000in}}%
\pgfusepath{clip}%
\pgfsetrectcap%
\pgfsetroundjoin%
\pgfsetlinewidth{1.505625pt}%
\definecolor{currentstroke}{rgb}{0.631373,0.788235,0.956863}%
\pgfsetstrokecolor{currentstroke}%
\pgfsetstrokeopacity{0.200000}%
\pgfsetdash{}{0pt}%
\pgfpathmoveto{\pgfqpoint{3.304052in}{3.367949in}}%
\pgfpathlineto{\pgfqpoint{3.401164in}{1.890602in}}%
\pgfusepath{stroke}%
\end{pgfscope}%
\begin{pgfscope}%
\pgfpathrectangle{\pgfqpoint{0.481978in}{0.331635in}}{\pgfqpoint{4.960000in}{3.696000in}}%
\pgfusepath{clip}%
\pgfsetrectcap%
\pgfsetroundjoin%
\pgfsetlinewidth{1.505625pt}%
\definecolor{currentstroke}{rgb}{0.631373,0.788235,0.956863}%
\pgfsetstrokecolor{currentstroke}%
\pgfsetstrokeopacity{0.200000}%
\pgfsetdash{}{0pt}%
\pgfpathmoveto{\pgfqpoint{3.656864in}{2.557888in}}%
\pgfpathlineto{\pgfqpoint{3.401164in}{1.890602in}}%
\pgfusepath{stroke}%
\end{pgfscope}%
\begin{pgfscope}%
\pgfpathrectangle{\pgfqpoint{0.481978in}{0.331635in}}{\pgfqpoint{4.960000in}{3.696000in}}%
\pgfusepath{clip}%
\pgfsetrectcap%
\pgfsetroundjoin%
\pgfsetlinewidth{1.505625pt}%
\definecolor{currentstroke}{rgb}{0.631373,0.788235,0.956863}%
\pgfsetstrokecolor{currentstroke}%
\pgfsetstrokeopacity{0.200000}%
\pgfsetdash{}{0pt}%
\pgfpathmoveto{\pgfqpoint{2.945780in}{0.583774in}}%
\pgfpathlineto{\pgfqpoint{3.401164in}{1.890602in}}%
\pgfusepath{stroke}%
\end{pgfscope}%
\begin{pgfscope}%
\pgfpathrectangle{\pgfqpoint{0.481978in}{0.331635in}}{\pgfqpoint{4.960000in}{3.696000in}}%
\pgfusepath{clip}%
\pgfsetrectcap%
\pgfsetroundjoin%
\pgfsetlinewidth{1.505625pt}%
\definecolor{currentstroke}{rgb}{0.631373,0.788235,0.956863}%
\pgfsetstrokecolor{currentstroke}%
\pgfsetstrokeopacity{0.200000}%
\pgfsetdash{}{0pt}%
\pgfpathmoveto{\pgfqpoint{3.781439in}{2.800808in}}%
\pgfpathlineto{\pgfqpoint{3.401164in}{1.890602in}}%
\pgfusepath{stroke}%
\end{pgfscope}%
\begin{pgfscope}%
\pgfpathrectangle{\pgfqpoint{0.481978in}{0.331635in}}{\pgfqpoint{4.960000in}{3.696000in}}%
\pgfusepath{clip}%
\pgfsetrectcap%
\pgfsetroundjoin%
\pgfsetlinewidth{1.505625pt}%
\definecolor{currentstroke}{rgb}{0.631373,0.788235,0.956863}%
\pgfsetstrokecolor{currentstroke}%
\pgfsetstrokeopacity{0.200000}%
\pgfsetdash{}{0pt}%
\pgfpathmoveto{\pgfqpoint{3.557003in}{0.752949in}}%
\pgfpathlineto{\pgfqpoint{3.401164in}{1.890602in}}%
\pgfusepath{stroke}%
\end{pgfscope}%
\begin{pgfscope}%
\pgfpathrectangle{\pgfqpoint{0.481978in}{0.331635in}}{\pgfqpoint{4.960000in}{3.696000in}}%
\pgfusepath{clip}%
\pgfsetrectcap%
\pgfsetroundjoin%
\pgfsetlinewidth{1.505625pt}%
\definecolor{currentstroke}{rgb}{0.631373,0.788235,0.956863}%
\pgfsetstrokecolor{currentstroke}%
\pgfsetstrokeopacity{0.200000}%
\pgfsetdash{}{0pt}%
\pgfpathmoveto{\pgfqpoint{2.792035in}{1.014348in}}%
\pgfpathlineto{\pgfqpoint{3.401164in}{1.890602in}}%
\pgfusepath{stroke}%
\end{pgfscope}%
\begin{pgfscope}%
\pgfpathrectangle{\pgfqpoint{0.481978in}{0.331635in}}{\pgfqpoint{4.960000in}{3.696000in}}%
\pgfusepath{clip}%
\pgfsetrectcap%
\pgfsetroundjoin%
\pgfsetlinewidth{1.505625pt}%
\definecolor{currentstroke}{rgb}{0.631373,0.788235,0.956863}%
\pgfsetstrokecolor{currentstroke}%
\pgfsetstrokeopacity{0.200000}%
\pgfsetdash{}{0pt}%
\pgfpathmoveto{\pgfqpoint{4.082567in}{1.741519in}}%
\pgfpathlineto{\pgfqpoint{3.401164in}{1.890602in}}%
\pgfusepath{stroke}%
\end{pgfscope}%
\begin{pgfscope}%
\pgfpathrectangle{\pgfqpoint{0.481978in}{0.331635in}}{\pgfqpoint{4.960000in}{3.696000in}}%
\pgfusepath{clip}%
\pgfsetrectcap%
\pgfsetroundjoin%
\pgfsetlinewidth{1.505625pt}%
\definecolor{currentstroke}{rgb}{0.631373,0.788235,0.956863}%
\pgfsetstrokecolor{currentstroke}%
\pgfsetstrokeopacity{0.200000}%
\pgfsetdash{}{0pt}%
\pgfpathmoveto{\pgfqpoint{3.232867in}{0.917040in}}%
\pgfpathlineto{\pgfqpoint{3.401164in}{1.890602in}}%
\pgfusepath{stroke}%
\end{pgfscope}%
\begin{pgfscope}%
\pgfpathrectangle{\pgfqpoint{0.481978in}{0.331635in}}{\pgfqpoint{4.960000in}{3.696000in}}%
\pgfusepath{clip}%
\pgfsetrectcap%
\pgfsetroundjoin%
\pgfsetlinewidth{1.505625pt}%
\definecolor{currentstroke}{rgb}{0.631373,0.788235,0.956863}%
\pgfsetstrokecolor{currentstroke}%
\pgfsetstrokeopacity{0.200000}%
\pgfsetdash{}{0pt}%
\pgfpathmoveto{\pgfqpoint{3.719174in}{2.482665in}}%
\pgfpathlineto{\pgfqpoint{3.401164in}{1.890602in}}%
\pgfusepath{stroke}%
\end{pgfscope}%
\begin{pgfscope}%
\pgfpathrectangle{\pgfqpoint{0.481978in}{0.331635in}}{\pgfqpoint{4.960000in}{3.696000in}}%
\pgfusepath{clip}%
\pgfsetrectcap%
\pgfsetroundjoin%
\pgfsetlinewidth{1.505625pt}%
\definecolor{currentstroke}{rgb}{0.631373,0.788235,0.956863}%
\pgfsetstrokecolor{currentstroke}%
\pgfsetstrokeopacity{0.200000}%
\pgfsetdash{}{0pt}%
\pgfpathmoveto{\pgfqpoint{2.663289in}{2.552137in}}%
\pgfpathlineto{\pgfqpoint{3.401164in}{1.890602in}}%
\pgfusepath{stroke}%
\end{pgfscope}%
\begin{pgfscope}%
\pgfpathrectangle{\pgfqpoint{0.481978in}{0.331635in}}{\pgfqpoint{4.960000in}{3.696000in}}%
\pgfusepath{clip}%
\pgfsetrectcap%
\pgfsetroundjoin%
\pgfsetlinewidth{1.505625pt}%
\definecolor{currentstroke}{rgb}{0.631373,0.788235,0.956863}%
\pgfsetstrokecolor{currentstroke}%
\pgfsetstrokeopacity{0.200000}%
\pgfsetdash{}{0pt}%
\pgfpathmoveto{\pgfqpoint{3.485653in}{1.240202in}}%
\pgfpathlineto{\pgfqpoint{3.401164in}{1.890602in}}%
\pgfusepath{stroke}%
\end{pgfscope}%
\begin{pgfscope}%
\pgfpathrectangle{\pgfqpoint{0.481978in}{0.331635in}}{\pgfqpoint{4.960000in}{3.696000in}}%
\pgfusepath{clip}%
\pgfsetrectcap%
\pgfsetroundjoin%
\pgfsetlinewidth{1.505625pt}%
\definecolor{currentstroke}{rgb}{0.631373,0.788235,0.956863}%
\pgfsetstrokecolor{currentstroke}%
\pgfsetstrokeopacity{0.200000}%
\pgfsetdash{}{0pt}%
\pgfpathmoveto{\pgfqpoint{2.981550in}{0.514023in}}%
\pgfpathlineto{\pgfqpoint{3.401164in}{1.890602in}}%
\pgfusepath{stroke}%
\end{pgfscope}%
\begin{pgfscope}%
\pgfpathrectangle{\pgfqpoint{0.481978in}{0.331635in}}{\pgfqpoint{4.960000in}{3.696000in}}%
\pgfusepath{clip}%
\pgfsetrectcap%
\pgfsetroundjoin%
\pgfsetlinewidth{1.505625pt}%
\definecolor{currentstroke}{rgb}{0.631373,0.788235,0.956863}%
\pgfsetstrokecolor{currentstroke}%
\pgfsetstrokeopacity{0.200000}%
\pgfsetdash{}{0pt}%
\pgfpathmoveto{\pgfqpoint{3.622294in}{2.048663in}}%
\pgfpathlineto{\pgfqpoint{3.401164in}{1.890602in}}%
\pgfusepath{stroke}%
\end{pgfscope}%
\begin{pgfscope}%
\pgfpathrectangle{\pgfqpoint{0.481978in}{0.331635in}}{\pgfqpoint{4.960000in}{3.696000in}}%
\pgfusepath{clip}%
\pgfsetrectcap%
\pgfsetroundjoin%
\pgfsetlinewidth{1.505625pt}%
\definecolor{currentstroke}{rgb}{0.631373,0.788235,0.956863}%
\pgfsetstrokecolor{currentstroke}%
\pgfsetstrokeopacity{0.200000}%
\pgfsetdash{}{0pt}%
\pgfpathmoveto{\pgfqpoint{4.139050in}{1.358919in}}%
\pgfpathlineto{\pgfqpoint{3.401164in}{1.890602in}}%
\pgfusepath{stroke}%
\end{pgfscope}%
\begin{pgfscope}%
\pgfpathrectangle{\pgfqpoint{0.481978in}{0.331635in}}{\pgfqpoint{4.960000in}{3.696000in}}%
\pgfusepath{clip}%
\pgfsetrectcap%
\pgfsetroundjoin%
\pgfsetlinewidth{1.505625pt}%
\definecolor{currentstroke}{rgb}{0.631373,0.788235,0.956863}%
\pgfsetstrokecolor{currentstroke}%
\pgfsetstrokeopacity{0.200000}%
\pgfsetdash{}{0pt}%
\pgfpathmoveto{\pgfqpoint{2.205469in}{1.020681in}}%
\pgfpathlineto{\pgfqpoint{3.401164in}{1.890602in}}%
\pgfusepath{stroke}%
\end{pgfscope}%
\begin{pgfscope}%
\pgfpathrectangle{\pgfqpoint{0.481978in}{0.331635in}}{\pgfqpoint{4.960000in}{3.696000in}}%
\pgfusepath{clip}%
\pgfsetrectcap%
\pgfsetroundjoin%
\pgfsetlinewidth{1.505625pt}%
\definecolor{currentstroke}{rgb}{0.631373,0.788235,0.956863}%
\pgfsetstrokecolor{currentstroke}%
\pgfsetstrokeopacity{0.200000}%
\pgfsetdash{}{0pt}%
\pgfpathmoveto{\pgfqpoint{2.990475in}{3.280190in}}%
\pgfpathlineto{\pgfqpoint{3.401164in}{1.890602in}}%
\pgfusepath{stroke}%
\end{pgfscope}%
\begin{pgfscope}%
\pgfpathrectangle{\pgfqpoint{0.481978in}{0.331635in}}{\pgfqpoint{4.960000in}{3.696000in}}%
\pgfusepath{clip}%
\pgfsetrectcap%
\pgfsetroundjoin%
\pgfsetlinewidth{1.505625pt}%
\definecolor{currentstroke}{rgb}{0.631373,0.788235,0.956863}%
\pgfsetstrokecolor{currentstroke}%
\pgfsetstrokeopacity{0.200000}%
\pgfsetdash{}{0pt}%
\pgfpathmoveto{\pgfqpoint{2.666430in}{0.851785in}}%
\pgfpathlineto{\pgfqpoint{3.401164in}{1.890602in}}%
\pgfusepath{stroke}%
\end{pgfscope}%
\begin{pgfscope}%
\pgfpathrectangle{\pgfqpoint{0.481978in}{0.331635in}}{\pgfqpoint{4.960000in}{3.696000in}}%
\pgfusepath{clip}%
\pgfsetrectcap%
\pgfsetroundjoin%
\pgfsetlinewidth{1.505625pt}%
\definecolor{currentstroke}{rgb}{0.631373,0.788235,0.956863}%
\pgfsetstrokecolor{currentstroke}%
\pgfsetstrokeopacity{0.200000}%
\pgfsetdash{}{0pt}%
\pgfpathmoveto{\pgfqpoint{2.845022in}{2.177755in}}%
\pgfpathlineto{\pgfqpoint{3.401164in}{1.890602in}}%
\pgfusepath{stroke}%
\end{pgfscope}%
\begin{pgfscope}%
\pgfpathrectangle{\pgfqpoint{0.481978in}{0.331635in}}{\pgfqpoint{4.960000in}{3.696000in}}%
\pgfusepath{clip}%
\pgfsetrectcap%
\pgfsetroundjoin%
\pgfsetlinewidth{1.505625pt}%
\definecolor{currentstroke}{rgb}{0.631373,0.788235,0.956863}%
\pgfsetstrokecolor{currentstroke}%
\pgfsetstrokeopacity{0.200000}%
\pgfsetdash{}{0pt}%
\pgfpathmoveto{\pgfqpoint{4.914434in}{2.316668in}}%
\pgfpathlineto{\pgfqpoint{3.401164in}{1.890602in}}%
\pgfusepath{stroke}%
\end{pgfscope}%
\begin{pgfscope}%
\pgfpathrectangle{\pgfqpoint{0.481978in}{0.331635in}}{\pgfqpoint{4.960000in}{3.696000in}}%
\pgfusepath{clip}%
\pgfsetrectcap%
\pgfsetroundjoin%
\pgfsetlinewidth{1.505625pt}%
\definecolor{currentstroke}{rgb}{0.631373,0.788235,0.956863}%
\pgfsetstrokecolor{currentstroke}%
\pgfsetstrokeopacity{0.200000}%
\pgfsetdash{}{0pt}%
\pgfpathmoveto{\pgfqpoint{3.626594in}{1.714800in}}%
\pgfpathlineto{\pgfqpoint{3.401164in}{1.890602in}}%
\pgfusepath{stroke}%
\end{pgfscope}%
\begin{pgfscope}%
\pgfpathrectangle{\pgfqpoint{0.481978in}{0.331635in}}{\pgfqpoint{4.960000in}{3.696000in}}%
\pgfusepath{clip}%
\pgfsetrectcap%
\pgfsetroundjoin%
\pgfsetlinewidth{1.505625pt}%
\definecolor{currentstroke}{rgb}{0.631373,0.788235,0.956863}%
\pgfsetstrokecolor{currentstroke}%
\pgfsetstrokeopacity{0.200000}%
\pgfsetdash{}{0pt}%
\pgfpathmoveto{\pgfqpoint{2.797292in}{1.373177in}}%
\pgfpathlineto{\pgfqpoint{3.401164in}{1.890602in}}%
\pgfusepath{stroke}%
\end{pgfscope}%
\begin{pgfscope}%
\pgfpathrectangle{\pgfqpoint{0.481978in}{0.331635in}}{\pgfqpoint{4.960000in}{3.696000in}}%
\pgfusepath{clip}%
\pgfsetrectcap%
\pgfsetroundjoin%
\pgfsetlinewidth{1.505625pt}%
\definecolor{currentstroke}{rgb}{0.631373,0.788235,0.956863}%
\pgfsetstrokecolor{currentstroke}%
\pgfsetstrokeopacity{0.200000}%
\pgfsetdash{}{0pt}%
\pgfpathmoveto{\pgfqpoint{2.640193in}{0.561274in}}%
\pgfpathlineto{\pgfqpoint{3.401164in}{1.890602in}}%
\pgfusepath{stroke}%
\end{pgfscope}%
\begin{pgfscope}%
\pgfpathrectangle{\pgfqpoint{0.481978in}{0.331635in}}{\pgfqpoint{4.960000in}{3.696000in}}%
\pgfusepath{clip}%
\pgfsetrectcap%
\pgfsetroundjoin%
\pgfsetlinewidth{1.505625pt}%
\definecolor{currentstroke}{rgb}{0.631373,0.788235,0.956863}%
\pgfsetstrokecolor{currentstroke}%
\pgfsetstrokeopacity{0.200000}%
\pgfsetdash{}{0pt}%
\pgfpathmoveto{\pgfqpoint{3.001265in}{1.206555in}}%
\pgfpathlineto{\pgfqpoint{3.401164in}{1.890602in}}%
\pgfusepath{stroke}%
\end{pgfscope}%
\begin{pgfscope}%
\pgfpathrectangle{\pgfqpoint{0.481978in}{0.331635in}}{\pgfqpoint{4.960000in}{3.696000in}}%
\pgfusepath{clip}%
\pgfsetrectcap%
\pgfsetroundjoin%
\pgfsetlinewidth{1.505625pt}%
\definecolor{currentstroke}{rgb}{0.631373,0.788235,0.956863}%
\pgfsetstrokecolor{currentstroke}%
\pgfsetstrokeopacity{0.200000}%
\pgfsetdash{}{0pt}%
\pgfpathmoveto{\pgfqpoint{2.921042in}{1.050180in}}%
\pgfpathlineto{\pgfqpoint{3.401164in}{1.890602in}}%
\pgfusepath{stroke}%
\end{pgfscope}%
\begin{pgfscope}%
\pgfpathrectangle{\pgfqpoint{0.481978in}{0.331635in}}{\pgfqpoint{4.960000in}{3.696000in}}%
\pgfusepath{clip}%
\pgfsetrectcap%
\pgfsetroundjoin%
\pgfsetlinewidth{1.505625pt}%
\definecolor{currentstroke}{rgb}{0.631373,0.788235,0.956863}%
\pgfsetstrokecolor{currentstroke}%
\pgfsetstrokeopacity{0.200000}%
\pgfsetdash{}{0pt}%
\pgfpathmoveto{\pgfqpoint{2.393692in}{1.280695in}}%
\pgfpathlineto{\pgfqpoint{3.401164in}{1.890602in}}%
\pgfusepath{stroke}%
\end{pgfscope}%
\begin{pgfscope}%
\pgfpathrectangle{\pgfqpoint{0.481978in}{0.331635in}}{\pgfqpoint{4.960000in}{3.696000in}}%
\pgfusepath{clip}%
\pgfsetrectcap%
\pgfsetroundjoin%
\pgfsetlinewidth{1.505625pt}%
\definecolor{currentstroke}{rgb}{0.631373,0.788235,0.956863}%
\pgfsetstrokecolor{currentstroke}%
\pgfsetstrokeopacity{0.200000}%
\pgfsetdash{}{0pt}%
\pgfpathmoveto{\pgfqpoint{3.120031in}{2.761551in}}%
\pgfpathlineto{\pgfqpoint{3.401164in}{1.890602in}}%
\pgfusepath{stroke}%
\end{pgfscope}%
\begin{pgfscope}%
\pgfpathrectangle{\pgfqpoint{0.481978in}{0.331635in}}{\pgfqpoint{4.960000in}{3.696000in}}%
\pgfusepath{clip}%
\pgfsetrectcap%
\pgfsetroundjoin%
\pgfsetlinewidth{1.505625pt}%
\definecolor{currentstroke}{rgb}{0.631373,0.788235,0.956863}%
\pgfsetstrokecolor{currentstroke}%
\pgfsetstrokeopacity{0.200000}%
\pgfsetdash{}{0pt}%
\pgfpathmoveto{\pgfqpoint{4.510528in}{2.407226in}}%
\pgfpathlineto{\pgfqpoint{3.401164in}{1.890602in}}%
\pgfusepath{stroke}%
\end{pgfscope}%
\begin{pgfscope}%
\pgfpathrectangle{\pgfqpoint{0.481978in}{0.331635in}}{\pgfqpoint{4.960000in}{3.696000in}}%
\pgfusepath{clip}%
\pgfsetrectcap%
\pgfsetroundjoin%
\pgfsetlinewidth{1.505625pt}%
\definecolor{currentstroke}{rgb}{0.631373,0.788235,0.956863}%
\pgfsetstrokecolor{currentstroke}%
\pgfsetstrokeopacity{0.200000}%
\pgfsetdash{}{0pt}%
\pgfpathmoveto{\pgfqpoint{3.942534in}{3.253270in}}%
\pgfpathlineto{\pgfqpoint{3.401164in}{1.890602in}}%
\pgfusepath{stroke}%
\end{pgfscope}%
\begin{pgfscope}%
\pgfpathrectangle{\pgfqpoint{0.481978in}{0.331635in}}{\pgfqpoint{4.960000in}{3.696000in}}%
\pgfusepath{clip}%
\pgfsetrectcap%
\pgfsetroundjoin%
\pgfsetlinewidth{1.505625pt}%
\definecolor{currentstroke}{rgb}{0.631373,0.788235,0.956863}%
\pgfsetstrokecolor{currentstroke}%
\pgfsetstrokeopacity{0.200000}%
\pgfsetdash{}{0pt}%
\pgfpathmoveto{\pgfqpoint{4.138057in}{3.274652in}}%
\pgfpathlineto{\pgfqpoint{3.401164in}{1.890602in}}%
\pgfusepath{stroke}%
\end{pgfscope}%
\begin{pgfscope}%
\pgfpathrectangle{\pgfqpoint{0.481978in}{0.331635in}}{\pgfqpoint{4.960000in}{3.696000in}}%
\pgfusepath{clip}%
\pgfsetrectcap%
\pgfsetroundjoin%
\pgfsetlinewidth{1.505625pt}%
\definecolor{currentstroke}{rgb}{0.631373,0.788235,0.956863}%
\pgfsetstrokecolor{currentstroke}%
\pgfsetstrokeopacity{0.200000}%
\pgfsetdash{}{0pt}%
\pgfpathmoveto{\pgfqpoint{2.830135in}{0.993456in}}%
\pgfpathlineto{\pgfqpoint{3.401164in}{1.890602in}}%
\pgfusepath{stroke}%
\end{pgfscope}%
\begin{pgfscope}%
\pgfpathrectangle{\pgfqpoint{0.481978in}{0.331635in}}{\pgfqpoint{4.960000in}{3.696000in}}%
\pgfusepath{clip}%
\pgfsetrectcap%
\pgfsetroundjoin%
\pgfsetlinewidth{1.505625pt}%
\definecolor{currentstroke}{rgb}{0.631373,0.788235,0.956863}%
\pgfsetstrokecolor{currentstroke}%
\pgfsetstrokeopacity{0.200000}%
\pgfsetdash{}{0pt}%
\pgfpathmoveto{\pgfqpoint{2.773539in}{1.774363in}}%
\pgfpathlineto{\pgfqpoint{3.401164in}{1.890602in}}%
\pgfusepath{stroke}%
\end{pgfscope}%
\begin{pgfscope}%
\pgfpathrectangle{\pgfqpoint{0.481978in}{0.331635in}}{\pgfqpoint{4.960000in}{3.696000in}}%
\pgfusepath{clip}%
\pgfsetrectcap%
\pgfsetroundjoin%
\pgfsetlinewidth{1.505625pt}%
\definecolor{currentstroke}{rgb}{0.631373,0.788235,0.956863}%
\pgfsetstrokecolor{currentstroke}%
\pgfsetstrokeopacity{0.200000}%
\pgfsetdash{}{0pt}%
\pgfpathmoveto{\pgfqpoint{3.305756in}{1.533732in}}%
\pgfpathlineto{\pgfqpoint{3.401164in}{1.890602in}}%
\pgfusepath{stroke}%
\end{pgfscope}%
\begin{pgfscope}%
\pgfpathrectangle{\pgfqpoint{0.481978in}{0.331635in}}{\pgfqpoint{4.960000in}{3.696000in}}%
\pgfusepath{clip}%
\pgfsetrectcap%
\pgfsetroundjoin%
\pgfsetlinewidth{1.505625pt}%
\definecolor{currentstroke}{rgb}{0.631373,0.788235,0.956863}%
\pgfsetstrokecolor{currentstroke}%
\pgfsetstrokeopacity{0.200000}%
\pgfsetdash{}{0pt}%
\pgfpathmoveto{\pgfqpoint{3.729492in}{2.109582in}}%
\pgfpathlineto{\pgfqpoint{3.401164in}{1.890602in}}%
\pgfusepath{stroke}%
\end{pgfscope}%
\begin{pgfscope}%
\pgfpathrectangle{\pgfqpoint{0.481978in}{0.331635in}}{\pgfqpoint{4.960000in}{3.696000in}}%
\pgfusepath{clip}%
\pgfsetrectcap%
\pgfsetroundjoin%
\pgfsetlinewidth{1.505625pt}%
\definecolor{currentstroke}{rgb}{0.631373,0.788235,0.956863}%
\pgfsetstrokecolor{currentstroke}%
\pgfsetstrokeopacity{0.200000}%
\pgfsetdash{}{0pt}%
\pgfpathmoveto{\pgfqpoint{3.105792in}{0.686919in}}%
\pgfpathlineto{\pgfqpoint{3.401164in}{1.890602in}}%
\pgfusepath{stroke}%
\end{pgfscope}%
\begin{pgfscope}%
\pgfpathrectangle{\pgfqpoint{0.481978in}{0.331635in}}{\pgfqpoint{4.960000in}{3.696000in}}%
\pgfusepath{clip}%
\pgfsetrectcap%
\pgfsetroundjoin%
\pgfsetlinewidth{1.505625pt}%
\definecolor{currentstroke}{rgb}{0.631373,0.788235,0.956863}%
\pgfsetstrokecolor{currentstroke}%
\pgfsetstrokeopacity{0.200000}%
\pgfsetdash{}{0pt}%
\pgfpathmoveto{\pgfqpoint{3.126698in}{1.759983in}}%
\pgfpathlineto{\pgfqpoint{3.401164in}{1.890602in}}%
\pgfusepath{stroke}%
\end{pgfscope}%
\begin{pgfscope}%
\pgfpathrectangle{\pgfqpoint{0.481978in}{0.331635in}}{\pgfqpoint{4.960000in}{3.696000in}}%
\pgfusepath{clip}%
\pgfsetrectcap%
\pgfsetroundjoin%
\pgfsetlinewidth{1.505625pt}%
\definecolor{currentstroke}{rgb}{0.631373,0.788235,0.956863}%
\pgfsetstrokecolor{currentstroke}%
\pgfsetstrokeopacity{0.200000}%
\pgfsetdash{}{0pt}%
\pgfpathmoveto{\pgfqpoint{3.394486in}{1.171114in}}%
\pgfpathlineto{\pgfqpoint{3.401164in}{1.890602in}}%
\pgfusepath{stroke}%
\end{pgfscope}%
\begin{pgfscope}%
\pgfpathrectangle{\pgfqpoint{0.481978in}{0.331635in}}{\pgfqpoint{4.960000in}{3.696000in}}%
\pgfusepath{clip}%
\pgfsetrectcap%
\pgfsetroundjoin%
\pgfsetlinewidth{1.505625pt}%
\definecolor{currentstroke}{rgb}{0.631373,0.788235,0.956863}%
\pgfsetstrokecolor{currentstroke}%
\pgfsetstrokeopacity{0.200000}%
\pgfsetdash{}{0pt}%
\pgfpathmoveto{\pgfqpoint{3.272709in}{2.093083in}}%
\pgfpathlineto{\pgfqpoint{3.401164in}{1.890602in}}%
\pgfusepath{stroke}%
\end{pgfscope}%
\begin{pgfscope}%
\pgfpathrectangle{\pgfqpoint{0.481978in}{0.331635in}}{\pgfqpoint{4.960000in}{3.696000in}}%
\pgfusepath{clip}%
\pgfsetrectcap%
\pgfsetroundjoin%
\pgfsetlinewidth{1.505625pt}%
\definecolor{currentstroke}{rgb}{0.631373,0.788235,0.956863}%
\pgfsetstrokecolor{currentstroke}%
\pgfsetstrokeopacity{0.200000}%
\pgfsetdash{}{0pt}%
\pgfpathmoveto{\pgfqpoint{1.830270in}{0.993362in}}%
\pgfpathlineto{\pgfqpoint{3.401164in}{1.890602in}}%
\pgfusepath{stroke}%
\end{pgfscope}%
\begin{pgfscope}%
\pgfpathrectangle{\pgfqpoint{0.481978in}{0.331635in}}{\pgfqpoint{4.960000in}{3.696000in}}%
\pgfusepath{clip}%
\pgfsetrectcap%
\pgfsetroundjoin%
\pgfsetlinewidth{1.505625pt}%
\definecolor{currentstroke}{rgb}{0.631373,0.788235,0.956863}%
\pgfsetstrokecolor{currentstroke}%
\pgfsetstrokeopacity{0.200000}%
\pgfsetdash{}{0pt}%
\pgfpathmoveto{\pgfqpoint{2.450737in}{0.990101in}}%
\pgfpathlineto{\pgfqpoint{3.401164in}{1.890602in}}%
\pgfusepath{stroke}%
\end{pgfscope}%
\begin{pgfscope}%
\pgfpathrectangle{\pgfqpoint{0.481978in}{0.331635in}}{\pgfqpoint{4.960000in}{3.696000in}}%
\pgfusepath{clip}%
\pgfsetrectcap%
\pgfsetroundjoin%
\pgfsetlinewidth{1.505625pt}%
\definecolor{currentstroke}{rgb}{0.631373,0.788235,0.956863}%
\pgfsetstrokecolor{currentstroke}%
\pgfsetstrokeopacity{0.200000}%
\pgfsetdash{}{0pt}%
\pgfpathmoveto{\pgfqpoint{3.452671in}{0.676285in}}%
\pgfpathlineto{\pgfqpoint{3.401164in}{1.890602in}}%
\pgfusepath{stroke}%
\end{pgfscope}%
\begin{pgfscope}%
\pgfpathrectangle{\pgfqpoint{0.481978in}{0.331635in}}{\pgfqpoint{4.960000in}{3.696000in}}%
\pgfusepath{clip}%
\pgfsetrectcap%
\pgfsetroundjoin%
\pgfsetlinewidth{1.505625pt}%
\definecolor{currentstroke}{rgb}{0.631373,0.788235,0.956863}%
\pgfsetstrokecolor{currentstroke}%
\pgfsetstrokeopacity{0.200000}%
\pgfsetdash{}{0pt}%
\pgfpathmoveto{\pgfqpoint{2.908589in}{0.653209in}}%
\pgfpathlineto{\pgfqpoint{3.401164in}{1.890602in}}%
\pgfusepath{stroke}%
\end{pgfscope}%
\begin{pgfscope}%
\pgfpathrectangle{\pgfqpoint{0.481978in}{0.331635in}}{\pgfqpoint{4.960000in}{3.696000in}}%
\pgfusepath{clip}%
\pgfsetrectcap%
\pgfsetroundjoin%
\pgfsetlinewidth{1.505625pt}%
\definecolor{currentstroke}{rgb}{0.631373,0.788235,0.956863}%
\pgfsetstrokecolor{currentstroke}%
\pgfsetstrokeopacity{0.200000}%
\pgfsetdash{}{0pt}%
\pgfpathmoveto{\pgfqpoint{3.525361in}{2.836750in}}%
\pgfpathlineto{\pgfqpoint{3.401164in}{1.890602in}}%
\pgfusepath{stroke}%
\end{pgfscope}%
\begin{pgfscope}%
\pgfpathrectangle{\pgfqpoint{0.481978in}{0.331635in}}{\pgfqpoint{4.960000in}{3.696000in}}%
\pgfusepath{clip}%
\pgfsetrectcap%
\pgfsetroundjoin%
\pgfsetlinewidth{1.505625pt}%
\definecolor{currentstroke}{rgb}{0.631373,0.788235,0.956863}%
\pgfsetstrokecolor{currentstroke}%
\pgfsetstrokeopacity{0.200000}%
\pgfsetdash{}{0pt}%
\pgfpathmoveto{\pgfqpoint{3.422212in}{2.712644in}}%
\pgfpathlineto{\pgfqpoint{3.401164in}{1.890602in}}%
\pgfusepath{stroke}%
\end{pgfscope}%
\begin{pgfscope}%
\pgfpathrectangle{\pgfqpoint{0.481978in}{0.331635in}}{\pgfqpoint{4.960000in}{3.696000in}}%
\pgfusepath{clip}%
\pgfsetrectcap%
\pgfsetroundjoin%
\pgfsetlinewidth{1.505625pt}%
\definecolor{currentstroke}{rgb}{0.631373,0.788235,0.956863}%
\pgfsetstrokecolor{currentstroke}%
\pgfsetstrokeopacity{0.200000}%
\pgfsetdash{}{0pt}%
\pgfpathmoveto{\pgfqpoint{3.435148in}{2.499654in}}%
\pgfpathlineto{\pgfqpoint{3.401164in}{1.890602in}}%
\pgfusepath{stroke}%
\end{pgfscope}%
\begin{pgfscope}%
\pgfpathrectangle{\pgfqpoint{0.481978in}{0.331635in}}{\pgfqpoint{4.960000in}{3.696000in}}%
\pgfusepath{clip}%
\pgfsetrectcap%
\pgfsetroundjoin%
\pgfsetlinewidth{1.505625pt}%
\definecolor{currentstroke}{rgb}{0.631373,0.788235,0.956863}%
\pgfsetstrokecolor{currentstroke}%
\pgfsetstrokeopacity{0.200000}%
\pgfsetdash{}{0pt}%
\pgfpathmoveto{\pgfqpoint{3.319694in}{2.356556in}}%
\pgfpathlineto{\pgfqpoint{3.401164in}{1.890602in}}%
\pgfusepath{stroke}%
\end{pgfscope}%
\begin{pgfscope}%
\pgfpathrectangle{\pgfqpoint{0.481978in}{0.331635in}}{\pgfqpoint{4.960000in}{3.696000in}}%
\pgfusepath{clip}%
\pgfsetrectcap%
\pgfsetroundjoin%
\pgfsetlinewidth{1.505625pt}%
\definecolor{currentstroke}{rgb}{0.631373,0.788235,0.956863}%
\pgfsetstrokecolor{currentstroke}%
\pgfsetstrokeopacity{0.200000}%
\pgfsetdash{}{0pt}%
\pgfpathmoveto{\pgfqpoint{1.843632in}{2.351200in}}%
\pgfpathlineto{\pgfqpoint{3.401164in}{1.890602in}}%
\pgfusepath{stroke}%
\end{pgfscope}%
\begin{pgfscope}%
\pgfpathrectangle{\pgfqpoint{0.481978in}{0.331635in}}{\pgfqpoint{4.960000in}{3.696000in}}%
\pgfusepath{clip}%
\pgfsetrectcap%
\pgfsetroundjoin%
\pgfsetlinewidth{1.505625pt}%
\definecolor{currentstroke}{rgb}{0.631373,0.788235,0.956863}%
\pgfsetstrokecolor{currentstroke}%
\pgfsetstrokeopacity{0.200000}%
\pgfsetdash{}{0pt}%
\pgfpathmoveto{\pgfqpoint{3.396710in}{2.236419in}}%
\pgfpathlineto{\pgfqpoint{3.401164in}{1.890602in}}%
\pgfusepath{stroke}%
\end{pgfscope}%
\begin{pgfscope}%
\pgfpathrectangle{\pgfqpoint{0.481978in}{0.331635in}}{\pgfqpoint{4.960000in}{3.696000in}}%
\pgfusepath{clip}%
\pgfsetrectcap%
\pgfsetroundjoin%
\pgfsetlinewidth{1.505625pt}%
\definecolor{currentstroke}{rgb}{0.631373,0.788235,0.956863}%
\pgfsetstrokecolor{currentstroke}%
\pgfsetstrokeopacity{0.200000}%
\pgfsetdash{}{0pt}%
\pgfpathmoveto{\pgfqpoint{2.212263in}{2.188841in}}%
\pgfpathlineto{\pgfqpoint{3.401164in}{1.890602in}}%
\pgfusepath{stroke}%
\end{pgfscope}%
\begin{pgfscope}%
\pgfpathrectangle{\pgfqpoint{0.481978in}{0.331635in}}{\pgfqpoint{4.960000in}{3.696000in}}%
\pgfusepath{clip}%
\pgfsetrectcap%
\pgfsetroundjoin%
\pgfsetlinewidth{1.505625pt}%
\definecolor{currentstroke}{rgb}{0.631373,0.788235,0.956863}%
\pgfsetstrokecolor{currentstroke}%
\pgfsetstrokeopacity{0.200000}%
\pgfsetdash{}{0pt}%
\pgfpathmoveto{\pgfqpoint{4.964209in}{2.320271in}}%
\pgfpathlineto{\pgfqpoint{3.401164in}{1.890602in}}%
\pgfusepath{stroke}%
\end{pgfscope}%
\begin{pgfscope}%
\pgfpathrectangle{\pgfqpoint{0.481978in}{0.331635in}}{\pgfqpoint{4.960000in}{3.696000in}}%
\pgfusepath{clip}%
\pgfsetrectcap%
\pgfsetroundjoin%
\pgfsetlinewidth{1.505625pt}%
\definecolor{currentstroke}{rgb}{0.631373,0.788235,0.956863}%
\pgfsetstrokecolor{currentstroke}%
\pgfsetstrokeopacity{0.200000}%
\pgfsetdash{}{0pt}%
\pgfpathmoveto{\pgfqpoint{3.279740in}{0.578800in}}%
\pgfpathlineto{\pgfqpoint{3.401164in}{1.890602in}}%
\pgfusepath{stroke}%
\end{pgfscope}%
\begin{pgfscope}%
\pgfpathrectangle{\pgfqpoint{0.481978in}{0.331635in}}{\pgfqpoint{4.960000in}{3.696000in}}%
\pgfusepath{clip}%
\pgfsetrectcap%
\pgfsetroundjoin%
\pgfsetlinewidth{1.505625pt}%
\definecolor{currentstroke}{rgb}{0.631373,0.788235,0.956863}%
\pgfsetstrokecolor{currentstroke}%
\pgfsetstrokeopacity{0.200000}%
\pgfsetdash{}{0pt}%
\pgfpathmoveto{\pgfqpoint{4.204936in}{2.221056in}}%
\pgfpathlineto{\pgfqpoint{3.401164in}{1.890602in}}%
\pgfusepath{stroke}%
\end{pgfscope}%
\begin{pgfscope}%
\pgfpathrectangle{\pgfqpoint{0.481978in}{0.331635in}}{\pgfqpoint{4.960000in}{3.696000in}}%
\pgfusepath{clip}%
\pgfsetrectcap%
\pgfsetroundjoin%
\pgfsetlinewidth{1.505625pt}%
\definecolor{currentstroke}{rgb}{0.631373,0.788235,0.956863}%
\pgfsetstrokecolor{currentstroke}%
\pgfsetstrokeopacity{0.200000}%
\pgfsetdash{}{0pt}%
\pgfpathmoveto{\pgfqpoint{4.440145in}{1.620531in}}%
\pgfpathlineto{\pgfqpoint{3.401164in}{1.890602in}}%
\pgfusepath{stroke}%
\end{pgfscope}%
\begin{pgfscope}%
\pgfpathrectangle{\pgfqpoint{0.481978in}{0.331635in}}{\pgfqpoint{4.960000in}{3.696000in}}%
\pgfusepath{clip}%
\pgfsetrectcap%
\pgfsetroundjoin%
\pgfsetlinewidth{1.505625pt}%
\definecolor{currentstroke}{rgb}{0.631373,0.788235,0.956863}%
\pgfsetstrokecolor{currentstroke}%
\pgfsetstrokeopacity{0.200000}%
\pgfsetdash{}{0pt}%
\pgfpathmoveto{\pgfqpoint{1.851390in}{0.984126in}}%
\pgfpathlineto{\pgfqpoint{3.401164in}{1.890602in}}%
\pgfusepath{stroke}%
\end{pgfscope}%
\begin{pgfscope}%
\pgfpathrectangle{\pgfqpoint{0.481978in}{0.331635in}}{\pgfqpoint{4.960000in}{3.696000in}}%
\pgfusepath{clip}%
\pgfsetrectcap%
\pgfsetroundjoin%
\pgfsetlinewidth{1.505625pt}%
\definecolor{currentstroke}{rgb}{0.631373,0.788235,0.956863}%
\pgfsetstrokecolor{currentstroke}%
\pgfsetstrokeopacity{0.200000}%
\pgfsetdash{}{0pt}%
\pgfpathmoveto{\pgfqpoint{3.151439in}{1.105908in}}%
\pgfpathlineto{\pgfqpoint{3.401164in}{1.890602in}}%
\pgfusepath{stroke}%
\end{pgfscope}%
\begin{pgfscope}%
\pgfpathrectangle{\pgfqpoint{0.481978in}{0.331635in}}{\pgfqpoint{4.960000in}{3.696000in}}%
\pgfusepath{clip}%
\pgfsetrectcap%
\pgfsetroundjoin%
\pgfsetlinewidth{1.505625pt}%
\definecolor{currentstroke}{rgb}{0.631373,0.788235,0.956863}%
\pgfsetstrokecolor{currentstroke}%
\pgfsetstrokeopacity{0.200000}%
\pgfsetdash{}{0pt}%
\pgfpathmoveto{\pgfqpoint{2.882809in}{3.021574in}}%
\pgfpathlineto{\pgfqpoint{3.401164in}{1.890602in}}%
\pgfusepath{stroke}%
\end{pgfscope}%
\begin{pgfscope}%
\pgfpathrectangle{\pgfqpoint{0.481978in}{0.331635in}}{\pgfqpoint{4.960000in}{3.696000in}}%
\pgfusepath{clip}%
\pgfsetrectcap%
\pgfsetroundjoin%
\pgfsetlinewidth{1.505625pt}%
\definecolor{currentstroke}{rgb}{0.631373,0.788235,0.956863}%
\pgfsetstrokecolor{currentstroke}%
\pgfsetstrokeopacity{0.200000}%
\pgfsetdash{}{0pt}%
\pgfpathmoveto{\pgfqpoint{3.349813in}{1.688718in}}%
\pgfpathlineto{\pgfqpoint{3.401164in}{1.890602in}}%
\pgfusepath{stroke}%
\end{pgfscope}%
\begin{pgfscope}%
\pgfpathrectangle{\pgfqpoint{0.481978in}{0.331635in}}{\pgfqpoint{4.960000in}{3.696000in}}%
\pgfusepath{clip}%
\pgfsetrectcap%
\pgfsetroundjoin%
\pgfsetlinewidth{1.505625pt}%
\definecolor{currentstroke}{rgb}{0.631373,0.788235,0.956863}%
\pgfsetstrokecolor{currentstroke}%
\pgfsetstrokeopacity{0.200000}%
\pgfsetdash{}{0pt}%
\pgfpathmoveto{\pgfqpoint{2.941881in}{0.912023in}}%
\pgfpathlineto{\pgfqpoint{3.401164in}{1.890602in}}%
\pgfusepath{stroke}%
\end{pgfscope}%
\begin{pgfscope}%
\pgfpathrectangle{\pgfqpoint{0.481978in}{0.331635in}}{\pgfqpoint{4.960000in}{3.696000in}}%
\pgfusepath{clip}%
\pgfsetrectcap%
\pgfsetroundjoin%
\pgfsetlinewidth{1.505625pt}%
\definecolor{currentstroke}{rgb}{0.631373,0.788235,0.956863}%
\pgfsetstrokecolor{currentstroke}%
\pgfsetstrokeopacity{0.200000}%
\pgfsetdash{}{0pt}%
\pgfpathmoveto{\pgfqpoint{3.692146in}{2.382105in}}%
\pgfpathlineto{\pgfqpoint{3.401164in}{1.890602in}}%
\pgfusepath{stroke}%
\end{pgfscope}%
\begin{pgfscope}%
\pgfpathrectangle{\pgfqpoint{0.481978in}{0.331635in}}{\pgfqpoint{4.960000in}{3.696000in}}%
\pgfusepath{clip}%
\pgfsetrectcap%
\pgfsetroundjoin%
\pgfsetlinewidth{1.505625pt}%
\definecolor{currentstroke}{rgb}{0.631373,0.788235,0.956863}%
\pgfsetstrokecolor{currentstroke}%
\pgfsetstrokeopacity{0.200000}%
\pgfsetdash{}{0pt}%
\pgfpathmoveto{\pgfqpoint{2.675899in}{1.443444in}}%
\pgfpathlineto{\pgfqpoint{3.401164in}{1.890602in}}%
\pgfusepath{stroke}%
\end{pgfscope}%
\begin{pgfscope}%
\pgfpathrectangle{\pgfqpoint{0.481978in}{0.331635in}}{\pgfqpoint{4.960000in}{3.696000in}}%
\pgfusepath{clip}%
\pgfsetrectcap%
\pgfsetroundjoin%
\pgfsetlinewidth{1.505625pt}%
\definecolor{currentstroke}{rgb}{0.631373,0.788235,0.956863}%
\pgfsetstrokecolor{currentstroke}%
\pgfsetstrokeopacity{0.200000}%
\pgfsetdash{}{0pt}%
\pgfpathmoveto{\pgfqpoint{1.662123in}{1.420270in}}%
\pgfpathlineto{\pgfqpoint{3.401164in}{1.890602in}}%
\pgfusepath{stroke}%
\end{pgfscope}%
\begin{pgfscope}%
\pgfpathrectangle{\pgfqpoint{0.481978in}{0.331635in}}{\pgfqpoint{4.960000in}{3.696000in}}%
\pgfusepath{clip}%
\pgfsetrectcap%
\pgfsetroundjoin%
\pgfsetlinewidth{1.505625pt}%
\definecolor{currentstroke}{rgb}{0.631373,0.788235,0.956863}%
\pgfsetstrokecolor{currentstroke}%
\pgfsetstrokeopacity{0.200000}%
\pgfsetdash{}{0pt}%
\pgfpathmoveto{\pgfqpoint{3.493611in}{1.947127in}}%
\pgfpathlineto{\pgfqpoint{3.401164in}{1.890602in}}%
\pgfusepath{stroke}%
\end{pgfscope}%
\begin{pgfscope}%
\pgfpathrectangle{\pgfqpoint{0.481978in}{0.331635in}}{\pgfqpoint{4.960000in}{3.696000in}}%
\pgfusepath{clip}%
\pgfsetrectcap%
\pgfsetroundjoin%
\pgfsetlinewidth{1.505625pt}%
\definecolor{currentstroke}{rgb}{0.631373,0.788235,0.956863}%
\pgfsetstrokecolor{currentstroke}%
\pgfsetstrokeopacity{0.200000}%
\pgfsetdash{}{0pt}%
\pgfpathmoveto{\pgfqpoint{2.998121in}{1.009636in}}%
\pgfpathlineto{\pgfqpoint{3.401164in}{1.890602in}}%
\pgfusepath{stroke}%
\end{pgfscope}%
\begin{pgfscope}%
\pgfpathrectangle{\pgfqpoint{0.481978in}{0.331635in}}{\pgfqpoint{4.960000in}{3.696000in}}%
\pgfusepath{clip}%
\pgfsetrectcap%
\pgfsetroundjoin%
\pgfsetlinewidth{1.505625pt}%
\definecolor{currentstroke}{rgb}{0.631373,0.788235,0.956863}%
\pgfsetstrokecolor{currentstroke}%
\pgfsetstrokeopacity{0.200000}%
\pgfsetdash{}{0pt}%
\pgfpathmoveto{\pgfqpoint{2.017138in}{0.899956in}}%
\pgfpathlineto{\pgfqpoint{3.401164in}{1.890602in}}%
\pgfusepath{stroke}%
\end{pgfscope}%
\begin{pgfscope}%
\pgfpathrectangle{\pgfqpoint{0.481978in}{0.331635in}}{\pgfqpoint{4.960000in}{3.696000in}}%
\pgfusepath{clip}%
\pgfsetrectcap%
\pgfsetroundjoin%
\pgfsetlinewidth{1.505625pt}%
\definecolor{currentstroke}{rgb}{0.631373,0.788235,0.956863}%
\pgfsetstrokecolor{currentstroke}%
\pgfsetstrokeopacity{0.200000}%
\pgfsetdash{}{0pt}%
\pgfpathmoveto{\pgfqpoint{2.584736in}{1.083083in}}%
\pgfpathlineto{\pgfqpoint{3.401164in}{1.890602in}}%
\pgfusepath{stroke}%
\end{pgfscope}%
\begin{pgfscope}%
\pgfpathrectangle{\pgfqpoint{0.481978in}{0.331635in}}{\pgfqpoint{4.960000in}{3.696000in}}%
\pgfusepath{clip}%
\pgfsetrectcap%
\pgfsetroundjoin%
\pgfsetlinewidth{1.505625pt}%
\definecolor{currentstroke}{rgb}{0.631373,0.788235,0.956863}%
\pgfsetstrokecolor{currentstroke}%
\pgfsetstrokeopacity{0.200000}%
\pgfsetdash{}{0pt}%
\pgfpathmoveto{\pgfqpoint{2.772380in}{0.951027in}}%
\pgfpathlineto{\pgfqpoint{3.401164in}{1.890602in}}%
\pgfusepath{stroke}%
\end{pgfscope}%
\begin{pgfscope}%
\pgfpathrectangle{\pgfqpoint{0.481978in}{0.331635in}}{\pgfqpoint{4.960000in}{3.696000in}}%
\pgfusepath{clip}%
\pgfsetrectcap%
\pgfsetroundjoin%
\pgfsetlinewidth{1.505625pt}%
\definecolor{currentstroke}{rgb}{0.631373,0.788235,0.956863}%
\pgfsetstrokecolor{currentstroke}%
\pgfsetstrokeopacity{0.200000}%
\pgfsetdash{}{0pt}%
\pgfpathmoveto{\pgfqpoint{3.839237in}{1.806847in}}%
\pgfpathlineto{\pgfqpoint{3.401164in}{1.890602in}}%
\pgfusepath{stroke}%
\end{pgfscope}%
\begin{pgfscope}%
\pgfpathrectangle{\pgfqpoint{0.481978in}{0.331635in}}{\pgfqpoint{4.960000in}{3.696000in}}%
\pgfusepath{clip}%
\pgfsetrectcap%
\pgfsetroundjoin%
\pgfsetlinewidth{1.505625pt}%
\definecolor{currentstroke}{rgb}{0.631373,0.788235,0.956863}%
\pgfsetstrokecolor{currentstroke}%
\pgfsetstrokeopacity{0.200000}%
\pgfsetdash{}{0pt}%
\pgfpathmoveto{\pgfqpoint{3.923632in}{3.464039in}}%
\pgfpathlineto{\pgfqpoint{3.401164in}{1.890602in}}%
\pgfusepath{stroke}%
\end{pgfscope}%
\begin{pgfscope}%
\pgfpathrectangle{\pgfqpoint{0.481978in}{0.331635in}}{\pgfqpoint{4.960000in}{3.696000in}}%
\pgfusepath{clip}%
\pgfsetrectcap%
\pgfsetroundjoin%
\pgfsetlinewidth{1.505625pt}%
\definecolor{currentstroke}{rgb}{0.631373,0.788235,0.956863}%
\pgfsetstrokecolor{currentstroke}%
\pgfsetstrokeopacity{0.200000}%
\pgfsetdash{}{0pt}%
\pgfpathmoveto{\pgfqpoint{3.682114in}{1.870308in}}%
\pgfpathlineto{\pgfqpoint{3.401164in}{1.890602in}}%
\pgfusepath{stroke}%
\end{pgfscope}%
\begin{pgfscope}%
\pgfpathrectangle{\pgfqpoint{0.481978in}{0.331635in}}{\pgfqpoint{4.960000in}{3.696000in}}%
\pgfusepath{clip}%
\pgfsetrectcap%
\pgfsetroundjoin%
\pgfsetlinewidth{1.505625pt}%
\definecolor{currentstroke}{rgb}{0.631373,0.788235,0.956863}%
\pgfsetstrokecolor{currentstroke}%
\pgfsetstrokeopacity{0.200000}%
\pgfsetdash{}{0pt}%
\pgfpathmoveto{\pgfqpoint{3.060682in}{1.672629in}}%
\pgfpathlineto{\pgfqpoint{3.401164in}{1.890602in}}%
\pgfusepath{stroke}%
\end{pgfscope}%
\begin{pgfscope}%
\pgfpathrectangle{\pgfqpoint{0.481978in}{0.331635in}}{\pgfqpoint{4.960000in}{3.696000in}}%
\pgfusepath{clip}%
\pgfsetrectcap%
\pgfsetroundjoin%
\pgfsetlinewidth{1.505625pt}%
\definecolor{currentstroke}{rgb}{0.631373,0.788235,0.956863}%
\pgfsetstrokecolor{currentstroke}%
\pgfsetstrokeopacity{0.200000}%
\pgfsetdash{}{0pt}%
\pgfpathmoveto{\pgfqpoint{4.145626in}{2.830323in}}%
\pgfpathlineto{\pgfqpoint{3.401164in}{1.890602in}}%
\pgfusepath{stroke}%
\end{pgfscope}%
\begin{pgfscope}%
\pgfpathrectangle{\pgfqpoint{0.481978in}{0.331635in}}{\pgfqpoint{4.960000in}{3.696000in}}%
\pgfusepath{clip}%
\pgfsetrectcap%
\pgfsetroundjoin%
\pgfsetlinewidth{1.505625pt}%
\definecolor{currentstroke}{rgb}{0.631373,0.788235,0.956863}%
\pgfsetstrokecolor{currentstroke}%
\pgfsetstrokeopacity{0.200000}%
\pgfsetdash{}{0pt}%
\pgfpathmoveto{\pgfqpoint{2.477464in}{0.769971in}}%
\pgfpathlineto{\pgfqpoint{3.401164in}{1.890602in}}%
\pgfusepath{stroke}%
\end{pgfscope}%
\begin{pgfscope}%
\pgfpathrectangle{\pgfqpoint{0.481978in}{0.331635in}}{\pgfqpoint{4.960000in}{3.696000in}}%
\pgfusepath{clip}%
\pgfsetrectcap%
\pgfsetroundjoin%
\pgfsetlinewidth{1.505625pt}%
\definecolor{currentstroke}{rgb}{0.631373,0.788235,0.956863}%
\pgfsetstrokecolor{currentstroke}%
\pgfsetstrokeopacity{0.200000}%
\pgfsetdash{}{0pt}%
\pgfpathmoveto{\pgfqpoint{3.293279in}{2.612242in}}%
\pgfpathlineto{\pgfqpoint{3.401164in}{1.890602in}}%
\pgfusepath{stroke}%
\end{pgfscope}%
\begin{pgfscope}%
\pgfpathrectangle{\pgfqpoint{0.481978in}{0.331635in}}{\pgfqpoint{4.960000in}{3.696000in}}%
\pgfusepath{clip}%
\pgfsetrectcap%
\pgfsetroundjoin%
\pgfsetlinewidth{1.505625pt}%
\definecolor{currentstroke}{rgb}{0.631373,0.788235,0.956863}%
\pgfsetstrokecolor{currentstroke}%
\pgfsetstrokeopacity{0.200000}%
\pgfsetdash{}{0pt}%
\pgfpathmoveto{\pgfqpoint{2.485627in}{0.705547in}}%
\pgfpathlineto{\pgfqpoint{3.401164in}{1.890602in}}%
\pgfusepath{stroke}%
\end{pgfscope}%
\begin{pgfscope}%
\pgfpathrectangle{\pgfqpoint{0.481978in}{0.331635in}}{\pgfqpoint{4.960000in}{3.696000in}}%
\pgfusepath{clip}%
\pgfsetrectcap%
\pgfsetroundjoin%
\pgfsetlinewidth{1.505625pt}%
\definecolor{currentstroke}{rgb}{0.631373,0.788235,0.956863}%
\pgfsetstrokecolor{currentstroke}%
\pgfsetstrokeopacity{0.200000}%
\pgfsetdash{}{0pt}%
\pgfpathmoveto{\pgfqpoint{5.020912in}{2.315206in}}%
\pgfpathlineto{\pgfqpoint{3.401164in}{1.890602in}}%
\pgfusepath{stroke}%
\end{pgfscope}%
\begin{pgfscope}%
\pgfpathrectangle{\pgfqpoint{0.481978in}{0.331635in}}{\pgfqpoint{4.960000in}{3.696000in}}%
\pgfusepath{clip}%
\pgfsetrectcap%
\pgfsetroundjoin%
\pgfsetlinewidth{1.505625pt}%
\definecolor{currentstroke}{rgb}{0.631373,0.788235,0.956863}%
\pgfsetstrokecolor{currentstroke}%
\pgfsetstrokeopacity{0.200000}%
\pgfsetdash{}{0pt}%
\pgfpathmoveto{\pgfqpoint{4.054103in}{1.974737in}}%
\pgfpathlineto{\pgfqpoint{3.401164in}{1.890602in}}%
\pgfusepath{stroke}%
\end{pgfscope}%
\begin{pgfscope}%
\pgfpathrectangle{\pgfqpoint{0.481978in}{0.331635in}}{\pgfqpoint{4.960000in}{3.696000in}}%
\pgfusepath{clip}%
\pgfsetrectcap%
\pgfsetroundjoin%
\pgfsetlinewidth{1.505625pt}%
\definecolor{currentstroke}{rgb}{0.631373,0.788235,0.956863}%
\pgfsetstrokecolor{currentstroke}%
\pgfsetstrokeopacity{0.200000}%
\pgfsetdash{}{0pt}%
\pgfpathmoveto{\pgfqpoint{3.589725in}{1.564796in}}%
\pgfpathlineto{\pgfqpoint{3.401164in}{1.890602in}}%
\pgfusepath{stroke}%
\end{pgfscope}%
\begin{pgfscope}%
\pgfpathrectangle{\pgfqpoint{0.481978in}{0.331635in}}{\pgfqpoint{4.960000in}{3.696000in}}%
\pgfusepath{clip}%
\pgfsetrectcap%
\pgfsetroundjoin%
\pgfsetlinewidth{1.505625pt}%
\definecolor{currentstroke}{rgb}{0.631373,0.788235,0.956863}%
\pgfsetstrokecolor{currentstroke}%
\pgfsetstrokeopacity{0.200000}%
\pgfsetdash{}{0pt}%
\pgfpathmoveto{\pgfqpoint{2.685295in}{1.237940in}}%
\pgfpathlineto{\pgfqpoint{3.401164in}{1.890602in}}%
\pgfusepath{stroke}%
\end{pgfscope}%
\begin{pgfscope}%
\pgfpathrectangle{\pgfqpoint{0.481978in}{0.331635in}}{\pgfqpoint{4.960000in}{3.696000in}}%
\pgfusepath{clip}%
\pgfsetrectcap%
\pgfsetroundjoin%
\pgfsetlinewidth{1.505625pt}%
\definecolor{currentstroke}{rgb}{0.631373,0.788235,0.956863}%
\pgfsetstrokecolor{currentstroke}%
\pgfsetstrokeopacity{0.200000}%
\pgfsetdash{}{0pt}%
\pgfpathmoveto{\pgfqpoint{4.076072in}{3.105802in}}%
\pgfpathlineto{\pgfqpoint{3.401164in}{1.890602in}}%
\pgfusepath{stroke}%
\end{pgfscope}%
\begin{pgfscope}%
\pgfpathrectangle{\pgfqpoint{0.481978in}{0.331635in}}{\pgfqpoint{4.960000in}{3.696000in}}%
\pgfusepath{clip}%
\pgfsetrectcap%
\pgfsetroundjoin%
\pgfsetlinewidth{1.505625pt}%
\definecolor{currentstroke}{rgb}{0.631373,0.788235,0.956863}%
\pgfsetstrokecolor{currentstroke}%
\pgfsetstrokeopacity{0.200000}%
\pgfsetdash{}{0pt}%
\pgfpathmoveto{\pgfqpoint{4.014806in}{2.164575in}}%
\pgfpathlineto{\pgfqpoint{3.401164in}{1.890602in}}%
\pgfusepath{stroke}%
\end{pgfscope}%
\begin{pgfscope}%
\pgfpathrectangle{\pgfqpoint{0.481978in}{0.331635in}}{\pgfqpoint{4.960000in}{3.696000in}}%
\pgfusepath{clip}%
\pgfsetrectcap%
\pgfsetroundjoin%
\pgfsetlinewidth{1.505625pt}%
\definecolor{currentstroke}{rgb}{0.631373,0.788235,0.956863}%
\pgfsetstrokecolor{currentstroke}%
\pgfsetstrokeopacity{0.200000}%
\pgfsetdash{}{0pt}%
\pgfpathmoveto{\pgfqpoint{3.669689in}{2.329972in}}%
\pgfpathlineto{\pgfqpoint{3.401164in}{1.890602in}}%
\pgfusepath{stroke}%
\end{pgfscope}%
\begin{pgfscope}%
\pgfpathrectangle{\pgfqpoint{0.481978in}{0.331635in}}{\pgfqpoint{4.960000in}{3.696000in}}%
\pgfusepath{clip}%
\pgfsetrectcap%
\pgfsetroundjoin%
\pgfsetlinewidth{1.505625pt}%
\definecolor{currentstroke}{rgb}{0.631373,0.788235,0.956863}%
\pgfsetstrokecolor{currentstroke}%
\pgfsetstrokeopacity{0.200000}%
\pgfsetdash{}{0pt}%
\pgfpathmoveto{\pgfqpoint{3.792597in}{2.251662in}}%
\pgfpathlineto{\pgfqpoint{3.401164in}{1.890602in}}%
\pgfusepath{stroke}%
\end{pgfscope}%
\begin{pgfscope}%
\pgfpathrectangle{\pgfqpoint{0.481978in}{0.331635in}}{\pgfqpoint{4.960000in}{3.696000in}}%
\pgfusepath{clip}%
\pgfsetrectcap%
\pgfsetroundjoin%
\pgfsetlinewidth{1.505625pt}%
\definecolor{currentstroke}{rgb}{0.631373,0.788235,0.956863}%
\pgfsetstrokecolor{currentstroke}%
\pgfsetstrokeopacity{0.200000}%
\pgfsetdash{}{0pt}%
\pgfpathmoveto{\pgfqpoint{3.517960in}{2.908944in}}%
\pgfpathlineto{\pgfqpoint{3.401164in}{1.890602in}}%
\pgfusepath{stroke}%
\end{pgfscope}%
\begin{pgfscope}%
\pgfpathrectangle{\pgfqpoint{0.481978in}{0.331635in}}{\pgfqpoint{4.960000in}{3.696000in}}%
\pgfusepath{clip}%
\pgfsetrectcap%
\pgfsetroundjoin%
\pgfsetlinewidth{1.505625pt}%
\definecolor{currentstroke}{rgb}{0.631373,0.788235,0.956863}%
\pgfsetstrokecolor{currentstroke}%
\pgfsetstrokeopacity{0.200000}%
\pgfsetdash{}{0pt}%
\pgfpathmoveto{\pgfqpoint{2.684754in}{0.733642in}}%
\pgfpathlineto{\pgfqpoint{3.401164in}{1.890602in}}%
\pgfusepath{stroke}%
\end{pgfscope}%
\begin{pgfscope}%
\pgfpathrectangle{\pgfqpoint{0.481978in}{0.331635in}}{\pgfqpoint{4.960000in}{3.696000in}}%
\pgfusepath{clip}%
\pgfsetrectcap%
\pgfsetroundjoin%
\pgfsetlinewidth{1.505625pt}%
\definecolor{currentstroke}{rgb}{0.631373,0.788235,0.956863}%
\pgfsetstrokecolor{currentstroke}%
\pgfsetstrokeopacity{0.200000}%
\pgfsetdash{}{0pt}%
\pgfpathmoveto{\pgfqpoint{3.654791in}{2.781206in}}%
\pgfpathlineto{\pgfqpoint{3.401164in}{1.890602in}}%
\pgfusepath{stroke}%
\end{pgfscope}%
\begin{pgfscope}%
\pgfpathrectangle{\pgfqpoint{0.481978in}{0.331635in}}{\pgfqpoint{4.960000in}{3.696000in}}%
\pgfusepath{clip}%
\pgfsetrectcap%
\pgfsetroundjoin%
\pgfsetlinewidth{1.505625pt}%
\definecolor{currentstroke}{rgb}{0.631373,0.788235,0.956863}%
\pgfsetstrokecolor{currentstroke}%
\pgfsetstrokeopacity{0.200000}%
\pgfsetdash{}{0pt}%
\pgfpathmoveto{\pgfqpoint{3.969303in}{1.648094in}}%
\pgfpathlineto{\pgfqpoint{3.401164in}{1.890602in}}%
\pgfusepath{stroke}%
\end{pgfscope}%
\begin{pgfscope}%
\pgfpathrectangle{\pgfqpoint{0.481978in}{0.331635in}}{\pgfqpoint{4.960000in}{3.696000in}}%
\pgfusepath{clip}%
\pgfsetrectcap%
\pgfsetroundjoin%
\pgfsetlinewidth{1.505625pt}%
\definecolor{currentstroke}{rgb}{0.631373,0.788235,0.956863}%
\pgfsetstrokecolor{currentstroke}%
\pgfsetstrokeopacity{0.200000}%
\pgfsetdash{}{0pt}%
\pgfpathmoveto{\pgfqpoint{3.236827in}{2.234241in}}%
\pgfpathlineto{\pgfqpoint{3.401164in}{1.890602in}}%
\pgfusepath{stroke}%
\end{pgfscope}%
\begin{pgfscope}%
\pgfpathrectangle{\pgfqpoint{0.481978in}{0.331635in}}{\pgfqpoint{4.960000in}{3.696000in}}%
\pgfusepath{clip}%
\pgfsetrectcap%
\pgfsetroundjoin%
\pgfsetlinewidth{1.505625pt}%
\definecolor{currentstroke}{rgb}{0.631373,0.788235,0.956863}%
\pgfsetstrokecolor{currentstroke}%
\pgfsetstrokeopacity{0.200000}%
\pgfsetdash{}{0pt}%
\pgfpathmoveto{\pgfqpoint{4.281359in}{2.015083in}}%
\pgfpathlineto{\pgfqpoint{3.401164in}{1.890602in}}%
\pgfusepath{stroke}%
\end{pgfscope}%
\begin{pgfscope}%
\pgfpathrectangle{\pgfqpoint{0.481978in}{0.331635in}}{\pgfqpoint{4.960000in}{3.696000in}}%
\pgfusepath{clip}%
\pgfsetrectcap%
\pgfsetroundjoin%
\pgfsetlinewidth{1.505625pt}%
\definecolor{currentstroke}{rgb}{0.631373,0.788235,0.956863}%
\pgfsetstrokecolor{currentstroke}%
\pgfsetstrokeopacity{0.200000}%
\pgfsetdash{}{0pt}%
\pgfpathmoveto{\pgfqpoint{4.179788in}{1.802996in}}%
\pgfpathlineto{\pgfqpoint{3.401164in}{1.890602in}}%
\pgfusepath{stroke}%
\end{pgfscope}%
\begin{pgfscope}%
\pgfpathrectangle{\pgfqpoint{0.481978in}{0.331635in}}{\pgfqpoint{4.960000in}{3.696000in}}%
\pgfusepath{clip}%
\pgfsetrectcap%
\pgfsetroundjoin%
\pgfsetlinewidth{1.505625pt}%
\definecolor{currentstroke}{rgb}{0.631373,0.788235,0.956863}%
\pgfsetstrokecolor{currentstroke}%
\pgfsetstrokeopacity{0.200000}%
\pgfsetdash{}{0pt}%
\pgfpathmoveto{\pgfqpoint{4.022983in}{2.310383in}}%
\pgfpathlineto{\pgfqpoint{3.401164in}{1.890602in}}%
\pgfusepath{stroke}%
\end{pgfscope}%
\begin{pgfscope}%
\pgfpathrectangle{\pgfqpoint{0.481978in}{0.331635in}}{\pgfqpoint{4.960000in}{3.696000in}}%
\pgfusepath{clip}%
\pgfsetrectcap%
\pgfsetroundjoin%
\pgfsetlinewidth{1.505625pt}%
\definecolor{currentstroke}{rgb}{0.631373,0.788235,0.956863}%
\pgfsetstrokecolor{currentstroke}%
\pgfsetstrokeopacity{0.200000}%
\pgfsetdash{}{0pt}%
\pgfpathmoveto{\pgfqpoint{2.058588in}{0.834025in}}%
\pgfpathlineto{\pgfqpoint{3.401164in}{1.890602in}}%
\pgfusepath{stroke}%
\end{pgfscope}%
\begin{pgfscope}%
\pgfpathrectangle{\pgfqpoint{0.481978in}{0.331635in}}{\pgfqpoint{4.960000in}{3.696000in}}%
\pgfusepath{clip}%
\pgfsetrectcap%
\pgfsetroundjoin%
\pgfsetlinewidth{1.505625pt}%
\definecolor{currentstroke}{rgb}{0.631373,0.788235,0.956863}%
\pgfsetstrokecolor{currentstroke}%
\pgfsetstrokeopacity{0.200000}%
\pgfsetdash{}{0pt}%
\pgfpathmoveto{\pgfqpoint{4.242206in}{3.532939in}}%
\pgfpathlineto{\pgfqpoint{3.401164in}{1.890602in}}%
\pgfusepath{stroke}%
\end{pgfscope}%
\begin{pgfscope}%
\pgfpathrectangle{\pgfqpoint{0.481978in}{0.331635in}}{\pgfqpoint{4.960000in}{3.696000in}}%
\pgfusepath{clip}%
\pgfsetrectcap%
\pgfsetroundjoin%
\pgfsetlinewidth{1.505625pt}%
\definecolor{currentstroke}{rgb}{0.631373,0.788235,0.956863}%
\pgfsetstrokecolor{currentstroke}%
\pgfsetstrokeopacity{0.200000}%
\pgfsetdash{}{0pt}%
\pgfpathmoveto{\pgfqpoint{3.917773in}{2.710962in}}%
\pgfpathlineto{\pgfqpoint{3.401164in}{1.890602in}}%
\pgfusepath{stroke}%
\end{pgfscope}%
\begin{pgfscope}%
\pgfpathrectangle{\pgfqpoint{0.481978in}{0.331635in}}{\pgfqpoint{4.960000in}{3.696000in}}%
\pgfusepath{clip}%
\pgfsetrectcap%
\pgfsetroundjoin%
\pgfsetlinewidth{1.505625pt}%
\definecolor{currentstroke}{rgb}{0.631373,0.788235,0.956863}%
\pgfsetstrokecolor{currentstroke}%
\pgfsetstrokeopacity{0.200000}%
\pgfsetdash{}{0pt}%
\pgfpathmoveto{\pgfqpoint{3.910278in}{2.113831in}}%
\pgfpathlineto{\pgfqpoint{3.401164in}{1.890602in}}%
\pgfusepath{stroke}%
\end{pgfscope}%
\begin{pgfscope}%
\pgfpathrectangle{\pgfqpoint{0.481978in}{0.331635in}}{\pgfqpoint{4.960000in}{3.696000in}}%
\pgfusepath{clip}%
\pgfsetrectcap%
\pgfsetroundjoin%
\pgfsetlinewidth{1.505625pt}%
\definecolor{currentstroke}{rgb}{0.631373,0.788235,0.956863}%
\pgfsetstrokecolor{currentstroke}%
\pgfsetstrokeopacity{0.200000}%
\pgfsetdash{}{0pt}%
\pgfpathmoveto{\pgfqpoint{2.808566in}{1.282388in}}%
\pgfpathlineto{\pgfqpoint{3.401164in}{1.890602in}}%
\pgfusepath{stroke}%
\end{pgfscope}%
\begin{pgfscope}%
\pgfpathrectangle{\pgfqpoint{0.481978in}{0.331635in}}{\pgfqpoint{4.960000in}{3.696000in}}%
\pgfusepath{clip}%
\pgfsetrectcap%
\pgfsetroundjoin%
\pgfsetlinewidth{1.505625pt}%
\definecolor{currentstroke}{rgb}{0.631373,0.788235,0.956863}%
\pgfsetstrokecolor{currentstroke}%
\pgfsetstrokeopacity{0.200000}%
\pgfsetdash{}{0pt}%
\pgfpathmoveto{\pgfqpoint{3.721687in}{1.339714in}}%
\pgfpathlineto{\pgfqpoint{3.401164in}{1.890602in}}%
\pgfusepath{stroke}%
\end{pgfscope}%
\begin{pgfscope}%
\pgfpathrectangle{\pgfqpoint{0.481978in}{0.331635in}}{\pgfqpoint{4.960000in}{3.696000in}}%
\pgfusepath{clip}%
\pgfsetrectcap%
\pgfsetroundjoin%
\pgfsetlinewidth{1.505625pt}%
\definecolor{currentstroke}{rgb}{0.631373,0.788235,0.956863}%
\pgfsetstrokecolor{currentstroke}%
\pgfsetstrokeopacity{0.200000}%
\pgfsetdash{}{0pt}%
\pgfpathmoveto{\pgfqpoint{4.249049in}{3.540064in}}%
\pgfpathlineto{\pgfqpoint{3.401164in}{1.890602in}}%
\pgfusepath{stroke}%
\end{pgfscope}%
\begin{pgfscope}%
\pgfpathrectangle{\pgfqpoint{0.481978in}{0.331635in}}{\pgfqpoint{4.960000in}{3.696000in}}%
\pgfusepath{clip}%
\pgfsetrectcap%
\pgfsetroundjoin%
\pgfsetlinewidth{1.505625pt}%
\definecolor{currentstroke}{rgb}{0.631373,0.788235,0.956863}%
\pgfsetstrokecolor{currentstroke}%
\pgfsetstrokeopacity{0.200000}%
\pgfsetdash{}{0pt}%
\pgfpathmoveto{\pgfqpoint{4.318155in}{2.474009in}}%
\pgfpathlineto{\pgfqpoint{3.401164in}{1.890602in}}%
\pgfusepath{stroke}%
\end{pgfscope}%
\begin{pgfscope}%
\pgfpathrectangle{\pgfqpoint{0.481978in}{0.331635in}}{\pgfqpoint{4.960000in}{3.696000in}}%
\pgfusepath{clip}%
\pgfsetrectcap%
\pgfsetroundjoin%
\pgfsetlinewidth{1.505625pt}%
\definecolor{currentstroke}{rgb}{0.631373,0.788235,0.956863}%
\pgfsetstrokecolor{currentstroke}%
\pgfsetstrokeopacity{0.200000}%
\pgfsetdash{}{0pt}%
\pgfpathmoveto{\pgfqpoint{2.648273in}{0.845828in}}%
\pgfpathlineto{\pgfqpoint{3.401164in}{1.890602in}}%
\pgfusepath{stroke}%
\end{pgfscope}%
\begin{pgfscope}%
\pgfpathrectangle{\pgfqpoint{0.481978in}{0.331635in}}{\pgfqpoint{4.960000in}{3.696000in}}%
\pgfusepath{clip}%
\pgfsetrectcap%
\pgfsetroundjoin%
\pgfsetlinewidth{1.505625pt}%
\definecolor{currentstroke}{rgb}{0.631373,0.788235,0.956863}%
\pgfsetstrokecolor{currentstroke}%
\pgfsetstrokeopacity{0.200000}%
\pgfsetdash{}{0pt}%
\pgfpathmoveto{\pgfqpoint{3.960912in}{3.220127in}}%
\pgfpathlineto{\pgfqpoint{3.401164in}{1.890602in}}%
\pgfusepath{stroke}%
\end{pgfscope}%
\begin{pgfscope}%
\pgfpathrectangle{\pgfqpoint{0.481978in}{0.331635in}}{\pgfqpoint{4.960000in}{3.696000in}}%
\pgfusepath{clip}%
\pgfsetrectcap%
\pgfsetroundjoin%
\pgfsetlinewidth{1.505625pt}%
\definecolor{currentstroke}{rgb}{0.631373,0.788235,0.956863}%
\pgfsetstrokecolor{currentstroke}%
\pgfsetstrokeopacity{0.200000}%
\pgfsetdash{}{0pt}%
\pgfpathmoveto{\pgfqpoint{3.165961in}{0.657161in}}%
\pgfpathlineto{\pgfqpoint{3.401164in}{1.890602in}}%
\pgfusepath{stroke}%
\end{pgfscope}%
\begin{pgfscope}%
\pgfpathrectangle{\pgfqpoint{0.481978in}{0.331635in}}{\pgfqpoint{4.960000in}{3.696000in}}%
\pgfusepath{clip}%
\pgfsetrectcap%
\pgfsetroundjoin%
\pgfsetlinewidth{1.505625pt}%
\definecolor{currentstroke}{rgb}{0.631373,0.788235,0.956863}%
\pgfsetstrokecolor{currentstroke}%
\pgfsetstrokeopacity{0.200000}%
\pgfsetdash{}{0pt}%
\pgfpathmoveto{\pgfqpoint{3.531854in}{3.289827in}}%
\pgfpathlineto{\pgfqpoint{3.401164in}{1.890602in}}%
\pgfusepath{stroke}%
\end{pgfscope}%
\begin{pgfscope}%
\pgfpathrectangle{\pgfqpoint{0.481978in}{0.331635in}}{\pgfqpoint{4.960000in}{3.696000in}}%
\pgfusepath{clip}%
\pgfsetrectcap%
\pgfsetroundjoin%
\pgfsetlinewidth{1.505625pt}%
\definecolor{currentstroke}{rgb}{0.631373,0.788235,0.956863}%
\pgfsetstrokecolor{currentstroke}%
\pgfsetstrokeopacity{0.200000}%
\pgfsetdash{}{0pt}%
\pgfpathmoveto{\pgfqpoint{3.193034in}{0.499635in}}%
\pgfpathlineto{\pgfqpoint{3.401164in}{1.890602in}}%
\pgfusepath{stroke}%
\end{pgfscope}%
\begin{pgfscope}%
\pgfpathrectangle{\pgfqpoint{0.481978in}{0.331635in}}{\pgfqpoint{4.960000in}{3.696000in}}%
\pgfusepath{clip}%
\pgfsetrectcap%
\pgfsetroundjoin%
\pgfsetlinewidth{1.505625pt}%
\definecolor{currentstroke}{rgb}{0.631373,0.788235,0.956863}%
\pgfsetstrokecolor{currentstroke}%
\pgfsetstrokeopacity{0.200000}%
\pgfsetdash{}{0pt}%
\pgfpathmoveto{\pgfqpoint{2.811954in}{0.797646in}}%
\pgfpathlineto{\pgfqpoint{3.401164in}{1.890602in}}%
\pgfusepath{stroke}%
\end{pgfscope}%
\begin{pgfscope}%
\pgfpathrectangle{\pgfqpoint{0.481978in}{0.331635in}}{\pgfqpoint{4.960000in}{3.696000in}}%
\pgfusepath{clip}%
\pgfsetrectcap%
\pgfsetroundjoin%
\pgfsetlinewidth{1.505625pt}%
\definecolor{currentstroke}{rgb}{0.631373,0.788235,0.956863}%
\pgfsetstrokecolor{currentstroke}%
\pgfsetstrokeopacity{0.200000}%
\pgfsetdash{}{0pt}%
\pgfpathmoveto{\pgfqpoint{4.845268in}{3.340519in}}%
\pgfpathlineto{\pgfqpoint{3.401164in}{1.890602in}}%
\pgfusepath{stroke}%
\end{pgfscope}%
\begin{pgfscope}%
\pgfpathrectangle{\pgfqpoint{0.481978in}{0.331635in}}{\pgfqpoint{4.960000in}{3.696000in}}%
\pgfusepath{clip}%
\pgfsetrectcap%
\pgfsetroundjoin%
\pgfsetlinewidth{1.505625pt}%
\definecolor{currentstroke}{rgb}{0.631373,0.788235,0.956863}%
\pgfsetstrokecolor{currentstroke}%
\pgfsetstrokeopacity{0.200000}%
\pgfsetdash{}{0pt}%
\pgfpathmoveto{\pgfqpoint{4.682828in}{2.629975in}}%
\pgfpathlineto{\pgfqpoint{3.401164in}{1.890602in}}%
\pgfusepath{stroke}%
\end{pgfscope}%
\begin{pgfscope}%
\pgfpathrectangle{\pgfqpoint{0.481978in}{0.331635in}}{\pgfqpoint{4.960000in}{3.696000in}}%
\pgfusepath{clip}%
\pgfsetrectcap%
\pgfsetroundjoin%
\pgfsetlinewidth{1.505625pt}%
\definecolor{currentstroke}{rgb}{0.631373,0.788235,0.956863}%
\pgfsetstrokecolor{currentstroke}%
\pgfsetstrokeopacity{0.200000}%
\pgfsetdash{}{0pt}%
\pgfpathmoveto{\pgfqpoint{4.592334in}{2.050304in}}%
\pgfpathlineto{\pgfqpoint{3.401164in}{1.890602in}}%
\pgfusepath{stroke}%
\end{pgfscope}%
\begin{pgfscope}%
\pgfpathrectangle{\pgfqpoint{0.481978in}{0.331635in}}{\pgfqpoint{4.960000in}{3.696000in}}%
\pgfusepath{clip}%
\pgfsetrectcap%
\pgfsetroundjoin%
\pgfsetlinewidth{1.505625pt}%
\definecolor{currentstroke}{rgb}{0.631373,0.788235,0.956863}%
\pgfsetstrokecolor{currentstroke}%
\pgfsetstrokeopacity{0.200000}%
\pgfsetdash{}{0pt}%
\pgfpathmoveto{\pgfqpoint{3.044931in}{3.082189in}}%
\pgfpathlineto{\pgfqpoint{3.401164in}{1.890602in}}%
\pgfusepath{stroke}%
\end{pgfscope}%
\begin{pgfscope}%
\pgfpathrectangle{\pgfqpoint{0.481978in}{0.331635in}}{\pgfqpoint{4.960000in}{3.696000in}}%
\pgfusepath{clip}%
\pgfsetrectcap%
\pgfsetroundjoin%
\pgfsetlinewidth{1.505625pt}%
\definecolor{currentstroke}{rgb}{0.631373,0.788235,0.956863}%
\pgfsetstrokecolor{currentstroke}%
\pgfsetstrokeopacity{0.200000}%
\pgfsetdash{}{0pt}%
\pgfpathmoveto{\pgfqpoint{3.543143in}{1.995884in}}%
\pgfpathlineto{\pgfqpoint{3.401164in}{1.890602in}}%
\pgfusepath{stroke}%
\end{pgfscope}%
\begin{pgfscope}%
\pgfpathrectangle{\pgfqpoint{0.481978in}{0.331635in}}{\pgfqpoint{4.960000in}{3.696000in}}%
\pgfusepath{clip}%
\pgfsetrectcap%
\pgfsetroundjoin%
\pgfsetlinewidth{1.505625pt}%
\definecolor{currentstroke}{rgb}{0.631373,0.788235,0.956863}%
\pgfsetstrokecolor{currentstroke}%
\pgfsetstrokeopacity{0.200000}%
\pgfsetdash{}{0pt}%
\pgfpathmoveto{\pgfqpoint{3.150587in}{1.354777in}}%
\pgfpathlineto{\pgfqpoint{3.401164in}{1.890602in}}%
\pgfusepath{stroke}%
\end{pgfscope}%
\begin{pgfscope}%
\pgfpathrectangle{\pgfqpoint{0.481978in}{0.331635in}}{\pgfqpoint{4.960000in}{3.696000in}}%
\pgfusepath{clip}%
\pgfsetrectcap%
\pgfsetroundjoin%
\pgfsetlinewidth{1.505625pt}%
\definecolor{currentstroke}{rgb}{0.631373,0.788235,0.956863}%
\pgfsetstrokecolor{currentstroke}%
\pgfsetstrokeopacity{0.200000}%
\pgfsetdash{}{0pt}%
\pgfpathmoveto{\pgfqpoint{3.468441in}{1.844187in}}%
\pgfpathlineto{\pgfqpoint{3.401164in}{1.890602in}}%
\pgfusepath{stroke}%
\end{pgfscope}%
\begin{pgfscope}%
\pgfpathrectangle{\pgfqpoint{0.481978in}{0.331635in}}{\pgfqpoint{4.960000in}{3.696000in}}%
\pgfusepath{clip}%
\pgfsetrectcap%
\pgfsetroundjoin%
\pgfsetlinewidth{1.505625pt}%
\definecolor{currentstroke}{rgb}{0.631373,0.788235,0.956863}%
\pgfsetstrokecolor{currentstroke}%
\pgfsetstrokeopacity{0.200000}%
\pgfsetdash{}{0pt}%
\pgfpathmoveto{\pgfqpoint{4.101947in}{2.211390in}}%
\pgfpathlineto{\pgfqpoint{3.401164in}{1.890602in}}%
\pgfusepath{stroke}%
\end{pgfscope}%
\begin{pgfscope}%
\pgfpathrectangle{\pgfqpoint{0.481978in}{0.331635in}}{\pgfqpoint{4.960000in}{3.696000in}}%
\pgfusepath{clip}%
\pgfsetrectcap%
\pgfsetroundjoin%
\pgfsetlinewidth{1.505625pt}%
\definecolor{currentstroke}{rgb}{0.631373,0.788235,0.956863}%
\pgfsetstrokecolor{currentstroke}%
\pgfsetstrokeopacity{0.200000}%
\pgfsetdash{}{0pt}%
\pgfpathmoveto{\pgfqpoint{3.201064in}{2.633111in}}%
\pgfpathlineto{\pgfqpoint{3.401164in}{1.890602in}}%
\pgfusepath{stroke}%
\end{pgfscope}%
\begin{pgfscope}%
\pgfpathrectangle{\pgfqpoint{0.481978in}{0.331635in}}{\pgfqpoint{4.960000in}{3.696000in}}%
\pgfusepath{clip}%
\pgfsetrectcap%
\pgfsetroundjoin%
\pgfsetlinewidth{1.505625pt}%
\definecolor{currentstroke}{rgb}{0.631373,0.788235,0.956863}%
\pgfsetstrokecolor{currentstroke}%
\pgfsetstrokeopacity{0.200000}%
\pgfsetdash{}{0pt}%
\pgfpathmoveto{\pgfqpoint{3.278230in}{0.597377in}}%
\pgfpathlineto{\pgfqpoint{3.401164in}{1.890602in}}%
\pgfusepath{stroke}%
\end{pgfscope}%
\begin{pgfscope}%
\pgfpathrectangle{\pgfqpoint{0.481978in}{0.331635in}}{\pgfqpoint{4.960000in}{3.696000in}}%
\pgfusepath{clip}%
\pgfsetrectcap%
\pgfsetroundjoin%
\pgfsetlinewidth{1.505625pt}%
\definecolor{currentstroke}{rgb}{0.631373,0.788235,0.956863}%
\pgfsetstrokecolor{currentstroke}%
\pgfsetstrokeopacity{0.200000}%
\pgfsetdash{}{0pt}%
\pgfpathmoveto{\pgfqpoint{2.957637in}{0.825052in}}%
\pgfpathlineto{\pgfqpoint{3.401164in}{1.890602in}}%
\pgfusepath{stroke}%
\end{pgfscope}%
\begin{pgfscope}%
\pgfpathrectangle{\pgfqpoint{0.481978in}{0.331635in}}{\pgfqpoint{4.960000in}{3.696000in}}%
\pgfusepath{clip}%
\pgfsetrectcap%
\pgfsetroundjoin%
\pgfsetlinewidth{1.505625pt}%
\definecolor{currentstroke}{rgb}{0.631373,0.788235,0.956863}%
\pgfsetstrokecolor{currentstroke}%
\pgfsetstrokeopacity{0.200000}%
\pgfsetdash{}{0pt}%
\pgfpathmoveto{\pgfqpoint{3.039657in}{2.917567in}}%
\pgfpathlineto{\pgfqpoint{3.401164in}{1.890602in}}%
\pgfusepath{stroke}%
\end{pgfscope}%
\begin{pgfscope}%
\pgfpathrectangle{\pgfqpoint{0.481978in}{0.331635in}}{\pgfqpoint{4.960000in}{3.696000in}}%
\pgfusepath{clip}%
\pgfsetrectcap%
\pgfsetroundjoin%
\pgfsetlinewidth{1.505625pt}%
\definecolor{currentstroke}{rgb}{0.631373,0.788235,0.956863}%
\pgfsetstrokecolor{currentstroke}%
\pgfsetstrokeopacity{0.200000}%
\pgfsetdash{}{0pt}%
\pgfpathmoveto{\pgfqpoint{4.156124in}{2.494663in}}%
\pgfpathlineto{\pgfqpoint{3.401164in}{1.890602in}}%
\pgfusepath{stroke}%
\end{pgfscope}%
\begin{pgfscope}%
\pgfpathrectangle{\pgfqpoint{0.481978in}{0.331635in}}{\pgfqpoint{4.960000in}{3.696000in}}%
\pgfusepath{clip}%
\pgfsetrectcap%
\pgfsetroundjoin%
\pgfsetlinewidth{1.505625pt}%
\definecolor{currentstroke}{rgb}{0.631373,0.788235,0.956863}%
\pgfsetstrokecolor{currentstroke}%
\pgfsetstrokeopacity{0.200000}%
\pgfsetdash{}{0pt}%
\pgfpathmoveto{\pgfqpoint{2.647467in}{0.552852in}}%
\pgfpathlineto{\pgfqpoint{3.401164in}{1.890602in}}%
\pgfusepath{stroke}%
\end{pgfscope}%
\begin{pgfscope}%
\pgfpathrectangle{\pgfqpoint{0.481978in}{0.331635in}}{\pgfqpoint{4.960000in}{3.696000in}}%
\pgfusepath{clip}%
\pgfsetrectcap%
\pgfsetroundjoin%
\pgfsetlinewidth{1.505625pt}%
\definecolor{currentstroke}{rgb}{0.631373,0.788235,0.956863}%
\pgfsetstrokecolor{currentstroke}%
\pgfsetstrokeopacity{0.200000}%
\pgfsetdash{}{0pt}%
\pgfpathmoveto{\pgfqpoint{3.643574in}{3.858304in}}%
\pgfpathlineto{\pgfqpoint{3.401164in}{1.890602in}}%
\pgfusepath{stroke}%
\end{pgfscope}%
\begin{pgfscope}%
\pgfpathrectangle{\pgfqpoint{0.481978in}{0.331635in}}{\pgfqpoint{4.960000in}{3.696000in}}%
\pgfusepath{clip}%
\pgfsetrectcap%
\pgfsetroundjoin%
\pgfsetlinewidth{1.505625pt}%
\definecolor{currentstroke}{rgb}{0.631373,0.788235,0.956863}%
\pgfsetstrokecolor{currentstroke}%
\pgfsetstrokeopacity{0.200000}%
\pgfsetdash{}{0pt}%
\pgfpathmoveto{\pgfqpoint{3.610736in}{2.637901in}}%
\pgfpathlineto{\pgfqpoint{3.401164in}{1.890602in}}%
\pgfusepath{stroke}%
\end{pgfscope}%
\begin{pgfscope}%
\pgfpathrectangle{\pgfqpoint{0.481978in}{0.331635in}}{\pgfqpoint{4.960000in}{3.696000in}}%
\pgfusepath{clip}%
\pgfsetrectcap%
\pgfsetroundjoin%
\pgfsetlinewidth{1.505625pt}%
\definecolor{currentstroke}{rgb}{0.631373,0.788235,0.956863}%
\pgfsetstrokecolor{currentstroke}%
\pgfsetstrokeopacity{0.200000}%
\pgfsetdash{}{0pt}%
\pgfpathmoveto{\pgfqpoint{3.986586in}{2.369703in}}%
\pgfpathlineto{\pgfqpoint{3.401164in}{1.890602in}}%
\pgfusepath{stroke}%
\end{pgfscope}%
\begin{pgfscope}%
\pgfpathrectangle{\pgfqpoint{0.481978in}{0.331635in}}{\pgfqpoint{4.960000in}{3.696000in}}%
\pgfusepath{clip}%
\pgfsetrectcap%
\pgfsetroundjoin%
\pgfsetlinewidth{1.505625pt}%
\definecolor{currentstroke}{rgb}{0.631373,0.788235,0.956863}%
\pgfsetstrokecolor{currentstroke}%
\pgfsetstrokeopacity{0.200000}%
\pgfsetdash{}{0pt}%
\pgfpathmoveto{\pgfqpoint{3.408961in}{2.771952in}}%
\pgfpathlineto{\pgfqpoint{3.401164in}{1.890602in}}%
\pgfusepath{stroke}%
\end{pgfscope}%
\begin{pgfscope}%
\pgfpathrectangle{\pgfqpoint{0.481978in}{0.331635in}}{\pgfqpoint{4.960000in}{3.696000in}}%
\pgfusepath{clip}%
\pgfsetrectcap%
\pgfsetroundjoin%
\pgfsetlinewidth{1.505625pt}%
\definecolor{currentstroke}{rgb}{0.631373,0.788235,0.956863}%
\pgfsetstrokecolor{currentstroke}%
\pgfsetstrokeopacity{0.200000}%
\pgfsetdash{}{0pt}%
\pgfpathmoveto{\pgfqpoint{2.884018in}{1.300623in}}%
\pgfpathlineto{\pgfqpoint{3.401164in}{1.890602in}}%
\pgfusepath{stroke}%
\end{pgfscope}%
\begin{pgfscope}%
\pgfpathrectangle{\pgfqpoint{0.481978in}{0.331635in}}{\pgfqpoint{4.960000in}{3.696000in}}%
\pgfusepath{clip}%
\pgfsetrectcap%
\pgfsetroundjoin%
\pgfsetlinewidth{1.505625pt}%
\definecolor{currentstroke}{rgb}{0.631373,0.788235,0.956863}%
\pgfsetstrokecolor{currentstroke}%
\pgfsetstrokeopacity{0.200000}%
\pgfsetdash{}{0pt}%
\pgfpathmoveto{\pgfqpoint{2.310399in}{1.007056in}}%
\pgfpathlineto{\pgfqpoint{3.401164in}{1.890602in}}%
\pgfusepath{stroke}%
\end{pgfscope}%
\begin{pgfscope}%
\pgfpathrectangle{\pgfqpoint{0.481978in}{0.331635in}}{\pgfqpoint{4.960000in}{3.696000in}}%
\pgfusepath{clip}%
\pgfsetrectcap%
\pgfsetroundjoin%
\pgfsetlinewidth{1.505625pt}%
\definecolor{currentstroke}{rgb}{0.631373,0.788235,0.956863}%
\pgfsetstrokecolor{currentstroke}%
\pgfsetstrokeopacity{0.200000}%
\pgfsetdash{}{0pt}%
\pgfpathmoveto{\pgfqpoint{3.865457in}{2.284367in}}%
\pgfpathlineto{\pgfqpoint{3.401164in}{1.890602in}}%
\pgfusepath{stroke}%
\end{pgfscope}%
\begin{pgfscope}%
\pgfpathrectangle{\pgfqpoint{0.481978in}{0.331635in}}{\pgfqpoint{4.960000in}{3.696000in}}%
\pgfusepath{clip}%
\pgfsetrectcap%
\pgfsetroundjoin%
\pgfsetlinewidth{1.505625pt}%
\definecolor{currentstroke}{rgb}{0.631373,0.788235,0.956863}%
\pgfsetstrokecolor{currentstroke}%
\pgfsetstrokeopacity{0.200000}%
\pgfsetdash{}{0pt}%
\pgfpathmoveto{\pgfqpoint{2.629656in}{1.075804in}}%
\pgfpathlineto{\pgfqpoint{3.401164in}{1.890602in}}%
\pgfusepath{stroke}%
\end{pgfscope}%
\begin{pgfscope}%
\pgfpathrectangle{\pgfqpoint{0.481978in}{0.331635in}}{\pgfqpoint{4.960000in}{3.696000in}}%
\pgfusepath{clip}%
\pgfsetrectcap%
\pgfsetroundjoin%
\pgfsetlinewidth{1.505625pt}%
\definecolor{currentstroke}{rgb}{0.631373,0.788235,0.956863}%
\pgfsetstrokecolor{currentstroke}%
\pgfsetstrokeopacity{0.200000}%
\pgfsetdash{}{0pt}%
\pgfpathmoveto{\pgfqpoint{4.210235in}{2.131089in}}%
\pgfpathlineto{\pgfqpoint{3.401164in}{1.890602in}}%
\pgfusepath{stroke}%
\end{pgfscope}%
\begin{pgfscope}%
\pgfpathrectangle{\pgfqpoint{0.481978in}{0.331635in}}{\pgfqpoint{4.960000in}{3.696000in}}%
\pgfusepath{clip}%
\pgfsetrectcap%
\pgfsetroundjoin%
\pgfsetlinewidth{1.505625pt}%
\definecolor{currentstroke}{rgb}{0.631373,0.788235,0.956863}%
\pgfsetstrokecolor{currentstroke}%
\pgfsetstrokeopacity{0.200000}%
\pgfsetdash{}{0pt}%
\pgfpathmoveto{\pgfqpoint{4.728662in}{2.004918in}}%
\pgfpathlineto{\pgfqpoint{3.401164in}{1.890602in}}%
\pgfusepath{stroke}%
\end{pgfscope}%
\begin{pgfscope}%
\pgfpathrectangle{\pgfqpoint{0.481978in}{0.331635in}}{\pgfqpoint{4.960000in}{3.696000in}}%
\pgfusepath{clip}%
\pgfsetrectcap%
\pgfsetroundjoin%
\pgfsetlinewidth{1.505625pt}%
\definecolor{currentstroke}{rgb}{0.631373,0.788235,0.956863}%
\pgfsetstrokecolor{currentstroke}%
\pgfsetstrokeopacity{0.200000}%
\pgfsetdash{}{0pt}%
\pgfpathmoveto{\pgfqpoint{4.043870in}{2.060016in}}%
\pgfpathlineto{\pgfqpoint{3.401164in}{1.890602in}}%
\pgfusepath{stroke}%
\end{pgfscope}%
\begin{pgfscope}%
\pgfpathrectangle{\pgfqpoint{0.481978in}{0.331635in}}{\pgfqpoint{4.960000in}{3.696000in}}%
\pgfusepath{clip}%
\pgfsetrectcap%
\pgfsetroundjoin%
\pgfsetlinewidth{1.505625pt}%
\definecolor{currentstroke}{rgb}{0.631373,0.788235,0.956863}%
\pgfsetstrokecolor{currentstroke}%
\pgfsetstrokeopacity{0.200000}%
\pgfsetdash{}{0pt}%
\pgfpathmoveto{\pgfqpoint{3.893572in}{1.901153in}}%
\pgfpathlineto{\pgfqpoint{3.401164in}{1.890602in}}%
\pgfusepath{stroke}%
\end{pgfscope}%
\begin{pgfscope}%
\pgfpathrectangle{\pgfqpoint{0.481978in}{0.331635in}}{\pgfqpoint{4.960000in}{3.696000in}}%
\pgfusepath{clip}%
\pgfsetrectcap%
\pgfsetroundjoin%
\pgfsetlinewidth{1.505625pt}%
\definecolor{currentstroke}{rgb}{0.631373,0.788235,0.956863}%
\pgfsetstrokecolor{currentstroke}%
\pgfsetstrokeopacity{0.200000}%
\pgfsetdash{}{0pt}%
\pgfpathmoveto{\pgfqpoint{3.653090in}{0.834174in}}%
\pgfpathlineto{\pgfqpoint{3.401164in}{1.890602in}}%
\pgfusepath{stroke}%
\end{pgfscope}%
\begin{pgfscope}%
\pgfpathrectangle{\pgfqpoint{0.481978in}{0.331635in}}{\pgfqpoint{4.960000in}{3.696000in}}%
\pgfusepath{clip}%
\pgfsetrectcap%
\pgfsetroundjoin%
\pgfsetlinewidth{1.505625pt}%
\definecolor{currentstroke}{rgb}{0.631373,0.788235,0.956863}%
\pgfsetstrokecolor{currentstroke}%
\pgfsetstrokeopacity{0.200000}%
\pgfsetdash{}{0pt}%
\pgfpathmoveto{\pgfqpoint{3.631133in}{1.985028in}}%
\pgfpathlineto{\pgfqpoint{3.401164in}{1.890602in}}%
\pgfusepath{stroke}%
\end{pgfscope}%
\begin{pgfscope}%
\pgfpathrectangle{\pgfqpoint{0.481978in}{0.331635in}}{\pgfqpoint{4.960000in}{3.696000in}}%
\pgfusepath{clip}%
\pgfsetrectcap%
\pgfsetroundjoin%
\pgfsetlinewidth{1.505625pt}%
\definecolor{currentstroke}{rgb}{0.631373,0.788235,0.956863}%
\pgfsetstrokecolor{currentstroke}%
\pgfsetstrokeopacity{0.200000}%
\pgfsetdash{}{0pt}%
\pgfpathmoveto{\pgfqpoint{3.297302in}{2.549026in}}%
\pgfpathlineto{\pgfqpoint{3.401164in}{1.890602in}}%
\pgfusepath{stroke}%
\end{pgfscope}%
\begin{pgfscope}%
\pgfpathrectangle{\pgfqpoint{0.481978in}{0.331635in}}{\pgfqpoint{4.960000in}{3.696000in}}%
\pgfusepath{clip}%
\pgfsetrectcap%
\pgfsetroundjoin%
\pgfsetlinewidth{1.505625pt}%
\definecolor{currentstroke}{rgb}{0.631373,0.788235,0.956863}%
\pgfsetstrokecolor{currentstroke}%
\pgfsetstrokeopacity{0.200000}%
\pgfsetdash{}{0pt}%
\pgfpathmoveto{\pgfqpoint{4.495691in}{2.555723in}}%
\pgfpathlineto{\pgfqpoint{3.401164in}{1.890602in}}%
\pgfusepath{stroke}%
\end{pgfscope}%
\begin{pgfscope}%
\pgfpathrectangle{\pgfqpoint{0.481978in}{0.331635in}}{\pgfqpoint{4.960000in}{3.696000in}}%
\pgfusepath{clip}%
\pgfsetrectcap%
\pgfsetroundjoin%
\pgfsetlinewidth{1.505625pt}%
\definecolor{currentstroke}{rgb}{0.631373,0.788235,0.956863}%
\pgfsetstrokecolor{currentstroke}%
\pgfsetstrokeopacity{0.200000}%
\pgfsetdash{}{0pt}%
\pgfpathmoveto{\pgfqpoint{2.851486in}{3.019296in}}%
\pgfpathlineto{\pgfqpoint{3.401164in}{1.890602in}}%
\pgfusepath{stroke}%
\end{pgfscope}%
\begin{pgfscope}%
\pgfpathrectangle{\pgfqpoint{0.481978in}{0.331635in}}{\pgfqpoint{4.960000in}{3.696000in}}%
\pgfusepath{clip}%
\pgfsetrectcap%
\pgfsetroundjoin%
\pgfsetlinewidth{1.505625pt}%
\definecolor{currentstroke}{rgb}{0.631373,0.788235,0.956863}%
\pgfsetstrokecolor{currentstroke}%
\pgfsetstrokeopacity{0.200000}%
\pgfsetdash{}{0pt}%
\pgfpathmoveto{\pgfqpoint{3.014216in}{1.197012in}}%
\pgfpathlineto{\pgfqpoint{3.401164in}{1.890602in}}%
\pgfusepath{stroke}%
\end{pgfscope}%
\begin{pgfscope}%
\pgfpathrectangle{\pgfqpoint{0.481978in}{0.331635in}}{\pgfqpoint{4.960000in}{3.696000in}}%
\pgfusepath{clip}%
\pgfsetrectcap%
\pgfsetroundjoin%
\pgfsetlinewidth{1.505625pt}%
\definecolor{currentstroke}{rgb}{0.631373,0.788235,0.956863}%
\pgfsetstrokecolor{currentstroke}%
\pgfsetstrokeopacity{0.200000}%
\pgfsetdash{}{0pt}%
\pgfpathmoveto{\pgfqpoint{2.085729in}{2.283765in}}%
\pgfpathlineto{\pgfqpoint{3.401164in}{1.890602in}}%
\pgfusepath{stroke}%
\end{pgfscope}%
\begin{pgfscope}%
\pgfpathrectangle{\pgfqpoint{0.481978in}{0.331635in}}{\pgfqpoint{4.960000in}{3.696000in}}%
\pgfusepath{clip}%
\pgfsetrectcap%
\pgfsetroundjoin%
\pgfsetlinewidth{1.505625pt}%
\definecolor{currentstroke}{rgb}{0.631373,0.788235,0.956863}%
\pgfsetstrokecolor{currentstroke}%
\pgfsetstrokeopacity{0.200000}%
\pgfsetdash{}{0pt}%
\pgfpathmoveto{\pgfqpoint{4.776797in}{2.695720in}}%
\pgfpathlineto{\pgfqpoint{3.401164in}{1.890602in}}%
\pgfusepath{stroke}%
\end{pgfscope}%
\begin{pgfscope}%
\pgfpathrectangle{\pgfqpoint{0.481978in}{0.331635in}}{\pgfqpoint{4.960000in}{3.696000in}}%
\pgfusepath{clip}%
\pgfsetrectcap%
\pgfsetroundjoin%
\pgfsetlinewidth{1.505625pt}%
\definecolor{currentstroke}{rgb}{0.631373,0.788235,0.956863}%
\pgfsetstrokecolor{currentstroke}%
\pgfsetstrokeopacity{0.200000}%
\pgfsetdash{}{0pt}%
\pgfpathmoveto{\pgfqpoint{4.252880in}{2.352308in}}%
\pgfpathlineto{\pgfqpoint{3.401164in}{1.890602in}}%
\pgfusepath{stroke}%
\end{pgfscope}%
\begin{pgfscope}%
\pgfpathrectangle{\pgfqpoint{0.481978in}{0.331635in}}{\pgfqpoint{4.960000in}{3.696000in}}%
\pgfusepath{clip}%
\pgfsetrectcap%
\pgfsetroundjoin%
\pgfsetlinewidth{1.505625pt}%
\definecolor{currentstroke}{rgb}{0.631373,0.788235,0.956863}%
\pgfsetstrokecolor{currentstroke}%
\pgfsetstrokeopacity{0.200000}%
\pgfsetdash{}{0pt}%
\pgfpathmoveto{\pgfqpoint{3.598013in}{2.154393in}}%
\pgfpathlineto{\pgfqpoint{3.401164in}{1.890602in}}%
\pgfusepath{stroke}%
\end{pgfscope}%
\begin{pgfscope}%
\pgfpathrectangle{\pgfqpoint{0.481978in}{0.331635in}}{\pgfqpoint{4.960000in}{3.696000in}}%
\pgfusepath{clip}%
\pgfsetrectcap%
\pgfsetroundjoin%
\pgfsetlinewidth{1.505625pt}%
\definecolor{currentstroke}{rgb}{0.631373,0.788235,0.956863}%
\pgfsetstrokecolor{currentstroke}%
\pgfsetstrokeopacity{0.200000}%
\pgfsetdash{}{0pt}%
\pgfpathmoveto{\pgfqpoint{2.636586in}{1.176048in}}%
\pgfpathlineto{\pgfqpoint{3.401164in}{1.890602in}}%
\pgfusepath{stroke}%
\end{pgfscope}%
\begin{pgfscope}%
\pgfpathrectangle{\pgfqpoint{0.481978in}{0.331635in}}{\pgfqpoint{4.960000in}{3.696000in}}%
\pgfusepath{clip}%
\pgfsetrectcap%
\pgfsetroundjoin%
\pgfsetlinewidth{1.505625pt}%
\definecolor{currentstroke}{rgb}{0.631373,0.788235,0.956863}%
\pgfsetstrokecolor{currentstroke}%
\pgfsetstrokeopacity{0.200000}%
\pgfsetdash{}{0pt}%
\pgfpathmoveto{\pgfqpoint{3.161179in}{1.299286in}}%
\pgfpathlineto{\pgfqpoint{3.401164in}{1.890602in}}%
\pgfusepath{stroke}%
\end{pgfscope}%
\begin{pgfscope}%
\pgfpathrectangle{\pgfqpoint{0.481978in}{0.331635in}}{\pgfqpoint{4.960000in}{3.696000in}}%
\pgfusepath{clip}%
\pgfsetrectcap%
\pgfsetroundjoin%
\pgfsetlinewidth{1.505625pt}%
\definecolor{currentstroke}{rgb}{0.631373,0.788235,0.956863}%
\pgfsetstrokecolor{currentstroke}%
\pgfsetstrokeopacity{0.200000}%
\pgfsetdash{}{0pt}%
\pgfpathmoveto{\pgfqpoint{3.463201in}{2.423292in}}%
\pgfpathlineto{\pgfqpoint{3.401164in}{1.890602in}}%
\pgfusepath{stroke}%
\end{pgfscope}%
\begin{pgfscope}%
\pgfpathrectangle{\pgfqpoint{0.481978in}{0.331635in}}{\pgfqpoint{4.960000in}{3.696000in}}%
\pgfusepath{clip}%
\pgfsetrectcap%
\pgfsetroundjoin%
\pgfsetlinewidth{1.505625pt}%
\definecolor{currentstroke}{rgb}{0.631373,0.788235,0.956863}%
\pgfsetstrokecolor{currentstroke}%
\pgfsetstrokeopacity{0.200000}%
\pgfsetdash{}{0pt}%
\pgfpathmoveto{\pgfqpoint{3.775497in}{2.762642in}}%
\pgfpathlineto{\pgfqpoint{3.401164in}{1.890602in}}%
\pgfusepath{stroke}%
\end{pgfscope}%
\begin{pgfscope}%
\pgfpathrectangle{\pgfqpoint{0.481978in}{0.331635in}}{\pgfqpoint{4.960000in}{3.696000in}}%
\pgfusepath{clip}%
\pgfsetrectcap%
\pgfsetroundjoin%
\pgfsetlinewidth{1.505625pt}%
\definecolor{currentstroke}{rgb}{0.631373,0.788235,0.956863}%
\pgfsetstrokecolor{currentstroke}%
\pgfsetstrokeopacity{0.200000}%
\pgfsetdash{}{0pt}%
\pgfpathmoveto{\pgfqpoint{4.634580in}{2.830864in}}%
\pgfpathlineto{\pgfqpoint{3.401164in}{1.890602in}}%
\pgfusepath{stroke}%
\end{pgfscope}%
\begin{pgfscope}%
\pgfpathrectangle{\pgfqpoint{0.481978in}{0.331635in}}{\pgfqpoint{4.960000in}{3.696000in}}%
\pgfusepath{clip}%
\pgfsetrectcap%
\pgfsetroundjoin%
\pgfsetlinewidth{1.505625pt}%
\definecolor{currentstroke}{rgb}{0.631373,0.788235,0.956863}%
\pgfsetstrokecolor{currentstroke}%
\pgfsetstrokeopacity{0.200000}%
\pgfsetdash{}{0pt}%
\pgfpathmoveto{\pgfqpoint{4.538728in}{2.350653in}}%
\pgfpathlineto{\pgfqpoint{3.401164in}{1.890602in}}%
\pgfusepath{stroke}%
\end{pgfscope}%
\begin{pgfscope}%
\pgfpathrectangle{\pgfqpoint{0.481978in}{0.331635in}}{\pgfqpoint{4.960000in}{3.696000in}}%
\pgfusepath{clip}%
\pgfsetrectcap%
\pgfsetroundjoin%
\pgfsetlinewidth{1.505625pt}%
\definecolor{currentstroke}{rgb}{0.631373,0.788235,0.956863}%
\pgfsetstrokecolor{currentstroke}%
\pgfsetstrokeopacity{0.200000}%
\pgfsetdash{}{0pt}%
\pgfpathmoveto{\pgfqpoint{4.604558in}{2.503561in}}%
\pgfpathlineto{\pgfqpoint{3.401164in}{1.890602in}}%
\pgfusepath{stroke}%
\end{pgfscope}%
\begin{pgfscope}%
\pgfpathrectangle{\pgfqpoint{0.481978in}{0.331635in}}{\pgfqpoint{4.960000in}{3.696000in}}%
\pgfusepath{clip}%
\pgfsetrectcap%
\pgfsetroundjoin%
\pgfsetlinewidth{1.505625pt}%
\definecolor{currentstroke}{rgb}{0.631373,0.788235,0.956863}%
\pgfsetstrokecolor{currentstroke}%
\pgfsetstrokeopacity{0.200000}%
\pgfsetdash{}{0pt}%
\pgfpathmoveto{\pgfqpoint{3.551524in}{2.246992in}}%
\pgfpathlineto{\pgfqpoint{3.401164in}{1.890602in}}%
\pgfusepath{stroke}%
\end{pgfscope}%
\begin{pgfscope}%
\pgfpathrectangle{\pgfqpoint{0.481978in}{0.331635in}}{\pgfqpoint{4.960000in}{3.696000in}}%
\pgfusepath{clip}%
\pgfsetrectcap%
\pgfsetroundjoin%
\pgfsetlinewidth{1.505625pt}%
\definecolor{currentstroke}{rgb}{0.631373,0.788235,0.956863}%
\pgfsetstrokecolor{currentstroke}%
\pgfsetstrokeopacity{0.200000}%
\pgfsetdash{}{0pt}%
\pgfpathmoveto{\pgfqpoint{3.932814in}{2.499059in}}%
\pgfpathlineto{\pgfqpoint{3.401164in}{1.890602in}}%
\pgfusepath{stroke}%
\end{pgfscope}%
\begin{pgfscope}%
\pgfpathrectangle{\pgfqpoint{0.481978in}{0.331635in}}{\pgfqpoint{4.960000in}{3.696000in}}%
\pgfusepath{clip}%
\pgfsetrectcap%
\pgfsetroundjoin%
\pgfsetlinewidth{1.505625pt}%
\definecolor{currentstroke}{rgb}{0.631373,0.788235,0.956863}%
\pgfsetstrokecolor{currentstroke}%
\pgfsetstrokeopacity{0.200000}%
\pgfsetdash{}{0pt}%
\pgfpathmoveto{\pgfqpoint{0.707432in}{1.992795in}}%
\pgfpathlineto{\pgfqpoint{3.401164in}{1.890602in}}%
\pgfusepath{stroke}%
\end{pgfscope}%
\begin{pgfscope}%
\pgfpathrectangle{\pgfqpoint{0.481978in}{0.331635in}}{\pgfqpoint{4.960000in}{3.696000in}}%
\pgfusepath{clip}%
\pgfsetrectcap%
\pgfsetroundjoin%
\pgfsetlinewidth{1.505625pt}%
\definecolor{currentstroke}{rgb}{0.631373,0.788235,0.956863}%
\pgfsetstrokecolor{currentstroke}%
\pgfsetstrokeopacity{0.200000}%
\pgfsetdash{}{0pt}%
\pgfpathmoveto{\pgfqpoint{2.217985in}{1.047209in}}%
\pgfpathlineto{\pgfqpoint{3.401164in}{1.890602in}}%
\pgfusepath{stroke}%
\end{pgfscope}%
\begin{pgfscope}%
\pgfpathrectangle{\pgfqpoint{0.481978in}{0.331635in}}{\pgfqpoint{4.960000in}{3.696000in}}%
\pgfusepath{clip}%
\pgfsetrectcap%
\pgfsetroundjoin%
\pgfsetlinewidth{1.505625pt}%
\definecolor{currentstroke}{rgb}{0.631373,0.788235,0.956863}%
\pgfsetstrokecolor{currentstroke}%
\pgfsetstrokeopacity{0.200000}%
\pgfsetdash{}{0pt}%
\pgfpathmoveto{\pgfqpoint{2.428376in}{1.627701in}}%
\pgfpathlineto{\pgfqpoint{3.401164in}{1.890602in}}%
\pgfusepath{stroke}%
\end{pgfscope}%
\begin{pgfscope}%
\pgfpathrectangle{\pgfqpoint{0.481978in}{0.331635in}}{\pgfqpoint{4.960000in}{3.696000in}}%
\pgfusepath{clip}%
\pgfsetrectcap%
\pgfsetroundjoin%
\pgfsetlinewidth{1.505625pt}%
\definecolor{currentstroke}{rgb}{0.631373,0.788235,0.956863}%
\pgfsetstrokecolor{currentstroke}%
\pgfsetstrokeopacity{0.200000}%
\pgfsetdash{}{0pt}%
\pgfpathmoveto{\pgfqpoint{4.595189in}{3.207591in}}%
\pgfpathlineto{\pgfqpoint{3.401164in}{1.890602in}}%
\pgfusepath{stroke}%
\end{pgfscope}%
\begin{pgfscope}%
\pgfpathrectangle{\pgfqpoint{0.481978in}{0.331635in}}{\pgfqpoint{4.960000in}{3.696000in}}%
\pgfusepath{clip}%
\pgfsetrectcap%
\pgfsetroundjoin%
\pgfsetlinewidth{1.505625pt}%
\definecolor{currentstroke}{rgb}{0.631373,0.788235,0.956863}%
\pgfsetstrokecolor{currentstroke}%
\pgfsetstrokeopacity{0.200000}%
\pgfsetdash{}{0pt}%
\pgfpathmoveto{\pgfqpoint{4.555071in}{2.075740in}}%
\pgfpathlineto{\pgfqpoint{3.401164in}{1.890602in}}%
\pgfusepath{stroke}%
\end{pgfscope}%
\begin{pgfscope}%
\pgfpathrectangle{\pgfqpoint{0.481978in}{0.331635in}}{\pgfqpoint{4.960000in}{3.696000in}}%
\pgfusepath{clip}%
\pgfsetrectcap%
\pgfsetroundjoin%
\pgfsetlinewidth{1.505625pt}%
\definecolor{currentstroke}{rgb}{0.631373,0.788235,0.956863}%
\pgfsetstrokecolor{currentstroke}%
\pgfsetstrokeopacity{0.200000}%
\pgfsetdash{}{0pt}%
\pgfpathmoveto{\pgfqpoint{2.495370in}{2.161606in}}%
\pgfpathlineto{\pgfqpoint{3.401164in}{1.890602in}}%
\pgfusepath{stroke}%
\end{pgfscope}%
\begin{pgfscope}%
\pgfpathrectangle{\pgfqpoint{0.481978in}{0.331635in}}{\pgfqpoint{4.960000in}{3.696000in}}%
\pgfusepath{clip}%
\pgfsetrectcap%
\pgfsetroundjoin%
\pgfsetlinewidth{1.505625pt}%
\definecolor{currentstroke}{rgb}{0.631373,0.788235,0.956863}%
\pgfsetstrokecolor{currentstroke}%
\pgfsetstrokeopacity{0.200000}%
\pgfsetdash{}{0pt}%
\pgfpathmoveto{\pgfqpoint{3.272209in}{1.940893in}}%
\pgfpathlineto{\pgfqpoint{3.401164in}{1.890602in}}%
\pgfusepath{stroke}%
\end{pgfscope}%
\begin{pgfscope}%
\pgfpathrectangle{\pgfqpoint{0.481978in}{0.331635in}}{\pgfqpoint{4.960000in}{3.696000in}}%
\pgfusepath{clip}%
\pgfsetrectcap%
\pgfsetroundjoin%
\pgfsetlinewidth{1.505625pt}%
\definecolor{currentstroke}{rgb}{0.631373,0.788235,0.956863}%
\pgfsetstrokecolor{currentstroke}%
\pgfsetstrokeopacity{0.200000}%
\pgfsetdash{}{0pt}%
\pgfpathmoveto{\pgfqpoint{3.691598in}{1.707511in}}%
\pgfpathlineto{\pgfqpoint{3.401164in}{1.890602in}}%
\pgfusepath{stroke}%
\end{pgfscope}%
\begin{pgfscope}%
\pgfpathrectangle{\pgfqpoint{0.481978in}{0.331635in}}{\pgfqpoint{4.960000in}{3.696000in}}%
\pgfusepath{clip}%
\pgfsetrectcap%
\pgfsetroundjoin%
\pgfsetlinewidth{1.505625pt}%
\definecolor{currentstroke}{rgb}{0.631373,0.788235,0.956863}%
\pgfsetstrokecolor{currentstroke}%
\pgfsetstrokeopacity{0.200000}%
\pgfsetdash{}{0pt}%
\pgfpathmoveto{\pgfqpoint{3.738955in}{1.556533in}}%
\pgfpathlineto{\pgfqpoint{3.401164in}{1.890602in}}%
\pgfusepath{stroke}%
\end{pgfscope}%
\begin{pgfscope}%
\pgfpathrectangle{\pgfqpoint{0.481978in}{0.331635in}}{\pgfqpoint{4.960000in}{3.696000in}}%
\pgfusepath{clip}%
\pgfsetrectcap%
\pgfsetroundjoin%
\pgfsetlinewidth{1.505625pt}%
\definecolor{currentstroke}{rgb}{0.631373,0.788235,0.956863}%
\pgfsetstrokecolor{currentstroke}%
\pgfsetstrokeopacity{0.200000}%
\pgfsetdash{}{0pt}%
\pgfpathmoveto{\pgfqpoint{3.644297in}{3.859635in}}%
\pgfpathlineto{\pgfqpoint{3.401164in}{1.890602in}}%
\pgfusepath{stroke}%
\end{pgfscope}%
\begin{pgfscope}%
\pgfpathrectangle{\pgfqpoint{0.481978in}{0.331635in}}{\pgfqpoint{4.960000in}{3.696000in}}%
\pgfusepath{clip}%
\pgfsetrectcap%
\pgfsetroundjoin%
\pgfsetlinewidth{1.505625pt}%
\definecolor{currentstroke}{rgb}{0.631373,0.788235,0.956863}%
\pgfsetstrokecolor{currentstroke}%
\pgfsetstrokeopacity{0.200000}%
\pgfsetdash{}{0pt}%
\pgfpathmoveto{\pgfqpoint{4.157988in}{1.644874in}}%
\pgfpathlineto{\pgfqpoint{3.401164in}{1.890602in}}%
\pgfusepath{stroke}%
\end{pgfscope}%
\begin{pgfscope}%
\pgfpathrectangle{\pgfqpoint{0.481978in}{0.331635in}}{\pgfqpoint{4.960000in}{3.696000in}}%
\pgfusepath{clip}%
\pgfsetrectcap%
\pgfsetroundjoin%
\pgfsetlinewidth{1.505625pt}%
\definecolor{currentstroke}{rgb}{0.631373,0.788235,0.956863}%
\pgfsetstrokecolor{currentstroke}%
\pgfsetstrokeopacity{0.200000}%
\pgfsetdash{}{0pt}%
\pgfpathmoveto{\pgfqpoint{3.933198in}{1.336032in}}%
\pgfpathlineto{\pgfqpoint{3.401164in}{1.890602in}}%
\pgfusepath{stroke}%
\end{pgfscope}%
\begin{pgfscope}%
\pgfpathrectangle{\pgfqpoint{0.481978in}{0.331635in}}{\pgfqpoint{4.960000in}{3.696000in}}%
\pgfusepath{clip}%
\pgfsetrectcap%
\pgfsetroundjoin%
\pgfsetlinewidth{1.505625pt}%
\definecolor{currentstroke}{rgb}{0.631373,0.788235,0.956863}%
\pgfsetstrokecolor{currentstroke}%
\pgfsetstrokeopacity{0.200000}%
\pgfsetdash{}{0pt}%
\pgfpathmoveto{\pgfqpoint{3.435789in}{1.205521in}}%
\pgfpathlineto{\pgfqpoint{3.401164in}{1.890602in}}%
\pgfusepath{stroke}%
\end{pgfscope}%
\begin{pgfscope}%
\pgfpathrectangle{\pgfqpoint{0.481978in}{0.331635in}}{\pgfqpoint{4.960000in}{3.696000in}}%
\pgfusepath{clip}%
\pgfsetrectcap%
\pgfsetroundjoin%
\pgfsetlinewidth{1.505625pt}%
\definecolor{currentstroke}{rgb}{0.631373,0.788235,0.956863}%
\pgfsetstrokecolor{currentstroke}%
\pgfsetstrokeopacity{0.200000}%
\pgfsetdash{}{0pt}%
\pgfpathmoveto{\pgfqpoint{3.463887in}{2.615265in}}%
\pgfpathlineto{\pgfqpoint{3.401164in}{1.890602in}}%
\pgfusepath{stroke}%
\end{pgfscope}%
\begin{pgfscope}%
\pgfpathrectangle{\pgfqpoint{0.481978in}{0.331635in}}{\pgfqpoint{4.960000in}{3.696000in}}%
\pgfusepath{clip}%
\pgfsetrectcap%
\pgfsetroundjoin%
\pgfsetlinewidth{1.505625pt}%
\definecolor{currentstroke}{rgb}{0.631373,0.788235,0.956863}%
\pgfsetstrokecolor{currentstroke}%
\pgfsetstrokeopacity{0.200000}%
\pgfsetdash{}{0pt}%
\pgfpathmoveto{\pgfqpoint{3.504958in}{2.316448in}}%
\pgfpathlineto{\pgfqpoint{3.401164in}{1.890602in}}%
\pgfusepath{stroke}%
\end{pgfscope}%
\begin{pgfscope}%
\pgfpathrectangle{\pgfqpoint{0.481978in}{0.331635in}}{\pgfqpoint{4.960000in}{3.696000in}}%
\pgfusepath{clip}%
\pgfsetrectcap%
\pgfsetroundjoin%
\pgfsetlinewidth{1.505625pt}%
\definecolor{currentstroke}{rgb}{0.631373,0.788235,0.956863}%
\pgfsetstrokecolor{currentstroke}%
\pgfsetstrokeopacity{0.200000}%
\pgfsetdash{}{0pt}%
\pgfpathmoveto{\pgfqpoint{3.831055in}{1.153959in}}%
\pgfpathlineto{\pgfqpoint{3.401164in}{1.890602in}}%
\pgfusepath{stroke}%
\end{pgfscope}%
\begin{pgfscope}%
\pgfpathrectangle{\pgfqpoint{0.481978in}{0.331635in}}{\pgfqpoint{4.960000in}{3.696000in}}%
\pgfusepath{clip}%
\pgfsetrectcap%
\pgfsetroundjoin%
\pgfsetlinewidth{1.505625pt}%
\definecolor{currentstroke}{rgb}{0.631373,0.788235,0.956863}%
\pgfsetstrokecolor{currentstroke}%
\pgfsetstrokeopacity{0.200000}%
\pgfsetdash{}{0pt}%
\pgfpathmoveto{\pgfqpoint{3.307965in}{1.894343in}}%
\pgfpathlineto{\pgfqpoint{3.401164in}{1.890602in}}%
\pgfusepath{stroke}%
\end{pgfscope}%
\begin{pgfscope}%
\pgfpathrectangle{\pgfqpoint{0.481978in}{0.331635in}}{\pgfqpoint{4.960000in}{3.696000in}}%
\pgfusepath{clip}%
\pgfsetrectcap%
\pgfsetroundjoin%
\pgfsetlinewidth{1.505625pt}%
\definecolor{currentstroke}{rgb}{0.631373,0.788235,0.956863}%
\pgfsetstrokecolor{currentstroke}%
\pgfsetstrokeopacity{0.200000}%
\pgfsetdash{}{0pt}%
\pgfpathmoveto{\pgfqpoint{3.139271in}{1.894478in}}%
\pgfpathlineto{\pgfqpoint{3.401164in}{1.890602in}}%
\pgfusepath{stroke}%
\end{pgfscope}%
\begin{pgfscope}%
\pgfpathrectangle{\pgfqpoint{0.481978in}{0.331635in}}{\pgfqpoint{4.960000in}{3.696000in}}%
\pgfusepath{clip}%
\pgfsetrectcap%
\pgfsetroundjoin%
\pgfsetlinewidth{1.505625pt}%
\definecolor{currentstroke}{rgb}{0.631373,0.788235,0.956863}%
\pgfsetstrokecolor{currentstroke}%
\pgfsetstrokeopacity{0.200000}%
\pgfsetdash{}{0pt}%
\pgfpathmoveto{\pgfqpoint{2.687418in}{1.264424in}}%
\pgfpathlineto{\pgfqpoint{3.401164in}{1.890602in}}%
\pgfusepath{stroke}%
\end{pgfscope}%
\begin{pgfscope}%
\pgfpathrectangle{\pgfqpoint{0.481978in}{0.331635in}}{\pgfqpoint{4.960000in}{3.696000in}}%
\pgfusepath{clip}%
\pgfsetrectcap%
\pgfsetroundjoin%
\pgfsetlinewidth{1.505625pt}%
\definecolor{currentstroke}{rgb}{0.631373,0.788235,0.956863}%
\pgfsetstrokecolor{currentstroke}%
\pgfsetstrokeopacity{0.200000}%
\pgfsetdash{}{0pt}%
\pgfpathmoveto{\pgfqpoint{3.342745in}{2.312406in}}%
\pgfpathlineto{\pgfqpoint{3.401164in}{1.890602in}}%
\pgfusepath{stroke}%
\end{pgfscope}%
\begin{pgfscope}%
\pgfpathrectangle{\pgfqpoint{0.481978in}{0.331635in}}{\pgfqpoint{4.960000in}{3.696000in}}%
\pgfusepath{clip}%
\pgfsetrectcap%
\pgfsetroundjoin%
\pgfsetlinewidth{1.505625pt}%
\definecolor{currentstroke}{rgb}{0.631373,0.788235,0.956863}%
\pgfsetstrokecolor{currentstroke}%
\pgfsetstrokeopacity{0.200000}%
\pgfsetdash{}{0pt}%
\pgfpathmoveto{\pgfqpoint{2.957882in}{0.845042in}}%
\pgfpathlineto{\pgfqpoint{3.401164in}{1.890602in}}%
\pgfusepath{stroke}%
\end{pgfscope}%
\begin{pgfscope}%
\pgfpathrectangle{\pgfqpoint{0.481978in}{0.331635in}}{\pgfqpoint{4.960000in}{3.696000in}}%
\pgfusepath{clip}%
\pgfsetrectcap%
\pgfsetroundjoin%
\pgfsetlinewidth{1.505625pt}%
\definecolor{currentstroke}{rgb}{0.631373,0.788235,0.956863}%
\pgfsetstrokecolor{currentstroke}%
\pgfsetstrokeopacity{0.200000}%
\pgfsetdash{}{0pt}%
\pgfpathmoveto{\pgfqpoint{2.998166in}{1.605557in}}%
\pgfpathlineto{\pgfqpoint{3.401164in}{1.890602in}}%
\pgfusepath{stroke}%
\end{pgfscope}%
\begin{pgfscope}%
\pgfpathrectangle{\pgfqpoint{0.481978in}{0.331635in}}{\pgfqpoint{4.960000in}{3.696000in}}%
\pgfusepath{clip}%
\pgfsetrectcap%
\pgfsetroundjoin%
\pgfsetlinewidth{1.505625pt}%
\definecolor{currentstroke}{rgb}{0.631373,0.788235,0.956863}%
\pgfsetstrokecolor{currentstroke}%
\pgfsetstrokeopacity{0.200000}%
\pgfsetdash{}{0pt}%
\pgfpathmoveto{\pgfqpoint{3.419497in}{1.514212in}}%
\pgfpathlineto{\pgfqpoint{3.401164in}{1.890602in}}%
\pgfusepath{stroke}%
\end{pgfscope}%
\begin{pgfscope}%
\pgfpathrectangle{\pgfqpoint{0.481978in}{0.331635in}}{\pgfqpoint{4.960000in}{3.696000in}}%
\pgfusepath{clip}%
\pgfsetrectcap%
\pgfsetroundjoin%
\pgfsetlinewidth{1.505625pt}%
\definecolor{currentstroke}{rgb}{0.631373,0.788235,0.956863}%
\pgfsetstrokecolor{currentstroke}%
\pgfsetstrokeopacity{0.200000}%
\pgfsetdash{}{0pt}%
\pgfpathmoveto{\pgfqpoint{3.105860in}{0.753327in}}%
\pgfpathlineto{\pgfqpoint{3.401164in}{1.890602in}}%
\pgfusepath{stroke}%
\end{pgfscope}%
\begin{pgfscope}%
\pgfpathrectangle{\pgfqpoint{0.481978in}{0.331635in}}{\pgfqpoint{4.960000in}{3.696000in}}%
\pgfusepath{clip}%
\pgfsetrectcap%
\pgfsetroundjoin%
\pgfsetlinewidth{1.505625pt}%
\definecolor{currentstroke}{rgb}{0.631373,0.788235,0.956863}%
\pgfsetstrokecolor{currentstroke}%
\pgfsetstrokeopacity{0.200000}%
\pgfsetdash{}{0pt}%
\pgfpathmoveto{\pgfqpoint{3.366070in}{2.419394in}}%
\pgfpathlineto{\pgfqpoint{3.401164in}{1.890602in}}%
\pgfusepath{stroke}%
\end{pgfscope}%
\begin{pgfscope}%
\pgfpathrectangle{\pgfqpoint{0.481978in}{0.331635in}}{\pgfqpoint{4.960000in}{3.696000in}}%
\pgfusepath{clip}%
\pgfsetrectcap%
\pgfsetroundjoin%
\pgfsetlinewidth{1.505625pt}%
\definecolor{currentstroke}{rgb}{0.631373,0.788235,0.956863}%
\pgfsetstrokecolor{currentstroke}%
\pgfsetstrokeopacity{0.200000}%
\pgfsetdash{}{0pt}%
\pgfpathmoveto{\pgfqpoint{4.220389in}{2.723908in}}%
\pgfpathlineto{\pgfqpoint{3.401164in}{1.890602in}}%
\pgfusepath{stroke}%
\end{pgfscope}%
\begin{pgfscope}%
\pgfpathrectangle{\pgfqpoint{0.481978in}{0.331635in}}{\pgfqpoint{4.960000in}{3.696000in}}%
\pgfusepath{clip}%
\pgfsetrectcap%
\pgfsetroundjoin%
\pgfsetlinewidth{1.505625pt}%
\definecolor{currentstroke}{rgb}{0.631373,0.788235,0.956863}%
\pgfsetstrokecolor{currentstroke}%
\pgfsetstrokeopacity{0.200000}%
\pgfsetdash{}{0pt}%
\pgfpathmoveto{\pgfqpoint{2.413294in}{0.951901in}}%
\pgfpathlineto{\pgfqpoint{3.401164in}{1.890602in}}%
\pgfusepath{stroke}%
\end{pgfscope}%
\begin{pgfscope}%
\pgfpathrectangle{\pgfqpoint{0.481978in}{0.331635in}}{\pgfqpoint{4.960000in}{3.696000in}}%
\pgfusepath{clip}%
\pgfsetrectcap%
\pgfsetroundjoin%
\pgfsetlinewidth{1.505625pt}%
\definecolor{currentstroke}{rgb}{0.631373,0.788235,0.956863}%
\pgfsetstrokecolor{currentstroke}%
\pgfsetstrokeopacity{0.200000}%
\pgfsetdash{}{0pt}%
\pgfpathmoveto{\pgfqpoint{2.058354in}{0.802885in}}%
\pgfpathlineto{\pgfqpoint{3.401164in}{1.890602in}}%
\pgfusepath{stroke}%
\end{pgfscope}%
\begin{pgfscope}%
\pgfpathrectangle{\pgfqpoint{0.481978in}{0.331635in}}{\pgfqpoint{4.960000in}{3.696000in}}%
\pgfusepath{clip}%
\pgfsetrectcap%
\pgfsetroundjoin%
\pgfsetlinewidth{1.505625pt}%
\definecolor{currentstroke}{rgb}{0.631373,0.788235,0.956863}%
\pgfsetstrokecolor{currentstroke}%
\pgfsetstrokeopacity{0.200000}%
\pgfsetdash{}{0pt}%
\pgfpathmoveto{\pgfqpoint{3.476728in}{2.211577in}}%
\pgfpathlineto{\pgfqpoint{3.401164in}{1.890602in}}%
\pgfusepath{stroke}%
\end{pgfscope}%
\begin{pgfscope}%
\pgfpathrectangle{\pgfqpoint{0.481978in}{0.331635in}}{\pgfqpoint{4.960000in}{3.696000in}}%
\pgfusepath{clip}%
\pgfsetrectcap%
\pgfsetroundjoin%
\pgfsetlinewidth{1.505625pt}%
\definecolor{currentstroke}{rgb}{0.631373,0.788235,0.956863}%
\pgfsetstrokecolor{currentstroke}%
\pgfsetstrokeopacity{0.200000}%
\pgfsetdash{}{0pt}%
\pgfpathmoveto{\pgfqpoint{4.137524in}{2.476976in}}%
\pgfpathlineto{\pgfqpoint{3.401164in}{1.890602in}}%
\pgfusepath{stroke}%
\end{pgfscope}%
\begin{pgfscope}%
\pgfpathrectangle{\pgfqpoint{0.481978in}{0.331635in}}{\pgfqpoint{4.960000in}{3.696000in}}%
\pgfusepath{clip}%
\pgfsetrectcap%
\pgfsetroundjoin%
\pgfsetlinewidth{1.505625pt}%
\definecolor{currentstroke}{rgb}{0.631373,0.788235,0.956863}%
\pgfsetstrokecolor{currentstroke}%
\pgfsetstrokeopacity{0.200000}%
\pgfsetdash{}{0pt}%
\pgfpathmoveto{\pgfqpoint{4.244000in}{2.761618in}}%
\pgfpathlineto{\pgfqpoint{3.401164in}{1.890602in}}%
\pgfusepath{stroke}%
\end{pgfscope}%
\begin{pgfscope}%
\pgfpathrectangle{\pgfqpoint{0.481978in}{0.331635in}}{\pgfqpoint{4.960000in}{3.696000in}}%
\pgfusepath{clip}%
\pgfsetrectcap%
\pgfsetroundjoin%
\pgfsetlinewidth{1.505625pt}%
\definecolor{currentstroke}{rgb}{0.631373,0.788235,0.956863}%
\pgfsetstrokecolor{currentstroke}%
\pgfsetstrokeopacity{0.200000}%
\pgfsetdash{}{0pt}%
\pgfpathmoveto{\pgfqpoint{1.590748in}{1.274000in}}%
\pgfpathlineto{\pgfqpoint{3.401164in}{1.890602in}}%
\pgfusepath{stroke}%
\end{pgfscope}%
\begin{pgfscope}%
\pgfpathrectangle{\pgfqpoint{0.481978in}{0.331635in}}{\pgfqpoint{4.960000in}{3.696000in}}%
\pgfusepath{clip}%
\pgfsetrectcap%
\pgfsetroundjoin%
\pgfsetlinewidth{1.505625pt}%
\definecolor{currentstroke}{rgb}{0.631373,0.788235,0.956863}%
\pgfsetstrokecolor{currentstroke}%
\pgfsetstrokeopacity{0.200000}%
\pgfsetdash{}{0pt}%
\pgfpathmoveto{\pgfqpoint{4.029524in}{1.378576in}}%
\pgfpathlineto{\pgfqpoint{3.401164in}{1.890602in}}%
\pgfusepath{stroke}%
\end{pgfscope}%
\begin{pgfscope}%
\pgfpathrectangle{\pgfqpoint{0.481978in}{0.331635in}}{\pgfqpoint{4.960000in}{3.696000in}}%
\pgfusepath{clip}%
\pgfsetrectcap%
\pgfsetroundjoin%
\pgfsetlinewidth{1.505625pt}%
\definecolor{currentstroke}{rgb}{0.631373,0.788235,0.956863}%
\pgfsetstrokecolor{currentstroke}%
\pgfsetstrokeopacity{0.200000}%
\pgfsetdash{}{0pt}%
\pgfpathmoveto{\pgfqpoint{3.280459in}{2.221035in}}%
\pgfpathlineto{\pgfqpoint{3.401164in}{1.890602in}}%
\pgfusepath{stroke}%
\end{pgfscope}%
\begin{pgfscope}%
\pgfsetrectcap%
\pgfsetmiterjoin%
\pgfsetlinewidth{0.803000pt}%
\definecolor{currentstroke}{rgb}{0.000000,0.000000,0.000000}%
\pgfsetstrokecolor{currentstroke}%
\pgfsetdash{}{0pt}%
\pgfpathmoveto{\pgfqpoint{0.481978in}{0.331635in}}%
\pgfpathlineto{\pgfqpoint{0.481978in}{4.027635in}}%
\pgfusepath{stroke}%
\end{pgfscope}%
\begin{pgfscope}%
\pgfsetrectcap%
\pgfsetmiterjoin%
\pgfsetlinewidth{0.803000pt}%
\definecolor{currentstroke}{rgb}{0.000000,0.000000,0.000000}%
\pgfsetstrokecolor{currentstroke}%
\pgfsetdash{}{0pt}%
\pgfpathmoveto{\pgfqpoint{5.441978in}{0.331635in}}%
\pgfpathlineto{\pgfqpoint{5.441978in}{4.027635in}}%
\pgfusepath{stroke}%
\end{pgfscope}%
\begin{pgfscope}%
\pgfsetrectcap%
\pgfsetmiterjoin%
\pgfsetlinewidth{0.803000pt}%
\definecolor{currentstroke}{rgb}{0.000000,0.000000,0.000000}%
\pgfsetstrokecolor{currentstroke}%
\pgfsetdash{}{0pt}%
\pgfpathmoveto{\pgfqpoint{0.481978in}{0.331635in}}%
\pgfpathlineto{\pgfqpoint{5.441978in}{0.331635in}}%
\pgfusepath{stroke}%
\end{pgfscope}%
\begin{pgfscope}%
\pgfsetrectcap%
\pgfsetmiterjoin%
\pgfsetlinewidth{0.803000pt}%
\definecolor{currentstroke}{rgb}{0.000000,0.000000,0.000000}%
\pgfsetstrokecolor{currentstroke}%
\pgfsetdash{}{0pt}%
\pgfpathmoveto{\pgfqpoint{0.481978in}{4.027635in}}%
\pgfpathlineto{\pgfqpoint{5.441978in}{4.027635in}}%
\pgfusepath{stroke}%
\end{pgfscope}%
\begin{pgfscope}%
\definecolor{textcolor}{rgb}{0.000000,0.000000,0.000000}%
\pgfsetstrokecolor{textcolor}%
\pgfsetfillcolor{textcolor}%
\pgftext[x=2.961978in,y=4.110968in,,base]{\color{textcolor}\sffamily\fontsize{12.000000}{14.400000}\selectfont t-SNE for chair images (s2r3dfree\_textured)}%
\end{pgfscope}%
\begin{pgfscope}%
\pgfsetbuttcap%
\pgfsetmiterjoin%
\definecolor{currentfill}{rgb}{1.000000,1.000000,1.000000}%
\pgfsetfillcolor{currentfill}%
\pgfsetfillopacity{0.800000}%
\pgfsetlinewidth{1.003750pt}%
\definecolor{currentstroke}{rgb}{0.800000,0.800000,0.800000}%
\pgfsetstrokecolor{currentstroke}%
\pgfsetstrokeopacity{0.800000}%
\pgfsetdash{}{0pt}%
\pgfpathmoveto{\pgfqpoint{0.579200in}{0.401079in}}%
\pgfpathlineto{\pgfqpoint{2.351091in}{0.401079in}}%
\pgfpathquadraticcurveto{\pgfqpoint{2.378869in}{0.401079in}}{\pgfqpoint{2.378869in}{0.428857in}}%
\pgfpathlineto{\pgfqpoint{2.378869in}{0.826548in}}%
\pgfpathquadraticcurveto{\pgfqpoint{2.378869in}{0.854326in}}{\pgfqpoint{2.351091in}{0.854326in}}%
\pgfpathlineto{\pgfqpoint{0.579200in}{0.854326in}}%
\pgfpathquadraticcurveto{\pgfqpoint{0.551422in}{0.854326in}}{\pgfqpoint{0.551422in}{0.826548in}}%
\pgfpathlineto{\pgfqpoint{0.551422in}{0.428857in}}%
\pgfpathquadraticcurveto{\pgfqpoint{0.551422in}{0.401079in}}{\pgfqpoint{0.579200in}{0.401079in}}%
\pgfpathclose%
\pgfusepath{stroke,fill}%
\end{pgfscope}%
\begin{pgfscope}%
\pgfsetbuttcap%
\pgfsetroundjoin%
\definecolor{currentfill}{rgb}{1.000000,0.705882,0.509804}%
\pgfsetfillcolor{currentfill}%
\pgfsetlinewidth{1.003750pt}%
\definecolor{currentstroke}{rgb}{1.000000,0.705882,0.509804}%
\pgfsetstrokecolor{currentstroke}%
\pgfsetdash{}{0pt}%
\pgfsys@defobject{currentmarker}{\pgfqpoint{-0.041667in}{-0.041667in}}{\pgfqpoint{0.041667in}{0.041667in}}{%
\pgfpathmoveto{\pgfqpoint{0.000000in}{-0.041667in}}%
\pgfpathcurveto{\pgfqpoint{0.011050in}{-0.041667in}}{\pgfqpoint{0.021649in}{-0.037276in}}{\pgfqpoint{0.029463in}{-0.029463in}}%
\pgfpathcurveto{\pgfqpoint{0.037276in}{-0.021649in}}{\pgfqpoint{0.041667in}{-0.011050in}}{\pgfqpoint{0.041667in}{0.000000in}}%
\pgfpathcurveto{\pgfqpoint{0.041667in}{0.011050in}}{\pgfqpoint{0.037276in}{0.021649in}}{\pgfqpoint{0.029463in}{0.029463in}}%
\pgfpathcurveto{\pgfqpoint{0.021649in}{0.037276in}}{\pgfqpoint{0.011050in}{0.041667in}}{\pgfqpoint{0.000000in}{0.041667in}}%
\pgfpathcurveto{\pgfqpoint{-0.011050in}{0.041667in}}{\pgfqpoint{-0.021649in}{0.037276in}}{\pgfqpoint{-0.029463in}{0.029463in}}%
\pgfpathcurveto{\pgfqpoint{-0.037276in}{0.021649in}}{\pgfqpoint{-0.041667in}{0.011050in}}{\pgfqpoint{-0.041667in}{0.000000in}}%
\pgfpathcurveto{\pgfqpoint{-0.041667in}{-0.011050in}}{\pgfqpoint{-0.037276in}{-0.021649in}}{\pgfqpoint{-0.029463in}{-0.029463in}}%
\pgfpathcurveto{\pgfqpoint{-0.021649in}{-0.037276in}}{\pgfqpoint{-0.011050in}{-0.041667in}}{\pgfqpoint{0.000000in}{-0.041667in}}%
\pgfpathclose%
\pgfusepath{stroke,fill}%
}%
\begin{pgfscope}%
\pgfsys@transformshift{0.745867in}{0.729706in}%
\pgfsys@useobject{currentmarker}{}%
\end{pgfscope}%
\end{pgfscope}%
\begin{pgfscope}%
\definecolor{textcolor}{rgb}{0.000000,0.000000,0.000000}%
\pgfsetstrokecolor{textcolor}%
\pgfsetfillcolor{textcolor}%
\pgftext[x=0.995867in,y=0.693247in,left,base]{\color{textcolor}\sffamily\fontsize{10.000000}{12.000000}\selectfont Pix3D}%
\end{pgfscope}%
\begin{pgfscope}%
\pgfsetbuttcap%
\pgfsetroundjoin%
\definecolor{currentfill}{rgb}{0.631373,0.788235,0.956863}%
\pgfsetfillcolor{currentfill}%
\pgfsetlinewidth{1.003750pt}%
\definecolor{currentstroke}{rgb}{0.631373,0.788235,0.956863}%
\pgfsetstrokecolor{currentstroke}%
\pgfsetdash{}{0pt}%
\pgfsys@defobject{currentmarker}{\pgfqpoint{-0.041667in}{-0.041667in}}{\pgfqpoint{0.041667in}{0.041667in}}{%
\pgfpathmoveto{\pgfqpoint{0.000000in}{-0.041667in}}%
\pgfpathcurveto{\pgfqpoint{0.011050in}{-0.041667in}}{\pgfqpoint{0.021649in}{-0.037276in}}{\pgfqpoint{0.029463in}{-0.029463in}}%
\pgfpathcurveto{\pgfqpoint{0.037276in}{-0.021649in}}{\pgfqpoint{0.041667in}{-0.011050in}}{\pgfqpoint{0.041667in}{0.000000in}}%
\pgfpathcurveto{\pgfqpoint{0.041667in}{0.011050in}}{\pgfqpoint{0.037276in}{0.021649in}}{\pgfqpoint{0.029463in}{0.029463in}}%
\pgfpathcurveto{\pgfqpoint{0.021649in}{0.037276in}}{\pgfqpoint{0.011050in}{0.041667in}}{\pgfqpoint{0.000000in}{0.041667in}}%
\pgfpathcurveto{\pgfqpoint{-0.011050in}{0.041667in}}{\pgfqpoint{-0.021649in}{0.037276in}}{\pgfqpoint{-0.029463in}{0.029463in}}%
\pgfpathcurveto{\pgfqpoint{-0.037276in}{0.021649in}}{\pgfqpoint{-0.041667in}{0.011050in}}{\pgfqpoint{-0.041667in}{0.000000in}}%
\pgfpathcurveto{\pgfqpoint{-0.041667in}{-0.011050in}}{\pgfqpoint{-0.037276in}{-0.021649in}}{\pgfqpoint{-0.029463in}{-0.029463in}}%
\pgfpathcurveto{\pgfqpoint{-0.021649in}{-0.037276in}}{\pgfqpoint{-0.011050in}{-0.041667in}}{\pgfqpoint{0.000000in}{-0.041667in}}%
\pgfpathclose%
\pgfusepath{stroke,fill}%
}%
\begin{pgfscope}%
\pgfsys@transformshift{0.745867in}{0.525849in}%
\pgfsys@useobject{currentmarker}{}%
\end{pgfscope}%
\end{pgfscope}%
\begin{pgfscope}%
\definecolor{textcolor}{rgb}{0.000000,0.000000,0.000000}%
\pgfsetstrokecolor{textcolor}%
\pgfsetfillcolor{textcolor}%
\pgftext[x=0.995867in,y=0.489390in,left,base]{\color{textcolor}\sffamily\fontsize{10.000000}{12.000000}\selectfont s2r3dfree\_textured}%
\end{pgfscope}%
\end{pgfpicture}%
\makeatother%
\endgroup%
}
    \resizebox{0.49\linewidth}{6cm}{%% Creator: Matplotlib, PGF backend
%%
%% To include the figure in your LaTeX document, write
%%   \input{<filename>.pgf}
%%
%% Make sure the required packages are loaded in your preamble
%%   \usepackage{pgf}
%%
%% Figures using additional raster images can only be included by \input if
%% they are in the same directory as the main LaTeX file. For loading figures
%% from other directories you can use the `import` package
%%   \usepackage{import}
%%
%% and then include the figures with
%%   \import{<path to file>}{<filename>.pgf}
%%
%% Matplotlib used the following preamble
%%   \usepackage{fontspec}
%%   \setmainfont{DejaVuSerif.ttf}[Path=\detokenize{/Users/apple/opt/anaconda3/envs/kaolin/lib/python3.7/site-packages/matplotlib/mpl-data/fonts/ttf/}]
%%   \setsansfont{DejaVuSans.ttf}[Path=\detokenize{/Users/apple/opt/anaconda3/envs/kaolin/lib/python3.7/site-packages/matplotlib/mpl-data/fonts/ttf/}]
%%   \setmonofont{DejaVuSansMono.ttf}[Path=\detokenize{/Users/apple/opt/anaconda3/envs/kaolin/lib/python3.7/site-packages/matplotlib/mpl-data/fonts/ttf/}]
%%
\begingroup%
\makeatletter%
\begin{pgfpicture}%
\pgfpathrectangle{\pgfpointorigin}{\pgfqpoint{5.541978in}{4.337596in}}%
\pgfusepath{use as bounding box, clip}%
\begin{pgfscope}%
\pgfsetbuttcap%
\pgfsetmiterjoin%
\definecolor{currentfill}{rgb}{1.000000,1.000000,1.000000}%
\pgfsetfillcolor{currentfill}%
\pgfsetlinewidth{0.000000pt}%
\definecolor{currentstroke}{rgb}{1.000000,1.000000,1.000000}%
\pgfsetstrokecolor{currentstroke}%
\pgfsetdash{}{0pt}%
\pgfpathmoveto{\pgfqpoint{0.000000in}{0.000000in}}%
\pgfpathlineto{\pgfqpoint{5.541978in}{0.000000in}}%
\pgfpathlineto{\pgfqpoint{5.541978in}{4.337596in}}%
\pgfpathlineto{\pgfqpoint{0.000000in}{4.337596in}}%
\pgfpathclose%
\pgfusepath{fill}%
\end{pgfscope}%
\begin{pgfscope}%
\pgfsetbuttcap%
\pgfsetmiterjoin%
\definecolor{currentfill}{rgb}{1.000000,1.000000,1.000000}%
\pgfsetfillcolor{currentfill}%
\pgfsetlinewidth{0.000000pt}%
\definecolor{currentstroke}{rgb}{0.000000,0.000000,0.000000}%
\pgfsetstrokecolor{currentstroke}%
\pgfsetstrokeopacity{0.000000}%
\pgfsetdash{}{0pt}%
\pgfpathmoveto{\pgfqpoint{0.481978in}{0.331635in}}%
\pgfpathlineto{\pgfqpoint{5.441978in}{0.331635in}}%
\pgfpathlineto{\pgfqpoint{5.441978in}{4.027635in}}%
\pgfpathlineto{\pgfqpoint{0.481978in}{4.027635in}}%
\pgfpathclose%
\pgfusepath{fill}%
\end{pgfscope}%
\begin{pgfscope}%
\pgfpathrectangle{\pgfqpoint{0.481978in}{0.331635in}}{\pgfqpoint{4.960000in}{3.696000in}}%
\pgfusepath{clip}%
\pgfsetbuttcap%
\pgfsetroundjoin%
\definecolor{currentfill}{rgb}{1.000000,0.705882,0.509804}%
\pgfsetfillcolor{currentfill}%
\pgfsetlinewidth{0.481800pt}%
\definecolor{currentstroke}{rgb}{1.000000,1.000000,1.000000}%
\pgfsetstrokecolor{currentstroke}%
\pgfsetdash{}{0pt}%
\pgfpathmoveto{\pgfqpoint{0.707432in}{1.070082in}}%
\pgfpathcurveto{\pgfqpoint{0.718483in}{1.070082in}}{\pgfqpoint{0.729082in}{1.074472in}}{\pgfqpoint{0.736895in}{1.082286in}}%
\pgfpathcurveto{\pgfqpoint{0.744709in}{1.090100in}}{\pgfqpoint{0.749099in}{1.100699in}}{\pgfqpoint{0.749099in}{1.111749in}}%
\pgfpathcurveto{\pgfqpoint{0.749099in}{1.122799in}}{\pgfqpoint{0.744709in}{1.133398in}}{\pgfqpoint{0.736895in}{1.141212in}}%
\pgfpathcurveto{\pgfqpoint{0.729082in}{1.149025in}}{\pgfqpoint{0.718483in}{1.153416in}}{\pgfqpoint{0.707432in}{1.153416in}}%
\pgfpathcurveto{\pgfqpoint{0.696382in}{1.153416in}}{\pgfqpoint{0.685783in}{1.149025in}}{\pgfqpoint{0.677970in}{1.141212in}}%
\pgfpathcurveto{\pgfqpoint{0.670156in}{1.133398in}}{\pgfqpoint{0.665766in}{1.122799in}}{\pgfqpoint{0.665766in}{1.111749in}}%
\pgfpathcurveto{\pgfqpoint{0.665766in}{1.100699in}}{\pgfqpoint{0.670156in}{1.090100in}}{\pgfqpoint{0.677970in}{1.082286in}}%
\pgfpathcurveto{\pgfqpoint{0.685783in}{1.074472in}}{\pgfqpoint{0.696382in}{1.070082in}}{\pgfqpoint{0.707432in}{1.070082in}}%
\pgfpathclose%
\pgfusepath{stroke,fill}%
\end{pgfscope}%
\begin{pgfscope}%
\pgfpathrectangle{\pgfqpoint{0.481978in}{0.331635in}}{\pgfqpoint{4.960000in}{3.696000in}}%
\pgfusepath{clip}%
\pgfsetbuttcap%
\pgfsetroundjoin%
\definecolor{currentfill}{rgb}{1.000000,0.705882,0.509804}%
\pgfsetfillcolor{currentfill}%
\pgfsetlinewidth{0.481800pt}%
\definecolor{currentstroke}{rgb}{1.000000,1.000000,1.000000}%
\pgfsetstrokecolor{currentstroke}%
\pgfsetdash{}{0pt}%
\pgfpathmoveto{\pgfqpoint{4.098791in}{1.294060in}}%
\pgfpathcurveto{\pgfqpoint{4.109842in}{1.294060in}}{\pgfqpoint{4.120441in}{1.298450in}}{\pgfqpoint{4.128254in}{1.306264in}}%
\pgfpathcurveto{\pgfqpoint{4.136068in}{1.314078in}}{\pgfqpoint{4.140458in}{1.324677in}}{\pgfqpoint{4.140458in}{1.335727in}}%
\pgfpathcurveto{\pgfqpoint{4.140458in}{1.346777in}}{\pgfqpoint{4.136068in}{1.357376in}}{\pgfqpoint{4.128254in}{1.365190in}}%
\pgfpathcurveto{\pgfqpoint{4.120441in}{1.373003in}}{\pgfqpoint{4.109842in}{1.377393in}}{\pgfqpoint{4.098791in}{1.377393in}}%
\pgfpathcurveto{\pgfqpoint{4.087741in}{1.377393in}}{\pgfqpoint{4.077142in}{1.373003in}}{\pgfqpoint{4.069329in}{1.365190in}}%
\pgfpathcurveto{\pgfqpoint{4.061515in}{1.357376in}}{\pgfqpoint{4.057125in}{1.346777in}}{\pgfqpoint{4.057125in}{1.335727in}}%
\pgfpathcurveto{\pgfqpoint{4.057125in}{1.324677in}}{\pgfqpoint{4.061515in}{1.314078in}}{\pgfqpoint{4.069329in}{1.306264in}}%
\pgfpathcurveto{\pgfqpoint{4.077142in}{1.298450in}}{\pgfqpoint{4.087741in}{1.294060in}}{\pgfqpoint{4.098791in}{1.294060in}}%
\pgfpathclose%
\pgfusepath{stroke,fill}%
\end{pgfscope}%
\begin{pgfscope}%
\pgfpathrectangle{\pgfqpoint{0.481978in}{0.331635in}}{\pgfqpoint{4.960000in}{3.696000in}}%
\pgfusepath{clip}%
\pgfsetbuttcap%
\pgfsetroundjoin%
\definecolor{currentfill}{rgb}{1.000000,0.705882,0.509804}%
\pgfsetfillcolor{currentfill}%
\pgfsetlinewidth{0.481800pt}%
\definecolor{currentstroke}{rgb}{1.000000,1.000000,1.000000}%
\pgfsetstrokecolor{currentstroke}%
\pgfsetdash{}{0pt}%
\pgfpathmoveto{\pgfqpoint{4.100778in}{1.299962in}}%
\pgfpathcurveto{\pgfqpoint{4.111828in}{1.299962in}}{\pgfqpoint{4.122427in}{1.304352in}}{\pgfqpoint{4.130240in}{1.312165in}}%
\pgfpathcurveto{\pgfqpoint{4.138054in}{1.319979in}}{\pgfqpoint{4.142444in}{1.330578in}}{\pgfqpoint{4.142444in}{1.341628in}}%
\pgfpathcurveto{\pgfqpoint{4.142444in}{1.352678in}}{\pgfqpoint{4.138054in}{1.363277in}}{\pgfqpoint{4.130240in}{1.371091in}}%
\pgfpathcurveto{\pgfqpoint{4.122427in}{1.378905in}}{\pgfqpoint{4.111828in}{1.383295in}}{\pgfqpoint{4.100778in}{1.383295in}}%
\pgfpathcurveto{\pgfqpoint{4.089727in}{1.383295in}}{\pgfqpoint{4.079128in}{1.378905in}}{\pgfqpoint{4.071315in}{1.371091in}}%
\pgfpathcurveto{\pgfqpoint{4.063501in}{1.363277in}}{\pgfqpoint{4.059111in}{1.352678in}}{\pgfqpoint{4.059111in}{1.341628in}}%
\pgfpathcurveto{\pgfqpoint{4.059111in}{1.330578in}}{\pgfqpoint{4.063501in}{1.319979in}}{\pgfqpoint{4.071315in}{1.312165in}}%
\pgfpathcurveto{\pgfqpoint{4.079128in}{1.304352in}}{\pgfqpoint{4.089727in}{1.299962in}}{\pgfqpoint{4.100778in}{1.299962in}}%
\pgfpathclose%
\pgfusepath{stroke,fill}%
\end{pgfscope}%
\begin{pgfscope}%
\pgfpathrectangle{\pgfqpoint{0.481978in}{0.331635in}}{\pgfqpoint{4.960000in}{3.696000in}}%
\pgfusepath{clip}%
\pgfsetbuttcap%
\pgfsetroundjoin%
\definecolor{currentfill}{rgb}{1.000000,0.705882,0.509804}%
\pgfsetfillcolor{currentfill}%
\pgfsetlinewidth{0.481800pt}%
\definecolor{currentstroke}{rgb}{1.000000,1.000000,1.000000}%
\pgfsetstrokecolor{currentstroke}%
\pgfsetdash{}{0pt}%
\pgfpathmoveto{\pgfqpoint{1.950259in}{2.273421in}}%
\pgfpathcurveto{\pgfqpoint{1.961309in}{2.273421in}}{\pgfqpoint{1.971908in}{2.277811in}}{\pgfqpoint{1.979721in}{2.285625in}}%
\pgfpathcurveto{\pgfqpoint{1.987535in}{2.293439in}}{\pgfqpoint{1.991925in}{2.304038in}}{\pgfqpoint{1.991925in}{2.315088in}}%
\pgfpathcurveto{\pgfqpoint{1.991925in}{2.326138in}}{\pgfqpoint{1.987535in}{2.336737in}}{\pgfqpoint{1.979721in}{2.344551in}}%
\pgfpathcurveto{\pgfqpoint{1.971908in}{2.352364in}}{\pgfqpoint{1.961309in}{2.356754in}}{\pgfqpoint{1.950259in}{2.356754in}}%
\pgfpathcurveto{\pgfqpoint{1.939209in}{2.356754in}}{\pgfqpoint{1.928609in}{2.352364in}}{\pgfqpoint{1.920796in}{2.344551in}}%
\pgfpathcurveto{\pgfqpoint{1.912982in}{2.336737in}}{\pgfqpoint{1.908592in}{2.326138in}}{\pgfqpoint{1.908592in}{2.315088in}}%
\pgfpathcurveto{\pgfqpoint{1.908592in}{2.304038in}}{\pgfqpoint{1.912982in}{2.293439in}}{\pgfqpoint{1.920796in}{2.285625in}}%
\pgfpathcurveto{\pgfqpoint{1.928609in}{2.277811in}}{\pgfqpoint{1.939209in}{2.273421in}}{\pgfqpoint{1.950259in}{2.273421in}}%
\pgfpathclose%
\pgfusepath{stroke,fill}%
\end{pgfscope}%
\begin{pgfscope}%
\pgfpathrectangle{\pgfqpoint{0.481978in}{0.331635in}}{\pgfqpoint{4.960000in}{3.696000in}}%
\pgfusepath{clip}%
\pgfsetbuttcap%
\pgfsetroundjoin%
\definecolor{currentfill}{rgb}{1.000000,0.705882,0.509804}%
\pgfsetfillcolor{currentfill}%
\pgfsetlinewidth{0.481800pt}%
\definecolor{currentstroke}{rgb}{1.000000,1.000000,1.000000}%
\pgfsetstrokecolor{currentstroke}%
\pgfsetdash{}{0pt}%
\pgfpathmoveto{\pgfqpoint{3.271542in}{1.247155in}}%
\pgfpathcurveto{\pgfqpoint{3.282592in}{1.247155in}}{\pgfqpoint{3.293191in}{1.251545in}}{\pgfqpoint{3.301005in}{1.259359in}}%
\pgfpathcurveto{\pgfqpoint{3.308818in}{1.267172in}}{\pgfqpoint{3.313208in}{1.277771in}}{\pgfqpoint{3.313208in}{1.288822in}}%
\pgfpathcurveto{\pgfqpoint{3.313208in}{1.299872in}}{\pgfqpoint{3.308818in}{1.310471in}}{\pgfqpoint{3.301005in}{1.318284in}}%
\pgfpathcurveto{\pgfqpoint{3.293191in}{1.326098in}}{\pgfqpoint{3.282592in}{1.330488in}}{\pgfqpoint{3.271542in}{1.330488in}}%
\pgfpathcurveto{\pgfqpoint{3.260492in}{1.330488in}}{\pgfqpoint{3.249893in}{1.326098in}}{\pgfqpoint{3.242079in}{1.318284in}}%
\pgfpathcurveto{\pgfqpoint{3.234265in}{1.310471in}}{\pgfqpoint{3.229875in}{1.299872in}}{\pgfqpoint{3.229875in}{1.288822in}}%
\pgfpathcurveto{\pgfqpoint{3.229875in}{1.277771in}}{\pgfqpoint{3.234265in}{1.267172in}}{\pgfqpoint{3.242079in}{1.259359in}}%
\pgfpathcurveto{\pgfqpoint{3.249893in}{1.251545in}}{\pgfqpoint{3.260492in}{1.247155in}}{\pgfqpoint{3.271542in}{1.247155in}}%
\pgfpathclose%
\pgfusepath{stroke,fill}%
\end{pgfscope}%
\begin{pgfscope}%
\pgfpathrectangle{\pgfqpoint{0.481978in}{0.331635in}}{\pgfqpoint{4.960000in}{3.696000in}}%
\pgfusepath{clip}%
\pgfsetbuttcap%
\pgfsetroundjoin%
\definecolor{currentfill}{rgb}{1.000000,0.705882,0.509804}%
\pgfsetfillcolor{currentfill}%
\pgfsetlinewidth{0.481800pt}%
\definecolor{currentstroke}{rgb}{1.000000,1.000000,1.000000}%
\pgfsetstrokecolor{currentstroke}%
\pgfsetdash{}{0pt}%
\pgfpathmoveto{\pgfqpoint{3.075117in}{1.588119in}}%
\pgfpathcurveto{\pgfqpoint{3.086167in}{1.588119in}}{\pgfqpoint{3.096767in}{1.592509in}}{\pgfqpoint{3.104580in}{1.600322in}}%
\pgfpathcurveto{\pgfqpoint{3.112394in}{1.608136in}}{\pgfqpoint{3.116784in}{1.618735in}}{\pgfqpoint{3.116784in}{1.629785in}}%
\pgfpathcurveto{\pgfqpoint{3.116784in}{1.640835in}}{\pgfqpoint{3.112394in}{1.651434in}}{\pgfqpoint{3.104580in}{1.659248in}}%
\pgfpathcurveto{\pgfqpoint{3.096767in}{1.667062in}}{\pgfqpoint{3.086167in}{1.671452in}}{\pgfqpoint{3.075117in}{1.671452in}}%
\pgfpathcurveto{\pgfqpoint{3.064067in}{1.671452in}}{\pgfqpoint{3.053468in}{1.667062in}}{\pgfqpoint{3.045655in}{1.659248in}}%
\pgfpathcurveto{\pgfqpoint{3.037841in}{1.651434in}}{\pgfqpoint{3.033451in}{1.640835in}}{\pgfqpoint{3.033451in}{1.629785in}}%
\pgfpathcurveto{\pgfqpoint{3.033451in}{1.618735in}}{\pgfqpoint{3.037841in}{1.608136in}}{\pgfqpoint{3.045655in}{1.600322in}}%
\pgfpathcurveto{\pgfqpoint{3.053468in}{1.592509in}}{\pgfqpoint{3.064067in}{1.588119in}}{\pgfqpoint{3.075117in}{1.588119in}}%
\pgfpathclose%
\pgfusepath{stroke,fill}%
\end{pgfscope}%
\begin{pgfscope}%
\pgfpathrectangle{\pgfqpoint{0.481978in}{0.331635in}}{\pgfqpoint{4.960000in}{3.696000in}}%
\pgfusepath{clip}%
\pgfsetbuttcap%
\pgfsetroundjoin%
\definecolor{currentfill}{rgb}{1.000000,0.705882,0.509804}%
\pgfsetfillcolor{currentfill}%
\pgfsetlinewidth{0.481800pt}%
\definecolor{currentstroke}{rgb}{1.000000,1.000000,1.000000}%
\pgfsetstrokecolor{currentstroke}%
\pgfsetdash{}{0pt}%
\pgfpathmoveto{\pgfqpoint{2.808150in}{1.629284in}}%
\pgfpathcurveto{\pgfqpoint{2.819200in}{1.629284in}}{\pgfqpoint{2.829799in}{1.633674in}}{\pgfqpoint{2.837613in}{1.641488in}}%
\pgfpathcurveto{\pgfqpoint{2.845426in}{1.649301in}}{\pgfqpoint{2.849817in}{1.659900in}}{\pgfqpoint{2.849817in}{1.670951in}}%
\pgfpathcurveto{\pgfqpoint{2.849817in}{1.682001in}}{\pgfqpoint{2.845426in}{1.692600in}}{\pgfqpoint{2.837613in}{1.700413in}}%
\pgfpathcurveto{\pgfqpoint{2.829799in}{1.708227in}}{\pgfqpoint{2.819200in}{1.712617in}}{\pgfqpoint{2.808150in}{1.712617in}}%
\pgfpathcurveto{\pgfqpoint{2.797100in}{1.712617in}}{\pgfqpoint{2.786501in}{1.708227in}}{\pgfqpoint{2.778687in}{1.700413in}}%
\pgfpathcurveto{\pgfqpoint{2.770874in}{1.692600in}}{\pgfqpoint{2.766483in}{1.682001in}}{\pgfqpoint{2.766483in}{1.670951in}}%
\pgfpathcurveto{\pgfqpoint{2.766483in}{1.659900in}}{\pgfqpoint{2.770874in}{1.649301in}}{\pgfqpoint{2.778687in}{1.641488in}}%
\pgfpathcurveto{\pgfqpoint{2.786501in}{1.633674in}}{\pgfqpoint{2.797100in}{1.629284in}}{\pgfqpoint{2.808150in}{1.629284in}}%
\pgfpathclose%
\pgfusepath{stroke,fill}%
\end{pgfscope}%
\begin{pgfscope}%
\pgfpathrectangle{\pgfqpoint{0.481978in}{0.331635in}}{\pgfqpoint{4.960000in}{3.696000in}}%
\pgfusepath{clip}%
\pgfsetbuttcap%
\pgfsetroundjoin%
\definecolor{currentfill}{rgb}{1.000000,0.705882,0.509804}%
\pgfsetfillcolor{currentfill}%
\pgfsetlinewidth{0.481800pt}%
\definecolor{currentstroke}{rgb}{1.000000,1.000000,1.000000}%
\pgfsetstrokecolor{currentstroke}%
\pgfsetdash{}{0pt}%
\pgfpathmoveto{\pgfqpoint{2.960335in}{1.596534in}}%
\pgfpathcurveto{\pgfqpoint{2.971385in}{1.596534in}}{\pgfqpoint{2.981984in}{1.600924in}}{\pgfqpoint{2.989797in}{1.608738in}}%
\pgfpathcurveto{\pgfqpoint{2.997611in}{1.616552in}}{\pgfqpoint{3.002001in}{1.627151in}}{\pgfqpoint{3.002001in}{1.638201in}}%
\pgfpathcurveto{\pgfqpoint{3.002001in}{1.649251in}}{\pgfqpoint{2.997611in}{1.659850in}}{\pgfqpoint{2.989797in}{1.667664in}}%
\pgfpathcurveto{\pgfqpoint{2.981984in}{1.675477in}}{\pgfqpoint{2.971385in}{1.679867in}}{\pgfqpoint{2.960335in}{1.679867in}}%
\pgfpathcurveto{\pgfqpoint{2.949285in}{1.679867in}}{\pgfqpoint{2.938686in}{1.675477in}}{\pgfqpoint{2.930872in}{1.667664in}}%
\pgfpathcurveto{\pgfqpoint{2.923058in}{1.659850in}}{\pgfqpoint{2.918668in}{1.649251in}}{\pgfqpoint{2.918668in}{1.638201in}}%
\pgfpathcurveto{\pgfqpoint{2.918668in}{1.627151in}}{\pgfqpoint{2.923058in}{1.616552in}}{\pgfqpoint{2.930872in}{1.608738in}}%
\pgfpathcurveto{\pgfqpoint{2.938686in}{1.600924in}}{\pgfqpoint{2.949285in}{1.596534in}}{\pgfqpoint{2.960335in}{1.596534in}}%
\pgfpathclose%
\pgfusepath{stroke,fill}%
\end{pgfscope}%
\begin{pgfscope}%
\pgfpathrectangle{\pgfqpoint{0.481978in}{0.331635in}}{\pgfqpoint{4.960000in}{3.696000in}}%
\pgfusepath{clip}%
\pgfsetbuttcap%
\pgfsetroundjoin%
\definecolor{currentfill}{rgb}{1.000000,0.705882,0.509804}%
\pgfsetfillcolor{currentfill}%
\pgfsetlinewidth{0.481800pt}%
\definecolor{currentstroke}{rgb}{1.000000,1.000000,1.000000}%
\pgfsetstrokecolor{currentstroke}%
\pgfsetdash{}{0pt}%
\pgfpathmoveto{\pgfqpoint{4.016960in}{0.768395in}}%
\pgfpathcurveto{\pgfqpoint{4.028011in}{0.768395in}}{\pgfqpoint{4.038610in}{0.772785in}}{\pgfqpoint{4.046423in}{0.780598in}}%
\pgfpathcurveto{\pgfqpoint{4.054237in}{0.788412in}}{\pgfqpoint{4.058627in}{0.799011in}}{\pgfqpoint{4.058627in}{0.810061in}}%
\pgfpathcurveto{\pgfqpoint{4.058627in}{0.821111in}}{\pgfqpoint{4.054237in}{0.831710in}}{\pgfqpoint{4.046423in}{0.839524in}}%
\pgfpathcurveto{\pgfqpoint{4.038610in}{0.847338in}}{\pgfqpoint{4.028011in}{0.851728in}}{\pgfqpoint{4.016960in}{0.851728in}}%
\pgfpathcurveto{\pgfqpoint{4.005910in}{0.851728in}}{\pgfqpoint{3.995311in}{0.847338in}}{\pgfqpoint{3.987498in}{0.839524in}}%
\pgfpathcurveto{\pgfqpoint{3.979684in}{0.831710in}}{\pgfqpoint{3.975294in}{0.821111in}}{\pgfqpoint{3.975294in}{0.810061in}}%
\pgfpathcurveto{\pgfqpoint{3.975294in}{0.799011in}}{\pgfqpoint{3.979684in}{0.788412in}}{\pgfqpoint{3.987498in}{0.780598in}}%
\pgfpathcurveto{\pgfqpoint{3.995311in}{0.772785in}}{\pgfqpoint{4.005910in}{0.768395in}}{\pgfqpoint{4.016960in}{0.768395in}}%
\pgfpathclose%
\pgfusepath{stroke,fill}%
\end{pgfscope}%
\begin{pgfscope}%
\pgfpathrectangle{\pgfqpoint{0.481978in}{0.331635in}}{\pgfqpoint{4.960000in}{3.696000in}}%
\pgfusepath{clip}%
\pgfsetbuttcap%
\pgfsetroundjoin%
\definecolor{currentfill}{rgb}{1.000000,0.705882,0.509804}%
\pgfsetfillcolor{currentfill}%
\pgfsetlinewidth{0.481800pt}%
\definecolor{currentstroke}{rgb}{1.000000,1.000000,1.000000}%
\pgfsetstrokecolor{currentstroke}%
\pgfsetdash{}{0pt}%
\pgfpathmoveto{\pgfqpoint{3.729020in}{1.988940in}}%
\pgfpathcurveto{\pgfqpoint{3.740070in}{1.988940in}}{\pgfqpoint{3.750669in}{1.993330in}}{\pgfqpoint{3.758482in}{2.001144in}}%
\pgfpathcurveto{\pgfqpoint{3.766296in}{2.008957in}}{\pgfqpoint{3.770686in}{2.019556in}}{\pgfqpoint{3.770686in}{2.030607in}}%
\pgfpathcurveto{\pgfqpoint{3.770686in}{2.041657in}}{\pgfqpoint{3.766296in}{2.052256in}}{\pgfqpoint{3.758482in}{2.060069in}}%
\pgfpathcurveto{\pgfqpoint{3.750669in}{2.067883in}}{\pgfqpoint{3.740070in}{2.072273in}}{\pgfqpoint{3.729020in}{2.072273in}}%
\pgfpathcurveto{\pgfqpoint{3.717969in}{2.072273in}}{\pgfqpoint{3.707370in}{2.067883in}}{\pgfqpoint{3.699557in}{2.060069in}}%
\pgfpathcurveto{\pgfqpoint{3.691743in}{2.052256in}}{\pgfqpoint{3.687353in}{2.041657in}}{\pgfqpoint{3.687353in}{2.030607in}}%
\pgfpathcurveto{\pgfqpoint{3.687353in}{2.019556in}}{\pgfqpoint{3.691743in}{2.008957in}}{\pgfqpoint{3.699557in}{2.001144in}}%
\pgfpathcurveto{\pgfqpoint{3.707370in}{1.993330in}}{\pgfqpoint{3.717969in}{1.988940in}}{\pgfqpoint{3.729020in}{1.988940in}}%
\pgfpathclose%
\pgfusepath{stroke,fill}%
\end{pgfscope}%
\begin{pgfscope}%
\pgfpathrectangle{\pgfqpoint{0.481978in}{0.331635in}}{\pgfqpoint{4.960000in}{3.696000in}}%
\pgfusepath{clip}%
\pgfsetbuttcap%
\pgfsetroundjoin%
\definecolor{currentfill}{rgb}{1.000000,0.705882,0.509804}%
\pgfsetfillcolor{currentfill}%
\pgfsetlinewidth{0.481800pt}%
\definecolor{currentstroke}{rgb}{1.000000,1.000000,1.000000}%
\pgfsetstrokecolor{currentstroke}%
\pgfsetdash{}{0pt}%
\pgfpathmoveto{\pgfqpoint{1.438757in}{1.769810in}}%
\pgfpathcurveto{\pgfqpoint{1.449807in}{1.769810in}}{\pgfqpoint{1.460406in}{1.774200in}}{\pgfqpoint{1.468220in}{1.782014in}}%
\pgfpathcurveto{\pgfqpoint{1.476033in}{1.789827in}}{\pgfqpoint{1.480424in}{1.800427in}}{\pgfqpoint{1.480424in}{1.811477in}}%
\pgfpathcurveto{\pgfqpoint{1.480424in}{1.822527in}}{\pgfqpoint{1.476033in}{1.833126in}}{\pgfqpoint{1.468220in}{1.840939in}}%
\pgfpathcurveto{\pgfqpoint{1.460406in}{1.848753in}}{\pgfqpoint{1.449807in}{1.853143in}}{\pgfqpoint{1.438757in}{1.853143in}}%
\pgfpathcurveto{\pgfqpoint{1.427707in}{1.853143in}}{\pgfqpoint{1.417108in}{1.848753in}}{\pgfqpoint{1.409294in}{1.840939in}}%
\pgfpathcurveto{\pgfqpoint{1.401480in}{1.833126in}}{\pgfqpoint{1.397090in}{1.822527in}}{\pgfqpoint{1.397090in}{1.811477in}}%
\pgfpathcurveto{\pgfqpoint{1.397090in}{1.800427in}}{\pgfqpoint{1.401480in}{1.789827in}}{\pgfqpoint{1.409294in}{1.782014in}}%
\pgfpathcurveto{\pgfqpoint{1.417108in}{1.774200in}}{\pgfqpoint{1.427707in}{1.769810in}}{\pgfqpoint{1.438757in}{1.769810in}}%
\pgfpathclose%
\pgfusepath{stroke,fill}%
\end{pgfscope}%
\begin{pgfscope}%
\pgfpathrectangle{\pgfqpoint{0.481978in}{0.331635in}}{\pgfqpoint{4.960000in}{3.696000in}}%
\pgfusepath{clip}%
\pgfsetbuttcap%
\pgfsetroundjoin%
\definecolor{currentfill}{rgb}{1.000000,0.705882,0.509804}%
\pgfsetfillcolor{currentfill}%
\pgfsetlinewidth{0.481800pt}%
\definecolor{currentstroke}{rgb}{1.000000,1.000000,1.000000}%
\pgfsetstrokecolor{currentstroke}%
\pgfsetdash{}{0pt}%
\pgfpathmoveto{\pgfqpoint{1.871850in}{1.044929in}}%
\pgfpathcurveto{\pgfqpoint{1.882900in}{1.044929in}}{\pgfqpoint{1.893499in}{1.049319in}}{\pgfqpoint{1.901313in}{1.057132in}}%
\pgfpathcurveto{\pgfqpoint{1.909127in}{1.064946in}}{\pgfqpoint{1.913517in}{1.075545in}}{\pgfqpoint{1.913517in}{1.086595in}}%
\pgfpathcurveto{\pgfqpoint{1.913517in}{1.097645in}}{\pgfqpoint{1.909127in}{1.108244in}}{\pgfqpoint{1.901313in}{1.116058in}}%
\pgfpathcurveto{\pgfqpoint{1.893499in}{1.123872in}}{\pgfqpoint{1.882900in}{1.128262in}}{\pgfqpoint{1.871850in}{1.128262in}}%
\pgfpathcurveto{\pgfqpoint{1.860800in}{1.128262in}}{\pgfqpoint{1.850201in}{1.123872in}}{\pgfqpoint{1.842387in}{1.116058in}}%
\pgfpathcurveto{\pgfqpoint{1.834574in}{1.108244in}}{\pgfqpoint{1.830184in}{1.097645in}}{\pgfqpoint{1.830184in}{1.086595in}}%
\pgfpathcurveto{\pgfqpoint{1.830184in}{1.075545in}}{\pgfqpoint{1.834574in}{1.064946in}}{\pgfqpoint{1.842387in}{1.057132in}}%
\pgfpathcurveto{\pgfqpoint{1.850201in}{1.049319in}}{\pgfqpoint{1.860800in}{1.044929in}}{\pgfqpoint{1.871850in}{1.044929in}}%
\pgfpathclose%
\pgfusepath{stroke,fill}%
\end{pgfscope}%
\begin{pgfscope}%
\pgfpathrectangle{\pgfqpoint{0.481978in}{0.331635in}}{\pgfqpoint{4.960000in}{3.696000in}}%
\pgfusepath{clip}%
\pgfsetbuttcap%
\pgfsetroundjoin%
\definecolor{currentfill}{rgb}{1.000000,0.705882,0.509804}%
\pgfsetfillcolor{currentfill}%
\pgfsetlinewidth{0.481800pt}%
\definecolor{currentstroke}{rgb}{1.000000,1.000000,1.000000}%
\pgfsetstrokecolor{currentstroke}%
\pgfsetdash{}{0pt}%
\pgfpathmoveto{\pgfqpoint{3.268642in}{0.995432in}}%
\pgfpathcurveto{\pgfqpoint{3.279692in}{0.995432in}}{\pgfqpoint{3.290291in}{0.999823in}}{\pgfqpoint{3.298105in}{1.007636in}}%
\pgfpathcurveto{\pgfqpoint{3.305918in}{1.015450in}}{\pgfqpoint{3.310309in}{1.026049in}}{\pgfqpoint{3.310309in}{1.037099in}}%
\pgfpathcurveto{\pgfqpoint{3.310309in}{1.048149in}}{\pgfqpoint{3.305918in}{1.058748in}}{\pgfqpoint{3.298105in}{1.066562in}}%
\pgfpathcurveto{\pgfqpoint{3.290291in}{1.074376in}}{\pgfqpoint{3.279692in}{1.078766in}}{\pgfqpoint{3.268642in}{1.078766in}}%
\pgfpathcurveto{\pgfqpoint{3.257592in}{1.078766in}}{\pgfqpoint{3.246993in}{1.074376in}}{\pgfqpoint{3.239179in}{1.066562in}}%
\pgfpathcurveto{\pgfqpoint{3.231366in}{1.058748in}}{\pgfqpoint{3.226975in}{1.048149in}}{\pgfqpoint{3.226975in}{1.037099in}}%
\pgfpathcurveto{\pgfqpoint{3.226975in}{1.026049in}}{\pgfqpoint{3.231366in}{1.015450in}}{\pgfqpoint{3.239179in}{1.007636in}}%
\pgfpathcurveto{\pgfqpoint{3.246993in}{0.999823in}}{\pgfqpoint{3.257592in}{0.995432in}}{\pgfqpoint{3.268642in}{0.995432in}}%
\pgfpathclose%
\pgfusepath{stroke,fill}%
\end{pgfscope}%
\begin{pgfscope}%
\pgfpathrectangle{\pgfqpoint{0.481978in}{0.331635in}}{\pgfqpoint{4.960000in}{3.696000in}}%
\pgfusepath{clip}%
\pgfsetbuttcap%
\pgfsetroundjoin%
\definecolor{currentfill}{rgb}{1.000000,0.705882,0.509804}%
\pgfsetfillcolor{currentfill}%
\pgfsetlinewidth{0.481800pt}%
\definecolor{currentstroke}{rgb}{1.000000,1.000000,1.000000}%
\pgfsetstrokecolor{currentstroke}%
\pgfsetdash{}{0pt}%
\pgfpathmoveto{\pgfqpoint{1.604093in}{2.378959in}}%
\pgfpathcurveto{\pgfqpoint{1.615143in}{2.378959in}}{\pgfqpoint{1.625742in}{2.383349in}}{\pgfqpoint{1.633556in}{2.391163in}}%
\pgfpathcurveto{\pgfqpoint{1.641369in}{2.398977in}}{\pgfqpoint{1.645760in}{2.409576in}}{\pgfqpoint{1.645760in}{2.420626in}}%
\pgfpathcurveto{\pgfqpoint{1.645760in}{2.431676in}}{\pgfqpoint{1.641369in}{2.442275in}}{\pgfqpoint{1.633556in}{2.450089in}}%
\pgfpathcurveto{\pgfqpoint{1.625742in}{2.457902in}}{\pgfqpoint{1.615143in}{2.462292in}}{\pgfqpoint{1.604093in}{2.462292in}}%
\pgfpathcurveto{\pgfqpoint{1.593043in}{2.462292in}}{\pgfqpoint{1.582444in}{2.457902in}}{\pgfqpoint{1.574630in}{2.450089in}}%
\pgfpathcurveto{\pgfqpoint{1.566817in}{2.442275in}}{\pgfqpoint{1.562426in}{2.431676in}}{\pgfqpoint{1.562426in}{2.420626in}}%
\pgfpathcurveto{\pgfqpoint{1.562426in}{2.409576in}}{\pgfqpoint{1.566817in}{2.398977in}}{\pgfqpoint{1.574630in}{2.391163in}}%
\pgfpathcurveto{\pgfqpoint{1.582444in}{2.383349in}}{\pgfqpoint{1.593043in}{2.378959in}}{\pgfqpoint{1.604093in}{2.378959in}}%
\pgfpathclose%
\pgfusepath{stroke,fill}%
\end{pgfscope}%
\begin{pgfscope}%
\pgfpathrectangle{\pgfqpoint{0.481978in}{0.331635in}}{\pgfqpoint{4.960000in}{3.696000in}}%
\pgfusepath{clip}%
\pgfsetbuttcap%
\pgfsetroundjoin%
\definecolor{currentfill}{rgb}{1.000000,0.705882,0.509804}%
\pgfsetfillcolor{currentfill}%
\pgfsetlinewidth{0.481800pt}%
\definecolor{currentstroke}{rgb}{1.000000,1.000000,1.000000}%
\pgfsetstrokecolor{currentstroke}%
\pgfsetdash{}{0pt}%
\pgfpathmoveto{\pgfqpoint{2.189596in}{2.361120in}}%
\pgfpathcurveto{\pgfqpoint{2.200646in}{2.361120in}}{\pgfqpoint{2.211245in}{2.365510in}}{\pgfqpoint{2.219059in}{2.373324in}}%
\pgfpathcurveto{\pgfqpoint{2.226873in}{2.381137in}}{\pgfqpoint{2.231263in}{2.391736in}}{\pgfqpoint{2.231263in}{2.402786in}}%
\pgfpathcurveto{\pgfqpoint{2.231263in}{2.413837in}}{\pgfqpoint{2.226873in}{2.424436in}}{\pgfqpoint{2.219059in}{2.432249in}}%
\pgfpathcurveto{\pgfqpoint{2.211245in}{2.440063in}}{\pgfqpoint{2.200646in}{2.444453in}}{\pgfqpoint{2.189596in}{2.444453in}}%
\pgfpathcurveto{\pgfqpoint{2.178546in}{2.444453in}}{\pgfqpoint{2.167947in}{2.440063in}}{\pgfqpoint{2.160133in}{2.432249in}}%
\pgfpathcurveto{\pgfqpoint{2.152320in}{2.424436in}}{\pgfqpoint{2.147930in}{2.413837in}}{\pgfqpoint{2.147930in}{2.402786in}}%
\pgfpathcurveto{\pgfqpoint{2.147930in}{2.391736in}}{\pgfqpoint{2.152320in}{2.381137in}}{\pgfqpoint{2.160133in}{2.373324in}}%
\pgfpathcurveto{\pgfqpoint{2.167947in}{2.365510in}}{\pgfqpoint{2.178546in}{2.361120in}}{\pgfqpoint{2.189596in}{2.361120in}}%
\pgfpathclose%
\pgfusepath{stroke,fill}%
\end{pgfscope}%
\begin{pgfscope}%
\pgfpathrectangle{\pgfqpoint{0.481978in}{0.331635in}}{\pgfqpoint{4.960000in}{3.696000in}}%
\pgfusepath{clip}%
\pgfsetbuttcap%
\pgfsetroundjoin%
\definecolor{currentfill}{rgb}{1.000000,0.705882,0.509804}%
\pgfsetfillcolor{currentfill}%
\pgfsetlinewidth{0.481800pt}%
\definecolor{currentstroke}{rgb}{1.000000,1.000000,1.000000}%
\pgfsetstrokecolor{currentstroke}%
\pgfsetdash{}{0pt}%
\pgfpathmoveto{\pgfqpoint{2.992675in}{0.988910in}}%
\pgfpathcurveto{\pgfqpoint{3.003725in}{0.988910in}}{\pgfqpoint{3.014324in}{0.993300in}}{\pgfqpoint{3.022138in}{1.001114in}}%
\pgfpathcurveto{\pgfqpoint{3.029952in}{1.008927in}}{\pgfqpoint{3.034342in}{1.019526in}}{\pgfqpoint{3.034342in}{1.030576in}}%
\pgfpathcurveto{\pgfqpoint{3.034342in}{1.041627in}}{\pgfqpoint{3.029952in}{1.052226in}}{\pgfqpoint{3.022138in}{1.060039in}}%
\pgfpathcurveto{\pgfqpoint{3.014324in}{1.067853in}}{\pgfqpoint{3.003725in}{1.072243in}}{\pgfqpoint{2.992675in}{1.072243in}}%
\pgfpathcurveto{\pgfqpoint{2.981625in}{1.072243in}}{\pgfqpoint{2.971026in}{1.067853in}}{\pgfqpoint{2.963213in}{1.060039in}}%
\pgfpathcurveto{\pgfqpoint{2.955399in}{1.052226in}}{\pgfqpoint{2.951009in}{1.041627in}}{\pgfqpoint{2.951009in}{1.030576in}}%
\pgfpathcurveto{\pgfqpoint{2.951009in}{1.019526in}}{\pgfqpoint{2.955399in}{1.008927in}}{\pgfqpoint{2.963213in}{1.001114in}}%
\pgfpathcurveto{\pgfqpoint{2.971026in}{0.993300in}}{\pgfqpoint{2.981625in}{0.988910in}}{\pgfqpoint{2.992675in}{0.988910in}}%
\pgfpathclose%
\pgfusepath{stroke,fill}%
\end{pgfscope}%
\begin{pgfscope}%
\pgfpathrectangle{\pgfqpoint{0.481978in}{0.331635in}}{\pgfqpoint{4.960000in}{3.696000in}}%
\pgfusepath{clip}%
\pgfsetbuttcap%
\pgfsetroundjoin%
\definecolor{currentfill}{rgb}{1.000000,0.705882,0.509804}%
\pgfsetfillcolor{currentfill}%
\pgfsetlinewidth{0.481800pt}%
\definecolor{currentstroke}{rgb}{1.000000,1.000000,1.000000}%
\pgfsetstrokecolor{currentstroke}%
\pgfsetdash{}{0pt}%
\pgfpathmoveto{\pgfqpoint{3.959148in}{1.488991in}}%
\pgfpathcurveto{\pgfqpoint{3.970198in}{1.488991in}}{\pgfqpoint{3.980797in}{1.493381in}}{\pgfqpoint{3.988611in}{1.501195in}}%
\pgfpathcurveto{\pgfqpoint{3.996424in}{1.509008in}}{\pgfqpoint{4.000815in}{1.519607in}}{\pgfqpoint{4.000815in}{1.530657in}}%
\pgfpathcurveto{\pgfqpoint{4.000815in}{1.541707in}}{\pgfqpoint{3.996424in}{1.552307in}}{\pgfqpoint{3.988611in}{1.560120in}}%
\pgfpathcurveto{\pgfqpoint{3.980797in}{1.567934in}}{\pgfqpoint{3.970198in}{1.572324in}}{\pgfqpoint{3.959148in}{1.572324in}}%
\pgfpathcurveto{\pgfqpoint{3.948098in}{1.572324in}}{\pgfqpoint{3.937499in}{1.567934in}}{\pgfqpoint{3.929685in}{1.560120in}}%
\pgfpathcurveto{\pgfqpoint{3.921872in}{1.552307in}}{\pgfqpoint{3.917481in}{1.541707in}}{\pgfqpoint{3.917481in}{1.530657in}}%
\pgfpathcurveto{\pgfqpoint{3.917481in}{1.519607in}}{\pgfqpoint{3.921872in}{1.509008in}}{\pgfqpoint{3.929685in}{1.501195in}}%
\pgfpathcurveto{\pgfqpoint{3.937499in}{1.493381in}}{\pgfqpoint{3.948098in}{1.488991in}}{\pgfqpoint{3.959148in}{1.488991in}}%
\pgfpathclose%
\pgfusepath{stroke,fill}%
\end{pgfscope}%
\begin{pgfscope}%
\pgfpathrectangle{\pgfqpoint{0.481978in}{0.331635in}}{\pgfqpoint{4.960000in}{3.696000in}}%
\pgfusepath{clip}%
\pgfsetbuttcap%
\pgfsetroundjoin%
\definecolor{currentfill}{rgb}{1.000000,0.705882,0.509804}%
\pgfsetfillcolor{currentfill}%
\pgfsetlinewidth{0.481800pt}%
\definecolor{currentstroke}{rgb}{1.000000,1.000000,1.000000}%
\pgfsetstrokecolor{currentstroke}%
\pgfsetdash{}{0pt}%
\pgfpathmoveto{\pgfqpoint{0.949934in}{1.385039in}}%
\pgfpathcurveto{\pgfqpoint{0.960984in}{1.385039in}}{\pgfqpoint{0.971583in}{1.389429in}}{\pgfqpoint{0.979396in}{1.397243in}}%
\pgfpathcurveto{\pgfqpoint{0.987210in}{1.405056in}}{\pgfqpoint{0.991600in}{1.415655in}}{\pgfqpoint{0.991600in}{1.426706in}}%
\pgfpathcurveto{\pgfqpoint{0.991600in}{1.437756in}}{\pgfqpoint{0.987210in}{1.448355in}}{\pgfqpoint{0.979396in}{1.456168in}}%
\pgfpathcurveto{\pgfqpoint{0.971583in}{1.463982in}}{\pgfqpoint{0.960984in}{1.468372in}}{\pgfqpoint{0.949934in}{1.468372in}}%
\pgfpathcurveto{\pgfqpoint{0.938884in}{1.468372in}}{\pgfqpoint{0.928284in}{1.463982in}}{\pgfqpoint{0.920471in}{1.456168in}}%
\pgfpathcurveto{\pgfqpoint{0.912657in}{1.448355in}}{\pgfqpoint{0.908267in}{1.437756in}}{\pgfqpoint{0.908267in}{1.426706in}}%
\pgfpathcurveto{\pgfqpoint{0.908267in}{1.415655in}}{\pgfqpoint{0.912657in}{1.405056in}}{\pgfqpoint{0.920471in}{1.397243in}}%
\pgfpathcurveto{\pgfqpoint{0.928284in}{1.389429in}}{\pgfqpoint{0.938884in}{1.385039in}}{\pgfqpoint{0.949934in}{1.385039in}}%
\pgfpathclose%
\pgfusepath{stroke,fill}%
\end{pgfscope}%
\begin{pgfscope}%
\pgfpathrectangle{\pgfqpoint{0.481978in}{0.331635in}}{\pgfqpoint{4.960000in}{3.696000in}}%
\pgfusepath{clip}%
\pgfsetbuttcap%
\pgfsetroundjoin%
\definecolor{currentfill}{rgb}{1.000000,0.705882,0.509804}%
\pgfsetfillcolor{currentfill}%
\pgfsetlinewidth{0.481800pt}%
\definecolor{currentstroke}{rgb}{1.000000,1.000000,1.000000}%
\pgfsetstrokecolor{currentstroke}%
\pgfsetdash{}{0pt}%
\pgfpathmoveto{\pgfqpoint{2.969152in}{1.105826in}}%
\pgfpathcurveto{\pgfqpoint{2.980203in}{1.105826in}}{\pgfqpoint{2.990802in}{1.110217in}}{\pgfqpoint{2.998615in}{1.118030in}}%
\pgfpathcurveto{\pgfqpoint{3.006429in}{1.125844in}}{\pgfqpoint{3.010819in}{1.136443in}}{\pgfqpoint{3.010819in}{1.147493in}}%
\pgfpathcurveto{\pgfqpoint{3.010819in}{1.158543in}}{\pgfqpoint{3.006429in}{1.169142in}}{\pgfqpoint{2.998615in}{1.176956in}}%
\pgfpathcurveto{\pgfqpoint{2.990802in}{1.184770in}}{\pgfqpoint{2.980203in}{1.189160in}}{\pgfqpoint{2.969152in}{1.189160in}}%
\pgfpathcurveto{\pgfqpoint{2.958102in}{1.189160in}}{\pgfqpoint{2.947503in}{1.184770in}}{\pgfqpoint{2.939690in}{1.176956in}}%
\pgfpathcurveto{\pgfqpoint{2.931876in}{1.169142in}}{\pgfqpoint{2.927486in}{1.158543in}}{\pgfqpoint{2.927486in}{1.147493in}}%
\pgfpathcurveto{\pgfqpoint{2.927486in}{1.136443in}}{\pgfqpoint{2.931876in}{1.125844in}}{\pgfqpoint{2.939690in}{1.118030in}}%
\pgfpathcurveto{\pgfqpoint{2.947503in}{1.110217in}}{\pgfqpoint{2.958102in}{1.105826in}}{\pgfqpoint{2.969152in}{1.105826in}}%
\pgfpathclose%
\pgfusepath{stroke,fill}%
\end{pgfscope}%
\begin{pgfscope}%
\pgfpathrectangle{\pgfqpoint{0.481978in}{0.331635in}}{\pgfqpoint{4.960000in}{3.696000in}}%
\pgfusepath{clip}%
\pgfsetbuttcap%
\pgfsetroundjoin%
\definecolor{currentfill}{rgb}{1.000000,0.705882,0.509804}%
\pgfsetfillcolor{currentfill}%
\pgfsetlinewidth{0.481800pt}%
\definecolor{currentstroke}{rgb}{1.000000,1.000000,1.000000}%
\pgfsetstrokecolor{currentstroke}%
\pgfsetdash{}{0pt}%
\pgfpathmoveto{\pgfqpoint{3.965536in}{1.867029in}}%
\pgfpathcurveto{\pgfqpoint{3.976586in}{1.867029in}}{\pgfqpoint{3.987185in}{1.871419in}}{\pgfqpoint{3.994999in}{1.879233in}}%
\pgfpathcurveto{\pgfqpoint{4.002812in}{1.887046in}}{\pgfqpoint{4.007202in}{1.897645in}}{\pgfqpoint{4.007202in}{1.908696in}}%
\pgfpathcurveto{\pgfqpoint{4.007202in}{1.919746in}}{\pgfqpoint{4.002812in}{1.930345in}}{\pgfqpoint{3.994999in}{1.938158in}}%
\pgfpathcurveto{\pgfqpoint{3.987185in}{1.945972in}}{\pgfqpoint{3.976586in}{1.950362in}}{\pgfqpoint{3.965536in}{1.950362in}}%
\pgfpathcurveto{\pgfqpoint{3.954486in}{1.950362in}}{\pgfqpoint{3.943887in}{1.945972in}}{\pgfqpoint{3.936073in}{1.938158in}}%
\pgfpathcurveto{\pgfqpoint{3.928259in}{1.930345in}}{\pgfqpoint{3.923869in}{1.919746in}}{\pgfqpoint{3.923869in}{1.908696in}}%
\pgfpathcurveto{\pgfqpoint{3.923869in}{1.897645in}}{\pgfqpoint{3.928259in}{1.887046in}}{\pgfqpoint{3.936073in}{1.879233in}}%
\pgfpathcurveto{\pgfqpoint{3.943887in}{1.871419in}}{\pgfqpoint{3.954486in}{1.867029in}}{\pgfqpoint{3.965536in}{1.867029in}}%
\pgfpathclose%
\pgfusepath{stroke,fill}%
\end{pgfscope}%
\begin{pgfscope}%
\pgfpathrectangle{\pgfqpoint{0.481978in}{0.331635in}}{\pgfqpoint{4.960000in}{3.696000in}}%
\pgfusepath{clip}%
\pgfsetbuttcap%
\pgfsetroundjoin%
\definecolor{currentfill}{rgb}{1.000000,0.705882,0.509804}%
\pgfsetfillcolor{currentfill}%
\pgfsetlinewidth{0.481800pt}%
\definecolor{currentstroke}{rgb}{1.000000,1.000000,1.000000}%
\pgfsetstrokecolor{currentstroke}%
\pgfsetdash{}{0pt}%
\pgfpathmoveto{\pgfqpoint{2.841011in}{1.152390in}}%
\pgfpathcurveto{\pgfqpoint{2.852061in}{1.152390in}}{\pgfqpoint{2.862660in}{1.156780in}}{\pgfqpoint{2.870473in}{1.164594in}}%
\pgfpathcurveto{\pgfqpoint{2.878287in}{1.172407in}}{\pgfqpoint{2.882677in}{1.183006in}}{\pgfqpoint{2.882677in}{1.194057in}}%
\pgfpathcurveto{\pgfqpoint{2.882677in}{1.205107in}}{\pgfqpoint{2.878287in}{1.215706in}}{\pgfqpoint{2.870473in}{1.223519in}}%
\pgfpathcurveto{\pgfqpoint{2.862660in}{1.231333in}}{\pgfqpoint{2.852061in}{1.235723in}}{\pgfqpoint{2.841011in}{1.235723in}}%
\pgfpathcurveto{\pgfqpoint{2.829960in}{1.235723in}}{\pgfqpoint{2.819361in}{1.231333in}}{\pgfqpoint{2.811548in}{1.223519in}}%
\pgfpathcurveto{\pgfqpoint{2.803734in}{1.215706in}}{\pgfqpoint{2.799344in}{1.205107in}}{\pgfqpoint{2.799344in}{1.194057in}}%
\pgfpathcurveto{\pgfqpoint{2.799344in}{1.183006in}}{\pgfqpoint{2.803734in}{1.172407in}}{\pgfqpoint{2.811548in}{1.164594in}}%
\pgfpathcurveto{\pgfqpoint{2.819361in}{1.156780in}}{\pgfqpoint{2.829960in}{1.152390in}}{\pgfqpoint{2.841011in}{1.152390in}}%
\pgfpathclose%
\pgfusepath{stroke,fill}%
\end{pgfscope}%
\begin{pgfscope}%
\pgfpathrectangle{\pgfqpoint{0.481978in}{0.331635in}}{\pgfqpoint{4.960000in}{3.696000in}}%
\pgfusepath{clip}%
\pgfsetbuttcap%
\pgfsetroundjoin%
\definecolor{currentfill}{rgb}{1.000000,0.705882,0.509804}%
\pgfsetfillcolor{currentfill}%
\pgfsetlinewidth{0.481800pt}%
\definecolor{currentstroke}{rgb}{1.000000,1.000000,1.000000}%
\pgfsetstrokecolor{currentstroke}%
\pgfsetdash{}{0pt}%
\pgfpathmoveto{\pgfqpoint{2.627352in}{1.637186in}}%
\pgfpathcurveto{\pgfqpoint{2.638402in}{1.637186in}}{\pgfqpoint{2.649001in}{1.641577in}}{\pgfqpoint{2.656815in}{1.649390in}}%
\pgfpathcurveto{\pgfqpoint{2.664628in}{1.657204in}}{\pgfqpoint{2.669019in}{1.667803in}}{\pgfqpoint{2.669019in}{1.678853in}}%
\pgfpathcurveto{\pgfqpoint{2.669019in}{1.689903in}}{\pgfqpoint{2.664628in}{1.700502in}}{\pgfqpoint{2.656815in}{1.708316in}}%
\pgfpathcurveto{\pgfqpoint{2.649001in}{1.716129in}}{\pgfqpoint{2.638402in}{1.720520in}}{\pgfqpoint{2.627352in}{1.720520in}}%
\pgfpathcurveto{\pgfqpoint{2.616302in}{1.720520in}}{\pgfqpoint{2.605703in}{1.716129in}}{\pgfqpoint{2.597889in}{1.708316in}}%
\pgfpathcurveto{\pgfqpoint{2.590076in}{1.700502in}}{\pgfqpoint{2.585685in}{1.689903in}}{\pgfqpoint{2.585685in}{1.678853in}}%
\pgfpathcurveto{\pgfqpoint{2.585685in}{1.667803in}}{\pgfqpoint{2.590076in}{1.657204in}}{\pgfqpoint{2.597889in}{1.649390in}}%
\pgfpathcurveto{\pgfqpoint{2.605703in}{1.641577in}}{\pgfqpoint{2.616302in}{1.637186in}}{\pgfqpoint{2.627352in}{1.637186in}}%
\pgfpathclose%
\pgfusepath{stroke,fill}%
\end{pgfscope}%
\begin{pgfscope}%
\pgfpathrectangle{\pgfqpoint{0.481978in}{0.331635in}}{\pgfqpoint{4.960000in}{3.696000in}}%
\pgfusepath{clip}%
\pgfsetbuttcap%
\pgfsetroundjoin%
\definecolor{currentfill}{rgb}{1.000000,0.705882,0.509804}%
\pgfsetfillcolor{currentfill}%
\pgfsetlinewidth{0.481800pt}%
\definecolor{currentstroke}{rgb}{1.000000,1.000000,1.000000}%
\pgfsetstrokecolor{currentstroke}%
\pgfsetdash{}{0pt}%
\pgfpathmoveto{\pgfqpoint{2.924999in}{1.755131in}}%
\pgfpathcurveto{\pgfqpoint{2.936049in}{1.755131in}}{\pgfqpoint{2.946648in}{1.759521in}}{\pgfqpoint{2.954462in}{1.767335in}}%
\pgfpathcurveto{\pgfqpoint{2.962275in}{1.775149in}}{\pgfqpoint{2.966666in}{1.785748in}}{\pgfqpoint{2.966666in}{1.796798in}}%
\pgfpathcurveto{\pgfqpoint{2.966666in}{1.807848in}}{\pgfqpoint{2.962275in}{1.818447in}}{\pgfqpoint{2.954462in}{1.826260in}}%
\pgfpathcurveto{\pgfqpoint{2.946648in}{1.834074in}}{\pgfqpoint{2.936049in}{1.838464in}}{\pgfqpoint{2.924999in}{1.838464in}}%
\pgfpathcurveto{\pgfqpoint{2.913949in}{1.838464in}}{\pgfqpoint{2.903350in}{1.834074in}}{\pgfqpoint{2.895536in}{1.826260in}}%
\pgfpathcurveto{\pgfqpoint{2.887723in}{1.818447in}}{\pgfqpoint{2.883332in}{1.807848in}}{\pgfqpoint{2.883332in}{1.796798in}}%
\pgfpathcurveto{\pgfqpoint{2.883332in}{1.785748in}}{\pgfqpoint{2.887723in}{1.775149in}}{\pgfqpoint{2.895536in}{1.767335in}}%
\pgfpathcurveto{\pgfqpoint{2.903350in}{1.759521in}}{\pgfqpoint{2.913949in}{1.755131in}}{\pgfqpoint{2.924999in}{1.755131in}}%
\pgfpathclose%
\pgfusepath{stroke,fill}%
\end{pgfscope}%
\begin{pgfscope}%
\pgfpathrectangle{\pgfqpoint{0.481978in}{0.331635in}}{\pgfqpoint{4.960000in}{3.696000in}}%
\pgfusepath{clip}%
\pgfsetbuttcap%
\pgfsetroundjoin%
\definecolor{currentfill}{rgb}{1.000000,0.705882,0.509804}%
\pgfsetfillcolor{currentfill}%
\pgfsetlinewidth{0.481800pt}%
\definecolor{currentstroke}{rgb}{1.000000,1.000000,1.000000}%
\pgfsetstrokecolor{currentstroke}%
\pgfsetdash{}{0pt}%
\pgfpathmoveto{\pgfqpoint{3.232405in}{2.193190in}}%
\pgfpathcurveto{\pgfqpoint{3.243455in}{2.193190in}}{\pgfqpoint{3.254054in}{2.197580in}}{\pgfqpoint{3.261867in}{2.205394in}}%
\pgfpathcurveto{\pgfqpoint{3.269681in}{2.213207in}}{\pgfqpoint{3.274071in}{2.223806in}}{\pgfqpoint{3.274071in}{2.234856in}}%
\pgfpathcurveto{\pgfqpoint{3.274071in}{2.245907in}}{\pgfqpoint{3.269681in}{2.256506in}}{\pgfqpoint{3.261867in}{2.264319in}}%
\pgfpathcurveto{\pgfqpoint{3.254054in}{2.272133in}}{\pgfqpoint{3.243455in}{2.276523in}}{\pgfqpoint{3.232405in}{2.276523in}}%
\pgfpathcurveto{\pgfqpoint{3.221354in}{2.276523in}}{\pgfqpoint{3.210755in}{2.272133in}}{\pgfqpoint{3.202942in}{2.264319in}}%
\pgfpathcurveto{\pgfqpoint{3.195128in}{2.256506in}}{\pgfqpoint{3.190738in}{2.245907in}}{\pgfqpoint{3.190738in}{2.234856in}}%
\pgfpathcurveto{\pgfqpoint{3.190738in}{2.223806in}}{\pgfqpoint{3.195128in}{2.213207in}}{\pgfqpoint{3.202942in}{2.205394in}}%
\pgfpathcurveto{\pgfqpoint{3.210755in}{2.197580in}}{\pgfqpoint{3.221354in}{2.193190in}}{\pgfqpoint{3.232405in}{2.193190in}}%
\pgfpathclose%
\pgfusepath{stroke,fill}%
\end{pgfscope}%
\begin{pgfscope}%
\pgfpathrectangle{\pgfqpoint{0.481978in}{0.331635in}}{\pgfqpoint{4.960000in}{3.696000in}}%
\pgfusepath{clip}%
\pgfsetbuttcap%
\pgfsetroundjoin%
\definecolor{currentfill}{rgb}{1.000000,0.705882,0.509804}%
\pgfsetfillcolor{currentfill}%
\pgfsetlinewidth{0.481800pt}%
\definecolor{currentstroke}{rgb}{1.000000,1.000000,1.000000}%
\pgfsetstrokecolor{currentstroke}%
\pgfsetdash{}{0pt}%
\pgfpathmoveto{\pgfqpoint{4.169916in}{1.049137in}}%
\pgfpathcurveto{\pgfqpoint{4.180966in}{1.049137in}}{\pgfqpoint{4.191565in}{1.053527in}}{\pgfqpoint{4.199378in}{1.061341in}}%
\pgfpathcurveto{\pgfqpoint{4.207192in}{1.069154in}}{\pgfqpoint{4.211582in}{1.079753in}}{\pgfqpoint{4.211582in}{1.090804in}}%
\pgfpathcurveto{\pgfqpoint{4.211582in}{1.101854in}}{\pgfqpoint{4.207192in}{1.112453in}}{\pgfqpoint{4.199378in}{1.120266in}}%
\pgfpathcurveto{\pgfqpoint{4.191565in}{1.128080in}}{\pgfqpoint{4.180966in}{1.132470in}}{\pgfqpoint{4.169916in}{1.132470in}}%
\pgfpathcurveto{\pgfqpoint{4.158866in}{1.132470in}}{\pgfqpoint{4.148267in}{1.128080in}}{\pgfqpoint{4.140453in}{1.120266in}}%
\pgfpathcurveto{\pgfqpoint{4.132639in}{1.112453in}}{\pgfqpoint{4.128249in}{1.101854in}}{\pgfqpoint{4.128249in}{1.090804in}}%
\pgfpathcurveto{\pgfqpoint{4.128249in}{1.079753in}}{\pgfqpoint{4.132639in}{1.069154in}}{\pgfqpoint{4.140453in}{1.061341in}}%
\pgfpathcurveto{\pgfqpoint{4.148267in}{1.053527in}}{\pgfqpoint{4.158866in}{1.049137in}}{\pgfqpoint{4.169916in}{1.049137in}}%
\pgfpathclose%
\pgfusepath{stroke,fill}%
\end{pgfscope}%
\begin{pgfscope}%
\pgfpathrectangle{\pgfqpoint{0.481978in}{0.331635in}}{\pgfqpoint{4.960000in}{3.696000in}}%
\pgfusepath{clip}%
\pgfsetbuttcap%
\pgfsetroundjoin%
\definecolor{currentfill}{rgb}{1.000000,0.705882,0.509804}%
\pgfsetfillcolor{currentfill}%
\pgfsetlinewidth{0.481800pt}%
\definecolor{currentstroke}{rgb}{1.000000,1.000000,1.000000}%
\pgfsetstrokecolor{currentstroke}%
\pgfsetdash{}{0pt}%
\pgfpathmoveto{\pgfqpoint{1.353814in}{1.749203in}}%
\pgfpathcurveto{\pgfqpoint{1.364864in}{1.749203in}}{\pgfqpoint{1.375463in}{1.753593in}}{\pgfqpoint{1.383277in}{1.761406in}}%
\pgfpathcurveto{\pgfqpoint{1.391091in}{1.769220in}}{\pgfqpoint{1.395481in}{1.779819in}}{\pgfqpoint{1.395481in}{1.790869in}}%
\pgfpathcurveto{\pgfqpoint{1.395481in}{1.801919in}}{\pgfqpoint{1.391091in}{1.812518in}}{\pgfqpoint{1.383277in}{1.820332in}}%
\pgfpathcurveto{\pgfqpoint{1.375463in}{1.828146in}}{\pgfqpoint{1.364864in}{1.832536in}}{\pgfqpoint{1.353814in}{1.832536in}}%
\pgfpathcurveto{\pgfqpoint{1.342764in}{1.832536in}}{\pgfqpoint{1.332165in}{1.828146in}}{\pgfqpoint{1.324352in}{1.820332in}}%
\pgfpathcurveto{\pgfqpoint{1.316538in}{1.812518in}}{\pgfqpoint{1.312148in}{1.801919in}}{\pgfqpoint{1.312148in}{1.790869in}}%
\pgfpathcurveto{\pgfqpoint{1.312148in}{1.779819in}}{\pgfqpoint{1.316538in}{1.769220in}}{\pgfqpoint{1.324352in}{1.761406in}}%
\pgfpathcurveto{\pgfqpoint{1.332165in}{1.753593in}}{\pgfqpoint{1.342764in}{1.749203in}}{\pgfqpoint{1.353814in}{1.749203in}}%
\pgfpathclose%
\pgfusepath{stroke,fill}%
\end{pgfscope}%
\begin{pgfscope}%
\pgfpathrectangle{\pgfqpoint{0.481978in}{0.331635in}}{\pgfqpoint{4.960000in}{3.696000in}}%
\pgfusepath{clip}%
\pgfsetbuttcap%
\pgfsetroundjoin%
\definecolor{currentfill}{rgb}{1.000000,0.705882,0.509804}%
\pgfsetfillcolor{currentfill}%
\pgfsetlinewidth{0.481800pt}%
\definecolor{currentstroke}{rgb}{1.000000,1.000000,1.000000}%
\pgfsetstrokecolor{currentstroke}%
\pgfsetdash{}{0pt}%
\pgfpathmoveto{\pgfqpoint{2.797827in}{0.707618in}}%
\pgfpathcurveto{\pgfqpoint{2.808877in}{0.707618in}}{\pgfqpoint{2.819476in}{0.712009in}}{\pgfqpoint{2.827290in}{0.719822in}}%
\pgfpathcurveto{\pgfqpoint{2.835104in}{0.727636in}}{\pgfqpoint{2.839494in}{0.738235in}}{\pgfqpoint{2.839494in}{0.749285in}}%
\pgfpathcurveto{\pgfqpoint{2.839494in}{0.760335in}}{\pgfqpoint{2.835104in}{0.770934in}}{\pgfqpoint{2.827290in}{0.778748in}}%
\pgfpathcurveto{\pgfqpoint{2.819476in}{0.786561in}}{\pgfqpoint{2.808877in}{0.790952in}}{\pgfqpoint{2.797827in}{0.790952in}}%
\pgfpathcurveto{\pgfqpoint{2.786777in}{0.790952in}}{\pgfqpoint{2.776178in}{0.786561in}}{\pgfqpoint{2.768364in}{0.778748in}}%
\pgfpathcurveto{\pgfqpoint{2.760551in}{0.770934in}}{\pgfqpoint{2.756161in}{0.760335in}}{\pgfqpoint{2.756161in}{0.749285in}}%
\pgfpathcurveto{\pgfqpoint{2.756161in}{0.738235in}}{\pgfqpoint{2.760551in}{0.727636in}}{\pgfqpoint{2.768364in}{0.719822in}}%
\pgfpathcurveto{\pgfqpoint{2.776178in}{0.712009in}}{\pgfqpoint{2.786777in}{0.707618in}}{\pgfqpoint{2.797827in}{0.707618in}}%
\pgfpathclose%
\pgfusepath{stroke,fill}%
\end{pgfscope}%
\begin{pgfscope}%
\pgfpathrectangle{\pgfqpoint{0.481978in}{0.331635in}}{\pgfqpoint{4.960000in}{3.696000in}}%
\pgfusepath{clip}%
\pgfsetbuttcap%
\pgfsetroundjoin%
\definecolor{currentfill}{rgb}{1.000000,0.705882,0.509804}%
\pgfsetfillcolor{currentfill}%
\pgfsetlinewidth{0.481800pt}%
\definecolor{currentstroke}{rgb}{1.000000,1.000000,1.000000}%
\pgfsetstrokecolor{currentstroke}%
\pgfsetdash{}{0pt}%
\pgfpathmoveto{\pgfqpoint{4.034424in}{0.769609in}}%
\pgfpathcurveto{\pgfqpoint{4.045474in}{0.769609in}}{\pgfqpoint{4.056073in}{0.773999in}}{\pgfqpoint{4.063887in}{0.781812in}}%
\pgfpathcurveto{\pgfqpoint{4.071700in}{0.789626in}}{\pgfqpoint{4.076090in}{0.800225in}}{\pgfqpoint{4.076090in}{0.811275in}}%
\pgfpathcurveto{\pgfqpoint{4.076090in}{0.822325in}}{\pgfqpoint{4.071700in}{0.832924in}}{\pgfqpoint{4.063887in}{0.840738in}}%
\pgfpathcurveto{\pgfqpoint{4.056073in}{0.848552in}}{\pgfqpoint{4.045474in}{0.852942in}}{\pgfqpoint{4.034424in}{0.852942in}}%
\pgfpathcurveto{\pgfqpoint{4.023374in}{0.852942in}}{\pgfqpoint{4.012775in}{0.848552in}}{\pgfqpoint{4.004961in}{0.840738in}}%
\pgfpathcurveto{\pgfqpoint{3.997147in}{0.832924in}}{\pgfqpoint{3.992757in}{0.822325in}}{\pgfqpoint{3.992757in}{0.811275in}}%
\pgfpathcurveto{\pgfqpoint{3.992757in}{0.800225in}}{\pgfqpoint{3.997147in}{0.789626in}}{\pgfqpoint{4.004961in}{0.781812in}}%
\pgfpathcurveto{\pgfqpoint{4.012775in}{0.773999in}}{\pgfqpoint{4.023374in}{0.769609in}}{\pgfqpoint{4.034424in}{0.769609in}}%
\pgfpathclose%
\pgfusepath{stroke,fill}%
\end{pgfscope}%
\begin{pgfscope}%
\pgfpathrectangle{\pgfqpoint{0.481978in}{0.331635in}}{\pgfqpoint{4.960000in}{3.696000in}}%
\pgfusepath{clip}%
\pgfsetbuttcap%
\pgfsetroundjoin%
\definecolor{currentfill}{rgb}{1.000000,0.705882,0.509804}%
\pgfsetfillcolor{currentfill}%
\pgfsetlinewidth{0.481800pt}%
\definecolor{currentstroke}{rgb}{1.000000,1.000000,1.000000}%
\pgfsetstrokecolor{currentstroke}%
\pgfsetdash{}{0pt}%
\pgfpathmoveto{\pgfqpoint{3.284231in}{1.095952in}}%
\pgfpathcurveto{\pgfqpoint{3.295281in}{1.095952in}}{\pgfqpoint{3.305880in}{1.100343in}}{\pgfqpoint{3.313694in}{1.108156in}}%
\pgfpathcurveto{\pgfqpoint{3.321507in}{1.115970in}}{\pgfqpoint{3.325898in}{1.126569in}}{\pgfqpoint{3.325898in}{1.137619in}}%
\pgfpathcurveto{\pgfqpoint{3.325898in}{1.148669in}}{\pgfqpoint{3.321507in}{1.159268in}}{\pgfqpoint{3.313694in}{1.167082in}}%
\pgfpathcurveto{\pgfqpoint{3.305880in}{1.174895in}}{\pgfqpoint{3.295281in}{1.179286in}}{\pgfqpoint{3.284231in}{1.179286in}}%
\pgfpathcurveto{\pgfqpoint{3.273181in}{1.179286in}}{\pgfqpoint{3.262582in}{1.174895in}}{\pgfqpoint{3.254768in}{1.167082in}}%
\pgfpathcurveto{\pgfqpoint{3.246955in}{1.159268in}}{\pgfqpoint{3.242564in}{1.148669in}}{\pgfqpoint{3.242564in}{1.137619in}}%
\pgfpathcurveto{\pgfqpoint{3.242564in}{1.126569in}}{\pgfqpoint{3.246955in}{1.115970in}}{\pgfqpoint{3.254768in}{1.108156in}}%
\pgfpathcurveto{\pgfqpoint{3.262582in}{1.100343in}}{\pgfqpoint{3.273181in}{1.095952in}}{\pgfqpoint{3.284231in}{1.095952in}}%
\pgfpathclose%
\pgfusepath{stroke,fill}%
\end{pgfscope}%
\begin{pgfscope}%
\pgfpathrectangle{\pgfqpoint{0.481978in}{0.331635in}}{\pgfqpoint{4.960000in}{3.696000in}}%
\pgfusepath{clip}%
\pgfsetbuttcap%
\pgfsetroundjoin%
\definecolor{currentfill}{rgb}{1.000000,0.705882,0.509804}%
\pgfsetfillcolor{currentfill}%
\pgfsetlinewidth{0.481800pt}%
\definecolor{currentstroke}{rgb}{1.000000,1.000000,1.000000}%
\pgfsetstrokecolor{currentstroke}%
\pgfsetdash{}{0pt}%
\pgfpathmoveto{\pgfqpoint{1.962719in}{0.904117in}}%
\pgfpathcurveto{\pgfqpoint{1.973769in}{0.904117in}}{\pgfqpoint{1.984368in}{0.908508in}}{\pgfqpoint{1.992181in}{0.916321in}}%
\pgfpathcurveto{\pgfqpoint{1.999995in}{0.924135in}}{\pgfqpoint{2.004385in}{0.934734in}}{\pgfqpoint{2.004385in}{0.945784in}}%
\pgfpathcurveto{\pgfqpoint{2.004385in}{0.956834in}}{\pgfqpoint{1.999995in}{0.967433in}}{\pgfqpoint{1.992181in}{0.975247in}}%
\pgfpathcurveto{\pgfqpoint{1.984368in}{0.983061in}}{\pgfqpoint{1.973769in}{0.987451in}}{\pgfqpoint{1.962719in}{0.987451in}}%
\pgfpathcurveto{\pgfqpoint{1.951669in}{0.987451in}}{\pgfqpoint{1.941070in}{0.983061in}}{\pgfqpoint{1.933256in}{0.975247in}}%
\pgfpathcurveto{\pgfqpoint{1.925442in}{0.967433in}}{\pgfqpoint{1.921052in}{0.956834in}}{\pgfqpoint{1.921052in}{0.945784in}}%
\pgfpathcurveto{\pgfqpoint{1.921052in}{0.934734in}}{\pgfqpoint{1.925442in}{0.924135in}}{\pgfqpoint{1.933256in}{0.916321in}}%
\pgfpathcurveto{\pgfqpoint{1.941070in}{0.908508in}}{\pgfqpoint{1.951669in}{0.904117in}}{\pgfqpoint{1.962719in}{0.904117in}}%
\pgfpathclose%
\pgfusepath{stroke,fill}%
\end{pgfscope}%
\begin{pgfscope}%
\pgfpathrectangle{\pgfqpoint{0.481978in}{0.331635in}}{\pgfqpoint{4.960000in}{3.696000in}}%
\pgfusepath{clip}%
\pgfsetbuttcap%
\pgfsetroundjoin%
\definecolor{currentfill}{rgb}{1.000000,0.705882,0.509804}%
\pgfsetfillcolor{currentfill}%
\pgfsetlinewidth{0.481800pt}%
\definecolor{currentstroke}{rgb}{1.000000,1.000000,1.000000}%
\pgfsetstrokecolor{currentstroke}%
\pgfsetdash{}{0pt}%
\pgfpathmoveto{\pgfqpoint{1.914501in}{1.102227in}}%
\pgfpathcurveto{\pgfqpoint{1.925551in}{1.102227in}}{\pgfqpoint{1.936150in}{1.106617in}}{\pgfqpoint{1.943964in}{1.114431in}}%
\pgfpathcurveto{\pgfqpoint{1.951777in}{1.122244in}}{\pgfqpoint{1.956168in}{1.132843in}}{\pgfqpoint{1.956168in}{1.143893in}}%
\pgfpathcurveto{\pgfqpoint{1.956168in}{1.154943in}}{\pgfqpoint{1.951777in}{1.165542in}}{\pgfqpoint{1.943964in}{1.173356in}}%
\pgfpathcurveto{\pgfqpoint{1.936150in}{1.181170in}}{\pgfqpoint{1.925551in}{1.185560in}}{\pgfqpoint{1.914501in}{1.185560in}}%
\pgfpathcurveto{\pgfqpoint{1.903451in}{1.185560in}}{\pgfqpoint{1.892852in}{1.181170in}}{\pgfqpoint{1.885038in}{1.173356in}}%
\pgfpathcurveto{\pgfqpoint{1.877225in}{1.165542in}}{\pgfqpoint{1.872834in}{1.154943in}}{\pgfqpoint{1.872834in}{1.143893in}}%
\pgfpathcurveto{\pgfqpoint{1.872834in}{1.132843in}}{\pgfqpoint{1.877225in}{1.122244in}}{\pgfqpoint{1.885038in}{1.114431in}}%
\pgfpathcurveto{\pgfqpoint{1.892852in}{1.106617in}}{\pgfqpoint{1.903451in}{1.102227in}}{\pgfqpoint{1.914501in}{1.102227in}}%
\pgfpathclose%
\pgfusepath{stroke,fill}%
\end{pgfscope}%
\begin{pgfscope}%
\pgfpathrectangle{\pgfqpoint{0.481978in}{0.331635in}}{\pgfqpoint{4.960000in}{3.696000in}}%
\pgfusepath{clip}%
\pgfsetbuttcap%
\pgfsetroundjoin%
\definecolor{currentfill}{rgb}{1.000000,0.705882,0.509804}%
\pgfsetfillcolor{currentfill}%
\pgfsetlinewidth{0.481800pt}%
\definecolor{currentstroke}{rgb}{1.000000,1.000000,1.000000}%
\pgfsetstrokecolor{currentstroke}%
\pgfsetdash{}{0pt}%
\pgfpathmoveto{\pgfqpoint{3.620395in}{2.099986in}}%
\pgfpathcurveto{\pgfqpoint{3.631445in}{2.099986in}}{\pgfqpoint{3.642044in}{2.104376in}}{\pgfqpoint{3.649858in}{2.112190in}}%
\pgfpathcurveto{\pgfqpoint{3.657671in}{2.120003in}}{\pgfqpoint{3.662062in}{2.130603in}}{\pgfqpoint{3.662062in}{2.141653in}}%
\pgfpathcurveto{\pgfqpoint{3.662062in}{2.152703in}}{\pgfqpoint{3.657671in}{2.163302in}}{\pgfqpoint{3.649858in}{2.171115in}}%
\pgfpathcurveto{\pgfqpoint{3.642044in}{2.178929in}}{\pgfqpoint{3.631445in}{2.183319in}}{\pgfqpoint{3.620395in}{2.183319in}}%
\pgfpathcurveto{\pgfqpoint{3.609345in}{2.183319in}}{\pgfqpoint{3.598746in}{2.178929in}}{\pgfqpoint{3.590932in}{2.171115in}}%
\pgfpathcurveto{\pgfqpoint{3.583119in}{2.163302in}}{\pgfqpoint{3.578728in}{2.152703in}}{\pgfqpoint{3.578728in}{2.141653in}}%
\pgfpathcurveto{\pgfqpoint{3.578728in}{2.130603in}}{\pgfqpoint{3.583119in}{2.120003in}}{\pgfqpoint{3.590932in}{2.112190in}}%
\pgfpathcurveto{\pgfqpoint{3.598746in}{2.104376in}}{\pgfqpoint{3.609345in}{2.099986in}}{\pgfqpoint{3.620395in}{2.099986in}}%
\pgfpathclose%
\pgfusepath{stroke,fill}%
\end{pgfscope}%
\begin{pgfscope}%
\pgfpathrectangle{\pgfqpoint{0.481978in}{0.331635in}}{\pgfqpoint{4.960000in}{3.696000in}}%
\pgfusepath{clip}%
\pgfsetbuttcap%
\pgfsetroundjoin%
\definecolor{currentfill}{rgb}{1.000000,0.705882,0.509804}%
\pgfsetfillcolor{currentfill}%
\pgfsetlinewidth{0.481800pt}%
\definecolor{currentstroke}{rgb}{1.000000,1.000000,1.000000}%
\pgfsetstrokecolor{currentstroke}%
\pgfsetdash{}{0pt}%
\pgfpathmoveto{\pgfqpoint{4.241494in}{1.238235in}}%
\pgfpathcurveto{\pgfqpoint{4.252544in}{1.238235in}}{\pgfqpoint{4.263143in}{1.242626in}}{\pgfqpoint{4.270957in}{1.250439in}}%
\pgfpathcurveto{\pgfqpoint{4.278770in}{1.258253in}}{\pgfqpoint{4.283161in}{1.268852in}}{\pgfqpoint{4.283161in}{1.279902in}}%
\pgfpathcurveto{\pgfqpoint{4.283161in}{1.290952in}}{\pgfqpoint{4.278770in}{1.301551in}}{\pgfqpoint{4.270957in}{1.309365in}}%
\pgfpathcurveto{\pgfqpoint{4.263143in}{1.317178in}}{\pgfqpoint{4.252544in}{1.321569in}}{\pgfqpoint{4.241494in}{1.321569in}}%
\pgfpathcurveto{\pgfqpoint{4.230444in}{1.321569in}}{\pgfqpoint{4.219845in}{1.317178in}}{\pgfqpoint{4.212031in}{1.309365in}}%
\pgfpathcurveto{\pgfqpoint{4.204218in}{1.301551in}}{\pgfqpoint{4.199827in}{1.290952in}}{\pgfqpoint{4.199827in}{1.279902in}}%
\pgfpathcurveto{\pgfqpoint{4.199827in}{1.268852in}}{\pgfqpoint{4.204218in}{1.258253in}}{\pgfqpoint{4.212031in}{1.250439in}}%
\pgfpathcurveto{\pgfqpoint{4.219845in}{1.242626in}}{\pgfqpoint{4.230444in}{1.238235in}}{\pgfqpoint{4.241494in}{1.238235in}}%
\pgfpathclose%
\pgfusepath{stroke,fill}%
\end{pgfscope}%
\begin{pgfscope}%
\pgfpathrectangle{\pgfqpoint{0.481978in}{0.331635in}}{\pgfqpoint{4.960000in}{3.696000in}}%
\pgfusepath{clip}%
\pgfsetbuttcap%
\pgfsetroundjoin%
\definecolor{currentfill}{rgb}{1.000000,0.705882,0.509804}%
\pgfsetfillcolor{currentfill}%
\pgfsetlinewidth{0.481800pt}%
\definecolor{currentstroke}{rgb}{1.000000,1.000000,1.000000}%
\pgfsetstrokecolor{currentstroke}%
\pgfsetdash{}{0pt}%
\pgfpathmoveto{\pgfqpoint{3.645474in}{1.190805in}}%
\pgfpathcurveto{\pgfqpoint{3.656525in}{1.190805in}}{\pgfqpoint{3.667124in}{1.195195in}}{\pgfqpoint{3.674937in}{1.203008in}}%
\pgfpathcurveto{\pgfqpoint{3.682751in}{1.210822in}}{\pgfqpoint{3.687141in}{1.221421in}}{\pgfqpoint{3.687141in}{1.232471in}}%
\pgfpathcurveto{\pgfqpoint{3.687141in}{1.243521in}}{\pgfqpoint{3.682751in}{1.254120in}}{\pgfqpoint{3.674937in}{1.261934in}}%
\pgfpathcurveto{\pgfqpoint{3.667124in}{1.269748in}}{\pgfqpoint{3.656525in}{1.274138in}}{\pgfqpoint{3.645474in}{1.274138in}}%
\pgfpathcurveto{\pgfqpoint{3.634424in}{1.274138in}}{\pgfqpoint{3.623825in}{1.269748in}}{\pgfqpoint{3.616012in}{1.261934in}}%
\pgfpathcurveto{\pgfqpoint{3.608198in}{1.254120in}}{\pgfqpoint{3.603808in}{1.243521in}}{\pgfqpoint{3.603808in}{1.232471in}}%
\pgfpathcurveto{\pgfqpoint{3.603808in}{1.221421in}}{\pgfqpoint{3.608198in}{1.210822in}}{\pgfqpoint{3.616012in}{1.203008in}}%
\pgfpathcurveto{\pgfqpoint{3.623825in}{1.195195in}}{\pgfqpoint{3.634424in}{1.190805in}}{\pgfqpoint{3.645474in}{1.190805in}}%
\pgfpathclose%
\pgfusepath{stroke,fill}%
\end{pgfscope}%
\begin{pgfscope}%
\pgfpathrectangle{\pgfqpoint{0.481978in}{0.331635in}}{\pgfqpoint{4.960000in}{3.696000in}}%
\pgfusepath{clip}%
\pgfsetbuttcap%
\pgfsetroundjoin%
\definecolor{currentfill}{rgb}{1.000000,0.705882,0.509804}%
\pgfsetfillcolor{currentfill}%
\pgfsetlinewidth{0.481800pt}%
\definecolor{currentstroke}{rgb}{1.000000,1.000000,1.000000}%
\pgfsetstrokecolor{currentstroke}%
\pgfsetdash{}{0pt}%
\pgfpathmoveto{\pgfqpoint{1.922492in}{1.866064in}}%
\pgfpathcurveto{\pgfqpoint{1.933542in}{1.866064in}}{\pgfqpoint{1.944141in}{1.870454in}}{\pgfqpoint{1.951954in}{1.878268in}}%
\pgfpathcurveto{\pgfqpoint{1.959768in}{1.886081in}}{\pgfqpoint{1.964158in}{1.896680in}}{\pgfqpoint{1.964158in}{1.907730in}}%
\pgfpathcurveto{\pgfqpoint{1.964158in}{1.918781in}}{\pgfqpoint{1.959768in}{1.929380in}}{\pgfqpoint{1.951954in}{1.937193in}}%
\pgfpathcurveto{\pgfqpoint{1.944141in}{1.945007in}}{\pgfqpoint{1.933542in}{1.949397in}}{\pgfqpoint{1.922492in}{1.949397in}}%
\pgfpathcurveto{\pgfqpoint{1.911441in}{1.949397in}}{\pgfqpoint{1.900842in}{1.945007in}}{\pgfqpoint{1.893029in}{1.937193in}}%
\pgfpathcurveto{\pgfqpoint{1.885215in}{1.929380in}}{\pgfqpoint{1.880825in}{1.918781in}}{\pgfqpoint{1.880825in}{1.907730in}}%
\pgfpathcurveto{\pgfqpoint{1.880825in}{1.896680in}}{\pgfqpoint{1.885215in}{1.886081in}}{\pgfqpoint{1.893029in}{1.878268in}}%
\pgfpathcurveto{\pgfqpoint{1.900842in}{1.870454in}}{\pgfqpoint{1.911441in}{1.866064in}}{\pgfqpoint{1.922492in}{1.866064in}}%
\pgfpathclose%
\pgfusepath{stroke,fill}%
\end{pgfscope}%
\begin{pgfscope}%
\pgfpathrectangle{\pgfqpoint{0.481978in}{0.331635in}}{\pgfqpoint{4.960000in}{3.696000in}}%
\pgfusepath{clip}%
\pgfsetbuttcap%
\pgfsetroundjoin%
\definecolor{currentfill}{rgb}{1.000000,0.705882,0.509804}%
\pgfsetfillcolor{currentfill}%
\pgfsetlinewidth{0.481800pt}%
\definecolor{currentstroke}{rgb}{1.000000,1.000000,1.000000}%
\pgfsetstrokecolor{currentstroke}%
\pgfsetdash{}{0pt}%
\pgfpathmoveto{\pgfqpoint{3.872683in}{1.985660in}}%
\pgfpathcurveto{\pgfqpoint{3.883733in}{1.985660in}}{\pgfqpoint{3.894332in}{1.990051in}}{\pgfqpoint{3.902145in}{1.997864in}}%
\pgfpathcurveto{\pgfqpoint{3.909959in}{2.005678in}}{\pgfqpoint{3.914349in}{2.016277in}}{\pgfqpoint{3.914349in}{2.027327in}}%
\pgfpathcurveto{\pgfqpoint{3.914349in}{2.038377in}}{\pgfqpoint{3.909959in}{2.048976in}}{\pgfqpoint{3.902145in}{2.056790in}}%
\pgfpathcurveto{\pgfqpoint{3.894332in}{2.064603in}}{\pgfqpoint{3.883733in}{2.068994in}}{\pgfqpoint{3.872683in}{2.068994in}}%
\pgfpathcurveto{\pgfqpoint{3.861632in}{2.068994in}}{\pgfqpoint{3.851033in}{2.064603in}}{\pgfqpoint{3.843220in}{2.056790in}}%
\pgfpathcurveto{\pgfqpoint{3.835406in}{2.048976in}}{\pgfqpoint{3.831016in}{2.038377in}}{\pgfqpoint{3.831016in}{2.027327in}}%
\pgfpathcurveto{\pgfqpoint{3.831016in}{2.016277in}}{\pgfqpoint{3.835406in}{2.005678in}}{\pgfqpoint{3.843220in}{1.997864in}}%
\pgfpathcurveto{\pgfqpoint{3.851033in}{1.990051in}}{\pgfqpoint{3.861632in}{1.985660in}}{\pgfqpoint{3.872683in}{1.985660in}}%
\pgfpathclose%
\pgfusepath{stroke,fill}%
\end{pgfscope}%
\begin{pgfscope}%
\pgfpathrectangle{\pgfqpoint{0.481978in}{0.331635in}}{\pgfqpoint{4.960000in}{3.696000in}}%
\pgfusepath{clip}%
\pgfsetbuttcap%
\pgfsetroundjoin%
\definecolor{currentfill}{rgb}{1.000000,0.705882,0.509804}%
\pgfsetfillcolor{currentfill}%
\pgfsetlinewidth{0.481800pt}%
\definecolor{currentstroke}{rgb}{1.000000,1.000000,1.000000}%
\pgfsetstrokecolor{currentstroke}%
\pgfsetdash{}{0pt}%
\pgfpathmoveto{\pgfqpoint{3.030639in}{0.904506in}}%
\pgfpathcurveto{\pgfqpoint{3.041689in}{0.904506in}}{\pgfqpoint{3.052288in}{0.908896in}}{\pgfqpoint{3.060101in}{0.916710in}}%
\pgfpathcurveto{\pgfqpoint{3.067915in}{0.924523in}}{\pgfqpoint{3.072305in}{0.935122in}}{\pgfqpoint{3.072305in}{0.946172in}}%
\pgfpathcurveto{\pgfqpoint{3.072305in}{0.957222in}}{\pgfqpoint{3.067915in}{0.967822in}}{\pgfqpoint{3.060101in}{0.975635in}}%
\pgfpathcurveto{\pgfqpoint{3.052288in}{0.983449in}}{\pgfqpoint{3.041689in}{0.987839in}}{\pgfqpoint{3.030639in}{0.987839in}}%
\pgfpathcurveto{\pgfqpoint{3.019589in}{0.987839in}}{\pgfqpoint{3.008990in}{0.983449in}}{\pgfqpoint{3.001176in}{0.975635in}}%
\pgfpathcurveto{\pgfqpoint{2.993362in}{0.967822in}}{\pgfqpoint{2.988972in}{0.957222in}}{\pgfqpoint{2.988972in}{0.946172in}}%
\pgfpathcurveto{\pgfqpoint{2.988972in}{0.935122in}}{\pgfqpoint{2.993362in}{0.924523in}}{\pgfqpoint{3.001176in}{0.916710in}}%
\pgfpathcurveto{\pgfqpoint{3.008990in}{0.908896in}}{\pgfqpoint{3.019589in}{0.904506in}}{\pgfqpoint{3.030639in}{0.904506in}}%
\pgfpathclose%
\pgfusepath{stroke,fill}%
\end{pgfscope}%
\begin{pgfscope}%
\pgfpathrectangle{\pgfqpoint{0.481978in}{0.331635in}}{\pgfqpoint{4.960000in}{3.696000in}}%
\pgfusepath{clip}%
\pgfsetbuttcap%
\pgfsetroundjoin%
\definecolor{currentfill}{rgb}{1.000000,0.705882,0.509804}%
\pgfsetfillcolor{currentfill}%
\pgfsetlinewidth{0.481800pt}%
\definecolor{currentstroke}{rgb}{1.000000,1.000000,1.000000}%
\pgfsetstrokecolor{currentstroke}%
\pgfsetdash{}{0pt}%
\pgfpathmoveto{\pgfqpoint{4.470487in}{1.771245in}}%
\pgfpathcurveto{\pgfqpoint{4.481537in}{1.771245in}}{\pgfqpoint{4.492136in}{1.775635in}}{\pgfqpoint{4.499950in}{1.783449in}}%
\pgfpathcurveto{\pgfqpoint{4.507763in}{1.791262in}}{\pgfqpoint{4.512153in}{1.801861in}}{\pgfqpoint{4.512153in}{1.812911in}}%
\pgfpathcurveto{\pgfqpoint{4.512153in}{1.823962in}}{\pgfqpoint{4.507763in}{1.834561in}}{\pgfqpoint{4.499950in}{1.842374in}}%
\pgfpathcurveto{\pgfqpoint{4.492136in}{1.850188in}}{\pgfqpoint{4.481537in}{1.854578in}}{\pgfqpoint{4.470487in}{1.854578in}}%
\pgfpathcurveto{\pgfqpoint{4.459437in}{1.854578in}}{\pgfqpoint{4.448838in}{1.850188in}}{\pgfqpoint{4.441024in}{1.842374in}}%
\pgfpathcurveto{\pgfqpoint{4.433210in}{1.834561in}}{\pgfqpoint{4.428820in}{1.823962in}}{\pgfqpoint{4.428820in}{1.812911in}}%
\pgfpathcurveto{\pgfqpoint{4.428820in}{1.801861in}}{\pgfqpoint{4.433210in}{1.791262in}}{\pgfqpoint{4.441024in}{1.783449in}}%
\pgfpathcurveto{\pgfqpoint{4.448838in}{1.775635in}}{\pgfqpoint{4.459437in}{1.771245in}}{\pgfqpoint{4.470487in}{1.771245in}}%
\pgfpathclose%
\pgfusepath{stroke,fill}%
\end{pgfscope}%
\begin{pgfscope}%
\pgfpathrectangle{\pgfqpoint{0.481978in}{0.331635in}}{\pgfqpoint{4.960000in}{3.696000in}}%
\pgfusepath{clip}%
\pgfsetbuttcap%
\pgfsetroundjoin%
\definecolor{currentfill}{rgb}{1.000000,0.705882,0.509804}%
\pgfsetfillcolor{currentfill}%
\pgfsetlinewidth{0.481800pt}%
\definecolor{currentstroke}{rgb}{1.000000,1.000000,1.000000}%
\pgfsetstrokecolor{currentstroke}%
\pgfsetdash{}{0pt}%
\pgfpathmoveto{\pgfqpoint{2.286558in}{0.732729in}}%
\pgfpathcurveto{\pgfqpoint{2.297608in}{0.732729in}}{\pgfqpoint{2.308207in}{0.737119in}}{\pgfqpoint{2.316021in}{0.744933in}}%
\pgfpathcurveto{\pgfqpoint{2.323834in}{0.752746in}}{\pgfqpoint{2.328225in}{0.763345in}}{\pgfqpoint{2.328225in}{0.774395in}}%
\pgfpathcurveto{\pgfqpoint{2.328225in}{0.785445in}}{\pgfqpoint{2.323834in}{0.796045in}}{\pgfqpoint{2.316021in}{0.803858in}}%
\pgfpathcurveto{\pgfqpoint{2.308207in}{0.811672in}}{\pgfqpoint{2.297608in}{0.816062in}}{\pgfqpoint{2.286558in}{0.816062in}}%
\pgfpathcurveto{\pgfqpoint{2.275508in}{0.816062in}}{\pgfqpoint{2.264909in}{0.811672in}}{\pgfqpoint{2.257095in}{0.803858in}}%
\pgfpathcurveto{\pgfqpoint{2.249282in}{0.796045in}}{\pgfqpoint{2.244891in}{0.785445in}}{\pgfqpoint{2.244891in}{0.774395in}}%
\pgfpathcurveto{\pgfqpoint{2.244891in}{0.763345in}}{\pgfqpoint{2.249282in}{0.752746in}}{\pgfqpoint{2.257095in}{0.744933in}}%
\pgfpathcurveto{\pgfqpoint{2.264909in}{0.737119in}}{\pgfqpoint{2.275508in}{0.732729in}}{\pgfqpoint{2.286558in}{0.732729in}}%
\pgfpathclose%
\pgfusepath{stroke,fill}%
\end{pgfscope}%
\begin{pgfscope}%
\pgfpathrectangle{\pgfqpoint{0.481978in}{0.331635in}}{\pgfqpoint{4.960000in}{3.696000in}}%
\pgfusepath{clip}%
\pgfsetbuttcap%
\pgfsetroundjoin%
\definecolor{currentfill}{rgb}{1.000000,0.705882,0.509804}%
\pgfsetfillcolor{currentfill}%
\pgfsetlinewidth{0.481800pt}%
\definecolor{currentstroke}{rgb}{1.000000,1.000000,1.000000}%
\pgfsetstrokecolor{currentstroke}%
\pgfsetdash{}{0pt}%
\pgfpathmoveto{\pgfqpoint{4.296984in}{2.081305in}}%
\pgfpathcurveto{\pgfqpoint{4.308034in}{2.081305in}}{\pgfqpoint{4.318633in}{2.085696in}}{\pgfqpoint{4.326447in}{2.093509in}}%
\pgfpathcurveto{\pgfqpoint{4.334260in}{2.101323in}}{\pgfqpoint{4.338650in}{2.111922in}}{\pgfqpoint{4.338650in}{2.122972in}}%
\pgfpathcurveto{\pgfqpoint{4.338650in}{2.134022in}}{\pgfqpoint{4.334260in}{2.144621in}}{\pgfqpoint{4.326447in}{2.152435in}}%
\pgfpathcurveto{\pgfqpoint{4.318633in}{2.160249in}}{\pgfqpoint{4.308034in}{2.164639in}}{\pgfqpoint{4.296984in}{2.164639in}}%
\pgfpathcurveto{\pgfqpoint{4.285934in}{2.164639in}}{\pgfqpoint{4.275335in}{2.160249in}}{\pgfqpoint{4.267521in}{2.152435in}}%
\pgfpathcurveto{\pgfqpoint{4.259707in}{2.144621in}}{\pgfqpoint{4.255317in}{2.134022in}}{\pgfqpoint{4.255317in}{2.122972in}}%
\pgfpathcurveto{\pgfqpoint{4.255317in}{2.111922in}}{\pgfqpoint{4.259707in}{2.101323in}}{\pgfqpoint{4.267521in}{2.093509in}}%
\pgfpathcurveto{\pgfqpoint{4.275335in}{2.085696in}}{\pgfqpoint{4.285934in}{2.081305in}}{\pgfqpoint{4.296984in}{2.081305in}}%
\pgfpathclose%
\pgfusepath{stroke,fill}%
\end{pgfscope}%
\begin{pgfscope}%
\pgfpathrectangle{\pgfqpoint{0.481978in}{0.331635in}}{\pgfqpoint{4.960000in}{3.696000in}}%
\pgfusepath{clip}%
\pgfsetbuttcap%
\pgfsetroundjoin%
\definecolor{currentfill}{rgb}{1.000000,0.705882,0.509804}%
\pgfsetfillcolor{currentfill}%
\pgfsetlinewidth{0.481800pt}%
\definecolor{currentstroke}{rgb}{1.000000,1.000000,1.000000}%
\pgfsetstrokecolor{currentstroke}%
\pgfsetdash{}{0pt}%
\pgfpathmoveto{\pgfqpoint{3.619173in}{1.333258in}}%
\pgfpathcurveto{\pgfqpoint{3.630223in}{1.333258in}}{\pgfqpoint{3.640822in}{1.337648in}}{\pgfqpoint{3.648636in}{1.345462in}}%
\pgfpathcurveto{\pgfqpoint{3.656450in}{1.353276in}}{\pgfqpoint{3.660840in}{1.363875in}}{\pgfqpoint{3.660840in}{1.374925in}}%
\pgfpathcurveto{\pgfqpoint{3.660840in}{1.385975in}}{\pgfqpoint{3.656450in}{1.396574in}}{\pgfqpoint{3.648636in}{1.404388in}}%
\pgfpathcurveto{\pgfqpoint{3.640822in}{1.412201in}}{\pgfqpoint{3.630223in}{1.416592in}}{\pgfqpoint{3.619173in}{1.416592in}}%
\pgfpathcurveto{\pgfqpoint{3.608123in}{1.416592in}}{\pgfqpoint{3.597524in}{1.412201in}}{\pgfqpoint{3.589710in}{1.404388in}}%
\pgfpathcurveto{\pgfqpoint{3.581897in}{1.396574in}}{\pgfqpoint{3.577506in}{1.385975in}}{\pgfqpoint{3.577506in}{1.374925in}}%
\pgfpathcurveto{\pgfqpoint{3.577506in}{1.363875in}}{\pgfqpoint{3.581897in}{1.353276in}}{\pgfqpoint{3.589710in}{1.345462in}}%
\pgfpathcurveto{\pgfqpoint{3.597524in}{1.337648in}}{\pgfqpoint{3.608123in}{1.333258in}}{\pgfqpoint{3.619173in}{1.333258in}}%
\pgfpathclose%
\pgfusepath{stroke,fill}%
\end{pgfscope}%
\begin{pgfscope}%
\pgfpathrectangle{\pgfqpoint{0.481978in}{0.331635in}}{\pgfqpoint{4.960000in}{3.696000in}}%
\pgfusepath{clip}%
\pgfsetbuttcap%
\pgfsetroundjoin%
\definecolor{currentfill}{rgb}{1.000000,0.705882,0.509804}%
\pgfsetfillcolor{currentfill}%
\pgfsetlinewidth{0.481800pt}%
\definecolor{currentstroke}{rgb}{1.000000,1.000000,1.000000}%
\pgfsetstrokecolor{currentstroke}%
\pgfsetdash{}{0pt}%
\pgfpathmoveto{\pgfqpoint{0.776093in}{1.064160in}}%
\pgfpathcurveto{\pgfqpoint{0.787143in}{1.064160in}}{\pgfqpoint{0.797742in}{1.068550in}}{\pgfqpoint{0.805556in}{1.076364in}}%
\pgfpathcurveto{\pgfqpoint{0.813370in}{1.084177in}}{\pgfqpoint{0.817760in}{1.094776in}}{\pgfqpoint{0.817760in}{1.105827in}}%
\pgfpathcurveto{\pgfqpoint{0.817760in}{1.116877in}}{\pgfqpoint{0.813370in}{1.127476in}}{\pgfqpoint{0.805556in}{1.135289in}}%
\pgfpathcurveto{\pgfqpoint{0.797742in}{1.143103in}}{\pgfqpoint{0.787143in}{1.147493in}}{\pgfqpoint{0.776093in}{1.147493in}}%
\pgfpathcurveto{\pgfqpoint{0.765043in}{1.147493in}}{\pgfqpoint{0.754444in}{1.143103in}}{\pgfqpoint{0.746631in}{1.135289in}}%
\pgfpathcurveto{\pgfqpoint{0.738817in}{1.127476in}}{\pgfqpoint{0.734427in}{1.116877in}}{\pgfqpoint{0.734427in}{1.105827in}}%
\pgfpathcurveto{\pgfqpoint{0.734427in}{1.094776in}}{\pgfqpoint{0.738817in}{1.084177in}}{\pgfqpoint{0.746631in}{1.076364in}}%
\pgfpathcurveto{\pgfqpoint{0.754444in}{1.068550in}}{\pgfqpoint{0.765043in}{1.064160in}}{\pgfqpoint{0.776093in}{1.064160in}}%
\pgfpathclose%
\pgfusepath{stroke,fill}%
\end{pgfscope}%
\begin{pgfscope}%
\pgfpathrectangle{\pgfqpoint{0.481978in}{0.331635in}}{\pgfqpoint{4.960000in}{3.696000in}}%
\pgfusepath{clip}%
\pgfsetbuttcap%
\pgfsetroundjoin%
\definecolor{currentfill}{rgb}{1.000000,0.705882,0.509804}%
\pgfsetfillcolor{currentfill}%
\pgfsetlinewidth{0.481800pt}%
\definecolor{currentstroke}{rgb}{1.000000,1.000000,1.000000}%
\pgfsetstrokecolor{currentstroke}%
\pgfsetdash{}{0pt}%
\pgfpathmoveto{\pgfqpoint{2.997617in}{1.817225in}}%
\pgfpathcurveto{\pgfqpoint{3.008667in}{1.817225in}}{\pgfqpoint{3.019266in}{1.821616in}}{\pgfqpoint{3.027080in}{1.829429in}}%
\pgfpathcurveto{\pgfqpoint{3.034894in}{1.837243in}}{\pgfqpoint{3.039284in}{1.847842in}}{\pgfqpoint{3.039284in}{1.858892in}}%
\pgfpathcurveto{\pgfqpoint{3.039284in}{1.869942in}}{\pgfqpoint{3.034894in}{1.880541in}}{\pgfqpoint{3.027080in}{1.888355in}}%
\pgfpathcurveto{\pgfqpoint{3.019266in}{1.896168in}}{\pgfqpoint{3.008667in}{1.900559in}}{\pgfqpoint{2.997617in}{1.900559in}}%
\pgfpathcurveto{\pgfqpoint{2.986567in}{1.900559in}}{\pgfqpoint{2.975968in}{1.896168in}}{\pgfqpoint{2.968154in}{1.888355in}}%
\pgfpathcurveto{\pgfqpoint{2.960341in}{1.880541in}}{\pgfqpoint{2.955951in}{1.869942in}}{\pgfqpoint{2.955951in}{1.858892in}}%
\pgfpathcurveto{\pgfqpoint{2.955951in}{1.847842in}}{\pgfqpoint{2.960341in}{1.837243in}}{\pgfqpoint{2.968154in}{1.829429in}}%
\pgfpathcurveto{\pgfqpoint{2.975968in}{1.821616in}}{\pgfqpoint{2.986567in}{1.817225in}}{\pgfqpoint{2.997617in}{1.817225in}}%
\pgfpathclose%
\pgfusepath{stroke,fill}%
\end{pgfscope}%
\begin{pgfscope}%
\pgfpathrectangle{\pgfqpoint{0.481978in}{0.331635in}}{\pgfqpoint{4.960000in}{3.696000in}}%
\pgfusepath{clip}%
\pgfsetbuttcap%
\pgfsetroundjoin%
\definecolor{currentfill}{rgb}{1.000000,0.705882,0.509804}%
\pgfsetfillcolor{currentfill}%
\pgfsetlinewidth{0.481800pt}%
\definecolor{currentstroke}{rgb}{1.000000,1.000000,1.000000}%
\pgfsetstrokecolor{currentstroke}%
\pgfsetdash{}{0pt}%
\pgfpathmoveto{\pgfqpoint{3.008821in}{1.406995in}}%
\pgfpathcurveto{\pgfqpoint{3.019871in}{1.406995in}}{\pgfqpoint{3.030470in}{1.411386in}}{\pgfqpoint{3.038284in}{1.419199in}}%
\pgfpathcurveto{\pgfqpoint{3.046098in}{1.427013in}}{\pgfqpoint{3.050488in}{1.437612in}}{\pgfqpoint{3.050488in}{1.448662in}}%
\pgfpathcurveto{\pgfqpoint{3.050488in}{1.459712in}}{\pgfqpoint{3.046098in}{1.470311in}}{\pgfqpoint{3.038284in}{1.478125in}}%
\pgfpathcurveto{\pgfqpoint{3.030470in}{1.485938in}}{\pgfqpoint{3.019871in}{1.490329in}}{\pgfqpoint{3.008821in}{1.490329in}}%
\pgfpathcurveto{\pgfqpoint{2.997771in}{1.490329in}}{\pgfqpoint{2.987172in}{1.485938in}}{\pgfqpoint{2.979359in}{1.478125in}}%
\pgfpathcurveto{\pgfqpoint{2.971545in}{1.470311in}}{\pgfqpoint{2.967155in}{1.459712in}}{\pgfqpoint{2.967155in}{1.448662in}}%
\pgfpathcurveto{\pgfqpoint{2.967155in}{1.437612in}}{\pgfqpoint{2.971545in}{1.427013in}}{\pgfqpoint{2.979359in}{1.419199in}}%
\pgfpathcurveto{\pgfqpoint{2.987172in}{1.411386in}}{\pgfqpoint{2.997771in}{1.406995in}}{\pgfqpoint{3.008821in}{1.406995in}}%
\pgfpathclose%
\pgfusepath{stroke,fill}%
\end{pgfscope}%
\begin{pgfscope}%
\pgfpathrectangle{\pgfqpoint{0.481978in}{0.331635in}}{\pgfqpoint{4.960000in}{3.696000in}}%
\pgfusepath{clip}%
\pgfsetbuttcap%
\pgfsetroundjoin%
\definecolor{currentfill}{rgb}{1.000000,0.705882,0.509804}%
\pgfsetfillcolor{currentfill}%
\pgfsetlinewidth{0.481800pt}%
\definecolor{currentstroke}{rgb}{1.000000,1.000000,1.000000}%
\pgfsetstrokecolor{currentstroke}%
\pgfsetdash{}{0pt}%
\pgfpathmoveto{\pgfqpoint{1.670611in}{2.167003in}}%
\pgfpathcurveto{\pgfqpoint{1.681661in}{2.167003in}}{\pgfqpoint{1.692260in}{2.171393in}}{\pgfqpoint{1.700073in}{2.179207in}}%
\pgfpathcurveto{\pgfqpoint{1.707887in}{2.187021in}}{\pgfqpoint{1.712277in}{2.197620in}}{\pgfqpoint{1.712277in}{2.208670in}}%
\pgfpathcurveto{\pgfqpoint{1.712277in}{2.219720in}}{\pgfqpoint{1.707887in}{2.230319in}}{\pgfqpoint{1.700073in}{2.238133in}}%
\pgfpathcurveto{\pgfqpoint{1.692260in}{2.245946in}}{\pgfqpoint{1.681661in}{2.250337in}}{\pgfqpoint{1.670611in}{2.250337in}}%
\pgfpathcurveto{\pgfqpoint{1.659560in}{2.250337in}}{\pgfqpoint{1.648961in}{2.245946in}}{\pgfqpoint{1.641148in}{2.238133in}}%
\pgfpathcurveto{\pgfqpoint{1.633334in}{2.230319in}}{\pgfqpoint{1.628944in}{2.219720in}}{\pgfqpoint{1.628944in}{2.208670in}}%
\pgfpathcurveto{\pgfqpoint{1.628944in}{2.197620in}}{\pgfqpoint{1.633334in}{2.187021in}}{\pgfqpoint{1.641148in}{2.179207in}}%
\pgfpathcurveto{\pgfqpoint{1.648961in}{2.171393in}}{\pgfqpoint{1.659560in}{2.167003in}}{\pgfqpoint{1.670611in}{2.167003in}}%
\pgfpathclose%
\pgfusepath{stroke,fill}%
\end{pgfscope}%
\begin{pgfscope}%
\pgfpathrectangle{\pgfqpoint{0.481978in}{0.331635in}}{\pgfqpoint{4.960000in}{3.696000in}}%
\pgfusepath{clip}%
\pgfsetbuttcap%
\pgfsetroundjoin%
\definecolor{currentfill}{rgb}{1.000000,0.705882,0.509804}%
\pgfsetfillcolor{currentfill}%
\pgfsetlinewidth{0.481800pt}%
\definecolor{currentstroke}{rgb}{1.000000,1.000000,1.000000}%
\pgfsetstrokecolor{currentstroke}%
\pgfsetdash{}{0pt}%
\pgfpathmoveto{\pgfqpoint{0.919551in}{2.023154in}}%
\pgfpathcurveto{\pgfqpoint{0.930601in}{2.023154in}}{\pgfqpoint{0.941200in}{2.027544in}}{\pgfqpoint{0.949013in}{2.035357in}}%
\pgfpathcurveto{\pgfqpoint{0.956827in}{2.043171in}}{\pgfqpoint{0.961217in}{2.053770in}}{\pgfqpoint{0.961217in}{2.064820in}}%
\pgfpathcurveto{\pgfqpoint{0.961217in}{2.075870in}}{\pgfqpoint{0.956827in}{2.086469in}}{\pgfqpoint{0.949013in}{2.094283in}}%
\pgfpathcurveto{\pgfqpoint{0.941200in}{2.102097in}}{\pgfqpoint{0.930601in}{2.106487in}}{\pgfqpoint{0.919551in}{2.106487in}}%
\pgfpathcurveto{\pgfqpoint{0.908500in}{2.106487in}}{\pgfqpoint{0.897901in}{2.102097in}}{\pgfqpoint{0.890088in}{2.094283in}}%
\pgfpathcurveto{\pgfqpoint{0.882274in}{2.086469in}}{\pgfqpoint{0.877884in}{2.075870in}}{\pgfqpoint{0.877884in}{2.064820in}}%
\pgfpathcurveto{\pgfqpoint{0.877884in}{2.053770in}}{\pgfqpoint{0.882274in}{2.043171in}}{\pgfqpoint{0.890088in}{2.035357in}}%
\pgfpathcurveto{\pgfqpoint{0.897901in}{2.027544in}}{\pgfqpoint{0.908500in}{2.023154in}}{\pgfqpoint{0.919551in}{2.023154in}}%
\pgfpathclose%
\pgfusepath{stroke,fill}%
\end{pgfscope}%
\begin{pgfscope}%
\pgfpathrectangle{\pgfqpoint{0.481978in}{0.331635in}}{\pgfqpoint{4.960000in}{3.696000in}}%
\pgfusepath{clip}%
\pgfsetbuttcap%
\pgfsetroundjoin%
\definecolor{currentfill}{rgb}{1.000000,0.705882,0.509804}%
\pgfsetfillcolor{currentfill}%
\pgfsetlinewidth{0.481800pt}%
\definecolor{currentstroke}{rgb}{1.000000,1.000000,1.000000}%
\pgfsetstrokecolor{currentstroke}%
\pgfsetdash{}{0pt}%
\pgfpathmoveto{\pgfqpoint{2.983995in}{0.924591in}}%
\pgfpathcurveto{\pgfqpoint{2.995046in}{0.924591in}}{\pgfqpoint{3.005645in}{0.928981in}}{\pgfqpoint{3.013458in}{0.936795in}}%
\pgfpathcurveto{\pgfqpoint{3.021272in}{0.944608in}}{\pgfqpoint{3.025662in}{0.955207in}}{\pgfqpoint{3.025662in}{0.966258in}}%
\pgfpathcurveto{\pgfqpoint{3.025662in}{0.977308in}}{\pgfqpoint{3.021272in}{0.987907in}}{\pgfqpoint{3.013458in}{0.995720in}}%
\pgfpathcurveto{\pgfqpoint{3.005645in}{1.003534in}}{\pgfqpoint{2.995046in}{1.007924in}}{\pgfqpoint{2.983995in}{1.007924in}}%
\pgfpathcurveto{\pgfqpoint{2.972945in}{1.007924in}}{\pgfqpoint{2.962346in}{1.003534in}}{\pgfqpoint{2.954533in}{0.995720in}}%
\pgfpathcurveto{\pgfqpoint{2.946719in}{0.987907in}}{\pgfqpoint{2.942329in}{0.977308in}}{\pgfqpoint{2.942329in}{0.966258in}}%
\pgfpathcurveto{\pgfqpoint{2.942329in}{0.955207in}}{\pgfqpoint{2.946719in}{0.944608in}}{\pgfqpoint{2.954533in}{0.936795in}}%
\pgfpathcurveto{\pgfqpoint{2.962346in}{0.928981in}}{\pgfqpoint{2.972945in}{0.924591in}}{\pgfqpoint{2.983995in}{0.924591in}}%
\pgfpathclose%
\pgfusepath{stroke,fill}%
\end{pgfscope}%
\begin{pgfscope}%
\pgfpathrectangle{\pgfqpoint{0.481978in}{0.331635in}}{\pgfqpoint{4.960000in}{3.696000in}}%
\pgfusepath{clip}%
\pgfsetbuttcap%
\pgfsetroundjoin%
\definecolor{currentfill}{rgb}{1.000000,0.705882,0.509804}%
\pgfsetfillcolor{currentfill}%
\pgfsetlinewidth{0.481800pt}%
\definecolor{currentstroke}{rgb}{1.000000,1.000000,1.000000}%
\pgfsetstrokecolor{currentstroke}%
\pgfsetdash{}{0pt}%
\pgfpathmoveto{\pgfqpoint{2.557995in}{1.651007in}}%
\pgfpathcurveto{\pgfqpoint{2.569045in}{1.651007in}}{\pgfqpoint{2.579644in}{1.655397in}}{\pgfqpoint{2.587458in}{1.663211in}}%
\pgfpathcurveto{\pgfqpoint{2.595271in}{1.671024in}}{\pgfqpoint{2.599661in}{1.681623in}}{\pgfqpoint{2.599661in}{1.692674in}}%
\pgfpathcurveto{\pgfqpoint{2.599661in}{1.703724in}}{\pgfqpoint{2.595271in}{1.714323in}}{\pgfqpoint{2.587458in}{1.722136in}}%
\pgfpathcurveto{\pgfqpoint{2.579644in}{1.729950in}}{\pgfqpoint{2.569045in}{1.734340in}}{\pgfqpoint{2.557995in}{1.734340in}}%
\pgfpathcurveto{\pgfqpoint{2.546945in}{1.734340in}}{\pgfqpoint{2.536346in}{1.729950in}}{\pgfqpoint{2.528532in}{1.722136in}}%
\pgfpathcurveto{\pgfqpoint{2.520718in}{1.714323in}}{\pgfqpoint{2.516328in}{1.703724in}}{\pgfqpoint{2.516328in}{1.692674in}}%
\pgfpathcurveto{\pgfqpoint{2.516328in}{1.681623in}}{\pgfqpoint{2.520718in}{1.671024in}}{\pgfqpoint{2.528532in}{1.663211in}}%
\pgfpathcurveto{\pgfqpoint{2.536346in}{1.655397in}}{\pgfqpoint{2.546945in}{1.651007in}}{\pgfqpoint{2.557995in}{1.651007in}}%
\pgfpathclose%
\pgfusepath{stroke,fill}%
\end{pgfscope}%
\begin{pgfscope}%
\pgfpathrectangle{\pgfqpoint{0.481978in}{0.331635in}}{\pgfqpoint{4.960000in}{3.696000in}}%
\pgfusepath{clip}%
\pgfsetbuttcap%
\pgfsetroundjoin%
\definecolor{currentfill}{rgb}{1.000000,0.705882,0.509804}%
\pgfsetfillcolor{currentfill}%
\pgfsetlinewidth{0.481800pt}%
\definecolor{currentstroke}{rgb}{1.000000,1.000000,1.000000}%
\pgfsetstrokecolor{currentstroke}%
\pgfsetdash{}{0pt}%
\pgfpathmoveto{\pgfqpoint{1.537027in}{2.393580in}}%
\pgfpathcurveto{\pgfqpoint{1.548077in}{2.393580in}}{\pgfqpoint{1.558676in}{2.397970in}}{\pgfqpoint{1.566489in}{2.405784in}}%
\pgfpathcurveto{\pgfqpoint{1.574303in}{2.413598in}}{\pgfqpoint{1.578693in}{2.424197in}}{\pgfqpoint{1.578693in}{2.435247in}}%
\pgfpathcurveto{\pgfqpoint{1.578693in}{2.446297in}}{\pgfqpoint{1.574303in}{2.456896in}}{\pgfqpoint{1.566489in}{2.464709in}}%
\pgfpathcurveto{\pgfqpoint{1.558676in}{2.472523in}}{\pgfqpoint{1.548077in}{2.476913in}}{\pgfqpoint{1.537027in}{2.476913in}}%
\pgfpathcurveto{\pgfqpoint{1.525976in}{2.476913in}}{\pgfqpoint{1.515377in}{2.472523in}}{\pgfqpoint{1.507564in}{2.464709in}}%
\pgfpathcurveto{\pgfqpoint{1.499750in}{2.456896in}}{\pgfqpoint{1.495360in}{2.446297in}}{\pgfqpoint{1.495360in}{2.435247in}}%
\pgfpathcurveto{\pgfqpoint{1.495360in}{2.424197in}}{\pgfqpoint{1.499750in}{2.413598in}}{\pgfqpoint{1.507564in}{2.405784in}}%
\pgfpathcurveto{\pgfqpoint{1.515377in}{2.397970in}}{\pgfqpoint{1.525976in}{2.393580in}}{\pgfqpoint{1.537027in}{2.393580in}}%
\pgfpathclose%
\pgfusepath{stroke,fill}%
\end{pgfscope}%
\begin{pgfscope}%
\pgfpathrectangle{\pgfqpoint{0.481978in}{0.331635in}}{\pgfqpoint{4.960000in}{3.696000in}}%
\pgfusepath{clip}%
\pgfsetbuttcap%
\pgfsetroundjoin%
\definecolor{currentfill}{rgb}{1.000000,0.705882,0.509804}%
\pgfsetfillcolor{currentfill}%
\pgfsetlinewidth{0.481800pt}%
\definecolor{currentstroke}{rgb}{1.000000,1.000000,1.000000}%
\pgfsetstrokecolor{currentstroke}%
\pgfsetdash{}{0pt}%
\pgfpathmoveto{\pgfqpoint{1.984657in}{1.248164in}}%
\pgfpathcurveto{\pgfqpoint{1.995707in}{1.248164in}}{\pgfqpoint{2.006306in}{1.252555in}}{\pgfqpoint{2.014120in}{1.260368in}}%
\pgfpathcurveto{\pgfqpoint{2.021933in}{1.268182in}}{\pgfqpoint{2.026323in}{1.278781in}}{\pgfqpoint{2.026323in}{1.289831in}}%
\pgfpathcurveto{\pgfqpoint{2.026323in}{1.300881in}}{\pgfqpoint{2.021933in}{1.311480in}}{\pgfqpoint{2.014120in}{1.319294in}}%
\pgfpathcurveto{\pgfqpoint{2.006306in}{1.327107in}}{\pgfqpoint{1.995707in}{1.331498in}}{\pgfqpoint{1.984657in}{1.331498in}}%
\pgfpathcurveto{\pgfqpoint{1.973607in}{1.331498in}}{\pgfqpoint{1.963008in}{1.327107in}}{\pgfqpoint{1.955194in}{1.319294in}}%
\pgfpathcurveto{\pgfqpoint{1.947380in}{1.311480in}}{\pgfqpoint{1.942990in}{1.300881in}}{\pgfqpoint{1.942990in}{1.289831in}}%
\pgfpathcurveto{\pgfqpoint{1.942990in}{1.278781in}}{\pgfqpoint{1.947380in}{1.268182in}}{\pgfqpoint{1.955194in}{1.260368in}}%
\pgfpathcurveto{\pgfqpoint{1.963008in}{1.252555in}}{\pgfqpoint{1.973607in}{1.248164in}}{\pgfqpoint{1.984657in}{1.248164in}}%
\pgfpathclose%
\pgfusepath{stroke,fill}%
\end{pgfscope}%
\begin{pgfscope}%
\pgfpathrectangle{\pgfqpoint{0.481978in}{0.331635in}}{\pgfqpoint{4.960000in}{3.696000in}}%
\pgfusepath{clip}%
\pgfsetbuttcap%
\pgfsetroundjoin%
\definecolor{currentfill}{rgb}{1.000000,0.705882,0.509804}%
\pgfsetfillcolor{currentfill}%
\pgfsetlinewidth{0.481800pt}%
\definecolor{currentstroke}{rgb}{1.000000,1.000000,1.000000}%
\pgfsetstrokecolor{currentstroke}%
\pgfsetdash{}{0pt}%
\pgfpathmoveto{\pgfqpoint{0.997912in}{1.292018in}}%
\pgfpathcurveto{\pgfqpoint{1.008962in}{1.292018in}}{\pgfqpoint{1.019561in}{1.296408in}}{\pgfqpoint{1.027375in}{1.304222in}}%
\pgfpathcurveto{\pgfqpoint{1.035189in}{1.312035in}}{\pgfqpoint{1.039579in}{1.322634in}}{\pgfqpoint{1.039579in}{1.333684in}}%
\pgfpathcurveto{\pgfqpoint{1.039579in}{1.344735in}}{\pgfqpoint{1.035189in}{1.355334in}}{\pgfqpoint{1.027375in}{1.363147in}}%
\pgfpathcurveto{\pgfqpoint{1.019561in}{1.370961in}}{\pgfqpoint{1.008962in}{1.375351in}}{\pgfqpoint{0.997912in}{1.375351in}}%
\pgfpathcurveto{\pgfqpoint{0.986862in}{1.375351in}}{\pgfqpoint{0.976263in}{1.370961in}}{\pgfqpoint{0.968449in}{1.363147in}}%
\pgfpathcurveto{\pgfqpoint{0.960636in}{1.355334in}}{\pgfqpoint{0.956245in}{1.344735in}}{\pgfqpoint{0.956245in}{1.333684in}}%
\pgfpathcurveto{\pgfqpoint{0.956245in}{1.322634in}}{\pgfqpoint{0.960636in}{1.312035in}}{\pgfqpoint{0.968449in}{1.304222in}}%
\pgfpathcurveto{\pgfqpoint{0.976263in}{1.296408in}}{\pgfqpoint{0.986862in}{1.292018in}}{\pgfqpoint{0.997912in}{1.292018in}}%
\pgfpathclose%
\pgfusepath{stroke,fill}%
\end{pgfscope}%
\begin{pgfscope}%
\pgfpathrectangle{\pgfqpoint{0.481978in}{0.331635in}}{\pgfqpoint{4.960000in}{3.696000in}}%
\pgfusepath{clip}%
\pgfsetbuttcap%
\pgfsetroundjoin%
\definecolor{currentfill}{rgb}{1.000000,0.705882,0.509804}%
\pgfsetfillcolor{currentfill}%
\pgfsetlinewidth{0.481800pt}%
\definecolor{currentstroke}{rgb}{1.000000,1.000000,1.000000}%
\pgfsetstrokecolor{currentstroke}%
\pgfsetdash{}{0pt}%
\pgfpathmoveto{\pgfqpoint{3.107756in}{0.457968in}}%
\pgfpathcurveto{\pgfqpoint{3.118806in}{0.457968in}}{\pgfqpoint{3.129405in}{0.462359in}}{\pgfqpoint{3.137219in}{0.470172in}}%
\pgfpathcurveto{\pgfqpoint{3.145032in}{0.477986in}}{\pgfqpoint{3.149423in}{0.488585in}}{\pgfqpoint{3.149423in}{0.499635in}}%
\pgfpathcurveto{\pgfqpoint{3.149423in}{0.510685in}}{\pgfqpoint{3.145032in}{0.521284in}}{\pgfqpoint{3.137219in}{0.529098in}}%
\pgfpathcurveto{\pgfqpoint{3.129405in}{0.536911in}}{\pgfqpoint{3.118806in}{0.541302in}}{\pgfqpoint{3.107756in}{0.541302in}}%
\pgfpathcurveto{\pgfqpoint{3.096706in}{0.541302in}}{\pgfqpoint{3.086107in}{0.536911in}}{\pgfqpoint{3.078293in}{0.529098in}}%
\pgfpathcurveto{\pgfqpoint{3.070479in}{0.521284in}}{\pgfqpoint{3.066089in}{0.510685in}}{\pgfqpoint{3.066089in}{0.499635in}}%
\pgfpathcurveto{\pgfqpoint{3.066089in}{0.488585in}}{\pgfqpoint{3.070479in}{0.477986in}}{\pgfqpoint{3.078293in}{0.470172in}}%
\pgfpathcurveto{\pgfqpoint{3.086107in}{0.462359in}}{\pgfqpoint{3.096706in}{0.457968in}}{\pgfqpoint{3.107756in}{0.457968in}}%
\pgfpathclose%
\pgfusepath{stroke,fill}%
\end{pgfscope}%
\begin{pgfscope}%
\pgfpathrectangle{\pgfqpoint{0.481978in}{0.331635in}}{\pgfqpoint{4.960000in}{3.696000in}}%
\pgfusepath{clip}%
\pgfsetbuttcap%
\pgfsetroundjoin%
\definecolor{currentfill}{rgb}{1.000000,0.705882,0.509804}%
\pgfsetfillcolor{currentfill}%
\pgfsetlinewidth{0.481800pt}%
\definecolor{currentstroke}{rgb}{1.000000,1.000000,1.000000}%
\pgfsetstrokecolor{currentstroke}%
\pgfsetdash{}{0pt}%
\pgfpathmoveto{\pgfqpoint{2.937197in}{0.957378in}}%
\pgfpathcurveto{\pgfqpoint{2.948247in}{0.957378in}}{\pgfqpoint{2.958846in}{0.961769in}}{\pgfqpoint{2.966660in}{0.969582in}}%
\pgfpathcurveto{\pgfqpoint{2.974474in}{0.977396in}}{\pgfqpoint{2.978864in}{0.987995in}}{\pgfqpoint{2.978864in}{0.999045in}}%
\pgfpathcurveto{\pgfqpoint{2.978864in}{1.010095in}}{\pgfqpoint{2.974474in}{1.020694in}}{\pgfqpoint{2.966660in}{1.028508in}}%
\pgfpathcurveto{\pgfqpoint{2.958846in}{1.036322in}}{\pgfqpoint{2.948247in}{1.040712in}}{\pgfqpoint{2.937197in}{1.040712in}}%
\pgfpathcurveto{\pgfqpoint{2.926147in}{1.040712in}}{\pgfqpoint{2.915548in}{1.036322in}}{\pgfqpoint{2.907735in}{1.028508in}}%
\pgfpathcurveto{\pgfqpoint{2.899921in}{1.020694in}}{\pgfqpoint{2.895531in}{1.010095in}}{\pgfqpoint{2.895531in}{0.999045in}}%
\pgfpathcurveto{\pgfqpoint{2.895531in}{0.987995in}}{\pgfqpoint{2.899921in}{0.977396in}}{\pgfqpoint{2.907735in}{0.969582in}}%
\pgfpathcurveto{\pgfqpoint{2.915548in}{0.961769in}}{\pgfqpoint{2.926147in}{0.957378in}}{\pgfqpoint{2.937197in}{0.957378in}}%
\pgfpathclose%
\pgfusepath{stroke,fill}%
\end{pgfscope}%
\begin{pgfscope}%
\pgfpathrectangle{\pgfqpoint{0.481978in}{0.331635in}}{\pgfqpoint{4.960000in}{3.696000in}}%
\pgfusepath{clip}%
\pgfsetbuttcap%
\pgfsetroundjoin%
\definecolor{currentfill}{rgb}{1.000000,0.705882,0.509804}%
\pgfsetfillcolor{currentfill}%
\pgfsetlinewidth{0.481800pt}%
\definecolor{currentstroke}{rgb}{1.000000,1.000000,1.000000}%
\pgfsetstrokecolor{currentstroke}%
\pgfsetdash{}{0pt}%
\pgfpathmoveto{\pgfqpoint{2.717812in}{1.334459in}}%
\pgfpathcurveto{\pgfqpoint{2.728862in}{1.334459in}}{\pgfqpoint{2.739461in}{1.338850in}}{\pgfqpoint{2.747274in}{1.346663in}}%
\pgfpathcurveto{\pgfqpoint{2.755088in}{1.354477in}}{\pgfqpoint{2.759478in}{1.365076in}}{\pgfqpoint{2.759478in}{1.376126in}}%
\pgfpathcurveto{\pgfqpoint{2.759478in}{1.387176in}}{\pgfqpoint{2.755088in}{1.397775in}}{\pgfqpoint{2.747274in}{1.405589in}}%
\pgfpathcurveto{\pgfqpoint{2.739461in}{1.413402in}}{\pgfqpoint{2.728862in}{1.417793in}}{\pgfqpoint{2.717812in}{1.417793in}}%
\pgfpathcurveto{\pgfqpoint{2.706762in}{1.417793in}}{\pgfqpoint{2.696163in}{1.413402in}}{\pgfqpoint{2.688349in}{1.405589in}}%
\pgfpathcurveto{\pgfqpoint{2.680535in}{1.397775in}}{\pgfqpoint{2.676145in}{1.387176in}}{\pgfqpoint{2.676145in}{1.376126in}}%
\pgfpathcurveto{\pgfqpoint{2.676145in}{1.365076in}}{\pgfqpoint{2.680535in}{1.354477in}}{\pgfqpoint{2.688349in}{1.346663in}}%
\pgfpathcurveto{\pgfqpoint{2.696163in}{1.338850in}}{\pgfqpoint{2.706762in}{1.334459in}}{\pgfqpoint{2.717812in}{1.334459in}}%
\pgfpathclose%
\pgfusepath{stroke,fill}%
\end{pgfscope}%
\begin{pgfscope}%
\pgfpathrectangle{\pgfqpoint{0.481978in}{0.331635in}}{\pgfqpoint{4.960000in}{3.696000in}}%
\pgfusepath{clip}%
\pgfsetbuttcap%
\pgfsetroundjoin%
\definecolor{currentfill}{rgb}{1.000000,0.705882,0.509804}%
\pgfsetfillcolor{currentfill}%
\pgfsetlinewidth{0.481800pt}%
\definecolor{currentstroke}{rgb}{1.000000,1.000000,1.000000}%
\pgfsetstrokecolor{currentstroke}%
\pgfsetdash{}{0pt}%
\pgfpathmoveto{\pgfqpoint{2.821264in}{1.490696in}}%
\pgfpathcurveto{\pgfqpoint{2.832314in}{1.490696in}}{\pgfqpoint{2.842913in}{1.495087in}}{\pgfqpoint{2.850726in}{1.502900in}}%
\pgfpathcurveto{\pgfqpoint{2.858540in}{1.510714in}}{\pgfqpoint{2.862930in}{1.521313in}}{\pgfqpoint{2.862930in}{1.532363in}}%
\pgfpathcurveto{\pgfqpoint{2.862930in}{1.543413in}}{\pgfqpoint{2.858540in}{1.554012in}}{\pgfqpoint{2.850726in}{1.561826in}}%
\pgfpathcurveto{\pgfqpoint{2.842913in}{1.569639in}}{\pgfqpoint{2.832314in}{1.574030in}}{\pgfqpoint{2.821264in}{1.574030in}}%
\pgfpathcurveto{\pgfqpoint{2.810213in}{1.574030in}}{\pgfqpoint{2.799614in}{1.569639in}}{\pgfqpoint{2.791801in}{1.561826in}}%
\pgfpathcurveto{\pgfqpoint{2.783987in}{1.554012in}}{\pgfqpoint{2.779597in}{1.543413in}}{\pgfqpoint{2.779597in}{1.532363in}}%
\pgfpathcurveto{\pgfqpoint{2.779597in}{1.521313in}}{\pgfqpoint{2.783987in}{1.510714in}}{\pgfqpoint{2.791801in}{1.502900in}}%
\pgfpathcurveto{\pgfqpoint{2.799614in}{1.495087in}}{\pgfqpoint{2.810213in}{1.490696in}}{\pgfqpoint{2.821264in}{1.490696in}}%
\pgfpathclose%
\pgfusepath{stroke,fill}%
\end{pgfscope}%
\begin{pgfscope}%
\pgfpathrectangle{\pgfqpoint{0.481978in}{0.331635in}}{\pgfqpoint{4.960000in}{3.696000in}}%
\pgfusepath{clip}%
\pgfsetbuttcap%
\pgfsetroundjoin%
\definecolor{currentfill}{rgb}{1.000000,0.705882,0.509804}%
\pgfsetfillcolor{currentfill}%
\pgfsetlinewidth{0.481800pt}%
\definecolor{currentstroke}{rgb}{1.000000,1.000000,1.000000}%
\pgfsetstrokecolor{currentstroke}%
\pgfsetdash{}{0pt}%
\pgfpathmoveto{\pgfqpoint{1.545663in}{0.832605in}}%
\pgfpathcurveto{\pgfqpoint{1.556713in}{0.832605in}}{\pgfqpoint{1.567312in}{0.836995in}}{\pgfqpoint{1.575126in}{0.844809in}}%
\pgfpathcurveto{\pgfqpoint{1.582940in}{0.852623in}}{\pgfqpoint{1.587330in}{0.863222in}}{\pgfqpoint{1.587330in}{0.874272in}}%
\pgfpathcurveto{\pgfqpoint{1.587330in}{0.885322in}}{\pgfqpoint{1.582940in}{0.895921in}}{\pgfqpoint{1.575126in}{0.903735in}}%
\pgfpathcurveto{\pgfqpoint{1.567312in}{0.911548in}}{\pgfqpoint{1.556713in}{0.915939in}}{\pgfqpoint{1.545663in}{0.915939in}}%
\pgfpathcurveto{\pgfqpoint{1.534613in}{0.915939in}}{\pgfqpoint{1.524014in}{0.911548in}}{\pgfqpoint{1.516200in}{0.903735in}}%
\pgfpathcurveto{\pgfqpoint{1.508387in}{0.895921in}}{\pgfqpoint{1.503996in}{0.885322in}}{\pgfqpoint{1.503996in}{0.874272in}}%
\pgfpathcurveto{\pgfqpoint{1.503996in}{0.863222in}}{\pgfqpoint{1.508387in}{0.852623in}}{\pgfqpoint{1.516200in}{0.844809in}}%
\pgfpathcurveto{\pgfqpoint{1.524014in}{0.836995in}}{\pgfqpoint{1.534613in}{0.832605in}}{\pgfqpoint{1.545663in}{0.832605in}}%
\pgfpathclose%
\pgfusepath{stroke,fill}%
\end{pgfscope}%
\begin{pgfscope}%
\pgfpathrectangle{\pgfqpoint{0.481978in}{0.331635in}}{\pgfqpoint{4.960000in}{3.696000in}}%
\pgfusepath{clip}%
\pgfsetbuttcap%
\pgfsetroundjoin%
\definecolor{currentfill}{rgb}{1.000000,0.705882,0.509804}%
\pgfsetfillcolor{currentfill}%
\pgfsetlinewidth{0.481800pt}%
\definecolor{currentstroke}{rgb}{1.000000,1.000000,1.000000}%
\pgfsetstrokecolor{currentstroke}%
\pgfsetdash{}{0pt}%
\pgfpathmoveto{\pgfqpoint{1.189172in}{1.658649in}}%
\pgfpathcurveto{\pgfqpoint{1.200222in}{1.658649in}}{\pgfqpoint{1.210821in}{1.663040in}}{\pgfqpoint{1.218635in}{1.670853in}}%
\pgfpathcurveto{\pgfqpoint{1.226448in}{1.678667in}}{\pgfqpoint{1.230838in}{1.689266in}}{\pgfqpoint{1.230838in}{1.700316in}}%
\pgfpathcurveto{\pgfqpoint{1.230838in}{1.711366in}}{\pgfqpoint{1.226448in}{1.721965in}}{\pgfqpoint{1.218635in}{1.729779in}}%
\pgfpathcurveto{\pgfqpoint{1.210821in}{1.737592in}}{\pgfqpoint{1.200222in}{1.741983in}}{\pgfqpoint{1.189172in}{1.741983in}}%
\pgfpathcurveto{\pgfqpoint{1.178122in}{1.741983in}}{\pgfqpoint{1.167523in}{1.737592in}}{\pgfqpoint{1.159709in}{1.729779in}}%
\pgfpathcurveto{\pgfqpoint{1.151895in}{1.721965in}}{\pgfqpoint{1.147505in}{1.711366in}}{\pgfqpoint{1.147505in}{1.700316in}}%
\pgfpathcurveto{\pgfqpoint{1.147505in}{1.689266in}}{\pgfqpoint{1.151895in}{1.678667in}}{\pgfqpoint{1.159709in}{1.670853in}}%
\pgfpathcurveto{\pgfqpoint{1.167523in}{1.663040in}}{\pgfqpoint{1.178122in}{1.658649in}}{\pgfqpoint{1.189172in}{1.658649in}}%
\pgfpathclose%
\pgfusepath{stroke,fill}%
\end{pgfscope}%
\begin{pgfscope}%
\pgfpathrectangle{\pgfqpoint{0.481978in}{0.331635in}}{\pgfqpoint{4.960000in}{3.696000in}}%
\pgfusepath{clip}%
\pgfsetbuttcap%
\pgfsetroundjoin%
\definecolor{currentfill}{rgb}{1.000000,0.705882,0.509804}%
\pgfsetfillcolor{currentfill}%
\pgfsetlinewidth{0.481800pt}%
\definecolor{currentstroke}{rgb}{1.000000,1.000000,1.000000}%
\pgfsetstrokecolor{currentstroke}%
\pgfsetdash{}{0pt}%
\pgfpathmoveto{\pgfqpoint{1.517112in}{1.927067in}}%
\pgfpathcurveto{\pgfqpoint{1.528162in}{1.927067in}}{\pgfqpoint{1.538761in}{1.931458in}}{\pgfqpoint{1.546574in}{1.939271in}}%
\pgfpathcurveto{\pgfqpoint{1.554388in}{1.947085in}}{\pgfqpoint{1.558778in}{1.957684in}}{\pgfqpoint{1.558778in}{1.968734in}}%
\pgfpathcurveto{\pgfqpoint{1.558778in}{1.979784in}}{\pgfqpoint{1.554388in}{1.990383in}}{\pgfqpoint{1.546574in}{1.998197in}}%
\pgfpathcurveto{\pgfqpoint{1.538761in}{2.006010in}}{\pgfqpoint{1.528162in}{2.010401in}}{\pgfqpoint{1.517112in}{2.010401in}}%
\pgfpathcurveto{\pgfqpoint{1.506062in}{2.010401in}}{\pgfqpoint{1.495463in}{2.006010in}}{\pgfqpoint{1.487649in}{1.998197in}}%
\pgfpathcurveto{\pgfqpoint{1.479835in}{1.990383in}}{\pgfqpoint{1.475445in}{1.979784in}}{\pgfqpoint{1.475445in}{1.968734in}}%
\pgfpathcurveto{\pgfqpoint{1.475445in}{1.957684in}}{\pgfqpoint{1.479835in}{1.947085in}}{\pgfqpoint{1.487649in}{1.939271in}}%
\pgfpathcurveto{\pgfqpoint{1.495463in}{1.931458in}}{\pgfqpoint{1.506062in}{1.927067in}}{\pgfqpoint{1.517112in}{1.927067in}}%
\pgfpathclose%
\pgfusepath{stroke,fill}%
\end{pgfscope}%
\begin{pgfscope}%
\pgfpathrectangle{\pgfqpoint{0.481978in}{0.331635in}}{\pgfqpoint{4.960000in}{3.696000in}}%
\pgfusepath{clip}%
\pgfsetbuttcap%
\pgfsetroundjoin%
\definecolor{currentfill}{rgb}{1.000000,0.705882,0.509804}%
\pgfsetfillcolor{currentfill}%
\pgfsetlinewidth{0.481800pt}%
\definecolor{currentstroke}{rgb}{1.000000,1.000000,1.000000}%
\pgfsetstrokecolor{currentstroke}%
\pgfsetdash{}{0pt}%
\pgfpathmoveto{\pgfqpoint{3.227064in}{2.191048in}}%
\pgfpathcurveto{\pgfqpoint{3.238115in}{2.191048in}}{\pgfqpoint{3.248714in}{2.195439in}}{\pgfqpoint{3.256527in}{2.203252in}}%
\pgfpathcurveto{\pgfqpoint{3.264341in}{2.211066in}}{\pgfqpoint{3.268731in}{2.221665in}}{\pgfqpoint{3.268731in}{2.232715in}}%
\pgfpathcurveto{\pgfqpoint{3.268731in}{2.243765in}}{\pgfqpoint{3.264341in}{2.254364in}}{\pgfqpoint{3.256527in}{2.262178in}}%
\pgfpathcurveto{\pgfqpoint{3.248714in}{2.269992in}}{\pgfqpoint{3.238115in}{2.274382in}}{\pgfqpoint{3.227064in}{2.274382in}}%
\pgfpathcurveto{\pgfqpoint{3.216014in}{2.274382in}}{\pgfqpoint{3.205415in}{2.269992in}}{\pgfqpoint{3.197602in}{2.262178in}}%
\pgfpathcurveto{\pgfqpoint{3.189788in}{2.254364in}}{\pgfqpoint{3.185398in}{2.243765in}}{\pgfqpoint{3.185398in}{2.232715in}}%
\pgfpathcurveto{\pgfqpoint{3.185398in}{2.221665in}}{\pgfqpoint{3.189788in}{2.211066in}}{\pgfqpoint{3.197602in}{2.203252in}}%
\pgfpathcurveto{\pgfqpoint{3.205415in}{2.195439in}}{\pgfqpoint{3.216014in}{2.191048in}}{\pgfqpoint{3.227064in}{2.191048in}}%
\pgfpathclose%
\pgfusepath{stroke,fill}%
\end{pgfscope}%
\begin{pgfscope}%
\pgfpathrectangle{\pgfqpoint{0.481978in}{0.331635in}}{\pgfqpoint{4.960000in}{3.696000in}}%
\pgfusepath{clip}%
\pgfsetbuttcap%
\pgfsetroundjoin%
\definecolor{currentfill}{rgb}{1.000000,0.705882,0.509804}%
\pgfsetfillcolor{currentfill}%
\pgfsetlinewidth{0.481800pt}%
\definecolor{currentstroke}{rgb}{1.000000,1.000000,1.000000}%
\pgfsetstrokecolor{currentstroke}%
\pgfsetdash{}{0pt}%
\pgfpathmoveto{\pgfqpoint{0.992160in}{0.921648in}}%
\pgfpathcurveto{\pgfqpoint{1.003210in}{0.921648in}}{\pgfqpoint{1.013809in}{0.926038in}}{\pgfqpoint{1.021622in}{0.933852in}}%
\pgfpathcurveto{\pgfqpoint{1.029436in}{0.941666in}}{\pgfqpoint{1.033826in}{0.952265in}}{\pgfqpoint{1.033826in}{0.963315in}}%
\pgfpathcurveto{\pgfqpoint{1.033826in}{0.974365in}}{\pgfqpoint{1.029436in}{0.984964in}}{\pgfqpoint{1.021622in}{0.992778in}}%
\pgfpathcurveto{\pgfqpoint{1.013809in}{1.000591in}}{\pgfqpoint{1.003210in}{1.004981in}}{\pgfqpoint{0.992160in}{1.004981in}}%
\pgfpathcurveto{\pgfqpoint{0.981110in}{1.004981in}}{\pgfqpoint{0.970510in}{1.000591in}}{\pgfqpoint{0.962697in}{0.992778in}}%
\pgfpathcurveto{\pgfqpoint{0.954883in}{0.984964in}}{\pgfqpoint{0.950493in}{0.974365in}}{\pgfqpoint{0.950493in}{0.963315in}}%
\pgfpathcurveto{\pgfqpoint{0.950493in}{0.952265in}}{\pgfqpoint{0.954883in}{0.941666in}}{\pgfqpoint{0.962697in}{0.933852in}}%
\pgfpathcurveto{\pgfqpoint{0.970510in}{0.926038in}}{\pgfqpoint{0.981110in}{0.921648in}}{\pgfqpoint{0.992160in}{0.921648in}}%
\pgfpathclose%
\pgfusepath{stroke,fill}%
\end{pgfscope}%
\begin{pgfscope}%
\pgfpathrectangle{\pgfqpoint{0.481978in}{0.331635in}}{\pgfqpoint{4.960000in}{3.696000in}}%
\pgfusepath{clip}%
\pgfsetbuttcap%
\pgfsetroundjoin%
\definecolor{currentfill}{rgb}{1.000000,0.705882,0.509804}%
\pgfsetfillcolor{currentfill}%
\pgfsetlinewidth{0.481800pt}%
\definecolor{currentstroke}{rgb}{1.000000,1.000000,1.000000}%
\pgfsetstrokecolor{currentstroke}%
\pgfsetdash{}{0pt}%
\pgfpathmoveto{\pgfqpoint{1.940136in}{1.604951in}}%
\pgfpathcurveto{\pgfqpoint{1.951186in}{1.604951in}}{\pgfqpoint{1.961785in}{1.609341in}}{\pgfqpoint{1.969599in}{1.617155in}}%
\pgfpathcurveto{\pgfqpoint{1.977412in}{1.624969in}}{\pgfqpoint{1.981803in}{1.635568in}}{\pgfqpoint{1.981803in}{1.646618in}}%
\pgfpathcurveto{\pgfqpoint{1.981803in}{1.657668in}}{\pgfqpoint{1.977412in}{1.668267in}}{\pgfqpoint{1.969599in}{1.676081in}}%
\pgfpathcurveto{\pgfqpoint{1.961785in}{1.683894in}}{\pgfqpoint{1.951186in}{1.688284in}}{\pgfqpoint{1.940136in}{1.688284in}}%
\pgfpathcurveto{\pgfqpoint{1.929086in}{1.688284in}}{\pgfqpoint{1.918487in}{1.683894in}}{\pgfqpoint{1.910673in}{1.676081in}}%
\pgfpathcurveto{\pgfqpoint{1.902860in}{1.668267in}}{\pgfqpoint{1.898469in}{1.657668in}}{\pgfqpoint{1.898469in}{1.646618in}}%
\pgfpathcurveto{\pgfqpoint{1.898469in}{1.635568in}}{\pgfqpoint{1.902860in}{1.624969in}}{\pgfqpoint{1.910673in}{1.617155in}}%
\pgfpathcurveto{\pgfqpoint{1.918487in}{1.609341in}}{\pgfqpoint{1.929086in}{1.604951in}}{\pgfqpoint{1.940136in}{1.604951in}}%
\pgfpathclose%
\pgfusepath{stroke,fill}%
\end{pgfscope}%
\begin{pgfscope}%
\pgfpathrectangle{\pgfqpoint{0.481978in}{0.331635in}}{\pgfqpoint{4.960000in}{3.696000in}}%
\pgfusepath{clip}%
\pgfsetbuttcap%
\pgfsetroundjoin%
\definecolor{currentfill}{rgb}{1.000000,0.705882,0.509804}%
\pgfsetfillcolor{currentfill}%
\pgfsetlinewidth{0.481800pt}%
\definecolor{currentstroke}{rgb}{1.000000,1.000000,1.000000}%
\pgfsetstrokecolor{currentstroke}%
\pgfsetdash{}{0pt}%
\pgfpathmoveto{\pgfqpoint{2.536499in}{0.577238in}}%
\pgfpathcurveto{\pgfqpoint{2.547549in}{0.577238in}}{\pgfqpoint{2.558148in}{0.581628in}}{\pgfqpoint{2.565961in}{0.589442in}}%
\pgfpathcurveto{\pgfqpoint{2.573775in}{0.597255in}}{\pgfqpoint{2.578165in}{0.607854in}}{\pgfqpoint{2.578165in}{0.618904in}}%
\pgfpathcurveto{\pgfqpoint{2.578165in}{0.629955in}}{\pgfqpoint{2.573775in}{0.640554in}}{\pgfqpoint{2.565961in}{0.648367in}}%
\pgfpathcurveto{\pgfqpoint{2.558148in}{0.656181in}}{\pgfqpoint{2.547549in}{0.660571in}}{\pgfqpoint{2.536499in}{0.660571in}}%
\pgfpathcurveto{\pgfqpoint{2.525448in}{0.660571in}}{\pgfqpoint{2.514849in}{0.656181in}}{\pgfqpoint{2.507036in}{0.648367in}}%
\pgfpathcurveto{\pgfqpoint{2.499222in}{0.640554in}}{\pgfqpoint{2.494832in}{0.629955in}}{\pgfqpoint{2.494832in}{0.618904in}}%
\pgfpathcurveto{\pgfqpoint{2.494832in}{0.607854in}}{\pgfqpoint{2.499222in}{0.597255in}}{\pgfqpoint{2.507036in}{0.589442in}}%
\pgfpathcurveto{\pgfqpoint{2.514849in}{0.581628in}}{\pgfqpoint{2.525448in}{0.577238in}}{\pgfqpoint{2.536499in}{0.577238in}}%
\pgfpathclose%
\pgfusepath{stroke,fill}%
\end{pgfscope}%
\begin{pgfscope}%
\pgfpathrectangle{\pgfqpoint{0.481978in}{0.331635in}}{\pgfqpoint{4.960000in}{3.696000in}}%
\pgfusepath{clip}%
\pgfsetbuttcap%
\pgfsetroundjoin%
\definecolor{currentfill}{rgb}{1.000000,0.705882,0.509804}%
\pgfsetfillcolor{currentfill}%
\pgfsetlinewidth{0.481800pt}%
\definecolor{currentstroke}{rgb}{1.000000,1.000000,1.000000}%
\pgfsetstrokecolor{currentstroke}%
\pgfsetdash{}{0pt}%
\pgfpathmoveto{\pgfqpoint{3.676022in}{1.324735in}}%
\pgfpathcurveto{\pgfqpoint{3.687072in}{1.324735in}}{\pgfqpoint{3.697671in}{1.329126in}}{\pgfqpoint{3.705485in}{1.336939in}}%
\pgfpathcurveto{\pgfqpoint{3.713298in}{1.344753in}}{\pgfqpoint{3.717689in}{1.355352in}}{\pgfqpoint{3.717689in}{1.366402in}}%
\pgfpathcurveto{\pgfqpoint{3.717689in}{1.377452in}}{\pgfqpoint{3.713298in}{1.388051in}}{\pgfqpoint{3.705485in}{1.395865in}}%
\pgfpathcurveto{\pgfqpoint{3.697671in}{1.403678in}}{\pgfqpoint{3.687072in}{1.408069in}}{\pgfqpoint{3.676022in}{1.408069in}}%
\pgfpathcurveto{\pgfqpoint{3.664972in}{1.408069in}}{\pgfqpoint{3.654373in}{1.403678in}}{\pgfqpoint{3.646559in}{1.395865in}}%
\pgfpathcurveto{\pgfqpoint{3.638745in}{1.388051in}}{\pgfqpoint{3.634355in}{1.377452in}}{\pgfqpoint{3.634355in}{1.366402in}}%
\pgfpathcurveto{\pgfqpoint{3.634355in}{1.355352in}}{\pgfqpoint{3.638745in}{1.344753in}}{\pgfqpoint{3.646559in}{1.336939in}}%
\pgfpathcurveto{\pgfqpoint{3.654373in}{1.329126in}}{\pgfqpoint{3.664972in}{1.324735in}}{\pgfqpoint{3.676022in}{1.324735in}}%
\pgfpathclose%
\pgfusepath{stroke,fill}%
\end{pgfscope}%
\begin{pgfscope}%
\pgfpathrectangle{\pgfqpoint{0.481978in}{0.331635in}}{\pgfqpoint{4.960000in}{3.696000in}}%
\pgfusepath{clip}%
\pgfsetbuttcap%
\pgfsetroundjoin%
\definecolor{currentfill}{rgb}{1.000000,0.705882,0.509804}%
\pgfsetfillcolor{currentfill}%
\pgfsetlinewidth{0.481800pt}%
\definecolor{currentstroke}{rgb}{1.000000,1.000000,1.000000}%
\pgfsetstrokecolor{currentstroke}%
\pgfsetdash{}{0pt}%
\pgfpathmoveto{\pgfqpoint{1.888129in}{1.984023in}}%
\pgfpathcurveto{\pgfqpoint{1.899179in}{1.984023in}}{\pgfqpoint{1.909778in}{1.988413in}}{\pgfqpoint{1.917592in}{1.996227in}}%
\pgfpathcurveto{\pgfqpoint{1.925406in}{2.004040in}}{\pgfqpoint{1.929796in}{2.014639in}}{\pgfqpoint{1.929796in}{2.025689in}}%
\pgfpathcurveto{\pgfqpoint{1.929796in}{2.036739in}}{\pgfqpoint{1.925406in}{2.047338in}}{\pgfqpoint{1.917592in}{2.055152in}}%
\pgfpathcurveto{\pgfqpoint{1.909778in}{2.062966in}}{\pgfqpoint{1.899179in}{2.067356in}}{\pgfqpoint{1.888129in}{2.067356in}}%
\pgfpathcurveto{\pgfqpoint{1.877079in}{2.067356in}}{\pgfqpoint{1.866480in}{2.062966in}}{\pgfqpoint{1.858666in}{2.055152in}}%
\pgfpathcurveto{\pgfqpoint{1.850853in}{2.047338in}}{\pgfqpoint{1.846463in}{2.036739in}}{\pgfqpoint{1.846463in}{2.025689in}}%
\pgfpathcurveto{\pgfqpoint{1.846463in}{2.014639in}}{\pgfqpoint{1.850853in}{2.004040in}}{\pgfqpoint{1.858666in}{1.996227in}}%
\pgfpathcurveto{\pgfqpoint{1.866480in}{1.988413in}}{\pgfqpoint{1.877079in}{1.984023in}}{\pgfqpoint{1.888129in}{1.984023in}}%
\pgfpathclose%
\pgfusepath{stroke,fill}%
\end{pgfscope}%
\begin{pgfscope}%
\pgfpathrectangle{\pgfqpoint{0.481978in}{0.331635in}}{\pgfqpoint{4.960000in}{3.696000in}}%
\pgfusepath{clip}%
\pgfsetbuttcap%
\pgfsetroundjoin%
\definecolor{currentfill}{rgb}{1.000000,0.705882,0.509804}%
\pgfsetfillcolor{currentfill}%
\pgfsetlinewidth{0.481800pt}%
\definecolor{currentstroke}{rgb}{1.000000,1.000000,1.000000}%
\pgfsetstrokecolor{currentstroke}%
\pgfsetdash{}{0pt}%
\pgfpathmoveto{\pgfqpoint{2.521771in}{1.475041in}}%
\pgfpathcurveto{\pgfqpoint{2.532821in}{1.475041in}}{\pgfqpoint{2.543420in}{1.479431in}}{\pgfqpoint{2.551234in}{1.487244in}}%
\pgfpathcurveto{\pgfqpoint{2.559048in}{1.495058in}}{\pgfqpoint{2.563438in}{1.505657in}}{\pgfqpoint{2.563438in}{1.516707in}}%
\pgfpathcurveto{\pgfqpoint{2.563438in}{1.527757in}}{\pgfqpoint{2.559048in}{1.538356in}}{\pgfqpoint{2.551234in}{1.546170in}}%
\pgfpathcurveto{\pgfqpoint{2.543420in}{1.553984in}}{\pgfqpoint{2.532821in}{1.558374in}}{\pgfqpoint{2.521771in}{1.558374in}}%
\pgfpathcurveto{\pgfqpoint{2.510721in}{1.558374in}}{\pgfqpoint{2.500122in}{1.553984in}}{\pgfqpoint{2.492309in}{1.546170in}}%
\pgfpathcurveto{\pgfqpoint{2.484495in}{1.538356in}}{\pgfqpoint{2.480105in}{1.527757in}}{\pgfqpoint{2.480105in}{1.516707in}}%
\pgfpathcurveto{\pgfqpoint{2.480105in}{1.505657in}}{\pgfqpoint{2.484495in}{1.495058in}}{\pgfqpoint{2.492309in}{1.487244in}}%
\pgfpathcurveto{\pgfqpoint{2.500122in}{1.479431in}}{\pgfqpoint{2.510721in}{1.475041in}}{\pgfqpoint{2.521771in}{1.475041in}}%
\pgfpathclose%
\pgfusepath{stroke,fill}%
\end{pgfscope}%
\begin{pgfscope}%
\pgfpathrectangle{\pgfqpoint{0.481978in}{0.331635in}}{\pgfqpoint{4.960000in}{3.696000in}}%
\pgfusepath{clip}%
\pgfsetbuttcap%
\pgfsetroundjoin%
\definecolor{currentfill}{rgb}{1.000000,0.705882,0.509804}%
\pgfsetfillcolor{currentfill}%
\pgfsetlinewidth{0.481800pt}%
\definecolor{currentstroke}{rgb}{1.000000,1.000000,1.000000}%
\pgfsetstrokecolor{currentstroke}%
\pgfsetdash{}{0pt}%
\pgfpathmoveto{\pgfqpoint{3.803581in}{2.114281in}}%
\pgfpathcurveto{\pgfqpoint{3.814631in}{2.114281in}}{\pgfqpoint{3.825230in}{2.118671in}}{\pgfqpoint{3.833043in}{2.126485in}}%
\pgfpathcurveto{\pgfqpoint{3.840857in}{2.134298in}}{\pgfqpoint{3.845247in}{2.144897in}}{\pgfqpoint{3.845247in}{2.155947in}}%
\pgfpathcurveto{\pgfqpoint{3.845247in}{2.166998in}}{\pgfqpoint{3.840857in}{2.177597in}}{\pgfqpoint{3.833043in}{2.185410in}}%
\pgfpathcurveto{\pgfqpoint{3.825230in}{2.193224in}}{\pgfqpoint{3.814631in}{2.197614in}}{\pgfqpoint{3.803581in}{2.197614in}}%
\pgfpathcurveto{\pgfqpoint{3.792531in}{2.197614in}}{\pgfqpoint{3.781931in}{2.193224in}}{\pgfqpoint{3.774118in}{2.185410in}}%
\pgfpathcurveto{\pgfqpoint{3.766304in}{2.177597in}}{\pgfqpoint{3.761914in}{2.166998in}}{\pgfqpoint{3.761914in}{2.155947in}}%
\pgfpathcurveto{\pgfqpoint{3.761914in}{2.144897in}}{\pgfqpoint{3.766304in}{2.134298in}}{\pgfqpoint{3.774118in}{2.126485in}}%
\pgfpathcurveto{\pgfqpoint{3.781931in}{2.118671in}}{\pgfqpoint{3.792531in}{2.114281in}}{\pgfqpoint{3.803581in}{2.114281in}}%
\pgfpathclose%
\pgfusepath{stroke,fill}%
\end{pgfscope}%
\begin{pgfscope}%
\pgfpathrectangle{\pgfqpoint{0.481978in}{0.331635in}}{\pgfqpoint{4.960000in}{3.696000in}}%
\pgfusepath{clip}%
\pgfsetbuttcap%
\pgfsetroundjoin%
\definecolor{currentfill}{rgb}{1.000000,0.705882,0.509804}%
\pgfsetfillcolor{currentfill}%
\pgfsetlinewidth{0.481800pt}%
\definecolor{currentstroke}{rgb}{1.000000,1.000000,1.000000}%
\pgfsetstrokecolor{currentstroke}%
\pgfsetdash{}{0pt}%
\pgfpathmoveto{\pgfqpoint{4.055052in}{2.041342in}}%
\pgfpathcurveto{\pgfqpoint{4.066102in}{2.041342in}}{\pgfqpoint{4.076701in}{2.045732in}}{\pgfqpoint{4.084515in}{2.053546in}}%
\pgfpathcurveto{\pgfqpoint{4.092328in}{2.061360in}}{\pgfqpoint{4.096719in}{2.071959in}}{\pgfqpoint{4.096719in}{2.083009in}}%
\pgfpathcurveto{\pgfqpoint{4.096719in}{2.094059in}}{\pgfqpoint{4.092328in}{2.104658in}}{\pgfqpoint{4.084515in}{2.112472in}}%
\pgfpathcurveto{\pgfqpoint{4.076701in}{2.120285in}}{\pgfqpoint{4.066102in}{2.124675in}}{\pgfqpoint{4.055052in}{2.124675in}}%
\pgfpathcurveto{\pgfqpoint{4.044002in}{2.124675in}}{\pgfqpoint{4.033403in}{2.120285in}}{\pgfqpoint{4.025589in}{2.112472in}}%
\pgfpathcurveto{\pgfqpoint{4.017776in}{2.104658in}}{\pgfqpoint{4.013385in}{2.094059in}}{\pgfqpoint{4.013385in}{2.083009in}}%
\pgfpathcurveto{\pgfqpoint{4.013385in}{2.071959in}}{\pgfqpoint{4.017776in}{2.061360in}}{\pgfqpoint{4.025589in}{2.053546in}}%
\pgfpathcurveto{\pgfqpoint{4.033403in}{2.045732in}}{\pgfqpoint{4.044002in}{2.041342in}}{\pgfqpoint{4.055052in}{2.041342in}}%
\pgfpathclose%
\pgfusepath{stroke,fill}%
\end{pgfscope}%
\begin{pgfscope}%
\pgfpathrectangle{\pgfqpoint{0.481978in}{0.331635in}}{\pgfqpoint{4.960000in}{3.696000in}}%
\pgfusepath{clip}%
\pgfsetbuttcap%
\pgfsetroundjoin%
\definecolor{currentfill}{rgb}{1.000000,0.705882,0.509804}%
\pgfsetfillcolor{currentfill}%
\pgfsetlinewidth{0.481800pt}%
\definecolor{currentstroke}{rgb}{1.000000,1.000000,1.000000}%
\pgfsetstrokecolor{currentstroke}%
\pgfsetdash{}{0pt}%
\pgfpathmoveto{\pgfqpoint{4.049936in}{2.132515in}}%
\pgfpathcurveto{\pgfqpoint{4.060986in}{2.132515in}}{\pgfqpoint{4.071585in}{2.136905in}}{\pgfqpoint{4.079399in}{2.144718in}}%
\pgfpathcurveto{\pgfqpoint{4.087213in}{2.152532in}}{\pgfqpoint{4.091603in}{2.163131in}}{\pgfqpoint{4.091603in}{2.174181in}}%
\pgfpathcurveto{\pgfqpoint{4.091603in}{2.185231in}}{\pgfqpoint{4.087213in}{2.195830in}}{\pgfqpoint{4.079399in}{2.203644in}}%
\pgfpathcurveto{\pgfqpoint{4.071585in}{2.211458in}}{\pgfqpoint{4.060986in}{2.215848in}}{\pgfqpoint{4.049936in}{2.215848in}}%
\pgfpathcurveto{\pgfqpoint{4.038886in}{2.215848in}}{\pgfqpoint{4.028287in}{2.211458in}}{\pgfqpoint{4.020473in}{2.203644in}}%
\pgfpathcurveto{\pgfqpoint{4.012660in}{2.195830in}}{\pgfqpoint{4.008269in}{2.185231in}}{\pgfqpoint{4.008269in}{2.174181in}}%
\pgfpathcurveto{\pgfqpoint{4.008269in}{2.163131in}}{\pgfqpoint{4.012660in}{2.152532in}}{\pgfqpoint{4.020473in}{2.144718in}}%
\pgfpathcurveto{\pgfqpoint{4.028287in}{2.136905in}}{\pgfqpoint{4.038886in}{2.132515in}}{\pgfqpoint{4.049936in}{2.132515in}}%
\pgfpathclose%
\pgfusepath{stroke,fill}%
\end{pgfscope}%
\begin{pgfscope}%
\pgfpathrectangle{\pgfqpoint{0.481978in}{0.331635in}}{\pgfqpoint{4.960000in}{3.696000in}}%
\pgfusepath{clip}%
\pgfsetbuttcap%
\pgfsetroundjoin%
\definecolor{currentfill}{rgb}{1.000000,0.705882,0.509804}%
\pgfsetfillcolor{currentfill}%
\pgfsetlinewidth{0.481800pt}%
\definecolor{currentstroke}{rgb}{1.000000,1.000000,1.000000}%
\pgfsetstrokecolor{currentstroke}%
\pgfsetdash{}{0pt}%
\pgfpathmoveto{\pgfqpoint{3.796579in}{1.953593in}}%
\pgfpathcurveto{\pgfqpoint{3.807629in}{1.953593in}}{\pgfqpoint{3.818228in}{1.957983in}}{\pgfqpoint{3.826041in}{1.965797in}}%
\pgfpathcurveto{\pgfqpoint{3.833855in}{1.973610in}}{\pgfqpoint{3.838245in}{1.984209in}}{\pgfqpoint{3.838245in}{1.995259in}}%
\pgfpathcurveto{\pgfqpoint{3.838245in}{2.006310in}}{\pgfqpoint{3.833855in}{2.016909in}}{\pgfqpoint{3.826041in}{2.024722in}}%
\pgfpathcurveto{\pgfqpoint{3.818228in}{2.032536in}}{\pgfqpoint{3.807629in}{2.036926in}}{\pgfqpoint{3.796579in}{2.036926in}}%
\pgfpathcurveto{\pgfqpoint{3.785528in}{2.036926in}}{\pgfqpoint{3.774929in}{2.032536in}}{\pgfqpoint{3.767116in}{2.024722in}}%
\pgfpathcurveto{\pgfqpoint{3.759302in}{2.016909in}}{\pgfqpoint{3.754912in}{2.006310in}}{\pgfqpoint{3.754912in}{1.995259in}}%
\pgfpathcurveto{\pgfqpoint{3.754912in}{1.984209in}}{\pgfqpoint{3.759302in}{1.973610in}}{\pgfqpoint{3.767116in}{1.965797in}}%
\pgfpathcurveto{\pgfqpoint{3.774929in}{1.957983in}}{\pgfqpoint{3.785528in}{1.953593in}}{\pgfqpoint{3.796579in}{1.953593in}}%
\pgfpathclose%
\pgfusepath{stroke,fill}%
\end{pgfscope}%
\begin{pgfscope}%
\pgfpathrectangle{\pgfqpoint{0.481978in}{0.331635in}}{\pgfqpoint{4.960000in}{3.696000in}}%
\pgfusepath{clip}%
\pgfsetbuttcap%
\pgfsetroundjoin%
\definecolor{currentfill}{rgb}{1.000000,0.705882,0.509804}%
\pgfsetfillcolor{currentfill}%
\pgfsetlinewidth{0.481800pt}%
\definecolor{currentstroke}{rgb}{1.000000,1.000000,1.000000}%
\pgfsetstrokecolor{currentstroke}%
\pgfsetdash{}{0pt}%
\pgfpathmoveto{\pgfqpoint{1.651432in}{1.012515in}}%
\pgfpathcurveto{\pgfqpoint{1.662482in}{1.012515in}}{\pgfqpoint{1.673081in}{1.016905in}}{\pgfqpoint{1.680895in}{1.024719in}}%
\pgfpathcurveto{\pgfqpoint{1.688708in}{1.032533in}}{\pgfqpoint{1.693098in}{1.043132in}}{\pgfqpoint{1.693098in}{1.054182in}}%
\pgfpathcurveto{\pgfqpoint{1.693098in}{1.065232in}}{\pgfqpoint{1.688708in}{1.075831in}}{\pgfqpoint{1.680895in}{1.083645in}}%
\pgfpathcurveto{\pgfqpoint{1.673081in}{1.091458in}}{\pgfqpoint{1.662482in}{1.095848in}}{\pgfqpoint{1.651432in}{1.095848in}}%
\pgfpathcurveto{\pgfqpoint{1.640382in}{1.095848in}}{\pgfqpoint{1.629783in}{1.091458in}}{\pgfqpoint{1.621969in}{1.083645in}}%
\pgfpathcurveto{\pgfqpoint{1.614155in}{1.075831in}}{\pgfqpoint{1.609765in}{1.065232in}}{\pgfqpoint{1.609765in}{1.054182in}}%
\pgfpathcurveto{\pgfqpoint{1.609765in}{1.043132in}}{\pgfqpoint{1.614155in}{1.032533in}}{\pgfqpoint{1.621969in}{1.024719in}}%
\pgfpathcurveto{\pgfqpoint{1.629783in}{1.016905in}}{\pgfqpoint{1.640382in}{1.012515in}}{\pgfqpoint{1.651432in}{1.012515in}}%
\pgfpathclose%
\pgfusepath{stroke,fill}%
\end{pgfscope}%
\begin{pgfscope}%
\pgfpathrectangle{\pgfqpoint{0.481978in}{0.331635in}}{\pgfqpoint{4.960000in}{3.696000in}}%
\pgfusepath{clip}%
\pgfsetbuttcap%
\pgfsetroundjoin%
\definecolor{currentfill}{rgb}{1.000000,0.705882,0.509804}%
\pgfsetfillcolor{currentfill}%
\pgfsetlinewidth{0.481800pt}%
\definecolor{currentstroke}{rgb}{1.000000,1.000000,1.000000}%
\pgfsetstrokecolor{currentstroke}%
\pgfsetdash{}{0pt}%
\pgfpathmoveto{\pgfqpoint{2.047564in}{1.063962in}}%
\pgfpathcurveto{\pgfqpoint{2.058614in}{1.063962in}}{\pgfqpoint{2.069213in}{1.068352in}}{\pgfqpoint{2.077026in}{1.076166in}}%
\pgfpathcurveto{\pgfqpoint{2.084840in}{1.083979in}}{\pgfqpoint{2.089230in}{1.094578in}}{\pgfqpoint{2.089230in}{1.105628in}}%
\pgfpathcurveto{\pgfqpoint{2.089230in}{1.116679in}}{\pgfqpoint{2.084840in}{1.127278in}}{\pgfqpoint{2.077026in}{1.135091in}}%
\pgfpathcurveto{\pgfqpoint{2.069213in}{1.142905in}}{\pgfqpoint{2.058614in}{1.147295in}}{\pgfqpoint{2.047564in}{1.147295in}}%
\pgfpathcurveto{\pgfqpoint{2.036513in}{1.147295in}}{\pgfqpoint{2.025914in}{1.142905in}}{\pgfqpoint{2.018101in}{1.135091in}}%
\pgfpathcurveto{\pgfqpoint{2.010287in}{1.127278in}}{\pgfqpoint{2.005897in}{1.116679in}}{\pgfqpoint{2.005897in}{1.105628in}}%
\pgfpathcurveto{\pgfqpoint{2.005897in}{1.094578in}}{\pgfqpoint{2.010287in}{1.083979in}}{\pgfqpoint{2.018101in}{1.076166in}}%
\pgfpathcurveto{\pgfqpoint{2.025914in}{1.068352in}}{\pgfqpoint{2.036513in}{1.063962in}}{\pgfqpoint{2.047564in}{1.063962in}}%
\pgfpathclose%
\pgfusepath{stroke,fill}%
\end{pgfscope}%
\begin{pgfscope}%
\pgfpathrectangle{\pgfqpoint{0.481978in}{0.331635in}}{\pgfqpoint{4.960000in}{3.696000in}}%
\pgfusepath{clip}%
\pgfsetbuttcap%
\pgfsetroundjoin%
\definecolor{currentfill}{rgb}{1.000000,0.705882,0.509804}%
\pgfsetfillcolor{currentfill}%
\pgfsetlinewidth{0.481800pt}%
\definecolor{currentstroke}{rgb}{1.000000,1.000000,1.000000}%
\pgfsetstrokecolor{currentstroke}%
\pgfsetdash{}{0pt}%
\pgfpathmoveto{\pgfqpoint{3.474036in}{1.680290in}}%
\pgfpathcurveto{\pgfqpoint{3.485086in}{1.680290in}}{\pgfqpoint{3.495686in}{1.684680in}}{\pgfqpoint{3.503499in}{1.692494in}}%
\pgfpathcurveto{\pgfqpoint{3.511313in}{1.700308in}}{\pgfqpoint{3.515703in}{1.710907in}}{\pgfqpoint{3.515703in}{1.721957in}}%
\pgfpathcurveto{\pgfqpoint{3.515703in}{1.733007in}}{\pgfqpoint{3.511313in}{1.743606in}}{\pgfqpoint{3.503499in}{1.751420in}}%
\pgfpathcurveto{\pgfqpoint{3.495686in}{1.759233in}}{\pgfqpoint{3.485086in}{1.763624in}}{\pgfqpoint{3.474036in}{1.763624in}}%
\pgfpathcurveto{\pgfqpoint{3.462986in}{1.763624in}}{\pgfqpoint{3.452387in}{1.759233in}}{\pgfqpoint{3.444574in}{1.751420in}}%
\pgfpathcurveto{\pgfqpoint{3.436760in}{1.743606in}}{\pgfqpoint{3.432370in}{1.733007in}}{\pgfqpoint{3.432370in}{1.721957in}}%
\pgfpathcurveto{\pgfqpoint{3.432370in}{1.710907in}}{\pgfqpoint{3.436760in}{1.700308in}}{\pgfqpoint{3.444574in}{1.692494in}}%
\pgfpathcurveto{\pgfqpoint{3.452387in}{1.684680in}}{\pgfqpoint{3.462986in}{1.680290in}}{\pgfqpoint{3.474036in}{1.680290in}}%
\pgfpathclose%
\pgfusepath{stroke,fill}%
\end{pgfscope}%
\begin{pgfscope}%
\pgfpathrectangle{\pgfqpoint{0.481978in}{0.331635in}}{\pgfqpoint{4.960000in}{3.696000in}}%
\pgfusepath{clip}%
\pgfsetbuttcap%
\pgfsetroundjoin%
\definecolor{currentfill}{rgb}{1.000000,0.705882,0.509804}%
\pgfsetfillcolor{currentfill}%
\pgfsetlinewidth{0.481800pt}%
\definecolor{currentstroke}{rgb}{1.000000,1.000000,1.000000}%
\pgfsetstrokecolor{currentstroke}%
\pgfsetdash{}{0pt}%
\pgfpathmoveto{\pgfqpoint{3.381704in}{0.698317in}}%
\pgfpathcurveto{\pgfqpoint{3.392754in}{0.698317in}}{\pgfqpoint{3.403353in}{0.702708in}}{\pgfqpoint{3.411167in}{0.710521in}}%
\pgfpathcurveto{\pgfqpoint{3.418980in}{0.718335in}}{\pgfqpoint{3.423371in}{0.728934in}}{\pgfqpoint{3.423371in}{0.739984in}}%
\pgfpathcurveto{\pgfqpoint{3.423371in}{0.751034in}}{\pgfqpoint{3.418980in}{0.761633in}}{\pgfqpoint{3.411167in}{0.769447in}}%
\pgfpathcurveto{\pgfqpoint{3.403353in}{0.777261in}}{\pgfqpoint{3.392754in}{0.781651in}}{\pgfqpoint{3.381704in}{0.781651in}}%
\pgfpathcurveto{\pgfqpoint{3.370654in}{0.781651in}}{\pgfqpoint{3.360055in}{0.777261in}}{\pgfqpoint{3.352241in}{0.769447in}}%
\pgfpathcurveto{\pgfqpoint{3.344428in}{0.761633in}}{\pgfqpoint{3.340037in}{0.751034in}}{\pgfqpoint{3.340037in}{0.739984in}}%
\pgfpathcurveto{\pgfqpoint{3.340037in}{0.728934in}}{\pgfqpoint{3.344428in}{0.718335in}}{\pgfqpoint{3.352241in}{0.710521in}}%
\pgfpathcurveto{\pgfqpoint{3.360055in}{0.702708in}}{\pgfqpoint{3.370654in}{0.698317in}}{\pgfqpoint{3.381704in}{0.698317in}}%
\pgfpathclose%
\pgfusepath{stroke,fill}%
\end{pgfscope}%
\begin{pgfscope}%
\pgfpathrectangle{\pgfqpoint{0.481978in}{0.331635in}}{\pgfqpoint{4.960000in}{3.696000in}}%
\pgfusepath{clip}%
\pgfsetbuttcap%
\pgfsetroundjoin%
\definecolor{currentfill}{rgb}{1.000000,0.705882,0.509804}%
\pgfsetfillcolor{currentfill}%
\pgfsetlinewidth{0.481800pt}%
\definecolor{currentstroke}{rgb}{1.000000,1.000000,1.000000}%
\pgfsetstrokecolor{currentstroke}%
\pgfsetdash{}{0pt}%
\pgfpathmoveto{\pgfqpoint{4.262882in}{2.216937in}}%
\pgfpathcurveto{\pgfqpoint{4.273932in}{2.216937in}}{\pgfqpoint{4.284531in}{2.221327in}}{\pgfqpoint{4.292344in}{2.229141in}}%
\pgfpathcurveto{\pgfqpoint{4.300158in}{2.236954in}}{\pgfqpoint{4.304548in}{2.247553in}}{\pgfqpoint{4.304548in}{2.258603in}}%
\pgfpathcurveto{\pgfqpoint{4.304548in}{2.269654in}}{\pgfqpoint{4.300158in}{2.280253in}}{\pgfqpoint{4.292344in}{2.288066in}}%
\pgfpathcurveto{\pgfqpoint{4.284531in}{2.295880in}}{\pgfqpoint{4.273932in}{2.300270in}}{\pgfqpoint{4.262882in}{2.300270in}}%
\pgfpathcurveto{\pgfqpoint{4.251832in}{2.300270in}}{\pgfqpoint{4.241232in}{2.295880in}}{\pgfqpoint{4.233419in}{2.288066in}}%
\pgfpathcurveto{\pgfqpoint{4.225605in}{2.280253in}}{\pgfqpoint{4.221215in}{2.269654in}}{\pgfqpoint{4.221215in}{2.258603in}}%
\pgfpathcurveto{\pgfqpoint{4.221215in}{2.247553in}}{\pgfqpoint{4.225605in}{2.236954in}}{\pgfqpoint{4.233419in}{2.229141in}}%
\pgfpathcurveto{\pgfqpoint{4.241232in}{2.221327in}}{\pgfqpoint{4.251832in}{2.216937in}}{\pgfqpoint{4.262882in}{2.216937in}}%
\pgfpathclose%
\pgfusepath{stroke,fill}%
\end{pgfscope}%
\begin{pgfscope}%
\pgfpathrectangle{\pgfqpoint{0.481978in}{0.331635in}}{\pgfqpoint{4.960000in}{3.696000in}}%
\pgfusepath{clip}%
\pgfsetbuttcap%
\pgfsetroundjoin%
\definecolor{currentfill}{rgb}{1.000000,0.705882,0.509804}%
\pgfsetfillcolor{currentfill}%
\pgfsetlinewidth{0.481800pt}%
\definecolor{currentstroke}{rgb}{1.000000,1.000000,1.000000}%
\pgfsetstrokecolor{currentstroke}%
\pgfsetdash{}{0pt}%
\pgfpathmoveto{\pgfqpoint{3.708000in}{1.865949in}}%
\pgfpathcurveto{\pgfqpoint{3.719050in}{1.865949in}}{\pgfqpoint{3.729649in}{1.870339in}}{\pgfqpoint{3.737462in}{1.878153in}}%
\pgfpathcurveto{\pgfqpoint{3.745276in}{1.885966in}}{\pgfqpoint{3.749666in}{1.896565in}}{\pgfqpoint{3.749666in}{1.907615in}}%
\pgfpathcurveto{\pgfqpoint{3.749666in}{1.918666in}}{\pgfqpoint{3.745276in}{1.929265in}}{\pgfqpoint{3.737462in}{1.937078in}}%
\pgfpathcurveto{\pgfqpoint{3.729649in}{1.944892in}}{\pgfqpoint{3.719050in}{1.949282in}}{\pgfqpoint{3.708000in}{1.949282in}}%
\pgfpathcurveto{\pgfqpoint{3.696950in}{1.949282in}}{\pgfqpoint{3.686351in}{1.944892in}}{\pgfqpoint{3.678537in}{1.937078in}}%
\pgfpathcurveto{\pgfqpoint{3.670723in}{1.929265in}}{\pgfqpoint{3.666333in}{1.918666in}}{\pgfqpoint{3.666333in}{1.907615in}}%
\pgfpathcurveto{\pgfqpoint{3.666333in}{1.896565in}}{\pgfqpoint{3.670723in}{1.885966in}}{\pgfqpoint{3.678537in}{1.878153in}}%
\pgfpathcurveto{\pgfqpoint{3.686351in}{1.870339in}}{\pgfqpoint{3.696950in}{1.865949in}}{\pgfqpoint{3.708000in}{1.865949in}}%
\pgfpathclose%
\pgfusepath{stroke,fill}%
\end{pgfscope}%
\begin{pgfscope}%
\pgfpathrectangle{\pgfqpoint{0.481978in}{0.331635in}}{\pgfqpoint{4.960000in}{3.696000in}}%
\pgfusepath{clip}%
\pgfsetbuttcap%
\pgfsetroundjoin%
\definecolor{currentfill}{rgb}{1.000000,0.705882,0.509804}%
\pgfsetfillcolor{currentfill}%
\pgfsetlinewidth{0.481800pt}%
\definecolor{currentstroke}{rgb}{1.000000,1.000000,1.000000}%
\pgfsetstrokecolor{currentstroke}%
\pgfsetdash{}{0pt}%
\pgfpathmoveto{\pgfqpoint{1.240808in}{0.974487in}}%
\pgfpathcurveto{\pgfqpoint{1.251859in}{0.974487in}}{\pgfqpoint{1.262458in}{0.978877in}}{\pgfqpoint{1.270271in}{0.986690in}}%
\pgfpathcurveto{\pgfqpoint{1.278085in}{0.994504in}}{\pgfqpoint{1.282475in}{1.005103in}}{\pgfqpoint{1.282475in}{1.016153in}}%
\pgfpathcurveto{\pgfqpoint{1.282475in}{1.027203in}}{\pgfqpoint{1.278085in}{1.037802in}}{\pgfqpoint{1.270271in}{1.045616in}}%
\pgfpathcurveto{\pgfqpoint{1.262458in}{1.053430in}}{\pgfqpoint{1.251859in}{1.057820in}}{\pgfqpoint{1.240808in}{1.057820in}}%
\pgfpathcurveto{\pgfqpoint{1.229758in}{1.057820in}}{\pgfqpoint{1.219159in}{1.053430in}}{\pgfqpoint{1.211346in}{1.045616in}}%
\pgfpathcurveto{\pgfqpoint{1.203532in}{1.037802in}}{\pgfqpoint{1.199142in}{1.027203in}}{\pgfqpoint{1.199142in}{1.016153in}}%
\pgfpathcurveto{\pgfqpoint{1.199142in}{1.005103in}}{\pgfqpoint{1.203532in}{0.994504in}}{\pgfqpoint{1.211346in}{0.986690in}}%
\pgfpathcurveto{\pgfqpoint{1.219159in}{0.978877in}}{\pgfqpoint{1.229758in}{0.974487in}}{\pgfqpoint{1.240808in}{0.974487in}}%
\pgfpathclose%
\pgfusepath{stroke,fill}%
\end{pgfscope}%
\begin{pgfscope}%
\pgfpathrectangle{\pgfqpoint{0.481978in}{0.331635in}}{\pgfqpoint{4.960000in}{3.696000in}}%
\pgfusepath{clip}%
\pgfsetbuttcap%
\pgfsetroundjoin%
\definecolor{currentfill}{rgb}{1.000000,0.705882,0.509804}%
\pgfsetfillcolor{currentfill}%
\pgfsetlinewidth{0.481800pt}%
\definecolor{currentstroke}{rgb}{1.000000,1.000000,1.000000}%
\pgfsetstrokecolor{currentstroke}%
\pgfsetdash{}{0pt}%
\pgfpathmoveto{\pgfqpoint{1.938067in}{1.072657in}}%
\pgfpathcurveto{\pgfqpoint{1.949117in}{1.072657in}}{\pgfqpoint{1.959716in}{1.077047in}}{\pgfqpoint{1.967530in}{1.084860in}}%
\pgfpathcurveto{\pgfqpoint{1.975344in}{1.092674in}}{\pgfqpoint{1.979734in}{1.103273in}}{\pgfqpoint{1.979734in}{1.114323in}}%
\pgfpathcurveto{\pgfqpoint{1.979734in}{1.125373in}}{\pgfqpoint{1.975344in}{1.135972in}}{\pgfqpoint{1.967530in}{1.143786in}}%
\pgfpathcurveto{\pgfqpoint{1.959716in}{1.151600in}}{\pgfqpoint{1.949117in}{1.155990in}}{\pgfqpoint{1.938067in}{1.155990in}}%
\pgfpathcurveto{\pgfqpoint{1.927017in}{1.155990in}}{\pgfqpoint{1.916418in}{1.151600in}}{\pgfqpoint{1.908604in}{1.143786in}}%
\pgfpathcurveto{\pgfqpoint{1.900791in}{1.135972in}}{\pgfqpoint{1.896401in}{1.125373in}}{\pgfqpoint{1.896401in}{1.114323in}}%
\pgfpathcurveto{\pgfqpoint{1.896401in}{1.103273in}}{\pgfqpoint{1.900791in}{1.092674in}}{\pgfqpoint{1.908604in}{1.084860in}}%
\pgfpathcurveto{\pgfqpoint{1.916418in}{1.077047in}}{\pgfqpoint{1.927017in}{1.072657in}}{\pgfqpoint{1.938067in}{1.072657in}}%
\pgfpathclose%
\pgfusepath{stroke,fill}%
\end{pgfscope}%
\begin{pgfscope}%
\pgfpathrectangle{\pgfqpoint{0.481978in}{0.331635in}}{\pgfqpoint{4.960000in}{3.696000in}}%
\pgfusepath{clip}%
\pgfsetbuttcap%
\pgfsetroundjoin%
\definecolor{currentfill}{rgb}{1.000000,0.705882,0.509804}%
\pgfsetfillcolor{currentfill}%
\pgfsetlinewidth{0.481800pt}%
\definecolor{currentstroke}{rgb}{1.000000,1.000000,1.000000}%
\pgfsetstrokecolor{currentstroke}%
\pgfsetdash{}{0pt}%
\pgfpathmoveto{\pgfqpoint{3.368392in}{2.003194in}}%
\pgfpathcurveto{\pgfqpoint{3.379442in}{2.003194in}}{\pgfqpoint{3.390041in}{2.007585in}}{\pgfqpoint{3.397855in}{2.015398in}}%
\pgfpathcurveto{\pgfqpoint{3.405668in}{2.023212in}}{\pgfqpoint{3.410059in}{2.033811in}}{\pgfqpoint{3.410059in}{2.044861in}}%
\pgfpathcurveto{\pgfqpoint{3.410059in}{2.055911in}}{\pgfqpoint{3.405668in}{2.066510in}}{\pgfqpoint{3.397855in}{2.074324in}}%
\pgfpathcurveto{\pgfqpoint{3.390041in}{2.082137in}}{\pgfqpoint{3.379442in}{2.086528in}}{\pgfqpoint{3.368392in}{2.086528in}}%
\pgfpathcurveto{\pgfqpoint{3.357342in}{2.086528in}}{\pgfqpoint{3.346743in}{2.082137in}}{\pgfqpoint{3.338929in}{2.074324in}}%
\pgfpathcurveto{\pgfqpoint{3.331116in}{2.066510in}}{\pgfqpoint{3.326725in}{2.055911in}}{\pgfqpoint{3.326725in}{2.044861in}}%
\pgfpathcurveto{\pgfqpoint{3.326725in}{2.033811in}}{\pgfqpoint{3.331116in}{2.023212in}}{\pgfqpoint{3.338929in}{2.015398in}}%
\pgfpathcurveto{\pgfqpoint{3.346743in}{2.007585in}}{\pgfqpoint{3.357342in}{2.003194in}}{\pgfqpoint{3.368392in}{2.003194in}}%
\pgfpathclose%
\pgfusepath{stroke,fill}%
\end{pgfscope}%
\begin{pgfscope}%
\pgfpathrectangle{\pgfqpoint{0.481978in}{0.331635in}}{\pgfqpoint{4.960000in}{3.696000in}}%
\pgfusepath{clip}%
\pgfsetbuttcap%
\pgfsetroundjoin%
\definecolor{currentfill}{rgb}{1.000000,0.705882,0.509804}%
\pgfsetfillcolor{currentfill}%
\pgfsetlinewidth{0.481800pt}%
\definecolor{currentstroke}{rgb}{1.000000,1.000000,1.000000}%
\pgfsetstrokecolor{currentstroke}%
\pgfsetdash{}{0pt}%
\pgfpathmoveto{\pgfqpoint{1.306901in}{1.179964in}}%
\pgfpathcurveto{\pgfqpoint{1.317952in}{1.179964in}}{\pgfqpoint{1.328551in}{1.184354in}}{\pgfqpoint{1.336364in}{1.192168in}}%
\pgfpathcurveto{\pgfqpoint{1.344178in}{1.199981in}}{\pgfqpoint{1.348568in}{1.210580in}}{\pgfqpoint{1.348568in}{1.221631in}}%
\pgfpathcurveto{\pgfqpoint{1.348568in}{1.232681in}}{\pgfqpoint{1.344178in}{1.243280in}}{\pgfqpoint{1.336364in}{1.251093in}}%
\pgfpathcurveto{\pgfqpoint{1.328551in}{1.258907in}}{\pgfqpoint{1.317952in}{1.263297in}}{\pgfqpoint{1.306901in}{1.263297in}}%
\pgfpathcurveto{\pgfqpoint{1.295851in}{1.263297in}}{\pgfqpoint{1.285252in}{1.258907in}}{\pgfqpoint{1.277439in}{1.251093in}}%
\pgfpathcurveto{\pgfqpoint{1.269625in}{1.243280in}}{\pgfqpoint{1.265235in}{1.232681in}}{\pgfqpoint{1.265235in}{1.221631in}}%
\pgfpathcurveto{\pgfqpoint{1.265235in}{1.210580in}}{\pgfqpoint{1.269625in}{1.199981in}}{\pgfqpoint{1.277439in}{1.192168in}}%
\pgfpathcurveto{\pgfqpoint{1.285252in}{1.184354in}}{\pgfqpoint{1.295851in}{1.179964in}}{\pgfqpoint{1.306901in}{1.179964in}}%
\pgfpathclose%
\pgfusepath{stroke,fill}%
\end{pgfscope}%
\begin{pgfscope}%
\pgfpathrectangle{\pgfqpoint{0.481978in}{0.331635in}}{\pgfqpoint{4.960000in}{3.696000in}}%
\pgfusepath{clip}%
\pgfsetbuttcap%
\pgfsetroundjoin%
\definecolor{currentfill}{rgb}{1.000000,0.705882,0.509804}%
\pgfsetfillcolor{currentfill}%
\pgfsetlinewidth{0.481800pt}%
\definecolor{currentstroke}{rgb}{1.000000,1.000000,1.000000}%
\pgfsetstrokecolor{currentstroke}%
\pgfsetdash{}{0pt}%
\pgfpathmoveto{\pgfqpoint{1.770432in}{1.202339in}}%
\pgfpathcurveto{\pgfqpoint{1.781482in}{1.202339in}}{\pgfqpoint{1.792081in}{1.206730in}}{\pgfqpoint{1.799895in}{1.214543in}}%
\pgfpathcurveto{\pgfqpoint{1.807709in}{1.222357in}}{\pgfqpoint{1.812099in}{1.232956in}}{\pgfqpoint{1.812099in}{1.244006in}}%
\pgfpathcurveto{\pgfqpoint{1.812099in}{1.255056in}}{\pgfqpoint{1.807709in}{1.265655in}}{\pgfqpoint{1.799895in}{1.273469in}}%
\pgfpathcurveto{\pgfqpoint{1.792081in}{1.281283in}}{\pgfqpoint{1.781482in}{1.285673in}}{\pgfqpoint{1.770432in}{1.285673in}}%
\pgfpathcurveto{\pgfqpoint{1.759382in}{1.285673in}}{\pgfqpoint{1.748783in}{1.281283in}}{\pgfqpoint{1.740970in}{1.273469in}}%
\pgfpathcurveto{\pgfqpoint{1.733156in}{1.265655in}}{\pgfqpoint{1.728766in}{1.255056in}}{\pgfqpoint{1.728766in}{1.244006in}}%
\pgfpathcurveto{\pgfqpoint{1.728766in}{1.232956in}}{\pgfqpoint{1.733156in}{1.222357in}}{\pgfqpoint{1.740970in}{1.214543in}}%
\pgfpathcurveto{\pgfqpoint{1.748783in}{1.206730in}}{\pgfqpoint{1.759382in}{1.202339in}}{\pgfqpoint{1.770432in}{1.202339in}}%
\pgfpathclose%
\pgfusepath{stroke,fill}%
\end{pgfscope}%
\begin{pgfscope}%
\pgfpathrectangle{\pgfqpoint{0.481978in}{0.331635in}}{\pgfqpoint{4.960000in}{3.696000in}}%
\pgfusepath{clip}%
\pgfsetbuttcap%
\pgfsetroundjoin%
\definecolor{currentfill}{rgb}{1.000000,0.705882,0.509804}%
\pgfsetfillcolor{currentfill}%
\pgfsetlinewidth{0.481800pt}%
\definecolor{currentstroke}{rgb}{1.000000,1.000000,1.000000}%
\pgfsetstrokecolor{currentstroke}%
\pgfsetdash{}{0pt}%
\pgfpathmoveto{\pgfqpoint{2.441773in}{1.154386in}}%
\pgfpathcurveto{\pgfqpoint{2.452824in}{1.154386in}}{\pgfqpoint{2.463423in}{1.158777in}}{\pgfqpoint{2.471236in}{1.166590in}}%
\pgfpathcurveto{\pgfqpoint{2.479050in}{1.174404in}}{\pgfqpoint{2.483440in}{1.185003in}}{\pgfqpoint{2.483440in}{1.196053in}}%
\pgfpathcurveto{\pgfqpoint{2.483440in}{1.207103in}}{\pgfqpoint{2.479050in}{1.217702in}}{\pgfqpoint{2.471236in}{1.225516in}}%
\pgfpathcurveto{\pgfqpoint{2.463423in}{1.233329in}}{\pgfqpoint{2.452824in}{1.237720in}}{\pgfqpoint{2.441773in}{1.237720in}}%
\pgfpathcurveto{\pgfqpoint{2.430723in}{1.237720in}}{\pgfqpoint{2.420124in}{1.233329in}}{\pgfqpoint{2.412311in}{1.225516in}}%
\pgfpathcurveto{\pgfqpoint{2.404497in}{1.217702in}}{\pgfqpoint{2.400107in}{1.207103in}}{\pgfqpoint{2.400107in}{1.196053in}}%
\pgfpathcurveto{\pgfqpoint{2.400107in}{1.185003in}}{\pgfqpoint{2.404497in}{1.174404in}}{\pgfqpoint{2.412311in}{1.166590in}}%
\pgfpathcurveto{\pgfqpoint{2.420124in}{1.158777in}}{\pgfqpoint{2.430723in}{1.154386in}}{\pgfqpoint{2.441773in}{1.154386in}}%
\pgfpathclose%
\pgfusepath{stroke,fill}%
\end{pgfscope}%
\begin{pgfscope}%
\pgfpathrectangle{\pgfqpoint{0.481978in}{0.331635in}}{\pgfqpoint{4.960000in}{3.696000in}}%
\pgfusepath{clip}%
\pgfsetbuttcap%
\pgfsetroundjoin%
\definecolor{currentfill}{rgb}{1.000000,0.705882,0.509804}%
\pgfsetfillcolor{currentfill}%
\pgfsetlinewidth{0.481800pt}%
\definecolor{currentstroke}{rgb}{1.000000,1.000000,1.000000}%
\pgfsetstrokecolor{currentstroke}%
\pgfsetdash{}{0pt}%
\pgfpathmoveto{\pgfqpoint{3.449131in}{1.237230in}}%
\pgfpathcurveto{\pgfqpoint{3.460181in}{1.237230in}}{\pgfqpoint{3.470780in}{1.241620in}}{\pgfqpoint{3.478593in}{1.249434in}}%
\pgfpathcurveto{\pgfqpoint{3.486407in}{1.257248in}}{\pgfqpoint{3.490797in}{1.267847in}}{\pgfqpoint{3.490797in}{1.278897in}}%
\pgfpathcurveto{\pgfqpoint{3.490797in}{1.289947in}}{\pgfqpoint{3.486407in}{1.300546in}}{\pgfqpoint{3.478593in}{1.308360in}}%
\pgfpathcurveto{\pgfqpoint{3.470780in}{1.316173in}}{\pgfqpoint{3.460181in}{1.320564in}}{\pgfqpoint{3.449131in}{1.320564in}}%
\pgfpathcurveto{\pgfqpoint{3.438080in}{1.320564in}}{\pgfqpoint{3.427481in}{1.316173in}}{\pgfqpoint{3.419668in}{1.308360in}}%
\pgfpathcurveto{\pgfqpoint{3.411854in}{1.300546in}}{\pgfqpoint{3.407464in}{1.289947in}}{\pgfqpoint{3.407464in}{1.278897in}}%
\pgfpathcurveto{\pgfqpoint{3.407464in}{1.267847in}}{\pgfqpoint{3.411854in}{1.257248in}}{\pgfqpoint{3.419668in}{1.249434in}}%
\pgfpathcurveto{\pgfqpoint{3.427481in}{1.241620in}}{\pgfqpoint{3.438080in}{1.237230in}}{\pgfqpoint{3.449131in}{1.237230in}}%
\pgfpathclose%
\pgfusepath{stroke,fill}%
\end{pgfscope}%
\begin{pgfscope}%
\pgfpathrectangle{\pgfqpoint{0.481978in}{0.331635in}}{\pgfqpoint{4.960000in}{3.696000in}}%
\pgfusepath{clip}%
\pgfsetbuttcap%
\pgfsetroundjoin%
\definecolor{currentfill}{rgb}{1.000000,0.705882,0.509804}%
\pgfsetfillcolor{currentfill}%
\pgfsetlinewidth{0.481800pt}%
\definecolor{currentstroke}{rgb}{1.000000,1.000000,1.000000}%
\pgfsetstrokecolor{currentstroke}%
\pgfsetdash{}{0pt}%
\pgfpathmoveto{\pgfqpoint{2.071777in}{1.357788in}}%
\pgfpathcurveto{\pgfqpoint{2.082828in}{1.357788in}}{\pgfqpoint{2.093427in}{1.362178in}}{\pgfqpoint{2.101240in}{1.369991in}}%
\pgfpathcurveto{\pgfqpoint{2.109054in}{1.377805in}}{\pgfqpoint{2.113444in}{1.388404in}}{\pgfqpoint{2.113444in}{1.399454in}}%
\pgfpathcurveto{\pgfqpoint{2.113444in}{1.410504in}}{\pgfqpoint{2.109054in}{1.421103in}}{\pgfqpoint{2.101240in}{1.428917in}}%
\pgfpathcurveto{\pgfqpoint{2.093427in}{1.436731in}}{\pgfqpoint{2.082828in}{1.441121in}}{\pgfqpoint{2.071777in}{1.441121in}}%
\pgfpathcurveto{\pgfqpoint{2.060727in}{1.441121in}}{\pgfqpoint{2.050128in}{1.436731in}}{\pgfqpoint{2.042315in}{1.428917in}}%
\pgfpathcurveto{\pgfqpoint{2.034501in}{1.421103in}}{\pgfqpoint{2.030111in}{1.410504in}}{\pgfqpoint{2.030111in}{1.399454in}}%
\pgfpathcurveto{\pgfqpoint{2.030111in}{1.388404in}}{\pgfqpoint{2.034501in}{1.377805in}}{\pgfqpoint{2.042315in}{1.369991in}}%
\pgfpathcurveto{\pgfqpoint{2.050128in}{1.362178in}}{\pgfqpoint{2.060727in}{1.357788in}}{\pgfqpoint{2.071777in}{1.357788in}}%
\pgfpathclose%
\pgfusepath{stroke,fill}%
\end{pgfscope}%
\begin{pgfscope}%
\pgfpathrectangle{\pgfqpoint{0.481978in}{0.331635in}}{\pgfqpoint{4.960000in}{3.696000in}}%
\pgfusepath{clip}%
\pgfsetbuttcap%
\pgfsetroundjoin%
\definecolor{currentfill}{rgb}{1.000000,0.705882,0.509804}%
\pgfsetfillcolor{currentfill}%
\pgfsetlinewidth{0.481800pt}%
\definecolor{currentstroke}{rgb}{1.000000,1.000000,1.000000}%
\pgfsetstrokecolor{currentstroke}%
\pgfsetdash{}{0pt}%
\pgfpathmoveto{\pgfqpoint{2.389920in}{0.678234in}}%
\pgfpathcurveto{\pgfqpoint{2.400970in}{0.678234in}}{\pgfqpoint{2.411569in}{0.682625in}}{\pgfqpoint{2.419383in}{0.690438in}}%
\pgfpathcurveto{\pgfqpoint{2.427196in}{0.698252in}}{\pgfqpoint{2.431586in}{0.708851in}}{\pgfqpoint{2.431586in}{0.719901in}}%
\pgfpathcurveto{\pgfqpoint{2.431586in}{0.730951in}}{\pgfqpoint{2.427196in}{0.741550in}}{\pgfqpoint{2.419383in}{0.749364in}}%
\pgfpathcurveto{\pgfqpoint{2.411569in}{0.757177in}}{\pgfqpoint{2.400970in}{0.761568in}}{\pgfqpoint{2.389920in}{0.761568in}}%
\pgfpathcurveto{\pgfqpoint{2.378870in}{0.761568in}}{\pgfqpoint{2.368271in}{0.757177in}}{\pgfqpoint{2.360457in}{0.749364in}}%
\pgfpathcurveto{\pgfqpoint{2.352643in}{0.741550in}}{\pgfqpoint{2.348253in}{0.730951in}}{\pgfqpoint{2.348253in}{0.719901in}}%
\pgfpathcurveto{\pgfqpoint{2.348253in}{0.708851in}}{\pgfqpoint{2.352643in}{0.698252in}}{\pgfqpoint{2.360457in}{0.690438in}}%
\pgfpathcurveto{\pgfqpoint{2.368271in}{0.682625in}}{\pgfqpoint{2.378870in}{0.678234in}}{\pgfqpoint{2.389920in}{0.678234in}}%
\pgfpathclose%
\pgfusepath{stroke,fill}%
\end{pgfscope}%
\begin{pgfscope}%
\pgfpathrectangle{\pgfqpoint{0.481978in}{0.331635in}}{\pgfqpoint{4.960000in}{3.696000in}}%
\pgfusepath{clip}%
\pgfsetbuttcap%
\pgfsetroundjoin%
\definecolor{currentfill}{rgb}{1.000000,0.705882,0.509804}%
\pgfsetfillcolor{currentfill}%
\pgfsetlinewidth{0.481800pt}%
\definecolor{currentstroke}{rgb}{1.000000,1.000000,1.000000}%
\pgfsetstrokecolor{currentstroke}%
\pgfsetdash{}{0pt}%
\pgfpathmoveto{\pgfqpoint{2.528165in}{0.877290in}}%
\pgfpathcurveto{\pgfqpoint{2.539215in}{0.877290in}}{\pgfqpoint{2.549814in}{0.881680in}}{\pgfqpoint{2.557628in}{0.889494in}}%
\pgfpathcurveto{\pgfqpoint{2.565442in}{0.897308in}}{\pgfqpoint{2.569832in}{0.907907in}}{\pgfqpoint{2.569832in}{0.918957in}}%
\pgfpathcurveto{\pgfqpoint{2.569832in}{0.930007in}}{\pgfqpoint{2.565442in}{0.940606in}}{\pgfqpoint{2.557628in}{0.948419in}}%
\pgfpathcurveto{\pgfqpoint{2.549814in}{0.956233in}}{\pgfqpoint{2.539215in}{0.960623in}}{\pgfqpoint{2.528165in}{0.960623in}}%
\pgfpathcurveto{\pgfqpoint{2.517115in}{0.960623in}}{\pgfqpoint{2.506516in}{0.956233in}}{\pgfqpoint{2.498702in}{0.948419in}}%
\pgfpathcurveto{\pgfqpoint{2.490889in}{0.940606in}}{\pgfqpoint{2.486498in}{0.930007in}}{\pgfqpoint{2.486498in}{0.918957in}}%
\pgfpathcurveto{\pgfqpoint{2.486498in}{0.907907in}}{\pgfqpoint{2.490889in}{0.897308in}}{\pgfqpoint{2.498702in}{0.889494in}}%
\pgfpathcurveto{\pgfqpoint{2.506516in}{0.881680in}}{\pgfqpoint{2.517115in}{0.877290in}}{\pgfqpoint{2.528165in}{0.877290in}}%
\pgfpathclose%
\pgfusepath{stroke,fill}%
\end{pgfscope}%
\begin{pgfscope}%
\pgfpathrectangle{\pgfqpoint{0.481978in}{0.331635in}}{\pgfqpoint{4.960000in}{3.696000in}}%
\pgfusepath{clip}%
\pgfsetbuttcap%
\pgfsetroundjoin%
\definecolor{currentfill}{rgb}{1.000000,0.705882,0.509804}%
\pgfsetfillcolor{currentfill}%
\pgfsetlinewidth{0.481800pt}%
\definecolor{currentstroke}{rgb}{1.000000,1.000000,1.000000}%
\pgfsetstrokecolor{currentstroke}%
\pgfsetdash{}{0pt}%
\pgfpathmoveto{\pgfqpoint{3.056217in}{0.804988in}}%
\pgfpathcurveto{\pgfqpoint{3.067267in}{0.804988in}}{\pgfqpoint{3.077866in}{0.809379in}}{\pgfqpoint{3.085680in}{0.817192in}}%
\pgfpathcurveto{\pgfqpoint{3.093493in}{0.825006in}}{\pgfqpoint{3.097884in}{0.835605in}}{\pgfqpoint{3.097884in}{0.846655in}}%
\pgfpathcurveto{\pgfqpoint{3.097884in}{0.857705in}}{\pgfqpoint{3.093493in}{0.868304in}}{\pgfqpoint{3.085680in}{0.876118in}}%
\pgfpathcurveto{\pgfqpoint{3.077866in}{0.883932in}}{\pgfqpoint{3.067267in}{0.888322in}}{\pgfqpoint{3.056217in}{0.888322in}}%
\pgfpathcurveto{\pgfqpoint{3.045167in}{0.888322in}}{\pgfqpoint{3.034568in}{0.883932in}}{\pgfqpoint{3.026754in}{0.876118in}}%
\pgfpathcurveto{\pgfqpoint{3.018941in}{0.868304in}}{\pgfqpoint{3.014550in}{0.857705in}}{\pgfqpoint{3.014550in}{0.846655in}}%
\pgfpathcurveto{\pgfqpoint{3.014550in}{0.835605in}}{\pgfqpoint{3.018941in}{0.825006in}}{\pgfqpoint{3.026754in}{0.817192in}}%
\pgfpathcurveto{\pgfqpoint{3.034568in}{0.809379in}}{\pgfqpoint{3.045167in}{0.804988in}}{\pgfqpoint{3.056217in}{0.804988in}}%
\pgfpathclose%
\pgfusepath{stroke,fill}%
\end{pgfscope}%
\begin{pgfscope}%
\pgfpathrectangle{\pgfqpoint{0.481978in}{0.331635in}}{\pgfqpoint{4.960000in}{3.696000in}}%
\pgfusepath{clip}%
\pgfsetbuttcap%
\pgfsetroundjoin%
\definecolor{currentfill}{rgb}{1.000000,0.705882,0.509804}%
\pgfsetfillcolor{currentfill}%
\pgfsetlinewidth{0.481800pt}%
\definecolor{currentstroke}{rgb}{1.000000,1.000000,1.000000}%
\pgfsetstrokecolor{currentstroke}%
\pgfsetdash{}{0pt}%
\pgfpathmoveto{\pgfqpoint{3.701071in}{1.646263in}}%
\pgfpathcurveto{\pgfqpoint{3.712121in}{1.646263in}}{\pgfqpoint{3.722720in}{1.650653in}}{\pgfqpoint{3.730534in}{1.658467in}}%
\pgfpathcurveto{\pgfqpoint{3.738347in}{1.666281in}}{\pgfqpoint{3.742738in}{1.676880in}}{\pgfqpoint{3.742738in}{1.687930in}}%
\pgfpathcurveto{\pgfqpoint{3.742738in}{1.698980in}}{\pgfqpoint{3.738347in}{1.709579in}}{\pgfqpoint{3.730534in}{1.717393in}}%
\pgfpathcurveto{\pgfqpoint{3.722720in}{1.725206in}}{\pgfqpoint{3.712121in}{1.729596in}}{\pgfqpoint{3.701071in}{1.729596in}}%
\pgfpathcurveto{\pgfqpoint{3.690021in}{1.729596in}}{\pgfqpoint{3.679422in}{1.725206in}}{\pgfqpoint{3.671608in}{1.717393in}}%
\pgfpathcurveto{\pgfqpoint{3.663794in}{1.709579in}}{\pgfqpoint{3.659404in}{1.698980in}}{\pgfqpoint{3.659404in}{1.687930in}}%
\pgfpathcurveto{\pgfqpoint{3.659404in}{1.676880in}}{\pgfqpoint{3.663794in}{1.666281in}}{\pgfqpoint{3.671608in}{1.658467in}}%
\pgfpathcurveto{\pgfqpoint{3.679422in}{1.650653in}}{\pgfqpoint{3.690021in}{1.646263in}}{\pgfqpoint{3.701071in}{1.646263in}}%
\pgfpathclose%
\pgfusepath{stroke,fill}%
\end{pgfscope}%
\begin{pgfscope}%
\pgfpathrectangle{\pgfqpoint{0.481978in}{0.331635in}}{\pgfqpoint{4.960000in}{3.696000in}}%
\pgfusepath{clip}%
\pgfsetbuttcap%
\pgfsetroundjoin%
\definecolor{currentfill}{rgb}{1.000000,0.705882,0.509804}%
\pgfsetfillcolor{currentfill}%
\pgfsetlinewidth{0.481800pt}%
\definecolor{currentstroke}{rgb}{1.000000,1.000000,1.000000}%
\pgfsetstrokecolor{currentstroke}%
\pgfsetdash{}{0pt}%
\pgfpathmoveto{\pgfqpoint{1.975829in}{2.268323in}}%
\pgfpathcurveto{\pgfqpoint{1.986879in}{2.268323in}}{\pgfqpoint{1.997478in}{2.272714in}}{\pgfqpoint{2.005292in}{2.280527in}}%
\pgfpathcurveto{\pgfqpoint{2.013105in}{2.288341in}}{\pgfqpoint{2.017496in}{2.298940in}}{\pgfqpoint{2.017496in}{2.309990in}}%
\pgfpathcurveto{\pgfqpoint{2.017496in}{2.321040in}}{\pgfqpoint{2.013105in}{2.331639in}}{\pgfqpoint{2.005292in}{2.339453in}}%
\pgfpathcurveto{\pgfqpoint{1.997478in}{2.347267in}}{\pgfqpoint{1.986879in}{2.351657in}}{\pgfqpoint{1.975829in}{2.351657in}}%
\pgfpathcurveto{\pgfqpoint{1.964779in}{2.351657in}}{\pgfqpoint{1.954180in}{2.347267in}}{\pgfqpoint{1.946366in}{2.339453in}}%
\pgfpathcurveto{\pgfqpoint{1.938552in}{2.331639in}}{\pgfqpoint{1.934162in}{2.321040in}}{\pgfqpoint{1.934162in}{2.309990in}}%
\pgfpathcurveto{\pgfqpoint{1.934162in}{2.298940in}}{\pgfqpoint{1.938552in}{2.288341in}}{\pgfqpoint{1.946366in}{2.280527in}}%
\pgfpathcurveto{\pgfqpoint{1.954180in}{2.272714in}}{\pgfqpoint{1.964779in}{2.268323in}}{\pgfqpoint{1.975829in}{2.268323in}}%
\pgfpathclose%
\pgfusepath{stroke,fill}%
\end{pgfscope}%
\begin{pgfscope}%
\pgfpathrectangle{\pgfqpoint{0.481978in}{0.331635in}}{\pgfqpoint{4.960000in}{3.696000in}}%
\pgfusepath{clip}%
\pgfsetbuttcap%
\pgfsetroundjoin%
\definecolor{currentfill}{rgb}{1.000000,0.705882,0.509804}%
\pgfsetfillcolor{currentfill}%
\pgfsetlinewidth{0.481800pt}%
\definecolor{currentstroke}{rgb}{1.000000,1.000000,1.000000}%
\pgfsetstrokecolor{currentstroke}%
\pgfsetdash{}{0pt}%
\pgfpathmoveto{\pgfqpoint{4.081865in}{0.749513in}}%
\pgfpathcurveto{\pgfqpoint{4.092915in}{0.749513in}}{\pgfqpoint{4.103514in}{0.753903in}}{\pgfqpoint{4.111328in}{0.761717in}}%
\pgfpathcurveto{\pgfqpoint{4.119141in}{0.769530in}}{\pgfqpoint{4.123532in}{0.780129in}}{\pgfqpoint{4.123532in}{0.791180in}}%
\pgfpathcurveto{\pgfqpoint{4.123532in}{0.802230in}}{\pgfqpoint{4.119141in}{0.812829in}}{\pgfqpoint{4.111328in}{0.820642in}}%
\pgfpathcurveto{\pgfqpoint{4.103514in}{0.828456in}}{\pgfqpoint{4.092915in}{0.832846in}}{\pgfqpoint{4.081865in}{0.832846in}}%
\pgfpathcurveto{\pgfqpoint{4.070815in}{0.832846in}}{\pgfqpoint{4.060216in}{0.828456in}}{\pgfqpoint{4.052402in}{0.820642in}}%
\pgfpathcurveto{\pgfqpoint{4.044589in}{0.812829in}}{\pgfqpoint{4.040198in}{0.802230in}}{\pgfqpoint{4.040198in}{0.791180in}}%
\pgfpathcurveto{\pgfqpoint{4.040198in}{0.780129in}}{\pgfqpoint{4.044589in}{0.769530in}}{\pgfqpoint{4.052402in}{0.761717in}}%
\pgfpathcurveto{\pgfqpoint{4.060216in}{0.753903in}}{\pgfqpoint{4.070815in}{0.749513in}}{\pgfqpoint{4.081865in}{0.749513in}}%
\pgfpathclose%
\pgfusepath{stroke,fill}%
\end{pgfscope}%
\begin{pgfscope}%
\pgfpathrectangle{\pgfqpoint{0.481978in}{0.331635in}}{\pgfqpoint{4.960000in}{3.696000in}}%
\pgfusepath{clip}%
\pgfsetbuttcap%
\pgfsetroundjoin%
\definecolor{currentfill}{rgb}{1.000000,0.705882,0.509804}%
\pgfsetfillcolor{currentfill}%
\pgfsetlinewidth{0.481800pt}%
\definecolor{currentstroke}{rgb}{1.000000,1.000000,1.000000}%
\pgfsetstrokecolor{currentstroke}%
\pgfsetdash{}{0pt}%
\pgfpathmoveto{\pgfqpoint{1.776731in}{1.202183in}}%
\pgfpathcurveto{\pgfqpoint{1.787781in}{1.202183in}}{\pgfqpoint{1.798380in}{1.206574in}}{\pgfqpoint{1.806194in}{1.214387in}}%
\pgfpathcurveto{\pgfqpoint{1.814008in}{1.222201in}}{\pgfqpoint{1.818398in}{1.232800in}}{\pgfqpoint{1.818398in}{1.243850in}}%
\pgfpathcurveto{\pgfqpoint{1.818398in}{1.254900in}}{\pgfqpoint{1.814008in}{1.265499in}}{\pgfqpoint{1.806194in}{1.273313in}}%
\pgfpathcurveto{\pgfqpoint{1.798380in}{1.281127in}}{\pgfqpoint{1.787781in}{1.285517in}}{\pgfqpoint{1.776731in}{1.285517in}}%
\pgfpathcurveto{\pgfqpoint{1.765681in}{1.285517in}}{\pgfqpoint{1.755082in}{1.281127in}}{\pgfqpoint{1.747269in}{1.273313in}}%
\pgfpathcurveto{\pgfqpoint{1.739455in}{1.265499in}}{\pgfqpoint{1.735065in}{1.254900in}}{\pgfqpoint{1.735065in}{1.243850in}}%
\pgfpathcurveto{\pgfqpoint{1.735065in}{1.232800in}}{\pgfqpoint{1.739455in}{1.222201in}}{\pgfqpoint{1.747269in}{1.214387in}}%
\pgfpathcurveto{\pgfqpoint{1.755082in}{1.206574in}}{\pgfqpoint{1.765681in}{1.202183in}}{\pgfqpoint{1.776731in}{1.202183in}}%
\pgfpathclose%
\pgfusepath{stroke,fill}%
\end{pgfscope}%
\begin{pgfscope}%
\pgfpathrectangle{\pgfqpoint{0.481978in}{0.331635in}}{\pgfqpoint{4.960000in}{3.696000in}}%
\pgfusepath{clip}%
\pgfsetbuttcap%
\pgfsetroundjoin%
\definecolor{currentfill}{rgb}{1.000000,0.705882,0.509804}%
\pgfsetfillcolor{currentfill}%
\pgfsetlinewidth{0.481800pt}%
\definecolor{currentstroke}{rgb}{1.000000,1.000000,1.000000}%
\pgfsetstrokecolor{currentstroke}%
\pgfsetdash{}{0pt}%
\pgfpathmoveto{\pgfqpoint{3.534146in}{1.060583in}}%
\pgfpathcurveto{\pgfqpoint{3.545196in}{1.060583in}}{\pgfqpoint{3.555795in}{1.064973in}}{\pgfqpoint{3.563609in}{1.072787in}}%
\pgfpathcurveto{\pgfqpoint{3.571423in}{1.080600in}}{\pgfqpoint{3.575813in}{1.091199in}}{\pgfqpoint{3.575813in}{1.102249in}}%
\pgfpathcurveto{\pgfqpoint{3.575813in}{1.113300in}}{\pgfqpoint{3.571423in}{1.123899in}}{\pgfqpoint{3.563609in}{1.131712in}}%
\pgfpathcurveto{\pgfqpoint{3.555795in}{1.139526in}}{\pgfqpoint{3.545196in}{1.143916in}}{\pgfqpoint{3.534146in}{1.143916in}}%
\pgfpathcurveto{\pgfqpoint{3.523096in}{1.143916in}}{\pgfqpoint{3.512497in}{1.139526in}}{\pgfqpoint{3.504683in}{1.131712in}}%
\pgfpathcurveto{\pgfqpoint{3.496870in}{1.123899in}}{\pgfqpoint{3.492479in}{1.113300in}}{\pgfqpoint{3.492479in}{1.102249in}}%
\pgfpathcurveto{\pgfqpoint{3.492479in}{1.091199in}}{\pgfqpoint{3.496870in}{1.080600in}}{\pgfqpoint{3.504683in}{1.072787in}}%
\pgfpathcurveto{\pgfqpoint{3.512497in}{1.064973in}}{\pgfqpoint{3.523096in}{1.060583in}}{\pgfqpoint{3.534146in}{1.060583in}}%
\pgfpathclose%
\pgfusepath{stroke,fill}%
\end{pgfscope}%
\begin{pgfscope}%
\pgfpathrectangle{\pgfqpoint{0.481978in}{0.331635in}}{\pgfqpoint{4.960000in}{3.696000in}}%
\pgfusepath{clip}%
\pgfsetbuttcap%
\pgfsetroundjoin%
\definecolor{currentfill}{rgb}{1.000000,0.705882,0.509804}%
\pgfsetfillcolor{currentfill}%
\pgfsetlinewidth{0.481800pt}%
\definecolor{currentstroke}{rgb}{1.000000,1.000000,1.000000}%
\pgfsetstrokecolor{currentstroke}%
\pgfsetdash{}{0pt}%
\pgfpathmoveto{\pgfqpoint{3.490400in}{1.550475in}}%
\pgfpathcurveto{\pgfqpoint{3.501450in}{1.550475in}}{\pgfqpoint{3.512049in}{1.554866in}}{\pgfqpoint{3.519863in}{1.562679in}}%
\pgfpathcurveto{\pgfqpoint{3.527677in}{1.570493in}}{\pgfqpoint{3.532067in}{1.581092in}}{\pgfqpoint{3.532067in}{1.592142in}}%
\pgfpathcurveto{\pgfqpoint{3.532067in}{1.603192in}}{\pgfqpoint{3.527677in}{1.613791in}}{\pgfqpoint{3.519863in}{1.621605in}}%
\pgfpathcurveto{\pgfqpoint{3.512049in}{1.629418in}}{\pgfqpoint{3.501450in}{1.633809in}}{\pgfqpoint{3.490400in}{1.633809in}}%
\pgfpathcurveto{\pgfqpoint{3.479350in}{1.633809in}}{\pgfqpoint{3.468751in}{1.629418in}}{\pgfqpoint{3.460937in}{1.621605in}}%
\pgfpathcurveto{\pgfqpoint{3.453124in}{1.613791in}}{\pgfqpoint{3.448734in}{1.603192in}}{\pgfqpoint{3.448734in}{1.592142in}}%
\pgfpathcurveto{\pgfqpoint{3.448734in}{1.581092in}}{\pgfqpoint{3.453124in}{1.570493in}}{\pgfqpoint{3.460937in}{1.562679in}}%
\pgfpathcurveto{\pgfqpoint{3.468751in}{1.554866in}}{\pgfqpoint{3.479350in}{1.550475in}}{\pgfqpoint{3.490400in}{1.550475in}}%
\pgfpathclose%
\pgfusepath{stroke,fill}%
\end{pgfscope}%
\begin{pgfscope}%
\pgfpathrectangle{\pgfqpoint{0.481978in}{0.331635in}}{\pgfqpoint{4.960000in}{3.696000in}}%
\pgfusepath{clip}%
\pgfsetbuttcap%
\pgfsetroundjoin%
\definecolor{currentfill}{rgb}{1.000000,0.705882,0.509804}%
\pgfsetfillcolor{currentfill}%
\pgfsetlinewidth{0.481800pt}%
\definecolor{currentstroke}{rgb}{1.000000,1.000000,1.000000}%
\pgfsetstrokecolor{currentstroke}%
\pgfsetdash{}{0pt}%
\pgfpathmoveto{\pgfqpoint{2.087048in}{0.583135in}}%
\pgfpathcurveto{\pgfqpoint{2.098098in}{0.583135in}}{\pgfqpoint{2.108697in}{0.587526in}}{\pgfqpoint{2.116511in}{0.595339in}}%
\pgfpathcurveto{\pgfqpoint{2.124324in}{0.603153in}}{\pgfqpoint{2.128714in}{0.613752in}}{\pgfqpoint{2.128714in}{0.624802in}}%
\pgfpathcurveto{\pgfqpoint{2.128714in}{0.635852in}}{\pgfqpoint{2.124324in}{0.646451in}}{\pgfqpoint{2.116511in}{0.654265in}}%
\pgfpathcurveto{\pgfqpoint{2.108697in}{0.662079in}}{\pgfqpoint{2.098098in}{0.666469in}}{\pgfqpoint{2.087048in}{0.666469in}}%
\pgfpathcurveto{\pgfqpoint{2.075998in}{0.666469in}}{\pgfqpoint{2.065399in}{0.662079in}}{\pgfqpoint{2.057585in}{0.654265in}}%
\pgfpathcurveto{\pgfqpoint{2.049771in}{0.646451in}}{\pgfqpoint{2.045381in}{0.635852in}}{\pgfqpoint{2.045381in}{0.624802in}}%
\pgfpathcurveto{\pgfqpoint{2.045381in}{0.613752in}}{\pgfqpoint{2.049771in}{0.603153in}}{\pgfqpoint{2.057585in}{0.595339in}}%
\pgfpathcurveto{\pgfqpoint{2.065399in}{0.587526in}}{\pgfqpoint{2.075998in}{0.583135in}}{\pgfqpoint{2.087048in}{0.583135in}}%
\pgfpathclose%
\pgfusepath{stroke,fill}%
\end{pgfscope}%
\begin{pgfscope}%
\pgfpathrectangle{\pgfqpoint{0.481978in}{0.331635in}}{\pgfqpoint{4.960000in}{3.696000in}}%
\pgfusepath{clip}%
\pgfsetbuttcap%
\pgfsetroundjoin%
\definecolor{currentfill}{rgb}{1.000000,0.705882,0.509804}%
\pgfsetfillcolor{currentfill}%
\pgfsetlinewidth{0.481800pt}%
\definecolor{currentstroke}{rgb}{1.000000,1.000000,1.000000}%
\pgfsetstrokecolor{currentstroke}%
\pgfsetdash{}{0pt}%
\pgfpathmoveto{\pgfqpoint{2.593014in}{0.675649in}}%
\pgfpathcurveto{\pgfqpoint{2.604064in}{0.675649in}}{\pgfqpoint{2.614663in}{0.680039in}}{\pgfqpoint{2.622477in}{0.687852in}}%
\pgfpathcurveto{\pgfqpoint{2.630290in}{0.695666in}}{\pgfqpoint{2.634680in}{0.706265in}}{\pgfqpoint{2.634680in}{0.717315in}}%
\pgfpathcurveto{\pgfqpoint{2.634680in}{0.728365in}}{\pgfqpoint{2.630290in}{0.738964in}}{\pgfqpoint{2.622477in}{0.746778in}}%
\pgfpathcurveto{\pgfqpoint{2.614663in}{0.754592in}}{\pgfqpoint{2.604064in}{0.758982in}}{\pgfqpoint{2.593014in}{0.758982in}}%
\pgfpathcurveto{\pgfqpoint{2.581964in}{0.758982in}}{\pgfqpoint{2.571365in}{0.754592in}}{\pgfqpoint{2.563551in}{0.746778in}}%
\pgfpathcurveto{\pgfqpoint{2.555737in}{0.738964in}}{\pgfqpoint{2.551347in}{0.728365in}}{\pgfqpoint{2.551347in}{0.717315in}}%
\pgfpathcurveto{\pgfqpoint{2.551347in}{0.706265in}}{\pgfqpoint{2.555737in}{0.695666in}}{\pgfqpoint{2.563551in}{0.687852in}}%
\pgfpathcurveto{\pgfqpoint{2.571365in}{0.680039in}}{\pgfqpoint{2.581964in}{0.675649in}}{\pgfqpoint{2.593014in}{0.675649in}}%
\pgfpathclose%
\pgfusepath{stroke,fill}%
\end{pgfscope}%
\begin{pgfscope}%
\pgfpathrectangle{\pgfqpoint{0.481978in}{0.331635in}}{\pgfqpoint{4.960000in}{3.696000in}}%
\pgfusepath{clip}%
\pgfsetbuttcap%
\pgfsetroundjoin%
\definecolor{currentfill}{rgb}{1.000000,0.705882,0.509804}%
\pgfsetfillcolor{currentfill}%
\pgfsetlinewidth{0.481800pt}%
\definecolor{currentstroke}{rgb}{1.000000,1.000000,1.000000}%
\pgfsetstrokecolor{currentstroke}%
\pgfsetdash{}{0pt}%
\pgfpathmoveto{\pgfqpoint{1.487659in}{1.665662in}}%
\pgfpathcurveto{\pgfqpoint{1.498709in}{1.665662in}}{\pgfqpoint{1.509308in}{1.670053in}}{\pgfqpoint{1.517121in}{1.677866in}}%
\pgfpathcurveto{\pgfqpoint{1.524935in}{1.685680in}}{\pgfqpoint{1.529325in}{1.696279in}}{\pgfqpoint{1.529325in}{1.707329in}}%
\pgfpathcurveto{\pgfqpoint{1.529325in}{1.718379in}}{\pgfqpoint{1.524935in}{1.728978in}}{\pgfqpoint{1.517121in}{1.736792in}}%
\pgfpathcurveto{\pgfqpoint{1.509308in}{1.744606in}}{\pgfqpoint{1.498709in}{1.748996in}}{\pgfqpoint{1.487659in}{1.748996in}}%
\pgfpathcurveto{\pgfqpoint{1.476608in}{1.748996in}}{\pgfqpoint{1.466009in}{1.744606in}}{\pgfqpoint{1.458196in}{1.736792in}}%
\pgfpathcurveto{\pgfqpoint{1.450382in}{1.728978in}}{\pgfqpoint{1.445992in}{1.718379in}}{\pgfqpoint{1.445992in}{1.707329in}}%
\pgfpathcurveto{\pgfqpoint{1.445992in}{1.696279in}}{\pgfqpoint{1.450382in}{1.685680in}}{\pgfqpoint{1.458196in}{1.677866in}}%
\pgfpathcurveto{\pgfqpoint{1.466009in}{1.670053in}}{\pgfqpoint{1.476608in}{1.665662in}}{\pgfqpoint{1.487659in}{1.665662in}}%
\pgfpathclose%
\pgfusepath{stroke,fill}%
\end{pgfscope}%
\begin{pgfscope}%
\pgfpathrectangle{\pgfqpoint{0.481978in}{0.331635in}}{\pgfqpoint{4.960000in}{3.696000in}}%
\pgfusepath{clip}%
\pgfsetbuttcap%
\pgfsetroundjoin%
\definecolor{currentfill}{rgb}{1.000000,0.705882,0.509804}%
\pgfsetfillcolor{currentfill}%
\pgfsetlinewidth{0.481800pt}%
\definecolor{currentstroke}{rgb}{1.000000,1.000000,1.000000}%
\pgfsetstrokecolor{currentstroke}%
\pgfsetdash{}{0pt}%
\pgfpathmoveto{\pgfqpoint{4.289496in}{3.101106in}}%
\pgfpathcurveto{\pgfqpoint{4.300546in}{3.101106in}}{\pgfqpoint{4.311145in}{3.105497in}}{\pgfqpoint{4.318959in}{3.113310in}}%
\pgfpathcurveto{\pgfqpoint{4.326773in}{3.121124in}}{\pgfqpoint{4.331163in}{3.131723in}}{\pgfqpoint{4.331163in}{3.142773in}}%
\pgfpathcurveto{\pgfqpoint{4.331163in}{3.153823in}}{\pgfqpoint{4.326773in}{3.164422in}}{\pgfqpoint{4.318959in}{3.172236in}}%
\pgfpathcurveto{\pgfqpoint{4.311145in}{3.180050in}}{\pgfqpoint{4.300546in}{3.184440in}}{\pgfqpoint{4.289496in}{3.184440in}}%
\pgfpathcurveto{\pgfqpoint{4.278446in}{3.184440in}}{\pgfqpoint{4.267847in}{3.180050in}}{\pgfqpoint{4.260033in}{3.172236in}}%
\pgfpathcurveto{\pgfqpoint{4.252220in}{3.164422in}}{\pgfqpoint{4.247830in}{3.153823in}}{\pgfqpoint{4.247830in}{3.142773in}}%
\pgfpathcurveto{\pgfqpoint{4.247830in}{3.131723in}}{\pgfqpoint{4.252220in}{3.121124in}}{\pgfqpoint{4.260033in}{3.113310in}}%
\pgfpathcurveto{\pgfqpoint{4.267847in}{3.105497in}}{\pgfqpoint{4.278446in}{3.101106in}}{\pgfqpoint{4.289496in}{3.101106in}}%
\pgfpathclose%
\pgfusepath{stroke,fill}%
\end{pgfscope}%
\begin{pgfscope}%
\pgfpathrectangle{\pgfqpoint{0.481978in}{0.331635in}}{\pgfqpoint{4.960000in}{3.696000in}}%
\pgfusepath{clip}%
\pgfsetbuttcap%
\pgfsetroundjoin%
\definecolor{currentfill}{rgb}{1.000000,0.705882,0.509804}%
\pgfsetfillcolor{currentfill}%
\pgfsetlinewidth{0.481800pt}%
\definecolor{currentstroke}{rgb}{1.000000,1.000000,1.000000}%
\pgfsetstrokecolor{currentstroke}%
\pgfsetdash{}{0pt}%
\pgfpathmoveto{\pgfqpoint{3.290746in}{1.665093in}}%
\pgfpathcurveto{\pgfqpoint{3.301796in}{1.665093in}}{\pgfqpoint{3.312395in}{1.669484in}}{\pgfqpoint{3.320209in}{1.677297in}}%
\pgfpathcurveto{\pgfqpoint{3.328023in}{1.685111in}}{\pgfqpoint{3.332413in}{1.695710in}}{\pgfqpoint{3.332413in}{1.706760in}}%
\pgfpathcurveto{\pgfqpoint{3.332413in}{1.717810in}}{\pgfqpoint{3.328023in}{1.728409in}}{\pgfqpoint{3.320209in}{1.736223in}}%
\pgfpathcurveto{\pgfqpoint{3.312395in}{1.744036in}}{\pgfqpoint{3.301796in}{1.748427in}}{\pgfqpoint{3.290746in}{1.748427in}}%
\pgfpathcurveto{\pgfqpoint{3.279696in}{1.748427in}}{\pgfqpoint{3.269097in}{1.744036in}}{\pgfqpoint{3.261283in}{1.736223in}}%
\pgfpathcurveto{\pgfqpoint{3.253470in}{1.728409in}}{\pgfqpoint{3.249079in}{1.717810in}}{\pgfqpoint{3.249079in}{1.706760in}}%
\pgfpathcurveto{\pgfqpoint{3.249079in}{1.695710in}}{\pgfqpoint{3.253470in}{1.685111in}}{\pgfqpoint{3.261283in}{1.677297in}}%
\pgfpathcurveto{\pgfqpoint{3.269097in}{1.669484in}}{\pgfqpoint{3.279696in}{1.665093in}}{\pgfqpoint{3.290746in}{1.665093in}}%
\pgfpathclose%
\pgfusepath{stroke,fill}%
\end{pgfscope}%
\begin{pgfscope}%
\pgfpathrectangle{\pgfqpoint{0.481978in}{0.331635in}}{\pgfqpoint{4.960000in}{3.696000in}}%
\pgfusepath{clip}%
\pgfsetbuttcap%
\pgfsetroundjoin%
\definecolor{currentfill}{rgb}{1.000000,0.705882,0.509804}%
\pgfsetfillcolor{currentfill}%
\pgfsetlinewidth{0.481800pt}%
\definecolor{currentstroke}{rgb}{1.000000,1.000000,1.000000}%
\pgfsetstrokecolor{currentstroke}%
\pgfsetdash{}{0pt}%
\pgfpathmoveto{\pgfqpoint{4.255842in}{2.092888in}}%
\pgfpathcurveto{\pgfqpoint{4.266892in}{2.092888in}}{\pgfqpoint{4.277491in}{2.097278in}}{\pgfqpoint{4.285305in}{2.105092in}}%
\pgfpathcurveto{\pgfqpoint{4.293118in}{2.112905in}}{\pgfqpoint{4.297508in}{2.123504in}}{\pgfqpoint{4.297508in}{2.134555in}}%
\pgfpathcurveto{\pgfqpoint{4.297508in}{2.145605in}}{\pgfqpoint{4.293118in}{2.156204in}}{\pgfqpoint{4.285305in}{2.164017in}}%
\pgfpathcurveto{\pgfqpoint{4.277491in}{2.171831in}}{\pgfqpoint{4.266892in}{2.176221in}}{\pgfqpoint{4.255842in}{2.176221in}}%
\pgfpathcurveto{\pgfqpoint{4.244792in}{2.176221in}}{\pgfqpoint{4.234193in}{2.171831in}}{\pgfqpoint{4.226379in}{2.164017in}}%
\pgfpathcurveto{\pgfqpoint{4.218565in}{2.156204in}}{\pgfqpoint{4.214175in}{2.145605in}}{\pgfqpoint{4.214175in}{2.134555in}}%
\pgfpathcurveto{\pgfqpoint{4.214175in}{2.123504in}}{\pgfqpoint{4.218565in}{2.112905in}}{\pgfqpoint{4.226379in}{2.105092in}}%
\pgfpathcurveto{\pgfqpoint{4.234193in}{2.097278in}}{\pgfqpoint{4.244792in}{2.092888in}}{\pgfqpoint{4.255842in}{2.092888in}}%
\pgfpathclose%
\pgfusepath{stroke,fill}%
\end{pgfscope}%
\begin{pgfscope}%
\pgfpathrectangle{\pgfqpoint{0.481978in}{0.331635in}}{\pgfqpoint{4.960000in}{3.696000in}}%
\pgfusepath{clip}%
\pgfsetbuttcap%
\pgfsetroundjoin%
\definecolor{currentfill}{rgb}{1.000000,0.705882,0.509804}%
\pgfsetfillcolor{currentfill}%
\pgfsetlinewidth{0.481800pt}%
\definecolor{currentstroke}{rgb}{1.000000,1.000000,1.000000}%
\pgfsetstrokecolor{currentstroke}%
\pgfsetdash{}{0pt}%
\pgfpathmoveto{\pgfqpoint{2.913924in}{1.126149in}}%
\pgfpathcurveto{\pgfqpoint{2.924974in}{1.126149in}}{\pgfqpoint{2.935573in}{1.130540in}}{\pgfqpoint{2.943387in}{1.138353in}}%
\pgfpathcurveto{\pgfqpoint{2.951201in}{1.146167in}}{\pgfqpoint{2.955591in}{1.156766in}}{\pgfqpoint{2.955591in}{1.167816in}}%
\pgfpathcurveto{\pgfqpoint{2.955591in}{1.178866in}}{\pgfqpoint{2.951201in}{1.189465in}}{\pgfqpoint{2.943387in}{1.197279in}}%
\pgfpathcurveto{\pgfqpoint{2.935573in}{1.205092in}}{\pgfqpoint{2.924974in}{1.209483in}}{\pgfqpoint{2.913924in}{1.209483in}}%
\pgfpathcurveto{\pgfqpoint{2.902874in}{1.209483in}}{\pgfqpoint{2.892275in}{1.205092in}}{\pgfqpoint{2.884461in}{1.197279in}}%
\pgfpathcurveto{\pgfqpoint{2.876648in}{1.189465in}}{\pgfqpoint{2.872258in}{1.178866in}}{\pgfqpoint{2.872258in}{1.167816in}}%
\pgfpathcurveto{\pgfqpoint{2.872258in}{1.156766in}}{\pgfqpoint{2.876648in}{1.146167in}}{\pgfqpoint{2.884461in}{1.138353in}}%
\pgfpathcurveto{\pgfqpoint{2.892275in}{1.130540in}}{\pgfqpoint{2.902874in}{1.126149in}}{\pgfqpoint{2.913924in}{1.126149in}}%
\pgfpathclose%
\pgfusepath{stroke,fill}%
\end{pgfscope}%
\begin{pgfscope}%
\pgfpathrectangle{\pgfqpoint{0.481978in}{0.331635in}}{\pgfqpoint{4.960000in}{3.696000in}}%
\pgfusepath{clip}%
\pgfsetbuttcap%
\pgfsetroundjoin%
\definecolor{currentfill}{rgb}{1.000000,0.705882,0.509804}%
\pgfsetfillcolor{currentfill}%
\pgfsetlinewidth{0.481800pt}%
\definecolor{currentstroke}{rgb}{1.000000,1.000000,1.000000}%
\pgfsetstrokecolor{currentstroke}%
\pgfsetdash{}{0pt}%
\pgfpathmoveto{\pgfqpoint{3.437504in}{1.279029in}}%
\pgfpathcurveto{\pgfqpoint{3.448554in}{1.279029in}}{\pgfqpoint{3.459153in}{1.283419in}}{\pgfqpoint{3.466967in}{1.291233in}}%
\pgfpathcurveto{\pgfqpoint{3.474780in}{1.299046in}}{\pgfqpoint{3.479171in}{1.309645in}}{\pgfqpoint{3.479171in}{1.320696in}}%
\pgfpathcurveto{\pgfqpoint{3.479171in}{1.331746in}}{\pgfqpoint{3.474780in}{1.342345in}}{\pgfqpoint{3.466967in}{1.350158in}}%
\pgfpathcurveto{\pgfqpoint{3.459153in}{1.357972in}}{\pgfqpoint{3.448554in}{1.362362in}}{\pgfqpoint{3.437504in}{1.362362in}}%
\pgfpathcurveto{\pgfqpoint{3.426454in}{1.362362in}}{\pgfqpoint{3.415855in}{1.357972in}}{\pgfqpoint{3.408041in}{1.350158in}}%
\pgfpathcurveto{\pgfqpoint{3.400227in}{1.342345in}}{\pgfqpoint{3.395837in}{1.331746in}}{\pgfqpoint{3.395837in}{1.320696in}}%
\pgfpathcurveto{\pgfqpoint{3.395837in}{1.309645in}}{\pgfqpoint{3.400227in}{1.299046in}}{\pgfqpoint{3.408041in}{1.291233in}}%
\pgfpathcurveto{\pgfqpoint{3.415855in}{1.283419in}}{\pgfqpoint{3.426454in}{1.279029in}}{\pgfqpoint{3.437504in}{1.279029in}}%
\pgfpathclose%
\pgfusepath{stroke,fill}%
\end{pgfscope}%
\begin{pgfscope}%
\pgfpathrectangle{\pgfqpoint{0.481978in}{0.331635in}}{\pgfqpoint{4.960000in}{3.696000in}}%
\pgfusepath{clip}%
\pgfsetbuttcap%
\pgfsetroundjoin%
\definecolor{currentfill}{rgb}{1.000000,0.705882,0.509804}%
\pgfsetfillcolor{currentfill}%
\pgfsetlinewidth{0.481800pt}%
\definecolor{currentstroke}{rgb}{1.000000,1.000000,1.000000}%
\pgfsetstrokecolor{currentstroke}%
\pgfsetdash{}{0pt}%
\pgfpathmoveto{\pgfqpoint{2.388897in}{1.430944in}}%
\pgfpathcurveto{\pgfqpoint{2.399948in}{1.430944in}}{\pgfqpoint{2.410547in}{1.435334in}}{\pgfqpoint{2.418360in}{1.443148in}}%
\pgfpathcurveto{\pgfqpoint{2.426174in}{1.450962in}}{\pgfqpoint{2.430564in}{1.461561in}}{\pgfqpoint{2.430564in}{1.472611in}}%
\pgfpathcurveto{\pgfqpoint{2.430564in}{1.483661in}}{\pgfqpoint{2.426174in}{1.494260in}}{\pgfqpoint{2.418360in}{1.502074in}}%
\pgfpathcurveto{\pgfqpoint{2.410547in}{1.509887in}}{\pgfqpoint{2.399948in}{1.514277in}}{\pgfqpoint{2.388897in}{1.514277in}}%
\pgfpathcurveto{\pgfqpoint{2.377847in}{1.514277in}}{\pgfqpoint{2.367248in}{1.509887in}}{\pgfqpoint{2.359435in}{1.502074in}}%
\pgfpathcurveto{\pgfqpoint{2.351621in}{1.494260in}}{\pgfqpoint{2.347231in}{1.483661in}}{\pgfqpoint{2.347231in}{1.472611in}}%
\pgfpathcurveto{\pgfqpoint{2.347231in}{1.461561in}}{\pgfqpoint{2.351621in}{1.450962in}}{\pgfqpoint{2.359435in}{1.443148in}}%
\pgfpathcurveto{\pgfqpoint{2.367248in}{1.435334in}}{\pgfqpoint{2.377847in}{1.430944in}}{\pgfqpoint{2.388897in}{1.430944in}}%
\pgfpathclose%
\pgfusepath{stroke,fill}%
\end{pgfscope}%
\begin{pgfscope}%
\pgfpathrectangle{\pgfqpoint{0.481978in}{0.331635in}}{\pgfqpoint{4.960000in}{3.696000in}}%
\pgfusepath{clip}%
\pgfsetbuttcap%
\pgfsetroundjoin%
\definecolor{currentfill}{rgb}{1.000000,0.705882,0.509804}%
\pgfsetfillcolor{currentfill}%
\pgfsetlinewidth{0.481800pt}%
\definecolor{currentstroke}{rgb}{1.000000,1.000000,1.000000}%
\pgfsetstrokecolor{currentstroke}%
\pgfsetdash{}{0pt}%
\pgfpathmoveto{\pgfqpoint{1.220987in}{1.743405in}}%
\pgfpathcurveto{\pgfqpoint{1.232037in}{1.743405in}}{\pgfqpoint{1.242636in}{1.747796in}}{\pgfqpoint{1.250450in}{1.755609in}}%
\pgfpathcurveto{\pgfqpoint{1.258263in}{1.763423in}}{\pgfqpoint{1.262654in}{1.774022in}}{\pgfqpoint{1.262654in}{1.785072in}}%
\pgfpathcurveto{\pgfqpoint{1.262654in}{1.796122in}}{\pgfqpoint{1.258263in}{1.806721in}}{\pgfqpoint{1.250450in}{1.814535in}}%
\pgfpathcurveto{\pgfqpoint{1.242636in}{1.822348in}}{\pgfqpoint{1.232037in}{1.826739in}}{\pgfqpoint{1.220987in}{1.826739in}}%
\pgfpathcurveto{\pgfqpoint{1.209937in}{1.826739in}}{\pgfqpoint{1.199338in}{1.822348in}}{\pgfqpoint{1.191524in}{1.814535in}}%
\pgfpathcurveto{\pgfqpoint{1.183711in}{1.806721in}}{\pgfqpoint{1.179320in}{1.796122in}}{\pgfqpoint{1.179320in}{1.785072in}}%
\pgfpathcurveto{\pgfqpoint{1.179320in}{1.774022in}}{\pgfqpoint{1.183711in}{1.763423in}}{\pgfqpoint{1.191524in}{1.755609in}}%
\pgfpathcurveto{\pgfqpoint{1.199338in}{1.747796in}}{\pgfqpoint{1.209937in}{1.743405in}}{\pgfqpoint{1.220987in}{1.743405in}}%
\pgfpathclose%
\pgfusepath{stroke,fill}%
\end{pgfscope}%
\begin{pgfscope}%
\pgfpathrectangle{\pgfqpoint{0.481978in}{0.331635in}}{\pgfqpoint{4.960000in}{3.696000in}}%
\pgfusepath{clip}%
\pgfsetbuttcap%
\pgfsetroundjoin%
\definecolor{currentfill}{rgb}{1.000000,0.705882,0.509804}%
\pgfsetfillcolor{currentfill}%
\pgfsetlinewidth{0.481800pt}%
\definecolor{currentstroke}{rgb}{1.000000,1.000000,1.000000}%
\pgfsetstrokecolor{currentstroke}%
\pgfsetdash{}{0pt}%
\pgfpathmoveto{\pgfqpoint{2.981752in}{1.496053in}}%
\pgfpathcurveto{\pgfqpoint{2.992802in}{1.496053in}}{\pgfqpoint{3.003401in}{1.500443in}}{\pgfqpoint{3.011215in}{1.508256in}}%
\pgfpathcurveto{\pgfqpoint{3.019028in}{1.516070in}}{\pgfqpoint{3.023419in}{1.526669in}}{\pgfqpoint{3.023419in}{1.537719in}}%
\pgfpathcurveto{\pgfqpoint{3.023419in}{1.548769in}}{\pgfqpoint{3.019028in}{1.559368in}}{\pgfqpoint{3.011215in}{1.567182in}}%
\pgfpathcurveto{\pgfqpoint{3.003401in}{1.574996in}}{\pgfqpoint{2.992802in}{1.579386in}}{\pgfqpoint{2.981752in}{1.579386in}}%
\pgfpathcurveto{\pgfqpoint{2.970702in}{1.579386in}}{\pgfqpoint{2.960103in}{1.574996in}}{\pgfqpoint{2.952289in}{1.567182in}}%
\pgfpathcurveto{\pgfqpoint{2.944475in}{1.559368in}}{\pgfqpoint{2.940085in}{1.548769in}}{\pgfqpoint{2.940085in}{1.537719in}}%
\pgfpathcurveto{\pgfqpoint{2.940085in}{1.526669in}}{\pgfqpoint{2.944475in}{1.516070in}}{\pgfqpoint{2.952289in}{1.508256in}}%
\pgfpathcurveto{\pgfqpoint{2.960103in}{1.500443in}}{\pgfqpoint{2.970702in}{1.496053in}}{\pgfqpoint{2.981752in}{1.496053in}}%
\pgfpathclose%
\pgfusepath{stroke,fill}%
\end{pgfscope}%
\begin{pgfscope}%
\pgfpathrectangle{\pgfqpoint{0.481978in}{0.331635in}}{\pgfqpoint{4.960000in}{3.696000in}}%
\pgfusepath{clip}%
\pgfsetbuttcap%
\pgfsetroundjoin%
\definecolor{currentfill}{rgb}{1.000000,0.705882,0.509804}%
\pgfsetfillcolor{currentfill}%
\pgfsetlinewidth{0.481800pt}%
\definecolor{currentstroke}{rgb}{1.000000,1.000000,1.000000}%
\pgfsetstrokecolor{currentstroke}%
\pgfsetdash{}{0pt}%
\pgfpathmoveto{\pgfqpoint{4.627519in}{2.176750in}}%
\pgfpathcurveto{\pgfqpoint{4.638570in}{2.176750in}}{\pgfqpoint{4.649169in}{2.181140in}}{\pgfqpoint{4.656982in}{2.188954in}}%
\pgfpathcurveto{\pgfqpoint{4.664796in}{2.196768in}}{\pgfqpoint{4.669186in}{2.207367in}}{\pgfqpoint{4.669186in}{2.218417in}}%
\pgfpathcurveto{\pgfqpoint{4.669186in}{2.229467in}}{\pgfqpoint{4.664796in}{2.240066in}}{\pgfqpoint{4.656982in}{2.247880in}}%
\pgfpathcurveto{\pgfqpoint{4.649169in}{2.255693in}}{\pgfqpoint{4.638570in}{2.260084in}}{\pgfqpoint{4.627519in}{2.260084in}}%
\pgfpathcurveto{\pgfqpoint{4.616469in}{2.260084in}}{\pgfqpoint{4.605870in}{2.255693in}}{\pgfqpoint{4.598057in}{2.247880in}}%
\pgfpathcurveto{\pgfqpoint{4.590243in}{2.240066in}}{\pgfqpoint{4.585853in}{2.229467in}}{\pgfqpoint{4.585853in}{2.218417in}}%
\pgfpathcurveto{\pgfqpoint{4.585853in}{2.207367in}}{\pgfqpoint{4.590243in}{2.196768in}}{\pgfqpoint{4.598057in}{2.188954in}}%
\pgfpathcurveto{\pgfqpoint{4.605870in}{2.181140in}}{\pgfqpoint{4.616469in}{2.176750in}}{\pgfqpoint{4.627519in}{2.176750in}}%
\pgfpathclose%
\pgfusepath{stroke,fill}%
\end{pgfscope}%
\begin{pgfscope}%
\pgfpathrectangle{\pgfqpoint{0.481978in}{0.331635in}}{\pgfqpoint{4.960000in}{3.696000in}}%
\pgfusepath{clip}%
\pgfsetbuttcap%
\pgfsetroundjoin%
\definecolor{currentfill}{rgb}{1.000000,0.705882,0.509804}%
\pgfsetfillcolor{currentfill}%
\pgfsetlinewidth{0.481800pt}%
\definecolor{currentstroke}{rgb}{1.000000,1.000000,1.000000}%
\pgfsetstrokecolor{currentstroke}%
\pgfsetdash{}{0pt}%
\pgfpathmoveto{\pgfqpoint{2.759699in}{1.442163in}}%
\pgfpathcurveto{\pgfqpoint{2.770749in}{1.442163in}}{\pgfqpoint{2.781348in}{1.446553in}}{\pgfqpoint{2.789161in}{1.454367in}}%
\pgfpathcurveto{\pgfqpoint{2.796975in}{1.462180in}}{\pgfqpoint{2.801365in}{1.472780in}}{\pgfqpoint{2.801365in}{1.483830in}}%
\pgfpathcurveto{\pgfqpoint{2.801365in}{1.494880in}}{\pgfqpoint{2.796975in}{1.505479in}}{\pgfqpoint{2.789161in}{1.513292in}}%
\pgfpathcurveto{\pgfqpoint{2.781348in}{1.521106in}}{\pgfqpoint{2.770749in}{1.525496in}}{\pgfqpoint{2.759699in}{1.525496in}}%
\pgfpathcurveto{\pgfqpoint{2.748648in}{1.525496in}}{\pgfqpoint{2.738049in}{1.521106in}}{\pgfqpoint{2.730236in}{1.513292in}}%
\pgfpathcurveto{\pgfqpoint{2.722422in}{1.505479in}}{\pgfqpoint{2.718032in}{1.494880in}}{\pgfqpoint{2.718032in}{1.483830in}}%
\pgfpathcurveto{\pgfqpoint{2.718032in}{1.472780in}}{\pgfqpoint{2.722422in}{1.462180in}}{\pgfqpoint{2.730236in}{1.454367in}}%
\pgfpathcurveto{\pgfqpoint{2.738049in}{1.446553in}}{\pgfqpoint{2.748648in}{1.442163in}}{\pgfqpoint{2.759699in}{1.442163in}}%
\pgfpathclose%
\pgfusepath{stroke,fill}%
\end{pgfscope}%
\begin{pgfscope}%
\pgfpathrectangle{\pgfqpoint{0.481978in}{0.331635in}}{\pgfqpoint{4.960000in}{3.696000in}}%
\pgfusepath{clip}%
\pgfsetbuttcap%
\pgfsetroundjoin%
\definecolor{currentfill}{rgb}{1.000000,0.705882,0.509804}%
\pgfsetfillcolor{currentfill}%
\pgfsetlinewidth{0.481800pt}%
\definecolor{currentstroke}{rgb}{1.000000,1.000000,1.000000}%
\pgfsetstrokecolor{currentstroke}%
\pgfsetdash{}{0pt}%
\pgfpathmoveto{\pgfqpoint{3.467601in}{1.095952in}}%
\pgfpathcurveto{\pgfqpoint{3.478651in}{1.095952in}}{\pgfqpoint{3.489250in}{1.100342in}}{\pgfqpoint{3.497064in}{1.108155in}}%
\pgfpathcurveto{\pgfqpoint{3.504878in}{1.115969in}}{\pgfqpoint{3.509268in}{1.126568in}}{\pgfqpoint{3.509268in}{1.137618in}}%
\pgfpathcurveto{\pgfqpoint{3.509268in}{1.148668in}}{\pgfqpoint{3.504878in}{1.159267in}}{\pgfqpoint{3.497064in}{1.167081in}}%
\pgfpathcurveto{\pgfqpoint{3.489250in}{1.174895in}}{\pgfqpoint{3.478651in}{1.179285in}}{\pgfqpoint{3.467601in}{1.179285in}}%
\pgfpathcurveto{\pgfqpoint{3.456551in}{1.179285in}}{\pgfqpoint{3.445952in}{1.174895in}}{\pgfqpoint{3.438138in}{1.167081in}}%
\pgfpathcurveto{\pgfqpoint{3.430325in}{1.159267in}}{\pgfqpoint{3.425935in}{1.148668in}}{\pgfqpoint{3.425935in}{1.137618in}}%
\pgfpathcurveto{\pgfqpoint{3.425935in}{1.126568in}}{\pgfqpoint{3.430325in}{1.115969in}}{\pgfqpoint{3.438138in}{1.108155in}}%
\pgfpathcurveto{\pgfqpoint{3.445952in}{1.100342in}}{\pgfqpoint{3.456551in}{1.095952in}}{\pgfqpoint{3.467601in}{1.095952in}}%
\pgfpathclose%
\pgfusepath{stroke,fill}%
\end{pgfscope}%
\begin{pgfscope}%
\pgfpathrectangle{\pgfqpoint{0.481978in}{0.331635in}}{\pgfqpoint{4.960000in}{3.696000in}}%
\pgfusepath{clip}%
\pgfsetbuttcap%
\pgfsetroundjoin%
\definecolor{currentfill}{rgb}{1.000000,0.705882,0.509804}%
\pgfsetfillcolor{currentfill}%
\pgfsetlinewidth{0.481800pt}%
\definecolor{currentstroke}{rgb}{1.000000,1.000000,1.000000}%
\pgfsetstrokecolor{currentstroke}%
\pgfsetdash{}{0pt}%
\pgfpathmoveto{\pgfqpoint{1.541084in}{1.598857in}}%
\pgfpathcurveto{\pgfqpoint{1.552134in}{1.598857in}}{\pgfqpoint{1.562733in}{1.603248in}}{\pgfqpoint{1.570547in}{1.611061in}}%
\pgfpathcurveto{\pgfqpoint{1.578361in}{1.618875in}}{\pgfqpoint{1.582751in}{1.629474in}}{\pgfqpoint{1.582751in}{1.640524in}}%
\pgfpathcurveto{\pgfqpoint{1.582751in}{1.651574in}}{\pgfqpoint{1.578361in}{1.662173in}}{\pgfqpoint{1.570547in}{1.669987in}}%
\pgfpathcurveto{\pgfqpoint{1.562733in}{1.677800in}}{\pgfqpoint{1.552134in}{1.682191in}}{\pgfqpoint{1.541084in}{1.682191in}}%
\pgfpathcurveto{\pgfqpoint{1.530034in}{1.682191in}}{\pgfqpoint{1.519435in}{1.677800in}}{\pgfqpoint{1.511622in}{1.669987in}}%
\pgfpathcurveto{\pgfqpoint{1.503808in}{1.662173in}}{\pgfqpoint{1.499418in}{1.651574in}}{\pgfqpoint{1.499418in}{1.640524in}}%
\pgfpathcurveto{\pgfqpoint{1.499418in}{1.629474in}}{\pgfqpoint{1.503808in}{1.618875in}}{\pgfqpoint{1.511622in}{1.611061in}}%
\pgfpathcurveto{\pgfqpoint{1.519435in}{1.603248in}}{\pgfqpoint{1.530034in}{1.598857in}}{\pgfqpoint{1.541084in}{1.598857in}}%
\pgfpathclose%
\pgfusepath{stroke,fill}%
\end{pgfscope}%
\begin{pgfscope}%
\pgfpathrectangle{\pgfqpoint{0.481978in}{0.331635in}}{\pgfqpoint{4.960000in}{3.696000in}}%
\pgfusepath{clip}%
\pgfsetbuttcap%
\pgfsetroundjoin%
\definecolor{currentfill}{rgb}{1.000000,0.705882,0.509804}%
\pgfsetfillcolor{currentfill}%
\pgfsetlinewidth{0.481800pt}%
\definecolor{currentstroke}{rgb}{1.000000,1.000000,1.000000}%
\pgfsetstrokecolor{currentstroke}%
\pgfsetdash{}{0pt}%
\pgfpathmoveto{\pgfqpoint{3.501411in}{2.237385in}}%
\pgfpathcurveto{\pgfqpoint{3.512461in}{2.237385in}}{\pgfqpoint{3.523060in}{2.241775in}}{\pgfqpoint{3.530874in}{2.249589in}}%
\pgfpathcurveto{\pgfqpoint{3.538687in}{2.257403in}}{\pgfqpoint{3.543077in}{2.268002in}}{\pgfqpoint{3.543077in}{2.279052in}}%
\pgfpathcurveto{\pgfqpoint{3.543077in}{2.290102in}}{\pgfqpoint{3.538687in}{2.300701in}}{\pgfqpoint{3.530874in}{2.308515in}}%
\pgfpathcurveto{\pgfqpoint{3.523060in}{2.316328in}}{\pgfqpoint{3.512461in}{2.320718in}}{\pgfqpoint{3.501411in}{2.320718in}}%
\pgfpathcurveto{\pgfqpoint{3.490361in}{2.320718in}}{\pgfqpoint{3.479762in}{2.316328in}}{\pgfqpoint{3.471948in}{2.308515in}}%
\pgfpathcurveto{\pgfqpoint{3.464134in}{2.300701in}}{\pgfqpoint{3.459744in}{2.290102in}}{\pgfqpoint{3.459744in}{2.279052in}}%
\pgfpathcurveto{\pgfqpoint{3.459744in}{2.268002in}}{\pgfqpoint{3.464134in}{2.257403in}}{\pgfqpoint{3.471948in}{2.249589in}}%
\pgfpathcurveto{\pgfqpoint{3.479762in}{2.241775in}}{\pgfqpoint{3.490361in}{2.237385in}}{\pgfqpoint{3.501411in}{2.237385in}}%
\pgfpathclose%
\pgfusepath{stroke,fill}%
\end{pgfscope}%
\begin{pgfscope}%
\pgfpathrectangle{\pgfqpoint{0.481978in}{0.331635in}}{\pgfqpoint{4.960000in}{3.696000in}}%
\pgfusepath{clip}%
\pgfsetbuttcap%
\pgfsetroundjoin%
\definecolor{currentfill}{rgb}{1.000000,0.705882,0.509804}%
\pgfsetfillcolor{currentfill}%
\pgfsetlinewidth{0.481800pt}%
\definecolor{currentstroke}{rgb}{1.000000,1.000000,1.000000}%
\pgfsetstrokecolor{currentstroke}%
\pgfsetdash{}{0pt}%
\pgfpathmoveto{\pgfqpoint{3.324129in}{1.499035in}}%
\pgfpathcurveto{\pgfqpoint{3.335179in}{1.499035in}}{\pgfqpoint{3.345778in}{1.503426in}}{\pgfqpoint{3.353592in}{1.511239in}}%
\pgfpathcurveto{\pgfqpoint{3.361405in}{1.519053in}}{\pgfqpoint{3.365796in}{1.529652in}}{\pgfqpoint{3.365796in}{1.540702in}}%
\pgfpathcurveto{\pgfqpoint{3.365796in}{1.551752in}}{\pgfqpoint{3.361405in}{1.562351in}}{\pgfqpoint{3.353592in}{1.570165in}}%
\pgfpathcurveto{\pgfqpoint{3.345778in}{1.577979in}}{\pgfqpoint{3.335179in}{1.582369in}}{\pgfqpoint{3.324129in}{1.582369in}}%
\pgfpathcurveto{\pgfqpoint{3.313079in}{1.582369in}}{\pgfqpoint{3.302480in}{1.577979in}}{\pgfqpoint{3.294666in}{1.570165in}}%
\pgfpathcurveto{\pgfqpoint{3.286852in}{1.562351in}}{\pgfqpoint{3.282462in}{1.551752in}}{\pgfqpoint{3.282462in}{1.540702in}}%
\pgfpathcurveto{\pgfqpoint{3.282462in}{1.529652in}}{\pgfqpoint{3.286852in}{1.519053in}}{\pgfqpoint{3.294666in}{1.511239in}}%
\pgfpathcurveto{\pgfqpoint{3.302480in}{1.503426in}}{\pgfqpoint{3.313079in}{1.499035in}}{\pgfqpoint{3.324129in}{1.499035in}}%
\pgfpathclose%
\pgfusepath{stroke,fill}%
\end{pgfscope}%
\begin{pgfscope}%
\pgfpathrectangle{\pgfqpoint{0.481978in}{0.331635in}}{\pgfqpoint{4.960000in}{3.696000in}}%
\pgfusepath{clip}%
\pgfsetbuttcap%
\pgfsetroundjoin%
\definecolor{currentfill}{rgb}{1.000000,0.705882,0.509804}%
\pgfsetfillcolor{currentfill}%
\pgfsetlinewidth{0.481800pt}%
\definecolor{currentstroke}{rgb}{1.000000,1.000000,1.000000}%
\pgfsetstrokecolor{currentstroke}%
\pgfsetdash{}{0pt}%
\pgfpathmoveto{\pgfqpoint{3.150076in}{1.329362in}}%
\pgfpathcurveto{\pgfqpoint{3.161127in}{1.329362in}}{\pgfqpoint{3.171726in}{1.333752in}}{\pgfqpoint{3.179539in}{1.341566in}}%
\pgfpathcurveto{\pgfqpoint{3.187353in}{1.349379in}}{\pgfqpoint{3.191743in}{1.359978in}}{\pgfqpoint{3.191743in}{1.371028in}}%
\pgfpathcurveto{\pgfqpoint{3.191743in}{1.382079in}}{\pgfqpoint{3.187353in}{1.392678in}}{\pgfqpoint{3.179539in}{1.400491in}}%
\pgfpathcurveto{\pgfqpoint{3.171726in}{1.408305in}}{\pgfqpoint{3.161127in}{1.412695in}}{\pgfqpoint{3.150076in}{1.412695in}}%
\pgfpathcurveto{\pgfqpoint{3.139026in}{1.412695in}}{\pgfqpoint{3.128427in}{1.408305in}}{\pgfqpoint{3.120614in}{1.400491in}}%
\pgfpathcurveto{\pgfqpoint{3.112800in}{1.392678in}}{\pgfqpoint{3.108410in}{1.382079in}}{\pgfqpoint{3.108410in}{1.371028in}}%
\pgfpathcurveto{\pgfqpoint{3.108410in}{1.359978in}}{\pgfqpoint{3.112800in}{1.349379in}}{\pgfqpoint{3.120614in}{1.341566in}}%
\pgfpathcurveto{\pgfqpoint{3.128427in}{1.333752in}}{\pgfqpoint{3.139026in}{1.329362in}}{\pgfqpoint{3.150076in}{1.329362in}}%
\pgfpathclose%
\pgfusepath{stroke,fill}%
\end{pgfscope}%
\begin{pgfscope}%
\pgfpathrectangle{\pgfqpoint{0.481978in}{0.331635in}}{\pgfqpoint{4.960000in}{3.696000in}}%
\pgfusepath{clip}%
\pgfsetbuttcap%
\pgfsetroundjoin%
\definecolor{currentfill}{rgb}{1.000000,0.705882,0.509804}%
\pgfsetfillcolor{currentfill}%
\pgfsetlinewidth{0.481800pt}%
\definecolor{currentstroke}{rgb}{1.000000,1.000000,1.000000}%
\pgfsetstrokecolor{currentstroke}%
\pgfsetdash{}{0pt}%
\pgfpathmoveto{\pgfqpoint{2.909368in}{1.428075in}}%
\pgfpathcurveto{\pgfqpoint{2.920418in}{1.428075in}}{\pgfqpoint{2.931017in}{1.432465in}}{\pgfqpoint{2.938830in}{1.440279in}}%
\pgfpathcurveto{\pgfqpoint{2.946644in}{1.448093in}}{\pgfqpoint{2.951034in}{1.458692in}}{\pgfqpoint{2.951034in}{1.469742in}}%
\pgfpathcurveto{\pgfqpoint{2.951034in}{1.480792in}}{\pgfqpoint{2.946644in}{1.491391in}}{\pgfqpoint{2.938830in}{1.499205in}}%
\pgfpathcurveto{\pgfqpoint{2.931017in}{1.507018in}}{\pgfqpoint{2.920418in}{1.511408in}}{\pgfqpoint{2.909368in}{1.511408in}}%
\pgfpathcurveto{\pgfqpoint{2.898318in}{1.511408in}}{\pgfqpoint{2.887719in}{1.507018in}}{\pgfqpoint{2.879905in}{1.499205in}}%
\pgfpathcurveto{\pgfqpoint{2.872091in}{1.491391in}}{\pgfqpoint{2.867701in}{1.480792in}}{\pgfqpoint{2.867701in}{1.469742in}}%
\pgfpathcurveto{\pgfqpoint{2.867701in}{1.458692in}}{\pgfqpoint{2.872091in}{1.448093in}}{\pgfqpoint{2.879905in}{1.440279in}}%
\pgfpathcurveto{\pgfqpoint{2.887719in}{1.432465in}}{\pgfqpoint{2.898318in}{1.428075in}}{\pgfqpoint{2.909368in}{1.428075in}}%
\pgfpathclose%
\pgfusepath{stroke,fill}%
\end{pgfscope}%
\begin{pgfscope}%
\pgfpathrectangle{\pgfqpoint{0.481978in}{0.331635in}}{\pgfqpoint{4.960000in}{3.696000in}}%
\pgfusepath{clip}%
\pgfsetbuttcap%
\pgfsetroundjoin%
\definecolor{currentfill}{rgb}{1.000000,0.705882,0.509804}%
\pgfsetfillcolor{currentfill}%
\pgfsetlinewidth{0.481800pt}%
\definecolor{currentstroke}{rgb}{1.000000,1.000000,1.000000}%
\pgfsetstrokecolor{currentstroke}%
\pgfsetdash{}{0pt}%
\pgfpathmoveto{\pgfqpoint{1.954857in}{1.010997in}}%
\pgfpathcurveto{\pgfqpoint{1.965907in}{1.010997in}}{\pgfqpoint{1.976506in}{1.015387in}}{\pgfqpoint{1.984319in}{1.023200in}}%
\pgfpathcurveto{\pgfqpoint{1.992133in}{1.031014in}}{\pgfqpoint{1.996523in}{1.041613in}}{\pgfqpoint{1.996523in}{1.052663in}}%
\pgfpathcurveto{\pgfqpoint{1.996523in}{1.063713in}}{\pgfqpoint{1.992133in}{1.074312in}}{\pgfqpoint{1.984319in}{1.082126in}}%
\pgfpathcurveto{\pgfqpoint{1.976506in}{1.089940in}}{\pgfqpoint{1.965907in}{1.094330in}}{\pgfqpoint{1.954857in}{1.094330in}}%
\pgfpathcurveto{\pgfqpoint{1.943807in}{1.094330in}}{\pgfqpoint{1.933208in}{1.089940in}}{\pgfqpoint{1.925394in}{1.082126in}}%
\pgfpathcurveto{\pgfqpoint{1.917580in}{1.074312in}}{\pgfqpoint{1.913190in}{1.063713in}}{\pgfqpoint{1.913190in}{1.052663in}}%
\pgfpathcurveto{\pgfqpoint{1.913190in}{1.041613in}}{\pgfqpoint{1.917580in}{1.031014in}}{\pgfqpoint{1.925394in}{1.023200in}}%
\pgfpathcurveto{\pgfqpoint{1.933208in}{1.015387in}}{\pgfqpoint{1.943807in}{1.010997in}}{\pgfqpoint{1.954857in}{1.010997in}}%
\pgfpathclose%
\pgfusepath{stroke,fill}%
\end{pgfscope}%
\begin{pgfscope}%
\pgfpathrectangle{\pgfqpoint{0.481978in}{0.331635in}}{\pgfqpoint{4.960000in}{3.696000in}}%
\pgfusepath{clip}%
\pgfsetbuttcap%
\pgfsetroundjoin%
\definecolor{currentfill}{rgb}{1.000000,0.705882,0.509804}%
\pgfsetfillcolor{currentfill}%
\pgfsetlinewidth{0.481800pt}%
\definecolor{currentstroke}{rgb}{1.000000,1.000000,1.000000}%
\pgfsetstrokecolor{currentstroke}%
\pgfsetdash{}{0pt}%
\pgfpathmoveto{\pgfqpoint{3.586905in}{1.557950in}}%
\pgfpathcurveto{\pgfqpoint{3.597955in}{1.557950in}}{\pgfqpoint{3.608554in}{1.562340in}}{\pgfqpoint{3.616367in}{1.570154in}}%
\pgfpathcurveto{\pgfqpoint{3.624181in}{1.577967in}}{\pgfqpoint{3.628571in}{1.588566in}}{\pgfqpoint{3.628571in}{1.599616in}}%
\pgfpathcurveto{\pgfqpoint{3.628571in}{1.610667in}}{\pgfqpoint{3.624181in}{1.621266in}}{\pgfqpoint{3.616367in}{1.629079in}}%
\pgfpathcurveto{\pgfqpoint{3.608554in}{1.636893in}}{\pgfqpoint{3.597955in}{1.641283in}}{\pgfqpoint{3.586905in}{1.641283in}}%
\pgfpathcurveto{\pgfqpoint{3.575855in}{1.641283in}}{\pgfqpoint{3.565256in}{1.636893in}}{\pgfqpoint{3.557442in}{1.629079in}}%
\pgfpathcurveto{\pgfqpoint{3.549628in}{1.621266in}}{\pgfqpoint{3.545238in}{1.610667in}}{\pgfqpoint{3.545238in}{1.599616in}}%
\pgfpathcurveto{\pgfqpoint{3.545238in}{1.588566in}}{\pgfqpoint{3.549628in}{1.577967in}}{\pgfqpoint{3.557442in}{1.570154in}}%
\pgfpathcurveto{\pgfqpoint{3.565256in}{1.562340in}}{\pgfqpoint{3.575855in}{1.557950in}}{\pgfqpoint{3.586905in}{1.557950in}}%
\pgfpathclose%
\pgfusepath{stroke,fill}%
\end{pgfscope}%
\begin{pgfscope}%
\pgfpathrectangle{\pgfqpoint{0.481978in}{0.331635in}}{\pgfqpoint{4.960000in}{3.696000in}}%
\pgfusepath{clip}%
\pgfsetbuttcap%
\pgfsetroundjoin%
\definecolor{currentfill}{rgb}{1.000000,0.705882,0.509804}%
\pgfsetfillcolor{currentfill}%
\pgfsetlinewidth{0.481800pt}%
\definecolor{currentstroke}{rgb}{1.000000,1.000000,1.000000}%
\pgfsetstrokecolor{currentstroke}%
\pgfsetdash{}{0pt}%
\pgfpathmoveto{\pgfqpoint{3.867423in}{2.893676in}}%
\pgfpathcurveto{\pgfqpoint{3.878473in}{2.893676in}}{\pgfqpoint{3.889072in}{2.898067in}}{\pgfqpoint{3.896886in}{2.905880in}}%
\pgfpathcurveto{\pgfqpoint{3.904699in}{2.913694in}}{\pgfqpoint{3.909089in}{2.924293in}}{\pgfqpoint{3.909089in}{2.935343in}}%
\pgfpathcurveto{\pgfqpoint{3.909089in}{2.946393in}}{\pgfqpoint{3.904699in}{2.956992in}}{\pgfqpoint{3.896886in}{2.964806in}}%
\pgfpathcurveto{\pgfqpoint{3.889072in}{2.972619in}}{\pgfqpoint{3.878473in}{2.977010in}}{\pgfqpoint{3.867423in}{2.977010in}}%
\pgfpathcurveto{\pgfqpoint{3.856373in}{2.977010in}}{\pgfqpoint{3.845774in}{2.972619in}}{\pgfqpoint{3.837960in}{2.964806in}}%
\pgfpathcurveto{\pgfqpoint{3.830146in}{2.956992in}}{\pgfqpoint{3.825756in}{2.946393in}}{\pgfqpoint{3.825756in}{2.935343in}}%
\pgfpathcurveto{\pgfqpoint{3.825756in}{2.924293in}}{\pgfqpoint{3.830146in}{2.913694in}}{\pgfqpoint{3.837960in}{2.905880in}}%
\pgfpathcurveto{\pgfqpoint{3.845774in}{2.898067in}}{\pgfqpoint{3.856373in}{2.893676in}}{\pgfqpoint{3.867423in}{2.893676in}}%
\pgfpathclose%
\pgfusepath{stroke,fill}%
\end{pgfscope}%
\begin{pgfscope}%
\pgfpathrectangle{\pgfqpoint{0.481978in}{0.331635in}}{\pgfqpoint{4.960000in}{3.696000in}}%
\pgfusepath{clip}%
\pgfsetbuttcap%
\pgfsetroundjoin%
\definecolor{currentfill}{rgb}{1.000000,0.705882,0.509804}%
\pgfsetfillcolor{currentfill}%
\pgfsetlinewidth{0.481800pt}%
\definecolor{currentstroke}{rgb}{1.000000,1.000000,1.000000}%
\pgfsetstrokecolor{currentstroke}%
\pgfsetdash{}{0pt}%
\pgfpathmoveto{\pgfqpoint{1.032296in}{1.270369in}}%
\pgfpathcurveto{\pgfqpoint{1.043346in}{1.270369in}}{\pgfqpoint{1.053945in}{1.274760in}}{\pgfqpoint{1.061758in}{1.282573in}}%
\pgfpathcurveto{\pgfqpoint{1.069572in}{1.290387in}}{\pgfqpoint{1.073962in}{1.300986in}}{\pgfqpoint{1.073962in}{1.312036in}}%
\pgfpathcurveto{\pgfqpoint{1.073962in}{1.323086in}}{\pgfqpoint{1.069572in}{1.333685in}}{\pgfqpoint{1.061758in}{1.341499in}}%
\pgfpathcurveto{\pgfqpoint{1.053945in}{1.349312in}}{\pgfqpoint{1.043346in}{1.353703in}}{\pgfqpoint{1.032296in}{1.353703in}}%
\pgfpathcurveto{\pgfqpoint{1.021246in}{1.353703in}}{\pgfqpoint{1.010646in}{1.349312in}}{\pgfqpoint{1.002833in}{1.341499in}}%
\pgfpathcurveto{\pgfqpoint{0.995019in}{1.333685in}}{\pgfqpoint{0.990629in}{1.323086in}}{\pgfqpoint{0.990629in}{1.312036in}}%
\pgfpathcurveto{\pgfqpoint{0.990629in}{1.300986in}}{\pgfqpoint{0.995019in}{1.290387in}}{\pgfqpoint{1.002833in}{1.282573in}}%
\pgfpathcurveto{\pgfqpoint{1.010646in}{1.274760in}}{\pgfqpoint{1.021246in}{1.270369in}}{\pgfqpoint{1.032296in}{1.270369in}}%
\pgfpathclose%
\pgfusepath{stroke,fill}%
\end{pgfscope}%
\begin{pgfscope}%
\pgfpathrectangle{\pgfqpoint{0.481978in}{0.331635in}}{\pgfqpoint{4.960000in}{3.696000in}}%
\pgfusepath{clip}%
\pgfsetbuttcap%
\pgfsetroundjoin%
\definecolor{currentfill}{rgb}{1.000000,0.705882,0.509804}%
\pgfsetfillcolor{currentfill}%
\pgfsetlinewidth{0.481800pt}%
\definecolor{currentstroke}{rgb}{1.000000,1.000000,1.000000}%
\pgfsetstrokecolor{currentstroke}%
\pgfsetdash{}{0pt}%
\pgfpathmoveto{\pgfqpoint{1.491483in}{1.522307in}}%
\pgfpathcurveto{\pgfqpoint{1.502533in}{1.522307in}}{\pgfqpoint{1.513133in}{1.526697in}}{\pgfqpoint{1.520946in}{1.534511in}}%
\pgfpathcurveto{\pgfqpoint{1.528760in}{1.542324in}}{\pgfqpoint{1.533150in}{1.552923in}}{\pgfqpoint{1.533150in}{1.563973in}}%
\pgfpathcurveto{\pgfqpoint{1.533150in}{1.575023in}}{\pgfqpoint{1.528760in}{1.585622in}}{\pgfqpoint{1.520946in}{1.593436in}}%
\pgfpathcurveto{\pgfqpoint{1.513133in}{1.601250in}}{\pgfqpoint{1.502533in}{1.605640in}}{\pgfqpoint{1.491483in}{1.605640in}}%
\pgfpathcurveto{\pgfqpoint{1.480433in}{1.605640in}}{\pgfqpoint{1.469834in}{1.601250in}}{\pgfqpoint{1.462021in}{1.593436in}}%
\pgfpathcurveto{\pgfqpoint{1.454207in}{1.585622in}}{\pgfqpoint{1.449817in}{1.575023in}}{\pgfqpoint{1.449817in}{1.563973in}}%
\pgfpathcurveto{\pgfqpoint{1.449817in}{1.552923in}}{\pgfqpoint{1.454207in}{1.542324in}}{\pgfqpoint{1.462021in}{1.534511in}}%
\pgfpathcurveto{\pgfqpoint{1.469834in}{1.526697in}}{\pgfqpoint{1.480433in}{1.522307in}}{\pgfqpoint{1.491483in}{1.522307in}}%
\pgfpathclose%
\pgfusepath{stroke,fill}%
\end{pgfscope}%
\begin{pgfscope}%
\pgfpathrectangle{\pgfqpoint{0.481978in}{0.331635in}}{\pgfqpoint{4.960000in}{3.696000in}}%
\pgfusepath{clip}%
\pgfsetbuttcap%
\pgfsetroundjoin%
\definecolor{currentfill}{rgb}{1.000000,0.705882,0.509804}%
\pgfsetfillcolor{currentfill}%
\pgfsetlinewidth{0.481800pt}%
\definecolor{currentstroke}{rgb}{1.000000,1.000000,1.000000}%
\pgfsetstrokecolor{currentstroke}%
\pgfsetdash{}{0pt}%
\pgfpathmoveto{\pgfqpoint{2.597836in}{1.377394in}}%
\pgfpathcurveto{\pgfqpoint{2.608886in}{1.377394in}}{\pgfqpoint{2.619485in}{1.381784in}}{\pgfqpoint{2.627299in}{1.389598in}}%
\pgfpathcurveto{\pgfqpoint{2.635113in}{1.397411in}}{\pgfqpoint{2.639503in}{1.408010in}}{\pgfqpoint{2.639503in}{1.419060in}}%
\pgfpathcurveto{\pgfqpoint{2.639503in}{1.430111in}}{\pgfqpoint{2.635113in}{1.440710in}}{\pgfqpoint{2.627299in}{1.448523in}}%
\pgfpathcurveto{\pgfqpoint{2.619485in}{1.456337in}}{\pgfqpoint{2.608886in}{1.460727in}}{\pgfqpoint{2.597836in}{1.460727in}}%
\pgfpathcurveto{\pgfqpoint{2.586786in}{1.460727in}}{\pgfqpoint{2.576187in}{1.456337in}}{\pgfqpoint{2.568373in}{1.448523in}}%
\pgfpathcurveto{\pgfqpoint{2.560560in}{1.440710in}}{\pgfqpoint{2.556169in}{1.430111in}}{\pgfqpoint{2.556169in}{1.419060in}}%
\pgfpathcurveto{\pgfqpoint{2.556169in}{1.408010in}}{\pgfqpoint{2.560560in}{1.397411in}}{\pgfqpoint{2.568373in}{1.389598in}}%
\pgfpathcurveto{\pgfqpoint{2.576187in}{1.381784in}}{\pgfqpoint{2.586786in}{1.377394in}}{\pgfqpoint{2.597836in}{1.377394in}}%
\pgfpathclose%
\pgfusepath{stroke,fill}%
\end{pgfscope}%
\begin{pgfscope}%
\pgfpathrectangle{\pgfqpoint{0.481978in}{0.331635in}}{\pgfqpoint{4.960000in}{3.696000in}}%
\pgfusepath{clip}%
\pgfsetbuttcap%
\pgfsetroundjoin%
\definecolor{currentfill}{rgb}{1.000000,0.705882,0.509804}%
\pgfsetfillcolor{currentfill}%
\pgfsetlinewidth{0.481800pt}%
\definecolor{currentstroke}{rgb}{1.000000,1.000000,1.000000}%
\pgfsetstrokecolor{currentstroke}%
\pgfsetdash{}{0pt}%
\pgfpathmoveto{\pgfqpoint{2.673632in}{1.518656in}}%
\pgfpathcurveto{\pgfqpoint{2.684682in}{1.518656in}}{\pgfqpoint{2.695281in}{1.523047in}}{\pgfqpoint{2.703095in}{1.530860in}}%
\pgfpathcurveto{\pgfqpoint{2.710909in}{1.538674in}}{\pgfqpoint{2.715299in}{1.549273in}}{\pgfqpoint{2.715299in}{1.560323in}}%
\pgfpathcurveto{\pgfqpoint{2.715299in}{1.571373in}}{\pgfqpoint{2.710909in}{1.581972in}}{\pgfqpoint{2.703095in}{1.589786in}}%
\pgfpathcurveto{\pgfqpoint{2.695281in}{1.597599in}}{\pgfqpoint{2.684682in}{1.601990in}}{\pgfqpoint{2.673632in}{1.601990in}}%
\pgfpathcurveto{\pgfqpoint{2.662582in}{1.601990in}}{\pgfqpoint{2.651983in}{1.597599in}}{\pgfqpoint{2.644169in}{1.589786in}}%
\pgfpathcurveto{\pgfqpoint{2.636356in}{1.581972in}}{\pgfqpoint{2.631965in}{1.571373in}}{\pgfqpoint{2.631965in}{1.560323in}}%
\pgfpathcurveto{\pgfqpoint{2.631965in}{1.549273in}}{\pgfqpoint{2.636356in}{1.538674in}}{\pgfqpoint{2.644169in}{1.530860in}}%
\pgfpathcurveto{\pgfqpoint{2.651983in}{1.523047in}}{\pgfqpoint{2.662582in}{1.518656in}}{\pgfqpoint{2.673632in}{1.518656in}}%
\pgfpathclose%
\pgfusepath{stroke,fill}%
\end{pgfscope}%
\begin{pgfscope}%
\pgfpathrectangle{\pgfqpoint{0.481978in}{0.331635in}}{\pgfqpoint{4.960000in}{3.696000in}}%
\pgfusepath{clip}%
\pgfsetbuttcap%
\pgfsetroundjoin%
\definecolor{currentfill}{rgb}{1.000000,0.705882,0.509804}%
\pgfsetfillcolor{currentfill}%
\pgfsetlinewidth{0.481800pt}%
\definecolor{currentstroke}{rgb}{1.000000,1.000000,1.000000}%
\pgfsetstrokecolor{currentstroke}%
\pgfsetdash{}{0pt}%
\pgfpathmoveto{\pgfqpoint{3.059811in}{1.229853in}}%
\pgfpathcurveto{\pgfqpoint{3.070861in}{1.229853in}}{\pgfqpoint{3.081460in}{1.234244in}}{\pgfqpoint{3.089274in}{1.242057in}}%
\pgfpathcurveto{\pgfqpoint{3.097088in}{1.249871in}}{\pgfqpoint{3.101478in}{1.260470in}}{\pgfqpoint{3.101478in}{1.271520in}}%
\pgfpathcurveto{\pgfqpoint{3.101478in}{1.282570in}}{\pgfqpoint{3.097088in}{1.293169in}}{\pgfqpoint{3.089274in}{1.300983in}}%
\pgfpathcurveto{\pgfqpoint{3.081460in}{1.308797in}}{\pgfqpoint{3.070861in}{1.313187in}}{\pgfqpoint{3.059811in}{1.313187in}}%
\pgfpathcurveto{\pgfqpoint{3.048761in}{1.313187in}}{\pgfqpoint{3.038162in}{1.308797in}}{\pgfqpoint{3.030349in}{1.300983in}}%
\pgfpathcurveto{\pgfqpoint{3.022535in}{1.293169in}}{\pgfqpoint{3.018145in}{1.282570in}}{\pgfqpoint{3.018145in}{1.271520in}}%
\pgfpathcurveto{\pgfqpoint{3.018145in}{1.260470in}}{\pgfqpoint{3.022535in}{1.249871in}}{\pgfqpoint{3.030349in}{1.242057in}}%
\pgfpathcurveto{\pgfqpoint{3.038162in}{1.234244in}}{\pgfqpoint{3.048761in}{1.229853in}}{\pgfqpoint{3.059811in}{1.229853in}}%
\pgfpathclose%
\pgfusepath{stroke,fill}%
\end{pgfscope}%
\begin{pgfscope}%
\pgfpathrectangle{\pgfqpoint{0.481978in}{0.331635in}}{\pgfqpoint{4.960000in}{3.696000in}}%
\pgfusepath{clip}%
\pgfsetbuttcap%
\pgfsetroundjoin%
\definecolor{currentfill}{rgb}{1.000000,0.705882,0.509804}%
\pgfsetfillcolor{currentfill}%
\pgfsetlinewidth{0.481800pt}%
\definecolor{currentstroke}{rgb}{1.000000,1.000000,1.000000}%
\pgfsetstrokecolor{currentstroke}%
\pgfsetdash{}{0pt}%
\pgfpathmoveto{\pgfqpoint{2.580966in}{0.678102in}}%
\pgfpathcurveto{\pgfqpoint{2.592016in}{0.678102in}}{\pgfqpoint{2.602615in}{0.682493in}}{\pgfqpoint{2.610429in}{0.690306in}}%
\pgfpathcurveto{\pgfqpoint{2.618243in}{0.698120in}}{\pgfqpoint{2.622633in}{0.708719in}}{\pgfqpoint{2.622633in}{0.719769in}}%
\pgfpathcurveto{\pgfqpoint{2.622633in}{0.730819in}}{\pgfqpoint{2.618243in}{0.741418in}}{\pgfqpoint{2.610429in}{0.749232in}}%
\pgfpathcurveto{\pgfqpoint{2.602615in}{0.757045in}}{\pgfqpoint{2.592016in}{0.761436in}}{\pgfqpoint{2.580966in}{0.761436in}}%
\pgfpathcurveto{\pgfqpoint{2.569916in}{0.761436in}}{\pgfqpoint{2.559317in}{0.757045in}}{\pgfqpoint{2.551503in}{0.749232in}}%
\pgfpathcurveto{\pgfqpoint{2.543690in}{0.741418in}}{\pgfqpoint{2.539300in}{0.730819in}}{\pgfqpoint{2.539300in}{0.719769in}}%
\pgfpathcurveto{\pgfqpoint{2.539300in}{0.708719in}}{\pgfqpoint{2.543690in}{0.698120in}}{\pgfqpoint{2.551503in}{0.690306in}}%
\pgfpathcurveto{\pgfqpoint{2.559317in}{0.682493in}}{\pgfqpoint{2.569916in}{0.678102in}}{\pgfqpoint{2.580966in}{0.678102in}}%
\pgfpathclose%
\pgfusepath{stroke,fill}%
\end{pgfscope}%
\begin{pgfscope}%
\pgfpathrectangle{\pgfqpoint{0.481978in}{0.331635in}}{\pgfqpoint{4.960000in}{3.696000in}}%
\pgfusepath{clip}%
\pgfsetbuttcap%
\pgfsetroundjoin%
\definecolor{currentfill}{rgb}{1.000000,0.705882,0.509804}%
\pgfsetfillcolor{currentfill}%
\pgfsetlinewidth{0.481800pt}%
\definecolor{currentstroke}{rgb}{1.000000,1.000000,1.000000}%
\pgfsetstrokecolor{currentstroke}%
\pgfsetdash{}{0pt}%
\pgfpathmoveto{\pgfqpoint{3.696596in}{1.569069in}}%
\pgfpathcurveto{\pgfqpoint{3.707647in}{1.569069in}}{\pgfqpoint{3.718246in}{1.573460in}}{\pgfqpoint{3.726059in}{1.581273in}}%
\pgfpathcurveto{\pgfqpoint{3.733873in}{1.589087in}}{\pgfqpoint{3.738263in}{1.599686in}}{\pgfqpoint{3.738263in}{1.610736in}}%
\pgfpathcurveto{\pgfqpoint{3.738263in}{1.621786in}}{\pgfqpoint{3.733873in}{1.632385in}}{\pgfqpoint{3.726059in}{1.640199in}}%
\pgfpathcurveto{\pgfqpoint{3.718246in}{1.648013in}}{\pgfqpoint{3.707647in}{1.652403in}}{\pgfqpoint{3.696596in}{1.652403in}}%
\pgfpathcurveto{\pgfqpoint{3.685546in}{1.652403in}}{\pgfqpoint{3.674947in}{1.648013in}}{\pgfqpoint{3.667134in}{1.640199in}}%
\pgfpathcurveto{\pgfqpoint{3.659320in}{1.632385in}}{\pgfqpoint{3.654930in}{1.621786in}}{\pgfqpoint{3.654930in}{1.610736in}}%
\pgfpathcurveto{\pgfqpoint{3.654930in}{1.599686in}}{\pgfqpoint{3.659320in}{1.589087in}}{\pgfqpoint{3.667134in}{1.581273in}}%
\pgfpathcurveto{\pgfqpoint{3.674947in}{1.573460in}}{\pgfqpoint{3.685546in}{1.569069in}}{\pgfqpoint{3.696596in}{1.569069in}}%
\pgfpathclose%
\pgfusepath{stroke,fill}%
\end{pgfscope}%
\begin{pgfscope}%
\pgfpathrectangle{\pgfqpoint{0.481978in}{0.331635in}}{\pgfqpoint{4.960000in}{3.696000in}}%
\pgfusepath{clip}%
\pgfsetbuttcap%
\pgfsetroundjoin%
\definecolor{currentfill}{rgb}{1.000000,0.705882,0.509804}%
\pgfsetfillcolor{currentfill}%
\pgfsetlinewidth{0.481800pt}%
\definecolor{currentstroke}{rgb}{1.000000,1.000000,1.000000}%
\pgfsetstrokecolor{currentstroke}%
\pgfsetdash{}{0pt}%
\pgfpathmoveto{\pgfqpoint{1.955794in}{1.505175in}}%
\pgfpathcurveto{\pgfqpoint{1.966844in}{1.505175in}}{\pgfqpoint{1.977443in}{1.509565in}}{\pgfqpoint{1.985257in}{1.517379in}}%
\pgfpathcurveto{\pgfqpoint{1.993070in}{1.525193in}}{\pgfqpoint{1.997461in}{1.535792in}}{\pgfqpoint{1.997461in}{1.546842in}}%
\pgfpathcurveto{\pgfqpoint{1.997461in}{1.557892in}}{\pgfqpoint{1.993070in}{1.568491in}}{\pgfqpoint{1.985257in}{1.576304in}}%
\pgfpathcurveto{\pgfqpoint{1.977443in}{1.584118in}}{\pgfqpoint{1.966844in}{1.588508in}}{\pgfqpoint{1.955794in}{1.588508in}}%
\pgfpathcurveto{\pgfqpoint{1.944744in}{1.588508in}}{\pgfqpoint{1.934145in}{1.584118in}}{\pgfqpoint{1.926331in}{1.576304in}}%
\pgfpathcurveto{\pgfqpoint{1.918517in}{1.568491in}}{\pgfqpoint{1.914127in}{1.557892in}}{\pgfqpoint{1.914127in}{1.546842in}}%
\pgfpathcurveto{\pgfqpoint{1.914127in}{1.535792in}}{\pgfqpoint{1.918517in}{1.525193in}}{\pgfqpoint{1.926331in}{1.517379in}}%
\pgfpathcurveto{\pgfqpoint{1.934145in}{1.509565in}}{\pgfqpoint{1.944744in}{1.505175in}}{\pgfqpoint{1.955794in}{1.505175in}}%
\pgfpathclose%
\pgfusepath{stroke,fill}%
\end{pgfscope}%
\begin{pgfscope}%
\pgfpathrectangle{\pgfqpoint{0.481978in}{0.331635in}}{\pgfqpoint{4.960000in}{3.696000in}}%
\pgfusepath{clip}%
\pgfsetbuttcap%
\pgfsetroundjoin%
\definecolor{currentfill}{rgb}{1.000000,0.705882,0.509804}%
\pgfsetfillcolor{currentfill}%
\pgfsetlinewidth{0.481800pt}%
\definecolor{currentstroke}{rgb}{1.000000,1.000000,1.000000}%
\pgfsetstrokecolor{currentstroke}%
\pgfsetdash{}{0pt}%
\pgfpathmoveto{\pgfqpoint{3.741977in}{2.095884in}}%
\pgfpathcurveto{\pgfqpoint{3.753027in}{2.095884in}}{\pgfqpoint{3.763626in}{2.100274in}}{\pgfqpoint{3.771439in}{2.108088in}}%
\pgfpathcurveto{\pgfqpoint{3.779253in}{2.115901in}}{\pgfqpoint{3.783643in}{2.126500in}}{\pgfqpoint{3.783643in}{2.137550in}}%
\pgfpathcurveto{\pgfqpoint{3.783643in}{2.148601in}}{\pgfqpoint{3.779253in}{2.159200in}}{\pgfqpoint{3.771439in}{2.167013in}}%
\pgfpathcurveto{\pgfqpoint{3.763626in}{2.174827in}}{\pgfqpoint{3.753027in}{2.179217in}}{\pgfqpoint{3.741977in}{2.179217in}}%
\pgfpathcurveto{\pgfqpoint{3.730926in}{2.179217in}}{\pgfqpoint{3.720327in}{2.174827in}}{\pgfqpoint{3.712514in}{2.167013in}}%
\pgfpathcurveto{\pgfqpoint{3.704700in}{2.159200in}}{\pgfqpoint{3.700310in}{2.148601in}}{\pgfqpoint{3.700310in}{2.137550in}}%
\pgfpathcurveto{\pgfqpoint{3.700310in}{2.126500in}}{\pgfqpoint{3.704700in}{2.115901in}}{\pgfqpoint{3.712514in}{2.108088in}}%
\pgfpathcurveto{\pgfqpoint{3.720327in}{2.100274in}}{\pgfqpoint{3.730926in}{2.095884in}}{\pgfqpoint{3.741977in}{2.095884in}}%
\pgfpathclose%
\pgfusepath{stroke,fill}%
\end{pgfscope}%
\begin{pgfscope}%
\pgfpathrectangle{\pgfqpoint{0.481978in}{0.331635in}}{\pgfqpoint{4.960000in}{3.696000in}}%
\pgfusepath{clip}%
\pgfsetbuttcap%
\pgfsetroundjoin%
\definecolor{currentfill}{rgb}{1.000000,0.705882,0.509804}%
\pgfsetfillcolor{currentfill}%
\pgfsetlinewidth{0.481800pt}%
\definecolor{currentstroke}{rgb}{1.000000,1.000000,1.000000}%
\pgfsetstrokecolor{currentstroke}%
\pgfsetdash{}{0pt}%
\pgfpathmoveto{\pgfqpoint{1.231668in}{1.015867in}}%
\pgfpathcurveto{\pgfqpoint{1.242718in}{1.015867in}}{\pgfqpoint{1.253317in}{1.020258in}}{\pgfqpoint{1.261131in}{1.028071in}}%
\pgfpathcurveto{\pgfqpoint{1.268945in}{1.035885in}}{\pgfqpoint{1.273335in}{1.046484in}}{\pgfqpoint{1.273335in}{1.057534in}}%
\pgfpathcurveto{\pgfqpoint{1.273335in}{1.068584in}}{\pgfqpoint{1.268945in}{1.079183in}}{\pgfqpoint{1.261131in}{1.086997in}}%
\pgfpathcurveto{\pgfqpoint{1.253317in}{1.094810in}}{\pgfqpoint{1.242718in}{1.099201in}}{\pgfqpoint{1.231668in}{1.099201in}}%
\pgfpathcurveto{\pgfqpoint{1.220618in}{1.099201in}}{\pgfqpoint{1.210019in}{1.094810in}}{\pgfqpoint{1.202205in}{1.086997in}}%
\pgfpathcurveto{\pgfqpoint{1.194392in}{1.079183in}}{\pgfqpoint{1.190001in}{1.068584in}}{\pgfqpoint{1.190001in}{1.057534in}}%
\pgfpathcurveto{\pgfqpoint{1.190001in}{1.046484in}}{\pgfqpoint{1.194392in}{1.035885in}}{\pgfqpoint{1.202205in}{1.028071in}}%
\pgfpathcurveto{\pgfqpoint{1.210019in}{1.020258in}}{\pgfqpoint{1.220618in}{1.015867in}}{\pgfqpoint{1.231668in}{1.015867in}}%
\pgfpathclose%
\pgfusepath{stroke,fill}%
\end{pgfscope}%
\begin{pgfscope}%
\pgfpathrectangle{\pgfqpoint{0.481978in}{0.331635in}}{\pgfqpoint{4.960000in}{3.696000in}}%
\pgfusepath{clip}%
\pgfsetbuttcap%
\pgfsetroundjoin%
\definecolor{currentfill}{rgb}{1.000000,0.705882,0.509804}%
\pgfsetfillcolor{currentfill}%
\pgfsetlinewidth{0.481800pt}%
\definecolor{currentstroke}{rgb}{1.000000,1.000000,1.000000}%
\pgfsetstrokecolor{currentstroke}%
\pgfsetdash{}{0pt}%
\pgfpathmoveto{\pgfqpoint{1.514045in}{1.425244in}}%
\pgfpathcurveto{\pgfqpoint{1.525095in}{1.425244in}}{\pgfqpoint{1.535694in}{1.429634in}}{\pgfqpoint{1.543508in}{1.437448in}}%
\pgfpathcurveto{\pgfqpoint{1.551321in}{1.445261in}}{\pgfqpoint{1.555712in}{1.455860in}}{\pgfqpoint{1.555712in}{1.466910in}}%
\pgfpathcurveto{\pgfqpoint{1.555712in}{1.477961in}}{\pgfqpoint{1.551321in}{1.488560in}}{\pgfqpoint{1.543508in}{1.496373in}}%
\pgfpathcurveto{\pgfqpoint{1.535694in}{1.504187in}}{\pgfqpoint{1.525095in}{1.508577in}}{\pgfqpoint{1.514045in}{1.508577in}}%
\pgfpathcurveto{\pgfqpoint{1.502995in}{1.508577in}}{\pgfqpoint{1.492396in}{1.504187in}}{\pgfqpoint{1.484582in}{1.496373in}}%
\pgfpathcurveto{\pgfqpoint{1.476769in}{1.488560in}}{\pgfqpoint{1.472378in}{1.477961in}}{\pgfqpoint{1.472378in}{1.466910in}}%
\pgfpathcurveto{\pgfqpoint{1.472378in}{1.455860in}}{\pgfqpoint{1.476769in}{1.445261in}}{\pgfqpoint{1.484582in}{1.437448in}}%
\pgfpathcurveto{\pgfqpoint{1.492396in}{1.429634in}}{\pgfqpoint{1.502995in}{1.425244in}}{\pgfqpoint{1.514045in}{1.425244in}}%
\pgfpathclose%
\pgfusepath{stroke,fill}%
\end{pgfscope}%
\begin{pgfscope}%
\pgfpathrectangle{\pgfqpoint{0.481978in}{0.331635in}}{\pgfqpoint{4.960000in}{3.696000in}}%
\pgfusepath{clip}%
\pgfsetbuttcap%
\pgfsetroundjoin%
\definecolor{currentfill}{rgb}{1.000000,0.705882,0.509804}%
\pgfsetfillcolor{currentfill}%
\pgfsetlinewidth{0.481800pt}%
\definecolor{currentstroke}{rgb}{1.000000,1.000000,1.000000}%
\pgfsetstrokecolor{currentstroke}%
\pgfsetdash{}{0pt}%
\pgfpathmoveto{\pgfqpoint{2.097374in}{2.081029in}}%
\pgfpathcurveto{\pgfqpoint{2.108424in}{2.081029in}}{\pgfqpoint{2.119023in}{2.085419in}}{\pgfqpoint{2.126837in}{2.093232in}}%
\pgfpathcurveto{\pgfqpoint{2.134650in}{2.101046in}}{\pgfqpoint{2.139041in}{2.111645in}}{\pgfqpoint{2.139041in}{2.122695in}}%
\pgfpathcurveto{\pgfqpoint{2.139041in}{2.133745in}}{\pgfqpoint{2.134650in}{2.144344in}}{\pgfqpoint{2.126837in}{2.152158in}}%
\pgfpathcurveto{\pgfqpoint{2.119023in}{2.159972in}}{\pgfqpoint{2.108424in}{2.164362in}}{\pgfqpoint{2.097374in}{2.164362in}}%
\pgfpathcurveto{\pgfqpoint{2.086324in}{2.164362in}}{\pgfqpoint{2.075725in}{2.159972in}}{\pgfqpoint{2.067911in}{2.152158in}}%
\pgfpathcurveto{\pgfqpoint{2.060098in}{2.144344in}}{\pgfqpoint{2.055707in}{2.133745in}}{\pgfqpoint{2.055707in}{2.122695in}}%
\pgfpathcurveto{\pgfqpoint{2.055707in}{2.111645in}}{\pgfqpoint{2.060098in}{2.101046in}}{\pgfqpoint{2.067911in}{2.093232in}}%
\pgfpathcurveto{\pgfqpoint{2.075725in}{2.085419in}}{\pgfqpoint{2.086324in}{2.081029in}}{\pgfqpoint{2.097374in}{2.081029in}}%
\pgfpathclose%
\pgfusepath{stroke,fill}%
\end{pgfscope}%
\begin{pgfscope}%
\pgfpathrectangle{\pgfqpoint{0.481978in}{0.331635in}}{\pgfqpoint{4.960000in}{3.696000in}}%
\pgfusepath{clip}%
\pgfsetbuttcap%
\pgfsetroundjoin%
\definecolor{currentfill}{rgb}{1.000000,0.705882,0.509804}%
\pgfsetfillcolor{currentfill}%
\pgfsetlinewidth{0.481800pt}%
\definecolor{currentstroke}{rgb}{1.000000,1.000000,1.000000}%
\pgfsetstrokecolor{currentstroke}%
\pgfsetdash{}{0pt}%
\pgfpathmoveto{\pgfqpoint{3.862616in}{1.171086in}}%
\pgfpathcurveto{\pgfqpoint{3.873666in}{1.171086in}}{\pgfqpoint{3.884265in}{1.175477in}}{\pgfqpoint{3.892079in}{1.183290in}}%
\pgfpathcurveto{\pgfqpoint{3.899892in}{1.191104in}}{\pgfqpoint{3.904282in}{1.201703in}}{\pgfqpoint{3.904282in}{1.212753in}}%
\pgfpathcurveto{\pgfqpoint{3.904282in}{1.223803in}}{\pgfqpoint{3.899892in}{1.234402in}}{\pgfqpoint{3.892079in}{1.242216in}}%
\pgfpathcurveto{\pgfqpoint{3.884265in}{1.250029in}}{\pgfqpoint{3.873666in}{1.254420in}}{\pgfqpoint{3.862616in}{1.254420in}}%
\pgfpathcurveto{\pgfqpoint{3.851566in}{1.254420in}}{\pgfqpoint{3.840967in}{1.250029in}}{\pgfqpoint{3.833153in}{1.242216in}}%
\pgfpathcurveto{\pgfqpoint{3.825339in}{1.234402in}}{\pgfqpoint{3.820949in}{1.223803in}}{\pgfqpoint{3.820949in}{1.212753in}}%
\pgfpathcurveto{\pgfqpoint{3.820949in}{1.201703in}}{\pgfqpoint{3.825339in}{1.191104in}}{\pgfqpoint{3.833153in}{1.183290in}}%
\pgfpathcurveto{\pgfqpoint{3.840967in}{1.175477in}}{\pgfqpoint{3.851566in}{1.171086in}}{\pgfqpoint{3.862616in}{1.171086in}}%
\pgfpathclose%
\pgfusepath{stroke,fill}%
\end{pgfscope}%
\begin{pgfscope}%
\pgfpathrectangle{\pgfqpoint{0.481978in}{0.331635in}}{\pgfqpoint{4.960000in}{3.696000in}}%
\pgfusepath{clip}%
\pgfsetbuttcap%
\pgfsetroundjoin%
\definecolor{currentfill}{rgb}{1.000000,0.705882,0.509804}%
\pgfsetfillcolor{currentfill}%
\pgfsetlinewidth{0.481800pt}%
\definecolor{currentstroke}{rgb}{1.000000,1.000000,1.000000}%
\pgfsetstrokecolor{currentstroke}%
\pgfsetdash{}{0pt}%
\pgfpathmoveto{\pgfqpoint{2.141974in}{1.759922in}}%
\pgfpathcurveto{\pgfqpoint{2.153024in}{1.759922in}}{\pgfqpoint{2.163623in}{1.764312in}}{\pgfqpoint{2.171437in}{1.772126in}}%
\pgfpathcurveto{\pgfqpoint{2.179251in}{1.779940in}}{\pgfqpoint{2.183641in}{1.790539in}}{\pgfqpoint{2.183641in}{1.801589in}}%
\pgfpathcurveto{\pgfqpoint{2.183641in}{1.812639in}}{\pgfqpoint{2.179251in}{1.823238in}}{\pgfqpoint{2.171437in}{1.831052in}}%
\pgfpathcurveto{\pgfqpoint{2.163623in}{1.838865in}}{\pgfqpoint{2.153024in}{1.843256in}}{\pgfqpoint{2.141974in}{1.843256in}}%
\pgfpathcurveto{\pgfqpoint{2.130924in}{1.843256in}}{\pgfqpoint{2.120325in}{1.838865in}}{\pgfqpoint{2.112511in}{1.831052in}}%
\pgfpathcurveto{\pgfqpoint{2.104698in}{1.823238in}}{\pgfqpoint{2.100308in}{1.812639in}}{\pgfqpoint{2.100308in}{1.801589in}}%
\pgfpathcurveto{\pgfqpoint{2.100308in}{1.790539in}}{\pgfqpoint{2.104698in}{1.779940in}}{\pgfqpoint{2.112511in}{1.772126in}}%
\pgfpathcurveto{\pgfqpoint{2.120325in}{1.764312in}}{\pgfqpoint{2.130924in}{1.759922in}}{\pgfqpoint{2.141974in}{1.759922in}}%
\pgfpathclose%
\pgfusepath{stroke,fill}%
\end{pgfscope}%
\begin{pgfscope}%
\pgfpathrectangle{\pgfqpoint{0.481978in}{0.331635in}}{\pgfqpoint{4.960000in}{3.696000in}}%
\pgfusepath{clip}%
\pgfsetbuttcap%
\pgfsetroundjoin%
\definecolor{currentfill}{rgb}{1.000000,0.705882,0.509804}%
\pgfsetfillcolor{currentfill}%
\pgfsetlinewidth{0.481800pt}%
\definecolor{currentstroke}{rgb}{1.000000,1.000000,1.000000}%
\pgfsetstrokecolor{currentstroke}%
\pgfsetdash{}{0pt}%
\pgfpathmoveto{\pgfqpoint{2.743580in}{1.005505in}}%
\pgfpathcurveto{\pgfqpoint{2.754630in}{1.005505in}}{\pgfqpoint{2.765229in}{1.009896in}}{\pgfqpoint{2.773043in}{1.017709in}}%
\pgfpathcurveto{\pgfqpoint{2.780857in}{1.025523in}}{\pgfqpoint{2.785247in}{1.036122in}}{\pgfqpoint{2.785247in}{1.047172in}}%
\pgfpathcurveto{\pgfqpoint{2.785247in}{1.058222in}}{\pgfqpoint{2.780857in}{1.068821in}}{\pgfqpoint{2.773043in}{1.076635in}}%
\pgfpathcurveto{\pgfqpoint{2.765229in}{1.084448in}}{\pgfqpoint{2.754630in}{1.088839in}}{\pgfqpoint{2.743580in}{1.088839in}}%
\pgfpathcurveto{\pgfqpoint{2.732530in}{1.088839in}}{\pgfqpoint{2.721931in}{1.084448in}}{\pgfqpoint{2.714117in}{1.076635in}}%
\pgfpathcurveto{\pgfqpoint{2.706304in}{1.068821in}}{\pgfqpoint{2.701913in}{1.058222in}}{\pgfqpoint{2.701913in}{1.047172in}}%
\pgfpathcurveto{\pgfqpoint{2.701913in}{1.036122in}}{\pgfqpoint{2.706304in}{1.025523in}}{\pgfqpoint{2.714117in}{1.017709in}}%
\pgfpathcurveto{\pgfqpoint{2.721931in}{1.009896in}}{\pgfqpoint{2.732530in}{1.005505in}}{\pgfqpoint{2.743580in}{1.005505in}}%
\pgfpathclose%
\pgfusepath{stroke,fill}%
\end{pgfscope}%
\begin{pgfscope}%
\pgfpathrectangle{\pgfqpoint{0.481978in}{0.331635in}}{\pgfqpoint{4.960000in}{3.696000in}}%
\pgfusepath{clip}%
\pgfsetbuttcap%
\pgfsetroundjoin%
\definecolor{currentfill}{rgb}{1.000000,0.705882,0.509804}%
\pgfsetfillcolor{currentfill}%
\pgfsetlinewidth{0.481800pt}%
\definecolor{currentstroke}{rgb}{1.000000,1.000000,1.000000}%
\pgfsetstrokecolor{currentstroke}%
\pgfsetdash{}{0pt}%
\pgfpathmoveto{\pgfqpoint{0.945106in}{1.384528in}}%
\pgfpathcurveto{\pgfqpoint{0.956156in}{1.384528in}}{\pgfqpoint{0.966755in}{1.388919in}}{\pgfqpoint{0.974569in}{1.396732in}}%
\pgfpathcurveto{\pgfqpoint{0.982383in}{1.404546in}}{\pgfqpoint{0.986773in}{1.415145in}}{\pgfqpoint{0.986773in}{1.426195in}}%
\pgfpathcurveto{\pgfqpoint{0.986773in}{1.437245in}}{\pgfqpoint{0.982383in}{1.447844in}}{\pgfqpoint{0.974569in}{1.455658in}}%
\pgfpathcurveto{\pgfqpoint{0.966755in}{1.463471in}}{\pgfqpoint{0.956156in}{1.467862in}}{\pgfqpoint{0.945106in}{1.467862in}}%
\pgfpathcurveto{\pgfqpoint{0.934056in}{1.467862in}}{\pgfqpoint{0.923457in}{1.463471in}}{\pgfqpoint{0.915643in}{1.455658in}}%
\pgfpathcurveto{\pgfqpoint{0.907830in}{1.447844in}}{\pgfqpoint{0.903439in}{1.437245in}}{\pgfqpoint{0.903439in}{1.426195in}}%
\pgfpathcurveto{\pgfqpoint{0.903439in}{1.415145in}}{\pgfqpoint{0.907830in}{1.404546in}}{\pgfqpoint{0.915643in}{1.396732in}}%
\pgfpathcurveto{\pgfqpoint{0.923457in}{1.388919in}}{\pgfqpoint{0.934056in}{1.384528in}}{\pgfqpoint{0.945106in}{1.384528in}}%
\pgfpathclose%
\pgfusepath{stroke,fill}%
\end{pgfscope}%
\begin{pgfscope}%
\pgfpathrectangle{\pgfqpoint{0.481978in}{0.331635in}}{\pgfqpoint{4.960000in}{3.696000in}}%
\pgfusepath{clip}%
\pgfsetbuttcap%
\pgfsetroundjoin%
\definecolor{currentfill}{rgb}{1.000000,0.705882,0.509804}%
\pgfsetfillcolor{currentfill}%
\pgfsetlinewidth{0.481800pt}%
\definecolor{currentstroke}{rgb}{1.000000,1.000000,1.000000}%
\pgfsetstrokecolor{currentstroke}%
\pgfsetdash{}{0pt}%
\pgfpathmoveto{\pgfqpoint{2.371014in}{1.284221in}}%
\pgfpathcurveto{\pgfqpoint{2.382064in}{1.284221in}}{\pgfqpoint{2.392663in}{1.288611in}}{\pgfqpoint{2.400477in}{1.296425in}}%
\pgfpathcurveto{\pgfqpoint{2.408291in}{1.304239in}}{\pgfqpoint{2.412681in}{1.314838in}}{\pgfqpoint{2.412681in}{1.325888in}}%
\pgfpathcurveto{\pgfqpoint{2.412681in}{1.336938in}}{\pgfqpoint{2.408291in}{1.347537in}}{\pgfqpoint{2.400477in}{1.355351in}}%
\pgfpathcurveto{\pgfqpoint{2.392663in}{1.363164in}}{\pgfqpoint{2.382064in}{1.367554in}}{\pgfqpoint{2.371014in}{1.367554in}}%
\pgfpathcurveto{\pgfqpoint{2.359964in}{1.367554in}}{\pgfqpoint{2.349365in}{1.363164in}}{\pgfqpoint{2.341551in}{1.355351in}}%
\pgfpathcurveto{\pgfqpoint{2.333738in}{1.347537in}}{\pgfqpoint{2.329347in}{1.336938in}}{\pgfqpoint{2.329347in}{1.325888in}}%
\pgfpathcurveto{\pgfqpoint{2.329347in}{1.314838in}}{\pgfqpoint{2.333738in}{1.304239in}}{\pgfqpoint{2.341551in}{1.296425in}}%
\pgfpathcurveto{\pgfqpoint{2.349365in}{1.288611in}}{\pgfqpoint{2.359964in}{1.284221in}}{\pgfqpoint{2.371014in}{1.284221in}}%
\pgfpathclose%
\pgfusepath{stroke,fill}%
\end{pgfscope}%
\begin{pgfscope}%
\pgfpathrectangle{\pgfqpoint{0.481978in}{0.331635in}}{\pgfqpoint{4.960000in}{3.696000in}}%
\pgfusepath{clip}%
\pgfsetbuttcap%
\pgfsetroundjoin%
\definecolor{currentfill}{rgb}{1.000000,0.705882,0.509804}%
\pgfsetfillcolor{currentfill}%
\pgfsetlinewidth{0.481800pt}%
\definecolor{currentstroke}{rgb}{1.000000,1.000000,1.000000}%
\pgfsetstrokecolor{currentstroke}%
\pgfsetdash{}{0pt}%
\pgfpathmoveto{\pgfqpoint{4.359096in}{1.272999in}}%
\pgfpathcurveto{\pgfqpoint{4.370146in}{1.272999in}}{\pgfqpoint{4.380745in}{1.277389in}}{\pgfqpoint{4.388559in}{1.285203in}}%
\pgfpathcurveto{\pgfqpoint{4.396372in}{1.293016in}}{\pgfqpoint{4.400763in}{1.303615in}}{\pgfqpoint{4.400763in}{1.314665in}}%
\pgfpathcurveto{\pgfqpoint{4.400763in}{1.325715in}}{\pgfqpoint{4.396372in}{1.336315in}}{\pgfqpoint{4.388559in}{1.344128in}}%
\pgfpathcurveto{\pgfqpoint{4.380745in}{1.351942in}}{\pgfqpoint{4.370146in}{1.356332in}}{\pgfqpoint{4.359096in}{1.356332in}}%
\pgfpathcurveto{\pgfqpoint{4.348046in}{1.356332in}}{\pgfqpoint{4.337447in}{1.351942in}}{\pgfqpoint{4.329633in}{1.344128in}}%
\pgfpathcurveto{\pgfqpoint{4.321819in}{1.336315in}}{\pgfqpoint{4.317429in}{1.325715in}}{\pgfqpoint{4.317429in}{1.314665in}}%
\pgfpathcurveto{\pgfqpoint{4.317429in}{1.303615in}}{\pgfqpoint{4.321819in}{1.293016in}}{\pgfqpoint{4.329633in}{1.285203in}}%
\pgfpathcurveto{\pgfqpoint{4.337447in}{1.277389in}}{\pgfqpoint{4.348046in}{1.272999in}}{\pgfqpoint{4.359096in}{1.272999in}}%
\pgfpathclose%
\pgfusepath{stroke,fill}%
\end{pgfscope}%
\begin{pgfscope}%
\pgfpathrectangle{\pgfqpoint{0.481978in}{0.331635in}}{\pgfqpoint{4.960000in}{3.696000in}}%
\pgfusepath{clip}%
\pgfsetbuttcap%
\pgfsetroundjoin%
\definecolor{currentfill}{rgb}{1.000000,0.705882,0.509804}%
\pgfsetfillcolor{currentfill}%
\pgfsetlinewidth{0.481800pt}%
\definecolor{currentstroke}{rgb}{1.000000,1.000000,1.000000}%
\pgfsetstrokecolor{currentstroke}%
\pgfsetdash{}{0pt}%
\pgfpathmoveto{\pgfqpoint{2.282850in}{0.571139in}}%
\pgfpathcurveto{\pgfqpoint{2.293900in}{0.571139in}}{\pgfqpoint{2.304499in}{0.575529in}}{\pgfqpoint{2.312313in}{0.583343in}}%
\pgfpathcurveto{\pgfqpoint{2.320127in}{0.591156in}}{\pgfqpoint{2.324517in}{0.601755in}}{\pgfqpoint{2.324517in}{0.612805in}}%
\pgfpathcurveto{\pgfqpoint{2.324517in}{0.623856in}}{\pgfqpoint{2.320127in}{0.634455in}}{\pgfqpoint{2.312313in}{0.642268in}}%
\pgfpathcurveto{\pgfqpoint{2.304499in}{0.650082in}}{\pgfqpoint{2.293900in}{0.654472in}}{\pgfqpoint{2.282850in}{0.654472in}}%
\pgfpathcurveto{\pgfqpoint{2.271800in}{0.654472in}}{\pgfqpoint{2.261201in}{0.650082in}}{\pgfqpoint{2.253387in}{0.642268in}}%
\pgfpathcurveto{\pgfqpoint{2.245574in}{0.634455in}}{\pgfqpoint{2.241183in}{0.623856in}}{\pgfqpoint{2.241183in}{0.612805in}}%
\pgfpathcurveto{\pgfqpoint{2.241183in}{0.601755in}}{\pgfqpoint{2.245574in}{0.591156in}}{\pgfqpoint{2.253387in}{0.583343in}}%
\pgfpathcurveto{\pgfqpoint{2.261201in}{0.575529in}}{\pgfqpoint{2.271800in}{0.571139in}}{\pgfqpoint{2.282850in}{0.571139in}}%
\pgfpathclose%
\pgfusepath{stroke,fill}%
\end{pgfscope}%
\begin{pgfscope}%
\pgfpathrectangle{\pgfqpoint{0.481978in}{0.331635in}}{\pgfqpoint{4.960000in}{3.696000in}}%
\pgfusepath{clip}%
\pgfsetbuttcap%
\pgfsetroundjoin%
\definecolor{currentfill}{rgb}{1.000000,0.705882,0.509804}%
\pgfsetfillcolor{currentfill}%
\pgfsetlinewidth{0.481800pt}%
\definecolor{currentstroke}{rgb}{1.000000,1.000000,1.000000}%
\pgfsetstrokecolor{currentstroke}%
\pgfsetdash{}{0pt}%
\pgfpathmoveto{\pgfqpoint{2.633929in}{1.424731in}}%
\pgfpathcurveto{\pgfqpoint{2.644979in}{1.424731in}}{\pgfqpoint{2.655578in}{1.429121in}}{\pgfqpoint{2.663392in}{1.436935in}}%
\pgfpathcurveto{\pgfqpoint{2.671205in}{1.444748in}}{\pgfqpoint{2.675596in}{1.455347in}}{\pgfqpoint{2.675596in}{1.466397in}}%
\pgfpathcurveto{\pgfqpoint{2.675596in}{1.477448in}}{\pgfqpoint{2.671205in}{1.488047in}}{\pgfqpoint{2.663392in}{1.495860in}}%
\pgfpathcurveto{\pgfqpoint{2.655578in}{1.503674in}}{\pgfqpoint{2.644979in}{1.508064in}}{\pgfqpoint{2.633929in}{1.508064in}}%
\pgfpathcurveto{\pgfqpoint{2.622879in}{1.508064in}}{\pgfqpoint{2.612280in}{1.503674in}}{\pgfqpoint{2.604466in}{1.495860in}}%
\pgfpathcurveto{\pgfqpoint{2.596653in}{1.488047in}}{\pgfqpoint{2.592262in}{1.477448in}}{\pgfqpoint{2.592262in}{1.466397in}}%
\pgfpathcurveto{\pgfqpoint{2.592262in}{1.455347in}}{\pgfqpoint{2.596653in}{1.444748in}}{\pgfqpoint{2.604466in}{1.436935in}}%
\pgfpathcurveto{\pgfqpoint{2.612280in}{1.429121in}}{\pgfqpoint{2.622879in}{1.424731in}}{\pgfqpoint{2.633929in}{1.424731in}}%
\pgfpathclose%
\pgfusepath{stroke,fill}%
\end{pgfscope}%
\begin{pgfscope}%
\pgfpathrectangle{\pgfqpoint{0.481978in}{0.331635in}}{\pgfqpoint{4.960000in}{3.696000in}}%
\pgfusepath{clip}%
\pgfsetbuttcap%
\pgfsetroundjoin%
\definecolor{currentfill}{rgb}{1.000000,0.705882,0.509804}%
\pgfsetfillcolor{currentfill}%
\pgfsetlinewidth{0.481800pt}%
\definecolor{currentstroke}{rgb}{1.000000,1.000000,1.000000}%
\pgfsetstrokecolor{currentstroke}%
\pgfsetdash{}{0pt}%
\pgfpathmoveto{\pgfqpoint{2.154059in}{1.110120in}}%
\pgfpathcurveto{\pgfqpoint{2.165109in}{1.110120in}}{\pgfqpoint{2.175708in}{1.114510in}}{\pgfqpoint{2.183522in}{1.122324in}}%
\pgfpathcurveto{\pgfqpoint{2.191335in}{1.130137in}}{\pgfqpoint{2.195726in}{1.140736in}}{\pgfqpoint{2.195726in}{1.151787in}}%
\pgfpathcurveto{\pgfqpoint{2.195726in}{1.162837in}}{\pgfqpoint{2.191335in}{1.173436in}}{\pgfqpoint{2.183522in}{1.181249in}}%
\pgfpathcurveto{\pgfqpoint{2.175708in}{1.189063in}}{\pgfqpoint{2.165109in}{1.193453in}}{\pgfqpoint{2.154059in}{1.193453in}}%
\pgfpathcurveto{\pgfqpoint{2.143009in}{1.193453in}}{\pgfqpoint{2.132410in}{1.189063in}}{\pgfqpoint{2.124596in}{1.181249in}}%
\pgfpathcurveto{\pgfqpoint{2.116783in}{1.173436in}}{\pgfqpoint{2.112392in}{1.162837in}}{\pgfqpoint{2.112392in}{1.151787in}}%
\pgfpathcurveto{\pgfqpoint{2.112392in}{1.140736in}}{\pgfqpoint{2.116783in}{1.130137in}}{\pgfqpoint{2.124596in}{1.122324in}}%
\pgfpathcurveto{\pgfqpoint{2.132410in}{1.114510in}}{\pgfqpoint{2.143009in}{1.110120in}}{\pgfqpoint{2.154059in}{1.110120in}}%
\pgfpathclose%
\pgfusepath{stroke,fill}%
\end{pgfscope}%
\begin{pgfscope}%
\pgfpathrectangle{\pgfqpoint{0.481978in}{0.331635in}}{\pgfqpoint{4.960000in}{3.696000in}}%
\pgfusepath{clip}%
\pgfsetbuttcap%
\pgfsetroundjoin%
\definecolor{currentfill}{rgb}{1.000000,0.705882,0.509804}%
\pgfsetfillcolor{currentfill}%
\pgfsetlinewidth{0.481800pt}%
\definecolor{currentstroke}{rgb}{1.000000,1.000000,1.000000}%
\pgfsetstrokecolor{currentstroke}%
\pgfsetdash{}{0pt}%
\pgfpathmoveto{\pgfqpoint{2.225441in}{1.763870in}}%
\pgfpathcurveto{\pgfqpoint{2.236492in}{1.763870in}}{\pgfqpoint{2.247091in}{1.768260in}}{\pgfqpoint{2.254904in}{1.776074in}}%
\pgfpathcurveto{\pgfqpoint{2.262718in}{1.783887in}}{\pgfqpoint{2.267108in}{1.794486in}}{\pgfqpoint{2.267108in}{1.805536in}}%
\pgfpathcurveto{\pgfqpoint{2.267108in}{1.816587in}}{\pgfqpoint{2.262718in}{1.827186in}}{\pgfqpoint{2.254904in}{1.834999in}}%
\pgfpathcurveto{\pgfqpoint{2.247091in}{1.842813in}}{\pgfqpoint{2.236492in}{1.847203in}}{\pgfqpoint{2.225441in}{1.847203in}}%
\pgfpathcurveto{\pgfqpoint{2.214391in}{1.847203in}}{\pgfqpoint{2.203792in}{1.842813in}}{\pgfqpoint{2.195979in}{1.834999in}}%
\pgfpathcurveto{\pgfqpoint{2.188165in}{1.827186in}}{\pgfqpoint{2.183775in}{1.816587in}}{\pgfqpoint{2.183775in}{1.805536in}}%
\pgfpathcurveto{\pgfqpoint{2.183775in}{1.794486in}}{\pgfqpoint{2.188165in}{1.783887in}}{\pgfqpoint{2.195979in}{1.776074in}}%
\pgfpathcurveto{\pgfqpoint{2.203792in}{1.768260in}}{\pgfqpoint{2.214391in}{1.763870in}}{\pgfqpoint{2.225441in}{1.763870in}}%
\pgfpathclose%
\pgfusepath{stroke,fill}%
\end{pgfscope}%
\begin{pgfscope}%
\pgfpathrectangle{\pgfqpoint{0.481978in}{0.331635in}}{\pgfqpoint{4.960000in}{3.696000in}}%
\pgfusepath{clip}%
\pgfsetbuttcap%
\pgfsetroundjoin%
\definecolor{currentfill}{rgb}{1.000000,0.705882,0.509804}%
\pgfsetfillcolor{currentfill}%
\pgfsetlinewidth{0.481800pt}%
\definecolor{currentstroke}{rgb}{1.000000,1.000000,1.000000}%
\pgfsetstrokecolor{currentstroke}%
\pgfsetdash{}{0pt}%
\pgfpathmoveto{\pgfqpoint{1.360201in}{2.145745in}}%
\pgfpathcurveto{\pgfqpoint{1.371251in}{2.145745in}}{\pgfqpoint{1.381850in}{2.150136in}}{\pgfqpoint{1.389664in}{2.157949in}}%
\pgfpathcurveto{\pgfqpoint{1.397477in}{2.165763in}}{\pgfqpoint{1.401868in}{2.176362in}}{\pgfqpoint{1.401868in}{2.187412in}}%
\pgfpathcurveto{\pgfqpoint{1.401868in}{2.198462in}}{\pgfqpoint{1.397477in}{2.209061in}}{\pgfqpoint{1.389664in}{2.216875in}}%
\pgfpathcurveto{\pgfqpoint{1.381850in}{2.224688in}}{\pgfqpoint{1.371251in}{2.229079in}}{\pgfqpoint{1.360201in}{2.229079in}}%
\pgfpathcurveto{\pgfqpoint{1.349151in}{2.229079in}}{\pgfqpoint{1.338552in}{2.224688in}}{\pgfqpoint{1.330738in}{2.216875in}}%
\pgfpathcurveto{\pgfqpoint{1.322924in}{2.209061in}}{\pgfqpoint{1.318534in}{2.198462in}}{\pgfqpoint{1.318534in}{2.187412in}}%
\pgfpathcurveto{\pgfqpoint{1.318534in}{2.176362in}}{\pgfqpoint{1.322924in}{2.165763in}}{\pgfqpoint{1.330738in}{2.157949in}}%
\pgfpathcurveto{\pgfqpoint{1.338552in}{2.150136in}}{\pgfqpoint{1.349151in}{2.145745in}}{\pgfqpoint{1.360201in}{2.145745in}}%
\pgfpathclose%
\pgfusepath{stroke,fill}%
\end{pgfscope}%
\begin{pgfscope}%
\pgfpathrectangle{\pgfqpoint{0.481978in}{0.331635in}}{\pgfqpoint{4.960000in}{3.696000in}}%
\pgfusepath{clip}%
\pgfsetbuttcap%
\pgfsetroundjoin%
\definecolor{currentfill}{rgb}{1.000000,0.705882,0.509804}%
\pgfsetfillcolor{currentfill}%
\pgfsetlinewidth{0.481800pt}%
\definecolor{currentstroke}{rgb}{1.000000,1.000000,1.000000}%
\pgfsetstrokecolor{currentstroke}%
\pgfsetdash{}{0pt}%
\pgfpathmoveto{\pgfqpoint{2.128386in}{0.902866in}}%
\pgfpathcurveto{\pgfqpoint{2.139436in}{0.902866in}}{\pgfqpoint{2.150035in}{0.907256in}}{\pgfqpoint{2.157848in}{0.915070in}}%
\pgfpathcurveto{\pgfqpoint{2.165662in}{0.922884in}}{\pgfqpoint{2.170052in}{0.933483in}}{\pgfqpoint{2.170052in}{0.944533in}}%
\pgfpathcurveto{\pgfqpoint{2.170052in}{0.955583in}}{\pgfqpoint{2.165662in}{0.966182in}}{\pgfqpoint{2.157848in}{0.973996in}}%
\pgfpathcurveto{\pgfqpoint{2.150035in}{0.981809in}}{\pgfqpoint{2.139436in}{0.986199in}}{\pgfqpoint{2.128386in}{0.986199in}}%
\pgfpathcurveto{\pgfqpoint{2.117336in}{0.986199in}}{\pgfqpoint{2.106736in}{0.981809in}}{\pgfqpoint{2.098923in}{0.973996in}}%
\pgfpathcurveto{\pgfqpoint{2.091109in}{0.966182in}}{\pgfqpoint{2.086719in}{0.955583in}}{\pgfqpoint{2.086719in}{0.944533in}}%
\pgfpathcurveto{\pgfqpoint{2.086719in}{0.933483in}}{\pgfqpoint{2.091109in}{0.922884in}}{\pgfqpoint{2.098923in}{0.915070in}}%
\pgfpathcurveto{\pgfqpoint{2.106736in}{0.907256in}}{\pgfqpoint{2.117336in}{0.902866in}}{\pgfqpoint{2.128386in}{0.902866in}}%
\pgfpathclose%
\pgfusepath{stroke,fill}%
\end{pgfscope}%
\begin{pgfscope}%
\pgfpathrectangle{\pgfqpoint{0.481978in}{0.331635in}}{\pgfqpoint{4.960000in}{3.696000in}}%
\pgfusepath{clip}%
\pgfsetbuttcap%
\pgfsetroundjoin%
\definecolor{currentfill}{rgb}{1.000000,0.705882,0.509804}%
\pgfsetfillcolor{currentfill}%
\pgfsetlinewidth{0.481800pt}%
\definecolor{currentstroke}{rgb}{1.000000,1.000000,1.000000}%
\pgfsetstrokecolor{currentstroke}%
\pgfsetdash{}{0pt}%
\pgfpathmoveto{\pgfqpoint{3.108327in}{1.029646in}}%
\pgfpathcurveto{\pgfqpoint{3.119377in}{1.029646in}}{\pgfqpoint{3.129976in}{1.034036in}}{\pgfqpoint{3.137790in}{1.041850in}}%
\pgfpathcurveto{\pgfqpoint{3.145604in}{1.049663in}}{\pgfqpoint{3.149994in}{1.060262in}}{\pgfqpoint{3.149994in}{1.071312in}}%
\pgfpathcurveto{\pgfqpoint{3.149994in}{1.082363in}}{\pgfqpoint{3.145604in}{1.092962in}}{\pgfqpoint{3.137790in}{1.100775in}}%
\pgfpathcurveto{\pgfqpoint{3.129976in}{1.108589in}}{\pgfqpoint{3.119377in}{1.112979in}}{\pgfqpoint{3.108327in}{1.112979in}}%
\pgfpathcurveto{\pgfqpoint{3.097277in}{1.112979in}}{\pgfqpoint{3.086678in}{1.108589in}}{\pgfqpoint{3.078864in}{1.100775in}}%
\pgfpathcurveto{\pgfqpoint{3.071051in}{1.092962in}}{\pgfqpoint{3.066661in}{1.082363in}}{\pgfqpoint{3.066661in}{1.071312in}}%
\pgfpathcurveto{\pgfqpoint{3.066661in}{1.060262in}}{\pgfqpoint{3.071051in}{1.049663in}}{\pgfqpoint{3.078864in}{1.041850in}}%
\pgfpathcurveto{\pgfqpoint{3.086678in}{1.034036in}}{\pgfqpoint{3.097277in}{1.029646in}}{\pgfqpoint{3.108327in}{1.029646in}}%
\pgfpathclose%
\pgfusepath{stroke,fill}%
\end{pgfscope}%
\begin{pgfscope}%
\pgfpathrectangle{\pgfqpoint{0.481978in}{0.331635in}}{\pgfqpoint{4.960000in}{3.696000in}}%
\pgfusepath{clip}%
\pgfsetbuttcap%
\pgfsetroundjoin%
\definecolor{currentfill}{rgb}{1.000000,0.705882,0.509804}%
\pgfsetfillcolor{currentfill}%
\pgfsetlinewidth{0.481800pt}%
\definecolor{currentstroke}{rgb}{1.000000,1.000000,1.000000}%
\pgfsetstrokecolor{currentstroke}%
\pgfsetdash{}{0pt}%
\pgfpathmoveto{\pgfqpoint{1.466879in}{1.631298in}}%
\pgfpathcurveto{\pgfqpoint{1.477929in}{1.631298in}}{\pgfqpoint{1.488528in}{1.635689in}}{\pgfqpoint{1.496342in}{1.643502in}}%
\pgfpathcurveto{\pgfqpoint{1.504156in}{1.651316in}}{\pgfqpoint{1.508546in}{1.661915in}}{\pgfqpoint{1.508546in}{1.672965in}}%
\pgfpathcurveto{\pgfqpoint{1.508546in}{1.684015in}}{\pgfqpoint{1.504156in}{1.694614in}}{\pgfqpoint{1.496342in}{1.702428in}}%
\pgfpathcurveto{\pgfqpoint{1.488528in}{1.710241in}}{\pgfqpoint{1.477929in}{1.714632in}}{\pgfqpoint{1.466879in}{1.714632in}}%
\pgfpathcurveto{\pgfqpoint{1.455829in}{1.714632in}}{\pgfqpoint{1.445230in}{1.710241in}}{\pgfqpoint{1.437416in}{1.702428in}}%
\pgfpathcurveto{\pgfqpoint{1.429603in}{1.694614in}}{\pgfqpoint{1.425212in}{1.684015in}}{\pgfqpoint{1.425212in}{1.672965in}}%
\pgfpathcurveto{\pgfqpoint{1.425212in}{1.661915in}}{\pgfqpoint{1.429603in}{1.651316in}}{\pgfqpoint{1.437416in}{1.643502in}}%
\pgfpathcurveto{\pgfqpoint{1.445230in}{1.635689in}}{\pgfqpoint{1.455829in}{1.631298in}}{\pgfqpoint{1.466879in}{1.631298in}}%
\pgfpathclose%
\pgfusepath{stroke,fill}%
\end{pgfscope}%
\begin{pgfscope}%
\pgfpathrectangle{\pgfqpoint{0.481978in}{0.331635in}}{\pgfqpoint{4.960000in}{3.696000in}}%
\pgfusepath{clip}%
\pgfsetbuttcap%
\pgfsetroundjoin%
\definecolor{currentfill}{rgb}{1.000000,0.705882,0.509804}%
\pgfsetfillcolor{currentfill}%
\pgfsetlinewidth{0.481800pt}%
\definecolor{currentstroke}{rgb}{1.000000,1.000000,1.000000}%
\pgfsetstrokecolor{currentstroke}%
\pgfsetdash{}{0pt}%
\pgfpathmoveto{\pgfqpoint{2.321041in}{1.343466in}}%
\pgfpathcurveto{\pgfqpoint{2.332091in}{1.343466in}}{\pgfqpoint{2.342690in}{1.347857in}}{\pgfqpoint{2.350504in}{1.355670in}}%
\pgfpathcurveto{\pgfqpoint{2.358317in}{1.363484in}}{\pgfqpoint{2.362708in}{1.374083in}}{\pgfqpoint{2.362708in}{1.385133in}}%
\pgfpathcurveto{\pgfqpoint{2.362708in}{1.396183in}}{\pgfqpoint{2.358317in}{1.406782in}}{\pgfqpoint{2.350504in}{1.414596in}}%
\pgfpathcurveto{\pgfqpoint{2.342690in}{1.422409in}}{\pgfqpoint{2.332091in}{1.426800in}}{\pgfqpoint{2.321041in}{1.426800in}}%
\pgfpathcurveto{\pgfqpoint{2.309991in}{1.426800in}}{\pgfqpoint{2.299392in}{1.422409in}}{\pgfqpoint{2.291578in}{1.414596in}}%
\pgfpathcurveto{\pgfqpoint{2.283764in}{1.406782in}}{\pgfqpoint{2.279374in}{1.396183in}}{\pgfqpoint{2.279374in}{1.385133in}}%
\pgfpathcurveto{\pgfqpoint{2.279374in}{1.374083in}}{\pgfqpoint{2.283764in}{1.363484in}}{\pgfqpoint{2.291578in}{1.355670in}}%
\pgfpathcurveto{\pgfqpoint{2.299392in}{1.347857in}}{\pgfqpoint{2.309991in}{1.343466in}}{\pgfqpoint{2.321041in}{1.343466in}}%
\pgfpathclose%
\pgfusepath{stroke,fill}%
\end{pgfscope}%
\begin{pgfscope}%
\pgfpathrectangle{\pgfqpoint{0.481978in}{0.331635in}}{\pgfqpoint{4.960000in}{3.696000in}}%
\pgfusepath{clip}%
\pgfsetbuttcap%
\pgfsetroundjoin%
\definecolor{currentfill}{rgb}{1.000000,0.705882,0.509804}%
\pgfsetfillcolor{currentfill}%
\pgfsetlinewidth{0.481800pt}%
\definecolor{currentstroke}{rgb}{1.000000,1.000000,1.000000}%
\pgfsetstrokecolor{currentstroke}%
\pgfsetdash{}{0pt}%
\pgfpathmoveto{\pgfqpoint{3.747692in}{1.442802in}}%
\pgfpathcurveto{\pgfqpoint{3.758742in}{1.442802in}}{\pgfqpoint{3.769341in}{1.447192in}}{\pgfqpoint{3.777155in}{1.455005in}}%
\pgfpathcurveto{\pgfqpoint{3.784968in}{1.462819in}}{\pgfqpoint{3.789359in}{1.473418in}}{\pgfqpoint{3.789359in}{1.484468in}}%
\pgfpathcurveto{\pgfqpoint{3.789359in}{1.495518in}}{\pgfqpoint{3.784968in}{1.506117in}}{\pgfqpoint{3.777155in}{1.513931in}}%
\pgfpathcurveto{\pgfqpoint{3.769341in}{1.521745in}}{\pgfqpoint{3.758742in}{1.526135in}}{\pgfqpoint{3.747692in}{1.526135in}}%
\pgfpathcurveto{\pgfqpoint{3.736642in}{1.526135in}}{\pgfqpoint{3.726043in}{1.521745in}}{\pgfqpoint{3.718229in}{1.513931in}}%
\pgfpathcurveto{\pgfqpoint{3.710416in}{1.506117in}}{\pgfqpoint{3.706025in}{1.495518in}}{\pgfqpoint{3.706025in}{1.484468in}}%
\pgfpathcurveto{\pgfqpoint{3.706025in}{1.473418in}}{\pgfqpoint{3.710416in}{1.462819in}}{\pgfqpoint{3.718229in}{1.455005in}}%
\pgfpathcurveto{\pgfqpoint{3.726043in}{1.447192in}}{\pgfqpoint{3.736642in}{1.442802in}}{\pgfqpoint{3.747692in}{1.442802in}}%
\pgfpathclose%
\pgfusepath{stroke,fill}%
\end{pgfscope}%
\begin{pgfscope}%
\pgfpathrectangle{\pgfqpoint{0.481978in}{0.331635in}}{\pgfqpoint{4.960000in}{3.696000in}}%
\pgfusepath{clip}%
\pgfsetbuttcap%
\pgfsetroundjoin%
\definecolor{currentfill}{rgb}{1.000000,0.705882,0.509804}%
\pgfsetfillcolor{currentfill}%
\pgfsetlinewidth{0.481800pt}%
\definecolor{currentstroke}{rgb}{1.000000,1.000000,1.000000}%
\pgfsetstrokecolor{currentstroke}%
\pgfsetdash{}{0pt}%
\pgfpathmoveto{\pgfqpoint{0.914552in}{2.017857in}}%
\pgfpathcurveto{\pgfqpoint{0.925602in}{2.017857in}}{\pgfqpoint{0.936201in}{2.022248in}}{\pgfqpoint{0.944014in}{2.030061in}}%
\pgfpathcurveto{\pgfqpoint{0.951828in}{2.037875in}}{\pgfqpoint{0.956218in}{2.048474in}}{\pgfqpoint{0.956218in}{2.059524in}}%
\pgfpathcurveto{\pgfqpoint{0.956218in}{2.070574in}}{\pgfqpoint{0.951828in}{2.081173in}}{\pgfqpoint{0.944014in}{2.088987in}}%
\pgfpathcurveto{\pgfqpoint{0.936201in}{2.096801in}}{\pgfqpoint{0.925602in}{2.101191in}}{\pgfqpoint{0.914552in}{2.101191in}}%
\pgfpathcurveto{\pgfqpoint{0.903502in}{2.101191in}}{\pgfqpoint{0.892903in}{2.096801in}}{\pgfqpoint{0.885089in}{2.088987in}}%
\pgfpathcurveto{\pgfqpoint{0.877275in}{2.081173in}}{\pgfqpoint{0.872885in}{2.070574in}}{\pgfqpoint{0.872885in}{2.059524in}}%
\pgfpathcurveto{\pgfqpoint{0.872885in}{2.048474in}}{\pgfqpoint{0.877275in}{2.037875in}}{\pgfqpoint{0.885089in}{2.030061in}}%
\pgfpathcurveto{\pgfqpoint{0.892903in}{2.022248in}}{\pgfqpoint{0.903502in}{2.017857in}}{\pgfqpoint{0.914552in}{2.017857in}}%
\pgfpathclose%
\pgfusepath{stroke,fill}%
\end{pgfscope}%
\begin{pgfscope}%
\pgfpathrectangle{\pgfqpoint{0.481978in}{0.331635in}}{\pgfqpoint{4.960000in}{3.696000in}}%
\pgfusepath{clip}%
\pgfsetbuttcap%
\pgfsetroundjoin%
\definecolor{currentfill}{rgb}{1.000000,0.705882,0.509804}%
\pgfsetfillcolor{currentfill}%
\pgfsetlinewidth{0.481800pt}%
\definecolor{currentstroke}{rgb}{1.000000,1.000000,1.000000}%
\pgfsetstrokecolor{currentstroke}%
\pgfsetdash{}{0pt}%
\pgfpathmoveto{\pgfqpoint{2.729676in}{1.720457in}}%
\pgfpathcurveto{\pgfqpoint{2.740726in}{1.720457in}}{\pgfqpoint{2.751325in}{1.724847in}}{\pgfqpoint{2.759139in}{1.732661in}}%
\pgfpathcurveto{\pgfqpoint{2.766952in}{1.740474in}}{\pgfqpoint{2.771343in}{1.751073in}}{\pgfqpoint{2.771343in}{1.762123in}}%
\pgfpathcurveto{\pgfqpoint{2.771343in}{1.773174in}}{\pgfqpoint{2.766952in}{1.783773in}}{\pgfqpoint{2.759139in}{1.791586in}}%
\pgfpathcurveto{\pgfqpoint{2.751325in}{1.799400in}}{\pgfqpoint{2.740726in}{1.803790in}}{\pgfqpoint{2.729676in}{1.803790in}}%
\pgfpathcurveto{\pgfqpoint{2.718626in}{1.803790in}}{\pgfqpoint{2.708027in}{1.799400in}}{\pgfqpoint{2.700213in}{1.791586in}}%
\pgfpathcurveto{\pgfqpoint{2.692400in}{1.783773in}}{\pgfqpoint{2.688009in}{1.773174in}}{\pgfqpoint{2.688009in}{1.762123in}}%
\pgfpathcurveto{\pgfqpoint{2.688009in}{1.751073in}}{\pgfqpoint{2.692400in}{1.740474in}}{\pgfqpoint{2.700213in}{1.732661in}}%
\pgfpathcurveto{\pgfqpoint{2.708027in}{1.724847in}}{\pgfqpoint{2.718626in}{1.720457in}}{\pgfqpoint{2.729676in}{1.720457in}}%
\pgfpathclose%
\pgfusepath{stroke,fill}%
\end{pgfscope}%
\begin{pgfscope}%
\pgfpathrectangle{\pgfqpoint{0.481978in}{0.331635in}}{\pgfqpoint{4.960000in}{3.696000in}}%
\pgfusepath{clip}%
\pgfsetbuttcap%
\pgfsetroundjoin%
\definecolor{currentfill}{rgb}{1.000000,0.705882,0.509804}%
\pgfsetfillcolor{currentfill}%
\pgfsetlinewidth{0.481800pt}%
\definecolor{currentstroke}{rgb}{1.000000,1.000000,1.000000}%
\pgfsetstrokecolor{currentstroke}%
\pgfsetdash{}{0pt}%
\pgfpathmoveto{\pgfqpoint{1.522149in}{1.710169in}}%
\pgfpathcurveto{\pgfqpoint{1.533199in}{1.710169in}}{\pgfqpoint{1.543798in}{1.714559in}}{\pgfqpoint{1.551612in}{1.722373in}}%
\pgfpathcurveto{\pgfqpoint{1.559426in}{1.730186in}}{\pgfqpoint{1.563816in}{1.740785in}}{\pgfqpoint{1.563816in}{1.751836in}}%
\pgfpathcurveto{\pgfqpoint{1.563816in}{1.762886in}}{\pgfqpoint{1.559426in}{1.773485in}}{\pgfqpoint{1.551612in}{1.781298in}}%
\pgfpathcurveto{\pgfqpoint{1.543798in}{1.789112in}}{\pgfqpoint{1.533199in}{1.793502in}}{\pgfqpoint{1.522149in}{1.793502in}}%
\pgfpathcurveto{\pgfqpoint{1.511099in}{1.793502in}}{\pgfqpoint{1.500500in}{1.789112in}}{\pgfqpoint{1.492686in}{1.781298in}}%
\pgfpathcurveto{\pgfqpoint{1.484873in}{1.773485in}}{\pgfqpoint{1.480483in}{1.762886in}}{\pgfqpoint{1.480483in}{1.751836in}}%
\pgfpathcurveto{\pgfqpoint{1.480483in}{1.740785in}}{\pgfqpoint{1.484873in}{1.730186in}}{\pgfqpoint{1.492686in}{1.722373in}}%
\pgfpathcurveto{\pgfqpoint{1.500500in}{1.714559in}}{\pgfqpoint{1.511099in}{1.710169in}}{\pgfqpoint{1.522149in}{1.710169in}}%
\pgfpathclose%
\pgfusepath{stroke,fill}%
\end{pgfscope}%
\begin{pgfscope}%
\pgfpathrectangle{\pgfqpoint{0.481978in}{0.331635in}}{\pgfqpoint{4.960000in}{3.696000in}}%
\pgfusepath{clip}%
\pgfsetbuttcap%
\pgfsetroundjoin%
\definecolor{currentfill}{rgb}{1.000000,0.705882,0.509804}%
\pgfsetfillcolor{currentfill}%
\pgfsetlinewidth{0.481800pt}%
\definecolor{currentstroke}{rgb}{1.000000,1.000000,1.000000}%
\pgfsetstrokecolor{currentstroke}%
\pgfsetdash{}{0pt}%
\pgfpathmoveto{\pgfqpoint{3.401713in}{1.454007in}}%
\pgfpathcurveto{\pgfqpoint{3.412763in}{1.454007in}}{\pgfqpoint{3.423362in}{1.458397in}}{\pgfqpoint{3.431176in}{1.466211in}}%
\pgfpathcurveto{\pgfqpoint{3.438990in}{1.474024in}}{\pgfqpoint{3.443380in}{1.484623in}}{\pgfqpoint{3.443380in}{1.495674in}}%
\pgfpathcurveto{\pgfqpoint{3.443380in}{1.506724in}}{\pgfqpoint{3.438990in}{1.517323in}}{\pgfqpoint{3.431176in}{1.525136in}}%
\pgfpathcurveto{\pgfqpoint{3.423362in}{1.532950in}}{\pgfqpoint{3.412763in}{1.537340in}}{\pgfqpoint{3.401713in}{1.537340in}}%
\pgfpathcurveto{\pgfqpoint{3.390663in}{1.537340in}}{\pgfqpoint{3.380064in}{1.532950in}}{\pgfqpoint{3.372250in}{1.525136in}}%
\pgfpathcurveto{\pgfqpoint{3.364437in}{1.517323in}}{\pgfqpoint{3.360047in}{1.506724in}}{\pgfqpoint{3.360047in}{1.495674in}}%
\pgfpathcurveto{\pgfqpoint{3.360047in}{1.484623in}}{\pgfqpoint{3.364437in}{1.474024in}}{\pgfqpoint{3.372250in}{1.466211in}}%
\pgfpathcurveto{\pgfqpoint{3.380064in}{1.458397in}}{\pgfqpoint{3.390663in}{1.454007in}}{\pgfqpoint{3.401713in}{1.454007in}}%
\pgfpathclose%
\pgfusepath{stroke,fill}%
\end{pgfscope}%
\begin{pgfscope}%
\pgfpathrectangle{\pgfqpoint{0.481978in}{0.331635in}}{\pgfqpoint{4.960000in}{3.696000in}}%
\pgfusepath{clip}%
\pgfsetbuttcap%
\pgfsetroundjoin%
\definecolor{currentfill}{rgb}{1.000000,0.705882,0.509804}%
\pgfsetfillcolor{currentfill}%
\pgfsetlinewidth{0.481800pt}%
\definecolor{currentstroke}{rgb}{1.000000,1.000000,1.000000}%
\pgfsetstrokecolor{currentstroke}%
\pgfsetdash{}{0pt}%
\pgfpathmoveto{\pgfqpoint{2.343714in}{1.520962in}}%
\pgfpathcurveto{\pgfqpoint{2.354764in}{1.520962in}}{\pgfqpoint{2.365363in}{1.525352in}}{\pgfqpoint{2.373176in}{1.533166in}}%
\pgfpathcurveto{\pgfqpoint{2.380990in}{1.540979in}}{\pgfqpoint{2.385380in}{1.551578in}}{\pgfqpoint{2.385380in}{1.562628in}}%
\pgfpathcurveto{\pgfqpoint{2.385380in}{1.573678in}}{\pgfqpoint{2.380990in}{1.584278in}}{\pgfqpoint{2.373176in}{1.592091in}}%
\pgfpathcurveto{\pgfqpoint{2.365363in}{1.599905in}}{\pgfqpoint{2.354764in}{1.604295in}}{\pgfqpoint{2.343714in}{1.604295in}}%
\pgfpathcurveto{\pgfqpoint{2.332663in}{1.604295in}}{\pgfqpoint{2.322064in}{1.599905in}}{\pgfqpoint{2.314251in}{1.592091in}}%
\pgfpathcurveto{\pgfqpoint{2.306437in}{1.584278in}}{\pgfqpoint{2.302047in}{1.573678in}}{\pgfqpoint{2.302047in}{1.562628in}}%
\pgfpathcurveto{\pgfqpoint{2.302047in}{1.551578in}}{\pgfqpoint{2.306437in}{1.540979in}}{\pgfqpoint{2.314251in}{1.533166in}}%
\pgfpathcurveto{\pgfqpoint{2.322064in}{1.525352in}}{\pgfqpoint{2.332663in}{1.520962in}}{\pgfqpoint{2.343714in}{1.520962in}}%
\pgfpathclose%
\pgfusepath{stroke,fill}%
\end{pgfscope}%
\begin{pgfscope}%
\pgfpathrectangle{\pgfqpoint{0.481978in}{0.331635in}}{\pgfqpoint{4.960000in}{3.696000in}}%
\pgfusepath{clip}%
\pgfsetbuttcap%
\pgfsetroundjoin%
\definecolor{currentfill}{rgb}{1.000000,0.705882,0.509804}%
\pgfsetfillcolor{currentfill}%
\pgfsetlinewidth{0.481800pt}%
\definecolor{currentstroke}{rgb}{1.000000,1.000000,1.000000}%
\pgfsetstrokecolor{currentstroke}%
\pgfsetdash{}{0pt}%
\pgfpathmoveto{\pgfqpoint{3.396230in}{1.321680in}}%
\pgfpathcurveto{\pgfqpoint{3.407280in}{1.321680in}}{\pgfqpoint{3.417879in}{1.326070in}}{\pgfqpoint{3.425693in}{1.333884in}}%
\pgfpathcurveto{\pgfqpoint{3.433507in}{1.341698in}}{\pgfqpoint{3.437897in}{1.352297in}}{\pgfqpoint{3.437897in}{1.363347in}}%
\pgfpathcurveto{\pgfqpoint{3.437897in}{1.374397in}}{\pgfqpoint{3.433507in}{1.384996in}}{\pgfqpoint{3.425693in}{1.392810in}}%
\pgfpathcurveto{\pgfqpoint{3.417879in}{1.400623in}}{\pgfqpoint{3.407280in}{1.405013in}}{\pgfqpoint{3.396230in}{1.405013in}}%
\pgfpathcurveto{\pgfqpoint{3.385180in}{1.405013in}}{\pgfqpoint{3.374581in}{1.400623in}}{\pgfqpoint{3.366767in}{1.392810in}}%
\pgfpathcurveto{\pgfqpoint{3.358954in}{1.384996in}}{\pgfqpoint{3.354563in}{1.374397in}}{\pgfqpoint{3.354563in}{1.363347in}}%
\pgfpathcurveto{\pgfqpoint{3.354563in}{1.352297in}}{\pgfqpoint{3.358954in}{1.341698in}}{\pgfqpoint{3.366767in}{1.333884in}}%
\pgfpathcurveto{\pgfqpoint{3.374581in}{1.326070in}}{\pgfqpoint{3.385180in}{1.321680in}}{\pgfqpoint{3.396230in}{1.321680in}}%
\pgfpathclose%
\pgfusepath{stroke,fill}%
\end{pgfscope}%
\begin{pgfscope}%
\pgfpathrectangle{\pgfqpoint{0.481978in}{0.331635in}}{\pgfqpoint{4.960000in}{3.696000in}}%
\pgfusepath{clip}%
\pgfsetbuttcap%
\pgfsetroundjoin%
\definecolor{currentfill}{rgb}{1.000000,0.705882,0.509804}%
\pgfsetfillcolor{currentfill}%
\pgfsetlinewidth{0.481800pt}%
\definecolor{currentstroke}{rgb}{1.000000,1.000000,1.000000}%
\pgfsetstrokecolor{currentstroke}%
\pgfsetdash{}{0pt}%
\pgfpathmoveto{\pgfqpoint{0.724960in}{1.307404in}}%
\pgfpathcurveto{\pgfqpoint{0.736010in}{1.307404in}}{\pgfqpoint{0.746609in}{1.311794in}}{\pgfqpoint{0.754423in}{1.319607in}}%
\pgfpathcurveto{\pgfqpoint{0.762236in}{1.327421in}}{\pgfqpoint{0.766627in}{1.338020in}}{\pgfqpoint{0.766627in}{1.349070in}}%
\pgfpathcurveto{\pgfqpoint{0.766627in}{1.360120in}}{\pgfqpoint{0.762236in}{1.370719in}}{\pgfqpoint{0.754423in}{1.378533in}}%
\pgfpathcurveto{\pgfqpoint{0.746609in}{1.386347in}}{\pgfqpoint{0.736010in}{1.390737in}}{\pgfqpoint{0.724960in}{1.390737in}}%
\pgfpathcurveto{\pgfqpoint{0.713910in}{1.390737in}}{\pgfqpoint{0.703311in}{1.386347in}}{\pgfqpoint{0.695497in}{1.378533in}}%
\pgfpathcurveto{\pgfqpoint{0.687684in}{1.370719in}}{\pgfqpoint{0.683293in}{1.360120in}}{\pgfqpoint{0.683293in}{1.349070in}}%
\pgfpathcurveto{\pgfqpoint{0.683293in}{1.338020in}}{\pgfqpoint{0.687684in}{1.327421in}}{\pgfqpoint{0.695497in}{1.319607in}}%
\pgfpathcurveto{\pgfqpoint{0.703311in}{1.311794in}}{\pgfqpoint{0.713910in}{1.307404in}}{\pgfqpoint{0.724960in}{1.307404in}}%
\pgfpathclose%
\pgfusepath{stroke,fill}%
\end{pgfscope}%
\begin{pgfscope}%
\pgfpathrectangle{\pgfqpoint{0.481978in}{0.331635in}}{\pgfqpoint{4.960000in}{3.696000in}}%
\pgfusepath{clip}%
\pgfsetbuttcap%
\pgfsetroundjoin%
\definecolor{currentfill}{rgb}{1.000000,0.705882,0.509804}%
\pgfsetfillcolor{currentfill}%
\pgfsetlinewidth{0.481800pt}%
\definecolor{currentstroke}{rgb}{1.000000,1.000000,1.000000}%
\pgfsetstrokecolor{currentstroke}%
\pgfsetdash{}{0pt}%
\pgfpathmoveto{\pgfqpoint{1.932203in}{0.874574in}}%
\pgfpathcurveto{\pgfqpoint{1.943253in}{0.874574in}}{\pgfqpoint{1.953852in}{0.878964in}}{\pgfqpoint{1.961666in}{0.886778in}}%
\pgfpathcurveto{\pgfqpoint{1.969479in}{0.894591in}}{\pgfqpoint{1.973870in}{0.905190in}}{\pgfqpoint{1.973870in}{0.916240in}}%
\pgfpathcurveto{\pgfqpoint{1.973870in}{0.927290in}}{\pgfqpoint{1.969479in}{0.937889in}}{\pgfqpoint{1.961666in}{0.945703in}}%
\pgfpathcurveto{\pgfqpoint{1.953852in}{0.953517in}}{\pgfqpoint{1.943253in}{0.957907in}}{\pgfqpoint{1.932203in}{0.957907in}}%
\pgfpathcurveto{\pgfqpoint{1.921153in}{0.957907in}}{\pgfqpoint{1.910554in}{0.953517in}}{\pgfqpoint{1.902740in}{0.945703in}}%
\pgfpathcurveto{\pgfqpoint{1.894926in}{0.937889in}}{\pgfqpoint{1.890536in}{0.927290in}}{\pgfqpoint{1.890536in}{0.916240in}}%
\pgfpathcurveto{\pgfqpoint{1.890536in}{0.905190in}}{\pgfqpoint{1.894926in}{0.894591in}}{\pgfqpoint{1.902740in}{0.886778in}}%
\pgfpathcurveto{\pgfqpoint{1.910554in}{0.878964in}}{\pgfqpoint{1.921153in}{0.874574in}}{\pgfqpoint{1.932203in}{0.874574in}}%
\pgfpathclose%
\pgfusepath{stroke,fill}%
\end{pgfscope}%
\begin{pgfscope}%
\pgfpathrectangle{\pgfqpoint{0.481978in}{0.331635in}}{\pgfqpoint{4.960000in}{3.696000in}}%
\pgfusepath{clip}%
\pgfsetbuttcap%
\pgfsetroundjoin%
\definecolor{currentfill}{rgb}{1.000000,0.705882,0.509804}%
\pgfsetfillcolor{currentfill}%
\pgfsetlinewidth{0.481800pt}%
\definecolor{currentstroke}{rgb}{1.000000,1.000000,1.000000}%
\pgfsetstrokecolor{currentstroke}%
\pgfsetdash{}{0pt}%
\pgfpathmoveto{\pgfqpoint{3.426668in}{0.636626in}}%
\pgfpathcurveto{\pgfqpoint{3.437718in}{0.636626in}}{\pgfqpoint{3.448317in}{0.641016in}}{\pgfqpoint{3.456130in}{0.648830in}}%
\pgfpathcurveto{\pgfqpoint{3.463944in}{0.656643in}}{\pgfqpoint{3.468334in}{0.667242in}}{\pgfqpoint{3.468334in}{0.678293in}}%
\pgfpathcurveto{\pgfqpoint{3.468334in}{0.689343in}}{\pgfqpoint{3.463944in}{0.699942in}}{\pgfqpoint{3.456130in}{0.707755in}}%
\pgfpathcurveto{\pgfqpoint{3.448317in}{0.715569in}}{\pgfqpoint{3.437718in}{0.719959in}}{\pgfqpoint{3.426668in}{0.719959in}}%
\pgfpathcurveto{\pgfqpoint{3.415618in}{0.719959in}}{\pgfqpoint{3.405019in}{0.715569in}}{\pgfqpoint{3.397205in}{0.707755in}}%
\pgfpathcurveto{\pgfqpoint{3.389391in}{0.699942in}}{\pgfqpoint{3.385001in}{0.689343in}}{\pgfqpoint{3.385001in}{0.678293in}}%
\pgfpathcurveto{\pgfqpoint{3.385001in}{0.667242in}}{\pgfqpoint{3.389391in}{0.656643in}}{\pgfqpoint{3.397205in}{0.648830in}}%
\pgfpathcurveto{\pgfqpoint{3.405019in}{0.641016in}}{\pgfqpoint{3.415618in}{0.636626in}}{\pgfqpoint{3.426668in}{0.636626in}}%
\pgfpathclose%
\pgfusepath{stroke,fill}%
\end{pgfscope}%
\begin{pgfscope}%
\pgfpathrectangle{\pgfqpoint{0.481978in}{0.331635in}}{\pgfqpoint{4.960000in}{3.696000in}}%
\pgfusepath{clip}%
\pgfsetbuttcap%
\pgfsetroundjoin%
\definecolor{currentfill}{rgb}{1.000000,0.705882,0.509804}%
\pgfsetfillcolor{currentfill}%
\pgfsetlinewidth{0.481800pt}%
\definecolor{currentstroke}{rgb}{1.000000,1.000000,1.000000}%
\pgfsetstrokecolor{currentstroke}%
\pgfsetdash{}{0pt}%
\pgfpathmoveto{\pgfqpoint{3.978978in}{1.101820in}}%
\pgfpathcurveto{\pgfqpoint{3.990029in}{1.101820in}}{\pgfqpoint{4.000628in}{1.106211in}}{\pgfqpoint{4.008441in}{1.114024in}}%
\pgfpathcurveto{\pgfqpoint{4.016255in}{1.121838in}}{\pgfqpoint{4.020645in}{1.132437in}}{\pgfqpoint{4.020645in}{1.143487in}}%
\pgfpathcurveto{\pgfqpoint{4.020645in}{1.154537in}}{\pgfqpoint{4.016255in}{1.165136in}}{\pgfqpoint{4.008441in}{1.172950in}}%
\pgfpathcurveto{\pgfqpoint{4.000628in}{1.180763in}}{\pgfqpoint{3.990029in}{1.185154in}}{\pgfqpoint{3.978978in}{1.185154in}}%
\pgfpathcurveto{\pgfqpoint{3.967928in}{1.185154in}}{\pgfqpoint{3.957329in}{1.180763in}}{\pgfqpoint{3.949516in}{1.172950in}}%
\pgfpathcurveto{\pgfqpoint{3.941702in}{1.165136in}}{\pgfqpoint{3.937312in}{1.154537in}}{\pgfqpoint{3.937312in}{1.143487in}}%
\pgfpathcurveto{\pgfqpoint{3.937312in}{1.132437in}}{\pgfqpoint{3.941702in}{1.121838in}}{\pgfqpoint{3.949516in}{1.114024in}}%
\pgfpathcurveto{\pgfqpoint{3.957329in}{1.106211in}}{\pgfqpoint{3.967928in}{1.101820in}}{\pgfqpoint{3.978978in}{1.101820in}}%
\pgfpathclose%
\pgfusepath{stroke,fill}%
\end{pgfscope}%
\begin{pgfscope}%
\pgfpathrectangle{\pgfqpoint{0.481978in}{0.331635in}}{\pgfqpoint{4.960000in}{3.696000in}}%
\pgfusepath{clip}%
\pgfsetbuttcap%
\pgfsetroundjoin%
\definecolor{currentfill}{rgb}{1.000000,0.705882,0.509804}%
\pgfsetfillcolor{currentfill}%
\pgfsetlinewidth{0.481800pt}%
\definecolor{currentstroke}{rgb}{1.000000,1.000000,1.000000}%
\pgfsetstrokecolor{currentstroke}%
\pgfsetdash{}{0pt}%
\pgfpathmoveto{\pgfqpoint{2.935351in}{1.222051in}}%
\pgfpathcurveto{\pgfqpoint{2.946401in}{1.222051in}}{\pgfqpoint{2.957000in}{1.226442in}}{\pgfqpoint{2.964814in}{1.234255in}}%
\pgfpathcurveto{\pgfqpoint{2.972627in}{1.242069in}}{\pgfqpoint{2.977018in}{1.252668in}}{\pgfqpoint{2.977018in}{1.263718in}}%
\pgfpathcurveto{\pgfqpoint{2.977018in}{1.274768in}}{\pgfqpoint{2.972627in}{1.285367in}}{\pgfqpoint{2.964814in}{1.293181in}}%
\pgfpathcurveto{\pgfqpoint{2.957000in}{1.300994in}}{\pgfqpoint{2.946401in}{1.305385in}}{\pgfqpoint{2.935351in}{1.305385in}}%
\pgfpathcurveto{\pgfqpoint{2.924301in}{1.305385in}}{\pgfqpoint{2.913702in}{1.300994in}}{\pgfqpoint{2.905888in}{1.293181in}}%
\pgfpathcurveto{\pgfqpoint{2.898075in}{1.285367in}}{\pgfqpoint{2.893684in}{1.274768in}}{\pgfqpoint{2.893684in}{1.263718in}}%
\pgfpathcurveto{\pgfqpoint{2.893684in}{1.252668in}}{\pgfqpoint{2.898075in}{1.242069in}}{\pgfqpoint{2.905888in}{1.234255in}}%
\pgfpathcurveto{\pgfqpoint{2.913702in}{1.226442in}}{\pgfqpoint{2.924301in}{1.222051in}}{\pgfqpoint{2.935351in}{1.222051in}}%
\pgfpathclose%
\pgfusepath{stroke,fill}%
\end{pgfscope}%
\begin{pgfscope}%
\pgfpathrectangle{\pgfqpoint{0.481978in}{0.331635in}}{\pgfqpoint{4.960000in}{3.696000in}}%
\pgfusepath{clip}%
\pgfsetbuttcap%
\pgfsetroundjoin%
\definecolor{currentfill}{rgb}{1.000000,0.705882,0.509804}%
\pgfsetfillcolor{currentfill}%
\pgfsetlinewidth{0.481800pt}%
\definecolor{currentstroke}{rgb}{1.000000,1.000000,1.000000}%
\pgfsetstrokecolor{currentstroke}%
\pgfsetdash{}{0pt}%
\pgfpathmoveto{\pgfqpoint{3.300863in}{1.095727in}}%
\pgfpathcurveto{\pgfqpoint{3.311913in}{1.095727in}}{\pgfqpoint{3.322512in}{1.100117in}}{\pgfqpoint{3.330325in}{1.107931in}}%
\pgfpathcurveto{\pgfqpoint{3.338139in}{1.115744in}}{\pgfqpoint{3.342529in}{1.126343in}}{\pgfqpoint{3.342529in}{1.137393in}}%
\pgfpathcurveto{\pgfqpoint{3.342529in}{1.148443in}}{\pgfqpoint{3.338139in}{1.159042in}}{\pgfqpoint{3.330325in}{1.166856in}}%
\pgfpathcurveto{\pgfqpoint{3.322512in}{1.174670in}}{\pgfqpoint{3.311913in}{1.179060in}}{\pgfqpoint{3.300863in}{1.179060in}}%
\pgfpathcurveto{\pgfqpoint{3.289813in}{1.179060in}}{\pgfqpoint{3.279214in}{1.174670in}}{\pgfqpoint{3.271400in}{1.166856in}}%
\pgfpathcurveto{\pgfqpoint{3.263586in}{1.159042in}}{\pgfqpoint{3.259196in}{1.148443in}}{\pgfqpoint{3.259196in}{1.137393in}}%
\pgfpathcurveto{\pgfqpoint{3.259196in}{1.126343in}}{\pgfqpoint{3.263586in}{1.115744in}}{\pgfqpoint{3.271400in}{1.107931in}}%
\pgfpathcurveto{\pgfqpoint{3.279214in}{1.100117in}}{\pgfqpoint{3.289813in}{1.095727in}}{\pgfqpoint{3.300863in}{1.095727in}}%
\pgfpathclose%
\pgfusepath{stroke,fill}%
\end{pgfscope}%
\begin{pgfscope}%
\pgfpathrectangle{\pgfqpoint{0.481978in}{0.331635in}}{\pgfqpoint{4.960000in}{3.696000in}}%
\pgfusepath{clip}%
\pgfsetbuttcap%
\pgfsetroundjoin%
\definecolor{currentfill}{rgb}{1.000000,0.705882,0.509804}%
\pgfsetfillcolor{currentfill}%
\pgfsetlinewidth{0.481800pt}%
\definecolor{currentstroke}{rgb}{1.000000,1.000000,1.000000}%
\pgfsetstrokecolor{currentstroke}%
\pgfsetdash{}{0pt}%
\pgfpathmoveto{\pgfqpoint{3.264488in}{1.827032in}}%
\pgfpathcurveto{\pgfqpoint{3.275538in}{1.827032in}}{\pgfqpoint{3.286137in}{1.831423in}}{\pgfqpoint{3.293951in}{1.839236in}}%
\pgfpathcurveto{\pgfqpoint{3.301764in}{1.847050in}}{\pgfqpoint{3.306154in}{1.857649in}}{\pgfqpoint{3.306154in}{1.868699in}}%
\pgfpathcurveto{\pgfqpoint{3.306154in}{1.879749in}}{\pgfqpoint{3.301764in}{1.890348in}}{\pgfqpoint{3.293951in}{1.898162in}}%
\pgfpathcurveto{\pgfqpoint{3.286137in}{1.905975in}}{\pgfqpoint{3.275538in}{1.910366in}}{\pgfqpoint{3.264488in}{1.910366in}}%
\pgfpathcurveto{\pgfqpoint{3.253438in}{1.910366in}}{\pgfqpoint{3.242839in}{1.905975in}}{\pgfqpoint{3.235025in}{1.898162in}}%
\pgfpathcurveto{\pgfqpoint{3.227211in}{1.890348in}}{\pgfqpoint{3.222821in}{1.879749in}}{\pgfqpoint{3.222821in}{1.868699in}}%
\pgfpathcurveto{\pgfqpoint{3.222821in}{1.857649in}}{\pgfqpoint{3.227211in}{1.847050in}}{\pgfqpoint{3.235025in}{1.839236in}}%
\pgfpathcurveto{\pgfqpoint{3.242839in}{1.831423in}}{\pgfqpoint{3.253438in}{1.827032in}}{\pgfqpoint{3.264488in}{1.827032in}}%
\pgfpathclose%
\pgfusepath{stroke,fill}%
\end{pgfscope}%
\begin{pgfscope}%
\pgfpathrectangle{\pgfqpoint{0.481978in}{0.331635in}}{\pgfqpoint{4.960000in}{3.696000in}}%
\pgfusepath{clip}%
\pgfsetbuttcap%
\pgfsetroundjoin%
\definecolor{currentfill}{rgb}{1.000000,0.705882,0.509804}%
\pgfsetfillcolor{currentfill}%
\pgfsetlinewidth{0.481800pt}%
\definecolor{currentstroke}{rgb}{1.000000,1.000000,1.000000}%
\pgfsetstrokecolor{currentstroke}%
\pgfsetdash{}{0pt}%
\pgfpathmoveto{\pgfqpoint{2.950725in}{1.306996in}}%
\pgfpathcurveto{\pgfqpoint{2.961775in}{1.306996in}}{\pgfqpoint{2.972374in}{1.311387in}}{\pgfqpoint{2.980188in}{1.319200in}}%
\pgfpathcurveto{\pgfqpoint{2.988001in}{1.327014in}}{\pgfqpoint{2.992392in}{1.337613in}}{\pgfqpoint{2.992392in}{1.348663in}}%
\pgfpathcurveto{\pgfqpoint{2.992392in}{1.359713in}}{\pgfqpoint{2.988001in}{1.370312in}}{\pgfqpoint{2.980188in}{1.378126in}}%
\pgfpathcurveto{\pgfqpoint{2.972374in}{1.385939in}}{\pgfqpoint{2.961775in}{1.390330in}}{\pgfqpoint{2.950725in}{1.390330in}}%
\pgfpathcurveto{\pgfqpoint{2.939675in}{1.390330in}}{\pgfqpoint{2.929076in}{1.385939in}}{\pgfqpoint{2.921262in}{1.378126in}}%
\pgfpathcurveto{\pgfqpoint{2.913448in}{1.370312in}}{\pgfqpoint{2.909058in}{1.359713in}}{\pgfqpoint{2.909058in}{1.348663in}}%
\pgfpathcurveto{\pgfqpoint{2.909058in}{1.337613in}}{\pgfqpoint{2.913448in}{1.327014in}}{\pgfqpoint{2.921262in}{1.319200in}}%
\pgfpathcurveto{\pgfqpoint{2.929076in}{1.311387in}}{\pgfqpoint{2.939675in}{1.306996in}}{\pgfqpoint{2.950725in}{1.306996in}}%
\pgfpathclose%
\pgfusepath{stroke,fill}%
\end{pgfscope}%
\begin{pgfscope}%
\pgfpathrectangle{\pgfqpoint{0.481978in}{0.331635in}}{\pgfqpoint{4.960000in}{3.696000in}}%
\pgfusepath{clip}%
\pgfsetbuttcap%
\pgfsetroundjoin%
\definecolor{currentfill}{rgb}{1.000000,0.705882,0.509804}%
\pgfsetfillcolor{currentfill}%
\pgfsetlinewidth{0.481800pt}%
\definecolor{currentstroke}{rgb}{1.000000,1.000000,1.000000}%
\pgfsetstrokecolor{currentstroke}%
\pgfsetdash{}{0pt}%
\pgfpathmoveto{\pgfqpoint{0.903986in}{1.056710in}}%
\pgfpathcurveto{\pgfqpoint{0.915036in}{1.056710in}}{\pgfqpoint{0.925635in}{1.061100in}}{\pgfqpoint{0.933449in}{1.068913in}}%
\pgfpathcurveto{\pgfqpoint{0.941263in}{1.076727in}}{\pgfqpoint{0.945653in}{1.087326in}}{\pgfqpoint{0.945653in}{1.098376in}}%
\pgfpathcurveto{\pgfqpoint{0.945653in}{1.109426in}}{\pgfqpoint{0.941263in}{1.120025in}}{\pgfqpoint{0.933449in}{1.127839in}}%
\pgfpathcurveto{\pgfqpoint{0.925635in}{1.135653in}}{\pgfqpoint{0.915036in}{1.140043in}}{\pgfqpoint{0.903986in}{1.140043in}}%
\pgfpathcurveto{\pgfqpoint{0.892936in}{1.140043in}}{\pgfqpoint{0.882337in}{1.135653in}}{\pgfqpoint{0.874523in}{1.127839in}}%
\pgfpathcurveto{\pgfqpoint{0.866710in}{1.120025in}}{\pgfqpoint{0.862320in}{1.109426in}}{\pgfqpoint{0.862320in}{1.098376in}}%
\pgfpathcurveto{\pgfqpoint{0.862320in}{1.087326in}}{\pgfqpoint{0.866710in}{1.076727in}}{\pgfqpoint{0.874523in}{1.068913in}}%
\pgfpathcurveto{\pgfqpoint{0.882337in}{1.061100in}}{\pgfqpoint{0.892936in}{1.056710in}}{\pgfqpoint{0.903986in}{1.056710in}}%
\pgfpathclose%
\pgfusepath{stroke,fill}%
\end{pgfscope}%
\begin{pgfscope}%
\pgfpathrectangle{\pgfqpoint{0.481978in}{0.331635in}}{\pgfqpoint{4.960000in}{3.696000in}}%
\pgfusepath{clip}%
\pgfsetbuttcap%
\pgfsetroundjoin%
\definecolor{currentfill}{rgb}{1.000000,0.705882,0.509804}%
\pgfsetfillcolor{currentfill}%
\pgfsetlinewidth{0.481800pt}%
\definecolor{currentstroke}{rgb}{1.000000,1.000000,1.000000}%
\pgfsetstrokecolor{currentstroke}%
\pgfsetdash{}{0pt}%
\pgfpathmoveto{\pgfqpoint{1.301328in}{0.972923in}}%
\pgfpathcurveto{\pgfqpoint{1.312378in}{0.972923in}}{\pgfqpoint{1.322977in}{0.977313in}}{\pgfqpoint{1.330791in}{0.985126in}}%
\pgfpathcurveto{\pgfqpoint{1.338605in}{0.992940in}}{\pgfqpoint{1.342995in}{1.003539in}}{\pgfqpoint{1.342995in}{1.014589in}}%
\pgfpathcurveto{\pgfqpoint{1.342995in}{1.025639in}}{\pgfqpoint{1.338605in}{1.036238in}}{\pgfqpoint{1.330791in}{1.044052in}}%
\pgfpathcurveto{\pgfqpoint{1.322977in}{1.051866in}}{\pgfqpoint{1.312378in}{1.056256in}}{\pgfqpoint{1.301328in}{1.056256in}}%
\pgfpathcurveto{\pgfqpoint{1.290278in}{1.056256in}}{\pgfqpoint{1.279679in}{1.051866in}}{\pgfqpoint{1.271865in}{1.044052in}}%
\pgfpathcurveto{\pgfqpoint{1.264052in}{1.036238in}}{\pgfqpoint{1.259662in}{1.025639in}}{\pgfqpoint{1.259662in}{1.014589in}}%
\pgfpathcurveto{\pgfqpoint{1.259662in}{1.003539in}}{\pgfqpoint{1.264052in}{0.992940in}}{\pgfqpoint{1.271865in}{0.985126in}}%
\pgfpathcurveto{\pgfqpoint{1.279679in}{0.977313in}}{\pgfqpoint{1.290278in}{0.972923in}}{\pgfqpoint{1.301328in}{0.972923in}}%
\pgfpathclose%
\pgfusepath{stroke,fill}%
\end{pgfscope}%
\begin{pgfscope}%
\pgfpathrectangle{\pgfqpoint{0.481978in}{0.331635in}}{\pgfqpoint{4.960000in}{3.696000in}}%
\pgfusepath{clip}%
\pgfsetbuttcap%
\pgfsetroundjoin%
\definecolor{currentfill}{rgb}{1.000000,0.705882,0.509804}%
\pgfsetfillcolor{currentfill}%
\pgfsetlinewidth{0.481800pt}%
\definecolor{currentstroke}{rgb}{1.000000,1.000000,1.000000}%
\pgfsetstrokecolor{currentstroke}%
\pgfsetdash{}{0pt}%
\pgfpathmoveto{\pgfqpoint{0.928531in}{1.058501in}}%
\pgfpathcurveto{\pgfqpoint{0.939581in}{1.058501in}}{\pgfqpoint{0.950180in}{1.062892in}}{\pgfqpoint{0.957994in}{1.070705in}}%
\pgfpathcurveto{\pgfqpoint{0.965808in}{1.078519in}}{\pgfqpoint{0.970198in}{1.089118in}}{\pgfqpoint{0.970198in}{1.100168in}}%
\pgfpathcurveto{\pgfqpoint{0.970198in}{1.111218in}}{\pgfqpoint{0.965808in}{1.121817in}}{\pgfqpoint{0.957994in}{1.129631in}}%
\pgfpathcurveto{\pgfqpoint{0.950180in}{1.137445in}}{\pgfqpoint{0.939581in}{1.141835in}}{\pgfqpoint{0.928531in}{1.141835in}}%
\pgfpathcurveto{\pgfqpoint{0.917481in}{1.141835in}}{\pgfqpoint{0.906882in}{1.137445in}}{\pgfqpoint{0.899068in}{1.129631in}}%
\pgfpathcurveto{\pgfqpoint{0.891255in}{1.121817in}}{\pgfqpoint{0.886865in}{1.111218in}}{\pgfqpoint{0.886865in}{1.100168in}}%
\pgfpathcurveto{\pgfqpoint{0.886865in}{1.089118in}}{\pgfqpoint{0.891255in}{1.078519in}}{\pgfqpoint{0.899068in}{1.070705in}}%
\pgfpathcurveto{\pgfqpoint{0.906882in}{1.062892in}}{\pgfqpoint{0.917481in}{1.058501in}}{\pgfqpoint{0.928531in}{1.058501in}}%
\pgfpathclose%
\pgfusepath{stroke,fill}%
\end{pgfscope}%
\begin{pgfscope}%
\pgfpathrectangle{\pgfqpoint{0.481978in}{0.331635in}}{\pgfqpoint{4.960000in}{3.696000in}}%
\pgfusepath{clip}%
\pgfsetbuttcap%
\pgfsetroundjoin%
\definecolor{currentfill}{rgb}{1.000000,0.705882,0.509804}%
\pgfsetfillcolor{currentfill}%
\pgfsetlinewidth{0.481800pt}%
\definecolor{currentstroke}{rgb}{1.000000,1.000000,1.000000}%
\pgfsetstrokecolor{currentstroke}%
\pgfsetdash{}{0pt}%
\pgfpathmoveto{\pgfqpoint{3.055823in}{2.266295in}}%
\pgfpathcurveto{\pgfqpoint{3.066873in}{2.266295in}}{\pgfqpoint{3.077472in}{2.270685in}}{\pgfqpoint{3.085285in}{2.278499in}}%
\pgfpathcurveto{\pgfqpoint{3.093099in}{2.286312in}}{\pgfqpoint{3.097489in}{2.296911in}}{\pgfqpoint{3.097489in}{2.307962in}}%
\pgfpathcurveto{\pgfqpoint{3.097489in}{2.319012in}}{\pgfqpoint{3.093099in}{2.329611in}}{\pgfqpoint{3.085285in}{2.337424in}}%
\pgfpathcurveto{\pgfqpoint{3.077472in}{2.345238in}}{\pgfqpoint{3.066873in}{2.349628in}}{\pgfqpoint{3.055823in}{2.349628in}}%
\pgfpathcurveto{\pgfqpoint{3.044773in}{2.349628in}}{\pgfqpoint{3.034174in}{2.345238in}}{\pgfqpoint{3.026360in}{2.337424in}}%
\pgfpathcurveto{\pgfqpoint{3.018546in}{2.329611in}}{\pgfqpoint{3.014156in}{2.319012in}}{\pgfqpoint{3.014156in}{2.307962in}}%
\pgfpathcurveto{\pgfqpoint{3.014156in}{2.296911in}}{\pgfqpoint{3.018546in}{2.286312in}}{\pgfqpoint{3.026360in}{2.278499in}}%
\pgfpathcurveto{\pgfqpoint{3.034174in}{2.270685in}}{\pgfqpoint{3.044773in}{2.266295in}}{\pgfqpoint{3.055823in}{2.266295in}}%
\pgfpathclose%
\pgfusepath{stroke,fill}%
\end{pgfscope}%
\begin{pgfscope}%
\pgfpathrectangle{\pgfqpoint{0.481978in}{0.331635in}}{\pgfqpoint{4.960000in}{3.696000in}}%
\pgfusepath{clip}%
\pgfsetbuttcap%
\pgfsetroundjoin%
\definecolor{currentfill}{rgb}{1.000000,0.705882,0.509804}%
\pgfsetfillcolor{currentfill}%
\pgfsetlinewidth{0.481800pt}%
\definecolor{currentstroke}{rgb}{1.000000,1.000000,1.000000}%
\pgfsetstrokecolor{currentstroke}%
\pgfsetdash{}{0pt}%
\pgfpathmoveto{\pgfqpoint{2.430224in}{1.649443in}}%
\pgfpathcurveto{\pgfqpoint{2.441274in}{1.649443in}}{\pgfqpoint{2.451873in}{1.653834in}}{\pgfqpoint{2.459687in}{1.661647in}}%
\pgfpathcurveto{\pgfqpoint{2.467500in}{1.669461in}}{\pgfqpoint{2.471891in}{1.680060in}}{\pgfqpoint{2.471891in}{1.691110in}}%
\pgfpathcurveto{\pgfqpoint{2.471891in}{1.702160in}}{\pgfqpoint{2.467500in}{1.712759in}}{\pgfqpoint{2.459687in}{1.720573in}}%
\pgfpathcurveto{\pgfqpoint{2.451873in}{1.728386in}}{\pgfqpoint{2.441274in}{1.732777in}}{\pgfqpoint{2.430224in}{1.732777in}}%
\pgfpathcurveto{\pgfqpoint{2.419174in}{1.732777in}}{\pgfqpoint{2.408575in}{1.728386in}}{\pgfqpoint{2.400761in}{1.720573in}}%
\pgfpathcurveto{\pgfqpoint{2.392948in}{1.712759in}}{\pgfqpoint{2.388557in}{1.702160in}}{\pgfqpoint{2.388557in}{1.691110in}}%
\pgfpathcurveto{\pgfqpoint{2.388557in}{1.680060in}}{\pgfqpoint{2.392948in}{1.669461in}}{\pgfqpoint{2.400761in}{1.661647in}}%
\pgfpathcurveto{\pgfqpoint{2.408575in}{1.653834in}}{\pgfqpoint{2.419174in}{1.649443in}}{\pgfqpoint{2.430224in}{1.649443in}}%
\pgfpathclose%
\pgfusepath{stroke,fill}%
\end{pgfscope}%
\begin{pgfscope}%
\pgfpathrectangle{\pgfqpoint{0.481978in}{0.331635in}}{\pgfqpoint{4.960000in}{3.696000in}}%
\pgfusepath{clip}%
\pgfsetbuttcap%
\pgfsetroundjoin%
\definecolor{currentfill}{rgb}{1.000000,0.705882,0.509804}%
\pgfsetfillcolor{currentfill}%
\pgfsetlinewidth{0.481800pt}%
\definecolor{currentstroke}{rgb}{1.000000,1.000000,1.000000}%
\pgfsetstrokecolor{currentstroke}%
\pgfsetdash{}{0pt}%
\pgfpathmoveto{\pgfqpoint{1.452210in}{1.086686in}}%
\pgfpathcurveto{\pgfqpoint{1.463260in}{1.086686in}}{\pgfqpoint{1.473859in}{1.091076in}}{\pgfqpoint{1.481673in}{1.098890in}}%
\pgfpathcurveto{\pgfqpoint{1.489486in}{1.106703in}}{\pgfqpoint{1.493877in}{1.117302in}}{\pgfqpoint{1.493877in}{1.128352in}}%
\pgfpathcurveto{\pgfqpoint{1.493877in}{1.139403in}}{\pgfqpoint{1.489486in}{1.150002in}}{\pgfqpoint{1.481673in}{1.157815in}}%
\pgfpathcurveto{\pgfqpoint{1.473859in}{1.165629in}}{\pgfqpoint{1.463260in}{1.170019in}}{\pgfqpoint{1.452210in}{1.170019in}}%
\pgfpathcurveto{\pgfqpoint{1.441160in}{1.170019in}}{\pgfqpoint{1.430561in}{1.165629in}}{\pgfqpoint{1.422747in}{1.157815in}}%
\pgfpathcurveto{\pgfqpoint{1.414934in}{1.150002in}}{\pgfqpoint{1.410543in}{1.139403in}}{\pgfqpoint{1.410543in}{1.128352in}}%
\pgfpathcurveto{\pgfqpoint{1.410543in}{1.117302in}}{\pgfqpoint{1.414934in}{1.106703in}}{\pgfqpoint{1.422747in}{1.098890in}}%
\pgfpathcurveto{\pgfqpoint{1.430561in}{1.091076in}}{\pgfqpoint{1.441160in}{1.086686in}}{\pgfqpoint{1.452210in}{1.086686in}}%
\pgfpathclose%
\pgfusepath{stroke,fill}%
\end{pgfscope}%
\begin{pgfscope}%
\pgfpathrectangle{\pgfqpoint{0.481978in}{0.331635in}}{\pgfqpoint{4.960000in}{3.696000in}}%
\pgfusepath{clip}%
\pgfsetbuttcap%
\pgfsetroundjoin%
\definecolor{currentfill}{rgb}{1.000000,0.705882,0.509804}%
\pgfsetfillcolor{currentfill}%
\pgfsetlinewidth{0.481800pt}%
\definecolor{currentstroke}{rgb}{1.000000,1.000000,1.000000}%
\pgfsetstrokecolor{currentstroke}%
\pgfsetdash{}{0pt}%
\pgfpathmoveto{\pgfqpoint{2.433780in}{0.917235in}}%
\pgfpathcurveto{\pgfqpoint{2.444830in}{0.917235in}}{\pgfqpoint{2.455429in}{0.921625in}}{\pgfqpoint{2.463242in}{0.929439in}}%
\pgfpathcurveto{\pgfqpoint{2.471056in}{0.937252in}}{\pgfqpoint{2.475446in}{0.947851in}}{\pgfqpoint{2.475446in}{0.958901in}}%
\pgfpathcurveto{\pgfqpoint{2.475446in}{0.969952in}}{\pgfqpoint{2.471056in}{0.980551in}}{\pgfqpoint{2.463242in}{0.988364in}}%
\pgfpathcurveto{\pgfqpoint{2.455429in}{0.996178in}}{\pgfqpoint{2.444830in}{1.000568in}}{\pgfqpoint{2.433780in}{1.000568in}}%
\pgfpathcurveto{\pgfqpoint{2.422730in}{1.000568in}}{\pgfqpoint{2.412130in}{0.996178in}}{\pgfqpoint{2.404317in}{0.988364in}}%
\pgfpathcurveto{\pgfqpoint{2.396503in}{0.980551in}}{\pgfqpoint{2.392113in}{0.969952in}}{\pgfqpoint{2.392113in}{0.958901in}}%
\pgfpathcurveto{\pgfqpoint{2.392113in}{0.947851in}}{\pgfqpoint{2.396503in}{0.937252in}}{\pgfqpoint{2.404317in}{0.929439in}}%
\pgfpathcurveto{\pgfqpoint{2.412130in}{0.921625in}}{\pgfqpoint{2.422730in}{0.917235in}}{\pgfqpoint{2.433780in}{0.917235in}}%
\pgfpathclose%
\pgfusepath{stroke,fill}%
\end{pgfscope}%
\begin{pgfscope}%
\pgfpathrectangle{\pgfqpoint{0.481978in}{0.331635in}}{\pgfqpoint{4.960000in}{3.696000in}}%
\pgfusepath{clip}%
\pgfsetbuttcap%
\pgfsetroundjoin%
\definecolor{currentfill}{rgb}{1.000000,0.705882,0.509804}%
\pgfsetfillcolor{currentfill}%
\pgfsetlinewidth{0.481800pt}%
\definecolor{currentstroke}{rgb}{1.000000,1.000000,1.000000}%
\pgfsetstrokecolor{currentstroke}%
\pgfsetdash{}{0pt}%
\pgfpathmoveto{\pgfqpoint{3.576377in}{1.343946in}}%
\pgfpathcurveto{\pgfqpoint{3.587427in}{1.343946in}}{\pgfqpoint{3.598026in}{1.348336in}}{\pgfqpoint{3.605840in}{1.356149in}}%
\pgfpathcurveto{\pgfqpoint{3.613653in}{1.363963in}}{\pgfqpoint{3.618044in}{1.374562in}}{\pgfqpoint{3.618044in}{1.385612in}}%
\pgfpathcurveto{\pgfqpoint{3.618044in}{1.396662in}}{\pgfqpoint{3.613653in}{1.407261in}}{\pgfqpoint{3.605840in}{1.415075in}}%
\pgfpathcurveto{\pgfqpoint{3.598026in}{1.422889in}}{\pgfqpoint{3.587427in}{1.427279in}}{\pgfqpoint{3.576377in}{1.427279in}}%
\pgfpathcurveto{\pgfqpoint{3.565327in}{1.427279in}}{\pgfqpoint{3.554728in}{1.422889in}}{\pgfqpoint{3.546914in}{1.415075in}}%
\pgfpathcurveto{\pgfqpoint{3.539100in}{1.407261in}}{\pgfqpoint{3.534710in}{1.396662in}}{\pgfqpoint{3.534710in}{1.385612in}}%
\pgfpathcurveto{\pgfqpoint{3.534710in}{1.374562in}}{\pgfqpoint{3.539100in}{1.363963in}}{\pgfqpoint{3.546914in}{1.356149in}}%
\pgfpathcurveto{\pgfqpoint{3.554728in}{1.348336in}}{\pgfqpoint{3.565327in}{1.343946in}}{\pgfqpoint{3.576377in}{1.343946in}}%
\pgfpathclose%
\pgfusepath{stroke,fill}%
\end{pgfscope}%
\begin{pgfscope}%
\pgfpathrectangle{\pgfqpoint{0.481978in}{0.331635in}}{\pgfqpoint{4.960000in}{3.696000in}}%
\pgfusepath{clip}%
\pgfsetbuttcap%
\pgfsetroundjoin%
\definecolor{currentfill}{rgb}{1.000000,0.705882,0.509804}%
\pgfsetfillcolor{currentfill}%
\pgfsetlinewidth{0.481800pt}%
\definecolor{currentstroke}{rgb}{1.000000,1.000000,1.000000}%
\pgfsetstrokecolor{currentstroke}%
\pgfsetdash{}{0pt}%
\pgfpathmoveto{\pgfqpoint{3.896697in}{2.155982in}}%
\pgfpathcurveto{\pgfqpoint{3.907747in}{2.155982in}}{\pgfqpoint{3.918346in}{2.160372in}}{\pgfqpoint{3.926160in}{2.168186in}}%
\pgfpathcurveto{\pgfqpoint{3.933973in}{2.175999in}}{\pgfqpoint{3.938364in}{2.186598in}}{\pgfqpoint{3.938364in}{2.197648in}}%
\pgfpathcurveto{\pgfqpoint{3.938364in}{2.208698in}}{\pgfqpoint{3.933973in}{2.219297in}}{\pgfqpoint{3.926160in}{2.227111in}}%
\pgfpathcurveto{\pgfqpoint{3.918346in}{2.234925in}}{\pgfqpoint{3.907747in}{2.239315in}}{\pgfqpoint{3.896697in}{2.239315in}}%
\pgfpathcurveto{\pgfqpoint{3.885647in}{2.239315in}}{\pgfqpoint{3.875048in}{2.234925in}}{\pgfqpoint{3.867234in}{2.227111in}}%
\pgfpathcurveto{\pgfqpoint{3.859420in}{2.219297in}}{\pgfqpoint{3.855030in}{2.208698in}}{\pgfqpoint{3.855030in}{2.197648in}}%
\pgfpathcurveto{\pgfqpoint{3.855030in}{2.186598in}}{\pgfqpoint{3.859420in}{2.175999in}}{\pgfqpoint{3.867234in}{2.168186in}}%
\pgfpathcurveto{\pgfqpoint{3.875048in}{2.160372in}}{\pgfqpoint{3.885647in}{2.155982in}}{\pgfqpoint{3.896697in}{2.155982in}}%
\pgfpathclose%
\pgfusepath{stroke,fill}%
\end{pgfscope}%
\begin{pgfscope}%
\pgfpathrectangle{\pgfqpoint{0.481978in}{0.331635in}}{\pgfqpoint{4.960000in}{3.696000in}}%
\pgfusepath{clip}%
\pgfsetbuttcap%
\pgfsetroundjoin%
\definecolor{currentfill}{rgb}{1.000000,0.705882,0.509804}%
\pgfsetfillcolor{currentfill}%
\pgfsetlinewidth{0.481800pt}%
\definecolor{currentstroke}{rgb}{1.000000,1.000000,1.000000}%
\pgfsetstrokecolor{currentstroke}%
\pgfsetdash{}{0pt}%
\pgfpathmoveto{\pgfqpoint{1.305107in}{1.512415in}}%
\pgfpathcurveto{\pgfqpoint{1.316157in}{1.512415in}}{\pgfqpoint{1.326756in}{1.516805in}}{\pgfqpoint{1.334570in}{1.524619in}}%
\pgfpathcurveto{\pgfqpoint{1.342384in}{1.532432in}}{\pgfqpoint{1.346774in}{1.543031in}}{\pgfqpoint{1.346774in}{1.554081in}}%
\pgfpathcurveto{\pgfqpoint{1.346774in}{1.565131in}}{\pgfqpoint{1.342384in}{1.575730in}}{\pgfqpoint{1.334570in}{1.583544in}}%
\pgfpathcurveto{\pgfqpoint{1.326756in}{1.591358in}}{\pgfqpoint{1.316157in}{1.595748in}}{\pgfqpoint{1.305107in}{1.595748in}}%
\pgfpathcurveto{\pgfqpoint{1.294057in}{1.595748in}}{\pgfqpoint{1.283458in}{1.591358in}}{\pgfqpoint{1.275644in}{1.583544in}}%
\pgfpathcurveto{\pgfqpoint{1.267831in}{1.575730in}}{\pgfqpoint{1.263441in}{1.565131in}}{\pgfqpoint{1.263441in}{1.554081in}}%
\pgfpathcurveto{\pgfqpoint{1.263441in}{1.543031in}}{\pgfqpoint{1.267831in}{1.532432in}}{\pgfqpoint{1.275644in}{1.524619in}}%
\pgfpathcurveto{\pgfqpoint{1.283458in}{1.516805in}}{\pgfqpoint{1.294057in}{1.512415in}}{\pgfqpoint{1.305107in}{1.512415in}}%
\pgfpathclose%
\pgfusepath{stroke,fill}%
\end{pgfscope}%
\begin{pgfscope}%
\pgfpathrectangle{\pgfqpoint{0.481978in}{0.331635in}}{\pgfqpoint{4.960000in}{3.696000in}}%
\pgfusepath{clip}%
\pgfsetbuttcap%
\pgfsetroundjoin%
\definecolor{currentfill}{rgb}{1.000000,0.705882,0.509804}%
\pgfsetfillcolor{currentfill}%
\pgfsetlinewidth{0.481800pt}%
\definecolor{currentstroke}{rgb}{1.000000,1.000000,1.000000}%
\pgfsetstrokecolor{currentstroke}%
\pgfsetdash{}{0pt}%
\pgfpathmoveto{\pgfqpoint{3.232967in}{1.012132in}}%
\pgfpathcurveto{\pgfqpoint{3.244017in}{1.012132in}}{\pgfqpoint{3.254616in}{1.016522in}}{\pgfqpoint{3.262430in}{1.024336in}}%
\pgfpathcurveto{\pgfqpoint{3.270244in}{1.032150in}}{\pgfqpoint{3.274634in}{1.042749in}}{\pgfqpoint{3.274634in}{1.053799in}}%
\pgfpathcurveto{\pgfqpoint{3.274634in}{1.064849in}}{\pgfqpoint{3.270244in}{1.075448in}}{\pgfqpoint{3.262430in}{1.083261in}}%
\pgfpathcurveto{\pgfqpoint{3.254616in}{1.091075in}}{\pgfqpoint{3.244017in}{1.095465in}}{\pgfqpoint{3.232967in}{1.095465in}}%
\pgfpathcurveto{\pgfqpoint{3.221917in}{1.095465in}}{\pgfqpoint{3.211318in}{1.091075in}}{\pgfqpoint{3.203504in}{1.083261in}}%
\pgfpathcurveto{\pgfqpoint{3.195691in}{1.075448in}}{\pgfqpoint{3.191301in}{1.064849in}}{\pgfqpoint{3.191301in}{1.053799in}}%
\pgfpathcurveto{\pgfqpoint{3.191301in}{1.042749in}}{\pgfqpoint{3.195691in}{1.032150in}}{\pgfqpoint{3.203504in}{1.024336in}}%
\pgfpathcurveto{\pgfqpoint{3.211318in}{1.016522in}}{\pgfqpoint{3.221917in}{1.012132in}}{\pgfqpoint{3.232967in}{1.012132in}}%
\pgfpathclose%
\pgfusepath{stroke,fill}%
\end{pgfscope}%
\begin{pgfscope}%
\pgfpathrectangle{\pgfqpoint{0.481978in}{0.331635in}}{\pgfqpoint{4.960000in}{3.696000in}}%
\pgfusepath{clip}%
\pgfsetbuttcap%
\pgfsetroundjoin%
\definecolor{currentfill}{rgb}{1.000000,0.705882,0.509804}%
\pgfsetfillcolor{currentfill}%
\pgfsetlinewidth{0.481800pt}%
\definecolor{currentstroke}{rgb}{1.000000,1.000000,1.000000}%
\pgfsetstrokecolor{currentstroke}%
\pgfsetdash{}{0pt}%
\pgfpathmoveto{\pgfqpoint{1.365241in}{1.852608in}}%
\pgfpathcurveto{\pgfqpoint{1.376291in}{1.852608in}}{\pgfqpoint{1.386890in}{1.856998in}}{\pgfqpoint{1.394704in}{1.864812in}}%
\pgfpathcurveto{\pgfqpoint{1.402518in}{1.872626in}}{\pgfqpoint{1.406908in}{1.883225in}}{\pgfqpoint{1.406908in}{1.894275in}}%
\pgfpathcurveto{\pgfqpoint{1.406908in}{1.905325in}}{\pgfqpoint{1.402518in}{1.915924in}}{\pgfqpoint{1.394704in}{1.923737in}}%
\pgfpathcurveto{\pgfqpoint{1.386890in}{1.931551in}}{\pgfqpoint{1.376291in}{1.935941in}}{\pgfqpoint{1.365241in}{1.935941in}}%
\pgfpathcurveto{\pgfqpoint{1.354191in}{1.935941in}}{\pgfqpoint{1.343592in}{1.931551in}}{\pgfqpoint{1.335778in}{1.923737in}}%
\pgfpathcurveto{\pgfqpoint{1.327965in}{1.915924in}}{\pgfqpoint{1.323575in}{1.905325in}}{\pgfqpoint{1.323575in}{1.894275in}}%
\pgfpathcurveto{\pgfqpoint{1.323575in}{1.883225in}}{\pgfqpoint{1.327965in}{1.872626in}}{\pgfqpoint{1.335778in}{1.864812in}}%
\pgfpathcurveto{\pgfqpoint{1.343592in}{1.856998in}}{\pgfqpoint{1.354191in}{1.852608in}}{\pgfqpoint{1.365241in}{1.852608in}}%
\pgfpathclose%
\pgfusepath{stroke,fill}%
\end{pgfscope}%
\begin{pgfscope}%
\pgfpathrectangle{\pgfqpoint{0.481978in}{0.331635in}}{\pgfqpoint{4.960000in}{3.696000in}}%
\pgfusepath{clip}%
\pgfsetbuttcap%
\pgfsetroundjoin%
\definecolor{currentfill}{rgb}{1.000000,0.705882,0.509804}%
\pgfsetfillcolor{currentfill}%
\pgfsetlinewidth{0.481800pt}%
\definecolor{currentstroke}{rgb}{1.000000,1.000000,1.000000}%
\pgfsetstrokecolor{currentstroke}%
\pgfsetdash{}{0pt}%
\pgfpathmoveto{\pgfqpoint{5.159612in}{2.495590in}}%
\pgfpathcurveto{\pgfqpoint{5.170663in}{2.495590in}}{\pgfqpoint{5.181262in}{2.499980in}}{\pgfqpoint{5.189075in}{2.507794in}}%
\pgfpathcurveto{\pgfqpoint{5.196889in}{2.515607in}}{\pgfqpoint{5.201279in}{2.526206in}}{\pgfqpoint{5.201279in}{2.537256in}}%
\pgfpathcurveto{\pgfqpoint{5.201279in}{2.548307in}}{\pgfqpoint{5.196889in}{2.558906in}}{\pgfqpoint{5.189075in}{2.566719in}}%
\pgfpathcurveto{\pgfqpoint{5.181262in}{2.574533in}}{\pgfqpoint{5.170663in}{2.578923in}}{\pgfqpoint{5.159612in}{2.578923in}}%
\pgfpathcurveto{\pgfqpoint{5.148562in}{2.578923in}}{\pgfqpoint{5.137963in}{2.574533in}}{\pgfqpoint{5.130150in}{2.566719in}}%
\pgfpathcurveto{\pgfqpoint{5.122336in}{2.558906in}}{\pgfqpoint{5.117946in}{2.548307in}}{\pgfqpoint{5.117946in}{2.537256in}}%
\pgfpathcurveto{\pgfqpoint{5.117946in}{2.526206in}}{\pgfqpoint{5.122336in}{2.515607in}}{\pgfqpoint{5.130150in}{2.507794in}}%
\pgfpathcurveto{\pgfqpoint{5.137963in}{2.499980in}}{\pgfqpoint{5.148562in}{2.495590in}}{\pgfqpoint{5.159612in}{2.495590in}}%
\pgfpathclose%
\pgfusepath{stroke,fill}%
\end{pgfscope}%
\begin{pgfscope}%
\pgfpathrectangle{\pgfqpoint{0.481978in}{0.331635in}}{\pgfqpoint{4.960000in}{3.696000in}}%
\pgfusepath{clip}%
\pgfsetbuttcap%
\pgfsetroundjoin%
\definecolor{currentfill}{rgb}{1.000000,0.705882,0.509804}%
\pgfsetfillcolor{currentfill}%
\pgfsetlinewidth{0.481800pt}%
\definecolor{currentstroke}{rgb}{1.000000,1.000000,1.000000}%
\pgfsetstrokecolor{currentstroke}%
\pgfsetdash{}{0pt}%
\pgfpathmoveto{\pgfqpoint{3.523183in}{0.958833in}}%
\pgfpathcurveto{\pgfqpoint{3.534233in}{0.958833in}}{\pgfqpoint{3.544832in}{0.963223in}}{\pgfqpoint{3.552646in}{0.971037in}}%
\pgfpathcurveto{\pgfqpoint{3.560459in}{0.978850in}}{\pgfqpoint{3.564850in}{0.989449in}}{\pgfqpoint{3.564850in}{1.000500in}}%
\pgfpathcurveto{\pgfqpoint{3.564850in}{1.011550in}}{\pgfqpoint{3.560459in}{1.022149in}}{\pgfqpoint{3.552646in}{1.029962in}}%
\pgfpathcurveto{\pgfqpoint{3.544832in}{1.037776in}}{\pgfqpoint{3.534233in}{1.042166in}}{\pgfqpoint{3.523183in}{1.042166in}}%
\pgfpathcurveto{\pgfqpoint{3.512133in}{1.042166in}}{\pgfqpoint{3.501534in}{1.037776in}}{\pgfqpoint{3.493720in}{1.029962in}}%
\pgfpathcurveto{\pgfqpoint{3.485906in}{1.022149in}}{\pgfqpoint{3.481516in}{1.011550in}}{\pgfqpoint{3.481516in}{1.000500in}}%
\pgfpathcurveto{\pgfqpoint{3.481516in}{0.989449in}}{\pgfqpoint{3.485906in}{0.978850in}}{\pgfqpoint{3.493720in}{0.971037in}}%
\pgfpathcurveto{\pgfqpoint{3.501534in}{0.963223in}}{\pgfqpoint{3.512133in}{0.958833in}}{\pgfqpoint{3.523183in}{0.958833in}}%
\pgfpathclose%
\pgfusepath{stroke,fill}%
\end{pgfscope}%
\begin{pgfscope}%
\pgfpathrectangle{\pgfqpoint{0.481978in}{0.331635in}}{\pgfqpoint{4.960000in}{3.696000in}}%
\pgfusepath{clip}%
\pgfsetbuttcap%
\pgfsetroundjoin%
\definecolor{currentfill}{rgb}{1.000000,0.705882,0.509804}%
\pgfsetfillcolor{currentfill}%
\pgfsetlinewidth{0.481800pt}%
\definecolor{currentstroke}{rgb}{1.000000,1.000000,1.000000}%
\pgfsetstrokecolor{currentstroke}%
\pgfsetdash{}{0pt}%
\pgfpathmoveto{\pgfqpoint{3.238366in}{1.367125in}}%
\pgfpathcurveto{\pgfqpoint{3.249416in}{1.367125in}}{\pgfqpoint{3.260015in}{1.371516in}}{\pgfqpoint{3.267829in}{1.379329in}}%
\pgfpathcurveto{\pgfqpoint{3.275642in}{1.387143in}}{\pgfqpoint{3.280033in}{1.397742in}}{\pgfqpoint{3.280033in}{1.408792in}}%
\pgfpathcurveto{\pgfqpoint{3.280033in}{1.419842in}}{\pgfqpoint{3.275642in}{1.430441in}}{\pgfqpoint{3.267829in}{1.438255in}}%
\pgfpathcurveto{\pgfqpoint{3.260015in}{1.446069in}}{\pgfqpoint{3.249416in}{1.450459in}}{\pgfqpoint{3.238366in}{1.450459in}}%
\pgfpathcurveto{\pgfqpoint{3.227316in}{1.450459in}}{\pgfqpoint{3.216717in}{1.446069in}}{\pgfqpoint{3.208903in}{1.438255in}}%
\pgfpathcurveto{\pgfqpoint{3.201090in}{1.430441in}}{\pgfqpoint{3.196699in}{1.419842in}}{\pgfqpoint{3.196699in}{1.408792in}}%
\pgfpathcurveto{\pgfqpoint{3.196699in}{1.397742in}}{\pgfqpoint{3.201090in}{1.387143in}}{\pgfqpoint{3.208903in}{1.379329in}}%
\pgfpathcurveto{\pgfqpoint{3.216717in}{1.371516in}}{\pgfqpoint{3.227316in}{1.367125in}}{\pgfqpoint{3.238366in}{1.367125in}}%
\pgfpathclose%
\pgfusepath{stroke,fill}%
\end{pgfscope}%
\begin{pgfscope}%
\pgfpathrectangle{\pgfqpoint{0.481978in}{0.331635in}}{\pgfqpoint{4.960000in}{3.696000in}}%
\pgfusepath{clip}%
\pgfsetbuttcap%
\pgfsetroundjoin%
\definecolor{currentfill}{rgb}{1.000000,0.705882,0.509804}%
\pgfsetfillcolor{currentfill}%
\pgfsetlinewidth{0.481800pt}%
\definecolor{currentstroke}{rgb}{1.000000,1.000000,1.000000}%
\pgfsetstrokecolor{currentstroke}%
\pgfsetdash{}{0pt}%
\pgfpathmoveto{\pgfqpoint{1.615727in}{0.741931in}}%
\pgfpathcurveto{\pgfqpoint{1.626777in}{0.741931in}}{\pgfqpoint{1.637376in}{0.746321in}}{\pgfqpoint{1.645190in}{0.754135in}}%
\pgfpathcurveto{\pgfqpoint{1.653003in}{0.761948in}}{\pgfqpoint{1.657394in}{0.772547in}}{\pgfqpoint{1.657394in}{0.783597in}}%
\pgfpathcurveto{\pgfqpoint{1.657394in}{0.794648in}}{\pgfqpoint{1.653003in}{0.805247in}}{\pgfqpoint{1.645190in}{0.813060in}}%
\pgfpathcurveto{\pgfqpoint{1.637376in}{0.820874in}}{\pgfqpoint{1.626777in}{0.825264in}}{\pgfqpoint{1.615727in}{0.825264in}}%
\pgfpathcurveto{\pgfqpoint{1.604677in}{0.825264in}}{\pgfqpoint{1.594078in}{0.820874in}}{\pgfqpoint{1.586264in}{0.813060in}}%
\pgfpathcurveto{\pgfqpoint{1.578451in}{0.805247in}}{\pgfqpoint{1.574060in}{0.794648in}}{\pgfqpoint{1.574060in}{0.783597in}}%
\pgfpathcurveto{\pgfqpoint{1.574060in}{0.772547in}}{\pgfqpoint{1.578451in}{0.761948in}}{\pgfqpoint{1.586264in}{0.754135in}}%
\pgfpathcurveto{\pgfqpoint{1.594078in}{0.746321in}}{\pgfqpoint{1.604677in}{0.741931in}}{\pgfqpoint{1.615727in}{0.741931in}}%
\pgfpathclose%
\pgfusepath{stroke,fill}%
\end{pgfscope}%
\begin{pgfscope}%
\pgfpathrectangle{\pgfqpoint{0.481978in}{0.331635in}}{\pgfqpoint{4.960000in}{3.696000in}}%
\pgfusepath{clip}%
\pgfsetbuttcap%
\pgfsetroundjoin%
\definecolor{currentfill}{rgb}{1.000000,0.705882,0.509804}%
\pgfsetfillcolor{currentfill}%
\pgfsetlinewidth{0.481800pt}%
\definecolor{currentstroke}{rgb}{1.000000,1.000000,1.000000}%
\pgfsetstrokecolor{currentstroke}%
\pgfsetdash{}{0pt}%
\pgfpathmoveto{\pgfqpoint{3.378019in}{0.832086in}}%
\pgfpathcurveto{\pgfqpoint{3.389069in}{0.832086in}}{\pgfqpoint{3.399668in}{0.836477in}}{\pgfqpoint{3.407482in}{0.844290in}}%
\pgfpathcurveto{\pgfqpoint{3.415295in}{0.852104in}}{\pgfqpoint{3.419686in}{0.862703in}}{\pgfqpoint{3.419686in}{0.873753in}}%
\pgfpathcurveto{\pgfqpoint{3.419686in}{0.884803in}}{\pgfqpoint{3.415295in}{0.895402in}}{\pgfqpoint{3.407482in}{0.903216in}}%
\pgfpathcurveto{\pgfqpoint{3.399668in}{0.911029in}}{\pgfqpoint{3.389069in}{0.915420in}}{\pgfqpoint{3.378019in}{0.915420in}}%
\pgfpathcurveto{\pgfqpoint{3.366969in}{0.915420in}}{\pgfqpoint{3.356370in}{0.911029in}}{\pgfqpoint{3.348556in}{0.903216in}}%
\pgfpathcurveto{\pgfqpoint{3.340743in}{0.895402in}}{\pgfqpoint{3.336352in}{0.884803in}}{\pgfqpoint{3.336352in}{0.873753in}}%
\pgfpathcurveto{\pgfqpoint{3.336352in}{0.862703in}}{\pgfqpoint{3.340743in}{0.852104in}}{\pgfqpoint{3.348556in}{0.844290in}}%
\pgfpathcurveto{\pgfqpoint{3.356370in}{0.836477in}}{\pgfqpoint{3.366969in}{0.832086in}}{\pgfqpoint{3.378019in}{0.832086in}}%
\pgfpathclose%
\pgfusepath{stroke,fill}%
\end{pgfscope}%
\begin{pgfscope}%
\pgfpathrectangle{\pgfqpoint{0.481978in}{0.331635in}}{\pgfqpoint{4.960000in}{3.696000in}}%
\pgfusepath{clip}%
\pgfsetbuttcap%
\pgfsetroundjoin%
\definecolor{currentfill}{rgb}{1.000000,0.705882,0.509804}%
\pgfsetfillcolor{currentfill}%
\pgfsetlinewidth{0.481800pt}%
\definecolor{currentstroke}{rgb}{1.000000,1.000000,1.000000}%
\pgfsetstrokecolor{currentstroke}%
\pgfsetdash{}{0pt}%
\pgfpathmoveto{\pgfqpoint{4.559563in}{1.933486in}}%
\pgfpathcurveto{\pgfqpoint{4.570613in}{1.933486in}}{\pgfqpoint{4.581212in}{1.937876in}}{\pgfqpoint{4.589025in}{1.945690in}}%
\pgfpathcurveto{\pgfqpoint{4.596839in}{1.953504in}}{\pgfqpoint{4.601229in}{1.964103in}}{\pgfqpoint{4.601229in}{1.975153in}}%
\pgfpathcurveto{\pgfqpoint{4.601229in}{1.986203in}}{\pgfqpoint{4.596839in}{1.996802in}}{\pgfqpoint{4.589025in}{2.004616in}}%
\pgfpathcurveto{\pgfqpoint{4.581212in}{2.012429in}}{\pgfqpoint{4.570613in}{2.016819in}}{\pgfqpoint{4.559563in}{2.016819in}}%
\pgfpathcurveto{\pgfqpoint{4.548512in}{2.016819in}}{\pgfqpoint{4.537913in}{2.012429in}}{\pgfqpoint{4.530100in}{2.004616in}}%
\pgfpathcurveto{\pgfqpoint{4.522286in}{1.996802in}}{\pgfqpoint{4.517896in}{1.986203in}}{\pgfqpoint{4.517896in}{1.975153in}}%
\pgfpathcurveto{\pgfqpoint{4.517896in}{1.964103in}}{\pgfqpoint{4.522286in}{1.953504in}}{\pgfqpoint{4.530100in}{1.945690in}}%
\pgfpathcurveto{\pgfqpoint{4.537913in}{1.937876in}}{\pgfqpoint{4.548512in}{1.933486in}}{\pgfqpoint{4.559563in}{1.933486in}}%
\pgfpathclose%
\pgfusepath{stroke,fill}%
\end{pgfscope}%
\begin{pgfscope}%
\pgfpathrectangle{\pgfqpoint{0.481978in}{0.331635in}}{\pgfqpoint{4.960000in}{3.696000in}}%
\pgfusepath{clip}%
\pgfsetbuttcap%
\pgfsetroundjoin%
\definecolor{currentfill}{rgb}{1.000000,0.705882,0.509804}%
\pgfsetfillcolor{currentfill}%
\pgfsetlinewidth{0.481800pt}%
\definecolor{currentstroke}{rgb}{1.000000,1.000000,1.000000}%
\pgfsetstrokecolor{currentstroke}%
\pgfsetdash{}{0pt}%
\pgfpathmoveto{\pgfqpoint{3.101805in}{1.583508in}}%
\pgfpathcurveto{\pgfqpoint{3.112855in}{1.583508in}}{\pgfqpoint{3.123454in}{1.587899in}}{\pgfqpoint{3.131268in}{1.595712in}}%
\pgfpathcurveto{\pgfqpoint{3.139081in}{1.603526in}}{\pgfqpoint{3.143472in}{1.614125in}}{\pgfqpoint{3.143472in}{1.625175in}}%
\pgfpathcurveto{\pgfqpoint{3.143472in}{1.636225in}}{\pgfqpoint{3.139081in}{1.646824in}}{\pgfqpoint{3.131268in}{1.654638in}}%
\pgfpathcurveto{\pgfqpoint{3.123454in}{1.662451in}}{\pgfqpoint{3.112855in}{1.666842in}}{\pgfqpoint{3.101805in}{1.666842in}}%
\pgfpathcurveto{\pgfqpoint{3.090755in}{1.666842in}}{\pgfqpoint{3.080156in}{1.662451in}}{\pgfqpoint{3.072342in}{1.654638in}}%
\pgfpathcurveto{\pgfqpoint{3.064529in}{1.646824in}}{\pgfqpoint{3.060138in}{1.636225in}}{\pgfqpoint{3.060138in}{1.625175in}}%
\pgfpathcurveto{\pgfqpoint{3.060138in}{1.614125in}}{\pgfqpoint{3.064529in}{1.603526in}}{\pgfqpoint{3.072342in}{1.595712in}}%
\pgfpathcurveto{\pgfqpoint{3.080156in}{1.587899in}}{\pgfqpoint{3.090755in}{1.583508in}}{\pgfqpoint{3.101805in}{1.583508in}}%
\pgfpathclose%
\pgfusepath{stroke,fill}%
\end{pgfscope}%
\begin{pgfscope}%
\pgfpathrectangle{\pgfqpoint{0.481978in}{0.331635in}}{\pgfqpoint{4.960000in}{3.696000in}}%
\pgfusepath{clip}%
\pgfsetbuttcap%
\pgfsetroundjoin%
\definecolor{currentfill}{rgb}{1.000000,0.705882,0.509804}%
\pgfsetfillcolor{currentfill}%
\pgfsetlinewidth{0.481800pt}%
\definecolor{currentstroke}{rgb}{1.000000,1.000000,1.000000}%
\pgfsetstrokecolor{currentstroke}%
\pgfsetdash{}{0pt}%
\pgfpathmoveto{\pgfqpoint{2.525832in}{1.364753in}}%
\pgfpathcurveto{\pgfqpoint{2.536882in}{1.364753in}}{\pgfqpoint{2.547481in}{1.369143in}}{\pgfqpoint{2.555294in}{1.376957in}}%
\pgfpathcurveto{\pgfqpoint{2.563108in}{1.384771in}}{\pgfqpoint{2.567498in}{1.395370in}}{\pgfqpoint{2.567498in}{1.406420in}}%
\pgfpathcurveto{\pgfqpoint{2.567498in}{1.417470in}}{\pgfqpoint{2.563108in}{1.428069in}}{\pgfqpoint{2.555294in}{1.435882in}}%
\pgfpathcurveto{\pgfqpoint{2.547481in}{1.443696in}}{\pgfqpoint{2.536882in}{1.448086in}}{\pgfqpoint{2.525832in}{1.448086in}}%
\pgfpathcurveto{\pgfqpoint{2.514781in}{1.448086in}}{\pgfqpoint{2.504182in}{1.443696in}}{\pgfqpoint{2.496369in}{1.435882in}}%
\pgfpathcurveto{\pgfqpoint{2.488555in}{1.428069in}}{\pgfqpoint{2.484165in}{1.417470in}}{\pgfqpoint{2.484165in}{1.406420in}}%
\pgfpathcurveto{\pgfqpoint{2.484165in}{1.395370in}}{\pgfqpoint{2.488555in}{1.384771in}}{\pgfqpoint{2.496369in}{1.376957in}}%
\pgfpathcurveto{\pgfqpoint{2.504182in}{1.369143in}}{\pgfqpoint{2.514781in}{1.364753in}}{\pgfqpoint{2.525832in}{1.364753in}}%
\pgfpathclose%
\pgfusepath{stroke,fill}%
\end{pgfscope}%
\begin{pgfscope}%
\pgfpathrectangle{\pgfqpoint{0.481978in}{0.331635in}}{\pgfqpoint{4.960000in}{3.696000in}}%
\pgfusepath{clip}%
\pgfsetbuttcap%
\pgfsetroundjoin%
\definecolor{currentfill}{rgb}{1.000000,0.705882,0.509804}%
\pgfsetfillcolor{currentfill}%
\pgfsetlinewidth{0.481800pt}%
\definecolor{currentstroke}{rgb}{1.000000,1.000000,1.000000}%
\pgfsetstrokecolor{currentstroke}%
\pgfsetdash{}{0pt}%
\pgfpathmoveto{\pgfqpoint{4.772753in}{2.716322in}}%
\pgfpathcurveto{\pgfqpoint{4.783804in}{2.716322in}}{\pgfqpoint{4.794403in}{2.720712in}}{\pgfqpoint{4.802216in}{2.728525in}}%
\pgfpathcurveto{\pgfqpoint{4.810030in}{2.736339in}}{\pgfqpoint{4.814420in}{2.746938in}}{\pgfqpoint{4.814420in}{2.757988in}}%
\pgfpathcurveto{\pgfqpoint{4.814420in}{2.769038in}}{\pgfqpoint{4.810030in}{2.779637in}}{\pgfqpoint{4.802216in}{2.787451in}}%
\pgfpathcurveto{\pgfqpoint{4.794403in}{2.795265in}}{\pgfqpoint{4.783804in}{2.799655in}}{\pgfqpoint{4.772753in}{2.799655in}}%
\pgfpathcurveto{\pgfqpoint{4.761703in}{2.799655in}}{\pgfqpoint{4.751104in}{2.795265in}}{\pgfqpoint{4.743291in}{2.787451in}}%
\pgfpathcurveto{\pgfqpoint{4.735477in}{2.779637in}}{\pgfqpoint{4.731087in}{2.769038in}}{\pgfqpoint{4.731087in}{2.757988in}}%
\pgfpathcurveto{\pgfqpoint{4.731087in}{2.746938in}}{\pgfqpoint{4.735477in}{2.736339in}}{\pgfqpoint{4.743291in}{2.728525in}}%
\pgfpathcurveto{\pgfqpoint{4.751104in}{2.720712in}}{\pgfqpoint{4.761703in}{2.716322in}}{\pgfqpoint{4.772753in}{2.716322in}}%
\pgfpathclose%
\pgfusepath{stroke,fill}%
\end{pgfscope}%
\begin{pgfscope}%
\pgfpathrectangle{\pgfqpoint{0.481978in}{0.331635in}}{\pgfqpoint{4.960000in}{3.696000in}}%
\pgfusepath{clip}%
\pgfsetbuttcap%
\pgfsetroundjoin%
\definecolor{currentfill}{rgb}{1.000000,0.705882,0.509804}%
\pgfsetfillcolor{currentfill}%
\pgfsetlinewidth{0.481800pt}%
\definecolor{currentstroke}{rgb}{1.000000,1.000000,1.000000}%
\pgfsetstrokecolor{currentstroke}%
\pgfsetdash{}{0pt}%
\pgfpathmoveto{\pgfqpoint{2.977732in}{0.517611in}}%
\pgfpathcurveto{\pgfqpoint{2.988782in}{0.517611in}}{\pgfqpoint{2.999381in}{0.522002in}}{\pgfqpoint{3.007195in}{0.529815in}}%
\pgfpathcurveto{\pgfqpoint{3.015009in}{0.537629in}}{\pgfqpoint{3.019399in}{0.548228in}}{\pgfqpoint{3.019399in}{0.559278in}}%
\pgfpathcurveto{\pgfqpoint{3.019399in}{0.570328in}}{\pgfqpoint{3.015009in}{0.580927in}}{\pgfqpoint{3.007195in}{0.588741in}}%
\pgfpathcurveto{\pgfqpoint{2.999381in}{0.596554in}}{\pgfqpoint{2.988782in}{0.600945in}}{\pgfqpoint{2.977732in}{0.600945in}}%
\pgfpathcurveto{\pgfqpoint{2.966682in}{0.600945in}}{\pgfqpoint{2.956083in}{0.596554in}}{\pgfqpoint{2.948269in}{0.588741in}}%
\pgfpathcurveto{\pgfqpoint{2.940456in}{0.580927in}}{\pgfqpoint{2.936065in}{0.570328in}}{\pgfqpoint{2.936065in}{0.559278in}}%
\pgfpathcurveto{\pgfqpoint{2.936065in}{0.548228in}}{\pgfqpoint{2.940456in}{0.537629in}}{\pgfqpoint{2.948269in}{0.529815in}}%
\pgfpathcurveto{\pgfqpoint{2.956083in}{0.522002in}}{\pgfqpoint{2.966682in}{0.517611in}}{\pgfqpoint{2.977732in}{0.517611in}}%
\pgfpathclose%
\pgfusepath{stroke,fill}%
\end{pgfscope}%
\begin{pgfscope}%
\pgfpathrectangle{\pgfqpoint{0.481978in}{0.331635in}}{\pgfqpoint{4.960000in}{3.696000in}}%
\pgfusepath{clip}%
\pgfsetbuttcap%
\pgfsetroundjoin%
\definecolor{currentfill}{rgb}{1.000000,0.705882,0.509804}%
\pgfsetfillcolor{currentfill}%
\pgfsetlinewidth{0.481800pt}%
\definecolor{currentstroke}{rgb}{1.000000,1.000000,1.000000}%
\pgfsetstrokecolor{currentstroke}%
\pgfsetdash{}{0pt}%
\pgfpathmoveto{\pgfqpoint{3.368601in}{0.690201in}}%
\pgfpathcurveto{\pgfqpoint{3.379651in}{0.690201in}}{\pgfqpoint{3.390250in}{0.694591in}}{\pgfqpoint{3.398064in}{0.702405in}}%
\pgfpathcurveto{\pgfqpoint{3.405877in}{0.710219in}}{\pgfqpoint{3.410268in}{0.720818in}}{\pgfqpoint{3.410268in}{0.731868in}}%
\pgfpathcurveto{\pgfqpoint{3.410268in}{0.742918in}}{\pgfqpoint{3.405877in}{0.753517in}}{\pgfqpoint{3.398064in}{0.761330in}}%
\pgfpathcurveto{\pgfqpoint{3.390250in}{0.769144in}}{\pgfqpoint{3.379651in}{0.773534in}}{\pgfqpoint{3.368601in}{0.773534in}}%
\pgfpathcurveto{\pgfqpoint{3.357551in}{0.773534in}}{\pgfqpoint{3.346952in}{0.769144in}}{\pgfqpoint{3.339138in}{0.761330in}}%
\pgfpathcurveto{\pgfqpoint{3.331324in}{0.753517in}}{\pgfqpoint{3.326934in}{0.742918in}}{\pgfqpoint{3.326934in}{0.731868in}}%
\pgfpathcurveto{\pgfqpoint{3.326934in}{0.720818in}}{\pgfqpoint{3.331324in}{0.710219in}}{\pgfqpoint{3.339138in}{0.702405in}}%
\pgfpathcurveto{\pgfqpoint{3.346952in}{0.694591in}}{\pgfqpoint{3.357551in}{0.690201in}}{\pgfqpoint{3.368601in}{0.690201in}}%
\pgfpathclose%
\pgfusepath{stroke,fill}%
\end{pgfscope}%
\begin{pgfscope}%
\pgfpathrectangle{\pgfqpoint{0.481978in}{0.331635in}}{\pgfqpoint{4.960000in}{3.696000in}}%
\pgfusepath{clip}%
\pgfsetbuttcap%
\pgfsetroundjoin%
\definecolor{currentfill}{rgb}{1.000000,0.705882,0.509804}%
\pgfsetfillcolor{currentfill}%
\pgfsetlinewidth{0.481800pt}%
\definecolor{currentstroke}{rgb}{1.000000,1.000000,1.000000}%
\pgfsetstrokecolor{currentstroke}%
\pgfsetdash{}{0pt}%
\pgfpathmoveto{\pgfqpoint{3.341619in}{2.386497in}}%
\pgfpathcurveto{\pgfqpoint{3.352669in}{2.386497in}}{\pgfqpoint{3.363268in}{2.390887in}}{\pgfqpoint{3.371082in}{2.398701in}}%
\pgfpathcurveto{\pgfqpoint{3.378896in}{2.406515in}}{\pgfqpoint{3.383286in}{2.417114in}}{\pgfqpoint{3.383286in}{2.428164in}}%
\pgfpathcurveto{\pgfqpoint{3.383286in}{2.439214in}}{\pgfqpoint{3.378896in}{2.449813in}}{\pgfqpoint{3.371082in}{2.457626in}}%
\pgfpathcurveto{\pgfqpoint{3.363268in}{2.465440in}}{\pgfqpoint{3.352669in}{2.469830in}}{\pgfqpoint{3.341619in}{2.469830in}}%
\pgfpathcurveto{\pgfqpoint{3.330569in}{2.469830in}}{\pgfqpoint{3.319970in}{2.465440in}}{\pgfqpoint{3.312156in}{2.457626in}}%
\pgfpathcurveto{\pgfqpoint{3.304343in}{2.449813in}}{\pgfqpoint{3.299953in}{2.439214in}}{\pgfqpoint{3.299953in}{2.428164in}}%
\pgfpathcurveto{\pgfqpoint{3.299953in}{2.417114in}}{\pgfqpoint{3.304343in}{2.406515in}}{\pgfqpoint{3.312156in}{2.398701in}}%
\pgfpathcurveto{\pgfqpoint{3.319970in}{2.390887in}}{\pgfqpoint{3.330569in}{2.386497in}}{\pgfqpoint{3.341619in}{2.386497in}}%
\pgfpathclose%
\pgfusepath{stroke,fill}%
\end{pgfscope}%
\begin{pgfscope}%
\pgfpathrectangle{\pgfqpoint{0.481978in}{0.331635in}}{\pgfqpoint{4.960000in}{3.696000in}}%
\pgfusepath{clip}%
\pgfsetbuttcap%
\pgfsetroundjoin%
\definecolor{currentfill}{rgb}{1.000000,0.705882,0.509804}%
\pgfsetfillcolor{currentfill}%
\pgfsetlinewidth{0.481800pt}%
\definecolor{currentstroke}{rgb}{1.000000,1.000000,1.000000}%
\pgfsetstrokecolor{currentstroke}%
\pgfsetdash{}{0pt}%
\pgfpathmoveto{\pgfqpoint{1.366891in}{1.195406in}}%
\pgfpathcurveto{\pgfqpoint{1.377941in}{1.195406in}}{\pgfqpoint{1.388540in}{1.199797in}}{\pgfqpoint{1.396354in}{1.207610in}}%
\pgfpathcurveto{\pgfqpoint{1.404168in}{1.215424in}}{\pgfqpoint{1.408558in}{1.226023in}}{\pgfqpoint{1.408558in}{1.237073in}}%
\pgfpathcurveto{\pgfqpoint{1.408558in}{1.248123in}}{\pgfqpoint{1.404168in}{1.258722in}}{\pgfqpoint{1.396354in}{1.266536in}}%
\pgfpathcurveto{\pgfqpoint{1.388540in}{1.274349in}}{\pgfqpoint{1.377941in}{1.278740in}}{\pgfqpoint{1.366891in}{1.278740in}}%
\pgfpathcurveto{\pgfqpoint{1.355841in}{1.278740in}}{\pgfqpoint{1.345242in}{1.274349in}}{\pgfqpoint{1.337429in}{1.266536in}}%
\pgfpathcurveto{\pgfqpoint{1.329615in}{1.258722in}}{\pgfqpoint{1.325225in}{1.248123in}}{\pgfqpoint{1.325225in}{1.237073in}}%
\pgfpathcurveto{\pgfqpoint{1.325225in}{1.226023in}}{\pgfqpoint{1.329615in}{1.215424in}}{\pgfqpoint{1.337429in}{1.207610in}}%
\pgfpathcurveto{\pgfqpoint{1.345242in}{1.199797in}}{\pgfqpoint{1.355841in}{1.195406in}}{\pgfqpoint{1.366891in}{1.195406in}}%
\pgfpathclose%
\pgfusepath{stroke,fill}%
\end{pgfscope}%
\begin{pgfscope}%
\pgfpathrectangle{\pgfqpoint{0.481978in}{0.331635in}}{\pgfqpoint{4.960000in}{3.696000in}}%
\pgfusepath{clip}%
\pgfsetbuttcap%
\pgfsetroundjoin%
\definecolor{currentfill}{rgb}{1.000000,0.705882,0.509804}%
\pgfsetfillcolor{currentfill}%
\pgfsetlinewidth{0.481800pt}%
\definecolor{currentstroke}{rgb}{1.000000,1.000000,1.000000}%
\pgfsetstrokecolor{currentstroke}%
\pgfsetdash{}{0pt}%
\pgfpathmoveto{\pgfqpoint{1.566891in}{1.750957in}}%
\pgfpathcurveto{\pgfqpoint{1.577941in}{1.750957in}}{\pgfqpoint{1.588540in}{1.755348in}}{\pgfqpoint{1.596353in}{1.763161in}}%
\pgfpathcurveto{\pgfqpoint{1.604167in}{1.770975in}}{\pgfqpoint{1.608557in}{1.781574in}}{\pgfqpoint{1.608557in}{1.792624in}}%
\pgfpathcurveto{\pgfqpoint{1.608557in}{1.803674in}}{\pgfqpoint{1.604167in}{1.814273in}}{\pgfqpoint{1.596353in}{1.822087in}}%
\pgfpathcurveto{\pgfqpoint{1.588540in}{1.829900in}}{\pgfqpoint{1.577941in}{1.834291in}}{\pgfqpoint{1.566891in}{1.834291in}}%
\pgfpathcurveto{\pgfqpoint{1.555840in}{1.834291in}}{\pgfqpoint{1.545241in}{1.829900in}}{\pgfqpoint{1.537428in}{1.822087in}}%
\pgfpathcurveto{\pgfqpoint{1.529614in}{1.814273in}}{\pgfqpoint{1.525224in}{1.803674in}}{\pgfqpoint{1.525224in}{1.792624in}}%
\pgfpathcurveto{\pgfqpoint{1.525224in}{1.781574in}}{\pgfqpoint{1.529614in}{1.770975in}}{\pgfqpoint{1.537428in}{1.763161in}}%
\pgfpathcurveto{\pgfqpoint{1.545241in}{1.755348in}}{\pgfqpoint{1.555840in}{1.750957in}}{\pgfqpoint{1.566891in}{1.750957in}}%
\pgfpathclose%
\pgfusepath{stroke,fill}%
\end{pgfscope}%
\begin{pgfscope}%
\pgfpathrectangle{\pgfqpoint{0.481978in}{0.331635in}}{\pgfqpoint{4.960000in}{3.696000in}}%
\pgfusepath{clip}%
\pgfsetbuttcap%
\pgfsetroundjoin%
\definecolor{currentfill}{rgb}{1.000000,0.705882,0.509804}%
\pgfsetfillcolor{currentfill}%
\pgfsetlinewidth{0.481800pt}%
\definecolor{currentstroke}{rgb}{1.000000,1.000000,1.000000}%
\pgfsetstrokecolor{currentstroke}%
\pgfsetdash{}{0pt}%
\pgfpathmoveto{\pgfqpoint{2.255178in}{0.512178in}}%
\pgfpathcurveto{\pgfqpoint{2.266229in}{0.512178in}}{\pgfqpoint{2.276828in}{0.516568in}}{\pgfqpoint{2.284641in}{0.524382in}}%
\pgfpathcurveto{\pgfqpoint{2.292455in}{0.532196in}}{\pgfqpoint{2.296845in}{0.542795in}}{\pgfqpoint{2.296845in}{0.553845in}}%
\pgfpathcurveto{\pgfqpoint{2.296845in}{0.564895in}}{\pgfqpoint{2.292455in}{0.575494in}}{\pgfqpoint{2.284641in}{0.583308in}}%
\pgfpathcurveto{\pgfqpoint{2.276828in}{0.591121in}}{\pgfqpoint{2.266229in}{0.595511in}}{\pgfqpoint{2.255178in}{0.595511in}}%
\pgfpathcurveto{\pgfqpoint{2.244128in}{0.595511in}}{\pgfqpoint{2.233529in}{0.591121in}}{\pgfqpoint{2.225716in}{0.583308in}}%
\pgfpathcurveto{\pgfqpoint{2.217902in}{0.575494in}}{\pgfqpoint{2.213512in}{0.564895in}}{\pgfqpoint{2.213512in}{0.553845in}}%
\pgfpathcurveto{\pgfqpoint{2.213512in}{0.542795in}}{\pgfqpoint{2.217902in}{0.532196in}}{\pgfqpoint{2.225716in}{0.524382in}}%
\pgfpathcurveto{\pgfqpoint{2.233529in}{0.516568in}}{\pgfqpoint{2.244128in}{0.512178in}}{\pgfqpoint{2.255178in}{0.512178in}}%
\pgfpathclose%
\pgfusepath{stroke,fill}%
\end{pgfscope}%
\begin{pgfscope}%
\pgfpathrectangle{\pgfqpoint{0.481978in}{0.331635in}}{\pgfqpoint{4.960000in}{3.696000in}}%
\pgfusepath{clip}%
\pgfsetbuttcap%
\pgfsetroundjoin%
\definecolor{currentfill}{rgb}{1.000000,0.705882,0.509804}%
\pgfsetfillcolor{currentfill}%
\pgfsetlinewidth{0.481800pt}%
\definecolor{currentstroke}{rgb}{1.000000,1.000000,1.000000}%
\pgfsetstrokecolor{currentstroke}%
\pgfsetdash{}{0pt}%
\pgfpathmoveto{\pgfqpoint{4.642349in}{2.176754in}}%
\pgfpathcurveto{\pgfqpoint{4.653399in}{2.176754in}}{\pgfqpoint{4.663998in}{2.181144in}}{\pgfqpoint{4.671812in}{2.188958in}}%
\pgfpathcurveto{\pgfqpoint{4.679625in}{2.196771in}}{\pgfqpoint{4.684016in}{2.207370in}}{\pgfqpoint{4.684016in}{2.218420in}}%
\pgfpathcurveto{\pgfqpoint{4.684016in}{2.229471in}}{\pgfqpoint{4.679625in}{2.240070in}}{\pgfqpoint{4.671812in}{2.247883in}}%
\pgfpathcurveto{\pgfqpoint{4.663998in}{2.255697in}}{\pgfqpoint{4.653399in}{2.260087in}}{\pgfqpoint{4.642349in}{2.260087in}}%
\pgfpathcurveto{\pgfqpoint{4.631299in}{2.260087in}}{\pgfqpoint{4.620700in}{2.255697in}}{\pgfqpoint{4.612886in}{2.247883in}}%
\pgfpathcurveto{\pgfqpoint{4.605073in}{2.240070in}}{\pgfqpoint{4.600682in}{2.229471in}}{\pgfqpoint{4.600682in}{2.218420in}}%
\pgfpathcurveto{\pgfqpoint{4.600682in}{2.207370in}}{\pgfqpoint{4.605073in}{2.196771in}}{\pgfqpoint{4.612886in}{2.188958in}}%
\pgfpathcurveto{\pgfqpoint{4.620700in}{2.181144in}}{\pgfqpoint{4.631299in}{2.176754in}}{\pgfqpoint{4.642349in}{2.176754in}}%
\pgfpathclose%
\pgfusepath{stroke,fill}%
\end{pgfscope}%
\begin{pgfscope}%
\pgfpathrectangle{\pgfqpoint{0.481978in}{0.331635in}}{\pgfqpoint{4.960000in}{3.696000in}}%
\pgfusepath{clip}%
\pgfsetbuttcap%
\pgfsetroundjoin%
\definecolor{currentfill}{rgb}{1.000000,0.705882,0.509804}%
\pgfsetfillcolor{currentfill}%
\pgfsetlinewidth{0.481800pt}%
\definecolor{currentstroke}{rgb}{1.000000,1.000000,1.000000}%
\pgfsetstrokecolor{currentstroke}%
\pgfsetdash{}{0pt}%
\pgfpathmoveto{\pgfqpoint{1.557977in}{0.970104in}}%
\pgfpathcurveto{\pgfqpoint{1.569027in}{0.970104in}}{\pgfqpoint{1.579626in}{0.974495in}}{\pgfqpoint{1.587439in}{0.982308in}}%
\pgfpathcurveto{\pgfqpoint{1.595253in}{0.990122in}}{\pgfqpoint{1.599643in}{1.000721in}}{\pgfqpoint{1.599643in}{1.011771in}}%
\pgfpathcurveto{\pgfqpoint{1.599643in}{1.022821in}}{\pgfqpoint{1.595253in}{1.033420in}}{\pgfqpoint{1.587439in}{1.041234in}}%
\pgfpathcurveto{\pgfqpoint{1.579626in}{1.049047in}}{\pgfqpoint{1.569027in}{1.053438in}}{\pgfqpoint{1.557977in}{1.053438in}}%
\pgfpathcurveto{\pgfqpoint{1.546927in}{1.053438in}}{\pgfqpoint{1.536328in}{1.049047in}}{\pgfqpoint{1.528514in}{1.041234in}}%
\pgfpathcurveto{\pgfqpoint{1.520700in}{1.033420in}}{\pgfqpoint{1.516310in}{1.022821in}}{\pgfqpoint{1.516310in}{1.011771in}}%
\pgfpathcurveto{\pgfqpoint{1.516310in}{1.000721in}}{\pgfqpoint{1.520700in}{0.990122in}}{\pgfqpoint{1.528514in}{0.982308in}}%
\pgfpathcurveto{\pgfqpoint{1.536328in}{0.974495in}}{\pgfqpoint{1.546927in}{0.970104in}}{\pgfqpoint{1.557977in}{0.970104in}}%
\pgfpathclose%
\pgfusepath{stroke,fill}%
\end{pgfscope}%
\begin{pgfscope}%
\pgfpathrectangle{\pgfqpoint{0.481978in}{0.331635in}}{\pgfqpoint{4.960000in}{3.696000in}}%
\pgfusepath{clip}%
\pgfsetbuttcap%
\pgfsetroundjoin%
\definecolor{currentfill}{rgb}{1.000000,0.705882,0.509804}%
\pgfsetfillcolor{currentfill}%
\pgfsetlinewidth{0.481800pt}%
\definecolor{currentstroke}{rgb}{1.000000,1.000000,1.000000}%
\pgfsetstrokecolor{currentstroke}%
\pgfsetdash{}{0pt}%
\pgfpathmoveto{\pgfqpoint{3.724605in}{0.758587in}}%
\pgfpathcurveto{\pgfqpoint{3.735655in}{0.758587in}}{\pgfqpoint{3.746254in}{0.762977in}}{\pgfqpoint{3.754067in}{0.770791in}}%
\pgfpathcurveto{\pgfqpoint{3.761881in}{0.778604in}}{\pgfqpoint{3.766271in}{0.789203in}}{\pgfqpoint{3.766271in}{0.800254in}}%
\pgfpathcurveto{\pgfqpoint{3.766271in}{0.811304in}}{\pgfqpoint{3.761881in}{0.821903in}}{\pgfqpoint{3.754067in}{0.829716in}}%
\pgfpathcurveto{\pgfqpoint{3.746254in}{0.837530in}}{\pgfqpoint{3.735655in}{0.841920in}}{\pgfqpoint{3.724605in}{0.841920in}}%
\pgfpathcurveto{\pgfqpoint{3.713554in}{0.841920in}}{\pgfqpoint{3.702955in}{0.837530in}}{\pgfqpoint{3.695142in}{0.829716in}}%
\pgfpathcurveto{\pgfqpoint{3.687328in}{0.821903in}}{\pgfqpoint{3.682938in}{0.811304in}}{\pgfqpoint{3.682938in}{0.800254in}}%
\pgfpathcurveto{\pgfqpoint{3.682938in}{0.789203in}}{\pgfqpoint{3.687328in}{0.778604in}}{\pgfqpoint{3.695142in}{0.770791in}}%
\pgfpathcurveto{\pgfqpoint{3.702955in}{0.762977in}}{\pgfqpoint{3.713554in}{0.758587in}}{\pgfqpoint{3.724605in}{0.758587in}}%
\pgfpathclose%
\pgfusepath{stroke,fill}%
\end{pgfscope}%
\begin{pgfscope}%
\pgfpathrectangle{\pgfqpoint{0.481978in}{0.331635in}}{\pgfqpoint{4.960000in}{3.696000in}}%
\pgfusepath{clip}%
\pgfsetbuttcap%
\pgfsetroundjoin%
\definecolor{currentfill}{rgb}{1.000000,0.705882,0.509804}%
\pgfsetfillcolor{currentfill}%
\pgfsetlinewidth{0.481800pt}%
\definecolor{currentstroke}{rgb}{1.000000,1.000000,1.000000}%
\pgfsetstrokecolor{currentstroke}%
\pgfsetdash{}{0pt}%
\pgfpathmoveto{\pgfqpoint{2.287234in}{0.726757in}}%
\pgfpathcurveto{\pgfqpoint{2.298284in}{0.726757in}}{\pgfqpoint{2.308883in}{0.731148in}}{\pgfqpoint{2.316697in}{0.738961in}}%
\pgfpathcurveto{\pgfqpoint{2.324511in}{0.746775in}}{\pgfqpoint{2.328901in}{0.757374in}}{\pgfqpoint{2.328901in}{0.768424in}}%
\pgfpathcurveto{\pgfqpoint{2.328901in}{0.779474in}}{\pgfqpoint{2.324511in}{0.790073in}}{\pgfqpoint{2.316697in}{0.797887in}}%
\pgfpathcurveto{\pgfqpoint{2.308883in}{0.805700in}}{\pgfqpoint{2.298284in}{0.810091in}}{\pgfqpoint{2.287234in}{0.810091in}}%
\pgfpathcurveto{\pgfqpoint{2.276184in}{0.810091in}}{\pgfqpoint{2.265585in}{0.805700in}}{\pgfqpoint{2.257771in}{0.797887in}}%
\pgfpathcurveto{\pgfqpoint{2.249958in}{0.790073in}}{\pgfqpoint{2.245568in}{0.779474in}}{\pgfqpoint{2.245568in}{0.768424in}}%
\pgfpathcurveto{\pgfqpoint{2.245568in}{0.757374in}}{\pgfqpoint{2.249958in}{0.746775in}}{\pgfqpoint{2.257771in}{0.738961in}}%
\pgfpathcurveto{\pgfqpoint{2.265585in}{0.731148in}}{\pgfqpoint{2.276184in}{0.726757in}}{\pgfqpoint{2.287234in}{0.726757in}}%
\pgfpathclose%
\pgfusepath{stroke,fill}%
\end{pgfscope}%
\begin{pgfscope}%
\pgfpathrectangle{\pgfqpoint{0.481978in}{0.331635in}}{\pgfqpoint{4.960000in}{3.696000in}}%
\pgfusepath{clip}%
\pgfsetbuttcap%
\pgfsetroundjoin%
\definecolor{currentfill}{rgb}{1.000000,0.705882,0.509804}%
\pgfsetfillcolor{currentfill}%
\pgfsetlinewidth{0.481800pt}%
\definecolor{currentstroke}{rgb}{1.000000,1.000000,1.000000}%
\pgfsetstrokecolor{currentstroke}%
\pgfsetdash{}{0pt}%
\pgfpathmoveto{\pgfqpoint{3.617579in}{1.766817in}}%
\pgfpathcurveto{\pgfqpoint{3.628629in}{1.766817in}}{\pgfqpoint{3.639228in}{1.771207in}}{\pgfqpoint{3.647042in}{1.779021in}}%
\pgfpathcurveto{\pgfqpoint{3.654856in}{1.786834in}}{\pgfqpoint{3.659246in}{1.797434in}}{\pgfqpoint{3.659246in}{1.808484in}}%
\pgfpathcurveto{\pgfqpoint{3.659246in}{1.819534in}}{\pgfqpoint{3.654856in}{1.830133in}}{\pgfqpoint{3.647042in}{1.837946in}}%
\pgfpathcurveto{\pgfqpoint{3.639228in}{1.845760in}}{\pgfqpoint{3.628629in}{1.850150in}}{\pgfqpoint{3.617579in}{1.850150in}}%
\pgfpathcurveto{\pgfqpoint{3.606529in}{1.850150in}}{\pgfqpoint{3.595930in}{1.845760in}}{\pgfqpoint{3.588116in}{1.837946in}}%
\pgfpathcurveto{\pgfqpoint{3.580303in}{1.830133in}}{\pgfqpoint{3.575913in}{1.819534in}}{\pgfqpoint{3.575913in}{1.808484in}}%
\pgfpathcurveto{\pgfqpoint{3.575913in}{1.797434in}}{\pgfqpoint{3.580303in}{1.786834in}}{\pgfqpoint{3.588116in}{1.779021in}}%
\pgfpathcurveto{\pgfqpoint{3.595930in}{1.771207in}}{\pgfqpoint{3.606529in}{1.766817in}}{\pgfqpoint{3.617579in}{1.766817in}}%
\pgfpathclose%
\pgfusepath{stroke,fill}%
\end{pgfscope}%
\begin{pgfscope}%
\pgfpathrectangle{\pgfqpoint{0.481978in}{0.331635in}}{\pgfqpoint{4.960000in}{3.696000in}}%
\pgfusepath{clip}%
\pgfsetbuttcap%
\pgfsetroundjoin%
\definecolor{currentfill}{rgb}{1.000000,0.705882,0.509804}%
\pgfsetfillcolor{currentfill}%
\pgfsetlinewidth{0.481800pt}%
\definecolor{currentstroke}{rgb}{1.000000,1.000000,1.000000}%
\pgfsetstrokecolor{currentstroke}%
\pgfsetdash{}{0pt}%
\pgfpathmoveto{\pgfqpoint{1.421203in}{2.404275in}}%
\pgfpathcurveto{\pgfqpoint{1.432253in}{2.404275in}}{\pgfqpoint{1.442852in}{2.408665in}}{\pgfqpoint{1.450666in}{2.416479in}}%
\pgfpathcurveto{\pgfqpoint{1.458479in}{2.424292in}}{\pgfqpoint{1.462870in}{2.434891in}}{\pgfqpoint{1.462870in}{2.445941in}}%
\pgfpathcurveto{\pgfqpoint{1.462870in}{2.456991in}}{\pgfqpoint{1.458479in}{2.467590in}}{\pgfqpoint{1.450666in}{2.475404in}}%
\pgfpathcurveto{\pgfqpoint{1.442852in}{2.483218in}}{\pgfqpoint{1.432253in}{2.487608in}}{\pgfqpoint{1.421203in}{2.487608in}}%
\pgfpathcurveto{\pgfqpoint{1.410153in}{2.487608in}}{\pgfqpoint{1.399554in}{2.483218in}}{\pgfqpoint{1.391740in}{2.475404in}}%
\pgfpathcurveto{\pgfqpoint{1.383926in}{2.467590in}}{\pgfqpoint{1.379536in}{2.456991in}}{\pgfqpoint{1.379536in}{2.445941in}}%
\pgfpathcurveto{\pgfqpoint{1.379536in}{2.434891in}}{\pgfqpoint{1.383926in}{2.424292in}}{\pgfqpoint{1.391740in}{2.416479in}}%
\pgfpathcurveto{\pgfqpoint{1.399554in}{2.408665in}}{\pgfqpoint{1.410153in}{2.404275in}}{\pgfqpoint{1.421203in}{2.404275in}}%
\pgfpathclose%
\pgfusepath{stroke,fill}%
\end{pgfscope}%
\begin{pgfscope}%
\pgfpathrectangle{\pgfqpoint{0.481978in}{0.331635in}}{\pgfqpoint{4.960000in}{3.696000in}}%
\pgfusepath{clip}%
\pgfsetbuttcap%
\pgfsetroundjoin%
\definecolor{currentfill}{rgb}{1.000000,0.705882,0.509804}%
\pgfsetfillcolor{currentfill}%
\pgfsetlinewidth{0.481800pt}%
\definecolor{currentstroke}{rgb}{1.000000,1.000000,1.000000}%
\pgfsetstrokecolor{currentstroke}%
\pgfsetdash{}{0pt}%
\pgfpathmoveto{\pgfqpoint{4.246861in}{2.186263in}}%
\pgfpathcurveto{\pgfqpoint{4.257911in}{2.186263in}}{\pgfqpoint{4.268510in}{2.190653in}}{\pgfqpoint{4.276323in}{2.198467in}}%
\pgfpathcurveto{\pgfqpoint{4.284137in}{2.206280in}}{\pgfqpoint{4.288527in}{2.216879in}}{\pgfqpoint{4.288527in}{2.227930in}}%
\pgfpathcurveto{\pgfqpoint{4.288527in}{2.238980in}}{\pgfqpoint{4.284137in}{2.249579in}}{\pgfqpoint{4.276323in}{2.257392in}}%
\pgfpathcurveto{\pgfqpoint{4.268510in}{2.265206in}}{\pgfqpoint{4.257911in}{2.269596in}}{\pgfqpoint{4.246861in}{2.269596in}}%
\pgfpathcurveto{\pgfqpoint{4.235811in}{2.269596in}}{\pgfqpoint{4.225211in}{2.265206in}}{\pgfqpoint{4.217398in}{2.257392in}}%
\pgfpathcurveto{\pgfqpoint{4.209584in}{2.249579in}}{\pgfqpoint{4.205194in}{2.238980in}}{\pgfqpoint{4.205194in}{2.227930in}}%
\pgfpathcurveto{\pgfqpoint{4.205194in}{2.216879in}}{\pgfqpoint{4.209584in}{2.206280in}}{\pgfqpoint{4.217398in}{2.198467in}}%
\pgfpathcurveto{\pgfqpoint{4.225211in}{2.190653in}}{\pgfqpoint{4.235811in}{2.186263in}}{\pgfqpoint{4.246861in}{2.186263in}}%
\pgfpathclose%
\pgfusepath{stroke,fill}%
\end{pgfscope}%
\begin{pgfscope}%
\pgfpathrectangle{\pgfqpoint{0.481978in}{0.331635in}}{\pgfqpoint{4.960000in}{3.696000in}}%
\pgfusepath{clip}%
\pgfsetbuttcap%
\pgfsetroundjoin%
\definecolor{currentfill}{rgb}{1.000000,0.705882,0.509804}%
\pgfsetfillcolor{currentfill}%
\pgfsetlinewidth{0.481800pt}%
\definecolor{currentstroke}{rgb}{1.000000,1.000000,1.000000}%
\pgfsetstrokecolor{currentstroke}%
\pgfsetdash{}{0pt}%
\pgfpathmoveto{\pgfqpoint{4.750602in}{1.367953in}}%
\pgfpathcurveto{\pgfqpoint{4.761653in}{1.367953in}}{\pgfqpoint{4.772252in}{1.372344in}}{\pgfqpoint{4.780065in}{1.380157in}}%
\pgfpathcurveto{\pgfqpoint{4.787879in}{1.387971in}}{\pgfqpoint{4.792269in}{1.398570in}}{\pgfqpoint{4.792269in}{1.409620in}}%
\pgfpathcurveto{\pgfqpoint{4.792269in}{1.420670in}}{\pgfqpoint{4.787879in}{1.431269in}}{\pgfqpoint{4.780065in}{1.439083in}}%
\pgfpathcurveto{\pgfqpoint{4.772252in}{1.446897in}}{\pgfqpoint{4.761653in}{1.451287in}}{\pgfqpoint{4.750602in}{1.451287in}}%
\pgfpathcurveto{\pgfqpoint{4.739552in}{1.451287in}}{\pgfqpoint{4.728953in}{1.446897in}}{\pgfqpoint{4.721140in}{1.439083in}}%
\pgfpathcurveto{\pgfqpoint{4.713326in}{1.431269in}}{\pgfqpoint{4.708936in}{1.420670in}}{\pgfqpoint{4.708936in}{1.409620in}}%
\pgfpathcurveto{\pgfqpoint{4.708936in}{1.398570in}}{\pgfqpoint{4.713326in}{1.387971in}}{\pgfqpoint{4.721140in}{1.380157in}}%
\pgfpathcurveto{\pgfqpoint{4.728953in}{1.372344in}}{\pgfqpoint{4.739552in}{1.367953in}}{\pgfqpoint{4.750602in}{1.367953in}}%
\pgfpathclose%
\pgfusepath{stroke,fill}%
\end{pgfscope}%
\begin{pgfscope}%
\pgfpathrectangle{\pgfqpoint{0.481978in}{0.331635in}}{\pgfqpoint{4.960000in}{3.696000in}}%
\pgfusepath{clip}%
\pgfsetbuttcap%
\pgfsetroundjoin%
\definecolor{currentfill}{rgb}{1.000000,0.705882,0.509804}%
\pgfsetfillcolor{currentfill}%
\pgfsetlinewidth{0.481800pt}%
\definecolor{currentstroke}{rgb}{1.000000,1.000000,1.000000}%
\pgfsetstrokecolor{currentstroke}%
\pgfsetdash{}{0pt}%
\pgfpathmoveto{\pgfqpoint{2.419077in}{1.035564in}}%
\pgfpathcurveto{\pgfqpoint{2.430128in}{1.035564in}}{\pgfqpoint{2.440727in}{1.039954in}}{\pgfqpoint{2.448540in}{1.047768in}}%
\pgfpathcurveto{\pgfqpoint{2.456354in}{1.055581in}}{\pgfqpoint{2.460744in}{1.066180in}}{\pgfqpoint{2.460744in}{1.077231in}}%
\pgfpathcurveto{\pgfqpoint{2.460744in}{1.088281in}}{\pgfqpoint{2.456354in}{1.098880in}}{\pgfqpoint{2.448540in}{1.106693in}}%
\pgfpathcurveto{\pgfqpoint{2.440727in}{1.114507in}}{\pgfqpoint{2.430128in}{1.118897in}}{\pgfqpoint{2.419077in}{1.118897in}}%
\pgfpathcurveto{\pgfqpoint{2.408027in}{1.118897in}}{\pgfqpoint{2.397428in}{1.114507in}}{\pgfqpoint{2.389615in}{1.106693in}}%
\pgfpathcurveto{\pgfqpoint{2.381801in}{1.098880in}}{\pgfqpoint{2.377411in}{1.088281in}}{\pgfqpoint{2.377411in}{1.077231in}}%
\pgfpathcurveto{\pgfqpoint{2.377411in}{1.066180in}}{\pgfqpoint{2.381801in}{1.055581in}}{\pgfqpoint{2.389615in}{1.047768in}}%
\pgfpathcurveto{\pgfqpoint{2.397428in}{1.039954in}}{\pgfqpoint{2.408027in}{1.035564in}}{\pgfqpoint{2.419077in}{1.035564in}}%
\pgfpathclose%
\pgfusepath{stroke,fill}%
\end{pgfscope}%
\begin{pgfscope}%
\pgfpathrectangle{\pgfqpoint{0.481978in}{0.331635in}}{\pgfqpoint{4.960000in}{3.696000in}}%
\pgfusepath{clip}%
\pgfsetbuttcap%
\pgfsetroundjoin%
\definecolor{currentfill}{rgb}{1.000000,0.705882,0.509804}%
\pgfsetfillcolor{currentfill}%
\pgfsetlinewidth{0.481800pt}%
\definecolor{currentstroke}{rgb}{1.000000,1.000000,1.000000}%
\pgfsetstrokecolor{currentstroke}%
\pgfsetdash{}{0pt}%
\pgfpathmoveto{\pgfqpoint{2.071657in}{1.374907in}}%
\pgfpathcurveto{\pgfqpoint{2.082707in}{1.374907in}}{\pgfqpoint{2.093306in}{1.379298in}}{\pgfqpoint{2.101120in}{1.387111in}}%
\pgfpathcurveto{\pgfqpoint{2.108933in}{1.394925in}}{\pgfqpoint{2.113324in}{1.405524in}}{\pgfqpoint{2.113324in}{1.416574in}}%
\pgfpathcurveto{\pgfqpoint{2.113324in}{1.427624in}}{\pgfqpoint{2.108933in}{1.438223in}}{\pgfqpoint{2.101120in}{1.446037in}}%
\pgfpathcurveto{\pgfqpoint{2.093306in}{1.453850in}}{\pgfqpoint{2.082707in}{1.458241in}}{\pgfqpoint{2.071657in}{1.458241in}}%
\pgfpathcurveto{\pgfqpoint{2.060607in}{1.458241in}}{\pgfqpoint{2.050008in}{1.453850in}}{\pgfqpoint{2.042194in}{1.446037in}}%
\pgfpathcurveto{\pgfqpoint{2.034381in}{1.438223in}}{\pgfqpoint{2.029990in}{1.427624in}}{\pgfqpoint{2.029990in}{1.416574in}}%
\pgfpathcurveto{\pgfqpoint{2.029990in}{1.405524in}}{\pgfqpoint{2.034381in}{1.394925in}}{\pgfqpoint{2.042194in}{1.387111in}}%
\pgfpathcurveto{\pgfqpoint{2.050008in}{1.379298in}}{\pgfqpoint{2.060607in}{1.374907in}}{\pgfqpoint{2.071657in}{1.374907in}}%
\pgfpathclose%
\pgfusepath{stroke,fill}%
\end{pgfscope}%
\begin{pgfscope}%
\pgfpathrectangle{\pgfqpoint{0.481978in}{0.331635in}}{\pgfqpoint{4.960000in}{3.696000in}}%
\pgfusepath{clip}%
\pgfsetbuttcap%
\pgfsetroundjoin%
\definecolor{currentfill}{rgb}{1.000000,0.705882,0.509804}%
\pgfsetfillcolor{currentfill}%
\pgfsetlinewidth{0.481800pt}%
\definecolor{currentstroke}{rgb}{1.000000,1.000000,1.000000}%
\pgfsetstrokecolor{currentstroke}%
\pgfsetdash{}{0pt}%
\pgfpathmoveto{\pgfqpoint{2.329525in}{1.208523in}}%
\pgfpathcurveto{\pgfqpoint{2.340575in}{1.208523in}}{\pgfqpoint{2.351174in}{1.212913in}}{\pgfqpoint{2.358987in}{1.220727in}}%
\pgfpathcurveto{\pgfqpoint{2.366801in}{1.228541in}}{\pgfqpoint{2.371191in}{1.239140in}}{\pgfqpoint{2.371191in}{1.250190in}}%
\pgfpathcurveto{\pgfqpoint{2.371191in}{1.261240in}}{\pgfqpoint{2.366801in}{1.271839in}}{\pgfqpoint{2.358987in}{1.279652in}}%
\pgfpathcurveto{\pgfqpoint{2.351174in}{1.287466in}}{\pgfqpoint{2.340575in}{1.291856in}}{\pgfqpoint{2.329525in}{1.291856in}}%
\pgfpathcurveto{\pgfqpoint{2.318475in}{1.291856in}}{\pgfqpoint{2.307875in}{1.287466in}}{\pgfqpoint{2.300062in}{1.279652in}}%
\pgfpathcurveto{\pgfqpoint{2.292248in}{1.271839in}}{\pgfqpoint{2.287858in}{1.261240in}}{\pgfqpoint{2.287858in}{1.250190in}}%
\pgfpathcurveto{\pgfqpoint{2.287858in}{1.239140in}}{\pgfqpoint{2.292248in}{1.228541in}}{\pgfqpoint{2.300062in}{1.220727in}}%
\pgfpathcurveto{\pgfqpoint{2.307875in}{1.212913in}}{\pgfqpoint{2.318475in}{1.208523in}}{\pgfqpoint{2.329525in}{1.208523in}}%
\pgfpathclose%
\pgfusepath{stroke,fill}%
\end{pgfscope}%
\begin{pgfscope}%
\pgfpathrectangle{\pgfqpoint{0.481978in}{0.331635in}}{\pgfqpoint{4.960000in}{3.696000in}}%
\pgfusepath{clip}%
\pgfsetbuttcap%
\pgfsetroundjoin%
\definecolor{currentfill}{rgb}{1.000000,0.705882,0.509804}%
\pgfsetfillcolor{currentfill}%
\pgfsetlinewidth{0.481800pt}%
\definecolor{currentstroke}{rgb}{1.000000,1.000000,1.000000}%
\pgfsetstrokecolor{currentstroke}%
\pgfsetdash{}{0pt}%
\pgfpathmoveto{\pgfqpoint{1.401436in}{1.119624in}}%
\pgfpathcurveto{\pgfqpoint{1.412486in}{1.119624in}}{\pgfqpoint{1.423085in}{1.124014in}}{\pgfqpoint{1.430899in}{1.131828in}}%
\pgfpathcurveto{\pgfqpoint{1.438712in}{1.139641in}}{\pgfqpoint{1.443102in}{1.150240in}}{\pgfqpoint{1.443102in}{1.161290in}}%
\pgfpathcurveto{\pgfqpoint{1.443102in}{1.172341in}}{\pgfqpoint{1.438712in}{1.182940in}}{\pgfqpoint{1.430899in}{1.190753in}}%
\pgfpathcurveto{\pgfqpoint{1.423085in}{1.198567in}}{\pgfqpoint{1.412486in}{1.202957in}}{\pgfqpoint{1.401436in}{1.202957in}}%
\pgfpathcurveto{\pgfqpoint{1.390386in}{1.202957in}}{\pgfqpoint{1.379787in}{1.198567in}}{\pgfqpoint{1.371973in}{1.190753in}}%
\pgfpathcurveto{\pgfqpoint{1.364159in}{1.182940in}}{\pgfqpoint{1.359769in}{1.172341in}}{\pgfqpoint{1.359769in}{1.161290in}}%
\pgfpathcurveto{\pgfqpoint{1.359769in}{1.150240in}}{\pgfqpoint{1.364159in}{1.139641in}}{\pgfqpoint{1.371973in}{1.131828in}}%
\pgfpathcurveto{\pgfqpoint{1.379787in}{1.124014in}}{\pgfqpoint{1.390386in}{1.119624in}}{\pgfqpoint{1.401436in}{1.119624in}}%
\pgfpathclose%
\pgfusepath{stroke,fill}%
\end{pgfscope}%
\begin{pgfscope}%
\pgfpathrectangle{\pgfqpoint{0.481978in}{0.331635in}}{\pgfqpoint{4.960000in}{3.696000in}}%
\pgfusepath{clip}%
\pgfsetbuttcap%
\pgfsetroundjoin%
\definecolor{currentfill}{rgb}{1.000000,0.705882,0.509804}%
\pgfsetfillcolor{currentfill}%
\pgfsetlinewidth{0.481800pt}%
\definecolor{currentstroke}{rgb}{1.000000,1.000000,1.000000}%
\pgfsetstrokecolor{currentstroke}%
\pgfsetdash{}{0pt}%
\pgfpathmoveto{\pgfqpoint{2.654322in}{1.212058in}}%
\pgfpathcurveto{\pgfqpoint{2.665372in}{1.212058in}}{\pgfqpoint{2.675971in}{1.216448in}}{\pgfqpoint{2.683785in}{1.224262in}}%
\pgfpathcurveto{\pgfqpoint{2.691599in}{1.232076in}}{\pgfqpoint{2.695989in}{1.242675in}}{\pgfqpoint{2.695989in}{1.253725in}}%
\pgfpathcurveto{\pgfqpoint{2.695989in}{1.264775in}}{\pgfqpoint{2.691599in}{1.275374in}}{\pgfqpoint{2.683785in}{1.283188in}}%
\pgfpathcurveto{\pgfqpoint{2.675971in}{1.291001in}}{\pgfqpoint{2.665372in}{1.295391in}}{\pgfqpoint{2.654322in}{1.295391in}}%
\pgfpathcurveto{\pgfqpoint{2.643272in}{1.295391in}}{\pgfqpoint{2.632673in}{1.291001in}}{\pgfqpoint{2.624860in}{1.283188in}}%
\pgfpathcurveto{\pgfqpoint{2.617046in}{1.275374in}}{\pgfqpoint{2.612656in}{1.264775in}}{\pgfqpoint{2.612656in}{1.253725in}}%
\pgfpathcurveto{\pgfqpoint{2.612656in}{1.242675in}}{\pgfqpoint{2.617046in}{1.232076in}}{\pgfqpoint{2.624860in}{1.224262in}}%
\pgfpathcurveto{\pgfqpoint{2.632673in}{1.216448in}}{\pgfqpoint{2.643272in}{1.212058in}}{\pgfqpoint{2.654322in}{1.212058in}}%
\pgfpathclose%
\pgfusepath{stroke,fill}%
\end{pgfscope}%
\begin{pgfscope}%
\pgfpathrectangle{\pgfqpoint{0.481978in}{0.331635in}}{\pgfqpoint{4.960000in}{3.696000in}}%
\pgfusepath{clip}%
\pgfsetbuttcap%
\pgfsetroundjoin%
\definecolor{currentfill}{rgb}{1.000000,0.705882,0.509804}%
\pgfsetfillcolor{currentfill}%
\pgfsetlinewidth{0.481800pt}%
\definecolor{currentstroke}{rgb}{1.000000,1.000000,1.000000}%
\pgfsetstrokecolor{currentstroke}%
\pgfsetdash{}{0pt}%
\pgfpathmoveto{\pgfqpoint{3.220258in}{2.305329in}}%
\pgfpathcurveto{\pgfqpoint{3.231308in}{2.305329in}}{\pgfqpoint{3.241907in}{2.309719in}}{\pgfqpoint{3.249721in}{2.317532in}}%
\pgfpathcurveto{\pgfqpoint{3.257535in}{2.325346in}}{\pgfqpoint{3.261925in}{2.335945in}}{\pgfqpoint{3.261925in}{2.346995in}}%
\pgfpathcurveto{\pgfqpoint{3.261925in}{2.358045in}}{\pgfqpoint{3.257535in}{2.368644in}}{\pgfqpoint{3.249721in}{2.376458in}}%
\pgfpathcurveto{\pgfqpoint{3.241907in}{2.384272in}}{\pgfqpoint{3.231308in}{2.388662in}}{\pgfqpoint{3.220258in}{2.388662in}}%
\pgfpathcurveto{\pgfqpoint{3.209208in}{2.388662in}}{\pgfqpoint{3.198609in}{2.384272in}}{\pgfqpoint{3.190796in}{2.376458in}}%
\pgfpathcurveto{\pgfqpoint{3.182982in}{2.368644in}}{\pgfqpoint{3.178592in}{2.358045in}}{\pgfqpoint{3.178592in}{2.346995in}}%
\pgfpathcurveto{\pgfqpoint{3.178592in}{2.335945in}}{\pgfqpoint{3.182982in}{2.325346in}}{\pgfqpoint{3.190796in}{2.317532in}}%
\pgfpathcurveto{\pgfqpoint{3.198609in}{2.309719in}}{\pgfqpoint{3.209208in}{2.305329in}}{\pgfqpoint{3.220258in}{2.305329in}}%
\pgfpathclose%
\pgfusepath{stroke,fill}%
\end{pgfscope}%
\begin{pgfscope}%
\pgfpathrectangle{\pgfqpoint{0.481978in}{0.331635in}}{\pgfqpoint{4.960000in}{3.696000in}}%
\pgfusepath{clip}%
\pgfsetbuttcap%
\pgfsetroundjoin%
\definecolor{currentfill}{rgb}{1.000000,0.705882,0.509804}%
\pgfsetfillcolor{currentfill}%
\pgfsetlinewidth{0.481800pt}%
\definecolor{currentstroke}{rgb}{1.000000,1.000000,1.000000}%
\pgfsetstrokecolor{currentstroke}%
\pgfsetdash{}{0pt}%
\pgfpathmoveto{\pgfqpoint{3.187385in}{1.467770in}}%
\pgfpathcurveto{\pgfqpoint{3.198435in}{1.467770in}}{\pgfqpoint{3.209034in}{1.472160in}}{\pgfqpoint{3.216848in}{1.479974in}}%
\pgfpathcurveto{\pgfqpoint{3.224661in}{1.487788in}}{\pgfqpoint{3.229051in}{1.498387in}}{\pgfqpoint{3.229051in}{1.509437in}}%
\pgfpathcurveto{\pgfqpoint{3.229051in}{1.520487in}}{\pgfqpoint{3.224661in}{1.531086in}}{\pgfqpoint{3.216848in}{1.538900in}}%
\pgfpathcurveto{\pgfqpoint{3.209034in}{1.546713in}}{\pgfqpoint{3.198435in}{1.551104in}}{\pgfqpoint{3.187385in}{1.551104in}}%
\pgfpathcurveto{\pgfqpoint{3.176335in}{1.551104in}}{\pgfqpoint{3.165736in}{1.546713in}}{\pgfqpoint{3.157922in}{1.538900in}}%
\pgfpathcurveto{\pgfqpoint{3.150108in}{1.531086in}}{\pgfqpoint{3.145718in}{1.520487in}}{\pgfqpoint{3.145718in}{1.509437in}}%
\pgfpathcurveto{\pgfqpoint{3.145718in}{1.498387in}}{\pgfqpoint{3.150108in}{1.487788in}}{\pgfqpoint{3.157922in}{1.479974in}}%
\pgfpathcurveto{\pgfqpoint{3.165736in}{1.472160in}}{\pgfqpoint{3.176335in}{1.467770in}}{\pgfqpoint{3.187385in}{1.467770in}}%
\pgfpathclose%
\pgfusepath{stroke,fill}%
\end{pgfscope}%
\begin{pgfscope}%
\pgfpathrectangle{\pgfqpoint{0.481978in}{0.331635in}}{\pgfqpoint{4.960000in}{3.696000in}}%
\pgfusepath{clip}%
\pgfsetbuttcap%
\pgfsetroundjoin%
\definecolor{currentfill}{rgb}{1.000000,0.705882,0.509804}%
\pgfsetfillcolor{currentfill}%
\pgfsetlinewidth{0.481800pt}%
\definecolor{currentstroke}{rgb}{1.000000,1.000000,1.000000}%
\pgfsetstrokecolor{currentstroke}%
\pgfsetdash{}{0pt}%
\pgfpathmoveto{\pgfqpoint{2.953726in}{1.357274in}}%
\pgfpathcurveto{\pgfqpoint{2.964776in}{1.357274in}}{\pgfqpoint{2.975375in}{1.361664in}}{\pgfqpoint{2.983188in}{1.369478in}}%
\pgfpathcurveto{\pgfqpoint{2.991002in}{1.377292in}}{\pgfqpoint{2.995392in}{1.387891in}}{\pgfqpoint{2.995392in}{1.398941in}}%
\pgfpathcurveto{\pgfqpoint{2.995392in}{1.409991in}}{\pgfqpoint{2.991002in}{1.420590in}}{\pgfqpoint{2.983188in}{1.428404in}}%
\pgfpathcurveto{\pgfqpoint{2.975375in}{1.436217in}}{\pgfqpoint{2.964776in}{1.440608in}}{\pgfqpoint{2.953726in}{1.440608in}}%
\pgfpathcurveto{\pgfqpoint{2.942676in}{1.440608in}}{\pgfqpoint{2.932077in}{1.436217in}}{\pgfqpoint{2.924263in}{1.428404in}}%
\pgfpathcurveto{\pgfqpoint{2.916449in}{1.420590in}}{\pgfqpoint{2.912059in}{1.409991in}}{\pgfqpoint{2.912059in}{1.398941in}}%
\pgfpathcurveto{\pgfqpoint{2.912059in}{1.387891in}}{\pgfqpoint{2.916449in}{1.377292in}}{\pgfqpoint{2.924263in}{1.369478in}}%
\pgfpathcurveto{\pgfqpoint{2.932077in}{1.361664in}}{\pgfqpoint{2.942676in}{1.357274in}}{\pgfqpoint{2.953726in}{1.357274in}}%
\pgfpathclose%
\pgfusepath{stroke,fill}%
\end{pgfscope}%
\begin{pgfscope}%
\pgfpathrectangle{\pgfqpoint{0.481978in}{0.331635in}}{\pgfqpoint{4.960000in}{3.696000in}}%
\pgfusepath{clip}%
\pgfsetbuttcap%
\pgfsetroundjoin%
\definecolor{currentfill}{rgb}{1.000000,0.705882,0.509804}%
\pgfsetfillcolor{currentfill}%
\pgfsetlinewidth{0.481800pt}%
\definecolor{currentstroke}{rgb}{1.000000,1.000000,1.000000}%
\pgfsetstrokecolor{currentstroke}%
\pgfsetdash{}{0pt}%
\pgfpathmoveto{\pgfqpoint{1.511141in}{1.421539in}}%
\pgfpathcurveto{\pgfqpoint{1.522191in}{1.421539in}}{\pgfqpoint{1.532790in}{1.425929in}}{\pgfqpoint{1.540604in}{1.433743in}}%
\pgfpathcurveto{\pgfqpoint{1.548418in}{1.441556in}}{\pgfqpoint{1.552808in}{1.452155in}}{\pgfqpoint{1.552808in}{1.463205in}}%
\pgfpathcurveto{\pgfqpoint{1.552808in}{1.474256in}}{\pgfqpoint{1.548418in}{1.484855in}}{\pgfqpoint{1.540604in}{1.492668in}}%
\pgfpathcurveto{\pgfqpoint{1.532790in}{1.500482in}}{\pgfqpoint{1.522191in}{1.504872in}}{\pgfqpoint{1.511141in}{1.504872in}}%
\pgfpathcurveto{\pgfqpoint{1.500091in}{1.504872in}}{\pgfqpoint{1.489492in}{1.500482in}}{\pgfqpoint{1.481679in}{1.492668in}}%
\pgfpathcurveto{\pgfqpoint{1.473865in}{1.484855in}}{\pgfqpoint{1.469475in}{1.474256in}}{\pgfqpoint{1.469475in}{1.463205in}}%
\pgfpathcurveto{\pgfqpoint{1.469475in}{1.452155in}}{\pgfqpoint{1.473865in}{1.441556in}}{\pgfqpoint{1.481679in}{1.433743in}}%
\pgfpathcurveto{\pgfqpoint{1.489492in}{1.425929in}}{\pgfqpoint{1.500091in}{1.421539in}}{\pgfqpoint{1.511141in}{1.421539in}}%
\pgfpathclose%
\pgfusepath{stroke,fill}%
\end{pgfscope}%
\begin{pgfscope}%
\pgfpathrectangle{\pgfqpoint{0.481978in}{0.331635in}}{\pgfqpoint{4.960000in}{3.696000in}}%
\pgfusepath{clip}%
\pgfsetbuttcap%
\pgfsetroundjoin%
\definecolor{currentfill}{rgb}{0.631373,0.788235,0.956863}%
\pgfsetfillcolor{currentfill}%
\pgfsetlinewidth{0.481800pt}%
\definecolor{currentstroke}{rgb}{1.000000,1.000000,1.000000}%
\pgfsetstrokecolor{currentstroke}%
\pgfsetdash{}{0pt}%
\pgfpathmoveto{\pgfqpoint{3.803556in}{3.429951in}}%
\pgfpathcurveto{\pgfqpoint{3.814606in}{3.429951in}}{\pgfqpoint{3.825205in}{3.434341in}}{\pgfqpoint{3.833018in}{3.442155in}}%
\pgfpathcurveto{\pgfqpoint{3.840832in}{3.449968in}}{\pgfqpoint{3.845222in}{3.460567in}}{\pgfqpoint{3.845222in}{3.471617in}}%
\pgfpathcurveto{\pgfqpoint{3.845222in}{3.482667in}}{\pgfqpoint{3.840832in}{3.493267in}}{\pgfqpoint{3.833018in}{3.501080in}}%
\pgfpathcurveto{\pgfqpoint{3.825205in}{3.508894in}}{\pgfqpoint{3.814606in}{3.513284in}}{\pgfqpoint{3.803556in}{3.513284in}}%
\pgfpathcurveto{\pgfqpoint{3.792505in}{3.513284in}}{\pgfqpoint{3.781906in}{3.508894in}}{\pgfqpoint{3.774093in}{3.501080in}}%
\pgfpathcurveto{\pgfqpoint{3.766279in}{3.493267in}}{\pgfqpoint{3.761889in}{3.482667in}}{\pgfqpoint{3.761889in}{3.471617in}}%
\pgfpathcurveto{\pgfqpoint{3.761889in}{3.460567in}}{\pgfqpoint{3.766279in}{3.449968in}}{\pgfqpoint{3.774093in}{3.442155in}}%
\pgfpathcurveto{\pgfqpoint{3.781906in}{3.434341in}}{\pgfqpoint{3.792505in}{3.429951in}}{\pgfqpoint{3.803556in}{3.429951in}}%
\pgfpathclose%
\pgfusepath{stroke,fill}%
\end{pgfscope}%
\begin{pgfscope}%
\pgfpathrectangle{\pgfqpoint{0.481978in}{0.331635in}}{\pgfqpoint{4.960000in}{3.696000in}}%
\pgfusepath{clip}%
\pgfsetbuttcap%
\pgfsetroundjoin%
\definecolor{currentfill}{rgb}{0.631373,0.788235,0.956863}%
\pgfsetfillcolor{currentfill}%
\pgfsetlinewidth{0.481800pt}%
\definecolor{currentstroke}{rgb}{1.000000,1.000000,1.000000}%
\pgfsetstrokecolor{currentstroke}%
\pgfsetdash{}{0pt}%
\pgfpathmoveto{\pgfqpoint{4.322934in}{3.384214in}}%
\pgfpathcurveto{\pgfqpoint{4.333984in}{3.384214in}}{\pgfqpoint{4.344583in}{3.388604in}}{\pgfqpoint{4.352397in}{3.396418in}}%
\pgfpathcurveto{\pgfqpoint{4.360210in}{3.404231in}}{\pgfqpoint{4.364601in}{3.414830in}}{\pgfqpoint{4.364601in}{3.425880in}}%
\pgfpathcurveto{\pgfqpoint{4.364601in}{3.436930in}}{\pgfqpoint{4.360210in}{3.447530in}}{\pgfqpoint{4.352397in}{3.455343in}}%
\pgfpathcurveto{\pgfqpoint{4.344583in}{3.463157in}}{\pgfqpoint{4.333984in}{3.467547in}}{\pgfqpoint{4.322934in}{3.467547in}}%
\pgfpathcurveto{\pgfqpoint{4.311884in}{3.467547in}}{\pgfqpoint{4.301285in}{3.463157in}}{\pgfqpoint{4.293471in}{3.455343in}}%
\pgfpathcurveto{\pgfqpoint{4.285657in}{3.447530in}}{\pgfqpoint{4.281267in}{3.436930in}}{\pgfqpoint{4.281267in}{3.425880in}}%
\pgfpathcurveto{\pgfqpoint{4.281267in}{3.414830in}}{\pgfqpoint{4.285657in}{3.404231in}}{\pgfqpoint{4.293471in}{3.396418in}}%
\pgfpathcurveto{\pgfqpoint{4.301285in}{3.388604in}}{\pgfqpoint{4.311884in}{3.384214in}}{\pgfqpoint{4.322934in}{3.384214in}}%
\pgfpathclose%
\pgfusepath{stroke,fill}%
\end{pgfscope}%
\begin{pgfscope}%
\pgfpathrectangle{\pgfqpoint{0.481978in}{0.331635in}}{\pgfqpoint{4.960000in}{3.696000in}}%
\pgfusepath{clip}%
\pgfsetbuttcap%
\pgfsetroundjoin%
\definecolor{currentfill}{rgb}{0.631373,0.788235,0.956863}%
\pgfsetfillcolor{currentfill}%
\pgfsetlinewidth{0.481800pt}%
\definecolor{currentstroke}{rgb}{1.000000,1.000000,1.000000}%
\pgfsetstrokecolor{currentstroke}%
\pgfsetdash{}{0pt}%
\pgfpathmoveto{\pgfqpoint{4.275707in}{1.496705in}}%
\pgfpathcurveto{\pgfqpoint{4.286757in}{1.496705in}}{\pgfqpoint{4.297356in}{1.501096in}}{\pgfqpoint{4.305169in}{1.508909in}}%
\pgfpathcurveto{\pgfqpoint{4.312983in}{1.516723in}}{\pgfqpoint{4.317373in}{1.527322in}}{\pgfqpoint{4.317373in}{1.538372in}}%
\pgfpathcurveto{\pgfqpoint{4.317373in}{1.549422in}}{\pgfqpoint{4.312983in}{1.560021in}}{\pgfqpoint{4.305169in}{1.567835in}}%
\pgfpathcurveto{\pgfqpoint{4.297356in}{1.575649in}}{\pgfqpoint{4.286757in}{1.580039in}}{\pgfqpoint{4.275707in}{1.580039in}}%
\pgfpathcurveto{\pgfqpoint{4.264657in}{1.580039in}}{\pgfqpoint{4.254058in}{1.575649in}}{\pgfqpoint{4.246244in}{1.567835in}}%
\pgfpathcurveto{\pgfqpoint{4.238430in}{1.560021in}}{\pgfqpoint{4.234040in}{1.549422in}}{\pgfqpoint{4.234040in}{1.538372in}}%
\pgfpathcurveto{\pgfqpoint{4.234040in}{1.527322in}}{\pgfqpoint{4.238430in}{1.516723in}}{\pgfqpoint{4.246244in}{1.508909in}}%
\pgfpathcurveto{\pgfqpoint{4.254058in}{1.501096in}}{\pgfqpoint{4.264657in}{1.496705in}}{\pgfqpoint{4.275707in}{1.496705in}}%
\pgfpathclose%
\pgfusepath{stroke,fill}%
\end{pgfscope}%
\begin{pgfscope}%
\pgfpathrectangle{\pgfqpoint{0.481978in}{0.331635in}}{\pgfqpoint{4.960000in}{3.696000in}}%
\pgfusepath{clip}%
\pgfsetbuttcap%
\pgfsetroundjoin%
\definecolor{currentfill}{rgb}{0.631373,0.788235,0.956863}%
\pgfsetfillcolor{currentfill}%
\pgfsetlinewidth{0.481800pt}%
\definecolor{currentstroke}{rgb}{1.000000,1.000000,1.000000}%
\pgfsetstrokecolor{currentstroke}%
\pgfsetdash{}{0pt}%
\pgfpathmoveto{\pgfqpoint{2.539406in}{2.195070in}}%
\pgfpathcurveto{\pgfqpoint{2.550456in}{2.195070in}}{\pgfqpoint{2.561055in}{2.199460in}}{\pgfqpoint{2.568869in}{2.207274in}}%
\pgfpathcurveto{\pgfqpoint{2.576682in}{2.215087in}}{\pgfqpoint{2.581073in}{2.225686in}}{\pgfqpoint{2.581073in}{2.236736in}}%
\pgfpathcurveto{\pgfqpoint{2.581073in}{2.247787in}}{\pgfqpoint{2.576682in}{2.258386in}}{\pgfqpoint{2.568869in}{2.266199in}}%
\pgfpathcurveto{\pgfqpoint{2.561055in}{2.274013in}}{\pgfqpoint{2.550456in}{2.278403in}}{\pgfqpoint{2.539406in}{2.278403in}}%
\pgfpathcurveto{\pgfqpoint{2.528356in}{2.278403in}}{\pgfqpoint{2.517757in}{2.274013in}}{\pgfqpoint{2.509943in}{2.266199in}}%
\pgfpathcurveto{\pgfqpoint{2.502130in}{2.258386in}}{\pgfqpoint{2.497739in}{2.247787in}}{\pgfqpoint{2.497739in}{2.236736in}}%
\pgfpathcurveto{\pgfqpoint{2.497739in}{2.225686in}}{\pgfqpoint{2.502130in}{2.215087in}}{\pgfqpoint{2.509943in}{2.207274in}}%
\pgfpathcurveto{\pgfqpoint{2.517757in}{2.199460in}}{\pgfqpoint{2.528356in}{2.195070in}}{\pgfqpoint{2.539406in}{2.195070in}}%
\pgfpathclose%
\pgfusepath{stroke,fill}%
\end{pgfscope}%
\begin{pgfscope}%
\pgfpathrectangle{\pgfqpoint{0.481978in}{0.331635in}}{\pgfqpoint{4.960000in}{3.696000in}}%
\pgfusepath{clip}%
\pgfsetbuttcap%
\pgfsetroundjoin%
\definecolor{currentfill}{rgb}{0.631373,0.788235,0.956863}%
\pgfsetfillcolor{currentfill}%
\pgfsetlinewidth{0.481800pt}%
\definecolor{currentstroke}{rgb}{1.000000,1.000000,1.000000}%
\pgfsetstrokecolor{currentstroke}%
\pgfsetdash{}{0pt}%
\pgfpathmoveto{\pgfqpoint{3.882882in}{1.083587in}}%
\pgfpathcurveto{\pgfqpoint{3.893932in}{1.083587in}}{\pgfqpoint{3.904531in}{1.087978in}}{\pgfqpoint{3.912345in}{1.095791in}}%
\pgfpathcurveto{\pgfqpoint{3.920158in}{1.103605in}}{\pgfqpoint{3.924548in}{1.114204in}}{\pgfqpoint{3.924548in}{1.125254in}}%
\pgfpathcurveto{\pgfqpoint{3.924548in}{1.136304in}}{\pgfqpoint{3.920158in}{1.146903in}}{\pgfqpoint{3.912345in}{1.154717in}}%
\pgfpathcurveto{\pgfqpoint{3.904531in}{1.162531in}}{\pgfqpoint{3.893932in}{1.166921in}}{\pgfqpoint{3.882882in}{1.166921in}}%
\pgfpathcurveto{\pgfqpoint{3.871832in}{1.166921in}}{\pgfqpoint{3.861233in}{1.162531in}}{\pgfqpoint{3.853419in}{1.154717in}}%
\pgfpathcurveto{\pgfqpoint{3.845605in}{1.146903in}}{\pgfqpoint{3.841215in}{1.136304in}}{\pgfqpoint{3.841215in}{1.125254in}}%
\pgfpathcurveto{\pgfqpoint{3.841215in}{1.114204in}}{\pgfqpoint{3.845605in}{1.103605in}}{\pgfqpoint{3.853419in}{1.095791in}}%
\pgfpathcurveto{\pgfqpoint{3.861233in}{1.087978in}}{\pgfqpoint{3.871832in}{1.083587in}}{\pgfqpoint{3.882882in}{1.083587in}}%
\pgfpathclose%
\pgfusepath{stroke,fill}%
\end{pgfscope}%
\begin{pgfscope}%
\pgfpathrectangle{\pgfqpoint{0.481978in}{0.331635in}}{\pgfqpoint{4.960000in}{3.696000in}}%
\pgfusepath{clip}%
\pgfsetbuttcap%
\pgfsetroundjoin%
\definecolor{currentfill}{rgb}{0.631373,0.788235,0.956863}%
\pgfsetfillcolor{currentfill}%
\pgfsetlinewidth{0.481800pt}%
\definecolor{currentstroke}{rgb}{1.000000,1.000000,1.000000}%
\pgfsetstrokecolor{currentstroke}%
\pgfsetdash{}{0pt}%
\pgfpathmoveto{\pgfqpoint{2.491685in}{2.969889in}}%
\pgfpathcurveto{\pgfqpoint{2.502735in}{2.969889in}}{\pgfqpoint{2.513334in}{2.974280in}}{\pgfqpoint{2.521148in}{2.982093in}}%
\pgfpathcurveto{\pgfqpoint{2.528962in}{2.989907in}}{\pgfqpoint{2.533352in}{3.000506in}}{\pgfqpoint{2.533352in}{3.011556in}}%
\pgfpathcurveto{\pgfqpoint{2.533352in}{3.022606in}}{\pgfqpoint{2.528962in}{3.033205in}}{\pgfqpoint{2.521148in}{3.041019in}}%
\pgfpathcurveto{\pgfqpoint{2.513334in}{3.048832in}}{\pgfqpoint{2.502735in}{3.053223in}}{\pgfqpoint{2.491685in}{3.053223in}}%
\pgfpathcurveto{\pgfqpoint{2.480635in}{3.053223in}}{\pgfqpoint{2.470036in}{3.048832in}}{\pgfqpoint{2.462222in}{3.041019in}}%
\pgfpathcurveto{\pgfqpoint{2.454409in}{3.033205in}}{\pgfqpoint{2.450019in}{3.022606in}}{\pgfqpoint{2.450019in}{3.011556in}}%
\pgfpathcurveto{\pgfqpoint{2.450019in}{3.000506in}}{\pgfqpoint{2.454409in}{2.989907in}}{\pgfqpoint{2.462222in}{2.982093in}}%
\pgfpathcurveto{\pgfqpoint{2.470036in}{2.974280in}}{\pgfqpoint{2.480635in}{2.969889in}}{\pgfqpoint{2.491685in}{2.969889in}}%
\pgfpathclose%
\pgfusepath{stroke,fill}%
\end{pgfscope}%
\begin{pgfscope}%
\pgfpathrectangle{\pgfqpoint{0.481978in}{0.331635in}}{\pgfqpoint{4.960000in}{3.696000in}}%
\pgfusepath{clip}%
\pgfsetbuttcap%
\pgfsetroundjoin%
\definecolor{currentfill}{rgb}{0.631373,0.788235,0.956863}%
\pgfsetfillcolor{currentfill}%
\pgfsetlinewidth{0.481800pt}%
\definecolor{currentstroke}{rgb}{1.000000,1.000000,1.000000}%
\pgfsetstrokecolor{currentstroke}%
\pgfsetdash{}{0pt}%
\pgfpathmoveto{\pgfqpoint{4.588483in}{3.068165in}}%
\pgfpathcurveto{\pgfqpoint{4.599533in}{3.068165in}}{\pgfqpoint{4.610133in}{3.072555in}}{\pgfqpoint{4.617946in}{3.080369in}}%
\pgfpathcurveto{\pgfqpoint{4.625760in}{3.088183in}}{\pgfqpoint{4.630150in}{3.098782in}}{\pgfqpoint{4.630150in}{3.109832in}}%
\pgfpathcurveto{\pgfqpoint{4.630150in}{3.120882in}}{\pgfqpoint{4.625760in}{3.131481in}}{\pgfqpoint{4.617946in}{3.139295in}}%
\pgfpathcurveto{\pgfqpoint{4.610133in}{3.147108in}}{\pgfqpoint{4.599533in}{3.151499in}}{\pgfqpoint{4.588483in}{3.151499in}}%
\pgfpathcurveto{\pgfqpoint{4.577433in}{3.151499in}}{\pgfqpoint{4.566834in}{3.147108in}}{\pgfqpoint{4.559021in}{3.139295in}}%
\pgfpathcurveto{\pgfqpoint{4.551207in}{3.131481in}}{\pgfqpoint{4.546817in}{3.120882in}}{\pgfqpoint{4.546817in}{3.109832in}}%
\pgfpathcurveto{\pgfqpoint{4.546817in}{3.098782in}}{\pgfqpoint{4.551207in}{3.088183in}}{\pgfqpoint{4.559021in}{3.080369in}}%
\pgfpathcurveto{\pgfqpoint{4.566834in}{3.072555in}}{\pgfqpoint{4.577433in}{3.068165in}}{\pgfqpoint{4.588483in}{3.068165in}}%
\pgfpathclose%
\pgfusepath{stroke,fill}%
\end{pgfscope}%
\begin{pgfscope}%
\pgfpathrectangle{\pgfqpoint{0.481978in}{0.331635in}}{\pgfqpoint{4.960000in}{3.696000in}}%
\pgfusepath{clip}%
\pgfsetbuttcap%
\pgfsetroundjoin%
\definecolor{currentfill}{rgb}{0.631373,0.788235,0.956863}%
\pgfsetfillcolor{currentfill}%
\pgfsetlinewidth{0.481800pt}%
\definecolor{currentstroke}{rgb}{1.000000,1.000000,1.000000}%
\pgfsetstrokecolor{currentstroke}%
\pgfsetdash{}{0pt}%
\pgfpathmoveto{\pgfqpoint{3.769223in}{3.309870in}}%
\pgfpathcurveto{\pgfqpoint{3.780273in}{3.309870in}}{\pgfqpoint{3.790872in}{3.314261in}}{\pgfqpoint{3.798686in}{3.322074in}}%
\pgfpathcurveto{\pgfqpoint{3.806499in}{3.329888in}}{\pgfqpoint{3.810890in}{3.340487in}}{\pgfqpoint{3.810890in}{3.351537in}}%
\pgfpathcurveto{\pgfqpoint{3.810890in}{3.362587in}}{\pgfqpoint{3.806499in}{3.373186in}}{\pgfqpoint{3.798686in}{3.381000in}}%
\pgfpathcurveto{\pgfqpoint{3.790872in}{3.388813in}}{\pgfqpoint{3.780273in}{3.393204in}}{\pgfqpoint{3.769223in}{3.393204in}}%
\pgfpathcurveto{\pgfqpoint{3.758173in}{3.393204in}}{\pgfqpoint{3.747574in}{3.388813in}}{\pgfqpoint{3.739760in}{3.381000in}}%
\pgfpathcurveto{\pgfqpoint{3.731947in}{3.373186in}}{\pgfqpoint{3.727556in}{3.362587in}}{\pgfqpoint{3.727556in}{3.351537in}}%
\pgfpathcurveto{\pgfqpoint{3.727556in}{3.340487in}}{\pgfqpoint{3.731947in}{3.329888in}}{\pgfqpoint{3.739760in}{3.322074in}}%
\pgfpathcurveto{\pgfqpoint{3.747574in}{3.314261in}}{\pgfqpoint{3.758173in}{3.309870in}}{\pgfqpoint{3.769223in}{3.309870in}}%
\pgfpathclose%
\pgfusepath{stroke,fill}%
\end{pgfscope}%
\begin{pgfscope}%
\pgfpathrectangle{\pgfqpoint{0.481978in}{0.331635in}}{\pgfqpoint{4.960000in}{3.696000in}}%
\pgfusepath{clip}%
\pgfsetbuttcap%
\pgfsetroundjoin%
\definecolor{currentfill}{rgb}{0.631373,0.788235,0.956863}%
\pgfsetfillcolor{currentfill}%
\pgfsetlinewidth{0.481800pt}%
\definecolor{currentstroke}{rgb}{1.000000,1.000000,1.000000}%
\pgfsetstrokecolor{currentstroke}%
\pgfsetdash{}{0pt}%
\pgfpathmoveto{\pgfqpoint{3.771620in}{2.534229in}}%
\pgfpathcurveto{\pgfqpoint{3.782670in}{2.534229in}}{\pgfqpoint{3.793269in}{2.538620in}}{\pgfqpoint{3.801082in}{2.546433in}}%
\pgfpathcurveto{\pgfqpoint{3.808896in}{2.554247in}}{\pgfqpoint{3.813286in}{2.564846in}}{\pgfqpoint{3.813286in}{2.575896in}}%
\pgfpathcurveto{\pgfqpoint{3.813286in}{2.586946in}}{\pgfqpoint{3.808896in}{2.597545in}}{\pgfqpoint{3.801082in}{2.605359in}}%
\pgfpathcurveto{\pgfqpoint{3.793269in}{2.613173in}}{\pgfqpoint{3.782670in}{2.617563in}}{\pgfqpoint{3.771620in}{2.617563in}}%
\pgfpathcurveto{\pgfqpoint{3.760569in}{2.617563in}}{\pgfqpoint{3.749970in}{2.613173in}}{\pgfqpoint{3.742157in}{2.605359in}}%
\pgfpathcurveto{\pgfqpoint{3.734343in}{2.597545in}}{\pgfqpoint{3.729953in}{2.586946in}}{\pgfqpoint{3.729953in}{2.575896in}}%
\pgfpathcurveto{\pgfqpoint{3.729953in}{2.564846in}}{\pgfqpoint{3.734343in}{2.554247in}}{\pgfqpoint{3.742157in}{2.546433in}}%
\pgfpathcurveto{\pgfqpoint{3.749970in}{2.538620in}}{\pgfqpoint{3.760569in}{2.534229in}}{\pgfqpoint{3.771620in}{2.534229in}}%
\pgfpathclose%
\pgfusepath{stroke,fill}%
\end{pgfscope}%
\begin{pgfscope}%
\pgfpathrectangle{\pgfqpoint{0.481978in}{0.331635in}}{\pgfqpoint{4.960000in}{3.696000in}}%
\pgfusepath{clip}%
\pgfsetbuttcap%
\pgfsetroundjoin%
\definecolor{currentfill}{rgb}{0.631373,0.788235,0.956863}%
\pgfsetfillcolor{currentfill}%
\pgfsetlinewidth{0.481800pt}%
\definecolor{currentstroke}{rgb}{1.000000,1.000000,1.000000}%
\pgfsetstrokecolor{currentstroke}%
\pgfsetdash{}{0pt}%
\pgfpathmoveto{\pgfqpoint{2.859335in}{2.774740in}}%
\pgfpathcurveto{\pgfqpoint{2.870385in}{2.774740in}}{\pgfqpoint{2.880984in}{2.779130in}}{\pgfqpoint{2.888798in}{2.786944in}}%
\pgfpathcurveto{\pgfqpoint{2.896611in}{2.794757in}}{\pgfqpoint{2.901002in}{2.805356in}}{\pgfqpoint{2.901002in}{2.816407in}}%
\pgfpathcurveto{\pgfqpoint{2.901002in}{2.827457in}}{\pgfqpoint{2.896611in}{2.838056in}}{\pgfqpoint{2.888798in}{2.845869in}}%
\pgfpathcurveto{\pgfqpoint{2.880984in}{2.853683in}}{\pgfqpoint{2.870385in}{2.858073in}}{\pgfqpoint{2.859335in}{2.858073in}}%
\pgfpathcurveto{\pgfqpoint{2.848285in}{2.858073in}}{\pgfqpoint{2.837686in}{2.853683in}}{\pgfqpoint{2.829872in}{2.845869in}}%
\pgfpathcurveto{\pgfqpoint{2.822059in}{2.838056in}}{\pgfqpoint{2.817668in}{2.827457in}}{\pgfqpoint{2.817668in}{2.816407in}}%
\pgfpathcurveto{\pgfqpoint{2.817668in}{2.805356in}}{\pgfqpoint{2.822059in}{2.794757in}}{\pgfqpoint{2.829872in}{2.786944in}}%
\pgfpathcurveto{\pgfqpoint{2.837686in}{2.779130in}}{\pgfqpoint{2.848285in}{2.774740in}}{\pgfqpoint{2.859335in}{2.774740in}}%
\pgfpathclose%
\pgfusepath{stroke,fill}%
\end{pgfscope}%
\begin{pgfscope}%
\pgfpathrectangle{\pgfqpoint{0.481978in}{0.331635in}}{\pgfqpoint{4.960000in}{3.696000in}}%
\pgfusepath{clip}%
\pgfsetbuttcap%
\pgfsetroundjoin%
\definecolor{currentfill}{rgb}{0.631373,0.788235,0.956863}%
\pgfsetfillcolor{currentfill}%
\pgfsetlinewidth{0.481800pt}%
\definecolor{currentstroke}{rgb}{1.000000,1.000000,1.000000}%
\pgfsetstrokecolor{currentstroke}%
\pgfsetdash{}{0pt}%
\pgfpathmoveto{\pgfqpoint{2.367225in}{1.846658in}}%
\pgfpathcurveto{\pgfqpoint{2.378276in}{1.846658in}}{\pgfqpoint{2.388875in}{1.851048in}}{\pgfqpoint{2.396688in}{1.858862in}}%
\pgfpathcurveto{\pgfqpoint{2.404502in}{1.866675in}}{\pgfqpoint{2.408892in}{1.877274in}}{\pgfqpoint{2.408892in}{1.888324in}}%
\pgfpathcurveto{\pgfqpoint{2.408892in}{1.899375in}}{\pgfqpoint{2.404502in}{1.909974in}}{\pgfqpoint{2.396688in}{1.917787in}}%
\pgfpathcurveto{\pgfqpoint{2.388875in}{1.925601in}}{\pgfqpoint{2.378276in}{1.929991in}}{\pgfqpoint{2.367225in}{1.929991in}}%
\pgfpathcurveto{\pgfqpoint{2.356175in}{1.929991in}}{\pgfqpoint{2.345576in}{1.925601in}}{\pgfqpoint{2.337763in}{1.917787in}}%
\pgfpathcurveto{\pgfqpoint{2.329949in}{1.909974in}}{\pgfqpoint{2.325559in}{1.899375in}}{\pgfqpoint{2.325559in}{1.888324in}}%
\pgfpathcurveto{\pgfqpoint{2.325559in}{1.877274in}}{\pgfqpoint{2.329949in}{1.866675in}}{\pgfqpoint{2.337763in}{1.858862in}}%
\pgfpathcurveto{\pgfqpoint{2.345576in}{1.851048in}}{\pgfqpoint{2.356175in}{1.846658in}}{\pgfqpoint{2.367225in}{1.846658in}}%
\pgfpathclose%
\pgfusepath{stroke,fill}%
\end{pgfscope}%
\begin{pgfscope}%
\pgfpathrectangle{\pgfqpoint{0.481978in}{0.331635in}}{\pgfqpoint{4.960000in}{3.696000in}}%
\pgfusepath{clip}%
\pgfsetbuttcap%
\pgfsetroundjoin%
\definecolor{currentfill}{rgb}{0.631373,0.788235,0.956863}%
\pgfsetfillcolor{currentfill}%
\pgfsetlinewidth{0.481800pt}%
\definecolor{currentstroke}{rgb}{1.000000,1.000000,1.000000}%
\pgfsetstrokecolor{currentstroke}%
\pgfsetdash{}{0pt}%
\pgfpathmoveto{\pgfqpoint{3.918937in}{3.290737in}}%
\pgfpathcurveto{\pgfqpoint{3.929987in}{3.290737in}}{\pgfqpoint{3.940587in}{3.295127in}}{\pgfqpoint{3.948400in}{3.302941in}}%
\pgfpathcurveto{\pgfqpoint{3.956214in}{3.310754in}}{\pgfqpoint{3.960604in}{3.321353in}}{\pgfqpoint{3.960604in}{3.332403in}}%
\pgfpathcurveto{\pgfqpoint{3.960604in}{3.343454in}}{\pgfqpoint{3.956214in}{3.354053in}}{\pgfqpoint{3.948400in}{3.361866in}}%
\pgfpathcurveto{\pgfqpoint{3.940587in}{3.369680in}}{\pgfqpoint{3.929987in}{3.374070in}}{\pgfqpoint{3.918937in}{3.374070in}}%
\pgfpathcurveto{\pgfqpoint{3.907887in}{3.374070in}}{\pgfqpoint{3.897288in}{3.369680in}}{\pgfqpoint{3.889475in}{3.361866in}}%
\pgfpathcurveto{\pgfqpoint{3.881661in}{3.354053in}}{\pgfqpoint{3.877271in}{3.343454in}}{\pgfqpoint{3.877271in}{3.332403in}}%
\pgfpathcurveto{\pgfqpoint{3.877271in}{3.321353in}}{\pgfqpoint{3.881661in}{3.310754in}}{\pgfqpoint{3.889475in}{3.302941in}}%
\pgfpathcurveto{\pgfqpoint{3.897288in}{3.295127in}}{\pgfqpoint{3.907887in}{3.290737in}}{\pgfqpoint{3.918937in}{3.290737in}}%
\pgfpathclose%
\pgfusepath{stroke,fill}%
\end{pgfscope}%
\begin{pgfscope}%
\pgfpathrectangle{\pgfqpoint{0.481978in}{0.331635in}}{\pgfqpoint{4.960000in}{3.696000in}}%
\pgfusepath{clip}%
\pgfsetbuttcap%
\pgfsetroundjoin%
\definecolor{currentfill}{rgb}{0.631373,0.788235,0.956863}%
\pgfsetfillcolor{currentfill}%
\pgfsetlinewidth{0.481800pt}%
\definecolor{currentstroke}{rgb}{1.000000,1.000000,1.000000}%
\pgfsetstrokecolor{currentstroke}%
\pgfsetdash{}{0pt}%
\pgfpathmoveto{\pgfqpoint{2.336229in}{2.915287in}}%
\pgfpathcurveto{\pgfqpoint{2.347279in}{2.915287in}}{\pgfqpoint{2.357878in}{2.919677in}}{\pgfqpoint{2.365692in}{2.927491in}}%
\pgfpathcurveto{\pgfqpoint{2.373505in}{2.935305in}}{\pgfqpoint{2.377896in}{2.945904in}}{\pgfqpoint{2.377896in}{2.956954in}}%
\pgfpathcurveto{\pgfqpoint{2.377896in}{2.968004in}}{\pgfqpoint{2.373505in}{2.978603in}}{\pgfqpoint{2.365692in}{2.986416in}}%
\pgfpathcurveto{\pgfqpoint{2.357878in}{2.994230in}}{\pgfqpoint{2.347279in}{2.998620in}}{\pgfqpoint{2.336229in}{2.998620in}}%
\pgfpathcurveto{\pgfqpoint{2.325179in}{2.998620in}}{\pgfqpoint{2.314580in}{2.994230in}}{\pgfqpoint{2.306766in}{2.986416in}}%
\pgfpathcurveto{\pgfqpoint{2.298952in}{2.978603in}}{\pgfqpoint{2.294562in}{2.968004in}}{\pgfqpoint{2.294562in}{2.956954in}}%
\pgfpathcurveto{\pgfqpoint{2.294562in}{2.945904in}}{\pgfqpoint{2.298952in}{2.935305in}}{\pgfqpoint{2.306766in}{2.927491in}}%
\pgfpathcurveto{\pgfqpoint{2.314580in}{2.919677in}}{\pgfqpoint{2.325179in}{2.915287in}}{\pgfqpoint{2.336229in}{2.915287in}}%
\pgfpathclose%
\pgfusepath{stroke,fill}%
\end{pgfscope}%
\begin{pgfscope}%
\pgfpathrectangle{\pgfqpoint{0.481978in}{0.331635in}}{\pgfqpoint{4.960000in}{3.696000in}}%
\pgfusepath{clip}%
\pgfsetbuttcap%
\pgfsetroundjoin%
\definecolor{currentfill}{rgb}{0.631373,0.788235,0.956863}%
\pgfsetfillcolor{currentfill}%
\pgfsetlinewidth{0.481800pt}%
\definecolor{currentstroke}{rgb}{1.000000,1.000000,1.000000}%
\pgfsetstrokecolor{currentstroke}%
\pgfsetdash{}{0pt}%
\pgfpathmoveto{\pgfqpoint{3.992759in}{2.233710in}}%
\pgfpathcurveto{\pgfqpoint{4.003809in}{2.233710in}}{\pgfqpoint{4.014408in}{2.238100in}}{\pgfqpoint{4.022222in}{2.245914in}}%
\pgfpathcurveto{\pgfqpoint{4.030036in}{2.253728in}}{\pgfqpoint{4.034426in}{2.264327in}}{\pgfqpoint{4.034426in}{2.275377in}}%
\pgfpathcurveto{\pgfqpoint{4.034426in}{2.286427in}}{\pgfqpoint{4.030036in}{2.297026in}}{\pgfqpoint{4.022222in}{2.304840in}}%
\pgfpathcurveto{\pgfqpoint{4.014408in}{2.312653in}}{\pgfqpoint{4.003809in}{2.317043in}}{\pgfqpoint{3.992759in}{2.317043in}}%
\pgfpathcurveto{\pgfqpoint{3.981709in}{2.317043in}}{\pgfqpoint{3.971110in}{2.312653in}}{\pgfqpoint{3.963297in}{2.304840in}}%
\pgfpathcurveto{\pgfqpoint{3.955483in}{2.297026in}}{\pgfqpoint{3.951093in}{2.286427in}}{\pgfqpoint{3.951093in}{2.275377in}}%
\pgfpathcurveto{\pgfqpoint{3.951093in}{2.264327in}}{\pgfqpoint{3.955483in}{2.253728in}}{\pgfqpoint{3.963297in}{2.245914in}}%
\pgfpathcurveto{\pgfqpoint{3.971110in}{2.238100in}}{\pgfqpoint{3.981709in}{2.233710in}}{\pgfqpoint{3.992759in}{2.233710in}}%
\pgfpathclose%
\pgfusepath{stroke,fill}%
\end{pgfscope}%
\begin{pgfscope}%
\pgfpathrectangle{\pgfqpoint{0.481978in}{0.331635in}}{\pgfqpoint{4.960000in}{3.696000in}}%
\pgfusepath{clip}%
\pgfsetbuttcap%
\pgfsetroundjoin%
\definecolor{currentfill}{rgb}{0.631373,0.788235,0.956863}%
\pgfsetfillcolor{currentfill}%
\pgfsetlinewidth{0.481800pt}%
\definecolor{currentstroke}{rgb}{1.000000,1.000000,1.000000}%
\pgfsetstrokecolor{currentstroke}%
\pgfsetdash{}{0pt}%
\pgfpathmoveto{\pgfqpoint{3.957359in}{3.370486in}}%
\pgfpathcurveto{\pgfqpoint{3.968409in}{3.370486in}}{\pgfqpoint{3.979008in}{3.374876in}}{\pgfqpoint{3.986822in}{3.382690in}}%
\pgfpathcurveto{\pgfqpoint{3.994635in}{3.390503in}}{\pgfqpoint{3.999026in}{3.401102in}}{\pgfqpoint{3.999026in}{3.412152in}}%
\pgfpathcurveto{\pgfqpoint{3.999026in}{3.423203in}}{\pgfqpoint{3.994635in}{3.433802in}}{\pgfqpoint{3.986822in}{3.441615in}}%
\pgfpathcurveto{\pgfqpoint{3.979008in}{3.449429in}}{\pgfqpoint{3.968409in}{3.453819in}}{\pgfqpoint{3.957359in}{3.453819in}}%
\pgfpathcurveto{\pgfqpoint{3.946309in}{3.453819in}}{\pgfqpoint{3.935710in}{3.449429in}}{\pgfqpoint{3.927896in}{3.441615in}}%
\pgfpathcurveto{\pgfqpoint{3.920083in}{3.433802in}}{\pgfqpoint{3.915692in}{3.423203in}}{\pgfqpoint{3.915692in}{3.412152in}}%
\pgfpathcurveto{\pgfqpoint{3.915692in}{3.401102in}}{\pgfqpoint{3.920083in}{3.390503in}}{\pgfqpoint{3.927896in}{3.382690in}}%
\pgfpathcurveto{\pgfqpoint{3.935710in}{3.374876in}}{\pgfqpoint{3.946309in}{3.370486in}}{\pgfqpoint{3.957359in}{3.370486in}}%
\pgfpathclose%
\pgfusepath{stroke,fill}%
\end{pgfscope}%
\begin{pgfscope}%
\pgfpathrectangle{\pgfqpoint{0.481978in}{0.331635in}}{\pgfqpoint{4.960000in}{3.696000in}}%
\pgfusepath{clip}%
\pgfsetbuttcap%
\pgfsetroundjoin%
\definecolor{currentfill}{rgb}{0.631373,0.788235,0.956863}%
\pgfsetfillcolor{currentfill}%
\pgfsetlinewidth{0.481800pt}%
\definecolor{currentstroke}{rgb}{1.000000,1.000000,1.000000}%
\pgfsetstrokecolor{currentstroke}%
\pgfsetdash{}{0pt}%
\pgfpathmoveto{\pgfqpoint{3.883880in}{1.750107in}}%
\pgfpathcurveto{\pgfqpoint{3.894930in}{1.750107in}}{\pgfqpoint{3.905529in}{1.754498in}}{\pgfqpoint{3.913343in}{1.762311in}}%
\pgfpathcurveto{\pgfqpoint{3.921157in}{1.770125in}}{\pgfqpoint{3.925547in}{1.780724in}}{\pgfqpoint{3.925547in}{1.791774in}}%
\pgfpathcurveto{\pgfqpoint{3.925547in}{1.802824in}}{\pgfqpoint{3.921157in}{1.813423in}}{\pgfqpoint{3.913343in}{1.821237in}}%
\pgfpathcurveto{\pgfqpoint{3.905529in}{1.829051in}}{\pgfqpoint{3.894930in}{1.833441in}}{\pgfqpoint{3.883880in}{1.833441in}}%
\pgfpathcurveto{\pgfqpoint{3.872830in}{1.833441in}}{\pgfqpoint{3.862231in}{1.829051in}}{\pgfqpoint{3.854417in}{1.821237in}}%
\pgfpathcurveto{\pgfqpoint{3.846604in}{1.813423in}}{\pgfqpoint{3.842214in}{1.802824in}}{\pgfqpoint{3.842214in}{1.791774in}}%
\pgfpathcurveto{\pgfqpoint{3.842214in}{1.780724in}}{\pgfqpoint{3.846604in}{1.770125in}}{\pgfqpoint{3.854417in}{1.762311in}}%
\pgfpathcurveto{\pgfqpoint{3.862231in}{1.754498in}}{\pgfqpoint{3.872830in}{1.750107in}}{\pgfqpoint{3.883880in}{1.750107in}}%
\pgfpathclose%
\pgfusepath{stroke,fill}%
\end{pgfscope}%
\begin{pgfscope}%
\pgfpathrectangle{\pgfqpoint{0.481978in}{0.331635in}}{\pgfqpoint{4.960000in}{3.696000in}}%
\pgfusepath{clip}%
\pgfsetbuttcap%
\pgfsetroundjoin%
\definecolor{currentfill}{rgb}{0.631373,0.788235,0.956863}%
\pgfsetfillcolor{currentfill}%
\pgfsetlinewidth{0.481800pt}%
\definecolor{currentstroke}{rgb}{1.000000,1.000000,1.000000}%
\pgfsetstrokecolor{currentstroke}%
\pgfsetdash{}{0pt}%
\pgfpathmoveto{\pgfqpoint{1.951129in}{1.532050in}}%
\pgfpathcurveto{\pgfqpoint{1.962179in}{1.532050in}}{\pgfqpoint{1.972778in}{1.536440in}}{\pgfqpoint{1.980592in}{1.544254in}}%
\pgfpathcurveto{\pgfqpoint{1.988405in}{1.552067in}}{\pgfqpoint{1.992796in}{1.562666in}}{\pgfqpoint{1.992796in}{1.573716in}}%
\pgfpathcurveto{\pgfqpoint{1.992796in}{1.584767in}}{\pgfqpoint{1.988405in}{1.595366in}}{\pgfqpoint{1.980592in}{1.603179in}}%
\pgfpathcurveto{\pgfqpoint{1.972778in}{1.610993in}}{\pgfqpoint{1.962179in}{1.615383in}}{\pgfqpoint{1.951129in}{1.615383in}}%
\pgfpathcurveto{\pgfqpoint{1.940079in}{1.615383in}}{\pgfqpoint{1.929480in}{1.610993in}}{\pgfqpoint{1.921666in}{1.603179in}}%
\pgfpathcurveto{\pgfqpoint{1.913853in}{1.595366in}}{\pgfqpoint{1.909462in}{1.584767in}}{\pgfqpoint{1.909462in}{1.573716in}}%
\pgfpathcurveto{\pgfqpoint{1.909462in}{1.562666in}}{\pgfqpoint{1.913853in}{1.552067in}}{\pgfqpoint{1.921666in}{1.544254in}}%
\pgfpathcurveto{\pgfqpoint{1.929480in}{1.536440in}}{\pgfqpoint{1.940079in}{1.532050in}}{\pgfqpoint{1.951129in}{1.532050in}}%
\pgfpathclose%
\pgfusepath{stroke,fill}%
\end{pgfscope}%
\begin{pgfscope}%
\pgfpathrectangle{\pgfqpoint{0.481978in}{0.331635in}}{\pgfqpoint{4.960000in}{3.696000in}}%
\pgfusepath{clip}%
\pgfsetbuttcap%
\pgfsetroundjoin%
\definecolor{currentfill}{rgb}{0.631373,0.788235,0.956863}%
\pgfsetfillcolor{currentfill}%
\pgfsetlinewidth{0.481800pt}%
\definecolor{currentstroke}{rgb}{1.000000,1.000000,1.000000}%
\pgfsetstrokecolor{currentstroke}%
\pgfsetdash{}{0pt}%
\pgfpathmoveto{\pgfqpoint{4.682335in}{1.475520in}}%
\pgfpathcurveto{\pgfqpoint{4.693385in}{1.475520in}}{\pgfqpoint{4.703984in}{1.479910in}}{\pgfqpoint{4.711798in}{1.487724in}}%
\pgfpathcurveto{\pgfqpoint{4.719611in}{1.495538in}}{\pgfqpoint{4.724002in}{1.506137in}}{\pgfqpoint{4.724002in}{1.517187in}}%
\pgfpathcurveto{\pgfqpoint{4.724002in}{1.528237in}}{\pgfqpoint{4.719611in}{1.538836in}}{\pgfqpoint{4.711798in}{1.546650in}}%
\pgfpathcurveto{\pgfqpoint{4.703984in}{1.554463in}}{\pgfqpoint{4.693385in}{1.558854in}}{\pgfqpoint{4.682335in}{1.558854in}}%
\pgfpathcurveto{\pgfqpoint{4.671285in}{1.558854in}}{\pgfqpoint{4.660686in}{1.554463in}}{\pgfqpoint{4.652872in}{1.546650in}}%
\pgfpathcurveto{\pgfqpoint{4.645059in}{1.538836in}}{\pgfqpoint{4.640668in}{1.528237in}}{\pgfqpoint{4.640668in}{1.517187in}}%
\pgfpathcurveto{\pgfqpoint{4.640668in}{1.506137in}}{\pgfqpoint{4.645059in}{1.495538in}}{\pgfqpoint{4.652872in}{1.487724in}}%
\pgfpathcurveto{\pgfqpoint{4.660686in}{1.479910in}}{\pgfqpoint{4.671285in}{1.475520in}}{\pgfqpoint{4.682335in}{1.475520in}}%
\pgfpathclose%
\pgfusepath{stroke,fill}%
\end{pgfscope}%
\begin{pgfscope}%
\pgfpathrectangle{\pgfqpoint{0.481978in}{0.331635in}}{\pgfqpoint{4.960000in}{3.696000in}}%
\pgfusepath{clip}%
\pgfsetbuttcap%
\pgfsetroundjoin%
\definecolor{currentfill}{rgb}{0.631373,0.788235,0.956863}%
\pgfsetfillcolor{currentfill}%
\pgfsetlinewidth{0.481800pt}%
\definecolor{currentstroke}{rgb}{1.000000,1.000000,1.000000}%
\pgfsetstrokecolor{currentstroke}%
\pgfsetdash{}{0pt}%
\pgfpathmoveto{\pgfqpoint{2.799814in}{3.505099in}}%
\pgfpathcurveto{\pgfqpoint{2.810864in}{3.505099in}}{\pgfqpoint{2.821463in}{3.509489in}}{\pgfqpoint{2.829277in}{3.517302in}}%
\pgfpathcurveto{\pgfqpoint{2.837090in}{3.525116in}}{\pgfqpoint{2.841481in}{3.535715in}}{\pgfqpoint{2.841481in}{3.546765in}}%
\pgfpathcurveto{\pgfqpoint{2.841481in}{3.557815in}}{\pgfqpoint{2.837090in}{3.568414in}}{\pgfqpoint{2.829277in}{3.576228in}}%
\pgfpathcurveto{\pgfqpoint{2.821463in}{3.584042in}}{\pgfqpoint{2.810864in}{3.588432in}}{\pgfqpoint{2.799814in}{3.588432in}}%
\pgfpathcurveto{\pgfqpoint{2.788764in}{3.588432in}}{\pgfqpoint{2.778165in}{3.584042in}}{\pgfqpoint{2.770351in}{3.576228in}}%
\pgfpathcurveto{\pgfqpoint{2.762538in}{3.568414in}}{\pgfqpoint{2.758147in}{3.557815in}}{\pgfqpoint{2.758147in}{3.546765in}}%
\pgfpathcurveto{\pgfqpoint{2.758147in}{3.535715in}}{\pgfqpoint{2.762538in}{3.525116in}}{\pgfqpoint{2.770351in}{3.517302in}}%
\pgfpathcurveto{\pgfqpoint{2.778165in}{3.509489in}}{\pgfqpoint{2.788764in}{3.505099in}}{\pgfqpoint{2.799814in}{3.505099in}}%
\pgfpathclose%
\pgfusepath{stroke,fill}%
\end{pgfscope}%
\begin{pgfscope}%
\pgfpathrectangle{\pgfqpoint{0.481978in}{0.331635in}}{\pgfqpoint{4.960000in}{3.696000in}}%
\pgfusepath{clip}%
\pgfsetbuttcap%
\pgfsetroundjoin%
\definecolor{currentfill}{rgb}{0.631373,0.788235,0.956863}%
\pgfsetfillcolor{currentfill}%
\pgfsetlinewidth{0.481800pt}%
\definecolor{currentstroke}{rgb}{1.000000,1.000000,1.000000}%
\pgfsetstrokecolor{currentstroke}%
\pgfsetdash{}{0pt}%
\pgfpathmoveto{\pgfqpoint{3.682647in}{2.958816in}}%
\pgfpathcurveto{\pgfqpoint{3.693697in}{2.958816in}}{\pgfqpoint{3.704296in}{2.963206in}}{\pgfqpoint{3.712110in}{2.971020in}}%
\pgfpathcurveto{\pgfqpoint{3.719923in}{2.978834in}}{\pgfqpoint{3.724313in}{2.989433in}}{\pgfqpoint{3.724313in}{3.000483in}}%
\pgfpathcurveto{\pgfqpoint{3.724313in}{3.011533in}}{\pgfqpoint{3.719923in}{3.022132in}}{\pgfqpoint{3.712110in}{3.029946in}}%
\pgfpathcurveto{\pgfqpoint{3.704296in}{3.037759in}}{\pgfqpoint{3.693697in}{3.042149in}}{\pgfqpoint{3.682647in}{3.042149in}}%
\pgfpathcurveto{\pgfqpoint{3.671597in}{3.042149in}}{\pgfqpoint{3.660998in}{3.037759in}}{\pgfqpoint{3.653184in}{3.029946in}}%
\pgfpathcurveto{\pgfqpoint{3.645370in}{3.022132in}}{\pgfqpoint{3.640980in}{3.011533in}}{\pgfqpoint{3.640980in}{3.000483in}}%
\pgfpathcurveto{\pgfqpoint{3.640980in}{2.989433in}}{\pgfqpoint{3.645370in}{2.978834in}}{\pgfqpoint{3.653184in}{2.971020in}}%
\pgfpathcurveto{\pgfqpoint{3.660998in}{2.963206in}}{\pgfqpoint{3.671597in}{2.958816in}}{\pgfqpoint{3.682647in}{2.958816in}}%
\pgfpathclose%
\pgfusepath{stroke,fill}%
\end{pgfscope}%
\begin{pgfscope}%
\pgfpathrectangle{\pgfqpoint{0.481978in}{0.331635in}}{\pgfqpoint{4.960000in}{3.696000in}}%
\pgfusepath{clip}%
\pgfsetbuttcap%
\pgfsetroundjoin%
\definecolor{currentfill}{rgb}{0.631373,0.788235,0.956863}%
\pgfsetfillcolor{currentfill}%
\pgfsetlinewidth{0.481800pt}%
\definecolor{currentstroke}{rgb}{1.000000,1.000000,1.000000}%
\pgfsetstrokecolor{currentstroke}%
\pgfsetdash{}{0pt}%
\pgfpathmoveto{\pgfqpoint{3.082225in}{2.104850in}}%
\pgfpathcurveto{\pgfqpoint{3.093275in}{2.104850in}}{\pgfqpoint{3.103874in}{2.109240in}}{\pgfqpoint{3.111688in}{2.117053in}}%
\pgfpathcurveto{\pgfqpoint{3.119501in}{2.124867in}}{\pgfqpoint{3.123892in}{2.135466in}}{\pgfqpoint{3.123892in}{2.146516in}}%
\pgfpathcurveto{\pgfqpoint{3.123892in}{2.157566in}}{\pgfqpoint{3.119501in}{2.168165in}}{\pgfqpoint{3.111688in}{2.175979in}}%
\pgfpathcurveto{\pgfqpoint{3.103874in}{2.183793in}}{\pgfqpoint{3.093275in}{2.188183in}}{\pgfqpoint{3.082225in}{2.188183in}}%
\pgfpathcurveto{\pgfqpoint{3.071175in}{2.188183in}}{\pgfqpoint{3.060576in}{2.183793in}}{\pgfqpoint{3.052762in}{2.175979in}}%
\pgfpathcurveto{\pgfqpoint{3.044949in}{2.168165in}}{\pgfqpoint{3.040558in}{2.157566in}}{\pgfqpoint{3.040558in}{2.146516in}}%
\pgfpathcurveto{\pgfqpoint{3.040558in}{2.135466in}}{\pgfqpoint{3.044949in}{2.124867in}}{\pgfqpoint{3.052762in}{2.117053in}}%
\pgfpathcurveto{\pgfqpoint{3.060576in}{2.109240in}}{\pgfqpoint{3.071175in}{2.104850in}}{\pgfqpoint{3.082225in}{2.104850in}}%
\pgfpathclose%
\pgfusepath{stroke,fill}%
\end{pgfscope}%
\begin{pgfscope}%
\pgfpathrectangle{\pgfqpoint{0.481978in}{0.331635in}}{\pgfqpoint{4.960000in}{3.696000in}}%
\pgfusepath{clip}%
\pgfsetbuttcap%
\pgfsetroundjoin%
\definecolor{currentfill}{rgb}{0.631373,0.788235,0.956863}%
\pgfsetfillcolor{currentfill}%
\pgfsetlinewidth{0.481800pt}%
\definecolor{currentstroke}{rgb}{1.000000,1.000000,1.000000}%
\pgfsetstrokecolor{currentstroke}%
\pgfsetdash{}{0pt}%
\pgfpathmoveto{\pgfqpoint{4.103401in}{3.449560in}}%
\pgfpathcurveto{\pgfqpoint{4.114451in}{3.449560in}}{\pgfqpoint{4.125050in}{3.453950in}}{\pgfqpoint{4.132863in}{3.461763in}}%
\pgfpathcurveto{\pgfqpoint{4.140677in}{3.469577in}}{\pgfqpoint{4.145067in}{3.480176in}}{\pgfqpoint{4.145067in}{3.491226in}}%
\pgfpathcurveto{\pgfqpoint{4.145067in}{3.502276in}}{\pgfqpoint{4.140677in}{3.512875in}}{\pgfqpoint{4.132863in}{3.520689in}}%
\pgfpathcurveto{\pgfqpoint{4.125050in}{3.528503in}}{\pgfqpoint{4.114451in}{3.532893in}}{\pgfqpoint{4.103401in}{3.532893in}}%
\pgfpathcurveto{\pgfqpoint{4.092350in}{3.532893in}}{\pgfqpoint{4.081751in}{3.528503in}}{\pgfqpoint{4.073938in}{3.520689in}}%
\pgfpathcurveto{\pgfqpoint{4.066124in}{3.512875in}}{\pgfqpoint{4.061734in}{3.502276in}}{\pgfqpoint{4.061734in}{3.491226in}}%
\pgfpathcurveto{\pgfqpoint{4.061734in}{3.480176in}}{\pgfqpoint{4.066124in}{3.469577in}}{\pgfqpoint{4.073938in}{3.461763in}}%
\pgfpathcurveto{\pgfqpoint{4.081751in}{3.453950in}}{\pgfqpoint{4.092350in}{3.449560in}}{\pgfqpoint{4.103401in}{3.449560in}}%
\pgfpathclose%
\pgfusepath{stroke,fill}%
\end{pgfscope}%
\begin{pgfscope}%
\pgfpathrectangle{\pgfqpoint{0.481978in}{0.331635in}}{\pgfqpoint{4.960000in}{3.696000in}}%
\pgfusepath{clip}%
\pgfsetbuttcap%
\pgfsetroundjoin%
\definecolor{currentfill}{rgb}{0.631373,0.788235,0.956863}%
\pgfsetfillcolor{currentfill}%
\pgfsetlinewidth{0.481800pt}%
\definecolor{currentstroke}{rgb}{1.000000,1.000000,1.000000}%
\pgfsetstrokecolor{currentstroke}%
\pgfsetdash{}{0pt}%
\pgfpathmoveto{\pgfqpoint{2.742987in}{3.037917in}}%
\pgfpathcurveto{\pgfqpoint{2.754037in}{3.037917in}}{\pgfqpoint{2.764636in}{3.042307in}}{\pgfqpoint{2.772450in}{3.050121in}}%
\pgfpathcurveto{\pgfqpoint{2.780264in}{3.057935in}}{\pgfqpoint{2.784654in}{3.068534in}}{\pgfqpoint{2.784654in}{3.079584in}}%
\pgfpathcurveto{\pgfqpoint{2.784654in}{3.090634in}}{\pgfqpoint{2.780264in}{3.101233in}}{\pgfqpoint{2.772450in}{3.109047in}}%
\pgfpathcurveto{\pgfqpoint{2.764636in}{3.116860in}}{\pgfqpoint{2.754037in}{3.121250in}}{\pgfqpoint{2.742987in}{3.121250in}}%
\pgfpathcurveto{\pgfqpoint{2.731937in}{3.121250in}}{\pgfqpoint{2.721338in}{3.116860in}}{\pgfqpoint{2.713524in}{3.109047in}}%
\pgfpathcurveto{\pgfqpoint{2.705711in}{3.101233in}}{\pgfqpoint{2.701320in}{3.090634in}}{\pgfqpoint{2.701320in}{3.079584in}}%
\pgfpathcurveto{\pgfqpoint{2.701320in}{3.068534in}}{\pgfqpoint{2.705711in}{3.057935in}}{\pgfqpoint{2.713524in}{3.050121in}}%
\pgfpathcurveto{\pgfqpoint{2.721338in}{3.042307in}}{\pgfqpoint{2.731937in}{3.037917in}}{\pgfqpoint{2.742987in}{3.037917in}}%
\pgfpathclose%
\pgfusepath{stroke,fill}%
\end{pgfscope}%
\begin{pgfscope}%
\pgfpathrectangle{\pgfqpoint{0.481978in}{0.331635in}}{\pgfqpoint{4.960000in}{3.696000in}}%
\pgfusepath{clip}%
\pgfsetbuttcap%
\pgfsetroundjoin%
\definecolor{currentfill}{rgb}{0.631373,0.788235,0.956863}%
\pgfsetfillcolor{currentfill}%
\pgfsetlinewidth{0.481800pt}%
\definecolor{currentstroke}{rgb}{1.000000,1.000000,1.000000}%
\pgfsetstrokecolor{currentstroke}%
\pgfsetdash{}{0pt}%
\pgfpathmoveto{\pgfqpoint{5.163838in}{1.927348in}}%
\pgfpathcurveto{\pgfqpoint{5.174888in}{1.927348in}}{\pgfqpoint{5.185487in}{1.931739in}}{\pgfqpoint{5.193301in}{1.939552in}}%
\pgfpathcurveto{\pgfqpoint{5.201114in}{1.947366in}}{\pgfqpoint{5.205505in}{1.957965in}}{\pgfqpoint{5.205505in}{1.969015in}}%
\pgfpathcurveto{\pgfqpoint{5.205505in}{1.980065in}}{\pgfqpoint{5.201114in}{1.990664in}}{\pgfqpoint{5.193301in}{1.998478in}}%
\pgfpathcurveto{\pgfqpoint{5.185487in}{2.006291in}}{\pgfqpoint{5.174888in}{2.010682in}}{\pgfqpoint{5.163838in}{2.010682in}}%
\pgfpathcurveto{\pgfqpoint{5.152788in}{2.010682in}}{\pgfqpoint{5.142189in}{2.006291in}}{\pgfqpoint{5.134375in}{1.998478in}}%
\pgfpathcurveto{\pgfqpoint{5.126562in}{1.990664in}}{\pgfqpoint{5.122171in}{1.980065in}}{\pgfqpoint{5.122171in}{1.969015in}}%
\pgfpathcurveto{\pgfqpoint{5.122171in}{1.957965in}}{\pgfqpoint{5.126562in}{1.947366in}}{\pgfqpoint{5.134375in}{1.939552in}}%
\pgfpathcurveto{\pgfqpoint{5.142189in}{1.931739in}}{\pgfqpoint{5.152788in}{1.927348in}}{\pgfqpoint{5.163838in}{1.927348in}}%
\pgfpathclose%
\pgfusepath{stroke,fill}%
\end{pgfscope}%
\begin{pgfscope}%
\pgfpathrectangle{\pgfqpoint{0.481978in}{0.331635in}}{\pgfqpoint{4.960000in}{3.696000in}}%
\pgfusepath{clip}%
\pgfsetbuttcap%
\pgfsetroundjoin%
\definecolor{currentfill}{rgb}{0.631373,0.788235,0.956863}%
\pgfsetfillcolor{currentfill}%
\pgfsetlinewidth{0.481800pt}%
\definecolor{currentstroke}{rgb}{1.000000,1.000000,1.000000}%
\pgfsetstrokecolor{currentstroke}%
\pgfsetdash{}{0pt}%
\pgfpathmoveto{\pgfqpoint{4.188026in}{1.904931in}}%
\pgfpathcurveto{\pgfqpoint{4.199076in}{1.904931in}}{\pgfqpoint{4.209675in}{1.909322in}}{\pgfqpoint{4.217489in}{1.917135in}}%
\pgfpathcurveto{\pgfqpoint{4.225302in}{1.924949in}}{\pgfqpoint{4.229693in}{1.935548in}}{\pgfqpoint{4.229693in}{1.946598in}}%
\pgfpathcurveto{\pgfqpoint{4.229693in}{1.957648in}}{\pgfqpoint{4.225302in}{1.968247in}}{\pgfqpoint{4.217489in}{1.976061in}}%
\pgfpathcurveto{\pgfqpoint{4.209675in}{1.983874in}}{\pgfqpoint{4.199076in}{1.988265in}}{\pgfqpoint{4.188026in}{1.988265in}}%
\pgfpathcurveto{\pgfqpoint{4.176976in}{1.988265in}}{\pgfqpoint{4.166377in}{1.983874in}}{\pgfqpoint{4.158563in}{1.976061in}}%
\pgfpathcurveto{\pgfqpoint{4.150750in}{1.968247in}}{\pgfqpoint{4.146359in}{1.957648in}}{\pgfqpoint{4.146359in}{1.946598in}}%
\pgfpathcurveto{\pgfqpoint{4.146359in}{1.935548in}}{\pgfqpoint{4.150750in}{1.924949in}}{\pgfqpoint{4.158563in}{1.917135in}}%
\pgfpathcurveto{\pgfqpoint{4.166377in}{1.909322in}}{\pgfqpoint{4.176976in}{1.904931in}}{\pgfqpoint{4.188026in}{1.904931in}}%
\pgfpathclose%
\pgfusepath{stroke,fill}%
\end{pgfscope}%
\begin{pgfscope}%
\pgfpathrectangle{\pgfqpoint{0.481978in}{0.331635in}}{\pgfqpoint{4.960000in}{3.696000in}}%
\pgfusepath{clip}%
\pgfsetbuttcap%
\pgfsetroundjoin%
\definecolor{currentfill}{rgb}{0.631373,0.788235,0.956863}%
\pgfsetfillcolor{currentfill}%
\pgfsetlinewidth{0.481800pt}%
\definecolor{currentstroke}{rgb}{1.000000,1.000000,1.000000}%
\pgfsetstrokecolor{currentstroke}%
\pgfsetdash{}{0pt}%
\pgfpathmoveto{\pgfqpoint{2.239764in}{2.970570in}}%
\pgfpathcurveto{\pgfqpoint{2.250815in}{2.970570in}}{\pgfqpoint{2.261414in}{2.974960in}}{\pgfqpoint{2.269227in}{2.982774in}}%
\pgfpathcurveto{\pgfqpoint{2.277041in}{2.990588in}}{\pgfqpoint{2.281431in}{3.001187in}}{\pgfqpoint{2.281431in}{3.012237in}}%
\pgfpathcurveto{\pgfqpoint{2.281431in}{3.023287in}}{\pgfqpoint{2.277041in}{3.033886in}}{\pgfqpoint{2.269227in}{3.041700in}}%
\pgfpathcurveto{\pgfqpoint{2.261414in}{3.049513in}}{\pgfqpoint{2.250815in}{3.053904in}}{\pgfqpoint{2.239764in}{3.053904in}}%
\pgfpathcurveto{\pgfqpoint{2.228714in}{3.053904in}}{\pgfqpoint{2.218115in}{3.049513in}}{\pgfqpoint{2.210302in}{3.041700in}}%
\pgfpathcurveto{\pgfqpoint{2.202488in}{3.033886in}}{\pgfqpoint{2.198098in}{3.023287in}}{\pgfqpoint{2.198098in}{3.012237in}}%
\pgfpathcurveto{\pgfqpoint{2.198098in}{3.001187in}}{\pgfqpoint{2.202488in}{2.990588in}}{\pgfqpoint{2.210302in}{2.982774in}}%
\pgfpathcurveto{\pgfqpoint{2.218115in}{2.974960in}}{\pgfqpoint{2.228714in}{2.970570in}}{\pgfqpoint{2.239764in}{2.970570in}}%
\pgfpathclose%
\pgfusepath{stroke,fill}%
\end{pgfscope}%
\begin{pgfscope}%
\pgfpathrectangle{\pgfqpoint{0.481978in}{0.331635in}}{\pgfqpoint{4.960000in}{3.696000in}}%
\pgfusepath{clip}%
\pgfsetbuttcap%
\pgfsetroundjoin%
\definecolor{currentfill}{rgb}{0.631373,0.788235,0.956863}%
\pgfsetfillcolor{currentfill}%
\pgfsetlinewidth{0.481800pt}%
\definecolor{currentstroke}{rgb}{1.000000,1.000000,1.000000}%
\pgfsetstrokecolor{currentstroke}%
\pgfsetdash{}{0pt}%
\pgfpathmoveto{\pgfqpoint{4.178036in}{2.675409in}}%
\pgfpathcurveto{\pgfqpoint{4.189086in}{2.675409in}}{\pgfqpoint{4.199685in}{2.679799in}}{\pgfqpoint{4.207499in}{2.687613in}}%
\pgfpathcurveto{\pgfqpoint{4.215312in}{2.695427in}}{\pgfqpoint{4.219703in}{2.706026in}}{\pgfqpoint{4.219703in}{2.717076in}}%
\pgfpathcurveto{\pgfqpoint{4.219703in}{2.728126in}}{\pgfqpoint{4.215312in}{2.738725in}}{\pgfqpoint{4.207499in}{2.746538in}}%
\pgfpathcurveto{\pgfqpoint{4.199685in}{2.754352in}}{\pgfqpoint{4.189086in}{2.758742in}}{\pgfqpoint{4.178036in}{2.758742in}}%
\pgfpathcurveto{\pgfqpoint{4.166986in}{2.758742in}}{\pgfqpoint{4.156387in}{2.754352in}}{\pgfqpoint{4.148573in}{2.746538in}}%
\pgfpathcurveto{\pgfqpoint{4.140760in}{2.738725in}}{\pgfqpoint{4.136369in}{2.728126in}}{\pgfqpoint{4.136369in}{2.717076in}}%
\pgfpathcurveto{\pgfqpoint{4.136369in}{2.706026in}}{\pgfqpoint{4.140760in}{2.695427in}}{\pgfqpoint{4.148573in}{2.687613in}}%
\pgfpathcurveto{\pgfqpoint{4.156387in}{2.679799in}}{\pgfqpoint{4.166986in}{2.675409in}}{\pgfqpoint{4.178036in}{2.675409in}}%
\pgfpathclose%
\pgfusepath{stroke,fill}%
\end{pgfscope}%
\begin{pgfscope}%
\pgfpathrectangle{\pgfqpoint{0.481978in}{0.331635in}}{\pgfqpoint{4.960000in}{3.696000in}}%
\pgfusepath{clip}%
\pgfsetbuttcap%
\pgfsetroundjoin%
\definecolor{currentfill}{rgb}{0.631373,0.788235,0.956863}%
\pgfsetfillcolor{currentfill}%
\pgfsetlinewidth{0.481800pt}%
\definecolor{currentstroke}{rgb}{1.000000,1.000000,1.000000}%
\pgfsetstrokecolor{currentstroke}%
\pgfsetdash{}{0pt}%
\pgfpathmoveto{\pgfqpoint{1.988321in}{2.938239in}}%
\pgfpathcurveto{\pgfqpoint{1.999371in}{2.938239in}}{\pgfqpoint{2.009970in}{2.942629in}}{\pgfqpoint{2.017784in}{2.950443in}}%
\pgfpathcurveto{\pgfqpoint{2.025597in}{2.958257in}}{\pgfqpoint{2.029988in}{2.968856in}}{\pgfqpoint{2.029988in}{2.979906in}}%
\pgfpathcurveto{\pgfqpoint{2.029988in}{2.990956in}}{\pgfqpoint{2.025597in}{3.001555in}}{\pgfqpoint{2.017784in}{3.009368in}}%
\pgfpathcurveto{\pgfqpoint{2.009970in}{3.017182in}}{\pgfqpoint{1.999371in}{3.021572in}}{\pgfqpoint{1.988321in}{3.021572in}}%
\pgfpathcurveto{\pgfqpoint{1.977271in}{3.021572in}}{\pgfqpoint{1.966672in}{3.017182in}}{\pgfqpoint{1.958858in}{3.009368in}}%
\pgfpathcurveto{\pgfqpoint{1.951044in}{3.001555in}}{\pgfqpoint{1.946654in}{2.990956in}}{\pgfqpoint{1.946654in}{2.979906in}}%
\pgfpathcurveto{\pgfqpoint{1.946654in}{2.968856in}}{\pgfqpoint{1.951044in}{2.958257in}}{\pgfqpoint{1.958858in}{2.950443in}}%
\pgfpathcurveto{\pgfqpoint{1.966672in}{2.942629in}}{\pgfqpoint{1.977271in}{2.938239in}}{\pgfqpoint{1.988321in}{2.938239in}}%
\pgfpathclose%
\pgfusepath{stroke,fill}%
\end{pgfscope}%
\begin{pgfscope}%
\pgfpathrectangle{\pgfqpoint{0.481978in}{0.331635in}}{\pgfqpoint{4.960000in}{3.696000in}}%
\pgfusepath{clip}%
\pgfsetbuttcap%
\pgfsetroundjoin%
\definecolor{currentfill}{rgb}{0.631373,0.788235,0.956863}%
\pgfsetfillcolor{currentfill}%
\pgfsetlinewidth{0.481800pt}%
\definecolor{currentstroke}{rgb}{1.000000,1.000000,1.000000}%
\pgfsetstrokecolor{currentstroke}%
\pgfsetdash{}{0pt}%
\pgfpathmoveto{\pgfqpoint{3.750673in}{3.133666in}}%
\pgfpathcurveto{\pgfqpoint{3.761723in}{3.133666in}}{\pgfqpoint{3.772322in}{3.138056in}}{\pgfqpoint{3.780135in}{3.145870in}}%
\pgfpathcurveto{\pgfqpoint{3.787949in}{3.153684in}}{\pgfqpoint{3.792339in}{3.164283in}}{\pgfqpoint{3.792339in}{3.175333in}}%
\pgfpathcurveto{\pgfqpoint{3.792339in}{3.186383in}}{\pgfqpoint{3.787949in}{3.196982in}}{\pgfqpoint{3.780135in}{3.204796in}}%
\pgfpathcurveto{\pgfqpoint{3.772322in}{3.212609in}}{\pgfqpoint{3.761723in}{3.216999in}}{\pgfqpoint{3.750673in}{3.216999in}}%
\pgfpathcurveto{\pgfqpoint{3.739623in}{3.216999in}}{\pgfqpoint{3.729023in}{3.212609in}}{\pgfqpoint{3.721210in}{3.204796in}}%
\pgfpathcurveto{\pgfqpoint{3.713396in}{3.196982in}}{\pgfqpoint{3.709006in}{3.186383in}}{\pgfqpoint{3.709006in}{3.175333in}}%
\pgfpathcurveto{\pgfqpoint{3.709006in}{3.164283in}}{\pgfqpoint{3.713396in}{3.153684in}}{\pgfqpoint{3.721210in}{3.145870in}}%
\pgfpathcurveto{\pgfqpoint{3.729023in}{3.138056in}}{\pgfqpoint{3.739623in}{3.133666in}}{\pgfqpoint{3.750673in}{3.133666in}}%
\pgfpathclose%
\pgfusepath{stroke,fill}%
\end{pgfscope}%
\begin{pgfscope}%
\pgfpathrectangle{\pgfqpoint{0.481978in}{0.331635in}}{\pgfqpoint{4.960000in}{3.696000in}}%
\pgfusepath{clip}%
\pgfsetbuttcap%
\pgfsetroundjoin%
\definecolor{currentfill}{rgb}{0.631373,0.788235,0.956863}%
\pgfsetfillcolor{currentfill}%
\pgfsetlinewidth{0.481800pt}%
\definecolor{currentstroke}{rgb}{1.000000,1.000000,1.000000}%
\pgfsetstrokecolor{currentstroke}%
\pgfsetdash{}{0pt}%
\pgfpathmoveto{\pgfqpoint{3.215015in}{3.105643in}}%
\pgfpathcurveto{\pgfqpoint{3.226065in}{3.105643in}}{\pgfqpoint{3.236664in}{3.110034in}}{\pgfqpoint{3.244477in}{3.117847in}}%
\pgfpathcurveto{\pgfqpoint{3.252291in}{3.125661in}}{\pgfqpoint{3.256681in}{3.136260in}}{\pgfqpoint{3.256681in}{3.147310in}}%
\pgfpathcurveto{\pgfqpoint{3.256681in}{3.158360in}}{\pgfqpoint{3.252291in}{3.168959in}}{\pgfqpoint{3.244477in}{3.176773in}}%
\pgfpathcurveto{\pgfqpoint{3.236664in}{3.184586in}}{\pgfqpoint{3.226065in}{3.188977in}}{\pgfqpoint{3.215015in}{3.188977in}}%
\pgfpathcurveto{\pgfqpoint{3.203964in}{3.188977in}}{\pgfqpoint{3.193365in}{3.184586in}}{\pgfqpoint{3.185552in}{3.176773in}}%
\pgfpathcurveto{\pgfqpoint{3.177738in}{3.168959in}}{\pgfqpoint{3.173348in}{3.158360in}}{\pgfqpoint{3.173348in}{3.147310in}}%
\pgfpathcurveto{\pgfqpoint{3.173348in}{3.136260in}}{\pgfqpoint{3.177738in}{3.125661in}}{\pgfqpoint{3.185552in}{3.117847in}}%
\pgfpathcurveto{\pgfqpoint{3.193365in}{3.110034in}}{\pgfqpoint{3.203964in}{3.105643in}}{\pgfqpoint{3.215015in}{3.105643in}}%
\pgfpathclose%
\pgfusepath{stroke,fill}%
\end{pgfscope}%
\begin{pgfscope}%
\pgfpathrectangle{\pgfqpoint{0.481978in}{0.331635in}}{\pgfqpoint{4.960000in}{3.696000in}}%
\pgfusepath{clip}%
\pgfsetbuttcap%
\pgfsetroundjoin%
\definecolor{currentfill}{rgb}{0.631373,0.788235,0.956863}%
\pgfsetfillcolor{currentfill}%
\pgfsetlinewidth{0.481800pt}%
\definecolor{currentstroke}{rgb}{1.000000,1.000000,1.000000}%
\pgfsetstrokecolor{currentstroke}%
\pgfsetdash{}{0pt}%
\pgfpathmoveto{\pgfqpoint{2.941688in}{2.598879in}}%
\pgfpathcurveto{\pgfqpoint{2.952738in}{2.598879in}}{\pgfqpoint{2.963337in}{2.603269in}}{\pgfqpoint{2.971150in}{2.611083in}}%
\pgfpathcurveto{\pgfqpoint{2.978964in}{2.618896in}}{\pgfqpoint{2.983354in}{2.629496in}}{\pgfqpoint{2.983354in}{2.640546in}}%
\pgfpathcurveto{\pgfqpoint{2.983354in}{2.651596in}}{\pgfqpoint{2.978964in}{2.662195in}}{\pgfqpoint{2.971150in}{2.670008in}}%
\pgfpathcurveto{\pgfqpoint{2.963337in}{2.677822in}}{\pgfqpoint{2.952738in}{2.682212in}}{\pgfqpoint{2.941688in}{2.682212in}}%
\pgfpathcurveto{\pgfqpoint{2.930637in}{2.682212in}}{\pgfqpoint{2.920038in}{2.677822in}}{\pgfqpoint{2.912225in}{2.670008in}}%
\pgfpathcurveto{\pgfqpoint{2.904411in}{2.662195in}}{\pgfqpoint{2.900021in}{2.651596in}}{\pgfqpoint{2.900021in}{2.640546in}}%
\pgfpathcurveto{\pgfqpoint{2.900021in}{2.629496in}}{\pgfqpoint{2.904411in}{2.618896in}}{\pgfqpoint{2.912225in}{2.611083in}}%
\pgfpathcurveto{\pgfqpoint{2.920038in}{2.603269in}}{\pgfqpoint{2.930637in}{2.598879in}}{\pgfqpoint{2.941688in}{2.598879in}}%
\pgfpathclose%
\pgfusepath{stroke,fill}%
\end{pgfscope}%
\begin{pgfscope}%
\pgfpathrectangle{\pgfqpoint{0.481978in}{0.331635in}}{\pgfqpoint{4.960000in}{3.696000in}}%
\pgfusepath{clip}%
\pgfsetbuttcap%
\pgfsetroundjoin%
\definecolor{currentfill}{rgb}{0.631373,0.788235,0.956863}%
\pgfsetfillcolor{currentfill}%
\pgfsetlinewidth{0.481800pt}%
\definecolor{currentstroke}{rgb}{1.000000,1.000000,1.000000}%
\pgfsetstrokecolor{currentstroke}%
\pgfsetdash{}{0pt}%
\pgfpathmoveto{\pgfqpoint{1.596447in}{2.885043in}}%
\pgfpathcurveto{\pgfqpoint{1.607498in}{2.885043in}}{\pgfqpoint{1.618097in}{2.889433in}}{\pgfqpoint{1.625910in}{2.897247in}}%
\pgfpathcurveto{\pgfqpoint{1.633724in}{2.905060in}}{\pgfqpoint{1.638114in}{2.915659in}}{\pgfqpoint{1.638114in}{2.926709in}}%
\pgfpathcurveto{\pgfqpoint{1.638114in}{2.937759in}}{\pgfqpoint{1.633724in}{2.948359in}}{\pgfqpoint{1.625910in}{2.956172in}}%
\pgfpathcurveto{\pgfqpoint{1.618097in}{2.963986in}}{\pgfqpoint{1.607498in}{2.968376in}}{\pgfqpoint{1.596447in}{2.968376in}}%
\pgfpathcurveto{\pgfqpoint{1.585397in}{2.968376in}}{\pgfqpoint{1.574798in}{2.963986in}}{\pgfqpoint{1.566985in}{2.956172in}}%
\pgfpathcurveto{\pgfqpoint{1.559171in}{2.948359in}}{\pgfqpoint{1.554781in}{2.937759in}}{\pgfqpoint{1.554781in}{2.926709in}}%
\pgfpathcurveto{\pgfqpoint{1.554781in}{2.915659in}}{\pgfqpoint{1.559171in}{2.905060in}}{\pgfqpoint{1.566985in}{2.897247in}}%
\pgfpathcurveto{\pgfqpoint{1.574798in}{2.889433in}}{\pgfqpoint{1.585397in}{2.885043in}}{\pgfqpoint{1.596447in}{2.885043in}}%
\pgfpathclose%
\pgfusepath{stroke,fill}%
\end{pgfscope}%
\begin{pgfscope}%
\pgfpathrectangle{\pgfqpoint{0.481978in}{0.331635in}}{\pgfqpoint{4.960000in}{3.696000in}}%
\pgfusepath{clip}%
\pgfsetbuttcap%
\pgfsetroundjoin%
\definecolor{currentfill}{rgb}{0.631373,0.788235,0.956863}%
\pgfsetfillcolor{currentfill}%
\pgfsetlinewidth{0.481800pt}%
\definecolor{currentstroke}{rgb}{1.000000,1.000000,1.000000}%
\pgfsetstrokecolor{currentstroke}%
\pgfsetdash{}{0pt}%
\pgfpathmoveto{\pgfqpoint{4.684035in}{1.644776in}}%
\pgfpathcurveto{\pgfqpoint{4.695085in}{1.644776in}}{\pgfqpoint{4.705684in}{1.649167in}}{\pgfqpoint{4.713498in}{1.656980in}}%
\pgfpathcurveto{\pgfqpoint{4.721311in}{1.664794in}}{\pgfqpoint{4.725702in}{1.675393in}}{\pgfqpoint{4.725702in}{1.686443in}}%
\pgfpathcurveto{\pgfqpoint{4.725702in}{1.697493in}}{\pgfqpoint{4.721311in}{1.708092in}}{\pgfqpoint{4.713498in}{1.715906in}}%
\pgfpathcurveto{\pgfqpoint{4.705684in}{1.723720in}}{\pgfqpoint{4.695085in}{1.728110in}}{\pgfqpoint{4.684035in}{1.728110in}}%
\pgfpathcurveto{\pgfqpoint{4.672985in}{1.728110in}}{\pgfqpoint{4.662386in}{1.723720in}}{\pgfqpoint{4.654572in}{1.715906in}}%
\pgfpathcurveto{\pgfqpoint{4.646759in}{1.708092in}}{\pgfqpoint{4.642368in}{1.697493in}}{\pgfqpoint{4.642368in}{1.686443in}}%
\pgfpathcurveto{\pgfqpoint{4.642368in}{1.675393in}}{\pgfqpoint{4.646759in}{1.664794in}}{\pgfqpoint{4.654572in}{1.656980in}}%
\pgfpathcurveto{\pgfqpoint{4.662386in}{1.649167in}}{\pgfqpoint{4.672985in}{1.644776in}}{\pgfqpoint{4.684035in}{1.644776in}}%
\pgfpathclose%
\pgfusepath{stroke,fill}%
\end{pgfscope}%
\begin{pgfscope}%
\pgfpathrectangle{\pgfqpoint{0.481978in}{0.331635in}}{\pgfqpoint{4.960000in}{3.696000in}}%
\pgfusepath{clip}%
\pgfsetbuttcap%
\pgfsetroundjoin%
\definecolor{currentfill}{rgb}{0.631373,0.788235,0.956863}%
\pgfsetfillcolor{currentfill}%
\pgfsetlinewidth{0.481800pt}%
\definecolor{currentstroke}{rgb}{1.000000,1.000000,1.000000}%
\pgfsetstrokecolor{currentstroke}%
\pgfsetdash{}{0pt}%
\pgfpathmoveto{\pgfqpoint{3.104244in}{3.632491in}}%
\pgfpathcurveto{\pgfqpoint{3.115294in}{3.632491in}}{\pgfqpoint{3.125893in}{3.636881in}}{\pgfqpoint{3.133707in}{3.644695in}}%
\pgfpathcurveto{\pgfqpoint{3.141520in}{3.652509in}}{\pgfqpoint{3.145911in}{3.663108in}}{\pgfqpoint{3.145911in}{3.674158in}}%
\pgfpathcurveto{\pgfqpoint{3.145911in}{3.685208in}}{\pgfqpoint{3.141520in}{3.695807in}}{\pgfqpoint{3.133707in}{3.703620in}}%
\pgfpathcurveto{\pgfqpoint{3.125893in}{3.711434in}}{\pgfqpoint{3.115294in}{3.715824in}}{\pgfqpoint{3.104244in}{3.715824in}}%
\pgfpathcurveto{\pgfqpoint{3.093194in}{3.715824in}}{\pgfqpoint{3.082595in}{3.711434in}}{\pgfqpoint{3.074781in}{3.703620in}}%
\pgfpathcurveto{\pgfqpoint{3.066967in}{3.695807in}}{\pgfqpoint{3.062577in}{3.685208in}}{\pgfqpoint{3.062577in}{3.674158in}}%
\pgfpathcurveto{\pgfqpoint{3.062577in}{3.663108in}}{\pgfqpoint{3.066967in}{3.652509in}}{\pgfqpoint{3.074781in}{3.644695in}}%
\pgfpathcurveto{\pgfqpoint{3.082595in}{3.636881in}}{\pgfqpoint{3.093194in}{3.632491in}}{\pgfqpoint{3.104244in}{3.632491in}}%
\pgfpathclose%
\pgfusepath{stroke,fill}%
\end{pgfscope}%
\begin{pgfscope}%
\pgfpathrectangle{\pgfqpoint{0.481978in}{0.331635in}}{\pgfqpoint{4.960000in}{3.696000in}}%
\pgfusepath{clip}%
\pgfsetbuttcap%
\pgfsetroundjoin%
\definecolor{currentfill}{rgb}{0.631373,0.788235,0.956863}%
\pgfsetfillcolor{currentfill}%
\pgfsetlinewidth{0.481800pt}%
\definecolor{currentstroke}{rgb}{1.000000,1.000000,1.000000}%
\pgfsetstrokecolor{currentstroke}%
\pgfsetdash{}{0pt}%
\pgfpathmoveto{\pgfqpoint{3.784793in}{3.347308in}}%
\pgfpathcurveto{\pgfqpoint{3.795843in}{3.347308in}}{\pgfqpoint{3.806442in}{3.351699in}}{\pgfqpoint{3.814256in}{3.359512in}}%
\pgfpathcurveto{\pgfqpoint{3.822069in}{3.367326in}}{\pgfqpoint{3.826460in}{3.377925in}}{\pgfqpoint{3.826460in}{3.388975in}}%
\pgfpathcurveto{\pgfqpoint{3.826460in}{3.400025in}}{\pgfqpoint{3.822069in}{3.410624in}}{\pgfqpoint{3.814256in}{3.418438in}}%
\pgfpathcurveto{\pgfqpoint{3.806442in}{3.426251in}}{\pgfqpoint{3.795843in}{3.430642in}}{\pgfqpoint{3.784793in}{3.430642in}}%
\pgfpathcurveto{\pgfqpoint{3.773743in}{3.430642in}}{\pgfqpoint{3.763144in}{3.426251in}}{\pgfqpoint{3.755330in}{3.418438in}}%
\pgfpathcurveto{\pgfqpoint{3.747517in}{3.410624in}}{\pgfqpoint{3.743126in}{3.400025in}}{\pgfqpoint{3.743126in}{3.388975in}}%
\pgfpathcurveto{\pgfqpoint{3.743126in}{3.377925in}}{\pgfqpoint{3.747517in}{3.367326in}}{\pgfqpoint{3.755330in}{3.359512in}}%
\pgfpathcurveto{\pgfqpoint{3.763144in}{3.351699in}}{\pgfqpoint{3.773743in}{3.347308in}}{\pgfqpoint{3.784793in}{3.347308in}}%
\pgfpathclose%
\pgfusepath{stroke,fill}%
\end{pgfscope}%
\begin{pgfscope}%
\pgfpathrectangle{\pgfqpoint{0.481978in}{0.331635in}}{\pgfqpoint{4.960000in}{3.696000in}}%
\pgfusepath{clip}%
\pgfsetbuttcap%
\pgfsetroundjoin%
\definecolor{currentfill}{rgb}{0.631373,0.788235,0.956863}%
\pgfsetfillcolor{currentfill}%
\pgfsetlinewidth{0.481800pt}%
\definecolor{currentstroke}{rgb}{1.000000,1.000000,1.000000}%
\pgfsetstrokecolor{currentstroke}%
\pgfsetdash{}{0pt}%
\pgfpathmoveto{\pgfqpoint{3.316835in}{3.144994in}}%
\pgfpathcurveto{\pgfqpoint{3.327885in}{3.144994in}}{\pgfqpoint{3.338484in}{3.149385in}}{\pgfqpoint{3.346297in}{3.157198in}}%
\pgfpathcurveto{\pgfqpoint{3.354111in}{3.165012in}}{\pgfqpoint{3.358501in}{3.175611in}}{\pgfqpoint{3.358501in}{3.186661in}}%
\pgfpathcurveto{\pgfqpoint{3.358501in}{3.197711in}}{\pgfqpoint{3.354111in}{3.208310in}}{\pgfqpoint{3.346297in}{3.216124in}}%
\pgfpathcurveto{\pgfqpoint{3.338484in}{3.223938in}}{\pgfqpoint{3.327885in}{3.228328in}}{\pgfqpoint{3.316835in}{3.228328in}}%
\pgfpathcurveto{\pgfqpoint{3.305785in}{3.228328in}}{\pgfqpoint{3.295185in}{3.223938in}}{\pgfqpoint{3.287372in}{3.216124in}}%
\pgfpathcurveto{\pgfqpoint{3.279558in}{3.208310in}}{\pgfqpoint{3.275168in}{3.197711in}}{\pgfqpoint{3.275168in}{3.186661in}}%
\pgfpathcurveto{\pgfqpoint{3.275168in}{3.175611in}}{\pgfqpoint{3.279558in}{3.165012in}}{\pgfqpoint{3.287372in}{3.157198in}}%
\pgfpathcurveto{\pgfqpoint{3.295185in}{3.149385in}}{\pgfqpoint{3.305785in}{3.144994in}}{\pgfqpoint{3.316835in}{3.144994in}}%
\pgfpathclose%
\pgfusepath{stroke,fill}%
\end{pgfscope}%
\begin{pgfscope}%
\pgfpathrectangle{\pgfqpoint{0.481978in}{0.331635in}}{\pgfqpoint{4.960000in}{3.696000in}}%
\pgfusepath{clip}%
\pgfsetbuttcap%
\pgfsetroundjoin%
\definecolor{currentfill}{rgb}{0.631373,0.788235,0.956863}%
\pgfsetfillcolor{currentfill}%
\pgfsetlinewidth{0.481800pt}%
\definecolor{currentstroke}{rgb}{1.000000,1.000000,1.000000}%
\pgfsetstrokecolor{currentstroke}%
\pgfsetdash{}{0pt}%
\pgfpathmoveto{\pgfqpoint{4.721051in}{3.330202in}}%
\pgfpathcurveto{\pgfqpoint{4.732101in}{3.330202in}}{\pgfqpoint{4.742700in}{3.334592in}}{\pgfqpoint{4.750513in}{3.342406in}}%
\pgfpathcurveto{\pgfqpoint{4.758327in}{3.350219in}}{\pgfqpoint{4.762717in}{3.360818in}}{\pgfqpoint{4.762717in}{3.371868in}}%
\pgfpathcurveto{\pgfqpoint{4.762717in}{3.382918in}}{\pgfqpoint{4.758327in}{3.393517in}}{\pgfqpoint{4.750513in}{3.401331in}}%
\pgfpathcurveto{\pgfqpoint{4.742700in}{3.409145in}}{\pgfqpoint{4.732101in}{3.413535in}}{\pgfqpoint{4.721051in}{3.413535in}}%
\pgfpathcurveto{\pgfqpoint{4.710000in}{3.413535in}}{\pgfqpoint{4.699401in}{3.409145in}}{\pgfqpoint{4.691588in}{3.401331in}}%
\pgfpathcurveto{\pgfqpoint{4.683774in}{3.393517in}}{\pgfqpoint{4.679384in}{3.382918in}}{\pgfqpoint{4.679384in}{3.371868in}}%
\pgfpathcurveto{\pgfqpoint{4.679384in}{3.360818in}}{\pgfqpoint{4.683774in}{3.350219in}}{\pgfqpoint{4.691588in}{3.342406in}}%
\pgfpathcurveto{\pgfqpoint{4.699401in}{3.334592in}}{\pgfqpoint{4.710000in}{3.330202in}}{\pgfqpoint{4.721051in}{3.330202in}}%
\pgfpathclose%
\pgfusepath{stroke,fill}%
\end{pgfscope}%
\begin{pgfscope}%
\pgfpathrectangle{\pgfqpoint{0.481978in}{0.331635in}}{\pgfqpoint{4.960000in}{3.696000in}}%
\pgfusepath{clip}%
\pgfsetbuttcap%
\pgfsetroundjoin%
\definecolor{currentfill}{rgb}{0.631373,0.788235,0.956863}%
\pgfsetfillcolor{currentfill}%
\pgfsetlinewidth{0.481800pt}%
\definecolor{currentstroke}{rgb}{1.000000,1.000000,1.000000}%
\pgfsetstrokecolor{currentstroke}%
\pgfsetdash{}{0pt}%
\pgfpathmoveto{\pgfqpoint{3.034645in}{3.332590in}}%
\pgfpathcurveto{\pgfqpoint{3.045695in}{3.332590in}}{\pgfqpoint{3.056294in}{3.336980in}}{\pgfqpoint{3.064108in}{3.344794in}}%
\pgfpathcurveto{\pgfqpoint{3.071921in}{3.352608in}}{\pgfqpoint{3.076311in}{3.363207in}}{\pgfqpoint{3.076311in}{3.374257in}}%
\pgfpathcurveto{\pgfqpoint{3.076311in}{3.385307in}}{\pgfqpoint{3.071921in}{3.395906in}}{\pgfqpoint{3.064108in}{3.403719in}}%
\pgfpathcurveto{\pgfqpoint{3.056294in}{3.411533in}}{\pgfqpoint{3.045695in}{3.415923in}}{\pgfqpoint{3.034645in}{3.415923in}}%
\pgfpathcurveto{\pgfqpoint{3.023595in}{3.415923in}}{\pgfqpoint{3.012996in}{3.411533in}}{\pgfqpoint{3.005182in}{3.403719in}}%
\pgfpathcurveto{\pgfqpoint{2.997368in}{3.395906in}}{\pgfqpoint{2.992978in}{3.385307in}}{\pgfqpoint{2.992978in}{3.374257in}}%
\pgfpathcurveto{\pgfqpoint{2.992978in}{3.363207in}}{\pgfqpoint{2.997368in}{3.352608in}}{\pgfqpoint{3.005182in}{3.344794in}}%
\pgfpathcurveto{\pgfqpoint{3.012996in}{3.336980in}}{\pgfqpoint{3.023595in}{3.332590in}}{\pgfqpoint{3.034645in}{3.332590in}}%
\pgfpathclose%
\pgfusepath{stroke,fill}%
\end{pgfscope}%
\begin{pgfscope}%
\pgfpathrectangle{\pgfqpoint{0.481978in}{0.331635in}}{\pgfqpoint{4.960000in}{3.696000in}}%
\pgfusepath{clip}%
\pgfsetbuttcap%
\pgfsetroundjoin%
\definecolor{currentfill}{rgb}{0.631373,0.788235,0.956863}%
\pgfsetfillcolor{currentfill}%
\pgfsetlinewidth{0.481800pt}%
\definecolor{currentstroke}{rgb}{1.000000,1.000000,1.000000}%
\pgfsetstrokecolor{currentstroke}%
\pgfsetdash{}{0pt}%
\pgfpathmoveto{\pgfqpoint{3.045472in}{2.492935in}}%
\pgfpathcurveto{\pgfqpoint{3.056522in}{2.492935in}}{\pgfqpoint{3.067121in}{2.497325in}}{\pgfqpoint{3.074935in}{2.505139in}}%
\pgfpathcurveto{\pgfqpoint{3.082748in}{2.512952in}}{\pgfqpoint{3.087139in}{2.523552in}}{\pgfqpoint{3.087139in}{2.534602in}}%
\pgfpathcurveto{\pgfqpoint{3.087139in}{2.545652in}}{\pgfqpoint{3.082748in}{2.556251in}}{\pgfqpoint{3.074935in}{2.564064in}}%
\pgfpathcurveto{\pgfqpoint{3.067121in}{2.571878in}}{\pgfqpoint{3.056522in}{2.576268in}}{\pgfqpoint{3.045472in}{2.576268in}}%
\pgfpathcurveto{\pgfqpoint{3.034422in}{2.576268in}}{\pgfqpoint{3.023823in}{2.571878in}}{\pgfqpoint{3.016009in}{2.564064in}}%
\pgfpathcurveto{\pgfqpoint{3.008196in}{2.556251in}}{\pgfqpoint{3.003805in}{2.545652in}}{\pgfqpoint{3.003805in}{2.534602in}}%
\pgfpathcurveto{\pgfqpoint{3.003805in}{2.523552in}}{\pgfqpoint{3.008196in}{2.512952in}}{\pgfqpoint{3.016009in}{2.505139in}}%
\pgfpathcurveto{\pgfqpoint{3.023823in}{2.497325in}}{\pgfqpoint{3.034422in}{2.492935in}}{\pgfqpoint{3.045472in}{2.492935in}}%
\pgfpathclose%
\pgfusepath{stroke,fill}%
\end{pgfscope}%
\begin{pgfscope}%
\pgfpathrectangle{\pgfqpoint{0.481978in}{0.331635in}}{\pgfqpoint{4.960000in}{3.696000in}}%
\pgfusepath{clip}%
\pgfsetbuttcap%
\pgfsetroundjoin%
\definecolor{currentfill}{rgb}{0.631373,0.788235,0.956863}%
\pgfsetfillcolor{currentfill}%
\pgfsetlinewidth{0.481800pt}%
\definecolor{currentstroke}{rgb}{1.000000,1.000000,1.000000}%
\pgfsetstrokecolor{currentstroke}%
\pgfsetdash{}{0pt}%
\pgfpathmoveto{\pgfqpoint{2.612555in}{2.060166in}}%
\pgfpathcurveto{\pgfqpoint{2.623605in}{2.060166in}}{\pgfqpoint{2.634204in}{2.064556in}}{\pgfqpoint{2.642018in}{2.072370in}}%
\pgfpathcurveto{\pgfqpoint{2.649831in}{2.080183in}}{\pgfqpoint{2.654222in}{2.090783in}}{\pgfqpoint{2.654222in}{2.101833in}}%
\pgfpathcurveto{\pgfqpoint{2.654222in}{2.112883in}}{\pgfqpoint{2.649831in}{2.123482in}}{\pgfqpoint{2.642018in}{2.131295in}}%
\pgfpathcurveto{\pgfqpoint{2.634204in}{2.139109in}}{\pgfqpoint{2.623605in}{2.143499in}}{\pgfqpoint{2.612555in}{2.143499in}}%
\pgfpathcurveto{\pgfqpoint{2.601505in}{2.143499in}}{\pgfqpoint{2.590906in}{2.139109in}}{\pgfqpoint{2.583092in}{2.131295in}}%
\pgfpathcurveto{\pgfqpoint{2.575279in}{2.123482in}}{\pgfqpoint{2.570888in}{2.112883in}}{\pgfqpoint{2.570888in}{2.101833in}}%
\pgfpathcurveto{\pgfqpoint{2.570888in}{2.090783in}}{\pgfqpoint{2.575279in}{2.080183in}}{\pgfqpoint{2.583092in}{2.072370in}}%
\pgfpathcurveto{\pgfqpoint{2.590906in}{2.064556in}}{\pgfqpoint{2.601505in}{2.060166in}}{\pgfqpoint{2.612555in}{2.060166in}}%
\pgfpathclose%
\pgfusepath{stroke,fill}%
\end{pgfscope}%
\begin{pgfscope}%
\pgfpathrectangle{\pgfqpoint{0.481978in}{0.331635in}}{\pgfqpoint{4.960000in}{3.696000in}}%
\pgfusepath{clip}%
\pgfsetbuttcap%
\pgfsetroundjoin%
\definecolor{currentfill}{rgb}{0.631373,0.788235,0.956863}%
\pgfsetfillcolor{currentfill}%
\pgfsetlinewidth{0.481800pt}%
\definecolor{currentstroke}{rgb}{1.000000,1.000000,1.000000}%
\pgfsetstrokecolor{currentstroke}%
\pgfsetdash{}{0pt}%
\pgfpathmoveto{\pgfqpoint{4.340254in}{1.748633in}}%
\pgfpathcurveto{\pgfqpoint{4.351304in}{1.748633in}}{\pgfqpoint{4.361903in}{1.753023in}}{\pgfqpoint{4.369717in}{1.760837in}}%
\pgfpathcurveto{\pgfqpoint{4.377530in}{1.768650in}}{\pgfqpoint{4.381921in}{1.779249in}}{\pgfqpoint{4.381921in}{1.790299in}}%
\pgfpathcurveto{\pgfqpoint{4.381921in}{1.801350in}}{\pgfqpoint{4.377530in}{1.811949in}}{\pgfqpoint{4.369717in}{1.819762in}}%
\pgfpathcurveto{\pgfqpoint{4.361903in}{1.827576in}}{\pgfqpoint{4.351304in}{1.831966in}}{\pgfqpoint{4.340254in}{1.831966in}}%
\pgfpathcurveto{\pgfqpoint{4.329204in}{1.831966in}}{\pgfqpoint{4.318605in}{1.827576in}}{\pgfqpoint{4.310791in}{1.819762in}}%
\pgfpathcurveto{\pgfqpoint{4.302977in}{1.811949in}}{\pgfqpoint{4.298587in}{1.801350in}}{\pgfqpoint{4.298587in}{1.790299in}}%
\pgfpathcurveto{\pgfqpoint{4.298587in}{1.779249in}}{\pgfqpoint{4.302977in}{1.768650in}}{\pgfqpoint{4.310791in}{1.760837in}}%
\pgfpathcurveto{\pgfqpoint{4.318605in}{1.753023in}}{\pgfqpoint{4.329204in}{1.748633in}}{\pgfqpoint{4.340254in}{1.748633in}}%
\pgfpathclose%
\pgfusepath{stroke,fill}%
\end{pgfscope}%
\begin{pgfscope}%
\pgfpathrectangle{\pgfqpoint{0.481978in}{0.331635in}}{\pgfqpoint{4.960000in}{3.696000in}}%
\pgfusepath{clip}%
\pgfsetbuttcap%
\pgfsetroundjoin%
\definecolor{currentfill}{rgb}{0.631373,0.788235,0.956863}%
\pgfsetfillcolor{currentfill}%
\pgfsetlinewidth{0.481800pt}%
\definecolor{currentstroke}{rgb}{1.000000,1.000000,1.000000}%
\pgfsetstrokecolor{currentstroke}%
\pgfsetdash{}{0pt}%
\pgfpathmoveto{\pgfqpoint{2.740137in}{2.297528in}}%
\pgfpathcurveto{\pgfqpoint{2.751187in}{2.297528in}}{\pgfqpoint{2.761786in}{2.301918in}}{\pgfqpoint{2.769600in}{2.309732in}}%
\pgfpathcurveto{\pgfqpoint{2.777414in}{2.317546in}}{\pgfqpoint{2.781804in}{2.328145in}}{\pgfqpoint{2.781804in}{2.339195in}}%
\pgfpathcurveto{\pgfqpoint{2.781804in}{2.350245in}}{\pgfqpoint{2.777414in}{2.360844in}}{\pgfqpoint{2.769600in}{2.368658in}}%
\pgfpathcurveto{\pgfqpoint{2.761786in}{2.376471in}}{\pgfqpoint{2.751187in}{2.380861in}}{\pgfqpoint{2.740137in}{2.380861in}}%
\pgfpathcurveto{\pgfqpoint{2.729087in}{2.380861in}}{\pgfqpoint{2.718488in}{2.376471in}}{\pgfqpoint{2.710674in}{2.368658in}}%
\pgfpathcurveto{\pgfqpoint{2.702861in}{2.360844in}}{\pgfqpoint{2.698470in}{2.350245in}}{\pgfqpoint{2.698470in}{2.339195in}}%
\pgfpathcurveto{\pgfqpoint{2.698470in}{2.328145in}}{\pgfqpoint{2.702861in}{2.317546in}}{\pgfqpoint{2.710674in}{2.309732in}}%
\pgfpathcurveto{\pgfqpoint{2.718488in}{2.301918in}}{\pgfqpoint{2.729087in}{2.297528in}}{\pgfqpoint{2.740137in}{2.297528in}}%
\pgfpathclose%
\pgfusepath{stroke,fill}%
\end{pgfscope}%
\begin{pgfscope}%
\pgfpathrectangle{\pgfqpoint{0.481978in}{0.331635in}}{\pgfqpoint{4.960000in}{3.696000in}}%
\pgfusepath{clip}%
\pgfsetbuttcap%
\pgfsetroundjoin%
\definecolor{currentfill}{rgb}{0.631373,0.788235,0.956863}%
\pgfsetfillcolor{currentfill}%
\pgfsetlinewidth{0.481800pt}%
\definecolor{currentstroke}{rgb}{1.000000,1.000000,1.000000}%
\pgfsetstrokecolor{currentstroke}%
\pgfsetdash{}{0pt}%
\pgfpathmoveto{\pgfqpoint{2.185010in}{2.539065in}}%
\pgfpathcurveto{\pgfqpoint{2.196061in}{2.539065in}}{\pgfqpoint{2.206660in}{2.543455in}}{\pgfqpoint{2.214473in}{2.551269in}}%
\pgfpathcurveto{\pgfqpoint{2.222287in}{2.559082in}}{\pgfqpoint{2.226677in}{2.569681in}}{\pgfqpoint{2.226677in}{2.580732in}}%
\pgfpathcurveto{\pgfqpoint{2.226677in}{2.591782in}}{\pgfqpoint{2.222287in}{2.602381in}}{\pgfqpoint{2.214473in}{2.610194in}}%
\pgfpathcurveto{\pgfqpoint{2.206660in}{2.618008in}}{\pgfqpoint{2.196061in}{2.622398in}}{\pgfqpoint{2.185010in}{2.622398in}}%
\pgfpathcurveto{\pgfqpoint{2.173960in}{2.622398in}}{\pgfqpoint{2.163361in}{2.618008in}}{\pgfqpoint{2.155548in}{2.610194in}}%
\pgfpathcurveto{\pgfqpoint{2.147734in}{2.602381in}}{\pgfqpoint{2.143344in}{2.591782in}}{\pgfqpoint{2.143344in}{2.580732in}}%
\pgfpathcurveto{\pgfqpoint{2.143344in}{2.569681in}}{\pgfqpoint{2.147734in}{2.559082in}}{\pgfqpoint{2.155548in}{2.551269in}}%
\pgfpathcurveto{\pgfqpoint{2.163361in}{2.543455in}}{\pgfqpoint{2.173960in}{2.539065in}}{\pgfqpoint{2.185010in}{2.539065in}}%
\pgfpathclose%
\pgfusepath{stroke,fill}%
\end{pgfscope}%
\begin{pgfscope}%
\pgfpathrectangle{\pgfqpoint{0.481978in}{0.331635in}}{\pgfqpoint{4.960000in}{3.696000in}}%
\pgfusepath{clip}%
\pgfsetbuttcap%
\pgfsetroundjoin%
\definecolor{currentfill}{rgb}{0.631373,0.788235,0.956863}%
\pgfsetfillcolor{currentfill}%
\pgfsetlinewidth{0.481800pt}%
\definecolor{currentstroke}{rgb}{1.000000,1.000000,1.000000}%
\pgfsetstrokecolor{currentstroke}%
\pgfsetdash{}{0pt}%
\pgfpathmoveto{\pgfqpoint{5.216523in}{1.903676in}}%
\pgfpathcurveto{\pgfqpoint{5.227574in}{1.903676in}}{\pgfqpoint{5.238173in}{1.908066in}}{\pgfqpoint{5.245986in}{1.915879in}}%
\pgfpathcurveto{\pgfqpoint{5.253800in}{1.923693in}}{\pgfqpoint{5.258190in}{1.934292in}}{\pgfqpoint{5.258190in}{1.945342in}}%
\pgfpathcurveto{\pgfqpoint{5.258190in}{1.956392in}}{\pgfqpoint{5.253800in}{1.966991in}}{\pgfqpoint{5.245986in}{1.974805in}}%
\pgfpathcurveto{\pgfqpoint{5.238173in}{1.982619in}}{\pgfqpoint{5.227574in}{1.987009in}}{\pgfqpoint{5.216523in}{1.987009in}}%
\pgfpathcurveto{\pgfqpoint{5.205473in}{1.987009in}}{\pgfqpoint{5.194874in}{1.982619in}}{\pgfqpoint{5.187061in}{1.974805in}}%
\pgfpathcurveto{\pgfqpoint{5.179247in}{1.966991in}}{\pgfqpoint{5.174857in}{1.956392in}}{\pgfqpoint{5.174857in}{1.945342in}}%
\pgfpathcurveto{\pgfqpoint{5.174857in}{1.934292in}}{\pgfqpoint{5.179247in}{1.923693in}}{\pgfqpoint{5.187061in}{1.915879in}}%
\pgfpathcurveto{\pgfqpoint{5.194874in}{1.908066in}}{\pgfqpoint{5.205473in}{1.903676in}}{\pgfqpoint{5.216523in}{1.903676in}}%
\pgfpathclose%
\pgfusepath{stroke,fill}%
\end{pgfscope}%
\begin{pgfscope}%
\pgfpathrectangle{\pgfqpoint{0.481978in}{0.331635in}}{\pgfqpoint{4.960000in}{3.696000in}}%
\pgfusepath{clip}%
\pgfsetbuttcap%
\pgfsetroundjoin%
\definecolor{currentfill}{rgb}{0.631373,0.788235,0.956863}%
\pgfsetfillcolor{currentfill}%
\pgfsetlinewidth{0.481800pt}%
\definecolor{currentstroke}{rgb}{1.000000,1.000000,1.000000}%
\pgfsetstrokecolor{currentstroke}%
\pgfsetdash{}{0pt}%
\pgfpathmoveto{\pgfqpoint{2.950569in}{1.679875in}}%
\pgfpathcurveto{\pgfqpoint{2.961619in}{1.679875in}}{\pgfqpoint{2.972218in}{1.684265in}}{\pgfqpoint{2.980032in}{1.692079in}}%
\pgfpathcurveto{\pgfqpoint{2.987846in}{1.699892in}}{\pgfqpoint{2.992236in}{1.710491in}}{\pgfqpoint{2.992236in}{1.721541in}}%
\pgfpathcurveto{\pgfqpoint{2.992236in}{1.732591in}}{\pgfqpoint{2.987846in}{1.743190in}}{\pgfqpoint{2.980032in}{1.751004in}}%
\pgfpathcurveto{\pgfqpoint{2.972218in}{1.758818in}}{\pgfqpoint{2.961619in}{1.763208in}}{\pgfqpoint{2.950569in}{1.763208in}}%
\pgfpathcurveto{\pgfqpoint{2.939519in}{1.763208in}}{\pgfqpoint{2.928920in}{1.758818in}}{\pgfqpoint{2.921106in}{1.751004in}}%
\pgfpathcurveto{\pgfqpoint{2.913293in}{1.743190in}}{\pgfqpoint{2.908902in}{1.732591in}}{\pgfqpoint{2.908902in}{1.721541in}}%
\pgfpathcurveto{\pgfqpoint{2.908902in}{1.710491in}}{\pgfqpoint{2.913293in}{1.699892in}}{\pgfqpoint{2.921106in}{1.692079in}}%
\pgfpathcurveto{\pgfqpoint{2.928920in}{1.684265in}}{\pgfqpoint{2.939519in}{1.679875in}}{\pgfqpoint{2.950569in}{1.679875in}}%
\pgfpathclose%
\pgfusepath{stroke,fill}%
\end{pgfscope}%
\begin{pgfscope}%
\pgfpathrectangle{\pgfqpoint{0.481978in}{0.331635in}}{\pgfqpoint{4.960000in}{3.696000in}}%
\pgfusepath{clip}%
\pgfsetbuttcap%
\pgfsetroundjoin%
\definecolor{currentfill}{rgb}{0.631373,0.788235,0.956863}%
\pgfsetfillcolor{currentfill}%
\pgfsetlinewidth{0.481800pt}%
\definecolor{currentstroke}{rgb}{1.000000,1.000000,1.000000}%
\pgfsetstrokecolor{currentstroke}%
\pgfsetdash{}{0pt}%
\pgfpathmoveto{\pgfqpoint{2.746965in}{3.164102in}}%
\pgfpathcurveto{\pgfqpoint{2.758015in}{3.164102in}}{\pgfqpoint{2.768614in}{3.168493in}}{\pgfqpoint{2.776427in}{3.176306in}}%
\pgfpathcurveto{\pgfqpoint{2.784241in}{3.184120in}}{\pgfqpoint{2.788631in}{3.194719in}}{\pgfqpoint{2.788631in}{3.205769in}}%
\pgfpathcurveto{\pgfqpoint{2.788631in}{3.216819in}}{\pgfqpoint{2.784241in}{3.227418in}}{\pgfqpoint{2.776427in}{3.235232in}}%
\pgfpathcurveto{\pgfqpoint{2.768614in}{3.243045in}}{\pgfqpoint{2.758015in}{3.247436in}}{\pgfqpoint{2.746965in}{3.247436in}}%
\pgfpathcurveto{\pgfqpoint{2.735914in}{3.247436in}}{\pgfqpoint{2.725315in}{3.243045in}}{\pgfqpoint{2.717502in}{3.235232in}}%
\pgfpathcurveto{\pgfqpoint{2.709688in}{3.227418in}}{\pgfqpoint{2.705298in}{3.216819in}}{\pgfqpoint{2.705298in}{3.205769in}}%
\pgfpathcurveto{\pgfqpoint{2.705298in}{3.194719in}}{\pgfqpoint{2.709688in}{3.184120in}}{\pgfqpoint{2.717502in}{3.176306in}}%
\pgfpathcurveto{\pgfqpoint{2.725315in}{3.168493in}}{\pgfqpoint{2.735914in}{3.164102in}}{\pgfqpoint{2.746965in}{3.164102in}}%
\pgfpathclose%
\pgfusepath{stroke,fill}%
\end{pgfscope}%
\begin{pgfscope}%
\pgfpathrectangle{\pgfqpoint{0.481978in}{0.331635in}}{\pgfqpoint{4.960000in}{3.696000in}}%
\pgfusepath{clip}%
\pgfsetbuttcap%
\pgfsetroundjoin%
\definecolor{currentfill}{rgb}{0.631373,0.788235,0.956863}%
\pgfsetfillcolor{currentfill}%
\pgfsetlinewidth{0.481800pt}%
\definecolor{currentstroke}{rgb}{1.000000,1.000000,1.000000}%
\pgfsetstrokecolor{currentstroke}%
\pgfsetdash{}{0pt}%
\pgfpathmoveto{\pgfqpoint{3.493134in}{2.974236in}}%
\pgfpathcurveto{\pgfqpoint{3.504184in}{2.974236in}}{\pgfqpoint{3.514783in}{2.978627in}}{\pgfqpoint{3.522596in}{2.986440in}}%
\pgfpathcurveto{\pgfqpoint{3.530410in}{2.994254in}}{\pgfqpoint{3.534800in}{3.004853in}}{\pgfqpoint{3.534800in}{3.015903in}}%
\pgfpathcurveto{\pgfqpoint{3.534800in}{3.026953in}}{\pgfqpoint{3.530410in}{3.037552in}}{\pgfqpoint{3.522596in}{3.045366in}}%
\pgfpathcurveto{\pgfqpoint{3.514783in}{3.053179in}}{\pgfqpoint{3.504184in}{3.057570in}}{\pgfqpoint{3.493134in}{3.057570in}}%
\pgfpathcurveto{\pgfqpoint{3.482084in}{3.057570in}}{\pgfqpoint{3.471484in}{3.053179in}}{\pgfqpoint{3.463671in}{3.045366in}}%
\pgfpathcurveto{\pgfqpoint{3.455857in}{3.037552in}}{\pgfqpoint{3.451467in}{3.026953in}}{\pgfqpoint{3.451467in}{3.015903in}}%
\pgfpathcurveto{\pgfqpoint{3.451467in}{3.004853in}}{\pgfqpoint{3.455857in}{2.994254in}}{\pgfqpoint{3.463671in}{2.986440in}}%
\pgfpathcurveto{\pgfqpoint{3.471484in}{2.978627in}}{\pgfqpoint{3.482084in}{2.974236in}}{\pgfqpoint{3.493134in}{2.974236in}}%
\pgfpathclose%
\pgfusepath{stroke,fill}%
\end{pgfscope}%
\begin{pgfscope}%
\pgfpathrectangle{\pgfqpoint{0.481978in}{0.331635in}}{\pgfqpoint{4.960000in}{3.696000in}}%
\pgfusepath{clip}%
\pgfsetbuttcap%
\pgfsetroundjoin%
\definecolor{currentfill}{rgb}{0.631373,0.788235,0.956863}%
\pgfsetfillcolor{currentfill}%
\pgfsetlinewidth{0.481800pt}%
\definecolor{currentstroke}{rgb}{1.000000,1.000000,1.000000}%
\pgfsetstrokecolor{currentstroke}%
\pgfsetdash{}{0pt}%
\pgfpathmoveto{\pgfqpoint{2.306544in}{2.150373in}}%
\pgfpathcurveto{\pgfqpoint{2.317594in}{2.150373in}}{\pgfqpoint{2.328193in}{2.154763in}}{\pgfqpoint{2.336007in}{2.162577in}}%
\pgfpathcurveto{\pgfqpoint{2.343821in}{2.170390in}}{\pgfqpoint{2.348211in}{2.180989in}}{\pgfqpoint{2.348211in}{2.192039in}}%
\pgfpathcurveto{\pgfqpoint{2.348211in}{2.203090in}}{\pgfqpoint{2.343821in}{2.213689in}}{\pgfqpoint{2.336007in}{2.221502in}}%
\pgfpathcurveto{\pgfqpoint{2.328193in}{2.229316in}}{\pgfqpoint{2.317594in}{2.233706in}}{\pgfqpoint{2.306544in}{2.233706in}}%
\pgfpathcurveto{\pgfqpoint{2.295494in}{2.233706in}}{\pgfqpoint{2.284895in}{2.229316in}}{\pgfqpoint{2.277081in}{2.221502in}}%
\pgfpathcurveto{\pgfqpoint{2.269268in}{2.213689in}}{\pgfqpoint{2.264878in}{2.203090in}}{\pgfqpoint{2.264878in}{2.192039in}}%
\pgfpathcurveto{\pgfqpoint{2.264878in}{2.180989in}}{\pgfqpoint{2.269268in}{2.170390in}}{\pgfqpoint{2.277081in}{2.162577in}}%
\pgfpathcurveto{\pgfqpoint{2.284895in}{2.154763in}}{\pgfqpoint{2.295494in}{2.150373in}}{\pgfqpoint{2.306544in}{2.150373in}}%
\pgfpathclose%
\pgfusepath{stroke,fill}%
\end{pgfscope}%
\begin{pgfscope}%
\pgfpathrectangle{\pgfqpoint{0.481978in}{0.331635in}}{\pgfqpoint{4.960000in}{3.696000in}}%
\pgfusepath{clip}%
\pgfsetbuttcap%
\pgfsetroundjoin%
\definecolor{currentfill}{rgb}{0.631373,0.788235,0.956863}%
\pgfsetfillcolor{currentfill}%
\pgfsetlinewidth{0.481800pt}%
\definecolor{currentstroke}{rgb}{1.000000,1.000000,1.000000}%
\pgfsetstrokecolor{currentstroke}%
\pgfsetdash{}{0pt}%
\pgfpathmoveto{\pgfqpoint{2.610617in}{2.106005in}}%
\pgfpathcurveto{\pgfqpoint{2.621667in}{2.106005in}}{\pgfqpoint{2.632266in}{2.110396in}}{\pgfqpoint{2.640080in}{2.118209in}}%
\pgfpathcurveto{\pgfqpoint{2.647894in}{2.126023in}}{\pgfqpoint{2.652284in}{2.136622in}}{\pgfqpoint{2.652284in}{2.147672in}}%
\pgfpathcurveto{\pgfqpoint{2.652284in}{2.158722in}}{\pgfqpoint{2.647894in}{2.169321in}}{\pgfqpoint{2.640080in}{2.177135in}}%
\pgfpathcurveto{\pgfqpoint{2.632266in}{2.184948in}}{\pgfqpoint{2.621667in}{2.189339in}}{\pgfqpoint{2.610617in}{2.189339in}}%
\pgfpathcurveto{\pgfqpoint{2.599567in}{2.189339in}}{\pgfqpoint{2.588968in}{2.184948in}}{\pgfqpoint{2.581154in}{2.177135in}}%
\pgfpathcurveto{\pgfqpoint{2.573341in}{2.169321in}}{\pgfqpoint{2.568950in}{2.158722in}}{\pgfqpoint{2.568950in}{2.147672in}}%
\pgfpathcurveto{\pgfqpoint{2.568950in}{2.136622in}}{\pgfqpoint{2.573341in}{2.126023in}}{\pgfqpoint{2.581154in}{2.118209in}}%
\pgfpathcurveto{\pgfqpoint{2.588968in}{2.110396in}}{\pgfqpoint{2.599567in}{2.106005in}}{\pgfqpoint{2.610617in}{2.106005in}}%
\pgfpathclose%
\pgfusepath{stroke,fill}%
\end{pgfscope}%
\begin{pgfscope}%
\pgfpathrectangle{\pgfqpoint{0.481978in}{0.331635in}}{\pgfqpoint{4.960000in}{3.696000in}}%
\pgfusepath{clip}%
\pgfsetbuttcap%
\pgfsetroundjoin%
\definecolor{currentfill}{rgb}{0.631373,0.788235,0.956863}%
\pgfsetfillcolor{currentfill}%
\pgfsetlinewidth{0.481800pt}%
\definecolor{currentstroke}{rgb}{1.000000,1.000000,1.000000}%
\pgfsetstrokecolor{currentstroke}%
\pgfsetdash{}{0pt}%
\pgfpathmoveto{\pgfqpoint{3.281173in}{2.522705in}}%
\pgfpathcurveto{\pgfqpoint{3.292223in}{2.522705in}}{\pgfqpoint{3.302822in}{2.527095in}}{\pgfqpoint{3.310636in}{2.534908in}}%
\pgfpathcurveto{\pgfqpoint{3.318449in}{2.542722in}}{\pgfqpoint{3.322839in}{2.553321in}}{\pgfqpoint{3.322839in}{2.564371in}}%
\pgfpathcurveto{\pgfqpoint{3.322839in}{2.575421in}}{\pgfqpoint{3.318449in}{2.586020in}}{\pgfqpoint{3.310636in}{2.593834in}}%
\pgfpathcurveto{\pgfqpoint{3.302822in}{2.601648in}}{\pgfqpoint{3.292223in}{2.606038in}}{\pgfqpoint{3.281173in}{2.606038in}}%
\pgfpathcurveto{\pgfqpoint{3.270123in}{2.606038in}}{\pgfqpoint{3.259524in}{2.601648in}}{\pgfqpoint{3.251710in}{2.593834in}}%
\pgfpathcurveto{\pgfqpoint{3.243896in}{2.586020in}}{\pgfqpoint{3.239506in}{2.575421in}}{\pgfqpoint{3.239506in}{2.564371in}}%
\pgfpathcurveto{\pgfqpoint{3.239506in}{2.553321in}}{\pgfqpoint{3.243896in}{2.542722in}}{\pgfqpoint{3.251710in}{2.534908in}}%
\pgfpathcurveto{\pgfqpoint{3.259524in}{2.527095in}}{\pgfqpoint{3.270123in}{2.522705in}}{\pgfqpoint{3.281173in}{2.522705in}}%
\pgfpathclose%
\pgfusepath{stroke,fill}%
\end{pgfscope}%
\begin{pgfscope}%
\pgfpathrectangle{\pgfqpoint{0.481978in}{0.331635in}}{\pgfqpoint{4.960000in}{3.696000in}}%
\pgfusepath{clip}%
\pgfsetbuttcap%
\pgfsetroundjoin%
\definecolor{currentfill}{rgb}{0.631373,0.788235,0.956863}%
\pgfsetfillcolor{currentfill}%
\pgfsetlinewidth{0.481800pt}%
\definecolor{currentstroke}{rgb}{1.000000,1.000000,1.000000}%
\pgfsetstrokecolor{currentstroke}%
\pgfsetdash{}{0pt}%
\pgfpathmoveto{\pgfqpoint{3.154017in}{3.250502in}}%
\pgfpathcurveto{\pgfqpoint{3.165067in}{3.250502in}}{\pgfqpoint{3.175666in}{3.254892in}}{\pgfqpoint{3.183480in}{3.262706in}}%
\pgfpathcurveto{\pgfqpoint{3.191294in}{3.270519in}}{\pgfqpoint{3.195684in}{3.281118in}}{\pgfqpoint{3.195684in}{3.292168in}}%
\pgfpathcurveto{\pgfqpoint{3.195684in}{3.303218in}}{\pgfqpoint{3.191294in}{3.313818in}}{\pgfqpoint{3.183480in}{3.321631in}}%
\pgfpathcurveto{\pgfqpoint{3.175666in}{3.329445in}}{\pgfqpoint{3.165067in}{3.333835in}}{\pgfqpoint{3.154017in}{3.333835in}}%
\pgfpathcurveto{\pgfqpoint{3.142967in}{3.333835in}}{\pgfqpoint{3.132368in}{3.329445in}}{\pgfqpoint{3.124554in}{3.321631in}}%
\pgfpathcurveto{\pgfqpoint{3.116741in}{3.313818in}}{\pgfqpoint{3.112351in}{3.303218in}}{\pgfqpoint{3.112351in}{3.292168in}}%
\pgfpathcurveto{\pgfqpoint{3.112351in}{3.281118in}}{\pgfqpoint{3.116741in}{3.270519in}}{\pgfqpoint{3.124554in}{3.262706in}}%
\pgfpathcurveto{\pgfqpoint{3.132368in}{3.254892in}}{\pgfqpoint{3.142967in}{3.250502in}}{\pgfqpoint{3.154017in}{3.250502in}}%
\pgfpathclose%
\pgfusepath{stroke,fill}%
\end{pgfscope}%
\begin{pgfscope}%
\pgfpathrectangle{\pgfqpoint{0.481978in}{0.331635in}}{\pgfqpoint{4.960000in}{3.696000in}}%
\pgfusepath{clip}%
\pgfsetbuttcap%
\pgfsetroundjoin%
\definecolor{currentfill}{rgb}{0.631373,0.788235,0.956863}%
\pgfsetfillcolor{currentfill}%
\pgfsetlinewidth{0.481800pt}%
\definecolor{currentstroke}{rgb}{1.000000,1.000000,1.000000}%
\pgfsetstrokecolor{currentstroke}%
\pgfsetdash{}{0pt}%
\pgfpathmoveto{\pgfqpoint{2.374813in}{2.415767in}}%
\pgfpathcurveto{\pgfqpoint{2.385863in}{2.415767in}}{\pgfqpoint{2.396463in}{2.420157in}}{\pgfqpoint{2.404276in}{2.427971in}}%
\pgfpathcurveto{\pgfqpoint{2.412090in}{2.435785in}}{\pgfqpoint{2.416480in}{2.446384in}}{\pgfqpoint{2.416480in}{2.457434in}}%
\pgfpathcurveto{\pgfqpoint{2.416480in}{2.468484in}}{\pgfqpoint{2.412090in}{2.479083in}}{\pgfqpoint{2.404276in}{2.486897in}}%
\pgfpathcurveto{\pgfqpoint{2.396463in}{2.494710in}}{\pgfqpoint{2.385863in}{2.499100in}}{\pgfqpoint{2.374813in}{2.499100in}}%
\pgfpathcurveto{\pgfqpoint{2.363763in}{2.499100in}}{\pgfqpoint{2.353164in}{2.494710in}}{\pgfqpoint{2.345351in}{2.486897in}}%
\pgfpathcurveto{\pgfqpoint{2.337537in}{2.479083in}}{\pgfqpoint{2.333147in}{2.468484in}}{\pgfqpoint{2.333147in}{2.457434in}}%
\pgfpathcurveto{\pgfqpoint{2.333147in}{2.446384in}}{\pgfqpoint{2.337537in}{2.435785in}}{\pgfqpoint{2.345351in}{2.427971in}}%
\pgfpathcurveto{\pgfqpoint{2.353164in}{2.420157in}}{\pgfqpoint{2.363763in}{2.415767in}}{\pgfqpoint{2.374813in}{2.415767in}}%
\pgfpathclose%
\pgfusepath{stroke,fill}%
\end{pgfscope}%
\begin{pgfscope}%
\pgfpathrectangle{\pgfqpoint{0.481978in}{0.331635in}}{\pgfqpoint{4.960000in}{3.696000in}}%
\pgfusepath{clip}%
\pgfsetbuttcap%
\pgfsetroundjoin%
\definecolor{currentfill}{rgb}{0.631373,0.788235,0.956863}%
\pgfsetfillcolor{currentfill}%
\pgfsetlinewidth{0.481800pt}%
\definecolor{currentstroke}{rgb}{1.000000,1.000000,1.000000}%
\pgfsetstrokecolor{currentstroke}%
\pgfsetdash{}{0pt}%
\pgfpathmoveto{\pgfqpoint{3.516196in}{3.088443in}}%
\pgfpathcurveto{\pgfqpoint{3.527246in}{3.088443in}}{\pgfqpoint{3.537845in}{3.092834in}}{\pgfqpoint{3.545659in}{3.100647in}}%
\pgfpathcurveto{\pgfqpoint{3.553473in}{3.108461in}}{\pgfqpoint{3.557863in}{3.119060in}}{\pgfqpoint{3.557863in}{3.130110in}}%
\pgfpathcurveto{\pgfqpoint{3.557863in}{3.141160in}}{\pgfqpoint{3.553473in}{3.151759in}}{\pgfqpoint{3.545659in}{3.159573in}}%
\pgfpathcurveto{\pgfqpoint{3.537845in}{3.167386in}}{\pgfqpoint{3.527246in}{3.171777in}}{\pgfqpoint{3.516196in}{3.171777in}}%
\pgfpathcurveto{\pgfqpoint{3.505146in}{3.171777in}}{\pgfqpoint{3.494547in}{3.167386in}}{\pgfqpoint{3.486734in}{3.159573in}}%
\pgfpathcurveto{\pgfqpoint{3.478920in}{3.151759in}}{\pgfqpoint{3.474530in}{3.141160in}}{\pgfqpoint{3.474530in}{3.130110in}}%
\pgfpathcurveto{\pgfqpoint{3.474530in}{3.119060in}}{\pgfqpoint{3.478920in}{3.108461in}}{\pgfqpoint{3.486734in}{3.100647in}}%
\pgfpathcurveto{\pgfqpoint{3.494547in}{3.092834in}}{\pgfqpoint{3.505146in}{3.088443in}}{\pgfqpoint{3.516196in}{3.088443in}}%
\pgfpathclose%
\pgfusepath{stroke,fill}%
\end{pgfscope}%
\begin{pgfscope}%
\pgfpathrectangle{\pgfqpoint{0.481978in}{0.331635in}}{\pgfqpoint{4.960000in}{3.696000in}}%
\pgfusepath{clip}%
\pgfsetbuttcap%
\pgfsetroundjoin%
\definecolor{currentfill}{rgb}{0.631373,0.788235,0.956863}%
\pgfsetfillcolor{currentfill}%
\pgfsetlinewidth{0.481800pt}%
\definecolor{currentstroke}{rgb}{1.000000,1.000000,1.000000}%
\pgfsetstrokecolor{currentstroke}%
\pgfsetdash{}{0pt}%
\pgfpathmoveto{\pgfqpoint{2.937519in}{3.043976in}}%
\pgfpathcurveto{\pgfqpoint{2.948569in}{3.043976in}}{\pgfqpoint{2.959168in}{3.048366in}}{\pgfqpoint{2.966981in}{3.056179in}}%
\pgfpathcurveto{\pgfqpoint{2.974795in}{3.063993in}}{\pgfqpoint{2.979185in}{3.074592in}}{\pgfqpoint{2.979185in}{3.085642in}}%
\pgfpathcurveto{\pgfqpoint{2.979185in}{3.096692in}}{\pgfqpoint{2.974795in}{3.107291in}}{\pgfqpoint{2.966981in}{3.115105in}}%
\pgfpathcurveto{\pgfqpoint{2.959168in}{3.122919in}}{\pgfqpoint{2.948569in}{3.127309in}}{\pgfqpoint{2.937519in}{3.127309in}}%
\pgfpathcurveto{\pgfqpoint{2.926468in}{3.127309in}}{\pgfqpoint{2.915869in}{3.122919in}}{\pgfqpoint{2.908056in}{3.115105in}}%
\pgfpathcurveto{\pgfqpoint{2.900242in}{3.107291in}}{\pgfqpoint{2.895852in}{3.096692in}}{\pgfqpoint{2.895852in}{3.085642in}}%
\pgfpathcurveto{\pgfqpoint{2.895852in}{3.074592in}}{\pgfqpoint{2.900242in}{3.063993in}}{\pgfqpoint{2.908056in}{3.056179in}}%
\pgfpathcurveto{\pgfqpoint{2.915869in}{3.048366in}}{\pgfqpoint{2.926468in}{3.043976in}}{\pgfqpoint{2.937519in}{3.043976in}}%
\pgfpathclose%
\pgfusepath{stroke,fill}%
\end{pgfscope}%
\begin{pgfscope}%
\pgfpathrectangle{\pgfqpoint{0.481978in}{0.331635in}}{\pgfqpoint{4.960000in}{3.696000in}}%
\pgfusepath{clip}%
\pgfsetbuttcap%
\pgfsetroundjoin%
\definecolor{currentfill}{rgb}{0.631373,0.788235,0.956863}%
\pgfsetfillcolor{currentfill}%
\pgfsetlinewidth{0.481800pt}%
\definecolor{currentstroke}{rgb}{1.000000,1.000000,1.000000}%
\pgfsetstrokecolor{currentstroke}%
\pgfsetdash{}{0pt}%
\pgfpathmoveto{\pgfqpoint{2.518830in}{2.529730in}}%
\pgfpathcurveto{\pgfqpoint{2.529880in}{2.529730in}}{\pgfqpoint{2.540480in}{2.534120in}}{\pgfqpoint{2.548293in}{2.541933in}}%
\pgfpathcurveto{\pgfqpoint{2.556107in}{2.549747in}}{\pgfqpoint{2.560497in}{2.560346in}}{\pgfqpoint{2.560497in}{2.571396in}}%
\pgfpathcurveto{\pgfqpoint{2.560497in}{2.582446in}}{\pgfqpoint{2.556107in}{2.593045in}}{\pgfqpoint{2.548293in}{2.600859in}}%
\pgfpathcurveto{\pgfqpoint{2.540480in}{2.608673in}}{\pgfqpoint{2.529880in}{2.613063in}}{\pgfqpoint{2.518830in}{2.613063in}}%
\pgfpathcurveto{\pgfqpoint{2.507780in}{2.613063in}}{\pgfqpoint{2.497181in}{2.608673in}}{\pgfqpoint{2.489368in}{2.600859in}}%
\pgfpathcurveto{\pgfqpoint{2.481554in}{2.593045in}}{\pgfqpoint{2.477164in}{2.582446in}}{\pgfqpoint{2.477164in}{2.571396in}}%
\pgfpathcurveto{\pgfqpoint{2.477164in}{2.560346in}}{\pgfqpoint{2.481554in}{2.549747in}}{\pgfqpoint{2.489368in}{2.541933in}}%
\pgfpathcurveto{\pgfqpoint{2.497181in}{2.534120in}}{\pgfqpoint{2.507780in}{2.529730in}}{\pgfqpoint{2.518830in}{2.529730in}}%
\pgfpathclose%
\pgfusepath{stroke,fill}%
\end{pgfscope}%
\begin{pgfscope}%
\pgfpathrectangle{\pgfqpoint{0.481978in}{0.331635in}}{\pgfqpoint{4.960000in}{3.696000in}}%
\pgfusepath{clip}%
\pgfsetbuttcap%
\pgfsetroundjoin%
\definecolor{currentfill}{rgb}{0.631373,0.788235,0.956863}%
\pgfsetfillcolor{currentfill}%
\pgfsetlinewidth{0.481800pt}%
\definecolor{currentstroke}{rgb}{1.000000,1.000000,1.000000}%
\pgfsetstrokecolor{currentstroke}%
\pgfsetdash{}{0pt}%
\pgfpathmoveto{\pgfqpoint{2.552065in}{2.752170in}}%
\pgfpathcurveto{\pgfqpoint{2.563115in}{2.752170in}}{\pgfqpoint{2.573714in}{2.756560in}}{\pgfqpoint{2.581528in}{2.764373in}}%
\pgfpathcurveto{\pgfqpoint{2.589341in}{2.772187in}}{\pgfqpoint{2.593732in}{2.782786in}}{\pgfqpoint{2.593732in}{2.793836in}}%
\pgfpathcurveto{\pgfqpoint{2.593732in}{2.804886in}}{\pgfqpoint{2.589341in}{2.815485in}}{\pgfqpoint{2.581528in}{2.823299in}}%
\pgfpathcurveto{\pgfqpoint{2.573714in}{2.831113in}}{\pgfqpoint{2.563115in}{2.835503in}}{\pgfqpoint{2.552065in}{2.835503in}}%
\pgfpathcurveto{\pgfqpoint{2.541015in}{2.835503in}}{\pgfqpoint{2.530416in}{2.831113in}}{\pgfqpoint{2.522602in}{2.823299in}}%
\pgfpathcurveto{\pgfqpoint{2.514789in}{2.815485in}}{\pgfqpoint{2.510398in}{2.804886in}}{\pgfqpoint{2.510398in}{2.793836in}}%
\pgfpathcurveto{\pgfqpoint{2.510398in}{2.782786in}}{\pgfqpoint{2.514789in}{2.772187in}}{\pgfqpoint{2.522602in}{2.764373in}}%
\pgfpathcurveto{\pgfqpoint{2.530416in}{2.756560in}}{\pgfqpoint{2.541015in}{2.752170in}}{\pgfqpoint{2.552065in}{2.752170in}}%
\pgfpathclose%
\pgfusepath{stroke,fill}%
\end{pgfscope}%
\begin{pgfscope}%
\pgfpathrectangle{\pgfqpoint{0.481978in}{0.331635in}}{\pgfqpoint{4.960000in}{3.696000in}}%
\pgfusepath{clip}%
\pgfsetbuttcap%
\pgfsetroundjoin%
\definecolor{currentfill}{rgb}{0.631373,0.788235,0.956863}%
\pgfsetfillcolor{currentfill}%
\pgfsetlinewidth{0.481800pt}%
\definecolor{currentstroke}{rgb}{1.000000,1.000000,1.000000}%
\pgfsetstrokecolor{currentstroke}%
\pgfsetdash{}{0pt}%
\pgfpathmoveto{\pgfqpoint{3.713491in}{3.532190in}}%
\pgfpathcurveto{\pgfqpoint{3.724541in}{3.532190in}}{\pgfqpoint{3.735140in}{3.536580in}}{\pgfqpoint{3.742954in}{3.544394in}}%
\pgfpathcurveto{\pgfqpoint{3.750768in}{3.552207in}}{\pgfqpoint{3.755158in}{3.562806in}}{\pgfqpoint{3.755158in}{3.573856in}}%
\pgfpathcurveto{\pgfqpoint{3.755158in}{3.584907in}}{\pgfqpoint{3.750768in}{3.595506in}}{\pgfqpoint{3.742954in}{3.603319in}}%
\pgfpathcurveto{\pgfqpoint{3.735140in}{3.611133in}}{\pgfqpoint{3.724541in}{3.615523in}}{\pgfqpoint{3.713491in}{3.615523in}}%
\pgfpathcurveto{\pgfqpoint{3.702441in}{3.615523in}}{\pgfqpoint{3.691842in}{3.611133in}}{\pgfqpoint{3.684028in}{3.603319in}}%
\pgfpathcurveto{\pgfqpoint{3.676215in}{3.595506in}}{\pgfqpoint{3.671825in}{3.584907in}}{\pgfqpoint{3.671825in}{3.573856in}}%
\pgfpathcurveto{\pgfqpoint{3.671825in}{3.562806in}}{\pgfqpoint{3.676215in}{3.552207in}}{\pgfqpoint{3.684028in}{3.544394in}}%
\pgfpathcurveto{\pgfqpoint{3.691842in}{3.536580in}}{\pgfqpoint{3.702441in}{3.532190in}}{\pgfqpoint{3.713491in}{3.532190in}}%
\pgfpathclose%
\pgfusepath{stroke,fill}%
\end{pgfscope}%
\begin{pgfscope}%
\pgfpathrectangle{\pgfqpoint{0.481978in}{0.331635in}}{\pgfqpoint{4.960000in}{3.696000in}}%
\pgfusepath{clip}%
\pgfsetbuttcap%
\pgfsetroundjoin%
\definecolor{currentfill}{rgb}{0.631373,0.788235,0.956863}%
\pgfsetfillcolor{currentfill}%
\pgfsetlinewidth{0.481800pt}%
\definecolor{currentstroke}{rgb}{1.000000,1.000000,1.000000}%
\pgfsetstrokecolor{currentstroke}%
\pgfsetdash{}{0pt}%
\pgfpathmoveto{\pgfqpoint{4.462105in}{1.669518in}}%
\pgfpathcurveto{\pgfqpoint{4.473155in}{1.669518in}}{\pgfqpoint{4.483754in}{1.673909in}}{\pgfqpoint{4.491567in}{1.681722in}}%
\pgfpathcurveto{\pgfqpoint{4.499381in}{1.689536in}}{\pgfqpoint{4.503771in}{1.700135in}}{\pgfqpoint{4.503771in}{1.711185in}}%
\pgfpathcurveto{\pgfqpoint{4.503771in}{1.722235in}}{\pgfqpoint{4.499381in}{1.732834in}}{\pgfqpoint{4.491567in}{1.740648in}}%
\pgfpathcurveto{\pgfqpoint{4.483754in}{1.748462in}}{\pgfqpoint{4.473155in}{1.752852in}}{\pgfqpoint{4.462105in}{1.752852in}}%
\pgfpathcurveto{\pgfqpoint{4.451054in}{1.752852in}}{\pgfqpoint{4.440455in}{1.748462in}}{\pgfqpoint{4.432642in}{1.740648in}}%
\pgfpathcurveto{\pgfqpoint{4.424828in}{1.732834in}}{\pgfqpoint{4.420438in}{1.722235in}}{\pgfqpoint{4.420438in}{1.711185in}}%
\pgfpathcurveto{\pgfqpoint{4.420438in}{1.700135in}}{\pgfqpoint{4.424828in}{1.689536in}}{\pgfqpoint{4.432642in}{1.681722in}}%
\pgfpathcurveto{\pgfqpoint{4.440455in}{1.673909in}}{\pgfqpoint{4.451054in}{1.669518in}}{\pgfqpoint{4.462105in}{1.669518in}}%
\pgfpathclose%
\pgfusepath{stroke,fill}%
\end{pgfscope}%
\begin{pgfscope}%
\pgfpathrectangle{\pgfqpoint{0.481978in}{0.331635in}}{\pgfqpoint{4.960000in}{3.696000in}}%
\pgfusepath{clip}%
\pgfsetbuttcap%
\pgfsetroundjoin%
\definecolor{currentfill}{rgb}{0.631373,0.788235,0.956863}%
\pgfsetfillcolor{currentfill}%
\pgfsetlinewidth{0.481800pt}%
\definecolor{currentstroke}{rgb}{1.000000,1.000000,1.000000}%
\pgfsetstrokecolor{currentstroke}%
\pgfsetdash{}{0pt}%
\pgfpathmoveto{\pgfqpoint{2.614726in}{2.876624in}}%
\pgfpathcurveto{\pgfqpoint{2.625776in}{2.876624in}}{\pgfqpoint{2.636375in}{2.881014in}}{\pgfqpoint{2.644189in}{2.888828in}}%
\pgfpathcurveto{\pgfqpoint{2.652003in}{2.896642in}}{\pgfqpoint{2.656393in}{2.907241in}}{\pgfqpoint{2.656393in}{2.918291in}}%
\pgfpathcurveto{\pgfqpoint{2.656393in}{2.929341in}}{\pgfqpoint{2.652003in}{2.939940in}}{\pgfqpoint{2.644189in}{2.947754in}}%
\pgfpathcurveto{\pgfqpoint{2.636375in}{2.955567in}}{\pgfqpoint{2.625776in}{2.959958in}}{\pgfqpoint{2.614726in}{2.959958in}}%
\pgfpathcurveto{\pgfqpoint{2.603676in}{2.959958in}}{\pgfqpoint{2.593077in}{2.955567in}}{\pgfqpoint{2.585263in}{2.947754in}}%
\pgfpathcurveto{\pgfqpoint{2.577450in}{2.939940in}}{\pgfqpoint{2.573060in}{2.929341in}}{\pgfqpoint{2.573060in}{2.918291in}}%
\pgfpathcurveto{\pgfqpoint{2.573060in}{2.907241in}}{\pgfqpoint{2.577450in}{2.896642in}}{\pgfqpoint{2.585263in}{2.888828in}}%
\pgfpathcurveto{\pgfqpoint{2.593077in}{2.881014in}}{\pgfqpoint{2.603676in}{2.876624in}}{\pgfqpoint{2.614726in}{2.876624in}}%
\pgfpathclose%
\pgfusepath{stroke,fill}%
\end{pgfscope}%
\begin{pgfscope}%
\pgfpathrectangle{\pgfqpoint{0.481978in}{0.331635in}}{\pgfqpoint{4.960000in}{3.696000in}}%
\pgfusepath{clip}%
\pgfsetbuttcap%
\pgfsetroundjoin%
\definecolor{currentfill}{rgb}{0.631373,0.788235,0.956863}%
\pgfsetfillcolor{currentfill}%
\pgfsetlinewidth{0.481800pt}%
\definecolor{currentstroke}{rgb}{1.000000,1.000000,1.000000}%
\pgfsetstrokecolor{currentstroke}%
\pgfsetdash{}{0pt}%
\pgfpathmoveto{\pgfqpoint{4.943248in}{1.885086in}}%
\pgfpathcurveto{\pgfqpoint{4.954299in}{1.885086in}}{\pgfqpoint{4.964898in}{1.889476in}}{\pgfqpoint{4.972711in}{1.897290in}}%
\pgfpathcurveto{\pgfqpoint{4.980525in}{1.905104in}}{\pgfqpoint{4.984915in}{1.915703in}}{\pgfqpoint{4.984915in}{1.926753in}}%
\pgfpathcurveto{\pgfqpoint{4.984915in}{1.937803in}}{\pgfqpoint{4.980525in}{1.948402in}}{\pgfqpoint{4.972711in}{1.956216in}}%
\pgfpathcurveto{\pgfqpoint{4.964898in}{1.964029in}}{\pgfqpoint{4.954299in}{1.968420in}}{\pgfqpoint{4.943248in}{1.968420in}}%
\pgfpathcurveto{\pgfqpoint{4.932198in}{1.968420in}}{\pgfqpoint{4.921599in}{1.964029in}}{\pgfqpoint{4.913786in}{1.956216in}}%
\pgfpathcurveto{\pgfqpoint{4.905972in}{1.948402in}}{\pgfqpoint{4.901582in}{1.937803in}}{\pgfqpoint{4.901582in}{1.926753in}}%
\pgfpathcurveto{\pgfqpoint{4.901582in}{1.915703in}}{\pgfqpoint{4.905972in}{1.905104in}}{\pgfqpoint{4.913786in}{1.897290in}}%
\pgfpathcurveto{\pgfqpoint{4.921599in}{1.889476in}}{\pgfqpoint{4.932198in}{1.885086in}}{\pgfqpoint{4.943248in}{1.885086in}}%
\pgfpathclose%
\pgfusepath{stroke,fill}%
\end{pgfscope}%
\begin{pgfscope}%
\pgfpathrectangle{\pgfqpoint{0.481978in}{0.331635in}}{\pgfqpoint{4.960000in}{3.696000in}}%
\pgfusepath{clip}%
\pgfsetbuttcap%
\pgfsetroundjoin%
\definecolor{currentfill}{rgb}{0.631373,0.788235,0.956863}%
\pgfsetfillcolor{currentfill}%
\pgfsetlinewidth{0.481800pt}%
\definecolor{currentstroke}{rgb}{1.000000,1.000000,1.000000}%
\pgfsetstrokecolor{currentstroke}%
\pgfsetdash{}{0pt}%
\pgfpathmoveto{\pgfqpoint{3.155268in}{3.720455in}}%
\pgfpathcurveto{\pgfqpoint{3.166318in}{3.720455in}}{\pgfqpoint{3.176917in}{3.724845in}}{\pgfqpoint{3.184730in}{3.732659in}}%
\pgfpathcurveto{\pgfqpoint{3.192544in}{3.740473in}}{\pgfqpoint{3.196934in}{3.751072in}}{\pgfqpoint{3.196934in}{3.762122in}}%
\pgfpathcurveto{\pgfqpoint{3.196934in}{3.773172in}}{\pgfqpoint{3.192544in}{3.783771in}}{\pgfqpoint{3.184730in}{3.791584in}}%
\pgfpathcurveto{\pgfqpoint{3.176917in}{3.799398in}}{\pgfqpoint{3.166318in}{3.803788in}}{\pgfqpoint{3.155268in}{3.803788in}}%
\pgfpathcurveto{\pgfqpoint{3.144218in}{3.803788in}}{\pgfqpoint{3.133619in}{3.799398in}}{\pgfqpoint{3.125805in}{3.791584in}}%
\pgfpathcurveto{\pgfqpoint{3.117991in}{3.783771in}}{\pgfqpoint{3.113601in}{3.773172in}}{\pgfqpoint{3.113601in}{3.762122in}}%
\pgfpathcurveto{\pgfqpoint{3.113601in}{3.751072in}}{\pgfqpoint{3.117991in}{3.740473in}}{\pgfqpoint{3.125805in}{3.732659in}}%
\pgfpathcurveto{\pgfqpoint{3.133619in}{3.724845in}}{\pgfqpoint{3.144218in}{3.720455in}}{\pgfqpoint{3.155268in}{3.720455in}}%
\pgfpathclose%
\pgfusepath{stroke,fill}%
\end{pgfscope}%
\begin{pgfscope}%
\pgfpathrectangle{\pgfqpoint{0.481978in}{0.331635in}}{\pgfqpoint{4.960000in}{3.696000in}}%
\pgfusepath{clip}%
\pgfsetbuttcap%
\pgfsetroundjoin%
\definecolor{currentfill}{rgb}{0.631373,0.788235,0.956863}%
\pgfsetfillcolor{currentfill}%
\pgfsetlinewidth{0.481800pt}%
\definecolor{currentstroke}{rgb}{1.000000,1.000000,1.000000}%
\pgfsetstrokecolor{currentstroke}%
\pgfsetdash{}{0pt}%
\pgfpathmoveto{\pgfqpoint{3.161490in}{2.894710in}}%
\pgfpathcurveto{\pgfqpoint{3.172540in}{2.894710in}}{\pgfqpoint{3.183139in}{2.899100in}}{\pgfqpoint{3.190953in}{2.906913in}}%
\pgfpathcurveto{\pgfqpoint{3.198766in}{2.914727in}}{\pgfqpoint{3.203156in}{2.925326in}}{\pgfqpoint{3.203156in}{2.936376in}}%
\pgfpathcurveto{\pgfqpoint{3.203156in}{2.947426in}}{\pgfqpoint{3.198766in}{2.958025in}}{\pgfqpoint{3.190953in}{2.965839in}}%
\pgfpathcurveto{\pgfqpoint{3.183139in}{2.973653in}}{\pgfqpoint{3.172540in}{2.978043in}}{\pgfqpoint{3.161490in}{2.978043in}}%
\pgfpathcurveto{\pgfqpoint{3.150440in}{2.978043in}}{\pgfqpoint{3.139841in}{2.973653in}}{\pgfqpoint{3.132027in}{2.965839in}}%
\pgfpathcurveto{\pgfqpoint{3.124213in}{2.958025in}}{\pgfqpoint{3.119823in}{2.947426in}}{\pgfqpoint{3.119823in}{2.936376in}}%
\pgfpathcurveto{\pgfqpoint{3.119823in}{2.925326in}}{\pgfqpoint{3.124213in}{2.914727in}}{\pgfqpoint{3.132027in}{2.906913in}}%
\pgfpathcurveto{\pgfqpoint{3.139841in}{2.899100in}}{\pgfqpoint{3.150440in}{2.894710in}}{\pgfqpoint{3.161490in}{2.894710in}}%
\pgfpathclose%
\pgfusepath{stroke,fill}%
\end{pgfscope}%
\begin{pgfscope}%
\pgfpathrectangle{\pgfqpoint{0.481978in}{0.331635in}}{\pgfqpoint{4.960000in}{3.696000in}}%
\pgfusepath{clip}%
\pgfsetbuttcap%
\pgfsetroundjoin%
\definecolor{currentfill}{rgb}{0.631373,0.788235,0.956863}%
\pgfsetfillcolor{currentfill}%
\pgfsetlinewidth{0.481800pt}%
\definecolor{currentstroke}{rgb}{1.000000,1.000000,1.000000}%
\pgfsetstrokecolor{currentstroke}%
\pgfsetdash{}{0pt}%
\pgfpathmoveto{\pgfqpoint{3.406915in}{1.797065in}}%
\pgfpathcurveto{\pgfqpoint{3.417965in}{1.797065in}}{\pgfqpoint{3.428564in}{1.801456in}}{\pgfqpoint{3.436378in}{1.809269in}}%
\pgfpathcurveto{\pgfqpoint{3.444192in}{1.817083in}}{\pgfqpoint{3.448582in}{1.827682in}}{\pgfqpoint{3.448582in}{1.838732in}}%
\pgfpathcurveto{\pgfqpoint{3.448582in}{1.849782in}}{\pgfqpoint{3.444192in}{1.860381in}}{\pgfqpoint{3.436378in}{1.868195in}}%
\pgfpathcurveto{\pgfqpoint{3.428564in}{1.876009in}}{\pgfqpoint{3.417965in}{1.880399in}}{\pgfqpoint{3.406915in}{1.880399in}}%
\pgfpathcurveto{\pgfqpoint{3.395865in}{1.880399in}}{\pgfqpoint{3.385266in}{1.876009in}}{\pgfqpoint{3.377452in}{1.868195in}}%
\pgfpathcurveto{\pgfqpoint{3.369639in}{1.860381in}}{\pgfqpoint{3.365248in}{1.849782in}}{\pgfqpoint{3.365248in}{1.838732in}}%
\pgfpathcurveto{\pgfqpoint{3.365248in}{1.827682in}}{\pgfqpoint{3.369639in}{1.817083in}}{\pgfqpoint{3.377452in}{1.809269in}}%
\pgfpathcurveto{\pgfqpoint{3.385266in}{1.801456in}}{\pgfqpoint{3.395865in}{1.797065in}}{\pgfqpoint{3.406915in}{1.797065in}}%
\pgfpathclose%
\pgfusepath{stroke,fill}%
\end{pgfscope}%
\begin{pgfscope}%
\pgfpathrectangle{\pgfqpoint{0.481978in}{0.331635in}}{\pgfqpoint{4.960000in}{3.696000in}}%
\pgfusepath{clip}%
\pgfsetbuttcap%
\pgfsetroundjoin%
\definecolor{currentfill}{rgb}{0.631373,0.788235,0.956863}%
\pgfsetfillcolor{currentfill}%
\pgfsetlinewidth{0.481800pt}%
\definecolor{currentstroke}{rgb}{1.000000,1.000000,1.000000}%
\pgfsetstrokecolor{currentstroke}%
\pgfsetdash{}{0pt}%
\pgfpathmoveto{\pgfqpoint{1.632626in}{2.698726in}}%
\pgfpathcurveto{\pgfqpoint{1.643676in}{2.698726in}}{\pgfqpoint{1.654275in}{2.703116in}}{\pgfqpoint{1.662088in}{2.710930in}}%
\pgfpathcurveto{\pgfqpoint{1.669902in}{2.718743in}}{\pgfqpoint{1.674292in}{2.729342in}}{\pgfqpoint{1.674292in}{2.740393in}}%
\pgfpathcurveto{\pgfqpoint{1.674292in}{2.751443in}}{\pgfqpoint{1.669902in}{2.762042in}}{\pgfqpoint{1.662088in}{2.769855in}}%
\pgfpathcurveto{\pgfqpoint{1.654275in}{2.777669in}}{\pgfqpoint{1.643676in}{2.782059in}}{\pgfqpoint{1.632626in}{2.782059in}}%
\pgfpathcurveto{\pgfqpoint{1.621575in}{2.782059in}}{\pgfqpoint{1.610976in}{2.777669in}}{\pgfqpoint{1.603163in}{2.769855in}}%
\pgfpathcurveto{\pgfqpoint{1.595349in}{2.762042in}}{\pgfqpoint{1.590959in}{2.751443in}}{\pgfqpoint{1.590959in}{2.740393in}}%
\pgfpathcurveto{\pgfqpoint{1.590959in}{2.729342in}}{\pgfqpoint{1.595349in}{2.718743in}}{\pgfqpoint{1.603163in}{2.710930in}}%
\pgfpathcurveto{\pgfqpoint{1.610976in}{2.703116in}}{\pgfqpoint{1.621575in}{2.698726in}}{\pgfqpoint{1.632626in}{2.698726in}}%
\pgfpathclose%
\pgfusepath{stroke,fill}%
\end{pgfscope}%
\begin{pgfscope}%
\pgfpathrectangle{\pgfqpoint{0.481978in}{0.331635in}}{\pgfqpoint{4.960000in}{3.696000in}}%
\pgfusepath{clip}%
\pgfsetbuttcap%
\pgfsetroundjoin%
\definecolor{currentfill}{rgb}{0.631373,0.788235,0.956863}%
\pgfsetfillcolor{currentfill}%
\pgfsetlinewidth{0.481800pt}%
\definecolor{currentstroke}{rgb}{1.000000,1.000000,1.000000}%
\pgfsetstrokecolor{currentstroke}%
\pgfsetdash{}{0pt}%
\pgfpathmoveto{\pgfqpoint{4.178469in}{2.391567in}}%
\pgfpathcurveto{\pgfqpoint{4.189519in}{2.391567in}}{\pgfqpoint{4.200118in}{2.395957in}}{\pgfqpoint{4.207932in}{2.403771in}}%
\pgfpathcurveto{\pgfqpoint{4.215745in}{2.411585in}}{\pgfqpoint{4.220136in}{2.422184in}}{\pgfqpoint{4.220136in}{2.433234in}}%
\pgfpathcurveto{\pgfqpoint{4.220136in}{2.444284in}}{\pgfqpoint{4.215745in}{2.454883in}}{\pgfqpoint{4.207932in}{2.462697in}}%
\pgfpathcurveto{\pgfqpoint{4.200118in}{2.470510in}}{\pgfqpoint{4.189519in}{2.474901in}}{\pgfqpoint{4.178469in}{2.474901in}}%
\pgfpathcurveto{\pgfqpoint{4.167419in}{2.474901in}}{\pgfqpoint{4.156820in}{2.470510in}}{\pgfqpoint{4.149006in}{2.462697in}}%
\pgfpathcurveto{\pgfqpoint{4.141193in}{2.454883in}}{\pgfqpoint{4.136802in}{2.444284in}}{\pgfqpoint{4.136802in}{2.433234in}}%
\pgfpathcurveto{\pgfqpoint{4.136802in}{2.422184in}}{\pgfqpoint{4.141193in}{2.411585in}}{\pgfqpoint{4.149006in}{2.403771in}}%
\pgfpathcurveto{\pgfqpoint{4.156820in}{2.395957in}}{\pgfqpoint{4.167419in}{2.391567in}}{\pgfqpoint{4.178469in}{2.391567in}}%
\pgfpathclose%
\pgfusepath{stroke,fill}%
\end{pgfscope}%
\begin{pgfscope}%
\pgfpathrectangle{\pgfqpoint{0.481978in}{0.331635in}}{\pgfqpoint{4.960000in}{3.696000in}}%
\pgfusepath{clip}%
\pgfsetbuttcap%
\pgfsetroundjoin%
\definecolor{currentfill}{rgb}{0.631373,0.788235,0.956863}%
\pgfsetfillcolor{currentfill}%
\pgfsetlinewidth{0.481800pt}%
\definecolor{currentstroke}{rgb}{1.000000,1.000000,1.000000}%
\pgfsetstrokecolor{currentstroke}%
\pgfsetdash{}{0pt}%
\pgfpathmoveto{\pgfqpoint{4.067149in}{3.523980in}}%
\pgfpathcurveto{\pgfqpoint{4.078199in}{3.523980in}}{\pgfqpoint{4.088798in}{3.528370in}}{\pgfqpoint{4.096612in}{3.536183in}}%
\pgfpathcurveto{\pgfqpoint{4.104426in}{3.543997in}}{\pgfqpoint{4.108816in}{3.554596in}}{\pgfqpoint{4.108816in}{3.565646in}}%
\pgfpathcurveto{\pgfqpoint{4.108816in}{3.576696in}}{\pgfqpoint{4.104426in}{3.587295in}}{\pgfqpoint{4.096612in}{3.595109in}}%
\pgfpathcurveto{\pgfqpoint{4.088798in}{3.602923in}}{\pgfqpoint{4.078199in}{3.607313in}}{\pgfqpoint{4.067149in}{3.607313in}}%
\pgfpathcurveto{\pgfqpoint{4.056099in}{3.607313in}}{\pgfqpoint{4.045500in}{3.602923in}}{\pgfqpoint{4.037686in}{3.595109in}}%
\pgfpathcurveto{\pgfqpoint{4.029873in}{3.587295in}}{\pgfqpoint{4.025483in}{3.576696in}}{\pgfqpoint{4.025483in}{3.565646in}}%
\pgfpathcurveto{\pgfqpoint{4.025483in}{3.554596in}}{\pgfqpoint{4.029873in}{3.543997in}}{\pgfqpoint{4.037686in}{3.536183in}}%
\pgfpathcurveto{\pgfqpoint{4.045500in}{3.528370in}}{\pgfqpoint{4.056099in}{3.523980in}}{\pgfqpoint{4.067149in}{3.523980in}}%
\pgfpathclose%
\pgfusepath{stroke,fill}%
\end{pgfscope}%
\begin{pgfscope}%
\pgfpathrectangle{\pgfqpoint{0.481978in}{0.331635in}}{\pgfqpoint{4.960000in}{3.696000in}}%
\pgfusepath{clip}%
\pgfsetbuttcap%
\pgfsetroundjoin%
\definecolor{currentfill}{rgb}{0.631373,0.788235,0.956863}%
\pgfsetfillcolor{currentfill}%
\pgfsetlinewidth{0.481800pt}%
\definecolor{currentstroke}{rgb}{1.000000,1.000000,1.000000}%
\pgfsetstrokecolor{currentstroke}%
\pgfsetdash{}{0pt}%
\pgfpathmoveto{\pgfqpoint{3.848878in}{2.413780in}}%
\pgfpathcurveto{\pgfqpoint{3.859928in}{2.413780in}}{\pgfqpoint{3.870527in}{2.418170in}}{\pgfqpoint{3.878341in}{2.425984in}}%
\pgfpathcurveto{\pgfqpoint{3.886154in}{2.433798in}}{\pgfqpoint{3.890544in}{2.444397in}}{\pgfqpoint{3.890544in}{2.455447in}}%
\pgfpathcurveto{\pgfqpoint{3.890544in}{2.466497in}}{\pgfqpoint{3.886154in}{2.477096in}}{\pgfqpoint{3.878341in}{2.484910in}}%
\pgfpathcurveto{\pgfqpoint{3.870527in}{2.492723in}}{\pgfqpoint{3.859928in}{2.497113in}}{\pgfqpoint{3.848878in}{2.497113in}}%
\pgfpathcurveto{\pgfqpoint{3.837828in}{2.497113in}}{\pgfqpoint{3.827229in}{2.492723in}}{\pgfqpoint{3.819415in}{2.484910in}}%
\pgfpathcurveto{\pgfqpoint{3.811601in}{2.477096in}}{\pgfqpoint{3.807211in}{2.466497in}}{\pgfqpoint{3.807211in}{2.455447in}}%
\pgfpathcurveto{\pgfqpoint{3.807211in}{2.444397in}}{\pgfqpoint{3.811601in}{2.433798in}}{\pgfqpoint{3.819415in}{2.425984in}}%
\pgfpathcurveto{\pgfqpoint{3.827229in}{2.418170in}}{\pgfqpoint{3.837828in}{2.413780in}}{\pgfqpoint{3.848878in}{2.413780in}}%
\pgfpathclose%
\pgfusepath{stroke,fill}%
\end{pgfscope}%
\begin{pgfscope}%
\pgfpathrectangle{\pgfqpoint{0.481978in}{0.331635in}}{\pgfqpoint{4.960000in}{3.696000in}}%
\pgfusepath{clip}%
\pgfsetbuttcap%
\pgfsetroundjoin%
\definecolor{currentfill}{rgb}{0.631373,0.788235,0.956863}%
\pgfsetfillcolor{currentfill}%
\pgfsetlinewidth{0.481800pt}%
\definecolor{currentstroke}{rgb}{1.000000,1.000000,1.000000}%
\pgfsetstrokecolor{currentstroke}%
\pgfsetdash{}{0pt}%
\pgfpathmoveto{\pgfqpoint{4.547031in}{1.653861in}}%
\pgfpathcurveto{\pgfqpoint{4.558081in}{1.653861in}}{\pgfqpoint{4.568680in}{1.658252in}}{\pgfqpoint{4.576494in}{1.666065in}}%
\pgfpathcurveto{\pgfqpoint{4.584307in}{1.673879in}}{\pgfqpoint{4.588698in}{1.684478in}}{\pgfqpoint{4.588698in}{1.695528in}}%
\pgfpathcurveto{\pgfqpoint{4.588698in}{1.706578in}}{\pgfqpoint{4.584307in}{1.717177in}}{\pgfqpoint{4.576494in}{1.724991in}}%
\pgfpathcurveto{\pgfqpoint{4.568680in}{1.732804in}}{\pgfqpoint{4.558081in}{1.737195in}}{\pgfqpoint{4.547031in}{1.737195in}}%
\pgfpathcurveto{\pgfqpoint{4.535981in}{1.737195in}}{\pgfqpoint{4.525382in}{1.732804in}}{\pgfqpoint{4.517568in}{1.724991in}}%
\pgfpathcurveto{\pgfqpoint{4.509755in}{1.717177in}}{\pgfqpoint{4.505364in}{1.706578in}}{\pgfqpoint{4.505364in}{1.695528in}}%
\pgfpathcurveto{\pgfqpoint{4.505364in}{1.684478in}}{\pgfqpoint{4.509755in}{1.673879in}}{\pgfqpoint{4.517568in}{1.666065in}}%
\pgfpathcurveto{\pgfqpoint{4.525382in}{1.658252in}}{\pgfqpoint{4.535981in}{1.653861in}}{\pgfqpoint{4.547031in}{1.653861in}}%
\pgfpathclose%
\pgfusepath{stroke,fill}%
\end{pgfscope}%
\begin{pgfscope}%
\pgfpathrectangle{\pgfqpoint{0.481978in}{0.331635in}}{\pgfqpoint{4.960000in}{3.696000in}}%
\pgfusepath{clip}%
\pgfsetbuttcap%
\pgfsetroundjoin%
\definecolor{currentfill}{rgb}{0.631373,0.788235,0.956863}%
\pgfsetfillcolor{currentfill}%
\pgfsetlinewidth{0.481800pt}%
\definecolor{currentstroke}{rgb}{1.000000,1.000000,1.000000}%
\pgfsetstrokecolor{currentstroke}%
\pgfsetdash{}{0pt}%
\pgfpathmoveto{\pgfqpoint{4.126614in}{1.710655in}}%
\pgfpathcurveto{\pgfqpoint{4.137665in}{1.710655in}}{\pgfqpoint{4.148264in}{1.715046in}}{\pgfqpoint{4.156077in}{1.722859in}}%
\pgfpathcurveto{\pgfqpoint{4.163891in}{1.730673in}}{\pgfqpoint{4.168281in}{1.741272in}}{\pgfqpoint{4.168281in}{1.752322in}}%
\pgfpathcurveto{\pgfqpoint{4.168281in}{1.763372in}}{\pgfqpoint{4.163891in}{1.773971in}}{\pgfqpoint{4.156077in}{1.781785in}}%
\pgfpathcurveto{\pgfqpoint{4.148264in}{1.789599in}}{\pgfqpoint{4.137665in}{1.793989in}}{\pgfqpoint{4.126614in}{1.793989in}}%
\pgfpathcurveto{\pgfqpoint{4.115564in}{1.793989in}}{\pgfqpoint{4.104965in}{1.789599in}}{\pgfqpoint{4.097152in}{1.781785in}}%
\pgfpathcurveto{\pgfqpoint{4.089338in}{1.773971in}}{\pgfqpoint{4.084948in}{1.763372in}}{\pgfqpoint{4.084948in}{1.752322in}}%
\pgfpathcurveto{\pgfqpoint{4.084948in}{1.741272in}}{\pgfqpoint{4.089338in}{1.730673in}}{\pgfqpoint{4.097152in}{1.722859in}}%
\pgfpathcurveto{\pgfqpoint{4.104965in}{1.715046in}}{\pgfqpoint{4.115564in}{1.710655in}}{\pgfqpoint{4.126614in}{1.710655in}}%
\pgfpathclose%
\pgfusepath{stroke,fill}%
\end{pgfscope}%
\begin{pgfscope}%
\pgfpathrectangle{\pgfqpoint{0.481978in}{0.331635in}}{\pgfqpoint{4.960000in}{3.696000in}}%
\pgfusepath{clip}%
\pgfsetbuttcap%
\pgfsetroundjoin%
\definecolor{currentfill}{rgb}{0.631373,0.788235,0.956863}%
\pgfsetfillcolor{currentfill}%
\pgfsetlinewidth{0.481800pt}%
\definecolor{currentstroke}{rgb}{1.000000,1.000000,1.000000}%
\pgfsetstrokecolor{currentstroke}%
\pgfsetdash{}{0pt}%
\pgfpathmoveto{\pgfqpoint{2.866000in}{1.887449in}}%
\pgfpathcurveto{\pgfqpoint{2.877050in}{1.887449in}}{\pgfqpoint{2.887649in}{1.891840in}}{\pgfqpoint{2.895463in}{1.899653in}}%
\pgfpathcurveto{\pgfqpoint{2.903277in}{1.907467in}}{\pgfqpoint{2.907667in}{1.918066in}}{\pgfqpoint{2.907667in}{1.929116in}}%
\pgfpathcurveto{\pgfqpoint{2.907667in}{1.940166in}}{\pgfqpoint{2.903277in}{1.950765in}}{\pgfqpoint{2.895463in}{1.958579in}}%
\pgfpathcurveto{\pgfqpoint{2.887649in}{1.966392in}}{\pgfqpoint{2.877050in}{1.970783in}}{\pgfqpoint{2.866000in}{1.970783in}}%
\pgfpathcurveto{\pgfqpoint{2.854950in}{1.970783in}}{\pgfqpoint{2.844351in}{1.966392in}}{\pgfqpoint{2.836538in}{1.958579in}}%
\pgfpathcurveto{\pgfqpoint{2.828724in}{1.950765in}}{\pgfqpoint{2.824334in}{1.940166in}}{\pgfqpoint{2.824334in}{1.929116in}}%
\pgfpathcurveto{\pgfqpoint{2.824334in}{1.918066in}}{\pgfqpoint{2.828724in}{1.907467in}}{\pgfqpoint{2.836538in}{1.899653in}}%
\pgfpathcurveto{\pgfqpoint{2.844351in}{1.891840in}}{\pgfqpoint{2.854950in}{1.887449in}}{\pgfqpoint{2.866000in}{1.887449in}}%
\pgfpathclose%
\pgfusepath{stroke,fill}%
\end{pgfscope}%
\begin{pgfscope}%
\pgfpathrectangle{\pgfqpoint{0.481978in}{0.331635in}}{\pgfqpoint{4.960000in}{3.696000in}}%
\pgfusepath{clip}%
\pgfsetbuttcap%
\pgfsetroundjoin%
\definecolor{currentfill}{rgb}{0.631373,0.788235,0.956863}%
\pgfsetfillcolor{currentfill}%
\pgfsetlinewidth{0.481800pt}%
\definecolor{currentstroke}{rgb}{1.000000,1.000000,1.000000}%
\pgfsetstrokecolor{currentstroke}%
\pgfsetdash{}{0pt}%
\pgfpathmoveto{\pgfqpoint{3.835758in}{2.599830in}}%
\pgfpathcurveto{\pgfqpoint{3.846809in}{2.599830in}}{\pgfqpoint{3.857408in}{2.604220in}}{\pgfqpoint{3.865221in}{2.612034in}}%
\pgfpathcurveto{\pgfqpoint{3.873035in}{2.619848in}}{\pgfqpoint{3.877425in}{2.630447in}}{\pgfqpoint{3.877425in}{2.641497in}}%
\pgfpathcurveto{\pgfqpoint{3.877425in}{2.652547in}}{\pgfqpoint{3.873035in}{2.663146in}}{\pgfqpoint{3.865221in}{2.670960in}}%
\pgfpathcurveto{\pgfqpoint{3.857408in}{2.678773in}}{\pgfqpoint{3.846809in}{2.683163in}}{\pgfqpoint{3.835758in}{2.683163in}}%
\pgfpathcurveto{\pgfqpoint{3.824708in}{2.683163in}}{\pgfqpoint{3.814109in}{2.678773in}}{\pgfqpoint{3.806296in}{2.670960in}}%
\pgfpathcurveto{\pgfqpoint{3.798482in}{2.663146in}}{\pgfqpoint{3.794092in}{2.652547in}}{\pgfqpoint{3.794092in}{2.641497in}}%
\pgfpathcurveto{\pgfqpoint{3.794092in}{2.630447in}}{\pgfqpoint{3.798482in}{2.619848in}}{\pgfqpoint{3.806296in}{2.612034in}}%
\pgfpathcurveto{\pgfqpoint{3.814109in}{2.604220in}}{\pgfqpoint{3.824708in}{2.599830in}}{\pgfqpoint{3.835758in}{2.599830in}}%
\pgfpathclose%
\pgfusepath{stroke,fill}%
\end{pgfscope}%
\begin{pgfscope}%
\pgfpathrectangle{\pgfqpoint{0.481978in}{0.331635in}}{\pgfqpoint{4.960000in}{3.696000in}}%
\pgfusepath{clip}%
\pgfsetbuttcap%
\pgfsetroundjoin%
\definecolor{currentfill}{rgb}{0.631373,0.788235,0.956863}%
\pgfsetfillcolor{currentfill}%
\pgfsetlinewidth{0.481800pt}%
\definecolor{currentstroke}{rgb}{1.000000,1.000000,1.000000}%
\pgfsetstrokecolor{currentstroke}%
\pgfsetdash{}{0pt}%
\pgfpathmoveto{\pgfqpoint{1.829229in}{2.163381in}}%
\pgfpathcurveto{\pgfqpoint{1.840280in}{2.163381in}}{\pgfqpoint{1.850879in}{2.167771in}}{\pgfqpoint{1.858692in}{2.175585in}}%
\pgfpathcurveto{\pgfqpoint{1.866506in}{2.183398in}}{\pgfqpoint{1.870896in}{2.193997in}}{\pgfqpoint{1.870896in}{2.205047in}}%
\pgfpathcurveto{\pgfqpoint{1.870896in}{2.216098in}}{\pgfqpoint{1.866506in}{2.226697in}}{\pgfqpoint{1.858692in}{2.234510in}}%
\pgfpathcurveto{\pgfqpoint{1.850879in}{2.242324in}}{\pgfqpoint{1.840280in}{2.246714in}}{\pgfqpoint{1.829229in}{2.246714in}}%
\pgfpathcurveto{\pgfqpoint{1.818179in}{2.246714in}}{\pgfqpoint{1.807580in}{2.242324in}}{\pgfqpoint{1.799767in}{2.234510in}}%
\pgfpathcurveto{\pgfqpoint{1.791953in}{2.226697in}}{\pgfqpoint{1.787563in}{2.216098in}}{\pgfqpoint{1.787563in}{2.205047in}}%
\pgfpathcurveto{\pgfqpoint{1.787563in}{2.193997in}}{\pgfqpoint{1.791953in}{2.183398in}}{\pgfqpoint{1.799767in}{2.175585in}}%
\pgfpathcurveto{\pgfqpoint{1.807580in}{2.167771in}}{\pgfqpoint{1.818179in}{2.163381in}}{\pgfqpoint{1.829229in}{2.163381in}}%
\pgfpathclose%
\pgfusepath{stroke,fill}%
\end{pgfscope}%
\begin{pgfscope}%
\pgfpathrectangle{\pgfqpoint{0.481978in}{0.331635in}}{\pgfqpoint{4.960000in}{3.696000in}}%
\pgfusepath{clip}%
\pgfsetbuttcap%
\pgfsetroundjoin%
\definecolor{currentfill}{rgb}{0.631373,0.788235,0.956863}%
\pgfsetfillcolor{currentfill}%
\pgfsetlinewidth{0.481800pt}%
\definecolor{currentstroke}{rgb}{1.000000,1.000000,1.000000}%
\pgfsetstrokecolor{currentstroke}%
\pgfsetdash{}{0pt}%
\pgfpathmoveto{\pgfqpoint{3.218175in}{1.933624in}}%
\pgfpathcurveto{\pgfqpoint{3.229225in}{1.933624in}}{\pgfqpoint{3.239824in}{1.938014in}}{\pgfqpoint{3.247638in}{1.945828in}}%
\pgfpathcurveto{\pgfqpoint{3.255452in}{1.953642in}}{\pgfqpoint{3.259842in}{1.964241in}}{\pgfqpoint{3.259842in}{1.975291in}}%
\pgfpathcurveto{\pgfqpoint{3.259842in}{1.986341in}}{\pgfqpoint{3.255452in}{1.996940in}}{\pgfqpoint{3.247638in}{2.004754in}}%
\pgfpathcurveto{\pgfqpoint{3.239824in}{2.012567in}}{\pgfqpoint{3.229225in}{2.016958in}}{\pgfqpoint{3.218175in}{2.016958in}}%
\pgfpathcurveto{\pgfqpoint{3.207125in}{2.016958in}}{\pgfqpoint{3.196526in}{2.012567in}}{\pgfqpoint{3.188713in}{2.004754in}}%
\pgfpathcurveto{\pgfqpoint{3.180899in}{1.996940in}}{\pgfqpoint{3.176509in}{1.986341in}}{\pgfqpoint{3.176509in}{1.975291in}}%
\pgfpathcurveto{\pgfqpoint{3.176509in}{1.964241in}}{\pgfqpoint{3.180899in}{1.953642in}}{\pgfqpoint{3.188713in}{1.945828in}}%
\pgfpathcurveto{\pgfqpoint{3.196526in}{1.938014in}}{\pgfqpoint{3.207125in}{1.933624in}}{\pgfqpoint{3.218175in}{1.933624in}}%
\pgfpathclose%
\pgfusepath{stroke,fill}%
\end{pgfscope}%
\begin{pgfscope}%
\pgfpathrectangle{\pgfqpoint{0.481978in}{0.331635in}}{\pgfqpoint{4.960000in}{3.696000in}}%
\pgfusepath{clip}%
\pgfsetbuttcap%
\pgfsetroundjoin%
\definecolor{currentfill}{rgb}{0.631373,0.788235,0.956863}%
\pgfsetfillcolor{currentfill}%
\pgfsetlinewidth{0.481800pt}%
\definecolor{currentstroke}{rgb}{1.000000,1.000000,1.000000}%
\pgfsetstrokecolor{currentstroke}%
\pgfsetdash{}{0pt}%
\pgfpathmoveto{\pgfqpoint{3.143077in}{2.884819in}}%
\pgfpathcurveto{\pgfqpoint{3.154127in}{2.884819in}}{\pgfqpoint{3.164726in}{2.889209in}}{\pgfqpoint{3.172540in}{2.897022in}}%
\pgfpathcurveto{\pgfqpoint{3.180353in}{2.904836in}}{\pgfqpoint{3.184744in}{2.915435in}}{\pgfqpoint{3.184744in}{2.926485in}}%
\pgfpathcurveto{\pgfqpoint{3.184744in}{2.937535in}}{\pgfqpoint{3.180353in}{2.948134in}}{\pgfqpoint{3.172540in}{2.955948in}}%
\pgfpathcurveto{\pgfqpoint{3.164726in}{2.963762in}}{\pgfqpoint{3.154127in}{2.968152in}}{\pgfqpoint{3.143077in}{2.968152in}}%
\pgfpathcurveto{\pgfqpoint{3.132027in}{2.968152in}}{\pgfqpoint{3.121428in}{2.963762in}}{\pgfqpoint{3.113614in}{2.955948in}}%
\pgfpathcurveto{\pgfqpoint{3.105800in}{2.948134in}}{\pgfqpoint{3.101410in}{2.937535in}}{\pgfqpoint{3.101410in}{2.926485in}}%
\pgfpathcurveto{\pgfqpoint{3.101410in}{2.915435in}}{\pgfqpoint{3.105800in}{2.904836in}}{\pgfqpoint{3.113614in}{2.897022in}}%
\pgfpathcurveto{\pgfqpoint{3.121428in}{2.889209in}}{\pgfqpoint{3.132027in}{2.884819in}}{\pgfqpoint{3.143077in}{2.884819in}}%
\pgfpathclose%
\pgfusepath{stroke,fill}%
\end{pgfscope}%
\begin{pgfscope}%
\pgfpathrectangle{\pgfqpoint{0.481978in}{0.331635in}}{\pgfqpoint{4.960000in}{3.696000in}}%
\pgfusepath{clip}%
\pgfsetbuttcap%
\pgfsetroundjoin%
\definecolor{currentfill}{rgb}{0.631373,0.788235,0.956863}%
\pgfsetfillcolor{currentfill}%
\pgfsetlinewidth{0.481800pt}%
\definecolor{currentstroke}{rgb}{1.000000,1.000000,1.000000}%
\pgfsetstrokecolor{currentstroke}%
\pgfsetdash{}{0pt}%
\pgfpathmoveto{\pgfqpoint{2.735315in}{3.557061in}}%
\pgfpathcurveto{\pgfqpoint{2.746365in}{3.557061in}}{\pgfqpoint{2.756965in}{3.561451in}}{\pgfqpoint{2.764778in}{3.569264in}}%
\pgfpathcurveto{\pgfqpoint{2.772592in}{3.577078in}}{\pgfqpoint{2.776982in}{3.587677in}}{\pgfqpoint{2.776982in}{3.598727in}}%
\pgfpathcurveto{\pgfqpoint{2.776982in}{3.609777in}}{\pgfqpoint{2.772592in}{3.620376in}}{\pgfqpoint{2.764778in}{3.628190in}}%
\pgfpathcurveto{\pgfqpoint{2.756965in}{3.636004in}}{\pgfqpoint{2.746365in}{3.640394in}}{\pgfqpoint{2.735315in}{3.640394in}}%
\pgfpathcurveto{\pgfqpoint{2.724265in}{3.640394in}}{\pgfqpoint{2.713666in}{3.636004in}}{\pgfqpoint{2.705853in}{3.628190in}}%
\pgfpathcurveto{\pgfqpoint{2.698039in}{3.620376in}}{\pgfqpoint{2.693649in}{3.609777in}}{\pgfqpoint{2.693649in}{3.598727in}}%
\pgfpathcurveto{\pgfqpoint{2.693649in}{3.587677in}}{\pgfqpoint{2.698039in}{3.577078in}}{\pgfqpoint{2.705853in}{3.569264in}}%
\pgfpathcurveto{\pgfqpoint{2.713666in}{3.561451in}}{\pgfqpoint{2.724265in}{3.557061in}}{\pgfqpoint{2.735315in}{3.557061in}}%
\pgfpathclose%
\pgfusepath{stroke,fill}%
\end{pgfscope}%
\begin{pgfscope}%
\pgfpathrectangle{\pgfqpoint{0.481978in}{0.331635in}}{\pgfqpoint{4.960000in}{3.696000in}}%
\pgfusepath{clip}%
\pgfsetbuttcap%
\pgfsetroundjoin%
\definecolor{currentfill}{rgb}{0.631373,0.788235,0.956863}%
\pgfsetfillcolor{currentfill}%
\pgfsetlinewidth{0.481800pt}%
\definecolor{currentstroke}{rgb}{1.000000,1.000000,1.000000}%
\pgfsetstrokecolor{currentstroke}%
\pgfsetdash{}{0pt}%
\pgfpathmoveto{\pgfqpoint{2.631413in}{3.257032in}}%
\pgfpathcurveto{\pgfqpoint{2.642463in}{3.257032in}}{\pgfqpoint{2.653062in}{3.261422in}}{\pgfqpoint{2.660875in}{3.269236in}}%
\pgfpathcurveto{\pgfqpoint{2.668689in}{3.277049in}}{\pgfqpoint{2.673079in}{3.287648in}}{\pgfqpoint{2.673079in}{3.298699in}}%
\pgfpathcurveto{\pgfqpoint{2.673079in}{3.309749in}}{\pgfqpoint{2.668689in}{3.320348in}}{\pgfqpoint{2.660875in}{3.328161in}}%
\pgfpathcurveto{\pgfqpoint{2.653062in}{3.335975in}}{\pgfqpoint{2.642463in}{3.340365in}}{\pgfqpoint{2.631413in}{3.340365in}}%
\pgfpathcurveto{\pgfqpoint{2.620363in}{3.340365in}}{\pgfqpoint{2.609763in}{3.335975in}}{\pgfqpoint{2.601950in}{3.328161in}}%
\pgfpathcurveto{\pgfqpoint{2.594136in}{3.320348in}}{\pgfqpoint{2.589746in}{3.309749in}}{\pgfqpoint{2.589746in}{3.298699in}}%
\pgfpathcurveto{\pgfqpoint{2.589746in}{3.287648in}}{\pgfqpoint{2.594136in}{3.277049in}}{\pgfqpoint{2.601950in}{3.269236in}}%
\pgfpathcurveto{\pgfqpoint{2.609763in}{3.261422in}}{\pgfqpoint{2.620363in}{3.257032in}}{\pgfqpoint{2.631413in}{3.257032in}}%
\pgfpathclose%
\pgfusepath{stroke,fill}%
\end{pgfscope}%
\begin{pgfscope}%
\pgfpathrectangle{\pgfqpoint{0.481978in}{0.331635in}}{\pgfqpoint{4.960000in}{3.696000in}}%
\pgfusepath{clip}%
\pgfsetbuttcap%
\pgfsetroundjoin%
\definecolor{currentfill}{rgb}{0.631373,0.788235,0.956863}%
\pgfsetfillcolor{currentfill}%
\pgfsetlinewidth{0.481800pt}%
\definecolor{currentstroke}{rgb}{1.000000,1.000000,1.000000}%
\pgfsetstrokecolor{currentstroke}%
\pgfsetdash{}{0pt}%
\pgfpathmoveto{\pgfqpoint{2.918834in}{3.165808in}}%
\pgfpathcurveto{\pgfqpoint{2.929884in}{3.165808in}}{\pgfqpoint{2.940483in}{3.170198in}}{\pgfqpoint{2.948297in}{3.178012in}}%
\pgfpathcurveto{\pgfqpoint{2.956111in}{3.185826in}}{\pgfqpoint{2.960501in}{3.196425in}}{\pgfqpoint{2.960501in}{3.207475in}}%
\pgfpathcurveto{\pgfqpoint{2.960501in}{3.218525in}}{\pgfqpoint{2.956111in}{3.229124in}}{\pgfqpoint{2.948297in}{3.236938in}}%
\pgfpathcurveto{\pgfqpoint{2.940483in}{3.244751in}}{\pgfqpoint{2.929884in}{3.249141in}}{\pgfqpoint{2.918834in}{3.249141in}}%
\pgfpathcurveto{\pgfqpoint{2.907784in}{3.249141in}}{\pgfqpoint{2.897185in}{3.244751in}}{\pgfqpoint{2.889371in}{3.236938in}}%
\pgfpathcurveto{\pgfqpoint{2.881558in}{3.229124in}}{\pgfqpoint{2.877167in}{3.218525in}}{\pgfqpoint{2.877167in}{3.207475in}}%
\pgfpathcurveto{\pgfqpoint{2.877167in}{3.196425in}}{\pgfqpoint{2.881558in}{3.185826in}}{\pgfqpoint{2.889371in}{3.178012in}}%
\pgfpathcurveto{\pgfqpoint{2.897185in}{3.170198in}}{\pgfqpoint{2.907784in}{3.165808in}}{\pgfqpoint{2.918834in}{3.165808in}}%
\pgfpathclose%
\pgfusepath{stroke,fill}%
\end{pgfscope}%
\begin{pgfscope}%
\pgfpathrectangle{\pgfqpoint{0.481978in}{0.331635in}}{\pgfqpoint{4.960000in}{3.696000in}}%
\pgfusepath{clip}%
\pgfsetbuttcap%
\pgfsetroundjoin%
\definecolor{currentfill}{rgb}{0.631373,0.788235,0.956863}%
\pgfsetfillcolor{currentfill}%
\pgfsetlinewidth{0.481800pt}%
\definecolor{currentstroke}{rgb}{1.000000,1.000000,1.000000}%
\pgfsetstrokecolor{currentstroke}%
\pgfsetdash{}{0pt}%
\pgfpathmoveto{\pgfqpoint{4.008265in}{3.062382in}}%
\pgfpathcurveto{\pgfqpoint{4.019315in}{3.062382in}}{\pgfqpoint{4.029914in}{3.066772in}}{\pgfqpoint{4.037728in}{3.074586in}}%
\pgfpathcurveto{\pgfqpoint{4.045542in}{3.082400in}}{\pgfqpoint{4.049932in}{3.092999in}}{\pgfqpoint{4.049932in}{3.104049in}}%
\pgfpathcurveto{\pgfqpoint{4.049932in}{3.115099in}}{\pgfqpoint{4.045542in}{3.125698in}}{\pgfqpoint{4.037728in}{3.133512in}}%
\pgfpathcurveto{\pgfqpoint{4.029914in}{3.141325in}}{\pgfqpoint{4.019315in}{3.145715in}}{\pgfqpoint{4.008265in}{3.145715in}}%
\pgfpathcurveto{\pgfqpoint{3.997215in}{3.145715in}}{\pgfqpoint{3.986616in}{3.141325in}}{\pgfqpoint{3.978802in}{3.133512in}}%
\pgfpathcurveto{\pgfqpoint{3.970989in}{3.125698in}}{\pgfqpoint{3.966599in}{3.115099in}}{\pgfqpoint{3.966599in}{3.104049in}}%
\pgfpathcurveto{\pgfqpoint{3.966599in}{3.092999in}}{\pgfqpoint{3.970989in}{3.082400in}}{\pgfqpoint{3.978802in}{3.074586in}}%
\pgfpathcurveto{\pgfqpoint{3.986616in}{3.066772in}}{\pgfqpoint{3.997215in}{3.062382in}}{\pgfqpoint{4.008265in}{3.062382in}}%
\pgfpathclose%
\pgfusepath{stroke,fill}%
\end{pgfscope}%
\begin{pgfscope}%
\pgfpathrectangle{\pgfqpoint{0.481978in}{0.331635in}}{\pgfqpoint{4.960000in}{3.696000in}}%
\pgfusepath{clip}%
\pgfsetbuttcap%
\pgfsetroundjoin%
\definecolor{currentfill}{rgb}{0.631373,0.788235,0.956863}%
\pgfsetfillcolor{currentfill}%
\pgfsetlinewidth{0.481800pt}%
\definecolor{currentstroke}{rgb}{1.000000,1.000000,1.000000}%
\pgfsetstrokecolor{currentstroke}%
\pgfsetdash{}{0pt}%
\pgfpathmoveto{\pgfqpoint{4.676024in}{3.348349in}}%
\pgfpathcurveto{\pgfqpoint{4.687074in}{3.348349in}}{\pgfqpoint{4.697673in}{3.352739in}}{\pgfqpoint{4.705486in}{3.360553in}}%
\pgfpathcurveto{\pgfqpoint{4.713300in}{3.368367in}}{\pgfqpoint{4.717690in}{3.378966in}}{\pgfqpoint{4.717690in}{3.390016in}}%
\pgfpathcurveto{\pgfqpoint{4.717690in}{3.401066in}}{\pgfqpoint{4.713300in}{3.411665in}}{\pgfqpoint{4.705486in}{3.419479in}}%
\pgfpathcurveto{\pgfqpoint{4.697673in}{3.427292in}}{\pgfqpoint{4.687074in}{3.431683in}}{\pgfqpoint{4.676024in}{3.431683in}}%
\pgfpathcurveto{\pgfqpoint{4.664974in}{3.431683in}}{\pgfqpoint{4.654375in}{3.427292in}}{\pgfqpoint{4.646561in}{3.419479in}}%
\pgfpathcurveto{\pgfqpoint{4.638747in}{3.411665in}}{\pgfqpoint{4.634357in}{3.401066in}}{\pgfqpoint{4.634357in}{3.390016in}}%
\pgfpathcurveto{\pgfqpoint{4.634357in}{3.378966in}}{\pgfqpoint{4.638747in}{3.368367in}}{\pgfqpoint{4.646561in}{3.360553in}}%
\pgfpathcurveto{\pgfqpoint{4.654375in}{3.352739in}}{\pgfqpoint{4.664974in}{3.348349in}}{\pgfqpoint{4.676024in}{3.348349in}}%
\pgfpathclose%
\pgfusepath{stroke,fill}%
\end{pgfscope}%
\begin{pgfscope}%
\pgfpathrectangle{\pgfqpoint{0.481978in}{0.331635in}}{\pgfqpoint{4.960000in}{3.696000in}}%
\pgfusepath{clip}%
\pgfsetbuttcap%
\pgfsetroundjoin%
\definecolor{currentfill}{rgb}{0.631373,0.788235,0.956863}%
\pgfsetfillcolor{currentfill}%
\pgfsetlinewidth{0.481800pt}%
\definecolor{currentstroke}{rgb}{1.000000,1.000000,1.000000}%
\pgfsetstrokecolor{currentstroke}%
\pgfsetdash{}{0pt}%
\pgfpathmoveto{\pgfqpoint{2.835001in}{2.658791in}}%
\pgfpathcurveto{\pgfqpoint{2.846051in}{2.658791in}}{\pgfqpoint{2.856650in}{2.663181in}}{\pgfqpoint{2.864464in}{2.670995in}}%
\pgfpathcurveto{\pgfqpoint{2.872277in}{2.678808in}}{\pgfqpoint{2.876668in}{2.689407in}}{\pgfqpoint{2.876668in}{2.700457in}}%
\pgfpathcurveto{\pgfqpoint{2.876668in}{2.711508in}}{\pgfqpoint{2.872277in}{2.722107in}}{\pgfqpoint{2.864464in}{2.729920in}}%
\pgfpathcurveto{\pgfqpoint{2.856650in}{2.737734in}}{\pgfqpoint{2.846051in}{2.742124in}}{\pgfqpoint{2.835001in}{2.742124in}}%
\pgfpathcurveto{\pgfqpoint{2.823951in}{2.742124in}}{\pgfqpoint{2.813352in}{2.737734in}}{\pgfqpoint{2.805538in}{2.729920in}}%
\pgfpathcurveto{\pgfqpoint{2.797725in}{2.722107in}}{\pgfqpoint{2.793334in}{2.711508in}}{\pgfqpoint{2.793334in}{2.700457in}}%
\pgfpathcurveto{\pgfqpoint{2.793334in}{2.689407in}}{\pgfqpoint{2.797725in}{2.678808in}}{\pgfqpoint{2.805538in}{2.670995in}}%
\pgfpathcurveto{\pgfqpoint{2.813352in}{2.663181in}}{\pgfqpoint{2.823951in}{2.658791in}}{\pgfqpoint{2.835001in}{2.658791in}}%
\pgfpathclose%
\pgfusepath{stroke,fill}%
\end{pgfscope}%
\begin{pgfscope}%
\pgfpathrectangle{\pgfqpoint{0.481978in}{0.331635in}}{\pgfqpoint{4.960000in}{3.696000in}}%
\pgfusepath{clip}%
\pgfsetbuttcap%
\pgfsetroundjoin%
\definecolor{currentfill}{rgb}{0.631373,0.788235,0.956863}%
\pgfsetfillcolor{currentfill}%
\pgfsetlinewidth{0.481800pt}%
\definecolor{currentstroke}{rgb}{1.000000,1.000000,1.000000}%
\pgfsetstrokecolor{currentstroke}%
\pgfsetdash{}{0pt}%
\pgfpathmoveto{\pgfqpoint{3.385446in}{3.243604in}}%
\pgfpathcurveto{\pgfqpoint{3.396496in}{3.243604in}}{\pgfqpoint{3.407095in}{3.247995in}}{\pgfqpoint{3.414908in}{3.255808in}}%
\pgfpathcurveto{\pgfqpoint{3.422722in}{3.263622in}}{\pgfqpoint{3.427112in}{3.274221in}}{\pgfqpoint{3.427112in}{3.285271in}}%
\pgfpathcurveto{\pgfqpoint{3.427112in}{3.296321in}}{\pgfqpoint{3.422722in}{3.306920in}}{\pgfqpoint{3.414908in}{3.314734in}}%
\pgfpathcurveto{\pgfqpoint{3.407095in}{3.322547in}}{\pgfqpoint{3.396496in}{3.326938in}}{\pgfqpoint{3.385446in}{3.326938in}}%
\pgfpathcurveto{\pgfqpoint{3.374395in}{3.326938in}}{\pgfqpoint{3.363796in}{3.322547in}}{\pgfqpoint{3.355983in}{3.314734in}}%
\pgfpathcurveto{\pgfqpoint{3.348169in}{3.306920in}}{\pgfqpoint{3.343779in}{3.296321in}}{\pgfqpoint{3.343779in}{3.285271in}}%
\pgfpathcurveto{\pgfqpoint{3.343779in}{3.274221in}}{\pgfqpoint{3.348169in}{3.263622in}}{\pgfqpoint{3.355983in}{3.255808in}}%
\pgfpathcurveto{\pgfqpoint{3.363796in}{3.247995in}}{\pgfqpoint{3.374395in}{3.243604in}}{\pgfqpoint{3.385446in}{3.243604in}}%
\pgfpathclose%
\pgfusepath{stroke,fill}%
\end{pgfscope}%
\begin{pgfscope}%
\pgfpathrectangle{\pgfqpoint{0.481978in}{0.331635in}}{\pgfqpoint{4.960000in}{3.696000in}}%
\pgfusepath{clip}%
\pgfsetbuttcap%
\pgfsetroundjoin%
\definecolor{currentfill}{rgb}{0.631373,0.788235,0.956863}%
\pgfsetfillcolor{currentfill}%
\pgfsetlinewidth{0.481800pt}%
\definecolor{currentstroke}{rgb}{1.000000,1.000000,1.000000}%
\pgfsetstrokecolor{currentstroke}%
\pgfsetdash{}{0pt}%
\pgfpathmoveto{\pgfqpoint{3.070415in}{1.825620in}}%
\pgfpathcurveto{\pgfqpoint{3.081465in}{1.825620in}}{\pgfqpoint{3.092064in}{1.830010in}}{\pgfqpoint{3.099878in}{1.837824in}}%
\pgfpathcurveto{\pgfqpoint{3.107691in}{1.845637in}}{\pgfqpoint{3.112082in}{1.856236in}}{\pgfqpoint{3.112082in}{1.867287in}}%
\pgfpathcurveto{\pgfqpoint{3.112082in}{1.878337in}}{\pgfqpoint{3.107691in}{1.888936in}}{\pgfqpoint{3.099878in}{1.896749in}}%
\pgfpathcurveto{\pgfqpoint{3.092064in}{1.904563in}}{\pgfqpoint{3.081465in}{1.908953in}}{\pgfqpoint{3.070415in}{1.908953in}}%
\pgfpathcurveto{\pgfqpoint{3.059365in}{1.908953in}}{\pgfqpoint{3.048766in}{1.904563in}}{\pgfqpoint{3.040952in}{1.896749in}}%
\pgfpathcurveto{\pgfqpoint{3.033138in}{1.888936in}}{\pgfqpoint{3.028748in}{1.878337in}}{\pgfqpoint{3.028748in}{1.867287in}}%
\pgfpathcurveto{\pgfqpoint{3.028748in}{1.856236in}}{\pgfqpoint{3.033138in}{1.845637in}}{\pgfqpoint{3.040952in}{1.837824in}}%
\pgfpathcurveto{\pgfqpoint{3.048766in}{1.830010in}}{\pgfqpoint{3.059365in}{1.825620in}}{\pgfqpoint{3.070415in}{1.825620in}}%
\pgfpathclose%
\pgfusepath{stroke,fill}%
\end{pgfscope}%
\begin{pgfscope}%
\pgfpathrectangle{\pgfqpoint{0.481978in}{0.331635in}}{\pgfqpoint{4.960000in}{3.696000in}}%
\pgfusepath{clip}%
\pgfsetbuttcap%
\pgfsetroundjoin%
\definecolor{currentfill}{rgb}{0.631373,0.788235,0.956863}%
\pgfsetfillcolor{currentfill}%
\pgfsetlinewidth{0.481800pt}%
\definecolor{currentstroke}{rgb}{1.000000,1.000000,1.000000}%
\pgfsetstrokecolor{currentstroke}%
\pgfsetdash{}{0pt}%
\pgfpathmoveto{\pgfqpoint{4.063842in}{3.443875in}}%
\pgfpathcurveto{\pgfqpoint{4.074893in}{3.443875in}}{\pgfqpoint{4.085492in}{3.448265in}}{\pgfqpoint{4.093305in}{3.456079in}}%
\pgfpathcurveto{\pgfqpoint{4.101119in}{3.463892in}}{\pgfqpoint{4.105509in}{3.474491in}}{\pgfqpoint{4.105509in}{3.485542in}}%
\pgfpathcurveto{\pgfqpoint{4.105509in}{3.496592in}}{\pgfqpoint{4.101119in}{3.507191in}}{\pgfqpoint{4.093305in}{3.515004in}}%
\pgfpathcurveto{\pgfqpoint{4.085492in}{3.522818in}}{\pgfqpoint{4.074893in}{3.527208in}}{\pgfqpoint{4.063842in}{3.527208in}}%
\pgfpathcurveto{\pgfqpoint{4.052792in}{3.527208in}}{\pgfqpoint{4.042193in}{3.522818in}}{\pgfqpoint{4.034380in}{3.515004in}}%
\pgfpathcurveto{\pgfqpoint{4.026566in}{3.507191in}}{\pgfqpoint{4.022176in}{3.496592in}}{\pgfqpoint{4.022176in}{3.485542in}}%
\pgfpathcurveto{\pgfqpoint{4.022176in}{3.474491in}}{\pgfqpoint{4.026566in}{3.463892in}}{\pgfqpoint{4.034380in}{3.456079in}}%
\pgfpathcurveto{\pgfqpoint{4.042193in}{3.448265in}}{\pgfqpoint{4.052792in}{3.443875in}}{\pgfqpoint{4.063842in}{3.443875in}}%
\pgfpathclose%
\pgfusepath{stroke,fill}%
\end{pgfscope}%
\begin{pgfscope}%
\pgfpathrectangle{\pgfqpoint{0.481978in}{0.331635in}}{\pgfqpoint{4.960000in}{3.696000in}}%
\pgfusepath{clip}%
\pgfsetbuttcap%
\pgfsetroundjoin%
\definecolor{currentfill}{rgb}{0.631373,0.788235,0.956863}%
\pgfsetfillcolor{currentfill}%
\pgfsetlinewidth{0.481800pt}%
\definecolor{currentstroke}{rgb}{1.000000,1.000000,1.000000}%
\pgfsetstrokecolor{currentstroke}%
\pgfsetdash{}{0pt}%
\pgfpathmoveto{\pgfqpoint{2.721231in}{1.992805in}}%
\pgfpathcurveto{\pgfqpoint{2.732281in}{1.992805in}}{\pgfqpoint{2.742880in}{1.997196in}}{\pgfqpoint{2.750694in}{2.005009in}}%
\pgfpathcurveto{\pgfqpoint{2.758507in}{2.012823in}}{\pgfqpoint{2.762898in}{2.023422in}}{\pgfqpoint{2.762898in}{2.034472in}}%
\pgfpathcurveto{\pgfqpoint{2.762898in}{2.045522in}}{\pgfqpoint{2.758507in}{2.056121in}}{\pgfqpoint{2.750694in}{2.063935in}}%
\pgfpathcurveto{\pgfqpoint{2.742880in}{2.071748in}}{\pgfqpoint{2.732281in}{2.076139in}}{\pgfqpoint{2.721231in}{2.076139in}}%
\pgfpathcurveto{\pgfqpoint{2.710181in}{2.076139in}}{\pgfqpoint{2.699582in}{2.071748in}}{\pgfqpoint{2.691768in}{2.063935in}}%
\pgfpathcurveto{\pgfqpoint{2.683955in}{2.056121in}}{\pgfqpoint{2.679564in}{2.045522in}}{\pgfqpoint{2.679564in}{2.034472in}}%
\pgfpathcurveto{\pgfqpoint{2.679564in}{2.023422in}}{\pgfqpoint{2.683955in}{2.012823in}}{\pgfqpoint{2.691768in}{2.005009in}}%
\pgfpathcurveto{\pgfqpoint{2.699582in}{1.997196in}}{\pgfqpoint{2.710181in}{1.992805in}}{\pgfqpoint{2.721231in}{1.992805in}}%
\pgfpathclose%
\pgfusepath{stroke,fill}%
\end{pgfscope}%
\begin{pgfscope}%
\pgfpathrectangle{\pgfqpoint{0.481978in}{0.331635in}}{\pgfqpoint{4.960000in}{3.696000in}}%
\pgfusepath{clip}%
\pgfsetbuttcap%
\pgfsetroundjoin%
\definecolor{currentfill}{rgb}{0.631373,0.788235,0.956863}%
\pgfsetfillcolor{currentfill}%
\pgfsetlinewidth{0.481800pt}%
\definecolor{currentstroke}{rgb}{1.000000,1.000000,1.000000}%
\pgfsetstrokecolor{currentstroke}%
\pgfsetdash{}{0pt}%
\pgfpathmoveto{\pgfqpoint{2.841211in}{3.085396in}}%
\pgfpathcurveto{\pgfqpoint{2.852261in}{3.085396in}}{\pgfqpoint{2.862860in}{3.089786in}}{\pgfqpoint{2.870673in}{3.097600in}}%
\pgfpathcurveto{\pgfqpoint{2.878487in}{3.105414in}}{\pgfqpoint{2.882877in}{3.116013in}}{\pgfqpoint{2.882877in}{3.127063in}}%
\pgfpathcurveto{\pgfqpoint{2.882877in}{3.138113in}}{\pgfqpoint{2.878487in}{3.148712in}}{\pgfqpoint{2.870673in}{3.156525in}}%
\pgfpathcurveto{\pgfqpoint{2.862860in}{3.164339in}}{\pgfqpoint{2.852261in}{3.168729in}}{\pgfqpoint{2.841211in}{3.168729in}}%
\pgfpathcurveto{\pgfqpoint{2.830161in}{3.168729in}}{\pgfqpoint{2.819561in}{3.164339in}}{\pgfqpoint{2.811748in}{3.156525in}}%
\pgfpathcurveto{\pgfqpoint{2.803934in}{3.148712in}}{\pgfqpoint{2.799544in}{3.138113in}}{\pgfqpoint{2.799544in}{3.127063in}}%
\pgfpathcurveto{\pgfqpoint{2.799544in}{3.116013in}}{\pgfqpoint{2.803934in}{3.105414in}}{\pgfqpoint{2.811748in}{3.097600in}}%
\pgfpathcurveto{\pgfqpoint{2.819561in}{3.089786in}}{\pgfqpoint{2.830161in}{3.085396in}}{\pgfqpoint{2.841211in}{3.085396in}}%
\pgfpathclose%
\pgfusepath{stroke,fill}%
\end{pgfscope}%
\begin{pgfscope}%
\pgfpathrectangle{\pgfqpoint{0.481978in}{0.331635in}}{\pgfqpoint{4.960000in}{3.696000in}}%
\pgfusepath{clip}%
\pgfsetbuttcap%
\pgfsetroundjoin%
\definecolor{currentfill}{rgb}{0.631373,0.788235,0.956863}%
\pgfsetfillcolor{currentfill}%
\pgfsetlinewidth{0.481800pt}%
\definecolor{currentstroke}{rgb}{1.000000,1.000000,1.000000}%
\pgfsetstrokecolor{currentstroke}%
\pgfsetdash{}{0pt}%
\pgfpathmoveto{\pgfqpoint{3.319157in}{2.068250in}}%
\pgfpathcurveto{\pgfqpoint{3.330208in}{2.068250in}}{\pgfqpoint{3.340807in}{2.072641in}}{\pgfqpoint{3.348620in}{2.080454in}}%
\pgfpathcurveto{\pgfqpoint{3.356434in}{2.088268in}}{\pgfqpoint{3.360824in}{2.098867in}}{\pgfqpoint{3.360824in}{2.109917in}}%
\pgfpathcurveto{\pgfqpoint{3.360824in}{2.120967in}}{\pgfqpoint{3.356434in}{2.131566in}}{\pgfqpoint{3.348620in}{2.139380in}}%
\pgfpathcurveto{\pgfqpoint{3.340807in}{2.147193in}}{\pgfqpoint{3.330208in}{2.151584in}}{\pgfqpoint{3.319157in}{2.151584in}}%
\pgfpathcurveto{\pgfqpoint{3.308107in}{2.151584in}}{\pgfqpoint{3.297508in}{2.147193in}}{\pgfqpoint{3.289695in}{2.139380in}}%
\pgfpathcurveto{\pgfqpoint{3.281881in}{2.131566in}}{\pgfqpoint{3.277491in}{2.120967in}}{\pgfqpoint{3.277491in}{2.109917in}}%
\pgfpathcurveto{\pgfqpoint{3.277491in}{2.098867in}}{\pgfqpoint{3.281881in}{2.088268in}}{\pgfqpoint{3.289695in}{2.080454in}}%
\pgfpathcurveto{\pgfqpoint{3.297508in}{2.072641in}}{\pgfqpoint{3.308107in}{2.068250in}}{\pgfqpoint{3.319157in}{2.068250in}}%
\pgfpathclose%
\pgfusepath{stroke,fill}%
\end{pgfscope}%
\begin{pgfscope}%
\pgfpathrectangle{\pgfqpoint{0.481978in}{0.331635in}}{\pgfqpoint{4.960000in}{3.696000in}}%
\pgfusepath{clip}%
\pgfsetbuttcap%
\pgfsetroundjoin%
\definecolor{currentfill}{rgb}{0.631373,0.788235,0.956863}%
\pgfsetfillcolor{currentfill}%
\pgfsetlinewidth{0.481800pt}%
\definecolor{currentstroke}{rgb}{1.000000,1.000000,1.000000}%
\pgfsetstrokecolor{currentstroke}%
\pgfsetdash{}{0pt}%
\pgfpathmoveto{\pgfqpoint{3.234413in}{2.770307in}}%
\pgfpathcurveto{\pgfqpoint{3.245463in}{2.770307in}}{\pgfqpoint{3.256062in}{2.774697in}}{\pgfqpoint{3.263875in}{2.782511in}}%
\pgfpathcurveto{\pgfqpoint{3.271689in}{2.790324in}}{\pgfqpoint{3.276079in}{2.800923in}}{\pgfqpoint{3.276079in}{2.811974in}}%
\pgfpathcurveto{\pgfqpoint{3.276079in}{2.823024in}}{\pgfqpoint{3.271689in}{2.833623in}}{\pgfqpoint{3.263875in}{2.841436in}}%
\pgfpathcurveto{\pgfqpoint{3.256062in}{2.849250in}}{\pgfqpoint{3.245463in}{2.853640in}}{\pgfqpoint{3.234413in}{2.853640in}}%
\pgfpathcurveto{\pgfqpoint{3.223363in}{2.853640in}}{\pgfqpoint{3.212764in}{2.849250in}}{\pgfqpoint{3.204950in}{2.841436in}}%
\pgfpathcurveto{\pgfqpoint{3.197136in}{2.833623in}}{\pgfqpoint{3.192746in}{2.823024in}}{\pgfqpoint{3.192746in}{2.811974in}}%
\pgfpathcurveto{\pgfqpoint{3.192746in}{2.800923in}}{\pgfqpoint{3.197136in}{2.790324in}}{\pgfqpoint{3.204950in}{2.782511in}}%
\pgfpathcurveto{\pgfqpoint{3.212764in}{2.774697in}}{\pgfqpoint{3.223363in}{2.770307in}}{\pgfqpoint{3.234413in}{2.770307in}}%
\pgfpathclose%
\pgfusepath{stroke,fill}%
\end{pgfscope}%
\begin{pgfscope}%
\pgfpathrectangle{\pgfqpoint{0.481978in}{0.331635in}}{\pgfqpoint{4.960000in}{3.696000in}}%
\pgfusepath{clip}%
\pgfsetbuttcap%
\pgfsetroundjoin%
\definecolor{currentfill}{rgb}{0.631373,0.788235,0.956863}%
\pgfsetfillcolor{currentfill}%
\pgfsetlinewidth{0.481800pt}%
\definecolor{currentstroke}{rgb}{1.000000,1.000000,1.000000}%
\pgfsetstrokecolor{currentstroke}%
\pgfsetdash{}{0pt}%
\pgfpathmoveto{\pgfqpoint{3.645927in}{2.290242in}}%
\pgfpathcurveto{\pgfqpoint{3.656977in}{2.290242in}}{\pgfqpoint{3.667576in}{2.294632in}}{\pgfqpoint{3.675389in}{2.302445in}}%
\pgfpathcurveto{\pgfqpoint{3.683203in}{2.310259in}}{\pgfqpoint{3.687593in}{2.320858in}}{\pgfqpoint{3.687593in}{2.331908in}}%
\pgfpathcurveto{\pgfqpoint{3.687593in}{2.342958in}}{\pgfqpoint{3.683203in}{2.353557in}}{\pgfqpoint{3.675389in}{2.361371in}}%
\pgfpathcurveto{\pgfqpoint{3.667576in}{2.369185in}}{\pgfqpoint{3.656977in}{2.373575in}}{\pgfqpoint{3.645927in}{2.373575in}}%
\pgfpathcurveto{\pgfqpoint{3.634877in}{2.373575in}}{\pgfqpoint{3.624277in}{2.369185in}}{\pgfqpoint{3.616464in}{2.361371in}}%
\pgfpathcurveto{\pgfqpoint{3.608650in}{2.353557in}}{\pgfqpoint{3.604260in}{2.342958in}}{\pgfqpoint{3.604260in}{2.331908in}}%
\pgfpathcurveto{\pgfqpoint{3.604260in}{2.320858in}}{\pgfqpoint{3.608650in}{2.310259in}}{\pgfqpoint{3.616464in}{2.302445in}}%
\pgfpathcurveto{\pgfqpoint{3.624277in}{2.294632in}}{\pgfqpoint{3.634877in}{2.290242in}}{\pgfqpoint{3.645927in}{2.290242in}}%
\pgfpathclose%
\pgfusepath{stroke,fill}%
\end{pgfscope}%
\begin{pgfscope}%
\pgfpathrectangle{\pgfqpoint{0.481978in}{0.331635in}}{\pgfqpoint{4.960000in}{3.696000in}}%
\pgfusepath{clip}%
\pgfsetbuttcap%
\pgfsetroundjoin%
\definecolor{currentfill}{rgb}{0.631373,0.788235,0.956863}%
\pgfsetfillcolor{currentfill}%
\pgfsetlinewidth{0.481800pt}%
\definecolor{currentstroke}{rgb}{1.000000,1.000000,1.000000}%
\pgfsetstrokecolor{currentstroke}%
\pgfsetdash{}{0pt}%
\pgfpathmoveto{\pgfqpoint{4.276918in}{3.517980in}}%
\pgfpathcurveto{\pgfqpoint{4.287968in}{3.517980in}}{\pgfqpoint{4.298567in}{3.522371in}}{\pgfqpoint{4.306381in}{3.530184in}}%
\pgfpathcurveto{\pgfqpoint{4.314195in}{3.537998in}}{\pgfqpoint{4.318585in}{3.548597in}}{\pgfqpoint{4.318585in}{3.559647in}}%
\pgfpathcurveto{\pgfqpoint{4.318585in}{3.570697in}}{\pgfqpoint{4.314195in}{3.581296in}}{\pgfqpoint{4.306381in}{3.589110in}}%
\pgfpathcurveto{\pgfqpoint{4.298567in}{3.596923in}}{\pgfqpoint{4.287968in}{3.601314in}}{\pgfqpoint{4.276918in}{3.601314in}}%
\pgfpathcurveto{\pgfqpoint{4.265868in}{3.601314in}}{\pgfqpoint{4.255269in}{3.596923in}}{\pgfqpoint{4.247456in}{3.589110in}}%
\pgfpathcurveto{\pgfqpoint{4.239642in}{3.581296in}}{\pgfqpoint{4.235252in}{3.570697in}}{\pgfqpoint{4.235252in}{3.559647in}}%
\pgfpathcurveto{\pgfqpoint{4.235252in}{3.548597in}}{\pgfqpoint{4.239642in}{3.537998in}}{\pgfqpoint{4.247456in}{3.530184in}}%
\pgfpathcurveto{\pgfqpoint{4.255269in}{3.522371in}}{\pgfqpoint{4.265868in}{3.517980in}}{\pgfqpoint{4.276918in}{3.517980in}}%
\pgfpathclose%
\pgfusepath{stroke,fill}%
\end{pgfscope}%
\begin{pgfscope}%
\pgfpathrectangle{\pgfqpoint{0.481978in}{0.331635in}}{\pgfqpoint{4.960000in}{3.696000in}}%
\pgfusepath{clip}%
\pgfsetbuttcap%
\pgfsetroundjoin%
\definecolor{currentfill}{rgb}{0.631373,0.788235,0.956863}%
\pgfsetfillcolor{currentfill}%
\pgfsetlinewidth{0.481800pt}%
\definecolor{currentstroke}{rgb}{1.000000,1.000000,1.000000}%
\pgfsetstrokecolor{currentstroke}%
\pgfsetdash{}{0pt}%
\pgfpathmoveto{\pgfqpoint{2.740941in}{2.191716in}}%
\pgfpathcurveto{\pgfqpoint{2.751991in}{2.191716in}}{\pgfqpoint{2.762590in}{2.196107in}}{\pgfqpoint{2.770403in}{2.203920in}}%
\pgfpathcurveto{\pgfqpoint{2.778217in}{2.211734in}}{\pgfqpoint{2.782607in}{2.222333in}}{\pgfqpoint{2.782607in}{2.233383in}}%
\pgfpathcurveto{\pgfqpoint{2.782607in}{2.244433in}}{\pgfqpoint{2.778217in}{2.255032in}}{\pgfqpoint{2.770403in}{2.262846in}}%
\pgfpathcurveto{\pgfqpoint{2.762590in}{2.270660in}}{\pgfqpoint{2.751991in}{2.275050in}}{\pgfqpoint{2.740941in}{2.275050in}}%
\pgfpathcurveto{\pgfqpoint{2.729890in}{2.275050in}}{\pgfqpoint{2.719291in}{2.270660in}}{\pgfqpoint{2.711478in}{2.262846in}}%
\pgfpathcurveto{\pgfqpoint{2.703664in}{2.255032in}}{\pgfqpoint{2.699274in}{2.244433in}}{\pgfqpoint{2.699274in}{2.233383in}}%
\pgfpathcurveto{\pgfqpoint{2.699274in}{2.222333in}}{\pgfqpoint{2.703664in}{2.211734in}}{\pgfqpoint{2.711478in}{2.203920in}}%
\pgfpathcurveto{\pgfqpoint{2.719291in}{2.196107in}}{\pgfqpoint{2.729890in}{2.191716in}}{\pgfqpoint{2.740941in}{2.191716in}}%
\pgfpathclose%
\pgfusepath{stroke,fill}%
\end{pgfscope}%
\begin{pgfscope}%
\pgfpathrectangle{\pgfqpoint{0.481978in}{0.331635in}}{\pgfqpoint{4.960000in}{3.696000in}}%
\pgfusepath{clip}%
\pgfsetbuttcap%
\pgfsetroundjoin%
\definecolor{currentfill}{rgb}{0.631373,0.788235,0.956863}%
\pgfsetfillcolor{currentfill}%
\pgfsetlinewidth{0.481800pt}%
\definecolor{currentstroke}{rgb}{1.000000,1.000000,1.000000}%
\pgfsetstrokecolor{currentstroke}%
\pgfsetdash{}{0pt}%
\pgfpathmoveto{\pgfqpoint{2.346492in}{2.274565in}}%
\pgfpathcurveto{\pgfqpoint{2.357542in}{2.274565in}}{\pgfqpoint{2.368141in}{2.278955in}}{\pgfqpoint{2.375955in}{2.286769in}}%
\pgfpathcurveto{\pgfqpoint{2.383768in}{2.294583in}}{\pgfqpoint{2.388159in}{2.305182in}}{\pgfqpoint{2.388159in}{2.316232in}}%
\pgfpathcurveto{\pgfqpoint{2.388159in}{2.327282in}}{\pgfqpoint{2.383768in}{2.337881in}}{\pgfqpoint{2.375955in}{2.345695in}}%
\pgfpathcurveto{\pgfqpoint{2.368141in}{2.353508in}}{\pgfqpoint{2.357542in}{2.357898in}}{\pgfqpoint{2.346492in}{2.357898in}}%
\pgfpathcurveto{\pgfqpoint{2.335442in}{2.357898in}}{\pgfqpoint{2.324843in}{2.353508in}}{\pgfqpoint{2.317029in}{2.345695in}}%
\pgfpathcurveto{\pgfqpoint{2.309216in}{2.337881in}}{\pgfqpoint{2.304825in}{2.327282in}}{\pgfqpoint{2.304825in}{2.316232in}}%
\pgfpathcurveto{\pgfqpoint{2.304825in}{2.305182in}}{\pgfqpoint{2.309216in}{2.294583in}}{\pgfqpoint{2.317029in}{2.286769in}}%
\pgfpathcurveto{\pgfqpoint{2.324843in}{2.278955in}}{\pgfqpoint{2.335442in}{2.274565in}}{\pgfqpoint{2.346492in}{2.274565in}}%
\pgfpathclose%
\pgfusepath{stroke,fill}%
\end{pgfscope}%
\begin{pgfscope}%
\pgfpathrectangle{\pgfqpoint{0.481978in}{0.331635in}}{\pgfqpoint{4.960000in}{3.696000in}}%
\pgfusepath{clip}%
\pgfsetbuttcap%
\pgfsetroundjoin%
\definecolor{currentfill}{rgb}{0.631373,0.788235,0.956863}%
\pgfsetfillcolor{currentfill}%
\pgfsetlinewidth{0.481800pt}%
\definecolor{currentstroke}{rgb}{1.000000,1.000000,1.000000}%
\pgfsetstrokecolor{currentstroke}%
\pgfsetdash{}{0pt}%
\pgfpathmoveto{\pgfqpoint{2.250683in}{1.897030in}}%
\pgfpathcurveto{\pgfqpoint{2.261734in}{1.897030in}}{\pgfqpoint{2.272333in}{1.901421in}}{\pgfqpoint{2.280146in}{1.909234in}}%
\pgfpathcurveto{\pgfqpoint{2.287960in}{1.917048in}}{\pgfqpoint{2.292350in}{1.927647in}}{\pgfqpoint{2.292350in}{1.938697in}}%
\pgfpathcurveto{\pgfqpoint{2.292350in}{1.949747in}}{\pgfqpoint{2.287960in}{1.960346in}}{\pgfqpoint{2.280146in}{1.968160in}}%
\pgfpathcurveto{\pgfqpoint{2.272333in}{1.975973in}}{\pgfqpoint{2.261734in}{1.980364in}}{\pgfqpoint{2.250683in}{1.980364in}}%
\pgfpathcurveto{\pgfqpoint{2.239633in}{1.980364in}}{\pgfqpoint{2.229034in}{1.975973in}}{\pgfqpoint{2.221221in}{1.968160in}}%
\pgfpathcurveto{\pgfqpoint{2.213407in}{1.960346in}}{\pgfqpoint{2.209017in}{1.949747in}}{\pgfqpoint{2.209017in}{1.938697in}}%
\pgfpathcurveto{\pgfqpoint{2.209017in}{1.927647in}}{\pgfqpoint{2.213407in}{1.917048in}}{\pgfqpoint{2.221221in}{1.909234in}}%
\pgfpathcurveto{\pgfqpoint{2.229034in}{1.901421in}}{\pgfqpoint{2.239633in}{1.897030in}}{\pgfqpoint{2.250683in}{1.897030in}}%
\pgfpathclose%
\pgfusepath{stroke,fill}%
\end{pgfscope}%
\begin{pgfscope}%
\pgfpathrectangle{\pgfqpoint{0.481978in}{0.331635in}}{\pgfqpoint{4.960000in}{3.696000in}}%
\pgfusepath{clip}%
\pgfsetbuttcap%
\pgfsetroundjoin%
\definecolor{currentfill}{rgb}{0.631373,0.788235,0.956863}%
\pgfsetfillcolor{currentfill}%
\pgfsetlinewidth{0.481800pt}%
\definecolor{currentstroke}{rgb}{1.000000,1.000000,1.000000}%
\pgfsetstrokecolor{currentstroke}%
\pgfsetdash{}{0pt}%
\pgfpathmoveto{\pgfqpoint{2.500865in}{1.749430in}}%
\pgfpathcurveto{\pgfqpoint{2.511915in}{1.749430in}}{\pgfqpoint{2.522514in}{1.753820in}}{\pgfqpoint{2.530328in}{1.761633in}}%
\pgfpathcurveto{\pgfqpoint{2.538141in}{1.769447in}}{\pgfqpoint{2.542532in}{1.780046in}}{\pgfqpoint{2.542532in}{1.791096in}}%
\pgfpathcurveto{\pgfqpoint{2.542532in}{1.802146in}}{\pgfqpoint{2.538141in}{1.812745in}}{\pgfqpoint{2.530328in}{1.820559in}}%
\pgfpathcurveto{\pgfqpoint{2.522514in}{1.828373in}}{\pgfqpoint{2.511915in}{1.832763in}}{\pgfqpoint{2.500865in}{1.832763in}}%
\pgfpathcurveto{\pgfqpoint{2.489815in}{1.832763in}}{\pgfqpoint{2.479216in}{1.828373in}}{\pgfqpoint{2.471402in}{1.820559in}}%
\pgfpathcurveto{\pgfqpoint{2.463589in}{1.812745in}}{\pgfqpoint{2.459198in}{1.802146in}}{\pgfqpoint{2.459198in}{1.791096in}}%
\pgfpathcurveto{\pgfqpoint{2.459198in}{1.780046in}}{\pgfqpoint{2.463589in}{1.769447in}}{\pgfqpoint{2.471402in}{1.761633in}}%
\pgfpathcurveto{\pgfqpoint{2.479216in}{1.753820in}}{\pgfqpoint{2.489815in}{1.749430in}}{\pgfqpoint{2.500865in}{1.749430in}}%
\pgfpathclose%
\pgfusepath{stroke,fill}%
\end{pgfscope}%
\begin{pgfscope}%
\pgfpathrectangle{\pgfqpoint{0.481978in}{0.331635in}}{\pgfqpoint{4.960000in}{3.696000in}}%
\pgfusepath{clip}%
\pgfsetbuttcap%
\pgfsetroundjoin%
\definecolor{currentfill}{rgb}{0.631373,0.788235,0.956863}%
\pgfsetfillcolor{currentfill}%
\pgfsetlinewidth{0.481800pt}%
\definecolor{currentstroke}{rgb}{1.000000,1.000000,1.000000}%
\pgfsetstrokecolor{currentstroke}%
\pgfsetdash{}{0pt}%
\pgfpathmoveto{\pgfqpoint{2.611315in}{2.320377in}}%
\pgfpathcurveto{\pgfqpoint{2.622365in}{2.320377in}}{\pgfqpoint{2.632964in}{2.324768in}}{\pgfqpoint{2.640778in}{2.332581in}}%
\pgfpathcurveto{\pgfqpoint{2.648591in}{2.340395in}}{\pgfqpoint{2.652982in}{2.350994in}}{\pgfqpoint{2.652982in}{2.362044in}}%
\pgfpathcurveto{\pgfqpoint{2.652982in}{2.373094in}}{\pgfqpoint{2.648591in}{2.383693in}}{\pgfqpoint{2.640778in}{2.391507in}}%
\pgfpathcurveto{\pgfqpoint{2.632964in}{2.399320in}}{\pgfqpoint{2.622365in}{2.403711in}}{\pgfqpoint{2.611315in}{2.403711in}}%
\pgfpathcurveto{\pgfqpoint{2.600265in}{2.403711in}}{\pgfqpoint{2.589666in}{2.399320in}}{\pgfqpoint{2.581852in}{2.391507in}}%
\pgfpathcurveto{\pgfqpoint{2.574039in}{2.383693in}}{\pgfqpoint{2.569648in}{2.373094in}}{\pgfqpoint{2.569648in}{2.362044in}}%
\pgfpathcurveto{\pgfqpoint{2.569648in}{2.350994in}}{\pgfqpoint{2.574039in}{2.340395in}}{\pgfqpoint{2.581852in}{2.332581in}}%
\pgfpathcurveto{\pgfqpoint{2.589666in}{2.324768in}}{\pgfqpoint{2.600265in}{2.320377in}}{\pgfqpoint{2.611315in}{2.320377in}}%
\pgfpathclose%
\pgfusepath{stroke,fill}%
\end{pgfscope}%
\begin{pgfscope}%
\pgfpathrectangle{\pgfqpoint{0.481978in}{0.331635in}}{\pgfqpoint{4.960000in}{3.696000in}}%
\pgfusepath{clip}%
\pgfsetbuttcap%
\pgfsetroundjoin%
\definecolor{currentfill}{rgb}{0.631373,0.788235,0.956863}%
\pgfsetfillcolor{currentfill}%
\pgfsetlinewidth{0.481800pt}%
\definecolor{currentstroke}{rgb}{1.000000,1.000000,1.000000}%
\pgfsetstrokecolor{currentstroke}%
\pgfsetdash{}{0pt}%
\pgfpathmoveto{\pgfqpoint{4.658256in}{3.474573in}}%
\pgfpathcurveto{\pgfqpoint{4.669306in}{3.474573in}}{\pgfqpoint{4.679905in}{3.478963in}}{\pgfqpoint{4.687718in}{3.486777in}}%
\pgfpathcurveto{\pgfqpoint{4.695532in}{3.494590in}}{\pgfqpoint{4.699922in}{3.505189in}}{\pgfqpoint{4.699922in}{3.516239in}}%
\pgfpathcurveto{\pgfqpoint{4.699922in}{3.527290in}}{\pgfqpoint{4.695532in}{3.537889in}}{\pgfqpoint{4.687718in}{3.545702in}}%
\pgfpathcurveto{\pgfqpoint{4.679905in}{3.553516in}}{\pgfqpoint{4.669306in}{3.557906in}}{\pgfqpoint{4.658256in}{3.557906in}}%
\pgfpathcurveto{\pgfqpoint{4.647206in}{3.557906in}}{\pgfqpoint{4.636606in}{3.553516in}}{\pgfqpoint{4.628793in}{3.545702in}}%
\pgfpathcurveto{\pgfqpoint{4.620979in}{3.537889in}}{\pgfqpoint{4.616589in}{3.527290in}}{\pgfqpoint{4.616589in}{3.516239in}}%
\pgfpathcurveto{\pgfqpoint{4.616589in}{3.505189in}}{\pgfqpoint{4.620979in}{3.494590in}}{\pgfqpoint{4.628793in}{3.486777in}}%
\pgfpathcurveto{\pgfqpoint{4.636606in}{3.478963in}}{\pgfqpoint{4.647206in}{3.474573in}}{\pgfqpoint{4.658256in}{3.474573in}}%
\pgfpathclose%
\pgfusepath{stroke,fill}%
\end{pgfscope}%
\begin{pgfscope}%
\pgfpathrectangle{\pgfqpoint{0.481978in}{0.331635in}}{\pgfqpoint{4.960000in}{3.696000in}}%
\pgfusepath{clip}%
\pgfsetbuttcap%
\pgfsetroundjoin%
\definecolor{currentfill}{rgb}{0.631373,0.788235,0.956863}%
\pgfsetfillcolor{currentfill}%
\pgfsetlinewidth{0.481800pt}%
\definecolor{currentstroke}{rgb}{1.000000,1.000000,1.000000}%
\pgfsetstrokecolor{currentstroke}%
\pgfsetdash{}{0pt}%
\pgfpathmoveto{\pgfqpoint{3.034776in}{1.765671in}}%
\pgfpathcurveto{\pgfqpoint{3.045826in}{1.765671in}}{\pgfqpoint{3.056425in}{1.770061in}}{\pgfqpoint{3.064238in}{1.777875in}}%
\pgfpathcurveto{\pgfqpoint{3.072052in}{1.785689in}}{\pgfqpoint{3.076442in}{1.796288in}}{\pgfqpoint{3.076442in}{1.807338in}}%
\pgfpathcurveto{\pgfqpoint{3.076442in}{1.818388in}}{\pgfqpoint{3.072052in}{1.828987in}}{\pgfqpoint{3.064238in}{1.836801in}}%
\pgfpathcurveto{\pgfqpoint{3.056425in}{1.844614in}}{\pgfqpoint{3.045826in}{1.849005in}}{\pgfqpoint{3.034776in}{1.849005in}}%
\pgfpathcurveto{\pgfqpoint{3.023725in}{1.849005in}}{\pgfqpoint{3.013126in}{1.844614in}}{\pgfqpoint{3.005313in}{1.836801in}}%
\pgfpathcurveto{\pgfqpoint{2.997499in}{1.828987in}}{\pgfqpoint{2.993109in}{1.818388in}}{\pgfqpoint{2.993109in}{1.807338in}}%
\pgfpathcurveto{\pgfqpoint{2.993109in}{1.796288in}}{\pgfqpoint{2.997499in}{1.785689in}}{\pgfqpoint{3.005313in}{1.777875in}}%
\pgfpathcurveto{\pgfqpoint{3.013126in}{1.770061in}}{\pgfqpoint{3.023725in}{1.765671in}}{\pgfqpoint{3.034776in}{1.765671in}}%
\pgfpathclose%
\pgfusepath{stroke,fill}%
\end{pgfscope}%
\begin{pgfscope}%
\pgfpathrectangle{\pgfqpoint{0.481978in}{0.331635in}}{\pgfqpoint{4.960000in}{3.696000in}}%
\pgfusepath{clip}%
\pgfsetbuttcap%
\pgfsetroundjoin%
\definecolor{currentfill}{rgb}{0.631373,0.788235,0.956863}%
\pgfsetfillcolor{currentfill}%
\pgfsetlinewidth{0.481800pt}%
\definecolor{currentstroke}{rgb}{1.000000,1.000000,1.000000}%
\pgfsetstrokecolor{currentstroke}%
\pgfsetdash{}{0pt}%
\pgfpathmoveto{\pgfqpoint{3.122341in}{2.769773in}}%
\pgfpathcurveto{\pgfqpoint{3.133391in}{2.769773in}}{\pgfqpoint{3.143990in}{2.774163in}}{\pgfqpoint{3.151804in}{2.781976in}}%
\pgfpathcurveto{\pgfqpoint{3.159617in}{2.789790in}}{\pgfqpoint{3.164008in}{2.800389in}}{\pgfqpoint{3.164008in}{2.811439in}}%
\pgfpathcurveto{\pgfqpoint{3.164008in}{2.822489in}}{\pgfqpoint{3.159617in}{2.833088in}}{\pgfqpoint{3.151804in}{2.840902in}}%
\pgfpathcurveto{\pgfqpoint{3.143990in}{2.848716in}}{\pgfqpoint{3.133391in}{2.853106in}}{\pgfqpoint{3.122341in}{2.853106in}}%
\pgfpathcurveto{\pgfqpoint{3.111291in}{2.853106in}}{\pgfqpoint{3.100692in}{2.848716in}}{\pgfqpoint{3.092878in}{2.840902in}}%
\pgfpathcurveto{\pgfqpoint{3.085065in}{2.833088in}}{\pgfqpoint{3.080674in}{2.822489in}}{\pgfqpoint{3.080674in}{2.811439in}}%
\pgfpathcurveto{\pgfqpoint{3.080674in}{2.800389in}}{\pgfqpoint{3.085065in}{2.789790in}}{\pgfqpoint{3.092878in}{2.781976in}}%
\pgfpathcurveto{\pgfqpoint{3.100692in}{2.774163in}}{\pgfqpoint{3.111291in}{2.769773in}}{\pgfqpoint{3.122341in}{2.769773in}}%
\pgfpathclose%
\pgfusepath{stroke,fill}%
\end{pgfscope}%
\begin{pgfscope}%
\pgfpathrectangle{\pgfqpoint{0.481978in}{0.331635in}}{\pgfqpoint{4.960000in}{3.696000in}}%
\pgfusepath{clip}%
\pgfsetbuttcap%
\pgfsetroundjoin%
\definecolor{currentfill}{rgb}{0.631373,0.788235,0.956863}%
\pgfsetfillcolor{currentfill}%
\pgfsetlinewidth{0.481800pt}%
\definecolor{currentstroke}{rgb}{1.000000,1.000000,1.000000}%
\pgfsetstrokecolor{currentstroke}%
\pgfsetdash{}{0pt}%
\pgfpathmoveto{\pgfqpoint{2.762823in}{3.520965in}}%
\pgfpathcurveto{\pgfqpoint{2.773874in}{3.520965in}}{\pgfqpoint{2.784473in}{3.525355in}}{\pgfqpoint{2.792286in}{3.533169in}}%
\pgfpathcurveto{\pgfqpoint{2.800100in}{3.540983in}}{\pgfqpoint{2.804490in}{3.551582in}}{\pgfqpoint{2.804490in}{3.562632in}}%
\pgfpathcurveto{\pgfqpoint{2.804490in}{3.573682in}}{\pgfqpoint{2.800100in}{3.584281in}}{\pgfqpoint{2.792286in}{3.592094in}}%
\pgfpathcurveto{\pgfqpoint{2.784473in}{3.599908in}}{\pgfqpoint{2.773874in}{3.604298in}}{\pgfqpoint{2.762823in}{3.604298in}}%
\pgfpathcurveto{\pgfqpoint{2.751773in}{3.604298in}}{\pgfqpoint{2.741174in}{3.599908in}}{\pgfqpoint{2.733361in}{3.592094in}}%
\pgfpathcurveto{\pgfqpoint{2.725547in}{3.584281in}}{\pgfqpoint{2.721157in}{3.573682in}}{\pgfqpoint{2.721157in}{3.562632in}}%
\pgfpathcurveto{\pgfqpoint{2.721157in}{3.551582in}}{\pgfqpoint{2.725547in}{3.540983in}}{\pgfqpoint{2.733361in}{3.533169in}}%
\pgfpathcurveto{\pgfqpoint{2.741174in}{3.525355in}}{\pgfqpoint{2.751773in}{3.520965in}}{\pgfqpoint{2.762823in}{3.520965in}}%
\pgfpathclose%
\pgfusepath{stroke,fill}%
\end{pgfscope}%
\begin{pgfscope}%
\pgfpathrectangle{\pgfqpoint{0.481978in}{0.331635in}}{\pgfqpoint{4.960000in}{3.696000in}}%
\pgfusepath{clip}%
\pgfsetbuttcap%
\pgfsetroundjoin%
\definecolor{currentfill}{rgb}{0.631373,0.788235,0.956863}%
\pgfsetfillcolor{currentfill}%
\pgfsetlinewidth{0.481800pt}%
\definecolor{currentstroke}{rgb}{1.000000,1.000000,1.000000}%
\pgfsetstrokecolor{currentstroke}%
\pgfsetdash{}{0pt}%
\pgfpathmoveto{\pgfqpoint{2.859601in}{2.492247in}}%
\pgfpathcurveto{\pgfqpoint{2.870652in}{2.492247in}}{\pgfqpoint{2.881251in}{2.496637in}}{\pgfqpoint{2.889064in}{2.504451in}}%
\pgfpathcurveto{\pgfqpoint{2.896878in}{2.512264in}}{\pgfqpoint{2.901268in}{2.522863in}}{\pgfqpoint{2.901268in}{2.533913in}}%
\pgfpathcurveto{\pgfqpoint{2.901268in}{2.544963in}}{\pgfqpoint{2.896878in}{2.555563in}}{\pgfqpoint{2.889064in}{2.563376in}}%
\pgfpathcurveto{\pgfqpoint{2.881251in}{2.571190in}}{\pgfqpoint{2.870652in}{2.575580in}}{\pgfqpoint{2.859601in}{2.575580in}}%
\pgfpathcurveto{\pgfqpoint{2.848551in}{2.575580in}}{\pgfqpoint{2.837952in}{2.571190in}}{\pgfqpoint{2.830139in}{2.563376in}}%
\pgfpathcurveto{\pgfqpoint{2.822325in}{2.555563in}}{\pgfqpoint{2.817935in}{2.544963in}}{\pgfqpoint{2.817935in}{2.533913in}}%
\pgfpathcurveto{\pgfqpoint{2.817935in}{2.522863in}}{\pgfqpoint{2.822325in}{2.512264in}}{\pgfqpoint{2.830139in}{2.504451in}}%
\pgfpathcurveto{\pgfqpoint{2.837952in}{2.496637in}}{\pgfqpoint{2.848551in}{2.492247in}}{\pgfqpoint{2.859601in}{2.492247in}}%
\pgfpathclose%
\pgfusepath{stroke,fill}%
\end{pgfscope}%
\begin{pgfscope}%
\pgfpathrectangle{\pgfqpoint{0.481978in}{0.331635in}}{\pgfqpoint{4.960000in}{3.696000in}}%
\pgfusepath{clip}%
\pgfsetbuttcap%
\pgfsetroundjoin%
\definecolor{currentfill}{rgb}{0.631373,0.788235,0.956863}%
\pgfsetfillcolor{currentfill}%
\pgfsetlinewidth{0.481800pt}%
\definecolor{currentstroke}{rgb}{1.000000,1.000000,1.000000}%
\pgfsetstrokecolor{currentstroke}%
\pgfsetdash{}{0pt}%
\pgfpathmoveto{\pgfqpoint{3.643947in}{3.353991in}}%
\pgfpathcurveto{\pgfqpoint{3.654997in}{3.353991in}}{\pgfqpoint{3.665596in}{3.358381in}}{\pgfqpoint{3.673409in}{3.366195in}}%
\pgfpathcurveto{\pgfqpoint{3.681223in}{3.374009in}}{\pgfqpoint{3.685613in}{3.384608in}}{\pgfqpoint{3.685613in}{3.395658in}}%
\pgfpathcurveto{\pgfqpoint{3.685613in}{3.406708in}}{\pgfqpoint{3.681223in}{3.417307in}}{\pgfqpoint{3.673409in}{3.425121in}}%
\pgfpathcurveto{\pgfqpoint{3.665596in}{3.432934in}}{\pgfqpoint{3.654997in}{3.437324in}}{\pgfqpoint{3.643947in}{3.437324in}}%
\pgfpathcurveto{\pgfqpoint{3.632896in}{3.437324in}}{\pgfqpoint{3.622297in}{3.432934in}}{\pgfqpoint{3.614484in}{3.425121in}}%
\pgfpathcurveto{\pgfqpoint{3.606670in}{3.417307in}}{\pgfqpoint{3.602280in}{3.406708in}}{\pgfqpoint{3.602280in}{3.395658in}}%
\pgfpathcurveto{\pgfqpoint{3.602280in}{3.384608in}}{\pgfqpoint{3.606670in}{3.374009in}}{\pgfqpoint{3.614484in}{3.366195in}}%
\pgfpathcurveto{\pgfqpoint{3.622297in}{3.358381in}}{\pgfqpoint{3.632896in}{3.353991in}}{\pgfqpoint{3.643947in}{3.353991in}}%
\pgfpathclose%
\pgfusepath{stroke,fill}%
\end{pgfscope}%
\begin{pgfscope}%
\pgfpathrectangle{\pgfqpoint{0.481978in}{0.331635in}}{\pgfqpoint{4.960000in}{3.696000in}}%
\pgfusepath{clip}%
\pgfsetbuttcap%
\pgfsetroundjoin%
\definecolor{currentfill}{rgb}{0.631373,0.788235,0.956863}%
\pgfsetfillcolor{currentfill}%
\pgfsetlinewidth{0.481800pt}%
\definecolor{currentstroke}{rgb}{1.000000,1.000000,1.000000}%
\pgfsetstrokecolor{currentstroke}%
\pgfsetdash{}{0pt}%
\pgfpathmoveto{\pgfqpoint{3.539644in}{3.100877in}}%
\pgfpathcurveto{\pgfqpoint{3.550695in}{3.100877in}}{\pgfqpoint{3.561294in}{3.105267in}}{\pgfqpoint{3.569107in}{3.113081in}}%
\pgfpathcurveto{\pgfqpoint{3.576921in}{3.120895in}}{\pgfqpoint{3.581311in}{3.131494in}}{\pgfqpoint{3.581311in}{3.142544in}}%
\pgfpathcurveto{\pgfqpoint{3.581311in}{3.153594in}}{\pgfqpoint{3.576921in}{3.164193in}}{\pgfqpoint{3.569107in}{3.172007in}}%
\pgfpathcurveto{\pgfqpoint{3.561294in}{3.179820in}}{\pgfqpoint{3.550695in}{3.184211in}}{\pgfqpoint{3.539644in}{3.184211in}}%
\pgfpathcurveto{\pgfqpoint{3.528594in}{3.184211in}}{\pgfqpoint{3.517995in}{3.179820in}}{\pgfqpoint{3.510182in}{3.172007in}}%
\pgfpathcurveto{\pgfqpoint{3.502368in}{3.164193in}}{\pgfqpoint{3.497978in}{3.153594in}}{\pgfqpoint{3.497978in}{3.142544in}}%
\pgfpathcurveto{\pgfqpoint{3.497978in}{3.131494in}}{\pgfqpoint{3.502368in}{3.120895in}}{\pgfqpoint{3.510182in}{3.113081in}}%
\pgfpathcurveto{\pgfqpoint{3.517995in}{3.105267in}}{\pgfqpoint{3.528594in}{3.100877in}}{\pgfqpoint{3.539644in}{3.100877in}}%
\pgfpathclose%
\pgfusepath{stroke,fill}%
\end{pgfscope}%
\begin{pgfscope}%
\pgfpathrectangle{\pgfqpoint{0.481978in}{0.331635in}}{\pgfqpoint{4.960000in}{3.696000in}}%
\pgfusepath{clip}%
\pgfsetbuttcap%
\pgfsetroundjoin%
\definecolor{currentfill}{rgb}{0.631373,0.788235,0.956863}%
\pgfsetfillcolor{currentfill}%
\pgfsetlinewidth{0.481800pt}%
\definecolor{currentstroke}{rgb}{1.000000,1.000000,1.000000}%
\pgfsetstrokecolor{currentstroke}%
\pgfsetdash{}{0pt}%
\pgfpathmoveto{\pgfqpoint{4.151040in}{3.236589in}}%
\pgfpathcurveto{\pgfqpoint{4.162090in}{3.236589in}}{\pgfqpoint{4.172689in}{3.240979in}}{\pgfqpoint{4.180503in}{3.248793in}}%
\pgfpathcurveto{\pgfqpoint{4.188317in}{3.256606in}}{\pgfqpoint{4.192707in}{3.267205in}}{\pgfqpoint{4.192707in}{3.278256in}}%
\pgfpathcurveto{\pgfqpoint{4.192707in}{3.289306in}}{\pgfqpoint{4.188317in}{3.299905in}}{\pgfqpoint{4.180503in}{3.307718in}}%
\pgfpathcurveto{\pgfqpoint{4.172689in}{3.315532in}}{\pgfqpoint{4.162090in}{3.319922in}}{\pgfqpoint{4.151040in}{3.319922in}}%
\pgfpathcurveto{\pgfqpoint{4.139990in}{3.319922in}}{\pgfqpoint{4.129391in}{3.315532in}}{\pgfqpoint{4.121578in}{3.307718in}}%
\pgfpathcurveto{\pgfqpoint{4.113764in}{3.299905in}}{\pgfqpoint{4.109374in}{3.289306in}}{\pgfqpoint{4.109374in}{3.278256in}}%
\pgfpathcurveto{\pgfqpoint{4.109374in}{3.267205in}}{\pgfqpoint{4.113764in}{3.256606in}}{\pgfqpoint{4.121578in}{3.248793in}}%
\pgfpathcurveto{\pgfqpoint{4.129391in}{3.240979in}}{\pgfqpoint{4.139990in}{3.236589in}}{\pgfqpoint{4.151040in}{3.236589in}}%
\pgfpathclose%
\pgfusepath{stroke,fill}%
\end{pgfscope}%
\begin{pgfscope}%
\pgfpathrectangle{\pgfqpoint{0.481978in}{0.331635in}}{\pgfqpoint{4.960000in}{3.696000in}}%
\pgfusepath{clip}%
\pgfsetbuttcap%
\pgfsetroundjoin%
\definecolor{currentfill}{rgb}{0.631373,0.788235,0.956863}%
\pgfsetfillcolor{currentfill}%
\pgfsetlinewidth{0.481800pt}%
\definecolor{currentstroke}{rgb}{1.000000,1.000000,1.000000}%
\pgfsetstrokecolor{currentstroke}%
\pgfsetdash{}{0pt}%
\pgfpathmoveto{\pgfqpoint{2.718198in}{2.043552in}}%
\pgfpathcurveto{\pgfqpoint{2.729248in}{2.043552in}}{\pgfqpoint{2.739847in}{2.047943in}}{\pgfqpoint{2.747661in}{2.055756in}}%
\pgfpathcurveto{\pgfqpoint{2.755474in}{2.063570in}}{\pgfqpoint{2.759865in}{2.074169in}}{\pgfqpoint{2.759865in}{2.085219in}}%
\pgfpathcurveto{\pgfqpoint{2.759865in}{2.096269in}}{\pgfqpoint{2.755474in}{2.106868in}}{\pgfqpoint{2.747661in}{2.114682in}}%
\pgfpathcurveto{\pgfqpoint{2.739847in}{2.122495in}}{\pgfqpoint{2.729248in}{2.126886in}}{\pgfqpoint{2.718198in}{2.126886in}}%
\pgfpathcurveto{\pgfqpoint{2.707148in}{2.126886in}}{\pgfqpoint{2.696549in}{2.122495in}}{\pgfqpoint{2.688735in}{2.114682in}}%
\pgfpathcurveto{\pgfqpoint{2.680921in}{2.106868in}}{\pgfqpoint{2.676531in}{2.096269in}}{\pgfqpoint{2.676531in}{2.085219in}}%
\pgfpathcurveto{\pgfqpoint{2.676531in}{2.074169in}}{\pgfqpoint{2.680921in}{2.063570in}}{\pgfqpoint{2.688735in}{2.055756in}}%
\pgfpathcurveto{\pgfqpoint{2.696549in}{2.047943in}}{\pgfqpoint{2.707148in}{2.043552in}}{\pgfqpoint{2.718198in}{2.043552in}}%
\pgfpathclose%
\pgfusepath{stroke,fill}%
\end{pgfscope}%
\begin{pgfscope}%
\pgfpathrectangle{\pgfqpoint{0.481978in}{0.331635in}}{\pgfqpoint{4.960000in}{3.696000in}}%
\pgfusepath{clip}%
\pgfsetbuttcap%
\pgfsetroundjoin%
\definecolor{currentfill}{rgb}{0.631373,0.788235,0.956863}%
\pgfsetfillcolor{currentfill}%
\pgfsetlinewidth{0.481800pt}%
\definecolor{currentstroke}{rgb}{1.000000,1.000000,1.000000}%
\pgfsetstrokecolor{currentstroke}%
\pgfsetdash{}{0pt}%
\pgfpathmoveto{\pgfqpoint{3.451581in}{1.890769in}}%
\pgfpathcurveto{\pgfqpoint{3.462632in}{1.890769in}}{\pgfqpoint{3.473231in}{1.895159in}}{\pgfqpoint{3.481044in}{1.902973in}}%
\pgfpathcurveto{\pgfqpoint{3.488858in}{1.910787in}}{\pgfqpoint{3.493248in}{1.921386in}}{\pgfqpoint{3.493248in}{1.932436in}}%
\pgfpathcurveto{\pgfqpoint{3.493248in}{1.943486in}}{\pgfqpoint{3.488858in}{1.954085in}}{\pgfqpoint{3.481044in}{1.961899in}}%
\pgfpathcurveto{\pgfqpoint{3.473231in}{1.969712in}}{\pgfqpoint{3.462632in}{1.974102in}}{\pgfqpoint{3.451581in}{1.974102in}}%
\pgfpathcurveto{\pgfqpoint{3.440531in}{1.974102in}}{\pgfqpoint{3.429932in}{1.969712in}}{\pgfqpoint{3.422119in}{1.961899in}}%
\pgfpathcurveto{\pgfqpoint{3.414305in}{1.954085in}}{\pgfqpoint{3.409915in}{1.943486in}}{\pgfqpoint{3.409915in}{1.932436in}}%
\pgfpathcurveto{\pgfqpoint{3.409915in}{1.921386in}}{\pgfqpoint{3.414305in}{1.910787in}}{\pgfqpoint{3.422119in}{1.902973in}}%
\pgfpathcurveto{\pgfqpoint{3.429932in}{1.895159in}}{\pgfqpoint{3.440531in}{1.890769in}}{\pgfqpoint{3.451581in}{1.890769in}}%
\pgfpathclose%
\pgfusepath{stroke,fill}%
\end{pgfscope}%
\begin{pgfscope}%
\pgfpathrectangle{\pgfqpoint{0.481978in}{0.331635in}}{\pgfqpoint{4.960000in}{3.696000in}}%
\pgfusepath{clip}%
\pgfsetbuttcap%
\pgfsetroundjoin%
\definecolor{currentfill}{rgb}{0.631373,0.788235,0.956863}%
\pgfsetfillcolor{currentfill}%
\pgfsetlinewidth{0.481800pt}%
\definecolor{currentstroke}{rgb}{1.000000,1.000000,1.000000}%
\pgfsetstrokecolor{currentstroke}%
\pgfsetdash{}{0pt}%
\pgfpathmoveto{\pgfqpoint{2.582592in}{1.895693in}}%
\pgfpathcurveto{\pgfqpoint{2.593642in}{1.895693in}}{\pgfqpoint{2.604241in}{1.900083in}}{\pgfqpoint{2.612055in}{1.907897in}}%
\pgfpathcurveto{\pgfqpoint{2.619869in}{1.915710in}}{\pgfqpoint{2.624259in}{1.926309in}}{\pgfqpoint{2.624259in}{1.937360in}}%
\pgfpathcurveto{\pgfqpoint{2.624259in}{1.948410in}}{\pgfqpoint{2.619869in}{1.959009in}}{\pgfqpoint{2.612055in}{1.966822in}}%
\pgfpathcurveto{\pgfqpoint{2.604241in}{1.974636in}}{\pgfqpoint{2.593642in}{1.979026in}}{\pgfqpoint{2.582592in}{1.979026in}}%
\pgfpathcurveto{\pgfqpoint{2.571542in}{1.979026in}}{\pgfqpoint{2.560943in}{1.974636in}}{\pgfqpoint{2.553130in}{1.966822in}}%
\pgfpathcurveto{\pgfqpoint{2.545316in}{1.959009in}}{\pgfqpoint{2.540926in}{1.948410in}}{\pgfqpoint{2.540926in}{1.937360in}}%
\pgfpathcurveto{\pgfqpoint{2.540926in}{1.926309in}}{\pgfqpoint{2.545316in}{1.915710in}}{\pgfqpoint{2.553130in}{1.907897in}}%
\pgfpathcurveto{\pgfqpoint{2.560943in}{1.900083in}}{\pgfqpoint{2.571542in}{1.895693in}}{\pgfqpoint{2.582592in}{1.895693in}}%
\pgfpathclose%
\pgfusepath{stroke,fill}%
\end{pgfscope}%
\begin{pgfscope}%
\pgfpathrectangle{\pgfqpoint{0.481978in}{0.331635in}}{\pgfqpoint{4.960000in}{3.696000in}}%
\pgfusepath{clip}%
\pgfsetbuttcap%
\pgfsetroundjoin%
\definecolor{currentfill}{rgb}{0.631373,0.788235,0.956863}%
\pgfsetfillcolor{currentfill}%
\pgfsetlinewidth{0.481800pt}%
\definecolor{currentstroke}{rgb}{1.000000,1.000000,1.000000}%
\pgfsetstrokecolor{currentstroke}%
\pgfsetdash{}{0pt}%
\pgfpathmoveto{\pgfqpoint{3.290958in}{3.404279in}}%
\pgfpathcurveto{\pgfqpoint{3.302008in}{3.404279in}}{\pgfqpoint{3.312607in}{3.408669in}}{\pgfqpoint{3.320421in}{3.416483in}}%
\pgfpathcurveto{\pgfqpoint{3.328234in}{3.424297in}}{\pgfqpoint{3.332625in}{3.434896in}}{\pgfqpoint{3.332625in}{3.445946in}}%
\pgfpathcurveto{\pgfqpoint{3.332625in}{3.456996in}}{\pgfqpoint{3.328234in}{3.467595in}}{\pgfqpoint{3.320421in}{3.475409in}}%
\pgfpathcurveto{\pgfqpoint{3.312607in}{3.483222in}}{\pgfqpoint{3.302008in}{3.487612in}}{\pgfqpoint{3.290958in}{3.487612in}}%
\pgfpathcurveto{\pgfqpoint{3.279908in}{3.487612in}}{\pgfqpoint{3.269309in}{3.483222in}}{\pgfqpoint{3.261495in}{3.475409in}}%
\pgfpathcurveto{\pgfqpoint{3.253682in}{3.467595in}}{\pgfqpoint{3.249291in}{3.456996in}}{\pgfqpoint{3.249291in}{3.445946in}}%
\pgfpathcurveto{\pgfqpoint{3.249291in}{3.434896in}}{\pgfqpoint{3.253682in}{3.424297in}}{\pgfqpoint{3.261495in}{3.416483in}}%
\pgfpathcurveto{\pgfqpoint{3.269309in}{3.408669in}}{\pgfqpoint{3.279908in}{3.404279in}}{\pgfqpoint{3.290958in}{3.404279in}}%
\pgfpathclose%
\pgfusepath{stroke,fill}%
\end{pgfscope}%
\begin{pgfscope}%
\pgfpathrectangle{\pgfqpoint{0.481978in}{0.331635in}}{\pgfqpoint{4.960000in}{3.696000in}}%
\pgfusepath{clip}%
\pgfsetbuttcap%
\pgfsetroundjoin%
\definecolor{currentfill}{rgb}{0.631373,0.788235,0.956863}%
\pgfsetfillcolor{currentfill}%
\pgfsetlinewidth{0.481800pt}%
\definecolor{currentstroke}{rgb}{1.000000,1.000000,1.000000}%
\pgfsetstrokecolor{currentstroke}%
\pgfsetdash{}{0pt}%
\pgfpathmoveto{\pgfqpoint{3.511086in}{2.607519in}}%
\pgfpathcurveto{\pgfqpoint{3.522136in}{2.607519in}}{\pgfqpoint{3.532735in}{2.611909in}}{\pgfqpoint{3.540548in}{2.619723in}}%
\pgfpathcurveto{\pgfqpoint{3.548362in}{2.627536in}}{\pgfqpoint{3.552752in}{2.638135in}}{\pgfqpoint{3.552752in}{2.649185in}}%
\pgfpathcurveto{\pgfqpoint{3.552752in}{2.660235in}}{\pgfqpoint{3.548362in}{2.670834in}}{\pgfqpoint{3.540548in}{2.678648in}}%
\pgfpathcurveto{\pgfqpoint{3.532735in}{2.686462in}}{\pgfqpoint{3.522136in}{2.690852in}}{\pgfqpoint{3.511086in}{2.690852in}}%
\pgfpathcurveto{\pgfqpoint{3.500036in}{2.690852in}}{\pgfqpoint{3.489437in}{2.686462in}}{\pgfqpoint{3.481623in}{2.678648in}}%
\pgfpathcurveto{\pgfqpoint{3.473809in}{2.670834in}}{\pgfqpoint{3.469419in}{2.660235in}}{\pgfqpoint{3.469419in}{2.649185in}}%
\pgfpathcurveto{\pgfqpoint{3.469419in}{2.638135in}}{\pgfqpoint{3.473809in}{2.627536in}}{\pgfqpoint{3.481623in}{2.619723in}}%
\pgfpathcurveto{\pgfqpoint{3.489437in}{2.611909in}}{\pgfqpoint{3.500036in}{2.607519in}}{\pgfqpoint{3.511086in}{2.607519in}}%
\pgfpathclose%
\pgfusepath{stroke,fill}%
\end{pgfscope}%
\begin{pgfscope}%
\pgfpathrectangle{\pgfqpoint{0.481978in}{0.331635in}}{\pgfqpoint{4.960000in}{3.696000in}}%
\pgfusepath{clip}%
\pgfsetbuttcap%
\pgfsetroundjoin%
\definecolor{currentfill}{rgb}{0.631373,0.788235,0.956863}%
\pgfsetfillcolor{currentfill}%
\pgfsetlinewidth{0.481800pt}%
\definecolor{currentstroke}{rgb}{1.000000,1.000000,1.000000}%
\pgfsetstrokecolor{currentstroke}%
\pgfsetdash{}{0pt}%
\pgfpathmoveto{\pgfqpoint{3.595851in}{3.424232in}}%
\pgfpathcurveto{\pgfqpoint{3.606901in}{3.424232in}}{\pgfqpoint{3.617500in}{3.428622in}}{\pgfqpoint{3.625314in}{3.436436in}}%
\pgfpathcurveto{\pgfqpoint{3.633128in}{3.444249in}}{\pgfqpoint{3.637518in}{3.454848in}}{\pgfqpoint{3.637518in}{3.465898in}}%
\pgfpathcurveto{\pgfqpoint{3.637518in}{3.476948in}}{\pgfqpoint{3.633128in}{3.487547in}}{\pgfqpoint{3.625314in}{3.495361in}}%
\pgfpathcurveto{\pgfqpoint{3.617500in}{3.503175in}}{\pgfqpoint{3.606901in}{3.507565in}}{\pgfqpoint{3.595851in}{3.507565in}}%
\pgfpathcurveto{\pgfqpoint{3.584801in}{3.507565in}}{\pgfqpoint{3.574202in}{3.503175in}}{\pgfqpoint{3.566389in}{3.495361in}}%
\pgfpathcurveto{\pgfqpoint{3.558575in}{3.487547in}}{\pgfqpoint{3.554185in}{3.476948in}}{\pgfqpoint{3.554185in}{3.465898in}}%
\pgfpathcurveto{\pgfqpoint{3.554185in}{3.454848in}}{\pgfqpoint{3.558575in}{3.444249in}}{\pgfqpoint{3.566389in}{3.436436in}}%
\pgfpathcurveto{\pgfqpoint{3.574202in}{3.428622in}}{\pgfqpoint{3.584801in}{3.424232in}}{\pgfqpoint{3.595851in}{3.424232in}}%
\pgfpathclose%
\pgfusepath{stroke,fill}%
\end{pgfscope}%
\begin{pgfscope}%
\pgfpathrectangle{\pgfqpoint{0.481978in}{0.331635in}}{\pgfqpoint{4.960000in}{3.696000in}}%
\pgfusepath{clip}%
\pgfsetbuttcap%
\pgfsetroundjoin%
\definecolor{currentfill}{rgb}{0.631373,0.788235,0.956863}%
\pgfsetfillcolor{currentfill}%
\pgfsetlinewidth{0.481800pt}%
\definecolor{currentstroke}{rgb}{1.000000,1.000000,1.000000}%
\pgfsetstrokecolor{currentstroke}%
\pgfsetdash{}{0pt}%
\pgfpathmoveto{\pgfqpoint{2.932962in}{2.172264in}}%
\pgfpathcurveto{\pgfqpoint{2.944012in}{2.172264in}}{\pgfqpoint{2.954611in}{2.176655in}}{\pgfqpoint{2.962425in}{2.184468in}}%
\pgfpathcurveto{\pgfqpoint{2.970238in}{2.192282in}}{\pgfqpoint{2.974628in}{2.202881in}}{\pgfqpoint{2.974628in}{2.213931in}}%
\pgfpathcurveto{\pgfqpoint{2.974628in}{2.224981in}}{\pgfqpoint{2.970238in}{2.235580in}}{\pgfqpoint{2.962425in}{2.243394in}}%
\pgfpathcurveto{\pgfqpoint{2.954611in}{2.251207in}}{\pgfqpoint{2.944012in}{2.255598in}}{\pgfqpoint{2.932962in}{2.255598in}}%
\pgfpathcurveto{\pgfqpoint{2.921912in}{2.255598in}}{\pgfqpoint{2.911313in}{2.251207in}}{\pgfqpoint{2.903499in}{2.243394in}}%
\pgfpathcurveto{\pgfqpoint{2.895685in}{2.235580in}}{\pgfqpoint{2.891295in}{2.224981in}}{\pgfqpoint{2.891295in}{2.213931in}}%
\pgfpathcurveto{\pgfqpoint{2.891295in}{2.202881in}}{\pgfqpoint{2.895685in}{2.192282in}}{\pgfqpoint{2.903499in}{2.184468in}}%
\pgfpathcurveto{\pgfqpoint{2.911313in}{2.176655in}}{\pgfqpoint{2.921912in}{2.172264in}}{\pgfqpoint{2.932962in}{2.172264in}}%
\pgfpathclose%
\pgfusepath{stroke,fill}%
\end{pgfscope}%
\begin{pgfscope}%
\pgfpathrectangle{\pgfqpoint{0.481978in}{0.331635in}}{\pgfqpoint{4.960000in}{3.696000in}}%
\pgfusepath{clip}%
\pgfsetbuttcap%
\pgfsetroundjoin%
\definecolor{currentfill}{rgb}{0.631373,0.788235,0.956863}%
\pgfsetfillcolor{currentfill}%
\pgfsetlinewidth{0.481800pt}%
\definecolor{currentstroke}{rgb}{1.000000,1.000000,1.000000}%
\pgfsetstrokecolor{currentstroke}%
\pgfsetdash{}{0pt}%
\pgfpathmoveto{\pgfqpoint{3.369825in}{2.722044in}}%
\pgfpathcurveto{\pgfqpoint{3.380875in}{2.722044in}}{\pgfqpoint{3.391474in}{2.726434in}}{\pgfqpoint{3.399288in}{2.734248in}}%
\pgfpathcurveto{\pgfqpoint{3.407101in}{2.742062in}}{\pgfqpoint{3.411492in}{2.752661in}}{\pgfqpoint{3.411492in}{2.763711in}}%
\pgfpathcurveto{\pgfqpoint{3.411492in}{2.774761in}}{\pgfqpoint{3.407101in}{2.785360in}}{\pgfqpoint{3.399288in}{2.793174in}}%
\pgfpathcurveto{\pgfqpoint{3.391474in}{2.800987in}}{\pgfqpoint{3.380875in}{2.805377in}}{\pgfqpoint{3.369825in}{2.805377in}}%
\pgfpathcurveto{\pgfqpoint{3.358775in}{2.805377in}}{\pgfqpoint{3.348176in}{2.800987in}}{\pgfqpoint{3.340362in}{2.793174in}}%
\pgfpathcurveto{\pgfqpoint{3.332549in}{2.785360in}}{\pgfqpoint{3.328158in}{2.774761in}}{\pgfqpoint{3.328158in}{2.763711in}}%
\pgfpathcurveto{\pgfqpoint{3.328158in}{2.752661in}}{\pgfqpoint{3.332549in}{2.742062in}}{\pgfqpoint{3.340362in}{2.734248in}}%
\pgfpathcurveto{\pgfqpoint{3.348176in}{2.726434in}}{\pgfqpoint{3.358775in}{2.722044in}}{\pgfqpoint{3.369825in}{2.722044in}}%
\pgfpathclose%
\pgfusepath{stroke,fill}%
\end{pgfscope}%
\begin{pgfscope}%
\pgfpathrectangle{\pgfqpoint{0.481978in}{0.331635in}}{\pgfqpoint{4.960000in}{3.696000in}}%
\pgfusepath{clip}%
\pgfsetbuttcap%
\pgfsetroundjoin%
\definecolor{currentfill}{rgb}{0.631373,0.788235,0.956863}%
\pgfsetfillcolor{currentfill}%
\pgfsetlinewidth{0.481800pt}%
\definecolor{currentstroke}{rgb}{1.000000,1.000000,1.000000}%
\pgfsetstrokecolor{currentstroke}%
\pgfsetdash{}{0pt}%
\pgfpathmoveto{\pgfqpoint{3.062456in}{1.928177in}}%
\pgfpathcurveto{\pgfqpoint{3.073506in}{1.928177in}}{\pgfqpoint{3.084106in}{1.932568in}}{\pgfqpoint{3.091919in}{1.940381in}}%
\pgfpathcurveto{\pgfqpoint{3.099733in}{1.948195in}}{\pgfqpoint{3.104123in}{1.958794in}}{\pgfqpoint{3.104123in}{1.969844in}}%
\pgfpathcurveto{\pgfqpoint{3.104123in}{1.980894in}}{\pgfqpoint{3.099733in}{1.991493in}}{\pgfqpoint{3.091919in}{1.999307in}}%
\pgfpathcurveto{\pgfqpoint{3.084106in}{2.007120in}}{\pgfqpoint{3.073506in}{2.011511in}}{\pgfqpoint{3.062456in}{2.011511in}}%
\pgfpathcurveto{\pgfqpoint{3.051406in}{2.011511in}}{\pgfqpoint{3.040807in}{2.007120in}}{\pgfqpoint{3.032994in}{1.999307in}}%
\pgfpathcurveto{\pgfqpoint{3.025180in}{1.991493in}}{\pgfqpoint{3.020790in}{1.980894in}}{\pgfqpoint{3.020790in}{1.969844in}}%
\pgfpathcurveto{\pgfqpoint{3.020790in}{1.958794in}}{\pgfqpoint{3.025180in}{1.948195in}}{\pgfqpoint{3.032994in}{1.940381in}}%
\pgfpathcurveto{\pgfqpoint{3.040807in}{1.932568in}}{\pgfqpoint{3.051406in}{1.928177in}}{\pgfqpoint{3.062456in}{1.928177in}}%
\pgfpathclose%
\pgfusepath{stroke,fill}%
\end{pgfscope}%
\begin{pgfscope}%
\pgfpathrectangle{\pgfqpoint{0.481978in}{0.331635in}}{\pgfqpoint{4.960000in}{3.696000in}}%
\pgfusepath{clip}%
\pgfsetbuttcap%
\pgfsetroundjoin%
\definecolor{currentfill}{rgb}{0.631373,0.788235,0.956863}%
\pgfsetfillcolor{currentfill}%
\pgfsetlinewidth{0.481800pt}%
\definecolor{currentstroke}{rgb}{1.000000,1.000000,1.000000}%
\pgfsetstrokecolor{currentstroke}%
\pgfsetdash{}{0pt}%
\pgfpathmoveto{\pgfqpoint{4.682831in}{3.483020in}}%
\pgfpathcurveto{\pgfqpoint{4.693881in}{3.483020in}}{\pgfqpoint{4.704480in}{3.487410in}}{\pgfqpoint{4.712294in}{3.495224in}}%
\pgfpathcurveto{\pgfqpoint{4.720108in}{3.503038in}}{\pgfqpoint{4.724498in}{3.513637in}}{\pgfqpoint{4.724498in}{3.524687in}}%
\pgfpathcurveto{\pgfqpoint{4.724498in}{3.535737in}}{\pgfqpoint{4.720108in}{3.546336in}}{\pgfqpoint{4.712294in}{3.554150in}}%
\pgfpathcurveto{\pgfqpoint{4.704480in}{3.561963in}}{\pgfqpoint{4.693881in}{3.566353in}}{\pgfqpoint{4.682831in}{3.566353in}}%
\pgfpathcurveto{\pgfqpoint{4.671781in}{3.566353in}}{\pgfqpoint{4.661182in}{3.561963in}}{\pgfqpoint{4.653368in}{3.554150in}}%
\pgfpathcurveto{\pgfqpoint{4.645555in}{3.546336in}}{\pgfqpoint{4.641164in}{3.535737in}}{\pgfqpoint{4.641164in}{3.524687in}}%
\pgfpathcurveto{\pgfqpoint{4.641164in}{3.513637in}}{\pgfqpoint{4.645555in}{3.503038in}}{\pgfqpoint{4.653368in}{3.495224in}}%
\pgfpathcurveto{\pgfqpoint{4.661182in}{3.487410in}}{\pgfqpoint{4.671781in}{3.483020in}}{\pgfqpoint{4.682831in}{3.483020in}}%
\pgfpathclose%
\pgfusepath{stroke,fill}%
\end{pgfscope}%
\begin{pgfscope}%
\pgfpathrectangle{\pgfqpoint{0.481978in}{0.331635in}}{\pgfqpoint{4.960000in}{3.696000in}}%
\pgfusepath{clip}%
\pgfsetbuttcap%
\pgfsetroundjoin%
\definecolor{currentfill}{rgb}{0.631373,0.788235,0.956863}%
\pgfsetfillcolor{currentfill}%
\pgfsetlinewidth{0.481800pt}%
\definecolor{currentstroke}{rgb}{1.000000,1.000000,1.000000}%
\pgfsetstrokecolor{currentstroke}%
\pgfsetdash{}{0pt}%
\pgfpathmoveto{\pgfqpoint{3.205444in}{3.635403in}}%
\pgfpathcurveto{\pgfqpoint{3.216494in}{3.635403in}}{\pgfqpoint{3.227093in}{3.639793in}}{\pgfqpoint{3.234907in}{3.647607in}}%
\pgfpathcurveto{\pgfqpoint{3.242721in}{3.655421in}}{\pgfqpoint{3.247111in}{3.666020in}}{\pgfqpoint{3.247111in}{3.677070in}}%
\pgfpathcurveto{\pgfqpoint{3.247111in}{3.688120in}}{\pgfqpoint{3.242721in}{3.698719in}}{\pgfqpoint{3.234907in}{3.706533in}}%
\pgfpathcurveto{\pgfqpoint{3.227093in}{3.714346in}}{\pgfqpoint{3.216494in}{3.718737in}}{\pgfqpoint{3.205444in}{3.718737in}}%
\pgfpathcurveto{\pgfqpoint{3.194394in}{3.718737in}}{\pgfqpoint{3.183795in}{3.714346in}}{\pgfqpoint{3.175982in}{3.706533in}}%
\pgfpathcurveto{\pgfqpoint{3.168168in}{3.698719in}}{\pgfqpoint{3.163778in}{3.688120in}}{\pgfqpoint{3.163778in}{3.677070in}}%
\pgfpathcurveto{\pgfqpoint{3.163778in}{3.666020in}}{\pgfqpoint{3.168168in}{3.655421in}}{\pgfqpoint{3.175982in}{3.647607in}}%
\pgfpathcurveto{\pgfqpoint{3.183795in}{3.639793in}}{\pgfqpoint{3.194394in}{3.635403in}}{\pgfqpoint{3.205444in}{3.635403in}}%
\pgfpathclose%
\pgfusepath{stroke,fill}%
\end{pgfscope}%
\begin{pgfscope}%
\pgfpathrectangle{\pgfqpoint{0.481978in}{0.331635in}}{\pgfqpoint{4.960000in}{3.696000in}}%
\pgfusepath{clip}%
\pgfsetbuttcap%
\pgfsetroundjoin%
\definecolor{currentfill}{rgb}{0.631373,0.788235,0.956863}%
\pgfsetfillcolor{currentfill}%
\pgfsetlinewidth{0.481800pt}%
\definecolor{currentstroke}{rgb}{1.000000,1.000000,1.000000}%
\pgfsetstrokecolor{currentstroke}%
\pgfsetdash{}{0pt}%
\pgfpathmoveto{\pgfqpoint{3.649476in}{3.204734in}}%
\pgfpathcurveto{\pgfqpoint{3.660526in}{3.204734in}}{\pgfqpoint{3.671125in}{3.209124in}}{\pgfqpoint{3.678939in}{3.216937in}}%
\pgfpathcurveto{\pgfqpoint{3.686752in}{3.224751in}}{\pgfqpoint{3.691142in}{3.235350in}}{\pgfqpoint{3.691142in}{3.246400in}}%
\pgfpathcurveto{\pgfqpoint{3.691142in}{3.257450in}}{\pgfqpoint{3.686752in}{3.268049in}}{\pgfqpoint{3.678939in}{3.275863in}}%
\pgfpathcurveto{\pgfqpoint{3.671125in}{3.283677in}}{\pgfqpoint{3.660526in}{3.288067in}}{\pgfqpoint{3.649476in}{3.288067in}}%
\pgfpathcurveto{\pgfqpoint{3.638426in}{3.288067in}}{\pgfqpoint{3.627827in}{3.283677in}}{\pgfqpoint{3.620013in}{3.275863in}}%
\pgfpathcurveto{\pgfqpoint{3.612199in}{3.268049in}}{\pgfqpoint{3.607809in}{3.257450in}}{\pgfqpoint{3.607809in}{3.246400in}}%
\pgfpathcurveto{\pgfqpoint{3.607809in}{3.235350in}}{\pgfqpoint{3.612199in}{3.224751in}}{\pgfqpoint{3.620013in}{3.216937in}}%
\pgfpathcurveto{\pgfqpoint{3.627827in}{3.209124in}}{\pgfqpoint{3.638426in}{3.204734in}}{\pgfqpoint{3.649476in}{3.204734in}}%
\pgfpathclose%
\pgfusepath{stroke,fill}%
\end{pgfscope}%
\begin{pgfscope}%
\pgfpathrectangle{\pgfqpoint{0.481978in}{0.331635in}}{\pgfqpoint{4.960000in}{3.696000in}}%
\pgfusepath{clip}%
\pgfsetbuttcap%
\pgfsetroundjoin%
\definecolor{currentfill}{rgb}{0.631373,0.788235,0.956863}%
\pgfsetfillcolor{currentfill}%
\pgfsetlinewidth{0.481800pt}%
\definecolor{currentstroke}{rgb}{1.000000,1.000000,1.000000}%
\pgfsetstrokecolor{currentstroke}%
\pgfsetdash{}{0pt}%
\pgfpathmoveto{\pgfqpoint{2.930723in}{2.096317in}}%
\pgfpathcurveto{\pgfqpoint{2.941773in}{2.096317in}}{\pgfqpoint{2.952372in}{2.100707in}}{\pgfqpoint{2.960186in}{2.108521in}}%
\pgfpathcurveto{\pgfqpoint{2.968000in}{2.116335in}}{\pgfqpoint{2.972390in}{2.126934in}}{\pgfqpoint{2.972390in}{2.137984in}}%
\pgfpathcurveto{\pgfqpoint{2.972390in}{2.149034in}}{\pgfqpoint{2.968000in}{2.159633in}}{\pgfqpoint{2.960186in}{2.167447in}}%
\pgfpathcurveto{\pgfqpoint{2.952372in}{2.175260in}}{\pgfqpoint{2.941773in}{2.179650in}}{\pgfqpoint{2.930723in}{2.179650in}}%
\pgfpathcurveto{\pgfqpoint{2.919673in}{2.179650in}}{\pgfqpoint{2.909074in}{2.175260in}}{\pgfqpoint{2.901261in}{2.167447in}}%
\pgfpathcurveto{\pgfqpoint{2.893447in}{2.159633in}}{\pgfqpoint{2.889057in}{2.149034in}}{\pgfqpoint{2.889057in}{2.137984in}}%
\pgfpathcurveto{\pgfqpoint{2.889057in}{2.126934in}}{\pgfqpoint{2.893447in}{2.116335in}}{\pgfqpoint{2.901261in}{2.108521in}}%
\pgfpathcurveto{\pgfqpoint{2.909074in}{2.100707in}}{\pgfqpoint{2.919673in}{2.096317in}}{\pgfqpoint{2.930723in}{2.096317in}}%
\pgfpathclose%
\pgfusepath{stroke,fill}%
\end{pgfscope}%
\begin{pgfscope}%
\pgfpathrectangle{\pgfqpoint{0.481978in}{0.331635in}}{\pgfqpoint{4.960000in}{3.696000in}}%
\pgfusepath{clip}%
\pgfsetbuttcap%
\pgfsetroundjoin%
\definecolor{currentfill}{rgb}{0.631373,0.788235,0.956863}%
\pgfsetfillcolor{currentfill}%
\pgfsetlinewidth{0.481800pt}%
\definecolor{currentstroke}{rgb}{1.000000,1.000000,1.000000}%
\pgfsetstrokecolor{currentstroke}%
\pgfsetdash{}{0pt}%
\pgfpathmoveto{\pgfqpoint{4.204178in}{1.641652in}}%
\pgfpathcurveto{\pgfqpoint{4.215229in}{1.641652in}}{\pgfqpoint{4.225828in}{1.646042in}}{\pgfqpoint{4.233641in}{1.653856in}}%
\pgfpathcurveto{\pgfqpoint{4.241455in}{1.661669in}}{\pgfqpoint{4.245845in}{1.672268in}}{\pgfqpoint{4.245845in}{1.683319in}}%
\pgfpathcurveto{\pgfqpoint{4.245845in}{1.694369in}}{\pgfqpoint{4.241455in}{1.704968in}}{\pgfqpoint{4.233641in}{1.712781in}}%
\pgfpathcurveto{\pgfqpoint{4.225828in}{1.720595in}}{\pgfqpoint{4.215229in}{1.724985in}}{\pgfqpoint{4.204178in}{1.724985in}}%
\pgfpathcurveto{\pgfqpoint{4.193128in}{1.724985in}}{\pgfqpoint{4.182529in}{1.720595in}}{\pgfqpoint{4.174716in}{1.712781in}}%
\pgfpathcurveto{\pgfqpoint{4.166902in}{1.704968in}}{\pgfqpoint{4.162512in}{1.694369in}}{\pgfqpoint{4.162512in}{1.683319in}}%
\pgfpathcurveto{\pgfqpoint{4.162512in}{1.672268in}}{\pgfqpoint{4.166902in}{1.661669in}}{\pgfqpoint{4.174716in}{1.653856in}}%
\pgfpathcurveto{\pgfqpoint{4.182529in}{1.646042in}}{\pgfqpoint{4.193128in}{1.641652in}}{\pgfqpoint{4.204178in}{1.641652in}}%
\pgfpathclose%
\pgfusepath{stroke,fill}%
\end{pgfscope}%
\begin{pgfscope}%
\pgfpathrectangle{\pgfqpoint{0.481978in}{0.331635in}}{\pgfqpoint{4.960000in}{3.696000in}}%
\pgfusepath{clip}%
\pgfsetbuttcap%
\pgfsetroundjoin%
\definecolor{currentfill}{rgb}{0.631373,0.788235,0.956863}%
\pgfsetfillcolor{currentfill}%
\pgfsetlinewidth{0.481800pt}%
\definecolor{currentstroke}{rgb}{1.000000,1.000000,1.000000}%
\pgfsetstrokecolor{currentstroke}%
\pgfsetdash{}{0pt}%
\pgfpathmoveto{\pgfqpoint{4.511769in}{3.119665in}}%
\pgfpathcurveto{\pgfqpoint{4.522819in}{3.119665in}}{\pgfqpoint{4.533418in}{3.124055in}}{\pgfqpoint{4.541232in}{3.131869in}}%
\pgfpathcurveto{\pgfqpoint{4.549046in}{3.139682in}}{\pgfqpoint{4.553436in}{3.150281in}}{\pgfqpoint{4.553436in}{3.161331in}}%
\pgfpathcurveto{\pgfqpoint{4.553436in}{3.172381in}}{\pgfqpoint{4.549046in}{3.182980in}}{\pgfqpoint{4.541232in}{3.190794in}}%
\pgfpathcurveto{\pgfqpoint{4.533418in}{3.198608in}}{\pgfqpoint{4.522819in}{3.202998in}}{\pgfqpoint{4.511769in}{3.202998in}}%
\pgfpathcurveto{\pgfqpoint{4.500719in}{3.202998in}}{\pgfqpoint{4.490120in}{3.198608in}}{\pgfqpoint{4.482306in}{3.190794in}}%
\pgfpathcurveto{\pgfqpoint{4.474493in}{3.182980in}}{\pgfqpoint{4.470103in}{3.172381in}}{\pgfqpoint{4.470103in}{3.161331in}}%
\pgfpathcurveto{\pgfqpoint{4.470103in}{3.150281in}}{\pgfqpoint{4.474493in}{3.139682in}}{\pgfqpoint{4.482306in}{3.131869in}}%
\pgfpathcurveto{\pgfqpoint{4.490120in}{3.124055in}}{\pgfqpoint{4.500719in}{3.119665in}}{\pgfqpoint{4.511769in}{3.119665in}}%
\pgfpathclose%
\pgfusepath{stroke,fill}%
\end{pgfscope}%
\begin{pgfscope}%
\pgfpathrectangle{\pgfqpoint{0.481978in}{0.331635in}}{\pgfqpoint{4.960000in}{3.696000in}}%
\pgfusepath{clip}%
\pgfsetbuttcap%
\pgfsetroundjoin%
\definecolor{currentfill}{rgb}{0.631373,0.788235,0.956863}%
\pgfsetfillcolor{currentfill}%
\pgfsetlinewidth{0.481800pt}%
\definecolor{currentstroke}{rgb}{1.000000,1.000000,1.000000}%
\pgfsetstrokecolor{currentstroke}%
\pgfsetdash{}{0pt}%
\pgfpathmoveto{\pgfqpoint{2.560493in}{1.976906in}}%
\pgfpathcurveto{\pgfqpoint{2.571543in}{1.976906in}}{\pgfqpoint{2.582142in}{1.981296in}}{\pgfqpoint{2.589956in}{1.989110in}}%
\pgfpathcurveto{\pgfqpoint{2.597769in}{1.996924in}}{\pgfqpoint{2.602160in}{2.007523in}}{\pgfqpoint{2.602160in}{2.018573in}}%
\pgfpathcurveto{\pgfqpoint{2.602160in}{2.029623in}}{\pgfqpoint{2.597769in}{2.040222in}}{\pgfqpoint{2.589956in}{2.048036in}}%
\pgfpathcurveto{\pgfqpoint{2.582142in}{2.055849in}}{\pgfqpoint{2.571543in}{2.060239in}}{\pgfqpoint{2.560493in}{2.060239in}}%
\pgfpathcurveto{\pgfqpoint{2.549443in}{2.060239in}}{\pgfqpoint{2.538844in}{2.055849in}}{\pgfqpoint{2.531030in}{2.048036in}}%
\pgfpathcurveto{\pgfqpoint{2.523217in}{2.040222in}}{\pgfqpoint{2.518826in}{2.029623in}}{\pgfqpoint{2.518826in}{2.018573in}}%
\pgfpathcurveto{\pgfqpoint{2.518826in}{2.007523in}}{\pgfqpoint{2.523217in}{1.996924in}}{\pgfqpoint{2.531030in}{1.989110in}}%
\pgfpathcurveto{\pgfqpoint{2.538844in}{1.981296in}}{\pgfqpoint{2.549443in}{1.976906in}}{\pgfqpoint{2.560493in}{1.976906in}}%
\pgfpathclose%
\pgfusepath{stroke,fill}%
\end{pgfscope}%
\begin{pgfscope}%
\pgfpathrectangle{\pgfqpoint{0.481978in}{0.331635in}}{\pgfqpoint{4.960000in}{3.696000in}}%
\pgfusepath{clip}%
\pgfsetbuttcap%
\pgfsetroundjoin%
\definecolor{currentfill}{rgb}{0.631373,0.788235,0.956863}%
\pgfsetfillcolor{currentfill}%
\pgfsetlinewidth{0.481800pt}%
\definecolor{currentstroke}{rgb}{1.000000,1.000000,1.000000}%
\pgfsetstrokecolor{currentstroke}%
\pgfsetdash{}{0pt}%
\pgfpathmoveto{\pgfqpoint{2.942503in}{2.371582in}}%
\pgfpathcurveto{\pgfqpoint{2.953553in}{2.371582in}}{\pgfqpoint{2.964152in}{2.375972in}}{\pgfqpoint{2.971965in}{2.383786in}}%
\pgfpathcurveto{\pgfqpoint{2.979779in}{2.391600in}}{\pgfqpoint{2.984169in}{2.402199in}}{\pgfqpoint{2.984169in}{2.413249in}}%
\pgfpathcurveto{\pgfqpoint{2.984169in}{2.424299in}}{\pgfqpoint{2.979779in}{2.434898in}}{\pgfqpoint{2.971965in}{2.442712in}}%
\pgfpathcurveto{\pgfqpoint{2.964152in}{2.450525in}}{\pgfqpoint{2.953553in}{2.454916in}}{\pgfqpoint{2.942503in}{2.454916in}}%
\pgfpathcurveto{\pgfqpoint{2.931453in}{2.454916in}}{\pgfqpoint{2.920853in}{2.450525in}}{\pgfqpoint{2.913040in}{2.442712in}}%
\pgfpathcurveto{\pgfqpoint{2.905226in}{2.434898in}}{\pgfqpoint{2.900836in}{2.424299in}}{\pgfqpoint{2.900836in}{2.413249in}}%
\pgfpathcurveto{\pgfqpoint{2.900836in}{2.402199in}}{\pgfqpoint{2.905226in}{2.391600in}}{\pgfqpoint{2.913040in}{2.383786in}}%
\pgfpathcurveto{\pgfqpoint{2.920853in}{2.375972in}}{\pgfqpoint{2.931453in}{2.371582in}}{\pgfqpoint{2.942503in}{2.371582in}}%
\pgfpathclose%
\pgfusepath{stroke,fill}%
\end{pgfscope}%
\begin{pgfscope}%
\pgfpathrectangle{\pgfqpoint{0.481978in}{0.331635in}}{\pgfqpoint{4.960000in}{3.696000in}}%
\pgfusepath{clip}%
\pgfsetbuttcap%
\pgfsetroundjoin%
\definecolor{currentfill}{rgb}{0.631373,0.788235,0.956863}%
\pgfsetfillcolor{currentfill}%
\pgfsetlinewidth{0.481800pt}%
\definecolor{currentstroke}{rgb}{1.000000,1.000000,1.000000}%
\pgfsetstrokecolor{currentstroke}%
\pgfsetdash{}{0pt}%
\pgfpathmoveto{\pgfqpoint{3.195028in}{1.718003in}}%
\pgfpathcurveto{\pgfqpoint{3.206078in}{1.718003in}}{\pgfqpoint{3.216677in}{1.722393in}}{\pgfqpoint{3.224491in}{1.730207in}}%
\pgfpathcurveto{\pgfqpoint{3.232305in}{1.738021in}}{\pgfqpoint{3.236695in}{1.748620in}}{\pgfqpoint{3.236695in}{1.759670in}}%
\pgfpathcurveto{\pgfqpoint{3.236695in}{1.770720in}}{\pgfqpoint{3.232305in}{1.781319in}}{\pgfqpoint{3.224491in}{1.789133in}}%
\pgfpathcurveto{\pgfqpoint{3.216677in}{1.796946in}}{\pgfqpoint{3.206078in}{1.801337in}}{\pgfqpoint{3.195028in}{1.801337in}}%
\pgfpathcurveto{\pgfqpoint{3.183978in}{1.801337in}}{\pgfqpoint{3.173379in}{1.796946in}}{\pgfqpoint{3.165565in}{1.789133in}}%
\pgfpathcurveto{\pgfqpoint{3.157752in}{1.781319in}}{\pgfqpoint{3.153362in}{1.770720in}}{\pgfqpoint{3.153362in}{1.759670in}}%
\pgfpathcurveto{\pgfqpoint{3.153362in}{1.748620in}}{\pgfqpoint{3.157752in}{1.738021in}}{\pgfqpoint{3.165565in}{1.730207in}}%
\pgfpathcurveto{\pgfqpoint{3.173379in}{1.722393in}}{\pgfqpoint{3.183978in}{1.718003in}}{\pgfqpoint{3.195028in}{1.718003in}}%
\pgfpathclose%
\pgfusepath{stroke,fill}%
\end{pgfscope}%
\begin{pgfscope}%
\pgfpathrectangle{\pgfqpoint{0.481978in}{0.331635in}}{\pgfqpoint{4.960000in}{3.696000in}}%
\pgfusepath{clip}%
\pgfsetbuttcap%
\pgfsetroundjoin%
\definecolor{currentfill}{rgb}{0.631373,0.788235,0.956863}%
\pgfsetfillcolor{currentfill}%
\pgfsetlinewidth{0.481800pt}%
\definecolor{currentstroke}{rgb}{1.000000,1.000000,1.000000}%
\pgfsetstrokecolor{currentstroke}%
\pgfsetdash{}{0pt}%
\pgfpathmoveto{\pgfqpoint{2.587739in}{2.574840in}}%
\pgfpathcurveto{\pgfqpoint{2.598789in}{2.574840in}}{\pgfqpoint{2.609388in}{2.579230in}}{\pgfqpoint{2.617202in}{2.587044in}}%
\pgfpathcurveto{\pgfqpoint{2.625015in}{2.594857in}}{\pgfqpoint{2.629406in}{2.605456in}}{\pgfqpoint{2.629406in}{2.616506in}}%
\pgfpathcurveto{\pgfqpoint{2.629406in}{2.627556in}}{\pgfqpoint{2.625015in}{2.638156in}}{\pgfqpoint{2.617202in}{2.645969in}}%
\pgfpathcurveto{\pgfqpoint{2.609388in}{2.653783in}}{\pgfqpoint{2.598789in}{2.658173in}}{\pgfqpoint{2.587739in}{2.658173in}}%
\pgfpathcurveto{\pgfqpoint{2.576689in}{2.658173in}}{\pgfqpoint{2.566090in}{2.653783in}}{\pgfqpoint{2.558276in}{2.645969in}}%
\pgfpathcurveto{\pgfqpoint{2.550463in}{2.638156in}}{\pgfqpoint{2.546072in}{2.627556in}}{\pgfqpoint{2.546072in}{2.616506in}}%
\pgfpathcurveto{\pgfqpoint{2.546072in}{2.605456in}}{\pgfqpoint{2.550463in}{2.594857in}}{\pgfqpoint{2.558276in}{2.587044in}}%
\pgfpathcurveto{\pgfqpoint{2.566090in}{2.579230in}}{\pgfqpoint{2.576689in}{2.574840in}}{\pgfqpoint{2.587739in}{2.574840in}}%
\pgfpathclose%
\pgfusepath{stroke,fill}%
\end{pgfscope}%
\begin{pgfscope}%
\pgfpathrectangle{\pgfqpoint{0.481978in}{0.331635in}}{\pgfqpoint{4.960000in}{3.696000in}}%
\pgfusepath{clip}%
\pgfsetbuttcap%
\pgfsetroundjoin%
\definecolor{currentfill}{rgb}{0.631373,0.788235,0.956863}%
\pgfsetfillcolor{currentfill}%
\pgfsetlinewidth{0.481800pt}%
\definecolor{currentstroke}{rgb}{1.000000,1.000000,1.000000}%
\pgfsetstrokecolor{currentstroke}%
\pgfsetdash{}{0pt}%
\pgfpathmoveto{\pgfqpoint{3.162498in}{2.421435in}}%
\pgfpathcurveto{\pgfqpoint{3.173548in}{2.421435in}}{\pgfqpoint{3.184147in}{2.425825in}}{\pgfqpoint{3.191960in}{2.433639in}}%
\pgfpathcurveto{\pgfqpoint{3.199774in}{2.441452in}}{\pgfqpoint{3.204164in}{2.452052in}}{\pgfqpoint{3.204164in}{2.463102in}}%
\pgfpathcurveto{\pgfqpoint{3.204164in}{2.474152in}}{\pgfqpoint{3.199774in}{2.484751in}}{\pgfqpoint{3.191960in}{2.492564in}}%
\pgfpathcurveto{\pgfqpoint{3.184147in}{2.500378in}}{\pgfqpoint{3.173548in}{2.504768in}}{\pgfqpoint{3.162498in}{2.504768in}}%
\pgfpathcurveto{\pgfqpoint{3.151448in}{2.504768in}}{\pgfqpoint{3.140849in}{2.500378in}}{\pgfqpoint{3.133035in}{2.492564in}}%
\pgfpathcurveto{\pgfqpoint{3.125221in}{2.484751in}}{\pgfqpoint{3.120831in}{2.474152in}}{\pgfqpoint{3.120831in}{2.463102in}}%
\pgfpathcurveto{\pgfqpoint{3.120831in}{2.452052in}}{\pgfqpoint{3.125221in}{2.441452in}}{\pgfqpoint{3.133035in}{2.433639in}}%
\pgfpathcurveto{\pgfqpoint{3.140849in}{2.425825in}}{\pgfqpoint{3.151448in}{2.421435in}}{\pgfqpoint{3.162498in}{2.421435in}}%
\pgfpathclose%
\pgfusepath{stroke,fill}%
\end{pgfscope}%
\begin{pgfscope}%
\pgfpathrectangle{\pgfqpoint{0.481978in}{0.331635in}}{\pgfqpoint{4.960000in}{3.696000in}}%
\pgfusepath{clip}%
\pgfsetbuttcap%
\pgfsetroundjoin%
\definecolor{currentfill}{rgb}{0.631373,0.788235,0.956863}%
\pgfsetfillcolor{currentfill}%
\pgfsetlinewidth{0.481800pt}%
\definecolor{currentstroke}{rgb}{1.000000,1.000000,1.000000}%
\pgfsetstrokecolor{currentstroke}%
\pgfsetdash{}{0pt}%
\pgfpathmoveto{\pgfqpoint{2.743558in}{2.131229in}}%
\pgfpathcurveto{\pgfqpoint{2.754608in}{2.131229in}}{\pgfqpoint{2.765207in}{2.135620in}}{\pgfqpoint{2.773021in}{2.143433in}}%
\pgfpathcurveto{\pgfqpoint{2.780835in}{2.151247in}}{\pgfqpoint{2.785225in}{2.161846in}}{\pgfqpoint{2.785225in}{2.172896in}}%
\pgfpathcurveto{\pgfqpoint{2.785225in}{2.183946in}}{\pgfqpoint{2.780835in}{2.194545in}}{\pgfqpoint{2.773021in}{2.202359in}}%
\pgfpathcurveto{\pgfqpoint{2.765207in}{2.210172in}}{\pgfqpoint{2.754608in}{2.214563in}}{\pgfqpoint{2.743558in}{2.214563in}}%
\pgfpathcurveto{\pgfqpoint{2.732508in}{2.214563in}}{\pgfqpoint{2.721909in}{2.210172in}}{\pgfqpoint{2.714095in}{2.202359in}}%
\pgfpathcurveto{\pgfqpoint{2.706282in}{2.194545in}}{\pgfqpoint{2.701891in}{2.183946in}}{\pgfqpoint{2.701891in}{2.172896in}}%
\pgfpathcurveto{\pgfqpoint{2.701891in}{2.161846in}}{\pgfqpoint{2.706282in}{2.151247in}}{\pgfqpoint{2.714095in}{2.143433in}}%
\pgfpathcurveto{\pgfqpoint{2.721909in}{2.135620in}}{\pgfqpoint{2.732508in}{2.131229in}}{\pgfqpoint{2.743558in}{2.131229in}}%
\pgfpathclose%
\pgfusepath{stroke,fill}%
\end{pgfscope}%
\begin{pgfscope}%
\pgfpathrectangle{\pgfqpoint{0.481978in}{0.331635in}}{\pgfqpoint{4.960000in}{3.696000in}}%
\pgfusepath{clip}%
\pgfsetbuttcap%
\pgfsetroundjoin%
\definecolor{currentfill}{rgb}{0.631373,0.788235,0.956863}%
\pgfsetfillcolor{currentfill}%
\pgfsetlinewidth{0.481800pt}%
\definecolor{currentstroke}{rgb}{1.000000,1.000000,1.000000}%
\pgfsetstrokecolor{currentstroke}%
\pgfsetdash{}{0pt}%
\pgfpathmoveto{\pgfqpoint{3.019641in}{3.060947in}}%
\pgfpathcurveto{\pgfqpoint{3.030691in}{3.060947in}}{\pgfqpoint{3.041290in}{3.065338in}}{\pgfqpoint{3.049103in}{3.073151in}}%
\pgfpathcurveto{\pgfqpoint{3.056917in}{3.080965in}}{\pgfqpoint{3.061307in}{3.091564in}}{\pgfqpoint{3.061307in}{3.102614in}}%
\pgfpathcurveto{\pgfqpoint{3.061307in}{3.113664in}}{\pgfqpoint{3.056917in}{3.124263in}}{\pgfqpoint{3.049103in}{3.132077in}}%
\pgfpathcurveto{\pgfqpoint{3.041290in}{3.139890in}}{\pgfqpoint{3.030691in}{3.144281in}}{\pgfqpoint{3.019641in}{3.144281in}}%
\pgfpathcurveto{\pgfqpoint{3.008591in}{3.144281in}}{\pgfqpoint{2.997991in}{3.139890in}}{\pgfqpoint{2.990178in}{3.132077in}}%
\pgfpathcurveto{\pgfqpoint{2.982364in}{3.124263in}}{\pgfqpoint{2.977974in}{3.113664in}}{\pgfqpoint{2.977974in}{3.102614in}}%
\pgfpathcurveto{\pgfqpoint{2.977974in}{3.091564in}}{\pgfqpoint{2.982364in}{3.080965in}}{\pgfqpoint{2.990178in}{3.073151in}}%
\pgfpathcurveto{\pgfqpoint{2.997991in}{3.065338in}}{\pgfqpoint{3.008591in}{3.060947in}}{\pgfqpoint{3.019641in}{3.060947in}}%
\pgfpathclose%
\pgfusepath{stroke,fill}%
\end{pgfscope}%
\begin{pgfscope}%
\pgfpathrectangle{\pgfqpoint{0.481978in}{0.331635in}}{\pgfqpoint{4.960000in}{3.696000in}}%
\pgfusepath{clip}%
\pgfsetbuttcap%
\pgfsetroundjoin%
\definecolor{currentfill}{rgb}{0.631373,0.788235,0.956863}%
\pgfsetfillcolor{currentfill}%
\pgfsetlinewidth{0.481800pt}%
\definecolor{currentstroke}{rgb}{1.000000,1.000000,1.000000}%
\pgfsetstrokecolor{currentstroke}%
\pgfsetdash{}{0pt}%
\pgfpathmoveto{\pgfqpoint{3.607583in}{3.286174in}}%
\pgfpathcurveto{\pgfqpoint{3.618634in}{3.286174in}}{\pgfqpoint{3.629233in}{3.290564in}}{\pgfqpoint{3.637046in}{3.298378in}}%
\pgfpathcurveto{\pgfqpoint{3.644860in}{3.306191in}}{\pgfqpoint{3.649250in}{3.316790in}}{\pgfqpoint{3.649250in}{3.327840in}}%
\pgfpathcurveto{\pgfqpoint{3.649250in}{3.338891in}}{\pgfqpoint{3.644860in}{3.349490in}}{\pgfqpoint{3.637046in}{3.357303in}}%
\pgfpathcurveto{\pgfqpoint{3.629233in}{3.365117in}}{\pgfqpoint{3.618634in}{3.369507in}}{\pgfqpoint{3.607583in}{3.369507in}}%
\pgfpathcurveto{\pgfqpoint{3.596533in}{3.369507in}}{\pgfqpoint{3.585934in}{3.365117in}}{\pgfqpoint{3.578121in}{3.357303in}}%
\pgfpathcurveto{\pgfqpoint{3.570307in}{3.349490in}}{\pgfqpoint{3.565917in}{3.338891in}}{\pgfqpoint{3.565917in}{3.327840in}}%
\pgfpathcurveto{\pgfqpoint{3.565917in}{3.316790in}}{\pgfqpoint{3.570307in}{3.306191in}}{\pgfqpoint{3.578121in}{3.298378in}}%
\pgfpathcurveto{\pgfqpoint{3.585934in}{3.290564in}}{\pgfqpoint{3.596533in}{3.286174in}}{\pgfqpoint{3.607583in}{3.286174in}}%
\pgfpathclose%
\pgfusepath{stroke,fill}%
\end{pgfscope}%
\begin{pgfscope}%
\pgfpathrectangle{\pgfqpoint{0.481978in}{0.331635in}}{\pgfqpoint{4.960000in}{3.696000in}}%
\pgfusepath{clip}%
\pgfsetbuttcap%
\pgfsetroundjoin%
\definecolor{currentfill}{rgb}{0.631373,0.788235,0.956863}%
\pgfsetfillcolor{currentfill}%
\pgfsetlinewidth{0.481800pt}%
\definecolor{currentstroke}{rgb}{1.000000,1.000000,1.000000}%
\pgfsetstrokecolor{currentstroke}%
\pgfsetdash{}{0pt}%
\pgfpathmoveto{\pgfqpoint{3.004506in}{1.898579in}}%
\pgfpathcurveto{\pgfqpoint{3.015556in}{1.898579in}}{\pgfqpoint{3.026156in}{1.902969in}}{\pgfqpoint{3.033969in}{1.910783in}}%
\pgfpathcurveto{\pgfqpoint{3.041783in}{1.918596in}}{\pgfqpoint{3.046173in}{1.929195in}}{\pgfqpoint{3.046173in}{1.940246in}}%
\pgfpathcurveto{\pgfqpoint{3.046173in}{1.951296in}}{\pgfqpoint{3.041783in}{1.961895in}}{\pgfqpoint{3.033969in}{1.969708in}}%
\pgfpathcurveto{\pgfqpoint{3.026156in}{1.977522in}}{\pgfqpoint{3.015556in}{1.981912in}}{\pgfqpoint{3.004506in}{1.981912in}}%
\pgfpathcurveto{\pgfqpoint{2.993456in}{1.981912in}}{\pgfqpoint{2.982857in}{1.977522in}}{\pgfqpoint{2.975044in}{1.969708in}}%
\pgfpathcurveto{\pgfqpoint{2.967230in}{1.961895in}}{\pgfqpoint{2.962840in}{1.951296in}}{\pgfqpoint{2.962840in}{1.940246in}}%
\pgfpathcurveto{\pgfqpoint{2.962840in}{1.929195in}}{\pgfqpoint{2.967230in}{1.918596in}}{\pgfqpoint{2.975044in}{1.910783in}}%
\pgfpathcurveto{\pgfqpoint{2.982857in}{1.902969in}}{\pgfqpoint{2.993456in}{1.898579in}}{\pgfqpoint{3.004506in}{1.898579in}}%
\pgfpathclose%
\pgfusepath{stroke,fill}%
\end{pgfscope}%
\begin{pgfscope}%
\pgfpathrectangle{\pgfqpoint{0.481978in}{0.331635in}}{\pgfqpoint{4.960000in}{3.696000in}}%
\pgfusepath{clip}%
\pgfsetbuttcap%
\pgfsetroundjoin%
\definecolor{currentfill}{rgb}{0.631373,0.788235,0.956863}%
\pgfsetfillcolor{currentfill}%
\pgfsetlinewidth{0.481800pt}%
\definecolor{currentstroke}{rgb}{1.000000,1.000000,1.000000}%
\pgfsetstrokecolor{currentstroke}%
\pgfsetdash{}{0pt}%
\pgfpathmoveto{\pgfqpoint{3.007147in}{2.009282in}}%
\pgfpathcurveto{\pgfqpoint{3.018197in}{2.009282in}}{\pgfqpoint{3.028796in}{2.013673in}}{\pgfqpoint{3.036610in}{2.021486in}}%
\pgfpathcurveto{\pgfqpoint{3.044423in}{2.029300in}}{\pgfqpoint{3.048814in}{2.039899in}}{\pgfqpoint{3.048814in}{2.050949in}}%
\pgfpathcurveto{\pgfqpoint{3.048814in}{2.061999in}}{\pgfqpoint{3.044423in}{2.072598in}}{\pgfqpoint{3.036610in}{2.080412in}}%
\pgfpathcurveto{\pgfqpoint{3.028796in}{2.088225in}}{\pgfqpoint{3.018197in}{2.092616in}}{\pgfqpoint{3.007147in}{2.092616in}}%
\pgfpathcurveto{\pgfqpoint{2.996097in}{2.092616in}}{\pgfqpoint{2.985498in}{2.088225in}}{\pgfqpoint{2.977684in}{2.080412in}}%
\pgfpathcurveto{\pgfqpoint{2.969871in}{2.072598in}}{\pgfqpoint{2.965480in}{2.061999in}}{\pgfqpoint{2.965480in}{2.050949in}}%
\pgfpathcurveto{\pgfqpoint{2.965480in}{2.039899in}}{\pgfqpoint{2.969871in}{2.029300in}}{\pgfqpoint{2.977684in}{2.021486in}}%
\pgfpathcurveto{\pgfqpoint{2.985498in}{2.013673in}}{\pgfqpoint{2.996097in}{2.009282in}}{\pgfqpoint{3.007147in}{2.009282in}}%
\pgfpathclose%
\pgfusepath{stroke,fill}%
\end{pgfscope}%
\begin{pgfscope}%
\pgfpathrectangle{\pgfqpoint{0.481978in}{0.331635in}}{\pgfqpoint{4.960000in}{3.696000in}}%
\pgfusepath{clip}%
\pgfsetbuttcap%
\pgfsetroundjoin%
\definecolor{currentfill}{rgb}{0.631373,0.788235,0.956863}%
\pgfsetfillcolor{currentfill}%
\pgfsetlinewidth{0.481800pt}%
\definecolor{currentstroke}{rgb}{1.000000,1.000000,1.000000}%
\pgfsetstrokecolor{currentstroke}%
\pgfsetdash{}{0pt}%
\pgfpathmoveto{\pgfqpoint{3.903183in}{3.526373in}}%
\pgfpathcurveto{\pgfqpoint{3.914233in}{3.526373in}}{\pgfqpoint{3.924832in}{3.530763in}}{\pgfqpoint{3.932646in}{3.538577in}}%
\pgfpathcurveto{\pgfqpoint{3.940459in}{3.546391in}}{\pgfqpoint{3.944850in}{3.556990in}}{\pgfqpoint{3.944850in}{3.568040in}}%
\pgfpathcurveto{\pgfqpoint{3.944850in}{3.579090in}}{\pgfqpoint{3.940459in}{3.589689in}}{\pgfqpoint{3.932646in}{3.597503in}}%
\pgfpathcurveto{\pgfqpoint{3.924832in}{3.605316in}}{\pgfqpoint{3.914233in}{3.609706in}}{\pgfqpoint{3.903183in}{3.609706in}}%
\pgfpathcurveto{\pgfqpoint{3.892133in}{3.609706in}}{\pgfqpoint{3.881534in}{3.605316in}}{\pgfqpoint{3.873720in}{3.597503in}}%
\pgfpathcurveto{\pgfqpoint{3.865906in}{3.589689in}}{\pgfqpoint{3.861516in}{3.579090in}}{\pgfqpoint{3.861516in}{3.568040in}}%
\pgfpathcurveto{\pgfqpoint{3.861516in}{3.556990in}}{\pgfqpoint{3.865906in}{3.546391in}}{\pgfqpoint{3.873720in}{3.538577in}}%
\pgfpathcurveto{\pgfqpoint{3.881534in}{3.530763in}}{\pgfqpoint{3.892133in}{3.526373in}}{\pgfqpoint{3.903183in}{3.526373in}}%
\pgfpathclose%
\pgfusepath{stroke,fill}%
\end{pgfscope}%
\begin{pgfscope}%
\pgfpathrectangle{\pgfqpoint{0.481978in}{0.331635in}}{\pgfqpoint{4.960000in}{3.696000in}}%
\pgfusepath{clip}%
\pgfsetbuttcap%
\pgfsetroundjoin%
\definecolor{currentfill}{rgb}{0.631373,0.788235,0.956863}%
\pgfsetfillcolor{currentfill}%
\pgfsetlinewidth{0.481800pt}%
\definecolor{currentstroke}{rgb}{1.000000,1.000000,1.000000}%
\pgfsetstrokecolor{currentstroke}%
\pgfsetdash{}{0pt}%
\pgfpathmoveto{\pgfqpoint{3.640068in}{2.408446in}}%
\pgfpathcurveto{\pgfqpoint{3.651118in}{2.408446in}}{\pgfqpoint{3.661717in}{2.412837in}}{\pgfqpoint{3.669531in}{2.420650in}}%
\pgfpathcurveto{\pgfqpoint{3.677344in}{2.428464in}}{\pgfqpoint{3.681735in}{2.439063in}}{\pgfqpoint{3.681735in}{2.450113in}}%
\pgfpathcurveto{\pgfqpoint{3.681735in}{2.461163in}}{\pgfqpoint{3.677344in}{2.471762in}}{\pgfqpoint{3.669531in}{2.479576in}}%
\pgfpathcurveto{\pgfqpoint{3.661717in}{2.487389in}}{\pgfqpoint{3.651118in}{2.491780in}}{\pgfqpoint{3.640068in}{2.491780in}}%
\pgfpathcurveto{\pgfqpoint{3.629018in}{2.491780in}}{\pgfqpoint{3.618419in}{2.487389in}}{\pgfqpoint{3.610605in}{2.479576in}}%
\pgfpathcurveto{\pgfqpoint{3.602792in}{2.471762in}}{\pgfqpoint{3.598401in}{2.461163in}}{\pgfqpoint{3.598401in}{2.450113in}}%
\pgfpathcurveto{\pgfqpoint{3.598401in}{2.439063in}}{\pgfqpoint{3.602792in}{2.428464in}}{\pgfqpoint{3.610605in}{2.420650in}}%
\pgfpathcurveto{\pgfqpoint{3.618419in}{2.412837in}}{\pgfqpoint{3.629018in}{2.408446in}}{\pgfqpoint{3.640068in}{2.408446in}}%
\pgfpathclose%
\pgfusepath{stroke,fill}%
\end{pgfscope}%
\begin{pgfscope}%
\pgfpathrectangle{\pgfqpoint{0.481978in}{0.331635in}}{\pgfqpoint{4.960000in}{3.696000in}}%
\pgfusepath{clip}%
\pgfsetbuttcap%
\pgfsetroundjoin%
\definecolor{currentfill}{rgb}{0.631373,0.788235,0.956863}%
\pgfsetfillcolor{currentfill}%
\pgfsetlinewidth{0.481800pt}%
\definecolor{currentstroke}{rgb}{1.000000,1.000000,1.000000}%
\pgfsetstrokecolor{currentstroke}%
\pgfsetdash{}{0pt}%
\pgfpathmoveto{\pgfqpoint{2.787922in}{1.841209in}}%
\pgfpathcurveto{\pgfqpoint{2.798972in}{1.841209in}}{\pgfqpoint{2.809571in}{1.845599in}}{\pgfqpoint{2.817384in}{1.853413in}}%
\pgfpathcurveto{\pgfqpoint{2.825198in}{1.861226in}}{\pgfqpoint{2.829588in}{1.871825in}}{\pgfqpoint{2.829588in}{1.882876in}}%
\pgfpathcurveto{\pgfqpoint{2.829588in}{1.893926in}}{\pgfqpoint{2.825198in}{1.904525in}}{\pgfqpoint{2.817384in}{1.912338in}}%
\pgfpathcurveto{\pgfqpoint{2.809571in}{1.920152in}}{\pgfqpoint{2.798972in}{1.924542in}}{\pgfqpoint{2.787922in}{1.924542in}}%
\pgfpathcurveto{\pgfqpoint{2.776871in}{1.924542in}}{\pgfqpoint{2.766272in}{1.920152in}}{\pgfqpoint{2.758459in}{1.912338in}}%
\pgfpathcurveto{\pgfqpoint{2.750645in}{1.904525in}}{\pgfqpoint{2.746255in}{1.893926in}}{\pgfqpoint{2.746255in}{1.882876in}}%
\pgfpathcurveto{\pgfqpoint{2.746255in}{1.871825in}}{\pgfqpoint{2.750645in}{1.861226in}}{\pgfqpoint{2.758459in}{1.853413in}}%
\pgfpathcurveto{\pgfqpoint{2.766272in}{1.845599in}}{\pgfqpoint{2.776871in}{1.841209in}}{\pgfqpoint{2.787922in}{1.841209in}}%
\pgfpathclose%
\pgfusepath{stroke,fill}%
\end{pgfscope}%
\begin{pgfscope}%
\pgfpathrectangle{\pgfqpoint{0.481978in}{0.331635in}}{\pgfqpoint{4.960000in}{3.696000in}}%
\pgfusepath{clip}%
\pgfsetbuttcap%
\pgfsetroundjoin%
\definecolor{currentfill}{rgb}{0.631373,0.788235,0.956863}%
\pgfsetfillcolor{currentfill}%
\pgfsetlinewidth{0.481800pt}%
\definecolor{currentstroke}{rgb}{1.000000,1.000000,1.000000}%
\pgfsetstrokecolor{currentstroke}%
\pgfsetdash{}{0pt}%
\pgfpathmoveto{\pgfqpoint{3.431218in}{2.774537in}}%
\pgfpathcurveto{\pgfqpoint{3.442268in}{2.774537in}}{\pgfqpoint{3.452867in}{2.778927in}}{\pgfqpoint{3.460681in}{2.786741in}}%
\pgfpathcurveto{\pgfqpoint{3.468494in}{2.794555in}}{\pgfqpoint{3.472885in}{2.805154in}}{\pgfqpoint{3.472885in}{2.816204in}}%
\pgfpathcurveto{\pgfqpoint{3.472885in}{2.827254in}}{\pgfqpoint{3.468494in}{2.837853in}}{\pgfqpoint{3.460681in}{2.845667in}}%
\pgfpathcurveto{\pgfqpoint{3.452867in}{2.853480in}}{\pgfqpoint{3.442268in}{2.857871in}}{\pgfqpoint{3.431218in}{2.857871in}}%
\pgfpathcurveto{\pgfqpoint{3.420168in}{2.857871in}}{\pgfqpoint{3.409569in}{2.853480in}}{\pgfqpoint{3.401755in}{2.845667in}}%
\pgfpathcurveto{\pgfqpoint{3.393941in}{2.837853in}}{\pgfqpoint{3.389551in}{2.827254in}}{\pgfqpoint{3.389551in}{2.816204in}}%
\pgfpathcurveto{\pgfqpoint{3.389551in}{2.805154in}}{\pgfqpoint{3.393941in}{2.794555in}}{\pgfqpoint{3.401755in}{2.786741in}}%
\pgfpathcurveto{\pgfqpoint{3.409569in}{2.778927in}}{\pgfqpoint{3.420168in}{2.774537in}}{\pgfqpoint{3.431218in}{2.774537in}}%
\pgfpathclose%
\pgfusepath{stroke,fill}%
\end{pgfscope}%
\begin{pgfscope}%
\pgfpathrectangle{\pgfqpoint{0.481978in}{0.331635in}}{\pgfqpoint{4.960000in}{3.696000in}}%
\pgfusepath{clip}%
\pgfsetbuttcap%
\pgfsetroundjoin%
\definecolor{currentfill}{rgb}{0.631373,0.788235,0.956863}%
\pgfsetfillcolor{currentfill}%
\pgfsetlinewidth{0.481800pt}%
\definecolor{currentstroke}{rgb}{1.000000,1.000000,1.000000}%
\pgfsetstrokecolor{currentstroke}%
\pgfsetdash{}{0pt}%
\pgfpathmoveto{\pgfqpoint{2.771163in}{2.424041in}}%
\pgfpathcurveto{\pgfqpoint{2.782213in}{2.424041in}}{\pgfqpoint{2.792812in}{2.428432in}}{\pgfqpoint{2.800626in}{2.436245in}}%
\pgfpathcurveto{\pgfqpoint{2.808440in}{2.444059in}}{\pgfqpoint{2.812830in}{2.454658in}}{\pgfqpoint{2.812830in}{2.465708in}}%
\pgfpathcurveto{\pgfqpoint{2.812830in}{2.476758in}}{\pgfqpoint{2.808440in}{2.487357in}}{\pgfqpoint{2.800626in}{2.495171in}}%
\pgfpathcurveto{\pgfqpoint{2.792812in}{2.502984in}}{\pgfqpoint{2.782213in}{2.507375in}}{\pgfqpoint{2.771163in}{2.507375in}}%
\pgfpathcurveto{\pgfqpoint{2.760113in}{2.507375in}}{\pgfqpoint{2.749514in}{2.502984in}}{\pgfqpoint{2.741700in}{2.495171in}}%
\pgfpathcurveto{\pgfqpoint{2.733887in}{2.487357in}}{\pgfqpoint{2.729497in}{2.476758in}}{\pgfqpoint{2.729497in}{2.465708in}}%
\pgfpathcurveto{\pgfqpoint{2.729497in}{2.454658in}}{\pgfqpoint{2.733887in}{2.444059in}}{\pgfqpoint{2.741700in}{2.436245in}}%
\pgfpathcurveto{\pgfqpoint{2.749514in}{2.428432in}}{\pgfqpoint{2.760113in}{2.424041in}}{\pgfqpoint{2.771163in}{2.424041in}}%
\pgfpathclose%
\pgfusepath{stroke,fill}%
\end{pgfscope}%
\begin{pgfscope}%
\pgfpathrectangle{\pgfqpoint{0.481978in}{0.331635in}}{\pgfqpoint{4.960000in}{3.696000in}}%
\pgfusepath{clip}%
\pgfsetbuttcap%
\pgfsetroundjoin%
\definecolor{currentfill}{rgb}{0.631373,0.788235,0.956863}%
\pgfsetfillcolor{currentfill}%
\pgfsetlinewidth{0.481800pt}%
\definecolor{currentstroke}{rgb}{1.000000,1.000000,1.000000}%
\pgfsetstrokecolor{currentstroke}%
\pgfsetdash{}{0pt}%
\pgfpathmoveto{\pgfqpoint{3.272656in}{3.414119in}}%
\pgfpathcurveto{\pgfqpoint{3.283706in}{3.414119in}}{\pgfqpoint{3.294305in}{3.418509in}}{\pgfqpoint{3.302119in}{3.426323in}}%
\pgfpathcurveto{\pgfqpoint{3.309932in}{3.434136in}}{\pgfqpoint{3.314323in}{3.444735in}}{\pgfqpoint{3.314323in}{3.455785in}}%
\pgfpathcurveto{\pgfqpoint{3.314323in}{3.466836in}}{\pgfqpoint{3.309932in}{3.477435in}}{\pgfqpoint{3.302119in}{3.485248in}}%
\pgfpathcurveto{\pgfqpoint{3.294305in}{3.493062in}}{\pgfqpoint{3.283706in}{3.497452in}}{\pgfqpoint{3.272656in}{3.497452in}}%
\pgfpathcurveto{\pgfqpoint{3.261606in}{3.497452in}}{\pgfqpoint{3.251007in}{3.493062in}}{\pgfqpoint{3.243193in}{3.485248in}}%
\pgfpathcurveto{\pgfqpoint{3.235380in}{3.477435in}}{\pgfqpoint{3.230989in}{3.466836in}}{\pgfqpoint{3.230989in}{3.455785in}}%
\pgfpathcurveto{\pgfqpoint{3.230989in}{3.444735in}}{\pgfqpoint{3.235380in}{3.434136in}}{\pgfqpoint{3.243193in}{3.426323in}}%
\pgfpathcurveto{\pgfqpoint{3.251007in}{3.418509in}}{\pgfqpoint{3.261606in}{3.414119in}}{\pgfqpoint{3.272656in}{3.414119in}}%
\pgfpathclose%
\pgfusepath{stroke,fill}%
\end{pgfscope}%
\begin{pgfscope}%
\pgfpathrectangle{\pgfqpoint{0.481978in}{0.331635in}}{\pgfqpoint{4.960000in}{3.696000in}}%
\pgfusepath{clip}%
\pgfsetbuttcap%
\pgfsetroundjoin%
\definecolor{currentfill}{rgb}{0.631373,0.788235,0.956863}%
\pgfsetfillcolor{currentfill}%
\pgfsetlinewidth{0.481800pt}%
\definecolor{currentstroke}{rgb}{1.000000,1.000000,1.000000}%
\pgfsetstrokecolor{currentstroke}%
\pgfsetdash{}{0pt}%
\pgfpathmoveto{\pgfqpoint{2.358876in}{2.719293in}}%
\pgfpathcurveto{\pgfqpoint{2.369926in}{2.719293in}}{\pgfqpoint{2.380525in}{2.723684in}}{\pgfqpoint{2.388339in}{2.731497in}}%
\pgfpathcurveto{\pgfqpoint{2.396153in}{2.739311in}}{\pgfqpoint{2.400543in}{2.749910in}}{\pgfqpoint{2.400543in}{2.760960in}}%
\pgfpathcurveto{\pgfqpoint{2.400543in}{2.772010in}}{\pgfqpoint{2.396153in}{2.782609in}}{\pgfqpoint{2.388339in}{2.790423in}}%
\pgfpathcurveto{\pgfqpoint{2.380525in}{2.798236in}}{\pgfqpoint{2.369926in}{2.802627in}}{\pgfqpoint{2.358876in}{2.802627in}}%
\pgfpathcurveto{\pgfqpoint{2.347826in}{2.802627in}}{\pgfqpoint{2.337227in}{2.798236in}}{\pgfqpoint{2.329414in}{2.790423in}}%
\pgfpathcurveto{\pgfqpoint{2.321600in}{2.782609in}}{\pgfqpoint{2.317210in}{2.772010in}}{\pgfqpoint{2.317210in}{2.760960in}}%
\pgfpathcurveto{\pgfqpoint{2.317210in}{2.749910in}}{\pgfqpoint{2.321600in}{2.739311in}}{\pgfqpoint{2.329414in}{2.731497in}}%
\pgfpathcurveto{\pgfqpoint{2.337227in}{2.723684in}}{\pgfqpoint{2.347826in}{2.719293in}}{\pgfqpoint{2.358876in}{2.719293in}}%
\pgfpathclose%
\pgfusepath{stroke,fill}%
\end{pgfscope}%
\begin{pgfscope}%
\pgfpathrectangle{\pgfqpoint{0.481978in}{0.331635in}}{\pgfqpoint{4.960000in}{3.696000in}}%
\pgfusepath{clip}%
\pgfsetbuttcap%
\pgfsetroundjoin%
\definecolor{currentfill}{rgb}{0.631373,0.788235,0.956863}%
\pgfsetfillcolor{currentfill}%
\pgfsetlinewidth{0.481800pt}%
\definecolor{currentstroke}{rgb}{1.000000,1.000000,1.000000}%
\pgfsetstrokecolor{currentstroke}%
\pgfsetdash{}{0pt}%
\pgfpathmoveto{\pgfqpoint{2.499804in}{2.236777in}}%
\pgfpathcurveto{\pgfqpoint{2.510854in}{2.236777in}}{\pgfqpoint{2.521453in}{2.241167in}}{\pgfqpoint{2.529267in}{2.248981in}}%
\pgfpathcurveto{\pgfqpoint{2.537080in}{2.256794in}}{\pgfqpoint{2.541471in}{2.267394in}}{\pgfqpoint{2.541471in}{2.278444in}}%
\pgfpathcurveto{\pgfqpoint{2.541471in}{2.289494in}}{\pgfqpoint{2.537080in}{2.300093in}}{\pgfqpoint{2.529267in}{2.307906in}}%
\pgfpathcurveto{\pgfqpoint{2.521453in}{2.315720in}}{\pgfqpoint{2.510854in}{2.320110in}}{\pgfqpoint{2.499804in}{2.320110in}}%
\pgfpathcurveto{\pgfqpoint{2.488754in}{2.320110in}}{\pgfqpoint{2.478155in}{2.315720in}}{\pgfqpoint{2.470341in}{2.307906in}}%
\pgfpathcurveto{\pgfqpoint{2.462527in}{2.300093in}}{\pgfqpoint{2.458137in}{2.289494in}}{\pgfqpoint{2.458137in}{2.278444in}}%
\pgfpathcurveto{\pgfqpoint{2.458137in}{2.267394in}}{\pgfqpoint{2.462527in}{2.256794in}}{\pgfqpoint{2.470341in}{2.248981in}}%
\pgfpathcurveto{\pgfqpoint{2.478155in}{2.241167in}}{\pgfqpoint{2.488754in}{2.236777in}}{\pgfqpoint{2.499804in}{2.236777in}}%
\pgfpathclose%
\pgfusepath{stroke,fill}%
\end{pgfscope}%
\begin{pgfscope}%
\pgfpathrectangle{\pgfqpoint{0.481978in}{0.331635in}}{\pgfqpoint{4.960000in}{3.696000in}}%
\pgfusepath{clip}%
\pgfsetbuttcap%
\pgfsetroundjoin%
\definecolor{currentfill}{rgb}{0.631373,0.788235,0.956863}%
\pgfsetfillcolor{currentfill}%
\pgfsetlinewidth{0.481800pt}%
\definecolor{currentstroke}{rgb}{1.000000,1.000000,1.000000}%
\pgfsetstrokecolor{currentstroke}%
\pgfsetdash{}{0pt}%
\pgfpathmoveto{\pgfqpoint{2.497801in}{3.035043in}}%
\pgfpathcurveto{\pgfqpoint{2.508851in}{3.035043in}}{\pgfqpoint{2.519450in}{3.039433in}}{\pgfqpoint{2.527264in}{3.047247in}}%
\pgfpathcurveto{\pgfqpoint{2.535077in}{3.055060in}}{\pgfqpoint{2.539468in}{3.065659in}}{\pgfqpoint{2.539468in}{3.076709in}}%
\pgfpathcurveto{\pgfqpoint{2.539468in}{3.087759in}}{\pgfqpoint{2.535077in}{3.098358in}}{\pgfqpoint{2.527264in}{3.106172in}}%
\pgfpathcurveto{\pgfqpoint{2.519450in}{3.113986in}}{\pgfqpoint{2.508851in}{3.118376in}}{\pgfqpoint{2.497801in}{3.118376in}}%
\pgfpathcurveto{\pgfqpoint{2.486751in}{3.118376in}}{\pgfqpoint{2.476152in}{3.113986in}}{\pgfqpoint{2.468338in}{3.106172in}}%
\pgfpathcurveto{\pgfqpoint{2.460525in}{3.098358in}}{\pgfqpoint{2.456134in}{3.087759in}}{\pgfqpoint{2.456134in}{3.076709in}}%
\pgfpathcurveto{\pgfqpoint{2.456134in}{3.065659in}}{\pgfqpoint{2.460525in}{3.055060in}}{\pgfqpoint{2.468338in}{3.047247in}}%
\pgfpathcurveto{\pgfqpoint{2.476152in}{3.039433in}}{\pgfqpoint{2.486751in}{3.035043in}}{\pgfqpoint{2.497801in}{3.035043in}}%
\pgfpathclose%
\pgfusepath{stroke,fill}%
\end{pgfscope}%
\begin{pgfscope}%
\pgfpathrectangle{\pgfqpoint{0.481978in}{0.331635in}}{\pgfqpoint{4.960000in}{3.696000in}}%
\pgfusepath{clip}%
\pgfsetbuttcap%
\pgfsetroundjoin%
\definecolor{currentfill}{rgb}{0.631373,0.788235,0.956863}%
\pgfsetfillcolor{currentfill}%
\pgfsetlinewidth{0.481800pt}%
\definecolor{currentstroke}{rgb}{1.000000,1.000000,1.000000}%
\pgfsetstrokecolor{currentstroke}%
\pgfsetdash{}{0pt}%
\pgfpathmoveto{\pgfqpoint{2.841725in}{1.785964in}}%
\pgfpathcurveto{\pgfqpoint{2.852775in}{1.785964in}}{\pgfqpoint{2.863374in}{1.790354in}}{\pgfqpoint{2.871188in}{1.798168in}}%
\pgfpathcurveto{\pgfqpoint{2.879001in}{1.805981in}}{\pgfqpoint{2.883391in}{1.816580in}}{\pgfqpoint{2.883391in}{1.827631in}}%
\pgfpathcurveto{\pgfqpoint{2.883391in}{1.838681in}}{\pgfqpoint{2.879001in}{1.849280in}}{\pgfqpoint{2.871188in}{1.857093in}}%
\pgfpathcurveto{\pgfqpoint{2.863374in}{1.864907in}}{\pgfqpoint{2.852775in}{1.869297in}}{\pgfqpoint{2.841725in}{1.869297in}}%
\pgfpathcurveto{\pgfqpoint{2.830675in}{1.869297in}}{\pgfqpoint{2.820076in}{1.864907in}}{\pgfqpoint{2.812262in}{1.857093in}}%
\pgfpathcurveto{\pgfqpoint{2.804448in}{1.849280in}}{\pgfqpoint{2.800058in}{1.838681in}}{\pgfqpoint{2.800058in}{1.827631in}}%
\pgfpathcurveto{\pgfqpoint{2.800058in}{1.816580in}}{\pgfqpoint{2.804448in}{1.805981in}}{\pgfqpoint{2.812262in}{1.798168in}}%
\pgfpathcurveto{\pgfqpoint{2.820076in}{1.790354in}}{\pgfqpoint{2.830675in}{1.785964in}}{\pgfqpoint{2.841725in}{1.785964in}}%
\pgfpathclose%
\pgfusepath{stroke,fill}%
\end{pgfscope}%
\begin{pgfscope}%
\pgfpathrectangle{\pgfqpoint{0.481978in}{0.331635in}}{\pgfqpoint{4.960000in}{3.696000in}}%
\pgfusepath{clip}%
\pgfsetbuttcap%
\pgfsetroundjoin%
\definecolor{currentfill}{rgb}{0.631373,0.788235,0.956863}%
\pgfsetfillcolor{currentfill}%
\pgfsetlinewidth{0.481800pt}%
\definecolor{currentstroke}{rgb}{1.000000,1.000000,1.000000}%
\pgfsetstrokecolor{currentstroke}%
\pgfsetdash{}{0pt}%
\pgfpathmoveto{\pgfqpoint{4.583715in}{3.393414in}}%
\pgfpathcurveto{\pgfqpoint{4.594765in}{3.393414in}}{\pgfqpoint{4.605364in}{3.397805in}}{\pgfqpoint{4.613178in}{3.405618in}}%
\pgfpathcurveto{\pgfqpoint{4.620992in}{3.413432in}}{\pgfqpoint{4.625382in}{3.424031in}}{\pgfqpoint{4.625382in}{3.435081in}}%
\pgfpathcurveto{\pgfqpoint{4.625382in}{3.446131in}}{\pgfqpoint{4.620992in}{3.456730in}}{\pgfqpoint{4.613178in}{3.464544in}}%
\pgfpathcurveto{\pgfqpoint{4.605364in}{3.472357in}}{\pgfqpoint{4.594765in}{3.476748in}}{\pgfqpoint{4.583715in}{3.476748in}}%
\pgfpathcurveto{\pgfqpoint{4.572665in}{3.476748in}}{\pgfqpoint{4.562066in}{3.472357in}}{\pgfqpoint{4.554252in}{3.464544in}}%
\pgfpathcurveto{\pgfqpoint{4.546439in}{3.456730in}}{\pgfqpoint{4.542048in}{3.446131in}}{\pgfqpoint{4.542048in}{3.435081in}}%
\pgfpathcurveto{\pgfqpoint{4.542048in}{3.424031in}}{\pgfqpoint{4.546439in}{3.413432in}}{\pgfqpoint{4.554252in}{3.405618in}}%
\pgfpathcurveto{\pgfqpoint{4.562066in}{3.397805in}}{\pgfqpoint{4.572665in}{3.393414in}}{\pgfqpoint{4.583715in}{3.393414in}}%
\pgfpathclose%
\pgfusepath{stroke,fill}%
\end{pgfscope}%
\begin{pgfscope}%
\pgfpathrectangle{\pgfqpoint{0.481978in}{0.331635in}}{\pgfqpoint{4.960000in}{3.696000in}}%
\pgfusepath{clip}%
\pgfsetbuttcap%
\pgfsetroundjoin%
\definecolor{currentfill}{rgb}{0.631373,0.788235,0.956863}%
\pgfsetfillcolor{currentfill}%
\pgfsetlinewidth{0.481800pt}%
\definecolor{currentstroke}{rgb}{1.000000,1.000000,1.000000}%
\pgfsetstrokecolor{currentstroke}%
\pgfsetdash{}{0pt}%
\pgfpathmoveto{\pgfqpoint{4.090488in}{3.298309in}}%
\pgfpathcurveto{\pgfqpoint{4.101538in}{3.298309in}}{\pgfqpoint{4.112137in}{3.302699in}}{\pgfqpoint{4.119951in}{3.310513in}}%
\pgfpathcurveto{\pgfqpoint{4.127765in}{3.318326in}}{\pgfqpoint{4.132155in}{3.328925in}}{\pgfqpoint{4.132155in}{3.339975in}}%
\pgfpathcurveto{\pgfqpoint{4.132155in}{3.351026in}}{\pgfqpoint{4.127765in}{3.361625in}}{\pgfqpoint{4.119951in}{3.369438in}}%
\pgfpathcurveto{\pgfqpoint{4.112137in}{3.377252in}}{\pgfqpoint{4.101538in}{3.381642in}}{\pgfqpoint{4.090488in}{3.381642in}}%
\pgfpathcurveto{\pgfqpoint{4.079438in}{3.381642in}}{\pgfqpoint{4.068839in}{3.377252in}}{\pgfqpoint{4.061025in}{3.369438in}}%
\pgfpathcurveto{\pgfqpoint{4.053212in}{3.361625in}}{\pgfqpoint{4.048822in}{3.351026in}}{\pgfqpoint{4.048822in}{3.339975in}}%
\pgfpathcurveto{\pgfqpoint{4.048822in}{3.328925in}}{\pgfqpoint{4.053212in}{3.318326in}}{\pgfqpoint{4.061025in}{3.310513in}}%
\pgfpathcurveto{\pgfqpoint{4.068839in}{3.302699in}}{\pgfqpoint{4.079438in}{3.298309in}}{\pgfqpoint{4.090488in}{3.298309in}}%
\pgfpathclose%
\pgfusepath{stroke,fill}%
\end{pgfscope}%
\begin{pgfscope}%
\pgfpathrectangle{\pgfqpoint{0.481978in}{0.331635in}}{\pgfqpoint{4.960000in}{3.696000in}}%
\pgfusepath{clip}%
\pgfsetbuttcap%
\pgfsetroundjoin%
\definecolor{currentfill}{rgb}{0.631373,0.788235,0.956863}%
\pgfsetfillcolor{currentfill}%
\pgfsetlinewidth{0.481800pt}%
\definecolor{currentstroke}{rgb}{1.000000,1.000000,1.000000}%
\pgfsetstrokecolor{currentstroke}%
\pgfsetdash{}{0pt}%
\pgfpathmoveto{\pgfqpoint{3.754263in}{2.387869in}}%
\pgfpathcurveto{\pgfqpoint{3.765313in}{2.387869in}}{\pgfqpoint{3.775912in}{2.392259in}}{\pgfqpoint{3.783726in}{2.400073in}}%
\pgfpathcurveto{\pgfqpoint{3.791539in}{2.407887in}}{\pgfqpoint{3.795930in}{2.418486in}}{\pgfqpoint{3.795930in}{2.429536in}}%
\pgfpathcurveto{\pgfqpoint{3.795930in}{2.440586in}}{\pgfqpoint{3.791539in}{2.451185in}}{\pgfqpoint{3.783726in}{2.458999in}}%
\pgfpathcurveto{\pgfqpoint{3.775912in}{2.466812in}}{\pgfqpoint{3.765313in}{2.471202in}}{\pgfqpoint{3.754263in}{2.471202in}}%
\pgfpathcurveto{\pgfqpoint{3.743213in}{2.471202in}}{\pgfqpoint{3.732614in}{2.466812in}}{\pgfqpoint{3.724800in}{2.458999in}}%
\pgfpathcurveto{\pgfqpoint{3.716986in}{2.451185in}}{\pgfqpoint{3.712596in}{2.440586in}}{\pgfqpoint{3.712596in}{2.429536in}}%
\pgfpathcurveto{\pgfqpoint{3.712596in}{2.418486in}}{\pgfqpoint{3.716986in}{2.407887in}}{\pgfqpoint{3.724800in}{2.400073in}}%
\pgfpathcurveto{\pgfqpoint{3.732614in}{2.392259in}}{\pgfqpoint{3.743213in}{2.387869in}}{\pgfqpoint{3.754263in}{2.387869in}}%
\pgfpathclose%
\pgfusepath{stroke,fill}%
\end{pgfscope}%
\begin{pgfscope}%
\pgfpathrectangle{\pgfqpoint{0.481978in}{0.331635in}}{\pgfqpoint{4.960000in}{3.696000in}}%
\pgfusepath{clip}%
\pgfsetbuttcap%
\pgfsetroundjoin%
\definecolor{currentfill}{rgb}{0.631373,0.788235,0.956863}%
\pgfsetfillcolor{currentfill}%
\pgfsetlinewidth{0.481800pt}%
\definecolor{currentstroke}{rgb}{1.000000,1.000000,1.000000}%
\pgfsetstrokecolor{currentstroke}%
\pgfsetdash{}{0pt}%
\pgfpathmoveto{\pgfqpoint{3.020955in}{2.272757in}}%
\pgfpathcurveto{\pgfqpoint{3.032005in}{2.272757in}}{\pgfqpoint{3.042604in}{2.277147in}}{\pgfqpoint{3.050418in}{2.284961in}}%
\pgfpathcurveto{\pgfqpoint{3.058232in}{2.292774in}}{\pgfqpoint{3.062622in}{2.303373in}}{\pgfqpoint{3.062622in}{2.314424in}}%
\pgfpathcurveto{\pgfqpoint{3.062622in}{2.325474in}}{\pgfqpoint{3.058232in}{2.336073in}}{\pgfqpoint{3.050418in}{2.343886in}}%
\pgfpathcurveto{\pgfqpoint{3.042604in}{2.351700in}}{\pgfqpoint{3.032005in}{2.356090in}}{\pgfqpoint{3.020955in}{2.356090in}}%
\pgfpathcurveto{\pgfqpoint{3.009905in}{2.356090in}}{\pgfqpoint{2.999306in}{2.351700in}}{\pgfqpoint{2.991493in}{2.343886in}}%
\pgfpathcurveto{\pgfqpoint{2.983679in}{2.336073in}}{\pgfqpoint{2.979289in}{2.325474in}}{\pgfqpoint{2.979289in}{2.314424in}}%
\pgfpathcurveto{\pgfqpoint{2.979289in}{2.303373in}}{\pgfqpoint{2.983679in}{2.292774in}}{\pgfqpoint{2.991493in}{2.284961in}}%
\pgfpathcurveto{\pgfqpoint{2.999306in}{2.277147in}}{\pgfqpoint{3.009905in}{2.272757in}}{\pgfqpoint{3.020955in}{2.272757in}}%
\pgfpathclose%
\pgfusepath{stroke,fill}%
\end{pgfscope}%
\begin{pgfscope}%
\pgfpathrectangle{\pgfqpoint{0.481978in}{0.331635in}}{\pgfqpoint{4.960000in}{3.696000in}}%
\pgfusepath{clip}%
\pgfsetbuttcap%
\pgfsetroundjoin%
\definecolor{currentfill}{rgb}{0.631373,0.788235,0.956863}%
\pgfsetfillcolor{currentfill}%
\pgfsetlinewidth{0.481800pt}%
\definecolor{currentstroke}{rgb}{1.000000,1.000000,1.000000}%
\pgfsetstrokecolor{currentstroke}%
\pgfsetdash{}{0pt}%
\pgfpathmoveto{\pgfqpoint{3.823159in}{1.011936in}}%
\pgfpathcurveto{\pgfqpoint{3.834209in}{1.011936in}}{\pgfqpoint{3.844808in}{1.016326in}}{\pgfqpoint{3.852622in}{1.024140in}}%
\pgfpathcurveto{\pgfqpoint{3.860435in}{1.031953in}}{\pgfqpoint{3.864826in}{1.042552in}}{\pgfqpoint{3.864826in}{1.053603in}}%
\pgfpathcurveto{\pgfqpoint{3.864826in}{1.064653in}}{\pgfqpoint{3.860435in}{1.075252in}}{\pgfqpoint{3.852622in}{1.083065in}}%
\pgfpathcurveto{\pgfqpoint{3.844808in}{1.090879in}}{\pgfqpoint{3.834209in}{1.095269in}}{\pgfqpoint{3.823159in}{1.095269in}}%
\pgfpathcurveto{\pgfqpoint{3.812109in}{1.095269in}}{\pgfqpoint{3.801510in}{1.090879in}}{\pgfqpoint{3.793696in}{1.083065in}}%
\pgfpathcurveto{\pgfqpoint{3.785883in}{1.075252in}}{\pgfqpoint{3.781492in}{1.064653in}}{\pgfqpoint{3.781492in}{1.053603in}}%
\pgfpathcurveto{\pgfqpoint{3.781492in}{1.042552in}}{\pgfqpoint{3.785883in}{1.031953in}}{\pgfqpoint{3.793696in}{1.024140in}}%
\pgfpathcurveto{\pgfqpoint{3.801510in}{1.016326in}}{\pgfqpoint{3.812109in}{1.011936in}}{\pgfqpoint{3.823159in}{1.011936in}}%
\pgfpathclose%
\pgfusepath{stroke,fill}%
\end{pgfscope}%
\begin{pgfscope}%
\pgfpathrectangle{\pgfqpoint{0.481978in}{0.331635in}}{\pgfqpoint{4.960000in}{3.696000in}}%
\pgfusepath{clip}%
\pgfsetbuttcap%
\pgfsetroundjoin%
\definecolor{currentfill}{rgb}{0.631373,0.788235,0.956863}%
\pgfsetfillcolor{currentfill}%
\pgfsetlinewidth{0.481800pt}%
\definecolor{currentstroke}{rgb}{1.000000,1.000000,1.000000}%
\pgfsetstrokecolor{currentstroke}%
\pgfsetdash{}{0pt}%
\pgfpathmoveto{\pgfqpoint{3.618546in}{1.932239in}}%
\pgfpathcurveto{\pgfqpoint{3.629596in}{1.932239in}}{\pgfqpoint{3.640195in}{1.936629in}}{\pgfqpoint{3.648008in}{1.944443in}}%
\pgfpathcurveto{\pgfqpoint{3.655822in}{1.952256in}}{\pgfqpoint{3.660212in}{1.962855in}}{\pgfqpoint{3.660212in}{1.973905in}}%
\pgfpathcurveto{\pgfqpoint{3.660212in}{1.984956in}}{\pgfqpoint{3.655822in}{1.995555in}}{\pgfqpoint{3.648008in}{2.003368in}}%
\pgfpathcurveto{\pgfqpoint{3.640195in}{2.011182in}}{\pgfqpoint{3.629596in}{2.015572in}}{\pgfqpoint{3.618546in}{2.015572in}}%
\pgfpathcurveto{\pgfqpoint{3.607496in}{2.015572in}}{\pgfqpoint{3.596896in}{2.011182in}}{\pgfqpoint{3.589083in}{2.003368in}}%
\pgfpathcurveto{\pgfqpoint{3.581269in}{1.995555in}}{\pgfqpoint{3.576879in}{1.984956in}}{\pgfqpoint{3.576879in}{1.973905in}}%
\pgfpathcurveto{\pgfqpoint{3.576879in}{1.962855in}}{\pgfqpoint{3.581269in}{1.952256in}}{\pgfqpoint{3.589083in}{1.944443in}}%
\pgfpathcurveto{\pgfqpoint{3.596896in}{1.936629in}}{\pgfqpoint{3.607496in}{1.932239in}}{\pgfqpoint{3.618546in}{1.932239in}}%
\pgfpathclose%
\pgfusepath{stroke,fill}%
\end{pgfscope}%
\begin{pgfscope}%
\pgfpathrectangle{\pgfqpoint{0.481978in}{0.331635in}}{\pgfqpoint{4.960000in}{3.696000in}}%
\pgfusepath{clip}%
\pgfsetbuttcap%
\pgfsetroundjoin%
\definecolor{currentfill}{rgb}{0.631373,0.788235,0.956863}%
\pgfsetfillcolor{currentfill}%
\pgfsetlinewidth{0.481800pt}%
\definecolor{currentstroke}{rgb}{1.000000,1.000000,1.000000}%
\pgfsetstrokecolor{currentstroke}%
\pgfsetdash{}{0pt}%
\pgfpathmoveto{\pgfqpoint{2.396516in}{1.850907in}}%
\pgfpathcurveto{\pgfqpoint{2.407566in}{1.850907in}}{\pgfqpoint{2.418165in}{1.855298in}}{\pgfqpoint{2.425979in}{1.863111in}}%
\pgfpathcurveto{\pgfqpoint{2.433793in}{1.870925in}}{\pgfqpoint{2.438183in}{1.881524in}}{\pgfqpoint{2.438183in}{1.892574in}}%
\pgfpathcurveto{\pgfqpoint{2.438183in}{1.903624in}}{\pgfqpoint{2.433793in}{1.914223in}}{\pgfqpoint{2.425979in}{1.922037in}}%
\pgfpathcurveto{\pgfqpoint{2.418165in}{1.929850in}}{\pgfqpoint{2.407566in}{1.934241in}}{\pgfqpoint{2.396516in}{1.934241in}}%
\pgfpathcurveto{\pgfqpoint{2.385466in}{1.934241in}}{\pgfqpoint{2.374867in}{1.929850in}}{\pgfqpoint{2.367053in}{1.922037in}}%
\pgfpathcurveto{\pgfqpoint{2.359240in}{1.914223in}}{\pgfqpoint{2.354849in}{1.903624in}}{\pgfqpoint{2.354849in}{1.892574in}}%
\pgfpathcurveto{\pgfqpoint{2.354849in}{1.881524in}}{\pgfqpoint{2.359240in}{1.870925in}}{\pgfqpoint{2.367053in}{1.863111in}}%
\pgfpathcurveto{\pgfqpoint{2.374867in}{1.855298in}}{\pgfqpoint{2.385466in}{1.850907in}}{\pgfqpoint{2.396516in}{1.850907in}}%
\pgfpathclose%
\pgfusepath{stroke,fill}%
\end{pgfscope}%
\begin{pgfscope}%
\pgfpathrectangle{\pgfqpoint{0.481978in}{0.331635in}}{\pgfqpoint{4.960000in}{3.696000in}}%
\pgfusepath{clip}%
\pgfsetbuttcap%
\pgfsetroundjoin%
\definecolor{currentfill}{rgb}{0.631373,0.788235,0.956863}%
\pgfsetfillcolor{currentfill}%
\pgfsetlinewidth{0.481800pt}%
\definecolor{currentstroke}{rgb}{1.000000,1.000000,1.000000}%
\pgfsetstrokecolor{currentstroke}%
\pgfsetdash{}{0pt}%
\pgfpathmoveto{\pgfqpoint{2.127524in}{2.946523in}}%
\pgfpathcurveto{\pgfqpoint{2.138574in}{2.946523in}}{\pgfqpoint{2.149173in}{2.950913in}}{\pgfqpoint{2.156986in}{2.958727in}}%
\pgfpathcurveto{\pgfqpoint{2.164800in}{2.966540in}}{\pgfqpoint{2.169190in}{2.977139in}}{\pgfqpoint{2.169190in}{2.988190in}}%
\pgfpathcurveto{\pgfqpoint{2.169190in}{2.999240in}}{\pgfqpoint{2.164800in}{3.009839in}}{\pgfqpoint{2.156986in}{3.017652in}}%
\pgfpathcurveto{\pgfqpoint{2.149173in}{3.025466in}}{\pgfqpoint{2.138574in}{3.029856in}}{\pgfqpoint{2.127524in}{3.029856in}}%
\pgfpathcurveto{\pgfqpoint{2.116474in}{3.029856in}}{\pgfqpoint{2.105875in}{3.025466in}}{\pgfqpoint{2.098061in}{3.017652in}}%
\pgfpathcurveto{\pgfqpoint{2.090247in}{3.009839in}}{\pgfqpoint{2.085857in}{2.999240in}}{\pgfqpoint{2.085857in}{2.988190in}}%
\pgfpathcurveto{\pgfqpoint{2.085857in}{2.977139in}}{\pgfqpoint{2.090247in}{2.966540in}}{\pgfqpoint{2.098061in}{2.958727in}}%
\pgfpathcurveto{\pgfqpoint{2.105875in}{2.950913in}}{\pgfqpoint{2.116474in}{2.946523in}}{\pgfqpoint{2.127524in}{2.946523in}}%
\pgfpathclose%
\pgfusepath{stroke,fill}%
\end{pgfscope}%
\begin{pgfscope}%
\pgfpathrectangle{\pgfqpoint{0.481978in}{0.331635in}}{\pgfqpoint{4.960000in}{3.696000in}}%
\pgfusepath{clip}%
\pgfsetbuttcap%
\pgfsetroundjoin%
\definecolor{currentfill}{rgb}{0.631373,0.788235,0.956863}%
\pgfsetfillcolor{currentfill}%
\pgfsetlinewidth{0.481800pt}%
\definecolor{currentstroke}{rgb}{1.000000,1.000000,1.000000}%
\pgfsetstrokecolor{currentstroke}%
\pgfsetdash{}{0pt}%
\pgfpathmoveto{\pgfqpoint{2.834203in}{1.998929in}}%
\pgfpathcurveto{\pgfqpoint{2.845253in}{1.998929in}}{\pgfqpoint{2.855852in}{2.003319in}}{\pgfqpoint{2.863666in}{2.011133in}}%
\pgfpathcurveto{\pgfqpoint{2.871480in}{2.018947in}}{\pgfqpoint{2.875870in}{2.029546in}}{\pgfqpoint{2.875870in}{2.040596in}}%
\pgfpathcurveto{\pgfqpoint{2.875870in}{2.051646in}}{\pgfqpoint{2.871480in}{2.062245in}}{\pgfqpoint{2.863666in}{2.070059in}}%
\pgfpathcurveto{\pgfqpoint{2.855852in}{2.077872in}}{\pgfqpoint{2.845253in}{2.082263in}}{\pgfqpoint{2.834203in}{2.082263in}}%
\pgfpathcurveto{\pgfqpoint{2.823153in}{2.082263in}}{\pgfqpoint{2.812554in}{2.077872in}}{\pgfqpoint{2.804740in}{2.070059in}}%
\pgfpathcurveto{\pgfqpoint{2.796927in}{2.062245in}}{\pgfqpoint{2.792537in}{2.051646in}}{\pgfqpoint{2.792537in}{2.040596in}}%
\pgfpathcurveto{\pgfqpoint{2.792537in}{2.029546in}}{\pgfqpoint{2.796927in}{2.018947in}}{\pgfqpoint{2.804740in}{2.011133in}}%
\pgfpathcurveto{\pgfqpoint{2.812554in}{2.003319in}}{\pgfqpoint{2.823153in}{1.998929in}}{\pgfqpoint{2.834203in}{1.998929in}}%
\pgfpathclose%
\pgfusepath{stroke,fill}%
\end{pgfscope}%
\begin{pgfscope}%
\pgfpathrectangle{\pgfqpoint{0.481978in}{0.331635in}}{\pgfqpoint{4.960000in}{3.696000in}}%
\pgfusepath{clip}%
\pgfsetbuttcap%
\pgfsetroundjoin%
\definecolor{currentfill}{rgb}{0.631373,0.788235,0.956863}%
\pgfsetfillcolor{currentfill}%
\pgfsetlinewidth{0.481800pt}%
\definecolor{currentstroke}{rgb}{1.000000,1.000000,1.000000}%
\pgfsetstrokecolor{currentstroke}%
\pgfsetdash{}{0pt}%
\pgfpathmoveto{\pgfqpoint{3.622997in}{3.704089in}}%
\pgfpathcurveto{\pgfqpoint{3.634047in}{3.704089in}}{\pgfqpoint{3.644646in}{3.708479in}}{\pgfqpoint{3.652460in}{3.716293in}}%
\pgfpathcurveto{\pgfqpoint{3.660273in}{3.724106in}}{\pgfqpoint{3.664664in}{3.734705in}}{\pgfqpoint{3.664664in}{3.745756in}}%
\pgfpathcurveto{\pgfqpoint{3.664664in}{3.756806in}}{\pgfqpoint{3.660273in}{3.767405in}}{\pgfqpoint{3.652460in}{3.775218in}}%
\pgfpathcurveto{\pgfqpoint{3.644646in}{3.783032in}}{\pgfqpoint{3.634047in}{3.787422in}}{\pgfqpoint{3.622997in}{3.787422in}}%
\pgfpathcurveto{\pgfqpoint{3.611947in}{3.787422in}}{\pgfqpoint{3.601348in}{3.783032in}}{\pgfqpoint{3.593534in}{3.775218in}}%
\pgfpathcurveto{\pgfqpoint{3.585721in}{3.767405in}}{\pgfqpoint{3.581330in}{3.756806in}}{\pgfqpoint{3.581330in}{3.745756in}}%
\pgfpathcurveto{\pgfqpoint{3.581330in}{3.734705in}}{\pgfqpoint{3.585721in}{3.724106in}}{\pgfqpoint{3.593534in}{3.716293in}}%
\pgfpathcurveto{\pgfqpoint{3.601348in}{3.708479in}}{\pgfqpoint{3.611947in}{3.704089in}}{\pgfqpoint{3.622997in}{3.704089in}}%
\pgfpathclose%
\pgfusepath{stroke,fill}%
\end{pgfscope}%
\begin{pgfscope}%
\pgfpathrectangle{\pgfqpoint{0.481978in}{0.331635in}}{\pgfqpoint{4.960000in}{3.696000in}}%
\pgfusepath{clip}%
\pgfsetbuttcap%
\pgfsetroundjoin%
\definecolor{currentfill}{rgb}{0.631373,0.788235,0.956863}%
\pgfsetfillcolor{currentfill}%
\pgfsetlinewidth{0.481800pt}%
\definecolor{currentstroke}{rgb}{1.000000,1.000000,1.000000}%
\pgfsetstrokecolor{currentstroke}%
\pgfsetdash{}{0pt}%
\pgfpathmoveto{\pgfqpoint{4.452381in}{3.250606in}}%
\pgfpathcurveto{\pgfqpoint{4.463431in}{3.250606in}}{\pgfqpoint{4.474030in}{3.254997in}}{\pgfqpoint{4.481844in}{3.262810in}}%
\pgfpathcurveto{\pgfqpoint{4.489657in}{3.270624in}}{\pgfqpoint{4.494048in}{3.281223in}}{\pgfqpoint{4.494048in}{3.292273in}}%
\pgfpathcurveto{\pgfqpoint{4.494048in}{3.303323in}}{\pgfqpoint{4.489657in}{3.313922in}}{\pgfqpoint{4.481844in}{3.321736in}}%
\pgfpathcurveto{\pgfqpoint{4.474030in}{3.329549in}}{\pgfqpoint{4.463431in}{3.333940in}}{\pgfqpoint{4.452381in}{3.333940in}}%
\pgfpathcurveto{\pgfqpoint{4.441331in}{3.333940in}}{\pgfqpoint{4.430732in}{3.329549in}}{\pgfqpoint{4.422918in}{3.321736in}}%
\pgfpathcurveto{\pgfqpoint{4.415105in}{3.313922in}}{\pgfqpoint{4.410714in}{3.303323in}}{\pgfqpoint{4.410714in}{3.292273in}}%
\pgfpathcurveto{\pgfqpoint{4.410714in}{3.281223in}}{\pgfqpoint{4.415105in}{3.270624in}}{\pgfqpoint{4.422918in}{3.262810in}}%
\pgfpathcurveto{\pgfqpoint{4.430732in}{3.254997in}}{\pgfqpoint{4.441331in}{3.250606in}}{\pgfqpoint{4.452381in}{3.250606in}}%
\pgfpathclose%
\pgfusepath{stroke,fill}%
\end{pgfscope}%
\begin{pgfscope}%
\pgfpathrectangle{\pgfqpoint{0.481978in}{0.331635in}}{\pgfqpoint{4.960000in}{3.696000in}}%
\pgfusepath{clip}%
\pgfsetbuttcap%
\pgfsetroundjoin%
\definecolor{currentfill}{rgb}{0.631373,0.788235,0.956863}%
\pgfsetfillcolor{currentfill}%
\pgfsetlinewidth{0.481800pt}%
\definecolor{currentstroke}{rgb}{1.000000,1.000000,1.000000}%
\pgfsetstrokecolor{currentstroke}%
\pgfsetdash{}{0pt}%
\pgfpathmoveto{\pgfqpoint{3.564111in}{2.417782in}}%
\pgfpathcurveto{\pgfqpoint{3.575161in}{2.417782in}}{\pgfqpoint{3.585760in}{2.422172in}}{\pgfqpoint{3.593573in}{2.429986in}}%
\pgfpathcurveto{\pgfqpoint{3.601387in}{2.437799in}}{\pgfqpoint{3.605777in}{2.448398in}}{\pgfqpoint{3.605777in}{2.459448in}}%
\pgfpathcurveto{\pgfqpoint{3.605777in}{2.470498in}}{\pgfqpoint{3.601387in}{2.481097in}}{\pgfqpoint{3.593573in}{2.488911in}}%
\pgfpathcurveto{\pgfqpoint{3.585760in}{2.496725in}}{\pgfqpoint{3.575161in}{2.501115in}}{\pgfqpoint{3.564111in}{2.501115in}}%
\pgfpathcurveto{\pgfqpoint{3.553060in}{2.501115in}}{\pgfqpoint{3.542461in}{2.496725in}}{\pgfqpoint{3.534648in}{2.488911in}}%
\pgfpathcurveto{\pgfqpoint{3.526834in}{2.481097in}}{\pgfqpoint{3.522444in}{2.470498in}}{\pgfqpoint{3.522444in}{2.459448in}}%
\pgfpathcurveto{\pgfqpoint{3.522444in}{2.448398in}}{\pgfqpoint{3.526834in}{2.437799in}}{\pgfqpoint{3.534648in}{2.429986in}}%
\pgfpathcurveto{\pgfqpoint{3.542461in}{2.422172in}}{\pgfqpoint{3.553060in}{2.417782in}}{\pgfqpoint{3.564111in}{2.417782in}}%
\pgfpathclose%
\pgfusepath{stroke,fill}%
\end{pgfscope}%
\begin{pgfscope}%
\pgfpathrectangle{\pgfqpoint{0.481978in}{0.331635in}}{\pgfqpoint{4.960000in}{3.696000in}}%
\pgfusepath{clip}%
\pgfsetbuttcap%
\pgfsetroundjoin%
\definecolor{currentfill}{rgb}{0.631373,0.788235,0.956863}%
\pgfsetfillcolor{currentfill}%
\pgfsetlinewidth{0.481800pt}%
\definecolor{currentstroke}{rgb}{1.000000,1.000000,1.000000}%
\pgfsetstrokecolor{currentstroke}%
\pgfsetdash{}{0pt}%
\pgfpathmoveto{\pgfqpoint{4.549652in}{2.430798in}}%
\pgfpathcurveto{\pgfqpoint{4.560702in}{2.430798in}}{\pgfqpoint{4.571301in}{2.435188in}}{\pgfqpoint{4.579115in}{2.443002in}}%
\pgfpathcurveto{\pgfqpoint{4.586929in}{2.450815in}}{\pgfqpoint{4.591319in}{2.461414in}}{\pgfqpoint{4.591319in}{2.472464in}}%
\pgfpathcurveto{\pgfqpoint{4.591319in}{2.483514in}}{\pgfqpoint{4.586929in}{2.494113in}}{\pgfqpoint{4.579115in}{2.501927in}}%
\pgfpathcurveto{\pgfqpoint{4.571301in}{2.509741in}}{\pgfqpoint{4.560702in}{2.514131in}}{\pgfqpoint{4.549652in}{2.514131in}}%
\pgfpathcurveto{\pgfqpoint{4.538602in}{2.514131in}}{\pgfqpoint{4.528003in}{2.509741in}}{\pgfqpoint{4.520190in}{2.501927in}}%
\pgfpathcurveto{\pgfqpoint{4.512376in}{2.494113in}}{\pgfqpoint{4.507986in}{2.483514in}}{\pgfqpoint{4.507986in}{2.472464in}}%
\pgfpathcurveto{\pgfqpoint{4.507986in}{2.461414in}}{\pgfqpoint{4.512376in}{2.450815in}}{\pgfqpoint{4.520190in}{2.443002in}}%
\pgfpathcurveto{\pgfqpoint{4.528003in}{2.435188in}}{\pgfqpoint{4.538602in}{2.430798in}}{\pgfqpoint{4.549652in}{2.430798in}}%
\pgfpathclose%
\pgfusepath{stroke,fill}%
\end{pgfscope}%
\begin{pgfscope}%
\pgfpathrectangle{\pgfqpoint{0.481978in}{0.331635in}}{\pgfqpoint{4.960000in}{3.696000in}}%
\pgfusepath{clip}%
\pgfsetbuttcap%
\pgfsetroundjoin%
\definecolor{currentfill}{rgb}{0.631373,0.788235,0.956863}%
\pgfsetfillcolor{currentfill}%
\pgfsetlinewidth{0.481800pt}%
\definecolor{currentstroke}{rgb}{1.000000,1.000000,1.000000}%
\pgfsetstrokecolor{currentstroke}%
\pgfsetdash{}{0pt}%
\pgfpathmoveto{\pgfqpoint{2.673695in}{1.800761in}}%
\pgfpathcurveto{\pgfqpoint{2.684745in}{1.800761in}}{\pgfqpoint{2.695344in}{1.805152in}}{\pgfqpoint{2.703158in}{1.812965in}}%
\pgfpathcurveto{\pgfqpoint{2.710972in}{1.820779in}}{\pgfqpoint{2.715362in}{1.831378in}}{\pgfqpoint{2.715362in}{1.842428in}}%
\pgfpathcurveto{\pgfqpoint{2.715362in}{1.853478in}}{\pgfqpoint{2.710972in}{1.864077in}}{\pgfqpoint{2.703158in}{1.871891in}}%
\pgfpathcurveto{\pgfqpoint{2.695344in}{1.879704in}}{\pgfqpoint{2.684745in}{1.884095in}}{\pgfqpoint{2.673695in}{1.884095in}}%
\pgfpathcurveto{\pgfqpoint{2.662645in}{1.884095in}}{\pgfqpoint{2.652046in}{1.879704in}}{\pgfqpoint{2.644233in}{1.871891in}}%
\pgfpathcurveto{\pgfqpoint{2.636419in}{1.864077in}}{\pgfqpoint{2.632029in}{1.853478in}}{\pgfqpoint{2.632029in}{1.842428in}}%
\pgfpathcurveto{\pgfqpoint{2.632029in}{1.831378in}}{\pgfqpoint{2.636419in}{1.820779in}}{\pgfqpoint{2.644233in}{1.812965in}}%
\pgfpathcurveto{\pgfqpoint{2.652046in}{1.805152in}}{\pgfqpoint{2.662645in}{1.800761in}}{\pgfqpoint{2.673695in}{1.800761in}}%
\pgfpathclose%
\pgfusepath{stroke,fill}%
\end{pgfscope}%
\begin{pgfscope}%
\pgfpathrectangle{\pgfqpoint{0.481978in}{0.331635in}}{\pgfqpoint{4.960000in}{3.696000in}}%
\pgfusepath{clip}%
\pgfsetbuttcap%
\pgfsetroundjoin%
\definecolor{currentfill}{rgb}{0.631373,0.788235,0.956863}%
\pgfsetfillcolor{currentfill}%
\pgfsetlinewidth{0.481800pt}%
\definecolor{currentstroke}{rgb}{1.000000,1.000000,1.000000}%
\pgfsetstrokecolor{currentstroke}%
\pgfsetdash{}{0pt}%
\pgfpathmoveto{\pgfqpoint{3.995642in}{3.613340in}}%
\pgfpathcurveto{\pgfqpoint{4.006692in}{3.613340in}}{\pgfqpoint{4.017291in}{3.617730in}}{\pgfqpoint{4.025105in}{3.625544in}}%
\pgfpathcurveto{\pgfqpoint{4.032918in}{3.633358in}}{\pgfqpoint{4.037309in}{3.643957in}}{\pgfqpoint{4.037309in}{3.655007in}}%
\pgfpathcurveto{\pgfqpoint{4.037309in}{3.666057in}}{\pgfqpoint{4.032918in}{3.676656in}}{\pgfqpoint{4.025105in}{3.684470in}}%
\pgfpathcurveto{\pgfqpoint{4.017291in}{3.692283in}}{\pgfqpoint{4.006692in}{3.696673in}}{\pgfqpoint{3.995642in}{3.696673in}}%
\pgfpathcurveto{\pgfqpoint{3.984592in}{3.696673in}}{\pgfqpoint{3.973993in}{3.692283in}}{\pgfqpoint{3.966179in}{3.684470in}}%
\pgfpathcurveto{\pgfqpoint{3.958366in}{3.676656in}}{\pgfqpoint{3.953975in}{3.666057in}}{\pgfqpoint{3.953975in}{3.655007in}}%
\pgfpathcurveto{\pgfqpoint{3.953975in}{3.643957in}}{\pgfqpoint{3.958366in}{3.633358in}}{\pgfqpoint{3.966179in}{3.625544in}}%
\pgfpathcurveto{\pgfqpoint{3.973993in}{3.617730in}}{\pgfqpoint{3.984592in}{3.613340in}}{\pgfqpoint{3.995642in}{3.613340in}}%
\pgfpathclose%
\pgfusepath{stroke,fill}%
\end{pgfscope}%
\begin{pgfscope}%
\pgfpathrectangle{\pgfqpoint{0.481978in}{0.331635in}}{\pgfqpoint{4.960000in}{3.696000in}}%
\pgfusepath{clip}%
\pgfsetbuttcap%
\pgfsetroundjoin%
\definecolor{currentfill}{rgb}{0.631373,0.788235,0.956863}%
\pgfsetfillcolor{currentfill}%
\pgfsetlinewidth{0.481800pt}%
\definecolor{currentstroke}{rgb}{1.000000,1.000000,1.000000}%
\pgfsetstrokecolor{currentstroke}%
\pgfsetdash{}{0pt}%
\pgfpathmoveto{\pgfqpoint{2.233135in}{1.979396in}}%
\pgfpathcurveto{\pgfqpoint{2.244186in}{1.979396in}}{\pgfqpoint{2.254785in}{1.983786in}}{\pgfqpoint{2.262598in}{1.991599in}}%
\pgfpathcurveto{\pgfqpoint{2.270412in}{1.999413in}}{\pgfqpoint{2.274802in}{2.010012in}}{\pgfqpoint{2.274802in}{2.021062in}}%
\pgfpathcurveto{\pgfqpoint{2.274802in}{2.032112in}}{\pgfqpoint{2.270412in}{2.042711in}}{\pgfqpoint{2.262598in}{2.050525in}}%
\pgfpathcurveto{\pgfqpoint{2.254785in}{2.058339in}}{\pgfqpoint{2.244186in}{2.062729in}}{\pgfqpoint{2.233135in}{2.062729in}}%
\pgfpathcurveto{\pgfqpoint{2.222085in}{2.062729in}}{\pgfqpoint{2.211486in}{2.058339in}}{\pgfqpoint{2.203673in}{2.050525in}}%
\pgfpathcurveto{\pgfqpoint{2.195859in}{2.042711in}}{\pgfqpoint{2.191469in}{2.032112in}}{\pgfqpoint{2.191469in}{2.021062in}}%
\pgfpathcurveto{\pgfqpoint{2.191469in}{2.010012in}}{\pgfqpoint{2.195859in}{1.999413in}}{\pgfqpoint{2.203673in}{1.991599in}}%
\pgfpathcurveto{\pgfqpoint{2.211486in}{1.983786in}}{\pgfqpoint{2.222085in}{1.979396in}}{\pgfqpoint{2.233135in}{1.979396in}}%
\pgfpathclose%
\pgfusepath{stroke,fill}%
\end{pgfscope}%
\begin{pgfscope}%
\pgfpathrectangle{\pgfqpoint{0.481978in}{0.331635in}}{\pgfqpoint{4.960000in}{3.696000in}}%
\pgfusepath{clip}%
\pgfsetbuttcap%
\pgfsetroundjoin%
\definecolor{currentfill}{rgb}{0.631373,0.788235,0.956863}%
\pgfsetfillcolor{currentfill}%
\pgfsetlinewidth{0.481800pt}%
\definecolor{currentstroke}{rgb}{1.000000,1.000000,1.000000}%
\pgfsetstrokecolor{currentstroke}%
\pgfsetdash{}{0pt}%
\pgfpathmoveto{\pgfqpoint{4.693644in}{3.288172in}}%
\pgfpathcurveto{\pgfqpoint{4.704694in}{3.288172in}}{\pgfqpoint{4.715293in}{3.292563in}}{\pgfqpoint{4.723107in}{3.300376in}}%
\pgfpathcurveto{\pgfqpoint{4.730920in}{3.308190in}}{\pgfqpoint{4.735311in}{3.318789in}}{\pgfqpoint{4.735311in}{3.329839in}}%
\pgfpathcurveto{\pgfqpoint{4.735311in}{3.340889in}}{\pgfqpoint{4.730920in}{3.351488in}}{\pgfqpoint{4.723107in}{3.359302in}}%
\pgfpathcurveto{\pgfqpoint{4.715293in}{3.367115in}}{\pgfqpoint{4.704694in}{3.371506in}}{\pgfqpoint{4.693644in}{3.371506in}}%
\pgfpathcurveto{\pgfqpoint{4.682594in}{3.371506in}}{\pgfqpoint{4.671995in}{3.367115in}}{\pgfqpoint{4.664181in}{3.359302in}}%
\pgfpathcurveto{\pgfqpoint{4.656368in}{3.351488in}}{\pgfqpoint{4.651977in}{3.340889in}}{\pgfqpoint{4.651977in}{3.329839in}}%
\pgfpathcurveto{\pgfqpoint{4.651977in}{3.318789in}}{\pgfqpoint{4.656368in}{3.308190in}}{\pgfqpoint{4.664181in}{3.300376in}}%
\pgfpathcurveto{\pgfqpoint{4.671995in}{3.292563in}}{\pgfqpoint{4.682594in}{3.288172in}}{\pgfqpoint{4.693644in}{3.288172in}}%
\pgfpathclose%
\pgfusepath{stroke,fill}%
\end{pgfscope}%
\begin{pgfscope}%
\pgfpathrectangle{\pgfqpoint{0.481978in}{0.331635in}}{\pgfqpoint{4.960000in}{3.696000in}}%
\pgfusepath{clip}%
\pgfsetbuttcap%
\pgfsetroundjoin%
\definecolor{currentfill}{rgb}{0.631373,0.788235,0.956863}%
\pgfsetfillcolor{currentfill}%
\pgfsetlinewidth{0.481800pt}%
\definecolor{currentstroke}{rgb}{1.000000,1.000000,1.000000}%
\pgfsetstrokecolor{currentstroke}%
\pgfsetdash{}{0pt}%
\pgfpathmoveto{\pgfqpoint{2.693461in}{2.896888in}}%
\pgfpathcurveto{\pgfqpoint{2.704511in}{2.896888in}}{\pgfqpoint{2.715110in}{2.901279in}}{\pgfqpoint{2.722923in}{2.909092in}}%
\pgfpathcurveto{\pgfqpoint{2.730737in}{2.916906in}}{\pgfqpoint{2.735127in}{2.927505in}}{\pgfqpoint{2.735127in}{2.938555in}}%
\pgfpathcurveto{\pgfqpoint{2.735127in}{2.949605in}}{\pgfqpoint{2.730737in}{2.960204in}}{\pgfqpoint{2.722923in}{2.968018in}}%
\pgfpathcurveto{\pgfqpoint{2.715110in}{2.975831in}}{\pgfqpoint{2.704511in}{2.980222in}}{\pgfqpoint{2.693461in}{2.980222in}}%
\pgfpathcurveto{\pgfqpoint{2.682410in}{2.980222in}}{\pgfqpoint{2.671811in}{2.975831in}}{\pgfqpoint{2.663998in}{2.968018in}}%
\pgfpathcurveto{\pgfqpoint{2.656184in}{2.960204in}}{\pgfqpoint{2.651794in}{2.949605in}}{\pgfqpoint{2.651794in}{2.938555in}}%
\pgfpathcurveto{\pgfqpoint{2.651794in}{2.927505in}}{\pgfqpoint{2.656184in}{2.916906in}}{\pgfqpoint{2.663998in}{2.909092in}}%
\pgfpathcurveto{\pgfqpoint{2.671811in}{2.901279in}}{\pgfqpoint{2.682410in}{2.896888in}}{\pgfqpoint{2.693461in}{2.896888in}}%
\pgfpathclose%
\pgfusepath{stroke,fill}%
\end{pgfscope}%
\begin{pgfscope}%
\pgfpathrectangle{\pgfqpoint{0.481978in}{0.331635in}}{\pgfqpoint{4.960000in}{3.696000in}}%
\pgfusepath{clip}%
\pgfsetbuttcap%
\pgfsetroundjoin%
\definecolor{currentfill}{rgb}{0.631373,0.788235,0.956863}%
\pgfsetfillcolor{currentfill}%
\pgfsetlinewidth{0.481800pt}%
\definecolor{currentstroke}{rgb}{1.000000,1.000000,1.000000}%
\pgfsetstrokecolor{currentstroke}%
\pgfsetdash{}{0pt}%
\pgfpathmoveto{\pgfqpoint{3.990681in}{1.655737in}}%
\pgfpathcurveto{\pgfqpoint{4.001731in}{1.655737in}}{\pgfqpoint{4.012330in}{1.660127in}}{\pgfqpoint{4.020144in}{1.667941in}}%
\pgfpathcurveto{\pgfqpoint{4.027958in}{1.675755in}}{\pgfqpoint{4.032348in}{1.686354in}}{\pgfqpoint{4.032348in}{1.697404in}}%
\pgfpathcurveto{\pgfqpoint{4.032348in}{1.708454in}}{\pgfqpoint{4.027958in}{1.719053in}}{\pgfqpoint{4.020144in}{1.726867in}}%
\pgfpathcurveto{\pgfqpoint{4.012330in}{1.734680in}}{\pgfqpoint{4.001731in}{1.739070in}}{\pgfqpoint{3.990681in}{1.739070in}}%
\pgfpathcurveto{\pgfqpoint{3.979631in}{1.739070in}}{\pgfqpoint{3.969032in}{1.734680in}}{\pgfqpoint{3.961219in}{1.726867in}}%
\pgfpathcurveto{\pgfqpoint{3.953405in}{1.719053in}}{\pgfqpoint{3.949015in}{1.708454in}}{\pgfqpoint{3.949015in}{1.697404in}}%
\pgfpathcurveto{\pgfqpoint{3.949015in}{1.686354in}}{\pgfqpoint{3.953405in}{1.675755in}}{\pgfqpoint{3.961219in}{1.667941in}}%
\pgfpathcurveto{\pgfqpoint{3.969032in}{1.660127in}}{\pgfqpoint{3.979631in}{1.655737in}}{\pgfqpoint{3.990681in}{1.655737in}}%
\pgfpathclose%
\pgfusepath{stroke,fill}%
\end{pgfscope}%
\begin{pgfscope}%
\pgfpathrectangle{\pgfqpoint{0.481978in}{0.331635in}}{\pgfqpoint{4.960000in}{3.696000in}}%
\pgfusepath{clip}%
\pgfsetbuttcap%
\pgfsetroundjoin%
\definecolor{currentfill}{rgb}{0.631373,0.788235,0.956863}%
\pgfsetfillcolor{currentfill}%
\pgfsetlinewidth{0.481800pt}%
\definecolor{currentstroke}{rgb}{1.000000,1.000000,1.000000}%
\pgfsetstrokecolor{currentstroke}%
\pgfsetdash{}{0pt}%
\pgfpathmoveto{\pgfqpoint{3.841627in}{3.188521in}}%
\pgfpathcurveto{\pgfqpoint{3.852677in}{3.188521in}}{\pgfqpoint{3.863276in}{3.192911in}}{\pgfqpoint{3.871090in}{3.200725in}}%
\pgfpathcurveto{\pgfqpoint{3.878903in}{3.208539in}}{\pgfqpoint{3.883293in}{3.219138in}}{\pgfqpoint{3.883293in}{3.230188in}}%
\pgfpathcurveto{\pgfqpoint{3.883293in}{3.241238in}}{\pgfqpoint{3.878903in}{3.251837in}}{\pgfqpoint{3.871090in}{3.259651in}}%
\pgfpathcurveto{\pgfqpoint{3.863276in}{3.267464in}}{\pgfqpoint{3.852677in}{3.271854in}}{\pgfqpoint{3.841627in}{3.271854in}}%
\pgfpathcurveto{\pgfqpoint{3.830577in}{3.271854in}}{\pgfqpoint{3.819978in}{3.267464in}}{\pgfqpoint{3.812164in}{3.259651in}}%
\pgfpathcurveto{\pgfqpoint{3.804350in}{3.251837in}}{\pgfqpoint{3.799960in}{3.241238in}}{\pgfqpoint{3.799960in}{3.230188in}}%
\pgfpathcurveto{\pgfqpoint{3.799960in}{3.219138in}}{\pgfqpoint{3.804350in}{3.208539in}}{\pgfqpoint{3.812164in}{3.200725in}}%
\pgfpathcurveto{\pgfqpoint{3.819978in}{3.192911in}}{\pgfqpoint{3.830577in}{3.188521in}}{\pgfqpoint{3.841627in}{3.188521in}}%
\pgfpathclose%
\pgfusepath{stroke,fill}%
\end{pgfscope}%
\begin{pgfscope}%
\pgfpathrectangle{\pgfqpoint{0.481978in}{0.331635in}}{\pgfqpoint{4.960000in}{3.696000in}}%
\pgfusepath{clip}%
\pgfsetbuttcap%
\pgfsetroundjoin%
\definecolor{currentfill}{rgb}{0.631373,0.788235,0.956863}%
\pgfsetfillcolor{currentfill}%
\pgfsetlinewidth{0.481800pt}%
\definecolor{currentstroke}{rgb}{1.000000,1.000000,1.000000}%
\pgfsetstrokecolor{currentstroke}%
\pgfsetdash{}{0pt}%
\pgfpathmoveto{\pgfqpoint{1.920925in}{2.528492in}}%
\pgfpathcurveto{\pgfqpoint{1.931975in}{2.528492in}}{\pgfqpoint{1.942574in}{2.532882in}}{\pgfqpoint{1.950388in}{2.540696in}}%
\pgfpathcurveto{\pgfqpoint{1.958201in}{2.548509in}}{\pgfqpoint{1.962592in}{2.559108in}}{\pgfqpoint{1.962592in}{2.570158in}}%
\pgfpathcurveto{\pgfqpoint{1.962592in}{2.581208in}}{\pgfqpoint{1.958201in}{2.591807in}}{\pgfqpoint{1.950388in}{2.599621in}}%
\pgfpathcurveto{\pgfqpoint{1.942574in}{2.607435in}}{\pgfqpoint{1.931975in}{2.611825in}}{\pgfqpoint{1.920925in}{2.611825in}}%
\pgfpathcurveto{\pgfqpoint{1.909875in}{2.611825in}}{\pgfqpoint{1.899276in}{2.607435in}}{\pgfqpoint{1.891462in}{2.599621in}}%
\pgfpathcurveto{\pgfqpoint{1.883649in}{2.591807in}}{\pgfqpoint{1.879258in}{2.581208in}}{\pgfqpoint{1.879258in}{2.570158in}}%
\pgfpathcurveto{\pgfqpoint{1.879258in}{2.559108in}}{\pgfqpoint{1.883649in}{2.548509in}}{\pgfqpoint{1.891462in}{2.540696in}}%
\pgfpathcurveto{\pgfqpoint{1.899276in}{2.532882in}}{\pgfqpoint{1.909875in}{2.528492in}}{\pgfqpoint{1.920925in}{2.528492in}}%
\pgfpathclose%
\pgfusepath{stroke,fill}%
\end{pgfscope}%
\begin{pgfscope}%
\pgfpathrectangle{\pgfqpoint{0.481978in}{0.331635in}}{\pgfqpoint{4.960000in}{3.696000in}}%
\pgfusepath{clip}%
\pgfsetbuttcap%
\pgfsetroundjoin%
\definecolor{currentfill}{rgb}{0.631373,0.788235,0.956863}%
\pgfsetfillcolor{currentfill}%
\pgfsetlinewidth{0.481800pt}%
\definecolor{currentstroke}{rgb}{1.000000,1.000000,1.000000}%
\pgfsetstrokecolor{currentstroke}%
\pgfsetdash{}{0pt}%
\pgfpathmoveto{\pgfqpoint{3.357008in}{2.084148in}}%
\pgfpathcurveto{\pgfqpoint{3.368058in}{2.084148in}}{\pgfqpoint{3.378657in}{2.088538in}}{\pgfqpoint{3.386471in}{2.096352in}}%
\pgfpathcurveto{\pgfqpoint{3.394285in}{2.104165in}}{\pgfqpoint{3.398675in}{2.114764in}}{\pgfqpoint{3.398675in}{2.125815in}}%
\pgfpathcurveto{\pgfqpoint{3.398675in}{2.136865in}}{\pgfqpoint{3.394285in}{2.147464in}}{\pgfqpoint{3.386471in}{2.155277in}}%
\pgfpathcurveto{\pgfqpoint{3.378657in}{2.163091in}}{\pgfqpoint{3.368058in}{2.167481in}}{\pgfqpoint{3.357008in}{2.167481in}}%
\pgfpathcurveto{\pgfqpoint{3.345958in}{2.167481in}}{\pgfqpoint{3.335359in}{2.163091in}}{\pgfqpoint{3.327545in}{2.155277in}}%
\pgfpathcurveto{\pgfqpoint{3.319732in}{2.147464in}}{\pgfqpoint{3.315342in}{2.136865in}}{\pgfqpoint{3.315342in}{2.125815in}}%
\pgfpathcurveto{\pgfqpoint{3.315342in}{2.114764in}}{\pgfqpoint{3.319732in}{2.104165in}}{\pgfqpoint{3.327545in}{2.096352in}}%
\pgfpathcurveto{\pgfqpoint{3.335359in}{2.088538in}}{\pgfqpoint{3.345958in}{2.084148in}}{\pgfqpoint{3.357008in}{2.084148in}}%
\pgfpathclose%
\pgfusepath{stroke,fill}%
\end{pgfscope}%
\begin{pgfscope}%
\pgfpathrectangle{\pgfqpoint{0.481978in}{0.331635in}}{\pgfqpoint{4.960000in}{3.696000in}}%
\pgfusepath{clip}%
\pgfsetbuttcap%
\pgfsetroundjoin%
\definecolor{currentfill}{rgb}{0.631373,0.788235,0.956863}%
\pgfsetfillcolor{currentfill}%
\pgfsetlinewidth{0.481800pt}%
\definecolor{currentstroke}{rgb}{1.000000,1.000000,1.000000}%
\pgfsetstrokecolor{currentstroke}%
\pgfsetdash{}{0pt}%
\pgfpathmoveto{\pgfqpoint{2.424795in}{1.964714in}}%
\pgfpathcurveto{\pgfqpoint{2.435845in}{1.964714in}}{\pgfqpoint{2.446444in}{1.969104in}}{\pgfqpoint{2.454257in}{1.976918in}}%
\pgfpathcurveto{\pgfqpoint{2.462071in}{1.984731in}}{\pgfqpoint{2.466461in}{1.995330in}}{\pgfqpoint{2.466461in}{2.006380in}}%
\pgfpathcurveto{\pgfqpoint{2.466461in}{2.017431in}}{\pgfqpoint{2.462071in}{2.028030in}}{\pgfqpoint{2.454257in}{2.035843in}}%
\pgfpathcurveto{\pgfqpoint{2.446444in}{2.043657in}}{\pgfqpoint{2.435845in}{2.048047in}}{\pgfqpoint{2.424795in}{2.048047in}}%
\pgfpathcurveto{\pgfqpoint{2.413744in}{2.048047in}}{\pgfqpoint{2.403145in}{2.043657in}}{\pgfqpoint{2.395332in}{2.035843in}}%
\pgfpathcurveto{\pgfqpoint{2.387518in}{2.028030in}}{\pgfqpoint{2.383128in}{2.017431in}}{\pgfqpoint{2.383128in}{2.006380in}}%
\pgfpathcurveto{\pgfqpoint{2.383128in}{1.995330in}}{\pgfqpoint{2.387518in}{1.984731in}}{\pgfqpoint{2.395332in}{1.976918in}}%
\pgfpathcurveto{\pgfqpoint{2.403145in}{1.969104in}}{\pgfqpoint{2.413744in}{1.964714in}}{\pgfqpoint{2.424795in}{1.964714in}}%
\pgfpathclose%
\pgfusepath{stroke,fill}%
\end{pgfscope}%
\begin{pgfscope}%
\pgfpathrectangle{\pgfqpoint{0.481978in}{0.331635in}}{\pgfqpoint{4.960000in}{3.696000in}}%
\pgfusepath{clip}%
\pgfsetbuttcap%
\pgfsetroundjoin%
\definecolor{currentfill}{rgb}{0.631373,0.788235,0.956863}%
\pgfsetfillcolor{currentfill}%
\pgfsetlinewidth{0.481800pt}%
\definecolor{currentstroke}{rgb}{1.000000,1.000000,1.000000}%
\pgfsetstrokecolor{currentstroke}%
\pgfsetdash{}{0pt}%
\pgfpathmoveto{\pgfqpoint{2.825282in}{2.225316in}}%
\pgfpathcurveto{\pgfqpoint{2.836332in}{2.225316in}}{\pgfqpoint{2.846931in}{2.229706in}}{\pgfqpoint{2.854745in}{2.237520in}}%
\pgfpathcurveto{\pgfqpoint{2.862559in}{2.245333in}}{\pgfqpoint{2.866949in}{2.255932in}}{\pgfqpoint{2.866949in}{2.266982in}}%
\pgfpathcurveto{\pgfqpoint{2.866949in}{2.278033in}}{\pgfqpoint{2.862559in}{2.288632in}}{\pgfqpoint{2.854745in}{2.296445in}}%
\pgfpathcurveto{\pgfqpoint{2.846931in}{2.304259in}}{\pgfqpoint{2.836332in}{2.308649in}}{\pgfqpoint{2.825282in}{2.308649in}}%
\pgfpathcurveto{\pgfqpoint{2.814232in}{2.308649in}}{\pgfqpoint{2.803633in}{2.304259in}}{\pgfqpoint{2.795819in}{2.296445in}}%
\pgfpathcurveto{\pgfqpoint{2.788006in}{2.288632in}}{\pgfqpoint{2.783615in}{2.278033in}}{\pgfqpoint{2.783615in}{2.266982in}}%
\pgfpathcurveto{\pgfqpoint{2.783615in}{2.255932in}}{\pgfqpoint{2.788006in}{2.245333in}}{\pgfqpoint{2.795819in}{2.237520in}}%
\pgfpathcurveto{\pgfqpoint{2.803633in}{2.229706in}}{\pgfqpoint{2.814232in}{2.225316in}}{\pgfqpoint{2.825282in}{2.225316in}}%
\pgfpathclose%
\pgfusepath{stroke,fill}%
\end{pgfscope}%
\begin{pgfscope}%
\pgfpathrectangle{\pgfqpoint{0.481978in}{0.331635in}}{\pgfqpoint{4.960000in}{3.696000in}}%
\pgfusepath{clip}%
\pgfsetbuttcap%
\pgfsetroundjoin%
\definecolor{currentfill}{rgb}{0.631373,0.788235,0.956863}%
\pgfsetfillcolor{currentfill}%
\pgfsetlinewidth{0.481800pt}%
\definecolor{currentstroke}{rgb}{1.000000,1.000000,1.000000}%
\pgfsetstrokecolor{currentstroke}%
\pgfsetdash{}{0pt}%
\pgfpathmoveto{\pgfqpoint{3.718236in}{2.964121in}}%
\pgfpathcurveto{\pgfqpoint{3.729286in}{2.964121in}}{\pgfqpoint{3.739885in}{2.968512in}}{\pgfqpoint{3.747699in}{2.976325in}}%
\pgfpathcurveto{\pgfqpoint{3.755512in}{2.984139in}}{\pgfqpoint{3.759903in}{2.994738in}}{\pgfqpoint{3.759903in}{3.005788in}}%
\pgfpathcurveto{\pgfqpoint{3.759903in}{3.016838in}}{\pgfqpoint{3.755512in}{3.027437in}}{\pgfqpoint{3.747699in}{3.035251in}}%
\pgfpathcurveto{\pgfqpoint{3.739885in}{3.043064in}}{\pgfqpoint{3.729286in}{3.047455in}}{\pgfqpoint{3.718236in}{3.047455in}}%
\pgfpathcurveto{\pgfqpoint{3.707186in}{3.047455in}}{\pgfqpoint{3.696587in}{3.043064in}}{\pgfqpoint{3.688773in}{3.035251in}}%
\pgfpathcurveto{\pgfqpoint{3.680960in}{3.027437in}}{\pgfqpoint{3.676569in}{3.016838in}}{\pgfqpoint{3.676569in}{3.005788in}}%
\pgfpathcurveto{\pgfqpoint{3.676569in}{2.994738in}}{\pgfqpoint{3.680960in}{2.984139in}}{\pgfqpoint{3.688773in}{2.976325in}}%
\pgfpathcurveto{\pgfqpoint{3.696587in}{2.968512in}}{\pgfqpoint{3.707186in}{2.964121in}}{\pgfqpoint{3.718236in}{2.964121in}}%
\pgfpathclose%
\pgfusepath{stroke,fill}%
\end{pgfscope}%
\begin{pgfscope}%
\pgfpathrectangle{\pgfqpoint{0.481978in}{0.331635in}}{\pgfqpoint{4.960000in}{3.696000in}}%
\pgfusepath{clip}%
\pgfsetbuttcap%
\pgfsetroundjoin%
\definecolor{currentfill}{rgb}{0.631373,0.788235,0.956863}%
\pgfsetfillcolor{currentfill}%
\pgfsetlinewidth{0.481800pt}%
\definecolor{currentstroke}{rgb}{1.000000,1.000000,1.000000}%
\pgfsetstrokecolor{currentstroke}%
\pgfsetdash{}{0pt}%
\pgfpathmoveto{\pgfqpoint{1.979955in}{2.529250in}}%
\pgfpathcurveto{\pgfqpoint{1.991005in}{2.529250in}}{\pgfqpoint{2.001604in}{2.533640in}}{\pgfqpoint{2.009418in}{2.541454in}}%
\pgfpathcurveto{\pgfqpoint{2.017232in}{2.549267in}}{\pgfqpoint{2.021622in}{2.559866in}}{\pgfqpoint{2.021622in}{2.570916in}}%
\pgfpathcurveto{\pgfqpoint{2.021622in}{2.581966in}}{\pgfqpoint{2.017232in}{2.592565in}}{\pgfqpoint{2.009418in}{2.600379in}}%
\pgfpathcurveto{\pgfqpoint{2.001604in}{2.608193in}}{\pgfqpoint{1.991005in}{2.612583in}}{\pgfqpoint{1.979955in}{2.612583in}}%
\pgfpathcurveto{\pgfqpoint{1.968905in}{2.612583in}}{\pgfqpoint{1.958306in}{2.608193in}}{\pgfqpoint{1.950492in}{2.600379in}}%
\pgfpathcurveto{\pgfqpoint{1.942679in}{2.592565in}}{\pgfqpoint{1.938289in}{2.581966in}}{\pgfqpoint{1.938289in}{2.570916in}}%
\pgfpathcurveto{\pgfqpoint{1.938289in}{2.559866in}}{\pgfqpoint{1.942679in}{2.549267in}}{\pgfqpoint{1.950492in}{2.541454in}}%
\pgfpathcurveto{\pgfqpoint{1.958306in}{2.533640in}}{\pgfqpoint{1.968905in}{2.529250in}}{\pgfqpoint{1.979955in}{2.529250in}}%
\pgfpathclose%
\pgfusepath{stroke,fill}%
\end{pgfscope}%
\begin{pgfscope}%
\pgfpathrectangle{\pgfqpoint{0.481978in}{0.331635in}}{\pgfqpoint{4.960000in}{3.696000in}}%
\pgfusepath{clip}%
\pgfsetbuttcap%
\pgfsetroundjoin%
\definecolor{currentfill}{rgb}{0.631373,0.788235,0.956863}%
\pgfsetfillcolor{currentfill}%
\pgfsetlinewidth{0.481800pt}%
\definecolor{currentstroke}{rgb}{1.000000,1.000000,1.000000}%
\pgfsetstrokecolor{currentstroke}%
\pgfsetdash{}{0pt}%
\pgfpathmoveto{\pgfqpoint{3.620308in}{3.705524in}}%
\pgfpathcurveto{\pgfqpoint{3.631358in}{3.705524in}}{\pgfqpoint{3.641957in}{3.709914in}}{\pgfqpoint{3.649771in}{3.717728in}}%
\pgfpathcurveto{\pgfqpoint{3.657585in}{3.725541in}}{\pgfqpoint{3.661975in}{3.736140in}}{\pgfqpoint{3.661975in}{3.747190in}}%
\pgfpathcurveto{\pgfqpoint{3.661975in}{3.758241in}}{\pgfqpoint{3.657585in}{3.768840in}}{\pgfqpoint{3.649771in}{3.776653in}}%
\pgfpathcurveto{\pgfqpoint{3.641957in}{3.784467in}}{\pgfqpoint{3.631358in}{3.788857in}}{\pgfqpoint{3.620308in}{3.788857in}}%
\pgfpathcurveto{\pgfqpoint{3.609258in}{3.788857in}}{\pgfqpoint{3.598659in}{3.784467in}}{\pgfqpoint{3.590845in}{3.776653in}}%
\pgfpathcurveto{\pgfqpoint{3.583032in}{3.768840in}}{\pgfqpoint{3.578642in}{3.758241in}}{\pgfqpoint{3.578642in}{3.747190in}}%
\pgfpathcurveto{\pgfqpoint{3.578642in}{3.736140in}}{\pgfqpoint{3.583032in}{3.725541in}}{\pgfqpoint{3.590845in}{3.717728in}}%
\pgfpathcurveto{\pgfqpoint{3.598659in}{3.709914in}}{\pgfqpoint{3.609258in}{3.705524in}}{\pgfqpoint{3.620308in}{3.705524in}}%
\pgfpathclose%
\pgfusepath{stroke,fill}%
\end{pgfscope}%
\begin{pgfscope}%
\pgfpathrectangle{\pgfqpoint{0.481978in}{0.331635in}}{\pgfqpoint{4.960000in}{3.696000in}}%
\pgfusepath{clip}%
\pgfsetbuttcap%
\pgfsetroundjoin%
\definecolor{currentfill}{rgb}{0.631373,0.788235,0.956863}%
\pgfsetfillcolor{currentfill}%
\pgfsetlinewidth{0.481800pt}%
\definecolor{currentstroke}{rgb}{1.000000,1.000000,1.000000}%
\pgfsetstrokecolor{currentstroke}%
\pgfsetdash{}{0pt}%
\pgfpathmoveto{\pgfqpoint{3.766377in}{3.141694in}}%
\pgfpathcurveto{\pgfqpoint{3.777427in}{3.141694in}}{\pgfqpoint{3.788026in}{3.146084in}}{\pgfqpoint{3.795840in}{3.153898in}}%
\pgfpathcurveto{\pgfqpoint{3.803653in}{3.161711in}}{\pgfqpoint{3.808044in}{3.172310in}}{\pgfqpoint{3.808044in}{3.183360in}}%
\pgfpathcurveto{\pgfqpoint{3.808044in}{3.194411in}}{\pgfqpoint{3.803653in}{3.205010in}}{\pgfqpoint{3.795840in}{3.212823in}}%
\pgfpathcurveto{\pgfqpoint{3.788026in}{3.220637in}}{\pgfqpoint{3.777427in}{3.225027in}}{\pgfqpoint{3.766377in}{3.225027in}}%
\pgfpathcurveto{\pgfqpoint{3.755327in}{3.225027in}}{\pgfqpoint{3.744728in}{3.220637in}}{\pgfqpoint{3.736914in}{3.212823in}}%
\pgfpathcurveto{\pgfqpoint{3.729101in}{3.205010in}}{\pgfqpoint{3.724710in}{3.194411in}}{\pgfqpoint{3.724710in}{3.183360in}}%
\pgfpathcurveto{\pgfqpoint{3.724710in}{3.172310in}}{\pgfqpoint{3.729101in}{3.161711in}}{\pgfqpoint{3.736914in}{3.153898in}}%
\pgfpathcurveto{\pgfqpoint{3.744728in}{3.146084in}}{\pgfqpoint{3.755327in}{3.141694in}}{\pgfqpoint{3.766377in}{3.141694in}}%
\pgfpathclose%
\pgfusepath{stroke,fill}%
\end{pgfscope}%
\begin{pgfscope}%
\pgfpathrectangle{\pgfqpoint{0.481978in}{0.331635in}}{\pgfqpoint{4.960000in}{3.696000in}}%
\pgfusepath{clip}%
\pgfsetbuttcap%
\pgfsetroundjoin%
\definecolor{currentfill}{rgb}{0.631373,0.788235,0.956863}%
\pgfsetfillcolor{currentfill}%
\pgfsetlinewidth{0.481800pt}%
\definecolor{currentstroke}{rgb}{1.000000,1.000000,1.000000}%
\pgfsetstrokecolor{currentstroke}%
\pgfsetdash{}{0pt}%
\pgfpathmoveto{\pgfqpoint{3.211425in}{2.523302in}}%
\pgfpathcurveto{\pgfqpoint{3.222475in}{2.523302in}}{\pgfqpoint{3.233074in}{2.527692in}}{\pgfqpoint{3.240888in}{2.535506in}}%
\pgfpathcurveto{\pgfqpoint{3.248702in}{2.543319in}}{\pgfqpoint{3.253092in}{2.553918in}}{\pgfqpoint{3.253092in}{2.564968in}}%
\pgfpathcurveto{\pgfqpoint{3.253092in}{2.576018in}}{\pgfqpoint{3.248702in}{2.586618in}}{\pgfqpoint{3.240888in}{2.594431in}}%
\pgfpathcurveto{\pgfqpoint{3.233074in}{2.602245in}}{\pgfqpoint{3.222475in}{2.606635in}}{\pgfqpoint{3.211425in}{2.606635in}}%
\pgfpathcurveto{\pgfqpoint{3.200375in}{2.606635in}}{\pgfqpoint{3.189776in}{2.602245in}}{\pgfqpoint{3.181962in}{2.594431in}}%
\pgfpathcurveto{\pgfqpoint{3.174149in}{2.586618in}}{\pgfqpoint{3.169758in}{2.576018in}}{\pgfqpoint{3.169758in}{2.564968in}}%
\pgfpathcurveto{\pgfqpoint{3.169758in}{2.553918in}}{\pgfqpoint{3.174149in}{2.543319in}}{\pgfqpoint{3.181962in}{2.535506in}}%
\pgfpathcurveto{\pgfqpoint{3.189776in}{2.527692in}}{\pgfqpoint{3.200375in}{2.523302in}}{\pgfqpoint{3.211425in}{2.523302in}}%
\pgfpathclose%
\pgfusepath{stroke,fill}%
\end{pgfscope}%
\begin{pgfscope}%
\pgfpathrectangle{\pgfqpoint{0.481978in}{0.331635in}}{\pgfqpoint{4.960000in}{3.696000in}}%
\pgfusepath{clip}%
\pgfsetbuttcap%
\pgfsetroundjoin%
\definecolor{currentfill}{rgb}{0.631373,0.788235,0.956863}%
\pgfsetfillcolor{currentfill}%
\pgfsetlinewidth{0.481800pt}%
\definecolor{currentstroke}{rgb}{1.000000,1.000000,1.000000}%
\pgfsetstrokecolor{currentstroke}%
\pgfsetdash{}{0pt}%
\pgfpathmoveto{\pgfqpoint{3.346413in}{1.837042in}}%
\pgfpathcurveto{\pgfqpoint{3.357463in}{1.837042in}}{\pgfqpoint{3.368062in}{1.841433in}}{\pgfqpoint{3.375875in}{1.849246in}}%
\pgfpathcurveto{\pgfqpoint{3.383689in}{1.857060in}}{\pgfqpoint{3.388079in}{1.867659in}}{\pgfqpoint{3.388079in}{1.878709in}}%
\pgfpathcurveto{\pgfqpoint{3.388079in}{1.889759in}}{\pgfqpoint{3.383689in}{1.900358in}}{\pgfqpoint{3.375875in}{1.908172in}}%
\pgfpathcurveto{\pgfqpoint{3.368062in}{1.915986in}}{\pgfqpoint{3.357463in}{1.920376in}}{\pgfqpoint{3.346413in}{1.920376in}}%
\pgfpathcurveto{\pgfqpoint{3.335362in}{1.920376in}}{\pgfqpoint{3.324763in}{1.915986in}}{\pgfqpoint{3.316950in}{1.908172in}}%
\pgfpathcurveto{\pgfqpoint{3.309136in}{1.900358in}}{\pgfqpoint{3.304746in}{1.889759in}}{\pgfqpoint{3.304746in}{1.878709in}}%
\pgfpathcurveto{\pgfqpoint{3.304746in}{1.867659in}}{\pgfqpoint{3.309136in}{1.857060in}}{\pgfqpoint{3.316950in}{1.849246in}}%
\pgfpathcurveto{\pgfqpoint{3.324763in}{1.841433in}}{\pgfqpoint{3.335362in}{1.837042in}}{\pgfqpoint{3.346413in}{1.837042in}}%
\pgfpathclose%
\pgfusepath{stroke,fill}%
\end{pgfscope}%
\begin{pgfscope}%
\pgfpathrectangle{\pgfqpoint{0.481978in}{0.331635in}}{\pgfqpoint{4.960000in}{3.696000in}}%
\pgfusepath{clip}%
\pgfsetbuttcap%
\pgfsetroundjoin%
\definecolor{currentfill}{rgb}{0.631373,0.788235,0.956863}%
\pgfsetfillcolor{currentfill}%
\pgfsetlinewidth{0.481800pt}%
\definecolor{currentstroke}{rgb}{1.000000,1.000000,1.000000}%
\pgfsetstrokecolor{currentstroke}%
\pgfsetdash{}{0pt}%
\pgfpathmoveto{\pgfqpoint{2.443805in}{2.129211in}}%
\pgfpathcurveto{\pgfqpoint{2.454855in}{2.129211in}}{\pgfqpoint{2.465454in}{2.133601in}}{\pgfqpoint{2.473267in}{2.141415in}}%
\pgfpathcurveto{\pgfqpoint{2.481081in}{2.149228in}}{\pgfqpoint{2.485471in}{2.159827in}}{\pgfqpoint{2.485471in}{2.170878in}}%
\pgfpathcurveto{\pgfqpoint{2.485471in}{2.181928in}}{\pgfqpoint{2.481081in}{2.192527in}}{\pgfqpoint{2.473267in}{2.200340in}}%
\pgfpathcurveto{\pgfqpoint{2.465454in}{2.208154in}}{\pgfqpoint{2.454855in}{2.212544in}}{\pgfqpoint{2.443805in}{2.212544in}}%
\pgfpathcurveto{\pgfqpoint{2.432755in}{2.212544in}}{\pgfqpoint{2.422156in}{2.208154in}}{\pgfqpoint{2.414342in}{2.200340in}}%
\pgfpathcurveto{\pgfqpoint{2.406528in}{2.192527in}}{\pgfqpoint{2.402138in}{2.181928in}}{\pgfqpoint{2.402138in}{2.170878in}}%
\pgfpathcurveto{\pgfqpoint{2.402138in}{2.159827in}}{\pgfqpoint{2.406528in}{2.149228in}}{\pgfqpoint{2.414342in}{2.141415in}}%
\pgfpathcurveto{\pgfqpoint{2.422156in}{2.133601in}}{\pgfqpoint{2.432755in}{2.129211in}}{\pgfqpoint{2.443805in}{2.129211in}}%
\pgfpathclose%
\pgfusepath{stroke,fill}%
\end{pgfscope}%
\begin{pgfscope}%
\pgfpathrectangle{\pgfqpoint{0.481978in}{0.331635in}}{\pgfqpoint{4.960000in}{3.696000in}}%
\pgfusepath{clip}%
\pgfsetbuttcap%
\pgfsetroundjoin%
\definecolor{currentfill}{rgb}{0.631373,0.788235,0.956863}%
\pgfsetfillcolor{currentfill}%
\pgfsetlinewidth{0.481800pt}%
\definecolor{currentstroke}{rgb}{1.000000,1.000000,1.000000}%
\pgfsetstrokecolor{currentstroke}%
\pgfsetdash{}{0pt}%
\pgfpathmoveto{\pgfqpoint{2.510852in}{2.840094in}}%
\pgfpathcurveto{\pgfqpoint{2.521902in}{2.840094in}}{\pgfqpoint{2.532501in}{2.844484in}}{\pgfqpoint{2.540315in}{2.852298in}}%
\pgfpathcurveto{\pgfqpoint{2.548129in}{2.860112in}}{\pgfqpoint{2.552519in}{2.870711in}}{\pgfqpoint{2.552519in}{2.881761in}}%
\pgfpathcurveto{\pgfqpoint{2.552519in}{2.892811in}}{\pgfqpoint{2.548129in}{2.903410in}}{\pgfqpoint{2.540315in}{2.911223in}}%
\pgfpathcurveto{\pgfqpoint{2.532501in}{2.919037in}}{\pgfqpoint{2.521902in}{2.923427in}}{\pgfqpoint{2.510852in}{2.923427in}}%
\pgfpathcurveto{\pgfqpoint{2.499802in}{2.923427in}}{\pgfqpoint{2.489203in}{2.919037in}}{\pgfqpoint{2.481390in}{2.911223in}}%
\pgfpathcurveto{\pgfqpoint{2.473576in}{2.903410in}}{\pgfqpoint{2.469186in}{2.892811in}}{\pgfqpoint{2.469186in}{2.881761in}}%
\pgfpathcurveto{\pgfqpoint{2.469186in}{2.870711in}}{\pgfqpoint{2.473576in}{2.860112in}}{\pgfqpoint{2.481390in}{2.852298in}}%
\pgfpathcurveto{\pgfqpoint{2.489203in}{2.844484in}}{\pgfqpoint{2.499802in}{2.840094in}}{\pgfqpoint{2.510852in}{2.840094in}}%
\pgfpathclose%
\pgfusepath{stroke,fill}%
\end{pgfscope}%
\begin{pgfscope}%
\pgfpathrectangle{\pgfqpoint{0.481978in}{0.331635in}}{\pgfqpoint{4.960000in}{3.696000in}}%
\pgfusepath{clip}%
\pgfsetbuttcap%
\pgfsetroundjoin%
\definecolor{currentfill}{rgb}{0.631373,0.788235,0.956863}%
\pgfsetfillcolor{currentfill}%
\pgfsetlinewidth{0.481800pt}%
\definecolor{currentstroke}{rgb}{1.000000,1.000000,1.000000}%
\pgfsetstrokecolor{currentstroke}%
\pgfsetdash{}{0pt}%
\pgfpathmoveto{\pgfqpoint{2.573389in}{2.318973in}}%
\pgfpathcurveto{\pgfqpoint{2.584439in}{2.318973in}}{\pgfqpoint{2.595039in}{2.323363in}}{\pgfqpoint{2.602852in}{2.331177in}}%
\pgfpathcurveto{\pgfqpoint{2.610666in}{2.338990in}}{\pgfqpoint{2.615056in}{2.349589in}}{\pgfqpoint{2.615056in}{2.360640in}}%
\pgfpathcurveto{\pgfqpoint{2.615056in}{2.371690in}}{\pgfqpoint{2.610666in}{2.382289in}}{\pgfqpoint{2.602852in}{2.390102in}}%
\pgfpathcurveto{\pgfqpoint{2.595039in}{2.397916in}}{\pgfqpoint{2.584439in}{2.402306in}}{\pgfqpoint{2.573389in}{2.402306in}}%
\pgfpathcurveto{\pgfqpoint{2.562339in}{2.402306in}}{\pgfqpoint{2.551740in}{2.397916in}}{\pgfqpoint{2.543927in}{2.390102in}}%
\pgfpathcurveto{\pgfqpoint{2.536113in}{2.382289in}}{\pgfqpoint{2.531723in}{2.371690in}}{\pgfqpoint{2.531723in}{2.360640in}}%
\pgfpathcurveto{\pgfqpoint{2.531723in}{2.349589in}}{\pgfqpoint{2.536113in}{2.338990in}}{\pgfqpoint{2.543927in}{2.331177in}}%
\pgfpathcurveto{\pgfqpoint{2.551740in}{2.323363in}}{\pgfqpoint{2.562339in}{2.318973in}}{\pgfqpoint{2.573389in}{2.318973in}}%
\pgfpathclose%
\pgfusepath{stroke,fill}%
\end{pgfscope}%
\begin{pgfscope}%
\pgfpathrectangle{\pgfqpoint{0.481978in}{0.331635in}}{\pgfqpoint{4.960000in}{3.696000in}}%
\pgfusepath{clip}%
\pgfsetbuttcap%
\pgfsetroundjoin%
\definecolor{currentfill}{rgb}{0.631373,0.788235,0.956863}%
\pgfsetfillcolor{currentfill}%
\pgfsetlinewidth{0.481800pt}%
\definecolor{currentstroke}{rgb}{1.000000,1.000000,1.000000}%
\pgfsetstrokecolor{currentstroke}%
\pgfsetdash{}{0pt}%
\pgfpathmoveto{\pgfqpoint{3.746336in}{3.601220in}}%
\pgfpathcurveto{\pgfqpoint{3.757386in}{3.601220in}}{\pgfqpoint{3.767985in}{3.605610in}}{\pgfqpoint{3.775799in}{3.613424in}}%
\pgfpathcurveto{\pgfqpoint{3.783612in}{3.621238in}}{\pgfqpoint{3.788003in}{3.631837in}}{\pgfqpoint{3.788003in}{3.642887in}}%
\pgfpathcurveto{\pgfqpoint{3.788003in}{3.653937in}}{\pgfqpoint{3.783612in}{3.664536in}}{\pgfqpoint{3.775799in}{3.672350in}}%
\pgfpathcurveto{\pgfqpoint{3.767985in}{3.680163in}}{\pgfqpoint{3.757386in}{3.684554in}}{\pgfqpoint{3.746336in}{3.684554in}}%
\pgfpathcurveto{\pgfqpoint{3.735286in}{3.684554in}}{\pgfqpoint{3.724687in}{3.680163in}}{\pgfqpoint{3.716873in}{3.672350in}}%
\pgfpathcurveto{\pgfqpoint{3.709060in}{3.664536in}}{\pgfqpoint{3.704669in}{3.653937in}}{\pgfqpoint{3.704669in}{3.642887in}}%
\pgfpathcurveto{\pgfqpoint{3.704669in}{3.631837in}}{\pgfqpoint{3.709060in}{3.621238in}}{\pgfqpoint{3.716873in}{3.613424in}}%
\pgfpathcurveto{\pgfqpoint{3.724687in}{3.605610in}}{\pgfqpoint{3.735286in}{3.601220in}}{\pgfqpoint{3.746336in}{3.601220in}}%
\pgfpathclose%
\pgfusepath{stroke,fill}%
\end{pgfscope}%
\begin{pgfscope}%
\pgfpathrectangle{\pgfqpoint{0.481978in}{0.331635in}}{\pgfqpoint{4.960000in}{3.696000in}}%
\pgfusepath{clip}%
\pgfsetbuttcap%
\pgfsetroundjoin%
\definecolor{currentfill}{rgb}{0.631373,0.788235,0.956863}%
\pgfsetfillcolor{currentfill}%
\pgfsetlinewidth{0.481800pt}%
\definecolor{currentstroke}{rgb}{1.000000,1.000000,1.000000}%
\pgfsetstrokecolor{currentstroke}%
\pgfsetdash{}{0pt}%
\pgfpathmoveto{\pgfqpoint{3.949319in}{3.426978in}}%
\pgfpathcurveto{\pgfqpoint{3.960369in}{3.426978in}}{\pgfqpoint{3.970968in}{3.431368in}}{\pgfqpoint{3.978782in}{3.439182in}}%
\pgfpathcurveto{\pgfqpoint{3.986595in}{3.446995in}}{\pgfqpoint{3.990986in}{3.457594in}}{\pgfqpoint{3.990986in}{3.468645in}}%
\pgfpathcurveto{\pgfqpoint{3.990986in}{3.479695in}}{\pgfqpoint{3.986595in}{3.490294in}}{\pgfqpoint{3.978782in}{3.498107in}}%
\pgfpathcurveto{\pgfqpoint{3.970968in}{3.505921in}}{\pgfqpoint{3.960369in}{3.510311in}}{\pgfqpoint{3.949319in}{3.510311in}}%
\pgfpathcurveto{\pgfqpoint{3.938269in}{3.510311in}}{\pgfqpoint{3.927670in}{3.505921in}}{\pgfqpoint{3.919856in}{3.498107in}}%
\pgfpathcurveto{\pgfqpoint{3.912043in}{3.490294in}}{\pgfqpoint{3.907652in}{3.479695in}}{\pgfqpoint{3.907652in}{3.468645in}}%
\pgfpathcurveto{\pgfqpoint{3.907652in}{3.457594in}}{\pgfqpoint{3.912043in}{3.446995in}}{\pgfqpoint{3.919856in}{3.439182in}}%
\pgfpathcurveto{\pgfqpoint{3.927670in}{3.431368in}}{\pgfqpoint{3.938269in}{3.426978in}}{\pgfqpoint{3.949319in}{3.426978in}}%
\pgfpathclose%
\pgfusepath{stroke,fill}%
\end{pgfscope}%
\begin{pgfscope}%
\pgfpathrectangle{\pgfqpoint{0.481978in}{0.331635in}}{\pgfqpoint{4.960000in}{3.696000in}}%
\pgfusepath{clip}%
\pgfsetbuttcap%
\pgfsetroundjoin%
\definecolor{currentfill}{rgb}{0.631373,0.788235,0.956863}%
\pgfsetfillcolor{currentfill}%
\pgfsetlinewidth{0.481800pt}%
\definecolor{currentstroke}{rgb}{1.000000,1.000000,1.000000}%
\pgfsetstrokecolor{currentstroke}%
\pgfsetdash{}{0pt}%
\pgfpathmoveto{\pgfqpoint{3.824287in}{2.308785in}}%
\pgfpathcurveto{\pgfqpoint{3.835337in}{2.308785in}}{\pgfqpoint{3.845936in}{2.313176in}}{\pgfqpoint{3.853749in}{2.320989in}}%
\pgfpathcurveto{\pgfqpoint{3.861563in}{2.328803in}}{\pgfqpoint{3.865953in}{2.339402in}}{\pgfqpoint{3.865953in}{2.350452in}}%
\pgfpathcurveto{\pgfqpoint{3.865953in}{2.361502in}}{\pgfqpoint{3.861563in}{2.372101in}}{\pgfqpoint{3.853749in}{2.379915in}}%
\pgfpathcurveto{\pgfqpoint{3.845936in}{2.387728in}}{\pgfqpoint{3.835337in}{2.392119in}}{\pgfqpoint{3.824287in}{2.392119in}}%
\pgfpathcurveto{\pgfqpoint{3.813236in}{2.392119in}}{\pgfqpoint{3.802637in}{2.387728in}}{\pgfqpoint{3.794824in}{2.379915in}}%
\pgfpathcurveto{\pgfqpoint{3.787010in}{2.372101in}}{\pgfqpoint{3.782620in}{2.361502in}}{\pgfqpoint{3.782620in}{2.350452in}}%
\pgfpathcurveto{\pgfqpoint{3.782620in}{2.339402in}}{\pgfqpoint{3.787010in}{2.328803in}}{\pgfqpoint{3.794824in}{2.320989in}}%
\pgfpathcurveto{\pgfqpoint{3.802637in}{2.313176in}}{\pgfqpoint{3.813236in}{2.308785in}}{\pgfqpoint{3.824287in}{2.308785in}}%
\pgfpathclose%
\pgfusepath{stroke,fill}%
\end{pgfscope}%
\begin{pgfscope}%
\pgfpathrectangle{\pgfqpoint{0.481978in}{0.331635in}}{\pgfqpoint{4.960000in}{3.696000in}}%
\pgfusepath{clip}%
\pgfsetbuttcap%
\pgfsetroundjoin%
\definecolor{currentfill}{rgb}{0.631373,0.788235,0.956863}%
\pgfsetfillcolor{currentfill}%
\pgfsetlinewidth{0.481800pt}%
\definecolor{currentstroke}{rgb}{1.000000,1.000000,1.000000}%
\pgfsetstrokecolor{currentstroke}%
\pgfsetdash{}{0pt}%
\pgfpathmoveto{\pgfqpoint{2.636201in}{1.090561in}}%
\pgfpathcurveto{\pgfqpoint{2.647252in}{1.090561in}}{\pgfqpoint{2.657851in}{1.094951in}}{\pgfqpoint{2.665664in}{1.102765in}}%
\pgfpathcurveto{\pgfqpoint{2.673478in}{1.110578in}}{\pgfqpoint{2.677868in}{1.121177in}}{\pgfqpoint{2.677868in}{1.132227in}}%
\pgfpathcurveto{\pgfqpoint{2.677868in}{1.143277in}}{\pgfqpoint{2.673478in}{1.153876in}}{\pgfqpoint{2.665664in}{1.161690in}}%
\pgfpathcurveto{\pgfqpoint{2.657851in}{1.169504in}}{\pgfqpoint{2.647252in}{1.173894in}}{\pgfqpoint{2.636201in}{1.173894in}}%
\pgfpathcurveto{\pgfqpoint{2.625151in}{1.173894in}}{\pgfqpoint{2.614552in}{1.169504in}}{\pgfqpoint{2.606739in}{1.161690in}}%
\pgfpathcurveto{\pgfqpoint{2.598925in}{1.153876in}}{\pgfqpoint{2.594535in}{1.143277in}}{\pgfqpoint{2.594535in}{1.132227in}}%
\pgfpathcurveto{\pgfqpoint{2.594535in}{1.121177in}}{\pgfqpoint{2.598925in}{1.110578in}}{\pgfqpoint{2.606739in}{1.102765in}}%
\pgfpathcurveto{\pgfqpoint{2.614552in}{1.094951in}}{\pgfqpoint{2.625151in}{1.090561in}}{\pgfqpoint{2.636201in}{1.090561in}}%
\pgfpathclose%
\pgfusepath{stroke,fill}%
\end{pgfscope}%
\begin{pgfscope}%
\pgfpathrectangle{\pgfqpoint{0.481978in}{0.331635in}}{\pgfqpoint{4.960000in}{3.696000in}}%
\pgfusepath{clip}%
\pgfsetbuttcap%
\pgfsetroundjoin%
\definecolor{currentfill}{rgb}{0.631373,0.788235,0.956863}%
\pgfsetfillcolor{currentfill}%
\pgfsetlinewidth{0.481800pt}%
\definecolor{currentstroke}{rgb}{1.000000,1.000000,1.000000}%
\pgfsetstrokecolor{currentstroke}%
\pgfsetdash{}{0pt}%
\pgfpathmoveto{\pgfqpoint{3.807379in}{3.733147in}}%
\pgfpathcurveto{\pgfqpoint{3.818429in}{3.733147in}}{\pgfqpoint{3.829028in}{3.737537in}}{\pgfqpoint{3.836842in}{3.745351in}}%
\pgfpathcurveto{\pgfqpoint{3.844656in}{3.753164in}}{\pgfqpoint{3.849046in}{3.763764in}}{\pgfqpoint{3.849046in}{3.774814in}}%
\pgfpathcurveto{\pgfqpoint{3.849046in}{3.785864in}}{\pgfqpoint{3.844656in}{3.796463in}}{\pgfqpoint{3.836842in}{3.804276in}}%
\pgfpathcurveto{\pgfqpoint{3.829028in}{3.812090in}}{\pgfqpoint{3.818429in}{3.816480in}}{\pgfqpoint{3.807379in}{3.816480in}}%
\pgfpathcurveto{\pgfqpoint{3.796329in}{3.816480in}}{\pgfqpoint{3.785730in}{3.812090in}}{\pgfqpoint{3.777916in}{3.804276in}}%
\pgfpathcurveto{\pgfqpoint{3.770103in}{3.796463in}}{\pgfqpoint{3.765712in}{3.785864in}}{\pgfqpoint{3.765712in}{3.774814in}}%
\pgfpathcurveto{\pgfqpoint{3.765712in}{3.763764in}}{\pgfqpoint{3.770103in}{3.753164in}}{\pgfqpoint{3.777916in}{3.745351in}}%
\pgfpathcurveto{\pgfqpoint{3.785730in}{3.737537in}}{\pgfqpoint{3.796329in}{3.733147in}}{\pgfqpoint{3.807379in}{3.733147in}}%
\pgfpathclose%
\pgfusepath{stroke,fill}%
\end{pgfscope}%
\begin{pgfscope}%
\pgfpathrectangle{\pgfqpoint{0.481978in}{0.331635in}}{\pgfqpoint{4.960000in}{3.696000in}}%
\pgfusepath{clip}%
\pgfsetbuttcap%
\pgfsetroundjoin%
\definecolor{currentfill}{rgb}{0.631373,0.788235,0.956863}%
\pgfsetfillcolor{currentfill}%
\pgfsetlinewidth{0.481800pt}%
\definecolor{currentstroke}{rgb}{1.000000,1.000000,1.000000}%
\pgfsetstrokecolor{currentstroke}%
\pgfsetdash{}{0pt}%
\pgfpathmoveto{\pgfqpoint{3.979307in}{3.311948in}}%
\pgfpathcurveto{\pgfqpoint{3.990357in}{3.311948in}}{\pgfqpoint{4.000956in}{3.316338in}}{\pgfqpoint{4.008770in}{3.324152in}}%
\pgfpathcurveto{\pgfqpoint{4.016584in}{3.331965in}}{\pgfqpoint{4.020974in}{3.342564in}}{\pgfqpoint{4.020974in}{3.353614in}}%
\pgfpathcurveto{\pgfqpoint{4.020974in}{3.364664in}}{\pgfqpoint{4.016584in}{3.375263in}}{\pgfqpoint{4.008770in}{3.383077in}}%
\pgfpathcurveto{\pgfqpoint{4.000956in}{3.390891in}}{\pgfqpoint{3.990357in}{3.395281in}}{\pgfqpoint{3.979307in}{3.395281in}}%
\pgfpathcurveto{\pgfqpoint{3.968257in}{3.395281in}}{\pgfqpoint{3.957658in}{3.390891in}}{\pgfqpoint{3.949844in}{3.383077in}}%
\pgfpathcurveto{\pgfqpoint{3.942031in}{3.375263in}}{\pgfqpoint{3.937641in}{3.364664in}}{\pgfqpoint{3.937641in}{3.353614in}}%
\pgfpathcurveto{\pgfqpoint{3.937641in}{3.342564in}}{\pgfqpoint{3.942031in}{3.331965in}}{\pgfqpoint{3.949844in}{3.324152in}}%
\pgfpathcurveto{\pgfqpoint{3.957658in}{3.316338in}}{\pgfqpoint{3.968257in}{3.311948in}}{\pgfqpoint{3.979307in}{3.311948in}}%
\pgfpathclose%
\pgfusepath{stroke,fill}%
\end{pgfscope}%
\begin{pgfscope}%
\pgfpathrectangle{\pgfqpoint{0.481978in}{0.331635in}}{\pgfqpoint{4.960000in}{3.696000in}}%
\pgfusepath{clip}%
\pgfsetbuttcap%
\pgfsetroundjoin%
\definecolor{currentfill}{rgb}{0.631373,0.788235,0.956863}%
\pgfsetfillcolor{currentfill}%
\pgfsetlinewidth{0.481800pt}%
\definecolor{currentstroke}{rgb}{1.000000,1.000000,1.000000}%
\pgfsetstrokecolor{currentstroke}%
\pgfsetdash{}{0pt}%
\pgfpathmoveto{\pgfqpoint{3.363069in}{3.817968in}}%
\pgfpathcurveto{\pgfqpoint{3.374120in}{3.817968in}}{\pgfqpoint{3.384719in}{3.822359in}}{\pgfqpoint{3.392532in}{3.830172in}}%
\pgfpathcurveto{\pgfqpoint{3.400346in}{3.837986in}}{\pgfqpoint{3.404736in}{3.848585in}}{\pgfqpoint{3.404736in}{3.859635in}}%
\pgfpathcurveto{\pgfqpoint{3.404736in}{3.870685in}}{\pgfqpoint{3.400346in}{3.881284in}}{\pgfqpoint{3.392532in}{3.889098in}}%
\pgfpathcurveto{\pgfqpoint{3.384719in}{3.896911in}}{\pgfqpoint{3.374120in}{3.901302in}}{\pgfqpoint{3.363069in}{3.901302in}}%
\pgfpathcurveto{\pgfqpoint{3.352019in}{3.901302in}}{\pgfqpoint{3.341420in}{3.896911in}}{\pgfqpoint{3.333607in}{3.889098in}}%
\pgfpathcurveto{\pgfqpoint{3.325793in}{3.881284in}}{\pgfqpoint{3.321403in}{3.870685in}}{\pgfqpoint{3.321403in}{3.859635in}}%
\pgfpathcurveto{\pgfqpoint{3.321403in}{3.848585in}}{\pgfqpoint{3.325793in}{3.837986in}}{\pgfqpoint{3.333607in}{3.830172in}}%
\pgfpathcurveto{\pgfqpoint{3.341420in}{3.822359in}}{\pgfqpoint{3.352019in}{3.817968in}}{\pgfqpoint{3.363069in}{3.817968in}}%
\pgfpathclose%
\pgfusepath{stroke,fill}%
\end{pgfscope}%
\begin{pgfscope}%
\pgfpathrectangle{\pgfqpoint{0.481978in}{0.331635in}}{\pgfqpoint{4.960000in}{3.696000in}}%
\pgfusepath{clip}%
\pgfsetbuttcap%
\pgfsetroundjoin%
\definecolor{currentfill}{rgb}{0.631373,0.788235,0.956863}%
\pgfsetfillcolor{currentfill}%
\pgfsetlinewidth{0.481800pt}%
\definecolor{currentstroke}{rgb}{1.000000,1.000000,1.000000}%
\pgfsetstrokecolor{currentstroke}%
\pgfsetdash{}{0pt}%
\pgfpathmoveto{\pgfqpoint{3.678431in}{2.744732in}}%
\pgfpathcurveto{\pgfqpoint{3.689481in}{2.744732in}}{\pgfqpoint{3.700080in}{2.749123in}}{\pgfqpoint{3.707894in}{2.756936in}}%
\pgfpathcurveto{\pgfqpoint{3.715708in}{2.764750in}}{\pgfqpoint{3.720098in}{2.775349in}}{\pgfqpoint{3.720098in}{2.786399in}}%
\pgfpathcurveto{\pgfqpoint{3.720098in}{2.797449in}}{\pgfqpoint{3.715708in}{2.808048in}}{\pgfqpoint{3.707894in}{2.815862in}}%
\pgfpathcurveto{\pgfqpoint{3.700080in}{2.823675in}}{\pgfqpoint{3.689481in}{2.828066in}}{\pgfqpoint{3.678431in}{2.828066in}}%
\pgfpathcurveto{\pgfqpoint{3.667381in}{2.828066in}}{\pgfqpoint{3.656782in}{2.823675in}}{\pgfqpoint{3.648968in}{2.815862in}}%
\pgfpathcurveto{\pgfqpoint{3.641155in}{2.808048in}}{\pgfqpoint{3.636765in}{2.797449in}}{\pgfqpoint{3.636765in}{2.786399in}}%
\pgfpathcurveto{\pgfqpoint{3.636765in}{2.775349in}}{\pgfqpoint{3.641155in}{2.764750in}}{\pgfqpoint{3.648968in}{2.756936in}}%
\pgfpathcurveto{\pgfqpoint{3.656782in}{2.749123in}}{\pgfqpoint{3.667381in}{2.744732in}}{\pgfqpoint{3.678431in}{2.744732in}}%
\pgfpathclose%
\pgfusepath{stroke,fill}%
\end{pgfscope}%
\begin{pgfscope}%
\pgfpathrectangle{\pgfqpoint{0.481978in}{0.331635in}}{\pgfqpoint{4.960000in}{3.696000in}}%
\pgfusepath{clip}%
\pgfsetbuttcap%
\pgfsetroundjoin%
\definecolor{currentfill}{rgb}{0.631373,0.788235,0.956863}%
\pgfsetfillcolor{currentfill}%
\pgfsetlinewidth{0.481800pt}%
\definecolor{currentstroke}{rgb}{1.000000,1.000000,1.000000}%
\pgfsetstrokecolor{currentstroke}%
\pgfsetdash{}{0pt}%
\pgfpathmoveto{\pgfqpoint{3.253066in}{1.724544in}}%
\pgfpathcurveto{\pgfqpoint{3.264116in}{1.724544in}}{\pgfqpoint{3.274715in}{1.728934in}}{\pgfqpoint{3.282528in}{1.736748in}}%
\pgfpathcurveto{\pgfqpoint{3.290342in}{1.744561in}}{\pgfqpoint{3.294732in}{1.755160in}}{\pgfqpoint{3.294732in}{1.766211in}}%
\pgfpathcurveto{\pgfqpoint{3.294732in}{1.777261in}}{\pgfqpoint{3.290342in}{1.787860in}}{\pgfqpoint{3.282528in}{1.795673in}}%
\pgfpathcurveto{\pgfqpoint{3.274715in}{1.803487in}}{\pgfqpoint{3.264116in}{1.807877in}}{\pgfqpoint{3.253066in}{1.807877in}}%
\pgfpathcurveto{\pgfqpoint{3.242015in}{1.807877in}}{\pgfqpoint{3.231416in}{1.803487in}}{\pgfqpoint{3.223603in}{1.795673in}}%
\pgfpathcurveto{\pgfqpoint{3.215789in}{1.787860in}}{\pgfqpoint{3.211399in}{1.777261in}}{\pgfqpoint{3.211399in}{1.766211in}}%
\pgfpathcurveto{\pgfqpoint{3.211399in}{1.755160in}}{\pgfqpoint{3.215789in}{1.744561in}}{\pgfqpoint{3.223603in}{1.736748in}}%
\pgfpathcurveto{\pgfqpoint{3.231416in}{1.728934in}}{\pgfqpoint{3.242015in}{1.724544in}}{\pgfqpoint{3.253066in}{1.724544in}}%
\pgfpathclose%
\pgfusepath{stroke,fill}%
\end{pgfscope}%
\begin{pgfscope}%
\pgfpathrectangle{\pgfqpoint{0.481978in}{0.331635in}}{\pgfqpoint{4.960000in}{3.696000in}}%
\pgfusepath{clip}%
\pgfsetbuttcap%
\pgfsetroundjoin%
\definecolor{currentfill}{rgb}{0.631373,0.788235,0.956863}%
\pgfsetfillcolor{currentfill}%
\pgfsetlinewidth{0.481800pt}%
\definecolor{currentstroke}{rgb}{1.000000,1.000000,1.000000}%
\pgfsetstrokecolor{currentstroke}%
\pgfsetdash{}{0pt}%
\pgfpathmoveto{\pgfqpoint{4.487130in}{3.158027in}}%
\pgfpathcurveto{\pgfqpoint{4.498180in}{3.158027in}}{\pgfqpoint{4.508779in}{3.162417in}}{\pgfqpoint{4.516593in}{3.170231in}}%
\pgfpathcurveto{\pgfqpoint{4.524407in}{3.178044in}}{\pgfqpoint{4.528797in}{3.188643in}}{\pgfqpoint{4.528797in}{3.199693in}}%
\pgfpathcurveto{\pgfqpoint{4.528797in}{3.210744in}}{\pgfqpoint{4.524407in}{3.221343in}}{\pgfqpoint{4.516593in}{3.229156in}}%
\pgfpathcurveto{\pgfqpoint{4.508779in}{3.236970in}}{\pgfqpoint{4.498180in}{3.241360in}}{\pgfqpoint{4.487130in}{3.241360in}}%
\pgfpathcurveto{\pgfqpoint{4.476080in}{3.241360in}}{\pgfqpoint{4.465481in}{3.236970in}}{\pgfqpoint{4.457667in}{3.229156in}}%
\pgfpathcurveto{\pgfqpoint{4.449854in}{3.221343in}}{\pgfqpoint{4.445464in}{3.210744in}}{\pgfqpoint{4.445464in}{3.199693in}}%
\pgfpathcurveto{\pgfqpoint{4.445464in}{3.188643in}}{\pgfqpoint{4.449854in}{3.178044in}}{\pgfqpoint{4.457667in}{3.170231in}}%
\pgfpathcurveto{\pgfqpoint{4.465481in}{3.162417in}}{\pgfqpoint{4.476080in}{3.158027in}}{\pgfqpoint{4.487130in}{3.158027in}}%
\pgfpathclose%
\pgfusepath{stroke,fill}%
\end{pgfscope}%
\begin{pgfscope}%
\pgfpathrectangle{\pgfqpoint{0.481978in}{0.331635in}}{\pgfqpoint{4.960000in}{3.696000in}}%
\pgfusepath{clip}%
\pgfsetbuttcap%
\pgfsetroundjoin%
\definecolor{currentfill}{rgb}{0.631373,0.788235,0.956863}%
\pgfsetfillcolor{currentfill}%
\pgfsetlinewidth{0.481800pt}%
\definecolor{currentstroke}{rgb}{1.000000,1.000000,1.000000}%
\pgfsetstrokecolor{currentstroke}%
\pgfsetdash{}{0pt}%
\pgfpathmoveto{\pgfqpoint{2.748902in}{3.161249in}}%
\pgfpathcurveto{\pgfqpoint{2.759952in}{3.161249in}}{\pgfqpoint{2.770551in}{3.165640in}}{\pgfqpoint{2.778364in}{3.173453in}}%
\pgfpathcurveto{\pgfqpoint{2.786178in}{3.181267in}}{\pgfqpoint{2.790568in}{3.191866in}}{\pgfqpoint{2.790568in}{3.202916in}}%
\pgfpathcurveto{\pgfqpoint{2.790568in}{3.213966in}}{\pgfqpoint{2.786178in}{3.224565in}}{\pgfqpoint{2.778364in}{3.232379in}}%
\pgfpathcurveto{\pgfqpoint{2.770551in}{3.240192in}}{\pgfqpoint{2.759952in}{3.244583in}}{\pgfqpoint{2.748902in}{3.244583in}}%
\pgfpathcurveto{\pgfqpoint{2.737851in}{3.244583in}}{\pgfqpoint{2.727252in}{3.240192in}}{\pgfqpoint{2.719439in}{3.232379in}}%
\pgfpathcurveto{\pgfqpoint{2.711625in}{3.224565in}}{\pgfqpoint{2.707235in}{3.213966in}}{\pgfqpoint{2.707235in}{3.202916in}}%
\pgfpathcurveto{\pgfqpoint{2.707235in}{3.191866in}}{\pgfqpoint{2.711625in}{3.181267in}}{\pgfqpoint{2.719439in}{3.173453in}}%
\pgfpathcurveto{\pgfqpoint{2.727252in}{3.165640in}}{\pgfqpoint{2.737851in}{3.161249in}}{\pgfqpoint{2.748902in}{3.161249in}}%
\pgfpathclose%
\pgfusepath{stroke,fill}%
\end{pgfscope}%
\begin{pgfscope}%
\pgfpathrectangle{\pgfqpoint{0.481978in}{0.331635in}}{\pgfqpoint{4.960000in}{3.696000in}}%
\pgfusepath{clip}%
\pgfsetbuttcap%
\pgfsetroundjoin%
\definecolor{currentfill}{rgb}{0.631373,0.788235,0.956863}%
\pgfsetfillcolor{currentfill}%
\pgfsetlinewidth{0.481800pt}%
\definecolor{currentstroke}{rgb}{1.000000,1.000000,1.000000}%
\pgfsetstrokecolor{currentstroke}%
\pgfsetdash{}{0pt}%
\pgfpathmoveto{\pgfqpoint{3.134956in}{2.627324in}}%
\pgfpathcurveto{\pgfqpoint{3.146006in}{2.627324in}}{\pgfqpoint{3.156605in}{2.631714in}}{\pgfqpoint{3.164418in}{2.639528in}}%
\pgfpathcurveto{\pgfqpoint{3.172232in}{2.647342in}}{\pgfqpoint{3.176622in}{2.657941in}}{\pgfqpoint{3.176622in}{2.668991in}}%
\pgfpathcurveto{\pgfqpoint{3.176622in}{2.680041in}}{\pgfqpoint{3.172232in}{2.690640in}}{\pgfqpoint{3.164418in}{2.698453in}}%
\pgfpathcurveto{\pgfqpoint{3.156605in}{2.706267in}}{\pgfqpoint{3.146006in}{2.710657in}}{\pgfqpoint{3.134956in}{2.710657in}}%
\pgfpathcurveto{\pgfqpoint{3.123906in}{2.710657in}}{\pgfqpoint{3.113306in}{2.706267in}}{\pgfqpoint{3.105493in}{2.698453in}}%
\pgfpathcurveto{\pgfqpoint{3.097679in}{2.690640in}}{\pgfqpoint{3.093289in}{2.680041in}}{\pgfqpoint{3.093289in}{2.668991in}}%
\pgfpathcurveto{\pgfqpoint{3.093289in}{2.657941in}}{\pgfqpoint{3.097679in}{2.647342in}}{\pgfqpoint{3.105493in}{2.639528in}}%
\pgfpathcurveto{\pgfqpoint{3.113306in}{2.631714in}}{\pgfqpoint{3.123906in}{2.627324in}}{\pgfqpoint{3.134956in}{2.627324in}}%
\pgfpathclose%
\pgfusepath{stroke,fill}%
\end{pgfscope}%
\begin{pgfscope}%
\pgfpathrectangle{\pgfqpoint{0.481978in}{0.331635in}}{\pgfqpoint{4.960000in}{3.696000in}}%
\pgfusepath{clip}%
\pgfsetbuttcap%
\pgfsetroundjoin%
\definecolor{currentfill}{rgb}{0.631373,0.788235,0.956863}%
\pgfsetfillcolor{currentfill}%
\pgfsetlinewidth{0.481800pt}%
\definecolor{currentstroke}{rgb}{1.000000,1.000000,1.000000}%
\pgfsetstrokecolor{currentstroke}%
\pgfsetdash{}{0pt}%
\pgfpathmoveto{\pgfqpoint{1.770446in}{1.336443in}}%
\pgfpathcurveto{\pgfqpoint{1.781496in}{1.336443in}}{\pgfqpoint{1.792095in}{1.340833in}}{\pgfqpoint{1.799909in}{1.348647in}}%
\pgfpathcurveto{\pgfqpoint{1.807722in}{1.356461in}}{\pgfqpoint{1.812113in}{1.367060in}}{\pgfqpoint{1.812113in}{1.378110in}}%
\pgfpathcurveto{\pgfqpoint{1.812113in}{1.389160in}}{\pgfqpoint{1.807722in}{1.399759in}}{\pgfqpoint{1.799909in}{1.407573in}}%
\pgfpathcurveto{\pgfqpoint{1.792095in}{1.415386in}}{\pgfqpoint{1.781496in}{1.419776in}}{\pgfqpoint{1.770446in}{1.419776in}}%
\pgfpathcurveto{\pgfqpoint{1.759396in}{1.419776in}}{\pgfqpoint{1.748797in}{1.415386in}}{\pgfqpoint{1.740983in}{1.407573in}}%
\pgfpathcurveto{\pgfqpoint{1.733170in}{1.399759in}}{\pgfqpoint{1.728779in}{1.389160in}}{\pgfqpoint{1.728779in}{1.378110in}}%
\pgfpathcurveto{\pgfqpoint{1.728779in}{1.367060in}}{\pgfqpoint{1.733170in}{1.356461in}}{\pgfqpoint{1.740983in}{1.348647in}}%
\pgfpathcurveto{\pgfqpoint{1.748797in}{1.340833in}}{\pgfqpoint{1.759396in}{1.336443in}}{\pgfqpoint{1.770446in}{1.336443in}}%
\pgfpathclose%
\pgfusepath{stroke,fill}%
\end{pgfscope}%
\begin{pgfscope}%
\pgfpathrectangle{\pgfqpoint{0.481978in}{0.331635in}}{\pgfqpoint{4.960000in}{3.696000in}}%
\pgfusepath{clip}%
\pgfsetbuttcap%
\pgfsetroundjoin%
\definecolor{currentfill}{rgb}{0.631373,0.788235,0.956863}%
\pgfsetfillcolor{currentfill}%
\pgfsetlinewidth{0.481800pt}%
\definecolor{currentstroke}{rgb}{1.000000,1.000000,1.000000}%
\pgfsetstrokecolor{currentstroke}%
\pgfsetdash{}{0pt}%
\pgfpathmoveto{\pgfqpoint{3.227485in}{3.050950in}}%
\pgfpathcurveto{\pgfqpoint{3.238535in}{3.050950in}}{\pgfqpoint{3.249134in}{3.055340in}}{\pgfqpoint{3.256947in}{3.063154in}}%
\pgfpathcurveto{\pgfqpoint{3.264761in}{3.070968in}}{\pgfqpoint{3.269151in}{3.081567in}}{\pgfqpoint{3.269151in}{3.092617in}}%
\pgfpathcurveto{\pgfqpoint{3.269151in}{3.103667in}}{\pgfqpoint{3.264761in}{3.114266in}}{\pgfqpoint{3.256947in}{3.122080in}}%
\pgfpathcurveto{\pgfqpoint{3.249134in}{3.129893in}}{\pgfqpoint{3.238535in}{3.134284in}}{\pgfqpoint{3.227485in}{3.134284in}}%
\pgfpathcurveto{\pgfqpoint{3.216435in}{3.134284in}}{\pgfqpoint{3.205835in}{3.129893in}}{\pgfqpoint{3.198022in}{3.122080in}}%
\pgfpathcurveto{\pgfqpoint{3.190208in}{3.114266in}}{\pgfqpoint{3.185818in}{3.103667in}}{\pgfqpoint{3.185818in}{3.092617in}}%
\pgfpathcurveto{\pgfqpoint{3.185818in}{3.081567in}}{\pgfqpoint{3.190208in}{3.070968in}}{\pgfqpoint{3.198022in}{3.063154in}}%
\pgfpathcurveto{\pgfqpoint{3.205835in}{3.055340in}}{\pgfqpoint{3.216435in}{3.050950in}}{\pgfqpoint{3.227485in}{3.050950in}}%
\pgfpathclose%
\pgfusepath{stroke,fill}%
\end{pgfscope}%
\begin{pgfscope}%
\pgfpathrectangle{\pgfqpoint{0.481978in}{0.331635in}}{\pgfqpoint{4.960000in}{3.696000in}}%
\pgfusepath{clip}%
\pgfsetbuttcap%
\pgfsetroundjoin%
\definecolor{currentfill}{rgb}{0.631373,0.788235,0.956863}%
\pgfsetfillcolor{currentfill}%
\pgfsetlinewidth{0.481800pt}%
\definecolor{currentstroke}{rgb}{1.000000,1.000000,1.000000}%
\pgfsetstrokecolor{currentstroke}%
\pgfsetdash{}{0pt}%
\pgfpathmoveto{\pgfqpoint{3.036063in}{2.342956in}}%
\pgfpathcurveto{\pgfqpoint{3.047114in}{2.342956in}}{\pgfqpoint{3.057713in}{2.347347in}}{\pgfqpoint{3.065526in}{2.355160in}}%
\pgfpathcurveto{\pgfqpoint{3.073340in}{2.362974in}}{\pgfqpoint{3.077730in}{2.373573in}}{\pgfqpoint{3.077730in}{2.384623in}}%
\pgfpathcurveto{\pgfqpoint{3.077730in}{2.395673in}}{\pgfqpoint{3.073340in}{2.406272in}}{\pgfqpoint{3.065526in}{2.414086in}}%
\pgfpathcurveto{\pgfqpoint{3.057713in}{2.421899in}}{\pgfqpoint{3.047114in}{2.426290in}}{\pgfqpoint{3.036063in}{2.426290in}}%
\pgfpathcurveto{\pgfqpoint{3.025013in}{2.426290in}}{\pgfqpoint{3.014414in}{2.421899in}}{\pgfqpoint{3.006601in}{2.414086in}}%
\pgfpathcurveto{\pgfqpoint{2.998787in}{2.406272in}}{\pgfqpoint{2.994397in}{2.395673in}}{\pgfqpoint{2.994397in}{2.384623in}}%
\pgfpathcurveto{\pgfqpoint{2.994397in}{2.373573in}}{\pgfqpoint{2.998787in}{2.362974in}}{\pgfqpoint{3.006601in}{2.355160in}}%
\pgfpathcurveto{\pgfqpoint{3.014414in}{2.347347in}}{\pgfqpoint{3.025013in}{2.342956in}}{\pgfqpoint{3.036063in}{2.342956in}}%
\pgfpathclose%
\pgfusepath{stroke,fill}%
\end{pgfscope}%
\begin{pgfscope}%
\pgfpathrectangle{\pgfqpoint{0.481978in}{0.331635in}}{\pgfqpoint{4.960000in}{3.696000in}}%
\pgfusepath{clip}%
\pgfsetbuttcap%
\pgfsetroundjoin%
\definecolor{currentfill}{rgb}{0.631373,0.788235,0.956863}%
\pgfsetfillcolor{currentfill}%
\pgfsetlinewidth{0.481800pt}%
\definecolor{currentstroke}{rgb}{1.000000,1.000000,1.000000}%
\pgfsetstrokecolor{currentstroke}%
\pgfsetdash{}{0pt}%
\pgfpathmoveto{\pgfqpoint{3.957887in}{3.344506in}}%
\pgfpathcurveto{\pgfqpoint{3.968937in}{3.344506in}}{\pgfqpoint{3.979536in}{3.348896in}}{\pgfqpoint{3.987349in}{3.356710in}}%
\pgfpathcurveto{\pgfqpoint{3.995163in}{3.364523in}}{\pgfqpoint{3.999553in}{3.375122in}}{\pgfqpoint{3.999553in}{3.386172in}}%
\pgfpathcurveto{\pgfqpoint{3.999553in}{3.397223in}}{\pgfqpoint{3.995163in}{3.407822in}}{\pgfqpoint{3.987349in}{3.415635in}}%
\pgfpathcurveto{\pgfqpoint{3.979536in}{3.423449in}}{\pgfqpoint{3.968937in}{3.427839in}}{\pgfqpoint{3.957887in}{3.427839in}}%
\pgfpathcurveto{\pgfqpoint{3.946837in}{3.427839in}}{\pgfqpoint{3.936238in}{3.423449in}}{\pgfqpoint{3.928424in}{3.415635in}}%
\pgfpathcurveto{\pgfqpoint{3.920610in}{3.407822in}}{\pgfqpoint{3.916220in}{3.397223in}}{\pgfqpoint{3.916220in}{3.386172in}}%
\pgfpathcurveto{\pgfqpoint{3.916220in}{3.375122in}}{\pgfqpoint{3.920610in}{3.364523in}}{\pgfqpoint{3.928424in}{3.356710in}}%
\pgfpathcurveto{\pgfqpoint{3.936238in}{3.348896in}}{\pgfqpoint{3.946837in}{3.344506in}}{\pgfqpoint{3.957887in}{3.344506in}}%
\pgfpathclose%
\pgfusepath{stroke,fill}%
\end{pgfscope}%
\begin{pgfscope}%
\pgfpathrectangle{\pgfqpoint{0.481978in}{0.331635in}}{\pgfqpoint{4.960000in}{3.696000in}}%
\pgfusepath{clip}%
\pgfsetbuttcap%
\pgfsetroundjoin%
\definecolor{currentfill}{rgb}{0.631373,0.788235,0.956863}%
\pgfsetfillcolor{currentfill}%
\pgfsetlinewidth{0.481800pt}%
\definecolor{currentstroke}{rgb}{1.000000,1.000000,1.000000}%
\pgfsetstrokecolor{currentstroke}%
\pgfsetdash{}{0pt}%
\pgfpathmoveto{\pgfqpoint{2.208851in}{1.612980in}}%
\pgfpathcurveto{\pgfqpoint{2.219901in}{1.612980in}}{\pgfqpoint{2.230500in}{1.617370in}}{\pgfqpoint{2.238314in}{1.625184in}}%
\pgfpathcurveto{\pgfqpoint{2.246128in}{1.632997in}}{\pgfqpoint{2.250518in}{1.643596in}}{\pgfqpoint{2.250518in}{1.654647in}}%
\pgfpathcurveto{\pgfqpoint{2.250518in}{1.665697in}}{\pgfqpoint{2.246128in}{1.676296in}}{\pgfqpoint{2.238314in}{1.684109in}}%
\pgfpathcurveto{\pgfqpoint{2.230500in}{1.691923in}}{\pgfqpoint{2.219901in}{1.696313in}}{\pgfqpoint{2.208851in}{1.696313in}}%
\pgfpathcurveto{\pgfqpoint{2.197801in}{1.696313in}}{\pgfqpoint{2.187202in}{1.691923in}}{\pgfqpoint{2.179388in}{1.684109in}}%
\pgfpathcurveto{\pgfqpoint{2.171575in}{1.676296in}}{\pgfqpoint{2.167184in}{1.665697in}}{\pgfqpoint{2.167184in}{1.654647in}}%
\pgfpathcurveto{\pgfqpoint{2.167184in}{1.643596in}}{\pgfqpoint{2.171575in}{1.632997in}}{\pgfqpoint{2.179388in}{1.625184in}}%
\pgfpathcurveto{\pgfqpoint{2.187202in}{1.617370in}}{\pgfqpoint{2.197801in}{1.612980in}}{\pgfqpoint{2.208851in}{1.612980in}}%
\pgfpathclose%
\pgfusepath{stroke,fill}%
\end{pgfscope}%
\begin{pgfscope}%
\pgfpathrectangle{\pgfqpoint{0.481978in}{0.331635in}}{\pgfqpoint{4.960000in}{3.696000in}}%
\pgfusepath{clip}%
\pgfsetbuttcap%
\pgfsetroundjoin%
\definecolor{currentfill}{rgb}{0.631373,0.788235,0.956863}%
\pgfsetfillcolor{currentfill}%
\pgfsetlinewidth{0.481800pt}%
\definecolor{currentstroke}{rgb}{1.000000,1.000000,1.000000}%
\pgfsetstrokecolor{currentstroke}%
\pgfsetdash{}{0pt}%
\pgfpathmoveto{\pgfqpoint{2.170244in}{3.128931in}}%
\pgfpathcurveto{\pgfqpoint{2.181294in}{3.128931in}}{\pgfqpoint{2.191893in}{3.133321in}}{\pgfqpoint{2.199706in}{3.141134in}}%
\pgfpathcurveto{\pgfqpoint{2.207520in}{3.148948in}}{\pgfqpoint{2.211910in}{3.159547in}}{\pgfqpoint{2.211910in}{3.170597in}}%
\pgfpathcurveto{\pgfqpoint{2.211910in}{3.181647in}}{\pgfqpoint{2.207520in}{3.192246in}}{\pgfqpoint{2.199706in}{3.200060in}}%
\pgfpathcurveto{\pgfqpoint{2.191893in}{3.207874in}}{\pgfqpoint{2.181294in}{3.212264in}}{\pgfqpoint{2.170244in}{3.212264in}}%
\pgfpathcurveto{\pgfqpoint{2.159193in}{3.212264in}}{\pgfqpoint{2.148594in}{3.207874in}}{\pgfqpoint{2.140781in}{3.200060in}}%
\pgfpathcurveto{\pgfqpoint{2.132967in}{3.192246in}}{\pgfqpoint{2.128577in}{3.181647in}}{\pgfqpoint{2.128577in}{3.170597in}}%
\pgfpathcurveto{\pgfqpoint{2.128577in}{3.159547in}}{\pgfqpoint{2.132967in}{3.148948in}}{\pgfqpoint{2.140781in}{3.141134in}}%
\pgfpathcurveto{\pgfqpoint{2.148594in}{3.133321in}}{\pgfqpoint{2.159193in}{3.128931in}}{\pgfqpoint{2.170244in}{3.128931in}}%
\pgfpathclose%
\pgfusepath{stroke,fill}%
\end{pgfscope}%
\begin{pgfscope}%
\pgfpathrectangle{\pgfqpoint{0.481978in}{0.331635in}}{\pgfqpoint{4.960000in}{3.696000in}}%
\pgfusepath{clip}%
\pgfsetbuttcap%
\pgfsetroundjoin%
\definecolor{currentfill}{rgb}{0.631373,0.788235,0.956863}%
\pgfsetfillcolor{currentfill}%
\pgfsetlinewidth{0.481800pt}%
\definecolor{currentstroke}{rgb}{1.000000,1.000000,1.000000}%
\pgfsetstrokecolor{currentstroke}%
\pgfsetdash{}{0pt}%
\pgfpathmoveto{\pgfqpoint{4.488576in}{1.663080in}}%
\pgfpathcurveto{\pgfqpoint{4.499626in}{1.663080in}}{\pgfqpoint{4.510225in}{1.667470in}}{\pgfqpoint{4.518039in}{1.675284in}}%
\pgfpathcurveto{\pgfqpoint{4.525853in}{1.683097in}}{\pgfqpoint{4.530243in}{1.693696in}}{\pgfqpoint{4.530243in}{1.704747in}}%
\pgfpathcurveto{\pgfqpoint{4.530243in}{1.715797in}}{\pgfqpoint{4.525853in}{1.726396in}}{\pgfqpoint{4.518039in}{1.734209in}}%
\pgfpathcurveto{\pgfqpoint{4.510225in}{1.742023in}}{\pgfqpoint{4.499626in}{1.746413in}}{\pgfqpoint{4.488576in}{1.746413in}}%
\pgfpathcurveto{\pgfqpoint{4.477526in}{1.746413in}}{\pgfqpoint{4.466927in}{1.742023in}}{\pgfqpoint{4.459113in}{1.734209in}}%
\pgfpathcurveto{\pgfqpoint{4.451300in}{1.726396in}}{\pgfqpoint{4.446910in}{1.715797in}}{\pgfqpoint{4.446910in}{1.704747in}}%
\pgfpathcurveto{\pgfqpoint{4.446910in}{1.693696in}}{\pgfqpoint{4.451300in}{1.683097in}}{\pgfqpoint{4.459113in}{1.675284in}}%
\pgfpathcurveto{\pgfqpoint{4.466927in}{1.667470in}}{\pgfqpoint{4.477526in}{1.663080in}}{\pgfqpoint{4.488576in}{1.663080in}}%
\pgfpathclose%
\pgfusepath{stroke,fill}%
\end{pgfscope}%
\begin{pgfscope}%
\pgfpathrectangle{\pgfqpoint{0.481978in}{0.331635in}}{\pgfqpoint{4.960000in}{3.696000in}}%
\pgfusepath{clip}%
\pgfsetbuttcap%
\pgfsetroundjoin%
\definecolor{currentfill}{rgb}{0.631373,0.788235,0.956863}%
\pgfsetfillcolor{currentfill}%
\pgfsetlinewidth{0.481800pt}%
\definecolor{currentstroke}{rgb}{1.000000,1.000000,1.000000}%
\pgfsetstrokecolor{currentstroke}%
\pgfsetdash{}{0pt}%
\pgfpathmoveto{\pgfqpoint{2.302687in}{1.837466in}}%
\pgfpathcurveto{\pgfqpoint{2.313737in}{1.837466in}}{\pgfqpoint{2.324336in}{1.841856in}}{\pgfqpoint{2.332150in}{1.849670in}}%
\pgfpathcurveto{\pgfqpoint{2.339963in}{1.857483in}}{\pgfqpoint{2.344354in}{1.868082in}}{\pgfqpoint{2.344354in}{1.879132in}}%
\pgfpathcurveto{\pgfqpoint{2.344354in}{1.890183in}}{\pgfqpoint{2.339963in}{1.900782in}}{\pgfqpoint{2.332150in}{1.908595in}}%
\pgfpathcurveto{\pgfqpoint{2.324336in}{1.916409in}}{\pgfqpoint{2.313737in}{1.920799in}}{\pgfqpoint{2.302687in}{1.920799in}}%
\pgfpathcurveto{\pgfqpoint{2.291637in}{1.920799in}}{\pgfqpoint{2.281038in}{1.916409in}}{\pgfqpoint{2.273224in}{1.908595in}}%
\pgfpathcurveto{\pgfqpoint{2.265411in}{1.900782in}}{\pgfqpoint{2.261020in}{1.890183in}}{\pgfqpoint{2.261020in}{1.879132in}}%
\pgfpathcurveto{\pgfqpoint{2.261020in}{1.868082in}}{\pgfqpoint{2.265411in}{1.857483in}}{\pgfqpoint{2.273224in}{1.849670in}}%
\pgfpathcurveto{\pgfqpoint{2.281038in}{1.841856in}}{\pgfqpoint{2.291637in}{1.837466in}}{\pgfqpoint{2.302687in}{1.837466in}}%
\pgfpathclose%
\pgfusepath{stroke,fill}%
\end{pgfscope}%
\begin{pgfscope}%
\pgfpathrectangle{\pgfqpoint{0.481978in}{0.331635in}}{\pgfqpoint{4.960000in}{3.696000in}}%
\pgfusepath{clip}%
\pgfsetbuttcap%
\pgfsetroundjoin%
\definecolor{currentfill}{rgb}{0.631373,0.788235,0.956863}%
\pgfsetfillcolor{currentfill}%
\pgfsetlinewidth{0.481800pt}%
\definecolor{currentstroke}{rgb}{1.000000,1.000000,1.000000}%
\pgfsetstrokecolor{currentstroke}%
\pgfsetdash{}{0pt}%
\pgfpathmoveto{\pgfqpoint{4.203402in}{3.606501in}}%
\pgfpathcurveto{\pgfqpoint{4.214452in}{3.606501in}}{\pgfqpoint{4.225051in}{3.610891in}}{\pgfqpoint{4.232865in}{3.618705in}}%
\pgfpathcurveto{\pgfqpoint{4.240678in}{3.626519in}}{\pgfqpoint{4.245068in}{3.637118in}}{\pgfqpoint{4.245068in}{3.648168in}}%
\pgfpathcurveto{\pgfqpoint{4.245068in}{3.659218in}}{\pgfqpoint{4.240678in}{3.669817in}}{\pgfqpoint{4.232865in}{3.677631in}}%
\pgfpathcurveto{\pgfqpoint{4.225051in}{3.685444in}}{\pgfqpoint{4.214452in}{3.689834in}}{\pgfqpoint{4.203402in}{3.689834in}}%
\pgfpathcurveto{\pgfqpoint{4.192352in}{3.689834in}}{\pgfqpoint{4.181753in}{3.685444in}}{\pgfqpoint{4.173939in}{3.677631in}}%
\pgfpathcurveto{\pgfqpoint{4.166125in}{3.669817in}}{\pgfqpoint{4.161735in}{3.659218in}}{\pgfqpoint{4.161735in}{3.648168in}}%
\pgfpathcurveto{\pgfqpoint{4.161735in}{3.637118in}}{\pgfqpoint{4.166125in}{3.626519in}}{\pgfqpoint{4.173939in}{3.618705in}}%
\pgfpathcurveto{\pgfqpoint{4.181753in}{3.610891in}}{\pgfqpoint{4.192352in}{3.606501in}}{\pgfqpoint{4.203402in}{3.606501in}}%
\pgfpathclose%
\pgfusepath{stroke,fill}%
\end{pgfscope}%
\begin{pgfscope}%
\pgfpathrectangle{\pgfqpoint{0.481978in}{0.331635in}}{\pgfqpoint{4.960000in}{3.696000in}}%
\pgfusepath{clip}%
\pgfsetbuttcap%
\pgfsetroundjoin%
\definecolor{currentfill}{rgb}{0.631373,0.788235,0.956863}%
\pgfsetfillcolor{currentfill}%
\pgfsetlinewidth{0.481800pt}%
\definecolor{currentstroke}{rgb}{1.000000,1.000000,1.000000}%
\pgfsetstrokecolor{currentstroke}%
\pgfsetdash{}{0pt}%
\pgfpathmoveto{\pgfqpoint{2.511684in}{2.533155in}}%
\pgfpathcurveto{\pgfqpoint{2.522734in}{2.533155in}}{\pgfqpoint{2.533333in}{2.537546in}}{\pgfqpoint{2.541147in}{2.545359in}}%
\pgfpathcurveto{\pgfqpoint{2.548961in}{2.553173in}}{\pgfqpoint{2.553351in}{2.563772in}}{\pgfqpoint{2.553351in}{2.574822in}}%
\pgfpathcurveto{\pgfqpoint{2.553351in}{2.585872in}}{\pgfqpoint{2.548961in}{2.596471in}}{\pgfqpoint{2.541147in}{2.604285in}}%
\pgfpathcurveto{\pgfqpoint{2.533333in}{2.612099in}}{\pgfqpoint{2.522734in}{2.616489in}}{\pgfqpoint{2.511684in}{2.616489in}}%
\pgfpathcurveto{\pgfqpoint{2.500634in}{2.616489in}}{\pgfqpoint{2.490035in}{2.612099in}}{\pgfqpoint{2.482221in}{2.604285in}}%
\pgfpathcurveto{\pgfqpoint{2.474408in}{2.596471in}}{\pgfqpoint{2.470018in}{2.585872in}}{\pgfqpoint{2.470018in}{2.574822in}}%
\pgfpathcurveto{\pgfqpoint{2.470018in}{2.563772in}}{\pgfqpoint{2.474408in}{2.553173in}}{\pgfqpoint{2.482221in}{2.545359in}}%
\pgfpathcurveto{\pgfqpoint{2.490035in}{2.537546in}}{\pgfqpoint{2.500634in}{2.533155in}}{\pgfqpoint{2.511684in}{2.533155in}}%
\pgfpathclose%
\pgfusepath{stroke,fill}%
\end{pgfscope}%
\begin{pgfscope}%
\pgfpathrectangle{\pgfqpoint{0.481978in}{0.331635in}}{\pgfqpoint{4.960000in}{3.696000in}}%
\pgfusepath{clip}%
\pgfsetbuttcap%
\pgfsetroundjoin%
\definecolor{currentfill}{rgb}{0.631373,0.788235,0.956863}%
\pgfsetfillcolor{currentfill}%
\pgfsetlinewidth{0.481800pt}%
\definecolor{currentstroke}{rgb}{1.000000,1.000000,1.000000}%
\pgfsetstrokecolor{currentstroke}%
\pgfsetdash{}{0pt}%
\pgfpathmoveto{\pgfqpoint{5.185848in}{1.957032in}}%
\pgfpathcurveto{\pgfqpoint{5.196898in}{1.957032in}}{\pgfqpoint{5.207497in}{1.961422in}}{\pgfqpoint{5.215310in}{1.969236in}}%
\pgfpathcurveto{\pgfqpoint{5.223124in}{1.977050in}}{\pgfqpoint{5.227514in}{1.987649in}}{\pgfqpoint{5.227514in}{1.998699in}}%
\pgfpathcurveto{\pgfqpoint{5.227514in}{2.009749in}}{\pgfqpoint{5.223124in}{2.020348in}}{\pgfqpoint{5.215310in}{2.028161in}}%
\pgfpathcurveto{\pgfqpoint{5.207497in}{2.035975in}}{\pgfqpoint{5.196898in}{2.040365in}}{\pgfqpoint{5.185848in}{2.040365in}}%
\pgfpathcurveto{\pgfqpoint{5.174798in}{2.040365in}}{\pgfqpoint{5.164199in}{2.035975in}}{\pgfqpoint{5.156385in}{2.028161in}}%
\pgfpathcurveto{\pgfqpoint{5.148571in}{2.020348in}}{\pgfqpoint{5.144181in}{2.009749in}}{\pgfqpoint{5.144181in}{1.998699in}}%
\pgfpathcurveto{\pgfqpoint{5.144181in}{1.987649in}}{\pgfqpoint{5.148571in}{1.977050in}}{\pgfqpoint{5.156385in}{1.969236in}}%
\pgfpathcurveto{\pgfqpoint{5.164199in}{1.961422in}}{\pgfqpoint{5.174798in}{1.957032in}}{\pgfqpoint{5.185848in}{1.957032in}}%
\pgfpathclose%
\pgfusepath{stroke,fill}%
\end{pgfscope}%
\begin{pgfscope}%
\pgfpathrectangle{\pgfqpoint{0.481978in}{0.331635in}}{\pgfqpoint{4.960000in}{3.696000in}}%
\pgfusepath{clip}%
\pgfsetbuttcap%
\pgfsetroundjoin%
\definecolor{currentfill}{rgb}{0.631373,0.788235,0.956863}%
\pgfsetfillcolor{currentfill}%
\pgfsetlinewidth{0.481800pt}%
\definecolor{currentstroke}{rgb}{1.000000,1.000000,1.000000}%
\pgfsetstrokecolor{currentstroke}%
\pgfsetdash{}{0pt}%
\pgfpathmoveto{\pgfqpoint{4.321890in}{1.737976in}}%
\pgfpathcurveto{\pgfqpoint{4.332940in}{1.737976in}}{\pgfqpoint{4.343539in}{1.742367in}}{\pgfqpoint{4.351353in}{1.750180in}}%
\pgfpathcurveto{\pgfqpoint{4.359166in}{1.757994in}}{\pgfqpoint{4.363556in}{1.768593in}}{\pgfqpoint{4.363556in}{1.779643in}}%
\pgfpathcurveto{\pgfqpoint{4.363556in}{1.790693in}}{\pgfqpoint{4.359166in}{1.801292in}}{\pgfqpoint{4.351353in}{1.809106in}}%
\pgfpathcurveto{\pgfqpoint{4.343539in}{1.816919in}}{\pgfqpoint{4.332940in}{1.821310in}}{\pgfqpoint{4.321890in}{1.821310in}}%
\pgfpathcurveto{\pgfqpoint{4.310840in}{1.821310in}}{\pgfqpoint{4.300241in}{1.816919in}}{\pgfqpoint{4.292427in}{1.809106in}}%
\pgfpathcurveto{\pgfqpoint{4.284613in}{1.801292in}}{\pgfqpoint{4.280223in}{1.790693in}}{\pgfqpoint{4.280223in}{1.779643in}}%
\pgfpathcurveto{\pgfqpoint{4.280223in}{1.768593in}}{\pgfqpoint{4.284613in}{1.757994in}}{\pgfqpoint{4.292427in}{1.750180in}}%
\pgfpathcurveto{\pgfqpoint{4.300241in}{1.742367in}}{\pgfqpoint{4.310840in}{1.737976in}}{\pgfqpoint{4.321890in}{1.737976in}}%
\pgfpathclose%
\pgfusepath{stroke,fill}%
\end{pgfscope}%
\begin{pgfscope}%
\pgfpathrectangle{\pgfqpoint{0.481978in}{0.331635in}}{\pgfqpoint{4.960000in}{3.696000in}}%
\pgfusepath{clip}%
\pgfsetbuttcap%
\pgfsetroundjoin%
\definecolor{currentfill}{rgb}{0.631373,0.788235,0.956863}%
\pgfsetfillcolor{currentfill}%
\pgfsetlinewidth{0.481800pt}%
\definecolor{currentstroke}{rgb}{1.000000,1.000000,1.000000}%
\pgfsetstrokecolor{currentstroke}%
\pgfsetdash{}{0pt}%
\pgfpathmoveto{\pgfqpoint{3.416866in}{3.233091in}}%
\pgfpathcurveto{\pgfqpoint{3.427917in}{3.233091in}}{\pgfqpoint{3.438516in}{3.237481in}}{\pgfqpoint{3.446329in}{3.245295in}}%
\pgfpathcurveto{\pgfqpoint{3.454143in}{3.253108in}}{\pgfqpoint{3.458533in}{3.263707in}}{\pgfqpoint{3.458533in}{3.274758in}}%
\pgfpathcurveto{\pgfqpoint{3.458533in}{3.285808in}}{\pgfqpoint{3.454143in}{3.296407in}}{\pgfqpoint{3.446329in}{3.304220in}}%
\pgfpathcurveto{\pgfqpoint{3.438516in}{3.312034in}}{\pgfqpoint{3.427917in}{3.316424in}}{\pgfqpoint{3.416866in}{3.316424in}}%
\pgfpathcurveto{\pgfqpoint{3.405816in}{3.316424in}}{\pgfqpoint{3.395217in}{3.312034in}}{\pgfqpoint{3.387404in}{3.304220in}}%
\pgfpathcurveto{\pgfqpoint{3.379590in}{3.296407in}}{\pgfqpoint{3.375200in}{3.285808in}}{\pgfqpoint{3.375200in}{3.274758in}}%
\pgfpathcurveto{\pgfqpoint{3.375200in}{3.263707in}}{\pgfqpoint{3.379590in}{3.253108in}}{\pgfqpoint{3.387404in}{3.245295in}}%
\pgfpathcurveto{\pgfqpoint{3.395217in}{3.237481in}}{\pgfqpoint{3.405816in}{3.233091in}}{\pgfqpoint{3.416866in}{3.233091in}}%
\pgfpathclose%
\pgfusepath{stroke,fill}%
\end{pgfscope}%
\begin{pgfscope}%
\pgfpathrectangle{\pgfqpoint{0.481978in}{0.331635in}}{\pgfqpoint{4.960000in}{3.696000in}}%
\pgfusepath{clip}%
\pgfsetbuttcap%
\pgfsetroundjoin%
\definecolor{currentfill}{rgb}{0.631373,0.788235,0.956863}%
\pgfsetfillcolor{currentfill}%
\pgfsetlinewidth{0.481800pt}%
\definecolor{currentstroke}{rgb}{1.000000,1.000000,1.000000}%
\pgfsetstrokecolor{currentstroke}%
\pgfsetdash{}{0pt}%
\pgfpathmoveto{\pgfqpoint{2.720729in}{2.450804in}}%
\pgfpathcurveto{\pgfqpoint{2.731779in}{2.450804in}}{\pgfqpoint{2.742378in}{2.455194in}}{\pgfqpoint{2.750192in}{2.463008in}}%
\pgfpathcurveto{\pgfqpoint{2.758005in}{2.470821in}}{\pgfqpoint{2.762396in}{2.481420in}}{\pgfqpoint{2.762396in}{2.492470in}}%
\pgfpathcurveto{\pgfqpoint{2.762396in}{2.503521in}}{\pgfqpoint{2.758005in}{2.514120in}}{\pgfqpoint{2.750192in}{2.521933in}}%
\pgfpathcurveto{\pgfqpoint{2.742378in}{2.529747in}}{\pgfqpoint{2.731779in}{2.534137in}}{\pgfqpoint{2.720729in}{2.534137in}}%
\pgfpathcurveto{\pgfqpoint{2.709679in}{2.534137in}}{\pgfqpoint{2.699080in}{2.529747in}}{\pgfqpoint{2.691266in}{2.521933in}}%
\pgfpathcurveto{\pgfqpoint{2.683453in}{2.514120in}}{\pgfqpoint{2.679062in}{2.503521in}}{\pgfqpoint{2.679062in}{2.492470in}}%
\pgfpathcurveto{\pgfqpoint{2.679062in}{2.481420in}}{\pgfqpoint{2.683453in}{2.470821in}}{\pgfqpoint{2.691266in}{2.463008in}}%
\pgfpathcurveto{\pgfqpoint{2.699080in}{2.455194in}}{\pgfqpoint{2.709679in}{2.450804in}}{\pgfqpoint{2.720729in}{2.450804in}}%
\pgfpathclose%
\pgfusepath{stroke,fill}%
\end{pgfscope}%
\begin{pgfscope}%
\pgfpathrectangle{\pgfqpoint{0.481978in}{0.331635in}}{\pgfqpoint{4.960000in}{3.696000in}}%
\pgfusepath{clip}%
\pgfsetbuttcap%
\pgfsetroundjoin%
\definecolor{currentfill}{rgb}{0.631373,0.788235,0.956863}%
\pgfsetfillcolor{currentfill}%
\pgfsetlinewidth{0.481800pt}%
\definecolor{currentstroke}{rgb}{1.000000,1.000000,1.000000}%
\pgfsetstrokecolor{currentstroke}%
\pgfsetdash{}{0pt}%
\pgfpathmoveto{\pgfqpoint{4.178946in}{2.931307in}}%
\pgfpathcurveto{\pgfqpoint{4.189996in}{2.931307in}}{\pgfqpoint{4.200595in}{2.935697in}}{\pgfqpoint{4.208408in}{2.943511in}}%
\pgfpathcurveto{\pgfqpoint{4.216222in}{2.951324in}}{\pgfqpoint{4.220612in}{2.961923in}}{\pgfqpoint{4.220612in}{2.972973in}}%
\pgfpathcurveto{\pgfqpoint{4.220612in}{2.984024in}}{\pgfqpoint{4.216222in}{2.994623in}}{\pgfqpoint{4.208408in}{3.002436in}}%
\pgfpathcurveto{\pgfqpoint{4.200595in}{3.010250in}}{\pgfqpoint{4.189996in}{3.014640in}}{\pgfqpoint{4.178946in}{3.014640in}}%
\pgfpathcurveto{\pgfqpoint{4.167895in}{3.014640in}}{\pgfqpoint{4.157296in}{3.010250in}}{\pgfqpoint{4.149483in}{3.002436in}}%
\pgfpathcurveto{\pgfqpoint{4.141669in}{2.994623in}}{\pgfqpoint{4.137279in}{2.984024in}}{\pgfqpoint{4.137279in}{2.972973in}}%
\pgfpathcurveto{\pgfqpoint{4.137279in}{2.961923in}}{\pgfqpoint{4.141669in}{2.951324in}}{\pgfqpoint{4.149483in}{2.943511in}}%
\pgfpathcurveto{\pgfqpoint{4.157296in}{2.935697in}}{\pgfqpoint{4.167895in}{2.931307in}}{\pgfqpoint{4.178946in}{2.931307in}}%
\pgfpathclose%
\pgfusepath{stroke,fill}%
\end{pgfscope}%
\begin{pgfscope}%
\pgfpathrectangle{\pgfqpoint{0.481978in}{0.331635in}}{\pgfqpoint{4.960000in}{3.696000in}}%
\pgfusepath{clip}%
\pgfsetbuttcap%
\pgfsetroundjoin%
\definecolor{currentfill}{rgb}{0.631373,0.788235,0.956863}%
\pgfsetfillcolor{currentfill}%
\pgfsetlinewidth{0.481800pt}%
\definecolor{currentstroke}{rgb}{1.000000,1.000000,1.000000}%
\pgfsetstrokecolor{currentstroke}%
\pgfsetdash{}{0pt}%
\pgfpathmoveto{\pgfqpoint{2.733957in}{1.900211in}}%
\pgfpathcurveto{\pgfqpoint{2.745007in}{1.900211in}}{\pgfqpoint{2.755606in}{1.904601in}}{\pgfqpoint{2.763420in}{1.912415in}}%
\pgfpathcurveto{\pgfqpoint{2.771234in}{1.920228in}}{\pgfqpoint{2.775624in}{1.930827in}}{\pgfqpoint{2.775624in}{1.941878in}}%
\pgfpathcurveto{\pgfqpoint{2.775624in}{1.952928in}}{\pgfqpoint{2.771234in}{1.963527in}}{\pgfqpoint{2.763420in}{1.971340in}}%
\pgfpathcurveto{\pgfqpoint{2.755606in}{1.979154in}}{\pgfqpoint{2.745007in}{1.983544in}}{\pgfqpoint{2.733957in}{1.983544in}}%
\pgfpathcurveto{\pgfqpoint{2.722907in}{1.983544in}}{\pgfqpoint{2.712308in}{1.979154in}}{\pgfqpoint{2.704495in}{1.971340in}}%
\pgfpathcurveto{\pgfqpoint{2.696681in}{1.963527in}}{\pgfqpoint{2.692291in}{1.952928in}}{\pgfqpoint{2.692291in}{1.941878in}}%
\pgfpathcurveto{\pgfqpoint{2.692291in}{1.930827in}}{\pgfqpoint{2.696681in}{1.920228in}}{\pgfqpoint{2.704495in}{1.912415in}}%
\pgfpathcurveto{\pgfqpoint{2.712308in}{1.904601in}}{\pgfqpoint{2.722907in}{1.900211in}}{\pgfqpoint{2.733957in}{1.900211in}}%
\pgfpathclose%
\pgfusepath{stroke,fill}%
\end{pgfscope}%
\begin{pgfscope}%
\pgfpathrectangle{\pgfqpoint{0.481978in}{0.331635in}}{\pgfqpoint{4.960000in}{3.696000in}}%
\pgfusepath{clip}%
\pgfsetbuttcap%
\pgfsetroundjoin%
\definecolor{currentfill}{rgb}{0.631373,0.788235,0.956863}%
\pgfsetfillcolor{currentfill}%
\pgfsetlinewidth{0.481800pt}%
\definecolor{currentstroke}{rgb}{1.000000,1.000000,1.000000}%
\pgfsetstrokecolor{currentstroke}%
\pgfsetdash{}{0pt}%
\pgfpathmoveto{\pgfqpoint{2.037454in}{2.709513in}}%
\pgfpathcurveto{\pgfqpoint{2.048504in}{2.709513in}}{\pgfqpoint{2.059103in}{2.713904in}}{\pgfqpoint{2.066917in}{2.721717in}}%
\pgfpathcurveto{\pgfqpoint{2.074730in}{2.729531in}}{\pgfqpoint{2.079120in}{2.740130in}}{\pgfqpoint{2.079120in}{2.751180in}}%
\pgfpathcurveto{\pgfqpoint{2.079120in}{2.762230in}}{\pgfqpoint{2.074730in}{2.772829in}}{\pgfqpoint{2.066917in}{2.780643in}}%
\pgfpathcurveto{\pgfqpoint{2.059103in}{2.788456in}}{\pgfqpoint{2.048504in}{2.792847in}}{\pgfqpoint{2.037454in}{2.792847in}}%
\pgfpathcurveto{\pgfqpoint{2.026404in}{2.792847in}}{\pgfqpoint{2.015805in}{2.788456in}}{\pgfqpoint{2.007991in}{2.780643in}}%
\pgfpathcurveto{\pgfqpoint{2.000177in}{2.772829in}}{\pgfqpoint{1.995787in}{2.762230in}}{\pgfqpoint{1.995787in}{2.751180in}}%
\pgfpathcurveto{\pgfqpoint{1.995787in}{2.740130in}}{\pgfqpoint{2.000177in}{2.729531in}}{\pgfqpoint{2.007991in}{2.721717in}}%
\pgfpathcurveto{\pgfqpoint{2.015805in}{2.713904in}}{\pgfqpoint{2.026404in}{2.709513in}}{\pgfqpoint{2.037454in}{2.709513in}}%
\pgfpathclose%
\pgfusepath{stroke,fill}%
\end{pgfscope}%
\begin{pgfscope}%
\pgfpathrectangle{\pgfqpoint{0.481978in}{0.331635in}}{\pgfqpoint{4.960000in}{3.696000in}}%
\pgfusepath{clip}%
\pgfsetbuttcap%
\pgfsetroundjoin%
\definecolor{currentfill}{rgb}{0.631373,0.788235,0.956863}%
\pgfsetfillcolor{currentfill}%
\pgfsetlinewidth{0.481800pt}%
\definecolor{currentstroke}{rgb}{1.000000,1.000000,1.000000}%
\pgfsetstrokecolor{currentstroke}%
\pgfsetdash{}{0pt}%
\pgfpathmoveto{\pgfqpoint{4.624636in}{1.142162in}}%
\pgfpathcurveto{\pgfqpoint{4.635686in}{1.142162in}}{\pgfqpoint{4.646285in}{1.146552in}}{\pgfqpoint{4.654099in}{1.154366in}}%
\pgfpathcurveto{\pgfqpoint{4.661912in}{1.162179in}}{\pgfqpoint{4.666303in}{1.172778in}}{\pgfqpoint{4.666303in}{1.183829in}}%
\pgfpathcurveto{\pgfqpoint{4.666303in}{1.194879in}}{\pgfqpoint{4.661912in}{1.205478in}}{\pgfqpoint{4.654099in}{1.213291in}}%
\pgfpathcurveto{\pgfqpoint{4.646285in}{1.221105in}}{\pgfqpoint{4.635686in}{1.225495in}}{\pgfqpoint{4.624636in}{1.225495in}}%
\pgfpathcurveto{\pgfqpoint{4.613586in}{1.225495in}}{\pgfqpoint{4.602987in}{1.221105in}}{\pgfqpoint{4.595173in}{1.213291in}}%
\pgfpathcurveto{\pgfqpoint{4.587360in}{1.205478in}}{\pgfqpoint{4.582969in}{1.194879in}}{\pgfqpoint{4.582969in}{1.183829in}}%
\pgfpathcurveto{\pgfqpoint{4.582969in}{1.172778in}}{\pgfqpoint{4.587360in}{1.162179in}}{\pgfqpoint{4.595173in}{1.154366in}}%
\pgfpathcurveto{\pgfqpoint{4.602987in}{1.146552in}}{\pgfqpoint{4.613586in}{1.142162in}}{\pgfqpoint{4.624636in}{1.142162in}}%
\pgfpathclose%
\pgfusepath{stroke,fill}%
\end{pgfscope}%
\begin{pgfscope}%
\pgfpathrectangle{\pgfqpoint{0.481978in}{0.331635in}}{\pgfqpoint{4.960000in}{3.696000in}}%
\pgfusepath{clip}%
\pgfsetbuttcap%
\pgfsetroundjoin%
\definecolor{currentfill}{rgb}{1.000000,0.705882,0.509804}%
\pgfsetfillcolor{currentfill}%
\pgfsetlinewidth{1.003750pt}%
\definecolor{currentstroke}{rgb}{1.000000,0.705882,0.509804}%
\pgfsetstrokecolor{currentstroke}%
\pgfsetdash{}{0pt}%
\pgfsys@defobject{currentmarker}{\pgfqpoint{-0.041667in}{-0.041667in}}{\pgfqpoint{0.041667in}{0.041667in}}{%
\pgfpathmoveto{\pgfqpoint{0.000000in}{-0.041667in}}%
\pgfpathcurveto{\pgfqpoint{0.011050in}{-0.041667in}}{\pgfqpoint{0.021649in}{-0.037276in}}{\pgfqpoint{0.029463in}{-0.029463in}}%
\pgfpathcurveto{\pgfqpoint{0.037276in}{-0.021649in}}{\pgfqpoint{0.041667in}{-0.011050in}}{\pgfqpoint{0.041667in}{0.000000in}}%
\pgfpathcurveto{\pgfqpoint{0.041667in}{0.011050in}}{\pgfqpoint{0.037276in}{0.021649in}}{\pgfqpoint{0.029463in}{0.029463in}}%
\pgfpathcurveto{\pgfqpoint{0.021649in}{0.037276in}}{\pgfqpoint{0.011050in}{0.041667in}}{\pgfqpoint{0.000000in}{0.041667in}}%
\pgfpathcurveto{\pgfqpoint{-0.011050in}{0.041667in}}{\pgfqpoint{-0.021649in}{0.037276in}}{\pgfqpoint{-0.029463in}{0.029463in}}%
\pgfpathcurveto{\pgfqpoint{-0.037276in}{0.021649in}}{\pgfqpoint{-0.041667in}{0.011050in}}{\pgfqpoint{-0.041667in}{0.000000in}}%
\pgfpathcurveto{\pgfqpoint{-0.041667in}{-0.011050in}}{\pgfqpoint{-0.037276in}{-0.021649in}}{\pgfqpoint{-0.029463in}{-0.029463in}}%
\pgfpathcurveto{\pgfqpoint{-0.021649in}{-0.037276in}}{\pgfqpoint{-0.011050in}{-0.041667in}}{\pgfqpoint{0.000000in}{-0.041667in}}%
\pgfpathclose%
\pgfusepath{stroke,fill}%
}%
\end{pgfscope}%
\begin{pgfscope}%
\pgfpathrectangle{\pgfqpoint{0.481978in}{0.331635in}}{\pgfqpoint{4.960000in}{3.696000in}}%
\pgfusepath{clip}%
\pgfsetbuttcap%
\pgfsetroundjoin%
\definecolor{currentfill}{rgb}{0.631373,0.788235,0.956863}%
\pgfsetfillcolor{currentfill}%
\pgfsetlinewidth{1.003750pt}%
\definecolor{currentstroke}{rgb}{0.631373,0.788235,0.956863}%
\pgfsetstrokecolor{currentstroke}%
\pgfsetdash{}{0pt}%
\pgfsys@defobject{currentmarker}{\pgfqpoint{-0.041667in}{-0.041667in}}{\pgfqpoint{0.041667in}{0.041667in}}{%
\pgfpathmoveto{\pgfqpoint{0.000000in}{-0.041667in}}%
\pgfpathcurveto{\pgfqpoint{0.011050in}{-0.041667in}}{\pgfqpoint{0.021649in}{-0.037276in}}{\pgfqpoint{0.029463in}{-0.029463in}}%
\pgfpathcurveto{\pgfqpoint{0.037276in}{-0.021649in}}{\pgfqpoint{0.041667in}{-0.011050in}}{\pgfqpoint{0.041667in}{0.000000in}}%
\pgfpathcurveto{\pgfqpoint{0.041667in}{0.011050in}}{\pgfqpoint{0.037276in}{0.021649in}}{\pgfqpoint{0.029463in}{0.029463in}}%
\pgfpathcurveto{\pgfqpoint{0.021649in}{0.037276in}}{\pgfqpoint{0.011050in}{0.041667in}}{\pgfqpoint{0.000000in}{0.041667in}}%
\pgfpathcurveto{\pgfqpoint{-0.011050in}{0.041667in}}{\pgfqpoint{-0.021649in}{0.037276in}}{\pgfqpoint{-0.029463in}{0.029463in}}%
\pgfpathcurveto{\pgfqpoint{-0.037276in}{0.021649in}}{\pgfqpoint{-0.041667in}{0.011050in}}{\pgfqpoint{-0.041667in}{0.000000in}}%
\pgfpathcurveto{\pgfqpoint{-0.041667in}{-0.011050in}}{\pgfqpoint{-0.037276in}{-0.021649in}}{\pgfqpoint{-0.029463in}{-0.029463in}}%
\pgfpathcurveto{\pgfqpoint{-0.021649in}{-0.037276in}}{\pgfqpoint{-0.011050in}{-0.041667in}}{\pgfqpoint{0.000000in}{-0.041667in}}%
\pgfpathclose%
\pgfusepath{stroke,fill}%
}%
\end{pgfscope}%
\begin{pgfscope}%
\pgfsetbuttcap%
\pgfsetroundjoin%
\definecolor{currentfill}{rgb}{0.000000,0.000000,0.000000}%
\pgfsetfillcolor{currentfill}%
\pgfsetlinewidth{0.803000pt}%
\definecolor{currentstroke}{rgb}{0.000000,0.000000,0.000000}%
\pgfsetstrokecolor{currentstroke}%
\pgfsetdash{}{0pt}%
\pgfsys@defobject{currentmarker}{\pgfqpoint{0.000000in}{-0.048611in}}{\pgfqpoint{0.000000in}{0.000000in}}{%
\pgfpathmoveto{\pgfqpoint{0.000000in}{0.000000in}}%
\pgfpathlineto{\pgfqpoint{0.000000in}{-0.048611in}}%
\pgfusepath{stroke,fill}%
}%
\begin{pgfscope}%
\pgfsys@transformshift{1.215598in}{0.331635in}%
\pgfsys@useobject{currentmarker}{}%
\end{pgfscope}%
\end{pgfscope}%
\begin{pgfscope}%
\definecolor{textcolor}{rgb}{0.000000,0.000000,0.000000}%
\pgfsetstrokecolor{textcolor}%
\pgfsetfillcolor{textcolor}%
\pgftext[x=1.215598in,y=0.234413in,,top]{\color{textcolor}\sffamily\fontsize{10.000000}{12.000000}\selectfont \ensuremath{-}15}%
\end{pgfscope}%
\begin{pgfscope}%
\pgfsetbuttcap%
\pgfsetroundjoin%
\definecolor{currentfill}{rgb}{0.000000,0.000000,0.000000}%
\pgfsetfillcolor{currentfill}%
\pgfsetlinewidth{0.803000pt}%
\definecolor{currentstroke}{rgb}{0.000000,0.000000,0.000000}%
\pgfsetstrokecolor{currentstroke}%
\pgfsetdash{}{0pt}%
\pgfsys@defobject{currentmarker}{\pgfqpoint{0.000000in}{-0.048611in}}{\pgfqpoint{0.000000in}{0.000000in}}{%
\pgfpathmoveto{\pgfqpoint{0.000000in}{0.000000in}}%
\pgfpathlineto{\pgfqpoint{0.000000in}{-0.048611in}}%
\pgfusepath{stroke,fill}%
}%
\begin{pgfscope}%
\pgfsys@transformshift{1.995280in}{0.331635in}%
\pgfsys@useobject{currentmarker}{}%
\end{pgfscope}%
\end{pgfscope}%
\begin{pgfscope}%
\definecolor{textcolor}{rgb}{0.000000,0.000000,0.000000}%
\pgfsetstrokecolor{textcolor}%
\pgfsetfillcolor{textcolor}%
\pgftext[x=1.995280in,y=0.234413in,,top]{\color{textcolor}\sffamily\fontsize{10.000000}{12.000000}\selectfont \ensuremath{-}10}%
\end{pgfscope}%
\begin{pgfscope}%
\pgfsetbuttcap%
\pgfsetroundjoin%
\definecolor{currentfill}{rgb}{0.000000,0.000000,0.000000}%
\pgfsetfillcolor{currentfill}%
\pgfsetlinewidth{0.803000pt}%
\definecolor{currentstroke}{rgb}{0.000000,0.000000,0.000000}%
\pgfsetstrokecolor{currentstroke}%
\pgfsetdash{}{0pt}%
\pgfsys@defobject{currentmarker}{\pgfqpoint{0.000000in}{-0.048611in}}{\pgfqpoint{0.000000in}{0.000000in}}{%
\pgfpathmoveto{\pgfqpoint{0.000000in}{0.000000in}}%
\pgfpathlineto{\pgfqpoint{0.000000in}{-0.048611in}}%
\pgfusepath{stroke,fill}%
}%
\begin{pgfscope}%
\pgfsys@transformshift{2.774961in}{0.331635in}%
\pgfsys@useobject{currentmarker}{}%
\end{pgfscope}%
\end{pgfscope}%
\begin{pgfscope}%
\definecolor{textcolor}{rgb}{0.000000,0.000000,0.000000}%
\pgfsetstrokecolor{textcolor}%
\pgfsetfillcolor{textcolor}%
\pgftext[x=2.774961in,y=0.234413in,,top]{\color{textcolor}\sffamily\fontsize{10.000000}{12.000000}\selectfont \ensuremath{-}5}%
\end{pgfscope}%
\begin{pgfscope}%
\pgfsetbuttcap%
\pgfsetroundjoin%
\definecolor{currentfill}{rgb}{0.000000,0.000000,0.000000}%
\pgfsetfillcolor{currentfill}%
\pgfsetlinewidth{0.803000pt}%
\definecolor{currentstroke}{rgb}{0.000000,0.000000,0.000000}%
\pgfsetstrokecolor{currentstroke}%
\pgfsetdash{}{0pt}%
\pgfsys@defobject{currentmarker}{\pgfqpoint{0.000000in}{-0.048611in}}{\pgfqpoint{0.000000in}{0.000000in}}{%
\pgfpathmoveto{\pgfqpoint{0.000000in}{0.000000in}}%
\pgfpathlineto{\pgfqpoint{0.000000in}{-0.048611in}}%
\pgfusepath{stroke,fill}%
}%
\begin{pgfscope}%
\pgfsys@transformshift{3.554643in}{0.331635in}%
\pgfsys@useobject{currentmarker}{}%
\end{pgfscope}%
\end{pgfscope}%
\begin{pgfscope}%
\definecolor{textcolor}{rgb}{0.000000,0.000000,0.000000}%
\pgfsetstrokecolor{textcolor}%
\pgfsetfillcolor{textcolor}%
\pgftext[x=3.554643in,y=0.234413in,,top]{\color{textcolor}\sffamily\fontsize{10.000000}{12.000000}\selectfont 0}%
\end{pgfscope}%
\begin{pgfscope}%
\pgfsetbuttcap%
\pgfsetroundjoin%
\definecolor{currentfill}{rgb}{0.000000,0.000000,0.000000}%
\pgfsetfillcolor{currentfill}%
\pgfsetlinewidth{0.803000pt}%
\definecolor{currentstroke}{rgb}{0.000000,0.000000,0.000000}%
\pgfsetstrokecolor{currentstroke}%
\pgfsetdash{}{0pt}%
\pgfsys@defobject{currentmarker}{\pgfqpoint{0.000000in}{-0.048611in}}{\pgfqpoint{0.000000in}{0.000000in}}{%
\pgfpathmoveto{\pgfqpoint{0.000000in}{0.000000in}}%
\pgfpathlineto{\pgfqpoint{0.000000in}{-0.048611in}}%
\pgfusepath{stroke,fill}%
}%
\begin{pgfscope}%
\pgfsys@transformshift{4.334324in}{0.331635in}%
\pgfsys@useobject{currentmarker}{}%
\end{pgfscope}%
\end{pgfscope}%
\begin{pgfscope}%
\definecolor{textcolor}{rgb}{0.000000,0.000000,0.000000}%
\pgfsetstrokecolor{textcolor}%
\pgfsetfillcolor{textcolor}%
\pgftext[x=4.334324in,y=0.234413in,,top]{\color{textcolor}\sffamily\fontsize{10.000000}{12.000000}\selectfont 5}%
\end{pgfscope}%
\begin{pgfscope}%
\pgfsetbuttcap%
\pgfsetroundjoin%
\definecolor{currentfill}{rgb}{0.000000,0.000000,0.000000}%
\pgfsetfillcolor{currentfill}%
\pgfsetlinewidth{0.803000pt}%
\definecolor{currentstroke}{rgb}{0.000000,0.000000,0.000000}%
\pgfsetstrokecolor{currentstroke}%
\pgfsetdash{}{0pt}%
\pgfsys@defobject{currentmarker}{\pgfqpoint{0.000000in}{-0.048611in}}{\pgfqpoint{0.000000in}{0.000000in}}{%
\pgfpathmoveto{\pgfqpoint{0.000000in}{0.000000in}}%
\pgfpathlineto{\pgfqpoint{0.000000in}{-0.048611in}}%
\pgfusepath{stroke,fill}%
}%
\begin{pgfscope}%
\pgfsys@transformshift{5.114005in}{0.331635in}%
\pgfsys@useobject{currentmarker}{}%
\end{pgfscope}%
\end{pgfscope}%
\begin{pgfscope}%
\definecolor{textcolor}{rgb}{0.000000,0.000000,0.000000}%
\pgfsetstrokecolor{textcolor}%
\pgfsetfillcolor{textcolor}%
\pgftext[x=5.114005in,y=0.234413in,,top]{\color{textcolor}\sffamily\fontsize{10.000000}{12.000000}\selectfont 10}%
\end{pgfscope}%
\begin{pgfscope}%
\pgfsetbuttcap%
\pgfsetroundjoin%
\definecolor{currentfill}{rgb}{0.000000,0.000000,0.000000}%
\pgfsetfillcolor{currentfill}%
\pgfsetlinewidth{0.803000pt}%
\definecolor{currentstroke}{rgb}{0.000000,0.000000,0.000000}%
\pgfsetstrokecolor{currentstroke}%
\pgfsetdash{}{0pt}%
\pgfsys@defobject{currentmarker}{\pgfqpoint{-0.048611in}{0.000000in}}{\pgfqpoint{-0.000000in}{0.000000in}}{%
\pgfpathmoveto{\pgfqpoint{-0.000000in}{0.000000in}}%
\pgfpathlineto{\pgfqpoint{-0.048611in}{0.000000in}}%
\pgfusepath{stroke,fill}%
}%
\begin{pgfscope}%
\pgfsys@transformshift{0.481978in}{0.632347in}%
\pgfsys@useobject{currentmarker}{}%
\end{pgfscope}%
\end{pgfscope}%
\begin{pgfscope}%
\definecolor{textcolor}{rgb}{0.000000,0.000000,0.000000}%
\pgfsetstrokecolor{textcolor}%
\pgfsetfillcolor{textcolor}%
\pgftext[x=0.100000in, y=0.579585in, left, base]{\color{textcolor}\sffamily\fontsize{10.000000}{12.000000}\selectfont \ensuremath{-}20}%
\end{pgfscope}%
\begin{pgfscope}%
\pgfsetbuttcap%
\pgfsetroundjoin%
\definecolor{currentfill}{rgb}{0.000000,0.000000,0.000000}%
\pgfsetfillcolor{currentfill}%
\pgfsetlinewidth{0.803000pt}%
\definecolor{currentstroke}{rgb}{0.000000,0.000000,0.000000}%
\pgfsetstrokecolor{currentstroke}%
\pgfsetdash{}{0pt}%
\pgfsys@defobject{currentmarker}{\pgfqpoint{-0.048611in}{0.000000in}}{\pgfqpoint{-0.000000in}{0.000000in}}{%
\pgfpathmoveto{\pgfqpoint{-0.000000in}{0.000000in}}%
\pgfpathlineto{\pgfqpoint{-0.048611in}{0.000000in}}%
\pgfusepath{stroke,fill}%
}%
\begin{pgfscope}%
\pgfsys@transformshift{0.481978in}{1.049259in}%
\pgfsys@useobject{currentmarker}{}%
\end{pgfscope}%
\end{pgfscope}%
\begin{pgfscope}%
\definecolor{textcolor}{rgb}{0.000000,0.000000,0.000000}%
\pgfsetstrokecolor{textcolor}%
\pgfsetfillcolor{textcolor}%
\pgftext[x=0.100000in, y=0.996498in, left, base]{\color{textcolor}\sffamily\fontsize{10.000000}{12.000000}\selectfont \ensuremath{-}15}%
\end{pgfscope}%
\begin{pgfscope}%
\pgfsetbuttcap%
\pgfsetroundjoin%
\definecolor{currentfill}{rgb}{0.000000,0.000000,0.000000}%
\pgfsetfillcolor{currentfill}%
\pgfsetlinewidth{0.803000pt}%
\definecolor{currentstroke}{rgb}{0.000000,0.000000,0.000000}%
\pgfsetstrokecolor{currentstroke}%
\pgfsetdash{}{0pt}%
\pgfsys@defobject{currentmarker}{\pgfqpoint{-0.048611in}{0.000000in}}{\pgfqpoint{-0.000000in}{0.000000in}}{%
\pgfpathmoveto{\pgfqpoint{-0.000000in}{0.000000in}}%
\pgfpathlineto{\pgfqpoint{-0.048611in}{0.000000in}}%
\pgfusepath{stroke,fill}%
}%
\begin{pgfscope}%
\pgfsys@transformshift{0.481978in}{1.466172in}%
\pgfsys@useobject{currentmarker}{}%
\end{pgfscope}%
\end{pgfscope}%
\begin{pgfscope}%
\definecolor{textcolor}{rgb}{0.000000,0.000000,0.000000}%
\pgfsetstrokecolor{textcolor}%
\pgfsetfillcolor{textcolor}%
\pgftext[x=0.100000in, y=1.413410in, left, base]{\color{textcolor}\sffamily\fontsize{10.000000}{12.000000}\selectfont \ensuremath{-}10}%
\end{pgfscope}%
\begin{pgfscope}%
\pgfsetbuttcap%
\pgfsetroundjoin%
\definecolor{currentfill}{rgb}{0.000000,0.000000,0.000000}%
\pgfsetfillcolor{currentfill}%
\pgfsetlinewidth{0.803000pt}%
\definecolor{currentstroke}{rgb}{0.000000,0.000000,0.000000}%
\pgfsetstrokecolor{currentstroke}%
\pgfsetdash{}{0pt}%
\pgfsys@defobject{currentmarker}{\pgfqpoint{-0.048611in}{0.000000in}}{\pgfqpoint{-0.000000in}{0.000000in}}{%
\pgfpathmoveto{\pgfqpoint{-0.000000in}{0.000000in}}%
\pgfpathlineto{\pgfqpoint{-0.048611in}{0.000000in}}%
\pgfusepath{stroke,fill}%
}%
\begin{pgfscope}%
\pgfsys@transformshift{0.481978in}{1.883084in}%
\pgfsys@useobject{currentmarker}{}%
\end{pgfscope}%
\end{pgfscope}%
\begin{pgfscope}%
\definecolor{textcolor}{rgb}{0.000000,0.000000,0.000000}%
\pgfsetstrokecolor{textcolor}%
\pgfsetfillcolor{textcolor}%
\pgftext[x=0.188365in, y=1.830323in, left, base]{\color{textcolor}\sffamily\fontsize{10.000000}{12.000000}\selectfont \ensuremath{-}5}%
\end{pgfscope}%
\begin{pgfscope}%
\pgfsetbuttcap%
\pgfsetroundjoin%
\definecolor{currentfill}{rgb}{0.000000,0.000000,0.000000}%
\pgfsetfillcolor{currentfill}%
\pgfsetlinewidth{0.803000pt}%
\definecolor{currentstroke}{rgb}{0.000000,0.000000,0.000000}%
\pgfsetstrokecolor{currentstroke}%
\pgfsetdash{}{0pt}%
\pgfsys@defobject{currentmarker}{\pgfqpoint{-0.048611in}{0.000000in}}{\pgfqpoint{-0.000000in}{0.000000in}}{%
\pgfpathmoveto{\pgfqpoint{-0.000000in}{0.000000in}}%
\pgfpathlineto{\pgfqpoint{-0.048611in}{0.000000in}}%
\pgfusepath{stroke,fill}%
}%
\begin{pgfscope}%
\pgfsys@transformshift{0.481978in}{2.299997in}%
\pgfsys@useobject{currentmarker}{}%
\end{pgfscope}%
\end{pgfscope}%
\begin{pgfscope}%
\definecolor{textcolor}{rgb}{0.000000,0.000000,0.000000}%
\pgfsetstrokecolor{textcolor}%
\pgfsetfillcolor{textcolor}%
\pgftext[x=0.296390in, y=2.247235in, left, base]{\color{textcolor}\sffamily\fontsize{10.000000}{12.000000}\selectfont 0}%
\end{pgfscope}%
\begin{pgfscope}%
\pgfsetbuttcap%
\pgfsetroundjoin%
\definecolor{currentfill}{rgb}{0.000000,0.000000,0.000000}%
\pgfsetfillcolor{currentfill}%
\pgfsetlinewidth{0.803000pt}%
\definecolor{currentstroke}{rgb}{0.000000,0.000000,0.000000}%
\pgfsetstrokecolor{currentstroke}%
\pgfsetdash{}{0pt}%
\pgfsys@defobject{currentmarker}{\pgfqpoint{-0.048611in}{0.000000in}}{\pgfqpoint{-0.000000in}{0.000000in}}{%
\pgfpathmoveto{\pgfqpoint{-0.000000in}{0.000000in}}%
\pgfpathlineto{\pgfqpoint{-0.048611in}{0.000000in}}%
\pgfusepath{stroke,fill}%
}%
\begin{pgfscope}%
\pgfsys@transformshift{0.481978in}{2.716910in}%
\pgfsys@useobject{currentmarker}{}%
\end{pgfscope}%
\end{pgfscope}%
\begin{pgfscope}%
\definecolor{textcolor}{rgb}{0.000000,0.000000,0.000000}%
\pgfsetstrokecolor{textcolor}%
\pgfsetfillcolor{textcolor}%
\pgftext[x=0.296390in, y=2.664148in, left, base]{\color{textcolor}\sffamily\fontsize{10.000000}{12.000000}\selectfont 5}%
\end{pgfscope}%
\begin{pgfscope}%
\pgfsetbuttcap%
\pgfsetroundjoin%
\definecolor{currentfill}{rgb}{0.000000,0.000000,0.000000}%
\pgfsetfillcolor{currentfill}%
\pgfsetlinewidth{0.803000pt}%
\definecolor{currentstroke}{rgb}{0.000000,0.000000,0.000000}%
\pgfsetstrokecolor{currentstroke}%
\pgfsetdash{}{0pt}%
\pgfsys@defobject{currentmarker}{\pgfqpoint{-0.048611in}{0.000000in}}{\pgfqpoint{-0.000000in}{0.000000in}}{%
\pgfpathmoveto{\pgfqpoint{-0.000000in}{0.000000in}}%
\pgfpathlineto{\pgfqpoint{-0.048611in}{0.000000in}}%
\pgfusepath{stroke,fill}%
}%
\begin{pgfscope}%
\pgfsys@transformshift{0.481978in}{3.133822in}%
\pgfsys@useobject{currentmarker}{}%
\end{pgfscope}%
\end{pgfscope}%
\begin{pgfscope}%
\definecolor{textcolor}{rgb}{0.000000,0.000000,0.000000}%
\pgfsetstrokecolor{textcolor}%
\pgfsetfillcolor{textcolor}%
\pgftext[x=0.208025in, y=3.081061in, left, base]{\color{textcolor}\sffamily\fontsize{10.000000}{12.000000}\selectfont 10}%
\end{pgfscope}%
\begin{pgfscope}%
\pgfsetbuttcap%
\pgfsetroundjoin%
\definecolor{currentfill}{rgb}{0.000000,0.000000,0.000000}%
\pgfsetfillcolor{currentfill}%
\pgfsetlinewidth{0.803000pt}%
\definecolor{currentstroke}{rgb}{0.000000,0.000000,0.000000}%
\pgfsetstrokecolor{currentstroke}%
\pgfsetdash{}{0pt}%
\pgfsys@defobject{currentmarker}{\pgfqpoint{-0.048611in}{0.000000in}}{\pgfqpoint{-0.000000in}{0.000000in}}{%
\pgfpathmoveto{\pgfqpoint{-0.000000in}{0.000000in}}%
\pgfpathlineto{\pgfqpoint{-0.048611in}{0.000000in}}%
\pgfusepath{stroke,fill}%
}%
\begin{pgfscope}%
\pgfsys@transformshift{0.481978in}{3.550735in}%
\pgfsys@useobject{currentmarker}{}%
\end{pgfscope}%
\end{pgfscope}%
\begin{pgfscope}%
\definecolor{textcolor}{rgb}{0.000000,0.000000,0.000000}%
\pgfsetstrokecolor{textcolor}%
\pgfsetfillcolor{textcolor}%
\pgftext[x=0.208025in, y=3.497973in, left, base]{\color{textcolor}\sffamily\fontsize{10.000000}{12.000000}\selectfont 15}%
\end{pgfscope}%
\begin{pgfscope}%
\pgfsetbuttcap%
\pgfsetroundjoin%
\definecolor{currentfill}{rgb}{0.000000,0.000000,0.000000}%
\pgfsetfillcolor{currentfill}%
\pgfsetlinewidth{0.803000pt}%
\definecolor{currentstroke}{rgb}{0.000000,0.000000,0.000000}%
\pgfsetstrokecolor{currentstroke}%
\pgfsetdash{}{0pt}%
\pgfsys@defobject{currentmarker}{\pgfqpoint{-0.048611in}{0.000000in}}{\pgfqpoint{-0.000000in}{0.000000in}}{%
\pgfpathmoveto{\pgfqpoint{-0.000000in}{0.000000in}}%
\pgfpathlineto{\pgfqpoint{-0.048611in}{0.000000in}}%
\pgfusepath{stroke,fill}%
}%
\begin{pgfscope}%
\pgfsys@transformshift{0.481978in}{3.967647in}%
\pgfsys@useobject{currentmarker}{}%
\end{pgfscope}%
\end{pgfscope}%
\begin{pgfscope}%
\definecolor{textcolor}{rgb}{0.000000,0.000000,0.000000}%
\pgfsetstrokecolor{textcolor}%
\pgfsetfillcolor{textcolor}%
\pgftext[x=0.208025in, y=3.914886in, left, base]{\color{textcolor}\sffamily\fontsize{10.000000}{12.000000}\selectfont 20}%
\end{pgfscope}%
\begin{pgfscope}%
\pgfpathrectangle{\pgfqpoint{0.481978in}{0.331635in}}{\pgfqpoint{4.960000in}{3.696000in}}%
\pgfusepath{clip}%
\pgfsetrectcap%
\pgfsetroundjoin%
\pgfsetlinewidth{1.505625pt}%
\definecolor{currentstroke}{rgb}{1.000000,0.705882,0.509804}%
\pgfsetstrokecolor{currentstroke}%
\pgfsetstrokeopacity{0.800000}%
\pgfsetdash{}{0pt}%
\pgfpathmoveto{\pgfqpoint{0.707432in}{1.111749in}}%
\pgfpathlineto{\pgfqpoint{2.715488in}{1.479455in}}%
\pgfusepath{stroke}%
\end{pgfscope}%
\begin{pgfscope}%
\pgfpathrectangle{\pgfqpoint{0.481978in}{0.331635in}}{\pgfqpoint{4.960000in}{3.696000in}}%
\pgfusepath{clip}%
\pgfsetrectcap%
\pgfsetroundjoin%
\pgfsetlinewidth{1.505625pt}%
\definecolor{currentstroke}{rgb}{1.000000,0.705882,0.509804}%
\pgfsetstrokecolor{currentstroke}%
\pgfsetstrokeopacity{0.800000}%
\pgfsetdash{}{0pt}%
\pgfpathmoveto{\pgfqpoint{4.098791in}{1.335727in}}%
\pgfpathlineto{\pgfqpoint{2.715488in}{1.479455in}}%
\pgfusepath{stroke}%
\end{pgfscope}%
\begin{pgfscope}%
\pgfpathrectangle{\pgfqpoint{0.481978in}{0.331635in}}{\pgfqpoint{4.960000in}{3.696000in}}%
\pgfusepath{clip}%
\pgfsetrectcap%
\pgfsetroundjoin%
\pgfsetlinewidth{1.505625pt}%
\definecolor{currentstroke}{rgb}{1.000000,0.705882,0.509804}%
\pgfsetstrokecolor{currentstroke}%
\pgfsetstrokeopacity{0.800000}%
\pgfsetdash{}{0pt}%
\pgfpathmoveto{\pgfqpoint{4.100778in}{1.341628in}}%
\pgfpathlineto{\pgfqpoint{2.715488in}{1.479455in}}%
\pgfusepath{stroke}%
\end{pgfscope}%
\begin{pgfscope}%
\pgfpathrectangle{\pgfqpoint{0.481978in}{0.331635in}}{\pgfqpoint{4.960000in}{3.696000in}}%
\pgfusepath{clip}%
\pgfsetrectcap%
\pgfsetroundjoin%
\pgfsetlinewidth{1.505625pt}%
\definecolor{currentstroke}{rgb}{1.000000,0.705882,0.509804}%
\pgfsetstrokecolor{currentstroke}%
\pgfsetstrokeopacity{0.800000}%
\pgfsetdash{}{0pt}%
\pgfpathmoveto{\pgfqpoint{1.950259in}{2.315088in}}%
\pgfpathlineto{\pgfqpoint{2.715488in}{1.479455in}}%
\pgfusepath{stroke}%
\end{pgfscope}%
\begin{pgfscope}%
\pgfpathrectangle{\pgfqpoint{0.481978in}{0.331635in}}{\pgfqpoint{4.960000in}{3.696000in}}%
\pgfusepath{clip}%
\pgfsetrectcap%
\pgfsetroundjoin%
\pgfsetlinewidth{1.505625pt}%
\definecolor{currentstroke}{rgb}{1.000000,0.705882,0.509804}%
\pgfsetstrokecolor{currentstroke}%
\pgfsetstrokeopacity{0.800000}%
\pgfsetdash{}{0pt}%
\pgfpathmoveto{\pgfqpoint{3.271542in}{1.288822in}}%
\pgfpathlineto{\pgfqpoint{2.715488in}{1.479455in}}%
\pgfusepath{stroke}%
\end{pgfscope}%
\begin{pgfscope}%
\pgfpathrectangle{\pgfqpoint{0.481978in}{0.331635in}}{\pgfqpoint{4.960000in}{3.696000in}}%
\pgfusepath{clip}%
\pgfsetrectcap%
\pgfsetroundjoin%
\pgfsetlinewidth{1.505625pt}%
\definecolor{currentstroke}{rgb}{1.000000,0.705882,0.509804}%
\pgfsetstrokecolor{currentstroke}%
\pgfsetstrokeopacity{0.800000}%
\pgfsetdash{}{0pt}%
\pgfpathmoveto{\pgfqpoint{3.075117in}{1.629785in}}%
\pgfpathlineto{\pgfqpoint{2.715488in}{1.479455in}}%
\pgfusepath{stroke}%
\end{pgfscope}%
\begin{pgfscope}%
\pgfpathrectangle{\pgfqpoint{0.481978in}{0.331635in}}{\pgfqpoint{4.960000in}{3.696000in}}%
\pgfusepath{clip}%
\pgfsetrectcap%
\pgfsetroundjoin%
\pgfsetlinewidth{1.505625pt}%
\definecolor{currentstroke}{rgb}{1.000000,0.705882,0.509804}%
\pgfsetstrokecolor{currentstroke}%
\pgfsetstrokeopacity{0.800000}%
\pgfsetdash{}{0pt}%
\pgfpathmoveto{\pgfqpoint{2.808150in}{1.670951in}}%
\pgfpathlineto{\pgfqpoint{2.715488in}{1.479455in}}%
\pgfusepath{stroke}%
\end{pgfscope}%
\begin{pgfscope}%
\pgfpathrectangle{\pgfqpoint{0.481978in}{0.331635in}}{\pgfqpoint{4.960000in}{3.696000in}}%
\pgfusepath{clip}%
\pgfsetrectcap%
\pgfsetroundjoin%
\pgfsetlinewidth{1.505625pt}%
\definecolor{currentstroke}{rgb}{1.000000,0.705882,0.509804}%
\pgfsetstrokecolor{currentstroke}%
\pgfsetstrokeopacity{0.800000}%
\pgfsetdash{}{0pt}%
\pgfpathmoveto{\pgfqpoint{2.960335in}{1.638201in}}%
\pgfpathlineto{\pgfqpoint{2.715488in}{1.479455in}}%
\pgfusepath{stroke}%
\end{pgfscope}%
\begin{pgfscope}%
\pgfpathrectangle{\pgfqpoint{0.481978in}{0.331635in}}{\pgfqpoint{4.960000in}{3.696000in}}%
\pgfusepath{clip}%
\pgfsetrectcap%
\pgfsetroundjoin%
\pgfsetlinewidth{1.505625pt}%
\definecolor{currentstroke}{rgb}{1.000000,0.705882,0.509804}%
\pgfsetstrokecolor{currentstroke}%
\pgfsetstrokeopacity{0.800000}%
\pgfsetdash{}{0pt}%
\pgfpathmoveto{\pgfqpoint{4.016960in}{0.810061in}}%
\pgfpathlineto{\pgfqpoint{2.715488in}{1.479455in}}%
\pgfusepath{stroke}%
\end{pgfscope}%
\begin{pgfscope}%
\pgfpathrectangle{\pgfqpoint{0.481978in}{0.331635in}}{\pgfqpoint{4.960000in}{3.696000in}}%
\pgfusepath{clip}%
\pgfsetrectcap%
\pgfsetroundjoin%
\pgfsetlinewidth{1.505625pt}%
\definecolor{currentstroke}{rgb}{1.000000,0.705882,0.509804}%
\pgfsetstrokecolor{currentstroke}%
\pgfsetstrokeopacity{0.800000}%
\pgfsetdash{}{0pt}%
\pgfpathmoveto{\pgfqpoint{3.729020in}{2.030607in}}%
\pgfpathlineto{\pgfqpoint{2.715488in}{1.479455in}}%
\pgfusepath{stroke}%
\end{pgfscope}%
\begin{pgfscope}%
\pgfpathrectangle{\pgfqpoint{0.481978in}{0.331635in}}{\pgfqpoint{4.960000in}{3.696000in}}%
\pgfusepath{clip}%
\pgfsetrectcap%
\pgfsetroundjoin%
\pgfsetlinewidth{1.505625pt}%
\definecolor{currentstroke}{rgb}{1.000000,0.705882,0.509804}%
\pgfsetstrokecolor{currentstroke}%
\pgfsetstrokeopacity{0.800000}%
\pgfsetdash{}{0pt}%
\pgfpathmoveto{\pgfqpoint{1.438757in}{1.811477in}}%
\pgfpathlineto{\pgfqpoint{2.715488in}{1.479455in}}%
\pgfusepath{stroke}%
\end{pgfscope}%
\begin{pgfscope}%
\pgfpathrectangle{\pgfqpoint{0.481978in}{0.331635in}}{\pgfqpoint{4.960000in}{3.696000in}}%
\pgfusepath{clip}%
\pgfsetrectcap%
\pgfsetroundjoin%
\pgfsetlinewidth{1.505625pt}%
\definecolor{currentstroke}{rgb}{1.000000,0.705882,0.509804}%
\pgfsetstrokecolor{currentstroke}%
\pgfsetstrokeopacity{0.800000}%
\pgfsetdash{}{0pt}%
\pgfpathmoveto{\pgfqpoint{1.871850in}{1.086595in}}%
\pgfpathlineto{\pgfqpoint{2.715488in}{1.479455in}}%
\pgfusepath{stroke}%
\end{pgfscope}%
\begin{pgfscope}%
\pgfpathrectangle{\pgfqpoint{0.481978in}{0.331635in}}{\pgfqpoint{4.960000in}{3.696000in}}%
\pgfusepath{clip}%
\pgfsetrectcap%
\pgfsetroundjoin%
\pgfsetlinewidth{1.505625pt}%
\definecolor{currentstroke}{rgb}{1.000000,0.705882,0.509804}%
\pgfsetstrokecolor{currentstroke}%
\pgfsetstrokeopacity{0.800000}%
\pgfsetdash{}{0pt}%
\pgfpathmoveto{\pgfqpoint{3.268642in}{1.037099in}}%
\pgfpathlineto{\pgfqpoint{2.715488in}{1.479455in}}%
\pgfusepath{stroke}%
\end{pgfscope}%
\begin{pgfscope}%
\pgfpathrectangle{\pgfqpoint{0.481978in}{0.331635in}}{\pgfqpoint{4.960000in}{3.696000in}}%
\pgfusepath{clip}%
\pgfsetrectcap%
\pgfsetroundjoin%
\pgfsetlinewidth{1.505625pt}%
\definecolor{currentstroke}{rgb}{1.000000,0.705882,0.509804}%
\pgfsetstrokecolor{currentstroke}%
\pgfsetstrokeopacity{0.800000}%
\pgfsetdash{}{0pt}%
\pgfpathmoveto{\pgfqpoint{1.604093in}{2.420626in}}%
\pgfpathlineto{\pgfqpoint{2.715488in}{1.479455in}}%
\pgfusepath{stroke}%
\end{pgfscope}%
\begin{pgfscope}%
\pgfpathrectangle{\pgfqpoint{0.481978in}{0.331635in}}{\pgfqpoint{4.960000in}{3.696000in}}%
\pgfusepath{clip}%
\pgfsetrectcap%
\pgfsetroundjoin%
\pgfsetlinewidth{1.505625pt}%
\definecolor{currentstroke}{rgb}{1.000000,0.705882,0.509804}%
\pgfsetstrokecolor{currentstroke}%
\pgfsetstrokeopacity{0.800000}%
\pgfsetdash{}{0pt}%
\pgfpathmoveto{\pgfqpoint{2.189596in}{2.402786in}}%
\pgfpathlineto{\pgfqpoint{2.715488in}{1.479455in}}%
\pgfusepath{stroke}%
\end{pgfscope}%
\begin{pgfscope}%
\pgfpathrectangle{\pgfqpoint{0.481978in}{0.331635in}}{\pgfqpoint{4.960000in}{3.696000in}}%
\pgfusepath{clip}%
\pgfsetrectcap%
\pgfsetroundjoin%
\pgfsetlinewidth{1.505625pt}%
\definecolor{currentstroke}{rgb}{1.000000,0.705882,0.509804}%
\pgfsetstrokecolor{currentstroke}%
\pgfsetstrokeopacity{0.800000}%
\pgfsetdash{}{0pt}%
\pgfpathmoveto{\pgfqpoint{2.992675in}{1.030576in}}%
\pgfpathlineto{\pgfqpoint{2.715488in}{1.479455in}}%
\pgfusepath{stroke}%
\end{pgfscope}%
\begin{pgfscope}%
\pgfpathrectangle{\pgfqpoint{0.481978in}{0.331635in}}{\pgfqpoint{4.960000in}{3.696000in}}%
\pgfusepath{clip}%
\pgfsetrectcap%
\pgfsetroundjoin%
\pgfsetlinewidth{1.505625pt}%
\definecolor{currentstroke}{rgb}{1.000000,0.705882,0.509804}%
\pgfsetstrokecolor{currentstroke}%
\pgfsetstrokeopacity{0.800000}%
\pgfsetdash{}{0pt}%
\pgfpathmoveto{\pgfqpoint{3.959148in}{1.530657in}}%
\pgfpathlineto{\pgfqpoint{2.715488in}{1.479455in}}%
\pgfusepath{stroke}%
\end{pgfscope}%
\begin{pgfscope}%
\pgfpathrectangle{\pgfqpoint{0.481978in}{0.331635in}}{\pgfqpoint{4.960000in}{3.696000in}}%
\pgfusepath{clip}%
\pgfsetrectcap%
\pgfsetroundjoin%
\pgfsetlinewidth{1.505625pt}%
\definecolor{currentstroke}{rgb}{1.000000,0.705882,0.509804}%
\pgfsetstrokecolor{currentstroke}%
\pgfsetstrokeopacity{0.800000}%
\pgfsetdash{}{0pt}%
\pgfpathmoveto{\pgfqpoint{0.949934in}{1.426706in}}%
\pgfpathlineto{\pgfqpoint{2.715488in}{1.479455in}}%
\pgfusepath{stroke}%
\end{pgfscope}%
\begin{pgfscope}%
\pgfpathrectangle{\pgfqpoint{0.481978in}{0.331635in}}{\pgfqpoint{4.960000in}{3.696000in}}%
\pgfusepath{clip}%
\pgfsetrectcap%
\pgfsetroundjoin%
\pgfsetlinewidth{1.505625pt}%
\definecolor{currentstroke}{rgb}{1.000000,0.705882,0.509804}%
\pgfsetstrokecolor{currentstroke}%
\pgfsetstrokeopacity{0.800000}%
\pgfsetdash{}{0pt}%
\pgfpathmoveto{\pgfqpoint{2.969152in}{1.147493in}}%
\pgfpathlineto{\pgfqpoint{2.715488in}{1.479455in}}%
\pgfusepath{stroke}%
\end{pgfscope}%
\begin{pgfscope}%
\pgfpathrectangle{\pgfqpoint{0.481978in}{0.331635in}}{\pgfqpoint{4.960000in}{3.696000in}}%
\pgfusepath{clip}%
\pgfsetrectcap%
\pgfsetroundjoin%
\pgfsetlinewidth{1.505625pt}%
\definecolor{currentstroke}{rgb}{1.000000,0.705882,0.509804}%
\pgfsetstrokecolor{currentstroke}%
\pgfsetstrokeopacity{0.800000}%
\pgfsetdash{}{0pt}%
\pgfpathmoveto{\pgfqpoint{3.965536in}{1.908696in}}%
\pgfpathlineto{\pgfqpoint{2.715488in}{1.479455in}}%
\pgfusepath{stroke}%
\end{pgfscope}%
\begin{pgfscope}%
\pgfpathrectangle{\pgfqpoint{0.481978in}{0.331635in}}{\pgfqpoint{4.960000in}{3.696000in}}%
\pgfusepath{clip}%
\pgfsetrectcap%
\pgfsetroundjoin%
\pgfsetlinewidth{1.505625pt}%
\definecolor{currentstroke}{rgb}{1.000000,0.705882,0.509804}%
\pgfsetstrokecolor{currentstroke}%
\pgfsetstrokeopacity{0.800000}%
\pgfsetdash{}{0pt}%
\pgfpathmoveto{\pgfqpoint{2.841011in}{1.194057in}}%
\pgfpathlineto{\pgfqpoint{2.715488in}{1.479455in}}%
\pgfusepath{stroke}%
\end{pgfscope}%
\begin{pgfscope}%
\pgfpathrectangle{\pgfqpoint{0.481978in}{0.331635in}}{\pgfqpoint{4.960000in}{3.696000in}}%
\pgfusepath{clip}%
\pgfsetrectcap%
\pgfsetroundjoin%
\pgfsetlinewidth{1.505625pt}%
\definecolor{currentstroke}{rgb}{1.000000,0.705882,0.509804}%
\pgfsetstrokecolor{currentstroke}%
\pgfsetstrokeopacity{0.800000}%
\pgfsetdash{}{0pt}%
\pgfpathmoveto{\pgfqpoint{2.627352in}{1.678853in}}%
\pgfpathlineto{\pgfqpoint{2.715488in}{1.479455in}}%
\pgfusepath{stroke}%
\end{pgfscope}%
\begin{pgfscope}%
\pgfpathrectangle{\pgfqpoint{0.481978in}{0.331635in}}{\pgfqpoint{4.960000in}{3.696000in}}%
\pgfusepath{clip}%
\pgfsetrectcap%
\pgfsetroundjoin%
\pgfsetlinewidth{1.505625pt}%
\definecolor{currentstroke}{rgb}{1.000000,0.705882,0.509804}%
\pgfsetstrokecolor{currentstroke}%
\pgfsetstrokeopacity{0.800000}%
\pgfsetdash{}{0pt}%
\pgfpathmoveto{\pgfqpoint{2.924999in}{1.796798in}}%
\pgfpathlineto{\pgfqpoint{2.715488in}{1.479455in}}%
\pgfusepath{stroke}%
\end{pgfscope}%
\begin{pgfscope}%
\pgfpathrectangle{\pgfqpoint{0.481978in}{0.331635in}}{\pgfqpoint{4.960000in}{3.696000in}}%
\pgfusepath{clip}%
\pgfsetrectcap%
\pgfsetroundjoin%
\pgfsetlinewidth{1.505625pt}%
\definecolor{currentstroke}{rgb}{1.000000,0.705882,0.509804}%
\pgfsetstrokecolor{currentstroke}%
\pgfsetstrokeopacity{0.800000}%
\pgfsetdash{}{0pt}%
\pgfpathmoveto{\pgfqpoint{3.232405in}{2.234856in}}%
\pgfpathlineto{\pgfqpoint{2.715488in}{1.479455in}}%
\pgfusepath{stroke}%
\end{pgfscope}%
\begin{pgfscope}%
\pgfpathrectangle{\pgfqpoint{0.481978in}{0.331635in}}{\pgfqpoint{4.960000in}{3.696000in}}%
\pgfusepath{clip}%
\pgfsetrectcap%
\pgfsetroundjoin%
\pgfsetlinewidth{1.505625pt}%
\definecolor{currentstroke}{rgb}{1.000000,0.705882,0.509804}%
\pgfsetstrokecolor{currentstroke}%
\pgfsetstrokeopacity{0.800000}%
\pgfsetdash{}{0pt}%
\pgfpathmoveto{\pgfqpoint{4.169916in}{1.090804in}}%
\pgfpathlineto{\pgfqpoint{2.715488in}{1.479455in}}%
\pgfusepath{stroke}%
\end{pgfscope}%
\begin{pgfscope}%
\pgfpathrectangle{\pgfqpoint{0.481978in}{0.331635in}}{\pgfqpoint{4.960000in}{3.696000in}}%
\pgfusepath{clip}%
\pgfsetrectcap%
\pgfsetroundjoin%
\pgfsetlinewidth{1.505625pt}%
\definecolor{currentstroke}{rgb}{1.000000,0.705882,0.509804}%
\pgfsetstrokecolor{currentstroke}%
\pgfsetstrokeopacity{0.800000}%
\pgfsetdash{}{0pt}%
\pgfpathmoveto{\pgfqpoint{1.353814in}{1.790869in}}%
\pgfpathlineto{\pgfqpoint{2.715488in}{1.479455in}}%
\pgfusepath{stroke}%
\end{pgfscope}%
\begin{pgfscope}%
\pgfpathrectangle{\pgfqpoint{0.481978in}{0.331635in}}{\pgfqpoint{4.960000in}{3.696000in}}%
\pgfusepath{clip}%
\pgfsetrectcap%
\pgfsetroundjoin%
\pgfsetlinewidth{1.505625pt}%
\definecolor{currentstroke}{rgb}{1.000000,0.705882,0.509804}%
\pgfsetstrokecolor{currentstroke}%
\pgfsetstrokeopacity{0.800000}%
\pgfsetdash{}{0pt}%
\pgfpathmoveto{\pgfqpoint{2.797827in}{0.749285in}}%
\pgfpathlineto{\pgfqpoint{2.715488in}{1.479455in}}%
\pgfusepath{stroke}%
\end{pgfscope}%
\begin{pgfscope}%
\pgfpathrectangle{\pgfqpoint{0.481978in}{0.331635in}}{\pgfqpoint{4.960000in}{3.696000in}}%
\pgfusepath{clip}%
\pgfsetrectcap%
\pgfsetroundjoin%
\pgfsetlinewidth{1.505625pt}%
\definecolor{currentstroke}{rgb}{1.000000,0.705882,0.509804}%
\pgfsetstrokecolor{currentstroke}%
\pgfsetstrokeopacity{0.800000}%
\pgfsetdash{}{0pt}%
\pgfpathmoveto{\pgfqpoint{4.034424in}{0.811275in}}%
\pgfpathlineto{\pgfqpoint{2.715488in}{1.479455in}}%
\pgfusepath{stroke}%
\end{pgfscope}%
\begin{pgfscope}%
\pgfpathrectangle{\pgfqpoint{0.481978in}{0.331635in}}{\pgfqpoint{4.960000in}{3.696000in}}%
\pgfusepath{clip}%
\pgfsetrectcap%
\pgfsetroundjoin%
\pgfsetlinewidth{1.505625pt}%
\definecolor{currentstroke}{rgb}{1.000000,0.705882,0.509804}%
\pgfsetstrokecolor{currentstroke}%
\pgfsetstrokeopacity{0.800000}%
\pgfsetdash{}{0pt}%
\pgfpathmoveto{\pgfqpoint{3.284231in}{1.137619in}}%
\pgfpathlineto{\pgfqpoint{2.715488in}{1.479455in}}%
\pgfusepath{stroke}%
\end{pgfscope}%
\begin{pgfscope}%
\pgfpathrectangle{\pgfqpoint{0.481978in}{0.331635in}}{\pgfqpoint{4.960000in}{3.696000in}}%
\pgfusepath{clip}%
\pgfsetrectcap%
\pgfsetroundjoin%
\pgfsetlinewidth{1.505625pt}%
\definecolor{currentstroke}{rgb}{1.000000,0.705882,0.509804}%
\pgfsetstrokecolor{currentstroke}%
\pgfsetstrokeopacity{0.800000}%
\pgfsetdash{}{0pt}%
\pgfpathmoveto{\pgfqpoint{1.962719in}{0.945784in}}%
\pgfpathlineto{\pgfqpoint{2.715488in}{1.479455in}}%
\pgfusepath{stroke}%
\end{pgfscope}%
\begin{pgfscope}%
\pgfpathrectangle{\pgfqpoint{0.481978in}{0.331635in}}{\pgfqpoint{4.960000in}{3.696000in}}%
\pgfusepath{clip}%
\pgfsetrectcap%
\pgfsetroundjoin%
\pgfsetlinewidth{1.505625pt}%
\definecolor{currentstroke}{rgb}{1.000000,0.705882,0.509804}%
\pgfsetstrokecolor{currentstroke}%
\pgfsetstrokeopacity{0.800000}%
\pgfsetdash{}{0pt}%
\pgfpathmoveto{\pgfqpoint{1.914501in}{1.143893in}}%
\pgfpathlineto{\pgfqpoint{2.715488in}{1.479455in}}%
\pgfusepath{stroke}%
\end{pgfscope}%
\begin{pgfscope}%
\pgfpathrectangle{\pgfqpoint{0.481978in}{0.331635in}}{\pgfqpoint{4.960000in}{3.696000in}}%
\pgfusepath{clip}%
\pgfsetrectcap%
\pgfsetroundjoin%
\pgfsetlinewidth{1.505625pt}%
\definecolor{currentstroke}{rgb}{1.000000,0.705882,0.509804}%
\pgfsetstrokecolor{currentstroke}%
\pgfsetstrokeopacity{0.800000}%
\pgfsetdash{}{0pt}%
\pgfpathmoveto{\pgfqpoint{3.620395in}{2.141653in}}%
\pgfpathlineto{\pgfqpoint{2.715488in}{1.479455in}}%
\pgfusepath{stroke}%
\end{pgfscope}%
\begin{pgfscope}%
\pgfpathrectangle{\pgfqpoint{0.481978in}{0.331635in}}{\pgfqpoint{4.960000in}{3.696000in}}%
\pgfusepath{clip}%
\pgfsetrectcap%
\pgfsetroundjoin%
\pgfsetlinewidth{1.505625pt}%
\definecolor{currentstroke}{rgb}{1.000000,0.705882,0.509804}%
\pgfsetstrokecolor{currentstroke}%
\pgfsetstrokeopacity{0.800000}%
\pgfsetdash{}{0pt}%
\pgfpathmoveto{\pgfqpoint{4.241494in}{1.279902in}}%
\pgfpathlineto{\pgfqpoint{2.715488in}{1.479455in}}%
\pgfusepath{stroke}%
\end{pgfscope}%
\begin{pgfscope}%
\pgfpathrectangle{\pgfqpoint{0.481978in}{0.331635in}}{\pgfqpoint{4.960000in}{3.696000in}}%
\pgfusepath{clip}%
\pgfsetrectcap%
\pgfsetroundjoin%
\pgfsetlinewidth{1.505625pt}%
\definecolor{currentstroke}{rgb}{1.000000,0.705882,0.509804}%
\pgfsetstrokecolor{currentstroke}%
\pgfsetstrokeopacity{0.800000}%
\pgfsetdash{}{0pt}%
\pgfpathmoveto{\pgfqpoint{3.645474in}{1.232471in}}%
\pgfpathlineto{\pgfqpoint{2.715488in}{1.479455in}}%
\pgfusepath{stroke}%
\end{pgfscope}%
\begin{pgfscope}%
\pgfpathrectangle{\pgfqpoint{0.481978in}{0.331635in}}{\pgfqpoint{4.960000in}{3.696000in}}%
\pgfusepath{clip}%
\pgfsetrectcap%
\pgfsetroundjoin%
\pgfsetlinewidth{1.505625pt}%
\definecolor{currentstroke}{rgb}{1.000000,0.705882,0.509804}%
\pgfsetstrokecolor{currentstroke}%
\pgfsetstrokeopacity{0.800000}%
\pgfsetdash{}{0pt}%
\pgfpathmoveto{\pgfqpoint{1.922492in}{1.907730in}}%
\pgfpathlineto{\pgfqpoint{2.715488in}{1.479455in}}%
\pgfusepath{stroke}%
\end{pgfscope}%
\begin{pgfscope}%
\pgfpathrectangle{\pgfqpoint{0.481978in}{0.331635in}}{\pgfqpoint{4.960000in}{3.696000in}}%
\pgfusepath{clip}%
\pgfsetrectcap%
\pgfsetroundjoin%
\pgfsetlinewidth{1.505625pt}%
\definecolor{currentstroke}{rgb}{1.000000,0.705882,0.509804}%
\pgfsetstrokecolor{currentstroke}%
\pgfsetstrokeopacity{0.800000}%
\pgfsetdash{}{0pt}%
\pgfpathmoveto{\pgfqpoint{3.872683in}{2.027327in}}%
\pgfpathlineto{\pgfqpoint{2.715488in}{1.479455in}}%
\pgfusepath{stroke}%
\end{pgfscope}%
\begin{pgfscope}%
\pgfpathrectangle{\pgfqpoint{0.481978in}{0.331635in}}{\pgfqpoint{4.960000in}{3.696000in}}%
\pgfusepath{clip}%
\pgfsetrectcap%
\pgfsetroundjoin%
\pgfsetlinewidth{1.505625pt}%
\definecolor{currentstroke}{rgb}{1.000000,0.705882,0.509804}%
\pgfsetstrokecolor{currentstroke}%
\pgfsetstrokeopacity{0.800000}%
\pgfsetdash{}{0pt}%
\pgfpathmoveto{\pgfqpoint{3.030639in}{0.946172in}}%
\pgfpathlineto{\pgfqpoint{2.715488in}{1.479455in}}%
\pgfusepath{stroke}%
\end{pgfscope}%
\begin{pgfscope}%
\pgfpathrectangle{\pgfqpoint{0.481978in}{0.331635in}}{\pgfqpoint{4.960000in}{3.696000in}}%
\pgfusepath{clip}%
\pgfsetrectcap%
\pgfsetroundjoin%
\pgfsetlinewidth{1.505625pt}%
\definecolor{currentstroke}{rgb}{1.000000,0.705882,0.509804}%
\pgfsetstrokecolor{currentstroke}%
\pgfsetstrokeopacity{0.800000}%
\pgfsetdash{}{0pt}%
\pgfpathmoveto{\pgfqpoint{4.470487in}{1.812911in}}%
\pgfpathlineto{\pgfqpoint{2.715488in}{1.479455in}}%
\pgfusepath{stroke}%
\end{pgfscope}%
\begin{pgfscope}%
\pgfpathrectangle{\pgfqpoint{0.481978in}{0.331635in}}{\pgfqpoint{4.960000in}{3.696000in}}%
\pgfusepath{clip}%
\pgfsetrectcap%
\pgfsetroundjoin%
\pgfsetlinewidth{1.505625pt}%
\definecolor{currentstroke}{rgb}{1.000000,0.705882,0.509804}%
\pgfsetstrokecolor{currentstroke}%
\pgfsetstrokeopacity{0.800000}%
\pgfsetdash{}{0pt}%
\pgfpathmoveto{\pgfqpoint{2.286558in}{0.774395in}}%
\pgfpathlineto{\pgfqpoint{2.715488in}{1.479455in}}%
\pgfusepath{stroke}%
\end{pgfscope}%
\begin{pgfscope}%
\pgfpathrectangle{\pgfqpoint{0.481978in}{0.331635in}}{\pgfqpoint{4.960000in}{3.696000in}}%
\pgfusepath{clip}%
\pgfsetrectcap%
\pgfsetroundjoin%
\pgfsetlinewidth{1.505625pt}%
\definecolor{currentstroke}{rgb}{1.000000,0.705882,0.509804}%
\pgfsetstrokecolor{currentstroke}%
\pgfsetstrokeopacity{0.800000}%
\pgfsetdash{}{0pt}%
\pgfpathmoveto{\pgfqpoint{4.296984in}{2.122972in}}%
\pgfpathlineto{\pgfqpoint{2.715488in}{1.479455in}}%
\pgfusepath{stroke}%
\end{pgfscope}%
\begin{pgfscope}%
\pgfpathrectangle{\pgfqpoint{0.481978in}{0.331635in}}{\pgfqpoint{4.960000in}{3.696000in}}%
\pgfusepath{clip}%
\pgfsetrectcap%
\pgfsetroundjoin%
\pgfsetlinewidth{1.505625pt}%
\definecolor{currentstroke}{rgb}{1.000000,0.705882,0.509804}%
\pgfsetstrokecolor{currentstroke}%
\pgfsetstrokeopacity{0.800000}%
\pgfsetdash{}{0pt}%
\pgfpathmoveto{\pgfqpoint{3.619173in}{1.374925in}}%
\pgfpathlineto{\pgfqpoint{2.715488in}{1.479455in}}%
\pgfusepath{stroke}%
\end{pgfscope}%
\begin{pgfscope}%
\pgfpathrectangle{\pgfqpoint{0.481978in}{0.331635in}}{\pgfqpoint{4.960000in}{3.696000in}}%
\pgfusepath{clip}%
\pgfsetrectcap%
\pgfsetroundjoin%
\pgfsetlinewidth{1.505625pt}%
\definecolor{currentstroke}{rgb}{1.000000,0.705882,0.509804}%
\pgfsetstrokecolor{currentstroke}%
\pgfsetstrokeopacity{0.800000}%
\pgfsetdash{}{0pt}%
\pgfpathmoveto{\pgfqpoint{0.776093in}{1.105827in}}%
\pgfpathlineto{\pgfqpoint{2.715488in}{1.479455in}}%
\pgfusepath{stroke}%
\end{pgfscope}%
\begin{pgfscope}%
\pgfpathrectangle{\pgfqpoint{0.481978in}{0.331635in}}{\pgfqpoint{4.960000in}{3.696000in}}%
\pgfusepath{clip}%
\pgfsetrectcap%
\pgfsetroundjoin%
\pgfsetlinewidth{1.505625pt}%
\definecolor{currentstroke}{rgb}{1.000000,0.705882,0.509804}%
\pgfsetstrokecolor{currentstroke}%
\pgfsetstrokeopacity{0.800000}%
\pgfsetdash{}{0pt}%
\pgfpathmoveto{\pgfqpoint{2.997617in}{1.858892in}}%
\pgfpathlineto{\pgfqpoint{2.715488in}{1.479455in}}%
\pgfusepath{stroke}%
\end{pgfscope}%
\begin{pgfscope}%
\pgfpathrectangle{\pgfqpoint{0.481978in}{0.331635in}}{\pgfqpoint{4.960000in}{3.696000in}}%
\pgfusepath{clip}%
\pgfsetrectcap%
\pgfsetroundjoin%
\pgfsetlinewidth{1.505625pt}%
\definecolor{currentstroke}{rgb}{1.000000,0.705882,0.509804}%
\pgfsetstrokecolor{currentstroke}%
\pgfsetstrokeopacity{0.800000}%
\pgfsetdash{}{0pt}%
\pgfpathmoveto{\pgfqpoint{3.008821in}{1.448662in}}%
\pgfpathlineto{\pgfqpoint{2.715488in}{1.479455in}}%
\pgfusepath{stroke}%
\end{pgfscope}%
\begin{pgfscope}%
\pgfpathrectangle{\pgfqpoint{0.481978in}{0.331635in}}{\pgfqpoint{4.960000in}{3.696000in}}%
\pgfusepath{clip}%
\pgfsetrectcap%
\pgfsetroundjoin%
\pgfsetlinewidth{1.505625pt}%
\definecolor{currentstroke}{rgb}{1.000000,0.705882,0.509804}%
\pgfsetstrokecolor{currentstroke}%
\pgfsetstrokeopacity{0.800000}%
\pgfsetdash{}{0pt}%
\pgfpathmoveto{\pgfqpoint{1.670611in}{2.208670in}}%
\pgfpathlineto{\pgfqpoint{2.715488in}{1.479455in}}%
\pgfusepath{stroke}%
\end{pgfscope}%
\begin{pgfscope}%
\pgfpathrectangle{\pgfqpoint{0.481978in}{0.331635in}}{\pgfqpoint{4.960000in}{3.696000in}}%
\pgfusepath{clip}%
\pgfsetrectcap%
\pgfsetroundjoin%
\pgfsetlinewidth{1.505625pt}%
\definecolor{currentstroke}{rgb}{1.000000,0.705882,0.509804}%
\pgfsetstrokecolor{currentstroke}%
\pgfsetstrokeopacity{0.800000}%
\pgfsetdash{}{0pt}%
\pgfpathmoveto{\pgfqpoint{0.919551in}{2.064820in}}%
\pgfpathlineto{\pgfqpoint{2.715488in}{1.479455in}}%
\pgfusepath{stroke}%
\end{pgfscope}%
\begin{pgfscope}%
\pgfpathrectangle{\pgfqpoint{0.481978in}{0.331635in}}{\pgfqpoint{4.960000in}{3.696000in}}%
\pgfusepath{clip}%
\pgfsetrectcap%
\pgfsetroundjoin%
\pgfsetlinewidth{1.505625pt}%
\definecolor{currentstroke}{rgb}{1.000000,0.705882,0.509804}%
\pgfsetstrokecolor{currentstroke}%
\pgfsetstrokeopacity{0.800000}%
\pgfsetdash{}{0pt}%
\pgfpathmoveto{\pgfqpoint{2.983995in}{0.966258in}}%
\pgfpathlineto{\pgfqpoint{2.715488in}{1.479455in}}%
\pgfusepath{stroke}%
\end{pgfscope}%
\begin{pgfscope}%
\pgfpathrectangle{\pgfqpoint{0.481978in}{0.331635in}}{\pgfqpoint{4.960000in}{3.696000in}}%
\pgfusepath{clip}%
\pgfsetrectcap%
\pgfsetroundjoin%
\pgfsetlinewidth{1.505625pt}%
\definecolor{currentstroke}{rgb}{1.000000,0.705882,0.509804}%
\pgfsetstrokecolor{currentstroke}%
\pgfsetstrokeopacity{0.800000}%
\pgfsetdash{}{0pt}%
\pgfpathmoveto{\pgfqpoint{2.557995in}{1.692674in}}%
\pgfpathlineto{\pgfqpoint{2.715488in}{1.479455in}}%
\pgfusepath{stroke}%
\end{pgfscope}%
\begin{pgfscope}%
\pgfpathrectangle{\pgfqpoint{0.481978in}{0.331635in}}{\pgfqpoint{4.960000in}{3.696000in}}%
\pgfusepath{clip}%
\pgfsetrectcap%
\pgfsetroundjoin%
\pgfsetlinewidth{1.505625pt}%
\definecolor{currentstroke}{rgb}{1.000000,0.705882,0.509804}%
\pgfsetstrokecolor{currentstroke}%
\pgfsetstrokeopacity{0.800000}%
\pgfsetdash{}{0pt}%
\pgfpathmoveto{\pgfqpoint{1.537027in}{2.435247in}}%
\pgfpathlineto{\pgfqpoint{2.715488in}{1.479455in}}%
\pgfusepath{stroke}%
\end{pgfscope}%
\begin{pgfscope}%
\pgfpathrectangle{\pgfqpoint{0.481978in}{0.331635in}}{\pgfqpoint{4.960000in}{3.696000in}}%
\pgfusepath{clip}%
\pgfsetrectcap%
\pgfsetroundjoin%
\pgfsetlinewidth{1.505625pt}%
\definecolor{currentstroke}{rgb}{1.000000,0.705882,0.509804}%
\pgfsetstrokecolor{currentstroke}%
\pgfsetstrokeopacity{0.800000}%
\pgfsetdash{}{0pt}%
\pgfpathmoveto{\pgfqpoint{1.984657in}{1.289831in}}%
\pgfpathlineto{\pgfqpoint{2.715488in}{1.479455in}}%
\pgfusepath{stroke}%
\end{pgfscope}%
\begin{pgfscope}%
\pgfpathrectangle{\pgfqpoint{0.481978in}{0.331635in}}{\pgfqpoint{4.960000in}{3.696000in}}%
\pgfusepath{clip}%
\pgfsetrectcap%
\pgfsetroundjoin%
\pgfsetlinewidth{1.505625pt}%
\definecolor{currentstroke}{rgb}{1.000000,0.705882,0.509804}%
\pgfsetstrokecolor{currentstroke}%
\pgfsetstrokeopacity{0.800000}%
\pgfsetdash{}{0pt}%
\pgfpathmoveto{\pgfqpoint{0.997912in}{1.333684in}}%
\pgfpathlineto{\pgfqpoint{2.715488in}{1.479455in}}%
\pgfusepath{stroke}%
\end{pgfscope}%
\begin{pgfscope}%
\pgfpathrectangle{\pgfqpoint{0.481978in}{0.331635in}}{\pgfqpoint{4.960000in}{3.696000in}}%
\pgfusepath{clip}%
\pgfsetrectcap%
\pgfsetroundjoin%
\pgfsetlinewidth{1.505625pt}%
\definecolor{currentstroke}{rgb}{1.000000,0.705882,0.509804}%
\pgfsetstrokecolor{currentstroke}%
\pgfsetstrokeopacity{0.800000}%
\pgfsetdash{}{0pt}%
\pgfpathmoveto{\pgfqpoint{3.107756in}{0.499635in}}%
\pgfpathlineto{\pgfqpoint{2.715488in}{1.479455in}}%
\pgfusepath{stroke}%
\end{pgfscope}%
\begin{pgfscope}%
\pgfpathrectangle{\pgfqpoint{0.481978in}{0.331635in}}{\pgfqpoint{4.960000in}{3.696000in}}%
\pgfusepath{clip}%
\pgfsetrectcap%
\pgfsetroundjoin%
\pgfsetlinewidth{1.505625pt}%
\definecolor{currentstroke}{rgb}{1.000000,0.705882,0.509804}%
\pgfsetstrokecolor{currentstroke}%
\pgfsetstrokeopacity{0.800000}%
\pgfsetdash{}{0pt}%
\pgfpathmoveto{\pgfqpoint{2.937197in}{0.999045in}}%
\pgfpathlineto{\pgfqpoint{2.715488in}{1.479455in}}%
\pgfusepath{stroke}%
\end{pgfscope}%
\begin{pgfscope}%
\pgfpathrectangle{\pgfqpoint{0.481978in}{0.331635in}}{\pgfqpoint{4.960000in}{3.696000in}}%
\pgfusepath{clip}%
\pgfsetrectcap%
\pgfsetroundjoin%
\pgfsetlinewidth{1.505625pt}%
\definecolor{currentstroke}{rgb}{1.000000,0.705882,0.509804}%
\pgfsetstrokecolor{currentstroke}%
\pgfsetstrokeopacity{0.800000}%
\pgfsetdash{}{0pt}%
\pgfpathmoveto{\pgfqpoint{2.717812in}{1.376126in}}%
\pgfpathlineto{\pgfqpoint{2.715488in}{1.479455in}}%
\pgfusepath{stroke}%
\end{pgfscope}%
\begin{pgfscope}%
\pgfpathrectangle{\pgfqpoint{0.481978in}{0.331635in}}{\pgfqpoint{4.960000in}{3.696000in}}%
\pgfusepath{clip}%
\pgfsetrectcap%
\pgfsetroundjoin%
\pgfsetlinewidth{1.505625pt}%
\definecolor{currentstroke}{rgb}{1.000000,0.705882,0.509804}%
\pgfsetstrokecolor{currentstroke}%
\pgfsetstrokeopacity{0.800000}%
\pgfsetdash{}{0pt}%
\pgfpathmoveto{\pgfqpoint{2.821264in}{1.532363in}}%
\pgfpathlineto{\pgfqpoint{2.715488in}{1.479455in}}%
\pgfusepath{stroke}%
\end{pgfscope}%
\begin{pgfscope}%
\pgfpathrectangle{\pgfqpoint{0.481978in}{0.331635in}}{\pgfqpoint{4.960000in}{3.696000in}}%
\pgfusepath{clip}%
\pgfsetrectcap%
\pgfsetroundjoin%
\pgfsetlinewidth{1.505625pt}%
\definecolor{currentstroke}{rgb}{1.000000,0.705882,0.509804}%
\pgfsetstrokecolor{currentstroke}%
\pgfsetstrokeopacity{0.800000}%
\pgfsetdash{}{0pt}%
\pgfpathmoveto{\pgfqpoint{1.545663in}{0.874272in}}%
\pgfpathlineto{\pgfqpoint{2.715488in}{1.479455in}}%
\pgfusepath{stroke}%
\end{pgfscope}%
\begin{pgfscope}%
\pgfpathrectangle{\pgfqpoint{0.481978in}{0.331635in}}{\pgfqpoint{4.960000in}{3.696000in}}%
\pgfusepath{clip}%
\pgfsetrectcap%
\pgfsetroundjoin%
\pgfsetlinewidth{1.505625pt}%
\definecolor{currentstroke}{rgb}{1.000000,0.705882,0.509804}%
\pgfsetstrokecolor{currentstroke}%
\pgfsetstrokeopacity{0.800000}%
\pgfsetdash{}{0pt}%
\pgfpathmoveto{\pgfqpoint{1.189172in}{1.700316in}}%
\pgfpathlineto{\pgfqpoint{2.715488in}{1.479455in}}%
\pgfusepath{stroke}%
\end{pgfscope}%
\begin{pgfscope}%
\pgfpathrectangle{\pgfqpoint{0.481978in}{0.331635in}}{\pgfqpoint{4.960000in}{3.696000in}}%
\pgfusepath{clip}%
\pgfsetrectcap%
\pgfsetroundjoin%
\pgfsetlinewidth{1.505625pt}%
\definecolor{currentstroke}{rgb}{1.000000,0.705882,0.509804}%
\pgfsetstrokecolor{currentstroke}%
\pgfsetstrokeopacity{0.800000}%
\pgfsetdash{}{0pt}%
\pgfpathmoveto{\pgfqpoint{1.517112in}{1.968734in}}%
\pgfpathlineto{\pgfqpoint{2.715488in}{1.479455in}}%
\pgfusepath{stroke}%
\end{pgfscope}%
\begin{pgfscope}%
\pgfpathrectangle{\pgfqpoint{0.481978in}{0.331635in}}{\pgfqpoint{4.960000in}{3.696000in}}%
\pgfusepath{clip}%
\pgfsetrectcap%
\pgfsetroundjoin%
\pgfsetlinewidth{1.505625pt}%
\definecolor{currentstroke}{rgb}{1.000000,0.705882,0.509804}%
\pgfsetstrokecolor{currentstroke}%
\pgfsetstrokeopacity{0.800000}%
\pgfsetdash{}{0pt}%
\pgfpathmoveto{\pgfqpoint{3.227064in}{2.232715in}}%
\pgfpathlineto{\pgfqpoint{2.715488in}{1.479455in}}%
\pgfusepath{stroke}%
\end{pgfscope}%
\begin{pgfscope}%
\pgfpathrectangle{\pgfqpoint{0.481978in}{0.331635in}}{\pgfqpoint{4.960000in}{3.696000in}}%
\pgfusepath{clip}%
\pgfsetrectcap%
\pgfsetroundjoin%
\pgfsetlinewidth{1.505625pt}%
\definecolor{currentstroke}{rgb}{1.000000,0.705882,0.509804}%
\pgfsetstrokecolor{currentstroke}%
\pgfsetstrokeopacity{0.800000}%
\pgfsetdash{}{0pt}%
\pgfpathmoveto{\pgfqpoint{0.992160in}{0.963315in}}%
\pgfpathlineto{\pgfqpoint{2.715488in}{1.479455in}}%
\pgfusepath{stroke}%
\end{pgfscope}%
\begin{pgfscope}%
\pgfpathrectangle{\pgfqpoint{0.481978in}{0.331635in}}{\pgfqpoint{4.960000in}{3.696000in}}%
\pgfusepath{clip}%
\pgfsetrectcap%
\pgfsetroundjoin%
\pgfsetlinewidth{1.505625pt}%
\definecolor{currentstroke}{rgb}{1.000000,0.705882,0.509804}%
\pgfsetstrokecolor{currentstroke}%
\pgfsetstrokeopacity{0.800000}%
\pgfsetdash{}{0pt}%
\pgfpathmoveto{\pgfqpoint{1.940136in}{1.646618in}}%
\pgfpathlineto{\pgfqpoint{2.715488in}{1.479455in}}%
\pgfusepath{stroke}%
\end{pgfscope}%
\begin{pgfscope}%
\pgfpathrectangle{\pgfqpoint{0.481978in}{0.331635in}}{\pgfqpoint{4.960000in}{3.696000in}}%
\pgfusepath{clip}%
\pgfsetrectcap%
\pgfsetroundjoin%
\pgfsetlinewidth{1.505625pt}%
\definecolor{currentstroke}{rgb}{1.000000,0.705882,0.509804}%
\pgfsetstrokecolor{currentstroke}%
\pgfsetstrokeopacity{0.800000}%
\pgfsetdash{}{0pt}%
\pgfpathmoveto{\pgfqpoint{2.536499in}{0.618904in}}%
\pgfpathlineto{\pgfqpoint{2.715488in}{1.479455in}}%
\pgfusepath{stroke}%
\end{pgfscope}%
\begin{pgfscope}%
\pgfpathrectangle{\pgfqpoint{0.481978in}{0.331635in}}{\pgfqpoint{4.960000in}{3.696000in}}%
\pgfusepath{clip}%
\pgfsetrectcap%
\pgfsetroundjoin%
\pgfsetlinewidth{1.505625pt}%
\definecolor{currentstroke}{rgb}{1.000000,0.705882,0.509804}%
\pgfsetstrokecolor{currentstroke}%
\pgfsetstrokeopacity{0.800000}%
\pgfsetdash{}{0pt}%
\pgfpathmoveto{\pgfqpoint{3.676022in}{1.366402in}}%
\pgfpathlineto{\pgfqpoint{2.715488in}{1.479455in}}%
\pgfusepath{stroke}%
\end{pgfscope}%
\begin{pgfscope}%
\pgfpathrectangle{\pgfqpoint{0.481978in}{0.331635in}}{\pgfqpoint{4.960000in}{3.696000in}}%
\pgfusepath{clip}%
\pgfsetrectcap%
\pgfsetroundjoin%
\pgfsetlinewidth{1.505625pt}%
\definecolor{currentstroke}{rgb}{1.000000,0.705882,0.509804}%
\pgfsetstrokecolor{currentstroke}%
\pgfsetstrokeopacity{0.800000}%
\pgfsetdash{}{0pt}%
\pgfpathmoveto{\pgfqpoint{1.888129in}{2.025689in}}%
\pgfpathlineto{\pgfqpoint{2.715488in}{1.479455in}}%
\pgfusepath{stroke}%
\end{pgfscope}%
\begin{pgfscope}%
\pgfpathrectangle{\pgfqpoint{0.481978in}{0.331635in}}{\pgfqpoint{4.960000in}{3.696000in}}%
\pgfusepath{clip}%
\pgfsetrectcap%
\pgfsetroundjoin%
\pgfsetlinewidth{1.505625pt}%
\definecolor{currentstroke}{rgb}{1.000000,0.705882,0.509804}%
\pgfsetstrokecolor{currentstroke}%
\pgfsetstrokeopacity{0.800000}%
\pgfsetdash{}{0pt}%
\pgfpathmoveto{\pgfqpoint{2.521771in}{1.516707in}}%
\pgfpathlineto{\pgfqpoint{2.715488in}{1.479455in}}%
\pgfusepath{stroke}%
\end{pgfscope}%
\begin{pgfscope}%
\pgfpathrectangle{\pgfqpoint{0.481978in}{0.331635in}}{\pgfqpoint{4.960000in}{3.696000in}}%
\pgfusepath{clip}%
\pgfsetrectcap%
\pgfsetroundjoin%
\pgfsetlinewidth{1.505625pt}%
\definecolor{currentstroke}{rgb}{1.000000,0.705882,0.509804}%
\pgfsetstrokecolor{currentstroke}%
\pgfsetstrokeopacity{0.800000}%
\pgfsetdash{}{0pt}%
\pgfpathmoveto{\pgfqpoint{3.803581in}{2.155947in}}%
\pgfpathlineto{\pgfqpoint{2.715488in}{1.479455in}}%
\pgfusepath{stroke}%
\end{pgfscope}%
\begin{pgfscope}%
\pgfpathrectangle{\pgfqpoint{0.481978in}{0.331635in}}{\pgfqpoint{4.960000in}{3.696000in}}%
\pgfusepath{clip}%
\pgfsetrectcap%
\pgfsetroundjoin%
\pgfsetlinewidth{1.505625pt}%
\definecolor{currentstroke}{rgb}{1.000000,0.705882,0.509804}%
\pgfsetstrokecolor{currentstroke}%
\pgfsetstrokeopacity{0.800000}%
\pgfsetdash{}{0pt}%
\pgfpathmoveto{\pgfqpoint{4.055052in}{2.083009in}}%
\pgfpathlineto{\pgfqpoint{2.715488in}{1.479455in}}%
\pgfusepath{stroke}%
\end{pgfscope}%
\begin{pgfscope}%
\pgfpathrectangle{\pgfqpoint{0.481978in}{0.331635in}}{\pgfqpoint{4.960000in}{3.696000in}}%
\pgfusepath{clip}%
\pgfsetrectcap%
\pgfsetroundjoin%
\pgfsetlinewidth{1.505625pt}%
\definecolor{currentstroke}{rgb}{1.000000,0.705882,0.509804}%
\pgfsetstrokecolor{currentstroke}%
\pgfsetstrokeopacity{0.800000}%
\pgfsetdash{}{0pt}%
\pgfpathmoveto{\pgfqpoint{4.049936in}{2.174181in}}%
\pgfpathlineto{\pgfqpoint{2.715488in}{1.479455in}}%
\pgfusepath{stroke}%
\end{pgfscope}%
\begin{pgfscope}%
\pgfpathrectangle{\pgfqpoint{0.481978in}{0.331635in}}{\pgfqpoint{4.960000in}{3.696000in}}%
\pgfusepath{clip}%
\pgfsetrectcap%
\pgfsetroundjoin%
\pgfsetlinewidth{1.505625pt}%
\definecolor{currentstroke}{rgb}{1.000000,0.705882,0.509804}%
\pgfsetstrokecolor{currentstroke}%
\pgfsetstrokeopacity{0.800000}%
\pgfsetdash{}{0pt}%
\pgfpathmoveto{\pgfqpoint{3.796579in}{1.995259in}}%
\pgfpathlineto{\pgfqpoint{2.715488in}{1.479455in}}%
\pgfusepath{stroke}%
\end{pgfscope}%
\begin{pgfscope}%
\pgfpathrectangle{\pgfqpoint{0.481978in}{0.331635in}}{\pgfqpoint{4.960000in}{3.696000in}}%
\pgfusepath{clip}%
\pgfsetrectcap%
\pgfsetroundjoin%
\pgfsetlinewidth{1.505625pt}%
\definecolor{currentstroke}{rgb}{1.000000,0.705882,0.509804}%
\pgfsetstrokecolor{currentstroke}%
\pgfsetstrokeopacity{0.800000}%
\pgfsetdash{}{0pt}%
\pgfpathmoveto{\pgfqpoint{1.651432in}{1.054182in}}%
\pgfpathlineto{\pgfqpoint{2.715488in}{1.479455in}}%
\pgfusepath{stroke}%
\end{pgfscope}%
\begin{pgfscope}%
\pgfpathrectangle{\pgfqpoint{0.481978in}{0.331635in}}{\pgfqpoint{4.960000in}{3.696000in}}%
\pgfusepath{clip}%
\pgfsetrectcap%
\pgfsetroundjoin%
\pgfsetlinewidth{1.505625pt}%
\definecolor{currentstroke}{rgb}{1.000000,0.705882,0.509804}%
\pgfsetstrokecolor{currentstroke}%
\pgfsetstrokeopacity{0.800000}%
\pgfsetdash{}{0pt}%
\pgfpathmoveto{\pgfqpoint{2.047564in}{1.105628in}}%
\pgfpathlineto{\pgfqpoint{2.715488in}{1.479455in}}%
\pgfusepath{stroke}%
\end{pgfscope}%
\begin{pgfscope}%
\pgfpathrectangle{\pgfqpoint{0.481978in}{0.331635in}}{\pgfqpoint{4.960000in}{3.696000in}}%
\pgfusepath{clip}%
\pgfsetrectcap%
\pgfsetroundjoin%
\pgfsetlinewidth{1.505625pt}%
\definecolor{currentstroke}{rgb}{1.000000,0.705882,0.509804}%
\pgfsetstrokecolor{currentstroke}%
\pgfsetstrokeopacity{0.800000}%
\pgfsetdash{}{0pt}%
\pgfpathmoveto{\pgfqpoint{3.474036in}{1.721957in}}%
\pgfpathlineto{\pgfqpoint{2.715488in}{1.479455in}}%
\pgfusepath{stroke}%
\end{pgfscope}%
\begin{pgfscope}%
\pgfpathrectangle{\pgfqpoint{0.481978in}{0.331635in}}{\pgfqpoint{4.960000in}{3.696000in}}%
\pgfusepath{clip}%
\pgfsetrectcap%
\pgfsetroundjoin%
\pgfsetlinewidth{1.505625pt}%
\definecolor{currentstroke}{rgb}{1.000000,0.705882,0.509804}%
\pgfsetstrokecolor{currentstroke}%
\pgfsetstrokeopacity{0.800000}%
\pgfsetdash{}{0pt}%
\pgfpathmoveto{\pgfqpoint{3.381704in}{0.739984in}}%
\pgfpathlineto{\pgfqpoint{2.715488in}{1.479455in}}%
\pgfusepath{stroke}%
\end{pgfscope}%
\begin{pgfscope}%
\pgfpathrectangle{\pgfqpoint{0.481978in}{0.331635in}}{\pgfqpoint{4.960000in}{3.696000in}}%
\pgfusepath{clip}%
\pgfsetrectcap%
\pgfsetroundjoin%
\pgfsetlinewidth{1.505625pt}%
\definecolor{currentstroke}{rgb}{1.000000,0.705882,0.509804}%
\pgfsetstrokecolor{currentstroke}%
\pgfsetstrokeopacity{0.800000}%
\pgfsetdash{}{0pt}%
\pgfpathmoveto{\pgfqpoint{4.262882in}{2.258603in}}%
\pgfpathlineto{\pgfqpoint{2.715488in}{1.479455in}}%
\pgfusepath{stroke}%
\end{pgfscope}%
\begin{pgfscope}%
\pgfpathrectangle{\pgfqpoint{0.481978in}{0.331635in}}{\pgfqpoint{4.960000in}{3.696000in}}%
\pgfusepath{clip}%
\pgfsetrectcap%
\pgfsetroundjoin%
\pgfsetlinewidth{1.505625pt}%
\definecolor{currentstroke}{rgb}{1.000000,0.705882,0.509804}%
\pgfsetstrokecolor{currentstroke}%
\pgfsetstrokeopacity{0.800000}%
\pgfsetdash{}{0pt}%
\pgfpathmoveto{\pgfqpoint{3.708000in}{1.907615in}}%
\pgfpathlineto{\pgfqpoint{2.715488in}{1.479455in}}%
\pgfusepath{stroke}%
\end{pgfscope}%
\begin{pgfscope}%
\pgfpathrectangle{\pgfqpoint{0.481978in}{0.331635in}}{\pgfqpoint{4.960000in}{3.696000in}}%
\pgfusepath{clip}%
\pgfsetrectcap%
\pgfsetroundjoin%
\pgfsetlinewidth{1.505625pt}%
\definecolor{currentstroke}{rgb}{1.000000,0.705882,0.509804}%
\pgfsetstrokecolor{currentstroke}%
\pgfsetstrokeopacity{0.800000}%
\pgfsetdash{}{0pt}%
\pgfpathmoveto{\pgfqpoint{1.240808in}{1.016153in}}%
\pgfpathlineto{\pgfqpoint{2.715488in}{1.479455in}}%
\pgfusepath{stroke}%
\end{pgfscope}%
\begin{pgfscope}%
\pgfpathrectangle{\pgfqpoint{0.481978in}{0.331635in}}{\pgfqpoint{4.960000in}{3.696000in}}%
\pgfusepath{clip}%
\pgfsetrectcap%
\pgfsetroundjoin%
\pgfsetlinewidth{1.505625pt}%
\definecolor{currentstroke}{rgb}{1.000000,0.705882,0.509804}%
\pgfsetstrokecolor{currentstroke}%
\pgfsetstrokeopacity{0.800000}%
\pgfsetdash{}{0pt}%
\pgfpathmoveto{\pgfqpoint{1.938067in}{1.114323in}}%
\pgfpathlineto{\pgfqpoint{2.715488in}{1.479455in}}%
\pgfusepath{stroke}%
\end{pgfscope}%
\begin{pgfscope}%
\pgfpathrectangle{\pgfqpoint{0.481978in}{0.331635in}}{\pgfqpoint{4.960000in}{3.696000in}}%
\pgfusepath{clip}%
\pgfsetrectcap%
\pgfsetroundjoin%
\pgfsetlinewidth{1.505625pt}%
\definecolor{currentstroke}{rgb}{1.000000,0.705882,0.509804}%
\pgfsetstrokecolor{currentstroke}%
\pgfsetstrokeopacity{0.800000}%
\pgfsetdash{}{0pt}%
\pgfpathmoveto{\pgfqpoint{3.368392in}{2.044861in}}%
\pgfpathlineto{\pgfqpoint{2.715488in}{1.479455in}}%
\pgfusepath{stroke}%
\end{pgfscope}%
\begin{pgfscope}%
\pgfpathrectangle{\pgfqpoint{0.481978in}{0.331635in}}{\pgfqpoint{4.960000in}{3.696000in}}%
\pgfusepath{clip}%
\pgfsetrectcap%
\pgfsetroundjoin%
\pgfsetlinewidth{1.505625pt}%
\definecolor{currentstroke}{rgb}{1.000000,0.705882,0.509804}%
\pgfsetstrokecolor{currentstroke}%
\pgfsetstrokeopacity{0.800000}%
\pgfsetdash{}{0pt}%
\pgfpathmoveto{\pgfqpoint{1.306901in}{1.221631in}}%
\pgfpathlineto{\pgfqpoint{2.715488in}{1.479455in}}%
\pgfusepath{stroke}%
\end{pgfscope}%
\begin{pgfscope}%
\pgfpathrectangle{\pgfqpoint{0.481978in}{0.331635in}}{\pgfqpoint{4.960000in}{3.696000in}}%
\pgfusepath{clip}%
\pgfsetrectcap%
\pgfsetroundjoin%
\pgfsetlinewidth{1.505625pt}%
\definecolor{currentstroke}{rgb}{1.000000,0.705882,0.509804}%
\pgfsetstrokecolor{currentstroke}%
\pgfsetstrokeopacity{0.800000}%
\pgfsetdash{}{0pt}%
\pgfpathmoveto{\pgfqpoint{1.770432in}{1.244006in}}%
\pgfpathlineto{\pgfqpoint{2.715488in}{1.479455in}}%
\pgfusepath{stroke}%
\end{pgfscope}%
\begin{pgfscope}%
\pgfpathrectangle{\pgfqpoint{0.481978in}{0.331635in}}{\pgfqpoint{4.960000in}{3.696000in}}%
\pgfusepath{clip}%
\pgfsetrectcap%
\pgfsetroundjoin%
\pgfsetlinewidth{1.505625pt}%
\definecolor{currentstroke}{rgb}{1.000000,0.705882,0.509804}%
\pgfsetstrokecolor{currentstroke}%
\pgfsetstrokeopacity{0.800000}%
\pgfsetdash{}{0pt}%
\pgfpathmoveto{\pgfqpoint{2.441773in}{1.196053in}}%
\pgfpathlineto{\pgfqpoint{2.715488in}{1.479455in}}%
\pgfusepath{stroke}%
\end{pgfscope}%
\begin{pgfscope}%
\pgfpathrectangle{\pgfqpoint{0.481978in}{0.331635in}}{\pgfqpoint{4.960000in}{3.696000in}}%
\pgfusepath{clip}%
\pgfsetrectcap%
\pgfsetroundjoin%
\pgfsetlinewidth{1.505625pt}%
\definecolor{currentstroke}{rgb}{1.000000,0.705882,0.509804}%
\pgfsetstrokecolor{currentstroke}%
\pgfsetstrokeopacity{0.800000}%
\pgfsetdash{}{0pt}%
\pgfpathmoveto{\pgfqpoint{3.449131in}{1.278897in}}%
\pgfpathlineto{\pgfqpoint{2.715488in}{1.479455in}}%
\pgfusepath{stroke}%
\end{pgfscope}%
\begin{pgfscope}%
\pgfpathrectangle{\pgfqpoint{0.481978in}{0.331635in}}{\pgfqpoint{4.960000in}{3.696000in}}%
\pgfusepath{clip}%
\pgfsetrectcap%
\pgfsetroundjoin%
\pgfsetlinewidth{1.505625pt}%
\definecolor{currentstroke}{rgb}{1.000000,0.705882,0.509804}%
\pgfsetstrokecolor{currentstroke}%
\pgfsetstrokeopacity{0.800000}%
\pgfsetdash{}{0pt}%
\pgfpathmoveto{\pgfqpoint{2.071777in}{1.399454in}}%
\pgfpathlineto{\pgfqpoint{2.715488in}{1.479455in}}%
\pgfusepath{stroke}%
\end{pgfscope}%
\begin{pgfscope}%
\pgfpathrectangle{\pgfqpoint{0.481978in}{0.331635in}}{\pgfqpoint{4.960000in}{3.696000in}}%
\pgfusepath{clip}%
\pgfsetrectcap%
\pgfsetroundjoin%
\pgfsetlinewidth{1.505625pt}%
\definecolor{currentstroke}{rgb}{1.000000,0.705882,0.509804}%
\pgfsetstrokecolor{currentstroke}%
\pgfsetstrokeopacity{0.800000}%
\pgfsetdash{}{0pt}%
\pgfpathmoveto{\pgfqpoint{2.389920in}{0.719901in}}%
\pgfpathlineto{\pgfqpoint{2.715488in}{1.479455in}}%
\pgfusepath{stroke}%
\end{pgfscope}%
\begin{pgfscope}%
\pgfpathrectangle{\pgfqpoint{0.481978in}{0.331635in}}{\pgfqpoint{4.960000in}{3.696000in}}%
\pgfusepath{clip}%
\pgfsetrectcap%
\pgfsetroundjoin%
\pgfsetlinewidth{1.505625pt}%
\definecolor{currentstroke}{rgb}{1.000000,0.705882,0.509804}%
\pgfsetstrokecolor{currentstroke}%
\pgfsetstrokeopacity{0.800000}%
\pgfsetdash{}{0pt}%
\pgfpathmoveto{\pgfqpoint{2.528165in}{0.918957in}}%
\pgfpathlineto{\pgfqpoint{2.715488in}{1.479455in}}%
\pgfusepath{stroke}%
\end{pgfscope}%
\begin{pgfscope}%
\pgfpathrectangle{\pgfqpoint{0.481978in}{0.331635in}}{\pgfqpoint{4.960000in}{3.696000in}}%
\pgfusepath{clip}%
\pgfsetrectcap%
\pgfsetroundjoin%
\pgfsetlinewidth{1.505625pt}%
\definecolor{currentstroke}{rgb}{1.000000,0.705882,0.509804}%
\pgfsetstrokecolor{currentstroke}%
\pgfsetstrokeopacity{0.800000}%
\pgfsetdash{}{0pt}%
\pgfpathmoveto{\pgfqpoint{3.056217in}{0.846655in}}%
\pgfpathlineto{\pgfqpoint{2.715488in}{1.479455in}}%
\pgfusepath{stroke}%
\end{pgfscope}%
\begin{pgfscope}%
\pgfpathrectangle{\pgfqpoint{0.481978in}{0.331635in}}{\pgfqpoint{4.960000in}{3.696000in}}%
\pgfusepath{clip}%
\pgfsetrectcap%
\pgfsetroundjoin%
\pgfsetlinewidth{1.505625pt}%
\definecolor{currentstroke}{rgb}{1.000000,0.705882,0.509804}%
\pgfsetstrokecolor{currentstroke}%
\pgfsetstrokeopacity{0.800000}%
\pgfsetdash{}{0pt}%
\pgfpathmoveto{\pgfqpoint{3.701071in}{1.687930in}}%
\pgfpathlineto{\pgfqpoint{2.715488in}{1.479455in}}%
\pgfusepath{stroke}%
\end{pgfscope}%
\begin{pgfscope}%
\pgfpathrectangle{\pgfqpoint{0.481978in}{0.331635in}}{\pgfqpoint{4.960000in}{3.696000in}}%
\pgfusepath{clip}%
\pgfsetrectcap%
\pgfsetroundjoin%
\pgfsetlinewidth{1.505625pt}%
\definecolor{currentstroke}{rgb}{1.000000,0.705882,0.509804}%
\pgfsetstrokecolor{currentstroke}%
\pgfsetstrokeopacity{0.800000}%
\pgfsetdash{}{0pt}%
\pgfpathmoveto{\pgfqpoint{1.975829in}{2.309990in}}%
\pgfpathlineto{\pgfqpoint{2.715488in}{1.479455in}}%
\pgfusepath{stroke}%
\end{pgfscope}%
\begin{pgfscope}%
\pgfpathrectangle{\pgfqpoint{0.481978in}{0.331635in}}{\pgfqpoint{4.960000in}{3.696000in}}%
\pgfusepath{clip}%
\pgfsetrectcap%
\pgfsetroundjoin%
\pgfsetlinewidth{1.505625pt}%
\definecolor{currentstroke}{rgb}{1.000000,0.705882,0.509804}%
\pgfsetstrokecolor{currentstroke}%
\pgfsetstrokeopacity{0.800000}%
\pgfsetdash{}{0pt}%
\pgfpathmoveto{\pgfqpoint{4.081865in}{0.791180in}}%
\pgfpathlineto{\pgfqpoint{2.715488in}{1.479455in}}%
\pgfusepath{stroke}%
\end{pgfscope}%
\begin{pgfscope}%
\pgfpathrectangle{\pgfqpoint{0.481978in}{0.331635in}}{\pgfqpoint{4.960000in}{3.696000in}}%
\pgfusepath{clip}%
\pgfsetrectcap%
\pgfsetroundjoin%
\pgfsetlinewidth{1.505625pt}%
\definecolor{currentstroke}{rgb}{1.000000,0.705882,0.509804}%
\pgfsetstrokecolor{currentstroke}%
\pgfsetstrokeopacity{0.800000}%
\pgfsetdash{}{0pt}%
\pgfpathmoveto{\pgfqpoint{1.776731in}{1.243850in}}%
\pgfpathlineto{\pgfqpoint{2.715488in}{1.479455in}}%
\pgfusepath{stroke}%
\end{pgfscope}%
\begin{pgfscope}%
\pgfpathrectangle{\pgfqpoint{0.481978in}{0.331635in}}{\pgfqpoint{4.960000in}{3.696000in}}%
\pgfusepath{clip}%
\pgfsetrectcap%
\pgfsetroundjoin%
\pgfsetlinewidth{1.505625pt}%
\definecolor{currentstroke}{rgb}{1.000000,0.705882,0.509804}%
\pgfsetstrokecolor{currentstroke}%
\pgfsetstrokeopacity{0.800000}%
\pgfsetdash{}{0pt}%
\pgfpathmoveto{\pgfqpoint{3.534146in}{1.102249in}}%
\pgfpathlineto{\pgfqpoint{2.715488in}{1.479455in}}%
\pgfusepath{stroke}%
\end{pgfscope}%
\begin{pgfscope}%
\pgfpathrectangle{\pgfqpoint{0.481978in}{0.331635in}}{\pgfqpoint{4.960000in}{3.696000in}}%
\pgfusepath{clip}%
\pgfsetrectcap%
\pgfsetroundjoin%
\pgfsetlinewidth{1.505625pt}%
\definecolor{currentstroke}{rgb}{1.000000,0.705882,0.509804}%
\pgfsetstrokecolor{currentstroke}%
\pgfsetstrokeopacity{0.800000}%
\pgfsetdash{}{0pt}%
\pgfpathmoveto{\pgfqpoint{3.490400in}{1.592142in}}%
\pgfpathlineto{\pgfqpoint{2.715488in}{1.479455in}}%
\pgfusepath{stroke}%
\end{pgfscope}%
\begin{pgfscope}%
\pgfpathrectangle{\pgfqpoint{0.481978in}{0.331635in}}{\pgfqpoint{4.960000in}{3.696000in}}%
\pgfusepath{clip}%
\pgfsetrectcap%
\pgfsetroundjoin%
\pgfsetlinewidth{1.505625pt}%
\definecolor{currentstroke}{rgb}{1.000000,0.705882,0.509804}%
\pgfsetstrokecolor{currentstroke}%
\pgfsetstrokeopacity{0.800000}%
\pgfsetdash{}{0pt}%
\pgfpathmoveto{\pgfqpoint{2.087048in}{0.624802in}}%
\pgfpathlineto{\pgfqpoint{2.715488in}{1.479455in}}%
\pgfusepath{stroke}%
\end{pgfscope}%
\begin{pgfscope}%
\pgfpathrectangle{\pgfqpoint{0.481978in}{0.331635in}}{\pgfqpoint{4.960000in}{3.696000in}}%
\pgfusepath{clip}%
\pgfsetrectcap%
\pgfsetroundjoin%
\pgfsetlinewidth{1.505625pt}%
\definecolor{currentstroke}{rgb}{1.000000,0.705882,0.509804}%
\pgfsetstrokecolor{currentstroke}%
\pgfsetstrokeopacity{0.800000}%
\pgfsetdash{}{0pt}%
\pgfpathmoveto{\pgfqpoint{2.593014in}{0.717315in}}%
\pgfpathlineto{\pgfqpoint{2.715488in}{1.479455in}}%
\pgfusepath{stroke}%
\end{pgfscope}%
\begin{pgfscope}%
\pgfpathrectangle{\pgfqpoint{0.481978in}{0.331635in}}{\pgfqpoint{4.960000in}{3.696000in}}%
\pgfusepath{clip}%
\pgfsetrectcap%
\pgfsetroundjoin%
\pgfsetlinewidth{1.505625pt}%
\definecolor{currentstroke}{rgb}{1.000000,0.705882,0.509804}%
\pgfsetstrokecolor{currentstroke}%
\pgfsetstrokeopacity{0.800000}%
\pgfsetdash{}{0pt}%
\pgfpathmoveto{\pgfqpoint{1.487659in}{1.707329in}}%
\pgfpathlineto{\pgfqpoint{2.715488in}{1.479455in}}%
\pgfusepath{stroke}%
\end{pgfscope}%
\begin{pgfscope}%
\pgfpathrectangle{\pgfqpoint{0.481978in}{0.331635in}}{\pgfqpoint{4.960000in}{3.696000in}}%
\pgfusepath{clip}%
\pgfsetrectcap%
\pgfsetroundjoin%
\pgfsetlinewidth{1.505625pt}%
\definecolor{currentstroke}{rgb}{1.000000,0.705882,0.509804}%
\pgfsetstrokecolor{currentstroke}%
\pgfsetstrokeopacity{0.800000}%
\pgfsetdash{}{0pt}%
\pgfpathmoveto{\pgfqpoint{4.289496in}{3.142773in}}%
\pgfpathlineto{\pgfqpoint{2.715488in}{1.479455in}}%
\pgfusepath{stroke}%
\end{pgfscope}%
\begin{pgfscope}%
\pgfpathrectangle{\pgfqpoint{0.481978in}{0.331635in}}{\pgfqpoint{4.960000in}{3.696000in}}%
\pgfusepath{clip}%
\pgfsetrectcap%
\pgfsetroundjoin%
\pgfsetlinewidth{1.505625pt}%
\definecolor{currentstroke}{rgb}{1.000000,0.705882,0.509804}%
\pgfsetstrokecolor{currentstroke}%
\pgfsetstrokeopacity{0.800000}%
\pgfsetdash{}{0pt}%
\pgfpathmoveto{\pgfqpoint{3.290746in}{1.706760in}}%
\pgfpathlineto{\pgfqpoint{2.715488in}{1.479455in}}%
\pgfusepath{stroke}%
\end{pgfscope}%
\begin{pgfscope}%
\pgfpathrectangle{\pgfqpoint{0.481978in}{0.331635in}}{\pgfqpoint{4.960000in}{3.696000in}}%
\pgfusepath{clip}%
\pgfsetrectcap%
\pgfsetroundjoin%
\pgfsetlinewidth{1.505625pt}%
\definecolor{currentstroke}{rgb}{1.000000,0.705882,0.509804}%
\pgfsetstrokecolor{currentstroke}%
\pgfsetstrokeopacity{0.800000}%
\pgfsetdash{}{0pt}%
\pgfpathmoveto{\pgfqpoint{4.255842in}{2.134555in}}%
\pgfpathlineto{\pgfqpoint{2.715488in}{1.479455in}}%
\pgfusepath{stroke}%
\end{pgfscope}%
\begin{pgfscope}%
\pgfpathrectangle{\pgfqpoint{0.481978in}{0.331635in}}{\pgfqpoint{4.960000in}{3.696000in}}%
\pgfusepath{clip}%
\pgfsetrectcap%
\pgfsetroundjoin%
\pgfsetlinewidth{1.505625pt}%
\definecolor{currentstroke}{rgb}{1.000000,0.705882,0.509804}%
\pgfsetstrokecolor{currentstroke}%
\pgfsetstrokeopacity{0.800000}%
\pgfsetdash{}{0pt}%
\pgfpathmoveto{\pgfqpoint{2.913924in}{1.167816in}}%
\pgfpathlineto{\pgfqpoint{2.715488in}{1.479455in}}%
\pgfusepath{stroke}%
\end{pgfscope}%
\begin{pgfscope}%
\pgfpathrectangle{\pgfqpoint{0.481978in}{0.331635in}}{\pgfqpoint{4.960000in}{3.696000in}}%
\pgfusepath{clip}%
\pgfsetrectcap%
\pgfsetroundjoin%
\pgfsetlinewidth{1.505625pt}%
\definecolor{currentstroke}{rgb}{1.000000,0.705882,0.509804}%
\pgfsetstrokecolor{currentstroke}%
\pgfsetstrokeopacity{0.800000}%
\pgfsetdash{}{0pt}%
\pgfpathmoveto{\pgfqpoint{3.437504in}{1.320696in}}%
\pgfpathlineto{\pgfqpoint{2.715488in}{1.479455in}}%
\pgfusepath{stroke}%
\end{pgfscope}%
\begin{pgfscope}%
\pgfpathrectangle{\pgfqpoint{0.481978in}{0.331635in}}{\pgfqpoint{4.960000in}{3.696000in}}%
\pgfusepath{clip}%
\pgfsetrectcap%
\pgfsetroundjoin%
\pgfsetlinewidth{1.505625pt}%
\definecolor{currentstroke}{rgb}{1.000000,0.705882,0.509804}%
\pgfsetstrokecolor{currentstroke}%
\pgfsetstrokeopacity{0.800000}%
\pgfsetdash{}{0pt}%
\pgfpathmoveto{\pgfqpoint{2.388897in}{1.472611in}}%
\pgfpathlineto{\pgfqpoint{2.715488in}{1.479455in}}%
\pgfusepath{stroke}%
\end{pgfscope}%
\begin{pgfscope}%
\pgfpathrectangle{\pgfqpoint{0.481978in}{0.331635in}}{\pgfqpoint{4.960000in}{3.696000in}}%
\pgfusepath{clip}%
\pgfsetrectcap%
\pgfsetroundjoin%
\pgfsetlinewidth{1.505625pt}%
\definecolor{currentstroke}{rgb}{1.000000,0.705882,0.509804}%
\pgfsetstrokecolor{currentstroke}%
\pgfsetstrokeopacity{0.800000}%
\pgfsetdash{}{0pt}%
\pgfpathmoveto{\pgfqpoint{1.220987in}{1.785072in}}%
\pgfpathlineto{\pgfqpoint{2.715488in}{1.479455in}}%
\pgfusepath{stroke}%
\end{pgfscope}%
\begin{pgfscope}%
\pgfpathrectangle{\pgfqpoint{0.481978in}{0.331635in}}{\pgfqpoint{4.960000in}{3.696000in}}%
\pgfusepath{clip}%
\pgfsetrectcap%
\pgfsetroundjoin%
\pgfsetlinewidth{1.505625pt}%
\definecolor{currentstroke}{rgb}{1.000000,0.705882,0.509804}%
\pgfsetstrokecolor{currentstroke}%
\pgfsetstrokeopacity{0.800000}%
\pgfsetdash{}{0pt}%
\pgfpathmoveto{\pgfqpoint{2.981752in}{1.537719in}}%
\pgfpathlineto{\pgfqpoint{2.715488in}{1.479455in}}%
\pgfusepath{stroke}%
\end{pgfscope}%
\begin{pgfscope}%
\pgfpathrectangle{\pgfqpoint{0.481978in}{0.331635in}}{\pgfqpoint{4.960000in}{3.696000in}}%
\pgfusepath{clip}%
\pgfsetrectcap%
\pgfsetroundjoin%
\pgfsetlinewidth{1.505625pt}%
\definecolor{currentstroke}{rgb}{1.000000,0.705882,0.509804}%
\pgfsetstrokecolor{currentstroke}%
\pgfsetstrokeopacity{0.800000}%
\pgfsetdash{}{0pt}%
\pgfpathmoveto{\pgfqpoint{4.627519in}{2.218417in}}%
\pgfpathlineto{\pgfqpoint{2.715488in}{1.479455in}}%
\pgfusepath{stroke}%
\end{pgfscope}%
\begin{pgfscope}%
\pgfpathrectangle{\pgfqpoint{0.481978in}{0.331635in}}{\pgfqpoint{4.960000in}{3.696000in}}%
\pgfusepath{clip}%
\pgfsetrectcap%
\pgfsetroundjoin%
\pgfsetlinewidth{1.505625pt}%
\definecolor{currentstroke}{rgb}{1.000000,0.705882,0.509804}%
\pgfsetstrokecolor{currentstroke}%
\pgfsetstrokeopacity{0.800000}%
\pgfsetdash{}{0pt}%
\pgfpathmoveto{\pgfqpoint{2.759699in}{1.483830in}}%
\pgfpathlineto{\pgfqpoint{2.715488in}{1.479455in}}%
\pgfusepath{stroke}%
\end{pgfscope}%
\begin{pgfscope}%
\pgfpathrectangle{\pgfqpoint{0.481978in}{0.331635in}}{\pgfqpoint{4.960000in}{3.696000in}}%
\pgfusepath{clip}%
\pgfsetrectcap%
\pgfsetroundjoin%
\pgfsetlinewidth{1.505625pt}%
\definecolor{currentstroke}{rgb}{1.000000,0.705882,0.509804}%
\pgfsetstrokecolor{currentstroke}%
\pgfsetstrokeopacity{0.800000}%
\pgfsetdash{}{0pt}%
\pgfpathmoveto{\pgfqpoint{3.467601in}{1.137618in}}%
\pgfpathlineto{\pgfqpoint{2.715488in}{1.479455in}}%
\pgfusepath{stroke}%
\end{pgfscope}%
\begin{pgfscope}%
\pgfpathrectangle{\pgfqpoint{0.481978in}{0.331635in}}{\pgfqpoint{4.960000in}{3.696000in}}%
\pgfusepath{clip}%
\pgfsetrectcap%
\pgfsetroundjoin%
\pgfsetlinewidth{1.505625pt}%
\definecolor{currentstroke}{rgb}{1.000000,0.705882,0.509804}%
\pgfsetstrokecolor{currentstroke}%
\pgfsetstrokeopacity{0.800000}%
\pgfsetdash{}{0pt}%
\pgfpathmoveto{\pgfqpoint{1.541084in}{1.640524in}}%
\pgfpathlineto{\pgfqpoint{2.715488in}{1.479455in}}%
\pgfusepath{stroke}%
\end{pgfscope}%
\begin{pgfscope}%
\pgfpathrectangle{\pgfqpoint{0.481978in}{0.331635in}}{\pgfqpoint{4.960000in}{3.696000in}}%
\pgfusepath{clip}%
\pgfsetrectcap%
\pgfsetroundjoin%
\pgfsetlinewidth{1.505625pt}%
\definecolor{currentstroke}{rgb}{1.000000,0.705882,0.509804}%
\pgfsetstrokecolor{currentstroke}%
\pgfsetstrokeopacity{0.800000}%
\pgfsetdash{}{0pt}%
\pgfpathmoveto{\pgfqpoint{3.501411in}{2.279052in}}%
\pgfpathlineto{\pgfqpoint{2.715488in}{1.479455in}}%
\pgfusepath{stroke}%
\end{pgfscope}%
\begin{pgfscope}%
\pgfpathrectangle{\pgfqpoint{0.481978in}{0.331635in}}{\pgfqpoint{4.960000in}{3.696000in}}%
\pgfusepath{clip}%
\pgfsetrectcap%
\pgfsetroundjoin%
\pgfsetlinewidth{1.505625pt}%
\definecolor{currentstroke}{rgb}{1.000000,0.705882,0.509804}%
\pgfsetstrokecolor{currentstroke}%
\pgfsetstrokeopacity{0.800000}%
\pgfsetdash{}{0pt}%
\pgfpathmoveto{\pgfqpoint{3.324129in}{1.540702in}}%
\pgfpathlineto{\pgfqpoint{2.715488in}{1.479455in}}%
\pgfusepath{stroke}%
\end{pgfscope}%
\begin{pgfscope}%
\pgfpathrectangle{\pgfqpoint{0.481978in}{0.331635in}}{\pgfqpoint{4.960000in}{3.696000in}}%
\pgfusepath{clip}%
\pgfsetrectcap%
\pgfsetroundjoin%
\pgfsetlinewidth{1.505625pt}%
\definecolor{currentstroke}{rgb}{1.000000,0.705882,0.509804}%
\pgfsetstrokecolor{currentstroke}%
\pgfsetstrokeopacity{0.800000}%
\pgfsetdash{}{0pt}%
\pgfpathmoveto{\pgfqpoint{3.150076in}{1.371028in}}%
\pgfpathlineto{\pgfqpoint{2.715488in}{1.479455in}}%
\pgfusepath{stroke}%
\end{pgfscope}%
\begin{pgfscope}%
\pgfpathrectangle{\pgfqpoint{0.481978in}{0.331635in}}{\pgfqpoint{4.960000in}{3.696000in}}%
\pgfusepath{clip}%
\pgfsetrectcap%
\pgfsetroundjoin%
\pgfsetlinewidth{1.505625pt}%
\definecolor{currentstroke}{rgb}{1.000000,0.705882,0.509804}%
\pgfsetstrokecolor{currentstroke}%
\pgfsetstrokeopacity{0.800000}%
\pgfsetdash{}{0pt}%
\pgfpathmoveto{\pgfqpoint{2.909368in}{1.469742in}}%
\pgfpathlineto{\pgfqpoint{2.715488in}{1.479455in}}%
\pgfusepath{stroke}%
\end{pgfscope}%
\begin{pgfscope}%
\pgfpathrectangle{\pgfqpoint{0.481978in}{0.331635in}}{\pgfqpoint{4.960000in}{3.696000in}}%
\pgfusepath{clip}%
\pgfsetrectcap%
\pgfsetroundjoin%
\pgfsetlinewidth{1.505625pt}%
\definecolor{currentstroke}{rgb}{1.000000,0.705882,0.509804}%
\pgfsetstrokecolor{currentstroke}%
\pgfsetstrokeopacity{0.800000}%
\pgfsetdash{}{0pt}%
\pgfpathmoveto{\pgfqpoint{1.954857in}{1.052663in}}%
\pgfpathlineto{\pgfqpoint{2.715488in}{1.479455in}}%
\pgfusepath{stroke}%
\end{pgfscope}%
\begin{pgfscope}%
\pgfpathrectangle{\pgfqpoint{0.481978in}{0.331635in}}{\pgfqpoint{4.960000in}{3.696000in}}%
\pgfusepath{clip}%
\pgfsetrectcap%
\pgfsetroundjoin%
\pgfsetlinewidth{1.505625pt}%
\definecolor{currentstroke}{rgb}{1.000000,0.705882,0.509804}%
\pgfsetstrokecolor{currentstroke}%
\pgfsetstrokeopacity{0.800000}%
\pgfsetdash{}{0pt}%
\pgfpathmoveto{\pgfqpoint{3.586905in}{1.599616in}}%
\pgfpathlineto{\pgfqpoint{2.715488in}{1.479455in}}%
\pgfusepath{stroke}%
\end{pgfscope}%
\begin{pgfscope}%
\pgfpathrectangle{\pgfqpoint{0.481978in}{0.331635in}}{\pgfqpoint{4.960000in}{3.696000in}}%
\pgfusepath{clip}%
\pgfsetrectcap%
\pgfsetroundjoin%
\pgfsetlinewidth{1.505625pt}%
\definecolor{currentstroke}{rgb}{1.000000,0.705882,0.509804}%
\pgfsetstrokecolor{currentstroke}%
\pgfsetstrokeopacity{0.800000}%
\pgfsetdash{}{0pt}%
\pgfpathmoveto{\pgfqpoint{3.867423in}{2.935343in}}%
\pgfpathlineto{\pgfqpoint{2.715488in}{1.479455in}}%
\pgfusepath{stroke}%
\end{pgfscope}%
\begin{pgfscope}%
\pgfpathrectangle{\pgfqpoint{0.481978in}{0.331635in}}{\pgfqpoint{4.960000in}{3.696000in}}%
\pgfusepath{clip}%
\pgfsetrectcap%
\pgfsetroundjoin%
\pgfsetlinewidth{1.505625pt}%
\definecolor{currentstroke}{rgb}{1.000000,0.705882,0.509804}%
\pgfsetstrokecolor{currentstroke}%
\pgfsetstrokeopacity{0.800000}%
\pgfsetdash{}{0pt}%
\pgfpathmoveto{\pgfqpoint{1.032296in}{1.312036in}}%
\pgfpathlineto{\pgfqpoint{2.715488in}{1.479455in}}%
\pgfusepath{stroke}%
\end{pgfscope}%
\begin{pgfscope}%
\pgfpathrectangle{\pgfqpoint{0.481978in}{0.331635in}}{\pgfqpoint{4.960000in}{3.696000in}}%
\pgfusepath{clip}%
\pgfsetrectcap%
\pgfsetroundjoin%
\pgfsetlinewidth{1.505625pt}%
\definecolor{currentstroke}{rgb}{1.000000,0.705882,0.509804}%
\pgfsetstrokecolor{currentstroke}%
\pgfsetstrokeopacity{0.800000}%
\pgfsetdash{}{0pt}%
\pgfpathmoveto{\pgfqpoint{1.491483in}{1.563973in}}%
\pgfpathlineto{\pgfqpoint{2.715488in}{1.479455in}}%
\pgfusepath{stroke}%
\end{pgfscope}%
\begin{pgfscope}%
\pgfpathrectangle{\pgfqpoint{0.481978in}{0.331635in}}{\pgfqpoint{4.960000in}{3.696000in}}%
\pgfusepath{clip}%
\pgfsetrectcap%
\pgfsetroundjoin%
\pgfsetlinewidth{1.505625pt}%
\definecolor{currentstroke}{rgb}{1.000000,0.705882,0.509804}%
\pgfsetstrokecolor{currentstroke}%
\pgfsetstrokeopacity{0.800000}%
\pgfsetdash{}{0pt}%
\pgfpathmoveto{\pgfqpoint{2.597836in}{1.419060in}}%
\pgfpathlineto{\pgfqpoint{2.715488in}{1.479455in}}%
\pgfusepath{stroke}%
\end{pgfscope}%
\begin{pgfscope}%
\pgfpathrectangle{\pgfqpoint{0.481978in}{0.331635in}}{\pgfqpoint{4.960000in}{3.696000in}}%
\pgfusepath{clip}%
\pgfsetrectcap%
\pgfsetroundjoin%
\pgfsetlinewidth{1.505625pt}%
\definecolor{currentstroke}{rgb}{1.000000,0.705882,0.509804}%
\pgfsetstrokecolor{currentstroke}%
\pgfsetstrokeopacity{0.800000}%
\pgfsetdash{}{0pt}%
\pgfpathmoveto{\pgfqpoint{2.673632in}{1.560323in}}%
\pgfpathlineto{\pgfqpoint{2.715488in}{1.479455in}}%
\pgfusepath{stroke}%
\end{pgfscope}%
\begin{pgfscope}%
\pgfpathrectangle{\pgfqpoint{0.481978in}{0.331635in}}{\pgfqpoint{4.960000in}{3.696000in}}%
\pgfusepath{clip}%
\pgfsetrectcap%
\pgfsetroundjoin%
\pgfsetlinewidth{1.505625pt}%
\definecolor{currentstroke}{rgb}{1.000000,0.705882,0.509804}%
\pgfsetstrokecolor{currentstroke}%
\pgfsetstrokeopacity{0.800000}%
\pgfsetdash{}{0pt}%
\pgfpathmoveto{\pgfqpoint{3.059811in}{1.271520in}}%
\pgfpathlineto{\pgfqpoint{2.715488in}{1.479455in}}%
\pgfusepath{stroke}%
\end{pgfscope}%
\begin{pgfscope}%
\pgfpathrectangle{\pgfqpoint{0.481978in}{0.331635in}}{\pgfqpoint{4.960000in}{3.696000in}}%
\pgfusepath{clip}%
\pgfsetrectcap%
\pgfsetroundjoin%
\pgfsetlinewidth{1.505625pt}%
\definecolor{currentstroke}{rgb}{1.000000,0.705882,0.509804}%
\pgfsetstrokecolor{currentstroke}%
\pgfsetstrokeopacity{0.800000}%
\pgfsetdash{}{0pt}%
\pgfpathmoveto{\pgfqpoint{2.580966in}{0.719769in}}%
\pgfpathlineto{\pgfqpoint{2.715488in}{1.479455in}}%
\pgfusepath{stroke}%
\end{pgfscope}%
\begin{pgfscope}%
\pgfpathrectangle{\pgfqpoint{0.481978in}{0.331635in}}{\pgfqpoint{4.960000in}{3.696000in}}%
\pgfusepath{clip}%
\pgfsetrectcap%
\pgfsetroundjoin%
\pgfsetlinewidth{1.505625pt}%
\definecolor{currentstroke}{rgb}{1.000000,0.705882,0.509804}%
\pgfsetstrokecolor{currentstroke}%
\pgfsetstrokeopacity{0.800000}%
\pgfsetdash{}{0pt}%
\pgfpathmoveto{\pgfqpoint{3.696596in}{1.610736in}}%
\pgfpathlineto{\pgfqpoint{2.715488in}{1.479455in}}%
\pgfusepath{stroke}%
\end{pgfscope}%
\begin{pgfscope}%
\pgfpathrectangle{\pgfqpoint{0.481978in}{0.331635in}}{\pgfqpoint{4.960000in}{3.696000in}}%
\pgfusepath{clip}%
\pgfsetrectcap%
\pgfsetroundjoin%
\pgfsetlinewidth{1.505625pt}%
\definecolor{currentstroke}{rgb}{1.000000,0.705882,0.509804}%
\pgfsetstrokecolor{currentstroke}%
\pgfsetstrokeopacity{0.800000}%
\pgfsetdash{}{0pt}%
\pgfpathmoveto{\pgfqpoint{1.955794in}{1.546842in}}%
\pgfpathlineto{\pgfqpoint{2.715488in}{1.479455in}}%
\pgfusepath{stroke}%
\end{pgfscope}%
\begin{pgfscope}%
\pgfpathrectangle{\pgfqpoint{0.481978in}{0.331635in}}{\pgfqpoint{4.960000in}{3.696000in}}%
\pgfusepath{clip}%
\pgfsetrectcap%
\pgfsetroundjoin%
\pgfsetlinewidth{1.505625pt}%
\definecolor{currentstroke}{rgb}{1.000000,0.705882,0.509804}%
\pgfsetstrokecolor{currentstroke}%
\pgfsetstrokeopacity{0.800000}%
\pgfsetdash{}{0pt}%
\pgfpathmoveto{\pgfqpoint{3.741977in}{2.137550in}}%
\pgfpathlineto{\pgfqpoint{2.715488in}{1.479455in}}%
\pgfusepath{stroke}%
\end{pgfscope}%
\begin{pgfscope}%
\pgfpathrectangle{\pgfqpoint{0.481978in}{0.331635in}}{\pgfqpoint{4.960000in}{3.696000in}}%
\pgfusepath{clip}%
\pgfsetrectcap%
\pgfsetroundjoin%
\pgfsetlinewidth{1.505625pt}%
\definecolor{currentstroke}{rgb}{1.000000,0.705882,0.509804}%
\pgfsetstrokecolor{currentstroke}%
\pgfsetstrokeopacity{0.800000}%
\pgfsetdash{}{0pt}%
\pgfpathmoveto{\pgfqpoint{1.231668in}{1.057534in}}%
\pgfpathlineto{\pgfqpoint{2.715488in}{1.479455in}}%
\pgfusepath{stroke}%
\end{pgfscope}%
\begin{pgfscope}%
\pgfpathrectangle{\pgfqpoint{0.481978in}{0.331635in}}{\pgfqpoint{4.960000in}{3.696000in}}%
\pgfusepath{clip}%
\pgfsetrectcap%
\pgfsetroundjoin%
\pgfsetlinewidth{1.505625pt}%
\definecolor{currentstroke}{rgb}{1.000000,0.705882,0.509804}%
\pgfsetstrokecolor{currentstroke}%
\pgfsetstrokeopacity{0.800000}%
\pgfsetdash{}{0pt}%
\pgfpathmoveto{\pgfqpoint{1.514045in}{1.466910in}}%
\pgfpathlineto{\pgfqpoint{2.715488in}{1.479455in}}%
\pgfusepath{stroke}%
\end{pgfscope}%
\begin{pgfscope}%
\pgfpathrectangle{\pgfqpoint{0.481978in}{0.331635in}}{\pgfqpoint{4.960000in}{3.696000in}}%
\pgfusepath{clip}%
\pgfsetrectcap%
\pgfsetroundjoin%
\pgfsetlinewidth{1.505625pt}%
\definecolor{currentstroke}{rgb}{1.000000,0.705882,0.509804}%
\pgfsetstrokecolor{currentstroke}%
\pgfsetstrokeopacity{0.800000}%
\pgfsetdash{}{0pt}%
\pgfpathmoveto{\pgfqpoint{2.097374in}{2.122695in}}%
\pgfpathlineto{\pgfqpoint{2.715488in}{1.479455in}}%
\pgfusepath{stroke}%
\end{pgfscope}%
\begin{pgfscope}%
\pgfpathrectangle{\pgfqpoint{0.481978in}{0.331635in}}{\pgfqpoint{4.960000in}{3.696000in}}%
\pgfusepath{clip}%
\pgfsetrectcap%
\pgfsetroundjoin%
\pgfsetlinewidth{1.505625pt}%
\definecolor{currentstroke}{rgb}{1.000000,0.705882,0.509804}%
\pgfsetstrokecolor{currentstroke}%
\pgfsetstrokeopacity{0.800000}%
\pgfsetdash{}{0pt}%
\pgfpathmoveto{\pgfqpoint{3.862616in}{1.212753in}}%
\pgfpathlineto{\pgfqpoint{2.715488in}{1.479455in}}%
\pgfusepath{stroke}%
\end{pgfscope}%
\begin{pgfscope}%
\pgfpathrectangle{\pgfqpoint{0.481978in}{0.331635in}}{\pgfqpoint{4.960000in}{3.696000in}}%
\pgfusepath{clip}%
\pgfsetrectcap%
\pgfsetroundjoin%
\pgfsetlinewidth{1.505625pt}%
\definecolor{currentstroke}{rgb}{1.000000,0.705882,0.509804}%
\pgfsetstrokecolor{currentstroke}%
\pgfsetstrokeopacity{0.800000}%
\pgfsetdash{}{0pt}%
\pgfpathmoveto{\pgfqpoint{2.141974in}{1.801589in}}%
\pgfpathlineto{\pgfqpoint{2.715488in}{1.479455in}}%
\pgfusepath{stroke}%
\end{pgfscope}%
\begin{pgfscope}%
\pgfpathrectangle{\pgfqpoint{0.481978in}{0.331635in}}{\pgfqpoint{4.960000in}{3.696000in}}%
\pgfusepath{clip}%
\pgfsetrectcap%
\pgfsetroundjoin%
\pgfsetlinewidth{1.505625pt}%
\definecolor{currentstroke}{rgb}{1.000000,0.705882,0.509804}%
\pgfsetstrokecolor{currentstroke}%
\pgfsetstrokeopacity{0.800000}%
\pgfsetdash{}{0pt}%
\pgfpathmoveto{\pgfqpoint{2.743580in}{1.047172in}}%
\pgfpathlineto{\pgfqpoint{2.715488in}{1.479455in}}%
\pgfusepath{stroke}%
\end{pgfscope}%
\begin{pgfscope}%
\pgfpathrectangle{\pgfqpoint{0.481978in}{0.331635in}}{\pgfqpoint{4.960000in}{3.696000in}}%
\pgfusepath{clip}%
\pgfsetrectcap%
\pgfsetroundjoin%
\pgfsetlinewidth{1.505625pt}%
\definecolor{currentstroke}{rgb}{1.000000,0.705882,0.509804}%
\pgfsetstrokecolor{currentstroke}%
\pgfsetstrokeopacity{0.800000}%
\pgfsetdash{}{0pt}%
\pgfpathmoveto{\pgfqpoint{0.945106in}{1.426195in}}%
\pgfpathlineto{\pgfqpoint{2.715488in}{1.479455in}}%
\pgfusepath{stroke}%
\end{pgfscope}%
\begin{pgfscope}%
\pgfpathrectangle{\pgfqpoint{0.481978in}{0.331635in}}{\pgfqpoint{4.960000in}{3.696000in}}%
\pgfusepath{clip}%
\pgfsetrectcap%
\pgfsetroundjoin%
\pgfsetlinewidth{1.505625pt}%
\definecolor{currentstroke}{rgb}{1.000000,0.705882,0.509804}%
\pgfsetstrokecolor{currentstroke}%
\pgfsetstrokeopacity{0.800000}%
\pgfsetdash{}{0pt}%
\pgfpathmoveto{\pgfqpoint{2.371014in}{1.325888in}}%
\pgfpathlineto{\pgfqpoint{2.715488in}{1.479455in}}%
\pgfusepath{stroke}%
\end{pgfscope}%
\begin{pgfscope}%
\pgfpathrectangle{\pgfqpoint{0.481978in}{0.331635in}}{\pgfqpoint{4.960000in}{3.696000in}}%
\pgfusepath{clip}%
\pgfsetrectcap%
\pgfsetroundjoin%
\pgfsetlinewidth{1.505625pt}%
\definecolor{currentstroke}{rgb}{1.000000,0.705882,0.509804}%
\pgfsetstrokecolor{currentstroke}%
\pgfsetstrokeopacity{0.800000}%
\pgfsetdash{}{0pt}%
\pgfpathmoveto{\pgfqpoint{4.359096in}{1.314665in}}%
\pgfpathlineto{\pgfqpoint{2.715488in}{1.479455in}}%
\pgfusepath{stroke}%
\end{pgfscope}%
\begin{pgfscope}%
\pgfpathrectangle{\pgfqpoint{0.481978in}{0.331635in}}{\pgfqpoint{4.960000in}{3.696000in}}%
\pgfusepath{clip}%
\pgfsetrectcap%
\pgfsetroundjoin%
\pgfsetlinewidth{1.505625pt}%
\definecolor{currentstroke}{rgb}{1.000000,0.705882,0.509804}%
\pgfsetstrokecolor{currentstroke}%
\pgfsetstrokeopacity{0.800000}%
\pgfsetdash{}{0pt}%
\pgfpathmoveto{\pgfqpoint{2.282850in}{0.612805in}}%
\pgfpathlineto{\pgfqpoint{2.715488in}{1.479455in}}%
\pgfusepath{stroke}%
\end{pgfscope}%
\begin{pgfscope}%
\pgfpathrectangle{\pgfqpoint{0.481978in}{0.331635in}}{\pgfqpoint{4.960000in}{3.696000in}}%
\pgfusepath{clip}%
\pgfsetrectcap%
\pgfsetroundjoin%
\pgfsetlinewidth{1.505625pt}%
\definecolor{currentstroke}{rgb}{1.000000,0.705882,0.509804}%
\pgfsetstrokecolor{currentstroke}%
\pgfsetstrokeopacity{0.800000}%
\pgfsetdash{}{0pt}%
\pgfpathmoveto{\pgfqpoint{2.633929in}{1.466397in}}%
\pgfpathlineto{\pgfqpoint{2.715488in}{1.479455in}}%
\pgfusepath{stroke}%
\end{pgfscope}%
\begin{pgfscope}%
\pgfpathrectangle{\pgfqpoint{0.481978in}{0.331635in}}{\pgfqpoint{4.960000in}{3.696000in}}%
\pgfusepath{clip}%
\pgfsetrectcap%
\pgfsetroundjoin%
\pgfsetlinewidth{1.505625pt}%
\definecolor{currentstroke}{rgb}{1.000000,0.705882,0.509804}%
\pgfsetstrokecolor{currentstroke}%
\pgfsetstrokeopacity{0.800000}%
\pgfsetdash{}{0pt}%
\pgfpathmoveto{\pgfqpoint{2.154059in}{1.151787in}}%
\pgfpathlineto{\pgfqpoint{2.715488in}{1.479455in}}%
\pgfusepath{stroke}%
\end{pgfscope}%
\begin{pgfscope}%
\pgfpathrectangle{\pgfqpoint{0.481978in}{0.331635in}}{\pgfqpoint{4.960000in}{3.696000in}}%
\pgfusepath{clip}%
\pgfsetrectcap%
\pgfsetroundjoin%
\pgfsetlinewidth{1.505625pt}%
\definecolor{currentstroke}{rgb}{1.000000,0.705882,0.509804}%
\pgfsetstrokecolor{currentstroke}%
\pgfsetstrokeopacity{0.800000}%
\pgfsetdash{}{0pt}%
\pgfpathmoveto{\pgfqpoint{2.225441in}{1.805536in}}%
\pgfpathlineto{\pgfqpoint{2.715488in}{1.479455in}}%
\pgfusepath{stroke}%
\end{pgfscope}%
\begin{pgfscope}%
\pgfpathrectangle{\pgfqpoint{0.481978in}{0.331635in}}{\pgfqpoint{4.960000in}{3.696000in}}%
\pgfusepath{clip}%
\pgfsetrectcap%
\pgfsetroundjoin%
\pgfsetlinewidth{1.505625pt}%
\definecolor{currentstroke}{rgb}{1.000000,0.705882,0.509804}%
\pgfsetstrokecolor{currentstroke}%
\pgfsetstrokeopacity{0.800000}%
\pgfsetdash{}{0pt}%
\pgfpathmoveto{\pgfqpoint{1.360201in}{2.187412in}}%
\pgfpathlineto{\pgfqpoint{2.715488in}{1.479455in}}%
\pgfusepath{stroke}%
\end{pgfscope}%
\begin{pgfscope}%
\pgfpathrectangle{\pgfqpoint{0.481978in}{0.331635in}}{\pgfqpoint{4.960000in}{3.696000in}}%
\pgfusepath{clip}%
\pgfsetrectcap%
\pgfsetroundjoin%
\pgfsetlinewidth{1.505625pt}%
\definecolor{currentstroke}{rgb}{1.000000,0.705882,0.509804}%
\pgfsetstrokecolor{currentstroke}%
\pgfsetstrokeopacity{0.800000}%
\pgfsetdash{}{0pt}%
\pgfpathmoveto{\pgfqpoint{2.128386in}{0.944533in}}%
\pgfpathlineto{\pgfqpoint{2.715488in}{1.479455in}}%
\pgfusepath{stroke}%
\end{pgfscope}%
\begin{pgfscope}%
\pgfpathrectangle{\pgfqpoint{0.481978in}{0.331635in}}{\pgfqpoint{4.960000in}{3.696000in}}%
\pgfusepath{clip}%
\pgfsetrectcap%
\pgfsetroundjoin%
\pgfsetlinewidth{1.505625pt}%
\definecolor{currentstroke}{rgb}{1.000000,0.705882,0.509804}%
\pgfsetstrokecolor{currentstroke}%
\pgfsetstrokeopacity{0.800000}%
\pgfsetdash{}{0pt}%
\pgfpathmoveto{\pgfqpoint{3.108327in}{1.071312in}}%
\pgfpathlineto{\pgfqpoint{2.715488in}{1.479455in}}%
\pgfusepath{stroke}%
\end{pgfscope}%
\begin{pgfscope}%
\pgfpathrectangle{\pgfqpoint{0.481978in}{0.331635in}}{\pgfqpoint{4.960000in}{3.696000in}}%
\pgfusepath{clip}%
\pgfsetrectcap%
\pgfsetroundjoin%
\pgfsetlinewidth{1.505625pt}%
\definecolor{currentstroke}{rgb}{1.000000,0.705882,0.509804}%
\pgfsetstrokecolor{currentstroke}%
\pgfsetstrokeopacity{0.800000}%
\pgfsetdash{}{0pt}%
\pgfpathmoveto{\pgfqpoint{1.466879in}{1.672965in}}%
\pgfpathlineto{\pgfqpoint{2.715488in}{1.479455in}}%
\pgfusepath{stroke}%
\end{pgfscope}%
\begin{pgfscope}%
\pgfpathrectangle{\pgfqpoint{0.481978in}{0.331635in}}{\pgfqpoint{4.960000in}{3.696000in}}%
\pgfusepath{clip}%
\pgfsetrectcap%
\pgfsetroundjoin%
\pgfsetlinewidth{1.505625pt}%
\definecolor{currentstroke}{rgb}{1.000000,0.705882,0.509804}%
\pgfsetstrokecolor{currentstroke}%
\pgfsetstrokeopacity{0.800000}%
\pgfsetdash{}{0pt}%
\pgfpathmoveto{\pgfqpoint{2.321041in}{1.385133in}}%
\pgfpathlineto{\pgfqpoint{2.715488in}{1.479455in}}%
\pgfusepath{stroke}%
\end{pgfscope}%
\begin{pgfscope}%
\pgfpathrectangle{\pgfqpoint{0.481978in}{0.331635in}}{\pgfqpoint{4.960000in}{3.696000in}}%
\pgfusepath{clip}%
\pgfsetrectcap%
\pgfsetroundjoin%
\pgfsetlinewidth{1.505625pt}%
\definecolor{currentstroke}{rgb}{1.000000,0.705882,0.509804}%
\pgfsetstrokecolor{currentstroke}%
\pgfsetstrokeopacity{0.800000}%
\pgfsetdash{}{0pt}%
\pgfpathmoveto{\pgfqpoint{3.747692in}{1.484468in}}%
\pgfpathlineto{\pgfqpoint{2.715488in}{1.479455in}}%
\pgfusepath{stroke}%
\end{pgfscope}%
\begin{pgfscope}%
\pgfpathrectangle{\pgfqpoint{0.481978in}{0.331635in}}{\pgfqpoint{4.960000in}{3.696000in}}%
\pgfusepath{clip}%
\pgfsetrectcap%
\pgfsetroundjoin%
\pgfsetlinewidth{1.505625pt}%
\definecolor{currentstroke}{rgb}{1.000000,0.705882,0.509804}%
\pgfsetstrokecolor{currentstroke}%
\pgfsetstrokeopacity{0.800000}%
\pgfsetdash{}{0pt}%
\pgfpathmoveto{\pgfqpoint{0.914552in}{2.059524in}}%
\pgfpathlineto{\pgfqpoint{2.715488in}{1.479455in}}%
\pgfusepath{stroke}%
\end{pgfscope}%
\begin{pgfscope}%
\pgfpathrectangle{\pgfqpoint{0.481978in}{0.331635in}}{\pgfqpoint{4.960000in}{3.696000in}}%
\pgfusepath{clip}%
\pgfsetrectcap%
\pgfsetroundjoin%
\pgfsetlinewidth{1.505625pt}%
\definecolor{currentstroke}{rgb}{1.000000,0.705882,0.509804}%
\pgfsetstrokecolor{currentstroke}%
\pgfsetstrokeopacity{0.800000}%
\pgfsetdash{}{0pt}%
\pgfpathmoveto{\pgfqpoint{2.729676in}{1.762123in}}%
\pgfpathlineto{\pgfqpoint{2.715488in}{1.479455in}}%
\pgfusepath{stroke}%
\end{pgfscope}%
\begin{pgfscope}%
\pgfpathrectangle{\pgfqpoint{0.481978in}{0.331635in}}{\pgfqpoint{4.960000in}{3.696000in}}%
\pgfusepath{clip}%
\pgfsetrectcap%
\pgfsetroundjoin%
\pgfsetlinewidth{1.505625pt}%
\definecolor{currentstroke}{rgb}{1.000000,0.705882,0.509804}%
\pgfsetstrokecolor{currentstroke}%
\pgfsetstrokeopacity{0.800000}%
\pgfsetdash{}{0pt}%
\pgfpathmoveto{\pgfqpoint{1.522149in}{1.751836in}}%
\pgfpathlineto{\pgfqpoint{2.715488in}{1.479455in}}%
\pgfusepath{stroke}%
\end{pgfscope}%
\begin{pgfscope}%
\pgfpathrectangle{\pgfqpoint{0.481978in}{0.331635in}}{\pgfqpoint{4.960000in}{3.696000in}}%
\pgfusepath{clip}%
\pgfsetrectcap%
\pgfsetroundjoin%
\pgfsetlinewidth{1.505625pt}%
\definecolor{currentstroke}{rgb}{1.000000,0.705882,0.509804}%
\pgfsetstrokecolor{currentstroke}%
\pgfsetstrokeopacity{0.800000}%
\pgfsetdash{}{0pt}%
\pgfpathmoveto{\pgfqpoint{3.401713in}{1.495674in}}%
\pgfpathlineto{\pgfqpoint{2.715488in}{1.479455in}}%
\pgfusepath{stroke}%
\end{pgfscope}%
\begin{pgfscope}%
\pgfpathrectangle{\pgfqpoint{0.481978in}{0.331635in}}{\pgfqpoint{4.960000in}{3.696000in}}%
\pgfusepath{clip}%
\pgfsetrectcap%
\pgfsetroundjoin%
\pgfsetlinewidth{1.505625pt}%
\definecolor{currentstroke}{rgb}{1.000000,0.705882,0.509804}%
\pgfsetstrokecolor{currentstroke}%
\pgfsetstrokeopacity{0.800000}%
\pgfsetdash{}{0pt}%
\pgfpathmoveto{\pgfqpoint{2.343714in}{1.562628in}}%
\pgfpathlineto{\pgfqpoint{2.715488in}{1.479455in}}%
\pgfusepath{stroke}%
\end{pgfscope}%
\begin{pgfscope}%
\pgfpathrectangle{\pgfqpoint{0.481978in}{0.331635in}}{\pgfqpoint{4.960000in}{3.696000in}}%
\pgfusepath{clip}%
\pgfsetrectcap%
\pgfsetroundjoin%
\pgfsetlinewidth{1.505625pt}%
\definecolor{currentstroke}{rgb}{1.000000,0.705882,0.509804}%
\pgfsetstrokecolor{currentstroke}%
\pgfsetstrokeopacity{0.800000}%
\pgfsetdash{}{0pt}%
\pgfpathmoveto{\pgfqpoint{3.396230in}{1.363347in}}%
\pgfpathlineto{\pgfqpoint{2.715488in}{1.479455in}}%
\pgfusepath{stroke}%
\end{pgfscope}%
\begin{pgfscope}%
\pgfpathrectangle{\pgfqpoint{0.481978in}{0.331635in}}{\pgfqpoint{4.960000in}{3.696000in}}%
\pgfusepath{clip}%
\pgfsetrectcap%
\pgfsetroundjoin%
\pgfsetlinewidth{1.505625pt}%
\definecolor{currentstroke}{rgb}{1.000000,0.705882,0.509804}%
\pgfsetstrokecolor{currentstroke}%
\pgfsetstrokeopacity{0.800000}%
\pgfsetdash{}{0pt}%
\pgfpathmoveto{\pgfqpoint{0.724960in}{1.349070in}}%
\pgfpathlineto{\pgfqpoint{2.715488in}{1.479455in}}%
\pgfusepath{stroke}%
\end{pgfscope}%
\begin{pgfscope}%
\pgfpathrectangle{\pgfqpoint{0.481978in}{0.331635in}}{\pgfqpoint{4.960000in}{3.696000in}}%
\pgfusepath{clip}%
\pgfsetrectcap%
\pgfsetroundjoin%
\pgfsetlinewidth{1.505625pt}%
\definecolor{currentstroke}{rgb}{1.000000,0.705882,0.509804}%
\pgfsetstrokecolor{currentstroke}%
\pgfsetstrokeopacity{0.800000}%
\pgfsetdash{}{0pt}%
\pgfpathmoveto{\pgfqpoint{1.932203in}{0.916240in}}%
\pgfpathlineto{\pgfqpoint{2.715488in}{1.479455in}}%
\pgfusepath{stroke}%
\end{pgfscope}%
\begin{pgfscope}%
\pgfpathrectangle{\pgfqpoint{0.481978in}{0.331635in}}{\pgfqpoint{4.960000in}{3.696000in}}%
\pgfusepath{clip}%
\pgfsetrectcap%
\pgfsetroundjoin%
\pgfsetlinewidth{1.505625pt}%
\definecolor{currentstroke}{rgb}{1.000000,0.705882,0.509804}%
\pgfsetstrokecolor{currentstroke}%
\pgfsetstrokeopacity{0.800000}%
\pgfsetdash{}{0pt}%
\pgfpathmoveto{\pgfqpoint{3.426668in}{0.678293in}}%
\pgfpathlineto{\pgfqpoint{2.715488in}{1.479455in}}%
\pgfusepath{stroke}%
\end{pgfscope}%
\begin{pgfscope}%
\pgfpathrectangle{\pgfqpoint{0.481978in}{0.331635in}}{\pgfqpoint{4.960000in}{3.696000in}}%
\pgfusepath{clip}%
\pgfsetrectcap%
\pgfsetroundjoin%
\pgfsetlinewidth{1.505625pt}%
\definecolor{currentstroke}{rgb}{1.000000,0.705882,0.509804}%
\pgfsetstrokecolor{currentstroke}%
\pgfsetstrokeopacity{0.800000}%
\pgfsetdash{}{0pt}%
\pgfpathmoveto{\pgfqpoint{3.978978in}{1.143487in}}%
\pgfpathlineto{\pgfqpoint{2.715488in}{1.479455in}}%
\pgfusepath{stroke}%
\end{pgfscope}%
\begin{pgfscope}%
\pgfpathrectangle{\pgfqpoint{0.481978in}{0.331635in}}{\pgfqpoint{4.960000in}{3.696000in}}%
\pgfusepath{clip}%
\pgfsetrectcap%
\pgfsetroundjoin%
\pgfsetlinewidth{1.505625pt}%
\definecolor{currentstroke}{rgb}{1.000000,0.705882,0.509804}%
\pgfsetstrokecolor{currentstroke}%
\pgfsetstrokeopacity{0.800000}%
\pgfsetdash{}{0pt}%
\pgfpathmoveto{\pgfqpoint{2.935351in}{1.263718in}}%
\pgfpathlineto{\pgfqpoint{2.715488in}{1.479455in}}%
\pgfusepath{stroke}%
\end{pgfscope}%
\begin{pgfscope}%
\pgfpathrectangle{\pgfqpoint{0.481978in}{0.331635in}}{\pgfqpoint{4.960000in}{3.696000in}}%
\pgfusepath{clip}%
\pgfsetrectcap%
\pgfsetroundjoin%
\pgfsetlinewidth{1.505625pt}%
\definecolor{currentstroke}{rgb}{1.000000,0.705882,0.509804}%
\pgfsetstrokecolor{currentstroke}%
\pgfsetstrokeopacity{0.800000}%
\pgfsetdash{}{0pt}%
\pgfpathmoveto{\pgfqpoint{3.300863in}{1.137393in}}%
\pgfpathlineto{\pgfqpoint{2.715488in}{1.479455in}}%
\pgfusepath{stroke}%
\end{pgfscope}%
\begin{pgfscope}%
\pgfpathrectangle{\pgfqpoint{0.481978in}{0.331635in}}{\pgfqpoint{4.960000in}{3.696000in}}%
\pgfusepath{clip}%
\pgfsetrectcap%
\pgfsetroundjoin%
\pgfsetlinewidth{1.505625pt}%
\definecolor{currentstroke}{rgb}{1.000000,0.705882,0.509804}%
\pgfsetstrokecolor{currentstroke}%
\pgfsetstrokeopacity{0.800000}%
\pgfsetdash{}{0pt}%
\pgfpathmoveto{\pgfqpoint{3.264488in}{1.868699in}}%
\pgfpathlineto{\pgfqpoint{2.715488in}{1.479455in}}%
\pgfusepath{stroke}%
\end{pgfscope}%
\begin{pgfscope}%
\pgfpathrectangle{\pgfqpoint{0.481978in}{0.331635in}}{\pgfqpoint{4.960000in}{3.696000in}}%
\pgfusepath{clip}%
\pgfsetrectcap%
\pgfsetroundjoin%
\pgfsetlinewidth{1.505625pt}%
\definecolor{currentstroke}{rgb}{1.000000,0.705882,0.509804}%
\pgfsetstrokecolor{currentstroke}%
\pgfsetstrokeopacity{0.800000}%
\pgfsetdash{}{0pt}%
\pgfpathmoveto{\pgfqpoint{2.950725in}{1.348663in}}%
\pgfpathlineto{\pgfqpoint{2.715488in}{1.479455in}}%
\pgfusepath{stroke}%
\end{pgfscope}%
\begin{pgfscope}%
\pgfpathrectangle{\pgfqpoint{0.481978in}{0.331635in}}{\pgfqpoint{4.960000in}{3.696000in}}%
\pgfusepath{clip}%
\pgfsetrectcap%
\pgfsetroundjoin%
\pgfsetlinewidth{1.505625pt}%
\definecolor{currentstroke}{rgb}{1.000000,0.705882,0.509804}%
\pgfsetstrokecolor{currentstroke}%
\pgfsetstrokeopacity{0.800000}%
\pgfsetdash{}{0pt}%
\pgfpathmoveto{\pgfqpoint{0.903986in}{1.098376in}}%
\pgfpathlineto{\pgfqpoint{2.715488in}{1.479455in}}%
\pgfusepath{stroke}%
\end{pgfscope}%
\begin{pgfscope}%
\pgfpathrectangle{\pgfqpoint{0.481978in}{0.331635in}}{\pgfqpoint{4.960000in}{3.696000in}}%
\pgfusepath{clip}%
\pgfsetrectcap%
\pgfsetroundjoin%
\pgfsetlinewidth{1.505625pt}%
\definecolor{currentstroke}{rgb}{1.000000,0.705882,0.509804}%
\pgfsetstrokecolor{currentstroke}%
\pgfsetstrokeopacity{0.800000}%
\pgfsetdash{}{0pt}%
\pgfpathmoveto{\pgfqpoint{1.301328in}{1.014589in}}%
\pgfpathlineto{\pgfqpoint{2.715488in}{1.479455in}}%
\pgfusepath{stroke}%
\end{pgfscope}%
\begin{pgfscope}%
\pgfpathrectangle{\pgfqpoint{0.481978in}{0.331635in}}{\pgfqpoint{4.960000in}{3.696000in}}%
\pgfusepath{clip}%
\pgfsetrectcap%
\pgfsetroundjoin%
\pgfsetlinewidth{1.505625pt}%
\definecolor{currentstroke}{rgb}{1.000000,0.705882,0.509804}%
\pgfsetstrokecolor{currentstroke}%
\pgfsetstrokeopacity{0.800000}%
\pgfsetdash{}{0pt}%
\pgfpathmoveto{\pgfqpoint{0.928531in}{1.100168in}}%
\pgfpathlineto{\pgfqpoint{2.715488in}{1.479455in}}%
\pgfusepath{stroke}%
\end{pgfscope}%
\begin{pgfscope}%
\pgfpathrectangle{\pgfqpoint{0.481978in}{0.331635in}}{\pgfqpoint{4.960000in}{3.696000in}}%
\pgfusepath{clip}%
\pgfsetrectcap%
\pgfsetroundjoin%
\pgfsetlinewidth{1.505625pt}%
\definecolor{currentstroke}{rgb}{1.000000,0.705882,0.509804}%
\pgfsetstrokecolor{currentstroke}%
\pgfsetstrokeopacity{0.800000}%
\pgfsetdash{}{0pt}%
\pgfpathmoveto{\pgfqpoint{3.055823in}{2.307962in}}%
\pgfpathlineto{\pgfqpoint{2.715488in}{1.479455in}}%
\pgfusepath{stroke}%
\end{pgfscope}%
\begin{pgfscope}%
\pgfpathrectangle{\pgfqpoint{0.481978in}{0.331635in}}{\pgfqpoint{4.960000in}{3.696000in}}%
\pgfusepath{clip}%
\pgfsetrectcap%
\pgfsetroundjoin%
\pgfsetlinewidth{1.505625pt}%
\definecolor{currentstroke}{rgb}{1.000000,0.705882,0.509804}%
\pgfsetstrokecolor{currentstroke}%
\pgfsetstrokeopacity{0.800000}%
\pgfsetdash{}{0pt}%
\pgfpathmoveto{\pgfqpoint{2.430224in}{1.691110in}}%
\pgfpathlineto{\pgfqpoint{2.715488in}{1.479455in}}%
\pgfusepath{stroke}%
\end{pgfscope}%
\begin{pgfscope}%
\pgfpathrectangle{\pgfqpoint{0.481978in}{0.331635in}}{\pgfqpoint{4.960000in}{3.696000in}}%
\pgfusepath{clip}%
\pgfsetrectcap%
\pgfsetroundjoin%
\pgfsetlinewidth{1.505625pt}%
\definecolor{currentstroke}{rgb}{1.000000,0.705882,0.509804}%
\pgfsetstrokecolor{currentstroke}%
\pgfsetstrokeopacity{0.800000}%
\pgfsetdash{}{0pt}%
\pgfpathmoveto{\pgfqpoint{1.452210in}{1.128352in}}%
\pgfpathlineto{\pgfqpoint{2.715488in}{1.479455in}}%
\pgfusepath{stroke}%
\end{pgfscope}%
\begin{pgfscope}%
\pgfpathrectangle{\pgfqpoint{0.481978in}{0.331635in}}{\pgfqpoint{4.960000in}{3.696000in}}%
\pgfusepath{clip}%
\pgfsetrectcap%
\pgfsetroundjoin%
\pgfsetlinewidth{1.505625pt}%
\definecolor{currentstroke}{rgb}{1.000000,0.705882,0.509804}%
\pgfsetstrokecolor{currentstroke}%
\pgfsetstrokeopacity{0.800000}%
\pgfsetdash{}{0pt}%
\pgfpathmoveto{\pgfqpoint{2.433780in}{0.958901in}}%
\pgfpathlineto{\pgfqpoint{2.715488in}{1.479455in}}%
\pgfusepath{stroke}%
\end{pgfscope}%
\begin{pgfscope}%
\pgfpathrectangle{\pgfqpoint{0.481978in}{0.331635in}}{\pgfqpoint{4.960000in}{3.696000in}}%
\pgfusepath{clip}%
\pgfsetrectcap%
\pgfsetroundjoin%
\pgfsetlinewidth{1.505625pt}%
\definecolor{currentstroke}{rgb}{1.000000,0.705882,0.509804}%
\pgfsetstrokecolor{currentstroke}%
\pgfsetstrokeopacity{0.800000}%
\pgfsetdash{}{0pt}%
\pgfpathmoveto{\pgfqpoint{3.576377in}{1.385612in}}%
\pgfpathlineto{\pgfqpoint{2.715488in}{1.479455in}}%
\pgfusepath{stroke}%
\end{pgfscope}%
\begin{pgfscope}%
\pgfpathrectangle{\pgfqpoint{0.481978in}{0.331635in}}{\pgfqpoint{4.960000in}{3.696000in}}%
\pgfusepath{clip}%
\pgfsetrectcap%
\pgfsetroundjoin%
\pgfsetlinewidth{1.505625pt}%
\definecolor{currentstroke}{rgb}{1.000000,0.705882,0.509804}%
\pgfsetstrokecolor{currentstroke}%
\pgfsetstrokeopacity{0.800000}%
\pgfsetdash{}{0pt}%
\pgfpathmoveto{\pgfqpoint{3.896697in}{2.197648in}}%
\pgfpathlineto{\pgfqpoint{2.715488in}{1.479455in}}%
\pgfusepath{stroke}%
\end{pgfscope}%
\begin{pgfscope}%
\pgfpathrectangle{\pgfqpoint{0.481978in}{0.331635in}}{\pgfqpoint{4.960000in}{3.696000in}}%
\pgfusepath{clip}%
\pgfsetrectcap%
\pgfsetroundjoin%
\pgfsetlinewidth{1.505625pt}%
\definecolor{currentstroke}{rgb}{1.000000,0.705882,0.509804}%
\pgfsetstrokecolor{currentstroke}%
\pgfsetstrokeopacity{0.800000}%
\pgfsetdash{}{0pt}%
\pgfpathmoveto{\pgfqpoint{1.305107in}{1.554081in}}%
\pgfpathlineto{\pgfqpoint{2.715488in}{1.479455in}}%
\pgfusepath{stroke}%
\end{pgfscope}%
\begin{pgfscope}%
\pgfpathrectangle{\pgfqpoint{0.481978in}{0.331635in}}{\pgfqpoint{4.960000in}{3.696000in}}%
\pgfusepath{clip}%
\pgfsetrectcap%
\pgfsetroundjoin%
\pgfsetlinewidth{1.505625pt}%
\definecolor{currentstroke}{rgb}{1.000000,0.705882,0.509804}%
\pgfsetstrokecolor{currentstroke}%
\pgfsetstrokeopacity{0.800000}%
\pgfsetdash{}{0pt}%
\pgfpathmoveto{\pgfqpoint{3.232967in}{1.053799in}}%
\pgfpathlineto{\pgfqpoint{2.715488in}{1.479455in}}%
\pgfusepath{stroke}%
\end{pgfscope}%
\begin{pgfscope}%
\pgfpathrectangle{\pgfqpoint{0.481978in}{0.331635in}}{\pgfqpoint{4.960000in}{3.696000in}}%
\pgfusepath{clip}%
\pgfsetrectcap%
\pgfsetroundjoin%
\pgfsetlinewidth{1.505625pt}%
\definecolor{currentstroke}{rgb}{1.000000,0.705882,0.509804}%
\pgfsetstrokecolor{currentstroke}%
\pgfsetstrokeopacity{0.800000}%
\pgfsetdash{}{0pt}%
\pgfpathmoveto{\pgfqpoint{1.365241in}{1.894275in}}%
\pgfpathlineto{\pgfqpoint{2.715488in}{1.479455in}}%
\pgfusepath{stroke}%
\end{pgfscope}%
\begin{pgfscope}%
\pgfpathrectangle{\pgfqpoint{0.481978in}{0.331635in}}{\pgfqpoint{4.960000in}{3.696000in}}%
\pgfusepath{clip}%
\pgfsetrectcap%
\pgfsetroundjoin%
\pgfsetlinewidth{1.505625pt}%
\definecolor{currentstroke}{rgb}{1.000000,0.705882,0.509804}%
\pgfsetstrokecolor{currentstroke}%
\pgfsetstrokeopacity{0.800000}%
\pgfsetdash{}{0pt}%
\pgfpathmoveto{\pgfqpoint{5.159612in}{2.537256in}}%
\pgfpathlineto{\pgfqpoint{2.715488in}{1.479455in}}%
\pgfusepath{stroke}%
\end{pgfscope}%
\begin{pgfscope}%
\pgfpathrectangle{\pgfqpoint{0.481978in}{0.331635in}}{\pgfqpoint{4.960000in}{3.696000in}}%
\pgfusepath{clip}%
\pgfsetrectcap%
\pgfsetroundjoin%
\pgfsetlinewidth{1.505625pt}%
\definecolor{currentstroke}{rgb}{1.000000,0.705882,0.509804}%
\pgfsetstrokecolor{currentstroke}%
\pgfsetstrokeopacity{0.800000}%
\pgfsetdash{}{0pt}%
\pgfpathmoveto{\pgfqpoint{3.523183in}{1.000500in}}%
\pgfpathlineto{\pgfqpoint{2.715488in}{1.479455in}}%
\pgfusepath{stroke}%
\end{pgfscope}%
\begin{pgfscope}%
\pgfpathrectangle{\pgfqpoint{0.481978in}{0.331635in}}{\pgfqpoint{4.960000in}{3.696000in}}%
\pgfusepath{clip}%
\pgfsetrectcap%
\pgfsetroundjoin%
\pgfsetlinewidth{1.505625pt}%
\definecolor{currentstroke}{rgb}{1.000000,0.705882,0.509804}%
\pgfsetstrokecolor{currentstroke}%
\pgfsetstrokeopacity{0.800000}%
\pgfsetdash{}{0pt}%
\pgfpathmoveto{\pgfqpoint{3.238366in}{1.408792in}}%
\pgfpathlineto{\pgfqpoint{2.715488in}{1.479455in}}%
\pgfusepath{stroke}%
\end{pgfscope}%
\begin{pgfscope}%
\pgfpathrectangle{\pgfqpoint{0.481978in}{0.331635in}}{\pgfqpoint{4.960000in}{3.696000in}}%
\pgfusepath{clip}%
\pgfsetrectcap%
\pgfsetroundjoin%
\pgfsetlinewidth{1.505625pt}%
\definecolor{currentstroke}{rgb}{1.000000,0.705882,0.509804}%
\pgfsetstrokecolor{currentstroke}%
\pgfsetstrokeopacity{0.800000}%
\pgfsetdash{}{0pt}%
\pgfpathmoveto{\pgfqpoint{1.615727in}{0.783597in}}%
\pgfpathlineto{\pgfqpoint{2.715488in}{1.479455in}}%
\pgfusepath{stroke}%
\end{pgfscope}%
\begin{pgfscope}%
\pgfpathrectangle{\pgfqpoint{0.481978in}{0.331635in}}{\pgfqpoint{4.960000in}{3.696000in}}%
\pgfusepath{clip}%
\pgfsetrectcap%
\pgfsetroundjoin%
\pgfsetlinewidth{1.505625pt}%
\definecolor{currentstroke}{rgb}{1.000000,0.705882,0.509804}%
\pgfsetstrokecolor{currentstroke}%
\pgfsetstrokeopacity{0.800000}%
\pgfsetdash{}{0pt}%
\pgfpathmoveto{\pgfqpoint{3.378019in}{0.873753in}}%
\pgfpathlineto{\pgfqpoint{2.715488in}{1.479455in}}%
\pgfusepath{stroke}%
\end{pgfscope}%
\begin{pgfscope}%
\pgfpathrectangle{\pgfqpoint{0.481978in}{0.331635in}}{\pgfqpoint{4.960000in}{3.696000in}}%
\pgfusepath{clip}%
\pgfsetrectcap%
\pgfsetroundjoin%
\pgfsetlinewidth{1.505625pt}%
\definecolor{currentstroke}{rgb}{1.000000,0.705882,0.509804}%
\pgfsetstrokecolor{currentstroke}%
\pgfsetstrokeopacity{0.800000}%
\pgfsetdash{}{0pt}%
\pgfpathmoveto{\pgfqpoint{4.559563in}{1.975153in}}%
\pgfpathlineto{\pgfqpoint{2.715488in}{1.479455in}}%
\pgfusepath{stroke}%
\end{pgfscope}%
\begin{pgfscope}%
\pgfpathrectangle{\pgfqpoint{0.481978in}{0.331635in}}{\pgfqpoint{4.960000in}{3.696000in}}%
\pgfusepath{clip}%
\pgfsetrectcap%
\pgfsetroundjoin%
\pgfsetlinewidth{1.505625pt}%
\definecolor{currentstroke}{rgb}{1.000000,0.705882,0.509804}%
\pgfsetstrokecolor{currentstroke}%
\pgfsetstrokeopacity{0.800000}%
\pgfsetdash{}{0pt}%
\pgfpathmoveto{\pgfqpoint{3.101805in}{1.625175in}}%
\pgfpathlineto{\pgfqpoint{2.715488in}{1.479455in}}%
\pgfusepath{stroke}%
\end{pgfscope}%
\begin{pgfscope}%
\pgfpathrectangle{\pgfqpoint{0.481978in}{0.331635in}}{\pgfqpoint{4.960000in}{3.696000in}}%
\pgfusepath{clip}%
\pgfsetrectcap%
\pgfsetroundjoin%
\pgfsetlinewidth{1.505625pt}%
\definecolor{currentstroke}{rgb}{1.000000,0.705882,0.509804}%
\pgfsetstrokecolor{currentstroke}%
\pgfsetstrokeopacity{0.800000}%
\pgfsetdash{}{0pt}%
\pgfpathmoveto{\pgfqpoint{2.525832in}{1.406420in}}%
\pgfpathlineto{\pgfqpoint{2.715488in}{1.479455in}}%
\pgfusepath{stroke}%
\end{pgfscope}%
\begin{pgfscope}%
\pgfpathrectangle{\pgfqpoint{0.481978in}{0.331635in}}{\pgfqpoint{4.960000in}{3.696000in}}%
\pgfusepath{clip}%
\pgfsetrectcap%
\pgfsetroundjoin%
\pgfsetlinewidth{1.505625pt}%
\definecolor{currentstroke}{rgb}{1.000000,0.705882,0.509804}%
\pgfsetstrokecolor{currentstroke}%
\pgfsetstrokeopacity{0.800000}%
\pgfsetdash{}{0pt}%
\pgfpathmoveto{\pgfqpoint{4.772753in}{2.757988in}}%
\pgfpathlineto{\pgfqpoint{2.715488in}{1.479455in}}%
\pgfusepath{stroke}%
\end{pgfscope}%
\begin{pgfscope}%
\pgfpathrectangle{\pgfqpoint{0.481978in}{0.331635in}}{\pgfqpoint{4.960000in}{3.696000in}}%
\pgfusepath{clip}%
\pgfsetrectcap%
\pgfsetroundjoin%
\pgfsetlinewidth{1.505625pt}%
\definecolor{currentstroke}{rgb}{1.000000,0.705882,0.509804}%
\pgfsetstrokecolor{currentstroke}%
\pgfsetstrokeopacity{0.800000}%
\pgfsetdash{}{0pt}%
\pgfpathmoveto{\pgfqpoint{2.977732in}{0.559278in}}%
\pgfpathlineto{\pgfqpoint{2.715488in}{1.479455in}}%
\pgfusepath{stroke}%
\end{pgfscope}%
\begin{pgfscope}%
\pgfpathrectangle{\pgfqpoint{0.481978in}{0.331635in}}{\pgfqpoint{4.960000in}{3.696000in}}%
\pgfusepath{clip}%
\pgfsetrectcap%
\pgfsetroundjoin%
\pgfsetlinewidth{1.505625pt}%
\definecolor{currentstroke}{rgb}{1.000000,0.705882,0.509804}%
\pgfsetstrokecolor{currentstroke}%
\pgfsetstrokeopacity{0.800000}%
\pgfsetdash{}{0pt}%
\pgfpathmoveto{\pgfqpoint{3.368601in}{0.731868in}}%
\pgfpathlineto{\pgfqpoint{2.715488in}{1.479455in}}%
\pgfusepath{stroke}%
\end{pgfscope}%
\begin{pgfscope}%
\pgfpathrectangle{\pgfqpoint{0.481978in}{0.331635in}}{\pgfqpoint{4.960000in}{3.696000in}}%
\pgfusepath{clip}%
\pgfsetrectcap%
\pgfsetroundjoin%
\pgfsetlinewidth{1.505625pt}%
\definecolor{currentstroke}{rgb}{1.000000,0.705882,0.509804}%
\pgfsetstrokecolor{currentstroke}%
\pgfsetstrokeopacity{0.800000}%
\pgfsetdash{}{0pt}%
\pgfpathmoveto{\pgfqpoint{3.341619in}{2.428164in}}%
\pgfpathlineto{\pgfqpoint{2.715488in}{1.479455in}}%
\pgfusepath{stroke}%
\end{pgfscope}%
\begin{pgfscope}%
\pgfpathrectangle{\pgfqpoint{0.481978in}{0.331635in}}{\pgfqpoint{4.960000in}{3.696000in}}%
\pgfusepath{clip}%
\pgfsetrectcap%
\pgfsetroundjoin%
\pgfsetlinewidth{1.505625pt}%
\definecolor{currentstroke}{rgb}{1.000000,0.705882,0.509804}%
\pgfsetstrokecolor{currentstroke}%
\pgfsetstrokeopacity{0.800000}%
\pgfsetdash{}{0pt}%
\pgfpathmoveto{\pgfqpoint{1.366891in}{1.237073in}}%
\pgfpathlineto{\pgfqpoint{2.715488in}{1.479455in}}%
\pgfusepath{stroke}%
\end{pgfscope}%
\begin{pgfscope}%
\pgfpathrectangle{\pgfqpoint{0.481978in}{0.331635in}}{\pgfqpoint{4.960000in}{3.696000in}}%
\pgfusepath{clip}%
\pgfsetrectcap%
\pgfsetroundjoin%
\pgfsetlinewidth{1.505625pt}%
\definecolor{currentstroke}{rgb}{1.000000,0.705882,0.509804}%
\pgfsetstrokecolor{currentstroke}%
\pgfsetstrokeopacity{0.800000}%
\pgfsetdash{}{0pt}%
\pgfpathmoveto{\pgfqpoint{1.566891in}{1.792624in}}%
\pgfpathlineto{\pgfqpoint{2.715488in}{1.479455in}}%
\pgfusepath{stroke}%
\end{pgfscope}%
\begin{pgfscope}%
\pgfpathrectangle{\pgfqpoint{0.481978in}{0.331635in}}{\pgfqpoint{4.960000in}{3.696000in}}%
\pgfusepath{clip}%
\pgfsetrectcap%
\pgfsetroundjoin%
\pgfsetlinewidth{1.505625pt}%
\definecolor{currentstroke}{rgb}{1.000000,0.705882,0.509804}%
\pgfsetstrokecolor{currentstroke}%
\pgfsetstrokeopacity{0.800000}%
\pgfsetdash{}{0pt}%
\pgfpathmoveto{\pgfqpoint{2.255178in}{0.553845in}}%
\pgfpathlineto{\pgfqpoint{2.715488in}{1.479455in}}%
\pgfusepath{stroke}%
\end{pgfscope}%
\begin{pgfscope}%
\pgfpathrectangle{\pgfqpoint{0.481978in}{0.331635in}}{\pgfqpoint{4.960000in}{3.696000in}}%
\pgfusepath{clip}%
\pgfsetrectcap%
\pgfsetroundjoin%
\pgfsetlinewidth{1.505625pt}%
\definecolor{currentstroke}{rgb}{1.000000,0.705882,0.509804}%
\pgfsetstrokecolor{currentstroke}%
\pgfsetstrokeopacity{0.800000}%
\pgfsetdash{}{0pt}%
\pgfpathmoveto{\pgfqpoint{4.642349in}{2.218420in}}%
\pgfpathlineto{\pgfqpoint{2.715488in}{1.479455in}}%
\pgfusepath{stroke}%
\end{pgfscope}%
\begin{pgfscope}%
\pgfpathrectangle{\pgfqpoint{0.481978in}{0.331635in}}{\pgfqpoint{4.960000in}{3.696000in}}%
\pgfusepath{clip}%
\pgfsetrectcap%
\pgfsetroundjoin%
\pgfsetlinewidth{1.505625pt}%
\definecolor{currentstroke}{rgb}{1.000000,0.705882,0.509804}%
\pgfsetstrokecolor{currentstroke}%
\pgfsetstrokeopacity{0.800000}%
\pgfsetdash{}{0pt}%
\pgfpathmoveto{\pgfqpoint{1.557977in}{1.011771in}}%
\pgfpathlineto{\pgfqpoint{2.715488in}{1.479455in}}%
\pgfusepath{stroke}%
\end{pgfscope}%
\begin{pgfscope}%
\pgfpathrectangle{\pgfqpoint{0.481978in}{0.331635in}}{\pgfqpoint{4.960000in}{3.696000in}}%
\pgfusepath{clip}%
\pgfsetrectcap%
\pgfsetroundjoin%
\pgfsetlinewidth{1.505625pt}%
\definecolor{currentstroke}{rgb}{1.000000,0.705882,0.509804}%
\pgfsetstrokecolor{currentstroke}%
\pgfsetstrokeopacity{0.800000}%
\pgfsetdash{}{0pt}%
\pgfpathmoveto{\pgfqpoint{3.724605in}{0.800254in}}%
\pgfpathlineto{\pgfqpoint{2.715488in}{1.479455in}}%
\pgfusepath{stroke}%
\end{pgfscope}%
\begin{pgfscope}%
\pgfpathrectangle{\pgfqpoint{0.481978in}{0.331635in}}{\pgfqpoint{4.960000in}{3.696000in}}%
\pgfusepath{clip}%
\pgfsetrectcap%
\pgfsetroundjoin%
\pgfsetlinewidth{1.505625pt}%
\definecolor{currentstroke}{rgb}{1.000000,0.705882,0.509804}%
\pgfsetstrokecolor{currentstroke}%
\pgfsetstrokeopacity{0.800000}%
\pgfsetdash{}{0pt}%
\pgfpathmoveto{\pgfqpoint{2.287234in}{0.768424in}}%
\pgfpathlineto{\pgfqpoint{2.715488in}{1.479455in}}%
\pgfusepath{stroke}%
\end{pgfscope}%
\begin{pgfscope}%
\pgfpathrectangle{\pgfqpoint{0.481978in}{0.331635in}}{\pgfqpoint{4.960000in}{3.696000in}}%
\pgfusepath{clip}%
\pgfsetrectcap%
\pgfsetroundjoin%
\pgfsetlinewidth{1.505625pt}%
\definecolor{currentstroke}{rgb}{1.000000,0.705882,0.509804}%
\pgfsetstrokecolor{currentstroke}%
\pgfsetstrokeopacity{0.800000}%
\pgfsetdash{}{0pt}%
\pgfpathmoveto{\pgfqpoint{3.617579in}{1.808484in}}%
\pgfpathlineto{\pgfqpoint{2.715488in}{1.479455in}}%
\pgfusepath{stroke}%
\end{pgfscope}%
\begin{pgfscope}%
\pgfpathrectangle{\pgfqpoint{0.481978in}{0.331635in}}{\pgfqpoint{4.960000in}{3.696000in}}%
\pgfusepath{clip}%
\pgfsetrectcap%
\pgfsetroundjoin%
\pgfsetlinewidth{1.505625pt}%
\definecolor{currentstroke}{rgb}{1.000000,0.705882,0.509804}%
\pgfsetstrokecolor{currentstroke}%
\pgfsetstrokeopacity{0.800000}%
\pgfsetdash{}{0pt}%
\pgfpathmoveto{\pgfqpoint{1.421203in}{2.445941in}}%
\pgfpathlineto{\pgfqpoint{2.715488in}{1.479455in}}%
\pgfusepath{stroke}%
\end{pgfscope}%
\begin{pgfscope}%
\pgfpathrectangle{\pgfqpoint{0.481978in}{0.331635in}}{\pgfqpoint{4.960000in}{3.696000in}}%
\pgfusepath{clip}%
\pgfsetrectcap%
\pgfsetroundjoin%
\pgfsetlinewidth{1.505625pt}%
\definecolor{currentstroke}{rgb}{1.000000,0.705882,0.509804}%
\pgfsetstrokecolor{currentstroke}%
\pgfsetstrokeopacity{0.800000}%
\pgfsetdash{}{0pt}%
\pgfpathmoveto{\pgfqpoint{4.246861in}{2.227930in}}%
\pgfpathlineto{\pgfqpoint{2.715488in}{1.479455in}}%
\pgfusepath{stroke}%
\end{pgfscope}%
\begin{pgfscope}%
\pgfpathrectangle{\pgfqpoint{0.481978in}{0.331635in}}{\pgfqpoint{4.960000in}{3.696000in}}%
\pgfusepath{clip}%
\pgfsetrectcap%
\pgfsetroundjoin%
\pgfsetlinewidth{1.505625pt}%
\definecolor{currentstroke}{rgb}{1.000000,0.705882,0.509804}%
\pgfsetstrokecolor{currentstroke}%
\pgfsetstrokeopacity{0.800000}%
\pgfsetdash{}{0pt}%
\pgfpathmoveto{\pgfqpoint{4.750602in}{1.409620in}}%
\pgfpathlineto{\pgfqpoint{2.715488in}{1.479455in}}%
\pgfusepath{stroke}%
\end{pgfscope}%
\begin{pgfscope}%
\pgfpathrectangle{\pgfqpoint{0.481978in}{0.331635in}}{\pgfqpoint{4.960000in}{3.696000in}}%
\pgfusepath{clip}%
\pgfsetrectcap%
\pgfsetroundjoin%
\pgfsetlinewidth{1.505625pt}%
\definecolor{currentstroke}{rgb}{1.000000,0.705882,0.509804}%
\pgfsetstrokecolor{currentstroke}%
\pgfsetstrokeopacity{0.800000}%
\pgfsetdash{}{0pt}%
\pgfpathmoveto{\pgfqpoint{2.419077in}{1.077231in}}%
\pgfpathlineto{\pgfqpoint{2.715488in}{1.479455in}}%
\pgfusepath{stroke}%
\end{pgfscope}%
\begin{pgfscope}%
\pgfpathrectangle{\pgfqpoint{0.481978in}{0.331635in}}{\pgfqpoint{4.960000in}{3.696000in}}%
\pgfusepath{clip}%
\pgfsetrectcap%
\pgfsetroundjoin%
\pgfsetlinewidth{1.505625pt}%
\definecolor{currentstroke}{rgb}{1.000000,0.705882,0.509804}%
\pgfsetstrokecolor{currentstroke}%
\pgfsetstrokeopacity{0.800000}%
\pgfsetdash{}{0pt}%
\pgfpathmoveto{\pgfqpoint{2.071657in}{1.416574in}}%
\pgfpathlineto{\pgfqpoint{2.715488in}{1.479455in}}%
\pgfusepath{stroke}%
\end{pgfscope}%
\begin{pgfscope}%
\pgfpathrectangle{\pgfqpoint{0.481978in}{0.331635in}}{\pgfqpoint{4.960000in}{3.696000in}}%
\pgfusepath{clip}%
\pgfsetrectcap%
\pgfsetroundjoin%
\pgfsetlinewidth{1.505625pt}%
\definecolor{currentstroke}{rgb}{1.000000,0.705882,0.509804}%
\pgfsetstrokecolor{currentstroke}%
\pgfsetstrokeopacity{0.800000}%
\pgfsetdash{}{0pt}%
\pgfpathmoveto{\pgfqpoint{2.329525in}{1.250190in}}%
\pgfpathlineto{\pgfqpoint{2.715488in}{1.479455in}}%
\pgfusepath{stroke}%
\end{pgfscope}%
\begin{pgfscope}%
\pgfpathrectangle{\pgfqpoint{0.481978in}{0.331635in}}{\pgfqpoint{4.960000in}{3.696000in}}%
\pgfusepath{clip}%
\pgfsetrectcap%
\pgfsetroundjoin%
\pgfsetlinewidth{1.505625pt}%
\definecolor{currentstroke}{rgb}{1.000000,0.705882,0.509804}%
\pgfsetstrokecolor{currentstroke}%
\pgfsetstrokeopacity{0.800000}%
\pgfsetdash{}{0pt}%
\pgfpathmoveto{\pgfqpoint{1.401436in}{1.161290in}}%
\pgfpathlineto{\pgfqpoint{2.715488in}{1.479455in}}%
\pgfusepath{stroke}%
\end{pgfscope}%
\begin{pgfscope}%
\pgfpathrectangle{\pgfqpoint{0.481978in}{0.331635in}}{\pgfqpoint{4.960000in}{3.696000in}}%
\pgfusepath{clip}%
\pgfsetrectcap%
\pgfsetroundjoin%
\pgfsetlinewidth{1.505625pt}%
\definecolor{currentstroke}{rgb}{1.000000,0.705882,0.509804}%
\pgfsetstrokecolor{currentstroke}%
\pgfsetstrokeopacity{0.800000}%
\pgfsetdash{}{0pt}%
\pgfpathmoveto{\pgfqpoint{2.654322in}{1.253725in}}%
\pgfpathlineto{\pgfqpoint{2.715488in}{1.479455in}}%
\pgfusepath{stroke}%
\end{pgfscope}%
\begin{pgfscope}%
\pgfpathrectangle{\pgfqpoint{0.481978in}{0.331635in}}{\pgfqpoint{4.960000in}{3.696000in}}%
\pgfusepath{clip}%
\pgfsetrectcap%
\pgfsetroundjoin%
\pgfsetlinewidth{1.505625pt}%
\definecolor{currentstroke}{rgb}{1.000000,0.705882,0.509804}%
\pgfsetstrokecolor{currentstroke}%
\pgfsetstrokeopacity{0.800000}%
\pgfsetdash{}{0pt}%
\pgfpathmoveto{\pgfqpoint{3.220258in}{2.346995in}}%
\pgfpathlineto{\pgfqpoint{2.715488in}{1.479455in}}%
\pgfusepath{stroke}%
\end{pgfscope}%
\begin{pgfscope}%
\pgfpathrectangle{\pgfqpoint{0.481978in}{0.331635in}}{\pgfqpoint{4.960000in}{3.696000in}}%
\pgfusepath{clip}%
\pgfsetrectcap%
\pgfsetroundjoin%
\pgfsetlinewidth{1.505625pt}%
\definecolor{currentstroke}{rgb}{1.000000,0.705882,0.509804}%
\pgfsetstrokecolor{currentstroke}%
\pgfsetstrokeopacity{0.800000}%
\pgfsetdash{}{0pt}%
\pgfpathmoveto{\pgfqpoint{3.187385in}{1.509437in}}%
\pgfpathlineto{\pgfqpoint{2.715488in}{1.479455in}}%
\pgfusepath{stroke}%
\end{pgfscope}%
\begin{pgfscope}%
\pgfpathrectangle{\pgfqpoint{0.481978in}{0.331635in}}{\pgfqpoint{4.960000in}{3.696000in}}%
\pgfusepath{clip}%
\pgfsetrectcap%
\pgfsetroundjoin%
\pgfsetlinewidth{1.505625pt}%
\definecolor{currentstroke}{rgb}{1.000000,0.705882,0.509804}%
\pgfsetstrokecolor{currentstroke}%
\pgfsetstrokeopacity{0.800000}%
\pgfsetdash{}{0pt}%
\pgfpathmoveto{\pgfqpoint{2.953726in}{1.398941in}}%
\pgfpathlineto{\pgfqpoint{2.715488in}{1.479455in}}%
\pgfusepath{stroke}%
\end{pgfscope}%
\begin{pgfscope}%
\pgfpathrectangle{\pgfqpoint{0.481978in}{0.331635in}}{\pgfqpoint{4.960000in}{3.696000in}}%
\pgfusepath{clip}%
\pgfsetrectcap%
\pgfsetroundjoin%
\pgfsetlinewidth{1.505625pt}%
\definecolor{currentstroke}{rgb}{1.000000,0.705882,0.509804}%
\pgfsetstrokecolor{currentstroke}%
\pgfsetstrokeopacity{0.800000}%
\pgfsetdash{}{0pt}%
\pgfpathmoveto{\pgfqpoint{1.511141in}{1.463205in}}%
\pgfpathlineto{\pgfqpoint{2.715488in}{1.479455in}}%
\pgfusepath{stroke}%
\end{pgfscope}%
\begin{pgfscope}%
\pgfpathrectangle{\pgfqpoint{0.481978in}{0.331635in}}{\pgfqpoint{4.960000in}{3.696000in}}%
\pgfusepath{clip}%
\pgfsetrectcap%
\pgfsetroundjoin%
\pgfsetlinewidth{1.505625pt}%
\definecolor{currentstroke}{rgb}{0.631373,0.788235,0.956863}%
\pgfsetstrokecolor{currentstroke}%
\pgfsetstrokeopacity{0.200000}%
\pgfsetdash{}{0pt}%
\pgfpathmoveto{\pgfqpoint{3.803556in}{3.471617in}}%
\pgfpathlineto{\pgfqpoint{3.305661in}{2.636178in}}%
\pgfusepath{stroke}%
\end{pgfscope}%
\begin{pgfscope}%
\pgfpathrectangle{\pgfqpoint{0.481978in}{0.331635in}}{\pgfqpoint{4.960000in}{3.696000in}}%
\pgfusepath{clip}%
\pgfsetrectcap%
\pgfsetroundjoin%
\pgfsetlinewidth{1.505625pt}%
\definecolor{currentstroke}{rgb}{0.631373,0.788235,0.956863}%
\pgfsetstrokecolor{currentstroke}%
\pgfsetstrokeopacity{0.200000}%
\pgfsetdash{}{0pt}%
\pgfpathmoveto{\pgfqpoint{4.322934in}{3.425880in}}%
\pgfpathlineto{\pgfqpoint{3.305661in}{2.636178in}}%
\pgfusepath{stroke}%
\end{pgfscope}%
\begin{pgfscope}%
\pgfpathrectangle{\pgfqpoint{0.481978in}{0.331635in}}{\pgfqpoint{4.960000in}{3.696000in}}%
\pgfusepath{clip}%
\pgfsetrectcap%
\pgfsetroundjoin%
\pgfsetlinewidth{1.505625pt}%
\definecolor{currentstroke}{rgb}{0.631373,0.788235,0.956863}%
\pgfsetstrokecolor{currentstroke}%
\pgfsetstrokeopacity{0.200000}%
\pgfsetdash{}{0pt}%
\pgfpathmoveto{\pgfqpoint{4.275707in}{1.538372in}}%
\pgfpathlineto{\pgfqpoint{3.305661in}{2.636178in}}%
\pgfusepath{stroke}%
\end{pgfscope}%
\begin{pgfscope}%
\pgfpathrectangle{\pgfqpoint{0.481978in}{0.331635in}}{\pgfqpoint{4.960000in}{3.696000in}}%
\pgfusepath{clip}%
\pgfsetrectcap%
\pgfsetroundjoin%
\pgfsetlinewidth{1.505625pt}%
\definecolor{currentstroke}{rgb}{0.631373,0.788235,0.956863}%
\pgfsetstrokecolor{currentstroke}%
\pgfsetstrokeopacity{0.200000}%
\pgfsetdash{}{0pt}%
\pgfpathmoveto{\pgfqpoint{2.539406in}{2.236736in}}%
\pgfpathlineto{\pgfqpoint{3.305661in}{2.636178in}}%
\pgfusepath{stroke}%
\end{pgfscope}%
\begin{pgfscope}%
\pgfpathrectangle{\pgfqpoint{0.481978in}{0.331635in}}{\pgfqpoint{4.960000in}{3.696000in}}%
\pgfusepath{clip}%
\pgfsetrectcap%
\pgfsetroundjoin%
\pgfsetlinewidth{1.505625pt}%
\definecolor{currentstroke}{rgb}{0.631373,0.788235,0.956863}%
\pgfsetstrokecolor{currentstroke}%
\pgfsetstrokeopacity{0.200000}%
\pgfsetdash{}{0pt}%
\pgfpathmoveto{\pgfqpoint{3.882882in}{1.125254in}}%
\pgfpathlineto{\pgfqpoint{3.305661in}{2.636178in}}%
\pgfusepath{stroke}%
\end{pgfscope}%
\begin{pgfscope}%
\pgfpathrectangle{\pgfqpoint{0.481978in}{0.331635in}}{\pgfqpoint{4.960000in}{3.696000in}}%
\pgfusepath{clip}%
\pgfsetrectcap%
\pgfsetroundjoin%
\pgfsetlinewidth{1.505625pt}%
\definecolor{currentstroke}{rgb}{0.631373,0.788235,0.956863}%
\pgfsetstrokecolor{currentstroke}%
\pgfsetstrokeopacity{0.200000}%
\pgfsetdash{}{0pt}%
\pgfpathmoveto{\pgfqpoint{2.491685in}{3.011556in}}%
\pgfpathlineto{\pgfqpoint{3.305661in}{2.636178in}}%
\pgfusepath{stroke}%
\end{pgfscope}%
\begin{pgfscope}%
\pgfpathrectangle{\pgfqpoint{0.481978in}{0.331635in}}{\pgfqpoint{4.960000in}{3.696000in}}%
\pgfusepath{clip}%
\pgfsetrectcap%
\pgfsetroundjoin%
\pgfsetlinewidth{1.505625pt}%
\definecolor{currentstroke}{rgb}{0.631373,0.788235,0.956863}%
\pgfsetstrokecolor{currentstroke}%
\pgfsetstrokeopacity{0.200000}%
\pgfsetdash{}{0pt}%
\pgfpathmoveto{\pgfqpoint{4.588483in}{3.109832in}}%
\pgfpathlineto{\pgfqpoint{3.305661in}{2.636178in}}%
\pgfusepath{stroke}%
\end{pgfscope}%
\begin{pgfscope}%
\pgfpathrectangle{\pgfqpoint{0.481978in}{0.331635in}}{\pgfqpoint{4.960000in}{3.696000in}}%
\pgfusepath{clip}%
\pgfsetrectcap%
\pgfsetroundjoin%
\pgfsetlinewidth{1.505625pt}%
\definecolor{currentstroke}{rgb}{0.631373,0.788235,0.956863}%
\pgfsetstrokecolor{currentstroke}%
\pgfsetstrokeopacity{0.200000}%
\pgfsetdash{}{0pt}%
\pgfpathmoveto{\pgfqpoint{3.769223in}{3.351537in}}%
\pgfpathlineto{\pgfqpoint{3.305661in}{2.636178in}}%
\pgfusepath{stroke}%
\end{pgfscope}%
\begin{pgfscope}%
\pgfpathrectangle{\pgfqpoint{0.481978in}{0.331635in}}{\pgfqpoint{4.960000in}{3.696000in}}%
\pgfusepath{clip}%
\pgfsetrectcap%
\pgfsetroundjoin%
\pgfsetlinewidth{1.505625pt}%
\definecolor{currentstroke}{rgb}{0.631373,0.788235,0.956863}%
\pgfsetstrokecolor{currentstroke}%
\pgfsetstrokeopacity{0.200000}%
\pgfsetdash{}{0pt}%
\pgfpathmoveto{\pgfqpoint{3.771620in}{2.575896in}}%
\pgfpathlineto{\pgfqpoint{3.305661in}{2.636178in}}%
\pgfusepath{stroke}%
\end{pgfscope}%
\begin{pgfscope}%
\pgfpathrectangle{\pgfqpoint{0.481978in}{0.331635in}}{\pgfqpoint{4.960000in}{3.696000in}}%
\pgfusepath{clip}%
\pgfsetrectcap%
\pgfsetroundjoin%
\pgfsetlinewidth{1.505625pt}%
\definecolor{currentstroke}{rgb}{0.631373,0.788235,0.956863}%
\pgfsetstrokecolor{currentstroke}%
\pgfsetstrokeopacity{0.200000}%
\pgfsetdash{}{0pt}%
\pgfpathmoveto{\pgfqpoint{2.859335in}{2.816407in}}%
\pgfpathlineto{\pgfqpoint{3.305661in}{2.636178in}}%
\pgfusepath{stroke}%
\end{pgfscope}%
\begin{pgfscope}%
\pgfpathrectangle{\pgfqpoint{0.481978in}{0.331635in}}{\pgfqpoint{4.960000in}{3.696000in}}%
\pgfusepath{clip}%
\pgfsetrectcap%
\pgfsetroundjoin%
\pgfsetlinewidth{1.505625pt}%
\definecolor{currentstroke}{rgb}{0.631373,0.788235,0.956863}%
\pgfsetstrokecolor{currentstroke}%
\pgfsetstrokeopacity{0.200000}%
\pgfsetdash{}{0pt}%
\pgfpathmoveto{\pgfqpoint{2.367225in}{1.888324in}}%
\pgfpathlineto{\pgfqpoint{3.305661in}{2.636178in}}%
\pgfusepath{stroke}%
\end{pgfscope}%
\begin{pgfscope}%
\pgfpathrectangle{\pgfqpoint{0.481978in}{0.331635in}}{\pgfqpoint{4.960000in}{3.696000in}}%
\pgfusepath{clip}%
\pgfsetrectcap%
\pgfsetroundjoin%
\pgfsetlinewidth{1.505625pt}%
\definecolor{currentstroke}{rgb}{0.631373,0.788235,0.956863}%
\pgfsetstrokecolor{currentstroke}%
\pgfsetstrokeopacity{0.200000}%
\pgfsetdash{}{0pt}%
\pgfpathmoveto{\pgfqpoint{3.918937in}{3.332403in}}%
\pgfpathlineto{\pgfqpoint{3.305661in}{2.636178in}}%
\pgfusepath{stroke}%
\end{pgfscope}%
\begin{pgfscope}%
\pgfpathrectangle{\pgfqpoint{0.481978in}{0.331635in}}{\pgfqpoint{4.960000in}{3.696000in}}%
\pgfusepath{clip}%
\pgfsetrectcap%
\pgfsetroundjoin%
\pgfsetlinewidth{1.505625pt}%
\definecolor{currentstroke}{rgb}{0.631373,0.788235,0.956863}%
\pgfsetstrokecolor{currentstroke}%
\pgfsetstrokeopacity{0.200000}%
\pgfsetdash{}{0pt}%
\pgfpathmoveto{\pgfqpoint{2.336229in}{2.956954in}}%
\pgfpathlineto{\pgfqpoint{3.305661in}{2.636178in}}%
\pgfusepath{stroke}%
\end{pgfscope}%
\begin{pgfscope}%
\pgfpathrectangle{\pgfqpoint{0.481978in}{0.331635in}}{\pgfqpoint{4.960000in}{3.696000in}}%
\pgfusepath{clip}%
\pgfsetrectcap%
\pgfsetroundjoin%
\pgfsetlinewidth{1.505625pt}%
\definecolor{currentstroke}{rgb}{0.631373,0.788235,0.956863}%
\pgfsetstrokecolor{currentstroke}%
\pgfsetstrokeopacity{0.200000}%
\pgfsetdash{}{0pt}%
\pgfpathmoveto{\pgfqpoint{3.992759in}{2.275377in}}%
\pgfpathlineto{\pgfqpoint{3.305661in}{2.636178in}}%
\pgfusepath{stroke}%
\end{pgfscope}%
\begin{pgfscope}%
\pgfpathrectangle{\pgfqpoint{0.481978in}{0.331635in}}{\pgfqpoint{4.960000in}{3.696000in}}%
\pgfusepath{clip}%
\pgfsetrectcap%
\pgfsetroundjoin%
\pgfsetlinewidth{1.505625pt}%
\definecolor{currentstroke}{rgb}{0.631373,0.788235,0.956863}%
\pgfsetstrokecolor{currentstroke}%
\pgfsetstrokeopacity{0.200000}%
\pgfsetdash{}{0pt}%
\pgfpathmoveto{\pgfqpoint{3.957359in}{3.412152in}}%
\pgfpathlineto{\pgfqpoint{3.305661in}{2.636178in}}%
\pgfusepath{stroke}%
\end{pgfscope}%
\begin{pgfscope}%
\pgfpathrectangle{\pgfqpoint{0.481978in}{0.331635in}}{\pgfqpoint{4.960000in}{3.696000in}}%
\pgfusepath{clip}%
\pgfsetrectcap%
\pgfsetroundjoin%
\pgfsetlinewidth{1.505625pt}%
\definecolor{currentstroke}{rgb}{0.631373,0.788235,0.956863}%
\pgfsetstrokecolor{currentstroke}%
\pgfsetstrokeopacity{0.200000}%
\pgfsetdash{}{0pt}%
\pgfpathmoveto{\pgfqpoint{3.883880in}{1.791774in}}%
\pgfpathlineto{\pgfqpoint{3.305661in}{2.636178in}}%
\pgfusepath{stroke}%
\end{pgfscope}%
\begin{pgfscope}%
\pgfpathrectangle{\pgfqpoint{0.481978in}{0.331635in}}{\pgfqpoint{4.960000in}{3.696000in}}%
\pgfusepath{clip}%
\pgfsetrectcap%
\pgfsetroundjoin%
\pgfsetlinewidth{1.505625pt}%
\definecolor{currentstroke}{rgb}{0.631373,0.788235,0.956863}%
\pgfsetstrokecolor{currentstroke}%
\pgfsetstrokeopacity{0.200000}%
\pgfsetdash{}{0pt}%
\pgfpathmoveto{\pgfqpoint{1.951129in}{1.573716in}}%
\pgfpathlineto{\pgfqpoint{3.305661in}{2.636178in}}%
\pgfusepath{stroke}%
\end{pgfscope}%
\begin{pgfscope}%
\pgfpathrectangle{\pgfqpoint{0.481978in}{0.331635in}}{\pgfqpoint{4.960000in}{3.696000in}}%
\pgfusepath{clip}%
\pgfsetrectcap%
\pgfsetroundjoin%
\pgfsetlinewidth{1.505625pt}%
\definecolor{currentstroke}{rgb}{0.631373,0.788235,0.956863}%
\pgfsetstrokecolor{currentstroke}%
\pgfsetstrokeopacity{0.200000}%
\pgfsetdash{}{0pt}%
\pgfpathmoveto{\pgfqpoint{4.682335in}{1.517187in}}%
\pgfpathlineto{\pgfqpoint{3.305661in}{2.636178in}}%
\pgfusepath{stroke}%
\end{pgfscope}%
\begin{pgfscope}%
\pgfpathrectangle{\pgfqpoint{0.481978in}{0.331635in}}{\pgfqpoint{4.960000in}{3.696000in}}%
\pgfusepath{clip}%
\pgfsetrectcap%
\pgfsetroundjoin%
\pgfsetlinewidth{1.505625pt}%
\definecolor{currentstroke}{rgb}{0.631373,0.788235,0.956863}%
\pgfsetstrokecolor{currentstroke}%
\pgfsetstrokeopacity{0.200000}%
\pgfsetdash{}{0pt}%
\pgfpathmoveto{\pgfqpoint{2.799814in}{3.546765in}}%
\pgfpathlineto{\pgfqpoint{3.305661in}{2.636178in}}%
\pgfusepath{stroke}%
\end{pgfscope}%
\begin{pgfscope}%
\pgfpathrectangle{\pgfqpoint{0.481978in}{0.331635in}}{\pgfqpoint{4.960000in}{3.696000in}}%
\pgfusepath{clip}%
\pgfsetrectcap%
\pgfsetroundjoin%
\pgfsetlinewidth{1.505625pt}%
\definecolor{currentstroke}{rgb}{0.631373,0.788235,0.956863}%
\pgfsetstrokecolor{currentstroke}%
\pgfsetstrokeopacity{0.200000}%
\pgfsetdash{}{0pt}%
\pgfpathmoveto{\pgfqpoint{3.682647in}{3.000483in}}%
\pgfpathlineto{\pgfqpoint{3.305661in}{2.636178in}}%
\pgfusepath{stroke}%
\end{pgfscope}%
\begin{pgfscope}%
\pgfpathrectangle{\pgfqpoint{0.481978in}{0.331635in}}{\pgfqpoint{4.960000in}{3.696000in}}%
\pgfusepath{clip}%
\pgfsetrectcap%
\pgfsetroundjoin%
\pgfsetlinewidth{1.505625pt}%
\definecolor{currentstroke}{rgb}{0.631373,0.788235,0.956863}%
\pgfsetstrokecolor{currentstroke}%
\pgfsetstrokeopacity{0.200000}%
\pgfsetdash{}{0pt}%
\pgfpathmoveto{\pgfqpoint{3.082225in}{2.146516in}}%
\pgfpathlineto{\pgfqpoint{3.305661in}{2.636178in}}%
\pgfusepath{stroke}%
\end{pgfscope}%
\begin{pgfscope}%
\pgfpathrectangle{\pgfqpoint{0.481978in}{0.331635in}}{\pgfqpoint{4.960000in}{3.696000in}}%
\pgfusepath{clip}%
\pgfsetrectcap%
\pgfsetroundjoin%
\pgfsetlinewidth{1.505625pt}%
\definecolor{currentstroke}{rgb}{0.631373,0.788235,0.956863}%
\pgfsetstrokecolor{currentstroke}%
\pgfsetstrokeopacity{0.200000}%
\pgfsetdash{}{0pt}%
\pgfpathmoveto{\pgfqpoint{4.103401in}{3.491226in}}%
\pgfpathlineto{\pgfqpoint{3.305661in}{2.636178in}}%
\pgfusepath{stroke}%
\end{pgfscope}%
\begin{pgfscope}%
\pgfpathrectangle{\pgfqpoint{0.481978in}{0.331635in}}{\pgfqpoint{4.960000in}{3.696000in}}%
\pgfusepath{clip}%
\pgfsetrectcap%
\pgfsetroundjoin%
\pgfsetlinewidth{1.505625pt}%
\definecolor{currentstroke}{rgb}{0.631373,0.788235,0.956863}%
\pgfsetstrokecolor{currentstroke}%
\pgfsetstrokeopacity{0.200000}%
\pgfsetdash{}{0pt}%
\pgfpathmoveto{\pgfqpoint{2.742987in}{3.079584in}}%
\pgfpathlineto{\pgfqpoint{3.305661in}{2.636178in}}%
\pgfusepath{stroke}%
\end{pgfscope}%
\begin{pgfscope}%
\pgfpathrectangle{\pgfqpoint{0.481978in}{0.331635in}}{\pgfqpoint{4.960000in}{3.696000in}}%
\pgfusepath{clip}%
\pgfsetrectcap%
\pgfsetroundjoin%
\pgfsetlinewidth{1.505625pt}%
\definecolor{currentstroke}{rgb}{0.631373,0.788235,0.956863}%
\pgfsetstrokecolor{currentstroke}%
\pgfsetstrokeopacity{0.200000}%
\pgfsetdash{}{0pt}%
\pgfpathmoveto{\pgfqpoint{5.163838in}{1.969015in}}%
\pgfpathlineto{\pgfqpoint{3.305661in}{2.636178in}}%
\pgfusepath{stroke}%
\end{pgfscope}%
\begin{pgfscope}%
\pgfpathrectangle{\pgfqpoint{0.481978in}{0.331635in}}{\pgfqpoint{4.960000in}{3.696000in}}%
\pgfusepath{clip}%
\pgfsetrectcap%
\pgfsetroundjoin%
\pgfsetlinewidth{1.505625pt}%
\definecolor{currentstroke}{rgb}{0.631373,0.788235,0.956863}%
\pgfsetstrokecolor{currentstroke}%
\pgfsetstrokeopacity{0.200000}%
\pgfsetdash{}{0pt}%
\pgfpathmoveto{\pgfqpoint{4.188026in}{1.946598in}}%
\pgfpathlineto{\pgfqpoint{3.305661in}{2.636178in}}%
\pgfusepath{stroke}%
\end{pgfscope}%
\begin{pgfscope}%
\pgfpathrectangle{\pgfqpoint{0.481978in}{0.331635in}}{\pgfqpoint{4.960000in}{3.696000in}}%
\pgfusepath{clip}%
\pgfsetrectcap%
\pgfsetroundjoin%
\pgfsetlinewidth{1.505625pt}%
\definecolor{currentstroke}{rgb}{0.631373,0.788235,0.956863}%
\pgfsetstrokecolor{currentstroke}%
\pgfsetstrokeopacity{0.200000}%
\pgfsetdash{}{0pt}%
\pgfpathmoveto{\pgfqpoint{2.239764in}{3.012237in}}%
\pgfpathlineto{\pgfqpoint{3.305661in}{2.636178in}}%
\pgfusepath{stroke}%
\end{pgfscope}%
\begin{pgfscope}%
\pgfpathrectangle{\pgfqpoint{0.481978in}{0.331635in}}{\pgfqpoint{4.960000in}{3.696000in}}%
\pgfusepath{clip}%
\pgfsetrectcap%
\pgfsetroundjoin%
\pgfsetlinewidth{1.505625pt}%
\definecolor{currentstroke}{rgb}{0.631373,0.788235,0.956863}%
\pgfsetstrokecolor{currentstroke}%
\pgfsetstrokeopacity{0.200000}%
\pgfsetdash{}{0pt}%
\pgfpathmoveto{\pgfqpoint{4.178036in}{2.717076in}}%
\pgfpathlineto{\pgfqpoint{3.305661in}{2.636178in}}%
\pgfusepath{stroke}%
\end{pgfscope}%
\begin{pgfscope}%
\pgfpathrectangle{\pgfqpoint{0.481978in}{0.331635in}}{\pgfqpoint{4.960000in}{3.696000in}}%
\pgfusepath{clip}%
\pgfsetrectcap%
\pgfsetroundjoin%
\pgfsetlinewidth{1.505625pt}%
\definecolor{currentstroke}{rgb}{0.631373,0.788235,0.956863}%
\pgfsetstrokecolor{currentstroke}%
\pgfsetstrokeopacity{0.200000}%
\pgfsetdash{}{0pt}%
\pgfpathmoveto{\pgfqpoint{1.988321in}{2.979906in}}%
\pgfpathlineto{\pgfqpoint{3.305661in}{2.636178in}}%
\pgfusepath{stroke}%
\end{pgfscope}%
\begin{pgfscope}%
\pgfpathrectangle{\pgfqpoint{0.481978in}{0.331635in}}{\pgfqpoint{4.960000in}{3.696000in}}%
\pgfusepath{clip}%
\pgfsetrectcap%
\pgfsetroundjoin%
\pgfsetlinewidth{1.505625pt}%
\definecolor{currentstroke}{rgb}{0.631373,0.788235,0.956863}%
\pgfsetstrokecolor{currentstroke}%
\pgfsetstrokeopacity{0.200000}%
\pgfsetdash{}{0pt}%
\pgfpathmoveto{\pgfqpoint{3.750673in}{3.175333in}}%
\pgfpathlineto{\pgfqpoint{3.305661in}{2.636178in}}%
\pgfusepath{stroke}%
\end{pgfscope}%
\begin{pgfscope}%
\pgfpathrectangle{\pgfqpoint{0.481978in}{0.331635in}}{\pgfqpoint{4.960000in}{3.696000in}}%
\pgfusepath{clip}%
\pgfsetrectcap%
\pgfsetroundjoin%
\pgfsetlinewidth{1.505625pt}%
\definecolor{currentstroke}{rgb}{0.631373,0.788235,0.956863}%
\pgfsetstrokecolor{currentstroke}%
\pgfsetstrokeopacity{0.200000}%
\pgfsetdash{}{0pt}%
\pgfpathmoveto{\pgfqpoint{3.215015in}{3.147310in}}%
\pgfpathlineto{\pgfqpoint{3.305661in}{2.636178in}}%
\pgfusepath{stroke}%
\end{pgfscope}%
\begin{pgfscope}%
\pgfpathrectangle{\pgfqpoint{0.481978in}{0.331635in}}{\pgfqpoint{4.960000in}{3.696000in}}%
\pgfusepath{clip}%
\pgfsetrectcap%
\pgfsetroundjoin%
\pgfsetlinewidth{1.505625pt}%
\definecolor{currentstroke}{rgb}{0.631373,0.788235,0.956863}%
\pgfsetstrokecolor{currentstroke}%
\pgfsetstrokeopacity{0.200000}%
\pgfsetdash{}{0pt}%
\pgfpathmoveto{\pgfqpoint{2.941688in}{2.640546in}}%
\pgfpathlineto{\pgfqpoint{3.305661in}{2.636178in}}%
\pgfusepath{stroke}%
\end{pgfscope}%
\begin{pgfscope}%
\pgfpathrectangle{\pgfqpoint{0.481978in}{0.331635in}}{\pgfqpoint{4.960000in}{3.696000in}}%
\pgfusepath{clip}%
\pgfsetrectcap%
\pgfsetroundjoin%
\pgfsetlinewidth{1.505625pt}%
\definecolor{currentstroke}{rgb}{0.631373,0.788235,0.956863}%
\pgfsetstrokecolor{currentstroke}%
\pgfsetstrokeopacity{0.200000}%
\pgfsetdash{}{0pt}%
\pgfpathmoveto{\pgfqpoint{1.596447in}{2.926709in}}%
\pgfpathlineto{\pgfqpoint{3.305661in}{2.636178in}}%
\pgfusepath{stroke}%
\end{pgfscope}%
\begin{pgfscope}%
\pgfpathrectangle{\pgfqpoint{0.481978in}{0.331635in}}{\pgfqpoint{4.960000in}{3.696000in}}%
\pgfusepath{clip}%
\pgfsetrectcap%
\pgfsetroundjoin%
\pgfsetlinewidth{1.505625pt}%
\definecolor{currentstroke}{rgb}{0.631373,0.788235,0.956863}%
\pgfsetstrokecolor{currentstroke}%
\pgfsetstrokeopacity{0.200000}%
\pgfsetdash{}{0pt}%
\pgfpathmoveto{\pgfqpoint{4.684035in}{1.686443in}}%
\pgfpathlineto{\pgfqpoint{3.305661in}{2.636178in}}%
\pgfusepath{stroke}%
\end{pgfscope}%
\begin{pgfscope}%
\pgfpathrectangle{\pgfqpoint{0.481978in}{0.331635in}}{\pgfqpoint{4.960000in}{3.696000in}}%
\pgfusepath{clip}%
\pgfsetrectcap%
\pgfsetroundjoin%
\pgfsetlinewidth{1.505625pt}%
\definecolor{currentstroke}{rgb}{0.631373,0.788235,0.956863}%
\pgfsetstrokecolor{currentstroke}%
\pgfsetstrokeopacity{0.200000}%
\pgfsetdash{}{0pt}%
\pgfpathmoveto{\pgfqpoint{3.104244in}{3.674158in}}%
\pgfpathlineto{\pgfqpoint{3.305661in}{2.636178in}}%
\pgfusepath{stroke}%
\end{pgfscope}%
\begin{pgfscope}%
\pgfpathrectangle{\pgfqpoint{0.481978in}{0.331635in}}{\pgfqpoint{4.960000in}{3.696000in}}%
\pgfusepath{clip}%
\pgfsetrectcap%
\pgfsetroundjoin%
\pgfsetlinewidth{1.505625pt}%
\definecolor{currentstroke}{rgb}{0.631373,0.788235,0.956863}%
\pgfsetstrokecolor{currentstroke}%
\pgfsetstrokeopacity{0.200000}%
\pgfsetdash{}{0pt}%
\pgfpathmoveto{\pgfqpoint{3.784793in}{3.388975in}}%
\pgfpathlineto{\pgfqpoint{3.305661in}{2.636178in}}%
\pgfusepath{stroke}%
\end{pgfscope}%
\begin{pgfscope}%
\pgfpathrectangle{\pgfqpoint{0.481978in}{0.331635in}}{\pgfqpoint{4.960000in}{3.696000in}}%
\pgfusepath{clip}%
\pgfsetrectcap%
\pgfsetroundjoin%
\pgfsetlinewidth{1.505625pt}%
\definecolor{currentstroke}{rgb}{0.631373,0.788235,0.956863}%
\pgfsetstrokecolor{currentstroke}%
\pgfsetstrokeopacity{0.200000}%
\pgfsetdash{}{0pt}%
\pgfpathmoveto{\pgfqpoint{3.316835in}{3.186661in}}%
\pgfpathlineto{\pgfqpoint{3.305661in}{2.636178in}}%
\pgfusepath{stroke}%
\end{pgfscope}%
\begin{pgfscope}%
\pgfpathrectangle{\pgfqpoint{0.481978in}{0.331635in}}{\pgfqpoint{4.960000in}{3.696000in}}%
\pgfusepath{clip}%
\pgfsetrectcap%
\pgfsetroundjoin%
\pgfsetlinewidth{1.505625pt}%
\definecolor{currentstroke}{rgb}{0.631373,0.788235,0.956863}%
\pgfsetstrokecolor{currentstroke}%
\pgfsetstrokeopacity{0.200000}%
\pgfsetdash{}{0pt}%
\pgfpathmoveto{\pgfqpoint{4.721051in}{3.371868in}}%
\pgfpathlineto{\pgfqpoint{3.305661in}{2.636178in}}%
\pgfusepath{stroke}%
\end{pgfscope}%
\begin{pgfscope}%
\pgfpathrectangle{\pgfqpoint{0.481978in}{0.331635in}}{\pgfqpoint{4.960000in}{3.696000in}}%
\pgfusepath{clip}%
\pgfsetrectcap%
\pgfsetroundjoin%
\pgfsetlinewidth{1.505625pt}%
\definecolor{currentstroke}{rgb}{0.631373,0.788235,0.956863}%
\pgfsetstrokecolor{currentstroke}%
\pgfsetstrokeopacity{0.200000}%
\pgfsetdash{}{0pt}%
\pgfpathmoveto{\pgfqpoint{3.034645in}{3.374257in}}%
\pgfpathlineto{\pgfqpoint{3.305661in}{2.636178in}}%
\pgfusepath{stroke}%
\end{pgfscope}%
\begin{pgfscope}%
\pgfpathrectangle{\pgfqpoint{0.481978in}{0.331635in}}{\pgfqpoint{4.960000in}{3.696000in}}%
\pgfusepath{clip}%
\pgfsetrectcap%
\pgfsetroundjoin%
\pgfsetlinewidth{1.505625pt}%
\definecolor{currentstroke}{rgb}{0.631373,0.788235,0.956863}%
\pgfsetstrokecolor{currentstroke}%
\pgfsetstrokeopacity{0.200000}%
\pgfsetdash{}{0pt}%
\pgfpathmoveto{\pgfqpoint{3.045472in}{2.534602in}}%
\pgfpathlineto{\pgfqpoint{3.305661in}{2.636178in}}%
\pgfusepath{stroke}%
\end{pgfscope}%
\begin{pgfscope}%
\pgfpathrectangle{\pgfqpoint{0.481978in}{0.331635in}}{\pgfqpoint{4.960000in}{3.696000in}}%
\pgfusepath{clip}%
\pgfsetrectcap%
\pgfsetroundjoin%
\pgfsetlinewidth{1.505625pt}%
\definecolor{currentstroke}{rgb}{0.631373,0.788235,0.956863}%
\pgfsetstrokecolor{currentstroke}%
\pgfsetstrokeopacity{0.200000}%
\pgfsetdash{}{0pt}%
\pgfpathmoveto{\pgfqpoint{2.612555in}{2.101833in}}%
\pgfpathlineto{\pgfqpoint{3.305661in}{2.636178in}}%
\pgfusepath{stroke}%
\end{pgfscope}%
\begin{pgfscope}%
\pgfpathrectangle{\pgfqpoint{0.481978in}{0.331635in}}{\pgfqpoint{4.960000in}{3.696000in}}%
\pgfusepath{clip}%
\pgfsetrectcap%
\pgfsetroundjoin%
\pgfsetlinewidth{1.505625pt}%
\definecolor{currentstroke}{rgb}{0.631373,0.788235,0.956863}%
\pgfsetstrokecolor{currentstroke}%
\pgfsetstrokeopacity{0.200000}%
\pgfsetdash{}{0pt}%
\pgfpathmoveto{\pgfqpoint{4.340254in}{1.790299in}}%
\pgfpathlineto{\pgfqpoint{3.305661in}{2.636178in}}%
\pgfusepath{stroke}%
\end{pgfscope}%
\begin{pgfscope}%
\pgfpathrectangle{\pgfqpoint{0.481978in}{0.331635in}}{\pgfqpoint{4.960000in}{3.696000in}}%
\pgfusepath{clip}%
\pgfsetrectcap%
\pgfsetroundjoin%
\pgfsetlinewidth{1.505625pt}%
\definecolor{currentstroke}{rgb}{0.631373,0.788235,0.956863}%
\pgfsetstrokecolor{currentstroke}%
\pgfsetstrokeopacity{0.200000}%
\pgfsetdash{}{0pt}%
\pgfpathmoveto{\pgfqpoint{2.740137in}{2.339195in}}%
\pgfpathlineto{\pgfqpoint{3.305661in}{2.636178in}}%
\pgfusepath{stroke}%
\end{pgfscope}%
\begin{pgfscope}%
\pgfpathrectangle{\pgfqpoint{0.481978in}{0.331635in}}{\pgfqpoint{4.960000in}{3.696000in}}%
\pgfusepath{clip}%
\pgfsetrectcap%
\pgfsetroundjoin%
\pgfsetlinewidth{1.505625pt}%
\definecolor{currentstroke}{rgb}{0.631373,0.788235,0.956863}%
\pgfsetstrokecolor{currentstroke}%
\pgfsetstrokeopacity{0.200000}%
\pgfsetdash{}{0pt}%
\pgfpathmoveto{\pgfqpoint{2.185010in}{2.580732in}}%
\pgfpathlineto{\pgfqpoint{3.305661in}{2.636178in}}%
\pgfusepath{stroke}%
\end{pgfscope}%
\begin{pgfscope}%
\pgfpathrectangle{\pgfqpoint{0.481978in}{0.331635in}}{\pgfqpoint{4.960000in}{3.696000in}}%
\pgfusepath{clip}%
\pgfsetrectcap%
\pgfsetroundjoin%
\pgfsetlinewidth{1.505625pt}%
\definecolor{currentstroke}{rgb}{0.631373,0.788235,0.956863}%
\pgfsetstrokecolor{currentstroke}%
\pgfsetstrokeopacity{0.200000}%
\pgfsetdash{}{0pt}%
\pgfpathmoveto{\pgfqpoint{5.216523in}{1.945342in}}%
\pgfpathlineto{\pgfqpoint{3.305661in}{2.636178in}}%
\pgfusepath{stroke}%
\end{pgfscope}%
\begin{pgfscope}%
\pgfpathrectangle{\pgfqpoint{0.481978in}{0.331635in}}{\pgfqpoint{4.960000in}{3.696000in}}%
\pgfusepath{clip}%
\pgfsetrectcap%
\pgfsetroundjoin%
\pgfsetlinewidth{1.505625pt}%
\definecolor{currentstroke}{rgb}{0.631373,0.788235,0.956863}%
\pgfsetstrokecolor{currentstroke}%
\pgfsetstrokeopacity{0.200000}%
\pgfsetdash{}{0pt}%
\pgfpathmoveto{\pgfqpoint{2.950569in}{1.721541in}}%
\pgfpathlineto{\pgfqpoint{3.305661in}{2.636178in}}%
\pgfusepath{stroke}%
\end{pgfscope}%
\begin{pgfscope}%
\pgfpathrectangle{\pgfqpoint{0.481978in}{0.331635in}}{\pgfqpoint{4.960000in}{3.696000in}}%
\pgfusepath{clip}%
\pgfsetrectcap%
\pgfsetroundjoin%
\pgfsetlinewidth{1.505625pt}%
\definecolor{currentstroke}{rgb}{0.631373,0.788235,0.956863}%
\pgfsetstrokecolor{currentstroke}%
\pgfsetstrokeopacity{0.200000}%
\pgfsetdash{}{0pt}%
\pgfpathmoveto{\pgfqpoint{2.746965in}{3.205769in}}%
\pgfpathlineto{\pgfqpoint{3.305661in}{2.636178in}}%
\pgfusepath{stroke}%
\end{pgfscope}%
\begin{pgfscope}%
\pgfpathrectangle{\pgfqpoint{0.481978in}{0.331635in}}{\pgfqpoint{4.960000in}{3.696000in}}%
\pgfusepath{clip}%
\pgfsetrectcap%
\pgfsetroundjoin%
\pgfsetlinewidth{1.505625pt}%
\definecolor{currentstroke}{rgb}{0.631373,0.788235,0.956863}%
\pgfsetstrokecolor{currentstroke}%
\pgfsetstrokeopacity{0.200000}%
\pgfsetdash{}{0pt}%
\pgfpathmoveto{\pgfqpoint{3.493134in}{3.015903in}}%
\pgfpathlineto{\pgfqpoint{3.305661in}{2.636178in}}%
\pgfusepath{stroke}%
\end{pgfscope}%
\begin{pgfscope}%
\pgfpathrectangle{\pgfqpoint{0.481978in}{0.331635in}}{\pgfqpoint{4.960000in}{3.696000in}}%
\pgfusepath{clip}%
\pgfsetrectcap%
\pgfsetroundjoin%
\pgfsetlinewidth{1.505625pt}%
\definecolor{currentstroke}{rgb}{0.631373,0.788235,0.956863}%
\pgfsetstrokecolor{currentstroke}%
\pgfsetstrokeopacity{0.200000}%
\pgfsetdash{}{0pt}%
\pgfpathmoveto{\pgfqpoint{2.306544in}{2.192039in}}%
\pgfpathlineto{\pgfqpoint{3.305661in}{2.636178in}}%
\pgfusepath{stroke}%
\end{pgfscope}%
\begin{pgfscope}%
\pgfpathrectangle{\pgfqpoint{0.481978in}{0.331635in}}{\pgfqpoint{4.960000in}{3.696000in}}%
\pgfusepath{clip}%
\pgfsetrectcap%
\pgfsetroundjoin%
\pgfsetlinewidth{1.505625pt}%
\definecolor{currentstroke}{rgb}{0.631373,0.788235,0.956863}%
\pgfsetstrokecolor{currentstroke}%
\pgfsetstrokeopacity{0.200000}%
\pgfsetdash{}{0pt}%
\pgfpathmoveto{\pgfqpoint{2.610617in}{2.147672in}}%
\pgfpathlineto{\pgfqpoint{3.305661in}{2.636178in}}%
\pgfusepath{stroke}%
\end{pgfscope}%
\begin{pgfscope}%
\pgfpathrectangle{\pgfqpoint{0.481978in}{0.331635in}}{\pgfqpoint{4.960000in}{3.696000in}}%
\pgfusepath{clip}%
\pgfsetrectcap%
\pgfsetroundjoin%
\pgfsetlinewidth{1.505625pt}%
\definecolor{currentstroke}{rgb}{0.631373,0.788235,0.956863}%
\pgfsetstrokecolor{currentstroke}%
\pgfsetstrokeopacity{0.200000}%
\pgfsetdash{}{0pt}%
\pgfpathmoveto{\pgfqpoint{3.281173in}{2.564371in}}%
\pgfpathlineto{\pgfqpoint{3.305661in}{2.636178in}}%
\pgfusepath{stroke}%
\end{pgfscope}%
\begin{pgfscope}%
\pgfpathrectangle{\pgfqpoint{0.481978in}{0.331635in}}{\pgfqpoint{4.960000in}{3.696000in}}%
\pgfusepath{clip}%
\pgfsetrectcap%
\pgfsetroundjoin%
\pgfsetlinewidth{1.505625pt}%
\definecolor{currentstroke}{rgb}{0.631373,0.788235,0.956863}%
\pgfsetstrokecolor{currentstroke}%
\pgfsetstrokeopacity{0.200000}%
\pgfsetdash{}{0pt}%
\pgfpathmoveto{\pgfqpoint{3.154017in}{3.292168in}}%
\pgfpathlineto{\pgfqpoint{3.305661in}{2.636178in}}%
\pgfusepath{stroke}%
\end{pgfscope}%
\begin{pgfscope}%
\pgfpathrectangle{\pgfqpoint{0.481978in}{0.331635in}}{\pgfqpoint{4.960000in}{3.696000in}}%
\pgfusepath{clip}%
\pgfsetrectcap%
\pgfsetroundjoin%
\pgfsetlinewidth{1.505625pt}%
\definecolor{currentstroke}{rgb}{0.631373,0.788235,0.956863}%
\pgfsetstrokecolor{currentstroke}%
\pgfsetstrokeopacity{0.200000}%
\pgfsetdash{}{0pt}%
\pgfpathmoveto{\pgfqpoint{2.374813in}{2.457434in}}%
\pgfpathlineto{\pgfqpoint{3.305661in}{2.636178in}}%
\pgfusepath{stroke}%
\end{pgfscope}%
\begin{pgfscope}%
\pgfpathrectangle{\pgfqpoint{0.481978in}{0.331635in}}{\pgfqpoint{4.960000in}{3.696000in}}%
\pgfusepath{clip}%
\pgfsetrectcap%
\pgfsetroundjoin%
\pgfsetlinewidth{1.505625pt}%
\definecolor{currentstroke}{rgb}{0.631373,0.788235,0.956863}%
\pgfsetstrokecolor{currentstroke}%
\pgfsetstrokeopacity{0.200000}%
\pgfsetdash{}{0pt}%
\pgfpathmoveto{\pgfqpoint{3.516196in}{3.130110in}}%
\pgfpathlineto{\pgfqpoint{3.305661in}{2.636178in}}%
\pgfusepath{stroke}%
\end{pgfscope}%
\begin{pgfscope}%
\pgfpathrectangle{\pgfqpoint{0.481978in}{0.331635in}}{\pgfqpoint{4.960000in}{3.696000in}}%
\pgfusepath{clip}%
\pgfsetrectcap%
\pgfsetroundjoin%
\pgfsetlinewidth{1.505625pt}%
\definecolor{currentstroke}{rgb}{0.631373,0.788235,0.956863}%
\pgfsetstrokecolor{currentstroke}%
\pgfsetstrokeopacity{0.200000}%
\pgfsetdash{}{0pt}%
\pgfpathmoveto{\pgfqpoint{2.937519in}{3.085642in}}%
\pgfpathlineto{\pgfqpoint{3.305661in}{2.636178in}}%
\pgfusepath{stroke}%
\end{pgfscope}%
\begin{pgfscope}%
\pgfpathrectangle{\pgfqpoint{0.481978in}{0.331635in}}{\pgfqpoint{4.960000in}{3.696000in}}%
\pgfusepath{clip}%
\pgfsetrectcap%
\pgfsetroundjoin%
\pgfsetlinewidth{1.505625pt}%
\definecolor{currentstroke}{rgb}{0.631373,0.788235,0.956863}%
\pgfsetstrokecolor{currentstroke}%
\pgfsetstrokeopacity{0.200000}%
\pgfsetdash{}{0pt}%
\pgfpathmoveto{\pgfqpoint{2.518830in}{2.571396in}}%
\pgfpathlineto{\pgfqpoint{3.305661in}{2.636178in}}%
\pgfusepath{stroke}%
\end{pgfscope}%
\begin{pgfscope}%
\pgfpathrectangle{\pgfqpoint{0.481978in}{0.331635in}}{\pgfqpoint{4.960000in}{3.696000in}}%
\pgfusepath{clip}%
\pgfsetrectcap%
\pgfsetroundjoin%
\pgfsetlinewidth{1.505625pt}%
\definecolor{currentstroke}{rgb}{0.631373,0.788235,0.956863}%
\pgfsetstrokecolor{currentstroke}%
\pgfsetstrokeopacity{0.200000}%
\pgfsetdash{}{0pt}%
\pgfpathmoveto{\pgfqpoint{2.552065in}{2.793836in}}%
\pgfpathlineto{\pgfqpoint{3.305661in}{2.636178in}}%
\pgfusepath{stroke}%
\end{pgfscope}%
\begin{pgfscope}%
\pgfpathrectangle{\pgfqpoint{0.481978in}{0.331635in}}{\pgfqpoint{4.960000in}{3.696000in}}%
\pgfusepath{clip}%
\pgfsetrectcap%
\pgfsetroundjoin%
\pgfsetlinewidth{1.505625pt}%
\definecolor{currentstroke}{rgb}{0.631373,0.788235,0.956863}%
\pgfsetstrokecolor{currentstroke}%
\pgfsetstrokeopacity{0.200000}%
\pgfsetdash{}{0pt}%
\pgfpathmoveto{\pgfqpoint{3.713491in}{3.573856in}}%
\pgfpathlineto{\pgfqpoint{3.305661in}{2.636178in}}%
\pgfusepath{stroke}%
\end{pgfscope}%
\begin{pgfscope}%
\pgfpathrectangle{\pgfqpoint{0.481978in}{0.331635in}}{\pgfqpoint{4.960000in}{3.696000in}}%
\pgfusepath{clip}%
\pgfsetrectcap%
\pgfsetroundjoin%
\pgfsetlinewidth{1.505625pt}%
\definecolor{currentstroke}{rgb}{0.631373,0.788235,0.956863}%
\pgfsetstrokecolor{currentstroke}%
\pgfsetstrokeopacity{0.200000}%
\pgfsetdash{}{0pt}%
\pgfpathmoveto{\pgfqpoint{4.462105in}{1.711185in}}%
\pgfpathlineto{\pgfqpoint{3.305661in}{2.636178in}}%
\pgfusepath{stroke}%
\end{pgfscope}%
\begin{pgfscope}%
\pgfpathrectangle{\pgfqpoint{0.481978in}{0.331635in}}{\pgfqpoint{4.960000in}{3.696000in}}%
\pgfusepath{clip}%
\pgfsetrectcap%
\pgfsetroundjoin%
\pgfsetlinewidth{1.505625pt}%
\definecolor{currentstroke}{rgb}{0.631373,0.788235,0.956863}%
\pgfsetstrokecolor{currentstroke}%
\pgfsetstrokeopacity{0.200000}%
\pgfsetdash{}{0pt}%
\pgfpathmoveto{\pgfqpoint{2.614726in}{2.918291in}}%
\pgfpathlineto{\pgfqpoint{3.305661in}{2.636178in}}%
\pgfusepath{stroke}%
\end{pgfscope}%
\begin{pgfscope}%
\pgfpathrectangle{\pgfqpoint{0.481978in}{0.331635in}}{\pgfqpoint{4.960000in}{3.696000in}}%
\pgfusepath{clip}%
\pgfsetrectcap%
\pgfsetroundjoin%
\pgfsetlinewidth{1.505625pt}%
\definecolor{currentstroke}{rgb}{0.631373,0.788235,0.956863}%
\pgfsetstrokecolor{currentstroke}%
\pgfsetstrokeopacity{0.200000}%
\pgfsetdash{}{0pt}%
\pgfpathmoveto{\pgfqpoint{4.943248in}{1.926753in}}%
\pgfpathlineto{\pgfqpoint{3.305661in}{2.636178in}}%
\pgfusepath{stroke}%
\end{pgfscope}%
\begin{pgfscope}%
\pgfpathrectangle{\pgfqpoint{0.481978in}{0.331635in}}{\pgfqpoint{4.960000in}{3.696000in}}%
\pgfusepath{clip}%
\pgfsetrectcap%
\pgfsetroundjoin%
\pgfsetlinewidth{1.505625pt}%
\definecolor{currentstroke}{rgb}{0.631373,0.788235,0.956863}%
\pgfsetstrokecolor{currentstroke}%
\pgfsetstrokeopacity{0.200000}%
\pgfsetdash{}{0pt}%
\pgfpathmoveto{\pgfqpoint{3.155268in}{3.762122in}}%
\pgfpathlineto{\pgfqpoint{3.305661in}{2.636178in}}%
\pgfusepath{stroke}%
\end{pgfscope}%
\begin{pgfscope}%
\pgfpathrectangle{\pgfqpoint{0.481978in}{0.331635in}}{\pgfqpoint{4.960000in}{3.696000in}}%
\pgfusepath{clip}%
\pgfsetrectcap%
\pgfsetroundjoin%
\pgfsetlinewidth{1.505625pt}%
\definecolor{currentstroke}{rgb}{0.631373,0.788235,0.956863}%
\pgfsetstrokecolor{currentstroke}%
\pgfsetstrokeopacity{0.200000}%
\pgfsetdash{}{0pt}%
\pgfpathmoveto{\pgfqpoint{3.161490in}{2.936376in}}%
\pgfpathlineto{\pgfqpoint{3.305661in}{2.636178in}}%
\pgfusepath{stroke}%
\end{pgfscope}%
\begin{pgfscope}%
\pgfpathrectangle{\pgfqpoint{0.481978in}{0.331635in}}{\pgfqpoint{4.960000in}{3.696000in}}%
\pgfusepath{clip}%
\pgfsetrectcap%
\pgfsetroundjoin%
\pgfsetlinewidth{1.505625pt}%
\definecolor{currentstroke}{rgb}{0.631373,0.788235,0.956863}%
\pgfsetstrokecolor{currentstroke}%
\pgfsetstrokeopacity{0.200000}%
\pgfsetdash{}{0pt}%
\pgfpathmoveto{\pgfqpoint{3.406915in}{1.838732in}}%
\pgfpathlineto{\pgfqpoint{3.305661in}{2.636178in}}%
\pgfusepath{stroke}%
\end{pgfscope}%
\begin{pgfscope}%
\pgfpathrectangle{\pgfqpoint{0.481978in}{0.331635in}}{\pgfqpoint{4.960000in}{3.696000in}}%
\pgfusepath{clip}%
\pgfsetrectcap%
\pgfsetroundjoin%
\pgfsetlinewidth{1.505625pt}%
\definecolor{currentstroke}{rgb}{0.631373,0.788235,0.956863}%
\pgfsetstrokecolor{currentstroke}%
\pgfsetstrokeopacity{0.200000}%
\pgfsetdash{}{0pt}%
\pgfpathmoveto{\pgfqpoint{1.632626in}{2.740393in}}%
\pgfpathlineto{\pgfqpoint{3.305661in}{2.636178in}}%
\pgfusepath{stroke}%
\end{pgfscope}%
\begin{pgfscope}%
\pgfpathrectangle{\pgfqpoint{0.481978in}{0.331635in}}{\pgfqpoint{4.960000in}{3.696000in}}%
\pgfusepath{clip}%
\pgfsetrectcap%
\pgfsetroundjoin%
\pgfsetlinewidth{1.505625pt}%
\definecolor{currentstroke}{rgb}{0.631373,0.788235,0.956863}%
\pgfsetstrokecolor{currentstroke}%
\pgfsetstrokeopacity{0.200000}%
\pgfsetdash{}{0pt}%
\pgfpathmoveto{\pgfqpoint{4.178469in}{2.433234in}}%
\pgfpathlineto{\pgfqpoint{3.305661in}{2.636178in}}%
\pgfusepath{stroke}%
\end{pgfscope}%
\begin{pgfscope}%
\pgfpathrectangle{\pgfqpoint{0.481978in}{0.331635in}}{\pgfqpoint{4.960000in}{3.696000in}}%
\pgfusepath{clip}%
\pgfsetrectcap%
\pgfsetroundjoin%
\pgfsetlinewidth{1.505625pt}%
\definecolor{currentstroke}{rgb}{0.631373,0.788235,0.956863}%
\pgfsetstrokecolor{currentstroke}%
\pgfsetstrokeopacity{0.200000}%
\pgfsetdash{}{0pt}%
\pgfpathmoveto{\pgfqpoint{4.067149in}{3.565646in}}%
\pgfpathlineto{\pgfqpoint{3.305661in}{2.636178in}}%
\pgfusepath{stroke}%
\end{pgfscope}%
\begin{pgfscope}%
\pgfpathrectangle{\pgfqpoint{0.481978in}{0.331635in}}{\pgfqpoint{4.960000in}{3.696000in}}%
\pgfusepath{clip}%
\pgfsetrectcap%
\pgfsetroundjoin%
\pgfsetlinewidth{1.505625pt}%
\definecolor{currentstroke}{rgb}{0.631373,0.788235,0.956863}%
\pgfsetstrokecolor{currentstroke}%
\pgfsetstrokeopacity{0.200000}%
\pgfsetdash{}{0pt}%
\pgfpathmoveto{\pgfqpoint{3.848878in}{2.455447in}}%
\pgfpathlineto{\pgfqpoint{3.305661in}{2.636178in}}%
\pgfusepath{stroke}%
\end{pgfscope}%
\begin{pgfscope}%
\pgfpathrectangle{\pgfqpoint{0.481978in}{0.331635in}}{\pgfqpoint{4.960000in}{3.696000in}}%
\pgfusepath{clip}%
\pgfsetrectcap%
\pgfsetroundjoin%
\pgfsetlinewidth{1.505625pt}%
\definecolor{currentstroke}{rgb}{0.631373,0.788235,0.956863}%
\pgfsetstrokecolor{currentstroke}%
\pgfsetstrokeopacity{0.200000}%
\pgfsetdash{}{0pt}%
\pgfpathmoveto{\pgfqpoint{4.547031in}{1.695528in}}%
\pgfpathlineto{\pgfqpoint{3.305661in}{2.636178in}}%
\pgfusepath{stroke}%
\end{pgfscope}%
\begin{pgfscope}%
\pgfpathrectangle{\pgfqpoint{0.481978in}{0.331635in}}{\pgfqpoint{4.960000in}{3.696000in}}%
\pgfusepath{clip}%
\pgfsetrectcap%
\pgfsetroundjoin%
\pgfsetlinewidth{1.505625pt}%
\definecolor{currentstroke}{rgb}{0.631373,0.788235,0.956863}%
\pgfsetstrokecolor{currentstroke}%
\pgfsetstrokeopacity{0.200000}%
\pgfsetdash{}{0pt}%
\pgfpathmoveto{\pgfqpoint{4.126614in}{1.752322in}}%
\pgfpathlineto{\pgfqpoint{3.305661in}{2.636178in}}%
\pgfusepath{stroke}%
\end{pgfscope}%
\begin{pgfscope}%
\pgfpathrectangle{\pgfqpoint{0.481978in}{0.331635in}}{\pgfqpoint{4.960000in}{3.696000in}}%
\pgfusepath{clip}%
\pgfsetrectcap%
\pgfsetroundjoin%
\pgfsetlinewidth{1.505625pt}%
\definecolor{currentstroke}{rgb}{0.631373,0.788235,0.956863}%
\pgfsetstrokecolor{currentstroke}%
\pgfsetstrokeopacity{0.200000}%
\pgfsetdash{}{0pt}%
\pgfpathmoveto{\pgfqpoint{2.866000in}{1.929116in}}%
\pgfpathlineto{\pgfqpoint{3.305661in}{2.636178in}}%
\pgfusepath{stroke}%
\end{pgfscope}%
\begin{pgfscope}%
\pgfpathrectangle{\pgfqpoint{0.481978in}{0.331635in}}{\pgfqpoint{4.960000in}{3.696000in}}%
\pgfusepath{clip}%
\pgfsetrectcap%
\pgfsetroundjoin%
\pgfsetlinewidth{1.505625pt}%
\definecolor{currentstroke}{rgb}{0.631373,0.788235,0.956863}%
\pgfsetstrokecolor{currentstroke}%
\pgfsetstrokeopacity{0.200000}%
\pgfsetdash{}{0pt}%
\pgfpathmoveto{\pgfqpoint{3.835758in}{2.641497in}}%
\pgfpathlineto{\pgfqpoint{3.305661in}{2.636178in}}%
\pgfusepath{stroke}%
\end{pgfscope}%
\begin{pgfscope}%
\pgfpathrectangle{\pgfqpoint{0.481978in}{0.331635in}}{\pgfqpoint{4.960000in}{3.696000in}}%
\pgfusepath{clip}%
\pgfsetrectcap%
\pgfsetroundjoin%
\pgfsetlinewidth{1.505625pt}%
\definecolor{currentstroke}{rgb}{0.631373,0.788235,0.956863}%
\pgfsetstrokecolor{currentstroke}%
\pgfsetstrokeopacity{0.200000}%
\pgfsetdash{}{0pt}%
\pgfpathmoveto{\pgfqpoint{1.829229in}{2.205047in}}%
\pgfpathlineto{\pgfqpoint{3.305661in}{2.636178in}}%
\pgfusepath{stroke}%
\end{pgfscope}%
\begin{pgfscope}%
\pgfpathrectangle{\pgfqpoint{0.481978in}{0.331635in}}{\pgfqpoint{4.960000in}{3.696000in}}%
\pgfusepath{clip}%
\pgfsetrectcap%
\pgfsetroundjoin%
\pgfsetlinewidth{1.505625pt}%
\definecolor{currentstroke}{rgb}{0.631373,0.788235,0.956863}%
\pgfsetstrokecolor{currentstroke}%
\pgfsetstrokeopacity{0.200000}%
\pgfsetdash{}{0pt}%
\pgfpathmoveto{\pgfqpoint{3.218175in}{1.975291in}}%
\pgfpathlineto{\pgfqpoint{3.305661in}{2.636178in}}%
\pgfusepath{stroke}%
\end{pgfscope}%
\begin{pgfscope}%
\pgfpathrectangle{\pgfqpoint{0.481978in}{0.331635in}}{\pgfqpoint{4.960000in}{3.696000in}}%
\pgfusepath{clip}%
\pgfsetrectcap%
\pgfsetroundjoin%
\pgfsetlinewidth{1.505625pt}%
\definecolor{currentstroke}{rgb}{0.631373,0.788235,0.956863}%
\pgfsetstrokecolor{currentstroke}%
\pgfsetstrokeopacity{0.200000}%
\pgfsetdash{}{0pt}%
\pgfpathmoveto{\pgfqpoint{3.143077in}{2.926485in}}%
\pgfpathlineto{\pgfqpoint{3.305661in}{2.636178in}}%
\pgfusepath{stroke}%
\end{pgfscope}%
\begin{pgfscope}%
\pgfpathrectangle{\pgfqpoint{0.481978in}{0.331635in}}{\pgfqpoint{4.960000in}{3.696000in}}%
\pgfusepath{clip}%
\pgfsetrectcap%
\pgfsetroundjoin%
\pgfsetlinewidth{1.505625pt}%
\definecolor{currentstroke}{rgb}{0.631373,0.788235,0.956863}%
\pgfsetstrokecolor{currentstroke}%
\pgfsetstrokeopacity{0.200000}%
\pgfsetdash{}{0pt}%
\pgfpathmoveto{\pgfqpoint{2.735315in}{3.598727in}}%
\pgfpathlineto{\pgfqpoint{3.305661in}{2.636178in}}%
\pgfusepath{stroke}%
\end{pgfscope}%
\begin{pgfscope}%
\pgfpathrectangle{\pgfqpoint{0.481978in}{0.331635in}}{\pgfqpoint{4.960000in}{3.696000in}}%
\pgfusepath{clip}%
\pgfsetrectcap%
\pgfsetroundjoin%
\pgfsetlinewidth{1.505625pt}%
\definecolor{currentstroke}{rgb}{0.631373,0.788235,0.956863}%
\pgfsetstrokecolor{currentstroke}%
\pgfsetstrokeopacity{0.200000}%
\pgfsetdash{}{0pt}%
\pgfpathmoveto{\pgfqpoint{2.631413in}{3.298699in}}%
\pgfpathlineto{\pgfqpoint{3.305661in}{2.636178in}}%
\pgfusepath{stroke}%
\end{pgfscope}%
\begin{pgfscope}%
\pgfpathrectangle{\pgfqpoint{0.481978in}{0.331635in}}{\pgfqpoint{4.960000in}{3.696000in}}%
\pgfusepath{clip}%
\pgfsetrectcap%
\pgfsetroundjoin%
\pgfsetlinewidth{1.505625pt}%
\definecolor{currentstroke}{rgb}{0.631373,0.788235,0.956863}%
\pgfsetstrokecolor{currentstroke}%
\pgfsetstrokeopacity{0.200000}%
\pgfsetdash{}{0pt}%
\pgfpathmoveto{\pgfqpoint{2.918834in}{3.207475in}}%
\pgfpathlineto{\pgfqpoint{3.305661in}{2.636178in}}%
\pgfusepath{stroke}%
\end{pgfscope}%
\begin{pgfscope}%
\pgfpathrectangle{\pgfqpoint{0.481978in}{0.331635in}}{\pgfqpoint{4.960000in}{3.696000in}}%
\pgfusepath{clip}%
\pgfsetrectcap%
\pgfsetroundjoin%
\pgfsetlinewidth{1.505625pt}%
\definecolor{currentstroke}{rgb}{0.631373,0.788235,0.956863}%
\pgfsetstrokecolor{currentstroke}%
\pgfsetstrokeopacity{0.200000}%
\pgfsetdash{}{0pt}%
\pgfpathmoveto{\pgfqpoint{4.008265in}{3.104049in}}%
\pgfpathlineto{\pgfqpoint{3.305661in}{2.636178in}}%
\pgfusepath{stroke}%
\end{pgfscope}%
\begin{pgfscope}%
\pgfpathrectangle{\pgfqpoint{0.481978in}{0.331635in}}{\pgfqpoint{4.960000in}{3.696000in}}%
\pgfusepath{clip}%
\pgfsetrectcap%
\pgfsetroundjoin%
\pgfsetlinewidth{1.505625pt}%
\definecolor{currentstroke}{rgb}{0.631373,0.788235,0.956863}%
\pgfsetstrokecolor{currentstroke}%
\pgfsetstrokeopacity{0.200000}%
\pgfsetdash{}{0pt}%
\pgfpathmoveto{\pgfqpoint{4.676024in}{3.390016in}}%
\pgfpathlineto{\pgfqpoint{3.305661in}{2.636178in}}%
\pgfusepath{stroke}%
\end{pgfscope}%
\begin{pgfscope}%
\pgfpathrectangle{\pgfqpoint{0.481978in}{0.331635in}}{\pgfqpoint{4.960000in}{3.696000in}}%
\pgfusepath{clip}%
\pgfsetrectcap%
\pgfsetroundjoin%
\pgfsetlinewidth{1.505625pt}%
\definecolor{currentstroke}{rgb}{0.631373,0.788235,0.956863}%
\pgfsetstrokecolor{currentstroke}%
\pgfsetstrokeopacity{0.200000}%
\pgfsetdash{}{0pt}%
\pgfpathmoveto{\pgfqpoint{2.835001in}{2.700457in}}%
\pgfpathlineto{\pgfqpoint{3.305661in}{2.636178in}}%
\pgfusepath{stroke}%
\end{pgfscope}%
\begin{pgfscope}%
\pgfpathrectangle{\pgfqpoint{0.481978in}{0.331635in}}{\pgfqpoint{4.960000in}{3.696000in}}%
\pgfusepath{clip}%
\pgfsetrectcap%
\pgfsetroundjoin%
\pgfsetlinewidth{1.505625pt}%
\definecolor{currentstroke}{rgb}{0.631373,0.788235,0.956863}%
\pgfsetstrokecolor{currentstroke}%
\pgfsetstrokeopacity{0.200000}%
\pgfsetdash{}{0pt}%
\pgfpathmoveto{\pgfqpoint{3.385446in}{3.285271in}}%
\pgfpathlineto{\pgfqpoint{3.305661in}{2.636178in}}%
\pgfusepath{stroke}%
\end{pgfscope}%
\begin{pgfscope}%
\pgfpathrectangle{\pgfqpoint{0.481978in}{0.331635in}}{\pgfqpoint{4.960000in}{3.696000in}}%
\pgfusepath{clip}%
\pgfsetrectcap%
\pgfsetroundjoin%
\pgfsetlinewidth{1.505625pt}%
\definecolor{currentstroke}{rgb}{0.631373,0.788235,0.956863}%
\pgfsetstrokecolor{currentstroke}%
\pgfsetstrokeopacity{0.200000}%
\pgfsetdash{}{0pt}%
\pgfpathmoveto{\pgfqpoint{3.070415in}{1.867287in}}%
\pgfpathlineto{\pgfqpoint{3.305661in}{2.636178in}}%
\pgfusepath{stroke}%
\end{pgfscope}%
\begin{pgfscope}%
\pgfpathrectangle{\pgfqpoint{0.481978in}{0.331635in}}{\pgfqpoint{4.960000in}{3.696000in}}%
\pgfusepath{clip}%
\pgfsetrectcap%
\pgfsetroundjoin%
\pgfsetlinewidth{1.505625pt}%
\definecolor{currentstroke}{rgb}{0.631373,0.788235,0.956863}%
\pgfsetstrokecolor{currentstroke}%
\pgfsetstrokeopacity{0.200000}%
\pgfsetdash{}{0pt}%
\pgfpathmoveto{\pgfqpoint{4.063842in}{3.485542in}}%
\pgfpathlineto{\pgfqpoint{3.305661in}{2.636178in}}%
\pgfusepath{stroke}%
\end{pgfscope}%
\begin{pgfscope}%
\pgfpathrectangle{\pgfqpoint{0.481978in}{0.331635in}}{\pgfqpoint{4.960000in}{3.696000in}}%
\pgfusepath{clip}%
\pgfsetrectcap%
\pgfsetroundjoin%
\pgfsetlinewidth{1.505625pt}%
\definecolor{currentstroke}{rgb}{0.631373,0.788235,0.956863}%
\pgfsetstrokecolor{currentstroke}%
\pgfsetstrokeopacity{0.200000}%
\pgfsetdash{}{0pt}%
\pgfpathmoveto{\pgfqpoint{2.721231in}{2.034472in}}%
\pgfpathlineto{\pgfqpoint{3.305661in}{2.636178in}}%
\pgfusepath{stroke}%
\end{pgfscope}%
\begin{pgfscope}%
\pgfpathrectangle{\pgfqpoint{0.481978in}{0.331635in}}{\pgfqpoint{4.960000in}{3.696000in}}%
\pgfusepath{clip}%
\pgfsetrectcap%
\pgfsetroundjoin%
\pgfsetlinewidth{1.505625pt}%
\definecolor{currentstroke}{rgb}{0.631373,0.788235,0.956863}%
\pgfsetstrokecolor{currentstroke}%
\pgfsetstrokeopacity{0.200000}%
\pgfsetdash{}{0pt}%
\pgfpathmoveto{\pgfqpoint{2.841211in}{3.127063in}}%
\pgfpathlineto{\pgfqpoint{3.305661in}{2.636178in}}%
\pgfusepath{stroke}%
\end{pgfscope}%
\begin{pgfscope}%
\pgfpathrectangle{\pgfqpoint{0.481978in}{0.331635in}}{\pgfqpoint{4.960000in}{3.696000in}}%
\pgfusepath{clip}%
\pgfsetrectcap%
\pgfsetroundjoin%
\pgfsetlinewidth{1.505625pt}%
\definecolor{currentstroke}{rgb}{0.631373,0.788235,0.956863}%
\pgfsetstrokecolor{currentstroke}%
\pgfsetstrokeopacity{0.200000}%
\pgfsetdash{}{0pt}%
\pgfpathmoveto{\pgfqpoint{3.319157in}{2.109917in}}%
\pgfpathlineto{\pgfqpoint{3.305661in}{2.636178in}}%
\pgfusepath{stroke}%
\end{pgfscope}%
\begin{pgfscope}%
\pgfpathrectangle{\pgfqpoint{0.481978in}{0.331635in}}{\pgfqpoint{4.960000in}{3.696000in}}%
\pgfusepath{clip}%
\pgfsetrectcap%
\pgfsetroundjoin%
\pgfsetlinewidth{1.505625pt}%
\definecolor{currentstroke}{rgb}{0.631373,0.788235,0.956863}%
\pgfsetstrokecolor{currentstroke}%
\pgfsetstrokeopacity{0.200000}%
\pgfsetdash{}{0pt}%
\pgfpathmoveto{\pgfqpoint{3.234413in}{2.811974in}}%
\pgfpathlineto{\pgfqpoint{3.305661in}{2.636178in}}%
\pgfusepath{stroke}%
\end{pgfscope}%
\begin{pgfscope}%
\pgfpathrectangle{\pgfqpoint{0.481978in}{0.331635in}}{\pgfqpoint{4.960000in}{3.696000in}}%
\pgfusepath{clip}%
\pgfsetrectcap%
\pgfsetroundjoin%
\pgfsetlinewidth{1.505625pt}%
\definecolor{currentstroke}{rgb}{0.631373,0.788235,0.956863}%
\pgfsetstrokecolor{currentstroke}%
\pgfsetstrokeopacity{0.200000}%
\pgfsetdash{}{0pt}%
\pgfpathmoveto{\pgfqpoint{3.645927in}{2.331908in}}%
\pgfpathlineto{\pgfqpoint{3.305661in}{2.636178in}}%
\pgfusepath{stroke}%
\end{pgfscope}%
\begin{pgfscope}%
\pgfpathrectangle{\pgfqpoint{0.481978in}{0.331635in}}{\pgfqpoint{4.960000in}{3.696000in}}%
\pgfusepath{clip}%
\pgfsetrectcap%
\pgfsetroundjoin%
\pgfsetlinewidth{1.505625pt}%
\definecolor{currentstroke}{rgb}{0.631373,0.788235,0.956863}%
\pgfsetstrokecolor{currentstroke}%
\pgfsetstrokeopacity{0.200000}%
\pgfsetdash{}{0pt}%
\pgfpathmoveto{\pgfqpoint{4.276918in}{3.559647in}}%
\pgfpathlineto{\pgfqpoint{3.305661in}{2.636178in}}%
\pgfusepath{stroke}%
\end{pgfscope}%
\begin{pgfscope}%
\pgfpathrectangle{\pgfqpoint{0.481978in}{0.331635in}}{\pgfqpoint{4.960000in}{3.696000in}}%
\pgfusepath{clip}%
\pgfsetrectcap%
\pgfsetroundjoin%
\pgfsetlinewidth{1.505625pt}%
\definecolor{currentstroke}{rgb}{0.631373,0.788235,0.956863}%
\pgfsetstrokecolor{currentstroke}%
\pgfsetstrokeopacity{0.200000}%
\pgfsetdash{}{0pt}%
\pgfpathmoveto{\pgfqpoint{2.740941in}{2.233383in}}%
\pgfpathlineto{\pgfqpoint{3.305661in}{2.636178in}}%
\pgfusepath{stroke}%
\end{pgfscope}%
\begin{pgfscope}%
\pgfpathrectangle{\pgfqpoint{0.481978in}{0.331635in}}{\pgfqpoint{4.960000in}{3.696000in}}%
\pgfusepath{clip}%
\pgfsetrectcap%
\pgfsetroundjoin%
\pgfsetlinewidth{1.505625pt}%
\definecolor{currentstroke}{rgb}{0.631373,0.788235,0.956863}%
\pgfsetstrokecolor{currentstroke}%
\pgfsetstrokeopacity{0.200000}%
\pgfsetdash{}{0pt}%
\pgfpathmoveto{\pgfqpoint{2.346492in}{2.316232in}}%
\pgfpathlineto{\pgfqpoint{3.305661in}{2.636178in}}%
\pgfusepath{stroke}%
\end{pgfscope}%
\begin{pgfscope}%
\pgfpathrectangle{\pgfqpoint{0.481978in}{0.331635in}}{\pgfqpoint{4.960000in}{3.696000in}}%
\pgfusepath{clip}%
\pgfsetrectcap%
\pgfsetroundjoin%
\pgfsetlinewidth{1.505625pt}%
\definecolor{currentstroke}{rgb}{0.631373,0.788235,0.956863}%
\pgfsetstrokecolor{currentstroke}%
\pgfsetstrokeopacity{0.200000}%
\pgfsetdash{}{0pt}%
\pgfpathmoveto{\pgfqpoint{2.250683in}{1.938697in}}%
\pgfpathlineto{\pgfqpoint{3.305661in}{2.636178in}}%
\pgfusepath{stroke}%
\end{pgfscope}%
\begin{pgfscope}%
\pgfpathrectangle{\pgfqpoint{0.481978in}{0.331635in}}{\pgfqpoint{4.960000in}{3.696000in}}%
\pgfusepath{clip}%
\pgfsetrectcap%
\pgfsetroundjoin%
\pgfsetlinewidth{1.505625pt}%
\definecolor{currentstroke}{rgb}{0.631373,0.788235,0.956863}%
\pgfsetstrokecolor{currentstroke}%
\pgfsetstrokeopacity{0.200000}%
\pgfsetdash{}{0pt}%
\pgfpathmoveto{\pgfqpoint{2.500865in}{1.791096in}}%
\pgfpathlineto{\pgfqpoint{3.305661in}{2.636178in}}%
\pgfusepath{stroke}%
\end{pgfscope}%
\begin{pgfscope}%
\pgfpathrectangle{\pgfqpoint{0.481978in}{0.331635in}}{\pgfqpoint{4.960000in}{3.696000in}}%
\pgfusepath{clip}%
\pgfsetrectcap%
\pgfsetroundjoin%
\pgfsetlinewidth{1.505625pt}%
\definecolor{currentstroke}{rgb}{0.631373,0.788235,0.956863}%
\pgfsetstrokecolor{currentstroke}%
\pgfsetstrokeopacity{0.200000}%
\pgfsetdash{}{0pt}%
\pgfpathmoveto{\pgfqpoint{2.611315in}{2.362044in}}%
\pgfpathlineto{\pgfqpoint{3.305661in}{2.636178in}}%
\pgfusepath{stroke}%
\end{pgfscope}%
\begin{pgfscope}%
\pgfpathrectangle{\pgfqpoint{0.481978in}{0.331635in}}{\pgfqpoint{4.960000in}{3.696000in}}%
\pgfusepath{clip}%
\pgfsetrectcap%
\pgfsetroundjoin%
\pgfsetlinewidth{1.505625pt}%
\definecolor{currentstroke}{rgb}{0.631373,0.788235,0.956863}%
\pgfsetstrokecolor{currentstroke}%
\pgfsetstrokeopacity{0.200000}%
\pgfsetdash{}{0pt}%
\pgfpathmoveto{\pgfqpoint{4.658256in}{3.516239in}}%
\pgfpathlineto{\pgfqpoint{3.305661in}{2.636178in}}%
\pgfusepath{stroke}%
\end{pgfscope}%
\begin{pgfscope}%
\pgfpathrectangle{\pgfqpoint{0.481978in}{0.331635in}}{\pgfqpoint{4.960000in}{3.696000in}}%
\pgfusepath{clip}%
\pgfsetrectcap%
\pgfsetroundjoin%
\pgfsetlinewidth{1.505625pt}%
\definecolor{currentstroke}{rgb}{0.631373,0.788235,0.956863}%
\pgfsetstrokecolor{currentstroke}%
\pgfsetstrokeopacity{0.200000}%
\pgfsetdash{}{0pt}%
\pgfpathmoveto{\pgfqpoint{3.034776in}{1.807338in}}%
\pgfpathlineto{\pgfqpoint{3.305661in}{2.636178in}}%
\pgfusepath{stroke}%
\end{pgfscope}%
\begin{pgfscope}%
\pgfpathrectangle{\pgfqpoint{0.481978in}{0.331635in}}{\pgfqpoint{4.960000in}{3.696000in}}%
\pgfusepath{clip}%
\pgfsetrectcap%
\pgfsetroundjoin%
\pgfsetlinewidth{1.505625pt}%
\definecolor{currentstroke}{rgb}{0.631373,0.788235,0.956863}%
\pgfsetstrokecolor{currentstroke}%
\pgfsetstrokeopacity{0.200000}%
\pgfsetdash{}{0pt}%
\pgfpathmoveto{\pgfqpoint{3.122341in}{2.811439in}}%
\pgfpathlineto{\pgfqpoint{3.305661in}{2.636178in}}%
\pgfusepath{stroke}%
\end{pgfscope}%
\begin{pgfscope}%
\pgfpathrectangle{\pgfqpoint{0.481978in}{0.331635in}}{\pgfqpoint{4.960000in}{3.696000in}}%
\pgfusepath{clip}%
\pgfsetrectcap%
\pgfsetroundjoin%
\pgfsetlinewidth{1.505625pt}%
\definecolor{currentstroke}{rgb}{0.631373,0.788235,0.956863}%
\pgfsetstrokecolor{currentstroke}%
\pgfsetstrokeopacity{0.200000}%
\pgfsetdash{}{0pt}%
\pgfpathmoveto{\pgfqpoint{2.762823in}{3.562632in}}%
\pgfpathlineto{\pgfqpoint{3.305661in}{2.636178in}}%
\pgfusepath{stroke}%
\end{pgfscope}%
\begin{pgfscope}%
\pgfpathrectangle{\pgfqpoint{0.481978in}{0.331635in}}{\pgfqpoint{4.960000in}{3.696000in}}%
\pgfusepath{clip}%
\pgfsetrectcap%
\pgfsetroundjoin%
\pgfsetlinewidth{1.505625pt}%
\definecolor{currentstroke}{rgb}{0.631373,0.788235,0.956863}%
\pgfsetstrokecolor{currentstroke}%
\pgfsetstrokeopacity{0.200000}%
\pgfsetdash{}{0pt}%
\pgfpathmoveto{\pgfqpoint{2.859601in}{2.533913in}}%
\pgfpathlineto{\pgfqpoint{3.305661in}{2.636178in}}%
\pgfusepath{stroke}%
\end{pgfscope}%
\begin{pgfscope}%
\pgfpathrectangle{\pgfqpoint{0.481978in}{0.331635in}}{\pgfqpoint{4.960000in}{3.696000in}}%
\pgfusepath{clip}%
\pgfsetrectcap%
\pgfsetroundjoin%
\pgfsetlinewidth{1.505625pt}%
\definecolor{currentstroke}{rgb}{0.631373,0.788235,0.956863}%
\pgfsetstrokecolor{currentstroke}%
\pgfsetstrokeopacity{0.200000}%
\pgfsetdash{}{0pt}%
\pgfpathmoveto{\pgfqpoint{3.643947in}{3.395658in}}%
\pgfpathlineto{\pgfqpoint{3.305661in}{2.636178in}}%
\pgfusepath{stroke}%
\end{pgfscope}%
\begin{pgfscope}%
\pgfpathrectangle{\pgfqpoint{0.481978in}{0.331635in}}{\pgfqpoint{4.960000in}{3.696000in}}%
\pgfusepath{clip}%
\pgfsetrectcap%
\pgfsetroundjoin%
\pgfsetlinewidth{1.505625pt}%
\definecolor{currentstroke}{rgb}{0.631373,0.788235,0.956863}%
\pgfsetstrokecolor{currentstroke}%
\pgfsetstrokeopacity{0.200000}%
\pgfsetdash{}{0pt}%
\pgfpathmoveto{\pgfqpoint{3.539644in}{3.142544in}}%
\pgfpathlineto{\pgfqpoint{3.305661in}{2.636178in}}%
\pgfusepath{stroke}%
\end{pgfscope}%
\begin{pgfscope}%
\pgfpathrectangle{\pgfqpoint{0.481978in}{0.331635in}}{\pgfqpoint{4.960000in}{3.696000in}}%
\pgfusepath{clip}%
\pgfsetrectcap%
\pgfsetroundjoin%
\pgfsetlinewidth{1.505625pt}%
\definecolor{currentstroke}{rgb}{0.631373,0.788235,0.956863}%
\pgfsetstrokecolor{currentstroke}%
\pgfsetstrokeopacity{0.200000}%
\pgfsetdash{}{0pt}%
\pgfpathmoveto{\pgfqpoint{4.151040in}{3.278256in}}%
\pgfpathlineto{\pgfqpoint{3.305661in}{2.636178in}}%
\pgfusepath{stroke}%
\end{pgfscope}%
\begin{pgfscope}%
\pgfpathrectangle{\pgfqpoint{0.481978in}{0.331635in}}{\pgfqpoint{4.960000in}{3.696000in}}%
\pgfusepath{clip}%
\pgfsetrectcap%
\pgfsetroundjoin%
\pgfsetlinewidth{1.505625pt}%
\definecolor{currentstroke}{rgb}{0.631373,0.788235,0.956863}%
\pgfsetstrokecolor{currentstroke}%
\pgfsetstrokeopacity{0.200000}%
\pgfsetdash{}{0pt}%
\pgfpathmoveto{\pgfqpoint{2.718198in}{2.085219in}}%
\pgfpathlineto{\pgfqpoint{3.305661in}{2.636178in}}%
\pgfusepath{stroke}%
\end{pgfscope}%
\begin{pgfscope}%
\pgfpathrectangle{\pgfqpoint{0.481978in}{0.331635in}}{\pgfqpoint{4.960000in}{3.696000in}}%
\pgfusepath{clip}%
\pgfsetrectcap%
\pgfsetroundjoin%
\pgfsetlinewidth{1.505625pt}%
\definecolor{currentstroke}{rgb}{0.631373,0.788235,0.956863}%
\pgfsetstrokecolor{currentstroke}%
\pgfsetstrokeopacity{0.200000}%
\pgfsetdash{}{0pt}%
\pgfpathmoveto{\pgfqpoint{3.451581in}{1.932436in}}%
\pgfpathlineto{\pgfqpoint{3.305661in}{2.636178in}}%
\pgfusepath{stroke}%
\end{pgfscope}%
\begin{pgfscope}%
\pgfpathrectangle{\pgfqpoint{0.481978in}{0.331635in}}{\pgfqpoint{4.960000in}{3.696000in}}%
\pgfusepath{clip}%
\pgfsetrectcap%
\pgfsetroundjoin%
\pgfsetlinewidth{1.505625pt}%
\definecolor{currentstroke}{rgb}{0.631373,0.788235,0.956863}%
\pgfsetstrokecolor{currentstroke}%
\pgfsetstrokeopacity{0.200000}%
\pgfsetdash{}{0pt}%
\pgfpathmoveto{\pgfqpoint{2.582592in}{1.937360in}}%
\pgfpathlineto{\pgfqpoint{3.305661in}{2.636178in}}%
\pgfusepath{stroke}%
\end{pgfscope}%
\begin{pgfscope}%
\pgfpathrectangle{\pgfqpoint{0.481978in}{0.331635in}}{\pgfqpoint{4.960000in}{3.696000in}}%
\pgfusepath{clip}%
\pgfsetrectcap%
\pgfsetroundjoin%
\pgfsetlinewidth{1.505625pt}%
\definecolor{currentstroke}{rgb}{0.631373,0.788235,0.956863}%
\pgfsetstrokecolor{currentstroke}%
\pgfsetstrokeopacity{0.200000}%
\pgfsetdash{}{0pt}%
\pgfpathmoveto{\pgfqpoint{3.290958in}{3.445946in}}%
\pgfpathlineto{\pgfqpoint{3.305661in}{2.636178in}}%
\pgfusepath{stroke}%
\end{pgfscope}%
\begin{pgfscope}%
\pgfpathrectangle{\pgfqpoint{0.481978in}{0.331635in}}{\pgfqpoint{4.960000in}{3.696000in}}%
\pgfusepath{clip}%
\pgfsetrectcap%
\pgfsetroundjoin%
\pgfsetlinewidth{1.505625pt}%
\definecolor{currentstroke}{rgb}{0.631373,0.788235,0.956863}%
\pgfsetstrokecolor{currentstroke}%
\pgfsetstrokeopacity{0.200000}%
\pgfsetdash{}{0pt}%
\pgfpathmoveto{\pgfqpoint{3.511086in}{2.649185in}}%
\pgfpathlineto{\pgfqpoint{3.305661in}{2.636178in}}%
\pgfusepath{stroke}%
\end{pgfscope}%
\begin{pgfscope}%
\pgfpathrectangle{\pgfqpoint{0.481978in}{0.331635in}}{\pgfqpoint{4.960000in}{3.696000in}}%
\pgfusepath{clip}%
\pgfsetrectcap%
\pgfsetroundjoin%
\pgfsetlinewidth{1.505625pt}%
\definecolor{currentstroke}{rgb}{0.631373,0.788235,0.956863}%
\pgfsetstrokecolor{currentstroke}%
\pgfsetstrokeopacity{0.200000}%
\pgfsetdash{}{0pt}%
\pgfpathmoveto{\pgfqpoint{3.595851in}{3.465898in}}%
\pgfpathlineto{\pgfqpoint{3.305661in}{2.636178in}}%
\pgfusepath{stroke}%
\end{pgfscope}%
\begin{pgfscope}%
\pgfpathrectangle{\pgfqpoint{0.481978in}{0.331635in}}{\pgfqpoint{4.960000in}{3.696000in}}%
\pgfusepath{clip}%
\pgfsetrectcap%
\pgfsetroundjoin%
\pgfsetlinewidth{1.505625pt}%
\definecolor{currentstroke}{rgb}{0.631373,0.788235,0.956863}%
\pgfsetstrokecolor{currentstroke}%
\pgfsetstrokeopacity{0.200000}%
\pgfsetdash{}{0pt}%
\pgfpathmoveto{\pgfqpoint{2.932962in}{2.213931in}}%
\pgfpathlineto{\pgfqpoint{3.305661in}{2.636178in}}%
\pgfusepath{stroke}%
\end{pgfscope}%
\begin{pgfscope}%
\pgfpathrectangle{\pgfqpoint{0.481978in}{0.331635in}}{\pgfqpoint{4.960000in}{3.696000in}}%
\pgfusepath{clip}%
\pgfsetrectcap%
\pgfsetroundjoin%
\pgfsetlinewidth{1.505625pt}%
\definecolor{currentstroke}{rgb}{0.631373,0.788235,0.956863}%
\pgfsetstrokecolor{currentstroke}%
\pgfsetstrokeopacity{0.200000}%
\pgfsetdash{}{0pt}%
\pgfpathmoveto{\pgfqpoint{3.369825in}{2.763711in}}%
\pgfpathlineto{\pgfqpoint{3.305661in}{2.636178in}}%
\pgfusepath{stroke}%
\end{pgfscope}%
\begin{pgfscope}%
\pgfpathrectangle{\pgfqpoint{0.481978in}{0.331635in}}{\pgfqpoint{4.960000in}{3.696000in}}%
\pgfusepath{clip}%
\pgfsetrectcap%
\pgfsetroundjoin%
\pgfsetlinewidth{1.505625pt}%
\definecolor{currentstroke}{rgb}{0.631373,0.788235,0.956863}%
\pgfsetstrokecolor{currentstroke}%
\pgfsetstrokeopacity{0.200000}%
\pgfsetdash{}{0pt}%
\pgfpathmoveto{\pgfqpoint{3.062456in}{1.969844in}}%
\pgfpathlineto{\pgfqpoint{3.305661in}{2.636178in}}%
\pgfusepath{stroke}%
\end{pgfscope}%
\begin{pgfscope}%
\pgfpathrectangle{\pgfqpoint{0.481978in}{0.331635in}}{\pgfqpoint{4.960000in}{3.696000in}}%
\pgfusepath{clip}%
\pgfsetrectcap%
\pgfsetroundjoin%
\pgfsetlinewidth{1.505625pt}%
\definecolor{currentstroke}{rgb}{0.631373,0.788235,0.956863}%
\pgfsetstrokecolor{currentstroke}%
\pgfsetstrokeopacity{0.200000}%
\pgfsetdash{}{0pt}%
\pgfpathmoveto{\pgfqpoint{4.682831in}{3.524687in}}%
\pgfpathlineto{\pgfqpoint{3.305661in}{2.636178in}}%
\pgfusepath{stroke}%
\end{pgfscope}%
\begin{pgfscope}%
\pgfpathrectangle{\pgfqpoint{0.481978in}{0.331635in}}{\pgfqpoint{4.960000in}{3.696000in}}%
\pgfusepath{clip}%
\pgfsetrectcap%
\pgfsetroundjoin%
\pgfsetlinewidth{1.505625pt}%
\definecolor{currentstroke}{rgb}{0.631373,0.788235,0.956863}%
\pgfsetstrokecolor{currentstroke}%
\pgfsetstrokeopacity{0.200000}%
\pgfsetdash{}{0pt}%
\pgfpathmoveto{\pgfqpoint{3.205444in}{3.677070in}}%
\pgfpathlineto{\pgfqpoint{3.305661in}{2.636178in}}%
\pgfusepath{stroke}%
\end{pgfscope}%
\begin{pgfscope}%
\pgfpathrectangle{\pgfqpoint{0.481978in}{0.331635in}}{\pgfqpoint{4.960000in}{3.696000in}}%
\pgfusepath{clip}%
\pgfsetrectcap%
\pgfsetroundjoin%
\pgfsetlinewidth{1.505625pt}%
\definecolor{currentstroke}{rgb}{0.631373,0.788235,0.956863}%
\pgfsetstrokecolor{currentstroke}%
\pgfsetstrokeopacity{0.200000}%
\pgfsetdash{}{0pt}%
\pgfpathmoveto{\pgfqpoint{3.649476in}{3.246400in}}%
\pgfpathlineto{\pgfqpoint{3.305661in}{2.636178in}}%
\pgfusepath{stroke}%
\end{pgfscope}%
\begin{pgfscope}%
\pgfpathrectangle{\pgfqpoint{0.481978in}{0.331635in}}{\pgfqpoint{4.960000in}{3.696000in}}%
\pgfusepath{clip}%
\pgfsetrectcap%
\pgfsetroundjoin%
\pgfsetlinewidth{1.505625pt}%
\definecolor{currentstroke}{rgb}{0.631373,0.788235,0.956863}%
\pgfsetstrokecolor{currentstroke}%
\pgfsetstrokeopacity{0.200000}%
\pgfsetdash{}{0pt}%
\pgfpathmoveto{\pgfqpoint{2.930723in}{2.137984in}}%
\pgfpathlineto{\pgfqpoint{3.305661in}{2.636178in}}%
\pgfusepath{stroke}%
\end{pgfscope}%
\begin{pgfscope}%
\pgfpathrectangle{\pgfqpoint{0.481978in}{0.331635in}}{\pgfqpoint{4.960000in}{3.696000in}}%
\pgfusepath{clip}%
\pgfsetrectcap%
\pgfsetroundjoin%
\pgfsetlinewidth{1.505625pt}%
\definecolor{currentstroke}{rgb}{0.631373,0.788235,0.956863}%
\pgfsetstrokecolor{currentstroke}%
\pgfsetstrokeopacity{0.200000}%
\pgfsetdash{}{0pt}%
\pgfpathmoveto{\pgfqpoint{4.204178in}{1.683319in}}%
\pgfpathlineto{\pgfqpoint{3.305661in}{2.636178in}}%
\pgfusepath{stroke}%
\end{pgfscope}%
\begin{pgfscope}%
\pgfpathrectangle{\pgfqpoint{0.481978in}{0.331635in}}{\pgfqpoint{4.960000in}{3.696000in}}%
\pgfusepath{clip}%
\pgfsetrectcap%
\pgfsetroundjoin%
\pgfsetlinewidth{1.505625pt}%
\definecolor{currentstroke}{rgb}{0.631373,0.788235,0.956863}%
\pgfsetstrokecolor{currentstroke}%
\pgfsetstrokeopacity{0.200000}%
\pgfsetdash{}{0pt}%
\pgfpathmoveto{\pgfqpoint{4.511769in}{3.161331in}}%
\pgfpathlineto{\pgfqpoint{3.305661in}{2.636178in}}%
\pgfusepath{stroke}%
\end{pgfscope}%
\begin{pgfscope}%
\pgfpathrectangle{\pgfqpoint{0.481978in}{0.331635in}}{\pgfqpoint{4.960000in}{3.696000in}}%
\pgfusepath{clip}%
\pgfsetrectcap%
\pgfsetroundjoin%
\pgfsetlinewidth{1.505625pt}%
\definecolor{currentstroke}{rgb}{0.631373,0.788235,0.956863}%
\pgfsetstrokecolor{currentstroke}%
\pgfsetstrokeopacity{0.200000}%
\pgfsetdash{}{0pt}%
\pgfpathmoveto{\pgfqpoint{2.560493in}{2.018573in}}%
\pgfpathlineto{\pgfqpoint{3.305661in}{2.636178in}}%
\pgfusepath{stroke}%
\end{pgfscope}%
\begin{pgfscope}%
\pgfpathrectangle{\pgfqpoint{0.481978in}{0.331635in}}{\pgfqpoint{4.960000in}{3.696000in}}%
\pgfusepath{clip}%
\pgfsetrectcap%
\pgfsetroundjoin%
\pgfsetlinewidth{1.505625pt}%
\definecolor{currentstroke}{rgb}{0.631373,0.788235,0.956863}%
\pgfsetstrokecolor{currentstroke}%
\pgfsetstrokeopacity{0.200000}%
\pgfsetdash{}{0pt}%
\pgfpathmoveto{\pgfqpoint{2.942503in}{2.413249in}}%
\pgfpathlineto{\pgfqpoint{3.305661in}{2.636178in}}%
\pgfusepath{stroke}%
\end{pgfscope}%
\begin{pgfscope}%
\pgfpathrectangle{\pgfqpoint{0.481978in}{0.331635in}}{\pgfqpoint{4.960000in}{3.696000in}}%
\pgfusepath{clip}%
\pgfsetrectcap%
\pgfsetroundjoin%
\pgfsetlinewidth{1.505625pt}%
\definecolor{currentstroke}{rgb}{0.631373,0.788235,0.956863}%
\pgfsetstrokecolor{currentstroke}%
\pgfsetstrokeopacity{0.200000}%
\pgfsetdash{}{0pt}%
\pgfpathmoveto{\pgfqpoint{3.195028in}{1.759670in}}%
\pgfpathlineto{\pgfqpoint{3.305661in}{2.636178in}}%
\pgfusepath{stroke}%
\end{pgfscope}%
\begin{pgfscope}%
\pgfpathrectangle{\pgfqpoint{0.481978in}{0.331635in}}{\pgfqpoint{4.960000in}{3.696000in}}%
\pgfusepath{clip}%
\pgfsetrectcap%
\pgfsetroundjoin%
\pgfsetlinewidth{1.505625pt}%
\definecolor{currentstroke}{rgb}{0.631373,0.788235,0.956863}%
\pgfsetstrokecolor{currentstroke}%
\pgfsetstrokeopacity{0.200000}%
\pgfsetdash{}{0pt}%
\pgfpathmoveto{\pgfqpoint{2.587739in}{2.616506in}}%
\pgfpathlineto{\pgfqpoint{3.305661in}{2.636178in}}%
\pgfusepath{stroke}%
\end{pgfscope}%
\begin{pgfscope}%
\pgfpathrectangle{\pgfqpoint{0.481978in}{0.331635in}}{\pgfqpoint{4.960000in}{3.696000in}}%
\pgfusepath{clip}%
\pgfsetrectcap%
\pgfsetroundjoin%
\pgfsetlinewidth{1.505625pt}%
\definecolor{currentstroke}{rgb}{0.631373,0.788235,0.956863}%
\pgfsetstrokecolor{currentstroke}%
\pgfsetstrokeopacity{0.200000}%
\pgfsetdash{}{0pt}%
\pgfpathmoveto{\pgfqpoint{3.162498in}{2.463102in}}%
\pgfpathlineto{\pgfqpoint{3.305661in}{2.636178in}}%
\pgfusepath{stroke}%
\end{pgfscope}%
\begin{pgfscope}%
\pgfpathrectangle{\pgfqpoint{0.481978in}{0.331635in}}{\pgfqpoint{4.960000in}{3.696000in}}%
\pgfusepath{clip}%
\pgfsetrectcap%
\pgfsetroundjoin%
\pgfsetlinewidth{1.505625pt}%
\definecolor{currentstroke}{rgb}{0.631373,0.788235,0.956863}%
\pgfsetstrokecolor{currentstroke}%
\pgfsetstrokeopacity{0.200000}%
\pgfsetdash{}{0pt}%
\pgfpathmoveto{\pgfqpoint{2.743558in}{2.172896in}}%
\pgfpathlineto{\pgfqpoint{3.305661in}{2.636178in}}%
\pgfusepath{stroke}%
\end{pgfscope}%
\begin{pgfscope}%
\pgfpathrectangle{\pgfqpoint{0.481978in}{0.331635in}}{\pgfqpoint{4.960000in}{3.696000in}}%
\pgfusepath{clip}%
\pgfsetrectcap%
\pgfsetroundjoin%
\pgfsetlinewidth{1.505625pt}%
\definecolor{currentstroke}{rgb}{0.631373,0.788235,0.956863}%
\pgfsetstrokecolor{currentstroke}%
\pgfsetstrokeopacity{0.200000}%
\pgfsetdash{}{0pt}%
\pgfpathmoveto{\pgfqpoint{3.019641in}{3.102614in}}%
\pgfpathlineto{\pgfqpoint{3.305661in}{2.636178in}}%
\pgfusepath{stroke}%
\end{pgfscope}%
\begin{pgfscope}%
\pgfpathrectangle{\pgfqpoint{0.481978in}{0.331635in}}{\pgfqpoint{4.960000in}{3.696000in}}%
\pgfusepath{clip}%
\pgfsetrectcap%
\pgfsetroundjoin%
\pgfsetlinewidth{1.505625pt}%
\definecolor{currentstroke}{rgb}{0.631373,0.788235,0.956863}%
\pgfsetstrokecolor{currentstroke}%
\pgfsetstrokeopacity{0.200000}%
\pgfsetdash{}{0pt}%
\pgfpathmoveto{\pgfqpoint{3.607583in}{3.327840in}}%
\pgfpathlineto{\pgfqpoint{3.305661in}{2.636178in}}%
\pgfusepath{stroke}%
\end{pgfscope}%
\begin{pgfscope}%
\pgfpathrectangle{\pgfqpoint{0.481978in}{0.331635in}}{\pgfqpoint{4.960000in}{3.696000in}}%
\pgfusepath{clip}%
\pgfsetrectcap%
\pgfsetroundjoin%
\pgfsetlinewidth{1.505625pt}%
\definecolor{currentstroke}{rgb}{0.631373,0.788235,0.956863}%
\pgfsetstrokecolor{currentstroke}%
\pgfsetstrokeopacity{0.200000}%
\pgfsetdash{}{0pt}%
\pgfpathmoveto{\pgfqpoint{3.004506in}{1.940246in}}%
\pgfpathlineto{\pgfqpoint{3.305661in}{2.636178in}}%
\pgfusepath{stroke}%
\end{pgfscope}%
\begin{pgfscope}%
\pgfpathrectangle{\pgfqpoint{0.481978in}{0.331635in}}{\pgfqpoint{4.960000in}{3.696000in}}%
\pgfusepath{clip}%
\pgfsetrectcap%
\pgfsetroundjoin%
\pgfsetlinewidth{1.505625pt}%
\definecolor{currentstroke}{rgb}{0.631373,0.788235,0.956863}%
\pgfsetstrokecolor{currentstroke}%
\pgfsetstrokeopacity{0.200000}%
\pgfsetdash{}{0pt}%
\pgfpathmoveto{\pgfqpoint{3.007147in}{2.050949in}}%
\pgfpathlineto{\pgfqpoint{3.305661in}{2.636178in}}%
\pgfusepath{stroke}%
\end{pgfscope}%
\begin{pgfscope}%
\pgfpathrectangle{\pgfqpoint{0.481978in}{0.331635in}}{\pgfqpoint{4.960000in}{3.696000in}}%
\pgfusepath{clip}%
\pgfsetrectcap%
\pgfsetroundjoin%
\pgfsetlinewidth{1.505625pt}%
\definecolor{currentstroke}{rgb}{0.631373,0.788235,0.956863}%
\pgfsetstrokecolor{currentstroke}%
\pgfsetstrokeopacity{0.200000}%
\pgfsetdash{}{0pt}%
\pgfpathmoveto{\pgfqpoint{3.903183in}{3.568040in}}%
\pgfpathlineto{\pgfqpoint{3.305661in}{2.636178in}}%
\pgfusepath{stroke}%
\end{pgfscope}%
\begin{pgfscope}%
\pgfpathrectangle{\pgfqpoint{0.481978in}{0.331635in}}{\pgfqpoint{4.960000in}{3.696000in}}%
\pgfusepath{clip}%
\pgfsetrectcap%
\pgfsetroundjoin%
\pgfsetlinewidth{1.505625pt}%
\definecolor{currentstroke}{rgb}{0.631373,0.788235,0.956863}%
\pgfsetstrokecolor{currentstroke}%
\pgfsetstrokeopacity{0.200000}%
\pgfsetdash{}{0pt}%
\pgfpathmoveto{\pgfqpoint{3.640068in}{2.450113in}}%
\pgfpathlineto{\pgfqpoint{3.305661in}{2.636178in}}%
\pgfusepath{stroke}%
\end{pgfscope}%
\begin{pgfscope}%
\pgfpathrectangle{\pgfqpoint{0.481978in}{0.331635in}}{\pgfqpoint{4.960000in}{3.696000in}}%
\pgfusepath{clip}%
\pgfsetrectcap%
\pgfsetroundjoin%
\pgfsetlinewidth{1.505625pt}%
\definecolor{currentstroke}{rgb}{0.631373,0.788235,0.956863}%
\pgfsetstrokecolor{currentstroke}%
\pgfsetstrokeopacity{0.200000}%
\pgfsetdash{}{0pt}%
\pgfpathmoveto{\pgfqpoint{2.787922in}{1.882876in}}%
\pgfpathlineto{\pgfqpoint{3.305661in}{2.636178in}}%
\pgfusepath{stroke}%
\end{pgfscope}%
\begin{pgfscope}%
\pgfpathrectangle{\pgfqpoint{0.481978in}{0.331635in}}{\pgfqpoint{4.960000in}{3.696000in}}%
\pgfusepath{clip}%
\pgfsetrectcap%
\pgfsetroundjoin%
\pgfsetlinewidth{1.505625pt}%
\definecolor{currentstroke}{rgb}{0.631373,0.788235,0.956863}%
\pgfsetstrokecolor{currentstroke}%
\pgfsetstrokeopacity{0.200000}%
\pgfsetdash{}{0pt}%
\pgfpathmoveto{\pgfqpoint{3.431218in}{2.816204in}}%
\pgfpathlineto{\pgfqpoint{3.305661in}{2.636178in}}%
\pgfusepath{stroke}%
\end{pgfscope}%
\begin{pgfscope}%
\pgfpathrectangle{\pgfqpoint{0.481978in}{0.331635in}}{\pgfqpoint{4.960000in}{3.696000in}}%
\pgfusepath{clip}%
\pgfsetrectcap%
\pgfsetroundjoin%
\pgfsetlinewidth{1.505625pt}%
\definecolor{currentstroke}{rgb}{0.631373,0.788235,0.956863}%
\pgfsetstrokecolor{currentstroke}%
\pgfsetstrokeopacity{0.200000}%
\pgfsetdash{}{0pt}%
\pgfpathmoveto{\pgfqpoint{2.771163in}{2.465708in}}%
\pgfpathlineto{\pgfqpoint{3.305661in}{2.636178in}}%
\pgfusepath{stroke}%
\end{pgfscope}%
\begin{pgfscope}%
\pgfpathrectangle{\pgfqpoint{0.481978in}{0.331635in}}{\pgfqpoint{4.960000in}{3.696000in}}%
\pgfusepath{clip}%
\pgfsetrectcap%
\pgfsetroundjoin%
\pgfsetlinewidth{1.505625pt}%
\definecolor{currentstroke}{rgb}{0.631373,0.788235,0.956863}%
\pgfsetstrokecolor{currentstroke}%
\pgfsetstrokeopacity{0.200000}%
\pgfsetdash{}{0pt}%
\pgfpathmoveto{\pgfqpoint{3.272656in}{3.455785in}}%
\pgfpathlineto{\pgfqpoint{3.305661in}{2.636178in}}%
\pgfusepath{stroke}%
\end{pgfscope}%
\begin{pgfscope}%
\pgfpathrectangle{\pgfqpoint{0.481978in}{0.331635in}}{\pgfqpoint{4.960000in}{3.696000in}}%
\pgfusepath{clip}%
\pgfsetrectcap%
\pgfsetroundjoin%
\pgfsetlinewidth{1.505625pt}%
\definecolor{currentstroke}{rgb}{0.631373,0.788235,0.956863}%
\pgfsetstrokecolor{currentstroke}%
\pgfsetstrokeopacity{0.200000}%
\pgfsetdash{}{0pt}%
\pgfpathmoveto{\pgfqpoint{2.358876in}{2.760960in}}%
\pgfpathlineto{\pgfqpoint{3.305661in}{2.636178in}}%
\pgfusepath{stroke}%
\end{pgfscope}%
\begin{pgfscope}%
\pgfpathrectangle{\pgfqpoint{0.481978in}{0.331635in}}{\pgfqpoint{4.960000in}{3.696000in}}%
\pgfusepath{clip}%
\pgfsetrectcap%
\pgfsetroundjoin%
\pgfsetlinewidth{1.505625pt}%
\definecolor{currentstroke}{rgb}{0.631373,0.788235,0.956863}%
\pgfsetstrokecolor{currentstroke}%
\pgfsetstrokeopacity{0.200000}%
\pgfsetdash{}{0pt}%
\pgfpathmoveto{\pgfqpoint{2.499804in}{2.278444in}}%
\pgfpathlineto{\pgfqpoint{3.305661in}{2.636178in}}%
\pgfusepath{stroke}%
\end{pgfscope}%
\begin{pgfscope}%
\pgfpathrectangle{\pgfqpoint{0.481978in}{0.331635in}}{\pgfqpoint{4.960000in}{3.696000in}}%
\pgfusepath{clip}%
\pgfsetrectcap%
\pgfsetroundjoin%
\pgfsetlinewidth{1.505625pt}%
\definecolor{currentstroke}{rgb}{0.631373,0.788235,0.956863}%
\pgfsetstrokecolor{currentstroke}%
\pgfsetstrokeopacity{0.200000}%
\pgfsetdash{}{0pt}%
\pgfpathmoveto{\pgfqpoint{2.497801in}{3.076709in}}%
\pgfpathlineto{\pgfqpoint{3.305661in}{2.636178in}}%
\pgfusepath{stroke}%
\end{pgfscope}%
\begin{pgfscope}%
\pgfpathrectangle{\pgfqpoint{0.481978in}{0.331635in}}{\pgfqpoint{4.960000in}{3.696000in}}%
\pgfusepath{clip}%
\pgfsetrectcap%
\pgfsetroundjoin%
\pgfsetlinewidth{1.505625pt}%
\definecolor{currentstroke}{rgb}{0.631373,0.788235,0.956863}%
\pgfsetstrokecolor{currentstroke}%
\pgfsetstrokeopacity{0.200000}%
\pgfsetdash{}{0pt}%
\pgfpathmoveto{\pgfqpoint{2.841725in}{1.827631in}}%
\pgfpathlineto{\pgfqpoint{3.305661in}{2.636178in}}%
\pgfusepath{stroke}%
\end{pgfscope}%
\begin{pgfscope}%
\pgfpathrectangle{\pgfqpoint{0.481978in}{0.331635in}}{\pgfqpoint{4.960000in}{3.696000in}}%
\pgfusepath{clip}%
\pgfsetrectcap%
\pgfsetroundjoin%
\pgfsetlinewidth{1.505625pt}%
\definecolor{currentstroke}{rgb}{0.631373,0.788235,0.956863}%
\pgfsetstrokecolor{currentstroke}%
\pgfsetstrokeopacity{0.200000}%
\pgfsetdash{}{0pt}%
\pgfpathmoveto{\pgfqpoint{4.583715in}{3.435081in}}%
\pgfpathlineto{\pgfqpoint{3.305661in}{2.636178in}}%
\pgfusepath{stroke}%
\end{pgfscope}%
\begin{pgfscope}%
\pgfpathrectangle{\pgfqpoint{0.481978in}{0.331635in}}{\pgfqpoint{4.960000in}{3.696000in}}%
\pgfusepath{clip}%
\pgfsetrectcap%
\pgfsetroundjoin%
\pgfsetlinewidth{1.505625pt}%
\definecolor{currentstroke}{rgb}{0.631373,0.788235,0.956863}%
\pgfsetstrokecolor{currentstroke}%
\pgfsetstrokeopacity{0.200000}%
\pgfsetdash{}{0pt}%
\pgfpathmoveto{\pgfqpoint{4.090488in}{3.339975in}}%
\pgfpathlineto{\pgfqpoint{3.305661in}{2.636178in}}%
\pgfusepath{stroke}%
\end{pgfscope}%
\begin{pgfscope}%
\pgfpathrectangle{\pgfqpoint{0.481978in}{0.331635in}}{\pgfqpoint{4.960000in}{3.696000in}}%
\pgfusepath{clip}%
\pgfsetrectcap%
\pgfsetroundjoin%
\pgfsetlinewidth{1.505625pt}%
\definecolor{currentstroke}{rgb}{0.631373,0.788235,0.956863}%
\pgfsetstrokecolor{currentstroke}%
\pgfsetstrokeopacity{0.200000}%
\pgfsetdash{}{0pt}%
\pgfpathmoveto{\pgfqpoint{3.754263in}{2.429536in}}%
\pgfpathlineto{\pgfqpoint{3.305661in}{2.636178in}}%
\pgfusepath{stroke}%
\end{pgfscope}%
\begin{pgfscope}%
\pgfpathrectangle{\pgfqpoint{0.481978in}{0.331635in}}{\pgfqpoint{4.960000in}{3.696000in}}%
\pgfusepath{clip}%
\pgfsetrectcap%
\pgfsetroundjoin%
\pgfsetlinewidth{1.505625pt}%
\definecolor{currentstroke}{rgb}{0.631373,0.788235,0.956863}%
\pgfsetstrokecolor{currentstroke}%
\pgfsetstrokeopacity{0.200000}%
\pgfsetdash{}{0pt}%
\pgfpathmoveto{\pgfqpoint{3.020955in}{2.314424in}}%
\pgfpathlineto{\pgfqpoint{3.305661in}{2.636178in}}%
\pgfusepath{stroke}%
\end{pgfscope}%
\begin{pgfscope}%
\pgfpathrectangle{\pgfqpoint{0.481978in}{0.331635in}}{\pgfqpoint{4.960000in}{3.696000in}}%
\pgfusepath{clip}%
\pgfsetrectcap%
\pgfsetroundjoin%
\pgfsetlinewidth{1.505625pt}%
\definecolor{currentstroke}{rgb}{0.631373,0.788235,0.956863}%
\pgfsetstrokecolor{currentstroke}%
\pgfsetstrokeopacity{0.200000}%
\pgfsetdash{}{0pt}%
\pgfpathmoveto{\pgfqpoint{3.823159in}{1.053603in}}%
\pgfpathlineto{\pgfqpoint{3.305661in}{2.636178in}}%
\pgfusepath{stroke}%
\end{pgfscope}%
\begin{pgfscope}%
\pgfpathrectangle{\pgfqpoint{0.481978in}{0.331635in}}{\pgfqpoint{4.960000in}{3.696000in}}%
\pgfusepath{clip}%
\pgfsetrectcap%
\pgfsetroundjoin%
\pgfsetlinewidth{1.505625pt}%
\definecolor{currentstroke}{rgb}{0.631373,0.788235,0.956863}%
\pgfsetstrokecolor{currentstroke}%
\pgfsetstrokeopacity{0.200000}%
\pgfsetdash{}{0pt}%
\pgfpathmoveto{\pgfqpoint{3.618546in}{1.973905in}}%
\pgfpathlineto{\pgfqpoint{3.305661in}{2.636178in}}%
\pgfusepath{stroke}%
\end{pgfscope}%
\begin{pgfscope}%
\pgfpathrectangle{\pgfqpoint{0.481978in}{0.331635in}}{\pgfqpoint{4.960000in}{3.696000in}}%
\pgfusepath{clip}%
\pgfsetrectcap%
\pgfsetroundjoin%
\pgfsetlinewidth{1.505625pt}%
\definecolor{currentstroke}{rgb}{0.631373,0.788235,0.956863}%
\pgfsetstrokecolor{currentstroke}%
\pgfsetstrokeopacity{0.200000}%
\pgfsetdash{}{0pt}%
\pgfpathmoveto{\pgfqpoint{2.396516in}{1.892574in}}%
\pgfpathlineto{\pgfqpoint{3.305661in}{2.636178in}}%
\pgfusepath{stroke}%
\end{pgfscope}%
\begin{pgfscope}%
\pgfpathrectangle{\pgfqpoint{0.481978in}{0.331635in}}{\pgfqpoint{4.960000in}{3.696000in}}%
\pgfusepath{clip}%
\pgfsetrectcap%
\pgfsetroundjoin%
\pgfsetlinewidth{1.505625pt}%
\definecolor{currentstroke}{rgb}{0.631373,0.788235,0.956863}%
\pgfsetstrokecolor{currentstroke}%
\pgfsetstrokeopacity{0.200000}%
\pgfsetdash{}{0pt}%
\pgfpathmoveto{\pgfqpoint{2.127524in}{2.988190in}}%
\pgfpathlineto{\pgfqpoint{3.305661in}{2.636178in}}%
\pgfusepath{stroke}%
\end{pgfscope}%
\begin{pgfscope}%
\pgfpathrectangle{\pgfqpoint{0.481978in}{0.331635in}}{\pgfqpoint{4.960000in}{3.696000in}}%
\pgfusepath{clip}%
\pgfsetrectcap%
\pgfsetroundjoin%
\pgfsetlinewidth{1.505625pt}%
\definecolor{currentstroke}{rgb}{0.631373,0.788235,0.956863}%
\pgfsetstrokecolor{currentstroke}%
\pgfsetstrokeopacity{0.200000}%
\pgfsetdash{}{0pt}%
\pgfpathmoveto{\pgfqpoint{2.834203in}{2.040596in}}%
\pgfpathlineto{\pgfqpoint{3.305661in}{2.636178in}}%
\pgfusepath{stroke}%
\end{pgfscope}%
\begin{pgfscope}%
\pgfpathrectangle{\pgfqpoint{0.481978in}{0.331635in}}{\pgfqpoint{4.960000in}{3.696000in}}%
\pgfusepath{clip}%
\pgfsetrectcap%
\pgfsetroundjoin%
\pgfsetlinewidth{1.505625pt}%
\definecolor{currentstroke}{rgb}{0.631373,0.788235,0.956863}%
\pgfsetstrokecolor{currentstroke}%
\pgfsetstrokeopacity{0.200000}%
\pgfsetdash{}{0pt}%
\pgfpathmoveto{\pgfqpoint{3.622997in}{3.745756in}}%
\pgfpathlineto{\pgfqpoint{3.305661in}{2.636178in}}%
\pgfusepath{stroke}%
\end{pgfscope}%
\begin{pgfscope}%
\pgfpathrectangle{\pgfqpoint{0.481978in}{0.331635in}}{\pgfqpoint{4.960000in}{3.696000in}}%
\pgfusepath{clip}%
\pgfsetrectcap%
\pgfsetroundjoin%
\pgfsetlinewidth{1.505625pt}%
\definecolor{currentstroke}{rgb}{0.631373,0.788235,0.956863}%
\pgfsetstrokecolor{currentstroke}%
\pgfsetstrokeopacity{0.200000}%
\pgfsetdash{}{0pt}%
\pgfpathmoveto{\pgfqpoint{4.452381in}{3.292273in}}%
\pgfpathlineto{\pgfqpoint{3.305661in}{2.636178in}}%
\pgfusepath{stroke}%
\end{pgfscope}%
\begin{pgfscope}%
\pgfpathrectangle{\pgfqpoint{0.481978in}{0.331635in}}{\pgfqpoint{4.960000in}{3.696000in}}%
\pgfusepath{clip}%
\pgfsetrectcap%
\pgfsetroundjoin%
\pgfsetlinewidth{1.505625pt}%
\definecolor{currentstroke}{rgb}{0.631373,0.788235,0.956863}%
\pgfsetstrokecolor{currentstroke}%
\pgfsetstrokeopacity{0.200000}%
\pgfsetdash{}{0pt}%
\pgfpathmoveto{\pgfqpoint{3.564111in}{2.459448in}}%
\pgfpathlineto{\pgfqpoint{3.305661in}{2.636178in}}%
\pgfusepath{stroke}%
\end{pgfscope}%
\begin{pgfscope}%
\pgfpathrectangle{\pgfqpoint{0.481978in}{0.331635in}}{\pgfqpoint{4.960000in}{3.696000in}}%
\pgfusepath{clip}%
\pgfsetrectcap%
\pgfsetroundjoin%
\pgfsetlinewidth{1.505625pt}%
\definecolor{currentstroke}{rgb}{0.631373,0.788235,0.956863}%
\pgfsetstrokecolor{currentstroke}%
\pgfsetstrokeopacity{0.200000}%
\pgfsetdash{}{0pt}%
\pgfpathmoveto{\pgfqpoint{4.549652in}{2.472464in}}%
\pgfpathlineto{\pgfqpoint{3.305661in}{2.636178in}}%
\pgfusepath{stroke}%
\end{pgfscope}%
\begin{pgfscope}%
\pgfpathrectangle{\pgfqpoint{0.481978in}{0.331635in}}{\pgfqpoint{4.960000in}{3.696000in}}%
\pgfusepath{clip}%
\pgfsetrectcap%
\pgfsetroundjoin%
\pgfsetlinewidth{1.505625pt}%
\definecolor{currentstroke}{rgb}{0.631373,0.788235,0.956863}%
\pgfsetstrokecolor{currentstroke}%
\pgfsetstrokeopacity{0.200000}%
\pgfsetdash{}{0pt}%
\pgfpathmoveto{\pgfqpoint{2.673695in}{1.842428in}}%
\pgfpathlineto{\pgfqpoint{3.305661in}{2.636178in}}%
\pgfusepath{stroke}%
\end{pgfscope}%
\begin{pgfscope}%
\pgfpathrectangle{\pgfqpoint{0.481978in}{0.331635in}}{\pgfqpoint{4.960000in}{3.696000in}}%
\pgfusepath{clip}%
\pgfsetrectcap%
\pgfsetroundjoin%
\pgfsetlinewidth{1.505625pt}%
\definecolor{currentstroke}{rgb}{0.631373,0.788235,0.956863}%
\pgfsetstrokecolor{currentstroke}%
\pgfsetstrokeopacity{0.200000}%
\pgfsetdash{}{0pt}%
\pgfpathmoveto{\pgfqpoint{3.995642in}{3.655007in}}%
\pgfpathlineto{\pgfqpoint{3.305661in}{2.636178in}}%
\pgfusepath{stroke}%
\end{pgfscope}%
\begin{pgfscope}%
\pgfpathrectangle{\pgfqpoint{0.481978in}{0.331635in}}{\pgfqpoint{4.960000in}{3.696000in}}%
\pgfusepath{clip}%
\pgfsetrectcap%
\pgfsetroundjoin%
\pgfsetlinewidth{1.505625pt}%
\definecolor{currentstroke}{rgb}{0.631373,0.788235,0.956863}%
\pgfsetstrokecolor{currentstroke}%
\pgfsetstrokeopacity{0.200000}%
\pgfsetdash{}{0pt}%
\pgfpathmoveto{\pgfqpoint{2.233135in}{2.021062in}}%
\pgfpathlineto{\pgfqpoint{3.305661in}{2.636178in}}%
\pgfusepath{stroke}%
\end{pgfscope}%
\begin{pgfscope}%
\pgfpathrectangle{\pgfqpoint{0.481978in}{0.331635in}}{\pgfqpoint{4.960000in}{3.696000in}}%
\pgfusepath{clip}%
\pgfsetrectcap%
\pgfsetroundjoin%
\pgfsetlinewidth{1.505625pt}%
\definecolor{currentstroke}{rgb}{0.631373,0.788235,0.956863}%
\pgfsetstrokecolor{currentstroke}%
\pgfsetstrokeopacity{0.200000}%
\pgfsetdash{}{0pt}%
\pgfpathmoveto{\pgfqpoint{4.693644in}{3.329839in}}%
\pgfpathlineto{\pgfqpoint{3.305661in}{2.636178in}}%
\pgfusepath{stroke}%
\end{pgfscope}%
\begin{pgfscope}%
\pgfpathrectangle{\pgfqpoint{0.481978in}{0.331635in}}{\pgfqpoint{4.960000in}{3.696000in}}%
\pgfusepath{clip}%
\pgfsetrectcap%
\pgfsetroundjoin%
\pgfsetlinewidth{1.505625pt}%
\definecolor{currentstroke}{rgb}{0.631373,0.788235,0.956863}%
\pgfsetstrokecolor{currentstroke}%
\pgfsetstrokeopacity{0.200000}%
\pgfsetdash{}{0pt}%
\pgfpathmoveto{\pgfqpoint{2.693461in}{2.938555in}}%
\pgfpathlineto{\pgfqpoint{3.305661in}{2.636178in}}%
\pgfusepath{stroke}%
\end{pgfscope}%
\begin{pgfscope}%
\pgfpathrectangle{\pgfqpoint{0.481978in}{0.331635in}}{\pgfqpoint{4.960000in}{3.696000in}}%
\pgfusepath{clip}%
\pgfsetrectcap%
\pgfsetroundjoin%
\pgfsetlinewidth{1.505625pt}%
\definecolor{currentstroke}{rgb}{0.631373,0.788235,0.956863}%
\pgfsetstrokecolor{currentstroke}%
\pgfsetstrokeopacity{0.200000}%
\pgfsetdash{}{0pt}%
\pgfpathmoveto{\pgfqpoint{3.990681in}{1.697404in}}%
\pgfpathlineto{\pgfqpoint{3.305661in}{2.636178in}}%
\pgfusepath{stroke}%
\end{pgfscope}%
\begin{pgfscope}%
\pgfpathrectangle{\pgfqpoint{0.481978in}{0.331635in}}{\pgfqpoint{4.960000in}{3.696000in}}%
\pgfusepath{clip}%
\pgfsetrectcap%
\pgfsetroundjoin%
\pgfsetlinewidth{1.505625pt}%
\definecolor{currentstroke}{rgb}{0.631373,0.788235,0.956863}%
\pgfsetstrokecolor{currentstroke}%
\pgfsetstrokeopacity{0.200000}%
\pgfsetdash{}{0pt}%
\pgfpathmoveto{\pgfqpoint{3.841627in}{3.230188in}}%
\pgfpathlineto{\pgfqpoint{3.305661in}{2.636178in}}%
\pgfusepath{stroke}%
\end{pgfscope}%
\begin{pgfscope}%
\pgfpathrectangle{\pgfqpoint{0.481978in}{0.331635in}}{\pgfqpoint{4.960000in}{3.696000in}}%
\pgfusepath{clip}%
\pgfsetrectcap%
\pgfsetroundjoin%
\pgfsetlinewidth{1.505625pt}%
\definecolor{currentstroke}{rgb}{0.631373,0.788235,0.956863}%
\pgfsetstrokecolor{currentstroke}%
\pgfsetstrokeopacity{0.200000}%
\pgfsetdash{}{0pt}%
\pgfpathmoveto{\pgfqpoint{1.920925in}{2.570158in}}%
\pgfpathlineto{\pgfqpoint{3.305661in}{2.636178in}}%
\pgfusepath{stroke}%
\end{pgfscope}%
\begin{pgfscope}%
\pgfpathrectangle{\pgfqpoint{0.481978in}{0.331635in}}{\pgfqpoint{4.960000in}{3.696000in}}%
\pgfusepath{clip}%
\pgfsetrectcap%
\pgfsetroundjoin%
\pgfsetlinewidth{1.505625pt}%
\definecolor{currentstroke}{rgb}{0.631373,0.788235,0.956863}%
\pgfsetstrokecolor{currentstroke}%
\pgfsetstrokeopacity{0.200000}%
\pgfsetdash{}{0pt}%
\pgfpathmoveto{\pgfqpoint{3.357008in}{2.125815in}}%
\pgfpathlineto{\pgfqpoint{3.305661in}{2.636178in}}%
\pgfusepath{stroke}%
\end{pgfscope}%
\begin{pgfscope}%
\pgfpathrectangle{\pgfqpoint{0.481978in}{0.331635in}}{\pgfqpoint{4.960000in}{3.696000in}}%
\pgfusepath{clip}%
\pgfsetrectcap%
\pgfsetroundjoin%
\pgfsetlinewidth{1.505625pt}%
\definecolor{currentstroke}{rgb}{0.631373,0.788235,0.956863}%
\pgfsetstrokecolor{currentstroke}%
\pgfsetstrokeopacity{0.200000}%
\pgfsetdash{}{0pt}%
\pgfpathmoveto{\pgfqpoint{2.424795in}{2.006380in}}%
\pgfpathlineto{\pgfqpoint{3.305661in}{2.636178in}}%
\pgfusepath{stroke}%
\end{pgfscope}%
\begin{pgfscope}%
\pgfpathrectangle{\pgfqpoint{0.481978in}{0.331635in}}{\pgfqpoint{4.960000in}{3.696000in}}%
\pgfusepath{clip}%
\pgfsetrectcap%
\pgfsetroundjoin%
\pgfsetlinewidth{1.505625pt}%
\definecolor{currentstroke}{rgb}{0.631373,0.788235,0.956863}%
\pgfsetstrokecolor{currentstroke}%
\pgfsetstrokeopacity{0.200000}%
\pgfsetdash{}{0pt}%
\pgfpathmoveto{\pgfqpoint{2.825282in}{2.266982in}}%
\pgfpathlineto{\pgfqpoint{3.305661in}{2.636178in}}%
\pgfusepath{stroke}%
\end{pgfscope}%
\begin{pgfscope}%
\pgfpathrectangle{\pgfqpoint{0.481978in}{0.331635in}}{\pgfqpoint{4.960000in}{3.696000in}}%
\pgfusepath{clip}%
\pgfsetrectcap%
\pgfsetroundjoin%
\pgfsetlinewidth{1.505625pt}%
\definecolor{currentstroke}{rgb}{0.631373,0.788235,0.956863}%
\pgfsetstrokecolor{currentstroke}%
\pgfsetstrokeopacity{0.200000}%
\pgfsetdash{}{0pt}%
\pgfpathmoveto{\pgfqpoint{3.718236in}{3.005788in}}%
\pgfpathlineto{\pgfqpoint{3.305661in}{2.636178in}}%
\pgfusepath{stroke}%
\end{pgfscope}%
\begin{pgfscope}%
\pgfpathrectangle{\pgfqpoint{0.481978in}{0.331635in}}{\pgfqpoint{4.960000in}{3.696000in}}%
\pgfusepath{clip}%
\pgfsetrectcap%
\pgfsetroundjoin%
\pgfsetlinewidth{1.505625pt}%
\definecolor{currentstroke}{rgb}{0.631373,0.788235,0.956863}%
\pgfsetstrokecolor{currentstroke}%
\pgfsetstrokeopacity{0.200000}%
\pgfsetdash{}{0pt}%
\pgfpathmoveto{\pgfqpoint{1.979955in}{2.570916in}}%
\pgfpathlineto{\pgfqpoint{3.305661in}{2.636178in}}%
\pgfusepath{stroke}%
\end{pgfscope}%
\begin{pgfscope}%
\pgfpathrectangle{\pgfqpoint{0.481978in}{0.331635in}}{\pgfqpoint{4.960000in}{3.696000in}}%
\pgfusepath{clip}%
\pgfsetrectcap%
\pgfsetroundjoin%
\pgfsetlinewidth{1.505625pt}%
\definecolor{currentstroke}{rgb}{0.631373,0.788235,0.956863}%
\pgfsetstrokecolor{currentstroke}%
\pgfsetstrokeopacity{0.200000}%
\pgfsetdash{}{0pt}%
\pgfpathmoveto{\pgfqpoint{3.620308in}{3.747190in}}%
\pgfpathlineto{\pgfqpoint{3.305661in}{2.636178in}}%
\pgfusepath{stroke}%
\end{pgfscope}%
\begin{pgfscope}%
\pgfpathrectangle{\pgfqpoint{0.481978in}{0.331635in}}{\pgfqpoint{4.960000in}{3.696000in}}%
\pgfusepath{clip}%
\pgfsetrectcap%
\pgfsetroundjoin%
\pgfsetlinewidth{1.505625pt}%
\definecolor{currentstroke}{rgb}{0.631373,0.788235,0.956863}%
\pgfsetstrokecolor{currentstroke}%
\pgfsetstrokeopacity{0.200000}%
\pgfsetdash{}{0pt}%
\pgfpathmoveto{\pgfqpoint{3.766377in}{3.183360in}}%
\pgfpathlineto{\pgfqpoint{3.305661in}{2.636178in}}%
\pgfusepath{stroke}%
\end{pgfscope}%
\begin{pgfscope}%
\pgfpathrectangle{\pgfqpoint{0.481978in}{0.331635in}}{\pgfqpoint{4.960000in}{3.696000in}}%
\pgfusepath{clip}%
\pgfsetrectcap%
\pgfsetroundjoin%
\pgfsetlinewidth{1.505625pt}%
\definecolor{currentstroke}{rgb}{0.631373,0.788235,0.956863}%
\pgfsetstrokecolor{currentstroke}%
\pgfsetstrokeopacity{0.200000}%
\pgfsetdash{}{0pt}%
\pgfpathmoveto{\pgfqpoint{3.211425in}{2.564968in}}%
\pgfpathlineto{\pgfqpoint{3.305661in}{2.636178in}}%
\pgfusepath{stroke}%
\end{pgfscope}%
\begin{pgfscope}%
\pgfpathrectangle{\pgfqpoint{0.481978in}{0.331635in}}{\pgfqpoint{4.960000in}{3.696000in}}%
\pgfusepath{clip}%
\pgfsetrectcap%
\pgfsetroundjoin%
\pgfsetlinewidth{1.505625pt}%
\definecolor{currentstroke}{rgb}{0.631373,0.788235,0.956863}%
\pgfsetstrokecolor{currentstroke}%
\pgfsetstrokeopacity{0.200000}%
\pgfsetdash{}{0pt}%
\pgfpathmoveto{\pgfqpoint{3.346413in}{1.878709in}}%
\pgfpathlineto{\pgfqpoint{3.305661in}{2.636178in}}%
\pgfusepath{stroke}%
\end{pgfscope}%
\begin{pgfscope}%
\pgfpathrectangle{\pgfqpoint{0.481978in}{0.331635in}}{\pgfqpoint{4.960000in}{3.696000in}}%
\pgfusepath{clip}%
\pgfsetrectcap%
\pgfsetroundjoin%
\pgfsetlinewidth{1.505625pt}%
\definecolor{currentstroke}{rgb}{0.631373,0.788235,0.956863}%
\pgfsetstrokecolor{currentstroke}%
\pgfsetstrokeopacity{0.200000}%
\pgfsetdash{}{0pt}%
\pgfpathmoveto{\pgfqpoint{2.443805in}{2.170878in}}%
\pgfpathlineto{\pgfqpoint{3.305661in}{2.636178in}}%
\pgfusepath{stroke}%
\end{pgfscope}%
\begin{pgfscope}%
\pgfpathrectangle{\pgfqpoint{0.481978in}{0.331635in}}{\pgfqpoint{4.960000in}{3.696000in}}%
\pgfusepath{clip}%
\pgfsetrectcap%
\pgfsetroundjoin%
\pgfsetlinewidth{1.505625pt}%
\definecolor{currentstroke}{rgb}{0.631373,0.788235,0.956863}%
\pgfsetstrokecolor{currentstroke}%
\pgfsetstrokeopacity{0.200000}%
\pgfsetdash{}{0pt}%
\pgfpathmoveto{\pgfqpoint{2.510852in}{2.881761in}}%
\pgfpathlineto{\pgfqpoint{3.305661in}{2.636178in}}%
\pgfusepath{stroke}%
\end{pgfscope}%
\begin{pgfscope}%
\pgfpathrectangle{\pgfqpoint{0.481978in}{0.331635in}}{\pgfqpoint{4.960000in}{3.696000in}}%
\pgfusepath{clip}%
\pgfsetrectcap%
\pgfsetroundjoin%
\pgfsetlinewidth{1.505625pt}%
\definecolor{currentstroke}{rgb}{0.631373,0.788235,0.956863}%
\pgfsetstrokecolor{currentstroke}%
\pgfsetstrokeopacity{0.200000}%
\pgfsetdash{}{0pt}%
\pgfpathmoveto{\pgfqpoint{2.573389in}{2.360640in}}%
\pgfpathlineto{\pgfqpoint{3.305661in}{2.636178in}}%
\pgfusepath{stroke}%
\end{pgfscope}%
\begin{pgfscope}%
\pgfpathrectangle{\pgfqpoint{0.481978in}{0.331635in}}{\pgfqpoint{4.960000in}{3.696000in}}%
\pgfusepath{clip}%
\pgfsetrectcap%
\pgfsetroundjoin%
\pgfsetlinewidth{1.505625pt}%
\definecolor{currentstroke}{rgb}{0.631373,0.788235,0.956863}%
\pgfsetstrokecolor{currentstroke}%
\pgfsetstrokeopacity{0.200000}%
\pgfsetdash{}{0pt}%
\pgfpathmoveto{\pgfqpoint{3.746336in}{3.642887in}}%
\pgfpathlineto{\pgfqpoint{3.305661in}{2.636178in}}%
\pgfusepath{stroke}%
\end{pgfscope}%
\begin{pgfscope}%
\pgfpathrectangle{\pgfqpoint{0.481978in}{0.331635in}}{\pgfqpoint{4.960000in}{3.696000in}}%
\pgfusepath{clip}%
\pgfsetrectcap%
\pgfsetroundjoin%
\pgfsetlinewidth{1.505625pt}%
\definecolor{currentstroke}{rgb}{0.631373,0.788235,0.956863}%
\pgfsetstrokecolor{currentstroke}%
\pgfsetstrokeopacity{0.200000}%
\pgfsetdash{}{0pt}%
\pgfpathmoveto{\pgfqpoint{3.949319in}{3.468645in}}%
\pgfpathlineto{\pgfqpoint{3.305661in}{2.636178in}}%
\pgfusepath{stroke}%
\end{pgfscope}%
\begin{pgfscope}%
\pgfpathrectangle{\pgfqpoint{0.481978in}{0.331635in}}{\pgfqpoint{4.960000in}{3.696000in}}%
\pgfusepath{clip}%
\pgfsetrectcap%
\pgfsetroundjoin%
\pgfsetlinewidth{1.505625pt}%
\definecolor{currentstroke}{rgb}{0.631373,0.788235,0.956863}%
\pgfsetstrokecolor{currentstroke}%
\pgfsetstrokeopacity{0.200000}%
\pgfsetdash{}{0pt}%
\pgfpathmoveto{\pgfqpoint{3.824287in}{2.350452in}}%
\pgfpathlineto{\pgfqpoint{3.305661in}{2.636178in}}%
\pgfusepath{stroke}%
\end{pgfscope}%
\begin{pgfscope}%
\pgfpathrectangle{\pgfqpoint{0.481978in}{0.331635in}}{\pgfqpoint{4.960000in}{3.696000in}}%
\pgfusepath{clip}%
\pgfsetrectcap%
\pgfsetroundjoin%
\pgfsetlinewidth{1.505625pt}%
\definecolor{currentstroke}{rgb}{0.631373,0.788235,0.956863}%
\pgfsetstrokecolor{currentstroke}%
\pgfsetstrokeopacity{0.200000}%
\pgfsetdash{}{0pt}%
\pgfpathmoveto{\pgfqpoint{2.636201in}{1.132227in}}%
\pgfpathlineto{\pgfqpoint{3.305661in}{2.636178in}}%
\pgfusepath{stroke}%
\end{pgfscope}%
\begin{pgfscope}%
\pgfpathrectangle{\pgfqpoint{0.481978in}{0.331635in}}{\pgfqpoint{4.960000in}{3.696000in}}%
\pgfusepath{clip}%
\pgfsetrectcap%
\pgfsetroundjoin%
\pgfsetlinewidth{1.505625pt}%
\definecolor{currentstroke}{rgb}{0.631373,0.788235,0.956863}%
\pgfsetstrokecolor{currentstroke}%
\pgfsetstrokeopacity{0.200000}%
\pgfsetdash{}{0pt}%
\pgfpathmoveto{\pgfqpoint{3.807379in}{3.774814in}}%
\pgfpathlineto{\pgfqpoint{3.305661in}{2.636178in}}%
\pgfusepath{stroke}%
\end{pgfscope}%
\begin{pgfscope}%
\pgfpathrectangle{\pgfqpoint{0.481978in}{0.331635in}}{\pgfqpoint{4.960000in}{3.696000in}}%
\pgfusepath{clip}%
\pgfsetrectcap%
\pgfsetroundjoin%
\pgfsetlinewidth{1.505625pt}%
\definecolor{currentstroke}{rgb}{0.631373,0.788235,0.956863}%
\pgfsetstrokecolor{currentstroke}%
\pgfsetstrokeopacity{0.200000}%
\pgfsetdash{}{0pt}%
\pgfpathmoveto{\pgfqpoint{3.979307in}{3.353614in}}%
\pgfpathlineto{\pgfqpoint{3.305661in}{2.636178in}}%
\pgfusepath{stroke}%
\end{pgfscope}%
\begin{pgfscope}%
\pgfpathrectangle{\pgfqpoint{0.481978in}{0.331635in}}{\pgfqpoint{4.960000in}{3.696000in}}%
\pgfusepath{clip}%
\pgfsetrectcap%
\pgfsetroundjoin%
\pgfsetlinewidth{1.505625pt}%
\definecolor{currentstroke}{rgb}{0.631373,0.788235,0.956863}%
\pgfsetstrokecolor{currentstroke}%
\pgfsetstrokeopacity{0.200000}%
\pgfsetdash{}{0pt}%
\pgfpathmoveto{\pgfqpoint{3.363069in}{3.859635in}}%
\pgfpathlineto{\pgfqpoint{3.305661in}{2.636178in}}%
\pgfusepath{stroke}%
\end{pgfscope}%
\begin{pgfscope}%
\pgfpathrectangle{\pgfqpoint{0.481978in}{0.331635in}}{\pgfqpoint{4.960000in}{3.696000in}}%
\pgfusepath{clip}%
\pgfsetrectcap%
\pgfsetroundjoin%
\pgfsetlinewidth{1.505625pt}%
\definecolor{currentstroke}{rgb}{0.631373,0.788235,0.956863}%
\pgfsetstrokecolor{currentstroke}%
\pgfsetstrokeopacity{0.200000}%
\pgfsetdash{}{0pt}%
\pgfpathmoveto{\pgfqpoint{3.678431in}{2.786399in}}%
\pgfpathlineto{\pgfqpoint{3.305661in}{2.636178in}}%
\pgfusepath{stroke}%
\end{pgfscope}%
\begin{pgfscope}%
\pgfpathrectangle{\pgfqpoint{0.481978in}{0.331635in}}{\pgfqpoint{4.960000in}{3.696000in}}%
\pgfusepath{clip}%
\pgfsetrectcap%
\pgfsetroundjoin%
\pgfsetlinewidth{1.505625pt}%
\definecolor{currentstroke}{rgb}{0.631373,0.788235,0.956863}%
\pgfsetstrokecolor{currentstroke}%
\pgfsetstrokeopacity{0.200000}%
\pgfsetdash{}{0pt}%
\pgfpathmoveto{\pgfqpoint{3.253066in}{1.766211in}}%
\pgfpathlineto{\pgfqpoint{3.305661in}{2.636178in}}%
\pgfusepath{stroke}%
\end{pgfscope}%
\begin{pgfscope}%
\pgfpathrectangle{\pgfqpoint{0.481978in}{0.331635in}}{\pgfqpoint{4.960000in}{3.696000in}}%
\pgfusepath{clip}%
\pgfsetrectcap%
\pgfsetroundjoin%
\pgfsetlinewidth{1.505625pt}%
\definecolor{currentstroke}{rgb}{0.631373,0.788235,0.956863}%
\pgfsetstrokecolor{currentstroke}%
\pgfsetstrokeopacity{0.200000}%
\pgfsetdash{}{0pt}%
\pgfpathmoveto{\pgfqpoint{4.487130in}{3.199693in}}%
\pgfpathlineto{\pgfqpoint{3.305661in}{2.636178in}}%
\pgfusepath{stroke}%
\end{pgfscope}%
\begin{pgfscope}%
\pgfpathrectangle{\pgfqpoint{0.481978in}{0.331635in}}{\pgfqpoint{4.960000in}{3.696000in}}%
\pgfusepath{clip}%
\pgfsetrectcap%
\pgfsetroundjoin%
\pgfsetlinewidth{1.505625pt}%
\definecolor{currentstroke}{rgb}{0.631373,0.788235,0.956863}%
\pgfsetstrokecolor{currentstroke}%
\pgfsetstrokeopacity{0.200000}%
\pgfsetdash{}{0pt}%
\pgfpathmoveto{\pgfqpoint{2.748902in}{3.202916in}}%
\pgfpathlineto{\pgfqpoint{3.305661in}{2.636178in}}%
\pgfusepath{stroke}%
\end{pgfscope}%
\begin{pgfscope}%
\pgfpathrectangle{\pgfqpoint{0.481978in}{0.331635in}}{\pgfqpoint{4.960000in}{3.696000in}}%
\pgfusepath{clip}%
\pgfsetrectcap%
\pgfsetroundjoin%
\pgfsetlinewidth{1.505625pt}%
\definecolor{currentstroke}{rgb}{0.631373,0.788235,0.956863}%
\pgfsetstrokecolor{currentstroke}%
\pgfsetstrokeopacity{0.200000}%
\pgfsetdash{}{0pt}%
\pgfpathmoveto{\pgfqpoint{3.134956in}{2.668991in}}%
\pgfpathlineto{\pgfqpoint{3.305661in}{2.636178in}}%
\pgfusepath{stroke}%
\end{pgfscope}%
\begin{pgfscope}%
\pgfpathrectangle{\pgfqpoint{0.481978in}{0.331635in}}{\pgfqpoint{4.960000in}{3.696000in}}%
\pgfusepath{clip}%
\pgfsetrectcap%
\pgfsetroundjoin%
\pgfsetlinewidth{1.505625pt}%
\definecolor{currentstroke}{rgb}{0.631373,0.788235,0.956863}%
\pgfsetstrokecolor{currentstroke}%
\pgfsetstrokeopacity{0.200000}%
\pgfsetdash{}{0pt}%
\pgfpathmoveto{\pgfqpoint{1.770446in}{1.378110in}}%
\pgfpathlineto{\pgfqpoint{3.305661in}{2.636178in}}%
\pgfusepath{stroke}%
\end{pgfscope}%
\begin{pgfscope}%
\pgfpathrectangle{\pgfqpoint{0.481978in}{0.331635in}}{\pgfqpoint{4.960000in}{3.696000in}}%
\pgfusepath{clip}%
\pgfsetrectcap%
\pgfsetroundjoin%
\pgfsetlinewidth{1.505625pt}%
\definecolor{currentstroke}{rgb}{0.631373,0.788235,0.956863}%
\pgfsetstrokecolor{currentstroke}%
\pgfsetstrokeopacity{0.200000}%
\pgfsetdash{}{0pt}%
\pgfpathmoveto{\pgfqpoint{3.227485in}{3.092617in}}%
\pgfpathlineto{\pgfqpoint{3.305661in}{2.636178in}}%
\pgfusepath{stroke}%
\end{pgfscope}%
\begin{pgfscope}%
\pgfpathrectangle{\pgfqpoint{0.481978in}{0.331635in}}{\pgfqpoint{4.960000in}{3.696000in}}%
\pgfusepath{clip}%
\pgfsetrectcap%
\pgfsetroundjoin%
\pgfsetlinewidth{1.505625pt}%
\definecolor{currentstroke}{rgb}{0.631373,0.788235,0.956863}%
\pgfsetstrokecolor{currentstroke}%
\pgfsetstrokeopacity{0.200000}%
\pgfsetdash{}{0pt}%
\pgfpathmoveto{\pgfqpoint{3.036063in}{2.384623in}}%
\pgfpathlineto{\pgfqpoint{3.305661in}{2.636178in}}%
\pgfusepath{stroke}%
\end{pgfscope}%
\begin{pgfscope}%
\pgfpathrectangle{\pgfqpoint{0.481978in}{0.331635in}}{\pgfqpoint{4.960000in}{3.696000in}}%
\pgfusepath{clip}%
\pgfsetrectcap%
\pgfsetroundjoin%
\pgfsetlinewidth{1.505625pt}%
\definecolor{currentstroke}{rgb}{0.631373,0.788235,0.956863}%
\pgfsetstrokecolor{currentstroke}%
\pgfsetstrokeopacity{0.200000}%
\pgfsetdash{}{0pt}%
\pgfpathmoveto{\pgfqpoint{3.957887in}{3.386172in}}%
\pgfpathlineto{\pgfqpoint{3.305661in}{2.636178in}}%
\pgfusepath{stroke}%
\end{pgfscope}%
\begin{pgfscope}%
\pgfpathrectangle{\pgfqpoint{0.481978in}{0.331635in}}{\pgfqpoint{4.960000in}{3.696000in}}%
\pgfusepath{clip}%
\pgfsetrectcap%
\pgfsetroundjoin%
\pgfsetlinewidth{1.505625pt}%
\definecolor{currentstroke}{rgb}{0.631373,0.788235,0.956863}%
\pgfsetstrokecolor{currentstroke}%
\pgfsetstrokeopacity{0.200000}%
\pgfsetdash{}{0pt}%
\pgfpathmoveto{\pgfqpoint{2.208851in}{1.654647in}}%
\pgfpathlineto{\pgfqpoint{3.305661in}{2.636178in}}%
\pgfusepath{stroke}%
\end{pgfscope}%
\begin{pgfscope}%
\pgfpathrectangle{\pgfqpoint{0.481978in}{0.331635in}}{\pgfqpoint{4.960000in}{3.696000in}}%
\pgfusepath{clip}%
\pgfsetrectcap%
\pgfsetroundjoin%
\pgfsetlinewidth{1.505625pt}%
\definecolor{currentstroke}{rgb}{0.631373,0.788235,0.956863}%
\pgfsetstrokecolor{currentstroke}%
\pgfsetstrokeopacity{0.200000}%
\pgfsetdash{}{0pt}%
\pgfpathmoveto{\pgfqpoint{2.170244in}{3.170597in}}%
\pgfpathlineto{\pgfqpoint{3.305661in}{2.636178in}}%
\pgfusepath{stroke}%
\end{pgfscope}%
\begin{pgfscope}%
\pgfpathrectangle{\pgfqpoint{0.481978in}{0.331635in}}{\pgfqpoint{4.960000in}{3.696000in}}%
\pgfusepath{clip}%
\pgfsetrectcap%
\pgfsetroundjoin%
\pgfsetlinewidth{1.505625pt}%
\definecolor{currentstroke}{rgb}{0.631373,0.788235,0.956863}%
\pgfsetstrokecolor{currentstroke}%
\pgfsetstrokeopacity{0.200000}%
\pgfsetdash{}{0pt}%
\pgfpathmoveto{\pgfqpoint{4.488576in}{1.704747in}}%
\pgfpathlineto{\pgfqpoint{3.305661in}{2.636178in}}%
\pgfusepath{stroke}%
\end{pgfscope}%
\begin{pgfscope}%
\pgfpathrectangle{\pgfqpoint{0.481978in}{0.331635in}}{\pgfqpoint{4.960000in}{3.696000in}}%
\pgfusepath{clip}%
\pgfsetrectcap%
\pgfsetroundjoin%
\pgfsetlinewidth{1.505625pt}%
\definecolor{currentstroke}{rgb}{0.631373,0.788235,0.956863}%
\pgfsetstrokecolor{currentstroke}%
\pgfsetstrokeopacity{0.200000}%
\pgfsetdash{}{0pt}%
\pgfpathmoveto{\pgfqpoint{2.302687in}{1.879132in}}%
\pgfpathlineto{\pgfqpoint{3.305661in}{2.636178in}}%
\pgfusepath{stroke}%
\end{pgfscope}%
\begin{pgfscope}%
\pgfpathrectangle{\pgfqpoint{0.481978in}{0.331635in}}{\pgfqpoint{4.960000in}{3.696000in}}%
\pgfusepath{clip}%
\pgfsetrectcap%
\pgfsetroundjoin%
\pgfsetlinewidth{1.505625pt}%
\definecolor{currentstroke}{rgb}{0.631373,0.788235,0.956863}%
\pgfsetstrokecolor{currentstroke}%
\pgfsetstrokeopacity{0.200000}%
\pgfsetdash{}{0pt}%
\pgfpathmoveto{\pgfqpoint{4.203402in}{3.648168in}}%
\pgfpathlineto{\pgfqpoint{3.305661in}{2.636178in}}%
\pgfusepath{stroke}%
\end{pgfscope}%
\begin{pgfscope}%
\pgfpathrectangle{\pgfqpoint{0.481978in}{0.331635in}}{\pgfqpoint{4.960000in}{3.696000in}}%
\pgfusepath{clip}%
\pgfsetrectcap%
\pgfsetroundjoin%
\pgfsetlinewidth{1.505625pt}%
\definecolor{currentstroke}{rgb}{0.631373,0.788235,0.956863}%
\pgfsetstrokecolor{currentstroke}%
\pgfsetstrokeopacity{0.200000}%
\pgfsetdash{}{0pt}%
\pgfpathmoveto{\pgfqpoint{2.511684in}{2.574822in}}%
\pgfpathlineto{\pgfqpoint{3.305661in}{2.636178in}}%
\pgfusepath{stroke}%
\end{pgfscope}%
\begin{pgfscope}%
\pgfpathrectangle{\pgfqpoint{0.481978in}{0.331635in}}{\pgfqpoint{4.960000in}{3.696000in}}%
\pgfusepath{clip}%
\pgfsetrectcap%
\pgfsetroundjoin%
\pgfsetlinewidth{1.505625pt}%
\definecolor{currentstroke}{rgb}{0.631373,0.788235,0.956863}%
\pgfsetstrokecolor{currentstroke}%
\pgfsetstrokeopacity{0.200000}%
\pgfsetdash{}{0pt}%
\pgfpathmoveto{\pgfqpoint{5.185848in}{1.998699in}}%
\pgfpathlineto{\pgfqpoint{3.305661in}{2.636178in}}%
\pgfusepath{stroke}%
\end{pgfscope}%
\begin{pgfscope}%
\pgfpathrectangle{\pgfqpoint{0.481978in}{0.331635in}}{\pgfqpoint{4.960000in}{3.696000in}}%
\pgfusepath{clip}%
\pgfsetrectcap%
\pgfsetroundjoin%
\pgfsetlinewidth{1.505625pt}%
\definecolor{currentstroke}{rgb}{0.631373,0.788235,0.956863}%
\pgfsetstrokecolor{currentstroke}%
\pgfsetstrokeopacity{0.200000}%
\pgfsetdash{}{0pt}%
\pgfpathmoveto{\pgfqpoint{4.321890in}{1.779643in}}%
\pgfpathlineto{\pgfqpoint{3.305661in}{2.636178in}}%
\pgfusepath{stroke}%
\end{pgfscope}%
\begin{pgfscope}%
\pgfpathrectangle{\pgfqpoint{0.481978in}{0.331635in}}{\pgfqpoint{4.960000in}{3.696000in}}%
\pgfusepath{clip}%
\pgfsetrectcap%
\pgfsetroundjoin%
\pgfsetlinewidth{1.505625pt}%
\definecolor{currentstroke}{rgb}{0.631373,0.788235,0.956863}%
\pgfsetstrokecolor{currentstroke}%
\pgfsetstrokeopacity{0.200000}%
\pgfsetdash{}{0pt}%
\pgfpathmoveto{\pgfqpoint{3.416866in}{3.274758in}}%
\pgfpathlineto{\pgfqpoint{3.305661in}{2.636178in}}%
\pgfusepath{stroke}%
\end{pgfscope}%
\begin{pgfscope}%
\pgfpathrectangle{\pgfqpoint{0.481978in}{0.331635in}}{\pgfqpoint{4.960000in}{3.696000in}}%
\pgfusepath{clip}%
\pgfsetrectcap%
\pgfsetroundjoin%
\pgfsetlinewidth{1.505625pt}%
\definecolor{currentstroke}{rgb}{0.631373,0.788235,0.956863}%
\pgfsetstrokecolor{currentstroke}%
\pgfsetstrokeopacity{0.200000}%
\pgfsetdash{}{0pt}%
\pgfpathmoveto{\pgfqpoint{2.720729in}{2.492470in}}%
\pgfpathlineto{\pgfqpoint{3.305661in}{2.636178in}}%
\pgfusepath{stroke}%
\end{pgfscope}%
\begin{pgfscope}%
\pgfpathrectangle{\pgfqpoint{0.481978in}{0.331635in}}{\pgfqpoint{4.960000in}{3.696000in}}%
\pgfusepath{clip}%
\pgfsetrectcap%
\pgfsetroundjoin%
\pgfsetlinewidth{1.505625pt}%
\definecolor{currentstroke}{rgb}{0.631373,0.788235,0.956863}%
\pgfsetstrokecolor{currentstroke}%
\pgfsetstrokeopacity{0.200000}%
\pgfsetdash{}{0pt}%
\pgfpathmoveto{\pgfqpoint{4.178946in}{2.972973in}}%
\pgfpathlineto{\pgfqpoint{3.305661in}{2.636178in}}%
\pgfusepath{stroke}%
\end{pgfscope}%
\begin{pgfscope}%
\pgfpathrectangle{\pgfqpoint{0.481978in}{0.331635in}}{\pgfqpoint{4.960000in}{3.696000in}}%
\pgfusepath{clip}%
\pgfsetrectcap%
\pgfsetroundjoin%
\pgfsetlinewidth{1.505625pt}%
\definecolor{currentstroke}{rgb}{0.631373,0.788235,0.956863}%
\pgfsetstrokecolor{currentstroke}%
\pgfsetstrokeopacity{0.200000}%
\pgfsetdash{}{0pt}%
\pgfpathmoveto{\pgfqpoint{2.733957in}{1.941878in}}%
\pgfpathlineto{\pgfqpoint{3.305661in}{2.636178in}}%
\pgfusepath{stroke}%
\end{pgfscope}%
\begin{pgfscope}%
\pgfpathrectangle{\pgfqpoint{0.481978in}{0.331635in}}{\pgfqpoint{4.960000in}{3.696000in}}%
\pgfusepath{clip}%
\pgfsetrectcap%
\pgfsetroundjoin%
\pgfsetlinewidth{1.505625pt}%
\definecolor{currentstroke}{rgb}{0.631373,0.788235,0.956863}%
\pgfsetstrokecolor{currentstroke}%
\pgfsetstrokeopacity{0.200000}%
\pgfsetdash{}{0pt}%
\pgfpathmoveto{\pgfqpoint{2.037454in}{2.751180in}}%
\pgfpathlineto{\pgfqpoint{3.305661in}{2.636178in}}%
\pgfusepath{stroke}%
\end{pgfscope}%
\begin{pgfscope}%
\pgfpathrectangle{\pgfqpoint{0.481978in}{0.331635in}}{\pgfqpoint{4.960000in}{3.696000in}}%
\pgfusepath{clip}%
\pgfsetrectcap%
\pgfsetroundjoin%
\pgfsetlinewidth{1.505625pt}%
\definecolor{currentstroke}{rgb}{0.631373,0.788235,0.956863}%
\pgfsetstrokecolor{currentstroke}%
\pgfsetstrokeopacity{0.200000}%
\pgfsetdash{}{0pt}%
\pgfpathmoveto{\pgfqpoint{4.624636in}{1.183829in}}%
\pgfpathlineto{\pgfqpoint{3.305661in}{2.636178in}}%
\pgfusepath{stroke}%
\end{pgfscope}%
\begin{pgfscope}%
\pgfsetrectcap%
\pgfsetmiterjoin%
\pgfsetlinewidth{0.803000pt}%
\definecolor{currentstroke}{rgb}{0.000000,0.000000,0.000000}%
\pgfsetstrokecolor{currentstroke}%
\pgfsetdash{}{0pt}%
\pgfpathmoveto{\pgfqpoint{0.481978in}{0.331635in}}%
\pgfpathlineto{\pgfqpoint{0.481978in}{4.027635in}}%
\pgfusepath{stroke}%
\end{pgfscope}%
\begin{pgfscope}%
\pgfsetrectcap%
\pgfsetmiterjoin%
\pgfsetlinewidth{0.803000pt}%
\definecolor{currentstroke}{rgb}{0.000000,0.000000,0.000000}%
\pgfsetstrokecolor{currentstroke}%
\pgfsetdash{}{0pt}%
\pgfpathmoveto{\pgfqpoint{5.441978in}{0.331635in}}%
\pgfpathlineto{\pgfqpoint{5.441978in}{4.027635in}}%
\pgfusepath{stroke}%
\end{pgfscope}%
\begin{pgfscope}%
\pgfsetrectcap%
\pgfsetmiterjoin%
\pgfsetlinewidth{0.803000pt}%
\definecolor{currentstroke}{rgb}{0.000000,0.000000,0.000000}%
\pgfsetstrokecolor{currentstroke}%
\pgfsetdash{}{0pt}%
\pgfpathmoveto{\pgfqpoint{0.481978in}{0.331635in}}%
\pgfpathlineto{\pgfqpoint{5.441978in}{0.331635in}}%
\pgfusepath{stroke}%
\end{pgfscope}%
\begin{pgfscope}%
\pgfsetrectcap%
\pgfsetmiterjoin%
\pgfsetlinewidth{0.803000pt}%
\definecolor{currentstroke}{rgb}{0.000000,0.000000,0.000000}%
\pgfsetstrokecolor{currentstroke}%
\pgfsetdash{}{0pt}%
\pgfpathmoveto{\pgfqpoint{0.481978in}{4.027635in}}%
\pgfpathlineto{\pgfqpoint{5.441978in}{4.027635in}}%
\pgfusepath{stroke}%
\end{pgfscope}%
\begin{pgfscope}%
\definecolor{textcolor}{rgb}{0.000000,0.000000,0.000000}%
\pgfsetstrokecolor{textcolor}%
\pgfsetfillcolor{textcolor}%
\pgftext[x=2.961978in,y=4.110968in,,base]{\color{textcolor}\sffamily\fontsize{12.000000}{14.400000}\selectfont t-SNE for chair images (s2r3dfree\_textured\_light)}%
\end{pgfscope}%
\begin{pgfscope}%
\pgfsetbuttcap%
\pgfsetmiterjoin%
\definecolor{currentfill}{rgb}{1.000000,1.000000,1.000000}%
\pgfsetfillcolor{currentfill}%
\pgfsetfillopacity{0.800000}%
\pgfsetlinewidth{1.003750pt}%
\definecolor{currentstroke}{rgb}{0.800000,0.800000,0.800000}%
\pgfsetstrokecolor{currentstroke}%
\pgfsetstrokeopacity{0.800000}%
\pgfsetdash{}{0pt}%
\pgfpathmoveto{\pgfqpoint{0.579200in}{3.504944in}}%
\pgfpathlineto{\pgfqpoint{2.728357in}{3.504944in}}%
\pgfpathquadraticcurveto{\pgfqpoint{2.756134in}{3.504944in}}{\pgfqpoint{2.756134in}{3.532722in}}%
\pgfpathlineto{\pgfqpoint{2.756134in}{3.930413in}}%
\pgfpathquadraticcurveto{\pgfqpoint{2.756134in}{3.958191in}}{\pgfqpoint{2.728357in}{3.958191in}}%
\pgfpathlineto{\pgfqpoint{0.579200in}{3.958191in}}%
\pgfpathquadraticcurveto{\pgfqpoint{0.551422in}{3.958191in}}{\pgfqpoint{0.551422in}{3.930413in}}%
\pgfpathlineto{\pgfqpoint{0.551422in}{3.532722in}}%
\pgfpathquadraticcurveto{\pgfqpoint{0.551422in}{3.504944in}}{\pgfqpoint{0.579200in}{3.504944in}}%
\pgfpathclose%
\pgfusepath{stroke,fill}%
\end{pgfscope}%
\begin{pgfscope}%
\pgfsetbuttcap%
\pgfsetroundjoin%
\definecolor{currentfill}{rgb}{1.000000,0.705882,0.509804}%
\pgfsetfillcolor{currentfill}%
\pgfsetlinewidth{1.003750pt}%
\definecolor{currentstroke}{rgb}{1.000000,0.705882,0.509804}%
\pgfsetstrokecolor{currentstroke}%
\pgfsetdash{}{0pt}%
\pgfsys@defobject{currentmarker}{\pgfqpoint{-0.041667in}{-0.041667in}}{\pgfqpoint{0.041667in}{0.041667in}}{%
\pgfpathmoveto{\pgfqpoint{0.000000in}{-0.041667in}}%
\pgfpathcurveto{\pgfqpoint{0.011050in}{-0.041667in}}{\pgfqpoint{0.021649in}{-0.037276in}}{\pgfqpoint{0.029463in}{-0.029463in}}%
\pgfpathcurveto{\pgfqpoint{0.037276in}{-0.021649in}}{\pgfqpoint{0.041667in}{-0.011050in}}{\pgfqpoint{0.041667in}{0.000000in}}%
\pgfpathcurveto{\pgfqpoint{0.041667in}{0.011050in}}{\pgfqpoint{0.037276in}{0.021649in}}{\pgfqpoint{0.029463in}{0.029463in}}%
\pgfpathcurveto{\pgfqpoint{0.021649in}{0.037276in}}{\pgfqpoint{0.011050in}{0.041667in}}{\pgfqpoint{0.000000in}{0.041667in}}%
\pgfpathcurveto{\pgfqpoint{-0.011050in}{0.041667in}}{\pgfqpoint{-0.021649in}{0.037276in}}{\pgfqpoint{-0.029463in}{0.029463in}}%
\pgfpathcurveto{\pgfqpoint{-0.037276in}{0.021649in}}{\pgfqpoint{-0.041667in}{0.011050in}}{\pgfqpoint{-0.041667in}{0.000000in}}%
\pgfpathcurveto{\pgfqpoint{-0.041667in}{-0.011050in}}{\pgfqpoint{-0.037276in}{-0.021649in}}{\pgfqpoint{-0.029463in}{-0.029463in}}%
\pgfpathcurveto{\pgfqpoint{-0.021649in}{-0.037276in}}{\pgfqpoint{-0.011050in}{-0.041667in}}{\pgfqpoint{0.000000in}{-0.041667in}}%
\pgfpathclose%
\pgfusepath{stroke,fill}%
}%
\begin{pgfscope}%
\pgfsys@transformshift{0.745867in}{3.833570in}%
\pgfsys@useobject{currentmarker}{}%
\end{pgfscope}%
\end{pgfscope}%
\begin{pgfscope}%
\definecolor{textcolor}{rgb}{0.000000,0.000000,0.000000}%
\pgfsetstrokecolor{textcolor}%
\pgfsetfillcolor{textcolor}%
\pgftext[x=0.995867in,y=3.797112in,left,base]{\color{textcolor}\sffamily\fontsize{10.000000}{12.000000}\selectfont Pix3D}%
\end{pgfscope}%
\begin{pgfscope}%
\pgfsetbuttcap%
\pgfsetroundjoin%
\definecolor{currentfill}{rgb}{0.631373,0.788235,0.956863}%
\pgfsetfillcolor{currentfill}%
\pgfsetlinewidth{1.003750pt}%
\definecolor{currentstroke}{rgb}{0.631373,0.788235,0.956863}%
\pgfsetstrokecolor{currentstroke}%
\pgfsetdash{}{0pt}%
\pgfsys@defobject{currentmarker}{\pgfqpoint{-0.041667in}{-0.041667in}}{\pgfqpoint{0.041667in}{0.041667in}}{%
\pgfpathmoveto{\pgfqpoint{0.000000in}{-0.041667in}}%
\pgfpathcurveto{\pgfqpoint{0.011050in}{-0.041667in}}{\pgfqpoint{0.021649in}{-0.037276in}}{\pgfqpoint{0.029463in}{-0.029463in}}%
\pgfpathcurveto{\pgfqpoint{0.037276in}{-0.021649in}}{\pgfqpoint{0.041667in}{-0.011050in}}{\pgfqpoint{0.041667in}{0.000000in}}%
\pgfpathcurveto{\pgfqpoint{0.041667in}{0.011050in}}{\pgfqpoint{0.037276in}{0.021649in}}{\pgfqpoint{0.029463in}{0.029463in}}%
\pgfpathcurveto{\pgfqpoint{0.021649in}{0.037276in}}{\pgfqpoint{0.011050in}{0.041667in}}{\pgfqpoint{0.000000in}{0.041667in}}%
\pgfpathcurveto{\pgfqpoint{-0.011050in}{0.041667in}}{\pgfqpoint{-0.021649in}{0.037276in}}{\pgfqpoint{-0.029463in}{0.029463in}}%
\pgfpathcurveto{\pgfqpoint{-0.037276in}{0.021649in}}{\pgfqpoint{-0.041667in}{0.011050in}}{\pgfqpoint{-0.041667in}{0.000000in}}%
\pgfpathcurveto{\pgfqpoint{-0.041667in}{-0.011050in}}{\pgfqpoint{-0.037276in}{-0.021649in}}{\pgfqpoint{-0.029463in}{-0.029463in}}%
\pgfpathcurveto{\pgfqpoint{-0.021649in}{-0.037276in}}{\pgfqpoint{-0.011050in}{-0.041667in}}{\pgfqpoint{0.000000in}{-0.041667in}}%
\pgfpathclose%
\pgfusepath{stroke,fill}%
}%
\begin{pgfscope}%
\pgfsys@transformshift{0.745867in}{3.629713in}%
\pgfsys@useobject{currentmarker}{}%
\end{pgfscope}%
\end{pgfscope}%
\begin{pgfscope}%
\definecolor{textcolor}{rgb}{0.000000,0.000000,0.000000}%
\pgfsetstrokecolor{textcolor}%
\pgfsetfillcolor{textcolor}%
\pgftext[x=0.995867in,y=3.593255in,left,base]{\color{textcolor}\sffamily\fontsize{10.000000}{12.000000}\selectfont s2r3dfree\_textured\_light}%
\end{pgfscope}%
\end{pgfpicture}%
\makeatother%
\endgroup%
}\\
    \resizebox{0.49\linewidth}{6cm}{%% Creator: Matplotlib, PGF backend
%%
%% To include the figure in your LaTeX document, write
%%   \input{<filename>.pgf}
%%
%% Make sure the required packages are loaded in your preamble
%%   \usepackage{pgf}
%%
%% Figures using additional raster images can only be included by \input if
%% they are in the same directory as the main LaTeX file. For loading figures
%% from other directories you can use the `import` package
%%   \usepackage{import}
%%
%% and then include the figures with
%%   \import{<path to file>}{<filename>.pgf}
%%
%% Matplotlib used the following preamble
%%   \usepackage{fontspec}
%%   \setmainfont{DejaVuSerif.ttf}[Path=\detokenize{/Users/apple/opt/anaconda3/envs/kaolin/lib/python3.7/site-packages/matplotlib/mpl-data/fonts/ttf/}]
%%   \setsansfont{DejaVuSans.ttf}[Path=\detokenize{/Users/apple/opt/anaconda3/envs/kaolin/lib/python3.7/site-packages/matplotlib/mpl-data/fonts/ttf/}]
%%   \setmonofont{DejaVuSansMono.ttf}[Path=\detokenize{/Users/apple/opt/anaconda3/envs/kaolin/lib/python3.7/site-packages/matplotlib/mpl-data/fonts/ttf/}]
%%
\begingroup%
\makeatletter%
\begin{pgfpicture}%
\pgfpathrectangle{\pgfpointorigin}{\pgfqpoint{5.541978in}{4.337596in}}%
\pgfusepath{use as bounding box, clip}%
\begin{pgfscope}%
\pgfsetbuttcap%
\pgfsetmiterjoin%
\definecolor{currentfill}{rgb}{1.000000,1.000000,1.000000}%
\pgfsetfillcolor{currentfill}%
\pgfsetlinewidth{0.000000pt}%
\definecolor{currentstroke}{rgb}{1.000000,1.000000,1.000000}%
\pgfsetstrokecolor{currentstroke}%
\pgfsetdash{}{0pt}%
\pgfpathmoveto{\pgfqpoint{0.000000in}{0.000000in}}%
\pgfpathlineto{\pgfqpoint{5.541978in}{0.000000in}}%
\pgfpathlineto{\pgfqpoint{5.541978in}{4.337596in}}%
\pgfpathlineto{\pgfqpoint{0.000000in}{4.337596in}}%
\pgfpathclose%
\pgfusepath{fill}%
\end{pgfscope}%
\begin{pgfscope}%
\pgfsetbuttcap%
\pgfsetmiterjoin%
\definecolor{currentfill}{rgb}{1.000000,1.000000,1.000000}%
\pgfsetfillcolor{currentfill}%
\pgfsetlinewidth{0.000000pt}%
\definecolor{currentstroke}{rgb}{0.000000,0.000000,0.000000}%
\pgfsetstrokecolor{currentstroke}%
\pgfsetstrokeopacity{0.000000}%
\pgfsetdash{}{0pt}%
\pgfpathmoveto{\pgfqpoint{0.481978in}{0.331635in}}%
\pgfpathlineto{\pgfqpoint{5.441978in}{0.331635in}}%
\pgfpathlineto{\pgfqpoint{5.441978in}{4.027635in}}%
\pgfpathlineto{\pgfqpoint{0.481978in}{4.027635in}}%
\pgfpathclose%
\pgfusepath{fill}%
\end{pgfscope}%
\begin{pgfscope}%
\pgfpathrectangle{\pgfqpoint{0.481978in}{0.331635in}}{\pgfqpoint{4.960000in}{3.696000in}}%
\pgfusepath{clip}%
\pgfsetbuttcap%
\pgfsetroundjoin%
\definecolor{currentfill}{rgb}{1.000000,0.705882,0.509804}%
\pgfsetfillcolor{currentfill}%
\pgfsetlinewidth{0.481800pt}%
\definecolor{currentstroke}{rgb}{1.000000,1.000000,1.000000}%
\pgfsetstrokecolor{currentstroke}%
\pgfsetdash{}{0pt}%
\pgfpathmoveto{\pgfqpoint{4.627391in}{2.794143in}}%
\pgfpathcurveto{\pgfqpoint{4.638441in}{2.794143in}}{\pgfqpoint{4.649040in}{2.798533in}}{\pgfqpoint{4.656854in}{2.806347in}}%
\pgfpathcurveto{\pgfqpoint{4.664668in}{2.814160in}}{\pgfqpoint{4.669058in}{2.824759in}}{\pgfqpoint{4.669058in}{2.835809in}}%
\pgfpathcurveto{\pgfqpoint{4.669058in}{2.846859in}}{\pgfqpoint{4.664668in}{2.857459in}}{\pgfqpoint{4.656854in}{2.865272in}}%
\pgfpathcurveto{\pgfqpoint{4.649040in}{2.873086in}}{\pgfqpoint{4.638441in}{2.877476in}}{\pgfqpoint{4.627391in}{2.877476in}}%
\pgfpathcurveto{\pgfqpoint{4.616341in}{2.877476in}}{\pgfqpoint{4.605742in}{2.873086in}}{\pgfqpoint{4.597928in}{2.865272in}}%
\pgfpathcurveto{\pgfqpoint{4.590115in}{2.857459in}}{\pgfqpoint{4.585725in}{2.846859in}}{\pgfqpoint{4.585725in}{2.835809in}}%
\pgfpathcurveto{\pgfqpoint{4.585725in}{2.824759in}}{\pgfqpoint{4.590115in}{2.814160in}}{\pgfqpoint{4.597928in}{2.806347in}}%
\pgfpathcurveto{\pgfqpoint{4.605742in}{2.798533in}}{\pgfqpoint{4.616341in}{2.794143in}}{\pgfqpoint{4.627391in}{2.794143in}}%
\pgfpathclose%
\pgfusepath{stroke,fill}%
\end{pgfscope}%
\begin{pgfscope}%
\pgfpathrectangle{\pgfqpoint{0.481978in}{0.331635in}}{\pgfqpoint{4.960000in}{3.696000in}}%
\pgfusepath{clip}%
\pgfsetbuttcap%
\pgfsetroundjoin%
\definecolor{currentfill}{rgb}{1.000000,0.705882,0.509804}%
\pgfsetfillcolor{currentfill}%
\pgfsetlinewidth{0.481800pt}%
\definecolor{currentstroke}{rgb}{1.000000,1.000000,1.000000}%
\pgfsetstrokecolor{currentstroke}%
\pgfsetdash{}{0pt}%
\pgfpathmoveto{\pgfqpoint{2.475618in}{0.557840in}}%
\pgfpathcurveto{\pgfqpoint{2.486668in}{0.557840in}}{\pgfqpoint{2.497267in}{0.562231in}}{\pgfqpoint{2.505080in}{0.570044in}}%
\pgfpathcurveto{\pgfqpoint{2.512894in}{0.577858in}}{\pgfqpoint{2.517284in}{0.588457in}}{\pgfqpoint{2.517284in}{0.599507in}}%
\pgfpathcurveto{\pgfqpoint{2.517284in}{0.610557in}}{\pgfqpoint{2.512894in}{0.621156in}}{\pgfqpoint{2.505080in}{0.628970in}}%
\pgfpathcurveto{\pgfqpoint{2.497267in}{0.636783in}}{\pgfqpoint{2.486668in}{0.641174in}}{\pgfqpoint{2.475618in}{0.641174in}}%
\pgfpathcurveto{\pgfqpoint{2.464567in}{0.641174in}}{\pgfqpoint{2.453968in}{0.636783in}}{\pgfqpoint{2.446155in}{0.628970in}}%
\pgfpathcurveto{\pgfqpoint{2.438341in}{0.621156in}}{\pgfqpoint{2.433951in}{0.610557in}}{\pgfqpoint{2.433951in}{0.599507in}}%
\pgfpathcurveto{\pgfqpoint{2.433951in}{0.588457in}}{\pgfqpoint{2.438341in}{0.577858in}}{\pgfqpoint{2.446155in}{0.570044in}}%
\pgfpathcurveto{\pgfqpoint{2.453968in}{0.562231in}}{\pgfqpoint{2.464567in}{0.557840in}}{\pgfqpoint{2.475618in}{0.557840in}}%
\pgfpathclose%
\pgfusepath{stroke,fill}%
\end{pgfscope}%
\begin{pgfscope}%
\pgfpathrectangle{\pgfqpoint{0.481978in}{0.331635in}}{\pgfqpoint{4.960000in}{3.696000in}}%
\pgfusepath{clip}%
\pgfsetbuttcap%
\pgfsetroundjoin%
\definecolor{currentfill}{rgb}{1.000000,0.705882,0.509804}%
\pgfsetfillcolor{currentfill}%
\pgfsetlinewidth{0.481800pt}%
\definecolor{currentstroke}{rgb}{1.000000,1.000000,1.000000}%
\pgfsetstrokecolor{currentstroke}%
\pgfsetdash{}{0pt}%
\pgfpathmoveto{\pgfqpoint{2.464482in}{0.563520in}}%
\pgfpathcurveto{\pgfqpoint{2.475532in}{0.563520in}}{\pgfqpoint{2.486131in}{0.567910in}}{\pgfqpoint{2.493945in}{0.575724in}}%
\pgfpathcurveto{\pgfqpoint{2.501758in}{0.583537in}}{\pgfqpoint{2.506148in}{0.594136in}}{\pgfqpoint{2.506148in}{0.605186in}}%
\pgfpathcurveto{\pgfqpoint{2.506148in}{0.616237in}}{\pgfqpoint{2.501758in}{0.626836in}}{\pgfqpoint{2.493945in}{0.634649in}}%
\pgfpathcurveto{\pgfqpoint{2.486131in}{0.642463in}}{\pgfqpoint{2.475532in}{0.646853in}}{\pgfqpoint{2.464482in}{0.646853in}}%
\pgfpathcurveto{\pgfqpoint{2.453432in}{0.646853in}}{\pgfqpoint{2.442833in}{0.642463in}}{\pgfqpoint{2.435019in}{0.634649in}}%
\pgfpathcurveto{\pgfqpoint{2.427205in}{0.626836in}}{\pgfqpoint{2.422815in}{0.616237in}}{\pgfqpoint{2.422815in}{0.605186in}}%
\pgfpathcurveto{\pgfqpoint{2.422815in}{0.594136in}}{\pgfqpoint{2.427205in}{0.583537in}}{\pgfqpoint{2.435019in}{0.575724in}}%
\pgfpathcurveto{\pgfqpoint{2.442833in}{0.567910in}}{\pgfqpoint{2.453432in}{0.563520in}}{\pgfqpoint{2.464482in}{0.563520in}}%
\pgfpathclose%
\pgfusepath{stroke,fill}%
\end{pgfscope}%
\begin{pgfscope}%
\pgfpathrectangle{\pgfqpoint{0.481978in}{0.331635in}}{\pgfqpoint{4.960000in}{3.696000in}}%
\pgfusepath{clip}%
\pgfsetbuttcap%
\pgfsetroundjoin%
\definecolor{currentfill}{rgb}{1.000000,0.705882,0.509804}%
\pgfsetfillcolor{currentfill}%
\pgfsetlinewidth{0.481800pt}%
\definecolor{currentstroke}{rgb}{1.000000,1.000000,1.000000}%
\pgfsetstrokecolor{currentstroke}%
\pgfsetdash{}{0pt}%
\pgfpathmoveto{\pgfqpoint{1.079617in}{2.354450in}}%
\pgfpathcurveto{\pgfqpoint{1.090667in}{2.354450in}}{\pgfqpoint{1.101266in}{2.358840in}}{\pgfqpoint{1.109080in}{2.366654in}}%
\pgfpathcurveto{\pgfqpoint{1.116893in}{2.374468in}}{\pgfqpoint{1.121284in}{2.385067in}}{\pgfqpoint{1.121284in}{2.396117in}}%
\pgfpathcurveto{\pgfqpoint{1.121284in}{2.407167in}}{\pgfqpoint{1.116893in}{2.417766in}}{\pgfqpoint{1.109080in}{2.425580in}}%
\pgfpathcurveto{\pgfqpoint{1.101266in}{2.433393in}}{\pgfqpoint{1.090667in}{2.437784in}}{\pgfqpoint{1.079617in}{2.437784in}}%
\pgfpathcurveto{\pgfqpoint{1.068567in}{2.437784in}}{\pgfqpoint{1.057968in}{2.433393in}}{\pgfqpoint{1.050154in}{2.425580in}}%
\pgfpathcurveto{\pgfqpoint{1.042341in}{2.417766in}}{\pgfqpoint{1.037950in}{2.407167in}}{\pgfqpoint{1.037950in}{2.396117in}}%
\pgfpathcurveto{\pgfqpoint{1.037950in}{2.385067in}}{\pgfqpoint{1.042341in}{2.374468in}}{\pgfqpoint{1.050154in}{2.366654in}}%
\pgfpathcurveto{\pgfqpoint{1.057968in}{2.358840in}}{\pgfqpoint{1.068567in}{2.354450in}}{\pgfqpoint{1.079617in}{2.354450in}}%
\pgfpathclose%
\pgfusepath{stroke,fill}%
\end{pgfscope}%
\begin{pgfscope}%
\pgfpathrectangle{\pgfqpoint{0.481978in}{0.331635in}}{\pgfqpoint{4.960000in}{3.696000in}}%
\pgfusepath{clip}%
\pgfsetbuttcap%
\pgfsetroundjoin%
\definecolor{currentfill}{rgb}{1.000000,0.705882,0.509804}%
\pgfsetfillcolor{currentfill}%
\pgfsetlinewidth{0.481800pt}%
\definecolor{currentstroke}{rgb}{1.000000,1.000000,1.000000}%
\pgfsetstrokecolor{currentstroke}%
\pgfsetdash{}{0pt}%
\pgfpathmoveto{\pgfqpoint{3.029515in}{1.823140in}}%
\pgfpathcurveto{\pgfqpoint{3.040565in}{1.823140in}}{\pgfqpoint{3.051164in}{1.827530in}}{\pgfqpoint{3.058978in}{1.835344in}}%
\pgfpathcurveto{\pgfqpoint{3.066791in}{1.843157in}}{\pgfqpoint{3.071181in}{1.853756in}}{\pgfqpoint{3.071181in}{1.864807in}}%
\pgfpathcurveto{\pgfqpoint{3.071181in}{1.875857in}}{\pgfqpoint{3.066791in}{1.886456in}}{\pgfqpoint{3.058978in}{1.894269in}}%
\pgfpathcurveto{\pgfqpoint{3.051164in}{1.902083in}}{\pgfqpoint{3.040565in}{1.906473in}}{\pgfqpoint{3.029515in}{1.906473in}}%
\pgfpathcurveto{\pgfqpoint{3.018465in}{1.906473in}}{\pgfqpoint{3.007866in}{1.902083in}}{\pgfqpoint{3.000052in}{1.894269in}}%
\pgfpathcurveto{\pgfqpoint{2.992238in}{1.886456in}}{\pgfqpoint{2.987848in}{1.875857in}}{\pgfqpoint{2.987848in}{1.864807in}}%
\pgfpathcurveto{\pgfqpoint{2.987848in}{1.853756in}}{\pgfqpoint{2.992238in}{1.843157in}}{\pgfqpoint{3.000052in}{1.835344in}}%
\pgfpathcurveto{\pgfqpoint{3.007866in}{1.827530in}}{\pgfqpoint{3.018465in}{1.823140in}}{\pgfqpoint{3.029515in}{1.823140in}}%
\pgfpathclose%
\pgfusepath{stroke,fill}%
\end{pgfscope}%
\begin{pgfscope}%
\pgfpathrectangle{\pgfqpoint{0.481978in}{0.331635in}}{\pgfqpoint{4.960000in}{3.696000in}}%
\pgfusepath{clip}%
\pgfsetbuttcap%
\pgfsetroundjoin%
\definecolor{currentfill}{rgb}{1.000000,0.705882,0.509804}%
\pgfsetfillcolor{currentfill}%
\pgfsetlinewidth{0.481800pt}%
\definecolor{currentstroke}{rgb}{1.000000,1.000000,1.000000}%
\pgfsetstrokecolor{currentstroke}%
\pgfsetdash{}{0pt}%
\pgfpathmoveto{\pgfqpoint{2.736847in}{1.842423in}}%
\pgfpathcurveto{\pgfqpoint{2.747897in}{1.842423in}}{\pgfqpoint{2.758496in}{1.846813in}}{\pgfqpoint{2.766310in}{1.854627in}}%
\pgfpathcurveto{\pgfqpoint{2.774123in}{1.862440in}}{\pgfqpoint{2.778514in}{1.873039in}}{\pgfqpoint{2.778514in}{1.884090in}}%
\pgfpathcurveto{\pgfqpoint{2.778514in}{1.895140in}}{\pgfqpoint{2.774123in}{1.905739in}}{\pgfqpoint{2.766310in}{1.913552in}}%
\pgfpathcurveto{\pgfqpoint{2.758496in}{1.921366in}}{\pgfqpoint{2.747897in}{1.925756in}}{\pgfqpoint{2.736847in}{1.925756in}}%
\pgfpathcurveto{\pgfqpoint{2.725797in}{1.925756in}}{\pgfqpoint{2.715198in}{1.921366in}}{\pgfqpoint{2.707384in}{1.913552in}}%
\pgfpathcurveto{\pgfqpoint{2.699571in}{1.905739in}}{\pgfqpoint{2.695180in}{1.895140in}}{\pgfqpoint{2.695180in}{1.884090in}}%
\pgfpathcurveto{\pgfqpoint{2.695180in}{1.873039in}}{\pgfqpoint{2.699571in}{1.862440in}}{\pgfqpoint{2.707384in}{1.854627in}}%
\pgfpathcurveto{\pgfqpoint{2.715198in}{1.846813in}}{\pgfqpoint{2.725797in}{1.842423in}}{\pgfqpoint{2.736847in}{1.842423in}}%
\pgfpathclose%
\pgfusepath{stroke,fill}%
\end{pgfscope}%
\begin{pgfscope}%
\pgfpathrectangle{\pgfqpoint{0.481978in}{0.331635in}}{\pgfqpoint{4.960000in}{3.696000in}}%
\pgfusepath{clip}%
\pgfsetbuttcap%
\pgfsetroundjoin%
\definecolor{currentfill}{rgb}{1.000000,0.705882,0.509804}%
\pgfsetfillcolor{currentfill}%
\pgfsetlinewidth{0.481800pt}%
\definecolor{currentstroke}{rgb}{1.000000,1.000000,1.000000}%
\pgfsetstrokecolor{currentstroke}%
\pgfsetdash{}{0pt}%
\pgfpathmoveto{\pgfqpoint{2.064108in}{1.525433in}}%
\pgfpathcurveto{\pgfqpoint{2.075158in}{1.525433in}}{\pgfqpoint{2.085757in}{1.529824in}}{\pgfqpoint{2.093571in}{1.537637in}}%
\pgfpathcurveto{\pgfqpoint{2.101385in}{1.545451in}}{\pgfqpoint{2.105775in}{1.556050in}}{\pgfqpoint{2.105775in}{1.567100in}}%
\pgfpathcurveto{\pgfqpoint{2.105775in}{1.578150in}}{\pgfqpoint{2.101385in}{1.588749in}}{\pgfqpoint{2.093571in}{1.596563in}}%
\pgfpathcurveto{\pgfqpoint{2.085757in}{1.604377in}}{\pgfqpoint{2.075158in}{1.608767in}}{\pgfqpoint{2.064108in}{1.608767in}}%
\pgfpathcurveto{\pgfqpoint{2.053058in}{1.608767in}}{\pgfqpoint{2.042459in}{1.604377in}}{\pgfqpoint{2.034645in}{1.596563in}}%
\pgfpathcurveto{\pgfqpoint{2.026832in}{1.588749in}}{\pgfqpoint{2.022441in}{1.578150in}}{\pgfqpoint{2.022441in}{1.567100in}}%
\pgfpathcurveto{\pgfqpoint{2.022441in}{1.556050in}}{\pgfqpoint{2.026832in}{1.545451in}}{\pgfqpoint{2.034645in}{1.537637in}}%
\pgfpathcurveto{\pgfqpoint{2.042459in}{1.529824in}}{\pgfqpoint{2.053058in}{1.525433in}}{\pgfqpoint{2.064108in}{1.525433in}}%
\pgfpathclose%
\pgfusepath{stroke,fill}%
\end{pgfscope}%
\begin{pgfscope}%
\pgfpathrectangle{\pgfqpoint{0.481978in}{0.331635in}}{\pgfqpoint{4.960000in}{3.696000in}}%
\pgfusepath{clip}%
\pgfsetbuttcap%
\pgfsetroundjoin%
\definecolor{currentfill}{rgb}{1.000000,0.705882,0.509804}%
\pgfsetfillcolor{currentfill}%
\pgfsetlinewidth{0.481800pt}%
\definecolor{currentstroke}{rgb}{1.000000,1.000000,1.000000}%
\pgfsetstrokecolor{currentstroke}%
\pgfsetdash{}{0pt}%
\pgfpathmoveto{\pgfqpoint{2.838937in}{3.683595in}}%
\pgfpathcurveto{\pgfqpoint{2.849987in}{3.683595in}}{\pgfqpoint{2.860586in}{3.687986in}}{\pgfqpoint{2.868400in}{3.695799in}}%
\pgfpathcurveto{\pgfqpoint{2.876213in}{3.703613in}}{\pgfqpoint{2.880603in}{3.714212in}}{\pgfqpoint{2.880603in}{3.725262in}}%
\pgfpathcurveto{\pgfqpoint{2.880603in}{3.736312in}}{\pgfqpoint{2.876213in}{3.746911in}}{\pgfqpoint{2.868400in}{3.754725in}}%
\pgfpathcurveto{\pgfqpoint{2.860586in}{3.762538in}}{\pgfqpoint{2.849987in}{3.766929in}}{\pgfqpoint{2.838937in}{3.766929in}}%
\pgfpathcurveto{\pgfqpoint{2.827887in}{3.766929in}}{\pgfqpoint{2.817288in}{3.762538in}}{\pgfqpoint{2.809474in}{3.754725in}}%
\pgfpathcurveto{\pgfqpoint{2.801660in}{3.746911in}}{\pgfqpoint{2.797270in}{3.736312in}}{\pgfqpoint{2.797270in}{3.725262in}}%
\pgfpathcurveto{\pgfqpoint{2.797270in}{3.714212in}}{\pgfqpoint{2.801660in}{3.703613in}}{\pgfqpoint{2.809474in}{3.695799in}}%
\pgfpathcurveto{\pgfqpoint{2.817288in}{3.687986in}}{\pgfqpoint{2.827887in}{3.683595in}}{\pgfqpoint{2.838937in}{3.683595in}}%
\pgfpathclose%
\pgfusepath{stroke,fill}%
\end{pgfscope}%
\begin{pgfscope}%
\pgfpathrectangle{\pgfqpoint{0.481978in}{0.331635in}}{\pgfqpoint{4.960000in}{3.696000in}}%
\pgfusepath{clip}%
\pgfsetbuttcap%
\pgfsetroundjoin%
\definecolor{currentfill}{rgb}{1.000000,0.705882,0.509804}%
\pgfsetfillcolor{currentfill}%
\pgfsetlinewidth{0.481800pt}%
\definecolor{currentstroke}{rgb}{1.000000,1.000000,1.000000}%
\pgfsetstrokecolor{currentstroke}%
\pgfsetdash{}{0pt}%
\pgfpathmoveto{\pgfqpoint{2.585551in}{1.677600in}}%
\pgfpathcurveto{\pgfqpoint{2.596602in}{1.677600in}}{\pgfqpoint{2.607201in}{1.681990in}}{\pgfqpoint{2.615014in}{1.689804in}}%
\pgfpathcurveto{\pgfqpoint{2.622828in}{1.697618in}}{\pgfqpoint{2.627218in}{1.708217in}}{\pgfqpoint{2.627218in}{1.719267in}}%
\pgfpathcurveto{\pgfqpoint{2.627218in}{1.730317in}}{\pgfqpoint{2.622828in}{1.740916in}}{\pgfqpoint{2.615014in}{1.748730in}}%
\pgfpathcurveto{\pgfqpoint{2.607201in}{1.756543in}}{\pgfqpoint{2.596602in}{1.760933in}}{\pgfqpoint{2.585551in}{1.760933in}}%
\pgfpathcurveto{\pgfqpoint{2.574501in}{1.760933in}}{\pgfqpoint{2.563902in}{1.756543in}}{\pgfqpoint{2.556089in}{1.748730in}}%
\pgfpathcurveto{\pgfqpoint{2.548275in}{1.740916in}}{\pgfqpoint{2.543885in}{1.730317in}}{\pgfqpoint{2.543885in}{1.719267in}}%
\pgfpathcurveto{\pgfqpoint{2.543885in}{1.708217in}}{\pgfqpoint{2.548275in}{1.697618in}}{\pgfqpoint{2.556089in}{1.689804in}}%
\pgfpathcurveto{\pgfqpoint{2.563902in}{1.681990in}}{\pgfqpoint{2.574501in}{1.677600in}}{\pgfqpoint{2.585551in}{1.677600in}}%
\pgfpathclose%
\pgfusepath{stroke,fill}%
\end{pgfscope}%
\begin{pgfscope}%
\pgfpathrectangle{\pgfqpoint{0.481978in}{0.331635in}}{\pgfqpoint{4.960000in}{3.696000in}}%
\pgfusepath{clip}%
\pgfsetbuttcap%
\pgfsetroundjoin%
\definecolor{currentfill}{rgb}{1.000000,0.705882,0.509804}%
\pgfsetfillcolor{currentfill}%
\pgfsetlinewidth{0.481800pt}%
\definecolor{currentstroke}{rgb}{1.000000,1.000000,1.000000}%
\pgfsetstrokecolor{currentstroke}%
\pgfsetdash{}{0pt}%
\pgfpathmoveto{\pgfqpoint{1.240353in}{1.737647in}}%
\pgfpathcurveto{\pgfqpoint{1.251403in}{1.737647in}}{\pgfqpoint{1.262002in}{1.742037in}}{\pgfqpoint{1.269815in}{1.749850in}}%
\pgfpathcurveto{\pgfqpoint{1.277629in}{1.757664in}}{\pgfqpoint{1.282019in}{1.768263in}}{\pgfqpoint{1.282019in}{1.779313in}}%
\pgfpathcurveto{\pgfqpoint{1.282019in}{1.790363in}}{\pgfqpoint{1.277629in}{1.800962in}}{\pgfqpoint{1.269815in}{1.808776in}}%
\pgfpathcurveto{\pgfqpoint{1.262002in}{1.816590in}}{\pgfqpoint{1.251403in}{1.820980in}}{\pgfqpoint{1.240353in}{1.820980in}}%
\pgfpathcurveto{\pgfqpoint{1.229302in}{1.820980in}}{\pgfqpoint{1.218703in}{1.816590in}}{\pgfqpoint{1.210890in}{1.808776in}}%
\pgfpathcurveto{\pgfqpoint{1.203076in}{1.800962in}}{\pgfqpoint{1.198686in}{1.790363in}}{\pgfqpoint{1.198686in}{1.779313in}}%
\pgfpathcurveto{\pgfqpoint{1.198686in}{1.768263in}}{\pgfqpoint{1.203076in}{1.757664in}}{\pgfqpoint{1.210890in}{1.749850in}}%
\pgfpathcurveto{\pgfqpoint{1.218703in}{1.742037in}}{\pgfqpoint{1.229302in}{1.737647in}}{\pgfqpoint{1.240353in}{1.737647in}}%
\pgfpathclose%
\pgfusepath{stroke,fill}%
\end{pgfscope}%
\begin{pgfscope}%
\pgfpathrectangle{\pgfqpoint{0.481978in}{0.331635in}}{\pgfqpoint{4.960000in}{3.696000in}}%
\pgfusepath{clip}%
\pgfsetbuttcap%
\pgfsetroundjoin%
\definecolor{currentfill}{rgb}{1.000000,0.705882,0.509804}%
\pgfsetfillcolor{currentfill}%
\pgfsetlinewidth{0.481800pt}%
\definecolor{currentstroke}{rgb}{1.000000,1.000000,1.000000}%
\pgfsetstrokecolor{currentstroke}%
\pgfsetdash{}{0pt}%
\pgfpathmoveto{\pgfqpoint{3.982955in}{0.948500in}}%
\pgfpathcurveto{\pgfqpoint{3.994006in}{0.948500in}}{\pgfqpoint{4.004605in}{0.952890in}}{\pgfqpoint{4.012418in}{0.960704in}}%
\pgfpathcurveto{\pgfqpoint{4.020232in}{0.968517in}}{\pgfqpoint{4.024622in}{0.979117in}}{\pgfqpoint{4.024622in}{0.990167in}}%
\pgfpathcurveto{\pgfqpoint{4.024622in}{1.001217in}}{\pgfqpoint{4.020232in}{1.011816in}}{\pgfqpoint{4.012418in}{1.019629in}}%
\pgfpathcurveto{\pgfqpoint{4.004605in}{1.027443in}}{\pgfqpoint{3.994006in}{1.031833in}}{\pgfqpoint{3.982955in}{1.031833in}}%
\pgfpathcurveto{\pgfqpoint{3.971905in}{1.031833in}}{\pgfqpoint{3.961306in}{1.027443in}}{\pgfqpoint{3.953493in}{1.019629in}}%
\pgfpathcurveto{\pgfqpoint{3.945679in}{1.011816in}}{\pgfqpoint{3.941289in}{1.001217in}}{\pgfqpoint{3.941289in}{0.990167in}}%
\pgfpathcurveto{\pgfqpoint{3.941289in}{0.979117in}}{\pgfqpoint{3.945679in}{0.968517in}}{\pgfqpoint{3.953493in}{0.960704in}}%
\pgfpathcurveto{\pgfqpoint{3.961306in}{0.952890in}}{\pgfqpoint{3.971905in}{0.948500in}}{\pgfqpoint{3.982955in}{0.948500in}}%
\pgfpathclose%
\pgfusepath{stroke,fill}%
\end{pgfscope}%
\begin{pgfscope}%
\pgfpathrectangle{\pgfqpoint{0.481978in}{0.331635in}}{\pgfqpoint{4.960000in}{3.696000in}}%
\pgfusepath{clip}%
\pgfsetbuttcap%
\pgfsetroundjoin%
\definecolor{currentfill}{rgb}{1.000000,0.705882,0.509804}%
\pgfsetfillcolor{currentfill}%
\pgfsetlinewidth{0.481800pt}%
\definecolor{currentstroke}{rgb}{1.000000,1.000000,1.000000}%
\pgfsetstrokecolor{currentstroke}%
\pgfsetdash{}{0pt}%
\pgfpathmoveto{\pgfqpoint{3.988249in}{2.905149in}}%
\pgfpathcurveto{\pgfqpoint{3.999300in}{2.905149in}}{\pgfqpoint{4.009899in}{2.909540in}}{\pgfqpoint{4.017712in}{2.917353in}}%
\pgfpathcurveto{\pgfqpoint{4.025526in}{2.925167in}}{\pgfqpoint{4.029916in}{2.935766in}}{\pgfqpoint{4.029916in}{2.946816in}}%
\pgfpathcurveto{\pgfqpoint{4.029916in}{2.957866in}}{\pgfqpoint{4.025526in}{2.968465in}}{\pgfqpoint{4.017712in}{2.976279in}}%
\pgfpathcurveto{\pgfqpoint{4.009899in}{2.984093in}}{\pgfqpoint{3.999300in}{2.988483in}}{\pgfqpoint{3.988249in}{2.988483in}}%
\pgfpathcurveto{\pgfqpoint{3.977199in}{2.988483in}}{\pgfqpoint{3.966600in}{2.984093in}}{\pgfqpoint{3.958787in}{2.976279in}}%
\pgfpathcurveto{\pgfqpoint{3.950973in}{2.968465in}}{\pgfqpoint{3.946583in}{2.957866in}}{\pgfqpoint{3.946583in}{2.946816in}}%
\pgfpathcurveto{\pgfqpoint{3.946583in}{2.935766in}}{\pgfqpoint{3.950973in}{2.925167in}}{\pgfqpoint{3.958787in}{2.917353in}}%
\pgfpathcurveto{\pgfqpoint{3.966600in}{2.909540in}}{\pgfqpoint{3.977199in}{2.905149in}}{\pgfqpoint{3.988249in}{2.905149in}}%
\pgfpathclose%
\pgfusepath{stroke,fill}%
\end{pgfscope}%
\begin{pgfscope}%
\pgfpathrectangle{\pgfqpoint{0.481978in}{0.331635in}}{\pgfqpoint{4.960000in}{3.696000in}}%
\pgfusepath{clip}%
\pgfsetbuttcap%
\pgfsetroundjoin%
\definecolor{currentfill}{rgb}{1.000000,0.705882,0.509804}%
\pgfsetfillcolor{currentfill}%
\pgfsetlinewidth{0.481800pt}%
\definecolor{currentstroke}{rgb}{1.000000,1.000000,1.000000}%
\pgfsetstrokecolor{currentstroke}%
\pgfsetdash{}{0pt}%
\pgfpathmoveto{\pgfqpoint{3.185681in}{2.947875in}}%
\pgfpathcurveto{\pgfqpoint{3.196731in}{2.947875in}}{\pgfqpoint{3.207330in}{2.952266in}}{\pgfqpoint{3.215144in}{2.960079in}}%
\pgfpathcurveto{\pgfqpoint{3.222958in}{2.967893in}}{\pgfqpoint{3.227348in}{2.978492in}}{\pgfqpoint{3.227348in}{2.989542in}}%
\pgfpathcurveto{\pgfqpoint{3.227348in}{3.000592in}}{\pgfqpoint{3.222958in}{3.011191in}}{\pgfqpoint{3.215144in}{3.019005in}}%
\pgfpathcurveto{\pgfqpoint{3.207330in}{3.026818in}}{\pgfqpoint{3.196731in}{3.031209in}}{\pgfqpoint{3.185681in}{3.031209in}}%
\pgfpathcurveto{\pgfqpoint{3.174631in}{3.031209in}}{\pgfqpoint{3.164032in}{3.026818in}}{\pgfqpoint{3.156219in}{3.019005in}}%
\pgfpathcurveto{\pgfqpoint{3.148405in}{3.011191in}}{\pgfqpoint{3.144015in}{3.000592in}}{\pgfqpoint{3.144015in}{2.989542in}}%
\pgfpathcurveto{\pgfqpoint{3.144015in}{2.978492in}}{\pgfqpoint{3.148405in}{2.967893in}}{\pgfqpoint{3.156219in}{2.960079in}}%
\pgfpathcurveto{\pgfqpoint{3.164032in}{2.952266in}}{\pgfqpoint{3.174631in}{2.947875in}}{\pgfqpoint{3.185681in}{2.947875in}}%
\pgfpathclose%
\pgfusepath{stroke,fill}%
\end{pgfscope}%
\begin{pgfscope}%
\pgfpathrectangle{\pgfqpoint{0.481978in}{0.331635in}}{\pgfqpoint{4.960000in}{3.696000in}}%
\pgfusepath{clip}%
\pgfsetbuttcap%
\pgfsetroundjoin%
\definecolor{currentfill}{rgb}{1.000000,0.705882,0.509804}%
\pgfsetfillcolor{currentfill}%
\pgfsetlinewidth{0.481800pt}%
\definecolor{currentstroke}{rgb}{1.000000,1.000000,1.000000}%
\pgfsetstrokecolor{currentstroke}%
\pgfsetdash{}{0pt}%
\pgfpathmoveto{\pgfqpoint{0.996356in}{2.171413in}}%
\pgfpathcurveto{\pgfqpoint{1.007406in}{2.171413in}}{\pgfqpoint{1.018005in}{2.175803in}}{\pgfqpoint{1.025819in}{2.183617in}}%
\pgfpathcurveto{\pgfqpoint{1.033632in}{2.191430in}}{\pgfqpoint{1.038023in}{2.202029in}}{\pgfqpoint{1.038023in}{2.213080in}}%
\pgfpathcurveto{\pgfqpoint{1.038023in}{2.224130in}}{\pgfqpoint{1.033632in}{2.234729in}}{\pgfqpoint{1.025819in}{2.242542in}}%
\pgfpathcurveto{\pgfqpoint{1.018005in}{2.250356in}}{\pgfqpoint{1.007406in}{2.254746in}}{\pgfqpoint{0.996356in}{2.254746in}}%
\pgfpathcurveto{\pgfqpoint{0.985306in}{2.254746in}}{\pgfqpoint{0.974707in}{2.250356in}}{\pgfqpoint{0.966893in}{2.242542in}}%
\pgfpathcurveto{\pgfqpoint{0.959080in}{2.234729in}}{\pgfqpoint{0.954689in}{2.224130in}}{\pgfqpoint{0.954689in}{2.213080in}}%
\pgfpathcurveto{\pgfqpoint{0.954689in}{2.202029in}}{\pgfqpoint{0.959080in}{2.191430in}}{\pgfqpoint{0.966893in}{2.183617in}}%
\pgfpathcurveto{\pgfqpoint{0.974707in}{2.175803in}}{\pgfqpoint{0.985306in}{2.171413in}}{\pgfqpoint{0.996356in}{2.171413in}}%
\pgfpathclose%
\pgfusepath{stroke,fill}%
\end{pgfscope}%
\begin{pgfscope}%
\pgfpathrectangle{\pgfqpoint{0.481978in}{0.331635in}}{\pgfqpoint{4.960000in}{3.696000in}}%
\pgfusepath{clip}%
\pgfsetbuttcap%
\pgfsetroundjoin%
\definecolor{currentfill}{rgb}{1.000000,0.705882,0.509804}%
\pgfsetfillcolor{currentfill}%
\pgfsetlinewidth{0.481800pt}%
\definecolor{currentstroke}{rgb}{1.000000,1.000000,1.000000}%
\pgfsetstrokecolor{currentstroke}%
\pgfsetdash{}{0pt}%
\pgfpathmoveto{\pgfqpoint{1.152793in}{2.589146in}}%
\pgfpathcurveto{\pgfqpoint{1.163843in}{2.589146in}}{\pgfqpoint{1.174442in}{2.593536in}}{\pgfqpoint{1.182256in}{2.601350in}}%
\pgfpathcurveto{\pgfqpoint{1.190069in}{2.609164in}}{\pgfqpoint{1.194460in}{2.619763in}}{\pgfqpoint{1.194460in}{2.630813in}}%
\pgfpathcurveto{\pgfqpoint{1.194460in}{2.641863in}}{\pgfqpoint{1.190069in}{2.652462in}}{\pgfqpoint{1.182256in}{2.660276in}}%
\pgfpathcurveto{\pgfqpoint{1.174442in}{2.668089in}}{\pgfqpoint{1.163843in}{2.672479in}}{\pgfqpoint{1.152793in}{2.672479in}}%
\pgfpathcurveto{\pgfqpoint{1.141743in}{2.672479in}}{\pgfqpoint{1.131144in}{2.668089in}}{\pgfqpoint{1.123330in}{2.660276in}}%
\pgfpathcurveto{\pgfqpoint{1.115517in}{2.652462in}}{\pgfqpoint{1.111126in}{2.641863in}}{\pgfqpoint{1.111126in}{2.630813in}}%
\pgfpathcurveto{\pgfqpoint{1.111126in}{2.619763in}}{\pgfqpoint{1.115517in}{2.609164in}}{\pgfqpoint{1.123330in}{2.601350in}}%
\pgfpathcurveto{\pgfqpoint{1.131144in}{2.593536in}}{\pgfqpoint{1.141743in}{2.589146in}}{\pgfqpoint{1.152793in}{2.589146in}}%
\pgfpathclose%
\pgfusepath{stroke,fill}%
\end{pgfscope}%
\begin{pgfscope}%
\pgfpathrectangle{\pgfqpoint{0.481978in}{0.331635in}}{\pgfqpoint{4.960000in}{3.696000in}}%
\pgfusepath{clip}%
\pgfsetbuttcap%
\pgfsetroundjoin%
\definecolor{currentfill}{rgb}{1.000000,0.705882,0.509804}%
\pgfsetfillcolor{currentfill}%
\pgfsetlinewidth{0.481800pt}%
\definecolor{currentstroke}{rgb}{1.000000,1.000000,1.000000}%
\pgfsetstrokecolor{currentstroke}%
\pgfsetdash{}{0pt}%
\pgfpathmoveto{\pgfqpoint{3.244734in}{3.368506in}}%
\pgfpathcurveto{\pgfqpoint{3.255784in}{3.368506in}}{\pgfqpoint{3.266383in}{3.372897in}}{\pgfqpoint{3.274197in}{3.380710in}}%
\pgfpathcurveto{\pgfqpoint{3.282011in}{3.388524in}}{\pgfqpoint{3.286401in}{3.399123in}}{\pgfqpoint{3.286401in}{3.410173in}}%
\pgfpathcurveto{\pgfqpoint{3.286401in}{3.421223in}}{\pgfqpoint{3.282011in}{3.431822in}}{\pgfqpoint{3.274197in}{3.439636in}}%
\pgfpathcurveto{\pgfqpoint{3.266383in}{3.447449in}}{\pgfqpoint{3.255784in}{3.451840in}}{\pgfqpoint{3.244734in}{3.451840in}}%
\pgfpathcurveto{\pgfqpoint{3.233684in}{3.451840in}}{\pgfqpoint{3.223085in}{3.447449in}}{\pgfqpoint{3.215271in}{3.439636in}}%
\pgfpathcurveto{\pgfqpoint{3.207458in}{3.431822in}}{\pgfqpoint{3.203068in}{3.421223in}}{\pgfqpoint{3.203068in}{3.410173in}}%
\pgfpathcurveto{\pgfqpoint{3.203068in}{3.399123in}}{\pgfqpoint{3.207458in}{3.388524in}}{\pgfqpoint{3.215271in}{3.380710in}}%
\pgfpathcurveto{\pgfqpoint{3.223085in}{3.372897in}}{\pgfqpoint{3.233684in}{3.368506in}}{\pgfqpoint{3.244734in}{3.368506in}}%
\pgfpathclose%
\pgfusepath{stroke,fill}%
\end{pgfscope}%
\begin{pgfscope}%
\pgfpathrectangle{\pgfqpoint{0.481978in}{0.331635in}}{\pgfqpoint{4.960000in}{3.696000in}}%
\pgfusepath{clip}%
\pgfsetbuttcap%
\pgfsetroundjoin%
\definecolor{currentfill}{rgb}{1.000000,0.705882,0.509804}%
\pgfsetfillcolor{currentfill}%
\pgfsetlinewidth{0.481800pt}%
\definecolor{currentstroke}{rgb}{1.000000,1.000000,1.000000}%
\pgfsetstrokecolor{currentstroke}%
\pgfsetdash{}{0pt}%
\pgfpathmoveto{\pgfqpoint{2.824609in}{0.578820in}}%
\pgfpathcurveto{\pgfqpoint{2.835660in}{0.578820in}}{\pgfqpoint{2.846259in}{0.583210in}}{\pgfqpoint{2.854072in}{0.591024in}}%
\pgfpathcurveto{\pgfqpoint{2.861886in}{0.598837in}}{\pgfqpoint{2.866276in}{0.609436in}}{\pgfqpoint{2.866276in}{0.620486in}}%
\pgfpathcurveto{\pgfqpoint{2.866276in}{0.631537in}}{\pgfqpoint{2.861886in}{0.642136in}}{\pgfqpoint{2.854072in}{0.649949in}}%
\pgfpathcurveto{\pgfqpoint{2.846259in}{0.657763in}}{\pgfqpoint{2.835660in}{0.662153in}}{\pgfqpoint{2.824609in}{0.662153in}}%
\pgfpathcurveto{\pgfqpoint{2.813559in}{0.662153in}}{\pgfqpoint{2.802960in}{0.657763in}}{\pgfqpoint{2.795147in}{0.649949in}}%
\pgfpathcurveto{\pgfqpoint{2.787333in}{0.642136in}}{\pgfqpoint{2.782943in}{0.631537in}}{\pgfqpoint{2.782943in}{0.620486in}}%
\pgfpathcurveto{\pgfqpoint{2.782943in}{0.609436in}}{\pgfqpoint{2.787333in}{0.598837in}}{\pgfqpoint{2.795147in}{0.591024in}}%
\pgfpathcurveto{\pgfqpoint{2.802960in}{0.583210in}}{\pgfqpoint{2.813559in}{0.578820in}}{\pgfqpoint{2.824609in}{0.578820in}}%
\pgfpathclose%
\pgfusepath{stroke,fill}%
\end{pgfscope}%
\begin{pgfscope}%
\pgfpathrectangle{\pgfqpoint{0.481978in}{0.331635in}}{\pgfqpoint{4.960000in}{3.696000in}}%
\pgfusepath{clip}%
\pgfsetbuttcap%
\pgfsetroundjoin%
\definecolor{currentfill}{rgb}{1.000000,0.705882,0.509804}%
\pgfsetfillcolor{currentfill}%
\pgfsetlinewidth{0.481800pt}%
\definecolor{currentstroke}{rgb}{1.000000,1.000000,1.000000}%
\pgfsetstrokecolor{currentstroke}%
\pgfsetdash{}{0pt}%
\pgfpathmoveto{\pgfqpoint{3.965318in}{2.546716in}}%
\pgfpathcurveto{\pgfqpoint{3.976368in}{2.546716in}}{\pgfqpoint{3.986967in}{2.551106in}}{\pgfqpoint{3.994781in}{2.558920in}}%
\pgfpathcurveto{\pgfqpoint{4.002594in}{2.566734in}}{\pgfqpoint{4.006984in}{2.577333in}}{\pgfqpoint{4.006984in}{2.588383in}}%
\pgfpathcurveto{\pgfqpoint{4.006984in}{2.599433in}}{\pgfqpoint{4.002594in}{2.610032in}}{\pgfqpoint{3.994781in}{2.617846in}}%
\pgfpathcurveto{\pgfqpoint{3.986967in}{2.625659in}}{\pgfqpoint{3.976368in}{2.630050in}}{\pgfqpoint{3.965318in}{2.630050in}}%
\pgfpathcurveto{\pgfqpoint{3.954268in}{2.630050in}}{\pgfqpoint{3.943669in}{2.625659in}}{\pgfqpoint{3.935855in}{2.617846in}}%
\pgfpathcurveto{\pgfqpoint{3.928041in}{2.610032in}}{\pgfqpoint{3.923651in}{2.599433in}}{\pgfqpoint{3.923651in}{2.588383in}}%
\pgfpathcurveto{\pgfqpoint{3.923651in}{2.577333in}}{\pgfqpoint{3.928041in}{2.566734in}}{\pgfqpoint{3.935855in}{2.558920in}}%
\pgfpathcurveto{\pgfqpoint{3.943669in}{2.551106in}}{\pgfqpoint{3.954268in}{2.546716in}}{\pgfqpoint{3.965318in}{2.546716in}}%
\pgfpathclose%
\pgfusepath{stroke,fill}%
\end{pgfscope}%
\begin{pgfscope}%
\pgfpathrectangle{\pgfqpoint{0.481978in}{0.331635in}}{\pgfqpoint{4.960000in}{3.696000in}}%
\pgfusepath{clip}%
\pgfsetbuttcap%
\pgfsetroundjoin%
\definecolor{currentfill}{rgb}{1.000000,0.705882,0.509804}%
\pgfsetfillcolor{currentfill}%
\pgfsetlinewidth{0.481800pt}%
\definecolor{currentstroke}{rgb}{1.000000,1.000000,1.000000}%
\pgfsetstrokecolor{currentstroke}%
\pgfsetdash{}{0pt}%
\pgfpathmoveto{\pgfqpoint{3.018864in}{3.069749in}}%
\pgfpathcurveto{\pgfqpoint{3.029914in}{3.069749in}}{\pgfqpoint{3.040513in}{3.074139in}}{\pgfqpoint{3.048326in}{3.081953in}}%
\pgfpathcurveto{\pgfqpoint{3.056140in}{3.089766in}}{\pgfqpoint{3.060530in}{3.100365in}}{\pgfqpoint{3.060530in}{3.111415in}}%
\pgfpathcurveto{\pgfqpoint{3.060530in}{3.122466in}}{\pgfqpoint{3.056140in}{3.133065in}}{\pgfqpoint{3.048326in}{3.140878in}}%
\pgfpathcurveto{\pgfqpoint{3.040513in}{3.148692in}}{\pgfqpoint{3.029914in}{3.153082in}}{\pgfqpoint{3.018864in}{3.153082in}}%
\pgfpathcurveto{\pgfqpoint{3.007814in}{3.153082in}}{\pgfqpoint{2.997214in}{3.148692in}}{\pgfqpoint{2.989401in}{3.140878in}}%
\pgfpathcurveto{\pgfqpoint{2.981587in}{3.133065in}}{\pgfqpoint{2.977197in}{3.122466in}}{\pgfqpoint{2.977197in}{3.111415in}}%
\pgfpathcurveto{\pgfqpoint{2.977197in}{3.100365in}}{\pgfqpoint{2.981587in}{3.089766in}}{\pgfqpoint{2.989401in}{3.081953in}}%
\pgfpathcurveto{\pgfqpoint{2.997214in}{3.074139in}}{\pgfqpoint{3.007814in}{3.069749in}}{\pgfqpoint{3.018864in}{3.069749in}}%
\pgfpathclose%
\pgfusepath{stroke,fill}%
\end{pgfscope}%
\begin{pgfscope}%
\pgfpathrectangle{\pgfqpoint{0.481978in}{0.331635in}}{\pgfqpoint{4.960000in}{3.696000in}}%
\pgfusepath{clip}%
\pgfsetbuttcap%
\pgfsetroundjoin%
\definecolor{currentfill}{rgb}{1.000000,0.705882,0.509804}%
\pgfsetfillcolor{currentfill}%
\pgfsetlinewidth{0.481800pt}%
\definecolor{currentstroke}{rgb}{1.000000,1.000000,1.000000}%
\pgfsetstrokecolor{currentstroke}%
\pgfsetdash{}{0pt}%
\pgfpathmoveto{\pgfqpoint{1.498715in}{1.482040in}}%
\pgfpathcurveto{\pgfqpoint{1.509766in}{1.482040in}}{\pgfqpoint{1.520365in}{1.486430in}}{\pgfqpoint{1.528178in}{1.494244in}}%
\pgfpathcurveto{\pgfqpoint{1.535992in}{1.502058in}}{\pgfqpoint{1.540382in}{1.512657in}}{\pgfqpoint{1.540382in}{1.523707in}}%
\pgfpathcurveto{\pgfqpoint{1.540382in}{1.534757in}}{\pgfqpoint{1.535992in}{1.545356in}}{\pgfqpoint{1.528178in}{1.553170in}}%
\pgfpathcurveto{\pgfqpoint{1.520365in}{1.560983in}}{\pgfqpoint{1.509766in}{1.565374in}}{\pgfqpoint{1.498715in}{1.565374in}}%
\pgfpathcurveto{\pgfqpoint{1.487665in}{1.565374in}}{\pgfqpoint{1.477066in}{1.560983in}}{\pgfqpoint{1.469253in}{1.553170in}}%
\pgfpathcurveto{\pgfqpoint{1.461439in}{1.545356in}}{\pgfqpoint{1.457049in}{1.534757in}}{\pgfqpoint{1.457049in}{1.523707in}}%
\pgfpathcurveto{\pgfqpoint{1.457049in}{1.512657in}}{\pgfqpoint{1.461439in}{1.502058in}}{\pgfqpoint{1.469253in}{1.494244in}}%
\pgfpathcurveto{\pgfqpoint{1.477066in}{1.486430in}}{\pgfqpoint{1.487665in}{1.482040in}}{\pgfqpoint{1.498715in}{1.482040in}}%
\pgfpathclose%
\pgfusepath{stroke,fill}%
\end{pgfscope}%
\begin{pgfscope}%
\pgfpathrectangle{\pgfqpoint{0.481978in}{0.331635in}}{\pgfqpoint{4.960000in}{3.696000in}}%
\pgfusepath{clip}%
\pgfsetbuttcap%
\pgfsetroundjoin%
\definecolor{currentfill}{rgb}{1.000000,0.705882,0.509804}%
\pgfsetfillcolor{currentfill}%
\pgfsetlinewidth{0.481800pt}%
\definecolor{currentstroke}{rgb}{1.000000,1.000000,1.000000}%
\pgfsetstrokecolor{currentstroke}%
\pgfsetdash{}{0pt}%
\pgfpathmoveto{\pgfqpoint{2.639538in}{2.286841in}}%
\pgfpathcurveto{\pgfqpoint{2.650588in}{2.286841in}}{\pgfqpoint{2.661187in}{2.291231in}}{\pgfqpoint{2.669000in}{2.299045in}}%
\pgfpathcurveto{\pgfqpoint{2.676814in}{2.306858in}}{\pgfqpoint{2.681204in}{2.317457in}}{\pgfqpoint{2.681204in}{2.328507in}}%
\pgfpathcurveto{\pgfqpoint{2.681204in}{2.339558in}}{\pgfqpoint{2.676814in}{2.350157in}}{\pgfqpoint{2.669000in}{2.357970in}}%
\pgfpathcurveto{\pgfqpoint{2.661187in}{2.365784in}}{\pgfqpoint{2.650588in}{2.370174in}}{\pgfqpoint{2.639538in}{2.370174in}}%
\pgfpathcurveto{\pgfqpoint{2.628487in}{2.370174in}}{\pgfqpoint{2.617888in}{2.365784in}}{\pgfqpoint{2.610075in}{2.357970in}}%
\pgfpathcurveto{\pgfqpoint{2.602261in}{2.350157in}}{\pgfqpoint{2.597871in}{2.339558in}}{\pgfqpoint{2.597871in}{2.328507in}}%
\pgfpathcurveto{\pgfqpoint{2.597871in}{2.317457in}}{\pgfqpoint{2.602261in}{2.306858in}}{\pgfqpoint{2.610075in}{2.299045in}}%
\pgfpathcurveto{\pgfqpoint{2.617888in}{2.291231in}}{\pgfqpoint{2.628487in}{2.286841in}}{\pgfqpoint{2.639538in}{2.286841in}}%
\pgfpathclose%
\pgfusepath{stroke,fill}%
\end{pgfscope}%
\begin{pgfscope}%
\pgfpathrectangle{\pgfqpoint{0.481978in}{0.331635in}}{\pgfqpoint{4.960000in}{3.696000in}}%
\pgfusepath{clip}%
\pgfsetbuttcap%
\pgfsetroundjoin%
\definecolor{currentfill}{rgb}{1.000000,0.705882,0.509804}%
\pgfsetfillcolor{currentfill}%
\pgfsetlinewidth{0.481800pt}%
\definecolor{currentstroke}{rgb}{1.000000,1.000000,1.000000}%
\pgfsetstrokecolor{currentstroke}%
\pgfsetdash{}{0pt}%
\pgfpathmoveto{\pgfqpoint{2.371027in}{2.052417in}}%
\pgfpathcurveto{\pgfqpoint{2.382077in}{2.052417in}}{\pgfqpoint{2.392677in}{2.056808in}}{\pgfqpoint{2.400490in}{2.064621in}}%
\pgfpathcurveto{\pgfqpoint{2.408304in}{2.072435in}}{\pgfqpoint{2.412694in}{2.083034in}}{\pgfqpoint{2.412694in}{2.094084in}}%
\pgfpathcurveto{\pgfqpoint{2.412694in}{2.105134in}}{\pgfqpoint{2.408304in}{2.115733in}}{\pgfqpoint{2.400490in}{2.123547in}}%
\pgfpathcurveto{\pgfqpoint{2.392677in}{2.131361in}}{\pgfqpoint{2.382077in}{2.135751in}}{\pgfqpoint{2.371027in}{2.135751in}}%
\pgfpathcurveto{\pgfqpoint{2.359977in}{2.135751in}}{\pgfqpoint{2.349378in}{2.131361in}}{\pgfqpoint{2.341565in}{2.123547in}}%
\pgfpathcurveto{\pgfqpoint{2.333751in}{2.115733in}}{\pgfqpoint{2.329361in}{2.105134in}}{\pgfqpoint{2.329361in}{2.094084in}}%
\pgfpathcurveto{\pgfqpoint{2.329361in}{2.083034in}}{\pgfqpoint{2.333751in}{2.072435in}}{\pgfqpoint{2.341565in}{2.064621in}}%
\pgfpathcurveto{\pgfqpoint{2.349378in}{2.056808in}}{\pgfqpoint{2.359977in}{2.052417in}}{\pgfqpoint{2.371027in}{2.052417in}}%
\pgfpathclose%
\pgfusepath{stroke,fill}%
\end{pgfscope}%
\begin{pgfscope}%
\pgfpathrectangle{\pgfqpoint{0.481978in}{0.331635in}}{\pgfqpoint{4.960000in}{3.696000in}}%
\pgfusepath{clip}%
\pgfsetbuttcap%
\pgfsetroundjoin%
\definecolor{currentfill}{rgb}{1.000000,0.705882,0.509804}%
\pgfsetfillcolor{currentfill}%
\pgfsetlinewidth{0.481800pt}%
\definecolor{currentstroke}{rgb}{1.000000,1.000000,1.000000}%
\pgfsetstrokecolor{currentstroke}%
\pgfsetdash{}{0pt}%
\pgfpathmoveto{\pgfqpoint{2.555491in}{1.301171in}}%
\pgfpathcurveto{\pgfqpoint{2.566541in}{1.301171in}}{\pgfqpoint{2.577140in}{1.305561in}}{\pgfqpoint{2.584954in}{1.313375in}}%
\pgfpathcurveto{\pgfqpoint{2.592767in}{1.321188in}}{\pgfqpoint{2.597157in}{1.331787in}}{\pgfqpoint{2.597157in}{1.342838in}}%
\pgfpathcurveto{\pgfqpoint{2.597157in}{1.353888in}}{\pgfqpoint{2.592767in}{1.364487in}}{\pgfqpoint{2.584954in}{1.372300in}}%
\pgfpathcurveto{\pgfqpoint{2.577140in}{1.380114in}}{\pgfqpoint{2.566541in}{1.384504in}}{\pgfqpoint{2.555491in}{1.384504in}}%
\pgfpathcurveto{\pgfqpoint{2.544441in}{1.384504in}}{\pgfqpoint{2.533842in}{1.380114in}}{\pgfqpoint{2.526028in}{1.372300in}}%
\pgfpathcurveto{\pgfqpoint{2.518214in}{1.364487in}}{\pgfqpoint{2.513824in}{1.353888in}}{\pgfqpoint{2.513824in}{1.342838in}}%
\pgfpathcurveto{\pgfqpoint{2.513824in}{1.331787in}}{\pgfqpoint{2.518214in}{1.321188in}}{\pgfqpoint{2.526028in}{1.313375in}}%
\pgfpathcurveto{\pgfqpoint{2.533842in}{1.305561in}}{\pgfqpoint{2.544441in}{1.301171in}}{\pgfqpoint{2.555491in}{1.301171in}}%
\pgfpathclose%
\pgfusepath{stroke,fill}%
\end{pgfscope}%
\begin{pgfscope}%
\pgfpathrectangle{\pgfqpoint{0.481978in}{0.331635in}}{\pgfqpoint{4.960000in}{3.696000in}}%
\pgfusepath{clip}%
\pgfsetbuttcap%
\pgfsetroundjoin%
\definecolor{currentfill}{rgb}{1.000000,0.705882,0.509804}%
\pgfsetfillcolor{currentfill}%
\pgfsetlinewidth{0.481800pt}%
\definecolor{currentstroke}{rgb}{1.000000,1.000000,1.000000}%
\pgfsetstrokecolor{currentstroke}%
\pgfsetdash{}{0pt}%
\pgfpathmoveto{\pgfqpoint{4.217987in}{2.538643in}}%
\pgfpathcurveto{\pgfqpoint{4.229037in}{2.538643in}}{\pgfqpoint{4.239636in}{2.543033in}}{\pgfqpoint{4.247450in}{2.550847in}}%
\pgfpathcurveto{\pgfqpoint{4.255264in}{2.558661in}}{\pgfqpoint{4.259654in}{2.569260in}}{\pgfqpoint{4.259654in}{2.580310in}}%
\pgfpathcurveto{\pgfqpoint{4.259654in}{2.591360in}}{\pgfqpoint{4.255264in}{2.601959in}}{\pgfqpoint{4.247450in}{2.609772in}}%
\pgfpathcurveto{\pgfqpoint{4.239636in}{2.617586in}}{\pgfqpoint{4.229037in}{2.621976in}}{\pgfqpoint{4.217987in}{2.621976in}}%
\pgfpathcurveto{\pgfqpoint{4.206937in}{2.621976in}}{\pgfqpoint{4.196338in}{2.617586in}}{\pgfqpoint{4.188524in}{2.609772in}}%
\pgfpathcurveto{\pgfqpoint{4.180711in}{2.601959in}}{\pgfqpoint{4.176320in}{2.591360in}}{\pgfqpoint{4.176320in}{2.580310in}}%
\pgfpathcurveto{\pgfqpoint{4.176320in}{2.569260in}}{\pgfqpoint{4.180711in}{2.558661in}}{\pgfqpoint{4.188524in}{2.550847in}}%
\pgfpathcurveto{\pgfqpoint{4.196338in}{2.543033in}}{\pgfqpoint{4.206937in}{2.538643in}}{\pgfqpoint{4.217987in}{2.538643in}}%
\pgfpathclose%
\pgfusepath{stroke,fill}%
\end{pgfscope}%
\begin{pgfscope}%
\pgfpathrectangle{\pgfqpoint{0.481978in}{0.331635in}}{\pgfqpoint{4.960000in}{3.696000in}}%
\pgfusepath{clip}%
\pgfsetbuttcap%
\pgfsetroundjoin%
\definecolor{currentfill}{rgb}{1.000000,0.705882,0.509804}%
\pgfsetfillcolor{currentfill}%
\pgfsetlinewidth{0.481800pt}%
\definecolor{currentstroke}{rgb}{1.000000,1.000000,1.000000}%
\pgfsetstrokecolor{currentstroke}%
\pgfsetdash{}{0pt}%
\pgfpathmoveto{\pgfqpoint{1.306020in}{2.485800in}}%
\pgfpathcurveto{\pgfqpoint{1.317070in}{2.485800in}}{\pgfqpoint{1.327669in}{2.490190in}}{\pgfqpoint{1.335483in}{2.498004in}}%
\pgfpathcurveto{\pgfqpoint{1.343296in}{2.505818in}}{\pgfqpoint{1.347686in}{2.516417in}}{\pgfqpoint{1.347686in}{2.527467in}}%
\pgfpathcurveto{\pgfqpoint{1.347686in}{2.538517in}}{\pgfqpoint{1.343296in}{2.549116in}}{\pgfqpoint{1.335483in}{2.556930in}}%
\pgfpathcurveto{\pgfqpoint{1.327669in}{2.564743in}}{\pgfqpoint{1.317070in}{2.569134in}}{\pgfqpoint{1.306020in}{2.569134in}}%
\pgfpathcurveto{\pgfqpoint{1.294970in}{2.569134in}}{\pgfqpoint{1.284371in}{2.564743in}}{\pgfqpoint{1.276557in}{2.556930in}}%
\pgfpathcurveto{\pgfqpoint{1.268743in}{2.549116in}}{\pgfqpoint{1.264353in}{2.538517in}}{\pgfqpoint{1.264353in}{2.527467in}}%
\pgfpathcurveto{\pgfqpoint{1.264353in}{2.516417in}}{\pgfqpoint{1.268743in}{2.505818in}}{\pgfqpoint{1.276557in}{2.498004in}}%
\pgfpathcurveto{\pgfqpoint{1.284371in}{2.490190in}}{\pgfqpoint{1.294970in}{2.485800in}}{\pgfqpoint{1.306020in}{2.485800in}}%
\pgfpathclose%
\pgfusepath{stroke,fill}%
\end{pgfscope}%
\begin{pgfscope}%
\pgfpathrectangle{\pgfqpoint{0.481978in}{0.331635in}}{\pgfqpoint{4.960000in}{3.696000in}}%
\pgfusepath{clip}%
\pgfsetbuttcap%
\pgfsetroundjoin%
\definecolor{currentfill}{rgb}{1.000000,0.705882,0.509804}%
\pgfsetfillcolor{currentfill}%
\pgfsetlinewidth{0.481800pt}%
\definecolor{currentstroke}{rgb}{1.000000,1.000000,1.000000}%
\pgfsetstrokecolor{currentstroke}%
\pgfsetdash{}{0pt}%
\pgfpathmoveto{\pgfqpoint{4.013546in}{0.985137in}}%
\pgfpathcurveto{\pgfqpoint{4.024596in}{0.985137in}}{\pgfqpoint{4.035195in}{0.989527in}}{\pgfqpoint{4.043009in}{0.997340in}}%
\pgfpathcurveto{\pgfqpoint{4.050823in}{1.005154in}}{\pgfqpoint{4.055213in}{1.015753in}}{\pgfqpoint{4.055213in}{1.026803in}}%
\pgfpathcurveto{\pgfqpoint{4.055213in}{1.037853in}}{\pgfqpoint{4.050823in}{1.048452in}}{\pgfqpoint{4.043009in}{1.056266in}}%
\pgfpathcurveto{\pgfqpoint{4.035195in}{1.064080in}}{\pgfqpoint{4.024596in}{1.068470in}}{\pgfqpoint{4.013546in}{1.068470in}}%
\pgfpathcurveto{\pgfqpoint{4.002496in}{1.068470in}}{\pgfqpoint{3.991897in}{1.064080in}}{\pgfqpoint{3.984083in}{1.056266in}}%
\pgfpathcurveto{\pgfqpoint{3.976270in}{1.048452in}}{\pgfqpoint{3.971879in}{1.037853in}}{\pgfqpoint{3.971879in}{1.026803in}}%
\pgfpathcurveto{\pgfqpoint{3.971879in}{1.015753in}}{\pgfqpoint{3.976270in}{1.005154in}}{\pgfqpoint{3.984083in}{0.997340in}}%
\pgfpathcurveto{\pgfqpoint{3.991897in}{0.989527in}}{\pgfqpoint{4.002496in}{0.985137in}}{\pgfqpoint{4.013546in}{0.985137in}}%
\pgfpathclose%
\pgfusepath{stroke,fill}%
\end{pgfscope}%
\begin{pgfscope}%
\pgfpathrectangle{\pgfqpoint{0.481978in}{0.331635in}}{\pgfqpoint{4.960000in}{3.696000in}}%
\pgfusepath{clip}%
\pgfsetbuttcap%
\pgfsetroundjoin%
\definecolor{currentfill}{rgb}{1.000000,0.705882,0.509804}%
\pgfsetfillcolor{currentfill}%
\pgfsetlinewidth{0.481800pt}%
\definecolor{currentstroke}{rgb}{1.000000,1.000000,1.000000}%
\pgfsetstrokecolor{currentstroke}%
\pgfsetdash{}{0pt}%
\pgfpathmoveto{\pgfqpoint{3.031867in}{1.200718in}}%
\pgfpathcurveto{\pgfqpoint{3.042917in}{1.200718in}}{\pgfqpoint{3.053516in}{1.205109in}}{\pgfqpoint{3.061330in}{1.212922in}}%
\pgfpathcurveto{\pgfqpoint{3.069143in}{1.220736in}}{\pgfqpoint{3.073534in}{1.231335in}}{\pgfqpoint{3.073534in}{1.242385in}}%
\pgfpathcurveto{\pgfqpoint{3.073534in}{1.253435in}}{\pgfqpoint{3.069143in}{1.264034in}}{\pgfqpoint{3.061330in}{1.271848in}}%
\pgfpathcurveto{\pgfqpoint{3.053516in}{1.279661in}}{\pgfqpoint{3.042917in}{1.284052in}}{\pgfqpoint{3.031867in}{1.284052in}}%
\pgfpathcurveto{\pgfqpoint{3.020817in}{1.284052in}}{\pgfqpoint{3.010218in}{1.279661in}}{\pgfqpoint{3.002404in}{1.271848in}}%
\pgfpathcurveto{\pgfqpoint{2.994591in}{1.264034in}}{\pgfqpoint{2.990200in}{1.253435in}}{\pgfqpoint{2.990200in}{1.242385in}}%
\pgfpathcurveto{\pgfqpoint{2.990200in}{1.231335in}}{\pgfqpoint{2.994591in}{1.220736in}}{\pgfqpoint{3.002404in}{1.212922in}}%
\pgfpathcurveto{\pgfqpoint{3.010218in}{1.205109in}}{\pgfqpoint{3.020817in}{1.200718in}}{\pgfqpoint{3.031867in}{1.200718in}}%
\pgfpathclose%
\pgfusepath{stroke,fill}%
\end{pgfscope}%
\begin{pgfscope}%
\pgfpathrectangle{\pgfqpoint{0.481978in}{0.331635in}}{\pgfqpoint{4.960000in}{3.696000in}}%
\pgfusepath{clip}%
\pgfsetbuttcap%
\pgfsetroundjoin%
\definecolor{currentfill}{rgb}{1.000000,0.705882,0.509804}%
\pgfsetfillcolor{currentfill}%
\pgfsetlinewidth{0.481800pt}%
\definecolor{currentstroke}{rgb}{1.000000,1.000000,1.000000}%
\pgfsetstrokecolor{currentstroke}%
\pgfsetdash{}{0pt}%
\pgfpathmoveto{\pgfqpoint{2.140155in}{0.763193in}}%
\pgfpathcurveto{\pgfqpoint{2.151205in}{0.763193in}}{\pgfqpoint{2.161804in}{0.767583in}}{\pgfqpoint{2.169617in}{0.775397in}}%
\pgfpathcurveto{\pgfqpoint{2.177431in}{0.783210in}}{\pgfqpoint{2.181821in}{0.793809in}}{\pgfqpoint{2.181821in}{0.804859in}}%
\pgfpathcurveto{\pgfqpoint{2.181821in}{0.815910in}}{\pgfqpoint{2.177431in}{0.826509in}}{\pgfqpoint{2.169617in}{0.834322in}}%
\pgfpathcurveto{\pgfqpoint{2.161804in}{0.842136in}}{\pgfqpoint{2.151205in}{0.846526in}}{\pgfqpoint{2.140155in}{0.846526in}}%
\pgfpathcurveto{\pgfqpoint{2.129104in}{0.846526in}}{\pgfqpoint{2.118505in}{0.842136in}}{\pgfqpoint{2.110692in}{0.834322in}}%
\pgfpathcurveto{\pgfqpoint{2.102878in}{0.826509in}}{\pgfqpoint{2.098488in}{0.815910in}}{\pgfqpoint{2.098488in}{0.804859in}}%
\pgfpathcurveto{\pgfqpoint{2.098488in}{0.793809in}}{\pgfqpoint{2.102878in}{0.783210in}}{\pgfqpoint{2.110692in}{0.775397in}}%
\pgfpathcurveto{\pgfqpoint{2.118505in}{0.767583in}}{\pgfqpoint{2.129104in}{0.763193in}}{\pgfqpoint{2.140155in}{0.763193in}}%
\pgfpathclose%
\pgfusepath{stroke,fill}%
\end{pgfscope}%
\begin{pgfscope}%
\pgfpathrectangle{\pgfqpoint{0.481978in}{0.331635in}}{\pgfqpoint{4.960000in}{3.696000in}}%
\pgfusepath{clip}%
\pgfsetbuttcap%
\pgfsetroundjoin%
\definecolor{currentfill}{rgb}{1.000000,0.705882,0.509804}%
\pgfsetfillcolor{currentfill}%
\pgfsetlinewidth{0.481800pt}%
\definecolor{currentstroke}{rgb}{1.000000,1.000000,1.000000}%
\pgfsetstrokecolor{currentstroke}%
\pgfsetdash{}{0pt}%
\pgfpathmoveto{\pgfqpoint{3.374264in}{3.473227in}}%
\pgfpathcurveto{\pgfqpoint{3.385314in}{3.473227in}}{\pgfqpoint{3.395913in}{3.477617in}}{\pgfqpoint{3.403726in}{3.485431in}}%
\pgfpathcurveto{\pgfqpoint{3.411540in}{3.493244in}}{\pgfqpoint{3.415930in}{3.503843in}}{\pgfqpoint{3.415930in}{3.514894in}}%
\pgfpathcurveto{\pgfqpoint{3.415930in}{3.525944in}}{\pgfqpoint{3.411540in}{3.536543in}}{\pgfqpoint{3.403726in}{3.544356in}}%
\pgfpathcurveto{\pgfqpoint{3.395913in}{3.552170in}}{\pgfqpoint{3.385314in}{3.556560in}}{\pgfqpoint{3.374264in}{3.556560in}}%
\pgfpathcurveto{\pgfqpoint{3.363214in}{3.556560in}}{\pgfqpoint{3.352614in}{3.552170in}}{\pgfqpoint{3.344801in}{3.544356in}}%
\pgfpathcurveto{\pgfqpoint{3.336987in}{3.536543in}}{\pgfqpoint{3.332597in}{3.525944in}}{\pgfqpoint{3.332597in}{3.514894in}}%
\pgfpathcurveto{\pgfqpoint{3.332597in}{3.503843in}}{\pgfqpoint{3.336987in}{3.493244in}}{\pgfqpoint{3.344801in}{3.485431in}}%
\pgfpathcurveto{\pgfqpoint{3.352614in}{3.477617in}}{\pgfqpoint{3.363214in}{3.473227in}}{\pgfqpoint{3.374264in}{3.473227in}}%
\pgfpathclose%
\pgfusepath{stroke,fill}%
\end{pgfscope}%
\begin{pgfscope}%
\pgfpathrectangle{\pgfqpoint{0.481978in}{0.331635in}}{\pgfqpoint{4.960000in}{3.696000in}}%
\pgfusepath{clip}%
\pgfsetbuttcap%
\pgfsetroundjoin%
\definecolor{currentfill}{rgb}{1.000000,0.705882,0.509804}%
\pgfsetfillcolor{currentfill}%
\pgfsetlinewidth{0.481800pt}%
\definecolor{currentstroke}{rgb}{1.000000,1.000000,1.000000}%
\pgfsetstrokecolor{currentstroke}%
\pgfsetdash{}{0pt}%
\pgfpathmoveto{\pgfqpoint{3.767977in}{3.466684in}}%
\pgfpathcurveto{\pgfqpoint{3.779027in}{3.466684in}}{\pgfqpoint{3.789626in}{3.471074in}}{\pgfqpoint{3.797440in}{3.478888in}}%
\pgfpathcurveto{\pgfqpoint{3.805254in}{3.486702in}}{\pgfqpoint{3.809644in}{3.497301in}}{\pgfqpoint{3.809644in}{3.508351in}}%
\pgfpathcurveto{\pgfqpoint{3.809644in}{3.519401in}}{\pgfqpoint{3.805254in}{3.530000in}}{\pgfqpoint{3.797440in}{3.537814in}}%
\pgfpathcurveto{\pgfqpoint{3.789626in}{3.545627in}}{\pgfqpoint{3.779027in}{3.550017in}}{\pgfqpoint{3.767977in}{3.550017in}}%
\pgfpathcurveto{\pgfqpoint{3.756927in}{3.550017in}}{\pgfqpoint{3.746328in}{3.545627in}}{\pgfqpoint{3.738515in}{3.537814in}}%
\pgfpathcurveto{\pgfqpoint{3.730701in}{3.530000in}}{\pgfqpoint{3.726311in}{3.519401in}}{\pgfqpoint{3.726311in}{3.508351in}}%
\pgfpathcurveto{\pgfqpoint{3.726311in}{3.497301in}}{\pgfqpoint{3.730701in}{3.486702in}}{\pgfqpoint{3.738515in}{3.478888in}}%
\pgfpathcurveto{\pgfqpoint{3.746328in}{3.471074in}}{\pgfqpoint{3.756927in}{3.466684in}}{\pgfqpoint{3.767977in}{3.466684in}}%
\pgfpathclose%
\pgfusepath{stroke,fill}%
\end{pgfscope}%
\begin{pgfscope}%
\pgfpathrectangle{\pgfqpoint{0.481978in}{0.331635in}}{\pgfqpoint{4.960000in}{3.696000in}}%
\pgfusepath{clip}%
\pgfsetbuttcap%
\pgfsetroundjoin%
\definecolor{currentfill}{rgb}{1.000000,0.705882,0.509804}%
\pgfsetfillcolor{currentfill}%
\pgfsetlinewidth{0.481800pt}%
\definecolor{currentstroke}{rgb}{1.000000,1.000000,1.000000}%
\pgfsetstrokecolor{currentstroke}%
\pgfsetdash{}{0pt}%
\pgfpathmoveto{\pgfqpoint{3.773339in}{2.902883in}}%
\pgfpathcurveto{\pgfqpoint{3.784389in}{2.902883in}}{\pgfqpoint{3.794988in}{2.907273in}}{\pgfqpoint{3.802802in}{2.915086in}}%
\pgfpathcurveto{\pgfqpoint{3.810616in}{2.922900in}}{\pgfqpoint{3.815006in}{2.933499in}}{\pgfqpoint{3.815006in}{2.944549in}}%
\pgfpathcurveto{\pgfqpoint{3.815006in}{2.955599in}}{\pgfqpoint{3.810616in}{2.966198in}}{\pgfqpoint{3.802802in}{2.974012in}}%
\pgfpathcurveto{\pgfqpoint{3.794988in}{2.981826in}}{\pgfqpoint{3.784389in}{2.986216in}}{\pgfqpoint{3.773339in}{2.986216in}}%
\pgfpathcurveto{\pgfqpoint{3.762289in}{2.986216in}}{\pgfqpoint{3.751690in}{2.981826in}}{\pgfqpoint{3.743876in}{2.974012in}}%
\pgfpathcurveto{\pgfqpoint{3.736063in}{2.966198in}}{\pgfqpoint{3.731672in}{2.955599in}}{\pgfqpoint{3.731672in}{2.944549in}}%
\pgfpathcurveto{\pgfqpoint{3.731672in}{2.933499in}}{\pgfqpoint{3.736063in}{2.922900in}}{\pgfqpoint{3.743876in}{2.915086in}}%
\pgfpathcurveto{\pgfqpoint{3.751690in}{2.907273in}}{\pgfqpoint{3.762289in}{2.902883in}}{\pgfqpoint{3.773339in}{2.902883in}}%
\pgfpathclose%
\pgfusepath{stroke,fill}%
\end{pgfscope}%
\begin{pgfscope}%
\pgfpathrectangle{\pgfqpoint{0.481978in}{0.331635in}}{\pgfqpoint{4.960000in}{3.696000in}}%
\pgfusepath{clip}%
\pgfsetbuttcap%
\pgfsetroundjoin%
\definecolor{currentfill}{rgb}{1.000000,0.705882,0.509804}%
\pgfsetfillcolor{currentfill}%
\pgfsetlinewidth{0.481800pt}%
\definecolor{currentstroke}{rgb}{1.000000,1.000000,1.000000}%
\pgfsetstrokecolor{currentstroke}%
\pgfsetdash{}{0pt}%
\pgfpathmoveto{\pgfqpoint{1.255414in}{1.973761in}}%
\pgfpathcurveto{\pgfqpoint{1.266464in}{1.973761in}}{\pgfqpoint{1.277064in}{1.978152in}}{\pgfqpoint{1.284877in}{1.985965in}}%
\pgfpathcurveto{\pgfqpoint{1.292691in}{1.993779in}}{\pgfqpoint{1.297081in}{2.004378in}}{\pgfqpoint{1.297081in}{2.015428in}}%
\pgfpathcurveto{\pgfqpoint{1.297081in}{2.026478in}}{\pgfqpoint{1.292691in}{2.037077in}}{\pgfqpoint{1.284877in}{2.044891in}}%
\pgfpathcurveto{\pgfqpoint{1.277064in}{2.052704in}}{\pgfqpoint{1.266464in}{2.057095in}}{\pgfqpoint{1.255414in}{2.057095in}}%
\pgfpathcurveto{\pgfqpoint{1.244364in}{2.057095in}}{\pgfqpoint{1.233765in}{2.052704in}}{\pgfqpoint{1.225952in}{2.044891in}}%
\pgfpathcurveto{\pgfqpoint{1.218138in}{2.037077in}}{\pgfqpoint{1.213748in}{2.026478in}}{\pgfqpoint{1.213748in}{2.015428in}}%
\pgfpathcurveto{\pgfqpoint{1.213748in}{2.004378in}}{\pgfqpoint{1.218138in}{1.993779in}}{\pgfqpoint{1.225952in}{1.985965in}}%
\pgfpathcurveto{\pgfqpoint{1.233765in}{1.978152in}}{\pgfqpoint{1.244364in}{1.973761in}}{\pgfqpoint{1.255414in}{1.973761in}}%
\pgfpathclose%
\pgfusepath{stroke,fill}%
\end{pgfscope}%
\begin{pgfscope}%
\pgfpathrectangle{\pgfqpoint{0.481978in}{0.331635in}}{\pgfqpoint{4.960000in}{3.696000in}}%
\pgfusepath{clip}%
\pgfsetbuttcap%
\pgfsetroundjoin%
\definecolor{currentfill}{rgb}{1.000000,0.705882,0.509804}%
\pgfsetfillcolor{currentfill}%
\pgfsetlinewidth{0.481800pt}%
\definecolor{currentstroke}{rgb}{1.000000,1.000000,1.000000}%
\pgfsetstrokecolor{currentstroke}%
\pgfsetdash{}{0pt}%
\pgfpathmoveto{\pgfqpoint{2.406351in}{0.565994in}}%
\pgfpathcurveto{\pgfqpoint{2.417401in}{0.565994in}}{\pgfqpoint{2.428000in}{0.570384in}}{\pgfqpoint{2.435814in}{0.578198in}}%
\pgfpathcurveto{\pgfqpoint{2.443628in}{0.586011in}}{\pgfqpoint{2.448018in}{0.596610in}}{\pgfqpoint{2.448018in}{0.607661in}}%
\pgfpathcurveto{\pgfqpoint{2.448018in}{0.618711in}}{\pgfqpoint{2.443628in}{0.629310in}}{\pgfqpoint{2.435814in}{0.637123in}}%
\pgfpathcurveto{\pgfqpoint{2.428000in}{0.644937in}}{\pgfqpoint{2.417401in}{0.649327in}}{\pgfqpoint{2.406351in}{0.649327in}}%
\pgfpathcurveto{\pgfqpoint{2.395301in}{0.649327in}}{\pgfqpoint{2.384702in}{0.644937in}}{\pgfqpoint{2.376888in}{0.637123in}}%
\pgfpathcurveto{\pgfqpoint{2.369075in}{0.629310in}}{\pgfqpoint{2.364684in}{0.618711in}}{\pgfqpoint{2.364684in}{0.607661in}}%
\pgfpathcurveto{\pgfqpoint{2.364684in}{0.596610in}}{\pgfqpoint{2.369075in}{0.586011in}}{\pgfqpoint{2.376888in}{0.578198in}}%
\pgfpathcurveto{\pgfqpoint{2.384702in}{0.570384in}}{\pgfqpoint{2.395301in}{0.565994in}}{\pgfqpoint{2.406351in}{0.565994in}}%
\pgfpathclose%
\pgfusepath{stroke,fill}%
\end{pgfscope}%
\begin{pgfscope}%
\pgfpathrectangle{\pgfqpoint{0.481978in}{0.331635in}}{\pgfqpoint{4.960000in}{3.696000in}}%
\pgfusepath{clip}%
\pgfsetbuttcap%
\pgfsetroundjoin%
\definecolor{currentfill}{rgb}{1.000000,0.705882,0.509804}%
\pgfsetfillcolor{currentfill}%
\pgfsetlinewidth{0.481800pt}%
\definecolor{currentstroke}{rgb}{1.000000,1.000000,1.000000}%
\pgfsetstrokecolor{currentstroke}%
\pgfsetdash{}{0pt}%
\pgfpathmoveto{\pgfqpoint{2.404152in}{2.392618in}}%
\pgfpathcurveto{\pgfqpoint{2.415202in}{2.392618in}}{\pgfqpoint{2.425801in}{2.397009in}}{\pgfqpoint{2.433615in}{2.404822in}}%
\pgfpathcurveto{\pgfqpoint{2.441428in}{2.412636in}}{\pgfqpoint{2.445819in}{2.423235in}}{\pgfqpoint{2.445819in}{2.434285in}}%
\pgfpathcurveto{\pgfqpoint{2.445819in}{2.445335in}}{\pgfqpoint{2.441428in}{2.455934in}}{\pgfqpoint{2.433615in}{2.463748in}}%
\pgfpathcurveto{\pgfqpoint{2.425801in}{2.471561in}}{\pgfqpoint{2.415202in}{2.475952in}}{\pgfqpoint{2.404152in}{2.475952in}}%
\pgfpathcurveto{\pgfqpoint{2.393102in}{2.475952in}}{\pgfqpoint{2.382503in}{2.471561in}}{\pgfqpoint{2.374689in}{2.463748in}}%
\pgfpathcurveto{\pgfqpoint{2.366876in}{2.455934in}}{\pgfqpoint{2.362485in}{2.445335in}}{\pgfqpoint{2.362485in}{2.434285in}}%
\pgfpathcurveto{\pgfqpoint{2.362485in}{2.423235in}}{\pgfqpoint{2.366876in}{2.412636in}}{\pgfqpoint{2.374689in}{2.404822in}}%
\pgfpathcurveto{\pgfqpoint{2.382503in}{2.397009in}}{\pgfqpoint{2.393102in}{2.392618in}}{\pgfqpoint{2.404152in}{2.392618in}}%
\pgfpathclose%
\pgfusepath{stroke,fill}%
\end{pgfscope}%
\begin{pgfscope}%
\pgfpathrectangle{\pgfqpoint{0.481978in}{0.331635in}}{\pgfqpoint{4.960000in}{3.696000in}}%
\pgfusepath{clip}%
\pgfsetbuttcap%
\pgfsetroundjoin%
\definecolor{currentfill}{rgb}{1.000000,0.705882,0.509804}%
\pgfsetfillcolor{currentfill}%
\pgfsetlinewidth{0.481800pt}%
\definecolor{currentstroke}{rgb}{1.000000,1.000000,1.000000}%
\pgfsetstrokecolor{currentstroke}%
\pgfsetdash{}{0pt}%
\pgfpathmoveto{\pgfqpoint{1.286887in}{2.989286in}}%
\pgfpathcurveto{\pgfqpoint{1.297937in}{2.989286in}}{\pgfqpoint{1.308536in}{2.993676in}}{\pgfqpoint{1.316349in}{3.001490in}}%
\pgfpathcurveto{\pgfqpoint{1.324163in}{3.009303in}}{\pgfqpoint{1.328553in}{3.019902in}}{\pgfqpoint{1.328553in}{3.030953in}}%
\pgfpathcurveto{\pgfqpoint{1.328553in}{3.042003in}}{\pgfqpoint{1.324163in}{3.052602in}}{\pgfqpoint{1.316349in}{3.060415in}}%
\pgfpathcurveto{\pgfqpoint{1.308536in}{3.068229in}}{\pgfqpoint{1.297937in}{3.072619in}}{\pgfqpoint{1.286887in}{3.072619in}}%
\pgfpathcurveto{\pgfqpoint{1.275836in}{3.072619in}}{\pgfqpoint{1.265237in}{3.068229in}}{\pgfqpoint{1.257424in}{3.060415in}}%
\pgfpathcurveto{\pgfqpoint{1.249610in}{3.052602in}}{\pgfqpoint{1.245220in}{3.042003in}}{\pgfqpoint{1.245220in}{3.030953in}}%
\pgfpathcurveto{\pgfqpoint{1.245220in}{3.019902in}}{\pgfqpoint{1.249610in}{3.009303in}}{\pgfqpoint{1.257424in}{3.001490in}}%
\pgfpathcurveto{\pgfqpoint{1.265237in}{2.993676in}}{\pgfqpoint{1.275836in}{2.989286in}}{\pgfqpoint{1.286887in}{2.989286in}}%
\pgfpathclose%
\pgfusepath{stroke,fill}%
\end{pgfscope}%
\begin{pgfscope}%
\pgfpathrectangle{\pgfqpoint{0.481978in}{0.331635in}}{\pgfqpoint{4.960000in}{3.696000in}}%
\pgfusepath{clip}%
\pgfsetbuttcap%
\pgfsetroundjoin%
\definecolor{currentfill}{rgb}{1.000000,0.705882,0.509804}%
\pgfsetfillcolor{currentfill}%
\pgfsetlinewidth{0.481800pt}%
\definecolor{currentstroke}{rgb}{1.000000,1.000000,1.000000}%
\pgfsetstrokecolor{currentstroke}%
\pgfsetdash{}{0pt}%
\pgfpathmoveto{\pgfqpoint{0.714090in}{2.396830in}}%
\pgfpathcurveto{\pgfqpoint{0.725140in}{2.396830in}}{\pgfqpoint{0.735739in}{2.401220in}}{\pgfqpoint{0.743553in}{2.409034in}}%
\pgfpathcurveto{\pgfqpoint{0.751366in}{2.416847in}}{\pgfqpoint{0.755756in}{2.427446in}}{\pgfqpoint{0.755756in}{2.438496in}}%
\pgfpathcurveto{\pgfqpoint{0.755756in}{2.449546in}}{\pgfqpoint{0.751366in}{2.460145in}}{\pgfqpoint{0.743553in}{2.467959in}}%
\pgfpathcurveto{\pgfqpoint{0.735739in}{2.475773in}}{\pgfqpoint{0.725140in}{2.480163in}}{\pgfqpoint{0.714090in}{2.480163in}}%
\pgfpathcurveto{\pgfqpoint{0.703040in}{2.480163in}}{\pgfqpoint{0.692441in}{2.475773in}}{\pgfqpoint{0.684627in}{2.467959in}}%
\pgfpathcurveto{\pgfqpoint{0.676813in}{2.460145in}}{\pgfqpoint{0.672423in}{2.449546in}}{\pgfqpoint{0.672423in}{2.438496in}}%
\pgfpathcurveto{\pgfqpoint{0.672423in}{2.427446in}}{\pgfqpoint{0.676813in}{2.416847in}}{\pgfqpoint{0.684627in}{2.409034in}}%
\pgfpathcurveto{\pgfqpoint{0.692441in}{2.401220in}}{\pgfqpoint{0.703040in}{2.396830in}}{\pgfqpoint{0.714090in}{2.396830in}}%
\pgfpathclose%
\pgfusepath{stroke,fill}%
\end{pgfscope}%
\begin{pgfscope}%
\pgfpathrectangle{\pgfqpoint{0.481978in}{0.331635in}}{\pgfqpoint{4.960000in}{3.696000in}}%
\pgfusepath{clip}%
\pgfsetbuttcap%
\pgfsetroundjoin%
\definecolor{currentfill}{rgb}{1.000000,0.705882,0.509804}%
\pgfsetfillcolor{currentfill}%
\pgfsetlinewidth{0.481800pt}%
\definecolor{currentstroke}{rgb}{1.000000,1.000000,1.000000}%
\pgfsetstrokecolor{currentstroke}%
\pgfsetdash{}{0pt}%
\pgfpathmoveto{\pgfqpoint{3.131682in}{3.324801in}}%
\pgfpathcurveto{\pgfqpoint{3.142732in}{3.324801in}}{\pgfqpoint{3.153331in}{3.329191in}}{\pgfqpoint{3.161145in}{3.337005in}}%
\pgfpathcurveto{\pgfqpoint{3.168958in}{3.344818in}}{\pgfqpoint{3.173348in}{3.355417in}}{\pgfqpoint{3.173348in}{3.366467in}}%
\pgfpathcurveto{\pgfqpoint{3.173348in}{3.377517in}}{\pgfqpoint{3.168958in}{3.388117in}}{\pgfqpoint{3.161145in}{3.395930in}}%
\pgfpathcurveto{\pgfqpoint{3.153331in}{3.403744in}}{\pgfqpoint{3.142732in}{3.408134in}}{\pgfqpoint{3.131682in}{3.408134in}}%
\pgfpathcurveto{\pgfqpoint{3.120632in}{3.408134in}}{\pgfqpoint{3.110033in}{3.403744in}}{\pgfqpoint{3.102219in}{3.395930in}}%
\pgfpathcurveto{\pgfqpoint{3.094405in}{3.388117in}}{\pgfqpoint{3.090015in}{3.377517in}}{\pgfqpoint{3.090015in}{3.366467in}}%
\pgfpathcurveto{\pgfqpoint{3.090015in}{3.355417in}}{\pgfqpoint{3.094405in}{3.344818in}}{\pgfqpoint{3.102219in}{3.337005in}}%
\pgfpathcurveto{\pgfqpoint{3.110033in}{3.329191in}}{\pgfqpoint{3.120632in}{3.324801in}}{\pgfqpoint{3.131682in}{3.324801in}}%
\pgfpathclose%
\pgfusepath{stroke,fill}%
\end{pgfscope}%
\begin{pgfscope}%
\pgfpathrectangle{\pgfqpoint{0.481978in}{0.331635in}}{\pgfqpoint{4.960000in}{3.696000in}}%
\pgfusepath{clip}%
\pgfsetbuttcap%
\pgfsetroundjoin%
\definecolor{currentfill}{rgb}{1.000000,0.705882,0.509804}%
\pgfsetfillcolor{currentfill}%
\pgfsetlinewidth{0.481800pt}%
\definecolor{currentstroke}{rgb}{1.000000,1.000000,1.000000}%
\pgfsetstrokecolor{currentstroke}%
\pgfsetdash{}{0pt}%
\pgfpathmoveto{\pgfqpoint{3.213482in}{2.451282in}}%
\pgfpathcurveto{\pgfqpoint{3.224532in}{2.451282in}}{\pgfqpoint{3.235131in}{2.455673in}}{\pgfqpoint{3.242945in}{2.463486in}}%
\pgfpathcurveto{\pgfqpoint{3.250759in}{2.471300in}}{\pgfqpoint{3.255149in}{2.481899in}}{\pgfqpoint{3.255149in}{2.492949in}}%
\pgfpathcurveto{\pgfqpoint{3.255149in}{2.503999in}}{\pgfqpoint{3.250759in}{2.514598in}}{\pgfqpoint{3.242945in}{2.522412in}}%
\pgfpathcurveto{\pgfqpoint{3.235131in}{2.530225in}}{\pgfqpoint{3.224532in}{2.534616in}}{\pgfqpoint{3.213482in}{2.534616in}}%
\pgfpathcurveto{\pgfqpoint{3.202432in}{2.534616in}}{\pgfqpoint{3.191833in}{2.530225in}}{\pgfqpoint{3.184019in}{2.522412in}}%
\pgfpathcurveto{\pgfqpoint{3.176206in}{2.514598in}}{\pgfqpoint{3.171816in}{2.503999in}}{\pgfqpoint{3.171816in}{2.492949in}}%
\pgfpathcurveto{\pgfqpoint{3.171816in}{2.481899in}}{\pgfqpoint{3.176206in}{2.471300in}}{\pgfqpoint{3.184019in}{2.463486in}}%
\pgfpathcurveto{\pgfqpoint{3.191833in}{2.455673in}}{\pgfqpoint{3.202432in}{2.451282in}}{\pgfqpoint{3.213482in}{2.451282in}}%
\pgfpathclose%
\pgfusepath{stroke,fill}%
\end{pgfscope}%
\begin{pgfscope}%
\pgfpathrectangle{\pgfqpoint{0.481978in}{0.331635in}}{\pgfqpoint{4.960000in}{3.696000in}}%
\pgfusepath{clip}%
\pgfsetbuttcap%
\pgfsetroundjoin%
\definecolor{currentfill}{rgb}{1.000000,0.705882,0.509804}%
\pgfsetfillcolor{currentfill}%
\pgfsetlinewidth{0.481800pt}%
\definecolor{currentstroke}{rgb}{1.000000,1.000000,1.000000}%
\pgfsetstrokecolor{currentstroke}%
\pgfsetdash{}{0pt}%
\pgfpathmoveto{\pgfqpoint{3.222017in}{0.678565in}}%
\pgfpathcurveto{\pgfqpoint{3.233067in}{0.678565in}}{\pgfqpoint{3.243666in}{0.682955in}}{\pgfqpoint{3.251479in}{0.690769in}}%
\pgfpathcurveto{\pgfqpoint{3.259293in}{0.698582in}}{\pgfqpoint{3.263683in}{0.709181in}}{\pgfqpoint{3.263683in}{0.720231in}}%
\pgfpathcurveto{\pgfqpoint{3.263683in}{0.731281in}}{\pgfqpoint{3.259293in}{0.741881in}}{\pgfqpoint{3.251479in}{0.749694in}}%
\pgfpathcurveto{\pgfqpoint{3.243666in}{0.757508in}}{\pgfqpoint{3.233067in}{0.761898in}}{\pgfqpoint{3.222017in}{0.761898in}}%
\pgfpathcurveto{\pgfqpoint{3.210967in}{0.761898in}}{\pgfqpoint{3.200368in}{0.757508in}}{\pgfqpoint{3.192554in}{0.749694in}}%
\pgfpathcurveto{\pgfqpoint{3.184740in}{0.741881in}}{\pgfqpoint{3.180350in}{0.731281in}}{\pgfqpoint{3.180350in}{0.720231in}}%
\pgfpathcurveto{\pgfqpoint{3.180350in}{0.709181in}}{\pgfqpoint{3.184740in}{0.698582in}}{\pgfqpoint{3.192554in}{0.690769in}}%
\pgfpathcurveto{\pgfqpoint{3.200368in}{0.682955in}}{\pgfqpoint{3.210967in}{0.678565in}}{\pgfqpoint{3.222017in}{0.678565in}}%
\pgfpathclose%
\pgfusepath{stroke,fill}%
\end{pgfscope}%
\begin{pgfscope}%
\pgfpathrectangle{\pgfqpoint{0.481978in}{0.331635in}}{\pgfqpoint{4.960000in}{3.696000in}}%
\pgfusepath{clip}%
\pgfsetbuttcap%
\pgfsetroundjoin%
\definecolor{currentfill}{rgb}{1.000000,0.705882,0.509804}%
\pgfsetfillcolor{currentfill}%
\pgfsetlinewidth{0.481800pt}%
\definecolor{currentstroke}{rgb}{1.000000,1.000000,1.000000}%
\pgfsetstrokecolor{currentstroke}%
\pgfsetdash{}{0pt}%
\pgfpathmoveto{\pgfqpoint{0.707432in}{2.436040in}}%
\pgfpathcurveto{\pgfqpoint{0.718483in}{2.436040in}}{\pgfqpoint{0.729082in}{2.440430in}}{\pgfqpoint{0.736895in}{2.448243in}}%
\pgfpathcurveto{\pgfqpoint{0.744709in}{2.456057in}}{\pgfqpoint{0.749099in}{2.466656in}}{\pgfqpoint{0.749099in}{2.477706in}}%
\pgfpathcurveto{\pgfqpoint{0.749099in}{2.488756in}}{\pgfqpoint{0.744709in}{2.499355in}}{\pgfqpoint{0.736895in}{2.507169in}}%
\pgfpathcurveto{\pgfqpoint{0.729082in}{2.514983in}}{\pgfqpoint{0.718483in}{2.519373in}}{\pgfqpoint{0.707432in}{2.519373in}}%
\pgfpathcurveto{\pgfqpoint{0.696382in}{2.519373in}}{\pgfqpoint{0.685783in}{2.514983in}}{\pgfqpoint{0.677970in}{2.507169in}}%
\pgfpathcurveto{\pgfqpoint{0.670156in}{2.499355in}}{\pgfqpoint{0.665766in}{2.488756in}}{\pgfqpoint{0.665766in}{2.477706in}}%
\pgfpathcurveto{\pgfqpoint{0.665766in}{2.466656in}}{\pgfqpoint{0.670156in}{2.456057in}}{\pgfqpoint{0.677970in}{2.448243in}}%
\pgfpathcurveto{\pgfqpoint{0.685783in}{2.440430in}}{\pgfqpoint{0.696382in}{2.436040in}}{\pgfqpoint{0.707432in}{2.436040in}}%
\pgfpathclose%
\pgfusepath{stroke,fill}%
\end{pgfscope}%
\begin{pgfscope}%
\pgfpathrectangle{\pgfqpoint{0.481978in}{0.331635in}}{\pgfqpoint{4.960000in}{3.696000in}}%
\pgfusepath{clip}%
\pgfsetbuttcap%
\pgfsetroundjoin%
\definecolor{currentfill}{rgb}{1.000000,0.705882,0.509804}%
\pgfsetfillcolor{currentfill}%
\pgfsetlinewidth{0.481800pt}%
\definecolor{currentstroke}{rgb}{1.000000,1.000000,1.000000}%
\pgfsetstrokecolor{currentstroke}%
\pgfsetdash{}{0pt}%
\pgfpathmoveto{\pgfqpoint{3.262853in}{1.801662in}}%
\pgfpathcurveto{\pgfqpoint{3.273903in}{1.801662in}}{\pgfqpoint{3.284502in}{1.806052in}}{\pgfqpoint{3.292315in}{1.813865in}}%
\pgfpathcurveto{\pgfqpoint{3.300129in}{1.821679in}}{\pgfqpoint{3.304519in}{1.832278in}}{\pgfqpoint{3.304519in}{1.843328in}}%
\pgfpathcurveto{\pgfqpoint{3.304519in}{1.854378in}}{\pgfqpoint{3.300129in}{1.864977in}}{\pgfqpoint{3.292315in}{1.872791in}}%
\pgfpathcurveto{\pgfqpoint{3.284502in}{1.880605in}}{\pgfqpoint{3.273903in}{1.884995in}}{\pgfqpoint{3.262853in}{1.884995in}}%
\pgfpathcurveto{\pgfqpoint{3.251803in}{1.884995in}}{\pgfqpoint{3.241204in}{1.880605in}}{\pgfqpoint{3.233390in}{1.872791in}}%
\pgfpathcurveto{\pgfqpoint{3.225576in}{1.864977in}}{\pgfqpoint{3.221186in}{1.854378in}}{\pgfqpoint{3.221186in}{1.843328in}}%
\pgfpathcurveto{\pgfqpoint{3.221186in}{1.832278in}}{\pgfqpoint{3.225576in}{1.821679in}}{\pgfqpoint{3.233390in}{1.813865in}}%
\pgfpathcurveto{\pgfqpoint{3.241204in}{1.806052in}}{\pgfqpoint{3.251803in}{1.801662in}}{\pgfqpoint{3.262853in}{1.801662in}}%
\pgfpathclose%
\pgfusepath{stroke,fill}%
\end{pgfscope}%
\begin{pgfscope}%
\pgfpathrectangle{\pgfqpoint{0.481978in}{0.331635in}}{\pgfqpoint{4.960000in}{3.696000in}}%
\pgfusepath{clip}%
\pgfsetbuttcap%
\pgfsetroundjoin%
\definecolor{currentfill}{rgb}{1.000000,0.705882,0.509804}%
\pgfsetfillcolor{currentfill}%
\pgfsetlinewidth{0.481800pt}%
\definecolor{currentstroke}{rgb}{1.000000,1.000000,1.000000}%
\pgfsetstrokecolor{currentstroke}%
\pgfsetdash{}{0pt}%
\pgfpathmoveto{\pgfqpoint{4.437669in}{2.778285in}}%
\pgfpathcurveto{\pgfqpoint{4.448720in}{2.778285in}}{\pgfqpoint{4.459319in}{2.782675in}}{\pgfqpoint{4.467132in}{2.790489in}}%
\pgfpathcurveto{\pgfqpoint{4.474946in}{2.798303in}}{\pgfqpoint{4.479336in}{2.808902in}}{\pgfqpoint{4.479336in}{2.819952in}}%
\pgfpathcurveto{\pgfqpoint{4.479336in}{2.831002in}}{\pgfqpoint{4.474946in}{2.841601in}}{\pgfqpoint{4.467132in}{2.849415in}}%
\pgfpathcurveto{\pgfqpoint{4.459319in}{2.857228in}}{\pgfqpoint{4.448720in}{2.861619in}}{\pgfqpoint{4.437669in}{2.861619in}}%
\pgfpathcurveto{\pgfqpoint{4.426619in}{2.861619in}}{\pgfqpoint{4.416020in}{2.857228in}}{\pgfqpoint{4.408207in}{2.849415in}}%
\pgfpathcurveto{\pgfqpoint{4.400393in}{2.841601in}}{\pgfqpoint{4.396003in}{2.831002in}}{\pgfqpoint{4.396003in}{2.819952in}}%
\pgfpathcurveto{\pgfqpoint{4.396003in}{2.808902in}}{\pgfqpoint{4.400393in}{2.798303in}}{\pgfqpoint{4.408207in}{2.790489in}}%
\pgfpathcurveto{\pgfqpoint{4.416020in}{2.782675in}}{\pgfqpoint{4.426619in}{2.778285in}}{\pgfqpoint{4.437669in}{2.778285in}}%
\pgfpathclose%
\pgfusepath{stroke,fill}%
\end{pgfscope}%
\begin{pgfscope}%
\pgfpathrectangle{\pgfqpoint{0.481978in}{0.331635in}}{\pgfqpoint{4.960000in}{3.696000in}}%
\pgfusepath{clip}%
\pgfsetbuttcap%
\pgfsetroundjoin%
\definecolor{currentfill}{rgb}{1.000000,0.705882,0.509804}%
\pgfsetfillcolor{currentfill}%
\pgfsetlinewidth{0.481800pt}%
\definecolor{currentstroke}{rgb}{1.000000,1.000000,1.000000}%
\pgfsetstrokecolor{currentstroke}%
\pgfsetdash{}{0pt}%
\pgfpathmoveto{\pgfqpoint{2.059815in}{2.041778in}}%
\pgfpathcurveto{\pgfqpoint{2.070866in}{2.041778in}}{\pgfqpoint{2.081465in}{2.046169in}}{\pgfqpoint{2.089278in}{2.053982in}}%
\pgfpathcurveto{\pgfqpoint{2.097092in}{2.061796in}}{\pgfqpoint{2.101482in}{2.072395in}}{\pgfqpoint{2.101482in}{2.083445in}}%
\pgfpathcurveto{\pgfqpoint{2.101482in}{2.094495in}}{\pgfqpoint{2.097092in}{2.105094in}}{\pgfqpoint{2.089278in}{2.112908in}}%
\pgfpathcurveto{\pgfqpoint{2.081465in}{2.120722in}}{\pgfqpoint{2.070866in}{2.125112in}}{\pgfqpoint{2.059815in}{2.125112in}}%
\pgfpathcurveto{\pgfqpoint{2.048765in}{2.125112in}}{\pgfqpoint{2.038166in}{2.120722in}}{\pgfqpoint{2.030353in}{2.112908in}}%
\pgfpathcurveto{\pgfqpoint{2.022539in}{2.105094in}}{\pgfqpoint{2.018149in}{2.094495in}}{\pgfqpoint{2.018149in}{2.083445in}}%
\pgfpathcurveto{\pgfqpoint{2.018149in}{2.072395in}}{\pgfqpoint{2.022539in}{2.061796in}}{\pgfqpoint{2.030353in}{2.053982in}}%
\pgfpathcurveto{\pgfqpoint{2.038166in}{2.046169in}}{\pgfqpoint{2.048765in}{2.041778in}}{\pgfqpoint{2.059815in}{2.041778in}}%
\pgfpathclose%
\pgfusepath{stroke,fill}%
\end{pgfscope}%
\begin{pgfscope}%
\pgfpathrectangle{\pgfqpoint{0.481978in}{0.331635in}}{\pgfqpoint{4.960000in}{3.696000in}}%
\pgfusepath{clip}%
\pgfsetbuttcap%
\pgfsetroundjoin%
\definecolor{currentfill}{rgb}{1.000000,0.705882,0.509804}%
\pgfsetfillcolor{currentfill}%
\pgfsetlinewidth{0.481800pt}%
\definecolor{currentstroke}{rgb}{1.000000,1.000000,1.000000}%
\pgfsetstrokecolor{currentstroke}%
\pgfsetdash{}{0pt}%
\pgfpathmoveto{\pgfqpoint{1.667895in}{3.397630in}}%
\pgfpathcurveto{\pgfqpoint{1.678945in}{3.397630in}}{\pgfqpoint{1.689544in}{3.402020in}}{\pgfqpoint{1.697358in}{3.409834in}}%
\pgfpathcurveto{\pgfqpoint{1.705171in}{3.417647in}}{\pgfqpoint{1.709562in}{3.428246in}}{\pgfqpoint{1.709562in}{3.439296in}}%
\pgfpathcurveto{\pgfqpoint{1.709562in}{3.450347in}}{\pgfqpoint{1.705171in}{3.460946in}}{\pgfqpoint{1.697358in}{3.468759in}}%
\pgfpathcurveto{\pgfqpoint{1.689544in}{3.476573in}}{\pgfqpoint{1.678945in}{3.480963in}}{\pgfqpoint{1.667895in}{3.480963in}}%
\pgfpathcurveto{\pgfqpoint{1.656845in}{3.480963in}}{\pgfqpoint{1.646246in}{3.476573in}}{\pgfqpoint{1.638432in}{3.468759in}}%
\pgfpathcurveto{\pgfqpoint{1.630619in}{3.460946in}}{\pgfqpoint{1.626228in}{3.450347in}}{\pgfqpoint{1.626228in}{3.439296in}}%
\pgfpathcurveto{\pgfqpoint{1.626228in}{3.428246in}}{\pgfqpoint{1.630619in}{3.417647in}}{\pgfqpoint{1.638432in}{3.409834in}}%
\pgfpathcurveto{\pgfqpoint{1.646246in}{3.402020in}}{\pgfqpoint{1.656845in}{3.397630in}}{\pgfqpoint{1.667895in}{3.397630in}}%
\pgfpathclose%
\pgfusepath{stroke,fill}%
\end{pgfscope}%
\begin{pgfscope}%
\pgfpathrectangle{\pgfqpoint{0.481978in}{0.331635in}}{\pgfqpoint{4.960000in}{3.696000in}}%
\pgfusepath{clip}%
\pgfsetbuttcap%
\pgfsetroundjoin%
\definecolor{currentfill}{rgb}{1.000000,0.705882,0.509804}%
\pgfsetfillcolor{currentfill}%
\pgfsetlinewidth{0.481800pt}%
\definecolor{currentstroke}{rgb}{1.000000,1.000000,1.000000}%
\pgfsetstrokecolor{currentstroke}%
\pgfsetdash{}{0pt}%
\pgfpathmoveto{\pgfqpoint{1.976452in}{3.457671in}}%
\pgfpathcurveto{\pgfqpoint{1.987502in}{3.457671in}}{\pgfqpoint{1.998101in}{3.462061in}}{\pgfqpoint{2.005914in}{3.469875in}}%
\pgfpathcurveto{\pgfqpoint{2.013728in}{3.477689in}}{\pgfqpoint{2.018118in}{3.488288in}}{\pgfqpoint{2.018118in}{3.499338in}}%
\pgfpathcurveto{\pgfqpoint{2.018118in}{3.510388in}}{\pgfqpoint{2.013728in}{3.520987in}}{\pgfqpoint{2.005914in}{3.528801in}}%
\pgfpathcurveto{\pgfqpoint{1.998101in}{3.536614in}}{\pgfqpoint{1.987502in}{3.541005in}}{\pgfqpoint{1.976452in}{3.541005in}}%
\pgfpathcurveto{\pgfqpoint{1.965401in}{3.541005in}}{\pgfqpoint{1.954802in}{3.536614in}}{\pgfqpoint{1.946989in}{3.528801in}}%
\pgfpathcurveto{\pgfqpoint{1.939175in}{3.520987in}}{\pgfqpoint{1.934785in}{3.510388in}}{\pgfqpoint{1.934785in}{3.499338in}}%
\pgfpathcurveto{\pgfqpoint{1.934785in}{3.488288in}}{\pgfqpoint{1.939175in}{3.477689in}}{\pgfqpoint{1.946989in}{3.469875in}}%
\pgfpathcurveto{\pgfqpoint{1.954802in}{3.462061in}}{\pgfqpoint{1.965401in}{3.457671in}}{\pgfqpoint{1.976452in}{3.457671in}}%
\pgfpathclose%
\pgfusepath{stroke,fill}%
\end{pgfscope}%
\begin{pgfscope}%
\pgfpathrectangle{\pgfqpoint{0.481978in}{0.331635in}}{\pgfqpoint{4.960000in}{3.696000in}}%
\pgfusepath{clip}%
\pgfsetbuttcap%
\pgfsetroundjoin%
\definecolor{currentfill}{rgb}{1.000000,0.705882,0.509804}%
\pgfsetfillcolor{currentfill}%
\pgfsetlinewidth{0.481800pt}%
\definecolor{currentstroke}{rgb}{1.000000,1.000000,1.000000}%
\pgfsetstrokecolor{currentstroke}%
\pgfsetdash{}{0pt}%
\pgfpathmoveto{\pgfqpoint{2.471402in}{3.681877in}}%
\pgfpathcurveto{\pgfqpoint{2.482452in}{3.681877in}}{\pgfqpoint{2.493052in}{3.686268in}}{\pgfqpoint{2.500865in}{3.694081in}}%
\pgfpathcurveto{\pgfqpoint{2.508679in}{3.701895in}}{\pgfqpoint{2.513069in}{3.712494in}}{\pgfqpoint{2.513069in}{3.723544in}}%
\pgfpathcurveto{\pgfqpoint{2.513069in}{3.734594in}}{\pgfqpoint{2.508679in}{3.745193in}}{\pgfqpoint{2.500865in}{3.753007in}}%
\pgfpathcurveto{\pgfqpoint{2.493052in}{3.760820in}}{\pgfqpoint{2.482452in}{3.765211in}}{\pgfqpoint{2.471402in}{3.765211in}}%
\pgfpathcurveto{\pgfqpoint{2.460352in}{3.765211in}}{\pgfqpoint{2.449753in}{3.760820in}}{\pgfqpoint{2.441940in}{3.753007in}}%
\pgfpathcurveto{\pgfqpoint{2.434126in}{3.745193in}}{\pgfqpoint{2.429736in}{3.734594in}}{\pgfqpoint{2.429736in}{3.723544in}}%
\pgfpathcurveto{\pgfqpoint{2.429736in}{3.712494in}}{\pgfqpoint{2.434126in}{3.701895in}}{\pgfqpoint{2.441940in}{3.694081in}}%
\pgfpathcurveto{\pgfqpoint{2.449753in}{3.686268in}}{\pgfqpoint{2.460352in}{3.681877in}}{\pgfqpoint{2.471402in}{3.681877in}}%
\pgfpathclose%
\pgfusepath{stroke,fill}%
\end{pgfscope}%
\begin{pgfscope}%
\pgfpathrectangle{\pgfqpoint{0.481978in}{0.331635in}}{\pgfqpoint{4.960000in}{3.696000in}}%
\pgfusepath{clip}%
\pgfsetbuttcap%
\pgfsetroundjoin%
\definecolor{currentfill}{rgb}{1.000000,0.705882,0.509804}%
\pgfsetfillcolor{currentfill}%
\pgfsetlinewidth{0.481800pt}%
\definecolor{currentstroke}{rgb}{1.000000,1.000000,1.000000}%
\pgfsetstrokecolor{currentstroke}%
\pgfsetdash{}{0pt}%
\pgfpathmoveto{\pgfqpoint{3.142151in}{3.254916in}}%
\pgfpathcurveto{\pgfqpoint{3.153201in}{3.254916in}}{\pgfqpoint{3.163800in}{3.259306in}}{\pgfqpoint{3.171613in}{3.267120in}}%
\pgfpathcurveto{\pgfqpoint{3.179427in}{3.274934in}}{\pgfqpoint{3.183817in}{3.285533in}}{\pgfqpoint{3.183817in}{3.296583in}}%
\pgfpathcurveto{\pgfqpoint{3.183817in}{3.307633in}}{\pgfqpoint{3.179427in}{3.318232in}}{\pgfqpoint{3.171613in}{3.326046in}}%
\pgfpathcurveto{\pgfqpoint{3.163800in}{3.333859in}}{\pgfqpoint{3.153201in}{3.338249in}}{\pgfqpoint{3.142151in}{3.338249in}}%
\pgfpathcurveto{\pgfqpoint{3.131100in}{3.338249in}}{\pgfqpoint{3.120501in}{3.333859in}}{\pgfqpoint{3.112688in}{3.326046in}}%
\pgfpathcurveto{\pgfqpoint{3.104874in}{3.318232in}}{\pgfqpoint{3.100484in}{3.307633in}}{\pgfqpoint{3.100484in}{3.296583in}}%
\pgfpathcurveto{\pgfqpoint{3.100484in}{3.285533in}}{\pgfqpoint{3.104874in}{3.274934in}}{\pgfqpoint{3.112688in}{3.267120in}}%
\pgfpathcurveto{\pgfqpoint{3.120501in}{3.259306in}}{\pgfqpoint{3.131100in}{3.254916in}}{\pgfqpoint{3.142151in}{3.254916in}}%
\pgfpathclose%
\pgfusepath{stroke,fill}%
\end{pgfscope}%
\begin{pgfscope}%
\pgfpathrectangle{\pgfqpoint{0.481978in}{0.331635in}}{\pgfqpoint{4.960000in}{3.696000in}}%
\pgfusepath{clip}%
\pgfsetbuttcap%
\pgfsetroundjoin%
\definecolor{currentfill}{rgb}{1.000000,0.705882,0.509804}%
\pgfsetfillcolor{currentfill}%
\pgfsetlinewidth{0.481800pt}%
\definecolor{currentstroke}{rgb}{1.000000,1.000000,1.000000}%
\pgfsetstrokecolor{currentstroke}%
\pgfsetdash{}{0pt}%
\pgfpathmoveto{\pgfqpoint{2.296911in}{1.967206in}}%
\pgfpathcurveto{\pgfqpoint{2.307961in}{1.967206in}}{\pgfqpoint{2.318560in}{1.971597in}}{\pgfqpoint{2.326373in}{1.979410in}}%
\pgfpathcurveto{\pgfqpoint{2.334187in}{1.987224in}}{\pgfqpoint{2.338577in}{1.997823in}}{\pgfqpoint{2.338577in}{2.008873in}}%
\pgfpathcurveto{\pgfqpoint{2.338577in}{2.019923in}}{\pgfqpoint{2.334187in}{2.030522in}}{\pgfqpoint{2.326373in}{2.038336in}}%
\pgfpathcurveto{\pgfqpoint{2.318560in}{2.046149in}}{\pgfqpoint{2.307961in}{2.050540in}}{\pgfqpoint{2.296911in}{2.050540in}}%
\pgfpathcurveto{\pgfqpoint{2.285860in}{2.050540in}}{\pgfqpoint{2.275261in}{2.046149in}}{\pgfqpoint{2.267448in}{2.038336in}}%
\pgfpathcurveto{\pgfqpoint{2.259634in}{2.030522in}}{\pgfqpoint{2.255244in}{2.019923in}}{\pgfqpoint{2.255244in}{2.008873in}}%
\pgfpathcurveto{\pgfqpoint{2.255244in}{1.997823in}}{\pgfqpoint{2.259634in}{1.987224in}}{\pgfqpoint{2.267448in}{1.979410in}}%
\pgfpathcurveto{\pgfqpoint{2.275261in}{1.971597in}}{\pgfqpoint{2.285860in}{1.967206in}}{\pgfqpoint{2.296911in}{1.967206in}}%
\pgfpathclose%
\pgfusepath{stroke,fill}%
\end{pgfscope}%
\begin{pgfscope}%
\pgfpathrectangle{\pgfqpoint{0.481978in}{0.331635in}}{\pgfqpoint{4.960000in}{3.696000in}}%
\pgfusepath{clip}%
\pgfsetbuttcap%
\pgfsetroundjoin%
\definecolor{currentfill}{rgb}{1.000000,0.705882,0.509804}%
\pgfsetfillcolor{currentfill}%
\pgfsetlinewidth{0.481800pt}%
\definecolor{currentstroke}{rgb}{1.000000,1.000000,1.000000}%
\pgfsetstrokecolor{currentstroke}%
\pgfsetdash{}{0pt}%
\pgfpathmoveto{\pgfqpoint{0.967458in}{2.076347in}}%
\pgfpathcurveto{\pgfqpoint{0.978508in}{2.076347in}}{\pgfqpoint{0.989107in}{2.080737in}}{\pgfqpoint{0.996920in}{2.088550in}}%
\pgfpathcurveto{\pgfqpoint{1.004734in}{2.096364in}}{\pgfqpoint{1.009124in}{2.106963in}}{\pgfqpoint{1.009124in}{2.118013in}}%
\pgfpathcurveto{\pgfqpoint{1.009124in}{2.129063in}}{\pgfqpoint{1.004734in}{2.139662in}}{\pgfqpoint{0.996920in}{2.147476in}}%
\pgfpathcurveto{\pgfqpoint{0.989107in}{2.155290in}}{\pgfqpoint{0.978508in}{2.159680in}}{\pgfqpoint{0.967458in}{2.159680in}}%
\pgfpathcurveto{\pgfqpoint{0.956407in}{2.159680in}}{\pgfqpoint{0.945808in}{2.155290in}}{\pgfqpoint{0.937995in}{2.147476in}}%
\pgfpathcurveto{\pgfqpoint{0.930181in}{2.139662in}}{\pgfqpoint{0.925791in}{2.129063in}}{\pgfqpoint{0.925791in}{2.118013in}}%
\pgfpathcurveto{\pgfqpoint{0.925791in}{2.106963in}}{\pgfqpoint{0.930181in}{2.096364in}}{\pgfqpoint{0.937995in}{2.088550in}}%
\pgfpathcurveto{\pgfqpoint{0.945808in}{2.080737in}}{\pgfqpoint{0.956407in}{2.076347in}}{\pgfqpoint{0.967458in}{2.076347in}}%
\pgfpathclose%
\pgfusepath{stroke,fill}%
\end{pgfscope}%
\begin{pgfscope}%
\pgfpathrectangle{\pgfqpoint{0.481978in}{0.331635in}}{\pgfqpoint{4.960000in}{3.696000in}}%
\pgfusepath{clip}%
\pgfsetbuttcap%
\pgfsetroundjoin%
\definecolor{currentfill}{rgb}{1.000000,0.705882,0.509804}%
\pgfsetfillcolor{currentfill}%
\pgfsetlinewidth{0.481800pt}%
\definecolor{currentstroke}{rgb}{1.000000,1.000000,1.000000}%
\pgfsetstrokecolor{currentstroke}%
\pgfsetdash{}{0pt}%
\pgfpathmoveto{\pgfqpoint{3.223478in}{3.004135in}}%
\pgfpathcurveto{\pgfqpoint{3.234528in}{3.004135in}}{\pgfqpoint{3.245127in}{3.008525in}}{\pgfqpoint{3.252941in}{3.016339in}}%
\pgfpathcurveto{\pgfqpoint{3.260754in}{3.024153in}}{\pgfqpoint{3.265144in}{3.034752in}}{\pgfqpoint{3.265144in}{3.045802in}}%
\pgfpathcurveto{\pgfqpoint{3.265144in}{3.056852in}}{\pgfqpoint{3.260754in}{3.067451in}}{\pgfqpoint{3.252941in}{3.075265in}}%
\pgfpathcurveto{\pgfqpoint{3.245127in}{3.083078in}}{\pgfqpoint{3.234528in}{3.087468in}}{\pgfqpoint{3.223478in}{3.087468in}}%
\pgfpathcurveto{\pgfqpoint{3.212428in}{3.087468in}}{\pgfqpoint{3.201829in}{3.083078in}}{\pgfqpoint{3.194015in}{3.075265in}}%
\pgfpathcurveto{\pgfqpoint{3.186201in}{3.067451in}}{\pgfqpoint{3.181811in}{3.056852in}}{\pgfqpoint{3.181811in}{3.045802in}}%
\pgfpathcurveto{\pgfqpoint{3.181811in}{3.034752in}}{\pgfqpoint{3.186201in}{3.024153in}}{\pgfqpoint{3.194015in}{3.016339in}}%
\pgfpathcurveto{\pgfqpoint{3.201829in}{3.008525in}}{\pgfqpoint{3.212428in}{3.004135in}}{\pgfqpoint{3.223478in}{3.004135in}}%
\pgfpathclose%
\pgfusepath{stroke,fill}%
\end{pgfscope}%
\begin{pgfscope}%
\pgfpathrectangle{\pgfqpoint{0.481978in}{0.331635in}}{\pgfqpoint{4.960000in}{3.696000in}}%
\pgfusepath{clip}%
\pgfsetbuttcap%
\pgfsetroundjoin%
\definecolor{currentfill}{rgb}{1.000000,0.705882,0.509804}%
\pgfsetfillcolor{currentfill}%
\pgfsetlinewidth{0.481800pt}%
\definecolor{currentstroke}{rgb}{1.000000,1.000000,1.000000}%
\pgfsetstrokecolor{currentstroke}%
\pgfsetdash{}{0pt}%
\pgfpathmoveto{\pgfqpoint{3.997789in}{2.608530in}}%
\pgfpathcurveto{\pgfqpoint{4.008839in}{2.608530in}}{\pgfqpoint{4.019438in}{2.612921in}}{\pgfqpoint{4.027252in}{2.620734in}}%
\pgfpathcurveto{\pgfqpoint{4.035065in}{2.628548in}}{\pgfqpoint{4.039456in}{2.639147in}}{\pgfqpoint{4.039456in}{2.650197in}}%
\pgfpathcurveto{\pgfqpoint{4.039456in}{2.661247in}}{\pgfqpoint{4.035065in}{2.671846in}}{\pgfqpoint{4.027252in}{2.679660in}}%
\pgfpathcurveto{\pgfqpoint{4.019438in}{2.687473in}}{\pgfqpoint{4.008839in}{2.691864in}}{\pgfqpoint{3.997789in}{2.691864in}}%
\pgfpathcurveto{\pgfqpoint{3.986739in}{2.691864in}}{\pgfqpoint{3.976140in}{2.687473in}}{\pgfqpoint{3.968326in}{2.679660in}}%
\pgfpathcurveto{\pgfqpoint{3.960513in}{2.671846in}}{\pgfqpoint{3.956122in}{2.661247in}}{\pgfqpoint{3.956122in}{2.650197in}}%
\pgfpathcurveto{\pgfqpoint{3.956122in}{2.639147in}}{\pgfqpoint{3.960513in}{2.628548in}}{\pgfqpoint{3.968326in}{2.620734in}}%
\pgfpathcurveto{\pgfqpoint{3.976140in}{2.612921in}}{\pgfqpoint{3.986739in}{2.608530in}}{\pgfqpoint{3.997789in}{2.608530in}}%
\pgfpathclose%
\pgfusepath{stroke,fill}%
\end{pgfscope}%
\begin{pgfscope}%
\pgfpathrectangle{\pgfqpoint{0.481978in}{0.331635in}}{\pgfqpoint{4.960000in}{3.696000in}}%
\pgfusepath{clip}%
\pgfsetbuttcap%
\pgfsetroundjoin%
\definecolor{currentfill}{rgb}{1.000000,0.705882,0.509804}%
\pgfsetfillcolor{currentfill}%
\pgfsetlinewidth{0.481800pt}%
\definecolor{currentstroke}{rgb}{1.000000,1.000000,1.000000}%
\pgfsetstrokecolor{currentstroke}%
\pgfsetdash{}{0pt}%
\pgfpathmoveto{\pgfqpoint{3.630700in}{2.501288in}}%
\pgfpathcurveto{\pgfqpoint{3.641750in}{2.501288in}}{\pgfqpoint{3.652349in}{2.505678in}}{\pgfqpoint{3.660162in}{2.513492in}}%
\pgfpathcurveto{\pgfqpoint{3.667976in}{2.521306in}}{\pgfqpoint{3.672366in}{2.531905in}}{\pgfqpoint{3.672366in}{2.542955in}}%
\pgfpathcurveto{\pgfqpoint{3.672366in}{2.554005in}}{\pgfqpoint{3.667976in}{2.564604in}}{\pgfqpoint{3.660162in}{2.572417in}}%
\pgfpathcurveto{\pgfqpoint{3.652349in}{2.580231in}}{\pgfqpoint{3.641750in}{2.584621in}}{\pgfqpoint{3.630700in}{2.584621in}}%
\pgfpathcurveto{\pgfqpoint{3.619650in}{2.584621in}}{\pgfqpoint{3.609050in}{2.580231in}}{\pgfqpoint{3.601237in}{2.572417in}}%
\pgfpathcurveto{\pgfqpoint{3.593423in}{2.564604in}}{\pgfqpoint{3.589033in}{2.554005in}}{\pgfqpoint{3.589033in}{2.542955in}}%
\pgfpathcurveto{\pgfqpoint{3.589033in}{2.531905in}}{\pgfqpoint{3.593423in}{2.521306in}}{\pgfqpoint{3.601237in}{2.513492in}}%
\pgfpathcurveto{\pgfqpoint{3.609050in}{2.505678in}}{\pgfqpoint{3.619650in}{2.501288in}}{\pgfqpoint{3.630700in}{2.501288in}}%
\pgfpathclose%
\pgfusepath{stroke,fill}%
\end{pgfscope}%
\begin{pgfscope}%
\pgfpathrectangle{\pgfqpoint{0.481978in}{0.331635in}}{\pgfqpoint{4.960000in}{3.696000in}}%
\pgfusepath{clip}%
\pgfsetbuttcap%
\pgfsetroundjoin%
\definecolor{currentfill}{rgb}{1.000000,0.705882,0.509804}%
\pgfsetfillcolor{currentfill}%
\pgfsetlinewidth{0.481800pt}%
\definecolor{currentstroke}{rgb}{1.000000,1.000000,1.000000}%
\pgfsetstrokecolor{currentstroke}%
\pgfsetdash{}{0pt}%
\pgfpathmoveto{\pgfqpoint{3.192223in}{3.329042in}}%
\pgfpathcurveto{\pgfqpoint{3.203273in}{3.329042in}}{\pgfqpoint{3.213872in}{3.333432in}}{\pgfqpoint{3.221685in}{3.341246in}}%
\pgfpathcurveto{\pgfqpoint{3.229499in}{3.349060in}}{\pgfqpoint{3.233889in}{3.359659in}}{\pgfqpoint{3.233889in}{3.370709in}}%
\pgfpathcurveto{\pgfqpoint{3.233889in}{3.381759in}}{\pgfqpoint{3.229499in}{3.392358in}}{\pgfqpoint{3.221685in}{3.400172in}}%
\pgfpathcurveto{\pgfqpoint{3.213872in}{3.407985in}}{\pgfqpoint{3.203273in}{3.412375in}}{\pgfqpoint{3.192223in}{3.412375in}}%
\pgfpathcurveto{\pgfqpoint{3.181172in}{3.412375in}}{\pgfqpoint{3.170573in}{3.407985in}}{\pgfqpoint{3.162760in}{3.400172in}}%
\pgfpathcurveto{\pgfqpoint{3.154946in}{3.392358in}}{\pgfqpoint{3.150556in}{3.381759in}}{\pgfqpoint{3.150556in}{3.370709in}}%
\pgfpathcurveto{\pgfqpoint{3.150556in}{3.359659in}}{\pgfqpoint{3.154946in}{3.349060in}}{\pgfqpoint{3.162760in}{3.341246in}}%
\pgfpathcurveto{\pgfqpoint{3.170573in}{3.333432in}}{\pgfqpoint{3.181172in}{3.329042in}}{\pgfqpoint{3.192223in}{3.329042in}}%
\pgfpathclose%
\pgfusepath{stroke,fill}%
\end{pgfscope}%
\begin{pgfscope}%
\pgfpathrectangle{\pgfqpoint{0.481978in}{0.331635in}}{\pgfqpoint{4.960000in}{3.696000in}}%
\pgfusepath{clip}%
\pgfsetbuttcap%
\pgfsetroundjoin%
\definecolor{currentfill}{rgb}{1.000000,0.705882,0.509804}%
\pgfsetfillcolor{currentfill}%
\pgfsetlinewidth{0.481800pt}%
\definecolor{currentstroke}{rgb}{1.000000,1.000000,1.000000}%
\pgfsetstrokecolor{currentstroke}%
\pgfsetdash{}{0pt}%
\pgfpathmoveto{\pgfqpoint{2.811679in}{1.472289in}}%
\pgfpathcurveto{\pgfqpoint{2.822729in}{1.472289in}}{\pgfqpoint{2.833328in}{1.476679in}}{\pgfqpoint{2.841142in}{1.484493in}}%
\pgfpathcurveto{\pgfqpoint{2.848955in}{1.492306in}}{\pgfqpoint{2.853346in}{1.502905in}}{\pgfqpoint{2.853346in}{1.513956in}}%
\pgfpathcurveto{\pgfqpoint{2.853346in}{1.525006in}}{\pgfqpoint{2.848955in}{1.535605in}}{\pgfqpoint{2.841142in}{1.543418in}}%
\pgfpathcurveto{\pgfqpoint{2.833328in}{1.551232in}}{\pgfqpoint{2.822729in}{1.555622in}}{\pgfqpoint{2.811679in}{1.555622in}}%
\pgfpathcurveto{\pgfqpoint{2.800629in}{1.555622in}}{\pgfqpoint{2.790030in}{1.551232in}}{\pgfqpoint{2.782216in}{1.543418in}}%
\pgfpathcurveto{\pgfqpoint{2.774402in}{1.535605in}}{\pgfqpoint{2.770012in}{1.525006in}}{\pgfqpoint{2.770012in}{1.513956in}}%
\pgfpathcurveto{\pgfqpoint{2.770012in}{1.502905in}}{\pgfqpoint{2.774402in}{1.492306in}}{\pgfqpoint{2.782216in}{1.484493in}}%
\pgfpathcurveto{\pgfqpoint{2.790030in}{1.476679in}}{\pgfqpoint{2.800629in}{1.472289in}}{\pgfqpoint{2.811679in}{1.472289in}}%
\pgfpathclose%
\pgfusepath{stroke,fill}%
\end{pgfscope}%
\begin{pgfscope}%
\pgfpathrectangle{\pgfqpoint{0.481978in}{0.331635in}}{\pgfqpoint{4.960000in}{3.696000in}}%
\pgfusepath{clip}%
\pgfsetbuttcap%
\pgfsetroundjoin%
\definecolor{currentfill}{rgb}{1.000000,0.705882,0.509804}%
\pgfsetfillcolor{currentfill}%
\pgfsetlinewidth{0.481800pt}%
\definecolor{currentstroke}{rgb}{1.000000,1.000000,1.000000}%
\pgfsetstrokecolor{currentstroke}%
\pgfsetdash{}{0pt}%
\pgfpathmoveto{\pgfqpoint{2.645919in}{1.837066in}}%
\pgfpathcurveto{\pgfqpoint{2.656969in}{1.837066in}}{\pgfqpoint{2.667568in}{1.841457in}}{\pgfqpoint{2.675382in}{1.849270in}}%
\pgfpathcurveto{\pgfqpoint{2.683196in}{1.857084in}}{\pgfqpoint{2.687586in}{1.867683in}}{\pgfqpoint{2.687586in}{1.878733in}}%
\pgfpathcurveto{\pgfqpoint{2.687586in}{1.889783in}}{\pgfqpoint{2.683196in}{1.900382in}}{\pgfqpoint{2.675382in}{1.908196in}}%
\pgfpathcurveto{\pgfqpoint{2.667568in}{1.916009in}}{\pgfqpoint{2.656969in}{1.920400in}}{\pgfqpoint{2.645919in}{1.920400in}}%
\pgfpathcurveto{\pgfqpoint{2.634869in}{1.920400in}}{\pgfqpoint{2.624270in}{1.916009in}}{\pgfqpoint{2.616456in}{1.908196in}}%
\pgfpathcurveto{\pgfqpoint{2.608643in}{1.900382in}}{\pgfqpoint{2.604252in}{1.889783in}}{\pgfqpoint{2.604252in}{1.878733in}}%
\pgfpathcurveto{\pgfqpoint{2.604252in}{1.867683in}}{\pgfqpoint{2.608643in}{1.857084in}}{\pgfqpoint{2.616456in}{1.849270in}}%
\pgfpathcurveto{\pgfqpoint{2.624270in}{1.841457in}}{\pgfqpoint{2.634869in}{1.837066in}}{\pgfqpoint{2.645919in}{1.837066in}}%
\pgfpathclose%
\pgfusepath{stroke,fill}%
\end{pgfscope}%
\begin{pgfscope}%
\pgfpathrectangle{\pgfqpoint{0.481978in}{0.331635in}}{\pgfqpoint{4.960000in}{3.696000in}}%
\pgfusepath{clip}%
\pgfsetbuttcap%
\pgfsetroundjoin%
\definecolor{currentfill}{rgb}{1.000000,0.705882,0.509804}%
\pgfsetfillcolor{currentfill}%
\pgfsetlinewidth{0.481800pt}%
\definecolor{currentstroke}{rgb}{1.000000,1.000000,1.000000}%
\pgfsetstrokecolor{currentstroke}%
\pgfsetdash{}{0pt}%
\pgfpathmoveto{\pgfqpoint{4.106333in}{2.702347in}}%
\pgfpathcurveto{\pgfqpoint{4.117384in}{2.702347in}}{\pgfqpoint{4.127983in}{2.706737in}}{\pgfqpoint{4.135796in}{2.714551in}}%
\pgfpathcurveto{\pgfqpoint{4.143610in}{2.722364in}}{\pgfqpoint{4.148000in}{2.732963in}}{\pgfqpoint{4.148000in}{2.744013in}}%
\pgfpathcurveto{\pgfqpoint{4.148000in}{2.755064in}}{\pgfqpoint{4.143610in}{2.765663in}}{\pgfqpoint{4.135796in}{2.773476in}}%
\pgfpathcurveto{\pgfqpoint{4.127983in}{2.781290in}}{\pgfqpoint{4.117384in}{2.785680in}}{\pgfqpoint{4.106333in}{2.785680in}}%
\pgfpathcurveto{\pgfqpoint{4.095283in}{2.785680in}}{\pgfqpoint{4.084684in}{2.781290in}}{\pgfqpoint{4.076871in}{2.773476in}}%
\pgfpathcurveto{\pgfqpoint{4.069057in}{2.765663in}}{\pgfqpoint{4.064667in}{2.755064in}}{\pgfqpoint{4.064667in}{2.744013in}}%
\pgfpathcurveto{\pgfqpoint{4.064667in}{2.732963in}}{\pgfqpoint{4.069057in}{2.722364in}}{\pgfqpoint{4.076871in}{2.714551in}}%
\pgfpathcurveto{\pgfqpoint{4.084684in}{2.706737in}}{\pgfqpoint{4.095283in}{2.702347in}}{\pgfqpoint{4.106333in}{2.702347in}}%
\pgfpathclose%
\pgfusepath{stroke,fill}%
\end{pgfscope}%
\begin{pgfscope}%
\pgfpathrectangle{\pgfqpoint{0.481978in}{0.331635in}}{\pgfqpoint{4.960000in}{3.696000in}}%
\pgfusepath{clip}%
\pgfsetbuttcap%
\pgfsetroundjoin%
\definecolor{currentfill}{rgb}{1.000000,0.705882,0.509804}%
\pgfsetfillcolor{currentfill}%
\pgfsetlinewidth{0.481800pt}%
\definecolor{currentstroke}{rgb}{1.000000,1.000000,1.000000}%
\pgfsetstrokecolor{currentstroke}%
\pgfsetdash{}{0pt}%
\pgfpathmoveto{\pgfqpoint{3.981935in}{1.088723in}}%
\pgfpathcurveto{\pgfqpoint{3.992985in}{1.088723in}}{\pgfqpoint{4.003584in}{1.093114in}}{\pgfqpoint{4.011397in}{1.100927in}}%
\pgfpathcurveto{\pgfqpoint{4.019211in}{1.108741in}}{\pgfqpoint{4.023601in}{1.119340in}}{\pgfqpoint{4.023601in}{1.130390in}}%
\pgfpathcurveto{\pgfqpoint{4.023601in}{1.141440in}}{\pgfqpoint{4.019211in}{1.152039in}}{\pgfqpoint{4.011397in}{1.159853in}}%
\pgfpathcurveto{\pgfqpoint{4.003584in}{1.167666in}}{\pgfqpoint{3.992985in}{1.172057in}}{\pgfqpoint{3.981935in}{1.172057in}}%
\pgfpathcurveto{\pgfqpoint{3.970884in}{1.172057in}}{\pgfqpoint{3.960285in}{1.167666in}}{\pgfqpoint{3.952472in}{1.159853in}}%
\pgfpathcurveto{\pgfqpoint{3.944658in}{1.152039in}}{\pgfqpoint{3.940268in}{1.141440in}}{\pgfqpoint{3.940268in}{1.130390in}}%
\pgfpathcurveto{\pgfqpoint{3.940268in}{1.119340in}}{\pgfqpoint{3.944658in}{1.108741in}}{\pgfqpoint{3.952472in}{1.100927in}}%
\pgfpathcurveto{\pgfqpoint{3.960285in}{1.093114in}}{\pgfqpoint{3.970884in}{1.088723in}}{\pgfqpoint{3.981935in}{1.088723in}}%
\pgfpathclose%
\pgfusepath{stroke,fill}%
\end{pgfscope}%
\begin{pgfscope}%
\pgfpathrectangle{\pgfqpoint{0.481978in}{0.331635in}}{\pgfqpoint{4.960000in}{3.696000in}}%
\pgfusepath{clip}%
\pgfsetbuttcap%
\pgfsetroundjoin%
\definecolor{currentfill}{rgb}{1.000000,0.705882,0.509804}%
\pgfsetfillcolor{currentfill}%
\pgfsetlinewidth{0.481800pt}%
\definecolor{currentstroke}{rgb}{1.000000,1.000000,1.000000}%
\pgfsetstrokecolor{currentstroke}%
\pgfsetdash{}{0pt}%
\pgfpathmoveto{\pgfqpoint{3.806749in}{0.595278in}}%
\pgfpathcurveto{\pgfqpoint{3.817799in}{0.595278in}}{\pgfqpoint{3.828398in}{0.599668in}}{\pgfqpoint{3.836211in}{0.607482in}}%
\pgfpathcurveto{\pgfqpoint{3.844025in}{0.615296in}}{\pgfqpoint{3.848415in}{0.625895in}}{\pgfqpoint{3.848415in}{0.636945in}}%
\pgfpathcurveto{\pgfqpoint{3.848415in}{0.647995in}}{\pgfqpoint{3.844025in}{0.658594in}}{\pgfqpoint{3.836211in}{0.666408in}}%
\pgfpathcurveto{\pgfqpoint{3.828398in}{0.674221in}}{\pgfqpoint{3.817799in}{0.678612in}}{\pgfqpoint{3.806749in}{0.678612in}}%
\pgfpathcurveto{\pgfqpoint{3.795699in}{0.678612in}}{\pgfqpoint{3.785100in}{0.674221in}}{\pgfqpoint{3.777286in}{0.666408in}}%
\pgfpathcurveto{\pgfqpoint{3.769472in}{0.658594in}}{\pgfqpoint{3.765082in}{0.647995in}}{\pgfqpoint{3.765082in}{0.636945in}}%
\pgfpathcurveto{\pgfqpoint{3.765082in}{0.625895in}}{\pgfqpoint{3.769472in}{0.615296in}}{\pgfqpoint{3.777286in}{0.607482in}}%
\pgfpathcurveto{\pgfqpoint{3.785100in}{0.599668in}}{\pgfqpoint{3.795699in}{0.595278in}}{\pgfqpoint{3.806749in}{0.595278in}}%
\pgfpathclose%
\pgfusepath{stroke,fill}%
\end{pgfscope}%
\begin{pgfscope}%
\pgfpathrectangle{\pgfqpoint{0.481978in}{0.331635in}}{\pgfqpoint{4.960000in}{3.696000in}}%
\pgfusepath{clip}%
\pgfsetbuttcap%
\pgfsetroundjoin%
\definecolor{currentfill}{rgb}{1.000000,0.705882,0.509804}%
\pgfsetfillcolor{currentfill}%
\pgfsetlinewidth{0.481800pt}%
\definecolor{currentstroke}{rgb}{1.000000,1.000000,1.000000}%
\pgfsetstrokecolor{currentstroke}%
\pgfsetdash{}{0pt}%
\pgfpathmoveto{\pgfqpoint{4.230982in}{2.526749in}}%
\pgfpathcurveto{\pgfqpoint{4.242032in}{2.526749in}}{\pgfqpoint{4.252631in}{2.531140in}}{\pgfqpoint{4.260444in}{2.538953in}}%
\pgfpathcurveto{\pgfqpoint{4.268258in}{2.546767in}}{\pgfqpoint{4.272648in}{2.557366in}}{\pgfqpoint{4.272648in}{2.568416in}}%
\pgfpathcurveto{\pgfqpoint{4.272648in}{2.579466in}}{\pgfqpoint{4.268258in}{2.590065in}}{\pgfqpoint{4.260444in}{2.597879in}}%
\pgfpathcurveto{\pgfqpoint{4.252631in}{2.605693in}}{\pgfqpoint{4.242032in}{2.610083in}}{\pgfqpoint{4.230982in}{2.610083in}}%
\pgfpathcurveto{\pgfqpoint{4.219932in}{2.610083in}}{\pgfqpoint{4.209333in}{2.605693in}}{\pgfqpoint{4.201519in}{2.597879in}}%
\pgfpathcurveto{\pgfqpoint{4.193705in}{2.590065in}}{\pgfqpoint{4.189315in}{2.579466in}}{\pgfqpoint{4.189315in}{2.568416in}}%
\pgfpathcurveto{\pgfqpoint{4.189315in}{2.557366in}}{\pgfqpoint{4.193705in}{2.546767in}}{\pgfqpoint{4.201519in}{2.538953in}}%
\pgfpathcurveto{\pgfqpoint{4.209333in}{2.531140in}}{\pgfqpoint{4.219932in}{2.526749in}}{\pgfqpoint{4.230982in}{2.526749in}}%
\pgfpathclose%
\pgfusepath{stroke,fill}%
\end{pgfscope}%
\begin{pgfscope}%
\pgfpathrectangle{\pgfqpoint{0.481978in}{0.331635in}}{\pgfqpoint{4.960000in}{3.696000in}}%
\pgfusepath{clip}%
\pgfsetbuttcap%
\pgfsetroundjoin%
\definecolor{currentfill}{rgb}{1.000000,0.705882,0.509804}%
\pgfsetfillcolor{currentfill}%
\pgfsetlinewidth{0.481800pt}%
\definecolor{currentstroke}{rgb}{1.000000,1.000000,1.000000}%
\pgfsetstrokecolor{currentstroke}%
\pgfsetdash{}{0pt}%
\pgfpathmoveto{\pgfqpoint{3.956699in}{3.264325in}}%
\pgfpathcurveto{\pgfqpoint{3.967749in}{3.264325in}}{\pgfqpoint{3.978348in}{3.268715in}}{\pgfqpoint{3.986162in}{3.276529in}}%
\pgfpathcurveto{\pgfqpoint{3.993976in}{3.284342in}}{\pgfqpoint{3.998366in}{3.294941in}}{\pgfqpoint{3.998366in}{3.305992in}}%
\pgfpathcurveto{\pgfqpoint{3.998366in}{3.317042in}}{\pgfqpoint{3.993976in}{3.327641in}}{\pgfqpoint{3.986162in}{3.335454in}}%
\pgfpathcurveto{\pgfqpoint{3.978348in}{3.343268in}}{\pgfqpoint{3.967749in}{3.347658in}}{\pgfqpoint{3.956699in}{3.347658in}}%
\pgfpathcurveto{\pgfqpoint{3.945649in}{3.347658in}}{\pgfqpoint{3.935050in}{3.343268in}}{\pgfqpoint{3.927236in}{3.335454in}}%
\pgfpathcurveto{\pgfqpoint{3.919423in}{3.327641in}}{\pgfqpoint{3.915033in}{3.317042in}}{\pgfqpoint{3.915033in}{3.305992in}}%
\pgfpathcurveto{\pgfqpoint{3.915033in}{3.294941in}}{\pgfqpoint{3.919423in}{3.284342in}}{\pgfqpoint{3.927236in}{3.276529in}}%
\pgfpathcurveto{\pgfqpoint{3.935050in}{3.268715in}}{\pgfqpoint{3.945649in}{3.264325in}}{\pgfqpoint{3.956699in}{3.264325in}}%
\pgfpathclose%
\pgfusepath{stroke,fill}%
\end{pgfscope}%
\begin{pgfscope}%
\pgfpathrectangle{\pgfqpoint{0.481978in}{0.331635in}}{\pgfqpoint{4.960000in}{3.696000in}}%
\pgfusepath{clip}%
\pgfsetbuttcap%
\pgfsetroundjoin%
\definecolor{currentfill}{rgb}{1.000000,0.705882,0.509804}%
\pgfsetfillcolor{currentfill}%
\pgfsetlinewidth{0.481800pt}%
\definecolor{currentstroke}{rgb}{1.000000,1.000000,1.000000}%
\pgfsetstrokecolor{currentstroke}%
\pgfsetdash{}{0pt}%
\pgfpathmoveto{\pgfqpoint{0.924460in}{1.702200in}}%
\pgfpathcurveto{\pgfqpoint{0.935510in}{1.702200in}}{\pgfqpoint{0.946109in}{1.706590in}}{\pgfqpoint{0.953923in}{1.714404in}}%
\pgfpathcurveto{\pgfqpoint{0.961736in}{1.722217in}}{\pgfqpoint{0.966127in}{1.732816in}}{\pgfqpoint{0.966127in}{1.743866in}}%
\pgfpathcurveto{\pgfqpoint{0.966127in}{1.754916in}}{\pgfqpoint{0.961736in}{1.765515in}}{\pgfqpoint{0.953923in}{1.773329in}}%
\pgfpathcurveto{\pgfqpoint{0.946109in}{1.781143in}}{\pgfqpoint{0.935510in}{1.785533in}}{\pgfqpoint{0.924460in}{1.785533in}}%
\pgfpathcurveto{\pgfqpoint{0.913410in}{1.785533in}}{\pgfqpoint{0.902811in}{1.781143in}}{\pgfqpoint{0.894997in}{1.773329in}}%
\pgfpathcurveto{\pgfqpoint{0.887184in}{1.765515in}}{\pgfqpoint{0.882793in}{1.754916in}}{\pgfqpoint{0.882793in}{1.743866in}}%
\pgfpathcurveto{\pgfqpoint{0.882793in}{1.732816in}}{\pgfqpoint{0.887184in}{1.722217in}}{\pgfqpoint{0.894997in}{1.714404in}}%
\pgfpathcurveto{\pgfqpoint{0.902811in}{1.706590in}}{\pgfqpoint{0.913410in}{1.702200in}}{\pgfqpoint{0.924460in}{1.702200in}}%
\pgfpathclose%
\pgfusepath{stroke,fill}%
\end{pgfscope}%
\begin{pgfscope}%
\pgfpathrectangle{\pgfqpoint{0.481978in}{0.331635in}}{\pgfqpoint{4.960000in}{3.696000in}}%
\pgfusepath{clip}%
\pgfsetbuttcap%
\pgfsetroundjoin%
\definecolor{currentfill}{rgb}{1.000000,0.705882,0.509804}%
\pgfsetfillcolor{currentfill}%
\pgfsetlinewidth{0.481800pt}%
\definecolor{currentstroke}{rgb}{1.000000,1.000000,1.000000}%
\pgfsetstrokecolor{currentstroke}%
\pgfsetdash{}{0pt}%
\pgfpathmoveto{\pgfqpoint{3.140005in}{0.948151in}}%
\pgfpathcurveto{\pgfqpoint{3.151056in}{0.948151in}}{\pgfqpoint{3.161655in}{0.952541in}}{\pgfqpoint{3.169468in}{0.960355in}}%
\pgfpathcurveto{\pgfqpoint{3.177282in}{0.968168in}}{\pgfqpoint{3.181672in}{0.978767in}}{\pgfqpoint{3.181672in}{0.989817in}}%
\pgfpathcurveto{\pgfqpoint{3.181672in}{1.000868in}}{\pgfqpoint{3.177282in}{1.011467in}}{\pgfqpoint{3.169468in}{1.019280in}}%
\pgfpathcurveto{\pgfqpoint{3.161655in}{1.027094in}}{\pgfqpoint{3.151056in}{1.031484in}}{\pgfqpoint{3.140005in}{1.031484in}}%
\pgfpathcurveto{\pgfqpoint{3.128955in}{1.031484in}}{\pgfqpoint{3.118356in}{1.027094in}}{\pgfqpoint{3.110543in}{1.019280in}}%
\pgfpathcurveto{\pgfqpoint{3.102729in}{1.011467in}}{\pgfqpoint{3.098339in}{1.000868in}}{\pgfqpoint{3.098339in}{0.989817in}}%
\pgfpathcurveto{\pgfqpoint{3.098339in}{0.978767in}}{\pgfqpoint{3.102729in}{0.968168in}}{\pgfqpoint{3.110543in}{0.960355in}}%
\pgfpathcurveto{\pgfqpoint{3.118356in}{0.952541in}}{\pgfqpoint{3.128955in}{0.948151in}}{\pgfqpoint{3.140005in}{0.948151in}}%
\pgfpathclose%
\pgfusepath{stroke,fill}%
\end{pgfscope}%
\begin{pgfscope}%
\pgfpathrectangle{\pgfqpoint{0.481978in}{0.331635in}}{\pgfqpoint{4.960000in}{3.696000in}}%
\pgfusepath{clip}%
\pgfsetbuttcap%
\pgfsetroundjoin%
\definecolor{currentfill}{rgb}{1.000000,0.705882,0.509804}%
\pgfsetfillcolor{currentfill}%
\pgfsetlinewidth{0.481800pt}%
\definecolor{currentstroke}{rgb}{1.000000,1.000000,1.000000}%
\pgfsetstrokecolor{currentstroke}%
\pgfsetdash{}{0pt}%
\pgfpathmoveto{\pgfqpoint{2.463528in}{0.888754in}}%
\pgfpathcurveto{\pgfqpoint{2.474578in}{0.888754in}}{\pgfqpoint{2.485177in}{0.893144in}}{\pgfqpoint{2.492990in}{0.900958in}}%
\pgfpathcurveto{\pgfqpoint{2.500804in}{0.908771in}}{\pgfqpoint{2.505194in}{0.919370in}}{\pgfqpoint{2.505194in}{0.930421in}}%
\pgfpathcurveto{\pgfqpoint{2.505194in}{0.941471in}}{\pgfqpoint{2.500804in}{0.952070in}}{\pgfqpoint{2.492990in}{0.959883in}}%
\pgfpathcurveto{\pgfqpoint{2.485177in}{0.967697in}}{\pgfqpoint{2.474578in}{0.972087in}}{\pgfqpoint{2.463528in}{0.972087in}}%
\pgfpathcurveto{\pgfqpoint{2.452478in}{0.972087in}}{\pgfqpoint{2.441879in}{0.967697in}}{\pgfqpoint{2.434065in}{0.959883in}}%
\pgfpathcurveto{\pgfqpoint{2.426251in}{0.952070in}}{\pgfqpoint{2.421861in}{0.941471in}}{\pgfqpoint{2.421861in}{0.930421in}}%
\pgfpathcurveto{\pgfqpoint{2.421861in}{0.919370in}}{\pgfqpoint{2.426251in}{0.908771in}}{\pgfqpoint{2.434065in}{0.900958in}}%
\pgfpathcurveto{\pgfqpoint{2.441879in}{0.893144in}}{\pgfqpoint{2.452478in}{0.888754in}}{\pgfqpoint{2.463528in}{0.888754in}}%
\pgfpathclose%
\pgfusepath{stroke,fill}%
\end{pgfscope}%
\begin{pgfscope}%
\pgfpathrectangle{\pgfqpoint{0.481978in}{0.331635in}}{\pgfqpoint{4.960000in}{3.696000in}}%
\pgfusepath{clip}%
\pgfsetbuttcap%
\pgfsetroundjoin%
\definecolor{currentfill}{rgb}{1.000000,0.705882,0.509804}%
\pgfsetfillcolor{currentfill}%
\pgfsetlinewidth{0.481800pt}%
\definecolor{currentstroke}{rgb}{1.000000,1.000000,1.000000}%
\pgfsetstrokecolor{currentstroke}%
\pgfsetdash{}{0pt}%
\pgfpathmoveto{\pgfqpoint{1.980453in}{1.236887in}}%
\pgfpathcurveto{\pgfqpoint{1.991503in}{1.236887in}}{\pgfqpoint{2.002102in}{1.241278in}}{\pgfqpoint{2.009916in}{1.249091in}}%
\pgfpathcurveto{\pgfqpoint{2.017730in}{1.256905in}}{\pgfqpoint{2.022120in}{1.267504in}}{\pgfqpoint{2.022120in}{1.278554in}}%
\pgfpathcurveto{\pgfqpoint{2.022120in}{1.289604in}}{\pgfqpoint{2.017730in}{1.300203in}}{\pgfqpoint{2.009916in}{1.308017in}}%
\pgfpathcurveto{\pgfqpoint{2.002102in}{1.315830in}}{\pgfqpoint{1.991503in}{1.320221in}}{\pgfqpoint{1.980453in}{1.320221in}}%
\pgfpathcurveto{\pgfqpoint{1.969403in}{1.320221in}}{\pgfqpoint{1.958804in}{1.315830in}}{\pgfqpoint{1.950990in}{1.308017in}}%
\pgfpathcurveto{\pgfqpoint{1.943177in}{1.300203in}}{\pgfqpoint{1.938786in}{1.289604in}}{\pgfqpoint{1.938786in}{1.278554in}}%
\pgfpathcurveto{\pgfqpoint{1.938786in}{1.267504in}}{\pgfqpoint{1.943177in}{1.256905in}}{\pgfqpoint{1.950990in}{1.249091in}}%
\pgfpathcurveto{\pgfqpoint{1.958804in}{1.241278in}}{\pgfqpoint{1.969403in}{1.236887in}}{\pgfqpoint{1.980453in}{1.236887in}}%
\pgfpathclose%
\pgfusepath{stroke,fill}%
\end{pgfscope}%
\begin{pgfscope}%
\pgfpathrectangle{\pgfqpoint{0.481978in}{0.331635in}}{\pgfqpoint{4.960000in}{3.696000in}}%
\pgfusepath{clip}%
\pgfsetbuttcap%
\pgfsetroundjoin%
\definecolor{currentfill}{rgb}{1.000000,0.705882,0.509804}%
\pgfsetfillcolor{currentfill}%
\pgfsetlinewidth{0.481800pt}%
\definecolor{currentstroke}{rgb}{1.000000,1.000000,1.000000}%
\pgfsetstrokecolor{currentstroke}%
\pgfsetdash{}{0pt}%
\pgfpathmoveto{\pgfqpoint{2.136569in}{3.817968in}}%
\pgfpathcurveto{\pgfqpoint{2.147619in}{3.817968in}}{\pgfqpoint{2.158218in}{3.822359in}}{\pgfqpoint{2.166032in}{3.830172in}}%
\pgfpathcurveto{\pgfqpoint{2.173846in}{3.837986in}}{\pgfqpoint{2.178236in}{3.848585in}}{\pgfqpoint{2.178236in}{3.859635in}}%
\pgfpathcurveto{\pgfqpoint{2.178236in}{3.870685in}}{\pgfqpoint{2.173846in}{3.881284in}}{\pgfqpoint{2.166032in}{3.889098in}}%
\pgfpathcurveto{\pgfqpoint{2.158218in}{3.896911in}}{\pgfqpoint{2.147619in}{3.901302in}}{\pgfqpoint{2.136569in}{3.901302in}}%
\pgfpathcurveto{\pgfqpoint{2.125519in}{3.901302in}}{\pgfqpoint{2.114920in}{3.896911in}}{\pgfqpoint{2.107106in}{3.889098in}}%
\pgfpathcurveto{\pgfqpoint{2.099293in}{3.881284in}}{\pgfqpoint{2.094903in}{3.870685in}}{\pgfqpoint{2.094903in}{3.859635in}}%
\pgfpathcurveto{\pgfqpoint{2.094903in}{3.848585in}}{\pgfqpoint{2.099293in}{3.837986in}}{\pgfqpoint{2.107106in}{3.830172in}}%
\pgfpathcurveto{\pgfqpoint{2.114920in}{3.822359in}}{\pgfqpoint{2.125519in}{3.817968in}}{\pgfqpoint{2.136569in}{3.817968in}}%
\pgfpathclose%
\pgfusepath{stroke,fill}%
\end{pgfscope}%
\begin{pgfscope}%
\pgfpathrectangle{\pgfqpoint{0.481978in}{0.331635in}}{\pgfqpoint{4.960000in}{3.696000in}}%
\pgfusepath{clip}%
\pgfsetbuttcap%
\pgfsetroundjoin%
\definecolor{currentfill}{rgb}{1.000000,0.705882,0.509804}%
\pgfsetfillcolor{currentfill}%
\pgfsetlinewidth{0.481800pt}%
\definecolor{currentstroke}{rgb}{1.000000,1.000000,1.000000}%
\pgfsetstrokecolor{currentstroke}%
\pgfsetdash{}{0pt}%
\pgfpathmoveto{\pgfqpoint{1.304196in}{2.000084in}}%
\pgfpathcurveto{\pgfqpoint{1.315246in}{2.000084in}}{\pgfqpoint{1.325845in}{2.004474in}}{\pgfqpoint{1.333659in}{2.012288in}}%
\pgfpathcurveto{\pgfqpoint{1.341472in}{2.020101in}}{\pgfqpoint{1.345863in}{2.030700in}}{\pgfqpoint{1.345863in}{2.041751in}}%
\pgfpathcurveto{\pgfqpoint{1.345863in}{2.052801in}}{\pgfqpoint{1.341472in}{2.063400in}}{\pgfqpoint{1.333659in}{2.071213in}}%
\pgfpathcurveto{\pgfqpoint{1.325845in}{2.079027in}}{\pgfqpoint{1.315246in}{2.083417in}}{\pgfqpoint{1.304196in}{2.083417in}}%
\pgfpathcurveto{\pgfqpoint{1.293146in}{2.083417in}}{\pgfqpoint{1.282547in}{2.079027in}}{\pgfqpoint{1.274733in}{2.071213in}}%
\pgfpathcurveto{\pgfqpoint{1.266919in}{2.063400in}}{\pgfqpoint{1.262529in}{2.052801in}}{\pgfqpoint{1.262529in}{2.041751in}}%
\pgfpathcurveto{\pgfqpoint{1.262529in}{2.030700in}}{\pgfqpoint{1.266919in}{2.020101in}}{\pgfqpoint{1.274733in}{2.012288in}}%
\pgfpathcurveto{\pgfqpoint{1.282547in}{2.004474in}}{\pgfqpoint{1.293146in}{2.000084in}}{\pgfqpoint{1.304196in}{2.000084in}}%
\pgfpathclose%
\pgfusepath{stroke,fill}%
\end{pgfscope}%
\begin{pgfscope}%
\pgfpathrectangle{\pgfqpoint{0.481978in}{0.331635in}}{\pgfqpoint{4.960000in}{3.696000in}}%
\pgfusepath{clip}%
\pgfsetbuttcap%
\pgfsetroundjoin%
\definecolor{currentfill}{rgb}{1.000000,0.705882,0.509804}%
\pgfsetfillcolor{currentfill}%
\pgfsetlinewidth{0.481800pt}%
\definecolor{currentstroke}{rgb}{1.000000,1.000000,1.000000}%
\pgfsetstrokecolor{currentstroke}%
\pgfsetdash{}{0pt}%
\pgfpathmoveto{\pgfqpoint{0.790548in}{2.204861in}}%
\pgfpathcurveto{\pgfqpoint{0.801598in}{2.204861in}}{\pgfqpoint{0.812198in}{2.209252in}}{\pgfqpoint{0.820011in}{2.217065in}}%
\pgfpathcurveto{\pgfqpoint{0.827825in}{2.224879in}}{\pgfqpoint{0.832215in}{2.235478in}}{\pgfqpoint{0.832215in}{2.246528in}}%
\pgfpathcurveto{\pgfqpoint{0.832215in}{2.257578in}}{\pgfqpoint{0.827825in}{2.268177in}}{\pgfqpoint{0.820011in}{2.275991in}}%
\pgfpathcurveto{\pgfqpoint{0.812198in}{2.283804in}}{\pgfqpoint{0.801598in}{2.288195in}}{\pgfqpoint{0.790548in}{2.288195in}}%
\pgfpathcurveto{\pgfqpoint{0.779498in}{2.288195in}}{\pgfqpoint{0.768899in}{2.283804in}}{\pgfqpoint{0.761086in}{2.275991in}}%
\pgfpathcurveto{\pgfqpoint{0.753272in}{2.268177in}}{\pgfqpoint{0.748882in}{2.257578in}}{\pgfqpoint{0.748882in}{2.246528in}}%
\pgfpathcurveto{\pgfqpoint{0.748882in}{2.235478in}}{\pgfqpoint{0.753272in}{2.224879in}}{\pgfqpoint{0.761086in}{2.217065in}}%
\pgfpathcurveto{\pgfqpoint{0.768899in}{2.209252in}}{\pgfqpoint{0.779498in}{2.204861in}}{\pgfqpoint{0.790548in}{2.204861in}}%
\pgfpathclose%
\pgfusepath{stroke,fill}%
\end{pgfscope}%
\begin{pgfscope}%
\pgfpathrectangle{\pgfqpoint{0.481978in}{0.331635in}}{\pgfqpoint{4.960000in}{3.696000in}}%
\pgfusepath{clip}%
\pgfsetbuttcap%
\pgfsetroundjoin%
\definecolor{currentfill}{rgb}{1.000000,0.705882,0.509804}%
\pgfsetfillcolor{currentfill}%
\pgfsetlinewidth{0.481800pt}%
\definecolor{currentstroke}{rgb}{1.000000,1.000000,1.000000}%
\pgfsetstrokecolor{currentstroke}%
\pgfsetdash{}{0pt}%
\pgfpathmoveto{\pgfqpoint{1.170260in}{2.153310in}}%
\pgfpathcurveto{\pgfqpoint{1.181310in}{2.153310in}}{\pgfqpoint{1.191909in}{2.157700in}}{\pgfqpoint{1.199723in}{2.165514in}}%
\pgfpathcurveto{\pgfqpoint{1.207536in}{2.173327in}}{\pgfqpoint{1.211926in}{2.183926in}}{\pgfqpoint{1.211926in}{2.194977in}}%
\pgfpathcurveto{\pgfqpoint{1.211926in}{2.206027in}}{\pgfqpoint{1.207536in}{2.216626in}}{\pgfqpoint{1.199723in}{2.224439in}}%
\pgfpathcurveto{\pgfqpoint{1.191909in}{2.232253in}}{\pgfqpoint{1.181310in}{2.236643in}}{\pgfqpoint{1.170260in}{2.236643in}}%
\pgfpathcurveto{\pgfqpoint{1.159210in}{2.236643in}}{\pgfqpoint{1.148611in}{2.232253in}}{\pgfqpoint{1.140797in}{2.224439in}}%
\pgfpathcurveto{\pgfqpoint{1.132983in}{2.216626in}}{\pgfqpoint{1.128593in}{2.206027in}}{\pgfqpoint{1.128593in}{2.194977in}}%
\pgfpathcurveto{\pgfqpoint{1.128593in}{2.183926in}}{\pgfqpoint{1.132983in}{2.173327in}}{\pgfqpoint{1.140797in}{2.165514in}}%
\pgfpathcurveto{\pgfqpoint{1.148611in}{2.157700in}}{\pgfqpoint{1.159210in}{2.153310in}}{\pgfqpoint{1.170260in}{2.153310in}}%
\pgfpathclose%
\pgfusepath{stroke,fill}%
\end{pgfscope}%
\begin{pgfscope}%
\pgfpathrectangle{\pgfqpoint{0.481978in}{0.331635in}}{\pgfqpoint{4.960000in}{3.696000in}}%
\pgfusepath{clip}%
\pgfsetbuttcap%
\pgfsetroundjoin%
\definecolor{currentfill}{rgb}{1.000000,0.705882,0.509804}%
\pgfsetfillcolor{currentfill}%
\pgfsetlinewidth{0.481800pt}%
\definecolor{currentstroke}{rgb}{1.000000,1.000000,1.000000}%
\pgfsetstrokecolor{currentstroke}%
\pgfsetdash{}{0pt}%
\pgfpathmoveto{\pgfqpoint{1.295016in}{1.714660in}}%
\pgfpathcurveto{\pgfqpoint{1.306066in}{1.714660in}}{\pgfqpoint{1.316665in}{1.719050in}}{\pgfqpoint{1.324479in}{1.726864in}}%
\pgfpathcurveto{\pgfqpoint{1.332293in}{1.734678in}}{\pgfqpoint{1.336683in}{1.745277in}}{\pgfqpoint{1.336683in}{1.756327in}}%
\pgfpathcurveto{\pgfqpoint{1.336683in}{1.767377in}}{\pgfqpoint{1.332293in}{1.777976in}}{\pgfqpoint{1.324479in}{1.785790in}}%
\pgfpathcurveto{\pgfqpoint{1.316665in}{1.793603in}}{\pgfqpoint{1.306066in}{1.797994in}}{\pgfqpoint{1.295016in}{1.797994in}}%
\pgfpathcurveto{\pgfqpoint{1.283966in}{1.797994in}}{\pgfqpoint{1.273367in}{1.793603in}}{\pgfqpoint{1.265553in}{1.785790in}}%
\pgfpathcurveto{\pgfqpoint{1.257740in}{1.777976in}}{\pgfqpoint{1.253349in}{1.767377in}}{\pgfqpoint{1.253349in}{1.756327in}}%
\pgfpathcurveto{\pgfqpoint{1.253349in}{1.745277in}}{\pgfqpoint{1.257740in}{1.734678in}}{\pgfqpoint{1.265553in}{1.726864in}}%
\pgfpathcurveto{\pgfqpoint{1.273367in}{1.719050in}}{\pgfqpoint{1.283966in}{1.714660in}}{\pgfqpoint{1.295016in}{1.714660in}}%
\pgfpathclose%
\pgfusepath{stroke,fill}%
\end{pgfscope}%
\begin{pgfscope}%
\pgfpathrectangle{\pgfqpoint{0.481978in}{0.331635in}}{\pgfqpoint{4.960000in}{3.696000in}}%
\pgfusepath{clip}%
\pgfsetbuttcap%
\pgfsetroundjoin%
\definecolor{currentfill}{rgb}{1.000000,0.705882,0.509804}%
\pgfsetfillcolor{currentfill}%
\pgfsetlinewidth{0.481800pt}%
\definecolor{currentstroke}{rgb}{1.000000,1.000000,1.000000}%
\pgfsetstrokecolor{currentstroke}%
\pgfsetdash{}{0pt}%
\pgfpathmoveto{\pgfqpoint{4.058832in}{3.297705in}}%
\pgfpathcurveto{\pgfqpoint{4.069882in}{3.297705in}}{\pgfqpoint{4.080481in}{3.302095in}}{\pgfqpoint{4.088294in}{3.309909in}}%
\pgfpathcurveto{\pgfqpoint{4.096108in}{3.317722in}}{\pgfqpoint{4.100498in}{3.328321in}}{\pgfqpoint{4.100498in}{3.339371in}}%
\pgfpathcurveto{\pgfqpoint{4.100498in}{3.350422in}}{\pgfqpoint{4.096108in}{3.361021in}}{\pgfqpoint{4.088294in}{3.368834in}}%
\pgfpathcurveto{\pgfqpoint{4.080481in}{3.376648in}}{\pgfqpoint{4.069882in}{3.381038in}}{\pgfqpoint{4.058832in}{3.381038in}}%
\pgfpathcurveto{\pgfqpoint{4.047782in}{3.381038in}}{\pgfqpoint{4.037183in}{3.376648in}}{\pgfqpoint{4.029369in}{3.368834in}}%
\pgfpathcurveto{\pgfqpoint{4.021555in}{3.361021in}}{\pgfqpoint{4.017165in}{3.350422in}}{\pgfqpoint{4.017165in}{3.339371in}}%
\pgfpathcurveto{\pgfqpoint{4.017165in}{3.328321in}}{\pgfqpoint{4.021555in}{3.317722in}}{\pgfqpoint{4.029369in}{3.309909in}}%
\pgfpathcurveto{\pgfqpoint{4.037183in}{3.302095in}}{\pgfqpoint{4.047782in}{3.297705in}}{\pgfqpoint{4.058832in}{3.297705in}}%
\pgfpathclose%
\pgfusepath{stroke,fill}%
\end{pgfscope}%
\begin{pgfscope}%
\pgfpathrectangle{\pgfqpoint{0.481978in}{0.331635in}}{\pgfqpoint{4.960000in}{3.696000in}}%
\pgfusepath{clip}%
\pgfsetbuttcap%
\pgfsetroundjoin%
\definecolor{currentfill}{rgb}{1.000000,0.705882,0.509804}%
\pgfsetfillcolor{currentfill}%
\pgfsetlinewidth{0.481800pt}%
\definecolor{currentstroke}{rgb}{1.000000,1.000000,1.000000}%
\pgfsetstrokecolor{currentstroke}%
\pgfsetdash{}{0pt}%
\pgfpathmoveto{\pgfqpoint{3.682995in}{2.989736in}}%
\pgfpathcurveto{\pgfqpoint{3.694045in}{2.989736in}}{\pgfqpoint{3.704644in}{2.994126in}}{\pgfqpoint{3.712458in}{3.001940in}}%
\pgfpathcurveto{\pgfqpoint{3.720272in}{3.009753in}}{\pgfqpoint{3.724662in}{3.020352in}}{\pgfqpoint{3.724662in}{3.031402in}}%
\pgfpathcurveto{\pgfqpoint{3.724662in}{3.042453in}}{\pgfqpoint{3.720272in}{3.053052in}}{\pgfqpoint{3.712458in}{3.060865in}}%
\pgfpathcurveto{\pgfqpoint{3.704644in}{3.068679in}}{\pgfqpoint{3.694045in}{3.073069in}}{\pgfqpoint{3.682995in}{3.073069in}}%
\pgfpathcurveto{\pgfqpoint{3.671945in}{3.073069in}}{\pgfqpoint{3.661346in}{3.068679in}}{\pgfqpoint{3.653532in}{3.060865in}}%
\pgfpathcurveto{\pgfqpoint{3.645719in}{3.053052in}}{\pgfqpoint{3.641329in}{3.042453in}}{\pgfqpoint{3.641329in}{3.031402in}}%
\pgfpathcurveto{\pgfqpoint{3.641329in}{3.020352in}}{\pgfqpoint{3.645719in}{3.009753in}}{\pgfqpoint{3.653532in}{3.001940in}}%
\pgfpathcurveto{\pgfqpoint{3.661346in}{2.994126in}}{\pgfqpoint{3.671945in}{2.989736in}}{\pgfqpoint{3.682995in}{2.989736in}}%
\pgfpathclose%
\pgfusepath{stroke,fill}%
\end{pgfscope}%
\begin{pgfscope}%
\pgfpathrectangle{\pgfqpoint{0.481978in}{0.331635in}}{\pgfqpoint{4.960000in}{3.696000in}}%
\pgfusepath{clip}%
\pgfsetbuttcap%
\pgfsetroundjoin%
\definecolor{currentfill}{rgb}{1.000000,0.705882,0.509804}%
\pgfsetfillcolor{currentfill}%
\pgfsetlinewidth{0.481800pt}%
\definecolor{currentstroke}{rgb}{1.000000,1.000000,1.000000}%
\pgfsetstrokecolor{currentstroke}%
\pgfsetdash{}{0pt}%
\pgfpathmoveto{\pgfqpoint{2.309434in}{1.153550in}}%
\pgfpathcurveto{\pgfqpoint{2.320484in}{1.153550in}}{\pgfqpoint{2.331083in}{1.157941in}}{\pgfqpoint{2.338896in}{1.165754in}}%
\pgfpathcurveto{\pgfqpoint{2.346710in}{1.173568in}}{\pgfqpoint{2.351100in}{1.184167in}}{\pgfqpoint{2.351100in}{1.195217in}}%
\pgfpathcurveto{\pgfqpoint{2.351100in}{1.206267in}}{\pgfqpoint{2.346710in}{1.216866in}}{\pgfqpoint{2.338896in}{1.224680in}}%
\pgfpathcurveto{\pgfqpoint{2.331083in}{1.232493in}}{\pgfqpoint{2.320484in}{1.236884in}}{\pgfqpoint{2.309434in}{1.236884in}}%
\pgfpathcurveto{\pgfqpoint{2.298384in}{1.236884in}}{\pgfqpoint{2.287784in}{1.232493in}}{\pgfqpoint{2.279971in}{1.224680in}}%
\pgfpathcurveto{\pgfqpoint{2.272157in}{1.216866in}}{\pgfqpoint{2.267767in}{1.206267in}}{\pgfqpoint{2.267767in}{1.195217in}}%
\pgfpathcurveto{\pgfqpoint{2.267767in}{1.184167in}}{\pgfqpoint{2.272157in}{1.173568in}}{\pgfqpoint{2.279971in}{1.165754in}}%
\pgfpathcurveto{\pgfqpoint{2.287784in}{1.157941in}}{\pgfqpoint{2.298384in}{1.153550in}}{\pgfqpoint{2.309434in}{1.153550in}}%
\pgfpathclose%
\pgfusepath{stroke,fill}%
\end{pgfscope}%
\begin{pgfscope}%
\pgfpathrectangle{\pgfqpoint{0.481978in}{0.331635in}}{\pgfqpoint{4.960000in}{3.696000in}}%
\pgfusepath{clip}%
\pgfsetbuttcap%
\pgfsetroundjoin%
\definecolor{currentfill}{rgb}{1.000000,0.705882,0.509804}%
\pgfsetfillcolor{currentfill}%
\pgfsetlinewidth{0.481800pt}%
\definecolor{currentstroke}{rgb}{1.000000,1.000000,1.000000}%
\pgfsetstrokecolor{currentstroke}%
\pgfsetdash{}{0pt}%
\pgfpathmoveto{\pgfqpoint{4.283274in}{1.073470in}}%
\pgfpathcurveto{\pgfqpoint{4.294324in}{1.073470in}}{\pgfqpoint{4.304923in}{1.077860in}}{\pgfqpoint{4.312737in}{1.085674in}}%
\pgfpathcurveto{\pgfqpoint{4.320550in}{1.093487in}}{\pgfqpoint{4.324940in}{1.104086in}}{\pgfqpoint{4.324940in}{1.115136in}}%
\pgfpathcurveto{\pgfqpoint{4.324940in}{1.126186in}}{\pgfqpoint{4.320550in}{1.136785in}}{\pgfqpoint{4.312737in}{1.144599in}}%
\pgfpathcurveto{\pgfqpoint{4.304923in}{1.152413in}}{\pgfqpoint{4.294324in}{1.156803in}}{\pgfqpoint{4.283274in}{1.156803in}}%
\pgfpathcurveto{\pgfqpoint{4.272224in}{1.156803in}}{\pgfqpoint{4.261625in}{1.152413in}}{\pgfqpoint{4.253811in}{1.144599in}}%
\pgfpathcurveto{\pgfqpoint{4.245997in}{1.136785in}}{\pgfqpoint{4.241607in}{1.126186in}}{\pgfqpoint{4.241607in}{1.115136in}}%
\pgfpathcurveto{\pgfqpoint{4.241607in}{1.104086in}}{\pgfqpoint{4.245997in}{1.093487in}}{\pgfqpoint{4.253811in}{1.085674in}}%
\pgfpathcurveto{\pgfqpoint{4.261625in}{1.077860in}}{\pgfqpoint{4.272224in}{1.073470in}}{\pgfqpoint{4.283274in}{1.073470in}}%
\pgfpathclose%
\pgfusepath{stroke,fill}%
\end{pgfscope}%
\begin{pgfscope}%
\pgfpathrectangle{\pgfqpoint{0.481978in}{0.331635in}}{\pgfqpoint{4.960000in}{3.696000in}}%
\pgfusepath{clip}%
\pgfsetbuttcap%
\pgfsetroundjoin%
\definecolor{currentfill}{rgb}{1.000000,0.705882,0.509804}%
\pgfsetfillcolor{currentfill}%
\pgfsetlinewidth{0.481800pt}%
\definecolor{currentstroke}{rgb}{1.000000,1.000000,1.000000}%
\pgfsetstrokecolor{currentstroke}%
\pgfsetdash{}{0pt}%
\pgfpathmoveto{\pgfqpoint{1.048524in}{2.608904in}}%
\pgfpathcurveto{\pgfqpoint{1.059574in}{2.608904in}}{\pgfqpoint{1.070173in}{2.613295in}}{\pgfqpoint{1.077987in}{2.621108in}}%
\pgfpathcurveto{\pgfqpoint{1.085800in}{2.628922in}}{\pgfqpoint{1.090191in}{2.639521in}}{\pgfqpoint{1.090191in}{2.650571in}}%
\pgfpathcurveto{\pgfqpoint{1.090191in}{2.661621in}}{\pgfqpoint{1.085800in}{2.672220in}}{\pgfqpoint{1.077987in}{2.680034in}}%
\pgfpathcurveto{\pgfqpoint{1.070173in}{2.687847in}}{\pgfqpoint{1.059574in}{2.692238in}}{\pgfqpoint{1.048524in}{2.692238in}}%
\pgfpathcurveto{\pgfqpoint{1.037474in}{2.692238in}}{\pgfqpoint{1.026875in}{2.687847in}}{\pgfqpoint{1.019061in}{2.680034in}}%
\pgfpathcurveto{\pgfqpoint{1.011247in}{2.672220in}}{\pgfqpoint{1.006857in}{2.661621in}}{\pgfqpoint{1.006857in}{2.650571in}}%
\pgfpathcurveto{\pgfqpoint{1.006857in}{2.639521in}}{\pgfqpoint{1.011247in}{2.628922in}}{\pgfqpoint{1.019061in}{2.621108in}}%
\pgfpathcurveto{\pgfqpoint{1.026875in}{2.613295in}}{\pgfqpoint{1.037474in}{2.608904in}}{\pgfqpoint{1.048524in}{2.608904in}}%
\pgfpathclose%
\pgfusepath{stroke,fill}%
\end{pgfscope}%
\begin{pgfscope}%
\pgfpathrectangle{\pgfqpoint{0.481978in}{0.331635in}}{\pgfqpoint{4.960000in}{3.696000in}}%
\pgfusepath{clip}%
\pgfsetbuttcap%
\pgfsetroundjoin%
\definecolor{currentfill}{rgb}{1.000000,0.705882,0.509804}%
\pgfsetfillcolor{currentfill}%
\pgfsetlinewidth{0.481800pt}%
\definecolor{currentstroke}{rgb}{1.000000,1.000000,1.000000}%
\pgfsetstrokecolor{currentstroke}%
\pgfsetdash{}{0pt}%
\pgfpathmoveto{\pgfqpoint{1.199548in}{2.388633in}}%
\pgfpathcurveto{\pgfqpoint{1.210598in}{2.388633in}}{\pgfqpoint{1.221197in}{2.393023in}}{\pgfqpoint{1.229010in}{2.400836in}}%
\pgfpathcurveto{\pgfqpoint{1.236824in}{2.408650in}}{\pgfqpoint{1.241214in}{2.419249in}}{\pgfqpoint{1.241214in}{2.430299in}}%
\pgfpathcurveto{\pgfqpoint{1.241214in}{2.441349in}}{\pgfqpoint{1.236824in}{2.451948in}}{\pgfqpoint{1.229010in}{2.459762in}}%
\pgfpathcurveto{\pgfqpoint{1.221197in}{2.467576in}}{\pgfqpoint{1.210598in}{2.471966in}}{\pgfqpoint{1.199548in}{2.471966in}}%
\pgfpathcurveto{\pgfqpoint{1.188497in}{2.471966in}}{\pgfqpoint{1.177898in}{2.467576in}}{\pgfqpoint{1.170085in}{2.459762in}}%
\pgfpathcurveto{\pgfqpoint{1.162271in}{2.451948in}}{\pgfqpoint{1.157881in}{2.441349in}}{\pgfqpoint{1.157881in}{2.430299in}}%
\pgfpathcurveto{\pgfqpoint{1.157881in}{2.419249in}}{\pgfqpoint{1.162271in}{2.408650in}}{\pgfqpoint{1.170085in}{2.400836in}}%
\pgfpathcurveto{\pgfqpoint{1.177898in}{2.393023in}}{\pgfqpoint{1.188497in}{2.388633in}}{\pgfqpoint{1.199548in}{2.388633in}}%
\pgfpathclose%
\pgfusepath{stroke,fill}%
\end{pgfscope}%
\begin{pgfscope}%
\pgfpathrectangle{\pgfqpoint{0.481978in}{0.331635in}}{\pgfqpoint{4.960000in}{3.696000in}}%
\pgfusepath{clip}%
\pgfsetbuttcap%
\pgfsetroundjoin%
\definecolor{currentfill}{rgb}{1.000000,0.705882,0.509804}%
\pgfsetfillcolor{currentfill}%
\pgfsetlinewidth{0.481800pt}%
\definecolor{currentstroke}{rgb}{1.000000,1.000000,1.000000}%
\pgfsetstrokecolor{currentstroke}%
\pgfsetdash{}{0pt}%
\pgfpathmoveto{\pgfqpoint{4.506455in}{2.660190in}}%
\pgfpathcurveto{\pgfqpoint{4.517505in}{2.660190in}}{\pgfqpoint{4.528104in}{2.664581in}}{\pgfqpoint{4.535917in}{2.672394in}}%
\pgfpathcurveto{\pgfqpoint{4.543731in}{2.680208in}}{\pgfqpoint{4.548121in}{2.690807in}}{\pgfqpoint{4.548121in}{2.701857in}}%
\pgfpathcurveto{\pgfqpoint{4.548121in}{2.712907in}}{\pgfqpoint{4.543731in}{2.723506in}}{\pgfqpoint{4.535917in}{2.731320in}}%
\pgfpathcurveto{\pgfqpoint{4.528104in}{2.739134in}}{\pgfqpoint{4.517505in}{2.743524in}}{\pgfqpoint{4.506455in}{2.743524in}}%
\pgfpathcurveto{\pgfqpoint{4.495404in}{2.743524in}}{\pgfqpoint{4.484805in}{2.739134in}}{\pgfqpoint{4.476992in}{2.731320in}}%
\pgfpathcurveto{\pgfqpoint{4.469178in}{2.723506in}}{\pgfqpoint{4.464788in}{2.712907in}}{\pgfqpoint{4.464788in}{2.701857in}}%
\pgfpathcurveto{\pgfqpoint{4.464788in}{2.690807in}}{\pgfqpoint{4.469178in}{2.680208in}}{\pgfqpoint{4.476992in}{2.672394in}}%
\pgfpathcurveto{\pgfqpoint{4.484805in}{2.664581in}}{\pgfqpoint{4.495404in}{2.660190in}}{\pgfqpoint{4.506455in}{2.660190in}}%
\pgfpathclose%
\pgfusepath{stroke,fill}%
\end{pgfscope}%
\begin{pgfscope}%
\pgfpathrectangle{\pgfqpoint{0.481978in}{0.331635in}}{\pgfqpoint{4.960000in}{3.696000in}}%
\pgfusepath{clip}%
\pgfsetbuttcap%
\pgfsetroundjoin%
\definecolor{currentfill}{rgb}{1.000000,0.705882,0.509804}%
\pgfsetfillcolor{currentfill}%
\pgfsetlinewidth{0.481800pt}%
\definecolor{currentstroke}{rgb}{1.000000,1.000000,1.000000}%
\pgfsetstrokecolor{currentstroke}%
\pgfsetdash{}{0pt}%
\pgfpathmoveto{\pgfqpoint{3.805585in}{2.946464in}}%
\pgfpathcurveto{\pgfqpoint{3.816635in}{2.946464in}}{\pgfqpoint{3.827234in}{2.950854in}}{\pgfqpoint{3.835048in}{2.958668in}}%
\pgfpathcurveto{\pgfqpoint{3.842861in}{2.966482in}}{\pgfqpoint{3.847251in}{2.977081in}}{\pgfqpoint{3.847251in}{2.988131in}}%
\pgfpathcurveto{\pgfqpoint{3.847251in}{2.999181in}}{\pgfqpoint{3.842861in}{3.009780in}}{\pgfqpoint{3.835048in}{3.017594in}}%
\pgfpathcurveto{\pgfqpoint{3.827234in}{3.025407in}}{\pgfqpoint{3.816635in}{3.029798in}}{\pgfqpoint{3.805585in}{3.029798in}}%
\pgfpathcurveto{\pgfqpoint{3.794535in}{3.029798in}}{\pgfqpoint{3.783936in}{3.025407in}}{\pgfqpoint{3.776122in}{3.017594in}}%
\pgfpathcurveto{\pgfqpoint{3.768308in}{3.009780in}}{\pgfqpoint{3.763918in}{2.999181in}}{\pgfqpoint{3.763918in}{2.988131in}}%
\pgfpathcurveto{\pgfqpoint{3.763918in}{2.977081in}}{\pgfqpoint{3.768308in}{2.966482in}}{\pgfqpoint{3.776122in}{2.958668in}}%
\pgfpathcurveto{\pgfqpoint{3.783936in}{2.950854in}}{\pgfqpoint{3.794535in}{2.946464in}}{\pgfqpoint{3.805585in}{2.946464in}}%
\pgfpathclose%
\pgfusepath{stroke,fill}%
\end{pgfscope}%
\begin{pgfscope}%
\pgfpathrectangle{\pgfqpoint{0.481978in}{0.331635in}}{\pgfqpoint{4.960000in}{3.696000in}}%
\pgfusepath{clip}%
\pgfsetbuttcap%
\pgfsetroundjoin%
\definecolor{currentfill}{rgb}{1.000000,0.705882,0.509804}%
\pgfsetfillcolor{currentfill}%
\pgfsetlinewidth{0.481800pt}%
\definecolor{currentstroke}{rgb}{1.000000,1.000000,1.000000}%
\pgfsetstrokecolor{currentstroke}%
\pgfsetdash{}{0pt}%
\pgfpathmoveto{\pgfqpoint{1.394040in}{1.688848in}}%
\pgfpathcurveto{\pgfqpoint{1.405090in}{1.688848in}}{\pgfqpoint{1.415689in}{1.693238in}}{\pgfqpoint{1.423503in}{1.701052in}}%
\pgfpathcurveto{\pgfqpoint{1.431317in}{1.708865in}}{\pgfqpoint{1.435707in}{1.719464in}}{\pgfqpoint{1.435707in}{1.730514in}}%
\pgfpathcurveto{\pgfqpoint{1.435707in}{1.741565in}}{\pgfqpoint{1.431317in}{1.752164in}}{\pgfqpoint{1.423503in}{1.759977in}}%
\pgfpathcurveto{\pgfqpoint{1.415689in}{1.767791in}}{\pgfqpoint{1.405090in}{1.772181in}}{\pgfqpoint{1.394040in}{1.772181in}}%
\pgfpathcurveto{\pgfqpoint{1.382990in}{1.772181in}}{\pgfqpoint{1.372391in}{1.767791in}}{\pgfqpoint{1.364578in}{1.759977in}}%
\pgfpathcurveto{\pgfqpoint{1.356764in}{1.752164in}}{\pgfqpoint{1.352374in}{1.741565in}}{\pgfqpoint{1.352374in}{1.730514in}}%
\pgfpathcurveto{\pgfqpoint{1.352374in}{1.719464in}}{\pgfqpoint{1.356764in}{1.708865in}}{\pgfqpoint{1.364578in}{1.701052in}}%
\pgfpathcurveto{\pgfqpoint{1.372391in}{1.693238in}}{\pgfqpoint{1.382990in}{1.688848in}}{\pgfqpoint{1.394040in}{1.688848in}}%
\pgfpathclose%
\pgfusepath{stroke,fill}%
\end{pgfscope}%
\begin{pgfscope}%
\pgfpathrectangle{\pgfqpoint{0.481978in}{0.331635in}}{\pgfqpoint{4.960000in}{3.696000in}}%
\pgfusepath{clip}%
\pgfsetbuttcap%
\pgfsetroundjoin%
\definecolor{currentfill}{rgb}{1.000000,0.705882,0.509804}%
\pgfsetfillcolor{currentfill}%
\pgfsetlinewidth{0.481800pt}%
\definecolor{currentstroke}{rgb}{1.000000,1.000000,1.000000}%
\pgfsetstrokecolor{currentstroke}%
\pgfsetdash{}{0pt}%
\pgfpathmoveto{\pgfqpoint{4.326413in}{2.939642in}}%
\pgfpathcurveto{\pgfqpoint{4.337464in}{2.939642in}}{\pgfqpoint{4.348063in}{2.944032in}}{\pgfqpoint{4.355876in}{2.951846in}}%
\pgfpathcurveto{\pgfqpoint{4.363690in}{2.959659in}}{\pgfqpoint{4.368080in}{2.970258in}}{\pgfqpoint{4.368080in}{2.981308in}}%
\pgfpathcurveto{\pgfqpoint{4.368080in}{2.992358in}}{\pgfqpoint{4.363690in}{3.002957in}}{\pgfqpoint{4.355876in}{3.010771in}}%
\pgfpathcurveto{\pgfqpoint{4.348063in}{3.018585in}}{\pgfqpoint{4.337464in}{3.022975in}}{\pgfqpoint{4.326413in}{3.022975in}}%
\pgfpathcurveto{\pgfqpoint{4.315363in}{3.022975in}}{\pgfqpoint{4.304764in}{3.018585in}}{\pgfqpoint{4.296951in}{3.010771in}}%
\pgfpathcurveto{\pgfqpoint{4.289137in}{3.002957in}}{\pgfqpoint{4.284747in}{2.992358in}}{\pgfqpoint{4.284747in}{2.981308in}}%
\pgfpathcurveto{\pgfqpoint{4.284747in}{2.970258in}}{\pgfqpoint{4.289137in}{2.959659in}}{\pgfqpoint{4.296951in}{2.951846in}}%
\pgfpathcurveto{\pgfqpoint{4.304764in}{2.944032in}}{\pgfqpoint{4.315363in}{2.939642in}}{\pgfqpoint{4.326413in}{2.939642in}}%
\pgfpathclose%
\pgfusepath{stroke,fill}%
\end{pgfscope}%
\begin{pgfscope}%
\pgfpathrectangle{\pgfqpoint{0.481978in}{0.331635in}}{\pgfqpoint{4.960000in}{3.696000in}}%
\pgfusepath{clip}%
\pgfsetbuttcap%
\pgfsetroundjoin%
\definecolor{currentfill}{rgb}{1.000000,0.705882,0.509804}%
\pgfsetfillcolor{currentfill}%
\pgfsetlinewidth{0.481800pt}%
\definecolor{currentstroke}{rgb}{1.000000,1.000000,1.000000}%
\pgfsetstrokecolor{currentstroke}%
\pgfsetdash{}{0pt}%
\pgfpathmoveto{\pgfqpoint{3.457056in}{2.840711in}}%
\pgfpathcurveto{\pgfqpoint{3.468106in}{2.840711in}}{\pgfqpoint{3.478705in}{2.845101in}}{\pgfqpoint{3.486518in}{2.852915in}}%
\pgfpathcurveto{\pgfqpoint{3.494332in}{2.860728in}}{\pgfqpoint{3.498722in}{2.871328in}}{\pgfqpoint{3.498722in}{2.882378in}}%
\pgfpathcurveto{\pgfqpoint{3.498722in}{2.893428in}}{\pgfqpoint{3.494332in}{2.904027in}}{\pgfqpoint{3.486518in}{2.911840in}}%
\pgfpathcurveto{\pgfqpoint{3.478705in}{2.919654in}}{\pgfqpoint{3.468106in}{2.924044in}}{\pgfqpoint{3.457056in}{2.924044in}}%
\pgfpathcurveto{\pgfqpoint{3.446005in}{2.924044in}}{\pgfqpoint{3.435406in}{2.919654in}}{\pgfqpoint{3.427593in}{2.911840in}}%
\pgfpathcurveto{\pgfqpoint{3.419779in}{2.904027in}}{\pgfqpoint{3.415389in}{2.893428in}}{\pgfqpoint{3.415389in}{2.882378in}}%
\pgfpathcurveto{\pgfqpoint{3.415389in}{2.871328in}}{\pgfqpoint{3.419779in}{2.860728in}}{\pgfqpoint{3.427593in}{2.852915in}}%
\pgfpathcurveto{\pgfqpoint{3.435406in}{2.845101in}}{\pgfqpoint{3.446005in}{2.840711in}}{\pgfqpoint{3.457056in}{2.840711in}}%
\pgfpathclose%
\pgfusepath{stroke,fill}%
\end{pgfscope}%
\begin{pgfscope}%
\pgfpathrectangle{\pgfqpoint{0.481978in}{0.331635in}}{\pgfqpoint{4.960000in}{3.696000in}}%
\pgfusepath{clip}%
\pgfsetbuttcap%
\pgfsetroundjoin%
\definecolor{currentfill}{rgb}{1.000000,0.705882,0.509804}%
\pgfsetfillcolor{currentfill}%
\pgfsetlinewidth{0.481800pt}%
\definecolor{currentstroke}{rgb}{1.000000,1.000000,1.000000}%
\pgfsetstrokecolor{currentstroke}%
\pgfsetdash{}{0pt}%
\pgfpathmoveto{\pgfqpoint{3.818493in}{3.127111in}}%
\pgfpathcurveto{\pgfqpoint{3.829543in}{3.127111in}}{\pgfqpoint{3.840142in}{3.131501in}}{\pgfqpoint{3.847956in}{3.139315in}}%
\pgfpathcurveto{\pgfqpoint{3.855769in}{3.147128in}}{\pgfqpoint{3.860160in}{3.157728in}}{\pgfqpoint{3.860160in}{3.168778in}}%
\pgfpathcurveto{\pgfqpoint{3.860160in}{3.179828in}}{\pgfqpoint{3.855769in}{3.190427in}}{\pgfqpoint{3.847956in}{3.198240in}}%
\pgfpathcurveto{\pgfqpoint{3.840142in}{3.206054in}}{\pgfqpoint{3.829543in}{3.210444in}}{\pgfqpoint{3.818493in}{3.210444in}}%
\pgfpathcurveto{\pgfqpoint{3.807443in}{3.210444in}}{\pgfqpoint{3.796844in}{3.206054in}}{\pgfqpoint{3.789030in}{3.198240in}}%
\pgfpathcurveto{\pgfqpoint{3.781217in}{3.190427in}}{\pgfqpoint{3.776826in}{3.179828in}}{\pgfqpoint{3.776826in}{3.168778in}}%
\pgfpathcurveto{\pgfqpoint{3.776826in}{3.157728in}}{\pgfqpoint{3.781217in}{3.147128in}}{\pgfqpoint{3.789030in}{3.139315in}}%
\pgfpathcurveto{\pgfqpoint{3.796844in}{3.131501in}}{\pgfqpoint{3.807443in}{3.127111in}}{\pgfqpoint{3.818493in}{3.127111in}}%
\pgfpathclose%
\pgfusepath{stroke,fill}%
\end{pgfscope}%
\begin{pgfscope}%
\pgfpathrectangle{\pgfqpoint{0.481978in}{0.331635in}}{\pgfqpoint{4.960000in}{3.696000in}}%
\pgfusepath{clip}%
\pgfsetbuttcap%
\pgfsetroundjoin%
\definecolor{currentfill}{rgb}{1.000000,0.705882,0.509804}%
\pgfsetfillcolor{currentfill}%
\pgfsetlinewidth{0.481800pt}%
\definecolor{currentstroke}{rgb}{1.000000,1.000000,1.000000}%
\pgfsetstrokecolor{currentstroke}%
\pgfsetdash{}{0pt}%
\pgfpathmoveto{\pgfqpoint{3.050267in}{1.465288in}}%
\pgfpathcurveto{\pgfqpoint{3.061317in}{1.465288in}}{\pgfqpoint{3.071916in}{1.469679in}}{\pgfqpoint{3.079730in}{1.477492in}}%
\pgfpathcurveto{\pgfqpoint{3.087543in}{1.485306in}}{\pgfqpoint{3.091934in}{1.495905in}}{\pgfqpoint{3.091934in}{1.506955in}}%
\pgfpathcurveto{\pgfqpoint{3.091934in}{1.518005in}}{\pgfqpoint{3.087543in}{1.528604in}}{\pgfqpoint{3.079730in}{1.536418in}}%
\pgfpathcurveto{\pgfqpoint{3.071916in}{1.544231in}}{\pgfqpoint{3.061317in}{1.548622in}}{\pgfqpoint{3.050267in}{1.548622in}}%
\pgfpathcurveto{\pgfqpoint{3.039217in}{1.548622in}}{\pgfqpoint{3.028618in}{1.544231in}}{\pgfqpoint{3.020804in}{1.536418in}}%
\pgfpathcurveto{\pgfqpoint{3.012991in}{1.528604in}}{\pgfqpoint{3.008600in}{1.518005in}}{\pgfqpoint{3.008600in}{1.506955in}}%
\pgfpathcurveto{\pgfqpoint{3.008600in}{1.495905in}}{\pgfqpoint{3.012991in}{1.485306in}}{\pgfqpoint{3.020804in}{1.477492in}}%
\pgfpathcurveto{\pgfqpoint{3.028618in}{1.469679in}}{\pgfqpoint{3.039217in}{1.465288in}}{\pgfqpoint{3.050267in}{1.465288in}}%
\pgfpathclose%
\pgfusepath{stroke,fill}%
\end{pgfscope}%
\begin{pgfscope}%
\pgfpathrectangle{\pgfqpoint{0.481978in}{0.331635in}}{\pgfqpoint{4.960000in}{3.696000in}}%
\pgfusepath{clip}%
\pgfsetbuttcap%
\pgfsetroundjoin%
\definecolor{currentfill}{rgb}{1.000000,0.705882,0.509804}%
\pgfsetfillcolor{currentfill}%
\pgfsetlinewidth{0.481800pt}%
\definecolor{currentstroke}{rgb}{1.000000,1.000000,1.000000}%
\pgfsetstrokecolor{currentstroke}%
\pgfsetdash{}{0pt}%
\pgfpathmoveto{\pgfqpoint{2.971686in}{2.189413in}}%
\pgfpathcurveto{\pgfqpoint{2.982736in}{2.189413in}}{\pgfqpoint{2.993335in}{2.193804in}}{\pgfqpoint{3.001149in}{2.201617in}}%
\pgfpathcurveto{\pgfqpoint{3.008963in}{2.209431in}}{\pgfqpoint{3.013353in}{2.220030in}}{\pgfqpoint{3.013353in}{2.231080in}}%
\pgfpathcurveto{\pgfqpoint{3.013353in}{2.242130in}}{\pgfqpoint{3.008963in}{2.252729in}}{\pgfqpoint{3.001149in}{2.260543in}}%
\pgfpathcurveto{\pgfqpoint{2.993335in}{2.268356in}}{\pgfqpoint{2.982736in}{2.272747in}}{\pgfqpoint{2.971686in}{2.272747in}}%
\pgfpathcurveto{\pgfqpoint{2.960636in}{2.272747in}}{\pgfqpoint{2.950037in}{2.268356in}}{\pgfqpoint{2.942223in}{2.260543in}}%
\pgfpathcurveto{\pgfqpoint{2.934410in}{2.252729in}}{\pgfqpoint{2.930019in}{2.242130in}}{\pgfqpoint{2.930019in}{2.231080in}}%
\pgfpathcurveto{\pgfqpoint{2.930019in}{2.220030in}}{\pgfqpoint{2.934410in}{2.209431in}}{\pgfqpoint{2.942223in}{2.201617in}}%
\pgfpathcurveto{\pgfqpoint{2.950037in}{2.193804in}}{\pgfqpoint{2.960636in}{2.189413in}}{\pgfqpoint{2.971686in}{2.189413in}}%
\pgfpathclose%
\pgfusepath{stroke,fill}%
\end{pgfscope}%
\begin{pgfscope}%
\pgfpathrectangle{\pgfqpoint{0.481978in}{0.331635in}}{\pgfqpoint{4.960000in}{3.696000in}}%
\pgfusepath{clip}%
\pgfsetbuttcap%
\pgfsetroundjoin%
\definecolor{currentfill}{rgb}{1.000000,0.705882,0.509804}%
\pgfsetfillcolor{currentfill}%
\pgfsetlinewidth{0.481800pt}%
\definecolor{currentstroke}{rgb}{1.000000,1.000000,1.000000}%
\pgfsetstrokecolor{currentstroke}%
\pgfsetdash{}{0pt}%
\pgfpathmoveto{\pgfqpoint{3.212180in}{0.752277in}}%
\pgfpathcurveto{\pgfqpoint{3.223230in}{0.752277in}}{\pgfqpoint{3.233829in}{0.756667in}}{\pgfqpoint{3.241642in}{0.764481in}}%
\pgfpathcurveto{\pgfqpoint{3.249456in}{0.772295in}}{\pgfqpoint{3.253846in}{0.782894in}}{\pgfqpoint{3.253846in}{0.793944in}}%
\pgfpathcurveto{\pgfqpoint{3.253846in}{0.804994in}}{\pgfqpoint{3.249456in}{0.815593in}}{\pgfqpoint{3.241642in}{0.823407in}}%
\pgfpathcurveto{\pgfqpoint{3.233829in}{0.831220in}}{\pgfqpoint{3.223230in}{0.835611in}}{\pgfqpoint{3.212180in}{0.835611in}}%
\pgfpathcurveto{\pgfqpoint{3.201129in}{0.835611in}}{\pgfqpoint{3.190530in}{0.831220in}}{\pgfqpoint{3.182717in}{0.823407in}}%
\pgfpathcurveto{\pgfqpoint{3.174903in}{0.815593in}}{\pgfqpoint{3.170513in}{0.804994in}}{\pgfqpoint{3.170513in}{0.793944in}}%
\pgfpathcurveto{\pgfqpoint{3.170513in}{0.782894in}}{\pgfqpoint{3.174903in}{0.772295in}}{\pgfqpoint{3.182717in}{0.764481in}}%
\pgfpathcurveto{\pgfqpoint{3.190530in}{0.756667in}}{\pgfqpoint{3.201129in}{0.752277in}}{\pgfqpoint{3.212180in}{0.752277in}}%
\pgfpathclose%
\pgfusepath{stroke,fill}%
\end{pgfscope}%
\begin{pgfscope}%
\pgfpathrectangle{\pgfqpoint{0.481978in}{0.331635in}}{\pgfqpoint{4.960000in}{3.696000in}}%
\pgfusepath{clip}%
\pgfsetbuttcap%
\pgfsetroundjoin%
\definecolor{currentfill}{rgb}{1.000000,0.705882,0.509804}%
\pgfsetfillcolor{currentfill}%
\pgfsetlinewidth{0.481800pt}%
\definecolor{currentstroke}{rgb}{1.000000,1.000000,1.000000}%
\pgfsetstrokecolor{currentstroke}%
\pgfsetdash{}{0pt}%
\pgfpathmoveto{\pgfqpoint{3.186086in}{1.662133in}}%
\pgfpathcurveto{\pgfqpoint{3.197136in}{1.662133in}}{\pgfqpoint{3.207735in}{1.666524in}}{\pgfqpoint{3.215548in}{1.674337in}}%
\pgfpathcurveto{\pgfqpoint{3.223362in}{1.682151in}}{\pgfqpoint{3.227752in}{1.692750in}}{\pgfqpoint{3.227752in}{1.703800in}}%
\pgfpathcurveto{\pgfqpoint{3.227752in}{1.714850in}}{\pgfqpoint{3.223362in}{1.725449in}}{\pgfqpoint{3.215548in}{1.733263in}}%
\pgfpathcurveto{\pgfqpoint{3.207735in}{1.741076in}}{\pgfqpoint{3.197136in}{1.745467in}}{\pgfqpoint{3.186086in}{1.745467in}}%
\pgfpathcurveto{\pgfqpoint{3.175036in}{1.745467in}}{\pgfqpoint{3.164436in}{1.741076in}}{\pgfqpoint{3.156623in}{1.733263in}}%
\pgfpathcurveto{\pgfqpoint{3.148809in}{1.725449in}}{\pgfqpoint{3.144419in}{1.714850in}}{\pgfqpoint{3.144419in}{1.703800in}}%
\pgfpathcurveto{\pgfqpoint{3.144419in}{1.692750in}}{\pgfqpoint{3.148809in}{1.682151in}}{\pgfqpoint{3.156623in}{1.674337in}}%
\pgfpathcurveto{\pgfqpoint{3.164436in}{1.666524in}}{\pgfqpoint{3.175036in}{1.662133in}}{\pgfqpoint{3.186086in}{1.662133in}}%
\pgfpathclose%
\pgfusepath{stroke,fill}%
\end{pgfscope}%
\begin{pgfscope}%
\pgfpathrectangle{\pgfqpoint{0.481978in}{0.331635in}}{\pgfqpoint{4.960000in}{3.696000in}}%
\pgfusepath{clip}%
\pgfsetbuttcap%
\pgfsetroundjoin%
\definecolor{currentfill}{rgb}{1.000000,0.705882,0.509804}%
\pgfsetfillcolor{currentfill}%
\pgfsetlinewidth{0.481800pt}%
\definecolor{currentstroke}{rgb}{1.000000,1.000000,1.000000}%
\pgfsetstrokecolor{currentstroke}%
\pgfsetdash{}{0pt}%
\pgfpathmoveto{\pgfqpoint{3.777301in}{2.698807in}}%
\pgfpathcurveto{\pgfqpoint{3.788351in}{2.698807in}}{\pgfqpoint{3.798950in}{2.703197in}}{\pgfqpoint{3.806764in}{2.711010in}}%
\pgfpathcurveto{\pgfqpoint{3.814578in}{2.718824in}}{\pgfqpoint{3.818968in}{2.729423in}}{\pgfqpoint{3.818968in}{2.740473in}}%
\pgfpathcurveto{\pgfqpoint{3.818968in}{2.751523in}}{\pgfqpoint{3.814578in}{2.762122in}}{\pgfqpoint{3.806764in}{2.769936in}}%
\pgfpathcurveto{\pgfqpoint{3.798950in}{2.777750in}}{\pgfqpoint{3.788351in}{2.782140in}}{\pgfqpoint{3.777301in}{2.782140in}}%
\pgfpathcurveto{\pgfqpoint{3.766251in}{2.782140in}}{\pgfqpoint{3.755652in}{2.777750in}}{\pgfqpoint{3.747838in}{2.769936in}}%
\pgfpathcurveto{\pgfqpoint{3.740025in}{2.762122in}}{\pgfqpoint{3.735635in}{2.751523in}}{\pgfqpoint{3.735635in}{2.740473in}}%
\pgfpathcurveto{\pgfqpoint{3.735635in}{2.729423in}}{\pgfqpoint{3.740025in}{2.718824in}}{\pgfqpoint{3.747838in}{2.711010in}}%
\pgfpathcurveto{\pgfqpoint{3.755652in}{2.703197in}}{\pgfqpoint{3.766251in}{2.698807in}}{\pgfqpoint{3.777301in}{2.698807in}}%
\pgfpathclose%
\pgfusepath{stroke,fill}%
\end{pgfscope}%
\begin{pgfscope}%
\pgfpathrectangle{\pgfqpoint{0.481978in}{0.331635in}}{\pgfqpoint{4.960000in}{3.696000in}}%
\pgfusepath{clip}%
\pgfsetbuttcap%
\pgfsetroundjoin%
\definecolor{currentfill}{rgb}{1.000000,0.705882,0.509804}%
\pgfsetfillcolor{currentfill}%
\pgfsetlinewidth{0.481800pt}%
\definecolor{currentstroke}{rgb}{1.000000,1.000000,1.000000}%
\pgfsetstrokecolor{currentstroke}%
\pgfsetdash{}{0pt}%
\pgfpathmoveto{\pgfqpoint{2.325909in}{0.939003in}}%
\pgfpathcurveto{\pgfqpoint{2.336959in}{0.939003in}}{\pgfqpoint{2.347558in}{0.943393in}}{\pgfqpoint{2.355372in}{0.951207in}}%
\pgfpathcurveto{\pgfqpoint{2.363186in}{0.959020in}}{\pgfqpoint{2.367576in}{0.969619in}}{\pgfqpoint{2.367576in}{0.980669in}}%
\pgfpathcurveto{\pgfqpoint{2.367576in}{0.991719in}}{\pgfqpoint{2.363186in}{1.002318in}}{\pgfqpoint{2.355372in}{1.010132in}}%
\pgfpathcurveto{\pgfqpoint{2.347558in}{1.017946in}}{\pgfqpoint{2.336959in}{1.022336in}}{\pgfqpoint{2.325909in}{1.022336in}}%
\pgfpathcurveto{\pgfqpoint{2.314859in}{1.022336in}}{\pgfqpoint{2.304260in}{1.017946in}}{\pgfqpoint{2.296447in}{1.010132in}}%
\pgfpathcurveto{\pgfqpoint{2.288633in}{1.002318in}}{\pgfqpoint{2.284243in}{0.991719in}}{\pgfqpoint{2.284243in}{0.980669in}}%
\pgfpathcurveto{\pgfqpoint{2.284243in}{0.969619in}}{\pgfqpoint{2.288633in}{0.959020in}}{\pgfqpoint{2.296447in}{0.951207in}}%
\pgfpathcurveto{\pgfqpoint{2.304260in}{0.943393in}}{\pgfqpoint{2.314859in}{0.939003in}}{\pgfqpoint{2.325909in}{0.939003in}}%
\pgfpathclose%
\pgfusepath{stroke,fill}%
\end{pgfscope}%
\begin{pgfscope}%
\pgfpathrectangle{\pgfqpoint{0.481978in}{0.331635in}}{\pgfqpoint{4.960000in}{3.696000in}}%
\pgfusepath{clip}%
\pgfsetbuttcap%
\pgfsetroundjoin%
\definecolor{currentfill}{rgb}{1.000000,0.705882,0.509804}%
\pgfsetfillcolor{currentfill}%
\pgfsetlinewidth{0.481800pt}%
\definecolor{currentstroke}{rgb}{1.000000,1.000000,1.000000}%
\pgfsetstrokecolor{currentstroke}%
\pgfsetdash{}{0pt}%
\pgfpathmoveto{\pgfqpoint{1.678605in}{2.689663in}}%
\pgfpathcurveto{\pgfqpoint{1.689655in}{2.689663in}}{\pgfqpoint{1.700254in}{2.694054in}}{\pgfqpoint{1.708068in}{2.701867in}}%
\pgfpathcurveto{\pgfqpoint{1.715881in}{2.709681in}}{\pgfqpoint{1.720272in}{2.720280in}}{\pgfqpoint{1.720272in}{2.731330in}}%
\pgfpathcurveto{\pgfqpoint{1.720272in}{2.742380in}}{\pgfqpoint{1.715881in}{2.752979in}}{\pgfqpoint{1.708068in}{2.760793in}}%
\pgfpathcurveto{\pgfqpoint{1.700254in}{2.768606in}}{\pgfqpoint{1.689655in}{2.772997in}}{\pgfqpoint{1.678605in}{2.772997in}}%
\pgfpathcurveto{\pgfqpoint{1.667555in}{2.772997in}}{\pgfqpoint{1.656956in}{2.768606in}}{\pgfqpoint{1.649142in}{2.760793in}}%
\pgfpathcurveto{\pgfqpoint{1.641329in}{2.752979in}}{\pgfqpoint{1.636938in}{2.742380in}}{\pgfqpoint{1.636938in}{2.731330in}}%
\pgfpathcurveto{\pgfqpoint{1.636938in}{2.720280in}}{\pgfqpoint{1.641329in}{2.709681in}}{\pgfqpoint{1.649142in}{2.701867in}}%
\pgfpathcurveto{\pgfqpoint{1.656956in}{2.694054in}}{\pgfqpoint{1.667555in}{2.689663in}}{\pgfqpoint{1.678605in}{2.689663in}}%
\pgfpathclose%
\pgfusepath{stroke,fill}%
\end{pgfscope}%
\begin{pgfscope}%
\pgfpathrectangle{\pgfqpoint{0.481978in}{0.331635in}}{\pgfqpoint{4.960000in}{3.696000in}}%
\pgfusepath{clip}%
\pgfsetbuttcap%
\pgfsetroundjoin%
\definecolor{currentfill}{rgb}{1.000000,0.705882,0.509804}%
\pgfsetfillcolor{currentfill}%
\pgfsetlinewidth{0.481800pt}%
\definecolor{currentstroke}{rgb}{1.000000,1.000000,1.000000}%
\pgfsetstrokecolor{currentstroke}%
\pgfsetdash{}{0pt}%
\pgfpathmoveto{\pgfqpoint{2.077157in}{0.710207in}}%
\pgfpathcurveto{\pgfqpoint{2.088207in}{0.710207in}}{\pgfqpoint{2.098806in}{0.714597in}}{\pgfqpoint{2.106619in}{0.722411in}}%
\pgfpathcurveto{\pgfqpoint{2.114433in}{0.730224in}}{\pgfqpoint{2.118823in}{0.740823in}}{\pgfqpoint{2.118823in}{0.751873in}}%
\pgfpathcurveto{\pgfqpoint{2.118823in}{0.762924in}}{\pgfqpoint{2.114433in}{0.773523in}}{\pgfqpoint{2.106619in}{0.781336in}}%
\pgfpathcurveto{\pgfqpoint{2.098806in}{0.789150in}}{\pgfqpoint{2.088207in}{0.793540in}}{\pgfqpoint{2.077157in}{0.793540in}}%
\pgfpathcurveto{\pgfqpoint{2.066106in}{0.793540in}}{\pgfqpoint{2.055507in}{0.789150in}}{\pgfqpoint{2.047694in}{0.781336in}}%
\pgfpathcurveto{\pgfqpoint{2.039880in}{0.773523in}}{\pgfqpoint{2.035490in}{0.762924in}}{\pgfqpoint{2.035490in}{0.751873in}}%
\pgfpathcurveto{\pgfqpoint{2.035490in}{0.740823in}}{\pgfqpoint{2.039880in}{0.730224in}}{\pgfqpoint{2.047694in}{0.722411in}}%
\pgfpathcurveto{\pgfqpoint{2.055507in}{0.714597in}}{\pgfqpoint{2.066106in}{0.710207in}}{\pgfqpoint{2.077157in}{0.710207in}}%
\pgfpathclose%
\pgfusepath{stroke,fill}%
\end{pgfscope}%
\begin{pgfscope}%
\pgfpathrectangle{\pgfqpoint{0.481978in}{0.331635in}}{\pgfqpoint{4.960000in}{3.696000in}}%
\pgfusepath{clip}%
\pgfsetbuttcap%
\pgfsetroundjoin%
\definecolor{currentfill}{rgb}{1.000000,0.705882,0.509804}%
\pgfsetfillcolor{currentfill}%
\pgfsetlinewidth{0.481800pt}%
\definecolor{currentstroke}{rgb}{1.000000,1.000000,1.000000}%
\pgfsetstrokecolor{currentstroke}%
\pgfsetdash{}{0pt}%
\pgfpathmoveto{\pgfqpoint{3.318137in}{3.035569in}}%
\pgfpathcurveto{\pgfqpoint{3.329187in}{3.035569in}}{\pgfqpoint{3.339786in}{3.039960in}}{\pgfqpoint{3.347600in}{3.047773in}}%
\pgfpathcurveto{\pgfqpoint{3.355414in}{3.055587in}}{\pgfqpoint{3.359804in}{3.066186in}}{\pgfqpoint{3.359804in}{3.077236in}}%
\pgfpathcurveto{\pgfqpoint{3.359804in}{3.088286in}}{\pgfqpoint{3.355414in}{3.098885in}}{\pgfqpoint{3.347600in}{3.106699in}}%
\pgfpathcurveto{\pgfqpoint{3.339786in}{3.114512in}}{\pgfqpoint{3.329187in}{3.118903in}}{\pgfqpoint{3.318137in}{3.118903in}}%
\pgfpathcurveto{\pgfqpoint{3.307087in}{3.118903in}}{\pgfqpoint{3.296488in}{3.114512in}}{\pgfqpoint{3.288674in}{3.106699in}}%
\pgfpathcurveto{\pgfqpoint{3.280861in}{3.098885in}}{\pgfqpoint{3.276470in}{3.088286in}}{\pgfqpoint{3.276470in}{3.077236in}}%
\pgfpathcurveto{\pgfqpoint{3.276470in}{3.066186in}}{\pgfqpoint{3.280861in}{3.055587in}}{\pgfqpoint{3.288674in}{3.047773in}}%
\pgfpathcurveto{\pgfqpoint{3.296488in}{3.039960in}}{\pgfqpoint{3.307087in}{3.035569in}}{\pgfqpoint{3.318137in}{3.035569in}}%
\pgfpathclose%
\pgfusepath{stroke,fill}%
\end{pgfscope}%
\begin{pgfscope}%
\pgfpathrectangle{\pgfqpoint{0.481978in}{0.331635in}}{\pgfqpoint{4.960000in}{3.696000in}}%
\pgfusepath{clip}%
\pgfsetbuttcap%
\pgfsetroundjoin%
\definecolor{currentfill}{rgb}{1.000000,0.705882,0.509804}%
\pgfsetfillcolor{currentfill}%
\pgfsetlinewidth{0.481800pt}%
\definecolor{currentstroke}{rgb}{1.000000,1.000000,1.000000}%
\pgfsetstrokecolor{currentstroke}%
\pgfsetdash{}{0pt}%
\pgfpathmoveto{\pgfqpoint{1.775533in}{0.767035in}}%
\pgfpathcurveto{\pgfqpoint{1.786583in}{0.767035in}}{\pgfqpoint{1.797182in}{0.771425in}}{\pgfqpoint{1.804996in}{0.779239in}}%
\pgfpathcurveto{\pgfqpoint{1.812809in}{0.787053in}}{\pgfqpoint{1.817199in}{0.797652in}}{\pgfqpoint{1.817199in}{0.808702in}}%
\pgfpathcurveto{\pgfqpoint{1.817199in}{0.819752in}}{\pgfqpoint{1.812809in}{0.830351in}}{\pgfqpoint{1.804996in}{0.838165in}}%
\pgfpathcurveto{\pgfqpoint{1.797182in}{0.845978in}}{\pgfqpoint{1.786583in}{0.850368in}}{\pgfqpoint{1.775533in}{0.850368in}}%
\pgfpathcurveto{\pgfqpoint{1.764483in}{0.850368in}}{\pgfqpoint{1.753884in}{0.845978in}}{\pgfqpoint{1.746070in}{0.838165in}}%
\pgfpathcurveto{\pgfqpoint{1.738256in}{0.830351in}}{\pgfqpoint{1.733866in}{0.819752in}}{\pgfqpoint{1.733866in}{0.808702in}}%
\pgfpathcurveto{\pgfqpoint{1.733866in}{0.797652in}}{\pgfqpoint{1.738256in}{0.787053in}}{\pgfqpoint{1.746070in}{0.779239in}}%
\pgfpathcurveto{\pgfqpoint{1.753884in}{0.771425in}}{\pgfqpoint{1.764483in}{0.767035in}}{\pgfqpoint{1.775533in}{0.767035in}}%
\pgfpathclose%
\pgfusepath{stroke,fill}%
\end{pgfscope}%
\begin{pgfscope}%
\pgfpathrectangle{\pgfqpoint{0.481978in}{0.331635in}}{\pgfqpoint{4.960000in}{3.696000in}}%
\pgfusepath{clip}%
\pgfsetbuttcap%
\pgfsetroundjoin%
\definecolor{currentfill}{rgb}{1.000000,0.705882,0.509804}%
\pgfsetfillcolor{currentfill}%
\pgfsetlinewidth{0.481800pt}%
\definecolor{currentstroke}{rgb}{1.000000,1.000000,1.000000}%
\pgfsetstrokecolor{currentstroke}%
\pgfsetdash{}{0pt}%
\pgfpathmoveto{\pgfqpoint{1.765211in}{2.704215in}}%
\pgfpathcurveto{\pgfqpoint{1.776261in}{2.704215in}}{\pgfqpoint{1.786860in}{2.708606in}}{\pgfqpoint{1.794673in}{2.716419in}}%
\pgfpathcurveto{\pgfqpoint{1.802487in}{2.724233in}}{\pgfqpoint{1.806877in}{2.734832in}}{\pgfqpoint{1.806877in}{2.745882in}}%
\pgfpathcurveto{\pgfqpoint{1.806877in}{2.756932in}}{\pgfqpoint{1.802487in}{2.767531in}}{\pgfqpoint{1.794673in}{2.775345in}}%
\pgfpathcurveto{\pgfqpoint{1.786860in}{2.783158in}}{\pgfqpoint{1.776261in}{2.787549in}}{\pgfqpoint{1.765211in}{2.787549in}}%
\pgfpathcurveto{\pgfqpoint{1.754161in}{2.787549in}}{\pgfqpoint{1.743562in}{2.783158in}}{\pgfqpoint{1.735748in}{2.775345in}}%
\pgfpathcurveto{\pgfqpoint{1.727934in}{2.767531in}}{\pgfqpoint{1.723544in}{2.756932in}}{\pgfqpoint{1.723544in}{2.745882in}}%
\pgfpathcurveto{\pgfqpoint{1.723544in}{2.734832in}}{\pgfqpoint{1.727934in}{2.724233in}}{\pgfqpoint{1.735748in}{2.716419in}}%
\pgfpathcurveto{\pgfqpoint{1.743562in}{2.708606in}}{\pgfqpoint{1.754161in}{2.704215in}}{\pgfqpoint{1.765211in}{2.704215in}}%
\pgfpathclose%
\pgfusepath{stroke,fill}%
\end{pgfscope}%
\begin{pgfscope}%
\pgfpathrectangle{\pgfqpoint{0.481978in}{0.331635in}}{\pgfqpoint{4.960000in}{3.696000in}}%
\pgfusepath{clip}%
\pgfsetbuttcap%
\pgfsetroundjoin%
\definecolor{currentfill}{rgb}{1.000000,0.705882,0.509804}%
\pgfsetfillcolor{currentfill}%
\pgfsetlinewidth{0.481800pt}%
\definecolor{currentstroke}{rgb}{1.000000,1.000000,1.000000}%
\pgfsetstrokecolor{currentstroke}%
\pgfsetdash{}{0pt}%
\pgfpathmoveto{\pgfqpoint{3.513350in}{0.983127in}}%
\pgfpathcurveto{\pgfqpoint{3.524400in}{0.983127in}}{\pgfqpoint{3.534999in}{0.987518in}}{\pgfqpoint{3.542813in}{0.995331in}}%
\pgfpathcurveto{\pgfqpoint{3.550626in}{1.003145in}}{\pgfqpoint{3.555016in}{1.013744in}}{\pgfqpoint{3.555016in}{1.024794in}}%
\pgfpathcurveto{\pgfqpoint{3.555016in}{1.035844in}}{\pgfqpoint{3.550626in}{1.046443in}}{\pgfqpoint{3.542813in}{1.054257in}}%
\pgfpathcurveto{\pgfqpoint{3.534999in}{1.062071in}}{\pgfqpoint{3.524400in}{1.066461in}}{\pgfqpoint{3.513350in}{1.066461in}}%
\pgfpathcurveto{\pgfqpoint{3.502300in}{1.066461in}}{\pgfqpoint{3.491701in}{1.062071in}}{\pgfqpoint{3.483887in}{1.054257in}}%
\pgfpathcurveto{\pgfqpoint{3.476073in}{1.046443in}}{\pgfqpoint{3.471683in}{1.035844in}}{\pgfqpoint{3.471683in}{1.024794in}}%
\pgfpathcurveto{\pgfqpoint{3.471683in}{1.013744in}}{\pgfqpoint{3.476073in}{1.003145in}}{\pgfqpoint{3.483887in}{0.995331in}}%
\pgfpathcurveto{\pgfqpoint{3.491701in}{0.987518in}}{\pgfqpoint{3.502300in}{0.983127in}}{\pgfqpoint{3.513350in}{0.983127in}}%
\pgfpathclose%
\pgfusepath{stroke,fill}%
\end{pgfscope}%
\begin{pgfscope}%
\pgfpathrectangle{\pgfqpoint{0.481978in}{0.331635in}}{\pgfqpoint{4.960000in}{3.696000in}}%
\pgfusepath{clip}%
\pgfsetbuttcap%
\pgfsetroundjoin%
\definecolor{currentfill}{rgb}{1.000000,0.705882,0.509804}%
\pgfsetfillcolor{currentfill}%
\pgfsetlinewidth{0.481800pt}%
\definecolor{currentstroke}{rgb}{1.000000,1.000000,1.000000}%
\pgfsetstrokecolor{currentstroke}%
\pgfsetdash{}{0pt}%
\pgfpathmoveto{\pgfqpoint{3.224603in}{1.295434in}}%
\pgfpathcurveto{\pgfqpoint{3.235653in}{1.295434in}}{\pgfqpoint{3.246252in}{1.299825in}}{\pgfqpoint{3.254066in}{1.307638in}}%
\pgfpathcurveto{\pgfqpoint{3.261880in}{1.315452in}}{\pgfqpoint{3.266270in}{1.326051in}}{\pgfqpoint{3.266270in}{1.337101in}}%
\pgfpathcurveto{\pgfqpoint{3.266270in}{1.348151in}}{\pgfqpoint{3.261880in}{1.358750in}}{\pgfqpoint{3.254066in}{1.366564in}}%
\pgfpathcurveto{\pgfqpoint{3.246252in}{1.374377in}}{\pgfqpoint{3.235653in}{1.378768in}}{\pgfqpoint{3.224603in}{1.378768in}}%
\pgfpathcurveto{\pgfqpoint{3.213553in}{1.378768in}}{\pgfqpoint{3.202954in}{1.374377in}}{\pgfqpoint{3.195141in}{1.366564in}}%
\pgfpathcurveto{\pgfqpoint{3.187327in}{1.358750in}}{\pgfqpoint{3.182937in}{1.348151in}}{\pgfqpoint{3.182937in}{1.337101in}}%
\pgfpathcurveto{\pgfqpoint{3.182937in}{1.326051in}}{\pgfqpoint{3.187327in}{1.315452in}}{\pgfqpoint{3.195141in}{1.307638in}}%
\pgfpathcurveto{\pgfqpoint{3.202954in}{1.299825in}}{\pgfqpoint{3.213553in}{1.295434in}}{\pgfqpoint{3.224603in}{1.295434in}}%
\pgfpathclose%
\pgfusepath{stroke,fill}%
\end{pgfscope}%
\begin{pgfscope}%
\pgfpathrectangle{\pgfqpoint{0.481978in}{0.331635in}}{\pgfqpoint{4.960000in}{3.696000in}}%
\pgfusepath{clip}%
\pgfsetbuttcap%
\pgfsetroundjoin%
\definecolor{currentfill}{rgb}{1.000000,0.705882,0.509804}%
\pgfsetfillcolor{currentfill}%
\pgfsetlinewidth{0.481800pt}%
\definecolor{currentstroke}{rgb}{1.000000,1.000000,1.000000}%
\pgfsetstrokecolor{currentstroke}%
\pgfsetdash{}{0pt}%
\pgfpathmoveto{\pgfqpoint{4.033915in}{1.276398in}}%
\pgfpathcurveto{\pgfqpoint{4.044965in}{1.276398in}}{\pgfqpoint{4.055564in}{1.280789in}}{\pgfqpoint{4.063377in}{1.288602in}}%
\pgfpathcurveto{\pgfqpoint{4.071191in}{1.296416in}}{\pgfqpoint{4.075581in}{1.307015in}}{\pgfqpoint{4.075581in}{1.318065in}}%
\pgfpathcurveto{\pgfqpoint{4.075581in}{1.329115in}}{\pgfqpoint{4.071191in}{1.339714in}}{\pgfqpoint{4.063377in}{1.347528in}}%
\pgfpathcurveto{\pgfqpoint{4.055564in}{1.355342in}}{\pgfqpoint{4.044965in}{1.359732in}}{\pgfqpoint{4.033915in}{1.359732in}}%
\pgfpathcurveto{\pgfqpoint{4.022864in}{1.359732in}}{\pgfqpoint{4.012265in}{1.355342in}}{\pgfqpoint{4.004452in}{1.347528in}}%
\pgfpathcurveto{\pgfqpoint{3.996638in}{1.339714in}}{\pgfqpoint{3.992248in}{1.329115in}}{\pgfqpoint{3.992248in}{1.318065in}}%
\pgfpathcurveto{\pgfqpoint{3.992248in}{1.307015in}}{\pgfqpoint{3.996638in}{1.296416in}}{\pgfqpoint{4.004452in}{1.288602in}}%
\pgfpathcurveto{\pgfqpoint{4.012265in}{1.280789in}}{\pgfqpoint{4.022864in}{1.276398in}}{\pgfqpoint{4.033915in}{1.276398in}}%
\pgfpathclose%
\pgfusepath{stroke,fill}%
\end{pgfscope}%
\begin{pgfscope}%
\pgfpathrectangle{\pgfqpoint{0.481978in}{0.331635in}}{\pgfqpoint{4.960000in}{3.696000in}}%
\pgfusepath{clip}%
\pgfsetbuttcap%
\pgfsetroundjoin%
\definecolor{currentfill}{rgb}{1.000000,0.705882,0.509804}%
\pgfsetfillcolor{currentfill}%
\pgfsetlinewidth{0.481800pt}%
\definecolor{currentstroke}{rgb}{1.000000,1.000000,1.000000}%
\pgfsetstrokecolor{currentstroke}%
\pgfsetdash{}{0pt}%
\pgfpathmoveto{\pgfqpoint{2.683999in}{3.067872in}}%
\pgfpathcurveto{\pgfqpoint{2.695049in}{3.067872in}}{\pgfqpoint{2.705648in}{3.072262in}}{\pgfqpoint{2.713462in}{3.080075in}}%
\pgfpathcurveto{\pgfqpoint{2.721276in}{3.087889in}}{\pgfqpoint{2.725666in}{3.098488in}}{\pgfqpoint{2.725666in}{3.109538in}}%
\pgfpathcurveto{\pgfqpoint{2.725666in}{3.120588in}}{\pgfqpoint{2.721276in}{3.131187in}}{\pgfqpoint{2.713462in}{3.139001in}}%
\pgfpathcurveto{\pgfqpoint{2.705648in}{3.146815in}}{\pgfqpoint{2.695049in}{3.151205in}}{\pgfqpoint{2.683999in}{3.151205in}}%
\pgfpathcurveto{\pgfqpoint{2.672949in}{3.151205in}}{\pgfqpoint{2.662350in}{3.146815in}}{\pgfqpoint{2.654536in}{3.139001in}}%
\pgfpathcurveto{\pgfqpoint{2.646723in}{3.131187in}}{\pgfqpoint{2.642333in}{3.120588in}}{\pgfqpoint{2.642333in}{3.109538in}}%
\pgfpathcurveto{\pgfqpoint{2.642333in}{3.098488in}}{\pgfqpoint{2.646723in}{3.087889in}}{\pgfqpoint{2.654536in}{3.080075in}}%
\pgfpathcurveto{\pgfqpoint{2.662350in}{3.072262in}}{\pgfqpoint{2.672949in}{3.067872in}}{\pgfqpoint{2.683999in}{3.067872in}}%
\pgfpathclose%
\pgfusepath{stroke,fill}%
\end{pgfscope}%
\begin{pgfscope}%
\pgfpathrectangle{\pgfqpoint{0.481978in}{0.331635in}}{\pgfqpoint{4.960000in}{3.696000in}}%
\pgfusepath{clip}%
\pgfsetbuttcap%
\pgfsetroundjoin%
\definecolor{currentfill}{rgb}{1.000000,0.705882,0.509804}%
\pgfsetfillcolor{currentfill}%
\pgfsetlinewidth{0.481800pt}%
\definecolor{currentstroke}{rgb}{1.000000,1.000000,1.000000}%
\pgfsetstrokecolor{currentstroke}%
\pgfsetdash{}{0pt}%
\pgfpathmoveto{\pgfqpoint{2.795687in}{1.212664in}}%
\pgfpathcurveto{\pgfqpoint{2.806737in}{1.212664in}}{\pgfqpoint{2.817336in}{1.217054in}}{\pgfqpoint{2.825149in}{1.224868in}}%
\pgfpathcurveto{\pgfqpoint{2.832963in}{1.232681in}}{\pgfqpoint{2.837353in}{1.243280in}}{\pgfqpoint{2.837353in}{1.254330in}}%
\pgfpathcurveto{\pgfqpoint{2.837353in}{1.265381in}}{\pgfqpoint{2.832963in}{1.275980in}}{\pgfqpoint{2.825149in}{1.283793in}}%
\pgfpathcurveto{\pgfqpoint{2.817336in}{1.291607in}}{\pgfqpoint{2.806737in}{1.295997in}}{\pgfqpoint{2.795687in}{1.295997in}}%
\pgfpathcurveto{\pgfqpoint{2.784636in}{1.295997in}}{\pgfqpoint{2.774037in}{1.291607in}}{\pgfqpoint{2.766224in}{1.283793in}}%
\pgfpathcurveto{\pgfqpoint{2.758410in}{1.275980in}}{\pgfqpoint{2.754020in}{1.265381in}}{\pgfqpoint{2.754020in}{1.254330in}}%
\pgfpathcurveto{\pgfqpoint{2.754020in}{1.243280in}}{\pgfqpoint{2.758410in}{1.232681in}}{\pgfqpoint{2.766224in}{1.224868in}}%
\pgfpathcurveto{\pgfqpoint{2.774037in}{1.217054in}}{\pgfqpoint{2.784636in}{1.212664in}}{\pgfqpoint{2.795687in}{1.212664in}}%
\pgfpathclose%
\pgfusepath{stroke,fill}%
\end{pgfscope}%
\begin{pgfscope}%
\pgfpathrectangle{\pgfqpoint{0.481978in}{0.331635in}}{\pgfqpoint{4.960000in}{3.696000in}}%
\pgfusepath{clip}%
\pgfsetbuttcap%
\pgfsetroundjoin%
\definecolor{currentfill}{rgb}{1.000000,0.705882,0.509804}%
\pgfsetfillcolor{currentfill}%
\pgfsetlinewidth{0.481800pt}%
\definecolor{currentstroke}{rgb}{1.000000,1.000000,1.000000}%
\pgfsetstrokecolor{currentstroke}%
\pgfsetdash{}{0pt}%
\pgfpathmoveto{\pgfqpoint{0.720737in}{2.519626in}}%
\pgfpathcurveto{\pgfqpoint{0.731787in}{2.519626in}}{\pgfqpoint{0.742386in}{2.524016in}}{\pgfqpoint{0.750200in}{2.531829in}}%
\pgfpathcurveto{\pgfqpoint{0.758014in}{2.539643in}}{\pgfqpoint{0.762404in}{2.550242in}}{\pgfqpoint{0.762404in}{2.561292in}}%
\pgfpathcurveto{\pgfqpoint{0.762404in}{2.572342in}}{\pgfqpoint{0.758014in}{2.582941in}}{\pgfqpoint{0.750200in}{2.590755in}}%
\pgfpathcurveto{\pgfqpoint{0.742386in}{2.598569in}}{\pgfqpoint{0.731787in}{2.602959in}}{\pgfqpoint{0.720737in}{2.602959in}}%
\pgfpathcurveto{\pgfqpoint{0.709687in}{2.602959in}}{\pgfqpoint{0.699088in}{2.598569in}}{\pgfqpoint{0.691275in}{2.590755in}}%
\pgfpathcurveto{\pgfqpoint{0.683461in}{2.582941in}}{\pgfqpoint{0.679071in}{2.572342in}}{\pgfqpoint{0.679071in}{2.561292in}}%
\pgfpathcurveto{\pgfqpoint{0.679071in}{2.550242in}}{\pgfqpoint{0.683461in}{2.539643in}}{\pgfqpoint{0.691275in}{2.531829in}}%
\pgfpathcurveto{\pgfqpoint{0.699088in}{2.524016in}}{\pgfqpoint{0.709687in}{2.519626in}}{\pgfqpoint{0.720737in}{2.519626in}}%
\pgfpathclose%
\pgfusepath{stroke,fill}%
\end{pgfscope}%
\begin{pgfscope}%
\pgfpathrectangle{\pgfqpoint{0.481978in}{0.331635in}}{\pgfqpoint{4.960000in}{3.696000in}}%
\pgfusepath{clip}%
\pgfsetbuttcap%
\pgfsetroundjoin%
\definecolor{currentfill}{rgb}{1.000000,0.705882,0.509804}%
\pgfsetfillcolor{currentfill}%
\pgfsetlinewidth{0.481800pt}%
\definecolor{currentstroke}{rgb}{1.000000,1.000000,1.000000}%
\pgfsetstrokecolor{currentstroke}%
\pgfsetdash{}{0pt}%
\pgfpathmoveto{\pgfqpoint{3.665682in}{2.741612in}}%
\pgfpathcurveto{\pgfqpoint{3.676732in}{2.741612in}}{\pgfqpoint{3.687331in}{2.746002in}}{\pgfqpoint{3.695145in}{2.753816in}}%
\pgfpathcurveto{\pgfqpoint{3.702959in}{2.761629in}}{\pgfqpoint{3.707349in}{2.772228in}}{\pgfqpoint{3.707349in}{2.783278in}}%
\pgfpathcurveto{\pgfqpoint{3.707349in}{2.794328in}}{\pgfqpoint{3.702959in}{2.804927in}}{\pgfqpoint{3.695145in}{2.812741in}}%
\pgfpathcurveto{\pgfqpoint{3.687331in}{2.820555in}}{\pgfqpoint{3.676732in}{2.824945in}}{\pgfqpoint{3.665682in}{2.824945in}}%
\pgfpathcurveto{\pgfqpoint{3.654632in}{2.824945in}}{\pgfqpoint{3.644033in}{2.820555in}}{\pgfqpoint{3.636219in}{2.812741in}}%
\pgfpathcurveto{\pgfqpoint{3.628406in}{2.804927in}}{\pgfqpoint{3.624016in}{2.794328in}}{\pgfqpoint{3.624016in}{2.783278in}}%
\pgfpathcurveto{\pgfqpoint{3.624016in}{2.772228in}}{\pgfqpoint{3.628406in}{2.761629in}}{\pgfqpoint{3.636219in}{2.753816in}}%
\pgfpathcurveto{\pgfqpoint{3.644033in}{2.746002in}}{\pgfqpoint{3.654632in}{2.741612in}}{\pgfqpoint{3.665682in}{2.741612in}}%
\pgfpathclose%
\pgfusepath{stroke,fill}%
\end{pgfscope}%
\begin{pgfscope}%
\pgfpathrectangle{\pgfqpoint{0.481978in}{0.331635in}}{\pgfqpoint{4.960000in}{3.696000in}}%
\pgfusepath{clip}%
\pgfsetbuttcap%
\pgfsetroundjoin%
\definecolor{currentfill}{rgb}{1.000000,0.705882,0.509804}%
\pgfsetfillcolor{currentfill}%
\pgfsetlinewidth{0.481800pt}%
\definecolor{currentstroke}{rgb}{1.000000,1.000000,1.000000}%
\pgfsetstrokecolor{currentstroke}%
\pgfsetdash{}{0pt}%
\pgfpathmoveto{\pgfqpoint{2.960200in}{3.579036in}}%
\pgfpathcurveto{\pgfqpoint{2.971250in}{3.579036in}}{\pgfqpoint{2.981849in}{3.583426in}}{\pgfqpoint{2.989662in}{3.591240in}}%
\pgfpathcurveto{\pgfqpoint{2.997476in}{3.599053in}}{\pgfqpoint{3.001866in}{3.609652in}}{\pgfqpoint{3.001866in}{3.620703in}}%
\pgfpathcurveto{\pgfqpoint{3.001866in}{3.631753in}}{\pgfqpoint{2.997476in}{3.642352in}}{\pgfqpoint{2.989662in}{3.650165in}}%
\pgfpathcurveto{\pgfqpoint{2.981849in}{3.657979in}}{\pgfqpoint{2.971250in}{3.662369in}}{\pgfqpoint{2.960200in}{3.662369in}}%
\pgfpathcurveto{\pgfqpoint{2.949150in}{3.662369in}}{\pgfqpoint{2.938551in}{3.657979in}}{\pgfqpoint{2.930737in}{3.650165in}}%
\pgfpathcurveto{\pgfqpoint{2.922923in}{3.642352in}}{\pgfqpoint{2.918533in}{3.631753in}}{\pgfqpoint{2.918533in}{3.620703in}}%
\pgfpathcurveto{\pgfqpoint{2.918533in}{3.609652in}}{\pgfqpoint{2.922923in}{3.599053in}}{\pgfqpoint{2.930737in}{3.591240in}}%
\pgfpathcurveto{\pgfqpoint{2.938551in}{3.583426in}}{\pgfqpoint{2.949150in}{3.579036in}}{\pgfqpoint{2.960200in}{3.579036in}}%
\pgfpathclose%
\pgfusepath{stroke,fill}%
\end{pgfscope}%
\begin{pgfscope}%
\pgfpathrectangle{\pgfqpoint{0.481978in}{0.331635in}}{\pgfqpoint{4.960000in}{3.696000in}}%
\pgfusepath{clip}%
\pgfsetbuttcap%
\pgfsetroundjoin%
\definecolor{currentfill}{rgb}{1.000000,0.705882,0.509804}%
\pgfsetfillcolor{currentfill}%
\pgfsetlinewidth{0.481800pt}%
\definecolor{currentstroke}{rgb}{1.000000,1.000000,1.000000}%
\pgfsetstrokecolor{currentstroke}%
\pgfsetdash{}{0pt}%
\pgfpathmoveto{\pgfqpoint{3.125856in}{0.861942in}}%
\pgfpathcurveto{\pgfqpoint{3.136906in}{0.861942in}}{\pgfqpoint{3.147505in}{0.866332in}}{\pgfqpoint{3.155319in}{0.874146in}}%
\pgfpathcurveto{\pgfqpoint{3.163132in}{0.881959in}}{\pgfqpoint{3.167523in}{0.892558in}}{\pgfqpoint{3.167523in}{0.903608in}}%
\pgfpathcurveto{\pgfqpoint{3.167523in}{0.914659in}}{\pgfqpoint{3.163132in}{0.925258in}}{\pgfqpoint{3.155319in}{0.933071in}}%
\pgfpathcurveto{\pgfqpoint{3.147505in}{0.940885in}}{\pgfqpoint{3.136906in}{0.945275in}}{\pgfqpoint{3.125856in}{0.945275in}}%
\pgfpathcurveto{\pgfqpoint{3.114806in}{0.945275in}}{\pgfqpoint{3.104207in}{0.940885in}}{\pgfqpoint{3.096393in}{0.933071in}}%
\pgfpathcurveto{\pgfqpoint{3.088579in}{0.925258in}}{\pgfqpoint{3.084189in}{0.914659in}}{\pgfqpoint{3.084189in}{0.903608in}}%
\pgfpathcurveto{\pgfqpoint{3.084189in}{0.892558in}}{\pgfqpoint{3.088579in}{0.881959in}}{\pgfqpoint{3.096393in}{0.874146in}}%
\pgfpathcurveto{\pgfqpoint{3.104207in}{0.866332in}}{\pgfqpoint{3.114806in}{0.861942in}}{\pgfqpoint{3.125856in}{0.861942in}}%
\pgfpathclose%
\pgfusepath{stroke,fill}%
\end{pgfscope}%
\begin{pgfscope}%
\pgfpathrectangle{\pgfqpoint{0.481978in}{0.331635in}}{\pgfqpoint{4.960000in}{3.696000in}}%
\pgfusepath{clip}%
\pgfsetbuttcap%
\pgfsetroundjoin%
\definecolor{currentfill}{rgb}{1.000000,0.705882,0.509804}%
\pgfsetfillcolor{currentfill}%
\pgfsetlinewidth{0.481800pt}%
\definecolor{currentstroke}{rgb}{1.000000,1.000000,1.000000}%
\pgfsetstrokecolor{currentstroke}%
\pgfsetdash{}{0pt}%
\pgfpathmoveto{\pgfqpoint{3.821901in}{1.079128in}}%
\pgfpathcurveto{\pgfqpoint{3.832951in}{1.079128in}}{\pgfqpoint{3.843550in}{1.083519in}}{\pgfqpoint{3.851364in}{1.091332in}}%
\pgfpathcurveto{\pgfqpoint{3.859178in}{1.099146in}}{\pgfqpoint{3.863568in}{1.109745in}}{\pgfqpoint{3.863568in}{1.120795in}}%
\pgfpathcurveto{\pgfqpoint{3.863568in}{1.131845in}}{\pgfqpoint{3.859178in}{1.142444in}}{\pgfqpoint{3.851364in}{1.150258in}}%
\pgfpathcurveto{\pgfqpoint{3.843550in}{1.158071in}}{\pgfqpoint{3.832951in}{1.162462in}}{\pgfqpoint{3.821901in}{1.162462in}}%
\pgfpathcurveto{\pgfqpoint{3.810851in}{1.162462in}}{\pgfqpoint{3.800252in}{1.158071in}}{\pgfqpoint{3.792438in}{1.150258in}}%
\pgfpathcurveto{\pgfqpoint{3.784625in}{1.142444in}}{\pgfqpoint{3.780235in}{1.131845in}}{\pgfqpoint{3.780235in}{1.120795in}}%
\pgfpathcurveto{\pgfqpoint{3.780235in}{1.109745in}}{\pgfqpoint{3.784625in}{1.099146in}}{\pgfqpoint{3.792438in}{1.091332in}}%
\pgfpathcurveto{\pgfqpoint{3.800252in}{1.083519in}}{\pgfqpoint{3.810851in}{1.079128in}}{\pgfqpoint{3.821901in}{1.079128in}}%
\pgfpathclose%
\pgfusepath{stroke,fill}%
\end{pgfscope}%
\begin{pgfscope}%
\pgfpathrectangle{\pgfqpoint{0.481978in}{0.331635in}}{\pgfqpoint{4.960000in}{3.696000in}}%
\pgfusepath{clip}%
\pgfsetbuttcap%
\pgfsetroundjoin%
\definecolor{currentfill}{rgb}{1.000000,0.705882,0.509804}%
\pgfsetfillcolor{currentfill}%
\pgfsetlinewidth{0.481800pt}%
\definecolor{currentstroke}{rgb}{1.000000,1.000000,1.000000}%
\pgfsetstrokecolor{currentstroke}%
\pgfsetdash{}{0pt}%
\pgfpathmoveto{\pgfqpoint{2.608776in}{1.769045in}}%
\pgfpathcurveto{\pgfqpoint{2.619826in}{1.769045in}}{\pgfqpoint{2.630425in}{1.773436in}}{\pgfqpoint{2.638239in}{1.781249in}}%
\pgfpathcurveto{\pgfqpoint{2.646053in}{1.789063in}}{\pgfqpoint{2.650443in}{1.799662in}}{\pgfqpoint{2.650443in}{1.810712in}}%
\pgfpathcurveto{\pgfqpoint{2.650443in}{1.821762in}}{\pgfqpoint{2.646053in}{1.832361in}}{\pgfqpoint{2.638239in}{1.840175in}}%
\pgfpathcurveto{\pgfqpoint{2.630425in}{1.847989in}}{\pgfqpoint{2.619826in}{1.852379in}}{\pgfqpoint{2.608776in}{1.852379in}}%
\pgfpathcurveto{\pgfqpoint{2.597726in}{1.852379in}}{\pgfqpoint{2.587127in}{1.847989in}}{\pgfqpoint{2.579314in}{1.840175in}}%
\pgfpathcurveto{\pgfqpoint{2.571500in}{1.832361in}}{\pgfqpoint{2.567110in}{1.821762in}}{\pgfqpoint{2.567110in}{1.810712in}}%
\pgfpathcurveto{\pgfqpoint{2.567110in}{1.799662in}}{\pgfqpoint{2.571500in}{1.789063in}}{\pgfqpoint{2.579314in}{1.781249in}}%
\pgfpathcurveto{\pgfqpoint{2.587127in}{1.773436in}}{\pgfqpoint{2.597726in}{1.769045in}}{\pgfqpoint{2.608776in}{1.769045in}}%
\pgfpathclose%
\pgfusepath{stroke,fill}%
\end{pgfscope}%
\begin{pgfscope}%
\pgfpathrectangle{\pgfqpoint{0.481978in}{0.331635in}}{\pgfqpoint{4.960000in}{3.696000in}}%
\pgfusepath{clip}%
\pgfsetbuttcap%
\pgfsetroundjoin%
\definecolor{currentfill}{rgb}{1.000000,0.705882,0.509804}%
\pgfsetfillcolor{currentfill}%
\pgfsetlinewidth{0.481800pt}%
\definecolor{currentstroke}{rgb}{1.000000,1.000000,1.000000}%
\pgfsetstrokecolor{currentstroke}%
\pgfsetdash{}{0pt}%
\pgfpathmoveto{\pgfqpoint{0.983379in}{2.422005in}}%
\pgfpathcurveto{\pgfqpoint{0.994429in}{2.422005in}}{\pgfqpoint{1.005028in}{2.426395in}}{\pgfqpoint{1.012842in}{2.434209in}}%
\pgfpathcurveto{\pgfqpoint{1.020655in}{2.442022in}}{\pgfqpoint{1.025046in}{2.452621in}}{\pgfqpoint{1.025046in}{2.463671in}}%
\pgfpathcurveto{\pgfqpoint{1.025046in}{2.474721in}}{\pgfqpoint{1.020655in}{2.485320in}}{\pgfqpoint{1.012842in}{2.493134in}}%
\pgfpathcurveto{\pgfqpoint{1.005028in}{2.500948in}}{\pgfqpoint{0.994429in}{2.505338in}}{\pgfqpoint{0.983379in}{2.505338in}}%
\pgfpathcurveto{\pgfqpoint{0.972329in}{2.505338in}}{\pgfqpoint{0.961730in}{2.500948in}}{\pgfqpoint{0.953916in}{2.493134in}}%
\pgfpathcurveto{\pgfqpoint{0.946102in}{2.485320in}}{\pgfqpoint{0.941712in}{2.474721in}}{\pgfqpoint{0.941712in}{2.463671in}}%
\pgfpathcurveto{\pgfqpoint{0.941712in}{2.452621in}}{\pgfqpoint{0.946102in}{2.442022in}}{\pgfqpoint{0.953916in}{2.434209in}}%
\pgfpathcurveto{\pgfqpoint{0.961730in}{2.426395in}}{\pgfqpoint{0.972329in}{2.422005in}}{\pgfqpoint{0.983379in}{2.422005in}}%
\pgfpathclose%
\pgfusepath{stroke,fill}%
\end{pgfscope}%
\begin{pgfscope}%
\pgfpathrectangle{\pgfqpoint{0.481978in}{0.331635in}}{\pgfqpoint{4.960000in}{3.696000in}}%
\pgfusepath{clip}%
\pgfsetbuttcap%
\pgfsetroundjoin%
\definecolor{currentfill}{rgb}{1.000000,0.705882,0.509804}%
\pgfsetfillcolor{currentfill}%
\pgfsetlinewidth{0.481800pt}%
\definecolor{currentstroke}{rgb}{1.000000,1.000000,1.000000}%
\pgfsetstrokecolor{currentstroke}%
\pgfsetdash{}{0pt}%
\pgfpathmoveto{\pgfqpoint{2.508413in}{1.843686in}}%
\pgfpathcurveto{\pgfqpoint{2.519463in}{1.843686in}}{\pgfqpoint{2.530062in}{1.848076in}}{\pgfqpoint{2.537875in}{1.855890in}}%
\pgfpathcurveto{\pgfqpoint{2.545689in}{1.863703in}}{\pgfqpoint{2.550079in}{1.874302in}}{\pgfqpoint{2.550079in}{1.885352in}}%
\pgfpathcurveto{\pgfqpoint{2.550079in}{1.896402in}}{\pgfqpoint{2.545689in}{1.907002in}}{\pgfqpoint{2.537875in}{1.914815in}}%
\pgfpathcurveto{\pgfqpoint{2.530062in}{1.922629in}}{\pgfqpoint{2.519463in}{1.927019in}}{\pgfqpoint{2.508413in}{1.927019in}}%
\pgfpathcurveto{\pgfqpoint{2.497362in}{1.927019in}}{\pgfqpoint{2.486763in}{1.922629in}}{\pgfqpoint{2.478950in}{1.914815in}}%
\pgfpathcurveto{\pgfqpoint{2.471136in}{1.907002in}}{\pgfqpoint{2.466746in}{1.896402in}}{\pgfqpoint{2.466746in}{1.885352in}}%
\pgfpathcurveto{\pgfqpoint{2.466746in}{1.874302in}}{\pgfqpoint{2.471136in}{1.863703in}}{\pgfqpoint{2.478950in}{1.855890in}}%
\pgfpathcurveto{\pgfqpoint{2.486763in}{1.848076in}}{\pgfqpoint{2.497362in}{1.843686in}}{\pgfqpoint{2.508413in}{1.843686in}}%
\pgfpathclose%
\pgfusepath{stroke,fill}%
\end{pgfscope}%
\begin{pgfscope}%
\pgfpathrectangle{\pgfqpoint{0.481978in}{0.331635in}}{\pgfqpoint{4.960000in}{3.696000in}}%
\pgfusepath{clip}%
\pgfsetbuttcap%
\pgfsetroundjoin%
\definecolor{currentfill}{rgb}{1.000000,0.705882,0.509804}%
\pgfsetfillcolor{currentfill}%
\pgfsetlinewidth{0.481800pt}%
\definecolor{currentstroke}{rgb}{1.000000,1.000000,1.000000}%
\pgfsetstrokecolor{currentstroke}%
\pgfsetdash{}{0pt}%
\pgfpathmoveto{\pgfqpoint{2.999772in}{1.443741in}}%
\pgfpathcurveto{\pgfqpoint{3.010822in}{1.443741in}}{\pgfqpoint{3.021421in}{1.448132in}}{\pgfqpoint{3.029235in}{1.455945in}}%
\pgfpathcurveto{\pgfqpoint{3.037048in}{1.463759in}}{\pgfqpoint{3.041439in}{1.474358in}}{\pgfqpoint{3.041439in}{1.485408in}}%
\pgfpathcurveto{\pgfqpoint{3.041439in}{1.496458in}}{\pgfqpoint{3.037048in}{1.507057in}}{\pgfqpoint{3.029235in}{1.514871in}}%
\pgfpathcurveto{\pgfqpoint{3.021421in}{1.522684in}}{\pgfqpoint{3.010822in}{1.527075in}}{\pgfqpoint{2.999772in}{1.527075in}}%
\pgfpathcurveto{\pgfqpoint{2.988722in}{1.527075in}}{\pgfqpoint{2.978123in}{1.522684in}}{\pgfqpoint{2.970309in}{1.514871in}}%
\pgfpathcurveto{\pgfqpoint{2.962496in}{1.507057in}}{\pgfqpoint{2.958105in}{1.496458in}}{\pgfqpoint{2.958105in}{1.485408in}}%
\pgfpathcurveto{\pgfqpoint{2.958105in}{1.474358in}}{\pgfqpoint{2.962496in}{1.463759in}}{\pgfqpoint{2.970309in}{1.455945in}}%
\pgfpathcurveto{\pgfqpoint{2.978123in}{1.448132in}}{\pgfqpoint{2.988722in}{1.443741in}}{\pgfqpoint{2.999772in}{1.443741in}}%
\pgfpathclose%
\pgfusepath{stroke,fill}%
\end{pgfscope}%
\begin{pgfscope}%
\pgfpathrectangle{\pgfqpoint{0.481978in}{0.331635in}}{\pgfqpoint{4.960000in}{3.696000in}}%
\pgfusepath{clip}%
\pgfsetbuttcap%
\pgfsetroundjoin%
\definecolor{currentfill}{rgb}{1.000000,0.705882,0.509804}%
\pgfsetfillcolor{currentfill}%
\pgfsetlinewidth{0.481800pt}%
\definecolor{currentstroke}{rgb}{1.000000,1.000000,1.000000}%
\pgfsetstrokecolor{currentstroke}%
\pgfsetdash{}{0pt}%
\pgfpathmoveto{\pgfqpoint{4.146138in}{1.354972in}}%
\pgfpathcurveto{\pgfqpoint{4.157188in}{1.354972in}}{\pgfqpoint{4.167787in}{1.359362in}}{\pgfqpoint{4.175601in}{1.367176in}}%
\pgfpathcurveto{\pgfqpoint{4.183415in}{1.374990in}}{\pgfqpoint{4.187805in}{1.385589in}}{\pgfqpoint{4.187805in}{1.396639in}}%
\pgfpathcurveto{\pgfqpoint{4.187805in}{1.407689in}}{\pgfqpoint{4.183415in}{1.418288in}}{\pgfqpoint{4.175601in}{1.426102in}}%
\pgfpathcurveto{\pgfqpoint{4.167787in}{1.433915in}}{\pgfqpoint{4.157188in}{1.438305in}}{\pgfqpoint{4.146138in}{1.438305in}}%
\pgfpathcurveto{\pgfqpoint{4.135088in}{1.438305in}}{\pgfqpoint{4.124489in}{1.433915in}}{\pgfqpoint{4.116675in}{1.426102in}}%
\pgfpathcurveto{\pgfqpoint{4.108862in}{1.418288in}}{\pgfqpoint{4.104471in}{1.407689in}}{\pgfqpoint{4.104471in}{1.396639in}}%
\pgfpathcurveto{\pgfqpoint{4.104471in}{1.385589in}}{\pgfqpoint{4.108862in}{1.374990in}}{\pgfqpoint{4.116675in}{1.367176in}}%
\pgfpathcurveto{\pgfqpoint{4.124489in}{1.359362in}}{\pgfqpoint{4.135088in}{1.354972in}}{\pgfqpoint{4.146138in}{1.354972in}}%
\pgfpathclose%
\pgfusepath{stroke,fill}%
\end{pgfscope}%
\begin{pgfscope}%
\pgfpathrectangle{\pgfqpoint{0.481978in}{0.331635in}}{\pgfqpoint{4.960000in}{3.696000in}}%
\pgfusepath{clip}%
\pgfsetbuttcap%
\pgfsetroundjoin%
\definecolor{currentfill}{rgb}{1.000000,0.705882,0.509804}%
\pgfsetfillcolor{currentfill}%
\pgfsetlinewidth{0.481800pt}%
\definecolor{currentstroke}{rgb}{1.000000,1.000000,1.000000}%
\pgfsetstrokecolor{currentstroke}%
\pgfsetdash{}{0pt}%
\pgfpathmoveto{\pgfqpoint{2.588931in}{2.875848in}}%
\pgfpathcurveto{\pgfqpoint{2.599981in}{2.875848in}}{\pgfqpoint{2.610580in}{2.880238in}}{\pgfqpoint{2.618394in}{2.888052in}}%
\pgfpathcurveto{\pgfqpoint{2.626208in}{2.895866in}}{\pgfqpoint{2.630598in}{2.906465in}}{\pgfqpoint{2.630598in}{2.917515in}}%
\pgfpathcurveto{\pgfqpoint{2.630598in}{2.928565in}}{\pgfqpoint{2.626208in}{2.939164in}}{\pgfqpoint{2.618394in}{2.946978in}}%
\pgfpathcurveto{\pgfqpoint{2.610580in}{2.954791in}}{\pgfqpoint{2.599981in}{2.959181in}}{\pgfqpoint{2.588931in}{2.959181in}}%
\pgfpathcurveto{\pgfqpoint{2.577881in}{2.959181in}}{\pgfqpoint{2.567282in}{2.954791in}}{\pgfqpoint{2.559468in}{2.946978in}}%
\pgfpathcurveto{\pgfqpoint{2.551655in}{2.939164in}}{\pgfqpoint{2.547265in}{2.928565in}}{\pgfqpoint{2.547265in}{2.917515in}}%
\pgfpathcurveto{\pgfqpoint{2.547265in}{2.906465in}}{\pgfqpoint{2.551655in}{2.895866in}}{\pgfqpoint{2.559468in}{2.888052in}}%
\pgfpathcurveto{\pgfqpoint{2.567282in}{2.880238in}}{\pgfqpoint{2.577881in}{2.875848in}}{\pgfqpoint{2.588931in}{2.875848in}}%
\pgfpathclose%
\pgfusepath{stroke,fill}%
\end{pgfscope}%
\begin{pgfscope}%
\pgfpathrectangle{\pgfqpoint{0.481978in}{0.331635in}}{\pgfqpoint{4.960000in}{3.696000in}}%
\pgfusepath{clip}%
\pgfsetbuttcap%
\pgfsetroundjoin%
\definecolor{currentfill}{rgb}{1.000000,0.705882,0.509804}%
\pgfsetfillcolor{currentfill}%
\pgfsetlinewidth{0.481800pt}%
\definecolor{currentstroke}{rgb}{1.000000,1.000000,1.000000}%
\pgfsetstrokecolor{currentstroke}%
\pgfsetdash{}{0pt}%
\pgfpathmoveto{\pgfqpoint{2.749278in}{0.864200in}}%
\pgfpathcurveto{\pgfqpoint{2.760328in}{0.864200in}}{\pgfqpoint{2.770927in}{0.868590in}}{\pgfqpoint{2.778740in}{0.876404in}}%
\pgfpathcurveto{\pgfqpoint{2.786554in}{0.884218in}}{\pgfqpoint{2.790944in}{0.894817in}}{\pgfqpoint{2.790944in}{0.905867in}}%
\pgfpathcurveto{\pgfqpoint{2.790944in}{0.916917in}}{\pgfqpoint{2.786554in}{0.927516in}}{\pgfqpoint{2.778740in}{0.935330in}}%
\pgfpathcurveto{\pgfqpoint{2.770927in}{0.943143in}}{\pgfqpoint{2.760328in}{0.947533in}}{\pgfqpoint{2.749278in}{0.947533in}}%
\pgfpathcurveto{\pgfqpoint{2.738228in}{0.947533in}}{\pgfqpoint{2.727629in}{0.943143in}}{\pgfqpoint{2.719815in}{0.935330in}}%
\pgfpathcurveto{\pgfqpoint{2.712001in}{0.927516in}}{\pgfqpoint{2.707611in}{0.916917in}}{\pgfqpoint{2.707611in}{0.905867in}}%
\pgfpathcurveto{\pgfqpoint{2.707611in}{0.894817in}}{\pgfqpoint{2.712001in}{0.884218in}}{\pgfqpoint{2.719815in}{0.876404in}}%
\pgfpathcurveto{\pgfqpoint{2.727629in}{0.868590in}}{\pgfqpoint{2.738228in}{0.864200in}}{\pgfqpoint{2.749278in}{0.864200in}}%
\pgfpathclose%
\pgfusepath{stroke,fill}%
\end{pgfscope}%
\begin{pgfscope}%
\pgfpathrectangle{\pgfqpoint{0.481978in}{0.331635in}}{\pgfqpoint{4.960000in}{3.696000in}}%
\pgfusepath{clip}%
\pgfsetbuttcap%
\pgfsetroundjoin%
\definecolor{currentfill}{rgb}{1.000000,0.705882,0.509804}%
\pgfsetfillcolor{currentfill}%
\pgfsetlinewidth{0.481800pt}%
\definecolor{currentstroke}{rgb}{1.000000,1.000000,1.000000}%
\pgfsetstrokecolor{currentstroke}%
\pgfsetdash{}{0pt}%
\pgfpathmoveto{\pgfqpoint{2.852198in}{1.746271in}}%
\pgfpathcurveto{\pgfqpoint{2.863248in}{1.746271in}}{\pgfqpoint{2.873847in}{1.750661in}}{\pgfqpoint{2.881661in}{1.758475in}}%
\pgfpathcurveto{\pgfqpoint{2.889474in}{1.766289in}}{\pgfqpoint{2.893865in}{1.776888in}}{\pgfqpoint{2.893865in}{1.787938in}}%
\pgfpathcurveto{\pgfqpoint{2.893865in}{1.798988in}}{\pgfqpoint{2.889474in}{1.809587in}}{\pgfqpoint{2.881661in}{1.817401in}}%
\pgfpathcurveto{\pgfqpoint{2.873847in}{1.825214in}}{\pgfqpoint{2.863248in}{1.829604in}}{\pgfqpoint{2.852198in}{1.829604in}}%
\pgfpathcurveto{\pgfqpoint{2.841148in}{1.829604in}}{\pgfqpoint{2.830549in}{1.825214in}}{\pgfqpoint{2.822735in}{1.817401in}}%
\pgfpathcurveto{\pgfqpoint{2.814922in}{1.809587in}}{\pgfqpoint{2.810531in}{1.798988in}}{\pgfqpoint{2.810531in}{1.787938in}}%
\pgfpathcurveto{\pgfqpoint{2.810531in}{1.776888in}}{\pgfqpoint{2.814922in}{1.766289in}}{\pgfqpoint{2.822735in}{1.758475in}}%
\pgfpathcurveto{\pgfqpoint{2.830549in}{1.750661in}}{\pgfqpoint{2.841148in}{1.746271in}}{\pgfqpoint{2.852198in}{1.746271in}}%
\pgfpathclose%
\pgfusepath{stroke,fill}%
\end{pgfscope}%
\begin{pgfscope}%
\pgfpathrectangle{\pgfqpoint{0.481978in}{0.331635in}}{\pgfqpoint{4.960000in}{3.696000in}}%
\pgfusepath{clip}%
\pgfsetbuttcap%
\pgfsetroundjoin%
\definecolor{currentfill}{rgb}{1.000000,0.705882,0.509804}%
\pgfsetfillcolor{currentfill}%
\pgfsetlinewidth{0.481800pt}%
\definecolor{currentstroke}{rgb}{1.000000,1.000000,1.000000}%
\pgfsetstrokecolor{currentstroke}%
\pgfsetdash{}{0pt}%
\pgfpathmoveto{\pgfqpoint{3.111968in}{1.415947in}}%
\pgfpathcurveto{\pgfqpoint{3.123018in}{1.415947in}}{\pgfqpoint{3.133617in}{1.420337in}}{\pgfqpoint{3.141430in}{1.428151in}}%
\pgfpathcurveto{\pgfqpoint{3.149244in}{1.435965in}}{\pgfqpoint{3.153634in}{1.446564in}}{\pgfqpoint{3.153634in}{1.457614in}}%
\pgfpathcurveto{\pgfqpoint{3.153634in}{1.468664in}}{\pgfqpoint{3.149244in}{1.479263in}}{\pgfqpoint{3.141430in}{1.487077in}}%
\pgfpathcurveto{\pgfqpoint{3.133617in}{1.494890in}}{\pgfqpoint{3.123018in}{1.499280in}}{\pgfqpoint{3.111968in}{1.499280in}}%
\pgfpathcurveto{\pgfqpoint{3.100917in}{1.499280in}}{\pgfqpoint{3.090318in}{1.494890in}}{\pgfqpoint{3.082505in}{1.487077in}}%
\pgfpathcurveto{\pgfqpoint{3.074691in}{1.479263in}}{\pgfqpoint{3.070301in}{1.468664in}}{\pgfqpoint{3.070301in}{1.457614in}}%
\pgfpathcurveto{\pgfqpoint{3.070301in}{1.446564in}}{\pgfqpoint{3.074691in}{1.435965in}}{\pgfqpoint{3.082505in}{1.428151in}}%
\pgfpathcurveto{\pgfqpoint{3.090318in}{1.420337in}}{\pgfqpoint{3.100917in}{1.415947in}}{\pgfqpoint{3.111968in}{1.415947in}}%
\pgfpathclose%
\pgfusepath{stroke,fill}%
\end{pgfscope}%
\begin{pgfscope}%
\pgfpathrectangle{\pgfqpoint{0.481978in}{0.331635in}}{\pgfqpoint{4.960000in}{3.696000in}}%
\pgfusepath{clip}%
\pgfsetbuttcap%
\pgfsetroundjoin%
\definecolor{currentfill}{rgb}{1.000000,0.705882,0.509804}%
\pgfsetfillcolor{currentfill}%
\pgfsetlinewidth{0.481800pt}%
\definecolor{currentstroke}{rgb}{1.000000,1.000000,1.000000}%
\pgfsetstrokecolor{currentstroke}%
\pgfsetdash{}{0pt}%
\pgfpathmoveto{\pgfqpoint{3.717258in}{2.911112in}}%
\pgfpathcurveto{\pgfqpoint{3.728308in}{2.911112in}}{\pgfqpoint{3.738907in}{2.915502in}}{\pgfqpoint{3.746721in}{2.923316in}}%
\pgfpathcurveto{\pgfqpoint{3.754535in}{2.931129in}}{\pgfqpoint{3.758925in}{2.941728in}}{\pgfqpoint{3.758925in}{2.952778in}}%
\pgfpathcurveto{\pgfqpoint{3.758925in}{2.963828in}}{\pgfqpoint{3.754535in}{2.974427in}}{\pgfqpoint{3.746721in}{2.982241in}}%
\pgfpathcurveto{\pgfqpoint{3.738907in}{2.990055in}}{\pgfqpoint{3.728308in}{2.994445in}}{\pgfqpoint{3.717258in}{2.994445in}}%
\pgfpathcurveto{\pgfqpoint{3.706208in}{2.994445in}}{\pgfqpoint{3.695609in}{2.990055in}}{\pgfqpoint{3.687796in}{2.982241in}}%
\pgfpathcurveto{\pgfqpoint{3.679982in}{2.974427in}}{\pgfqpoint{3.675592in}{2.963828in}}{\pgfqpoint{3.675592in}{2.952778in}}%
\pgfpathcurveto{\pgfqpoint{3.675592in}{2.941728in}}{\pgfqpoint{3.679982in}{2.931129in}}{\pgfqpoint{3.687796in}{2.923316in}}%
\pgfpathcurveto{\pgfqpoint{3.695609in}{2.915502in}}{\pgfqpoint{3.706208in}{2.911112in}}{\pgfqpoint{3.717258in}{2.911112in}}%
\pgfpathclose%
\pgfusepath{stroke,fill}%
\end{pgfscope}%
\begin{pgfscope}%
\pgfpathrectangle{\pgfqpoint{0.481978in}{0.331635in}}{\pgfqpoint{4.960000in}{3.696000in}}%
\pgfusepath{clip}%
\pgfsetbuttcap%
\pgfsetroundjoin%
\definecolor{currentfill}{rgb}{1.000000,0.705882,0.509804}%
\pgfsetfillcolor{currentfill}%
\pgfsetlinewidth{0.481800pt}%
\definecolor{currentstroke}{rgb}{1.000000,1.000000,1.000000}%
\pgfsetstrokecolor{currentstroke}%
\pgfsetdash{}{0pt}%
\pgfpathmoveto{\pgfqpoint{2.466644in}{1.245969in}}%
\pgfpathcurveto{\pgfqpoint{2.477694in}{1.245969in}}{\pgfqpoint{2.488293in}{1.250359in}}{\pgfqpoint{2.496106in}{1.258172in}}%
\pgfpathcurveto{\pgfqpoint{2.503920in}{1.265986in}}{\pgfqpoint{2.508310in}{1.276585in}}{\pgfqpoint{2.508310in}{1.287635in}}%
\pgfpathcurveto{\pgfqpoint{2.508310in}{1.298685in}}{\pgfqpoint{2.503920in}{1.309284in}}{\pgfqpoint{2.496106in}{1.317098in}}%
\pgfpathcurveto{\pgfqpoint{2.488293in}{1.324912in}}{\pgfqpoint{2.477694in}{1.329302in}}{\pgfqpoint{2.466644in}{1.329302in}}%
\pgfpathcurveto{\pgfqpoint{2.455594in}{1.329302in}}{\pgfqpoint{2.444995in}{1.324912in}}{\pgfqpoint{2.437181in}{1.317098in}}%
\pgfpathcurveto{\pgfqpoint{2.429367in}{1.309284in}}{\pgfqpoint{2.424977in}{1.298685in}}{\pgfqpoint{2.424977in}{1.287635in}}%
\pgfpathcurveto{\pgfqpoint{2.424977in}{1.276585in}}{\pgfqpoint{2.429367in}{1.265986in}}{\pgfqpoint{2.437181in}{1.258172in}}%
\pgfpathcurveto{\pgfqpoint{2.444995in}{1.250359in}}{\pgfqpoint{2.455594in}{1.245969in}}{\pgfqpoint{2.466644in}{1.245969in}}%
\pgfpathclose%
\pgfusepath{stroke,fill}%
\end{pgfscope}%
\begin{pgfscope}%
\pgfpathrectangle{\pgfqpoint{0.481978in}{0.331635in}}{\pgfqpoint{4.960000in}{3.696000in}}%
\pgfusepath{clip}%
\pgfsetbuttcap%
\pgfsetroundjoin%
\definecolor{currentfill}{rgb}{1.000000,0.705882,0.509804}%
\pgfsetfillcolor{currentfill}%
\pgfsetlinewidth{0.481800pt}%
\definecolor{currentstroke}{rgb}{1.000000,1.000000,1.000000}%
\pgfsetstrokecolor{currentstroke}%
\pgfsetdash{}{0pt}%
\pgfpathmoveto{\pgfqpoint{2.803563in}{2.890478in}}%
\pgfpathcurveto{\pgfqpoint{2.814613in}{2.890478in}}{\pgfqpoint{2.825212in}{2.894868in}}{\pgfqpoint{2.833026in}{2.902682in}}%
\pgfpathcurveto{\pgfqpoint{2.840840in}{2.910495in}}{\pgfqpoint{2.845230in}{2.921094in}}{\pgfqpoint{2.845230in}{2.932145in}}%
\pgfpathcurveto{\pgfqpoint{2.845230in}{2.943195in}}{\pgfqpoint{2.840840in}{2.953794in}}{\pgfqpoint{2.833026in}{2.961607in}}%
\pgfpathcurveto{\pgfqpoint{2.825212in}{2.969421in}}{\pgfqpoint{2.814613in}{2.973811in}}{\pgfqpoint{2.803563in}{2.973811in}}%
\pgfpathcurveto{\pgfqpoint{2.792513in}{2.973811in}}{\pgfqpoint{2.781914in}{2.969421in}}{\pgfqpoint{2.774100in}{2.961607in}}%
\pgfpathcurveto{\pgfqpoint{2.766287in}{2.953794in}}{\pgfqpoint{2.761896in}{2.943195in}}{\pgfqpoint{2.761896in}{2.932145in}}%
\pgfpathcurveto{\pgfqpoint{2.761896in}{2.921094in}}{\pgfqpoint{2.766287in}{2.910495in}}{\pgfqpoint{2.774100in}{2.902682in}}%
\pgfpathcurveto{\pgfqpoint{2.781914in}{2.894868in}}{\pgfqpoint{2.792513in}{2.890478in}}{\pgfqpoint{2.803563in}{2.890478in}}%
\pgfpathclose%
\pgfusepath{stroke,fill}%
\end{pgfscope}%
\begin{pgfscope}%
\pgfpathrectangle{\pgfqpoint{0.481978in}{0.331635in}}{\pgfqpoint{4.960000in}{3.696000in}}%
\pgfusepath{clip}%
\pgfsetbuttcap%
\pgfsetroundjoin%
\definecolor{currentfill}{rgb}{1.000000,0.705882,0.509804}%
\pgfsetfillcolor{currentfill}%
\pgfsetlinewidth{0.481800pt}%
\definecolor{currentstroke}{rgb}{1.000000,1.000000,1.000000}%
\pgfsetstrokecolor{currentstroke}%
\pgfsetdash{}{0pt}%
\pgfpathmoveto{\pgfqpoint{3.672387in}{2.732861in}}%
\pgfpathcurveto{\pgfqpoint{3.683437in}{2.732861in}}{\pgfqpoint{3.694036in}{2.737251in}}{\pgfqpoint{3.701850in}{2.745065in}}%
\pgfpathcurveto{\pgfqpoint{3.709663in}{2.752878in}}{\pgfqpoint{3.714054in}{2.763478in}}{\pgfqpoint{3.714054in}{2.774528in}}%
\pgfpathcurveto{\pgfqpoint{3.714054in}{2.785578in}}{\pgfqpoint{3.709663in}{2.796177in}}{\pgfqpoint{3.701850in}{2.803990in}}%
\pgfpathcurveto{\pgfqpoint{3.694036in}{2.811804in}}{\pgfqpoint{3.683437in}{2.816194in}}{\pgfqpoint{3.672387in}{2.816194in}}%
\pgfpathcurveto{\pgfqpoint{3.661337in}{2.816194in}}{\pgfqpoint{3.650738in}{2.811804in}}{\pgfqpoint{3.642924in}{2.803990in}}%
\pgfpathcurveto{\pgfqpoint{3.635111in}{2.796177in}}{\pgfqpoint{3.630720in}{2.785578in}}{\pgfqpoint{3.630720in}{2.774528in}}%
\pgfpathcurveto{\pgfqpoint{3.630720in}{2.763478in}}{\pgfqpoint{3.635111in}{2.752878in}}{\pgfqpoint{3.642924in}{2.745065in}}%
\pgfpathcurveto{\pgfqpoint{3.650738in}{2.737251in}}{\pgfqpoint{3.661337in}{2.732861in}}{\pgfqpoint{3.672387in}{2.732861in}}%
\pgfpathclose%
\pgfusepath{stroke,fill}%
\end{pgfscope}%
\begin{pgfscope}%
\pgfpathrectangle{\pgfqpoint{0.481978in}{0.331635in}}{\pgfqpoint{4.960000in}{3.696000in}}%
\pgfusepath{clip}%
\pgfsetbuttcap%
\pgfsetroundjoin%
\definecolor{currentfill}{rgb}{1.000000,0.705882,0.509804}%
\pgfsetfillcolor{currentfill}%
\pgfsetlinewidth{0.481800pt}%
\definecolor{currentstroke}{rgb}{1.000000,1.000000,1.000000}%
\pgfsetstrokecolor{currentstroke}%
\pgfsetdash{}{0pt}%
\pgfpathmoveto{\pgfqpoint{4.195542in}{2.721693in}}%
\pgfpathcurveto{\pgfqpoint{4.206592in}{2.721693in}}{\pgfqpoint{4.217191in}{2.726083in}}{\pgfqpoint{4.225004in}{2.733896in}}%
\pgfpathcurveto{\pgfqpoint{4.232818in}{2.741710in}}{\pgfqpoint{4.237208in}{2.752309in}}{\pgfqpoint{4.237208in}{2.763359in}}%
\pgfpathcurveto{\pgfqpoint{4.237208in}{2.774409in}}{\pgfqpoint{4.232818in}{2.785008in}}{\pgfqpoint{4.225004in}{2.792822in}}%
\pgfpathcurveto{\pgfqpoint{4.217191in}{2.800636in}}{\pgfqpoint{4.206592in}{2.805026in}}{\pgfqpoint{4.195542in}{2.805026in}}%
\pgfpathcurveto{\pgfqpoint{4.184491in}{2.805026in}}{\pgfqpoint{4.173892in}{2.800636in}}{\pgfqpoint{4.166079in}{2.792822in}}%
\pgfpathcurveto{\pgfqpoint{4.158265in}{2.785008in}}{\pgfqpoint{4.153875in}{2.774409in}}{\pgfqpoint{4.153875in}{2.763359in}}%
\pgfpathcurveto{\pgfqpoint{4.153875in}{2.752309in}}{\pgfqpoint{4.158265in}{2.741710in}}{\pgfqpoint{4.166079in}{2.733896in}}%
\pgfpathcurveto{\pgfqpoint{4.173892in}{2.726083in}}{\pgfqpoint{4.184491in}{2.721693in}}{\pgfqpoint{4.195542in}{2.721693in}}%
\pgfpathclose%
\pgfusepath{stroke,fill}%
\end{pgfscope}%
\begin{pgfscope}%
\pgfpathrectangle{\pgfqpoint{0.481978in}{0.331635in}}{\pgfqpoint{4.960000in}{3.696000in}}%
\pgfusepath{clip}%
\pgfsetbuttcap%
\pgfsetroundjoin%
\definecolor{currentfill}{rgb}{1.000000,0.705882,0.509804}%
\pgfsetfillcolor{currentfill}%
\pgfsetlinewidth{0.481800pt}%
\definecolor{currentstroke}{rgb}{1.000000,1.000000,1.000000}%
\pgfsetstrokecolor{currentstroke}%
\pgfsetdash{}{0pt}%
\pgfpathmoveto{\pgfqpoint{2.935072in}{1.620984in}}%
\pgfpathcurveto{\pgfqpoint{2.946122in}{1.620984in}}{\pgfqpoint{2.956721in}{1.625374in}}{\pgfqpoint{2.964534in}{1.633188in}}%
\pgfpathcurveto{\pgfqpoint{2.972348in}{1.641002in}}{\pgfqpoint{2.976738in}{1.651601in}}{\pgfqpoint{2.976738in}{1.662651in}}%
\pgfpathcurveto{\pgfqpoint{2.976738in}{1.673701in}}{\pgfqpoint{2.972348in}{1.684300in}}{\pgfqpoint{2.964534in}{1.692114in}}%
\pgfpathcurveto{\pgfqpoint{2.956721in}{1.699927in}}{\pgfqpoint{2.946122in}{1.704318in}}{\pgfqpoint{2.935072in}{1.704318in}}%
\pgfpathcurveto{\pgfqpoint{2.924022in}{1.704318in}}{\pgfqpoint{2.913423in}{1.699927in}}{\pgfqpoint{2.905609in}{1.692114in}}%
\pgfpathcurveto{\pgfqpoint{2.897795in}{1.684300in}}{\pgfqpoint{2.893405in}{1.673701in}}{\pgfqpoint{2.893405in}{1.662651in}}%
\pgfpathcurveto{\pgfqpoint{2.893405in}{1.651601in}}{\pgfqpoint{2.897795in}{1.641002in}}{\pgfqpoint{2.905609in}{1.633188in}}%
\pgfpathcurveto{\pgfqpoint{2.913423in}{1.625374in}}{\pgfqpoint{2.924022in}{1.620984in}}{\pgfqpoint{2.935072in}{1.620984in}}%
\pgfpathclose%
\pgfusepath{stroke,fill}%
\end{pgfscope}%
\begin{pgfscope}%
\pgfpathrectangle{\pgfqpoint{0.481978in}{0.331635in}}{\pgfqpoint{4.960000in}{3.696000in}}%
\pgfusepath{clip}%
\pgfsetbuttcap%
\pgfsetroundjoin%
\definecolor{currentfill}{rgb}{1.000000,0.705882,0.509804}%
\pgfsetfillcolor{currentfill}%
\pgfsetlinewidth{0.481800pt}%
\definecolor{currentstroke}{rgb}{1.000000,1.000000,1.000000}%
\pgfsetstrokecolor{currentstroke}%
\pgfsetdash{}{0pt}%
\pgfpathmoveto{\pgfqpoint{2.676919in}{1.319416in}}%
\pgfpathcurveto{\pgfqpoint{2.687969in}{1.319416in}}{\pgfqpoint{2.698569in}{1.323806in}}{\pgfqpoint{2.706382in}{1.331619in}}%
\pgfpathcurveto{\pgfqpoint{2.714196in}{1.339433in}}{\pgfqpoint{2.718586in}{1.350032in}}{\pgfqpoint{2.718586in}{1.361082in}}%
\pgfpathcurveto{\pgfqpoint{2.718586in}{1.372132in}}{\pgfqpoint{2.714196in}{1.382731in}}{\pgfqpoint{2.706382in}{1.390545in}}%
\pgfpathcurveto{\pgfqpoint{2.698569in}{1.398359in}}{\pgfqpoint{2.687969in}{1.402749in}}{\pgfqpoint{2.676919in}{1.402749in}}%
\pgfpathcurveto{\pgfqpoint{2.665869in}{1.402749in}}{\pgfqpoint{2.655270in}{1.398359in}}{\pgfqpoint{2.647457in}{1.390545in}}%
\pgfpathcurveto{\pgfqpoint{2.639643in}{1.382731in}}{\pgfqpoint{2.635253in}{1.372132in}}{\pgfqpoint{2.635253in}{1.361082in}}%
\pgfpathcurveto{\pgfqpoint{2.635253in}{1.350032in}}{\pgfqpoint{2.639643in}{1.339433in}}{\pgfqpoint{2.647457in}{1.331619in}}%
\pgfpathcurveto{\pgfqpoint{2.655270in}{1.323806in}}{\pgfqpoint{2.665869in}{1.319416in}}{\pgfqpoint{2.676919in}{1.319416in}}%
\pgfpathclose%
\pgfusepath{stroke,fill}%
\end{pgfscope}%
\begin{pgfscope}%
\pgfpathrectangle{\pgfqpoint{0.481978in}{0.331635in}}{\pgfqpoint{4.960000in}{3.696000in}}%
\pgfusepath{clip}%
\pgfsetbuttcap%
\pgfsetroundjoin%
\definecolor{currentfill}{rgb}{1.000000,0.705882,0.509804}%
\pgfsetfillcolor{currentfill}%
\pgfsetlinewidth{0.481800pt}%
\definecolor{currentstroke}{rgb}{1.000000,1.000000,1.000000}%
\pgfsetstrokecolor{currentstroke}%
\pgfsetdash{}{0pt}%
\pgfpathmoveto{\pgfqpoint{2.794210in}{2.723700in}}%
\pgfpathcurveto{\pgfqpoint{2.805261in}{2.723700in}}{\pgfqpoint{2.815860in}{2.728090in}}{\pgfqpoint{2.823673in}{2.735904in}}%
\pgfpathcurveto{\pgfqpoint{2.831487in}{2.743718in}}{\pgfqpoint{2.835877in}{2.754317in}}{\pgfqpoint{2.835877in}{2.765367in}}%
\pgfpathcurveto{\pgfqpoint{2.835877in}{2.776417in}}{\pgfqpoint{2.831487in}{2.787016in}}{\pgfqpoint{2.823673in}{2.794830in}}%
\pgfpathcurveto{\pgfqpoint{2.815860in}{2.802643in}}{\pgfqpoint{2.805261in}{2.807033in}}{\pgfqpoint{2.794210in}{2.807033in}}%
\pgfpathcurveto{\pgfqpoint{2.783160in}{2.807033in}}{\pgfqpoint{2.772561in}{2.802643in}}{\pgfqpoint{2.764748in}{2.794830in}}%
\pgfpathcurveto{\pgfqpoint{2.756934in}{2.787016in}}{\pgfqpoint{2.752544in}{2.776417in}}{\pgfqpoint{2.752544in}{2.765367in}}%
\pgfpathcurveto{\pgfqpoint{2.752544in}{2.754317in}}{\pgfqpoint{2.756934in}{2.743718in}}{\pgfqpoint{2.764748in}{2.735904in}}%
\pgfpathcurveto{\pgfqpoint{2.772561in}{2.728090in}}{\pgfqpoint{2.783160in}{2.723700in}}{\pgfqpoint{2.794210in}{2.723700in}}%
\pgfpathclose%
\pgfusepath{stroke,fill}%
\end{pgfscope}%
\begin{pgfscope}%
\pgfpathrectangle{\pgfqpoint{0.481978in}{0.331635in}}{\pgfqpoint{4.960000in}{3.696000in}}%
\pgfusepath{clip}%
\pgfsetbuttcap%
\pgfsetroundjoin%
\definecolor{currentfill}{rgb}{1.000000,0.705882,0.509804}%
\pgfsetfillcolor{currentfill}%
\pgfsetlinewidth{0.481800pt}%
\definecolor{currentstroke}{rgb}{1.000000,1.000000,1.000000}%
\pgfsetstrokecolor{currentstroke}%
\pgfsetdash{}{0pt}%
\pgfpathmoveto{\pgfqpoint{3.191851in}{1.127460in}}%
\pgfpathcurveto{\pgfqpoint{3.202901in}{1.127460in}}{\pgfqpoint{3.213500in}{1.131850in}}{\pgfqpoint{3.221314in}{1.139664in}}%
\pgfpathcurveto{\pgfqpoint{3.229127in}{1.147477in}}{\pgfqpoint{3.233518in}{1.158076in}}{\pgfqpoint{3.233518in}{1.169127in}}%
\pgfpathcurveto{\pgfqpoint{3.233518in}{1.180177in}}{\pgfqpoint{3.229127in}{1.190776in}}{\pgfqpoint{3.221314in}{1.198589in}}%
\pgfpathcurveto{\pgfqpoint{3.213500in}{1.206403in}}{\pgfqpoint{3.202901in}{1.210793in}}{\pgfqpoint{3.191851in}{1.210793in}}%
\pgfpathcurveto{\pgfqpoint{3.180801in}{1.210793in}}{\pgfqpoint{3.170202in}{1.206403in}}{\pgfqpoint{3.162388in}{1.198589in}}%
\pgfpathcurveto{\pgfqpoint{3.154575in}{1.190776in}}{\pgfqpoint{3.150184in}{1.180177in}}{\pgfqpoint{3.150184in}{1.169127in}}%
\pgfpathcurveto{\pgfqpoint{3.150184in}{1.158076in}}{\pgfqpoint{3.154575in}{1.147477in}}{\pgfqpoint{3.162388in}{1.139664in}}%
\pgfpathcurveto{\pgfqpoint{3.170202in}{1.131850in}}{\pgfqpoint{3.180801in}{1.127460in}}{\pgfqpoint{3.191851in}{1.127460in}}%
\pgfpathclose%
\pgfusepath{stroke,fill}%
\end{pgfscope}%
\begin{pgfscope}%
\pgfpathrectangle{\pgfqpoint{0.481978in}{0.331635in}}{\pgfqpoint{4.960000in}{3.696000in}}%
\pgfusepath{clip}%
\pgfsetbuttcap%
\pgfsetroundjoin%
\definecolor{currentfill}{rgb}{1.000000,0.705882,0.509804}%
\pgfsetfillcolor{currentfill}%
\pgfsetlinewidth{0.481800pt}%
\definecolor{currentstroke}{rgb}{1.000000,1.000000,1.000000}%
\pgfsetstrokecolor{currentstroke}%
\pgfsetdash{}{0pt}%
\pgfpathmoveto{\pgfqpoint{2.491019in}{1.338041in}}%
\pgfpathcurveto{\pgfqpoint{2.502069in}{1.338041in}}{\pgfqpoint{2.512668in}{1.342431in}}{\pgfqpoint{2.520482in}{1.350245in}}%
\pgfpathcurveto{\pgfqpoint{2.528295in}{1.358059in}}{\pgfqpoint{2.532686in}{1.368658in}}{\pgfqpoint{2.532686in}{1.379708in}}%
\pgfpathcurveto{\pgfqpoint{2.532686in}{1.390758in}}{\pgfqpoint{2.528295in}{1.401357in}}{\pgfqpoint{2.520482in}{1.409170in}}%
\pgfpathcurveto{\pgfqpoint{2.512668in}{1.416984in}}{\pgfqpoint{2.502069in}{1.421374in}}{\pgfqpoint{2.491019in}{1.421374in}}%
\pgfpathcurveto{\pgfqpoint{2.479969in}{1.421374in}}{\pgfqpoint{2.469370in}{1.416984in}}{\pgfqpoint{2.461556in}{1.409170in}}%
\pgfpathcurveto{\pgfqpoint{2.453743in}{1.401357in}}{\pgfqpoint{2.449352in}{1.390758in}}{\pgfqpoint{2.449352in}{1.379708in}}%
\pgfpathcurveto{\pgfqpoint{2.449352in}{1.368658in}}{\pgfqpoint{2.453743in}{1.358059in}}{\pgfqpoint{2.461556in}{1.350245in}}%
\pgfpathcurveto{\pgfqpoint{2.469370in}{1.342431in}}{\pgfqpoint{2.479969in}{1.338041in}}{\pgfqpoint{2.491019in}{1.338041in}}%
\pgfpathclose%
\pgfusepath{stroke,fill}%
\end{pgfscope}%
\begin{pgfscope}%
\pgfpathrectangle{\pgfqpoint{0.481978in}{0.331635in}}{\pgfqpoint{4.960000in}{3.696000in}}%
\pgfusepath{clip}%
\pgfsetbuttcap%
\pgfsetroundjoin%
\definecolor{currentfill}{rgb}{1.000000,0.705882,0.509804}%
\pgfsetfillcolor{currentfill}%
\pgfsetlinewidth{0.481800pt}%
\definecolor{currentstroke}{rgb}{1.000000,1.000000,1.000000}%
\pgfsetstrokecolor{currentstroke}%
\pgfsetdash{}{0pt}%
\pgfpathmoveto{\pgfqpoint{0.871762in}{1.803600in}}%
\pgfpathcurveto{\pgfqpoint{0.882812in}{1.803600in}}{\pgfqpoint{0.893411in}{1.807990in}}{\pgfqpoint{0.901225in}{1.815804in}}%
\pgfpathcurveto{\pgfqpoint{0.909039in}{1.823617in}}{\pgfqpoint{0.913429in}{1.834216in}}{\pgfqpoint{0.913429in}{1.845266in}}%
\pgfpathcurveto{\pgfqpoint{0.913429in}{1.856316in}}{\pgfqpoint{0.909039in}{1.866915in}}{\pgfqpoint{0.901225in}{1.874729in}}%
\pgfpathcurveto{\pgfqpoint{0.893411in}{1.882543in}}{\pgfqpoint{0.882812in}{1.886933in}}{\pgfqpoint{0.871762in}{1.886933in}}%
\pgfpathcurveto{\pgfqpoint{0.860712in}{1.886933in}}{\pgfqpoint{0.850113in}{1.882543in}}{\pgfqpoint{0.842300in}{1.874729in}}%
\pgfpathcurveto{\pgfqpoint{0.834486in}{1.866915in}}{\pgfqpoint{0.830096in}{1.856316in}}{\pgfqpoint{0.830096in}{1.845266in}}%
\pgfpathcurveto{\pgfqpoint{0.830096in}{1.834216in}}{\pgfqpoint{0.834486in}{1.823617in}}{\pgfqpoint{0.842300in}{1.815804in}}%
\pgfpathcurveto{\pgfqpoint{0.850113in}{1.807990in}}{\pgfqpoint{0.860712in}{1.803600in}}{\pgfqpoint{0.871762in}{1.803600in}}%
\pgfpathclose%
\pgfusepath{stroke,fill}%
\end{pgfscope}%
\begin{pgfscope}%
\pgfpathrectangle{\pgfqpoint{0.481978in}{0.331635in}}{\pgfqpoint{4.960000in}{3.696000in}}%
\pgfusepath{clip}%
\pgfsetbuttcap%
\pgfsetroundjoin%
\definecolor{currentfill}{rgb}{1.000000,0.705882,0.509804}%
\pgfsetfillcolor{currentfill}%
\pgfsetlinewidth{0.481800pt}%
\definecolor{currentstroke}{rgb}{1.000000,1.000000,1.000000}%
\pgfsetstrokecolor{currentstroke}%
\pgfsetdash{}{0pt}%
\pgfpathmoveto{\pgfqpoint{1.354219in}{1.930655in}}%
\pgfpathcurveto{\pgfqpoint{1.365269in}{1.930655in}}{\pgfqpoint{1.375868in}{1.935046in}}{\pgfqpoint{1.383681in}{1.942859in}}%
\pgfpathcurveto{\pgfqpoint{1.391495in}{1.950673in}}{\pgfqpoint{1.395885in}{1.961272in}}{\pgfqpoint{1.395885in}{1.972322in}}%
\pgfpathcurveto{\pgfqpoint{1.395885in}{1.983372in}}{\pgfqpoint{1.391495in}{1.993971in}}{\pgfqpoint{1.383681in}{2.001785in}}%
\pgfpathcurveto{\pgfqpoint{1.375868in}{2.009599in}}{\pgfqpoint{1.365269in}{2.013989in}}{\pgfqpoint{1.354219in}{2.013989in}}%
\pgfpathcurveto{\pgfqpoint{1.343169in}{2.013989in}}{\pgfqpoint{1.332570in}{2.009599in}}{\pgfqpoint{1.324756in}{2.001785in}}%
\pgfpathcurveto{\pgfqpoint{1.316942in}{1.993971in}}{\pgfqpoint{1.312552in}{1.983372in}}{\pgfqpoint{1.312552in}{1.972322in}}%
\pgfpathcurveto{\pgfqpoint{1.312552in}{1.961272in}}{\pgfqpoint{1.316942in}{1.950673in}}{\pgfqpoint{1.324756in}{1.942859in}}%
\pgfpathcurveto{\pgfqpoint{1.332570in}{1.935046in}}{\pgfqpoint{1.343169in}{1.930655in}}{\pgfqpoint{1.354219in}{1.930655in}}%
\pgfpathclose%
\pgfusepath{stroke,fill}%
\end{pgfscope}%
\begin{pgfscope}%
\pgfpathrectangle{\pgfqpoint{0.481978in}{0.331635in}}{\pgfqpoint{4.960000in}{3.696000in}}%
\pgfusepath{clip}%
\pgfsetbuttcap%
\pgfsetroundjoin%
\definecolor{currentfill}{rgb}{1.000000,0.705882,0.509804}%
\pgfsetfillcolor{currentfill}%
\pgfsetlinewidth{0.481800pt}%
\definecolor{currentstroke}{rgb}{1.000000,1.000000,1.000000}%
\pgfsetstrokecolor{currentstroke}%
\pgfsetdash{}{0pt}%
\pgfpathmoveto{\pgfqpoint{4.550338in}{2.781428in}}%
\pgfpathcurveto{\pgfqpoint{4.561388in}{2.781428in}}{\pgfqpoint{4.571987in}{2.785818in}}{\pgfqpoint{4.579801in}{2.793632in}}%
\pgfpathcurveto{\pgfqpoint{4.587614in}{2.801446in}}{\pgfqpoint{4.592005in}{2.812045in}}{\pgfqpoint{4.592005in}{2.823095in}}%
\pgfpathcurveto{\pgfqpoint{4.592005in}{2.834145in}}{\pgfqpoint{4.587614in}{2.844744in}}{\pgfqpoint{4.579801in}{2.852558in}}%
\pgfpathcurveto{\pgfqpoint{4.571987in}{2.860371in}}{\pgfqpoint{4.561388in}{2.864762in}}{\pgfqpoint{4.550338in}{2.864762in}}%
\pgfpathcurveto{\pgfqpoint{4.539288in}{2.864762in}}{\pgfqpoint{4.528689in}{2.860371in}}{\pgfqpoint{4.520875in}{2.852558in}}%
\pgfpathcurveto{\pgfqpoint{4.513062in}{2.844744in}}{\pgfqpoint{4.508671in}{2.834145in}}{\pgfqpoint{4.508671in}{2.823095in}}%
\pgfpathcurveto{\pgfqpoint{4.508671in}{2.812045in}}{\pgfqpoint{4.513062in}{2.801446in}}{\pgfqpoint{4.520875in}{2.793632in}}%
\pgfpathcurveto{\pgfqpoint{4.528689in}{2.785818in}}{\pgfqpoint{4.539288in}{2.781428in}}{\pgfqpoint{4.550338in}{2.781428in}}%
\pgfpathclose%
\pgfusepath{stroke,fill}%
\end{pgfscope}%
\begin{pgfscope}%
\pgfpathrectangle{\pgfqpoint{0.481978in}{0.331635in}}{\pgfqpoint{4.960000in}{3.696000in}}%
\pgfusepath{clip}%
\pgfsetbuttcap%
\pgfsetroundjoin%
\definecolor{currentfill}{rgb}{1.000000,0.705882,0.509804}%
\pgfsetfillcolor{currentfill}%
\pgfsetlinewidth{0.481800pt}%
\definecolor{currentstroke}{rgb}{1.000000,1.000000,1.000000}%
\pgfsetstrokecolor{currentstroke}%
\pgfsetdash{}{0pt}%
\pgfpathmoveto{\pgfqpoint{4.171300in}{2.654743in}}%
\pgfpathcurveto{\pgfqpoint{4.182350in}{2.654743in}}{\pgfqpoint{4.192949in}{2.659133in}}{\pgfqpoint{4.200763in}{2.666947in}}%
\pgfpathcurveto{\pgfqpoint{4.208576in}{2.674760in}}{\pgfqpoint{4.212967in}{2.685359in}}{\pgfqpoint{4.212967in}{2.696409in}}%
\pgfpathcurveto{\pgfqpoint{4.212967in}{2.707460in}}{\pgfqpoint{4.208576in}{2.718059in}}{\pgfqpoint{4.200763in}{2.725872in}}%
\pgfpathcurveto{\pgfqpoint{4.192949in}{2.733686in}}{\pgfqpoint{4.182350in}{2.738076in}}{\pgfqpoint{4.171300in}{2.738076in}}%
\pgfpathcurveto{\pgfqpoint{4.160250in}{2.738076in}}{\pgfqpoint{4.149651in}{2.733686in}}{\pgfqpoint{4.141837in}{2.725872in}}%
\pgfpathcurveto{\pgfqpoint{4.134024in}{2.718059in}}{\pgfqpoint{4.129633in}{2.707460in}}{\pgfqpoint{4.129633in}{2.696409in}}%
\pgfpathcurveto{\pgfqpoint{4.129633in}{2.685359in}}{\pgfqpoint{4.134024in}{2.674760in}}{\pgfqpoint{4.141837in}{2.666947in}}%
\pgfpathcurveto{\pgfqpoint{4.149651in}{2.659133in}}{\pgfqpoint{4.160250in}{2.654743in}}{\pgfqpoint{4.171300in}{2.654743in}}%
\pgfpathclose%
\pgfusepath{stroke,fill}%
\end{pgfscope}%
\begin{pgfscope}%
\pgfpathrectangle{\pgfqpoint{0.481978in}{0.331635in}}{\pgfqpoint{4.960000in}{3.696000in}}%
\pgfusepath{clip}%
\pgfsetbuttcap%
\pgfsetroundjoin%
\definecolor{currentfill}{rgb}{1.000000,0.705882,0.509804}%
\pgfsetfillcolor{currentfill}%
\pgfsetlinewidth{0.481800pt}%
\definecolor{currentstroke}{rgb}{1.000000,1.000000,1.000000}%
\pgfsetstrokecolor{currentstroke}%
\pgfsetdash{}{0pt}%
\pgfpathmoveto{\pgfqpoint{1.370027in}{1.679456in}}%
\pgfpathcurveto{\pgfqpoint{1.381077in}{1.679456in}}{\pgfqpoint{1.391676in}{1.683846in}}{\pgfqpoint{1.399489in}{1.691659in}}%
\pgfpathcurveto{\pgfqpoint{1.407303in}{1.699473in}}{\pgfqpoint{1.411693in}{1.710072in}}{\pgfqpoint{1.411693in}{1.721122in}}%
\pgfpathcurveto{\pgfqpoint{1.411693in}{1.732172in}}{\pgfqpoint{1.407303in}{1.742771in}}{\pgfqpoint{1.399489in}{1.750585in}}%
\pgfpathcurveto{\pgfqpoint{1.391676in}{1.758399in}}{\pgfqpoint{1.381077in}{1.762789in}}{\pgfqpoint{1.370027in}{1.762789in}}%
\pgfpathcurveto{\pgfqpoint{1.358976in}{1.762789in}}{\pgfqpoint{1.348377in}{1.758399in}}{\pgfqpoint{1.340564in}{1.750585in}}%
\pgfpathcurveto{\pgfqpoint{1.332750in}{1.742771in}}{\pgfqpoint{1.328360in}{1.732172in}}{\pgfqpoint{1.328360in}{1.721122in}}%
\pgfpathcurveto{\pgfqpoint{1.328360in}{1.710072in}}{\pgfqpoint{1.332750in}{1.699473in}}{\pgfqpoint{1.340564in}{1.691659in}}%
\pgfpathcurveto{\pgfqpoint{1.348377in}{1.683846in}}{\pgfqpoint{1.358976in}{1.679456in}}{\pgfqpoint{1.370027in}{1.679456in}}%
\pgfpathclose%
\pgfusepath{stroke,fill}%
\end{pgfscope}%
\begin{pgfscope}%
\pgfpathrectangle{\pgfqpoint{0.481978in}{0.331635in}}{\pgfqpoint{4.960000in}{3.696000in}}%
\pgfusepath{clip}%
\pgfsetbuttcap%
\pgfsetroundjoin%
\definecolor{currentfill}{rgb}{1.000000,0.705882,0.509804}%
\pgfsetfillcolor{currentfill}%
\pgfsetlinewidth{0.481800pt}%
\definecolor{currentstroke}{rgb}{1.000000,1.000000,1.000000}%
\pgfsetstrokecolor{currentstroke}%
\pgfsetdash{}{0pt}%
\pgfpathmoveto{\pgfqpoint{5.216523in}{1.558638in}}%
\pgfpathcurveto{\pgfqpoint{5.227574in}{1.558638in}}{\pgfqpoint{5.238173in}{1.563028in}}{\pgfqpoint{5.245986in}{1.570842in}}%
\pgfpathcurveto{\pgfqpoint{5.253800in}{1.578655in}}{\pgfqpoint{5.258190in}{1.589254in}}{\pgfqpoint{5.258190in}{1.600304in}}%
\pgfpathcurveto{\pgfqpoint{5.258190in}{1.611354in}}{\pgfqpoint{5.253800in}{1.621953in}}{\pgfqpoint{5.245986in}{1.629767in}}%
\pgfpathcurveto{\pgfqpoint{5.238173in}{1.637581in}}{\pgfqpoint{5.227574in}{1.641971in}}{\pgfqpoint{5.216523in}{1.641971in}}%
\pgfpathcurveto{\pgfqpoint{5.205473in}{1.641971in}}{\pgfqpoint{5.194874in}{1.637581in}}{\pgfqpoint{5.187061in}{1.629767in}}%
\pgfpathcurveto{\pgfqpoint{5.179247in}{1.621953in}}{\pgfqpoint{5.174857in}{1.611354in}}{\pgfqpoint{5.174857in}{1.600304in}}%
\pgfpathcurveto{\pgfqpoint{5.174857in}{1.589254in}}{\pgfqpoint{5.179247in}{1.578655in}}{\pgfqpoint{5.187061in}{1.570842in}}%
\pgfpathcurveto{\pgfqpoint{5.194874in}{1.563028in}}{\pgfqpoint{5.205473in}{1.558638in}}{\pgfqpoint{5.216523in}{1.558638in}}%
\pgfpathclose%
\pgfusepath{stroke,fill}%
\end{pgfscope}%
\begin{pgfscope}%
\pgfpathrectangle{\pgfqpoint{0.481978in}{0.331635in}}{\pgfqpoint{4.960000in}{3.696000in}}%
\pgfusepath{clip}%
\pgfsetbuttcap%
\pgfsetroundjoin%
\definecolor{currentfill}{rgb}{1.000000,0.705882,0.509804}%
\pgfsetfillcolor{currentfill}%
\pgfsetlinewidth{0.481800pt}%
\definecolor{currentstroke}{rgb}{1.000000,1.000000,1.000000}%
\pgfsetstrokecolor{currentstroke}%
\pgfsetdash{}{0pt}%
\pgfpathmoveto{\pgfqpoint{2.099960in}{1.722444in}}%
\pgfpathcurveto{\pgfqpoint{2.111011in}{1.722444in}}{\pgfqpoint{2.121610in}{1.726834in}}{\pgfqpoint{2.129423in}{1.734647in}}%
\pgfpathcurveto{\pgfqpoint{2.137237in}{1.742461in}}{\pgfqpoint{2.141627in}{1.753060in}}{\pgfqpoint{2.141627in}{1.764110in}}%
\pgfpathcurveto{\pgfqpoint{2.141627in}{1.775160in}}{\pgfqpoint{2.137237in}{1.785759in}}{\pgfqpoint{2.129423in}{1.793573in}}%
\pgfpathcurveto{\pgfqpoint{2.121610in}{1.801387in}}{\pgfqpoint{2.111011in}{1.805777in}}{\pgfqpoint{2.099960in}{1.805777in}}%
\pgfpathcurveto{\pgfqpoint{2.088910in}{1.805777in}}{\pgfqpoint{2.078311in}{1.801387in}}{\pgfqpoint{2.070498in}{1.793573in}}%
\pgfpathcurveto{\pgfqpoint{2.062684in}{1.785759in}}{\pgfqpoint{2.058294in}{1.775160in}}{\pgfqpoint{2.058294in}{1.764110in}}%
\pgfpathcurveto{\pgfqpoint{2.058294in}{1.753060in}}{\pgfqpoint{2.062684in}{1.742461in}}{\pgfqpoint{2.070498in}{1.734647in}}%
\pgfpathcurveto{\pgfqpoint{2.078311in}{1.726834in}}{\pgfqpoint{2.088910in}{1.722444in}}{\pgfqpoint{2.099960in}{1.722444in}}%
\pgfpathclose%
\pgfusepath{stroke,fill}%
\end{pgfscope}%
\begin{pgfscope}%
\pgfpathrectangle{\pgfqpoint{0.481978in}{0.331635in}}{\pgfqpoint{4.960000in}{3.696000in}}%
\pgfusepath{clip}%
\pgfsetbuttcap%
\pgfsetroundjoin%
\definecolor{currentfill}{rgb}{1.000000,0.705882,0.509804}%
\pgfsetfillcolor{currentfill}%
\pgfsetlinewidth{0.481800pt}%
\definecolor{currentstroke}{rgb}{1.000000,1.000000,1.000000}%
\pgfsetstrokecolor{currentstroke}%
\pgfsetdash{}{0pt}%
\pgfpathmoveto{\pgfqpoint{3.292277in}{3.198403in}}%
\pgfpathcurveto{\pgfqpoint{3.303327in}{3.198403in}}{\pgfqpoint{3.313926in}{3.202793in}}{\pgfqpoint{3.321739in}{3.210607in}}%
\pgfpathcurveto{\pgfqpoint{3.329553in}{3.218421in}}{\pgfqpoint{3.333943in}{3.229020in}}{\pgfqpoint{3.333943in}{3.240070in}}%
\pgfpathcurveto{\pgfqpoint{3.333943in}{3.251120in}}{\pgfqpoint{3.329553in}{3.261719in}}{\pgfqpoint{3.321739in}{3.269533in}}%
\pgfpathcurveto{\pgfqpoint{3.313926in}{3.277346in}}{\pgfqpoint{3.303327in}{3.281736in}}{\pgfqpoint{3.292277in}{3.281736in}}%
\pgfpathcurveto{\pgfqpoint{3.281226in}{3.281736in}}{\pgfqpoint{3.270627in}{3.277346in}}{\pgfqpoint{3.262814in}{3.269533in}}%
\pgfpathcurveto{\pgfqpoint{3.255000in}{3.261719in}}{\pgfqpoint{3.250610in}{3.251120in}}{\pgfqpoint{3.250610in}{3.240070in}}%
\pgfpathcurveto{\pgfqpoint{3.250610in}{3.229020in}}{\pgfqpoint{3.255000in}{3.218421in}}{\pgfqpoint{3.262814in}{3.210607in}}%
\pgfpathcurveto{\pgfqpoint{3.270627in}{3.202793in}}{\pgfqpoint{3.281226in}{3.198403in}}{\pgfqpoint{3.292277in}{3.198403in}}%
\pgfpathclose%
\pgfusepath{stroke,fill}%
\end{pgfscope}%
\begin{pgfscope}%
\pgfpathrectangle{\pgfqpoint{0.481978in}{0.331635in}}{\pgfqpoint{4.960000in}{3.696000in}}%
\pgfusepath{clip}%
\pgfsetbuttcap%
\pgfsetroundjoin%
\definecolor{currentfill}{rgb}{1.000000,0.705882,0.509804}%
\pgfsetfillcolor{currentfill}%
\pgfsetlinewidth{0.481800pt}%
\definecolor{currentstroke}{rgb}{1.000000,1.000000,1.000000}%
\pgfsetstrokecolor{currentstroke}%
\pgfsetdash{}{0pt}%
\pgfpathmoveto{\pgfqpoint{3.994608in}{2.535787in}}%
\pgfpathcurveto{\pgfqpoint{4.005658in}{2.535787in}}{\pgfqpoint{4.016257in}{2.540177in}}{\pgfqpoint{4.024071in}{2.547991in}}%
\pgfpathcurveto{\pgfqpoint{4.031885in}{2.555805in}}{\pgfqpoint{4.036275in}{2.566404in}}{\pgfqpoint{4.036275in}{2.577454in}}%
\pgfpathcurveto{\pgfqpoint{4.036275in}{2.588504in}}{\pgfqpoint{4.031885in}{2.599103in}}{\pgfqpoint{4.024071in}{2.606916in}}%
\pgfpathcurveto{\pgfqpoint{4.016257in}{2.614730in}}{\pgfqpoint{4.005658in}{2.619120in}}{\pgfqpoint{3.994608in}{2.619120in}}%
\pgfpathcurveto{\pgfqpoint{3.983558in}{2.619120in}}{\pgfqpoint{3.972959in}{2.614730in}}{\pgfqpoint{3.965145in}{2.606916in}}%
\pgfpathcurveto{\pgfqpoint{3.957332in}{2.599103in}}{\pgfqpoint{3.952941in}{2.588504in}}{\pgfqpoint{3.952941in}{2.577454in}}%
\pgfpathcurveto{\pgfqpoint{3.952941in}{2.566404in}}{\pgfqpoint{3.957332in}{2.555805in}}{\pgfqpoint{3.965145in}{2.547991in}}%
\pgfpathcurveto{\pgfqpoint{3.972959in}{2.540177in}}{\pgfqpoint{3.983558in}{2.535787in}}{\pgfqpoint{3.994608in}{2.535787in}}%
\pgfpathclose%
\pgfusepath{stroke,fill}%
\end{pgfscope}%
\begin{pgfscope}%
\pgfpathrectangle{\pgfqpoint{0.481978in}{0.331635in}}{\pgfqpoint{4.960000in}{3.696000in}}%
\pgfusepath{clip}%
\pgfsetbuttcap%
\pgfsetroundjoin%
\definecolor{currentfill}{rgb}{1.000000,0.705882,0.509804}%
\pgfsetfillcolor{currentfill}%
\pgfsetlinewidth{0.481800pt}%
\definecolor{currentstroke}{rgb}{1.000000,1.000000,1.000000}%
\pgfsetstrokecolor{currentstroke}%
\pgfsetdash{}{0pt}%
\pgfpathmoveto{\pgfqpoint{2.854723in}{1.212831in}}%
\pgfpathcurveto{\pgfqpoint{2.865773in}{1.212831in}}{\pgfqpoint{2.876373in}{1.217221in}}{\pgfqpoint{2.884186in}{1.225035in}}%
\pgfpathcurveto{\pgfqpoint{2.892000in}{1.232849in}}{\pgfqpoint{2.896390in}{1.243448in}}{\pgfqpoint{2.896390in}{1.254498in}}%
\pgfpathcurveto{\pgfqpoint{2.896390in}{1.265548in}}{\pgfqpoint{2.892000in}{1.276147in}}{\pgfqpoint{2.884186in}{1.283961in}}%
\pgfpathcurveto{\pgfqpoint{2.876373in}{1.291774in}}{\pgfqpoint{2.865773in}{1.296165in}}{\pgfqpoint{2.854723in}{1.296165in}}%
\pgfpathcurveto{\pgfqpoint{2.843673in}{1.296165in}}{\pgfqpoint{2.833074in}{1.291774in}}{\pgfqpoint{2.825261in}{1.283961in}}%
\pgfpathcurveto{\pgfqpoint{2.817447in}{1.276147in}}{\pgfqpoint{2.813057in}{1.265548in}}{\pgfqpoint{2.813057in}{1.254498in}}%
\pgfpathcurveto{\pgfqpoint{2.813057in}{1.243448in}}{\pgfqpoint{2.817447in}{1.232849in}}{\pgfqpoint{2.825261in}{1.225035in}}%
\pgfpathcurveto{\pgfqpoint{2.833074in}{1.217221in}}{\pgfqpoint{2.843673in}{1.212831in}}{\pgfqpoint{2.854723in}{1.212831in}}%
\pgfpathclose%
\pgfusepath{stroke,fill}%
\end{pgfscope}%
\begin{pgfscope}%
\pgfpathrectangle{\pgfqpoint{0.481978in}{0.331635in}}{\pgfqpoint{4.960000in}{3.696000in}}%
\pgfusepath{clip}%
\pgfsetbuttcap%
\pgfsetroundjoin%
\definecolor{currentfill}{rgb}{1.000000,0.705882,0.509804}%
\pgfsetfillcolor{currentfill}%
\pgfsetlinewidth{0.481800pt}%
\definecolor{currentstroke}{rgb}{1.000000,1.000000,1.000000}%
\pgfsetstrokecolor{currentstroke}%
\pgfsetdash{}{0pt}%
\pgfpathmoveto{\pgfqpoint{1.725916in}{3.532275in}}%
\pgfpathcurveto{\pgfqpoint{1.736966in}{3.532275in}}{\pgfqpoint{1.747565in}{3.536665in}}{\pgfqpoint{1.755378in}{3.544478in}}%
\pgfpathcurveto{\pgfqpoint{1.763192in}{3.552292in}}{\pgfqpoint{1.767582in}{3.562891in}}{\pgfqpoint{1.767582in}{3.573941in}}%
\pgfpathcurveto{\pgfqpoint{1.767582in}{3.584991in}}{\pgfqpoint{1.763192in}{3.595590in}}{\pgfqpoint{1.755378in}{3.603404in}}%
\pgfpathcurveto{\pgfqpoint{1.747565in}{3.611218in}}{\pgfqpoint{1.736966in}{3.615608in}}{\pgfqpoint{1.725916in}{3.615608in}}%
\pgfpathcurveto{\pgfqpoint{1.714865in}{3.615608in}}{\pgfqpoint{1.704266in}{3.611218in}}{\pgfqpoint{1.696453in}{3.603404in}}%
\pgfpathcurveto{\pgfqpoint{1.688639in}{3.595590in}}{\pgfqpoint{1.684249in}{3.584991in}}{\pgfqpoint{1.684249in}{3.573941in}}%
\pgfpathcurveto{\pgfqpoint{1.684249in}{3.562891in}}{\pgfqpoint{1.688639in}{3.552292in}}{\pgfqpoint{1.696453in}{3.544478in}}%
\pgfpathcurveto{\pgfqpoint{1.704266in}{3.536665in}}{\pgfqpoint{1.714865in}{3.532275in}}{\pgfqpoint{1.725916in}{3.532275in}}%
\pgfpathclose%
\pgfusepath{stroke,fill}%
\end{pgfscope}%
\begin{pgfscope}%
\pgfpathrectangle{\pgfqpoint{0.481978in}{0.331635in}}{\pgfqpoint{4.960000in}{3.696000in}}%
\pgfusepath{clip}%
\pgfsetbuttcap%
\pgfsetroundjoin%
\definecolor{currentfill}{rgb}{1.000000,0.705882,0.509804}%
\pgfsetfillcolor{currentfill}%
\pgfsetlinewidth{0.481800pt}%
\definecolor{currentstroke}{rgb}{1.000000,1.000000,1.000000}%
\pgfsetstrokecolor{currentstroke}%
\pgfsetdash{}{0pt}%
\pgfpathmoveto{\pgfqpoint{3.356000in}{0.981517in}}%
\pgfpathcurveto{\pgfqpoint{3.367050in}{0.981517in}}{\pgfqpoint{3.377649in}{0.985907in}}{\pgfqpoint{3.385462in}{0.993721in}}%
\pgfpathcurveto{\pgfqpoint{3.393276in}{1.001534in}}{\pgfqpoint{3.397666in}{1.012133in}}{\pgfqpoint{3.397666in}{1.023183in}}%
\pgfpathcurveto{\pgfqpoint{3.397666in}{1.034233in}}{\pgfqpoint{3.393276in}{1.044833in}}{\pgfqpoint{3.385462in}{1.052646in}}%
\pgfpathcurveto{\pgfqpoint{3.377649in}{1.060460in}}{\pgfqpoint{3.367050in}{1.064850in}}{\pgfqpoint{3.356000in}{1.064850in}}%
\pgfpathcurveto{\pgfqpoint{3.344950in}{1.064850in}}{\pgfqpoint{3.334351in}{1.060460in}}{\pgfqpoint{3.326537in}{1.052646in}}%
\pgfpathcurveto{\pgfqpoint{3.318723in}{1.044833in}}{\pgfqpoint{3.314333in}{1.034233in}}{\pgfqpoint{3.314333in}{1.023183in}}%
\pgfpathcurveto{\pgfqpoint{3.314333in}{1.012133in}}{\pgfqpoint{3.318723in}{1.001534in}}{\pgfqpoint{3.326537in}{0.993721in}}%
\pgfpathcurveto{\pgfqpoint{3.334351in}{0.985907in}}{\pgfqpoint{3.344950in}{0.981517in}}{\pgfqpoint{3.356000in}{0.981517in}}%
\pgfpathclose%
\pgfusepath{stroke,fill}%
\end{pgfscope}%
\begin{pgfscope}%
\pgfpathrectangle{\pgfqpoint{0.481978in}{0.331635in}}{\pgfqpoint{4.960000in}{3.696000in}}%
\pgfusepath{clip}%
\pgfsetbuttcap%
\pgfsetroundjoin%
\definecolor{currentfill}{rgb}{1.000000,0.705882,0.509804}%
\pgfsetfillcolor{currentfill}%
\pgfsetlinewidth{0.481800pt}%
\definecolor{currentstroke}{rgb}{1.000000,1.000000,1.000000}%
\pgfsetstrokecolor{currentstroke}%
\pgfsetdash{}{0pt}%
\pgfpathmoveto{\pgfqpoint{2.839959in}{1.522006in}}%
\pgfpathcurveto{\pgfqpoint{2.851009in}{1.522006in}}{\pgfqpoint{2.861608in}{1.526396in}}{\pgfqpoint{2.869422in}{1.534210in}}%
\pgfpathcurveto{\pgfqpoint{2.877236in}{1.542023in}}{\pgfqpoint{2.881626in}{1.552622in}}{\pgfqpoint{2.881626in}{1.563673in}}%
\pgfpathcurveto{\pgfqpoint{2.881626in}{1.574723in}}{\pgfqpoint{2.877236in}{1.585322in}}{\pgfqpoint{2.869422in}{1.593135in}}%
\pgfpathcurveto{\pgfqpoint{2.861608in}{1.600949in}}{\pgfqpoint{2.851009in}{1.605339in}}{\pgfqpoint{2.839959in}{1.605339in}}%
\pgfpathcurveto{\pgfqpoint{2.828909in}{1.605339in}}{\pgfqpoint{2.818310in}{1.600949in}}{\pgfqpoint{2.810496in}{1.593135in}}%
\pgfpathcurveto{\pgfqpoint{2.802683in}{1.585322in}}{\pgfqpoint{2.798293in}{1.574723in}}{\pgfqpoint{2.798293in}{1.563673in}}%
\pgfpathcurveto{\pgfqpoint{2.798293in}{1.552622in}}{\pgfqpoint{2.802683in}{1.542023in}}{\pgfqpoint{2.810496in}{1.534210in}}%
\pgfpathcurveto{\pgfqpoint{2.818310in}{1.526396in}}{\pgfqpoint{2.828909in}{1.522006in}}{\pgfqpoint{2.839959in}{1.522006in}}%
\pgfpathclose%
\pgfusepath{stroke,fill}%
\end{pgfscope}%
\begin{pgfscope}%
\pgfpathrectangle{\pgfqpoint{0.481978in}{0.331635in}}{\pgfqpoint{4.960000in}{3.696000in}}%
\pgfusepath{clip}%
\pgfsetbuttcap%
\pgfsetroundjoin%
\definecolor{currentfill}{rgb}{1.000000,0.705882,0.509804}%
\pgfsetfillcolor{currentfill}%
\pgfsetlinewidth{0.481800pt}%
\definecolor{currentstroke}{rgb}{1.000000,1.000000,1.000000}%
\pgfsetstrokecolor{currentstroke}%
\pgfsetdash{}{0pt}%
\pgfpathmoveto{\pgfqpoint{3.583295in}{2.995671in}}%
\pgfpathcurveto{\pgfqpoint{3.594345in}{2.995671in}}{\pgfqpoint{3.604944in}{3.000061in}}{\pgfqpoint{3.612757in}{3.007875in}}%
\pgfpathcurveto{\pgfqpoint{3.620571in}{3.015688in}}{\pgfqpoint{3.624961in}{3.026287in}}{\pgfqpoint{3.624961in}{3.037337in}}%
\pgfpathcurveto{\pgfqpoint{3.624961in}{3.048387in}}{\pgfqpoint{3.620571in}{3.058986in}}{\pgfqpoint{3.612757in}{3.066800in}}%
\pgfpathcurveto{\pgfqpoint{3.604944in}{3.074614in}}{\pgfqpoint{3.594345in}{3.079004in}}{\pgfqpoint{3.583295in}{3.079004in}}%
\pgfpathcurveto{\pgfqpoint{3.572244in}{3.079004in}}{\pgfqpoint{3.561645in}{3.074614in}}{\pgfqpoint{3.553832in}{3.066800in}}%
\pgfpathcurveto{\pgfqpoint{3.546018in}{3.058986in}}{\pgfqpoint{3.541628in}{3.048387in}}{\pgfqpoint{3.541628in}{3.037337in}}%
\pgfpathcurveto{\pgfqpoint{3.541628in}{3.026287in}}{\pgfqpoint{3.546018in}{3.015688in}}{\pgfqpoint{3.553832in}{3.007875in}}%
\pgfpathcurveto{\pgfqpoint{3.561645in}{3.000061in}}{\pgfqpoint{3.572244in}{2.995671in}}{\pgfqpoint{3.583295in}{2.995671in}}%
\pgfpathclose%
\pgfusepath{stroke,fill}%
\end{pgfscope}%
\begin{pgfscope}%
\pgfpathrectangle{\pgfqpoint{0.481978in}{0.331635in}}{\pgfqpoint{4.960000in}{3.696000in}}%
\pgfusepath{clip}%
\pgfsetbuttcap%
\pgfsetroundjoin%
\definecolor{currentfill}{rgb}{1.000000,0.705882,0.509804}%
\pgfsetfillcolor{currentfill}%
\pgfsetlinewidth{0.481800pt}%
\definecolor{currentstroke}{rgb}{1.000000,1.000000,1.000000}%
\pgfsetstrokecolor{currentstroke}%
\pgfsetdash{}{0pt}%
\pgfpathmoveto{\pgfqpoint{2.238649in}{1.806504in}}%
\pgfpathcurveto{\pgfqpoint{2.249700in}{1.806504in}}{\pgfqpoint{2.260299in}{1.810894in}}{\pgfqpoint{2.268112in}{1.818708in}}%
\pgfpathcurveto{\pgfqpoint{2.275926in}{1.826522in}}{\pgfqpoint{2.280316in}{1.837121in}}{\pgfqpoint{2.280316in}{1.848171in}}%
\pgfpathcurveto{\pgfqpoint{2.280316in}{1.859221in}}{\pgfqpoint{2.275926in}{1.869820in}}{\pgfqpoint{2.268112in}{1.877634in}}%
\pgfpathcurveto{\pgfqpoint{2.260299in}{1.885447in}}{\pgfqpoint{2.249700in}{1.889837in}}{\pgfqpoint{2.238649in}{1.889837in}}%
\pgfpathcurveto{\pgfqpoint{2.227599in}{1.889837in}}{\pgfqpoint{2.217000in}{1.885447in}}{\pgfqpoint{2.209187in}{1.877634in}}%
\pgfpathcurveto{\pgfqpoint{2.201373in}{1.869820in}}{\pgfqpoint{2.196983in}{1.859221in}}{\pgfqpoint{2.196983in}{1.848171in}}%
\pgfpathcurveto{\pgfqpoint{2.196983in}{1.837121in}}{\pgfqpoint{2.201373in}{1.826522in}}{\pgfqpoint{2.209187in}{1.818708in}}%
\pgfpathcurveto{\pgfqpoint{2.217000in}{1.810894in}}{\pgfqpoint{2.227599in}{1.806504in}}{\pgfqpoint{2.238649in}{1.806504in}}%
\pgfpathclose%
\pgfusepath{stroke,fill}%
\end{pgfscope}%
\begin{pgfscope}%
\pgfpathrectangle{\pgfqpoint{0.481978in}{0.331635in}}{\pgfqpoint{4.960000in}{3.696000in}}%
\pgfusepath{clip}%
\pgfsetbuttcap%
\pgfsetroundjoin%
\definecolor{currentfill}{rgb}{1.000000,0.705882,0.509804}%
\pgfsetfillcolor{currentfill}%
\pgfsetlinewidth{0.481800pt}%
\definecolor{currentstroke}{rgb}{1.000000,1.000000,1.000000}%
\pgfsetstrokecolor{currentstroke}%
\pgfsetdash{}{0pt}%
\pgfpathmoveto{\pgfqpoint{1.441655in}{2.238397in}}%
\pgfpathcurveto{\pgfqpoint{1.452705in}{2.238397in}}{\pgfqpoint{1.463304in}{2.242787in}}{\pgfqpoint{1.471118in}{2.250601in}}%
\pgfpathcurveto{\pgfqpoint{1.478932in}{2.258414in}}{\pgfqpoint{1.483322in}{2.269013in}}{\pgfqpoint{1.483322in}{2.280064in}}%
\pgfpathcurveto{\pgfqpoint{1.483322in}{2.291114in}}{\pgfqpoint{1.478932in}{2.301713in}}{\pgfqpoint{1.471118in}{2.309526in}}%
\pgfpathcurveto{\pgfqpoint{1.463304in}{2.317340in}}{\pgfqpoint{1.452705in}{2.321730in}}{\pgfqpoint{1.441655in}{2.321730in}}%
\pgfpathcurveto{\pgfqpoint{1.430605in}{2.321730in}}{\pgfqpoint{1.420006in}{2.317340in}}{\pgfqpoint{1.412192in}{2.309526in}}%
\pgfpathcurveto{\pgfqpoint{1.404379in}{2.301713in}}{\pgfqpoint{1.399988in}{2.291114in}}{\pgfqpoint{1.399988in}{2.280064in}}%
\pgfpathcurveto{\pgfqpoint{1.399988in}{2.269013in}}{\pgfqpoint{1.404379in}{2.258414in}}{\pgfqpoint{1.412192in}{2.250601in}}%
\pgfpathcurveto{\pgfqpoint{1.420006in}{2.242787in}}{\pgfqpoint{1.430605in}{2.238397in}}{\pgfqpoint{1.441655in}{2.238397in}}%
\pgfpathclose%
\pgfusepath{stroke,fill}%
\end{pgfscope}%
\begin{pgfscope}%
\pgfpathrectangle{\pgfqpoint{0.481978in}{0.331635in}}{\pgfqpoint{4.960000in}{3.696000in}}%
\pgfusepath{clip}%
\pgfsetbuttcap%
\pgfsetroundjoin%
\definecolor{currentfill}{rgb}{1.000000,0.705882,0.509804}%
\pgfsetfillcolor{currentfill}%
\pgfsetlinewidth{0.481800pt}%
\definecolor{currentstroke}{rgb}{1.000000,1.000000,1.000000}%
\pgfsetstrokecolor{currentstroke}%
\pgfsetdash{}{0pt}%
\pgfpathmoveto{\pgfqpoint{4.344247in}{2.183345in}}%
\pgfpathcurveto{\pgfqpoint{4.355297in}{2.183345in}}{\pgfqpoint{4.365896in}{2.187735in}}{\pgfqpoint{4.373710in}{2.195548in}}%
\pgfpathcurveto{\pgfqpoint{4.381523in}{2.203362in}}{\pgfqpoint{4.385914in}{2.213961in}}{\pgfqpoint{4.385914in}{2.225011in}}%
\pgfpathcurveto{\pgfqpoint{4.385914in}{2.236061in}}{\pgfqpoint{4.381523in}{2.246660in}}{\pgfqpoint{4.373710in}{2.254474in}}%
\pgfpathcurveto{\pgfqpoint{4.365896in}{2.262288in}}{\pgfqpoint{4.355297in}{2.266678in}}{\pgfqpoint{4.344247in}{2.266678in}}%
\pgfpathcurveto{\pgfqpoint{4.333197in}{2.266678in}}{\pgfqpoint{4.322598in}{2.262288in}}{\pgfqpoint{4.314784in}{2.254474in}}%
\pgfpathcurveto{\pgfqpoint{4.306971in}{2.246660in}}{\pgfqpoint{4.302580in}{2.236061in}}{\pgfqpoint{4.302580in}{2.225011in}}%
\pgfpathcurveto{\pgfqpoint{4.302580in}{2.213961in}}{\pgfqpoint{4.306971in}{2.203362in}}{\pgfqpoint{4.314784in}{2.195548in}}%
\pgfpathcurveto{\pgfqpoint{4.322598in}{2.187735in}}{\pgfqpoint{4.333197in}{2.183345in}}{\pgfqpoint{4.344247in}{2.183345in}}%
\pgfpathclose%
\pgfusepath{stroke,fill}%
\end{pgfscope}%
\begin{pgfscope}%
\pgfpathrectangle{\pgfqpoint{0.481978in}{0.331635in}}{\pgfqpoint{4.960000in}{3.696000in}}%
\pgfusepath{clip}%
\pgfsetbuttcap%
\pgfsetroundjoin%
\definecolor{currentfill}{rgb}{1.000000,0.705882,0.509804}%
\pgfsetfillcolor{currentfill}%
\pgfsetlinewidth{0.481800pt}%
\definecolor{currentstroke}{rgb}{1.000000,1.000000,1.000000}%
\pgfsetstrokecolor{currentstroke}%
\pgfsetdash{}{0pt}%
\pgfpathmoveto{\pgfqpoint{3.012958in}{3.049811in}}%
\pgfpathcurveto{\pgfqpoint{3.024008in}{3.049811in}}{\pgfqpoint{3.034607in}{3.054202in}}{\pgfqpoint{3.042420in}{3.062015in}}%
\pgfpathcurveto{\pgfqpoint{3.050234in}{3.069829in}}{\pgfqpoint{3.054624in}{3.080428in}}{\pgfqpoint{3.054624in}{3.091478in}}%
\pgfpathcurveto{\pgfqpoint{3.054624in}{3.102528in}}{\pgfqpoint{3.050234in}{3.113127in}}{\pgfqpoint{3.042420in}{3.120941in}}%
\pgfpathcurveto{\pgfqpoint{3.034607in}{3.128755in}}{\pgfqpoint{3.024008in}{3.133145in}}{\pgfqpoint{3.012958in}{3.133145in}}%
\pgfpathcurveto{\pgfqpoint{3.001907in}{3.133145in}}{\pgfqpoint{2.991308in}{3.128755in}}{\pgfqpoint{2.983495in}{3.120941in}}%
\pgfpathcurveto{\pgfqpoint{2.975681in}{3.113127in}}{\pgfqpoint{2.971291in}{3.102528in}}{\pgfqpoint{2.971291in}{3.091478in}}%
\pgfpathcurveto{\pgfqpoint{2.971291in}{3.080428in}}{\pgfqpoint{2.975681in}{3.069829in}}{\pgfqpoint{2.983495in}{3.062015in}}%
\pgfpathcurveto{\pgfqpoint{2.991308in}{3.054202in}}{\pgfqpoint{3.001907in}{3.049811in}}{\pgfqpoint{3.012958in}{3.049811in}}%
\pgfpathclose%
\pgfusepath{stroke,fill}%
\end{pgfscope}%
\begin{pgfscope}%
\pgfpathrectangle{\pgfqpoint{0.481978in}{0.331635in}}{\pgfqpoint{4.960000in}{3.696000in}}%
\pgfusepath{clip}%
\pgfsetbuttcap%
\pgfsetroundjoin%
\definecolor{currentfill}{rgb}{1.000000,0.705882,0.509804}%
\pgfsetfillcolor{currentfill}%
\pgfsetlinewidth{0.481800pt}%
\definecolor{currentstroke}{rgb}{1.000000,1.000000,1.000000}%
\pgfsetstrokecolor{currentstroke}%
\pgfsetdash{}{0pt}%
\pgfpathmoveto{\pgfqpoint{4.097583in}{1.325283in}}%
\pgfpathcurveto{\pgfqpoint{4.108633in}{1.325283in}}{\pgfqpoint{4.119232in}{1.329674in}}{\pgfqpoint{4.127046in}{1.337487in}}%
\pgfpathcurveto{\pgfqpoint{4.134859in}{1.345301in}}{\pgfqpoint{4.139249in}{1.355900in}}{\pgfqpoint{4.139249in}{1.366950in}}%
\pgfpathcurveto{\pgfqpoint{4.139249in}{1.378000in}}{\pgfqpoint{4.134859in}{1.388599in}}{\pgfqpoint{4.127046in}{1.396413in}}%
\pgfpathcurveto{\pgfqpoint{4.119232in}{1.404226in}}{\pgfqpoint{4.108633in}{1.408617in}}{\pgfqpoint{4.097583in}{1.408617in}}%
\pgfpathcurveto{\pgfqpoint{4.086533in}{1.408617in}}{\pgfqpoint{4.075934in}{1.404226in}}{\pgfqpoint{4.068120in}{1.396413in}}%
\pgfpathcurveto{\pgfqpoint{4.060306in}{1.388599in}}{\pgfqpoint{4.055916in}{1.378000in}}{\pgfqpoint{4.055916in}{1.366950in}}%
\pgfpathcurveto{\pgfqpoint{4.055916in}{1.355900in}}{\pgfqpoint{4.060306in}{1.345301in}}{\pgfqpoint{4.068120in}{1.337487in}}%
\pgfpathcurveto{\pgfqpoint{4.075934in}{1.329674in}}{\pgfqpoint{4.086533in}{1.325283in}}{\pgfqpoint{4.097583in}{1.325283in}}%
\pgfpathclose%
\pgfusepath{stroke,fill}%
\end{pgfscope}%
\begin{pgfscope}%
\pgfpathrectangle{\pgfqpoint{0.481978in}{0.331635in}}{\pgfqpoint{4.960000in}{3.696000in}}%
\pgfusepath{clip}%
\pgfsetbuttcap%
\pgfsetroundjoin%
\definecolor{currentfill}{rgb}{1.000000,0.705882,0.509804}%
\pgfsetfillcolor{currentfill}%
\pgfsetlinewidth{0.481800pt}%
\definecolor{currentstroke}{rgb}{1.000000,1.000000,1.000000}%
\pgfsetstrokecolor{currentstroke}%
\pgfsetdash{}{0pt}%
\pgfpathmoveto{\pgfqpoint{2.247330in}{3.516427in}}%
\pgfpathcurveto{\pgfqpoint{2.258380in}{3.516427in}}{\pgfqpoint{2.268979in}{3.520817in}}{\pgfqpoint{2.276793in}{3.528631in}}%
\pgfpathcurveto{\pgfqpoint{2.284606in}{3.536444in}}{\pgfqpoint{2.288997in}{3.547043in}}{\pgfqpoint{2.288997in}{3.558093in}}%
\pgfpathcurveto{\pgfqpoint{2.288997in}{3.569144in}}{\pgfqpoint{2.284606in}{3.579743in}}{\pgfqpoint{2.276793in}{3.587556in}}%
\pgfpathcurveto{\pgfqpoint{2.268979in}{3.595370in}}{\pgfqpoint{2.258380in}{3.599760in}}{\pgfqpoint{2.247330in}{3.599760in}}%
\pgfpathcurveto{\pgfqpoint{2.236280in}{3.599760in}}{\pgfqpoint{2.225681in}{3.595370in}}{\pgfqpoint{2.217867in}{3.587556in}}%
\pgfpathcurveto{\pgfqpoint{2.210054in}{3.579743in}}{\pgfqpoint{2.205663in}{3.569144in}}{\pgfqpoint{2.205663in}{3.558093in}}%
\pgfpathcurveto{\pgfqpoint{2.205663in}{3.547043in}}{\pgfqpoint{2.210054in}{3.536444in}}{\pgfqpoint{2.217867in}{3.528631in}}%
\pgfpathcurveto{\pgfqpoint{2.225681in}{3.520817in}}{\pgfqpoint{2.236280in}{3.516427in}}{\pgfqpoint{2.247330in}{3.516427in}}%
\pgfpathclose%
\pgfusepath{stroke,fill}%
\end{pgfscope}%
\begin{pgfscope}%
\pgfpathrectangle{\pgfqpoint{0.481978in}{0.331635in}}{\pgfqpoint{4.960000in}{3.696000in}}%
\pgfusepath{clip}%
\pgfsetbuttcap%
\pgfsetroundjoin%
\definecolor{currentfill}{rgb}{1.000000,0.705882,0.509804}%
\pgfsetfillcolor{currentfill}%
\pgfsetlinewidth{0.481800pt}%
\definecolor{currentstroke}{rgb}{1.000000,1.000000,1.000000}%
\pgfsetstrokecolor{currentstroke}%
\pgfsetdash{}{0pt}%
\pgfpathmoveto{\pgfqpoint{2.664498in}{1.227828in}}%
\pgfpathcurveto{\pgfqpoint{2.675548in}{1.227828in}}{\pgfqpoint{2.686147in}{1.232218in}}{\pgfqpoint{2.693960in}{1.240032in}}%
\pgfpathcurveto{\pgfqpoint{2.701774in}{1.247845in}}{\pgfqpoint{2.706164in}{1.258444in}}{\pgfqpoint{2.706164in}{1.269495in}}%
\pgfpathcurveto{\pgfqpoint{2.706164in}{1.280545in}}{\pgfqpoint{2.701774in}{1.291144in}}{\pgfqpoint{2.693960in}{1.298957in}}%
\pgfpathcurveto{\pgfqpoint{2.686147in}{1.306771in}}{\pgfqpoint{2.675548in}{1.311161in}}{\pgfqpoint{2.664498in}{1.311161in}}%
\pgfpathcurveto{\pgfqpoint{2.653448in}{1.311161in}}{\pgfqpoint{2.642848in}{1.306771in}}{\pgfqpoint{2.635035in}{1.298957in}}%
\pgfpathcurveto{\pgfqpoint{2.627221in}{1.291144in}}{\pgfqpoint{2.622831in}{1.280545in}}{\pgfqpoint{2.622831in}{1.269495in}}%
\pgfpathcurveto{\pgfqpoint{2.622831in}{1.258444in}}{\pgfqpoint{2.627221in}{1.247845in}}{\pgfqpoint{2.635035in}{1.240032in}}%
\pgfpathcurveto{\pgfqpoint{2.642848in}{1.232218in}}{\pgfqpoint{2.653448in}{1.227828in}}{\pgfqpoint{2.664498in}{1.227828in}}%
\pgfpathclose%
\pgfusepath{stroke,fill}%
\end{pgfscope}%
\begin{pgfscope}%
\pgfpathrectangle{\pgfqpoint{0.481978in}{0.331635in}}{\pgfqpoint{4.960000in}{3.696000in}}%
\pgfusepath{clip}%
\pgfsetbuttcap%
\pgfsetroundjoin%
\definecolor{currentfill}{rgb}{1.000000,0.705882,0.509804}%
\pgfsetfillcolor{currentfill}%
\pgfsetlinewidth{0.481800pt}%
\definecolor{currentstroke}{rgb}{1.000000,1.000000,1.000000}%
\pgfsetstrokecolor{currentstroke}%
\pgfsetdash{}{0pt}%
\pgfpathmoveto{\pgfqpoint{2.466458in}{3.688349in}}%
\pgfpathcurveto{\pgfqpoint{2.477509in}{3.688349in}}{\pgfqpoint{2.488108in}{3.692740in}}{\pgfqpoint{2.495921in}{3.700553in}}%
\pgfpathcurveto{\pgfqpoint{2.503735in}{3.708367in}}{\pgfqpoint{2.508125in}{3.718966in}}{\pgfqpoint{2.508125in}{3.730016in}}%
\pgfpathcurveto{\pgfqpoint{2.508125in}{3.741066in}}{\pgfqpoint{2.503735in}{3.751665in}}{\pgfqpoint{2.495921in}{3.759479in}}%
\pgfpathcurveto{\pgfqpoint{2.488108in}{3.767293in}}{\pgfqpoint{2.477509in}{3.771683in}}{\pgfqpoint{2.466458in}{3.771683in}}%
\pgfpathcurveto{\pgfqpoint{2.455408in}{3.771683in}}{\pgfqpoint{2.444809in}{3.767293in}}{\pgfqpoint{2.436996in}{3.759479in}}%
\pgfpathcurveto{\pgfqpoint{2.429182in}{3.751665in}}{\pgfqpoint{2.424792in}{3.741066in}}{\pgfqpoint{2.424792in}{3.730016in}}%
\pgfpathcurveto{\pgfqpoint{2.424792in}{3.718966in}}{\pgfqpoint{2.429182in}{3.708367in}}{\pgfqpoint{2.436996in}{3.700553in}}%
\pgfpathcurveto{\pgfqpoint{2.444809in}{3.692740in}}{\pgfqpoint{2.455408in}{3.688349in}}{\pgfqpoint{2.466458in}{3.688349in}}%
\pgfpathclose%
\pgfusepath{stroke,fill}%
\end{pgfscope}%
\begin{pgfscope}%
\pgfpathrectangle{\pgfqpoint{0.481978in}{0.331635in}}{\pgfqpoint{4.960000in}{3.696000in}}%
\pgfusepath{clip}%
\pgfsetbuttcap%
\pgfsetroundjoin%
\definecolor{currentfill}{rgb}{1.000000,0.705882,0.509804}%
\pgfsetfillcolor{currentfill}%
\pgfsetlinewidth{0.481800pt}%
\definecolor{currentstroke}{rgb}{1.000000,1.000000,1.000000}%
\pgfsetstrokecolor{currentstroke}%
\pgfsetdash{}{0pt}%
\pgfpathmoveto{\pgfqpoint{2.511710in}{1.097432in}}%
\pgfpathcurveto{\pgfqpoint{2.522760in}{1.097432in}}{\pgfqpoint{2.533359in}{1.101823in}}{\pgfqpoint{2.541172in}{1.109636in}}%
\pgfpathcurveto{\pgfqpoint{2.548986in}{1.117450in}}{\pgfqpoint{2.553376in}{1.128049in}}{\pgfqpoint{2.553376in}{1.139099in}}%
\pgfpathcurveto{\pgfqpoint{2.553376in}{1.150149in}}{\pgfqpoint{2.548986in}{1.160748in}}{\pgfqpoint{2.541172in}{1.168562in}}%
\pgfpathcurveto{\pgfqpoint{2.533359in}{1.176375in}}{\pgfqpoint{2.522760in}{1.180766in}}{\pgfqpoint{2.511710in}{1.180766in}}%
\pgfpathcurveto{\pgfqpoint{2.500659in}{1.180766in}}{\pgfqpoint{2.490060in}{1.176375in}}{\pgfqpoint{2.482247in}{1.168562in}}%
\pgfpathcurveto{\pgfqpoint{2.474433in}{1.160748in}}{\pgfqpoint{2.470043in}{1.150149in}}{\pgfqpoint{2.470043in}{1.139099in}}%
\pgfpathcurveto{\pgfqpoint{2.470043in}{1.128049in}}{\pgfqpoint{2.474433in}{1.117450in}}{\pgfqpoint{2.482247in}{1.109636in}}%
\pgfpathcurveto{\pgfqpoint{2.490060in}{1.101823in}}{\pgfqpoint{2.500659in}{1.097432in}}{\pgfqpoint{2.511710in}{1.097432in}}%
\pgfpathclose%
\pgfusepath{stroke,fill}%
\end{pgfscope}%
\begin{pgfscope}%
\pgfpathrectangle{\pgfqpoint{0.481978in}{0.331635in}}{\pgfqpoint{4.960000in}{3.696000in}}%
\pgfusepath{clip}%
\pgfsetbuttcap%
\pgfsetroundjoin%
\definecolor{currentfill}{rgb}{1.000000,0.705882,0.509804}%
\pgfsetfillcolor{currentfill}%
\pgfsetlinewidth{0.481800pt}%
\definecolor{currentstroke}{rgb}{1.000000,1.000000,1.000000}%
\pgfsetstrokecolor{currentstroke}%
\pgfsetdash{}{0pt}%
\pgfpathmoveto{\pgfqpoint{3.946493in}{1.205734in}}%
\pgfpathcurveto{\pgfqpoint{3.957543in}{1.205734in}}{\pgfqpoint{3.968142in}{1.210125in}}{\pgfqpoint{3.975956in}{1.217938in}}%
\pgfpathcurveto{\pgfqpoint{3.983770in}{1.225752in}}{\pgfqpoint{3.988160in}{1.236351in}}{\pgfqpoint{3.988160in}{1.247401in}}%
\pgfpathcurveto{\pgfqpoint{3.988160in}{1.258451in}}{\pgfqpoint{3.983770in}{1.269050in}}{\pgfqpoint{3.975956in}{1.276864in}}%
\pgfpathcurveto{\pgfqpoint{3.968142in}{1.284677in}}{\pgfqpoint{3.957543in}{1.289068in}}{\pgfqpoint{3.946493in}{1.289068in}}%
\pgfpathcurveto{\pgfqpoint{3.935443in}{1.289068in}}{\pgfqpoint{3.924844in}{1.284677in}}{\pgfqpoint{3.917031in}{1.276864in}}%
\pgfpathcurveto{\pgfqpoint{3.909217in}{1.269050in}}{\pgfqpoint{3.904827in}{1.258451in}}{\pgfqpoint{3.904827in}{1.247401in}}%
\pgfpathcurveto{\pgfqpoint{3.904827in}{1.236351in}}{\pgfqpoint{3.909217in}{1.225752in}}{\pgfqpoint{3.917031in}{1.217938in}}%
\pgfpathcurveto{\pgfqpoint{3.924844in}{1.210125in}}{\pgfqpoint{3.935443in}{1.205734in}}{\pgfqpoint{3.946493in}{1.205734in}}%
\pgfpathclose%
\pgfusepath{stroke,fill}%
\end{pgfscope}%
\begin{pgfscope}%
\pgfpathrectangle{\pgfqpoint{0.481978in}{0.331635in}}{\pgfqpoint{4.960000in}{3.696000in}}%
\pgfusepath{clip}%
\pgfsetbuttcap%
\pgfsetroundjoin%
\definecolor{currentfill}{rgb}{1.000000,0.705882,0.509804}%
\pgfsetfillcolor{currentfill}%
\pgfsetlinewidth{0.481800pt}%
\definecolor{currentstroke}{rgb}{1.000000,1.000000,1.000000}%
\pgfsetstrokecolor{currentstroke}%
\pgfsetdash{}{0pt}%
\pgfpathmoveto{\pgfqpoint{3.308273in}{3.729300in}}%
\pgfpathcurveto{\pgfqpoint{3.319323in}{3.729300in}}{\pgfqpoint{3.329922in}{3.733691in}}{\pgfqpoint{3.337736in}{3.741504in}}%
\pgfpathcurveto{\pgfqpoint{3.345550in}{3.749318in}}{\pgfqpoint{3.349940in}{3.759917in}}{\pgfqpoint{3.349940in}{3.770967in}}%
\pgfpathcurveto{\pgfqpoint{3.349940in}{3.782017in}}{\pgfqpoint{3.345550in}{3.792616in}}{\pgfqpoint{3.337736in}{3.800430in}}%
\pgfpathcurveto{\pgfqpoint{3.329922in}{3.808243in}}{\pgfqpoint{3.319323in}{3.812634in}}{\pgfqpoint{3.308273in}{3.812634in}}%
\pgfpathcurveto{\pgfqpoint{3.297223in}{3.812634in}}{\pgfqpoint{3.286624in}{3.808243in}}{\pgfqpoint{3.278810in}{3.800430in}}%
\pgfpathcurveto{\pgfqpoint{3.270997in}{3.792616in}}{\pgfqpoint{3.266606in}{3.782017in}}{\pgfqpoint{3.266606in}{3.770967in}}%
\pgfpathcurveto{\pgfqpoint{3.266606in}{3.759917in}}{\pgfqpoint{3.270997in}{3.749318in}}{\pgfqpoint{3.278810in}{3.741504in}}%
\pgfpathcurveto{\pgfqpoint{3.286624in}{3.733691in}}{\pgfqpoint{3.297223in}{3.729300in}}{\pgfqpoint{3.308273in}{3.729300in}}%
\pgfpathclose%
\pgfusepath{stroke,fill}%
\end{pgfscope}%
\begin{pgfscope}%
\pgfpathrectangle{\pgfqpoint{0.481978in}{0.331635in}}{\pgfqpoint{4.960000in}{3.696000in}}%
\pgfusepath{clip}%
\pgfsetbuttcap%
\pgfsetroundjoin%
\definecolor{currentfill}{rgb}{1.000000,0.705882,0.509804}%
\pgfsetfillcolor{currentfill}%
\pgfsetlinewidth{0.481800pt}%
\definecolor{currentstroke}{rgb}{1.000000,1.000000,1.000000}%
\pgfsetstrokecolor{currentstroke}%
\pgfsetdash{}{0pt}%
\pgfpathmoveto{\pgfqpoint{3.625583in}{1.561410in}}%
\pgfpathcurveto{\pgfqpoint{3.636633in}{1.561410in}}{\pgfqpoint{3.647232in}{1.565800in}}{\pgfqpoint{3.655045in}{1.573613in}}%
\pgfpathcurveto{\pgfqpoint{3.662859in}{1.581427in}}{\pgfqpoint{3.667249in}{1.592026in}}{\pgfqpoint{3.667249in}{1.603076in}}%
\pgfpathcurveto{\pgfqpoint{3.667249in}{1.614126in}}{\pgfqpoint{3.662859in}{1.624725in}}{\pgfqpoint{3.655045in}{1.632539in}}%
\pgfpathcurveto{\pgfqpoint{3.647232in}{1.640353in}}{\pgfqpoint{3.636633in}{1.644743in}}{\pgfqpoint{3.625583in}{1.644743in}}%
\pgfpathcurveto{\pgfqpoint{3.614533in}{1.644743in}}{\pgfqpoint{3.603934in}{1.640353in}}{\pgfqpoint{3.596120in}{1.632539in}}%
\pgfpathcurveto{\pgfqpoint{3.588306in}{1.624725in}}{\pgfqpoint{3.583916in}{1.614126in}}{\pgfqpoint{3.583916in}{1.603076in}}%
\pgfpathcurveto{\pgfqpoint{3.583916in}{1.592026in}}{\pgfqpoint{3.588306in}{1.581427in}}{\pgfqpoint{3.596120in}{1.573613in}}%
\pgfpathcurveto{\pgfqpoint{3.603934in}{1.565800in}}{\pgfqpoint{3.614533in}{1.561410in}}{\pgfqpoint{3.625583in}{1.561410in}}%
\pgfpathclose%
\pgfusepath{stroke,fill}%
\end{pgfscope}%
\begin{pgfscope}%
\pgfpathrectangle{\pgfqpoint{0.481978in}{0.331635in}}{\pgfqpoint{4.960000in}{3.696000in}}%
\pgfusepath{clip}%
\pgfsetbuttcap%
\pgfsetroundjoin%
\definecolor{currentfill}{rgb}{1.000000,0.705882,0.509804}%
\pgfsetfillcolor{currentfill}%
\pgfsetlinewidth{0.481800pt}%
\definecolor{currentstroke}{rgb}{1.000000,1.000000,1.000000}%
\pgfsetstrokecolor{currentstroke}%
\pgfsetdash{}{0pt}%
\pgfpathmoveto{\pgfqpoint{2.731795in}{1.428227in}}%
\pgfpathcurveto{\pgfqpoint{2.742846in}{1.428227in}}{\pgfqpoint{2.753445in}{1.432618in}}{\pgfqpoint{2.761258in}{1.440431in}}%
\pgfpathcurveto{\pgfqpoint{2.769072in}{1.448245in}}{\pgfqpoint{2.773462in}{1.458844in}}{\pgfqpoint{2.773462in}{1.469894in}}%
\pgfpathcurveto{\pgfqpoint{2.773462in}{1.480944in}}{\pgfqpoint{2.769072in}{1.491543in}}{\pgfqpoint{2.761258in}{1.499357in}}%
\pgfpathcurveto{\pgfqpoint{2.753445in}{1.507170in}}{\pgfqpoint{2.742846in}{1.511561in}}{\pgfqpoint{2.731795in}{1.511561in}}%
\pgfpathcurveto{\pgfqpoint{2.720745in}{1.511561in}}{\pgfqpoint{2.710146in}{1.507170in}}{\pgfqpoint{2.702333in}{1.499357in}}%
\pgfpathcurveto{\pgfqpoint{2.694519in}{1.491543in}}{\pgfqpoint{2.690129in}{1.480944in}}{\pgfqpoint{2.690129in}{1.469894in}}%
\pgfpathcurveto{\pgfqpoint{2.690129in}{1.458844in}}{\pgfqpoint{2.694519in}{1.448245in}}{\pgfqpoint{2.702333in}{1.440431in}}%
\pgfpathcurveto{\pgfqpoint{2.710146in}{1.432618in}}{\pgfqpoint{2.720745in}{1.428227in}}{\pgfqpoint{2.731795in}{1.428227in}}%
\pgfpathclose%
\pgfusepath{stroke,fill}%
\end{pgfscope}%
\begin{pgfscope}%
\pgfpathrectangle{\pgfqpoint{0.481978in}{0.331635in}}{\pgfqpoint{4.960000in}{3.696000in}}%
\pgfusepath{clip}%
\pgfsetbuttcap%
\pgfsetroundjoin%
\definecolor{currentfill}{rgb}{1.000000,0.705882,0.509804}%
\pgfsetfillcolor{currentfill}%
\pgfsetlinewidth{0.481800pt}%
\definecolor{currentstroke}{rgb}{1.000000,1.000000,1.000000}%
\pgfsetstrokecolor{currentstroke}%
\pgfsetdash{}{0pt}%
\pgfpathmoveto{\pgfqpoint{3.870146in}{2.486326in}}%
\pgfpathcurveto{\pgfqpoint{3.881196in}{2.486326in}}{\pgfqpoint{3.891795in}{2.490716in}}{\pgfqpoint{3.899609in}{2.498530in}}%
\pgfpathcurveto{\pgfqpoint{3.907422in}{2.506343in}}{\pgfqpoint{3.911813in}{2.516942in}}{\pgfqpoint{3.911813in}{2.527993in}}%
\pgfpathcurveto{\pgfqpoint{3.911813in}{2.539043in}}{\pgfqpoint{3.907422in}{2.549642in}}{\pgfqpoint{3.899609in}{2.557455in}}%
\pgfpathcurveto{\pgfqpoint{3.891795in}{2.565269in}}{\pgfqpoint{3.881196in}{2.569659in}}{\pgfqpoint{3.870146in}{2.569659in}}%
\pgfpathcurveto{\pgfqpoint{3.859096in}{2.569659in}}{\pgfqpoint{3.848497in}{2.565269in}}{\pgfqpoint{3.840683in}{2.557455in}}%
\pgfpathcurveto{\pgfqpoint{3.832870in}{2.549642in}}{\pgfqpoint{3.828479in}{2.539043in}}{\pgfqpoint{3.828479in}{2.527993in}}%
\pgfpathcurveto{\pgfqpoint{3.828479in}{2.516942in}}{\pgfqpoint{3.832870in}{2.506343in}}{\pgfqpoint{3.840683in}{2.498530in}}%
\pgfpathcurveto{\pgfqpoint{3.848497in}{2.490716in}}{\pgfqpoint{3.859096in}{2.486326in}}{\pgfqpoint{3.870146in}{2.486326in}}%
\pgfpathclose%
\pgfusepath{stroke,fill}%
\end{pgfscope}%
\begin{pgfscope}%
\pgfpathrectangle{\pgfqpoint{0.481978in}{0.331635in}}{\pgfqpoint{4.960000in}{3.696000in}}%
\pgfusepath{clip}%
\pgfsetbuttcap%
\pgfsetroundjoin%
\definecolor{currentfill}{rgb}{1.000000,0.705882,0.509804}%
\pgfsetfillcolor{currentfill}%
\pgfsetlinewidth{0.481800pt}%
\definecolor{currentstroke}{rgb}{1.000000,1.000000,1.000000}%
\pgfsetstrokecolor{currentstroke}%
\pgfsetdash{}{0pt}%
\pgfpathmoveto{\pgfqpoint{3.946508in}{2.869579in}}%
\pgfpathcurveto{\pgfqpoint{3.957558in}{2.869579in}}{\pgfqpoint{3.968157in}{2.873970in}}{\pgfqpoint{3.975970in}{2.881783in}}%
\pgfpathcurveto{\pgfqpoint{3.983784in}{2.889597in}}{\pgfqpoint{3.988174in}{2.900196in}}{\pgfqpoint{3.988174in}{2.911246in}}%
\pgfpathcurveto{\pgfqpoint{3.988174in}{2.922296in}}{\pgfqpoint{3.983784in}{2.932895in}}{\pgfqpoint{3.975970in}{2.940709in}}%
\pgfpathcurveto{\pgfqpoint{3.968157in}{2.948522in}}{\pgfqpoint{3.957558in}{2.952913in}}{\pgfqpoint{3.946508in}{2.952913in}}%
\pgfpathcurveto{\pgfqpoint{3.935458in}{2.952913in}}{\pgfqpoint{3.924858in}{2.948522in}}{\pgfqpoint{3.917045in}{2.940709in}}%
\pgfpathcurveto{\pgfqpoint{3.909231in}{2.932895in}}{\pgfqpoint{3.904841in}{2.922296in}}{\pgfqpoint{3.904841in}{2.911246in}}%
\pgfpathcurveto{\pgfqpoint{3.904841in}{2.900196in}}{\pgfqpoint{3.909231in}{2.889597in}}{\pgfqpoint{3.917045in}{2.881783in}}%
\pgfpathcurveto{\pgfqpoint{3.924858in}{2.873970in}}{\pgfqpoint{3.935458in}{2.869579in}}{\pgfqpoint{3.946508in}{2.869579in}}%
\pgfpathclose%
\pgfusepath{stroke,fill}%
\end{pgfscope}%
\begin{pgfscope}%
\pgfpathrectangle{\pgfqpoint{0.481978in}{0.331635in}}{\pgfqpoint{4.960000in}{3.696000in}}%
\pgfusepath{clip}%
\pgfsetbuttcap%
\pgfsetroundjoin%
\definecolor{currentfill}{rgb}{1.000000,0.705882,0.509804}%
\pgfsetfillcolor{currentfill}%
\pgfsetlinewidth{0.481800pt}%
\definecolor{currentstroke}{rgb}{1.000000,1.000000,1.000000}%
\pgfsetstrokecolor{currentstroke}%
\pgfsetdash{}{0pt}%
\pgfpathmoveto{\pgfqpoint{4.398894in}{1.001098in}}%
\pgfpathcurveto{\pgfqpoint{4.409944in}{1.001098in}}{\pgfqpoint{4.420543in}{1.005488in}}{\pgfqpoint{4.428357in}{1.013302in}}%
\pgfpathcurveto{\pgfqpoint{4.436170in}{1.021115in}}{\pgfqpoint{4.440560in}{1.031714in}}{\pgfqpoint{4.440560in}{1.042765in}}%
\pgfpathcurveto{\pgfqpoint{4.440560in}{1.053815in}}{\pgfqpoint{4.436170in}{1.064414in}}{\pgfqpoint{4.428357in}{1.072227in}}%
\pgfpathcurveto{\pgfqpoint{4.420543in}{1.080041in}}{\pgfqpoint{4.409944in}{1.084431in}}{\pgfqpoint{4.398894in}{1.084431in}}%
\pgfpathcurveto{\pgfqpoint{4.387844in}{1.084431in}}{\pgfqpoint{4.377245in}{1.080041in}}{\pgfqpoint{4.369431in}{1.072227in}}%
\pgfpathcurveto{\pgfqpoint{4.361617in}{1.064414in}}{\pgfqpoint{4.357227in}{1.053815in}}{\pgfqpoint{4.357227in}{1.042765in}}%
\pgfpathcurveto{\pgfqpoint{4.357227in}{1.031714in}}{\pgfqpoint{4.361617in}{1.021115in}}{\pgfqpoint{4.369431in}{1.013302in}}%
\pgfpathcurveto{\pgfqpoint{4.377245in}{1.005488in}}{\pgfqpoint{4.387844in}{1.001098in}}{\pgfqpoint{4.398894in}{1.001098in}}%
\pgfpathclose%
\pgfusepath{stroke,fill}%
\end{pgfscope}%
\begin{pgfscope}%
\pgfpathrectangle{\pgfqpoint{0.481978in}{0.331635in}}{\pgfqpoint{4.960000in}{3.696000in}}%
\pgfusepath{clip}%
\pgfsetbuttcap%
\pgfsetroundjoin%
\definecolor{currentfill}{rgb}{1.000000,0.705882,0.509804}%
\pgfsetfillcolor{currentfill}%
\pgfsetlinewidth{0.481800pt}%
\definecolor{currentstroke}{rgb}{1.000000,1.000000,1.000000}%
\pgfsetstrokecolor{currentstroke}%
\pgfsetdash{}{0pt}%
\pgfpathmoveto{\pgfqpoint{5.211970in}{1.559972in}}%
\pgfpathcurveto{\pgfqpoint{5.223020in}{1.559972in}}{\pgfqpoint{5.233619in}{1.564362in}}{\pgfqpoint{5.241433in}{1.572176in}}%
\pgfpathcurveto{\pgfqpoint{5.249247in}{1.579990in}}{\pgfqpoint{5.253637in}{1.590589in}}{\pgfqpoint{5.253637in}{1.601639in}}%
\pgfpathcurveto{\pgfqpoint{5.253637in}{1.612689in}}{\pgfqpoint{5.249247in}{1.623288in}}{\pgfqpoint{5.241433in}{1.631102in}}%
\pgfpathcurveto{\pgfqpoint{5.233619in}{1.638915in}}{\pgfqpoint{5.223020in}{1.643305in}}{\pgfqpoint{5.211970in}{1.643305in}}%
\pgfpathcurveto{\pgfqpoint{5.200920in}{1.643305in}}{\pgfqpoint{5.190321in}{1.638915in}}{\pgfqpoint{5.182507in}{1.631102in}}%
\pgfpathcurveto{\pgfqpoint{5.174694in}{1.623288in}}{\pgfqpoint{5.170303in}{1.612689in}}{\pgfqpoint{5.170303in}{1.601639in}}%
\pgfpathcurveto{\pgfqpoint{5.170303in}{1.590589in}}{\pgfqpoint{5.174694in}{1.579990in}}{\pgfqpoint{5.182507in}{1.572176in}}%
\pgfpathcurveto{\pgfqpoint{5.190321in}{1.564362in}}{\pgfqpoint{5.200920in}{1.559972in}}{\pgfqpoint{5.211970in}{1.559972in}}%
\pgfpathclose%
\pgfusepath{stroke,fill}%
\end{pgfscope}%
\begin{pgfscope}%
\pgfpathrectangle{\pgfqpoint{0.481978in}{0.331635in}}{\pgfqpoint{4.960000in}{3.696000in}}%
\pgfusepath{clip}%
\pgfsetbuttcap%
\pgfsetroundjoin%
\definecolor{currentfill}{rgb}{1.000000,0.705882,0.509804}%
\pgfsetfillcolor{currentfill}%
\pgfsetlinewidth{0.481800pt}%
\definecolor{currentstroke}{rgb}{1.000000,1.000000,1.000000}%
\pgfsetstrokecolor{currentstroke}%
\pgfsetdash{}{0pt}%
\pgfpathmoveto{\pgfqpoint{2.587465in}{2.946300in}}%
\pgfpathcurveto{\pgfqpoint{2.598515in}{2.946300in}}{\pgfqpoint{2.609114in}{2.950690in}}{\pgfqpoint{2.616928in}{2.958504in}}%
\pgfpathcurveto{\pgfqpoint{2.624742in}{2.966318in}}{\pgfqpoint{2.629132in}{2.976917in}}{\pgfqpoint{2.629132in}{2.987967in}}%
\pgfpathcurveto{\pgfqpoint{2.629132in}{2.999017in}}{\pgfqpoint{2.624742in}{3.009616in}}{\pgfqpoint{2.616928in}{3.017430in}}%
\pgfpathcurveto{\pgfqpoint{2.609114in}{3.025243in}}{\pgfqpoint{2.598515in}{3.029634in}}{\pgfqpoint{2.587465in}{3.029634in}}%
\pgfpathcurveto{\pgfqpoint{2.576415in}{3.029634in}}{\pgfqpoint{2.565816in}{3.025243in}}{\pgfqpoint{2.558002in}{3.017430in}}%
\pgfpathcurveto{\pgfqpoint{2.550189in}{3.009616in}}{\pgfqpoint{2.545798in}{2.999017in}}{\pgfqpoint{2.545798in}{2.987967in}}%
\pgfpathcurveto{\pgfqpoint{2.545798in}{2.976917in}}{\pgfqpoint{2.550189in}{2.966318in}}{\pgfqpoint{2.558002in}{2.958504in}}%
\pgfpathcurveto{\pgfqpoint{2.565816in}{2.950690in}}{\pgfqpoint{2.576415in}{2.946300in}}{\pgfqpoint{2.587465in}{2.946300in}}%
\pgfpathclose%
\pgfusepath{stroke,fill}%
\end{pgfscope}%
\begin{pgfscope}%
\pgfpathrectangle{\pgfqpoint{0.481978in}{0.331635in}}{\pgfqpoint{4.960000in}{3.696000in}}%
\pgfusepath{clip}%
\pgfsetbuttcap%
\pgfsetroundjoin%
\definecolor{currentfill}{rgb}{1.000000,0.705882,0.509804}%
\pgfsetfillcolor{currentfill}%
\pgfsetlinewidth{0.481800pt}%
\definecolor{currentstroke}{rgb}{1.000000,1.000000,1.000000}%
\pgfsetstrokecolor{currentstroke}%
\pgfsetdash{}{0pt}%
\pgfpathmoveto{\pgfqpoint{3.371501in}{3.446000in}}%
\pgfpathcurveto{\pgfqpoint{3.382551in}{3.446000in}}{\pgfqpoint{3.393150in}{3.450391in}}{\pgfqpoint{3.400964in}{3.458204in}}%
\pgfpathcurveto{\pgfqpoint{3.408778in}{3.466018in}}{\pgfqpoint{3.413168in}{3.476617in}}{\pgfqpoint{3.413168in}{3.487667in}}%
\pgfpathcurveto{\pgfqpoint{3.413168in}{3.498717in}}{\pgfqpoint{3.408778in}{3.509316in}}{\pgfqpoint{3.400964in}{3.517130in}}%
\pgfpathcurveto{\pgfqpoint{3.393150in}{3.524943in}}{\pgfqpoint{3.382551in}{3.529334in}}{\pgfqpoint{3.371501in}{3.529334in}}%
\pgfpathcurveto{\pgfqpoint{3.360451in}{3.529334in}}{\pgfqpoint{3.349852in}{3.524943in}}{\pgfqpoint{3.342039in}{3.517130in}}%
\pgfpathcurveto{\pgfqpoint{3.334225in}{3.509316in}}{\pgfqpoint{3.329835in}{3.498717in}}{\pgfqpoint{3.329835in}{3.487667in}}%
\pgfpathcurveto{\pgfqpoint{3.329835in}{3.476617in}}{\pgfqpoint{3.334225in}{3.466018in}}{\pgfqpoint{3.342039in}{3.458204in}}%
\pgfpathcurveto{\pgfqpoint{3.349852in}{3.450391in}}{\pgfqpoint{3.360451in}{3.446000in}}{\pgfqpoint{3.371501in}{3.446000in}}%
\pgfpathclose%
\pgfusepath{stroke,fill}%
\end{pgfscope}%
\begin{pgfscope}%
\pgfpathrectangle{\pgfqpoint{0.481978in}{0.331635in}}{\pgfqpoint{4.960000in}{3.696000in}}%
\pgfusepath{clip}%
\pgfsetbuttcap%
\pgfsetroundjoin%
\definecolor{currentfill}{rgb}{1.000000,0.705882,0.509804}%
\pgfsetfillcolor{currentfill}%
\pgfsetlinewidth{0.481800pt}%
\definecolor{currentstroke}{rgb}{1.000000,1.000000,1.000000}%
\pgfsetstrokecolor{currentstroke}%
\pgfsetdash{}{0pt}%
\pgfpathmoveto{\pgfqpoint{1.618240in}{2.575901in}}%
\pgfpathcurveto{\pgfqpoint{1.629290in}{2.575901in}}{\pgfqpoint{1.639889in}{2.580292in}}{\pgfqpoint{1.647703in}{2.588105in}}%
\pgfpathcurveto{\pgfqpoint{1.655516in}{2.595919in}}{\pgfqpoint{1.659907in}{2.606518in}}{\pgfqpoint{1.659907in}{2.617568in}}%
\pgfpathcurveto{\pgfqpoint{1.659907in}{2.628618in}}{\pgfqpoint{1.655516in}{2.639217in}}{\pgfqpoint{1.647703in}{2.647031in}}%
\pgfpathcurveto{\pgfqpoint{1.639889in}{2.654844in}}{\pgfqpoint{1.629290in}{2.659235in}}{\pgfqpoint{1.618240in}{2.659235in}}%
\pgfpathcurveto{\pgfqpoint{1.607190in}{2.659235in}}{\pgfqpoint{1.596591in}{2.654844in}}{\pgfqpoint{1.588777in}{2.647031in}}%
\pgfpathcurveto{\pgfqpoint{1.580963in}{2.639217in}}{\pgfqpoint{1.576573in}{2.628618in}}{\pgfqpoint{1.576573in}{2.617568in}}%
\pgfpathcurveto{\pgfqpoint{1.576573in}{2.606518in}}{\pgfqpoint{1.580963in}{2.595919in}}{\pgfqpoint{1.588777in}{2.588105in}}%
\pgfpathcurveto{\pgfqpoint{1.596591in}{2.580292in}}{\pgfqpoint{1.607190in}{2.575901in}}{\pgfqpoint{1.618240in}{2.575901in}}%
\pgfpathclose%
\pgfusepath{stroke,fill}%
\end{pgfscope}%
\begin{pgfscope}%
\pgfpathrectangle{\pgfqpoint{0.481978in}{0.331635in}}{\pgfqpoint{4.960000in}{3.696000in}}%
\pgfusepath{clip}%
\pgfsetbuttcap%
\pgfsetroundjoin%
\definecolor{currentfill}{rgb}{1.000000,0.705882,0.509804}%
\pgfsetfillcolor{currentfill}%
\pgfsetlinewidth{0.481800pt}%
\definecolor{currentstroke}{rgb}{1.000000,1.000000,1.000000}%
\pgfsetstrokecolor{currentstroke}%
\pgfsetdash{}{0pt}%
\pgfpathmoveto{\pgfqpoint{2.745120in}{2.629232in}}%
\pgfpathcurveto{\pgfqpoint{2.756170in}{2.629232in}}{\pgfqpoint{2.766770in}{2.633622in}}{\pgfqpoint{2.774583in}{2.641436in}}%
\pgfpathcurveto{\pgfqpoint{2.782397in}{2.649249in}}{\pgfqpoint{2.786787in}{2.659848in}}{\pgfqpoint{2.786787in}{2.670898in}}%
\pgfpathcurveto{\pgfqpoint{2.786787in}{2.681948in}}{\pgfqpoint{2.782397in}{2.692548in}}{\pgfqpoint{2.774583in}{2.700361in}}%
\pgfpathcurveto{\pgfqpoint{2.766770in}{2.708175in}}{\pgfqpoint{2.756170in}{2.712565in}}{\pgfqpoint{2.745120in}{2.712565in}}%
\pgfpathcurveto{\pgfqpoint{2.734070in}{2.712565in}}{\pgfqpoint{2.723471in}{2.708175in}}{\pgfqpoint{2.715658in}{2.700361in}}%
\pgfpathcurveto{\pgfqpoint{2.707844in}{2.692548in}}{\pgfqpoint{2.703454in}{2.681948in}}{\pgfqpoint{2.703454in}{2.670898in}}%
\pgfpathcurveto{\pgfqpoint{2.703454in}{2.659848in}}{\pgfqpoint{2.707844in}{2.649249in}}{\pgfqpoint{2.715658in}{2.641436in}}%
\pgfpathcurveto{\pgfqpoint{2.723471in}{2.633622in}}{\pgfqpoint{2.734070in}{2.629232in}}{\pgfqpoint{2.745120in}{2.629232in}}%
\pgfpathclose%
\pgfusepath{stroke,fill}%
\end{pgfscope}%
\begin{pgfscope}%
\pgfpathrectangle{\pgfqpoint{0.481978in}{0.331635in}}{\pgfqpoint{4.960000in}{3.696000in}}%
\pgfusepath{clip}%
\pgfsetbuttcap%
\pgfsetroundjoin%
\definecolor{currentfill}{rgb}{1.000000,0.705882,0.509804}%
\pgfsetfillcolor{currentfill}%
\pgfsetlinewidth{0.481800pt}%
\definecolor{currentstroke}{rgb}{1.000000,1.000000,1.000000}%
\pgfsetstrokecolor{currentstroke}%
\pgfsetdash{}{0pt}%
\pgfpathmoveto{\pgfqpoint{4.493868in}{2.821020in}}%
\pgfpathcurveto{\pgfqpoint{4.504918in}{2.821020in}}{\pgfqpoint{4.515517in}{2.825410in}}{\pgfqpoint{4.523330in}{2.833224in}}%
\pgfpathcurveto{\pgfqpoint{4.531144in}{2.841038in}}{\pgfqpoint{4.535534in}{2.851637in}}{\pgfqpoint{4.535534in}{2.862687in}}%
\pgfpathcurveto{\pgfqpoint{4.535534in}{2.873737in}}{\pgfqpoint{4.531144in}{2.884336in}}{\pgfqpoint{4.523330in}{2.892150in}}%
\pgfpathcurveto{\pgfqpoint{4.515517in}{2.899963in}}{\pgfqpoint{4.504918in}{2.904354in}}{\pgfqpoint{4.493868in}{2.904354in}}%
\pgfpathcurveto{\pgfqpoint{4.482818in}{2.904354in}}{\pgfqpoint{4.472219in}{2.899963in}}{\pgfqpoint{4.464405in}{2.892150in}}%
\pgfpathcurveto{\pgfqpoint{4.456591in}{2.884336in}}{\pgfqpoint{4.452201in}{2.873737in}}{\pgfqpoint{4.452201in}{2.862687in}}%
\pgfpathcurveto{\pgfqpoint{4.452201in}{2.851637in}}{\pgfqpoint{4.456591in}{2.841038in}}{\pgfqpoint{4.464405in}{2.833224in}}%
\pgfpathcurveto{\pgfqpoint{4.472219in}{2.825410in}}{\pgfqpoint{4.482818in}{2.821020in}}{\pgfqpoint{4.493868in}{2.821020in}}%
\pgfpathclose%
\pgfusepath{stroke,fill}%
\end{pgfscope}%
\begin{pgfscope}%
\pgfpathrectangle{\pgfqpoint{0.481978in}{0.331635in}}{\pgfqpoint{4.960000in}{3.696000in}}%
\pgfusepath{clip}%
\pgfsetbuttcap%
\pgfsetroundjoin%
\definecolor{currentfill}{rgb}{1.000000,0.705882,0.509804}%
\pgfsetfillcolor{currentfill}%
\pgfsetlinewidth{0.481800pt}%
\definecolor{currentstroke}{rgb}{1.000000,1.000000,1.000000}%
\pgfsetstrokecolor{currentstroke}%
\pgfsetdash{}{0pt}%
\pgfpathmoveto{\pgfqpoint{4.183823in}{2.277733in}}%
\pgfpathcurveto{\pgfqpoint{4.194873in}{2.277733in}}{\pgfqpoint{4.205473in}{2.282123in}}{\pgfqpoint{4.213286in}{2.289936in}}%
\pgfpathcurveto{\pgfqpoint{4.221100in}{2.297750in}}{\pgfqpoint{4.225490in}{2.308349in}}{\pgfqpoint{4.225490in}{2.319399in}}%
\pgfpathcurveto{\pgfqpoint{4.225490in}{2.330449in}}{\pgfqpoint{4.221100in}{2.341048in}}{\pgfqpoint{4.213286in}{2.348862in}}%
\pgfpathcurveto{\pgfqpoint{4.205473in}{2.356676in}}{\pgfqpoint{4.194873in}{2.361066in}}{\pgfqpoint{4.183823in}{2.361066in}}%
\pgfpathcurveto{\pgfqpoint{4.172773in}{2.361066in}}{\pgfqpoint{4.162174in}{2.356676in}}{\pgfqpoint{4.154361in}{2.348862in}}%
\pgfpathcurveto{\pgfqpoint{4.146547in}{2.341048in}}{\pgfqpoint{4.142157in}{2.330449in}}{\pgfqpoint{4.142157in}{2.319399in}}%
\pgfpathcurveto{\pgfqpoint{4.142157in}{2.308349in}}{\pgfqpoint{4.146547in}{2.297750in}}{\pgfqpoint{4.154361in}{2.289936in}}%
\pgfpathcurveto{\pgfqpoint{4.162174in}{2.282123in}}{\pgfqpoint{4.172773in}{2.277733in}}{\pgfqpoint{4.183823in}{2.277733in}}%
\pgfpathclose%
\pgfusepath{stroke,fill}%
\end{pgfscope}%
\begin{pgfscope}%
\pgfpathrectangle{\pgfqpoint{0.481978in}{0.331635in}}{\pgfqpoint{4.960000in}{3.696000in}}%
\pgfusepath{clip}%
\pgfsetbuttcap%
\pgfsetroundjoin%
\definecolor{currentfill}{rgb}{1.000000,0.705882,0.509804}%
\pgfsetfillcolor{currentfill}%
\pgfsetlinewidth{0.481800pt}%
\definecolor{currentstroke}{rgb}{1.000000,1.000000,1.000000}%
\pgfsetstrokecolor{currentstroke}%
\pgfsetdash{}{0pt}%
\pgfpathmoveto{\pgfqpoint{4.457059in}{2.781734in}}%
\pgfpathcurveto{\pgfqpoint{4.468109in}{2.781734in}}{\pgfqpoint{4.478708in}{2.786125in}}{\pgfqpoint{4.486522in}{2.793938in}}%
\pgfpathcurveto{\pgfqpoint{4.494335in}{2.801752in}}{\pgfqpoint{4.498726in}{2.812351in}}{\pgfqpoint{4.498726in}{2.823401in}}%
\pgfpathcurveto{\pgfqpoint{4.498726in}{2.834451in}}{\pgfqpoint{4.494335in}{2.845050in}}{\pgfqpoint{4.486522in}{2.852864in}}%
\pgfpathcurveto{\pgfqpoint{4.478708in}{2.860677in}}{\pgfqpoint{4.468109in}{2.865068in}}{\pgfqpoint{4.457059in}{2.865068in}}%
\pgfpathcurveto{\pgfqpoint{4.446009in}{2.865068in}}{\pgfqpoint{4.435410in}{2.860677in}}{\pgfqpoint{4.427596in}{2.852864in}}%
\pgfpathcurveto{\pgfqpoint{4.419783in}{2.845050in}}{\pgfqpoint{4.415392in}{2.834451in}}{\pgfqpoint{4.415392in}{2.823401in}}%
\pgfpathcurveto{\pgfqpoint{4.415392in}{2.812351in}}{\pgfqpoint{4.419783in}{2.801752in}}{\pgfqpoint{4.427596in}{2.793938in}}%
\pgfpathcurveto{\pgfqpoint{4.435410in}{2.786125in}}{\pgfqpoint{4.446009in}{2.781734in}}{\pgfqpoint{4.457059in}{2.781734in}}%
\pgfpathclose%
\pgfusepath{stroke,fill}%
\end{pgfscope}%
\begin{pgfscope}%
\pgfpathrectangle{\pgfqpoint{0.481978in}{0.331635in}}{\pgfqpoint{4.960000in}{3.696000in}}%
\pgfusepath{clip}%
\pgfsetbuttcap%
\pgfsetroundjoin%
\definecolor{currentfill}{rgb}{1.000000,0.705882,0.509804}%
\pgfsetfillcolor{currentfill}%
\pgfsetlinewidth{0.481800pt}%
\definecolor{currentstroke}{rgb}{1.000000,1.000000,1.000000}%
\pgfsetstrokecolor{currentstroke}%
\pgfsetdash{}{0pt}%
\pgfpathmoveto{\pgfqpoint{0.873344in}{1.877073in}}%
\pgfpathcurveto{\pgfqpoint{0.884394in}{1.877073in}}{\pgfqpoint{0.894993in}{1.881463in}}{\pgfqpoint{0.902807in}{1.889277in}}%
\pgfpathcurveto{\pgfqpoint{0.910621in}{1.897090in}}{\pgfqpoint{0.915011in}{1.907689in}}{\pgfqpoint{0.915011in}{1.918739in}}%
\pgfpathcurveto{\pgfqpoint{0.915011in}{1.929790in}}{\pgfqpoint{0.910621in}{1.940389in}}{\pgfqpoint{0.902807in}{1.948202in}}%
\pgfpathcurveto{\pgfqpoint{0.894993in}{1.956016in}}{\pgfqpoint{0.884394in}{1.960406in}}{\pgfqpoint{0.873344in}{1.960406in}}%
\pgfpathcurveto{\pgfqpoint{0.862294in}{1.960406in}}{\pgfqpoint{0.851695in}{1.956016in}}{\pgfqpoint{0.843882in}{1.948202in}}%
\pgfpathcurveto{\pgfqpoint{0.836068in}{1.940389in}}{\pgfqpoint{0.831678in}{1.929790in}}{\pgfqpoint{0.831678in}{1.918739in}}%
\pgfpathcurveto{\pgfqpoint{0.831678in}{1.907689in}}{\pgfqpoint{0.836068in}{1.897090in}}{\pgfqpoint{0.843882in}{1.889277in}}%
\pgfpathcurveto{\pgfqpoint{0.851695in}{1.881463in}}{\pgfqpoint{0.862294in}{1.877073in}}{\pgfqpoint{0.873344in}{1.877073in}}%
\pgfpathclose%
\pgfusepath{stroke,fill}%
\end{pgfscope}%
\begin{pgfscope}%
\pgfpathrectangle{\pgfqpoint{0.481978in}{0.331635in}}{\pgfqpoint{4.960000in}{3.696000in}}%
\pgfusepath{clip}%
\pgfsetbuttcap%
\pgfsetroundjoin%
\definecolor{currentfill}{rgb}{1.000000,0.705882,0.509804}%
\pgfsetfillcolor{currentfill}%
\pgfsetlinewidth{0.481800pt}%
\definecolor{currentstroke}{rgb}{1.000000,1.000000,1.000000}%
\pgfsetstrokecolor{currentstroke}%
\pgfsetdash{}{0pt}%
\pgfpathmoveto{\pgfqpoint{2.384182in}{1.952402in}}%
\pgfpathcurveto{\pgfqpoint{2.395232in}{1.952402in}}{\pgfqpoint{2.405831in}{1.956793in}}{\pgfqpoint{2.413645in}{1.964606in}}%
\pgfpathcurveto{\pgfqpoint{2.421459in}{1.972420in}}{\pgfqpoint{2.425849in}{1.983019in}}{\pgfqpoint{2.425849in}{1.994069in}}%
\pgfpathcurveto{\pgfqpoint{2.425849in}{2.005119in}}{\pgfqpoint{2.421459in}{2.015718in}}{\pgfqpoint{2.413645in}{2.023532in}}%
\pgfpathcurveto{\pgfqpoint{2.405831in}{2.031345in}}{\pgfqpoint{2.395232in}{2.035736in}}{\pgfqpoint{2.384182in}{2.035736in}}%
\pgfpathcurveto{\pgfqpoint{2.373132in}{2.035736in}}{\pgfqpoint{2.362533in}{2.031345in}}{\pgfqpoint{2.354719in}{2.023532in}}%
\pgfpathcurveto{\pgfqpoint{2.346906in}{2.015718in}}{\pgfqpoint{2.342516in}{2.005119in}}{\pgfqpoint{2.342516in}{1.994069in}}%
\pgfpathcurveto{\pgfqpoint{2.342516in}{1.983019in}}{\pgfqpoint{2.346906in}{1.972420in}}{\pgfqpoint{2.354719in}{1.964606in}}%
\pgfpathcurveto{\pgfqpoint{2.362533in}{1.956793in}}{\pgfqpoint{2.373132in}{1.952402in}}{\pgfqpoint{2.384182in}{1.952402in}}%
\pgfpathclose%
\pgfusepath{stroke,fill}%
\end{pgfscope}%
\begin{pgfscope}%
\pgfpathrectangle{\pgfqpoint{0.481978in}{0.331635in}}{\pgfqpoint{4.960000in}{3.696000in}}%
\pgfusepath{clip}%
\pgfsetbuttcap%
\pgfsetroundjoin%
\definecolor{currentfill}{rgb}{1.000000,0.705882,0.509804}%
\pgfsetfillcolor{currentfill}%
\pgfsetlinewidth{0.481800pt}%
\definecolor{currentstroke}{rgb}{1.000000,1.000000,1.000000}%
\pgfsetstrokecolor{currentstroke}%
\pgfsetdash{}{0pt}%
\pgfpathmoveto{\pgfqpoint{4.149641in}{3.102313in}}%
\pgfpathcurveto{\pgfqpoint{4.160691in}{3.102313in}}{\pgfqpoint{4.171290in}{3.106703in}}{\pgfqpoint{4.179104in}{3.114517in}}%
\pgfpathcurveto{\pgfqpoint{4.186918in}{3.122331in}}{\pgfqpoint{4.191308in}{3.132930in}}{\pgfqpoint{4.191308in}{3.143980in}}%
\pgfpathcurveto{\pgfqpoint{4.191308in}{3.155030in}}{\pgfqpoint{4.186918in}{3.165629in}}{\pgfqpoint{4.179104in}{3.173443in}}%
\pgfpathcurveto{\pgfqpoint{4.171290in}{3.181256in}}{\pgfqpoint{4.160691in}{3.185647in}}{\pgfqpoint{4.149641in}{3.185647in}}%
\pgfpathcurveto{\pgfqpoint{4.138591in}{3.185647in}}{\pgfqpoint{4.127992in}{3.181256in}}{\pgfqpoint{4.120179in}{3.173443in}}%
\pgfpathcurveto{\pgfqpoint{4.112365in}{3.165629in}}{\pgfqpoint{4.107975in}{3.155030in}}{\pgfqpoint{4.107975in}{3.143980in}}%
\pgfpathcurveto{\pgfqpoint{4.107975in}{3.132930in}}{\pgfqpoint{4.112365in}{3.122331in}}{\pgfqpoint{4.120179in}{3.114517in}}%
\pgfpathcurveto{\pgfqpoint{4.127992in}{3.106703in}}{\pgfqpoint{4.138591in}{3.102313in}}{\pgfqpoint{4.149641in}{3.102313in}}%
\pgfpathclose%
\pgfusepath{stroke,fill}%
\end{pgfscope}%
\begin{pgfscope}%
\pgfpathrectangle{\pgfqpoint{0.481978in}{0.331635in}}{\pgfqpoint{4.960000in}{3.696000in}}%
\pgfusepath{clip}%
\pgfsetbuttcap%
\pgfsetroundjoin%
\definecolor{currentfill}{rgb}{1.000000,0.705882,0.509804}%
\pgfsetfillcolor{currentfill}%
\pgfsetlinewidth{0.481800pt}%
\definecolor{currentstroke}{rgb}{1.000000,1.000000,1.000000}%
\pgfsetstrokecolor{currentstroke}%
\pgfsetdash{}{0pt}%
\pgfpathmoveto{\pgfqpoint{3.033751in}{1.699816in}}%
\pgfpathcurveto{\pgfqpoint{3.044801in}{1.699816in}}{\pgfqpoint{3.055400in}{1.704206in}}{\pgfqpoint{3.063214in}{1.712020in}}%
\pgfpathcurveto{\pgfqpoint{3.071027in}{1.719833in}}{\pgfqpoint{3.075418in}{1.730432in}}{\pgfqpoint{3.075418in}{1.741482in}}%
\pgfpathcurveto{\pgfqpoint{3.075418in}{1.752532in}}{\pgfqpoint{3.071027in}{1.763132in}}{\pgfqpoint{3.063214in}{1.770945in}}%
\pgfpathcurveto{\pgfqpoint{3.055400in}{1.778759in}}{\pgfqpoint{3.044801in}{1.783149in}}{\pgfqpoint{3.033751in}{1.783149in}}%
\pgfpathcurveto{\pgfqpoint{3.022701in}{1.783149in}}{\pgfqpoint{3.012102in}{1.778759in}}{\pgfqpoint{3.004288in}{1.770945in}}%
\pgfpathcurveto{\pgfqpoint{2.996474in}{1.763132in}}{\pgfqpoint{2.992084in}{1.752532in}}{\pgfqpoint{2.992084in}{1.741482in}}%
\pgfpathcurveto{\pgfqpoint{2.992084in}{1.730432in}}{\pgfqpoint{2.996474in}{1.719833in}}{\pgfqpoint{3.004288in}{1.712020in}}%
\pgfpathcurveto{\pgfqpoint{3.012102in}{1.704206in}}{\pgfqpoint{3.022701in}{1.699816in}}{\pgfqpoint{3.033751in}{1.699816in}}%
\pgfpathclose%
\pgfusepath{stroke,fill}%
\end{pgfscope}%
\begin{pgfscope}%
\pgfpathrectangle{\pgfqpoint{0.481978in}{0.331635in}}{\pgfqpoint{4.960000in}{3.696000in}}%
\pgfusepath{clip}%
\pgfsetbuttcap%
\pgfsetroundjoin%
\definecolor{currentfill}{rgb}{1.000000,0.705882,0.509804}%
\pgfsetfillcolor{currentfill}%
\pgfsetlinewidth{0.481800pt}%
\definecolor{currentstroke}{rgb}{1.000000,1.000000,1.000000}%
\pgfsetstrokecolor{currentstroke}%
\pgfsetdash{}{0pt}%
\pgfpathmoveto{\pgfqpoint{2.729878in}{1.087110in}}%
\pgfpathcurveto{\pgfqpoint{2.740928in}{1.087110in}}{\pgfqpoint{2.751527in}{1.091500in}}{\pgfqpoint{2.759341in}{1.099314in}}%
\pgfpathcurveto{\pgfqpoint{2.767154in}{1.107127in}}{\pgfqpoint{2.771544in}{1.117726in}}{\pgfqpoint{2.771544in}{1.128776in}}%
\pgfpathcurveto{\pgfqpoint{2.771544in}{1.139827in}}{\pgfqpoint{2.767154in}{1.150426in}}{\pgfqpoint{2.759341in}{1.158239in}}%
\pgfpathcurveto{\pgfqpoint{2.751527in}{1.166053in}}{\pgfqpoint{2.740928in}{1.170443in}}{\pgfqpoint{2.729878in}{1.170443in}}%
\pgfpathcurveto{\pgfqpoint{2.718828in}{1.170443in}}{\pgfqpoint{2.708229in}{1.166053in}}{\pgfqpoint{2.700415in}{1.158239in}}%
\pgfpathcurveto{\pgfqpoint{2.692601in}{1.150426in}}{\pgfqpoint{2.688211in}{1.139827in}}{\pgfqpoint{2.688211in}{1.128776in}}%
\pgfpathcurveto{\pgfqpoint{2.688211in}{1.117726in}}{\pgfqpoint{2.692601in}{1.107127in}}{\pgfqpoint{2.700415in}{1.099314in}}%
\pgfpathcurveto{\pgfqpoint{2.708229in}{1.091500in}}{\pgfqpoint{2.718828in}{1.087110in}}{\pgfqpoint{2.729878in}{1.087110in}}%
\pgfpathclose%
\pgfusepath{stroke,fill}%
\end{pgfscope}%
\begin{pgfscope}%
\pgfpathrectangle{\pgfqpoint{0.481978in}{0.331635in}}{\pgfqpoint{4.960000in}{3.696000in}}%
\pgfusepath{clip}%
\pgfsetbuttcap%
\pgfsetroundjoin%
\definecolor{currentfill}{rgb}{1.000000,0.705882,0.509804}%
\pgfsetfillcolor{currentfill}%
\pgfsetlinewidth{0.481800pt}%
\definecolor{currentstroke}{rgb}{1.000000,1.000000,1.000000}%
\pgfsetstrokecolor{currentstroke}%
\pgfsetdash{}{0pt}%
\pgfpathmoveto{\pgfqpoint{1.117610in}{2.071317in}}%
\pgfpathcurveto{\pgfqpoint{1.128660in}{2.071317in}}{\pgfqpoint{1.139259in}{2.075708in}}{\pgfqpoint{1.147072in}{2.083521in}}%
\pgfpathcurveto{\pgfqpoint{1.154886in}{2.091335in}}{\pgfqpoint{1.159276in}{2.101934in}}{\pgfqpoint{1.159276in}{2.112984in}}%
\pgfpathcurveto{\pgfqpoint{1.159276in}{2.124034in}}{\pgfqpoint{1.154886in}{2.134633in}}{\pgfqpoint{1.147072in}{2.142447in}}%
\pgfpathcurveto{\pgfqpoint{1.139259in}{2.150261in}}{\pgfqpoint{1.128660in}{2.154651in}}{\pgfqpoint{1.117610in}{2.154651in}}%
\pgfpathcurveto{\pgfqpoint{1.106559in}{2.154651in}}{\pgfqpoint{1.095960in}{2.150261in}}{\pgfqpoint{1.088147in}{2.142447in}}%
\pgfpathcurveto{\pgfqpoint{1.080333in}{2.134633in}}{\pgfqpoint{1.075943in}{2.124034in}}{\pgfqpoint{1.075943in}{2.112984in}}%
\pgfpathcurveto{\pgfqpoint{1.075943in}{2.101934in}}{\pgfqpoint{1.080333in}{2.091335in}}{\pgfqpoint{1.088147in}{2.083521in}}%
\pgfpathcurveto{\pgfqpoint{1.095960in}{2.075708in}}{\pgfqpoint{1.106559in}{2.071317in}}{\pgfqpoint{1.117610in}{2.071317in}}%
\pgfpathclose%
\pgfusepath{stroke,fill}%
\end{pgfscope}%
\begin{pgfscope}%
\pgfpathrectangle{\pgfqpoint{0.481978in}{0.331635in}}{\pgfqpoint{4.960000in}{3.696000in}}%
\pgfusepath{clip}%
\pgfsetbuttcap%
\pgfsetroundjoin%
\definecolor{currentfill}{rgb}{1.000000,0.705882,0.509804}%
\pgfsetfillcolor{currentfill}%
\pgfsetlinewidth{0.481800pt}%
\definecolor{currentstroke}{rgb}{1.000000,1.000000,1.000000}%
\pgfsetstrokecolor{currentstroke}%
\pgfsetdash{}{0pt}%
\pgfpathmoveto{\pgfqpoint{4.325352in}{1.184844in}}%
\pgfpathcurveto{\pgfqpoint{4.336402in}{1.184844in}}{\pgfqpoint{4.347001in}{1.189235in}}{\pgfqpoint{4.354815in}{1.197048in}}%
\pgfpathcurveto{\pgfqpoint{4.362628in}{1.204862in}}{\pgfqpoint{4.367019in}{1.215461in}}{\pgfqpoint{4.367019in}{1.226511in}}%
\pgfpathcurveto{\pgfqpoint{4.367019in}{1.237561in}}{\pgfqpoint{4.362628in}{1.248160in}}{\pgfqpoint{4.354815in}{1.255974in}}%
\pgfpathcurveto{\pgfqpoint{4.347001in}{1.263787in}}{\pgfqpoint{4.336402in}{1.268178in}}{\pgfqpoint{4.325352in}{1.268178in}}%
\pgfpathcurveto{\pgfqpoint{4.314302in}{1.268178in}}{\pgfqpoint{4.303703in}{1.263787in}}{\pgfqpoint{4.295889in}{1.255974in}}%
\pgfpathcurveto{\pgfqpoint{4.288076in}{1.248160in}}{\pgfqpoint{4.283685in}{1.237561in}}{\pgfqpoint{4.283685in}{1.226511in}}%
\pgfpathcurveto{\pgfqpoint{4.283685in}{1.215461in}}{\pgfqpoint{4.288076in}{1.204862in}}{\pgfqpoint{4.295889in}{1.197048in}}%
\pgfpathcurveto{\pgfqpoint{4.303703in}{1.189235in}}{\pgfqpoint{4.314302in}{1.184844in}}{\pgfqpoint{4.325352in}{1.184844in}}%
\pgfpathclose%
\pgfusepath{stroke,fill}%
\end{pgfscope}%
\begin{pgfscope}%
\pgfpathrectangle{\pgfqpoint{0.481978in}{0.331635in}}{\pgfqpoint{4.960000in}{3.696000in}}%
\pgfusepath{clip}%
\pgfsetbuttcap%
\pgfsetroundjoin%
\definecolor{currentfill}{rgb}{1.000000,0.705882,0.509804}%
\pgfsetfillcolor{currentfill}%
\pgfsetlinewidth{0.481800pt}%
\definecolor{currentstroke}{rgb}{1.000000,1.000000,1.000000}%
\pgfsetstrokecolor{currentstroke}%
\pgfsetdash{}{0pt}%
\pgfpathmoveto{\pgfqpoint{3.380981in}{3.288300in}}%
\pgfpathcurveto{\pgfqpoint{3.392031in}{3.288300in}}{\pgfqpoint{3.402630in}{3.292690in}}{\pgfqpoint{3.410443in}{3.300504in}}%
\pgfpathcurveto{\pgfqpoint{3.418257in}{3.308318in}}{\pgfqpoint{3.422647in}{3.318917in}}{\pgfqpoint{3.422647in}{3.329967in}}%
\pgfpathcurveto{\pgfqpoint{3.422647in}{3.341017in}}{\pgfqpoint{3.418257in}{3.351616in}}{\pgfqpoint{3.410443in}{3.359430in}}%
\pgfpathcurveto{\pgfqpoint{3.402630in}{3.367243in}}{\pgfqpoint{3.392031in}{3.371634in}}{\pgfqpoint{3.380981in}{3.371634in}}%
\pgfpathcurveto{\pgfqpoint{3.369930in}{3.371634in}}{\pgfqpoint{3.359331in}{3.367243in}}{\pgfqpoint{3.351518in}{3.359430in}}%
\pgfpathcurveto{\pgfqpoint{3.343704in}{3.351616in}}{\pgfqpoint{3.339314in}{3.341017in}}{\pgfqpoint{3.339314in}{3.329967in}}%
\pgfpathcurveto{\pgfqpoint{3.339314in}{3.318917in}}{\pgfqpoint{3.343704in}{3.308318in}}{\pgfqpoint{3.351518in}{3.300504in}}%
\pgfpathcurveto{\pgfqpoint{3.359331in}{3.292690in}}{\pgfqpoint{3.369930in}{3.288300in}}{\pgfqpoint{3.380981in}{3.288300in}}%
\pgfpathclose%
\pgfusepath{stroke,fill}%
\end{pgfscope}%
\begin{pgfscope}%
\pgfpathrectangle{\pgfqpoint{0.481978in}{0.331635in}}{\pgfqpoint{4.960000in}{3.696000in}}%
\pgfusepath{clip}%
\pgfsetbuttcap%
\pgfsetroundjoin%
\definecolor{currentfill}{rgb}{1.000000,0.705882,0.509804}%
\pgfsetfillcolor{currentfill}%
\pgfsetlinewidth{0.481800pt}%
\definecolor{currentstroke}{rgb}{1.000000,1.000000,1.000000}%
\pgfsetstrokecolor{currentstroke}%
\pgfsetdash{}{0pt}%
\pgfpathmoveto{\pgfqpoint{3.963164in}{0.902601in}}%
\pgfpathcurveto{\pgfqpoint{3.974214in}{0.902601in}}{\pgfqpoint{3.984813in}{0.906991in}}{\pgfqpoint{3.992626in}{0.914805in}}%
\pgfpathcurveto{\pgfqpoint{4.000440in}{0.922619in}}{\pgfqpoint{4.004830in}{0.933218in}}{\pgfqpoint{4.004830in}{0.944268in}}%
\pgfpathcurveto{\pgfqpoint{4.004830in}{0.955318in}}{\pgfqpoint{4.000440in}{0.965917in}}{\pgfqpoint{3.992626in}{0.973731in}}%
\pgfpathcurveto{\pgfqpoint{3.984813in}{0.981544in}}{\pgfqpoint{3.974214in}{0.985934in}}{\pgfqpoint{3.963164in}{0.985934in}}%
\pgfpathcurveto{\pgfqpoint{3.952114in}{0.985934in}}{\pgfqpoint{3.941515in}{0.981544in}}{\pgfqpoint{3.933701in}{0.973731in}}%
\pgfpathcurveto{\pgfqpoint{3.925887in}{0.965917in}}{\pgfqpoint{3.921497in}{0.955318in}}{\pgfqpoint{3.921497in}{0.944268in}}%
\pgfpathcurveto{\pgfqpoint{3.921497in}{0.933218in}}{\pgfqpoint{3.925887in}{0.922619in}}{\pgfqpoint{3.933701in}{0.914805in}}%
\pgfpathcurveto{\pgfqpoint{3.941515in}{0.906991in}}{\pgfqpoint{3.952114in}{0.902601in}}{\pgfqpoint{3.963164in}{0.902601in}}%
\pgfpathclose%
\pgfusepath{stroke,fill}%
\end{pgfscope}%
\begin{pgfscope}%
\pgfpathrectangle{\pgfqpoint{0.481978in}{0.331635in}}{\pgfqpoint{4.960000in}{3.696000in}}%
\pgfusepath{clip}%
\pgfsetbuttcap%
\pgfsetroundjoin%
\definecolor{currentfill}{rgb}{1.000000,0.705882,0.509804}%
\pgfsetfillcolor{currentfill}%
\pgfsetlinewidth{0.481800pt}%
\definecolor{currentstroke}{rgb}{1.000000,1.000000,1.000000}%
\pgfsetstrokecolor{currentstroke}%
\pgfsetdash{}{0pt}%
\pgfpathmoveto{\pgfqpoint{4.651146in}{2.523949in}}%
\pgfpathcurveto{\pgfqpoint{4.662196in}{2.523949in}}{\pgfqpoint{4.672795in}{2.528339in}}{\pgfqpoint{4.680609in}{2.536153in}}%
\pgfpathcurveto{\pgfqpoint{4.688423in}{2.543966in}}{\pgfqpoint{4.692813in}{2.554565in}}{\pgfqpoint{4.692813in}{2.565615in}}%
\pgfpathcurveto{\pgfqpoint{4.692813in}{2.576665in}}{\pgfqpoint{4.688423in}{2.587264in}}{\pgfqpoint{4.680609in}{2.595078in}}%
\pgfpathcurveto{\pgfqpoint{4.672795in}{2.602892in}}{\pgfqpoint{4.662196in}{2.607282in}}{\pgfqpoint{4.651146in}{2.607282in}}%
\pgfpathcurveto{\pgfqpoint{4.640096in}{2.607282in}}{\pgfqpoint{4.629497in}{2.602892in}}{\pgfqpoint{4.621683in}{2.595078in}}%
\pgfpathcurveto{\pgfqpoint{4.613870in}{2.587264in}}{\pgfqpoint{4.609480in}{2.576665in}}{\pgfqpoint{4.609480in}{2.565615in}}%
\pgfpathcurveto{\pgfqpoint{4.609480in}{2.554565in}}{\pgfqpoint{4.613870in}{2.543966in}}{\pgfqpoint{4.621683in}{2.536153in}}%
\pgfpathcurveto{\pgfqpoint{4.629497in}{2.528339in}}{\pgfqpoint{4.640096in}{2.523949in}}{\pgfqpoint{4.651146in}{2.523949in}}%
\pgfpathclose%
\pgfusepath{stroke,fill}%
\end{pgfscope}%
\begin{pgfscope}%
\pgfpathrectangle{\pgfqpoint{0.481978in}{0.331635in}}{\pgfqpoint{4.960000in}{3.696000in}}%
\pgfusepath{clip}%
\pgfsetbuttcap%
\pgfsetroundjoin%
\definecolor{currentfill}{rgb}{1.000000,0.705882,0.509804}%
\pgfsetfillcolor{currentfill}%
\pgfsetlinewidth{0.481800pt}%
\definecolor{currentstroke}{rgb}{1.000000,1.000000,1.000000}%
\pgfsetstrokecolor{currentstroke}%
\pgfsetdash{}{0pt}%
\pgfpathmoveto{\pgfqpoint{2.366333in}{3.341327in}}%
\pgfpathcurveto{\pgfqpoint{2.377383in}{3.341327in}}{\pgfqpoint{2.387982in}{3.345717in}}{\pgfqpoint{2.395796in}{3.353531in}}%
\pgfpathcurveto{\pgfqpoint{2.403610in}{3.361345in}}{\pgfqpoint{2.408000in}{3.371944in}}{\pgfqpoint{2.408000in}{3.382994in}}%
\pgfpathcurveto{\pgfqpoint{2.408000in}{3.394044in}}{\pgfqpoint{2.403610in}{3.404643in}}{\pgfqpoint{2.395796in}{3.412456in}}%
\pgfpathcurveto{\pgfqpoint{2.387982in}{3.420270in}}{\pgfqpoint{2.377383in}{3.424660in}}{\pgfqpoint{2.366333in}{3.424660in}}%
\pgfpathcurveto{\pgfqpoint{2.355283in}{3.424660in}}{\pgfqpoint{2.344684in}{3.420270in}}{\pgfqpoint{2.336870in}{3.412456in}}%
\pgfpathcurveto{\pgfqpoint{2.329057in}{3.404643in}}{\pgfqpoint{2.324666in}{3.394044in}}{\pgfqpoint{2.324666in}{3.382994in}}%
\pgfpathcurveto{\pgfqpoint{2.324666in}{3.371944in}}{\pgfqpoint{2.329057in}{3.361345in}}{\pgfqpoint{2.336870in}{3.353531in}}%
\pgfpathcurveto{\pgfqpoint{2.344684in}{3.345717in}}{\pgfqpoint{2.355283in}{3.341327in}}{\pgfqpoint{2.366333in}{3.341327in}}%
\pgfpathclose%
\pgfusepath{stroke,fill}%
\end{pgfscope}%
\begin{pgfscope}%
\pgfpathrectangle{\pgfqpoint{0.481978in}{0.331635in}}{\pgfqpoint{4.960000in}{3.696000in}}%
\pgfusepath{clip}%
\pgfsetbuttcap%
\pgfsetroundjoin%
\definecolor{currentfill}{rgb}{1.000000,0.705882,0.509804}%
\pgfsetfillcolor{currentfill}%
\pgfsetlinewidth{0.481800pt}%
\definecolor{currentstroke}{rgb}{1.000000,1.000000,1.000000}%
\pgfsetstrokecolor{currentstroke}%
\pgfsetdash{}{0pt}%
\pgfpathmoveto{\pgfqpoint{3.263269in}{1.787512in}}%
\pgfpathcurveto{\pgfqpoint{3.274319in}{1.787512in}}{\pgfqpoint{3.284918in}{1.791903in}}{\pgfqpoint{3.292732in}{1.799716in}}%
\pgfpathcurveto{\pgfqpoint{3.300545in}{1.807530in}}{\pgfqpoint{3.304936in}{1.818129in}}{\pgfqpoint{3.304936in}{1.829179in}}%
\pgfpathcurveto{\pgfqpoint{3.304936in}{1.840229in}}{\pgfqpoint{3.300545in}{1.850828in}}{\pgfqpoint{3.292732in}{1.858642in}}%
\pgfpathcurveto{\pgfqpoint{3.284918in}{1.866456in}}{\pgfqpoint{3.274319in}{1.870846in}}{\pgfqpoint{3.263269in}{1.870846in}}%
\pgfpathcurveto{\pgfqpoint{3.252219in}{1.870846in}}{\pgfqpoint{3.241620in}{1.866456in}}{\pgfqpoint{3.233806in}{1.858642in}}%
\pgfpathcurveto{\pgfqpoint{3.225993in}{1.850828in}}{\pgfqpoint{3.221602in}{1.840229in}}{\pgfqpoint{3.221602in}{1.829179in}}%
\pgfpathcurveto{\pgfqpoint{3.221602in}{1.818129in}}{\pgfqpoint{3.225993in}{1.807530in}}{\pgfqpoint{3.233806in}{1.799716in}}%
\pgfpathcurveto{\pgfqpoint{3.241620in}{1.791903in}}{\pgfqpoint{3.252219in}{1.787512in}}{\pgfqpoint{3.263269in}{1.787512in}}%
\pgfpathclose%
\pgfusepath{stroke,fill}%
\end{pgfscope}%
\begin{pgfscope}%
\pgfpathrectangle{\pgfqpoint{0.481978in}{0.331635in}}{\pgfqpoint{4.960000in}{3.696000in}}%
\pgfusepath{clip}%
\pgfsetbuttcap%
\pgfsetroundjoin%
\definecolor{currentfill}{rgb}{1.000000,0.705882,0.509804}%
\pgfsetfillcolor{currentfill}%
\pgfsetlinewidth{0.481800pt}%
\definecolor{currentstroke}{rgb}{1.000000,1.000000,1.000000}%
\pgfsetstrokecolor{currentstroke}%
\pgfsetdash{}{0pt}%
\pgfpathmoveto{\pgfqpoint{3.402580in}{1.635537in}}%
\pgfpathcurveto{\pgfqpoint{3.413631in}{1.635537in}}{\pgfqpoint{3.424230in}{1.639927in}}{\pgfqpoint{3.432043in}{1.647741in}}%
\pgfpathcurveto{\pgfqpoint{3.439857in}{1.655554in}}{\pgfqpoint{3.444247in}{1.666153in}}{\pgfqpoint{3.444247in}{1.677203in}}%
\pgfpathcurveto{\pgfqpoint{3.444247in}{1.688254in}}{\pgfqpoint{3.439857in}{1.698853in}}{\pgfqpoint{3.432043in}{1.706666in}}%
\pgfpathcurveto{\pgfqpoint{3.424230in}{1.714480in}}{\pgfqpoint{3.413631in}{1.718870in}}{\pgfqpoint{3.402580in}{1.718870in}}%
\pgfpathcurveto{\pgfqpoint{3.391530in}{1.718870in}}{\pgfqpoint{3.380931in}{1.714480in}}{\pgfqpoint{3.373118in}{1.706666in}}%
\pgfpathcurveto{\pgfqpoint{3.365304in}{1.698853in}}{\pgfqpoint{3.360914in}{1.688254in}}{\pgfqpoint{3.360914in}{1.677203in}}%
\pgfpathcurveto{\pgfqpoint{3.360914in}{1.666153in}}{\pgfqpoint{3.365304in}{1.655554in}}{\pgfqpoint{3.373118in}{1.647741in}}%
\pgfpathcurveto{\pgfqpoint{3.380931in}{1.639927in}}{\pgfqpoint{3.391530in}{1.635537in}}{\pgfqpoint{3.402580in}{1.635537in}}%
\pgfpathclose%
\pgfusepath{stroke,fill}%
\end{pgfscope}%
\begin{pgfscope}%
\pgfpathrectangle{\pgfqpoint{0.481978in}{0.331635in}}{\pgfqpoint{4.960000in}{3.696000in}}%
\pgfusepath{clip}%
\pgfsetbuttcap%
\pgfsetroundjoin%
\definecolor{currentfill}{rgb}{1.000000,0.705882,0.509804}%
\pgfsetfillcolor{currentfill}%
\pgfsetlinewidth{0.481800pt}%
\definecolor{currentstroke}{rgb}{1.000000,1.000000,1.000000}%
\pgfsetstrokecolor{currentstroke}%
\pgfsetdash{}{0pt}%
\pgfpathmoveto{\pgfqpoint{2.928313in}{2.719747in}}%
\pgfpathcurveto{\pgfqpoint{2.939364in}{2.719747in}}{\pgfqpoint{2.949963in}{2.724137in}}{\pgfqpoint{2.957776in}{2.731951in}}%
\pgfpathcurveto{\pgfqpoint{2.965590in}{2.739764in}}{\pgfqpoint{2.969980in}{2.750363in}}{\pgfqpoint{2.969980in}{2.761414in}}%
\pgfpathcurveto{\pgfqpoint{2.969980in}{2.772464in}}{\pgfqpoint{2.965590in}{2.783063in}}{\pgfqpoint{2.957776in}{2.790876in}}%
\pgfpathcurveto{\pgfqpoint{2.949963in}{2.798690in}}{\pgfqpoint{2.939364in}{2.803080in}}{\pgfqpoint{2.928313in}{2.803080in}}%
\pgfpathcurveto{\pgfqpoint{2.917263in}{2.803080in}}{\pgfqpoint{2.906664in}{2.798690in}}{\pgfqpoint{2.898851in}{2.790876in}}%
\pgfpathcurveto{\pgfqpoint{2.891037in}{2.783063in}}{\pgfqpoint{2.886647in}{2.772464in}}{\pgfqpoint{2.886647in}{2.761414in}}%
\pgfpathcurveto{\pgfqpoint{2.886647in}{2.750363in}}{\pgfqpoint{2.891037in}{2.739764in}}{\pgfqpoint{2.898851in}{2.731951in}}%
\pgfpathcurveto{\pgfqpoint{2.906664in}{2.724137in}}{\pgfqpoint{2.917263in}{2.719747in}}{\pgfqpoint{2.928313in}{2.719747in}}%
\pgfpathclose%
\pgfusepath{stroke,fill}%
\end{pgfscope}%
\begin{pgfscope}%
\pgfpathrectangle{\pgfqpoint{0.481978in}{0.331635in}}{\pgfqpoint{4.960000in}{3.696000in}}%
\pgfusepath{clip}%
\pgfsetbuttcap%
\pgfsetroundjoin%
\definecolor{currentfill}{rgb}{1.000000,0.705882,0.509804}%
\pgfsetfillcolor{currentfill}%
\pgfsetlinewidth{0.481800pt}%
\definecolor{currentstroke}{rgb}{1.000000,1.000000,1.000000}%
\pgfsetstrokecolor{currentstroke}%
\pgfsetdash{}{0pt}%
\pgfpathmoveto{\pgfqpoint{4.201586in}{0.857799in}}%
\pgfpathcurveto{\pgfqpoint{4.212636in}{0.857799in}}{\pgfqpoint{4.223235in}{0.862189in}}{\pgfqpoint{4.231049in}{0.870003in}}%
\pgfpathcurveto{\pgfqpoint{4.238862in}{0.877816in}}{\pgfqpoint{4.243253in}{0.888416in}}{\pgfqpoint{4.243253in}{0.899466in}}%
\pgfpathcurveto{\pgfqpoint{4.243253in}{0.910516in}}{\pgfqpoint{4.238862in}{0.921115in}}{\pgfqpoint{4.231049in}{0.928928in}}%
\pgfpathcurveto{\pgfqpoint{4.223235in}{0.936742in}}{\pgfqpoint{4.212636in}{0.941132in}}{\pgfqpoint{4.201586in}{0.941132in}}%
\pgfpathcurveto{\pgfqpoint{4.190536in}{0.941132in}}{\pgfqpoint{4.179937in}{0.936742in}}{\pgfqpoint{4.172123in}{0.928928in}}%
\pgfpathcurveto{\pgfqpoint{4.164310in}{0.921115in}}{\pgfqpoint{4.159919in}{0.910516in}}{\pgfqpoint{4.159919in}{0.899466in}}%
\pgfpathcurveto{\pgfqpoint{4.159919in}{0.888416in}}{\pgfqpoint{4.164310in}{0.877816in}}{\pgfqpoint{4.172123in}{0.870003in}}%
\pgfpathcurveto{\pgfqpoint{4.179937in}{0.862189in}}{\pgfqpoint{4.190536in}{0.857799in}}{\pgfqpoint{4.201586in}{0.857799in}}%
\pgfpathclose%
\pgfusepath{stroke,fill}%
\end{pgfscope}%
\begin{pgfscope}%
\pgfpathrectangle{\pgfqpoint{0.481978in}{0.331635in}}{\pgfqpoint{4.960000in}{3.696000in}}%
\pgfusepath{clip}%
\pgfsetbuttcap%
\pgfsetroundjoin%
\definecolor{currentfill}{rgb}{1.000000,0.705882,0.509804}%
\pgfsetfillcolor{currentfill}%
\pgfsetlinewidth{0.481800pt}%
\definecolor{currentstroke}{rgb}{1.000000,1.000000,1.000000}%
\pgfsetstrokecolor{currentstroke}%
\pgfsetdash{}{0pt}%
\pgfpathmoveto{\pgfqpoint{2.639691in}{0.999613in}}%
\pgfpathcurveto{\pgfqpoint{2.650741in}{0.999613in}}{\pgfqpoint{2.661340in}{1.004004in}}{\pgfqpoint{2.669153in}{1.011817in}}%
\pgfpathcurveto{\pgfqpoint{2.676967in}{1.019631in}}{\pgfqpoint{2.681357in}{1.030230in}}{\pgfqpoint{2.681357in}{1.041280in}}%
\pgfpathcurveto{\pgfqpoint{2.681357in}{1.052330in}}{\pgfqpoint{2.676967in}{1.062929in}}{\pgfqpoint{2.669153in}{1.070743in}}%
\pgfpathcurveto{\pgfqpoint{2.661340in}{1.078557in}}{\pgfqpoint{2.650741in}{1.082947in}}{\pgfqpoint{2.639691in}{1.082947in}}%
\pgfpathcurveto{\pgfqpoint{2.628640in}{1.082947in}}{\pgfqpoint{2.618041in}{1.078557in}}{\pgfqpoint{2.610228in}{1.070743in}}%
\pgfpathcurveto{\pgfqpoint{2.602414in}{1.062929in}}{\pgfqpoint{2.598024in}{1.052330in}}{\pgfqpoint{2.598024in}{1.041280in}}%
\pgfpathcurveto{\pgfqpoint{2.598024in}{1.030230in}}{\pgfqpoint{2.602414in}{1.019631in}}{\pgfqpoint{2.610228in}{1.011817in}}%
\pgfpathcurveto{\pgfqpoint{2.618041in}{1.004004in}}{\pgfqpoint{2.628640in}{0.999613in}}{\pgfqpoint{2.639691in}{0.999613in}}%
\pgfpathclose%
\pgfusepath{stroke,fill}%
\end{pgfscope}%
\begin{pgfscope}%
\pgfpathrectangle{\pgfqpoint{0.481978in}{0.331635in}}{\pgfqpoint{4.960000in}{3.696000in}}%
\pgfusepath{clip}%
\pgfsetbuttcap%
\pgfsetroundjoin%
\definecolor{currentfill}{rgb}{1.000000,0.705882,0.509804}%
\pgfsetfillcolor{currentfill}%
\pgfsetlinewidth{0.481800pt}%
\definecolor{currentstroke}{rgb}{1.000000,1.000000,1.000000}%
\pgfsetstrokecolor{currentstroke}%
\pgfsetdash{}{0pt}%
\pgfpathmoveto{\pgfqpoint{2.932478in}{1.618178in}}%
\pgfpathcurveto{\pgfqpoint{2.943528in}{1.618178in}}{\pgfqpoint{2.954127in}{1.622568in}}{\pgfqpoint{2.961941in}{1.630382in}}%
\pgfpathcurveto{\pgfqpoint{2.969754in}{1.638196in}}{\pgfqpoint{2.974144in}{1.648795in}}{\pgfqpoint{2.974144in}{1.659845in}}%
\pgfpathcurveto{\pgfqpoint{2.974144in}{1.670895in}}{\pgfqpoint{2.969754in}{1.681494in}}{\pgfqpoint{2.961941in}{1.689308in}}%
\pgfpathcurveto{\pgfqpoint{2.954127in}{1.697121in}}{\pgfqpoint{2.943528in}{1.701512in}}{\pgfqpoint{2.932478in}{1.701512in}}%
\pgfpathcurveto{\pgfqpoint{2.921428in}{1.701512in}}{\pgfqpoint{2.910829in}{1.697121in}}{\pgfqpoint{2.903015in}{1.689308in}}%
\pgfpathcurveto{\pgfqpoint{2.895201in}{1.681494in}}{\pgfqpoint{2.890811in}{1.670895in}}{\pgfqpoint{2.890811in}{1.659845in}}%
\pgfpathcurveto{\pgfqpoint{2.890811in}{1.648795in}}{\pgfqpoint{2.895201in}{1.638196in}}{\pgfqpoint{2.903015in}{1.630382in}}%
\pgfpathcurveto{\pgfqpoint{2.910829in}{1.622568in}}{\pgfqpoint{2.921428in}{1.618178in}}{\pgfqpoint{2.932478in}{1.618178in}}%
\pgfpathclose%
\pgfusepath{stroke,fill}%
\end{pgfscope}%
\begin{pgfscope}%
\pgfpathrectangle{\pgfqpoint{0.481978in}{0.331635in}}{\pgfqpoint{4.960000in}{3.696000in}}%
\pgfusepath{clip}%
\pgfsetbuttcap%
\pgfsetroundjoin%
\definecolor{currentfill}{rgb}{1.000000,0.705882,0.509804}%
\pgfsetfillcolor{currentfill}%
\pgfsetlinewidth{0.481800pt}%
\definecolor{currentstroke}{rgb}{1.000000,1.000000,1.000000}%
\pgfsetstrokecolor{currentstroke}%
\pgfsetdash{}{0pt}%
\pgfpathmoveto{\pgfqpoint{4.484479in}{2.938711in}}%
\pgfpathcurveto{\pgfqpoint{4.495529in}{2.938711in}}{\pgfqpoint{4.506128in}{2.943101in}}{\pgfqpoint{4.513942in}{2.950915in}}%
\pgfpathcurveto{\pgfqpoint{4.521756in}{2.958729in}}{\pgfqpoint{4.526146in}{2.969328in}}{\pgfqpoint{4.526146in}{2.980378in}}%
\pgfpathcurveto{\pgfqpoint{4.526146in}{2.991428in}}{\pgfqpoint{4.521756in}{3.002027in}}{\pgfqpoint{4.513942in}{3.009840in}}%
\pgfpathcurveto{\pgfqpoint{4.506128in}{3.017654in}}{\pgfqpoint{4.495529in}{3.022044in}}{\pgfqpoint{4.484479in}{3.022044in}}%
\pgfpathcurveto{\pgfqpoint{4.473429in}{3.022044in}}{\pgfqpoint{4.462830in}{3.017654in}}{\pgfqpoint{4.455016in}{3.009840in}}%
\pgfpathcurveto{\pgfqpoint{4.447203in}{3.002027in}}{\pgfqpoint{4.442813in}{2.991428in}}{\pgfqpoint{4.442813in}{2.980378in}}%
\pgfpathcurveto{\pgfqpoint{4.442813in}{2.969328in}}{\pgfqpoint{4.447203in}{2.958729in}}{\pgfqpoint{4.455016in}{2.950915in}}%
\pgfpathcurveto{\pgfqpoint{4.462830in}{2.943101in}}{\pgfqpoint{4.473429in}{2.938711in}}{\pgfqpoint{4.484479in}{2.938711in}}%
\pgfpathclose%
\pgfusepath{stroke,fill}%
\end{pgfscope}%
\begin{pgfscope}%
\pgfpathrectangle{\pgfqpoint{0.481978in}{0.331635in}}{\pgfqpoint{4.960000in}{3.696000in}}%
\pgfusepath{clip}%
\pgfsetbuttcap%
\pgfsetroundjoin%
\definecolor{currentfill}{rgb}{1.000000,0.705882,0.509804}%
\pgfsetfillcolor{currentfill}%
\pgfsetlinewidth{0.481800pt}%
\definecolor{currentstroke}{rgb}{1.000000,1.000000,1.000000}%
\pgfsetstrokecolor{currentstroke}%
\pgfsetdash{}{0pt}%
\pgfpathmoveto{\pgfqpoint{1.750152in}{1.325224in}}%
\pgfpathcurveto{\pgfqpoint{1.761203in}{1.325224in}}{\pgfqpoint{1.771802in}{1.329614in}}{\pgfqpoint{1.779615in}{1.337428in}}%
\pgfpathcurveto{\pgfqpoint{1.787429in}{1.345241in}}{\pgfqpoint{1.791819in}{1.355840in}}{\pgfqpoint{1.791819in}{1.366890in}}%
\pgfpathcurveto{\pgfqpoint{1.791819in}{1.377940in}}{\pgfqpoint{1.787429in}{1.388539in}}{\pgfqpoint{1.779615in}{1.396353in}}%
\pgfpathcurveto{\pgfqpoint{1.771802in}{1.404167in}}{\pgfqpoint{1.761203in}{1.408557in}}{\pgfqpoint{1.750152in}{1.408557in}}%
\pgfpathcurveto{\pgfqpoint{1.739102in}{1.408557in}}{\pgfqpoint{1.728503in}{1.404167in}}{\pgfqpoint{1.720690in}{1.396353in}}%
\pgfpathcurveto{\pgfqpoint{1.712876in}{1.388539in}}{\pgfqpoint{1.708486in}{1.377940in}}{\pgfqpoint{1.708486in}{1.366890in}}%
\pgfpathcurveto{\pgfqpoint{1.708486in}{1.355840in}}{\pgfqpoint{1.712876in}{1.345241in}}{\pgfqpoint{1.720690in}{1.337428in}}%
\pgfpathcurveto{\pgfqpoint{1.728503in}{1.329614in}}{\pgfqpoint{1.739102in}{1.325224in}}{\pgfqpoint{1.750152in}{1.325224in}}%
\pgfpathclose%
\pgfusepath{stroke,fill}%
\end{pgfscope}%
\begin{pgfscope}%
\pgfpathrectangle{\pgfqpoint{0.481978in}{0.331635in}}{\pgfqpoint{4.960000in}{3.696000in}}%
\pgfusepath{clip}%
\pgfsetbuttcap%
\pgfsetroundjoin%
\definecolor{currentfill}{rgb}{1.000000,0.705882,0.509804}%
\pgfsetfillcolor{currentfill}%
\pgfsetlinewidth{0.481800pt}%
\definecolor{currentstroke}{rgb}{1.000000,1.000000,1.000000}%
\pgfsetstrokecolor{currentstroke}%
\pgfsetdash{}{0pt}%
\pgfpathmoveto{\pgfqpoint{4.368323in}{1.042044in}}%
\pgfpathcurveto{\pgfqpoint{4.379373in}{1.042044in}}{\pgfqpoint{4.389972in}{1.046435in}}{\pgfqpoint{4.397786in}{1.054248in}}%
\pgfpathcurveto{\pgfqpoint{4.405600in}{1.062062in}}{\pgfqpoint{4.409990in}{1.072661in}}{\pgfqpoint{4.409990in}{1.083711in}}%
\pgfpathcurveto{\pgfqpoint{4.409990in}{1.094761in}}{\pgfqpoint{4.405600in}{1.105360in}}{\pgfqpoint{4.397786in}{1.113174in}}%
\pgfpathcurveto{\pgfqpoint{4.389972in}{1.120987in}}{\pgfqpoint{4.379373in}{1.125378in}}{\pgfqpoint{4.368323in}{1.125378in}}%
\pgfpathcurveto{\pgfqpoint{4.357273in}{1.125378in}}{\pgfqpoint{4.346674in}{1.120987in}}{\pgfqpoint{4.338861in}{1.113174in}}%
\pgfpathcurveto{\pgfqpoint{4.331047in}{1.105360in}}{\pgfqpoint{4.326657in}{1.094761in}}{\pgfqpoint{4.326657in}{1.083711in}}%
\pgfpathcurveto{\pgfqpoint{4.326657in}{1.072661in}}{\pgfqpoint{4.331047in}{1.062062in}}{\pgfqpoint{4.338861in}{1.054248in}}%
\pgfpathcurveto{\pgfqpoint{4.346674in}{1.046435in}}{\pgfqpoint{4.357273in}{1.042044in}}{\pgfqpoint{4.368323in}{1.042044in}}%
\pgfpathclose%
\pgfusepath{stroke,fill}%
\end{pgfscope}%
\begin{pgfscope}%
\pgfpathrectangle{\pgfqpoint{0.481978in}{0.331635in}}{\pgfqpoint{4.960000in}{3.696000in}}%
\pgfusepath{clip}%
\pgfsetbuttcap%
\pgfsetroundjoin%
\definecolor{currentfill}{rgb}{1.000000,0.705882,0.509804}%
\pgfsetfillcolor{currentfill}%
\pgfsetlinewidth{0.481800pt}%
\definecolor{currentstroke}{rgb}{1.000000,1.000000,1.000000}%
\pgfsetstrokecolor{currentstroke}%
\pgfsetdash{}{0pt}%
\pgfpathmoveto{\pgfqpoint{2.870476in}{0.457968in}}%
\pgfpathcurveto{\pgfqpoint{2.881526in}{0.457968in}}{\pgfqpoint{2.892126in}{0.462359in}}{\pgfqpoint{2.899939in}{0.470172in}}%
\pgfpathcurveto{\pgfqpoint{2.907753in}{0.477986in}}{\pgfqpoint{2.912143in}{0.488585in}}{\pgfqpoint{2.912143in}{0.499635in}}%
\pgfpathcurveto{\pgfqpoint{2.912143in}{0.510685in}}{\pgfqpoint{2.907753in}{0.521284in}}{\pgfqpoint{2.899939in}{0.529098in}}%
\pgfpathcurveto{\pgfqpoint{2.892126in}{0.536911in}}{\pgfqpoint{2.881526in}{0.541302in}}{\pgfqpoint{2.870476in}{0.541302in}}%
\pgfpathcurveto{\pgfqpoint{2.859426in}{0.541302in}}{\pgfqpoint{2.848827in}{0.536911in}}{\pgfqpoint{2.841014in}{0.529098in}}%
\pgfpathcurveto{\pgfqpoint{2.833200in}{0.521284in}}{\pgfqpoint{2.828810in}{0.510685in}}{\pgfqpoint{2.828810in}{0.499635in}}%
\pgfpathcurveto{\pgfqpoint{2.828810in}{0.488585in}}{\pgfqpoint{2.833200in}{0.477986in}}{\pgfqpoint{2.841014in}{0.470172in}}%
\pgfpathcurveto{\pgfqpoint{2.848827in}{0.462359in}}{\pgfqpoint{2.859426in}{0.457968in}}{\pgfqpoint{2.870476in}{0.457968in}}%
\pgfpathclose%
\pgfusepath{stroke,fill}%
\end{pgfscope}%
\begin{pgfscope}%
\pgfpathrectangle{\pgfqpoint{0.481978in}{0.331635in}}{\pgfqpoint{4.960000in}{3.696000in}}%
\pgfusepath{clip}%
\pgfsetbuttcap%
\pgfsetroundjoin%
\definecolor{currentfill}{rgb}{1.000000,0.705882,0.509804}%
\pgfsetfillcolor{currentfill}%
\pgfsetlinewidth{0.481800pt}%
\definecolor{currentstroke}{rgb}{1.000000,1.000000,1.000000}%
\pgfsetstrokecolor{currentstroke}%
\pgfsetdash{}{0pt}%
\pgfpathmoveto{\pgfqpoint{4.249938in}{2.898156in}}%
\pgfpathcurveto{\pgfqpoint{4.260988in}{2.898156in}}{\pgfqpoint{4.271587in}{2.902546in}}{\pgfqpoint{4.279401in}{2.910360in}}%
\pgfpathcurveto{\pgfqpoint{4.287215in}{2.918173in}}{\pgfqpoint{4.291605in}{2.928772in}}{\pgfqpoint{4.291605in}{2.939822in}}%
\pgfpathcurveto{\pgfqpoint{4.291605in}{2.950873in}}{\pgfqpoint{4.287215in}{2.961472in}}{\pgfqpoint{4.279401in}{2.969285in}}%
\pgfpathcurveto{\pgfqpoint{4.271587in}{2.977099in}}{\pgfqpoint{4.260988in}{2.981489in}}{\pgfqpoint{4.249938in}{2.981489in}}%
\pgfpathcurveto{\pgfqpoint{4.238888in}{2.981489in}}{\pgfqpoint{4.228289in}{2.977099in}}{\pgfqpoint{4.220475in}{2.969285in}}%
\pgfpathcurveto{\pgfqpoint{4.212662in}{2.961472in}}{\pgfqpoint{4.208272in}{2.950873in}}{\pgfqpoint{4.208272in}{2.939822in}}%
\pgfpathcurveto{\pgfqpoint{4.208272in}{2.928772in}}{\pgfqpoint{4.212662in}{2.918173in}}{\pgfqpoint{4.220475in}{2.910360in}}%
\pgfpathcurveto{\pgfqpoint{4.228289in}{2.902546in}}{\pgfqpoint{4.238888in}{2.898156in}}{\pgfqpoint{4.249938in}{2.898156in}}%
\pgfpathclose%
\pgfusepath{stroke,fill}%
\end{pgfscope}%
\begin{pgfscope}%
\pgfpathrectangle{\pgfqpoint{0.481978in}{0.331635in}}{\pgfqpoint{4.960000in}{3.696000in}}%
\pgfusepath{clip}%
\pgfsetbuttcap%
\pgfsetroundjoin%
\definecolor{currentfill}{rgb}{1.000000,0.705882,0.509804}%
\pgfsetfillcolor{currentfill}%
\pgfsetlinewidth{0.481800pt}%
\definecolor{currentstroke}{rgb}{1.000000,1.000000,1.000000}%
\pgfsetstrokecolor{currentstroke}%
\pgfsetdash{}{0pt}%
\pgfpathmoveto{\pgfqpoint{3.827281in}{1.272784in}}%
\pgfpathcurveto{\pgfqpoint{3.838331in}{1.272784in}}{\pgfqpoint{3.848931in}{1.277174in}}{\pgfqpoint{3.856744in}{1.284987in}}%
\pgfpathcurveto{\pgfqpoint{3.864558in}{1.292801in}}{\pgfqpoint{3.868948in}{1.303400in}}{\pgfqpoint{3.868948in}{1.314450in}}%
\pgfpathcurveto{\pgfqpoint{3.868948in}{1.325500in}}{\pgfqpoint{3.864558in}{1.336099in}}{\pgfqpoint{3.856744in}{1.343913in}}%
\pgfpathcurveto{\pgfqpoint{3.848931in}{1.351727in}}{\pgfqpoint{3.838331in}{1.356117in}}{\pgfqpoint{3.827281in}{1.356117in}}%
\pgfpathcurveto{\pgfqpoint{3.816231in}{1.356117in}}{\pgfqpoint{3.805632in}{1.351727in}}{\pgfqpoint{3.797819in}{1.343913in}}%
\pgfpathcurveto{\pgfqpoint{3.790005in}{1.336099in}}{\pgfqpoint{3.785615in}{1.325500in}}{\pgfqpoint{3.785615in}{1.314450in}}%
\pgfpathcurveto{\pgfqpoint{3.785615in}{1.303400in}}{\pgfqpoint{3.790005in}{1.292801in}}{\pgfqpoint{3.797819in}{1.284987in}}%
\pgfpathcurveto{\pgfqpoint{3.805632in}{1.277174in}}{\pgfqpoint{3.816231in}{1.272784in}}{\pgfqpoint{3.827281in}{1.272784in}}%
\pgfpathclose%
\pgfusepath{stroke,fill}%
\end{pgfscope}%
\begin{pgfscope}%
\pgfpathrectangle{\pgfqpoint{0.481978in}{0.331635in}}{\pgfqpoint{4.960000in}{3.696000in}}%
\pgfusepath{clip}%
\pgfsetbuttcap%
\pgfsetroundjoin%
\definecolor{currentfill}{rgb}{1.000000,0.705882,0.509804}%
\pgfsetfillcolor{currentfill}%
\pgfsetlinewidth{0.481800pt}%
\definecolor{currentstroke}{rgb}{1.000000,1.000000,1.000000}%
\pgfsetstrokecolor{currentstroke}%
\pgfsetdash{}{0pt}%
\pgfpathmoveto{\pgfqpoint{3.454050in}{0.900857in}}%
\pgfpathcurveto{\pgfqpoint{3.465101in}{0.900857in}}{\pgfqpoint{3.475700in}{0.905247in}}{\pgfqpoint{3.483513in}{0.913061in}}%
\pgfpathcurveto{\pgfqpoint{3.491327in}{0.920875in}}{\pgfqpoint{3.495717in}{0.931474in}}{\pgfqpoint{3.495717in}{0.942524in}}%
\pgfpathcurveto{\pgfqpoint{3.495717in}{0.953574in}}{\pgfqpoint{3.491327in}{0.964173in}}{\pgfqpoint{3.483513in}{0.971986in}}%
\pgfpathcurveto{\pgfqpoint{3.475700in}{0.979800in}}{\pgfqpoint{3.465101in}{0.984190in}}{\pgfqpoint{3.454050in}{0.984190in}}%
\pgfpathcurveto{\pgfqpoint{3.443000in}{0.984190in}}{\pgfqpoint{3.432401in}{0.979800in}}{\pgfqpoint{3.424588in}{0.971986in}}%
\pgfpathcurveto{\pgfqpoint{3.416774in}{0.964173in}}{\pgfqpoint{3.412384in}{0.953574in}}{\pgfqpoint{3.412384in}{0.942524in}}%
\pgfpathcurveto{\pgfqpoint{3.412384in}{0.931474in}}{\pgfqpoint{3.416774in}{0.920875in}}{\pgfqpoint{3.424588in}{0.913061in}}%
\pgfpathcurveto{\pgfqpoint{3.432401in}{0.905247in}}{\pgfqpoint{3.443000in}{0.900857in}}{\pgfqpoint{3.454050in}{0.900857in}}%
\pgfpathclose%
\pgfusepath{stroke,fill}%
\end{pgfscope}%
\begin{pgfscope}%
\pgfpathrectangle{\pgfqpoint{0.481978in}{0.331635in}}{\pgfqpoint{4.960000in}{3.696000in}}%
\pgfusepath{clip}%
\pgfsetbuttcap%
\pgfsetroundjoin%
\definecolor{currentfill}{rgb}{1.000000,0.705882,0.509804}%
\pgfsetfillcolor{currentfill}%
\pgfsetlinewidth{0.481800pt}%
\definecolor{currentstroke}{rgb}{1.000000,1.000000,1.000000}%
\pgfsetstrokecolor{currentstroke}%
\pgfsetdash{}{0pt}%
\pgfpathmoveto{\pgfqpoint{1.042275in}{2.395382in}}%
\pgfpathcurveto{\pgfqpoint{1.053326in}{2.395382in}}{\pgfqpoint{1.063925in}{2.399772in}}{\pgfqpoint{1.071738in}{2.407586in}}%
\pgfpathcurveto{\pgfqpoint{1.079552in}{2.415400in}}{\pgfqpoint{1.083942in}{2.425999in}}{\pgfqpoint{1.083942in}{2.437049in}}%
\pgfpathcurveto{\pgfqpoint{1.083942in}{2.448099in}}{\pgfqpoint{1.079552in}{2.458698in}}{\pgfqpoint{1.071738in}{2.466512in}}%
\pgfpathcurveto{\pgfqpoint{1.063925in}{2.474325in}}{\pgfqpoint{1.053326in}{2.478716in}}{\pgfqpoint{1.042275in}{2.478716in}}%
\pgfpathcurveto{\pgfqpoint{1.031225in}{2.478716in}}{\pgfqpoint{1.020626in}{2.474325in}}{\pgfqpoint{1.012813in}{2.466512in}}%
\pgfpathcurveto{\pgfqpoint{1.004999in}{2.458698in}}{\pgfqpoint{1.000609in}{2.448099in}}{\pgfqpoint{1.000609in}{2.437049in}}%
\pgfpathcurveto{\pgfqpoint{1.000609in}{2.425999in}}{\pgfqpoint{1.004999in}{2.415400in}}{\pgfqpoint{1.012813in}{2.407586in}}%
\pgfpathcurveto{\pgfqpoint{1.020626in}{2.399772in}}{\pgfqpoint{1.031225in}{2.395382in}}{\pgfqpoint{1.042275in}{2.395382in}}%
\pgfpathclose%
\pgfusepath{stroke,fill}%
\end{pgfscope}%
\begin{pgfscope}%
\pgfpathrectangle{\pgfqpoint{0.481978in}{0.331635in}}{\pgfqpoint{4.960000in}{3.696000in}}%
\pgfusepath{clip}%
\pgfsetbuttcap%
\pgfsetroundjoin%
\definecolor{currentfill}{rgb}{1.000000,0.705882,0.509804}%
\pgfsetfillcolor{currentfill}%
\pgfsetlinewidth{0.481800pt}%
\definecolor{currentstroke}{rgb}{1.000000,1.000000,1.000000}%
\pgfsetstrokecolor{currentstroke}%
\pgfsetdash{}{0pt}%
\pgfpathmoveto{\pgfqpoint{3.907740in}{3.191215in}}%
\pgfpathcurveto{\pgfqpoint{3.918790in}{3.191215in}}{\pgfqpoint{3.929389in}{3.195605in}}{\pgfqpoint{3.937203in}{3.203419in}}%
\pgfpathcurveto{\pgfqpoint{3.945016in}{3.211232in}}{\pgfqpoint{3.949407in}{3.221832in}}{\pgfqpoint{3.949407in}{3.232882in}}%
\pgfpathcurveto{\pgfqpoint{3.949407in}{3.243932in}}{\pgfqpoint{3.945016in}{3.254531in}}{\pgfqpoint{3.937203in}{3.262344in}}%
\pgfpathcurveto{\pgfqpoint{3.929389in}{3.270158in}}{\pgfqpoint{3.918790in}{3.274548in}}{\pgfqpoint{3.907740in}{3.274548in}}%
\pgfpathcurveto{\pgfqpoint{3.896690in}{3.274548in}}{\pgfqpoint{3.886091in}{3.270158in}}{\pgfqpoint{3.878277in}{3.262344in}}%
\pgfpathcurveto{\pgfqpoint{3.870464in}{3.254531in}}{\pgfqpoint{3.866073in}{3.243932in}}{\pgfqpoint{3.866073in}{3.232882in}}%
\pgfpathcurveto{\pgfqpoint{3.866073in}{3.221832in}}{\pgfqpoint{3.870464in}{3.211232in}}{\pgfqpoint{3.878277in}{3.203419in}}%
\pgfpathcurveto{\pgfqpoint{3.886091in}{3.195605in}}{\pgfqpoint{3.896690in}{3.191215in}}{\pgfqpoint{3.907740in}{3.191215in}}%
\pgfpathclose%
\pgfusepath{stroke,fill}%
\end{pgfscope}%
\begin{pgfscope}%
\pgfpathrectangle{\pgfqpoint{0.481978in}{0.331635in}}{\pgfqpoint{4.960000in}{3.696000in}}%
\pgfusepath{clip}%
\pgfsetbuttcap%
\pgfsetroundjoin%
\definecolor{currentfill}{rgb}{1.000000,0.705882,0.509804}%
\pgfsetfillcolor{currentfill}%
\pgfsetlinewidth{0.481800pt}%
\definecolor{currentstroke}{rgb}{1.000000,1.000000,1.000000}%
\pgfsetstrokecolor{currentstroke}%
\pgfsetdash{}{0pt}%
\pgfpathmoveto{\pgfqpoint{3.085391in}{0.971036in}}%
\pgfpathcurveto{\pgfqpoint{3.096441in}{0.971036in}}{\pgfqpoint{3.107040in}{0.975427in}}{\pgfqpoint{3.114854in}{0.983240in}}%
\pgfpathcurveto{\pgfqpoint{3.122667in}{0.991054in}}{\pgfqpoint{3.127057in}{1.001653in}}{\pgfqpoint{3.127057in}{1.012703in}}%
\pgfpathcurveto{\pgfqpoint{3.127057in}{1.023753in}}{\pgfqpoint{3.122667in}{1.034352in}}{\pgfqpoint{3.114854in}{1.042166in}}%
\pgfpathcurveto{\pgfqpoint{3.107040in}{1.049979in}}{\pgfqpoint{3.096441in}{1.054370in}}{\pgfqpoint{3.085391in}{1.054370in}}%
\pgfpathcurveto{\pgfqpoint{3.074341in}{1.054370in}}{\pgfqpoint{3.063742in}{1.049979in}}{\pgfqpoint{3.055928in}{1.042166in}}%
\pgfpathcurveto{\pgfqpoint{3.048114in}{1.034352in}}{\pgfqpoint{3.043724in}{1.023753in}}{\pgfqpoint{3.043724in}{1.012703in}}%
\pgfpathcurveto{\pgfqpoint{3.043724in}{1.001653in}}{\pgfqpoint{3.048114in}{0.991054in}}{\pgfqpoint{3.055928in}{0.983240in}}%
\pgfpathcurveto{\pgfqpoint{3.063742in}{0.975427in}}{\pgfqpoint{3.074341in}{0.971036in}}{\pgfqpoint{3.085391in}{0.971036in}}%
\pgfpathclose%
\pgfusepath{stroke,fill}%
\end{pgfscope}%
\begin{pgfscope}%
\pgfpathrectangle{\pgfqpoint{0.481978in}{0.331635in}}{\pgfqpoint{4.960000in}{3.696000in}}%
\pgfusepath{clip}%
\pgfsetbuttcap%
\pgfsetroundjoin%
\definecolor{currentfill}{rgb}{1.000000,0.705882,0.509804}%
\pgfsetfillcolor{currentfill}%
\pgfsetlinewidth{0.481800pt}%
\definecolor{currentstroke}{rgb}{1.000000,1.000000,1.000000}%
\pgfsetstrokecolor{currentstroke}%
\pgfsetdash{}{0pt}%
\pgfpathmoveto{\pgfqpoint{3.219219in}{0.714690in}}%
\pgfpathcurveto{\pgfqpoint{3.230270in}{0.714690in}}{\pgfqpoint{3.240869in}{0.719080in}}{\pgfqpoint{3.248682in}{0.726894in}}%
\pgfpathcurveto{\pgfqpoint{3.256496in}{0.734707in}}{\pgfqpoint{3.260886in}{0.745306in}}{\pgfqpoint{3.260886in}{0.756356in}}%
\pgfpathcurveto{\pgfqpoint{3.260886in}{0.767407in}}{\pgfqpoint{3.256496in}{0.778006in}}{\pgfqpoint{3.248682in}{0.785819in}}%
\pgfpathcurveto{\pgfqpoint{3.240869in}{0.793633in}}{\pgfqpoint{3.230270in}{0.798023in}}{\pgfqpoint{3.219219in}{0.798023in}}%
\pgfpathcurveto{\pgfqpoint{3.208169in}{0.798023in}}{\pgfqpoint{3.197570in}{0.793633in}}{\pgfqpoint{3.189757in}{0.785819in}}%
\pgfpathcurveto{\pgfqpoint{3.181943in}{0.778006in}}{\pgfqpoint{3.177553in}{0.767407in}}{\pgfqpoint{3.177553in}{0.756356in}}%
\pgfpathcurveto{\pgfqpoint{3.177553in}{0.745306in}}{\pgfqpoint{3.181943in}{0.734707in}}{\pgfqpoint{3.189757in}{0.726894in}}%
\pgfpathcurveto{\pgfqpoint{3.197570in}{0.719080in}}{\pgfqpoint{3.208169in}{0.714690in}}{\pgfqpoint{3.219219in}{0.714690in}}%
\pgfpathclose%
\pgfusepath{stroke,fill}%
\end{pgfscope}%
\begin{pgfscope}%
\pgfpathrectangle{\pgfqpoint{0.481978in}{0.331635in}}{\pgfqpoint{4.960000in}{3.696000in}}%
\pgfusepath{clip}%
\pgfsetbuttcap%
\pgfsetroundjoin%
\definecolor{currentfill}{rgb}{1.000000,0.705882,0.509804}%
\pgfsetfillcolor{currentfill}%
\pgfsetlinewidth{0.481800pt}%
\definecolor{currentstroke}{rgb}{1.000000,1.000000,1.000000}%
\pgfsetstrokecolor{currentstroke}%
\pgfsetdash{}{0pt}%
\pgfpathmoveto{\pgfqpoint{1.298048in}{3.074471in}}%
\pgfpathcurveto{\pgfqpoint{1.309098in}{3.074471in}}{\pgfqpoint{1.319697in}{3.078861in}}{\pgfqpoint{1.327511in}{3.086674in}}%
\pgfpathcurveto{\pgfqpoint{1.335324in}{3.094488in}}{\pgfqpoint{1.339715in}{3.105087in}}{\pgfqpoint{1.339715in}{3.116137in}}%
\pgfpathcurveto{\pgfqpoint{1.339715in}{3.127187in}}{\pgfqpoint{1.335324in}{3.137786in}}{\pgfqpoint{1.327511in}{3.145600in}}%
\pgfpathcurveto{\pgfqpoint{1.319697in}{3.153414in}}{\pgfqpoint{1.309098in}{3.157804in}}{\pgfqpoint{1.298048in}{3.157804in}}%
\pgfpathcurveto{\pgfqpoint{1.286998in}{3.157804in}}{\pgfqpoint{1.276399in}{3.153414in}}{\pgfqpoint{1.268585in}{3.145600in}}%
\pgfpathcurveto{\pgfqpoint{1.260771in}{3.137786in}}{\pgfqpoint{1.256381in}{3.127187in}}{\pgfqpoint{1.256381in}{3.116137in}}%
\pgfpathcurveto{\pgfqpoint{1.256381in}{3.105087in}}{\pgfqpoint{1.260771in}{3.094488in}}{\pgfqpoint{1.268585in}{3.086674in}}%
\pgfpathcurveto{\pgfqpoint{1.276399in}{3.078861in}}{\pgfqpoint{1.286998in}{3.074471in}}{\pgfqpoint{1.298048in}{3.074471in}}%
\pgfpathclose%
\pgfusepath{stroke,fill}%
\end{pgfscope}%
\begin{pgfscope}%
\pgfpathrectangle{\pgfqpoint{0.481978in}{0.331635in}}{\pgfqpoint{4.960000in}{3.696000in}}%
\pgfusepath{clip}%
\pgfsetbuttcap%
\pgfsetroundjoin%
\definecolor{currentfill}{rgb}{1.000000,0.705882,0.509804}%
\pgfsetfillcolor{currentfill}%
\pgfsetlinewidth{0.481800pt}%
\definecolor{currentstroke}{rgb}{1.000000,1.000000,1.000000}%
\pgfsetstrokecolor{currentstroke}%
\pgfsetdash{}{0pt}%
\pgfpathmoveto{\pgfqpoint{0.816826in}{2.015765in}}%
\pgfpathcurveto{\pgfqpoint{0.827876in}{2.015765in}}{\pgfqpoint{0.838475in}{2.020156in}}{\pgfqpoint{0.846289in}{2.027969in}}%
\pgfpathcurveto{\pgfqpoint{0.854103in}{2.035783in}}{\pgfqpoint{0.858493in}{2.046382in}}{\pgfqpoint{0.858493in}{2.057432in}}%
\pgfpathcurveto{\pgfqpoint{0.858493in}{2.068482in}}{\pgfqpoint{0.854103in}{2.079081in}}{\pgfqpoint{0.846289in}{2.086895in}}%
\pgfpathcurveto{\pgfqpoint{0.838475in}{2.094709in}}{\pgfqpoint{0.827876in}{2.099099in}}{\pgfqpoint{0.816826in}{2.099099in}}%
\pgfpathcurveto{\pgfqpoint{0.805776in}{2.099099in}}{\pgfqpoint{0.795177in}{2.094709in}}{\pgfqpoint{0.787363in}{2.086895in}}%
\pgfpathcurveto{\pgfqpoint{0.779550in}{2.079081in}}{\pgfqpoint{0.775160in}{2.068482in}}{\pgfqpoint{0.775160in}{2.057432in}}%
\pgfpathcurveto{\pgfqpoint{0.775160in}{2.046382in}}{\pgfqpoint{0.779550in}{2.035783in}}{\pgfqpoint{0.787363in}{2.027969in}}%
\pgfpathcurveto{\pgfqpoint{0.795177in}{2.020156in}}{\pgfqpoint{0.805776in}{2.015765in}}{\pgfqpoint{0.816826in}{2.015765in}}%
\pgfpathclose%
\pgfusepath{stroke,fill}%
\end{pgfscope}%
\begin{pgfscope}%
\pgfpathrectangle{\pgfqpoint{0.481978in}{0.331635in}}{\pgfqpoint{4.960000in}{3.696000in}}%
\pgfusepath{clip}%
\pgfsetbuttcap%
\pgfsetroundjoin%
\definecolor{currentfill}{rgb}{1.000000,0.705882,0.509804}%
\pgfsetfillcolor{currentfill}%
\pgfsetlinewidth{0.481800pt}%
\definecolor{currentstroke}{rgb}{1.000000,1.000000,1.000000}%
\pgfsetstrokecolor{currentstroke}%
\pgfsetdash{}{0pt}%
\pgfpathmoveto{\pgfqpoint{0.744844in}{2.548440in}}%
\pgfpathcurveto{\pgfqpoint{0.755894in}{2.548440in}}{\pgfqpoint{0.766493in}{2.552830in}}{\pgfqpoint{0.774307in}{2.560644in}}%
\pgfpathcurveto{\pgfqpoint{0.782120in}{2.568457in}}{\pgfqpoint{0.786511in}{2.579056in}}{\pgfqpoint{0.786511in}{2.590106in}}%
\pgfpathcurveto{\pgfqpoint{0.786511in}{2.601156in}}{\pgfqpoint{0.782120in}{2.611755in}}{\pgfqpoint{0.774307in}{2.619569in}}%
\pgfpathcurveto{\pgfqpoint{0.766493in}{2.627383in}}{\pgfqpoint{0.755894in}{2.631773in}}{\pgfqpoint{0.744844in}{2.631773in}}%
\pgfpathcurveto{\pgfqpoint{0.733794in}{2.631773in}}{\pgfqpoint{0.723195in}{2.627383in}}{\pgfqpoint{0.715381in}{2.619569in}}%
\pgfpathcurveto{\pgfqpoint{0.707567in}{2.611755in}}{\pgfqpoint{0.703177in}{2.601156in}}{\pgfqpoint{0.703177in}{2.590106in}}%
\pgfpathcurveto{\pgfqpoint{0.703177in}{2.579056in}}{\pgfqpoint{0.707567in}{2.568457in}}{\pgfqpoint{0.715381in}{2.560644in}}%
\pgfpathcurveto{\pgfqpoint{0.723195in}{2.552830in}}{\pgfqpoint{0.733794in}{2.548440in}}{\pgfqpoint{0.744844in}{2.548440in}}%
\pgfpathclose%
\pgfusepath{stroke,fill}%
\end{pgfscope}%
\begin{pgfscope}%
\pgfpathrectangle{\pgfqpoint{0.481978in}{0.331635in}}{\pgfqpoint{4.960000in}{3.696000in}}%
\pgfusepath{clip}%
\pgfsetbuttcap%
\pgfsetroundjoin%
\definecolor{currentfill}{rgb}{1.000000,0.705882,0.509804}%
\pgfsetfillcolor{currentfill}%
\pgfsetlinewidth{0.481800pt}%
\definecolor{currentstroke}{rgb}{1.000000,1.000000,1.000000}%
\pgfsetstrokecolor{currentstroke}%
\pgfsetdash{}{0pt}%
\pgfpathmoveto{\pgfqpoint{2.676035in}{3.398274in}}%
\pgfpathcurveto{\pgfqpoint{2.687085in}{3.398274in}}{\pgfqpoint{2.697684in}{3.402664in}}{\pgfqpoint{2.705498in}{3.410478in}}%
\pgfpathcurveto{\pgfqpoint{2.713311in}{3.418291in}}{\pgfqpoint{2.717701in}{3.428891in}}{\pgfqpoint{2.717701in}{3.439941in}}%
\pgfpathcurveto{\pgfqpoint{2.717701in}{3.450991in}}{\pgfqpoint{2.713311in}{3.461590in}}{\pgfqpoint{2.705498in}{3.469403in}}%
\pgfpathcurveto{\pgfqpoint{2.697684in}{3.477217in}}{\pgfqpoint{2.687085in}{3.481607in}}{\pgfqpoint{2.676035in}{3.481607in}}%
\pgfpathcurveto{\pgfqpoint{2.664985in}{3.481607in}}{\pgfqpoint{2.654386in}{3.477217in}}{\pgfqpoint{2.646572in}{3.469403in}}%
\pgfpathcurveto{\pgfqpoint{2.638758in}{3.461590in}}{\pgfqpoint{2.634368in}{3.450991in}}{\pgfqpoint{2.634368in}{3.439941in}}%
\pgfpathcurveto{\pgfqpoint{2.634368in}{3.428891in}}{\pgfqpoint{2.638758in}{3.418291in}}{\pgfqpoint{2.646572in}{3.410478in}}%
\pgfpathcurveto{\pgfqpoint{2.654386in}{3.402664in}}{\pgfqpoint{2.664985in}{3.398274in}}{\pgfqpoint{2.676035in}{3.398274in}}%
\pgfpathclose%
\pgfusepath{stroke,fill}%
\end{pgfscope}%
\begin{pgfscope}%
\pgfpathrectangle{\pgfqpoint{0.481978in}{0.331635in}}{\pgfqpoint{4.960000in}{3.696000in}}%
\pgfusepath{clip}%
\pgfsetbuttcap%
\pgfsetroundjoin%
\definecolor{currentfill}{rgb}{1.000000,0.705882,0.509804}%
\pgfsetfillcolor{currentfill}%
\pgfsetlinewidth{0.481800pt}%
\definecolor{currentstroke}{rgb}{1.000000,1.000000,1.000000}%
\pgfsetstrokecolor{currentstroke}%
\pgfsetdash{}{0pt}%
\pgfpathmoveto{\pgfqpoint{3.635940in}{3.256584in}}%
\pgfpathcurveto{\pgfqpoint{3.646990in}{3.256584in}}{\pgfqpoint{3.657589in}{3.260974in}}{\pgfqpoint{3.665403in}{3.268788in}}%
\pgfpathcurveto{\pgfqpoint{3.673217in}{3.276602in}}{\pgfqpoint{3.677607in}{3.287201in}}{\pgfqpoint{3.677607in}{3.298251in}}%
\pgfpathcurveto{\pgfqpoint{3.677607in}{3.309301in}}{\pgfqpoint{3.673217in}{3.319900in}}{\pgfqpoint{3.665403in}{3.327713in}}%
\pgfpathcurveto{\pgfqpoint{3.657589in}{3.335527in}}{\pgfqpoint{3.646990in}{3.339917in}}{\pgfqpoint{3.635940in}{3.339917in}}%
\pgfpathcurveto{\pgfqpoint{3.624890in}{3.339917in}}{\pgfqpoint{3.614291in}{3.335527in}}{\pgfqpoint{3.606477in}{3.327713in}}%
\pgfpathcurveto{\pgfqpoint{3.598664in}{3.319900in}}{\pgfqpoint{3.594274in}{3.309301in}}{\pgfqpoint{3.594274in}{3.298251in}}%
\pgfpathcurveto{\pgfqpoint{3.594274in}{3.287201in}}{\pgfqpoint{3.598664in}{3.276602in}}{\pgfqpoint{3.606477in}{3.268788in}}%
\pgfpathcurveto{\pgfqpoint{3.614291in}{3.260974in}}{\pgfqpoint{3.624890in}{3.256584in}}{\pgfqpoint{3.635940in}{3.256584in}}%
\pgfpathclose%
\pgfusepath{stroke,fill}%
\end{pgfscope}%
\begin{pgfscope}%
\pgfpathrectangle{\pgfqpoint{0.481978in}{0.331635in}}{\pgfqpoint{4.960000in}{3.696000in}}%
\pgfusepath{clip}%
\pgfsetbuttcap%
\pgfsetroundjoin%
\definecolor{currentfill}{rgb}{1.000000,0.705882,0.509804}%
\pgfsetfillcolor{currentfill}%
\pgfsetlinewidth{0.481800pt}%
\definecolor{currentstroke}{rgb}{1.000000,1.000000,1.000000}%
\pgfsetstrokecolor{currentstroke}%
\pgfsetdash{}{0pt}%
\pgfpathmoveto{\pgfqpoint{2.951945in}{2.069516in}}%
\pgfpathcurveto{\pgfqpoint{2.962995in}{2.069516in}}{\pgfqpoint{2.973594in}{2.073907in}}{\pgfqpoint{2.981408in}{2.081720in}}%
\pgfpathcurveto{\pgfqpoint{2.989222in}{2.089534in}}{\pgfqpoint{2.993612in}{2.100133in}}{\pgfqpoint{2.993612in}{2.111183in}}%
\pgfpathcurveto{\pgfqpoint{2.993612in}{2.122233in}}{\pgfqpoint{2.989222in}{2.132832in}}{\pgfqpoint{2.981408in}{2.140646in}}%
\pgfpathcurveto{\pgfqpoint{2.973594in}{2.148459in}}{\pgfqpoint{2.962995in}{2.152850in}}{\pgfqpoint{2.951945in}{2.152850in}}%
\pgfpathcurveto{\pgfqpoint{2.940895in}{2.152850in}}{\pgfqpoint{2.930296in}{2.148459in}}{\pgfqpoint{2.922482in}{2.140646in}}%
\pgfpathcurveto{\pgfqpoint{2.914669in}{2.132832in}}{\pgfqpoint{2.910279in}{2.122233in}}{\pgfqpoint{2.910279in}{2.111183in}}%
\pgfpathcurveto{\pgfqpoint{2.910279in}{2.100133in}}{\pgfqpoint{2.914669in}{2.089534in}}{\pgfqpoint{2.922482in}{2.081720in}}%
\pgfpathcurveto{\pgfqpoint{2.930296in}{2.073907in}}{\pgfqpoint{2.940895in}{2.069516in}}{\pgfqpoint{2.951945in}{2.069516in}}%
\pgfpathclose%
\pgfusepath{stroke,fill}%
\end{pgfscope}%
\begin{pgfscope}%
\pgfpathrectangle{\pgfqpoint{0.481978in}{0.331635in}}{\pgfqpoint{4.960000in}{3.696000in}}%
\pgfusepath{clip}%
\pgfsetbuttcap%
\pgfsetroundjoin%
\definecolor{currentfill}{rgb}{1.000000,0.705882,0.509804}%
\pgfsetfillcolor{currentfill}%
\pgfsetlinewidth{0.481800pt}%
\definecolor{currentstroke}{rgb}{1.000000,1.000000,1.000000}%
\pgfsetstrokecolor{currentstroke}%
\pgfsetdash{}{0pt}%
\pgfpathmoveto{\pgfqpoint{2.897572in}{1.311117in}}%
\pgfpathcurveto{\pgfqpoint{2.908622in}{1.311117in}}{\pgfqpoint{2.919221in}{1.315508in}}{\pgfqpoint{2.927034in}{1.323321in}}%
\pgfpathcurveto{\pgfqpoint{2.934848in}{1.331135in}}{\pgfqpoint{2.939238in}{1.341734in}}{\pgfqpoint{2.939238in}{1.352784in}}%
\pgfpathcurveto{\pgfqpoint{2.939238in}{1.363834in}}{\pgfqpoint{2.934848in}{1.374433in}}{\pgfqpoint{2.927034in}{1.382247in}}%
\pgfpathcurveto{\pgfqpoint{2.919221in}{1.390060in}}{\pgfqpoint{2.908622in}{1.394451in}}{\pgfqpoint{2.897572in}{1.394451in}}%
\pgfpathcurveto{\pgfqpoint{2.886522in}{1.394451in}}{\pgfqpoint{2.875922in}{1.390060in}}{\pgfqpoint{2.868109in}{1.382247in}}%
\pgfpathcurveto{\pgfqpoint{2.860295in}{1.374433in}}{\pgfqpoint{2.855905in}{1.363834in}}{\pgfqpoint{2.855905in}{1.352784in}}%
\pgfpathcurveto{\pgfqpoint{2.855905in}{1.341734in}}{\pgfqpoint{2.860295in}{1.331135in}}{\pgfqpoint{2.868109in}{1.323321in}}%
\pgfpathcurveto{\pgfqpoint{2.875922in}{1.315508in}}{\pgfqpoint{2.886522in}{1.311117in}}{\pgfqpoint{2.897572in}{1.311117in}}%
\pgfpathclose%
\pgfusepath{stroke,fill}%
\end{pgfscope}%
\begin{pgfscope}%
\pgfpathrectangle{\pgfqpoint{0.481978in}{0.331635in}}{\pgfqpoint{4.960000in}{3.696000in}}%
\pgfusepath{clip}%
\pgfsetbuttcap%
\pgfsetroundjoin%
\definecolor{currentfill}{rgb}{1.000000,0.705882,0.509804}%
\pgfsetfillcolor{currentfill}%
\pgfsetlinewidth{0.481800pt}%
\definecolor{currentstroke}{rgb}{1.000000,1.000000,1.000000}%
\pgfsetstrokecolor{currentstroke}%
\pgfsetdash{}{0pt}%
\pgfpathmoveto{\pgfqpoint{4.132208in}{3.087990in}}%
\pgfpathcurveto{\pgfqpoint{4.143258in}{3.087990in}}{\pgfqpoint{4.153857in}{3.092380in}}{\pgfqpoint{4.161671in}{3.100194in}}%
\pgfpathcurveto{\pgfqpoint{4.169485in}{3.108007in}}{\pgfqpoint{4.173875in}{3.118606in}}{\pgfqpoint{4.173875in}{3.129656in}}%
\pgfpathcurveto{\pgfqpoint{4.173875in}{3.140707in}}{\pgfqpoint{4.169485in}{3.151306in}}{\pgfqpoint{4.161671in}{3.159119in}}%
\pgfpathcurveto{\pgfqpoint{4.153857in}{3.166933in}}{\pgfqpoint{4.143258in}{3.171323in}}{\pgfqpoint{4.132208in}{3.171323in}}%
\pgfpathcurveto{\pgfqpoint{4.121158in}{3.171323in}}{\pgfqpoint{4.110559in}{3.166933in}}{\pgfqpoint{4.102746in}{3.159119in}}%
\pgfpathcurveto{\pgfqpoint{4.094932in}{3.151306in}}{\pgfqpoint{4.090542in}{3.140707in}}{\pgfqpoint{4.090542in}{3.129656in}}%
\pgfpathcurveto{\pgfqpoint{4.090542in}{3.118606in}}{\pgfqpoint{4.094932in}{3.108007in}}{\pgfqpoint{4.102746in}{3.100194in}}%
\pgfpathcurveto{\pgfqpoint{4.110559in}{3.092380in}}{\pgfqpoint{4.121158in}{3.087990in}}{\pgfqpoint{4.132208in}{3.087990in}}%
\pgfpathclose%
\pgfusepath{stroke,fill}%
\end{pgfscope}%
\begin{pgfscope}%
\pgfpathrectangle{\pgfqpoint{0.481978in}{0.331635in}}{\pgfqpoint{4.960000in}{3.696000in}}%
\pgfusepath{clip}%
\pgfsetbuttcap%
\pgfsetroundjoin%
\definecolor{currentfill}{rgb}{1.000000,0.705882,0.509804}%
\pgfsetfillcolor{currentfill}%
\pgfsetlinewidth{0.481800pt}%
\definecolor{currentstroke}{rgb}{1.000000,1.000000,1.000000}%
\pgfsetstrokecolor{currentstroke}%
\pgfsetdash{}{0pt}%
\pgfpathmoveto{\pgfqpoint{3.639008in}{2.579169in}}%
\pgfpathcurveto{\pgfqpoint{3.650058in}{2.579169in}}{\pgfqpoint{3.660657in}{2.583559in}}{\pgfqpoint{3.668471in}{2.591373in}}%
\pgfpathcurveto{\pgfqpoint{3.676284in}{2.599186in}}{\pgfqpoint{3.680675in}{2.609785in}}{\pgfqpoint{3.680675in}{2.620835in}}%
\pgfpathcurveto{\pgfqpoint{3.680675in}{2.631886in}}{\pgfqpoint{3.676284in}{2.642485in}}{\pgfqpoint{3.668471in}{2.650298in}}%
\pgfpathcurveto{\pgfqpoint{3.660657in}{2.658112in}}{\pgfqpoint{3.650058in}{2.662502in}}{\pgfqpoint{3.639008in}{2.662502in}}%
\pgfpathcurveto{\pgfqpoint{3.627958in}{2.662502in}}{\pgfqpoint{3.617359in}{2.658112in}}{\pgfqpoint{3.609545in}{2.650298in}}%
\pgfpathcurveto{\pgfqpoint{3.601732in}{2.642485in}}{\pgfqpoint{3.597341in}{2.631886in}}{\pgfqpoint{3.597341in}{2.620835in}}%
\pgfpathcurveto{\pgfqpoint{3.597341in}{2.609785in}}{\pgfqpoint{3.601732in}{2.599186in}}{\pgfqpoint{3.609545in}{2.591373in}}%
\pgfpathcurveto{\pgfqpoint{3.617359in}{2.583559in}}{\pgfqpoint{3.627958in}{2.579169in}}{\pgfqpoint{3.639008in}{2.579169in}}%
\pgfpathclose%
\pgfusepath{stroke,fill}%
\end{pgfscope}%
\begin{pgfscope}%
\pgfpathrectangle{\pgfqpoint{0.481978in}{0.331635in}}{\pgfqpoint{4.960000in}{3.696000in}}%
\pgfusepath{clip}%
\pgfsetbuttcap%
\pgfsetroundjoin%
\definecolor{currentfill}{rgb}{1.000000,0.705882,0.509804}%
\pgfsetfillcolor{currentfill}%
\pgfsetlinewidth{0.481800pt}%
\definecolor{currentstroke}{rgb}{1.000000,1.000000,1.000000}%
\pgfsetstrokecolor{currentstroke}%
\pgfsetdash{}{0pt}%
\pgfpathmoveto{\pgfqpoint{2.882105in}{1.893452in}}%
\pgfpathcurveto{\pgfqpoint{2.893155in}{1.893452in}}{\pgfqpoint{2.903754in}{1.897842in}}{\pgfqpoint{2.911568in}{1.905656in}}%
\pgfpathcurveto{\pgfqpoint{2.919381in}{1.913469in}}{\pgfqpoint{2.923771in}{1.924068in}}{\pgfqpoint{2.923771in}{1.935118in}}%
\pgfpathcurveto{\pgfqpoint{2.923771in}{1.946168in}}{\pgfqpoint{2.919381in}{1.956767in}}{\pgfqpoint{2.911568in}{1.964581in}}%
\pgfpathcurveto{\pgfqpoint{2.903754in}{1.972395in}}{\pgfqpoint{2.893155in}{1.976785in}}{\pgfqpoint{2.882105in}{1.976785in}}%
\pgfpathcurveto{\pgfqpoint{2.871055in}{1.976785in}}{\pgfqpoint{2.860456in}{1.972395in}}{\pgfqpoint{2.852642in}{1.964581in}}%
\pgfpathcurveto{\pgfqpoint{2.844828in}{1.956767in}}{\pgfqpoint{2.840438in}{1.946168in}}{\pgfqpoint{2.840438in}{1.935118in}}%
\pgfpathcurveto{\pgfqpoint{2.840438in}{1.924068in}}{\pgfqpoint{2.844828in}{1.913469in}}{\pgfqpoint{2.852642in}{1.905656in}}%
\pgfpathcurveto{\pgfqpoint{2.860456in}{1.897842in}}{\pgfqpoint{2.871055in}{1.893452in}}{\pgfqpoint{2.882105in}{1.893452in}}%
\pgfpathclose%
\pgfusepath{stroke,fill}%
\end{pgfscope}%
\begin{pgfscope}%
\pgfpathrectangle{\pgfqpoint{0.481978in}{0.331635in}}{\pgfqpoint{4.960000in}{3.696000in}}%
\pgfusepath{clip}%
\pgfsetbuttcap%
\pgfsetroundjoin%
\definecolor{currentfill}{rgb}{1.000000,0.705882,0.509804}%
\pgfsetfillcolor{currentfill}%
\pgfsetlinewidth{0.481800pt}%
\definecolor{currentstroke}{rgb}{1.000000,1.000000,1.000000}%
\pgfsetstrokecolor{currentstroke}%
\pgfsetdash{}{0pt}%
\pgfpathmoveto{\pgfqpoint{3.145500in}{1.505261in}}%
\pgfpathcurveto{\pgfqpoint{3.156550in}{1.505261in}}{\pgfqpoint{3.167149in}{1.509651in}}{\pgfqpoint{3.174963in}{1.517465in}}%
\pgfpathcurveto{\pgfqpoint{3.182776in}{1.525279in}}{\pgfqpoint{3.187167in}{1.535878in}}{\pgfqpoint{3.187167in}{1.546928in}}%
\pgfpathcurveto{\pgfqpoint{3.187167in}{1.557978in}}{\pgfqpoint{3.182776in}{1.568577in}}{\pgfqpoint{3.174963in}{1.576391in}}%
\pgfpathcurveto{\pgfqpoint{3.167149in}{1.584204in}}{\pgfqpoint{3.156550in}{1.588594in}}{\pgfqpoint{3.145500in}{1.588594in}}%
\pgfpathcurveto{\pgfqpoint{3.134450in}{1.588594in}}{\pgfqpoint{3.123851in}{1.584204in}}{\pgfqpoint{3.116037in}{1.576391in}}%
\pgfpathcurveto{\pgfqpoint{3.108224in}{1.568577in}}{\pgfqpoint{3.103833in}{1.557978in}}{\pgfqpoint{3.103833in}{1.546928in}}%
\pgfpathcurveto{\pgfqpoint{3.103833in}{1.535878in}}{\pgfqpoint{3.108224in}{1.525279in}}{\pgfqpoint{3.116037in}{1.517465in}}%
\pgfpathcurveto{\pgfqpoint{3.123851in}{1.509651in}}{\pgfqpoint{3.134450in}{1.505261in}}{\pgfqpoint{3.145500in}{1.505261in}}%
\pgfpathclose%
\pgfusepath{stroke,fill}%
\end{pgfscope}%
\begin{pgfscope}%
\pgfpathrectangle{\pgfqpoint{0.481978in}{0.331635in}}{\pgfqpoint{4.960000in}{3.696000in}}%
\pgfusepath{clip}%
\pgfsetbuttcap%
\pgfsetroundjoin%
\definecolor{currentfill}{rgb}{1.000000,0.705882,0.509804}%
\pgfsetfillcolor{currentfill}%
\pgfsetlinewidth{0.481800pt}%
\definecolor{currentstroke}{rgb}{1.000000,1.000000,1.000000}%
\pgfsetstrokecolor{currentstroke}%
\pgfsetdash{}{0pt}%
\pgfpathmoveto{\pgfqpoint{2.747214in}{2.671724in}}%
\pgfpathcurveto{\pgfqpoint{2.758264in}{2.671724in}}{\pgfqpoint{2.768863in}{2.676114in}}{\pgfqpoint{2.776676in}{2.683928in}}%
\pgfpathcurveto{\pgfqpoint{2.784490in}{2.691742in}}{\pgfqpoint{2.788880in}{2.702341in}}{\pgfqpoint{2.788880in}{2.713391in}}%
\pgfpathcurveto{\pgfqpoint{2.788880in}{2.724441in}}{\pgfqpoint{2.784490in}{2.735040in}}{\pgfqpoint{2.776676in}{2.742854in}}%
\pgfpathcurveto{\pgfqpoint{2.768863in}{2.750667in}}{\pgfqpoint{2.758264in}{2.755058in}}{\pgfqpoint{2.747214in}{2.755058in}}%
\pgfpathcurveto{\pgfqpoint{2.736164in}{2.755058in}}{\pgfqpoint{2.725564in}{2.750667in}}{\pgfqpoint{2.717751in}{2.742854in}}%
\pgfpathcurveto{\pgfqpoint{2.709937in}{2.735040in}}{\pgfqpoint{2.705547in}{2.724441in}}{\pgfqpoint{2.705547in}{2.713391in}}%
\pgfpathcurveto{\pgfqpoint{2.705547in}{2.702341in}}{\pgfqpoint{2.709937in}{2.691742in}}{\pgfqpoint{2.717751in}{2.683928in}}%
\pgfpathcurveto{\pgfqpoint{2.725564in}{2.676114in}}{\pgfqpoint{2.736164in}{2.671724in}}{\pgfqpoint{2.747214in}{2.671724in}}%
\pgfpathclose%
\pgfusepath{stroke,fill}%
\end{pgfscope}%
\begin{pgfscope}%
\pgfpathrectangle{\pgfqpoint{0.481978in}{0.331635in}}{\pgfqpoint{4.960000in}{3.696000in}}%
\pgfusepath{clip}%
\pgfsetbuttcap%
\pgfsetroundjoin%
\definecolor{currentfill}{rgb}{1.000000,0.705882,0.509804}%
\pgfsetfillcolor{currentfill}%
\pgfsetlinewidth{0.481800pt}%
\definecolor{currentstroke}{rgb}{1.000000,1.000000,1.000000}%
\pgfsetstrokecolor{currentstroke}%
\pgfsetdash{}{0pt}%
\pgfpathmoveto{\pgfqpoint{4.181119in}{2.707213in}}%
\pgfpathcurveto{\pgfqpoint{4.192169in}{2.707213in}}{\pgfqpoint{4.202768in}{2.711603in}}{\pgfqpoint{4.210581in}{2.719417in}}%
\pgfpathcurveto{\pgfqpoint{4.218395in}{2.727230in}}{\pgfqpoint{4.222785in}{2.737829in}}{\pgfqpoint{4.222785in}{2.748879in}}%
\pgfpathcurveto{\pgfqpoint{4.222785in}{2.759929in}}{\pgfqpoint{4.218395in}{2.770528in}}{\pgfqpoint{4.210581in}{2.778342in}}%
\pgfpathcurveto{\pgfqpoint{4.202768in}{2.786156in}}{\pgfqpoint{4.192169in}{2.790546in}}{\pgfqpoint{4.181119in}{2.790546in}}%
\pgfpathcurveto{\pgfqpoint{4.170068in}{2.790546in}}{\pgfqpoint{4.159469in}{2.786156in}}{\pgfqpoint{4.151656in}{2.778342in}}%
\pgfpathcurveto{\pgfqpoint{4.143842in}{2.770528in}}{\pgfqpoint{4.139452in}{2.759929in}}{\pgfqpoint{4.139452in}{2.748879in}}%
\pgfpathcurveto{\pgfqpoint{4.139452in}{2.737829in}}{\pgfqpoint{4.143842in}{2.727230in}}{\pgfqpoint{4.151656in}{2.719417in}}%
\pgfpathcurveto{\pgfqpoint{4.159469in}{2.711603in}}{\pgfqpoint{4.170068in}{2.707213in}}{\pgfqpoint{4.181119in}{2.707213in}}%
\pgfpathclose%
\pgfusepath{stroke,fill}%
\end{pgfscope}%
\begin{pgfscope}%
\pgfpathrectangle{\pgfqpoint{0.481978in}{0.331635in}}{\pgfqpoint{4.960000in}{3.696000in}}%
\pgfusepath{clip}%
\pgfsetbuttcap%
\pgfsetroundjoin%
\definecolor{currentfill}{rgb}{0.631373,0.788235,0.956863}%
\pgfsetfillcolor{currentfill}%
\pgfsetlinewidth{0.481800pt}%
\definecolor{currentstroke}{rgb}{1.000000,1.000000,1.000000}%
\pgfsetstrokecolor{currentstroke}%
\pgfsetdash{}{0pt}%
\pgfpathmoveto{\pgfqpoint{1.411682in}{2.584264in}}%
\pgfpathcurveto{\pgfqpoint{1.422732in}{2.584264in}}{\pgfqpoint{1.433331in}{2.588654in}}{\pgfqpoint{1.441145in}{2.596467in}}%
\pgfpathcurveto{\pgfqpoint{1.448959in}{2.604281in}}{\pgfqpoint{1.453349in}{2.614880in}}{\pgfqpoint{1.453349in}{2.625930in}}%
\pgfpathcurveto{\pgfqpoint{1.453349in}{2.636980in}}{\pgfqpoint{1.448959in}{2.647579in}}{\pgfqpoint{1.441145in}{2.655393in}}%
\pgfpathcurveto{\pgfqpoint{1.433331in}{2.663207in}}{\pgfqpoint{1.422732in}{2.667597in}}{\pgfqpoint{1.411682in}{2.667597in}}%
\pgfpathcurveto{\pgfqpoint{1.400632in}{2.667597in}}{\pgfqpoint{1.390033in}{2.663207in}}{\pgfqpoint{1.382219in}{2.655393in}}%
\pgfpathcurveto{\pgfqpoint{1.374406in}{2.647579in}}{\pgfqpoint{1.370016in}{2.636980in}}{\pgfqpoint{1.370016in}{2.625930in}}%
\pgfpathcurveto{\pgfqpoint{1.370016in}{2.614880in}}{\pgfqpoint{1.374406in}{2.604281in}}{\pgfqpoint{1.382219in}{2.596467in}}%
\pgfpathcurveto{\pgfqpoint{1.390033in}{2.588654in}}{\pgfqpoint{1.400632in}{2.584264in}}{\pgfqpoint{1.411682in}{2.584264in}}%
\pgfpathclose%
\pgfusepath{stroke,fill}%
\end{pgfscope}%
\begin{pgfscope}%
\pgfpathrectangle{\pgfqpoint{0.481978in}{0.331635in}}{\pgfqpoint{4.960000in}{3.696000in}}%
\pgfusepath{clip}%
\pgfsetbuttcap%
\pgfsetroundjoin%
\definecolor{currentfill}{rgb}{0.631373,0.788235,0.956863}%
\pgfsetfillcolor{currentfill}%
\pgfsetlinewidth{0.481800pt}%
\definecolor{currentstroke}{rgb}{1.000000,1.000000,1.000000}%
\pgfsetstrokecolor{currentstroke}%
\pgfsetdash{}{0pt}%
\pgfpathmoveto{\pgfqpoint{1.733103in}{1.545063in}}%
\pgfpathcurveto{\pgfqpoint{1.744153in}{1.545063in}}{\pgfqpoint{1.754752in}{1.549453in}}{\pgfqpoint{1.762566in}{1.557266in}}%
\pgfpathcurveto{\pgfqpoint{1.770380in}{1.565080in}}{\pgfqpoint{1.774770in}{1.575679in}}{\pgfqpoint{1.774770in}{1.586729in}}%
\pgfpathcurveto{\pgfqpoint{1.774770in}{1.597779in}}{\pgfqpoint{1.770380in}{1.608378in}}{\pgfqpoint{1.762566in}{1.616192in}}%
\pgfpathcurveto{\pgfqpoint{1.754752in}{1.624006in}}{\pgfqpoint{1.744153in}{1.628396in}}{\pgfqpoint{1.733103in}{1.628396in}}%
\pgfpathcurveto{\pgfqpoint{1.722053in}{1.628396in}}{\pgfqpoint{1.711454in}{1.624006in}}{\pgfqpoint{1.703640in}{1.616192in}}%
\pgfpathcurveto{\pgfqpoint{1.695827in}{1.608378in}}{\pgfqpoint{1.691437in}{1.597779in}}{\pgfqpoint{1.691437in}{1.586729in}}%
\pgfpathcurveto{\pgfqpoint{1.691437in}{1.575679in}}{\pgfqpoint{1.695827in}{1.565080in}}{\pgfqpoint{1.703640in}{1.557266in}}%
\pgfpathcurveto{\pgfqpoint{1.711454in}{1.549453in}}{\pgfqpoint{1.722053in}{1.545063in}}{\pgfqpoint{1.733103in}{1.545063in}}%
\pgfpathclose%
\pgfusepath{stroke,fill}%
\end{pgfscope}%
\begin{pgfscope}%
\pgfpathrectangle{\pgfqpoint{0.481978in}{0.331635in}}{\pgfqpoint{4.960000in}{3.696000in}}%
\pgfusepath{clip}%
\pgfsetbuttcap%
\pgfsetroundjoin%
\definecolor{currentfill}{rgb}{0.631373,0.788235,0.956863}%
\pgfsetfillcolor{currentfill}%
\pgfsetlinewidth{0.481800pt}%
\definecolor{currentstroke}{rgb}{1.000000,1.000000,1.000000}%
\pgfsetstrokecolor{currentstroke}%
\pgfsetdash{}{0pt}%
\pgfpathmoveto{\pgfqpoint{1.627229in}{1.944004in}}%
\pgfpathcurveto{\pgfqpoint{1.638279in}{1.944004in}}{\pgfqpoint{1.648878in}{1.948394in}}{\pgfqpoint{1.656692in}{1.956208in}}%
\pgfpathcurveto{\pgfqpoint{1.664505in}{1.964021in}}{\pgfqpoint{1.668896in}{1.974620in}}{\pgfqpoint{1.668896in}{1.985670in}}%
\pgfpathcurveto{\pgfqpoint{1.668896in}{1.996721in}}{\pgfqpoint{1.664505in}{2.007320in}}{\pgfqpoint{1.656692in}{2.015133in}}%
\pgfpathcurveto{\pgfqpoint{1.648878in}{2.022947in}}{\pgfqpoint{1.638279in}{2.027337in}}{\pgfqpoint{1.627229in}{2.027337in}}%
\pgfpathcurveto{\pgfqpoint{1.616179in}{2.027337in}}{\pgfqpoint{1.605580in}{2.022947in}}{\pgfqpoint{1.597766in}{2.015133in}}%
\pgfpathcurveto{\pgfqpoint{1.589953in}{2.007320in}}{\pgfqpoint{1.585562in}{1.996721in}}{\pgfqpoint{1.585562in}{1.985670in}}%
\pgfpathcurveto{\pgfqpoint{1.585562in}{1.974620in}}{\pgfqpoint{1.589953in}{1.964021in}}{\pgfqpoint{1.597766in}{1.956208in}}%
\pgfpathcurveto{\pgfqpoint{1.605580in}{1.948394in}}{\pgfqpoint{1.616179in}{1.944004in}}{\pgfqpoint{1.627229in}{1.944004in}}%
\pgfpathclose%
\pgfusepath{stroke,fill}%
\end{pgfscope}%
\begin{pgfscope}%
\pgfpathrectangle{\pgfqpoint{0.481978in}{0.331635in}}{\pgfqpoint{4.960000in}{3.696000in}}%
\pgfusepath{clip}%
\pgfsetbuttcap%
\pgfsetroundjoin%
\definecolor{currentfill}{rgb}{0.631373,0.788235,0.956863}%
\pgfsetfillcolor{currentfill}%
\pgfsetlinewidth{0.481800pt}%
\definecolor{currentstroke}{rgb}{1.000000,1.000000,1.000000}%
\pgfsetstrokecolor{currentstroke}%
\pgfsetdash{}{0pt}%
\pgfpathmoveto{\pgfqpoint{2.033963in}{2.269910in}}%
\pgfpathcurveto{\pgfqpoint{2.045013in}{2.269910in}}{\pgfqpoint{2.055612in}{2.274300in}}{\pgfqpoint{2.063426in}{2.282114in}}%
\pgfpathcurveto{\pgfqpoint{2.071239in}{2.289928in}}{\pgfqpoint{2.075630in}{2.300527in}}{\pgfqpoint{2.075630in}{2.311577in}}%
\pgfpathcurveto{\pgfqpoint{2.075630in}{2.322627in}}{\pgfqpoint{2.071239in}{2.333226in}}{\pgfqpoint{2.063426in}{2.341040in}}%
\pgfpathcurveto{\pgfqpoint{2.055612in}{2.348853in}}{\pgfqpoint{2.045013in}{2.353243in}}{\pgfqpoint{2.033963in}{2.353243in}}%
\pgfpathcurveto{\pgfqpoint{2.022913in}{2.353243in}}{\pgfqpoint{2.012314in}{2.348853in}}{\pgfqpoint{2.004500in}{2.341040in}}%
\pgfpathcurveto{\pgfqpoint{1.996687in}{2.333226in}}{\pgfqpoint{1.992296in}{2.322627in}}{\pgfqpoint{1.992296in}{2.311577in}}%
\pgfpathcurveto{\pgfqpoint{1.992296in}{2.300527in}}{\pgfqpoint{1.996687in}{2.289928in}}{\pgfqpoint{2.004500in}{2.282114in}}%
\pgfpathcurveto{\pgfqpoint{2.012314in}{2.274300in}}{\pgfqpoint{2.022913in}{2.269910in}}{\pgfqpoint{2.033963in}{2.269910in}}%
\pgfpathclose%
\pgfusepath{stroke,fill}%
\end{pgfscope}%
\begin{pgfscope}%
\pgfpathrectangle{\pgfqpoint{0.481978in}{0.331635in}}{\pgfqpoint{4.960000in}{3.696000in}}%
\pgfusepath{clip}%
\pgfsetbuttcap%
\pgfsetroundjoin%
\definecolor{currentfill}{rgb}{0.631373,0.788235,0.956863}%
\pgfsetfillcolor{currentfill}%
\pgfsetlinewidth{0.481800pt}%
\definecolor{currentstroke}{rgb}{1.000000,1.000000,1.000000}%
\pgfsetstrokecolor{currentstroke}%
\pgfsetdash{}{0pt}%
\pgfpathmoveto{\pgfqpoint{2.328330in}{2.316727in}}%
\pgfpathcurveto{\pgfqpoint{2.339380in}{2.316727in}}{\pgfqpoint{2.349979in}{2.321117in}}{\pgfqpoint{2.357792in}{2.328931in}}%
\pgfpathcurveto{\pgfqpoint{2.365606in}{2.336744in}}{\pgfqpoint{2.369996in}{2.347343in}}{\pgfqpoint{2.369996in}{2.358393in}}%
\pgfpathcurveto{\pgfqpoint{2.369996in}{2.369444in}}{\pgfqpoint{2.365606in}{2.380043in}}{\pgfqpoint{2.357792in}{2.387856in}}%
\pgfpathcurveto{\pgfqpoint{2.349979in}{2.395670in}}{\pgfqpoint{2.339380in}{2.400060in}}{\pgfqpoint{2.328330in}{2.400060in}}%
\pgfpathcurveto{\pgfqpoint{2.317279in}{2.400060in}}{\pgfqpoint{2.306680in}{2.395670in}}{\pgfqpoint{2.298867in}{2.387856in}}%
\pgfpathcurveto{\pgfqpoint{2.291053in}{2.380043in}}{\pgfqpoint{2.286663in}{2.369444in}}{\pgfqpoint{2.286663in}{2.358393in}}%
\pgfpathcurveto{\pgfqpoint{2.286663in}{2.347343in}}{\pgfqpoint{2.291053in}{2.336744in}}{\pgfqpoint{2.298867in}{2.328931in}}%
\pgfpathcurveto{\pgfqpoint{2.306680in}{2.321117in}}{\pgfqpoint{2.317279in}{2.316727in}}{\pgfqpoint{2.328330in}{2.316727in}}%
\pgfpathclose%
\pgfusepath{stroke,fill}%
\end{pgfscope}%
\begin{pgfscope}%
\pgfpathrectangle{\pgfqpoint{0.481978in}{0.331635in}}{\pgfqpoint{4.960000in}{3.696000in}}%
\pgfusepath{clip}%
\pgfsetbuttcap%
\pgfsetroundjoin%
\definecolor{currentfill}{rgb}{0.631373,0.788235,0.956863}%
\pgfsetfillcolor{currentfill}%
\pgfsetlinewidth{0.481800pt}%
\definecolor{currentstroke}{rgb}{1.000000,1.000000,1.000000}%
\pgfsetstrokecolor{currentstroke}%
\pgfsetdash{}{0pt}%
\pgfpathmoveto{\pgfqpoint{4.500183in}{2.318758in}}%
\pgfpathcurveto{\pgfqpoint{4.511233in}{2.318758in}}{\pgfqpoint{4.521832in}{2.323148in}}{\pgfqpoint{4.529646in}{2.330962in}}%
\pgfpathcurveto{\pgfqpoint{4.537460in}{2.338775in}}{\pgfqpoint{4.541850in}{2.349374in}}{\pgfqpoint{4.541850in}{2.360424in}}%
\pgfpathcurveto{\pgfqpoint{4.541850in}{2.371475in}}{\pgfqpoint{4.537460in}{2.382074in}}{\pgfqpoint{4.529646in}{2.389887in}}%
\pgfpathcurveto{\pgfqpoint{4.521832in}{2.397701in}}{\pgfqpoint{4.511233in}{2.402091in}}{\pgfqpoint{4.500183in}{2.402091in}}%
\pgfpathcurveto{\pgfqpoint{4.489133in}{2.402091in}}{\pgfqpoint{4.478534in}{2.397701in}}{\pgfqpoint{4.470720in}{2.389887in}}%
\pgfpathcurveto{\pgfqpoint{4.462907in}{2.382074in}}{\pgfqpoint{4.458516in}{2.371475in}}{\pgfqpoint{4.458516in}{2.360424in}}%
\pgfpathcurveto{\pgfqpoint{4.458516in}{2.349374in}}{\pgfqpoint{4.462907in}{2.338775in}}{\pgfqpoint{4.470720in}{2.330962in}}%
\pgfpathcurveto{\pgfqpoint{4.478534in}{2.323148in}}{\pgfqpoint{4.489133in}{2.318758in}}{\pgfqpoint{4.500183in}{2.318758in}}%
\pgfpathclose%
\pgfusepath{stroke,fill}%
\end{pgfscope}%
\begin{pgfscope}%
\pgfpathrectangle{\pgfqpoint{0.481978in}{0.331635in}}{\pgfqpoint{4.960000in}{3.696000in}}%
\pgfusepath{clip}%
\pgfsetbuttcap%
\pgfsetroundjoin%
\definecolor{currentfill}{rgb}{0.631373,0.788235,0.956863}%
\pgfsetfillcolor{currentfill}%
\pgfsetlinewidth{0.481800pt}%
\definecolor{currentstroke}{rgb}{1.000000,1.000000,1.000000}%
\pgfsetstrokecolor{currentstroke}%
\pgfsetdash{}{0pt}%
\pgfpathmoveto{\pgfqpoint{3.254374in}{1.195259in}}%
\pgfpathcurveto{\pgfqpoint{3.265424in}{1.195259in}}{\pgfqpoint{3.276023in}{1.199649in}}{\pgfqpoint{3.283837in}{1.207463in}}%
\pgfpathcurveto{\pgfqpoint{3.291650in}{1.215276in}}{\pgfqpoint{3.296041in}{1.225875in}}{\pgfqpoint{3.296041in}{1.236925in}}%
\pgfpathcurveto{\pgfqpoint{3.296041in}{1.247975in}}{\pgfqpoint{3.291650in}{1.258575in}}{\pgfqpoint{3.283837in}{1.266388in}}%
\pgfpathcurveto{\pgfqpoint{3.276023in}{1.274202in}}{\pgfqpoint{3.265424in}{1.278592in}}{\pgfqpoint{3.254374in}{1.278592in}}%
\pgfpathcurveto{\pgfqpoint{3.243324in}{1.278592in}}{\pgfqpoint{3.232725in}{1.274202in}}{\pgfqpoint{3.224911in}{1.266388in}}%
\pgfpathcurveto{\pgfqpoint{3.217098in}{1.258575in}}{\pgfqpoint{3.212707in}{1.247975in}}{\pgfqpoint{3.212707in}{1.236925in}}%
\pgfpathcurveto{\pgfqpoint{3.212707in}{1.225875in}}{\pgfqpoint{3.217098in}{1.215276in}}{\pgfqpoint{3.224911in}{1.207463in}}%
\pgfpathcurveto{\pgfqpoint{3.232725in}{1.199649in}}{\pgfqpoint{3.243324in}{1.195259in}}{\pgfqpoint{3.254374in}{1.195259in}}%
\pgfpathclose%
\pgfusepath{stroke,fill}%
\end{pgfscope}%
\begin{pgfscope}%
\pgfpathrectangle{\pgfqpoint{0.481978in}{0.331635in}}{\pgfqpoint{4.960000in}{3.696000in}}%
\pgfusepath{clip}%
\pgfsetbuttcap%
\pgfsetroundjoin%
\definecolor{currentfill}{rgb}{0.631373,0.788235,0.956863}%
\pgfsetfillcolor{currentfill}%
\pgfsetlinewidth{0.481800pt}%
\definecolor{currentstroke}{rgb}{1.000000,1.000000,1.000000}%
\pgfsetstrokecolor{currentstroke}%
\pgfsetdash{}{0pt}%
\pgfpathmoveto{\pgfqpoint{2.136794in}{2.038567in}}%
\pgfpathcurveto{\pgfqpoint{2.147844in}{2.038567in}}{\pgfqpoint{2.158443in}{2.042957in}}{\pgfqpoint{2.166257in}{2.050771in}}%
\pgfpathcurveto{\pgfqpoint{2.174071in}{2.058585in}}{\pgfqpoint{2.178461in}{2.069184in}}{\pgfqpoint{2.178461in}{2.080234in}}%
\pgfpathcurveto{\pgfqpoint{2.178461in}{2.091284in}}{\pgfqpoint{2.174071in}{2.101883in}}{\pgfqpoint{2.166257in}{2.109696in}}%
\pgfpathcurveto{\pgfqpoint{2.158443in}{2.117510in}}{\pgfqpoint{2.147844in}{2.121900in}}{\pgfqpoint{2.136794in}{2.121900in}}%
\pgfpathcurveto{\pgfqpoint{2.125744in}{2.121900in}}{\pgfqpoint{2.115145in}{2.117510in}}{\pgfqpoint{2.107331in}{2.109696in}}%
\pgfpathcurveto{\pgfqpoint{2.099518in}{2.101883in}}{\pgfqpoint{2.095128in}{2.091284in}}{\pgfqpoint{2.095128in}{2.080234in}}%
\pgfpathcurveto{\pgfqpoint{2.095128in}{2.069184in}}{\pgfqpoint{2.099518in}{2.058585in}}{\pgfqpoint{2.107331in}{2.050771in}}%
\pgfpathcurveto{\pgfqpoint{2.115145in}{2.042957in}}{\pgfqpoint{2.125744in}{2.038567in}}{\pgfqpoint{2.136794in}{2.038567in}}%
\pgfpathclose%
\pgfusepath{stroke,fill}%
\end{pgfscope}%
\begin{pgfscope}%
\pgfpathrectangle{\pgfqpoint{0.481978in}{0.331635in}}{\pgfqpoint{4.960000in}{3.696000in}}%
\pgfusepath{clip}%
\pgfsetbuttcap%
\pgfsetroundjoin%
\definecolor{currentfill}{rgb}{0.631373,0.788235,0.956863}%
\pgfsetfillcolor{currentfill}%
\pgfsetlinewidth{0.481800pt}%
\definecolor{currentstroke}{rgb}{1.000000,1.000000,1.000000}%
\pgfsetstrokecolor{currentstroke}%
\pgfsetdash{}{0pt}%
\pgfpathmoveto{\pgfqpoint{3.643865in}{1.063609in}}%
\pgfpathcurveto{\pgfqpoint{3.654915in}{1.063609in}}{\pgfqpoint{3.665514in}{1.067999in}}{\pgfqpoint{3.673328in}{1.075813in}}%
\pgfpathcurveto{\pgfqpoint{3.681142in}{1.083627in}}{\pgfqpoint{3.685532in}{1.094226in}}{\pgfqpoint{3.685532in}{1.105276in}}%
\pgfpathcurveto{\pgfqpoint{3.685532in}{1.116326in}}{\pgfqpoint{3.681142in}{1.126925in}}{\pgfqpoint{3.673328in}{1.134739in}}%
\pgfpathcurveto{\pgfqpoint{3.665514in}{1.142552in}}{\pgfqpoint{3.654915in}{1.146943in}}{\pgfqpoint{3.643865in}{1.146943in}}%
\pgfpathcurveto{\pgfqpoint{3.632815in}{1.146943in}}{\pgfqpoint{3.622216in}{1.142552in}}{\pgfqpoint{3.614402in}{1.134739in}}%
\pgfpathcurveto{\pgfqpoint{3.606589in}{1.126925in}}{\pgfqpoint{3.602199in}{1.116326in}}{\pgfqpoint{3.602199in}{1.105276in}}%
\pgfpathcurveto{\pgfqpoint{3.602199in}{1.094226in}}{\pgfqpoint{3.606589in}{1.083627in}}{\pgfqpoint{3.614402in}{1.075813in}}%
\pgfpathcurveto{\pgfqpoint{3.622216in}{1.067999in}}{\pgfqpoint{3.632815in}{1.063609in}}{\pgfqpoint{3.643865in}{1.063609in}}%
\pgfpathclose%
\pgfusepath{stroke,fill}%
\end{pgfscope}%
\begin{pgfscope}%
\pgfpathrectangle{\pgfqpoint{0.481978in}{0.331635in}}{\pgfqpoint{4.960000in}{3.696000in}}%
\pgfusepath{clip}%
\pgfsetbuttcap%
\pgfsetroundjoin%
\definecolor{currentfill}{rgb}{0.631373,0.788235,0.956863}%
\pgfsetfillcolor{currentfill}%
\pgfsetlinewidth{0.481800pt}%
\definecolor{currentstroke}{rgb}{1.000000,1.000000,1.000000}%
\pgfsetstrokecolor{currentstroke}%
\pgfsetdash{}{0pt}%
\pgfpathmoveto{\pgfqpoint{3.577822in}{2.021481in}}%
\pgfpathcurveto{\pgfqpoint{3.588872in}{2.021481in}}{\pgfqpoint{3.599471in}{2.025872in}}{\pgfqpoint{3.607285in}{2.033685in}}%
\pgfpathcurveto{\pgfqpoint{3.615098in}{2.041499in}}{\pgfqpoint{3.619489in}{2.052098in}}{\pgfqpoint{3.619489in}{2.063148in}}%
\pgfpathcurveto{\pgfqpoint{3.619489in}{2.074198in}}{\pgfqpoint{3.615098in}{2.084797in}}{\pgfqpoint{3.607285in}{2.092611in}}%
\pgfpathcurveto{\pgfqpoint{3.599471in}{2.100425in}}{\pgfqpoint{3.588872in}{2.104815in}}{\pgfqpoint{3.577822in}{2.104815in}}%
\pgfpathcurveto{\pgfqpoint{3.566772in}{2.104815in}}{\pgfqpoint{3.556173in}{2.100425in}}{\pgfqpoint{3.548359in}{2.092611in}}%
\pgfpathcurveto{\pgfqpoint{3.540545in}{2.084797in}}{\pgfqpoint{3.536155in}{2.074198in}}{\pgfqpoint{3.536155in}{2.063148in}}%
\pgfpathcurveto{\pgfqpoint{3.536155in}{2.052098in}}{\pgfqpoint{3.540545in}{2.041499in}}{\pgfqpoint{3.548359in}{2.033685in}}%
\pgfpathcurveto{\pgfqpoint{3.556173in}{2.025872in}}{\pgfqpoint{3.566772in}{2.021481in}}{\pgfqpoint{3.577822in}{2.021481in}}%
\pgfpathclose%
\pgfusepath{stroke,fill}%
\end{pgfscope}%
\begin{pgfscope}%
\pgfpathrectangle{\pgfqpoint{0.481978in}{0.331635in}}{\pgfqpoint{4.960000in}{3.696000in}}%
\pgfusepath{clip}%
\pgfsetbuttcap%
\pgfsetroundjoin%
\definecolor{currentfill}{rgb}{0.631373,0.788235,0.956863}%
\pgfsetfillcolor{currentfill}%
\pgfsetlinewidth{0.481800pt}%
\definecolor{currentstroke}{rgb}{1.000000,1.000000,1.000000}%
\pgfsetstrokecolor{currentstroke}%
\pgfsetdash{}{0pt}%
\pgfpathmoveto{\pgfqpoint{1.483591in}{2.487058in}}%
\pgfpathcurveto{\pgfqpoint{1.494641in}{2.487058in}}{\pgfqpoint{1.505240in}{2.491449in}}{\pgfqpoint{1.513054in}{2.499262in}}%
\pgfpathcurveto{\pgfqpoint{1.520867in}{2.507076in}}{\pgfqpoint{1.525257in}{2.517675in}}{\pgfqpoint{1.525257in}{2.528725in}}%
\pgfpathcurveto{\pgfqpoint{1.525257in}{2.539775in}}{\pgfqpoint{1.520867in}{2.550374in}}{\pgfqpoint{1.513054in}{2.558188in}}%
\pgfpathcurveto{\pgfqpoint{1.505240in}{2.566001in}}{\pgfqpoint{1.494641in}{2.570392in}}{\pgfqpoint{1.483591in}{2.570392in}}%
\pgfpathcurveto{\pgfqpoint{1.472541in}{2.570392in}}{\pgfqpoint{1.461942in}{2.566001in}}{\pgfqpoint{1.454128in}{2.558188in}}%
\pgfpathcurveto{\pgfqpoint{1.446314in}{2.550374in}}{\pgfqpoint{1.441924in}{2.539775in}}{\pgfqpoint{1.441924in}{2.528725in}}%
\pgfpathcurveto{\pgfqpoint{1.441924in}{2.517675in}}{\pgfqpoint{1.446314in}{2.507076in}}{\pgfqpoint{1.454128in}{2.499262in}}%
\pgfpathcurveto{\pgfqpoint{1.461942in}{2.491449in}}{\pgfqpoint{1.472541in}{2.487058in}}{\pgfqpoint{1.483591in}{2.487058in}}%
\pgfpathclose%
\pgfusepath{stroke,fill}%
\end{pgfscope}%
\begin{pgfscope}%
\pgfpathrectangle{\pgfqpoint{0.481978in}{0.331635in}}{\pgfqpoint{4.960000in}{3.696000in}}%
\pgfusepath{clip}%
\pgfsetbuttcap%
\pgfsetroundjoin%
\definecolor{currentfill}{rgb}{0.631373,0.788235,0.956863}%
\pgfsetfillcolor{currentfill}%
\pgfsetlinewidth{0.481800pt}%
\definecolor{currentstroke}{rgb}{1.000000,1.000000,1.000000}%
\pgfsetstrokecolor{currentstroke}%
\pgfsetdash{}{0pt}%
\pgfpathmoveto{\pgfqpoint{2.576405in}{2.150585in}}%
\pgfpathcurveto{\pgfqpoint{2.587456in}{2.150585in}}{\pgfqpoint{2.598055in}{2.154975in}}{\pgfqpoint{2.605868in}{2.162788in}}%
\pgfpathcurveto{\pgfqpoint{2.613682in}{2.170602in}}{\pgfqpoint{2.618072in}{2.181201in}}{\pgfqpoint{2.618072in}{2.192251in}}%
\pgfpathcurveto{\pgfqpoint{2.618072in}{2.203301in}}{\pgfqpoint{2.613682in}{2.213900in}}{\pgfqpoint{2.605868in}{2.221714in}}%
\pgfpathcurveto{\pgfqpoint{2.598055in}{2.229528in}}{\pgfqpoint{2.587456in}{2.233918in}}{\pgfqpoint{2.576405in}{2.233918in}}%
\pgfpathcurveto{\pgfqpoint{2.565355in}{2.233918in}}{\pgfqpoint{2.554756in}{2.229528in}}{\pgfqpoint{2.546943in}{2.221714in}}%
\pgfpathcurveto{\pgfqpoint{2.539129in}{2.213900in}}{\pgfqpoint{2.534739in}{2.203301in}}{\pgfqpoint{2.534739in}{2.192251in}}%
\pgfpathcurveto{\pgfqpoint{2.534739in}{2.181201in}}{\pgfqpoint{2.539129in}{2.170602in}}{\pgfqpoint{2.546943in}{2.162788in}}%
\pgfpathcurveto{\pgfqpoint{2.554756in}{2.154975in}}{\pgfqpoint{2.565355in}{2.150585in}}{\pgfqpoint{2.576405in}{2.150585in}}%
\pgfpathclose%
\pgfusepath{stroke,fill}%
\end{pgfscope}%
\begin{pgfscope}%
\pgfpathrectangle{\pgfqpoint{0.481978in}{0.331635in}}{\pgfqpoint{4.960000in}{3.696000in}}%
\pgfusepath{clip}%
\pgfsetbuttcap%
\pgfsetroundjoin%
\definecolor{currentfill}{rgb}{0.631373,0.788235,0.956863}%
\pgfsetfillcolor{currentfill}%
\pgfsetlinewidth{0.481800pt}%
\definecolor{currentstroke}{rgb}{1.000000,1.000000,1.000000}%
\pgfsetstrokecolor{currentstroke}%
\pgfsetdash{}{0pt}%
\pgfpathmoveto{\pgfqpoint{2.282422in}{1.506179in}}%
\pgfpathcurveto{\pgfqpoint{2.293472in}{1.506179in}}{\pgfqpoint{2.304071in}{1.510569in}}{\pgfqpoint{2.311885in}{1.518382in}}%
\pgfpathcurveto{\pgfqpoint{2.319699in}{1.526196in}}{\pgfqpoint{2.324089in}{1.536795in}}{\pgfqpoint{2.324089in}{1.547845in}}%
\pgfpathcurveto{\pgfqpoint{2.324089in}{1.558895in}}{\pgfqpoint{2.319699in}{1.569494in}}{\pgfqpoint{2.311885in}{1.577308in}}%
\pgfpathcurveto{\pgfqpoint{2.304071in}{1.585122in}}{\pgfqpoint{2.293472in}{1.589512in}}{\pgfqpoint{2.282422in}{1.589512in}}%
\pgfpathcurveto{\pgfqpoint{2.271372in}{1.589512in}}{\pgfqpoint{2.260773in}{1.585122in}}{\pgfqpoint{2.252959in}{1.577308in}}%
\pgfpathcurveto{\pgfqpoint{2.245146in}{1.569494in}}{\pgfqpoint{2.240755in}{1.558895in}}{\pgfqpoint{2.240755in}{1.547845in}}%
\pgfpathcurveto{\pgfqpoint{2.240755in}{1.536795in}}{\pgfqpoint{2.245146in}{1.526196in}}{\pgfqpoint{2.252959in}{1.518382in}}%
\pgfpathcurveto{\pgfqpoint{2.260773in}{1.510569in}}{\pgfqpoint{2.271372in}{1.506179in}}{\pgfqpoint{2.282422in}{1.506179in}}%
\pgfpathclose%
\pgfusepath{stroke,fill}%
\end{pgfscope}%
\begin{pgfscope}%
\pgfpathrectangle{\pgfqpoint{0.481978in}{0.331635in}}{\pgfqpoint{4.960000in}{3.696000in}}%
\pgfusepath{clip}%
\pgfsetbuttcap%
\pgfsetroundjoin%
\definecolor{currentfill}{rgb}{0.631373,0.788235,0.956863}%
\pgfsetfillcolor{currentfill}%
\pgfsetlinewidth{0.481800pt}%
\definecolor{currentstroke}{rgb}{1.000000,1.000000,1.000000}%
\pgfsetstrokecolor{currentstroke}%
\pgfsetdash{}{0pt}%
\pgfpathmoveto{\pgfqpoint{1.547612in}{2.568752in}}%
\pgfpathcurveto{\pgfqpoint{1.558662in}{2.568752in}}{\pgfqpoint{1.569262in}{2.573142in}}{\pgfqpoint{1.577075in}{2.580956in}}%
\pgfpathcurveto{\pgfqpoint{1.584889in}{2.588769in}}{\pgfqpoint{1.589279in}{2.599368in}}{\pgfqpoint{1.589279in}{2.610418in}}%
\pgfpathcurveto{\pgfqpoint{1.589279in}{2.621469in}}{\pgfqpoint{1.584889in}{2.632068in}}{\pgfqpoint{1.577075in}{2.639881in}}%
\pgfpathcurveto{\pgfqpoint{1.569262in}{2.647695in}}{\pgfqpoint{1.558662in}{2.652085in}}{\pgfqpoint{1.547612in}{2.652085in}}%
\pgfpathcurveto{\pgfqpoint{1.536562in}{2.652085in}}{\pgfqpoint{1.525963in}{2.647695in}}{\pgfqpoint{1.518150in}{2.639881in}}%
\pgfpathcurveto{\pgfqpoint{1.510336in}{2.632068in}}{\pgfqpoint{1.505946in}{2.621469in}}{\pgfqpoint{1.505946in}{2.610418in}}%
\pgfpathcurveto{\pgfqpoint{1.505946in}{2.599368in}}{\pgfqpoint{1.510336in}{2.588769in}}{\pgfqpoint{1.518150in}{2.580956in}}%
\pgfpathcurveto{\pgfqpoint{1.525963in}{2.573142in}}{\pgfqpoint{1.536562in}{2.568752in}}{\pgfqpoint{1.547612in}{2.568752in}}%
\pgfpathclose%
\pgfusepath{stroke,fill}%
\end{pgfscope}%
\begin{pgfscope}%
\pgfpathrectangle{\pgfqpoint{0.481978in}{0.331635in}}{\pgfqpoint{4.960000in}{3.696000in}}%
\pgfusepath{clip}%
\pgfsetbuttcap%
\pgfsetroundjoin%
\definecolor{currentfill}{rgb}{0.631373,0.788235,0.956863}%
\pgfsetfillcolor{currentfill}%
\pgfsetlinewidth{0.481800pt}%
\definecolor{currentstroke}{rgb}{1.000000,1.000000,1.000000}%
\pgfsetstrokecolor{currentstroke}%
\pgfsetdash{}{0pt}%
\pgfpathmoveto{\pgfqpoint{2.898875in}{2.042295in}}%
\pgfpathcurveto{\pgfqpoint{2.909926in}{2.042295in}}{\pgfqpoint{2.920525in}{2.046685in}}{\pgfqpoint{2.928338in}{2.054498in}}%
\pgfpathcurveto{\pgfqpoint{2.936152in}{2.062312in}}{\pgfqpoint{2.940542in}{2.072911in}}{\pgfqpoint{2.940542in}{2.083961in}}%
\pgfpathcurveto{\pgfqpoint{2.940542in}{2.095011in}}{\pgfqpoint{2.936152in}{2.105610in}}{\pgfqpoint{2.928338in}{2.113424in}}%
\pgfpathcurveto{\pgfqpoint{2.920525in}{2.121238in}}{\pgfqpoint{2.909926in}{2.125628in}}{\pgfqpoint{2.898875in}{2.125628in}}%
\pgfpathcurveto{\pgfqpoint{2.887825in}{2.125628in}}{\pgfqpoint{2.877226in}{2.121238in}}{\pgfqpoint{2.869413in}{2.113424in}}%
\pgfpathcurveto{\pgfqpoint{2.861599in}{2.105610in}}{\pgfqpoint{2.857209in}{2.095011in}}{\pgfqpoint{2.857209in}{2.083961in}}%
\pgfpathcurveto{\pgfqpoint{2.857209in}{2.072911in}}{\pgfqpoint{2.861599in}{2.062312in}}{\pgfqpoint{2.869413in}{2.054498in}}%
\pgfpathcurveto{\pgfqpoint{2.877226in}{2.046685in}}{\pgfqpoint{2.887825in}{2.042295in}}{\pgfqpoint{2.898875in}{2.042295in}}%
\pgfpathclose%
\pgfusepath{stroke,fill}%
\end{pgfscope}%
\begin{pgfscope}%
\pgfpathrectangle{\pgfqpoint{0.481978in}{0.331635in}}{\pgfqpoint{4.960000in}{3.696000in}}%
\pgfusepath{clip}%
\pgfsetbuttcap%
\pgfsetroundjoin%
\definecolor{currentfill}{rgb}{0.631373,0.788235,0.956863}%
\pgfsetfillcolor{currentfill}%
\pgfsetlinewidth{0.481800pt}%
\definecolor{currentstroke}{rgb}{1.000000,1.000000,1.000000}%
\pgfsetstrokecolor{currentstroke}%
\pgfsetdash{}{0pt}%
\pgfpathmoveto{\pgfqpoint{2.562342in}{2.416100in}}%
\pgfpathcurveto{\pgfqpoint{2.573392in}{2.416100in}}{\pgfqpoint{2.583991in}{2.420490in}}{\pgfqpoint{2.591805in}{2.428304in}}%
\pgfpathcurveto{\pgfqpoint{2.599619in}{2.436117in}}{\pgfqpoint{2.604009in}{2.446716in}}{\pgfqpoint{2.604009in}{2.457766in}}%
\pgfpathcurveto{\pgfqpoint{2.604009in}{2.468817in}}{\pgfqpoint{2.599619in}{2.479416in}}{\pgfqpoint{2.591805in}{2.487229in}}%
\pgfpathcurveto{\pgfqpoint{2.583991in}{2.495043in}}{\pgfqpoint{2.573392in}{2.499433in}}{\pgfqpoint{2.562342in}{2.499433in}}%
\pgfpathcurveto{\pgfqpoint{2.551292in}{2.499433in}}{\pgfqpoint{2.540693in}{2.495043in}}{\pgfqpoint{2.532880in}{2.487229in}}%
\pgfpathcurveto{\pgfqpoint{2.525066in}{2.479416in}}{\pgfqpoint{2.520676in}{2.468817in}}{\pgfqpoint{2.520676in}{2.457766in}}%
\pgfpathcurveto{\pgfqpoint{2.520676in}{2.446716in}}{\pgfqpoint{2.525066in}{2.436117in}}{\pgfqpoint{2.532880in}{2.428304in}}%
\pgfpathcurveto{\pgfqpoint{2.540693in}{2.420490in}}{\pgfqpoint{2.551292in}{2.416100in}}{\pgfqpoint{2.562342in}{2.416100in}}%
\pgfpathclose%
\pgfusepath{stroke,fill}%
\end{pgfscope}%
\begin{pgfscope}%
\pgfpathrectangle{\pgfqpoint{0.481978in}{0.331635in}}{\pgfqpoint{4.960000in}{3.696000in}}%
\pgfusepath{clip}%
\pgfsetbuttcap%
\pgfsetroundjoin%
\definecolor{currentfill}{rgb}{0.631373,0.788235,0.956863}%
\pgfsetfillcolor{currentfill}%
\pgfsetlinewidth{0.481800pt}%
\definecolor{currentstroke}{rgb}{1.000000,1.000000,1.000000}%
\pgfsetstrokecolor{currentstroke}%
\pgfsetdash{}{0pt}%
\pgfpathmoveto{\pgfqpoint{2.562832in}{3.089432in}}%
\pgfpathcurveto{\pgfqpoint{2.573883in}{3.089432in}}{\pgfqpoint{2.584482in}{3.093822in}}{\pgfqpoint{2.592295in}{3.101636in}}%
\pgfpathcurveto{\pgfqpoint{2.600109in}{3.109449in}}{\pgfqpoint{2.604499in}{3.120048in}}{\pgfqpoint{2.604499in}{3.131098in}}%
\pgfpathcurveto{\pgfqpoint{2.604499in}{3.142148in}}{\pgfqpoint{2.600109in}{3.152747in}}{\pgfqpoint{2.592295in}{3.160561in}}%
\pgfpathcurveto{\pgfqpoint{2.584482in}{3.168375in}}{\pgfqpoint{2.573883in}{3.172765in}}{\pgfqpoint{2.562832in}{3.172765in}}%
\pgfpathcurveto{\pgfqpoint{2.551782in}{3.172765in}}{\pgfqpoint{2.541183in}{3.168375in}}{\pgfqpoint{2.533370in}{3.160561in}}%
\pgfpathcurveto{\pgfqpoint{2.525556in}{3.152747in}}{\pgfqpoint{2.521166in}{3.142148in}}{\pgfqpoint{2.521166in}{3.131098in}}%
\pgfpathcurveto{\pgfqpoint{2.521166in}{3.120048in}}{\pgfqpoint{2.525556in}{3.109449in}}{\pgfqpoint{2.533370in}{3.101636in}}%
\pgfpathcurveto{\pgfqpoint{2.541183in}{3.093822in}}{\pgfqpoint{2.551782in}{3.089432in}}{\pgfqpoint{2.562832in}{3.089432in}}%
\pgfpathclose%
\pgfusepath{stroke,fill}%
\end{pgfscope}%
\begin{pgfscope}%
\pgfpathrectangle{\pgfqpoint{0.481978in}{0.331635in}}{\pgfqpoint{4.960000in}{3.696000in}}%
\pgfusepath{clip}%
\pgfsetbuttcap%
\pgfsetroundjoin%
\definecolor{currentfill}{rgb}{0.631373,0.788235,0.956863}%
\pgfsetfillcolor{currentfill}%
\pgfsetlinewidth{0.481800pt}%
\definecolor{currentstroke}{rgb}{1.000000,1.000000,1.000000}%
\pgfsetstrokecolor{currentstroke}%
\pgfsetdash{}{0pt}%
\pgfpathmoveto{\pgfqpoint{2.194229in}{2.398448in}}%
\pgfpathcurveto{\pgfqpoint{2.205279in}{2.398448in}}{\pgfqpoint{2.215878in}{2.402839in}}{\pgfqpoint{2.223692in}{2.410652in}}%
\pgfpathcurveto{\pgfqpoint{2.231506in}{2.418466in}}{\pgfqpoint{2.235896in}{2.429065in}}{\pgfqpoint{2.235896in}{2.440115in}}%
\pgfpathcurveto{\pgfqpoint{2.235896in}{2.451165in}}{\pgfqpoint{2.231506in}{2.461764in}}{\pgfqpoint{2.223692in}{2.469578in}}%
\pgfpathcurveto{\pgfqpoint{2.215878in}{2.477391in}}{\pgfqpoint{2.205279in}{2.481782in}}{\pgfqpoint{2.194229in}{2.481782in}}%
\pgfpathcurveto{\pgfqpoint{2.183179in}{2.481782in}}{\pgfqpoint{2.172580in}{2.477391in}}{\pgfqpoint{2.164766in}{2.469578in}}%
\pgfpathcurveto{\pgfqpoint{2.156953in}{2.461764in}}{\pgfqpoint{2.152563in}{2.451165in}}{\pgfqpoint{2.152563in}{2.440115in}}%
\pgfpathcurveto{\pgfqpoint{2.152563in}{2.429065in}}{\pgfqpoint{2.156953in}{2.418466in}}{\pgfqpoint{2.164766in}{2.410652in}}%
\pgfpathcurveto{\pgfqpoint{2.172580in}{2.402839in}}{\pgfqpoint{2.183179in}{2.398448in}}{\pgfqpoint{2.194229in}{2.398448in}}%
\pgfpathclose%
\pgfusepath{stroke,fill}%
\end{pgfscope}%
\begin{pgfscope}%
\pgfpathrectangle{\pgfqpoint{0.481978in}{0.331635in}}{\pgfqpoint{4.960000in}{3.696000in}}%
\pgfusepath{clip}%
\pgfsetbuttcap%
\pgfsetroundjoin%
\definecolor{currentfill}{rgb}{0.631373,0.788235,0.956863}%
\pgfsetfillcolor{currentfill}%
\pgfsetlinewidth{0.481800pt}%
\definecolor{currentstroke}{rgb}{1.000000,1.000000,1.000000}%
\pgfsetstrokecolor{currentstroke}%
\pgfsetdash{}{0pt}%
\pgfpathmoveto{\pgfqpoint{4.484037in}{2.259451in}}%
\pgfpathcurveto{\pgfqpoint{4.495087in}{2.259451in}}{\pgfqpoint{4.505686in}{2.263841in}}{\pgfqpoint{4.513500in}{2.271655in}}%
\pgfpathcurveto{\pgfqpoint{4.521313in}{2.279468in}}{\pgfqpoint{4.525704in}{2.290067in}}{\pgfqpoint{4.525704in}{2.301118in}}%
\pgfpathcurveto{\pgfqpoint{4.525704in}{2.312168in}}{\pgfqpoint{4.521313in}{2.322767in}}{\pgfqpoint{4.513500in}{2.330580in}}%
\pgfpathcurveto{\pgfqpoint{4.505686in}{2.338394in}}{\pgfqpoint{4.495087in}{2.342784in}}{\pgfqpoint{4.484037in}{2.342784in}}%
\pgfpathcurveto{\pgfqpoint{4.472987in}{2.342784in}}{\pgfqpoint{4.462388in}{2.338394in}}{\pgfqpoint{4.454574in}{2.330580in}}%
\pgfpathcurveto{\pgfqpoint{4.446760in}{2.322767in}}{\pgfqpoint{4.442370in}{2.312168in}}{\pgfqpoint{4.442370in}{2.301118in}}%
\pgfpathcurveto{\pgfqpoint{4.442370in}{2.290067in}}{\pgfqpoint{4.446760in}{2.279468in}}{\pgfqpoint{4.454574in}{2.271655in}}%
\pgfpathcurveto{\pgfqpoint{4.462388in}{2.263841in}}{\pgfqpoint{4.472987in}{2.259451in}}{\pgfqpoint{4.484037in}{2.259451in}}%
\pgfpathclose%
\pgfusepath{stroke,fill}%
\end{pgfscope}%
\begin{pgfscope}%
\pgfpathrectangle{\pgfqpoint{0.481978in}{0.331635in}}{\pgfqpoint{4.960000in}{3.696000in}}%
\pgfusepath{clip}%
\pgfsetbuttcap%
\pgfsetroundjoin%
\definecolor{currentfill}{rgb}{0.631373,0.788235,0.956863}%
\pgfsetfillcolor{currentfill}%
\pgfsetlinewidth{0.481800pt}%
\definecolor{currentstroke}{rgb}{1.000000,1.000000,1.000000}%
\pgfsetstrokecolor{currentstroke}%
\pgfsetdash{}{0pt}%
\pgfpathmoveto{\pgfqpoint{2.713554in}{1.960460in}}%
\pgfpathcurveto{\pgfqpoint{2.724605in}{1.960460in}}{\pgfqpoint{2.735204in}{1.964850in}}{\pgfqpoint{2.743017in}{1.972664in}}%
\pgfpathcurveto{\pgfqpoint{2.750831in}{1.980477in}}{\pgfqpoint{2.755221in}{1.991076in}}{\pgfqpoint{2.755221in}{2.002126in}}%
\pgfpathcurveto{\pgfqpoint{2.755221in}{2.013177in}}{\pgfqpoint{2.750831in}{2.023776in}}{\pgfqpoint{2.743017in}{2.031589in}}%
\pgfpathcurveto{\pgfqpoint{2.735204in}{2.039403in}}{\pgfqpoint{2.724605in}{2.043793in}}{\pgfqpoint{2.713554in}{2.043793in}}%
\pgfpathcurveto{\pgfqpoint{2.702504in}{2.043793in}}{\pgfqpoint{2.691905in}{2.039403in}}{\pgfqpoint{2.684092in}{2.031589in}}%
\pgfpathcurveto{\pgfqpoint{2.676278in}{2.023776in}}{\pgfqpoint{2.671888in}{2.013177in}}{\pgfqpoint{2.671888in}{2.002126in}}%
\pgfpathcurveto{\pgfqpoint{2.671888in}{1.991076in}}{\pgfqpoint{2.676278in}{1.980477in}}{\pgfqpoint{2.684092in}{1.972664in}}%
\pgfpathcurveto{\pgfqpoint{2.691905in}{1.964850in}}{\pgfqpoint{2.702504in}{1.960460in}}{\pgfqpoint{2.713554in}{1.960460in}}%
\pgfpathclose%
\pgfusepath{stroke,fill}%
\end{pgfscope}%
\begin{pgfscope}%
\pgfpathrectangle{\pgfqpoint{0.481978in}{0.331635in}}{\pgfqpoint{4.960000in}{3.696000in}}%
\pgfusepath{clip}%
\pgfsetbuttcap%
\pgfsetroundjoin%
\definecolor{currentfill}{rgb}{0.631373,0.788235,0.956863}%
\pgfsetfillcolor{currentfill}%
\pgfsetlinewidth{0.481800pt}%
\definecolor{currentstroke}{rgb}{1.000000,1.000000,1.000000}%
\pgfsetstrokecolor{currentstroke}%
\pgfsetdash{}{0pt}%
\pgfpathmoveto{\pgfqpoint{2.577033in}{2.981696in}}%
\pgfpathcurveto{\pgfqpoint{2.588083in}{2.981696in}}{\pgfqpoint{2.598682in}{2.986086in}}{\pgfqpoint{2.606496in}{2.993900in}}%
\pgfpathcurveto{\pgfqpoint{2.614310in}{3.001713in}}{\pgfqpoint{2.618700in}{3.012312in}}{\pgfqpoint{2.618700in}{3.023362in}}%
\pgfpathcurveto{\pgfqpoint{2.618700in}{3.034412in}}{\pgfqpoint{2.614310in}{3.045012in}}{\pgfqpoint{2.606496in}{3.052825in}}%
\pgfpathcurveto{\pgfqpoint{2.598682in}{3.060639in}}{\pgfqpoint{2.588083in}{3.065029in}}{\pgfqpoint{2.577033in}{3.065029in}}%
\pgfpathcurveto{\pgfqpoint{2.565983in}{3.065029in}}{\pgfqpoint{2.555384in}{3.060639in}}{\pgfqpoint{2.547570in}{3.052825in}}%
\pgfpathcurveto{\pgfqpoint{2.539757in}{3.045012in}}{\pgfqpoint{2.535366in}{3.034412in}}{\pgfqpoint{2.535366in}{3.023362in}}%
\pgfpathcurveto{\pgfqpoint{2.535366in}{3.012312in}}{\pgfqpoint{2.539757in}{3.001713in}}{\pgfqpoint{2.547570in}{2.993900in}}%
\pgfpathcurveto{\pgfqpoint{2.555384in}{2.986086in}}{\pgfqpoint{2.565983in}{2.981696in}}{\pgfqpoint{2.577033in}{2.981696in}}%
\pgfpathclose%
\pgfusepath{stroke,fill}%
\end{pgfscope}%
\begin{pgfscope}%
\pgfpathrectangle{\pgfqpoint{0.481978in}{0.331635in}}{\pgfqpoint{4.960000in}{3.696000in}}%
\pgfusepath{clip}%
\pgfsetbuttcap%
\pgfsetroundjoin%
\definecolor{currentfill}{rgb}{0.631373,0.788235,0.956863}%
\pgfsetfillcolor{currentfill}%
\pgfsetlinewidth{0.481800pt}%
\definecolor{currentstroke}{rgb}{1.000000,1.000000,1.000000}%
\pgfsetstrokecolor{currentstroke}%
\pgfsetdash{}{0pt}%
\pgfpathmoveto{\pgfqpoint{2.245375in}{3.193863in}}%
\pgfpathcurveto{\pgfqpoint{2.256426in}{3.193863in}}{\pgfqpoint{2.267025in}{3.198253in}}{\pgfqpoint{2.274838in}{3.206067in}}%
\pgfpathcurveto{\pgfqpoint{2.282652in}{3.213881in}}{\pgfqpoint{2.287042in}{3.224480in}}{\pgfqpoint{2.287042in}{3.235530in}}%
\pgfpathcurveto{\pgfqpoint{2.287042in}{3.246580in}}{\pgfqpoint{2.282652in}{3.257179in}}{\pgfqpoint{2.274838in}{3.264993in}}%
\pgfpathcurveto{\pgfqpoint{2.267025in}{3.272806in}}{\pgfqpoint{2.256426in}{3.277196in}}{\pgfqpoint{2.245375in}{3.277196in}}%
\pgfpathcurveto{\pgfqpoint{2.234325in}{3.277196in}}{\pgfqpoint{2.223726in}{3.272806in}}{\pgfqpoint{2.215913in}{3.264993in}}%
\pgfpathcurveto{\pgfqpoint{2.208099in}{3.257179in}}{\pgfqpoint{2.203709in}{3.246580in}}{\pgfqpoint{2.203709in}{3.235530in}}%
\pgfpathcurveto{\pgfqpoint{2.203709in}{3.224480in}}{\pgfqpoint{2.208099in}{3.213881in}}{\pgfqpoint{2.215913in}{3.206067in}}%
\pgfpathcurveto{\pgfqpoint{2.223726in}{3.198253in}}{\pgfqpoint{2.234325in}{3.193863in}}{\pgfqpoint{2.245375in}{3.193863in}}%
\pgfpathclose%
\pgfusepath{stroke,fill}%
\end{pgfscope}%
\begin{pgfscope}%
\pgfpathrectangle{\pgfqpoint{0.481978in}{0.331635in}}{\pgfqpoint{4.960000in}{3.696000in}}%
\pgfusepath{clip}%
\pgfsetbuttcap%
\pgfsetroundjoin%
\definecolor{currentfill}{rgb}{0.631373,0.788235,0.956863}%
\pgfsetfillcolor{currentfill}%
\pgfsetlinewidth{0.481800pt}%
\definecolor{currentstroke}{rgb}{1.000000,1.000000,1.000000}%
\pgfsetstrokecolor{currentstroke}%
\pgfsetdash{}{0pt}%
\pgfpathmoveto{\pgfqpoint{1.712744in}{2.085248in}}%
\pgfpathcurveto{\pgfqpoint{1.723794in}{2.085248in}}{\pgfqpoint{1.734394in}{2.089638in}}{\pgfqpoint{1.742207in}{2.097452in}}%
\pgfpathcurveto{\pgfqpoint{1.750021in}{2.105265in}}{\pgfqpoint{1.754411in}{2.115864in}}{\pgfqpoint{1.754411in}{2.126914in}}%
\pgfpathcurveto{\pgfqpoint{1.754411in}{2.137965in}}{\pgfqpoint{1.750021in}{2.148564in}}{\pgfqpoint{1.742207in}{2.156377in}}%
\pgfpathcurveto{\pgfqpoint{1.734394in}{2.164191in}}{\pgfqpoint{1.723794in}{2.168581in}}{\pgfqpoint{1.712744in}{2.168581in}}%
\pgfpathcurveto{\pgfqpoint{1.701694in}{2.168581in}}{\pgfqpoint{1.691095in}{2.164191in}}{\pgfqpoint{1.683282in}{2.156377in}}%
\pgfpathcurveto{\pgfqpoint{1.675468in}{2.148564in}}{\pgfqpoint{1.671078in}{2.137965in}}{\pgfqpoint{1.671078in}{2.126914in}}%
\pgfpathcurveto{\pgfqpoint{1.671078in}{2.115864in}}{\pgfqpoint{1.675468in}{2.105265in}}{\pgfqpoint{1.683282in}{2.097452in}}%
\pgfpathcurveto{\pgfqpoint{1.691095in}{2.089638in}}{\pgfqpoint{1.701694in}{2.085248in}}{\pgfqpoint{1.712744in}{2.085248in}}%
\pgfpathclose%
\pgfusepath{stroke,fill}%
\end{pgfscope}%
\begin{pgfscope}%
\pgfpathrectangle{\pgfqpoint{0.481978in}{0.331635in}}{\pgfqpoint{4.960000in}{3.696000in}}%
\pgfusepath{clip}%
\pgfsetbuttcap%
\pgfsetroundjoin%
\definecolor{currentfill}{rgb}{0.631373,0.788235,0.956863}%
\pgfsetfillcolor{currentfill}%
\pgfsetlinewidth{0.481800pt}%
\definecolor{currentstroke}{rgb}{1.000000,1.000000,1.000000}%
\pgfsetstrokecolor{currentstroke}%
\pgfsetdash{}{0pt}%
\pgfpathmoveto{\pgfqpoint{1.834176in}{2.109509in}}%
\pgfpathcurveto{\pgfqpoint{1.845227in}{2.109509in}}{\pgfqpoint{1.855826in}{2.113899in}}{\pgfqpoint{1.863639in}{2.121713in}}%
\pgfpathcurveto{\pgfqpoint{1.871453in}{2.129527in}}{\pgfqpoint{1.875843in}{2.140126in}}{\pgfqpoint{1.875843in}{2.151176in}}%
\pgfpathcurveto{\pgfqpoint{1.875843in}{2.162226in}}{\pgfqpoint{1.871453in}{2.172825in}}{\pgfqpoint{1.863639in}{2.180639in}}%
\pgfpathcurveto{\pgfqpoint{1.855826in}{2.188452in}}{\pgfqpoint{1.845227in}{2.192842in}}{\pgfqpoint{1.834176in}{2.192842in}}%
\pgfpathcurveto{\pgfqpoint{1.823126in}{2.192842in}}{\pgfqpoint{1.812527in}{2.188452in}}{\pgfqpoint{1.804714in}{2.180639in}}%
\pgfpathcurveto{\pgfqpoint{1.796900in}{2.172825in}}{\pgfqpoint{1.792510in}{2.162226in}}{\pgfqpoint{1.792510in}{2.151176in}}%
\pgfpathcurveto{\pgfqpoint{1.792510in}{2.140126in}}{\pgfqpoint{1.796900in}{2.129527in}}{\pgfqpoint{1.804714in}{2.121713in}}%
\pgfpathcurveto{\pgfqpoint{1.812527in}{2.113899in}}{\pgfqpoint{1.823126in}{2.109509in}}{\pgfqpoint{1.834176in}{2.109509in}}%
\pgfpathclose%
\pgfusepath{stroke,fill}%
\end{pgfscope}%
\begin{pgfscope}%
\pgfpathrectangle{\pgfqpoint{0.481978in}{0.331635in}}{\pgfqpoint{4.960000in}{3.696000in}}%
\pgfusepath{clip}%
\pgfsetbuttcap%
\pgfsetroundjoin%
\definecolor{currentfill}{rgb}{0.631373,0.788235,0.956863}%
\pgfsetfillcolor{currentfill}%
\pgfsetlinewidth{0.481800pt}%
\definecolor{currentstroke}{rgb}{1.000000,1.000000,1.000000}%
\pgfsetstrokecolor{currentstroke}%
\pgfsetdash{}{0pt}%
\pgfpathmoveto{\pgfqpoint{1.866658in}{3.053476in}}%
\pgfpathcurveto{\pgfqpoint{1.877708in}{3.053476in}}{\pgfqpoint{1.888308in}{3.057866in}}{\pgfqpoint{1.896121in}{3.065680in}}%
\pgfpathcurveto{\pgfqpoint{1.903935in}{3.073493in}}{\pgfqpoint{1.908325in}{3.084092in}}{\pgfqpoint{1.908325in}{3.095143in}}%
\pgfpathcurveto{\pgfqpoint{1.908325in}{3.106193in}}{\pgfqpoint{1.903935in}{3.116792in}}{\pgfqpoint{1.896121in}{3.124605in}}%
\pgfpathcurveto{\pgfqpoint{1.888308in}{3.132419in}}{\pgfqpoint{1.877708in}{3.136809in}}{\pgfqpoint{1.866658in}{3.136809in}}%
\pgfpathcurveto{\pgfqpoint{1.855608in}{3.136809in}}{\pgfqpoint{1.845009in}{3.132419in}}{\pgfqpoint{1.837196in}{3.124605in}}%
\pgfpathcurveto{\pgfqpoint{1.829382in}{3.116792in}}{\pgfqpoint{1.824992in}{3.106193in}}{\pgfqpoint{1.824992in}{3.095143in}}%
\pgfpathcurveto{\pgfqpoint{1.824992in}{3.084092in}}{\pgfqpoint{1.829382in}{3.073493in}}{\pgfqpoint{1.837196in}{3.065680in}}%
\pgfpathcurveto{\pgfqpoint{1.845009in}{3.057866in}}{\pgfqpoint{1.855608in}{3.053476in}}{\pgfqpoint{1.866658in}{3.053476in}}%
\pgfpathclose%
\pgfusepath{stroke,fill}%
\end{pgfscope}%
\begin{pgfscope}%
\pgfpathrectangle{\pgfqpoint{0.481978in}{0.331635in}}{\pgfqpoint{4.960000in}{3.696000in}}%
\pgfusepath{clip}%
\pgfsetbuttcap%
\pgfsetroundjoin%
\definecolor{currentfill}{rgb}{0.631373,0.788235,0.956863}%
\pgfsetfillcolor{currentfill}%
\pgfsetlinewidth{0.481800pt}%
\definecolor{currentstroke}{rgb}{1.000000,1.000000,1.000000}%
\pgfsetstrokecolor{currentstroke}%
\pgfsetdash{}{0pt}%
\pgfpathmoveto{\pgfqpoint{2.039007in}{1.858524in}}%
\pgfpathcurveto{\pgfqpoint{2.050057in}{1.858524in}}{\pgfqpoint{2.060656in}{1.862915in}}{\pgfqpoint{2.068469in}{1.870728in}}%
\pgfpathcurveto{\pgfqpoint{2.076283in}{1.878542in}}{\pgfqpoint{2.080673in}{1.889141in}}{\pgfqpoint{2.080673in}{1.900191in}}%
\pgfpathcurveto{\pgfqpoint{2.080673in}{1.911241in}}{\pgfqpoint{2.076283in}{1.921840in}}{\pgfqpoint{2.068469in}{1.929654in}}%
\pgfpathcurveto{\pgfqpoint{2.060656in}{1.937467in}}{\pgfqpoint{2.050057in}{1.941858in}}{\pgfqpoint{2.039007in}{1.941858in}}%
\pgfpathcurveto{\pgfqpoint{2.027956in}{1.941858in}}{\pgfqpoint{2.017357in}{1.937467in}}{\pgfqpoint{2.009544in}{1.929654in}}%
\pgfpathcurveto{\pgfqpoint{2.001730in}{1.921840in}}{\pgfqpoint{1.997340in}{1.911241in}}{\pgfqpoint{1.997340in}{1.900191in}}%
\pgfpathcurveto{\pgfqpoint{1.997340in}{1.889141in}}{\pgfqpoint{2.001730in}{1.878542in}}{\pgfqpoint{2.009544in}{1.870728in}}%
\pgfpathcurveto{\pgfqpoint{2.017357in}{1.862915in}}{\pgfqpoint{2.027956in}{1.858524in}}{\pgfqpoint{2.039007in}{1.858524in}}%
\pgfpathclose%
\pgfusepath{stroke,fill}%
\end{pgfscope}%
\begin{pgfscope}%
\pgfpathrectangle{\pgfqpoint{0.481978in}{0.331635in}}{\pgfqpoint{4.960000in}{3.696000in}}%
\pgfusepath{clip}%
\pgfsetbuttcap%
\pgfsetroundjoin%
\definecolor{currentfill}{rgb}{0.631373,0.788235,0.956863}%
\pgfsetfillcolor{currentfill}%
\pgfsetlinewidth{0.481800pt}%
\definecolor{currentstroke}{rgb}{1.000000,1.000000,1.000000}%
\pgfsetstrokecolor{currentstroke}%
\pgfsetdash{}{0pt}%
\pgfpathmoveto{\pgfqpoint{2.111771in}{2.964856in}}%
\pgfpathcurveto{\pgfqpoint{2.122821in}{2.964856in}}{\pgfqpoint{2.133420in}{2.969246in}}{\pgfqpoint{2.141234in}{2.977060in}}%
\pgfpathcurveto{\pgfqpoint{2.149048in}{2.984873in}}{\pgfqpoint{2.153438in}{2.995472in}}{\pgfqpoint{2.153438in}{3.006523in}}%
\pgfpathcurveto{\pgfqpoint{2.153438in}{3.017573in}}{\pgfqpoint{2.149048in}{3.028172in}}{\pgfqpoint{2.141234in}{3.035985in}}%
\pgfpathcurveto{\pgfqpoint{2.133420in}{3.043799in}}{\pgfqpoint{2.122821in}{3.048189in}}{\pgfqpoint{2.111771in}{3.048189in}}%
\pgfpathcurveto{\pgfqpoint{2.100721in}{3.048189in}}{\pgfqpoint{2.090122in}{3.043799in}}{\pgfqpoint{2.082308in}{3.035985in}}%
\pgfpathcurveto{\pgfqpoint{2.074495in}{3.028172in}}{\pgfqpoint{2.070105in}{3.017573in}}{\pgfqpoint{2.070105in}{3.006523in}}%
\pgfpathcurveto{\pgfqpoint{2.070105in}{2.995472in}}{\pgfqpoint{2.074495in}{2.984873in}}{\pgfqpoint{2.082308in}{2.977060in}}%
\pgfpathcurveto{\pgfqpoint{2.090122in}{2.969246in}}{\pgfqpoint{2.100721in}{2.964856in}}{\pgfqpoint{2.111771in}{2.964856in}}%
\pgfpathclose%
\pgfusepath{stroke,fill}%
\end{pgfscope}%
\begin{pgfscope}%
\pgfpathrectangle{\pgfqpoint{0.481978in}{0.331635in}}{\pgfqpoint{4.960000in}{3.696000in}}%
\pgfusepath{clip}%
\pgfsetbuttcap%
\pgfsetroundjoin%
\definecolor{currentfill}{rgb}{0.631373,0.788235,0.956863}%
\pgfsetfillcolor{currentfill}%
\pgfsetlinewidth{0.481800pt}%
\definecolor{currentstroke}{rgb}{1.000000,1.000000,1.000000}%
\pgfsetstrokecolor{currentstroke}%
\pgfsetdash{}{0pt}%
\pgfpathmoveto{\pgfqpoint{2.007118in}{2.526727in}}%
\pgfpathcurveto{\pgfqpoint{2.018169in}{2.526727in}}{\pgfqpoint{2.028768in}{2.531117in}}{\pgfqpoint{2.036581in}{2.538931in}}%
\pgfpathcurveto{\pgfqpoint{2.044395in}{2.546744in}}{\pgfqpoint{2.048785in}{2.557343in}}{\pgfqpoint{2.048785in}{2.568393in}}%
\pgfpathcurveto{\pgfqpoint{2.048785in}{2.579444in}}{\pgfqpoint{2.044395in}{2.590043in}}{\pgfqpoint{2.036581in}{2.597856in}}%
\pgfpathcurveto{\pgfqpoint{2.028768in}{2.605670in}}{\pgfqpoint{2.018169in}{2.610060in}}{\pgfqpoint{2.007118in}{2.610060in}}%
\pgfpathcurveto{\pgfqpoint{1.996068in}{2.610060in}}{\pgfqpoint{1.985469in}{2.605670in}}{\pgfqpoint{1.977656in}{2.597856in}}%
\pgfpathcurveto{\pgfqpoint{1.969842in}{2.590043in}}{\pgfqpoint{1.965452in}{2.579444in}}{\pgfqpoint{1.965452in}{2.568393in}}%
\pgfpathcurveto{\pgfqpoint{1.965452in}{2.557343in}}{\pgfqpoint{1.969842in}{2.546744in}}{\pgfqpoint{1.977656in}{2.538931in}}%
\pgfpathcurveto{\pgfqpoint{1.985469in}{2.531117in}}{\pgfqpoint{1.996068in}{2.526727in}}{\pgfqpoint{2.007118in}{2.526727in}}%
\pgfpathclose%
\pgfusepath{stroke,fill}%
\end{pgfscope}%
\begin{pgfscope}%
\pgfpathrectangle{\pgfqpoint{0.481978in}{0.331635in}}{\pgfqpoint{4.960000in}{3.696000in}}%
\pgfusepath{clip}%
\pgfsetbuttcap%
\pgfsetroundjoin%
\definecolor{currentfill}{rgb}{0.631373,0.788235,0.956863}%
\pgfsetfillcolor{currentfill}%
\pgfsetlinewidth{0.481800pt}%
\definecolor{currentstroke}{rgb}{1.000000,1.000000,1.000000}%
\pgfsetstrokecolor{currentstroke}%
\pgfsetdash{}{0pt}%
\pgfpathmoveto{\pgfqpoint{2.777628in}{2.104041in}}%
\pgfpathcurveto{\pgfqpoint{2.788678in}{2.104041in}}{\pgfqpoint{2.799277in}{2.108432in}}{\pgfqpoint{2.807091in}{2.116245in}}%
\pgfpathcurveto{\pgfqpoint{2.814904in}{2.124059in}}{\pgfqpoint{2.819295in}{2.134658in}}{\pgfqpoint{2.819295in}{2.145708in}}%
\pgfpathcurveto{\pgfqpoint{2.819295in}{2.156758in}}{\pgfqpoint{2.814904in}{2.167357in}}{\pgfqpoint{2.807091in}{2.175171in}}%
\pgfpathcurveto{\pgfqpoint{2.799277in}{2.182984in}}{\pgfqpoint{2.788678in}{2.187375in}}{\pgfqpoint{2.777628in}{2.187375in}}%
\pgfpathcurveto{\pgfqpoint{2.766578in}{2.187375in}}{\pgfqpoint{2.755979in}{2.182984in}}{\pgfqpoint{2.748165in}{2.175171in}}%
\pgfpathcurveto{\pgfqpoint{2.740351in}{2.167357in}}{\pgfqpoint{2.735961in}{2.156758in}}{\pgfqpoint{2.735961in}{2.145708in}}%
\pgfpathcurveto{\pgfqpoint{2.735961in}{2.134658in}}{\pgfqpoint{2.740351in}{2.124059in}}{\pgfqpoint{2.748165in}{2.116245in}}%
\pgfpathcurveto{\pgfqpoint{2.755979in}{2.108432in}}{\pgfqpoint{2.766578in}{2.104041in}}{\pgfqpoint{2.777628in}{2.104041in}}%
\pgfpathclose%
\pgfusepath{stroke,fill}%
\end{pgfscope}%
\begin{pgfscope}%
\pgfpathrectangle{\pgfqpoint{0.481978in}{0.331635in}}{\pgfqpoint{4.960000in}{3.696000in}}%
\pgfusepath{clip}%
\pgfsetbuttcap%
\pgfsetroundjoin%
\definecolor{currentfill}{rgb}{0.631373,0.788235,0.956863}%
\pgfsetfillcolor{currentfill}%
\pgfsetlinewidth{0.481800pt}%
\definecolor{currentstroke}{rgb}{1.000000,1.000000,1.000000}%
\pgfsetstrokecolor{currentstroke}%
\pgfsetdash{}{0pt}%
\pgfpathmoveto{\pgfqpoint{4.038724in}{3.568411in}}%
\pgfpathcurveto{\pgfqpoint{4.049774in}{3.568411in}}{\pgfqpoint{4.060373in}{3.572801in}}{\pgfqpoint{4.068187in}{3.580615in}}%
\pgfpathcurveto{\pgfqpoint{4.076001in}{3.588429in}}{\pgfqpoint{4.080391in}{3.599028in}}{\pgfqpoint{4.080391in}{3.610078in}}%
\pgfpathcurveto{\pgfqpoint{4.080391in}{3.621128in}}{\pgfqpoint{4.076001in}{3.631727in}}{\pgfqpoint{4.068187in}{3.639541in}}%
\pgfpathcurveto{\pgfqpoint{4.060373in}{3.647354in}}{\pgfqpoint{4.049774in}{3.651744in}}{\pgfqpoint{4.038724in}{3.651744in}}%
\pgfpathcurveto{\pgfqpoint{4.027674in}{3.651744in}}{\pgfqpoint{4.017075in}{3.647354in}}{\pgfqpoint{4.009261in}{3.639541in}}%
\pgfpathcurveto{\pgfqpoint{4.001448in}{3.631727in}}{\pgfqpoint{3.997057in}{3.621128in}}{\pgfqpoint{3.997057in}{3.610078in}}%
\pgfpathcurveto{\pgfqpoint{3.997057in}{3.599028in}}{\pgfqpoint{4.001448in}{3.588429in}}{\pgfqpoint{4.009261in}{3.580615in}}%
\pgfpathcurveto{\pgfqpoint{4.017075in}{3.572801in}}{\pgfqpoint{4.027674in}{3.568411in}}{\pgfqpoint{4.038724in}{3.568411in}}%
\pgfpathclose%
\pgfusepath{stroke,fill}%
\end{pgfscope}%
\begin{pgfscope}%
\pgfpathrectangle{\pgfqpoint{0.481978in}{0.331635in}}{\pgfqpoint{4.960000in}{3.696000in}}%
\pgfusepath{clip}%
\pgfsetbuttcap%
\pgfsetroundjoin%
\definecolor{currentfill}{rgb}{0.631373,0.788235,0.956863}%
\pgfsetfillcolor{currentfill}%
\pgfsetlinewidth{0.481800pt}%
\definecolor{currentstroke}{rgb}{1.000000,1.000000,1.000000}%
\pgfsetstrokecolor{currentstroke}%
\pgfsetdash{}{0pt}%
\pgfpathmoveto{\pgfqpoint{2.255659in}{3.085941in}}%
\pgfpathcurveto{\pgfqpoint{2.266709in}{3.085941in}}{\pgfqpoint{2.277308in}{3.090332in}}{\pgfqpoint{2.285121in}{3.098145in}}%
\pgfpathcurveto{\pgfqpoint{2.292935in}{3.105959in}}{\pgfqpoint{2.297325in}{3.116558in}}{\pgfqpoint{2.297325in}{3.127608in}}%
\pgfpathcurveto{\pgfqpoint{2.297325in}{3.138658in}}{\pgfqpoint{2.292935in}{3.149257in}}{\pgfqpoint{2.285121in}{3.157071in}}%
\pgfpathcurveto{\pgfqpoint{2.277308in}{3.164884in}}{\pgfqpoint{2.266709in}{3.169275in}}{\pgfqpoint{2.255659in}{3.169275in}}%
\pgfpathcurveto{\pgfqpoint{2.244608in}{3.169275in}}{\pgfqpoint{2.234009in}{3.164884in}}{\pgfqpoint{2.226196in}{3.157071in}}%
\pgfpathcurveto{\pgfqpoint{2.218382in}{3.149257in}}{\pgfqpoint{2.213992in}{3.138658in}}{\pgfqpoint{2.213992in}{3.127608in}}%
\pgfpathcurveto{\pgfqpoint{2.213992in}{3.116558in}}{\pgfqpoint{2.218382in}{3.105959in}}{\pgfqpoint{2.226196in}{3.098145in}}%
\pgfpathcurveto{\pgfqpoint{2.234009in}{3.090332in}}{\pgfqpoint{2.244608in}{3.085941in}}{\pgfqpoint{2.255659in}{3.085941in}}%
\pgfpathclose%
\pgfusepath{stroke,fill}%
\end{pgfscope}%
\begin{pgfscope}%
\pgfpathrectangle{\pgfqpoint{0.481978in}{0.331635in}}{\pgfqpoint{4.960000in}{3.696000in}}%
\pgfusepath{clip}%
\pgfsetbuttcap%
\pgfsetroundjoin%
\definecolor{currentfill}{rgb}{0.631373,0.788235,0.956863}%
\pgfsetfillcolor{currentfill}%
\pgfsetlinewidth{0.481800pt}%
\definecolor{currentstroke}{rgb}{1.000000,1.000000,1.000000}%
\pgfsetstrokecolor{currentstroke}%
\pgfsetdash{}{0pt}%
\pgfpathmoveto{\pgfqpoint{4.419131in}{1.990045in}}%
\pgfpathcurveto{\pgfqpoint{4.430181in}{1.990045in}}{\pgfqpoint{4.440780in}{1.994435in}}{\pgfqpoint{4.448594in}{2.002248in}}%
\pgfpathcurveto{\pgfqpoint{4.456407in}{2.010062in}}{\pgfqpoint{4.460797in}{2.020661in}}{\pgfqpoint{4.460797in}{2.031711in}}%
\pgfpathcurveto{\pgfqpoint{4.460797in}{2.042761in}}{\pgfqpoint{4.456407in}{2.053360in}}{\pgfqpoint{4.448594in}{2.061174in}}%
\pgfpathcurveto{\pgfqpoint{4.440780in}{2.068988in}}{\pgfqpoint{4.430181in}{2.073378in}}{\pgfqpoint{4.419131in}{2.073378in}}%
\pgfpathcurveto{\pgfqpoint{4.408081in}{2.073378in}}{\pgfqpoint{4.397482in}{2.068988in}}{\pgfqpoint{4.389668in}{2.061174in}}%
\pgfpathcurveto{\pgfqpoint{4.381854in}{2.053360in}}{\pgfqpoint{4.377464in}{2.042761in}}{\pgfqpoint{4.377464in}{2.031711in}}%
\pgfpathcurveto{\pgfqpoint{4.377464in}{2.020661in}}{\pgfqpoint{4.381854in}{2.010062in}}{\pgfqpoint{4.389668in}{2.002248in}}%
\pgfpathcurveto{\pgfqpoint{4.397482in}{1.994435in}}{\pgfqpoint{4.408081in}{1.990045in}}{\pgfqpoint{4.419131in}{1.990045in}}%
\pgfpathclose%
\pgfusepath{stroke,fill}%
\end{pgfscope}%
\begin{pgfscope}%
\pgfpathrectangle{\pgfqpoint{0.481978in}{0.331635in}}{\pgfqpoint{4.960000in}{3.696000in}}%
\pgfusepath{clip}%
\pgfsetbuttcap%
\pgfsetroundjoin%
\definecolor{currentfill}{rgb}{0.631373,0.788235,0.956863}%
\pgfsetfillcolor{currentfill}%
\pgfsetlinewidth{0.481800pt}%
\definecolor{currentstroke}{rgb}{1.000000,1.000000,1.000000}%
\pgfsetstrokecolor{currentstroke}%
\pgfsetdash{}{0pt}%
\pgfpathmoveto{\pgfqpoint{2.577424in}{2.049683in}}%
\pgfpathcurveto{\pgfqpoint{2.588475in}{2.049683in}}{\pgfqpoint{2.599074in}{2.054074in}}{\pgfqpoint{2.606887in}{2.061887in}}%
\pgfpathcurveto{\pgfqpoint{2.614701in}{2.069701in}}{\pgfqpoint{2.619091in}{2.080300in}}{\pgfqpoint{2.619091in}{2.091350in}}%
\pgfpathcurveto{\pgfqpoint{2.619091in}{2.102400in}}{\pgfqpoint{2.614701in}{2.112999in}}{\pgfqpoint{2.606887in}{2.120813in}}%
\pgfpathcurveto{\pgfqpoint{2.599074in}{2.128626in}}{\pgfqpoint{2.588475in}{2.133017in}}{\pgfqpoint{2.577424in}{2.133017in}}%
\pgfpathcurveto{\pgfqpoint{2.566374in}{2.133017in}}{\pgfqpoint{2.555775in}{2.128626in}}{\pgfqpoint{2.547962in}{2.120813in}}%
\pgfpathcurveto{\pgfqpoint{2.540148in}{2.112999in}}{\pgfqpoint{2.535758in}{2.102400in}}{\pgfqpoint{2.535758in}{2.091350in}}%
\pgfpathcurveto{\pgfqpoint{2.535758in}{2.080300in}}{\pgfqpoint{2.540148in}{2.069701in}}{\pgfqpoint{2.547962in}{2.061887in}}%
\pgfpathcurveto{\pgfqpoint{2.555775in}{2.054074in}}{\pgfqpoint{2.566374in}{2.049683in}}{\pgfqpoint{2.577424in}{2.049683in}}%
\pgfpathclose%
\pgfusepath{stroke,fill}%
\end{pgfscope}%
\begin{pgfscope}%
\pgfpathrectangle{\pgfqpoint{0.481978in}{0.331635in}}{\pgfqpoint{4.960000in}{3.696000in}}%
\pgfusepath{clip}%
\pgfsetbuttcap%
\pgfsetroundjoin%
\definecolor{currentfill}{rgb}{0.631373,0.788235,0.956863}%
\pgfsetfillcolor{currentfill}%
\pgfsetlinewidth{0.481800pt}%
\definecolor{currentstroke}{rgb}{1.000000,1.000000,1.000000}%
\pgfsetstrokecolor{currentstroke}%
\pgfsetdash{}{0pt}%
\pgfpathmoveto{\pgfqpoint{4.214629in}{1.516429in}}%
\pgfpathcurveto{\pgfqpoint{4.225680in}{1.516429in}}{\pgfqpoint{4.236279in}{1.520819in}}{\pgfqpoint{4.244092in}{1.528633in}}%
\pgfpathcurveto{\pgfqpoint{4.251906in}{1.536447in}}{\pgfqpoint{4.256296in}{1.547046in}}{\pgfqpoint{4.256296in}{1.558096in}}%
\pgfpathcurveto{\pgfqpoint{4.256296in}{1.569146in}}{\pgfqpoint{4.251906in}{1.579745in}}{\pgfqpoint{4.244092in}{1.587559in}}%
\pgfpathcurveto{\pgfqpoint{4.236279in}{1.595372in}}{\pgfqpoint{4.225680in}{1.599762in}}{\pgfqpoint{4.214629in}{1.599762in}}%
\pgfpathcurveto{\pgfqpoint{4.203579in}{1.599762in}}{\pgfqpoint{4.192980in}{1.595372in}}{\pgfqpoint{4.185167in}{1.587559in}}%
\pgfpathcurveto{\pgfqpoint{4.177353in}{1.579745in}}{\pgfqpoint{4.172963in}{1.569146in}}{\pgfqpoint{4.172963in}{1.558096in}}%
\pgfpathcurveto{\pgfqpoint{4.172963in}{1.547046in}}{\pgfqpoint{4.177353in}{1.536447in}}{\pgfqpoint{4.185167in}{1.528633in}}%
\pgfpathcurveto{\pgfqpoint{4.192980in}{1.520819in}}{\pgfqpoint{4.203579in}{1.516429in}}{\pgfqpoint{4.214629in}{1.516429in}}%
\pgfpathclose%
\pgfusepath{stroke,fill}%
\end{pgfscope}%
\begin{pgfscope}%
\pgfpathrectangle{\pgfqpoint{0.481978in}{0.331635in}}{\pgfqpoint{4.960000in}{3.696000in}}%
\pgfusepath{clip}%
\pgfsetbuttcap%
\pgfsetroundjoin%
\definecolor{currentfill}{rgb}{0.631373,0.788235,0.956863}%
\pgfsetfillcolor{currentfill}%
\pgfsetlinewidth{0.481800pt}%
\definecolor{currentstroke}{rgb}{1.000000,1.000000,1.000000}%
\pgfsetstrokecolor{currentstroke}%
\pgfsetdash{}{0pt}%
\pgfpathmoveto{\pgfqpoint{1.520885in}{2.212527in}}%
\pgfpathcurveto{\pgfqpoint{1.531935in}{2.212527in}}{\pgfqpoint{1.542534in}{2.216917in}}{\pgfqpoint{1.550347in}{2.224731in}}%
\pgfpathcurveto{\pgfqpoint{1.558161in}{2.232544in}}{\pgfqpoint{1.562551in}{2.243143in}}{\pgfqpoint{1.562551in}{2.254193in}}%
\pgfpathcurveto{\pgfqpoint{1.562551in}{2.265243in}}{\pgfqpoint{1.558161in}{2.275842in}}{\pgfqpoint{1.550347in}{2.283656in}}%
\pgfpathcurveto{\pgfqpoint{1.542534in}{2.291470in}}{\pgfqpoint{1.531935in}{2.295860in}}{\pgfqpoint{1.520885in}{2.295860in}}%
\pgfpathcurveto{\pgfqpoint{1.509834in}{2.295860in}}{\pgfqpoint{1.499235in}{2.291470in}}{\pgfqpoint{1.491422in}{2.283656in}}%
\pgfpathcurveto{\pgfqpoint{1.483608in}{2.275842in}}{\pgfqpoint{1.479218in}{2.265243in}}{\pgfqpoint{1.479218in}{2.254193in}}%
\pgfpathcurveto{\pgfqpoint{1.479218in}{2.243143in}}{\pgfqpoint{1.483608in}{2.232544in}}{\pgfqpoint{1.491422in}{2.224731in}}%
\pgfpathcurveto{\pgfqpoint{1.499235in}{2.216917in}}{\pgfqpoint{1.509834in}{2.212527in}}{\pgfqpoint{1.520885in}{2.212527in}}%
\pgfpathclose%
\pgfusepath{stroke,fill}%
\end{pgfscope}%
\begin{pgfscope}%
\pgfpathrectangle{\pgfqpoint{0.481978in}{0.331635in}}{\pgfqpoint{4.960000in}{3.696000in}}%
\pgfusepath{clip}%
\pgfsetbuttcap%
\pgfsetroundjoin%
\definecolor{currentfill}{rgb}{0.631373,0.788235,0.956863}%
\pgfsetfillcolor{currentfill}%
\pgfsetlinewidth{0.481800pt}%
\definecolor{currentstroke}{rgb}{1.000000,1.000000,1.000000}%
\pgfsetstrokecolor{currentstroke}%
\pgfsetdash{}{0pt}%
\pgfpathmoveto{\pgfqpoint{2.504302in}{1.470920in}}%
\pgfpathcurveto{\pgfqpoint{2.515352in}{1.470920in}}{\pgfqpoint{2.525951in}{1.475310in}}{\pgfqpoint{2.533764in}{1.483124in}}%
\pgfpathcurveto{\pgfqpoint{2.541578in}{1.490938in}}{\pgfqpoint{2.545968in}{1.501537in}}{\pgfqpoint{2.545968in}{1.512587in}}%
\pgfpathcurveto{\pgfqpoint{2.545968in}{1.523637in}}{\pgfqpoint{2.541578in}{1.534236in}}{\pgfqpoint{2.533764in}{1.542049in}}%
\pgfpathcurveto{\pgfqpoint{2.525951in}{1.549863in}}{\pgfqpoint{2.515352in}{1.554253in}}{\pgfqpoint{2.504302in}{1.554253in}}%
\pgfpathcurveto{\pgfqpoint{2.493251in}{1.554253in}}{\pgfqpoint{2.482652in}{1.549863in}}{\pgfqpoint{2.474839in}{1.542049in}}%
\pgfpathcurveto{\pgfqpoint{2.467025in}{1.534236in}}{\pgfqpoint{2.462635in}{1.523637in}}{\pgfqpoint{2.462635in}{1.512587in}}%
\pgfpathcurveto{\pgfqpoint{2.462635in}{1.501537in}}{\pgfqpoint{2.467025in}{1.490938in}}{\pgfqpoint{2.474839in}{1.483124in}}%
\pgfpathcurveto{\pgfqpoint{2.482652in}{1.475310in}}{\pgfqpoint{2.493251in}{1.470920in}}{\pgfqpoint{2.504302in}{1.470920in}}%
\pgfpathclose%
\pgfusepath{stroke,fill}%
\end{pgfscope}%
\begin{pgfscope}%
\pgfpathrectangle{\pgfqpoint{0.481978in}{0.331635in}}{\pgfqpoint{4.960000in}{3.696000in}}%
\pgfusepath{clip}%
\pgfsetbuttcap%
\pgfsetroundjoin%
\definecolor{currentfill}{rgb}{0.631373,0.788235,0.956863}%
\pgfsetfillcolor{currentfill}%
\pgfsetlinewidth{0.481800pt}%
\definecolor{currentstroke}{rgb}{1.000000,1.000000,1.000000}%
\pgfsetstrokecolor{currentstroke}%
\pgfsetdash{}{0pt}%
\pgfpathmoveto{\pgfqpoint{1.834100in}{3.166033in}}%
\pgfpathcurveto{\pgfqpoint{1.845150in}{3.166033in}}{\pgfqpoint{1.855749in}{3.170423in}}{\pgfqpoint{1.863563in}{3.178237in}}%
\pgfpathcurveto{\pgfqpoint{1.871377in}{3.186050in}}{\pgfqpoint{1.875767in}{3.196649in}}{\pgfqpoint{1.875767in}{3.207700in}}%
\pgfpathcurveto{\pgfqpoint{1.875767in}{3.218750in}}{\pgfqpoint{1.871377in}{3.229349in}}{\pgfqpoint{1.863563in}{3.237162in}}%
\pgfpathcurveto{\pgfqpoint{1.855749in}{3.244976in}}{\pgfqpoint{1.845150in}{3.249366in}}{\pgfqpoint{1.834100in}{3.249366in}}%
\pgfpathcurveto{\pgfqpoint{1.823050in}{3.249366in}}{\pgfqpoint{1.812451in}{3.244976in}}{\pgfqpoint{1.804638in}{3.237162in}}%
\pgfpathcurveto{\pgfqpoint{1.796824in}{3.229349in}}{\pgfqpoint{1.792434in}{3.218750in}}{\pgfqpoint{1.792434in}{3.207700in}}%
\pgfpathcurveto{\pgfqpoint{1.792434in}{3.196649in}}{\pgfqpoint{1.796824in}{3.186050in}}{\pgfqpoint{1.804638in}{3.178237in}}%
\pgfpathcurveto{\pgfqpoint{1.812451in}{3.170423in}}{\pgfqpoint{1.823050in}{3.166033in}}{\pgfqpoint{1.834100in}{3.166033in}}%
\pgfpathclose%
\pgfusepath{stroke,fill}%
\end{pgfscope}%
\begin{pgfscope}%
\pgfpathrectangle{\pgfqpoint{0.481978in}{0.331635in}}{\pgfqpoint{4.960000in}{3.696000in}}%
\pgfusepath{clip}%
\pgfsetbuttcap%
\pgfsetroundjoin%
\definecolor{currentfill}{rgb}{0.631373,0.788235,0.956863}%
\pgfsetfillcolor{currentfill}%
\pgfsetlinewidth{0.481800pt}%
\definecolor{currentstroke}{rgb}{1.000000,1.000000,1.000000}%
\pgfsetstrokecolor{currentstroke}%
\pgfsetdash{}{0pt}%
\pgfpathmoveto{\pgfqpoint{2.129996in}{1.933933in}}%
\pgfpathcurveto{\pgfqpoint{2.141046in}{1.933933in}}{\pgfqpoint{2.151645in}{1.938323in}}{\pgfqpoint{2.159458in}{1.946137in}}%
\pgfpathcurveto{\pgfqpoint{2.167272in}{1.953950in}}{\pgfqpoint{2.171662in}{1.964549in}}{\pgfqpoint{2.171662in}{1.975599in}}%
\pgfpathcurveto{\pgfqpoint{2.171662in}{1.986650in}}{\pgfqpoint{2.167272in}{1.997249in}}{\pgfqpoint{2.159458in}{2.005062in}}%
\pgfpathcurveto{\pgfqpoint{2.151645in}{2.012876in}}{\pgfqpoint{2.141046in}{2.017266in}}{\pgfqpoint{2.129996in}{2.017266in}}%
\pgfpathcurveto{\pgfqpoint{2.118945in}{2.017266in}}{\pgfqpoint{2.108346in}{2.012876in}}{\pgfqpoint{2.100533in}{2.005062in}}%
\pgfpathcurveto{\pgfqpoint{2.092719in}{1.997249in}}{\pgfqpoint{2.088329in}{1.986650in}}{\pgfqpoint{2.088329in}{1.975599in}}%
\pgfpathcurveto{\pgfqpoint{2.088329in}{1.964549in}}{\pgfqpoint{2.092719in}{1.953950in}}{\pgfqpoint{2.100533in}{1.946137in}}%
\pgfpathcurveto{\pgfqpoint{2.108346in}{1.938323in}}{\pgfqpoint{2.118945in}{1.933933in}}{\pgfqpoint{2.129996in}{1.933933in}}%
\pgfpathclose%
\pgfusepath{stroke,fill}%
\end{pgfscope}%
\begin{pgfscope}%
\pgfpathrectangle{\pgfqpoint{0.481978in}{0.331635in}}{\pgfqpoint{4.960000in}{3.696000in}}%
\pgfusepath{clip}%
\pgfsetbuttcap%
\pgfsetroundjoin%
\definecolor{currentfill}{rgb}{0.631373,0.788235,0.956863}%
\pgfsetfillcolor{currentfill}%
\pgfsetlinewidth{0.481800pt}%
\definecolor{currentstroke}{rgb}{1.000000,1.000000,1.000000}%
\pgfsetstrokecolor{currentstroke}%
\pgfsetdash{}{0pt}%
\pgfpathmoveto{\pgfqpoint{1.577829in}{2.850830in}}%
\pgfpathcurveto{\pgfqpoint{1.588879in}{2.850830in}}{\pgfqpoint{1.599478in}{2.855220in}}{\pgfqpoint{1.607292in}{2.863034in}}%
\pgfpathcurveto{\pgfqpoint{1.615105in}{2.870848in}}{\pgfqpoint{1.619496in}{2.881447in}}{\pgfqpoint{1.619496in}{2.892497in}}%
\pgfpathcurveto{\pgfqpoint{1.619496in}{2.903547in}}{\pgfqpoint{1.615105in}{2.914146in}}{\pgfqpoint{1.607292in}{2.921959in}}%
\pgfpathcurveto{\pgfqpoint{1.599478in}{2.929773in}}{\pgfqpoint{1.588879in}{2.934163in}}{\pgfqpoint{1.577829in}{2.934163in}}%
\pgfpathcurveto{\pgfqpoint{1.566779in}{2.934163in}}{\pgfqpoint{1.556180in}{2.929773in}}{\pgfqpoint{1.548366in}{2.921959in}}%
\pgfpathcurveto{\pgfqpoint{1.540553in}{2.914146in}}{\pgfqpoint{1.536162in}{2.903547in}}{\pgfqpoint{1.536162in}{2.892497in}}%
\pgfpathcurveto{\pgfqpoint{1.536162in}{2.881447in}}{\pgfqpoint{1.540553in}{2.870848in}}{\pgfqpoint{1.548366in}{2.863034in}}%
\pgfpathcurveto{\pgfqpoint{1.556180in}{2.855220in}}{\pgfqpoint{1.566779in}{2.850830in}}{\pgfqpoint{1.577829in}{2.850830in}}%
\pgfpathclose%
\pgfusepath{stroke,fill}%
\end{pgfscope}%
\begin{pgfscope}%
\pgfpathrectangle{\pgfqpoint{0.481978in}{0.331635in}}{\pgfqpoint{4.960000in}{3.696000in}}%
\pgfusepath{clip}%
\pgfsetbuttcap%
\pgfsetroundjoin%
\definecolor{currentfill}{rgb}{0.631373,0.788235,0.956863}%
\pgfsetfillcolor{currentfill}%
\pgfsetlinewidth{0.481800pt}%
\definecolor{currentstroke}{rgb}{1.000000,1.000000,1.000000}%
\pgfsetstrokecolor{currentstroke}%
\pgfsetdash{}{0pt}%
\pgfpathmoveto{\pgfqpoint{2.000204in}{2.613808in}}%
\pgfpathcurveto{\pgfqpoint{2.011254in}{2.613808in}}{\pgfqpoint{2.021853in}{2.618198in}}{\pgfqpoint{2.029666in}{2.626012in}}%
\pgfpathcurveto{\pgfqpoint{2.037480in}{2.633826in}}{\pgfqpoint{2.041870in}{2.644425in}}{\pgfqpoint{2.041870in}{2.655475in}}%
\pgfpathcurveto{\pgfqpoint{2.041870in}{2.666525in}}{\pgfqpoint{2.037480in}{2.677124in}}{\pgfqpoint{2.029666in}{2.684938in}}%
\pgfpathcurveto{\pgfqpoint{2.021853in}{2.692751in}}{\pgfqpoint{2.011254in}{2.697141in}}{\pgfqpoint{2.000204in}{2.697141in}}%
\pgfpathcurveto{\pgfqpoint{1.989153in}{2.697141in}}{\pgfqpoint{1.978554in}{2.692751in}}{\pgfqpoint{1.970741in}{2.684938in}}%
\pgfpathcurveto{\pgfqpoint{1.962927in}{2.677124in}}{\pgfqpoint{1.958537in}{2.666525in}}{\pgfqpoint{1.958537in}{2.655475in}}%
\pgfpathcurveto{\pgfqpoint{1.958537in}{2.644425in}}{\pgfqpoint{1.962927in}{2.633826in}}{\pgfqpoint{1.970741in}{2.626012in}}%
\pgfpathcurveto{\pgfqpoint{1.978554in}{2.618198in}}{\pgfqpoint{1.989153in}{2.613808in}}{\pgfqpoint{2.000204in}{2.613808in}}%
\pgfpathclose%
\pgfusepath{stroke,fill}%
\end{pgfscope}%
\begin{pgfscope}%
\pgfpathrectangle{\pgfqpoint{0.481978in}{0.331635in}}{\pgfqpoint{4.960000in}{3.696000in}}%
\pgfusepath{clip}%
\pgfsetbuttcap%
\pgfsetroundjoin%
\definecolor{currentfill}{rgb}{0.631373,0.788235,0.956863}%
\pgfsetfillcolor{currentfill}%
\pgfsetlinewidth{0.481800pt}%
\definecolor{currentstroke}{rgb}{1.000000,1.000000,1.000000}%
\pgfsetstrokecolor{currentstroke}%
\pgfsetdash{}{0pt}%
\pgfpathmoveto{\pgfqpoint{2.727209in}{1.689443in}}%
\pgfpathcurveto{\pgfqpoint{2.738259in}{1.689443in}}{\pgfqpoint{2.748858in}{1.693833in}}{\pgfqpoint{2.756671in}{1.701646in}}%
\pgfpathcurveto{\pgfqpoint{2.764485in}{1.709460in}}{\pgfqpoint{2.768875in}{1.720059in}}{\pgfqpoint{2.768875in}{1.731109in}}%
\pgfpathcurveto{\pgfqpoint{2.768875in}{1.742159in}}{\pgfqpoint{2.764485in}{1.752758in}}{\pgfqpoint{2.756671in}{1.760572in}}%
\pgfpathcurveto{\pgfqpoint{2.748858in}{1.768386in}}{\pgfqpoint{2.738259in}{1.772776in}}{\pgfqpoint{2.727209in}{1.772776in}}%
\pgfpathcurveto{\pgfqpoint{2.716158in}{1.772776in}}{\pgfqpoint{2.705559in}{1.768386in}}{\pgfqpoint{2.697746in}{1.760572in}}%
\pgfpathcurveto{\pgfqpoint{2.689932in}{1.752758in}}{\pgfqpoint{2.685542in}{1.742159in}}{\pgfqpoint{2.685542in}{1.731109in}}%
\pgfpathcurveto{\pgfqpoint{2.685542in}{1.720059in}}{\pgfqpoint{2.689932in}{1.709460in}}{\pgfqpoint{2.697746in}{1.701646in}}%
\pgfpathcurveto{\pgfqpoint{2.705559in}{1.693833in}}{\pgfqpoint{2.716158in}{1.689443in}}{\pgfqpoint{2.727209in}{1.689443in}}%
\pgfpathclose%
\pgfusepath{stroke,fill}%
\end{pgfscope}%
\begin{pgfscope}%
\pgfpathrectangle{\pgfqpoint{0.481978in}{0.331635in}}{\pgfqpoint{4.960000in}{3.696000in}}%
\pgfusepath{clip}%
\pgfsetbuttcap%
\pgfsetroundjoin%
\definecolor{currentfill}{rgb}{0.631373,0.788235,0.956863}%
\pgfsetfillcolor{currentfill}%
\pgfsetlinewidth{0.481800pt}%
\definecolor{currentstroke}{rgb}{1.000000,1.000000,1.000000}%
\pgfsetstrokecolor{currentstroke}%
\pgfsetdash{}{0pt}%
\pgfpathmoveto{\pgfqpoint{1.087825in}{1.334177in}}%
\pgfpathcurveto{\pgfqpoint{1.098875in}{1.334177in}}{\pgfqpoint{1.109474in}{1.338567in}}{\pgfqpoint{1.117287in}{1.346380in}}%
\pgfpathcurveto{\pgfqpoint{1.125101in}{1.354194in}}{\pgfqpoint{1.129491in}{1.364793in}}{\pgfqpoint{1.129491in}{1.375843in}}%
\pgfpathcurveto{\pgfqpoint{1.129491in}{1.386893in}}{\pgfqpoint{1.125101in}{1.397492in}}{\pgfqpoint{1.117287in}{1.405306in}}%
\pgfpathcurveto{\pgfqpoint{1.109474in}{1.413120in}}{\pgfqpoint{1.098875in}{1.417510in}}{\pgfqpoint{1.087825in}{1.417510in}}%
\pgfpathcurveto{\pgfqpoint{1.076775in}{1.417510in}}{\pgfqpoint{1.066175in}{1.413120in}}{\pgfqpoint{1.058362in}{1.405306in}}%
\pgfpathcurveto{\pgfqpoint{1.050548in}{1.397492in}}{\pgfqpoint{1.046158in}{1.386893in}}{\pgfqpoint{1.046158in}{1.375843in}}%
\pgfpathcurveto{\pgfqpoint{1.046158in}{1.364793in}}{\pgfqpoint{1.050548in}{1.354194in}}{\pgfqpoint{1.058362in}{1.346380in}}%
\pgfpathcurveto{\pgfqpoint{1.066175in}{1.338567in}}{\pgfqpoint{1.076775in}{1.334177in}}{\pgfqpoint{1.087825in}{1.334177in}}%
\pgfpathclose%
\pgfusepath{stroke,fill}%
\end{pgfscope}%
\begin{pgfscope}%
\pgfpathrectangle{\pgfqpoint{0.481978in}{0.331635in}}{\pgfqpoint{4.960000in}{3.696000in}}%
\pgfusepath{clip}%
\pgfsetbuttcap%
\pgfsetroundjoin%
\definecolor{currentfill}{rgb}{0.631373,0.788235,0.956863}%
\pgfsetfillcolor{currentfill}%
\pgfsetlinewidth{0.481800pt}%
\definecolor{currentstroke}{rgb}{1.000000,1.000000,1.000000}%
\pgfsetstrokecolor{currentstroke}%
\pgfsetdash{}{0pt}%
\pgfpathmoveto{\pgfqpoint{2.974204in}{2.951308in}}%
\pgfpathcurveto{\pgfqpoint{2.985254in}{2.951308in}}{\pgfqpoint{2.995853in}{2.955699in}}{\pgfqpoint{3.003666in}{2.963512in}}%
\pgfpathcurveto{\pgfqpoint{3.011480in}{2.971326in}}{\pgfqpoint{3.015870in}{2.981925in}}{\pgfqpoint{3.015870in}{2.992975in}}%
\pgfpathcurveto{\pgfqpoint{3.015870in}{3.004025in}}{\pgfqpoint{3.011480in}{3.014624in}}{\pgfqpoint{3.003666in}{3.022438in}}%
\pgfpathcurveto{\pgfqpoint{2.995853in}{3.030252in}}{\pgfqpoint{2.985254in}{3.034642in}}{\pgfqpoint{2.974204in}{3.034642in}}%
\pgfpathcurveto{\pgfqpoint{2.963153in}{3.034642in}}{\pgfqpoint{2.952554in}{3.030252in}}{\pgfqpoint{2.944741in}{3.022438in}}%
\pgfpathcurveto{\pgfqpoint{2.936927in}{3.014624in}}{\pgfqpoint{2.932537in}{3.004025in}}{\pgfqpoint{2.932537in}{2.992975in}}%
\pgfpathcurveto{\pgfqpoint{2.932537in}{2.981925in}}{\pgfqpoint{2.936927in}{2.971326in}}{\pgfqpoint{2.944741in}{2.963512in}}%
\pgfpathcurveto{\pgfqpoint{2.952554in}{2.955699in}}{\pgfqpoint{2.963153in}{2.951308in}}{\pgfqpoint{2.974204in}{2.951308in}}%
\pgfpathclose%
\pgfusepath{stroke,fill}%
\end{pgfscope}%
\begin{pgfscope}%
\pgfpathrectangle{\pgfqpoint{0.481978in}{0.331635in}}{\pgfqpoint{4.960000in}{3.696000in}}%
\pgfusepath{clip}%
\pgfsetbuttcap%
\pgfsetroundjoin%
\definecolor{currentfill}{rgb}{0.631373,0.788235,0.956863}%
\pgfsetfillcolor{currentfill}%
\pgfsetlinewidth{0.481800pt}%
\definecolor{currentstroke}{rgb}{1.000000,1.000000,1.000000}%
\pgfsetstrokecolor{currentstroke}%
\pgfsetdash{}{0pt}%
\pgfpathmoveto{\pgfqpoint{1.447239in}{2.768138in}}%
\pgfpathcurveto{\pgfqpoint{1.458289in}{2.768138in}}{\pgfqpoint{1.468888in}{2.772528in}}{\pgfqpoint{1.476701in}{2.780342in}}%
\pgfpathcurveto{\pgfqpoint{1.484515in}{2.788155in}}{\pgfqpoint{1.488905in}{2.798754in}}{\pgfqpoint{1.488905in}{2.809804in}}%
\pgfpathcurveto{\pgfqpoint{1.488905in}{2.820854in}}{\pgfqpoint{1.484515in}{2.831453in}}{\pgfqpoint{1.476701in}{2.839267in}}%
\pgfpathcurveto{\pgfqpoint{1.468888in}{2.847081in}}{\pgfqpoint{1.458289in}{2.851471in}}{\pgfqpoint{1.447239in}{2.851471in}}%
\pgfpathcurveto{\pgfqpoint{1.436188in}{2.851471in}}{\pgfqpoint{1.425589in}{2.847081in}}{\pgfqpoint{1.417776in}{2.839267in}}%
\pgfpathcurveto{\pgfqpoint{1.409962in}{2.831453in}}{\pgfqpoint{1.405572in}{2.820854in}}{\pgfqpoint{1.405572in}{2.809804in}}%
\pgfpathcurveto{\pgfqpoint{1.405572in}{2.798754in}}{\pgfqpoint{1.409962in}{2.788155in}}{\pgfqpoint{1.417776in}{2.780342in}}%
\pgfpathcurveto{\pgfqpoint{1.425589in}{2.772528in}}{\pgfqpoint{1.436188in}{2.768138in}}{\pgfqpoint{1.447239in}{2.768138in}}%
\pgfpathclose%
\pgfusepath{stroke,fill}%
\end{pgfscope}%
\begin{pgfscope}%
\pgfpathrectangle{\pgfqpoint{0.481978in}{0.331635in}}{\pgfqpoint{4.960000in}{3.696000in}}%
\pgfusepath{clip}%
\pgfsetbuttcap%
\pgfsetroundjoin%
\definecolor{currentfill}{rgb}{0.631373,0.788235,0.956863}%
\pgfsetfillcolor{currentfill}%
\pgfsetlinewidth{0.481800pt}%
\definecolor{currentstroke}{rgb}{1.000000,1.000000,1.000000}%
\pgfsetstrokecolor{currentstroke}%
\pgfsetdash{}{0pt}%
\pgfpathmoveto{\pgfqpoint{1.582685in}{3.330027in}}%
\pgfpathcurveto{\pgfqpoint{1.593735in}{3.330027in}}{\pgfqpoint{1.604334in}{3.334418in}}{\pgfqpoint{1.612148in}{3.342231in}}%
\pgfpathcurveto{\pgfqpoint{1.619961in}{3.350045in}}{\pgfqpoint{1.624352in}{3.360644in}}{\pgfqpoint{1.624352in}{3.371694in}}%
\pgfpathcurveto{\pgfqpoint{1.624352in}{3.382744in}}{\pgfqpoint{1.619961in}{3.393343in}}{\pgfqpoint{1.612148in}{3.401157in}}%
\pgfpathcurveto{\pgfqpoint{1.604334in}{3.408970in}}{\pgfqpoint{1.593735in}{3.413361in}}{\pgfqpoint{1.582685in}{3.413361in}}%
\pgfpathcurveto{\pgfqpoint{1.571635in}{3.413361in}}{\pgfqpoint{1.561036in}{3.408970in}}{\pgfqpoint{1.553222in}{3.401157in}}%
\pgfpathcurveto{\pgfqpoint{1.545409in}{3.393343in}}{\pgfqpoint{1.541018in}{3.382744in}}{\pgfqpoint{1.541018in}{3.371694in}}%
\pgfpathcurveto{\pgfqpoint{1.541018in}{3.360644in}}{\pgfqpoint{1.545409in}{3.350045in}}{\pgfqpoint{1.553222in}{3.342231in}}%
\pgfpathcurveto{\pgfqpoint{1.561036in}{3.334418in}}{\pgfqpoint{1.571635in}{3.330027in}}{\pgfqpoint{1.582685in}{3.330027in}}%
\pgfpathclose%
\pgfusepath{stroke,fill}%
\end{pgfscope}%
\begin{pgfscope}%
\pgfpathrectangle{\pgfqpoint{0.481978in}{0.331635in}}{\pgfqpoint{4.960000in}{3.696000in}}%
\pgfusepath{clip}%
\pgfsetbuttcap%
\pgfsetroundjoin%
\definecolor{currentfill}{rgb}{0.631373,0.788235,0.956863}%
\pgfsetfillcolor{currentfill}%
\pgfsetlinewidth{0.481800pt}%
\definecolor{currentstroke}{rgb}{1.000000,1.000000,1.000000}%
\pgfsetstrokecolor{currentstroke}%
\pgfsetdash{}{0pt}%
\pgfpathmoveto{\pgfqpoint{2.565304in}{2.268789in}}%
\pgfpathcurveto{\pgfqpoint{2.576354in}{2.268789in}}{\pgfqpoint{2.586953in}{2.273179in}}{\pgfqpoint{2.594767in}{2.280992in}}%
\pgfpathcurveto{\pgfqpoint{2.602581in}{2.288806in}}{\pgfqpoint{2.606971in}{2.299405in}}{\pgfqpoint{2.606971in}{2.310455in}}%
\pgfpathcurveto{\pgfqpoint{2.606971in}{2.321505in}}{\pgfqpoint{2.602581in}{2.332104in}}{\pgfqpoint{2.594767in}{2.339918in}}%
\pgfpathcurveto{\pgfqpoint{2.586953in}{2.347732in}}{\pgfqpoint{2.576354in}{2.352122in}}{\pgfqpoint{2.565304in}{2.352122in}}%
\pgfpathcurveto{\pgfqpoint{2.554254in}{2.352122in}}{\pgfqpoint{2.543655in}{2.347732in}}{\pgfqpoint{2.535841in}{2.339918in}}%
\pgfpathcurveto{\pgfqpoint{2.528028in}{2.332104in}}{\pgfqpoint{2.523637in}{2.321505in}}{\pgfqpoint{2.523637in}{2.310455in}}%
\pgfpathcurveto{\pgfqpoint{2.523637in}{2.299405in}}{\pgfqpoint{2.528028in}{2.288806in}}{\pgfqpoint{2.535841in}{2.280992in}}%
\pgfpathcurveto{\pgfqpoint{2.543655in}{2.273179in}}{\pgfqpoint{2.554254in}{2.268789in}}{\pgfqpoint{2.565304in}{2.268789in}}%
\pgfpathclose%
\pgfusepath{stroke,fill}%
\end{pgfscope}%
\begin{pgfscope}%
\pgfpathrectangle{\pgfqpoint{0.481978in}{0.331635in}}{\pgfqpoint{4.960000in}{3.696000in}}%
\pgfusepath{clip}%
\pgfsetbuttcap%
\pgfsetroundjoin%
\definecolor{currentfill}{rgb}{0.631373,0.788235,0.956863}%
\pgfsetfillcolor{currentfill}%
\pgfsetlinewidth{0.481800pt}%
\definecolor{currentstroke}{rgb}{1.000000,1.000000,1.000000}%
\pgfsetstrokecolor{currentstroke}%
\pgfsetdash{}{0pt}%
\pgfpathmoveto{\pgfqpoint{1.949373in}{2.789207in}}%
\pgfpathcurveto{\pgfqpoint{1.960423in}{2.789207in}}{\pgfqpoint{1.971022in}{2.793597in}}{\pgfqpoint{1.978836in}{2.801411in}}%
\pgfpathcurveto{\pgfqpoint{1.986649in}{2.809225in}}{\pgfqpoint{1.991040in}{2.819824in}}{\pgfqpoint{1.991040in}{2.830874in}}%
\pgfpathcurveto{\pgfqpoint{1.991040in}{2.841924in}}{\pgfqpoint{1.986649in}{2.852523in}}{\pgfqpoint{1.978836in}{2.860337in}}%
\pgfpathcurveto{\pgfqpoint{1.971022in}{2.868150in}}{\pgfqpoint{1.960423in}{2.872541in}}{\pgfqpoint{1.949373in}{2.872541in}}%
\pgfpathcurveto{\pgfqpoint{1.938323in}{2.872541in}}{\pgfqpoint{1.927724in}{2.868150in}}{\pgfqpoint{1.919910in}{2.860337in}}%
\pgfpathcurveto{\pgfqpoint{1.912097in}{2.852523in}}{\pgfqpoint{1.907706in}{2.841924in}}{\pgfqpoint{1.907706in}{2.830874in}}%
\pgfpathcurveto{\pgfqpoint{1.907706in}{2.819824in}}{\pgfqpoint{1.912097in}{2.809225in}}{\pgfqpoint{1.919910in}{2.801411in}}%
\pgfpathcurveto{\pgfqpoint{1.927724in}{2.793597in}}{\pgfqpoint{1.938323in}{2.789207in}}{\pgfqpoint{1.949373in}{2.789207in}}%
\pgfpathclose%
\pgfusepath{stroke,fill}%
\end{pgfscope}%
\begin{pgfscope}%
\pgfpathrectangle{\pgfqpoint{0.481978in}{0.331635in}}{\pgfqpoint{4.960000in}{3.696000in}}%
\pgfusepath{clip}%
\pgfsetbuttcap%
\pgfsetroundjoin%
\definecolor{currentfill}{rgb}{0.631373,0.788235,0.956863}%
\pgfsetfillcolor{currentfill}%
\pgfsetlinewidth{0.481800pt}%
\definecolor{currentstroke}{rgb}{1.000000,1.000000,1.000000}%
\pgfsetstrokecolor{currentstroke}%
\pgfsetdash{}{0pt}%
\pgfpathmoveto{\pgfqpoint{2.241416in}{2.213228in}}%
\pgfpathcurveto{\pgfqpoint{2.252466in}{2.213228in}}{\pgfqpoint{2.263065in}{2.217619in}}{\pgfqpoint{2.270879in}{2.225432in}}%
\pgfpathcurveto{\pgfqpoint{2.278692in}{2.233246in}}{\pgfqpoint{2.283083in}{2.243845in}}{\pgfqpoint{2.283083in}{2.254895in}}%
\pgfpathcurveto{\pgfqpoint{2.283083in}{2.265945in}}{\pgfqpoint{2.278692in}{2.276544in}}{\pgfqpoint{2.270879in}{2.284358in}}%
\pgfpathcurveto{\pgfqpoint{2.263065in}{2.292171in}}{\pgfqpoint{2.252466in}{2.296562in}}{\pgfqpoint{2.241416in}{2.296562in}}%
\pgfpathcurveto{\pgfqpoint{2.230366in}{2.296562in}}{\pgfqpoint{2.219767in}{2.292171in}}{\pgfqpoint{2.211953in}{2.284358in}}%
\pgfpathcurveto{\pgfqpoint{2.204140in}{2.276544in}}{\pgfqpoint{2.199749in}{2.265945in}}{\pgfqpoint{2.199749in}{2.254895in}}%
\pgfpathcurveto{\pgfqpoint{2.199749in}{2.243845in}}{\pgfqpoint{2.204140in}{2.233246in}}{\pgfqpoint{2.211953in}{2.225432in}}%
\pgfpathcurveto{\pgfqpoint{2.219767in}{2.217619in}}{\pgfqpoint{2.230366in}{2.213228in}}{\pgfqpoint{2.241416in}{2.213228in}}%
\pgfpathclose%
\pgfusepath{stroke,fill}%
\end{pgfscope}%
\begin{pgfscope}%
\pgfpathrectangle{\pgfqpoint{0.481978in}{0.331635in}}{\pgfqpoint{4.960000in}{3.696000in}}%
\pgfusepath{clip}%
\pgfsetbuttcap%
\pgfsetroundjoin%
\definecolor{currentfill}{rgb}{0.631373,0.788235,0.956863}%
\pgfsetfillcolor{currentfill}%
\pgfsetlinewidth{0.481800pt}%
\definecolor{currentstroke}{rgb}{1.000000,1.000000,1.000000}%
\pgfsetstrokecolor{currentstroke}%
\pgfsetdash{}{0pt}%
\pgfpathmoveto{\pgfqpoint{3.525117in}{1.360576in}}%
\pgfpathcurveto{\pgfqpoint{3.536167in}{1.360576in}}{\pgfqpoint{3.546766in}{1.364966in}}{\pgfqpoint{3.554579in}{1.372780in}}%
\pgfpathcurveto{\pgfqpoint{3.562393in}{1.380593in}}{\pgfqpoint{3.566783in}{1.391192in}}{\pgfqpoint{3.566783in}{1.402243in}}%
\pgfpathcurveto{\pgfqpoint{3.566783in}{1.413293in}}{\pgfqpoint{3.562393in}{1.423892in}}{\pgfqpoint{3.554579in}{1.431705in}}%
\pgfpathcurveto{\pgfqpoint{3.546766in}{1.439519in}}{\pgfqpoint{3.536167in}{1.443909in}}{\pgfqpoint{3.525117in}{1.443909in}}%
\pgfpathcurveto{\pgfqpoint{3.514067in}{1.443909in}}{\pgfqpoint{3.503468in}{1.439519in}}{\pgfqpoint{3.495654in}{1.431705in}}%
\pgfpathcurveto{\pgfqpoint{3.487840in}{1.423892in}}{\pgfqpoint{3.483450in}{1.413293in}}{\pgfqpoint{3.483450in}{1.402243in}}%
\pgfpathcurveto{\pgfqpoint{3.483450in}{1.391192in}}{\pgfqpoint{3.487840in}{1.380593in}}{\pgfqpoint{3.495654in}{1.372780in}}%
\pgfpathcurveto{\pgfqpoint{3.503468in}{1.364966in}}{\pgfqpoint{3.514067in}{1.360576in}}{\pgfqpoint{3.525117in}{1.360576in}}%
\pgfpathclose%
\pgfusepath{stroke,fill}%
\end{pgfscope}%
\begin{pgfscope}%
\pgfpathrectangle{\pgfqpoint{0.481978in}{0.331635in}}{\pgfqpoint{4.960000in}{3.696000in}}%
\pgfusepath{clip}%
\pgfsetbuttcap%
\pgfsetroundjoin%
\definecolor{currentfill}{rgb}{0.631373,0.788235,0.956863}%
\pgfsetfillcolor{currentfill}%
\pgfsetlinewidth{0.481800pt}%
\definecolor{currentstroke}{rgb}{1.000000,1.000000,1.000000}%
\pgfsetstrokecolor{currentstroke}%
\pgfsetdash{}{0pt}%
\pgfpathmoveto{\pgfqpoint{3.771458in}{2.197603in}}%
\pgfpathcurveto{\pgfqpoint{3.782509in}{2.197603in}}{\pgfqpoint{3.793108in}{2.201993in}}{\pgfqpoint{3.800921in}{2.209807in}}%
\pgfpathcurveto{\pgfqpoint{3.808735in}{2.217620in}}{\pgfqpoint{3.813125in}{2.228219in}}{\pgfqpoint{3.813125in}{2.239269in}}%
\pgfpathcurveto{\pgfqpoint{3.813125in}{2.250320in}}{\pgfqpoint{3.808735in}{2.260919in}}{\pgfqpoint{3.800921in}{2.268732in}}%
\pgfpathcurveto{\pgfqpoint{3.793108in}{2.276546in}}{\pgfqpoint{3.782509in}{2.280936in}}{\pgfqpoint{3.771458in}{2.280936in}}%
\pgfpathcurveto{\pgfqpoint{3.760408in}{2.280936in}}{\pgfqpoint{3.749809in}{2.276546in}}{\pgfqpoint{3.741996in}{2.268732in}}%
\pgfpathcurveto{\pgfqpoint{3.734182in}{2.260919in}}{\pgfqpoint{3.729792in}{2.250320in}}{\pgfqpoint{3.729792in}{2.239269in}}%
\pgfpathcurveto{\pgfqpoint{3.729792in}{2.228219in}}{\pgfqpoint{3.734182in}{2.217620in}}{\pgfqpoint{3.741996in}{2.209807in}}%
\pgfpathcurveto{\pgfqpoint{3.749809in}{2.201993in}}{\pgfqpoint{3.760408in}{2.197603in}}{\pgfqpoint{3.771458in}{2.197603in}}%
\pgfpathclose%
\pgfusepath{stroke,fill}%
\end{pgfscope}%
\begin{pgfscope}%
\pgfpathrectangle{\pgfqpoint{0.481978in}{0.331635in}}{\pgfqpoint{4.960000in}{3.696000in}}%
\pgfusepath{clip}%
\pgfsetbuttcap%
\pgfsetroundjoin%
\definecolor{currentfill}{rgb}{0.631373,0.788235,0.956863}%
\pgfsetfillcolor{currentfill}%
\pgfsetlinewidth{0.481800pt}%
\definecolor{currentstroke}{rgb}{1.000000,1.000000,1.000000}%
\pgfsetstrokecolor{currentstroke}%
\pgfsetdash{}{0pt}%
\pgfpathmoveto{\pgfqpoint{2.157114in}{1.349032in}}%
\pgfpathcurveto{\pgfqpoint{2.168164in}{1.349032in}}{\pgfqpoint{2.178763in}{1.353422in}}{\pgfqpoint{2.186576in}{1.361236in}}%
\pgfpathcurveto{\pgfqpoint{2.194390in}{1.369049in}}{\pgfqpoint{2.198780in}{1.379649in}}{\pgfqpoint{2.198780in}{1.390699in}}%
\pgfpathcurveto{\pgfqpoint{2.198780in}{1.401749in}}{\pgfqpoint{2.194390in}{1.412348in}}{\pgfqpoint{2.186576in}{1.420161in}}%
\pgfpathcurveto{\pgfqpoint{2.178763in}{1.427975in}}{\pgfqpoint{2.168164in}{1.432365in}}{\pgfqpoint{2.157114in}{1.432365in}}%
\pgfpathcurveto{\pgfqpoint{2.146064in}{1.432365in}}{\pgfqpoint{2.135464in}{1.427975in}}{\pgfqpoint{2.127651in}{1.420161in}}%
\pgfpathcurveto{\pgfqpoint{2.119837in}{1.412348in}}{\pgfqpoint{2.115447in}{1.401749in}}{\pgfqpoint{2.115447in}{1.390699in}}%
\pgfpathcurveto{\pgfqpoint{2.115447in}{1.379649in}}{\pgfqpoint{2.119837in}{1.369049in}}{\pgfqpoint{2.127651in}{1.361236in}}%
\pgfpathcurveto{\pgfqpoint{2.135464in}{1.353422in}}{\pgfqpoint{2.146064in}{1.349032in}}{\pgfqpoint{2.157114in}{1.349032in}}%
\pgfpathclose%
\pgfusepath{stroke,fill}%
\end{pgfscope}%
\begin{pgfscope}%
\pgfpathrectangle{\pgfqpoint{0.481978in}{0.331635in}}{\pgfqpoint{4.960000in}{3.696000in}}%
\pgfusepath{clip}%
\pgfsetbuttcap%
\pgfsetroundjoin%
\definecolor{currentfill}{rgb}{0.631373,0.788235,0.956863}%
\pgfsetfillcolor{currentfill}%
\pgfsetlinewidth{0.481800pt}%
\definecolor{currentstroke}{rgb}{1.000000,1.000000,1.000000}%
\pgfsetstrokecolor{currentstroke}%
\pgfsetdash{}{0pt}%
\pgfpathmoveto{\pgfqpoint{1.828104in}{2.772115in}}%
\pgfpathcurveto{\pgfqpoint{1.839154in}{2.772115in}}{\pgfqpoint{1.849753in}{2.776506in}}{\pgfqpoint{1.857567in}{2.784319in}}%
\pgfpathcurveto{\pgfqpoint{1.865380in}{2.792133in}}{\pgfqpoint{1.869771in}{2.802732in}}{\pgfqpoint{1.869771in}{2.813782in}}%
\pgfpathcurveto{\pgfqpoint{1.869771in}{2.824832in}}{\pgfqpoint{1.865380in}{2.835431in}}{\pgfqpoint{1.857567in}{2.843245in}}%
\pgfpathcurveto{\pgfqpoint{1.849753in}{2.851058in}}{\pgfqpoint{1.839154in}{2.855449in}}{\pgfqpoint{1.828104in}{2.855449in}}%
\pgfpathcurveto{\pgfqpoint{1.817054in}{2.855449in}}{\pgfqpoint{1.806455in}{2.851058in}}{\pgfqpoint{1.798641in}{2.843245in}}%
\pgfpathcurveto{\pgfqpoint{1.790827in}{2.835431in}}{\pgfqpoint{1.786437in}{2.824832in}}{\pgfqpoint{1.786437in}{2.813782in}}%
\pgfpathcurveto{\pgfqpoint{1.786437in}{2.802732in}}{\pgfqpoint{1.790827in}{2.792133in}}{\pgfqpoint{1.798641in}{2.784319in}}%
\pgfpathcurveto{\pgfqpoint{1.806455in}{2.776506in}}{\pgfqpoint{1.817054in}{2.772115in}}{\pgfqpoint{1.828104in}{2.772115in}}%
\pgfpathclose%
\pgfusepath{stroke,fill}%
\end{pgfscope}%
\begin{pgfscope}%
\pgfpathrectangle{\pgfqpoint{0.481978in}{0.331635in}}{\pgfqpoint{4.960000in}{3.696000in}}%
\pgfusepath{clip}%
\pgfsetbuttcap%
\pgfsetroundjoin%
\definecolor{currentfill}{rgb}{0.631373,0.788235,0.956863}%
\pgfsetfillcolor{currentfill}%
\pgfsetlinewidth{0.481800pt}%
\definecolor{currentstroke}{rgb}{1.000000,1.000000,1.000000}%
\pgfsetstrokecolor{currentstroke}%
\pgfsetdash{}{0pt}%
\pgfpathmoveto{\pgfqpoint{2.853703in}{3.257160in}}%
\pgfpathcurveto{\pgfqpoint{2.864753in}{3.257160in}}{\pgfqpoint{2.875352in}{3.261551in}}{\pgfqpoint{2.883165in}{3.269364in}}%
\pgfpathcurveto{\pgfqpoint{2.890979in}{3.277178in}}{\pgfqpoint{2.895369in}{3.287777in}}{\pgfqpoint{2.895369in}{3.298827in}}%
\pgfpathcurveto{\pgfqpoint{2.895369in}{3.309877in}}{\pgfqpoint{2.890979in}{3.320476in}}{\pgfqpoint{2.883165in}{3.328290in}}%
\pgfpathcurveto{\pgfqpoint{2.875352in}{3.336103in}}{\pgfqpoint{2.864753in}{3.340494in}}{\pgfqpoint{2.853703in}{3.340494in}}%
\pgfpathcurveto{\pgfqpoint{2.842652in}{3.340494in}}{\pgfqpoint{2.832053in}{3.336103in}}{\pgfqpoint{2.824240in}{3.328290in}}%
\pgfpathcurveto{\pgfqpoint{2.816426in}{3.320476in}}{\pgfqpoint{2.812036in}{3.309877in}}{\pgfqpoint{2.812036in}{3.298827in}}%
\pgfpathcurveto{\pgfqpoint{2.812036in}{3.287777in}}{\pgfqpoint{2.816426in}{3.277178in}}{\pgfqpoint{2.824240in}{3.269364in}}%
\pgfpathcurveto{\pgfqpoint{2.832053in}{3.261551in}}{\pgfqpoint{2.842652in}{3.257160in}}{\pgfqpoint{2.853703in}{3.257160in}}%
\pgfpathclose%
\pgfusepath{stroke,fill}%
\end{pgfscope}%
\begin{pgfscope}%
\pgfpathrectangle{\pgfqpoint{0.481978in}{0.331635in}}{\pgfqpoint{4.960000in}{3.696000in}}%
\pgfusepath{clip}%
\pgfsetbuttcap%
\pgfsetroundjoin%
\definecolor{currentfill}{rgb}{0.631373,0.788235,0.956863}%
\pgfsetfillcolor{currentfill}%
\pgfsetlinewidth{0.481800pt}%
\definecolor{currentstroke}{rgb}{1.000000,1.000000,1.000000}%
\pgfsetstrokecolor{currentstroke}%
\pgfsetdash{}{0pt}%
\pgfpathmoveto{\pgfqpoint{2.512476in}{1.491436in}}%
\pgfpathcurveto{\pgfqpoint{2.523526in}{1.491436in}}{\pgfqpoint{2.534125in}{1.495827in}}{\pgfqpoint{2.541939in}{1.503640in}}%
\pgfpathcurveto{\pgfqpoint{2.549752in}{1.511454in}}{\pgfqpoint{2.554143in}{1.522053in}}{\pgfqpoint{2.554143in}{1.533103in}}%
\pgfpathcurveto{\pgfqpoint{2.554143in}{1.544153in}}{\pgfqpoint{2.549752in}{1.554752in}}{\pgfqpoint{2.541939in}{1.562566in}}%
\pgfpathcurveto{\pgfqpoint{2.534125in}{1.570379in}}{\pgfqpoint{2.523526in}{1.574770in}}{\pgfqpoint{2.512476in}{1.574770in}}%
\pgfpathcurveto{\pgfqpoint{2.501426in}{1.574770in}}{\pgfqpoint{2.490827in}{1.570379in}}{\pgfqpoint{2.483013in}{1.562566in}}%
\pgfpathcurveto{\pgfqpoint{2.475200in}{1.554752in}}{\pgfqpoint{2.470809in}{1.544153in}}{\pgfqpoint{2.470809in}{1.533103in}}%
\pgfpathcurveto{\pgfqpoint{2.470809in}{1.522053in}}{\pgfqpoint{2.475200in}{1.511454in}}{\pgfqpoint{2.483013in}{1.503640in}}%
\pgfpathcurveto{\pgfqpoint{2.490827in}{1.495827in}}{\pgfqpoint{2.501426in}{1.491436in}}{\pgfqpoint{2.512476in}{1.491436in}}%
\pgfpathclose%
\pgfusepath{stroke,fill}%
\end{pgfscope}%
\begin{pgfscope}%
\pgfpathrectangle{\pgfqpoint{0.481978in}{0.331635in}}{\pgfqpoint{4.960000in}{3.696000in}}%
\pgfusepath{clip}%
\pgfsetbuttcap%
\pgfsetroundjoin%
\definecolor{currentfill}{rgb}{0.631373,0.788235,0.956863}%
\pgfsetfillcolor{currentfill}%
\pgfsetlinewidth{0.481800pt}%
\definecolor{currentstroke}{rgb}{1.000000,1.000000,1.000000}%
\pgfsetstrokecolor{currentstroke}%
\pgfsetdash{}{0pt}%
\pgfpathmoveto{\pgfqpoint{2.414659in}{1.507761in}}%
\pgfpathcurveto{\pgfqpoint{2.425709in}{1.507761in}}{\pgfqpoint{2.436308in}{1.512151in}}{\pgfqpoint{2.444122in}{1.519965in}}%
\pgfpathcurveto{\pgfqpoint{2.451935in}{1.527778in}}{\pgfqpoint{2.456326in}{1.538377in}}{\pgfqpoint{2.456326in}{1.549428in}}%
\pgfpathcurveto{\pgfqpoint{2.456326in}{1.560478in}}{\pgfqpoint{2.451935in}{1.571077in}}{\pgfqpoint{2.444122in}{1.578890in}}%
\pgfpathcurveto{\pgfqpoint{2.436308in}{1.586704in}}{\pgfqpoint{2.425709in}{1.591094in}}{\pgfqpoint{2.414659in}{1.591094in}}%
\pgfpathcurveto{\pgfqpoint{2.403609in}{1.591094in}}{\pgfqpoint{2.393010in}{1.586704in}}{\pgfqpoint{2.385196in}{1.578890in}}%
\pgfpathcurveto{\pgfqpoint{2.377382in}{1.571077in}}{\pgfqpoint{2.372992in}{1.560478in}}{\pgfqpoint{2.372992in}{1.549428in}}%
\pgfpathcurveto{\pgfqpoint{2.372992in}{1.538377in}}{\pgfqpoint{2.377382in}{1.527778in}}{\pgfqpoint{2.385196in}{1.519965in}}%
\pgfpathcurveto{\pgfqpoint{2.393010in}{1.512151in}}{\pgfqpoint{2.403609in}{1.507761in}}{\pgfqpoint{2.414659in}{1.507761in}}%
\pgfpathclose%
\pgfusepath{stroke,fill}%
\end{pgfscope}%
\begin{pgfscope}%
\pgfpathrectangle{\pgfqpoint{0.481978in}{0.331635in}}{\pgfqpoint{4.960000in}{3.696000in}}%
\pgfusepath{clip}%
\pgfsetbuttcap%
\pgfsetroundjoin%
\definecolor{currentfill}{rgb}{0.631373,0.788235,0.956863}%
\pgfsetfillcolor{currentfill}%
\pgfsetlinewidth{0.481800pt}%
\definecolor{currentstroke}{rgb}{1.000000,1.000000,1.000000}%
\pgfsetstrokecolor{currentstroke}%
\pgfsetdash{}{0pt}%
\pgfpathmoveto{\pgfqpoint{2.518218in}{1.972560in}}%
\pgfpathcurveto{\pgfqpoint{2.529268in}{1.972560in}}{\pgfqpoint{2.539867in}{1.976951in}}{\pgfqpoint{2.547680in}{1.984764in}}%
\pgfpathcurveto{\pgfqpoint{2.555494in}{1.992578in}}{\pgfqpoint{2.559884in}{2.003177in}}{\pgfqpoint{2.559884in}{2.014227in}}%
\pgfpathcurveto{\pgfqpoint{2.559884in}{2.025277in}}{\pgfqpoint{2.555494in}{2.035876in}}{\pgfqpoint{2.547680in}{2.043690in}}%
\pgfpathcurveto{\pgfqpoint{2.539867in}{2.051504in}}{\pgfqpoint{2.529268in}{2.055894in}}{\pgfqpoint{2.518218in}{2.055894in}}%
\pgfpathcurveto{\pgfqpoint{2.507167in}{2.055894in}}{\pgfqpoint{2.496568in}{2.051504in}}{\pgfqpoint{2.488755in}{2.043690in}}%
\pgfpathcurveto{\pgfqpoint{2.480941in}{2.035876in}}{\pgfqpoint{2.476551in}{2.025277in}}{\pgfqpoint{2.476551in}{2.014227in}}%
\pgfpathcurveto{\pgfqpoint{2.476551in}{2.003177in}}{\pgfqpoint{2.480941in}{1.992578in}}{\pgfqpoint{2.488755in}{1.984764in}}%
\pgfpathcurveto{\pgfqpoint{2.496568in}{1.976951in}}{\pgfqpoint{2.507167in}{1.972560in}}{\pgfqpoint{2.518218in}{1.972560in}}%
\pgfpathclose%
\pgfusepath{stroke,fill}%
\end{pgfscope}%
\begin{pgfscope}%
\pgfpathrectangle{\pgfqpoint{0.481978in}{0.331635in}}{\pgfqpoint{4.960000in}{3.696000in}}%
\pgfusepath{clip}%
\pgfsetbuttcap%
\pgfsetroundjoin%
\definecolor{currentfill}{rgb}{0.631373,0.788235,0.956863}%
\pgfsetfillcolor{currentfill}%
\pgfsetlinewidth{0.481800pt}%
\definecolor{currentstroke}{rgb}{1.000000,1.000000,1.000000}%
\pgfsetstrokecolor{currentstroke}%
\pgfsetdash{}{0pt}%
\pgfpathmoveto{\pgfqpoint{3.567777in}{2.292092in}}%
\pgfpathcurveto{\pgfqpoint{3.578827in}{2.292092in}}{\pgfqpoint{3.589426in}{2.296482in}}{\pgfqpoint{3.597240in}{2.304296in}}%
\pgfpathcurveto{\pgfqpoint{3.605054in}{2.312110in}}{\pgfqpoint{3.609444in}{2.322709in}}{\pgfqpoint{3.609444in}{2.333759in}}%
\pgfpathcurveto{\pgfqpoint{3.609444in}{2.344809in}}{\pgfqpoint{3.605054in}{2.355408in}}{\pgfqpoint{3.597240in}{2.363222in}}%
\pgfpathcurveto{\pgfqpoint{3.589426in}{2.371035in}}{\pgfqpoint{3.578827in}{2.375426in}}{\pgfqpoint{3.567777in}{2.375426in}}%
\pgfpathcurveto{\pgfqpoint{3.556727in}{2.375426in}}{\pgfqpoint{3.546128in}{2.371035in}}{\pgfqpoint{3.538314in}{2.363222in}}%
\pgfpathcurveto{\pgfqpoint{3.530501in}{2.355408in}}{\pgfqpoint{3.526111in}{2.344809in}}{\pgfqpoint{3.526111in}{2.333759in}}%
\pgfpathcurveto{\pgfqpoint{3.526111in}{2.322709in}}{\pgfqpoint{3.530501in}{2.312110in}}{\pgfqpoint{3.538314in}{2.304296in}}%
\pgfpathcurveto{\pgfqpoint{3.546128in}{2.296482in}}{\pgfqpoint{3.556727in}{2.292092in}}{\pgfqpoint{3.567777in}{2.292092in}}%
\pgfpathclose%
\pgfusepath{stroke,fill}%
\end{pgfscope}%
\begin{pgfscope}%
\pgfpathrectangle{\pgfqpoint{0.481978in}{0.331635in}}{\pgfqpoint{4.960000in}{3.696000in}}%
\pgfusepath{clip}%
\pgfsetbuttcap%
\pgfsetroundjoin%
\definecolor{currentfill}{rgb}{0.631373,0.788235,0.956863}%
\pgfsetfillcolor{currentfill}%
\pgfsetlinewidth{0.481800pt}%
\definecolor{currentstroke}{rgb}{1.000000,1.000000,1.000000}%
\pgfsetstrokecolor{currentstroke}%
\pgfsetdash{}{0pt}%
\pgfpathmoveto{\pgfqpoint{2.525557in}{2.613078in}}%
\pgfpathcurveto{\pgfqpoint{2.536607in}{2.613078in}}{\pgfqpoint{2.547206in}{2.617468in}}{\pgfqpoint{2.555020in}{2.625282in}}%
\pgfpathcurveto{\pgfqpoint{2.562833in}{2.633095in}}{\pgfqpoint{2.567223in}{2.643694in}}{\pgfqpoint{2.567223in}{2.654744in}}%
\pgfpathcurveto{\pgfqpoint{2.567223in}{2.665794in}}{\pgfqpoint{2.562833in}{2.676393in}}{\pgfqpoint{2.555020in}{2.684207in}}%
\pgfpathcurveto{\pgfqpoint{2.547206in}{2.692021in}}{\pgfqpoint{2.536607in}{2.696411in}}{\pgfqpoint{2.525557in}{2.696411in}}%
\pgfpathcurveto{\pgfqpoint{2.514507in}{2.696411in}}{\pgfqpoint{2.503908in}{2.692021in}}{\pgfqpoint{2.496094in}{2.684207in}}%
\pgfpathcurveto{\pgfqpoint{2.488280in}{2.676393in}}{\pgfqpoint{2.483890in}{2.665794in}}{\pgfqpoint{2.483890in}{2.654744in}}%
\pgfpathcurveto{\pgfqpoint{2.483890in}{2.643694in}}{\pgfqpoint{2.488280in}{2.633095in}}{\pgfqpoint{2.496094in}{2.625282in}}%
\pgfpathcurveto{\pgfqpoint{2.503908in}{2.617468in}}{\pgfqpoint{2.514507in}{2.613078in}}{\pgfqpoint{2.525557in}{2.613078in}}%
\pgfpathclose%
\pgfusepath{stroke,fill}%
\end{pgfscope}%
\begin{pgfscope}%
\pgfpathrectangle{\pgfqpoint{0.481978in}{0.331635in}}{\pgfqpoint{4.960000in}{3.696000in}}%
\pgfusepath{clip}%
\pgfsetbuttcap%
\pgfsetroundjoin%
\definecolor{currentfill}{rgb}{0.631373,0.788235,0.956863}%
\pgfsetfillcolor{currentfill}%
\pgfsetlinewidth{0.481800pt}%
\definecolor{currentstroke}{rgb}{1.000000,1.000000,1.000000}%
\pgfsetstrokecolor{currentstroke}%
\pgfsetdash{}{0pt}%
\pgfpathmoveto{\pgfqpoint{2.649451in}{1.988106in}}%
\pgfpathcurveto{\pgfqpoint{2.660502in}{1.988106in}}{\pgfqpoint{2.671101in}{1.992496in}}{\pgfqpoint{2.678914in}{2.000310in}}%
\pgfpathcurveto{\pgfqpoint{2.686728in}{2.008123in}}{\pgfqpoint{2.691118in}{2.018722in}}{\pgfqpoint{2.691118in}{2.029773in}}%
\pgfpathcurveto{\pgfqpoint{2.691118in}{2.040823in}}{\pgfqpoint{2.686728in}{2.051422in}}{\pgfqpoint{2.678914in}{2.059235in}}%
\pgfpathcurveto{\pgfqpoint{2.671101in}{2.067049in}}{\pgfqpoint{2.660502in}{2.071439in}}{\pgfqpoint{2.649451in}{2.071439in}}%
\pgfpathcurveto{\pgfqpoint{2.638401in}{2.071439in}}{\pgfqpoint{2.627802in}{2.067049in}}{\pgfqpoint{2.619989in}{2.059235in}}%
\pgfpathcurveto{\pgfqpoint{2.612175in}{2.051422in}}{\pgfqpoint{2.607785in}{2.040823in}}{\pgfqpoint{2.607785in}{2.029773in}}%
\pgfpathcurveto{\pgfqpoint{2.607785in}{2.018722in}}{\pgfqpoint{2.612175in}{2.008123in}}{\pgfqpoint{2.619989in}{2.000310in}}%
\pgfpathcurveto{\pgfqpoint{2.627802in}{1.992496in}}{\pgfqpoint{2.638401in}{1.988106in}}{\pgfqpoint{2.649451in}{1.988106in}}%
\pgfpathclose%
\pgfusepath{stroke,fill}%
\end{pgfscope}%
\begin{pgfscope}%
\pgfpathrectangle{\pgfqpoint{0.481978in}{0.331635in}}{\pgfqpoint{4.960000in}{3.696000in}}%
\pgfusepath{clip}%
\pgfsetbuttcap%
\pgfsetroundjoin%
\definecolor{currentfill}{rgb}{0.631373,0.788235,0.956863}%
\pgfsetfillcolor{currentfill}%
\pgfsetlinewidth{0.481800pt}%
\definecolor{currentstroke}{rgb}{1.000000,1.000000,1.000000}%
\pgfsetstrokecolor{currentstroke}%
\pgfsetdash{}{0pt}%
\pgfpathmoveto{\pgfqpoint{2.313708in}{2.770604in}}%
\pgfpathcurveto{\pgfqpoint{2.324758in}{2.770604in}}{\pgfqpoint{2.335357in}{2.774995in}}{\pgfqpoint{2.343171in}{2.782808in}}%
\pgfpathcurveto{\pgfqpoint{2.350984in}{2.790622in}}{\pgfqpoint{2.355375in}{2.801221in}}{\pgfqpoint{2.355375in}{2.812271in}}%
\pgfpathcurveto{\pgfqpoint{2.355375in}{2.823321in}}{\pgfqpoint{2.350984in}{2.833920in}}{\pgfqpoint{2.343171in}{2.841734in}}%
\pgfpathcurveto{\pgfqpoint{2.335357in}{2.849547in}}{\pgfqpoint{2.324758in}{2.853938in}}{\pgfqpoint{2.313708in}{2.853938in}}%
\pgfpathcurveto{\pgfqpoint{2.302658in}{2.853938in}}{\pgfqpoint{2.292059in}{2.849547in}}{\pgfqpoint{2.284245in}{2.841734in}}%
\pgfpathcurveto{\pgfqpoint{2.276431in}{2.833920in}}{\pgfqpoint{2.272041in}{2.823321in}}{\pgfqpoint{2.272041in}{2.812271in}}%
\pgfpathcurveto{\pgfqpoint{2.272041in}{2.801221in}}{\pgfqpoint{2.276431in}{2.790622in}}{\pgfqpoint{2.284245in}{2.782808in}}%
\pgfpathcurveto{\pgfqpoint{2.292059in}{2.774995in}}{\pgfqpoint{2.302658in}{2.770604in}}{\pgfqpoint{2.313708in}{2.770604in}}%
\pgfpathclose%
\pgfusepath{stroke,fill}%
\end{pgfscope}%
\begin{pgfscope}%
\pgfpathrectangle{\pgfqpoint{0.481978in}{0.331635in}}{\pgfqpoint{4.960000in}{3.696000in}}%
\pgfusepath{clip}%
\pgfsetbuttcap%
\pgfsetroundjoin%
\definecolor{currentfill}{rgb}{0.631373,0.788235,0.956863}%
\pgfsetfillcolor{currentfill}%
\pgfsetlinewidth{0.481800pt}%
\definecolor{currentstroke}{rgb}{1.000000,1.000000,1.000000}%
\pgfsetstrokecolor{currentstroke}%
\pgfsetdash{}{0pt}%
\pgfpathmoveto{\pgfqpoint{2.006019in}{2.730109in}}%
\pgfpathcurveto{\pgfqpoint{2.017069in}{2.730109in}}{\pgfqpoint{2.027668in}{2.734499in}}{\pgfqpoint{2.035481in}{2.742313in}}%
\pgfpathcurveto{\pgfqpoint{2.043295in}{2.750127in}}{\pgfqpoint{2.047685in}{2.760726in}}{\pgfqpoint{2.047685in}{2.771776in}}%
\pgfpathcurveto{\pgfqpoint{2.047685in}{2.782826in}}{\pgfqpoint{2.043295in}{2.793425in}}{\pgfqpoint{2.035481in}{2.801239in}}%
\pgfpathcurveto{\pgfqpoint{2.027668in}{2.809052in}}{\pgfqpoint{2.017069in}{2.813443in}}{\pgfqpoint{2.006019in}{2.813443in}}%
\pgfpathcurveto{\pgfqpoint{1.994968in}{2.813443in}}{\pgfqpoint{1.984369in}{2.809052in}}{\pgfqpoint{1.976556in}{2.801239in}}%
\pgfpathcurveto{\pgfqpoint{1.968742in}{2.793425in}}{\pgfqpoint{1.964352in}{2.782826in}}{\pgfqpoint{1.964352in}{2.771776in}}%
\pgfpathcurveto{\pgfqpoint{1.964352in}{2.760726in}}{\pgfqpoint{1.968742in}{2.750127in}}{\pgfqpoint{1.976556in}{2.742313in}}%
\pgfpathcurveto{\pgfqpoint{1.984369in}{2.734499in}}{\pgfqpoint{1.994968in}{2.730109in}}{\pgfqpoint{2.006019in}{2.730109in}}%
\pgfpathclose%
\pgfusepath{stroke,fill}%
\end{pgfscope}%
\begin{pgfscope}%
\pgfpathrectangle{\pgfqpoint{0.481978in}{0.331635in}}{\pgfqpoint{4.960000in}{3.696000in}}%
\pgfusepath{clip}%
\pgfsetbuttcap%
\pgfsetroundjoin%
\definecolor{currentfill}{rgb}{0.631373,0.788235,0.956863}%
\pgfsetfillcolor{currentfill}%
\pgfsetlinewidth{0.481800pt}%
\definecolor{currentstroke}{rgb}{1.000000,1.000000,1.000000}%
\pgfsetstrokecolor{currentstroke}%
\pgfsetdash{}{0pt}%
\pgfpathmoveto{\pgfqpoint{2.334579in}{1.859019in}}%
\pgfpathcurveto{\pgfqpoint{2.345629in}{1.859019in}}{\pgfqpoint{2.356228in}{1.863409in}}{\pgfqpoint{2.364042in}{1.871223in}}%
\pgfpathcurveto{\pgfqpoint{2.371856in}{1.879036in}}{\pgfqpoint{2.376246in}{1.889635in}}{\pgfqpoint{2.376246in}{1.900686in}}%
\pgfpathcurveto{\pgfqpoint{2.376246in}{1.911736in}}{\pgfqpoint{2.371856in}{1.922335in}}{\pgfqpoint{2.364042in}{1.930148in}}%
\pgfpathcurveto{\pgfqpoint{2.356228in}{1.937962in}}{\pgfqpoint{2.345629in}{1.942352in}}{\pgfqpoint{2.334579in}{1.942352in}}%
\pgfpathcurveto{\pgfqpoint{2.323529in}{1.942352in}}{\pgfqpoint{2.312930in}{1.937962in}}{\pgfqpoint{2.305116in}{1.930148in}}%
\pgfpathcurveto{\pgfqpoint{2.297303in}{1.922335in}}{\pgfqpoint{2.292913in}{1.911736in}}{\pgfqpoint{2.292913in}{1.900686in}}%
\pgfpathcurveto{\pgfqpoint{2.292913in}{1.889635in}}{\pgfqpoint{2.297303in}{1.879036in}}{\pgfqpoint{2.305116in}{1.871223in}}%
\pgfpathcurveto{\pgfqpoint{2.312930in}{1.863409in}}{\pgfqpoint{2.323529in}{1.859019in}}{\pgfqpoint{2.334579in}{1.859019in}}%
\pgfpathclose%
\pgfusepath{stroke,fill}%
\end{pgfscope}%
\begin{pgfscope}%
\pgfpathrectangle{\pgfqpoint{0.481978in}{0.331635in}}{\pgfqpoint{4.960000in}{3.696000in}}%
\pgfusepath{clip}%
\pgfsetbuttcap%
\pgfsetroundjoin%
\definecolor{currentfill}{rgb}{0.631373,0.788235,0.956863}%
\pgfsetfillcolor{currentfill}%
\pgfsetlinewidth{0.481800pt}%
\definecolor{currentstroke}{rgb}{1.000000,1.000000,1.000000}%
\pgfsetstrokecolor{currentstroke}%
\pgfsetdash{}{0pt}%
\pgfpathmoveto{\pgfqpoint{2.056506in}{2.109275in}}%
\pgfpathcurveto{\pgfqpoint{2.067556in}{2.109275in}}{\pgfqpoint{2.078155in}{2.113665in}}{\pgfqpoint{2.085969in}{2.121479in}}%
\pgfpathcurveto{\pgfqpoint{2.093782in}{2.129293in}}{\pgfqpoint{2.098173in}{2.139892in}}{\pgfqpoint{2.098173in}{2.150942in}}%
\pgfpathcurveto{\pgfqpoint{2.098173in}{2.161992in}}{\pgfqpoint{2.093782in}{2.172591in}}{\pgfqpoint{2.085969in}{2.180405in}}%
\pgfpathcurveto{\pgfqpoint{2.078155in}{2.188218in}}{\pgfqpoint{2.067556in}{2.192608in}}{\pgfqpoint{2.056506in}{2.192608in}}%
\pgfpathcurveto{\pgfqpoint{2.045456in}{2.192608in}}{\pgfqpoint{2.034857in}{2.188218in}}{\pgfqpoint{2.027043in}{2.180405in}}%
\pgfpathcurveto{\pgfqpoint{2.019230in}{2.172591in}}{\pgfqpoint{2.014839in}{2.161992in}}{\pgfqpoint{2.014839in}{2.150942in}}%
\pgfpathcurveto{\pgfqpoint{2.014839in}{2.139892in}}{\pgfqpoint{2.019230in}{2.129293in}}{\pgfqpoint{2.027043in}{2.121479in}}%
\pgfpathcurveto{\pgfqpoint{2.034857in}{2.113665in}}{\pgfqpoint{2.045456in}{2.109275in}}{\pgfqpoint{2.056506in}{2.109275in}}%
\pgfpathclose%
\pgfusepath{stroke,fill}%
\end{pgfscope}%
\begin{pgfscope}%
\pgfpathrectangle{\pgfqpoint{0.481978in}{0.331635in}}{\pgfqpoint{4.960000in}{3.696000in}}%
\pgfusepath{clip}%
\pgfsetbuttcap%
\pgfsetroundjoin%
\definecolor{currentfill}{rgb}{0.631373,0.788235,0.956863}%
\pgfsetfillcolor{currentfill}%
\pgfsetlinewidth{0.481800pt}%
\definecolor{currentstroke}{rgb}{1.000000,1.000000,1.000000}%
\pgfsetstrokecolor{currentstroke}%
\pgfsetdash{}{0pt}%
\pgfpathmoveto{\pgfqpoint{2.475462in}{2.249626in}}%
\pgfpathcurveto{\pgfqpoint{2.486512in}{2.249626in}}{\pgfqpoint{2.497111in}{2.254016in}}{\pgfqpoint{2.504925in}{2.261830in}}%
\pgfpathcurveto{\pgfqpoint{2.512739in}{2.269643in}}{\pgfqpoint{2.517129in}{2.280242in}}{\pgfqpoint{2.517129in}{2.291293in}}%
\pgfpathcurveto{\pgfqpoint{2.517129in}{2.302343in}}{\pgfqpoint{2.512739in}{2.312942in}}{\pgfqpoint{2.504925in}{2.320755in}}%
\pgfpathcurveto{\pgfqpoint{2.497111in}{2.328569in}}{\pgfqpoint{2.486512in}{2.332959in}}{\pgfqpoint{2.475462in}{2.332959in}}%
\pgfpathcurveto{\pgfqpoint{2.464412in}{2.332959in}}{\pgfqpoint{2.453813in}{2.328569in}}{\pgfqpoint{2.446000in}{2.320755in}}%
\pgfpathcurveto{\pgfqpoint{2.438186in}{2.312942in}}{\pgfqpoint{2.433796in}{2.302343in}}{\pgfqpoint{2.433796in}{2.291293in}}%
\pgfpathcurveto{\pgfqpoint{2.433796in}{2.280242in}}{\pgfqpoint{2.438186in}{2.269643in}}{\pgfqpoint{2.446000in}{2.261830in}}%
\pgfpathcurveto{\pgfqpoint{2.453813in}{2.254016in}}{\pgfqpoint{2.464412in}{2.249626in}}{\pgfqpoint{2.475462in}{2.249626in}}%
\pgfpathclose%
\pgfusepath{stroke,fill}%
\end{pgfscope}%
\begin{pgfscope}%
\pgfpathrectangle{\pgfqpoint{0.481978in}{0.331635in}}{\pgfqpoint{4.960000in}{3.696000in}}%
\pgfusepath{clip}%
\pgfsetbuttcap%
\pgfsetroundjoin%
\definecolor{currentfill}{rgb}{0.631373,0.788235,0.956863}%
\pgfsetfillcolor{currentfill}%
\pgfsetlinewidth{0.481800pt}%
\definecolor{currentstroke}{rgb}{1.000000,1.000000,1.000000}%
\pgfsetstrokecolor{currentstroke}%
\pgfsetdash{}{0pt}%
\pgfpathmoveto{\pgfqpoint{2.457081in}{1.657079in}}%
\pgfpathcurveto{\pgfqpoint{2.468131in}{1.657079in}}{\pgfqpoint{2.478730in}{1.661469in}}{\pgfqpoint{2.486543in}{1.669283in}}%
\pgfpathcurveto{\pgfqpoint{2.494357in}{1.677097in}}{\pgfqpoint{2.498747in}{1.687696in}}{\pgfqpoint{2.498747in}{1.698746in}}%
\pgfpathcurveto{\pgfqpoint{2.498747in}{1.709796in}}{\pgfqpoint{2.494357in}{1.720395in}}{\pgfqpoint{2.486543in}{1.728209in}}%
\pgfpathcurveto{\pgfqpoint{2.478730in}{1.736022in}}{\pgfqpoint{2.468131in}{1.740412in}}{\pgfqpoint{2.457081in}{1.740412in}}%
\pgfpathcurveto{\pgfqpoint{2.446030in}{1.740412in}}{\pgfqpoint{2.435431in}{1.736022in}}{\pgfqpoint{2.427618in}{1.728209in}}%
\pgfpathcurveto{\pgfqpoint{2.419804in}{1.720395in}}{\pgfqpoint{2.415414in}{1.709796in}}{\pgfqpoint{2.415414in}{1.698746in}}%
\pgfpathcurveto{\pgfqpoint{2.415414in}{1.687696in}}{\pgfqpoint{2.419804in}{1.677097in}}{\pgfqpoint{2.427618in}{1.669283in}}%
\pgfpathcurveto{\pgfqpoint{2.435431in}{1.661469in}}{\pgfqpoint{2.446030in}{1.657079in}}{\pgfqpoint{2.457081in}{1.657079in}}%
\pgfpathclose%
\pgfusepath{stroke,fill}%
\end{pgfscope}%
\begin{pgfscope}%
\pgfpathrectangle{\pgfqpoint{0.481978in}{0.331635in}}{\pgfqpoint{4.960000in}{3.696000in}}%
\pgfusepath{clip}%
\pgfsetbuttcap%
\pgfsetroundjoin%
\definecolor{currentfill}{rgb}{0.631373,0.788235,0.956863}%
\pgfsetfillcolor{currentfill}%
\pgfsetlinewidth{0.481800pt}%
\definecolor{currentstroke}{rgb}{1.000000,1.000000,1.000000}%
\pgfsetstrokecolor{currentstroke}%
\pgfsetdash{}{0pt}%
\pgfpathmoveto{\pgfqpoint{2.810443in}{1.484448in}}%
\pgfpathcurveto{\pgfqpoint{2.821493in}{1.484448in}}{\pgfqpoint{2.832092in}{1.488838in}}{\pgfqpoint{2.839906in}{1.496651in}}%
\pgfpathcurveto{\pgfqpoint{2.847719in}{1.504465in}}{\pgfqpoint{2.852110in}{1.515064in}}{\pgfqpoint{2.852110in}{1.526114in}}%
\pgfpathcurveto{\pgfqpoint{2.852110in}{1.537164in}}{\pgfqpoint{2.847719in}{1.547763in}}{\pgfqpoint{2.839906in}{1.555577in}}%
\pgfpathcurveto{\pgfqpoint{2.832092in}{1.563391in}}{\pgfqpoint{2.821493in}{1.567781in}}{\pgfqpoint{2.810443in}{1.567781in}}%
\pgfpathcurveto{\pgfqpoint{2.799393in}{1.567781in}}{\pgfqpoint{2.788794in}{1.563391in}}{\pgfqpoint{2.780980in}{1.555577in}}%
\pgfpathcurveto{\pgfqpoint{2.773166in}{1.547763in}}{\pgfqpoint{2.768776in}{1.537164in}}{\pgfqpoint{2.768776in}{1.526114in}}%
\pgfpathcurveto{\pgfqpoint{2.768776in}{1.515064in}}{\pgfqpoint{2.773166in}{1.504465in}}{\pgfqpoint{2.780980in}{1.496651in}}%
\pgfpathcurveto{\pgfqpoint{2.788794in}{1.488838in}}{\pgfqpoint{2.799393in}{1.484448in}}{\pgfqpoint{2.810443in}{1.484448in}}%
\pgfpathclose%
\pgfusepath{stroke,fill}%
\end{pgfscope}%
\begin{pgfscope}%
\pgfpathrectangle{\pgfqpoint{0.481978in}{0.331635in}}{\pgfqpoint{4.960000in}{3.696000in}}%
\pgfusepath{clip}%
\pgfsetbuttcap%
\pgfsetroundjoin%
\definecolor{currentfill}{rgb}{0.631373,0.788235,0.956863}%
\pgfsetfillcolor{currentfill}%
\pgfsetlinewidth{0.481800pt}%
\definecolor{currentstroke}{rgb}{1.000000,1.000000,1.000000}%
\pgfsetstrokecolor{currentstroke}%
\pgfsetdash{}{0pt}%
\pgfpathmoveto{\pgfqpoint{3.346294in}{2.533383in}}%
\pgfpathcurveto{\pgfqpoint{3.357344in}{2.533383in}}{\pgfqpoint{3.367943in}{2.537774in}}{\pgfqpoint{3.375756in}{2.545587in}}%
\pgfpathcurveto{\pgfqpoint{3.383570in}{2.553401in}}{\pgfqpoint{3.387960in}{2.564000in}}{\pgfqpoint{3.387960in}{2.575050in}}%
\pgfpathcurveto{\pgfqpoint{3.387960in}{2.586100in}}{\pgfqpoint{3.383570in}{2.596699in}}{\pgfqpoint{3.375756in}{2.604513in}}%
\pgfpathcurveto{\pgfqpoint{3.367943in}{2.612326in}}{\pgfqpoint{3.357344in}{2.616717in}}{\pgfqpoint{3.346294in}{2.616717in}}%
\pgfpathcurveto{\pgfqpoint{3.335243in}{2.616717in}}{\pgfqpoint{3.324644in}{2.612326in}}{\pgfqpoint{3.316831in}{2.604513in}}%
\pgfpathcurveto{\pgfqpoint{3.309017in}{2.596699in}}{\pgfqpoint{3.304627in}{2.586100in}}{\pgfqpoint{3.304627in}{2.575050in}}%
\pgfpathcurveto{\pgfqpoint{3.304627in}{2.564000in}}{\pgfqpoint{3.309017in}{2.553401in}}{\pgfqpoint{3.316831in}{2.545587in}}%
\pgfpathcurveto{\pgfqpoint{3.324644in}{2.537774in}}{\pgfqpoint{3.335243in}{2.533383in}}{\pgfqpoint{3.346294in}{2.533383in}}%
\pgfpathclose%
\pgfusepath{stroke,fill}%
\end{pgfscope}%
\begin{pgfscope}%
\pgfpathrectangle{\pgfqpoint{0.481978in}{0.331635in}}{\pgfqpoint{4.960000in}{3.696000in}}%
\pgfusepath{clip}%
\pgfsetbuttcap%
\pgfsetroundjoin%
\definecolor{currentfill}{rgb}{0.631373,0.788235,0.956863}%
\pgfsetfillcolor{currentfill}%
\pgfsetlinewidth{0.481800pt}%
\definecolor{currentstroke}{rgb}{1.000000,1.000000,1.000000}%
\pgfsetstrokecolor{currentstroke}%
\pgfsetdash{}{0pt}%
\pgfpathmoveto{\pgfqpoint{2.257440in}{2.502028in}}%
\pgfpathcurveto{\pgfqpoint{2.268490in}{2.502028in}}{\pgfqpoint{2.279089in}{2.506418in}}{\pgfqpoint{2.286903in}{2.514231in}}%
\pgfpathcurveto{\pgfqpoint{2.294717in}{2.522045in}}{\pgfqpoint{2.299107in}{2.532644in}}{\pgfqpoint{2.299107in}{2.543694in}}%
\pgfpathcurveto{\pgfqpoint{2.299107in}{2.554744in}}{\pgfqpoint{2.294717in}{2.565343in}}{\pgfqpoint{2.286903in}{2.573157in}}%
\pgfpathcurveto{\pgfqpoint{2.279089in}{2.580971in}}{\pgfqpoint{2.268490in}{2.585361in}}{\pgfqpoint{2.257440in}{2.585361in}}%
\pgfpathcurveto{\pgfqpoint{2.246390in}{2.585361in}}{\pgfqpoint{2.235791in}{2.580971in}}{\pgfqpoint{2.227978in}{2.573157in}}%
\pgfpathcurveto{\pgfqpoint{2.220164in}{2.565343in}}{\pgfqpoint{2.215774in}{2.554744in}}{\pgfqpoint{2.215774in}{2.543694in}}%
\pgfpathcurveto{\pgfqpoint{2.215774in}{2.532644in}}{\pgfqpoint{2.220164in}{2.522045in}}{\pgfqpoint{2.227978in}{2.514231in}}%
\pgfpathcurveto{\pgfqpoint{2.235791in}{2.506418in}}{\pgfqpoint{2.246390in}{2.502028in}}{\pgfqpoint{2.257440in}{2.502028in}}%
\pgfpathclose%
\pgfusepath{stroke,fill}%
\end{pgfscope}%
\begin{pgfscope}%
\pgfpathrectangle{\pgfqpoint{0.481978in}{0.331635in}}{\pgfqpoint{4.960000in}{3.696000in}}%
\pgfusepath{clip}%
\pgfsetbuttcap%
\pgfsetroundjoin%
\definecolor{currentfill}{rgb}{0.631373,0.788235,0.956863}%
\pgfsetfillcolor{currentfill}%
\pgfsetlinewidth{0.481800pt}%
\definecolor{currentstroke}{rgb}{1.000000,1.000000,1.000000}%
\pgfsetstrokecolor{currentstroke}%
\pgfsetdash{}{0pt}%
\pgfpathmoveto{\pgfqpoint{3.149125in}{1.968639in}}%
\pgfpathcurveto{\pgfqpoint{3.160175in}{1.968639in}}{\pgfqpoint{3.170774in}{1.973030in}}{\pgfqpoint{3.178588in}{1.980843in}}%
\pgfpathcurveto{\pgfqpoint{3.186401in}{1.988657in}}{\pgfqpoint{3.190792in}{1.999256in}}{\pgfqpoint{3.190792in}{2.010306in}}%
\pgfpathcurveto{\pgfqpoint{3.190792in}{2.021356in}}{\pgfqpoint{3.186401in}{2.031955in}}{\pgfqpoint{3.178588in}{2.039769in}}%
\pgfpathcurveto{\pgfqpoint{3.170774in}{2.047582in}}{\pgfqpoint{3.160175in}{2.051973in}}{\pgfqpoint{3.149125in}{2.051973in}}%
\pgfpathcurveto{\pgfqpoint{3.138075in}{2.051973in}}{\pgfqpoint{3.127476in}{2.047582in}}{\pgfqpoint{3.119662in}{2.039769in}}%
\pgfpathcurveto{\pgfqpoint{3.111849in}{2.031955in}}{\pgfqpoint{3.107458in}{2.021356in}}{\pgfqpoint{3.107458in}{2.010306in}}%
\pgfpathcurveto{\pgfqpoint{3.107458in}{1.999256in}}{\pgfqpoint{3.111849in}{1.988657in}}{\pgfqpoint{3.119662in}{1.980843in}}%
\pgfpathcurveto{\pgfqpoint{3.127476in}{1.973030in}}{\pgfqpoint{3.138075in}{1.968639in}}{\pgfqpoint{3.149125in}{1.968639in}}%
\pgfpathclose%
\pgfusepath{stroke,fill}%
\end{pgfscope}%
\begin{pgfscope}%
\pgfpathrectangle{\pgfqpoint{0.481978in}{0.331635in}}{\pgfqpoint{4.960000in}{3.696000in}}%
\pgfusepath{clip}%
\pgfsetbuttcap%
\pgfsetroundjoin%
\definecolor{currentfill}{rgb}{0.631373,0.788235,0.956863}%
\pgfsetfillcolor{currentfill}%
\pgfsetlinewidth{0.481800pt}%
\definecolor{currentstroke}{rgb}{1.000000,1.000000,1.000000}%
\pgfsetstrokecolor{currentstroke}%
\pgfsetdash{}{0pt}%
\pgfpathmoveto{\pgfqpoint{4.291538in}{1.907302in}}%
\pgfpathcurveto{\pgfqpoint{4.302588in}{1.907302in}}{\pgfqpoint{4.313187in}{1.911692in}}{\pgfqpoint{4.321001in}{1.919506in}}%
\pgfpathcurveto{\pgfqpoint{4.328815in}{1.927319in}}{\pgfqpoint{4.333205in}{1.937918in}}{\pgfqpoint{4.333205in}{1.948968in}}%
\pgfpathcurveto{\pgfqpoint{4.333205in}{1.960018in}}{\pgfqpoint{4.328815in}{1.970618in}}{\pgfqpoint{4.321001in}{1.978431in}}%
\pgfpathcurveto{\pgfqpoint{4.313187in}{1.986245in}}{\pgfqpoint{4.302588in}{1.990635in}}{\pgfqpoint{4.291538in}{1.990635in}}%
\pgfpathcurveto{\pgfqpoint{4.280488in}{1.990635in}}{\pgfqpoint{4.269889in}{1.986245in}}{\pgfqpoint{4.262076in}{1.978431in}}%
\pgfpathcurveto{\pgfqpoint{4.254262in}{1.970618in}}{\pgfqpoint{4.249872in}{1.960018in}}{\pgfqpoint{4.249872in}{1.948968in}}%
\pgfpathcurveto{\pgfqpoint{4.249872in}{1.937918in}}{\pgfqpoint{4.254262in}{1.927319in}}{\pgfqpoint{4.262076in}{1.919506in}}%
\pgfpathcurveto{\pgfqpoint{4.269889in}{1.911692in}}{\pgfqpoint{4.280488in}{1.907302in}}{\pgfqpoint{4.291538in}{1.907302in}}%
\pgfpathclose%
\pgfusepath{stroke,fill}%
\end{pgfscope}%
\begin{pgfscope}%
\pgfpathrectangle{\pgfqpoint{0.481978in}{0.331635in}}{\pgfqpoint{4.960000in}{3.696000in}}%
\pgfusepath{clip}%
\pgfsetbuttcap%
\pgfsetroundjoin%
\definecolor{currentfill}{rgb}{0.631373,0.788235,0.956863}%
\pgfsetfillcolor{currentfill}%
\pgfsetlinewidth{0.481800pt}%
\definecolor{currentstroke}{rgb}{1.000000,1.000000,1.000000}%
\pgfsetstrokecolor{currentstroke}%
\pgfsetdash{}{0pt}%
\pgfpathmoveto{\pgfqpoint{2.299822in}{2.617076in}}%
\pgfpathcurveto{\pgfqpoint{2.310872in}{2.617076in}}{\pgfqpoint{2.321471in}{2.621467in}}{\pgfqpoint{2.329285in}{2.629280in}}%
\pgfpathcurveto{\pgfqpoint{2.337099in}{2.637094in}}{\pgfqpoint{2.341489in}{2.647693in}}{\pgfqpoint{2.341489in}{2.658743in}}%
\pgfpathcurveto{\pgfqpoint{2.341489in}{2.669793in}}{\pgfqpoint{2.337099in}{2.680392in}}{\pgfqpoint{2.329285in}{2.688206in}}%
\pgfpathcurveto{\pgfqpoint{2.321471in}{2.696019in}}{\pgfqpoint{2.310872in}{2.700410in}}{\pgfqpoint{2.299822in}{2.700410in}}%
\pgfpathcurveto{\pgfqpoint{2.288772in}{2.700410in}}{\pgfqpoint{2.278173in}{2.696019in}}{\pgfqpoint{2.270359in}{2.688206in}}%
\pgfpathcurveto{\pgfqpoint{2.262546in}{2.680392in}}{\pgfqpoint{2.258156in}{2.669793in}}{\pgfqpoint{2.258156in}{2.658743in}}%
\pgfpathcurveto{\pgfqpoint{2.258156in}{2.647693in}}{\pgfqpoint{2.262546in}{2.637094in}}{\pgfqpoint{2.270359in}{2.629280in}}%
\pgfpathcurveto{\pgfqpoint{2.278173in}{2.621467in}}{\pgfqpoint{2.288772in}{2.617076in}}{\pgfqpoint{2.299822in}{2.617076in}}%
\pgfpathclose%
\pgfusepath{stroke,fill}%
\end{pgfscope}%
\begin{pgfscope}%
\pgfpathrectangle{\pgfqpoint{0.481978in}{0.331635in}}{\pgfqpoint{4.960000in}{3.696000in}}%
\pgfusepath{clip}%
\pgfsetbuttcap%
\pgfsetroundjoin%
\definecolor{currentfill}{rgb}{0.631373,0.788235,0.956863}%
\pgfsetfillcolor{currentfill}%
\pgfsetlinewidth{0.481800pt}%
\definecolor{currentstroke}{rgb}{1.000000,1.000000,1.000000}%
\pgfsetstrokecolor{currentstroke}%
\pgfsetdash{}{0pt}%
\pgfpathmoveto{\pgfqpoint{2.768417in}{2.056598in}}%
\pgfpathcurveto{\pgfqpoint{2.779467in}{2.056598in}}{\pgfqpoint{2.790066in}{2.060989in}}{\pgfqpoint{2.797880in}{2.068802in}}%
\pgfpathcurveto{\pgfqpoint{2.805693in}{2.076616in}}{\pgfqpoint{2.810084in}{2.087215in}}{\pgfqpoint{2.810084in}{2.098265in}}%
\pgfpathcurveto{\pgfqpoint{2.810084in}{2.109315in}}{\pgfqpoint{2.805693in}{2.119914in}}{\pgfqpoint{2.797880in}{2.127728in}}%
\pgfpathcurveto{\pgfqpoint{2.790066in}{2.135541in}}{\pgfqpoint{2.779467in}{2.139932in}}{\pgfqpoint{2.768417in}{2.139932in}}%
\pgfpathcurveto{\pgfqpoint{2.757367in}{2.139932in}}{\pgfqpoint{2.746768in}{2.135541in}}{\pgfqpoint{2.738954in}{2.127728in}}%
\pgfpathcurveto{\pgfqpoint{2.731140in}{2.119914in}}{\pgfqpoint{2.726750in}{2.109315in}}{\pgfqpoint{2.726750in}{2.098265in}}%
\pgfpathcurveto{\pgfqpoint{2.726750in}{2.087215in}}{\pgfqpoint{2.731140in}{2.076616in}}{\pgfqpoint{2.738954in}{2.068802in}}%
\pgfpathcurveto{\pgfqpoint{2.746768in}{2.060989in}}{\pgfqpoint{2.757367in}{2.056598in}}{\pgfqpoint{2.768417in}{2.056598in}}%
\pgfpathclose%
\pgfusepath{stroke,fill}%
\end{pgfscope}%
\begin{pgfscope}%
\pgfpathrectangle{\pgfqpoint{0.481978in}{0.331635in}}{\pgfqpoint{4.960000in}{3.696000in}}%
\pgfusepath{clip}%
\pgfsetbuttcap%
\pgfsetroundjoin%
\definecolor{currentfill}{rgb}{0.631373,0.788235,0.956863}%
\pgfsetfillcolor{currentfill}%
\pgfsetlinewidth{0.481800pt}%
\definecolor{currentstroke}{rgb}{1.000000,1.000000,1.000000}%
\pgfsetstrokecolor{currentstroke}%
\pgfsetdash{}{0pt}%
\pgfpathmoveto{\pgfqpoint{2.195580in}{1.880246in}}%
\pgfpathcurveto{\pgfqpoint{2.206631in}{1.880246in}}{\pgfqpoint{2.217230in}{1.884636in}}{\pgfqpoint{2.225043in}{1.892450in}}%
\pgfpathcurveto{\pgfqpoint{2.232857in}{1.900264in}}{\pgfqpoint{2.237247in}{1.910863in}}{\pgfqpoint{2.237247in}{1.921913in}}%
\pgfpathcurveto{\pgfqpoint{2.237247in}{1.932963in}}{\pgfqpoint{2.232857in}{1.943562in}}{\pgfqpoint{2.225043in}{1.951376in}}%
\pgfpathcurveto{\pgfqpoint{2.217230in}{1.959189in}}{\pgfqpoint{2.206631in}{1.963579in}}{\pgfqpoint{2.195580in}{1.963579in}}%
\pgfpathcurveto{\pgfqpoint{2.184530in}{1.963579in}}{\pgfqpoint{2.173931in}{1.959189in}}{\pgfqpoint{2.166118in}{1.951376in}}%
\pgfpathcurveto{\pgfqpoint{2.158304in}{1.943562in}}{\pgfqpoint{2.153914in}{1.932963in}}{\pgfqpoint{2.153914in}{1.921913in}}%
\pgfpathcurveto{\pgfqpoint{2.153914in}{1.910863in}}{\pgfqpoint{2.158304in}{1.900264in}}{\pgfqpoint{2.166118in}{1.892450in}}%
\pgfpathcurveto{\pgfqpoint{2.173931in}{1.884636in}}{\pgfqpoint{2.184530in}{1.880246in}}{\pgfqpoint{2.195580in}{1.880246in}}%
\pgfpathclose%
\pgfusepath{stroke,fill}%
\end{pgfscope}%
\begin{pgfscope}%
\pgfpathrectangle{\pgfqpoint{0.481978in}{0.331635in}}{\pgfqpoint{4.960000in}{3.696000in}}%
\pgfusepath{clip}%
\pgfsetbuttcap%
\pgfsetroundjoin%
\definecolor{currentfill}{rgb}{0.631373,0.788235,0.956863}%
\pgfsetfillcolor{currentfill}%
\pgfsetlinewidth{0.481800pt}%
\definecolor{currentstroke}{rgb}{1.000000,1.000000,1.000000}%
\pgfsetstrokecolor{currentstroke}%
\pgfsetdash{}{0pt}%
\pgfpathmoveto{\pgfqpoint{1.746017in}{2.410775in}}%
\pgfpathcurveto{\pgfqpoint{1.757067in}{2.410775in}}{\pgfqpoint{1.767666in}{2.415165in}}{\pgfqpoint{1.775480in}{2.422979in}}%
\pgfpathcurveto{\pgfqpoint{1.783293in}{2.430792in}}{\pgfqpoint{1.787684in}{2.441391in}}{\pgfqpoint{1.787684in}{2.452441in}}%
\pgfpathcurveto{\pgfqpoint{1.787684in}{2.463491in}}{\pgfqpoint{1.783293in}{2.474090in}}{\pgfqpoint{1.775480in}{2.481904in}}%
\pgfpathcurveto{\pgfqpoint{1.767666in}{2.489718in}}{\pgfqpoint{1.757067in}{2.494108in}}{\pgfqpoint{1.746017in}{2.494108in}}%
\pgfpathcurveto{\pgfqpoint{1.734967in}{2.494108in}}{\pgfqpoint{1.724368in}{2.489718in}}{\pgfqpoint{1.716554in}{2.481904in}}%
\pgfpathcurveto{\pgfqpoint{1.708741in}{2.474090in}}{\pgfqpoint{1.704350in}{2.463491in}}{\pgfqpoint{1.704350in}{2.452441in}}%
\pgfpathcurveto{\pgfqpoint{1.704350in}{2.441391in}}{\pgfqpoint{1.708741in}{2.430792in}}{\pgfqpoint{1.716554in}{2.422979in}}%
\pgfpathcurveto{\pgfqpoint{1.724368in}{2.415165in}}{\pgfqpoint{1.734967in}{2.410775in}}{\pgfqpoint{1.746017in}{2.410775in}}%
\pgfpathclose%
\pgfusepath{stroke,fill}%
\end{pgfscope}%
\begin{pgfscope}%
\pgfpathrectangle{\pgfqpoint{0.481978in}{0.331635in}}{\pgfqpoint{4.960000in}{3.696000in}}%
\pgfusepath{clip}%
\pgfsetbuttcap%
\pgfsetroundjoin%
\definecolor{currentfill}{rgb}{0.631373,0.788235,0.956863}%
\pgfsetfillcolor{currentfill}%
\pgfsetlinewidth{0.481800pt}%
\definecolor{currentstroke}{rgb}{1.000000,1.000000,1.000000}%
\pgfsetstrokecolor{currentstroke}%
\pgfsetdash{}{0pt}%
\pgfpathmoveto{\pgfqpoint{2.652920in}{2.464691in}}%
\pgfpathcurveto{\pgfqpoint{2.663971in}{2.464691in}}{\pgfqpoint{2.674570in}{2.469081in}}{\pgfqpoint{2.682383in}{2.476895in}}%
\pgfpathcurveto{\pgfqpoint{2.690197in}{2.484709in}}{\pgfqpoint{2.694587in}{2.495308in}}{\pgfqpoint{2.694587in}{2.506358in}}%
\pgfpathcurveto{\pgfqpoint{2.694587in}{2.517408in}}{\pgfqpoint{2.690197in}{2.528007in}}{\pgfqpoint{2.682383in}{2.535821in}}%
\pgfpathcurveto{\pgfqpoint{2.674570in}{2.543634in}}{\pgfqpoint{2.663971in}{2.548024in}}{\pgfqpoint{2.652920in}{2.548024in}}%
\pgfpathcurveto{\pgfqpoint{2.641870in}{2.548024in}}{\pgfqpoint{2.631271in}{2.543634in}}{\pgfqpoint{2.623458in}{2.535821in}}%
\pgfpathcurveto{\pgfqpoint{2.615644in}{2.528007in}}{\pgfqpoint{2.611254in}{2.517408in}}{\pgfqpoint{2.611254in}{2.506358in}}%
\pgfpathcurveto{\pgfqpoint{2.611254in}{2.495308in}}{\pgfqpoint{2.615644in}{2.484709in}}{\pgfqpoint{2.623458in}{2.476895in}}%
\pgfpathcurveto{\pgfqpoint{2.631271in}{2.469081in}}{\pgfqpoint{2.641870in}{2.464691in}}{\pgfqpoint{2.652920in}{2.464691in}}%
\pgfpathclose%
\pgfusepath{stroke,fill}%
\end{pgfscope}%
\begin{pgfscope}%
\pgfpathrectangle{\pgfqpoint{0.481978in}{0.331635in}}{\pgfqpoint{4.960000in}{3.696000in}}%
\pgfusepath{clip}%
\pgfsetbuttcap%
\pgfsetroundjoin%
\definecolor{currentfill}{rgb}{0.631373,0.788235,0.956863}%
\pgfsetfillcolor{currentfill}%
\pgfsetlinewidth{0.481800pt}%
\definecolor{currentstroke}{rgb}{1.000000,1.000000,1.000000}%
\pgfsetstrokecolor{currentstroke}%
\pgfsetdash{}{0pt}%
\pgfpathmoveto{\pgfqpoint{2.837731in}{3.692181in}}%
\pgfpathcurveto{\pgfqpoint{2.848782in}{3.692181in}}{\pgfqpoint{2.859381in}{3.696572in}}{\pgfqpoint{2.867194in}{3.704385in}}%
\pgfpathcurveto{\pgfqpoint{2.875008in}{3.712199in}}{\pgfqpoint{2.879398in}{3.722798in}}{\pgfqpoint{2.879398in}{3.733848in}}%
\pgfpathcurveto{\pgfqpoint{2.879398in}{3.744898in}}{\pgfqpoint{2.875008in}{3.755497in}}{\pgfqpoint{2.867194in}{3.763311in}}%
\pgfpathcurveto{\pgfqpoint{2.859381in}{3.771125in}}{\pgfqpoint{2.848782in}{3.775515in}}{\pgfqpoint{2.837731in}{3.775515in}}%
\pgfpathcurveto{\pgfqpoint{2.826681in}{3.775515in}}{\pgfqpoint{2.816082in}{3.771125in}}{\pgfqpoint{2.808269in}{3.763311in}}%
\pgfpathcurveto{\pgfqpoint{2.800455in}{3.755497in}}{\pgfqpoint{2.796065in}{3.744898in}}{\pgfqpoint{2.796065in}{3.733848in}}%
\pgfpathcurveto{\pgfqpoint{2.796065in}{3.722798in}}{\pgfqpoint{2.800455in}{3.712199in}}{\pgfqpoint{2.808269in}{3.704385in}}%
\pgfpathcurveto{\pgfqpoint{2.816082in}{3.696572in}}{\pgfqpoint{2.826681in}{3.692181in}}{\pgfqpoint{2.837731in}{3.692181in}}%
\pgfpathclose%
\pgfusepath{stroke,fill}%
\end{pgfscope}%
\begin{pgfscope}%
\pgfpathrectangle{\pgfqpoint{0.481978in}{0.331635in}}{\pgfqpoint{4.960000in}{3.696000in}}%
\pgfusepath{clip}%
\pgfsetbuttcap%
\pgfsetroundjoin%
\definecolor{currentfill}{rgb}{0.631373,0.788235,0.956863}%
\pgfsetfillcolor{currentfill}%
\pgfsetlinewidth{0.481800pt}%
\definecolor{currentstroke}{rgb}{1.000000,1.000000,1.000000}%
\pgfsetstrokecolor{currentstroke}%
\pgfsetdash{}{0pt}%
\pgfpathmoveto{\pgfqpoint{3.329764in}{1.476993in}}%
\pgfpathcurveto{\pgfqpoint{3.340814in}{1.476993in}}{\pgfqpoint{3.351413in}{1.481383in}}{\pgfqpoint{3.359226in}{1.489196in}}%
\pgfpathcurveto{\pgfqpoint{3.367040in}{1.497010in}}{\pgfqpoint{3.371430in}{1.507609in}}{\pgfqpoint{3.371430in}{1.518659in}}%
\pgfpathcurveto{\pgfqpoint{3.371430in}{1.529709in}}{\pgfqpoint{3.367040in}{1.540308in}}{\pgfqpoint{3.359226in}{1.548122in}}%
\pgfpathcurveto{\pgfqpoint{3.351413in}{1.555936in}}{\pgfqpoint{3.340814in}{1.560326in}}{\pgfqpoint{3.329764in}{1.560326in}}%
\pgfpathcurveto{\pgfqpoint{3.318713in}{1.560326in}}{\pgfqpoint{3.308114in}{1.555936in}}{\pgfqpoint{3.300301in}{1.548122in}}%
\pgfpathcurveto{\pgfqpoint{3.292487in}{1.540308in}}{\pgfqpoint{3.288097in}{1.529709in}}{\pgfqpoint{3.288097in}{1.518659in}}%
\pgfpathcurveto{\pgfqpoint{3.288097in}{1.507609in}}{\pgfqpoint{3.292487in}{1.497010in}}{\pgfqpoint{3.300301in}{1.489196in}}%
\pgfpathcurveto{\pgfqpoint{3.308114in}{1.481383in}}{\pgfqpoint{3.318713in}{1.476993in}}{\pgfqpoint{3.329764in}{1.476993in}}%
\pgfpathclose%
\pgfusepath{stroke,fill}%
\end{pgfscope}%
\begin{pgfscope}%
\pgfpathrectangle{\pgfqpoint{0.481978in}{0.331635in}}{\pgfqpoint{4.960000in}{3.696000in}}%
\pgfusepath{clip}%
\pgfsetbuttcap%
\pgfsetroundjoin%
\definecolor{currentfill}{rgb}{0.631373,0.788235,0.956863}%
\pgfsetfillcolor{currentfill}%
\pgfsetlinewidth{0.481800pt}%
\definecolor{currentstroke}{rgb}{1.000000,1.000000,1.000000}%
\pgfsetstrokecolor{currentstroke}%
\pgfsetdash{}{0pt}%
\pgfpathmoveto{\pgfqpoint{1.917096in}{1.961250in}}%
\pgfpathcurveto{\pgfqpoint{1.928146in}{1.961250in}}{\pgfqpoint{1.938745in}{1.965640in}}{\pgfqpoint{1.946559in}{1.973453in}}%
\pgfpathcurveto{\pgfqpoint{1.954372in}{1.981267in}}{\pgfqpoint{1.958763in}{1.991866in}}{\pgfqpoint{1.958763in}{2.002916in}}%
\pgfpathcurveto{\pgfqpoint{1.958763in}{2.013966in}}{\pgfqpoint{1.954372in}{2.024565in}}{\pgfqpoint{1.946559in}{2.032379in}}%
\pgfpathcurveto{\pgfqpoint{1.938745in}{2.040193in}}{\pgfqpoint{1.928146in}{2.044583in}}{\pgfqpoint{1.917096in}{2.044583in}}%
\pgfpathcurveto{\pgfqpoint{1.906046in}{2.044583in}}{\pgfqpoint{1.895447in}{2.040193in}}{\pgfqpoint{1.887633in}{2.032379in}}%
\pgfpathcurveto{\pgfqpoint{1.879820in}{2.024565in}}{\pgfqpoint{1.875429in}{2.013966in}}{\pgfqpoint{1.875429in}{2.002916in}}%
\pgfpathcurveto{\pgfqpoint{1.875429in}{1.991866in}}{\pgfqpoint{1.879820in}{1.981267in}}{\pgfqpoint{1.887633in}{1.973453in}}%
\pgfpathcurveto{\pgfqpoint{1.895447in}{1.965640in}}{\pgfqpoint{1.906046in}{1.961250in}}{\pgfqpoint{1.917096in}{1.961250in}}%
\pgfpathclose%
\pgfusepath{stroke,fill}%
\end{pgfscope}%
\begin{pgfscope}%
\pgfpathrectangle{\pgfqpoint{0.481978in}{0.331635in}}{\pgfqpoint{4.960000in}{3.696000in}}%
\pgfusepath{clip}%
\pgfsetbuttcap%
\pgfsetroundjoin%
\definecolor{currentfill}{rgb}{0.631373,0.788235,0.956863}%
\pgfsetfillcolor{currentfill}%
\pgfsetlinewidth{0.481800pt}%
\definecolor{currentstroke}{rgb}{1.000000,1.000000,1.000000}%
\pgfsetstrokecolor{currentstroke}%
\pgfsetdash{}{0pt}%
\pgfpathmoveto{\pgfqpoint{2.723722in}{2.153697in}}%
\pgfpathcurveto{\pgfqpoint{2.734772in}{2.153697in}}{\pgfqpoint{2.745371in}{2.158087in}}{\pgfqpoint{2.753184in}{2.165901in}}%
\pgfpathcurveto{\pgfqpoint{2.760998in}{2.173714in}}{\pgfqpoint{2.765388in}{2.184313in}}{\pgfqpoint{2.765388in}{2.195364in}}%
\pgfpathcurveto{\pgfqpoint{2.765388in}{2.206414in}}{\pgfqpoint{2.760998in}{2.217013in}}{\pgfqpoint{2.753184in}{2.224826in}}%
\pgfpathcurveto{\pgfqpoint{2.745371in}{2.232640in}}{\pgfqpoint{2.734772in}{2.237030in}}{\pgfqpoint{2.723722in}{2.237030in}}%
\pgfpathcurveto{\pgfqpoint{2.712672in}{2.237030in}}{\pgfqpoint{2.702072in}{2.232640in}}{\pgfqpoint{2.694259in}{2.224826in}}%
\pgfpathcurveto{\pgfqpoint{2.686445in}{2.217013in}}{\pgfqpoint{2.682055in}{2.206414in}}{\pgfqpoint{2.682055in}{2.195364in}}%
\pgfpathcurveto{\pgfqpoint{2.682055in}{2.184313in}}{\pgfqpoint{2.686445in}{2.173714in}}{\pgfqpoint{2.694259in}{2.165901in}}%
\pgfpathcurveto{\pgfqpoint{2.702072in}{2.158087in}}{\pgfqpoint{2.712672in}{2.153697in}}{\pgfqpoint{2.723722in}{2.153697in}}%
\pgfpathclose%
\pgfusepath{stroke,fill}%
\end{pgfscope}%
\begin{pgfscope}%
\pgfpathrectangle{\pgfqpoint{0.481978in}{0.331635in}}{\pgfqpoint{4.960000in}{3.696000in}}%
\pgfusepath{clip}%
\pgfsetbuttcap%
\pgfsetroundjoin%
\definecolor{currentfill}{rgb}{0.631373,0.788235,0.956863}%
\pgfsetfillcolor{currentfill}%
\pgfsetlinewidth{0.481800pt}%
\definecolor{currentstroke}{rgb}{1.000000,1.000000,1.000000}%
\pgfsetstrokecolor{currentstroke}%
\pgfsetdash{}{0pt}%
\pgfpathmoveto{\pgfqpoint{1.791703in}{1.782544in}}%
\pgfpathcurveto{\pgfqpoint{1.802753in}{1.782544in}}{\pgfqpoint{1.813352in}{1.786934in}}{\pgfqpoint{1.821166in}{1.794748in}}%
\pgfpathcurveto{\pgfqpoint{1.828979in}{1.802561in}}{\pgfqpoint{1.833370in}{1.813160in}}{\pgfqpoint{1.833370in}{1.824210in}}%
\pgfpathcurveto{\pgfqpoint{1.833370in}{1.835260in}}{\pgfqpoint{1.828979in}{1.845860in}}{\pgfqpoint{1.821166in}{1.853673in}}%
\pgfpathcurveto{\pgfqpoint{1.813352in}{1.861487in}}{\pgfqpoint{1.802753in}{1.865877in}}{\pgfqpoint{1.791703in}{1.865877in}}%
\pgfpathcurveto{\pgfqpoint{1.780653in}{1.865877in}}{\pgfqpoint{1.770054in}{1.861487in}}{\pgfqpoint{1.762240in}{1.853673in}}%
\pgfpathcurveto{\pgfqpoint{1.754427in}{1.845860in}}{\pgfqpoint{1.750036in}{1.835260in}}{\pgfqpoint{1.750036in}{1.824210in}}%
\pgfpathcurveto{\pgfqpoint{1.750036in}{1.813160in}}{\pgfqpoint{1.754427in}{1.802561in}}{\pgfqpoint{1.762240in}{1.794748in}}%
\pgfpathcurveto{\pgfqpoint{1.770054in}{1.786934in}}{\pgfqpoint{1.780653in}{1.782544in}}{\pgfqpoint{1.791703in}{1.782544in}}%
\pgfpathclose%
\pgfusepath{stroke,fill}%
\end{pgfscope}%
\begin{pgfscope}%
\pgfpathrectangle{\pgfqpoint{0.481978in}{0.331635in}}{\pgfqpoint{4.960000in}{3.696000in}}%
\pgfusepath{clip}%
\pgfsetbuttcap%
\pgfsetroundjoin%
\definecolor{currentfill}{rgb}{0.631373,0.788235,0.956863}%
\pgfsetfillcolor{currentfill}%
\pgfsetlinewidth{0.481800pt}%
\definecolor{currentstroke}{rgb}{1.000000,1.000000,1.000000}%
\pgfsetstrokecolor{currentstroke}%
\pgfsetdash{}{0pt}%
\pgfpathmoveto{\pgfqpoint{2.820654in}{2.495890in}}%
\pgfpathcurveto{\pgfqpoint{2.831704in}{2.495890in}}{\pgfqpoint{2.842303in}{2.500281in}}{\pgfqpoint{2.850117in}{2.508094in}}%
\pgfpathcurveto{\pgfqpoint{2.857931in}{2.515908in}}{\pgfqpoint{2.862321in}{2.526507in}}{\pgfqpoint{2.862321in}{2.537557in}}%
\pgfpathcurveto{\pgfqpoint{2.862321in}{2.548607in}}{\pgfqpoint{2.857931in}{2.559206in}}{\pgfqpoint{2.850117in}{2.567020in}}%
\pgfpathcurveto{\pgfqpoint{2.842303in}{2.574833in}}{\pgfqpoint{2.831704in}{2.579224in}}{\pgfqpoint{2.820654in}{2.579224in}}%
\pgfpathcurveto{\pgfqpoint{2.809604in}{2.579224in}}{\pgfqpoint{2.799005in}{2.574833in}}{\pgfqpoint{2.791192in}{2.567020in}}%
\pgfpathcurveto{\pgfqpoint{2.783378in}{2.559206in}}{\pgfqpoint{2.778988in}{2.548607in}}{\pgfqpoint{2.778988in}{2.537557in}}%
\pgfpathcurveto{\pgfqpoint{2.778988in}{2.526507in}}{\pgfqpoint{2.783378in}{2.515908in}}{\pgfqpoint{2.791192in}{2.508094in}}%
\pgfpathcurveto{\pgfqpoint{2.799005in}{2.500281in}}{\pgfqpoint{2.809604in}{2.495890in}}{\pgfqpoint{2.820654in}{2.495890in}}%
\pgfpathclose%
\pgfusepath{stroke,fill}%
\end{pgfscope}%
\begin{pgfscope}%
\pgfpathrectangle{\pgfqpoint{0.481978in}{0.331635in}}{\pgfqpoint{4.960000in}{3.696000in}}%
\pgfusepath{clip}%
\pgfsetbuttcap%
\pgfsetroundjoin%
\definecolor{currentfill}{rgb}{0.631373,0.788235,0.956863}%
\pgfsetfillcolor{currentfill}%
\pgfsetlinewidth{0.481800pt}%
\definecolor{currentstroke}{rgb}{1.000000,1.000000,1.000000}%
\pgfsetstrokecolor{currentstroke}%
\pgfsetdash{}{0pt}%
\pgfpathmoveto{\pgfqpoint{3.101095in}{2.087910in}}%
\pgfpathcurveto{\pgfqpoint{3.112145in}{2.087910in}}{\pgfqpoint{3.122744in}{2.092300in}}{\pgfqpoint{3.130558in}{2.100114in}}%
\pgfpathcurveto{\pgfqpoint{3.138371in}{2.107927in}}{\pgfqpoint{3.142762in}{2.118526in}}{\pgfqpoint{3.142762in}{2.129576in}}%
\pgfpathcurveto{\pgfqpoint{3.142762in}{2.140627in}}{\pgfqpoint{3.138371in}{2.151226in}}{\pgfqpoint{3.130558in}{2.159039in}}%
\pgfpathcurveto{\pgfqpoint{3.122744in}{2.166853in}}{\pgfqpoint{3.112145in}{2.171243in}}{\pgfqpoint{3.101095in}{2.171243in}}%
\pgfpathcurveto{\pgfqpoint{3.090045in}{2.171243in}}{\pgfqpoint{3.079446in}{2.166853in}}{\pgfqpoint{3.071632in}{2.159039in}}%
\pgfpathcurveto{\pgfqpoint{3.063819in}{2.151226in}}{\pgfqpoint{3.059428in}{2.140627in}}{\pgfqpoint{3.059428in}{2.129576in}}%
\pgfpathcurveto{\pgfqpoint{3.059428in}{2.118526in}}{\pgfqpoint{3.063819in}{2.107927in}}{\pgfqpoint{3.071632in}{2.100114in}}%
\pgfpathcurveto{\pgfqpoint{3.079446in}{2.092300in}}{\pgfqpoint{3.090045in}{2.087910in}}{\pgfqpoint{3.101095in}{2.087910in}}%
\pgfpathclose%
\pgfusepath{stroke,fill}%
\end{pgfscope}%
\begin{pgfscope}%
\pgfpathrectangle{\pgfqpoint{0.481978in}{0.331635in}}{\pgfqpoint{4.960000in}{3.696000in}}%
\pgfusepath{clip}%
\pgfsetbuttcap%
\pgfsetroundjoin%
\definecolor{currentfill}{rgb}{0.631373,0.788235,0.956863}%
\pgfsetfillcolor{currentfill}%
\pgfsetlinewidth{0.481800pt}%
\definecolor{currentstroke}{rgb}{1.000000,1.000000,1.000000}%
\pgfsetstrokecolor{currentstroke}%
\pgfsetdash{}{0pt}%
\pgfpathmoveto{\pgfqpoint{3.852958in}{2.009978in}}%
\pgfpathcurveto{\pgfqpoint{3.864008in}{2.009978in}}{\pgfqpoint{3.874607in}{2.014368in}}{\pgfqpoint{3.882421in}{2.022182in}}%
\pgfpathcurveto{\pgfqpoint{3.890234in}{2.029995in}}{\pgfqpoint{3.894625in}{2.040594in}}{\pgfqpoint{3.894625in}{2.051644in}}%
\pgfpathcurveto{\pgfqpoint{3.894625in}{2.062695in}}{\pgfqpoint{3.890234in}{2.073294in}}{\pgfqpoint{3.882421in}{2.081107in}}%
\pgfpathcurveto{\pgfqpoint{3.874607in}{2.088921in}}{\pgfqpoint{3.864008in}{2.093311in}}{\pgfqpoint{3.852958in}{2.093311in}}%
\pgfpathcurveto{\pgfqpoint{3.841908in}{2.093311in}}{\pgfqpoint{3.831309in}{2.088921in}}{\pgfqpoint{3.823495in}{2.081107in}}%
\pgfpathcurveto{\pgfqpoint{3.815682in}{2.073294in}}{\pgfqpoint{3.811291in}{2.062695in}}{\pgfqpoint{3.811291in}{2.051644in}}%
\pgfpathcurveto{\pgfqpoint{3.811291in}{2.040594in}}{\pgfqpoint{3.815682in}{2.029995in}}{\pgfqpoint{3.823495in}{2.022182in}}%
\pgfpathcurveto{\pgfqpoint{3.831309in}{2.014368in}}{\pgfqpoint{3.841908in}{2.009978in}}{\pgfqpoint{3.852958in}{2.009978in}}%
\pgfpathclose%
\pgfusepath{stroke,fill}%
\end{pgfscope}%
\begin{pgfscope}%
\pgfpathrectangle{\pgfqpoint{0.481978in}{0.331635in}}{\pgfqpoint{4.960000in}{3.696000in}}%
\pgfusepath{clip}%
\pgfsetbuttcap%
\pgfsetroundjoin%
\definecolor{currentfill}{rgb}{0.631373,0.788235,0.956863}%
\pgfsetfillcolor{currentfill}%
\pgfsetlinewidth{0.481800pt}%
\definecolor{currentstroke}{rgb}{1.000000,1.000000,1.000000}%
\pgfsetstrokecolor{currentstroke}%
\pgfsetdash{}{0pt}%
\pgfpathmoveto{\pgfqpoint{2.967735in}{2.407368in}}%
\pgfpathcurveto{\pgfqpoint{2.978785in}{2.407368in}}{\pgfqpoint{2.989384in}{2.411759in}}{\pgfqpoint{2.997197in}{2.419572in}}%
\pgfpathcurveto{\pgfqpoint{3.005011in}{2.427386in}}{\pgfqpoint{3.009401in}{2.437985in}}{\pgfqpoint{3.009401in}{2.449035in}}%
\pgfpathcurveto{\pgfqpoint{3.009401in}{2.460085in}}{\pgfqpoint{3.005011in}{2.470684in}}{\pgfqpoint{2.997197in}{2.478498in}}%
\pgfpathcurveto{\pgfqpoint{2.989384in}{2.486311in}}{\pgfqpoint{2.978785in}{2.490702in}}{\pgfqpoint{2.967735in}{2.490702in}}%
\pgfpathcurveto{\pgfqpoint{2.956685in}{2.490702in}}{\pgfqpoint{2.946085in}{2.486311in}}{\pgfqpoint{2.938272in}{2.478498in}}%
\pgfpathcurveto{\pgfqpoint{2.930458in}{2.470684in}}{\pgfqpoint{2.926068in}{2.460085in}}{\pgfqpoint{2.926068in}{2.449035in}}%
\pgfpathcurveto{\pgfqpoint{2.926068in}{2.437985in}}{\pgfqpoint{2.930458in}{2.427386in}}{\pgfqpoint{2.938272in}{2.419572in}}%
\pgfpathcurveto{\pgfqpoint{2.946085in}{2.411759in}}{\pgfqpoint{2.956685in}{2.407368in}}{\pgfqpoint{2.967735in}{2.407368in}}%
\pgfpathclose%
\pgfusepath{stroke,fill}%
\end{pgfscope}%
\begin{pgfscope}%
\pgfpathrectangle{\pgfqpoint{0.481978in}{0.331635in}}{\pgfqpoint{4.960000in}{3.696000in}}%
\pgfusepath{clip}%
\pgfsetbuttcap%
\pgfsetroundjoin%
\definecolor{currentfill}{rgb}{0.631373,0.788235,0.956863}%
\pgfsetfillcolor{currentfill}%
\pgfsetlinewidth{0.481800pt}%
\definecolor{currentstroke}{rgb}{1.000000,1.000000,1.000000}%
\pgfsetstrokecolor{currentstroke}%
\pgfsetdash{}{0pt}%
\pgfpathmoveto{\pgfqpoint{3.877867in}{2.147441in}}%
\pgfpathcurveto{\pgfqpoint{3.888917in}{2.147441in}}{\pgfqpoint{3.899516in}{2.151832in}}{\pgfqpoint{3.907330in}{2.159645in}}%
\pgfpathcurveto{\pgfqpoint{3.915143in}{2.167459in}}{\pgfqpoint{3.919534in}{2.178058in}}{\pgfqpoint{3.919534in}{2.189108in}}%
\pgfpathcurveto{\pgfqpoint{3.919534in}{2.200158in}}{\pgfqpoint{3.915143in}{2.210757in}}{\pgfqpoint{3.907330in}{2.218571in}}%
\pgfpathcurveto{\pgfqpoint{3.899516in}{2.226384in}}{\pgfqpoint{3.888917in}{2.230775in}}{\pgfqpoint{3.877867in}{2.230775in}}%
\pgfpathcurveto{\pgfqpoint{3.866817in}{2.230775in}}{\pgfqpoint{3.856218in}{2.226384in}}{\pgfqpoint{3.848404in}{2.218571in}}%
\pgfpathcurveto{\pgfqpoint{3.840590in}{2.210757in}}{\pgfqpoint{3.836200in}{2.200158in}}{\pgfqpoint{3.836200in}{2.189108in}}%
\pgfpathcurveto{\pgfqpoint{3.836200in}{2.178058in}}{\pgfqpoint{3.840590in}{2.167459in}}{\pgfqpoint{3.848404in}{2.159645in}}%
\pgfpathcurveto{\pgfqpoint{3.856218in}{2.151832in}}{\pgfqpoint{3.866817in}{2.147441in}}{\pgfqpoint{3.877867in}{2.147441in}}%
\pgfpathclose%
\pgfusepath{stroke,fill}%
\end{pgfscope}%
\begin{pgfscope}%
\pgfpathrectangle{\pgfqpoint{0.481978in}{0.331635in}}{\pgfqpoint{4.960000in}{3.696000in}}%
\pgfusepath{clip}%
\pgfsetbuttcap%
\pgfsetroundjoin%
\definecolor{currentfill}{rgb}{0.631373,0.788235,0.956863}%
\pgfsetfillcolor{currentfill}%
\pgfsetlinewidth{0.481800pt}%
\definecolor{currentstroke}{rgb}{1.000000,1.000000,1.000000}%
\pgfsetstrokecolor{currentstroke}%
\pgfsetdash{}{0pt}%
\pgfpathmoveto{\pgfqpoint{2.639796in}{2.019690in}}%
\pgfpathcurveto{\pgfqpoint{2.650846in}{2.019690in}}{\pgfqpoint{2.661445in}{2.024080in}}{\pgfqpoint{2.669259in}{2.031893in}}%
\pgfpathcurveto{\pgfqpoint{2.677072in}{2.039707in}}{\pgfqpoint{2.681463in}{2.050306in}}{\pgfqpoint{2.681463in}{2.061356in}}%
\pgfpathcurveto{\pgfqpoint{2.681463in}{2.072406in}}{\pgfqpoint{2.677072in}{2.083005in}}{\pgfqpoint{2.669259in}{2.090819in}}%
\pgfpathcurveto{\pgfqpoint{2.661445in}{2.098633in}}{\pgfqpoint{2.650846in}{2.103023in}}{\pgfqpoint{2.639796in}{2.103023in}}%
\pgfpathcurveto{\pgfqpoint{2.628746in}{2.103023in}}{\pgfqpoint{2.618147in}{2.098633in}}{\pgfqpoint{2.610333in}{2.090819in}}%
\pgfpathcurveto{\pgfqpoint{2.602519in}{2.083005in}}{\pgfqpoint{2.598129in}{2.072406in}}{\pgfqpoint{2.598129in}{2.061356in}}%
\pgfpathcurveto{\pgfqpoint{2.598129in}{2.050306in}}{\pgfqpoint{2.602519in}{2.039707in}}{\pgfqpoint{2.610333in}{2.031893in}}%
\pgfpathcurveto{\pgfqpoint{2.618147in}{2.024080in}}{\pgfqpoint{2.628746in}{2.019690in}}{\pgfqpoint{2.639796in}{2.019690in}}%
\pgfpathclose%
\pgfusepath{stroke,fill}%
\end{pgfscope}%
\begin{pgfscope}%
\pgfpathrectangle{\pgfqpoint{0.481978in}{0.331635in}}{\pgfqpoint{4.960000in}{3.696000in}}%
\pgfusepath{clip}%
\pgfsetbuttcap%
\pgfsetroundjoin%
\definecolor{currentfill}{rgb}{0.631373,0.788235,0.956863}%
\pgfsetfillcolor{currentfill}%
\pgfsetlinewidth{0.481800pt}%
\definecolor{currentstroke}{rgb}{1.000000,1.000000,1.000000}%
\pgfsetstrokecolor{currentstroke}%
\pgfsetdash{}{0pt}%
\pgfpathmoveto{\pgfqpoint{3.080323in}{2.278504in}}%
\pgfpathcurveto{\pgfqpoint{3.091373in}{2.278504in}}{\pgfqpoint{3.101972in}{2.282894in}}{\pgfqpoint{3.109786in}{2.290708in}}%
\pgfpathcurveto{\pgfqpoint{3.117599in}{2.298522in}}{\pgfqpoint{3.121990in}{2.309121in}}{\pgfqpoint{3.121990in}{2.320171in}}%
\pgfpathcurveto{\pgfqpoint{3.121990in}{2.331221in}}{\pgfqpoint{3.117599in}{2.341820in}}{\pgfqpoint{3.109786in}{2.349634in}}%
\pgfpathcurveto{\pgfqpoint{3.101972in}{2.357447in}}{\pgfqpoint{3.091373in}{2.361837in}}{\pgfqpoint{3.080323in}{2.361837in}}%
\pgfpathcurveto{\pgfqpoint{3.069273in}{2.361837in}}{\pgfqpoint{3.058674in}{2.357447in}}{\pgfqpoint{3.050860in}{2.349634in}}%
\pgfpathcurveto{\pgfqpoint{3.043047in}{2.341820in}}{\pgfqpoint{3.038656in}{2.331221in}}{\pgfqpoint{3.038656in}{2.320171in}}%
\pgfpathcurveto{\pgfqpoint{3.038656in}{2.309121in}}{\pgfqpoint{3.043047in}{2.298522in}}{\pgfqpoint{3.050860in}{2.290708in}}%
\pgfpathcurveto{\pgfqpoint{3.058674in}{2.282894in}}{\pgfqpoint{3.069273in}{2.278504in}}{\pgfqpoint{3.080323in}{2.278504in}}%
\pgfpathclose%
\pgfusepath{stroke,fill}%
\end{pgfscope}%
\begin{pgfscope}%
\pgfpathrectangle{\pgfqpoint{0.481978in}{0.331635in}}{\pgfqpoint{4.960000in}{3.696000in}}%
\pgfusepath{clip}%
\pgfsetbuttcap%
\pgfsetroundjoin%
\definecolor{currentfill}{rgb}{0.631373,0.788235,0.956863}%
\pgfsetfillcolor{currentfill}%
\pgfsetlinewidth{0.481800pt}%
\definecolor{currentstroke}{rgb}{1.000000,1.000000,1.000000}%
\pgfsetstrokecolor{currentstroke}%
\pgfsetdash{}{0pt}%
\pgfpathmoveto{\pgfqpoint{4.504389in}{1.605777in}}%
\pgfpathcurveto{\pgfqpoint{4.515439in}{1.605777in}}{\pgfqpoint{4.526038in}{1.610168in}}{\pgfqpoint{4.533851in}{1.617981in}}%
\pgfpathcurveto{\pgfqpoint{4.541665in}{1.625795in}}{\pgfqpoint{4.546055in}{1.636394in}}{\pgfqpoint{4.546055in}{1.647444in}}%
\pgfpathcurveto{\pgfqpoint{4.546055in}{1.658494in}}{\pgfqpoint{4.541665in}{1.669093in}}{\pgfqpoint{4.533851in}{1.676907in}}%
\pgfpathcurveto{\pgfqpoint{4.526038in}{1.684720in}}{\pgfqpoint{4.515439in}{1.689111in}}{\pgfqpoint{4.504389in}{1.689111in}}%
\pgfpathcurveto{\pgfqpoint{4.493339in}{1.689111in}}{\pgfqpoint{4.482739in}{1.684720in}}{\pgfqpoint{4.474926in}{1.676907in}}%
\pgfpathcurveto{\pgfqpoint{4.467112in}{1.669093in}}{\pgfqpoint{4.462722in}{1.658494in}}{\pgfqpoint{4.462722in}{1.647444in}}%
\pgfpathcurveto{\pgfqpoint{4.462722in}{1.636394in}}{\pgfqpoint{4.467112in}{1.625795in}}{\pgfqpoint{4.474926in}{1.617981in}}%
\pgfpathcurveto{\pgfqpoint{4.482739in}{1.610168in}}{\pgfqpoint{4.493339in}{1.605777in}}{\pgfqpoint{4.504389in}{1.605777in}}%
\pgfpathclose%
\pgfusepath{stroke,fill}%
\end{pgfscope}%
\begin{pgfscope}%
\pgfpathrectangle{\pgfqpoint{0.481978in}{0.331635in}}{\pgfqpoint{4.960000in}{3.696000in}}%
\pgfusepath{clip}%
\pgfsetbuttcap%
\pgfsetroundjoin%
\definecolor{currentfill}{rgb}{0.631373,0.788235,0.956863}%
\pgfsetfillcolor{currentfill}%
\pgfsetlinewidth{0.481800pt}%
\definecolor{currentstroke}{rgb}{1.000000,1.000000,1.000000}%
\pgfsetstrokecolor{currentstroke}%
\pgfsetdash{}{0pt}%
\pgfpathmoveto{\pgfqpoint{2.599685in}{1.703662in}}%
\pgfpathcurveto{\pgfqpoint{2.610735in}{1.703662in}}{\pgfqpoint{2.621334in}{1.708052in}}{\pgfqpoint{2.629148in}{1.715866in}}%
\pgfpathcurveto{\pgfqpoint{2.636961in}{1.723680in}}{\pgfqpoint{2.641352in}{1.734279in}}{\pgfqpoint{2.641352in}{1.745329in}}%
\pgfpathcurveto{\pgfqpoint{2.641352in}{1.756379in}}{\pgfqpoint{2.636961in}{1.766978in}}{\pgfqpoint{2.629148in}{1.774792in}}%
\pgfpathcurveto{\pgfqpoint{2.621334in}{1.782605in}}{\pgfqpoint{2.610735in}{1.786995in}}{\pgfqpoint{2.599685in}{1.786995in}}%
\pgfpathcurveto{\pgfqpoint{2.588635in}{1.786995in}}{\pgfqpoint{2.578036in}{1.782605in}}{\pgfqpoint{2.570222in}{1.774792in}}%
\pgfpathcurveto{\pgfqpoint{2.562408in}{1.766978in}}{\pgfqpoint{2.558018in}{1.756379in}}{\pgfqpoint{2.558018in}{1.745329in}}%
\pgfpathcurveto{\pgfqpoint{2.558018in}{1.734279in}}{\pgfqpoint{2.562408in}{1.723680in}}{\pgfqpoint{2.570222in}{1.715866in}}%
\pgfpathcurveto{\pgfqpoint{2.578036in}{1.708052in}}{\pgfqpoint{2.588635in}{1.703662in}}{\pgfqpoint{2.599685in}{1.703662in}}%
\pgfpathclose%
\pgfusepath{stroke,fill}%
\end{pgfscope}%
\begin{pgfscope}%
\pgfpathrectangle{\pgfqpoint{0.481978in}{0.331635in}}{\pgfqpoint{4.960000in}{3.696000in}}%
\pgfusepath{clip}%
\pgfsetbuttcap%
\pgfsetroundjoin%
\definecolor{currentfill}{rgb}{0.631373,0.788235,0.956863}%
\pgfsetfillcolor{currentfill}%
\pgfsetlinewidth{0.481800pt}%
\definecolor{currentstroke}{rgb}{1.000000,1.000000,1.000000}%
\pgfsetstrokecolor{currentstroke}%
\pgfsetdash{}{0pt}%
\pgfpathmoveto{\pgfqpoint{2.396495in}{2.533789in}}%
\pgfpathcurveto{\pgfqpoint{2.407545in}{2.533789in}}{\pgfqpoint{2.418144in}{2.538179in}}{\pgfqpoint{2.425958in}{2.545993in}}%
\pgfpathcurveto{\pgfqpoint{2.433771in}{2.553807in}}{\pgfqpoint{2.438162in}{2.564406in}}{\pgfqpoint{2.438162in}{2.575456in}}%
\pgfpathcurveto{\pgfqpoint{2.438162in}{2.586506in}}{\pgfqpoint{2.433771in}{2.597105in}}{\pgfqpoint{2.425958in}{2.604918in}}%
\pgfpathcurveto{\pgfqpoint{2.418144in}{2.612732in}}{\pgfqpoint{2.407545in}{2.617122in}}{\pgfqpoint{2.396495in}{2.617122in}}%
\pgfpathcurveto{\pgfqpoint{2.385445in}{2.617122in}}{\pgfqpoint{2.374846in}{2.612732in}}{\pgfqpoint{2.367032in}{2.604918in}}%
\pgfpathcurveto{\pgfqpoint{2.359219in}{2.597105in}}{\pgfqpoint{2.354828in}{2.586506in}}{\pgfqpoint{2.354828in}{2.575456in}}%
\pgfpathcurveto{\pgfqpoint{2.354828in}{2.564406in}}{\pgfqpoint{2.359219in}{2.553807in}}{\pgfqpoint{2.367032in}{2.545993in}}%
\pgfpathcurveto{\pgfqpoint{2.374846in}{2.538179in}}{\pgfqpoint{2.385445in}{2.533789in}}{\pgfqpoint{2.396495in}{2.533789in}}%
\pgfpathclose%
\pgfusepath{stroke,fill}%
\end{pgfscope}%
\begin{pgfscope}%
\pgfpathrectangle{\pgfqpoint{0.481978in}{0.331635in}}{\pgfqpoint{4.960000in}{3.696000in}}%
\pgfusepath{clip}%
\pgfsetbuttcap%
\pgfsetroundjoin%
\definecolor{currentfill}{rgb}{0.631373,0.788235,0.956863}%
\pgfsetfillcolor{currentfill}%
\pgfsetlinewidth{0.481800pt}%
\definecolor{currentstroke}{rgb}{1.000000,1.000000,1.000000}%
\pgfsetstrokecolor{currentstroke}%
\pgfsetdash{}{0pt}%
\pgfpathmoveto{\pgfqpoint{3.101018in}{1.945189in}}%
\pgfpathcurveto{\pgfqpoint{3.112068in}{1.945189in}}{\pgfqpoint{3.122667in}{1.949579in}}{\pgfqpoint{3.130480in}{1.957393in}}%
\pgfpathcurveto{\pgfqpoint{3.138294in}{1.965206in}}{\pgfqpoint{3.142684in}{1.975805in}}{\pgfqpoint{3.142684in}{1.986855in}}%
\pgfpathcurveto{\pgfqpoint{3.142684in}{1.997906in}}{\pgfqpoint{3.138294in}{2.008505in}}{\pgfqpoint{3.130480in}{2.016318in}}%
\pgfpathcurveto{\pgfqpoint{3.122667in}{2.024132in}}{\pgfqpoint{3.112068in}{2.028522in}}{\pgfqpoint{3.101018in}{2.028522in}}%
\pgfpathcurveto{\pgfqpoint{3.089967in}{2.028522in}}{\pgfqpoint{3.079368in}{2.024132in}}{\pgfqpoint{3.071555in}{2.016318in}}%
\pgfpathcurveto{\pgfqpoint{3.063741in}{2.008505in}}{\pgfqpoint{3.059351in}{1.997906in}}{\pgfqpoint{3.059351in}{1.986855in}}%
\pgfpathcurveto{\pgfqpoint{3.059351in}{1.975805in}}{\pgfqpoint{3.063741in}{1.965206in}}{\pgfqpoint{3.071555in}{1.957393in}}%
\pgfpathcurveto{\pgfqpoint{3.079368in}{1.949579in}}{\pgfqpoint{3.089967in}{1.945189in}}{\pgfqpoint{3.101018in}{1.945189in}}%
\pgfpathclose%
\pgfusepath{stroke,fill}%
\end{pgfscope}%
\begin{pgfscope}%
\pgfpathrectangle{\pgfqpoint{0.481978in}{0.331635in}}{\pgfqpoint{4.960000in}{3.696000in}}%
\pgfusepath{clip}%
\pgfsetbuttcap%
\pgfsetroundjoin%
\definecolor{currentfill}{rgb}{0.631373,0.788235,0.956863}%
\pgfsetfillcolor{currentfill}%
\pgfsetlinewidth{0.481800pt}%
\definecolor{currentstroke}{rgb}{1.000000,1.000000,1.000000}%
\pgfsetstrokecolor{currentstroke}%
\pgfsetdash{}{0pt}%
\pgfpathmoveto{\pgfqpoint{3.826951in}{1.753990in}}%
\pgfpathcurveto{\pgfqpoint{3.838001in}{1.753990in}}{\pgfqpoint{3.848600in}{1.758380in}}{\pgfqpoint{3.856414in}{1.766193in}}%
\pgfpathcurveto{\pgfqpoint{3.864227in}{1.774007in}}{\pgfqpoint{3.868618in}{1.784606in}}{\pgfqpoint{3.868618in}{1.795656in}}%
\pgfpathcurveto{\pgfqpoint{3.868618in}{1.806706in}}{\pgfqpoint{3.864227in}{1.817305in}}{\pgfqpoint{3.856414in}{1.825119in}}%
\pgfpathcurveto{\pgfqpoint{3.848600in}{1.832933in}}{\pgfqpoint{3.838001in}{1.837323in}}{\pgfqpoint{3.826951in}{1.837323in}}%
\pgfpathcurveto{\pgfqpoint{3.815901in}{1.837323in}}{\pgfqpoint{3.805302in}{1.832933in}}{\pgfqpoint{3.797488in}{1.825119in}}%
\pgfpathcurveto{\pgfqpoint{3.789675in}{1.817305in}}{\pgfqpoint{3.785284in}{1.806706in}}{\pgfqpoint{3.785284in}{1.795656in}}%
\pgfpathcurveto{\pgfqpoint{3.785284in}{1.784606in}}{\pgfqpoint{3.789675in}{1.774007in}}{\pgfqpoint{3.797488in}{1.766193in}}%
\pgfpathcurveto{\pgfqpoint{3.805302in}{1.758380in}}{\pgfqpoint{3.815901in}{1.753990in}}{\pgfqpoint{3.826951in}{1.753990in}}%
\pgfpathclose%
\pgfusepath{stroke,fill}%
\end{pgfscope}%
\begin{pgfscope}%
\pgfpathrectangle{\pgfqpoint{0.481978in}{0.331635in}}{\pgfqpoint{4.960000in}{3.696000in}}%
\pgfusepath{clip}%
\pgfsetbuttcap%
\pgfsetroundjoin%
\definecolor{currentfill}{rgb}{0.631373,0.788235,0.956863}%
\pgfsetfillcolor{currentfill}%
\pgfsetlinewidth{0.481800pt}%
\definecolor{currentstroke}{rgb}{1.000000,1.000000,1.000000}%
\pgfsetstrokecolor{currentstroke}%
\pgfsetdash{}{0pt}%
\pgfpathmoveto{\pgfqpoint{3.258079in}{2.290621in}}%
\pgfpathcurveto{\pgfqpoint{3.269130in}{2.290621in}}{\pgfqpoint{3.279729in}{2.295011in}}{\pgfqpoint{3.287542in}{2.302825in}}%
\pgfpathcurveto{\pgfqpoint{3.295356in}{2.310639in}}{\pgfqpoint{3.299746in}{2.321238in}}{\pgfqpoint{3.299746in}{2.332288in}}%
\pgfpathcurveto{\pgfqpoint{3.299746in}{2.343338in}}{\pgfqpoint{3.295356in}{2.353937in}}{\pgfqpoint{3.287542in}{2.361751in}}%
\pgfpathcurveto{\pgfqpoint{3.279729in}{2.369564in}}{\pgfqpoint{3.269130in}{2.373955in}}{\pgfqpoint{3.258079in}{2.373955in}}%
\pgfpathcurveto{\pgfqpoint{3.247029in}{2.373955in}}{\pgfqpoint{3.236430in}{2.369564in}}{\pgfqpoint{3.228617in}{2.361751in}}%
\pgfpathcurveto{\pgfqpoint{3.220803in}{2.353937in}}{\pgfqpoint{3.216413in}{2.343338in}}{\pgfqpoint{3.216413in}{2.332288in}}%
\pgfpathcurveto{\pgfqpoint{3.216413in}{2.321238in}}{\pgfqpoint{3.220803in}{2.310639in}}{\pgfqpoint{3.228617in}{2.302825in}}%
\pgfpathcurveto{\pgfqpoint{3.236430in}{2.295011in}}{\pgfqpoint{3.247029in}{2.290621in}}{\pgfqpoint{3.258079in}{2.290621in}}%
\pgfpathclose%
\pgfusepath{stroke,fill}%
\end{pgfscope}%
\begin{pgfscope}%
\pgfpathrectangle{\pgfqpoint{0.481978in}{0.331635in}}{\pgfqpoint{4.960000in}{3.696000in}}%
\pgfusepath{clip}%
\pgfsetbuttcap%
\pgfsetroundjoin%
\definecolor{currentfill}{rgb}{0.631373,0.788235,0.956863}%
\pgfsetfillcolor{currentfill}%
\pgfsetlinewidth{0.481800pt}%
\definecolor{currentstroke}{rgb}{1.000000,1.000000,1.000000}%
\pgfsetstrokecolor{currentstroke}%
\pgfsetdash{}{0pt}%
\pgfpathmoveto{\pgfqpoint{3.546483in}{1.803458in}}%
\pgfpathcurveto{\pgfqpoint{3.557533in}{1.803458in}}{\pgfqpoint{3.568132in}{1.807848in}}{\pgfqpoint{3.575946in}{1.815662in}}%
\pgfpathcurveto{\pgfqpoint{3.583759in}{1.823475in}}{\pgfqpoint{3.588150in}{1.834074in}}{\pgfqpoint{3.588150in}{1.845124in}}%
\pgfpathcurveto{\pgfqpoint{3.588150in}{1.856175in}}{\pgfqpoint{3.583759in}{1.866774in}}{\pgfqpoint{3.575946in}{1.874587in}}%
\pgfpathcurveto{\pgfqpoint{3.568132in}{1.882401in}}{\pgfqpoint{3.557533in}{1.886791in}}{\pgfqpoint{3.546483in}{1.886791in}}%
\pgfpathcurveto{\pgfqpoint{3.535433in}{1.886791in}}{\pgfqpoint{3.524834in}{1.882401in}}{\pgfqpoint{3.517020in}{1.874587in}}%
\pgfpathcurveto{\pgfqpoint{3.509207in}{1.866774in}}{\pgfqpoint{3.504816in}{1.856175in}}{\pgfqpoint{3.504816in}{1.845124in}}%
\pgfpathcurveto{\pgfqpoint{3.504816in}{1.834074in}}{\pgfqpoint{3.509207in}{1.823475in}}{\pgfqpoint{3.517020in}{1.815662in}}%
\pgfpathcurveto{\pgfqpoint{3.524834in}{1.807848in}}{\pgfqpoint{3.535433in}{1.803458in}}{\pgfqpoint{3.546483in}{1.803458in}}%
\pgfpathclose%
\pgfusepath{stroke,fill}%
\end{pgfscope}%
\begin{pgfscope}%
\pgfpathrectangle{\pgfqpoint{0.481978in}{0.331635in}}{\pgfqpoint{4.960000in}{3.696000in}}%
\pgfusepath{clip}%
\pgfsetbuttcap%
\pgfsetroundjoin%
\definecolor{currentfill}{rgb}{0.631373,0.788235,0.956863}%
\pgfsetfillcolor{currentfill}%
\pgfsetlinewidth{0.481800pt}%
\definecolor{currentstroke}{rgb}{1.000000,1.000000,1.000000}%
\pgfsetstrokecolor{currentstroke}%
\pgfsetdash{}{0pt}%
\pgfpathmoveto{\pgfqpoint{2.353690in}{2.674871in}}%
\pgfpathcurveto{\pgfqpoint{2.364740in}{2.674871in}}{\pgfqpoint{2.375339in}{2.679261in}}{\pgfqpoint{2.383153in}{2.687075in}}%
\pgfpathcurveto{\pgfqpoint{2.390967in}{2.694888in}}{\pgfqpoint{2.395357in}{2.705487in}}{\pgfqpoint{2.395357in}{2.716538in}}%
\pgfpathcurveto{\pgfqpoint{2.395357in}{2.727588in}}{\pgfqpoint{2.390967in}{2.738187in}}{\pgfqpoint{2.383153in}{2.746000in}}%
\pgfpathcurveto{\pgfqpoint{2.375339in}{2.753814in}}{\pgfqpoint{2.364740in}{2.758204in}}{\pgfqpoint{2.353690in}{2.758204in}}%
\pgfpathcurveto{\pgfqpoint{2.342640in}{2.758204in}}{\pgfqpoint{2.332041in}{2.753814in}}{\pgfqpoint{2.324227in}{2.746000in}}%
\pgfpathcurveto{\pgfqpoint{2.316414in}{2.738187in}}{\pgfqpoint{2.312023in}{2.727588in}}{\pgfqpoint{2.312023in}{2.716538in}}%
\pgfpathcurveto{\pgfqpoint{2.312023in}{2.705487in}}{\pgfqpoint{2.316414in}{2.694888in}}{\pgfqpoint{2.324227in}{2.687075in}}%
\pgfpathcurveto{\pgfqpoint{2.332041in}{2.679261in}}{\pgfqpoint{2.342640in}{2.674871in}}{\pgfqpoint{2.353690in}{2.674871in}}%
\pgfpathclose%
\pgfusepath{stroke,fill}%
\end{pgfscope}%
\begin{pgfscope}%
\pgfpathrectangle{\pgfqpoint{0.481978in}{0.331635in}}{\pgfqpoint{4.960000in}{3.696000in}}%
\pgfusepath{clip}%
\pgfsetbuttcap%
\pgfsetroundjoin%
\definecolor{currentfill}{rgb}{0.631373,0.788235,0.956863}%
\pgfsetfillcolor{currentfill}%
\pgfsetlinewidth{0.481800pt}%
\definecolor{currentstroke}{rgb}{1.000000,1.000000,1.000000}%
\pgfsetstrokecolor{currentstroke}%
\pgfsetdash{}{0pt}%
\pgfpathmoveto{\pgfqpoint{4.157151in}{1.911417in}}%
\pgfpathcurveto{\pgfqpoint{4.168201in}{1.911417in}}{\pgfqpoint{4.178800in}{1.915808in}}{\pgfqpoint{4.186614in}{1.923621in}}%
\pgfpathcurveto{\pgfqpoint{4.194428in}{1.931435in}}{\pgfqpoint{4.198818in}{1.942034in}}{\pgfqpoint{4.198818in}{1.953084in}}%
\pgfpathcurveto{\pgfqpoint{4.198818in}{1.964134in}}{\pgfqpoint{4.194428in}{1.974733in}}{\pgfqpoint{4.186614in}{1.982547in}}%
\pgfpathcurveto{\pgfqpoint{4.178800in}{1.990361in}}{\pgfqpoint{4.168201in}{1.994751in}}{\pgfqpoint{4.157151in}{1.994751in}}%
\pgfpathcurveto{\pgfqpoint{4.146101in}{1.994751in}}{\pgfqpoint{4.135502in}{1.990361in}}{\pgfqpoint{4.127688in}{1.982547in}}%
\pgfpathcurveto{\pgfqpoint{4.119875in}{1.974733in}}{\pgfqpoint{4.115485in}{1.964134in}}{\pgfqpoint{4.115485in}{1.953084in}}%
\pgfpathcurveto{\pgfqpoint{4.115485in}{1.942034in}}{\pgfqpoint{4.119875in}{1.931435in}}{\pgfqpoint{4.127688in}{1.923621in}}%
\pgfpathcurveto{\pgfqpoint{4.135502in}{1.915808in}}{\pgfqpoint{4.146101in}{1.911417in}}{\pgfqpoint{4.157151in}{1.911417in}}%
\pgfpathclose%
\pgfusepath{stroke,fill}%
\end{pgfscope}%
\begin{pgfscope}%
\pgfpathrectangle{\pgfqpoint{0.481978in}{0.331635in}}{\pgfqpoint{4.960000in}{3.696000in}}%
\pgfusepath{clip}%
\pgfsetbuttcap%
\pgfsetroundjoin%
\definecolor{currentfill}{rgb}{0.631373,0.788235,0.956863}%
\pgfsetfillcolor{currentfill}%
\pgfsetlinewidth{0.481800pt}%
\definecolor{currentstroke}{rgb}{1.000000,1.000000,1.000000}%
\pgfsetstrokecolor{currentstroke}%
\pgfsetdash{}{0pt}%
\pgfpathmoveto{\pgfqpoint{1.950038in}{2.222556in}}%
\pgfpathcurveto{\pgfqpoint{1.961088in}{2.222556in}}{\pgfqpoint{1.971687in}{2.226946in}}{\pgfqpoint{1.979500in}{2.234760in}}%
\pgfpathcurveto{\pgfqpoint{1.987314in}{2.242573in}}{\pgfqpoint{1.991704in}{2.253172in}}{\pgfqpoint{1.991704in}{2.264223in}}%
\pgfpathcurveto{\pgfqpoint{1.991704in}{2.275273in}}{\pgfqpoint{1.987314in}{2.285872in}}{\pgfqpoint{1.979500in}{2.293685in}}%
\pgfpathcurveto{\pgfqpoint{1.971687in}{2.301499in}}{\pgfqpoint{1.961088in}{2.305889in}}{\pgfqpoint{1.950038in}{2.305889in}}%
\pgfpathcurveto{\pgfqpoint{1.938988in}{2.305889in}}{\pgfqpoint{1.928389in}{2.301499in}}{\pgfqpoint{1.920575in}{2.293685in}}%
\pgfpathcurveto{\pgfqpoint{1.912761in}{2.285872in}}{\pgfqpoint{1.908371in}{2.275273in}}{\pgfqpoint{1.908371in}{2.264223in}}%
\pgfpathcurveto{\pgfqpoint{1.908371in}{2.253172in}}{\pgfqpoint{1.912761in}{2.242573in}}{\pgfqpoint{1.920575in}{2.234760in}}%
\pgfpathcurveto{\pgfqpoint{1.928389in}{2.226946in}}{\pgfqpoint{1.938988in}{2.222556in}}{\pgfqpoint{1.950038in}{2.222556in}}%
\pgfpathclose%
\pgfusepath{stroke,fill}%
\end{pgfscope}%
\begin{pgfscope}%
\pgfpathrectangle{\pgfqpoint{0.481978in}{0.331635in}}{\pgfqpoint{4.960000in}{3.696000in}}%
\pgfusepath{clip}%
\pgfsetbuttcap%
\pgfsetroundjoin%
\definecolor{currentfill}{rgb}{0.631373,0.788235,0.956863}%
\pgfsetfillcolor{currentfill}%
\pgfsetlinewidth{0.481800pt}%
\definecolor{currentstroke}{rgb}{1.000000,1.000000,1.000000}%
\pgfsetstrokecolor{currentstroke}%
\pgfsetdash{}{0pt}%
\pgfpathmoveto{\pgfqpoint{1.917872in}{2.040095in}}%
\pgfpathcurveto{\pgfqpoint{1.928922in}{2.040095in}}{\pgfqpoint{1.939521in}{2.044485in}}{\pgfqpoint{1.947334in}{2.052299in}}%
\pgfpathcurveto{\pgfqpoint{1.955148in}{2.060112in}}{\pgfqpoint{1.959538in}{2.070712in}}{\pgfqpoint{1.959538in}{2.081762in}}%
\pgfpathcurveto{\pgfqpoint{1.959538in}{2.092812in}}{\pgfqpoint{1.955148in}{2.103411in}}{\pgfqpoint{1.947334in}{2.111224in}}%
\pgfpathcurveto{\pgfqpoint{1.939521in}{2.119038in}}{\pgfqpoint{1.928922in}{2.123428in}}{\pgfqpoint{1.917872in}{2.123428in}}%
\pgfpathcurveto{\pgfqpoint{1.906821in}{2.123428in}}{\pgfqpoint{1.896222in}{2.119038in}}{\pgfqpoint{1.888409in}{2.111224in}}%
\pgfpathcurveto{\pgfqpoint{1.880595in}{2.103411in}}{\pgfqpoint{1.876205in}{2.092812in}}{\pgfqpoint{1.876205in}{2.081762in}}%
\pgfpathcurveto{\pgfqpoint{1.876205in}{2.070712in}}{\pgfqpoint{1.880595in}{2.060112in}}{\pgfqpoint{1.888409in}{2.052299in}}%
\pgfpathcurveto{\pgfqpoint{1.896222in}{2.044485in}}{\pgfqpoint{1.906821in}{2.040095in}}{\pgfqpoint{1.917872in}{2.040095in}}%
\pgfpathclose%
\pgfusepath{stroke,fill}%
\end{pgfscope}%
\begin{pgfscope}%
\pgfpathrectangle{\pgfqpoint{0.481978in}{0.331635in}}{\pgfqpoint{4.960000in}{3.696000in}}%
\pgfusepath{clip}%
\pgfsetbuttcap%
\pgfsetroundjoin%
\definecolor{currentfill}{rgb}{0.631373,0.788235,0.956863}%
\pgfsetfillcolor{currentfill}%
\pgfsetlinewidth{0.481800pt}%
\definecolor{currentstroke}{rgb}{1.000000,1.000000,1.000000}%
\pgfsetstrokecolor{currentstroke}%
\pgfsetdash{}{0pt}%
\pgfpathmoveto{\pgfqpoint{1.997703in}{1.667693in}}%
\pgfpathcurveto{\pgfqpoint{2.008753in}{1.667693in}}{\pgfqpoint{2.019352in}{1.672083in}}{\pgfqpoint{2.027166in}{1.679897in}}%
\pgfpathcurveto{\pgfqpoint{2.034979in}{1.687711in}}{\pgfqpoint{2.039369in}{1.698310in}}{\pgfqpoint{2.039369in}{1.709360in}}%
\pgfpathcurveto{\pgfqpoint{2.039369in}{1.720410in}}{\pgfqpoint{2.034979in}{1.731009in}}{\pgfqpoint{2.027166in}{1.738822in}}%
\pgfpathcurveto{\pgfqpoint{2.019352in}{1.746636in}}{\pgfqpoint{2.008753in}{1.751026in}}{\pgfqpoint{1.997703in}{1.751026in}}%
\pgfpathcurveto{\pgfqpoint{1.986653in}{1.751026in}}{\pgfqpoint{1.976054in}{1.746636in}}{\pgfqpoint{1.968240in}{1.738822in}}%
\pgfpathcurveto{\pgfqpoint{1.960426in}{1.731009in}}{\pgfqpoint{1.956036in}{1.720410in}}{\pgfqpoint{1.956036in}{1.709360in}}%
\pgfpathcurveto{\pgfqpoint{1.956036in}{1.698310in}}{\pgfqpoint{1.960426in}{1.687711in}}{\pgfqpoint{1.968240in}{1.679897in}}%
\pgfpathcurveto{\pgfqpoint{1.976054in}{1.672083in}}{\pgfqpoint{1.986653in}{1.667693in}}{\pgfqpoint{1.997703in}{1.667693in}}%
\pgfpathclose%
\pgfusepath{stroke,fill}%
\end{pgfscope}%
\begin{pgfscope}%
\pgfpathrectangle{\pgfqpoint{0.481978in}{0.331635in}}{\pgfqpoint{4.960000in}{3.696000in}}%
\pgfusepath{clip}%
\pgfsetbuttcap%
\pgfsetroundjoin%
\definecolor{currentfill}{rgb}{0.631373,0.788235,0.956863}%
\pgfsetfillcolor{currentfill}%
\pgfsetlinewidth{0.481800pt}%
\definecolor{currentstroke}{rgb}{1.000000,1.000000,1.000000}%
\pgfsetstrokecolor{currentstroke}%
\pgfsetdash{}{0pt}%
\pgfpathmoveto{\pgfqpoint{1.471014in}{1.273025in}}%
\pgfpathcurveto{\pgfqpoint{1.482064in}{1.273025in}}{\pgfqpoint{1.492663in}{1.277415in}}{\pgfqpoint{1.500477in}{1.285229in}}%
\pgfpathcurveto{\pgfqpoint{1.508290in}{1.293042in}}{\pgfqpoint{1.512680in}{1.303641in}}{\pgfqpoint{1.512680in}{1.314692in}}%
\pgfpathcurveto{\pgfqpoint{1.512680in}{1.325742in}}{\pgfqpoint{1.508290in}{1.336341in}}{\pgfqpoint{1.500477in}{1.344154in}}%
\pgfpathcurveto{\pgfqpoint{1.492663in}{1.351968in}}{\pgfqpoint{1.482064in}{1.356358in}}{\pgfqpoint{1.471014in}{1.356358in}}%
\pgfpathcurveto{\pgfqpoint{1.459964in}{1.356358in}}{\pgfqpoint{1.449365in}{1.351968in}}{\pgfqpoint{1.441551in}{1.344154in}}%
\pgfpathcurveto{\pgfqpoint{1.433737in}{1.336341in}}{\pgfqpoint{1.429347in}{1.325742in}}{\pgfqpoint{1.429347in}{1.314692in}}%
\pgfpathcurveto{\pgfqpoint{1.429347in}{1.303641in}}{\pgfqpoint{1.433737in}{1.293042in}}{\pgfqpoint{1.441551in}{1.285229in}}%
\pgfpathcurveto{\pgfqpoint{1.449365in}{1.277415in}}{\pgfqpoint{1.459964in}{1.273025in}}{\pgfqpoint{1.471014in}{1.273025in}}%
\pgfpathclose%
\pgfusepath{stroke,fill}%
\end{pgfscope}%
\begin{pgfscope}%
\pgfpathrectangle{\pgfqpoint{0.481978in}{0.331635in}}{\pgfqpoint{4.960000in}{3.696000in}}%
\pgfusepath{clip}%
\pgfsetbuttcap%
\pgfsetroundjoin%
\definecolor{currentfill}{rgb}{0.631373,0.788235,0.956863}%
\pgfsetfillcolor{currentfill}%
\pgfsetlinewidth{0.481800pt}%
\definecolor{currentstroke}{rgb}{1.000000,1.000000,1.000000}%
\pgfsetstrokecolor{currentstroke}%
\pgfsetdash{}{0pt}%
\pgfpathmoveto{\pgfqpoint{2.468182in}{3.035940in}}%
\pgfpathcurveto{\pgfqpoint{2.479232in}{3.035940in}}{\pgfqpoint{2.489831in}{3.040330in}}{\pgfqpoint{2.497645in}{3.048144in}}%
\pgfpathcurveto{\pgfqpoint{2.505458in}{3.055958in}}{\pgfqpoint{2.509849in}{3.066557in}}{\pgfqpoint{2.509849in}{3.077607in}}%
\pgfpathcurveto{\pgfqpoint{2.509849in}{3.088657in}}{\pgfqpoint{2.505458in}{3.099256in}}{\pgfqpoint{2.497645in}{3.107069in}}%
\pgfpathcurveto{\pgfqpoint{2.489831in}{3.114883in}}{\pgfqpoint{2.479232in}{3.119273in}}{\pgfqpoint{2.468182in}{3.119273in}}%
\pgfpathcurveto{\pgfqpoint{2.457132in}{3.119273in}}{\pgfqpoint{2.446533in}{3.114883in}}{\pgfqpoint{2.438719in}{3.107069in}}%
\pgfpathcurveto{\pgfqpoint{2.430906in}{3.099256in}}{\pgfqpoint{2.426515in}{3.088657in}}{\pgfqpoint{2.426515in}{3.077607in}}%
\pgfpathcurveto{\pgfqpoint{2.426515in}{3.066557in}}{\pgfqpoint{2.430906in}{3.055958in}}{\pgfqpoint{2.438719in}{3.048144in}}%
\pgfpathcurveto{\pgfqpoint{2.446533in}{3.040330in}}{\pgfqpoint{2.457132in}{3.035940in}}{\pgfqpoint{2.468182in}{3.035940in}}%
\pgfpathclose%
\pgfusepath{stroke,fill}%
\end{pgfscope}%
\begin{pgfscope}%
\pgfpathrectangle{\pgfqpoint{0.481978in}{0.331635in}}{\pgfqpoint{4.960000in}{3.696000in}}%
\pgfusepath{clip}%
\pgfsetbuttcap%
\pgfsetroundjoin%
\definecolor{currentfill}{rgb}{0.631373,0.788235,0.956863}%
\pgfsetfillcolor{currentfill}%
\pgfsetlinewidth{0.481800pt}%
\definecolor{currentstroke}{rgb}{1.000000,1.000000,1.000000}%
\pgfsetstrokecolor{currentstroke}%
\pgfsetdash{}{0pt}%
\pgfpathmoveto{\pgfqpoint{3.820394in}{1.741181in}}%
\pgfpathcurveto{\pgfqpoint{3.831444in}{1.741181in}}{\pgfqpoint{3.842043in}{1.745571in}}{\pgfqpoint{3.849857in}{1.753385in}}%
\pgfpathcurveto{\pgfqpoint{3.857670in}{1.761198in}}{\pgfqpoint{3.862060in}{1.771797in}}{\pgfqpoint{3.862060in}{1.782848in}}%
\pgfpathcurveto{\pgfqpoint{3.862060in}{1.793898in}}{\pgfqpoint{3.857670in}{1.804497in}}{\pgfqpoint{3.849857in}{1.812310in}}%
\pgfpathcurveto{\pgfqpoint{3.842043in}{1.820124in}}{\pgfqpoint{3.831444in}{1.824514in}}{\pgfqpoint{3.820394in}{1.824514in}}%
\pgfpathcurveto{\pgfqpoint{3.809344in}{1.824514in}}{\pgfqpoint{3.798745in}{1.820124in}}{\pgfqpoint{3.790931in}{1.812310in}}%
\pgfpathcurveto{\pgfqpoint{3.783117in}{1.804497in}}{\pgfqpoint{3.778727in}{1.793898in}}{\pgfqpoint{3.778727in}{1.782848in}}%
\pgfpathcurveto{\pgfqpoint{3.778727in}{1.771797in}}{\pgfqpoint{3.783117in}{1.761198in}}{\pgfqpoint{3.790931in}{1.753385in}}%
\pgfpathcurveto{\pgfqpoint{3.798745in}{1.745571in}}{\pgfqpoint{3.809344in}{1.741181in}}{\pgfqpoint{3.820394in}{1.741181in}}%
\pgfpathclose%
\pgfusepath{stroke,fill}%
\end{pgfscope}%
\begin{pgfscope}%
\pgfpathrectangle{\pgfqpoint{0.481978in}{0.331635in}}{\pgfqpoint{4.960000in}{3.696000in}}%
\pgfusepath{clip}%
\pgfsetbuttcap%
\pgfsetroundjoin%
\definecolor{currentfill}{rgb}{0.631373,0.788235,0.956863}%
\pgfsetfillcolor{currentfill}%
\pgfsetlinewidth{0.481800pt}%
\definecolor{currentstroke}{rgb}{1.000000,1.000000,1.000000}%
\pgfsetstrokecolor{currentstroke}%
\pgfsetdash{}{0pt}%
\pgfpathmoveto{\pgfqpoint{1.893096in}{1.934162in}}%
\pgfpathcurveto{\pgfqpoint{1.904146in}{1.934162in}}{\pgfqpoint{1.914745in}{1.938552in}}{\pgfqpoint{1.922559in}{1.946366in}}%
\pgfpathcurveto{\pgfqpoint{1.930372in}{1.954180in}}{\pgfqpoint{1.934762in}{1.964779in}}{\pgfqpoint{1.934762in}{1.975829in}}%
\pgfpathcurveto{\pgfqpoint{1.934762in}{1.986879in}}{\pgfqpoint{1.930372in}{1.997478in}}{\pgfqpoint{1.922559in}{2.005292in}}%
\pgfpathcurveto{\pgfqpoint{1.914745in}{2.013105in}}{\pgfqpoint{1.904146in}{2.017495in}}{\pgfqpoint{1.893096in}{2.017495in}}%
\pgfpathcurveto{\pgfqpoint{1.882046in}{2.017495in}}{\pgfqpoint{1.871447in}{2.013105in}}{\pgfqpoint{1.863633in}{2.005292in}}%
\pgfpathcurveto{\pgfqpoint{1.855819in}{1.997478in}}{\pgfqpoint{1.851429in}{1.986879in}}{\pgfqpoint{1.851429in}{1.975829in}}%
\pgfpathcurveto{\pgfqpoint{1.851429in}{1.964779in}}{\pgfqpoint{1.855819in}{1.954180in}}{\pgfqpoint{1.863633in}{1.946366in}}%
\pgfpathcurveto{\pgfqpoint{1.871447in}{1.938552in}}{\pgfqpoint{1.882046in}{1.934162in}}{\pgfqpoint{1.893096in}{1.934162in}}%
\pgfpathclose%
\pgfusepath{stroke,fill}%
\end{pgfscope}%
\begin{pgfscope}%
\pgfpathrectangle{\pgfqpoint{0.481978in}{0.331635in}}{\pgfqpoint{4.960000in}{3.696000in}}%
\pgfusepath{clip}%
\pgfsetbuttcap%
\pgfsetroundjoin%
\definecolor{currentfill}{rgb}{0.631373,0.788235,0.956863}%
\pgfsetfillcolor{currentfill}%
\pgfsetlinewidth{0.481800pt}%
\definecolor{currentstroke}{rgb}{1.000000,1.000000,1.000000}%
\pgfsetstrokecolor{currentstroke}%
\pgfsetdash{}{0pt}%
\pgfpathmoveto{\pgfqpoint{2.234546in}{1.405989in}}%
\pgfpathcurveto{\pgfqpoint{2.245596in}{1.405989in}}{\pgfqpoint{2.256195in}{1.410379in}}{\pgfqpoint{2.264009in}{1.418193in}}%
\pgfpathcurveto{\pgfqpoint{2.271823in}{1.426006in}}{\pgfqpoint{2.276213in}{1.436605in}}{\pgfqpoint{2.276213in}{1.447655in}}%
\pgfpathcurveto{\pgfqpoint{2.276213in}{1.458706in}}{\pgfqpoint{2.271823in}{1.469305in}}{\pgfqpoint{2.264009in}{1.477118in}}%
\pgfpathcurveto{\pgfqpoint{2.256195in}{1.484932in}}{\pgfqpoint{2.245596in}{1.489322in}}{\pgfqpoint{2.234546in}{1.489322in}}%
\pgfpathcurveto{\pgfqpoint{2.223496in}{1.489322in}}{\pgfqpoint{2.212897in}{1.484932in}}{\pgfqpoint{2.205084in}{1.477118in}}%
\pgfpathcurveto{\pgfqpoint{2.197270in}{1.469305in}}{\pgfqpoint{2.192880in}{1.458706in}}{\pgfqpoint{2.192880in}{1.447655in}}%
\pgfpathcurveto{\pgfqpoint{2.192880in}{1.436605in}}{\pgfqpoint{2.197270in}{1.426006in}}{\pgfqpoint{2.205084in}{1.418193in}}%
\pgfpathcurveto{\pgfqpoint{2.212897in}{1.410379in}}{\pgfqpoint{2.223496in}{1.405989in}}{\pgfqpoint{2.234546in}{1.405989in}}%
\pgfpathclose%
\pgfusepath{stroke,fill}%
\end{pgfscope}%
\begin{pgfscope}%
\pgfpathrectangle{\pgfqpoint{0.481978in}{0.331635in}}{\pgfqpoint{4.960000in}{3.696000in}}%
\pgfusepath{clip}%
\pgfsetbuttcap%
\pgfsetroundjoin%
\definecolor{currentfill}{rgb}{0.631373,0.788235,0.956863}%
\pgfsetfillcolor{currentfill}%
\pgfsetlinewidth{0.481800pt}%
\definecolor{currentstroke}{rgb}{1.000000,1.000000,1.000000}%
\pgfsetstrokecolor{currentstroke}%
\pgfsetdash{}{0pt}%
\pgfpathmoveto{\pgfqpoint{4.328140in}{1.967569in}}%
\pgfpathcurveto{\pgfqpoint{4.339190in}{1.967569in}}{\pgfqpoint{4.349789in}{1.971959in}}{\pgfqpoint{4.357603in}{1.979773in}}%
\pgfpathcurveto{\pgfqpoint{4.365416in}{1.987586in}}{\pgfqpoint{4.369807in}{1.998185in}}{\pgfqpoint{4.369807in}{2.009235in}}%
\pgfpathcurveto{\pgfqpoint{4.369807in}{2.020286in}}{\pgfqpoint{4.365416in}{2.030885in}}{\pgfqpoint{4.357603in}{2.038698in}}%
\pgfpathcurveto{\pgfqpoint{4.349789in}{2.046512in}}{\pgfqpoint{4.339190in}{2.050902in}}{\pgfqpoint{4.328140in}{2.050902in}}%
\pgfpathcurveto{\pgfqpoint{4.317090in}{2.050902in}}{\pgfqpoint{4.306491in}{2.046512in}}{\pgfqpoint{4.298677in}{2.038698in}}%
\pgfpathcurveto{\pgfqpoint{4.290864in}{2.030885in}}{\pgfqpoint{4.286473in}{2.020286in}}{\pgfqpoint{4.286473in}{2.009235in}}%
\pgfpathcurveto{\pgfqpoint{4.286473in}{1.998185in}}{\pgfqpoint{4.290864in}{1.987586in}}{\pgfqpoint{4.298677in}{1.979773in}}%
\pgfpathcurveto{\pgfqpoint{4.306491in}{1.971959in}}{\pgfqpoint{4.317090in}{1.967569in}}{\pgfqpoint{4.328140in}{1.967569in}}%
\pgfpathclose%
\pgfusepath{stroke,fill}%
\end{pgfscope}%
\begin{pgfscope}%
\pgfpathrectangle{\pgfqpoint{0.481978in}{0.331635in}}{\pgfqpoint{4.960000in}{3.696000in}}%
\pgfusepath{clip}%
\pgfsetbuttcap%
\pgfsetroundjoin%
\definecolor{currentfill}{rgb}{0.631373,0.788235,0.956863}%
\pgfsetfillcolor{currentfill}%
\pgfsetlinewidth{0.481800pt}%
\definecolor{currentstroke}{rgb}{1.000000,1.000000,1.000000}%
\pgfsetstrokecolor{currentstroke}%
\pgfsetdash{}{0pt}%
\pgfpathmoveto{\pgfqpoint{1.873549in}{2.011988in}}%
\pgfpathcurveto{\pgfqpoint{1.884600in}{2.011988in}}{\pgfqpoint{1.895199in}{2.016378in}}{\pgfqpoint{1.903012in}{2.024192in}}%
\pgfpathcurveto{\pgfqpoint{1.910826in}{2.032006in}}{\pgfqpoint{1.915216in}{2.042605in}}{\pgfqpoint{1.915216in}{2.053655in}}%
\pgfpathcurveto{\pgfqpoint{1.915216in}{2.064705in}}{\pgfqpoint{1.910826in}{2.075304in}}{\pgfqpoint{1.903012in}{2.083118in}}%
\pgfpathcurveto{\pgfqpoint{1.895199in}{2.090931in}}{\pgfqpoint{1.884600in}{2.095322in}}{\pgfqpoint{1.873549in}{2.095322in}}%
\pgfpathcurveto{\pgfqpoint{1.862499in}{2.095322in}}{\pgfqpoint{1.851900in}{2.090931in}}{\pgfqpoint{1.844087in}{2.083118in}}%
\pgfpathcurveto{\pgfqpoint{1.836273in}{2.075304in}}{\pgfqpoint{1.831883in}{2.064705in}}{\pgfqpoint{1.831883in}{2.053655in}}%
\pgfpathcurveto{\pgfqpoint{1.831883in}{2.042605in}}{\pgfqpoint{1.836273in}{2.032006in}}{\pgfqpoint{1.844087in}{2.024192in}}%
\pgfpathcurveto{\pgfqpoint{1.851900in}{2.016378in}}{\pgfqpoint{1.862499in}{2.011988in}}{\pgfqpoint{1.873549in}{2.011988in}}%
\pgfpathclose%
\pgfusepath{stroke,fill}%
\end{pgfscope}%
\begin{pgfscope}%
\pgfpathrectangle{\pgfqpoint{0.481978in}{0.331635in}}{\pgfqpoint{4.960000in}{3.696000in}}%
\pgfusepath{clip}%
\pgfsetbuttcap%
\pgfsetroundjoin%
\definecolor{currentfill}{rgb}{0.631373,0.788235,0.956863}%
\pgfsetfillcolor{currentfill}%
\pgfsetlinewidth{0.481800pt}%
\definecolor{currentstroke}{rgb}{1.000000,1.000000,1.000000}%
\pgfsetstrokecolor{currentstroke}%
\pgfsetdash{}{0pt}%
\pgfpathmoveto{\pgfqpoint{2.899915in}{3.275151in}}%
\pgfpathcurveto{\pgfqpoint{2.910965in}{3.275151in}}{\pgfqpoint{2.921564in}{3.279541in}}{\pgfqpoint{2.929377in}{3.287355in}}%
\pgfpathcurveto{\pgfqpoint{2.937191in}{3.295168in}}{\pgfqpoint{2.941581in}{3.305767in}}{\pgfqpoint{2.941581in}{3.316818in}}%
\pgfpathcurveto{\pgfqpoint{2.941581in}{3.327868in}}{\pgfqpoint{2.937191in}{3.338467in}}{\pgfqpoint{2.929377in}{3.346280in}}%
\pgfpathcurveto{\pgfqpoint{2.921564in}{3.354094in}}{\pgfqpoint{2.910965in}{3.358484in}}{\pgfqpoint{2.899915in}{3.358484in}}%
\pgfpathcurveto{\pgfqpoint{2.888865in}{3.358484in}}{\pgfqpoint{2.878266in}{3.354094in}}{\pgfqpoint{2.870452in}{3.346280in}}%
\pgfpathcurveto{\pgfqpoint{2.862638in}{3.338467in}}{\pgfqpoint{2.858248in}{3.327868in}}{\pgfqpoint{2.858248in}{3.316818in}}%
\pgfpathcurveto{\pgfqpoint{2.858248in}{3.305767in}}{\pgfqpoint{2.862638in}{3.295168in}}{\pgfqpoint{2.870452in}{3.287355in}}%
\pgfpathcurveto{\pgfqpoint{2.878266in}{3.279541in}}{\pgfqpoint{2.888865in}{3.275151in}}{\pgfqpoint{2.899915in}{3.275151in}}%
\pgfpathclose%
\pgfusepath{stroke,fill}%
\end{pgfscope}%
\begin{pgfscope}%
\pgfpathrectangle{\pgfqpoint{0.481978in}{0.331635in}}{\pgfqpoint{4.960000in}{3.696000in}}%
\pgfusepath{clip}%
\pgfsetbuttcap%
\pgfsetroundjoin%
\definecolor{currentfill}{rgb}{0.631373,0.788235,0.956863}%
\pgfsetfillcolor{currentfill}%
\pgfsetlinewidth{0.481800pt}%
\definecolor{currentstroke}{rgb}{1.000000,1.000000,1.000000}%
\pgfsetstrokecolor{currentstroke}%
\pgfsetdash{}{0pt}%
\pgfpathmoveto{\pgfqpoint{2.149070in}{2.537965in}}%
\pgfpathcurveto{\pgfqpoint{2.160120in}{2.537965in}}{\pgfqpoint{2.170719in}{2.542355in}}{\pgfqpoint{2.178532in}{2.550169in}}%
\pgfpathcurveto{\pgfqpoint{2.186346in}{2.557982in}}{\pgfqpoint{2.190736in}{2.568581in}}{\pgfqpoint{2.190736in}{2.579631in}}%
\pgfpathcurveto{\pgfqpoint{2.190736in}{2.590681in}}{\pgfqpoint{2.186346in}{2.601281in}}{\pgfqpoint{2.178532in}{2.609094in}}%
\pgfpathcurveto{\pgfqpoint{2.170719in}{2.616908in}}{\pgfqpoint{2.160120in}{2.621298in}}{\pgfqpoint{2.149070in}{2.621298in}}%
\pgfpathcurveto{\pgfqpoint{2.138019in}{2.621298in}}{\pgfqpoint{2.127420in}{2.616908in}}{\pgfqpoint{2.119607in}{2.609094in}}%
\pgfpathcurveto{\pgfqpoint{2.111793in}{2.601281in}}{\pgfqpoint{2.107403in}{2.590681in}}{\pgfqpoint{2.107403in}{2.579631in}}%
\pgfpathcurveto{\pgfqpoint{2.107403in}{2.568581in}}{\pgfqpoint{2.111793in}{2.557982in}}{\pgfqpoint{2.119607in}{2.550169in}}%
\pgfpathcurveto{\pgfqpoint{2.127420in}{2.542355in}}{\pgfqpoint{2.138019in}{2.537965in}}{\pgfqpoint{2.149070in}{2.537965in}}%
\pgfpathclose%
\pgfusepath{stroke,fill}%
\end{pgfscope}%
\begin{pgfscope}%
\pgfpathrectangle{\pgfqpoint{0.481978in}{0.331635in}}{\pgfqpoint{4.960000in}{3.696000in}}%
\pgfusepath{clip}%
\pgfsetbuttcap%
\pgfsetroundjoin%
\definecolor{currentfill}{rgb}{0.631373,0.788235,0.956863}%
\pgfsetfillcolor{currentfill}%
\pgfsetlinewidth{0.481800pt}%
\definecolor{currentstroke}{rgb}{1.000000,1.000000,1.000000}%
\pgfsetstrokecolor{currentstroke}%
\pgfsetdash{}{0pt}%
\pgfpathmoveto{\pgfqpoint{3.237696in}{2.617393in}}%
\pgfpathcurveto{\pgfqpoint{3.248746in}{2.617393in}}{\pgfqpoint{3.259345in}{2.621783in}}{\pgfqpoint{3.267159in}{2.629597in}}%
\pgfpathcurveto{\pgfqpoint{3.274972in}{2.637410in}}{\pgfqpoint{3.279363in}{2.648009in}}{\pgfqpoint{3.279363in}{2.659060in}}%
\pgfpathcurveto{\pgfqpoint{3.279363in}{2.670110in}}{\pgfqpoint{3.274972in}{2.680709in}}{\pgfqpoint{3.267159in}{2.688522in}}%
\pgfpathcurveto{\pgfqpoint{3.259345in}{2.696336in}}{\pgfqpoint{3.248746in}{2.700726in}}{\pgfqpoint{3.237696in}{2.700726in}}%
\pgfpathcurveto{\pgfqpoint{3.226646in}{2.700726in}}{\pgfqpoint{3.216047in}{2.696336in}}{\pgfqpoint{3.208233in}{2.688522in}}%
\pgfpathcurveto{\pgfqpoint{3.200420in}{2.680709in}}{\pgfqpoint{3.196029in}{2.670110in}}{\pgfqpoint{3.196029in}{2.659060in}}%
\pgfpathcurveto{\pgfqpoint{3.196029in}{2.648009in}}{\pgfqpoint{3.200420in}{2.637410in}}{\pgfqpoint{3.208233in}{2.629597in}}%
\pgfpathcurveto{\pgfqpoint{3.216047in}{2.621783in}}{\pgfqpoint{3.226646in}{2.617393in}}{\pgfqpoint{3.237696in}{2.617393in}}%
\pgfpathclose%
\pgfusepath{stroke,fill}%
\end{pgfscope}%
\begin{pgfscope}%
\pgfpathrectangle{\pgfqpoint{0.481978in}{0.331635in}}{\pgfqpoint{4.960000in}{3.696000in}}%
\pgfusepath{clip}%
\pgfsetbuttcap%
\pgfsetroundjoin%
\definecolor{currentfill}{rgb}{0.631373,0.788235,0.956863}%
\pgfsetfillcolor{currentfill}%
\pgfsetlinewidth{0.481800pt}%
\definecolor{currentstroke}{rgb}{1.000000,1.000000,1.000000}%
\pgfsetstrokecolor{currentstroke}%
\pgfsetdash{}{0pt}%
\pgfpathmoveto{\pgfqpoint{2.401125in}{1.307687in}}%
\pgfpathcurveto{\pgfqpoint{2.412175in}{1.307687in}}{\pgfqpoint{2.422774in}{1.312077in}}{\pgfqpoint{2.430588in}{1.319891in}}%
\pgfpathcurveto{\pgfqpoint{2.438402in}{1.327705in}}{\pgfqpoint{2.442792in}{1.338304in}}{\pgfqpoint{2.442792in}{1.349354in}}%
\pgfpathcurveto{\pgfqpoint{2.442792in}{1.360404in}}{\pgfqpoint{2.438402in}{1.371003in}}{\pgfqpoint{2.430588in}{1.378817in}}%
\pgfpathcurveto{\pgfqpoint{2.422774in}{1.386630in}}{\pgfqpoint{2.412175in}{1.391020in}}{\pgfqpoint{2.401125in}{1.391020in}}%
\pgfpathcurveto{\pgfqpoint{2.390075in}{1.391020in}}{\pgfqpoint{2.379476in}{1.386630in}}{\pgfqpoint{2.371662in}{1.378817in}}%
\pgfpathcurveto{\pgfqpoint{2.363849in}{1.371003in}}{\pgfqpoint{2.359458in}{1.360404in}}{\pgfqpoint{2.359458in}{1.349354in}}%
\pgfpathcurveto{\pgfqpoint{2.359458in}{1.338304in}}{\pgfqpoint{2.363849in}{1.327705in}}{\pgfqpoint{2.371662in}{1.319891in}}%
\pgfpathcurveto{\pgfqpoint{2.379476in}{1.312077in}}{\pgfqpoint{2.390075in}{1.307687in}}{\pgfqpoint{2.401125in}{1.307687in}}%
\pgfpathclose%
\pgfusepath{stroke,fill}%
\end{pgfscope}%
\begin{pgfscope}%
\pgfpathrectangle{\pgfqpoint{0.481978in}{0.331635in}}{\pgfqpoint{4.960000in}{3.696000in}}%
\pgfusepath{clip}%
\pgfsetbuttcap%
\pgfsetroundjoin%
\definecolor{currentfill}{rgb}{0.631373,0.788235,0.956863}%
\pgfsetfillcolor{currentfill}%
\pgfsetlinewidth{0.481800pt}%
\definecolor{currentstroke}{rgb}{1.000000,1.000000,1.000000}%
\pgfsetstrokecolor{currentstroke}%
\pgfsetdash{}{0pt}%
\pgfpathmoveto{\pgfqpoint{4.130878in}{1.752719in}}%
\pgfpathcurveto{\pgfqpoint{4.141928in}{1.752719in}}{\pgfqpoint{4.152528in}{1.757109in}}{\pgfqpoint{4.160341in}{1.764923in}}%
\pgfpathcurveto{\pgfqpoint{4.168155in}{1.772737in}}{\pgfqpoint{4.172545in}{1.783336in}}{\pgfqpoint{4.172545in}{1.794386in}}%
\pgfpathcurveto{\pgfqpoint{4.172545in}{1.805436in}}{\pgfqpoint{4.168155in}{1.816035in}}{\pgfqpoint{4.160341in}{1.823848in}}%
\pgfpathcurveto{\pgfqpoint{4.152528in}{1.831662in}}{\pgfqpoint{4.141928in}{1.836052in}}{\pgfqpoint{4.130878in}{1.836052in}}%
\pgfpathcurveto{\pgfqpoint{4.119828in}{1.836052in}}{\pgfqpoint{4.109229in}{1.831662in}}{\pgfqpoint{4.101416in}{1.823848in}}%
\pgfpathcurveto{\pgfqpoint{4.093602in}{1.816035in}}{\pgfqpoint{4.089212in}{1.805436in}}{\pgfqpoint{4.089212in}{1.794386in}}%
\pgfpathcurveto{\pgfqpoint{4.089212in}{1.783336in}}{\pgfqpoint{4.093602in}{1.772737in}}{\pgfqpoint{4.101416in}{1.764923in}}%
\pgfpathcurveto{\pgfqpoint{4.109229in}{1.757109in}}{\pgfqpoint{4.119828in}{1.752719in}}{\pgfqpoint{4.130878in}{1.752719in}}%
\pgfpathclose%
\pgfusepath{stroke,fill}%
\end{pgfscope}%
\begin{pgfscope}%
\pgfpathrectangle{\pgfqpoint{0.481978in}{0.331635in}}{\pgfqpoint{4.960000in}{3.696000in}}%
\pgfusepath{clip}%
\pgfsetbuttcap%
\pgfsetroundjoin%
\definecolor{currentfill}{rgb}{0.631373,0.788235,0.956863}%
\pgfsetfillcolor{currentfill}%
\pgfsetlinewidth{0.481800pt}%
\definecolor{currentstroke}{rgb}{1.000000,1.000000,1.000000}%
\pgfsetstrokecolor{currentstroke}%
\pgfsetdash{}{0pt}%
\pgfpathmoveto{\pgfqpoint{3.635936in}{1.956774in}}%
\pgfpathcurveto{\pgfqpoint{3.646987in}{1.956774in}}{\pgfqpoint{3.657586in}{1.961164in}}{\pgfqpoint{3.665399in}{1.968978in}}%
\pgfpathcurveto{\pgfqpoint{3.673213in}{1.976791in}}{\pgfqpoint{3.677603in}{1.987390in}}{\pgfqpoint{3.677603in}{1.998440in}}%
\pgfpathcurveto{\pgfqpoint{3.677603in}{2.009490in}}{\pgfqpoint{3.673213in}{2.020090in}}{\pgfqpoint{3.665399in}{2.027903in}}%
\pgfpathcurveto{\pgfqpoint{3.657586in}{2.035717in}}{\pgfqpoint{3.646987in}{2.040107in}}{\pgfqpoint{3.635936in}{2.040107in}}%
\pgfpathcurveto{\pgfqpoint{3.624886in}{2.040107in}}{\pgfqpoint{3.614287in}{2.035717in}}{\pgfqpoint{3.606474in}{2.027903in}}%
\pgfpathcurveto{\pgfqpoint{3.598660in}{2.020090in}}{\pgfqpoint{3.594270in}{2.009490in}}{\pgfqpoint{3.594270in}{1.998440in}}%
\pgfpathcurveto{\pgfqpoint{3.594270in}{1.987390in}}{\pgfqpoint{3.598660in}{1.976791in}}{\pgfqpoint{3.606474in}{1.968978in}}%
\pgfpathcurveto{\pgfqpoint{3.614287in}{1.961164in}}{\pgfqpoint{3.624886in}{1.956774in}}{\pgfqpoint{3.635936in}{1.956774in}}%
\pgfpathclose%
\pgfusepath{stroke,fill}%
\end{pgfscope}%
\begin{pgfscope}%
\pgfpathrectangle{\pgfqpoint{0.481978in}{0.331635in}}{\pgfqpoint{4.960000in}{3.696000in}}%
\pgfusepath{clip}%
\pgfsetbuttcap%
\pgfsetroundjoin%
\definecolor{currentfill}{rgb}{0.631373,0.788235,0.956863}%
\pgfsetfillcolor{currentfill}%
\pgfsetlinewidth{0.481800pt}%
\definecolor{currentstroke}{rgb}{1.000000,1.000000,1.000000}%
\pgfsetstrokecolor{currentstroke}%
\pgfsetdash{}{0pt}%
\pgfpathmoveto{\pgfqpoint{2.542160in}{2.502002in}}%
\pgfpathcurveto{\pgfqpoint{2.553210in}{2.502002in}}{\pgfqpoint{2.563809in}{2.506392in}}{\pgfqpoint{2.571623in}{2.514206in}}%
\pgfpathcurveto{\pgfqpoint{2.579436in}{2.522020in}}{\pgfqpoint{2.583826in}{2.532619in}}{\pgfqpoint{2.583826in}{2.543669in}}%
\pgfpathcurveto{\pgfqpoint{2.583826in}{2.554719in}}{\pgfqpoint{2.579436in}{2.565318in}}{\pgfqpoint{2.571623in}{2.573132in}}%
\pgfpathcurveto{\pgfqpoint{2.563809in}{2.580945in}}{\pgfqpoint{2.553210in}{2.585335in}}{\pgfqpoint{2.542160in}{2.585335in}}%
\pgfpathcurveto{\pgfqpoint{2.531110in}{2.585335in}}{\pgfqpoint{2.520511in}{2.580945in}}{\pgfqpoint{2.512697in}{2.573132in}}%
\pgfpathcurveto{\pgfqpoint{2.504883in}{2.565318in}}{\pgfqpoint{2.500493in}{2.554719in}}{\pgfqpoint{2.500493in}{2.543669in}}%
\pgfpathcurveto{\pgfqpoint{2.500493in}{2.532619in}}{\pgfqpoint{2.504883in}{2.522020in}}{\pgfqpoint{2.512697in}{2.514206in}}%
\pgfpathcurveto{\pgfqpoint{2.520511in}{2.506392in}}{\pgfqpoint{2.531110in}{2.502002in}}{\pgfqpoint{2.542160in}{2.502002in}}%
\pgfpathclose%
\pgfusepath{stroke,fill}%
\end{pgfscope}%
\begin{pgfscope}%
\pgfpathrectangle{\pgfqpoint{0.481978in}{0.331635in}}{\pgfqpoint{4.960000in}{3.696000in}}%
\pgfusepath{clip}%
\pgfsetbuttcap%
\pgfsetroundjoin%
\definecolor{currentfill}{rgb}{0.631373,0.788235,0.956863}%
\pgfsetfillcolor{currentfill}%
\pgfsetlinewidth{0.481800pt}%
\definecolor{currentstroke}{rgb}{1.000000,1.000000,1.000000}%
\pgfsetstrokecolor{currentstroke}%
\pgfsetdash{}{0pt}%
\pgfpathmoveto{\pgfqpoint{3.421206in}{1.263997in}}%
\pgfpathcurveto{\pgfqpoint{3.432256in}{1.263997in}}{\pgfqpoint{3.442855in}{1.268387in}}{\pgfqpoint{3.450668in}{1.276201in}}%
\pgfpathcurveto{\pgfqpoint{3.458482in}{1.284015in}}{\pgfqpoint{3.462872in}{1.294614in}}{\pgfqpoint{3.462872in}{1.305664in}}%
\pgfpathcurveto{\pgfqpoint{3.462872in}{1.316714in}}{\pgfqpoint{3.458482in}{1.327313in}}{\pgfqpoint{3.450668in}{1.335127in}}%
\pgfpathcurveto{\pgfqpoint{3.442855in}{1.342940in}}{\pgfqpoint{3.432256in}{1.347330in}}{\pgfqpoint{3.421206in}{1.347330in}}%
\pgfpathcurveto{\pgfqpoint{3.410155in}{1.347330in}}{\pgfqpoint{3.399556in}{1.342940in}}{\pgfqpoint{3.391743in}{1.335127in}}%
\pgfpathcurveto{\pgfqpoint{3.383929in}{1.327313in}}{\pgfqpoint{3.379539in}{1.316714in}}{\pgfqpoint{3.379539in}{1.305664in}}%
\pgfpathcurveto{\pgfqpoint{3.379539in}{1.294614in}}{\pgfqpoint{3.383929in}{1.284015in}}{\pgfqpoint{3.391743in}{1.276201in}}%
\pgfpathcurveto{\pgfqpoint{3.399556in}{1.268387in}}{\pgfqpoint{3.410155in}{1.263997in}}{\pgfqpoint{3.421206in}{1.263997in}}%
\pgfpathclose%
\pgfusepath{stroke,fill}%
\end{pgfscope}%
\begin{pgfscope}%
\pgfpathrectangle{\pgfqpoint{0.481978in}{0.331635in}}{\pgfqpoint{4.960000in}{3.696000in}}%
\pgfusepath{clip}%
\pgfsetbuttcap%
\pgfsetroundjoin%
\definecolor{currentfill}{rgb}{0.631373,0.788235,0.956863}%
\pgfsetfillcolor{currentfill}%
\pgfsetlinewidth{0.481800pt}%
\definecolor{currentstroke}{rgb}{1.000000,1.000000,1.000000}%
\pgfsetstrokecolor{currentstroke}%
\pgfsetdash{}{0pt}%
\pgfpathmoveto{\pgfqpoint{3.349846in}{1.230815in}}%
\pgfpathcurveto{\pgfqpoint{3.360896in}{1.230815in}}{\pgfqpoint{3.371495in}{1.235205in}}{\pgfqpoint{3.379309in}{1.243019in}}%
\pgfpathcurveto{\pgfqpoint{3.387123in}{1.250833in}}{\pgfqpoint{3.391513in}{1.261432in}}{\pgfqpoint{3.391513in}{1.272482in}}%
\pgfpathcurveto{\pgfqpoint{3.391513in}{1.283532in}}{\pgfqpoint{3.387123in}{1.294131in}}{\pgfqpoint{3.379309in}{1.301945in}}%
\pgfpathcurveto{\pgfqpoint{3.371495in}{1.309758in}}{\pgfqpoint{3.360896in}{1.314149in}}{\pgfqpoint{3.349846in}{1.314149in}}%
\pgfpathcurveto{\pgfqpoint{3.338796in}{1.314149in}}{\pgfqpoint{3.328197in}{1.309758in}}{\pgfqpoint{3.320383in}{1.301945in}}%
\pgfpathcurveto{\pgfqpoint{3.312570in}{1.294131in}}{\pgfqpoint{3.308180in}{1.283532in}}{\pgfqpoint{3.308180in}{1.272482in}}%
\pgfpathcurveto{\pgfqpoint{3.308180in}{1.261432in}}{\pgfqpoint{3.312570in}{1.250833in}}{\pgfqpoint{3.320383in}{1.243019in}}%
\pgfpathcurveto{\pgfqpoint{3.328197in}{1.235205in}}{\pgfqpoint{3.338796in}{1.230815in}}{\pgfqpoint{3.349846in}{1.230815in}}%
\pgfpathclose%
\pgfusepath{stroke,fill}%
\end{pgfscope}%
\begin{pgfscope}%
\pgfpathrectangle{\pgfqpoint{0.481978in}{0.331635in}}{\pgfqpoint{4.960000in}{3.696000in}}%
\pgfusepath{clip}%
\pgfsetbuttcap%
\pgfsetroundjoin%
\definecolor{currentfill}{rgb}{0.631373,0.788235,0.956863}%
\pgfsetfillcolor{currentfill}%
\pgfsetlinewidth{0.481800pt}%
\definecolor{currentstroke}{rgb}{1.000000,1.000000,1.000000}%
\pgfsetstrokecolor{currentstroke}%
\pgfsetdash{}{0pt}%
\pgfpathmoveto{\pgfqpoint{2.216571in}{1.599785in}}%
\pgfpathcurveto{\pgfqpoint{2.227621in}{1.599785in}}{\pgfqpoint{2.238220in}{1.604176in}}{\pgfqpoint{2.246034in}{1.611989in}}%
\pgfpathcurveto{\pgfqpoint{2.253848in}{1.619803in}}{\pgfqpoint{2.258238in}{1.630402in}}{\pgfqpoint{2.258238in}{1.641452in}}%
\pgfpathcurveto{\pgfqpoint{2.258238in}{1.652502in}}{\pgfqpoint{2.253848in}{1.663101in}}{\pgfqpoint{2.246034in}{1.670915in}}%
\pgfpathcurveto{\pgfqpoint{2.238220in}{1.678728in}}{\pgfqpoint{2.227621in}{1.683119in}}{\pgfqpoint{2.216571in}{1.683119in}}%
\pgfpathcurveto{\pgfqpoint{2.205521in}{1.683119in}}{\pgfqpoint{2.194922in}{1.678728in}}{\pgfqpoint{2.187109in}{1.670915in}}%
\pgfpathcurveto{\pgfqpoint{2.179295in}{1.663101in}}{\pgfqpoint{2.174905in}{1.652502in}}{\pgfqpoint{2.174905in}{1.641452in}}%
\pgfpathcurveto{\pgfqpoint{2.174905in}{1.630402in}}{\pgfqpoint{2.179295in}{1.619803in}}{\pgfqpoint{2.187109in}{1.611989in}}%
\pgfpathcurveto{\pgfqpoint{2.194922in}{1.604176in}}{\pgfqpoint{2.205521in}{1.599785in}}{\pgfqpoint{2.216571in}{1.599785in}}%
\pgfpathclose%
\pgfusepath{stroke,fill}%
\end{pgfscope}%
\begin{pgfscope}%
\pgfpathrectangle{\pgfqpoint{0.481978in}{0.331635in}}{\pgfqpoint{4.960000in}{3.696000in}}%
\pgfusepath{clip}%
\pgfsetbuttcap%
\pgfsetroundjoin%
\definecolor{currentfill}{rgb}{0.631373,0.788235,0.956863}%
\pgfsetfillcolor{currentfill}%
\pgfsetlinewidth{0.481800pt}%
\definecolor{currentstroke}{rgb}{1.000000,1.000000,1.000000}%
\pgfsetstrokecolor{currentstroke}%
\pgfsetdash{}{0pt}%
\pgfpathmoveto{\pgfqpoint{1.969421in}{2.435703in}}%
\pgfpathcurveto{\pgfqpoint{1.980471in}{2.435703in}}{\pgfqpoint{1.991070in}{2.440093in}}{\pgfqpoint{1.998883in}{2.447907in}}%
\pgfpathcurveto{\pgfqpoint{2.006697in}{2.455721in}}{\pgfqpoint{2.011087in}{2.466320in}}{\pgfqpoint{2.011087in}{2.477370in}}%
\pgfpathcurveto{\pgfqpoint{2.011087in}{2.488420in}}{\pgfqpoint{2.006697in}{2.499019in}}{\pgfqpoint{1.998883in}{2.506833in}}%
\pgfpathcurveto{\pgfqpoint{1.991070in}{2.514646in}}{\pgfqpoint{1.980471in}{2.519037in}}{\pgfqpoint{1.969421in}{2.519037in}}%
\pgfpathcurveto{\pgfqpoint{1.958371in}{2.519037in}}{\pgfqpoint{1.947771in}{2.514646in}}{\pgfqpoint{1.939958in}{2.506833in}}%
\pgfpathcurveto{\pgfqpoint{1.932144in}{2.499019in}}{\pgfqpoint{1.927754in}{2.488420in}}{\pgfqpoint{1.927754in}{2.477370in}}%
\pgfpathcurveto{\pgfqpoint{1.927754in}{2.466320in}}{\pgfqpoint{1.932144in}{2.455721in}}{\pgfqpoint{1.939958in}{2.447907in}}%
\pgfpathcurveto{\pgfqpoint{1.947771in}{2.440093in}}{\pgfqpoint{1.958371in}{2.435703in}}{\pgfqpoint{1.969421in}{2.435703in}}%
\pgfpathclose%
\pgfusepath{stroke,fill}%
\end{pgfscope}%
\begin{pgfscope}%
\pgfpathrectangle{\pgfqpoint{0.481978in}{0.331635in}}{\pgfqpoint{4.960000in}{3.696000in}}%
\pgfusepath{clip}%
\pgfsetbuttcap%
\pgfsetroundjoin%
\definecolor{currentfill}{rgb}{0.631373,0.788235,0.956863}%
\pgfsetfillcolor{currentfill}%
\pgfsetlinewidth{0.481800pt}%
\definecolor{currentstroke}{rgb}{1.000000,1.000000,1.000000}%
\pgfsetstrokecolor{currentstroke}%
\pgfsetdash{}{0pt}%
\pgfpathmoveto{\pgfqpoint{3.679572in}{2.176972in}}%
\pgfpathcurveto{\pgfqpoint{3.690622in}{2.176972in}}{\pgfqpoint{3.701221in}{2.181362in}}{\pgfqpoint{3.709035in}{2.189176in}}%
\pgfpathcurveto{\pgfqpoint{3.716849in}{2.196989in}}{\pgfqpoint{3.721239in}{2.207588in}}{\pgfqpoint{3.721239in}{2.218638in}}%
\pgfpathcurveto{\pgfqpoint{3.721239in}{2.229689in}}{\pgfqpoint{3.716849in}{2.240288in}}{\pgfqpoint{3.709035in}{2.248101in}}%
\pgfpathcurveto{\pgfqpoint{3.701221in}{2.255915in}}{\pgfqpoint{3.690622in}{2.260305in}}{\pgfqpoint{3.679572in}{2.260305in}}%
\pgfpathcurveto{\pgfqpoint{3.668522in}{2.260305in}}{\pgfqpoint{3.657923in}{2.255915in}}{\pgfqpoint{3.650110in}{2.248101in}}%
\pgfpathcurveto{\pgfqpoint{3.642296in}{2.240288in}}{\pgfqpoint{3.637906in}{2.229689in}}{\pgfqpoint{3.637906in}{2.218638in}}%
\pgfpathcurveto{\pgfqpoint{3.637906in}{2.207588in}}{\pgfqpoint{3.642296in}{2.196989in}}{\pgfqpoint{3.650110in}{2.189176in}}%
\pgfpathcurveto{\pgfqpoint{3.657923in}{2.181362in}}{\pgfqpoint{3.668522in}{2.176972in}}{\pgfqpoint{3.679572in}{2.176972in}}%
\pgfpathclose%
\pgfusepath{stroke,fill}%
\end{pgfscope}%
\begin{pgfscope}%
\pgfpathrectangle{\pgfqpoint{0.481978in}{0.331635in}}{\pgfqpoint{4.960000in}{3.696000in}}%
\pgfusepath{clip}%
\pgfsetbuttcap%
\pgfsetroundjoin%
\definecolor{currentfill}{rgb}{0.631373,0.788235,0.956863}%
\pgfsetfillcolor{currentfill}%
\pgfsetlinewidth{0.481800pt}%
\definecolor{currentstroke}{rgb}{1.000000,1.000000,1.000000}%
\pgfsetstrokecolor{currentstroke}%
\pgfsetdash{}{0pt}%
\pgfpathmoveto{\pgfqpoint{2.384921in}{2.542120in}}%
\pgfpathcurveto{\pgfqpoint{2.395971in}{2.542120in}}{\pgfqpoint{2.406570in}{2.546510in}}{\pgfqpoint{2.414383in}{2.554323in}}%
\pgfpathcurveto{\pgfqpoint{2.422197in}{2.562137in}}{\pgfqpoint{2.426587in}{2.572736in}}{\pgfqpoint{2.426587in}{2.583786in}}%
\pgfpathcurveto{\pgfqpoint{2.426587in}{2.594836in}}{\pgfqpoint{2.422197in}{2.605435in}}{\pgfqpoint{2.414383in}{2.613249in}}%
\pgfpathcurveto{\pgfqpoint{2.406570in}{2.621063in}}{\pgfqpoint{2.395971in}{2.625453in}}{\pgfqpoint{2.384921in}{2.625453in}}%
\pgfpathcurveto{\pgfqpoint{2.373870in}{2.625453in}}{\pgfqpoint{2.363271in}{2.621063in}}{\pgfqpoint{2.355458in}{2.613249in}}%
\pgfpathcurveto{\pgfqpoint{2.347644in}{2.605435in}}{\pgfqpoint{2.343254in}{2.594836in}}{\pgfqpoint{2.343254in}{2.583786in}}%
\pgfpathcurveto{\pgfqpoint{2.343254in}{2.572736in}}{\pgfqpoint{2.347644in}{2.562137in}}{\pgfqpoint{2.355458in}{2.554323in}}%
\pgfpathcurveto{\pgfqpoint{2.363271in}{2.546510in}}{\pgfqpoint{2.373870in}{2.542120in}}{\pgfqpoint{2.384921in}{2.542120in}}%
\pgfpathclose%
\pgfusepath{stroke,fill}%
\end{pgfscope}%
\begin{pgfscope}%
\pgfpathrectangle{\pgfqpoint{0.481978in}{0.331635in}}{\pgfqpoint{4.960000in}{3.696000in}}%
\pgfusepath{clip}%
\pgfsetbuttcap%
\pgfsetroundjoin%
\definecolor{currentfill}{rgb}{0.631373,0.788235,0.956863}%
\pgfsetfillcolor{currentfill}%
\pgfsetlinewidth{0.481800pt}%
\definecolor{currentstroke}{rgb}{1.000000,1.000000,1.000000}%
\pgfsetstrokecolor{currentstroke}%
\pgfsetdash{}{0pt}%
\pgfpathmoveto{\pgfqpoint{3.016150in}{2.573069in}}%
\pgfpathcurveto{\pgfqpoint{3.027201in}{2.573069in}}{\pgfqpoint{3.037800in}{2.577459in}}{\pgfqpoint{3.045613in}{2.585273in}}%
\pgfpathcurveto{\pgfqpoint{3.053427in}{2.593086in}}{\pgfqpoint{3.057817in}{2.603686in}}{\pgfqpoint{3.057817in}{2.614736in}}%
\pgfpathcurveto{\pgfqpoint{3.057817in}{2.625786in}}{\pgfqpoint{3.053427in}{2.636385in}}{\pgfqpoint{3.045613in}{2.644198in}}%
\pgfpathcurveto{\pgfqpoint{3.037800in}{2.652012in}}{\pgfqpoint{3.027201in}{2.656402in}}{\pgfqpoint{3.016150in}{2.656402in}}%
\pgfpathcurveto{\pgfqpoint{3.005100in}{2.656402in}}{\pgfqpoint{2.994501in}{2.652012in}}{\pgfqpoint{2.986688in}{2.644198in}}%
\pgfpathcurveto{\pgfqpoint{2.978874in}{2.636385in}}{\pgfqpoint{2.974484in}{2.625786in}}{\pgfqpoint{2.974484in}{2.614736in}}%
\pgfpathcurveto{\pgfqpoint{2.974484in}{2.603686in}}{\pgfqpoint{2.978874in}{2.593086in}}{\pgfqpoint{2.986688in}{2.585273in}}%
\pgfpathcurveto{\pgfqpoint{2.994501in}{2.577459in}}{\pgfqpoint{3.005100in}{2.573069in}}{\pgfqpoint{3.016150in}{2.573069in}}%
\pgfpathclose%
\pgfusepath{stroke,fill}%
\end{pgfscope}%
\begin{pgfscope}%
\pgfpathrectangle{\pgfqpoint{0.481978in}{0.331635in}}{\pgfqpoint{4.960000in}{3.696000in}}%
\pgfusepath{clip}%
\pgfsetbuttcap%
\pgfsetroundjoin%
\definecolor{currentfill}{rgb}{0.631373,0.788235,0.956863}%
\pgfsetfillcolor{currentfill}%
\pgfsetlinewidth{0.481800pt}%
\definecolor{currentstroke}{rgb}{1.000000,1.000000,1.000000}%
\pgfsetstrokecolor{currentstroke}%
\pgfsetdash{}{0pt}%
\pgfpathmoveto{\pgfqpoint{3.388934in}{2.093998in}}%
\pgfpathcurveto{\pgfqpoint{3.399984in}{2.093998in}}{\pgfqpoint{3.410583in}{2.098388in}}{\pgfqpoint{3.418397in}{2.106202in}}%
\pgfpathcurveto{\pgfqpoint{3.426210in}{2.114015in}}{\pgfqpoint{3.430600in}{2.124614in}}{\pgfqpoint{3.430600in}{2.135665in}}%
\pgfpathcurveto{\pgfqpoint{3.430600in}{2.146715in}}{\pgfqpoint{3.426210in}{2.157314in}}{\pgfqpoint{3.418397in}{2.165127in}}%
\pgfpathcurveto{\pgfqpoint{3.410583in}{2.172941in}}{\pgfqpoint{3.399984in}{2.177331in}}{\pgfqpoint{3.388934in}{2.177331in}}%
\pgfpathcurveto{\pgfqpoint{3.377884in}{2.177331in}}{\pgfqpoint{3.367285in}{2.172941in}}{\pgfqpoint{3.359471in}{2.165127in}}%
\pgfpathcurveto{\pgfqpoint{3.351657in}{2.157314in}}{\pgfqpoint{3.347267in}{2.146715in}}{\pgfqpoint{3.347267in}{2.135665in}}%
\pgfpathcurveto{\pgfqpoint{3.347267in}{2.124614in}}{\pgfqpoint{3.351657in}{2.114015in}}{\pgfqpoint{3.359471in}{2.106202in}}%
\pgfpathcurveto{\pgfqpoint{3.367285in}{2.098388in}}{\pgfqpoint{3.377884in}{2.093998in}}{\pgfqpoint{3.388934in}{2.093998in}}%
\pgfpathclose%
\pgfusepath{stroke,fill}%
\end{pgfscope}%
\begin{pgfscope}%
\pgfpathrectangle{\pgfqpoint{0.481978in}{0.331635in}}{\pgfqpoint{4.960000in}{3.696000in}}%
\pgfusepath{clip}%
\pgfsetbuttcap%
\pgfsetroundjoin%
\definecolor{currentfill}{rgb}{0.631373,0.788235,0.956863}%
\pgfsetfillcolor{currentfill}%
\pgfsetlinewidth{0.481800pt}%
\definecolor{currentstroke}{rgb}{1.000000,1.000000,1.000000}%
\pgfsetstrokecolor{currentstroke}%
\pgfsetdash{}{0pt}%
\pgfpathmoveto{\pgfqpoint{2.019457in}{1.670745in}}%
\pgfpathcurveto{\pgfqpoint{2.030507in}{1.670745in}}{\pgfqpoint{2.041106in}{1.675135in}}{\pgfqpoint{2.048919in}{1.682949in}}%
\pgfpathcurveto{\pgfqpoint{2.056733in}{1.690763in}}{\pgfqpoint{2.061123in}{1.701362in}}{\pgfqpoint{2.061123in}{1.712412in}}%
\pgfpathcurveto{\pgfqpoint{2.061123in}{1.723462in}}{\pgfqpoint{2.056733in}{1.734061in}}{\pgfqpoint{2.048919in}{1.741874in}}%
\pgfpathcurveto{\pgfqpoint{2.041106in}{1.749688in}}{\pgfqpoint{2.030507in}{1.754078in}}{\pgfqpoint{2.019457in}{1.754078in}}%
\pgfpathcurveto{\pgfqpoint{2.008406in}{1.754078in}}{\pgfqpoint{1.997807in}{1.749688in}}{\pgfqpoint{1.989994in}{1.741874in}}%
\pgfpathcurveto{\pgfqpoint{1.982180in}{1.734061in}}{\pgfqpoint{1.977790in}{1.723462in}}{\pgfqpoint{1.977790in}{1.712412in}}%
\pgfpathcurveto{\pgfqpoint{1.977790in}{1.701362in}}{\pgfqpoint{1.982180in}{1.690763in}}{\pgfqpoint{1.989994in}{1.682949in}}%
\pgfpathcurveto{\pgfqpoint{1.997807in}{1.675135in}}{\pgfqpoint{2.008406in}{1.670745in}}{\pgfqpoint{2.019457in}{1.670745in}}%
\pgfpathclose%
\pgfusepath{stroke,fill}%
\end{pgfscope}%
\begin{pgfscope}%
\pgfpathrectangle{\pgfqpoint{0.481978in}{0.331635in}}{\pgfqpoint{4.960000in}{3.696000in}}%
\pgfusepath{clip}%
\pgfsetbuttcap%
\pgfsetroundjoin%
\definecolor{currentfill}{rgb}{0.631373,0.788235,0.956863}%
\pgfsetfillcolor{currentfill}%
\pgfsetlinewidth{0.481800pt}%
\definecolor{currentstroke}{rgb}{1.000000,1.000000,1.000000}%
\pgfsetstrokecolor{currentstroke}%
\pgfsetdash{}{0pt}%
\pgfpathmoveto{\pgfqpoint{2.390462in}{2.188888in}}%
\pgfpathcurveto{\pgfqpoint{2.401512in}{2.188888in}}{\pgfqpoint{2.412111in}{2.193278in}}{\pgfqpoint{2.419925in}{2.201092in}}%
\pgfpathcurveto{\pgfqpoint{2.427738in}{2.208905in}}{\pgfqpoint{2.432129in}{2.219505in}}{\pgfqpoint{2.432129in}{2.230555in}}%
\pgfpathcurveto{\pgfqpoint{2.432129in}{2.241605in}}{\pgfqpoint{2.427738in}{2.252204in}}{\pgfqpoint{2.419925in}{2.260017in}}%
\pgfpathcurveto{\pgfqpoint{2.412111in}{2.267831in}}{\pgfqpoint{2.401512in}{2.272221in}}{\pgfqpoint{2.390462in}{2.272221in}}%
\pgfpathcurveto{\pgfqpoint{2.379412in}{2.272221in}}{\pgfqpoint{2.368813in}{2.267831in}}{\pgfqpoint{2.360999in}{2.260017in}}%
\pgfpathcurveto{\pgfqpoint{2.353186in}{2.252204in}}{\pgfqpoint{2.348795in}{2.241605in}}{\pgfqpoint{2.348795in}{2.230555in}}%
\pgfpathcurveto{\pgfqpoint{2.348795in}{2.219505in}}{\pgfqpoint{2.353186in}{2.208905in}}{\pgfqpoint{2.360999in}{2.201092in}}%
\pgfpathcurveto{\pgfqpoint{2.368813in}{2.193278in}}{\pgfqpoint{2.379412in}{2.188888in}}{\pgfqpoint{2.390462in}{2.188888in}}%
\pgfpathclose%
\pgfusepath{stroke,fill}%
\end{pgfscope}%
\begin{pgfscope}%
\pgfpathrectangle{\pgfqpoint{0.481978in}{0.331635in}}{\pgfqpoint{4.960000in}{3.696000in}}%
\pgfusepath{clip}%
\pgfsetbuttcap%
\pgfsetroundjoin%
\definecolor{currentfill}{rgb}{0.631373,0.788235,0.956863}%
\pgfsetfillcolor{currentfill}%
\pgfsetlinewidth{0.481800pt}%
\definecolor{currentstroke}{rgb}{1.000000,1.000000,1.000000}%
\pgfsetstrokecolor{currentstroke}%
\pgfsetdash{}{0pt}%
\pgfpathmoveto{\pgfqpoint{1.913384in}{2.698676in}}%
\pgfpathcurveto{\pgfqpoint{1.924434in}{2.698676in}}{\pgfqpoint{1.935033in}{2.703066in}}{\pgfqpoint{1.942847in}{2.710880in}}%
\pgfpathcurveto{\pgfqpoint{1.950661in}{2.718693in}}{\pgfqpoint{1.955051in}{2.729292in}}{\pgfqpoint{1.955051in}{2.740342in}}%
\pgfpathcurveto{\pgfqpoint{1.955051in}{2.751392in}}{\pgfqpoint{1.950661in}{2.761992in}}{\pgfqpoint{1.942847in}{2.769805in}}%
\pgfpathcurveto{\pgfqpoint{1.935033in}{2.777619in}}{\pgfqpoint{1.924434in}{2.782009in}}{\pgfqpoint{1.913384in}{2.782009in}}%
\pgfpathcurveto{\pgfqpoint{1.902334in}{2.782009in}}{\pgfqpoint{1.891735in}{2.777619in}}{\pgfqpoint{1.883922in}{2.769805in}}%
\pgfpathcurveto{\pgfqpoint{1.876108in}{2.761992in}}{\pgfqpoint{1.871718in}{2.751392in}}{\pgfqpoint{1.871718in}{2.740342in}}%
\pgfpathcurveto{\pgfqpoint{1.871718in}{2.729292in}}{\pgfqpoint{1.876108in}{2.718693in}}{\pgfqpoint{1.883922in}{2.710880in}}%
\pgfpathcurveto{\pgfqpoint{1.891735in}{2.703066in}}{\pgfqpoint{1.902334in}{2.698676in}}{\pgfqpoint{1.913384in}{2.698676in}}%
\pgfpathclose%
\pgfusepath{stroke,fill}%
\end{pgfscope}%
\begin{pgfscope}%
\pgfpathrectangle{\pgfqpoint{0.481978in}{0.331635in}}{\pgfqpoint{4.960000in}{3.696000in}}%
\pgfusepath{clip}%
\pgfsetbuttcap%
\pgfsetroundjoin%
\definecolor{currentfill}{rgb}{0.631373,0.788235,0.956863}%
\pgfsetfillcolor{currentfill}%
\pgfsetlinewidth{0.481800pt}%
\definecolor{currentstroke}{rgb}{1.000000,1.000000,1.000000}%
\pgfsetstrokecolor{currentstroke}%
\pgfsetdash{}{0pt}%
\pgfpathmoveto{\pgfqpoint{2.832331in}{1.030980in}}%
\pgfpathcurveto{\pgfqpoint{2.843381in}{1.030980in}}{\pgfqpoint{2.853980in}{1.035370in}}{\pgfqpoint{2.861794in}{1.043184in}}%
\pgfpathcurveto{\pgfqpoint{2.869607in}{1.050998in}}{\pgfqpoint{2.873998in}{1.061597in}}{\pgfqpoint{2.873998in}{1.072647in}}%
\pgfpathcurveto{\pgfqpoint{2.873998in}{1.083697in}}{\pgfqpoint{2.869607in}{1.094296in}}{\pgfqpoint{2.861794in}{1.102110in}}%
\pgfpathcurveto{\pgfqpoint{2.853980in}{1.109923in}}{\pgfqpoint{2.843381in}{1.114314in}}{\pgfqpoint{2.832331in}{1.114314in}}%
\pgfpathcurveto{\pgfqpoint{2.821281in}{1.114314in}}{\pgfqpoint{2.810682in}{1.109923in}}{\pgfqpoint{2.802868in}{1.102110in}}%
\pgfpathcurveto{\pgfqpoint{2.795055in}{1.094296in}}{\pgfqpoint{2.790664in}{1.083697in}}{\pgfqpoint{2.790664in}{1.072647in}}%
\pgfpathcurveto{\pgfqpoint{2.790664in}{1.061597in}}{\pgfqpoint{2.795055in}{1.050998in}}{\pgfqpoint{2.802868in}{1.043184in}}%
\pgfpathcurveto{\pgfqpoint{2.810682in}{1.035370in}}{\pgfqpoint{2.821281in}{1.030980in}}{\pgfqpoint{2.832331in}{1.030980in}}%
\pgfpathclose%
\pgfusepath{stroke,fill}%
\end{pgfscope}%
\begin{pgfscope}%
\pgfpathrectangle{\pgfqpoint{0.481978in}{0.331635in}}{\pgfqpoint{4.960000in}{3.696000in}}%
\pgfusepath{clip}%
\pgfsetbuttcap%
\pgfsetroundjoin%
\definecolor{currentfill}{rgb}{0.631373,0.788235,0.956863}%
\pgfsetfillcolor{currentfill}%
\pgfsetlinewidth{0.481800pt}%
\definecolor{currentstroke}{rgb}{1.000000,1.000000,1.000000}%
\pgfsetstrokecolor{currentstroke}%
\pgfsetdash{}{0pt}%
\pgfpathmoveto{\pgfqpoint{4.022929in}{2.032670in}}%
\pgfpathcurveto{\pgfqpoint{4.033979in}{2.032670in}}{\pgfqpoint{4.044578in}{2.037060in}}{\pgfqpoint{4.052392in}{2.044874in}}%
\pgfpathcurveto{\pgfqpoint{4.060206in}{2.052688in}}{\pgfqpoint{4.064596in}{2.063287in}}{\pgfqpoint{4.064596in}{2.074337in}}%
\pgfpathcurveto{\pgfqpoint{4.064596in}{2.085387in}}{\pgfqpoint{4.060206in}{2.095986in}}{\pgfqpoint{4.052392in}{2.103800in}}%
\pgfpathcurveto{\pgfqpoint{4.044578in}{2.111613in}}{\pgfqpoint{4.033979in}{2.116003in}}{\pgfqpoint{4.022929in}{2.116003in}}%
\pgfpathcurveto{\pgfqpoint{4.011879in}{2.116003in}}{\pgfqpoint{4.001280in}{2.111613in}}{\pgfqpoint{3.993466in}{2.103800in}}%
\pgfpathcurveto{\pgfqpoint{3.985653in}{2.095986in}}{\pgfqpoint{3.981263in}{2.085387in}}{\pgfqpoint{3.981263in}{2.074337in}}%
\pgfpathcurveto{\pgfqpoint{3.981263in}{2.063287in}}{\pgfqpoint{3.985653in}{2.052688in}}{\pgfqpoint{3.993466in}{2.044874in}}%
\pgfpathcurveto{\pgfqpoint{4.001280in}{2.037060in}}{\pgfqpoint{4.011879in}{2.032670in}}{\pgfqpoint{4.022929in}{2.032670in}}%
\pgfpathclose%
\pgfusepath{stroke,fill}%
\end{pgfscope}%
\begin{pgfscope}%
\pgfpathrectangle{\pgfqpoint{0.481978in}{0.331635in}}{\pgfqpoint{4.960000in}{3.696000in}}%
\pgfusepath{clip}%
\pgfsetbuttcap%
\pgfsetroundjoin%
\definecolor{currentfill}{rgb}{0.631373,0.788235,0.956863}%
\pgfsetfillcolor{currentfill}%
\pgfsetlinewidth{0.481800pt}%
\definecolor{currentstroke}{rgb}{1.000000,1.000000,1.000000}%
\pgfsetstrokecolor{currentstroke}%
\pgfsetdash{}{0pt}%
\pgfpathmoveto{\pgfqpoint{2.755220in}{1.677582in}}%
\pgfpathcurveto{\pgfqpoint{2.766270in}{1.677582in}}{\pgfqpoint{2.776869in}{1.681973in}}{\pgfqpoint{2.784682in}{1.689786in}}%
\pgfpathcurveto{\pgfqpoint{2.792496in}{1.697600in}}{\pgfqpoint{2.796886in}{1.708199in}}{\pgfqpoint{2.796886in}{1.719249in}}%
\pgfpathcurveto{\pgfqpoint{2.796886in}{1.730299in}}{\pgfqpoint{2.792496in}{1.740898in}}{\pgfqpoint{2.784682in}{1.748712in}}%
\pgfpathcurveto{\pgfqpoint{2.776869in}{1.756525in}}{\pgfqpoint{2.766270in}{1.760916in}}{\pgfqpoint{2.755220in}{1.760916in}}%
\pgfpathcurveto{\pgfqpoint{2.744169in}{1.760916in}}{\pgfqpoint{2.733570in}{1.756525in}}{\pgfqpoint{2.725757in}{1.748712in}}%
\pgfpathcurveto{\pgfqpoint{2.717943in}{1.740898in}}{\pgfqpoint{2.713553in}{1.730299in}}{\pgfqpoint{2.713553in}{1.719249in}}%
\pgfpathcurveto{\pgfqpoint{2.713553in}{1.708199in}}{\pgfqpoint{2.717943in}{1.697600in}}{\pgfqpoint{2.725757in}{1.689786in}}%
\pgfpathcurveto{\pgfqpoint{2.733570in}{1.681973in}}{\pgfqpoint{2.744169in}{1.677582in}}{\pgfqpoint{2.755220in}{1.677582in}}%
\pgfpathclose%
\pgfusepath{stroke,fill}%
\end{pgfscope}%
\begin{pgfscope}%
\pgfpathrectangle{\pgfqpoint{0.481978in}{0.331635in}}{\pgfqpoint{4.960000in}{3.696000in}}%
\pgfusepath{clip}%
\pgfsetbuttcap%
\pgfsetroundjoin%
\definecolor{currentfill}{rgb}{0.631373,0.788235,0.956863}%
\pgfsetfillcolor{currentfill}%
\pgfsetlinewidth{0.481800pt}%
\definecolor{currentstroke}{rgb}{1.000000,1.000000,1.000000}%
\pgfsetstrokecolor{currentstroke}%
\pgfsetdash{}{0pt}%
\pgfpathmoveto{\pgfqpoint{2.232084in}{2.934814in}}%
\pgfpathcurveto{\pgfqpoint{2.243134in}{2.934814in}}{\pgfqpoint{2.253733in}{2.939204in}}{\pgfqpoint{2.261547in}{2.947018in}}%
\pgfpathcurveto{\pgfqpoint{2.269360in}{2.954832in}}{\pgfqpoint{2.273750in}{2.965431in}}{\pgfqpoint{2.273750in}{2.976481in}}%
\pgfpathcurveto{\pgfqpoint{2.273750in}{2.987531in}}{\pgfqpoint{2.269360in}{2.998130in}}{\pgfqpoint{2.261547in}{3.005943in}}%
\pgfpathcurveto{\pgfqpoint{2.253733in}{3.013757in}}{\pgfqpoint{2.243134in}{3.018147in}}{\pgfqpoint{2.232084in}{3.018147in}}%
\pgfpathcurveto{\pgfqpoint{2.221034in}{3.018147in}}{\pgfqpoint{2.210435in}{3.013757in}}{\pgfqpoint{2.202621in}{3.005943in}}%
\pgfpathcurveto{\pgfqpoint{2.194807in}{2.998130in}}{\pgfqpoint{2.190417in}{2.987531in}}{\pgfqpoint{2.190417in}{2.976481in}}%
\pgfpathcurveto{\pgfqpoint{2.190417in}{2.965431in}}{\pgfqpoint{2.194807in}{2.954832in}}{\pgfqpoint{2.202621in}{2.947018in}}%
\pgfpathcurveto{\pgfqpoint{2.210435in}{2.939204in}}{\pgfqpoint{2.221034in}{2.934814in}}{\pgfqpoint{2.232084in}{2.934814in}}%
\pgfpathclose%
\pgfusepath{stroke,fill}%
\end{pgfscope}%
\begin{pgfscope}%
\pgfpathrectangle{\pgfqpoint{0.481978in}{0.331635in}}{\pgfqpoint{4.960000in}{3.696000in}}%
\pgfusepath{clip}%
\pgfsetbuttcap%
\pgfsetroundjoin%
\definecolor{currentfill}{rgb}{0.631373,0.788235,0.956863}%
\pgfsetfillcolor{currentfill}%
\pgfsetlinewidth{0.481800pt}%
\definecolor{currentstroke}{rgb}{1.000000,1.000000,1.000000}%
\pgfsetstrokecolor{currentstroke}%
\pgfsetdash{}{0pt}%
\pgfpathmoveto{\pgfqpoint{2.086861in}{2.172947in}}%
\pgfpathcurveto{\pgfqpoint{2.097911in}{2.172947in}}{\pgfqpoint{2.108510in}{2.177338in}}{\pgfqpoint{2.116324in}{2.185151in}}%
\pgfpathcurveto{\pgfqpoint{2.124137in}{2.192965in}}{\pgfqpoint{2.128528in}{2.203564in}}{\pgfqpoint{2.128528in}{2.214614in}}%
\pgfpathcurveto{\pgfqpoint{2.128528in}{2.225664in}}{\pgfqpoint{2.124137in}{2.236263in}}{\pgfqpoint{2.116324in}{2.244077in}}%
\pgfpathcurveto{\pgfqpoint{2.108510in}{2.251890in}}{\pgfqpoint{2.097911in}{2.256281in}}{\pgfqpoint{2.086861in}{2.256281in}}%
\pgfpathcurveto{\pgfqpoint{2.075811in}{2.256281in}}{\pgfqpoint{2.065212in}{2.251890in}}{\pgfqpoint{2.057398in}{2.244077in}}%
\pgfpathcurveto{\pgfqpoint{2.049584in}{2.236263in}}{\pgfqpoint{2.045194in}{2.225664in}}{\pgfqpoint{2.045194in}{2.214614in}}%
\pgfpathcurveto{\pgfqpoint{2.045194in}{2.203564in}}{\pgfqpoint{2.049584in}{2.192965in}}{\pgfqpoint{2.057398in}{2.185151in}}%
\pgfpathcurveto{\pgfqpoint{2.065212in}{2.177338in}}{\pgfqpoint{2.075811in}{2.172947in}}{\pgfqpoint{2.086861in}{2.172947in}}%
\pgfpathclose%
\pgfusepath{stroke,fill}%
\end{pgfscope}%
\begin{pgfscope}%
\pgfpathrectangle{\pgfqpoint{0.481978in}{0.331635in}}{\pgfqpoint{4.960000in}{3.696000in}}%
\pgfusepath{clip}%
\pgfsetbuttcap%
\pgfsetroundjoin%
\definecolor{currentfill}{rgb}{0.631373,0.788235,0.956863}%
\pgfsetfillcolor{currentfill}%
\pgfsetlinewidth{0.481800pt}%
\definecolor{currentstroke}{rgb}{1.000000,1.000000,1.000000}%
\pgfsetstrokecolor{currentstroke}%
\pgfsetdash{}{0pt}%
\pgfpathmoveto{\pgfqpoint{1.970752in}{1.407367in}}%
\pgfpathcurveto{\pgfqpoint{1.981802in}{1.407367in}}{\pgfqpoint{1.992401in}{1.411757in}}{\pgfqpoint{2.000215in}{1.419571in}}%
\pgfpathcurveto{\pgfqpoint{2.008029in}{1.427384in}}{\pgfqpoint{2.012419in}{1.437983in}}{\pgfqpoint{2.012419in}{1.449033in}}%
\pgfpathcurveto{\pgfqpoint{2.012419in}{1.460084in}}{\pgfqpoint{2.008029in}{1.470683in}}{\pgfqpoint{2.000215in}{1.478496in}}%
\pgfpathcurveto{\pgfqpoint{1.992401in}{1.486310in}}{\pgfqpoint{1.981802in}{1.490700in}}{\pgfqpoint{1.970752in}{1.490700in}}%
\pgfpathcurveto{\pgfqpoint{1.959702in}{1.490700in}}{\pgfqpoint{1.949103in}{1.486310in}}{\pgfqpoint{1.941289in}{1.478496in}}%
\pgfpathcurveto{\pgfqpoint{1.933476in}{1.470683in}}{\pgfqpoint{1.929086in}{1.460084in}}{\pgfqpoint{1.929086in}{1.449033in}}%
\pgfpathcurveto{\pgfqpoint{1.929086in}{1.437983in}}{\pgfqpoint{1.933476in}{1.427384in}}{\pgfqpoint{1.941289in}{1.419571in}}%
\pgfpathcurveto{\pgfqpoint{1.949103in}{1.411757in}}{\pgfqpoint{1.959702in}{1.407367in}}{\pgfqpoint{1.970752in}{1.407367in}}%
\pgfpathclose%
\pgfusepath{stroke,fill}%
\end{pgfscope}%
\begin{pgfscope}%
\pgfpathrectangle{\pgfqpoint{0.481978in}{0.331635in}}{\pgfqpoint{4.960000in}{3.696000in}}%
\pgfusepath{clip}%
\pgfsetbuttcap%
\pgfsetroundjoin%
\definecolor{currentfill}{rgb}{0.631373,0.788235,0.956863}%
\pgfsetfillcolor{currentfill}%
\pgfsetlinewidth{0.481800pt}%
\definecolor{currentstroke}{rgb}{1.000000,1.000000,1.000000}%
\pgfsetstrokecolor{currentstroke}%
\pgfsetdash{}{0pt}%
\pgfpathmoveto{\pgfqpoint{3.222124in}{2.464639in}}%
\pgfpathcurveto{\pgfqpoint{3.233174in}{2.464639in}}{\pgfqpoint{3.243773in}{2.469029in}}{\pgfqpoint{3.251587in}{2.476842in}}%
\pgfpathcurveto{\pgfqpoint{3.259400in}{2.484656in}}{\pgfqpoint{3.263791in}{2.495255in}}{\pgfqpoint{3.263791in}{2.506305in}}%
\pgfpathcurveto{\pgfqpoint{3.263791in}{2.517355in}}{\pgfqpoint{3.259400in}{2.527954in}}{\pgfqpoint{3.251587in}{2.535768in}}%
\pgfpathcurveto{\pgfqpoint{3.243773in}{2.543582in}}{\pgfqpoint{3.233174in}{2.547972in}}{\pgfqpoint{3.222124in}{2.547972in}}%
\pgfpathcurveto{\pgfqpoint{3.211074in}{2.547972in}}{\pgfqpoint{3.200475in}{2.543582in}}{\pgfqpoint{3.192661in}{2.535768in}}%
\pgfpathcurveto{\pgfqpoint{3.184847in}{2.527954in}}{\pgfqpoint{3.180457in}{2.517355in}}{\pgfqpoint{3.180457in}{2.506305in}}%
\pgfpathcurveto{\pgfqpoint{3.180457in}{2.495255in}}{\pgfqpoint{3.184847in}{2.484656in}}{\pgfqpoint{3.192661in}{2.476842in}}%
\pgfpathcurveto{\pgfqpoint{3.200475in}{2.469029in}}{\pgfqpoint{3.211074in}{2.464639in}}{\pgfqpoint{3.222124in}{2.464639in}}%
\pgfpathclose%
\pgfusepath{stroke,fill}%
\end{pgfscope}%
\begin{pgfscope}%
\pgfpathrectangle{\pgfqpoint{0.481978in}{0.331635in}}{\pgfqpoint{4.960000in}{3.696000in}}%
\pgfusepath{clip}%
\pgfsetbuttcap%
\pgfsetroundjoin%
\definecolor{currentfill}{rgb}{0.631373,0.788235,0.956863}%
\pgfsetfillcolor{currentfill}%
\pgfsetlinewidth{0.481800pt}%
\definecolor{currentstroke}{rgb}{1.000000,1.000000,1.000000}%
\pgfsetstrokecolor{currentstroke}%
\pgfsetdash{}{0pt}%
\pgfpathmoveto{\pgfqpoint{4.691614in}{1.825388in}}%
\pgfpathcurveto{\pgfqpoint{4.702664in}{1.825388in}}{\pgfqpoint{4.713263in}{1.829778in}}{\pgfqpoint{4.721077in}{1.837592in}}%
\pgfpathcurveto{\pgfqpoint{4.728890in}{1.845405in}}{\pgfqpoint{4.733281in}{1.856004in}}{\pgfqpoint{4.733281in}{1.867055in}}%
\pgfpathcurveto{\pgfqpoint{4.733281in}{1.878105in}}{\pgfqpoint{4.728890in}{1.888704in}}{\pgfqpoint{4.721077in}{1.896517in}}%
\pgfpathcurveto{\pgfqpoint{4.713263in}{1.904331in}}{\pgfqpoint{4.702664in}{1.908721in}}{\pgfqpoint{4.691614in}{1.908721in}}%
\pgfpathcurveto{\pgfqpoint{4.680564in}{1.908721in}}{\pgfqpoint{4.669965in}{1.904331in}}{\pgfqpoint{4.662151in}{1.896517in}}%
\pgfpathcurveto{\pgfqpoint{4.654338in}{1.888704in}}{\pgfqpoint{4.649947in}{1.878105in}}{\pgfqpoint{4.649947in}{1.867055in}}%
\pgfpathcurveto{\pgfqpoint{4.649947in}{1.856004in}}{\pgfqpoint{4.654338in}{1.845405in}}{\pgfqpoint{4.662151in}{1.837592in}}%
\pgfpathcurveto{\pgfqpoint{4.669965in}{1.829778in}}{\pgfqpoint{4.680564in}{1.825388in}}{\pgfqpoint{4.691614in}{1.825388in}}%
\pgfpathclose%
\pgfusepath{stroke,fill}%
\end{pgfscope}%
\begin{pgfscope}%
\pgfpathrectangle{\pgfqpoint{0.481978in}{0.331635in}}{\pgfqpoint{4.960000in}{3.696000in}}%
\pgfusepath{clip}%
\pgfsetbuttcap%
\pgfsetroundjoin%
\definecolor{currentfill}{rgb}{0.631373,0.788235,0.956863}%
\pgfsetfillcolor{currentfill}%
\pgfsetlinewidth{0.481800pt}%
\definecolor{currentstroke}{rgb}{1.000000,1.000000,1.000000}%
\pgfsetstrokecolor{currentstroke}%
\pgfsetdash{}{0pt}%
\pgfpathmoveto{\pgfqpoint{4.017177in}{1.895669in}}%
\pgfpathcurveto{\pgfqpoint{4.028228in}{1.895669in}}{\pgfqpoint{4.038827in}{1.900059in}}{\pgfqpoint{4.046640in}{1.907873in}}%
\pgfpathcurveto{\pgfqpoint{4.054454in}{1.915686in}}{\pgfqpoint{4.058844in}{1.926285in}}{\pgfqpoint{4.058844in}{1.937335in}}%
\pgfpathcurveto{\pgfqpoint{4.058844in}{1.948386in}}{\pgfqpoint{4.054454in}{1.958985in}}{\pgfqpoint{4.046640in}{1.966798in}}%
\pgfpathcurveto{\pgfqpoint{4.038827in}{1.974612in}}{\pgfqpoint{4.028228in}{1.979002in}}{\pgfqpoint{4.017177in}{1.979002in}}%
\pgfpathcurveto{\pgfqpoint{4.006127in}{1.979002in}}{\pgfqpoint{3.995528in}{1.974612in}}{\pgfqpoint{3.987715in}{1.966798in}}%
\pgfpathcurveto{\pgfqpoint{3.979901in}{1.958985in}}{\pgfqpoint{3.975511in}{1.948386in}}{\pgfqpoint{3.975511in}{1.937335in}}%
\pgfpathcurveto{\pgfqpoint{3.975511in}{1.926285in}}{\pgfqpoint{3.979901in}{1.915686in}}{\pgfqpoint{3.987715in}{1.907873in}}%
\pgfpathcurveto{\pgfqpoint{3.995528in}{1.900059in}}{\pgfqpoint{4.006127in}{1.895669in}}{\pgfqpoint{4.017177in}{1.895669in}}%
\pgfpathclose%
\pgfusepath{stroke,fill}%
\end{pgfscope}%
\begin{pgfscope}%
\pgfpathrectangle{\pgfqpoint{0.481978in}{0.331635in}}{\pgfqpoint{4.960000in}{3.696000in}}%
\pgfusepath{clip}%
\pgfsetbuttcap%
\pgfsetroundjoin%
\definecolor{currentfill}{rgb}{0.631373,0.788235,0.956863}%
\pgfsetfillcolor{currentfill}%
\pgfsetlinewidth{0.481800pt}%
\definecolor{currentstroke}{rgb}{1.000000,1.000000,1.000000}%
\pgfsetstrokecolor{currentstroke}%
\pgfsetdash{}{0pt}%
\pgfpathmoveto{\pgfqpoint{2.369063in}{1.740082in}}%
\pgfpathcurveto{\pgfqpoint{2.380113in}{1.740082in}}{\pgfqpoint{2.390712in}{1.744472in}}{\pgfqpoint{2.398526in}{1.752286in}}%
\pgfpathcurveto{\pgfqpoint{2.406339in}{1.760099in}}{\pgfqpoint{2.410729in}{1.770698in}}{\pgfqpoint{2.410729in}{1.781748in}}%
\pgfpathcurveto{\pgfqpoint{2.410729in}{1.792799in}}{\pgfqpoint{2.406339in}{1.803398in}}{\pgfqpoint{2.398526in}{1.811211in}}%
\pgfpathcurveto{\pgfqpoint{2.390712in}{1.819025in}}{\pgfqpoint{2.380113in}{1.823415in}}{\pgfqpoint{2.369063in}{1.823415in}}%
\pgfpathcurveto{\pgfqpoint{2.358013in}{1.823415in}}{\pgfqpoint{2.347414in}{1.819025in}}{\pgfqpoint{2.339600in}{1.811211in}}%
\pgfpathcurveto{\pgfqpoint{2.331786in}{1.803398in}}{\pgfqpoint{2.327396in}{1.792799in}}{\pgfqpoint{2.327396in}{1.781748in}}%
\pgfpathcurveto{\pgfqpoint{2.327396in}{1.770698in}}{\pgfqpoint{2.331786in}{1.760099in}}{\pgfqpoint{2.339600in}{1.752286in}}%
\pgfpathcurveto{\pgfqpoint{2.347414in}{1.744472in}}{\pgfqpoint{2.358013in}{1.740082in}}{\pgfqpoint{2.369063in}{1.740082in}}%
\pgfpathclose%
\pgfusepath{stroke,fill}%
\end{pgfscope}%
\begin{pgfscope}%
\pgfpathrectangle{\pgfqpoint{0.481978in}{0.331635in}}{\pgfqpoint{4.960000in}{3.696000in}}%
\pgfusepath{clip}%
\pgfsetbuttcap%
\pgfsetroundjoin%
\definecolor{currentfill}{rgb}{0.631373,0.788235,0.956863}%
\pgfsetfillcolor{currentfill}%
\pgfsetlinewidth{0.481800pt}%
\definecolor{currentstroke}{rgb}{1.000000,1.000000,1.000000}%
\pgfsetstrokecolor{currentstroke}%
\pgfsetdash{}{0pt}%
\pgfpathmoveto{\pgfqpoint{2.075627in}{3.234178in}}%
\pgfpathcurveto{\pgfqpoint{2.086678in}{3.234178in}}{\pgfqpoint{2.097277in}{3.238568in}}{\pgfqpoint{2.105090in}{3.246382in}}%
\pgfpathcurveto{\pgfqpoint{2.112904in}{3.254196in}}{\pgfqpoint{2.117294in}{3.264795in}}{\pgfqpoint{2.117294in}{3.275845in}}%
\pgfpathcurveto{\pgfqpoint{2.117294in}{3.286895in}}{\pgfqpoint{2.112904in}{3.297494in}}{\pgfqpoint{2.105090in}{3.305307in}}%
\pgfpathcurveto{\pgfqpoint{2.097277in}{3.313121in}}{\pgfqpoint{2.086678in}{3.317511in}}{\pgfqpoint{2.075627in}{3.317511in}}%
\pgfpathcurveto{\pgfqpoint{2.064577in}{3.317511in}}{\pgfqpoint{2.053978in}{3.313121in}}{\pgfqpoint{2.046165in}{3.305307in}}%
\pgfpathcurveto{\pgfqpoint{2.038351in}{3.297494in}}{\pgfqpoint{2.033961in}{3.286895in}}{\pgfqpoint{2.033961in}{3.275845in}}%
\pgfpathcurveto{\pgfqpoint{2.033961in}{3.264795in}}{\pgfqpoint{2.038351in}{3.254196in}}{\pgfqpoint{2.046165in}{3.246382in}}%
\pgfpathcurveto{\pgfqpoint{2.053978in}{3.238568in}}{\pgfqpoint{2.064577in}{3.234178in}}{\pgfqpoint{2.075627in}{3.234178in}}%
\pgfpathclose%
\pgfusepath{stroke,fill}%
\end{pgfscope}%
\begin{pgfscope}%
\pgfpathrectangle{\pgfqpoint{0.481978in}{0.331635in}}{\pgfqpoint{4.960000in}{3.696000in}}%
\pgfusepath{clip}%
\pgfsetbuttcap%
\pgfsetroundjoin%
\definecolor{currentfill}{rgb}{0.631373,0.788235,0.956863}%
\pgfsetfillcolor{currentfill}%
\pgfsetlinewidth{0.481800pt}%
\definecolor{currentstroke}{rgb}{1.000000,1.000000,1.000000}%
\pgfsetstrokecolor{currentstroke}%
\pgfsetdash{}{0pt}%
\pgfpathmoveto{\pgfqpoint{2.323340in}{1.276686in}}%
\pgfpathcurveto{\pgfqpoint{2.334390in}{1.276686in}}{\pgfqpoint{2.344990in}{1.281077in}}{\pgfqpoint{2.352803in}{1.288890in}}%
\pgfpathcurveto{\pgfqpoint{2.360617in}{1.296704in}}{\pgfqpoint{2.365007in}{1.307303in}}{\pgfqpoint{2.365007in}{1.318353in}}%
\pgfpathcurveto{\pgfqpoint{2.365007in}{1.329403in}}{\pgfqpoint{2.360617in}{1.340002in}}{\pgfqpoint{2.352803in}{1.347816in}}%
\pgfpathcurveto{\pgfqpoint{2.344990in}{1.355629in}}{\pgfqpoint{2.334390in}{1.360020in}}{\pgfqpoint{2.323340in}{1.360020in}}%
\pgfpathcurveto{\pgfqpoint{2.312290in}{1.360020in}}{\pgfqpoint{2.301691in}{1.355629in}}{\pgfqpoint{2.293878in}{1.347816in}}%
\pgfpathcurveto{\pgfqpoint{2.286064in}{1.340002in}}{\pgfqpoint{2.281674in}{1.329403in}}{\pgfqpoint{2.281674in}{1.318353in}}%
\pgfpathcurveto{\pgfqpoint{2.281674in}{1.307303in}}{\pgfqpoint{2.286064in}{1.296704in}}{\pgfqpoint{2.293878in}{1.288890in}}%
\pgfpathcurveto{\pgfqpoint{2.301691in}{1.281077in}}{\pgfqpoint{2.312290in}{1.276686in}}{\pgfqpoint{2.323340in}{1.276686in}}%
\pgfpathclose%
\pgfusepath{stroke,fill}%
\end{pgfscope}%
\begin{pgfscope}%
\pgfpathrectangle{\pgfqpoint{0.481978in}{0.331635in}}{\pgfqpoint{4.960000in}{3.696000in}}%
\pgfusepath{clip}%
\pgfsetbuttcap%
\pgfsetroundjoin%
\definecolor{currentfill}{rgb}{0.631373,0.788235,0.956863}%
\pgfsetfillcolor{currentfill}%
\pgfsetlinewidth{0.481800pt}%
\definecolor{currentstroke}{rgb}{1.000000,1.000000,1.000000}%
\pgfsetstrokecolor{currentstroke}%
\pgfsetdash{}{0pt}%
\pgfpathmoveto{\pgfqpoint{1.355993in}{1.051233in}}%
\pgfpathcurveto{\pgfqpoint{1.367044in}{1.051233in}}{\pgfqpoint{1.377643in}{1.055623in}}{\pgfqpoint{1.385456in}{1.063436in}}%
\pgfpathcurveto{\pgfqpoint{1.393270in}{1.071250in}}{\pgfqpoint{1.397660in}{1.081849in}}{\pgfqpoint{1.397660in}{1.092899in}}%
\pgfpathcurveto{\pgfqpoint{1.397660in}{1.103949in}}{\pgfqpoint{1.393270in}{1.114548in}}{\pgfqpoint{1.385456in}{1.122362in}}%
\pgfpathcurveto{\pgfqpoint{1.377643in}{1.130176in}}{\pgfqpoint{1.367044in}{1.134566in}}{\pgfqpoint{1.355993in}{1.134566in}}%
\pgfpathcurveto{\pgfqpoint{1.344943in}{1.134566in}}{\pgfqpoint{1.334344in}{1.130176in}}{\pgfqpoint{1.326531in}{1.122362in}}%
\pgfpathcurveto{\pgfqpoint{1.318717in}{1.114548in}}{\pgfqpoint{1.314327in}{1.103949in}}{\pgfqpoint{1.314327in}{1.092899in}}%
\pgfpathcurveto{\pgfqpoint{1.314327in}{1.081849in}}{\pgfqpoint{1.318717in}{1.071250in}}{\pgfqpoint{1.326531in}{1.063436in}}%
\pgfpathcurveto{\pgfqpoint{1.334344in}{1.055623in}}{\pgfqpoint{1.344943in}{1.051233in}}{\pgfqpoint{1.355993in}{1.051233in}}%
\pgfpathclose%
\pgfusepath{stroke,fill}%
\end{pgfscope}%
\begin{pgfscope}%
\pgfpathrectangle{\pgfqpoint{0.481978in}{0.331635in}}{\pgfqpoint{4.960000in}{3.696000in}}%
\pgfusepath{clip}%
\pgfsetbuttcap%
\pgfsetroundjoin%
\definecolor{currentfill}{rgb}{0.631373,0.788235,0.956863}%
\pgfsetfillcolor{currentfill}%
\pgfsetlinewidth{0.481800pt}%
\definecolor{currentstroke}{rgb}{1.000000,1.000000,1.000000}%
\pgfsetstrokecolor{currentstroke}%
\pgfsetdash{}{0pt}%
\pgfpathmoveto{\pgfqpoint{4.108199in}{1.796453in}}%
\pgfpathcurveto{\pgfqpoint{4.119250in}{1.796453in}}{\pgfqpoint{4.129849in}{1.800844in}}{\pgfqpoint{4.137662in}{1.808657in}}%
\pgfpathcurveto{\pgfqpoint{4.145476in}{1.816471in}}{\pgfqpoint{4.149866in}{1.827070in}}{\pgfqpoint{4.149866in}{1.838120in}}%
\pgfpathcurveto{\pgfqpoint{4.149866in}{1.849170in}}{\pgfqpoint{4.145476in}{1.859769in}}{\pgfqpoint{4.137662in}{1.867583in}}%
\pgfpathcurveto{\pgfqpoint{4.129849in}{1.875396in}}{\pgfqpoint{4.119250in}{1.879787in}}{\pgfqpoint{4.108199in}{1.879787in}}%
\pgfpathcurveto{\pgfqpoint{4.097149in}{1.879787in}}{\pgfqpoint{4.086550in}{1.875396in}}{\pgfqpoint{4.078737in}{1.867583in}}%
\pgfpathcurveto{\pgfqpoint{4.070923in}{1.859769in}}{\pgfqpoint{4.066533in}{1.849170in}}{\pgfqpoint{4.066533in}{1.838120in}}%
\pgfpathcurveto{\pgfqpoint{4.066533in}{1.827070in}}{\pgfqpoint{4.070923in}{1.816471in}}{\pgfqpoint{4.078737in}{1.808657in}}%
\pgfpathcurveto{\pgfqpoint{4.086550in}{1.800844in}}{\pgfqpoint{4.097149in}{1.796453in}}{\pgfqpoint{4.108199in}{1.796453in}}%
\pgfpathclose%
\pgfusepath{stroke,fill}%
\end{pgfscope}%
\begin{pgfscope}%
\pgfpathrectangle{\pgfqpoint{0.481978in}{0.331635in}}{\pgfqpoint{4.960000in}{3.696000in}}%
\pgfusepath{clip}%
\pgfsetbuttcap%
\pgfsetroundjoin%
\definecolor{currentfill}{rgb}{0.631373,0.788235,0.956863}%
\pgfsetfillcolor{currentfill}%
\pgfsetlinewidth{0.481800pt}%
\definecolor{currentstroke}{rgb}{1.000000,1.000000,1.000000}%
\pgfsetstrokecolor{currentstroke}%
\pgfsetdash{}{0pt}%
\pgfpathmoveto{\pgfqpoint{2.295691in}{2.120819in}}%
\pgfpathcurveto{\pgfqpoint{2.306741in}{2.120819in}}{\pgfqpoint{2.317340in}{2.125209in}}{\pgfqpoint{2.325154in}{2.133022in}}%
\pgfpathcurveto{\pgfqpoint{2.332967in}{2.140836in}}{\pgfqpoint{2.337358in}{2.151435in}}{\pgfqpoint{2.337358in}{2.162485in}}%
\pgfpathcurveto{\pgfqpoint{2.337358in}{2.173535in}}{\pgfqpoint{2.332967in}{2.184134in}}{\pgfqpoint{2.325154in}{2.191948in}}%
\pgfpathcurveto{\pgfqpoint{2.317340in}{2.199762in}}{\pgfqpoint{2.306741in}{2.204152in}}{\pgfqpoint{2.295691in}{2.204152in}}%
\pgfpathcurveto{\pgfqpoint{2.284641in}{2.204152in}}{\pgfqpoint{2.274042in}{2.199762in}}{\pgfqpoint{2.266228in}{2.191948in}}%
\pgfpathcurveto{\pgfqpoint{2.258415in}{2.184134in}}{\pgfqpoint{2.254024in}{2.173535in}}{\pgfqpoint{2.254024in}{2.162485in}}%
\pgfpathcurveto{\pgfqpoint{2.254024in}{2.151435in}}{\pgfqpoint{2.258415in}{2.140836in}}{\pgfqpoint{2.266228in}{2.133022in}}%
\pgfpathcurveto{\pgfqpoint{2.274042in}{2.125209in}}{\pgfqpoint{2.284641in}{2.120819in}}{\pgfqpoint{2.295691in}{2.120819in}}%
\pgfpathclose%
\pgfusepath{stroke,fill}%
\end{pgfscope}%
\begin{pgfscope}%
\pgfpathrectangle{\pgfqpoint{0.481978in}{0.331635in}}{\pgfqpoint{4.960000in}{3.696000in}}%
\pgfusepath{clip}%
\pgfsetbuttcap%
\pgfsetroundjoin%
\definecolor{currentfill}{rgb}{0.631373,0.788235,0.956863}%
\pgfsetfillcolor{currentfill}%
\pgfsetlinewidth{0.481800pt}%
\definecolor{currentstroke}{rgb}{1.000000,1.000000,1.000000}%
\pgfsetstrokecolor{currentstroke}%
\pgfsetdash{}{0pt}%
\pgfpathmoveto{\pgfqpoint{1.851660in}{2.971810in}}%
\pgfpathcurveto{\pgfqpoint{1.862711in}{2.971810in}}{\pgfqpoint{1.873310in}{2.976200in}}{\pgfqpoint{1.881123in}{2.984014in}}%
\pgfpathcurveto{\pgfqpoint{1.888937in}{2.991828in}}{\pgfqpoint{1.893327in}{3.002427in}}{\pgfqpoint{1.893327in}{3.013477in}}%
\pgfpathcurveto{\pgfqpoint{1.893327in}{3.024527in}}{\pgfqpoint{1.888937in}{3.035126in}}{\pgfqpoint{1.881123in}{3.042940in}}%
\pgfpathcurveto{\pgfqpoint{1.873310in}{3.050753in}}{\pgfqpoint{1.862711in}{3.055144in}}{\pgfqpoint{1.851660in}{3.055144in}}%
\pgfpathcurveto{\pgfqpoint{1.840610in}{3.055144in}}{\pgfqpoint{1.830011in}{3.050753in}}{\pgfqpoint{1.822198in}{3.042940in}}%
\pgfpathcurveto{\pgfqpoint{1.814384in}{3.035126in}}{\pgfqpoint{1.809994in}{3.024527in}}{\pgfqpoint{1.809994in}{3.013477in}}%
\pgfpathcurveto{\pgfqpoint{1.809994in}{3.002427in}}{\pgfqpoint{1.814384in}{2.991828in}}{\pgfqpoint{1.822198in}{2.984014in}}%
\pgfpathcurveto{\pgfqpoint{1.830011in}{2.976200in}}{\pgfqpoint{1.840610in}{2.971810in}}{\pgfqpoint{1.851660in}{2.971810in}}%
\pgfpathclose%
\pgfusepath{stroke,fill}%
\end{pgfscope}%
\begin{pgfscope}%
\pgfpathrectangle{\pgfqpoint{0.481978in}{0.331635in}}{\pgfqpoint{4.960000in}{3.696000in}}%
\pgfusepath{clip}%
\pgfsetbuttcap%
\pgfsetroundjoin%
\definecolor{currentfill}{rgb}{0.631373,0.788235,0.956863}%
\pgfsetfillcolor{currentfill}%
\pgfsetlinewidth{0.481800pt}%
\definecolor{currentstroke}{rgb}{1.000000,1.000000,1.000000}%
\pgfsetstrokecolor{currentstroke}%
\pgfsetdash{}{0pt}%
\pgfpathmoveto{\pgfqpoint{3.459839in}{2.113590in}}%
\pgfpathcurveto{\pgfqpoint{3.470889in}{2.113590in}}{\pgfqpoint{3.481488in}{2.117980in}}{\pgfqpoint{3.489301in}{2.125794in}}%
\pgfpathcurveto{\pgfqpoint{3.497115in}{2.133607in}}{\pgfqpoint{3.501505in}{2.144206in}}{\pgfqpoint{3.501505in}{2.155256in}}%
\pgfpathcurveto{\pgfqpoint{3.501505in}{2.166307in}}{\pgfqpoint{3.497115in}{2.176906in}}{\pgfqpoint{3.489301in}{2.184719in}}%
\pgfpathcurveto{\pgfqpoint{3.481488in}{2.192533in}}{\pgfqpoint{3.470889in}{2.196923in}}{\pgfqpoint{3.459839in}{2.196923in}}%
\pgfpathcurveto{\pgfqpoint{3.448788in}{2.196923in}}{\pgfqpoint{3.438189in}{2.192533in}}{\pgfqpoint{3.430376in}{2.184719in}}%
\pgfpathcurveto{\pgfqpoint{3.422562in}{2.176906in}}{\pgfqpoint{3.418172in}{2.166307in}}{\pgfqpoint{3.418172in}{2.155256in}}%
\pgfpathcurveto{\pgfqpoint{3.418172in}{2.144206in}}{\pgfqpoint{3.422562in}{2.133607in}}{\pgfqpoint{3.430376in}{2.125794in}}%
\pgfpathcurveto{\pgfqpoint{3.438189in}{2.117980in}}{\pgfqpoint{3.448788in}{2.113590in}}{\pgfqpoint{3.459839in}{2.113590in}}%
\pgfpathclose%
\pgfusepath{stroke,fill}%
\end{pgfscope}%
\begin{pgfscope}%
\pgfpathrectangle{\pgfqpoint{0.481978in}{0.331635in}}{\pgfqpoint{4.960000in}{3.696000in}}%
\pgfusepath{clip}%
\pgfsetbuttcap%
\pgfsetroundjoin%
\definecolor{currentfill}{rgb}{0.631373,0.788235,0.956863}%
\pgfsetfillcolor{currentfill}%
\pgfsetlinewidth{0.481800pt}%
\definecolor{currentstroke}{rgb}{1.000000,1.000000,1.000000}%
\pgfsetstrokecolor{currentstroke}%
\pgfsetdash{}{0pt}%
\pgfpathmoveto{\pgfqpoint{4.203759in}{1.996357in}}%
\pgfpathcurveto{\pgfqpoint{4.214810in}{1.996357in}}{\pgfqpoint{4.225409in}{2.000747in}}{\pgfqpoint{4.233222in}{2.008561in}}%
\pgfpathcurveto{\pgfqpoint{4.241036in}{2.016375in}}{\pgfqpoint{4.245426in}{2.026974in}}{\pgfqpoint{4.245426in}{2.038024in}}%
\pgfpathcurveto{\pgfqpoint{4.245426in}{2.049074in}}{\pgfqpoint{4.241036in}{2.059673in}}{\pgfqpoint{4.233222in}{2.067487in}}%
\pgfpathcurveto{\pgfqpoint{4.225409in}{2.075300in}}{\pgfqpoint{4.214810in}{2.079690in}}{\pgfqpoint{4.203759in}{2.079690in}}%
\pgfpathcurveto{\pgfqpoint{4.192709in}{2.079690in}}{\pgfqpoint{4.182110in}{2.075300in}}{\pgfqpoint{4.174297in}{2.067487in}}%
\pgfpathcurveto{\pgfqpoint{4.166483in}{2.059673in}}{\pgfqpoint{4.162093in}{2.049074in}}{\pgfqpoint{4.162093in}{2.038024in}}%
\pgfpathcurveto{\pgfqpoint{4.162093in}{2.026974in}}{\pgfqpoint{4.166483in}{2.016375in}}{\pgfqpoint{4.174297in}{2.008561in}}%
\pgfpathcurveto{\pgfqpoint{4.182110in}{2.000747in}}{\pgfqpoint{4.192709in}{1.996357in}}{\pgfqpoint{4.203759in}{1.996357in}}%
\pgfpathclose%
\pgfusepath{stroke,fill}%
\end{pgfscope}%
\begin{pgfscope}%
\pgfpathrectangle{\pgfqpoint{0.481978in}{0.331635in}}{\pgfqpoint{4.960000in}{3.696000in}}%
\pgfusepath{clip}%
\pgfsetbuttcap%
\pgfsetroundjoin%
\definecolor{currentfill}{rgb}{0.631373,0.788235,0.956863}%
\pgfsetfillcolor{currentfill}%
\pgfsetlinewidth{0.481800pt}%
\definecolor{currentstroke}{rgb}{1.000000,1.000000,1.000000}%
\pgfsetstrokecolor{currentstroke}%
\pgfsetdash{}{0pt}%
\pgfpathmoveto{\pgfqpoint{3.504513in}{2.995039in}}%
\pgfpathcurveto{\pgfqpoint{3.515563in}{2.995039in}}{\pgfqpoint{3.526162in}{2.999429in}}{\pgfqpoint{3.533976in}{3.007243in}}%
\pgfpathcurveto{\pgfqpoint{3.541789in}{3.015057in}}{\pgfqpoint{3.546180in}{3.025656in}}{\pgfqpoint{3.546180in}{3.036706in}}%
\pgfpathcurveto{\pgfqpoint{3.546180in}{3.047756in}}{\pgfqpoint{3.541789in}{3.058355in}}{\pgfqpoint{3.533976in}{3.066169in}}%
\pgfpathcurveto{\pgfqpoint{3.526162in}{3.073982in}}{\pgfqpoint{3.515563in}{3.078372in}}{\pgfqpoint{3.504513in}{3.078372in}}%
\pgfpathcurveto{\pgfqpoint{3.493463in}{3.078372in}}{\pgfqpoint{3.482864in}{3.073982in}}{\pgfqpoint{3.475050in}{3.066169in}}%
\pgfpathcurveto{\pgfqpoint{3.467237in}{3.058355in}}{\pgfqpoint{3.462846in}{3.047756in}}{\pgfqpoint{3.462846in}{3.036706in}}%
\pgfpathcurveto{\pgfqpoint{3.462846in}{3.025656in}}{\pgfqpoint{3.467237in}{3.015057in}}{\pgfqpoint{3.475050in}{3.007243in}}%
\pgfpathcurveto{\pgfqpoint{3.482864in}{2.999429in}}{\pgfqpoint{3.493463in}{2.995039in}}{\pgfqpoint{3.504513in}{2.995039in}}%
\pgfpathclose%
\pgfusepath{stroke,fill}%
\end{pgfscope}%
\begin{pgfscope}%
\pgfpathrectangle{\pgfqpoint{0.481978in}{0.331635in}}{\pgfqpoint{4.960000in}{3.696000in}}%
\pgfusepath{clip}%
\pgfsetbuttcap%
\pgfsetroundjoin%
\definecolor{currentfill}{rgb}{0.631373,0.788235,0.956863}%
\pgfsetfillcolor{currentfill}%
\pgfsetlinewidth{0.481800pt}%
\definecolor{currentstroke}{rgb}{1.000000,1.000000,1.000000}%
\pgfsetstrokecolor{currentstroke}%
\pgfsetdash{}{0pt}%
\pgfpathmoveto{\pgfqpoint{3.435586in}{2.716506in}}%
\pgfpathcurveto{\pgfqpoint{3.446636in}{2.716506in}}{\pgfqpoint{3.457235in}{2.720896in}}{\pgfqpoint{3.465049in}{2.728709in}}%
\pgfpathcurveto{\pgfqpoint{3.472863in}{2.736523in}}{\pgfqpoint{3.477253in}{2.747122in}}{\pgfqpoint{3.477253in}{2.758172in}}%
\pgfpathcurveto{\pgfqpoint{3.477253in}{2.769222in}}{\pgfqpoint{3.472863in}{2.779821in}}{\pgfqpoint{3.465049in}{2.787635in}}%
\pgfpathcurveto{\pgfqpoint{3.457235in}{2.795449in}}{\pgfqpoint{3.446636in}{2.799839in}}{\pgfqpoint{3.435586in}{2.799839in}}%
\pgfpathcurveto{\pgfqpoint{3.424536in}{2.799839in}}{\pgfqpoint{3.413937in}{2.795449in}}{\pgfqpoint{3.406123in}{2.787635in}}%
\pgfpathcurveto{\pgfqpoint{3.398310in}{2.779821in}}{\pgfqpoint{3.393919in}{2.769222in}}{\pgfqpoint{3.393919in}{2.758172in}}%
\pgfpathcurveto{\pgfqpoint{3.393919in}{2.747122in}}{\pgfqpoint{3.398310in}{2.736523in}}{\pgfqpoint{3.406123in}{2.728709in}}%
\pgfpathcurveto{\pgfqpoint{3.413937in}{2.720896in}}{\pgfqpoint{3.424536in}{2.716506in}}{\pgfqpoint{3.435586in}{2.716506in}}%
\pgfpathclose%
\pgfusepath{stroke,fill}%
\end{pgfscope}%
\begin{pgfscope}%
\pgfpathrectangle{\pgfqpoint{0.481978in}{0.331635in}}{\pgfqpoint{4.960000in}{3.696000in}}%
\pgfusepath{clip}%
\pgfsetbuttcap%
\pgfsetroundjoin%
\definecolor{currentfill}{rgb}{0.631373,0.788235,0.956863}%
\pgfsetfillcolor{currentfill}%
\pgfsetlinewidth{0.481800pt}%
\definecolor{currentstroke}{rgb}{1.000000,1.000000,1.000000}%
\pgfsetstrokecolor{currentstroke}%
\pgfsetdash{}{0pt}%
\pgfpathmoveto{\pgfqpoint{2.298856in}{1.452187in}}%
\pgfpathcurveto{\pgfqpoint{2.309906in}{1.452187in}}{\pgfqpoint{2.320505in}{1.456578in}}{\pgfqpoint{2.328319in}{1.464391in}}%
\pgfpathcurveto{\pgfqpoint{2.336132in}{1.472205in}}{\pgfqpoint{2.340522in}{1.482804in}}{\pgfqpoint{2.340522in}{1.493854in}}%
\pgfpathcurveto{\pgfqpoint{2.340522in}{1.504904in}}{\pgfqpoint{2.336132in}{1.515503in}}{\pgfqpoint{2.328319in}{1.523317in}}%
\pgfpathcurveto{\pgfqpoint{2.320505in}{1.531130in}}{\pgfqpoint{2.309906in}{1.535521in}}{\pgfqpoint{2.298856in}{1.535521in}}%
\pgfpathcurveto{\pgfqpoint{2.287806in}{1.535521in}}{\pgfqpoint{2.277207in}{1.531130in}}{\pgfqpoint{2.269393in}{1.523317in}}%
\pgfpathcurveto{\pgfqpoint{2.261579in}{1.515503in}}{\pgfqpoint{2.257189in}{1.504904in}}{\pgfqpoint{2.257189in}{1.493854in}}%
\pgfpathcurveto{\pgfqpoint{2.257189in}{1.482804in}}{\pgfqpoint{2.261579in}{1.472205in}}{\pgfqpoint{2.269393in}{1.464391in}}%
\pgfpathcurveto{\pgfqpoint{2.277207in}{1.456578in}}{\pgfqpoint{2.287806in}{1.452187in}}{\pgfqpoint{2.298856in}{1.452187in}}%
\pgfpathclose%
\pgfusepath{stroke,fill}%
\end{pgfscope}%
\begin{pgfscope}%
\pgfpathrectangle{\pgfqpoint{0.481978in}{0.331635in}}{\pgfqpoint{4.960000in}{3.696000in}}%
\pgfusepath{clip}%
\pgfsetbuttcap%
\pgfsetroundjoin%
\definecolor{currentfill}{rgb}{0.631373,0.788235,0.956863}%
\pgfsetfillcolor{currentfill}%
\pgfsetlinewidth{0.481800pt}%
\definecolor{currentstroke}{rgb}{1.000000,1.000000,1.000000}%
\pgfsetstrokecolor{currentstroke}%
\pgfsetdash{}{0pt}%
\pgfpathmoveto{\pgfqpoint{1.716898in}{2.993268in}}%
\pgfpathcurveto{\pgfqpoint{1.727948in}{2.993268in}}{\pgfqpoint{1.738547in}{2.997658in}}{\pgfqpoint{1.746361in}{3.005472in}}%
\pgfpathcurveto{\pgfqpoint{1.754174in}{3.013286in}}{\pgfqpoint{1.758565in}{3.023885in}}{\pgfqpoint{1.758565in}{3.034935in}}%
\pgfpathcurveto{\pgfqpoint{1.758565in}{3.045985in}}{\pgfqpoint{1.754174in}{3.056584in}}{\pgfqpoint{1.746361in}{3.064398in}}%
\pgfpathcurveto{\pgfqpoint{1.738547in}{3.072211in}}{\pgfqpoint{1.727948in}{3.076601in}}{\pgfqpoint{1.716898in}{3.076601in}}%
\pgfpathcurveto{\pgfqpoint{1.705848in}{3.076601in}}{\pgfqpoint{1.695249in}{3.072211in}}{\pgfqpoint{1.687435in}{3.064398in}}%
\pgfpathcurveto{\pgfqpoint{1.679621in}{3.056584in}}{\pgfqpoint{1.675231in}{3.045985in}}{\pgfqpoint{1.675231in}{3.034935in}}%
\pgfpathcurveto{\pgfqpoint{1.675231in}{3.023885in}}{\pgfqpoint{1.679621in}{3.013286in}}{\pgfqpoint{1.687435in}{3.005472in}}%
\pgfpathcurveto{\pgfqpoint{1.695249in}{2.997658in}}{\pgfqpoint{1.705848in}{2.993268in}}{\pgfqpoint{1.716898in}{2.993268in}}%
\pgfpathclose%
\pgfusepath{stroke,fill}%
\end{pgfscope}%
\begin{pgfscope}%
\pgfpathrectangle{\pgfqpoint{0.481978in}{0.331635in}}{\pgfqpoint{4.960000in}{3.696000in}}%
\pgfusepath{clip}%
\pgfsetbuttcap%
\pgfsetroundjoin%
\definecolor{currentfill}{rgb}{0.631373,0.788235,0.956863}%
\pgfsetfillcolor{currentfill}%
\pgfsetlinewidth{0.481800pt}%
\definecolor{currentstroke}{rgb}{1.000000,1.000000,1.000000}%
\pgfsetstrokecolor{currentstroke}%
\pgfsetdash{}{0pt}%
\pgfpathmoveto{\pgfqpoint{2.944040in}{2.403393in}}%
\pgfpathcurveto{\pgfqpoint{2.955091in}{2.403393in}}{\pgfqpoint{2.965690in}{2.407783in}}{\pgfqpoint{2.973503in}{2.415597in}}%
\pgfpathcurveto{\pgfqpoint{2.981317in}{2.423411in}}{\pgfqpoint{2.985707in}{2.434010in}}{\pgfqpoint{2.985707in}{2.445060in}}%
\pgfpathcurveto{\pgfqpoint{2.985707in}{2.456110in}}{\pgfqpoint{2.981317in}{2.466709in}}{\pgfqpoint{2.973503in}{2.474522in}}%
\pgfpathcurveto{\pgfqpoint{2.965690in}{2.482336in}}{\pgfqpoint{2.955091in}{2.486726in}}{\pgfqpoint{2.944040in}{2.486726in}}%
\pgfpathcurveto{\pgfqpoint{2.932990in}{2.486726in}}{\pgfqpoint{2.922391in}{2.482336in}}{\pgfqpoint{2.914578in}{2.474522in}}%
\pgfpathcurveto{\pgfqpoint{2.906764in}{2.466709in}}{\pgfqpoint{2.902374in}{2.456110in}}{\pgfqpoint{2.902374in}{2.445060in}}%
\pgfpathcurveto{\pgfqpoint{2.902374in}{2.434010in}}{\pgfqpoint{2.906764in}{2.423411in}}{\pgfqpoint{2.914578in}{2.415597in}}%
\pgfpathcurveto{\pgfqpoint{2.922391in}{2.407783in}}{\pgfqpoint{2.932990in}{2.403393in}}{\pgfqpoint{2.944040in}{2.403393in}}%
\pgfpathclose%
\pgfusepath{stroke,fill}%
\end{pgfscope}%
\begin{pgfscope}%
\pgfpathrectangle{\pgfqpoint{0.481978in}{0.331635in}}{\pgfqpoint{4.960000in}{3.696000in}}%
\pgfusepath{clip}%
\pgfsetbuttcap%
\pgfsetroundjoin%
\definecolor{currentfill}{rgb}{0.631373,0.788235,0.956863}%
\pgfsetfillcolor{currentfill}%
\pgfsetlinewidth{0.481800pt}%
\definecolor{currentstroke}{rgb}{1.000000,1.000000,1.000000}%
\pgfsetstrokecolor{currentstroke}%
\pgfsetdash{}{0pt}%
\pgfpathmoveto{\pgfqpoint{2.677796in}{2.469856in}}%
\pgfpathcurveto{\pgfqpoint{2.688846in}{2.469856in}}{\pgfqpoint{2.699445in}{2.474247in}}{\pgfqpoint{2.707259in}{2.482060in}}%
\pgfpathcurveto{\pgfqpoint{2.715073in}{2.489874in}}{\pgfqpoint{2.719463in}{2.500473in}}{\pgfqpoint{2.719463in}{2.511523in}}%
\pgfpathcurveto{\pgfqpoint{2.719463in}{2.522573in}}{\pgfqpoint{2.715073in}{2.533172in}}{\pgfqpoint{2.707259in}{2.540986in}}%
\pgfpathcurveto{\pgfqpoint{2.699445in}{2.548799in}}{\pgfqpoint{2.688846in}{2.553190in}}{\pgfqpoint{2.677796in}{2.553190in}}%
\pgfpathcurveto{\pgfqpoint{2.666746in}{2.553190in}}{\pgfqpoint{2.656147in}{2.548799in}}{\pgfqpoint{2.648333in}{2.540986in}}%
\pgfpathcurveto{\pgfqpoint{2.640520in}{2.533172in}}{\pgfqpoint{2.636130in}{2.522573in}}{\pgfqpoint{2.636130in}{2.511523in}}%
\pgfpathcurveto{\pgfqpoint{2.636130in}{2.500473in}}{\pgfqpoint{2.640520in}{2.489874in}}{\pgfqpoint{2.648333in}{2.482060in}}%
\pgfpathcurveto{\pgfqpoint{2.656147in}{2.474247in}}{\pgfqpoint{2.666746in}{2.469856in}}{\pgfqpoint{2.677796in}{2.469856in}}%
\pgfpathclose%
\pgfusepath{stroke,fill}%
\end{pgfscope}%
\begin{pgfscope}%
\pgfpathrectangle{\pgfqpoint{0.481978in}{0.331635in}}{\pgfqpoint{4.960000in}{3.696000in}}%
\pgfusepath{clip}%
\pgfsetbuttcap%
\pgfsetroundjoin%
\definecolor{currentfill}{rgb}{0.631373,0.788235,0.956863}%
\pgfsetfillcolor{currentfill}%
\pgfsetlinewidth{0.481800pt}%
\definecolor{currentstroke}{rgb}{1.000000,1.000000,1.000000}%
\pgfsetstrokecolor{currentstroke}%
\pgfsetdash{}{0pt}%
\pgfpathmoveto{\pgfqpoint{3.131597in}{2.104559in}}%
\pgfpathcurveto{\pgfqpoint{3.142647in}{2.104559in}}{\pgfqpoint{3.153246in}{2.108949in}}{\pgfqpoint{3.161060in}{2.116763in}}%
\pgfpathcurveto{\pgfqpoint{3.168873in}{2.124576in}}{\pgfqpoint{3.173264in}{2.135175in}}{\pgfqpoint{3.173264in}{2.146225in}}%
\pgfpathcurveto{\pgfqpoint{3.173264in}{2.157275in}}{\pgfqpoint{3.168873in}{2.167874in}}{\pgfqpoint{3.161060in}{2.175688in}}%
\pgfpathcurveto{\pgfqpoint{3.153246in}{2.183502in}}{\pgfqpoint{3.142647in}{2.187892in}}{\pgfqpoint{3.131597in}{2.187892in}}%
\pgfpathcurveto{\pgfqpoint{3.120547in}{2.187892in}}{\pgfqpoint{3.109948in}{2.183502in}}{\pgfqpoint{3.102134in}{2.175688in}}%
\pgfpathcurveto{\pgfqpoint{3.094321in}{2.167874in}}{\pgfqpoint{3.089930in}{2.157275in}}{\pgfqpoint{3.089930in}{2.146225in}}%
\pgfpathcurveto{\pgfqpoint{3.089930in}{2.135175in}}{\pgfqpoint{3.094321in}{2.124576in}}{\pgfqpoint{3.102134in}{2.116763in}}%
\pgfpathcurveto{\pgfqpoint{3.109948in}{2.108949in}}{\pgfqpoint{3.120547in}{2.104559in}}{\pgfqpoint{3.131597in}{2.104559in}}%
\pgfpathclose%
\pgfusepath{stroke,fill}%
\end{pgfscope}%
\begin{pgfscope}%
\pgfpathrectangle{\pgfqpoint{0.481978in}{0.331635in}}{\pgfqpoint{4.960000in}{3.696000in}}%
\pgfusepath{clip}%
\pgfsetbuttcap%
\pgfsetroundjoin%
\definecolor{currentfill}{rgb}{0.631373,0.788235,0.956863}%
\pgfsetfillcolor{currentfill}%
\pgfsetlinewidth{0.481800pt}%
\definecolor{currentstroke}{rgb}{1.000000,1.000000,1.000000}%
\pgfsetstrokecolor{currentstroke}%
\pgfsetdash{}{0pt}%
\pgfpathmoveto{\pgfqpoint{2.013077in}{2.422125in}}%
\pgfpathcurveto{\pgfqpoint{2.024127in}{2.422125in}}{\pgfqpoint{2.034726in}{2.426515in}}{\pgfqpoint{2.042539in}{2.434329in}}%
\pgfpathcurveto{\pgfqpoint{2.050353in}{2.442143in}}{\pgfqpoint{2.054743in}{2.452742in}}{\pgfqpoint{2.054743in}{2.463792in}}%
\pgfpathcurveto{\pgfqpoint{2.054743in}{2.474842in}}{\pgfqpoint{2.050353in}{2.485441in}}{\pgfqpoint{2.042539in}{2.493255in}}%
\pgfpathcurveto{\pgfqpoint{2.034726in}{2.501068in}}{\pgfqpoint{2.024127in}{2.505458in}}{\pgfqpoint{2.013077in}{2.505458in}}%
\pgfpathcurveto{\pgfqpoint{2.002027in}{2.505458in}}{\pgfqpoint{1.991428in}{2.501068in}}{\pgfqpoint{1.983614in}{2.493255in}}%
\pgfpathcurveto{\pgfqpoint{1.975800in}{2.485441in}}{\pgfqpoint{1.971410in}{2.474842in}}{\pgfqpoint{1.971410in}{2.463792in}}%
\pgfpathcurveto{\pgfqpoint{1.971410in}{2.452742in}}{\pgfqpoint{1.975800in}{2.442143in}}{\pgfqpoint{1.983614in}{2.434329in}}%
\pgfpathcurveto{\pgfqpoint{1.991428in}{2.426515in}}{\pgfqpoint{2.002027in}{2.422125in}}{\pgfqpoint{2.013077in}{2.422125in}}%
\pgfpathclose%
\pgfusepath{stroke,fill}%
\end{pgfscope}%
\begin{pgfscope}%
\pgfpathrectangle{\pgfqpoint{0.481978in}{0.331635in}}{\pgfqpoint{4.960000in}{3.696000in}}%
\pgfusepath{clip}%
\pgfsetbuttcap%
\pgfsetroundjoin%
\definecolor{currentfill}{rgb}{0.631373,0.788235,0.956863}%
\pgfsetfillcolor{currentfill}%
\pgfsetlinewidth{0.481800pt}%
\definecolor{currentstroke}{rgb}{1.000000,1.000000,1.000000}%
\pgfsetstrokecolor{currentstroke}%
\pgfsetdash{}{0pt}%
\pgfpathmoveto{\pgfqpoint{2.508540in}{2.646379in}}%
\pgfpathcurveto{\pgfqpoint{2.519590in}{2.646379in}}{\pgfqpoint{2.530189in}{2.650769in}}{\pgfqpoint{2.538003in}{2.658583in}}%
\pgfpathcurveto{\pgfqpoint{2.545816in}{2.666396in}}{\pgfqpoint{2.550207in}{2.676995in}}{\pgfqpoint{2.550207in}{2.688045in}}%
\pgfpathcurveto{\pgfqpoint{2.550207in}{2.699095in}}{\pgfqpoint{2.545816in}{2.709695in}}{\pgfqpoint{2.538003in}{2.717508in}}%
\pgfpathcurveto{\pgfqpoint{2.530189in}{2.725322in}}{\pgfqpoint{2.519590in}{2.729712in}}{\pgfqpoint{2.508540in}{2.729712in}}%
\pgfpathcurveto{\pgfqpoint{2.497490in}{2.729712in}}{\pgfqpoint{2.486891in}{2.725322in}}{\pgfqpoint{2.479077in}{2.717508in}}%
\pgfpathcurveto{\pgfqpoint{2.471264in}{2.709695in}}{\pgfqpoint{2.466873in}{2.699095in}}{\pgfqpoint{2.466873in}{2.688045in}}%
\pgfpathcurveto{\pgfqpoint{2.466873in}{2.676995in}}{\pgfqpoint{2.471264in}{2.666396in}}{\pgfqpoint{2.479077in}{2.658583in}}%
\pgfpathcurveto{\pgfqpoint{2.486891in}{2.650769in}}{\pgfqpoint{2.497490in}{2.646379in}}{\pgfqpoint{2.508540in}{2.646379in}}%
\pgfpathclose%
\pgfusepath{stroke,fill}%
\end{pgfscope}%
\begin{pgfscope}%
\pgfpathrectangle{\pgfqpoint{0.481978in}{0.331635in}}{\pgfqpoint{4.960000in}{3.696000in}}%
\pgfusepath{clip}%
\pgfsetbuttcap%
\pgfsetroundjoin%
\definecolor{currentfill}{rgb}{0.631373,0.788235,0.956863}%
\pgfsetfillcolor{currentfill}%
\pgfsetlinewidth{0.481800pt}%
\definecolor{currentstroke}{rgb}{1.000000,1.000000,1.000000}%
\pgfsetstrokecolor{currentstroke}%
\pgfsetdash{}{0pt}%
\pgfpathmoveto{\pgfqpoint{2.185269in}{2.604248in}}%
\pgfpathcurveto{\pgfqpoint{2.196319in}{2.604248in}}{\pgfqpoint{2.206918in}{2.608638in}}{\pgfqpoint{2.214732in}{2.616452in}}%
\pgfpathcurveto{\pgfqpoint{2.222545in}{2.624266in}}{\pgfqpoint{2.226936in}{2.634865in}}{\pgfqpoint{2.226936in}{2.645915in}}%
\pgfpathcurveto{\pgfqpoint{2.226936in}{2.656965in}}{\pgfqpoint{2.222545in}{2.667564in}}{\pgfqpoint{2.214732in}{2.675377in}}%
\pgfpathcurveto{\pgfqpoint{2.206918in}{2.683191in}}{\pgfqpoint{2.196319in}{2.687581in}}{\pgfqpoint{2.185269in}{2.687581in}}%
\pgfpathcurveto{\pgfqpoint{2.174219in}{2.687581in}}{\pgfqpoint{2.163620in}{2.683191in}}{\pgfqpoint{2.155806in}{2.675377in}}%
\pgfpathcurveto{\pgfqpoint{2.147992in}{2.667564in}}{\pgfqpoint{2.143602in}{2.656965in}}{\pgfqpoint{2.143602in}{2.645915in}}%
\pgfpathcurveto{\pgfqpoint{2.143602in}{2.634865in}}{\pgfqpoint{2.147992in}{2.624266in}}{\pgfqpoint{2.155806in}{2.616452in}}%
\pgfpathcurveto{\pgfqpoint{2.163620in}{2.608638in}}{\pgfqpoint{2.174219in}{2.604248in}}{\pgfqpoint{2.185269in}{2.604248in}}%
\pgfpathclose%
\pgfusepath{stroke,fill}%
\end{pgfscope}%
\begin{pgfscope}%
\pgfpathrectangle{\pgfqpoint{0.481978in}{0.331635in}}{\pgfqpoint{4.960000in}{3.696000in}}%
\pgfusepath{clip}%
\pgfsetbuttcap%
\pgfsetroundjoin%
\definecolor{currentfill}{rgb}{0.631373,0.788235,0.956863}%
\pgfsetfillcolor{currentfill}%
\pgfsetlinewidth{0.481800pt}%
\definecolor{currentstroke}{rgb}{1.000000,1.000000,1.000000}%
\pgfsetstrokecolor{currentstroke}%
\pgfsetdash{}{0pt}%
\pgfpathmoveto{\pgfqpoint{1.877370in}{2.521227in}}%
\pgfpathcurveto{\pgfqpoint{1.888420in}{2.521227in}}{\pgfqpoint{1.899019in}{2.525617in}}{\pgfqpoint{1.906833in}{2.533431in}}%
\pgfpathcurveto{\pgfqpoint{1.914646in}{2.541245in}}{\pgfqpoint{1.919037in}{2.551844in}}{\pgfqpoint{1.919037in}{2.562894in}}%
\pgfpathcurveto{\pgfqpoint{1.919037in}{2.573944in}}{\pgfqpoint{1.914646in}{2.584543in}}{\pgfqpoint{1.906833in}{2.592357in}}%
\pgfpathcurveto{\pgfqpoint{1.899019in}{2.600170in}}{\pgfqpoint{1.888420in}{2.604561in}}{\pgfqpoint{1.877370in}{2.604561in}}%
\pgfpathcurveto{\pgfqpoint{1.866320in}{2.604561in}}{\pgfqpoint{1.855721in}{2.600170in}}{\pgfqpoint{1.847907in}{2.592357in}}%
\pgfpathcurveto{\pgfqpoint{1.840094in}{2.584543in}}{\pgfqpoint{1.835703in}{2.573944in}}{\pgfqpoint{1.835703in}{2.562894in}}%
\pgfpathcurveto{\pgfqpoint{1.835703in}{2.551844in}}{\pgfqpoint{1.840094in}{2.541245in}}{\pgfqpoint{1.847907in}{2.533431in}}%
\pgfpathcurveto{\pgfqpoint{1.855721in}{2.525617in}}{\pgfqpoint{1.866320in}{2.521227in}}{\pgfqpoint{1.877370in}{2.521227in}}%
\pgfpathclose%
\pgfusepath{stroke,fill}%
\end{pgfscope}%
\begin{pgfscope}%
\pgfpathrectangle{\pgfqpoint{0.481978in}{0.331635in}}{\pgfqpoint{4.960000in}{3.696000in}}%
\pgfusepath{clip}%
\pgfsetbuttcap%
\pgfsetroundjoin%
\definecolor{currentfill}{rgb}{0.631373,0.788235,0.956863}%
\pgfsetfillcolor{currentfill}%
\pgfsetlinewidth{0.481800pt}%
\definecolor{currentstroke}{rgb}{1.000000,1.000000,1.000000}%
\pgfsetstrokecolor{currentstroke}%
\pgfsetdash{}{0pt}%
\pgfpathmoveto{\pgfqpoint{2.612806in}{2.814576in}}%
\pgfpathcurveto{\pgfqpoint{2.623856in}{2.814576in}}{\pgfqpoint{2.634455in}{2.818967in}}{\pgfqpoint{2.642269in}{2.826780in}}%
\pgfpathcurveto{\pgfqpoint{2.650083in}{2.834594in}}{\pgfqpoint{2.654473in}{2.845193in}}{\pgfqpoint{2.654473in}{2.856243in}}%
\pgfpathcurveto{\pgfqpoint{2.654473in}{2.867293in}}{\pgfqpoint{2.650083in}{2.877892in}}{\pgfqpoint{2.642269in}{2.885706in}}%
\pgfpathcurveto{\pgfqpoint{2.634455in}{2.893520in}}{\pgfqpoint{2.623856in}{2.897910in}}{\pgfqpoint{2.612806in}{2.897910in}}%
\pgfpathcurveto{\pgfqpoint{2.601756in}{2.897910in}}{\pgfqpoint{2.591157in}{2.893520in}}{\pgfqpoint{2.583343in}{2.885706in}}%
\pgfpathcurveto{\pgfqpoint{2.575530in}{2.877892in}}{\pgfqpoint{2.571139in}{2.867293in}}{\pgfqpoint{2.571139in}{2.856243in}}%
\pgfpathcurveto{\pgfqpoint{2.571139in}{2.845193in}}{\pgfqpoint{2.575530in}{2.834594in}}{\pgfqpoint{2.583343in}{2.826780in}}%
\pgfpathcurveto{\pgfqpoint{2.591157in}{2.818967in}}{\pgfqpoint{2.601756in}{2.814576in}}{\pgfqpoint{2.612806in}{2.814576in}}%
\pgfpathclose%
\pgfusepath{stroke,fill}%
\end{pgfscope}%
\begin{pgfscope}%
\pgfpathrectangle{\pgfqpoint{0.481978in}{0.331635in}}{\pgfqpoint{4.960000in}{3.696000in}}%
\pgfusepath{clip}%
\pgfsetbuttcap%
\pgfsetroundjoin%
\definecolor{currentfill}{rgb}{0.631373,0.788235,0.956863}%
\pgfsetfillcolor{currentfill}%
\pgfsetlinewidth{0.481800pt}%
\definecolor{currentstroke}{rgb}{1.000000,1.000000,1.000000}%
\pgfsetstrokecolor{currentstroke}%
\pgfsetdash{}{0pt}%
\pgfpathmoveto{\pgfqpoint{2.285707in}{1.666283in}}%
\pgfpathcurveto{\pgfqpoint{2.296757in}{1.666283in}}{\pgfqpoint{2.307356in}{1.670674in}}{\pgfqpoint{2.315169in}{1.678487in}}%
\pgfpathcurveto{\pgfqpoint{2.322983in}{1.686301in}}{\pgfqpoint{2.327373in}{1.696900in}}{\pgfqpoint{2.327373in}{1.707950in}}%
\pgfpathcurveto{\pgfqpoint{2.327373in}{1.719000in}}{\pgfqpoint{2.322983in}{1.729599in}}{\pgfqpoint{2.315169in}{1.737413in}}%
\pgfpathcurveto{\pgfqpoint{2.307356in}{1.745227in}}{\pgfqpoint{2.296757in}{1.749617in}}{\pgfqpoint{2.285707in}{1.749617in}}%
\pgfpathcurveto{\pgfqpoint{2.274657in}{1.749617in}}{\pgfqpoint{2.264058in}{1.745227in}}{\pgfqpoint{2.256244in}{1.737413in}}%
\pgfpathcurveto{\pgfqpoint{2.248430in}{1.729599in}}{\pgfqpoint{2.244040in}{1.719000in}}{\pgfqpoint{2.244040in}{1.707950in}}%
\pgfpathcurveto{\pgfqpoint{2.244040in}{1.696900in}}{\pgfqpoint{2.248430in}{1.686301in}}{\pgfqpoint{2.256244in}{1.678487in}}%
\pgfpathcurveto{\pgfqpoint{2.264058in}{1.670674in}}{\pgfqpoint{2.274657in}{1.666283in}}{\pgfqpoint{2.285707in}{1.666283in}}%
\pgfpathclose%
\pgfusepath{stroke,fill}%
\end{pgfscope}%
\begin{pgfscope}%
\pgfpathrectangle{\pgfqpoint{0.481978in}{0.331635in}}{\pgfqpoint{4.960000in}{3.696000in}}%
\pgfusepath{clip}%
\pgfsetbuttcap%
\pgfsetroundjoin%
\definecolor{currentfill}{rgb}{0.631373,0.788235,0.956863}%
\pgfsetfillcolor{currentfill}%
\pgfsetlinewidth{0.481800pt}%
\definecolor{currentstroke}{rgb}{1.000000,1.000000,1.000000}%
\pgfsetstrokecolor{currentstroke}%
\pgfsetdash{}{0pt}%
\pgfpathmoveto{\pgfqpoint{1.585888in}{2.576023in}}%
\pgfpathcurveto{\pgfqpoint{1.596938in}{2.576023in}}{\pgfqpoint{1.607537in}{2.580413in}}{\pgfqpoint{1.615351in}{2.588227in}}%
\pgfpathcurveto{\pgfqpoint{1.623165in}{2.596041in}}{\pgfqpoint{1.627555in}{2.606640in}}{\pgfqpoint{1.627555in}{2.617690in}}%
\pgfpathcurveto{\pgfqpoint{1.627555in}{2.628740in}}{\pgfqpoint{1.623165in}{2.639339in}}{\pgfqpoint{1.615351in}{2.647153in}}%
\pgfpathcurveto{\pgfqpoint{1.607537in}{2.654966in}}{\pgfqpoint{1.596938in}{2.659356in}}{\pgfqpoint{1.585888in}{2.659356in}}%
\pgfpathcurveto{\pgfqpoint{1.574838in}{2.659356in}}{\pgfqpoint{1.564239in}{2.654966in}}{\pgfqpoint{1.556425in}{2.647153in}}%
\pgfpathcurveto{\pgfqpoint{1.548612in}{2.639339in}}{\pgfqpoint{1.544222in}{2.628740in}}{\pgfqpoint{1.544222in}{2.617690in}}%
\pgfpathcurveto{\pgfqpoint{1.544222in}{2.606640in}}{\pgfqpoint{1.548612in}{2.596041in}}{\pgfqpoint{1.556425in}{2.588227in}}%
\pgfpathcurveto{\pgfqpoint{1.564239in}{2.580413in}}{\pgfqpoint{1.574838in}{2.576023in}}{\pgfqpoint{1.585888in}{2.576023in}}%
\pgfpathclose%
\pgfusepath{stroke,fill}%
\end{pgfscope}%
\begin{pgfscope}%
\pgfpathrectangle{\pgfqpoint{0.481978in}{0.331635in}}{\pgfqpoint{4.960000in}{3.696000in}}%
\pgfusepath{clip}%
\pgfsetbuttcap%
\pgfsetroundjoin%
\definecolor{currentfill}{rgb}{0.631373,0.788235,0.956863}%
\pgfsetfillcolor{currentfill}%
\pgfsetlinewidth{0.481800pt}%
\definecolor{currentstroke}{rgb}{1.000000,1.000000,1.000000}%
\pgfsetstrokecolor{currentstroke}%
\pgfsetdash{}{0pt}%
\pgfpathmoveto{\pgfqpoint{4.113942in}{1.815612in}}%
\pgfpathcurveto{\pgfqpoint{4.124993in}{1.815612in}}{\pgfqpoint{4.135592in}{1.820002in}}{\pgfqpoint{4.143405in}{1.827816in}}%
\pgfpathcurveto{\pgfqpoint{4.151219in}{1.835629in}}{\pgfqpoint{4.155609in}{1.846228in}}{\pgfqpoint{4.155609in}{1.857278in}}%
\pgfpathcurveto{\pgfqpoint{4.155609in}{1.868329in}}{\pgfqpoint{4.151219in}{1.878928in}}{\pgfqpoint{4.143405in}{1.886741in}}%
\pgfpathcurveto{\pgfqpoint{4.135592in}{1.894555in}}{\pgfqpoint{4.124993in}{1.898945in}}{\pgfqpoint{4.113942in}{1.898945in}}%
\pgfpathcurveto{\pgfqpoint{4.102892in}{1.898945in}}{\pgfqpoint{4.092293in}{1.894555in}}{\pgfqpoint{4.084480in}{1.886741in}}%
\pgfpathcurveto{\pgfqpoint{4.076666in}{1.878928in}}{\pgfqpoint{4.072276in}{1.868329in}}{\pgfqpoint{4.072276in}{1.857278in}}%
\pgfpathcurveto{\pgfqpoint{4.072276in}{1.846228in}}{\pgfqpoint{4.076666in}{1.835629in}}{\pgfqpoint{4.084480in}{1.827816in}}%
\pgfpathcurveto{\pgfqpoint{4.092293in}{1.820002in}}{\pgfqpoint{4.102892in}{1.815612in}}{\pgfqpoint{4.113942in}{1.815612in}}%
\pgfpathclose%
\pgfusepath{stroke,fill}%
\end{pgfscope}%
\begin{pgfscope}%
\pgfpathrectangle{\pgfqpoint{0.481978in}{0.331635in}}{\pgfqpoint{4.960000in}{3.696000in}}%
\pgfusepath{clip}%
\pgfsetbuttcap%
\pgfsetroundjoin%
\definecolor{currentfill}{rgb}{0.631373,0.788235,0.956863}%
\pgfsetfillcolor{currentfill}%
\pgfsetlinewidth{0.481800pt}%
\definecolor{currentstroke}{rgb}{1.000000,1.000000,1.000000}%
\pgfsetstrokecolor{currentstroke}%
\pgfsetdash{}{0pt}%
\pgfpathmoveto{\pgfqpoint{2.608551in}{3.306649in}}%
\pgfpathcurveto{\pgfqpoint{2.619601in}{3.306649in}}{\pgfqpoint{2.630200in}{3.311039in}}{\pgfqpoint{2.638013in}{3.318853in}}%
\pgfpathcurveto{\pgfqpoint{2.645827in}{3.326667in}}{\pgfqpoint{2.650217in}{3.337266in}}{\pgfqpoint{2.650217in}{3.348316in}}%
\pgfpathcurveto{\pgfqpoint{2.650217in}{3.359366in}}{\pgfqpoint{2.645827in}{3.369965in}}{\pgfqpoint{2.638013in}{3.377779in}}%
\pgfpathcurveto{\pgfqpoint{2.630200in}{3.385592in}}{\pgfqpoint{2.619601in}{3.389982in}}{\pgfqpoint{2.608551in}{3.389982in}}%
\pgfpathcurveto{\pgfqpoint{2.597500in}{3.389982in}}{\pgfqpoint{2.586901in}{3.385592in}}{\pgfqpoint{2.579088in}{3.377779in}}%
\pgfpathcurveto{\pgfqpoint{2.571274in}{3.369965in}}{\pgfqpoint{2.566884in}{3.359366in}}{\pgfqpoint{2.566884in}{3.348316in}}%
\pgfpathcurveto{\pgfqpoint{2.566884in}{3.337266in}}{\pgfqpoint{2.571274in}{3.326667in}}{\pgfqpoint{2.579088in}{3.318853in}}%
\pgfpathcurveto{\pgfqpoint{2.586901in}{3.311039in}}{\pgfqpoint{2.597500in}{3.306649in}}{\pgfqpoint{2.608551in}{3.306649in}}%
\pgfpathclose%
\pgfusepath{stroke,fill}%
\end{pgfscope}%
\begin{pgfscope}%
\pgfpathrectangle{\pgfqpoint{0.481978in}{0.331635in}}{\pgfqpoint{4.960000in}{3.696000in}}%
\pgfusepath{clip}%
\pgfsetbuttcap%
\pgfsetroundjoin%
\definecolor{currentfill}{rgb}{0.631373,0.788235,0.956863}%
\pgfsetfillcolor{currentfill}%
\pgfsetlinewidth{0.481800pt}%
\definecolor{currentstroke}{rgb}{1.000000,1.000000,1.000000}%
\pgfsetstrokecolor{currentstroke}%
\pgfsetdash{}{0pt}%
\pgfpathmoveto{\pgfqpoint{1.832941in}{1.602670in}}%
\pgfpathcurveto{\pgfqpoint{1.843991in}{1.602670in}}{\pgfqpoint{1.854591in}{1.607060in}}{\pgfqpoint{1.862404in}{1.614874in}}%
\pgfpathcurveto{\pgfqpoint{1.870218in}{1.622687in}}{\pgfqpoint{1.874608in}{1.633286in}}{\pgfqpoint{1.874608in}{1.644336in}}%
\pgfpathcurveto{\pgfqpoint{1.874608in}{1.655387in}}{\pgfqpoint{1.870218in}{1.665986in}}{\pgfqpoint{1.862404in}{1.673799in}}%
\pgfpathcurveto{\pgfqpoint{1.854591in}{1.681613in}}{\pgfqpoint{1.843991in}{1.686003in}}{\pgfqpoint{1.832941in}{1.686003in}}%
\pgfpathcurveto{\pgfqpoint{1.821891in}{1.686003in}}{\pgfqpoint{1.811292in}{1.681613in}}{\pgfqpoint{1.803479in}{1.673799in}}%
\pgfpathcurveto{\pgfqpoint{1.795665in}{1.665986in}}{\pgfqpoint{1.791275in}{1.655387in}}{\pgfqpoint{1.791275in}{1.644336in}}%
\pgfpathcurveto{\pgfqpoint{1.791275in}{1.633286in}}{\pgfqpoint{1.795665in}{1.622687in}}{\pgfqpoint{1.803479in}{1.614874in}}%
\pgfpathcurveto{\pgfqpoint{1.811292in}{1.607060in}}{\pgfqpoint{1.821891in}{1.602670in}}{\pgfqpoint{1.832941in}{1.602670in}}%
\pgfpathclose%
\pgfusepath{stroke,fill}%
\end{pgfscope}%
\begin{pgfscope}%
\pgfpathrectangle{\pgfqpoint{0.481978in}{0.331635in}}{\pgfqpoint{4.960000in}{3.696000in}}%
\pgfusepath{clip}%
\pgfsetbuttcap%
\pgfsetroundjoin%
\definecolor{currentfill}{rgb}{0.631373,0.788235,0.956863}%
\pgfsetfillcolor{currentfill}%
\pgfsetlinewidth{0.481800pt}%
\definecolor{currentstroke}{rgb}{1.000000,1.000000,1.000000}%
\pgfsetstrokecolor{currentstroke}%
\pgfsetdash{}{0pt}%
\pgfpathmoveto{\pgfqpoint{3.758211in}{1.487165in}}%
\pgfpathcurveto{\pgfqpoint{3.769261in}{1.487165in}}{\pgfqpoint{3.779860in}{1.491555in}}{\pgfqpoint{3.787673in}{1.499369in}}%
\pgfpathcurveto{\pgfqpoint{3.795487in}{1.507182in}}{\pgfqpoint{3.799877in}{1.517781in}}{\pgfqpoint{3.799877in}{1.528832in}}%
\pgfpathcurveto{\pgfqpoint{3.799877in}{1.539882in}}{\pgfqpoint{3.795487in}{1.550481in}}{\pgfqpoint{3.787673in}{1.558294in}}%
\pgfpathcurveto{\pgfqpoint{3.779860in}{1.566108in}}{\pgfqpoint{3.769261in}{1.570498in}}{\pgfqpoint{3.758211in}{1.570498in}}%
\pgfpathcurveto{\pgfqpoint{3.747161in}{1.570498in}}{\pgfqpoint{3.736562in}{1.566108in}}{\pgfqpoint{3.728748in}{1.558294in}}%
\pgfpathcurveto{\pgfqpoint{3.720934in}{1.550481in}}{\pgfqpoint{3.716544in}{1.539882in}}{\pgfqpoint{3.716544in}{1.528832in}}%
\pgfpathcurveto{\pgfqpoint{3.716544in}{1.517781in}}{\pgfqpoint{3.720934in}{1.507182in}}{\pgfqpoint{3.728748in}{1.499369in}}%
\pgfpathcurveto{\pgfqpoint{3.736562in}{1.491555in}}{\pgfqpoint{3.747161in}{1.487165in}}{\pgfqpoint{3.758211in}{1.487165in}}%
\pgfpathclose%
\pgfusepath{stroke,fill}%
\end{pgfscope}%
\begin{pgfscope}%
\pgfpathrectangle{\pgfqpoint{0.481978in}{0.331635in}}{\pgfqpoint{4.960000in}{3.696000in}}%
\pgfusepath{clip}%
\pgfsetbuttcap%
\pgfsetroundjoin%
\definecolor{currentfill}{rgb}{0.631373,0.788235,0.956863}%
\pgfsetfillcolor{currentfill}%
\pgfsetlinewidth{0.481800pt}%
\definecolor{currentstroke}{rgb}{1.000000,1.000000,1.000000}%
\pgfsetstrokecolor{currentstroke}%
\pgfsetdash{}{0pt}%
\pgfpathmoveto{\pgfqpoint{2.408577in}{2.823665in}}%
\pgfpathcurveto{\pgfqpoint{2.419627in}{2.823665in}}{\pgfqpoint{2.430226in}{2.828055in}}{\pgfqpoint{2.438040in}{2.835869in}}%
\pgfpathcurveto{\pgfqpoint{2.445854in}{2.843682in}}{\pgfqpoint{2.450244in}{2.854281in}}{\pgfqpoint{2.450244in}{2.865331in}}%
\pgfpathcurveto{\pgfqpoint{2.450244in}{2.876382in}}{\pgfqpoint{2.445854in}{2.886981in}}{\pgfqpoint{2.438040in}{2.894794in}}%
\pgfpathcurveto{\pgfqpoint{2.430226in}{2.902608in}}{\pgfqpoint{2.419627in}{2.906998in}}{\pgfqpoint{2.408577in}{2.906998in}}%
\pgfpathcurveto{\pgfqpoint{2.397527in}{2.906998in}}{\pgfqpoint{2.386928in}{2.902608in}}{\pgfqpoint{2.379114in}{2.894794in}}%
\pgfpathcurveto{\pgfqpoint{2.371301in}{2.886981in}}{\pgfqpoint{2.366911in}{2.876382in}}{\pgfqpoint{2.366911in}{2.865331in}}%
\pgfpathcurveto{\pgfqpoint{2.366911in}{2.854281in}}{\pgfqpoint{2.371301in}{2.843682in}}{\pgfqpoint{2.379114in}{2.835869in}}%
\pgfpathcurveto{\pgfqpoint{2.386928in}{2.828055in}}{\pgfqpoint{2.397527in}{2.823665in}}{\pgfqpoint{2.408577in}{2.823665in}}%
\pgfpathclose%
\pgfusepath{stroke,fill}%
\end{pgfscope}%
\begin{pgfscope}%
\pgfpathrectangle{\pgfqpoint{0.481978in}{0.331635in}}{\pgfqpoint{4.960000in}{3.696000in}}%
\pgfusepath{clip}%
\pgfsetbuttcap%
\pgfsetroundjoin%
\definecolor{currentfill}{rgb}{0.631373,0.788235,0.956863}%
\pgfsetfillcolor{currentfill}%
\pgfsetlinewidth{0.481800pt}%
\definecolor{currentstroke}{rgb}{1.000000,1.000000,1.000000}%
\pgfsetstrokecolor{currentstroke}%
\pgfsetdash{}{0pt}%
\pgfpathmoveto{\pgfqpoint{3.296638in}{2.859409in}}%
\pgfpathcurveto{\pgfqpoint{3.307688in}{2.859409in}}{\pgfqpoint{3.318287in}{2.863799in}}{\pgfqpoint{3.326101in}{2.871613in}}%
\pgfpathcurveto{\pgfqpoint{3.333915in}{2.879427in}}{\pgfqpoint{3.338305in}{2.890026in}}{\pgfqpoint{3.338305in}{2.901076in}}%
\pgfpathcurveto{\pgfqpoint{3.338305in}{2.912126in}}{\pgfqpoint{3.333915in}{2.922725in}}{\pgfqpoint{3.326101in}{2.930539in}}%
\pgfpathcurveto{\pgfqpoint{3.318287in}{2.938352in}}{\pgfqpoint{3.307688in}{2.942743in}}{\pgfqpoint{3.296638in}{2.942743in}}%
\pgfpathcurveto{\pgfqpoint{3.285588in}{2.942743in}}{\pgfqpoint{3.274989in}{2.938352in}}{\pgfqpoint{3.267175in}{2.930539in}}%
\pgfpathcurveto{\pgfqpoint{3.259362in}{2.922725in}}{\pgfqpoint{3.254971in}{2.912126in}}{\pgfqpoint{3.254971in}{2.901076in}}%
\pgfpathcurveto{\pgfqpoint{3.254971in}{2.890026in}}{\pgfqpoint{3.259362in}{2.879427in}}{\pgfqpoint{3.267175in}{2.871613in}}%
\pgfpathcurveto{\pgfqpoint{3.274989in}{2.863799in}}{\pgfqpoint{3.285588in}{2.859409in}}{\pgfqpoint{3.296638in}{2.859409in}}%
\pgfpathclose%
\pgfusepath{stroke,fill}%
\end{pgfscope}%
\begin{pgfscope}%
\pgfpathrectangle{\pgfqpoint{0.481978in}{0.331635in}}{\pgfqpoint{4.960000in}{3.696000in}}%
\pgfusepath{clip}%
\pgfsetbuttcap%
\pgfsetroundjoin%
\definecolor{currentfill}{rgb}{0.631373,0.788235,0.956863}%
\pgfsetfillcolor{currentfill}%
\pgfsetlinewidth{0.481800pt}%
\definecolor{currentstroke}{rgb}{1.000000,1.000000,1.000000}%
\pgfsetstrokecolor{currentstroke}%
\pgfsetdash{}{0pt}%
\pgfpathmoveto{\pgfqpoint{1.439821in}{2.233033in}}%
\pgfpathcurveto{\pgfqpoint{1.450871in}{2.233033in}}{\pgfqpoint{1.461470in}{2.237424in}}{\pgfqpoint{1.469284in}{2.245237in}}%
\pgfpathcurveto{\pgfqpoint{1.477098in}{2.253051in}}{\pgfqpoint{1.481488in}{2.263650in}}{\pgfqpoint{1.481488in}{2.274700in}}%
\pgfpathcurveto{\pgfqpoint{1.481488in}{2.285750in}}{\pgfqpoint{1.477098in}{2.296349in}}{\pgfqpoint{1.469284in}{2.304163in}}%
\pgfpathcurveto{\pgfqpoint{1.461470in}{2.311976in}}{\pgfqpoint{1.450871in}{2.316367in}}{\pgfqpoint{1.439821in}{2.316367in}}%
\pgfpathcurveto{\pgfqpoint{1.428771in}{2.316367in}}{\pgfqpoint{1.418172in}{2.311976in}}{\pgfqpoint{1.410358in}{2.304163in}}%
\pgfpathcurveto{\pgfqpoint{1.402545in}{2.296349in}}{\pgfqpoint{1.398154in}{2.285750in}}{\pgfqpoint{1.398154in}{2.274700in}}%
\pgfpathcurveto{\pgfqpoint{1.398154in}{2.263650in}}{\pgfqpoint{1.402545in}{2.253051in}}{\pgfqpoint{1.410358in}{2.245237in}}%
\pgfpathcurveto{\pgfqpoint{1.418172in}{2.237424in}}{\pgfqpoint{1.428771in}{2.233033in}}{\pgfqpoint{1.439821in}{2.233033in}}%
\pgfpathclose%
\pgfusepath{stroke,fill}%
\end{pgfscope}%
\begin{pgfscope}%
\pgfpathrectangle{\pgfqpoint{0.481978in}{0.331635in}}{\pgfqpoint{4.960000in}{3.696000in}}%
\pgfusepath{clip}%
\pgfsetbuttcap%
\pgfsetroundjoin%
\definecolor{currentfill}{rgb}{0.631373,0.788235,0.956863}%
\pgfsetfillcolor{currentfill}%
\pgfsetlinewidth{0.481800pt}%
\definecolor{currentstroke}{rgb}{1.000000,1.000000,1.000000}%
\pgfsetstrokecolor{currentstroke}%
\pgfsetdash{}{0pt}%
\pgfpathmoveto{\pgfqpoint{3.276159in}{2.334258in}}%
\pgfpathcurveto{\pgfqpoint{3.287209in}{2.334258in}}{\pgfqpoint{3.297808in}{2.338649in}}{\pgfqpoint{3.305622in}{2.346462in}}%
\pgfpathcurveto{\pgfqpoint{3.313436in}{2.354276in}}{\pgfqpoint{3.317826in}{2.364875in}}{\pgfqpoint{3.317826in}{2.375925in}}%
\pgfpathcurveto{\pgfqpoint{3.317826in}{2.386975in}}{\pgfqpoint{3.313436in}{2.397574in}}{\pgfqpoint{3.305622in}{2.405388in}}%
\pgfpathcurveto{\pgfqpoint{3.297808in}{2.413202in}}{\pgfqpoint{3.287209in}{2.417592in}}{\pgfqpoint{3.276159in}{2.417592in}}%
\pgfpathcurveto{\pgfqpoint{3.265109in}{2.417592in}}{\pgfqpoint{3.254510in}{2.413202in}}{\pgfqpoint{3.246696in}{2.405388in}}%
\pgfpathcurveto{\pgfqpoint{3.238883in}{2.397574in}}{\pgfqpoint{3.234492in}{2.386975in}}{\pgfqpoint{3.234492in}{2.375925in}}%
\pgfpathcurveto{\pgfqpoint{3.234492in}{2.364875in}}{\pgfqpoint{3.238883in}{2.354276in}}{\pgfqpoint{3.246696in}{2.346462in}}%
\pgfpathcurveto{\pgfqpoint{3.254510in}{2.338649in}}{\pgfqpoint{3.265109in}{2.334258in}}{\pgfqpoint{3.276159in}{2.334258in}}%
\pgfpathclose%
\pgfusepath{stroke,fill}%
\end{pgfscope}%
\begin{pgfscope}%
\pgfpathrectangle{\pgfqpoint{0.481978in}{0.331635in}}{\pgfqpoint{4.960000in}{3.696000in}}%
\pgfusepath{clip}%
\pgfsetbuttcap%
\pgfsetroundjoin%
\definecolor{currentfill}{rgb}{0.631373,0.788235,0.956863}%
\pgfsetfillcolor{currentfill}%
\pgfsetlinewidth{0.481800pt}%
\definecolor{currentstroke}{rgb}{1.000000,1.000000,1.000000}%
\pgfsetstrokecolor{currentstroke}%
\pgfsetdash{}{0pt}%
\pgfpathmoveto{\pgfqpoint{1.912245in}{1.571402in}}%
\pgfpathcurveto{\pgfqpoint{1.923295in}{1.571402in}}{\pgfqpoint{1.933894in}{1.575792in}}{\pgfqpoint{1.941708in}{1.583606in}}%
\pgfpathcurveto{\pgfqpoint{1.949521in}{1.591420in}}{\pgfqpoint{1.953911in}{1.602019in}}{\pgfqpoint{1.953911in}{1.613069in}}%
\pgfpathcurveto{\pgfqpoint{1.953911in}{1.624119in}}{\pgfqpoint{1.949521in}{1.634718in}}{\pgfqpoint{1.941708in}{1.642532in}}%
\pgfpathcurveto{\pgfqpoint{1.933894in}{1.650345in}}{\pgfqpoint{1.923295in}{1.654735in}}{\pgfqpoint{1.912245in}{1.654735in}}%
\pgfpathcurveto{\pgfqpoint{1.901195in}{1.654735in}}{\pgfqpoint{1.890596in}{1.650345in}}{\pgfqpoint{1.882782in}{1.642532in}}%
\pgfpathcurveto{\pgfqpoint{1.874968in}{1.634718in}}{\pgfqpoint{1.870578in}{1.624119in}}{\pgfqpoint{1.870578in}{1.613069in}}%
\pgfpathcurveto{\pgfqpoint{1.870578in}{1.602019in}}{\pgfqpoint{1.874968in}{1.591420in}}{\pgfqpoint{1.882782in}{1.583606in}}%
\pgfpathcurveto{\pgfqpoint{1.890596in}{1.575792in}}{\pgfqpoint{1.901195in}{1.571402in}}{\pgfqpoint{1.912245in}{1.571402in}}%
\pgfpathclose%
\pgfusepath{stroke,fill}%
\end{pgfscope}%
\begin{pgfscope}%
\pgfpathrectangle{\pgfqpoint{0.481978in}{0.331635in}}{\pgfqpoint{4.960000in}{3.696000in}}%
\pgfusepath{clip}%
\pgfsetbuttcap%
\pgfsetroundjoin%
\definecolor{currentfill}{rgb}{0.631373,0.788235,0.956863}%
\pgfsetfillcolor{currentfill}%
\pgfsetlinewidth{0.481800pt}%
\definecolor{currentstroke}{rgb}{1.000000,1.000000,1.000000}%
\pgfsetstrokecolor{currentstroke}%
\pgfsetdash{}{0pt}%
\pgfpathmoveto{\pgfqpoint{4.503965in}{1.876754in}}%
\pgfpathcurveto{\pgfqpoint{4.515016in}{1.876754in}}{\pgfqpoint{4.525615in}{1.881144in}}{\pgfqpoint{4.533428in}{1.888957in}}%
\pgfpathcurveto{\pgfqpoint{4.541242in}{1.896771in}}{\pgfqpoint{4.545632in}{1.907370in}}{\pgfqpoint{4.545632in}{1.918420in}}%
\pgfpathcurveto{\pgfqpoint{4.545632in}{1.929470in}}{\pgfqpoint{4.541242in}{1.940069in}}{\pgfqpoint{4.533428in}{1.947883in}}%
\pgfpathcurveto{\pgfqpoint{4.525615in}{1.955697in}}{\pgfqpoint{4.515016in}{1.960087in}}{\pgfqpoint{4.503965in}{1.960087in}}%
\pgfpathcurveto{\pgfqpoint{4.492915in}{1.960087in}}{\pgfqpoint{4.482316in}{1.955697in}}{\pgfqpoint{4.474503in}{1.947883in}}%
\pgfpathcurveto{\pgfqpoint{4.466689in}{1.940069in}}{\pgfqpoint{4.462299in}{1.929470in}}{\pgfqpoint{4.462299in}{1.918420in}}%
\pgfpathcurveto{\pgfqpoint{4.462299in}{1.907370in}}{\pgfqpoint{4.466689in}{1.896771in}}{\pgfqpoint{4.474503in}{1.888957in}}%
\pgfpathcurveto{\pgfqpoint{4.482316in}{1.881144in}}{\pgfqpoint{4.492915in}{1.876754in}}{\pgfqpoint{4.503965in}{1.876754in}}%
\pgfpathclose%
\pgfusepath{stroke,fill}%
\end{pgfscope}%
\begin{pgfscope}%
\pgfpathrectangle{\pgfqpoint{0.481978in}{0.331635in}}{\pgfqpoint{4.960000in}{3.696000in}}%
\pgfusepath{clip}%
\pgfsetbuttcap%
\pgfsetroundjoin%
\definecolor{currentfill}{rgb}{0.631373,0.788235,0.956863}%
\pgfsetfillcolor{currentfill}%
\pgfsetlinewidth{0.481800pt}%
\definecolor{currentstroke}{rgb}{1.000000,1.000000,1.000000}%
\pgfsetstrokecolor{currentstroke}%
\pgfsetdash{}{0pt}%
\pgfpathmoveto{\pgfqpoint{1.288010in}{2.822750in}}%
\pgfpathcurveto{\pgfqpoint{1.299060in}{2.822750in}}{\pgfqpoint{1.309659in}{2.827141in}}{\pgfqpoint{1.317473in}{2.834954in}}%
\pgfpathcurveto{\pgfqpoint{1.325286in}{2.842768in}}{\pgfqpoint{1.329676in}{2.853367in}}{\pgfqpoint{1.329676in}{2.864417in}}%
\pgfpathcurveto{\pgfqpoint{1.329676in}{2.875467in}}{\pgfqpoint{1.325286in}{2.886066in}}{\pgfqpoint{1.317473in}{2.893880in}}%
\pgfpathcurveto{\pgfqpoint{1.309659in}{2.901694in}}{\pgfqpoint{1.299060in}{2.906084in}}{\pgfqpoint{1.288010in}{2.906084in}}%
\pgfpathcurveto{\pgfqpoint{1.276960in}{2.906084in}}{\pgfqpoint{1.266361in}{2.901694in}}{\pgfqpoint{1.258547in}{2.893880in}}%
\pgfpathcurveto{\pgfqpoint{1.250733in}{2.886066in}}{\pgfqpoint{1.246343in}{2.875467in}}{\pgfqpoint{1.246343in}{2.864417in}}%
\pgfpathcurveto{\pgfqpoint{1.246343in}{2.853367in}}{\pgfqpoint{1.250733in}{2.842768in}}{\pgfqpoint{1.258547in}{2.834954in}}%
\pgfpathcurveto{\pgfqpoint{1.266361in}{2.827141in}}{\pgfqpoint{1.276960in}{2.822750in}}{\pgfqpoint{1.288010in}{2.822750in}}%
\pgfpathclose%
\pgfusepath{stroke,fill}%
\end{pgfscope}%
\begin{pgfscope}%
\pgfpathrectangle{\pgfqpoint{0.481978in}{0.331635in}}{\pgfqpoint{4.960000in}{3.696000in}}%
\pgfusepath{clip}%
\pgfsetbuttcap%
\pgfsetroundjoin%
\definecolor{currentfill}{rgb}{0.631373,0.788235,0.956863}%
\pgfsetfillcolor{currentfill}%
\pgfsetlinewidth{0.481800pt}%
\definecolor{currentstroke}{rgb}{1.000000,1.000000,1.000000}%
\pgfsetstrokecolor{currentstroke}%
\pgfsetdash{}{0pt}%
\pgfpathmoveto{\pgfqpoint{2.779164in}{2.325964in}}%
\pgfpathcurveto{\pgfqpoint{2.790214in}{2.325964in}}{\pgfqpoint{2.800813in}{2.330354in}}{\pgfqpoint{2.808627in}{2.338168in}}%
\pgfpathcurveto{\pgfqpoint{2.816440in}{2.345981in}}{\pgfqpoint{2.820831in}{2.356580in}}{\pgfqpoint{2.820831in}{2.367631in}}%
\pgfpathcurveto{\pgfqpoint{2.820831in}{2.378681in}}{\pgfqpoint{2.816440in}{2.389280in}}{\pgfqpoint{2.808627in}{2.397093in}}%
\pgfpathcurveto{\pgfqpoint{2.800813in}{2.404907in}}{\pgfqpoint{2.790214in}{2.409297in}}{\pgfqpoint{2.779164in}{2.409297in}}%
\pgfpathcurveto{\pgfqpoint{2.768114in}{2.409297in}}{\pgfqpoint{2.757515in}{2.404907in}}{\pgfqpoint{2.749701in}{2.397093in}}%
\pgfpathcurveto{\pgfqpoint{2.741888in}{2.389280in}}{\pgfqpoint{2.737497in}{2.378681in}}{\pgfqpoint{2.737497in}{2.367631in}}%
\pgfpathcurveto{\pgfqpoint{2.737497in}{2.356580in}}{\pgfqpoint{2.741888in}{2.345981in}}{\pgfqpoint{2.749701in}{2.338168in}}%
\pgfpathcurveto{\pgfqpoint{2.757515in}{2.330354in}}{\pgfqpoint{2.768114in}{2.325964in}}{\pgfqpoint{2.779164in}{2.325964in}}%
\pgfpathclose%
\pgfusepath{stroke,fill}%
\end{pgfscope}%
\begin{pgfscope}%
\pgfpathrectangle{\pgfqpoint{0.481978in}{0.331635in}}{\pgfqpoint{4.960000in}{3.696000in}}%
\pgfusepath{clip}%
\pgfsetbuttcap%
\pgfsetroundjoin%
\definecolor{currentfill}{rgb}{0.631373,0.788235,0.956863}%
\pgfsetfillcolor{currentfill}%
\pgfsetlinewidth{0.481800pt}%
\definecolor{currentstroke}{rgb}{1.000000,1.000000,1.000000}%
\pgfsetstrokecolor{currentstroke}%
\pgfsetdash{}{0pt}%
\pgfpathmoveto{\pgfqpoint{1.292786in}{3.359039in}}%
\pgfpathcurveto{\pgfqpoint{1.303836in}{3.359039in}}{\pgfqpoint{1.314435in}{3.363430in}}{\pgfqpoint{1.322249in}{3.371243in}}%
\pgfpathcurveto{\pgfqpoint{1.330062in}{3.379057in}}{\pgfqpoint{1.334453in}{3.389656in}}{\pgfqpoint{1.334453in}{3.400706in}}%
\pgfpathcurveto{\pgfqpoint{1.334453in}{3.411756in}}{\pgfqpoint{1.330062in}{3.422355in}}{\pgfqpoint{1.322249in}{3.430169in}}%
\pgfpathcurveto{\pgfqpoint{1.314435in}{3.437983in}}{\pgfqpoint{1.303836in}{3.442373in}}{\pgfqpoint{1.292786in}{3.442373in}}%
\pgfpathcurveto{\pgfqpoint{1.281736in}{3.442373in}}{\pgfqpoint{1.271137in}{3.437983in}}{\pgfqpoint{1.263323in}{3.430169in}}%
\pgfpathcurveto{\pgfqpoint{1.255510in}{3.422355in}}{\pgfqpoint{1.251119in}{3.411756in}}{\pgfqpoint{1.251119in}{3.400706in}}%
\pgfpathcurveto{\pgfqpoint{1.251119in}{3.389656in}}{\pgfqpoint{1.255510in}{3.379057in}}{\pgfqpoint{1.263323in}{3.371243in}}%
\pgfpathcurveto{\pgfqpoint{1.271137in}{3.363430in}}{\pgfqpoint{1.281736in}{3.359039in}}{\pgfqpoint{1.292786in}{3.359039in}}%
\pgfpathclose%
\pgfusepath{stroke,fill}%
\end{pgfscope}%
\begin{pgfscope}%
\pgfpathrectangle{\pgfqpoint{0.481978in}{0.331635in}}{\pgfqpoint{4.960000in}{3.696000in}}%
\pgfusepath{clip}%
\pgfsetbuttcap%
\pgfsetroundjoin%
\definecolor{currentfill}{rgb}{0.631373,0.788235,0.956863}%
\pgfsetfillcolor{currentfill}%
\pgfsetlinewidth{0.481800pt}%
\definecolor{currentstroke}{rgb}{1.000000,1.000000,1.000000}%
\pgfsetstrokecolor{currentstroke}%
\pgfsetdash{}{0pt}%
\pgfpathmoveto{\pgfqpoint{2.210439in}{2.727334in}}%
\pgfpathcurveto{\pgfqpoint{2.221489in}{2.727334in}}{\pgfqpoint{2.232088in}{2.731725in}}{\pgfqpoint{2.239902in}{2.739538in}}%
\pgfpathcurveto{\pgfqpoint{2.247715in}{2.747352in}}{\pgfqpoint{2.252105in}{2.757951in}}{\pgfqpoint{2.252105in}{2.769001in}}%
\pgfpathcurveto{\pgfqpoint{2.252105in}{2.780051in}}{\pgfqpoint{2.247715in}{2.790650in}}{\pgfqpoint{2.239902in}{2.798464in}}%
\pgfpathcurveto{\pgfqpoint{2.232088in}{2.806277in}}{\pgfqpoint{2.221489in}{2.810668in}}{\pgfqpoint{2.210439in}{2.810668in}}%
\pgfpathcurveto{\pgfqpoint{2.199389in}{2.810668in}}{\pgfqpoint{2.188790in}{2.806277in}}{\pgfqpoint{2.180976in}{2.798464in}}%
\pgfpathcurveto{\pgfqpoint{2.173162in}{2.790650in}}{\pgfqpoint{2.168772in}{2.780051in}}{\pgfqpoint{2.168772in}{2.769001in}}%
\pgfpathcurveto{\pgfqpoint{2.168772in}{2.757951in}}{\pgfqpoint{2.173162in}{2.747352in}}{\pgfqpoint{2.180976in}{2.739538in}}%
\pgfpathcurveto{\pgfqpoint{2.188790in}{2.731725in}}{\pgfqpoint{2.199389in}{2.727334in}}{\pgfqpoint{2.210439in}{2.727334in}}%
\pgfpathclose%
\pgfusepath{stroke,fill}%
\end{pgfscope}%
\begin{pgfscope}%
\pgfpathrectangle{\pgfqpoint{0.481978in}{0.331635in}}{\pgfqpoint{4.960000in}{3.696000in}}%
\pgfusepath{clip}%
\pgfsetbuttcap%
\pgfsetroundjoin%
\definecolor{currentfill}{rgb}{0.631373,0.788235,0.956863}%
\pgfsetfillcolor{currentfill}%
\pgfsetlinewidth{0.481800pt}%
\definecolor{currentstroke}{rgb}{1.000000,1.000000,1.000000}%
\pgfsetstrokecolor{currentstroke}%
\pgfsetdash{}{0pt}%
\pgfpathmoveto{\pgfqpoint{4.510417in}{1.604972in}}%
\pgfpathcurveto{\pgfqpoint{4.521467in}{1.604972in}}{\pgfqpoint{4.532066in}{1.609363in}}{\pgfqpoint{4.539880in}{1.617176in}}%
\pgfpathcurveto{\pgfqpoint{4.547693in}{1.624990in}}{\pgfqpoint{4.552084in}{1.635589in}}{\pgfqpoint{4.552084in}{1.646639in}}%
\pgfpathcurveto{\pgfqpoint{4.552084in}{1.657689in}}{\pgfqpoint{4.547693in}{1.668288in}}{\pgfqpoint{4.539880in}{1.676102in}}%
\pgfpathcurveto{\pgfqpoint{4.532066in}{1.683916in}}{\pgfqpoint{4.521467in}{1.688306in}}{\pgfqpoint{4.510417in}{1.688306in}}%
\pgfpathcurveto{\pgfqpoint{4.499367in}{1.688306in}}{\pgfqpoint{4.488768in}{1.683916in}}{\pgfqpoint{4.480954in}{1.676102in}}%
\pgfpathcurveto{\pgfqpoint{4.473140in}{1.668288in}}{\pgfqpoint{4.468750in}{1.657689in}}{\pgfqpoint{4.468750in}{1.646639in}}%
\pgfpathcurveto{\pgfqpoint{4.468750in}{1.635589in}}{\pgfqpoint{4.473140in}{1.624990in}}{\pgfqpoint{4.480954in}{1.617176in}}%
\pgfpathcurveto{\pgfqpoint{4.488768in}{1.609363in}}{\pgfqpoint{4.499367in}{1.604972in}}{\pgfqpoint{4.510417in}{1.604972in}}%
\pgfpathclose%
\pgfusepath{stroke,fill}%
\end{pgfscope}%
\begin{pgfscope}%
\pgfpathrectangle{\pgfqpoint{0.481978in}{0.331635in}}{\pgfqpoint{4.960000in}{3.696000in}}%
\pgfusepath{clip}%
\pgfsetbuttcap%
\pgfsetroundjoin%
\definecolor{currentfill}{rgb}{0.631373,0.788235,0.956863}%
\pgfsetfillcolor{currentfill}%
\pgfsetlinewidth{0.481800pt}%
\definecolor{currentstroke}{rgb}{1.000000,1.000000,1.000000}%
\pgfsetstrokecolor{currentstroke}%
\pgfsetdash{}{0pt}%
\pgfpathmoveto{\pgfqpoint{1.850846in}{2.550657in}}%
\pgfpathcurveto{\pgfqpoint{1.861896in}{2.550657in}}{\pgfqpoint{1.872495in}{2.555048in}}{\pgfqpoint{1.880309in}{2.562861in}}%
\pgfpathcurveto{\pgfqpoint{1.888122in}{2.570675in}}{\pgfqpoint{1.892513in}{2.581274in}}{\pgfqpoint{1.892513in}{2.592324in}}%
\pgfpathcurveto{\pgfqpoint{1.892513in}{2.603374in}}{\pgfqpoint{1.888122in}{2.613973in}}{\pgfqpoint{1.880309in}{2.621787in}}%
\pgfpathcurveto{\pgfqpoint{1.872495in}{2.629601in}}{\pgfqpoint{1.861896in}{2.633991in}}{\pgfqpoint{1.850846in}{2.633991in}}%
\pgfpathcurveto{\pgfqpoint{1.839796in}{2.633991in}}{\pgfqpoint{1.829197in}{2.629601in}}{\pgfqpoint{1.821383in}{2.621787in}}%
\pgfpathcurveto{\pgfqpoint{1.813570in}{2.613973in}}{\pgfqpoint{1.809179in}{2.603374in}}{\pgfqpoint{1.809179in}{2.592324in}}%
\pgfpathcurveto{\pgfqpoint{1.809179in}{2.581274in}}{\pgfqpoint{1.813570in}{2.570675in}}{\pgfqpoint{1.821383in}{2.562861in}}%
\pgfpathcurveto{\pgfqpoint{1.829197in}{2.555048in}}{\pgfqpoint{1.839796in}{2.550657in}}{\pgfqpoint{1.850846in}{2.550657in}}%
\pgfpathclose%
\pgfusepath{stroke,fill}%
\end{pgfscope}%
\begin{pgfscope}%
\pgfpathrectangle{\pgfqpoint{0.481978in}{0.331635in}}{\pgfqpoint{4.960000in}{3.696000in}}%
\pgfusepath{clip}%
\pgfsetbuttcap%
\pgfsetroundjoin%
\definecolor{currentfill}{rgb}{0.631373,0.788235,0.956863}%
\pgfsetfillcolor{currentfill}%
\pgfsetlinewidth{0.481800pt}%
\definecolor{currentstroke}{rgb}{1.000000,1.000000,1.000000}%
\pgfsetstrokecolor{currentstroke}%
\pgfsetdash{}{0pt}%
\pgfpathmoveto{\pgfqpoint{3.762258in}{1.888141in}}%
\pgfpathcurveto{\pgfqpoint{3.773309in}{1.888141in}}{\pgfqpoint{3.783908in}{1.892532in}}{\pgfqpoint{3.791721in}{1.900345in}}%
\pgfpathcurveto{\pgfqpoint{3.799535in}{1.908159in}}{\pgfqpoint{3.803925in}{1.918758in}}{\pgfqpoint{3.803925in}{1.929808in}}%
\pgfpathcurveto{\pgfqpoint{3.803925in}{1.940858in}}{\pgfqpoint{3.799535in}{1.951457in}}{\pgfqpoint{3.791721in}{1.959271in}}%
\pgfpathcurveto{\pgfqpoint{3.783908in}{1.967084in}}{\pgfqpoint{3.773309in}{1.971475in}}{\pgfqpoint{3.762258in}{1.971475in}}%
\pgfpathcurveto{\pgfqpoint{3.751208in}{1.971475in}}{\pgfqpoint{3.740609in}{1.967084in}}{\pgfqpoint{3.732796in}{1.959271in}}%
\pgfpathcurveto{\pgfqpoint{3.724982in}{1.951457in}}{\pgfqpoint{3.720592in}{1.940858in}}{\pgfqpoint{3.720592in}{1.929808in}}%
\pgfpathcurveto{\pgfqpoint{3.720592in}{1.918758in}}{\pgfqpoint{3.724982in}{1.908159in}}{\pgfqpoint{3.732796in}{1.900345in}}%
\pgfpathcurveto{\pgfqpoint{3.740609in}{1.892532in}}{\pgfqpoint{3.751208in}{1.888141in}}{\pgfqpoint{3.762258in}{1.888141in}}%
\pgfpathclose%
\pgfusepath{stroke,fill}%
\end{pgfscope}%
\begin{pgfscope}%
\pgfpathrectangle{\pgfqpoint{0.481978in}{0.331635in}}{\pgfqpoint{4.960000in}{3.696000in}}%
\pgfusepath{clip}%
\pgfsetbuttcap%
\pgfsetroundjoin%
\definecolor{currentfill}{rgb}{0.631373,0.788235,0.956863}%
\pgfsetfillcolor{currentfill}%
\pgfsetlinewidth{0.481800pt}%
\definecolor{currentstroke}{rgb}{1.000000,1.000000,1.000000}%
\pgfsetstrokecolor{currentstroke}%
\pgfsetdash{}{0pt}%
\pgfpathmoveto{\pgfqpoint{1.702947in}{2.346253in}}%
\pgfpathcurveto{\pgfqpoint{1.713997in}{2.346253in}}{\pgfqpoint{1.724596in}{2.350643in}}{\pgfqpoint{1.732409in}{2.358457in}}%
\pgfpathcurveto{\pgfqpoint{1.740223in}{2.366271in}}{\pgfqpoint{1.744613in}{2.376870in}}{\pgfqpoint{1.744613in}{2.387920in}}%
\pgfpathcurveto{\pgfqpoint{1.744613in}{2.398970in}}{\pgfqpoint{1.740223in}{2.409569in}}{\pgfqpoint{1.732409in}{2.417383in}}%
\pgfpathcurveto{\pgfqpoint{1.724596in}{2.425196in}}{\pgfqpoint{1.713997in}{2.429586in}}{\pgfqpoint{1.702947in}{2.429586in}}%
\pgfpathcurveto{\pgfqpoint{1.691896in}{2.429586in}}{\pgfqpoint{1.681297in}{2.425196in}}{\pgfqpoint{1.673484in}{2.417383in}}%
\pgfpathcurveto{\pgfqpoint{1.665670in}{2.409569in}}{\pgfqpoint{1.661280in}{2.398970in}}{\pgfqpoint{1.661280in}{2.387920in}}%
\pgfpathcurveto{\pgfqpoint{1.661280in}{2.376870in}}{\pgfqpoint{1.665670in}{2.366271in}}{\pgfqpoint{1.673484in}{2.358457in}}%
\pgfpathcurveto{\pgfqpoint{1.681297in}{2.350643in}}{\pgfqpoint{1.691896in}{2.346253in}}{\pgfqpoint{1.702947in}{2.346253in}}%
\pgfpathclose%
\pgfusepath{stroke,fill}%
\end{pgfscope}%
\begin{pgfscope}%
\pgfpathrectangle{\pgfqpoint{0.481978in}{0.331635in}}{\pgfqpoint{4.960000in}{3.696000in}}%
\pgfusepath{clip}%
\pgfsetbuttcap%
\pgfsetroundjoin%
\definecolor{currentfill}{rgb}{0.631373,0.788235,0.956863}%
\pgfsetfillcolor{currentfill}%
\pgfsetlinewidth{0.481800pt}%
\definecolor{currentstroke}{rgb}{1.000000,1.000000,1.000000}%
\pgfsetstrokecolor{currentstroke}%
\pgfsetdash{}{0pt}%
\pgfpathmoveto{\pgfqpoint{1.901158in}{2.321910in}}%
\pgfpathcurveto{\pgfqpoint{1.912208in}{2.321910in}}{\pgfqpoint{1.922807in}{2.326300in}}{\pgfqpoint{1.930620in}{2.334114in}}%
\pgfpathcurveto{\pgfqpoint{1.938434in}{2.341927in}}{\pgfqpoint{1.942824in}{2.352526in}}{\pgfqpoint{1.942824in}{2.363576in}}%
\pgfpathcurveto{\pgfqpoint{1.942824in}{2.374626in}}{\pgfqpoint{1.938434in}{2.385225in}}{\pgfqpoint{1.930620in}{2.393039in}}%
\pgfpathcurveto{\pgfqpoint{1.922807in}{2.400853in}}{\pgfqpoint{1.912208in}{2.405243in}}{\pgfqpoint{1.901158in}{2.405243in}}%
\pgfpathcurveto{\pgfqpoint{1.890107in}{2.405243in}}{\pgfqpoint{1.879508in}{2.400853in}}{\pgfqpoint{1.871695in}{2.393039in}}%
\pgfpathcurveto{\pgfqpoint{1.863881in}{2.385225in}}{\pgfqpoint{1.859491in}{2.374626in}}{\pgfqpoint{1.859491in}{2.363576in}}%
\pgfpathcurveto{\pgfqpoint{1.859491in}{2.352526in}}{\pgfqpoint{1.863881in}{2.341927in}}{\pgfqpoint{1.871695in}{2.334114in}}%
\pgfpathcurveto{\pgfqpoint{1.879508in}{2.326300in}}{\pgfqpoint{1.890107in}{2.321910in}}{\pgfqpoint{1.901158in}{2.321910in}}%
\pgfpathclose%
\pgfusepath{stroke,fill}%
\end{pgfscope}%
\begin{pgfscope}%
\pgfpathrectangle{\pgfqpoint{0.481978in}{0.331635in}}{\pgfqpoint{4.960000in}{3.696000in}}%
\pgfusepath{clip}%
\pgfsetbuttcap%
\pgfsetroundjoin%
\definecolor{currentfill}{rgb}{0.631373,0.788235,0.956863}%
\pgfsetfillcolor{currentfill}%
\pgfsetlinewidth{0.481800pt}%
\definecolor{currentstroke}{rgb}{1.000000,1.000000,1.000000}%
\pgfsetstrokecolor{currentstroke}%
\pgfsetdash{}{0pt}%
\pgfpathmoveto{\pgfqpoint{1.478480in}{3.057004in}}%
\pgfpathcurveto{\pgfqpoint{1.489530in}{3.057004in}}{\pgfqpoint{1.500129in}{3.061394in}}{\pgfqpoint{1.507943in}{3.069207in}}%
\pgfpathcurveto{\pgfqpoint{1.515757in}{3.077021in}}{\pgfqpoint{1.520147in}{3.087620in}}{\pgfqpoint{1.520147in}{3.098670in}}%
\pgfpathcurveto{\pgfqpoint{1.520147in}{3.109720in}}{\pgfqpoint{1.515757in}{3.120319in}}{\pgfqpoint{1.507943in}{3.128133in}}%
\pgfpathcurveto{\pgfqpoint{1.500129in}{3.135947in}}{\pgfqpoint{1.489530in}{3.140337in}}{\pgfqpoint{1.478480in}{3.140337in}}%
\pgfpathcurveto{\pgfqpoint{1.467430in}{3.140337in}}{\pgfqpoint{1.456831in}{3.135947in}}{\pgfqpoint{1.449017in}{3.128133in}}%
\pgfpathcurveto{\pgfqpoint{1.441204in}{3.120319in}}{\pgfqpoint{1.436814in}{3.109720in}}{\pgfqpoint{1.436814in}{3.098670in}}%
\pgfpathcurveto{\pgfqpoint{1.436814in}{3.087620in}}{\pgfqpoint{1.441204in}{3.077021in}}{\pgfqpoint{1.449017in}{3.069207in}}%
\pgfpathcurveto{\pgfqpoint{1.456831in}{3.061394in}}{\pgfqpoint{1.467430in}{3.057004in}}{\pgfqpoint{1.478480in}{3.057004in}}%
\pgfpathclose%
\pgfusepath{stroke,fill}%
\end{pgfscope}%
\begin{pgfscope}%
\pgfpathrectangle{\pgfqpoint{0.481978in}{0.331635in}}{\pgfqpoint{4.960000in}{3.696000in}}%
\pgfusepath{clip}%
\pgfsetbuttcap%
\pgfsetroundjoin%
\definecolor{currentfill}{rgb}{0.631373,0.788235,0.956863}%
\pgfsetfillcolor{currentfill}%
\pgfsetlinewidth{0.481800pt}%
\definecolor{currentstroke}{rgb}{1.000000,1.000000,1.000000}%
\pgfsetstrokecolor{currentstroke}%
\pgfsetdash{}{0pt}%
\pgfpathmoveto{\pgfqpoint{2.422644in}{1.954620in}}%
\pgfpathcurveto{\pgfqpoint{2.433695in}{1.954620in}}{\pgfqpoint{2.444294in}{1.959010in}}{\pgfqpoint{2.452107in}{1.966824in}}%
\pgfpathcurveto{\pgfqpoint{2.459921in}{1.974638in}}{\pgfqpoint{2.464311in}{1.985237in}}{\pgfqpoint{2.464311in}{1.996287in}}%
\pgfpathcurveto{\pgfqpoint{2.464311in}{2.007337in}}{\pgfqpoint{2.459921in}{2.017936in}}{\pgfqpoint{2.452107in}{2.025749in}}%
\pgfpathcurveto{\pgfqpoint{2.444294in}{2.033563in}}{\pgfqpoint{2.433695in}{2.037953in}}{\pgfqpoint{2.422644in}{2.037953in}}%
\pgfpathcurveto{\pgfqpoint{2.411594in}{2.037953in}}{\pgfqpoint{2.400995in}{2.033563in}}{\pgfqpoint{2.393182in}{2.025749in}}%
\pgfpathcurveto{\pgfqpoint{2.385368in}{2.017936in}}{\pgfqpoint{2.380978in}{2.007337in}}{\pgfqpoint{2.380978in}{1.996287in}}%
\pgfpathcurveto{\pgfqpoint{2.380978in}{1.985237in}}{\pgfqpoint{2.385368in}{1.974638in}}{\pgfqpoint{2.393182in}{1.966824in}}%
\pgfpathcurveto{\pgfqpoint{2.400995in}{1.959010in}}{\pgfqpoint{2.411594in}{1.954620in}}{\pgfqpoint{2.422644in}{1.954620in}}%
\pgfpathclose%
\pgfusepath{stroke,fill}%
\end{pgfscope}%
\begin{pgfscope}%
\pgfpathrectangle{\pgfqpoint{0.481978in}{0.331635in}}{\pgfqpoint{4.960000in}{3.696000in}}%
\pgfusepath{clip}%
\pgfsetbuttcap%
\pgfsetroundjoin%
\definecolor{currentfill}{rgb}{0.631373,0.788235,0.956863}%
\pgfsetfillcolor{currentfill}%
\pgfsetlinewidth{0.481800pt}%
\definecolor{currentstroke}{rgb}{1.000000,1.000000,1.000000}%
\pgfsetstrokecolor{currentstroke}%
\pgfsetdash{}{0pt}%
\pgfpathmoveto{\pgfqpoint{2.256485in}{2.109900in}}%
\pgfpathcurveto{\pgfqpoint{2.267535in}{2.109900in}}{\pgfqpoint{2.278134in}{2.114290in}}{\pgfqpoint{2.285948in}{2.122104in}}%
\pgfpathcurveto{\pgfqpoint{2.293761in}{2.129918in}}{\pgfqpoint{2.298151in}{2.140517in}}{\pgfqpoint{2.298151in}{2.151567in}}%
\pgfpathcurveto{\pgfqpoint{2.298151in}{2.162617in}}{\pgfqpoint{2.293761in}{2.173216in}}{\pgfqpoint{2.285948in}{2.181029in}}%
\pgfpathcurveto{\pgfqpoint{2.278134in}{2.188843in}}{\pgfqpoint{2.267535in}{2.193233in}}{\pgfqpoint{2.256485in}{2.193233in}}%
\pgfpathcurveto{\pgfqpoint{2.245435in}{2.193233in}}{\pgfqpoint{2.234836in}{2.188843in}}{\pgfqpoint{2.227022in}{2.181029in}}%
\pgfpathcurveto{\pgfqpoint{2.219208in}{2.173216in}}{\pgfqpoint{2.214818in}{2.162617in}}{\pgfqpoint{2.214818in}{2.151567in}}%
\pgfpathcurveto{\pgfqpoint{2.214818in}{2.140517in}}{\pgfqpoint{2.219208in}{2.129918in}}{\pgfqpoint{2.227022in}{2.122104in}}%
\pgfpathcurveto{\pgfqpoint{2.234836in}{2.114290in}}{\pgfqpoint{2.245435in}{2.109900in}}{\pgfqpoint{2.256485in}{2.109900in}}%
\pgfpathclose%
\pgfusepath{stroke,fill}%
\end{pgfscope}%
\begin{pgfscope}%
\pgfpathrectangle{\pgfqpoint{0.481978in}{0.331635in}}{\pgfqpoint{4.960000in}{3.696000in}}%
\pgfusepath{clip}%
\pgfsetbuttcap%
\pgfsetroundjoin%
\definecolor{currentfill}{rgb}{0.631373,0.788235,0.956863}%
\pgfsetfillcolor{currentfill}%
\pgfsetlinewidth{0.481800pt}%
\definecolor{currentstroke}{rgb}{1.000000,1.000000,1.000000}%
\pgfsetstrokecolor{currentstroke}%
\pgfsetdash{}{0pt}%
\pgfpathmoveto{\pgfqpoint{2.912683in}{2.199737in}}%
\pgfpathcurveto{\pgfqpoint{2.923733in}{2.199737in}}{\pgfqpoint{2.934332in}{2.204127in}}{\pgfqpoint{2.942145in}{2.211941in}}%
\pgfpathcurveto{\pgfqpoint{2.949959in}{2.219754in}}{\pgfqpoint{2.954349in}{2.230353in}}{\pgfqpoint{2.954349in}{2.241403in}}%
\pgfpathcurveto{\pgfqpoint{2.954349in}{2.252453in}}{\pgfqpoint{2.949959in}{2.263053in}}{\pgfqpoint{2.942145in}{2.270866in}}%
\pgfpathcurveto{\pgfqpoint{2.934332in}{2.278680in}}{\pgfqpoint{2.923733in}{2.283070in}}{\pgfqpoint{2.912683in}{2.283070in}}%
\pgfpathcurveto{\pgfqpoint{2.901633in}{2.283070in}}{\pgfqpoint{2.891034in}{2.278680in}}{\pgfqpoint{2.883220in}{2.270866in}}%
\pgfpathcurveto{\pgfqpoint{2.875406in}{2.263053in}}{\pgfqpoint{2.871016in}{2.252453in}}{\pgfqpoint{2.871016in}{2.241403in}}%
\pgfpathcurveto{\pgfqpoint{2.871016in}{2.230353in}}{\pgfqpoint{2.875406in}{2.219754in}}{\pgfqpoint{2.883220in}{2.211941in}}%
\pgfpathcurveto{\pgfqpoint{2.891034in}{2.204127in}}{\pgfqpoint{2.901633in}{2.199737in}}{\pgfqpoint{2.912683in}{2.199737in}}%
\pgfpathclose%
\pgfusepath{stroke,fill}%
\end{pgfscope}%
\begin{pgfscope}%
\pgfpathrectangle{\pgfqpoint{0.481978in}{0.331635in}}{\pgfqpoint{4.960000in}{3.696000in}}%
\pgfusepath{clip}%
\pgfsetbuttcap%
\pgfsetroundjoin%
\definecolor{currentfill}{rgb}{0.631373,0.788235,0.956863}%
\pgfsetfillcolor{currentfill}%
\pgfsetlinewidth{0.481800pt}%
\definecolor{currentstroke}{rgb}{1.000000,1.000000,1.000000}%
\pgfsetstrokecolor{currentstroke}%
\pgfsetdash{}{0pt}%
\pgfpathmoveto{\pgfqpoint{2.056790in}{2.344223in}}%
\pgfpathcurveto{\pgfqpoint{2.067841in}{2.344223in}}{\pgfqpoint{2.078440in}{2.348614in}}{\pgfqpoint{2.086253in}{2.356427in}}%
\pgfpathcurveto{\pgfqpoint{2.094067in}{2.364241in}}{\pgfqpoint{2.098457in}{2.374840in}}{\pgfqpoint{2.098457in}{2.385890in}}%
\pgfpathcurveto{\pgfqpoint{2.098457in}{2.396940in}}{\pgfqpoint{2.094067in}{2.407539in}}{\pgfqpoint{2.086253in}{2.415353in}}%
\pgfpathcurveto{\pgfqpoint{2.078440in}{2.423167in}}{\pgfqpoint{2.067841in}{2.427557in}}{\pgfqpoint{2.056790in}{2.427557in}}%
\pgfpathcurveto{\pgfqpoint{2.045740in}{2.427557in}}{\pgfqpoint{2.035141in}{2.423167in}}{\pgfqpoint{2.027328in}{2.415353in}}%
\pgfpathcurveto{\pgfqpoint{2.019514in}{2.407539in}}{\pgfqpoint{2.015124in}{2.396940in}}{\pgfqpoint{2.015124in}{2.385890in}}%
\pgfpathcurveto{\pgfqpoint{2.015124in}{2.374840in}}{\pgfqpoint{2.019514in}{2.364241in}}{\pgfqpoint{2.027328in}{2.356427in}}%
\pgfpathcurveto{\pgfqpoint{2.035141in}{2.348614in}}{\pgfqpoint{2.045740in}{2.344223in}}{\pgfqpoint{2.056790in}{2.344223in}}%
\pgfpathclose%
\pgfusepath{stroke,fill}%
\end{pgfscope}%
\begin{pgfscope}%
\pgfpathrectangle{\pgfqpoint{0.481978in}{0.331635in}}{\pgfqpoint{4.960000in}{3.696000in}}%
\pgfusepath{clip}%
\pgfsetbuttcap%
\pgfsetroundjoin%
\definecolor{currentfill}{rgb}{0.631373,0.788235,0.956863}%
\pgfsetfillcolor{currentfill}%
\pgfsetlinewidth{0.481800pt}%
\definecolor{currentstroke}{rgb}{1.000000,1.000000,1.000000}%
\pgfsetstrokecolor{currentstroke}%
\pgfsetdash{}{0pt}%
\pgfpathmoveto{\pgfqpoint{1.735980in}{1.068886in}}%
\pgfpathcurveto{\pgfqpoint{1.747030in}{1.068886in}}{\pgfqpoint{1.757630in}{1.073277in}}{\pgfqpoint{1.765443in}{1.081090in}}%
\pgfpathcurveto{\pgfqpoint{1.773257in}{1.088904in}}{\pgfqpoint{1.777647in}{1.099503in}}{\pgfqpoint{1.777647in}{1.110553in}}%
\pgfpathcurveto{\pgfqpoint{1.777647in}{1.121603in}}{\pgfqpoint{1.773257in}{1.132202in}}{\pgfqpoint{1.765443in}{1.140016in}}%
\pgfpathcurveto{\pgfqpoint{1.757630in}{1.147829in}}{\pgfqpoint{1.747030in}{1.152220in}}{\pgfqpoint{1.735980in}{1.152220in}}%
\pgfpathcurveto{\pgfqpoint{1.724930in}{1.152220in}}{\pgfqpoint{1.714331in}{1.147829in}}{\pgfqpoint{1.706518in}{1.140016in}}%
\pgfpathcurveto{\pgfqpoint{1.698704in}{1.132202in}}{\pgfqpoint{1.694314in}{1.121603in}}{\pgfqpoint{1.694314in}{1.110553in}}%
\pgfpathcurveto{\pgfqpoint{1.694314in}{1.099503in}}{\pgfqpoint{1.698704in}{1.088904in}}{\pgfqpoint{1.706518in}{1.081090in}}%
\pgfpathcurveto{\pgfqpoint{1.714331in}{1.073277in}}{\pgfqpoint{1.724930in}{1.068886in}}{\pgfqpoint{1.735980in}{1.068886in}}%
\pgfpathclose%
\pgfusepath{stroke,fill}%
\end{pgfscope}%
\begin{pgfscope}%
\pgfpathrectangle{\pgfqpoint{0.481978in}{0.331635in}}{\pgfqpoint{4.960000in}{3.696000in}}%
\pgfusepath{clip}%
\pgfsetbuttcap%
\pgfsetroundjoin%
\definecolor{currentfill}{rgb}{0.631373,0.788235,0.956863}%
\pgfsetfillcolor{currentfill}%
\pgfsetlinewidth{0.481800pt}%
\definecolor{currentstroke}{rgb}{1.000000,1.000000,1.000000}%
\pgfsetstrokecolor{currentstroke}%
\pgfsetdash{}{0pt}%
\pgfpathmoveto{\pgfqpoint{1.740395in}{2.440636in}}%
\pgfpathcurveto{\pgfqpoint{1.751445in}{2.440636in}}{\pgfqpoint{1.762044in}{2.445026in}}{\pgfqpoint{1.769858in}{2.452840in}}%
\pgfpathcurveto{\pgfqpoint{1.777672in}{2.460653in}}{\pgfqpoint{1.782062in}{2.471252in}}{\pgfqpoint{1.782062in}{2.482302in}}%
\pgfpathcurveto{\pgfqpoint{1.782062in}{2.493353in}}{\pgfqpoint{1.777672in}{2.503952in}}{\pgfqpoint{1.769858in}{2.511765in}}%
\pgfpathcurveto{\pgfqpoint{1.762044in}{2.519579in}}{\pgfqpoint{1.751445in}{2.523969in}}{\pgfqpoint{1.740395in}{2.523969in}}%
\pgfpathcurveto{\pgfqpoint{1.729345in}{2.523969in}}{\pgfqpoint{1.718746in}{2.519579in}}{\pgfqpoint{1.710932in}{2.511765in}}%
\pgfpathcurveto{\pgfqpoint{1.703119in}{2.503952in}}{\pgfqpoint{1.698729in}{2.493353in}}{\pgfqpoint{1.698729in}{2.482302in}}%
\pgfpathcurveto{\pgfqpoint{1.698729in}{2.471252in}}{\pgfqpoint{1.703119in}{2.460653in}}{\pgfqpoint{1.710932in}{2.452840in}}%
\pgfpathcurveto{\pgfqpoint{1.718746in}{2.445026in}}{\pgfqpoint{1.729345in}{2.440636in}}{\pgfqpoint{1.740395in}{2.440636in}}%
\pgfpathclose%
\pgfusepath{stroke,fill}%
\end{pgfscope}%
\begin{pgfscope}%
\pgfpathrectangle{\pgfqpoint{0.481978in}{0.331635in}}{\pgfqpoint{4.960000in}{3.696000in}}%
\pgfusepath{clip}%
\pgfsetbuttcap%
\pgfsetroundjoin%
\definecolor{currentfill}{rgb}{0.631373,0.788235,0.956863}%
\pgfsetfillcolor{currentfill}%
\pgfsetlinewidth{0.481800pt}%
\definecolor{currentstroke}{rgb}{1.000000,1.000000,1.000000}%
\pgfsetstrokecolor{currentstroke}%
\pgfsetdash{}{0pt}%
\pgfpathmoveto{\pgfqpoint{4.146877in}{2.019622in}}%
\pgfpathcurveto{\pgfqpoint{4.157927in}{2.019622in}}{\pgfqpoint{4.168526in}{2.024012in}}{\pgfqpoint{4.176340in}{2.031826in}}%
\pgfpathcurveto{\pgfqpoint{4.184154in}{2.039639in}}{\pgfqpoint{4.188544in}{2.050238in}}{\pgfqpoint{4.188544in}{2.061288in}}%
\pgfpathcurveto{\pgfqpoint{4.188544in}{2.072339in}}{\pgfqpoint{4.184154in}{2.082938in}}{\pgfqpoint{4.176340in}{2.090751in}}%
\pgfpathcurveto{\pgfqpoint{4.168526in}{2.098565in}}{\pgfqpoint{4.157927in}{2.102955in}}{\pgfqpoint{4.146877in}{2.102955in}}%
\pgfpathcurveto{\pgfqpoint{4.135827in}{2.102955in}}{\pgfqpoint{4.125228in}{2.098565in}}{\pgfqpoint{4.117414in}{2.090751in}}%
\pgfpathcurveto{\pgfqpoint{4.109601in}{2.082938in}}{\pgfqpoint{4.105211in}{2.072339in}}{\pgfqpoint{4.105211in}{2.061288in}}%
\pgfpathcurveto{\pgfqpoint{4.105211in}{2.050238in}}{\pgfqpoint{4.109601in}{2.039639in}}{\pgfqpoint{4.117414in}{2.031826in}}%
\pgfpathcurveto{\pgfqpoint{4.125228in}{2.024012in}}{\pgfqpoint{4.135827in}{2.019622in}}{\pgfqpoint{4.146877in}{2.019622in}}%
\pgfpathclose%
\pgfusepath{stroke,fill}%
\end{pgfscope}%
\begin{pgfscope}%
\pgfpathrectangle{\pgfqpoint{0.481978in}{0.331635in}}{\pgfqpoint{4.960000in}{3.696000in}}%
\pgfusepath{clip}%
\pgfsetbuttcap%
\pgfsetroundjoin%
\definecolor{currentfill}{rgb}{0.631373,0.788235,0.956863}%
\pgfsetfillcolor{currentfill}%
\pgfsetlinewidth{0.481800pt}%
\definecolor{currentstroke}{rgb}{1.000000,1.000000,1.000000}%
\pgfsetstrokecolor{currentstroke}%
\pgfsetdash{}{0pt}%
\pgfpathmoveto{\pgfqpoint{2.320750in}{2.305820in}}%
\pgfpathcurveto{\pgfqpoint{2.331800in}{2.305820in}}{\pgfqpoint{2.342400in}{2.310210in}}{\pgfqpoint{2.350213in}{2.318023in}}%
\pgfpathcurveto{\pgfqpoint{2.358027in}{2.325837in}}{\pgfqpoint{2.362417in}{2.336436in}}{\pgfqpoint{2.362417in}{2.347486in}}%
\pgfpathcurveto{\pgfqpoint{2.362417in}{2.358536in}}{\pgfqpoint{2.358027in}{2.369135in}}{\pgfqpoint{2.350213in}{2.376949in}}%
\pgfpathcurveto{\pgfqpoint{2.342400in}{2.384763in}}{\pgfqpoint{2.331800in}{2.389153in}}{\pgfqpoint{2.320750in}{2.389153in}}%
\pgfpathcurveto{\pgfqpoint{2.309700in}{2.389153in}}{\pgfqpoint{2.299101in}{2.384763in}}{\pgfqpoint{2.291288in}{2.376949in}}%
\pgfpathcurveto{\pgfqpoint{2.283474in}{2.369135in}}{\pgfqpoint{2.279084in}{2.358536in}}{\pgfqpoint{2.279084in}{2.347486in}}%
\pgfpathcurveto{\pgfqpoint{2.279084in}{2.336436in}}{\pgfqpoint{2.283474in}{2.325837in}}{\pgfqpoint{2.291288in}{2.318023in}}%
\pgfpathcurveto{\pgfqpoint{2.299101in}{2.310210in}}{\pgfqpoint{2.309700in}{2.305820in}}{\pgfqpoint{2.320750in}{2.305820in}}%
\pgfpathclose%
\pgfusepath{stroke,fill}%
\end{pgfscope}%
\begin{pgfscope}%
\pgfpathrectangle{\pgfqpoint{0.481978in}{0.331635in}}{\pgfqpoint{4.960000in}{3.696000in}}%
\pgfusepath{clip}%
\pgfsetbuttcap%
\pgfsetroundjoin%
\definecolor{currentfill}{rgb}{0.631373,0.788235,0.956863}%
\pgfsetfillcolor{currentfill}%
\pgfsetlinewidth{0.481800pt}%
\definecolor{currentstroke}{rgb}{1.000000,1.000000,1.000000}%
\pgfsetstrokecolor{currentstroke}%
\pgfsetdash{}{0pt}%
\pgfpathmoveto{\pgfqpoint{1.648423in}{1.703724in}}%
\pgfpathcurveto{\pgfqpoint{1.659473in}{1.703724in}}{\pgfqpoint{1.670072in}{1.708114in}}{\pgfqpoint{1.677886in}{1.715927in}}%
\pgfpathcurveto{\pgfqpoint{1.685700in}{1.723741in}}{\pgfqpoint{1.690090in}{1.734340in}}{\pgfqpoint{1.690090in}{1.745390in}}%
\pgfpathcurveto{\pgfqpoint{1.690090in}{1.756440in}}{\pgfqpoint{1.685700in}{1.767039in}}{\pgfqpoint{1.677886in}{1.774853in}}%
\pgfpathcurveto{\pgfqpoint{1.670072in}{1.782667in}}{\pgfqpoint{1.659473in}{1.787057in}}{\pgfqpoint{1.648423in}{1.787057in}}%
\pgfpathcurveto{\pgfqpoint{1.637373in}{1.787057in}}{\pgfqpoint{1.626774in}{1.782667in}}{\pgfqpoint{1.618960in}{1.774853in}}%
\pgfpathcurveto{\pgfqpoint{1.611147in}{1.767039in}}{\pgfqpoint{1.606757in}{1.756440in}}{\pgfqpoint{1.606757in}{1.745390in}}%
\pgfpathcurveto{\pgfqpoint{1.606757in}{1.734340in}}{\pgfqpoint{1.611147in}{1.723741in}}{\pgfqpoint{1.618960in}{1.715927in}}%
\pgfpathcurveto{\pgfqpoint{1.626774in}{1.708114in}}{\pgfqpoint{1.637373in}{1.703724in}}{\pgfqpoint{1.648423in}{1.703724in}}%
\pgfpathclose%
\pgfusepath{stroke,fill}%
\end{pgfscope}%
\begin{pgfscope}%
\pgfpathrectangle{\pgfqpoint{0.481978in}{0.331635in}}{\pgfqpoint{4.960000in}{3.696000in}}%
\pgfusepath{clip}%
\pgfsetbuttcap%
\pgfsetroundjoin%
\definecolor{currentfill}{rgb}{0.631373,0.788235,0.956863}%
\pgfsetfillcolor{currentfill}%
\pgfsetlinewidth{0.481800pt}%
\definecolor{currentstroke}{rgb}{1.000000,1.000000,1.000000}%
\pgfsetstrokecolor{currentstroke}%
\pgfsetdash{}{0pt}%
\pgfpathmoveto{\pgfqpoint{3.265583in}{2.602654in}}%
\pgfpathcurveto{\pgfqpoint{3.276633in}{2.602654in}}{\pgfqpoint{3.287232in}{2.607044in}}{\pgfqpoint{3.295046in}{2.614858in}}%
\pgfpathcurveto{\pgfqpoint{3.302859in}{2.622672in}}{\pgfqpoint{3.307250in}{2.633271in}}{\pgfqpoint{3.307250in}{2.644321in}}%
\pgfpathcurveto{\pgfqpoint{3.307250in}{2.655371in}}{\pgfqpoint{3.302859in}{2.665970in}}{\pgfqpoint{3.295046in}{2.673783in}}%
\pgfpathcurveto{\pgfqpoint{3.287232in}{2.681597in}}{\pgfqpoint{3.276633in}{2.685987in}}{\pgfqpoint{3.265583in}{2.685987in}}%
\pgfpathcurveto{\pgfqpoint{3.254533in}{2.685987in}}{\pgfqpoint{3.243934in}{2.681597in}}{\pgfqpoint{3.236120in}{2.673783in}}%
\pgfpathcurveto{\pgfqpoint{3.228307in}{2.665970in}}{\pgfqpoint{3.223916in}{2.655371in}}{\pgfqpoint{3.223916in}{2.644321in}}%
\pgfpathcurveto{\pgfqpoint{3.223916in}{2.633271in}}{\pgfqpoint{3.228307in}{2.622672in}}{\pgfqpoint{3.236120in}{2.614858in}}%
\pgfpathcurveto{\pgfqpoint{3.243934in}{2.607044in}}{\pgfqpoint{3.254533in}{2.602654in}}{\pgfqpoint{3.265583in}{2.602654in}}%
\pgfpathclose%
\pgfusepath{stroke,fill}%
\end{pgfscope}%
\begin{pgfscope}%
\pgfpathrectangle{\pgfqpoint{0.481978in}{0.331635in}}{\pgfqpoint{4.960000in}{3.696000in}}%
\pgfusepath{clip}%
\pgfsetbuttcap%
\pgfsetroundjoin%
\definecolor{currentfill}{rgb}{0.631373,0.788235,0.956863}%
\pgfsetfillcolor{currentfill}%
\pgfsetlinewidth{0.481800pt}%
\definecolor{currentstroke}{rgb}{1.000000,1.000000,1.000000}%
\pgfsetstrokecolor{currentstroke}%
\pgfsetdash{}{0pt}%
\pgfpathmoveto{\pgfqpoint{3.394690in}{2.425391in}}%
\pgfpathcurveto{\pgfqpoint{3.405740in}{2.425391in}}{\pgfqpoint{3.416339in}{2.429781in}}{\pgfqpoint{3.424153in}{2.437595in}}%
\pgfpathcurveto{\pgfqpoint{3.431966in}{2.445409in}}{\pgfqpoint{3.436356in}{2.456008in}}{\pgfqpoint{3.436356in}{2.467058in}}%
\pgfpathcurveto{\pgfqpoint{3.436356in}{2.478108in}}{\pgfqpoint{3.431966in}{2.488707in}}{\pgfqpoint{3.424153in}{2.496521in}}%
\pgfpathcurveto{\pgfqpoint{3.416339in}{2.504334in}}{\pgfqpoint{3.405740in}{2.508724in}}{\pgfqpoint{3.394690in}{2.508724in}}%
\pgfpathcurveto{\pgfqpoint{3.383640in}{2.508724in}}{\pgfqpoint{3.373041in}{2.504334in}}{\pgfqpoint{3.365227in}{2.496521in}}%
\pgfpathcurveto{\pgfqpoint{3.357413in}{2.488707in}}{\pgfqpoint{3.353023in}{2.478108in}}{\pgfqpoint{3.353023in}{2.467058in}}%
\pgfpathcurveto{\pgfqpoint{3.353023in}{2.456008in}}{\pgfqpoint{3.357413in}{2.445409in}}{\pgfqpoint{3.365227in}{2.437595in}}%
\pgfpathcurveto{\pgfqpoint{3.373041in}{2.429781in}}{\pgfqpoint{3.383640in}{2.425391in}}{\pgfqpoint{3.394690in}{2.425391in}}%
\pgfpathclose%
\pgfusepath{stroke,fill}%
\end{pgfscope}%
\begin{pgfscope}%
\pgfpathrectangle{\pgfqpoint{0.481978in}{0.331635in}}{\pgfqpoint{4.960000in}{3.696000in}}%
\pgfusepath{clip}%
\pgfsetbuttcap%
\pgfsetroundjoin%
\definecolor{currentfill}{rgb}{0.631373,0.788235,0.956863}%
\pgfsetfillcolor{currentfill}%
\pgfsetlinewidth{0.481800pt}%
\definecolor{currentstroke}{rgb}{1.000000,1.000000,1.000000}%
\pgfsetstrokecolor{currentstroke}%
\pgfsetdash{}{0pt}%
\pgfpathmoveto{\pgfqpoint{1.330445in}{2.820863in}}%
\pgfpathcurveto{\pgfqpoint{1.341495in}{2.820863in}}{\pgfqpoint{1.352094in}{2.825254in}}{\pgfqpoint{1.359907in}{2.833067in}}%
\pgfpathcurveto{\pgfqpoint{1.367721in}{2.840881in}}{\pgfqpoint{1.372111in}{2.851480in}}{\pgfqpoint{1.372111in}{2.862530in}}%
\pgfpathcurveto{\pgfqpoint{1.372111in}{2.873580in}}{\pgfqpoint{1.367721in}{2.884179in}}{\pgfqpoint{1.359907in}{2.891993in}}%
\pgfpathcurveto{\pgfqpoint{1.352094in}{2.899807in}}{\pgfqpoint{1.341495in}{2.904197in}}{\pgfqpoint{1.330445in}{2.904197in}}%
\pgfpathcurveto{\pgfqpoint{1.319394in}{2.904197in}}{\pgfqpoint{1.308795in}{2.899807in}}{\pgfqpoint{1.300982in}{2.891993in}}%
\pgfpathcurveto{\pgfqpoint{1.293168in}{2.884179in}}{\pgfqpoint{1.288778in}{2.873580in}}{\pgfqpoint{1.288778in}{2.862530in}}%
\pgfpathcurveto{\pgfqpoint{1.288778in}{2.851480in}}{\pgfqpoint{1.293168in}{2.840881in}}{\pgfqpoint{1.300982in}{2.833067in}}%
\pgfpathcurveto{\pgfqpoint{1.308795in}{2.825254in}}{\pgfqpoint{1.319394in}{2.820863in}}{\pgfqpoint{1.330445in}{2.820863in}}%
\pgfpathclose%
\pgfusepath{stroke,fill}%
\end{pgfscope}%
\begin{pgfscope}%
\pgfpathrectangle{\pgfqpoint{0.481978in}{0.331635in}}{\pgfqpoint{4.960000in}{3.696000in}}%
\pgfusepath{clip}%
\pgfsetbuttcap%
\pgfsetroundjoin%
\definecolor{currentfill}{rgb}{0.631373,0.788235,0.956863}%
\pgfsetfillcolor{currentfill}%
\pgfsetlinewidth{0.481800pt}%
\definecolor{currentstroke}{rgb}{1.000000,1.000000,1.000000}%
\pgfsetstrokecolor{currentstroke}%
\pgfsetdash{}{0pt}%
\pgfpathmoveto{\pgfqpoint{3.086945in}{2.783583in}}%
\pgfpathcurveto{\pgfqpoint{3.097995in}{2.783583in}}{\pgfqpoint{3.108594in}{2.787973in}}{\pgfqpoint{3.116408in}{2.795787in}}%
\pgfpathcurveto{\pgfqpoint{3.124222in}{2.803601in}}{\pgfqpoint{3.128612in}{2.814200in}}{\pgfqpoint{3.128612in}{2.825250in}}%
\pgfpathcurveto{\pgfqpoint{3.128612in}{2.836300in}}{\pgfqpoint{3.124222in}{2.846899in}}{\pgfqpoint{3.116408in}{2.854713in}}%
\pgfpathcurveto{\pgfqpoint{3.108594in}{2.862526in}}{\pgfqpoint{3.097995in}{2.866916in}}{\pgfqpoint{3.086945in}{2.866916in}}%
\pgfpathcurveto{\pgfqpoint{3.075895in}{2.866916in}}{\pgfqpoint{3.065296in}{2.862526in}}{\pgfqpoint{3.057482in}{2.854713in}}%
\pgfpathcurveto{\pgfqpoint{3.049669in}{2.846899in}}{\pgfqpoint{3.045278in}{2.836300in}}{\pgfqpoint{3.045278in}{2.825250in}}%
\pgfpathcurveto{\pgfqpoint{3.045278in}{2.814200in}}{\pgfqpoint{3.049669in}{2.803601in}}{\pgfqpoint{3.057482in}{2.795787in}}%
\pgfpathcurveto{\pgfqpoint{3.065296in}{2.787973in}}{\pgfqpoint{3.075895in}{2.783583in}}{\pgfqpoint{3.086945in}{2.783583in}}%
\pgfpathclose%
\pgfusepath{stroke,fill}%
\end{pgfscope}%
\begin{pgfscope}%
\pgfpathrectangle{\pgfqpoint{0.481978in}{0.331635in}}{\pgfqpoint{4.960000in}{3.696000in}}%
\pgfusepath{clip}%
\pgfsetbuttcap%
\pgfsetroundjoin%
\definecolor{currentfill}{rgb}{0.631373,0.788235,0.956863}%
\pgfsetfillcolor{currentfill}%
\pgfsetlinewidth{0.481800pt}%
\definecolor{currentstroke}{rgb}{1.000000,1.000000,1.000000}%
\pgfsetstrokecolor{currentstroke}%
\pgfsetdash{}{0pt}%
\pgfpathmoveto{\pgfqpoint{2.062589in}{1.019028in}}%
\pgfpathcurveto{\pgfqpoint{2.073639in}{1.019028in}}{\pgfqpoint{2.084238in}{1.023419in}}{\pgfqpoint{2.092052in}{1.031232in}}%
\pgfpathcurveto{\pgfqpoint{2.099865in}{1.039046in}}{\pgfqpoint{2.104256in}{1.049645in}}{\pgfqpoint{2.104256in}{1.060695in}}%
\pgfpathcurveto{\pgfqpoint{2.104256in}{1.071745in}}{\pgfqpoint{2.099865in}{1.082344in}}{\pgfqpoint{2.092052in}{1.090158in}}%
\pgfpathcurveto{\pgfqpoint{2.084238in}{1.097971in}}{\pgfqpoint{2.073639in}{1.102362in}}{\pgfqpoint{2.062589in}{1.102362in}}%
\pgfpathcurveto{\pgfqpoint{2.051539in}{1.102362in}}{\pgfqpoint{2.040940in}{1.097971in}}{\pgfqpoint{2.033126in}{1.090158in}}%
\pgfpathcurveto{\pgfqpoint{2.025312in}{1.082344in}}{\pgfqpoint{2.020922in}{1.071745in}}{\pgfqpoint{2.020922in}{1.060695in}}%
\pgfpathcurveto{\pgfqpoint{2.020922in}{1.049645in}}{\pgfqpoint{2.025312in}{1.039046in}}{\pgfqpoint{2.033126in}{1.031232in}}%
\pgfpathcurveto{\pgfqpoint{2.040940in}{1.023419in}}{\pgfqpoint{2.051539in}{1.019028in}}{\pgfqpoint{2.062589in}{1.019028in}}%
\pgfpathclose%
\pgfusepath{stroke,fill}%
\end{pgfscope}%
\begin{pgfscope}%
\pgfpathrectangle{\pgfqpoint{0.481978in}{0.331635in}}{\pgfqpoint{4.960000in}{3.696000in}}%
\pgfusepath{clip}%
\pgfsetbuttcap%
\pgfsetroundjoin%
\definecolor{currentfill}{rgb}{0.631373,0.788235,0.956863}%
\pgfsetfillcolor{currentfill}%
\pgfsetlinewidth{0.481800pt}%
\definecolor{currentstroke}{rgb}{1.000000,1.000000,1.000000}%
\pgfsetstrokecolor{currentstroke}%
\pgfsetdash{}{0pt}%
\pgfpathmoveto{\pgfqpoint{4.685082in}{1.827434in}}%
\pgfpathcurveto{\pgfqpoint{4.696132in}{1.827434in}}{\pgfqpoint{4.706731in}{1.831825in}}{\pgfqpoint{4.714544in}{1.839638in}}%
\pgfpathcurveto{\pgfqpoint{4.722358in}{1.847452in}}{\pgfqpoint{4.726748in}{1.858051in}}{\pgfqpoint{4.726748in}{1.869101in}}%
\pgfpathcurveto{\pgfqpoint{4.726748in}{1.880151in}}{\pgfqpoint{4.722358in}{1.890750in}}{\pgfqpoint{4.714544in}{1.898564in}}%
\pgfpathcurveto{\pgfqpoint{4.706731in}{1.906377in}}{\pgfqpoint{4.696132in}{1.910768in}}{\pgfqpoint{4.685082in}{1.910768in}}%
\pgfpathcurveto{\pgfqpoint{4.674032in}{1.910768in}}{\pgfqpoint{4.663433in}{1.906377in}}{\pgfqpoint{4.655619in}{1.898564in}}%
\pgfpathcurveto{\pgfqpoint{4.647805in}{1.890750in}}{\pgfqpoint{4.643415in}{1.880151in}}{\pgfqpoint{4.643415in}{1.869101in}}%
\pgfpathcurveto{\pgfqpoint{4.643415in}{1.858051in}}{\pgfqpoint{4.647805in}{1.847452in}}{\pgfqpoint{4.655619in}{1.839638in}}%
\pgfpathcurveto{\pgfqpoint{4.663433in}{1.831825in}}{\pgfqpoint{4.674032in}{1.827434in}}{\pgfqpoint{4.685082in}{1.827434in}}%
\pgfpathclose%
\pgfusepath{stroke,fill}%
\end{pgfscope}%
\begin{pgfscope}%
\pgfpathrectangle{\pgfqpoint{0.481978in}{0.331635in}}{\pgfqpoint{4.960000in}{3.696000in}}%
\pgfusepath{clip}%
\pgfsetbuttcap%
\pgfsetroundjoin%
\definecolor{currentfill}{rgb}{0.631373,0.788235,0.956863}%
\pgfsetfillcolor{currentfill}%
\pgfsetlinewidth{0.481800pt}%
\definecolor{currentstroke}{rgb}{1.000000,1.000000,1.000000}%
\pgfsetstrokecolor{currentstroke}%
\pgfsetdash{}{0pt}%
\pgfpathmoveto{\pgfqpoint{3.302393in}{2.444366in}}%
\pgfpathcurveto{\pgfqpoint{3.313443in}{2.444366in}}{\pgfqpoint{3.324042in}{2.448756in}}{\pgfqpoint{3.331855in}{2.456570in}}%
\pgfpathcurveto{\pgfqpoint{3.339669in}{2.464383in}}{\pgfqpoint{3.344059in}{2.474982in}}{\pgfqpoint{3.344059in}{2.486033in}}%
\pgfpathcurveto{\pgfqpoint{3.344059in}{2.497083in}}{\pgfqpoint{3.339669in}{2.507682in}}{\pgfqpoint{3.331855in}{2.515495in}}%
\pgfpathcurveto{\pgfqpoint{3.324042in}{2.523309in}}{\pgfqpoint{3.313443in}{2.527699in}}{\pgfqpoint{3.302393in}{2.527699in}}%
\pgfpathcurveto{\pgfqpoint{3.291343in}{2.527699in}}{\pgfqpoint{3.280744in}{2.523309in}}{\pgfqpoint{3.272930in}{2.515495in}}%
\pgfpathcurveto{\pgfqpoint{3.265116in}{2.507682in}}{\pgfqpoint{3.260726in}{2.497083in}}{\pgfqpoint{3.260726in}{2.486033in}}%
\pgfpathcurveto{\pgfqpoint{3.260726in}{2.474982in}}{\pgfqpoint{3.265116in}{2.464383in}}{\pgfqpoint{3.272930in}{2.456570in}}%
\pgfpathcurveto{\pgfqpoint{3.280744in}{2.448756in}}{\pgfqpoint{3.291343in}{2.444366in}}{\pgfqpoint{3.302393in}{2.444366in}}%
\pgfpathclose%
\pgfusepath{stroke,fill}%
\end{pgfscope}%
\begin{pgfscope}%
\pgfpathrectangle{\pgfqpoint{0.481978in}{0.331635in}}{\pgfqpoint{4.960000in}{3.696000in}}%
\pgfusepath{clip}%
\pgfsetbuttcap%
\pgfsetroundjoin%
\definecolor{currentfill}{rgb}{0.631373,0.788235,0.956863}%
\pgfsetfillcolor{currentfill}%
\pgfsetlinewidth{0.481800pt}%
\definecolor{currentstroke}{rgb}{1.000000,1.000000,1.000000}%
\pgfsetstrokecolor{currentstroke}%
\pgfsetdash{}{0pt}%
\pgfpathmoveto{\pgfqpoint{1.858759in}{2.638519in}}%
\pgfpathcurveto{\pgfqpoint{1.869809in}{2.638519in}}{\pgfqpoint{1.880408in}{2.642910in}}{\pgfqpoint{1.888221in}{2.650723in}}%
\pgfpathcurveto{\pgfqpoint{1.896035in}{2.658537in}}{\pgfqpoint{1.900425in}{2.669136in}}{\pgfqpoint{1.900425in}{2.680186in}}%
\pgfpathcurveto{\pgfqpoint{1.900425in}{2.691236in}}{\pgfqpoint{1.896035in}{2.701835in}}{\pgfqpoint{1.888221in}{2.709649in}}%
\pgfpathcurveto{\pgfqpoint{1.880408in}{2.717462in}}{\pgfqpoint{1.869809in}{2.721853in}}{\pgfqpoint{1.858759in}{2.721853in}}%
\pgfpathcurveto{\pgfqpoint{1.847708in}{2.721853in}}{\pgfqpoint{1.837109in}{2.717462in}}{\pgfqpoint{1.829296in}{2.709649in}}%
\pgfpathcurveto{\pgfqpoint{1.821482in}{2.701835in}}{\pgfqpoint{1.817092in}{2.691236in}}{\pgfqpoint{1.817092in}{2.680186in}}%
\pgfpathcurveto{\pgfqpoint{1.817092in}{2.669136in}}{\pgfqpoint{1.821482in}{2.658537in}}{\pgfqpoint{1.829296in}{2.650723in}}%
\pgfpathcurveto{\pgfqpoint{1.837109in}{2.642910in}}{\pgfqpoint{1.847708in}{2.638519in}}{\pgfqpoint{1.858759in}{2.638519in}}%
\pgfpathclose%
\pgfusepath{stroke,fill}%
\end{pgfscope}%
\begin{pgfscope}%
\pgfpathrectangle{\pgfqpoint{0.481978in}{0.331635in}}{\pgfqpoint{4.960000in}{3.696000in}}%
\pgfusepath{clip}%
\pgfsetbuttcap%
\pgfsetroundjoin%
\definecolor{currentfill}{rgb}{0.631373,0.788235,0.956863}%
\pgfsetfillcolor{currentfill}%
\pgfsetlinewidth{0.481800pt}%
\definecolor{currentstroke}{rgb}{1.000000,1.000000,1.000000}%
\pgfsetstrokecolor{currentstroke}%
\pgfsetdash{}{0pt}%
\pgfpathmoveto{\pgfqpoint{3.174628in}{2.149293in}}%
\pgfpathcurveto{\pgfqpoint{3.185678in}{2.149293in}}{\pgfqpoint{3.196277in}{2.153683in}}{\pgfqpoint{3.204091in}{2.161497in}}%
\pgfpathcurveto{\pgfqpoint{3.211904in}{2.169310in}}{\pgfqpoint{3.216295in}{2.179910in}}{\pgfqpoint{3.216295in}{2.190960in}}%
\pgfpathcurveto{\pgfqpoint{3.216295in}{2.202010in}}{\pgfqpoint{3.211904in}{2.212609in}}{\pgfqpoint{3.204091in}{2.220422in}}%
\pgfpathcurveto{\pgfqpoint{3.196277in}{2.228236in}}{\pgfqpoint{3.185678in}{2.232626in}}{\pgfqpoint{3.174628in}{2.232626in}}%
\pgfpathcurveto{\pgfqpoint{3.163578in}{2.232626in}}{\pgfqpoint{3.152979in}{2.228236in}}{\pgfqpoint{3.145165in}{2.220422in}}%
\pgfpathcurveto{\pgfqpoint{3.137352in}{2.212609in}}{\pgfqpoint{3.132961in}{2.202010in}}{\pgfqpoint{3.132961in}{2.190960in}}%
\pgfpathcurveto{\pgfqpoint{3.132961in}{2.179910in}}{\pgfqpoint{3.137352in}{2.169310in}}{\pgfqpoint{3.145165in}{2.161497in}}%
\pgfpathcurveto{\pgfqpoint{3.152979in}{2.153683in}}{\pgfqpoint{3.163578in}{2.149293in}}{\pgfqpoint{3.174628in}{2.149293in}}%
\pgfpathclose%
\pgfusepath{stroke,fill}%
\end{pgfscope}%
\begin{pgfscope}%
\pgfpathrectangle{\pgfqpoint{0.481978in}{0.331635in}}{\pgfqpoint{4.960000in}{3.696000in}}%
\pgfusepath{clip}%
\pgfsetbuttcap%
\pgfsetroundjoin%
\definecolor{currentfill}{rgb}{0.631373,0.788235,0.956863}%
\pgfsetfillcolor{currentfill}%
\pgfsetlinewidth{0.481800pt}%
\definecolor{currentstroke}{rgb}{1.000000,1.000000,1.000000}%
\pgfsetstrokecolor{currentstroke}%
\pgfsetdash{}{0pt}%
\pgfpathmoveto{\pgfqpoint{3.835461in}{1.648080in}}%
\pgfpathcurveto{\pgfqpoint{3.846511in}{1.648080in}}{\pgfqpoint{3.857110in}{1.652470in}}{\pgfqpoint{3.864924in}{1.660284in}}%
\pgfpathcurveto{\pgfqpoint{3.872738in}{1.668098in}}{\pgfqpoint{3.877128in}{1.678697in}}{\pgfqpoint{3.877128in}{1.689747in}}%
\pgfpathcurveto{\pgfqpoint{3.877128in}{1.700797in}}{\pgfqpoint{3.872738in}{1.711396in}}{\pgfqpoint{3.864924in}{1.719210in}}%
\pgfpathcurveto{\pgfqpoint{3.857110in}{1.727023in}}{\pgfqpoint{3.846511in}{1.731413in}}{\pgfqpoint{3.835461in}{1.731413in}}%
\pgfpathcurveto{\pgfqpoint{3.824411in}{1.731413in}}{\pgfqpoint{3.813812in}{1.727023in}}{\pgfqpoint{3.805998in}{1.719210in}}%
\pgfpathcurveto{\pgfqpoint{3.798185in}{1.711396in}}{\pgfqpoint{3.793794in}{1.700797in}}{\pgfqpoint{3.793794in}{1.689747in}}%
\pgfpathcurveto{\pgfqpoint{3.793794in}{1.678697in}}{\pgfqpoint{3.798185in}{1.668098in}}{\pgfqpoint{3.805998in}{1.660284in}}%
\pgfpathcurveto{\pgfqpoint{3.813812in}{1.652470in}}{\pgfqpoint{3.824411in}{1.648080in}}{\pgfqpoint{3.835461in}{1.648080in}}%
\pgfpathclose%
\pgfusepath{stroke,fill}%
\end{pgfscope}%
\begin{pgfscope}%
\pgfpathrectangle{\pgfqpoint{0.481978in}{0.331635in}}{\pgfqpoint{4.960000in}{3.696000in}}%
\pgfusepath{clip}%
\pgfsetbuttcap%
\pgfsetroundjoin%
\definecolor{currentfill}{rgb}{0.631373,0.788235,0.956863}%
\pgfsetfillcolor{currentfill}%
\pgfsetlinewidth{0.481800pt}%
\definecolor{currentstroke}{rgb}{1.000000,1.000000,1.000000}%
\pgfsetstrokecolor{currentstroke}%
\pgfsetdash{}{0pt}%
\pgfpathmoveto{\pgfqpoint{1.054043in}{2.906818in}}%
\pgfpathcurveto{\pgfqpoint{1.065093in}{2.906818in}}{\pgfqpoint{1.075692in}{2.911208in}}{\pgfqpoint{1.083506in}{2.919022in}}%
\pgfpathcurveto{\pgfqpoint{1.091319in}{2.926836in}}{\pgfqpoint{1.095709in}{2.937435in}}{\pgfqpoint{1.095709in}{2.948485in}}%
\pgfpathcurveto{\pgfqpoint{1.095709in}{2.959535in}}{\pgfqpoint{1.091319in}{2.970134in}}{\pgfqpoint{1.083506in}{2.977948in}}%
\pgfpathcurveto{\pgfqpoint{1.075692in}{2.985761in}}{\pgfqpoint{1.065093in}{2.990152in}}{\pgfqpoint{1.054043in}{2.990152in}}%
\pgfpathcurveto{\pgfqpoint{1.042993in}{2.990152in}}{\pgfqpoint{1.032394in}{2.985761in}}{\pgfqpoint{1.024580in}{2.977948in}}%
\pgfpathcurveto{\pgfqpoint{1.016766in}{2.970134in}}{\pgfqpoint{1.012376in}{2.959535in}}{\pgfqpoint{1.012376in}{2.948485in}}%
\pgfpathcurveto{\pgfqpoint{1.012376in}{2.937435in}}{\pgfqpoint{1.016766in}{2.926836in}}{\pgfqpoint{1.024580in}{2.919022in}}%
\pgfpathcurveto{\pgfqpoint{1.032394in}{2.911208in}}{\pgfqpoint{1.042993in}{2.906818in}}{\pgfqpoint{1.054043in}{2.906818in}}%
\pgfpathclose%
\pgfusepath{stroke,fill}%
\end{pgfscope}%
\begin{pgfscope}%
\pgfpathrectangle{\pgfqpoint{0.481978in}{0.331635in}}{\pgfqpoint{4.960000in}{3.696000in}}%
\pgfusepath{clip}%
\pgfsetbuttcap%
\pgfsetroundjoin%
\definecolor{currentfill}{rgb}{0.631373,0.788235,0.956863}%
\pgfsetfillcolor{currentfill}%
\pgfsetlinewidth{0.481800pt}%
\definecolor{currentstroke}{rgb}{1.000000,1.000000,1.000000}%
\pgfsetstrokecolor{currentstroke}%
\pgfsetdash{}{0pt}%
\pgfpathmoveto{\pgfqpoint{2.707868in}{2.174624in}}%
\pgfpathcurveto{\pgfqpoint{2.718918in}{2.174624in}}{\pgfqpoint{2.729517in}{2.179014in}}{\pgfqpoint{2.737330in}{2.186828in}}%
\pgfpathcurveto{\pgfqpoint{2.745144in}{2.194641in}}{\pgfqpoint{2.749534in}{2.205240in}}{\pgfqpoint{2.749534in}{2.216291in}}%
\pgfpathcurveto{\pgfqpoint{2.749534in}{2.227341in}}{\pgfqpoint{2.745144in}{2.237940in}}{\pgfqpoint{2.737330in}{2.245753in}}%
\pgfpathcurveto{\pgfqpoint{2.729517in}{2.253567in}}{\pgfqpoint{2.718918in}{2.257957in}}{\pgfqpoint{2.707868in}{2.257957in}}%
\pgfpathcurveto{\pgfqpoint{2.696817in}{2.257957in}}{\pgfqpoint{2.686218in}{2.253567in}}{\pgfqpoint{2.678405in}{2.245753in}}%
\pgfpathcurveto{\pgfqpoint{2.670591in}{2.237940in}}{\pgfqpoint{2.666201in}{2.227341in}}{\pgfqpoint{2.666201in}{2.216291in}}%
\pgfpathcurveto{\pgfqpoint{2.666201in}{2.205240in}}{\pgfqpoint{2.670591in}{2.194641in}}{\pgfqpoint{2.678405in}{2.186828in}}%
\pgfpathcurveto{\pgfqpoint{2.686218in}{2.179014in}}{\pgfqpoint{2.696817in}{2.174624in}}{\pgfqpoint{2.707868in}{2.174624in}}%
\pgfpathclose%
\pgfusepath{stroke,fill}%
\end{pgfscope}%
\begin{pgfscope}%
\pgfpathrectangle{\pgfqpoint{0.481978in}{0.331635in}}{\pgfqpoint{4.960000in}{3.696000in}}%
\pgfusepath{clip}%
\pgfsetbuttcap%
\pgfsetroundjoin%
\definecolor{currentfill}{rgb}{0.631373,0.788235,0.956863}%
\pgfsetfillcolor{currentfill}%
\pgfsetlinewidth{0.481800pt}%
\definecolor{currentstroke}{rgb}{1.000000,1.000000,1.000000}%
\pgfsetstrokecolor{currentstroke}%
\pgfsetdash{}{0pt}%
\pgfpathmoveto{\pgfqpoint{1.569643in}{3.288340in}}%
\pgfpathcurveto{\pgfqpoint{1.580693in}{3.288340in}}{\pgfqpoint{1.591292in}{3.292730in}}{\pgfqpoint{1.599106in}{3.300543in}}%
\pgfpathcurveto{\pgfqpoint{1.606919in}{3.308357in}}{\pgfqpoint{1.611309in}{3.318956in}}{\pgfqpoint{1.611309in}{3.330006in}}%
\pgfpathcurveto{\pgfqpoint{1.611309in}{3.341056in}}{\pgfqpoint{1.606919in}{3.351655in}}{\pgfqpoint{1.599106in}{3.359469in}}%
\pgfpathcurveto{\pgfqpoint{1.591292in}{3.367283in}}{\pgfqpoint{1.580693in}{3.371673in}}{\pgfqpoint{1.569643in}{3.371673in}}%
\pgfpathcurveto{\pgfqpoint{1.558593in}{3.371673in}}{\pgfqpoint{1.547994in}{3.367283in}}{\pgfqpoint{1.540180in}{3.359469in}}%
\pgfpathcurveto{\pgfqpoint{1.532366in}{3.351655in}}{\pgfqpoint{1.527976in}{3.341056in}}{\pgfqpoint{1.527976in}{3.330006in}}%
\pgfpathcurveto{\pgfqpoint{1.527976in}{3.318956in}}{\pgfqpoint{1.532366in}{3.308357in}}{\pgfqpoint{1.540180in}{3.300543in}}%
\pgfpathcurveto{\pgfqpoint{1.547994in}{3.292730in}}{\pgfqpoint{1.558593in}{3.288340in}}{\pgfqpoint{1.569643in}{3.288340in}}%
\pgfpathclose%
\pgfusepath{stroke,fill}%
\end{pgfscope}%
\begin{pgfscope}%
\pgfpathrectangle{\pgfqpoint{0.481978in}{0.331635in}}{\pgfqpoint{4.960000in}{3.696000in}}%
\pgfusepath{clip}%
\pgfsetbuttcap%
\pgfsetroundjoin%
\definecolor{currentfill}{rgb}{0.631373,0.788235,0.956863}%
\pgfsetfillcolor{currentfill}%
\pgfsetlinewidth{0.481800pt}%
\definecolor{currentstroke}{rgb}{1.000000,1.000000,1.000000}%
\pgfsetstrokecolor{currentstroke}%
\pgfsetdash{}{0pt}%
\pgfpathmoveto{\pgfqpoint{3.714123in}{1.126284in}}%
\pgfpathcurveto{\pgfqpoint{3.725173in}{1.126284in}}{\pgfqpoint{3.735772in}{1.130674in}}{\pgfqpoint{3.743585in}{1.138488in}}%
\pgfpathcurveto{\pgfqpoint{3.751399in}{1.146301in}}{\pgfqpoint{3.755789in}{1.156900in}}{\pgfqpoint{3.755789in}{1.167951in}}%
\pgfpathcurveto{\pgfqpoint{3.755789in}{1.179001in}}{\pgfqpoint{3.751399in}{1.189600in}}{\pgfqpoint{3.743585in}{1.197413in}}%
\pgfpathcurveto{\pgfqpoint{3.735772in}{1.205227in}}{\pgfqpoint{3.725173in}{1.209617in}}{\pgfqpoint{3.714123in}{1.209617in}}%
\pgfpathcurveto{\pgfqpoint{3.703072in}{1.209617in}}{\pgfqpoint{3.692473in}{1.205227in}}{\pgfqpoint{3.684660in}{1.197413in}}%
\pgfpathcurveto{\pgfqpoint{3.676846in}{1.189600in}}{\pgfqpoint{3.672456in}{1.179001in}}{\pgfqpoint{3.672456in}{1.167951in}}%
\pgfpathcurveto{\pgfqpoint{3.672456in}{1.156900in}}{\pgfqpoint{3.676846in}{1.146301in}}{\pgfqpoint{3.684660in}{1.138488in}}%
\pgfpathcurveto{\pgfqpoint{3.692473in}{1.130674in}}{\pgfqpoint{3.703072in}{1.126284in}}{\pgfqpoint{3.714123in}{1.126284in}}%
\pgfpathclose%
\pgfusepath{stroke,fill}%
\end{pgfscope}%
\begin{pgfscope}%
\pgfpathrectangle{\pgfqpoint{0.481978in}{0.331635in}}{\pgfqpoint{4.960000in}{3.696000in}}%
\pgfusepath{clip}%
\pgfsetbuttcap%
\pgfsetroundjoin%
\definecolor{currentfill}{rgb}{0.631373,0.788235,0.956863}%
\pgfsetfillcolor{currentfill}%
\pgfsetlinewidth{0.481800pt}%
\definecolor{currentstroke}{rgb}{1.000000,1.000000,1.000000}%
\pgfsetstrokecolor{currentstroke}%
\pgfsetdash{}{0pt}%
\pgfpathmoveto{\pgfqpoint{1.884764in}{2.984917in}}%
\pgfpathcurveto{\pgfqpoint{1.895814in}{2.984917in}}{\pgfqpoint{1.906413in}{2.989307in}}{\pgfqpoint{1.914227in}{2.997121in}}%
\pgfpathcurveto{\pgfqpoint{1.922040in}{3.004935in}}{\pgfqpoint{1.926431in}{3.015534in}}{\pgfqpoint{1.926431in}{3.026584in}}%
\pgfpathcurveto{\pgfqpoint{1.926431in}{3.037634in}}{\pgfqpoint{1.922040in}{3.048233in}}{\pgfqpoint{1.914227in}{3.056047in}}%
\pgfpathcurveto{\pgfqpoint{1.906413in}{3.063860in}}{\pgfqpoint{1.895814in}{3.068251in}}{\pgfqpoint{1.884764in}{3.068251in}}%
\pgfpathcurveto{\pgfqpoint{1.873714in}{3.068251in}}{\pgfqpoint{1.863115in}{3.063860in}}{\pgfqpoint{1.855301in}{3.056047in}}%
\pgfpathcurveto{\pgfqpoint{1.847488in}{3.048233in}}{\pgfqpoint{1.843097in}{3.037634in}}{\pgfqpoint{1.843097in}{3.026584in}}%
\pgfpathcurveto{\pgfqpoint{1.843097in}{3.015534in}}{\pgfqpoint{1.847488in}{3.004935in}}{\pgfqpoint{1.855301in}{2.997121in}}%
\pgfpathcurveto{\pgfqpoint{1.863115in}{2.989307in}}{\pgfqpoint{1.873714in}{2.984917in}}{\pgfqpoint{1.884764in}{2.984917in}}%
\pgfpathclose%
\pgfusepath{stroke,fill}%
\end{pgfscope}%
\begin{pgfscope}%
\pgfpathrectangle{\pgfqpoint{0.481978in}{0.331635in}}{\pgfqpoint{4.960000in}{3.696000in}}%
\pgfusepath{clip}%
\pgfsetbuttcap%
\pgfsetroundjoin%
\definecolor{currentfill}{rgb}{1.000000,0.705882,0.509804}%
\pgfsetfillcolor{currentfill}%
\pgfsetlinewidth{1.003750pt}%
\definecolor{currentstroke}{rgb}{1.000000,0.705882,0.509804}%
\pgfsetstrokecolor{currentstroke}%
\pgfsetdash{}{0pt}%
\pgfsys@defobject{currentmarker}{\pgfqpoint{-0.041667in}{-0.041667in}}{\pgfqpoint{0.041667in}{0.041667in}}{%
\pgfpathmoveto{\pgfqpoint{0.000000in}{-0.041667in}}%
\pgfpathcurveto{\pgfqpoint{0.011050in}{-0.041667in}}{\pgfqpoint{0.021649in}{-0.037276in}}{\pgfqpoint{0.029463in}{-0.029463in}}%
\pgfpathcurveto{\pgfqpoint{0.037276in}{-0.021649in}}{\pgfqpoint{0.041667in}{-0.011050in}}{\pgfqpoint{0.041667in}{0.000000in}}%
\pgfpathcurveto{\pgfqpoint{0.041667in}{0.011050in}}{\pgfqpoint{0.037276in}{0.021649in}}{\pgfqpoint{0.029463in}{0.029463in}}%
\pgfpathcurveto{\pgfqpoint{0.021649in}{0.037276in}}{\pgfqpoint{0.011050in}{0.041667in}}{\pgfqpoint{0.000000in}{0.041667in}}%
\pgfpathcurveto{\pgfqpoint{-0.011050in}{0.041667in}}{\pgfqpoint{-0.021649in}{0.037276in}}{\pgfqpoint{-0.029463in}{0.029463in}}%
\pgfpathcurveto{\pgfqpoint{-0.037276in}{0.021649in}}{\pgfqpoint{-0.041667in}{0.011050in}}{\pgfqpoint{-0.041667in}{0.000000in}}%
\pgfpathcurveto{\pgfqpoint{-0.041667in}{-0.011050in}}{\pgfqpoint{-0.037276in}{-0.021649in}}{\pgfqpoint{-0.029463in}{-0.029463in}}%
\pgfpathcurveto{\pgfqpoint{-0.021649in}{-0.037276in}}{\pgfqpoint{-0.011050in}{-0.041667in}}{\pgfqpoint{0.000000in}{-0.041667in}}%
\pgfpathclose%
\pgfusepath{stroke,fill}%
}%
\end{pgfscope}%
\begin{pgfscope}%
\pgfpathrectangle{\pgfqpoint{0.481978in}{0.331635in}}{\pgfqpoint{4.960000in}{3.696000in}}%
\pgfusepath{clip}%
\pgfsetbuttcap%
\pgfsetroundjoin%
\definecolor{currentfill}{rgb}{0.631373,0.788235,0.956863}%
\pgfsetfillcolor{currentfill}%
\pgfsetlinewidth{1.003750pt}%
\definecolor{currentstroke}{rgb}{0.631373,0.788235,0.956863}%
\pgfsetstrokecolor{currentstroke}%
\pgfsetdash{}{0pt}%
\pgfsys@defobject{currentmarker}{\pgfqpoint{-0.041667in}{-0.041667in}}{\pgfqpoint{0.041667in}{0.041667in}}{%
\pgfpathmoveto{\pgfqpoint{0.000000in}{-0.041667in}}%
\pgfpathcurveto{\pgfqpoint{0.011050in}{-0.041667in}}{\pgfqpoint{0.021649in}{-0.037276in}}{\pgfqpoint{0.029463in}{-0.029463in}}%
\pgfpathcurveto{\pgfqpoint{0.037276in}{-0.021649in}}{\pgfqpoint{0.041667in}{-0.011050in}}{\pgfqpoint{0.041667in}{0.000000in}}%
\pgfpathcurveto{\pgfqpoint{0.041667in}{0.011050in}}{\pgfqpoint{0.037276in}{0.021649in}}{\pgfqpoint{0.029463in}{0.029463in}}%
\pgfpathcurveto{\pgfqpoint{0.021649in}{0.037276in}}{\pgfqpoint{0.011050in}{0.041667in}}{\pgfqpoint{0.000000in}{0.041667in}}%
\pgfpathcurveto{\pgfqpoint{-0.011050in}{0.041667in}}{\pgfqpoint{-0.021649in}{0.037276in}}{\pgfqpoint{-0.029463in}{0.029463in}}%
\pgfpathcurveto{\pgfqpoint{-0.037276in}{0.021649in}}{\pgfqpoint{-0.041667in}{0.011050in}}{\pgfqpoint{-0.041667in}{0.000000in}}%
\pgfpathcurveto{\pgfqpoint{-0.041667in}{-0.011050in}}{\pgfqpoint{-0.037276in}{-0.021649in}}{\pgfqpoint{-0.029463in}{-0.029463in}}%
\pgfpathcurveto{\pgfqpoint{-0.021649in}{-0.037276in}}{\pgfqpoint{-0.011050in}{-0.041667in}}{\pgfqpoint{0.000000in}{-0.041667in}}%
\pgfpathclose%
\pgfusepath{stroke,fill}%
}%
\end{pgfscope}%
\begin{pgfscope}%
\pgfsetbuttcap%
\pgfsetroundjoin%
\definecolor{currentfill}{rgb}{0.000000,0.000000,0.000000}%
\pgfsetfillcolor{currentfill}%
\pgfsetlinewidth{0.803000pt}%
\definecolor{currentstroke}{rgb}{0.000000,0.000000,0.000000}%
\pgfsetstrokecolor{currentstroke}%
\pgfsetdash{}{0pt}%
\pgfsys@defobject{currentmarker}{\pgfqpoint{0.000000in}{-0.048611in}}{\pgfqpoint{0.000000in}{0.000000in}}{%
\pgfpathmoveto{\pgfqpoint{0.000000in}{0.000000in}}%
\pgfpathlineto{\pgfqpoint{0.000000in}{-0.048611in}}%
\pgfusepath{stroke,fill}%
}%
\begin{pgfscope}%
\pgfsys@transformshift{1.098296in}{0.331635in}%
\pgfsys@useobject{currentmarker}{}%
\end{pgfscope}%
\end{pgfscope}%
\begin{pgfscope}%
\definecolor{textcolor}{rgb}{0.000000,0.000000,0.000000}%
\pgfsetstrokecolor{textcolor}%
\pgfsetfillcolor{textcolor}%
\pgftext[x=1.098296in,y=0.234413in,,top]{\color{textcolor}\sffamily\fontsize{10.000000}{12.000000}\selectfont \ensuremath{-}15}%
\end{pgfscope}%
\begin{pgfscope}%
\pgfsetbuttcap%
\pgfsetroundjoin%
\definecolor{currentfill}{rgb}{0.000000,0.000000,0.000000}%
\pgfsetfillcolor{currentfill}%
\pgfsetlinewidth{0.803000pt}%
\definecolor{currentstroke}{rgb}{0.000000,0.000000,0.000000}%
\pgfsetstrokecolor{currentstroke}%
\pgfsetdash{}{0pt}%
\pgfsys@defobject{currentmarker}{\pgfqpoint{0.000000in}{-0.048611in}}{\pgfqpoint{0.000000in}{0.000000in}}{%
\pgfpathmoveto{\pgfqpoint{0.000000in}{0.000000in}}%
\pgfpathlineto{\pgfqpoint{0.000000in}{-0.048611in}}%
\pgfusepath{stroke,fill}%
}%
\begin{pgfscope}%
\pgfsys@transformshift{1.786833in}{0.331635in}%
\pgfsys@useobject{currentmarker}{}%
\end{pgfscope}%
\end{pgfscope}%
\begin{pgfscope}%
\definecolor{textcolor}{rgb}{0.000000,0.000000,0.000000}%
\pgfsetstrokecolor{textcolor}%
\pgfsetfillcolor{textcolor}%
\pgftext[x=1.786833in,y=0.234413in,,top]{\color{textcolor}\sffamily\fontsize{10.000000}{12.000000}\selectfont \ensuremath{-}10}%
\end{pgfscope}%
\begin{pgfscope}%
\pgfsetbuttcap%
\pgfsetroundjoin%
\definecolor{currentfill}{rgb}{0.000000,0.000000,0.000000}%
\pgfsetfillcolor{currentfill}%
\pgfsetlinewidth{0.803000pt}%
\definecolor{currentstroke}{rgb}{0.000000,0.000000,0.000000}%
\pgfsetstrokecolor{currentstroke}%
\pgfsetdash{}{0pt}%
\pgfsys@defobject{currentmarker}{\pgfqpoint{0.000000in}{-0.048611in}}{\pgfqpoint{0.000000in}{0.000000in}}{%
\pgfpathmoveto{\pgfqpoint{0.000000in}{0.000000in}}%
\pgfpathlineto{\pgfqpoint{0.000000in}{-0.048611in}}%
\pgfusepath{stroke,fill}%
}%
\begin{pgfscope}%
\pgfsys@transformshift{2.475370in}{0.331635in}%
\pgfsys@useobject{currentmarker}{}%
\end{pgfscope}%
\end{pgfscope}%
\begin{pgfscope}%
\definecolor{textcolor}{rgb}{0.000000,0.000000,0.000000}%
\pgfsetstrokecolor{textcolor}%
\pgfsetfillcolor{textcolor}%
\pgftext[x=2.475370in,y=0.234413in,,top]{\color{textcolor}\sffamily\fontsize{10.000000}{12.000000}\selectfont \ensuremath{-}5}%
\end{pgfscope}%
\begin{pgfscope}%
\pgfsetbuttcap%
\pgfsetroundjoin%
\definecolor{currentfill}{rgb}{0.000000,0.000000,0.000000}%
\pgfsetfillcolor{currentfill}%
\pgfsetlinewidth{0.803000pt}%
\definecolor{currentstroke}{rgb}{0.000000,0.000000,0.000000}%
\pgfsetstrokecolor{currentstroke}%
\pgfsetdash{}{0pt}%
\pgfsys@defobject{currentmarker}{\pgfqpoint{0.000000in}{-0.048611in}}{\pgfqpoint{0.000000in}{0.000000in}}{%
\pgfpathmoveto{\pgfqpoint{0.000000in}{0.000000in}}%
\pgfpathlineto{\pgfqpoint{0.000000in}{-0.048611in}}%
\pgfusepath{stroke,fill}%
}%
\begin{pgfscope}%
\pgfsys@transformshift{3.163906in}{0.331635in}%
\pgfsys@useobject{currentmarker}{}%
\end{pgfscope}%
\end{pgfscope}%
\begin{pgfscope}%
\definecolor{textcolor}{rgb}{0.000000,0.000000,0.000000}%
\pgfsetstrokecolor{textcolor}%
\pgfsetfillcolor{textcolor}%
\pgftext[x=3.163906in,y=0.234413in,,top]{\color{textcolor}\sffamily\fontsize{10.000000}{12.000000}\selectfont 0}%
\end{pgfscope}%
\begin{pgfscope}%
\pgfsetbuttcap%
\pgfsetroundjoin%
\definecolor{currentfill}{rgb}{0.000000,0.000000,0.000000}%
\pgfsetfillcolor{currentfill}%
\pgfsetlinewidth{0.803000pt}%
\definecolor{currentstroke}{rgb}{0.000000,0.000000,0.000000}%
\pgfsetstrokecolor{currentstroke}%
\pgfsetdash{}{0pt}%
\pgfsys@defobject{currentmarker}{\pgfqpoint{0.000000in}{-0.048611in}}{\pgfqpoint{0.000000in}{0.000000in}}{%
\pgfpathmoveto{\pgfqpoint{0.000000in}{0.000000in}}%
\pgfpathlineto{\pgfqpoint{0.000000in}{-0.048611in}}%
\pgfusepath{stroke,fill}%
}%
\begin{pgfscope}%
\pgfsys@transformshift{3.852443in}{0.331635in}%
\pgfsys@useobject{currentmarker}{}%
\end{pgfscope}%
\end{pgfscope}%
\begin{pgfscope}%
\definecolor{textcolor}{rgb}{0.000000,0.000000,0.000000}%
\pgfsetstrokecolor{textcolor}%
\pgfsetfillcolor{textcolor}%
\pgftext[x=3.852443in,y=0.234413in,,top]{\color{textcolor}\sffamily\fontsize{10.000000}{12.000000}\selectfont 5}%
\end{pgfscope}%
\begin{pgfscope}%
\pgfsetbuttcap%
\pgfsetroundjoin%
\definecolor{currentfill}{rgb}{0.000000,0.000000,0.000000}%
\pgfsetfillcolor{currentfill}%
\pgfsetlinewidth{0.803000pt}%
\definecolor{currentstroke}{rgb}{0.000000,0.000000,0.000000}%
\pgfsetstrokecolor{currentstroke}%
\pgfsetdash{}{0pt}%
\pgfsys@defobject{currentmarker}{\pgfqpoint{0.000000in}{-0.048611in}}{\pgfqpoint{0.000000in}{0.000000in}}{%
\pgfpathmoveto{\pgfqpoint{0.000000in}{0.000000in}}%
\pgfpathlineto{\pgfqpoint{0.000000in}{-0.048611in}}%
\pgfusepath{stroke,fill}%
}%
\begin{pgfscope}%
\pgfsys@transformshift{4.540980in}{0.331635in}%
\pgfsys@useobject{currentmarker}{}%
\end{pgfscope}%
\end{pgfscope}%
\begin{pgfscope}%
\definecolor{textcolor}{rgb}{0.000000,0.000000,0.000000}%
\pgfsetstrokecolor{textcolor}%
\pgfsetfillcolor{textcolor}%
\pgftext[x=4.540980in,y=0.234413in,,top]{\color{textcolor}\sffamily\fontsize{10.000000}{12.000000}\selectfont 10}%
\end{pgfscope}%
\begin{pgfscope}%
\pgfsetbuttcap%
\pgfsetroundjoin%
\definecolor{currentfill}{rgb}{0.000000,0.000000,0.000000}%
\pgfsetfillcolor{currentfill}%
\pgfsetlinewidth{0.803000pt}%
\definecolor{currentstroke}{rgb}{0.000000,0.000000,0.000000}%
\pgfsetstrokecolor{currentstroke}%
\pgfsetdash{}{0pt}%
\pgfsys@defobject{currentmarker}{\pgfqpoint{0.000000in}{-0.048611in}}{\pgfqpoint{0.000000in}{0.000000in}}{%
\pgfpathmoveto{\pgfqpoint{0.000000in}{0.000000in}}%
\pgfpathlineto{\pgfqpoint{0.000000in}{-0.048611in}}%
\pgfusepath{stroke,fill}%
}%
\begin{pgfscope}%
\pgfsys@transformshift{5.229516in}{0.331635in}%
\pgfsys@useobject{currentmarker}{}%
\end{pgfscope}%
\end{pgfscope}%
\begin{pgfscope}%
\definecolor{textcolor}{rgb}{0.000000,0.000000,0.000000}%
\pgfsetstrokecolor{textcolor}%
\pgfsetfillcolor{textcolor}%
\pgftext[x=5.229516in,y=0.234413in,,top]{\color{textcolor}\sffamily\fontsize{10.000000}{12.000000}\selectfont 15}%
\end{pgfscope}%
\begin{pgfscope}%
\pgfsetbuttcap%
\pgfsetroundjoin%
\definecolor{currentfill}{rgb}{0.000000,0.000000,0.000000}%
\pgfsetfillcolor{currentfill}%
\pgfsetlinewidth{0.803000pt}%
\definecolor{currentstroke}{rgb}{0.000000,0.000000,0.000000}%
\pgfsetstrokecolor{currentstroke}%
\pgfsetdash{}{0pt}%
\pgfsys@defobject{currentmarker}{\pgfqpoint{-0.048611in}{0.000000in}}{\pgfqpoint{-0.000000in}{0.000000in}}{%
\pgfpathmoveto{\pgfqpoint{-0.000000in}{0.000000in}}%
\pgfpathlineto{\pgfqpoint{-0.048611in}{0.000000in}}%
\pgfusepath{stroke,fill}%
}%
\begin{pgfscope}%
\pgfsys@transformshift{0.481978in}{0.384508in}%
\pgfsys@useobject{currentmarker}{}%
\end{pgfscope}%
\end{pgfscope}%
\begin{pgfscope}%
\definecolor{textcolor}{rgb}{0.000000,0.000000,0.000000}%
\pgfsetstrokecolor{textcolor}%
\pgfsetfillcolor{textcolor}%
\pgftext[x=0.100000in, y=0.331747in, left, base]{\color{textcolor}\sffamily\fontsize{10.000000}{12.000000}\selectfont \ensuremath{-}15}%
\end{pgfscope}%
\begin{pgfscope}%
\pgfsetbuttcap%
\pgfsetroundjoin%
\definecolor{currentfill}{rgb}{0.000000,0.000000,0.000000}%
\pgfsetfillcolor{currentfill}%
\pgfsetlinewidth{0.803000pt}%
\definecolor{currentstroke}{rgb}{0.000000,0.000000,0.000000}%
\pgfsetstrokecolor{currentstroke}%
\pgfsetdash{}{0pt}%
\pgfsys@defobject{currentmarker}{\pgfqpoint{-0.048611in}{0.000000in}}{\pgfqpoint{-0.000000in}{0.000000in}}{%
\pgfpathmoveto{\pgfqpoint{-0.000000in}{0.000000in}}%
\pgfpathlineto{\pgfqpoint{-0.048611in}{0.000000in}}%
\pgfusepath{stroke,fill}%
}%
\begin{pgfscope}%
\pgfsys@transformshift{0.481978in}{1.018143in}%
\pgfsys@useobject{currentmarker}{}%
\end{pgfscope}%
\end{pgfscope}%
\begin{pgfscope}%
\definecolor{textcolor}{rgb}{0.000000,0.000000,0.000000}%
\pgfsetstrokecolor{textcolor}%
\pgfsetfillcolor{textcolor}%
\pgftext[x=0.100000in, y=0.965381in, left, base]{\color{textcolor}\sffamily\fontsize{10.000000}{12.000000}\selectfont \ensuremath{-}10}%
\end{pgfscope}%
\begin{pgfscope}%
\pgfsetbuttcap%
\pgfsetroundjoin%
\definecolor{currentfill}{rgb}{0.000000,0.000000,0.000000}%
\pgfsetfillcolor{currentfill}%
\pgfsetlinewidth{0.803000pt}%
\definecolor{currentstroke}{rgb}{0.000000,0.000000,0.000000}%
\pgfsetstrokecolor{currentstroke}%
\pgfsetdash{}{0pt}%
\pgfsys@defobject{currentmarker}{\pgfqpoint{-0.048611in}{0.000000in}}{\pgfqpoint{-0.000000in}{0.000000in}}{%
\pgfpathmoveto{\pgfqpoint{-0.000000in}{0.000000in}}%
\pgfpathlineto{\pgfqpoint{-0.048611in}{0.000000in}}%
\pgfusepath{stroke,fill}%
}%
\begin{pgfscope}%
\pgfsys@transformshift{0.481978in}{1.651777in}%
\pgfsys@useobject{currentmarker}{}%
\end{pgfscope}%
\end{pgfscope}%
\begin{pgfscope}%
\definecolor{textcolor}{rgb}{0.000000,0.000000,0.000000}%
\pgfsetstrokecolor{textcolor}%
\pgfsetfillcolor{textcolor}%
\pgftext[x=0.188365in, y=1.599016in, left, base]{\color{textcolor}\sffamily\fontsize{10.000000}{12.000000}\selectfont \ensuremath{-}5}%
\end{pgfscope}%
\begin{pgfscope}%
\pgfsetbuttcap%
\pgfsetroundjoin%
\definecolor{currentfill}{rgb}{0.000000,0.000000,0.000000}%
\pgfsetfillcolor{currentfill}%
\pgfsetlinewidth{0.803000pt}%
\definecolor{currentstroke}{rgb}{0.000000,0.000000,0.000000}%
\pgfsetstrokecolor{currentstroke}%
\pgfsetdash{}{0pt}%
\pgfsys@defobject{currentmarker}{\pgfqpoint{-0.048611in}{0.000000in}}{\pgfqpoint{-0.000000in}{0.000000in}}{%
\pgfpathmoveto{\pgfqpoint{-0.000000in}{0.000000in}}%
\pgfpathlineto{\pgfqpoint{-0.048611in}{0.000000in}}%
\pgfusepath{stroke,fill}%
}%
\begin{pgfscope}%
\pgfsys@transformshift{0.481978in}{2.285412in}%
\pgfsys@useobject{currentmarker}{}%
\end{pgfscope}%
\end{pgfscope}%
\begin{pgfscope}%
\definecolor{textcolor}{rgb}{0.000000,0.000000,0.000000}%
\pgfsetstrokecolor{textcolor}%
\pgfsetfillcolor{textcolor}%
\pgftext[x=0.296390in, y=2.232650in, left, base]{\color{textcolor}\sffamily\fontsize{10.000000}{12.000000}\selectfont 0}%
\end{pgfscope}%
\begin{pgfscope}%
\pgfsetbuttcap%
\pgfsetroundjoin%
\definecolor{currentfill}{rgb}{0.000000,0.000000,0.000000}%
\pgfsetfillcolor{currentfill}%
\pgfsetlinewidth{0.803000pt}%
\definecolor{currentstroke}{rgb}{0.000000,0.000000,0.000000}%
\pgfsetstrokecolor{currentstroke}%
\pgfsetdash{}{0pt}%
\pgfsys@defobject{currentmarker}{\pgfqpoint{-0.048611in}{0.000000in}}{\pgfqpoint{-0.000000in}{0.000000in}}{%
\pgfpathmoveto{\pgfqpoint{-0.000000in}{0.000000in}}%
\pgfpathlineto{\pgfqpoint{-0.048611in}{0.000000in}}%
\pgfusepath{stroke,fill}%
}%
\begin{pgfscope}%
\pgfsys@transformshift{0.481978in}{2.919046in}%
\pgfsys@useobject{currentmarker}{}%
\end{pgfscope}%
\end{pgfscope}%
\begin{pgfscope}%
\definecolor{textcolor}{rgb}{0.000000,0.000000,0.000000}%
\pgfsetstrokecolor{textcolor}%
\pgfsetfillcolor{textcolor}%
\pgftext[x=0.296390in, y=2.866285in, left, base]{\color{textcolor}\sffamily\fontsize{10.000000}{12.000000}\selectfont 5}%
\end{pgfscope}%
\begin{pgfscope}%
\pgfsetbuttcap%
\pgfsetroundjoin%
\definecolor{currentfill}{rgb}{0.000000,0.000000,0.000000}%
\pgfsetfillcolor{currentfill}%
\pgfsetlinewidth{0.803000pt}%
\definecolor{currentstroke}{rgb}{0.000000,0.000000,0.000000}%
\pgfsetstrokecolor{currentstroke}%
\pgfsetdash{}{0pt}%
\pgfsys@defobject{currentmarker}{\pgfqpoint{-0.048611in}{0.000000in}}{\pgfqpoint{-0.000000in}{0.000000in}}{%
\pgfpathmoveto{\pgfqpoint{-0.000000in}{0.000000in}}%
\pgfpathlineto{\pgfqpoint{-0.048611in}{0.000000in}}%
\pgfusepath{stroke,fill}%
}%
\begin{pgfscope}%
\pgfsys@transformshift{0.481978in}{3.552681in}%
\pgfsys@useobject{currentmarker}{}%
\end{pgfscope}%
\end{pgfscope}%
\begin{pgfscope}%
\definecolor{textcolor}{rgb}{0.000000,0.000000,0.000000}%
\pgfsetstrokecolor{textcolor}%
\pgfsetfillcolor{textcolor}%
\pgftext[x=0.208025in, y=3.499919in, left, base]{\color{textcolor}\sffamily\fontsize{10.000000}{12.000000}\selectfont 10}%
\end{pgfscope}%
\begin{pgfscope}%
\pgfpathrectangle{\pgfqpoint{0.481978in}{0.331635in}}{\pgfqpoint{4.960000in}{3.696000in}}%
\pgfusepath{clip}%
\pgfsetrectcap%
\pgfsetroundjoin%
\pgfsetlinewidth{1.505625pt}%
\definecolor{currentstroke}{rgb}{1.000000,0.705882,0.509804}%
\pgfsetstrokecolor{currentstroke}%
\pgfsetstrokeopacity{0.800000}%
\pgfsetdash{}{0pt}%
\pgfpathmoveto{\pgfqpoint{4.627391in}{2.835809in}}%
\pgfpathlineto{\pgfqpoint{2.899136in}{2.154863in}}%
\pgfusepath{stroke}%
\end{pgfscope}%
\begin{pgfscope}%
\pgfpathrectangle{\pgfqpoint{0.481978in}{0.331635in}}{\pgfqpoint{4.960000in}{3.696000in}}%
\pgfusepath{clip}%
\pgfsetrectcap%
\pgfsetroundjoin%
\pgfsetlinewidth{1.505625pt}%
\definecolor{currentstroke}{rgb}{1.000000,0.705882,0.509804}%
\pgfsetstrokecolor{currentstroke}%
\pgfsetstrokeopacity{0.800000}%
\pgfsetdash{}{0pt}%
\pgfpathmoveto{\pgfqpoint{2.475618in}{0.599507in}}%
\pgfpathlineto{\pgfqpoint{2.899136in}{2.154863in}}%
\pgfusepath{stroke}%
\end{pgfscope}%
\begin{pgfscope}%
\pgfpathrectangle{\pgfqpoint{0.481978in}{0.331635in}}{\pgfqpoint{4.960000in}{3.696000in}}%
\pgfusepath{clip}%
\pgfsetrectcap%
\pgfsetroundjoin%
\pgfsetlinewidth{1.505625pt}%
\definecolor{currentstroke}{rgb}{1.000000,0.705882,0.509804}%
\pgfsetstrokecolor{currentstroke}%
\pgfsetstrokeopacity{0.800000}%
\pgfsetdash{}{0pt}%
\pgfpathmoveto{\pgfqpoint{2.464482in}{0.605186in}}%
\pgfpathlineto{\pgfqpoint{2.899136in}{2.154863in}}%
\pgfusepath{stroke}%
\end{pgfscope}%
\begin{pgfscope}%
\pgfpathrectangle{\pgfqpoint{0.481978in}{0.331635in}}{\pgfqpoint{4.960000in}{3.696000in}}%
\pgfusepath{clip}%
\pgfsetrectcap%
\pgfsetroundjoin%
\pgfsetlinewidth{1.505625pt}%
\definecolor{currentstroke}{rgb}{1.000000,0.705882,0.509804}%
\pgfsetstrokecolor{currentstroke}%
\pgfsetstrokeopacity{0.800000}%
\pgfsetdash{}{0pt}%
\pgfpathmoveto{\pgfqpoint{1.079617in}{2.396117in}}%
\pgfpathlineto{\pgfqpoint{2.899136in}{2.154863in}}%
\pgfusepath{stroke}%
\end{pgfscope}%
\begin{pgfscope}%
\pgfpathrectangle{\pgfqpoint{0.481978in}{0.331635in}}{\pgfqpoint{4.960000in}{3.696000in}}%
\pgfusepath{clip}%
\pgfsetrectcap%
\pgfsetroundjoin%
\pgfsetlinewidth{1.505625pt}%
\definecolor{currentstroke}{rgb}{1.000000,0.705882,0.509804}%
\pgfsetstrokecolor{currentstroke}%
\pgfsetstrokeopacity{0.800000}%
\pgfsetdash{}{0pt}%
\pgfpathmoveto{\pgfqpoint{3.029515in}{1.864807in}}%
\pgfpathlineto{\pgfqpoint{2.899136in}{2.154863in}}%
\pgfusepath{stroke}%
\end{pgfscope}%
\begin{pgfscope}%
\pgfpathrectangle{\pgfqpoint{0.481978in}{0.331635in}}{\pgfqpoint{4.960000in}{3.696000in}}%
\pgfusepath{clip}%
\pgfsetrectcap%
\pgfsetroundjoin%
\pgfsetlinewidth{1.505625pt}%
\definecolor{currentstroke}{rgb}{1.000000,0.705882,0.509804}%
\pgfsetstrokecolor{currentstroke}%
\pgfsetstrokeopacity{0.800000}%
\pgfsetdash{}{0pt}%
\pgfpathmoveto{\pgfqpoint{2.736847in}{1.884090in}}%
\pgfpathlineto{\pgfqpoint{2.899136in}{2.154863in}}%
\pgfusepath{stroke}%
\end{pgfscope}%
\begin{pgfscope}%
\pgfpathrectangle{\pgfqpoint{0.481978in}{0.331635in}}{\pgfqpoint{4.960000in}{3.696000in}}%
\pgfusepath{clip}%
\pgfsetrectcap%
\pgfsetroundjoin%
\pgfsetlinewidth{1.505625pt}%
\definecolor{currentstroke}{rgb}{1.000000,0.705882,0.509804}%
\pgfsetstrokecolor{currentstroke}%
\pgfsetstrokeopacity{0.800000}%
\pgfsetdash{}{0pt}%
\pgfpathmoveto{\pgfqpoint{2.064108in}{1.567100in}}%
\pgfpathlineto{\pgfqpoint{2.899136in}{2.154863in}}%
\pgfusepath{stroke}%
\end{pgfscope}%
\begin{pgfscope}%
\pgfpathrectangle{\pgfqpoint{0.481978in}{0.331635in}}{\pgfqpoint{4.960000in}{3.696000in}}%
\pgfusepath{clip}%
\pgfsetrectcap%
\pgfsetroundjoin%
\pgfsetlinewidth{1.505625pt}%
\definecolor{currentstroke}{rgb}{1.000000,0.705882,0.509804}%
\pgfsetstrokecolor{currentstroke}%
\pgfsetstrokeopacity{0.800000}%
\pgfsetdash{}{0pt}%
\pgfpathmoveto{\pgfqpoint{2.838937in}{3.725262in}}%
\pgfpathlineto{\pgfqpoint{2.899136in}{2.154863in}}%
\pgfusepath{stroke}%
\end{pgfscope}%
\begin{pgfscope}%
\pgfpathrectangle{\pgfqpoint{0.481978in}{0.331635in}}{\pgfqpoint{4.960000in}{3.696000in}}%
\pgfusepath{clip}%
\pgfsetrectcap%
\pgfsetroundjoin%
\pgfsetlinewidth{1.505625pt}%
\definecolor{currentstroke}{rgb}{1.000000,0.705882,0.509804}%
\pgfsetstrokecolor{currentstroke}%
\pgfsetstrokeopacity{0.800000}%
\pgfsetdash{}{0pt}%
\pgfpathmoveto{\pgfqpoint{2.585551in}{1.719267in}}%
\pgfpathlineto{\pgfqpoint{2.899136in}{2.154863in}}%
\pgfusepath{stroke}%
\end{pgfscope}%
\begin{pgfscope}%
\pgfpathrectangle{\pgfqpoint{0.481978in}{0.331635in}}{\pgfqpoint{4.960000in}{3.696000in}}%
\pgfusepath{clip}%
\pgfsetrectcap%
\pgfsetroundjoin%
\pgfsetlinewidth{1.505625pt}%
\definecolor{currentstroke}{rgb}{1.000000,0.705882,0.509804}%
\pgfsetstrokecolor{currentstroke}%
\pgfsetstrokeopacity{0.800000}%
\pgfsetdash{}{0pt}%
\pgfpathmoveto{\pgfqpoint{1.240353in}{1.779313in}}%
\pgfpathlineto{\pgfqpoint{2.899136in}{2.154863in}}%
\pgfusepath{stroke}%
\end{pgfscope}%
\begin{pgfscope}%
\pgfpathrectangle{\pgfqpoint{0.481978in}{0.331635in}}{\pgfqpoint{4.960000in}{3.696000in}}%
\pgfusepath{clip}%
\pgfsetrectcap%
\pgfsetroundjoin%
\pgfsetlinewidth{1.505625pt}%
\definecolor{currentstroke}{rgb}{1.000000,0.705882,0.509804}%
\pgfsetstrokecolor{currentstroke}%
\pgfsetstrokeopacity{0.800000}%
\pgfsetdash{}{0pt}%
\pgfpathmoveto{\pgfqpoint{3.982955in}{0.990167in}}%
\pgfpathlineto{\pgfqpoint{2.899136in}{2.154863in}}%
\pgfusepath{stroke}%
\end{pgfscope}%
\begin{pgfscope}%
\pgfpathrectangle{\pgfqpoint{0.481978in}{0.331635in}}{\pgfqpoint{4.960000in}{3.696000in}}%
\pgfusepath{clip}%
\pgfsetrectcap%
\pgfsetroundjoin%
\pgfsetlinewidth{1.505625pt}%
\definecolor{currentstroke}{rgb}{1.000000,0.705882,0.509804}%
\pgfsetstrokecolor{currentstroke}%
\pgfsetstrokeopacity{0.800000}%
\pgfsetdash{}{0pt}%
\pgfpathmoveto{\pgfqpoint{3.988249in}{2.946816in}}%
\pgfpathlineto{\pgfqpoint{2.899136in}{2.154863in}}%
\pgfusepath{stroke}%
\end{pgfscope}%
\begin{pgfscope}%
\pgfpathrectangle{\pgfqpoint{0.481978in}{0.331635in}}{\pgfqpoint{4.960000in}{3.696000in}}%
\pgfusepath{clip}%
\pgfsetrectcap%
\pgfsetroundjoin%
\pgfsetlinewidth{1.505625pt}%
\definecolor{currentstroke}{rgb}{1.000000,0.705882,0.509804}%
\pgfsetstrokecolor{currentstroke}%
\pgfsetstrokeopacity{0.800000}%
\pgfsetdash{}{0pt}%
\pgfpathmoveto{\pgfqpoint{3.185681in}{2.989542in}}%
\pgfpathlineto{\pgfqpoint{2.899136in}{2.154863in}}%
\pgfusepath{stroke}%
\end{pgfscope}%
\begin{pgfscope}%
\pgfpathrectangle{\pgfqpoint{0.481978in}{0.331635in}}{\pgfqpoint{4.960000in}{3.696000in}}%
\pgfusepath{clip}%
\pgfsetrectcap%
\pgfsetroundjoin%
\pgfsetlinewidth{1.505625pt}%
\definecolor{currentstroke}{rgb}{1.000000,0.705882,0.509804}%
\pgfsetstrokecolor{currentstroke}%
\pgfsetstrokeopacity{0.800000}%
\pgfsetdash{}{0pt}%
\pgfpathmoveto{\pgfqpoint{0.996356in}{2.213080in}}%
\pgfpathlineto{\pgfqpoint{2.899136in}{2.154863in}}%
\pgfusepath{stroke}%
\end{pgfscope}%
\begin{pgfscope}%
\pgfpathrectangle{\pgfqpoint{0.481978in}{0.331635in}}{\pgfqpoint{4.960000in}{3.696000in}}%
\pgfusepath{clip}%
\pgfsetrectcap%
\pgfsetroundjoin%
\pgfsetlinewidth{1.505625pt}%
\definecolor{currentstroke}{rgb}{1.000000,0.705882,0.509804}%
\pgfsetstrokecolor{currentstroke}%
\pgfsetstrokeopacity{0.800000}%
\pgfsetdash{}{0pt}%
\pgfpathmoveto{\pgfqpoint{1.152793in}{2.630813in}}%
\pgfpathlineto{\pgfqpoint{2.899136in}{2.154863in}}%
\pgfusepath{stroke}%
\end{pgfscope}%
\begin{pgfscope}%
\pgfpathrectangle{\pgfqpoint{0.481978in}{0.331635in}}{\pgfqpoint{4.960000in}{3.696000in}}%
\pgfusepath{clip}%
\pgfsetrectcap%
\pgfsetroundjoin%
\pgfsetlinewidth{1.505625pt}%
\definecolor{currentstroke}{rgb}{1.000000,0.705882,0.509804}%
\pgfsetstrokecolor{currentstroke}%
\pgfsetstrokeopacity{0.800000}%
\pgfsetdash{}{0pt}%
\pgfpathmoveto{\pgfqpoint{3.244734in}{3.410173in}}%
\pgfpathlineto{\pgfqpoint{2.899136in}{2.154863in}}%
\pgfusepath{stroke}%
\end{pgfscope}%
\begin{pgfscope}%
\pgfpathrectangle{\pgfqpoint{0.481978in}{0.331635in}}{\pgfqpoint{4.960000in}{3.696000in}}%
\pgfusepath{clip}%
\pgfsetrectcap%
\pgfsetroundjoin%
\pgfsetlinewidth{1.505625pt}%
\definecolor{currentstroke}{rgb}{1.000000,0.705882,0.509804}%
\pgfsetstrokecolor{currentstroke}%
\pgfsetstrokeopacity{0.800000}%
\pgfsetdash{}{0pt}%
\pgfpathmoveto{\pgfqpoint{2.824609in}{0.620486in}}%
\pgfpathlineto{\pgfqpoint{2.899136in}{2.154863in}}%
\pgfusepath{stroke}%
\end{pgfscope}%
\begin{pgfscope}%
\pgfpathrectangle{\pgfqpoint{0.481978in}{0.331635in}}{\pgfqpoint{4.960000in}{3.696000in}}%
\pgfusepath{clip}%
\pgfsetrectcap%
\pgfsetroundjoin%
\pgfsetlinewidth{1.505625pt}%
\definecolor{currentstroke}{rgb}{1.000000,0.705882,0.509804}%
\pgfsetstrokecolor{currentstroke}%
\pgfsetstrokeopacity{0.800000}%
\pgfsetdash{}{0pt}%
\pgfpathmoveto{\pgfqpoint{3.965318in}{2.588383in}}%
\pgfpathlineto{\pgfqpoint{2.899136in}{2.154863in}}%
\pgfusepath{stroke}%
\end{pgfscope}%
\begin{pgfscope}%
\pgfpathrectangle{\pgfqpoint{0.481978in}{0.331635in}}{\pgfqpoint{4.960000in}{3.696000in}}%
\pgfusepath{clip}%
\pgfsetrectcap%
\pgfsetroundjoin%
\pgfsetlinewidth{1.505625pt}%
\definecolor{currentstroke}{rgb}{1.000000,0.705882,0.509804}%
\pgfsetstrokecolor{currentstroke}%
\pgfsetstrokeopacity{0.800000}%
\pgfsetdash{}{0pt}%
\pgfpathmoveto{\pgfqpoint{3.018864in}{3.111415in}}%
\pgfpathlineto{\pgfqpoint{2.899136in}{2.154863in}}%
\pgfusepath{stroke}%
\end{pgfscope}%
\begin{pgfscope}%
\pgfpathrectangle{\pgfqpoint{0.481978in}{0.331635in}}{\pgfqpoint{4.960000in}{3.696000in}}%
\pgfusepath{clip}%
\pgfsetrectcap%
\pgfsetroundjoin%
\pgfsetlinewidth{1.505625pt}%
\definecolor{currentstroke}{rgb}{1.000000,0.705882,0.509804}%
\pgfsetstrokecolor{currentstroke}%
\pgfsetstrokeopacity{0.800000}%
\pgfsetdash{}{0pt}%
\pgfpathmoveto{\pgfqpoint{1.498715in}{1.523707in}}%
\pgfpathlineto{\pgfqpoint{2.899136in}{2.154863in}}%
\pgfusepath{stroke}%
\end{pgfscope}%
\begin{pgfscope}%
\pgfpathrectangle{\pgfqpoint{0.481978in}{0.331635in}}{\pgfqpoint{4.960000in}{3.696000in}}%
\pgfusepath{clip}%
\pgfsetrectcap%
\pgfsetroundjoin%
\pgfsetlinewidth{1.505625pt}%
\definecolor{currentstroke}{rgb}{1.000000,0.705882,0.509804}%
\pgfsetstrokecolor{currentstroke}%
\pgfsetstrokeopacity{0.800000}%
\pgfsetdash{}{0pt}%
\pgfpathmoveto{\pgfqpoint{2.639538in}{2.328507in}}%
\pgfpathlineto{\pgfqpoint{2.899136in}{2.154863in}}%
\pgfusepath{stroke}%
\end{pgfscope}%
\begin{pgfscope}%
\pgfpathrectangle{\pgfqpoint{0.481978in}{0.331635in}}{\pgfqpoint{4.960000in}{3.696000in}}%
\pgfusepath{clip}%
\pgfsetrectcap%
\pgfsetroundjoin%
\pgfsetlinewidth{1.505625pt}%
\definecolor{currentstroke}{rgb}{1.000000,0.705882,0.509804}%
\pgfsetstrokecolor{currentstroke}%
\pgfsetstrokeopacity{0.800000}%
\pgfsetdash{}{0pt}%
\pgfpathmoveto{\pgfqpoint{2.371027in}{2.094084in}}%
\pgfpathlineto{\pgfqpoint{2.899136in}{2.154863in}}%
\pgfusepath{stroke}%
\end{pgfscope}%
\begin{pgfscope}%
\pgfpathrectangle{\pgfqpoint{0.481978in}{0.331635in}}{\pgfqpoint{4.960000in}{3.696000in}}%
\pgfusepath{clip}%
\pgfsetrectcap%
\pgfsetroundjoin%
\pgfsetlinewidth{1.505625pt}%
\definecolor{currentstroke}{rgb}{1.000000,0.705882,0.509804}%
\pgfsetstrokecolor{currentstroke}%
\pgfsetstrokeopacity{0.800000}%
\pgfsetdash{}{0pt}%
\pgfpathmoveto{\pgfqpoint{2.555491in}{1.342838in}}%
\pgfpathlineto{\pgfqpoint{2.899136in}{2.154863in}}%
\pgfusepath{stroke}%
\end{pgfscope}%
\begin{pgfscope}%
\pgfpathrectangle{\pgfqpoint{0.481978in}{0.331635in}}{\pgfqpoint{4.960000in}{3.696000in}}%
\pgfusepath{clip}%
\pgfsetrectcap%
\pgfsetroundjoin%
\pgfsetlinewidth{1.505625pt}%
\definecolor{currentstroke}{rgb}{1.000000,0.705882,0.509804}%
\pgfsetstrokecolor{currentstroke}%
\pgfsetstrokeopacity{0.800000}%
\pgfsetdash{}{0pt}%
\pgfpathmoveto{\pgfqpoint{4.217987in}{2.580310in}}%
\pgfpathlineto{\pgfqpoint{2.899136in}{2.154863in}}%
\pgfusepath{stroke}%
\end{pgfscope}%
\begin{pgfscope}%
\pgfpathrectangle{\pgfqpoint{0.481978in}{0.331635in}}{\pgfqpoint{4.960000in}{3.696000in}}%
\pgfusepath{clip}%
\pgfsetrectcap%
\pgfsetroundjoin%
\pgfsetlinewidth{1.505625pt}%
\definecolor{currentstroke}{rgb}{1.000000,0.705882,0.509804}%
\pgfsetstrokecolor{currentstroke}%
\pgfsetstrokeopacity{0.800000}%
\pgfsetdash{}{0pt}%
\pgfpathmoveto{\pgfqpoint{1.306020in}{2.527467in}}%
\pgfpathlineto{\pgfqpoint{2.899136in}{2.154863in}}%
\pgfusepath{stroke}%
\end{pgfscope}%
\begin{pgfscope}%
\pgfpathrectangle{\pgfqpoint{0.481978in}{0.331635in}}{\pgfqpoint{4.960000in}{3.696000in}}%
\pgfusepath{clip}%
\pgfsetrectcap%
\pgfsetroundjoin%
\pgfsetlinewidth{1.505625pt}%
\definecolor{currentstroke}{rgb}{1.000000,0.705882,0.509804}%
\pgfsetstrokecolor{currentstroke}%
\pgfsetstrokeopacity{0.800000}%
\pgfsetdash{}{0pt}%
\pgfpathmoveto{\pgfqpoint{4.013546in}{1.026803in}}%
\pgfpathlineto{\pgfqpoint{2.899136in}{2.154863in}}%
\pgfusepath{stroke}%
\end{pgfscope}%
\begin{pgfscope}%
\pgfpathrectangle{\pgfqpoint{0.481978in}{0.331635in}}{\pgfqpoint{4.960000in}{3.696000in}}%
\pgfusepath{clip}%
\pgfsetrectcap%
\pgfsetroundjoin%
\pgfsetlinewidth{1.505625pt}%
\definecolor{currentstroke}{rgb}{1.000000,0.705882,0.509804}%
\pgfsetstrokecolor{currentstroke}%
\pgfsetstrokeopacity{0.800000}%
\pgfsetdash{}{0pt}%
\pgfpathmoveto{\pgfqpoint{3.031867in}{1.242385in}}%
\pgfpathlineto{\pgfqpoint{2.899136in}{2.154863in}}%
\pgfusepath{stroke}%
\end{pgfscope}%
\begin{pgfscope}%
\pgfpathrectangle{\pgfqpoint{0.481978in}{0.331635in}}{\pgfqpoint{4.960000in}{3.696000in}}%
\pgfusepath{clip}%
\pgfsetrectcap%
\pgfsetroundjoin%
\pgfsetlinewidth{1.505625pt}%
\definecolor{currentstroke}{rgb}{1.000000,0.705882,0.509804}%
\pgfsetstrokecolor{currentstroke}%
\pgfsetstrokeopacity{0.800000}%
\pgfsetdash{}{0pt}%
\pgfpathmoveto{\pgfqpoint{2.140155in}{0.804859in}}%
\pgfpathlineto{\pgfqpoint{2.899136in}{2.154863in}}%
\pgfusepath{stroke}%
\end{pgfscope}%
\begin{pgfscope}%
\pgfpathrectangle{\pgfqpoint{0.481978in}{0.331635in}}{\pgfqpoint{4.960000in}{3.696000in}}%
\pgfusepath{clip}%
\pgfsetrectcap%
\pgfsetroundjoin%
\pgfsetlinewidth{1.505625pt}%
\definecolor{currentstroke}{rgb}{1.000000,0.705882,0.509804}%
\pgfsetstrokecolor{currentstroke}%
\pgfsetstrokeopacity{0.800000}%
\pgfsetdash{}{0pt}%
\pgfpathmoveto{\pgfqpoint{3.374264in}{3.514894in}}%
\pgfpathlineto{\pgfqpoint{2.899136in}{2.154863in}}%
\pgfusepath{stroke}%
\end{pgfscope}%
\begin{pgfscope}%
\pgfpathrectangle{\pgfqpoint{0.481978in}{0.331635in}}{\pgfqpoint{4.960000in}{3.696000in}}%
\pgfusepath{clip}%
\pgfsetrectcap%
\pgfsetroundjoin%
\pgfsetlinewidth{1.505625pt}%
\definecolor{currentstroke}{rgb}{1.000000,0.705882,0.509804}%
\pgfsetstrokecolor{currentstroke}%
\pgfsetstrokeopacity{0.800000}%
\pgfsetdash{}{0pt}%
\pgfpathmoveto{\pgfqpoint{3.767977in}{3.508351in}}%
\pgfpathlineto{\pgfqpoint{2.899136in}{2.154863in}}%
\pgfusepath{stroke}%
\end{pgfscope}%
\begin{pgfscope}%
\pgfpathrectangle{\pgfqpoint{0.481978in}{0.331635in}}{\pgfqpoint{4.960000in}{3.696000in}}%
\pgfusepath{clip}%
\pgfsetrectcap%
\pgfsetroundjoin%
\pgfsetlinewidth{1.505625pt}%
\definecolor{currentstroke}{rgb}{1.000000,0.705882,0.509804}%
\pgfsetstrokecolor{currentstroke}%
\pgfsetstrokeopacity{0.800000}%
\pgfsetdash{}{0pt}%
\pgfpathmoveto{\pgfqpoint{3.773339in}{2.944549in}}%
\pgfpathlineto{\pgfqpoint{2.899136in}{2.154863in}}%
\pgfusepath{stroke}%
\end{pgfscope}%
\begin{pgfscope}%
\pgfpathrectangle{\pgfqpoint{0.481978in}{0.331635in}}{\pgfqpoint{4.960000in}{3.696000in}}%
\pgfusepath{clip}%
\pgfsetrectcap%
\pgfsetroundjoin%
\pgfsetlinewidth{1.505625pt}%
\definecolor{currentstroke}{rgb}{1.000000,0.705882,0.509804}%
\pgfsetstrokecolor{currentstroke}%
\pgfsetstrokeopacity{0.800000}%
\pgfsetdash{}{0pt}%
\pgfpathmoveto{\pgfqpoint{1.255414in}{2.015428in}}%
\pgfpathlineto{\pgfqpoint{2.899136in}{2.154863in}}%
\pgfusepath{stroke}%
\end{pgfscope}%
\begin{pgfscope}%
\pgfpathrectangle{\pgfqpoint{0.481978in}{0.331635in}}{\pgfqpoint{4.960000in}{3.696000in}}%
\pgfusepath{clip}%
\pgfsetrectcap%
\pgfsetroundjoin%
\pgfsetlinewidth{1.505625pt}%
\definecolor{currentstroke}{rgb}{1.000000,0.705882,0.509804}%
\pgfsetstrokecolor{currentstroke}%
\pgfsetstrokeopacity{0.800000}%
\pgfsetdash{}{0pt}%
\pgfpathmoveto{\pgfqpoint{2.406351in}{0.607661in}}%
\pgfpathlineto{\pgfqpoint{2.899136in}{2.154863in}}%
\pgfusepath{stroke}%
\end{pgfscope}%
\begin{pgfscope}%
\pgfpathrectangle{\pgfqpoint{0.481978in}{0.331635in}}{\pgfqpoint{4.960000in}{3.696000in}}%
\pgfusepath{clip}%
\pgfsetrectcap%
\pgfsetroundjoin%
\pgfsetlinewidth{1.505625pt}%
\definecolor{currentstroke}{rgb}{1.000000,0.705882,0.509804}%
\pgfsetstrokecolor{currentstroke}%
\pgfsetstrokeopacity{0.800000}%
\pgfsetdash{}{0pt}%
\pgfpathmoveto{\pgfqpoint{2.404152in}{2.434285in}}%
\pgfpathlineto{\pgfqpoint{2.899136in}{2.154863in}}%
\pgfusepath{stroke}%
\end{pgfscope}%
\begin{pgfscope}%
\pgfpathrectangle{\pgfqpoint{0.481978in}{0.331635in}}{\pgfqpoint{4.960000in}{3.696000in}}%
\pgfusepath{clip}%
\pgfsetrectcap%
\pgfsetroundjoin%
\pgfsetlinewidth{1.505625pt}%
\definecolor{currentstroke}{rgb}{1.000000,0.705882,0.509804}%
\pgfsetstrokecolor{currentstroke}%
\pgfsetstrokeopacity{0.800000}%
\pgfsetdash{}{0pt}%
\pgfpathmoveto{\pgfqpoint{1.286887in}{3.030953in}}%
\pgfpathlineto{\pgfqpoint{2.899136in}{2.154863in}}%
\pgfusepath{stroke}%
\end{pgfscope}%
\begin{pgfscope}%
\pgfpathrectangle{\pgfqpoint{0.481978in}{0.331635in}}{\pgfqpoint{4.960000in}{3.696000in}}%
\pgfusepath{clip}%
\pgfsetrectcap%
\pgfsetroundjoin%
\pgfsetlinewidth{1.505625pt}%
\definecolor{currentstroke}{rgb}{1.000000,0.705882,0.509804}%
\pgfsetstrokecolor{currentstroke}%
\pgfsetstrokeopacity{0.800000}%
\pgfsetdash{}{0pt}%
\pgfpathmoveto{\pgfqpoint{0.714090in}{2.438496in}}%
\pgfpathlineto{\pgfqpoint{2.899136in}{2.154863in}}%
\pgfusepath{stroke}%
\end{pgfscope}%
\begin{pgfscope}%
\pgfpathrectangle{\pgfqpoint{0.481978in}{0.331635in}}{\pgfqpoint{4.960000in}{3.696000in}}%
\pgfusepath{clip}%
\pgfsetrectcap%
\pgfsetroundjoin%
\pgfsetlinewidth{1.505625pt}%
\definecolor{currentstroke}{rgb}{1.000000,0.705882,0.509804}%
\pgfsetstrokecolor{currentstroke}%
\pgfsetstrokeopacity{0.800000}%
\pgfsetdash{}{0pt}%
\pgfpathmoveto{\pgfqpoint{3.131682in}{3.366467in}}%
\pgfpathlineto{\pgfqpoint{2.899136in}{2.154863in}}%
\pgfusepath{stroke}%
\end{pgfscope}%
\begin{pgfscope}%
\pgfpathrectangle{\pgfqpoint{0.481978in}{0.331635in}}{\pgfqpoint{4.960000in}{3.696000in}}%
\pgfusepath{clip}%
\pgfsetrectcap%
\pgfsetroundjoin%
\pgfsetlinewidth{1.505625pt}%
\definecolor{currentstroke}{rgb}{1.000000,0.705882,0.509804}%
\pgfsetstrokecolor{currentstroke}%
\pgfsetstrokeopacity{0.800000}%
\pgfsetdash{}{0pt}%
\pgfpathmoveto{\pgfqpoint{3.213482in}{2.492949in}}%
\pgfpathlineto{\pgfqpoint{2.899136in}{2.154863in}}%
\pgfusepath{stroke}%
\end{pgfscope}%
\begin{pgfscope}%
\pgfpathrectangle{\pgfqpoint{0.481978in}{0.331635in}}{\pgfqpoint{4.960000in}{3.696000in}}%
\pgfusepath{clip}%
\pgfsetrectcap%
\pgfsetroundjoin%
\pgfsetlinewidth{1.505625pt}%
\definecolor{currentstroke}{rgb}{1.000000,0.705882,0.509804}%
\pgfsetstrokecolor{currentstroke}%
\pgfsetstrokeopacity{0.800000}%
\pgfsetdash{}{0pt}%
\pgfpathmoveto{\pgfqpoint{3.222017in}{0.720231in}}%
\pgfpathlineto{\pgfqpoint{2.899136in}{2.154863in}}%
\pgfusepath{stroke}%
\end{pgfscope}%
\begin{pgfscope}%
\pgfpathrectangle{\pgfqpoint{0.481978in}{0.331635in}}{\pgfqpoint{4.960000in}{3.696000in}}%
\pgfusepath{clip}%
\pgfsetrectcap%
\pgfsetroundjoin%
\pgfsetlinewidth{1.505625pt}%
\definecolor{currentstroke}{rgb}{1.000000,0.705882,0.509804}%
\pgfsetstrokecolor{currentstroke}%
\pgfsetstrokeopacity{0.800000}%
\pgfsetdash{}{0pt}%
\pgfpathmoveto{\pgfqpoint{0.707432in}{2.477706in}}%
\pgfpathlineto{\pgfqpoint{2.899136in}{2.154863in}}%
\pgfusepath{stroke}%
\end{pgfscope}%
\begin{pgfscope}%
\pgfpathrectangle{\pgfqpoint{0.481978in}{0.331635in}}{\pgfqpoint{4.960000in}{3.696000in}}%
\pgfusepath{clip}%
\pgfsetrectcap%
\pgfsetroundjoin%
\pgfsetlinewidth{1.505625pt}%
\definecolor{currentstroke}{rgb}{1.000000,0.705882,0.509804}%
\pgfsetstrokecolor{currentstroke}%
\pgfsetstrokeopacity{0.800000}%
\pgfsetdash{}{0pt}%
\pgfpathmoveto{\pgfqpoint{3.262853in}{1.843328in}}%
\pgfpathlineto{\pgfqpoint{2.899136in}{2.154863in}}%
\pgfusepath{stroke}%
\end{pgfscope}%
\begin{pgfscope}%
\pgfpathrectangle{\pgfqpoint{0.481978in}{0.331635in}}{\pgfqpoint{4.960000in}{3.696000in}}%
\pgfusepath{clip}%
\pgfsetrectcap%
\pgfsetroundjoin%
\pgfsetlinewidth{1.505625pt}%
\definecolor{currentstroke}{rgb}{1.000000,0.705882,0.509804}%
\pgfsetstrokecolor{currentstroke}%
\pgfsetstrokeopacity{0.800000}%
\pgfsetdash{}{0pt}%
\pgfpathmoveto{\pgfqpoint{4.437669in}{2.819952in}}%
\pgfpathlineto{\pgfqpoint{2.899136in}{2.154863in}}%
\pgfusepath{stroke}%
\end{pgfscope}%
\begin{pgfscope}%
\pgfpathrectangle{\pgfqpoint{0.481978in}{0.331635in}}{\pgfqpoint{4.960000in}{3.696000in}}%
\pgfusepath{clip}%
\pgfsetrectcap%
\pgfsetroundjoin%
\pgfsetlinewidth{1.505625pt}%
\definecolor{currentstroke}{rgb}{1.000000,0.705882,0.509804}%
\pgfsetstrokecolor{currentstroke}%
\pgfsetstrokeopacity{0.800000}%
\pgfsetdash{}{0pt}%
\pgfpathmoveto{\pgfqpoint{2.059815in}{2.083445in}}%
\pgfpathlineto{\pgfqpoint{2.899136in}{2.154863in}}%
\pgfusepath{stroke}%
\end{pgfscope}%
\begin{pgfscope}%
\pgfpathrectangle{\pgfqpoint{0.481978in}{0.331635in}}{\pgfqpoint{4.960000in}{3.696000in}}%
\pgfusepath{clip}%
\pgfsetrectcap%
\pgfsetroundjoin%
\pgfsetlinewidth{1.505625pt}%
\definecolor{currentstroke}{rgb}{1.000000,0.705882,0.509804}%
\pgfsetstrokecolor{currentstroke}%
\pgfsetstrokeopacity{0.800000}%
\pgfsetdash{}{0pt}%
\pgfpathmoveto{\pgfqpoint{1.667895in}{3.439296in}}%
\pgfpathlineto{\pgfqpoint{2.899136in}{2.154863in}}%
\pgfusepath{stroke}%
\end{pgfscope}%
\begin{pgfscope}%
\pgfpathrectangle{\pgfqpoint{0.481978in}{0.331635in}}{\pgfqpoint{4.960000in}{3.696000in}}%
\pgfusepath{clip}%
\pgfsetrectcap%
\pgfsetroundjoin%
\pgfsetlinewidth{1.505625pt}%
\definecolor{currentstroke}{rgb}{1.000000,0.705882,0.509804}%
\pgfsetstrokecolor{currentstroke}%
\pgfsetstrokeopacity{0.800000}%
\pgfsetdash{}{0pt}%
\pgfpathmoveto{\pgfqpoint{1.976452in}{3.499338in}}%
\pgfpathlineto{\pgfqpoint{2.899136in}{2.154863in}}%
\pgfusepath{stroke}%
\end{pgfscope}%
\begin{pgfscope}%
\pgfpathrectangle{\pgfqpoint{0.481978in}{0.331635in}}{\pgfqpoint{4.960000in}{3.696000in}}%
\pgfusepath{clip}%
\pgfsetrectcap%
\pgfsetroundjoin%
\pgfsetlinewidth{1.505625pt}%
\definecolor{currentstroke}{rgb}{1.000000,0.705882,0.509804}%
\pgfsetstrokecolor{currentstroke}%
\pgfsetstrokeopacity{0.800000}%
\pgfsetdash{}{0pt}%
\pgfpathmoveto{\pgfqpoint{2.471402in}{3.723544in}}%
\pgfpathlineto{\pgfqpoint{2.899136in}{2.154863in}}%
\pgfusepath{stroke}%
\end{pgfscope}%
\begin{pgfscope}%
\pgfpathrectangle{\pgfqpoint{0.481978in}{0.331635in}}{\pgfqpoint{4.960000in}{3.696000in}}%
\pgfusepath{clip}%
\pgfsetrectcap%
\pgfsetroundjoin%
\pgfsetlinewidth{1.505625pt}%
\definecolor{currentstroke}{rgb}{1.000000,0.705882,0.509804}%
\pgfsetstrokecolor{currentstroke}%
\pgfsetstrokeopacity{0.800000}%
\pgfsetdash{}{0pt}%
\pgfpathmoveto{\pgfqpoint{3.142151in}{3.296583in}}%
\pgfpathlineto{\pgfqpoint{2.899136in}{2.154863in}}%
\pgfusepath{stroke}%
\end{pgfscope}%
\begin{pgfscope}%
\pgfpathrectangle{\pgfqpoint{0.481978in}{0.331635in}}{\pgfqpoint{4.960000in}{3.696000in}}%
\pgfusepath{clip}%
\pgfsetrectcap%
\pgfsetroundjoin%
\pgfsetlinewidth{1.505625pt}%
\definecolor{currentstroke}{rgb}{1.000000,0.705882,0.509804}%
\pgfsetstrokecolor{currentstroke}%
\pgfsetstrokeopacity{0.800000}%
\pgfsetdash{}{0pt}%
\pgfpathmoveto{\pgfqpoint{2.296911in}{2.008873in}}%
\pgfpathlineto{\pgfqpoint{2.899136in}{2.154863in}}%
\pgfusepath{stroke}%
\end{pgfscope}%
\begin{pgfscope}%
\pgfpathrectangle{\pgfqpoint{0.481978in}{0.331635in}}{\pgfqpoint{4.960000in}{3.696000in}}%
\pgfusepath{clip}%
\pgfsetrectcap%
\pgfsetroundjoin%
\pgfsetlinewidth{1.505625pt}%
\definecolor{currentstroke}{rgb}{1.000000,0.705882,0.509804}%
\pgfsetstrokecolor{currentstroke}%
\pgfsetstrokeopacity{0.800000}%
\pgfsetdash{}{0pt}%
\pgfpathmoveto{\pgfqpoint{0.967458in}{2.118013in}}%
\pgfpathlineto{\pgfqpoint{2.899136in}{2.154863in}}%
\pgfusepath{stroke}%
\end{pgfscope}%
\begin{pgfscope}%
\pgfpathrectangle{\pgfqpoint{0.481978in}{0.331635in}}{\pgfqpoint{4.960000in}{3.696000in}}%
\pgfusepath{clip}%
\pgfsetrectcap%
\pgfsetroundjoin%
\pgfsetlinewidth{1.505625pt}%
\definecolor{currentstroke}{rgb}{1.000000,0.705882,0.509804}%
\pgfsetstrokecolor{currentstroke}%
\pgfsetstrokeopacity{0.800000}%
\pgfsetdash{}{0pt}%
\pgfpathmoveto{\pgfqpoint{3.223478in}{3.045802in}}%
\pgfpathlineto{\pgfqpoint{2.899136in}{2.154863in}}%
\pgfusepath{stroke}%
\end{pgfscope}%
\begin{pgfscope}%
\pgfpathrectangle{\pgfqpoint{0.481978in}{0.331635in}}{\pgfqpoint{4.960000in}{3.696000in}}%
\pgfusepath{clip}%
\pgfsetrectcap%
\pgfsetroundjoin%
\pgfsetlinewidth{1.505625pt}%
\definecolor{currentstroke}{rgb}{1.000000,0.705882,0.509804}%
\pgfsetstrokecolor{currentstroke}%
\pgfsetstrokeopacity{0.800000}%
\pgfsetdash{}{0pt}%
\pgfpathmoveto{\pgfqpoint{3.997789in}{2.650197in}}%
\pgfpathlineto{\pgfqpoint{2.899136in}{2.154863in}}%
\pgfusepath{stroke}%
\end{pgfscope}%
\begin{pgfscope}%
\pgfpathrectangle{\pgfqpoint{0.481978in}{0.331635in}}{\pgfqpoint{4.960000in}{3.696000in}}%
\pgfusepath{clip}%
\pgfsetrectcap%
\pgfsetroundjoin%
\pgfsetlinewidth{1.505625pt}%
\definecolor{currentstroke}{rgb}{1.000000,0.705882,0.509804}%
\pgfsetstrokecolor{currentstroke}%
\pgfsetstrokeopacity{0.800000}%
\pgfsetdash{}{0pt}%
\pgfpathmoveto{\pgfqpoint{3.630700in}{2.542955in}}%
\pgfpathlineto{\pgfqpoint{2.899136in}{2.154863in}}%
\pgfusepath{stroke}%
\end{pgfscope}%
\begin{pgfscope}%
\pgfpathrectangle{\pgfqpoint{0.481978in}{0.331635in}}{\pgfqpoint{4.960000in}{3.696000in}}%
\pgfusepath{clip}%
\pgfsetrectcap%
\pgfsetroundjoin%
\pgfsetlinewidth{1.505625pt}%
\definecolor{currentstroke}{rgb}{1.000000,0.705882,0.509804}%
\pgfsetstrokecolor{currentstroke}%
\pgfsetstrokeopacity{0.800000}%
\pgfsetdash{}{0pt}%
\pgfpathmoveto{\pgfqpoint{3.192223in}{3.370709in}}%
\pgfpathlineto{\pgfqpoint{2.899136in}{2.154863in}}%
\pgfusepath{stroke}%
\end{pgfscope}%
\begin{pgfscope}%
\pgfpathrectangle{\pgfqpoint{0.481978in}{0.331635in}}{\pgfqpoint{4.960000in}{3.696000in}}%
\pgfusepath{clip}%
\pgfsetrectcap%
\pgfsetroundjoin%
\pgfsetlinewidth{1.505625pt}%
\definecolor{currentstroke}{rgb}{1.000000,0.705882,0.509804}%
\pgfsetstrokecolor{currentstroke}%
\pgfsetstrokeopacity{0.800000}%
\pgfsetdash{}{0pt}%
\pgfpathmoveto{\pgfqpoint{2.811679in}{1.513956in}}%
\pgfpathlineto{\pgfqpoint{2.899136in}{2.154863in}}%
\pgfusepath{stroke}%
\end{pgfscope}%
\begin{pgfscope}%
\pgfpathrectangle{\pgfqpoint{0.481978in}{0.331635in}}{\pgfqpoint{4.960000in}{3.696000in}}%
\pgfusepath{clip}%
\pgfsetrectcap%
\pgfsetroundjoin%
\pgfsetlinewidth{1.505625pt}%
\definecolor{currentstroke}{rgb}{1.000000,0.705882,0.509804}%
\pgfsetstrokecolor{currentstroke}%
\pgfsetstrokeopacity{0.800000}%
\pgfsetdash{}{0pt}%
\pgfpathmoveto{\pgfqpoint{2.645919in}{1.878733in}}%
\pgfpathlineto{\pgfqpoint{2.899136in}{2.154863in}}%
\pgfusepath{stroke}%
\end{pgfscope}%
\begin{pgfscope}%
\pgfpathrectangle{\pgfqpoint{0.481978in}{0.331635in}}{\pgfqpoint{4.960000in}{3.696000in}}%
\pgfusepath{clip}%
\pgfsetrectcap%
\pgfsetroundjoin%
\pgfsetlinewidth{1.505625pt}%
\definecolor{currentstroke}{rgb}{1.000000,0.705882,0.509804}%
\pgfsetstrokecolor{currentstroke}%
\pgfsetstrokeopacity{0.800000}%
\pgfsetdash{}{0pt}%
\pgfpathmoveto{\pgfqpoint{4.106333in}{2.744013in}}%
\pgfpathlineto{\pgfqpoint{2.899136in}{2.154863in}}%
\pgfusepath{stroke}%
\end{pgfscope}%
\begin{pgfscope}%
\pgfpathrectangle{\pgfqpoint{0.481978in}{0.331635in}}{\pgfqpoint{4.960000in}{3.696000in}}%
\pgfusepath{clip}%
\pgfsetrectcap%
\pgfsetroundjoin%
\pgfsetlinewidth{1.505625pt}%
\definecolor{currentstroke}{rgb}{1.000000,0.705882,0.509804}%
\pgfsetstrokecolor{currentstroke}%
\pgfsetstrokeopacity{0.800000}%
\pgfsetdash{}{0pt}%
\pgfpathmoveto{\pgfqpoint{3.981935in}{1.130390in}}%
\pgfpathlineto{\pgfqpoint{2.899136in}{2.154863in}}%
\pgfusepath{stroke}%
\end{pgfscope}%
\begin{pgfscope}%
\pgfpathrectangle{\pgfqpoint{0.481978in}{0.331635in}}{\pgfqpoint{4.960000in}{3.696000in}}%
\pgfusepath{clip}%
\pgfsetrectcap%
\pgfsetroundjoin%
\pgfsetlinewidth{1.505625pt}%
\definecolor{currentstroke}{rgb}{1.000000,0.705882,0.509804}%
\pgfsetstrokecolor{currentstroke}%
\pgfsetstrokeopacity{0.800000}%
\pgfsetdash{}{0pt}%
\pgfpathmoveto{\pgfqpoint{3.806749in}{0.636945in}}%
\pgfpathlineto{\pgfqpoint{2.899136in}{2.154863in}}%
\pgfusepath{stroke}%
\end{pgfscope}%
\begin{pgfscope}%
\pgfpathrectangle{\pgfqpoint{0.481978in}{0.331635in}}{\pgfqpoint{4.960000in}{3.696000in}}%
\pgfusepath{clip}%
\pgfsetrectcap%
\pgfsetroundjoin%
\pgfsetlinewidth{1.505625pt}%
\definecolor{currentstroke}{rgb}{1.000000,0.705882,0.509804}%
\pgfsetstrokecolor{currentstroke}%
\pgfsetstrokeopacity{0.800000}%
\pgfsetdash{}{0pt}%
\pgfpathmoveto{\pgfqpoint{4.230982in}{2.568416in}}%
\pgfpathlineto{\pgfqpoint{2.899136in}{2.154863in}}%
\pgfusepath{stroke}%
\end{pgfscope}%
\begin{pgfscope}%
\pgfpathrectangle{\pgfqpoint{0.481978in}{0.331635in}}{\pgfqpoint{4.960000in}{3.696000in}}%
\pgfusepath{clip}%
\pgfsetrectcap%
\pgfsetroundjoin%
\pgfsetlinewidth{1.505625pt}%
\definecolor{currentstroke}{rgb}{1.000000,0.705882,0.509804}%
\pgfsetstrokecolor{currentstroke}%
\pgfsetstrokeopacity{0.800000}%
\pgfsetdash{}{0pt}%
\pgfpathmoveto{\pgfqpoint{3.956699in}{3.305992in}}%
\pgfpathlineto{\pgfqpoint{2.899136in}{2.154863in}}%
\pgfusepath{stroke}%
\end{pgfscope}%
\begin{pgfscope}%
\pgfpathrectangle{\pgfqpoint{0.481978in}{0.331635in}}{\pgfqpoint{4.960000in}{3.696000in}}%
\pgfusepath{clip}%
\pgfsetrectcap%
\pgfsetroundjoin%
\pgfsetlinewidth{1.505625pt}%
\definecolor{currentstroke}{rgb}{1.000000,0.705882,0.509804}%
\pgfsetstrokecolor{currentstroke}%
\pgfsetstrokeopacity{0.800000}%
\pgfsetdash{}{0pt}%
\pgfpathmoveto{\pgfqpoint{0.924460in}{1.743866in}}%
\pgfpathlineto{\pgfqpoint{2.899136in}{2.154863in}}%
\pgfusepath{stroke}%
\end{pgfscope}%
\begin{pgfscope}%
\pgfpathrectangle{\pgfqpoint{0.481978in}{0.331635in}}{\pgfqpoint{4.960000in}{3.696000in}}%
\pgfusepath{clip}%
\pgfsetrectcap%
\pgfsetroundjoin%
\pgfsetlinewidth{1.505625pt}%
\definecolor{currentstroke}{rgb}{1.000000,0.705882,0.509804}%
\pgfsetstrokecolor{currentstroke}%
\pgfsetstrokeopacity{0.800000}%
\pgfsetdash{}{0pt}%
\pgfpathmoveto{\pgfqpoint{3.140005in}{0.989817in}}%
\pgfpathlineto{\pgfqpoint{2.899136in}{2.154863in}}%
\pgfusepath{stroke}%
\end{pgfscope}%
\begin{pgfscope}%
\pgfpathrectangle{\pgfqpoint{0.481978in}{0.331635in}}{\pgfqpoint{4.960000in}{3.696000in}}%
\pgfusepath{clip}%
\pgfsetrectcap%
\pgfsetroundjoin%
\pgfsetlinewidth{1.505625pt}%
\definecolor{currentstroke}{rgb}{1.000000,0.705882,0.509804}%
\pgfsetstrokecolor{currentstroke}%
\pgfsetstrokeopacity{0.800000}%
\pgfsetdash{}{0pt}%
\pgfpathmoveto{\pgfqpoint{2.463528in}{0.930421in}}%
\pgfpathlineto{\pgfqpoint{2.899136in}{2.154863in}}%
\pgfusepath{stroke}%
\end{pgfscope}%
\begin{pgfscope}%
\pgfpathrectangle{\pgfqpoint{0.481978in}{0.331635in}}{\pgfqpoint{4.960000in}{3.696000in}}%
\pgfusepath{clip}%
\pgfsetrectcap%
\pgfsetroundjoin%
\pgfsetlinewidth{1.505625pt}%
\definecolor{currentstroke}{rgb}{1.000000,0.705882,0.509804}%
\pgfsetstrokecolor{currentstroke}%
\pgfsetstrokeopacity{0.800000}%
\pgfsetdash{}{0pt}%
\pgfpathmoveto{\pgfqpoint{1.980453in}{1.278554in}}%
\pgfpathlineto{\pgfqpoint{2.899136in}{2.154863in}}%
\pgfusepath{stroke}%
\end{pgfscope}%
\begin{pgfscope}%
\pgfpathrectangle{\pgfqpoint{0.481978in}{0.331635in}}{\pgfqpoint{4.960000in}{3.696000in}}%
\pgfusepath{clip}%
\pgfsetrectcap%
\pgfsetroundjoin%
\pgfsetlinewidth{1.505625pt}%
\definecolor{currentstroke}{rgb}{1.000000,0.705882,0.509804}%
\pgfsetstrokecolor{currentstroke}%
\pgfsetstrokeopacity{0.800000}%
\pgfsetdash{}{0pt}%
\pgfpathmoveto{\pgfqpoint{2.136569in}{3.859635in}}%
\pgfpathlineto{\pgfqpoint{2.899136in}{2.154863in}}%
\pgfusepath{stroke}%
\end{pgfscope}%
\begin{pgfscope}%
\pgfpathrectangle{\pgfqpoint{0.481978in}{0.331635in}}{\pgfqpoint{4.960000in}{3.696000in}}%
\pgfusepath{clip}%
\pgfsetrectcap%
\pgfsetroundjoin%
\pgfsetlinewidth{1.505625pt}%
\definecolor{currentstroke}{rgb}{1.000000,0.705882,0.509804}%
\pgfsetstrokecolor{currentstroke}%
\pgfsetstrokeopacity{0.800000}%
\pgfsetdash{}{0pt}%
\pgfpathmoveto{\pgfqpoint{1.304196in}{2.041751in}}%
\pgfpathlineto{\pgfqpoint{2.899136in}{2.154863in}}%
\pgfusepath{stroke}%
\end{pgfscope}%
\begin{pgfscope}%
\pgfpathrectangle{\pgfqpoint{0.481978in}{0.331635in}}{\pgfqpoint{4.960000in}{3.696000in}}%
\pgfusepath{clip}%
\pgfsetrectcap%
\pgfsetroundjoin%
\pgfsetlinewidth{1.505625pt}%
\definecolor{currentstroke}{rgb}{1.000000,0.705882,0.509804}%
\pgfsetstrokecolor{currentstroke}%
\pgfsetstrokeopacity{0.800000}%
\pgfsetdash{}{0pt}%
\pgfpathmoveto{\pgfqpoint{0.790548in}{2.246528in}}%
\pgfpathlineto{\pgfqpoint{2.899136in}{2.154863in}}%
\pgfusepath{stroke}%
\end{pgfscope}%
\begin{pgfscope}%
\pgfpathrectangle{\pgfqpoint{0.481978in}{0.331635in}}{\pgfqpoint{4.960000in}{3.696000in}}%
\pgfusepath{clip}%
\pgfsetrectcap%
\pgfsetroundjoin%
\pgfsetlinewidth{1.505625pt}%
\definecolor{currentstroke}{rgb}{1.000000,0.705882,0.509804}%
\pgfsetstrokecolor{currentstroke}%
\pgfsetstrokeopacity{0.800000}%
\pgfsetdash{}{0pt}%
\pgfpathmoveto{\pgfqpoint{1.170260in}{2.194977in}}%
\pgfpathlineto{\pgfqpoint{2.899136in}{2.154863in}}%
\pgfusepath{stroke}%
\end{pgfscope}%
\begin{pgfscope}%
\pgfpathrectangle{\pgfqpoint{0.481978in}{0.331635in}}{\pgfqpoint{4.960000in}{3.696000in}}%
\pgfusepath{clip}%
\pgfsetrectcap%
\pgfsetroundjoin%
\pgfsetlinewidth{1.505625pt}%
\definecolor{currentstroke}{rgb}{1.000000,0.705882,0.509804}%
\pgfsetstrokecolor{currentstroke}%
\pgfsetstrokeopacity{0.800000}%
\pgfsetdash{}{0pt}%
\pgfpathmoveto{\pgfqpoint{1.295016in}{1.756327in}}%
\pgfpathlineto{\pgfqpoint{2.899136in}{2.154863in}}%
\pgfusepath{stroke}%
\end{pgfscope}%
\begin{pgfscope}%
\pgfpathrectangle{\pgfqpoint{0.481978in}{0.331635in}}{\pgfqpoint{4.960000in}{3.696000in}}%
\pgfusepath{clip}%
\pgfsetrectcap%
\pgfsetroundjoin%
\pgfsetlinewidth{1.505625pt}%
\definecolor{currentstroke}{rgb}{1.000000,0.705882,0.509804}%
\pgfsetstrokecolor{currentstroke}%
\pgfsetstrokeopacity{0.800000}%
\pgfsetdash{}{0pt}%
\pgfpathmoveto{\pgfqpoint{4.058832in}{3.339371in}}%
\pgfpathlineto{\pgfqpoint{2.899136in}{2.154863in}}%
\pgfusepath{stroke}%
\end{pgfscope}%
\begin{pgfscope}%
\pgfpathrectangle{\pgfqpoint{0.481978in}{0.331635in}}{\pgfqpoint{4.960000in}{3.696000in}}%
\pgfusepath{clip}%
\pgfsetrectcap%
\pgfsetroundjoin%
\pgfsetlinewidth{1.505625pt}%
\definecolor{currentstroke}{rgb}{1.000000,0.705882,0.509804}%
\pgfsetstrokecolor{currentstroke}%
\pgfsetstrokeopacity{0.800000}%
\pgfsetdash{}{0pt}%
\pgfpathmoveto{\pgfqpoint{3.682995in}{3.031402in}}%
\pgfpathlineto{\pgfqpoint{2.899136in}{2.154863in}}%
\pgfusepath{stroke}%
\end{pgfscope}%
\begin{pgfscope}%
\pgfpathrectangle{\pgfqpoint{0.481978in}{0.331635in}}{\pgfqpoint{4.960000in}{3.696000in}}%
\pgfusepath{clip}%
\pgfsetrectcap%
\pgfsetroundjoin%
\pgfsetlinewidth{1.505625pt}%
\definecolor{currentstroke}{rgb}{1.000000,0.705882,0.509804}%
\pgfsetstrokecolor{currentstroke}%
\pgfsetstrokeopacity{0.800000}%
\pgfsetdash{}{0pt}%
\pgfpathmoveto{\pgfqpoint{2.309434in}{1.195217in}}%
\pgfpathlineto{\pgfqpoint{2.899136in}{2.154863in}}%
\pgfusepath{stroke}%
\end{pgfscope}%
\begin{pgfscope}%
\pgfpathrectangle{\pgfqpoint{0.481978in}{0.331635in}}{\pgfqpoint{4.960000in}{3.696000in}}%
\pgfusepath{clip}%
\pgfsetrectcap%
\pgfsetroundjoin%
\pgfsetlinewidth{1.505625pt}%
\definecolor{currentstroke}{rgb}{1.000000,0.705882,0.509804}%
\pgfsetstrokecolor{currentstroke}%
\pgfsetstrokeopacity{0.800000}%
\pgfsetdash{}{0pt}%
\pgfpathmoveto{\pgfqpoint{4.283274in}{1.115136in}}%
\pgfpathlineto{\pgfqpoint{2.899136in}{2.154863in}}%
\pgfusepath{stroke}%
\end{pgfscope}%
\begin{pgfscope}%
\pgfpathrectangle{\pgfqpoint{0.481978in}{0.331635in}}{\pgfqpoint{4.960000in}{3.696000in}}%
\pgfusepath{clip}%
\pgfsetrectcap%
\pgfsetroundjoin%
\pgfsetlinewidth{1.505625pt}%
\definecolor{currentstroke}{rgb}{1.000000,0.705882,0.509804}%
\pgfsetstrokecolor{currentstroke}%
\pgfsetstrokeopacity{0.800000}%
\pgfsetdash{}{0pt}%
\pgfpathmoveto{\pgfqpoint{1.048524in}{2.650571in}}%
\pgfpathlineto{\pgfqpoint{2.899136in}{2.154863in}}%
\pgfusepath{stroke}%
\end{pgfscope}%
\begin{pgfscope}%
\pgfpathrectangle{\pgfqpoint{0.481978in}{0.331635in}}{\pgfqpoint{4.960000in}{3.696000in}}%
\pgfusepath{clip}%
\pgfsetrectcap%
\pgfsetroundjoin%
\pgfsetlinewidth{1.505625pt}%
\definecolor{currentstroke}{rgb}{1.000000,0.705882,0.509804}%
\pgfsetstrokecolor{currentstroke}%
\pgfsetstrokeopacity{0.800000}%
\pgfsetdash{}{0pt}%
\pgfpathmoveto{\pgfqpoint{1.199548in}{2.430299in}}%
\pgfpathlineto{\pgfqpoint{2.899136in}{2.154863in}}%
\pgfusepath{stroke}%
\end{pgfscope}%
\begin{pgfscope}%
\pgfpathrectangle{\pgfqpoint{0.481978in}{0.331635in}}{\pgfqpoint{4.960000in}{3.696000in}}%
\pgfusepath{clip}%
\pgfsetrectcap%
\pgfsetroundjoin%
\pgfsetlinewidth{1.505625pt}%
\definecolor{currentstroke}{rgb}{1.000000,0.705882,0.509804}%
\pgfsetstrokecolor{currentstroke}%
\pgfsetstrokeopacity{0.800000}%
\pgfsetdash{}{0pt}%
\pgfpathmoveto{\pgfqpoint{4.506455in}{2.701857in}}%
\pgfpathlineto{\pgfqpoint{2.899136in}{2.154863in}}%
\pgfusepath{stroke}%
\end{pgfscope}%
\begin{pgfscope}%
\pgfpathrectangle{\pgfqpoint{0.481978in}{0.331635in}}{\pgfqpoint{4.960000in}{3.696000in}}%
\pgfusepath{clip}%
\pgfsetrectcap%
\pgfsetroundjoin%
\pgfsetlinewidth{1.505625pt}%
\definecolor{currentstroke}{rgb}{1.000000,0.705882,0.509804}%
\pgfsetstrokecolor{currentstroke}%
\pgfsetstrokeopacity{0.800000}%
\pgfsetdash{}{0pt}%
\pgfpathmoveto{\pgfqpoint{3.805585in}{2.988131in}}%
\pgfpathlineto{\pgfqpoint{2.899136in}{2.154863in}}%
\pgfusepath{stroke}%
\end{pgfscope}%
\begin{pgfscope}%
\pgfpathrectangle{\pgfqpoint{0.481978in}{0.331635in}}{\pgfqpoint{4.960000in}{3.696000in}}%
\pgfusepath{clip}%
\pgfsetrectcap%
\pgfsetroundjoin%
\pgfsetlinewidth{1.505625pt}%
\definecolor{currentstroke}{rgb}{1.000000,0.705882,0.509804}%
\pgfsetstrokecolor{currentstroke}%
\pgfsetstrokeopacity{0.800000}%
\pgfsetdash{}{0pt}%
\pgfpathmoveto{\pgfqpoint{1.394040in}{1.730514in}}%
\pgfpathlineto{\pgfqpoint{2.899136in}{2.154863in}}%
\pgfusepath{stroke}%
\end{pgfscope}%
\begin{pgfscope}%
\pgfpathrectangle{\pgfqpoint{0.481978in}{0.331635in}}{\pgfqpoint{4.960000in}{3.696000in}}%
\pgfusepath{clip}%
\pgfsetrectcap%
\pgfsetroundjoin%
\pgfsetlinewidth{1.505625pt}%
\definecolor{currentstroke}{rgb}{1.000000,0.705882,0.509804}%
\pgfsetstrokecolor{currentstroke}%
\pgfsetstrokeopacity{0.800000}%
\pgfsetdash{}{0pt}%
\pgfpathmoveto{\pgfqpoint{4.326413in}{2.981308in}}%
\pgfpathlineto{\pgfqpoint{2.899136in}{2.154863in}}%
\pgfusepath{stroke}%
\end{pgfscope}%
\begin{pgfscope}%
\pgfpathrectangle{\pgfqpoint{0.481978in}{0.331635in}}{\pgfqpoint{4.960000in}{3.696000in}}%
\pgfusepath{clip}%
\pgfsetrectcap%
\pgfsetroundjoin%
\pgfsetlinewidth{1.505625pt}%
\definecolor{currentstroke}{rgb}{1.000000,0.705882,0.509804}%
\pgfsetstrokecolor{currentstroke}%
\pgfsetstrokeopacity{0.800000}%
\pgfsetdash{}{0pt}%
\pgfpathmoveto{\pgfqpoint{3.457056in}{2.882378in}}%
\pgfpathlineto{\pgfqpoint{2.899136in}{2.154863in}}%
\pgfusepath{stroke}%
\end{pgfscope}%
\begin{pgfscope}%
\pgfpathrectangle{\pgfqpoint{0.481978in}{0.331635in}}{\pgfqpoint{4.960000in}{3.696000in}}%
\pgfusepath{clip}%
\pgfsetrectcap%
\pgfsetroundjoin%
\pgfsetlinewidth{1.505625pt}%
\definecolor{currentstroke}{rgb}{1.000000,0.705882,0.509804}%
\pgfsetstrokecolor{currentstroke}%
\pgfsetstrokeopacity{0.800000}%
\pgfsetdash{}{0pt}%
\pgfpathmoveto{\pgfqpoint{3.818493in}{3.168778in}}%
\pgfpathlineto{\pgfqpoint{2.899136in}{2.154863in}}%
\pgfusepath{stroke}%
\end{pgfscope}%
\begin{pgfscope}%
\pgfpathrectangle{\pgfqpoint{0.481978in}{0.331635in}}{\pgfqpoint{4.960000in}{3.696000in}}%
\pgfusepath{clip}%
\pgfsetrectcap%
\pgfsetroundjoin%
\pgfsetlinewidth{1.505625pt}%
\definecolor{currentstroke}{rgb}{1.000000,0.705882,0.509804}%
\pgfsetstrokecolor{currentstroke}%
\pgfsetstrokeopacity{0.800000}%
\pgfsetdash{}{0pt}%
\pgfpathmoveto{\pgfqpoint{3.050267in}{1.506955in}}%
\pgfpathlineto{\pgfqpoint{2.899136in}{2.154863in}}%
\pgfusepath{stroke}%
\end{pgfscope}%
\begin{pgfscope}%
\pgfpathrectangle{\pgfqpoint{0.481978in}{0.331635in}}{\pgfqpoint{4.960000in}{3.696000in}}%
\pgfusepath{clip}%
\pgfsetrectcap%
\pgfsetroundjoin%
\pgfsetlinewidth{1.505625pt}%
\definecolor{currentstroke}{rgb}{1.000000,0.705882,0.509804}%
\pgfsetstrokecolor{currentstroke}%
\pgfsetstrokeopacity{0.800000}%
\pgfsetdash{}{0pt}%
\pgfpathmoveto{\pgfqpoint{2.971686in}{2.231080in}}%
\pgfpathlineto{\pgfqpoint{2.899136in}{2.154863in}}%
\pgfusepath{stroke}%
\end{pgfscope}%
\begin{pgfscope}%
\pgfpathrectangle{\pgfqpoint{0.481978in}{0.331635in}}{\pgfqpoint{4.960000in}{3.696000in}}%
\pgfusepath{clip}%
\pgfsetrectcap%
\pgfsetroundjoin%
\pgfsetlinewidth{1.505625pt}%
\definecolor{currentstroke}{rgb}{1.000000,0.705882,0.509804}%
\pgfsetstrokecolor{currentstroke}%
\pgfsetstrokeopacity{0.800000}%
\pgfsetdash{}{0pt}%
\pgfpathmoveto{\pgfqpoint{3.212180in}{0.793944in}}%
\pgfpathlineto{\pgfqpoint{2.899136in}{2.154863in}}%
\pgfusepath{stroke}%
\end{pgfscope}%
\begin{pgfscope}%
\pgfpathrectangle{\pgfqpoint{0.481978in}{0.331635in}}{\pgfqpoint{4.960000in}{3.696000in}}%
\pgfusepath{clip}%
\pgfsetrectcap%
\pgfsetroundjoin%
\pgfsetlinewidth{1.505625pt}%
\definecolor{currentstroke}{rgb}{1.000000,0.705882,0.509804}%
\pgfsetstrokecolor{currentstroke}%
\pgfsetstrokeopacity{0.800000}%
\pgfsetdash{}{0pt}%
\pgfpathmoveto{\pgfqpoint{3.186086in}{1.703800in}}%
\pgfpathlineto{\pgfqpoint{2.899136in}{2.154863in}}%
\pgfusepath{stroke}%
\end{pgfscope}%
\begin{pgfscope}%
\pgfpathrectangle{\pgfqpoint{0.481978in}{0.331635in}}{\pgfqpoint{4.960000in}{3.696000in}}%
\pgfusepath{clip}%
\pgfsetrectcap%
\pgfsetroundjoin%
\pgfsetlinewidth{1.505625pt}%
\definecolor{currentstroke}{rgb}{1.000000,0.705882,0.509804}%
\pgfsetstrokecolor{currentstroke}%
\pgfsetstrokeopacity{0.800000}%
\pgfsetdash{}{0pt}%
\pgfpathmoveto{\pgfqpoint{3.777301in}{2.740473in}}%
\pgfpathlineto{\pgfqpoint{2.899136in}{2.154863in}}%
\pgfusepath{stroke}%
\end{pgfscope}%
\begin{pgfscope}%
\pgfpathrectangle{\pgfqpoint{0.481978in}{0.331635in}}{\pgfqpoint{4.960000in}{3.696000in}}%
\pgfusepath{clip}%
\pgfsetrectcap%
\pgfsetroundjoin%
\pgfsetlinewidth{1.505625pt}%
\definecolor{currentstroke}{rgb}{1.000000,0.705882,0.509804}%
\pgfsetstrokecolor{currentstroke}%
\pgfsetstrokeopacity{0.800000}%
\pgfsetdash{}{0pt}%
\pgfpathmoveto{\pgfqpoint{2.325909in}{0.980669in}}%
\pgfpathlineto{\pgfqpoint{2.899136in}{2.154863in}}%
\pgfusepath{stroke}%
\end{pgfscope}%
\begin{pgfscope}%
\pgfpathrectangle{\pgfqpoint{0.481978in}{0.331635in}}{\pgfqpoint{4.960000in}{3.696000in}}%
\pgfusepath{clip}%
\pgfsetrectcap%
\pgfsetroundjoin%
\pgfsetlinewidth{1.505625pt}%
\definecolor{currentstroke}{rgb}{1.000000,0.705882,0.509804}%
\pgfsetstrokecolor{currentstroke}%
\pgfsetstrokeopacity{0.800000}%
\pgfsetdash{}{0pt}%
\pgfpathmoveto{\pgfqpoint{1.678605in}{2.731330in}}%
\pgfpathlineto{\pgfqpoint{2.899136in}{2.154863in}}%
\pgfusepath{stroke}%
\end{pgfscope}%
\begin{pgfscope}%
\pgfpathrectangle{\pgfqpoint{0.481978in}{0.331635in}}{\pgfqpoint{4.960000in}{3.696000in}}%
\pgfusepath{clip}%
\pgfsetrectcap%
\pgfsetroundjoin%
\pgfsetlinewidth{1.505625pt}%
\definecolor{currentstroke}{rgb}{1.000000,0.705882,0.509804}%
\pgfsetstrokecolor{currentstroke}%
\pgfsetstrokeopacity{0.800000}%
\pgfsetdash{}{0pt}%
\pgfpathmoveto{\pgfqpoint{2.077157in}{0.751873in}}%
\pgfpathlineto{\pgfqpoint{2.899136in}{2.154863in}}%
\pgfusepath{stroke}%
\end{pgfscope}%
\begin{pgfscope}%
\pgfpathrectangle{\pgfqpoint{0.481978in}{0.331635in}}{\pgfqpoint{4.960000in}{3.696000in}}%
\pgfusepath{clip}%
\pgfsetrectcap%
\pgfsetroundjoin%
\pgfsetlinewidth{1.505625pt}%
\definecolor{currentstroke}{rgb}{1.000000,0.705882,0.509804}%
\pgfsetstrokecolor{currentstroke}%
\pgfsetstrokeopacity{0.800000}%
\pgfsetdash{}{0pt}%
\pgfpathmoveto{\pgfqpoint{3.318137in}{3.077236in}}%
\pgfpathlineto{\pgfqpoint{2.899136in}{2.154863in}}%
\pgfusepath{stroke}%
\end{pgfscope}%
\begin{pgfscope}%
\pgfpathrectangle{\pgfqpoint{0.481978in}{0.331635in}}{\pgfqpoint{4.960000in}{3.696000in}}%
\pgfusepath{clip}%
\pgfsetrectcap%
\pgfsetroundjoin%
\pgfsetlinewidth{1.505625pt}%
\definecolor{currentstroke}{rgb}{1.000000,0.705882,0.509804}%
\pgfsetstrokecolor{currentstroke}%
\pgfsetstrokeopacity{0.800000}%
\pgfsetdash{}{0pt}%
\pgfpathmoveto{\pgfqpoint{1.775533in}{0.808702in}}%
\pgfpathlineto{\pgfqpoint{2.899136in}{2.154863in}}%
\pgfusepath{stroke}%
\end{pgfscope}%
\begin{pgfscope}%
\pgfpathrectangle{\pgfqpoint{0.481978in}{0.331635in}}{\pgfqpoint{4.960000in}{3.696000in}}%
\pgfusepath{clip}%
\pgfsetrectcap%
\pgfsetroundjoin%
\pgfsetlinewidth{1.505625pt}%
\definecolor{currentstroke}{rgb}{1.000000,0.705882,0.509804}%
\pgfsetstrokecolor{currentstroke}%
\pgfsetstrokeopacity{0.800000}%
\pgfsetdash{}{0pt}%
\pgfpathmoveto{\pgfqpoint{1.765211in}{2.745882in}}%
\pgfpathlineto{\pgfqpoint{2.899136in}{2.154863in}}%
\pgfusepath{stroke}%
\end{pgfscope}%
\begin{pgfscope}%
\pgfpathrectangle{\pgfqpoint{0.481978in}{0.331635in}}{\pgfqpoint{4.960000in}{3.696000in}}%
\pgfusepath{clip}%
\pgfsetrectcap%
\pgfsetroundjoin%
\pgfsetlinewidth{1.505625pt}%
\definecolor{currentstroke}{rgb}{1.000000,0.705882,0.509804}%
\pgfsetstrokecolor{currentstroke}%
\pgfsetstrokeopacity{0.800000}%
\pgfsetdash{}{0pt}%
\pgfpathmoveto{\pgfqpoint{3.513350in}{1.024794in}}%
\pgfpathlineto{\pgfqpoint{2.899136in}{2.154863in}}%
\pgfusepath{stroke}%
\end{pgfscope}%
\begin{pgfscope}%
\pgfpathrectangle{\pgfqpoint{0.481978in}{0.331635in}}{\pgfqpoint{4.960000in}{3.696000in}}%
\pgfusepath{clip}%
\pgfsetrectcap%
\pgfsetroundjoin%
\pgfsetlinewidth{1.505625pt}%
\definecolor{currentstroke}{rgb}{1.000000,0.705882,0.509804}%
\pgfsetstrokecolor{currentstroke}%
\pgfsetstrokeopacity{0.800000}%
\pgfsetdash{}{0pt}%
\pgfpathmoveto{\pgfqpoint{3.224603in}{1.337101in}}%
\pgfpathlineto{\pgfqpoint{2.899136in}{2.154863in}}%
\pgfusepath{stroke}%
\end{pgfscope}%
\begin{pgfscope}%
\pgfpathrectangle{\pgfqpoint{0.481978in}{0.331635in}}{\pgfqpoint{4.960000in}{3.696000in}}%
\pgfusepath{clip}%
\pgfsetrectcap%
\pgfsetroundjoin%
\pgfsetlinewidth{1.505625pt}%
\definecolor{currentstroke}{rgb}{1.000000,0.705882,0.509804}%
\pgfsetstrokecolor{currentstroke}%
\pgfsetstrokeopacity{0.800000}%
\pgfsetdash{}{0pt}%
\pgfpathmoveto{\pgfqpoint{4.033915in}{1.318065in}}%
\pgfpathlineto{\pgfqpoint{2.899136in}{2.154863in}}%
\pgfusepath{stroke}%
\end{pgfscope}%
\begin{pgfscope}%
\pgfpathrectangle{\pgfqpoint{0.481978in}{0.331635in}}{\pgfqpoint{4.960000in}{3.696000in}}%
\pgfusepath{clip}%
\pgfsetrectcap%
\pgfsetroundjoin%
\pgfsetlinewidth{1.505625pt}%
\definecolor{currentstroke}{rgb}{1.000000,0.705882,0.509804}%
\pgfsetstrokecolor{currentstroke}%
\pgfsetstrokeopacity{0.800000}%
\pgfsetdash{}{0pt}%
\pgfpathmoveto{\pgfqpoint{2.683999in}{3.109538in}}%
\pgfpathlineto{\pgfqpoint{2.899136in}{2.154863in}}%
\pgfusepath{stroke}%
\end{pgfscope}%
\begin{pgfscope}%
\pgfpathrectangle{\pgfqpoint{0.481978in}{0.331635in}}{\pgfqpoint{4.960000in}{3.696000in}}%
\pgfusepath{clip}%
\pgfsetrectcap%
\pgfsetroundjoin%
\pgfsetlinewidth{1.505625pt}%
\definecolor{currentstroke}{rgb}{1.000000,0.705882,0.509804}%
\pgfsetstrokecolor{currentstroke}%
\pgfsetstrokeopacity{0.800000}%
\pgfsetdash{}{0pt}%
\pgfpathmoveto{\pgfqpoint{2.795687in}{1.254330in}}%
\pgfpathlineto{\pgfqpoint{2.899136in}{2.154863in}}%
\pgfusepath{stroke}%
\end{pgfscope}%
\begin{pgfscope}%
\pgfpathrectangle{\pgfqpoint{0.481978in}{0.331635in}}{\pgfqpoint{4.960000in}{3.696000in}}%
\pgfusepath{clip}%
\pgfsetrectcap%
\pgfsetroundjoin%
\pgfsetlinewidth{1.505625pt}%
\definecolor{currentstroke}{rgb}{1.000000,0.705882,0.509804}%
\pgfsetstrokecolor{currentstroke}%
\pgfsetstrokeopacity{0.800000}%
\pgfsetdash{}{0pt}%
\pgfpathmoveto{\pgfqpoint{0.720737in}{2.561292in}}%
\pgfpathlineto{\pgfqpoint{2.899136in}{2.154863in}}%
\pgfusepath{stroke}%
\end{pgfscope}%
\begin{pgfscope}%
\pgfpathrectangle{\pgfqpoint{0.481978in}{0.331635in}}{\pgfqpoint{4.960000in}{3.696000in}}%
\pgfusepath{clip}%
\pgfsetrectcap%
\pgfsetroundjoin%
\pgfsetlinewidth{1.505625pt}%
\definecolor{currentstroke}{rgb}{1.000000,0.705882,0.509804}%
\pgfsetstrokecolor{currentstroke}%
\pgfsetstrokeopacity{0.800000}%
\pgfsetdash{}{0pt}%
\pgfpathmoveto{\pgfqpoint{3.665682in}{2.783278in}}%
\pgfpathlineto{\pgfqpoint{2.899136in}{2.154863in}}%
\pgfusepath{stroke}%
\end{pgfscope}%
\begin{pgfscope}%
\pgfpathrectangle{\pgfqpoint{0.481978in}{0.331635in}}{\pgfqpoint{4.960000in}{3.696000in}}%
\pgfusepath{clip}%
\pgfsetrectcap%
\pgfsetroundjoin%
\pgfsetlinewidth{1.505625pt}%
\definecolor{currentstroke}{rgb}{1.000000,0.705882,0.509804}%
\pgfsetstrokecolor{currentstroke}%
\pgfsetstrokeopacity{0.800000}%
\pgfsetdash{}{0pt}%
\pgfpathmoveto{\pgfqpoint{2.960200in}{3.620703in}}%
\pgfpathlineto{\pgfqpoint{2.899136in}{2.154863in}}%
\pgfusepath{stroke}%
\end{pgfscope}%
\begin{pgfscope}%
\pgfpathrectangle{\pgfqpoint{0.481978in}{0.331635in}}{\pgfqpoint{4.960000in}{3.696000in}}%
\pgfusepath{clip}%
\pgfsetrectcap%
\pgfsetroundjoin%
\pgfsetlinewidth{1.505625pt}%
\definecolor{currentstroke}{rgb}{1.000000,0.705882,0.509804}%
\pgfsetstrokecolor{currentstroke}%
\pgfsetstrokeopacity{0.800000}%
\pgfsetdash{}{0pt}%
\pgfpathmoveto{\pgfqpoint{3.125856in}{0.903608in}}%
\pgfpathlineto{\pgfqpoint{2.899136in}{2.154863in}}%
\pgfusepath{stroke}%
\end{pgfscope}%
\begin{pgfscope}%
\pgfpathrectangle{\pgfqpoint{0.481978in}{0.331635in}}{\pgfqpoint{4.960000in}{3.696000in}}%
\pgfusepath{clip}%
\pgfsetrectcap%
\pgfsetroundjoin%
\pgfsetlinewidth{1.505625pt}%
\definecolor{currentstroke}{rgb}{1.000000,0.705882,0.509804}%
\pgfsetstrokecolor{currentstroke}%
\pgfsetstrokeopacity{0.800000}%
\pgfsetdash{}{0pt}%
\pgfpathmoveto{\pgfqpoint{3.821901in}{1.120795in}}%
\pgfpathlineto{\pgfqpoint{2.899136in}{2.154863in}}%
\pgfusepath{stroke}%
\end{pgfscope}%
\begin{pgfscope}%
\pgfpathrectangle{\pgfqpoint{0.481978in}{0.331635in}}{\pgfqpoint{4.960000in}{3.696000in}}%
\pgfusepath{clip}%
\pgfsetrectcap%
\pgfsetroundjoin%
\pgfsetlinewidth{1.505625pt}%
\definecolor{currentstroke}{rgb}{1.000000,0.705882,0.509804}%
\pgfsetstrokecolor{currentstroke}%
\pgfsetstrokeopacity{0.800000}%
\pgfsetdash{}{0pt}%
\pgfpathmoveto{\pgfqpoint{2.608776in}{1.810712in}}%
\pgfpathlineto{\pgfqpoint{2.899136in}{2.154863in}}%
\pgfusepath{stroke}%
\end{pgfscope}%
\begin{pgfscope}%
\pgfpathrectangle{\pgfqpoint{0.481978in}{0.331635in}}{\pgfqpoint{4.960000in}{3.696000in}}%
\pgfusepath{clip}%
\pgfsetrectcap%
\pgfsetroundjoin%
\pgfsetlinewidth{1.505625pt}%
\definecolor{currentstroke}{rgb}{1.000000,0.705882,0.509804}%
\pgfsetstrokecolor{currentstroke}%
\pgfsetstrokeopacity{0.800000}%
\pgfsetdash{}{0pt}%
\pgfpathmoveto{\pgfqpoint{0.983379in}{2.463671in}}%
\pgfpathlineto{\pgfqpoint{2.899136in}{2.154863in}}%
\pgfusepath{stroke}%
\end{pgfscope}%
\begin{pgfscope}%
\pgfpathrectangle{\pgfqpoint{0.481978in}{0.331635in}}{\pgfqpoint{4.960000in}{3.696000in}}%
\pgfusepath{clip}%
\pgfsetrectcap%
\pgfsetroundjoin%
\pgfsetlinewidth{1.505625pt}%
\definecolor{currentstroke}{rgb}{1.000000,0.705882,0.509804}%
\pgfsetstrokecolor{currentstroke}%
\pgfsetstrokeopacity{0.800000}%
\pgfsetdash{}{0pt}%
\pgfpathmoveto{\pgfqpoint{2.508413in}{1.885352in}}%
\pgfpathlineto{\pgfqpoint{2.899136in}{2.154863in}}%
\pgfusepath{stroke}%
\end{pgfscope}%
\begin{pgfscope}%
\pgfpathrectangle{\pgfqpoint{0.481978in}{0.331635in}}{\pgfqpoint{4.960000in}{3.696000in}}%
\pgfusepath{clip}%
\pgfsetrectcap%
\pgfsetroundjoin%
\pgfsetlinewidth{1.505625pt}%
\definecolor{currentstroke}{rgb}{1.000000,0.705882,0.509804}%
\pgfsetstrokecolor{currentstroke}%
\pgfsetstrokeopacity{0.800000}%
\pgfsetdash{}{0pt}%
\pgfpathmoveto{\pgfqpoint{2.999772in}{1.485408in}}%
\pgfpathlineto{\pgfqpoint{2.899136in}{2.154863in}}%
\pgfusepath{stroke}%
\end{pgfscope}%
\begin{pgfscope}%
\pgfpathrectangle{\pgfqpoint{0.481978in}{0.331635in}}{\pgfqpoint{4.960000in}{3.696000in}}%
\pgfusepath{clip}%
\pgfsetrectcap%
\pgfsetroundjoin%
\pgfsetlinewidth{1.505625pt}%
\definecolor{currentstroke}{rgb}{1.000000,0.705882,0.509804}%
\pgfsetstrokecolor{currentstroke}%
\pgfsetstrokeopacity{0.800000}%
\pgfsetdash{}{0pt}%
\pgfpathmoveto{\pgfqpoint{4.146138in}{1.396639in}}%
\pgfpathlineto{\pgfqpoint{2.899136in}{2.154863in}}%
\pgfusepath{stroke}%
\end{pgfscope}%
\begin{pgfscope}%
\pgfpathrectangle{\pgfqpoint{0.481978in}{0.331635in}}{\pgfqpoint{4.960000in}{3.696000in}}%
\pgfusepath{clip}%
\pgfsetrectcap%
\pgfsetroundjoin%
\pgfsetlinewidth{1.505625pt}%
\definecolor{currentstroke}{rgb}{1.000000,0.705882,0.509804}%
\pgfsetstrokecolor{currentstroke}%
\pgfsetstrokeopacity{0.800000}%
\pgfsetdash{}{0pt}%
\pgfpathmoveto{\pgfqpoint{2.588931in}{2.917515in}}%
\pgfpathlineto{\pgfqpoint{2.899136in}{2.154863in}}%
\pgfusepath{stroke}%
\end{pgfscope}%
\begin{pgfscope}%
\pgfpathrectangle{\pgfqpoint{0.481978in}{0.331635in}}{\pgfqpoint{4.960000in}{3.696000in}}%
\pgfusepath{clip}%
\pgfsetrectcap%
\pgfsetroundjoin%
\pgfsetlinewidth{1.505625pt}%
\definecolor{currentstroke}{rgb}{1.000000,0.705882,0.509804}%
\pgfsetstrokecolor{currentstroke}%
\pgfsetstrokeopacity{0.800000}%
\pgfsetdash{}{0pt}%
\pgfpathmoveto{\pgfqpoint{2.749278in}{0.905867in}}%
\pgfpathlineto{\pgfqpoint{2.899136in}{2.154863in}}%
\pgfusepath{stroke}%
\end{pgfscope}%
\begin{pgfscope}%
\pgfpathrectangle{\pgfqpoint{0.481978in}{0.331635in}}{\pgfqpoint{4.960000in}{3.696000in}}%
\pgfusepath{clip}%
\pgfsetrectcap%
\pgfsetroundjoin%
\pgfsetlinewidth{1.505625pt}%
\definecolor{currentstroke}{rgb}{1.000000,0.705882,0.509804}%
\pgfsetstrokecolor{currentstroke}%
\pgfsetstrokeopacity{0.800000}%
\pgfsetdash{}{0pt}%
\pgfpathmoveto{\pgfqpoint{2.852198in}{1.787938in}}%
\pgfpathlineto{\pgfqpoint{2.899136in}{2.154863in}}%
\pgfusepath{stroke}%
\end{pgfscope}%
\begin{pgfscope}%
\pgfpathrectangle{\pgfqpoint{0.481978in}{0.331635in}}{\pgfqpoint{4.960000in}{3.696000in}}%
\pgfusepath{clip}%
\pgfsetrectcap%
\pgfsetroundjoin%
\pgfsetlinewidth{1.505625pt}%
\definecolor{currentstroke}{rgb}{1.000000,0.705882,0.509804}%
\pgfsetstrokecolor{currentstroke}%
\pgfsetstrokeopacity{0.800000}%
\pgfsetdash{}{0pt}%
\pgfpathmoveto{\pgfqpoint{3.111968in}{1.457614in}}%
\pgfpathlineto{\pgfqpoint{2.899136in}{2.154863in}}%
\pgfusepath{stroke}%
\end{pgfscope}%
\begin{pgfscope}%
\pgfpathrectangle{\pgfqpoint{0.481978in}{0.331635in}}{\pgfqpoint{4.960000in}{3.696000in}}%
\pgfusepath{clip}%
\pgfsetrectcap%
\pgfsetroundjoin%
\pgfsetlinewidth{1.505625pt}%
\definecolor{currentstroke}{rgb}{1.000000,0.705882,0.509804}%
\pgfsetstrokecolor{currentstroke}%
\pgfsetstrokeopacity{0.800000}%
\pgfsetdash{}{0pt}%
\pgfpathmoveto{\pgfqpoint{3.717258in}{2.952778in}}%
\pgfpathlineto{\pgfqpoint{2.899136in}{2.154863in}}%
\pgfusepath{stroke}%
\end{pgfscope}%
\begin{pgfscope}%
\pgfpathrectangle{\pgfqpoint{0.481978in}{0.331635in}}{\pgfqpoint{4.960000in}{3.696000in}}%
\pgfusepath{clip}%
\pgfsetrectcap%
\pgfsetroundjoin%
\pgfsetlinewidth{1.505625pt}%
\definecolor{currentstroke}{rgb}{1.000000,0.705882,0.509804}%
\pgfsetstrokecolor{currentstroke}%
\pgfsetstrokeopacity{0.800000}%
\pgfsetdash{}{0pt}%
\pgfpathmoveto{\pgfqpoint{2.466644in}{1.287635in}}%
\pgfpathlineto{\pgfqpoint{2.899136in}{2.154863in}}%
\pgfusepath{stroke}%
\end{pgfscope}%
\begin{pgfscope}%
\pgfpathrectangle{\pgfqpoint{0.481978in}{0.331635in}}{\pgfqpoint{4.960000in}{3.696000in}}%
\pgfusepath{clip}%
\pgfsetrectcap%
\pgfsetroundjoin%
\pgfsetlinewidth{1.505625pt}%
\definecolor{currentstroke}{rgb}{1.000000,0.705882,0.509804}%
\pgfsetstrokecolor{currentstroke}%
\pgfsetstrokeopacity{0.800000}%
\pgfsetdash{}{0pt}%
\pgfpathmoveto{\pgfqpoint{2.803563in}{2.932145in}}%
\pgfpathlineto{\pgfqpoint{2.899136in}{2.154863in}}%
\pgfusepath{stroke}%
\end{pgfscope}%
\begin{pgfscope}%
\pgfpathrectangle{\pgfqpoint{0.481978in}{0.331635in}}{\pgfqpoint{4.960000in}{3.696000in}}%
\pgfusepath{clip}%
\pgfsetrectcap%
\pgfsetroundjoin%
\pgfsetlinewidth{1.505625pt}%
\definecolor{currentstroke}{rgb}{1.000000,0.705882,0.509804}%
\pgfsetstrokecolor{currentstroke}%
\pgfsetstrokeopacity{0.800000}%
\pgfsetdash{}{0pt}%
\pgfpathmoveto{\pgfqpoint{3.672387in}{2.774528in}}%
\pgfpathlineto{\pgfqpoint{2.899136in}{2.154863in}}%
\pgfusepath{stroke}%
\end{pgfscope}%
\begin{pgfscope}%
\pgfpathrectangle{\pgfqpoint{0.481978in}{0.331635in}}{\pgfqpoint{4.960000in}{3.696000in}}%
\pgfusepath{clip}%
\pgfsetrectcap%
\pgfsetroundjoin%
\pgfsetlinewidth{1.505625pt}%
\definecolor{currentstroke}{rgb}{1.000000,0.705882,0.509804}%
\pgfsetstrokecolor{currentstroke}%
\pgfsetstrokeopacity{0.800000}%
\pgfsetdash{}{0pt}%
\pgfpathmoveto{\pgfqpoint{4.195542in}{2.763359in}}%
\pgfpathlineto{\pgfqpoint{2.899136in}{2.154863in}}%
\pgfusepath{stroke}%
\end{pgfscope}%
\begin{pgfscope}%
\pgfpathrectangle{\pgfqpoint{0.481978in}{0.331635in}}{\pgfqpoint{4.960000in}{3.696000in}}%
\pgfusepath{clip}%
\pgfsetrectcap%
\pgfsetroundjoin%
\pgfsetlinewidth{1.505625pt}%
\definecolor{currentstroke}{rgb}{1.000000,0.705882,0.509804}%
\pgfsetstrokecolor{currentstroke}%
\pgfsetstrokeopacity{0.800000}%
\pgfsetdash{}{0pt}%
\pgfpathmoveto{\pgfqpoint{2.935072in}{1.662651in}}%
\pgfpathlineto{\pgfqpoint{2.899136in}{2.154863in}}%
\pgfusepath{stroke}%
\end{pgfscope}%
\begin{pgfscope}%
\pgfpathrectangle{\pgfqpoint{0.481978in}{0.331635in}}{\pgfqpoint{4.960000in}{3.696000in}}%
\pgfusepath{clip}%
\pgfsetrectcap%
\pgfsetroundjoin%
\pgfsetlinewidth{1.505625pt}%
\definecolor{currentstroke}{rgb}{1.000000,0.705882,0.509804}%
\pgfsetstrokecolor{currentstroke}%
\pgfsetstrokeopacity{0.800000}%
\pgfsetdash{}{0pt}%
\pgfpathmoveto{\pgfqpoint{2.676919in}{1.361082in}}%
\pgfpathlineto{\pgfqpoint{2.899136in}{2.154863in}}%
\pgfusepath{stroke}%
\end{pgfscope}%
\begin{pgfscope}%
\pgfpathrectangle{\pgfqpoint{0.481978in}{0.331635in}}{\pgfqpoint{4.960000in}{3.696000in}}%
\pgfusepath{clip}%
\pgfsetrectcap%
\pgfsetroundjoin%
\pgfsetlinewidth{1.505625pt}%
\definecolor{currentstroke}{rgb}{1.000000,0.705882,0.509804}%
\pgfsetstrokecolor{currentstroke}%
\pgfsetstrokeopacity{0.800000}%
\pgfsetdash{}{0pt}%
\pgfpathmoveto{\pgfqpoint{2.794210in}{2.765367in}}%
\pgfpathlineto{\pgfqpoint{2.899136in}{2.154863in}}%
\pgfusepath{stroke}%
\end{pgfscope}%
\begin{pgfscope}%
\pgfpathrectangle{\pgfqpoint{0.481978in}{0.331635in}}{\pgfqpoint{4.960000in}{3.696000in}}%
\pgfusepath{clip}%
\pgfsetrectcap%
\pgfsetroundjoin%
\pgfsetlinewidth{1.505625pt}%
\definecolor{currentstroke}{rgb}{1.000000,0.705882,0.509804}%
\pgfsetstrokecolor{currentstroke}%
\pgfsetstrokeopacity{0.800000}%
\pgfsetdash{}{0pt}%
\pgfpathmoveto{\pgfqpoint{3.191851in}{1.169127in}}%
\pgfpathlineto{\pgfqpoint{2.899136in}{2.154863in}}%
\pgfusepath{stroke}%
\end{pgfscope}%
\begin{pgfscope}%
\pgfpathrectangle{\pgfqpoint{0.481978in}{0.331635in}}{\pgfqpoint{4.960000in}{3.696000in}}%
\pgfusepath{clip}%
\pgfsetrectcap%
\pgfsetroundjoin%
\pgfsetlinewidth{1.505625pt}%
\definecolor{currentstroke}{rgb}{1.000000,0.705882,0.509804}%
\pgfsetstrokecolor{currentstroke}%
\pgfsetstrokeopacity{0.800000}%
\pgfsetdash{}{0pt}%
\pgfpathmoveto{\pgfqpoint{2.491019in}{1.379708in}}%
\pgfpathlineto{\pgfqpoint{2.899136in}{2.154863in}}%
\pgfusepath{stroke}%
\end{pgfscope}%
\begin{pgfscope}%
\pgfpathrectangle{\pgfqpoint{0.481978in}{0.331635in}}{\pgfqpoint{4.960000in}{3.696000in}}%
\pgfusepath{clip}%
\pgfsetrectcap%
\pgfsetroundjoin%
\pgfsetlinewidth{1.505625pt}%
\definecolor{currentstroke}{rgb}{1.000000,0.705882,0.509804}%
\pgfsetstrokecolor{currentstroke}%
\pgfsetstrokeopacity{0.800000}%
\pgfsetdash{}{0pt}%
\pgfpathmoveto{\pgfqpoint{0.871762in}{1.845266in}}%
\pgfpathlineto{\pgfqpoint{2.899136in}{2.154863in}}%
\pgfusepath{stroke}%
\end{pgfscope}%
\begin{pgfscope}%
\pgfpathrectangle{\pgfqpoint{0.481978in}{0.331635in}}{\pgfqpoint{4.960000in}{3.696000in}}%
\pgfusepath{clip}%
\pgfsetrectcap%
\pgfsetroundjoin%
\pgfsetlinewidth{1.505625pt}%
\definecolor{currentstroke}{rgb}{1.000000,0.705882,0.509804}%
\pgfsetstrokecolor{currentstroke}%
\pgfsetstrokeopacity{0.800000}%
\pgfsetdash{}{0pt}%
\pgfpathmoveto{\pgfqpoint{1.354219in}{1.972322in}}%
\pgfpathlineto{\pgfqpoint{2.899136in}{2.154863in}}%
\pgfusepath{stroke}%
\end{pgfscope}%
\begin{pgfscope}%
\pgfpathrectangle{\pgfqpoint{0.481978in}{0.331635in}}{\pgfqpoint{4.960000in}{3.696000in}}%
\pgfusepath{clip}%
\pgfsetrectcap%
\pgfsetroundjoin%
\pgfsetlinewidth{1.505625pt}%
\definecolor{currentstroke}{rgb}{1.000000,0.705882,0.509804}%
\pgfsetstrokecolor{currentstroke}%
\pgfsetstrokeopacity{0.800000}%
\pgfsetdash{}{0pt}%
\pgfpathmoveto{\pgfqpoint{4.550338in}{2.823095in}}%
\pgfpathlineto{\pgfqpoint{2.899136in}{2.154863in}}%
\pgfusepath{stroke}%
\end{pgfscope}%
\begin{pgfscope}%
\pgfpathrectangle{\pgfqpoint{0.481978in}{0.331635in}}{\pgfqpoint{4.960000in}{3.696000in}}%
\pgfusepath{clip}%
\pgfsetrectcap%
\pgfsetroundjoin%
\pgfsetlinewidth{1.505625pt}%
\definecolor{currentstroke}{rgb}{1.000000,0.705882,0.509804}%
\pgfsetstrokecolor{currentstroke}%
\pgfsetstrokeopacity{0.800000}%
\pgfsetdash{}{0pt}%
\pgfpathmoveto{\pgfqpoint{4.171300in}{2.696409in}}%
\pgfpathlineto{\pgfqpoint{2.899136in}{2.154863in}}%
\pgfusepath{stroke}%
\end{pgfscope}%
\begin{pgfscope}%
\pgfpathrectangle{\pgfqpoint{0.481978in}{0.331635in}}{\pgfqpoint{4.960000in}{3.696000in}}%
\pgfusepath{clip}%
\pgfsetrectcap%
\pgfsetroundjoin%
\pgfsetlinewidth{1.505625pt}%
\definecolor{currentstroke}{rgb}{1.000000,0.705882,0.509804}%
\pgfsetstrokecolor{currentstroke}%
\pgfsetstrokeopacity{0.800000}%
\pgfsetdash{}{0pt}%
\pgfpathmoveto{\pgfqpoint{1.370027in}{1.721122in}}%
\pgfpathlineto{\pgfqpoint{2.899136in}{2.154863in}}%
\pgfusepath{stroke}%
\end{pgfscope}%
\begin{pgfscope}%
\pgfpathrectangle{\pgfqpoint{0.481978in}{0.331635in}}{\pgfqpoint{4.960000in}{3.696000in}}%
\pgfusepath{clip}%
\pgfsetrectcap%
\pgfsetroundjoin%
\pgfsetlinewidth{1.505625pt}%
\definecolor{currentstroke}{rgb}{1.000000,0.705882,0.509804}%
\pgfsetstrokecolor{currentstroke}%
\pgfsetstrokeopacity{0.800000}%
\pgfsetdash{}{0pt}%
\pgfpathmoveto{\pgfqpoint{5.216523in}{1.600304in}}%
\pgfpathlineto{\pgfqpoint{2.899136in}{2.154863in}}%
\pgfusepath{stroke}%
\end{pgfscope}%
\begin{pgfscope}%
\pgfpathrectangle{\pgfqpoint{0.481978in}{0.331635in}}{\pgfqpoint{4.960000in}{3.696000in}}%
\pgfusepath{clip}%
\pgfsetrectcap%
\pgfsetroundjoin%
\pgfsetlinewidth{1.505625pt}%
\definecolor{currentstroke}{rgb}{1.000000,0.705882,0.509804}%
\pgfsetstrokecolor{currentstroke}%
\pgfsetstrokeopacity{0.800000}%
\pgfsetdash{}{0pt}%
\pgfpathmoveto{\pgfqpoint{2.099960in}{1.764110in}}%
\pgfpathlineto{\pgfqpoint{2.899136in}{2.154863in}}%
\pgfusepath{stroke}%
\end{pgfscope}%
\begin{pgfscope}%
\pgfpathrectangle{\pgfqpoint{0.481978in}{0.331635in}}{\pgfqpoint{4.960000in}{3.696000in}}%
\pgfusepath{clip}%
\pgfsetrectcap%
\pgfsetroundjoin%
\pgfsetlinewidth{1.505625pt}%
\definecolor{currentstroke}{rgb}{1.000000,0.705882,0.509804}%
\pgfsetstrokecolor{currentstroke}%
\pgfsetstrokeopacity{0.800000}%
\pgfsetdash{}{0pt}%
\pgfpathmoveto{\pgfqpoint{3.292277in}{3.240070in}}%
\pgfpathlineto{\pgfqpoint{2.899136in}{2.154863in}}%
\pgfusepath{stroke}%
\end{pgfscope}%
\begin{pgfscope}%
\pgfpathrectangle{\pgfqpoint{0.481978in}{0.331635in}}{\pgfqpoint{4.960000in}{3.696000in}}%
\pgfusepath{clip}%
\pgfsetrectcap%
\pgfsetroundjoin%
\pgfsetlinewidth{1.505625pt}%
\definecolor{currentstroke}{rgb}{1.000000,0.705882,0.509804}%
\pgfsetstrokecolor{currentstroke}%
\pgfsetstrokeopacity{0.800000}%
\pgfsetdash{}{0pt}%
\pgfpathmoveto{\pgfqpoint{3.994608in}{2.577454in}}%
\pgfpathlineto{\pgfqpoint{2.899136in}{2.154863in}}%
\pgfusepath{stroke}%
\end{pgfscope}%
\begin{pgfscope}%
\pgfpathrectangle{\pgfqpoint{0.481978in}{0.331635in}}{\pgfqpoint{4.960000in}{3.696000in}}%
\pgfusepath{clip}%
\pgfsetrectcap%
\pgfsetroundjoin%
\pgfsetlinewidth{1.505625pt}%
\definecolor{currentstroke}{rgb}{1.000000,0.705882,0.509804}%
\pgfsetstrokecolor{currentstroke}%
\pgfsetstrokeopacity{0.800000}%
\pgfsetdash{}{0pt}%
\pgfpathmoveto{\pgfqpoint{2.854723in}{1.254498in}}%
\pgfpathlineto{\pgfqpoint{2.899136in}{2.154863in}}%
\pgfusepath{stroke}%
\end{pgfscope}%
\begin{pgfscope}%
\pgfpathrectangle{\pgfqpoint{0.481978in}{0.331635in}}{\pgfqpoint{4.960000in}{3.696000in}}%
\pgfusepath{clip}%
\pgfsetrectcap%
\pgfsetroundjoin%
\pgfsetlinewidth{1.505625pt}%
\definecolor{currentstroke}{rgb}{1.000000,0.705882,0.509804}%
\pgfsetstrokecolor{currentstroke}%
\pgfsetstrokeopacity{0.800000}%
\pgfsetdash{}{0pt}%
\pgfpathmoveto{\pgfqpoint{1.725916in}{3.573941in}}%
\pgfpathlineto{\pgfqpoint{2.899136in}{2.154863in}}%
\pgfusepath{stroke}%
\end{pgfscope}%
\begin{pgfscope}%
\pgfpathrectangle{\pgfqpoint{0.481978in}{0.331635in}}{\pgfqpoint{4.960000in}{3.696000in}}%
\pgfusepath{clip}%
\pgfsetrectcap%
\pgfsetroundjoin%
\pgfsetlinewidth{1.505625pt}%
\definecolor{currentstroke}{rgb}{1.000000,0.705882,0.509804}%
\pgfsetstrokecolor{currentstroke}%
\pgfsetstrokeopacity{0.800000}%
\pgfsetdash{}{0pt}%
\pgfpathmoveto{\pgfqpoint{3.356000in}{1.023183in}}%
\pgfpathlineto{\pgfqpoint{2.899136in}{2.154863in}}%
\pgfusepath{stroke}%
\end{pgfscope}%
\begin{pgfscope}%
\pgfpathrectangle{\pgfqpoint{0.481978in}{0.331635in}}{\pgfqpoint{4.960000in}{3.696000in}}%
\pgfusepath{clip}%
\pgfsetrectcap%
\pgfsetroundjoin%
\pgfsetlinewidth{1.505625pt}%
\definecolor{currentstroke}{rgb}{1.000000,0.705882,0.509804}%
\pgfsetstrokecolor{currentstroke}%
\pgfsetstrokeopacity{0.800000}%
\pgfsetdash{}{0pt}%
\pgfpathmoveto{\pgfqpoint{2.839959in}{1.563673in}}%
\pgfpathlineto{\pgfqpoint{2.899136in}{2.154863in}}%
\pgfusepath{stroke}%
\end{pgfscope}%
\begin{pgfscope}%
\pgfpathrectangle{\pgfqpoint{0.481978in}{0.331635in}}{\pgfqpoint{4.960000in}{3.696000in}}%
\pgfusepath{clip}%
\pgfsetrectcap%
\pgfsetroundjoin%
\pgfsetlinewidth{1.505625pt}%
\definecolor{currentstroke}{rgb}{1.000000,0.705882,0.509804}%
\pgfsetstrokecolor{currentstroke}%
\pgfsetstrokeopacity{0.800000}%
\pgfsetdash{}{0pt}%
\pgfpathmoveto{\pgfqpoint{3.583295in}{3.037337in}}%
\pgfpathlineto{\pgfqpoint{2.899136in}{2.154863in}}%
\pgfusepath{stroke}%
\end{pgfscope}%
\begin{pgfscope}%
\pgfpathrectangle{\pgfqpoint{0.481978in}{0.331635in}}{\pgfqpoint{4.960000in}{3.696000in}}%
\pgfusepath{clip}%
\pgfsetrectcap%
\pgfsetroundjoin%
\pgfsetlinewidth{1.505625pt}%
\definecolor{currentstroke}{rgb}{1.000000,0.705882,0.509804}%
\pgfsetstrokecolor{currentstroke}%
\pgfsetstrokeopacity{0.800000}%
\pgfsetdash{}{0pt}%
\pgfpathmoveto{\pgfqpoint{2.238649in}{1.848171in}}%
\pgfpathlineto{\pgfqpoint{2.899136in}{2.154863in}}%
\pgfusepath{stroke}%
\end{pgfscope}%
\begin{pgfscope}%
\pgfpathrectangle{\pgfqpoint{0.481978in}{0.331635in}}{\pgfqpoint{4.960000in}{3.696000in}}%
\pgfusepath{clip}%
\pgfsetrectcap%
\pgfsetroundjoin%
\pgfsetlinewidth{1.505625pt}%
\definecolor{currentstroke}{rgb}{1.000000,0.705882,0.509804}%
\pgfsetstrokecolor{currentstroke}%
\pgfsetstrokeopacity{0.800000}%
\pgfsetdash{}{0pt}%
\pgfpathmoveto{\pgfqpoint{1.441655in}{2.280064in}}%
\pgfpathlineto{\pgfqpoint{2.899136in}{2.154863in}}%
\pgfusepath{stroke}%
\end{pgfscope}%
\begin{pgfscope}%
\pgfpathrectangle{\pgfqpoint{0.481978in}{0.331635in}}{\pgfqpoint{4.960000in}{3.696000in}}%
\pgfusepath{clip}%
\pgfsetrectcap%
\pgfsetroundjoin%
\pgfsetlinewidth{1.505625pt}%
\definecolor{currentstroke}{rgb}{1.000000,0.705882,0.509804}%
\pgfsetstrokecolor{currentstroke}%
\pgfsetstrokeopacity{0.800000}%
\pgfsetdash{}{0pt}%
\pgfpathmoveto{\pgfqpoint{4.344247in}{2.225011in}}%
\pgfpathlineto{\pgfqpoint{2.899136in}{2.154863in}}%
\pgfusepath{stroke}%
\end{pgfscope}%
\begin{pgfscope}%
\pgfpathrectangle{\pgfqpoint{0.481978in}{0.331635in}}{\pgfqpoint{4.960000in}{3.696000in}}%
\pgfusepath{clip}%
\pgfsetrectcap%
\pgfsetroundjoin%
\pgfsetlinewidth{1.505625pt}%
\definecolor{currentstroke}{rgb}{1.000000,0.705882,0.509804}%
\pgfsetstrokecolor{currentstroke}%
\pgfsetstrokeopacity{0.800000}%
\pgfsetdash{}{0pt}%
\pgfpathmoveto{\pgfqpoint{3.012958in}{3.091478in}}%
\pgfpathlineto{\pgfqpoint{2.899136in}{2.154863in}}%
\pgfusepath{stroke}%
\end{pgfscope}%
\begin{pgfscope}%
\pgfpathrectangle{\pgfqpoint{0.481978in}{0.331635in}}{\pgfqpoint{4.960000in}{3.696000in}}%
\pgfusepath{clip}%
\pgfsetrectcap%
\pgfsetroundjoin%
\pgfsetlinewidth{1.505625pt}%
\definecolor{currentstroke}{rgb}{1.000000,0.705882,0.509804}%
\pgfsetstrokecolor{currentstroke}%
\pgfsetstrokeopacity{0.800000}%
\pgfsetdash{}{0pt}%
\pgfpathmoveto{\pgfqpoint{4.097583in}{1.366950in}}%
\pgfpathlineto{\pgfqpoint{2.899136in}{2.154863in}}%
\pgfusepath{stroke}%
\end{pgfscope}%
\begin{pgfscope}%
\pgfpathrectangle{\pgfqpoint{0.481978in}{0.331635in}}{\pgfqpoint{4.960000in}{3.696000in}}%
\pgfusepath{clip}%
\pgfsetrectcap%
\pgfsetroundjoin%
\pgfsetlinewidth{1.505625pt}%
\definecolor{currentstroke}{rgb}{1.000000,0.705882,0.509804}%
\pgfsetstrokecolor{currentstroke}%
\pgfsetstrokeopacity{0.800000}%
\pgfsetdash{}{0pt}%
\pgfpathmoveto{\pgfqpoint{2.247330in}{3.558093in}}%
\pgfpathlineto{\pgfqpoint{2.899136in}{2.154863in}}%
\pgfusepath{stroke}%
\end{pgfscope}%
\begin{pgfscope}%
\pgfpathrectangle{\pgfqpoint{0.481978in}{0.331635in}}{\pgfqpoint{4.960000in}{3.696000in}}%
\pgfusepath{clip}%
\pgfsetrectcap%
\pgfsetroundjoin%
\pgfsetlinewidth{1.505625pt}%
\definecolor{currentstroke}{rgb}{1.000000,0.705882,0.509804}%
\pgfsetstrokecolor{currentstroke}%
\pgfsetstrokeopacity{0.800000}%
\pgfsetdash{}{0pt}%
\pgfpathmoveto{\pgfqpoint{2.664498in}{1.269495in}}%
\pgfpathlineto{\pgfqpoint{2.899136in}{2.154863in}}%
\pgfusepath{stroke}%
\end{pgfscope}%
\begin{pgfscope}%
\pgfpathrectangle{\pgfqpoint{0.481978in}{0.331635in}}{\pgfqpoint{4.960000in}{3.696000in}}%
\pgfusepath{clip}%
\pgfsetrectcap%
\pgfsetroundjoin%
\pgfsetlinewidth{1.505625pt}%
\definecolor{currentstroke}{rgb}{1.000000,0.705882,0.509804}%
\pgfsetstrokecolor{currentstroke}%
\pgfsetstrokeopacity{0.800000}%
\pgfsetdash{}{0pt}%
\pgfpathmoveto{\pgfqpoint{2.466458in}{3.730016in}}%
\pgfpathlineto{\pgfqpoint{2.899136in}{2.154863in}}%
\pgfusepath{stroke}%
\end{pgfscope}%
\begin{pgfscope}%
\pgfpathrectangle{\pgfqpoint{0.481978in}{0.331635in}}{\pgfqpoint{4.960000in}{3.696000in}}%
\pgfusepath{clip}%
\pgfsetrectcap%
\pgfsetroundjoin%
\pgfsetlinewidth{1.505625pt}%
\definecolor{currentstroke}{rgb}{1.000000,0.705882,0.509804}%
\pgfsetstrokecolor{currentstroke}%
\pgfsetstrokeopacity{0.800000}%
\pgfsetdash{}{0pt}%
\pgfpathmoveto{\pgfqpoint{2.511710in}{1.139099in}}%
\pgfpathlineto{\pgfqpoint{2.899136in}{2.154863in}}%
\pgfusepath{stroke}%
\end{pgfscope}%
\begin{pgfscope}%
\pgfpathrectangle{\pgfqpoint{0.481978in}{0.331635in}}{\pgfqpoint{4.960000in}{3.696000in}}%
\pgfusepath{clip}%
\pgfsetrectcap%
\pgfsetroundjoin%
\pgfsetlinewidth{1.505625pt}%
\definecolor{currentstroke}{rgb}{1.000000,0.705882,0.509804}%
\pgfsetstrokecolor{currentstroke}%
\pgfsetstrokeopacity{0.800000}%
\pgfsetdash{}{0pt}%
\pgfpathmoveto{\pgfqpoint{3.946493in}{1.247401in}}%
\pgfpathlineto{\pgfqpoint{2.899136in}{2.154863in}}%
\pgfusepath{stroke}%
\end{pgfscope}%
\begin{pgfscope}%
\pgfpathrectangle{\pgfqpoint{0.481978in}{0.331635in}}{\pgfqpoint{4.960000in}{3.696000in}}%
\pgfusepath{clip}%
\pgfsetrectcap%
\pgfsetroundjoin%
\pgfsetlinewidth{1.505625pt}%
\definecolor{currentstroke}{rgb}{1.000000,0.705882,0.509804}%
\pgfsetstrokecolor{currentstroke}%
\pgfsetstrokeopacity{0.800000}%
\pgfsetdash{}{0pt}%
\pgfpathmoveto{\pgfqpoint{3.308273in}{3.770967in}}%
\pgfpathlineto{\pgfqpoint{2.899136in}{2.154863in}}%
\pgfusepath{stroke}%
\end{pgfscope}%
\begin{pgfscope}%
\pgfpathrectangle{\pgfqpoint{0.481978in}{0.331635in}}{\pgfqpoint{4.960000in}{3.696000in}}%
\pgfusepath{clip}%
\pgfsetrectcap%
\pgfsetroundjoin%
\pgfsetlinewidth{1.505625pt}%
\definecolor{currentstroke}{rgb}{1.000000,0.705882,0.509804}%
\pgfsetstrokecolor{currentstroke}%
\pgfsetstrokeopacity{0.800000}%
\pgfsetdash{}{0pt}%
\pgfpathmoveto{\pgfqpoint{3.625583in}{1.603076in}}%
\pgfpathlineto{\pgfqpoint{2.899136in}{2.154863in}}%
\pgfusepath{stroke}%
\end{pgfscope}%
\begin{pgfscope}%
\pgfpathrectangle{\pgfqpoint{0.481978in}{0.331635in}}{\pgfqpoint{4.960000in}{3.696000in}}%
\pgfusepath{clip}%
\pgfsetrectcap%
\pgfsetroundjoin%
\pgfsetlinewidth{1.505625pt}%
\definecolor{currentstroke}{rgb}{1.000000,0.705882,0.509804}%
\pgfsetstrokecolor{currentstroke}%
\pgfsetstrokeopacity{0.800000}%
\pgfsetdash{}{0pt}%
\pgfpathmoveto{\pgfqpoint{2.731795in}{1.469894in}}%
\pgfpathlineto{\pgfqpoint{2.899136in}{2.154863in}}%
\pgfusepath{stroke}%
\end{pgfscope}%
\begin{pgfscope}%
\pgfpathrectangle{\pgfqpoint{0.481978in}{0.331635in}}{\pgfqpoint{4.960000in}{3.696000in}}%
\pgfusepath{clip}%
\pgfsetrectcap%
\pgfsetroundjoin%
\pgfsetlinewidth{1.505625pt}%
\definecolor{currentstroke}{rgb}{1.000000,0.705882,0.509804}%
\pgfsetstrokecolor{currentstroke}%
\pgfsetstrokeopacity{0.800000}%
\pgfsetdash{}{0pt}%
\pgfpathmoveto{\pgfqpoint{3.870146in}{2.527993in}}%
\pgfpathlineto{\pgfqpoint{2.899136in}{2.154863in}}%
\pgfusepath{stroke}%
\end{pgfscope}%
\begin{pgfscope}%
\pgfpathrectangle{\pgfqpoint{0.481978in}{0.331635in}}{\pgfqpoint{4.960000in}{3.696000in}}%
\pgfusepath{clip}%
\pgfsetrectcap%
\pgfsetroundjoin%
\pgfsetlinewidth{1.505625pt}%
\definecolor{currentstroke}{rgb}{1.000000,0.705882,0.509804}%
\pgfsetstrokecolor{currentstroke}%
\pgfsetstrokeopacity{0.800000}%
\pgfsetdash{}{0pt}%
\pgfpathmoveto{\pgfqpoint{3.946508in}{2.911246in}}%
\pgfpathlineto{\pgfqpoint{2.899136in}{2.154863in}}%
\pgfusepath{stroke}%
\end{pgfscope}%
\begin{pgfscope}%
\pgfpathrectangle{\pgfqpoint{0.481978in}{0.331635in}}{\pgfqpoint{4.960000in}{3.696000in}}%
\pgfusepath{clip}%
\pgfsetrectcap%
\pgfsetroundjoin%
\pgfsetlinewidth{1.505625pt}%
\definecolor{currentstroke}{rgb}{1.000000,0.705882,0.509804}%
\pgfsetstrokecolor{currentstroke}%
\pgfsetstrokeopacity{0.800000}%
\pgfsetdash{}{0pt}%
\pgfpathmoveto{\pgfqpoint{4.398894in}{1.042765in}}%
\pgfpathlineto{\pgfqpoint{2.899136in}{2.154863in}}%
\pgfusepath{stroke}%
\end{pgfscope}%
\begin{pgfscope}%
\pgfpathrectangle{\pgfqpoint{0.481978in}{0.331635in}}{\pgfqpoint{4.960000in}{3.696000in}}%
\pgfusepath{clip}%
\pgfsetrectcap%
\pgfsetroundjoin%
\pgfsetlinewidth{1.505625pt}%
\definecolor{currentstroke}{rgb}{1.000000,0.705882,0.509804}%
\pgfsetstrokecolor{currentstroke}%
\pgfsetstrokeopacity{0.800000}%
\pgfsetdash{}{0pt}%
\pgfpathmoveto{\pgfqpoint{5.211970in}{1.601639in}}%
\pgfpathlineto{\pgfqpoint{2.899136in}{2.154863in}}%
\pgfusepath{stroke}%
\end{pgfscope}%
\begin{pgfscope}%
\pgfpathrectangle{\pgfqpoint{0.481978in}{0.331635in}}{\pgfqpoint{4.960000in}{3.696000in}}%
\pgfusepath{clip}%
\pgfsetrectcap%
\pgfsetroundjoin%
\pgfsetlinewidth{1.505625pt}%
\definecolor{currentstroke}{rgb}{1.000000,0.705882,0.509804}%
\pgfsetstrokecolor{currentstroke}%
\pgfsetstrokeopacity{0.800000}%
\pgfsetdash{}{0pt}%
\pgfpathmoveto{\pgfqpoint{2.587465in}{2.987967in}}%
\pgfpathlineto{\pgfqpoint{2.899136in}{2.154863in}}%
\pgfusepath{stroke}%
\end{pgfscope}%
\begin{pgfscope}%
\pgfpathrectangle{\pgfqpoint{0.481978in}{0.331635in}}{\pgfqpoint{4.960000in}{3.696000in}}%
\pgfusepath{clip}%
\pgfsetrectcap%
\pgfsetroundjoin%
\pgfsetlinewidth{1.505625pt}%
\definecolor{currentstroke}{rgb}{1.000000,0.705882,0.509804}%
\pgfsetstrokecolor{currentstroke}%
\pgfsetstrokeopacity{0.800000}%
\pgfsetdash{}{0pt}%
\pgfpathmoveto{\pgfqpoint{3.371501in}{3.487667in}}%
\pgfpathlineto{\pgfqpoint{2.899136in}{2.154863in}}%
\pgfusepath{stroke}%
\end{pgfscope}%
\begin{pgfscope}%
\pgfpathrectangle{\pgfqpoint{0.481978in}{0.331635in}}{\pgfqpoint{4.960000in}{3.696000in}}%
\pgfusepath{clip}%
\pgfsetrectcap%
\pgfsetroundjoin%
\pgfsetlinewidth{1.505625pt}%
\definecolor{currentstroke}{rgb}{1.000000,0.705882,0.509804}%
\pgfsetstrokecolor{currentstroke}%
\pgfsetstrokeopacity{0.800000}%
\pgfsetdash{}{0pt}%
\pgfpathmoveto{\pgfqpoint{1.618240in}{2.617568in}}%
\pgfpathlineto{\pgfqpoint{2.899136in}{2.154863in}}%
\pgfusepath{stroke}%
\end{pgfscope}%
\begin{pgfscope}%
\pgfpathrectangle{\pgfqpoint{0.481978in}{0.331635in}}{\pgfqpoint{4.960000in}{3.696000in}}%
\pgfusepath{clip}%
\pgfsetrectcap%
\pgfsetroundjoin%
\pgfsetlinewidth{1.505625pt}%
\definecolor{currentstroke}{rgb}{1.000000,0.705882,0.509804}%
\pgfsetstrokecolor{currentstroke}%
\pgfsetstrokeopacity{0.800000}%
\pgfsetdash{}{0pt}%
\pgfpathmoveto{\pgfqpoint{2.745120in}{2.670898in}}%
\pgfpathlineto{\pgfqpoint{2.899136in}{2.154863in}}%
\pgfusepath{stroke}%
\end{pgfscope}%
\begin{pgfscope}%
\pgfpathrectangle{\pgfqpoint{0.481978in}{0.331635in}}{\pgfqpoint{4.960000in}{3.696000in}}%
\pgfusepath{clip}%
\pgfsetrectcap%
\pgfsetroundjoin%
\pgfsetlinewidth{1.505625pt}%
\definecolor{currentstroke}{rgb}{1.000000,0.705882,0.509804}%
\pgfsetstrokecolor{currentstroke}%
\pgfsetstrokeopacity{0.800000}%
\pgfsetdash{}{0pt}%
\pgfpathmoveto{\pgfqpoint{4.493868in}{2.862687in}}%
\pgfpathlineto{\pgfqpoint{2.899136in}{2.154863in}}%
\pgfusepath{stroke}%
\end{pgfscope}%
\begin{pgfscope}%
\pgfpathrectangle{\pgfqpoint{0.481978in}{0.331635in}}{\pgfqpoint{4.960000in}{3.696000in}}%
\pgfusepath{clip}%
\pgfsetrectcap%
\pgfsetroundjoin%
\pgfsetlinewidth{1.505625pt}%
\definecolor{currentstroke}{rgb}{1.000000,0.705882,0.509804}%
\pgfsetstrokecolor{currentstroke}%
\pgfsetstrokeopacity{0.800000}%
\pgfsetdash{}{0pt}%
\pgfpathmoveto{\pgfqpoint{4.183823in}{2.319399in}}%
\pgfpathlineto{\pgfqpoint{2.899136in}{2.154863in}}%
\pgfusepath{stroke}%
\end{pgfscope}%
\begin{pgfscope}%
\pgfpathrectangle{\pgfqpoint{0.481978in}{0.331635in}}{\pgfqpoint{4.960000in}{3.696000in}}%
\pgfusepath{clip}%
\pgfsetrectcap%
\pgfsetroundjoin%
\pgfsetlinewidth{1.505625pt}%
\definecolor{currentstroke}{rgb}{1.000000,0.705882,0.509804}%
\pgfsetstrokecolor{currentstroke}%
\pgfsetstrokeopacity{0.800000}%
\pgfsetdash{}{0pt}%
\pgfpathmoveto{\pgfqpoint{4.457059in}{2.823401in}}%
\pgfpathlineto{\pgfqpoint{2.899136in}{2.154863in}}%
\pgfusepath{stroke}%
\end{pgfscope}%
\begin{pgfscope}%
\pgfpathrectangle{\pgfqpoint{0.481978in}{0.331635in}}{\pgfqpoint{4.960000in}{3.696000in}}%
\pgfusepath{clip}%
\pgfsetrectcap%
\pgfsetroundjoin%
\pgfsetlinewidth{1.505625pt}%
\definecolor{currentstroke}{rgb}{1.000000,0.705882,0.509804}%
\pgfsetstrokecolor{currentstroke}%
\pgfsetstrokeopacity{0.800000}%
\pgfsetdash{}{0pt}%
\pgfpathmoveto{\pgfqpoint{0.873344in}{1.918739in}}%
\pgfpathlineto{\pgfqpoint{2.899136in}{2.154863in}}%
\pgfusepath{stroke}%
\end{pgfscope}%
\begin{pgfscope}%
\pgfpathrectangle{\pgfqpoint{0.481978in}{0.331635in}}{\pgfqpoint{4.960000in}{3.696000in}}%
\pgfusepath{clip}%
\pgfsetrectcap%
\pgfsetroundjoin%
\pgfsetlinewidth{1.505625pt}%
\definecolor{currentstroke}{rgb}{1.000000,0.705882,0.509804}%
\pgfsetstrokecolor{currentstroke}%
\pgfsetstrokeopacity{0.800000}%
\pgfsetdash{}{0pt}%
\pgfpathmoveto{\pgfqpoint{2.384182in}{1.994069in}}%
\pgfpathlineto{\pgfqpoint{2.899136in}{2.154863in}}%
\pgfusepath{stroke}%
\end{pgfscope}%
\begin{pgfscope}%
\pgfpathrectangle{\pgfqpoint{0.481978in}{0.331635in}}{\pgfqpoint{4.960000in}{3.696000in}}%
\pgfusepath{clip}%
\pgfsetrectcap%
\pgfsetroundjoin%
\pgfsetlinewidth{1.505625pt}%
\definecolor{currentstroke}{rgb}{1.000000,0.705882,0.509804}%
\pgfsetstrokecolor{currentstroke}%
\pgfsetstrokeopacity{0.800000}%
\pgfsetdash{}{0pt}%
\pgfpathmoveto{\pgfqpoint{4.149641in}{3.143980in}}%
\pgfpathlineto{\pgfqpoint{2.899136in}{2.154863in}}%
\pgfusepath{stroke}%
\end{pgfscope}%
\begin{pgfscope}%
\pgfpathrectangle{\pgfqpoint{0.481978in}{0.331635in}}{\pgfqpoint{4.960000in}{3.696000in}}%
\pgfusepath{clip}%
\pgfsetrectcap%
\pgfsetroundjoin%
\pgfsetlinewidth{1.505625pt}%
\definecolor{currentstroke}{rgb}{1.000000,0.705882,0.509804}%
\pgfsetstrokecolor{currentstroke}%
\pgfsetstrokeopacity{0.800000}%
\pgfsetdash{}{0pt}%
\pgfpathmoveto{\pgfqpoint{3.033751in}{1.741482in}}%
\pgfpathlineto{\pgfqpoint{2.899136in}{2.154863in}}%
\pgfusepath{stroke}%
\end{pgfscope}%
\begin{pgfscope}%
\pgfpathrectangle{\pgfqpoint{0.481978in}{0.331635in}}{\pgfqpoint{4.960000in}{3.696000in}}%
\pgfusepath{clip}%
\pgfsetrectcap%
\pgfsetroundjoin%
\pgfsetlinewidth{1.505625pt}%
\definecolor{currentstroke}{rgb}{1.000000,0.705882,0.509804}%
\pgfsetstrokecolor{currentstroke}%
\pgfsetstrokeopacity{0.800000}%
\pgfsetdash{}{0pt}%
\pgfpathmoveto{\pgfqpoint{2.729878in}{1.128776in}}%
\pgfpathlineto{\pgfqpoint{2.899136in}{2.154863in}}%
\pgfusepath{stroke}%
\end{pgfscope}%
\begin{pgfscope}%
\pgfpathrectangle{\pgfqpoint{0.481978in}{0.331635in}}{\pgfqpoint{4.960000in}{3.696000in}}%
\pgfusepath{clip}%
\pgfsetrectcap%
\pgfsetroundjoin%
\pgfsetlinewidth{1.505625pt}%
\definecolor{currentstroke}{rgb}{1.000000,0.705882,0.509804}%
\pgfsetstrokecolor{currentstroke}%
\pgfsetstrokeopacity{0.800000}%
\pgfsetdash{}{0pt}%
\pgfpathmoveto{\pgfqpoint{1.117610in}{2.112984in}}%
\pgfpathlineto{\pgfqpoint{2.899136in}{2.154863in}}%
\pgfusepath{stroke}%
\end{pgfscope}%
\begin{pgfscope}%
\pgfpathrectangle{\pgfqpoint{0.481978in}{0.331635in}}{\pgfqpoint{4.960000in}{3.696000in}}%
\pgfusepath{clip}%
\pgfsetrectcap%
\pgfsetroundjoin%
\pgfsetlinewidth{1.505625pt}%
\definecolor{currentstroke}{rgb}{1.000000,0.705882,0.509804}%
\pgfsetstrokecolor{currentstroke}%
\pgfsetstrokeopacity{0.800000}%
\pgfsetdash{}{0pt}%
\pgfpathmoveto{\pgfqpoint{4.325352in}{1.226511in}}%
\pgfpathlineto{\pgfqpoint{2.899136in}{2.154863in}}%
\pgfusepath{stroke}%
\end{pgfscope}%
\begin{pgfscope}%
\pgfpathrectangle{\pgfqpoint{0.481978in}{0.331635in}}{\pgfqpoint{4.960000in}{3.696000in}}%
\pgfusepath{clip}%
\pgfsetrectcap%
\pgfsetroundjoin%
\pgfsetlinewidth{1.505625pt}%
\definecolor{currentstroke}{rgb}{1.000000,0.705882,0.509804}%
\pgfsetstrokecolor{currentstroke}%
\pgfsetstrokeopacity{0.800000}%
\pgfsetdash{}{0pt}%
\pgfpathmoveto{\pgfqpoint{3.380981in}{3.329967in}}%
\pgfpathlineto{\pgfqpoint{2.899136in}{2.154863in}}%
\pgfusepath{stroke}%
\end{pgfscope}%
\begin{pgfscope}%
\pgfpathrectangle{\pgfqpoint{0.481978in}{0.331635in}}{\pgfqpoint{4.960000in}{3.696000in}}%
\pgfusepath{clip}%
\pgfsetrectcap%
\pgfsetroundjoin%
\pgfsetlinewidth{1.505625pt}%
\definecolor{currentstroke}{rgb}{1.000000,0.705882,0.509804}%
\pgfsetstrokecolor{currentstroke}%
\pgfsetstrokeopacity{0.800000}%
\pgfsetdash{}{0pt}%
\pgfpathmoveto{\pgfqpoint{3.963164in}{0.944268in}}%
\pgfpathlineto{\pgfqpoint{2.899136in}{2.154863in}}%
\pgfusepath{stroke}%
\end{pgfscope}%
\begin{pgfscope}%
\pgfpathrectangle{\pgfqpoint{0.481978in}{0.331635in}}{\pgfqpoint{4.960000in}{3.696000in}}%
\pgfusepath{clip}%
\pgfsetrectcap%
\pgfsetroundjoin%
\pgfsetlinewidth{1.505625pt}%
\definecolor{currentstroke}{rgb}{1.000000,0.705882,0.509804}%
\pgfsetstrokecolor{currentstroke}%
\pgfsetstrokeopacity{0.800000}%
\pgfsetdash{}{0pt}%
\pgfpathmoveto{\pgfqpoint{4.651146in}{2.565615in}}%
\pgfpathlineto{\pgfqpoint{2.899136in}{2.154863in}}%
\pgfusepath{stroke}%
\end{pgfscope}%
\begin{pgfscope}%
\pgfpathrectangle{\pgfqpoint{0.481978in}{0.331635in}}{\pgfqpoint{4.960000in}{3.696000in}}%
\pgfusepath{clip}%
\pgfsetrectcap%
\pgfsetroundjoin%
\pgfsetlinewidth{1.505625pt}%
\definecolor{currentstroke}{rgb}{1.000000,0.705882,0.509804}%
\pgfsetstrokecolor{currentstroke}%
\pgfsetstrokeopacity{0.800000}%
\pgfsetdash{}{0pt}%
\pgfpathmoveto{\pgfqpoint{2.366333in}{3.382994in}}%
\pgfpathlineto{\pgfqpoint{2.899136in}{2.154863in}}%
\pgfusepath{stroke}%
\end{pgfscope}%
\begin{pgfscope}%
\pgfpathrectangle{\pgfqpoint{0.481978in}{0.331635in}}{\pgfqpoint{4.960000in}{3.696000in}}%
\pgfusepath{clip}%
\pgfsetrectcap%
\pgfsetroundjoin%
\pgfsetlinewidth{1.505625pt}%
\definecolor{currentstroke}{rgb}{1.000000,0.705882,0.509804}%
\pgfsetstrokecolor{currentstroke}%
\pgfsetstrokeopacity{0.800000}%
\pgfsetdash{}{0pt}%
\pgfpathmoveto{\pgfqpoint{3.263269in}{1.829179in}}%
\pgfpathlineto{\pgfqpoint{2.899136in}{2.154863in}}%
\pgfusepath{stroke}%
\end{pgfscope}%
\begin{pgfscope}%
\pgfpathrectangle{\pgfqpoint{0.481978in}{0.331635in}}{\pgfqpoint{4.960000in}{3.696000in}}%
\pgfusepath{clip}%
\pgfsetrectcap%
\pgfsetroundjoin%
\pgfsetlinewidth{1.505625pt}%
\definecolor{currentstroke}{rgb}{1.000000,0.705882,0.509804}%
\pgfsetstrokecolor{currentstroke}%
\pgfsetstrokeopacity{0.800000}%
\pgfsetdash{}{0pt}%
\pgfpathmoveto{\pgfqpoint{3.402580in}{1.677203in}}%
\pgfpathlineto{\pgfqpoint{2.899136in}{2.154863in}}%
\pgfusepath{stroke}%
\end{pgfscope}%
\begin{pgfscope}%
\pgfpathrectangle{\pgfqpoint{0.481978in}{0.331635in}}{\pgfqpoint{4.960000in}{3.696000in}}%
\pgfusepath{clip}%
\pgfsetrectcap%
\pgfsetroundjoin%
\pgfsetlinewidth{1.505625pt}%
\definecolor{currentstroke}{rgb}{1.000000,0.705882,0.509804}%
\pgfsetstrokecolor{currentstroke}%
\pgfsetstrokeopacity{0.800000}%
\pgfsetdash{}{0pt}%
\pgfpathmoveto{\pgfqpoint{2.928313in}{2.761414in}}%
\pgfpathlineto{\pgfqpoint{2.899136in}{2.154863in}}%
\pgfusepath{stroke}%
\end{pgfscope}%
\begin{pgfscope}%
\pgfpathrectangle{\pgfqpoint{0.481978in}{0.331635in}}{\pgfqpoint{4.960000in}{3.696000in}}%
\pgfusepath{clip}%
\pgfsetrectcap%
\pgfsetroundjoin%
\pgfsetlinewidth{1.505625pt}%
\definecolor{currentstroke}{rgb}{1.000000,0.705882,0.509804}%
\pgfsetstrokecolor{currentstroke}%
\pgfsetstrokeopacity{0.800000}%
\pgfsetdash{}{0pt}%
\pgfpathmoveto{\pgfqpoint{4.201586in}{0.899466in}}%
\pgfpathlineto{\pgfqpoint{2.899136in}{2.154863in}}%
\pgfusepath{stroke}%
\end{pgfscope}%
\begin{pgfscope}%
\pgfpathrectangle{\pgfqpoint{0.481978in}{0.331635in}}{\pgfqpoint{4.960000in}{3.696000in}}%
\pgfusepath{clip}%
\pgfsetrectcap%
\pgfsetroundjoin%
\pgfsetlinewidth{1.505625pt}%
\definecolor{currentstroke}{rgb}{1.000000,0.705882,0.509804}%
\pgfsetstrokecolor{currentstroke}%
\pgfsetstrokeopacity{0.800000}%
\pgfsetdash{}{0pt}%
\pgfpathmoveto{\pgfqpoint{2.639691in}{1.041280in}}%
\pgfpathlineto{\pgfqpoint{2.899136in}{2.154863in}}%
\pgfusepath{stroke}%
\end{pgfscope}%
\begin{pgfscope}%
\pgfpathrectangle{\pgfqpoint{0.481978in}{0.331635in}}{\pgfqpoint{4.960000in}{3.696000in}}%
\pgfusepath{clip}%
\pgfsetrectcap%
\pgfsetroundjoin%
\pgfsetlinewidth{1.505625pt}%
\definecolor{currentstroke}{rgb}{1.000000,0.705882,0.509804}%
\pgfsetstrokecolor{currentstroke}%
\pgfsetstrokeopacity{0.800000}%
\pgfsetdash{}{0pt}%
\pgfpathmoveto{\pgfqpoint{2.932478in}{1.659845in}}%
\pgfpathlineto{\pgfqpoint{2.899136in}{2.154863in}}%
\pgfusepath{stroke}%
\end{pgfscope}%
\begin{pgfscope}%
\pgfpathrectangle{\pgfqpoint{0.481978in}{0.331635in}}{\pgfqpoint{4.960000in}{3.696000in}}%
\pgfusepath{clip}%
\pgfsetrectcap%
\pgfsetroundjoin%
\pgfsetlinewidth{1.505625pt}%
\definecolor{currentstroke}{rgb}{1.000000,0.705882,0.509804}%
\pgfsetstrokecolor{currentstroke}%
\pgfsetstrokeopacity{0.800000}%
\pgfsetdash{}{0pt}%
\pgfpathmoveto{\pgfqpoint{4.484479in}{2.980378in}}%
\pgfpathlineto{\pgfqpoint{2.899136in}{2.154863in}}%
\pgfusepath{stroke}%
\end{pgfscope}%
\begin{pgfscope}%
\pgfpathrectangle{\pgfqpoint{0.481978in}{0.331635in}}{\pgfqpoint{4.960000in}{3.696000in}}%
\pgfusepath{clip}%
\pgfsetrectcap%
\pgfsetroundjoin%
\pgfsetlinewidth{1.505625pt}%
\definecolor{currentstroke}{rgb}{1.000000,0.705882,0.509804}%
\pgfsetstrokecolor{currentstroke}%
\pgfsetstrokeopacity{0.800000}%
\pgfsetdash{}{0pt}%
\pgfpathmoveto{\pgfqpoint{1.750152in}{1.366890in}}%
\pgfpathlineto{\pgfqpoint{2.899136in}{2.154863in}}%
\pgfusepath{stroke}%
\end{pgfscope}%
\begin{pgfscope}%
\pgfpathrectangle{\pgfqpoint{0.481978in}{0.331635in}}{\pgfqpoint{4.960000in}{3.696000in}}%
\pgfusepath{clip}%
\pgfsetrectcap%
\pgfsetroundjoin%
\pgfsetlinewidth{1.505625pt}%
\definecolor{currentstroke}{rgb}{1.000000,0.705882,0.509804}%
\pgfsetstrokecolor{currentstroke}%
\pgfsetstrokeopacity{0.800000}%
\pgfsetdash{}{0pt}%
\pgfpathmoveto{\pgfqpoint{4.368323in}{1.083711in}}%
\pgfpathlineto{\pgfqpoint{2.899136in}{2.154863in}}%
\pgfusepath{stroke}%
\end{pgfscope}%
\begin{pgfscope}%
\pgfpathrectangle{\pgfqpoint{0.481978in}{0.331635in}}{\pgfqpoint{4.960000in}{3.696000in}}%
\pgfusepath{clip}%
\pgfsetrectcap%
\pgfsetroundjoin%
\pgfsetlinewidth{1.505625pt}%
\definecolor{currentstroke}{rgb}{1.000000,0.705882,0.509804}%
\pgfsetstrokecolor{currentstroke}%
\pgfsetstrokeopacity{0.800000}%
\pgfsetdash{}{0pt}%
\pgfpathmoveto{\pgfqpoint{2.870476in}{0.499635in}}%
\pgfpathlineto{\pgfqpoint{2.899136in}{2.154863in}}%
\pgfusepath{stroke}%
\end{pgfscope}%
\begin{pgfscope}%
\pgfpathrectangle{\pgfqpoint{0.481978in}{0.331635in}}{\pgfqpoint{4.960000in}{3.696000in}}%
\pgfusepath{clip}%
\pgfsetrectcap%
\pgfsetroundjoin%
\pgfsetlinewidth{1.505625pt}%
\definecolor{currentstroke}{rgb}{1.000000,0.705882,0.509804}%
\pgfsetstrokecolor{currentstroke}%
\pgfsetstrokeopacity{0.800000}%
\pgfsetdash{}{0pt}%
\pgfpathmoveto{\pgfqpoint{4.249938in}{2.939822in}}%
\pgfpathlineto{\pgfqpoint{2.899136in}{2.154863in}}%
\pgfusepath{stroke}%
\end{pgfscope}%
\begin{pgfscope}%
\pgfpathrectangle{\pgfqpoint{0.481978in}{0.331635in}}{\pgfqpoint{4.960000in}{3.696000in}}%
\pgfusepath{clip}%
\pgfsetrectcap%
\pgfsetroundjoin%
\pgfsetlinewidth{1.505625pt}%
\definecolor{currentstroke}{rgb}{1.000000,0.705882,0.509804}%
\pgfsetstrokecolor{currentstroke}%
\pgfsetstrokeopacity{0.800000}%
\pgfsetdash{}{0pt}%
\pgfpathmoveto{\pgfqpoint{3.827281in}{1.314450in}}%
\pgfpathlineto{\pgfqpoint{2.899136in}{2.154863in}}%
\pgfusepath{stroke}%
\end{pgfscope}%
\begin{pgfscope}%
\pgfpathrectangle{\pgfqpoint{0.481978in}{0.331635in}}{\pgfqpoint{4.960000in}{3.696000in}}%
\pgfusepath{clip}%
\pgfsetrectcap%
\pgfsetroundjoin%
\pgfsetlinewidth{1.505625pt}%
\definecolor{currentstroke}{rgb}{1.000000,0.705882,0.509804}%
\pgfsetstrokecolor{currentstroke}%
\pgfsetstrokeopacity{0.800000}%
\pgfsetdash{}{0pt}%
\pgfpathmoveto{\pgfqpoint{3.454050in}{0.942524in}}%
\pgfpathlineto{\pgfqpoint{2.899136in}{2.154863in}}%
\pgfusepath{stroke}%
\end{pgfscope}%
\begin{pgfscope}%
\pgfpathrectangle{\pgfqpoint{0.481978in}{0.331635in}}{\pgfqpoint{4.960000in}{3.696000in}}%
\pgfusepath{clip}%
\pgfsetrectcap%
\pgfsetroundjoin%
\pgfsetlinewidth{1.505625pt}%
\definecolor{currentstroke}{rgb}{1.000000,0.705882,0.509804}%
\pgfsetstrokecolor{currentstroke}%
\pgfsetstrokeopacity{0.800000}%
\pgfsetdash{}{0pt}%
\pgfpathmoveto{\pgfqpoint{1.042275in}{2.437049in}}%
\pgfpathlineto{\pgfqpoint{2.899136in}{2.154863in}}%
\pgfusepath{stroke}%
\end{pgfscope}%
\begin{pgfscope}%
\pgfpathrectangle{\pgfqpoint{0.481978in}{0.331635in}}{\pgfqpoint{4.960000in}{3.696000in}}%
\pgfusepath{clip}%
\pgfsetrectcap%
\pgfsetroundjoin%
\pgfsetlinewidth{1.505625pt}%
\definecolor{currentstroke}{rgb}{1.000000,0.705882,0.509804}%
\pgfsetstrokecolor{currentstroke}%
\pgfsetstrokeopacity{0.800000}%
\pgfsetdash{}{0pt}%
\pgfpathmoveto{\pgfqpoint{3.907740in}{3.232882in}}%
\pgfpathlineto{\pgfqpoint{2.899136in}{2.154863in}}%
\pgfusepath{stroke}%
\end{pgfscope}%
\begin{pgfscope}%
\pgfpathrectangle{\pgfqpoint{0.481978in}{0.331635in}}{\pgfqpoint{4.960000in}{3.696000in}}%
\pgfusepath{clip}%
\pgfsetrectcap%
\pgfsetroundjoin%
\pgfsetlinewidth{1.505625pt}%
\definecolor{currentstroke}{rgb}{1.000000,0.705882,0.509804}%
\pgfsetstrokecolor{currentstroke}%
\pgfsetstrokeopacity{0.800000}%
\pgfsetdash{}{0pt}%
\pgfpathmoveto{\pgfqpoint{3.085391in}{1.012703in}}%
\pgfpathlineto{\pgfqpoint{2.899136in}{2.154863in}}%
\pgfusepath{stroke}%
\end{pgfscope}%
\begin{pgfscope}%
\pgfpathrectangle{\pgfqpoint{0.481978in}{0.331635in}}{\pgfqpoint{4.960000in}{3.696000in}}%
\pgfusepath{clip}%
\pgfsetrectcap%
\pgfsetroundjoin%
\pgfsetlinewidth{1.505625pt}%
\definecolor{currentstroke}{rgb}{1.000000,0.705882,0.509804}%
\pgfsetstrokecolor{currentstroke}%
\pgfsetstrokeopacity{0.800000}%
\pgfsetdash{}{0pt}%
\pgfpathmoveto{\pgfqpoint{3.219219in}{0.756356in}}%
\pgfpathlineto{\pgfqpoint{2.899136in}{2.154863in}}%
\pgfusepath{stroke}%
\end{pgfscope}%
\begin{pgfscope}%
\pgfpathrectangle{\pgfqpoint{0.481978in}{0.331635in}}{\pgfqpoint{4.960000in}{3.696000in}}%
\pgfusepath{clip}%
\pgfsetrectcap%
\pgfsetroundjoin%
\pgfsetlinewidth{1.505625pt}%
\definecolor{currentstroke}{rgb}{1.000000,0.705882,0.509804}%
\pgfsetstrokecolor{currentstroke}%
\pgfsetstrokeopacity{0.800000}%
\pgfsetdash{}{0pt}%
\pgfpathmoveto{\pgfqpoint{1.298048in}{3.116137in}}%
\pgfpathlineto{\pgfqpoint{2.899136in}{2.154863in}}%
\pgfusepath{stroke}%
\end{pgfscope}%
\begin{pgfscope}%
\pgfpathrectangle{\pgfqpoint{0.481978in}{0.331635in}}{\pgfqpoint{4.960000in}{3.696000in}}%
\pgfusepath{clip}%
\pgfsetrectcap%
\pgfsetroundjoin%
\pgfsetlinewidth{1.505625pt}%
\definecolor{currentstroke}{rgb}{1.000000,0.705882,0.509804}%
\pgfsetstrokecolor{currentstroke}%
\pgfsetstrokeopacity{0.800000}%
\pgfsetdash{}{0pt}%
\pgfpathmoveto{\pgfqpoint{0.816826in}{2.057432in}}%
\pgfpathlineto{\pgfqpoint{2.899136in}{2.154863in}}%
\pgfusepath{stroke}%
\end{pgfscope}%
\begin{pgfscope}%
\pgfpathrectangle{\pgfqpoint{0.481978in}{0.331635in}}{\pgfqpoint{4.960000in}{3.696000in}}%
\pgfusepath{clip}%
\pgfsetrectcap%
\pgfsetroundjoin%
\pgfsetlinewidth{1.505625pt}%
\definecolor{currentstroke}{rgb}{1.000000,0.705882,0.509804}%
\pgfsetstrokecolor{currentstroke}%
\pgfsetstrokeopacity{0.800000}%
\pgfsetdash{}{0pt}%
\pgfpathmoveto{\pgfqpoint{0.744844in}{2.590106in}}%
\pgfpathlineto{\pgfqpoint{2.899136in}{2.154863in}}%
\pgfusepath{stroke}%
\end{pgfscope}%
\begin{pgfscope}%
\pgfpathrectangle{\pgfqpoint{0.481978in}{0.331635in}}{\pgfqpoint{4.960000in}{3.696000in}}%
\pgfusepath{clip}%
\pgfsetrectcap%
\pgfsetroundjoin%
\pgfsetlinewidth{1.505625pt}%
\definecolor{currentstroke}{rgb}{1.000000,0.705882,0.509804}%
\pgfsetstrokecolor{currentstroke}%
\pgfsetstrokeopacity{0.800000}%
\pgfsetdash{}{0pt}%
\pgfpathmoveto{\pgfqpoint{2.676035in}{3.439941in}}%
\pgfpathlineto{\pgfqpoint{2.899136in}{2.154863in}}%
\pgfusepath{stroke}%
\end{pgfscope}%
\begin{pgfscope}%
\pgfpathrectangle{\pgfqpoint{0.481978in}{0.331635in}}{\pgfqpoint{4.960000in}{3.696000in}}%
\pgfusepath{clip}%
\pgfsetrectcap%
\pgfsetroundjoin%
\pgfsetlinewidth{1.505625pt}%
\definecolor{currentstroke}{rgb}{1.000000,0.705882,0.509804}%
\pgfsetstrokecolor{currentstroke}%
\pgfsetstrokeopacity{0.800000}%
\pgfsetdash{}{0pt}%
\pgfpathmoveto{\pgfqpoint{3.635940in}{3.298251in}}%
\pgfpathlineto{\pgfqpoint{2.899136in}{2.154863in}}%
\pgfusepath{stroke}%
\end{pgfscope}%
\begin{pgfscope}%
\pgfpathrectangle{\pgfqpoint{0.481978in}{0.331635in}}{\pgfqpoint{4.960000in}{3.696000in}}%
\pgfusepath{clip}%
\pgfsetrectcap%
\pgfsetroundjoin%
\pgfsetlinewidth{1.505625pt}%
\definecolor{currentstroke}{rgb}{1.000000,0.705882,0.509804}%
\pgfsetstrokecolor{currentstroke}%
\pgfsetstrokeopacity{0.800000}%
\pgfsetdash{}{0pt}%
\pgfpathmoveto{\pgfqpoint{2.951945in}{2.111183in}}%
\pgfpathlineto{\pgfqpoint{2.899136in}{2.154863in}}%
\pgfusepath{stroke}%
\end{pgfscope}%
\begin{pgfscope}%
\pgfpathrectangle{\pgfqpoint{0.481978in}{0.331635in}}{\pgfqpoint{4.960000in}{3.696000in}}%
\pgfusepath{clip}%
\pgfsetrectcap%
\pgfsetroundjoin%
\pgfsetlinewidth{1.505625pt}%
\definecolor{currentstroke}{rgb}{1.000000,0.705882,0.509804}%
\pgfsetstrokecolor{currentstroke}%
\pgfsetstrokeopacity{0.800000}%
\pgfsetdash{}{0pt}%
\pgfpathmoveto{\pgfqpoint{2.897572in}{1.352784in}}%
\pgfpathlineto{\pgfqpoint{2.899136in}{2.154863in}}%
\pgfusepath{stroke}%
\end{pgfscope}%
\begin{pgfscope}%
\pgfpathrectangle{\pgfqpoint{0.481978in}{0.331635in}}{\pgfqpoint{4.960000in}{3.696000in}}%
\pgfusepath{clip}%
\pgfsetrectcap%
\pgfsetroundjoin%
\pgfsetlinewidth{1.505625pt}%
\definecolor{currentstroke}{rgb}{1.000000,0.705882,0.509804}%
\pgfsetstrokecolor{currentstroke}%
\pgfsetstrokeopacity{0.800000}%
\pgfsetdash{}{0pt}%
\pgfpathmoveto{\pgfqpoint{4.132208in}{3.129656in}}%
\pgfpathlineto{\pgfqpoint{2.899136in}{2.154863in}}%
\pgfusepath{stroke}%
\end{pgfscope}%
\begin{pgfscope}%
\pgfpathrectangle{\pgfqpoint{0.481978in}{0.331635in}}{\pgfqpoint{4.960000in}{3.696000in}}%
\pgfusepath{clip}%
\pgfsetrectcap%
\pgfsetroundjoin%
\pgfsetlinewidth{1.505625pt}%
\definecolor{currentstroke}{rgb}{1.000000,0.705882,0.509804}%
\pgfsetstrokecolor{currentstroke}%
\pgfsetstrokeopacity{0.800000}%
\pgfsetdash{}{0pt}%
\pgfpathmoveto{\pgfqpoint{3.639008in}{2.620835in}}%
\pgfpathlineto{\pgfqpoint{2.899136in}{2.154863in}}%
\pgfusepath{stroke}%
\end{pgfscope}%
\begin{pgfscope}%
\pgfpathrectangle{\pgfqpoint{0.481978in}{0.331635in}}{\pgfqpoint{4.960000in}{3.696000in}}%
\pgfusepath{clip}%
\pgfsetrectcap%
\pgfsetroundjoin%
\pgfsetlinewidth{1.505625pt}%
\definecolor{currentstroke}{rgb}{1.000000,0.705882,0.509804}%
\pgfsetstrokecolor{currentstroke}%
\pgfsetstrokeopacity{0.800000}%
\pgfsetdash{}{0pt}%
\pgfpathmoveto{\pgfqpoint{2.882105in}{1.935118in}}%
\pgfpathlineto{\pgfqpoint{2.899136in}{2.154863in}}%
\pgfusepath{stroke}%
\end{pgfscope}%
\begin{pgfscope}%
\pgfpathrectangle{\pgfqpoint{0.481978in}{0.331635in}}{\pgfqpoint{4.960000in}{3.696000in}}%
\pgfusepath{clip}%
\pgfsetrectcap%
\pgfsetroundjoin%
\pgfsetlinewidth{1.505625pt}%
\definecolor{currentstroke}{rgb}{1.000000,0.705882,0.509804}%
\pgfsetstrokecolor{currentstroke}%
\pgfsetstrokeopacity{0.800000}%
\pgfsetdash{}{0pt}%
\pgfpathmoveto{\pgfqpoint{3.145500in}{1.546928in}}%
\pgfpathlineto{\pgfqpoint{2.899136in}{2.154863in}}%
\pgfusepath{stroke}%
\end{pgfscope}%
\begin{pgfscope}%
\pgfpathrectangle{\pgfqpoint{0.481978in}{0.331635in}}{\pgfqpoint{4.960000in}{3.696000in}}%
\pgfusepath{clip}%
\pgfsetrectcap%
\pgfsetroundjoin%
\pgfsetlinewidth{1.505625pt}%
\definecolor{currentstroke}{rgb}{1.000000,0.705882,0.509804}%
\pgfsetstrokecolor{currentstroke}%
\pgfsetstrokeopacity{0.800000}%
\pgfsetdash{}{0pt}%
\pgfpathmoveto{\pgfqpoint{2.747214in}{2.713391in}}%
\pgfpathlineto{\pgfqpoint{2.899136in}{2.154863in}}%
\pgfusepath{stroke}%
\end{pgfscope}%
\begin{pgfscope}%
\pgfpathrectangle{\pgfqpoint{0.481978in}{0.331635in}}{\pgfqpoint{4.960000in}{3.696000in}}%
\pgfusepath{clip}%
\pgfsetrectcap%
\pgfsetroundjoin%
\pgfsetlinewidth{1.505625pt}%
\definecolor{currentstroke}{rgb}{1.000000,0.705882,0.509804}%
\pgfsetstrokecolor{currentstroke}%
\pgfsetstrokeopacity{0.800000}%
\pgfsetdash{}{0pt}%
\pgfpathmoveto{\pgfqpoint{4.181119in}{2.748879in}}%
\pgfpathlineto{\pgfqpoint{2.899136in}{2.154863in}}%
\pgfusepath{stroke}%
\end{pgfscope}%
\begin{pgfscope}%
\pgfpathrectangle{\pgfqpoint{0.481978in}{0.331635in}}{\pgfqpoint{4.960000in}{3.696000in}}%
\pgfusepath{clip}%
\pgfsetrectcap%
\pgfsetroundjoin%
\pgfsetlinewidth{1.505625pt}%
\definecolor{currentstroke}{rgb}{0.631373,0.788235,0.956863}%
\pgfsetstrokecolor{currentstroke}%
\pgfsetstrokeopacity{0.200000}%
\pgfsetdash{}{0pt}%
\pgfpathmoveto{\pgfqpoint{1.411682in}{2.625930in}}%
\pgfpathlineto{\pgfqpoint{2.639955in}{2.248424in}}%
\pgfusepath{stroke}%
\end{pgfscope}%
\begin{pgfscope}%
\pgfpathrectangle{\pgfqpoint{0.481978in}{0.331635in}}{\pgfqpoint{4.960000in}{3.696000in}}%
\pgfusepath{clip}%
\pgfsetrectcap%
\pgfsetroundjoin%
\pgfsetlinewidth{1.505625pt}%
\definecolor{currentstroke}{rgb}{0.631373,0.788235,0.956863}%
\pgfsetstrokecolor{currentstroke}%
\pgfsetstrokeopacity{0.200000}%
\pgfsetdash{}{0pt}%
\pgfpathmoveto{\pgfqpoint{1.733103in}{1.586729in}}%
\pgfpathlineto{\pgfqpoint{2.639955in}{2.248424in}}%
\pgfusepath{stroke}%
\end{pgfscope}%
\begin{pgfscope}%
\pgfpathrectangle{\pgfqpoint{0.481978in}{0.331635in}}{\pgfqpoint{4.960000in}{3.696000in}}%
\pgfusepath{clip}%
\pgfsetrectcap%
\pgfsetroundjoin%
\pgfsetlinewidth{1.505625pt}%
\definecolor{currentstroke}{rgb}{0.631373,0.788235,0.956863}%
\pgfsetstrokecolor{currentstroke}%
\pgfsetstrokeopacity{0.200000}%
\pgfsetdash{}{0pt}%
\pgfpathmoveto{\pgfqpoint{1.627229in}{1.985670in}}%
\pgfpathlineto{\pgfqpoint{2.639955in}{2.248424in}}%
\pgfusepath{stroke}%
\end{pgfscope}%
\begin{pgfscope}%
\pgfpathrectangle{\pgfqpoint{0.481978in}{0.331635in}}{\pgfqpoint{4.960000in}{3.696000in}}%
\pgfusepath{clip}%
\pgfsetrectcap%
\pgfsetroundjoin%
\pgfsetlinewidth{1.505625pt}%
\definecolor{currentstroke}{rgb}{0.631373,0.788235,0.956863}%
\pgfsetstrokecolor{currentstroke}%
\pgfsetstrokeopacity{0.200000}%
\pgfsetdash{}{0pt}%
\pgfpathmoveto{\pgfqpoint{2.033963in}{2.311577in}}%
\pgfpathlineto{\pgfqpoint{2.639955in}{2.248424in}}%
\pgfusepath{stroke}%
\end{pgfscope}%
\begin{pgfscope}%
\pgfpathrectangle{\pgfqpoint{0.481978in}{0.331635in}}{\pgfqpoint{4.960000in}{3.696000in}}%
\pgfusepath{clip}%
\pgfsetrectcap%
\pgfsetroundjoin%
\pgfsetlinewidth{1.505625pt}%
\definecolor{currentstroke}{rgb}{0.631373,0.788235,0.956863}%
\pgfsetstrokecolor{currentstroke}%
\pgfsetstrokeopacity{0.200000}%
\pgfsetdash{}{0pt}%
\pgfpathmoveto{\pgfqpoint{2.328330in}{2.358393in}}%
\pgfpathlineto{\pgfqpoint{2.639955in}{2.248424in}}%
\pgfusepath{stroke}%
\end{pgfscope}%
\begin{pgfscope}%
\pgfpathrectangle{\pgfqpoint{0.481978in}{0.331635in}}{\pgfqpoint{4.960000in}{3.696000in}}%
\pgfusepath{clip}%
\pgfsetrectcap%
\pgfsetroundjoin%
\pgfsetlinewidth{1.505625pt}%
\definecolor{currentstroke}{rgb}{0.631373,0.788235,0.956863}%
\pgfsetstrokecolor{currentstroke}%
\pgfsetstrokeopacity{0.200000}%
\pgfsetdash{}{0pt}%
\pgfpathmoveto{\pgfqpoint{4.500183in}{2.360424in}}%
\pgfpathlineto{\pgfqpoint{2.639955in}{2.248424in}}%
\pgfusepath{stroke}%
\end{pgfscope}%
\begin{pgfscope}%
\pgfpathrectangle{\pgfqpoint{0.481978in}{0.331635in}}{\pgfqpoint{4.960000in}{3.696000in}}%
\pgfusepath{clip}%
\pgfsetrectcap%
\pgfsetroundjoin%
\pgfsetlinewidth{1.505625pt}%
\definecolor{currentstroke}{rgb}{0.631373,0.788235,0.956863}%
\pgfsetstrokecolor{currentstroke}%
\pgfsetstrokeopacity{0.200000}%
\pgfsetdash{}{0pt}%
\pgfpathmoveto{\pgfqpoint{3.254374in}{1.236925in}}%
\pgfpathlineto{\pgfqpoint{2.639955in}{2.248424in}}%
\pgfusepath{stroke}%
\end{pgfscope}%
\begin{pgfscope}%
\pgfpathrectangle{\pgfqpoint{0.481978in}{0.331635in}}{\pgfqpoint{4.960000in}{3.696000in}}%
\pgfusepath{clip}%
\pgfsetrectcap%
\pgfsetroundjoin%
\pgfsetlinewidth{1.505625pt}%
\definecolor{currentstroke}{rgb}{0.631373,0.788235,0.956863}%
\pgfsetstrokecolor{currentstroke}%
\pgfsetstrokeopacity{0.200000}%
\pgfsetdash{}{0pt}%
\pgfpathmoveto{\pgfqpoint{2.136794in}{2.080234in}}%
\pgfpathlineto{\pgfqpoint{2.639955in}{2.248424in}}%
\pgfusepath{stroke}%
\end{pgfscope}%
\begin{pgfscope}%
\pgfpathrectangle{\pgfqpoint{0.481978in}{0.331635in}}{\pgfqpoint{4.960000in}{3.696000in}}%
\pgfusepath{clip}%
\pgfsetrectcap%
\pgfsetroundjoin%
\pgfsetlinewidth{1.505625pt}%
\definecolor{currentstroke}{rgb}{0.631373,0.788235,0.956863}%
\pgfsetstrokecolor{currentstroke}%
\pgfsetstrokeopacity{0.200000}%
\pgfsetdash{}{0pt}%
\pgfpathmoveto{\pgfqpoint{3.643865in}{1.105276in}}%
\pgfpathlineto{\pgfqpoint{2.639955in}{2.248424in}}%
\pgfusepath{stroke}%
\end{pgfscope}%
\begin{pgfscope}%
\pgfpathrectangle{\pgfqpoint{0.481978in}{0.331635in}}{\pgfqpoint{4.960000in}{3.696000in}}%
\pgfusepath{clip}%
\pgfsetrectcap%
\pgfsetroundjoin%
\pgfsetlinewidth{1.505625pt}%
\definecolor{currentstroke}{rgb}{0.631373,0.788235,0.956863}%
\pgfsetstrokecolor{currentstroke}%
\pgfsetstrokeopacity{0.200000}%
\pgfsetdash{}{0pt}%
\pgfpathmoveto{\pgfqpoint{3.577822in}{2.063148in}}%
\pgfpathlineto{\pgfqpoint{2.639955in}{2.248424in}}%
\pgfusepath{stroke}%
\end{pgfscope}%
\begin{pgfscope}%
\pgfpathrectangle{\pgfqpoint{0.481978in}{0.331635in}}{\pgfqpoint{4.960000in}{3.696000in}}%
\pgfusepath{clip}%
\pgfsetrectcap%
\pgfsetroundjoin%
\pgfsetlinewidth{1.505625pt}%
\definecolor{currentstroke}{rgb}{0.631373,0.788235,0.956863}%
\pgfsetstrokecolor{currentstroke}%
\pgfsetstrokeopacity{0.200000}%
\pgfsetdash{}{0pt}%
\pgfpathmoveto{\pgfqpoint{1.483591in}{2.528725in}}%
\pgfpathlineto{\pgfqpoint{2.639955in}{2.248424in}}%
\pgfusepath{stroke}%
\end{pgfscope}%
\begin{pgfscope}%
\pgfpathrectangle{\pgfqpoint{0.481978in}{0.331635in}}{\pgfqpoint{4.960000in}{3.696000in}}%
\pgfusepath{clip}%
\pgfsetrectcap%
\pgfsetroundjoin%
\pgfsetlinewidth{1.505625pt}%
\definecolor{currentstroke}{rgb}{0.631373,0.788235,0.956863}%
\pgfsetstrokecolor{currentstroke}%
\pgfsetstrokeopacity{0.200000}%
\pgfsetdash{}{0pt}%
\pgfpathmoveto{\pgfqpoint{2.576405in}{2.192251in}}%
\pgfpathlineto{\pgfqpoint{2.639955in}{2.248424in}}%
\pgfusepath{stroke}%
\end{pgfscope}%
\begin{pgfscope}%
\pgfpathrectangle{\pgfqpoint{0.481978in}{0.331635in}}{\pgfqpoint{4.960000in}{3.696000in}}%
\pgfusepath{clip}%
\pgfsetrectcap%
\pgfsetroundjoin%
\pgfsetlinewidth{1.505625pt}%
\definecolor{currentstroke}{rgb}{0.631373,0.788235,0.956863}%
\pgfsetstrokecolor{currentstroke}%
\pgfsetstrokeopacity{0.200000}%
\pgfsetdash{}{0pt}%
\pgfpathmoveto{\pgfqpoint{2.282422in}{1.547845in}}%
\pgfpathlineto{\pgfqpoint{2.639955in}{2.248424in}}%
\pgfusepath{stroke}%
\end{pgfscope}%
\begin{pgfscope}%
\pgfpathrectangle{\pgfqpoint{0.481978in}{0.331635in}}{\pgfqpoint{4.960000in}{3.696000in}}%
\pgfusepath{clip}%
\pgfsetrectcap%
\pgfsetroundjoin%
\pgfsetlinewidth{1.505625pt}%
\definecolor{currentstroke}{rgb}{0.631373,0.788235,0.956863}%
\pgfsetstrokecolor{currentstroke}%
\pgfsetstrokeopacity{0.200000}%
\pgfsetdash{}{0pt}%
\pgfpathmoveto{\pgfqpoint{1.547612in}{2.610418in}}%
\pgfpathlineto{\pgfqpoint{2.639955in}{2.248424in}}%
\pgfusepath{stroke}%
\end{pgfscope}%
\begin{pgfscope}%
\pgfpathrectangle{\pgfqpoint{0.481978in}{0.331635in}}{\pgfqpoint{4.960000in}{3.696000in}}%
\pgfusepath{clip}%
\pgfsetrectcap%
\pgfsetroundjoin%
\pgfsetlinewidth{1.505625pt}%
\definecolor{currentstroke}{rgb}{0.631373,0.788235,0.956863}%
\pgfsetstrokecolor{currentstroke}%
\pgfsetstrokeopacity{0.200000}%
\pgfsetdash{}{0pt}%
\pgfpathmoveto{\pgfqpoint{2.898875in}{2.083961in}}%
\pgfpathlineto{\pgfqpoint{2.639955in}{2.248424in}}%
\pgfusepath{stroke}%
\end{pgfscope}%
\begin{pgfscope}%
\pgfpathrectangle{\pgfqpoint{0.481978in}{0.331635in}}{\pgfqpoint{4.960000in}{3.696000in}}%
\pgfusepath{clip}%
\pgfsetrectcap%
\pgfsetroundjoin%
\pgfsetlinewidth{1.505625pt}%
\definecolor{currentstroke}{rgb}{0.631373,0.788235,0.956863}%
\pgfsetstrokecolor{currentstroke}%
\pgfsetstrokeopacity{0.200000}%
\pgfsetdash{}{0pt}%
\pgfpathmoveto{\pgfqpoint{2.562342in}{2.457766in}}%
\pgfpathlineto{\pgfqpoint{2.639955in}{2.248424in}}%
\pgfusepath{stroke}%
\end{pgfscope}%
\begin{pgfscope}%
\pgfpathrectangle{\pgfqpoint{0.481978in}{0.331635in}}{\pgfqpoint{4.960000in}{3.696000in}}%
\pgfusepath{clip}%
\pgfsetrectcap%
\pgfsetroundjoin%
\pgfsetlinewidth{1.505625pt}%
\definecolor{currentstroke}{rgb}{0.631373,0.788235,0.956863}%
\pgfsetstrokecolor{currentstroke}%
\pgfsetstrokeopacity{0.200000}%
\pgfsetdash{}{0pt}%
\pgfpathmoveto{\pgfqpoint{2.562832in}{3.131098in}}%
\pgfpathlineto{\pgfqpoint{2.639955in}{2.248424in}}%
\pgfusepath{stroke}%
\end{pgfscope}%
\begin{pgfscope}%
\pgfpathrectangle{\pgfqpoint{0.481978in}{0.331635in}}{\pgfqpoint{4.960000in}{3.696000in}}%
\pgfusepath{clip}%
\pgfsetrectcap%
\pgfsetroundjoin%
\pgfsetlinewidth{1.505625pt}%
\definecolor{currentstroke}{rgb}{0.631373,0.788235,0.956863}%
\pgfsetstrokecolor{currentstroke}%
\pgfsetstrokeopacity{0.200000}%
\pgfsetdash{}{0pt}%
\pgfpathmoveto{\pgfqpoint{2.194229in}{2.440115in}}%
\pgfpathlineto{\pgfqpoint{2.639955in}{2.248424in}}%
\pgfusepath{stroke}%
\end{pgfscope}%
\begin{pgfscope}%
\pgfpathrectangle{\pgfqpoint{0.481978in}{0.331635in}}{\pgfqpoint{4.960000in}{3.696000in}}%
\pgfusepath{clip}%
\pgfsetrectcap%
\pgfsetroundjoin%
\pgfsetlinewidth{1.505625pt}%
\definecolor{currentstroke}{rgb}{0.631373,0.788235,0.956863}%
\pgfsetstrokecolor{currentstroke}%
\pgfsetstrokeopacity{0.200000}%
\pgfsetdash{}{0pt}%
\pgfpathmoveto{\pgfqpoint{4.484037in}{2.301118in}}%
\pgfpathlineto{\pgfqpoint{2.639955in}{2.248424in}}%
\pgfusepath{stroke}%
\end{pgfscope}%
\begin{pgfscope}%
\pgfpathrectangle{\pgfqpoint{0.481978in}{0.331635in}}{\pgfqpoint{4.960000in}{3.696000in}}%
\pgfusepath{clip}%
\pgfsetrectcap%
\pgfsetroundjoin%
\pgfsetlinewidth{1.505625pt}%
\definecolor{currentstroke}{rgb}{0.631373,0.788235,0.956863}%
\pgfsetstrokecolor{currentstroke}%
\pgfsetstrokeopacity{0.200000}%
\pgfsetdash{}{0pt}%
\pgfpathmoveto{\pgfqpoint{2.713554in}{2.002126in}}%
\pgfpathlineto{\pgfqpoint{2.639955in}{2.248424in}}%
\pgfusepath{stroke}%
\end{pgfscope}%
\begin{pgfscope}%
\pgfpathrectangle{\pgfqpoint{0.481978in}{0.331635in}}{\pgfqpoint{4.960000in}{3.696000in}}%
\pgfusepath{clip}%
\pgfsetrectcap%
\pgfsetroundjoin%
\pgfsetlinewidth{1.505625pt}%
\definecolor{currentstroke}{rgb}{0.631373,0.788235,0.956863}%
\pgfsetstrokecolor{currentstroke}%
\pgfsetstrokeopacity{0.200000}%
\pgfsetdash{}{0pt}%
\pgfpathmoveto{\pgfqpoint{2.577033in}{3.023362in}}%
\pgfpathlineto{\pgfqpoint{2.639955in}{2.248424in}}%
\pgfusepath{stroke}%
\end{pgfscope}%
\begin{pgfscope}%
\pgfpathrectangle{\pgfqpoint{0.481978in}{0.331635in}}{\pgfqpoint{4.960000in}{3.696000in}}%
\pgfusepath{clip}%
\pgfsetrectcap%
\pgfsetroundjoin%
\pgfsetlinewidth{1.505625pt}%
\definecolor{currentstroke}{rgb}{0.631373,0.788235,0.956863}%
\pgfsetstrokecolor{currentstroke}%
\pgfsetstrokeopacity{0.200000}%
\pgfsetdash{}{0pt}%
\pgfpathmoveto{\pgfqpoint{2.245375in}{3.235530in}}%
\pgfpathlineto{\pgfqpoint{2.639955in}{2.248424in}}%
\pgfusepath{stroke}%
\end{pgfscope}%
\begin{pgfscope}%
\pgfpathrectangle{\pgfqpoint{0.481978in}{0.331635in}}{\pgfqpoint{4.960000in}{3.696000in}}%
\pgfusepath{clip}%
\pgfsetrectcap%
\pgfsetroundjoin%
\pgfsetlinewidth{1.505625pt}%
\definecolor{currentstroke}{rgb}{0.631373,0.788235,0.956863}%
\pgfsetstrokecolor{currentstroke}%
\pgfsetstrokeopacity{0.200000}%
\pgfsetdash{}{0pt}%
\pgfpathmoveto{\pgfqpoint{1.712744in}{2.126914in}}%
\pgfpathlineto{\pgfqpoint{2.639955in}{2.248424in}}%
\pgfusepath{stroke}%
\end{pgfscope}%
\begin{pgfscope}%
\pgfpathrectangle{\pgfqpoint{0.481978in}{0.331635in}}{\pgfqpoint{4.960000in}{3.696000in}}%
\pgfusepath{clip}%
\pgfsetrectcap%
\pgfsetroundjoin%
\pgfsetlinewidth{1.505625pt}%
\definecolor{currentstroke}{rgb}{0.631373,0.788235,0.956863}%
\pgfsetstrokecolor{currentstroke}%
\pgfsetstrokeopacity{0.200000}%
\pgfsetdash{}{0pt}%
\pgfpathmoveto{\pgfqpoint{1.834176in}{2.151176in}}%
\pgfpathlineto{\pgfqpoint{2.639955in}{2.248424in}}%
\pgfusepath{stroke}%
\end{pgfscope}%
\begin{pgfscope}%
\pgfpathrectangle{\pgfqpoint{0.481978in}{0.331635in}}{\pgfqpoint{4.960000in}{3.696000in}}%
\pgfusepath{clip}%
\pgfsetrectcap%
\pgfsetroundjoin%
\pgfsetlinewidth{1.505625pt}%
\definecolor{currentstroke}{rgb}{0.631373,0.788235,0.956863}%
\pgfsetstrokecolor{currentstroke}%
\pgfsetstrokeopacity{0.200000}%
\pgfsetdash{}{0pt}%
\pgfpathmoveto{\pgfqpoint{1.866658in}{3.095143in}}%
\pgfpathlineto{\pgfqpoint{2.639955in}{2.248424in}}%
\pgfusepath{stroke}%
\end{pgfscope}%
\begin{pgfscope}%
\pgfpathrectangle{\pgfqpoint{0.481978in}{0.331635in}}{\pgfqpoint{4.960000in}{3.696000in}}%
\pgfusepath{clip}%
\pgfsetrectcap%
\pgfsetroundjoin%
\pgfsetlinewidth{1.505625pt}%
\definecolor{currentstroke}{rgb}{0.631373,0.788235,0.956863}%
\pgfsetstrokecolor{currentstroke}%
\pgfsetstrokeopacity{0.200000}%
\pgfsetdash{}{0pt}%
\pgfpathmoveto{\pgfqpoint{2.039007in}{1.900191in}}%
\pgfpathlineto{\pgfqpoint{2.639955in}{2.248424in}}%
\pgfusepath{stroke}%
\end{pgfscope}%
\begin{pgfscope}%
\pgfpathrectangle{\pgfqpoint{0.481978in}{0.331635in}}{\pgfqpoint{4.960000in}{3.696000in}}%
\pgfusepath{clip}%
\pgfsetrectcap%
\pgfsetroundjoin%
\pgfsetlinewidth{1.505625pt}%
\definecolor{currentstroke}{rgb}{0.631373,0.788235,0.956863}%
\pgfsetstrokecolor{currentstroke}%
\pgfsetstrokeopacity{0.200000}%
\pgfsetdash{}{0pt}%
\pgfpathmoveto{\pgfqpoint{2.111771in}{3.006523in}}%
\pgfpathlineto{\pgfqpoint{2.639955in}{2.248424in}}%
\pgfusepath{stroke}%
\end{pgfscope}%
\begin{pgfscope}%
\pgfpathrectangle{\pgfqpoint{0.481978in}{0.331635in}}{\pgfqpoint{4.960000in}{3.696000in}}%
\pgfusepath{clip}%
\pgfsetrectcap%
\pgfsetroundjoin%
\pgfsetlinewidth{1.505625pt}%
\definecolor{currentstroke}{rgb}{0.631373,0.788235,0.956863}%
\pgfsetstrokecolor{currentstroke}%
\pgfsetstrokeopacity{0.200000}%
\pgfsetdash{}{0pt}%
\pgfpathmoveto{\pgfqpoint{2.007118in}{2.568393in}}%
\pgfpathlineto{\pgfqpoint{2.639955in}{2.248424in}}%
\pgfusepath{stroke}%
\end{pgfscope}%
\begin{pgfscope}%
\pgfpathrectangle{\pgfqpoint{0.481978in}{0.331635in}}{\pgfqpoint{4.960000in}{3.696000in}}%
\pgfusepath{clip}%
\pgfsetrectcap%
\pgfsetroundjoin%
\pgfsetlinewidth{1.505625pt}%
\definecolor{currentstroke}{rgb}{0.631373,0.788235,0.956863}%
\pgfsetstrokecolor{currentstroke}%
\pgfsetstrokeopacity{0.200000}%
\pgfsetdash{}{0pt}%
\pgfpathmoveto{\pgfqpoint{2.777628in}{2.145708in}}%
\pgfpathlineto{\pgfqpoint{2.639955in}{2.248424in}}%
\pgfusepath{stroke}%
\end{pgfscope}%
\begin{pgfscope}%
\pgfpathrectangle{\pgfqpoint{0.481978in}{0.331635in}}{\pgfqpoint{4.960000in}{3.696000in}}%
\pgfusepath{clip}%
\pgfsetrectcap%
\pgfsetroundjoin%
\pgfsetlinewidth{1.505625pt}%
\definecolor{currentstroke}{rgb}{0.631373,0.788235,0.956863}%
\pgfsetstrokecolor{currentstroke}%
\pgfsetstrokeopacity{0.200000}%
\pgfsetdash{}{0pt}%
\pgfpathmoveto{\pgfqpoint{4.038724in}{3.610078in}}%
\pgfpathlineto{\pgfqpoint{2.639955in}{2.248424in}}%
\pgfusepath{stroke}%
\end{pgfscope}%
\begin{pgfscope}%
\pgfpathrectangle{\pgfqpoint{0.481978in}{0.331635in}}{\pgfqpoint{4.960000in}{3.696000in}}%
\pgfusepath{clip}%
\pgfsetrectcap%
\pgfsetroundjoin%
\pgfsetlinewidth{1.505625pt}%
\definecolor{currentstroke}{rgb}{0.631373,0.788235,0.956863}%
\pgfsetstrokecolor{currentstroke}%
\pgfsetstrokeopacity{0.200000}%
\pgfsetdash{}{0pt}%
\pgfpathmoveto{\pgfqpoint{2.255659in}{3.127608in}}%
\pgfpathlineto{\pgfqpoint{2.639955in}{2.248424in}}%
\pgfusepath{stroke}%
\end{pgfscope}%
\begin{pgfscope}%
\pgfpathrectangle{\pgfqpoint{0.481978in}{0.331635in}}{\pgfqpoint{4.960000in}{3.696000in}}%
\pgfusepath{clip}%
\pgfsetrectcap%
\pgfsetroundjoin%
\pgfsetlinewidth{1.505625pt}%
\definecolor{currentstroke}{rgb}{0.631373,0.788235,0.956863}%
\pgfsetstrokecolor{currentstroke}%
\pgfsetstrokeopacity{0.200000}%
\pgfsetdash{}{0pt}%
\pgfpathmoveto{\pgfqpoint{4.419131in}{2.031711in}}%
\pgfpathlineto{\pgfqpoint{2.639955in}{2.248424in}}%
\pgfusepath{stroke}%
\end{pgfscope}%
\begin{pgfscope}%
\pgfpathrectangle{\pgfqpoint{0.481978in}{0.331635in}}{\pgfqpoint{4.960000in}{3.696000in}}%
\pgfusepath{clip}%
\pgfsetrectcap%
\pgfsetroundjoin%
\pgfsetlinewidth{1.505625pt}%
\definecolor{currentstroke}{rgb}{0.631373,0.788235,0.956863}%
\pgfsetstrokecolor{currentstroke}%
\pgfsetstrokeopacity{0.200000}%
\pgfsetdash{}{0pt}%
\pgfpathmoveto{\pgfqpoint{2.577424in}{2.091350in}}%
\pgfpathlineto{\pgfqpoint{2.639955in}{2.248424in}}%
\pgfusepath{stroke}%
\end{pgfscope}%
\begin{pgfscope}%
\pgfpathrectangle{\pgfqpoint{0.481978in}{0.331635in}}{\pgfqpoint{4.960000in}{3.696000in}}%
\pgfusepath{clip}%
\pgfsetrectcap%
\pgfsetroundjoin%
\pgfsetlinewidth{1.505625pt}%
\definecolor{currentstroke}{rgb}{0.631373,0.788235,0.956863}%
\pgfsetstrokecolor{currentstroke}%
\pgfsetstrokeopacity{0.200000}%
\pgfsetdash{}{0pt}%
\pgfpathmoveto{\pgfqpoint{4.214629in}{1.558096in}}%
\pgfpathlineto{\pgfqpoint{2.639955in}{2.248424in}}%
\pgfusepath{stroke}%
\end{pgfscope}%
\begin{pgfscope}%
\pgfpathrectangle{\pgfqpoint{0.481978in}{0.331635in}}{\pgfqpoint{4.960000in}{3.696000in}}%
\pgfusepath{clip}%
\pgfsetrectcap%
\pgfsetroundjoin%
\pgfsetlinewidth{1.505625pt}%
\definecolor{currentstroke}{rgb}{0.631373,0.788235,0.956863}%
\pgfsetstrokecolor{currentstroke}%
\pgfsetstrokeopacity{0.200000}%
\pgfsetdash{}{0pt}%
\pgfpathmoveto{\pgfqpoint{1.520885in}{2.254193in}}%
\pgfpathlineto{\pgfqpoint{2.639955in}{2.248424in}}%
\pgfusepath{stroke}%
\end{pgfscope}%
\begin{pgfscope}%
\pgfpathrectangle{\pgfqpoint{0.481978in}{0.331635in}}{\pgfqpoint{4.960000in}{3.696000in}}%
\pgfusepath{clip}%
\pgfsetrectcap%
\pgfsetroundjoin%
\pgfsetlinewidth{1.505625pt}%
\definecolor{currentstroke}{rgb}{0.631373,0.788235,0.956863}%
\pgfsetstrokecolor{currentstroke}%
\pgfsetstrokeopacity{0.200000}%
\pgfsetdash{}{0pt}%
\pgfpathmoveto{\pgfqpoint{2.504302in}{1.512587in}}%
\pgfpathlineto{\pgfqpoint{2.639955in}{2.248424in}}%
\pgfusepath{stroke}%
\end{pgfscope}%
\begin{pgfscope}%
\pgfpathrectangle{\pgfqpoint{0.481978in}{0.331635in}}{\pgfqpoint{4.960000in}{3.696000in}}%
\pgfusepath{clip}%
\pgfsetrectcap%
\pgfsetroundjoin%
\pgfsetlinewidth{1.505625pt}%
\definecolor{currentstroke}{rgb}{0.631373,0.788235,0.956863}%
\pgfsetstrokecolor{currentstroke}%
\pgfsetstrokeopacity{0.200000}%
\pgfsetdash{}{0pt}%
\pgfpathmoveto{\pgfqpoint{1.834100in}{3.207700in}}%
\pgfpathlineto{\pgfqpoint{2.639955in}{2.248424in}}%
\pgfusepath{stroke}%
\end{pgfscope}%
\begin{pgfscope}%
\pgfpathrectangle{\pgfqpoint{0.481978in}{0.331635in}}{\pgfqpoint{4.960000in}{3.696000in}}%
\pgfusepath{clip}%
\pgfsetrectcap%
\pgfsetroundjoin%
\pgfsetlinewidth{1.505625pt}%
\definecolor{currentstroke}{rgb}{0.631373,0.788235,0.956863}%
\pgfsetstrokecolor{currentstroke}%
\pgfsetstrokeopacity{0.200000}%
\pgfsetdash{}{0pt}%
\pgfpathmoveto{\pgfqpoint{2.129996in}{1.975599in}}%
\pgfpathlineto{\pgfqpoint{2.639955in}{2.248424in}}%
\pgfusepath{stroke}%
\end{pgfscope}%
\begin{pgfscope}%
\pgfpathrectangle{\pgfqpoint{0.481978in}{0.331635in}}{\pgfqpoint{4.960000in}{3.696000in}}%
\pgfusepath{clip}%
\pgfsetrectcap%
\pgfsetroundjoin%
\pgfsetlinewidth{1.505625pt}%
\definecolor{currentstroke}{rgb}{0.631373,0.788235,0.956863}%
\pgfsetstrokecolor{currentstroke}%
\pgfsetstrokeopacity{0.200000}%
\pgfsetdash{}{0pt}%
\pgfpathmoveto{\pgfqpoint{1.577829in}{2.892497in}}%
\pgfpathlineto{\pgfqpoint{2.639955in}{2.248424in}}%
\pgfusepath{stroke}%
\end{pgfscope}%
\begin{pgfscope}%
\pgfpathrectangle{\pgfqpoint{0.481978in}{0.331635in}}{\pgfqpoint{4.960000in}{3.696000in}}%
\pgfusepath{clip}%
\pgfsetrectcap%
\pgfsetroundjoin%
\pgfsetlinewidth{1.505625pt}%
\definecolor{currentstroke}{rgb}{0.631373,0.788235,0.956863}%
\pgfsetstrokecolor{currentstroke}%
\pgfsetstrokeopacity{0.200000}%
\pgfsetdash{}{0pt}%
\pgfpathmoveto{\pgfqpoint{2.000204in}{2.655475in}}%
\pgfpathlineto{\pgfqpoint{2.639955in}{2.248424in}}%
\pgfusepath{stroke}%
\end{pgfscope}%
\begin{pgfscope}%
\pgfpathrectangle{\pgfqpoint{0.481978in}{0.331635in}}{\pgfqpoint{4.960000in}{3.696000in}}%
\pgfusepath{clip}%
\pgfsetrectcap%
\pgfsetroundjoin%
\pgfsetlinewidth{1.505625pt}%
\definecolor{currentstroke}{rgb}{0.631373,0.788235,0.956863}%
\pgfsetstrokecolor{currentstroke}%
\pgfsetstrokeopacity{0.200000}%
\pgfsetdash{}{0pt}%
\pgfpathmoveto{\pgfqpoint{2.727209in}{1.731109in}}%
\pgfpathlineto{\pgfqpoint{2.639955in}{2.248424in}}%
\pgfusepath{stroke}%
\end{pgfscope}%
\begin{pgfscope}%
\pgfpathrectangle{\pgfqpoint{0.481978in}{0.331635in}}{\pgfqpoint{4.960000in}{3.696000in}}%
\pgfusepath{clip}%
\pgfsetrectcap%
\pgfsetroundjoin%
\pgfsetlinewidth{1.505625pt}%
\definecolor{currentstroke}{rgb}{0.631373,0.788235,0.956863}%
\pgfsetstrokecolor{currentstroke}%
\pgfsetstrokeopacity{0.200000}%
\pgfsetdash{}{0pt}%
\pgfpathmoveto{\pgfqpoint{1.087825in}{1.375843in}}%
\pgfpathlineto{\pgfqpoint{2.639955in}{2.248424in}}%
\pgfusepath{stroke}%
\end{pgfscope}%
\begin{pgfscope}%
\pgfpathrectangle{\pgfqpoint{0.481978in}{0.331635in}}{\pgfqpoint{4.960000in}{3.696000in}}%
\pgfusepath{clip}%
\pgfsetrectcap%
\pgfsetroundjoin%
\pgfsetlinewidth{1.505625pt}%
\definecolor{currentstroke}{rgb}{0.631373,0.788235,0.956863}%
\pgfsetstrokecolor{currentstroke}%
\pgfsetstrokeopacity{0.200000}%
\pgfsetdash{}{0pt}%
\pgfpathmoveto{\pgfqpoint{2.974204in}{2.992975in}}%
\pgfpathlineto{\pgfqpoint{2.639955in}{2.248424in}}%
\pgfusepath{stroke}%
\end{pgfscope}%
\begin{pgfscope}%
\pgfpathrectangle{\pgfqpoint{0.481978in}{0.331635in}}{\pgfqpoint{4.960000in}{3.696000in}}%
\pgfusepath{clip}%
\pgfsetrectcap%
\pgfsetroundjoin%
\pgfsetlinewidth{1.505625pt}%
\definecolor{currentstroke}{rgb}{0.631373,0.788235,0.956863}%
\pgfsetstrokecolor{currentstroke}%
\pgfsetstrokeopacity{0.200000}%
\pgfsetdash{}{0pt}%
\pgfpathmoveto{\pgfqpoint{1.447239in}{2.809804in}}%
\pgfpathlineto{\pgfqpoint{2.639955in}{2.248424in}}%
\pgfusepath{stroke}%
\end{pgfscope}%
\begin{pgfscope}%
\pgfpathrectangle{\pgfqpoint{0.481978in}{0.331635in}}{\pgfqpoint{4.960000in}{3.696000in}}%
\pgfusepath{clip}%
\pgfsetrectcap%
\pgfsetroundjoin%
\pgfsetlinewidth{1.505625pt}%
\definecolor{currentstroke}{rgb}{0.631373,0.788235,0.956863}%
\pgfsetstrokecolor{currentstroke}%
\pgfsetstrokeopacity{0.200000}%
\pgfsetdash{}{0pt}%
\pgfpathmoveto{\pgfqpoint{1.582685in}{3.371694in}}%
\pgfpathlineto{\pgfqpoint{2.639955in}{2.248424in}}%
\pgfusepath{stroke}%
\end{pgfscope}%
\begin{pgfscope}%
\pgfpathrectangle{\pgfqpoint{0.481978in}{0.331635in}}{\pgfqpoint{4.960000in}{3.696000in}}%
\pgfusepath{clip}%
\pgfsetrectcap%
\pgfsetroundjoin%
\pgfsetlinewidth{1.505625pt}%
\definecolor{currentstroke}{rgb}{0.631373,0.788235,0.956863}%
\pgfsetstrokecolor{currentstroke}%
\pgfsetstrokeopacity{0.200000}%
\pgfsetdash{}{0pt}%
\pgfpathmoveto{\pgfqpoint{2.565304in}{2.310455in}}%
\pgfpathlineto{\pgfqpoint{2.639955in}{2.248424in}}%
\pgfusepath{stroke}%
\end{pgfscope}%
\begin{pgfscope}%
\pgfpathrectangle{\pgfqpoint{0.481978in}{0.331635in}}{\pgfqpoint{4.960000in}{3.696000in}}%
\pgfusepath{clip}%
\pgfsetrectcap%
\pgfsetroundjoin%
\pgfsetlinewidth{1.505625pt}%
\definecolor{currentstroke}{rgb}{0.631373,0.788235,0.956863}%
\pgfsetstrokecolor{currentstroke}%
\pgfsetstrokeopacity{0.200000}%
\pgfsetdash{}{0pt}%
\pgfpathmoveto{\pgfqpoint{1.949373in}{2.830874in}}%
\pgfpathlineto{\pgfqpoint{2.639955in}{2.248424in}}%
\pgfusepath{stroke}%
\end{pgfscope}%
\begin{pgfscope}%
\pgfpathrectangle{\pgfqpoint{0.481978in}{0.331635in}}{\pgfqpoint{4.960000in}{3.696000in}}%
\pgfusepath{clip}%
\pgfsetrectcap%
\pgfsetroundjoin%
\pgfsetlinewidth{1.505625pt}%
\definecolor{currentstroke}{rgb}{0.631373,0.788235,0.956863}%
\pgfsetstrokecolor{currentstroke}%
\pgfsetstrokeopacity{0.200000}%
\pgfsetdash{}{0pt}%
\pgfpathmoveto{\pgfqpoint{2.241416in}{2.254895in}}%
\pgfpathlineto{\pgfqpoint{2.639955in}{2.248424in}}%
\pgfusepath{stroke}%
\end{pgfscope}%
\begin{pgfscope}%
\pgfpathrectangle{\pgfqpoint{0.481978in}{0.331635in}}{\pgfqpoint{4.960000in}{3.696000in}}%
\pgfusepath{clip}%
\pgfsetrectcap%
\pgfsetroundjoin%
\pgfsetlinewidth{1.505625pt}%
\definecolor{currentstroke}{rgb}{0.631373,0.788235,0.956863}%
\pgfsetstrokecolor{currentstroke}%
\pgfsetstrokeopacity{0.200000}%
\pgfsetdash{}{0pt}%
\pgfpathmoveto{\pgfqpoint{3.525117in}{1.402243in}}%
\pgfpathlineto{\pgfqpoint{2.639955in}{2.248424in}}%
\pgfusepath{stroke}%
\end{pgfscope}%
\begin{pgfscope}%
\pgfpathrectangle{\pgfqpoint{0.481978in}{0.331635in}}{\pgfqpoint{4.960000in}{3.696000in}}%
\pgfusepath{clip}%
\pgfsetrectcap%
\pgfsetroundjoin%
\pgfsetlinewidth{1.505625pt}%
\definecolor{currentstroke}{rgb}{0.631373,0.788235,0.956863}%
\pgfsetstrokecolor{currentstroke}%
\pgfsetstrokeopacity{0.200000}%
\pgfsetdash{}{0pt}%
\pgfpathmoveto{\pgfqpoint{3.771458in}{2.239269in}}%
\pgfpathlineto{\pgfqpoint{2.639955in}{2.248424in}}%
\pgfusepath{stroke}%
\end{pgfscope}%
\begin{pgfscope}%
\pgfpathrectangle{\pgfqpoint{0.481978in}{0.331635in}}{\pgfqpoint{4.960000in}{3.696000in}}%
\pgfusepath{clip}%
\pgfsetrectcap%
\pgfsetroundjoin%
\pgfsetlinewidth{1.505625pt}%
\definecolor{currentstroke}{rgb}{0.631373,0.788235,0.956863}%
\pgfsetstrokecolor{currentstroke}%
\pgfsetstrokeopacity{0.200000}%
\pgfsetdash{}{0pt}%
\pgfpathmoveto{\pgfqpoint{2.157114in}{1.390699in}}%
\pgfpathlineto{\pgfqpoint{2.639955in}{2.248424in}}%
\pgfusepath{stroke}%
\end{pgfscope}%
\begin{pgfscope}%
\pgfpathrectangle{\pgfqpoint{0.481978in}{0.331635in}}{\pgfqpoint{4.960000in}{3.696000in}}%
\pgfusepath{clip}%
\pgfsetrectcap%
\pgfsetroundjoin%
\pgfsetlinewidth{1.505625pt}%
\definecolor{currentstroke}{rgb}{0.631373,0.788235,0.956863}%
\pgfsetstrokecolor{currentstroke}%
\pgfsetstrokeopacity{0.200000}%
\pgfsetdash{}{0pt}%
\pgfpathmoveto{\pgfqpoint{1.828104in}{2.813782in}}%
\pgfpathlineto{\pgfqpoint{2.639955in}{2.248424in}}%
\pgfusepath{stroke}%
\end{pgfscope}%
\begin{pgfscope}%
\pgfpathrectangle{\pgfqpoint{0.481978in}{0.331635in}}{\pgfqpoint{4.960000in}{3.696000in}}%
\pgfusepath{clip}%
\pgfsetrectcap%
\pgfsetroundjoin%
\pgfsetlinewidth{1.505625pt}%
\definecolor{currentstroke}{rgb}{0.631373,0.788235,0.956863}%
\pgfsetstrokecolor{currentstroke}%
\pgfsetstrokeopacity{0.200000}%
\pgfsetdash{}{0pt}%
\pgfpathmoveto{\pgfqpoint{2.853703in}{3.298827in}}%
\pgfpathlineto{\pgfqpoint{2.639955in}{2.248424in}}%
\pgfusepath{stroke}%
\end{pgfscope}%
\begin{pgfscope}%
\pgfpathrectangle{\pgfqpoint{0.481978in}{0.331635in}}{\pgfqpoint{4.960000in}{3.696000in}}%
\pgfusepath{clip}%
\pgfsetrectcap%
\pgfsetroundjoin%
\pgfsetlinewidth{1.505625pt}%
\definecolor{currentstroke}{rgb}{0.631373,0.788235,0.956863}%
\pgfsetstrokecolor{currentstroke}%
\pgfsetstrokeopacity{0.200000}%
\pgfsetdash{}{0pt}%
\pgfpathmoveto{\pgfqpoint{2.512476in}{1.533103in}}%
\pgfpathlineto{\pgfqpoint{2.639955in}{2.248424in}}%
\pgfusepath{stroke}%
\end{pgfscope}%
\begin{pgfscope}%
\pgfpathrectangle{\pgfqpoint{0.481978in}{0.331635in}}{\pgfqpoint{4.960000in}{3.696000in}}%
\pgfusepath{clip}%
\pgfsetrectcap%
\pgfsetroundjoin%
\pgfsetlinewidth{1.505625pt}%
\definecolor{currentstroke}{rgb}{0.631373,0.788235,0.956863}%
\pgfsetstrokecolor{currentstroke}%
\pgfsetstrokeopacity{0.200000}%
\pgfsetdash{}{0pt}%
\pgfpathmoveto{\pgfqpoint{2.414659in}{1.549428in}}%
\pgfpathlineto{\pgfqpoint{2.639955in}{2.248424in}}%
\pgfusepath{stroke}%
\end{pgfscope}%
\begin{pgfscope}%
\pgfpathrectangle{\pgfqpoint{0.481978in}{0.331635in}}{\pgfqpoint{4.960000in}{3.696000in}}%
\pgfusepath{clip}%
\pgfsetrectcap%
\pgfsetroundjoin%
\pgfsetlinewidth{1.505625pt}%
\definecolor{currentstroke}{rgb}{0.631373,0.788235,0.956863}%
\pgfsetstrokecolor{currentstroke}%
\pgfsetstrokeopacity{0.200000}%
\pgfsetdash{}{0pt}%
\pgfpathmoveto{\pgfqpoint{2.518218in}{2.014227in}}%
\pgfpathlineto{\pgfqpoint{2.639955in}{2.248424in}}%
\pgfusepath{stroke}%
\end{pgfscope}%
\begin{pgfscope}%
\pgfpathrectangle{\pgfqpoint{0.481978in}{0.331635in}}{\pgfqpoint{4.960000in}{3.696000in}}%
\pgfusepath{clip}%
\pgfsetrectcap%
\pgfsetroundjoin%
\pgfsetlinewidth{1.505625pt}%
\definecolor{currentstroke}{rgb}{0.631373,0.788235,0.956863}%
\pgfsetstrokecolor{currentstroke}%
\pgfsetstrokeopacity{0.200000}%
\pgfsetdash{}{0pt}%
\pgfpathmoveto{\pgfqpoint{3.567777in}{2.333759in}}%
\pgfpathlineto{\pgfqpoint{2.639955in}{2.248424in}}%
\pgfusepath{stroke}%
\end{pgfscope}%
\begin{pgfscope}%
\pgfpathrectangle{\pgfqpoint{0.481978in}{0.331635in}}{\pgfqpoint{4.960000in}{3.696000in}}%
\pgfusepath{clip}%
\pgfsetrectcap%
\pgfsetroundjoin%
\pgfsetlinewidth{1.505625pt}%
\definecolor{currentstroke}{rgb}{0.631373,0.788235,0.956863}%
\pgfsetstrokecolor{currentstroke}%
\pgfsetstrokeopacity{0.200000}%
\pgfsetdash{}{0pt}%
\pgfpathmoveto{\pgfqpoint{2.525557in}{2.654744in}}%
\pgfpathlineto{\pgfqpoint{2.639955in}{2.248424in}}%
\pgfusepath{stroke}%
\end{pgfscope}%
\begin{pgfscope}%
\pgfpathrectangle{\pgfqpoint{0.481978in}{0.331635in}}{\pgfqpoint{4.960000in}{3.696000in}}%
\pgfusepath{clip}%
\pgfsetrectcap%
\pgfsetroundjoin%
\pgfsetlinewidth{1.505625pt}%
\definecolor{currentstroke}{rgb}{0.631373,0.788235,0.956863}%
\pgfsetstrokecolor{currentstroke}%
\pgfsetstrokeopacity{0.200000}%
\pgfsetdash{}{0pt}%
\pgfpathmoveto{\pgfqpoint{2.649451in}{2.029773in}}%
\pgfpathlineto{\pgfqpoint{2.639955in}{2.248424in}}%
\pgfusepath{stroke}%
\end{pgfscope}%
\begin{pgfscope}%
\pgfpathrectangle{\pgfqpoint{0.481978in}{0.331635in}}{\pgfqpoint{4.960000in}{3.696000in}}%
\pgfusepath{clip}%
\pgfsetrectcap%
\pgfsetroundjoin%
\pgfsetlinewidth{1.505625pt}%
\definecolor{currentstroke}{rgb}{0.631373,0.788235,0.956863}%
\pgfsetstrokecolor{currentstroke}%
\pgfsetstrokeopacity{0.200000}%
\pgfsetdash{}{0pt}%
\pgfpathmoveto{\pgfqpoint{2.313708in}{2.812271in}}%
\pgfpathlineto{\pgfqpoint{2.639955in}{2.248424in}}%
\pgfusepath{stroke}%
\end{pgfscope}%
\begin{pgfscope}%
\pgfpathrectangle{\pgfqpoint{0.481978in}{0.331635in}}{\pgfqpoint{4.960000in}{3.696000in}}%
\pgfusepath{clip}%
\pgfsetrectcap%
\pgfsetroundjoin%
\pgfsetlinewidth{1.505625pt}%
\definecolor{currentstroke}{rgb}{0.631373,0.788235,0.956863}%
\pgfsetstrokecolor{currentstroke}%
\pgfsetstrokeopacity{0.200000}%
\pgfsetdash{}{0pt}%
\pgfpathmoveto{\pgfqpoint{2.006019in}{2.771776in}}%
\pgfpathlineto{\pgfqpoint{2.639955in}{2.248424in}}%
\pgfusepath{stroke}%
\end{pgfscope}%
\begin{pgfscope}%
\pgfpathrectangle{\pgfqpoint{0.481978in}{0.331635in}}{\pgfqpoint{4.960000in}{3.696000in}}%
\pgfusepath{clip}%
\pgfsetrectcap%
\pgfsetroundjoin%
\pgfsetlinewidth{1.505625pt}%
\definecolor{currentstroke}{rgb}{0.631373,0.788235,0.956863}%
\pgfsetstrokecolor{currentstroke}%
\pgfsetstrokeopacity{0.200000}%
\pgfsetdash{}{0pt}%
\pgfpathmoveto{\pgfqpoint{2.334579in}{1.900686in}}%
\pgfpathlineto{\pgfqpoint{2.639955in}{2.248424in}}%
\pgfusepath{stroke}%
\end{pgfscope}%
\begin{pgfscope}%
\pgfpathrectangle{\pgfqpoint{0.481978in}{0.331635in}}{\pgfqpoint{4.960000in}{3.696000in}}%
\pgfusepath{clip}%
\pgfsetrectcap%
\pgfsetroundjoin%
\pgfsetlinewidth{1.505625pt}%
\definecolor{currentstroke}{rgb}{0.631373,0.788235,0.956863}%
\pgfsetstrokecolor{currentstroke}%
\pgfsetstrokeopacity{0.200000}%
\pgfsetdash{}{0pt}%
\pgfpathmoveto{\pgfqpoint{2.056506in}{2.150942in}}%
\pgfpathlineto{\pgfqpoint{2.639955in}{2.248424in}}%
\pgfusepath{stroke}%
\end{pgfscope}%
\begin{pgfscope}%
\pgfpathrectangle{\pgfqpoint{0.481978in}{0.331635in}}{\pgfqpoint{4.960000in}{3.696000in}}%
\pgfusepath{clip}%
\pgfsetrectcap%
\pgfsetroundjoin%
\pgfsetlinewidth{1.505625pt}%
\definecolor{currentstroke}{rgb}{0.631373,0.788235,0.956863}%
\pgfsetstrokecolor{currentstroke}%
\pgfsetstrokeopacity{0.200000}%
\pgfsetdash{}{0pt}%
\pgfpathmoveto{\pgfqpoint{2.475462in}{2.291293in}}%
\pgfpathlineto{\pgfqpoint{2.639955in}{2.248424in}}%
\pgfusepath{stroke}%
\end{pgfscope}%
\begin{pgfscope}%
\pgfpathrectangle{\pgfqpoint{0.481978in}{0.331635in}}{\pgfqpoint{4.960000in}{3.696000in}}%
\pgfusepath{clip}%
\pgfsetrectcap%
\pgfsetroundjoin%
\pgfsetlinewidth{1.505625pt}%
\definecolor{currentstroke}{rgb}{0.631373,0.788235,0.956863}%
\pgfsetstrokecolor{currentstroke}%
\pgfsetstrokeopacity{0.200000}%
\pgfsetdash{}{0pt}%
\pgfpathmoveto{\pgfqpoint{2.457081in}{1.698746in}}%
\pgfpathlineto{\pgfqpoint{2.639955in}{2.248424in}}%
\pgfusepath{stroke}%
\end{pgfscope}%
\begin{pgfscope}%
\pgfpathrectangle{\pgfqpoint{0.481978in}{0.331635in}}{\pgfqpoint{4.960000in}{3.696000in}}%
\pgfusepath{clip}%
\pgfsetrectcap%
\pgfsetroundjoin%
\pgfsetlinewidth{1.505625pt}%
\definecolor{currentstroke}{rgb}{0.631373,0.788235,0.956863}%
\pgfsetstrokecolor{currentstroke}%
\pgfsetstrokeopacity{0.200000}%
\pgfsetdash{}{0pt}%
\pgfpathmoveto{\pgfqpoint{2.810443in}{1.526114in}}%
\pgfpathlineto{\pgfqpoint{2.639955in}{2.248424in}}%
\pgfusepath{stroke}%
\end{pgfscope}%
\begin{pgfscope}%
\pgfpathrectangle{\pgfqpoint{0.481978in}{0.331635in}}{\pgfqpoint{4.960000in}{3.696000in}}%
\pgfusepath{clip}%
\pgfsetrectcap%
\pgfsetroundjoin%
\pgfsetlinewidth{1.505625pt}%
\definecolor{currentstroke}{rgb}{0.631373,0.788235,0.956863}%
\pgfsetstrokecolor{currentstroke}%
\pgfsetstrokeopacity{0.200000}%
\pgfsetdash{}{0pt}%
\pgfpathmoveto{\pgfqpoint{3.346294in}{2.575050in}}%
\pgfpathlineto{\pgfqpoint{2.639955in}{2.248424in}}%
\pgfusepath{stroke}%
\end{pgfscope}%
\begin{pgfscope}%
\pgfpathrectangle{\pgfqpoint{0.481978in}{0.331635in}}{\pgfqpoint{4.960000in}{3.696000in}}%
\pgfusepath{clip}%
\pgfsetrectcap%
\pgfsetroundjoin%
\pgfsetlinewidth{1.505625pt}%
\definecolor{currentstroke}{rgb}{0.631373,0.788235,0.956863}%
\pgfsetstrokecolor{currentstroke}%
\pgfsetstrokeopacity{0.200000}%
\pgfsetdash{}{0pt}%
\pgfpathmoveto{\pgfqpoint{2.257440in}{2.543694in}}%
\pgfpathlineto{\pgfqpoint{2.639955in}{2.248424in}}%
\pgfusepath{stroke}%
\end{pgfscope}%
\begin{pgfscope}%
\pgfpathrectangle{\pgfqpoint{0.481978in}{0.331635in}}{\pgfqpoint{4.960000in}{3.696000in}}%
\pgfusepath{clip}%
\pgfsetrectcap%
\pgfsetroundjoin%
\pgfsetlinewidth{1.505625pt}%
\definecolor{currentstroke}{rgb}{0.631373,0.788235,0.956863}%
\pgfsetstrokecolor{currentstroke}%
\pgfsetstrokeopacity{0.200000}%
\pgfsetdash{}{0pt}%
\pgfpathmoveto{\pgfqpoint{3.149125in}{2.010306in}}%
\pgfpathlineto{\pgfqpoint{2.639955in}{2.248424in}}%
\pgfusepath{stroke}%
\end{pgfscope}%
\begin{pgfscope}%
\pgfpathrectangle{\pgfqpoint{0.481978in}{0.331635in}}{\pgfqpoint{4.960000in}{3.696000in}}%
\pgfusepath{clip}%
\pgfsetrectcap%
\pgfsetroundjoin%
\pgfsetlinewidth{1.505625pt}%
\definecolor{currentstroke}{rgb}{0.631373,0.788235,0.956863}%
\pgfsetstrokecolor{currentstroke}%
\pgfsetstrokeopacity{0.200000}%
\pgfsetdash{}{0pt}%
\pgfpathmoveto{\pgfqpoint{4.291538in}{1.948968in}}%
\pgfpathlineto{\pgfqpoint{2.639955in}{2.248424in}}%
\pgfusepath{stroke}%
\end{pgfscope}%
\begin{pgfscope}%
\pgfpathrectangle{\pgfqpoint{0.481978in}{0.331635in}}{\pgfqpoint{4.960000in}{3.696000in}}%
\pgfusepath{clip}%
\pgfsetrectcap%
\pgfsetroundjoin%
\pgfsetlinewidth{1.505625pt}%
\definecolor{currentstroke}{rgb}{0.631373,0.788235,0.956863}%
\pgfsetstrokecolor{currentstroke}%
\pgfsetstrokeopacity{0.200000}%
\pgfsetdash{}{0pt}%
\pgfpathmoveto{\pgfqpoint{2.299822in}{2.658743in}}%
\pgfpathlineto{\pgfqpoint{2.639955in}{2.248424in}}%
\pgfusepath{stroke}%
\end{pgfscope}%
\begin{pgfscope}%
\pgfpathrectangle{\pgfqpoint{0.481978in}{0.331635in}}{\pgfqpoint{4.960000in}{3.696000in}}%
\pgfusepath{clip}%
\pgfsetrectcap%
\pgfsetroundjoin%
\pgfsetlinewidth{1.505625pt}%
\definecolor{currentstroke}{rgb}{0.631373,0.788235,0.956863}%
\pgfsetstrokecolor{currentstroke}%
\pgfsetstrokeopacity{0.200000}%
\pgfsetdash{}{0pt}%
\pgfpathmoveto{\pgfqpoint{2.768417in}{2.098265in}}%
\pgfpathlineto{\pgfqpoint{2.639955in}{2.248424in}}%
\pgfusepath{stroke}%
\end{pgfscope}%
\begin{pgfscope}%
\pgfpathrectangle{\pgfqpoint{0.481978in}{0.331635in}}{\pgfqpoint{4.960000in}{3.696000in}}%
\pgfusepath{clip}%
\pgfsetrectcap%
\pgfsetroundjoin%
\pgfsetlinewidth{1.505625pt}%
\definecolor{currentstroke}{rgb}{0.631373,0.788235,0.956863}%
\pgfsetstrokecolor{currentstroke}%
\pgfsetstrokeopacity{0.200000}%
\pgfsetdash{}{0pt}%
\pgfpathmoveto{\pgfqpoint{2.195580in}{1.921913in}}%
\pgfpathlineto{\pgfqpoint{2.639955in}{2.248424in}}%
\pgfusepath{stroke}%
\end{pgfscope}%
\begin{pgfscope}%
\pgfpathrectangle{\pgfqpoint{0.481978in}{0.331635in}}{\pgfqpoint{4.960000in}{3.696000in}}%
\pgfusepath{clip}%
\pgfsetrectcap%
\pgfsetroundjoin%
\pgfsetlinewidth{1.505625pt}%
\definecolor{currentstroke}{rgb}{0.631373,0.788235,0.956863}%
\pgfsetstrokecolor{currentstroke}%
\pgfsetstrokeopacity{0.200000}%
\pgfsetdash{}{0pt}%
\pgfpathmoveto{\pgfqpoint{1.746017in}{2.452441in}}%
\pgfpathlineto{\pgfqpoint{2.639955in}{2.248424in}}%
\pgfusepath{stroke}%
\end{pgfscope}%
\begin{pgfscope}%
\pgfpathrectangle{\pgfqpoint{0.481978in}{0.331635in}}{\pgfqpoint{4.960000in}{3.696000in}}%
\pgfusepath{clip}%
\pgfsetrectcap%
\pgfsetroundjoin%
\pgfsetlinewidth{1.505625pt}%
\definecolor{currentstroke}{rgb}{0.631373,0.788235,0.956863}%
\pgfsetstrokecolor{currentstroke}%
\pgfsetstrokeopacity{0.200000}%
\pgfsetdash{}{0pt}%
\pgfpathmoveto{\pgfqpoint{2.652920in}{2.506358in}}%
\pgfpathlineto{\pgfqpoint{2.639955in}{2.248424in}}%
\pgfusepath{stroke}%
\end{pgfscope}%
\begin{pgfscope}%
\pgfpathrectangle{\pgfqpoint{0.481978in}{0.331635in}}{\pgfqpoint{4.960000in}{3.696000in}}%
\pgfusepath{clip}%
\pgfsetrectcap%
\pgfsetroundjoin%
\pgfsetlinewidth{1.505625pt}%
\definecolor{currentstroke}{rgb}{0.631373,0.788235,0.956863}%
\pgfsetstrokecolor{currentstroke}%
\pgfsetstrokeopacity{0.200000}%
\pgfsetdash{}{0pt}%
\pgfpathmoveto{\pgfqpoint{2.837731in}{3.733848in}}%
\pgfpathlineto{\pgfqpoint{2.639955in}{2.248424in}}%
\pgfusepath{stroke}%
\end{pgfscope}%
\begin{pgfscope}%
\pgfpathrectangle{\pgfqpoint{0.481978in}{0.331635in}}{\pgfqpoint{4.960000in}{3.696000in}}%
\pgfusepath{clip}%
\pgfsetrectcap%
\pgfsetroundjoin%
\pgfsetlinewidth{1.505625pt}%
\definecolor{currentstroke}{rgb}{0.631373,0.788235,0.956863}%
\pgfsetstrokecolor{currentstroke}%
\pgfsetstrokeopacity{0.200000}%
\pgfsetdash{}{0pt}%
\pgfpathmoveto{\pgfqpoint{3.329764in}{1.518659in}}%
\pgfpathlineto{\pgfqpoint{2.639955in}{2.248424in}}%
\pgfusepath{stroke}%
\end{pgfscope}%
\begin{pgfscope}%
\pgfpathrectangle{\pgfqpoint{0.481978in}{0.331635in}}{\pgfqpoint{4.960000in}{3.696000in}}%
\pgfusepath{clip}%
\pgfsetrectcap%
\pgfsetroundjoin%
\pgfsetlinewidth{1.505625pt}%
\definecolor{currentstroke}{rgb}{0.631373,0.788235,0.956863}%
\pgfsetstrokecolor{currentstroke}%
\pgfsetstrokeopacity{0.200000}%
\pgfsetdash{}{0pt}%
\pgfpathmoveto{\pgfqpoint{1.917096in}{2.002916in}}%
\pgfpathlineto{\pgfqpoint{2.639955in}{2.248424in}}%
\pgfusepath{stroke}%
\end{pgfscope}%
\begin{pgfscope}%
\pgfpathrectangle{\pgfqpoint{0.481978in}{0.331635in}}{\pgfqpoint{4.960000in}{3.696000in}}%
\pgfusepath{clip}%
\pgfsetrectcap%
\pgfsetroundjoin%
\pgfsetlinewidth{1.505625pt}%
\definecolor{currentstroke}{rgb}{0.631373,0.788235,0.956863}%
\pgfsetstrokecolor{currentstroke}%
\pgfsetstrokeopacity{0.200000}%
\pgfsetdash{}{0pt}%
\pgfpathmoveto{\pgfqpoint{2.723722in}{2.195364in}}%
\pgfpathlineto{\pgfqpoint{2.639955in}{2.248424in}}%
\pgfusepath{stroke}%
\end{pgfscope}%
\begin{pgfscope}%
\pgfpathrectangle{\pgfqpoint{0.481978in}{0.331635in}}{\pgfqpoint{4.960000in}{3.696000in}}%
\pgfusepath{clip}%
\pgfsetrectcap%
\pgfsetroundjoin%
\pgfsetlinewidth{1.505625pt}%
\definecolor{currentstroke}{rgb}{0.631373,0.788235,0.956863}%
\pgfsetstrokecolor{currentstroke}%
\pgfsetstrokeopacity{0.200000}%
\pgfsetdash{}{0pt}%
\pgfpathmoveto{\pgfqpoint{1.791703in}{1.824210in}}%
\pgfpathlineto{\pgfqpoint{2.639955in}{2.248424in}}%
\pgfusepath{stroke}%
\end{pgfscope}%
\begin{pgfscope}%
\pgfpathrectangle{\pgfqpoint{0.481978in}{0.331635in}}{\pgfqpoint{4.960000in}{3.696000in}}%
\pgfusepath{clip}%
\pgfsetrectcap%
\pgfsetroundjoin%
\pgfsetlinewidth{1.505625pt}%
\definecolor{currentstroke}{rgb}{0.631373,0.788235,0.956863}%
\pgfsetstrokecolor{currentstroke}%
\pgfsetstrokeopacity{0.200000}%
\pgfsetdash{}{0pt}%
\pgfpathmoveto{\pgfqpoint{2.820654in}{2.537557in}}%
\pgfpathlineto{\pgfqpoint{2.639955in}{2.248424in}}%
\pgfusepath{stroke}%
\end{pgfscope}%
\begin{pgfscope}%
\pgfpathrectangle{\pgfqpoint{0.481978in}{0.331635in}}{\pgfqpoint{4.960000in}{3.696000in}}%
\pgfusepath{clip}%
\pgfsetrectcap%
\pgfsetroundjoin%
\pgfsetlinewidth{1.505625pt}%
\definecolor{currentstroke}{rgb}{0.631373,0.788235,0.956863}%
\pgfsetstrokecolor{currentstroke}%
\pgfsetstrokeopacity{0.200000}%
\pgfsetdash{}{0pt}%
\pgfpathmoveto{\pgfqpoint{3.101095in}{2.129576in}}%
\pgfpathlineto{\pgfqpoint{2.639955in}{2.248424in}}%
\pgfusepath{stroke}%
\end{pgfscope}%
\begin{pgfscope}%
\pgfpathrectangle{\pgfqpoint{0.481978in}{0.331635in}}{\pgfqpoint{4.960000in}{3.696000in}}%
\pgfusepath{clip}%
\pgfsetrectcap%
\pgfsetroundjoin%
\pgfsetlinewidth{1.505625pt}%
\definecolor{currentstroke}{rgb}{0.631373,0.788235,0.956863}%
\pgfsetstrokecolor{currentstroke}%
\pgfsetstrokeopacity{0.200000}%
\pgfsetdash{}{0pt}%
\pgfpathmoveto{\pgfqpoint{3.852958in}{2.051644in}}%
\pgfpathlineto{\pgfqpoint{2.639955in}{2.248424in}}%
\pgfusepath{stroke}%
\end{pgfscope}%
\begin{pgfscope}%
\pgfpathrectangle{\pgfqpoint{0.481978in}{0.331635in}}{\pgfqpoint{4.960000in}{3.696000in}}%
\pgfusepath{clip}%
\pgfsetrectcap%
\pgfsetroundjoin%
\pgfsetlinewidth{1.505625pt}%
\definecolor{currentstroke}{rgb}{0.631373,0.788235,0.956863}%
\pgfsetstrokecolor{currentstroke}%
\pgfsetstrokeopacity{0.200000}%
\pgfsetdash{}{0pt}%
\pgfpathmoveto{\pgfqpoint{2.967735in}{2.449035in}}%
\pgfpathlineto{\pgfqpoint{2.639955in}{2.248424in}}%
\pgfusepath{stroke}%
\end{pgfscope}%
\begin{pgfscope}%
\pgfpathrectangle{\pgfqpoint{0.481978in}{0.331635in}}{\pgfqpoint{4.960000in}{3.696000in}}%
\pgfusepath{clip}%
\pgfsetrectcap%
\pgfsetroundjoin%
\pgfsetlinewidth{1.505625pt}%
\definecolor{currentstroke}{rgb}{0.631373,0.788235,0.956863}%
\pgfsetstrokecolor{currentstroke}%
\pgfsetstrokeopacity{0.200000}%
\pgfsetdash{}{0pt}%
\pgfpathmoveto{\pgfqpoint{3.877867in}{2.189108in}}%
\pgfpathlineto{\pgfqpoint{2.639955in}{2.248424in}}%
\pgfusepath{stroke}%
\end{pgfscope}%
\begin{pgfscope}%
\pgfpathrectangle{\pgfqpoint{0.481978in}{0.331635in}}{\pgfqpoint{4.960000in}{3.696000in}}%
\pgfusepath{clip}%
\pgfsetrectcap%
\pgfsetroundjoin%
\pgfsetlinewidth{1.505625pt}%
\definecolor{currentstroke}{rgb}{0.631373,0.788235,0.956863}%
\pgfsetstrokecolor{currentstroke}%
\pgfsetstrokeopacity{0.200000}%
\pgfsetdash{}{0pt}%
\pgfpathmoveto{\pgfqpoint{2.639796in}{2.061356in}}%
\pgfpathlineto{\pgfqpoint{2.639955in}{2.248424in}}%
\pgfusepath{stroke}%
\end{pgfscope}%
\begin{pgfscope}%
\pgfpathrectangle{\pgfqpoint{0.481978in}{0.331635in}}{\pgfqpoint{4.960000in}{3.696000in}}%
\pgfusepath{clip}%
\pgfsetrectcap%
\pgfsetroundjoin%
\pgfsetlinewidth{1.505625pt}%
\definecolor{currentstroke}{rgb}{0.631373,0.788235,0.956863}%
\pgfsetstrokecolor{currentstroke}%
\pgfsetstrokeopacity{0.200000}%
\pgfsetdash{}{0pt}%
\pgfpathmoveto{\pgfqpoint{3.080323in}{2.320171in}}%
\pgfpathlineto{\pgfqpoint{2.639955in}{2.248424in}}%
\pgfusepath{stroke}%
\end{pgfscope}%
\begin{pgfscope}%
\pgfpathrectangle{\pgfqpoint{0.481978in}{0.331635in}}{\pgfqpoint{4.960000in}{3.696000in}}%
\pgfusepath{clip}%
\pgfsetrectcap%
\pgfsetroundjoin%
\pgfsetlinewidth{1.505625pt}%
\definecolor{currentstroke}{rgb}{0.631373,0.788235,0.956863}%
\pgfsetstrokecolor{currentstroke}%
\pgfsetstrokeopacity{0.200000}%
\pgfsetdash{}{0pt}%
\pgfpathmoveto{\pgfqpoint{4.504389in}{1.647444in}}%
\pgfpathlineto{\pgfqpoint{2.639955in}{2.248424in}}%
\pgfusepath{stroke}%
\end{pgfscope}%
\begin{pgfscope}%
\pgfpathrectangle{\pgfqpoint{0.481978in}{0.331635in}}{\pgfqpoint{4.960000in}{3.696000in}}%
\pgfusepath{clip}%
\pgfsetrectcap%
\pgfsetroundjoin%
\pgfsetlinewidth{1.505625pt}%
\definecolor{currentstroke}{rgb}{0.631373,0.788235,0.956863}%
\pgfsetstrokecolor{currentstroke}%
\pgfsetstrokeopacity{0.200000}%
\pgfsetdash{}{0pt}%
\pgfpathmoveto{\pgfqpoint{2.599685in}{1.745329in}}%
\pgfpathlineto{\pgfqpoint{2.639955in}{2.248424in}}%
\pgfusepath{stroke}%
\end{pgfscope}%
\begin{pgfscope}%
\pgfpathrectangle{\pgfqpoint{0.481978in}{0.331635in}}{\pgfqpoint{4.960000in}{3.696000in}}%
\pgfusepath{clip}%
\pgfsetrectcap%
\pgfsetroundjoin%
\pgfsetlinewidth{1.505625pt}%
\definecolor{currentstroke}{rgb}{0.631373,0.788235,0.956863}%
\pgfsetstrokecolor{currentstroke}%
\pgfsetstrokeopacity{0.200000}%
\pgfsetdash{}{0pt}%
\pgfpathmoveto{\pgfqpoint{2.396495in}{2.575456in}}%
\pgfpathlineto{\pgfqpoint{2.639955in}{2.248424in}}%
\pgfusepath{stroke}%
\end{pgfscope}%
\begin{pgfscope}%
\pgfpathrectangle{\pgfqpoint{0.481978in}{0.331635in}}{\pgfqpoint{4.960000in}{3.696000in}}%
\pgfusepath{clip}%
\pgfsetrectcap%
\pgfsetroundjoin%
\pgfsetlinewidth{1.505625pt}%
\definecolor{currentstroke}{rgb}{0.631373,0.788235,0.956863}%
\pgfsetstrokecolor{currentstroke}%
\pgfsetstrokeopacity{0.200000}%
\pgfsetdash{}{0pt}%
\pgfpathmoveto{\pgfqpoint{3.101018in}{1.986855in}}%
\pgfpathlineto{\pgfqpoint{2.639955in}{2.248424in}}%
\pgfusepath{stroke}%
\end{pgfscope}%
\begin{pgfscope}%
\pgfpathrectangle{\pgfqpoint{0.481978in}{0.331635in}}{\pgfqpoint{4.960000in}{3.696000in}}%
\pgfusepath{clip}%
\pgfsetrectcap%
\pgfsetroundjoin%
\pgfsetlinewidth{1.505625pt}%
\definecolor{currentstroke}{rgb}{0.631373,0.788235,0.956863}%
\pgfsetstrokecolor{currentstroke}%
\pgfsetstrokeopacity{0.200000}%
\pgfsetdash{}{0pt}%
\pgfpathmoveto{\pgfqpoint{3.826951in}{1.795656in}}%
\pgfpathlineto{\pgfqpoint{2.639955in}{2.248424in}}%
\pgfusepath{stroke}%
\end{pgfscope}%
\begin{pgfscope}%
\pgfpathrectangle{\pgfqpoint{0.481978in}{0.331635in}}{\pgfqpoint{4.960000in}{3.696000in}}%
\pgfusepath{clip}%
\pgfsetrectcap%
\pgfsetroundjoin%
\pgfsetlinewidth{1.505625pt}%
\definecolor{currentstroke}{rgb}{0.631373,0.788235,0.956863}%
\pgfsetstrokecolor{currentstroke}%
\pgfsetstrokeopacity{0.200000}%
\pgfsetdash{}{0pt}%
\pgfpathmoveto{\pgfqpoint{3.258079in}{2.332288in}}%
\pgfpathlineto{\pgfqpoint{2.639955in}{2.248424in}}%
\pgfusepath{stroke}%
\end{pgfscope}%
\begin{pgfscope}%
\pgfpathrectangle{\pgfqpoint{0.481978in}{0.331635in}}{\pgfqpoint{4.960000in}{3.696000in}}%
\pgfusepath{clip}%
\pgfsetrectcap%
\pgfsetroundjoin%
\pgfsetlinewidth{1.505625pt}%
\definecolor{currentstroke}{rgb}{0.631373,0.788235,0.956863}%
\pgfsetstrokecolor{currentstroke}%
\pgfsetstrokeopacity{0.200000}%
\pgfsetdash{}{0pt}%
\pgfpathmoveto{\pgfqpoint{3.546483in}{1.845124in}}%
\pgfpathlineto{\pgfqpoint{2.639955in}{2.248424in}}%
\pgfusepath{stroke}%
\end{pgfscope}%
\begin{pgfscope}%
\pgfpathrectangle{\pgfqpoint{0.481978in}{0.331635in}}{\pgfqpoint{4.960000in}{3.696000in}}%
\pgfusepath{clip}%
\pgfsetrectcap%
\pgfsetroundjoin%
\pgfsetlinewidth{1.505625pt}%
\definecolor{currentstroke}{rgb}{0.631373,0.788235,0.956863}%
\pgfsetstrokecolor{currentstroke}%
\pgfsetstrokeopacity{0.200000}%
\pgfsetdash{}{0pt}%
\pgfpathmoveto{\pgfqpoint{2.353690in}{2.716538in}}%
\pgfpathlineto{\pgfqpoint{2.639955in}{2.248424in}}%
\pgfusepath{stroke}%
\end{pgfscope}%
\begin{pgfscope}%
\pgfpathrectangle{\pgfqpoint{0.481978in}{0.331635in}}{\pgfqpoint{4.960000in}{3.696000in}}%
\pgfusepath{clip}%
\pgfsetrectcap%
\pgfsetroundjoin%
\pgfsetlinewidth{1.505625pt}%
\definecolor{currentstroke}{rgb}{0.631373,0.788235,0.956863}%
\pgfsetstrokecolor{currentstroke}%
\pgfsetstrokeopacity{0.200000}%
\pgfsetdash{}{0pt}%
\pgfpathmoveto{\pgfqpoint{4.157151in}{1.953084in}}%
\pgfpathlineto{\pgfqpoint{2.639955in}{2.248424in}}%
\pgfusepath{stroke}%
\end{pgfscope}%
\begin{pgfscope}%
\pgfpathrectangle{\pgfqpoint{0.481978in}{0.331635in}}{\pgfqpoint{4.960000in}{3.696000in}}%
\pgfusepath{clip}%
\pgfsetrectcap%
\pgfsetroundjoin%
\pgfsetlinewidth{1.505625pt}%
\definecolor{currentstroke}{rgb}{0.631373,0.788235,0.956863}%
\pgfsetstrokecolor{currentstroke}%
\pgfsetstrokeopacity{0.200000}%
\pgfsetdash{}{0pt}%
\pgfpathmoveto{\pgfqpoint{1.950038in}{2.264223in}}%
\pgfpathlineto{\pgfqpoint{2.639955in}{2.248424in}}%
\pgfusepath{stroke}%
\end{pgfscope}%
\begin{pgfscope}%
\pgfpathrectangle{\pgfqpoint{0.481978in}{0.331635in}}{\pgfqpoint{4.960000in}{3.696000in}}%
\pgfusepath{clip}%
\pgfsetrectcap%
\pgfsetroundjoin%
\pgfsetlinewidth{1.505625pt}%
\definecolor{currentstroke}{rgb}{0.631373,0.788235,0.956863}%
\pgfsetstrokecolor{currentstroke}%
\pgfsetstrokeopacity{0.200000}%
\pgfsetdash{}{0pt}%
\pgfpathmoveto{\pgfqpoint{1.917872in}{2.081762in}}%
\pgfpathlineto{\pgfqpoint{2.639955in}{2.248424in}}%
\pgfusepath{stroke}%
\end{pgfscope}%
\begin{pgfscope}%
\pgfpathrectangle{\pgfqpoint{0.481978in}{0.331635in}}{\pgfqpoint{4.960000in}{3.696000in}}%
\pgfusepath{clip}%
\pgfsetrectcap%
\pgfsetroundjoin%
\pgfsetlinewidth{1.505625pt}%
\definecolor{currentstroke}{rgb}{0.631373,0.788235,0.956863}%
\pgfsetstrokecolor{currentstroke}%
\pgfsetstrokeopacity{0.200000}%
\pgfsetdash{}{0pt}%
\pgfpathmoveto{\pgfqpoint{1.997703in}{1.709360in}}%
\pgfpathlineto{\pgfqpoint{2.639955in}{2.248424in}}%
\pgfusepath{stroke}%
\end{pgfscope}%
\begin{pgfscope}%
\pgfpathrectangle{\pgfqpoint{0.481978in}{0.331635in}}{\pgfqpoint{4.960000in}{3.696000in}}%
\pgfusepath{clip}%
\pgfsetrectcap%
\pgfsetroundjoin%
\pgfsetlinewidth{1.505625pt}%
\definecolor{currentstroke}{rgb}{0.631373,0.788235,0.956863}%
\pgfsetstrokecolor{currentstroke}%
\pgfsetstrokeopacity{0.200000}%
\pgfsetdash{}{0pt}%
\pgfpathmoveto{\pgfqpoint{1.471014in}{1.314692in}}%
\pgfpathlineto{\pgfqpoint{2.639955in}{2.248424in}}%
\pgfusepath{stroke}%
\end{pgfscope}%
\begin{pgfscope}%
\pgfpathrectangle{\pgfqpoint{0.481978in}{0.331635in}}{\pgfqpoint{4.960000in}{3.696000in}}%
\pgfusepath{clip}%
\pgfsetrectcap%
\pgfsetroundjoin%
\pgfsetlinewidth{1.505625pt}%
\definecolor{currentstroke}{rgb}{0.631373,0.788235,0.956863}%
\pgfsetstrokecolor{currentstroke}%
\pgfsetstrokeopacity{0.200000}%
\pgfsetdash{}{0pt}%
\pgfpathmoveto{\pgfqpoint{2.468182in}{3.077607in}}%
\pgfpathlineto{\pgfqpoint{2.639955in}{2.248424in}}%
\pgfusepath{stroke}%
\end{pgfscope}%
\begin{pgfscope}%
\pgfpathrectangle{\pgfqpoint{0.481978in}{0.331635in}}{\pgfqpoint{4.960000in}{3.696000in}}%
\pgfusepath{clip}%
\pgfsetrectcap%
\pgfsetroundjoin%
\pgfsetlinewidth{1.505625pt}%
\definecolor{currentstroke}{rgb}{0.631373,0.788235,0.956863}%
\pgfsetstrokecolor{currentstroke}%
\pgfsetstrokeopacity{0.200000}%
\pgfsetdash{}{0pt}%
\pgfpathmoveto{\pgfqpoint{3.820394in}{1.782848in}}%
\pgfpathlineto{\pgfqpoint{2.639955in}{2.248424in}}%
\pgfusepath{stroke}%
\end{pgfscope}%
\begin{pgfscope}%
\pgfpathrectangle{\pgfqpoint{0.481978in}{0.331635in}}{\pgfqpoint{4.960000in}{3.696000in}}%
\pgfusepath{clip}%
\pgfsetrectcap%
\pgfsetroundjoin%
\pgfsetlinewidth{1.505625pt}%
\definecolor{currentstroke}{rgb}{0.631373,0.788235,0.956863}%
\pgfsetstrokecolor{currentstroke}%
\pgfsetstrokeopacity{0.200000}%
\pgfsetdash{}{0pt}%
\pgfpathmoveto{\pgfqpoint{1.893096in}{1.975829in}}%
\pgfpathlineto{\pgfqpoint{2.639955in}{2.248424in}}%
\pgfusepath{stroke}%
\end{pgfscope}%
\begin{pgfscope}%
\pgfpathrectangle{\pgfqpoint{0.481978in}{0.331635in}}{\pgfqpoint{4.960000in}{3.696000in}}%
\pgfusepath{clip}%
\pgfsetrectcap%
\pgfsetroundjoin%
\pgfsetlinewidth{1.505625pt}%
\definecolor{currentstroke}{rgb}{0.631373,0.788235,0.956863}%
\pgfsetstrokecolor{currentstroke}%
\pgfsetstrokeopacity{0.200000}%
\pgfsetdash{}{0pt}%
\pgfpathmoveto{\pgfqpoint{2.234546in}{1.447655in}}%
\pgfpathlineto{\pgfqpoint{2.639955in}{2.248424in}}%
\pgfusepath{stroke}%
\end{pgfscope}%
\begin{pgfscope}%
\pgfpathrectangle{\pgfqpoint{0.481978in}{0.331635in}}{\pgfqpoint{4.960000in}{3.696000in}}%
\pgfusepath{clip}%
\pgfsetrectcap%
\pgfsetroundjoin%
\pgfsetlinewidth{1.505625pt}%
\definecolor{currentstroke}{rgb}{0.631373,0.788235,0.956863}%
\pgfsetstrokecolor{currentstroke}%
\pgfsetstrokeopacity{0.200000}%
\pgfsetdash{}{0pt}%
\pgfpathmoveto{\pgfqpoint{4.328140in}{2.009235in}}%
\pgfpathlineto{\pgfqpoint{2.639955in}{2.248424in}}%
\pgfusepath{stroke}%
\end{pgfscope}%
\begin{pgfscope}%
\pgfpathrectangle{\pgfqpoint{0.481978in}{0.331635in}}{\pgfqpoint{4.960000in}{3.696000in}}%
\pgfusepath{clip}%
\pgfsetrectcap%
\pgfsetroundjoin%
\pgfsetlinewidth{1.505625pt}%
\definecolor{currentstroke}{rgb}{0.631373,0.788235,0.956863}%
\pgfsetstrokecolor{currentstroke}%
\pgfsetstrokeopacity{0.200000}%
\pgfsetdash{}{0pt}%
\pgfpathmoveto{\pgfqpoint{1.873549in}{2.053655in}}%
\pgfpathlineto{\pgfqpoint{2.639955in}{2.248424in}}%
\pgfusepath{stroke}%
\end{pgfscope}%
\begin{pgfscope}%
\pgfpathrectangle{\pgfqpoint{0.481978in}{0.331635in}}{\pgfqpoint{4.960000in}{3.696000in}}%
\pgfusepath{clip}%
\pgfsetrectcap%
\pgfsetroundjoin%
\pgfsetlinewidth{1.505625pt}%
\definecolor{currentstroke}{rgb}{0.631373,0.788235,0.956863}%
\pgfsetstrokecolor{currentstroke}%
\pgfsetstrokeopacity{0.200000}%
\pgfsetdash{}{0pt}%
\pgfpathmoveto{\pgfqpoint{2.899915in}{3.316818in}}%
\pgfpathlineto{\pgfqpoint{2.639955in}{2.248424in}}%
\pgfusepath{stroke}%
\end{pgfscope}%
\begin{pgfscope}%
\pgfpathrectangle{\pgfqpoint{0.481978in}{0.331635in}}{\pgfqpoint{4.960000in}{3.696000in}}%
\pgfusepath{clip}%
\pgfsetrectcap%
\pgfsetroundjoin%
\pgfsetlinewidth{1.505625pt}%
\definecolor{currentstroke}{rgb}{0.631373,0.788235,0.956863}%
\pgfsetstrokecolor{currentstroke}%
\pgfsetstrokeopacity{0.200000}%
\pgfsetdash{}{0pt}%
\pgfpathmoveto{\pgfqpoint{2.149070in}{2.579631in}}%
\pgfpathlineto{\pgfqpoint{2.639955in}{2.248424in}}%
\pgfusepath{stroke}%
\end{pgfscope}%
\begin{pgfscope}%
\pgfpathrectangle{\pgfqpoint{0.481978in}{0.331635in}}{\pgfqpoint{4.960000in}{3.696000in}}%
\pgfusepath{clip}%
\pgfsetrectcap%
\pgfsetroundjoin%
\pgfsetlinewidth{1.505625pt}%
\definecolor{currentstroke}{rgb}{0.631373,0.788235,0.956863}%
\pgfsetstrokecolor{currentstroke}%
\pgfsetstrokeopacity{0.200000}%
\pgfsetdash{}{0pt}%
\pgfpathmoveto{\pgfqpoint{3.237696in}{2.659060in}}%
\pgfpathlineto{\pgfqpoint{2.639955in}{2.248424in}}%
\pgfusepath{stroke}%
\end{pgfscope}%
\begin{pgfscope}%
\pgfpathrectangle{\pgfqpoint{0.481978in}{0.331635in}}{\pgfqpoint{4.960000in}{3.696000in}}%
\pgfusepath{clip}%
\pgfsetrectcap%
\pgfsetroundjoin%
\pgfsetlinewidth{1.505625pt}%
\definecolor{currentstroke}{rgb}{0.631373,0.788235,0.956863}%
\pgfsetstrokecolor{currentstroke}%
\pgfsetstrokeopacity{0.200000}%
\pgfsetdash{}{0pt}%
\pgfpathmoveto{\pgfqpoint{2.401125in}{1.349354in}}%
\pgfpathlineto{\pgfqpoint{2.639955in}{2.248424in}}%
\pgfusepath{stroke}%
\end{pgfscope}%
\begin{pgfscope}%
\pgfpathrectangle{\pgfqpoint{0.481978in}{0.331635in}}{\pgfqpoint{4.960000in}{3.696000in}}%
\pgfusepath{clip}%
\pgfsetrectcap%
\pgfsetroundjoin%
\pgfsetlinewidth{1.505625pt}%
\definecolor{currentstroke}{rgb}{0.631373,0.788235,0.956863}%
\pgfsetstrokecolor{currentstroke}%
\pgfsetstrokeopacity{0.200000}%
\pgfsetdash{}{0pt}%
\pgfpathmoveto{\pgfqpoint{4.130878in}{1.794386in}}%
\pgfpathlineto{\pgfqpoint{2.639955in}{2.248424in}}%
\pgfusepath{stroke}%
\end{pgfscope}%
\begin{pgfscope}%
\pgfpathrectangle{\pgfqpoint{0.481978in}{0.331635in}}{\pgfqpoint{4.960000in}{3.696000in}}%
\pgfusepath{clip}%
\pgfsetrectcap%
\pgfsetroundjoin%
\pgfsetlinewidth{1.505625pt}%
\definecolor{currentstroke}{rgb}{0.631373,0.788235,0.956863}%
\pgfsetstrokecolor{currentstroke}%
\pgfsetstrokeopacity{0.200000}%
\pgfsetdash{}{0pt}%
\pgfpathmoveto{\pgfqpoint{3.635936in}{1.998440in}}%
\pgfpathlineto{\pgfqpoint{2.639955in}{2.248424in}}%
\pgfusepath{stroke}%
\end{pgfscope}%
\begin{pgfscope}%
\pgfpathrectangle{\pgfqpoint{0.481978in}{0.331635in}}{\pgfqpoint{4.960000in}{3.696000in}}%
\pgfusepath{clip}%
\pgfsetrectcap%
\pgfsetroundjoin%
\pgfsetlinewidth{1.505625pt}%
\definecolor{currentstroke}{rgb}{0.631373,0.788235,0.956863}%
\pgfsetstrokecolor{currentstroke}%
\pgfsetstrokeopacity{0.200000}%
\pgfsetdash{}{0pt}%
\pgfpathmoveto{\pgfqpoint{2.542160in}{2.543669in}}%
\pgfpathlineto{\pgfqpoint{2.639955in}{2.248424in}}%
\pgfusepath{stroke}%
\end{pgfscope}%
\begin{pgfscope}%
\pgfpathrectangle{\pgfqpoint{0.481978in}{0.331635in}}{\pgfqpoint{4.960000in}{3.696000in}}%
\pgfusepath{clip}%
\pgfsetrectcap%
\pgfsetroundjoin%
\pgfsetlinewidth{1.505625pt}%
\definecolor{currentstroke}{rgb}{0.631373,0.788235,0.956863}%
\pgfsetstrokecolor{currentstroke}%
\pgfsetstrokeopacity{0.200000}%
\pgfsetdash{}{0pt}%
\pgfpathmoveto{\pgfqpoint{3.421206in}{1.305664in}}%
\pgfpathlineto{\pgfqpoint{2.639955in}{2.248424in}}%
\pgfusepath{stroke}%
\end{pgfscope}%
\begin{pgfscope}%
\pgfpathrectangle{\pgfqpoint{0.481978in}{0.331635in}}{\pgfqpoint{4.960000in}{3.696000in}}%
\pgfusepath{clip}%
\pgfsetrectcap%
\pgfsetroundjoin%
\pgfsetlinewidth{1.505625pt}%
\definecolor{currentstroke}{rgb}{0.631373,0.788235,0.956863}%
\pgfsetstrokecolor{currentstroke}%
\pgfsetstrokeopacity{0.200000}%
\pgfsetdash{}{0pt}%
\pgfpathmoveto{\pgfqpoint{3.349846in}{1.272482in}}%
\pgfpathlineto{\pgfqpoint{2.639955in}{2.248424in}}%
\pgfusepath{stroke}%
\end{pgfscope}%
\begin{pgfscope}%
\pgfpathrectangle{\pgfqpoint{0.481978in}{0.331635in}}{\pgfqpoint{4.960000in}{3.696000in}}%
\pgfusepath{clip}%
\pgfsetrectcap%
\pgfsetroundjoin%
\pgfsetlinewidth{1.505625pt}%
\definecolor{currentstroke}{rgb}{0.631373,0.788235,0.956863}%
\pgfsetstrokecolor{currentstroke}%
\pgfsetstrokeopacity{0.200000}%
\pgfsetdash{}{0pt}%
\pgfpathmoveto{\pgfqpoint{2.216571in}{1.641452in}}%
\pgfpathlineto{\pgfqpoint{2.639955in}{2.248424in}}%
\pgfusepath{stroke}%
\end{pgfscope}%
\begin{pgfscope}%
\pgfpathrectangle{\pgfqpoint{0.481978in}{0.331635in}}{\pgfqpoint{4.960000in}{3.696000in}}%
\pgfusepath{clip}%
\pgfsetrectcap%
\pgfsetroundjoin%
\pgfsetlinewidth{1.505625pt}%
\definecolor{currentstroke}{rgb}{0.631373,0.788235,0.956863}%
\pgfsetstrokecolor{currentstroke}%
\pgfsetstrokeopacity{0.200000}%
\pgfsetdash{}{0pt}%
\pgfpathmoveto{\pgfqpoint{1.969421in}{2.477370in}}%
\pgfpathlineto{\pgfqpoint{2.639955in}{2.248424in}}%
\pgfusepath{stroke}%
\end{pgfscope}%
\begin{pgfscope}%
\pgfpathrectangle{\pgfqpoint{0.481978in}{0.331635in}}{\pgfqpoint{4.960000in}{3.696000in}}%
\pgfusepath{clip}%
\pgfsetrectcap%
\pgfsetroundjoin%
\pgfsetlinewidth{1.505625pt}%
\definecolor{currentstroke}{rgb}{0.631373,0.788235,0.956863}%
\pgfsetstrokecolor{currentstroke}%
\pgfsetstrokeopacity{0.200000}%
\pgfsetdash{}{0pt}%
\pgfpathmoveto{\pgfqpoint{3.679572in}{2.218638in}}%
\pgfpathlineto{\pgfqpoint{2.639955in}{2.248424in}}%
\pgfusepath{stroke}%
\end{pgfscope}%
\begin{pgfscope}%
\pgfpathrectangle{\pgfqpoint{0.481978in}{0.331635in}}{\pgfqpoint{4.960000in}{3.696000in}}%
\pgfusepath{clip}%
\pgfsetrectcap%
\pgfsetroundjoin%
\pgfsetlinewidth{1.505625pt}%
\definecolor{currentstroke}{rgb}{0.631373,0.788235,0.956863}%
\pgfsetstrokecolor{currentstroke}%
\pgfsetstrokeopacity{0.200000}%
\pgfsetdash{}{0pt}%
\pgfpathmoveto{\pgfqpoint{2.384921in}{2.583786in}}%
\pgfpathlineto{\pgfqpoint{2.639955in}{2.248424in}}%
\pgfusepath{stroke}%
\end{pgfscope}%
\begin{pgfscope}%
\pgfpathrectangle{\pgfqpoint{0.481978in}{0.331635in}}{\pgfqpoint{4.960000in}{3.696000in}}%
\pgfusepath{clip}%
\pgfsetrectcap%
\pgfsetroundjoin%
\pgfsetlinewidth{1.505625pt}%
\definecolor{currentstroke}{rgb}{0.631373,0.788235,0.956863}%
\pgfsetstrokecolor{currentstroke}%
\pgfsetstrokeopacity{0.200000}%
\pgfsetdash{}{0pt}%
\pgfpathmoveto{\pgfqpoint{3.016150in}{2.614736in}}%
\pgfpathlineto{\pgfqpoint{2.639955in}{2.248424in}}%
\pgfusepath{stroke}%
\end{pgfscope}%
\begin{pgfscope}%
\pgfpathrectangle{\pgfqpoint{0.481978in}{0.331635in}}{\pgfqpoint{4.960000in}{3.696000in}}%
\pgfusepath{clip}%
\pgfsetrectcap%
\pgfsetroundjoin%
\pgfsetlinewidth{1.505625pt}%
\definecolor{currentstroke}{rgb}{0.631373,0.788235,0.956863}%
\pgfsetstrokecolor{currentstroke}%
\pgfsetstrokeopacity{0.200000}%
\pgfsetdash{}{0pt}%
\pgfpathmoveto{\pgfqpoint{3.388934in}{2.135665in}}%
\pgfpathlineto{\pgfqpoint{2.639955in}{2.248424in}}%
\pgfusepath{stroke}%
\end{pgfscope}%
\begin{pgfscope}%
\pgfpathrectangle{\pgfqpoint{0.481978in}{0.331635in}}{\pgfqpoint{4.960000in}{3.696000in}}%
\pgfusepath{clip}%
\pgfsetrectcap%
\pgfsetroundjoin%
\pgfsetlinewidth{1.505625pt}%
\definecolor{currentstroke}{rgb}{0.631373,0.788235,0.956863}%
\pgfsetstrokecolor{currentstroke}%
\pgfsetstrokeopacity{0.200000}%
\pgfsetdash{}{0pt}%
\pgfpathmoveto{\pgfqpoint{2.019457in}{1.712412in}}%
\pgfpathlineto{\pgfqpoint{2.639955in}{2.248424in}}%
\pgfusepath{stroke}%
\end{pgfscope}%
\begin{pgfscope}%
\pgfpathrectangle{\pgfqpoint{0.481978in}{0.331635in}}{\pgfqpoint{4.960000in}{3.696000in}}%
\pgfusepath{clip}%
\pgfsetrectcap%
\pgfsetroundjoin%
\pgfsetlinewidth{1.505625pt}%
\definecolor{currentstroke}{rgb}{0.631373,0.788235,0.956863}%
\pgfsetstrokecolor{currentstroke}%
\pgfsetstrokeopacity{0.200000}%
\pgfsetdash{}{0pt}%
\pgfpathmoveto{\pgfqpoint{2.390462in}{2.230555in}}%
\pgfpathlineto{\pgfqpoint{2.639955in}{2.248424in}}%
\pgfusepath{stroke}%
\end{pgfscope}%
\begin{pgfscope}%
\pgfpathrectangle{\pgfqpoint{0.481978in}{0.331635in}}{\pgfqpoint{4.960000in}{3.696000in}}%
\pgfusepath{clip}%
\pgfsetrectcap%
\pgfsetroundjoin%
\pgfsetlinewidth{1.505625pt}%
\definecolor{currentstroke}{rgb}{0.631373,0.788235,0.956863}%
\pgfsetstrokecolor{currentstroke}%
\pgfsetstrokeopacity{0.200000}%
\pgfsetdash{}{0pt}%
\pgfpathmoveto{\pgfqpoint{1.913384in}{2.740342in}}%
\pgfpathlineto{\pgfqpoint{2.639955in}{2.248424in}}%
\pgfusepath{stroke}%
\end{pgfscope}%
\begin{pgfscope}%
\pgfpathrectangle{\pgfqpoint{0.481978in}{0.331635in}}{\pgfqpoint{4.960000in}{3.696000in}}%
\pgfusepath{clip}%
\pgfsetrectcap%
\pgfsetroundjoin%
\pgfsetlinewidth{1.505625pt}%
\definecolor{currentstroke}{rgb}{0.631373,0.788235,0.956863}%
\pgfsetstrokecolor{currentstroke}%
\pgfsetstrokeopacity{0.200000}%
\pgfsetdash{}{0pt}%
\pgfpathmoveto{\pgfqpoint{2.832331in}{1.072647in}}%
\pgfpathlineto{\pgfqpoint{2.639955in}{2.248424in}}%
\pgfusepath{stroke}%
\end{pgfscope}%
\begin{pgfscope}%
\pgfpathrectangle{\pgfqpoint{0.481978in}{0.331635in}}{\pgfqpoint{4.960000in}{3.696000in}}%
\pgfusepath{clip}%
\pgfsetrectcap%
\pgfsetroundjoin%
\pgfsetlinewidth{1.505625pt}%
\definecolor{currentstroke}{rgb}{0.631373,0.788235,0.956863}%
\pgfsetstrokecolor{currentstroke}%
\pgfsetstrokeopacity{0.200000}%
\pgfsetdash{}{0pt}%
\pgfpathmoveto{\pgfqpoint{4.022929in}{2.074337in}}%
\pgfpathlineto{\pgfqpoint{2.639955in}{2.248424in}}%
\pgfusepath{stroke}%
\end{pgfscope}%
\begin{pgfscope}%
\pgfpathrectangle{\pgfqpoint{0.481978in}{0.331635in}}{\pgfqpoint{4.960000in}{3.696000in}}%
\pgfusepath{clip}%
\pgfsetrectcap%
\pgfsetroundjoin%
\pgfsetlinewidth{1.505625pt}%
\definecolor{currentstroke}{rgb}{0.631373,0.788235,0.956863}%
\pgfsetstrokecolor{currentstroke}%
\pgfsetstrokeopacity{0.200000}%
\pgfsetdash{}{0pt}%
\pgfpathmoveto{\pgfqpoint{2.755220in}{1.719249in}}%
\pgfpathlineto{\pgfqpoint{2.639955in}{2.248424in}}%
\pgfusepath{stroke}%
\end{pgfscope}%
\begin{pgfscope}%
\pgfpathrectangle{\pgfqpoint{0.481978in}{0.331635in}}{\pgfqpoint{4.960000in}{3.696000in}}%
\pgfusepath{clip}%
\pgfsetrectcap%
\pgfsetroundjoin%
\pgfsetlinewidth{1.505625pt}%
\definecolor{currentstroke}{rgb}{0.631373,0.788235,0.956863}%
\pgfsetstrokecolor{currentstroke}%
\pgfsetstrokeopacity{0.200000}%
\pgfsetdash{}{0pt}%
\pgfpathmoveto{\pgfqpoint{2.232084in}{2.976481in}}%
\pgfpathlineto{\pgfqpoint{2.639955in}{2.248424in}}%
\pgfusepath{stroke}%
\end{pgfscope}%
\begin{pgfscope}%
\pgfpathrectangle{\pgfqpoint{0.481978in}{0.331635in}}{\pgfqpoint{4.960000in}{3.696000in}}%
\pgfusepath{clip}%
\pgfsetrectcap%
\pgfsetroundjoin%
\pgfsetlinewidth{1.505625pt}%
\definecolor{currentstroke}{rgb}{0.631373,0.788235,0.956863}%
\pgfsetstrokecolor{currentstroke}%
\pgfsetstrokeopacity{0.200000}%
\pgfsetdash{}{0pt}%
\pgfpathmoveto{\pgfqpoint{2.086861in}{2.214614in}}%
\pgfpathlineto{\pgfqpoint{2.639955in}{2.248424in}}%
\pgfusepath{stroke}%
\end{pgfscope}%
\begin{pgfscope}%
\pgfpathrectangle{\pgfqpoint{0.481978in}{0.331635in}}{\pgfqpoint{4.960000in}{3.696000in}}%
\pgfusepath{clip}%
\pgfsetrectcap%
\pgfsetroundjoin%
\pgfsetlinewidth{1.505625pt}%
\definecolor{currentstroke}{rgb}{0.631373,0.788235,0.956863}%
\pgfsetstrokecolor{currentstroke}%
\pgfsetstrokeopacity{0.200000}%
\pgfsetdash{}{0pt}%
\pgfpathmoveto{\pgfqpoint{1.970752in}{1.449033in}}%
\pgfpathlineto{\pgfqpoint{2.639955in}{2.248424in}}%
\pgfusepath{stroke}%
\end{pgfscope}%
\begin{pgfscope}%
\pgfpathrectangle{\pgfqpoint{0.481978in}{0.331635in}}{\pgfqpoint{4.960000in}{3.696000in}}%
\pgfusepath{clip}%
\pgfsetrectcap%
\pgfsetroundjoin%
\pgfsetlinewidth{1.505625pt}%
\definecolor{currentstroke}{rgb}{0.631373,0.788235,0.956863}%
\pgfsetstrokecolor{currentstroke}%
\pgfsetstrokeopacity{0.200000}%
\pgfsetdash{}{0pt}%
\pgfpathmoveto{\pgfqpoint{3.222124in}{2.506305in}}%
\pgfpathlineto{\pgfqpoint{2.639955in}{2.248424in}}%
\pgfusepath{stroke}%
\end{pgfscope}%
\begin{pgfscope}%
\pgfpathrectangle{\pgfqpoint{0.481978in}{0.331635in}}{\pgfqpoint{4.960000in}{3.696000in}}%
\pgfusepath{clip}%
\pgfsetrectcap%
\pgfsetroundjoin%
\pgfsetlinewidth{1.505625pt}%
\definecolor{currentstroke}{rgb}{0.631373,0.788235,0.956863}%
\pgfsetstrokecolor{currentstroke}%
\pgfsetstrokeopacity{0.200000}%
\pgfsetdash{}{0pt}%
\pgfpathmoveto{\pgfqpoint{4.691614in}{1.867055in}}%
\pgfpathlineto{\pgfqpoint{2.639955in}{2.248424in}}%
\pgfusepath{stroke}%
\end{pgfscope}%
\begin{pgfscope}%
\pgfpathrectangle{\pgfqpoint{0.481978in}{0.331635in}}{\pgfqpoint{4.960000in}{3.696000in}}%
\pgfusepath{clip}%
\pgfsetrectcap%
\pgfsetroundjoin%
\pgfsetlinewidth{1.505625pt}%
\definecolor{currentstroke}{rgb}{0.631373,0.788235,0.956863}%
\pgfsetstrokecolor{currentstroke}%
\pgfsetstrokeopacity{0.200000}%
\pgfsetdash{}{0pt}%
\pgfpathmoveto{\pgfqpoint{4.017177in}{1.937335in}}%
\pgfpathlineto{\pgfqpoint{2.639955in}{2.248424in}}%
\pgfusepath{stroke}%
\end{pgfscope}%
\begin{pgfscope}%
\pgfpathrectangle{\pgfqpoint{0.481978in}{0.331635in}}{\pgfqpoint{4.960000in}{3.696000in}}%
\pgfusepath{clip}%
\pgfsetrectcap%
\pgfsetroundjoin%
\pgfsetlinewidth{1.505625pt}%
\definecolor{currentstroke}{rgb}{0.631373,0.788235,0.956863}%
\pgfsetstrokecolor{currentstroke}%
\pgfsetstrokeopacity{0.200000}%
\pgfsetdash{}{0pt}%
\pgfpathmoveto{\pgfqpoint{2.369063in}{1.781748in}}%
\pgfpathlineto{\pgfqpoint{2.639955in}{2.248424in}}%
\pgfusepath{stroke}%
\end{pgfscope}%
\begin{pgfscope}%
\pgfpathrectangle{\pgfqpoint{0.481978in}{0.331635in}}{\pgfqpoint{4.960000in}{3.696000in}}%
\pgfusepath{clip}%
\pgfsetrectcap%
\pgfsetroundjoin%
\pgfsetlinewidth{1.505625pt}%
\definecolor{currentstroke}{rgb}{0.631373,0.788235,0.956863}%
\pgfsetstrokecolor{currentstroke}%
\pgfsetstrokeopacity{0.200000}%
\pgfsetdash{}{0pt}%
\pgfpathmoveto{\pgfqpoint{2.075627in}{3.275845in}}%
\pgfpathlineto{\pgfqpoint{2.639955in}{2.248424in}}%
\pgfusepath{stroke}%
\end{pgfscope}%
\begin{pgfscope}%
\pgfpathrectangle{\pgfqpoint{0.481978in}{0.331635in}}{\pgfqpoint{4.960000in}{3.696000in}}%
\pgfusepath{clip}%
\pgfsetrectcap%
\pgfsetroundjoin%
\pgfsetlinewidth{1.505625pt}%
\definecolor{currentstroke}{rgb}{0.631373,0.788235,0.956863}%
\pgfsetstrokecolor{currentstroke}%
\pgfsetstrokeopacity{0.200000}%
\pgfsetdash{}{0pt}%
\pgfpathmoveto{\pgfqpoint{2.323340in}{1.318353in}}%
\pgfpathlineto{\pgfqpoint{2.639955in}{2.248424in}}%
\pgfusepath{stroke}%
\end{pgfscope}%
\begin{pgfscope}%
\pgfpathrectangle{\pgfqpoint{0.481978in}{0.331635in}}{\pgfqpoint{4.960000in}{3.696000in}}%
\pgfusepath{clip}%
\pgfsetrectcap%
\pgfsetroundjoin%
\pgfsetlinewidth{1.505625pt}%
\definecolor{currentstroke}{rgb}{0.631373,0.788235,0.956863}%
\pgfsetstrokecolor{currentstroke}%
\pgfsetstrokeopacity{0.200000}%
\pgfsetdash{}{0pt}%
\pgfpathmoveto{\pgfqpoint{1.355993in}{1.092899in}}%
\pgfpathlineto{\pgfqpoint{2.639955in}{2.248424in}}%
\pgfusepath{stroke}%
\end{pgfscope}%
\begin{pgfscope}%
\pgfpathrectangle{\pgfqpoint{0.481978in}{0.331635in}}{\pgfqpoint{4.960000in}{3.696000in}}%
\pgfusepath{clip}%
\pgfsetrectcap%
\pgfsetroundjoin%
\pgfsetlinewidth{1.505625pt}%
\definecolor{currentstroke}{rgb}{0.631373,0.788235,0.956863}%
\pgfsetstrokecolor{currentstroke}%
\pgfsetstrokeopacity{0.200000}%
\pgfsetdash{}{0pt}%
\pgfpathmoveto{\pgfqpoint{4.108199in}{1.838120in}}%
\pgfpathlineto{\pgfqpoint{2.639955in}{2.248424in}}%
\pgfusepath{stroke}%
\end{pgfscope}%
\begin{pgfscope}%
\pgfpathrectangle{\pgfqpoint{0.481978in}{0.331635in}}{\pgfqpoint{4.960000in}{3.696000in}}%
\pgfusepath{clip}%
\pgfsetrectcap%
\pgfsetroundjoin%
\pgfsetlinewidth{1.505625pt}%
\definecolor{currentstroke}{rgb}{0.631373,0.788235,0.956863}%
\pgfsetstrokecolor{currentstroke}%
\pgfsetstrokeopacity{0.200000}%
\pgfsetdash{}{0pt}%
\pgfpathmoveto{\pgfqpoint{2.295691in}{2.162485in}}%
\pgfpathlineto{\pgfqpoint{2.639955in}{2.248424in}}%
\pgfusepath{stroke}%
\end{pgfscope}%
\begin{pgfscope}%
\pgfpathrectangle{\pgfqpoint{0.481978in}{0.331635in}}{\pgfqpoint{4.960000in}{3.696000in}}%
\pgfusepath{clip}%
\pgfsetrectcap%
\pgfsetroundjoin%
\pgfsetlinewidth{1.505625pt}%
\definecolor{currentstroke}{rgb}{0.631373,0.788235,0.956863}%
\pgfsetstrokecolor{currentstroke}%
\pgfsetstrokeopacity{0.200000}%
\pgfsetdash{}{0pt}%
\pgfpathmoveto{\pgfqpoint{1.851660in}{3.013477in}}%
\pgfpathlineto{\pgfqpoint{2.639955in}{2.248424in}}%
\pgfusepath{stroke}%
\end{pgfscope}%
\begin{pgfscope}%
\pgfpathrectangle{\pgfqpoint{0.481978in}{0.331635in}}{\pgfqpoint{4.960000in}{3.696000in}}%
\pgfusepath{clip}%
\pgfsetrectcap%
\pgfsetroundjoin%
\pgfsetlinewidth{1.505625pt}%
\definecolor{currentstroke}{rgb}{0.631373,0.788235,0.956863}%
\pgfsetstrokecolor{currentstroke}%
\pgfsetstrokeopacity{0.200000}%
\pgfsetdash{}{0pt}%
\pgfpathmoveto{\pgfqpoint{3.459839in}{2.155256in}}%
\pgfpathlineto{\pgfqpoint{2.639955in}{2.248424in}}%
\pgfusepath{stroke}%
\end{pgfscope}%
\begin{pgfscope}%
\pgfpathrectangle{\pgfqpoint{0.481978in}{0.331635in}}{\pgfqpoint{4.960000in}{3.696000in}}%
\pgfusepath{clip}%
\pgfsetrectcap%
\pgfsetroundjoin%
\pgfsetlinewidth{1.505625pt}%
\definecolor{currentstroke}{rgb}{0.631373,0.788235,0.956863}%
\pgfsetstrokecolor{currentstroke}%
\pgfsetstrokeopacity{0.200000}%
\pgfsetdash{}{0pt}%
\pgfpathmoveto{\pgfqpoint{4.203759in}{2.038024in}}%
\pgfpathlineto{\pgfqpoint{2.639955in}{2.248424in}}%
\pgfusepath{stroke}%
\end{pgfscope}%
\begin{pgfscope}%
\pgfpathrectangle{\pgfqpoint{0.481978in}{0.331635in}}{\pgfqpoint{4.960000in}{3.696000in}}%
\pgfusepath{clip}%
\pgfsetrectcap%
\pgfsetroundjoin%
\pgfsetlinewidth{1.505625pt}%
\definecolor{currentstroke}{rgb}{0.631373,0.788235,0.956863}%
\pgfsetstrokecolor{currentstroke}%
\pgfsetstrokeopacity{0.200000}%
\pgfsetdash{}{0pt}%
\pgfpathmoveto{\pgfqpoint{3.504513in}{3.036706in}}%
\pgfpathlineto{\pgfqpoint{2.639955in}{2.248424in}}%
\pgfusepath{stroke}%
\end{pgfscope}%
\begin{pgfscope}%
\pgfpathrectangle{\pgfqpoint{0.481978in}{0.331635in}}{\pgfqpoint{4.960000in}{3.696000in}}%
\pgfusepath{clip}%
\pgfsetrectcap%
\pgfsetroundjoin%
\pgfsetlinewidth{1.505625pt}%
\definecolor{currentstroke}{rgb}{0.631373,0.788235,0.956863}%
\pgfsetstrokecolor{currentstroke}%
\pgfsetstrokeopacity{0.200000}%
\pgfsetdash{}{0pt}%
\pgfpathmoveto{\pgfqpoint{3.435586in}{2.758172in}}%
\pgfpathlineto{\pgfqpoint{2.639955in}{2.248424in}}%
\pgfusepath{stroke}%
\end{pgfscope}%
\begin{pgfscope}%
\pgfpathrectangle{\pgfqpoint{0.481978in}{0.331635in}}{\pgfqpoint{4.960000in}{3.696000in}}%
\pgfusepath{clip}%
\pgfsetrectcap%
\pgfsetroundjoin%
\pgfsetlinewidth{1.505625pt}%
\definecolor{currentstroke}{rgb}{0.631373,0.788235,0.956863}%
\pgfsetstrokecolor{currentstroke}%
\pgfsetstrokeopacity{0.200000}%
\pgfsetdash{}{0pt}%
\pgfpathmoveto{\pgfqpoint{2.298856in}{1.493854in}}%
\pgfpathlineto{\pgfqpoint{2.639955in}{2.248424in}}%
\pgfusepath{stroke}%
\end{pgfscope}%
\begin{pgfscope}%
\pgfpathrectangle{\pgfqpoint{0.481978in}{0.331635in}}{\pgfqpoint{4.960000in}{3.696000in}}%
\pgfusepath{clip}%
\pgfsetrectcap%
\pgfsetroundjoin%
\pgfsetlinewidth{1.505625pt}%
\definecolor{currentstroke}{rgb}{0.631373,0.788235,0.956863}%
\pgfsetstrokecolor{currentstroke}%
\pgfsetstrokeopacity{0.200000}%
\pgfsetdash{}{0pt}%
\pgfpathmoveto{\pgfqpoint{1.716898in}{3.034935in}}%
\pgfpathlineto{\pgfqpoint{2.639955in}{2.248424in}}%
\pgfusepath{stroke}%
\end{pgfscope}%
\begin{pgfscope}%
\pgfpathrectangle{\pgfqpoint{0.481978in}{0.331635in}}{\pgfqpoint{4.960000in}{3.696000in}}%
\pgfusepath{clip}%
\pgfsetrectcap%
\pgfsetroundjoin%
\pgfsetlinewidth{1.505625pt}%
\definecolor{currentstroke}{rgb}{0.631373,0.788235,0.956863}%
\pgfsetstrokecolor{currentstroke}%
\pgfsetstrokeopacity{0.200000}%
\pgfsetdash{}{0pt}%
\pgfpathmoveto{\pgfqpoint{2.944040in}{2.445060in}}%
\pgfpathlineto{\pgfqpoint{2.639955in}{2.248424in}}%
\pgfusepath{stroke}%
\end{pgfscope}%
\begin{pgfscope}%
\pgfpathrectangle{\pgfqpoint{0.481978in}{0.331635in}}{\pgfqpoint{4.960000in}{3.696000in}}%
\pgfusepath{clip}%
\pgfsetrectcap%
\pgfsetroundjoin%
\pgfsetlinewidth{1.505625pt}%
\definecolor{currentstroke}{rgb}{0.631373,0.788235,0.956863}%
\pgfsetstrokecolor{currentstroke}%
\pgfsetstrokeopacity{0.200000}%
\pgfsetdash{}{0pt}%
\pgfpathmoveto{\pgfqpoint{2.677796in}{2.511523in}}%
\pgfpathlineto{\pgfqpoint{2.639955in}{2.248424in}}%
\pgfusepath{stroke}%
\end{pgfscope}%
\begin{pgfscope}%
\pgfpathrectangle{\pgfqpoint{0.481978in}{0.331635in}}{\pgfqpoint{4.960000in}{3.696000in}}%
\pgfusepath{clip}%
\pgfsetrectcap%
\pgfsetroundjoin%
\pgfsetlinewidth{1.505625pt}%
\definecolor{currentstroke}{rgb}{0.631373,0.788235,0.956863}%
\pgfsetstrokecolor{currentstroke}%
\pgfsetstrokeopacity{0.200000}%
\pgfsetdash{}{0pt}%
\pgfpathmoveto{\pgfqpoint{3.131597in}{2.146225in}}%
\pgfpathlineto{\pgfqpoint{2.639955in}{2.248424in}}%
\pgfusepath{stroke}%
\end{pgfscope}%
\begin{pgfscope}%
\pgfpathrectangle{\pgfqpoint{0.481978in}{0.331635in}}{\pgfqpoint{4.960000in}{3.696000in}}%
\pgfusepath{clip}%
\pgfsetrectcap%
\pgfsetroundjoin%
\pgfsetlinewidth{1.505625pt}%
\definecolor{currentstroke}{rgb}{0.631373,0.788235,0.956863}%
\pgfsetstrokecolor{currentstroke}%
\pgfsetstrokeopacity{0.200000}%
\pgfsetdash{}{0pt}%
\pgfpathmoveto{\pgfqpoint{2.013077in}{2.463792in}}%
\pgfpathlineto{\pgfqpoint{2.639955in}{2.248424in}}%
\pgfusepath{stroke}%
\end{pgfscope}%
\begin{pgfscope}%
\pgfpathrectangle{\pgfqpoint{0.481978in}{0.331635in}}{\pgfqpoint{4.960000in}{3.696000in}}%
\pgfusepath{clip}%
\pgfsetrectcap%
\pgfsetroundjoin%
\pgfsetlinewidth{1.505625pt}%
\definecolor{currentstroke}{rgb}{0.631373,0.788235,0.956863}%
\pgfsetstrokecolor{currentstroke}%
\pgfsetstrokeopacity{0.200000}%
\pgfsetdash{}{0pt}%
\pgfpathmoveto{\pgfqpoint{2.508540in}{2.688045in}}%
\pgfpathlineto{\pgfqpoint{2.639955in}{2.248424in}}%
\pgfusepath{stroke}%
\end{pgfscope}%
\begin{pgfscope}%
\pgfpathrectangle{\pgfqpoint{0.481978in}{0.331635in}}{\pgfqpoint{4.960000in}{3.696000in}}%
\pgfusepath{clip}%
\pgfsetrectcap%
\pgfsetroundjoin%
\pgfsetlinewidth{1.505625pt}%
\definecolor{currentstroke}{rgb}{0.631373,0.788235,0.956863}%
\pgfsetstrokecolor{currentstroke}%
\pgfsetstrokeopacity{0.200000}%
\pgfsetdash{}{0pt}%
\pgfpathmoveto{\pgfqpoint{2.185269in}{2.645915in}}%
\pgfpathlineto{\pgfqpoint{2.639955in}{2.248424in}}%
\pgfusepath{stroke}%
\end{pgfscope}%
\begin{pgfscope}%
\pgfpathrectangle{\pgfqpoint{0.481978in}{0.331635in}}{\pgfqpoint{4.960000in}{3.696000in}}%
\pgfusepath{clip}%
\pgfsetrectcap%
\pgfsetroundjoin%
\pgfsetlinewidth{1.505625pt}%
\definecolor{currentstroke}{rgb}{0.631373,0.788235,0.956863}%
\pgfsetstrokecolor{currentstroke}%
\pgfsetstrokeopacity{0.200000}%
\pgfsetdash{}{0pt}%
\pgfpathmoveto{\pgfqpoint{1.877370in}{2.562894in}}%
\pgfpathlineto{\pgfqpoint{2.639955in}{2.248424in}}%
\pgfusepath{stroke}%
\end{pgfscope}%
\begin{pgfscope}%
\pgfpathrectangle{\pgfqpoint{0.481978in}{0.331635in}}{\pgfqpoint{4.960000in}{3.696000in}}%
\pgfusepath{clip}%
\pgfsetrectcap%
\pgfsetroundjoin%
\pgfsetlinewidth{1.505625pt}%
\definecolor{currentstroke}{rgb}{0.631373,0.788235,0.956863}%
\pgfsetstrokecolor{currentstroke}%
\pgfsetstrokeopacity{0.200000}%
\pgfsetdash{}{0pt}%
\pgfpathmoveto{\pgfqpoint{2.612806in}{2.856243in}}%
\pgfpathlineto{\pgfqpoint{2.639955in}{2.248424in}}%
\pgfusepath{stroke}%
\end{pgfscope}%
\begin{pgfscope}%
\pgfpathrectangle{\pgfqpoint{0.481978in}{0.331635in}}{\pgfqpoint{4.960000in}{3.696000in}}%
\pgfusepath{clip}%
\pgfsetrectcap%
\pgfsetroundjoin%
\pgfsetlinewidth{1.505625pt}%
\definecolor{currentstroke}{rgb}{0.631373,0.788235,0.956863}%
\pgfsetstrokecolor{currentstroke}%
\pgfsetstrokeopacity{0.200000}%
\pgfsetdash{}{0pt}%
\pgfpathmoveto{\pgfqpoint{2.285707in}{1.707950in}}%
\pgfpathlineto{\pgfqpoint{2.639955in}{2.248424in}}%
\pgfusepath{stroke}%
\end{pgfscope}%
\begin{pgfscope}%
\pgfpathrectangle{\pgfqpoint{0.481978in}{0.331635in}}{\pgfqpoint{4.960000in}{3.696000in}}%
\pgfusepath{clip}%
\pgfsetrectcap%
\pgfsetroundjoin%
\pgfsetlinewidth{1.505625pt}%
\definecolor{currentstroke}{rgb}{0.631373,0.788235,0.956863}%
\pgfsetstrokecolor{currentstroke}%
\pgfsetstrokeopacity{0.200000}%
\pgfsetdash{}{0pt}%
\pgfpathmoveto{\pgfqpoint{1.585888in}{2.617690in}}%
\pgfpathlineto{\pgfqpoint{2.639955in}{2.248424in}}%
\pgfusepath{stroke}%
\end{pgfscope}%
\begin{pgfscope}%
\pgfpathrectangle{\pgfqpoint{0.481978in}{0.331635in}}{\pgfqpoint{4.960000in}{3.696000in}}%
\pgfusepath{clip}%
\pgfsetrectcap%
\pgfsetroundjoin%
\pgfsetlinewidth{1.505625pt}%
\definecolor{currentstroke}{rgb}{0.631373,0.788235,0.956863}%
\pgfsetstrokecolor{currentstroke}%
\pgfsetstrokeopacity{0.200000}%
\pgfsetdash{}{0pt}%
\pgfpathmoveto{\pgfqpoint{4.113942in}{1.857278in}}%
\pgfpathlineto{\pgfqpoint{2.639955in}{2.248424in}}%
\pgfusepath{stroke}%
\end{pgfscope}%
\begin{pgfscope}%
\pgfpathrectangle{\pgfqpoint{0.481978in}{0.331635in}}{\pgfqpoint{4.960000in}{3.696000in}}%
\pgfusepath{clip}%
\pgfsetrectcap%
\pgfsetroundjoin%
\pgfsetlinewidth{1.505625pt}%
\definecolor{currentstroke}{rgb}{0.631373,0.788235,0.956863}%
\pgfsetstrokecolor{currentstroke}%
\pgfsetstrokeopacity{0.200000}%
\pgfsetdash{}{0pt}%
\pgfpathmoveto{\pgfqpoint{2.608551in}{3.348316in}}%
\pgfpathlineto{\pgfqpoint{2.639955in}{2.248424in}}%
\pgfusepath{stroke}%
\end{pgfscope}%
\begin{pgfscope}%
\pgfpathrectangle{\pgfqpoint{0.481978in}{0.331635in}}{\pgfqpoint{4.960000in}{3.696000in}}%
\pgfusepath{clip}%
\pgfsetrectcap%
\pgfsetroundjoin%
\pgfsetlinewidth{1.505625pt}%
\definecolor{currentstroke}{rgb}{0.631373,0.788235,0.956863}%
\pgfsetstrokecolor{currentstroke}%
\pgfsetstrokeopacity{0.200000}%
\pgfsetdash{}{0pt}%
\pgfpathmoveto{\pgfqpoint{1.832941in}{1.644336in}}%
\pgfpathlineto{\pgfqpoint{2.639955in}{2.248424in}}%
\pgfusepath{stroke}%
\end{pgfscope}%
\begin{pgfscope}%
\pgfpathrectangle{\pgfqpoint{0.481978in}{0.331635in}}{\pgfqpoint{4.960000in}{3.696000in}}%
\pgfusepath{clip}%
\pgfsetrectcap%
\pgfsetroundjoin%
\pgfsetlinewidth{1.505625pt}%
\definecolor{currentstroke}{rgb}{0.631373,0.788235,0.956863}%
\pgfsetstrokecolor{currentstroke}%
\pgfsetstrokeopacity{0.200000}%
\pgfsetdash{}{0pt}%
\pgfpathmoveto{\pgfqpoint{3.758211in}{1.528832in}}%
\pgfpathlineto{\pgfqpoint{2.639955in}{2.248424in}}%
\pgfusepath{stroke}%
\end{pgfscope}%
\begin{pgfscope}%
\pgfpathrectangle{\pgfqpoint{0.481978in}{0.331635in}}{\pgfqpoint{4.960000in}{3.696000in}}%
\pgfusepath{clip}%
\pgfsetrectcap%
\pgfsetroundjoin%
\pgfsetlinewidth{1.505625pt}%
\definecolor{currentstroke}{rgb}{0.631373,0.788235,0.956863}%
\pgfsetstrokecolor{currentstroke}%
\pgfsetstrokeopacity{0.200000}%
\pgfsetdash{}{0pt}%
\pgfpathmoveto{\pgfqpoint{2.408577in}{2.865331in}}%
\pgfpathlineto{\pgfqpoint{2.639955in}{2.248424in}}%
\pgfusepath{stroke}%
\end{pgfscope}%
\begin{pgfscope}%
\pgfpathrectangle{\pgfqpoint{0.481978in}{0.331635in}}{\pgfqpoint{4.960000in}{3.696000in}}%
\pgfusepath{clip}%
\pgfsetrectcap%
\pgfsetroundjoin%
\pgfsetlinewidth{1.505625pt}%
\definecolor{currentstroke}{rgb}{0.631373,0.788235,0.956863}%
\pgfsetstrokecolor{currentstroke}%
\pgfsetstrokeopacity{0.200000}%
\pgfsetdash{}{0pt}%
\pgfpathmoveto{\pgfqpoint{3.296638in}{2.901076in}}%
\pgfpathlineto{\pgfqpoint{2.639955in}{2.248424in}}%
\pgfusepath{stroke}%
\end{pgfscope}%
\begin{pgfscope}%
\pgfpathrectangle{\pgfqpoint{0.481978in}{0.331635in}}{\pgfqpoint{4.960000in}{3.696000in}}%
\pgfusepath{clip}%
\pgfsetrectcap%
\pgfsetroundjoin%
\pgfsetlinewidth{1.505625pt}%
\definecolor{currentstroke}{rgb}{0.631373,0.788235,0.956863}%
\pgfsetstrokecolor{currentstroke}%
\pgfsetstrokeopacity{0.200000}%
\pgfsetdash{}{0pt}%
\pgfpathmoveto{\pgfqpoint{1.439821in}{2.274700in}}%
\pgfpathlineto{\pgfqpoint{2.639955in}{2.248424in}}%
\pgfusepath{stroke}%
\end{pgfscope}%
\begin{pgfscope}%
\pgfpathrectangle{\pgfqpoint{0.481978in}{0.331635in}}{\pgfqpoint{4.960000in}{3.696000in}}%
\pgfusepath{clip}%
\pgfsetrectcap%
\pgfsetroundjoin%
\pgfsetlinewidth{1.505625pt}%
\definecolor{currentstroke}{rgb}{0.631373,0.788235,0.956863}%
\pgfsetstrokecolor{currentstroke}%
\pgfsetstrokeopacity{0.200000}%
\pgfsetdash{}{0pt}%
\pgfpathmoveto{\pgfqpoint{3.276159in}{2.375925in}}%
\pgfpathlineto{\pgfqpoint{2.639955in}{2.248424in}}%
\pgfusepath{stroke}%
\end{pgfscope}%
\begin{pgfscope}%
\pgfpathrectangle{\pgfqpoint{0.481978in}{0.331635in}}{\pgfqpoint{4.960000in}{3.696000in}}%
\pgfusepath{clip}%
\pgfsetrectcap%
\pgfsetroundjoin%
\pgfsetlinewidth{1.505625pt}%
\definecolor{currentstroke}{rgb}{0.631373,0.788235,0.956863}%
\pgfsetstrokecolor{currentstroke}%
\pgfsetstrokeopacity{0.200000}%
\pgfsetdash{}{0pt}%
\pgfpathmoveto{\pgfqpoint{1.912245in}{1.613069in}}%
\pgfpathlineto{\pgfqpoint{2.639955in}{2.248424in}}%
\pgfusepath{stroke}%
\end{pgfscope}%
\begin{pgfscope}%
\pgfpathrectangle{\pgfqpoint{0.481978in}{0.331635in}}{\pgfqpoint{4.960000in}{3.696000in}}%
\pgfusepath{clip}%
\pgfsetrectcap%
\pgfsetroundjoin%
\pgfsetlinewidth{1.505625pt}%
\definecolor{currentstroke}{rgb}{0.631373,0.788235,0.956863}%
\pgfsetstrokecolor{currentstroke}%
\pgfsetstrokeopacity{0.200000}%
\pgfsetdash{}{0pt}%
\pgfpathmoveto{\pgfqpoint{4.503965in}{1.918420in}}%
\pgfpathlineto{\pgfqpoint{2.639955in}{2.248424in}}%
\pgfusepath{stroke}%
\end{pgfscope}%
\begin{pgfscope}%
\pgfpathrectangle{\pgfqpoint{0.481978in}{0.331635in}}{\pgfqpoint{4.960000in}{3.696000in}}%
\pgfusepath{clip}%
\pgfsetrectcap%
\pgfsetroundjoin%
\pgfsetlinewidth{1.505625pt}%
\definecolor{currentstroke}{rgb}{0.631373,0.788235,0.956863}%
\pgfsetstrokecolor{currentstroke}%
\pgfsetstrokeopacity{0.200000}%
\pgfsetdash{}{0pt}%
\pgfpathmoveto{\pgfqpoint{1.288010in}{2.864417in}}%
\pgfpathlineto{\pgfqpoint{2.639955in}{2.248424in}}%
\pgfusepath{stroke}%
\end{pgfscope}%
\begin{pgfscope}%
\pgfpathrectangle{\pgfqpoint{0.481978in}{0.331635in}}{\pgfqpoint{4.960000in}{3.696000in}}%
\pgfusepath{clip}%
\pgfsetrectcap%
\pgfsetroundjoin%
\pgfsetlinewidth{1.505625pt}%
\definecolor{currentstroke}{rgb}{0.631373,0.788235,0.956863}%
\pgfsetstrokecolor{currentstroke}%
\pgfsetstrokeopacity{0.200000}%
\pgfsetdash{}{0pt}%
\pgfpathmoveto{\pgfqpoint{2.779164in}{2.367631in}}%
\pgfpathlineto{\pgfqpoint{2.639955in}{2.248424in}}%
\pgfusepath{stroke}%
\end{pgfscope}%
\begin{pgfscope}%
\pgfpathrectangle{\pgfqpoint{0.481978in}{0.331635in}}{\pgfqpoint{4.960000in}{3.696000in}}%
\pgfusepath{clip}%
\pgfsetrectcap%
\pgfsetroundjoin%
\pgfsetlinewidth{1.505625pt}%
\definecolor{currentstroke}{rgb}{0.631373,0.788235,0.956863}%
\pgfsetstrokecolor{currentstroke}%
\pgfsetstrokeopacity{0.200000}%
\pgfsetdash{}{0pt}%
\pgfpathmoveto{\pgfqpoint{1.292786in}{3.400706in}}%
\pgfpathlineto{\pgfqpoint{2.639955in}{2.248424in}}%
\pgfusepath{stroke}%
\end{pgfscope}%
\begin{pgfscope}%
\pgfpathrectangle{\pgfqpoint{0.481978in}{0.331635in}}{\pgfqpoint{4.960000in}{3.696000in}}%
\pgfusepath{clip}%
\pgfsetrectcap%
\pgfsetroundjoin%
\pgfsetlinewidth{1.505625pt}%
\definecolor{currentstroke}{rgb}{0.631373,0.788235,0.956863}%
\pgfsetstrokecolor{currentstroke}%
\pgfsetstrokeopacity{0.200000}%
\pgfsetdash{}{0pt}%
\pgfpathmoveto{\pgfqpoint{2.210439in}{2.769001in}}%
\pgfpathlineto{\pgfqpoint{2.639955in}{2.248424in}}%
\pgfusepath{stroke}%
\end{pgfscope}%
\begin{pgfscope}%
\pgfpathrectangle{\pgfqpoint{0.481978in}{0.331635in}}{\pgfqpoint{4.960000in}{3.696000in}}%
\pgfusepath{clip}%
\pgfsetrectcap%
\pgfsetroundjoin%
\pgfsetlinewidth{1.505625pt}%
\definecolor{currentstroke}{rgb}{0.631373,0.788235,0.956863}%
\pgfsetstrokecolor{currentstroke}%
\pgfsetstrokeopacity{0.200000}%
\pgfsetdash{}{0pt}%
\pgfpathmoveto{\pgfqpoint{4.510417in}{1.646639in}}%
\pgfpathlineto{\pgfqpoint{2.639955in}{2.248424in}}%
\pgfusepath{stroke}%
\end{pgfscope}%
\begin{pgfscope}%
\pgfpathrectangle{\pgfqpoint{0.481978in}{0.331635in}}{\pgfqpoint{4.960000in}{3.696000in}}%
\pgfusepath{clip}%
\pgfsetrectcap%
\pgfsetroundjoin%
\pgfsetlinewidth{1.505625pt}%
\definecolor{currentstroke}{rgb}{0.631373,0.788235,0.956863}%
\pgfsetstrokecolor{currentstroke}%
\pgfsetstrokeopacity{0.200000}%
\pgfsetdash{}{0pt}%
\pgfpathmoveto{\pgfqpoint{1.850846in}{2.592324in}}%
\pgfpathlineto{\pgfqpoint{2.639955in}{2.248424in}}%
\pgfusepath{stroke}%
\end{pgfscope}%
\begin{pgfscope}%
\pgfpathrectangle{\pgfqpoint{0.481978in}{0.331635in}}{\pgfqpoint{4.960000in}{3.696000in}}%
\pgfusepath{clip}%
\pgfsetrectcap%
\pgfsetroundjoin%
\pgfsetlinewidth{1.505625pt}%
\definecolor{currentstroke}{rgb}{0.631373,0.788235,0.956863}%
\pgfsetstrokecolor{currentstroke}%
\pgfsetstrokeopacity{0.200000}%
\pgfsetdash{}{0pt}%
\pgfpathmoveto{\pgfqpoint{3.762258in}{1.929808in}}%
\pgfpathlineto{\pgfqpoint{2.639955in}{2.248424in}}%
\pgfusepath{stroke}%
\end{pgfscope}%
\begin{pgfscope}%
\pgfpathrectangle{\pgfqpoint{0.481978in}{0.331635in}}{\pgfqpoint{4.960000in}{3.696000in}}%
\pgfusepath{clip}%
\pgfsetrectcap%
\pgfsetroundjoin%
\pgfsetlinewidth{1.505625pt}%
\definecolor{currentstroke}{rgb}{0.631373,0.788235,0.956863}%
\pgfsetstrokecolor{currentstroke}%
\pgfsetstrokeopacity{0.200000}%
\pgfsetdash{}{0pt}%
\pgfpathmoveto{\pgfqpoint{1.702947in}{2.387920in}}%
\pgfpathlineto{\pgfqpoint{2.639955in}{2.248424in}}%
\pgfusepath{stroke}%
\end{pgfscope}%
\begin{pgfscope}%
\pgfpathrectangle{\pgfqpoint{0.481978in}{0.331635in}}{\pgfqpoint{4.960000in}{3.696000in}}%
\pgfusepath{clip}%
\pgfsetrectcap%
\pgfsetroundjoin%
\pgfsetlinewidth{1.505625pt}%
\definecolor{currentstroke}{rgb}{0.631373,0.788235,0.956863}%
\pgfsetstrokecolor{currentstroke}%
\pgfsetstrokeopacity{0.200000}%
\pgfsetdash{}{0pt}%
\pgfpathmoveto{\pgfqpoint{1.901158in}{2.363576in}}%
\pgfpathlineto{\pgfqpoint{2.639955in}{2.248424in}}%
\pgfusepath{stroke}%
\end{pgfscope}%
\begin{pgfscope}%
\pgfpathrectangle{\pgfqpoint{0.481978in}{0.331635in}}{\pgfqpoint{4.960000in}{3.696000in}}%
\pgfusepath{clip}%
\pgfsetrectcap%
\pgfsetroundjoin%
\pgfsetlinewidth{1.505625pt}%
\definecolor{currentstroke}{rgb}{0.631373,0.788235,0.956863}%
\pgfsetstrokecolor{currentstroke}%
\pgfsetstrokeopacity{0.200000}%
\pgfsetdash{}{0pt}%
\pgfpathmoveto{\pgfqpoint{1.478480in}{3.098670in}}%
\pgfpathlineto{\pgfqpoint{2.639955in}{2.248424in}}%
\pgfusepath{stroke}%
\end{pgfscope}%
\begin{pgfscope}%
\pgfpathrectangle{\pgfqpoint{0.481978in}{0.331635in}}{\pgfqpoint{4.960000in}{3.696000in}}%
\pgfusepath{clip}%
\pgfsetrectcap%
\pgfsetroundjoin%
\pgfsetlinewidth{1.505625pt}%
\definecolor{currentstroke}{rgb}{0.631373,0.788235,0.956863}%
\pgfsetstrokecolor{currentstroke}%
\pgfsetstrokeopacity{0.200000}%
\pgfsetdash{}{0pt}%
\pgfpathmoveto{\pgfqpoint{2.422644in}{1.996287in}}%
\pgfpathlineto{\pgfqpoint{2.639955in}{2.248424in}}%
\pgfusepath{stroke}%
\end{pgfscope}%
\begin{pgfscope}%
\pgfpathrectangle{\pgfqpoint{0.481978in}{0.331635in}}{\pgfqpoint{4.960000in}{3.696000in}}%
\pgfusepath{clip}%
\pgfsetrectcap%
\pgfsetroundjoin%
\pgfsetlinewidth{1.505625pt}%
\definecolor{currentstroke}{rgb}{0.631373,0.788235,0.956863}%
\pgfsetstrokecolor{currentstroke}%
\pgfsetstrokeopacity{0.200000}%
\pgfsetdash{}{0pt}%
\pgfpathmoveto{\pgfqpoint{2.256485in}{2.151567in}}%
\pgfpathlineto{\pgfqpoint{2.639955in}{2.248424in}}%
\pgfusepath{stroke}%
\end{pgfscope}%
\begin{pgfscope}%
\pgfpathrectangle{\pgfqpoint{0.481978in}{0.331635in}}{\pgfqpoint{4.960000in}{3.696000in}}%
\pgfusepath{clip}%
\pgfsetrectcap%
\pgfsetroundjoin%
\pgfsetlinewidth{1.505625pt}%
\definecolor{currentstroke}{rgb}{0.631373,0.788235,0.956863}%
\pgfsetstrokecolor{currentstroke}%
\pgfsetstrokeopacity{0.200000}%
\pgfsetdash{}{0pt}%
\pgfpathmoveto{\pgfqpoint{2.912683in}{2.241403in}}%
\pgfpathlineto{\pgfqpoint{2.639955in}{2.248424in}}%
\pgfusepath{stroke}%
\end{pgfscope}%
\begin{pgfscope}%
\pgfpathrectangle{\pgfqpoint{0.481978in}{0.331635in}}{\pgfqpoint{4.960000in}{3.696000in}}%
\pgfusepath{clip}%
\pgfsetrectcap%
\pgfsetroundjoin%
\pgfsetlinewidth{1.505625pt}%
\definecolor{currentstroke}{rgb}{0.631373,0.788235,0.956863}%
\pgfsetstrokecolor{currentstroke}%
\pgfsetstrokeopacity{0.200000}%
\pgfsetdash{}{0pt}%
\pgfpathmoveto{\pgfqpoint{2.056790in}{2.385890in}}%
\pgfpathlineto{\pgfqpoint{2.639955in}{2.248424in}}%
\pgfusepath{stroke}%
\end{pgfscope}%
\begin{pgfscope}%
\pgfpathrectangle{\pgfqpoint{0.481978in}{0.331635in}}{\pgfqpoint{4.960000in}{3.696000in}}%
\pgfusepath{clip}%
\pgfsetrectcap%
\pgfsetroundjoin%
\pgfsetlinewidth{1.505625pt}%
\definecolor{currentstroke}{rgb}{0.631373,0.788235,0.956863}%
\pgfsetstrokecolor{currentstroke}%
\pgfsetstrokeopacity{0.200000}%
\pgfsetdash{}{0pt}%
\pgfpathmoveto{\pgfqpoint{1.735980in}{1.110553in}}%
\pgfpathlineto{\pgfqpoint{2.639955in}{2.248424in}}%
\pgfusepath{stroke}%
\end{pgfscope}%
\begin{pgfscope}%
\pgfpathrectangle{\pgfqpoint{0.481978in}{0.331635in}}{\pgfqpoint{4.960000in}{3.696000in}}%
\pgfusepath{clip}%
\pgfsetrectcap%
\pgfsetroundjoin%
\pgfsetlinewidth{1.505625pt}%
\definecolor{currentstroke}{rgb}{0.631373,0.788235,0.956863}%
\pgfsetstrokecolor{currentstroke}%
\pgfsetstrokeopacity{0.200000}%
\pgfsetdash{}{0pt}%
\pgfpathmoveto{\pgfqpoint{1.740395in}{2.482302in}}%
\pgfpathlineto{\pgfqpoint{2.639955in}{2.248424in}}%
\pgfusepath{stroke}%
\end{pgfscope}%
\begin{pgfscope}%
\pgfpathrectangle{\pgfqpoint{0.481978in}{0.331635in}}{\pgfqpoint{4.960000in}{3.696000in}}%
\pgfusepath{clip}%
\pgfsetrectcap%
\pgfsetroundjoin%
\pgfsetlinewidth{1.505625pt}%
\definecolor{currentstroke}{rgb}{0.631373,0.788235,0.956863}%
\pgfsetstrokecolor{currentstroke}%
\pgfsetstrokeopacity{0.200000}%
\pgfsetdash{}{0pt}%
\pgfpathmoveto{\pgfqpoint{4.146877in}{2.061288in}}%
\pgfpathlineto{\pgfqpoint{2.639955in}{2.248424in}}%
\pgfusepath{stroke}%
\end{pgfscope}%
\begin{pgfscope}%
\pgfpathrectangle{\pgfqpoint{0.481978in}{0.331635in}}{\pgfqpoint{4.960000in}{3.696000in}}%
\pgfusepath{clip}%
\pgfsetrectcap%
\pgfsetroundjoin%
\pgfsetlinewidth{1.505625pt}%
\definecolor{currentstroke}{rgb}{0.631373,0.788235,0.956863}%
\pgfsetstrokecolor{currentstroke}%
\pgfsetstrokeopacity{0.200000}%
\pgfsetdash{}{0pt}%
\pgfpathmoveto{\pgfqpoint{2.320750in}{2.347486in}}%
\pgfpathlineto{\pgfqpoint{2.639955in}{2.248424in}}%
\pgfusepath{stroke}%
\end{pgfscope}%
\begin{pgfscope}%
\pgfpathrectangle{\pgfqpoint{0.481978in}{0.331635in}}{\pgfqpoint{4.960000in}{3.696000in}}%
\pgfusepath{clip}%
\pgfsetrectcap%
\pgfsetroundjoin%
\pgfsetlinewidth{1.505625pt}%
\definecolor{currentstroke}{rgb}{0.631373,0.788235,0.956863}%
\pgfsetstrokecolor{currentstroke}%
\pgfsetstrokeopacity{0.200000}%
\pgfsetdash{}{0pt}%
\pgfpathmoveto{\pgfqpoint{1.648423in}{1.745390in}}%
\pgfpathlineto{\pgfqpoint{2.639955in}{2.248424in}}%
\pgfusepath{stroke}%
\end{pgfscope}%
\begin{pgfscope}%
\pgfpathrectangle{\pgfqpoint{0.481978in}{0.331635in}}{\pgfqpoint{4.960000in}{3.696000in}}%
\pgfusepath{clip}%
\pgfsetrectcap%
\pgfsetroundjoin%
\pgfsetlinewidth{1.505625pt}%
\definecolor{currentstroke}{rgb}{0.631373,0.788235,0.956863}%
\pgfsetstrokecolor{currentstroke}%
\pgfsetstrokeopacity{0.200000}%
\pgfsetdash{}{0pt}%
\pgfpathmoveto{\pgfqpoint{3.265583in}{2.644321in}}%
\pgfpathlineto{\pgfqpoint{2.639955in}{2.248424in}}%
\pgfusepath{stroke}%
\end{pgfscope}%
\begin{pgfscope}%
\pgfpathrectangle{\pgfqpoint{0.481978in}{0.331635in}}{\pgfqpoint{4.960000in}{3.696000in}}%
\pgfusepath{clip}%
\pgfsetrectcap%
\pgfsetroundjoin%
\pgfsetlinewidth{1.505625pt}%
\definecolor{currentstroke}{rgb}{0.631373,0.788235,0.956863}%
\pgfsetstrokecolor{currentstroke}%
\pgfsetstrokeopacity{0.200000}%
\pgfsetdash{}{0pt}%
\pgfpathmoveto{\pgfqpoint{3.394690in}{2.467058in}}%
\pgfpathlineto{\pgfqpoint{2.639955in}{2.248424in}}%
\pgfusepath{stroke}%
\end{pgfscope}%
\begin{pgfscope}%
\pgfpathrectangle{\pgfqpoint{0.481978in}{0.331635in}}{\pgfqpoint{4.960000in}{3.696000in}}%
\pgfusepath{clip}%
\pgfsetrectcap%
\pgfsetroundjoin%
\pgfsetlinewidth{1.505625pt}%
\definecolor{currentstroke}{rgb}{0.631373,0.788235,0.956863}%
\pgfsetstrokecolor{currentstroke}%
\pgfsetstrokeopacity{0.200000}%
\pgfsetdash{}{0pt}%
\pgfpathmoveto{\pgfqpoint{1.330445in}{2.862530in}}%
\pgfpathlineto{\pgfqpoint{2.639955in}{2.248424in}}%
\pgfusepath{stroke}%
\end{pgfscope}%
\begin{pgfscope}%
\pgfpathrectangle{\pgfqpoint{0.481978in}{0.331635in}}{\pgfqpoint{4.960000in}{3.696000in}}%
\pgfusepath{clip}%
\pgfsetrectcap%
\pgfsetroundjoin%
\pgfsetlinewidth{1.505625pt}%
\definecolor{currentstroke}{rgb}{0.631373,0.788235,0.956863}%
\pgfsetstrokecolor{currentstroke}%
\pgfsetstrokeopacity{0.200000}%
\pgfsetdash{}{0pt}%
\pgfpathmoveto{\pgfqpoint{3.086945in}{2.825250in}}%
\pgfpathlineto{\pgfqpoint{2.639955in}{2.248424in}}%
\pgfusepath{stroke}%
\end{pgfscope}%
\begin{pgfscope}%
\pgfpathrectangle{\pgfqpoint{0.481978in}{0.331635in}}{\pgfqpoint{4.960000in}{3.696000in}}%
\pgfusepath{clip}%
\pgfsetrectcap%
\pgfsetroundjoin%
\pgfsetlinewidth{1.505625pt}%
\definecolor{currentstroke}{rgb}{0.631373,0.788235,0.956863}%
\pgfsetstrokecolor{currentstroke}%
\pgfsetstrokeopacity{0.200000}%
\pgfsetdash{}{0pt}%
\pgfpathmoveto{\pgfqpoint{2.062589in}{1.060695in}}%
\pgfpathlineto{\pgfqpoint{2.639955in}{2.248424in}}%
\pgfusepath{stroke}%
\end{pgfscope}%
\begin{pgfscope}%
\pgfpathrectangle{\pgfqpoint{0.481978in}{0.331635in}}{\pgfqpoint{4.960000in}{3.696000in}}%
\pgfusepath{clip}%
\pgfsetrectcap%
\pgfsetroundjoin%
\pgfsetlinewidth{1.505625pt}%
\definecolor{currentstroke}{rgb}{0.631373,0.788235,0.956863}%
\pgfsetstrokecolor{currentstroke}%
\pgfsetstrokeopacity{0.200000}%
\pgfsetdash{}{0pt}%
\pgfpathmoveto{\pgfqpoint{4.685082in}{1.869101in}}%
\pgfpathlineto{\pgfqpoint{2.639955in}{2.248424in}}%
\pgfusepath{stroke}%
\end{pgfscope}%
\begin{pgfscope}%
\pgfpathrectangle{\pgfqpoint{0.481978in}{0.331635in}}{\pgfqpoint{4.960000in}{3.696000in}}%
\pgfusepath{clip}%
\pgfsetrectcap%
\pgfsetroundjoin%
\pgfsetlinewidth{1.505625pt}%
\definecolor{currentstroke}{rgb}{0.631373,0.788235,0.956863}%
\pgfsetstrokecolor{currentstroke}%
\pgfsetstrokeopacity{0.200000}%
\pgfsetdash{}{0pt}%
\pgfpathmoveto{\pgfqpoint{3.302393in}{2.486033in}}%
\pgfpathlineto{\pgfqpoint{2.639955in}{2.248424in}}%
\pgfusepath{stroke}%
\end{pgfscope}%
\begin{pgfscope}%
\pgfpathrectangle{\pgfqpoint{0.481978in}{0.331635in}}{\pgfqpoint{4.960000in}{3.696000in}}%
\pgfusepath{clip}%
\pgfsetrectcap%
\pgfsetroundjoin%
\pgfsetlinewidth{1.505625pt}%
\definecolor{currentstroke}{rgb}{0.631373,0.788235,0.956863}%
\pgfsetstrokecolor{currentstroke}%
\pgfsetstrokeopacity{0.200000}%
\pgfsetdash{}{0pt}%
\pgfpathmoveto{\pgfqpoint{1.858759in}{2.680186in}}%
\pgfpathlineto{\pgfqpoint{2.639955in}{2.248424in}}%
\pgfusepath{stroke}%
\end{pgfscope}%
\begin{pgfscope}%
\pgfpathrectangle{\pgfqpoint{0.481978in}{0.331635in}}{\pgfqpoint{4.960000in}{3.696000in}}%
\pgfusepath{clip}%
\pgfsetrectcap%
\pgfsetroundjoin%
\pgfsetlinewidth{1.505625pt}%
\definecolor{currentstroke}{rgb}{0.631373,0.788235,0.956863}%
\pgfsetstrokecolor{currentstroke}%
\pgfsetstrokeopacity{0.200000}%
\pgfsetdash{}{0pt}%
\pgfpathmoveto{\pgfqpoint{3.174628in}{2.190960in}}%
\pgfpathlineto{\pgfqpoint{2.639955in}{2.248424in}}%
\pgfusepath{stroke}%
\end{pgfscope}%
\begin{pgfscope}%
\pgfpathrectangle{\pgfqpoint{0.481978in}{0.331635in}}{\pgfqpoint{4.960000in}{3.696000in}}%
\pgfusepath{clip}%
\pgfsetrectcap%
\pgfsetroundjoin%
\pgfsetlinewidth{1.505625pt}%
\definecolor{currentstroke}{rgb}{0.631373,0.788235,0.956863}%
\pgfsetstrokecolor{currentstroke}%
\pgfsetstrokeopacity{0.200000}%
\pgfsetdash{}{0pt}%
\pgfpathmoveto{\pgfqpoint{3.835461in}{1.689747in}}%
\pgfpathlineto{\pgfqpoint{2.639955in}{2.248424in}}%
\pgfusepath{stroke}%
\end{pgfscope}%
\begin{pgfscope}%
\pgfpathrectangle{\pgfqpoint{0.481978in}{0.331635in}}{\pgfqpoint{4.960000in}{3.696000in}}%
\pgfusepath{clip}%
\pgfsetrectcap%
\pgfsetroundjoin%
\pgfsetlinewidth{1.505625pt}%
\definecolor{currentstroke}{rgb}{0.631373,0.788235,0.956863}%
\pgfsetstrokecolor{currentstroke}%
\pgfsetstrokeopacity{0.200000}%
\pgfsetdash{}{0pt}%
\pgfpathmoveto{\pgfqpoint{1.054043in}{2.948485in}}%
\pgfpathlineto{\pgfqpoint{2.639955in}{2.248424in}}%
\pgfusepath{stroke}%
\end{pgfscope}%
\begin{pgfscope}%
\pgfpathrectangle{\pgfqpoint{0.481978in}{0.331635in}}{\pgfqpoint{4.960000in}{3.696000in}}%
\pgfusepath{clip}%
\pgfsetrectcap%
\pgfsetroundjoin%
\pgfsetlinewidth{1.505625pt}%
\definecolor{currentstroke}{rgb}{0.631373,0.788235,0.956863}%
\pgfsetstrokecolor{currentstroke}%
\pgfsetstrokeopacity{0.200000}%
\pgfsetdash{}{0pt}%
\pgfpathmoveto{\pgfqpoint{2.707868in}{2.216291in}}%
\pgfpathlineto{\pgfqpoint{2.639955in}{2.248424in}}%
\pgfusepath{stroke}%
\end{pgfscope}%
\begin{pgfscope}%
\pgfpathrectangle{\pgfqpoint{0.481978in}{0.331635in}}{\pgfqpoint{4.960000in}{3.696000in}}%
\pgfusepath{clip}%
\pgfsetrectcap%
\pgfsetroundjoin%
\pgfsetlinewidth{1.505625pt}%
\definecolor{currentstroke}{rgb}{0.631373,0.788235,0.956863}%
\pgfsetstrokecolor{currentstroke}%
\pgfsetstrokeopacity{0.200000}%
\pgfsetdash{}{0pt}%
\pgfpathmoveto{\pgfqpoint{1.569643in}{3.330006in}}%
\pgfpathlineto{\pgfqpoint{2.639955in}{2.248424in}}%
\pgfusepath{stroke}%
\end{pgfscope}%
\begin{pgfscope}%
\pgfpathrectangle{\pgfqpoint{0.481978in}{0.331635in}}{\pgfqpoint{4.960000in}{3.696000in}}%
\pgfusepath{clip}%
\pgfsetrectcap%
\pgfsetroundjoin%
\pgfsetlinewidth{1.505625pt}%
\definecolor{currentstroke}{rgb}{0.631373,0.788235,0.956863}%
\pgfsetstrokecolor{currentstroke}%
\pgfsetstrokeopacity{0.200000}%
\pgfsetdash{}{0pt}%
\pgfpathmoveto{\pgfqpoint{3.714123in}{1.167951in}}%
\pgfpathlineto{\pgfqpoint{2.639955in}{2.248424in}}%
\pgfusepath{stroke}%
\end{pgfscope}%
\begin{pgfscope}%
\pgfpathrectangle{\pgfqpoint{0.481978in}{0.331635in}}{\pgfqpoint{4.960000in}{3.696000in}}%
\pgfusepath{clip}%
\pgfsetrectcap%
\pgfsetroundjoin%
\pgfsetlinewidth{1.505625pt}%
\definecolor{currentstroke}{rgb}{0.631373,0.788235,0.956863}%
\pgfsetstrokecolor{currentstroke}%
\pgfsetstrokeopacity{0.200000}%
\pgfsetdash{}{0pt}%
\pgfpathmoveto{\pgfqpoint{1.884764in}{3.026584in}}%
\pgfpathlineto{\pgfqpoint{2.639955in}{2.248424in}}%
\pgfusepath{stroke}%
\end{pgfscope}%
\begin{pgfscope}%
\pgfsetrectcap%
\pgfsetmiterjoin%
\pgfsetlinewidth{0.803000pt}%
\definecolor{currentstroke}{rgb}{0.000000,0.000000,0.000000}%
\pgfsetstrokecolor{currentstroke}%
\pgfsetdash{}{0pt}%
\pgfpathmoveto{\pgfqpoint{0.481978in}{0.331635in}}%
\pgfpathlineto{\pgfqpoint{0.481978in}{4.027635in}}%
\pgfusepath{stroke}%
\end{pgfscope}%
\begin{pgfscope}%
\pgfsetrectcap%
\pgfsetmiterjoin%
\pgfsetlinewidth{0.803000pt}%
\definecolor{currentstroke}{rgb}{0.000000,0.000000,0.000000}%
\pgfsetstrokecolor{currentstroke}%
\pgfsetdash{}{0pt}%
\pgfpathmoveto{\pgfqpoint{5.441978in}{0.331635in}}%
\pgfpathlineto{\pgfqpoint{5.441978in}{4.027635in}}%
\pgfusepath{stroke}%
\end{pgfscope}%
\begin{pgfscope}%
\pgfsetrectcap%
\pgfsetmiterjoin%
\pgfsetlinewidth{0.803000pt}%
\definecolor{currentstroke}{rgb}{0.000000,0.000000,0.000000}%
\pgfsetstrokecolor{currentstroke}%
\pgfsetdash{}{0pt}%
\pgfpathmoveto{\pgfqpoint{0.481978in}{0.331635in}}%
\pgfpathlineto{\pgfqpoint{5.441978in}{0.331635in}}%
\pgfusepath{stroke}%
\end{pgfscope}%
\begin{pgfscope}%
\pgfsetrectcap%
\pgfsetmiterjoin%
\pgfsetlinewidth{0.803000pt}%
\definecolor{currentstroke}{rgb}{0.000000,0.000000,0.000000}%
\pgfsetstrokecolor{currentstroke}%
\pgfsetdash{}{0pt}%
\pgfpathmoveto{\pgfqpoint{0.481978in}{4.027635in}}%
\pgfpathlineto{\pgfqpoint{5.441978in}{4.027635in}}%
\pgfusepath{stroke}%
\end{pgfscope}%
\begin{pgfscope}%
\definecolor{textcolor}{rgb}{0.000000,0.000000,0.000000}%
\pgfsetstrokecolor{textcolor}%
\pgfsetfillcolor{textcolor}%
\pgftext[x=2.961978in,y=4.110968in,,base]{\color{textcolor}\sffamily\fontsize{12.000000}{14.400000}\selectfont t-SNE for chair images (s2r3dfree\_multi-object)}%
\end{pgfscope}%
\begin{pgfscope}%
\pgfsetbuttcap%
\pgfsetmiterjoin%
\definecolor{currentfill}{rgb}{1.000000,1.000000,1.000000}%
\pgfsetfillcolor{currentfill}%
\pgfsetfillopacity{0.800000}%
\pgfsetlinewidth{1.003750pt}%
\definecolor{currentstroke}{rgb}{0.800000,0.800000,0.800000}%
\pgfsetstrokecolor{currentstroke}%
\pgfsetstrokeopacity{0.800000}%
\pgfsetdash{}{0pt}%
\pgfpathmoveto{\pgfqpoint{3.327028in}{0.401079in}}%
\pgfpathlineto{\pgfqpoint{5.344756in}{0.401079in}}%
\pgfpathquadraticcurveto{\pgfqpoint{5.372533in}{0.401079in}}{\pgfqpoint{5.372533in}{0.428857in}}%
\pgfpathlineto{\pgfqpoint{5.372533in}{0.826548in}}%
\pgfpathquadraticcurveto{\pgfqpoint{5.372533in}{0.854326in}}{\pgfqpoint{5.344756in}{0.854326in}}%
\pgfpathlineto{\pgfqpoint{3.327028in}{0.854326in}}%
\pgfpathquadraticcurveto{\pgfqpoint{3.299251in}{0.854326in}}{\pgfqpoint{3.299251in}{0.826548in}}%
\pgfpathlineto{\pgfqpoint{3.299251in}{0.428857in}}%
\pgfpathquadraticcurveto{\pgfqpoint{3.299251in}{0.401079in}}{\pgfqpoint{3.327028in}{0.401079in}}%
\pgfpathclose%
\pgfusepath{stroke,fill}%
\end{pgfscope}%
\begin{pgfscope}%
\pgfsetbuttcap%
\pgfsetroundjoin%
\definecolor{currentfill}{rgb}{1.000000,0.705882,0.509804}%
\pgfsetfillcolor{currentfill}%
\pgfsetlinewidth{1.003750pt}%
\definecolor{currentstroke}{rgb}{1.000000,0.705882,0.509804}%
\pgfsetstrokecolor{currentstroke}%
\pgfsetdash{}{0pt}%
\pgfsys@defobject{currentmarker}{\pgfqpoint{-0.041667in}{-0.041667in}}{\pgfqpoint{0.041667in}{0.041667in}}{%
\pgfpathmoveto{\pgfqpoint{0.000000in}{-0.041667in}}%
\pgfpathcurveto{\pgfqpoint{0.011050in}{-0.041667in}}{\pgfqpoint{0.021649in}{-0.037276in}}{\pgfqpoint{0.029463in}{-0.029463in}}%
\pgfpathcurveto{\pgfqpoint{0.037276in}{-0.021649in}}{\pgfqpoint{0.041667in}{-0.011050in}}{\pgfqpoint{0.041667in}{0.000000in}}%
\pgfpathcurveto{\pgfqpoint{0.041667in}{0.011050in}}{\pgfqpoint{0.037276in}{0.021649in}}{\pgfqpoint{0.029463in}{0.029463in}}%
\pgfpathcurveto{\pgfqpoint{0.021649in}{0.037276in}}{\pgfqpoint{0.011050in}{0.041667in}}{\pgfqpoint{0.000000in}{0.041667in}}%
\pgfpathcurveto{\pgfqpoint{-0.011050in}{0.041667in}}{\pgfqpoint{-0.021649in}{0.037276in}}{\pgfqpoint{-0.029463in}{0.029463in}}%
\pgfpathcurveto{\pgfqpoint{-0.037276in}{0.021649in}}{\pgfqpoint{-0.041667in}{0.011050in}}{\pgfqpoint{-0.041667in}{0.000000in}}%
\pgfpathcurveto{\pgfqpoint{-0.041667in}{-0.011050in}}{\pgfqpoint{-0.037276in}{-0.021649in}}{\pgfqpoint{-0.029463in}{-0.029463in}}%
\pgfpathcurveto{\pgfqpoint{-0.021649in}{-0.037276in}}{\pgfqpoint{-0.011050in}{-0.041667in}}{\pgfqpoint{0.000000in}{-0.041667in}}%
\pgfpathclose%
\pgfusepath{stroke,fill}%
}%
\begin{pgfscope}%
\pgfsys@transformshift{3.493695in}{0.729706in}%
\pgfsys@useobject{currentmarker}{}%
\end{pgfscope}%
\end{pgfscope}%
\begin{pgfscope}%
\definecolor{textcolor}{rgb}{0.000000,0.000000,0.000000}%
\pgfsetstrokecolor{textcolor}%
\pgfsetfillcolor{textcolor}%
\pgftext[x=3.743695in,y=0.693247in,left,base]{\color{textcolor}\sffamily\fontsize{10.000000}{12.000000}\selectfont Pix3D}%
\end{pgfscope}%
\begin{pgfscope}%
\pgfsetbuttcap%
\pgfsetroundjoin%
\definecolor{currentfill}{rgb}{0.631373,0.788235,0.956863}%
\pgfsetfillcolor{currentfill}%
\pgfsetlinewidth{1.003750pt}%
\definecolor{currentstroke}{rgb}{0.631373,0.788235,0.956863}%
\pgfsetstrokecolor{currentstroke}%
\pgfsetdash{}{0pt}%
\pgfsys@defobject{currentmarker}{\pgfqpoint{-0.041667in}{-0.041667in}}{\pgfqpoint{0.041667in}{0.041667in}}{%
\pgfpathmoveto{\pgfqpoint{0.000000in}{-0.041667in}}%
\pgfpathcurveto{\pgfqpoint{0.011050in}{-0.041667in}}{\pgfqpoint{0.021649in}{-0.037276in}}{\pgfqpoint{0.029463in}{-0.029463in}}%
\pgfpathcurveto{\pgfqpoint{0.037276in}{-0.021649in}}{\pgfqpoint{0.041667in}{-0.011050in}}{\pgfqpoint{0.041667in}{0.000000in}}%
\pgfpathcurveto{\pgfqpoint{0.041667in}{0.011050in}}{\pgfqpoint{0.037276in}{0.021649in}}{\pgfqpoint{0.029463in}{0.029463in}}%
\pgfpathcurveto{\pgfqpoint{0.021649in}{0.037276in}}{\pgfqpoint{0.011050in}{0.041667in}}{\pgfqpoint{0.000000in}{0.041667in}}%
\pgfpathcurveto{\pgfqpoint{-0.011050in}{0.041667in}}{\pgfqpoint{-0.021649in}{0.037276in}}{\pgfqpoint{-0.029463in}{0.029463in}}%
\pgfpathcurveto{\pgfqpoint{-0.037276in}{0.021649in}}{\pgfqpoint{-0.041667in}{0.011050in}}{\pgfqpoint{-0.041667in}{0.000000in}}%
\pgfpathcurveto{\pgfqpoint{-0.041667in}{-0.011050in}}{\pgfqpoint{-0.037276in}{-0.021649in}}{\pgfqpoint{-0.029463in}{-0.029463in}}%
\pgfpathcurveto{\pgfqpoint{-0.021649in}{-0.037276in}}{\pgfqpoint{-0.011050in}{-0.041667in}}{\pgfqpoint{0.000000in}{-0.041667in}}%
\pgfpathclose%
\pgfusepath{stroke,fill}%
}%
\begin{pgfscope}%
\pgfsys@transformshift{3.493695in}{0.525849in}%
\pgfsys@useobject{currentmarker}{}%
\end{pgfscope}%
\end{pgfscope}%
\begin{pgfscope}%
\definecolor{textcolor}{rgb}{0.000000,0.000000,0.000000}%
\pgfsetstrokecolor{textcolor}%
\pgfsetfillcolor{textcolor}%
\pgftext[x=3.743695in,y=0.489390in,left,base]{\color{textcolor}\sffamily\fontsize{10.000000}{12.000000}\selectfont s2r3dfree\_multi-object}%
\end{pgfscope}%
\end{pgfpicture}%
\makeatother%
\endgroup%
}
    \resizebox{0.49\linewidth}{6cm}{%% Creator: Matplotlib, PGF backend
%%
%% To include the figure in your LaTeX document, write
%%   \input{<filename>.pgf}
%%
%% Make sure the required packages are loaded in your preamble
%%   \usepackage{pgf}
%%
%% Figures using additional raster images can only be included by \input if
%% they are in the same directory as the main LaTeX file. For loading figures
%% from other directories you can use the `import` package
%%   \usepackage{import}
%%
%% and then include the figures with
%%   \import{<path to file>}{<filename>.pgf}
%%
%% Matplotlib used the following preamble
%%   \usepackage{fontspec}
%%   \setmainfont{DejaVuSerif.ttf}[Path=\detokenize{/Users/apple/opt/anaconda3/envs/kaolin/lib/python3.7/site-packages/matplotlib/mpl-data/fonts/ttf/}]
%%   \setsansfont{DejaVuSans.ttf}[Path=\detokenize{/Users/apple/opt/anaconda3/envs/kaolin/lib/python3.7/site-packages/matplotlib/mpl-data/fonts/ttf/}]
%%   \setmonofont{DejaVuSansMono.ttf}[Path=\detokenize{/Users/apple/opt/anaconda3/envs/kaolin/lib/python3.7/site-packages/matplotlib/mpl-data/fonts/ttf/}]
%%
\begingroup%
\makeatletter%
\begin{pgfpicture}%
\pgfpathrectangle{\pgfpointorigin}{\pgfqpoint{12.476162in}{8.341596in}}%
\pgfusepath{use as bounding box, clip}%
\begin{pgfscope}%
\pgfsetbuttcap%
\pgfsetmiterjoin%
\definecolor{currentfill}{rgb}{1.000000,1.000000,1.000000}%
\pgfsetfillcolor{currentfill}%
\pgfsetlinewidth{0.000000pt}%
\definecolor{currentstroke}{rgb}{1.000000,1.000000,1.000000}%
\pgfsetstrokecolor{currentstroke}%
\pgfsetdash{}{0pt}%
\pgfpathmoveto{\pgfqpoint{0.000000in}{0.000000in}}%
\pgfpathlineto{\pgfqpoint{12.476162in}{0.000000in}}%
\pgfpathlineto{\pgfqpoint{12.476162in}{8.341596in}}%
\pgfpathlineto{\pgfqpoint{0.000000in}{8.341596in}}%
\pgfpathclose%
\pgfusepath{fill}%
\end{pgfscope}%
\begin{pgfscope}%
\pgfsetbuttcap%
\pgfsetmiterjoin%
\definecolor{currentfill}{rgb}{1.000000,1.000000,1.000000}%
\pgfsetfillcolor{currentfill}%
\pgfsetlinewidth{0.000000pt}%
\definecolor{currentstroke}{rgb}{0.000000,0.000000,0.000000}%
\pgfsetstrokecolor{currentstroke}%
\pgfsetstrokeopacity{0.000000}%
\pgfsetdash{}{0pt}%
\pgfpathmoveto{\pgfqpoint{0.481978in}{0.331635in}}%
\pgfpathlineto{\pgfqpoint{9.781978in}{0.331635in}}%
\pgfpathlineto{\pgfqpoint{9.781978in}{8.031635in}}%
\pgfpathlineto{\pgfqpoint{0.481978in}{8.031635in}}%
\pgfpathclose%
\pgfusepath{fill}%
\end{pgfscope}%
\begin{pgfscope}%
\pgfpathrectangle{\pgfqpoint{0.481978in}{0.331635in}}{\pgfqpoint{9.300000in}{7.700000in}}%
\pgfusepath{clip}%
\pgfsetbuttcap%
\pgfsetroundjoin%
\definecolor{currentfill}{rgb}{0.631373,0.788235,0.956863}%
\pgfsetfillcolor{currentfill}%
\pgfsetlinewidth{0.481800pt}%
\definecolor{currentstroke}{rgb}{1.000000,1.000000,1.000000}%
\pgfsetstrokecolor{currentstroke}%
\pgfsetdash{}{0pt}%
\pgfpathmoveto{\pgfqpoint{3.350250in}{0.795870in}}%
\pgfpathcurveto{\pgfqpoint{3.361300in}{0.795870in}}{\pgfqpoint{3.371899in}{0.800261in}}{\pgfqpoint{3.379713in}{0.808074in}}%
\pgfpathcurveto{\pgfqpoint{3.387527in}{0.815888in}}{\pgfqpoint{3.391917in}{0.826487in}}{\pgfqpoint{3.391917in}{0.837537in}}%
\pgfpathcurveto{\pgfqpoint{3.391917in}{0.848587in}}{\pgfqpoint{3.387527in}{0.859186in}}{\pgfqpoint{3.379713in}{0.867000in}}%
\pgfpathcurveto{\pgfqpoint{3.371899in}{0.874813in}}{\pgfqpoint{3.361300in}{0.879204in}}{\pgfqpoint{3.350250in}{0.879204in}}%
\pgfpathcurveto{\pgfqpoint{3.339200in}{0.879204in}}{\pgfqpoint{3.328601in}{0.874813in}}{\pgfqpoint{3.320788in}{0.867000in}}%
\pgfpathcurveto{\pgfqpoint{3.312974in}{0.859186in}}{\pgfqpoint{3.308584in}{0.848587in}}{\pgfqpoint{3.308584in}{0.837537in}}%
\pgfpathcurveto{\pgfqpoint{3.308584in}{0.826487in}}{\pgfqpoint{3.312974in}{0.815888in}}{\pgfqpoint{3.320788in}{0.808074in}}%
\pgfpathcurveto{\pgfqpoint{3.328601in}{0.800261in}}{\pgfqpoint{3.339200in}{0.795870in}}{\pgfqpoint{3.350250in}{0.795870in}}%
\pgfpathclose%
\pgfusepath{stroke,fill}%
\end{pgfscope}%
\begin{pgfscope}%
\pgfpathrectangle{\pgfqpoint{0.481978in}{0.331635in}}{\pgfqpoint{9.300000in}{7.700000in}}%
\pgfusepath{clip}%
\pgfsetbuttcap%
\pgfsetroundjoin%
\definecolor{currentfill}{rgb}{0.631373,0.788235,0.956863}%
\pgfsetfillcolor{currentfill}%
\pgfsetlinewidth{0.481800pt}%
\definecolor{currentstroke}{rgb}{1.000000,1.000000,1.000000}%
\pgfsetstrokecolor{currentstroke}%
\pgfsetdash{}{0pt}%
\pgfpathmoveto{\pgfqpoint{3.059115in}{5.220279in}}%
\pgfpathcurveto{\pgfqpoint{3.070165in}{5.220279in}}{\pgfqpoint{3.080764in}{5.224669in}}{\pgfqpoint{3.088578in}{5.232482in}}%
\pgfpathcurveto{\pgfqpoint{3.096392in}{5.240296in}}{\pgfqpoint{3.100782in}{5.250895in}}{\pgfqpoint{3.100782in}{5.261945in}}%
\pgfpathcurveto{\pgfqpoint{3.100782in}{5.272995in}}{\pgfqpoint{3.096392in}{5.283594in}}{\pgfqpoint{3.088578in}{5.291408in}}%
\pgfpathcurveto{\pgfqpoint{3.080764in}{5.299222in}}{\pgfqpoint{3.070165in}{5.303612in}}{\pgfqpoint{3.059115in}{5.303612in}}%
\pgfpathcurveto{\pgfqpoint{3.048065in}{5.303612in}}{\pgfqpoint{3.037466in}{5.299222in}}{\pgfqpoint{3.029652in}{5.291408in}}%
\pgfpathcurveto{\pgfqpoint{3.021839in}{5.283594in}}{\pgfqpoint{3.017448in}{5.272995in}}{\pgfqpoint{3.017448in}{5.261945in}}%
\pgfpathcurveto{\pgfqpoint{3.017448in}{5.250895in}}{\pgfqpoint{3.021839in}{5.240296in}}{\pgfqpoint{3.029652in}{5.232482in}}%
\pgfpathcurveto{\pgfqpoint{3.037466in}{5.224669in}}{\pgfqpoint{3.048065in}{5.220279in}}{\pgfqpoint{3.059115in}{5.220279in}}%
\pgfpathclose%
\pgfusepath{stroke,fill}%
\end{pgfscope}%
\begin{pgfscope}%
\pgfpathrectangle{\pgfqpoint{0.481978in}{0.331635in}}{\pgfqpoint{9.300000in}{7.700000in}}%
\pgfusepath{clip}%
\pgfsetbuttcap%
\pgfsetroundjoin%
\definecolor{currentfill}{rgb}{0.631373,0.788235,0.956863}%
\pgfsetfillcolor{currentfill}%
\pgfsetlinewidth{0.481800pt}%
\definecolor{currentstroke}{rgb}{1.000000,1.000000,1.000000}%
\pgfsetstrokecolor{currentstroke}%
\pgfsetdash{}{0pt}%
\pgfpathmoveto{\pgfqpoint{4.845549in}{3.728687in}}%
\pgfpathcurveto{\pgfqpoint{4.856599in}{3.728687in}}{\pgfqpoint{4.867198in}{3.733077in}}{\pgfqpoint{4.875011in}{3.740891in}}%
\pgfpathcurveto{\pgfqpoint{4.882825in}{3.748705in}}{\pgfqpoint{4.887215in}{3.759304in}}{\pgfqpoint{4.887215in}{3.770354in}}%
\pgfpathcurveto{\pgfqpoint{4.887215in}{3.781404in}}{\pgfqpoint{4.882825in}{3.792003in}}{\pgfqpoint{4.875011in}{3.799817in}}%
\pgfpathcurveto{\pgfqpoint{4.867198in}{3.807630in}}{\pgfqpoint{4.856599in}{3.812021in}}{\pgfqpoint{4.845549in}{3.812021in}}%
\pgfpathcurveto{\pgfqpoint{4.834499in}{3.812021in}}{\pgfqpoint{4.823900in}{3.807630in}}{\pgfqpoint{4.816086in}{3.799817in}}%
\pgfpathcurveto{\pgfqpoint{4.808272in}{3.792003in}}{\pgfqpoint{4.803882in}{3.781404in}}{\pgfqpoint{4.803882in}{3.770354in}}%
\pgfpathcurveto{\pgfqpoint{4.803882in}{3.759304in}}{\pgfqpoint{4.808272in}{3.748705in}}{\pgfqpoint{4.816086in}{3.740891in}}%
\pgfpathcurveto{\pgfqpoint{4.823900in}{3.733077in}}{\pgfqpoint{4.834499in}{3.728687in}}{\pgfqpoint{4.845549in}{3.728687in}}%
\pgfpathclose%
\pgfusepath{stroke,fill}%
\end{pgfscope}%
\begin{pgfscope}%
\pgfpathrectangle{\pgfqpoint{0.481978in}{0.331635in}}{\pgfqpoint{9.300000in}{7.700000in}}%
\pgfusepath{clip}%
\pgfsetbuttcap%
\pgfsetroundjoin%
\definecolor{currentfill}{rgb}{0.631373,0.788235,0.956863}%
\pgfsetfillcolor{currentfill}%
\pgfsetlinewidth{0.481800pt}%
\definecolor{currentstroke}{rgb}{1.000000,1.000000,1.000000}%
\pgfsetstrokecolor{currentstroke}%
\pgfsetdash{}{0pt}%
\pgfpathmoveto{\pgfqpoint{1.304480in}{3.048221in}}%
\pgfpathcurveto{\pgfqpoint{1.315530in}{3.048221in}}{\pgfqpoint{1.326129in}{3.052611in}}{\pgfqpoint{1.333942in}{3.060424in}}%
\pgfpathcurveto{\pgfqpoint{1.341756in}{3.068238in}}{\pgfqpoint{1.346146in}{3.078837in}}{\pgfqpoint{1.346146in}{3.089887in}}%
\pgfpathcurveto{\pgfqpoint{1.346146in}{3.100937in}}{\pgfqpoint{1.341756in}{3.111536in}}{\pgfqpoint{1.333942in}{3.119350in}}%
\pgfpathcurveto{\pgfqpoint{1.326129in}{3.127164in}}{\pgfqpoint{1.315530in}{3.131554in}}{\pgfqpoint{1.304480in}{3.131554in}}%
\pgfpathcurveto{\pgfqpoint{1.293429in}{3.131554in}}{\pgfqpoint{1.282830in}{3.127164in}}{\pgfqpoint{1.275017in}{3.119350in}}%
\pgfpathcurveto{\pgfqpoint{1.267203in}{3.111536in}}{\pgfqpoint{1.262813in}{3.100937in}}{\pgfqpoint{1.262813in}{3.089887in}}%
\pgfpathcurveto{\pgfqpoint{1.262813in}{3.078837in}}{\pgfqpoint{1.267203in}{3.068238in}}{\pgfqpoint{1.275017in}{3.060424in}}%
\pgfpathcurveto{\pgfqpoint{1.282830in}{3.052611in}}{\pgfqpoint{1.293429in}{3.048221in}}{\pgfqpoint{1.304480in}{3.048221in}}%
\pgfpathclose%
\pgfusepath{stroke,fill}%
\end{pgfscope}%
\begin{pgfscope}%
\pgfpathrectangle{\pgfqpoint{0.481978in}{0.331635in}}{\pgfqpoint{9.300000in}{7.700000in}}%
\pgfusepath{clip}%
\pgfsetbuttcap%
\pgfsetroundjoin%
\definecolor{currentfill}{rgb}{0.631373,0.788235,0.956863}%
\pgfsetfillcolor{currentfill}%
\pgfsetlinewidth{0.481800pt}%
\definecolor{currentstroke}{rgb}{1.000000,1.000000,1.000000}%
\pgfsetstrokecolor{currentstroke}%
\pgfsetdash{}{0pt}%
\pgfpathmoveto{\pgfqpoint{4.583472in}{2.392940in}}%
\pgfpathcurveto{\pgfqpoint{4.594522in}{2.392940in}}{\pgfqpoint{4.605121in}{2.397330in}}{\pgfqpoint{4.612935in}{2.405144in}}%
\pgfpathcurveto{\pgfqpoint{4.620748in}{2.412957in}}{\pgfqpoint{4.625139in}{2.423556in}}{\pgfqpoint{4.625139in}{2.434606in}}%
\pgfpathcurveto{\pgfqpoint{4.625139in}{2.445657in}}{\pgfqpoint{4.620748in}{2.456256in}}{\pgfqpoint{4.612935in}{2.464069in}}%
\pgfpathcurveto{\pgfqpoint{4.605121in}{2.471883in}}{\pgfqpoint{4.594522in}{2.476273in}}{\pgfqpoint{4.583472in}{2.476273in}}%
\pgfpathcurveto{\pgfqpoint{4.572422in}{2.476273in}}{\pgfqpoint{4.561823in}{2.471883in}}{\pgfqpoint{4.554009in}{2.464069in}}%
\pgfpathcurveto{\pgfqpoint{4.546196in}{2.456256in}}{\pgfqpoint{4.541805in}{2.445657in}}{\pgfqpoint{4.541805in}{2.434606in}}%
\pgfpathcurveto{\pgfqpoint{4.541805in}{2.423556in}}{\pgfqpoint{4.546196in}{2.412957in}}{\pgfqpoint{4.554009in}{2.405144in}}%
\pgfpathcurveto{\pgfqpoint{4.561823in}{2.397330in}}{\pgfqpoint{4.572422in}{2.392940in}}{\pgfqpoint{4.583472in}{2.392940in}}%
\pgfpathclose%
\pgfusepath{stroke,fill}%
\end{pgfscope}%
\begin{pgfscope}%
\pgfpathrectangle{\pgfqpoint{0.481978in}{0.331635in}}{\pgfqpoint{9.300000in}{7.700000in}}%
\pgfusepath{clip}%
\pgfsetbuttcap%
\pgfsetroundjoin%
\definecolor{currentfill}{rgb}{0.631373,0.788235,0.956863}%
\pgfsetfillcolor{currentfill}%
\pgfsetlinewidth{0.481800pt}%
\definecolor{currentstroke}{rgb}{1.000000,1.000000,1.000000}%
\pgfsetstrokecolor{currentstroke}%
\pgfsetdash{}{0pt}%
\pgfpathmoveto{\pgfqpoint{5.509106in}{5.260891in}}%
\pgfpathcurveto{\pgfqpoint{5.520156in}{5.260891in}}{\pgfqpoint{5.530755in}{5.265281in}}{\pgfqpoint{5.538568in}{5.273094in}}%
\pgfpathcurveto{\pgfqpoint{5.546382in}{5.280908in}}{\pgfqpoint{5.550772in}{5.291507in}}{\pgfqpoint{5.550772in}{5.302557in}}%
\pgfpathcurveto{\pgfqpoint{5.550772in}{5.313607in}}{\pgfqpoint{5.546382in}{5.324206in}}{\pgfqpoint{5.538568in}{5.332020in}}%
\pgfpathcurveto{\pgfqpoint{5.530755in}{5.339834in}}{\pgfqpoint{5.520156in}{5.344224in}}{\pgfqpoint{5.509106in}{5.344224in}}%
\pgfpathcurveto{\pgfqpoint{5.498055in}{5.344224in}}{\pgfqpoint{5.487456in}{5.339834in}}{\pgfqpoint{5.479643in}{5.332020in}}%
\pgfpathcurveto{\pgfqpoint{5.471829in}{5.324206in}}{\pgfqpoint{5.467439in}{5.313607in}}{\pgfqpoint{5.467439in}{5.302557in}}%
\pgfpathcurveto{\pgfqpoint{5.467439in}{5.291507in}}{\pgfqpoint{5.471829in}{5.280908in}}{\pgfqpoint{5.479643in}{5.273094in}}%
\pgfpathcurveto{\pgfqpoint{5.487456in}{5.265281in}}{\pgfqpoint{5.498055in}{5.260891in}}{\pgfqpoint{5.509106in}{5.260891in}}%
\pgfpathclose%
\pgfusepath{stroke,fill}%
\end{pgfscope}%
\begin{pgfscope}%
\pgfpathrectangle{\pgfqpoint{0.481978in}{0.331635in}}{\pgfqpoint{9.300000in}{7.700000in}}%
\pgfusepath{clip}%
\pgfsetbuttcap%
\pgfsetroundjoin%
\definecolor{currentfill}{rgb}{0.631373,0.788235,0.956863}%
\pgfsetfillcolor{currentfill}%
\pgfsetlinewidth{0.481800pt}%
\definecolor{currentstroke}{rgb}{1.000000,1.000000,1.000000}%
\pgfsetstrokecolor{currentstroke}%
\pgfsetdash{}{0pt}%
\pgfpathmoveto{\pgfqpoint{1.270538in}{3.021766in}}%
\pgfpathcurveto{\pgfqpoint{1.281588in}{3.021766in}}{\pgfqpoint{1.292187in}{3.026156in}}{\pgfqpoint{1.300001in}{3.033970in}}%
\pgfpathcurveto{\pgfqpoint{1.307815in}{3.041783in}}{\pgfqpoint{1.312205in}{3.052382in}}{\pgfqpoint{1.312205in}{3.063432in}}%
\pgfpathcurveto{\pgfqpoint{1.312205in}{3.074483in}}{\pgfqpoint{1.307815in}{3.085082in}}{\pgfqpoint{1.300001in}{3.092895in}}%
\pgfpathcurveto{\pgfqpoint{1.292187in}{3.100709in}}{\pgfqpoint{1.281588in}{3.105099in}}{\pgfqpoint{1.270538in}{3.105099in}}%
\pgfpathcurveto{\pgfqpoint{1.259488in}{3.105099in}}{\pgfqpoint{1.248889in}{3.100709in}}{\pgfqpoint{1.241075in}{3.092895in}}%
\pgfpathcurveto{\pgfqpoint{1.233262in}{3.085082in}}{\pgfqpoint{1.228872in}{3.074483in}}{\pgfqpoint{1.228872in}{3.063432in}}%
\pgfpathcurveto{\pgfqpoint{1.228872in}{3.052382in}}{\pgfqpoint{1.233262in}{3.041783in}}{\pgfqpoint{1.241075in}{3.033970in}}%
\pgfpathcurveto{\pgfqpoint{1.248889in}{3.026156in}}{\pgfqpoint{1.259488in}{3.021766in}}{\pgfqpoint{1.270538in}{3.021766in}}%
\pgfpathclose%
\pgfusepath{stroke,fill}%
\end{pgfscope}%
\begin{pgfscope}%
\pgfpathrectangle{\pgfqpoint{0.481978in}{0.331635in}}{\pgfqpoint{9.300000in}{7.700000in}}%
\pgfusepath{clip}%
\pgfsetbuttcap%
\pgfsetroundjoin%
\definecolor{currentfill}{rgb}{0.631373,0.788235,0.956863}%
\pgfsetfillcolor{currentfill}%
\pgfsetlinewidth{0.481800pt}%
\definecolor{currentstroke}{rgb}{1.000000,1.000000,1.000000}%
\pgfsetstrokecolor{currentstroke}%
\pgfsetdash{}{0pt}%
\pgfpathmoveto{\pgfqpoint{4.233419in}{2.502329in}}%
\pgfpathcurveto{\pgfqpoint{4.244470in}{2.502329in}}{\pgfqpoint{4.255069in}{2.506720in}}{\pgfqpoint{4.262882in}{2.514533in}}%
\pgfpathcurveto{\pgfqpoint{4.270696in}{2.522347in}}{\pgfqpoint{4.275086in}{2.532946in}}{\pgfqpoint{4.275086in}{2.543996in}}%
\pgfpathcurveto{\pgfqpoint{4.275086in}{2.555046in}}{\pgfqpoint{4.270696in}{2.565645in}}{\pgfqpoint{4.262882in}{2.573459in}}%
\pgfpathcurveto{\pgfqpoint{4.255069in}{2.581272in}}{\pgfqpoint{4.244470in}{2.585663in}}{\pgfqpoint{4.233419in}{2.585663in}}%
\pgfpathcurveto{\pgfqpoint{4.222369in}{2.585663in}}{\pgfqpoint{4.211770in}{2.581272in}}{\pgfqpoint{4.203957in}{2.573459in}}%
\pgfpathcurveto{\pgfqpoint{4.196143in}{2.565645in}}{\pgfqpoint{4.191753in}{2.555046in}}{\pgfqpoint{4.191753in}{2.543996in}}%
\pgfpathcurveto{\pgfqpoint{4.191753in}{2.532946in}}{\pgfqpoint{4.196143in}{2.522347in}}{\pgfqpoint{4.203957in}{2.514533in}}%
\pgfpathcurveto{\pgfqpoint{4.211770in}{2.506720in}}{\pgfqpoint{4.222369in}{2.502329in}}{\pgfqpoint{4.233419in}{2.502329in}}%
\pgfpathclose%
\pgfusepath{stroke,fill}%
\end{pgfscope}%
\begin{pgfscope}%
\pgfpathrectangle{\pgfqpoint{0.481978in}{0.331635in}}{\pgfqpoint{9.300000in}{7.700000in}}%
\pgfusepath{clip}%
\pgfsetbuttcap%
\pgfsetroundjoin%
\definecolor{currentfill}{rgb}{0.631373,0.788235,0.956863}%
\pgfsetfillcolor{currentfill}%
\pgfsetlinewidth{0.481800pt}%
\definecolor{currentstroke}{rgb}{1.000000,1.000000,1.000000}%
\pgfsetstrokecolor{currentstroke}%
\pgfsetdash{}{0pt}%
\pgfpathmoveto{\pgfqpoint{2.873447in}{4.546373in}}%
\pgfpathcurveto{\pgfqpoint{2.884497in}{4.546373in}}{\pgfqpoint{2.895096in}{4.550764in}}{\pgfqpoint{2.902910in}{4.558577in}}%
\pgfpathcurveto{\pgfqpoint{2.910723in}{4.566391in}}{\pgfqpoint{2.915114in}{4.576990in}}{\pgfqpoint{2.915114in}{4.588040in}}%
\pgfpathcurveto{\pgfqpoint{2.915114in}{4.599090in}}{\pgfqpoint{2.910723in}{4.609689in}}{\pgfqpoint{2.902910in}{4.617503in}}%
\pgfpathcurveto{\pgfqpoint{2.895096in}{4.625317in}}{\pgfqpoint{2.884497in}{4.629707in}}{\pgfqpoint{2.873447in}{4.629707in}}%
\pgfpathcurveto{\pgfqpoint{2.862397in}{4.629707in}}{\pgfqpoint{2.851798in}{4.625317in}}{\pgfqpoint{2.843984in}{4.617503in}}%
\pgfpathcurveto{\pgfqpoint{2.836171in}{4.609689in}}{\pgfqpoint{2.831780in}{4.599090in}}{\pgfqpoint{2.831780in}{4.588040in}}%
\pgfpathcurveto{\pgfqpoint{2.831780in}{4.576990in}}{\pgfqpoint{2.836171in}{4.566391in}}{\pgfqpoint{2.843984in}{4.558577in}}%
\pgfpathcurveto{\pgfqpoint{2.851798in}{4.550764in}}{\pgfqpoint{2.862397in}{4.546373in}}{\pgfqpoint{2.873447in}{4.546373in}}%
\pgfpathclose%
\pgfusepath{stroke,fill}%
\end{pgfscope}%
\begin{pgfscope}%
\pgfpathrectangle{\pgfqpoint{0.481978in}{0.331635in}}{\pgfqpoint{9.300000in}{7.700000in}}%
\pgfusepath{clip}%
\pgfsetbuttcap%
\pgfsetroundjoin%
\definecolor{currentfill}{rgb}{0.631373,0.788235,0.956863}%
\pgfsetfillcolor{currentfill}%
\pgfsetlinewidth{0.481800pt}%
\definecolor{currentstroke}{rgb}{1.000000,1.000000,1.000000}%
\pgfsetstrokecolor{currentstroke}%
\pgfsetdash{}{0pt}%
\pgfpathmoveto{\pgfqpoint{6.038124in}{7.639968in}}%
\pgfpathcurveto{\pgfqpoint{6.049174in}{7.639968in}}{\pgfqpoint{6.059773in}{7.644359in}}{\pgfqpoint{6.067587in}{7.652172in}}%
\pgfpathcurveto{\pgfqpoint{6.075401in}{7.659986in}}{\pgfqpoint{6.079791in}{7.670585in}}{\pgfqpoint{6.079791in}{7.681635in}}%
\pgfpathcurveto{\pgfqpoint{6.079791in}{7.692685in}}{\pgfqpoint{6.075401in}{7.703284in}}{\pgfqpoint{6.067587in}{7.711098in}}%
\pgfpathcurveto{\pgfqpoint{6.059773in}{7.718911in}}{\pgfqpoint{6.049174in}{7.723302in}}{\pgfqpoint{6.038124in}{7.723302in}}%
\pgfpathcurveto{\pgfqpoint{6.027074in}{7.723302in}}{\pgfqpoint{6.016475in}{7.718911in}}{\pgfqpoint{6.008661in}{7.711098in}}%
\pgfpathcurveto{\pgfqpoint{6.000848in}{7.703284in}}{\pgfqpoint{5.996458in}{7.692685in}}{\pgfqpoint{5.996458in}{7.681635in}}%
\pgfpathcurveto{\pgfqpoint{5.996458in}{7.670585in}}{\pgfqpoint{6.000848in}{7.659986in}}{\pgfqpoint{6.008661in}{7.652172in}}%
\pgfpathcurveto{\pgfqpoint{6.016475in}{7.644359in}}{\pgfqpoint{6.027074in}{7.639968in}}{\pgfqpoint{6.038124in}{7.639968in}}%
\pgfpathclose%
\pgfusepath{stroke,fill}%
\end{pgfscope}%
\begin{pgfscope}%
\pgfpathrectangle{\pgfqpoint{0.481978in}{0.331635in}}{\pgfqpoint{9.300000in}{7.700000in}}%
\pgfusepath{clip}%
\pgfsetbuttcap%
\pgfsetroundjoin%
\definecolor{currentfill}{rgb}{0.631373,0.788235,0.956863}%
\pgfsetfillcolor{currentfill}%
\pgfsetlinewidth{0.481800pt}%
\definecolor{currentstroke}{rgb}{1.000000,1.000000,1.000000}%
\pgfsetstrokecolor{currentstroke}%
\pgfsetdash{}{0pt}%
\pgfpathmoveto{\pgfqpoint{6.068535in}{7.007729in}}%
\pgfpathcurveto{\pgfqpoint{6.079585in}{7.007729in}}{\pgfqpoint{6.090184in}{7.012120in}}{\pgfqpoint{6.097998in}{7.019933in}}%
\pgfpathcurveto{\pgfqpoint{6.105812in}{7.027747in}}{\pgfqpoint{6.110202in}{7.038346in}}{\pgfqpoint{6.110202in}{7.049396in}}%
\pgfpathcurveto{\pgfqpoint{6.110202in}{7.060446in}}{\pgfqpoint{6.105812in}{7.071045in}}{\pgfqpoint{6.097998in}{7.078859in}}%
\pgfpathcurveto{\pgfqpoint{6.090184in}{7.086672in}}{\pgfqpoint{6.079585in}{7.091063in}}{\pgfqpoint{6.068535in}{7.091063in}}%
\pgfpathcurveto{\pgfqpoint{6.057485in}{7.091063in}}{\pgfqpoint{6.046886in}{7.086672in}}{\pgfqpoint{6.039072in}{7.078859in}}%
\pgfpathcurveto{\pgfqpoint{6.031259in}{7.071045in}}{\pgfqpoint{6.026869in}{7.060446in}}{\pgfqpoint{6.026869in}{7.049396in}}%
\pgfpathcurveto{\pgfqpoint{6.026869in}{7.038346in}}{\pgfqpoint{6.031259in}{7.027747in}}{\pgfqpoint{6.039072in}{7.019933in}}%
\pgfpathcurveto{\pgfqpoint{6.046886in}{7.012120in}}{\pgfqpoint{6.057485in}{7.007729in}}{\pgfqpoint{6.068535in}{7.007729in}}%
\pgfpathclose%
\pgfusepath{stroke,fill}%
\end{pgfscope}%
\begin{pgfscope}%
\pgfpathrectangle{\pgfqpoint{0.481978in}{0.331635in}}{\pgfqpoint{9.300000in}{7.700000in}}%
\pgfusepath{clip}%
\pgfsetbuttcap%
\pgfsetroundjoin%
\definecolor{currentfill}{rgb}{0.631373,0.788235,0.956863}%
\pgfsetfillcolor{currentfill}%
\pgfsetlinewidth{0.481800pt}%
\definecolor{currentstroke}{rgb}{1.000000,1.000000,1.000000}%
\pgfsetstrokecolor{currentstroke}%
\pgfsetdash{}{0pt}%
\pgfpathmoveto{\pgfqpoint{0.904705in}{2.664956in}}%
\pgfpathcurveto{\pgfqpoint{0.915755in}{2.664956in}}{\pgfqpoint{0.926354in}{2.669346in}}{\pgfqpoint{0.934168in}{2.677160in}}%
\pgfpathcurveto{\pgfqpoint{0.941982in}{2.684973in}}{\pgfqpoint{0.946372in}{2.695572in}}{\pgfqpoint{0.946372in}{2.706623in}}%
\pgfpathcurveto{\pgfqpoint{0.946372in}{2.717673in}}{\pgfqpoint{0.941982in}{2.728272in}}{\pgfqpoint{0.934168in}{2.736085in}}%
\pgfpathcurveto{\pgfqpoint{0.926354in}{2.743899in}}{\pgfqpoint{0.915755in}{2.748289in}}{\pgfqpoint{0.904705in}{2.748289in}}%
\pgfpathcurveto{\pgfqpoint{0.893655in}{2.748289in}}{\pgfqpoint{0.883056in}{2.743899in}}{\pgfqpoint{0.875242in}{2.736085in}}%
\pgfpathcurveto{\pgfqpoint{0.867429in}{2.728272in}}{\pgfqpoint{0.863039in}{2.717673in}}{\pgfqpoint{0.863039in}{2.706623in}}%
\pgfpathcurveto{\pgfqpoint{0.863039in}{2.695572in}}{\pgfqpoint{0.867429in}{2.684973in}}{\pgfqpoint{0.875242in}{2.677160in}}%
\pgfpathcurveto{\pgfqpoint{0.883056in}{2.669346in}}{\pgfqpoint{0.893655in}{2.664956in}}{\pgfqpoint{0.904705in}{2.664956in}}%
\pgfpathclose%
\pgfusepath{stroke,fill}%
\end{pgfscope}%
\begin{pgfscope}%
\pgfpathrectangle{\pgfqpoint{0.481978in}{0.331635in}}{\pgfqpoint{9.300000in}{7.700000in}}%
\pgfusepath{clip}%
\pgfsetbuttcap%
\pgfsetroundjoin%
\definecolor{currentfill}{rgb}{0.631373,0.788235,0.956863}%
\pgfsetfillcolor{currentfill}%
\pgfsetlinewidth{0.481800pt}%
\definecolor{currentstroke}{rgb}{1.000000,1.000000,1.000000}%
\pgfsetstrokecolor{currentstroke}%
\pgfsetdash{}{0pt}%
\pgfpathmoveto{\pgfqpoint{3.355697in}{0.882879in}}%
\pgfpathcurveto{\pgfqpoint{3.366747in}{0.882879in}}{\pgfqpoint{3.377346in}{0.887270in}}{\pgfqpoint{3.385160in}{0.895083in}}%
\pgfpathcurveto{\pgfqpoint{3.392973in}{0.902897in}}{\pgfqpoint{3.397364in}{0.913496in}}{\pgfqpoint{3.397364in}{0.924546in}}%
\pgfpathcurveto{\pgfqpoint{3.397364in}{0.935596in}}{\pgfqpoint{3.392973in}{0.946195in}}{\pgfqpoint{3.385160in}{0.954009in}}%
\pgfpathcurveto{\pgfqpoint{3.377346in}{0.961823in}}{\pgfqpoint{3.366747in}{0.966213in}}{\pgfqpoint{3.355697in}{0.966213in}}%
\pgfpathcurveto{\pgfqpoint{3.344647in}{0.966213in}}{\pgfqpoint{3.334048in}{0.961823in}}{\pgfqpoint{3.326234in}{0.954009in}}%
\pgfpathcurveto{\pgfqpoint{3.318420in}{0.946195in}}{\pgfqpoint{3.314030in}{0.935596in}}{\pgfqpoint{3.314030in}{0.924546in}}%
\pgfpathcurveto{\pgfqpoint{3.314030in}{0.913496in}}{\pgfqpoint{3.318420in}{0.902897in}}{\pgfqpoint{3.326234in}{0.895083in}}%
\pgfpathcurveto{\pgfqpoint{3.334048in}{0.887270in}}{\pgfqpoint{3.344647in}{0.882879in}}{\pgfqpoint{3.355697in}{0.882879in}}%
\pgfpathclose%
\pgfusepath{stroke,fill}%
\end{pgfscope}%
\begin{pgfscope}%
\pgfpathrectangle{\pgfqpoint{0.481978in}{0.331635in}}{\pgfqpoint{9.300000in}{7.700000in}}%
\pgfusepath{clip}%
\pgfsetbuttcap%
\pgfsetroundjoin%
\definecolor{currentfill}{rgb}{0.631373,0.788235,0.956863}%
\pgfsetfillcolor{currentfill}%
\pgfsetlinewidth{0.481800pt}%
\definecolor{currentstroke}{rgb}{1.000000,1.000000,1.000000}%
\pgfsetstrokecolor{currentstroke}%
\pgfsetdash{}{0pt}%
\pgfpathmoveto{\pgfqpoint{1.822383in}{3.039349in}}%
\pgfpathcurveto{\pgfqpoint{1.833433in}{3.039349in}}{\pgfqpoint{1.844032in}{3.043739in}}{\pgfqpoint{1.851846in}{3.051553in}}%
\pgfpathcurveto{\pgfqpoint{1.859659in}{3.059367in}}{\pgfqpoint{1.864050in}{3.069966in}}{\pgfqpoint{1.864050in}{3.081016in}}%
\pgfpathcurveto{\pgfqpoint{1.864050in}{3.092066in}}{\pgfqpoint{1.859659in}{3.102665in}}{\pgfqpoint{1.851846in}{3.110479in}}%
\pgfpathcurveto{\pgfqpoint{1.844032in}{3.118292in}}{\pgfqpoint{1.833433in}{3.122682in}}{\pgfqpoint{1.822383in}{3.122682in}}%
\pgfpathcurveto{\pgfqpoint{1.811333in}{3.122682in}}{\pgfqpoint{1.800734in}{3.118292in}}{\pgfqpoint{1.792920in}{3.110479in}}%
\pgfpathcurveto{\pgfqpoint{1.785107in}{3.102665in}}{\pgfqpoint{1.780716in}{3.092066in}}{\pgfqpoint{1.780716in}{3.081016in}}%
\pgfpathcurveto{\pgfqpoint{1.780716in}{3.069966in}}{\pgfqpoint{1.785107in}{3.059367in}}{\pgfqpoint{1.792920in}{3.051553in}}%
\pgfpathcurveto{\pgfqpoint{1.800734in}{3.043739in}}{\pgfqpoint{1.811333in}{3.039349in}}{\pgfqpoint{1.822383in}{3.039349in}}%
\pgfpathclose%
\pgfusepath{stroke,fill}%
\end{pgfscope}%
\begin{pgfscope}%
\pgfpathrectangle{\pgfqpoint{0.481978in}{0.331635in}}{\pgfqpoint{9.300000in}{7.700000in}}%
\pgfusepath{clip}%
\pgfsetbuttcap%
\pgfsetroundjoin%
\definecolor{currentfill}{rgb}{0.631373,0.788235,0.956863}%
\pgfsetfillcolor{currentfill}%
\pgfsetlinewidth{0.481800pt}%
\definecolor{currentstroke}{rgb}{1.000000,1.000000,1.000000}%
\pgfsetstrokecolor{currentstroke}%
\pgfsetdash{}{0pt}%
\pgfpathmoveto{\pgfqpoint{6.431702in}{7.269912in}}%
\pgfpathcurveto{\pgfqpoint{6.442753in}{7.269912in}}{\pgfqpoint{6.453352in}{7.274302in}}{\pgfqpoint{6.461165in}{7.282116in}}%
\pgfpathcurveto{\pgfqpoint{6.468979in}{7.289929in}}{\pgfqpoint{6.473369in}{7.300528in}}{\pgfqpoint{6.473369in}{7.311578in}}%
\pgfpathcurveto{\pgfqpoint{6.473369in}{7.322629in}}{\pgfqpoint{6.468979in}{7.333228in}}{\pgfqpoint{6.461165in}{7.341041in}}%
\pgfpathcurveto{\pgfqpoint{6.453352in}{7.348855in}}{\pgfqpoint{6.442753in}{7.353245in}}{\pgfqpoint{6.431702in}{7.353245in}}%
\pgfpathcurveto{\pgfqpoint{6.420652in}{7.353245in}}{\pgfqpoint{6.410053in}{7.348855in}}{\pgfqpoint{6.402240in}{7.341041in}}%
\pgfpathcurveto{\pgfqpoint{6.394426in}{7.333228in}}{\pgfqpoint{6.390036in}{7.322629in}}{\pgfqpoint{6.390036in}{7.311578in}}%
\pgfpathcurveto{\pgfqpoint{6.390036in}{7.300528in}}{\pgfqpoint{6.394426in}{7.289929in}}{\pgfqpoint{6.402240in}{7.282116in}}%
\pgfpathcurveto{\pgfqpoint{6.410053in}{7.274302in}}{\pgfqpoint{6.420652in}{7.269912in}}{\pgfqpoint{6.431702in}{7.269912in}}%
\pgfpathclose%
\pgfusepath{stroke,fill}%
\end{pgfscope}%
\begin{pgfscope}%
\pgfpathrectangle{\pgfqpoint{0.481978in}{0.331635in}}{\pgfqpoint{9.300000in}{7.700000in}}%
\pgfusepath{clip}%
\pgfsetbuttcap%
\pgfsetroundjoin%
\definecolor{currentfill}{rgb}{0.631373,0.788235,0.956863}%
\pgfsetfillcolor{currentfill}%
\pgfsetlinewidth{0.481800pt}%
\definecolor{currentstroke}{rgb}{1.000000,1.000000,1.000000}%
\pgfsetstrokecolor{currentstroke}%
\pgfsetdash{}{0pt}%
\pgfpathmoveto{\pgfqpoint{1.601212in}{5.246053in}}%
\pgfpathcurveto{\pgfqpoint{1.612262in}{5.246053in}}{\pgfqpoint{1.622861in}{5.250443in}}{\pgfqpoint{1.630675in}{5.258257in}}%
\pgfpathcurveto{\pgfqpoint{1.638488in}{5.266070in}}{\pgfqpoint{1.642878in}{5.276669in}}{\pgfqpoint{1.642878in}{5.287720in}}%
\pgfpathcurveto{\pgfqpoint{1.642878in}{5.298770in}}{\pgfqpoint{1.638488in}{5.309369in}}{\pgfqpoint{1.630675in}{5.317182in}}%
\pgfpathcurveto{\pgfqpoint{1.622861in}{5.324996in}}{\pgfqpoint{1.612262in}{5.329386in}}{\pgfqpoint{1.601212in}{5.329386in}}%
\pgfpathcurveto{\pgfqpoint{1.590162in}{5.329386in}}{\pgfqpoint{1.579563in}{5.324996in}}{\pgfqpoint{1.571749in}{5.317182in}}%
\pgfpathcurveto{\pgfqpoint{1.563935in}{5.309369in}}{\pgfqpoint{1.559545in}{5.298770in}}{\pgfqpoint{1.559545in}{5.287720in}}%
\pgfpathcurveto{\pgfqpoint{1.559545in}{5.276669in}}{\pgfqpoint{1.563935in}{5.266070in}}{\pgfqpoint{1.571749in}{5.258257in}}%
\pgfpathcurveto{\pgfqpoint{1.579563in}{5.250443in}}{\pgfqpoint{1.590162in}{5.246053in}}{\pgfqpoint{1.601212in}{5.246053in}}%
\pgfpathclose%
\pgfusepath{stroke,fill}%
\end{pgfscope}%
\begin{pgfscope}%
\pgfpathrectangle{\pgfqpoint{0.481978in}{0.331635in}}{\pgfqpoint{9.300000in}{7.700000in}}%
\pgfusepath{clip}%
\pgfsetbuttcap%
\pgfsetroundjoin%
\definecolor{currentfill}{rgb}{0.631373,0.788235,0.956863}%
\pgfsetfillcolor{currentfill}%
\pgfsetlinewidth{0.481800pt}%
\definecolor{currentstroke}{rgb}{1.000000,1.000000,1.000000}%
\pgfsetstrokecolor{currentstroke}%
\pgfsetdash{}{0pt}%
\pgfpathmoveto{\pgfqpoint{2.280847in}{6.304822in}}%
\pgfpathcurveto{\pgfqpoint{2.291897in}{6.304822in}}{\pgfqpoint{2.302496in}{6.309212in}}{\pgfqpoint{2.310310in}{6.317026in}}%
\pgfpathcurveto{\pgfqpoint{2.318123in}{6.324840in}}{\pgfqpoint{2.322514in}{6.335439in}}{\pgfqpoint{2.322514in}{6.346489in}}%
\pgfpathcurveto{\pgfqpoint{2.322514in}{6.357539in}}{\pgfqpoint{2.318123in}{6.368138in}}{\pgfqpoint{2.310310in}{6.375951in}}%
\pgfpathcurveto{\pgfqpoint{2.302496in}{6.383765in}}{\pgfqpoint{2.291897in}{6.388155in}}{\pgfqpoint{2.280847in}{6.388155in}}%
\pgfpathcurveto{\pgfqpoint{2.269797in}{6.388155in}}{\pgfqpoint{2.259198in}{6.383765in}}{\pgfqpoint{2.251384in}{6.375951in}}%
\pgfpathcurveto{\pgfqpoint{2.243570in}{6.368138in}}{\pgfqpoint{2.239180in}{6.357539in}}{\pgfqpoint{2.239180in}{6.346489in}}%
\pgfpathcurveto{\pgfqpoint{2.239180in}{6.335439in}}{\pgfqpoint{2.243570in}{6.324840in}}{\pgfqpoint{2.251384in}{6.317026in}}%
\pgfpathcurveto{\pgfqpoint{2.259198in}{6.309212in}}{\pgfqpoint{2.269797in}{6.304822in}}{\pgfqpoint{2.280847in}{6.304822in}}%
\pgfpathclose%
\pgfusepath{stroke,fill}%
\end{pgfscope}%
\begin{pgfscope}%
\pgfpathrectangle{\pgfqpoint{0.481978in}{0.331635in}}{\pgfqpoint{9.300000in}{7.700000in}}%
\pgfusepath{clip}%
\pgfsetbuttcap%
\pgfsetroundjoin%
\definecolor{currentfill}{rgb}{0.631373,0.788235,0.956863}%
\pgfsetfillcolor{currentfill}%
\pgfsetlinewidth{0.481800pt}%
\definecolor{currentstroke}{rgb}{1.000000,1.000000,1.000000}%
\pgfsetstrokecolor{currentstroke}%
\pgfsetdash{}{0pt}%
\pgfpathmoveto{\pgfqpoint{1.914988in}{4.693993in}}%
\pgfpathcurveto{\pgfqpoint{1.926039in}{4.693993in}}{\pgfqpoint{1.936638in}{4.698383in}}{\pgfqpoint{1.944451in}{4.706197in}}%
\pgfpathcurveto{\pgfqpoint{1.952265in}{4.714010in}}{\pgfqpoint{1.956655in}{4.724609in}}{\pgfqpoint{1.956655in}{4.735660in}}%
\pgfpathcurveto{\pgfqpoint{1.956655in}{4.746710in}}{\pgfqpoint{1.952265in}{4.757309in}}{\pgfqpoint{1.944451in}{4.765122in}}%
\pgfpathcurveto{\pgfqpoint{1.936638in}{4.772936in}}{\pgfqpoint{1.926039in}{4.777326in}}{\pgfqpoint{1.914988in}{4.777326in}}%
\pgfpathcurveto{\pgfqpoint{1.903938in}{4.777326in}}{\pgfqpoint{1.893339in}{4.772936in}}{\pgfqpoint{1.885526in}{4.765122in}}%
\pgfpathcurveto{\pgfqpoint{1.877712in}{4.757309in}}{\pgfqpoint{1.873322in}{4.746710in}}{\pgfqpoint{1.873322in}{4.735660in}}%
\pgfpathcurveto{\pgfqpoint{1.873322in}{4.724609in}}{\pgfqpoint{1.877712in}{4.714010in}}{\pgfqpoint{1.885526in}{4.706197in}}%
\pgfpathcurveto{\pgfqpoint{1.893339in}{4.698383in}}{\pgfqpoint{1.903938in}{4.693993in}}{\pgfqpoint{1.914988in}{4.693993in}}%
\pgfpathclose%
\pgfusepath{stroke,fill}%
\end{pgfscope}%
\begin{pgfscope}%
\pgfpathrectangle{\pgfqpoint{0.481978in}{0.331635in}}{\pgfqpoint{9.300000in}{7.700000in}}%
\pgfusepath{clip}%
\pgfsetbuttcap%
\pgfsetroundjoin%
\definecolor{currentfill}{rgb}{0.631373,0.788235,0.956863}%
\pgfsetfillcolor{currentfill}%
\pgfsetlinewidth{0.481800pt}%
\definecolor{currentstroke}{rgb}{1.000000,1.000000,1.000000}%
\pgfsetstrokecolor{currentstroke}%
\pgfsetdash{}{0pt}%
\pgfpathmoveto{\pgfqpoint{3.751753in}{1.411582in}}%
\pgfpathcurveto{\pgfqpoint{3.762803in}{1.411582in}}{\pgfqpoint{3.773402in}{1.415972in}}{\pgfqpoint{3.781215in}{1.423785in}}%
\pgfpathcurveto{\pgfqpoint{3.789029in}{1.431599in}}{\pgfqpoint{3.793419in}{1.442198in}}{\pgfqpoint{3.793419in}{1.453248in}}%
\pgfpathcurveto{\pgfqpoint{3.793419in}{1.464298in}}{\pgfqpoint{3.789029in}{1.474897in}}{\pgfqpoint{3.781215in}{1.482711in}}%
\pgfpathcurveto{\pgfqpoint{3.773402in}{1.490525in}}{\pgfqpoint{3.762803in}{1.494915in}}{\pgfqpoint{3.751753in}{1.494915in}}%
\pgfpathcurveto{\pgfqpoint{3.740702in}{1.494915in}}{\pgfqpoint{3.730103in}{1.490525in}}{\pgfqpoint{3.722290in}{1.482711in}}%
\pgfpathcurveto{\pgfqpoint{3.714476in}{1.474897in}}{\pgfqpoint{3.710086in}{1.464298in}}{\pgfqpoint{3.710086in}{1.453248in}}%
\pgfpathcurveto{\pgfqpoint{3.710086in}{1.442198in}}{\pgfqpoint{3.714476in}{1.431599in}}{\pgfqpoint{3.722290in}{1.423785in}}%
\pgfpathcurveto{\pgfqpoint{3.730103in}{1.415972in}}{\pgfqpoint{3.740702in}{1.411582in}}{\pgfqpoint{3.751753in}{1.411582in}}%
\pgfpathclose%
\pgfusepath{stroke,fill}%
\end{pgfscope}%
\begin{pgfscope}%
\pgfpathrectangle{\pgfqpoint{0.481978in}{0.331635in}}{\pgfqpoint{9.300000in}{7.700000in}}%
\pgfusepath{clip}%
\pgfsetbuttcap%
\pgfsetroundjoin%
\definecolor{currentfill}{rgb}{0.631373,0.788235,0.956863}%
\pgfsetfillcolor{currentfill}%
\pgfsetlinewidth{0.481800pt}%
\definecolor{currentstroke}{rgb}{1.000000,1.000000,1.000000}%
\pgfsetstrokecolor{currentstroke}%
\pgfsetdash{}{0pt}%
\pgfpathmoveto{\pgfqpoint{5.753665in}{7.213922in}}%
\pgfpathcurveto{\pgfqpoint{5.764715in}{7.213922in}}{\pgfqpoint{5.775314in}{7.218313in}}{\pgfqpoint{5.783128in}{7.226126in}}%
\pgfpathcurveto{\pgfqpoint{5.790942in}{7.233940in}}{\pgfqpoint{5.795332in}{7.244539in}}{\pgfqpoint{5.795332in}{7.255589in}}%
\pgfpathcurveto{\pgfqpoint{5.795332in}{7.266639in}}{\pgfqpoint{5.790942in}{7.277238in}}{\pgfqpoint{5.783128in}{7.285052in}}%
\pgfpathcurveto{\pgfqpoint{5.775314in}{7.292865in}}{\pgfqpoint{5.764715in}{7.297256in}}{\pgfqpoint{5.753665in}{7.297256in}}%
\pgfpathcurveto{\pgfqpoint{5.742615in}{7.297256in}}{\pgfqpoint{5.732016in}{7.292865in}}{\pgfqpoint{5.724203in}{7.285052in}}%
\pgfpathcurveto{\pgfqpoint{5.716389in}{7.277238in}}{\pgfqpoint{5.711999in}{7.266639in}}{\pgfqpoint{5.711999in}{7.255589in}}%
\pgfpathcurveto{\pgfqpoint{5.711999in}{7.244539in}}{\pgfqpoint{5.716389in}{7.233940in}}{\pgfqpoint{5.724203in}{7.226126in}}%
\pgfpathcurveto{\pgfqpoint{5.732016in}{7.218313in}}{\pgfqpoint{5.742615in}{7.213922in}}{\pgfqpoint{5.753665in}{7.213922in}}%
\pgfpathclose%
\pgfusepath{stroke,fill}%
\end{pgfscope}%
\begin{pgfscope}%
\pgfpathrectangle{\pgfqpoint{0.481978in}{0.331635in}}{\pgfqpoint{9.300000in}{7.700000in}}%
\pgfusepath{clip}%
\pgfsetbuttcap%
\pgfsetroundjoin%
\definecolor{currentfill}{rgb}{0.631373,0.788235,0.956863}%
\pgfsetfillcolor{currentfill}%
\pgfsetlinewidth{0.481800pt}%
\definecolor{currentstroke}{rgb}{1.000000,1.000000,1.000000}%
\pgfsetstrokecolor{currentstroke}%
\pgfsetdash{}{0pt}%
\pgfpathmoveto{\pgfqpoint{5.495955in}{5.191634in}}%
\pgfpathcurveto{\pgfqpoint{5.507005in}{5.191634in}}{\pgfqpoint{5.517604in}{5.196024in}}{\pgfqpoint{5.525417in}{5.203837in}}%
\pgfpathcurveto{\pgfqpoint{5.533231in}{5.211651in}}{\pgfqpoint{5.537621in}{5.222250in}}{\pgfqpoint{5.537621in}{5.233300in}}%
\pgfpathcurveto{\pgfqpoint{5.537621in}{5.244350in}}{\pgfqpoint{5.533231in}{5.254949in}}{\pgfqpoint{5.525417in}{5.262763in}}%
\pgfpathcurveto{\pgfqpoint{5.517604in}{5.270577in}}{\pgfqpoint{5.507005in}{5.274967in}}{\pgfqpoint{5.495955in}{5.274967in}}%
\pgfpathcurveto{\pgfqpoint{5.484904in}{5.274967in}}{\pgfqpoint{5.474305in}{5.270577in}}{\pgfqpoint{5.466492in}{5.262763in}}%
\pgfpathcurveto{\pgfqpoint{5.458678in}{5.254949in}}{\pgfqpoint{5.454288in}{5.244350in}}{\pgfqpoint{5.454288in}{5.233300in}}%
\pgfpathcurveto{\pgfqpoint{5.454288in}{5.222250in}}{\pgfqpoint{5.458678in}{5.211651in}}{\pgfqpoint{5.466492in}{5.203837in}}%
\pgfpathcurveto{\pgfqpoint{5.474305in}{5.196024in}}{\pgfqpoint{5.484904in}{5.191634in}}{\pgfqpoint{5.495955in}{5.191634in}}%
\pgfpathclose%
\pgfusepath{stroke,fill}%
\end{pgfscope}%
\begin{pgfscope}%
\pgfpathrectangle{\pgfqpoint{0.481978in}{0.331635in}}{\pgfqpoint{9.300000in}{7.700000in}}%
\pgfusepath{clip}%
\pgfsetbuttcap%
\pgfsetroundjoin%
\definecolor{currentfill}{rgb}{0.631373,0.788235,0.956863}%
\pgfsetfillcolor{currentfill}%
\pgfsetlinewidth{0.481800pt}%
\definecolor{currentstroke}{rgb}{1.000000,1.000000,1.000000}%
\pgfsetstrokecolor{currentstroke}%
\pgfsetdash{}{0pt}%
\pgfpathmoveto{\pgfqpoint{4.238672in}{4.793073in}}%
\pgfpathcurveto{\pgfqpoint{4.249723in}{4.793073in}}{\pgfqpoint{4.260322in}{4.797463in}}{\pgfqpoint{4.268135in}{4.805277in}}%
\pgfpathcurveto{\pgfqpoint{4.275949in}{4.813090in}}{\pgfqpoint{4.280339in}{4.823689in}}{\pgfqpoint{4.280339in}{4.834740in}}%
\pgfpathcurveto{\pgfqpoint{4.280339in}{4.845790in}}{\pgfqpoint{4.275949in}{4.856389in}}{\pgfqpoint{4.268135in}{4.864202in}}%
\pgfpathcurveto{\pgfqpoint{4.260322in}{4.872016in}}{\pgfqpoint{4.249723in}{4.876406in}}{\pgfqpoint{4.238672in}{4.876406in}}%
\pgfpathcurveto{\pgfqpoint{4.227622in}{4.876406in}}{\pgfqpoint{4.217023in}{4.872016in}}{\pgfqpoint{4.209210in}{4.864202in}}%
\pgfpathcurveto{\pgfqpoint{4.201396in}{4.856389in}}{\pgfqpoint{4.197006in}{4.845790in}}{\pgfqpoint{4.197006in}{4.834740in}}%
\pgfpathcurveto{\pgfqpoint{4.197006in}{4.823689in}}{\pgfqpoint{4.201396in}{4.813090in}}{\pgfqpoint{4.209210in}{4.805277in}}%
\pgfpathcurveto{\pgfqpoint{4.217023in}{4.797463in}}{\pgfqpoint{4.227622in}{4.793073in}}{\pgfqpoint{4.238672in}{4.793073in}}%
\pgfpathclose%
\pgfusepath{stroke,fill}%
\end{pgfscope}%
\begin{pgfscope}%
\pgfpathrectangle{\pgfqpoint{0.481978in}{0.331635in}}{\pgfqpoint{9.300000in}{7.700000in}}%
\pgfusepath{clip}%
\pgfsetbuttcap%
\pgfsetroundjoin%
\definecolor{currentfill}{rgb}{0.631373,0.788235,0.956863}%
\pgfsetfillcolor{currentfill}%
\pgfsetlinewidth{0.481800pt}%
\definecolor{currentstroke}{rgb}{1.000000,1.000000,1.000000}%
\pgfsetstrokecolor{currentstroke}%
\pgfsetdash{}{0pt}%
\pgfpathmoveto{\pgfqpoint{6.175837in}{7.474137in}}%
\pgfpathcurveto{\pgfqpoint{6.186887in}{7.474137in}}{\pgfqpoint{6.197486in}{7.478527in}}{\pgfqpoint{6.205300in}{7.486341in}}%
\pgfpathcurveto{\pgfqpoint{6.213114in}{7.494154in}}{\pgfqpoint{6.217504in}{7.504753in}}{\pgfqpoint{6.217504in}{7.515803in}}%
\pgfpathcurveto{\pgfqpoint{6.217504in}{7.526854in}}{\pgfqpoint{6.213114in}{7.537453in}}{\pgfqpoint{6.205300in}{7.545266in}}%
\pgfpathcurveto{\pgfqpoint{6.197486in}{7.553080in}}{\pgfqpoint{6.186887in}{7.557470in}}{\pgfqpoint{6.175837in}{7.557470in}}%
\pgfpathcurveto{\pgfqpoint{6.164787in}{7.557470in}}{\pgfqpoint{6.154188in}{7.553080in}}{\pgfqpoint{6.146374in}{7.545266in}}%
\pgfpathcurveto{\pgfqpoint{6.138561in}{7.537453in}}{\pgfqpoint{6.134170in}{7.526854in}}{\pgfqpoint{6.134170in}{7.515803in}}%
\pgfpathcurveto{\pgfqpoint{6.134170in}{7.504753in}}{\pgfqpoint{6.138561in}{7.494154in}}{\pgfqpoint{6.146374in}{7.486341in}}%
\pgfpathcurveto{\pgfqpoint{6.154188in}{7.478527in}}{\pgfqpoint{6.164787in}{7.474137in}}{\pgfqpoint{6.175837in}{7.474137in}}%
\pgfpathclose%
\pgfusepath{stroke,fill}%
\end{pgfscope}%
\begin{pgfscope}%
\pgfpathrectangle{\pgfqpoint{0.481978in}{0.331635in}}{\pgfqpoint{9.300000in}{7.700000in}}%
\pgfusepath{clip}%
\pgfsetbuttcap%
\pgfsetroundjoin%
\definecolor{currentfill}{rgb}{0.631373,0.788235,0.956863}%
\pgfsetfillcolor{currentfill}%
\pgfsetlinewidth{0.481800pt}%
\definecolor{currentstroke}{rgb}{1.000000,1.000000,1.000000}%
\pgfsetstrokecolor{currentstroke}%
\pgfsetdash{}{0pt}%
\pgfpathmoveto{\pgfqpoint{4.277708in}{2.774798in}}%
\pgfpathcurveto{\pgfqpoint{4.288759in}{2.774798in}}{\pgfqpoint{4.299358in}{2.779188in}}{\pgfqpoint{4.307171in}{2.787002in}}%
\pgfpathcurveto{\pgfqpoint{4.314985in}{2.794816in}}{\pgfqpoint{4.319375in}{2.805415in}}{\pgfqpoint{4.319375in}{2.816465in}}%
\pgfpathcurveto{\pgfqpoint{4.319375in}{2.827515in}}{\pgfqpoint{4.314985in}{2.838114in}}{\pgfqpoint{4.307171in}{2.845928in}}%
\pgfpathcurveto{\pgfqpoint{4.299358in}{2.853741in}}{\pgfqpoint{4.288759in}{2.858131in}}{\pgfqpoint{4.277708in}{2.858131in}}%
\pgfpathcurveto{\pgfqpoint{4.266658in}{2.858131in}}{\pgfqpoint{4.256059in}{2.853741in}}{\pgfqpoint{4.248246in}{2.845928in}}%
\pgfpathcurveto{\pgfqpoint{4.240432in}{2.838114in}}{\pgfqpoint{4.236042in}{2.827515in}}{\pgfqpoint{4.236042in}{2.816465in}}%
\pgfpathcurveto{\pgfqpoint{4.236042in}{2.805415in}}{\pgfqpoint{4.240432in}{2.794816in}}{\pgfqpoint{4.248246in}{2.787002in}}%
\pgfpathcurveto{\pgfqpoint{4.256059in}{2.779188in}}{\pgfqpoint{4.266658in}{2.774798in}}{\pgfqpoint{4.277708in}{2.774798in}}%
\pgfpathclose%
\pgfusepath{stroke,fill}%
\end{pgfscope}%
\begin{pgfscope}%
\pgfpathrectangle{\pgfqpoint{0.481978in}{0.331635in}}{\pgfqpoint{9.300000in}{7.700000in}}%
\pgfusepath{clip}%
\pgfsetbuttcap%
\pgfsetroundjoin%
\definecolor{currentfill}{rgb}{0.631373,0.788235,0.956863}%
\pgfsetfillcolor{currentfill}%
\pgfsetlinewidth{0.481800pt}%
\definecolor{currentstroke}{rgb}{1.000000,1.000000,1.000000}%
\pgfsetstrokecolor{currentstroke}%
\pgfsetdash{}{0pt}%
\pgfpathmoveto{\pgfqpoint{1.180315in}{4.828736in}}%
\pgfpathcurveto{\pgfqpoint{1.191365in}{4.828736in}}{\pgfqpoint{1.201964in}{4.833126in}}{\pgfqpoint{1.209778in}{4.840939in}}%
\pgfpathcurveto{\pgfqpoint{1.217592in}{4.848753in}}{\pgfqpoint{1.221982in}{4.859352in}}{\pgfqpoint{1.221982in}{4.870402in}}%
\pgfpathcurveto{\pgfqpoint{1.221982in}{4.881452in}}{\pgfqpoint{1.217592in}{4.892051in}}{\pgfqpoint{1.209778in}{4.899865in}}%
\pgfpathcurveto{\pgfqpoint{1.201964in}{4.907679in}}{\pgfqpoint{1.191365in}{4.912069in}}{\pgfqpoint{1.180315in}{4.912069in}}%
\pgfpathcurveto{\pgfqpoint{1.169265in}{4.912069in}}{\pgfqpoint{1.158666in}{4.907679in}}{\pgfqpoint{1.150852in}{4.899865in}}%
\pgfpathcurveto{\pgfqpoint{1.143039in}{4.892051in}}{\pgfqpoint{1.138649in}{4.881452in}}{\pgfqpoint{1.138649in}{4.870402in}}%
\pgfpathcurveto{\pgfqpoint{1.138649in}{4.859352in}}{\pgfqpoint{1.143039in}{4.848753in}}{\pgfqpoint{1.150852in}{4.840939in}}%
\pgfpathcurveto{\pgfqpoint{1.158666in}{4.833126in}}{\pgfqpoint{1.169265in}{4.828736in}}{\pgfqpoint{1.180315in}{4.828736in}}%
\pgfpathclose%
\pgfusepath{stroke,fill}%
\end{pgfscope}%
\begin{pgfscope}%
\pgfpathrectangle{\pgfqpoint{0.481978in}{0.331635in}}{\pgfqpoint{9.300000in}{7.700000in}}%
\pgfusepath{clip}%
\pgfsetbuttcap%
\pgfsetroundjoin%
\definecolor{currentfill}{rgb}{0.631373,0.788235,0.956863}%
\pgfsetfillcolor{currentfill}%
\pgfsetlinewidth{0.481800pt}%
\definecolor{currentstroke}{rgb}{1.000000,1.000000,1.000000}%
\pgfsetstrokecolor{currentstroke}%
\pgfsetdash{}{0pt}%
\pgfpathmoveto{\pgfqpoint{4.575953in}{4.795945in}}%
\pgfpathcurveto{\pgfqpoint{4.587003in}{4.795945in}}{\pgfqpoint{4.597602in}{4.800336in}}{\pgfqpoint{4.605416in}{4.808149in}}%
\pgfpathcurveto{\pgfqpoint{4.613229in}{4.815963in}}{\pgfqpoint{4.617619in}{4.826562in}}{\pgfqpoint{4.617619in}{4.837612in}}%
\pgfpathcurveto{\pgfqpoint{4.617619in}{4.848662in}}{\pgfqpoint{4.613229in}{4.859261in}}{\pgfqpoint{4.605416in}{4.867075in}}%
\pgfpathcurveto{\pgfqpoint{4.597602in}{4.874888in}}{\pgfqpoint{4.587003in}{4.879279in}}{\pgfqpoint{4.575953in}{4.879279in}}%
\pgfpathcurveto{\pgfqpoint{4.564903in}{4.879279in}}{\pgfqpoint{4.554304in}{4.874888in}}{\pgfqpoint{4.546490in}{4.867075in}}%
\pgfpathcurveto{\pgfqpoint{4.538676in}{4.859261in}}{\pgfqpoint{4.534286in}{4.848662in}}{\pgfqpoint{4.534286in}{4.837612in}}%
\pgfpathcurveto{\pgfqpoint{4.534286in}{4.826562in}}{\pgfqpoint{4.538676in}{4.815963in}}{\pgfqpoint{4.546490in}{4.808149in}}%
\pgfpathcurveto{\pgfqpoint{4.554304in}{4.800336in}}{\pgfqpoint{4.564903in}{4.795945in}}{\pgfqpoint{4.575953in}{4.795945in}}%
\pgfpathclose%
\pgfusepath{stroke,fill}%
\end{pgfscope}%
\begin{pgfscope}%
\pgfpathrectangle{\pgfqpoint{0.481978in}{0.331635in}}{\pgfqpoint{9.300000in}{7.700000in}}%
\pgfusepath{clip}%
\pgfsetbuttcap%
\pgfsetroundjoin%
\definecolor{currentfill}{rgb}{0.631373,0.788235,0.956863}%
\pgfsetfillcolor{currentfill}%
\pgfsetlinewidth{0.481800pt}%
\definecolor{currentstroke}{rgb}{1.000000,1.000000,1.000000}%
\pgfsetstrokecolor{currentstroke}%
\pgfsetdash{}{0pt}%
\pgfpathmoveto{\pgfqpoint{7.415126in}{4.531322in}}%
\pgfpathcurveto{\pgfqpoint{7.426176in}{4.531322in}}{\pgfqpoint{7.436775in}{4.535712in}}{\pgfqpoint{7.444589in}{4.543526in}}%
\pgfpathcurveto{\pgfqpoint{7.452402in}{4.551340in}}{\pgfqpoint{7.456793in}{4.561939in}}{\pgfqpoint{7.456793in}{4.572989in}}%
\pgfpathcurveto{\pgfqpoint{7.456793in}{4.584039in}}{\pgfqpoint{7.452402in}{4.594638in}}{\pgfqpoint{7.444589in}{4.602452in}}%
\pgfpathcurveto{\pgfqpoint{7.436775in}{4.610265in}}{\pgfqpoint{7.426176in}{4.614655in}}{\pgfqpoint{7.415126in}{4.614655in}}%
\pgfpathcurveto{\pgfqpoint{7.404076in}{4.614655in}}{\pgfqpoint{7.393477in}{4.610265in}}{\pgfqpoint{7.385663in}{4.602452in}}%
\pgfpathcurveto{\pgfqpoint{7.377850in}{4.594638in}}{\pgfqpoint{7.373459in}{4.584039in}}{\pgfqpoint{7.373459in}{4.572989in}}%
\pgfpathcurveto{\pgfqpoint{7.373459in}{4.561939in}}{\pgfqpoint{7.377850in}{4.551340in}}{\pgfqpoint{7.385663in}{4.543526in}}%
\pgfpathcurveto{\pgfqpoint{7.393477in}{4.535712in}}{\pgfqpoint{7.404076in}{4.531322in}}{\pgfqpoint{7.415126in}{4.531322in}}%
\pgfpathclose%
\pgfusepath{stroke,fill}%
\end{pgfscope}%
\begin{pgfscope}%
\pgfpathrectangle{\pgfqpoint{0.481978in}{0.331635in}}{\pgfqpoint{9.300000in}{7.700000in}}%
\pgfusepath{clip}%
\pgfsetbuttcap%
\pgfsetroundjoin%
\definecolor{currentfill}{rgb}{0.631373,0.788235,0.956863}%
\pgfsetfillcolor{currentfill}%
\pgfsetlinewidth{0.481800pt}%
\definecolor{currentstroke}{rgb}{1.000000,1.000000,1.000000}%
\pgfsetstrokecolor{currentstroke}%
\pgfsetdash{}{0pt}%
\pgfpathmoveto{\pgfqpoint{5.743162in}{7.207909in}}%
\pgfpathcurveto{\pgfqpoint{5.754212in}{7.207909in}}{\pgfqpoint{5.764811in}{7.212299in}}{\pgfqpoint{5.772625in}{7.220113in}}%
\pgfpathcurveto{\pgfqpoint{5.780439in}{7.227926in}}{\pgfqpoint{5.784829in}{7.238525in}}{\pgfqpoint{5.784829in}{7.249576in}}%
\pgfpathcurveto{\pgfqpoint{5.784829in}{7.260626in}}{\pgfqpoint{5.780439in}{7.271225in}}{\pgfqpoint{5.772625in}{7.279038in}}%
\pgfpathcurveto{\pgfqpoint{5.764811in}{7.286852in}}{\pgfqpoint{5.754212in}{7.291242in}}{\pgfqpoint{5.743162in}{7.291242in}}%
\pgfpathcurveto{\pgfqpoint{5.732112in}{7.291242in}}{\pgfqpoint{5.721513in}{7.286852in}}{\pgfqpoint{5.713699in}{7.279038in}}%
\pgfpathcurveto{\pgfqpoint{5.705886in}{7.271225in}}{\pgfqpoint{5.701495in}{7.260626in}}{\pgfqpoint{5.701495in}{7.249576in}}%
\pgfpathcurveto{\pgfqpoint{5.701495in}{7.238525in}}{\pgfqpoint{5.705886in}{7.227926in}}{\pgfqpoint{5.713699in}{7.220113in}}%
\pgfpathcurveto{\pgfqpoint{5.721513in}{7.212299in}}{\pgfqpoint{5.732112in}{7.207909in}}{\pgfqpoint{5.743162in}{7.207909in}}%
\pgfpathclose%
\pgfusepath{stroke,fill}%
\end{pgfscope}%
\begin{pgfscope}%
\pgfpathrectangle{\pgfqpoint{0.481978in}{0.331635in}}{\pgfqpoint{9.300000in}{7.700000in}}%
\pgfusepath{clip}%
\pgfsetbuttcap%
\pgfsetroundjoin%
\definecolor{currentfill}{rgb}{0.631373,0.788235,0.956863}%
\pgfsetfillcolor{currentfill}%
\pgfsetlinewidth{0.481800pt}%
\definecolor{currentstroke}{rgb}{1.000000,1.000000,1.000000}%
\pgfsetstrokecolor{currentstroke}%
\pgfsetdash{}{0pt}%
\pgfpathmoveto{\pgfqpoint{3.731632in}{6.321424in}}%
\pgfpathcurveto{\pgfqpoint{3.742682in}{6.321424in}}{\pgfqpoint{3.753281in}{6.325815in}}{\pgfqpoint{3.761095in}{6.333628in}}%
\pgfpathcurveto{\pgfqpoint{3.768908in}{6.341442in}}{\pgfqpoint{3.773299in}{6.352041in}}{\pgfqpoint{3.773299in}{6.363091in}}%
\pgfpathcurveto{\pgfqpoint{3.773299in}{6.374141in}}{\pgfqpoint{3.768908in}{6.384740in}}{\pgfqpoint{3.761095in}{6.392554in}}%
\pgfpathcurveto{\pgfqpoint{3.753281in}{6.400367in}}{\pgfqpoint{3.742682in}{6.404758in}}{\pgfqpoint{3.731632in}{6.404758in}}%
\pgfpathcurveto{\pgfqpoint{3.720582in}{6.404758in}}{\pgfqpoint{3.709983in}{6.400367in}}{\pgfqpoint{3.702169in}{6.392554in}}%
\pgfpathcurveto{\pgfqpoint{3.694356in}{6.384740in}}{\pgfqpoint{3.689965in}{6.374141in}}{\pgfqpoint{3.689965in}{6.363091in}}%
\pgfpathcurveto{\pgfqpoint{3.689965in}{6.352041in}}{\pgfqpoint{3.694356in}{6.341442in}}{\pgfqpoint{3.702169in}{6.333628in}}%
\pgfpathcurveto{\pgfqpoint{3.709983in}{6.325815in}}{\pgfqpoint{3.720582in}{6.321424in}}{\pgfqpoint{3.731632in}{6.321424in}}%
\pgfpathclose%
\pgfusepath{stroke,fill}%
\end{pgfscope}%
\begin{pgfscope}%
\pgfpathrectangle{\pgfqpoint{0.481978in}{0.331635in}}{\pgfqpoint{9.300000in}{7.700000in}}%
\pgfusepath{clip}%
\pgfsetbuttcap%
\pgfsetroundjoin%
\definecolor{currentfill}{rgb}{0.631373,0.788235,0.956863}%
\pgfsetfillcolor{currentfill}%
\pgfsetlinewidth{0.481800pt}%
\definecolor{currentstroke}{rgb}{1.000000,1.000000,1.000000}%
\pgfsetstrokecolor{currentstroke}%
\pgfsetdash{}{0pt}%
\pgfpathmoveto{\pgfqpoint{3.354108in}{0.639968in}}%
\pgfpathcurveto{\pgfqpoint{3.365158in}{0.639968in}}{\pgfqpoint{3.375757in}{0.644359in}}{\pgfqpoint{3.383571in}{0.652172in}}%
\pgfpathcurveto{\pgfqpoint{3.391384in}{0.659986in}}{\pgfqpoint{3.395775in}{0.670585in}}{\pgfqpoint{3.395775in}{0.681635in}}%
\pgfpathcurveto{\pgfqpoint{3.395775in}{0.692685in}}{\pgfqpoint{3.391384in}{0.703284in}}{\pgfqpoint{3.383571in}{0.711098in}}%
\pgfpathcurveto{\pgfqpoint{3.375757in}{0.718911in}}{\pgfqpoint{3.365158in}{0.723302in}}{\pgfqpoint{3.354108in}{0.723302in}}%
\pgfpathcurveto{\pgfqpoint{3.343058in}{0.723302in}}{\pgfqpoint{3.332459in}{0.718911in}}{\pgfqpoint{3.324645in}{0.711098in}}%
\pgfpathcurveto{\pgfqpoint{3.316832in}{0.703284in}}{\pgfqpoint{3.312441in}{0.692685in}}{\pgfqpoint{3.312441in}{0.681635in}}%
\pgfpathcurveto{\pgfqpoint{3.312441in}{0.670585in}}{\pgfqpoint{3.316832in}{0.659986in}}{\pgfqpoint{3.324645in}{0.652172in}}%
\pgfpathcurveto{\pgfqpoint{3.332459in}{0.644359in}}{\pgfqpoint{3.343058in}{0.639968in}}{\pgfqpoint{3.354108in}{0.639968in}}%
\pgfpathclose%
\pgfusepath{stroke,fill}%
\end{pgfscope}%
\begin{pgfscope}%
\pgfpathrectangle{\pgfqpoint{0.481978in}{0.331635in}}{\pgfqpoint{9.300000in}{7.700000in}}%
\pgfusepath{clip}%
\pgfsetbuttcap%
\pgfsetroundjoin%
\definecolor{currentfill}{rgb}{0.631373,0.788235,0.956863}%
\pgfsetfillcolor{currentfill}%
\pgfsetlinewidth{0.481800pt}%
\definecolor{currentstroke}{rgb}{1.000000,1.000000,1.000000}%
\pgfsetstrokecolor{currentstroke}%
\pgfsetdash{}{0pt}%
\pgfpathmoveto{\pgfqpoint{2.507727in}{5.202892in}}%
\pgfpathcurveto{\pgfqpoint{2.518777in}{5.202892in}}{\pgfqpoint{2.529376in}{5.207282in}}{\pgfqpoint{2.537190in}{5.215096in}}%
\pgfpathcurveto{\pgfqpoint{2.545003in}{5.222910in}}{\pgfqpoint{2.549394in}{5.233509in}}{\pgfqpoint{2.549394in}{5.244559in}}%
\pgfpathcurveto{\pgfqpoint{2.549394in}{5.255609in}}{\pgfqpoint{2.545003in}{5.266208in}}{\pgfqpoint{2.537190in}{5.274022in}}%
\pgfpathcurveto{\pgfqpoint{2.529376in}{5.281835in}}{\pgfqpoint{2.518777in}{5.286225in}}{\pgfqpoint{2.507727in}{5.286225in}}%
\pgfpathcurveto{\pgfqpoint{2.496677in}{5.286225in}}{\pgfqpoint{2.486078in}{5.281835in}}{\pgfqpoint{2.478264in}{5.274022in}}%
\pgfpathcurveto{\pgfqpoint{2.470451in}{5.266208in}}{\pgfqpoint{2.466060in}{5.255609in}}{\pgfqpoint{2.466060in}{5.244559in}}%
\pgfpathcurveto{\pgfqpoint{2.466060in}{5.233509in}}{\pgfqpoint{2.470451in}{5.222910in}}{\pgfqpoint{2.478264in}{5.215096in}}%
\pgfpathcurveto{\pgfqpoint{2.486078in}{5.207282in}}{\pgfqpoint{2.496677in}{5.202892in}}{\pgfqpoint{2.507727in}{5.202892in}}%
\pgfpathclose%
\pgfusepath{stroke,fill}%
\end{pgfscope}%
\begin{pgfscope}%
\pgfpathrectangle{\pgfqpoint{0.481978in}{0.331635in}}{\pgfqpoint{9.300000in}{7.700000in}}%
\pgfusepath{clip}%
\pgfsetbuttcap%
\pgfsetroundjoin%
\definecolor{currentfill}{rgb}{0.631373,0.788235,0.956863}%
\pgfsetfillcolor{currentfill}%
\pgfsetlinewidth{0.481800pt}%
\definecolor{currentstroke}{rgb}{1.000000,1.000000,1.000000}%
\pgfsetstrokecolor{currentstroke}%
\pgfsetdash{}{0pt}%
\pgfpathmoveto{\pgfqpoint{3.345563in}{4.675887in}}%
\pgfpathcurveto{\pgfqpoint{3.356613in}{4.675887in}}{\pgfqpoint{3.367212in}{4.680278in}}{\pgfqpoint{3.375026in}{4.688091in}}%
\pgfpathcurveto{\pgfqpoint{3.382839in}{4.695905in}}{\pgfqpoint{3.387230in}{4.706504in}}{\pgfqpoint{3.387230in}{4.717554in}}%
\pgfpathcurveto{\pgfqpoint{3.387230in}{4.728604in}}{\pgfqpoint{3.382839in}{4.739203in}}{\pgfqpoint{3.375026in}{4.747017in}}%
\pgfpathcurveto{\pgfqpoint{3.367212in}{4.754830in}}{\pgfqpoint{3.356613in}{4.759221in}}{\pgfqpoint{3.345563in}{4.759221in}}%
\pgfpathcurveto{\pgfqpoint{3.334513in}{4.759221in}}{\pgfqpoint{3.323914in}{4.754830in}}{\pgfqpoint{3.316100in}{4.747017in}}%
\pgfpathcurveto{\pgfqpoint{3.308287in}{4.739203in}}{\pgfqpoint{3.303896in}{4.728604in}}{\pgfqpoint{3.303896in}{4.717554in}}%
\pgfpathcurveto{\pgfqpoint{3.303896in}{4.706504in}}{\pgfqpoint{3.308287in}{4.695905in}}{\pgfqpoint{3.316100in}{4.688091in}}%
\pgfpathcurveto{\pgfqpoint{3.323914in}{4.680278in}}{\pgfqpoint{3.334513in}{4.675887in}}{\pgfqpoint{3.345563in}{4.675887in}}%
\pgfpathclose%
\pgfusepath{stroke,fill}%
\end{pgfscope}%
\begin{pgfscope}%
\pgfpathrectangle{\pgfqpoint{0.481978in}{0.331635in}}{\pgfqpoint{9.300000in}{7.700000in}}%
\pgfusepath{clip}%
\pgfsetbuttcap%
\pgfsetroundjoin%
\definecolor{currentfill}{rgb}{0.631373,0.788235,0.956863}%
\pgfsetfillcolor{currentfill}%
\pgfsetlinewidth{0.481800pt}%
\definecolor{currentstroke}{rgb}{1.000000,1.000000,1.000000}%
\pgfsetstrokecolor{currentstroke}%
\pgfsetdash{}{0pt}%
\pgfpathmoveto{\pgfqpoint{5.829195in}{5.212750in}}%
\pgfpathcurveto{\pgfqpoint{5.840246in}{5.212750in}}{\pgfqpoint{5.850845in}{5.217140in}}{\pgfqpoint{5.858658in}{5.224954in}}%
\pgfpathcurveto{\pgfqpoint{5.866472in}{5.232767in}}{\pgfqpoint{5.870862in}{5.243367in}}{\pgfqpoint{5.870862in}{5.254417in}}%
\pgfpathcurveto{\pgfqpoint{5.870862in}{5.265467in}}{\pgfqpoint{5.866472in}{5.276066in}}{\pgfqpoint{5.858658in}{5.283879in}}%
\pgfpathcurveto{\pgfqpoint{5.850845in}{5.291693in}}{\pgfqpoint{5.840246in}{5.296083in}}{\pgfqpoint{5.829195in}{5.296083in}}%
\pgfpathcurveto{\pgfqpoint{5.818145in}{5.296083in}}{\pgfqpoint{5.807546in}{5.291693in}}{\pgfqpoint{5.799733in}{5.283879in}}%
\pgfpathcurveto{\pgfqpoint{5.791919in}{5.276066in}}{\pgfqpoint{5.787529in}{5.265467in}}{\pgfqpoint{5.787529in}{5.254417in}}%
\pgfpathcurveto{\pgfqpoint{5.787529in}{5.243367in}}{\pgfqpoint{5.791919in}{5.232767in}}{\pgfqpoint{5.799733in}{5.224954in}}%
\pgfpathcurveto{\pgfqpoint{5.807546in}{5.217140in}}{\pgfqpoint{5.818145in}{5.212750in}}{\pgfqpoint{5.829195in}{5.212750in}}%
\pgfpathclose%
\pgfusepath{stroke,fill}%
\end{pgfscope}%
\begin{pgfscope}%
\pgfpathrectangle{\pgfqpoint{0.481978in}{0.331635in}}{\pgfqpoint{9.300000in}{7.700000in}}%
\pgfusepath{clip}%
\pgfsetbuttcap%
\pgfsetroundjoin%
\definecolor{currentfill}{rgb}{0.631373,0.788235,0.956863}%
\pgfsetfillcolor{currentfill}%
\pgfsetlinewidth{0.481800pt}%
\definecolor{currentstroke}{rgb}{1.000000,1.000000,1.000000}%
\pgfsetstrokecolor{currentstroke}%
\pgfsetdash{}{0pt}%
\pgfpathmoveto{\pgfqpoint{1.266407in}{2.502917in}}%
\pgfpathcurveto{\pgfqpoint{1.277457in}{2.502917in}}{\pgfqpoint{1.288056in}{2.507308in}}{\pgfqpoint{1.295870in}{2.515121in}}%
\pgfpathcurveto{\pgfqpoint{1.303683in}{2.522935in}}{\pgfqpoint{1.308074in}{2.533534in}}{\pgfqpoint{1.308074in}{2.544584in}}%
\pgfpathcurveto{\pgfqpoint{1.308074in}{2.555634in}}{\pgfqpoint{1.303683in}{2.566233in}}{\pgfqpoint{1.295870in}{2.574047in}}%
\pgfpathcurveto{\pgfqpoint{1.288056in}{2.581861in}}{\pgfqpoint{1.277457in}{2.586251in}}{\pgfqpoint{1.266407in}{2.586251in}}%
\pgfpathcurveto{\pgfqpoint{1.255357in}{2.586251in}}{\pgfqpoint{1.244758in}{2.581861in}}{\pgfqpoint{1.236944in}{2.574047in}}%
\pgfpathcurveto{\pgfqpoint{1.229131in}{2.566233in}}{\pgfqpoint{1.224740in}{2.555634in}}{\pgfqpoint{1.224740in}{2.544584in}}%
\pgfpathcurveto{\pgfqpoint{1.224740in}{2.533534in}}{\pgfqpoint{1.229131in}{2.522935in}}{\pgfqpoint{1.236944in}{2.515121in}}%
\pgfpathcurveto{\pgfqpoint{1.244758in}{2.507308in}}{\pgfqpoint{1.255357in}{2.502917in}}{\pgfqpoint{1.266407in}{2.502917in}}%
\pgfpathclose%
\pgfusepath{stroke,fill}%
\end{pgfscope}%
\begin{pgfscope}%
\pgfpathrectangle{\pgfqpoint{0.481978in}{0.331635in}}{\pgfqpoint{9.300000in}{7.700000in}}%
\pgfusepath{clip}%
\pgfsetbuttcap%
\pgfsetroundjoin%
\definecolor{currentfill}{rgb}{0.631373,0.788235,0.956863}%
\pgfsetfillcolor{currentfill}%
\pgfsetlinewidth{0.481800pt}%
\definecolor{currentstroke}{rgb}{1.000000,1.000000,1.000000}%
\pgfsetstrokecolor{currentstroke}%
\pgfsetdash{}{0pt}%
\pgfpathmoveto{\pgfqpoint{1.280257in}{3.702244in}}%
\pgfpathcurveto{\pgfqpoint{1.291307in}{3.702244in}}{\pgfqpoint{1.301906in}{3.706634in}}{\pgfqpoint{1.309720in}{3.714448in}}%
\pgfpathcurveto{\pgfqpoint{1.317533in}{3.722261in}}{\pgfqpoint{1.321923in}{3.732860in}}{\pgfqpoint{1.321923in}{3.743911in}}%
\pgfpathcurveto{\pgfqpoint{1.321923in}{3.754961in}}{\pgfqpoint{1.317533in}{3.765560in}}{\pgfqpoint{1.309720in}{3.773373in}}%
\pgfpathcurveto{\pgfqpoint{1.301906in}{3.781187in}}{\pgfqpoint{1.291307in}{3.785577in}}{\pgfqpoint{1.280257in}{3.785577in}}%
\pgfpathcurveto{\pgfqpoint{1.269207in}{3.785577in}}{\pgfqpoint{1.258608in}{3.781187in}}{\pgfqpoint{1.250794in}{3.773373in}}%
\pgfpathcurveto{\pgfqpoint{1.242980in}{3.765560in}}{\pgfqpoint{1.238590in}{3.754961in}}{\pgfqpoint{1.238590in}{3.743911in}}%
\pgfpathcurveto{\pgfqpoint{1.238590in}{3.732860in}}{\pgfqpoint{1.242980in}{3.722261in}}{\pgfqpoint{1.250794in}{3.714448in}}%
\pgfpathcurveto{\pgfqpoint{1.258608in}{3.706634in}}{\pgfqpoint{1.269207in}{3.702244in}}{\pgfqpoint{1.280257in}{3.702244in}}%
\pgfpathclose%
\pgfusepath{stroke,fill}%
\end{pgfscope}%
\begin{pgfscope}%
\pgfpathrectangle{\pgfqpoint{0.481978in}{0.331635in}}{\pgfqpoint{9.300000in}{7.700000in}}%
\pgfusepath{clip}%
\pgfsetbuttcap%
\pgfsetroundjoin%
\definecolor{currentfill}{rgb}{0.631373,0.788235,0.956863}%
\pgfsetfillcolor{currentfill}%
\pgfsetlinewidth{0.481800pt}%
\definecolor{currentstroke}{rgb}{1.000000,1.000000,1.000000}%
\pgfsetstrokecolor{currentstroke}%
\pgfsetdash{}{0pt}%
\pgfpathmoveto{\pgfqpoint{1.366790in}{2.436232in}}%
\pgfpathcurveto{\pgfqpoint{1.377840in}{2.436232in}}{\pgfqpoint{1.388439in}{2.440622in}}{\pgfqpoint{1.396252in}{2.448436in}}%
\pgfpathcurveto{\pgfqpoint{1.404066in}{2.456249in}}{\pgfqpoint{1.408456in}{2.466848in}}{\pgfqpoint{1.408456in}{2.477899in}}%
\pgfpathcurveto{\pgfqpoint{1.408456in}{2.488949in}}{\pgfqpoint{1.404066in}{2.499548in}}{\pgfqpoint{1.396252in}{2.507361in}}%
\pgfpathcurveto{\pgfqpoint{1.388439in}{2.515175in}}{\pgfqpoint{1.377840in}{2.519565in}}{\pgfqpoint{1.366790in}{2.519565in}}%
\pgfpathcurveto{\pgfqpoint{1.355740in}{2.519565in}}{\pgfqpoint{1.345141in}{2.515175in}}{\pgfqpoint{1.337327in}{2.507361in}}%
\pgfpathcurveto{\pgfqpoint{1.329513in}{2.499548in}}{\pgfqpoint{1.325123in}{2.488949in}}{\pgfqpoint{1.325123in}{2.477899in}}%
\pgfpathcurveto{\pgfqpoint{1.325123in}{2.466848in}}{\pgfqpoint{1.329513in}{2.456249in}}{\pgfqpoint{1.337327in}{2.448436in}}%
\pgfpathcurveto{\pgfqpoint{1.345141in}{2.440622in}}{\pgfqpoint{1.355740in}{2.436232in}}{\pgfqpoint{1.366790in}{2.436232in}}%
\pgfpathclose%
\pgfusepath{stroke,fill}%
\end{pgfscope}%
\begin{pgfscope}%
\pgfpathrectangle{\pgfqpoint{0.481978in}{0.331635in}}{\pgfqpoint{9.300000in}{7.700000in}}%
\pgfusepath{clip}%
\pgfsetbuttcap%
\pgfsetroundjoin%
\definecolor{currentfill}{rgb}{0.631373,0.788235,0.956863}%
\pgfsetfillcolor{currentfill}%
\pgfsetlinewidth{0.481800pt}%
\definecolor{currentstroke}{rgb}{1.000000,1.000000,1.000000}%
\pgfsetstrokecolor{currentstroke}%
\pgfsetdash{}{0pt}%
\pgfpathmoveto{\pgfqpoint{5.131403in}{3.775556in}}%
\pgfpathcurveto{\pgfqpoint{5.142453in}{3.775556in}}{\pgfqpoint{5.153052in}{3.779946in}}{\pgfqpoint{5.160866in}{3.787760in}}%
\pgfpathcurveto{\pgfqpoint{5.168679in}{3.795573in}}{\pgfqpoint{5.173070in}{3.806172in}}{\pgfqpoint{5.173070in}{3.817222in}}%
\pgfpathcurveto{\pgfqpoint{5.173070in}{3.828273in}}{\pgfqpoint{5.168679in}{3.838872in}}{\pgfqpoint{5.160866in}{3.846685in}}%
\pgfpathcurveto{\pgfqpoint{5.153052in}{3.854499in}}{\pgfqpoint{5.142453in}{3.858889in}}{\pgfqpoint{5.131403in}{3.858889in}}%
\pgfpathcurveto{\pgfqpoint{5.120353in}{3.858889in}}{\pgfqpoint{5.109754in}{3.854499in}}{\pgfqpoint{5.101940in}{3.846685in}}%
\pgfpathcurveto{\pgfqpoint{5.094127in}{3.838872in}}{\pgfqpoint{5.089736in}{3.828273in}}{\pgfqpoint{5.089736in}{3.817222in}}%
\pgfpathcurveto{\pgfqpoint{5.089736in}{3.806172in}}{\pgfqpoint{5.094127in}{3.795573in}}{\pgfqpoint{5.101940in}{3.787760in}}%
\pgfpathcurveto{\pgfqpoint{5.109754in}{3.779946in}}{\pgfqpoint{5.120353in}{3.775556in}}{\pgfqpoint{5.131403in}{3.775556in}}%
\pgfpathclose%
\pgfusepath{stroke,fill}%
\end{pgfscope}%
\begin{pgfscope}%
\pgfpathrectangle{\pgfqpoint{0.481978in}{0.331635in}}{\pgfqpoint{9.300000in}{7.700000in}}%
\pgfusepath{clip}%
\pgfsetbuttcap%
\pgfsetroundjoin%
\definecolor{currentfill}{rgb}{0.631373,0.788235,0.956863}%
\pgfsetfillcolor{currentfill}%
\pgfsetlinewidth{0.481800pt}%
\definecolor{currentstroke}{rgb}{1.000000,1.000000,1.000000}%
\pgfsetstrokecolor{currentstroke}%
\pgfsetdash{}{0pt}%
\pgfpathmoveto{\pgfqpoint{6.077197in}{6.974685in}}%
\pgfpathcurveto{\pgfqpoint{6.088248in}{6.974685in}}{\pgfqpoint{6.098847in}{6.979075in}}{\pgfqpoint{6.106660in}{6.986889in}}%
\pgfpathcurveto{\pgfqpoint{6.114474in}{6.994702in}}{\pgfqpoint{6.118864in}{7.005301in}}{\pgfqpoint{6.118864in}{7.016351in}}%
\pgfpathcurveto{\pgfqpoint{6.118864in}{7.027402in}}{\pgfqpoint{6.114474in}{7.038001in}}{\pgfqpoint{6.106660in}{7.045814in}}%
\pgfpathcurveto{\pgfqpoint{6.098847in}{7.053628in}}{\pgfqpoint{6.088248in}{7.058018in}}{\pgfqpoint{6.077197in}{7.058018in}}%
\pgfpathcurveto{\pgfqpoint{6.066147in}{7.058018in}}{\pgfqpoint{6.055548in}{7.053628in}}{\pgfqpoint{6.047735in}{7.045814in}}%
\pgfpathcurveto{\pgfqpoint{6.039921in}{7.038001in}}{\pgfqpoint{6.035531in}{7.027402in}}{\pgfqpoint{6.035531in}{7.016351in}}%
\pgfpathcurveto{\pgfqpoint{6.035531in}{7.005301in}}{\pgfqpoint{6.039921in}{6.994702in}}{\pgfqpoint{6.047735in}{6.986889in}}%
\pgfpathcurveto{\pgfqpoint{6.055548in}{6.979075in}}{\pgfqpoint{6.066147in}{6.974685in}}{\pgfqpoint{6.077197in}{6.974685in}}%
\pgfpathclose%
\pgfusepath{stroke,fill}%
\end{pgfscope}%
\begin{pgfscope}%
\pgfpathrectangle{\pgfqpoint{0.481978in}{0.331635in}}{\pgfqpoint{9.300000in}{7.700000in}}%
\pgfusepath{clip}%
\pgfsetbuttcap%
\pgfsetroundjoin%
\definecolor{currentfill}{rgb}{0.631373,0.788235,0.956863}%
\pgfsetfillcolor{currentfill}%
\pgfsetlinewidth{0.481800pt}%
\definecolor{currentstroke}{rgb}{1.000000,1.000000,1.000000}%
\pgfsetstrokecolor{currentstroke}%
\pgfsetdash{}{0pt}%
\pgfpathmoveto{\pgfqpoint{5.267507in}{3.437471in}}%
\pgfpathcurveto{\pgfqpoint{5.278557in}{3.437471in}}{\pgfqpoint{5.289156in}{3.441861in}}{\pgfqpoint{5.296970in}{3.449674in}}%
\pgfpathcurveto{\pgfqpoint{5.304783in}{3.457488in}}{\pgfqpoint{5.309173in}{3.468087in}}{\pgfqpoint{5.309173in}{3.479137in}}%
\pgfpathcurveto{\pgfqpoint{5.309173in}{3.490187in}}{\pgfqpoint{5.304783in}{3.500786in}}{\pgfqpoint{5.296970in}{3.508600in}}%
\pgfpathcurveto{\pgfqpoint{5.289156in}{3.516414in}}{\pgfqpoint{5.278557in}{3.520804in}}{\pgfqpoint{5.267507in}{3.520804in}}%
\pgfpathcurveto{\pgfqpoint{5.256457in}{3.520804in}}{\pgfqpoint{5.245858in}{3.516414in}}{\pgfqpoint{5.238044in}{3.508600in}}%
\pgfpathcurveto{\pgfqpoint{5.230230in}{3.500786in}}{\pgfqpoint{5.225840in}{3.490187in}}{\pgfqpoint{5.225840in}{3.479137in}}%
\pgfpathcurveto{\pgfqpoint{5.225840in}{3.468087in}}{\pgfqpoint{5.230230in}{3.457488in}}{\pgfqpoint{5.238044in}{3.449674in}}%
\pgfpathcurveto{\pgfqpoint{5.245858in}{3.441861in}}{\pgfqpoint{5.256457in}{3.437471in}}{\pgfqpoint{5.267507in}{3.437471in}}%
\pgfpathclose%
\pgfusepath{stroke,fill}%
\end{pgfscope}%
\begin{pgfscope}%
\pgfpathrectangle{\pgfqpoint{0.481978in}{0.331635in}}{\pgfqpoint{9.300000in}{7.700000in}}%
\pgfusepath{clip}%
\pgfsetbuttcap%
\pgfsetroundjoin%
\definecolor{currentfill}{rgb}{0.631373,0.788235,0.956863}%
\pgfsetfillcolor{currentfill}%
\pgfsetlinewidth{0.481800pt}%
\definecolor{currentstroke}{rgb}{1.000000,1.000000,1.000000}%
\pgfsetstrokecolor{currentstroke}%
\pgfsetdash{}{0pt}%
\pgfpathmoveto{\pgfqpoint{5.422432in}{4.371502in}}%
\pgfpathcurveto{\pgfqpoint{5.433483in}{4.371502in}}{\pgfqpoint{5.444082in}{4.375892in}}{\pgfqpoint{5.451895in}{4.383705in}}%
\pgfpathcurveto{\pgfqpoint{5.459709in}{4.391519in}}{\pgfqpoint{5.464099in}{4.402118in}}{\pgfqpoint{5.464099in}{4.413168in}}%
\pgfpathcurveto{\pgfqpoint{5.464099in}{4.424218in}}{\pgfqpoint{5.459709in}{4.434817in}}{\pgfqpoint{5.451895in}{4.442631in}}%
\pgfpathcurveto{\pgfqpoint{5.444082in}{4.450445in}}{\pgfqpoint{5.433483in}{4.454835in}}{\pgfqpoint{5.422432in}{4.454835in}}%
\pgfpathcurveto{\pgfqpoint{5.411382in}{4.454835in}}{\pgfqpoint{5.400783in}{4.450445in}}{\pgfqpoint{5.392970in}{4.442631in}}%
\pgfpathcurveto{\pgfqpoint{5.385156in}{4.434817in}}{\pgfqpoint{5.380766in}{4.424218in}}{\pgfqpoint{5.380766in}{4.413168in}}%
\pgfpathcurveto{\pgfqpoint{5.380766in}{4.402118in}}{\pgfqpoint{5.385156in}{4.391519in}}{\pgfqpoint{5.392970in}{4.383705in}}%
\pgfpathcurveto{\pgfqpoint{5.400783in}{4.375892in}}{\pgfqpoint{5.411382in}{4.371502in}}{\pgfqpoint{5.422432in}{4.371502in}}%
\pgfpathclose%
\pgfusepath{stroke,fill}%
\end{pgfscope}%
\begin{pgfscope}%
\pgfpathrectangle{\pgfqpoint{0.481978in}{0.331635in}}{\pgfqpoint{9.300000in}{7.700000in}}%
\pgfusepath{clip}%
\pgfsetbuttcap%
\pgfsetroundjoin%
\definecolor{currentfill}{rgb}{0.631373,0.788235,0.956863}%
\pgfsetfillcolor{currentfill}%
\pgfsetlinewidth{0.481800pt}%
\definecolor{currentstroke}{rgb}{1.000000,1.000000,1.000000}%
\pgfsetstrokecolor{currentstroke}%
\pgfsetdash{}{0pt}%
\pgfpathmoveto{\pgfqpoint{4.576144in}{3.852406in}}%
\pgfpathcurveto{\pgfqpoint{4.587194in}{3.852406in}}{\pgfqpoint{4.597793in}{3.856796in}}{\pgfqpoint{4.605606in}{3.864609in}}%
\pgfpathcurveto{\pgfqpoint{4.613420in}{3.872423in}}{\pgfqpoint{4.617810in}{3.883022in}}{\pgfqpoint{4.617810in}{3.894072in}}%
\pgfpathcurveto{\pgfqpoint{4.617810in}{3.905122in}}{\pgfqpoint{4.613420in}{3.915721in}}{\pgfqpoint{4.605606in}{3.923535in}}%
\pgfpathcurveto{\pgfqpoint{4.597793in}{3.931349in}}{\pgfqpoint{4.587194in}{3.935739in}}{\pgfqpoint{4.576144in}{3.935739in}}%
\pgfpathcurveto{\pgfqpoint{4.565093in}{3.935739in}}{\pgfqpoint{4.554494in}{3.931349in}}{\pgfqpoint{4.546681in}{3.923535in}}%
\pgfpathcurveto{\pgfqpoint{4.538867in}{3.915721in}}{\pgfqpoint{4.534477in}{3.905122in}}{\pgfqpoint{4.534477in}{3.894072in}}%
\pgfpathcurveto{\pgfqpoint{4.534477in}{3.883022in}}{\pgfqpoint{4.538867in}{3.872423in}}{\pgfqpoint{4.546681in}{3.864609in}}%
\pgfpathcurveto{\pgfqpoint{4.554494in}{3.856796in}}{\pgfqpoint{4.565093in}{3.852406in}}{\pgfqpoint{4.576144in}{3.852406in}}%
\pgfpathclose%
\pgfusepath{stroke,fill}%
\end{pgfscope}%
\begin{pgfscope}%
\pgfpathrectangle{\pgfqpoint{0.481978in}{0.331635in}}{\pgfqpoint{9.300000in}{7.700000in}}%
\pgfusepath{clip}%
\pgfsetbuttcap%
\pgfsetroundjoin%
\definecolor{currentfill}{rgb}{0.631373,0.788235,0.956863}%
\pgfsetfillcolor{currentfill}%
\pgfsetlinewidth{0.481800pt}%
\definecolor{currentstroke}{rgb}{1.000000,1.000000,1.000000}%
\pgfsetstrokecolor{currentstroke}%
\pgfsetdash{}{0pt}%
\pgfpathmoveto{\pgfqpoint{2.109185in}{3.342395in}}%
\pgfpathcurveto{\pgfqpoint{2.120235in}{3.342395in}}{\pgfqpoint{2.130834in}{3.346786in}}{\pgfqpoint{2.138648in}{3.354599in}}%
\pgfpathcurveto{\pgfqpoint{2.146461in}{3.362413in}}{\pgfqpoint{2.150852in}{3.373012in}}{\pgfqpoint{2.150852in}{3.384062in}}%
\pgfpathcurveto{\pgfqpoint{2.150852in}{3.395112in}}{\pgfqpoint{2.146461in}{3.405711in}}{\pgfqpoint{2.138648in}{3.413525in}}%
\pgfpathcurveto{\pgfqpoint{2.130834in}{3.421338in}}{\pgfqpoint{2.120235in}{3.425729in}}{\pgfqpoint{2.109185in}{3.425729in}}%
\pgfpathcurveto{\pgfqpoint{2.098135in}{3.425729in}}{\pgfqpoint{2.087536in}{3.421338in}}{\pgfqpoint{2.079722in}{3.413525in}}%
\pgfpathcurveto{\pgfqpoint{2.071908in}{3.405711in}}{\pgfqpoint{2.067518in}{3.395112in}}{\pgfqpoint{2.067518in}{3.384062in}}%
\pgfpathcurveto{\pgfqpoint{2.067518in}{3.373012in}}{\pgfqpoint{2.071908in}{3.362413in}}{\pgfqpoint{2.079722in}{3.354599in}}%
\pgfpathcurveto{\pgfqpoint{2.087536in}{3.346786in}}{\pgfqpoint{2.098135in}{3.342395in}}{\pgfqpoint{2.109185in}{3.342395in}}%
\pgfpathclose%
\pgfusepath{stroke,fill}%
\end{pgfscope}%
\begin{pgfscope}%
\pgfpathrectangle{\pgfqpoint{0.481978in}{0.331635in}}{\pgfqpoint{9.300000in}{7.700000in}}%
\pgfusepath{clip}%
\pgfsetbuttcap%
\pgfsetroundjoin%
\definecolor{currentfill}{rgb}{0.631373,0.788235,0.956863}%
\pgfsetfillcolor{currentfill}%
\pgfsetlinewidth{0.481800pt}%
\definecolor{currentstroke}{rgb}{1.000000,1.000000,1.000000}%
\pgfsetstrokecolor{currentstroke}%
\pgfsetdash{}{0pt}%
\pgfpathmoveto{\pgfqpoint{3.045397in}{4.041120in}}%
\pgfpathcurveto{\pgfqpoint{3.056447in}{4.041120in}}{\pgfqpoint{3.067046in}{4.045510in}}{\pgfqpoint{3.074860in}{4.053324in}}%
\pgfpathcurveto{\pgfqpoint{3.082673in}{4.061137in}}{\pgfqpoint{3.087063in}{4.071736in}}{\pgfqpoint{3.087063in}{4.082787in}}%
\pgfpathcurveto{\pgfqpoint{3.087063in}{4.093837in}}{\pgfqpoint{3.082673in}{4.104436in}}{\pgfqpoint{3.074860in}{4.112249in}}%
\pgfpathcurveto{\pgfqpoint{3.067046in}{4.120063in}}{\pgfqpoint{3.056447in}{4.124453in}}{\pgfqpoint{3.045397in}{4.124453in}}%
\pgfpathcurveto{\pgfqpoint{3.034347in}{4.124453in}}{\pgfqpoint{3.023748in}{4.120063in}}{\pgfqpoint{3.015934in}{4.112249in}}%
\pgfpathcurveto{\pgfqpoint{3.008120in}{4.104436in}}{\pgfqpoint{3.003730in}{4.093837in}}{\pgfqpoint{3.003730in}{4.082787in}}%
\pgfpathcurveto{\pgfqpoint{3.003730in}{4.071736in}}{\pgfqpoint{3.008120in}{4.061137in}}{\pgfqpoint{3.015934in}{4.053324in}}%
\pgfpathcurveto{\pgfqpoint{3.023748in}{4.045510in}}{\pgfqpoint{3.034347in}{4.041120in}}{\pgfqpoint{3.045397in}{4.041120in}}%
\pgfpathclose%
\pgfusepath{stroke,fill}%
\end{pgfscope}%
\begin{pgfscope}%
\pgfpathrectangle{\pgfqpoint{0.481978in}{0.331635in}}{\pgfqpoint{9.300000in}{7.700000in}}%
\pgfusepath{clip}%
\pgfsetbuttcap%
\pgfsetroundjoin%
\definecolor{currentfill}{rgb}{0.631373,0.788235,0.956863}%
\pgfsetfillcolor{currentfill}%
\pgfsetlinewidth{0.481800pt}%
\definecolor{currentstroke}{rgb}{1.000000,1.000000,1.000000}%
\pgfsetstrokecolor{currentstroke}%
\pgfsetdash{}{0pt}%
\pgfpathmoveto{\pgfqpoint{2.937522in}{5.069906in}}%
\pgfpathcurveto{\pgfqpoint{2.948572in}{5.069906in}}{\pgfqpoint{2.959171in}{5.074297in}}{\pgfqpoint{2.966985in}{5.082110in}}%
\pgfpathcurveto{\pgfqpoint{2.974799in}{5.089924in}}{\pgfqpoint{2.979189in}{5.100523in}}{\pgfqpoint{2.979189in}{5.111573in}}%
\pgfpathcurveto{\pgfqpoint{2.979189in}{5.122623in}}{\pgfqpoint{2.974799in}{5.133222in}}{\pgfqpoint{2.966985in}{5.141036in}}%
\pgfpathcurveto{\pgfqpoint{2.959171in}{5.148849in}}{\pgfqpoint{2.948572in}{5.153240in}}{\pgfqpoint{2.937522in}{5.153240in}}%
\pgfpathcurveto{\pgfqpoint{2.926472in}{5.153240in}}{\pgfqpoint{2.915873in}{5.148849in}}{\pgfqpoint{2.908059in}{5.141036in}}%
\pgfpathcurveto{\pgfqpoint{2.900246in}{5.133222in}}{\pgfqpoint{2.895856in}{5.122623in}}{\pgfqpoint{2.895856in}{5.111573in}}%
\pgfpathcurveto{\pgfqpoint{2.895856in}{5.100523in}}{\pgfqpoint{2.900246in}{5.089924in}}{\pgfqpoint{2.908059in}{5.082110in}}%
\pgfpathcurveto{\pgfqpoint{2.915873in}{5.074297in}}{\pgfqpoint{2.926472in}{5.069906in}}{\pgfqpoint{2.937522in}{5.069906in}}%
\pgfpathclose%
\pgfusepath{stroke,fill}%
\end{pgfscope}%
\begin{pgfscope}%
\pgfpathrectangle{\pgfqpoint{0.481978in}{0.331635in}}{\pgfqpoint{9.300000in}{7.700000in}}%
\pgfusepath{clip}%
\pgfsetbuttcap%
\pgfsetroundjoin%
\definecolor{currentfill}{rgb}{0.631373,0.788235,0.956863}%
\pgfsetfillcolor{currentfill}%
\pgfsetlinewidth{0.481800pt}%
\definecolor{currentstroke}{rgb}{1.000000,1.000000,1.000000}%
\pgfsetstrokecolor{currentstroke}%
\pgfsetdash{}{0pt}%
\pgfpathmoveto{\pgfqpoint{2.204208in}{6.210084in}}%
\pgfpathcurveto{\pgfqpoint{2.215259in}{6.210084in}}{\pgfqpoint{2.225858in}{6.214475in}}{\pgfqpoint{2.233671in}{6.222288in}}%
\pgfpathcurveto{\pgfqpoint{2.241485in}{6.230102in}}{\pgfqpoint{2.245875in}{6.240701in}}{\pgfqpoint{2.245875in}{6.251751in}}%
\pgfpathcurveto{\pgfqpoint{2.245875in}{6.262801in}}{\pgfqpoint{2.241485in}{6.273400in}}{\pgfqpoint{2.233671in}{6.281214in}}%
\pgfpathcurveto{\pgfqpoint{2.225858in}{6.289027in}}{\pgfqpoint{2.215259in}{6.293418in}}{\pgfqpoint{2.204208in}{6.293418in}}%
\pgfpathcurveto{\pgfqpoint{2.193158in}{6.293418in}}{\pgfqpoint{2.182559in}{6.289027in}}{\pgfqpoint{2.174746in}{6.281214in}}%
\pgfpathcurveto{\pgfqpoint{2.166932in}{6.273400in}}{\pgfqpoint{2.162542in}{6.262801in}}{\pgfqpoint{2.162542in}{6.251751in}}%
\pgfpathcurveto{\pgfqpoint{2.162542in}{6.240701in}}{\pgfqpoint{2.166932in}{6.230102in}}{\pgfqpoint{2.174746in}{6.222288in}}%
\pgfpathcurveto{\pgfqpoint{2.182559in}{6.214475in}}{\pgfqpoint{2.193158in}{6.210084in}}{\pgfqpoint{2.204208in}{6.210084in}}%
\pgfpathclose%
\pgfusepath{stroke,fill}%
\end{pgfscope}%
\begin{pgfscope}%
\pgfpathrectangle{\pgfqpoint{0.481978in}{0.331635in}}{\pgfqpoint{9.300000in}{7.700000in}}%
\pgfusepath{clip}%
\pgfsetbuttcap%
\pgfsetroundjoin%
\definecolor{currentfill}{rgb}{0.631373,0.788235,0.956863}%
\pgfsetfillcolor{currentfill}%
\pgfsetlinewidth{0.481800pt}%
\definecolor{currentstroke}{rgb}{1.000000,1.000000,1.000000}%
\pgfsetstrokecolor{currentstroke}%
\pgfsetdash{}{0pt}%
\pgfpathmoveto{\pgfqpoint{3.780219in}{6.346116in}}%
\pgfpathcurveto{\pgfqpoint{3.791269in}{6.346116in}}{\pgfqpoint{3.801868in}{6.350506in}}{\pgfqpoint{3.809681in}{6.358320in}}%
\pgfpathcurveto{\pgfqpoint{3.817495in}{6.366133in}}{\pgfqpoint{3.821885in}{6.376732in}}{\pgfqpoint{3.821885in}{6.387783in}}%
\pgfpathcurveto{\pgfqpoint{3.821885in}{6.398833in}}{\pgfqpoint{3.817495in}{6.409432in}}{\pgfqpoint{3.809681in}{6.417245in}}%
\pgfpathcurveto{\pgfqpoint{3.801868in}{6.425059in}}{\pgfqpoint{3.791269in}{6.429449in}}{\pgfqpoint{3.780219in}{6.429449in}}%
\pgfpathcurveto{\pgfqpoint{3.769168in}{6.429449in}}{\pgfqpoint{3.758569in}{6.425059in}}{\pgfqpoint{3.750756in}{6.417245in}}%
\pgfpathcurveto{\pgfqpoint{3.742942in}{6.409432in}}{\pgfqpoint{3.738552in}{6.398833in}}{\pgfqpoint{3.738552in}{6.387783in}}%
\pgfpathcurveto{\pgfqpoint{3.738552in}{6.376732in}}{\pgfqpoint{3.742942in}{6.366133in}}{\pgfqpoint{3.750756in}{6.358320in}}%
\pgfpathcurveto{\pgfqpoint{3.758569in}{6.350506in}}{\pgfqpoint{3.769168in}{6.346116in}}{\pgfqpoint{3.780219in}{6.346116in}}%
\pgfpathclose%
\pgfusepath{stroke,fill}%
\end{pgfscope}%
\begin{pgfscope}%
\pgfpathrectangle{\pgfqpoint{0.481978in}{0.331635in}}{\pgfqpoint{9.300000in}{7.700000in}}%
\pgfusepath{clip}%
\pgfsetbuttcap%
\pgfsetroundjoin%
\definecolor{currentfill}{rgb}{0.631373,0.788235,0.956863}%
\pgfsetfillcolor{currentfill}%
\pgfsetlinewidth{0.481800pt}%
\definecolor{currentstroke}{rgb}{1.000000,1.000000,1.000000}%
\pgfsetstrokecolor{currentstroke}%
\pgfsetdash{}{0pt}%
\pgfpathmoveto{\pgfqpoint{4.814839in}{2.753029in}}%
\pgfpathcurveto{\pgfqpoint{4.825890in}{2.753029in}}{\pgfqpoint{4.836489in}{2.757419in}}{\pgfqpoint{4.844302in}{2.765233in}}%
\pgfpathcurveto{\pgfqpoint{4.852116in}{2.773046in}}{\pgfqpoint{4.856506in}{2.783646in}}{\pgfqpoint{4.856506in}{2.794696in}}%
\pgfpathcurveto{\pgfqpoint{4.856506in}{2.805746in}}{\pgfqpoint{4.852116in}{2.816345in}}{\pgfqpoint{4.844302in}{2.824158in}}%
\pgfpathcurveto{\pgfqpoint{4.836489in}{2.831972in}}{\pgfqpoint{4.825890in}{2.836362in}}{\pgfqpoint{4.814839in}{2.836362in}}%
\pgfpathcurveto{\pgfqpoint{4.803789in}{2.836362in}}{\pgfqpoint{4.793190in}{2.831972in}}{\pgfqpoint{4.785377in}{2.824158in}}%
\pgfpathcurveto{\pgfqpoint{4.777563in}{2.816345in}}{\pgfqpoint{4.773173in}{2.805746in}}{\pgfqpoint{4.773173in}{2.794696in}}%
\pgfpathcurveto{\pgfqpoint{4.773173in}{2.783646in}}{\pgfqpoint{4.777563in}{2.773046in}}{\pgfqpoint{4.785377in}{2.765233in}}%
\pgfpathcurveto{\pgfqpoint{4.793190in}{2.757419in}}{\pgfqpoint{4.803789in}{2.753029in}}{\pgfqpoint{4.814839in}{2.753029in}}%
\pgfpathclose%
\pgfusepath{stroke,fill}%
\end{pgfscope}%
\begin{pgfscope}%
\pgfpathrectangle{\pgfqpoint{0.481978in}{0.331635in}}{\pgfqpoint{9.300000in}{7.700000in}}%
\pgfusepath{clip}%
\pgfsetbuttcap%
\pgfsetroundjoin%
\definecolor{currentfill}{rgb}{0.631373,0.788235,0.956863}%
\pgfsetfillcolor{currentfill}%
\pgfsetlinewidth{0.481800pt}%
\definecolor{currentstroke}{rgb}{1.000000,1.000000,1.000000}%
\pgfsetstrokecolor{currentstroke}%
\pgfsetdash{}{0pt}%
\pgfpathmoveto{\pgfqpoint{2.753007in}{3.827926in}}%
\pgfpathcurveto{\pgfqpoint{2.764057in}{3.827926in}}{\pgfqpoint{2.774656in}{3.832316in}}{\pgfqpoint{2.782470in}{3.840130in}}%
\pgfpathcurveto{\pgfqpoint{2.790284in}{3.847944in}}{\pgfqpoint{2.794674in}{3.858543in}}{\pgfqpoint{2.794674in}{3.869593in}}%
\pgfpathcurveto{\pgfqpoint{2.794674in}{3.880643in}}{\pgfqpoint{2.790284in}{3.891242in}}{\pgfqpoint{2.782470in}{3.899056in}}%
\pgfpathcurveto{\pgfqpoint{2.774656in}{3.906869in}}{\pgfqpoint{2.764057in}{3.911259in}}{\pgfqpoint{2.753007in}{3.911259in}}%
\pgfpathcurveto{\pgfqpoint{2.741957in}{3.911259in}}{\pgfqpoint{2.731358in}{3.906869in}}{\pgfqpoint{2.723544in}{3.899056in}}%
\pgfpathcurveto{\pgfqpoint{2.715731in}{3.891242in}}{\pgfqpoint{2.711341in}{3.880643in}}{\pgfqpoint{2.711341in}{3.869593in}}%
\pgfpathcurveto{\pgfqpoint{2.711341in}{3.858543in}}{\pgfqpoint{2.715731in}{3.847944in}}{\pgfqpoint{2.723544in}{3.840130in}}%
\pgfpathcurveto{\pgfqpoint{2.731358in}{3.832316in}}{\pgfqpoint{2.741957in}{3.827926in}}{\pgfqpoint{2.753007in}{3.827926in}}%
\pgfpathclose%
\pgfusepath{stroke,fill}%
\end{pgfscope}%
\begin{pgfscope}%
\pgfpathrectangle{\pgfqpoint{0.481978in}{0.331635in}}{\pgfqpoint{9.300000in}{7.700000in}}%
\pgfusepath{clip}%
\pgfsetbuttcap%
\pgfsetroundjoin%
\definecolor{currentfill}{rgb}{0.631373,0.788235,0.956863}%
\pgfsetfillcolor{currentfill}%
\pgfsetlinewidth{0.481800pt}%
\definecolor{currentstroke}{rgb}{1.000000,1.000000,1.000000}%
\pgfsetstrokecolor{currentstroke}%
\pgfsetdash{}{0pt}%
\pgfpathmoveto{\pgfqpoint{2.663463in}{5.659614in}}%
\pgfpathcurveto{\pgfqpoint{2.674513in}{5.659614in}}{\pgfqpoint{2.685112in}{5.664004in}}{\pgfqpoint{2.692925in}{5.671818in}}%
\pgfpathcurveto{\pgfqpoint{2.700739in}{5.679631in}}{\pgfqpoint{2.705129in}{5.690230in}}{\pgfqpoint{2.705129in}{5.701280in}}%
\pgfpathcurveto{\pgfqpoint{2.705129in}{5.712330in}}{\pgfqpoint{2.700739in}{5.722929in}}{\pgfqpoint{2.692925in}{5.730743in}}%
\pgfpathcurveto{\pgfqpoint{2.685112in}{5.738557in}}{\pgfqpoint{2.674513in}{5.742947in}}{\pgfqpoint{2.663463in}{5.742947in}}%
\pgfpathcurveto{\pgfqpoint{2.652412in}{5.742947in}}{\pgfqpoint{2.641813in}{5.738557in}}{\pgfqpoint{2.634000in}{5.730743in}}%
\pgfpathcurveto{\pgfqpoint{2.626186in}{5.722929in}}{\pgfqpoint{2.621796in}{5.712330in}}{\pgfqpoint{2.621796in}{5.701280in}}%
\pgfpathcurveto{\pgfqpoint{2.621796in}{5.690230in}}{\pgfqpoint{2.626186in}{5.679631in}}{\pgfqpoint{2.634000in}{5.671818in}}%
\pgfpathcurveto{\pgfqpoint{2.641813in}{5.664004in}}{\pgfqpoint{2.652412in}{5.659614in}}{\pgfqpoint{2.663463in}{5.659614in}}%
\pgfpathclose%
\pgfusepath{stroke,fill}%
\end{pgfscope}%
\begin{pgfscope}%
\pgfpathrectangle{\pgfqpoint{0.481978in}{0.331635in}}{\pgfqpoint{9.300000in}{7.700000in}}%
\pgfusepath{clip}%
\pgfsetbuttcap%
\pgfsetroundjoin%
\definecolor{currentfill}{rgb}{0.631373,0.788235,0.956863}%
\pgfsetfillcolor{currentfill}%
\pgfsetlinewidth{0.481800pt}%
\definecolor{currentstroke}{rgb}{1.000000,1.000000,1.000000}%
\pgfsetstrokecolor{currentstroke}%
\pgfsetdash{}{0pt}%
\pgfpathmoveto{\pgfqpoint{6.114065in}{7.438633in}}%
\pgfpathcurveto{\pgfqpoint{6.125115in}{7.438633in}}{\pgfqpoint{6.135714in}{7.443023in}}{\pgfqpoint{6.143528in}{7.450837in}}%
\pgfpathcurveto{\pgfqpoint{6.151342in}{7.458650in}}{\pgfqpoint{6.155732in}{7.469249in}}{\pgfqpoint{6.155732in}{7.480299in}}%
\pgfpathcurveto{\pgfqpoint{6.155732in}{7.491349in}}{\pgfqpoint{6.151342in}{7.501949in}}{\pgfqpoint{6.143528in}{7.509762in}}%
\pgfpathcurveto{\pgfqpoint{6.135714in}{7.517576in}}{\pgfqpoint{6.125115in}{7.521966in}}{\pgfqpoint{6.114065in}{7.521966in}}%
\pgfpathcurveto{\pgfqpoint{6.103015in}{7.521966in}}{\pgfqpoint{6.092416in}{7.517576in}}{\pgfqpoint{6.084602in}{7.509762in}}%
\pgfpathcurveto{\pgfqpoint{6.076789in}{7.501949in}}{\pgfqpoint{6.072398in}{7.491349in}}{\pgfqpoint{6.072398in}{7.480299in}}%
\pgfpathcurveto{\pgfqpoint{6.072398in}{7.469249in}}{\pgfqpoint{6.076789in}{7.458650in}}{\pgfqpoint{6.084602in}{7.450837in}}%
\pgfpathcurveto{\pgfqpoint{6.092416in}{7.443023in}}{\pgfqpoint{6.103015in}{7.438633in}}{\pgfqpoint{6.114065in}{7.438633in}}%
\pgfpathclose%
\pgfusepath{stroke,fill}%
\end{pgfscope}%
\begin{pgfscope}%
\pgfpathrectangle{\pgfqpoint{0.481978in}{0.331635in}}{\pgfqpoint{9.300000in}{7.700000in}}%
\pgfusepath{clip}%
\pgfsetbuttcap%
\pgfsetroundjoin%
\definecolor{currentfill}{rgb}{1.000000,0.705882,0.509804}%
\pgfsetfillcolor{currentfill}%
\pgfsetlinewidth{0.481800pt}%
\definecolor{currentstroke}{rgb}{1.000000,1.000000,1.000000}%
\pgfsetstrokecolor{currentstroke}%
\pgfsetdash{}{0pt}%
\pgfpathmoveto{\pgfqpoint{3.154468in}{2.446661in}}%
\pgfpathcurveto{\pgfqpoint{3.165519in}{2.446661in}}{\pgfqpoint{3.176118in}{2.451052in}}{\pgfqpoint{3.183931in}{2.458865in}}%
\pgfpathcurveto{\pgfqpoint{3.191745in}{2.466679in}}{\pgfqpoint{3.196135in}{2.477278in}}{\pgfqpoint{3.196135in}{2.488328in}}%
\pgfpathcurveto{\pgfqpoint{3.196135in}{2.499378in}}{\pgfqpoint{3.191745in}{2.509977in}}{\pgfqpoint{3.183931in}{2.517791in}}%
\pgfpathcurveto{\pgfqpoint{3.176118in}{2.525604in}}{\pgfqpoint{3.165519in}{2.529995in}}{\pgfqpoint{3.154468in}{2.529995in}}%
\pgfpathcurveto{\pgfqpoint{3.143418in}{2.529995in}}{\pgfqpoint{3.132819in}{2.525604in}}{\pgfqpoint{3.125006in}{2.517791in}}%
\pgfpathcurveto{\pgfqpoint{3.117192in}{2.509977in}}{\pgfqpoint{3.112802in}{2.499378in}}{\pgfqpoint{3.112802in}{2.488328in}}%
\pgfpathcurveto{\pgfqpoint{3.112802in}{2.477278in}}{\pgfqpoint{3.117192in}{2.466679in}}{\pgfqpoint{3.125006in}{2.458865in}}%
\pgfpathcurveto{\pgfqpoint{3.132819in}{2.451052in}}{\pgfqpoint{3.143418in}{2.446661in}}{\pgfqpoint{3.154468in}{2.446661in}}%
\pgfpathclose%
\pgfusepath{stroke,fill}%
\end{pgfscope}%
\begin{pgfscope}%
\pgfpathrectangle{\pgfqpoint{0.481978in}{0.331635in}}{\pgfqpoint{9.300000in}{7.700000in}}%
\pgfusepath{clip}%
\pgfsetbuttcap%
\pgfsetroundjoin%
\definecolor{currentfill}{rgb}{1.000000,0.705882,0.509804}%
\pgfsetfillcolor{currentfill}%
\pgfsetlinewidth{0.481800pt}%
\definecolor{currentstroke}{rgb}{1.000000,1.000000,1.000000}%
\pgfsetstrokecolor{currentstroke}%
\pgfsetdash{}{0pt}%
\pgfpathmoveto{\pgfqpoint{1.545207in}{2.058337in}}%
\pgfpathcurveto{\pgfqpoint{1.556257in}{2.058337in}}{\pgfqpoint{1.566856in}{2.062727in}}{\pgfqpoint{1.574670in}{2.070541in}}%
\pgfpathcurveto{\pgfqpoint{1.582484in}{2.078355in}}{\pgfqpoint{1.586874in}{2.088954in}}{\pgfqpoint{1.586874in}{2.100004in}}%
\pgfpathcurveto{\pgfqpoint{1.586874in}{2.111054in}}{\pgfqpoint{1.582484in}{2.121653in}}{\pgfqpoint{1.574670in}{2.129467in}}%
\pgfpathcurveto{\pgfqpoint{1.566856in}{2.137280in}}{\pgfqpoint{1.556257in}{2.141671in}}{\pgfqpoint{1.545207in}{2.141671in}}%
\pgfpathcurveto{\pgfqpoint{1.534157in}{2.141671in}}{\pgfqpoint{1.523558in}{2.137280in}}{\pgfqpoint{1.515744in}{2.129467in}}%
\pgfpathcurveto{\pgfqpoint{1.507931in}{2.121653in}}{\pgfqpoint{1.503541in}{2.111054in}}{\pgfqpoint{1.503541in}{2.100004in}}%
\pgfpathcurveto{\pgfqpoint{1.503541in}{2.088954in}}{\pgfqpoint{1.507931in}{2.078355in}}{\pgfqpoint{1.515744in}{2.070541in}}%
\pgfpathcurveto{\pgfqpoint{1.523558in}{2.062727in}}{\pgfqpoint{1.534157in}{2.058337in}}{\pgfqpoint{1.545207in}{2.058337in}}%
\pgfpathclose%
\pgfusepath{stroke,fill}%
\end{pgfscope}%
\begin{pgfscope}%
\pgfpathrectangle{\pgfqpoint{0.481978in}{0.331635in}}{\pgfqpoint{9.300000in}{7.700000in}}%
\pgfusepath{clip}%
\pgfsetbuttcap%
\pgfsetroundjoin%
\definecolor{currentfill}{rgb}{1.000000,0.705882,0.509804}%
\pgfsetfillcolor{currentfill}%
\pgfsetlinewidth{0.481800pt}%
\definecolor{currentstroke}{rgb}{1.000000,1.000000,1.000000}%
\pgfsetstrokecolor{currentstroke}%
\pgfsetdash{}{0pt}%
\pgfpathmoveto{\pgfqpoint{3.307338in}{3.365821in}}%
\pgfpathcurveto{\pgfqpoint{3.318388in}{3.365821in}}{\pgfqpoint{3.328987in}{3.370211in}}{\pgfqpoint{3.336801in}{3.378025in}}%
\pgfpathcurveto{\pgfqpoint{3.344614in}{3.385838in}}{\pgfqpoint{3.349005in}{3.396437in}}{\pgfqpoint{3.349005in}{3.407488in}}%
\pgfpathcurveto{\pgfqpoint{3.349005in}{3.418538in}}{\pgfqpoint{3.344614in}{3.429137in}}{\pgfqpoint{3.336801in}{3.436950in}}%
\pgfpathcurveto{\pgfqpoint{3.328987in}{3.444764in}}{\pgfqpoint{3.318388in}{3.449154in}}{\pgfqpoint{3.307338in}{3.449154in}}%
\pgfpathcurveto{\pgfqpoint{3.296288in}{3.449154in}}{\pgfqpoint{3.285689in}{3.444764in}}{\pgfqpoint{3.277875in}{3.436950in}}%
\pgfpathcurveto{\pgfqpoint{3.270062in}{3.429137in}}{\pgfqpoint{3.265671in}{3.418538in}}{\pgfqpoint{3.265671in}{3.407488in}}%
\pgfpathcurveto{\pgfqpoint{3.265671in}{3.396437in}}{\pgfqpoint{3.270062in}{3.385838in}}{\pgfqpoint{3.277875in}{3.378025in}}%
\pgfpathcurveto{\pgfqpoint{3.285689in}{3.370211in}}{\pgfqpoint{3.296288in}{3.365821in}}{\pgfqpoint{3.307338in}{3.365821in}}%
\pgfpathclose%
\pgfusepath{stroke,fill}%
\end{pgfscope}%
\begin{pgfscope}%
\pgfpathrectangle{\pgfqpoint{0.481978in}{0.331635in}}{\pgfqpoint{9.300000in}{7.700000in}}%
\pgfusepath{clip}%
\pgfsetbuttcap%
\pgfsetroundjoin%
\definecolor{currentfill}{rgb}{1.000000,0.705882,0.509804}%
\pgfsetfillcolor{currentfill}%
\pgfsetlinewidth{0.481800pt}%
\definecolor{currentstroke}{rgb}{1.000000,1.000000,1.000000}%
\pgfsetstrokecolor{currentstroke}%
\pgfsetdash{}{0pt}%
\pgfpathmoveto{\pgfqpoint{5.143128in}{7.146530in}}%
\pgfpathcurveto{\pgfqpoint{5.154178in}{7.146530in}}{\pgfqpoint{5.164777in}{7.150920in}}{\pgfqpoint{5.172591in}{7.158734in}}%
\pgfpathcurveto{\pgfqpoint{5.180405in}{7.166547in}}{\pgfqpoint{5.184795in}{7.177146in}}{\pgfqpoint{5.184795in}{7.188196in}}%
\pgfpathcurveto{\pgfqpoint{5.184795in}{7.199247in}}{\pgfqpoint{5.180405in}{7.209846in}}{\pgfqpoint{5.172591in}{7.217659in}}%
\pgfpathcurveto{\pgfqpoint{5.164777in}{7.225473in}}{\pgfqpoint{5.154178in}{7.229863in}}{\pgfqpoint{5.143128in}{7.229863in}}%
\pgfpathcurveto{\pgfqpoint{5.132078in}{7.229863in}}{\pgfqpoint{5.121479in}{7.225473in}}{\pgfqpoint{5.113665in}{7.217659in}}%
\pgfpathcurveto{\pgfqpoint{5.105852in}{7.209846in}}{\pgfqpoint{5.101461in}{7.199247in}}{\pgfqpoint{5.101461in}{7.188196in}}%
\pgfpathcurveto{\pgfqpoint{5.101461in}{7.177146in}}{\pgfqpoint{5.105852in}{7.166547in}}{\pgfqpoint{5.113665in}{7.158734in}}%
\pgfpathcurveto{\pgfqpoint{5.121479in}{7.150920in}}{\pgfqpoint{5.132078in}{7.146530in}}{\pgfqpoint{5.143128in}{7.146530in}}%
\pgfpathclose%
\pgfusepath{stroke,fill}%
\end{pgfscope}%
\begin{pgfscope}%
\pgfpathrectangle{\pgfqpoint{0.481978in}{0.331635in}}{\pgfqpoint{9.300000in}{7.700000in}}%
\pgfusepath{clip}%
\pgfsetbuttcap%
\pgfsetroundjoin%
\definecolor{currentfill}{rgb}{1.000000,0.705882,0.509804}%
\pgfsetfillcolor{currentfill}%
\pgfsetlinewidth{0.481800pt}%
\definecolor{currentstroke}{rgb}{1.000000,1.000000,1.000000}%
\pgfsetstrokecolor{currentstroke}%
\pgfsetdash{}{0pt}%
\pgfpathmoveto{\pgfqpoint{3.238886in}{1.903284in}}%
\pgfpathcurveto{\pgfqpoint{3.249936in}{1.903284in}}{\pgfqpoint{3.260535in}{1.907675in}}{\pgfqpoint{3.268349in}{1.915488in}}%
\pgfpathcurveto{\pgfqpoint{3.276162in}{1.923302in}}{\pgfqpoint{3.280553in}{1.933901in}}{\pgfqpoint{3.280553in}{1.944951in}}%
\pgfpathcurveto{\pgfqpoint{3.280553in}{1.956001in}}{\pgfqpoint{3.276162in}{1.966600in}}{\pgfqpoint{3.268349in}{1.974414in}}%
\pgfpathcurveto{\pgfqpoint{3.260535in}{1.982227in}}{\pgfqpoint{3.249936in}{1.986618in}}{\pgfqpoint{3.238886in}{1.986618in}}%
\pgfpathcurveto{\pgfqpoint{3.227836in}{1.986618in}}{\pgfqpoint{3.217237in}{1.982227in}}{\pgfqpoint{3.209423in}{1.974414in}}%
\pgfpathcurveto{\pgfqpoint{3.201610in}{1.966600in}}{\pgfqpoint{3.197219in}{1.956001in}}{\pgfqpoint{3.197219in}{1.944951in}}%
\pgfpathcurveto{\pgfqpoint{3.197219in}{1.933901in}}{\pgfqpoint{3.201610in}{1.923302in}}{\pgfqpoint{3.209423in}{1.915488in}}%
\pgfpathcurveto{\pgfqpoint{3.217237in}{1.907675in}}{\pgfqpoint{3.227836in}{1.903284in}}{\pgfqpoint{3.238886in}{1.903284in}}%
\pgfpathclose%
\pgfusepath{stroke,fill}%
\end{pgfscope}%
\begin{pgfscope}%
\pgfpathrectangle{\pgfqpoint{0.481978in}{0.331635in}}{\pgfqpoint{9.300000in}{7.700000in}}%
\pgfusepath{clip}%
\pgfsetbuttcap%
\pgfsetroundjoin%
\definecolor{currentfill}{rgb}{1.000000,0.705882,0.509804}%
\pgfsetfillcolor{currentfill}%
\pgfsetlinewidth{0.481800pt}%
\definecolor{currentstroke}{rgb}{1.000000,1.000000,1.000000}%
\pgfsetstrokecolor{currentstroke}%
\pgfsetdash{}{0pt}%
\pgfpathmoveto{\pgfqpoint{6.096137in}{1.233341in}}%
\pgfpathcurveto{\pgfqpoint{6.107187in}{1.233341in}}{\pgfqpoint{6.117786in}{1.237731in}}{\pgfqpoint{6.125600in}{1.245545in}}%
\pgfpathcurveto{\pgfqpoint{6.133413in}{1.253359in}}{\pgfqpoint{6.137804in}{1.263958in}}{\pgfqpoint{6.137804in}{1.275008in}}%
\pgfpathcurveto{\pgfqpoint{6.137804in}{1.286058in}}{\pgfqpoint{6.133413in}{1.296657in}}{\pgfqpoint{6.125600in}{1.304471in}}%
\pgfpathcurveto{\pgfqpoint{6.117786in}{1.312284in}}{\pgfqpoint{6.107187in}{1.316675in}}{\pgfqpoint{6.096137in}{1.316675in}}%
\pgfpathcurveto{\pgfqpoint{6.085087in}{1.316675in}}{\pgfqpoint{6.074488in}{1.312284in}}{\pgfqpoint{6.066674in}{1.304471in}}%
\pgfpathcurveto{\pgfqpoint{6.058861in}{1.296657in}}{\pgfqpoint{6.054470in}{1.286058in}}{\pgfqpoint{6.054470in}{1.275008in}}%
\pgfpathcurveto{\pgfqpoint{6.054470in}{1.263958in}}{\pgfqpoint{6.058861in}{1.253359in}}{\pgfqpoint{6.066674in}{1.245545in}}%
\pgfpathcurveto{\pgfqpoint{6.074488in}{1.237731in}}{\pgfqpoint{6.085087in}{1.233341in}}{\pgfqpoint{6.096137in}{1.233341in}}%
\pgfpathclose%
\pgfusepath{stroke,fill}%
\end{pgfscope}%
\begin{pgfscope}%
\pgfpathrectangle{\pgfqpoint{0.481978in}{0.331635in}}{\pgfqpoint{9.300000in}{7.700000in}}%
\pgfusepath{clip}%
\pgfsetbuttcap%
\pgfsetroundjoin%
\definecolor{currentfill}{rgb}{1.000000,0.705882,0.509804}%
\pgfsetfillcolor{currentfill}%
\pgfsetlinewidth{0.481800pt}%
\definecolor{currentstroke}{rgb}{1.000000,1.000000,1.000000}%
\pgfsetstrokecolor{currentstroke}%
\pgfsetdash{}{0pt}%
\pgfpathmoveto{\pgfqpoint{8.892538in}{6.000024in}}%
\pgfpathcurveto{\pgfqpoint{8.903589in}{6.000024in}}{\pgfqpoint{8.914188in}{6.004414in}}{\pgfqpoint{8.922001in}{6.012228in}}%
\pgfpathcurveto{\pgfqpoint{8.929815in}{6.020041in}}{\pgfqpoint{8.934205in}{6.030640in}}{\pgfqpoint{8.934205in}{6.041691in}}%
\pgfpathcurveto{\pgfqpoint{8.934205in}{6.052741in}}{\pgfqpoint{8.929815in}{6.063340in}}{\pgfqpoint{8.922001in}{6.071153in}}%
\pgfpathcurveto{\pgfqpoint{8.914188in}{6.078967in}}{\pgfqpoint{8.903589in}{6.083357in}}{\pgfqpoint{8.892538in}{6.083357in}}%
\pgfpathcurveto{\pgfqpoint{8.881488in}{6.083357in}}{\pgfqpoint{8.870889in}{6.078967in}}{\pgfqpoint{8.863076in}{6.071153in}}%
\pgfpathcurveto{\pgfqpoint{8.855262in}{6.063340in}}{\pgfqpoint{8.850872in}{6.052741in}}{\pgfqpoint{8.850872in}{6.041691in}}%
\pgfpathcurveto{\pgfqpoint{8.850872in}{6.030640in}}{\pgfqpoint{8.855262in}{6.020041in}}{\pgfqpoint{8.863076in}{6.012228in}}%
\pgfpathcurveto{\pgfqpoint{8.870889in}{6.004414in}}{\pgfqpoint{8.881488in}{6.000024in}}{\pgfqpoint{8.892538in}{6.000024in}}%
\pgfpathclose%
\pgfusepath{stroke,fill}%
\end{pgfscope}%
\begin{pgfscope}%
\pgfpathrectangle{\pgfqpoint{0.481978in}{0.331635in}}{\pgfqpoint{9.300000in}{7.700000in}}%
\pgfusepath{clip}%
\pgfsetbuttcap%
\pgfsetroundjoin%
\definecolor{currentfill}{rgb}{1.000000,0.705882,0.509804}%
\pgfsetfillcolor{currentfill}%
\pgfsetlinewidth{0.481800pt}%
\definecolor{currentstroke}{rgb}{1.000000,1.000000,1.000000}%
\pgfsetstrokecolor{currentstroke}%
\pgfsetdash{}{0pt}%
\pgfpathmoveto{\pgfqpoint{8.792161in}{5.432877in}}%
\pgfpathcurveto{\pgfqpoint{8.803212in}{5.432877in}}{\pgfqpoint{8.813811in}{5.437267in}}{\pgfqpoint{8.821624in}{5.445081in}}%
\pgfpathcurveto{\pgfqpoint{8.829438in}{5.452895in}}{\pgfqpoint{8.833828in}{5.463494in}}{\pgfqpoint{8.833828in}{5.474544in}}%
\pgfpathcurveto{\pgfqpoint{8.833828in}{5.485594in}}{\pgfqpoint{8.829438in}{5.496193in}}{\pgfqpoint{8.821624in}{5.504006in}}%
\pgfpathcurveto{\pgfqpoint{8.813811in}{5.511820in}}{\pgfqpoint{8.803212in}{5.516210in}}{\pgfqpoint{8.792161in}{5.516210in}}%
\pgfpathcurveto{\pgfqpoint{8.781111in}{5.516210in}}{\pgfqpoint{8.770512in}{5.511820in}}{\pgfqpoint{8.762699in}{5.504006in}}%
\pgfpathcurveto{\pgfqpoint{8.754885in}{5.496193in}}{\pgfqpoint{8.750495in}{5.485594in}}{\pgfqpoint{8.750495in}{5.474544in}}%
\pgfpathcurveto{\pgfqpoint{8.750495in}{5.463494in}}{\pgfqpoint{8.754885in}{5.452895in}}{\pgfqpoint{8.762699in}{5.445081in}}%
\pgfpathcurveto{\pgfqpoint{8.770512in}{5.437267in}}{\pgfqpoint{8.781111in}{5.432877in}}{\pgfqpoint{8.792161in}{5.432877in}}%
\pgfpathclose%
\pgfusepath{stroke,fill}%
\end{pgfscope}%
\begin{pgfscope}%
\pgfpathrectangle{\pgfqpoint{0.481978in}{0.331635in}}{\pgfqpoint{9.300000in}{7.700000in}}%
\pgfusepath{clip}%
\pgfsetbuttcap%
\pgfsetroundjoin%
\definecolor{currentfill}{rgb}{1.000000,0.705882,0.509804}%
\pgfsetfillcolor{currentfill}%
\pgfsetlinewidth{0.481800pt}%
\definecolor{currentstroke}{rgb}{1.000000,1.000000,1.000000}%
\pgfsetstrokecolor{currentstroke}%
\pgfsetdash{}{0pt}%
\pgfpathmoveto{\pgfqpoint{4.884502in}{4.767853in}}%
\pgfpathcurveto{\pgfqpoint{4.895552in}{4.767853in}}{\pgfqpoint{4.906151in}{4.772243in}}{\pgfqpoint{4.913965in}{4.780057in}}%
\pgfpathcurveto{\pgfqpoint{4.921778in}{4.787870in}}{\pgfqpoint{4.926169in}{4.798469in}}{\pgfqpoint{4.926169in}{4.809519in}}%
\pgfpathcurveto{\pgfqpoint{4.926169in}{4.820570in}}{\pgfqpoint{4.921778in}{4.831169in}}{\pgfqpoint{4.913965in}{4.838982in}}%
\pgfpathcurveto{\pgfqpoint{4.906151in}{4.846796in}}{\pgfqpoint{4.895552in}{4.851186in}}{\pgfqpoint{4.884502in}{4.851186in}}%
\pgfpathcurveto{\pgfqpoint{4.873452in}{4.851186in}}{\pgfqpoint{4.862853in}{4.846796in}}{\pgfqpoint{4.855039in}{4.838982in}}%
\pgfpathcurveto{\pgfqpoint{4.847226in}{4.831169in}}{\pgfqpoint{4.842835in}{4.820570in}}{\pgfqpoint{4.842835in}{4.809519in}}%
\pgfpathcurveto{\pgfqpoint{4.842835in}{4.798469in}}{\pgfqpoint{4.847226in}{4.787870in}}{\pgfqpoint{4.855039in}{4.780057in}}%
\pgfpathcurveto{\pgfqpoint{4.862853in}{4.772243in}}{\pgfqpoint{4.873452in}{4.767853in}}{\pgfqpoint{4.884502in}{4.767853in}}%
\pgfpathclose%
\pgfusepath{stroke,fill}%
\end{pgfscope}%
\begin{pgfscope}%
\pgfpathrectangle{\pgfqpoint{0.481978in}{0.331635in}}{\pgfqpoint{9.300000in}{7.700000in}}%
\pgfusepath{clip}%
\pgfsetbuttcap%
\pgfsetroundjoin%
\definecolor{currentfill}{rgb}{1.000000,0.705882,0.509804}%
\pgfsetfillcolor{currentfill}%
\pgfsetlinewidth{0.481800pt}%
\definecolor{currentstroke}{rgb}{1.000000,1.000000,1.000000}%
\pgfsetstrokecolor{currentstroke}%
\pgfsetdash{}{0pt}%
\pgfpathmoveto{\pgfqpoint{8.839374in}{4.799422in}}%
\pgfpathcurveto{\pgfqpoint{8.850424in}{4.799422in}}{\pgfqpoint{8.861023in}{4.803812in}}{\pgfqpoint{8.868836in}{4.811626in}}%
\pgfpathcurveto{\pgfqpoint{8.876650in}{4.819439in}}{\pgfqpoint{8.881040in}{4.830038in}}{\pgfqpoint{8.881040in}{4.841088in}}%
\pgfpathcurveto{\pgfqpoint{8.881040in}{4.852138in}}{\pgfqpoint{8.876650in}{4.862737in}}{\pgfqpoint{8.868836in}{4.870551in}}%
\pgfpathcurveto{\pgfqpoint{8.861023in}{4.878365in}}{\pgfqpoint{8.850424in}{4.882755in}}{\pgfqpoint{8.839374in}{4.882755in}}%
\pgfpathcurveto{\pgfqpoint{8.828324in}{4.882755in}}{\pgfqpoint{8.817724in}{4.878365in}}{\pgfqpoint{8.809911in}{4.870551in}}%
\pgfpathcurveto{\pgfqpoint{8.802097in}{4.862737in}}{\pgfqpoint{8.797707in}{4.852138in}}{\pgfqpoint{8.797707in}{4.841088in}}%
\pgfpathcurveto{\pgfqpoint{8.797707in}{4.830038in}}{\pgfqpoint{8.802097in}{4.819439in}}{\pgfqpoint{8.809911in}{4.811626in}}%
\pgfpathcurveto{\pgfqpoint{8.817724in}{4.803812in}}{\pgfqpoint{8.828324in}{4.799422in}}{\pgfqpoint{8.839374in}{4.799422in}}%
\pgfpathclose%
\pgfusepath{stroke,fill}%
\end{pgfscope}%
\begin{pgfscope}%
\pgfpathrectangle{\pgfqpoint{0.481978in}{0.331635in}}{\pgfqpoint{9.300000in}{7.700000in}}%
\pgfusepath{clip}%
\pgfsetbuttcap%
\pgfsetroundjoin%
\definecolor{currentfill}{rgb}{1.000000,0.705882,0.509804}%
\pgfsetfillcolor{currentfill}%
\pgfsetlinewidth{0.481800pt}%
\definecolor{currentstroke}{rgb}{1.000000,1.000000,1.000000}%
\pgfsetstrokecolor{currentstroke}%
\pgfsetdash{}{0pt}%
\pgfpathmoveto{\pgfqpoint{8.645625in}{5.963537in}}%
\pgfpathcurveto{\pgfqpoint{8.656675in}{5.963537in}}{\pgfqpoint{8.667274in}{5.967928in}}{\pgfqpoint{8.675088in}{5.975741in}}%
\pgfpathcurveto{\pgfqpoint{8.682902in}{5.983555in}}{\pgfqpoint{8.687292in}{5.994154in}}{\pgfqpoint{8.687292in}{6.005204in}}%
\pgfpathcurveto{\pgfqpoint{8.687292in}{6.016254in}}{\pgfqpoint{8.682902in}{6.026853in}}{\pgfqpoint{8.675088in}{6.034667in}}%
\pgfpathcurveto{\pgfqpoint{8.667274in}{6.042480in}}{\pgfqpoint{8.656675in}{6.046871in}}{\pgfqpoint{8.645625in}{6.046871in}}%
\pgfpathcurveto{\pgfqpoint{8.634575in}{6.046871in}}{\pgfqpoint{8.623976in}{6.042480in}}{\pgfqpoint{8.616162in}{6.034667in}}%
\pgfpathcurveto{\pgfqpoint{8.608349in}{6.026853in}}{\pgfqpoint{8.603959in}{6.016254in}}{\pgfqpoint{8.603959in}{6.005204in}}%
\pgfpathcurveto{\pgfqpoint{8.603959in}{5.994154in}}{\pgfqpoint{8.608349in}{5.983555in}}{\pgfqpoint{8.616162in}{5.975741in}}%
\pgfpathcurveto{\pgfqpoint{8.623976in}{5.967928in}}{\pgfqpoint{8.634575in}{5.963537in}}{\pgfqpoint{8.645625in}{5.963537in}}%
\pgfpathclose%
\pgfusepath{stroke,fill}%
\end{pgfscope}%
\begin{pgfscope}%
\pgfpathrectangle{\pgfqpoint{0.481978in}{0.331635in}}{\pgfqpoint{9.300000in}{7.700000in}}%
\pgfusepath{clip}%
\pgfsetbuttcap%
\pgfsetroundjoin%
\definecolor{currentfill}{rgb}{1.000000,0.705882,0.509804}%
\pgfsetfillcolor{currentfill}%
\pgfsetlinewidth{0.481800pt}%
\definecolor{currentstroke}{rgb}{1.000000,1.000000,1.000000}%
\pgfsetstrokecolor{currentstroke}%
\pgfsetdash{}{0pt}%
\pgfpathmoveto{\pgfqpoint{1.541644in}{2.029501in}}%
\pgfpathcurveto{\pgfqpoint{1.552694in}{2.029501in}}{\pgfqpoint{1.563293in}{2.033891in}}{\pgfqpoint{1.571107in}{2.041705in}}%
\pgfpathcurveto{\pgfqpoint{1.578921in}{2.049518in}}{\pgfqpoint{1.583311in}{2.060117in}}{\pgfqpoint{1.583311in}{2.071167in}}%
\pgfpathcurveto{\pgfqpoint{1.583311in}{2.082218in}}{\pgfqpoint{1.578921in}{2.092817in}}{\pgfqpoint{1.571107in}{2.100630in}}%
\pgfpathcurveto{\pgfqpoint{1.563293in}{2.108444in}}{\pgfqpoint{1.552694in}{2.112834in}}{\pgfqpoint{1.541644in}{2.112834in}}%
\pgfpathcurveto{\pgfqpoint{1.530594in}{2.112834in}}{\pgfqpoint{1.519995in}{2.108444in}}{\pgfqpoint{1.512181in}{2.100630in}}%
\pgfpathcurveto{\pgfqpoint{1.504368in}{2.092817in}}{\pgfqpoint{1.499978in}{2.082218in}}{\pgfqpoint{1.499978in}{2.071167in}}%
\pgfpathcurveto{\pgfqpoint{1.499978in}{2.060117in}}{\pgfqpoint{1.504368in}{2.049518in}}{\pgfqpoint{1.512181in}{2.041705in}}%
\pgfpathcurveto{\pgfqpoint{1.519995in}{2.033891in}}{\pgfqpoint{1.530594in}{2.029501in}}{\pgfqpoint{1.541644in}{2.029501in}}%
\pgfpathclose%
\pgfusepath{stroke,fill}%
\end{pgfscope}%
\begin{pgfscope}%
\pgfpathrectangle{\pgfqpoint{0.481978in}{0.331635in}}{\pgfqpoint{9.300000in}{7.700000in}}%
\pgfusepath{clip}%
\pgfsetbuttcap%
\pgfsetroundjoin%
\definecolor{currentfill}{rgb}{1.000000,0.705882,0.509804}%
\pgfsetfillcolor{currentfill}%
\pgfsetlinewidth{0.481800pt}%
\definecolor{currentstroke}{rgb}{1.000000,1.000000,1.000000}%
\pgfsetstrokecolor{currentstroke}%
\pgfsetdash{}{0pt}%
\pgfpathmoveto{\pgfqpoint{8.938916in}{5.368764in}}%
\pgfpathcurveto{\pgfqpoint{8.949966in}{5.368764in}}{\pgfqpoint{8.960565in}{5.373154in}}{\pgfqpoint{8.968379in}{5.380967in}}%
\pgfpathcurveto{\pgfqpoint{8.976193in}{5.388781in}}{\pgfqpoint{8.980583in}{5.399380in}}{\pgfqpoint{8.980583in}{5.410430in}}%
\pgfpathcurveto{\pgfqpoint{8.980583in}{5.421480in}}{\pgfqpoint{8.976193in}{5.432079in}}{\pgfqpoint{8.968379in}{5.439893in}}%
\pgfpathcurveto{\pgfqpoint{8.960565in}{5.447707in}}{\pgfqpoint{8.949966in}{5.452097in}}{\pgfqpoint{8.938916in}{5.452097in}}%
\pgfpathcurveto{\pgfqpoint{8.927866in}{5.452097in}}{\pgfqpoint{8.917267in}{5.447707in}}{\pgfqpoint{8.909453in}{5.439893in}}%
\pgfpathcurveto{\pgfqpoint{8.901640in}{5.432079in}}{\pgfqpoint{8.897250in}{5.421480in}}{\pgfqpoint{8.897250in}{5.410430in}}%
\pgfpathcurveto{\pgfqpoint{8.897250in}{5.399380in}}{\pgfqpoint{8.901640in}{5.388781in}}{\pgfqpoint{8.909453in}{5.380967in}}%
\pgfpathcurveto{\pgfqpoint{8.917267in}{5.373154in}}{\pgfqpoint{8.927866in}{5.368764in}}{\pgfqpoint{8.938916in}{5.368764in}}%
\pgfpathclose%
\pgfusepath{stroke,fill}%
\end{pgfscope}%
\begin{pgfscope}%
\pgfpathrectangle{\pgfqpoint{0.481978in}{0.331635in}}{\pgfqpoint{9.300000in}{7.700000in}}%
\pgfusepath{clip}%
\pgfsetbuttcap%
\pgfsetroundjoin%
\definecolor{currentfill}{rgb}{1.000000,0.705882,0.509804}%
\pgfsetfillcolor{currentfill}%
\pgfsetlinewidth{0.481800pt}%
\definecolor{currentstroke}{rgb}{1.000000,1.000000,1.000000}%
\pgfsetstrokecolor{currentstroke}%
\pgfsetdash{}{0pt}%
\pgfpathmoveto{\pgfqpoint{3.682482in}{4.956310in}}%
\pgfpathcurveto{\pgfqpoint{3.693532in}{4.956310in}}{\pgfqpoint{3.704131in}{4.960700in}}{\pgfqpoint{3.711945in}{4.968514in}}%
\pgfpathcurveto{\pgfqpoint{3.719759in}{4.976328in}}{\pgfqpoint{3.724149in}{4.986927in}}{\pgfqpoint{3.724149in}{4.997977in}}%
\pgfpathcurveto{\pgfqpoint{3.724149in}{5.009027in}}{\pgfqpoint{3.719759in}{5.019626in}}{\pgfqpoint{3.711945in}{5.027440in}}%
\pgfpathcurveto{\pgfqpoint{3.704131in}{5.035253in}}{\pgfqpoint{3.693532in}{5.039644in}}{\pgfqpoint{3.682482in}{5.039644in}}%
\pgfpathcurveto{\pgfqpoint{3.671432in}{5.039644in}}{\pgfqpoint{3.660833in}{5.035253in}}{\pgfqpoint{3.653019in}{5.027440in}}%
\pgfpathcurveto{\pgfqpoint{3.645206in}{5.019626in}}{\pgfqpoint{3.640816in}{5.009027in}}{\pgfqpoint{3.640816in}{4.997977in}}%
\pgfpathcurveto{\pgfqpoint{3.640816in}{4.986927in}}{\pgfqpoint{3.645206in}{4.976328in}}{\pgfqpoint{3.653019in}{4.968514in}}%
\pgfpathcurveto{\pgfqpoint{3.660833in}{4.960700in}}{\pgfqpoint{3.671432in}{4.956310in}}{\pgfqpoint{3.682482in}{4.956310in}}%
\pgfpathclose%
\pgfusepath{stroke,fill}%
\end{pgfscope}%
\begin{pgfscope}%
\pgfpathrectangle{\pgfqpoint{0.481978in}{0.331635in}}{\pgfqpoint{9.300000in}{7.700000in}}%
\pgfusepath{clip}%
\pgfsetbuttcap%
\pgfsetroundjoin%
\definecolor{currentfill}{rgb}{1.000000,0.705882,0.509804}%
\pgfsetfillcolor{currentfill}%
\pgfsetlinewidth{0.481800pt}%
\definecolor{currentstroke}{rgb}{1.000000,1.000000,1.000000}%
\pgfsetstrokecolor{currentstroke}%
\pgfsetdash{}{0pt}%
\pgfpathmoveto{\pgfqpoint{7.449113in}{2.447310in}}%
\pgfpathcurveto{\pgfqpoint{7.460163in}{2.447310in}}{\pgfqpoint{7.470762in}{2.451700in}}{\pgfqpoint{7.478576in}{2.459514in}}%
\pgfpathcurveto{\pgfqpoint{7.486390in}{2.467327in}}{\pgfqpoint{7.490780in}{2.477927in}}{\pgfqpoint{7.490780in}{2.488977in}}%
\pgfpathcurveto{\pgfqpoint{7.490780in}{2.500027in}}{\pgfqpoint{7.486390in}{2.510626in}}{\pgfqpoint{7.478576in}{2.518439in}}%
\pgfpathcurveto{\pgfqpoint{7.470762in}{2.526253in}}{\pgfqpoint{7.460163in}{2.530643in}}{\pgfqpoint{7.449113in}{2.530643in}}%
\pgfpathcurveto{\pgfqpoint{7.438063in}{2.530643in}}{\pgfqpoint{7.427464in}{2.526253in}}{\pgfqpoint{7.419650in}{2.518439in}}%
\pgfpathcurveto{\pgfqpoint{7.411837in}{2.510626in}}{\pgfqpoint{7.407447in}{2.500027in}}{\pgfqpoint{7.407447in}{2.488977in}}%
\pgfpathcurveto{\pgfqpoint{7.407447in}{2.477927in}}{\pgfqpoint{7.411837in}{2.467327in}}{\pgfqpoint{7.419650in}{2.459514in}}%
\pgfpathcurveto{\pgfqpoint{7.427464in}{2.451700in}}{\pgfqpoint{7.438063in}{2.447310in}}{\pgfqpoint{7.449113in}{2.447310in}}%
\pgfpathclose%
\pgfusepath{stroke,fill}%
\end{pgfscope}%
\begin{pgfscope}%
\pgfpathrectangle{\pgfqpoint{0.481978in}{0.331635in}}{\pgfqpoint{9.300000in}{7.700000in}}%
\pgfusepath{clip}%
\pgfsetbuttcap%
\pgfsetroundjoin%
\definecolor{currentfill}{rgb}{1.000000,0.705882,0.509804}%
\pgfsetfillcolor{currentfill}%
\pgfsetlinewidth{0.481800pt}%
\definecolor{currentstroke}{rgb}{1.000000,1.000000,1.000000}%
\pgfsetstrokecolor{currentstroke}%
\pgfsetdash{}{0pt}%
\pgfpathmoveto{\pgfqpoint{8.325442in}{5.151331in}}%
\pgfpathcurveto{\pgfqpoint{8.336492in}{5.151331in}}{\pgfqpoint{8.347091in}{5.155721in}}{\pgfqpoint{8.354904in}{5.163535in}}%
\pgfpathcurveto{\pgfqpoint{8.362718in}{5.171348in}}{\pgfqpoint{8.367108in}{5.181947in}}{\pgfqpoint{8.367108in}{5.192998in}}%
\pgfpathcurveto{\pgfqpoint{8.367108in}{5.204048in}}{\pgfqpoint{8.362718in}{5.214647in}}{\pgfqpoint{8.354904in}{5.222460in}}%
\pgfpathcurveto{\pgfqpoint{8.347091in}{5.230274in}}{\pgfqpoint{8.336492in}{5.234664in}}{\pgfqpoint{8.325442in}{5.234664in}}%
\pgfpathcurveto{\pgfqpoint{8.314392in}{5.234664in}}{\pgfqpoint{8.303793in}{5.230274in}}{\pgfqpoint{8.295979in}{5.222460in}}%
\pgfpathcurveto{\pgfqpoint{8.288165in}{5.214647in}}{\pgfqpoint{8.283775in}{5.204048in}}{\pgfqpoint{8.283775in}{5.192998in}}%
\pgfpathcurveto{\pgfqpoint{8.283775in}{5.181947in}}{\pgfqpoint{8.288165in}{5.171348in}}{\pgfqpoint{8.295979in}{5.163535in}}%
\pgfpathcurveto{\pgfqpoint{8.303793in}{5.155721in}}{\pgfqpoint{8.314392in}{5.151331in}}{\pgfqpoint{8.325442in}{5.151331in}}%
\pgfpathclose%
\pgfusepath{stroke,fill}%
\end{pgfscope}%
\begin{pgfscope}%
\pgfpathrectangle{\pgfqpoint{0.481978in}{0.331635in}}{\pgfqpoint{9.300000in}{7.700000in}}%
\pgfusepath{clip}%
\pgfsetbuttcap%
\pgfsetroundjoin%
\definecolor{currentfill}{rgb}{1.000000,0.705882,0.509804}%
\pgfsetfillcolor{currentfill}%
\pgfsetlinewidth{0.481800pt}%
\definecolor{currentstroke}{rgb}{1.000000,1.000000,1.000000}%
\pgfsetstrokecolor{currentstroke}%
\pgfsetdash{}{0pt}%
\pgfpathmoveto{\pgfqpoint{7.578449in}{4.350693in}}%
\pgfpathcurveto{\pgfqpoint{7.589499in}{4.350693in}}{\pgfqpoint{7.600098in}{4.355084in}}{\pgfqpoint{7.607912in}{4.362897in}}%
\pgfpathcurveto{\pgfqpoint{7.615726in}{4.370711in}}{\pgfqpoint{7.620116in}{4.381310in}}{\pgfqpoint{7.620116in}{4.392360in}}%
\pgfpathcurveto{\pgfqpoint{7.620116in}{4.403410in}}{\pgfqpoint{7.615726in}{4.414009in}}{\pgfqpoint{7.607912in}{4.421823in}}%
\pgfpathcurveto{\pgfqpoint{7.600098in}{4.429636in}}{\pgfqpoint{7.589499in}{4.434027in}}{\pgfqpoint{7.578449in}{4.434027in}}%
\pgfpathcurveto{\pgfqpoint{7.567399in}{4.434027in}}{\pgfqpoint{7.556800in}{4.429636in}}{\pgfqpoint{7.548986in}{4.421823in}}%
\pgfpathcurveto{\pgfqpoint{7.541173in}{4.414009in}}{\pgfqpoint{7.536783in}{4.403410in}}{\pgfqpoint{7.536783in}{4.392360in}}%
\pgfpathcurveto{\pgfqpoint{7.536783in}{4.381310in}}{\pgfqpoint{7.541173in}{4.370711in}}{\pgfqpoint{7.548986in}{4.362897in}}%
\pgfpathcurveto{\pgfqpoint{7.556800in}{4.355084in}}{\pgfqpoint{7.567399in}{4.350693in}}{\pgfqpoint{7.578449in}{4.350693in}}%
\pgfpathclose%
\pgfusepath{stroke,fill}%
\end{pgfscope}%
\begin{pgfscope}%
\pgfpathrectangle{\pgfqpoint{0.481978in}{0.331635in}}{\pgfqpoint{9.300000in}{7.700000in}}%
\pgfusepath{clip}%
\pgfsetbuttcap%
\pgfsetroundjoin%
\definecolor{currentfill}{rgb}{1.000000,0.705882,0.509804}%
\pgfsetfillcolor{currentfill}%
\pgfsetlinewidth{0.481800pt}%
\definecolor{currentstroke}{rgb}{1.000000,1.000000,1.000000}%
\pgfsetstrokecolor{currentstroke}%
\pgfsetdash{}{0pt}%
\pgfpathmoveto{\pgfqpoint{6.440915in}{3.359023in}}%
\pgfpathcurveto{\pgfqpoint{6.451966in}{3.359023in}}{\pgfqpoint{6.462565in}{3.363413in}}{\pgfqpoint{6.470378in}{3.371227in}}%
\pgfpathcurveto{\pgfqpoint{6.478192in}{3.379041in}}{\pgfqpoint{6.482582in}{3.389640in}}{\pgfqpoint{6.482582in}{3.400690in}}%
\pgfpathcurveto{\pgfqpoint{6.482582in}{3.411740in}}{\pgfqpoint{6.478192in}{3.422339in}}{\pgfqpoint{6.470378in}{3.430152in}}%
\pgfpathcurveto{\pgfqpoint{6.462565in}{3.437966in}}{\pgfqpoint{6.451966in}{3.442356in}}{\pgfqpoint{6.440915in}{3.442356in}}%
\pgfpathcurveto{\pgfqpoint{6.429865in}{3.442356in}}{\pgfqpoint{6.419266in}{3.437966in}}{\pgfqpoint{6.411453in}{3.430152in}}%
\pgfpathcurveto{\pgfqpoint{6.403639in}{3.422339in}}{\pgfqpoint{6.399249in}{3.411740in}}{\pgfqpoint{6.399249in}{3.400690in}}%
\pgfpathcurveto{\pgfqpoint{6.399249in}{3.389640in}}{\pgfqpoint{6.403639in}{3.379041in}}{\pgfqpoint{6.411453in}{3.371227in}}%
\pgfpathcurveto{\pgfqpoint{6.419266in}{3.363413in}}{\pgfqpoint{6.429865in}{3.359023in}}{\pgfqpoint{6.440915in}{3.359023in}}%
\pgfpathclose%
\pgfusepath{stroke,fill}%
\end{pgfscope}%
\begin{pgfscope}%
\pgfpathrectangle{\pgfqpoint{0.481978in}{0.331635in}}{\pgfqpoint{9.300000in}{7.700000in}}%
\pgfusepath{clip}%
\pgfsetbuttcap%
\pgfsetroundjoin%
\definecolor{currentfill}{rgb}{1.000000,0.705882,0.509804}%
\pgfsetfillcolor{currentfill}%
\pgfsetlinewidth{0.481800pt}%
\definecolor{currentstroke}{rgb}{1.000000,1.000000,1.000000}%
\pgfsetstrokecolor{currentstroke}%
\pgfsetdash{}{0pt}%
\pgfpathmoveto{\pgfqpoint{3.469279in}{3.827559in}}%
\pgfpathcurveto{\pgfqpoint{3.480329in}{3.827559in}}{\pgfqpoint{3.490928in}{3.831949in}}{\pgfqpoint{3.498742in}{3.839763in}}%
\pgfpathcurveto{\pgfqpoint{3.506555in}{3.847576in}}{\pgfqpoint{3.510946in}{3.858175in}}{\pgfqpoint{3.510946in}{3.869225in}}%
\pgfpathcurveto{\pgfqpoint{3.510946in}{3.880275in}}{\pgfqpoint{3.506555in}{3.890874in}}{\pgfqpoint{3.498742in}{3.898688in}}%
\pgfpathcurveto{\pgfqpoint{3.490928in}{3.906502in}}{\pgfqpoint{3.480329in}{3.910892in}}{\pgfqpoint{3.469279in}{3.910892in}}%
\pgfpathcurveto{\pgfqpoint{3.458229in}{3.910892in}}{\pgfqpoint{3.447630in}{3.906502in}}{\pgfqpoint{3.439816in}{3.898688in}}%
\pgfpathcurveto{\pgfqpoint{3.432003in}{3.890874in}}{\pgfqpoint{3.427612in}{3.880275in}}{\pgfqpoint{3.427612in}{3.869225in}}%
\pgfpathcurveto{\pgfqpoint{3.427612in}{3.858175in}}{\pgfqpoint{3.432003in}{3.847576in}}{\pgfqpoint{3.439816in}{3.839763in}}%
\pgfpathcurveto{\pgfqpoint{3.447630in}{3.831949in}}{\pgfqpoint{3.458229in}{3.827559in}}{\pgfqpoint{3.469279in}{3.827559in}}%
\pgfpathclose%
\pgfusepath{stroke,fill}%
\end{pgfscope}%
\begin{pgfscope}%
\pgfpathrectangle{\pgfqpoint{0.481978in}{0.331635in}}{\pgfqpoint{9.300000in}{7.700000in}}%
\pgfusepath{clip}%
\pgfsetbuttcap%
\pgfsetroundjoin%
\definecolor{currentfill}{rgb}{1.000000,0.705882,0.509804}%
\pgfsetfillcolor{currentfill}%
\pgfsetlinewidth{0.481800pt}%
\definecolor{currentstroke}{rgb}{1.000000,1.000000,1.000000}%
\pgfsetstrokecolor{currentstroke}%
\pgfsetdash{}{0pt}%
\pgfpathmoveto{\pgfqpoint{6.295661in}{1.566834in}}%
\pgfpathcurveto{\pgfqpoint{6.306711in}{1.566834in}}{\pgfqpoint{6.317310in}{1.571225in}}{\pgfqpoint{6.325124in}{1.579038in}}%
\pgfpathcurveto{\pgfqpoint{6.332938in}{1.586852in}}{\pgfqpoint{6.337328in}{1.597451in}}{\pgfqpoint{6.337328in}{1.608501in}}%
\pgfpathcurveto{\pgfqpoint{6.337328in}{1.619551in}}{\pgfqpoint{6.332938in}{1.630150in}}{\pgfqpoint{6.325124in}{1.637964in}}%
\pgfpathcurveto{\pgfqpoint{6.317310in}{1.645777in}}{\pgfqpoint{6.306711in}{1.650168in}}{\pgfqpoint{6.295661in}{1.650168in}}%
\pgfpathcurveto{\pgfqpoint{6.284611in}{1.650168in}}{\pgfqpoint{6.274012in}{1.645777in}}{\pgfqpoint{6.266198in}{1.637964in}}%
\pgfpathcurveto{\pgfqpoint{6.258385in}{1.630150in}}{\pgfqpoint{6.253994in}{1.619551in}}{\pgfqpoint{6.253994in}{1.608501in}}%
\pgfpathcurveto{\pgfqpoint{6.253994in}{1.597451in}}{\pgfqpoint{6.258385in}{1.586852in}}{\pgfqpoint{6.266198in}{1.579038in}}%
\pgfpathcurveto{\pgfqpoint{6.274012in}{1.571225in}}{\pgfqpoint{6.284611in}{1.566834in}}{\pgfqpoint{6.295661in}{1.566834in}}%
\pgfpathclose%
\pgfusepath{stroke,fill}%
\end{pgfscope}%
\begin{pgfscope}%
\pgfpathrectangle{\pgfqpoint{0.481978in}{0.331635in}}{\pgfqpoint{9.300000in}{7.700000in}}%
\pgfusepath{clip}%
\pgfsetbuttcap%
\pgfsetroundjoin%
\definecolor{currentfill}{rgb}{1.000000,0.705882,0.509804}%
\pgfsetfillcolor{currentfill}%
\pgfsetlinewidth{0.481800pt}%
\definecolor{currentstroke}{rgb}{1.000000,1.000000,1.000000}%
\pgfsetstrokecolor{currentstroke}%
\pgfsetdash{}{0pt}%
\pgfpathmoveto{\pgfqpoint{3.458025in}{3.938973in}}%
\pgfpathcurveto{\pgfqpoint{3.469075in}{3.938973in}}{\pgfqpoint{3.479674in}{3.943364in}}{\pgfqpoint{3.487487in}{3.951177in}}%
\pgfpathcurveto{\pgfqpoint{3.495301in}{3.958991in}}{\pgfqpoint{3.499691in}{3.969590in}}{\pgfqpoint{3.499691in}{3.980640in}}%
\pgfpathcurveto{\pgfqpoint{3.499691in}{3.991690in}}{\pgfqpoint{3.495301in}{4.002289in}}{\pgfqpoint{3.487487in}{4.010103in}}%
\pgfpathcurveto{\pgfqpoint{3.479674in}{4.017916in}}{\pgfqpoint{3.469075in}{4.022307in}}{\pgfqpoint{3.458025in}{4.022307in}}%
\pgfpathcurveto{\pgfqpoint{3.446974in}{4.022307in}}{\pgfqpoint{3.436375in}{4.017916in}}{\pgfqpoint{3.428562in}{4.010103in}}%
\pgfpathcurveto{\pgfqpoint{3.420748in}{4.002289in}}{\pgfqpoint{3.416358in}{3.991690in}}{\pgfqpoint{3.416358in}{3.980640in}}%
\pgfpathcurveto{\pgfqpoint{3.416358in}{3.969590in}}{\pgfqpoint{3.420748in}{3.958991in}}{\pgfqpoint{3.428562in}{3.951177in}}%
\pgfpathcurveto{\pgfqpoint{3.436375in}{3.943364in}}{\pgfqpoint{3.446974in}{3.938973in}}{\pgfqpoint{3.458025in}{3.938973in}}%
\pgfpathclose%
\pgfusepath{stroke,fill}%
\end{pgfscope}%
\begin{pgfscope}%
\pgfpathrectangle{\pgfqpoint{0.481978in}{0.331635in}}{\pgfqpoint{9.300000in}{7.700000in}}%
\pgfusepath{clip}%
\pgfsetbuttcap%
\pgfsetroundjoin%
\definecolor{currentfill}{rgb}{1.000000,0.705882,0.509804}%
\pgfsetfillcolor{currentfill}%
\pgfsetlinewidth{0.481800pt}%
\definecolor{currentstroke}{rgb}{1.000000,1.000000,1.000000}%
\pgfsetstrokecolor{currentstroke}%
\pgfsetdash{}{0pt}%
\pgfpathmoveto{\pgfqpoint{4.270877in}{5.382117in}}%
\pgfpathcurveto{\pgfqpoint{4.281927in}{5.382117in}}{\pgfqpoint{4.292526in}{5.386508in}}{\pgfqpoint{4.300340in}{5.394321in}}%
\pgfpathcurveto{\pgfqpoint{4.308153in}{5.402135in}}{\pgfqpoint{4.312543in}{5.412734in}}{\pgfqpoint{4.312543in}{5.423784in}}%
\pgfpathcurveto{\pgfqpoint{4.312543in}{5.434834in}}{\pgfqpoint{4.308153in}{5.445433in}}{\pgfqpoint{4.300340in}{5.453247in}}%
\pgfpathcurveto{\pgfqpoint{4.292526in}{5.461060in}}{\pgfqpoint{4.281927in}{5.465451in}}{\pgfqpoint{4.270877in}{5.465451in}}%
\pgfpathcurveto{\pgfqpoint{4.259827in}{5.465451in}}{\pgfqpoint{4.249228in}{5.461060in}}{\pgfqpoint{4.241414in}{5.453247in}}%
\pgfpathcurveto{\pgfqpoint{4.233600in}{5.445433in}}{\pgfqpoint{4.229210in}{5.434834in}}{\pgfqpoint{4.229210in}{5.423784in}}%
\pgfpathcurveto{\pgfqpoint{4.229210in}{5.412734in}}{\pgfqpoint{4.233600in}{5.402135in}}{\pgfqpoint{4.241414in}{5.394321in}}%
\pgfpathcurveto{\pgfqpoint{4.249228in}{5.386508in}}{\pgfqpoint{4.259827in}{5.382117in}}{\pgfqpoint{4.270877in}{5.382117in}}%
\pgfpathclose%
\pgfusepath{stroke,fill}%
\end{pgfscope}%
\begin{pgfscope}%
\pgfpathrectangle{\pgfqpoint{0.481978in}{0.331635in}}{\pgfqpoint{9.300000in}{7.700000in}}%
\pgfusepath{clip}%
\pgfsetbuttcap%
\pgfsetroundjoin%
\definecolor{currentfill}{rgb}{1.000000,0.705882,0.509804}%
\pgfsetfillcolor{currentfill}%
\pgfsetlinewidth{0.481800pt}%
\definecolor{currentstroke}{rgb}{1.000000,1.000000,1.000000}%
\pgfsetstrokecolor{currentstroke}%
\pgfsetdash{}{0pt}%
\pgfpathmoveto{\pgfqpoint{5.749725in}{1.916657in}}%
\pgfpathcurveto{\pgfqpoint{5.760775in}{1.916657in}}{\pgfqpoint{5.771374in}{1.921047in}}{\pgfqpoint{5.779188in}{1.928861in}}%
\pgfpathcurveto{\pgfqpoint{5.787002in}{1.936675in}}{\pgfqpoint{5.791392in}{1.947274in}}{\pgfqpoint{5.791392in}{1.958324in}}%
\pgfpathcurveto{\pgfqpoint{5.791392in}{1.969374in}}{\pgfqpoint{5.787002in}{1.979973in}}{\pgfqpoint{5.779188in}{1.987787in}}%
\pgfpathcurveto{\pgfqpoint{5.771374in}{1.995600in}}{\pgfqpoint{5.760775in}{1.999990in}}{\pgfqpoint{5.749725in}{1.999990in}}%
\pgfpathcurveto{\pgfqpoint{5.738675in}{1.999990in}}{\pgfqpoint{5.728076in}{1.995600in}}{\pgfqpoint{5.720262in}{1.987787in}}%
\pgfpathcurveto{\pgfqpoint{5.712449in}{1.979973in}}{\pgfqpoint{5.708059in}{1.969374in}}{\pgfqpoint{5.708059in}{1.958324in}}%
\pgfpathcurveto{\pgfqpoint{5.708059in}{1.947274in}}{\pgfqpoint{5.712449in}{1.936675in}}{\pgfqpoint{5.720262in}{1.928861in}}%
\pgfpathcurveto{\pgfqpoint{5.728076in}{1.921047in}}{\pgfqpoint{5.738675in}{1.916657in}}{\pgfqpoint{5.749725in}{1.916657in}}%
\pgfpathclose%
\pgfusepath{stroke,fill}%
\end{pgfscope}%
\begin{pgfscope}%
\pgfpathrectangle{\pgfqpoint{0.481978in}{0.331635in}}{\pgfqpoint{9.300000in}{7.700000in}}%
\pgfusepath{clip}%
\pgfsetbuttcap%
\pgfsetroundjoin%
\definecolor{currentfill}{rgb}{1.000000,0.705882,0.509804}%
\pgfsetfillcolor{currentfill}%
\pgfsetlinewidth{0.481800pt}%
\definecolor{currentstroke}{rgb}{1.000000,1.000000,1.000000}%
\pgfsetstrokecolor{currentstroke}%
\pgfsetdash{}{0pt}%
\pgfpathmoveto{\pgfqpoint{2.458775in}{1.687705in}}%
\pgfpathcurveto{\pgfqpoint{2.469825in}{1.687705in}}{\pgfqpoint{2.480424in}{1.692095in}}{\pgfqpoint{2.488238in}{1.699909in}}%
\pgfpathcurveto{\pgfqpoint{2.496051in}{1.707723in}}{\pgfqpoint{2.500442in}{1.718322in}}{\pgfqpoint{2.500442in}{1.729372in}}%
\pgfpathcurveto{\pgfqpoint{2.500442in}{1.740422in}}{\pgfqpoint{2.496051in}{1.751021in}}{\pgfqpoint{2.488238in}{1.758835in}}%
\pgfpathcurveto{\pgfqpoint{2.480424in}{1.766648in}}{\pgfqpoint{2.469825in}{1.771038in}}{\pgfqpoint{2.458775in}{1.771038in}}%
\pgfpathcurveto{\pgfqpoint{2.447725in}{1.771038in}}{\pgfqpoint{2.437126in}{1.766648in}}{\pgfqpoint{2.429312in}{1.758835in}}%
\pgfpathcurveto{\pgfqpoint{2.421499in}{1.751021in}}{\pgfqpoint{2.417108in}{1.740422in}}{\pgfqpoint{2.417108in}{1.729372in}}%
\pgfpathcurveto{\pgfqpoint{2.417108in}{1.718322in}}{\pgfqpoint{2.421499in}{1.707723in}}{\pgfqpoint{2.429312in}{1.699909in}}%
\pgfpathcurveto{\pgfqpoint{2.437126in}{1.692095in}}{\pgfqpoint{2.447725in}{1.687705in}}{\pgfqpoint{2.458775in}{1.687705in}}%
\pgfpathclose%
\pgfusepath{stroke,fill}%
\end{pgfscope}%
\begin{pgfscope}%
\pgfpathrectangle{\pgfqpoint{0.481978in}{0.331635in}}{\pgfqpoint{9.300000in}{7.700000in}}%
\pgfusepath{clip}%
\pgfsetbuttcap%
\pgfsetroundjoin%
\definecolor{currentfill}{rgb}{1.000000,0.705882,0.509804}%
\pgfsetfillcolor{currentfill}%
\pgfsetlinewidth{0.481800pt}%
\definecolor{currentstroke}{rgb}{1.000000,1.000000,1.000000}%
\pgfsetstrokecolor{currentstroke}%
\pgfsetdash{}{0pt}%
\pgfpathmoveto{\pgfqpoint{4.692250in}{5.263851in}}%
\pgfpathcurveto{\pgfqpoint{4.703300in}{5.263851in}}{\pgfqpoint{4.713899in}{5.268242in}}{\pgfqpoint{4.721713in}{5.276055in}}%
\pgfpathcurveto{\pgfqpoint{4.729527in}{5.283869in}}{\pgfqpoint{4.733917in}{5.294468in}}{\pgfqpoint{4.733917in}{5.305518in}}%
\pgfpathcurveto{\pgfqpoint{4.733917in}{5.316568in}}{\pgfqpoint{4.729527in}{5.327167in}}{\pgfqpoint{4.721713in}{5.334981in}}%
\pgfpathcurveto{\pgfqpoint{4.713899in}{5.342795in}}{\pgfqpoint{4.703300in}{5.347185in}}{\pgfqpoint{4.692250in}{5.347185in}}%
\pgfpathcurveto{\pgfqpoint{4.681200in}{5.347185in}}{\pgfqpoint{4.670601in}{5.342795in}}{\pgfqpoint{4.662787in}{5.334981in}}%
\pgfpathcurveto{\pgfqpoint{4.654974in}{5.327167in}}{\pgfqpoint{4.650584in}{5.316568in}}{\pgfqpoint{4.650584in}{5.305518in}}%
\pgfpathcurveto{\pgfqpoint{4.650584in}{5.294468in}}{\pgfqpoint{4.654974in}{5.283869in}}{\pgfqpoint{4.662787in}{5.276055in}}%
\pgfpathcurveto{\pgfqpoint{4.670601in}{5.268242in}}{\pgfqpoint{4.681200in}{5.263851in}}{\pgfqpoint{4.692250in}{5.263851in}}%
\pgfpathclose%
\pgfusepath{stroke,fill}%
\end{pgfscope}%
\begin{pgfscope}%
\pgfpathrectangle{\pgfqpoint{0.481978in}{0.331635in}}{\pgfqpoint{9.300000in}{7.700000in}}%
\pgfusepath{clip}%
\pgfsetbuttcap%
\pgfsetroundjoin%
\definecolor{currentfill}{rgb}{1.000000,0.705882,0.509804}%
\pgfsetfillcolor{currentfill}%
\pgfsetlinewidth{0.481800pt}%
\definecolor{currentstroke}{rgb}{1.000000,1.000000,1.000000}%
\pgfsetstrokecolor{currentstroke}%
\pgfsetdash{}{0pt}%
\pgfpathmoveto{\pgfqpoint{7.916388in}{4.181299in}}%
\pgfpathcurveto{\pgfqpoint{7.927438in}{4.181299in}}{\pgfqpoint{7.938037in}{4.185689in}}{\pgfqpoint{7.945851in}{4.193503in}}%
\pgfpathcurveto{\pgfqpoint{7.953665in}{4.201317in}}{\pgfqpoint{7.958055in}{4.211916in}}{\pgfqpoint{7.958055in}{4.222966in}}%
\pgfpathcurveto{\pgfqpoint{7.958055in}{4.234016in}}{\pgfqpoint{7.953665in}{4.244615in}}{\pgfqpoint{7.945851in}{4.252429in}}%
\pgfpathcurveto{\pgfqpoint{7.938037in}{4.260242in}}{\pgfqpoint{7.927438in}{4.264632in}}{\pgfqpoint{7.916388in}{4.264632in}}%
\pgfpathcurveto{\pgfqpoint{7.905338in}{4.264632in}}{\pgfqpoint{7.894739in}{4.260242in}}{\pgfqpoint{7.886926in}{4.252429in}}%
\pgfpathcurveto{\pgfqpoint{7.879112in}{4.244615in}}{\pgfqpoint{7.874722in}{4.234016in}}{\pgfqpoint{7.874722in}{4.222966in}}%
\pgfpathcurveto{\pgfqpoint{7.874722in}{4.211916in}}{\pgfqpoint{7.879112in}{4.201317in}}{\pgfqpoint{7.886926in}{4.193503in}}%
\pgfpathcurveto{\pgfqpoint{7.894739in}{4.185689in}}{\pgfqpoint{7.905338in}{4.181299in}}{\pgfqpoint{7.916388in}{4.181299in}}%
\pgfpathclose%
\pgfusepath{stroke,fill}%
\end{pgfscope}%
\begin{pgfscope}%
\pgfpathrectangle{\pgfqpoint{0.481978in}{0.331635in}}{\pgfqpoint{9.300000in}{7.700000in}}%
\pgfusepath{clip}%
\pgfsetbuttcap%
\pgfsetroundjoin%
\definecolor{currentfill}{rgb}{1.000000,0.705882,0.509804}%
\pgfsetfillcolor{currentfill}%
\pgfsetlinewidth{0.481800pt}%
\definecolor{currentstroke}{rgb}{1.000000,1.000000,1.000000}%
\pgfsetstrokecolor{currentstroke}%
\pgfsetdash{}{0pt}%
\pgfpathmoveto{\pgfqpoint{4.752028in}{3.418430in}}%
\pgfpathcurveto{\pgfqpoint{4.763078in}{3.418430in}}{\pgfqpoint{4.773677in}{3.422821in}}{\pgfqpoint{4.781491in}{3.430634in}}%
\pgfpathcurveto{\pgfqpoint{4.789304in}{3.438448in}}{\pgfqpoint{4.793694in}{3.449047in}}{\pgfqpoint{4.793694in}{3.460097in}}%
\pgfpathcurveto{\pgfqpoint{4.793694in}{3.471147in}}{\pgfqpoint{4.789304in}{3.481746in}}{\pgfqpoint{4.781491in}{3.489560in}}%
\pgfpathcurveto{\pgfqpoint{4.773677in}{3.497374in}}{\pgfqpoint{4.763078in}{3.501764in}}{\pgfqpoint{4.752028in}{3.501764in}}%
\pgfpathcurveto{\pgfqpoint{4.740978in}{3.501764in}}{\pgfqpoint{4.730379in}{3.497374in}}{\pgfqpoint{4.722565in}{3.489560in}}%
\pgfpathcurveto{\pgfqpoint{4.714751in}{3.481746in}}{\pgfqpoint{4.710361in}{3.471147in}}{\pgfqpoint{4.710361in}{3.460097in}}%
\pgfpathcurveto{\pgfqpoint{4.710361in}{3.449047in}}{\pgfqpoint{4.714751in}{3.438448in}}{\pgfqpoint{4.722565in}{3.430634in}}%
\pgfpathcurveto{\pgfqpoint{4.730379in}{3.422821in}}{\pgfqpoint{4.740978in}{3.418430in}}{\pgfqpoint{4.752028in}{3.418430in}}%
\pgfpathclose%
\pgfusepath{stroke,fill}%
\end{pgfscope}%
\begin{pgfscope}%
\pgfpathrectangle{\pgfqpoint{0.481978in}{0.331635in}}{\pgfqpoint{9.300000in}{7.700000in}}%
\pgfusepath{clip}%
\pgfsetbuttcap%
\pgfsetroundjoin%
\definecolor{currentfill}{rgb}{1.000000,0.705882,0.509804}%
\pgfsetfillcolor{currentfill}%
\pgfsetlinewidth{0.481800pt}%
\definecolor{currentstroke}{rgb}{1.000000,1.000000,1.000000}%
\pgfsetstrokecolor{currentstroke}%
\pgfsetdash{}{0pt}%
\pgfpathmoveto{\pgfqpoint{8.514292in}{5.224514in}}%
\pgfpathcurveto{\pgfqpoint{8.525342in}{5.224514in}}{\pgfqpoint{8.535941in}{5.228904in}}{\pgfqpoint{8.543755in}{5.236717in}}%
\pgfpathcurveto{\pgfqpoint{8.551569in}{5.244531in}}{\pgfqpoint{8.555959in}{5.255130in}}{\pgfqpoint{8.555959in}{5.266180in}}%
\pgfpathcurveto{\pgfqpoint{8.555959in}{5.277230in}}{\pgfqpoint{8.551569in}{5.287829in}}{\pgfqpoint{8.543755in}{5.295643in}}%
\pgfpathcurveto{\pgfqpoint{8.535941in}{5.303457in}}{\pgfqpoint{8.525342in}{5.307847in}}{\pgfqpoint{8.514292in}{5.307847in}}%
\pgfpathcurveto{\pgfqpoint{8.503242in}{5.307847in}}{\pgfqpoint{8.492643in}{5.303457in}}{\pgfqpoint{8.484829in}{5.295643in}}%
\pgfpathcurveto{\pgfqpoint{8.477016in}{5.287829in}}{\pgfqpoint{8.472626in}{5.277230in}}{\pgfqpoint{8.472626in}{5.266180in}}%
\pgfpathcurveto{\pgfqpoint{8.472626in}{5.255130in}}{\pgfqpoint{8.477016in}{5.244531in}}{\pgfqpoint{8.484829in}{5.236717in}}%
\pgfpathcurveto{\pgfqpoint{8.492643in}{5.228904in}}{\pgfqpoint{8.503242in}{5.224514in}}{\pgfqpoint{8.514292in}{5.224514in}}%
\pgfpathclose%
\pgfusepath{stroke,fill}%
\end{pgfscope}%
\begin{pgfscope}%
\pgfpathrectangle{\pgfqpoint{0.481978in}{0.331635in}}{\pgfqpoint{9.300000in}{7.700000in}}%
\pgfusepath{clip}%
\pgfsetbuttcap%
\pgfsetroundjoin%
\definecolor{currentfill}{rgb}{1.000000,0.705882,0.509804}%
\pgfsetfillcolor{currentfill}%
\pgfsetlinewidth{0.481800pt}%
\definecolor{currentstroke}{rgb}{1.000000,1.000000,1.000000}%
\pgfsetstrokecolor{currentstroke}%
\pgfsetdash{}{0pt}%
\pgfpathmoveto{\pgfqpoint{3.988115in}{1.904248in}}%
\pgfpathcurveto{\pgfqpoint{3.999165in}{1.904248in}}{\pgfqpoint{4.009764in}{1.908639in}}{\pgfqpoint{4.017578in}{1.916452in}}%
\pgfpathcurveto{\pgfqpoint{4.025392in}{1.924266in}}{\pgfqpoint{4.029782in}{1.934865in}}{\pgfqpoint{4.029782in}{1.945915in}}%
\pgfpathcurveto{\pgfqpoint{4.029782in}{1.956965in}}{\pgfqpoint{4.025392in}{1.967564in}}{\pgfqpoint{4.017578in}{1.975378in}}%
\pgfpathcurveto{\pgfqpoint{4.009764in}{1.983191in}}{\pgfqpoint{3.999165in}{1.987582in}}{\pgfqpoint{3.988115in}{1.987582in}}%
\pgfpathcurveto{\pgfqpoint{3.977065in}{1.987582in}}{\pgfqpoint{3.966466in}{1.983191in}}{\pgfqpoint{3.958652in}{1.975378in}}%
\pgfpathcurveto{\pgfqpoint{3.950839in}{1.967564in}}{\pgfqpoint{3.946448in}{1.956965in}}{\pgfqpoint{3.946448in}{1.945915in}}%
\pgfpathcurveto{\pgfqpoint{3.946448in}{1.934865in}}{\pgfqpoint{3.950839in}{1.924266in}}{\pgfqpoint{3.958652in}{1.916452in}}%
\pgfpathcurveto{\pgfqpoint{3.966466in}{1.908639in}}{\pgfqpoint{3.977065in}{1.904248in}}{\pgfqpoint{3.988115in}{1.904248in}}%
\pgfpathclose%
\pgfusepath{stroke,fill}%
\end{pgfscope}%
\begin{pgfscope}%
\pgfpathrectangle{\pgfqpoint{0.481978in}{0.331635in}}{\pgfqpoint{9.300000in}{7.700000in}}%
\pgfusepath{clip}%
\pgfsetbuttcap%
\pgfsetroundjoin%
\definecolor{currentfill}{rgb}{1.000000,0.705882,0.509804}%
\pgfsetfillcolor{currentfill}%
\pgfsetlinewidth{0.481800pt}%
\definecolor{currentstroke}{rgb}{1.000000,1.000000,1.000000}%
\pgfsetstrokecolor{currentstroke}%
\pgfsetdash{}{0pt}%
\pgfpathmoveto{\pgfqpoint{3.637460in}{4.439078in}}%
\pgfpathcurveto{\pgfqpoint{3.648510in}{4.439078in}}{\pgfqpoint{3.659109in}{4.443469in}}{\pgfqpoint{3.666923in}{4.451282in}}%
\pgfpathcurveto{\pgfqpoint{3.674737in}{4.459096in}}{\pgfqpoint{3.679127in}{4.469695in}}{\pgfqpoint{3.679127in}{4.480745in}}%
\pgfpathcurveto{\pgfqpoint{3.679127in}{4.491795in}}{\pgfqpoint{3.674737in}{4.502394in}}{\pgfqpoint{3.666923in}{4.510208in}}%
\pgfpathcurveto{\pgfqpoint{3.659109in}{4.518022in}}{\pgfqpoint{3.648510in}{4.522412in}}{\pgfqpoint{3.637460in}{4.522412in}}%
\pgfpathcurveto{\pgfqpoint{3.626410in}{4.522412in}}{\pgfqpoint{3.615811in}{4.518022in}}{\pgfqpoint{3.607997in}{4.510208in}}%
\pgfpathcurveto{\pgfqpoint{3.600184in}{4.502394in}}{\pgfqpoint{3.595794in}{4.491795in}}{\pgfqpoint{3.595794in}{4.480745in}}%
\pgfpathcurveto{\pgfqpoint{3.595794in}{4.469695in}}{\pgfqpoint{3.600184in}{4.459096in}}{\pgfqpoint{3.607997in}{4.451282in}}%
\pgfpathcurveto{\pgfqpoint{3.615811in}{4.443469in}}{\pgfqpoint{3.626410in}{4.439078in}}{\pgfqpoint{3.637460in}{4.439078in}}%
\pgfpathclose%
\pgfusepath{stroke,fill}%
\end{pgfscope}%
\begin{pgfscope}%
\pgfpathrectangle{\pgfqpoint{0.481978in}{0.331635in}}{\pgfqpoint{9.300000in}{7.700000in}}%
\pgfusepath{clip}%
\pgfsetbuttcap%
\pgfsetroundjoin%
\definecolor{currentfill}{rgb}{1.000000,0.705882,0.509804}%
\pgfsetfillcolor{currentfill}%
\pgfsetlinewidth{0.481800pt}%
\definecolor{currentstroke}{rgb}{1.000000,1.000000,1.000000}%
\pgfsetstrokecolor{currentstroke}%
\pgfsetdash{}{0pt}%
\pgfpathmoveto{\pgfqpoint{3.666577in}{4.453388in}}%
\pgfpathcurveto{\pgfqpoint{3.677627in}{4.453388in}}{\pgfqpoint{3.688226in}{4.457779in}}{\pgfqpoint{3.696040in}{4.465592in}}%
\pgfpathcurveto{\pgfqpoint{3.703854in}{4.473406in}}{\pgfqpoint{3.708244in}{4.484005in}}{\pgfqpoint{3.708244in}{4.495055in}}%
\pgfpathcurveto{\pgfqpoint{3.708244in}{4.506105in}}{\pgfqpoint{3.703854in}{4.516704in}}{\pgfqpoint{3.696040in}{4.524518in}}%
\pgfpathcurveto{\pgfqpoint{3.688226in}{4.532331in}}{\pgfqpoint{3.677627in}{4.536722in}}{\pgfqpoint{3.666577in}{4.536722in}}%
\pgfpathcurveto{\pgfqpoint{3.655527in}{4.536722in}}{\pgfqpoint{3.644928in}{4.532331in}}{\pgfqpoint{3.637114in}{4.524518in}}%
\pgfpathcurveto{\pgfqpoint{3.629301in}{4.516704in}}{\pgfqpoint{3.624910in}{4.506105in}}{\pgfqpoint{3.624910in}{4.495055in}}%
\pgfpathcurveto{\pgfqpoint{3.624910in}{4.484005in}}{\pgfqpoint{3.629301in}{4.473406in}}{\pgfqpoint{3.637114in}{4.465592in}}%
\pgfpathcurveto{\pgfqpoint{3.644928in}{4.457779in}}{\pgfqpoint{3.655527in}{4.453388in}}{\pgfqpoint{3.666577in}{4.453388in}}%
\pgfpathclose%
\pgfusepath{stroke,fill}%
\end{pgfscope}%
\begin{pgfscope}%
\pgfpathrectangle{\pgfqpoint{0.481978in}{0.331635in}}{\pgfqpoint{9.300000in}{7.700000in}}%
\pgfusepath{clip}%
\pgfsetbuttcap%
\pgfsetroundjoin%
\definecolor{currentfill}{rgb}{1.000000,0.705882,0.509804}%
\pgfsetfillcolor{currentfill}%
\pgfsetlinewidth{0.481800pt}%
\definecolor{currentstroke}{rgb}{1.000000,1.000000,1.000000}%
\pgfsetstrokecolor{currentstroke}%
\pgfsetdash{}{0pt}%
\pgfpathmoveto{\pgfqpoint{3.344754in}{3.212776in}}%
\pgfpathcurveto{\pgfqpoint{3.355805in}{3.212776in}}{\pgfqpoint{3.366404in}{3.217166in}}{\pgfqpoint{3.374217in}{3.224980in}}%
\pgfpathcurveto{\pgfqpoint{3.382031in}{3.232794in}}{\pgfqpoint{3.386421in}{3.243393in}}{\pgfqpoint{3.386421in}{3.254443in}}%
\pgfpathcurveto{\pgfqpoint{3.386421in}{3.265493in}}{\pgfqpoint{3.382031in}{3.276092in}}{\pgfqpoint{3.374217in}{3.283905in}}%
\pgfpathcurveto{\pgfqpoint{3.366404in}{3.291719in}}{\pgfqpoint{3.355805in}{3.296109in}}{\pgfqpoint{3.344754in}{3.296109in}}%
\pgfpathcurveto{\pgfqpoint{3.333704in}{3.296109in}}{\pgfqpoint{3.323105in}{3.291719in}}{\pgfqpoint{3.315292in}{3.283905in}}%
\pgfpathcurveto{\pgfqpoint{3.307478in}{3.276092in}}{\pgfqpoint{3.303088in}{3.265493in}}{\pgfqpoint{3.303088in}{3.254443in}}%
\pgfpathcurveto{\pgfqpoint{3.303088in}{3.243393in}}{\pgfqpoint{3.307478in}{3.232794in}}{\pgfqpoint{3.315292in}{3.224980in}}%
\pgfpathcurveto{\pgfqpoint{3.323105in}{3.217166in}}{\pgfqpoint{3.333704in}{3.212776in}}{\pgfqpoint{3.344754in}{3.212776in}}%
\pgfpathclose%
\pgfusepath{stroke,fill}%
\end{pgfscope}%
\begin{pgfscope}%
\pgfpathrectangle{\pgfqpoint{0.481978in}{0.331635in}}{\pgfqpoint{9.300000in}{7.700000in}}%
\pgfusepath{clip}%
\pgfsetbuttcap%
\pgfsetroundjoin%
\definecolor{currentfill}{rgb}{1.000000,0.705882,0.509804}%
\pgfsetfillcolor{currentfill}%
\pgfsetlinewidth{0.481800pt}%
\definecolor{currentstroke}{rgb}{1.000000,1.000000,1.000000}%
\pgfsetstrokecolor{currentstroke}%
\pgfsetdash{}{0pt}%
\pgfpathmoveto{\pgfqpoint{7.394228in}{2.316863in}}%
\pgfpathcurveto{\pgfqpoint{7.405278in}{2.316863in}}{\pgfqpoint{7.415877in}{2.321253in}}{\pgfqpoint{7.423691in}{2.329066in}}%
\pgfpathcurveto{\pgfqpoint{7.431504in}{2.336880in}}{\pgfqpoint{7.435894in}{2.347479in}}{\pgfqpoint{7.435894in}{2.358529in}}%
\pgfpathcurveto{\pgfqpoint{7.435894in}{2.369579in}}{\pgfqpoint{7.431504in}{2.380178in}}{\pgfqpoint{7.423691in}{2.387992in}}%
\pgfpathcurveto{\pgfqpoint{7.415877in}{2.395806in}}{\pgfqpoint{7.405278in}{2.400196in}}{\pgfqpoint{7.394228in}{2.400196in}}%
\pgfpathcurveto{\pgfqpoint{7.383178in}{2.400196in}}{\pgfqpoint{7.372579in}{2.395806in}}{\pgfqpoint{7.364765in}{2.387992in}}%
\pgfpathcurveto{\pgfqpoint{7.356951in}{2.380178in}}{\pgfqpoint{7.352561in}{2.369579in}}{\pgfqpoint{7.352561in}{2.358529in}}%
\pgfpathcurveto{\pgfqpoint{7.352561in}{2.347479in}}{\pgfqpoint{7.356951in}{2.336880in}}{\pgfqpoint{7.364765in}{2.329066in}}%
\pgfpathcurveto{\pgfqpoint{7.372579in}{2.321253in}}{\pgfqpoint{7.383178in}{2.316863in}}{\pgfqpoint{7.394228in}{2.316863in}}%
\pgfpathclose%
\pgfusepath{stroke,fill}%
\end{pgfscope}%
\begin{pgfscope}%
\pgfpathrectangle{\pgfqpoint{0.481978in}{0.331635in}}{\pgfqpoint{9.300000in}{7.700000in}}%
\pgfusepath{clip}%
\pgfsetbuttcap%
\pgfsetroundjoin%
\definecolor{currentfill}{rgb}{1.000000,0.705882,0.509804}%
\pgfsetfillcolor{currentfill}%
\pgfsetlinewidth{0.481800pt}%
\definecolor{currentstroke}{rgb}{1.000000,1.000000,1.000000}%
\pgfsetstrokecolor{currentstroke}%
\pgfsetdash{}{0pt}%
\pgfpathmoveto{\pgfqpoint{2.694400in}{2.093625in}}%
\pgfpathcurveto{\pgfqpoint{2.705451in}{2.093625in}}{\pgfqpoint{2.716050in}{2.098015in}}{\pgfqpoint{2.723863in}{2.105829in}}%
\pgfpathcurveto{\pgfqpoint{2.731677in}{2.113642in}}{\pgfqpoint{2.736067in}{2.124241in}}{\pgfqpoint{2.736067in}{2.135292in}}%
\pgfpathcurveto{\pgfqpoint{2.736067in}{2.146342in}}{\pgfqpoint{2.731677in}{2.156941in}}{\pgfqpoint{2.723863in}{2.164754in}}%
\pgfpathcurveto{\pgfqpoint{2.716050in}{2.172568in}}{\pgfqpoint{2.705451in}{2.176958in}}{\pgfqpoint{2.694400in}{2.176958in}}%
\pgfpathcurveto{\pgfqpoint{2.683350in}{2.176958in}}{\pgfqpoint{2.672751in}{2.172568in}}{\pgfqpoint{2.664938in}{2.164754in}}%
\pgfpathcurveto{\pgfqpoint{2.657124in}{2.156941in}}{\pgfqpoint{2.652734in}{2.146342in}}{\pgfqpoint{2.652734in}{2.135292in}}%
\pgfpathcurveto{\pgfqpoint{2.652734in}{2.124241in}}{\pgfqpoint{2.657124in}{2.113642in}}{\pgfqpoint{2.664938in}{2.105829in}}%
\pgfpathcurveto{\pgfqpoint{2.672751in}{2.098015in}}{\pgfqpoint{2.683350in}{2.093625in}}{\pgfqpoint{2.694400in}{2.093625in}}%
\pgfpathclose%
\pgfusepath{stroke,fill}%
\end{pgfscope}%
\begin{pgfscope}%
\pgfpathrectangle{\pgfqpoint{0.481978in}{0.331635in}}{\pgfqpoint{9.300000in}{7.700000in}}%
\pgfusepath{clip}%
\pgfsetbuttcap%
\pgfsetroundjoin%
\definecolor{currentfill}{rgb}{1.000000,0.705882,0.509804}%
\pgfsetfillcolor{currentfill}%
\pgfsetlinewidth{0.481800pt}%
\definecolor{currentstroke}{rgb}{1.000000,1.000000,1.000000}%
\pgfsetstrokecolor{currentstroke}%
\pgfsetdash{}{0pt}%
\pgfpathmoveto{\pgfqpoint{5.916550in}{1.501443in}}%
\pgfpathcurveto{\pgfqpoint{5.927600in}{1.501443in}}{\pgfqpoint{5.938199in}{1.505833in}}{\pgfqpoint{5.946013in}{1.513647in}}%
\pgfpathcurveto{\pgfqpoint{5.953826in}{1.521460in}}{\pgfqpoint{5.958217in}{1.532059in}}{\pgfqpoint{5.958217in}{1.543110in}}%
\pgfpathcurveto{\pgfqpoint{5.958217in}{1.554160in}}{\pgfqpoint{5.953826in}{1.564759in}}{\pgfqpoint{5.946013in}{1.572572in}}%
\pgfpathcurveto{\pgfqpoint{5.938199in}{1.580386in}}{\pgfqpoint{5.927600in}{1.584776in}}{\pgfqpoint{5.916550in}{1.584776in}}%
\pgfpathcurveto{\pgfqpoint{5.905500in}{1.584776in}}{\pgfqpoint{5.894901in}{1.580386in}}{\pgfqpoint{5.887087in}{1.572572in}}%
\pgfpathcurveto{\pgfqpoint{5.879274in}{1.564759in}}{\pgfqpoint{5.874883in}{1.554160in}}{\pgfqpoint{5.874883in}{1.543110in}}%
\pgfpathcurveto{\pgfqpoint{5.874883in}{1.532059in}}{\pgfqpoint{5.879274in}{1.521460in}}{\pgfqpoint{5.887087in}{1.513647in}}%
\pgfpathcurveto{\pgfqpoint{5.894901in}{1.505833in}}{\pgfqpoint{5.905500in}{1.501443in}}{\pgfqpoint{5.916550in}{1.501443in}}%
\pgfpathclose%
\pgfusepath{stroke,fill}%
\end{pgfscope}%
\begin{pgfscope}%
\pgfpathrectangle{\pgfqpoint{0.481978in}{0.331635in}}{\pgfqpoint{9.300000in}{7.700000in}}%
\pgfusepath{clip}%
\pgfsetbuttcap%
\pgfsetroundjoin%
\definecolor{currentfill}{rgb}{1.000000,0.705882,0.509804}%
\pgfsetfillcolor{currentfill}%
\pgfsetlinewidth{0.481800pt}%
\definecolor{currentstroke}{rgb}{1.000000,1.000000,1.000000}%
\pgfsetstrokecolor{currentstroke}%
\pgfsetdash{}{0pt}%
\pgfpathmoveto{\pgfqpoint{6.083860in}{2.927174in}}%
\pgfpathcurveto{\pgfqpoint{6.094910in}{2.927174in}}{\pgfqpoint{6.105509in}{2.931564in}}{\pgfqpoint{6.113322in}{2.939378in}}%
\pgfpathcurveto{\pgfqpoint{6.121136in}{2.947192in}}{\pgfqpoint{6.125526in}{2.957791in}}{\pgfqpoint{6.125526in}{2.968841in}}%
\pgfpathcurveto{\pgfqpoint{6.125526in}{2.979891in}}{\pgfqpoint{6.121136in}{2.990490in}}{\pgfqpoint{6.113322in}{2.998304in}}%
\pgfpathcurveto{\pgfqpoint{6.105509in}{3.006117in}}{\pgfqpoint{6.094910in}{3.010507in}}{\pgfqpoint{6.083860in}{3.010507in}}%
\pgfpathcurveto{\pgfqpoint{6.072810in}{3.010507in}}{\pgfqpoint{6.062211in}{3.006117in}}{\pgfqpoint{6.054397in}{2.998304in}}%
\pgfpathcurveto{\pgfqpoint{6.046583in}{2.990490in}}{\pgfqpoint{6.042193in}{2.979891in}}{\pgfqpoint{6.042193in}{2.968841in}}%
\pgfpathcurveto{\pgfqpoint{6.042193in}{2.957791in}}{\pgfqpoint{6.046583in}{2.947192in}}{\pgfqpoint{6.054397in}{2.939378in}}%
\pgfpathcurveto{\pgfqpoint{6.062211in}{2.931564in}}{\pgfqpoint{6.072810in}{2.927174in}}{\pgfqpoint{6.083860in}{2.927174in}}%
\pgfpathclose%
\pgfusepath{stroke,fill}%
\end{pgfscope}%
\begin{pgfscope}%
\pgfpathrectangle{\pgfqpoint{0.481978in}{0.331635in}}{\pgfqpoint{9.300000in}{7.700000in}}%
\pgfusepath{clip}%
\pgfsetbuttcap%
\pgfsetroundjoin%
\definecolor{currentfill}{rgb}{1.000000,0.705882,0.509804}%
\pgfsetfillcolor{currentfill}%
\pgfsetlinewidth{0.481800pt}%
\definecolor{currentstroke}{rgb}{1.000000,1.000000,1.000000}%
\pgfsetstrokecolor{currentstroke}%
\pgfsetdash{}{0pt}%
\pgfpathmoveto{\pgfqpoint{4.639314in}{3.250673in}}%
\pgfpathcurveto{\pgfqpoint{4.650364in}{3.250673in}}{\pgfqpoint{4.660963in}{3.255063in}}{\pgfqpoint{4.668776in}{3.262877in}}%
\pgfpathcurveto{\pgfqpoint{4.676590in}{3.270690in}}{\pgfqpoint{4.680980in}{3.281289in}}{\pgfqpoint{4.680980in}{3.292339in}}%
\pgfpathcurveto{\pgfqpoint{4.680980in}{3.303390in}}{\pgfqpoint{4.676590in}{3.313989in}}{\pgfqpoint{4.668776in}{3.321802in}}%
\pgfpathcurveto{\pgfqpoint{4.660963in}{3.329616in}}{\pgfqpoint{4.650364in}{3.334006in}}{\pgfqpoint{4.639314in}{3.334006in}}%
\pgfpathcurveto{\pgfqpoint{4.628264in}{3.334006in}}{\pgfqpoint{4.617665in}{3.329616in}}{\pgfqpoint{4.609851in}{3.321802in}}%
\pgfpathcurveto{\pgfqpoint{4.602037in}{3.313989in}}{\pgfqpoint{4.597647in}{3.303390in}}{\pgfqpoint{4.597647in}{3.292339in}}%
\pgfpathcurveto{\pgfqpoint{4.597647in}{3.281289in}}{\pgfqpoint{4.602037in}{3.270690in}}{\pgfqpoint{4.609851in}{3.262877in}}%
\pgfpathcurveto{\pgfqpoint{4.617665in}{3.255063in}}{\pgfqpoint{4.628264in}{3.250673in}}{\pgfqpoint{4.639314in}{3.250673in}}%
\pgfpathclose%
\pgfusepath{stroke,fill}%
\end{pgfscope}%
\begin{pgfscope}%
\pgfpathrectangle{\pgfqpoint{0.481978in}{0.331635in}}{\pgfqpoint{9.300000in}{7.700000in}}%
\pgfusepath{clip}%
\pgfsetbuttcap%
\pgfsetroundjoin%
\definecolor{currentfill}{rgb}{1.000000,0.705882,0.509804}%
\pgfsetfillcolor{currentfill}%
\pgfsetlinewidth{0.481800pt}%
\definecolor{currentstroke}{rgb}{1.000000,1.000000,1.000000}%
\pgfsetstrokecolor{currentstroke}%
\pgfsetdash{}{0pt}%
\pgfpathmoveto{\pgfqpoint{8.886711in}{5.704878in}}%
\pgfpathcurveto{\pgfqpoint{8.897762in}{5.704878in}}{\pgfqpoint{8.908361in}{5.709268in}}{\pgfqpoint{8.916174in}{5.717082in}}%
\pgfpathcurveto{\pgfqpoint{8.923988in}{5.724895in}}{\pgfqpoint{8.928378in}{5.735494in}}{\pgfqpoint{8.928378in}{5.746544in}}%
\pgfpathcurveto{\pgfqpoint{8.928378in}{5.757595in}}{\pgfqpoint{8.923988in}{5.768194in}}{\pgfqpoint{8.916174in}{5.776007in}}%
\pgfpathcurveto{\pgfqpoint{8.908361in}{5.783821in}}{\pgfqpoint{8.897762in}{5.788211in}}{\pgfqpoint{8.886711in}{5.788211in}}%
\pgfpathcurveto{\pgfqpoint{8.875661in}{5.788211in}}{\pgfqpoint{8.865062in}{5.783821in}}{\pgfqpoint{8.857249in}{5.776007in}}%
\pgfpathcurveto{\pgfqpoint{8.849435in}{5.768194in}}{\pgfqpoint{8.845045in}{5.757595in}}{\pgfqpoint{8.845045in}{5.746544in}}%
\pgfpathcurveto{\pgfqpoint{8.845045in}{5.735494in}}{\pgfqpoint{8.849435in}{5.724895in}}{\pgfqpoint{8.857249in}{5.717082in}}%
\pgfpathcurveto{\pgfqpoint{8.865062in}{5.709268in}}{\pgfqpoint{8.875661in}{5.704878in}}{\pgfqpoint{8.886711in}{5.704878in}}%
\pgfpathclose%
\pgfusepath{stroke,fill}%
\end{pgfscope}%
\begin{pgfscope}%
\pgfpathrectangle{\pgfqpoint{0.481978in}{0.331635in}}{\pgfqpoint{9.300000in}{7.700000in}}%
\pgfusepath{clip}%
\pgfsetbuttcap%
\pgfsetroundjoin%
\definecolor{currentfill}{rgb}{1.000000,0.705882,0.509804}%
\pgfsetfillcolor{currentfill}%
\pgfsetlinewidth{0.481800pt}%
\definecolor{currentstroke}{rgb}{1.000000,1.000000,1.000000}%
\pgfsetstrokecolor{currentstroke}%
\pgfsetdash{}{0pt}%
\pgfpathmoveto{\pgfqpoint{3.960687in}{2.268367in}}%
\pgfpathcurveto{\pgfqpoint{3.971737in}{2.268367in}}{\pgfqpoint{3.982336in}{2.272757in}}{\pgfqpoint{3.990150in}{2.280571in}}%
\pgfpathcurveto{\pgfqpoint{3.997963in}{2.288384in}}{\pgfqpoint{4.002353in}{2.298983in}}{\pgfqpoint{4.002353in}{2.310033in}}%
\pgfpathcurveto{\pgfqpoint{4.002353in}{2.321084in}}{\pgfqpoint{3.997963in}{2.331683in}}{\pgfqpoint{3.990150in}{2.339496in}}%
\pgfpathcurveto{\pgfqpoint{3.982336in}{2.347310in}}{\pgfqpoint{3.971737in}{2.351700in}}{\pgfqpoint{3.960687in}{2.351700in}}%
\pgfpathcurveto{\pgfqpoint{3.949637in}{2.351700in}}{\pgfqpoint{3.939038in}{2.347310in}}{\pgfqpoint{3.931224in}{2.339496in}}%
\pgfpathcurveto{\pgfqpoint{3.923410in}{2.331683in}}{\pgfqpoint{3.919020in}{2.321084in}}{\pgfqpoint{3.919020in}{2.310033in}}%
\pgfpathcurveto{\pgfqpoint{3.919020in}{2.298983in}}{\pgfqpoint{3.923410in}{2.288384in}}{\pgfqpoint{3.931224in}{2.280571in}}%
\pgfpathcurveto{\pgfqpoint{3.939038in}{2.272757in}}{\pgfqpoint{3.949637in}{2.268367in}}{\pgfqpoint{3.960687in}{2.268367in}}%
\pgfpathclose%
\pgfusepath{stroke,fill}%
\end{pgfscope}%
\begin{pgfscope}%
\pgfpathrectangle{\pgfqpoint{0.481978in}{0.331635in}}{\pgfqpoint{9.300000in}{7.700000in}}%
\pgfusepath{clip}%
\pgfsetbuttcap%
\pgfsetroundjoin%
\definecolor{currentfill}{rgb}{1.000000,0.705882,0.509804}%
\pgfsetfillcolor{currentfill}%
\pgfsetlinewidth{0.481800pt}%
\definecolor{currentstroke}{rgb}{1.000000,1.000000,1.000000}%
\pgfsetstrokecolor{currentstroke}%
\pgfsetdash{}{0pt}%
\pgfpathmoveto{\pgfqpoint{3.907332in}{1.719292in}}%
\pgfpathcurveto{\pgfqpoint{3.918382in}{1.719292in}}{\pgfqpoint{3.928981in}{1.723682in}}{\pgfqpoint{3.936795in}{1.731496in}}%
\pgfpathcurveto{\pgfqpoint{3.944608in}{1.739309in}}{\pgfqpoint{3.948999in}{1.749908in}}{\pgfqpoint{3.948999in}{1.760958in}}%
\pgfpathcurveto{\pgfqpoint{3.948999in}{1.772009in}}{\pgfqpoint{3.944608in}{1.782608in}}{\pgfqpoint{3.936795in}{1.790421in}}%
\pgfpathcurveto{\pgfqpoint{3.928981in}{1.798235in}}{\pgfqpoint{3.918382in}{1.802625in}}{\pgfqpoint{3.907332in}{1.802625in}}%
\pgfpathcurveto{\pgfqpoint{3.896282in}{1.802625in}}{\pgfqpoint{3.885683in}{1.798235in}}{\pgfqpoint{3.877869in}{1.790421in}}%
\pgfpathcurveto{\pgfqpoint{3.870055in}{1.782608in}}{\pgfqpoint{3.865665in}{1.772009in}}{\pgfqpoint{3.865665in}{1.760958in}}%
\pgfpathcurveto{\pgfqpoint{3.865665in}{1.749908in}}{\pgfqpoint{3.870055in}{1.739309in}}{\pgfqpoint{3.877869in}{1.731496in}}%
\pgfpathcurveto{\pgfqpoint{3.885683in}{1.723682in}}{\pgfqpoint{3.896282in}{1.719292in}}{\pgfqpoint{3.907332in}{1.719292in}}%
\pgfpathclose%
\pgfusepath{stroke,fill}%
\end{pgfscope}%
\begin{pgfscope}%
\pgfpathrectangle{\pgfqpoint{0.481978in}{0.331635in}}{\pgfqpoint{9.300000in}{7.700000in}}%
\pgfusepath{clip}%
\pgfsetbuttcap%
\pgfsetroundjoin%
\definecolor{currentfill}{rgb}{1.000000,0.705882,0.509804}%
\pgfsetfillcolor{currentfill}%
\pgfsetlinewidth{0.481800pt}%
\definecolor{currentstroke}{rgb}{1.000000,1.000000,1.000000}%
\pgfsetstrokecolor{currentstroke}%
\pgfsetdash{}{0pt}%
\pgfpathmoveto{\pgfqpoint{4.745564in}{5.335530in}}%
\pgfpathcurveto{\pgfqpoint{4.756614in}{5.335530in}}{\pgfqpoint{4.767213in}{5.339920in}}{\pgfqpoint{4.775026in}{5.347734in}}%
\pgfpathcurveto{\pgfqpoint{4.782840in}{5.355548in}}{\pgfqpoint{4.787230in}{5.366147in}}{\pgfqpoint{4.787230in}{5.377197in}}%
\pgfpathcurveto{\pgfqpoint{4.787230in}{5.388247in}}{\pgfqpoint{4.782840in}{5.398846in}}{\pgfqpoint{4.775026in}{5.406660in}}%
\pgfpathcurveto{\pgfqpoint{4.767213in}{5.414473in}}{\pgfqpoint{4.756614in}{5.418864in}}{\pgfqpoint{4.745564in}{5.418864in}}%
\pgfpathcurveto{\pgfqpoint{4.734513in}{5.418864in}}{\pgfqpoint{4.723914in}{5.414473in}}{\pgfqpoint{4.716101in}{5.406660in}}%
\pgfpathcurveto{\pgfqpoint{4.708287in}{5.398846in}}{\pgfqpoint{4.703897in}{5.388247in}}{\pgfqpoint{4.703897in}{5.377197in}}%
\pgfpathcurveto{\pgfqpoint{4.703897in}{5.366147in}}{\pgfqpoint{4.708287in}{5.355548in}}{\pgfqpoint{4.716101in}{5.347734in}}%
\pgfpathcurveto{\pgfqpoint{4.723914in}{5.339920in}}{\pgfqpoint{4.734513in}{5.335530in}}{\pgfqpoint{4.745564in}{5.335530in}}%
\pgfpathclose%
\pgfusepath{stroke,fill}%
\end{pgfscope}%
\begin{pgfscope}%
\pgfpathrectangle{\pgfqpoint{0.481978in}{0.331635in}}{\pgfqpoint{9.300000in}{7.700000in}}%
\pgfusepath{clip}%
\pgfsetbuttcap%
\pgfsetroundjoin%
\definecolor{currentfill}{rgb}{1.000000,0.705882,0.509804}%
\pgfsetfillcolor{currentfill}%
\pgfsetlinewidth{0.481800pt}%
\definecolor{currentstroke}{rgb}{1.000000,1.000000,1.000000}%
\pgfsetstrokecolor{currentstroke}%
\pgfsetdash{}{0pt}%
\pgfpathmoveto{\pgfqpoint{4.104700in}{5.418936in}}%
\pgfpathcurveto{\pgfqpoint{4.115750in}{5.418936in}}{\pgfqpoint{4.126350in}{5.423326in}}{\pgfqpoint{4.134163in}{5.431139in}}%
\pgfpathcurveto{\pgfqpoint{4.141977in}{5.438953in}}{\pgfqpoint{4.146367in}{5.449552in}}{\pgfqpoint{4.146367in}{5.460602in}}%
\pgfpathcurveto{\pgfqpoint{4.146367in}{5.471652in}}{\pgfqpoint{4.141977in}{5.482251in}}{\pgfqpoint{4.134163in}{5.490065in}}%
\pgfpathcurveto{\pgfqpoint{4.126350in}{5.497879in}}{\pgfqpoint{4.115750in}{5.502269in}}{\pgfqpoint{4.104700in}{5.502269in}}%
\pgfpathcurveto{\pgfqpoint{4.093650in}{5.502269in}}{\pgfqpoint{4.083051in}{5.497879in}}{\pgfqpoint{4.075238in}{5.490065in}}%
\pgfpathcurveto{\pgfqpoint{4.067424in}{5.482251in}}{\pgfqpoint{4.063034in}{5.471652in}}{\pgfqpoint{4.063034in}{5.460602in}}%
\pgfpathcurveto{\pgfqpoint{4.063034in}{5.449552in}}{\pgfqpoint{4.067424in}{5.438953in}}{\pgfqpoint{4.075238in}{5.431139in}}%
\pgfpathcurveto{\pgfqpoint{4.083051in}{5.423326in}}{\pgfqpoint{4.093650in}{5.418936in}}{\pgfqpoint{4.104700in}{5.418936in}}%
\pgfpathclose%
\pgfusepath{stroke,fill}%
\end{pgfscope}%
\begin{pgfscope}%
\pgfpathrectangle{\pgfqpoint{0.481978in}{0.331635in}}{\pgfqpoint{9.300000in}{7.700000in}}%
\pgfusepath{clip}%
\pgfsetbuttcap%
\pgfsetroundjoin%
\definecolor{currentfill}{rgb}{1.000000,0.705882,0.509804}%
\pgfsetfillcolor{currentfill}%
\pgfsetlinewidth{0.481800pt}%
\definecolor{currentstroke}{rgb}{1.000000,1.000000,1.000000}%
\pgfsetstrokecolor{currentstroke}%
\pgfsetdash{}{0pt}%
\pgfpathmoveto{\pgfqpoint{8.529282in}{4.532862in}}%
\pgfpathcurveto{\pgfqpoint{8.540332in}{4.532862in}}{\pgfqpoint{8.550931in}{4.537253in}}{\pgfqpoint{8.558745in}{4.545066in}}%
\pgfpathcurveto{\pgfqpoint{8.566558in}{4.552880in}}{\pgfqpoint{8.570948in}{4.563479in}}{\pgfqpoint{8.570948in}{4.574529in}}%
\pgfpathcurveto{\pgfqpoint{8.570948in}{4.585579in}}{\pgfqpoint{8.566558in}{4.596178in}}{\pgfqpoint{8.558745in}{4.603992in}}%
\pgfpathcurveto{\pgfqpoint{8.550931in}{4.611806in}}{\pgfqpoint{8.540332in}{4.616196in}}{\pgfqpoint{8.529282in}{4.616196in}}%
\pgfpathcurveto{\pgfqpoint{8.518232in}{4.616196in}}{\pgfqpoint{8.507633in}{4.611806in}}{\pgfqpoint{8.499819in}{4.603992in}}%
\pgfpathcurveto{\pgfqpoint{8.492005in}{4.596178in}}{\pgfqpoint{8.487615in}{4.585579in}}{\pgfqpoint{8.487615in}{4.574529in}}%
\pgfpathcurveto{\pgfqpoint{8.487615in}{4.563479in}}{\pgfqpoint{8.492005in}{4.552880in}}{\pgfqpoint{8.499819in}{4.545066in}}%
\pgfpathcurveto{\pgfqpoint{8.507633in}{4.537253in}}{\pgfqpoint{8.518232in}{4.532862in}}{\pgfqpoint{8.529282in}{4.532862in}}%
\pgfpathclose%
\pgfusepath{stroke,fill}%
\end{pgfscope}%
\begin{pgfscope}%
\pgfpathrectangle{\pgfqpoint{0.481978in}{0.331635in}}{\pgfqpoint{9.300000in}{7.700000in}}%
\pgfusepath{clip}%
\pgfsetbuttcap%
\pgfsetroundjoin%
\definecolor{currentfill}{rgb}{1.000000,0.705882,0.509804}%
\pgfsetfillcolor{currentfill}%
\pgfsetlinewidth{0.481800pt}%
\definecolor{currentstroke}{rgb}{1.000000,1.000000,1.000000}%
\pgfsetstrokecolor{currentstroke}%
\pgfsetdash{}{0pt}%
\pgfpathmoveto{\pgfqpoint{7.907970in}{3.973013in}}%
\pgfpathcurveto{\pgfqpoint{7.919021in}{3.973013in}}{\pgfqpoint{7.929620in}{3.977403in}}{\pgfqpoint{7.937433in}{3.985217in}}%
\pgfpathcurveto{\pgfqpoint{7.945247in}{3.993030in}}{\pgfqpoint{7.949637in}{4.003629in}}{\pgfqpoint{7.949637in}{4.014679in}}%
\pgfpathcurveto{\pgfqpoint{7.949637in}{4.025730in}}{\pgfqpoint{7.945247in}{4.036329in}}{\pgfqpoint{7.937433in}{4.044142in}}%
\pgfpathcurveto{\pgfqpoint{7.929620in}{4.051956in}}{\pgfqpoint{7.919021in}{4.056346in}}{\pgfqpoint{7.907970in}{4.056346in}}%
\pgfpathcurveto{\pgfqpoint{7.896920in}{4.056346in}}{\pgfqpoint{7.886321in}{4.051956in}}{\pgfqpoint{7.878508in}{4.044142in}}%
\pgfpathcurveto{\pgfqpoint{7.870694in}{4.036329in}}{\pgfqpoint{7.866304in}{4.025730in}}{\pgfqpoint{7.866304in}{4.014679in}}%
\pgfpathcurveto{\pgfqpoint{7.866304in}{4.003629in}}{\pgfqpoint{7.870694in}{3.993030in}}{\pgfqpoint{7.878508in}{3.985217in}}%
\pgfpathcurveto{\pgfqpoint{7.886321in}{3.977403in}}{\pgfqpoint{7.896920in}{3.973013in}}{\pgfqpoint{7.907970in}{3.973013in}}%
\pgfpathclose%
\pgfusepath{stroke,fill}%
\end{pgfscope}%
\begin{pgfscope}%
\pgfpathrectangle{\pgfqpoint{0.481978in}{0.331635in}}{\pgfqpoint{9.300000in}{7.700000in}}%
\pgfusepath{clip}%
\pgfsetbuttcap%
\pgfsetroundjoin%
\definecolor{currentfill}{rgb}{1.000000,0.705882,0.509804}%
\pgfsetfillcolor{currentfill}%
\pgfsetlinewidth{0.481800pt}%
\definecolor{currentstroke}{rgb}{1.000000,1.000000,1.000000}%
\pgfsetstrokecolor{currentstroke}%
\pgfsetdash{}{0pt}%
\pgfpathmoveto{\pgfqpoint{8.433076in}{4.518523in}}%
\pgfpathcurveto{\pgfqpoint{8.444126in}{4.518523in}}{\pgfqpoint{8.454725in}{4.522914in}}{\pgfqpoint{8.462538in}{4.530727in}}%
\pgfpathcurveto{\pgfqpoint{8.470352in}{4.538541in}}{\pgfqpoint{8.474742in}{4.549140in}}{\pgfqpoint{8.474742in}{4.560190in}}%
\pgfpathcurveto{\pgfqpoint{8.474742in}{4.571240in}}{\pgfqpoint{8.470352in}{4.581839in}}{\pgfqpoint{8.462538in}{4.589653in}}%
\pgfpathcurveto{\pgfqpoint{8.454725in}{4.597467in}}{\pgfqpoint{8.444126in}{4.601857in}}{\pgfqpoint{8.433076in}{4.601857in}}%
\pgfpathcurveto{\pgfqpoint{8.422025in}{4.601857in}}{\pgfqpoint{8.411426in}{4.597467in}}{\pgfqpoint{8.403613in}{4.589653in}}%
\pgfpathcurveto{\pgfqpoint{8.395799in}{4.581839in}}{\pgfqpoint{8.391409in}{4.571240in}}{\pgfqpoint{8.391409in}{4.560190in}}%
\pgfpathcurveto{\pgfqpoint{8.391409in}{4.549140in}}{\pgfqpoint{8.395799in}{4.538541in}}{\pgfqpoint{8.403613in}{4.530727in}}%
\pgfpathcurveto{\pgfqpoint{8.411426in}{4.522914in}}{\pgfqpoint{8.422025in}{4.518523in}}{\pgfqpoint{8.433076in}{4.518523in}}%
\pgfpathclose%
\pgfusepath{stroke,fill}%
\end{pgfscope}%
\begin{pgfscope}%
\pgfpathrectangle{\pgfqpoint{0.481978in}{0.331635in}}{\pgfqpoint{9.300000in}{7.700000in}}%
\pgfusepath{clip}%
\pgfsetbuttcap%
\pgfsetroundjoin%
\definecolor{currentfill}{rgb}{1.000000,0.705882,0.509804}%
\pgfsetfillcolor{currentfill}%
\pgfsetlinewidth{0.481800pt}%
\definecolor{currentstroke}{rgb}{1.000000,1.000000,1.000000}%
\pgfsetstrokecolor{currentstroke}%
\pgfsetdash{}{0pt}%
\pgfpathmoveto{\pgfqpoint{3.128738in}{2.053690in}}%
\pgfpathcurveto{\pgfqpoint{3.139789in}{2.053690in}}{\pgfqpoint{3.150388in}{2.058080in}}{\pgfqpoint{3.158201in}{2.065894in}}%
\pgfpathcurveto{\pgfqpoint{3.166015in}{2.073708in}}{\pgfqpoint{3.170405in}{2.084307in}}{\pgfqpoint{3.170405in}{2.095357in}}%
\pgfpathcurveto{\pgfqpoint{3.170405in}{2.106407in}}{\pgfqpoint{3.166015in}{2.117006in}}{\pgfqpoint{3.158201in}{2.124820in}}%
\pgfpathcurveto{\pgfqpoint{3.150388in}{2.132633in}}{\pgfqpoint{3.139789in}{2.137024in}}{\pgfqpoint{3.128738in}{2.137024in}}%
\pgfpathcurveto{\pgfqpoint{3.117688in}{2.137024in}}{\pgfqpoint{3.107089in}{2.132633in}}{\pgfqpoint{3.099276in}{2.124820in}}%
\pgfpathcurveto{\pgfqpoint{3.091462in}{2.117006in}}{\pgfqpoint{3.087072in}{2.106407in}}{\pgfqpoint{3.087072in}{2.095357in}}%
\pgfpathcurveto{\pgfqpoint{3.087072in}{2.084307in}}{\pgfqpoint{3.091462in}{2.073708in}}{\pgfqpoint{3.099276in}{2.065894in}}%
\pgfpathcurveto{\pgfqpoint{3.107089in}{2.058080in}}{\pgfqpoint{3.117688in}{2.053690in}}{\pgfqpoint{3.128738in}{2.053690in}}%
\pgfpathclose%
\pgfusepath{stroke,fill}%
\end{pgfscope}%
\begin{pgfscope}%
\pgfpathrectangle{\pgfqpoint{0.481978in}{0.331635in}}{\pgfqpoint{9.300000in}{7.700000in}}%
\pgfusepath{clip}%
\pgfsetbuttcap%
\pgfsetroundjoin%
\definecolor{currentfill}{rgb}{1.000000,0.705882,0.509804}%
\pgfsetfillcolor{currentfill}%
\pgfsetlinewidth{0.481800pt}%
\definecolor{currentstroke}{rgb}{1.000000,1.000000,1.000000}%
\pgfsetstrokecolor{currentstroke}%
\pgfsetdash{}{0pt}%
\pgfpathmoveto{\pgfqpoint{6.750608in}{3.030787in}}%
\pgfpathcurveto{\pgfqpoint{6.761658in}{3.030787in}}{\pgfqpoint{6.772257in}{3.035178in}}{\pgfqpoint{6.780070in}{3.042991in}}%
\pgfpathcurveto{\pgfqpoint{6.787884in}{3.050805in}}{\pgfqpoint{6.792274in}{3.061404in}}{\pgfqpoint{6.792274in}{3.072454in}}%
\pgfpathcurveto{\pgfqpoint{6.792274in}{3.083504in}}{\pgfqpoint{6.787884in}{3.094103in}}{\pgfqpoint{6.780070in}{3.101917in}}%
\pgfpathcurveto{\pgfqpoint{6.772257in}{3.109730in}}{\pgfqpoint{6.761658in}{3.114121in}}{\pgfqpoint{6.750608in}{3.114121in}}%
\pgfpathcurveto{\pgfqpoint{6.739557in}{3.114121in}}{\pgfqpoint{6.728958in}{3.109730in}}{\pgfqpoint{6.721145in}{3.101917in}}%
\pgfpathcurveto{\pgfqpoint{6.713331in}{3.094103in}}{\pgfqpoint{6.708941in}{3.083504in}}{\pgfqpoint{6.708941in}{3.072454in}}%
\pgfpathcurveto{\pgfqpoint{6.708941in}{3.061404in}}{\pgfqpoint{6.713331in}{3.050805in}}{\pgfqpoint{6.721145in}{3.042991in}}%
\pgfpathcurveto{\pgfqpoint{6.728958in}{3.035178in}}{\pgfqpoint{6.739557in}{3.030787in}}{\pgfqpoint{6.750608in}{3.030787in}}%
\pgfpathclose%
\pgfusepath{stroke,fill}%
\end{pgfscope}%
\begin{pgfscope}%
\pgfpathrectangle{\pgfqpoint{0.481978in}{0.331635in}}{\pgfqpoint{9.300000in}{7.700000in}}%
\pgfusepath{clip}%
\pgfsetbuttcap%
\pgfsetroundjoin%
\definecolor{currentfill}{rgb}{1.000000,0.705882,0.509804}%
\pgfsetfillcolor{currentfill}%
\pgfsetlinewidth{0.481800pt}%
\definecolor{currentstroke}{rgb}{1.000000,1.000000,1.000000}%
\pgfsetstrokecolor{currentstroke}%
\pgfsetdash{}{0pt}%
\pgfpathmoveto{\pgfqpoint{3.434837in}{4.030228in}}%
\pgfpathcurveto{\pgfqpoint{3.445887in}{4.030228in}}{\pgfqpoint{3.456486in}{4.034618in}}{\pgfqpoint{3.464300in}{4.042432in}}%
\pgfpathcurveto{\pgfqpoint{3.472114in}{4.050245in}}{\pgfqpoint{3.476504in}{4.060844in}}{\pgfqpoint{3.476504in}{4.071894in}}%
\pgfpathcurveto{\pgfqpoint{3.476504in}{4.082945in}}{\pgfqpoint{3.472114in}{4.093544in}}{\pgfqpoint{3.464300in}{4.101357in}}%
\pgfpathcurveto{\pgfqpoint{3.456486in}{4.109171in}}{\pgfqpoint{3.445887in}{4.113561in}}{\pgfqpoint{3.434837in}{4.113561in}}%
\pgfpathcurveto{\pgfqpoint{3.423787in}{4.113561in}}{\pgfqpoint{3.413188in}{4.109171in}}{\pgfqpoint{3.405375in}{4.101357in}}%
\pgfpathcurveto{\pgfqpoint{3.397561in}{4.093544in}}{\pgfqpoint{3.393171in}{4.082945in}}{\pgfqpoint{3.393171in}{4.071894in}}%
\pgfpathcurveto{\pgfqpoint{3.393171in}{4.060844in}}{\pgfqpoint{3.397561in}{4.050245in}}{\pgfqpoint{3.405375in}{4.042432in}}%
\pgfpathcurveto{\pgfqpoint{3.413188in}{4.034618in}}{\pgfqpoint{3.423787in}{4.030228in}}{\pgfqpoint{3.434837in}{4.030228in}}%
\pgfpathclose%
\pgfusepath{stroke,fill}%
\end{pgfscope}%
\begin{pgfscope}%
\pgfpathrectangle{\pgfqpoint{0.481978in}{0.331635in}}{\pgfqpoint{9.300000in}{7.700000in}}%
\pgfusepath{clip}%
\pgfsetbuttcap%
\pgfsetroundjoin%
\definecolor{currentfill}{rgb}{1.000000,0.705882,0.509804}%
\pgfsetfillcolor{currentfill}%
\pgfsetlinewidth{0.481800pt}%
\definecolor{currentstroke}{rgb}{1.000000,1.000000,1.000000}%
\pgfsetstrokecolor{currentstroke}%
\pgfsetdash{}{0pt}%
\pgfpathmoveto{\pgfqpoint{2.745959in}{2.280244in}}%
\pgfpathcurveto{\pgfqpoint{2.757010in}{2.280244in}}{\pgfqpoint{2.767609in}{2.284634in}}{\pgfqpoint{2.775422in}{2.292448in}}%
\pgfpathcurveto{\pgfqpoint{2.783236in}{2.300262in}}{\pgfqpoint{2.787626in}{2.310861in}}{\pgfqpoint{2.787626in}{2.321911in}}%
\pgfpathcurveto{\pgfqpoint{2.787626in}{2.332961in}}{\pgfqpoint{2.783236in}{2.343560in}}{\pgfqpoint{2.775422in}{2.351374in}}%
\pgfpathcurveto{\pgfqpoint{2.767609in}{2.359187in}}{\pgfqpoint{2.757010in}{2.363577in}}{\pgfqpoint{2.745959in}{2.363577in}}%
\pgfpathcurveto{\pgfqpoint{2.734909in}{2.363577in}}{\pgfqpoint{2.724310in}{2.359187in}}{\pgfqpoint{2.716497in}{2.351374in}}%
\pgfpathcurveto{\pgfqpoint{2.708683in}{2.343560in}}{\pgfqpoint{2.704293in}{2.332961in}}{\pgfqpoint{2.704293in}{2.321911in}}%
\pgfpathcurveto{\pgfqpoint{2.704293in}{2.310861in}}{\pgfqpoint{2.708683in}{2.300262in}}{\pgfqpoint{2.716497in}{2.292448in}}%
\pgfpathcurveto{\pgfqpoint{2.724310in}{2.284634in}}{\pgfqpoint{2.734909in}{2.280244in}}{\pgfqpoint{2.745959in}{2.280244in}}%
\pgfpathclose%
\pgfusepath{stroke,fill}%
\end{pgfscope}%
\begin{pgfscope}%
\pgfpathrectangle{\pgfqpoint{0.481978in}{0.331635in}}{\pgfqpoint{9.300000in}{7.700000in}}%
\pgfusepath{clip}%
\pgfsetbuttcap%
\pgfsetroundjoin%
\definecolor{currentfill}{rgb}{1.000000,0.705882,0.509804}%
\pgfsetfillcolor{currentfill}%
\pgfsetlinewidth{0.481800pt}%
\definecolor{currentstroke}{rgb}{1.000000,1.000000,1.000000}%
\pgfsetstrokecolor{currentstroke}%
\pgfsetdash{}{0pt}%
\pgfpathmoveto{\pgfqpoint{8.714562in}{5.295792in}}%
\pgfpathcurveto{\pgfqpoint{8.725612in}{5.295792in}}{\pgfqpoint{8.736211in}{5.300182in}}{\pgfqpoint{8.744024in}{5.307996in}}%
\pgfpathcurveto{\pgfqpoint{8.751838in}{5.315809in}}{\pgfqpoint{8.756228in}{5.326408in}}{\pgfqpoint{8.756228in}{5.337458in}}%
\pgfpathcurveto{\pgfqpoint{8.756228in}{5.348509in}}{\pgfqpoint{8.751838in}{5.359108in}}{\pgfqpoint{8.744024in}{5.366921in}}%
\pgfpathcurveto{\pgfqpoint{8.736211in}{5.374735in}}{\pgfqpoint{8.725612in}{5.379125in}}{\pgfqpoint{8.714562in}{5.379125in}}%
\pgfpathcurveto{\pgfqpoint{8.703512in}{5.379125in}}{\pgfqpoint{8.692913in}{5.374735in}}{\pgfqpoint{8.685099in}{5.366921in}}%
\pgfpathcurveto{\pgfqpoint{8.677285in}{5.359108in}}{\pgfqpoint{8.672895in}{5.348509in}}{\pgfqpoint{8.672895in}{5.337458in}}%
\pgfpathcurveto{\pgfqpoint{8.672895in}{5.326408in}}{\pgfqpoint{8.677285in}{5.315809in}}{\pgfqpoint{8.685099in}{5.307996in}}%
\pgfpathcurveto{\pgfqpoint{8.692913in}{5.300182in}}{\pgfqpoint{8.703512in}{5.295792in}}{\pgfqpoint{8.714562in}{5.295792in}}%
\pgfpathclose%
\pgfusepath{stroke,fill}%
\end{pgfscope}%
\begin{pgfscope}%
\pgfpathrectangle{\pgfqpoint{0.481978in}{0.331635in}}{\pgfqpoint{9.300000in}{7.700000in}}%
\pgfusepath{clip}%
\pgfsetbuttcap%
\pgfsetroundjoin%
\definecolor{currentfill}{rgb}{0.552941,0.898039,0.631373}%
\pgfsetfillcolor{currentfill}%
\pgfsetlinewidth{0.481800pt}%
\definecolor{currentstroke}{rgb}{1.000000,1.000000,1.000000}%
\pgfsetstrokecolor{currentstroke}%
\pgfsetdash{}{0pt}%
\pgfpathmoveto{\pgfqpoint{5.163299in}{7.301501in}}%
\pgfpathcurveto{\pgfqpoint{5.174349in}{7.301501in}}{\pgfqpoint{5.184948in}{7.305892in}}{\pgfqpoint{5.192762in}{7.313705in}}%
\pgfpathcurveto{\pgfqpoint{5.200575in}{7.321519in}}{\pgfqpoint{5.204966in}{7.332118in}}{\pgfqpoint{5.204966in}{7.343168in}}%
\pgfpathcurveto{\pgfqpoint{5.204966in}{7.354218in}}{\pgfqpoint{5.200575in}{7.364817in}}{\pgfqpoint{5.192762in}{7.372631in}}%
\pgfpathcurveto{\pgfqpoint{5.184948in}{7.380445in}}{\pgfqpoint{5.174349in}{7.384835in}}{\pgfqpoint{5.163299in}{7.384835in}}%
\pgfpathcurveto{\pgfqpoint{5.152249in}{7.384835in}}{\pgfqpoint{5.141650in}{7.380445in}}{\pgfqpoint{5.133836in}{7.372631in}}%
\pgfpathcurveto{\pgfqpoint{5.126023in}{7.364817in}}{\pgfqpoint{5.121632in}{7.354218in}}{\pgfqpoint{5.121632in}{7.343168in}}%
\pgfpathcurveto{\pgfqpoint{5.121632in}{7.332118in}}{\pgfqpoint{5.126023in}{7.321519in}}{\pgfqpoint{5.133836in}{7.313705in}}%
\pgfpathcurveto{\pgfqpoint{5.141650in}{7.305892in}}{\pgfqpoint{5.152249in}{7.301501in}}{\pgfqpoint{5.163299in}{7.301501in}}%
\pgfpathclose%
\pgfusepath{stroke,fill}%
\end{pgfscope}%
\begin{pgfscope}%
\pgfpathrectangle{\pgfqpoint{0.481978in}{0.331635in}}{\pgfqpoint{9.300000in}{7.700000in}}%
\pgfusepath{clip}%
\pgfsetbuttcap%
\pgfsetroundjoin%
\definecolor{currentfill}{rgb}{0.552941,0.898039,0.631373}%
\pgfsetfillcolor{currentfill}%
\pgfsetlinewidth{0.481800pt}%
\definecolor{currentstroke}{rgb}{1.000000,1.000000,1.000000}%
\pgfsetstrokecolor{currentstroke}%
\pgfsetdash{}{0pt}%
\pgfpathmoveto{\pgfqpoint{3.761360in}{5.722980in}}%
\pgfpathcurveto{\pgfqpoint{3.772410in}{5.722980in}}{\pgfqpoint{3.783009in}{5.727370in}}{\pgfqpoint{3.790823in}{5.735183in}}%
\pgfpathcurveto{\pgfqpoint{3.798637in}{5.742997in}}{\pgfqpoint{3.803027in}{5.753596in}}{\pgfqpoint{3.803027in}{5.764646in}}%
\pgfpathcurveto{\pgfqpoint{3.803027in}{5.775696in}}{\pgfqpoint{3.798637in}{5.786295in}}{\pgfqpoint{3.790823in}{5.794109in}}%
\pgfpathcurveto{\pgfqpoint{3.783009in}{5.801923in}}{\pgfqpoint{3.772410in}{5.806313in}}{\pgfqpoint{3.761360in}{5.806313in}}%
\pgfpathcurveto{\pgfqpoint{3.750310in}{5.806313in}}{\pgfqpoint{3.739711in}{5.801923in}}{\pgfqpoint{3.731897in}{5.794109in}}%
\pgfpathcurveto{\pgfqpoint{3.724084in}{5.786295in}}{\pgfqpoint{3.719693in}{5.775696in}}{\pgfqpoint{3.719693in}{5.764646in}}%
\pgfpathcurveto{\pgfqpoint{3.719693in}{5.753596in}}{\pgfqpoint{3.724084in}{5.742997in}}{\pgfqpoint{3.731897in}{5.735183in}}%
\pgfpathcurveto{\pgfqpoint{3.739711in}{5.727370in}}{\pgfqpoint{3.750310in}{5.722980in}}{\pgfqpoint{3.761360in}{5.722980in}}%
\pgfpathclose%
\pgfusepath{stroke,fill}%
\end{pgfscope}%
\begin{pgfscope}%
\pgfpathrectangle{\pgfqpoint{0.481978in}{0.331635in}}{\pgfqpoint{9.300000in}{7.700000in}}%
\pgfusepath{clip}%
\pgfsetbuttcap%
\pgfsetroundjoin%
\definecolor{currentfill}{rgb}{0.552941,0.898039,0.631373}%
\pgfsetfillcolor{currentfill}%
\pgfsetlinewidth{0.481800pt}%
\definecolor{currentstroke}{rgb}{1.000000,1.000000,1.000000}%
\pgfsetstrokecolor{currentstroke}%
\pgfsetdash{}{0pt}%
\pgfpathmoveto{\pgfqpoint{3.580350in}{3.046636in}}%
\pgfpathcurveto{\pgfqpoint{3.591400in}{3.046636in}}{\pgfqpoint{3.601999in}{3.051026in}}{\pgfqpoint{3.609813in}{3.058840in}}%
\pgfpathcurveto{\pgfqpoint{3.617626in}{3.066654in}}{\pgfqpoint{3.622017in}{3.077253in}}{\pgfqpoint{3.622017in}{3.088303in}}%
\pgfpathcurveto{\pgfqpoint{3.622017in}{3.099353in}}{\pgfqpoint{3.617626in}{3.109952in}}{\pgfqpoint{3.609813in}{3.117765in}}%
\pgfpathcurveto{\pgfqpoint{3.601999in}{3.125579in}}{\pgfqpoint{3.591400in}{3.129969in}}{\pgfqpoint{3.580350in}{3.129969in}}%
\pgfpathcurveto{\pgfqpoint{3.569300in}{3.129969in}}{\pgfqpoint{3.558701in}{3.125579in}}{\pgfqpoint{3.550887in}{3.117765in}}%
\pgfpathcurveto{\pgfqpoint{3.543073in}{3.109952in}}{\pgfqpoint{3.538683in}{3.099353in}}{\pgfqpoint{3.538683in}{3.088303in}}%
\pgfpathcurveto{\pgfqpoint{3.538683in}{3.077253in}}{\pgfqpoint{3.543073in}{3.066654in}}{\pgfqpoint{3.550887in}{3.058840in}}%
\pgfpathcurveto{\pgfqpoint{3.558701in}{3.051026in}}{\pgfqpoint{3.569300in}{3.046636in}}{\pgfqpoint{3.580350in}{3.046636in}}%
\pgfpathclose%
\pgfusepath{stroke,fill}%
\end{pgfscope}%
\begin{pgfscope}%
\pgfpathrectangle{\pgfqpoint{0.481978in}{0.331635in}}{\pgfqpoint{9.300000in}{7.700000in}}%
\pgfusepath{clip}%
\pgfsetbuttcap%
\pgfsetroundjoin%
\definecolor{currentfill}{rgb}{0.552941,0.898039,0.631373}%
\pgfsetfillcolor{currentfill}%
\pgfsetlinewidth{0.481800pt}%
\definecolor{currentstroke}{rgb}{1.000000,1.000000,1.000000}%
\pgfsetstrokecolor{currentstroke}%
\pgfsetdash{}{0pt}%
\pgfpathmoveto{\pgfqpoint{8.242673in}{4.179758in}}%
\pgfpathcurveto{\pgfqpoint{8.253723in}{4.179758in}}{\pgfqpoint{8.264322in}{4.184148in}}{\pgfqpoint{8.272136in}{4.191962in}}%
\pgfpathcurveto{\pgfqpoint{8.279949in}{4.199775in}}{\pgfqpoint{8.284340in}{4.210374in}}{\pgfqpoint{8.284340in}{4.221424in}}%
\pgfpathcurveto{\pgfqpoint{8.284340in}{4.232475in}}{\pgfqpoint{8.279949in}{4.243074in}}{\pgfqpoint{8.272136in}{4.250887in}}%
\pgfpathcurveto{\pgfqpoint{8.264322in}{4.258701in}}{\pgfqpoint{8.253723in}{4.263091in}}{\pgfqpoint{8.242673in}{4.263091in}}%
\pgfpathcurveto{\pgfqpoint{8.231623in}{4.263091in}}{\pgfqpoint{8.221024in}{4.258701in}}{\pgfqpoint{8.213210in}{4.250887in}}%
\pgfpathcurveto{\pgfqpoint{8.205396in}{4.243074in}}{\pgfqpoint{8.201006in}{4.232475in}}{\pgfqpoint{8.201006in}{4.221424in}}%
\pgfpathcurveto{\pgfqpoint{8.201006in}{4.210374in}}{\pgfqpoint{8.205396in}{4.199775in}}{\pgfqpoint{8.213210in}{4.191962in}}%
\pgfpathcurveto{\pgfqpoint{8.221024in}{4.184148in}}{\pgfqpoint{8.231623in}{4.179758in}}{\pgfqpoint{8.242673in}{4.179758in}}%
\pgfpathclose%
\pgfusepath{stroke,fill}%
\end{pgfscope}%
\begin{pgfscope}%
\pgfpathrectangle{\pgfqpoint{0.481978in}{0.331635in}}{\pgfqpoint{9.300000in}{7.700000in}}%
\pgfusepath{clip}%
\pgfsetbuttcap%
\pgfsetroundjoin%
\definecolor{currentfill}{rgb}{0.552941,0.898039,0.631373}%
\pgfsetfillcolor{currentfill}%
\pgfsetlinewidth{0.481800pt}%
\definecolor{currentstroke}{rgb}{1.000000,1.000000,1.000000}%
\pgfsetstrokecolor{currentstroke}%
\pgfsetdash{}{0pt}%
\pgfpathmoveto{\pgfqpoint{4.433234in}{5.029314in}}%
\pgfpathcurveto{\pgfqpoint{4.444284in}{5.029314in}}{\pgfqpoint{4.454883in}{5.033704in}}{\pgfqpoint{4.462696in}{5.041518in}}%
\pgfpathcurveto{\pgfqpoint{4.470510in}{5.049332in}}{\pgfqpoint{4.474900in}{5.059931in}}{\pgfqpoint{4.474900in}{5.070981in}}%
\pgfpathcurveto{\pgfqpoint{4.474900in}{5.082031in}}{\pgfqpoint{4.470510in}{5.092630in}}{\pgfqpoint{4.462696in}{5.100444in}}%
\pgfpathcurveto{\pgfqpoint{4.454883in}{5.108257in}}{\pgfqpoint{4.444284in}{5.112648in}}{\pgfqpoint{4.433234in}{5.112648in}}%
\pgfpathcurveto{\pgfqpoint{4.422183in}{5.112648in}}{\pgfqpoint{4.411584in}{5.108257in}}{\pgfqpoint{4.403771in}{5.100444in}}%
\pgfpathcurveto{\pgfqpoint{4.395957in}{5.092630in}}{\pgfqpoint{4.391567in}{5.082031in}}{\pgfqpoint{4.391567in}{5.070981in}}%
\pgfpathcurveto{\pgfqpoint{4.391567in}{5.059931in}}{\pgfqpoint{4.395957in}{5.049332in}}{\pgfqpoint{4.403771in}{5.041518in}}%
\pgfpathcurveto{\pgfqpoint{4.411584in}{5.033704in}}{\pgfqpoint{4.422183in}{5.029314in}}{\pgfqpoint{4.433234in}{5.029314in}}%
\pgfpathclose%
\pgfusepath{stroke,fill}%
\end{pgfscope}%
\begin{pgfscope}%
\pgfpathrectangle{\pgfqpoint{0.481978in}{0.331635in}}{\pgfqpoint{9.300000in}{7.700000in}}%
\pgfusepath{clip}%
\pgfsetbuttcap%
\pgfsetroundjoin%
\definecolor{currentfill}{rgb}{0.552941,0.898039,0.631373}%
\pgfsetfillcolor{currentfill}%
\pgfsetlinewidth{0.481800pt}%
\definecolor{currentstroke}{rgb}{1.000000,1.000000,1.000000}%
\pgfsetstrokecolor{currentstroke}%
\pgfsetdash{}{0pt}%
\pgfpathmoveto{\pgfqpoint{9.194655in}{5.622269in}}%
\pgfpathcurveto{\pgfqpoint{9.205705in}{5.622269in}}{\pgfqpoint{9.216304in}{5.626659in}}{\pgfqpoint{9.224117in}{5.634473in}}%
\pgfpathcurveto{\pgfqpoint{9.231931in}{5.642286in}}{\pgfqpoint{9.236321in}{5.652885in}}{\pgfqpoint{9.236321in}{5.663935in}}%
\pgfpathcurveto{\pgfqpoint{9.236321in}{5.674985in}}{\pgfqpoint{9.231931in}{5.685584in}}{\pgfqpoint{9.224117in}{5.693398in}}%
\pgfpathcurveto{\pgfqpoint{9.216304in}{5.701212in}}{\pgfqpoint{9.205705in}{5.705602in}}{\pgfqpoint{9.194655in}{5.705602in}}%
\pgfpathcurveto{\pgfqpoint{9.183604in}{5.705602in}}{\pgfqpoint{9.173005in}{5.701212in}}{\pgfqpoint{9.165192in}{5.693398in}}%
\pgfpathcurveto{\pgfqpoint{9.157378in}{5.685584in}}{\pgfqpoint{9.152988in}{5.674985in}}{\pgfqpoint{9.152988in}{5.663935in}}%
\pgfpathcurveto{\pgfqpoint{9.152988in}{5.652885in}}{\pgfqpoint{9.157378in}{5.642286in}}{\pgfqpoint{9.165192in}{5.634473in}}%
\pgfpathcurveto{\pgfqpoint{9.173005in}{5.626659in}}{\pgfqpoint{9.183604in}{5.622269in}}{\pgfqpoint{9.194655in}{5.622269in}}%
\pgfpathclose%
\pgfusepath{stroke,fill}%
\end{pgfscope}%
\begin{pgfscope}%
\pgfpathrectangle{\pgfqpoint{0.481978in}{0.331635in}}{\pgfqpoint{9.300000in}{7.700000in}}%
\pgfusepath{clip}%
\pgfsetbuttcap%
\pgfsetroundjoin%
\definecolor{currentfill}{rgb}{0.552941,0.898039,0.631373}%
\pgfsetfillcolor{currentfill}%
\pgfsetlinewidth{0.481800pt}%
\definecolor{currentstroke}{rgb}{1.000000,1.000000,1.000000}%
\pgfsetstrokecolor{currentstroke}%
\pgfsetdash{}{0pt}%
\pgfpathmoveto{\pgfqpoint{9.032589in}{5.038453in}}%
\pgfpathcurveto{\pgfqpoint{9.043639in}{5.038453in}}{\pgfqpoint{9.054238in}{5.042843in}}{\pgfqpoint{9.062052in}{5.050657in}}%
\pgfpathcurveto{\pgfqpoint{9.069866in}{5.058470in}}{\pgfqpoint{9.074256in}{5.069069in}}{\pgfqpoint{9.074256in}{5.080119in}}%
\pgfpathcurveto{\pgfqpoint{9.074256in}{5.091169in}}{\pgfqpoint{9.069866in}{5.101769in}}{\pgfqpoint{9.062052in}{5.109582in}}%
\pgfpathcurveto{\pgfqpoint{9.054238in}{5.117396in}}{\pgfqpoint{9.043639in}{5.121786in}}{\pgfqpoint{9.032589in}{5.121786in}}%
\pgfpathcurveto{\pgfqpoint{9.021539in}{5.121786in}}{\pgfqpoint{9.010940in}{5.117396in}}{\pgfqpoint{9.003127in}{5.109582in}}%
\pgfpathcurveto{\pgfqpoint{8.995313in}{5.101769in}}{\pgfqpoint{8.990923in}{5.091169in}}{\pgfqpoint{8.990923in}{5.080119in}}%
\pgfpathcurveto{\pgfqpoint{8.990923in}{5.069069in}}{\pgfqpoint{8.995313in}{5.058470in}}{\pgfqpoint{9.003127in}{5.050657in}}%
\pgfpathcurveto{\pgfqpoint{9.010940in}{5.042843in}}{\pgfqpoint{9.021539in}{5.038453in}}{\pgfqpoint{9.032589in}{5.038453in}}%
\pgfpathclose%
\pgfusepath{stroke,fill}%
\end{pgfscope}%
\begin{pgfscope}%
\pgfpathrectangle{\pgfqpoint{0.481978in}{0.331635in}}{\pgfqpoint{9.300000in}{7.700000in}}%
\pgfusepath{clip}%
\pgfsetbuttcap%
\pgfsetroundjoin%
\definecolor{currentfill}{rgb}{0.552941,0.898039,0.631373}%
\pgfsetfillcolor{currentfill}%
\pgfsetlinewidth{0.481800pt}%
\definecolor{currentstroke}{rgb}{1.000000,1.000000,1.000000}%
\pgfsetstrokecolor{currentstroke}%
\pgfsetdash{}{0pt}%
\pgfpathmoveto{\pgfqpoint{3.942708in}{5.631876in}}%
\pgfpathcurveto{\pgfqpoint{3.953758in}{5.631876in}}{\pgfqpoint{3.964357in}{5.636267in}}{\pgfqpoint{3.972171in}{5.644080in}}%
\pgfpathcurveto{\pgfqpoint{3.979985in}{5.651894in}}{\pgfqpoint{3.984375in}{5.662493in}}{\pgfqpoint{3.984375in}{5.673543in}}%
\pgfpathcurveto{\pgfqpoint{3.984375in}{5.684593in}}{\pgfqpoint{3.979985in}{5.695192in}}{\pgfqpoint{3.972171in}{5.703006in}}%
\pgfpathcurveto{\pgfqpoint{3.964357in}{5.710819in}}{\pgfqpoint{3.953758in}{5.715210in}}{\pgfqpoint{3.942708in}{5.715210in}}%
\pgfpathcurveto{\pgfqpoint{3.931658in}{5.715210in}}{\pgfqpoint{3.921059in}{5.710819in}}{\pgfqpoint{3.913245in}{5.703006in}}%
\pgfpathcurveto{\pgfqpoint{3.905432in}{5.695192in}}{\pgfqpoint{3.901042in}{5.684593in}}{\pgfqpoint{3.901042in}{5.673543in}}%
\pgfpathcurveto{\pgfqpoint{3.901042in}{5.662493in}}{\pgfqpoint{3.905432in}{5.651894in}}{\pgfqpoint{3.913245in}{5.644080in}}%
\pgfpathcurveto{\pgfqpoint{3.921059in}{5.636267in}}{\pgfqpoint{3.931658in}{5.631876in}}{\pgfqpoint{3.942708in}{5.631876in}}%
\pgfpathclose%
\pgfusepath{stroke,fill}%
\end{pgfscope}%
\begin{pgfscope}%
\pgfpathrectangle{\pgfqpoint{0.481978in}{0.331635in}}{\pgfqpoint{9.300000in}{7.700000in}}%
\pgfusepath{clip}%
\pgfsetbuttcap%
\pgfsetroundjoin%
\definecolor{currentfill}{rgb}{0.552941,0.898039,0.631373}%
\pgfsetfillcolor{currentfill}%
\pgfsetlinewidth{0.481800pt}%
\definecolor{currentstroke}{rgb}{1.000000,1.000000,1.000000}%
\pgfsetstrokecolor{currentstroke}%
\pgfsetdash{}{0pt}%
\pgfpathmoveto{\pgfqpoint{5.874679in}{2.681112in}}%
\pgfpathcurveto{\pgfqpoint{5.885729in}{2.681112in}}{\pgfqpoint{5.896328in}{2.685502in}}{\pgfqpoint{5.904141in}{2.693316in}}%
\pgfpathcurveto{\pgfqpoint{5.911955in}{2.701129in}}{\pgfqpoint{5.916345in}{2.711728in}}{\pgfqpoint{5.916345in}{2.722779in}}%
\pgfpathcurveto{\pgfqpoint{5.916345in}{2.733829in}}{\pgfqpoint{5.911955in}{2.744428in}}{\pgfqpoint{5.904141in}{2.752241in}}%
\pgfpathcurveto{\pgfqpoint{5.896328in}{2.760055in}}{\pgfqpoint{5.885729in}{2.764445in}}{\pgfqpoint{5.874679in}{2.764445in}}%
\pgfpathcurveto{\pgfqpoint{5.863628in}{2.764445in}}{\pgfqpoint{5.853029in}{2.760055in}}{\pgfqpoint{5.845216in}{2.752241in}}%
\pgfpathcurveto{\pgfqpoint{5.837402in}{2.744428in}}{\pgfqpoint{5.833012in}{2.733829in}}{\pgfqpoint{5.833012in}{2.722779in}}%
\pgfpathcurveto{\pgfqpoint{5.833012in}{2.711728in}}{\pgfqpoint{5.837402in}{2.701129in}}{\pgfqpoint{5.845216in}{2.693316in}}%
\pgfpathcurveto{\pgfqpoint{5.853029in}{2.685502in}}{\pgfqpoint{5.863628in}{2.681112in}}{\pgfqpoint{5.874679in}{2.681112in}}%
\pgfpathclose%
\pgfusepath{stroke,fill}%
\end{pgfscope}%
\begin{pgfscope}%
\pgfpathrectangle{\pgfqpoint{0.481978in}{0.331635in}}{\pgfqpoint{9.300000in}{7.700000in}}%
\pgfusepath{clip}%
\pgfsetbuttcap%
\pgfsetroundjoin%
\definecolor{currentfill}{rgb}{0.552941,0.898039,0.631373}%
\pgfsetfillcolor{currentfill}%
\pgfsetlinewidth{0.481800pt}%
\definecolor{currentstroke}{rgb}{1.000000,1.000000,1.000000}%
\pgfsetstrokecolor{currentstroke}%
\pgfsetdash{}{0pt}%
\pgfpathmoveto{\pgfqpoint{6.449863in}{3.139229in}}%
\pgfpathcurveto{\pgfqpoint{6.460914in}{3.139229in}}{\pgfqpoint{6.471513in}{3.143620in}}{\pgfqpoint{6.479326in}{3.151433in}}%
\pgfpathcurveto{\pgfqpoint{6.487140in}{3.159247in}}{\pgfqpoint{6.491530in}{3.169846in}}{\pgfqpoint{6.491530in}{3.180896in}}%
\pgfpathcurveto{\pgfqpoint{6.491530in}{3.191946in}}{\pgfqpoint{6.487140in}{3.202545in}}{\pgfqpoint{6.479326in}{3.210359in}}%
\pgfpathcurveto{\pgfqpoint{6.471513in}{3.218173in}}{\pgfqpoint{6.460914in}{3.222563in}}{\pgfqpoint{6.449863in}{3.222563in}}%
\pgfpathcurveto{\pgfqpoint{6.438813in}{3.222563in}}{\pgfqpoint{6.428214in}{3.218173in}}{\pgfqpoint{6.420401in}{3.210359in}}%
\pgfpathcurveto{\pgfqpoint{6.412587in}{3.202545in}}{\pgfqpoint{6.408197in}{3.191946in}}{\pgfqpoint{6.408197in}{3.180896in}}%
\pgfpathcurveto{\pgfqpoint{6.408197in}{3.169846in}}{\pgfqpoint{6.412587in}{3.159247in}}{\pgfqpoint{6.420401in}{3.151433in}}%
\pgfpathcurveto{\pgfqpoint{6.428214in}{3.143620in}}{\pgfqpoint{6.438813in}{3.139229in}}{\pgfqpoint{6.449863in}{3.139229in}}%
\pgfpathclose%
\pgfusepath{stroke,fill}%
\end{pgfscope}%
\begin{pgfscope}%
\pgfpathrectangle{\pgfqpoint{0.481978in}{0.331635in}}{\pgfqpoint{9.300000in}{7.700000in}}%
\pgfusepath{clip}%
\pgfsetbuttcap%
\pgfsetroundjoin%
\definecolor{currentfill}{rgb}{0.552941,0.898039,0.631373}%
\pgfsetfillcolor{currentfill}%
\pgfsetlinewidth{0.481800pt}%
\definecolor{currentstroke}{rgb}{1.000000,1.000000,1.000000}%
\pgfsetstrokecolor{currentstroke}%
\pgfsetdash{}{0pt}%
\pgfpathmoveto{\pgfqpoint{2.941597in}{6.325321in}}%
\pgfpathcurveto{\pgfqpoint{2.952647in}{6.325321in}}{\pgfqpoint{2.963246in}{6.329712in}}{\pgfqpoint{2.971060in}{6.337525in}}%
\pgfpathcurveto{\pgfqpoint{2.978874in}{6.345339in}}{\pgfqpoint{2.983264in}{6.355938in}}{\pgfqpoint{2.983264in}{6.366988in}}%
\pgfpathcurveto{\pgfqpoint{2.983264in}{6.378038in}}{\pgfqpoint{2.978874in}{6.388637in}}{\pgfqpoint{2.971060in}{6.396451in}}%
\pgfpathcurveto{\pgfqpoint{2.963246in}{6.404264in}}{\pgfqpoint{2.952647in}{6.408655in}}{\pgfqpoint{2.941597in}{6.408655in}}%
\pgfpathcurveto{\pgfqpoint{2.930547in}{6.408655in}}{\pgfqpoint{2.919948in}{6.404264in}}{\pgfqpoint{2.912135in}{6.396451in}}%
\pgfpathcurveto{\pgfqpoint{2.904321in}{6.388637in}}{\pgfqpoint{2.899931in}{6.378038in}}{\pgfqpoint{2.899931in}{6.366988in}}%
\pgfpathcurveto{\pgfqpoint{2.899931in}{6.355938in}}{\pgfqpoint{2.904321in}{6.345339in}}{\pgfqpoint{2.912135in}{6.337525in}}%
\pgfpathcurveto{\pgfqpoint{2.919948in}{6.329712in}}{\pgfqpoint{2.930547in}{6.325321in}}{\pgfqpoint{2.941597in}{6.325321in}}%
\pgfpathclose%
\pgfusepath{stroke,fill}%
\end{pgfscope}%
\begin{pgfscope}%
\pgfpathrectangle{\pgfqpoint{0.481978in}{0.331635in}}{\pgfqpoint{9.300000in}{7.700000in}}%
\pgfusepath{clip}%
\pgfsetbuttcap%
\pgfsetroundjoin%
\definecolor{currentfill}{rgb}{0.552941,0.898039,0.631373}%
\pgfsetfillcolor{currentfill}%
\pgfsetlinewidth{0.481800pt}%
\definecolor{currentstroke}{rgb}{1.000000,1.000000,1.000000}%
\pgfsetstrokecolor{currentstroke}%
\pgfsetdash{}{0pt}%
\pgfpathmoveto{\pgfqpoint{4.671581in}{6.073104in}}%
\pgfpathcurveto{\pgfqpoint{4.682631in}{6.073104in}}{\pgfqpoint{4.693230in}{6.077495in}}{\pgfqpoint{4.701044in}{6.085308in}}%
\pgfpathcurveto{\pgfqpoint{4.708857in}{6.093122in}}{\pgfqpoint{4.713248in}{6.103721in}}{\pgfqpoint{4.713248in}{6.114771in}}%
\pgfpathcurveto{\pgfqpoint{4.713248in}{6.125821in}}{\pgfqpoint{4.708857in}{6.136420in}}{\pgfqpoint{4.701044in}{6.144234in}}%
\pgfpathcurveto{\pgfqpoint{4.693230in}{6.152047in}}{\pgfqpoint{4.682631in}{6.156438in}}{\pgfqpoint{4.671581in}{6.156438in}}%
\pgfpathcurveto{\pgfqpoint{4.660531in}{6.156438in}}{\pgfqpoint{4.649932in}{6.152047in}}{\pgfqpoint{4.642118in}{6.144234in}}%
\pgfpathcurveto{\pgfqpoint{4.634304in}{6.136420in}}{\pgfqpoint{4.629914in}{6.125821in}}{\pgfqpoint{4.629914in}{6.114771in}}%
\pgfpathcurveto{\pgfqpoint{4.629914in}{6.103721in}}{\pgfqpoint{4.634304in}{6.093122in}}{\pgfqpoint{4.642118in}{6.085308in}}%
\pgfpathcurveto{\pgfqpoint{4.649932in}{6.077495in}}{\pgfqpoint{4.660531in}{6.073104in}}{\pgfqpoint{4.671581in}{6.073104in}}%
\pgfpathclose%
\pgfusepath{stroke,fill}%
\end{pgfscope}%
\begin{pgfscope}%
\pgfpathrectangle{\pgfqpoint{0.481978in}{0.331635in}}{\pgfqpoint{9.300000in}{7.700000in}}%
\pgfusepath{clip}%
\pgfsetbuttcap%
\pgfsetroundjoin%
\definecolor{currentfill}{rgb}{0.552941,0.898039,0.631373}%
\pgfsetfillcolor{currentfill}%
\pgfsetlinewidth{0.481800pt}%
\definecolor{currentstroke}{rgb}{1.000000,1.000000,1.000000}%
\pgfsetstrokecolor{currentstroke}%
\pgfsetdash{}{0pt}%
\pgfpathmoveto{\pgfqpoint{3.259508in}{2.916456in}}%
\pgfpathcurveto{\pgfqpoint{3.270559in}{2.916456in}}{\pgfqpoint{3.281158in}{2.920846in}}{\pgfqpoint{3.288971in}{2.928660in}}%
\pgfpathcurveto{\pgfqpoint{3.296785in}{2.936473in}}{\pgfqpoint{3.301175in}{2.947072in}}{\pgfqpoint{3.301175in}{2.958122in}}%
\pgfpathcurveto{\pgfqpoint{3.301175in}{2.969172in}}{\pgfqpoint{3.296785in}{2.979771in}}{\pgfqpoint{3.288971in}{2.987585in}}%
\pgfpathcurveto{\pgfqpoint{3.281158in}{2.995399in}}{\pgfqpoint{3.270559in}{2.999789in}}{\pgfqpoint{3.259508in}{2.999789in}}%
\pgfpathcurveto{\pgfqpoint{3.248458in}{2.999789in}}{\pgfqpoint{3.237859in}{2.995399in}}{\pgfqpoint{3.230046in}{2.987585in}}%
\pgfpathcurveto{\pgfqpoint{3.222232in}{2.979771in}}{\pgfqpoint{3.217842in}{2.969172in}}{\pgfqpoint{3.217842in}{2.958122in}}%
\pgfpathcurveto{\pgfqpoint{3.217842in}{2.947072in}}{\pgfqpoint{3.222232in}{2.936473in}}{\pgfqpoint{3.230046in}{2.928660in}}%
\pgfpathcurveto{\pgfqpoint{3.237859in}{2.920846in}}{\pgfqpoint{3.248458in}{2.916456in}}{\pgfqpoint{3.259508in}{2.916456in}}%
\pgfpathclose%
\pgfusepath{stroke,fill}%
\end{pgfscope}%
\begin{pgfscope}%
\pgfpathrectangle{\pgfqpoint{0.481978in}{0.331635in}}{\pgfqpoint{9.300000in}{7.700000in}}%
\pgfusepath{clip}%
\pgfsetbuttcap%
\pgfsetroundjoin%
\definecolor{currentfill}{rgb}{0.552941,0.898039,0.631373}%
\pgfsetfillcolor{currentfill}%
\pgfsetlinewidth{0.481800pt}%
\definecolor{currentstroke}{rgb}{1.000000,1.000000,1.000000}%
\pgfsetstrokecolor{currentstroke}%
\pgfsetdash{}{0pt}%
\pgfpathmoveto{\pgfqpoint{7.805062in}{3.402898in}}%
\pgfpathcurveto{\pgfqpoint{7.816112in}{3.402898in}}{\pgfqpoint{7.826711in}{3.407288in}}{\pgfqpoint{7.834525in}{3.415101in}}%
\pgfpathcurveto{\pgfqpoint{7.842338in}{3.422915in}}{\pgfqpoint{7.846728in}{3.433514in}}{\pgfqpoint{7.846728in}{3.444564in}}%
\pgfpathcurveto{\pgfqpoint{7.846728in}{3.455614in}}{\pgfqpoint{7.842338in}{3.466213in}}{\pgfqpoint{7.834525in}{3.474027in}}%
\pgfpathcurveto{\pgfqpoint{7.826711in}{3.481841in}}{\pgfqpoint{7.816112in}{3.486231in}}{\pgfqpoint{7.805062in}{3.486231in}}%
\pgfpathcurveto{\pgfqpoint{7.794012in}{3.486231in}}{\pgfqpoint{7.783413in}{3.481841in}}{\pgfqpoint{7.775599in}{3.474027in}}%
\pgfpathcurveto{\pgfqpoint{7.767785in}{3.466213in}}{\pgfqpoint{7.763395in}{3.455614in}}{\pgfqpoint{7.763395in}{3.444564in}}%
\pgfpathcurveto{\pgfqpoint{7.763395in}{3.433514in}}{\pgfqpoint{7.767785in}{3.422915in}}{\pgfqpoint{7.775599in}{3.415101in}}%
\pgfpathcurveto{\pgfqpoint{7.783413in}{3.407288in}}{\pgfqpoint{7.794012in}{3.402898in}}{\pgfqpoint{7.805062in}{3.402898in}}%
\pgfpathclose%
\pgfusepath{stroke,fill}%
\end{pgfscope}%
\begin{pgfscope}%
\pgfpathrectangle{\pgfqpoint{0.481978in}{0.331635in}}{\pgfqpoint{9.300000in}{7.700000in}}%
\pgfusepath{clip}%
\pgfsetbuttcap%
\pgfsetroundjoin%
\definecolor{currentfill}{rgb}{0.552941,0.898039,0.631373}%
\pgfsetfillcolor{currentfill}%
\pgfsetlinewidth{0.481800pt}%
\definecolor{currentstroke}{rgb}{1.000000,1.000000,1.000000}%
\pgfsetstrokecolor{currentstroke}%
\pgfsetdash{}{0pt}%
\pgfpathmoveto{\pgfqpoint{4.750388in}{7.153024in}}%
\pgfpathcurveto{\pgfqpoint{4.761438in}{7.153024in}}{\pgfqpoint{4.772037in}{7.157414in}}{\pgfqpoint{4.779850in}{7.165228in}}%
\pgfpathcurveto{\pgfqpoint{4.787664in}{7.173042in}}{\pgfqpoint{4.792054in}{7.183641in}}{\pgfqpoint{4.792054in}{7.194691in}}%
\pgfpathcurveto{\pgfqpoint{4.792054in}{7.205741in}}{\pgfqpoint{4.787664in}{7.216340in}}{\pgfqpoint{4.779850in}{7.224154in}}%
\pgfpathcurveto{\pgfqpoint{4.772037in}{7.231967in}}{\pgfqpoint{4.761438in}{7.236357in}}{\pgfqpoint{4.750388in}{7.236357in}}%
\pgfpathcurveto{\pgfqpoint{4.739338in}{7.236357in}}{\pgfqpoint{4.728739in}{7.231967in}}{\pgfqpoint{4.720925in}{7.224154in}}%
\pgfpathcurveto{\pgfqpoint{4.713111in}{7.216340in}}{\pgfqpoint{4.708721in}{7.205741in}}{\pgfqpoint{4.708721in}{7.194691in}}%
\pgfpathcurveto{\pgfqpoint{4.708721in}{7.183641in}}{\pgfqpoint{4.713111in}{7.173042in}}{\pgfqpoint{4.720925in}{7.165228in}}%
\pgfpathcurveto{\pgfqpoint{4.728739in}{7.157414in}}{\pgfqpoint{4.739338in}{7.153024in}}{\pgfqpoint{4.750388in}{7.153024in}}%
\pgfpathclose%
\pgfusepath{stroke,fill}%
\end{pgfscope}%
\begin{pgfscope}%
\pgfpathrectangle{\pgfqpoint{0.481978in}{0.331635in}}{\pgfqpoint{9.300000in}{7.700000in}}%
\pgfusepath{clip}%
\pgfsetbuttcap%
\pgfsetroundjoin%
\definecolor{currentfill}{rgb}{0.552941,0.898039,0.631373}%
\pgfsetfillcolor{currentfill}%
\pgfsetlinewidth{0.481800pt}%
\definecolor{currentstroke}{rgb}{1.000000,1.000000,1.000000}%
\pgfsetstrokecolor{currentstroke}%
\pgfsetdash{}{0pt}%
\pgfpathmoveto{\pgfqpoint{4.386424in}{5.529843in}}%
\pgfpathcurveto{\pgfqpoint{4.397474in}{5.529843in}}{\pgfqpoint{4.408073in}{5.534233in}}{\pgfqpoint{4.415887in}{5.542047in}}%
\pgfpathcurveto{\pgfqpoint{4.423700in}{5.549861in}}{\pgfqpoint{4.428091in}{5.560460in}}{\pgfqpoint{4.428091in}{5.571510in}}%
\pgfpathcurveto{\pgfqpoint{4.428091in}{5.582560in}}{\pgfqpoint{4.423700in}{5.593159in}}{\pgfqpoint{4.415887in}{5.600972in}}%
\pgfpathcurveto{\pgfqpoint{4.408073in}{5.608786in}}{\pgfqpoint{4.397474in}{5.613176in}}{\pgfqpoint{4.386424in}{5.613176in}}%
\pgfpathcurveto{\pgfqpoint{4.375374in}{5.613176in}}{\pgfqpoint{4.364775in}{5.608786in}}{\pgfqpoint{4.356961in}{5.600972in}}%
\pgfpathcurveto{\pgfqpoint{4.349148in}{5.593159in}}{\pgfqpoint{4.344757in}{5.582560in}}{\pgfqpoint{4.344757in}{5.571510in}}%
\pgfpathcurveto{\pgfqpoint{4.344757in}{5.560460in}}{\pgfqpoint{4.349148in}{5.549861in}}{\pgfqpoint{4.356961in}{5.542047in}}%
\pgfpathcurveto{\pgfqpoint{4.364775in}{5.534233in}}{\pgfqpoint{4.375374in}{5.529843in}}{\pgfqpoint{4.386424in}{5.529843in}}%
\pgfpathclose%
\pgfusepath{stroke,fill}%
\end{pgfscope}%
\begin{pgfscope}%
\pgfpathrectangle{\pgfqpoint{0.481978in}{0.331635in}}{\pgfqpoint{9.300000in}{7.700000in}}%
\pgfusepath{clip}%
\pgfsetbuttcap%
\pgfsetroundjoin%
\definecolor{currentfill}{rgb}{0.552941,0.898039,0.631373}%
\pgfsetfillcolor{currentfill}%
\pgfsetlinewidth{0.481800pt}%
\definecolor{currentstroke}{rgb}{1.000000,1.000000,1.000000}%
\pgfsetstrokecolor{currentstroke}%
\pgfsetdash{}{0pt}%
\pgfpathmoveto{\pgfqpoint{4.878045in}{6.061659in}}%
\pgfpathcurveto{\pgfqpoint{4.889095in}{6.061659in}}{\pgfqpoint{4.899694in}{6.066049in}}{\pgfqpoint{4.907508in}{6.073863in}}%
\pgfpathcurveto{\pgfqpoint{4.915321in}{6.081677in}}{\pgfqpoint{4.919712in}{6.092276in}}{\pgfqpoint{4.919712in}{6.103326in}}%
\pgfpathcurveto{\pgfqpoint{4.919712in}{6.114376in}}{\pgfqpoint{4.915321in}{6.124975in}}{\pgfqpoint{4.907508in}{6.132789in}}%
\pgfpathcurveto{\pgfqpoint{4.899694in}{6.140602in}}{\pgfqpoint{4.889095in}{6.144993in}}{\pgfqpoint{4.878045in}{6.144993in}}%
\pgfpathcurveto{\pgfqpoint{4.866995in}{6.144993in}}{\pgfqpoint{4.856396in}{6.140602in}}{\pgfqpoint{4.848582in}{6.132789in}}%
\pgfpathcurveto{\pgfqpoint{4.840769in}{6.124975in}}{\pgfqpoint{4.836378in}{6.114376in}}{\pgfqpoint{4.836378in}{6.103326in}}%
\pgfpathcurveto{\pgfqpoint{4.836378in}{6.092276in}}{\pgfqpoint{4.840769in}{6.081677in}}{\pgfqpoint{4.848582in}{6.073863in}}%
\pgfpathcurveto{\pgfqpoint{4.856396in}{6.066049in}}{\pgfqpoint{4.866995in}{6.061659in}}{\pgfqpoint{4.878045in}{6.061659in}}%
\pgfpathclose%
\pgfusepath{stroke,fill}%
\end{pgfscope}%
\begin{pgfscope}%
\pgfpathrectangle{\pgfqpoint{0.481978in}{0.331635in}}{\pgfqpoint{9.300000in}{7.700000in}}%
\pgfusepath{clip}%
\pgfsetbuttcap%
\pgfsetroundjoin%
\definecolor{currentfill}{rgb}{0.552941,0.898039,0.631373}%
\pgfsetfillcolor{currentfill}%
\pgfsetlinewidth{0.481800pt}%
\definecolor{currentstroke}{rgb}{1.000000,1.000000,1.000000}%
\pgfsetstrokecolor{currentstroke}%
\pgfsetdash{}{0pt}%
\pgfpathmoveto{\pgfqpoint{9.259474in}{5.157624in}}%
\pgfpathcurveto{\pgfqpoint{9.270524in}{5.157624in}}{\pgfqpoint{9.281123in}{5.162014in}}{\pgfqpoint{9.288937in}{5.169827in}}%
\pgfpathcurveto{\pgfqpoint{9.296750in}{5.177641in}}{\pgfqpoint{9.301140in}{5.188240in}}{\pgfqpoint{9.301140in}{5.199290in}}%
\pgfpathcurveto{\pgfqpoint{9.301140in}{5.210340in}}{\pgfqpoint{9.296750in}{5.220939in}}{\pgfqpoint{9.288937in}{5.228753in}}%
\pgfpathcurveto{\pgfqpoint{9.281123in}{5.236567in}}{\pgfqpoint{9.270524in}{5.240957in}}{\pgfqpoint{9.259474in}{5.240957in}}%
\pgfpathcurveto{\pgfqpoint{9.248424in}{5.240957in}}{\pgfqpoint{9.237825in}{5.236567in}}{\pgfqpoint{9.230011in}{5.228753in}}%
\pgfpathcurveto{\pgfqpoint{9.222197in}{5.220939in}}{\pgfqpoint{9.217807in}{5.210340in}}{\pgfqpoint{9.217807in}{5.199290in}}%
\pgfpathcurveto{\pgfqpoint{9.217807in}{5.188240in}}{\pgfqpoint{9.222197in}{5.177641in}}{\pgfqpoint{9.230011in}{5.169827in}}%
\pgfpathcurveto{\pgfqpoint{9.237825in}{5.162014in}}{\pgfqpoint{9.248424in}{5.157624in}}{\pgfqpoint{9.259474in}{5.157624in}}%
\pgfpathclose%
\pgfusepath{stroke,fill}%
\end{pgfscope}%
\begin{pgfscope}%
\pgfpathrectangle{\pgfqpoint{0.481978in}{0.331635in}}{\pgfqpoint{9.300000in}{7.700000in}}%
\pgfusepath{clip}%
\pgfsetbuttcap%
\pgfsetroundjoin%
\definecolor{currentfill}{rgb}{0.552941,0.898039,0.631373}%
\pgfsetfillcolor{currentfill}%
\pgfsetlinewidth{0.481800pt}%
\definecolor{currentstroke}{rgb}{1.000000,1.000000,1.000000}%
\pgfsetstrokecolor{currentstroke}%
\pgfsetdash{}{0pt}%
\pgfpathmoveto{\pgfqpoint{8.569592in}{4.750652in}}%
\pgfpathcurveto{\pgfqpoint{8.580643in}{4.750652in}}{\pgfqpoint{8.591242in}{4.755042in}}{\pgfqpoint{8.599055in}{4.762856in}}%
\pgfpathcurveto{\pgfqpoint{8.606869in}{4.770670in}}{\pgfqpoint{8.611259in}{4.781269in}}{\pgfqpoint{8.611259in}{4.792319in}}%
\pgfpathcurveto{\pgfqpoint{8.611259in}{4.803369in}}{\pgfqpoint{8.606869in}{4.813968in}}{\pgfqpoint{8.599055in}{4.821782in}}%
\pgfpathcurveto{\pgfqpoint{8.591242in}{4.829595in}}{\pgfqpoint{8.580643in}{4.833985in}}{\pgfqpoint{8.569592in}{4.833985in}}%
\pgfpathcurveto{\pgfqpoint{8.558542in}{4.833985in}}{\pgfqpoint{8.547943in}{4.829595in}}{\pgfqpoint{8.540130in}{4.821782in}}%
\pgfpathcurveto{\pgfqpoint{8.532316in}{4.813968in}}{\pgfqpoint{8.527926in}{4.803369in}}{\pgfqpoint{8.527926in}{4.792319in}}%
\pgfpathcurveto{\pgfqpoint{8.527926in}{4.781269in}}{\pgfqpoint{8.532316in}{4.770670in}}{\pgfqpoint{8.540130in}{4.762856in}}%
\pgfpathcurveto{\pgfqpoint{8.547943in}{4.755042in}}{\pgfqpoint{8.558542in}{4.750652in}}{\pgfqpoint{8.569592in}{4.750652in}}%
\pgfpathclose%
\pgfusepath{stroke,fill}%
\end{pgfscope}%
\begin{pgfscope}%
\pgfpathrectangle{\pgfqpoint{0.481978in}{0.331635in}}{\pgfqpoint{9.300000in}{7.700000in}}%
\pgfusepath{clip}%
\pgfsetbuttcap%
\pgfsetroundjoin%
\definecolor{currentfill}{rgb}{0.552941,0.898039,0.631373}%
\pgfsetfillcolor{currentfill}%
\pgfsetlinewidth{0.481800pt}%
\definecolor{currentstroke}{rgb}{1.000000,1.000000,1.000000}%
\pgfsetstrokecolor{currentstroke}%
\pgfsetdash{}{0pt}%
\pgfpathmoveto{\pgfqpoint{8.786772in}{4.529927in}}%
\pgfpathcurveto{\pgfqpoint{8.797823in}{4.529927in}}{\pgfqpoint{8.808422in}{4.534317in}}{\pgfqpoint{8.816235in}{4.542131in}}%
\pgfpathcurveto{\pgfqpoint{8.824049in}{4.549944in}}{\pgfqpoint{8.828439in}{4.560543in}}{\pgfqpoint{8.828439in}{4.571593in}}%
\pgfpathcurveto{\pgfqpoint{8.828439in}{4.582644in}}{\pgfqpoint{8.824049in}{4.593243in}}{\pgfqpoint{8.816235in}{4.601056in}}%
\pgfpathcurveto{\pgfqpoint{8.808422in}{4.608870in}}{\pgfqpoint{8.797823in}{4.613260in}}{\pgfqpoint{8.786772in}{4.613260in}}%
\pgfpathcurveto{\pgfqpoint{8.775722in}{4.613260in}}{\pgfqpoint{8.765123in}{4.608870in}}{\pgfqpoint{8.757310in}{4.601056in}}%
\pgfpathcurveto{\pgfqpoint{8.749496in}{4.593243in}}{\pgfqpoint{8.745106in}{4.582644in}}{\pgfqpoint{8.745106in}{4.571593in}}%
\pgfpathcurveto{\pgfqpoint{8.745106in}{4.560543in}}{\pgfqpoint{8.749496in}{4.549944in}}{\pgfqpoint{8.757310in}{4.542131in}}%
\pgfpathcurveto{\pgfqpoint{8.765123in}{4.534317in}}{\pgfqpoint{8.775722in}{4.529927in}}{\pgfqpoint{8.786772in}{4.529927in}}%
\pgfpathclose%
\pgfusepath{stroke,fill}%
\end{pgfscope}%
\begin{pgfscope}%
\pgfpathrectangle{\pgfqpoint{0.481978in}{0.331635in}}{\pgfqpoint{9.300000in}{7.700000in}}%
\pgfusepath{clip}%
\pgfsetbuttcap%
\pgfsetroundjoin%
\definecolor{currentfill}{rgb}{0.552941,0.898039,0.631373}%
\pgfsetfillcolor{currentfill}%
\pgfsetlinewidth{0.481800pt}%
\definecolor{currentstroke}{rgb}{1.000000,1.000000,1.000000}%
\pgfsetstrokecolor{currentstroke}%
\pgfsetdash{}{0pt}%
\pgfpathmoveto{\pgfqpoint{7.733235in}{4.864578in}}%
\pgfpathcurveto{\pgfqpoint{7.744285in}{4.864578in}}{\pgfqpoint{7.754884in}{4.868968in}}{\pgfqpoint{7.762698in}{4.876782in}}%
\pgfpathcurveto{\pgfqpoint{7.770512in}{4.884595in}}{\pgfqpoint{7.774902in}{4.895194in}}{\pgfqpoint{7.774902in}{4.906244in}}%
\pgfpathcurveto{\pgfqpoint{7.774902in}{4.917294in}}{\pgfqpoint{7.770512in}{4.927893in}}{\pgfqpoint{7.762698in}{4.935707in}}%
\pgfpathcurveto{\pgfqpoint{7.754884in}{4.943521in}}{\pgfqpoint{7.744285in}{4.947911in}}{\pgfqpoint{7.733235in}{4.947911in}}%
\pgfpathcurveto{\pgfqpoint{7.722185in}{4.947911in}}{\pgfqpoint{7.711586in}{4.943521in}}{\pgfqpoint{7.703772in}{4.935707in}}%
\pgfpathcurveto{\pgfqpoint{7.695959in}{4.927893in}}{\pgfqpoint{7.691568in}{4.917294in}}{\pgfqpoint{7.691568in}{4.906244in}}%
\pgfpathcurveto{\pgfqpoint{7.691568in}{4.895194in}}{\pgfqpoint{7.695959in}{4.884595in}}{\pgfqpoint{7.703772in}{4.876782in}}%
\pgfpathcurveto{\pgfqpoint{7.711586in}{4.868968in}}{\pgfqpoint{7.722185in}{4.864578in}}{\pgfqpoint{7.733235in}{4.864578in}}%
\pgfpathclose%
\pgfusepath{stroke,fill}%
\end{pgfscope}%
\begin{pgfscope}%
\pgfpathrectangle{\pgfqpoint{0.481978in}{0.331635in}}{\pgfqpoint{9.300000in}{7.700000in}}%
\pgfusepath{clip}%
\pgfsetbuttcap%
\pgfsetroundjoin%
\definecolor{currentfill}{rgb}{0.552941,0.898039,0.631373}%
\pgfsetfillcolor{currentfill}%
\pgfsetlinewidth{0.481800pt}%
\definecolor{currentstroke}{rgb}{1.000000,1.000000,1.000000}%
\pgfsetstrokecolor{currentstroke}%
\pgfsetdash{}{0pt}%
\pgfpathmoveto{\pgfqpoint{5.023248in}{5.790519in}}%
\pgfpathcurveto{\pgfqpoint{5.034298in}{5.790519in}}{\pgfqpoint{5.044897in}{5.794910in}}{\pgfqpoint{5.052710in}{5.802723in}}%
\pgfpathcurveto{\pgfqpoint{5.060524in}{5.810537in}}{\pgfqpoint{5.064914in}{5.821136in}}{\pgfqpoint{5.064914in}{5.832186in}}%
\pgfpathcurveto{\pgfqpoint{5.064914in}{5.843236in}}{\pgfqpoint{5.060524in}{5.853835in}}{\pgfqpoint{5.052710in}{5.861649in}}%
\pgfpathcurveto{\pgfqpoint{5.044897in}{5.869462in}}{\pgfqpoint{5.034298in}{5.873853in}}{\pgfqpoint{5.023248in}{5.873853in}}%
\pgfpathcurveto{\pgfqpoint{5.012198in}{5.873853in}}{\pgfqpoint{5.001599in}{5.869462in}}{\pgfqpoint{4.993785in}{5.861649in}}%
\pgfpathcurveto{\pgfqpoint{4.985971in}{5.853835in}}{\pgfqpoint{4.981581in}{5.843236in}}{\pgfqpoint{4.981581in}{5.832186in}}%
\pgfpathcurveto{\pgfqpoint{4.981581in}{5.821136in}}{\pgfqpoint{4.985971in}{5.810537in}}{\pgfqpoint{4.993785in}{5.802723in}}%
\pgfpathcurveto{\pgfqpoint{5.001599in}{5.794910in}}{\pgfqpoint{5.012198in}{5.790519in}}{\pgfqpoint{5.023248in}{5.790519in}}%
\pgfpathclose%
\pgfusepath{stroke,fill}%
\end{pgfscope}%
\begin{pgfscope}%
\pgfpathrectangle{\pgfqpoint{0.481978in}{0.331635in}}{\pgfqpoint{9.300000in}{7.700000in}}%
\pgfusepath{clip}%
\pgfsetbuttcap%
\pgfsetroundjoin%
\definecolor{currentfill}{rgb}{0.552941,0.898039,0.631373}%
\pgfsetfillcolor{currentfill}%
\pgfsetlinewidth{0.481800pt}%
\definecolor{currentstroke}{rgb}{1.000000,1.000000,1.000000}%
\pgfsetstrokecolor{currentstroke}%
\pgfsetdash{}{0pt}%
\pgfpathmoveto{\pgfqpoint{4.955252in}{6.232287in}}%
\pgfpathcurveto{\pgfqpoint{4.966302in}{6.232287in}}{\pgfqpoint{4.976901in}{6.236677in}}{\pgfqpoint{4.984715in}{6.244490in}}%
\pgfpathcurveto{\pgfqpoint{4.992528in}{6.252304in}}{\pgfqpoint{4.996918in}{6.262903in}}{\pgfqpoint{4.996918in}{6.273953in}}%
\pgfpathcurveto{\pgfqpoint{4.996918in}{6.285003in}}{\pgfqpoint{4.992528in}{6.295602in}}{\pgfqpoint{4.984715in}{6.303416in}}%
\pgfpathcurveto{\pgfqpoint{4.976901in}{6.311230in}}{\pgfqpoint{4.966302in}{6.315620in}}{\pgfqpoint{4.955252in}{6.315620in}}%
\pgfpathcurveto{\pgfqpoint{4.944202in}{6.315620in}}{\pgfqpoint{4.933603in}{6.311230in}}{\pgfqpoint{4.925789in}{6.303416in}}%
\pgfpathcurveto{\pgfqpoint{4.917975in}{6.295602in}}{\pgfqpoint{4.913585in}{6.285003in}}{\pgfqpoint{4.913585in}{6.273953in}}%
\pgfpathcurveto{\pgfqpoint{4.913585in}{6.262903in}}{\pgfqpoint{4.917975in}{6.252304in}}{\pgfqpoint{4.925789in}{6.244490in}}%
\pgfpathcurveto{\pgfqpoint{4.933603in}{6.236677in}}{\pgfqpoint{4.944202in}{6.232287in}}{\pgfqpoint{4.955252in}{6.232287in}}%
\pgfpathclose%
\pgfusepath{stroke,fill}%
\end{pgfscope}%
\begin{pgfscope}%
\pgfpathrectangle{\pgfqpoint{0.481978in}{0.331635in}}{\pgfqpoint{9.300000in}{7.700000in}}%
\pgfusepath{clip}%
\pgfsetbuttcap%
\pgfsetroundjoin%
\definecolor{currentfill}{rgb}{0.552941,0.898039,0.631373}%
\pgfsetfillcolor{currentfill}%
\pgfsetlinewidth{0.481800pt}%
\definecolor{currentstroke}{rgb}{1.000000,1.000000,1.000000}%
\pgfsetstrokecolor{currentstroke}%
\pgfsetdash{}{0pt}%
\pgfpathmoveto{\pgfqpoint{7.433098in}{2.223941in}}%
\pgfpathcurveto{\pgfqpoint{7.444148in}{2.223941in}}{\pgfqpoint{7.454747in}{2.228332in}}{\pgfqpoint{7.462561in}{2.236145in}}%
\pgfpathcurveto{\pgfqpoint{7.470375in}{2.243959in}}{\pgfqpoint{7.474765in}{2.254558in}}{\pgfqpoint{7.474765in}{2.265608in}}%
\pgfpathcurveto{\pgfqpoint{7.474765in}{2.276658in}}{\pgfqpoint{7.470375in}{2.287257in}}{\pgfqpoint{7.462561in}{2.295071in}}%
\pgfpathcurveto{\pgfqpoint{7.454747in}{2.302884in}}{\pgfqpoint{7.444148in}{2.307275in}}{\pgfqpoint{7.433098in}{2.307275in}}%
\pgfpathcurveto{\pgfqpoint{7.422048in}{2.307275in}}{\pgfqpoint{7.411449in}{2.302884in}}{\pgfqpoint{7.403635in}{2.295071in}}%
\pgfpathcurveto{\pgfqpoint{7.395822in}{2.287257in}}{\pgfqpoint{7.391431in}{2.276658in}}{\pgfqpoint{7.391431in}{2.265608in}}%
\pgfpathcurveto{\pgfqpoint{7.391431in}{2.254558in}}{\pgfqpoint{7.395822in}{2.243959in}}{\pgfqpoint{7.403635in}{2.236145in}}%
\pgfpathcurveto{\pgfqpoint{7.411449in}{2.228332in}}{\pgfqpoint{7.422048in}{2.223941in}}{\pgfqpoint{7.433098in}{2.223941in}}%
\pgfpathclose%
\pgfusepath{stroke,fill}%
\end{pgfscope}%
\begin{pgfscope}%
\pgfpathrectangle{\pgfqpoint{0.481978in}{0.331635in}}{\pgfqpoint{9.300000in}{7.700000in}}%
\pgfusepath{clip}%
\pgfsetbuttcap%
\pgfsetroundjoin%
\definecolor{currentfill}{rgb}{0.552941,0.898039,0.631373}%
\pgfsetfillcolor{currentfill}%
\pgfsetlinewidth{0.481800pt}%
\definecolor{currentstroke}{rgb}{1.000000,1.000000,1.000000}%
\pgfsetstrokecolor{currentstroke}%
\pgfsetdash{}{0pt}%
\pgfpathmoveto{\pgfqpoint{8.232070in}{5.773420in}}%
\pgfpathcurveto{\pgfqpoint{8.243120in}{5.773420in}}{\pgfqpoint{8.253719in}{5.777810in}}{\pgfqpoint{8.261532in}{5.785624in}}%
\pgfpathcurveto{\pgfqpoint{8.269346in}{5.793438in}}{\pgfqpoint{8.273736in}{5.804037in}}{\pgfqpoint{8.273736in}{5.815087in}}%
\pgfpathcurveto{\pgfqpoint{8.273736in}{5.826137in}}{\pgfqpoint{8.269346in}{5.836736in}}{\pgfqpoint{8.261532in}{5.844550in}}%
\pgfpathcurveto{\pgfqpoint{8.253719in}{5.852363in}}{\pgfqpoint{8.243120in}{5.856753in}}{\pgfqpoint{8.232070in}{5.856753in}}%
\pgfpathcurveto{\pgfqpoint{8.221019in}{5.856753in}}{\pgfqpoint{8.210420in}{5.852363in}}{\pgfqpoint{8.202607in}{5.844550in}}%
\pgfpathcurveto{\pgfqpoint{8.194793in}{5.836736in}}{\pgfqpoint{8.190403in}{5.826137in}}{\pgfqpoint{8.190403in}{5.815087in}}%
\pgfpathcurveto{\pgfqpoint{8.190403in}{5.804037in}}{\pgfqpoint{8.194793in}{5.793438in}}{\pgfqpoint{8.202607in}{5.785624in}}%
\pgfpathcurveto{\pgfqpoint{8.210420in}{5.777810in}}{\pgfqpoint{8.221019in}{5.773420in}}{\pgfqpoint{8.232070in}{5.773420in}}%
\pgfpathclose%
\pgfusepath{stroke,fill}%
\end{pgfscope}%
\begin{pgfscope}%
\pgfpathrectangle{\pgfqpoint{0.481978in}{0.331635in}}{\pgfqpoint{9.300000in}{7.700000in}}%
\pgfusepath{clip}%
\pgfsetbuttcap%
\pgfsetroundjoin%
\definecolor{currentfill}{rgb}{0.552941,0.898039,0.631373}%
\pgfsetfillcolor{currentfill}%
\pgfsetlinewidth{0.481800pt}%
\definecolor{currentstroke}{rgb}{1.000000,1.000000,1.000000}%
\pgfsetstrokecolor{currentstroke}%
\pgfsetdash{}{0pt}%
\pgfpathmoveto{\pgfqpoint{3.011222in}{1.643530in}}%
\pgfpathcurveto{\pgfqpoint{3.022272in}{1.643530in}}{\pgfqpoint{3.032871in}{1.647920in}}{\pgfqpoint{3.040685in}{1.655734in}}%
\pgfpathcurveto{\pgfqpoint{3.048498in}{1.663547in}}{\pgfqpoint{3.052889in}{1.674146in}}{\pgfqpoint{3.052889in}{1.685197in}}%
\pgfpathcurveto{\pgfqpoint{3.052889in}{1.696247in}}{\pgfqpoint{3.048498in}{1.706846in}}{\pgfqpoint{3.040685in}{1.714659in}}%
\pgfpathcurveto{\pgfqpoint{3.032871in}{1.722473in}}{\pgfqpoint{3.022272in}{1.726863in}}{\pgfqpoint{3.011222in}{1.726863in}}%
\pgfpathcurveto{\pgfqpoint{3.000172in}{1.726863in}}{\pgfqpoint{2.989573in}{1.722473in}}{\pgfqpoint{2.981759in}{1.714659in}}%
\pgfpathcurveto{\pgfqpoint{2.973946in}{1.706846in}}{\pgfqpoint{2.969555in}{1.696247in}}{\pgfqpoint{2.969555in}{1.685197in}}%
\pgfpathcurveto{\pgfqpoint{2.969555in}{1.674146in}}{\pgfqpoint{2.973946in}{1.663547in}}{\pgfqpoint{2.981759in}{1.655734in}}%
\pgfpathcurveto{\pgfqpoint{2.989573in}{1.647920in}}{\pgfqpoint{3.000172in}{1.643530in}}{\pgfqpoint{3.011222in}{1.643530in}}%
\pgfpathclose%
\pgfusepath{stroke,fill}%
\end{pgfscope}%
\begin{pgfscope}%
\pgfpathrectangle{\pgfqpoint{0.481978in}{0.331635in}}{\pgfqpoint{9.300000in}{7.700000in}}%
\pgfusepath{clip}%
\pgfsetbuttcap%
\pgfsetroundjoin%
\definecolor{currentfill}{rgb}{0.552941,0.898039,0.631373}%
\pgfsetfillcolor{currentfill}%
\pgfsetlinewidth{0.481800pt}%
\definecolor{currentstroke}{rgb}{1.000000,1.000000,1.000000}%
\pgfsetstrokecolor{currentstroke}%
\pgfsetdash{}{0pt}%
\pgfpathmoveto{\pgfqpoint{4.782892in}{6.107152in}}%
\pgfpathcurveto{\pgfqpoint{4.793942in}{6.107152in}}{\pgfqpoint{4.804541in}{6.111542in}}{\pgfqpoint{4.812355in}{6.119355in}}%
\pgfpathcurveto{\pgfqpoint{4.820169in}{6.127169in}}{\pgfqpoint{4.824559in}{6.137768in}}{\pgfqpoint{4.824559in}{6.148818in}}%
\pgfpathcurveto{\pgfqpoint{4.824559in}{6.159868in}}{\pgfqpoint{4.820169in}{6.170467in}}{\pgfqpoint{4.812355in}{6.178281in}}%
\pgfpathcurveto{\pgfqpoint{4.804541in}{6.186095in}}{\pgfqpoint{4.793942in}{6.190485in}}{\pgfqpoint{4.782892in}{6.190485in}}%
\pgfpathcurveto{\pgfqpoint{4.771842in}{6.190485in}}{\pgfqpoint{4.761243in}{6.186095in}}{\pgfqpoint{4.753429in}{6.178281in}}%
\pgfpathcurveto{\pgfqpoint{4.745616in}{6.170467in}}{\pgfqpoint{4.741225in}{6.159868in}}{\pgfqpoint{4.741225in}{6.148818in}}%
\pgfpathcurveto{\pgfqpoint{4.741225in}{6.137768in}}{\pgfqpoint{4.745616in}{6.127169in}}{\pgfqpoint{4.753429in}{6.119355in}}%
\pgfpathcurveto{\pgfqpoint{4.761243in}{6.111542in}}{\pgfqpoint{4.771842in}{6.107152in}}{\pgfqpoint{4.782892in}{6.107152in}}%
\pgfpathclose%
\pgfusepath{stroke,fill}%
\end{pgfscope}%
\begin{pgfscope}%
\pgfpathrectangle{\pgfqpoint{0.481978in}{0.331635in}}{\pgfqpoint{9.300000in}{7.700000in}}%
\pgfusepath{clip}%
\pgfsetbuttcap%
\pgfsetroundjoin%
\definecolor{currentfill}{rgb}{0.552941,0.898039,0.631373}%
\pgfsetfillcolor{currentfill}%
\pgfsetlinewidth{0.481800pt}%
\definecolor{currentstroke}{rgb}{1.000000,1.000000,1.000000}%
\pgfsetstrokecolor{currentstroke}%
\pgfsetdash{}{0pt}%
\pgfpathmoveto{\pgfqpoint{3.628307in}{5.255411in}}%
\pgfpathcurveto{\pgfqpoint{3.639357in}{5.255411in}}{\pgfqpoint{3.649956in}{5.259801in}}{\pgfqpoint{3.657770in}{5.267615in}}%
\pgfpathcurveto{\pgfqpoint{3.665584in}{5.275428in}}{\pgfqpoint{3.669974in}{5.286028in}}{\pgfqpoint{3.669974in}{5.297078in}}%
\pgfpathcurveto{\pgfqpoint{3.669974in}{5.308128in}}{\pgfqpoint{3.665584in}{5.318727in}}{\pgfqpoint{3.657770in}{5.326540in}}%
\pgfpathcurveto{\pgfqpoint{3.649956in}{5.334354in}}{\pgfqpoint{3.639357in}{5.338744in}}{\pgfqpoint{3.628307in}{5.338744in}}%
\pgfpathcurveto{\pgfqpoint{3.617257in}{5.338744in}}{\pgfqpoint{3.606658in}{5.334354in}}{\pgfqpoint{3.598845in}{5.326540in}}%
\pgfpathcurveto{\pgfqpoint{3.591031in}{5.318727in}}{\pgfqpoint{3.586641in}{5.308128in}}{\pgfqpoint{3.586641in}{5.297078in}}%
\pgfpathcurveto{\pgfqpoint{3.586641in}{5.286028in}}{\pgfqpoint{3.591031in}{5.275428in}}{\pgfqpoint{3.598845in}{5.267615in}}%
\pgfpathcurveto{\pgfqpoint{3.606658in}{5.259801in}}{\pgfqpoint{3.617257in}{5.255411in}}{\pgfqpoint{3.628307in}{5.255411in}}%
\pgfpathclose%
\pgfusepath{stroke,fill}%
\end{pgfscope}%
\begin{pgfscope}%
\pgfpathrectangle{\pgfqpoint{0.481978in}{0.331635in}}{\pgfqpoint{9.300000in}{7.700000in}}%
\pgfusepath{clip}%
\pgfsetbuttcap%
\pgfsetroundjoin%
\definecolor{currentfill}{rgb}{0.552941,0.898039,0.631373}%
\pgfsetfillcolor{currentfill}%
\pgfsetlinewidth{0.481800pt}%
\definecolor{currentstroke}{rgb}{1.000000,1.000000,1.000000}%
\pgfsetstrokecolor{currentstroke}%
\pgfsetdash{}{0pt}%
\pgfpathmoveto{\pgfqpoint{3.687739in}{5.199691in}}%
\pgfpathcurveto{\pgfqpoint{3.698789in}{5.199691in}}{\pgfqpoint{3.709388in}{5.204081in}}{\pgfqpoint{3.717201in}{5.211895in}}%
\pgfpathcurveto{\pgfqpoint{3.725015in}{5.219708in}}{\pgfqpoint{3.729405in}{5.230307in}}{\pgfqpoint{3.729405in}{5.241357in}}%
\pgfpathcurveto{\pgfqpoint{3.729405in}{5.252408in}}{\pgfqpoint{3.725015in}{5.263007in}}{\pgfqpoint{3.717201in}{5.270820in}}%
\pgfpathcurveto{\pgfqpoint{3.709388in}{5.278634in}}{\pgfqpoint{3.698789in}{5.283024in}}{\pgfqpoint{3.687739in}{5.283024in}}%
\pgfpathcurveto{\pgfqpoint{3.676689in}{5.283024in}}{\pgfqpoint{3.666090in}{5.278634in}}{\pgfqpoint{3.658276in}{5.270820in}}%
\pgfpathcurveto{\pgfqpoint{3.650462in}{5.263007in}}{\pgfqpoint{3.646072in}{5.252408in}}{\pgfqpoint{3.646072in}{5.241357in}}%
\pgfpathcurveto{\pgfqpoint{3.646072in}{5.230307in}}{\pgfqpoint{3.650462in}{5.219708in}}{\pgfqpoint{3.658276in}{5.211895in}}%
\pgfpathcurveto{\pgfqpoint{3.666090in}{5.204081in}}{\pgfqpoint{3.676689in}{5.199691in}}{\pgfqpoint{3.687739in}{5.199691in}}%
\pgfpathclose%
\pgfusepath{stroke,fill}%
\end{pgfscope}%
\begin{pgfscope}%
\pgfpathrectangle{\pgfqpoint{0.481978in}{0.331635in}}{\pgfqpoint{9.300000in}{7.700000in}}%
\pgfusepath{clip}%
\pgfsetbuttcap%
\pgfsetroundjoin%
\definecolor{currentfill}{rgb}{0.552941,0.898039,0.631373}%
\pgfsetfillcolor{currentfill}%
\pgfsetlinewidth{0.481800pt}%
\definecolor{currentstroke}{rgb}{1.000000,1.000000,1.000000}%
\pgfsetstrokecolor{currentstroke}%
\pgfsetdash{}{0pt}%
\pgfpathmoveto{\pgfqpoint{7.445589in}{2.785107in}}%
\pgfpathcurveto{\pgfqpoint{7.456639in}{2.785107in}}{\pgfqpoint{7.467238in}{2.789497in}}{\pgfqpoint{7.475051in}{2.797311in}}%
\pgfpathcurveto{\pgfqpoint{7.482865in}{2.805124in}}{\pgfqpoint{7.487255in}{2.815723in}}{\pgfqpoint{7.487255in}{2.826774in}}%
\pgfpathcurveto{\pgfqpoint{7.487255in}{2.837824in}}{\pgfqpoint{7.482865in}{2.848423in}}{\pgfqpoint{7.475051in}{2.856236in}}%
\pgfpathcurveto{\pgfqpoint{7.467238in}{2.864050in}}{\pgfqpoint{7.456639in}{2.868440in}}{\pgfqpoint{7.445589in}{2.868440in}}%
\pgfpathcurveto{\pgfqpoint{7.434539in}{2.868440in}}{\pgfqpoint{7.423940in}{2.864050in}}{\pgfqpoint{7.416126in}{2.856236in}}%
\pgfpathcurveto{\pgfqpoint{7.408312in}{2.848423in}}{\pgfqpoint{7.403922in}{2.837824in}}{\pgfqpoint{7.403922in}{2.826774in}}%
\pgfpathcurveto{\pgfqpoint{7.403922in}{2.815723in}}{\pgfqpoint{7.408312in}{2.805124in}}{\pgfqpoint{7.416126in}{2.797311in}}%
\pgfpathcurveto{\pgfqpoint{7.423940in}{2.789497in}}{\pgfqpoint{7.434539in}{2.785107in}}{\pgfqpoint{7.445589in}{2.785107in}}%
\pgfpathclose%
\pgfusepath{stroke,fill}%
\end{pgfscope}%
\begin{pgfscope}%
\pgfpathrectangle{\pgfqpoint{0.481978in}{0.331635in}}{\pgfqpoint{9.300000in}{7.700000in}}%
\pgfusepath{clip}%
\pgfsetbuttcap%
\pgfsetroundjoin%
\definecolor{currentfill}{rgb}{0.552941,0.898039,0.631373}%
\pgfsetfillcolor{currentfill}%
\pgfsetlinewidth{0.481800pt}%
\definecolor{currentstroke}{rgb}{1.000000,1.000000,1.000000}%
\pgfsetstrokecolor{currentstroke}%
\pgfsetdash{}{0pt}%
\pgfpathmoveto{\pgfqpoint{8.934791in}{5.101472in}}%
\pgfpathcurveto{\pgfqpoint{8.945841in}{5.101472in}}{\pgfqpoint{8.956440in}{5.105862in}}{\pgfqpoint{8.964253in}{5.113676in}}%
\pgfpathcurveto{\pgfqpoint{8.972067in}{5.121489in}}{\pgfqpoint{8.976457in}{5.132088in}}{\pgfqpoint{8.976457in}{5.143139in}}%
\pgfpathcurveto{\pgfqpoint{8.976457in}{5.154189in}}{\pgfqpoint{8.972067in}{5.164788in}}{\pgfqpoint{8.964253in}{5.172601in}}%
\pgfpathcurveto{\pgfqpoint{8.956440in}{5.180415in}}{\pgfqpoint{8.945841in}{5.184805in}}{\pgfqpoint{8.934791in}{5.184805in}}%
\pgfpathcurveto{\pgfqpoint{8.923741in}{5.184805in}}{\pgfqpoint{8.913142in}{5.180415in}}{\pgfqpoint{8.905328in}{5.172601in}}%
\pgfpathcurveto{\pgfqpoint{8.897514in}{5.164788in}}{\pgfqpoint{8.893124in}{5.154189in}}{\pgfqpoint{8.893124in}{5.143139in}}%
\pgfpathcurveto{\pgfqpoint{8.893124in}{5.132088in}}{\pgfqpoint{8.897514in}{5.121489in}}{\pgfqpoint{8.905328in}{5.113676in}}%
\pgfpathcurveto{\pgfqpoint{8.913142in}{5.105862in}}{\pgfqpoint{8.923741in}{5.101472in}}{\pgfqpoint{8.934791in}{5.101472in}}%
\pgfpathclose%
\pgfusepath{stroke,fill}%
\end{pgfscope}%
\begin{pgfscope}%
\pgfpathrectangle{\pgfqpoint{0.481978in}{0.331635in}}{\pgfqpoint{9.300000in}{7.700000in}}%
\pgfusepath{clip}%
\pgfsetbuttcap%
\pgfsetroundjoin%
\definecolor{currentfill}{rgb}{0.552941,0.898039,0.631373}%
\pgfsetfillcolor{currentfill}%
\pgfsetlinewidth{0.481800pt}%
\definecolor{currentstroke}{rgb}{1.000000,1.000000,1.000000}%
\pgfsetstrokecolor{currentstroke}%
\pgfsetdash{}{0pt}%
\pgfpathmoveto{\pgfqpoint{3.508572in}{3.680026in}}%
\pgfpathcurveto{\pgfqpoint{3.519622in}{3.680026in}}{\pgfqpoint{3.530221in}{3.684417in}}{\pgfqpoint{3.538035in}{3.692230in}}%
\pgfpathcurveto{\pgfqpoint{3.545848in}{3.700044in}}{\pgfqpoint{3.550239in}{3.710643in}}{\pgfqpoint{3.550239in}{3.721693in}}%
\pgfpathcurveto{\pgfqpoint{3.550239in}{3.732743in}}{\pgfqpoint{3.545848in}{3.743342in}}{\pgfqpoint{3.538035in}{3.751156in}}%
\pgfpathcurveto{\pgfqpoint{3.530221in}{3.758969in}}{\pgfqpoint{3.519622in}{3.763360in}}{\pgfqpoint{3.508572in}{3.763360in}}%
\pgfpathcurveto{\pgfqpoint{3.497522in}{3.763360in}}{\pgfqpoint{3.486923in}{3.758969in}}{\pgfqpoint{3.479109in}{3.751156in}}%
\pgfpathcurveto{\pgfqpoint{3.471295in}{3.743342in}}{\pgfqpoint{3.466905in}{3.732743in}}{\pgfqpoint{3.466905in}{3.721693in}}%
\pgfpathcurveto{\pgfqpoint{3.466905in}{3.710643in}}{\pgfqpoint{3.471295in}{3.700044in}}{\pgfqpoint{3.479109in}{3.692230in}}%
\pgfpathcurveto{\pgfqpoint{3.486923in}{3.684417in}}{\pgfqpoint{3.497522in}{3.680026in}}{\pgfqpoint{3.508572in}{3.680026in}}%
\pgfpathclose%
\pgfusepath{stroke,fill}%
\end{pgfscope}%
\begin{pgfscope}%
\pgfpathrectangle{\pgfqpoint{0.481978in}{0.331635in}}{\pgfqpoint{9.300000in}{7.700000in}}%
\pgfusepath{clip}%
\pgfsetbuttcap%
\pgfsetroundjoin%
\definecolor{currentfill}{rgb}{0.552941,0.898039,0.631373}%
\pgfsetfillcolor{currentfill}%
\pgfsetlinewidth{0.481800pt}%
\definecolor{currentstroke}{rgb}{1.000000,1.000000,1.000000}%
\pgfsetstrokecolor{currentstroke}%
\pgfsetdash{}{0pt}%
\pgfpathmoveto{\pgfqpoint{4.192874in}{1.517858in}}%
\pgfpathcurveto{\pgfqpoint{4.203924in}{1.517858in}}{\pgfqpoint{4.214523in}{1.522248in}}{\pgfqpoint{4.222337in}{1.530062in}}%
\pgfpathcurveto{\pgfqpoint{4.230151in}{1.537875in}}{\pgfqpoint{4.234541in}{1.548474in}}{\pgfqpoint{4.234541in}{1.559524in}}%
\pgfpathcurveto{\pgfqpoint{4.234541in}{1.570575in}}{\pgfqpoint{4.230151in}{1.581174in}}{\pgfqpoint{4.222337in}{1.588987in}}%
\pgfpathcurveto{\pgfqpoint{4.214523in}{1.596801in}}{\pgfqpoint{4.203924in}{1.601191in}}{\pgfqpoint{4.192874in}{1.601191in}}%
\pgfpathcurveto{\pgfqpoint{4.181824in}{1.601191in}}{\pgfqpoint{4.171225in}{1.596801in}}{\pgfqpoint{4.163411in}{1.588987in}}%
\pgfpathcurveto{\pgfqpoint{4.155598in}{1.581174in}}{\pgfqpoint{4.151207in}{1.570575in}}{\pgfqpoint{4.151207in}{1.559524in}}%
\pgfpathcurveto{\pgfqpoint{4.151207in}{1.548474in}}{\pgfqpoint{4.155598in}{1.537875in}}{\pgfqpoint{4.163411in}{1.530062in}}%
\pgfpathcurveto{\pgfqpoint{4.171225in}{1.522248in}}{\pgfqpoint{4.181824in}{1.517858in}}{\pgfqpoint{4.192874in}{1.517858in}}%
\pgfpathclose%
\pgfusepath{stroke,fill}%
\end{pgfscope}%
\begin{pgfscope}%
\pgfpathrectangle{\pgfqpoint{0.481978in}{0.331635in}}{\pgfqpoint{9.300000in}{7.700000in}}%
\pgfusepath{clip}%
\pgfsetbuttcap%
\pgfsetroundjoin%
\definecolor{currentfill}{rgb}{0.552941,0.898039,0.631373}%
\pgfsetfillcolor{currentfill}%
\pgfsetlinewidth{0.481800pt}%
\definecolor{currentstroke}{rgb}{1.000000,1.000000,1.000000}%
\pgfsetstrokecolor{currentstroke}%
\pgfsetdash{}{0pt}%
\pgfpathmoveto{\pgfqpoint{4.001930in}{4.990969in}}%
\pgfpathcurveto{\pgfqpoint{4.012980in}{4.990969in}}{\pgfqpoint{4.023579in}{4.995360in}}{\pgfqpoint{4.031393in}{5.003173in}}%
\pgfpathcurveto{\pgfqpoint{4.039206in}{5.010987in}}{\pgfqpoint{4.043597in}{5.021586in}}{\pgfqpoint{4.043597in}{5.032636in}}%
\pgfpathcurveto{\pgfqpoint{4.043597in}{5.043686in}}{\pgfqpoint{4.039206in}{5.054285in}}{\pgfqpoint{4.031393in}{5.062099in}}%
\pgfpathcurveto{\pgfqpoint{4.023579in}{5.069912in}}{\pgfqpoint{4.012980in}{5.074303in}}{\pgfqpoint{4.001930in}{5.074303in}}%
\pgfpathcurveto{\pgfqpoint{3.990880in}{5.074303in}}{\pgfqpoint{3.980281in}{5.069912in}}{\pgfqpoint{3.972467in}{5.062099in}}%
\pgfpathcurveto{\pgfqpoint{3.964654in}{5.054285in}}{\pgfqpoint{3.960263in}{5.043686in}}{\pgfqpoint{3.960263in}{5.032636in}}%
\pgfpathcurveto{\pgfqpoint{3.960263in}{5.021586in}}{\pgfqpoint{3.964654in}{5.010987in}}{\pgfqpoint{3.972467in}{5.003173in}}%
\pgfpathcurveto{\pgfqpoint{3.980281in}{4.995360in}}{\pgfqpoint{3.990880in}{4.990969in}}{\pgfqpoint{4.001930in}{4.990969in}}%
\pgfpathclose%
\pgfusepath{stroke,fill}%
\end{pgfscope}%
\begin{pgfscope}%
\pgfpathrectangle{\pgfqpoint{0.481978in}{0.331635in}}{\pgfqpoint{9.300000in}{7.700000in}}%
\pgfusepath{clip}%
\pgfsetbuttcap%
\pgfsetroundjoin%
\definecolor{currentfill}{rgb}{0.552941,0.898039,0.631373}%
\pgfsetfillcolor{currentfill}%
\pgfsetlinewidth{0.481800pt}%
\definecolor{currentstroke}{rgb}{1.000000,1.000000,1.000000}%
\pgfsetstrokecolor{currentstroke}%
\pgfsetdash{}{0pt}%
\pgfpathmoveto{\pgfqpoint{6.060999in}{1.751246in}}%
\pgfpathcurveto{\pgfqpoint{6.072049in}{1.751246in}}{\pgfqpoint{6.082648in}{1.755636in}}{\pgfqpoint{6.090462in}{1.763449in}}%
\pgfpathcurveto{\pgfqpoint{6.098275in}{1.771263in}}{\pgfqpoint{6.102666in}{1.781862in}}{\pgfqpoint{6.102666in}{1.792912in}}%
\pgfpathcurveto{\pgfqpoint{6.102666in}{1.803962in}}{\pgfqpoint{6.098275in}{1.814561in}}{\pgfqpoint{6.090462in}{1.822375in}}%
\pgfpathcurveto{\pgfqpoint{6.082648in}{1.830189in}}{\pgfqpoint{6.072049in}{1.834579in}}{\pgfqpoint{6.060999in}{1.834579in}}%
\pgfpathcurveto{\pgfqpoint{6.049949in}{1.834579in}}{\pgfqpoint{6.039350in}{1.830189in}}{\pgfqpoint{6.031536in}{1.822375in}}%
\pgfpathcurveto{\pgfqpoint{6.023723in}{1.814561in}}{\pgfqpoint{6.019332in}{1.803962in}}{\pgfqpoint{6.019332in}{1.792912in}}%
\pgfpathcurveto{\pgfqpoint{6.019332in}{1.781862in}}{\pgfqpoint{6.023723in}{1.771263in}}{\pgfqpoint{6.031536in}{1.763449in}}%
\pgfpathcurveto{\pgfqpoint{6.039350in}{1.755636in}}{\pgfqpoint{6.049949in}{1.751246in}}{\pgfqpoint{6.060999in}{1.751246in}}%
\pgfpathclose%
\pgfusepath{stroke,fill}%
\end{pgfscope}%
\begin{pgfscope}%
\pgfpathrectangle{\pgfqpoint{0.481978in}{0.331635in}}{\pgfqpoint{9.300000in}{7.700000in}}%
\pgfusepath{clip}%
\pgfsetbuttcap%
\pgfsetroundjoin%
\definecolor{currentfill}{rgb}{0.552941,0.898039,0.631373}%
\pgfsetfillcolor{currentfill}%
\pgfsetlinewidth{0.481800pt}%
\definecolor{currentstroke}{rgb}{1.000000,1.000000,1.000000}%
\pgfsetstrokecolor{currentstroke}%
\pgfsetdash{}{0pt}%
\pgfpathmoveto{\pgfqpoint{2.449894in}{1.656799in}}%
\pgfpathcurveto{\pgfqpoint{2.460944in}{1.656799in}}{\pgfqpoint{2.471543in}{1.661190in}}{\pgfqpoint{2.479357in}{1.669003in}}%
\pgfpathcurveto{\pgfqpoint{2.487170in}{1.676817in}}{\pgfqpoint{2.491561in}{1.687416in}}{\pgfqpoint{2.491561in}{1.698466in}}%
\pgfpathcurveto{\pgfqpoint{2.491561in}{1.709516in}}{\pgfqpoint{2.487170in}{1.720115in}}{\pgfqpoint{2.479357in}{1.727929in}}%
\pgfpathcurveto{\pgfqpoint{2.471543in}{1.735742in}}{\pgfqpoint{2.460944in}{1.740133in}}{\pgfqpoint{2.449894in}{1.740133in}}%
\pgfpathcurveto{\pgfqpoint{2.438844in}{1.740133in}}{\pgfqpoint{2.428245in}{1.735742in}}{\pgfqpoint{2.420431in}{1.727929in}}%
\pgfpathcurveto{\pgfqpoint{2.412617in}{1.720115in}}{\pgfqpoint{2.408227in}{1.709516in}}{\pgfqpoint{2.408227in}{1.698466in}}%
\pgfpathcurveto{\pgfqpoint{2.408227in}{1.687416in}}{\pgfqpoint{2.412617in}{1.676817in}}{\pgfqpoint{2.420431in}{1.669003in}}%
\pgfpathcurveto{\pgfqpoint{2.428245in}{1.661190in}}{\pgfqpoint{2.438844in}{1.656799in}}{\pgfqpoint{2.449894in}{1.656799in}}%
\pgfpathclose%
\pgfusepath{stroke,fill}%
\end{pgfscope}%
\begin{pgfscope}%
\pgfpathrectangle{\pgfqpoint{0.481978in}{0.331635in}}{\pgfqpoint{9.300000in}{7.700000in}}%
\pgfusepath{clip}%
\pgfsetbuttcap%
\pgfsetroundjoin%
\definecolor{currentfill}{rgb}{0.552941,0.898039,0.631373}%
\pgfsetfillcolor{currentfill}%
\pgfsetlinewidth{0.481800pt}%
\definecolor{currentstroke}{rgb}{1.000000,1.000000,1.000000}%
\pgfsetstrokecolor{currentstroke}%
\pgfsetdash{}{0pt}%
\pgfpathmoveto{\pgfqpoint{7.028940in}{4.709424in}}%
\pgfpathcurveto{\pgfqpoint{7.039990in}{4.709424in}}{\pgfqpoint{7.050589in}{4.713814in}}{\pgfqpoint{7.058402in}{4.721628in}}%
\pgfpathcurveto{\pgfqpoint{7.066216in}{4.729441in}}{\pgfqpoint{7.070606in}{4.740040in}}{\pgfqpoint{7.070606in}{4.751090in}}%
\pgfpathcurveto{\pgfqpoint{7.070606in}{4.762141in}}{\pgfqpoint{7.066216in}{4.772740in}}{\pgfqpoint{7.058402in}{4.780553in}}%
\pgfpathcurveto{\pgfqpoint{7.050589in}{4.788367in}}{\pgfqpoint{7.039990in}{4.792757in}}{\pgfqpoint{7.028940in}{4.792757in}}%
\pgfpathcurveto{\pgfqpoint{7.017890in}{4.792757in}}{\pgfqpoint{7.007290in}{4.788367in}}{\pgfqpoint{6.999477in}{4.780553in}}%
\pgfpathcurveto{\pgfqpoint{6.991663in}{4.772740in}}{\pgfqpoint{6.987273in}{4.762141in}}{\pgfqpoint{6.987273in}{4.751090in}}%
\pgfpathcurveto{\pgfqpoint{6.987273in}{4.740040in}}{\pgfqpoint{6.991663in}{4.729441in}}{\pgfqpoint{6.999477in}{4.721628in}}%
\pgfpathcurveto{\pgfqpoint{7.007290in}{4.713814in}}{\pgfqpoint{7.017890in}{4.709424in}}{\pgfqpoint{7.028940in}{4.709424in}}%
\pgfpathclose%
\pgfusepath{stroke,fill}%
\end{pgfscope}%
\begin{pgfscope}%
\pgfpathrectangle{\pgfqpoint{0.481978in}{0.331635in}}{\pgfqpoint{9.300000in}{7.700000in}}%
\pgfusepath{clip}%
\pgfsetbuttcap%
\pgfsetroundjoin%
\definecolor{currentfill}{rgb}{0.552941,0.898039,0.631373}%
\pgfsetfillcolor{currentfill}%
\pgfsetlinewidth{0.481800pt}%
\definecolor{currentstroke}{rgb}{1.000000,1.000000,1.000000}%
\pgfsetstrokecolor{currentstroke}%
\pgfsetdash{}{0pt}%
\pgfpathmoveto{\pgfqpoint{5.259017in}{7.554084in}}%
\pgfpathcurveto{\pgfqpoint{5.270067in}{7.554084in}}{\pgfqpoint{5.280666in}{7.558475in}}{\pgfqpoint{5.288480in}{7.566288in}}%
\pgfpathcurveto{\pgfqpoint{5.296294in}{7.574102in}}{\pgfqpoint{5.300684in}{7.584701in}}{\pgfqpoint{5.300684in}{7.595751in}}%
\pgfpathcurveto{\pgfqpoint{5.300684in}{7.606801in}}{\pgfqpoint{5.296294in}{7.617400in}}{\pgfqpoint{5.288480in}{7.625214in}}%
\pgfpathcurveto{\pgfqpoint{5.280666in}{7.633027in}}{\pgfqpoint{5.270067in}{7.637418in}}{\pgfqpoint{5.259017in}{7.637418in}}%
\pgfpathcurveto{\pgfqpoint{5.247967in}{7.637418in}}{\pgfqpoint{5.237368in}{7.633027in}}{\pgfqpoint{5.229554in}{7.625214in}}%
\pgfpathcurveto{\pgfqpoint{5.221741in}{7.617400in}}{\pgfqpoint{5.217350in}{7.606801in}}{\pgfqpoint{5.217350in}{7.595751in}}%
\pgfpathcurveto{\pgfqpoint{5.217350in}{7.584701in}}{\pgfqpoint{5.221741in}{7.574102in}}{\pgfqpoint{5.229554in}{7.566288in}}%
\pgfpathcurveto{\pgfqpoint{5.237368in}{7.558475in}}{\pgfqpoint{5.247967in}{7.554084in}}{\pgfqpoint{5.259017in}{7.554084in}}%
\pgfpathclose%
\pgfusepath{stroke,fill}%
\end{pgfscope}%
\begin{pgfscope}%
\pgfpathrectangle{\pgfqpoint{0.481978in}{0.331635in}}{\pgfqpoint{9.300000in}{7.700000in}}%
\pgfusepath{clip}%
\pgfsetbuttcap%
\pgfsetroundjoin%
\definecolor{currentfill}{rgb}{0.552941,0.898039,0.631373}%
\pgfsetfillcolor{currentfill}%
\pgfsetlinewidth{0.481800pt}%
\definecolor{currentstroke}{rgb}{1.000000,1.000000,1.000000}%
\pgfsetstrokecolor{currentstroke}%
\pgfsetdash{}{0pt}%
\pgfpathmoveto{\pgfqpoint{9.184137in}{5.064662in}}%
\pgfpathcurveto{\pgfqpoint{9.195187in}{5.064662in}}{\pgfqpoint{9.205786in}{5.069052in}}{\pgfqpoint{9.213600in}{5.076866in}}%
\pgfpathcurveto{\pgfqpoint{9.221414in}{5.084680in}}{\pgfqpoint{9.225804in}{5.095279in}}{\pgfqpoint{9.225804in}{5.106329in}}%
\pgfpathcurveto{\pgfqpoint{9.225804in}{5.117379in}}{\pgfqpoint{9.221414in}{5.127978in}}{\pgfqpoint{9.213600in}{5.135792in}}%
\pgfpathcurveto{\pgfqpoint{9.205786in}{5.143605in}}{\pgfqpoint{9.195187in}{5.147995in}}{\pgfqpoint{9.184137in}{5.147995in}}%
\pgfpathcurveto{\pgfqpoint{9.173087in}{5.147995in}}{\pgfqpoint{9.162488in}{5.143605in}}{\pgfqpoint{9.154675in}{5.135792in}}%
\pgfpathcurveto{\pgfqpoint{9.146861in}{5.127978in}}{\pgfqpoint{9.142471in}{5.117379in}}{\pgfqpoint{9.142471in}{5.106329in}}%
\pgfpathcurveto{\pgfqpoint{9.142471in}{5.095279in}}{\pgfqpoint{9.146861in}{5.084680in}}{\pgfqpoint{9.154675in}{5.076866in}}%
\pgfpathcurveto{\pgfqpoint{9.162488in}{5.069052in}}{\pgfqpoint{9.173087in}{5.064662in}}{\pgfqpoint{9.184137in}{5.064662in}}%
\pgfpathclose%
\pgfusepath{stroke,fill}%
\end{pgfscope}%
\begin{pgfscope}%
\pgfpathrectangle{\pgfqpoint{0.481978in}{0.331635in}}{\pgfqpoint{9.300000in}{7.700000in}}%
\pgfusepath{clip}%
\pgfsetbuttcap%
\pgfsetroundjoin%
\definecolor{currentfill}{rgb}{0.552941,0.898039,0.631373}%
\pgfsetfillcolor{currentfill}%
\pgfsetlinewidth{0.481800pt}%
\definecolor{currentstroke}{rgb}{1.000000,1.000000,1.000000}%
\pgfsetstrokecolor{currentstroke}%
\pgfsetdash{}{0pt}%
\pgfpathmoveto{\pgfqpoint{8.944873in}{6.252081in}}%
\pgfpathcurveto{\pgfqpoint{8.955923in}{6.252081in}}{\pgfqpoint{8.966522in}{6.256471in}}{\pgfqpoint{8.974336in}{6.264285in}}%
\pgfpathcurveto{\pgfqpoint{8.982149in}{6.272098in}}{\pgfqpoint{8.986540in}{6.282697in}}{\pgfqpoint{8.986540in}{6.293748in}}%
\pgfpathcurveto{\pgfqpoint{8.986540in}{6.304798in}}{\pgfqpoint{8.982149in}{6.315397in}}{\pgfqpoint{8.974336in}{6.323210in}}%
\pgfpathcurveto{\pgfqpoint{8.966522in}{6.331024in}}{\pgfqpoint{8.955923in}{6.335414in}}{\pgfqpoint{8.944873in}{6.335414in}}%
\pgfpathcurveto{\pgfqpoint{8.933823in}{6.335414in}}{\pgfqpoint{8.923224in}{6.331024in}}{\pgfqpoint{8.915410in}{6.323210in}}%
\pgfpathcurveto{\pgfqpoint{8.907597in}{6.315397in}}{\pgfqpoint{8.903206in}{6.304798in}}{\pgfqpoint{8.903206in}{6.293748in}}%
\pgfpathcurveto{\pgfqpoint{8.903206in}{6.282697in}}{\pgfqpoint{8.907597in}{6.272098in}}{\pgfqpoint{8.915410in}{6.264285in}}%
\pgfpathcurveto{\pgfqpoint{8.923224in}{6.256471in}}{\pgfqpoint{8.933823in}{6.252081in}}{\pgfqpoint{8.944873in}{6.252081in}}%
\pgfpathclose%
\pgfusepath{stroke,fill}%
\end{pgfscope}%
\begin{pgfscope}%
\pgfpathrectangle{\pgfqpoint{0.481978in}{0.331635in}}{\pgfqpoint{9.300000in}{7.700000in}}%
\pgfusepath{clip}%
\pgfsetbuttcap%
\pgfsetroundjoin%
\definecolor{currentfill}{rgb}{0.552941,0.898039,0.631373}%
\pgfsetfillcolor{currentfill}%
\pgfsetlinewidth{0.481800pt}%
\definecolor{currentstroke}{rgb}{1.000000,1.000000,1.000000}%
\pgfsetstrokecolor{currentstroke}%
\pgfsetdash{}{0pt}%
\pgfpathmoveto{\pgfqpoint{7.733013in}{3.449529in}}%
\pgfpathcurveto{\pgfqpoint{7.744063in}{3.449529in}}{\pgfqpoint{7.754662in}{3.453919in}}{\pgfqpoint{7.762476in}{3.461733in}}%
\pgfpathcurveto{\pgfqpoint{7.770289in}{3.469547in}}{\pgfqpoint{7.774679in}{3.480146in}}{\pgfqpoint{7.774679in}{3.491196in}}%
\pgfpathcurveto{\pgfqpoint{7.774679in}{3.502246in}}{\pgfqpoint{7.770289in}{3.512845in}}{\pgfqpoint{7.762476in}{3.520659in}}%
\pgfpathcurveto{\pgfqpoint{7.754662in}{3.528472in}}{\pgfqpoint{7.744063in}{3.532862in}}{\pgfqpoint{7.733013in}{3.532862in}}%
\pgfpathcurveto{\pgfqpoint{7.721963in}{3.532862in}}{\pgfqpoint{7.711364in}{3.528472in}}{\pgfqpoint{7.703550in}{3.520659in}}%
\pgfpathcurveto{\pgfqpoint{7.695736in}{3.512845in}}{\pgfqpoint{7.691346in}{3.502246in}}{\pgfqpoint{7.691346in}{3.491196in}}%
\pgfpathcurveto{\pgfqpoint{7.691346in}{3.480146in}}{\pgfqpoint{7.695736in}{3.469547in}}{\pgfqpoint{7.703550in}{3.461733in}}%
\pgfpathcurveto{\pgfqpoint{7.711364in}{3.453919in}}{\pgfqpoint{7.721963in}{3.449529in}}{\pgfqpoint{7.733013in}{3.449529in}}%
\pgfpathclose%
\pgfusepath{stroke,fill}%
\end{pgfscope}%
\begin{pgfscope}%
\pgfpathrectangle{\pgfqpoint{0.481978in}{0.331635in}}{\pgfqpoint{9.300000in}{7.700000in}}%
\pgfusepath{clip}%
\pgfsetbuttcap%
\pgfsetroundjoin%
\definecolor{currentfill}{rgb}{0.552941,0.898039,0.631373}%
\pgfsetfillcolor{currentfill}%
\pgfsetlinewidth{0.481800pt}%
\definecolor{currentstroke}{rgb}{1.000000,1.000000,1.000000}%
\pgfsetstrokecolor{currentstroke}%
\pgfsetdash{}{0pt}%
\pgfpathmoveto{\pgfqpoint{5.907061in}{2.059171in}}%
\pgfpathcurveto{\pgfqpoint{5.918111in}{2.059171in}}{\pgfqpoint{5.928710in}{2.063562in}}{\pgfqpoint{5.936524in}{2.071375in}}%
\pgfpathcurveto{\pgfqpoint{5.944338in}{2.079189in}}{\pgfqpoint{5.948728in}{2.089788in}}{\pgfqpoint{5.948728in}{2.100838in}}%
\pgfpathcurveto{\pgfqpoint{5.948728in}{2.111888in}}{\pgfqpoint{5.944338in}{2.122487in}}{\pgfqpoint{5.936524in}{2.130301in}}%
\pgfpathcurveto{\pgfqpoint{5.928710in}{2.138114in}}{\pgfqpoint{5.918111in}{2.142505in}}{\pgfqpoint{5.907061in}{2.142505in}}%
\pgfpathcurveto{\pgfqpoint{5.896011in}{2.142505in}}{\pgfqpoint{5.885412in}{2.138114in}}{\pgfqpoint{5.877598in}{2.130301in}}%
\pgfpathcurveto{\pgfqpoint{5.869785in}{2.122487in}}{\pgfqpoint{5.865395in}{2.111888in}}{\pgfqpoint{5.865395in}{2.100838in}}%
\pgfpathcurveto{\pgfqpoint{5.865395in}{2.089788in}}{\pgfqpoint{5.869785in}{2.079189in}}{\pgfqpoint{5.877598in}{2.071375in}}%
\pgfpathcurveto{\pgfqpoint{5.885412in}{2.063562in}}{\pgfqpoint{5.896011in}{2.059171in}}{\pgfqpoint{5.907061in}{2.059171in}}%
\pgfpathclose%
\pgfusepath{stroke,fill}%
\end{pgfscope}%
\begin{pgfscope}%
\pgfpathrectangle{\pgfqpoint{0.481978in}{0.331635in}}{\pgfqpoint{9.300000in}{7.700000in}}%
\pgfusepath{clip}%
\pgfsetbuttcap%
\pgfsetroundjoin%
\definecolor{currentfill}{rgb}{0.552941,0.898039,0.631373}%
\pgfsetfillcolor{currentfill}%
\pgfsetlinewidth{0.481800pt}%
\definecolor{currentstroke}{rgb}{1.000000,1.000000,1.000000}%
\pgfsetstrokecolor{currentstroke}%
\pgfsetdash{}{0pt}%
\pgfpathmoveto{\pgfqpoint{4.839240in}{5.869672in}}%
\pgfpathcurveto{\pgfqpoint{4.850291in}{5.869672in}}{\pgfqpoint{4.860890in}{5.874062in}}{\pgfqpoint{4.868703in}{5.881876in}}%
\pgfpathcurveto{\pgfqpoint{4.876517in}{5.889689in}}{\pgfqpoint{4.880907in}{5.900288in}}{\pgfqpoint{4.880907in}{5.911338in}}%
\pgfpathcurveto{\pgfqpoint{4.880907in}{5.922389in}}{\pgfqpoint{4.876517in}{5.932988in}}{\pgfqpoint{4.868703in}{5.940801in}}%
\pgfpathcurveto{\pgfqpoint{4.860890in}{5.948615in}}{\pgfqpoint{4.850291in}{5.953005in}}{\pgfqpoint{4.839240in}{5.953005in}}%
\pgfpathcurveto{\pgfqpoint{4.828190in}{5.953005in}}{\pgfqpoint{4.817591in}{5.948615in}}{\pgfqpoint{4.809778in}{5.940801in}}%
\pgfpathcurveto{\pgfqpoint{4.801964in}{5.932988in}}{\pgfqpoint{4.797574in}{5.922389in}}{\pgfqpoint{4.797574in}{5.911338in}}%
\pgfpathcurveto{\pgfqpoint{4.797574in}{5.900288in}}{\pgfqpoint{4.801964in}{5.889689in}}{\pgfqpoint{4.809778in}{5.881876in}}%
\pgfpathcurveto{\pgfqpoint{4.817591in}{5.874062in}}{\pgfqpoint{4.828190in}{5.869672in}}{\pgfqpoint{4.839240in}{5.869672in}}%
\pgfpathclose%
\pgfusepath{stroke,fill}%
\end{pgfscope}%
\begin{pgfscope}%
\pgfpathrectangle{\pgfqpoint{0.481978in}{0.331635in}}{\pgfqpoint{9.300000in}{7.700000in}}%
\pgfusepath{clip}%
\pgfsetbuttcap%
\pgfsetroundjoin%
\definecolor{currentfill}{rgb}{0.552941,0.898039,0.631373}%
\pgfsetfillcolor{currentfill}%
\pgfsetlinewidth{0.481800pt}%
\definecolor{currentstroke}{rgb}{1.000000,1.000000,1.000000}%
\pgfsetstrokecolor{currentstroke}%
\pgfsetdash{}{0pt}%
\pgfpathmoveto{\pgfqpoint{2.381245in}{2.262941in}}%
\pgfpathcurveto{\pgfqpoint{2.392295in}{2.262941in}}{\pgfqpoint{2.402894in}{2.267331in}}{\pgfqpoint{2.410708in}{2.275144in}}%
\pgfpathcurveto{\pgfqpoint{2.418522in}{2.282958in}}{\pgfqpoint{2.422912in}{2.293557in}}{\pgfqpoint{2.422912in}{2.304607in}}%
\pgfpathcurveto{\pgfqpoint{2.422912in}{2.315657in}}{\pgfqpoint{2.418522in}{2.326256in}}{\pgfqpoint{2.410708in}{2.334070in}}%
\pgfpathcurveto{\pgfqpoint{2.402894in}{2.341884in}}{\pgfqpoint{2.392295in}{2.346274in}}{\pgfqpoint{2.381245in}{2.346274in}}%
\pgfpathcurveto{\pgfqpoint{2.370195in}{2.346274in}}{\pgfqpoint{2.359596in}{2.341884in}}{\pgfqpoint{2.351782in}{2.334070in}}%
\pgfpathcurveto{\pgfqpoint{2.343969in}{2.326256in}}{\pgfqpoint{2.339579in}{2.315657in}}{\pgfqpoint{2.339579in}{2.304607in}}%
\pgfpathcurveto{\pgfqpoint{2.339579in}{2.293557in}}{\pgfqpoint{2.343969in}{2.282958in}}{\pgfqpoint{2.351782in}{2.275144in}}%
\pgfpathcurveto{\pgfqpoint{2.359596in}{2.267331in}}{\pgfqpoint{2.370195in}{2.262941in}}{\pgfqpoint{2.381245in}{2.262941in}}%
\pgfpathclose%
\pgfusepath{stroke,fill}%
\end{pgfscope}%
\begin{pgfscope}%
\pgfpathrectangle{\pgfqpoint{0.481978in}{0.331635in}}{\pgfqpoint{9.300000in}{7.700000in}}%
\pgfusepath{clip}%
\pgfsetbuttcap%
\pgfsetroundjoin%
\definecolor{currentfill}{rgb}{0.552941,0.898039,0.631373}%
\pgfsetfillcolor{currentfill}%
\pgfsetlinewidth{0.481800pt}%
\definecolor{currentstroke}{rgb}{1.000000,1.000000,1.000000}%
\pgfsetstrokecolor{currentstroke}%
\pgfsetdash{}{0pt}%
\pgfpathmoveto{\pgfqpoint{7.266400in}{2.360768in}}%
\pgfpathcurveto{\pgfqpoint{7.277450in}{2.360768in}}{\pgfqpoint{7.288049in}{2.365158in}}{\pgfqpoint{7.295863in}{2.372972in}}%
\pgfpathcurveto{\pgfqpoint{7.303677in}{2.380785in}}{\pgfqpoint{7.308067in}{2.391384in}}{\pgfqpoint{7.308067in}{2.402434in}}%
\pgfpathcurveto{\pgfqpoint{7.308067in}{2.413484in}}{\pgfqpoint{7.303677in}{2.424083in}}{\pgfqpoint{7.295863in}{2.431897in}}%
\pgfpathcurveto{\pgfqpoint{7.288049in}{2.439711in}}{\pgfqpoint{7.277450in}{2.444101in}}{\pgfqpoint{7.266400in}{2.444101in}}%
\pgfpathcurveto{\pgfqpoint{7.255350in}{2.444101in}}{\pgfqpoint{7.244751in}{2.439711in}}{\pgfqpoint{7.236937in}{2.431897in}}%
\pgfpathcurveto{\pgfqpoint{7.229124in}{2.424083in}}{\pgfqpoint{7.224734in}{2.413484in}}{\pgfqpoint{7.224734in}{2.402434in}}%
\pgfpathcurveto{\pgfqpoint{7.224734in}{2.391384in}}{\pgfqpoint{7.229124in}{2.380785in}}{\pgfqpoint{7.236937in}{2.372972in}}%
\pgfpathcurveto{\pgfqpoint{7.244751in}{2.365158in}}{\pgfqpoint{7.255350in}{2.360768in}}{\pgfqpoint{7.266400in}{2.360768in}}%
\pgfpathclose%
\pgfusepath{stroke,fill}%
\end{pgfscope}%
\begin{pgfscope}%
\pgfpathrectangle{\pgfqpoint{0.481978in}{0.331635in}}{\pgfqpoint{9.300000in}{7.700000in}}%
\pgfusepath{clip}%
\pgfsetbuttcap%
\pgfsetroundjoin%
\definecolor{currentfill}{rgb}{0.552941,0.898039,0.631373}%
\pgfsetfillcolor{currentfill}%
\pgfsetlinewidth{0.481800pt}%
\definecolor{currentstroke}{rgb}{1.000000,1.000000,1.000000}%
\pgfsetstrokecolor{currentstroke}%
\pgfsetdash{}{0pt}%
\pgfpathmoveto{\pgfqpoint{3.410609in}{3.339609in}}%
\pgfpathcurveto{\pgfqpoint{3.421659in}{3.339609in}}{\pgfqpoint{3.432258in}{3.344000in}}{\pgfqpoint{3.440072in}{3.351813in}}%
\pgfpathcurveto{\pgfqpoint{3.447885in}{3.359627in}}{\pgfqpoint{3.452276in}{3.370226in}}{\pgfqpoint{3.452276in}{3.381276in}}%
\pgfpathcurveto{\pgfqpoint{3.452276in}{3.392326in}}{\pgfqpoint{3.447885in}{3.402925in}}{\pgfqpoint{3.440072in}{3.410739in}}%
\pgfpathcurveto{\pgfqpoint{3.432258in}{3.418553in}}{\pgfqpoint{3.421659in}{3.422943in}}{\pgfqpoint{3.410609in}{3.422943in}}%
\pgfpathcurveto{\pgfqpoint{3.399559in}{3.422943in}}{\pgfqpoint{3.388960in}{3.418553in}}{\pgfqpoint{3.381146in}{3.410739in}}%
\pgfpathcurveto{\pgfqpoint{3.373333in}{3.402925in}}{\pgfqpoint{3.368942in}{3.392326in}}{\pgfqpoint{3.368942in}{3.381276in}}%
\pgfpathcurveto{\pgfqpoint{3.368942in}{3.370226in}}{\pgfqpoint{3.373333in}{3.359627in}}{\pgfqpoint{3.381146in}{3.351813in}}%
\pgfpathcurveto{\pgfqpoint{3.388960in}{3.344000in}}{\pgfqpoint{3.399559in}{3.339609in}}{\pgfqpoint{3.410609in}{3.339609in}}%
\pgfpathclose%
\pgfusepath{stroke,fill}%
\end{pgfscope}%
\begin{pgfscope}%
\pgfpathrectangle{\pgfqpoint{0.481978in}{0.331635in}}{\pgfqpoint{9.300000in}{7.700000in}}%
\pgfusepath{clip}%
\pgfsetbuttcap%
\pgfsetroundjoin%
\definecolor{currentfill}{rgb}{0.552941,0.898039,0.631373}%
\pgfsetfillcolor{currentfill}%
\pgfsetlinewidth{0.481800pt}%
\definecolor{currentstroke}{rgb}{1.000000,1.000000,1.000000}%
\pgfsetstrokecolor{currentstroke}%
\pgfsetdash{}{0pt}%
\pgfpathmoveto{\pgfqpoint{9.044900in}{5.269063in}}%
\pgfpathcurveto{\pgfqpoint{9.055950in}{5.269063in}}{\pgfqpoint{9.066549in}{5.273453in}}{\pgfqpoint{9.074363in}{5.281267in}}%
\pgfpathcurveto{\pgfqpoint{9.082176in}{5.289081in}}{\pgfqpoint{9.086566in}{5.299680in}}{\pgfqpoint{9.086566in}{5.310730in}}%
\pgfpathcurveto{\pgfqpoint{9.086566in}{5.321780in}}{\pgfqpoint{9.082176in}{5.332379in}}{\pgfqpoint{9.074363in}{5.340192in}}%
\pgfpathcurveto{\pgfqpoint{9.066549in}{5.348006in}}{\pgfqpoint{9.055950in}{5.352396in}}{\pgfqpoint{9.044900in}{5.352396in}}%
\pgfpathcurveto{\pgfqpoint{9.033850in}{5.352396in}}{\pgfqpoint{9.023251in}{5.348006in}}{\pgfqpoint{9.015437in}{5.340192in}}%
\pgfpathcurveto{\pgfqpoint{9.007623in}{5.332379in}}{\pgfqpoint{9.003233in}{5.321780in}}{\pgfqpoint{9.003233in}{5.310730in}}%
\pgfpathcurveto{\pgfqpoint{9.003233in}{5.299680in}}{\pgfqpoint{9.007623in}{5.289081in}}{\pgfqpoint{9.015437in}{5.281267in}}%
\pgfpathcurveto{\pgfqpoint{9.023251in}{5.273453in}}{\pgfqpoint{9.033850in}{5.269063in}}{\pgfqpoint{9.044900in}{5.269063in}}%
\pgfpathclose%
\pgfusepath{stroke,fill}%
\end{pgfscope}%
\begin{pgfscope}%
\pgfpathrectangle{\pgfqpoint{0.481978in}{0.331635in}}{\pgfqpoint{9.300000in}{7.700000in}}%
\pgfusepath{clip}%
\pgfsetbuttcap%
\pgfsetroundjoin%
\definecolor{currentfill}{rgb}{0.552941,0.898039,0.631373}%
\pgfsetfillcolor{currentfill}%
\pgfsetlinewidth{0.481800pt}%
\definecolor{currentstroke}{rgb}{1.000000,1.000000,1.000000}%
\pgfsetstrokecolor{currentstroke}%
\pgfsetdash{}{0pt}%
\pgfpathmoveto{\pgfqpoint{8.024517in}{4.648323in}}%
\pgfpathcurveto{\pgfqpoint{8.035567in}{4.648323in}}{\pgfqpoint{8.046166in}{4.652713in}}{\pgfqpoint{8.053980in}{4.660527in}}%
\pgfpathcurveto{\pgfqpoint{8.061793in}{4.668341in}}{\pgfqpoint{8.066184in}{4.678940in}}{\pgfqpoint{8.066184in}{4.689990in}}%
\pgfpathcurveto{\pgfqpoint{8.066184in}{4.701040in}}{\pgfqpoint{8.061793in}{4.711639in}}{\pgfqpoint{8.053980in}{4.719453in}}%
\pgfpathcurveto{\pgfqpoint{8.046166in}{4.727266in}}{\pgfqpoint{8.035567in}{4.731656in}}{\pgfqpoint{8.024517in}{4.731656in}}%
\pgfpathcurveto{\pgfqpoint{8.013467in}{4.731656in}}{\pgfqpoint{8.002868in}{4.727266in}}{\pgfqpoint{7.995054in}{4.719453in}}%
\pgfpathcurveto{\pgfqpoint{7.987240in}{4.711639in}}{\pgfqpoint{7.982850in}{4.701040in}}{\pgfqpoint{7.982850in}{4.689990in}}%
\pgfpathcurveto{\pgfqpoint{7.982850in}{4.678940in}}{\pgfqpoint{7.987240in}{4.668341in}}{\pgfqpoint{7.995054in}{4.660527in}}%
\pgfpathcurveto{\pgfqpoint{8.002868in}{4.652713in}}{\pgfqpoint{8.013467in}{4.648323in}}{\pgfqpoint{8.024517in}{4.648323in}}%
\pgfpathclose%
\pgfusepath{stroke,fill}%
\end{pgfscope}%
\begin{pgfscope}%
\pgfpathrectangle{\pgfqpoint{0.481978in}{0.331635in}}{\pgfqpoint{9.300000in}{7.700000in}}%
\pgfusepath{clip}%
\pgfsetbuttcap%
\pgfsetroundjoin%
\definecolor{currentfill}{rgb}{0.552941,0.898039,0.631373}%
\pgfsetfillcolor{currentfill}%
\pgfsetlinewidth{0.481800pt}%
\definecolor{currentstroke}{rgb}{1.000000,1.000000,1.000000}%
\pgfsetstrokecolor{currentstroke}%
\pgfsetdash{}{0pt}%
\pgfpathmoveto{\pgfqpoint{3.710429in}{4.713864in}}%
\pgfpathcurveto{\pgfqpoint{3.721480in}{4.713864in}}{\pgfqpoint{3.732079in}{4.718254in}}{\pgfqpoint{3.739892in}{4.726068in}}%
\pgfpathcurveto{\pgfqpoint{3.747706in}{4.733881in}}{\pgfqpoint{3.752096in}{4.744480in}}{\pgfqpoint{3.752096in}{4.755530in}}%
\pgfpathcurveto{\pgfqpoint{3.752096in}{4.766580in}}{\pgfqpoint{3.747706in}{4.777179in}}{\pgfqpoint{3.739892in}{4.784993in}}%
\pgfpathcurveto{\pgfqpoint{3.732079in}{4.792807in}}{\pgfqpoint{3.721480in}{4.797197in}}{\pgfqpoint{3.710429in}{4.797197in}}%
\pgfpathcurveto{\pgfqpoint{3.699379in}{4.797197in}}{\pgfqpoint{3.688780in}{4.792807in}}{\pgfqpoint{3.680967in}{4.784993in}}%
\pgfpathcurveto{\pgfqpoint{3.673153in}{4.777179in}}{\pgfqpoint{3.668763in}{4.766580in}}{\pgfqpoint{3.668763in}{4.755530in}}%
\pgfpathcurveto{\pgfqpoint{3.668763in}{4.744480in}}{\pgfqpoint{3.673153in}{4.733881in}}{\pgfqpoint{3.680967in}{4.726068in}}%
\pgfpathcurveto{\pgfqpoint{3.688780in}{4.718254in}}{\pgfqpoint{3.699379in}{4.713864in}}{\pgfqpoint{3.710429in}{4.713864in}}%
\pgfpathclose%
\pgfusepath{stroke,fill}%
\end{pgfscope}%
\begin{pgfscope}%
\pgfpathrectangle{\pgfqpoint{0.481978in}{0.331635in}}{\pgfqpoint{9.300000in}{7.700000in}}%
\pgfusepath{clip}%
\pgfsetbuttcap%
\pgfsetroundjoin%
\definecolor{currentfill}{rgb}{0.552941,0.898039,0.631373}%
\pgfsetfillcolor{currentfill}%
\pgfsetlinewidth{0.481800pt}%
\definecolor{currentstroke}{rgb}{1.000000,1.000000,1.000000}%
\pgfsetstrokecolor{currentstroke}%
\pgfsetdash{}{0pt}%
\pgfpathmoveto{\pgfqpoint{5.690812in}{2.637048in}}%
\pgfpathcurveto{\pgfqpoint{5.701862in}{2.637048in}}{\pgfqpoint{5.712461in}{2.641438in}}{\pgfqpoint{5.720275in}{2.649252in}}%
\pgfpathcurveto{\pgfqpoint{5.728088in}{2.657065in}}{\pgfqpoint{5.732479in}{2.667664in}}{\pgfqpoint{5.732479in}{2.678714in}}%
\pgfpathcurveto{\pgfqpoint{5.732479in}{2.689765in}}{\pgfqpoint{5.728088in}{2.700364in}}{\pgfqpoint{5.720275in}{2.708177in}}%
\pgfpathcurveto{\pgfqpoint{5.712461in}{2.715991in}}{\pgfqpoint{5.701862in}{2.720381in}}{\pgfqpoint{5.690812in}{2.720381in}}%
\pgfpathcurveto{\pgfqpoint{5.679762in}{2.720381in}}{\pgfqpoint{5.669163in}{2.715991in}}{\pgfqpoint{5.661349in}{2.708177in}}%
\pgfpathcurveto{\pgfqpoint{5.653536in}{2.700364in}}{\pgfqpoint{5.649145in}{2.689765in}}{\pgfqpoint{5.649145in}{2.678714in}}%
\pgfpathcurveto{\pgfqpoint{5.649145in}{2.667664in}}{\pgfqpoint{5.653536in}{2.657065in}}{\pgfqpoint{5.661349in}{2.649252in}}%
\pgfpathcurveto{\pgfqpoint{5.669163in}{2.641438in}}{\pgfqpoint{5.679762in}{2.637048in}}{\pgfqpoint{5.690812in}{2.637048in}}%
\pgfpathclose%
\pgfusepath{stroke,fill}%
\end{pgfscope}%
\begin{pgfscope}%
\pgfpathrectangle{\pgfqpoint{0.481978in}{0.331635in}}{\pgfqpoint{9.300000in}{7.700000in}}%
\pgfusepath{clip}%
\pgfsetbuttcap%
\pgfsetroundjoin%
\definecolor{currentfill}{rgb}{1.000000,0.623529,0.607843}%
\pgfsetfillcolor{currentfill}%
\pgfsetlinewidth{0.481800pt}%
\definecolor{currentstroke}{rgb}{1.000000,1.000000,1.000000}%
\pgfsetstrokecolor{currentstroke}%
\pgfsetdash{}{0pt}%
\pgfpathmoveto{\pgfqpoint{7.737423in}{5.840523in}}%
\pgfpathcurveto{\pgfqpoint{7.748473in}{5.840523in}}{\pgfqpoint{7.759072in}{5.844913in}}{\pgfqpoint{7.766885in}{5.852727in}}%
\pgfpathcurveto{\pgfqpoint{7.774699in}{5.860541in}}{\pgfqpoint{7.779089in}{5.871140in}}{\pgfqpoint{7.779089in}{5.882190in}}%
\pgfpathcurveto{\pgfqpoint{7.779089in}{5.893240in}}{\pgfqpoint{7.774699in}{5.903839in}}{\pgfqpoint{7.766885in}{5.911653in}}%
\pgfpathcurveto{\pgfqpoint{7.759072in}{5.919466in}}{\pgfqpoint{7.748473in}{5.923856in}}{\pgfqpoint{7.737423in}{5.923856in}}%
\pgfpathcurveto{\pgfqpoint{7.726372in}{5.923856in}}{\pgfqpoint{7.715773in}{5.919466in}}{\pgfqpoint{7.707960in}{5.911653in}}%
\pgfpathcurveto{\pgfqpoint{7.700146in}{5.903839in}}{\pgfqpoint{7.695756in}{5.893240in}}{\pgfqpoint{7.695756in}{5.882190in}}%
\pgfpathcurveto{\pgfqpoint{7.695756in}{5.871140in}}{\pgfqpoint{7.700146in}{5.860541in}}{\pgfqpoint{7.707960in}{5.852727in}}%
\pgfpathcurveto{\pgfqpoint{7.715773in}{5.844913in}}{\pgfqpoint{7.726372in}{5.840523in}}{\pgfqpoint{7.737423in}{5.840523in}}%
\pgfpathclose%
\pgfusepath{stroke,fill}%
\end{pgfscope}%
\begin{pgfscope}%
\pgfpathrectangle{\pgfqpoint{0.481978in}{0.331635in}}{\pgfqpoint{9.300000in}{7.700000in}}%
\pgfusepath{clip}%
\pgfsetbuttcap%
\pgfsetroundjoin%
\definecolor{currentfill}{rgb}{1.000000,0.623529,0.607843}%
\pgfsetfillcolor{currentfill}%
\pgfsetlinewidth{0.481800pt}%
\definecolor{currentstroke}{rgb}{1.000000,1.000000,1.000000}%
\pgfsetstrokecolor{currentstroke}%
\pgfsetdash{}{0pt}%
\pgfpathmoveto{\pgfqpoint{8.478500in}{5.560595in}}%
\pgfpathcurveto{\pgfqpoint{8.489550in}{5.560595in}}{\pgfqpoint{8.500149in}{5.564985in}}{\pgfqpoint{8.507962in}{5.572799in}}%
\pgfpathcurveto{\pgfqpoint{8.515776in}{5.580613in}}{\pgfqpoint{8.520166in}{5.591212in}}{\pgfqpoint{8.520166in}{5.602262in}}%
\pgfpathcurveto{\pgfqpoint{8.520166in}{5.613312in}}{\pgfqpoint{8.515776in}{5.623911in}}{\pgfqpoint{8.507962in}{5.631724in}}%
\pgfpathcurveto{\pgfqpoint{8.500149in}{5.639538in}}{\pgfqpoint{8.489550in}{5.643928in}}{\pgfqpoint{8.478500in}{5.643928in}}%
\pgfpathcurveto{\pgfqpoint{8.467450in}{5.643928in}}{\pgfqpoint{8.456850in}{5.639538in}}{\pgfqpoint{8.449037in}{5.631724in}}%
\pgfpathcurveto{\pgfqpoint{8.441223in}{5.623911in}}{\pgfqpoint{8.436833in}{5.613312in}}{\pgfqpoint{8.436833in}{5.602262in}}%
\pgfpathcurveto{\pgfqpoint{8.436833in}{5.591212in}}{\pgfqpoint{8.441223in}{5.580613in}}{\pgfqpoint{8.449037in}{5.572799in}}%
\pgfpathcurveto{\pgfqpoint{8.456850in}{5.564985in}}{\pgfqpoint{8.467450in}{5.560595in}}{\pgfqpoint{8.478500in}{5.560595in}}%
\pgfpathclose%
\pgfusepath{stroke,fill}%
\end{pgfscope}%
\begin{pgfscope}%
\pgfpathrectangle{\pgfqpoint{0.481978in}{0.331635in}}{\pgfqpoint{9.300000in}{7.700000in}}%
\pgfusepath{clip}%
\pgfsetbuttcap%
\pgfsetroundjoin%
\definecolor{currentfill}{rgb}{1.000000,0.623529,0.607843}%
\pgfsetfillcolor{currentfill}%
\pgfsetlinewidth{0.481800pt}%
\definecolor{currentstroke}{rgb}{1.000000,1.000000,1.000000}%
\pgfsetstrokecolor{currentstroke}%
\pgfsetdash{}{0pt}%
\pgfpathmoveto{\pgfqpoint{5.299357in}{2.415253in}}%
\pgfpathcurveto{\pgfqpoint{5.310407in}{2.415253in}}{\pgfqpoint{5.321006in}{2.419644in}}{\pgfqpoint{5.328820in}{2.427457in}}%
\pgfpathcurveto{\pgfqpoint{5.336633in}{2.435271in}}{\pgfqpoint{5.341024in}{2.445870in}}{\pgfqpoint{5.341024in}{2.456920in}}%
\pgfpathcurveto{\pgfqpoint{5.341024in}{2.467970in}}{\pgfqpoint{5.336633in}{2.478569in}}{\pgfqpoint{5.328820in}{2.486383in}}%
\pgfpathcurveto{\pgfqpoint{5.321006in}{2.494196in}}{\pgfqpoint{5.310407in}{2.498587in}}{\pgfqpoint{5.299357in}{2.498587in}}%
\pgfpathcurveto{\pgfqpoint{5.288307in}{2.498587in}}{\pgfqpoint{5.277708in}{2.494196in}}{\pgfqpoint{5.269894in}{2.486383in}}%
\pgfpathcurveto{\pgfqpoint{5.262081in}{2.478569in}}{\pgfqpoint{5.257690in}{2.467970in}}{\pgfqpoint{5.257690in}{2.456920in}}%
\pgfpathcurveto{\pgfqpoint{5.257690in}{2.445870in}}{\pgfqpoint{5.262081in}{2.435271in}}{\pgfqpoint{5.269894in}{2.427457in}}%
\pgfpathcurveto{\pgfqpoint{5.277708in}{2.419644in}}{\pgfqpoint{5.288307in}{2.415253in}}{\pgfqpoint{5.299357in}{2.415253in}}%
\pgfpathclose%
\pgfusepath{stroke,fill}%
\end{pgfscope}%
\begin{pgfscope}%
\pgfpathrectangle{\pgfqpoint{0.481978in}{0.331635in}}{\pgfqpoint{9.300000in}{7.700000in}}%
\pgfusepath{clip}%
\pgfsetbuttcap%
\pgfsetroundjoin%
\definecolor{currentfill}{rgb}{1.000000,0.623529,0.607843}%
\pgfsetfillcolor{currentfill}%
\pgfsetlinewidth{0.481800pt}%
\definecolor{currentstroke}{rgb}{1.000000,1.000000,1.000000}%
\pgfsetstrokecolor{currentstroke}%
\pgfsetdash{}{0pt}%
\pgfpathmoveto{\pgfqpoint{5.050465in}{2.386257in}}%
\pgfpathcurveto{\pgfqpoint{5.061515in}{2.386257in}}{\pgfqpoint{5.072114in}{2.390648in}}{\pgfqpoint{5.079928in}{2.398461in}}%
\pgfpathcurveto{\pgfqpoint{5.087741in}{2.406275in}}{\pgfqpoint{5.092132in}{2.416874in}}{\pgfqpoint{5.092132in}{2.427924in}}%
\pgfpathcurveto{\pgfqpoint{5.092132in}{2.438974in}}{\pgfqpoint{5.087741in}{2.449573in}}{\pgfqpoint{5.079928in}{2.457387in}}%
\pgfpathcurveto{\pgfqpoint{5.072114in}{2.465200in}}{\pgfqpoint{5.061515in}{2.469591in}}{\pgfqpoint{5.050465in}{2.469591in}}%
\pgfpathcurveto{\pgfqpoint{5.039415in}{2.469591in}}{\pgfqpoint{5.028816in}{2.465200in}}{\pgfqpoint{5.021002in}{2.457387in}}%
\pgfpathcurveto{\pgfqpoint{5.013189in}{2.449573in}}{\pgfqpoint{5.008798in}{2.438974in}}{\pgfqpoint{5.008798in}{2.427924in}}%
\pgfpathcurveto{\pgfqpoint{5.008798in}{2.416874in}}{\pgfqpoint{5.013189in}{2.406275in}}{\pgfqpoint{5.021002in}{2.398461in}}%
\pgfpathcurveto{\pgfqpoint{5.028816in}{2.390648in}}{\pgfqpoint{5.039415in}{2.386257in}}{\pgfqpoint{5.050465in}{2.386257in}}%
\pgfpathclose%
\pgfusepath{stroke,fill}%
\end{pgfscope}%
\begin{pgfscope}%
\pgfpathrectangle{\pgfqpoint{0.481978in}{0.331635in}}{\pgfqpoint{9.300000in}{7.700000in}}%
\pgfusepath{clip}%
\pgfsetbuttcap%
\pgfsetroundjoin%
\definecolor{currentfill}{rgb}{1.000000,0.623529,0.607843}%
\pgfsetfillcolor{currentfill}%
\pgfsetlinewidth{0.481800pt}%
\definecolor{currentstroke}{rgb}{1.000000,1.000000,1.000000}%
\pgfsetstrokecolor{currentstroke}%
\pgfsetdash{}{0pt}%
\pgfpathmoveto{\pgfqpoint{4.424855in}{3.449766in}}%
\pgfpathcurveto{\pgfqpoint{4.435905in}{3.449766in}}{\pgfqpoint{4.446504in}{3.454156in}}{\pgfqpoint{4.454318in}{3.461970in}}%
\pgfpathcurveto{\pgfqpoint{4.462131in}{3.469783in}}{\pgfqpoint{4.466522in}{3.480382in}}{\pgfqpoint{4.466522in}{3.491433in}}%
\pgfpathcurveto{\pgfqpoint{4.466522in}{3.502483in}}{\pgfqpoint{4.462131in}{3.513082in}}{\pgfqpoint{4.454318in}{3.520895in}}%
\pgfpathcurveto{\pgfqpoint{4.446504in}{3.528709in}}{\pgfqpoint{4.435905in}{3.533099in}}{\pgfqpoint{4.424855in}{3.533099in}}%
\pgfpathcurveto{\pgfqpoint{4.413805in}{3.533099in}}{\pgfqpoint{4.403206in}{3.528709in}}{\pgfqpoint{4.395392in}{3.520895in}}%
\pgfpathcurveto{\pgfqpoint{4.387579in}{3.513082in}}{\pgfqpoint{4.383188in}{3.502483in}}{\pgfqpoint{4.383188in}{3.491433in}}%
\pgfpathcurveto{\pgfqpoint{4.383188in}{3.480382in}}{\pgfqpoint{4.387579in}{3.469783in}}{\pgfqpoint{4.395392in}{3.461970in}}%
\pgfpathcurveto{\pgfqpoint{4.403206in}{3.454156in}}{\pgfqpoint{4.413805in}{3.449766in}}{\pgfqpoint{4.424855in}{3.449766in}}%
\pgfpathclose%
\pgfusepath{stroke,fill}%
\end{pgfscope}%
\begin{pgfscope}%
\pgfpathrectangle{\pgfqpoint{0.481978in}{0.331635in}}{\pgfqpoint{9.300000in}{7.700000in}}%
\pgfusepath{clip}%
\pgfsetbuttcap%
\pgfsetroundjoin%
\definecolor{currentfill}{rgb}{1.000000,0.623529,0.607843}%
\pgfsetfillcolor{currentfill}%
\pgfsetlinewidth{0.481800pt}%
\definecolor{currentstroke}{rgb}{1.000000,1.000000,1.000000}%
\pgfsetstrokecolor{currentstroke}%
\pgfsetdash{}{0pt}%
\pgfpathmoveto{\pgfqpoint{4.703040in}{4.110492in}}%
\pgfpathcurveto{\pgfqpoint{4.714090in}{4.110492in}}{\pgfqpoint{4.724689in}{4.114882in}}{\pgfqpoint{4.732503in}{4.122696in}}%
\pgfpathcurveto{\pgfqpoint{4.740316in}{4.130509in}}{\pgfqpoint{4.744706in}{4.141108in}}{\pgfqpoint{4.744706in}{4.152158in}}%
\pgfpathcurveto{\pgfqpoint{4.744706in}{4.163209in}}{\pgfqpoint{4.740316in}{4.173808in}}{\pgfqpoint{4.732503in}{4.181621in}}%
\pgfpathcurveto{\pgfqpoint{4.724689in}{4.189435in}}{\pgfqpoint{4.714090in}{4.193825in}}{\pgfqpoint{4.703040in}{4.193825in}}%
\pgfpathcurveto{\pgfqpoint{4.691990in}{4.193825in}}{\pgfqpoint{4.681391in}{4.189435in}}{\pgfqpoint{4.673577in}{4.181621in}}%
\pgfpathcurveto{\pgfqpoint{4.665763in}{4.173808in}}{\pgfqpoint{4.661373in}{4.163209in}}{\pgfqpoint{4.661373in}{4.152158in}}%
\pgfpathcurveto{\pgfqpoint{4.661373in}{4.141108in}}{\pgfqpoint{4.665763in}{4.130509in}}{\pgfqpoint{4.673577in}{4.122696in}}%
\pgfpathcurveto{\pgfqpoint{4.681391in}{4.114882in}}{\pgfqpoint{4.691990in}{4.110492in}}{\pgfqpoint{4.703040in}{4.110492in}}%
\pgfpathclose%
\pgfusepath{stroke,fill}%
\end{pgfscope}%
\begin{pgfscope}%
\pgfpathrectangle{\pgfqpoint{0.481978in}{0.331635in}}{\pgfqpoint{9.300000in}{7.700000in}}%
\pgfusepath{clip}%
\pgfsetbuttcap%
\pgfsetroundjoin%
\definecolor{currentfill}{rgb}{1.000000,0.623529,0.607843}%
\pgfsetfillcolor{currentfill}%
\pgfsetlinewidth{0.481800pt}%
\definecolor{currentstroke}{rgb}{1.000000,1.000000,1.000000}%
\pgfsetstrokecolor{currentstroke}%
\pgfsetdash{}{0pt}%
\pgfpathmoveto{\pgfqpoint{2.809853in}{3.839407in}}%
\pgfpathcurveto{\pgfqpoint{2.820903in}{3.839407in}}{\pgfqpoint{2.831502in}{3.843797in}}{\pgfqpoint{2.839316in}{3.851610in}}%
\pgfpathcurveto{\pgfqpoint{2.847129in}{3.859424in}}{\pgfqpoint{2.851520in}{3.870023in}}{\pgfqpoint{2.851520in}{3.881073in}}%
\pgfpathcurveto{\pgfqpoint{2.851520in}{3.892123in}}{\pgfqpoint{2.847129in}{3.902722in}}{\pgfqpoint{2.839316in}{3.910536in}}%
\pgfpathcurveto{\pgfqpoint{2.831502in}{3.918350in}}{\pgfqpoint{2.820903in}{3.922740in}}{\pgfqpoint{2.809853in}{3.922740in}}%
\pgfpathcurveto{\pgfqpoint{2.798803in}{3.922740in}}{\pgfqpoint{2.788204in}{3.918350in}}{\pgfqpoint{2.780390in}{3.910536in}}%
\pgfpathcurveto{\pgfqpoint{2.772577in}{3.902722in}}{\pgfqpoint{2.768186in}{3.892123in}}{\pgfqpoint{2.768186in}{3.881073in}}%
\pgfpathcurveto{\pgfqpoint{2.768186in}{3.870023in}}{\pgfqpoint{2.772577in}{3.859424in}}{\pgfqpoint{2.780390in}{3.851610in}}%
\pgfpathcurveto{\pgfqpoint{2.788204in}{3.843797in}}{\pgfqpoint{2.798803in}{3.839407in}}{\pgfqpoint{2.809853in}{3.839407in}}%
\pgfpathclose%
\pgfusepath{stroke,fill}%
\end{pgfscope}%
\begin{pgfscope}%
\pgfpathrectangle{\pgfqpoint{0.481978in}{0.331635in}}{\pgfqpoint{9.300000in}{7.700000in}}%
\pgfusepath{clip}%
\pgfsetbuttcap%
\pgfsetroundjoin%
\definecolor{currentfill}{rgb}{1.000000,0.623529,0.607843}%
\pgfsetfillcolor{currentfill}%
\pgfsetlinewidth{0.481800pt}%
\definecolor{currentstroke}{rgb}{1.000000,1.000000,1.000000}%
\pgfsetstrokecolor{currentstroke}%
\pgfsetdash{}{0pt}%
\pgfpathmoveto{\pgfqpoint{3.971771in}{3.008119in}}%
\pgfpathcurveto{\pgfqpoint{3.982821in}{3.008119in}}{\pgfqpoint{3.993421in}{3.012509in}}{\pgfqpoint{4.001234in}{3.020323in}}%
\pgfpathcurveto{\pgfqpoint{4.009048in}{3.028136in}}{\pgfqpoint{4.013438in}{3.038735in}}{\pgfqpoint{4.013438in}{3.049785in}}%
\pgfpathcurveto{\pgfqpoint{4.013438in}{3.060835in}}{\pgfqpoint{4.009048in}{3.071435in}}{\pgfqpoint{4.001234in}{3.079248in}}%
\pgfpathcurveto{\pgfqpoint{3.993421in}{3.087062in}}{\pgfqpoint{3.982821in}{3.091452in}}{\pgfqpoint{3.971771in}{3.091452in}}%
\pgfpathcurveto{\pgfqpoint{3.960721in}{3.091452in}}{\pgfqpoint{3.950122in}{3.087062in}}{\pgfqpoint{3.942309in}{3.079248in}}%
\pgfpathcurveto{\pgfqpoint{3.934495in}{3.071435in}}{\pgfqpoint{3.930105in}{3.060835in}}{\pgfqpoint{3.930105in}{3.049785in}}%
\pgfpathcurveto{\pgfqpoint{3.930105in}{3.038735in}}{\pgfqpoint{3.934495in}{3.028136in}}{\pgfqpoint{3.942309in}{3.020323in}}%
\pgfpathcurveto{\pgfqpoint{3.950122in}{3.012509in}}{\pgfqpoint{3.960721in}{3.008119in}}{\pgfqpoint{3.971771in}{3.008119in}}%
\pgfpathclose%
\pgfusepath{stroke,fill}%
\end{pgfscope}%
\begin{pgfscope}%
\pgfpathrectangle{\pgfqpoint{0.481978in}{0.331635in}}{\pgfqpoint{9.300000in}{7.700000in}}%
\pgfusepath{clip}%
\pgfsetbuttcap%
\pgfsetroundjoin%
\definecolor{currentfill}{rgb}{1.000000,0.623529,0.607843}%
\pgfsetfillcolor{currentfill}%
\pgfsetlinewidth{0.481800pt}%
\definecolor{currentstroke}{rgb}{1.000000,1.000000,1.000000}%
\pgfsetstrokecolor{currentstroke}%
\pgfsetdash{}{0pt}%
\pgfpathmoveto{\pgfqpoint{7.681317in}{5.060668in}}%
\pgfpathcurveto{\pgfqpoint{7.692368in}{5.060668in}}{\pgfqpoint{7.702967in}{5.065058in}}{\pgfqpoint{7.710780in}{5.072872in}}%
\pgfpathcurveto{\pgfqpoint{7.718594in}{5.080685in}}{\pgfqpoint{7.722984in}{5.091284in}}{\pgfqpoint{7.722984in}{5.102335in}}%
\pgfpathcurveto{\pgfqpoint{7.722984in}{5.113385in}}{\pgfqpoint{7.718594in}{5.123984in}}{\pgfqpoint{7.710780in}{5.131797in}}%
\pgfpathcurveto{\pgfqpoint{7.702967in}{5.139611in}}{\pgfqpoint{7.692368in}{5.144001in}}{\pgfqpoint{7.681317in}{5.144001in}}%
\pgfpathcurveto{\pgfqpoint{7.670267in}{5.144001in}}{\pgfqpoint{7.659668in}{5.139611in}}{\pgfqpoint{7.651855in}{5.131797in}}%
\pgfpathcurveto{\pgfqpoint{7.644041in}{5.123984in}}{\pgfqpoint{7.639651in}{5.113385in}}{\pgfqpoint{7.639651in}{5.102335in}}%
\pgfpathcurveto{\pgfqpoint{7.639651in}{5.091284in}}{\pgfqpoint{7.644041in}{5.080685in}}{\pgfqpoint{7.651855in}{5.072872in}}%
\pgfpathcurveto{\pgfqpoint{7.659668in}{5.065058in}}{\pgfqpoint{7.670267in}{5.060668in}}{\pgfqpoint{7.681317in}{5.060668in}}%
\pgfpathclose%
\pgfusepath{stroke,fill}%
\end{pgfscope}%
\begin{pgfscope}%
\pgfpathrectangle{\pgfqpoint{0.481978in}{0.331635in}}{\pgfqpoint{9.300000in}{7.700000in}}%
\pgfusepath{clip}%
\pgfsetbuttcap%
\pgfsetroundjoin%
\definecolor{currentfill}{rgb}{1.000000,0.623529,0.607843}%
\pgfsetfillcolor{currentfill}%
\pgfsetlinewidth{0.481800pt}%
\definecolor{currentstroke}{rgb}{1.000000,1.000000,1.000000}%
\pgfsetstrokecolor{currentstroke}%
\pgfsetdash{}{0pt}%
\pgfpathmoveto{\pgfqpoint{4.922074in}{3.099400in}}%
\pgfpathcurveto{\pgfqpoint{4.933124in}{3.099400in}}{\pgfqpoint{4.943723in}{3.103790in}}{\pgfqpoint{4.951537in}{3.111604in}}%
\pgfpathcurveto{\pgfqpoint{4.959350in}{3.119417in}}{\pgfqpoint{4.963741in}{3.130016in}}{\pgfqpoint{4.963741in}{3.141067in}}%
\pgfpathcurveto{\pgfqpoint{4.963741in}{3.152117in}}{\pgfqpoint{4.959350in}{3.162716in}}{\pgfqpoint{4.951537in}{3.170529in}}%
\pgfpathcurveto{\pgfqpoint{4.943723in}{3.178343in}}{\pgfqpoint{4.933124in}{3.182733in}}{\pgfqpoint{4.922074in}{3.182733in}}%
\pgfpathcurveto{\pgfqpoint{4.911024in}{3.182733in}}{\pgfqpoint{4.900425in}{3.178343in}}{\pgfqpoint{4.892611in}{3.170529in}}%
\pgfpathcurveto{\pgfqpoint{4.884798in}{3.162716in}}{\pgfqpoint{4.880407in}{3.152117in}}{\pgfqpoint{4.880407in}{3.141067in}}%
\pgfpathcurveto{\pgfqpoint{4.880407in}{3.130016in}}{\pgfqpoint{4.884798in}{3.119417in}}{\pgfqpoint{4.892611in}{3.111604in}}%
\pgfpathcurveto{\pgfqpoint{4.900425in}{3.103790in}}{\pgfqpoint{4.911024in}{3.099400in}}{\pgfqpoint{4.922074in}{3.099400in}}%
\pgfpathclose%
\pgfusepath{stroke,fill}%
\end{pgfscope}%
\begin{pgfscope}%
\pgfpathrectangle{\pgfqpoint{0.481978in}{0.331635in}}{\pgfqpoint{9.300000in}{7.700000in}}%
\pgfusepath{clip}%
\pgfsetbuttcap%
\pgfsetroundjoin%
\definecolor{currentfill}{rgb}{1.000000,0.623529,0.607843}%
\pgfsetfillcolor{currentfill}%
\pgfsetlinewidth{0.481800pt}%
\definecolor{currentstroke}{rgb}{1.000000,1.000000,1.000000}%
\pgfsetstrokecolor{currentstroke}%
\pgfsetdash{}{0pt}%
\pgfpathmoveto{\pgfqpoint{4.349051in}{3.104492in}}%
\pgfpathcurveto{\pgfqpoint{4.360101in}{3.104492in}}{\pgfqpoint{4.370700in}{3.108882in}}{\pgfqpoint{4.378514in}{3.116696in}}%
\pgfpathcurveto{\pgfqpoint{4.386327in}{3.124510in}}{\pgfqpoint{4.390718in}{3.135109in}}{\pgfqpoint{4.390718in}{3.146159in}}%
\pgfpathcurveto{\pgfqpoint{4.390718in}{3.157209in}}{\pgfqpoint{4.386327in}{3.167808in}}{\pgfqpoint{4.378514in}{3.175621in}}%
\pgfpathcurveto{\pgfqpoint{4.370700in}{3.183435in}}{\pgfqpoint{4.360101in}{3.187825in}}{\pgfqpoint{4.349051in}{3.187825in}}%
\pgfpathcurveto{\pgfqpoint{4.338001in}{3.187825in}}{\pgfqpoint{4.327402in}{3.183435in}}{\pgfqpoint{4.319588in}{3.175621in}}%
\pgfpathcurveto{\pgfqpoint{4.311775in}{3.167808in}}{\pgfqpoint{4.307384in}{3.157209in}}{\pgfqpoint{4.307384in}{3.146159in}}%
\pgfpathcurveto{\pgfqpoint{4.307384in}{3.135109in}}{\pgfqpoint{4.311775in}{3.124510in}}{\pgfqpoint{4.319588in}{3.116696in}}%
\pgfpathcurveto{\pgfqpoint{4.327402in}{3.108882in}}{\pgfqpoint{4.338001in}{3.104492in}}{\pgfqpoint{4.349051in}{3.104492in}}%
\pgfpathclose%
\pgfusepath{stroke,fill}%
\end{pgfscope}%
\begin{pgfscope}%
\pgfpathrectangle{\pgfqpoint{0.481978in}{0.331635in}}{\pgfqpoint{9.300000in}{7.700000in}}%
\pgfusepath{clip}%
\pgfsetbuttcap%
\pgfsetroundjoin%
\definecolor{currentfill}{rgb}{1.000000,0.623529,0.607843}%
\pgfsetfillcolor{currentfill}%
\pgfsetlinewidth{0.481800pt}%
\definecolor{currentstroke}{rgb}{1.000000,1.000000,1.000000}%
\pgfsetstrokecolor{currentstroke}%
\pgfsetdash{}{0pt}%
\pgfpathmoveto{\pgfqpoint{4.285306in}{3.855591in}}%
\pgfpathcurveto{\pgfqpoint{4.296356in}{3.855591in}}{\pgfqpoint{4.306955in}{3.859981in}}{\pgfqpoint{4.314769in}{3.867794in}}%
\pgfpathcurveto{\pgfqpoint{4.322583in}{3.875608in}}{\pgfqpoint{4.326973in}{3.886207in}}{\pgfqpoint{4.326973in}{3.897257in}}%
\pgfpathcurveto{\pgfqpoint{4.326973in}{3.908307in}}{\pgfqpoint{4.322583in}{3.918906in}}{\pgfqpoint{4.314769in}{3.926720in}}%
\pgfpathcurveto{\pgfqpoint{4.306955in}{3.934534in}}{\pgfqpoint{4.296356in}{3.938924in}}{\pgfqpoint{4.285306in}{3.938924in}}%
\pgfpathcurveto{\pgfqpoint{4.274256in}{3.938924in}}{\pgfqpoint{4.263657in}{3.934534in}}{\pgfqpoint{4.255843in}{3.926720in}}%
\pgfpathcurveto{\pgfqpoint{4.248030in}{3.918906in}}{\pgfqpoint{4.243639in}{3.908307in}}{\pgfqpoint{4.243639in}{3.897257in}}%
\pgfpathcurveto{\pgfqpoint{4.243639in}{3.886207in}}{\pgfqpoint{4.248030in}{3.875608in}}{\pgfqpoint{4.255843in}{3.867794in}}%
\pgfpathcurveto{\pgfqpoint{4.263657in}{3.859981in}}{\pgfqpoint{4.274256in}{3.855591in}}{\pgfqpoint{4.285306in}{3.855591in}}%
\pgfpathclose%
\pgfusepath{stroke,fill}%
\end{pgfscope}%
\begin{pgfscope}%
\pgfpathrectangle{\pgfqpoint{0.481978in}{0.331635in}}{\pgfqpoint{9.300000in}{7.700000in}}%
\pgfusepath{clip}%
\pgfsetbuttcap%
\pgfsetroundjoin%
\definecolor{currentfill}{rgb}{1.000000,0.623529,0.607843}%
\pgfsetfillcolor{currentfill}%
\pgfsetlinewidth{0.481800pt}%
\definecolor{currentstroke}{rgb}{1.000000,1.000000,1.000000}%
\pgfsetstrokecolor{currentstroke}%
\pgfsetdash{}{0pt}%
\pgfpathmoveto{\pgfqpoint{5.628759in}{1.318174in}}%
\pgfpathcurveto{\pgfqpoint{5.639809in}{1.318174in}}{\pgfqpoint{5.650408in}{1.322564in}}{\pgfqpoint{5.658221in}{1.330378in}}%
\pgfpathcurveto{\pgfqpoint{5.666035in}{1.338191in}}{\pgfqpoint{5.670425in}{1.348790in}}{\pgfqpoint{5.670425in}{1.359840in}}%
\pgfpathcurveto{\pgfqpoint{5.670425in}{1.370890in}}{\pgfqpoint{5.666035in}{1.381489in}}{\pgfqpoint{5.658221in}{1.389303in}}%
\pgfpathcurveto{\pgfqpoint{5.650408in}{1.397117in}}{\pgfqpoint{5.639809in}{1.401507in}}{\pgfqpoint{5.628759in}{1.401507in}}%
\pgfpathcurveto{\pgfqpoint{5.617708in}{1.401507in}}{\pgfqpoint{5.607109in}{1.397117in}}{\pgfqpoint{5.599296in}{1.389303in}}%
\pgfpathcurveto{\pgfqpoint{5.591482in}{1.381489in}}{\pgfqpoint{5.587092in}{1.370890in}}{\pgfqpoint{5.587092in}{1.359840in}}%
\pgfpathcurveto{\pgfqpoint{5.587092in}{1.348790in}}{\pgfqpoint{5.591482in}{1.338191in}}{\pgfqpoint{5.599296in}{1.330378in}}%
\pgfpathcurveto{\pgfqpoint{5.607109in}{1.322564in}}{\pgfqpoint{5.617708in}{1.318174in}}{\pgfqpoint{5.628759in}{1.318174in}}%
\pgfpathclose%
\pgfusepath{stroke,fill}%
\end{pgfscope}%
\begin{pgfscope}%
\pgfpathrectangle{\pgfqpoint{0.481978in}{0.331635in}}{\pgfqpoint{9.300000in}{7.700000in}}%
\pgfusepath{clip}%
\pgfsetbuttcap%
\pgfsetroundjoin%
\definecolor{currentfill}{rgb}{1.000000,0.623529,0.607843}%
\pgfsetfillcolor{currentfill}%
\pgfsetlinewidth{0.481800pt}%
\definecolor{currentstroke}{rgb}{1.000000,1.000000,1.000000}%
\pgfsetstrokecolor{currentstroke}%
\pgfsetdash{}{0pt}%
\pgfpathmoveto{\pgfqpoint{2.208081in}{3.690280in}}%
\pgfpathcurveto{\pgfqpoint{2.219131in}{3.690280in}}{\pgfqpoint{2.229730in}{3.694670in}}{\pgfqpoint{2.237544in}{3.702484in}}%
\pgfpathcurveto{\pgfqpoint{2.245357in}{3.710298in}}{\pgfqpoint{2.249748in}{3.720897in}}{\pgfqpoint{2.249748in}{3.731947in}}%
\pgfpathcurveto{\pgfqpoint{2.249748in}{3.742997in}}{\pgfqpoint{2.245357in}{3.753596in}}{\pgfqpoint{2.237544in}{3.761410in}}%
\pgfpathcurveto{\pgfqpoint{2.229730in}{3.769223in}}{\pgfqpoint{2.219131in}{3.773614in}}{\pgfqpoint{2.208081in}{3.773614in}}%
\pgfpathcurveto{\pgfqpoint{2.197031in}{3.773614in}}{\pgfqpoint{2.186432in}{3.769223in}}{\pgfqpoint{2.178618in}{3.761410in}}%
\pgfpathcurveto{\pgfqpoint{2.170805in}{3.753596in}}{\pgfqpoint{2.166414in}{3.742997in}}{\pgfqpoint{2.166414in}{3.731947in}}%
\pgfpathcurveto{\pgfqpoint{2.166414in}{3.720897in}}{\pgfqpoint{2.170805in}{3.710298in}}{\pgfqpoint{2.178618in}{3.702484in}}%
\pgfpathcurveto{\pgfqpoint{2.186432in}{3.694670in}}{\pgfqpoint{2.197031in}{3.690280in}}{\pgfqpoint{2.208081in}{3.690280in}}%
\pgfpathclose%
\pgfusepath{stroke,fill}%
\end{pgfscope}%
\begin{pgfscope}%
\pgfpathrectangle{\pgfqpoint{0.481978in}{0.331635in}}{\pgfqpoint{9.300000in}{7.700000in}}%
\pgfusepath{clip}%
\pgfsetbuttcap%
\pgfsetroundjoin%
\definecolor{currentfill}{rgb}{1.000000,0.623529,0.607843}%
\pgfsetfillcolor{currentfill}%
\pgfsetlinewidth{0.481800pt}%
\definecolor{currentstroke}{rgb}{1.000000,1.000000,1.000000}%
\pgfsetstrokecolor{currentstroke}%
\pgfsetdash{}{0pt}%
\pgfpathmoveto{\pgfqpoint{5.627987in}{1.316150in}}%
\pgfpathcurveto{\pgfqpoint{5.639037in}{1.316150in}}{\pgfqpoint{5.649636in}{1.320540in}}{\pgfqpoint{5.657450in}{1.328354in}}%
\pgfpathcurveto{\pgfqpoint{5.665263in}{1.336167in}}{\pgfqpoint{5.669654in}{1.346766in}}{\pgfqpoint{5.669654in}{1.357817in}}%
\pgfpathcurveto{\pgfqpoint{5.669654in}{1.368867in}}{\pgfqpoint{5.665263in}{1.379466in}}{\pgfqpoint{5.657450in}{1.387279in}}%
\pgfpathcurveto{\pgfqpoint{5.649636in}{1.395093in}}{\pgfqpoint{5.639037in}{1.399483in}}{\pgfqpoint{5.627987in}{1.399483in}}%
\pgfpathcurveto{\pgfqpoint{5.616937in}{1.399483in}}{\pgfqpoint{5.606338in}{1.395093in}}{\pgfqpoint{5.598524in}{1.387279in}}%
\pgfpathcurveto{\pgfqpoint{5.590711in}{1.379466in}}{\pgfqpoint{5.586320in}{1.368867in}}{\pgfqpoint{5.586320in}{1.357817in}}%
\pgfpathcurveto{\pgfqpoint{5.586320in}{1.346766in}}{\pgfqpoint{5.590711in}{1.336167in}}{\pgfqpoint{5.598524in}{1.328354in}}%
\pgfpathcurveto{\pgfqpoint{5.606338in}{1.320540in}}{\pgfqpoint{5.616937in}{1.316150in}}{\pgfqpoint{5.627987in}{1.316150in}}%
\pgfpathclose%
\pgfusepath{stroke,fill}%
\end{pgfscope}%
\begin{pgfscope}%
\pgfpathrectangle{\pgfqpoint{0.481978in}{0.331635in}}{\pgfqpoint{9.300000in}{7.700000in}}%
\pgfusepath{clip}%
\pgfsetbuttcap%
\pgfsetroundjoin%
\definecolor{currentfill}{rgb}{1.000000,0.623529,0.607843}%
\pgfsetfillcolor{currentfill}%
\pgfsetlinewidth{0.481800pt}%
\definecolor{currentstroke}{rgb}{1.000000,1.000000,1.000000}%
\pgfsetstrokecolor{currentstroke}%
\pgfsetdash{}{0pt}%
\pgfpathmoveto{\pgfqpoint{7.678869in}{4.472682in}}%
\pgfpathcurveto{\pgfqpoint{7.689919in}{4.472682in}}{\pgfqpoint{7.700518in}{4.477072in}}{\pgfqpoint{7.708332in}{4.484886in}}%
\pgfpathcurveto{\pgfqpoint{7.716145in}{4.492700in}}{\pgfqpoint{7.720536in}{4.503299in}}{\pgfqpoint{7.720536in}{4.514349in}}%
\pgfpathcurveto{\pgfqpoint{7.720536in}{4.525399in}}{\pgfqpoint{7.716145in}{4.535998in}}{\pgfqpoint{7.708332in}{4.543812in}}%
\pgfpathcurveto{\pgfqpoint{7.700518in}{4.551625in}}{\pgfqpoint{7.689919in}{4.556015in}}{\pgfqpoint{7.678869in}{4.556015in}}%
\pgfpathcurveto{\pgfqpoint{7.667819in}{4.556015in}}{\pgfqpoint{7.657220in}{4.551625in}}{\pgfqpoint{7.649406in}{4.543812in}}%
\pgfpathcurveto{\pgfqpoint{7.641593in}{4.535998in}}{\pgfqpoint{7.637202in}{4.525399in}}{\pgfqpoint{7.637202in}{4.514349in}}%
\pgfpathcurveto{\pgfqpoint{7.637202in}{4.503299in}}{\pgfqpoint{7.641593in}{4.492700in}}{\pgfqpoint{7.649406in}{4.484886in}}%
\pgfpathcurveto{\pgfqpoint{7.657220in}{4.477072in}}{\pgfqpoint{7.667819in}{4.472682in}}{\pgfqpoint{7.678869in}{4.472682in}}%
\pgfpathclose%
\pgfusepath{stroke,fill}%
\end{pgfscope}%
\begin{pgfscope}%
\pgfpathrectangle{\pgfqpoint{0.481978in}{0.331635in}}{\pgfqpoint{9.300000in}{7.700000in}}%
\pgfusepath{clip}%
\pgfsetbuttcap%
\pgfsetroundjoin%
\definecolor{currentfill}{rgb}{1.000000,0.623529,0.607843}%
\pgfsetfillcolor{currentfill}%
\pgfsetlinewidth{0.481800pt}%
\definecolor{currentstroke}{rgb}{1.000000,1.000000,1.000000}%
\pgfsetstrokecolor{currentstroke}%
\pgfsetdash{}{0pt}%
\pgfpathmoveto{\pgfqpoint{3.663088in}{1.980706in}}%
\pgfpathcurveto{\pgfqpoint{3.674139in}{1.980706in}}{\pgfqpoint{3.684738in}{1.985096in}}{\pgfqpoint{3.692551in}{1.992910in}}%
\pgfpathcurveto{\pgfqpoint{3.700365in}{2.000723in}}{\pgfqpoint{3.704755in}{2.011322in}}{\pgfqpoint{3.704755in}{2.022372in}}%
\pgfpathcurveto{\pgfqpoint{3.704755in}{2.033423in}}{\pgfqpoint{3.700365in}{2.044022in}}{\pgfqpoint{3.692551in}{2.051835in}}%
\pgfpathcurveto{\pgfqpoint{3.684738in}{2.059649in}}{\pgfqpoint{3.674139in}{2.064039in}}{\pgfqpoint{3.663088in}{2.064039in}}%
\pgfpathcurveto{\pgfqpoint{3.652038in}{2.064039in}}{\pgfqpoint{3.641439in}{2.059649in}}{\pgfqpoint{3.633626in}{2.051835in}}%
\pgfpathcurveto{\pgfqpoint{3.625812in}{2.044022in}}{\pgfqpoint{3.621422in}{2.033423in}}{\pgfqpoint{3.621422in}{2.022372in}}%
\pgfpathcurveto{\pgfqpoint{3.621422in}{2.011322in}}{\pgfqpoint{3.625812in}{2.000723in}}{\pgfqpoint{3.633626in}{1.992910in}}%
\pgfpathcurveto{\pgfqpoint{3.641439in}{1.985096in}}{\pgfqpoint{3.652038in}{1.980706in}}{\pgfqpoint{3.663088in}{1.980706in}}%
\pgfpathclose%
\pgfusepath{stroke,fill}%
\end{pgfscope}%
\begin{pgfscope}%
\pgfpathrectangle{\pgfqpoint{0.481978in}{0.331635in}}{\pgfqpoint{9.300000in}{7.700000in}}%
\pgfusepath{clip}%
\pgfsetbuttcap%
\pgfsetroundjoin%
\definecolor{currentfill}{rgb}{1.000000,0.623529,0.607843}%
\pgfsetfillcolor{currentfill}%
\pgfsetlinewidth{0.481800pt}%
\definecolor{currentstroke}{rgb}{1.000000,1.000000,1.000000}%
\pgfsetstrokecolor{currentstroke}%
\pgfsetdash{}{0pt}%
\pgfpathmoveto{\pgfqpoint{2.974676in}{3.669425in}}%
\pgfpathcurveto{\pgfqpoint{2.985726in}{3.669425in}}{\pgfqpoint{2.996325in}{3.673815in}}{\pgfqpoint{3.004138in}{3.681629in}}%
\pgfpathcurveto{\pgfqpoint{3.011952in}{3.689442in}}{\pgfqpoint{3.016342in}{3.700041in}}{\pgfqpoint{3.016342in}{3.711091in}}%
\pgfpathcurveto{\pgfqpoint{3.016342in}{3.722141in}}{\pgfqpoint{3.011952in}{3.732741in}}{\pgfqpoint{3.004138in}{3.740554in}}%
\pgfpathcurveto{\pgfqpoint{2.996325in}{3.748368in}}{\pgfqpoint{2.985726in}{3.752758in}}{\pgfqpoint{2.974676in}{3.752758in}}%
\pgfpathcurveto{\pgfqpoint{2.963625in}{3.752758in}}{\pgfqpoint{2.953026in}{3.748368in}}{\pgfqpoint{2.945213in}{3.740554in}}%
\pgfpathcurveto{\pgfqpoint{2.937399in}{3.732741in}}{\pgfqpoint{2.933009in}{3.722141in}}{\pgfqpoint{2.933009in}{3.711091in}}%
\pgfpathcurveto{\pgfqpoint{2.933009in}{3.700041in}}{\pgfqpoint{2.937399in}{3.689442in}}{\pgfqpoint{2.945213in}{3.681629in}}%
\pgfpathcurveto{\pgfqpoint{2.953026in}{3.673815in}}{\pgfqpoint{2.963625in}{3.669425in}}{\pgfqpoint{2.974676in}{3.669425in}}%
\pgfpathclose%
\pgfusepath{stroke,fill}%
\end{pgfscope}%
\begin{pgfscope}%
\pgfpathrectangle{\pgfqpoint{0.481978in}{0.331635in}}{\pgfqpoint{9.300000in}{7.700000in}}%
\pgfusepath{clip}%
\pgfsetbuttcap%
\pgfsetroundjoin%
\definecolor{currentfill}{rgb}{1.000000,0.623529,0.607843}%
\pgfsetfillcolor{currentfill}%
\pgfsetlinewidth{0.481800pt}%
\definecolor{currentstroke}{rgb}{1.000000,1.000000,1.000000}%
\pgfsetstrokecolor{currentstroke}%
\pgfsetdash{}{0pt}%
\pgfpathmoveto{\pgfqpoint{4.287456in}{3.550314in}}%
\pgfpathcurveto{\pgfqpoint{4.298506in}{3.550314in}}{\pgfqpoint{4.309105in}{3.554704in}}{\pgfqpoint{4.316919in}{3.562518in}}%
\pgfpathcurveto{\pgfqpoint{4.324732in}{3.570331in}}{\pgfqpoint{4.329122in}{3.580930in}}{\pgfqpoint{4.329122in}{3.591981in}}%
\pgfpathcurveto{\pgfqpoint{4.329122in}{3.603031in}}{\pgfqpoint{4.324732in}{3.613630in}}{\pgfqpoint{4.316919in}{3.621443in}}%
\pgfpathcurveto{\pgfqpoint{4.309105in}{3.629257in}}{\pgfqpoint{4.298506in}{3.633647in}}{\pgfqpoint{4.287456in}{3.633647in}}%
\pgfpathcurveto{\pgfqpoint{4.276406in}{3.633647in}}{\pgfqpoint{4.265807in}{3.629257in}}{\pgfqpoint{4.257993in}{3.621443in}}%
\pgfpathcurveto{\pgfqpoint{4.250179in}{3.613630in}}{\pgfqpoint{4.245789in}{3.603031in}}{\pgfqpoint{4.245789in}{3.591981in}}%
\pgfpathcurveto{\pgfqpoint{4.245789in}{3.580930in}}{\pgfqpoint{4.250179in}{3.570331in}}{\pgfqpoint{4.257993in}{3.562518in}}%
\pgfpathcurveto{\pgfqpoint{4.265807in}{3.554704in}}{\pgfqpoint{4.276406in}{3.550314in}}{\pgfqpoint{4.287456in}{3.550314in}}%
\pgfpathclose%
\pgfusepath{stroke,fill}%
\end{pgfscope}%
\begin{pgfscope}%
\pgfpathrectangle{\pgfqpoint{0.481978in}{0.331635in}}{\pgfqpoint{9.300000in}{7.700000in}}%
\pgfusepath{clip}%
\pgfsetbuttcap%
\pgfsetroundjoin%
\definecolor{currentfill}{rgb}{1.000000,0.623529,0.607843}%
\pgfsetfillcolor{currentfill}%
\pgfsetlinewidth{0.481800pt}%
\definecolor{currentstroke}{rgb}{1.000000,1.000000,1.000000}%
\pgfsetstrokecolor{currentstroke}%
\pgfsetdash{}{0pt}%
\pgfpathmoveto{\pgfqpoint{1.967877in}{5.548168in}}%
\pgfpathcurveto{\pgfqpoint{1.978927in}{5.548168in}}{\pgfqpoint{1.989526in}{5.552558in}}{\pgfqpoint{1.997339in}{5.560372in}}%
\pgfpathcurveto{\pgfqpoint{2.005153in}{5.568185in}}{\pgfqpoint{2.009543in}{5.578784in}}{\pgfqpoint{2.009543in}{5.589834in}}%
\pgfpathcurveto{\pgfqpoint{2.009543in}{5.600885in}}{\pgfqpoint{2.005153in}{5.611484in}}{\pgfqpoint{1.997339in}{5.619297in}}%
\pgfpathcurveto{\pgfqpoint{1.989526in}{5.627111in}}{\pgfqpoint{1.978927in}{5.631501in}}{\pgfqpoint{1.967877in}{5.631501in}}%
\pgfpathcurveto{\pgfqpoint{1.956827in}{5.631501in}}{\pgfqpoint{1.946228in}{5.627111in}}{\pgfqpoint{1.938414in}{5.619297in}}%
\pgfpathcurveto{\pgfqpoint{1.930600in}{5.611484in}}{\pgfqpoint{1.926210in}{5.600885in}}{\pgfqpoint{1.926210in}{5.589834in}}%
\pgfpathcurveto{\pgfqpoint{1.926210in}{5.578784in}}{\pgfqpoint{1.930600in}{5.568185in}}{\pgfqpoint{1.938414in}{5.560372in}}%
\pgfpathcurveto{\pgfqpoint{1.946228in}{5.552558in}}{\pgfqpoint{1.956827in}{5.548168in}}{\pgfqpoint{1.967877in}{5.548168in}}%
\pgfpathclose%
\pgfusepath{stroke,fill}%
\end{pgfscope}%
\begin{pgfscope}%
\pgfpathrectangle{\pgfqpoint{0.481978in}{0.331635in}}{\pgfqpoint{9.300000in}{7.700000in}}%
\pgfusepath{clip}%
\pgfsetbuttcap%
\pgfsetroundjoin%
\definecolor{currentfill}{rgb}{1.000000,0.623529,0.607843}%
\pgfsetfillcolor{currentfill}%
\pgfsetlinewidth{0.481800pt}%
\definecolor{currentstroke}{rgb}{1.000000,1.000000,1.000000}%
\pgfsetstrokecolor{currentstroke}%
\pgfsetdash{}{0pt}%
\pgfpathmoveto{\pgfqpoint{3.570734in}{2.892749in}}%
\pgfpathcurveto{\pgfqpoint{3.581784in}{2.892749in}}{\pgfqpoint{3.592383in}{2.897139in}}{\pgfqpoint{3.600197in}{2.904953in}}%
\pgfpathcurveto{\pgfqpoint{3.608010in}{2.912766in}}{\pgfqpoint{3.612401in}{2.923365in}}{\pgfqpoint{3.612401in}{2.934416in}}%
\pgfpathcurveto{\pgfqpoint{3.612401in}{2.945466in}}{\pgfqpoint{3.608010in}{2.956065in}}{\pgfqpoint{3.600197in}{2.963878in}}%
\pgfpathcurveto{\pgfqpoint{3.592383in}{2.971692in}}{\pgfqpoint{3.581784in}{2.976082in}}{\pgfqpoint{3.570734in}{2.976082in}}%
\pgfpathcurveto{\pgfqpoint{3.559684in}{2.976082in}}{\pgfqpoint{3.549085in}{2.971692in}}{\pgfqpoint{3.541271in}{2.963878in}}%
\pgfpathcurveto{\pgfqpoint{3.533458in}{2.956065in}}{\pgfqpoint{3.529067in}{2.945466in}}{\pgfqpoint{3.529067in}{2.934416in}}%
\pgfpathcurveto{\pgfqpoint{3.529067in}{2.923365in}}{\pgfqpoint{3.533458in}{2.912766in}}{\pgfqpoint{3.541271in}{2.904953in}}%
\pgfpathcurveto{\pgfqpoint{3.549085in}{2.897139in}}{\pgfqpoint{3.559684in}{2.892749in}}{\pgfqpoint{3.570734in}{2.892749in}}%
\pgfpathclose%
\pgfusepath{stroke,fill}%
\end{pgfscope}%
\begin{pgfscope}%
\pgfpathrectangle{\pgfqpoint{0.481978in}{0.331635in}}{\pgfqpoint{9.300000in}{7.700000in}}%
\pgfusepath{clip}%
\pgfsetbuttcap%
\pgfsetroundjoin%
\definecolor{currentfill}{rgb}{1.000000,0.623529,0.607843}%
\pgfsetfillcolor{currentfill}%
\pgfsetlinewidth{0.481800pt}%
\definecolor{currentstroke}{rgb}{1.000000,1.000000,1.000000}%
\pgfsetstrokecolor{currentstroke}%
\pgfsetdash{}{0pt}%
\pgfpathmoveto{\pgfqpoint{5.661130in}{4.631812in}}%
\pgfpathcurveto{\pgfqpoint{5.672180in}{4.631812in}}{\pgfqpoint{5.682779in}{4.636202in}}{\pgfqpoint{5.690593in}{4.644015in}}%
\pgfpathcurveto{\pgfqpoint{5.698407in}{4.651829in}}{\pgfqpoint{5.702797in}{4.662428in}}{\pgfqpoint{5.702797in}{4.673478in}}%
\pgfpathcurveto{\pgfqpoint{5.702797in}{4.684528in}}{\pgfqpoint{5.698407in}{4.695127in}}{\pgfqpoint{5.690593in}{4.702941in}}%
\pgfpathcurveto{\pgfqpoint{5.682779in}{4.710755in}}{\pgfqpoint{5.672180in}{4.715145in}}{\pgfqpoint{5.661130in}{4.715145in}}%
\pgfpathcurveto{\pgfqpoint{5.650080in}{4.715145in}}{\pgfqpoint{5.639481in}{4.710755in}}{\pgfqpoint{5.631667in}{4.702941in}}%
\pgfpathcurveto{\pgfqpoint{5.623854in}{4.695127in}}{\pgfqpoint{5.619463in}{4.684528in}}{\pgfqpoint{5.619463in}{4.673478in}}%
\pgfpathcurveto{\pgfqpoint{5.619463in}{4.662428in}}{\pgfqpoint{5.623854in}{4.651829in}}{\pgfqpoint{5.631667in}{4.644015in}}%
\pgfpathcurveto{\pgfqpoint{5.639481in}{4.636202in}}{\pgfqpoint{5.650080in}{4.631812in}}{\pgfqpoint{5.661130in}{4.631812in}}%
\pgfpathclose%
\pgfusepath{stroke,fill}%
\end{pgfscope}%
\begin{pgfscope}%
\pgfpathrectangle{\pgfqpoint{0.481978in}{0.331635in}}{\pgfqpoint{9.300000in}{7.700000in}}%
\pgfusepath{clip}%
\pgfsetbuttcap%
\pgfsetroundjoin%
\definecolor{currentfill}{rgb}{1.000000,0.623529,0.607843}%
\pgfsetfillcolor{currentfill}%
\pgfsetlinewidth{0.481800pt}%
\definecolor{currentstroke}{rgb}{1.000000,1.000000,1.000000}%
\pgfsetstrokecolor{currentstroke}%
\pgfsetdash{}{0pt}%
\pgfpathmoveto{\pgfqpoint{4.720327in}{4.321572in}}%
\pgfpathcurveto{\pgfqpoint{4.731377in}{4.321572in}}{\pgfqpoint{4.741976in}{4.325963in}}{\pgfqpoint{4.749790in}{4.333776in}}%
\pgfpathcurveto{\pgfqpoint{4.757604in}{4.341590in}}{\pgfqpoint{4.761994in}{4.352189in}}{\pgfqpoint{4.761994in}{4.363239in}}%
\pgfpathcurveto{\pgfqpoint{4.761994in}{4.374289in}}{\pgfqpoint{4.757604in}{4.384888in}}{\pgfqpoint{4.749790in}{4.392702in}}%
\pgfpathcurveto{\pgfqpoint{4.741976in}{4.400515in}}{\pgfqpoint{4.731377in}{4.404906in}}{\pgfqpoint{4.720327in}{4.404906in}}%
\pgfpathcurveto{\pgfqpoint{4.709277in}{4.404906in}}{\pgfqpoint{4.698678in}{4.400515in}}{\pgfqpoint{4.690864in}{4.392702in}}%
\pgfpathcurveto{\pgfqpoint{4.683051in}{4.384888in}}{\pgfqpoint{4.678660in}{4.374289in}}{\pgfqpoint{4.678660in}{4.363239in}}%
\pgfpathcurveto{\pgfqpoint{4.678660in}{4.352189in}}{\pgfqpoint{4.683051in}{4.341590in}}{\pgfqpoint{4.690864in}{4.333776in}}%
\pgfpathcurveto{\pgfqpoint{4.698678in}{4.325963in}}{\pgfqpoint{4.709277in}{4.321572in}}{\pgfqpoint{4.720327in}{4.321572in}}%
\pgfpathclose%
\pgfusepath{stroke,fill}%
\end{pgfscope}%
\begin{pgfscope}%
\pgfpathrectangle{\pgfqpoint{0.481978in}{0.331635in}}{\pgfqpoint{9.300000in}{7.700000in}}%
\pgfusepath{clip}%
\pgfsetbuttcap%
\pgfsetroundjoin%
\definecolor{currentfill}{rgb}{1.000000,0.623529,0.607843}%
\pgfsetfillcolor{currentfill}%
\pgfsetlinewidth{0.481800pt}%
\definecolor{currentstroke}{rgb}{1.000000,1.000000,1.000000}%
\pgfsetstrokecolor{currentstroke}%
\pgfsetdash{}{0pt}%
\pgfpathmoveto{\pgfqpoint{2.339602in}{4.088730in}}%
\pgfpathcurveto{\pgfqpoint{2.350652in}{4.088730in}}{\pgfqpoint{2.361251in}{4.093120in}}{\pgfqpoint{2.369065in}{4.100934in}}%
\pgfpathcurveto{\pgfqpoint{2.376879in}{4.108747in}}{\pgfqpoint{2.381269in}{4.119346in}}{\pgfqpoint{2.381269in}{4.130396in}}%
\pgfpathcurveto{\pgfqpoint{2.381269in}{4.141446in}}{\pgfqpoint{2.376879in}{4.152045in}}{\pgfqpoint{2.369065in}{4.159859in}}%
\pgfpathcurveto{\pgfqpoint{2.361251in}{4.167673in}}{\pgfqpoint{2.350652in}{4.172063in}}{\pgfqpoint{2.339602in}{4.172063in}}%
\pgfpathcurveto{\pgfqpoint{2.328552in}{4.172063in}}{\pgfqpoint{2.317953in}{4.167673in}}{\pgfqpoint{2.310139in}{4.159859in}}%
\pgfpathcurveto{\pgfqpoint{2.302326in}{4.152045in}}{\pgfqpoint{2.297936in}{4.141446in}}{\pgfqpoint{2.297936in}{4.130396in}}%
\pgfpathcurveto{\pgfqpoint{2.297936in}{4.119346in}}{\pgfqpoint{2.302326in}{4.108747in}}{\pgfqpoint{2.310139in}{4.100934in}}%
\pgfpathcurveto{\pgfqpoint{2.317953in}{4.093120in}}{\pgfqpoint{2.328552in}{4.088730in}}{\pgfqpoint{2.339602in}{4.088730in}}%
\pgfpathclose%
\pgfusepath{stroke,fill}%
\end{pgfscope}%
\begin{pgfscope}%
\pgfpathrectangle{\pgfqpoint{0.481978in}{0.331635in}}{\pgfqpoint{9.300000in}{7.700000in}}%
\pgfusepath{clip}%
\pgfsetbuttcap%
\pgfsetroundjoin%
\definecolor{currentfill}{rgb}{1.000000,0.623529,0.607843}%
\pgfsetfillcolor{currentfill}%
\pgfsetlinewidth{0.481800pt}%
\definecolor{currentstroke}{rgb}{1.000000,1.000000,1.000000}%
\pgfsetstrokecolor{currentstroke}%
\pgfsetdash{}{0pt}%
\pgfpathmoveto{\pgfqpoint{7.776459in}{5.338891in}}%
\pgfpathcurveto{\pgfqpoint{7.787510in}{5.338891in}}{\pgfqpoint{7.798109in}{5.343282in}}{\pgfqpoint{7.805922in}{5.351095in}}%
\pgfpathcurveto{\pgfqpoint{7.813736in}{5.358909in}}{\pgfqpoint{7.818126in}{5.369508in}}{\pgfqpoint{7.818126in}{5.380558in}}%
\pgfpathcurveto{\pgfqpoint{7.818126in}{5.391608in}}{\pgfqpoint{7.813736in}{5.402207in}}{\pgfqpoint{7.805922in}{5.410021in}}%
\pgfpathcurveto{\pgfqpoint{7.798109in}{5.417834in}}{\pgfqpoint{7.787510in}{5.422225in}}{\pgfqpoint{7.776459in}{5.422225in}}%
\pgfpathcurveto{\pgfqpoint{7.765409in}{5.422225in}}{\pgfqpoint{7.754810in}{5.417834in}}{\pgfqpoint{7.746997in}{5.410021in}}%
\pgfpathcurveto{\pgfqpoint{7.739183in}{5.402207in}}{\pgfqpoint{7.734793in}{5.391608in}}{\pgfqpoint{7.734793in}{5.380558in}}%
\pgfpathcurveto{\pgfqpoint{7.734793in}{5.369508in}}{\pgfqpoint{7.739183in}{5.358909in}}{\pgfqpoint{7.746997in}{5.351095in}}%
\pgfpathcurveto{\pgfqpoint{7.754810in}{5.343282in}}{\pgfqpoint{7.765409in}{5.338891in}}{\pgfqpoint{7.776459in}{5.338891in}}%
\pgfpathclose%
\pgfusepath{stroke,fill}%
\end{pgfscope}%
\begin{pgfscope}%
\pgfpathrectangle{\pgfqpoint{0.481978in}{0.331635in}}{\pgfqpoint{9.300000in}{7.700000in}}%
\pgfusepath{clip}%
\pgfsetbuttcap%
\pgfsetroundjoin%
\definecolor{currentfill}{rgb}{1.000000,0.623529,0.607843}%
\pgfsetfillcolor{currentfill}%
\pgfsetlinewidth{0.481800pt}%
\definecolor{currentstroke}{rgb}{1.000000,1.000000,1.000000}%
\pgfsetstrokecolor{currentstroke}%
\pgfsetdash{}{0pt}%
\pgfpathmoveto{\pgfqpoint{5.832034in}{1.465477in}}%
\pgfpathcurveto{\pgfqpoint{5.843084in}{1.465477in}}{\pgfqpoint{5.853683in}{1.469868in}}{\pgfqpoint{5.861497in}{1.477681in}}%
\pgfpathcurveto{\pgfqpoint{5.869310in}{1.485495in}}{\pgfqpoint{5.873701in}{1.496094in}}{\pgfqpoint{5.873701in}{1.507144in}}%
\pgfpathcurveto{\pgfqpoint{5.873701in}{1.518194in}}{\pgfqpoint{5.869310in}{1.528793in}}{\pgfqpoint{5.861497in}{1.536607in}}%
\pgfpathcurveto{\pgfqpoint{5.853683in}{1.544421in}}{\pgfqpoint{5.843084in}{1.548811in}}{\pgfqpoint{5.832034in}{1.548811in}}%
\pgfpathcurveto{\pgfqpoint{5.820984in}{1.548811in}}{\pgfqpoint{5.810385in}{1.544421in}}{\pgfqpoint{5.802571in}{1.536607in}}%
\pgfpathcurveto{\pgfqpoint{5.794757in}{1.528793in}}{\pgfqpoint{5.790367in}{1.518194in}}{\pgfqpoint{5.790367in}{1.507144in}}%
\pgfpathcurveto{\pgfqpoint{5.790367in}{1.496094in}}{\pgfqpoint{5.794757in}{1.485495in}}{\pgfqpoint{5.802571in}{1.477681in}}%
\pgfpathcurveto{\pgfqpoint{5.810385in}{1.469868in}}{\pgfqpoint{5.820984in}{1.465477in}}{\pgfqpoint{5.832034in}{1.465477in}}%
\pgfpathclose%
\pgfusepath{stroke,fill}%
\end{pgfscope}%
\begin{pgfscope}%
\pgfpathrectangle{\pgfqpoint{0.481978in}{0.331635in}}{\pgfqpoint{9.300000in}{7.700000in}}%
\pgfusepath{clip}%
\pgfsetbuttcap%
\pgfsetroundjoin%
\definecolor{currentfill}{rgb}{1.000000,0.623529,0.607843}%
\pgfsetfillcolor{currentfill}%
\pgfsetlinewidth{0.481800pt}%
\definecolor{currentstroke}{rgb}{1.000000,1.000000,1.000000}%
\pgfsetstrokecolor{currentstroke}%
\pgfsetdash{}{0pt}%
\pgfpathmoveto{\pgfqpoint{8.011004in}{6.234740in}}%
\pgfpathcurveto{\pgfqpoint{8.022054in}{6.234740in}}{\pgfqpoint{8.032653in}{6.239130in}}{\pgfqpoint{8.040467in}{6.246944in}}%
\pgfpathcurveto{\pgfqpoint{8.048281in}{6.254757in}}{\pgfqpoint{8.052671in}{6.265356in}}{\pgfqpoint{8.052671in}{6.276407in}}%
\pgfpathcurveto{\pgfqpoint{8.052671in}{6.287457in}}{\pgfqpoint{8.048281in}{6.298056in}}{\pgfqpoint{8.040467in}{6.305869in}}%
\pgfpathcurveto{\pgfqpoint{8.032653in}{6.313683in}}{\pgfqpoint{8.022054in}{6.318073in}}{\pgfqpoint{8.011004in}{6.318073in}}%
\pgfpathcurveto{\pgfqpoint{7.999954in}{6.318073in}}{\pgfqpoint{7.989355in}{6.313683in}}{\pgfqpoint{7.981541in}{6.305869in}}%
\pgfpathcurveto{\pgfqpoint{7.973728in}{6.298056in}}{\pgfqpoint{7.969337in}{6.287457in}}{\pgfqpoint{7.969337in}{6.276407in}}%
\pgfpathcurveto{\pgfqpoint{7.969337in}{6.265356in}}{\pgfqpoint{7.973728in}{6.254757in}}{\pgfqpoint{7.981541in}{6.246944in}}%
\pgfpathcurveto{\pgfqpoint{7.989355in}{6.239130in}}{\pgfqpoint{7.999954in}{6.234740in}}{\pgfqpoint{8.011004in}{6.234740in}}%
\pgfpathclose%
\pgfusepath{stroke,fill}%
\end{pgfscope}%
\begin{pgfscope}%
\pgfpathrectangle{\pgfqpoint{0.481978in}{0.331635in}}{\pgfqpoint{9.300000in}{7.700000in}}%
\pgfusepath{clip}%
\pgfsetbuttcap%
\pgfsetroundjoin%
\definecolor{currentfill}{rgb}{1.000000,0.623529,0.607843}%
\pgfsetfillcolor{currentfill}%
\pgfsetlinewidth{0.481800pt}%
\definecolor{currentstroke}{rgb}{1.000000,1.000000,1.000000}%
\pgfsetstrokecolor{currentstroke}%
\pgfsetdash{}{0pt}%
\pgfpathmoveto{\pgfqpoint{2.908514in}{3.287743in}}%
\pgfpathcurveto{\pgfqpoint{2.919564in}{3.287743in}}{\pgfqpoint{2.930163in}{3.292133in}}{\pgfqpoint{2.937976in}{3.299947in}}%
\pgfpathcurveto{\pgfqpoint{2.945790in}{3.307760in}}{\pgfqpoint{2.950180in}{3.318359in}}{\pgfqpoint{2.950180in}{3.329410in}}%
\pgfpathcurveto{\pgfqpoint{2.950180in}{3.340460in}}{\pgfqpoint{2.945790in}{3.351059in}}{\pgfqpoint{2.937976in}{3.358872in}}%
\pgfpathcurveto{\pgfqpoint{2.930163in}{3.366686in}}{\pgfqpoint{2.919564in}{3.371076in}}{\pgfqpoint{2.908514in}{3.371076in}}%
\pgfpathcurveto{\pgfqpoint{2.897463in}{3.371076in}}{\pgfqpoint{2.886864in}{3.366686in}}{\pgfqpoint{2.879051in}{3.358872in}}%
\pgfpathcurveto{\pgfqpoint{2.871237in}{3.351059in}}{\pgfqpoint{2.866847in}{3.340460in}}{\pgfqpoint{2.866847in}{3.329410in}}%
\pgfpathcurveto{\pgfqpoint{2.866847in}{3.318359in}}{\pgfqpoint{2.871237in}{3.307760in}}{\pgfqpoint{2.879051in}{3.299947in}}%
\pgfpathcurveto{\pgfqpoint{2.886864in}{3.292133in}}{\pgfqpoint{2.897463in}{3.287743in}}{\pgfqpoint{2.908514in}{3.287743in}}%
\pgfpathclose%
\pgfusepath{stroke,fill}%
\end{pgfscope}%
\begin{pgfscope}%
\pgfpathrectangle{\pgfqpoint{0.481978in}{0.331635in}}{\pgfqpoint{9.300000in}{7.700000in}}%
\pgfusepath{clip}%
\pgfsetbuttcap%
\pgfsetroundjoin%
\definecolor{currentfill}{rgb}{1.000000,0.623529,0.607843}%
\pgfsetfillcolor{currentfill}%
\pgfsetlinewidth{0.481800pt}%
\definecolor{currentstroke}{rgb}{1.000000,1.000000,1.000000}%
\pgfsetstrokecolor{currentstroke}%
\pgfsetdash{}{0pt}%
\pgfpathmoveto{\pgfqpoint{6.465338in}{2.324944in}}%
\pgfpathcurveto{\pgfqpoint{6.476388in}{2.324944in}}{\pgfqpoint{6.486987in}{2.329335in}}{\pgfqpoint{6.494801in}{2.337148in}}%
\pgfpathcurveto{\pgfqpoint{6.502614in}{2.344962in}}{\pgfqpoint{6.507004in}{2.355561in}}{\pgfqpoint{6.507004in}{2.366611in}}%
\pgfpathcurveto{\pgfqpoint{6.507004in}{2.377661in}}{\pgfqpoint{6.502614in}{2.388260in}}{\pgfqpoint{6.494801in}{2.396074in}}%
\pgfpathcurveto{\pgfqpoint{6.486987in}{2.403887in}}{\pgfqpoint{6.476388in}{2.408278in}}{\pgfqpoint{6.465338in}{2.408278in}}%
\pgfpathcurveto{\pgfqpoint{6.454288in}{2.408278in}}{\pgfqpoint{6.443689in}{2.403887in}}{\pgfqpoint{6.435875in}{2.396074in}}%
\pgfpathcurveto{\pgfqpoint{6.428061in}{2.388260in}}{\pgfqpoint{6.423671in}{2.377661in}}{\pgfqpoint{6.423671in}{2.366611in}}%
\pgfpathcurveto{\pgfqpoint{6.423671in}{2.355561in}}{\pgfqpoint{6.428061in}{2.344962in}}{\pgfqpoint{6.435875in}{2.337148in}}%
\pgfpathcurveto{\pgfqpoint{6.443689in}{2.329335in}}{\pgfqpoint{6.454288in}{2.324944in}}{\pgfqpoint{6.465338in}{2.324944in}}%
\pgfpathclose%
\pgfusepath{stroke,fill}%
\end{pgfscope}%
\begin{pgfscope}%
\pgfpathrectangle{\pgfqpoint{0.481978in}{0.331635in}}{\pgfqpoint{9.300000in}{7.700000in}}%
\pgfusepath{clip}%
\pgfsetbuttcap%
\pgfsetroundjoin%
\definecolor{currentfill}{rgb}{1.000000,0.623529,0.607843}%
\pgfsetfillcolor{currentfill}%
\pgfsetlinewidth{0.481800pt}%
\definecolor{currentstroke}{rgb}{1.000000,1.000000,1.000000}%
\pgfsetstrokecolor{currentstroke}%
\pgfsetdash{}{0pt}%
\pgfpathmoveto{\pgfqpoint{8.106589in}{6.301680in}}%
\pgfpathcurveto{\pgfqpoint{8.117639in}{6.301680in}}{\pgfqpoint{8.128238in}{6.306070in}}{\pgfqpoint{8.136051in}{6.313883in}}%
\pgfpathcurveto{\pgfqpoint{8.143865in}{6.321697in}}{\pgfqpoint{8.148255in}{6.332296in}}{\pgfqpoint{8.148255in}{6.343346in}}%
\pgfpathcurveto{\pgfqpoint{8.148255in}{6.354396in}}{\pgfqpoint{8.143865in}{6.364995in}}{\pgfqpoint{8.136051in}{6.372809in}}%
\pgfpathcurveto{\pgfqpoint{8.128238in}{6.380623in}}{\pgfqpoint{8.117639in}{6.385013in}}{\pgfqpoint{8.106589in}{6.385013in}}%
\pgfpathcurveto{\pgfqpoint{8.095538in}{6.385013in}}{\pgfqpoint{8.084939in}{6.380623in}}{\pgfqpoint{8.077126in}{6.372809in}}%
\pgfpathcurveto{\pgfqpoint{8.069312in}{6.364995in}}{\pgfqpoint{8.064922in}{6.354396in}}{\pgfqpoint{8.064922in}{6.343346in}}%
\pgfpathcurveto{\pgfqpoint{8.064922in}{6.332296in}}{\pgfqpoint{8.069312in}{6.321697in}}{\pgfqpoint{8.077126in}{6.313883in}}%
\pgfpathcurveto{\pgfqpoint{8.084939in}{6.306070in}}{\pgfqpoint{8.095538in}{6.301680in}}{\pgfqpoint{8.106589in}{6.301680in}}%
\pgfpathclose%
\pgfusepath{stroke,fill}%
\end{pgfscope}%
\begin{pgfscope}%
\pgfpathrectangle{\pgfqpoint{0.481978in}{0.331635in}}{\pgfqpoint{9.300000in}{7.700000in}}%
\pgfusepath{clip}%
\pgfsetbuttcap%
\pgfsetroundjoin%
\definecolor{currentfill}{rgb}{1.000000,0.623529,0.607843}%
\pgfsetfillcolor{currentfill}%
\pgfsetlinewidth{0.481800pt}%
\definecolor{currentstroke}{rgb}{1.000000,1.000000,1.000000}%
\pgfsetstrokecolor{currentstroke}%
\pgfsetdash{}{0pt}%
\pgfpathmoveto{\pgfqpoint{7.991875in}{5.222385in}}%
\pgfpathcurveto{\pgfqpoint{8.002925in}{5.222385in}}{\pgfqpoint{8.013524in}{5.226776in}}{\pgfqpoint{8.021338in}{5.234589in}}%
\pgfpathcurveto{\pgfqpoint{8.029151in}{5.242403in}}{\pgfqpoint{8.033541in}{5.253002in}}{\pgfqpoint{8.033541in}{5.264052in}}%
\pgfpathcurveto{\pgfqpoint{8.033541in}{5.275102in}}{\pgfqpoint{8.029151in}{5.285701in}}{\pgfqpoint{8.021338in}{5.293515in}}%
\pgfpathcurveto{\pgfqpoint{8.013524in}{5.301328in}}{\pgfqpoint{8.002925in}{5.305719in}}{\pgfqpoint{7.991875in}{5.305719in}}%
\pgfpathcurveto{\pgfqpoint{7.980825in}{5.305719in}}{\pgfqpoint{7.970226in}{5.301328in}}{\pgfqpoint{7.962412in}{5.293515in}}%
\pgfpathcurveto{\pgfqpoint{7.954598in}{5.285701in}}{\pgfqpoint{7.950208in}{5.275102in}}{\pgfqpoint{7.950208in}{5.264052in}}%
\pgfpathcurveto{\pgfqpoint{7.950208in}{5.253002in}}{\pgfqpoint{7.954598in}{5.242403in}}{\pgfqpoint{7.962412in}{5.234589in}}%
\pgfpathcurveto{\pgfqpoint{7.970226in}{5.226776in}}{\pgfqpoint{7.980825in}{5.222385in}}{\pgfqpoint{7.991875in}{5.222385in}}%
\pgfpathclose%
\pgfusepath{stroke,fill}%
\end{pgfscope}%
\begin{pgfscope}%
\pgfpathrectangle{\pgfqpoint{0.481978in}{0.331635in}}{\pgfqpoint{9.300000in}{7.700000in}}%
\pgfusepath{clip}%
\pgfsetbuttcap%
\pgfsetroundjoin%
\definecolor{currentfill}{rgb}{1.000000,0.623529,0.607843}%
\pgfsetfillcolor{currentfill}%
\pgfsetlinewidth{0.481800pt}%
\definecolor{currentstroke}{rgb}{1.000000,1.000000,1.000000}%
\pgfsetstrokecolor{currentstroke}%
\pgfsetdash{}{0pt}%
\pgfpathmoveto{\pgfqpoint{7.423833in}{5.532170in}}%
\pgfpathcurveto{\pgfqpoint{7.434883in}{5.532170in}}{\pgfqpoint{7.445482in}{5.536561in}}{\pgfqpoint{7.453296in}{5.544374in}}%
\pgfpathcurveto{\pgfqpoint{7.461110in}{5.552188in}}{\pgfqpoint{7.465500in}{5.562787in}}{\pgfqpoint{7.465500in}{5.573837in}}%
\pgfpathcurveto{\pgfqpoint{7.465500in}{5.584887in}}{\pgfqpoint{7.461110in}{5.595486in}}{\pgfqpoint{7.453296in}{5.603300in}}%
\pgfpathcurveto{\pgfqpoint{7.445482in}{5.611113in}}{\pgfqpoint{7.434883in}{5.615504in}}{\pgfqpoint{7.423833in}{5.615504in}}%
\pgfpathcurveto{\pgfqpoint{7.412783in}{5.615504in}}{\pgfqpoint{7.402184in}{5.611113in}}{\pgfqpoint{7.394371in}{5.603300in}}%
\pgfpathcurveto{\pgfqpoint{7.386557in}{5.595486in}}{\pgfqpoint{7.382167in}{5.584887in}}{\pgfqpoint{7.382167in}{5.573837in}}%
\pgfpathcurveto{\pgfqpoint{7.382167in}{5.562787in}}{\pgfqpoint{7.386557in}{5.552188in}}{\pgfqpoint{7.394371in}{5.544374in}}%
\pgfpathcurveto{\pgfqpoint{7.402184in}{5.536561in}}{\pgfqpoint{7.412783in}{5.532170in}}{\pgfqpoint{7.423833in}{5.532170in}}%
\pgfpathclose%
\pgfusepath{stroke,fill}%
\end{pgfscope}%
\begin{pgfscope}%
\pgfpathrectangle{\pgfqpoint{0.481978in}{0.331635in}}{\pgfqpoint{9.300000in}{7.700000in}}%
\pgfusepath{clip}%
\pgfsetbuttcap%
\pgfsetroundjoin%
\definecolor{currentfill}{rgb}{1.000000,0.623529,0.607843}%
\pgfsetfillcolor{currentfill}%
\pgfsetlinewidth{0.481800pt}%
\definecolor{currentstroke}{rgb}{1.000000,1.000000,1.000000}%
\pgfsetstrokecolor{currentstroke}%
\pgfsetdash{}{0pt}%
\pgfpathmoveto{\pgfqpoint{4.192970in}{5.909242in}}%
\pgfpathcurveto{\pgfqpoint{4.204020in}{5.909242in}}{\pgfqpoint{4.214619in}{5.913632in}}{\pgfqpoint{4.222433in}{5.921445in}}%
\pgfpathcurveto{\pgfqpoint{4.230247in}{5.929259in}}{\pgfqpoint{4.234637in}{5.939858in}}{\pgfqpoint{4.234637in}{5.950908in}}%
\pgfpathcurveto{\pgfqpoint{4.234637in}{5.961958in}}{\pgfqpoint{4.230247in}{5.972557in}}{\pgfqpoint{4.222433in}{5.980371in}}%
\pgfpathcurveto{\pgfqpoint{4.214619in}{5.988185in}}{\pgfqpoint{4.204020in}{5.992575in}}{\pgfqpoint{4.192970in}{5.992575in}}%
\pgfpathcurveto{\pgfqpoint{4.181920in}{5.992575in}}{\pgfqpoint{4.171321in}{5.988185in}}{\pgfqpoint{4.163508in}{5.980371in}}%
\pgfpathcurveto{\pgfqpoint{4.155694in}{5.972557in}}{\pgfqpoint{4.151304in}{5.961958in}}{\pgfqpoint{4.151304in}{5.950908in}}%
\pgfpathcurveto{\pgfqpoint{4.151304in}{5.939858in}}{\pgfqpoint{4.155694in}{5.929259in}}{\pgfqpoint{4.163508in}{5.921445in}}%
\pgfpathcurveto{\pgfqpoint{4.171321in}{5.913632in}}{\pgfqpoint{4.181920in}{5.909242in}}{\pgfqpoint{4.192970in}{5.909242in}}%
\pgfpathclose%
\pgfusepath{stroke,fill}%
\end{pgfscope}%
\begin{pgfscope}%
\pgfpathrectangle{\pgfqpoint{0.481978in}{0.331635in}}{\pgfqpoint{9.300000in}{7.700000in}}%
\pgfusepath{clip}%
\pgfsetbuttcap%
\pgfsetroundjoin%
\definecolor{currentfill}{rgb}{1.000000,0.623529,0.607843}%
\pgfsetfillcolor{currentfill}%
\pgfsetlinewidth{0.481800pt}%
\definecolor{currentstroke}{rgb}{1.000000,1.000000,1.000000}%
\pgfsetstrokecolor{currentstroke}%
\pgfsetdash{}{0pt}%
\pgfpathmoveto{\pgfqpoint{6.770367in}{5.447632in}}%
\pgfpathcurveto{\pgfqpoint{6.781417in}{5.447632in}}{\pgfqpoint{6.792016in}{5.452022in}}{\pgfqpoint{6.799829in}{5.459836in}}%
\pgfpathcurveto{\pgfqpoint{6.807643in}{5.467649in}}{\pgfqpoint{6.812033in}{5.478248in}}{\pgfqpoint{6.812033in}{5.489298in}}%
\pgfpathcurveto{\pgfqpoint{6.812033in}{5.500349in}}{\pgfqpoint{6.807643in}{5.510948in}}{\pgfqpoint{6.799829in}{5.518761in}}%
\pgfpathcurveto{\pgfqpoint{6.792016in}{5.526575in}}{\pgfqpoint{6.781417in}{5.530965in}}{\pgfqpoint{6.770367in}{5.530965in}}%
\pgfpathcurveto{\pgfqpoint{6.759317in}{5.530965in}}{\pgfqpoint{6.748718in}{5.526575in}}{\pgfqpoint{6.740904in}{5.518761in}}%
\pgfpathcurveto{\pgfqpoint{6.733090in}{5.510948in}}{\pgfqpoint{6.728700in}{5.500349in}}{\pgfqpoint{6.728700in}{5.489298in}}%
\pgfpathcurveto{\pgfqpoint{6.728700in}{5.478248in}}{\pgfqpoint{6.733090in}{5.467649in}}{\pgfqpoint{6.740904in}{5.459836in}}%
\pgfpathcurveto{\pgfqpoint{6.748718in}{5.452022in}}{\pgfqpoint{6.759317in}{5.447632in}}{\pgfqpoint{6.770367in}{5.447632in}}%
\pgfpathclose%
\pgfusepath{stroke,fill}%
\end{pgfscope}%
\begin{pgfscope}%
\pgfpathrectangle{\pgfqpoint{0.481978in}{0.331635in}}{\pgfqpoint{9.300000in}{7.700000in}}%
\pgfusepath{clip}%
\pgfsetbuttcap%
\pgfsetroundjoin%
\definecolor{currentfill}{rgb}{1.000000,0.623529,0.607843}%
\pgfsetfillcolor{currentfill}%
\pgfsetlinewidth{0.481800pt}%
\definecolor{currentstroke}{rgb}{1.000000,1.000000,1.000000}%
\pgfsetstrokecolor{currentstroke}%
\pgfsetdash{}{0pt}%
\pgfpathmoveto{\pgfqpoint{7.446866in}{5.657018in}}%
\pgfpathcurveto{\pgfqpoint{7.457917in}{5.657018in}}{\pgfqpoint{7.468516in}{5.661409in}}{\pgfqpoint{7.476329in}{5.669222in}}%
\pgfpathcurveto{\pgfqpoint{7.484143in}{5.677036in}}{\pgfqpoint{7.488533in}{5.687635in}}{\pgfqpoint{7.488533in}{5.698685in}}%
\pgfpathcurveto{\pgfqpoint{7.488533in}{5.709735in}}{\pgfqpoint{7.484143in}{5.720334in}}{\pgfqpoint{7.476329in}{5.728148in}}%
\pgfpathcurveto{\pgfqpoint{7.468516in}{5.735961in}}{\pgfqpoint{7.457917in}{5.740352in}}{\pgfqpoint{7.446866in}{5.740352in}}%
\pgfpathcurveto{\pgfqpoint{7.435816in}{5.740352in}}{\pgfqpoint{7.425217in}{5.735961in}}{\pgfqpoint{7.417404in}{5.728148in}}%
\pgfpathcurveto{\pgfqpoint{7.409590in}{5.720334in}}{\pgfqpoint{7.405200in}{5.709735in}}{\pgfqpoint{7.405200in}{5.698685in}}%
\pgfpathcurveto{\pgfqpoint{7.405200in}{5.687635in}}{\pgfqpoint{7.409590in}{5.677036in}}{\pgfqpoint{7.417404in}{5.669222in}}%
\pgfpathcurveto{\pgfqpoint{7.425217in}{5.661409in}}{\pgfqpoint{7.435816in}{5.657018in}}{\pgfqpoint{7.446866in}{5.657018in}}%
\pgfpathclose%
\pgfusepath{stroke,fill}%
\end{pgfscope}%
\begin{pgfscope}%
\pgfpathrectangle{\pgfqpoint{0.481978in}{0.331635in}}{\pgfqpoint{9.300000in}{7.700000in}}%
\pgfusepath{clip}%
\pgfsetbuttcap%
\pgfsetroundjoin%
\definecolor{currentfill}{rgb}{1.000000,0.623529,0.607843}%
\pgfsetfillcolor{currentfill}%
\pgfsetlinewidth{0.481800pt}%
\definecolor{currentstroke}{rgb}{1.000000,1.000000,1.000000}%
\pgfsetstrokecolor{currentstroke}%
\pgfsetdash{}{0pt}%
\pgfpathmoveto{\pgfqpoint{4.422769in}{4.131207in}}%
\pgfpathcurveto{\pgfqpoint{4.433819in}{4.131207in}}{\pgfqpoint{4.444418in}{4.135598in}}{\pgfqpoint{4.452231in}{4.143411in}}%
\pgfpathcurveto{\pgfqpoint{4.460045in}{4.151225in}}{\pgfqpoint{4.464435in}{4.161824in}}{\pgfqpoint{4.464435in}{4.172874in}}%
\pgfpathcurveto{\pgfqpoint{4.464435in}{4.183924in}}{\pgfqpoint{4.460045in}{4.194523in}}{\pgfqpoint{4.452231in}{4.202337in}}%
\pgfpathcurveto{\pgfqpoint{4.444418in}{4.210150in}}{\pgfqpoint{4.433819in}{4.214541in}}{\pgfqpoint{4.422769in}{4.214541in}}%
\pgfpathcurveto{\pgfqpoint{4.411719in}{4.214541in}}{\pgfqpoint{4.401120in}{4.210150in}}{\pgfqpoint{4.393306in}{4.202337in}}%
\pgfpathcurveto{\pgfqpoint{4.385492in}{4.194523in}}{\pgfqpoint{4.381102in}{4.183924in}}{\pgfqpoint{4.381102in}{4.172874in}}%
\pgfpathcurveto{\pgfqpoint{4.381102in}{4.161824in}}{\pgfqpoint{4.385492in}{4.151225in}}{\pgfqpoint{4.393306in}{4.143411in}}%
\pgfpathcurveto{\pgfqpoint{4.401120in}{4.135598in}}{\pgfqpoint{4.411719in}{4.131207in}}{\pgfqpoint{4.422769in}{4.131207in}}%
\pgfpathclose%
\pgfusepath{stroke,fill}%
\end{pgfscope}%
\begin{pgfscope}%
\pgfpathrectangle{\pgfqpoint{0.481978in}{0.331635in}}{\pgfqpoint{9.300000in}{7.700000in}}%
\pgfusepath{clip}%
\pgfsetbuttcap%
\pgfsetroundjoin%
\definecolor{currentfill}{rgb}{1.000000,0.623529,0.607843}%
\pgfsetfillcolor{currentfill}%
\pgfsetlinewidth{0.481800pt}%
\definecolor{currentstroke}{rgb}{1.000000,1.000000,1.000000}%
\pgfsetstrokecolor{currentstroke}%
\pgfsetdash{}{0pt}%
\pgfpathmoveto{\pgfqpoint{4.632997in}{1.454063in}}%
\pgfpathcurveto{\pgfqpoint{4.644047in}{1.454063in}}{\pgfqpoint{4.654646in}{1.458453in}}{\pgfqpoint{4.662460in}{1.466266in}}%
\pgfpathcurveto{\pgfqpoint{4.670273in}{1.474080in}}{\pgfqpoint{4.674663in}{1.484679in}}{\pgfqpoint{4.674663in}{1.495729in}}%
\pgfpathcurveto{\pgfqpoint{4.674663in}{1.506779in}}{\pgfqpoint{4.670273in}{1.517378in}}{\pgfqpoint{4.662460in}{1.525192in}}%
\pgfpathcurveto{\pgfqpoint{4.654646in}{1.533006in}}{\pgfqpoint{4.644047in}{1.537396in}}{\pgfqpoint{4.632997in}{1.537396in}}%
\pgfpathcurveto{\pgfqpoint{4.621947in}{1.537396in}}{\pgfqpoint{4.611348in}{1.533006in}}{\pgfqpoint{4.603534in}{1.525192in}}%
\pgfpathcurveto{\pgfqpoint{4.595720in}{1.517378in}}{\pgfqpoint{4.591330in}{1.506779in}}{\pgfqpoint{4.591330in}{1.495729in}}%
\pgfpathcurveto{\pgfqpoint{4.591330in}{1.484679in}}{\pgfqpoint{4.595720in}{1.474080in}}{\pgfqpoint{4.603534in}{1.466266in}}%
\pgfpathcurveto{\pgfqpoint{4.611348in}{1.458453in}}{\pgfqpoint{4.621947in}{1.454063in}}{\pgfqpoint{4.632997in}{1.454063in}}%
\pgfpathclose%
\pgfusepath{stroke,fill}%
\end{pgfscope}%
\begin{pgfscope}%
\pgfpathrectangle{\pgfqpoint{0.481978in}{0.331635in}}{\pgfqpoint{9.300000in}{7.700000in}}%
\pgfusepath{clip}%
\pgfsetbuttcap%
\pgfsetroundjoin%
\definecolor{currentfill}{rgb}{1.000000,0.623529,0.607843}%
\pgfsetfillcolor{currentfill}%
\pgfsetlinewidth{0.481800pt}%
\definecolor{currentstroke}{rgb}{1.000000,1.000000,1.000000}%
\pgfsetstrokecolor{currentstroke}%
\pgfsetdash{}{0pt}%
\pgfpathmoveto{\pgfqpoint{3.319473in}{2.303496in}}%
\pgfpathcurveto{\pgfqpoint{3.330523in}{2.303496in}}{\pgfqpoint{3.341122in}{2.307886in}}{\pgfqpoint{3.348936in}{2.315699in}}%
\pgfpathcurveto{\pgfqpoint{3.356750in}{2.323513in}}{\pgfqpoint{3.361140in}{2.334112in}}{\pgfqpoint{3.361140in}{2.345162in}}%
\pgfpathcurveto{\pgfqpoint{3.361140in}{2.356212in}}{\pgfqpoint{3.356750in}{2.366811in}}{\pgfqpoint{3.348936in}{2.374625in}}%
\pgfpathcurveto{\pgfqpoint{3.341122in}{2.382439in}}{\pgfqpoint{3.330523in}{2.386829in}}{\pgfqpoint{3.319473in}{2.386829in}}%
\pgfpathcurveto{\pgfqpoint{3.308423in}{2.386829in}}{\pgfqpoint{3.297824in}{2.382439in}}{\pgfqpoint{3.290010in}{2.374625in}}%
\pgfpathcurveto{\pgfqpoint{3.282197in}{2.366811in}}{\pgfqpoint{3.277807in}{2.356212in}}{\pgfqpoint{3.277807in}{2.345162in}}%
\pgfpathcurveto{\pgfqpoint{3.277807in}{2.334112in}}{\pgfqpoint{3.282197in}{2.323513in}}{\pgfqpoint{3.290010in}{2.315699in}}%
\pgfpathcurveto{\pgfqpoint{3.297824in}{2.307886in}}{\pgfqpoint{3.308423in}{2.303496in}}{\pgfqpoint{3.319473in}{2.303496in}}%
\pgfpathclose%
\pgfusepath{stroke,fill}%
\end{pgfscope}%
\begin{pgfscope}%
\pgfpathrectangle{\pgfqpoint{0.481978in}{0.331635in}}{\pgfqpoint{9.300000in}{7.700000in}}%
\pgfusepath{clip}%
\pgfsetbuttcap%
\pgfsetroundjoin%
\definecolor{currentfill}{rgb}{1.000000,0.623529,0.607843}%
\pgfsetfillcolor{currentfill}%
\pgfsetlinewidth{0.481800pt}%
\definecolor{currentstroke}{rgb}{1.000000,1.000000,1.000000}%
\pgfsetstrokecolor{currentstroke}%
\pgfsetdash{}{0pt}%
\pgfpathmoveto{\pgfqpoint{3.661029in}{2.484883in}}%
\pgfpathcurveto{\pgfqpoint{3.672079in}{2.484883in}}{\pgfqpoint{3.682678in}{2.489274in}}{\pgfqpoint{3.690491in}{2.497087in}}%
\pgfpathcurveto{\pgfqpoint{3.698305in}{2.504901in}}{\pgfqpoint{3.702695in}{2.515500in}}{\pgfqpoint{3.702695in}{2.526550in}}%
\pgfpathcurveto{\pgfqpoint{3.702695in}{2.537600in}}{\pgfqpoint{3.698305in}{2.548199in}}{\pgfqpoint{3.690491in}{2.556013in}}%
\pgfpathcurveto{\pgfqpoint{3.682678in}{2.563826in}}{\pgfqpoint{3.672079in}{2.568217in}}{\pgfqpoint{3.661029in}{2.568217in}}%
\pgfpathcurveto{\pgfqpoint{3.649978in}{2.568217in}}{\pgfqpoint{3.639379in}{2.563826in}}{\pgfqpoint{3.631566in}{2.556013in}}%
\pgfpathcurveto{\pgfqpoint{3.623752in}{2.548199in}}{\pgfqpoint{3.619362in}{2.537600in}}{\pgfqpoint{3.619362in}{2.526550in}}%
\pgfpathcurveto{\pgfqpoint{3.619362in}{2.515500in}}{\pgfqpoint{3.623752in}{2.504901in}}{\pgfqpoint{3.631566in}{2.497087in}}%
\pgfpathcurveto{\pgfqpoint{3.639379in}{2.489274in}}{\pgfqpoint{3.649978in}{2.484883in}}{\pgfqpoint{3.661029in}{2.484883in}}%
\pgfpathclose%
\pgfusepath{stroke,fill}%
\end{pgfscope}%
\begin{pgfscope}%
\pgfpathrectangle{\pgfqpoint{0.481978in}{0.331635in}}{\pgfqpoint{9.300000in}{7.700000in}}%
\pgfusepath{clip}%
\pgfsetbuttcap%
\pgfsetroundjoin%
\definecolor{currentfill}{rgb}{1.000000,0.623529,0.607843}%
\pgfsetfillcolor{currentfill}%
\pgfsetlinewidth{0.481800pt}%
\definecolor{currentstroke}{rgb}{1.000000,1.000000,1.000000}%
\pgfsetstrokecolor{currentstroke}%
\pgfsetdash{}{0pt}%
\pgfpathmoveto{\pgfqpoint{3.845381in}{3.202561in}}%
\pgfpathcurveto{\pgfqpoint{3.856431in}{3.202561in}}{\pgfqpoint{3.867030in}{3.206951in}}{\pgfqpoint{3.874844in}{3.214765in}}%
\pgfpathcurveto{\pgfqpoint{3.882657in}{3.222578in}}{\pgfqpoint{3.887048in}{3.233177in}}{\pgfqpoint{3.887048in}{3.244228in}}%
\pgfpathcurveto{\pgfqpoint{3.887048in}{3.255278in}}{\pgfqpoint{3.882657in}{3.265877in}}{\pgfqpoint{3.874844in}{3.273690in}}%
\pgfpathcurveto{\pgfqpoint{3.867030in}{3.281504in}}{\pgfqpoint{3.856431in}{3.285894in}}{\pgfqpoint{3.845381in}{3.285894in}}%
\pgfpathcurveto{\pgfqpoint{3.834331in}{3.285894in}}{\pgfqpoint{3.823732in}{3.281504in}}{\pgfqpoint{3.815918in}{3.273690in}}%
\pgfpathcurveto{\pgfqpoint{3.808105in}{3.265877in}}{\pgfqpoint{3.803714in}{3.255278in}}{\pgfqpoint{3.803714in}{3.244228in}}%
\pgfpathcurveto{\pgfqpoint{3.803714in}{3.233177in}}{\pgfqpoint{3.808105in}{3.222578in}}{\pgfqpoint{3.815918in}{3.214765in}}%
\pgfpathcurveto{\pgfqpoint{3.823732in}{3.206951in}}{\pgfqpoint{3.834331in}{3.202561in}}{\pgfqpoint{3.845381in}{3.202561in}}%
\pgfpathclose%
\pgfusepath{stroke,fill}%
\end{pgfscope}%
\begin{pgfscope}%
\pgfpathrectangle{\pgfqpoint{0.481978in}{0.331635in}}{\pgfqpoint{9.300000in}{7.700000in}}%
\pgfusepath{clip}%
\pgfsetbuttcap%
\pgfsetroundjoin%
\definecolor{currentfill}{rgb}{1.000000,0.623529,0.607843}%
\pgfsetfillcolor{currentfill}%
\pgfsetlinewidth{0.481800pt}%
\definecolor{currentstroke}{rgb}{1.000000,1.000000,1.000000}%
\pgfsetstrokecolor{currentstroke}%
\pgfsetdash{}{0pt}%
\pgfpathmoveto{\pgfqpoint{4.050059in}{3.494463in}}%
\pgfpathcurveto{\pgfqpoint{4.061109in}{3.494463in}}{\pgfqpoint{4.071708in}{3.498853in}}{\pgfqpoint{4.079522in}{3.506667in}}%
\pgfpathcurveto{\pgfqpoint{4.087336in}{3.514480in}}{\pgfqpoint{4.091726in}{3.525079in}}{\pgfqpoint{4.091726in}{3.536130in}}%
\pgfpathcurveto{\pgfqpoint{4.091726in}{3.547180in}}{\pgfqpoint{4.087336in}{3.557779in}}{\pgfqpoint{4.079522in}{3.565592in}}%
\pgfpathcurveto{\pgfqpoint{4.071708in}{3.573406in}}{\pgfqpoint{4.061109in}{3.577796in}}{\pgfqpoint{4.050059in}{3.577796in}}%
\pgfpathcurveto{\pgfqpoint{4.039009in}{3.577796in}}{\pgfqpoint{4.028410in}{3.573406in}}{\pgfqpoint{4.020596in}{3.565592in}}%
\pgfpathcurveto{\pgfqpoint{4.012783in}{3.557779in}}{\pgfqpoint{4.008393in}{3.547180in}}{\pgfqpoint{4.008393in}{3.536130in}}%
\pgfpathcurveto{\pgfqpoint{4.008393in}{3.525079in}}{\pgfqpoint{4.012783in}{3.514480in}}{\pgfqpoint{4.020596in}{3.506667in}}%
\pgfpathcurveto{\pgfqpoint{4.028410in}{3.498853in}}{\pgfqpoint{4.039009in}{3.494463in}}{\pgfqpoint{4.050059in}{3.494463in}}%
\pgfpathclose%
\pgfusepath{stroke,fill}%
\end{pgfscope}%
\begin{pgfscope}%
\pgfpathrectangle{\pgfqpoint{0.481978in}{0.331635in}}{\pgfqpoint{9.300000in}{7.700000in}}%
\pgfusepath{clip}%
\pgfsetbuttcap%
\pgfsetroundjoin%
\definecolor{currentfill}{rgb}{1.000000,0.623529,0.607843}%
\pgfsetfillcolor{currentfill}%
\pgfsetlinewidth{0.481800pt}%
\definecolor{currentstroke}{rgb}{1.000000,1.000000,1.000000}%
\pgfsetstrokecolor{currentstroke}%
\pgfsetdash{}{0pt}%
\pgfpathmoveto{\pgfqpoint{4.925237in}{7.423806in}}%
\pgfpathcurveto{\pgfqpoint{4.936287in}{7.423806in}}{\pgfqpoint{4.946887in}{7.428197in}}{\pgfqpoint{4.954700in}{7.436010in}}%
\pgfpathcurveto{\pgfqpoint{4.962514in}{7.443824in}}{\pgfqpoint{4.966904in}{7.454423in}}{\pgfqpoint{4.966904in}{7.465473in}}%
\pgfpathcurveto{\pgfqpoint{4.966904in}{7.476523in}}{\pgfqpoint{4.962514in}{7.487122in}}{\pgfqpoint{4.954700in}{7.494936in}}%
\pgfpathcurveto{\pgfqpoint{4.946887in}{7.502749in}}{\pgfqpoint{4.936287in}{7.507140in}}{\pgfqpoint{4.925237in}{7.507140in}}%
\pgfpathcurveto{\pgfqpoint{4.914187in}{7.507140in}}{\pgfqpoint{4.903588in}{7.502749in}}{\pgfqpoint{4.895775in}{7.494936in}}%
\pgfpathcurveto{\pgfqpoint{4.887961in}{7.487122in}}{\pgfqpoint{4.883571in}{7.476523in}}{\pgfqpoint{4.883571in}{7.465473in}}%
\pgfpathcurveto{\pgfqpoint{4.883571in}{7.454423in}}{\pgfqpoint{4.887961in}{7.443824in}}{\pgfqpoint{4.895775in}{7.436010in}}%
\pgfpathcurveto{\pgfqpoint{4.903588in}{7.428197in}}{\pgfqpoint{4.914187in}{7.423806in}}{\pgfqpoint{4.925237in}{7.423806in}}%
\pgfpathclose%
\pgfusepath{stroke,fill}%
\end{pgfscope}%
\begin{pgfscope}%
\pgfpathrectangle{\pgfqpoint{0.481978in}{0.331635in}}{\pgfqpoint{9.300000in}{7.700000in}}%
\pgfusepath{clip}%
\pgfsetbuttcap%
\pgfsetroundjoin%
\definecolor{currentfill}{rgb}{1.000000,0.623529,0.607843}%
\pgfsetfillcolor{currentfill}%
\pgfsetlinewidth{0.481800pt}%
\definecolor{currentstroke}{rgb}{1.000000,1.000000,1.000000}%
\pgfsetstrokecolor{currentstroke}%
\pgfsetdash{}{0pt}%
\pgfpathmoveto{\pgfqpoint{6.991308in}{3.546324in}}%
\pgfpathcurveto{\pgfqpoint{7.002358in}{3.546324in}}{\pgfqpoint{7.012957in}{3.550714in}}{\pgfqpoint{7.020771in}{3.558528in}}%
\pgfpathcurveto{\pgfqpoint{7.028584in}{3.566341in}}{\pgfqpoint{7.032975in}{3.576940in}}{\pgfqpoint{7.032975in}{3.587991in}}%
\pgfpathcurveto{\pgfqpoint{7.032975in}{3.599041in}}{\pgfqpoint{7.028584in}{3.609640in}}{\pgfqpoint{7.020771in}{3.617453in}}%
\pgfpathcurveto{\pgfqpoint{7.012957in}{3.625267in}}{\pgfqpoint{7.002358in}{3.629657in}}{\pgfqpoint{6.991308in}{3.629657in}}%
\pgfpathcurveto{\pgfqpoint{6.980258in}{3.629657in}}{\pgfqpoint{6.969659in}{3.625267in}}{\pgfqpoint{6.961845in}{3.617453in}}%
\pgfpathcurveto{\pgfqpoint{6.954032in}{3.609640in}}{\pgfqpoint{6.949641in}{3.599041in}}{\pgfqpoint{6.949641in}{3.587991in}}%
\pgfpathcurveto{\pgfqpoint{6.949641in}{3.576940in}}{\pgfqpoint{6.954032in}{3.566341in}}{\pgfqpoint{6.961845in}{3.558528in}}%
\pgfpathcurveto{\pgfqpoint{6.969659in}{3.550714in}}{\pgfqpoint{6.980258in}{3.546324in}}{\pgfqpoint{6.991308in}{3.546324in}}%
\pgfpathclose%
\pgfusepath{stroke,fill}%
\end{pgfscope}%
\begin{pgfscope}%
\pgfpathrectangle{\pgfqpoint{0.481978in}{0.331635in}}{\pgfqpoint{9.300000in}{7.700000in}}%
\pgfusepath{clip}%
\pgfsetbuttcap%
\pgfsetroundjoin%
\definecolor{currentfill}{rgb}{1.000000,0.623529,0.607843}%
\pgfsetfillcolor{currentfill}%
\pgfsetlinewidth{0.481800pt}%
\definecolor{currentstroke}{rgb}{1.000000,1.000000,1.000000}%
\pgfsetstrokecolor{currentstroke}%
\pgfsetdash{}{0pt}%
\pgfpathmoveto{\pgfqpoint{8.618470in}{6.020902in}}%
\pgfpathcurveto{\pgfqpoint{8.629520in}{6.020902in}}{\pgfqpoint{8.640119in}{6.025293in}}{\pgfqpoint{8.647933in}{6.033106in}}%
\pgfpathcurveto{\pgfqpoint{8.655747in}{6.040920in}}{\pgfqpoint{8.660137in}{6.051519in}}{\pgfqpoint{8.660137in}{6.062569in}}%
\pgfpathcurveto{\pgfqpoint{8.660137in}{6.073619in}}{\pgfqpoint{8.655747in}{6.084218in}}{\pgfqpoint{8.647933in}{6.092032in}}%
\pgfpathcurveto{\pgfqpoint{8.640119in}{6.099845in}}{\pgfqpoint{8.629520in}{6.104236in}}{\pgfqpoint{8.618470in}{6.104236in}}%
\pgfpathcurveto{\pgfqpoint{8.607420in}{6.104236in}}{\pgfqpoint{8.596821in}{6.099845in}}{\pgfqpoint{8.589007in}{6.092032in}}%
\pgfpathcurveto{\pgfqpoint{8.581194in}{6.084218in}}{\pgfqpoint{8.576804in}{6.073619in}}{\pgfqpoint{8.576804in}{6.062569in}}%
\pgfpathcurveto{\pgfqpoint{8.576804in}{6.051519in}}{\pgfqpoint{8.581194in}{6.040920in}}{\pgfqpoint{8.589007in}{6.033106in}}%
\pgfpathcurveto{\pgfqpoint{8.596821in}{6.025293in}}{\pgfqpoint{8.607420in}{6.020902in}}{\pgfqpoint{8.618470in}{6.020902in}}%
\pgfpathclose%
\pgfusepath{stroke,fill}%
\end{pgfscope}%
\begin{pgfscope}%
\pgfpathrectangle{\pgfqpoint{0.481978in}{0.331635in}}{\pgfqpoint{9.300000in}{7.700000in}}%
\pgfusepath{clip}%
\pgfsetbuttcap%
\pgfsetroundjoin%
\definecolor{currentfill}{rgb}{1.000000,0.623529,0.607843}%
\pgfsetfillcolor{currentfill}%
\pgfsetlinewidth{0.481800pt}%
\definecolor{currentstroke}{rgb}{1.000000,1.000000,1.000000}%
\pgfsetstrokecolor{currentstroke}%
\pgfsetdash{}{0pt}%
\pgfpathmoveto{\pgfqpoint{7.337079in}{5.217206in}}%
\pgfpathcurveto{\pgfqpoint{7.348130in}{5.217206in}}{\pgfqpoint{7.358729in}{5.221596in}}{\pgfqpoint{7.366542in}{5.229410in}}%
\pgfpathcurveto{\pgfqpoint{7.374356in}{5.237223in}}{\pgfqpoint{7.378746in}{5.247822in}}{\pgfqpoint{7.378746in}{5.258872in}}%
\pgfpathcurveto{\pgfqpoint{7.378746in}{5.269922in}}{\pgfqpoint{7.374356in}{5.280521in}}{\pgfqpoint{7.366542in}{5.288335in}}%
\pgfpathcurveto{\pgfqpoint{7.358729in}{5.296149in}}{\pgfqpoint{7.348130in}{5.300539in}}{\pgfqpoint{7.337079in}{5.300539in}}%
\pgfpathcurveto{\pgfqpoint{7.326029in}{5.300539in}}{\pgfqpoint{7.315430in}{5.296149in}}{\pgfqpoint{7.307617in}{5.288335in}}%
\pgfpathcurveto{\pgfqpoint{7.299803in}{5.280521in}}{\pgfqpoint{7.295413in}{5.269922in}}{\pgfqpoint{7.295413in}{5.258872in}}%
\pgfpathcurveto{\pgfqpoint{7.295413in}{5.247822in}}{\pgfqpoint{7.299803in}{5.237223in}}{\pgfqpoint{7.307617in}{5.229410in}}%
\pgfpathcurveto{\pgfqpoint{7.315430in}{5.221596in}}{\pgfqpoint{7.326029in}{5.217206in}}{\pgfqpoint{7.337079in}{5.217206in}}%
\pgfpathclose%
\pgfusepath{stroke,fill}%
\end{pgfscope}%
\begin{pgfscope}%
\pgfpathrectangle{\pgfqpoint{0.481978in}{0.331635in}}{\pgfqpoint{9.300000in}{7.700000in}}%
\pgfusepath{clip}%
\pgfsetbuttcap%
\pgfsetroundjoin%
\definecolor{currentfill}{rgb}{1.000000,0.623529,0.607843}%
\pgfsetfillcolor{currentfill}%
\pgfsetlinewidth{0.481800pt}%
\definecolor{currentstroke}{rgb}{1.000000,1.000000,1.000000}%
\pgfsetstrokecolor{currentstroke}%
\pgfsetdash{}{0pt}%
\pgfpathmoveto{\pgfqpoint{4.553085in}{2.981640in}}%
\pgfpathcurveto{\pgfqpoint{4.564135in}{2.981640in}}{\pgfqpoint{4.574734in}{2.986030in}}{\pgfqpoint{4.582547in}{2.993844in}}%
\pgfpathcurveto{\pgfqpoint{4.590361in}{3.001658in}}{\pgfqpoint{4.594751in}{3.012257in}}{\pgfqpoint{4.594751in}{3.023307in}}%
\pgfpathcurveto{\pgfqpoint{4.594751in}{3.034357in}}{\pgfqpoint{4.590361in}{3.044956in}}{\pgfqpoint{4.582547in}{3.052769in}}%
\pgfpathcurveto{\pgfqpoint{4.574734in}{3.060583in}}{\pgfqpoint{4.564135in}{3.064973in}}{\pgfqpoint{4.553085in}{3.064973in}}%
\pgfpathcurveto{\pgfqpoint{4.542035in}{3.064973in}}{\pgfqpoint{4.531436in}{3.060583in}}{\pgfqpoint{4.523622in}{3.052769in}}%
\pgfpathcurveto{\pgfqpoint{4.515808in}{3.044956in}}{\pgfqpoint{4.511418in}{3.034357in}}{\pgfqpoint{4.511418in}{3.023307in}}%
\pgfpathcurveto{\pgfqpoint{4.511418in}{3.012257in}}{\pgfqpoint{4.515808in}{3.001658in}}{\pgfqpoint{4.523622in}{2.993844in}}%
\pgfpathcurveto{\pgfqpoint{4.531436in}{2.986030in}}{\pgfqpoint{4.542035in}{2.981640in}}{\pgfqpoint{4.553085in}{2.981640in}}%
\pgfpathclose%
\pgfusepath{stroke,fill}%
\end{pgfscope}%
\begin{pgfscope}%
\pgfpathrectangle{\pgfqpoint{0.481978in}{0.331635in}}{\pgfqpoint{9.300000in}{7.700000in}}%
\pgfusepath{clip}%
\pgfsetbuttcap%
\pgfsetroundjoin%
\definecolor{currentfill}{rgb}{1.000000,0.623529,0.607843}%
\pgfsetfillcolor{currentfill}%
\pgfsetlinewidth{0.481800pt}%
\definecolor{currentstroke}{rgb}{1.000000,1.000000,1.000000}%
\pgfsetstrokecolor{currentstroke}%
\pgfsetdash{}{0pt}%
\pgfpathmoveto{\pgfqpoint{9.095716in}{5.848566in}}%
\pgfpathcurveto{\pgfqpoint{9.106766in}{5.848566in}}{\pgfqpoint{9.117365in}{5.852956in}}{\pgfqpoint{9.125179in}{5.860770in}}%
\pgfpathcurveto{\pgfqpoint{9.132993in}{5.868584in}}{\pgfqpoint{9.137383in}{5.879183in}}{\pgfqpoint{9.137383in}{5.890233in}}%
\pgfpathcurveto{\pgfqpoint{9.137383in}{5.901283in}}{\pgfqpoint{9.132993in}{5.911882in}}{\pgfqpoint{9.125179in}{5.919696in}}%
\pgfpathcurveto{\pgfqpoint{9.117365in}{5.927509in}}{\pgfqpoint{9.106766in}{5.931899in}}{\pgfqpoint{9.095716in}{5.931899in}}%
\pgfpathcurveto{\pgfqpoint{9.084666in}{5.931899in}}{\pgfqpoint{9.074067in}{5.927509in}}{\pgfqpoint{9.066253in}{5.919696in}}%
\pgfpathcurveto{\pgfqpoint{9.058440in}{5.911882in}}{\pgfqpoint{9.054050in}{5.901283in}}{\pgfqpoint{9.054050in}{5.890233in}}%
\pgfpathcurveto{\pgfqpoint{9.054050in}{5.879183in}}{\pgfqpoint{9.058440in}{5.868584in}}{\pgfqpoint{9.066253in}{5.860770in}}%
\pgfpathcurveto{\pgfqpoint{9.074067in}{5.852956in}}{\pgfqpoint{9.084666in}{5.848566in}}{\pgfqpoint{9.095716in}{5.848566in}}%
\pgfpathclose%
\pgfusepath{stroke,fill}%
\end{pgfscope}%
\begin{pgfscope}%
\pgfpathrectangle{\pgfqpoint{0.481978in}{0.331635in}}{\pgfqpoint{9.300000in}{7.700000in}}%
\pgfusepath{clip}%
\pgfsetbuttcap%
\pgfsetroundjoin%
\definecolor{currentfill}{rgb}{1.000000,0.623529,0.607843}%
\pgfsetfillcolor{currentfill}%
\pgfsetlinewidth{0.481800pt}%
\definecolor{currentstroke}{rgb}{1.000000,1.000000,1.000000}%
\pgfsetstrokecolor{currentstroke}%
\pgfsetdash{}{0pt}%
\pgfpathmoveto{\pgfqpoint{2.403697in}{3.430362in}}%
\pgfpathcurveto{\pgfqpoint{2.414747in}{3.430362in}}{\pgfqpoint{2.425346in}{3.434752in}}{\pgfqpoint{2.433160in}{3.442566in}}%
\pgfpathcurveto{\pgfqpoint{2.440973in}{3.450380in}}{\pgfqpoint{2.445364in}{3.460979in}}{\pgfqpoint{2.445364in}{3.472029in}}%
\pgfpathcurveto{\pgfqpoint{2.445364in}{3.483079in}}{\pgfqpoint{2.440973in}{3.493678in}}{\pgfqpoint{2.433160in}{3.501491in}}%
\pgfpathcurveto{\pgfqpoint{2.425346in}{3.509305in}}{\pgfqpoint{2.414747in}{3.513695in}}{\pgfqpoint{2.403697in}{3.513695in}}%
\pgfpathcurveto{\pgfqpoint{2.392647in}{3.513695in}}{\pgfqpoint{2.382048in}{3.509305in}}{\pgfqpoint{2.374234in}{3.501491in}}%
\pgfpathcurveto{\pgfqpoint{2.366420in}{3.493678in}}{\pgfqpoint{2.362030in}{3.483079in}}{\pgfqpoint{2.362030in}{3.472029in}}%
\pgfpathcurveto{\pgfqpoint{2.362030in}{3.460979in}}{\pgfqpoint{2.366420in}{3.450380in}}{\pgfqpoint{2.374234in}{3.442566in}}%
\pgfpathcurveto{\pgfqpoint{2.382048in}{3.434752in}}{\pgfqpoint{2.392647in}{3.430362in}}{\pgfqpoint{2.403697in}{3.430362in}}%
\pgfpathclose%
\pgfusepath{stroke,fill}%
\end{pgfscope}%
\begin{pgfscope}%
\pgfpathrectangle{\pgfqpoint{0.481978in}{0.331635in}}{\pgfqpoint{9.300000in}{7.700000in}}%
\pgfusepath{clip}%
\pgfsetbuttcap%
\pgfsetroundjoin%
\definecolor{currentfill}{rgb}{1.000000,0.623529,0.607843}%
\pgfsetfillcolor{currentfill}%
\pgfsetlinewidth{0.481800pt}%
\definecolor{currentstroke}{rgb}{1.000000,1.000000,1.000000}%
\pgfsetstrokecolor{currentstroke}%
\pgfsetdash{}{0pt}%
\pgfpathmoveto{\pgfqpoint{7.815154in}{4.410412in}}%
\pgfpathcurveto{\pgfqpoint{7.826205in}{4.410412in}}{\pgfqpoint{7.836804in}{4.414802in}}{\pgfqpoint{7.844617in}{4.422616in}}%
\pgfpathcurveto{\pgfqpoint{7.852431in}{4.430429in}}{\pgfqpoint{7.856821in}{4.441028in}}{\pgfqpoint{7.856821in}{4.452079in}}%
\pgfpathcurveto{\pgfqpoint{7.856821in}{4.463129in}}{\pgfqpoint{7.852431in}{4.473728in}}{\pgfqpoint{7.844617in}{4.481541in}}%
\pgfpathcurveto{\pgfqpoint{7.836804in}{4.489355in}}{\pgfqpoint{7.826205in}{4.493745in}}{\pgfqpoint{7.815154in}{4.493745in}}%
\pgfpathcurveto{\pgfqpoint{7.804104in}{4.493745in}}{\pgfqpoint{7.793505in}{4.489355in}}{\pgfqpoint{7.785692in}{4.481541in}}%
\pgfpathcurveto{\pgfqpoint{7.777878in}{4.473728in}}{\pgfqpoint{7.773488in}{4.463129in}}{\pgfqpoint{7.773488in}{4.452079in}}%
\pgfpathcurveto{\pgfqpoint{7.773488in}{4.441028in}}{\pgfqpoint{7.777878in}{4.430429in}}{\pgfqpoint{7.785692in}{4.422616in}}%
\pgfpathcurveto{\pgfqpoint{7.793505in}{4.414802in}}{\pgfqpoint{7.804104in}{4.410412in}}{\pgfqpoint{7.815154in}{4.410412in}}%
\pgfpathclose%
\pgfusepath{stroke,fill}%
\end{pgfscope}%
\begin{pgfscope}%
\pgfpathrectangle{\pgfqpoint{0.481978in}{0.331635in}}{\pgfqpoint{9.300000in}{7.700000in}}%
\pgfusepath{clip}%
\pgfsetbuttcap%
\pgfsetroundjoin%
\definecolor{currentfill}{rgb}{1.000000,0.623529,0.607843}%
\pgfsetfillcolor{currentfill}%
\pgfsetlinewidth{0.481800pt}%
\definecolor{currentstroke}{rgb}{1.000000,1.000000,1.000000}%
\pgfsetstrokecolor{currentstroke}%
\pgfsetdash{}{0pt}%
\pgfpathmoveto{\pgfqpoint{4.473626in}{1.982566in}}%
\pgfpathcurveto{\pgfqpoint{4.484676in}{1.982566in}}{\pgfqpoint{4.495275in}{1.986956in}}{\pgfqpoint{4.503089in}{1.994770in}}%
\pgfpathcurveto{\pgfqpoint{4.510902in}{2.002583in}}{\pgfqpoint{4.515293in}{2.013182in}}{\pgfqpoint{4.515293in}{2.024232in}}%
\pgfpathcurveto{\pgfqpoint{4.515293in}{2.035282in}}{\pgfqpoint{4.510902in}{2.045881in}}{\pgfqpoint{4.503089in}{2.053695in}}%
\pgfpathcurveto{\pgfqpoint{4.495275in}{2.061509in}}{\pgfqpoint{4.484676in}{2.065899in}}{\pgfqpoint{4.473626in}{2.065899in}}%
\pgfpathcurveto{\pgfqpoint{4.462576in}{2.065899in}}{\pgfqpoint{4.451977in}{2.061509in}}{\pgfqpoint{4.444163in}{2.053695in}}%
\pgfpathcurveto{\pgfqpoint{4.436350in}{2.045881in}}{\pgfqpoint{4.431959in}{2.035282in}}{\pgfqpoint{4.431959in}{2.024232in}}%
\pgfpathcurveto{\pgfqpoint{4.431959in}{2.013182in}}{\pgfqpoint{4.436350in}{2.002583in}}{\pgfqpoint{4.444163in}{1.994770in}}%
\pgfpathcurveto{\pgfqpoint{4.451977in}{1.986956in}}{\pgfqpoint{4.462576in}{1.982566in}}{\pgfqpoint{4.473626in}{1.982566in}}%
\pgfpathclose%
\pgfusepath{stroke,fill}%
\end{pgfscope}%
\begin{pgfscope}%
\pgfpathrectangle{\pgfqpoint{0.481978in}{0.331635in}}{\pgfqpoint{9.300000in}{7.700000in}}%
\pgfusepath{clip}%
\pgfsetbuttcap%
\pgfsetroundjoin%
\definecolor{currentfill}{rgb}{0.815686,0.733333,1.000000}%
\pgfsetfillcolor{currentfill}%
\pgfsetlinewidth{0.481800pt}%
\definecolor{currentstroke}{rgb}{1.000000,1.000000,1.000000}%
\pgfsetstrokecolor{currentstroke}%
\pgfsetdash{}{0pt}%
\pgfpathmoveto{\pgfqpoint{7.152848in}{5.351280in}}%
\pgfpathcurveto{\pgfqpoint{7.163898in}{5.351280in}}{\pgfqpoint{7.174497in}{5.355670in}}{\pgfqpoint{7.182311in}{5.363484in}}%
\pgfpathcurveto{\pgfqpoint{7.190125in}{5.371297in}}{\pgfqpoint{7.194515in}{5.381896in}}{\pgfqpoint{7.194515in}{5.392946in}}%
\pgfpathcurveto{\pgfqpoint{7.194515in}{5.403996in}}{\pgfqpoint{7.190125in}{5.414596in}}{\pgfqpoint{7.182311in}{5.422409in}}%
\pgfpathcurveto{\pgfqpoint{7.174497in}{5.430223in}}{\pgfqpoint{7.163898in}{5.434613in}}{\pgfqpoint{7.152848in}{5.434613in}}%
\pgfpathcurveto{\pgfqpoint{7.141798in}{5.434613in}}{\pgfqpoint{7.131199in}{5.430223in}}{\pgfqpoint{7.123385in}{5.422409in}}%
\pgfpathcurveto{\pgfqpoint{7.115572in}{5.414596in}}{\pgfqpoint{7.111181in}{5.403996in}}{\pgfqpoint{7.111181in}{5.392946in}}%
\pgfpathcurveto{\pgfqpoint{7.111181in}{5.381896in}}{\pgfqpoint{7.115572in}{5.371297in}}{\pgfqpoint{7.123385in}{5.363484in}}%
\pgfpathcurveto{\pgfqpoint{7.131199in}{5.355670in}}{\pgfqpoint{7.141798in}{5.351280in}}{\pgfqpoint{7.152848in}{5.351280in}}%
\pgfpathclose%
\pgfusepath{stroke,fill}%
\end{pgfscope}%
\begin{pgfscope}%
\pgfpathrectangle{\pgfqpoint{0.481978in}{0.331635in}}{\pgfqpoint{9.300000in}{7.700000in}}%
\pgfusepath{clip}%
\pgfsetbuttcap%
\pgfsetroundjoin%
\definecolor{currentfill}{rgb}{0.815686,0.733333,1.000000}%
\pgfsetfillcolor{currentfill}%
\pgfsetlinewidth{0.481800pt}%
\definecolor{currentstroke}{rgb}{1.000000,1.000000,1.000000}%
\pgfsetstrokecolor{currentstroke}%
\pgfsetdash{}{0pt}%
\pgfpathmoveto{\pgfqpoint{7.702166in}{5.065536in}}%
\pgfpathcurveto{\pgfqpoint{7.713216in}{5.065536in}}{\pgfqpoint{7.723815in}{5.069927in}}{\pgfqpoint{7.731629in}{5.077740in}}%
\pgfpathcurveto{\pgfqpoint{7.739442in}{5.085554in}}{\pgfqpoint{7.743832in}{5.096153in}}{\pgfqpoint{7.743832in}{5.107203in}}%
\pgfpathcurveto{\pgfqpoint{7.743832in}{5.118253in}}{\pgfqpoint{7.739442in}{5.128852in}}{\pgfqpoint{7.731629in}{5.136666in}}%
\pgfpathcurveto{\pgfqpoint{7.723815in}{5.144480in}}{\pgfqpoint{7.713216in}{5.148870in}}{\pgfqpoint{7.702166in}{5.148870in}}%
\pgfpathcurveto{\pgfqpoint{7.691116in}{5.148870in}}{\pgfqpoint{7.680517in}{5.144480in}}{\pgfqpoint{7.672703in}{5.136666in}}%
\pgfpathcurveto{\pgfqpoint{7.664889in}{5.128852in}}{\pgfqpoint{7.660499in}{5.118253in}}{\pgfqpoint{7.660499in}{5.107203in}}%
\pgfpathcurveto{\pgfqpoint{7.660499in}{5.096153in}}{\pgfqpoint{7.664889in}{5.085554in}}{\pgfqpoint{7.672703in}{5.077740in}}%
\pgfpathcurveto{\pgfqpoint{7.680517in}{5.069927in}}{\pgfqpoint{7.691116in}{5.065536in}}{\pgfqpoint{7.702166in}{5.065536in}}%
\pgfpathclose%
\pgfusepath{stroke,fill}%
\end{pgfscope}%
\begin{pgfscope}%
\pgfpathrectangle{\pgfqpoint{0.481978in}{0.331635in}}{\pgfqpoint{9.300000in}{7.700000in}}%
\pgfusepath{clip}%
\pgfsetbuttcap%
\pgfsetroundjoin%
\definecolor{currentfill}{rgb}{0.815686,0.733333,1.000000}%
\pgfsetfillcolor{currentfill}%
\pgfsetlinewidth{0.481800pt}%
\definecolor{currentstroke}{rgb}{1.000000,1.000000,1.000000}%
\pgfsetstrokecolor{currentstroke}%
\pgfsetdash{}{0pt}%
\pgfpathmoveto{\pgfqpoint{2.872669in}{3.175352in}}%
\pgfpathcurveto{\pgfqpoint{2.883719in}{3.175352in}}{\pgfqpoint{2.894318in}{3.179742in}}{\pgfqpoint{2.902132in}{3.187556in}}%
\pgfpathcurveto{\pgfqpoint{2.909945in}{3.195370in}}{\pgfqpoint{2.914336in}{3.205969in}}{\pgfqpoint{2.914336in}{3.217019in}}%
\pgfpathcurveto{\pgfqpoint{2.914336in}{3.228069in}}{\pgfqpoint{2.909945in}{3.238668in}}{\pgfqpoint{2.902132in}{3.246482in}}%
\pgfpathcurveto{\pgfqpoint{2.894318in}{3.254295in}}{\pgfqpoint{2.883719in}{3.258686in}}{\pgfqpoint{2.872669in}{3.258686in}}%
\pgfpathcurveto{\pgfqpoint{2.861619in}{3.258686in}}{\pgfqpoint{2.851020in}{3.254295in}}{\pgfqpoint{2.843206in}{3.246482in}}%
\pgfpathcurveto{\pgfqpoint{2.835392in}{3.238668in}}{\pgfqpoint{2.831002in}{3.228069in}}{\pgfqpoint{2.831002in}{3.217019in}}%
\pgfpathcurveto{\pgfqpoint{2.831002in}{3.205969in}}{\pgfqpoint{2.835392in}{3.195370in}}{\pgfqpoint{2.843206in}{3.187556in}}%
\pgfpathcurveto{\pgfqpoint{2.851020in}{3.179742in}}{\pgfqpoint{2.861619in}{3.175352in}}{\pgfqpoint{2.872669in}{3.175352in}}%
\pgfpathclose%
\pgfusepath{stroke,fill}%
\end{pgfscope}%
\begin{pgfscope}%
\pgfpathrectangle{\pgfqpoint{0.481978in}{0.331635in}}{\pgfqpoint{9.300000in}{7.700000in}}%
\pgfusepath{clip}%
\pgfsetbuttcap%
\pgfsetroundjoin%
\definecolor{currentfill}{rgb}{0.815686,0.733333,1.000000}%
\pgfsetfillcolor{currentfill}%
\pgfsetlinewidth{0.481800pt}%
\definecolor{currentstroke}{rgb}{1.000000,1.000000,1.000000}%
\pgfsetstrokecolor{currentstroke}%
\pgfsetdash{}{0pt}%
\pgfpathmoveto{\pgfqpoint{3.619377in}{2.321220in}}%
\pgfpathcurveto{\pgfqpoint{3.630427in}{2.321220in}}{\pgfqpoint{3.641026in}{2.325610in}}{\pgfqpoint{3.648840in}{2.333423in}}%
\pgfpathcurveto{\pgfqpoint{3.656654in}{2.341237in}}{\pgfqpoint{3.661044in}{2.351836in}}{\pgfqpoint{3.661044in}{2.362886in}}%
\pgfpathcurveto{\pgfqpoint{3.661044in}{2.373936in}}{\pgfqpoint{3.656654in}{2.384535in}}{\pgfqpoint{3.648840in}{2.392349in}}%
\pgfpathcurveto{\pgfqpoint{3.641026in}{2.400163in}}{\pgfqpoint{3.630427in}{2.404553in}}{\pgfqpoint{3.619377in}{2.404553in}}%
\pgfpathcurveto{\pgfqpoint{3.608327in}{2.404553in}}{\pgfqpoint{3.597728in}{2.400163in}}{\pgfqpoint{3.589914in}{2.392349in}}%
\pgfpathcurveto{\pgfqpoint{3.582101in}{2.384535in}}{\pgfqpoint{3.577710in}{2.373936in}}{\pgfqpoint{3.577710in}{2.362886in}}%
\pgfpathcurveto{\pgfqpoint{3.577710in}{2.351836in}}{\pgfqpoint{3.582101in}{2.341237in}}{\pgfqpoint{3.589914in}{2.333423in}}%
\pgfpathcurveto{\pgfqpoint{3.597728in}{2.325610in}}{\pgfqpoint{3.608327in}{2.321220in}}{\pgfqpoint{3.619377in}{2.321220in}}%
\pgfpathclose%
\pgfusepath{stroke,fill}%
\end{pgfscope}%
\begin{pgfscope}%
\pgfpathrectangle{\pgfqpoint{0.481978in}{0.331635in}}{\pgfqpoint{9.300000in}{7.700000in}}%
\pgfusepath{clip}%
\pgfsetbuttcap%
\pgfsetroundjoin%
\definecolor{currentfill}{rgb}{0.815686,0.733333,1.000000}%
\pgfsetfillcolor{currentfill}%
\pgfsetlinewidth{0.481800pt}%
\definecolor{currentstroke}{rgb}{1.000000,1.000000,1.000000}%
\pgfsetstrokecolor{currentstroke}%
\pgfsetdash{}{0pt}%
\pgfpathmoveto{\pgfqpoint{3.448106in}{2.470564in}}%
\pgfpathcurveto{\pgfqpoint{3.459156in}{2.470564in}}{\pgfqpoint{3.469755in}{2.474954in}}{\pgfqpoint{3.477568in}{2.482767in}}%
\pgfpathcurveto{\pgfqpoint{3.485382in}{2.490581in}}{\pgfqpoint{3.489772in}{2.501180in}}{\pgfqpoint{3.489772in}{2.512230in}}%
\pgfpathcurveto{\pgfqpoint{3.489772in}{2.523280in}}{\pgfqpoint{3.485382in}{2.533879in}}{\pgfqpoint{3.477568in}{2.541693in}}%
\pgfpathcurveto{\pgfqpoint{3.469755in}{2.549507in}}{\pgfqpoint{3.459156in}{2.553897in}}{\pgfqpoint{3.448106in}{2.553897in}}%
\pgfpathcurveto{\pgfqpoint{3.437055in}{2.553897in}}{\pgfqpoint{3.426456in}{2.549507in}}{\pgfqpoint{3.418643in}{2.541693in}}%
\pgfpathcurveto{\pgfqpoint{3.410829in}{2.533879in}}{\pgfqpoint{3.406439in}{2.523280in}}{\pgfqpoint{3.406439in}{2.512230in}}%
\pgfpathcurveto{\pgfqpoint{3.406439in}{2.501180in}}{\pgfqpoint{3.410829in}{2.490581in}}{\pgfqpoint{3.418643in}{2.482767in}}%
\pgfpathcurveto{\pgfqpoint{3.426456in}{2.474954in}}{\pgfqpoint{3.437055in}{2.470564in}}{\pgfqpoint{3.448106in}{2.470564in}}%
\pgfpathclose%
\pgfusepath{stroke,fill}%
\end{pgfscope}%
\begin{pgfscope}%
\pgfpathrectangle{\pgfqpoint{0.481978in}{0.331635in}}{\pgfqpoint{9.300000in}{7.700000in}}%
\pgfusepath{clip}%
\pgfsetbuttcap%
\pgfsetroundjoin%
\definecolor{currentfill}{rgb}{0.815686,0.733333,1.000000}%
\pgfsetfillcolor{currentfill}%
\pgfsetlinewidth{0.481800pt}%
\definecolor{currentstroke}{rgb}{1.000000,1.000000,1.000000}%
\pgfsetstrokecolor{currentstroke}%
\pgfsetdash{}{0pt}%
\pgfpathmoveto{\pgfqpoint{6.968468in}{3.304397in}}%
\pgfpathcurveto{\pgfqpoint{6.979518in}{3.304397in}}{\pgfqpoint{6.990117in}{3.308788in}}{\pgfqpoint{6.997931in}{3.316601in}}%
\pgfpathcurveto{\pgfqpoint{7.005744in}{3.324415in}}{\pgfqpoint{7.010135in}{3.335014in}}{\pgfqpoint{7.010135in}{3.346064in}}%
\pgfpathcurveto{\pgfqpoint{7.010135in}{3.357114in}}{\pgfqpoint{7.005744in}{3.367713in}}{\pgfqpoint{6.997931in}{3.375527in}}%
\pgfpathcurveto{\pgfqpoint{6.990117in}{3.383340in}}{\pgfqpoint{6.979518in}{3.387731in}}{\pgfqpoint{6.968468in}{3.387731in}}%
\pgfpathcurveto{\pgfqpoint{6.957418in}{3.387731in}}{\pgfqpoint{6.946819in}{3.383340in}}{\pgfqpoint{6.939005in}{3.375527in}}%
\pgfpathcurveto{\pgfqpoint{6.931192in}{3.367713in}}{\pgfqpoint{6.926801in}{3.357114in}}{\pgfqpoint{6.926801in}{3.346064in}}%
\pgfpathcurveto{\pgfqpoint{6.926801in}{3.335014in}}{\pgfqpoint{6.931192in}{3.324415in}}{\pgfqpoint{6.939005in}{3.316601in}}%
\pgfpathcurveto{\pgfqpoint{6.946819in}{3.308788in}}{\pgfqpoint{6.957418in}{3.304397in}}{\pgfqpoint{6.968468in}{3.304397in}}%
\pgfpathclose%
\pgfusepath{stroke,fill}%
\end{pgfscope}%
\begin{pgfscope}%
\pgfpathrectangle{\pgfqpoint{0.481978in}{0.331635in}}{\pgfqpoint{9.300000in}{7.700000in}}%
\pgfusepath{clip}%
\pgfsetbuttcap%
\pgfsetroundjoin%
\definecolor{currentfill}{rgb}{0.815686,0.733333,1.000000}%
\pgfsetfillcolor{currentfill}%
\pgfsetlinewidth{0.481800pt}%
\definecolor{currentstroke}{rgb}{1.000000,1.000000,1.000000}%
\pgfsetstrokecolor{currentstroke}%
\pgfsetdash{}{0pt}%
\pgfpathmoveto{\pgfqpoint{3.260742in}{6.989602in}}%
\pgfpathcurveto{\pgfqpoint{3.271792in}{6.989602in}}{\pgfqpoint{3.282391in}{6.993992in}}{\pgfqpoint{3.290204in}{7.001806in}}%
\pgfpathcurveto{\pgfqpoint{3.298018in}{7.009620in}}{\pgfqpoint{3.302408in}{7.020219in}}{\pgfqpoint{3.302408in}{7.031269in}}%
\pgfpathcurveto{\pgfqpoint{3.302408in}{7.042319in}}{\pgfqpoint{3.298018in}{7.052918in}}{\pgfqpoint{3.290204in}{7.060732in}}%
\pgfpathcurveto{\pgfqpoint{3.282391in}{7.068545in}}{\pgfqpoint{3.271792in}{7.072935in}}{\pgfqpoint{3.260742in}{7.072935in}}%
\pgfpathcurveto{\pgfqpoint{3.249692in}{7.072935in}}{\pgfqpoint{3.239092in}{7.068545in}}{\pgfqpoint{3.231279in}{7.060732in}}%
\pgfpathcurveto{\pgfqpoint{3.223465in}{7.052918in}}{\pgfqpoint{3.219075in}{7.042319in}}{\pgfqpoint{3.219075in}{7.031269in}}%
\pgfpathcurveto{\pgfqpoint{3.219075in}{7.020219in}}{\pgfqpoint{3.223465in}{7.009620in}}{\pgfqpoint{3.231279in}{7.001806in}}%
\pgfpathcurveto{\pgfqpoint{3.239092in}{6.993992in}}{\pgfqpoint{3.249692in}{6.989602in}}{\pgfqpoint{3.260742in}{6.989602in}}%
\pgfpathclose%
\pgfusepath{stroke,fill}%
\end{pgfscope}%
\begin{pgfscope}%
\pgfpathrectangle{\pgfqpoint{0.481978in}{0.331635in}}{\pgfqpoint{9.300000in}{7.700000in}}%
\pgfusepath{clip}%
\pgfsetbuttcap%
\pgfsetroundjoin%
\definecolor{currentfill}{rgb}{0.815686,0.733333,1.000000}%
\pgfsetfillcolor{currentfill}%
\pgfsetlinewidth{0.481800pt}%
\definecolor{currentstroke}{rgb}{1.000000,1.000000,1.000000}%
\pgfsetstrokecolor{currentstroke}%
\pgfsetdash{}{0pt}%
\pgfpathmoveto{\pgfqpoint{3.172404in}{5.733347in}}%
\pgfpathcurveto{\pgfqpoint{3.183454in}{5.733347in}}{\pgfqpoint{3.194053in}{5.737738in}}{\pgfqpoint{3.201867in}{5.745551in}}%
\pgfpathcurveto{\pgfqpoint{3.209680in}{5.753365in}}{\pgfqpoint{3.214071in}{5.763964in}}{\pgfqpoint{3.214071in}{5.775014in}}%
\pgfpathcurveto{\pgfqpoint{3.214071in}{5.786064in}}{\pgfqpoint{3.209680in}{5.796663in}}{\pgfqpoint{3.201867in}{5.804477in}}%
\pgfpathcurveto{\pgfqpoint{3.194053in}{5.812290in}}{\pgfqpoint{3.183454in}{5.816681in}}{\pgfqpoint{3.172404in}{5.816681in}}%
\pgfpathcurveto{\pgfqpoint{3.161354in}{5.816681in}}{\pgfqpoint{3.150755in}{5.812290in}}{\pgfqpoint{3.142941in}{5.804477in}}%
\pgfpathcurveto{\pgfqpoint{3.135127in}{5.796663in}}{\pgfqpoint{3.130737in}{5.786064in}}{\pgfqpoint{3.130737in}{5.775014in}}%
\pgfpathcurveto{\pgfqpoint{3.130737in}{5.763964in}}{\pgfqpoint{3.135127in}{5.753365in}}{\pgfqpoint{3.142941in}{5.745551in}}%
\pgfpathcurveto{\pgfqpoint{3.150755in}{5.737738in}}{\pgfqpoint{3.161354in}{5.733347in}}{\pgfqpoint{3.172404in}{5.733347in}}%
\pgfpathclose%
\pgfusepath{stroke,fill}%
\end{pgfscope}%
\begin{pgfscope}%
\pgfpathrectangle{\pgfqpoint{0.481978in}{0.331635in}}{\pgfqpoint{9.300000in}{7.700000in}}%
\pgfusepath{clip}%
\pgfsetbuttcap%
\pgfsetroundjoin%
\definecolor{currentfill}{rgb}{0.815686,0.733333,1.000000}%
\pgfsetfillcolor{currentfill}%
\pgfsetlinewidth{0.481800pt}%
\definecolor{currentstroke}{rgb}{1.000000,1.000000,1.000000}%
\pgfsetstrokecolor{currentstroke}%
\pgfsetdash{}{0pt}%
\pgfpathmoveto{\pgfqpoint{2.953247in}{2.659230in}}%
\pgfpathcurveto{\pgfqpoint{2.964297in}{2.659230in}}{\pgfqpoint{2.974896in}{2.663621in}}{\pgfqpoint{2.982710in}{2.671434in}}%
\pgfpathcurveto{\pgfqpoint{2.990523in}{2.679248in}}{\pgfqpoint{2.994914in}{2.689847in}}{\pgfqpoint{2.994914in}{2.700897in}}%
\pgfpathcurveto{\pgfqpoint{2.994914in}{2.711947in}}{\pgfqpoint{2.990523in}{2.722546in}}{\pgfqpoint{2.982710in}{2.730360in}}%
\pgfpathcurveto{\pgfqpoint{2.974896in}{2.738173in}}{\pgfqpoint{2.964297in}{2.742564in}}{\pgfqpoint{2.953247in}{2.742564in}}%
\pgfpathcurveto{\pgfqpoint{2.942197in}{2.742564in}}{\pgfqpoint{2.931598in}{2.738173in}}{\pgfqpoint{2.923784in}{2.730360in}}%
\pgfpathcurveto{\pgfqpoint{2.915970in}{2.722546in}}{\pgfqpoint{2.911580in}{2.711947in}}{\pgfqpoint{2.911580in}{2.700897in}}%
\pgfpathcurveto{\pgfqpoint{2.911580in}{2.689847in}}{\pgfqpoint{2.915970in}{2.679248in}}{\pgfqpoint{2.923784in}{2.671434in}}%
\pgfpathcurveto{\pgfqpoint{2.931598in}{2.663621in}}{\pgfqpoint{2.942197in}{2.659230in}}{\pgfqpoint{2.953247in}{2.659230in}}%
\pgfpathclose%
\pgfusepath{stroke,fill}%
\end{pgfscope}%
\begin{pgfscope}%
\pgfpathrectangle{\pgfqpoint{0.481978in}{0.331635in}}{\pgfqpoint{9.300000in}{7.700000in}}%
\pgfusepath{clip}%
\pgfsetbuttcap%
\pgfsetroundjoin%
\definecolor{currentfill}{rgb}{0.815686,0.733333,1.000000}%
\pgfsetfillcolor{currentfill}%
\pgfsetlinewidth{0.481800pt}%
\definecolor{currentstroke}{rgb}{1.000000,1.000000,1.000000}%
\pgfsetstrokecolor{currentstroke}%
\pgfsetdash{}{0pt}%
\pgfpathmoveto{\pgfqpoint{8.116420in}{5.574617in}}%
\pgfpathcurveto{\pgfqpoint{8.127470in}{5.574617in}}{\pgfqpoint{8.138069in}{5.579007in}}{\pgfqpoint{8.145883in}{5.586821in}}%
\pgfpathcurveto{\pgfqpoint{8.153697in}{5.594635in}}{\pgfqpoint{8.158087in}{5.605234in}}{\pgfqpoint{8.158087in}{5.616284in}}%
\pgfpathcurveto{\pgfqpoint{8.158087in}{5.627334in}}{\pgfqpoint{8.153697in}{5.637933in}}{\pgfqpoint{8.145883in}{5.645747in}}%
\pgfpathcurveto{\pgfqpoint{8.138069in}{5.653560in}}{\pgfqpoint{8.127470in}{5.657950in}}{\pgfqpoint{8.116420in}{5.657950in}}%
\pgfpathcurveto{\pgfqpoint{8.105370in}{5.657950in}}{\pgfqpoint{8.094771in}{5.653560in}}{\pgfqpoint{8.086957in}{5.645747in}}%
\pgfpathcurveto{\pgfqpoint{8.079144in}{5.637933in}}{\pgfqpoint{8.074754in}{5.627334in}}{\pgfqpoint{8.074754in}{5.616284in}}%
\pgfpathcurveto{\pgfqpoint{8.074754in}{5.605234in}}{\pgfqpoint{8.079144in}{5.594635in}}{\pgfqpoint{8.086957in}{5.586821in}}%
\pgfpathcurveto{\pgfqpoint{8.094771in}{5.579007in}}{\pgfqpoint{8.105370in}{5.574617in}}{\pgfqpoint{8.116420in}{5.574617in}}%
\pgfpathclose%
\pgfusepath{stroke,fill}%
\end{pgfscope}%
\begin{pgfscope}%
\pgfpathrectangle{\pgfqpoint{0.481978in}{0.331635in}}{\pgfqpoint{9.300000in}{7.700000in}}%
\pgfusepath{clip}%
\pgfsetbuttcap%
\pgfsetroundjoin%
\definecolor{currentfill}{rgb}{0.815686,0.733333,1.000000}%
\pgfsetfillcolor{currentfill}%
\pgfsetlinewidth{0.481800pt}%
\definecolor{currentstroke}{rgb}{1.000000,1.000000,1.000000}%
\pgfsetstrokecolor{currentstroke}%
\pgfsetdash{}{0pt}%
\pgfpathmoveto{\pgfqpoint{6.906230in}{4.688829in}}%
\pgfpathcurveto{\pgfqpoint{6.917280in}{4.688829in}}{\pgfqpoint{6.927879in}{4.693220in}}{\pgfqpoint{6.935692in}{4.701033in}}%
\pgfpathcurveto{\pgfqpoint{6.943506in}{4.708847in}}{\pgfqpoint{6.947896in}{4.719446in}}{\pgfqpoint{6.947896in}{4.730496in}}%
\pgfpathcurveto{\pgfqpoint{6.947896in}{4.741546in}}{\pgfqpoint{6.943506in}{4.752145in}}{\pgfqpoint{6.935692in}{4.759959in}}%
\pgfpathcurveto{\pgfqpoint{6.927879in}{4.767772in}}{\pgfqpoint{6.917280in}{4.772163in}}{\pgfqpoint{6.906230in}{4.772163in}}%
\pgfpathcurveto{\pgfqpoint{6.895179in}{4.772163in}}{\pgfqpoint{6.884580in}{4.767772in}}{\pgfqpoint{6.876767in}{4.759959in}}%
\pgfpathcurveto{\pgfqpoint{6.868953in}{4.752145in}}{\pgfqpoint{6.864563in}{4.741546in}}{\pgfqpoint{6.864563in}{4.730496in}}%
\pgfpathcurveto{\pgfqpoint{6.864563in}{4.719446in}}{\pgfqpoint{6.868953in}{4.708847in}}{\pgfqpoint{6.876767in}{4.701033in}}%
\pgfpathcurveto{\pgfqpoint{6.884580in}{4.693220in}}{\pgfqpoint{6.895179in}{4.688829in}}{\pgfqpoint{6.906230in}{4.688829in}}%
\pgfpathclose%
\pgfusepath{stroke,fill}%
\end{pgfscope}%
\begin{pgfscope}%
\pgfpathrectangle{\pgfqpoint{0.481978in}{0.331635in}}{\pgfqpoint{9.300000in}{7.700000in}}%
\pgfusepath{clip}%
\pgfsetbuttcap%
\pgfsetroundjoin%
\definecolor{currentfill}{rgb}{0.815686,0.733333,1.000000}%
\pgfsetfillcolor{currentfill}%
\pgfsetlinewidth{0.481800pt}%
\definecolor{currentstroke}{rgb}{1.000000,1.000000,1.000000}%
\pgfsetstrokecolor{currentstroke}%
\pgfsetdash{}{0pt}%
\pgfpathmoveto{\pgfqpoint{3.815826in}{4.045962in}}%
\pgfpathcurveto{\pgfqpoint{3.826876in}{4.045962in}}{\pgfqpoint{3.837475in}{4.050352in}}{\pgfqpoint{3.845289in}{4.058166in}}%
\pgfpathcurveto{\pgfqpoint{3.853103in}{4.065979in}}{\pgfqpoint{3.857493in}{4.076578in}}{\pgfqpoint{3.857493in}{4.087628in}}%
\pgfpathcurveto{\pgfqpoint{3.857493in}{4.098678in}}{\pgfqpoint{3.853103in}{4.109278in}}{\pgfqpoint{3.845289in}{4.117091in}}%
\pgfpathcurveto{\pgfqpoint{3.837475in}{4.124905in}}{\pgfqpoint{3.826876in}{4.129295in}}{\pgfqpoint{3.815826in}{4.129295in}}%
\pgfpathcurveto{\pgfqpoint{3.804776in}{4.129295in}}{\pgfqpoint{3.794177in}{4.124905in}}{\pgfqpoint{3.786364in}{4.117091in}}%
\pgfpathcurveto{\pgfqpoint{3.778550in}{4.109278in}}{\pgfqpoint{3.774160in}{4.098678in}}{\pgfqpoint{3.774160in}{4.087628in}}%
\pgfpathcurveto{\pgfqpoint{3.774160in}{4.076578in}}{\pgfqpoint{3.778550in}{4.065979in}}{\pgfqpoint{3.786364in}{4.058166in}}%
\pgfpathcurveto{\pgfqpoint{3.794177in}{4.050352in}}{\pgfqpoint{3.804776in}{4.045962in}}{\pgfqpoint{3.815826in}{4.045962in}}%
\pgfpathclose%
\pgfusepath{stroke,fill}%
\end{pgfscope}%
\begin{pgfscope}%
\pgfpathrectangle{\pgfqpoint{0.481978in}{0.331635in}}{\pgfqpoint{9.300000in}{7.700000in}}%
\pgfusepath{clip}%
\pgfsetbuttcap%
\pgfsetroundjoin%
\definecolor{currentfill}{rgb}{0.815686,0.733333,1.000000}%
\pgfsetfillcolor{currentfill}%
\pgfsetlinewidth{0.481800pt}%
\definecolor{currentstroke}{rgb}{1.000000,1.000000,1.000000}%
\pgfsetstrokecolor{currentstroke}%
\pgfsetdash{}{0pt}%
\pgfpathmoveto{\pgfqpoint{6.962413in}{5.760379in}}%
\pgfpathcurveto{\pgfqpoint{6.973463in}{5.760379in}}{\pgfqpoint{6.984062in}{5.764769in}}{\pgfqpoint{6.991875in}{5.772582in}}%
\pgfpathcurveto{\pgfqpoint{6.999689in}{5.780396in}}{\pgfqpoint{7.004079in}{5.790995in}}{\pgfqpoint{7.004079in}{5.802045in}}%
\pgfpathcurveto{\pgfqpoint{7.004079in}{5.813095in}}{\pgfqpoint{6.999689in}{5.823694in}}{\pgfqpoint{6.991875in}{5.831508in}}%
\pgfpathcurveto{\pgfqpoint{6.984062in}{5.839322in}}{\pgfqpoint{6.973463in}{5.843712in}}{\pgfqpoint{6.962413in}{5.843712in}}%
\pgfpathcurveto{\pgfqpoint{6.951362in}{5.843712in}}{\pgfqpoint{6.940763in}{5.839322in}}{\pgfqpoint{6.932950in}{5.831508in}}%
\pgfpathcurveto{\pgfqpoint{6.925136in}{5.823694in}}{\pgfqpoint{6.920746in}{5.813095in}}{\pgfqpoint{6.920746in}{5.802045in}}%
\pgfpathcurveto{\pgfqpoint{6.920746in}{5.790995in}}{\pgfqpoint{6.925136in}{5.780396in}}{\pgfqpoint{6.932950in}{5.772582in}}%
\pgfpathcurveto{\pgfqpoint{6.940763in}{5.764769in}}{\pgfqpoint{6.951362in}{5.760379in}}{\pgfqpoint{6.962413in}{5.760379in}}%
\pgfpathclose%
\pgfusepath{stroke,fill}%
\end{pgfscope}%
\begin{pgfscope}%
\pgfpathrectangle{\pgfqpoint{0.481978in}{0.331635in}}{\pgfqpoint{9.300000in}{7.700000in}}%
\pgfusepath{clip}%
\pgfsetbuttcap%
\pgfsetroundjoin%
\definecolor{currentfill}{rgb}{0.815686,0.733333,1.000000}%
\pgfsetfillcolor{currentfill}%
\pgfsetlinewidth{0.481800pt}%
\definecolor{currentstroke}{rgb}{1.000000,1.000000,1.000000}%
\pgfsetstrokecolor{currentstroke}%
\pgfsetdash{}{0pt}%
\pgfpathmoveto{\pgfqpoint{7.236127in}{5.351377in}}%
\pgfpathcurveto{\pgfqpoint{7.247177in}{5.351377in}}{\pgfqpoint{7.257776in}{5.355768in}}{\pgfqpoint{7.265589in}{5.363581in}}%
\pgfpathcurveto{\pgfqpoint{7.273403in}{5.371395in}}{\pgfqpoint{7.277793in}{5.381994in}}{\pgfqpoint{7.277793in}{5.393044in}}%
\pgfpathcurveto{\pgfqpoint{7.277793in}{5.404094in}}{\pgfqpoint{7.273403in}{5.414693in}}{\pgfqpoint{7.265589in}{5.422507in}}%
\pgfpathcurveto{\pgfqpoint{7.257776in}{5.430321in}}{\pgfqpoint{7.247177in}{5.434711in}}{\pgfqpoint{7.236127in}{5.434711in}}%
\pgfpathcurveto{\pgfqpoint{7.225076in}{5.434711in}}{\pgfqpoint{7.214477in}{5.430321in}}{\pgfqpoint{7.206664in}{5.422507in}}%
\pgfpathcurveto{\pgfqpoint{7.198850in}{5.414693in}}{\pgfqpoint{7.194460in}{5.404094in}}{\pgfqpoint{7.194460in}{5.393044in}}%
\pgfpathcurveto{\pgfqpoint{7.194460in}{5.381994in}}{\pgfqpoint{7.198850in}{5.371395in}}{\pgfqpoint{7.206664in}{5.363581in}}%
\pgfpathcurveto{\pgfqpoint{7.214477in}{5.355768in}}{\pgfqpoint{7.225076in}{5.351377in}}{\pgfqpoint{7.236127in}{5.351377in}}%
\pgfpathclose%
\pgfusepath{stroke,fill}%
\end{pgfscope}%
\begin{pgfscope}%
\pgfpathrectangle{\pgfqpoint{0.481978in}{0.331635in}}{\pgfqpoint{9.300000in}{7.700000in}}%
\pgfusepath{clip}%
\pgfsetbuttcap%
\pgfsetroundjoin%
\definecolor{currentfill}{rgb}{0.815686,0.733333,1.000000}%
\pgfsetfillcolor{currentfill}%
\pgfsetlinewidth{0.481800pt}%
\definecolor{currentstroke}{rgb}{1.000000,1.000000,1.000000}%
\pgfsetstrokecolor{currentstroke}%
\pgfsetdash{}{0pt}%
\pgfpathmoveto{\pgfqpoint{8.116846in}{6.482719in}}%
\pgfpathcurveto{\pgfqpoint{8.127896in}{6.482719in}}{\pgfqpoint{8.138495in}{6.487109in}}{\pgfqpoint{8.146309in}{6.494923in}}%
\pgfpathcurveto{\pgfqpoint{8.154122in}{6.502737in}}{\pgfqpoint{8.158512in}{6.513336in}}{\pgfqpoint{8.158512in}{6.524386in}}%
\pgfpathcurveto{\pgfqpoint{8.158512in}{6.535436in}}{\pgfqpoint{8.154122in}{6.546035in}}{\pgfqpoint{8.146309in}{6.553849in}}%
\pgfpathcurveto{\pgfqpoint{8.138495in}{6.561662in}}{\pgfqpoint{8.127896in}{6.566052in}}{\pgfqpoint{8.116846in}{6.566052in}}%
\pgfpathcurveto{\pgfqpoint{8.105796in}{6.566052in}}{\pgfqpoint{8.095197in}{6.561662in}}{\pgfqpoint{8.087383in}{6.553849in}}%
\pgfpathcurveto{\pgfqpoint{8.079569in}{6.546035in}}{\pgfqpoint{8.075179in}{6.535436in}}{\pgfqpoint{8.075179in}{6.524386in}}%
\pgfpathcurveto{\pgfqpoint{8.075179in}{6.513336in}}{\pgfqpoint{8.079569in}{6.502737in}}{\pgfqpoint{8.087383in}{6.494923in}}%
\pgfpathcurveto{\pgfqpoint{8.095197in}{6.487109in}}{\pgfqpoint{8.105796in}{6.482719in}}{\pgfqpoint{8.116846in}{6.482719in}}%
\pgfpathclose%
\pgfusepath{stroke,fill}%
\end{pgfscope}%
\begin{pgfscope}%
\pgfpathrectangle{\pgfqpoint{0.481978in}{0.331635in}}{\pgfqpoint{9.300000in}{7.700000in}}%
\pgfusepath{clip}%
\pgfsetbuttcap%
\pgfsetroundjoin%
\definecolor{currentfill}{rgb}{0.815686,0.733333,1.000000}%
\pgfsetfillcolor{currentfill}%
\pgfsetlinewidth{0.481800pt}%
\definecolor{currentstroke}{rgb}{1.000000,1.000000,1.000000}%
\pgfsetstrokecolor{currentstroke}%
\pgfsetdash{}{0pt}%
\pgfpathmoveto{\pgfqpoint{3.449504in}{2.725034in}}%
\pgfpathcurveto{\pgfqpoint{3.460554in}{2.725034in}}{\pgfqpoint{3.471153in}{2.729424in}}{\pgfqpoint{3.478967in}{2.737238in}}%
\pgfpathcurveto{\pgfqpoint{3.486780in}{2.745051in}}{\pgfqpoint{3.491171in}{2.755651in}}{\pgfqpoint{3.491171in}{2.766701in}}%
\pgfpathcurveto{\pgfqpoint{3.491171in}{2.777751in}}{\pgfqpoint{3.486780in}{2.788350in}}{\pgfqpoint{3.478967in}{2.796163in}}%
\pgfpathcurveto{\pgfqpoint{3.471153in}{2.803977in}}{\pgfqpoint{3.460554in}{2.808367in}}{\pgfqpoint{3.449504in}{2.808367in}}%
\pgfpathcurveto{\pgfqpoint{3.438454in}{2.808367in}}{\pgfqpoint{3.427855in}{2.803977in}}{\pgfqpoint{3.420041in}{2.796163in}}%
\pgfpathcurveto{\pgfqpoint{3.412228in}{2.788350in}}{\pgfqpoint{3.407837in}{2.777751in}}{\pgfqpoint{3.407837in}{2.766701in}}%
\pgfpathcurveto{\pgfqpoint{3.407837in}{2.755651in}}{\pgfqpoint{3.412228in}{2.745051in}}{\pgfqpoint{3.420041in}{2.737238in}}%
\pgfpathcurveto{\pgfqpoint{3.427855in}{2.729424in}}{\pgfqpoint{3.438454in}{2.725034in}}{\pgfqpoint{3.449504in}{2.725034in}}%
\pgfpathclose%
\pgfusepath{stroke,fill}%
\end{pgfscope}%
\begin{pgfscope}%
\pgfpathrectangle{\pgfqpoint{0.481978in}{0.331635in}}{\pgfqpoint{9.300000in}{7.700000in}}%
\pgfusepath{clip}%
\pgfsetbuttcap%
\pgfsetroundjoin%
\definecolor{currentfill}{rgb}{0.815686,0.733333,1.000000}%
\pgfsetfillcolor{currentfill}%
\pgfsetlinewidth{0.481800pt}%
\definecolor{currentstroke}{rgb}{1.000000,1.000000,1.000000}%
\pgfsetstrokecolor{currentstroke}%
\pgfsetdash{}{0pt}%
\pgfpathmoveto{\pgfqpoint{4.947086in}{5.379971in}}%
\pgfpathcurveto{\pgfqpoint{4.958136in}{5.379971in}}{\pgfqpoint{4.968735in}{5.384361in}}{\pgfqpoint{4.976549in}{5.392175in}}%
\pgfpathcurveto{\pgfqpoint{4.984363in}{5.399988in}}{\pgfqpoint{4.988753in}{5.410587in}}{\pgfqpoint{4.988753in}{5.421637in}}%
\pgfpathcurveto{\pgfqpoint{4.988753in}{5.432688in}}{\pgfqpoint{4.984363in}{5.443287in}}{\pgfqpoint{4.976549in}{5.451100in}}%
\pgfpathcurveto{\pgfqpoint{4.968735in}{5.458914in}}{\pgfqpoint{4.958136in}{5.463304in}}{\pgfqpoint{4.947086in}{5.463304in}}%
\pgfpathcurveto{\pgfqpoint{4.936036in}{5.463304in}}{\pgfqpoint{4.925437in}{5.458914in}}{\pgfqpoint{4.917623in}{5.451100in}}%
\pgfpathcurveto{\pgfqpoint{4.909810in}{5.443287in}}{\pgfqpoint{4.905420in}{5.432688in}}{\pgfqpoint{4.905420in}{5.421637in}}%
\pgfpathcurveto{\pgfqpoint{4.905420in}{5.410587in}}{\pgfqpoint{4.909810in}{5.399988in}}{\pgfqpoint{4.917623in}{5.392175in}}%
\pgfpathcurveto{\pgfqpoint{4.925437in}{5.384361in}}{\pgfqpoint{4.936036in}{5.379971in}}{\pgfqpoint{4.947086in}{5.379971in}}%
\pgfpathclose%
\pgfusepath{stroke,fill}%
\end{pgfscope}%
\begin{pgfscope}%
\pgfpathrectangle{\pgfqpoint{0.481978in}{0.331635in}}{\pgfqpoint{9.300000in}{7.700000in}}%
\pgfusepath{clip}%
\pgfsetbuttcap%
\pgfsetroundjoin%
\definecolor{currentfill}{rgb}{0.815686,0.733333,1.000000}%
\pgfsetfillcolor{currentfill}%
\pgfsetlinewidth{0.481800pt}%
\definecolor{currentstroke}{rgb}{1.000000,1.000000,1.000000}%
\pgfsetstrokecolor{currentstroke}%
\pgfsetdash{}{0pt}%
\pgfpathmoveto{\pgfqpoint{6.925838in}{2.873806in}}%
\pgfpathcurveto{\pgfqpoint{6.936889in}{2.873806in}}{\pgfqpoint{6.947488in}{2.878197in}}{\pgfqpoint{6.955301in}{2.886010in}}%
\pgfpathcurveto{\pgfqpoint{6.963115in}{2.893824in}}{\pgfqpoint{6.967505in}{2.904423in}}{\pgfqpoint{6.967505in}{2.915473in}}%
\pgfpathcurveto{\pgfqpoint{6.967505in}{2.926523in}}{\pgfqpoint{6.963115in}{2.937122in}}{\pgfqpoint{6.955301in}{2.944936in}}%
\pgfpathcurveto{\pgfqpoint{6.947488in}{2.952749in}}{\pgfqpoint{6.936889in}{2.957140in}}{\pgfqpoint{6.925838in}{2.957140in}}%
\pgfpathcurveto{\pgfqpoint{6.914788in}{2.957140in}}{\pgfqpoint{6.904189in}{2.952749in}}{\pgfqpoint{6.896376in}{2.944936in}}%
\pgfpathcurveto{\pgfqpoint{6.888562in}{2.937122in}}{\pgfqpoint{6.884172in}{2.926523in}}{\pgfqpoint{6.884172in}{2.915473in}}%
\pgfpathcurveto{\pgfqpoint{6.884172in}{2.904423in}}{\pgfqpoint{6.888562in}{2.893824in}}{\pgfqpoint{6.896376in}{2.886010in}}%
\pgfpathcurveto{\pgfqpoint{6.904189in}{2.878197in}}{\pgfqpoint{6.914788in}{2.873806in}}{\pgfqpoint{6.925838in}{2.873806in}}%
\pgfpathclose%
\pgfusepath{stroke,fill}%
\end{pgfscope}%
\begin{pgfscope}%
\pgfpathrectangle{\pgfqpoint{0.481978in}{0.331635in}}{\pgfqpoint{9.300000in}{7.700000in}}%
\pgfusepath{clip}%
\pgfsetbuttcap%
\pgfsetroundjoin%
\definecolor{currentfill}{rgb}{0.815686,0.733333,1.000000}%
\pgfsetfillcolor{currentfill}%
\pgfsetlinewidth{0.481800pt}%
\definecolor{currentstroke}{rgb}{1.000000,1.000000,1.000000}%
\pgfsetstrokecolor{currentstroke}%
\pgfsetdash{}{0pt}%
\pgfpathmoveto{\pgfqpoint{7.338785in}{3.293368in}}%
\pgfpathcurveto{\pgfqpoint{7.349835in}{3.293368in}}{\pgfqpoint{7.360434in}{3.297758in}}{\pgfqpoint{7.368247in}{3.305572in}}%
\pgfpathcurveto{\pgfqpoint{7.376061in}{3.313385in}}{\pgfqpoint{7.380451in}{3.323984in}}{\pgfqpoint{7.380451in}{3.335035in}}%
\pgfpathcurveto{\pgfqpoint{7.380451in}{3.346085in}}{\pgfqpoint{7.376061in}{3.356684in}}{\pgfqpoint{7.368247in}{3.364497in}}%
\pgfpathcurveto{\pgfqpoint{7.360434in}{3.372311in}}{\pgfqpoint{7.349835in}{3.376701in}}{\pgfqpoint{7.338785in}{3.376701in}}%
\pgfpathcurveto{\pgfqpoint{7.327734in}{3.376701in}}{\pgfqpoint{7.317135in}{3.372311in}}{\pgfqpoint{7.309322in}{3.364497in}}%
\pgfpathcurveto{\pgfqpoint{7.301508in}{3.356684in}}{\pgfqpoint{7.297118in}{3.346085in}}{\pgfqpoint{7.297118in}{3.335035in}}%
\pgfpathcurveto{\pgfqpoint{7.297118in}{3.323984in}}{\pgfqpoint{7.301508in}{3.313385in}}{\pgfqpoint{7.309322in}{3.305572in}}%
\pgfpathcurveto{\pgfqpoint{7.317135in}{3.297758in}}{\pgfqpoint{7.327734in}{3.293368in}}{\pgfqpoint{7.338785in}{3.293368in}}%
\pgfpathclose%
\pgfusepath{stroke,fill}%
\end{pgfscope}%
\begin{pgfscope}%
\pgfpathrectangle{\pgfqpoint{0.481978in}{0.331635in}}{\pgfqpoint{9.300000in}{7.700000in}}%
\pgfusepath{clip}%
\pgfsetbuttcap%
\pgfsetroundjoin%
\definecolor{currentfill}{rgb}{0.815686,0.733333,1.000000}%
\pgfsetfillcolor{currentfill}%
\pgfsetlinewidth{0.481800pt}%
\definecolor{currentstroke}{rgb}{1.000000,1.000000,1.000000}%
\pgfsetstrokecolor{currentstroke}%
\pgfsetdash{}{0pt}%
\pgfpathmoveto{\pgfqpoint{3.955717in}{4.093278in}}%
\pgfpathcurveto{\pgfqpoint{3.966767in}{4.093278in}}{\pgfqpoint{3.977366in}{4.097669in}}{\pgfqpoint{3.985180in}{4.105482in}}%
\pgfpathcurveto{\pgfqpoint{3.992994in}{4.113296in}}{\pgfqpoint{3.997384in}{4.123895in}}{\pgfqpoint{3.997384in}{4.134945in}}%
\pgfpathcurveto{\pgfqpoint{3.997384in}{4.145995in}}{\pgfqpoint{3.992994in}{4.156594in}}{\pgfqpoint{3.985180in}{4.164408in}}%
\pgfpathcurveto{\pgfqpoint{3.977366in}{4.172221in}}{\pgfqpoint{3.966767in}{4.176612in}}{\pgfqpoint{3.955717in}{4.176612in}}%
\pgfpathcurveto{\pgfqpoint{3.944667in}{4.176612in}}{\pgfqpoint{3.934068in}{4.172221in}}{\pgfqpoint{3.926254in}{4.164408in}}%
\pgfpathcurveto{\pgfqpoint{3.918441in}{4.156594in}}{\pgfqpoint{3.914050in}{4.145995in}}{\pgfqpoint{3.914050in}{4.134945in}}%
\pgfpathcurveto{\pgfqpoint{3.914050in}{4.123895in}}{\pgfqpoint{3.918441in}{4.113296in}}{\pgfqpoint{3.926254in}{4.105482in}}%
\pgfpathcurveto{\pgfqpoint{3.934068in}{4.097669in}}{\pgfqpoint{3.944667in}{4.093278in}}{\pgfqpoint{3.955717in}{4.093278in}}%
\pgfpathclose%
\pgfusepath{stroke,fill}%
\end{pgfscope}%
\begin{pgfscope}%
\pgfpathrectangle{\pgfqpoint{0.481978in}{0.331635in}}{\pgfqpoint{9.300000in}{7.700000in}}%
\pgfusepath{clip}%
\pgfsetbuttcap%
\pgfsetroundjoin%
\definecolor{currentfill}{rgb}{0.815686,0.733333,1.000000}%
\pgfsetfillcolor{currentfill}%
\pgfsetlinewidth{0.481800pt}%
\definecolor{currentstroke}{rgb}{1.000000,1.000000,1.000000}%
\pgfsetstrokecolor{currentstroke}%
\pgfsetdash{}{0pt}%
\pgfpathmoveto{\pgfqpoint{2.160067in}{1.882672in}}%
\pgfpathcurveto{\pgfqpoint{2.171117in}{1.882672in}}{\pgfqpoint{2.181717in}{1.887062in}}{\pgfqpoint{2.189530in}{1.894876in}}%
\pgfpathcurveto{\pgfqpoint{2.197344in}{1.902689in}}{\pgfqpoint{2.201734in}{1.913288in}}{\pgfqpoint{2.201734in}{1.924338in}}%
\pgfpathcurveto{\pgfqpoint{2.201734in}{1.935389in}}{\pgfqpoint{2.197344in}{1.945988in}}{\pgfqpoint{2.189530in}{1.953801in}}%
\pgfpathcurveto{\pgfqpoint{2.181717in}{1.961615in}}{\pgfqpoint{2.171117in}{1.966005in}}{\pgfqpoint{2.160067in}{1.966005in}}%
\pgfpathcurveto{\pgfqpoint{2.149017in}{1.966005in}}{\pgfqpoint{2.138418in}{1.961615in}}{\pgfqpoint{2.130605in}{1.953801in}}%
\pgfpathcurveto{\pgfqpoint{2.122791in}{1.945988in}}{\pgfqpoint{2.118401in}{1.935389in}}{\pgfqpoint{2.118401in}{1.924338in}}%
\pgfpathcurveto{\pgfqpoint{2.118401in}{1.913288in}}{\pgfqpoint{2.122791in}{1.902689in}}{\pgfqpoint{2.130605in}{1.894876in}}%
\pgfpathcurveto{\pgfqpoint{2.138418in}{1.887062in}}{\pgfqpoint{2.149017in}{1.882672in}}{\pgfqpoint{2.160067in}{1.882672in}}%
\pgfpathclose%
\pgfusepath{stroke,fill}%
\end{pgfscope}%
\begin{pgfscope}%
\pgfpathrectangle{\pgfqpoint{0.481978in}{0.331635in}}{\pgfqpoint{9.300000in}{7.700000in}}%
\pgfusepath{clip}%
\pgfsetbuttcap%
\pgfsetroundjoin%
\definecolor{currentfill}{rgb}{0.815686,0.733333,1.000000}%
\pgfsetfillcolor{currentfill}%
\pgfsetlinewidth{0.481800pt}%
\definecolor{currentstroke}{rgb}{1.000000,1.000000,1.000000}%
\pgfsetstrokecolor{currentstroke}%
\pgfsetdash{}{0pt}%
\pgfpathmoveto{\pgfqpoint{4.198073in}{4.221155in}}%
\pgfpathcurveto{\pgfqpoint{4.209123in}{4.221155in}}{\pgfqpoint{4.219722in}{4.225546in}}{\pgfqpoint{4.227535in}{4.233359in}}%
\pgfpathcurveto{\pgfqpoint{4.235349in}{4.241173in}}{\pgfqpoint{4.239739in}{4.251772in}}{\pgfqpoint{4.239739in}{4.262822in}}%
\pgfpathcurveto{\pgfqpoint{4.239739in}{4.273872in}}{\pgfqpoint{4.235349in}{4.284471in}}{\pgfqpoint{4.227535in}{4.292285in}}%
\pgfpathcurveto{\pgfqpoint{4.219722in}{4.300098in}}{\pgfqpoint{4.209123in}{4.304489in}}{\pgfqpoint{4.198073in}{4.304489in}}%
\pgfpathcurveto{\pgfqpoint{4.187023in}{4.304489in}}{\pgfqpoint{4.176424in}{4.300098in}}{\pgfqpoint{4.168610in}{4.292285in}}%
\pgfpathcurveto{\pgfqpoint{4.160796in}{4.284471in}}{\pgfqpoint{4.156406in}{4.273872in}}{\pgfqpoint{4.156406in}{4.262822in}}%
\pgfpathcurveto{\pgfqpoint{4.156406in}{4.251772in}}{\pgfqpoint{4.160796in}{4.241173in}}{\pgfqpoint{4.168610in}{4.233359in}}%
\pgfpathcurveto{\pgfqpoint{4.176424in}{4.225546in}}{\pgfqpoint{4.187023in}{4.221155in}}{\pgfqpoint{4.198073in}{4.221155in}}%
\pgfpathclose%
\pgfusepath{stroke,fill}%
\end{pgfscope}%
\begin{pgfscope}%
\pgfpathrectangle{\pgfqpoint{0.481978in}{0.331635in}}{\pgfqpoint{9.300000in}{7.700000in}}%
\pgfusepath{clip}%
\pgfsetbuttcap%
\pgfsetroundjoin%
\definecolor{currentfill}{rgb}{0.815686,0.733333,1.000000}%
\pgfsetfillcolor{currentfill}%
\pgfsetlinewidth{0.481800pt}%
\definecolor{currentstroke}{rgb}{1.000000,1.000000,1.000000}%
\pgfsetstrokecolor{currentstroke}%
\pgfsetdash{}{0pt}%
\pgfpathmoveto{\pgfqpoint{3.029876in}{4.890266in}}%
\pgfpathcurveto{\pgfqpoint{3.040926in}{4.890266in}}{\pgfqpoint{3.051525in}{4.894657in}}{\pgfqpoint{3.059339in}{4.902470in}}%
\pgfpathcurveto{\pgfqpoint{3.067152in}{4.910284in}}{\pgfqpoint{3.071542in}{4.920883in}}{\pgfqpoint{3.071542in}{4.931933in}}%
\pgfpathcurveto{\pgfqpoint{3.071542in}{4.942983in}}{\pgfqpoint{3.067152in}{4.953582in}}{\pgfqpoint{3.059339in}{4.961396in}}%
\pgfpathcurveto{\pgfqpoint{3.051525in}{4.969209in}}{\pgfqpoint{3.040926in}{4.973600in}}{\pgfqpoint{3.029876in}{4.973600in}}%
\pgfpathcurveto{\pgfqpoint{3.018826in}{4.973600in}}{\pgfqpoint{3.008227in}{4.969209in}}{\pgfqpoint{3.000413in}{4.961396in}}%
\pgfpathcurveto{\pgfqpoint{2.992599in}{4.953582in}}{\pgfqpoint{2.988209in}{4.942983in}}{\pgfqpoint{2.988209in}{4.931933in}}%
\pgfpathcurveto{\pgfqpoint{2.988209in}{4.920883in}}{\pgfqpoint{2.992599in}{4.910284in}}{\pgfqpoint{3.000413in}{4.902470in}}%
\pgfpathcurveto{\pgfqpoint{3.008227in}{4.894657in}}{\pgfqpoint{3.018826in}{4.890266in}}{\pgfqpoint{3.029876in}{4.890266in}}%
\pgfpathclose%
\pgfusepath{stroke,fill}%
\end{pgfscope}%
\begin{pgfscope}%
\pgfpathrectangle{\pgfqpoint{0.481978in}{0.331635in}}{\pgfqpoint{9.300000in}{7.700000in}}%
\pgfusepath{clip}%
\pgfsetbuttcap%
\pgfsetroundjoin%
\definecolor{currentfill}{rgb}{0.815686,0.733333,1.000000}%
\pgfsetfillcolor{currentfill}%
\pgfsetlinewidth{0.481800pt}%
\definecolor{currentstroke}{rgb}{1.000000,1.000000,1.000000}%
\pgfsetstrokecolor{currentstroke}%
\pgfsetdash{}{0pt}%
\pgfpathmoveto{\pgfqpoint{1.310658in}{4.217106in}}%
\pgfpathcurveto{\pgfqpoint{1.321708in}{4.217106in}}{\pgfqpoint{1.332307in}{4.221497in}}{\pgfqpoint{1.340121in}{4.229310in}}%
\pgfpathcurveto{\pgfqpoint{1.347934in}{4.237124in}}{\pgfqpoint{1.352325in}{4.247723in}}{\pgfqpoint{1.352325in}{4.258773in}}%
\pgfpathcurveto{\pgfqpoint{1.352325in}{4.269823in}}{\pgfqpoint{1.347934in}{4.280422in}}{\pgfqpoint{1.340121in}{4.288236in}}%
\pgfpathcurveto{\pgfqpoint{1.332307in}{4.296050in}}{\pgfqpoint{1.321708in}{4.300440in}}{\pgfqpoint{1.310658in}{4.300440in}}%
\pgfpathcurveto{\pgfqpoint{1.299608in}{4.300440in}}{\pgfqpoint{1.289009in}{4.296050in}}{\pgfqpoint{1.281195in}{4.288236in}}%
\pgfpathcurveto{\pgfqpoint{1.273382in}{4.280422in}}{\pgfqpoint{1.268991in}{4.269823in}}{\pgfqpoint{1.268991in}{4.258773in}}%
\pgfpathcurveto{\pgfqpoint{1.268991in}{4.247723in}}{\pgfqpoint{1.273382in}{4.237124in}}{\pgfqpoint{1.281195in}{4.229310in}}%
\pgfpathcurveto{\pgfqpoint{1.289009in}{4.221497in}}{\pgfqpoint{1.299608in}{4.217106in}}{\pgfqpoint{1.310658in}{4.217106in}}%
\pgfpathclose%
\pgfusepath{stroke,fill}%
\end{pgfscope}%
\begin{pgfscope}%
\pgfpathrectangle{\pgfqpoint{0.481978in}{0.331635in}}{\pgfqpoint{9.300000in}{7.700000in}}%
\pgfusepath{clip}%
\pgfsetbuttcap%
\pgfsetroundjoin%
\definecolor{currentfill}{rgb}{0.815686,0.733333,1.000000}%
\pgfsetfillcolor{currentfill}%
\pgfsetlinewidth{0.481800pt}%
\definecolor{currentstroke}{rgb}{1.000000,1.000000,1.000000}%
\pgfsetstrokecolor{currentstroke}%
\pgfsetdash{}{0pt}%
\pgfpathmoveto{\pgfqpoint{5.061821in}{4.668455in}}%
\pgfpathcurveto{\pgfqpoint{5.072872in}{4.668455in}}{\pgfqpoint{5.083471in}{4.672845in}}{\pgfqpoint{5.091284in}{4.680659in}}%
\pgfpathcurveto{\pgfqpoint{5.099098in}{4.688473in}}{\pgfqpoint{5.103488in}{4.699072in}}{\pgfqpoint{5.103488in}{4.710122in}}%
\pgfpathcurveto{\pgfqpoint{5.103488in}{4.721172in}}{\pgfqpoint{5.099098in}{4.731771in}}{\pgfqpoint{5.091284in}{4.739585in}}%
\pgfpathcurveto{\pgfqpoint{5.083471in}{4.747398in}}{\pgfqpoint{5.072872in}{4.751788in}}{\pgfqpoint{5.061821in}{4.751788in}}%
\pgfpathcurveto{\pgfqpoint{5.050771in}{4.751788in}}{\pgfqpoint{5.040172in}{4.747398in}}{\pgfqpoint{5.032359in}{4.739585in}}%
\pgfpathcurveto{\pgfqpoint{5.024545in}{4.731771in}}{\pgfqpoint{5.020155in}{4.721172in}}{\pgfqpoint{5.020155in}{4.710122in}}%
\pgfpathcurveto{\pgfqpoint{5.020155in}{4.699072in}}{\pgfqpoint{5.024545in}{4.688473in}}{\pgfqpoint{5.032359in}{4.680659in}}%
\pgfpathcurveto{\pgfqpoint{5.040172in}{4.672845in}}{\pgfqpoint{5.050771in}{4.668455in}}{\pgfqpoint{5.061821in}{4.668455in}}%
\pgfpathclose%
\pgfusepath{stroke,fill}%
\end{pgfscope}%
\begin{pgfscope}%
\pgfpathrectangle{\pgfqpoint{0.481978in}{0.331635in}}{\pgfqpoint{9.300000in}{7.700000in}}%
\pgfusepath{clip}%
\pgfsetbuttcap%
\pgfsetroundjoin%
\definecolor{currentfill}{rgb}{0.815686,0.733333,1.000000}%
\pgfsetfillcolor{currentfill}%
\pgfsetlinewidth{0.481800pt}%
\definecolor{currentstroke}{rgb}{1.000000,1.000000,1.000000}%
\pgfsetstrokecolor{currentstroke}%
\pgfsetdash{}{0pt}%
\pgfpathmoveto{\pgfqpoint{5.046020in}{5.012374in}}%
\pgfpathcurveto{\pgfqpoint{5.057070in}{5.012374in}}{\pgfqpoint{5.067669in}{5.016764in}}{\pgfqpoint{5.075482in}{5.024578in}}%
\pgfpathcurveto{\pgfqpoint{5.083296in}{5.032391in}}{\pgfqpoint{5.087686in}{5.042990in}}{\pgfqpoint{5.087686in}{5.054040in}}%
\pgfpathcurveto{\pgfqpoint{5.087686in}{5.065091in}}{\pgfqpoint{5.083296in}{5.075690in}}{\pgfqpoint{5.075482in}{5.083503in}}%
\pgfpathcurveto{\pgfqpoint{5.067669in}{5.091317in}}{\pgfqpoint{5.057070in}{5.095707in}}{\pgfqpoint{5.046020in}{5.095707in}}%
\pgfpathcurveto{\pgfqpoint{5.034970in}{5.095707in}}{\pgfqpoint{5.024371in}{5.091317in}}{\pgfqpoint{5.016557in}{5.083503in}}%
\pgfpathcurveto{\pgfqpoint{5.008743in}{5.075690in}}{\pgfqpoint{5.004353in}{5.065091in}}{\pgfqpoint{5.004353in}{5.054040in}}%
\pgfpathcurveto{\pgfqpoint{5.004353in}{5.042990in}}{\pgfqpoint{5.008743in}{5.032391in}}{\pgfqpoint{5.016557in}{5.024578in}}%
\pgfpathcurveto{\pgfqpoint{5.024371in}{5.016764in}}{\pgfqpoint{5.034970in}{5.012374in}}{\pgfqpoint{5.046020in}{5.012374in}}%
\pgfpathclose%
\pgfusepath{stroke,fill}%
\end{pgfscope}%
\begin{pgfscope}%
\pgfpathrectangle{\pgfqpoint{0.481978in}{0.331635in}}{\pgfqpoint{9.300000in}{7.700000in}}%
\pgfusepath{clip}%
\pgfsetbuttcap%
\pgfsetroundjoin%
\definecolor{currentfill}{rgb}{0.815686,0.733333,1.000000}%
\pgfsetfillcolor{currentfill}%
\pgfsetlinewidth{0.481800pt}%
\definecolor{currentstroke}{rgb}{1.000000,1.000000,1.000000}%
\pgfsetstrokecolor{currentstroke}%
\pgfsetdash{}{0pt}%
\pgfpathmoveto{\pgfqpoint{7.569943in}{3.637722in}}%
\pgfpathcurveto{\pgfqpoint{7.580993in}{3.637722in}}{\pgfqpoint{7.591592in}{3.642112in}}{\pgfqpoint{7.599406in}{3.649926in}}%
\pgfpathcurveto{\pgfqpoint{7.607219in}{3.657740in}}{\pgfqpoint{7.611610in}{3.668339in}}{\pgfqpoint{7.611610in}{3.679389in}}%
\pgfpathcurveto{\pgfqpoint{7.611610in}{3.690439in}}{\pgfqpoint{7.607219in}{3.701038in}}{\pgfqpoint{7.599406in}{3.708851in}}%
\pgfpathcurveto{\pgfqpoint{7.591592in}{3.716665in}}{\pgfqpoint{7.580993in}{3.721055in}}{\pgfqpoint{7.569943in}{3.721055in}}%
\pgfpathcurveto{\pgfqpoint{7.558893in}{3.721055in}}{\pgfqpoint{7.548294in}{3.716665in}}{\pgfqpoint{7.540480in}{3.708851in}}%
\pgfpathcurveto{\pgfqpoint{7.532667in}{3.701038in}}{\pgfqpoint{7.528276in}{3.690439in}}{\pgfqpoint{7.528276in}{3.679389in}}%
\pgfpathcurveto{\pgfqpoint{7.528276in}{3.668339in}}{\pgfqpoint{7.532667in}{3.657740in}}{\pgfqpoint{7.540480in}{3.649926in}}%
\pgfpathcurveto{\pgfqpoint{7.548294in}{3.642112in}}{\pgfqpoint{7.558893in}{3.637722in}}{\pgfqpoint{7.569943in}{3.637722in}}%
\pgfpathclose%
\pgfusepath{stroke,fill}%
\end{pgfscope}%
\begin{pgfscope}%
\pgfpathrectangle{\pgfqpoint{0.481978in}{0.331635in}}{\pgfqpoint{9.300000in}{7.700000in}}%
\pgfusepath{clip}%
\pgfsetbuttcap%
\pgfsetroundjoin%
\definecolor{currentfill}{rgb}{0.815686,0.733333,1.000000}%
\pgfsetfillcolor{currentfill}%
\pgfsetlinewidth{0.481800pt}%
\definecolor{currentstroke}{rgb}{1.000000,1.000000,1.000000}%
\pgfsetstrokecolor{currentstroke}%
\pgfsetdash{}{0pt}%
\pgfpathmoveto{\pgfqpoint{7.674879in}{3.033872in}}%
\pgfpathcurveto{\pgfqpoint{7.685929in}{3.033872in}}{\pgfqpoint{7.696528in}{3.038262in}}{\pgfqpoint{7.704342in}{3.046076in}}%
\pgfpathcurveto{\pgfqpoint{7.712155in}{3.053889in}}{\pgfqpoint{7.716546in}{3.064489in}}{\pgfqpoint{7.716546in}{3.075539in}}%
\pgfpathcurveto{\pgfqpoint{7.716546in}{3.086589in}}{\pgfqpoint{7.712155in}{3.097188in}}{\pgfqpoint{7.704342in}{3.105001in}}%
\pgfpathcurveto{\pgfqpoint{7.696528in}{3.112815in}}{\pgfqpoint{7.685929in}{3.117205in}}{\pgfqpoint{7.674879in}{3.117205in}}%
\pgfpathcurveto{\pgfqpoint{7.663829in}{3.117205in}}{\pgfqpoint{7.653230in}{3.112815in}}{\pgfqpoint{7.645416in}{3.105001in}}%
\pgfpathcurveto{\pgfqpoint{7.637603in}{3.097188in}}{\pgfqpoint{7.633212in}{3.086589in}}{\pgfqpoint{7.633212in}{3.075539in}}%
\pgfpathcurveto{\pgfqpoint{7.633212in}{3.064489in}}{\pgfqpoint{7.637603in}{3.053889in}}{\pgfqpoint{7.645416in}{3.046076in}}%
\pgfpathcurveto{\pgfqpoint{7.653230in}{3.038262in}}{\pgfqpoint{7.663829in}{3.033872in}}{\pgfqpoint{7.674879in}{3.033872in}}%
\pgfpathclose%
\pgfusepath{stroke,fill}%
\end{pgfscope}%
\begin{pgfscope}%
\pgfpathrectangle{\pgfqpoint{0.481978in}{0.331635in}}{\pgfqpoint{9.300000in}{7.700000in}}%
\pgfusepath{clip}%
\pgfsetbuttcap%
\pgfsetroundjoin%
\definecolor{currentfill}{rgb}{0.815686,0.733333,1.000000}%
\pgfsetfillcolor{currentfill}%
\pgfsetlinewidth{0.481800pt}%
\definecolor{currentstroke}{rgb}{1.000000,1.000000,1.000000}%
\pgfsetstrokecolor{currentstroke}%
\pgfsetdash{}{0pt}%
\pgfpathmoveto{\pgfqpoint{5.428071in}{3.072967in}}%
\pgfpathcurveto{\pgfqpoint{5.439121in}{3.072967in}}{\pgfqpoint{5.449720in}{3.077357in}}{\pgfqpoint{5.457534in}{3.085171in}}%
\pgfpathcurveto{\pgfqpoint{5.465347in}{3.092985in}}{\pgfqpoint{5.469737in}{3.103584in}}{\pgfqpoint{5.469737in}{3.114634in}}%
\pgfpathcurveto{\pgfqpoint{5.469737in}{3.125684in}}{\pgfqpoint{5.465347in}{3.136283in}}{\pgfqpoint{5.457534in}{3.144097in}}%
\pgfpathcurveto{\pgfqpoint{5.449720in}{3.151910in}}{\pgfqpoint{5.439121in}{3.156300in}}{\pgfqpoint{5.428071in}{3.156300in}}%
\pgfpathcurveto{\pgfqpoint{5.417021in}{3.156300in}}{\pgfqpoint{5.406422in}{3.151910in}}{\pgfqpoint{5.398608in}{3.144097in}}%
\pgfpathcurveto{\pgfqpoint{5.390794in}{3.136283in}}{\pgfqpoint{5.386404in}{3.125684in}}{\pgfqpoint{5.386404in}{3.114634in}}%
\pgfpathcurveto{\pgfqpoint{5.386404in}{3.103584in}}{\pgfqpoint{5.390794in}{3.092985in}}{\pgfqpoint{5.398608in}{3.085171in}}%
\pgfpathcurveto{\pgfqpoint{5.406422in}{3.077357in}}{\pgfqpoint{5.417021in}{3.072967in}}{\pgfqpoint{5.428071in}{3.072967in}}%
\pgfpathclose%
\pgfusepath{stroke,fill}%
\end{pgfscope}%
\begin{pgfscope}%
\pgfpathrectangle{\pgfqpoint{0.481978in}{0.331635in}}{\pgfqpoint{9.300000in}{7.700000in}}%
\pgfusepath{clip}%
\pgfsetbuttcap%
\pgfsetroundjoin%
\definecolor{currentfill}{rgb}{0.815686,0.733333,1.000000}%
\pgfsetfillcolor{currentfill}%
\pgfsetlinewidth{0.481800pt}%
\definecolor{currentstroke}{rgb}{1.000000,1.000000,1.000000}%
\pgfsetstrokecolor{currentstroke}%
\pgfsetdash{}{0pt}%
\pgfpathmoveto{\pgfqpoint{2.403350in}{3.601610in}}%
\pgfpathcurveto{\pgfqpoint{2.414400in}{3.601610in}}{\pgfqpoint{2.425000in}{3.606000in}}{\pgfqpoint{2.432813in}{3.613814in}}%
\pgfpathcurveto{\pgfqpoint{2.440627in}{3.621627in}}{\pgfqpoint{2.445017in}{3.632226in}}{\pgfqpoint{2.445017in}{3.643277in}}%
\pgfpathcurveto{\pgfqpoint{2.445017in}{3.654327in}}{\pgfqpoint{2.440627in}{3.664926in}}{\pgfqpoint{2.432813in}{3.672739in}}%
\pgfpathcurveto{\pgfqpoint{2.425000in}{3.680553in}}{\pgfqpoint{2.414400in}{3.684943in}}{\pgfqpoint{2.403350in}{3.684943in}}%
\pgfpathcurveto{\pgfqpoint{2.392300in}{3.684943in}}{\pgfqpoint{2.381701in}{3.680553in}}{\pgfqpoint{2.373888in}{3.672739in}}%
\pgfpathcurveto{\pgfqpoint{2.366074in}{3.664926in}}{\pgfqpoint{2.361684in}{3.654327in}}{\pgfqpoint{2.361684in}{3.643277in}}%
\pgfpathcurveto{\pgfqpoint{2.361684in}{3.632226in}}{\pgfqpoint{2.366074in}{3.621627in}}{\pgfqpoint{2.373888in}{3.613814in}}%
\pgfpathcurveto{\pgfqpoint{2.381701in}{3.606000in}}{\pgfqpoint{2.392300in}{3.601610in}}{\pgfqpoint{2.403350in}{3.601610in}}%
\pgfpathclose%
\pgfusepath{stroke,fill}%
\end{pgfscope}%
\begin{pgfscope}%
\pgfpathrectangle{\pgfqpoint{0.481978in}{0.331635in}}{\pgfqpoint{9.300000in}{7.700000in}}%
\pgfusepath{clip}%
\pgfsetbuttcap%
\pgfsetroundjoin%
\definecolor{currentfill}{rgb}{0.815686,0.733333,1.000000}%
\pgfsetfillcolor{currentfill}%
\pgfsetlinewidth{0.481800pt}%
\definecolor{currentstroke}{rgb}{1.000000,1.000000,1.000000}%
\pgfsetstrokecolor{currentstroke}%
\pgfsetdash{}{0pt}%
\pgfpathmoveto{\pgfqpoint{6.041612in}{5.592475in}}%
\pgfpathcurveto{\pgfqpoint{6.052662in}{5.592475in}}{\pgfqpoint{6.063261in}{5.596865in}}{\pgfqpoint{6.071075in}{5.604679in}}%
\pgfpathcurveto{\pgfqpoint{6.078889in}{5.612492in}}{\pgfqpoint{6.083279in}{5.623091in}}{\pgfqpoint{6.083279in}{5.634142in}}%
\pgfpathcurveto{\pgfqpoint{6.083279in}{5.645192in}}{\pgfqpoint{6.078889in}{5.655791in}}{\pgfqpoint{6.071075in}{5.663604in}}%
\pgfpathcurveto{\pgfqpoint{6.063261in}{5.671418in}}{\pgfqpoint{6.052662in}{5.675808in}}{\pgfqpoint{6.041612in}{5.675808in}}%
\pgfpathcurveto{\pgfqpoint{6.030562in}{5.675808in}}{\pgfqpoint{6.019963in}{5.671418in}}{\pgfqpoint{6.012150in}{5.663604in}}%
\pgfpathcurveto{\pgfqpoint{6.004336in}{5.655791in}}{\pgfqpoint{5.999946in}{5.645192in}}{\pgfqpoint{5.999946in}{5.634142in}}%
\pgfpathcurveto{\pgfqpoint{5.999946in}{5.623091in}}{\pgfqpoint{6.004336in}{5.612492in}}{\pgfqpoint{6.012150in}{5.604679in}}%
\pgfpathcurveto{\pgfqpoint{6.019963in}{5.596865in}}{\pgfqpoint{6.030562in}{5.592475in}}{\pgfqpoint{6.041612in}{5.592475in}}%
\pgfpathclose%
\pgfusepath{stroke,fill}%
\end{pgfscope}%
\begin{pgfscope}%
\pgfpathrectangle{\pgfqpoint{0.481978in}{0.331635in}}{\pgfqpoint{9.300000in}{7.700000in}}%
\pgfusepath{clip}%
\pgfsetbuttcap%
\pgfsetroundjoin%
\definecolor{currentfill}{rgb}{0.815686,0.733333,1.000000}%
\pgfsetfillcolor{currentfill}%
\pgfsetlinewidth{0.481800pt}%
\definecolor{currentstroke}{rgb}{1.000000,1.000000,1.000000}%
\pgfsetstrokecolor{currentstroke}%
\pgfsetdash{}{0pt}%
\pgfpathmoveto{\pgfqpoint{8.383007in}{6.162090in}}%
\pgfpathcurveto{\pgfqpoint{8.394058in}{6.162090in}}{\pgfqpoint{8.404657in}{6.166480in}}{\pgfqpoint{8.412470in}{6.174294in}}%
\pgfpathcurveto{\pgfqpoint{8.420284in}{6.182107in}}{\pgfqpoint{8.424674in}{6.192706in}}{\pgfqpoint{8.424674in}{6.203756in}}%
\pgfpathcurveto{\pgfqpoint{8.424674in}{6.214807in}}{\pgfqpoint{8.420284in}{6.225406in}}{\pgfqpoint{8.412470in}{6.233219in}}%
\pgfpathcurveto{\pgfqpoint{8.404657in}{6.241033in}}{\pgfqpoint{8.394058in}{6.245423in}}{\pgfqpoint{8.383007in}{6.245423in}}%
\pgfpathcurveto{\pgfqpoint{8.371957in}{6.245423in}}{\pgfqpoint{8.361358in}{6.241033in}}{\pgfqpoint{8.353545in}{6.233219in}}%
\pgfpathcurveto{\pgfqpoint{8.345731in}{6.225406in}}{\pgfqpoint{8.341341in}{6.214807in}}{\pgfqpoint{8.341341in}{6.203756in}}%
\pgfpathcurveto{\pgfqpoint{8.341341in}{6.192706in}}{\pgfqpoint{8.345731in}{6.182107in}}{\pgfqpoint{8.353545in}{6.174294in}}%
\pgfpathcurveto{\pgfqpoint{8.361358in}{6.166480in}}{\pgfqpoint{8.371957in}{6.162090in}}{\pgfqpoint{8.383007in}{6.162090in}}%
\pgfpathclose%
\pgfusepath{stroke,fill}%
\end{pgfscope}%
\begin{pgfscope}%
\pgfpathrectangle{\pgfqpoint{0.481978in}{0.331635in}}{\pgfqpoint{9.300000in}{7.700000in}}%
\pgfusepath{clip}%
\pgfsetbuttcap%
\pgfsetroundjoin%
\definecolor{currentfill}{rgb}{0.815686,0.733333,1.000000}%
\pgfsetfillcolor{currentfill}%
\pgfsetlinewidth{0.481800pt}%
\definecolor{currentstroke}{rgb}{1.000000,1.000000,1.000000}%
\pgfsetstrokecolor{currentstroke}%
\pgfsetdash{}{0pt}%
\pgfpathmoveto{\pgfqpoint{4.398769in}{4.525825in}}%
\pgfpathcurveto{\pgfqpoint{4.409819in}{4.525825in}}{\pgfqpoint{4.420418in}{4.530215in}}{\pgfqpoint{4.428231in}{4.538029in}}%
\pgfpathcurveto{\pgfqpoint{4.436045in}{4.545842in}}{\pgfqpoint{4.440435in}{4.556441in}}{\pgfqpoint{4.440435in}{4.567491in}}%
\pgfpathcurveto{\pgfqpoint{4.440435in}{4.578542in}}{\pgfqpoint{4.436045in}{4.589141in}}{\pgfqpoint{4.428231in}{4.596954in}}%
\pgfpathcurveto{\pgfqpoint{4.420418in}{4.604768in}}{\pgfqpoint{4.409819in}{4.609158in}}{\pgfqpoint{4.398769in}{4.609158in}}%
\pgfpathcurveto{\pgfqpoint{4.387718in}{4.609158in}}{\pgfqpoint{4.377119in}{4.604768in}}{\pgfqpoint{4.369306in}{4.596954in}}%
\pgfpathcurveto{\pgfqpoint{4.361492in}{4.589141in}}{\pgfqpoint{4.357102in}{4.578542in}}{\pgfqpoint{4.357102in}{4.567491in}}%
\pgfpathcurveto{\pgfqpoint{4.357102in}{4.556441in}}{\pgfqpoint{4.361492in}{4.545842in}}{\pgfqpoint{4.369306in}{4.538029in}}%
\pgfpathcurveto{\pgfqpoint{4.377119in}{4.530215in}}{\pgfqpoint{4.387718in}{4.525825in}}{\pgfqpoint{4.398769in}{4.525825in}}%
\pgfpathclose%
\pgfusepath{stroke,fill}%
\end{pgfscope}%
\begin{pgfscope}%
\pgfpathrectangle{\pgfqpoint{0.481978in}{0.331635in}}{\pgfqpoint{9.300000in}{7.700000in}}%
\pgfusepath{clip}%
\pgfsetbuttcap%
\pgfsetroundjoin%
\definecolor{currentfill}{rgb}{0.815686,0.733333,1.000000}%
\pgfsetfillcolor{currentfill}%
\pgfsetlinewidth{0.481800pt}%
\definecolor{currentstroke}{rgb}{1.000000,1.000000,1.000000}%
\pgfsetstrokecolor{currentstroke}%
\pgfsetdash{}{0pt}%
\pgfpathmoveto{\pgfqpoint{8.682490in}{5.777703in}}%
\pgfpathcurveto{\pgfqpoint{8.693540in}{5.777703in}}{\pgfqpoint{8.704139in}{5.782093in}}{\pgfqpoint{8.711953in}{5.789906in}}%
\pgfpathcurveto{\pgfqpoint{8.719766in}{5.797720in}}{\pgfqpoint{8.724157in}{5.808319in}}{\pgfqpoint{8.724157in}{5.819369in}}%
\pgfpathcurveto{\pgfqpoint{8.724157in}{5.830419in}}{\pgfqpoint{8.719766in}{5.841018in}}{\pgfqpoint{8.711953in}{5.848832in}}%
\pgfpathcurveto{\pgfqpoint{8.704139in}{5.856646in}}{\pgfqpoint{8.693540in}{5.861036in}}{\pgfqpoint{8.682490in}{5.861036in}}%
\pgfpathcurveto{\pgfqpoint{8.671440in}{5.861036in}}{\pgfqpoint{8.660841in}{5.856646in}}{\pgfqpoint{8.653027in}{5.848832in}}%
\pgfpathcurveto{\pgfqpoint{8.645213in}{5.841018in}}{\pgfqpoint{8.640823in}{5.830419in}}{\pgfqpoint{8.640823in}{5.819369in}}%
\pgfpathcurveto{\pgfqpoint{8.640823in}{5.808319in}}{\pgfqpoint{8.645213in}{5.797720in}}{\pgfqpoint{8.653027in}{5.789906in}}%
\pgfpathcurveto{\pgfqpoint{8.660841in}{5.782093in}}{\pgfqpoint{8.671440in}{5.777703in}}{\pgfqpoint{8.682490in}{5.777703in}}%
\pgfpathclose%
\pgfusepath{stroke,fill}%
\end{pgfscope}%
\begin{pgfscope}%
\pgfpathrectangle{\pgfqpoint{0.481978in}{0.331635in}}{\pgfqpoint{9.300000in}{7.700000in}}%
\pgfusepath{clip}%
\pgfsetbuttcap%
\pgfsetroundjoin%
\definecolor{currentfill}{rgb}{0.815686,0.733333,1.000000}%
\pgfsetfillcolor{currentfill}%
\pgfsetlinewidth{0.481800pt}%
\definecolor{currentstroke}{rgb}{1.000000,1.000000,1.000000}%
\pgfsetstrokecolor{currentstroke}%
\pgfsetdash{}{0pt}%
\pgfpathmoveto{\pgfqpoint{5.664598in}{1.113364in}}%
\pgfpathcurveto{\pgfqpoint{5.675648in}{1.113364in}}{\pgfqpoint{5.686248in}{1.117754in}}{\pgfqpoint{5.694061in}{1.125568in}}%
\pgfpathcurveto{\pgfqpoint{5.701875in}{1.133382in}}{\pgfqpoint{5.706265in}{1.143981in}}{\pgfqpoint{5.706265in}{1.155031in}}%
\pgfpathcurveto{\pgfqpoint{5.706265in}{1.166081in}}{\pgfqpoint{5.701875in}{1.176680in}}{\pgfqpoint{5.694061in}{1.184494in}}%
\pgfpathcurveto{\pgfqpoint{5.686248in}{1.192307in}}{\pgfqpoint{5.675648in}{1.196698in}}{\pgfqpoint{5.664598in}{1.196698in}}%
\pgfpathcurveto{\pgfqpoint{5.653548in}{1.196698in}}{\pgfqpoint{5.642949in}{1.192307in}}{\pgfqpoint{5.635136in}{1.184494in}}%
\pgfpathcurveto{\pgfqpoint{5.627322in}{1.176680in}}{\pgfqpoint{5.622932in}{1.166081in}}{\pgfqpoint{5.622932in}{1.155031in}}%
\pgfpathcurveto{\pgfqpoint{5.622932in}{1.143981in}}{\pgfqpoint{5.627322in}{1.133382in}}{\pgfqpoint{5.635136in}{1.125568in}}%
\pgfpathcurveto{\pgfqpoint{5.642949in}{1.117754in}}{\pgfqpoint{5.653548in}{1.113364in}}{\pgfqpoint{5.664598in}{1.113364in}}%
\pgfpathclose%
\pgfusepath{stroke,fill}%
\end{pgfscope}%
\begin{pgfscope}%
\pgfpathrectangle{\pgfqpoint{0.481978in}{0.331635in}}{\pgfqpoint{9.300000in}{7.700000in}}%
\pgfusepath{clip}%
\pgfsetbuttcap%
\pgfsetroundjoin%
\definecolor{currentfill}{rgb}{0.815686,0.733333,1.000000}%
\pgfsetfillcolor{currentfill}%
\pgfsetlinewidth{0.481800pt}%
\definecolor{currentstroke}{rgb}{1.000000,1.000000,1.000000}%
\pgfsetstrokecolor{currentstroke}%
\pgfsetdash{}{0pt}%
\pgfpathmoveto{\pgfqpoint{1.939227in}{4.311857in}}%
\pgfpathcurveto{\pgfqpoint{1.950277in}{4.311857in}}{\pgfqpoint{1.960876in}{4.316247in}}{\pgfqpoint{1.968690in}{4.324061in}}%
\pgfpathcurveto{\pgfqpoint{1.976504in}{4.331874in}}{\pgfqpoint{1.980894in}{4.342473in}}{\pgfqpoint{1.980894in}{4.353523in}}%
\pgfpathcurveto{\pgfqpoint{1.980894in}{4.364574in}}{\pgfqpoint{1.976504in}{4.375173in}}{\pgfqpoint{1.968690in}{4.382986in}}%
\pgfpathcurveto{\pgfqpoint{1.960876in}{4.390800in}}{\pgfqpoint{1.950277in}{4.395190in}}{\pgfqpoint{1.939227in}{4.395190in}}%
\pgfpathcurveto{\pgfqpoint{1.928177in}{4.395190in}}{\pgfqpoint{1.917578in}{4.390800in}}{\pgfqpoint{1.909765in}{4.382986in}}%
\pgfpathcurveto{\pgfqpoint{1.901951in}{4.375173in}}{\pgfqpoint{1.897561in}{4.364574in}}{\pgfqpoint{1.897561in}{4.353523in}}%
\pgfpathcurveto{\pgfqpoint{1.897561in}{4.342473in}}{\pgfqpoint{1.901951in}{4.331874in}}{\pgfqpoint{1.909765in}{4.324061in}}%
\pgfpathcurveto{\pgfqpoint{1.917578in}{4.316247in}}{\pgfqpoint{1.928177in}{4.311857in}}{\pgfqpoint{1.939227in}{4.311857in}}%
\pgfpathclose%
\pgfusepath{stroke,fill}%
\end{pgfscope}%
\begin{pgfscope}%
\pgfpathrectangle{\pgfqpoint{0.481978in}{0.331635in}}{\pgfqpoint{9.300000in}{7.700000in}}%
\pgfusepath{clip}%
\pgfsetbuttcap%
\pgfsetroundjoin%
\definecolor{currentfill}{rgb}{0.815686,0.733333,1.000000}%
\pgfsetfillcolor{currentfill}%
\pgfsetlinewidth{0.481800pt}%
\definecolor{currentstroke}{rgb}{1.000000,1.000000,1.000000}%
\pgfsetstrokecolor{currentstroke}%
\pgfsetdash{}{0pt}%
\pgfpathmoveto{\pgfqpoint{6.777031in}{4.407810in}}%
\pgfpathcurveto{\pgfqpoint{6.788081in}{4.407810in}}{\pgfqpoint{6.798680in}{4.412200in}}{\pgfqpoint{6.806494in}{4.420014in}}%
\pgfpathcurveto{\pgfqpoint{6.814308in}{4.427828in}}{\pgfqpoint{6.818698in}{4.438427in}}{\pgfqpoint{6.818698in}{4.449477in}}%
\pgfpathcurveto{\pgfqpoint{6.818698in}{4.460527in}}{\pgfqpoint{6.814308in}{4.471126in}}{\pgfqpoint{6.806494in}{4.478939in}}%
\pgfpathcurveto{\pgfqpoint{6.798680in}{4.486753in}}{\pgfqpoint{6.788081in}{4.491143in}}{\pgfqpoint{6.777031in}{4.491143in}}%
\pgfpathcurveto{\pgfqpoint{6.765981in}{4.491143in}}{\pgfqpoint{6.755382in}{4.486753in}}{\pgfqpoint{6.747568in}{4.478939in}}%
\pgfpathcurveto{\pgfqpoint{6.739755in}{4.471126in}}{\pgfqpoint{6.735365in}{4.460527in}}{\pgfqpoint{6.735365in}{4.449477in}}%
\pgfpathcurveto{\pgfqpoint{6.735365in}{4.438427in}}{\pgfqpoint{6.739755in}{4.427828in}}{\pgfqpoint{6.747568in}{4.420014in}}%
\pgfpathcurveto{\pgfqpoint{6.755382in}{4.412200in}}{\pgfqpoint{6.765981in}{4.407810in}}{\pgfqpoint{6.777031in}{4.407810in}}%
\pgfpathclose%
\pgfusepath{stroke,fill}%
\end{pgfscope}%
\begin{pgfscope}%
\pgfpathrectangle{\pgfqpoint{0.481978in}{0.331635in}}{\pgfqpoint{9.300000in}{7.700000in}}%
\pgfusepath{clip}%
\pgfsetbuttcap%
\pgfsetroundjoin%
\definecolor{currentfill}{rgb}{0.815686,0.733333,1.000000}%
\pgfsetfillcolor{currentfill}%
\pgfsetlinewidth{0.481800pt}%
\definecolor{currentstroke}{rgb}{1.000000,1.000000,1.000000}%
\pgfsetstrokecolor{currentstroke}%
\pgfsetdash{}{0pt}%
\pgfpathmoveto{\pgfqpoint{3.798724in}{2.123955in}}%
\pgfpathcurveto{\pgfqpoint{3.809775in}{2.123955in}}{\pgfqpoint{3.820374in}{2.128345in}}{\pgfqpoint{3.828187in}{2.136158in}}%
\pgfpathcurveto{\pgfqpoint{3.836001in}{2.143972in}}{\pgfqpoint{3.840391in}{2.154571in}}{\pgfqpoint{3.840391in}{2.165621in}}%
\pgfpathcurveto{\pgfqpoint{3.840391in}{2.176671in}}{\pgfqpoint{3.836001in}{2.187270in}}{\pgfqpoint{3.828187in}{2.195084in}}%
\pgfpathcurveto{\pgfqpoint{3.820374in}{2.202898in}}{\pgfqpoint{3.809775in}{2.207288in}}{\pgfqpoint{3.798724in}{2.207288in}}%
\pgfpathcurveto{\pgfqpoint{3.787674in}{2.207288in}}{\pgfqpoint{3.777075in}{2.202898in}}{\pgfqpoint{3.769262in}{2.195084in}}%
\pgfpathcurveto{\pgfqpoint{3.761448in}{2.187270in}}{\pgfqpoint{3.757058in}{2.176671in}}{\pgfqpoint{3.757058in}{2.165621in}}%
\pgfpathcurveto{\pgfqpoint{3.757058in}{2.154571in}}{\pgfqpoint{3.761448in}{2.143972in}}{\pgfqpoint{3.769262in}{2.136158in}}%
\pgfpathcurveto{\pgfqpoint{3.777075in}{2.128345in}}{\pgfqpoint{3.787674in}{2.123955in}}{\pgfqpoint{3.798724in}{2.123955in}}%
\pgfpathclose%
\pgfusepath{stroke,fill}%
\end{pgfscope}%
\begin{pgfscope}%
\pgfpathrectangle{\pgfqpoint{0.481978in}{0.331635in}}{\pgfqpoint{9.300000in}{7.700000in}}%
\pgfusepath{clip}%
\pgfsetbuttcap%
\pgfsetroundjoin%
\definecolor{currentfill}{rgb}{0.815686,0.733333,1.000000}%
\pgfsetfillcolor{currentfill}%
\pgfsetlinewidth{0.481800pt}%
\definecolor{currentstroke}{rgb}{1.000000,1.000000,1.000000}%
\pgfsetstrokecolor{currentstroke}%
\pgfsetdash{}{0pt}%
\pgfpathmoveto{\pgfqpoint{1.738451in}{3.862058in}}%
\pgfpathcurveto{\pgfqpoint{1.749501in}{3.862058in}}{\pgfqpoint{1.760100in}{3.866448in}}{\pgfqpoint{1.767914in}{3.874262in}}%
\pgfpathcurveto{\pgfqpoint{1.775728in}{3.882075in}}{\pgfqpoint{1.780118in}{3.892674in}}{\pgfqpoint{1.780118in}{3.903724in}}%
\pgfpathcurveto{\pgfqpoint{1.780118in}{3.914774in}}{\pgfqpoint{1.775728in}{3.925373in}}{\pgfqpoint{1.767914in}{3.933187in}}%
\pgfpathcurveto{\pgfqpoint{1.760100in}{3.941001in}}{\pgfqpoint{1.749501in}{3.945391in}}{\pgfqpoint{1.738451in}{3.945391in}}%
\pgfpathcurveto{\pgfqpoint{1.727401in}{3.945391in}}{\pgfqpoint{1.716802in}{3.941001in}}{\pgfqpoint{1.708989in}{3.933187in}}%
\pgfpathcurveto{\pgfqpoint{1.701175in}{3.925373in}}{\pgfqpoint{1.696785in}{3.914774in}}{\pgfqpoint{1.696785in}{3.903724in}}%
\pgfpathcurveto{\pgfqpoint{1.696785in}{3.892674in}}{\pgfqpoint{1.701175in}{3.882075in}}{\pgfqpoint{1.708989in}{3.874262in}}%
\pgfpathcurveto{\pgfqpoint{1.716802in}{3.866448in}}{\pgfqpoint{1.727401in}{3.862058in}}{\pgfqpoint{1.738451in}{3.862058in}}%
\pgfpathclose%
\pgfusepath{stroke,fill}%
\end{pgfscope}%
\begin{pgfscope}%
\pgfpathrectangle{\pgfqpoint{0.481978in}{0.331635in}}{\pgfqpoint{9.300000in}{7.700000in}}%
\pgfusepath{clip}%
\pgfsetbuttcap%
\pgfsetroundjoin%
\definecolor{currentfill}{rgb}{0.815686,0.733333,1.000000}%
\pgfsetfillcolor{currentfill}%
\pgfsetlinewidth{0.481800pt}%
\definecolor{currentstroke}{rgb}{1.000000,1.000000,1.000000}%
\pgfsetstrokecolor{currentstroke}%
\pgfsetdash{}{0pt}%
\pgfpathmoveto{\pgfqpoint{8.508088in}{3.964215in}}%
\pgfpathcurveto{\pgfqpoint{8.519138in}{3.964215in}}{\pgfqpoint{8.529737in}{3.968605in}}{\pgfqpoint{8.537550in}{3.976419in}}%
\pgfpathcurveto{\pgfqpoint{8.545364in}{3.984232in}}{\pgfqpoint{8.549754in}{3.994831in}}{\pgfqpoint{8.549754in}{4.005881in}}%
\pgfpathcurveto{\pgfqpoint{8.549754in}{4.016932in}}{\pgfqpoint{8.545364in}{4.027531in}}{\pgfqpoint{8.537550in}{4.035344in}}%
\pgfpathcurveto{\pgfqpoint{8.529737in}{4.043158in}}{\pgfqpoint{8.519138in}{4.047548in}}{\pgfqpoint{8.508088in}{4.047548in}}%
\pgfpathcurveto{\pgfqpoint{8.497037in}{4.047548in}}{\pgfqpoint{8.486438in}{4.043158in}}{\pgfqpoint{8.478625in}{4.035344in}}%
\pgfpathcurveto{\pgfqpoint{8.470811in}{4.027531in}}{\pgfqpoint{8.466421in}{4.016932in}}{\pgfqpoint{8.466421in}{4.005881in}}%
\pgfpathcurveto{\pgfqpoint{8.466421in}{3.994831in}}{\pgfqpoint{8.470811in}{3.984232in}}{\pgfqpoint{8.478625in}{3.976419in}}%
\pgfpathcurveto{\pgfqpoint{8.486438in}{3.968605in}}{\pgfqpoint{8.497037in}{3.964215in}}{\pgfqpoint{8.508088in}{3.964215in}}%
\pgfpathclose%
\pgfusepath{stroke,fill}%
\end{pgfscope}%
\begin{pgfscope}%
\pgfpathrectangle{\pgfqpoint{0.481978in}{0.331635in}}{\pgfqpoint{9.300000in}{7.700000in}}%
\pgfusepath{clip}%
\pgfsetbuttcap%
\pgfsetroundjoin%
\definecolor{currentfill}{rgb}{0.815686,0.733333,1.000000}%
\pgfsetfillcolor{currentfill}%
\pgfsetlinewidth{0.481800pt}%
\definecolor{currentstroke}{rgb}{1.000000,1.000000,1.000000}%
\pgfsetstrokecolor{currentstroke}%
\pgfsetdash{}{0pt}%
\pgfpathmoveto{\pgfqpoint{6.837191in}{4.849315in}}%
\pgfpathcurveto{\pgfqpoint{6.848241in}{4.849315in}}{\pgfqpoint{6.858840in}{4.853705in}}{\pgfqpoint{6.866654in}{4.861519in}}%
\pgfpathcurveto{\pgfqpoint{6.874468in}{4.869333in}}{\pgfqpoint{6.878858in}{4.879932in}}{\pgfqpoint{6.878858in}{4.890982in}}%
\pgfpathcurveto{\pgfqpoint{6.878858in}{4.902032in}}{\pgfqpoint{6.874468in}{4.912631in}}{\pgfqpoint{6.866654in}{4.920444in}}%
\pgfpathcurveto{\pgfqpoint{6.858840in}{4.928258in}}{\pgfqpoint{6.848241in}{4.932648in}}{\pgfqpoint{6.837191in}{4.932648in}}%
\pgfpathcurveto{\pgfqpoint{6.826141in}{4.932648in}}{\pgfqpoint{6.815542in}{4.928258in}}{\pgfqpoint{6.807729in}{4.920444in}}%
\pgfpathcurveto{\pgfqpoint{6.799915in}{4.912631in}}{\pgfqpoint{6.795525in}{4.902032in}}{\pgfqpoint{6.795525in}{4.890982in}}%
\pgfpathcurveto{\pgfqpoint{6.795525in}{4.879932in}}{\pgfqpoint{6.799915in}{4.869333in}}{\pgfqpoint{6.807729in}{4.861519in}}%
\pgfpathcurveto{\pgfqpoint{6.815542in}{4.853705in}}{\pgfqpoint{6.826141in}{4.849315in}}{\pgfqpoint{6.837191in}{4.849315in}}%
\pgfpathclose%
\pgfusepath{stroke,fill}%
\end{pgfscope}%
\begin{pgfscope}%
\pgfpathrectangle{\pgfqpoint{0.481978in}{0.331635in}}{\pgfqpoint{9.300000in}{7.700000in}}%
\pgfusepath{clip}%
\pgfsetbuttcap%
\pgfsetroundjoin%
\definecolor{currentfill}{rgb}{0.815686,0.733333,1.000000}%
\pgfsetfillcolor{currentfill}%
\pgfsetlinewidth{0.481800pt}%
\definecolor{currentstroke}{rgb}{1.000000,1.000000,1.000000}%
\pgfsetstrokecolor{currentstroke}%
\pgfsetdash{}{0pt}%
\pgfpathmoveto{\pgfqpoint{4.101155in}{3.717957in}}%
\pgfpathcurveto{\pgfqpoint{4.112206in}{3.717957in}}{\pgfqpoint{4.122805in}{3.722347in}}{\pgfqpoint{4.130618in}{3.730161in}}%
\pgfpathcurveto{\pgfqpoint{4.138432in}{3.737975in}}{\pgfqpoint{4.142822in}{3.748574in}}{\pgfqpoint{4.142822in}{3.759624in}}%
\pgfpathcurveto{\pgfqpoint{4.142822in}{3.770674in}}{\pgfqpoint{4.138432in}{3.781273in}}{\pgfqpoint{4.130618in}{3.789086in}}%
\pgfpathcurveto{\pgfqpoint{4.122805in}{3.796900in}}{\pgfqpoint{4.112206in}{3.801290in}}{\pgfqpoint{4.101155in}{3.801290in}}%
\pgfpathcurveto{\pgfqpoint{4.090105in}{3.801290in}}{\pgfqpoint{4.079506in}{3.796900in}}{\pgfqpoint{4.071693in}{3.789086in}}%
\pgfpathcurveto{\pgfqpoint{4.063879in}{3.781273in}}{\pgfqpoint{4.059489in}{3.770674in}}{\pgfqpoint{4.059489in}{3.759624in}}%
\pgfpathcurveto{\pgfqpoint{4.059489in}{3.748574in}}{\pgfqpoint{4.063879in}{3.737975in}}{\pgfqpoint{4.071693in}{3.730161in}}%
\pgfpathcurveto{\pgfqpoint{4.079506in}{3.722347in}}{\pgfqpoint{4.090105in}{3.717957in}}{\pgfqpoint{4.101155in}{3.717957in}}%
\pgfpathclose%
\pgfusepath{stroke,fill}%
\end{pgfscope}%
\begin{pgfscope}%
\pgfpathrectangle{\pgfqpoint{0.481978in}{0.331635in}}{\pgfqpoint{9.300000in}{7.700000in}}%
\pgfusepath{clip}%
\pgfsetbuttcap%
\pgfsetroundjoin%
\definecolor{currentfill}{rgb}{0.815686,0.733333,1.000000}%
\pgfsetfillcolor{currentfill}%
\pgfsetlinewidth{0.481800pt}%
\definecolor{currentstroke}{rgb}{1.000000,1.000000,1.000000}%
\pgfsetstrokecolor{currentstroke}%
\pgfsetdash{}{0pt}%
\pgfpathmoveto{\pgfqpoint{7.340850in}{3.253024in}}%
\pgfpathcurveto{\pgfqpoint{7.351900in}{3.253024in}}{\pgfqpoint{7.362499in}{3.257414in}}{\pgfqpoint{7.370313in}{3.265227in}}%
\pgfpathcurveto{\pgfqpoint{7.378127in}{3.273041in}}{\pgfqpoint{7.382517in}{3.283640in}}{\pgfqpoint{7.382517in}{3.294690in}}%
\pgfpathcurveto{\pgfqpoint{7.382517in}{3.305740in}}{\pgfqpoint{7.378127in}{3.316339in}}{\pgfqpoint{7.370313in}{3.324153in}}%
\pgfpathcurveto{\pgfqpoint{7.362499in}{3.331967in}}{\pgfqpoint{7.351900in}{3.336357in}}{\pgfqpoint{7.340850in}{3.336357in}}%
\pgfpathcurveto{\pgfqpoint{7.329800in}{3.336357in}}{\pgfqpoint{7.319201in}{3.331967in}}{\pgfqpoint{7.311387in}{3.324153in}}%
\pgfpathcurveto{\pgfqpoint{7.303574in}{3.316339in}}{\pgfqpoint{7.299184in}{3.305740in}}{\pgfqpoint{7.299184in}{3.294690in}}%
\pgfpathcurveto{\pgfqpoint{7.299184in}{3.283640in}}{\pgfqpoint{7.303574in}{3.273041in}}{\pgfqpoint{7.311387in}{3.265227in}}%
\pgfpathcurveto{\pgfqpoint{7.319201in}{3.257414in}}{\pgfqpoint{7.329800in}{3.253024in}}{\pgfqpoint{7.340850in}{3.253024in}}%
\pgfpathclose%
\pgfusepath{stroke,fill}%
\end{pgfscope}%
\begin{pgfscope}%
\pgfpathrectangle{\pgfqpoint{0.481978in}{0.331635in}}{\pgfqpoint{9.300000in}{7.700000in}}%
\pgfusepath{clip}%
\pgfsetbuttcap%
\pgfsetroundjoin%
\definecolor{currentfill}{rgb}{0.815686,0.733333,1.000000}%
\pgfsetfillcolor{currentfill}%
\pgfsetlinewidth{0.481800pt}%
\definecolor{currentstroke}{rgb}{1.000000,1.000000,1.000000}%
\pgfsetstrokecolor{currentstroke}%
\pgfsetdash{}{0pt}%
\pgfpathmoveto{\pgfqpoint{4.878178in}{5.062333in}}%
\pgfpathcurveto{\pgfqpoint{4.889228in}{5.062333in}}{\pgfqpoint{4.899827in}{5.066724in}}{\pgfqpoint{4.907641in}{5.074537in}}%
\pgfpathcurveto{\pgfqpoint{4.915454in}{5.082351in}}{\pgfqpoint{4.919845in}{5.092950in}}{\pgfqpoint{4.919845in}{5.104000in}}%
\pgfpathcurveto{\pgfqpoint{4.919845in}{5.115050in}}{\pgfqpoint{4.915454in}{5.125649in}}{\pgfqpoint{4.907641in}{5.133463in}}%
\pgfpathcurveto{\pgfqpoint{4.899827in}{5.141277in}}{\pgfqpoint{4.889228in}{5.145667in}}{\pgfqpoint{4.878178in}{5.145667in}}%
\pgfpathcurveto{\pgfqpoint{4.867128in}{5.145667in}}{\pgfqpoint{4.856529in}{5.141277in}}{\pgfqpoint{4.848715in}{5.133463in}}%
\pgfpathcurveto{\pgfqpoint{4.840902in}{5.125649in}}{\pgfqpoint{4.836511in}{5.115050in}}{\pgfqpoint{4.836511in}{5.104000in}}%
\pgfpathcurveto{\pgfqpoint{4.836511in}{5.092950in}}{\pgfqpoint{4.840902in}{5.082351in}}{\pgfqpoint{4.848715in}{5.074537in}}%
\pgfpathcurveto{\pgfqpoint{4.856529in}{5.066724in}}{\pgfqpoint{4.867128in}{5.062333in}}{\pgfqpoint{4.878178in}{5.062333in}}%
\pgfpathclose%
\pgfusepath{stroke,fill}%
\end{pgfscope}%
\begin{pgfscope}%
\pgfpathrectangle{\pgfqpoint{0.481978in}{0.331635in}}{\pgfqpoint{9.300000in}{7.700000in}}%
\pgfusepath{clip}%
\pgfsetbuttcap%
\pgfsetroundjoin%
\definecolor{currentfill}{rgb}{0.815686,0.733333,1.000000}%
\pgfsetfillcolor{currentfill}%
\pgfsetlinewidth{0.481800pt}%
\definecolor{currentstroke}{rgb}{1.000000,1.000000,1.000000}%
\pgfsetstrokecolor{currentstroke}%
\pgfsetdash{}{0pt}%
\pgfpathmoveto{\pgfqpoint{3.093874in}{4.847172in}}%
\pgfpathcurveto{\pgfqpoint{3.104924in}{4.847172in}}{\pgfqpoint{3.115523in}{4.851563in}}{\pgfqpoint{3.123337in}{4.859376in}}%
\pgfpathcurveto{\pgfqpoint{3.131150in}{4.867190in}}{\pgfqpoint{3.135541in}{4.877789in}}{\pgfqpoint{3.135541in}{4.888839in}}%
\pgfpathcurveto{\pgfqpoint{3.135541in}{4.899889in}}{\pgfqpoint{3.131150in}{4.910488in}}{\pgfqpoint{3.123337in}{4.918302in}}%
\pgfpathcurveto{\pgfqpoint{3.115523in}{4.926115in}}{\pgfqpoint{3.104924in}{4.930506in}}{\pgfqpoint{3.093874in}{4.930506in}}%
\pgfpathcurveto{\pgfqpoint{3.082824in}{4.930506in}}{\pgfqpoint{3.072225in}{4.926115in}}{\pgfqpoint{3.064411in}{4.918302in}}%
\pgfpathcurveto{\pgfqpoint{3.056598in}{4.910488in}}{\pgfqpoint{3.052207in}{4.899889in}}{\pgfqpoint{3.052207in}{4.888839in}}%
\pgfpathcurveto{\pgfqpoint{3.052207in}{4.877789in}}{\pgfqpoint{3.056598in}{4.867190in}}{\pgfqpoint{3.064411in}{4.859376in}}%
\pgfpathcurveto{\pgfqpoint{3.072225in}{4.851563in}}{\pgfqpoint{3.082824in}{4.847172in}}{\pgfqpoint{3.093874in}{4.847172in}}%
\pgfpathclose%
\pgfusepath{stroke,fill}%
\end{pgfscope}%
\begin{pgfscope}%
\pgfpathrectangle{\pgfqpoint{0.481978in}{0.331635in}}{\pgfqpoint{9.300000in}{7.700000in}}%
\pgfusepath{clip}%
\pgfsetbuttcap%
\pgfsetroundjoin%
\definecolor{currentfill}{rgb}{0.815686,0.733333,1.000000}%
\pgfsetfillcolor{currentfill}%
\pgfsetlinewidth{0.481800pt}%
\definecolor{currentstroke}{rgb}{1.000000,1.000000,1.000000}%
\pgfsetstrokecolor{currentstroke}%
\pgfsetdash{}{0pt}%
\pgfpathmoveto{\pgfqpoint{5.273114in}{1.865984in}}%
\pgfpathcurveto{\pgfqpoint{5.284164in}{1.865984in}}{\pgfqpoint{5.294763in}{1.870375in}}{\pgfqpoint{5.302577in}{1.878188in}}%
\pgfpathcurveto{\pgfqpoint{5.310390in}{1.886002in}}{\pgfqpoint{5.314781in}{1.896601in}}{\pgfqpoint{5.314781in}{1.907651in}}%
\pgfpathcurveto{\pgfqpoint{5.314781in}{1.918701in}}{\pgfqpoint{5.310390in}{1.929300in}}{\pgfqpoint{5.302577in}{1.937114in}}%
\pgfpathcurveto{\pgfqpoint{5.294763in}{1.944927in}}{\pgfqpoint{5.284164in}{1.949318in}}{\pgfqpoint{5.273114in}{1.949318in}}%
\pgfpathcurveto{\pgfqpoint{5.262064in}{1.949318in}}{\pgfqpoint{5.251465in}{1.944927in}}{\pgfqpoint{5.243651in}{1.937114in}}%
\pgfpathcurveto{\pgfqpoint{5.235838in}{1.929300in}}{\pgfqpoint{5.231447in}{1.918701in}}{\pgfqpoint{5.231447in}{1.907651in}}%
\pgfpathcurveto{\pgfqpoint{5.231447in}{1.896601in}}{\pgfqpoint{5.235838in}{1.886002in}}{\pgfqpoint{5.243651in}{1.878188in}}%
\pgfpathcurveto{\pgfqpoint{5.251465in}{1.870375in}}{\pgfqpoint{5.262064in}{1.865984in}}{\pgfqpoint{5.273114in}{1.865984in}}%
\pgfpathclose%
\pgfusepath{stroke,fill}%
\end{pgfscope}%
\begin{pgfscope}%
\pgfpathrectangle{\pgfqpoint{0.481978in}{0.331635in}}{\pgfqpoint{9.300000in}{7.700000in}}%
\pgfusepath{clip}%
\pgfsetbuttcap%
\pgfsetroundjoin%
\definecolor{currentfill}{rgb}{0.815686,0.733333,1.000000}%
\pgfsetfillcolor{currentfill}%
\pgfsetlinewidth{0.481800pt}%
\definecolor{currentstroke}{rgb}{1.000000,1.000000,1.000000}%
\pgfsetstrokecolor{currentstroke}%
\pgfsetdash{}{0pt}%
\pgfpathmoveto{\pgfqpoint{8.330701in}{5.607165in}}%
\pgfpathcurveto{\pgfqpoint{8.341751in}{5.607165in}}{\pgfqpoint{8.352350in}{5.611556in}}{\pgfqpoint{8.360163in}{5.619369in}}%
\pgfpathcurveto{\pgfqpoint{8.367977in}{5.627183in}}{\pgfqpoint{8.372367in}{5.637782in}}{\pgfqpoint{8.372367in}{5.648832in}}%
\pgfpathcurveto{\pgfqpoint{8.372367in}{5.659882in}}{\pgfqpoint{8.367977in}{5.670481in}}{\pgfqpoint{8.360163in}{5.678295in}}%
\pgfpathcurveto{\pgfqpoint{8.352350in}{5.686109in}}{\pgfqpoint{8.341751in}{5.690499in}}{\pgfqpoint{8.330701in}{5.690499in}}%
\pgfpathcurveto{\pgfqpoint{8.319651in}{5.690499in}}{\pgfqpoint{8.309052in}{5.686109in}}{\pgfqpoint{8.301238in}{5.678295in}}%
\pgfpathcurveto{\pgfqpoint{8.293424in}{5.670481in}}{\pgfqpoint{8.289034in}{5.659882in}}{\pgfqpoint{8.289034in}{5.648832in}}%
\pgfpathcurveto{\pgfqpoint{8.289034in}{5.637782in}}{\pgfqpoint{8.293424in}{5.627183in}}{\pgfqpoint{8.301238in}{5.619369in}}%
\pgfpathcurveto{\pgfqpoint{8.309052in}{5.611556in}}{\pgfqpoint{8.319651in}{5.607165in}}{\pgfqpoint{8.330701in}{5.607165in}}%
\pgfpathclose%
\pgfusepath{stroke,fill}%
\end{pgfscope}%
\begin{pgfscope}%
\pgfpathrectangle{\pgfqpoint{0.481978in}{0.331635in}}{\pgfqpoint{9.300000in}{7.700000in}}%
\pgfusepath{clip}%
\pgfsetbuttcap%
\pgfsetroundjoin%
\definecolor{currentfill}{rgb}{0.815686,0.733333,1.000000}%
\pgfsetfillcolor{currentfill}%
\pgfsetlinewidth{0.481800pt}%
\definecolor{currentstroke}{rgb}{1.000000,1.000000,1.000000}%
\pgfsetstrokecolor{currentstroke}%
\pgfsetdash{}{0pt}%
\pgfpathmoveto{\pgfqpoint{7.364685in}{2.898601in}}%
\pgfpathcurveto{\pgfqpoint{7.375735in}{2.898601in}}{\pgfqpoint{7.386334in}{2.902991in}}{\pgfqpoint{7.394148in}{2.910805in}}%
\pgfpathcurveto{\pgfqpoint{7.401961in}{2.918618in}}{\pgfqpoint{7.406351in}{2.929217in}}{\pgfqpoint{7.406351in}{2.940268in}}%
\pgfpathcurveto{\pgfqpoint{7.406351in}{2.951318in}}{\pgfqpoint{7.401961in}{2.961917in}}{\pgfqpoint{7.394148in}{2.969730in}}%
\pgfpathcurveto{\pgfqpoint{7.386334in}{2.977544in}}{\pgfqpoint{7.375735in}{2.981934in}}{\pgfqpoint{7.364685in}{2.981934in}}%
\pgfpathcurveto{\pgfqpoint{7.353635in}{2.981934in}}{\pgfqpoint{7.343036in}{2.977544in}}{\pgfqpoint{7.335222in}{2.969730in}}%
\pgfpathcurveto{\pgfqpoint{7.327408in}{2.961917in}}{\pgfqpoint{7.323018in}{2.951318in}}{\pgfqpoint{7.323018in}{2.940268in}}%
\pgfpathcurveto{\pgfqpoint{7.323018in}{2.929217in}}{\pgfqpoint{7.327408in}{2.918618in}}{\pgfqpoint{7.335222in}{2.910805in}}%
\pgfpathcurveto{\pgfqpoint{7.343036in}{2.902991in}}{\pgfqpoint{7.353635in}{2.898601in}}{\pgfqpoint{7.364685in}{2.898601in}}%
\pgfpathclose%
\pgfusepath{stroke,fill}%
\end{pgfscope}%
\begin{pgfscope}%
\pgfpathrectangle{\pgfqpoint{0.481978in}{0.331635in}}{\pgfqpoint{9.300000in}{7.700000in}}%
\pgfusepath{clip}%
\pgfsetbuttcap%
\pgfsetroundjoin%
\definecolor{currentfill}{rgb}{0.815686,0.733333,1.000000}%
\pgfsetfillcolor{currentfill}%
\pgfsetlinewidth{0.481800pt}%
\definecolor{currentstroke}{rgb}{1.000000,1.000000,1.000000}%
\pgfsetstrokecolor{currentstroke}%
\pgfsetdash{}{0pt}%
\pgfpathmoveto{\pgfqpoint{2.332471in}{2.474830in}}%
\pgfpathcurveto{\pgfqpoint{2.343521in}{2.474830in}}{\pgfqpoint{2.354120in}{2.479221in}}{\pgfqpoint{2.361934in}{2.487034in}}%
\pgfpathcurveto{\pgfqpoint{2.369747in}{2.494848in}}{\pgfqpoint{2.374138in}{2.505447in}}{\pgfqpoint{2.374138in}{2.516497in}}%
\pgfpathcurveto{\pgfqpoint{2.374138in}{2.527547in}}{\pgfqpoint{2.369747in}{2.538146in}}{\pgfqpoint{2.361934in}{2.545960in}}%
\pgfpathcurveto{\pgfqpoint{2.354120in}{2.553773in}}{\pgfqpoint{2.343521in}{2.558164in}}{\pgfqpoint{2.332471in}{2.558164in}}%
\pgfpathcurveto{\pgfqpoint{2.321421in}{2.558164in}}{\pgfqpoint{2.310822in}{2.553773in}}{\pgfqpoint{2.303008in}{2.545960in}}%
\pgfpathcurveto{\pgfqpoint{2.295195in}{2.538146in}}{\pgfqpoint{2.290804in}{2.527547in}}{\pgfqpoint{2.290804in}{2.516497in}}%
\pgfpathcurveto{\pgfqpoint{2.290804in}{2.505447in}}{\pgfqpoint{2.295195in}{2.494848in}}{\pgfqpoint{2.303008in}{2.487034in}}%
\pgfpathcurveto{\pgfqpoint{2.310822in}{2.479221in}}{\pgfqpoint{2.321421in}{2.474830in}}{\pgfqpoint{2.332471in}{2.474830in}}%
\pgfpathclose%
\pgfusepath{stroke,fill}%
\end{pgfscope}%
\begin{pgfscope}%
\pgfpathrectangle{\pgfqpoint{0.481978in}{0.331635in}}{\pgfqpoint{9.300000in}{7.700000in}}%
\pgfusepath{clip}%
\pgfsetbuttcap%
\pgfsetroundjoin%
\definecolor{currentfill}{rgb}{0.815686,0.733333,1.000000}%
\pgfsetfillcolor{currentfill}%
\pgfsetlinewidth{0.481800pt}%
\definecolor{currentstroke}{rgb}{1.000000,1.000000,1.000000}%
\pgfsetstrokecolor{currentstroke}%
\pgfsetdash{}{0pt}%
\pgfpathmoveto{\pgfqpoint{6.392422in}{4.189674in}}%
\pgfpathcurveto{\pgfqpoint{6.403472in}{4.189674in}}{\pgfqpoint{6.414071in}{4.194065in}}{\pgfqpoint{6.421884in}{4.201878in}}%
\pgfpathcurveto{\pgfqpoint{6.429698in}{4.209692in}}{\pgfqpoint{6.434088in}{4.220291in}}{\pgfqpoint{6.434088in}{4.231341in}}%
\pgfpathcurveto{\pgfqpoint{6.434088in}{4.242391in}}{\pgfqpoint{6.429698in}{4.252990in}}{\pgfqpoint{6.421884in}{4.260804in}}%
\pgfpathcurveto{\pgfqpoint{6.414071in}{4.268617in}}{\pgfqpoint{6.403472in}{4.273008in}}{\pgfqpoint{6.392422in}{4.273008in}}%
\pgfpathcurveto{\pgfqpoint{6.381371in}{4.273008in}}{\pgfqpoint{6.370772in}{4.268617in}}{\pgfqpoint{6.362959in}{4.260804in}}%
\pgfpathcurveto{\pgfqpoint{6.355145in}{4.252990in}}{\pgfqpoint{6.350755in}{4.242391in}}{\pgfqpoint{6.350755in}{4.231341in}}%
\pgfpathcurveto{\pgfqpoint{6.350755in}{4.220291in}}{\pgfqpoint{6.355145in}{4.209692in}}{\pgfqpoint{6.362959in}{4.201878in}}%
\pgfpathcurveto{\pgfqpoint{6.370772in}{4.194065in}}{\pgfqpoint{6.381371in}{4.189674in}}{\pgfqpoint{6.392422in}{4.189674in}}%
\pgfpathclose%
\pgfusepath{stroke,fill}%
\end{pgfscope}%
\begin{pgfscope}%
\pgfpathrectangle{\pgfqpoint{0.481978in}{0.331635in}}{\pgfqpoint{9.300000in}{7.700000in}}%
\pgfusepath{clip}%
\pgfsetbuttcap%
\pgfsetroundjoin%
\definecolor{currentfill}{rgb}{0.870588,0.733333,0.607843}%
\pgfsetfillcolor{currentfill}%
\pgfsetlinewidth{0.481800pt}%
\definecolor{currentstroke}{rgb}{1.000000,1.000000,1.000000}%
\pgfsetstrokecolor{currentstroke}%
\pgfsetdash{}{0pt}%
\pgfpathmoveto{\pgfqpoint{3.278906in}{4.335174in}}%
\pgfpathcurveto{\pgfqpoint{3.289956in}{4.335174in}}{\pgfqpoint{3.300555in}{4.339564in}}{\pgfqpoint{3.308368in}{4.347377in}}%
\pgfpathcurveto{\pgfqpoint{3.316182in}{4.355191in}}{\pgfqpoint{3.320572in}{4.365790in}}{\pgfqpoint{3.320572in}{4.376840in}}%
\pgfpathcurveto{\pgfqpoint{3.320572in}{4.387890in}}{\pgfqpoint{3.316182in}{4.398489in}}{\pgfqpoint{3.308368in}{4.406303in}}%
\pgfpathcurveto{\pgfqpoint{3.300555in}{4.414117in}}{\pgfqpoint{3.289956in}{4.418507in}}{\pgfqpoint{3.278906in}{4.418507in}}%
\pgfpathcurveto{\pgfqpoint{3.267856in}{4.418507in}}{\pgfqpoint{3.257257in}{4.414117in}}{\pgfqpoint{3.249443in}{4.406303in}}%
\pgfpathcurveto{\pgfqpoint{3.241629in}{4.398489in}}{\pgfqpoint{3.237239in}{4.387890in}}{\pgfqpoint{3.237239in}{4.376840in}}%
\pgfpathcurveto{\pgfqpoint{3.237239in}{4.365790in}}{\pgfqpoint{3.241629in}{4.355191in}}{\pgfqpoint{3.249443in}{4.347377in}}%
\pgfpathcurveto{\pgfqpoint{3.257257in}{4.339564in}}{\pgfqpoint{3.267856in}{4.335174in}}{\pgfqpoint{3.278906in}{4.335174in}}%
\pgfpathclose%
\pgfusepath{stroke,fill}%
\end{pgfscope}%
\begin{pgfscope}%
\pgfpathrectangle{\pgfqpoint{0.481978in}{0.331635in}}{\pgfqpoint{9.300000in}{7.700000in}}%
\pgfusepath{clip}%
\pgfsetbuttcap%
\pgfsetroundjoin%
\definecolor{currentfill}{rgb}{0.870588,0.733333,0.607843}%
\pgfsetfillcolor{currentfill}%
\pgfsetlinewidth{0.481800pt}%
\definecolor{currentstroke}{rgb}{1.000000,1.000000,1.000000}%
\pgfsetstrokecolor{currentstroke}%
\pgfsetdash{}{0pt}%
\pgfpathmoveto{\pgfqpoint{1.806406in}{3.472401in}}%
\pgfpathcurveto{\pgfqpoint{1.817456in}{3.472401in}}{\pgfqpoint{1.828055in}{3.476791in}}{\pgfqpoint{1.835868in}{3.484605in}}%
\pgfpathcurveto{\pgfqpoint{1.843682in}{3.492418in}}{\pgfqpoint{1.848072in}{3.503017in}}{\pgfqpoint{1.848072in}{3.514068in}}%
\pgfpathcurveto{\pgfqpoint{1.848072in}{3.525118in}}{\pgfqpoint{1.843682in}{3.535717in}}{\pgfqpoint{1.835868in}{3.543530in}}%
\pgfpathcurveto{\pgfqpoint{1.828055in}{3.551344in}}{\pgfqpoint{1.817456in}{3.555734in}}{\pgfqpoint{1.806406in}{3.555734in}}%
\pgfpathcurveto{\pgfqpoint{1.795355in}{3.555734in}}{\pgfqpoint{1.784756in}{3.551344in}}{\pgfqpoint{1.776943in}{3.543530in}}%
\pgfpathcurveto{\pgfqpoint{1.769129in}{3.535717in}}{\pgfqpoint{1.764739in}{3.525118in}}{\pgfqpoint{1.764739in}{3.514068in}}%
\pgfpathcurveto{\pgfqpoint{1.764739in}{3.503017in}}{\pgfqpoint{1.769129in}{3.492418in}}{\pgfqpoint{1.776943in}{3.484605in}}%
\pgfpathcurveto{\pgfqpoint{1.784756in}{3.476791in}}{\pgfqpoint{1.795355in}{3.472401in}}{\pgfqpoint{1.806406in}{3.472401in}}%
\pgfpathclose%
\pgfusepath{stroke,fill}%
\end{pgfscope}%
\begin{pgfscope}%
\pgfpathrectangle{\pgfqpoint{0.481978in}{0.331635in}}{\pgfqpoint{9.300000in}{7.700000in}}%
\pgfusepath{clip}%
\pgfsetbuttcap%
\pgfsetroundjoin%
\definecolor{currentfill}{rgb}{0.870588,0.733333,0.607843}%
\pgfsetfillcolor{currentfill}%
\pgfsetlinewidth{0.481800pt}%
\definecolor{currentstroke}{rgb}{1.000000,1.000000,1.000000}%
\pgfsetstrokecolor{currentstroke}%
\pgfsetdash{}{0pt}%
\pgfpathmoveto{\pgfqpoint{3.901522in}{2.710673in}}%
\pgfpathcurveto{\pgfqpoint{3.912572in}{2.710673in}}{\pgfqpoint{3.923171in}{2.715064in}}{\pgfqpoint{3.930984in}{2.722877in}}%
\pgfpathcurveto{\pgfqpoint{3.938798in}{2.730691in}}{\pgfqpoint{3.943188in}{2.741290in}}{\pgfqpoint{3.943188in}{2.752340in}}%
\pgfpathcurveto{\pgfqpoint{3.943188in}{2.763390in}}{\pgfqpoint{3.938798in}{2.773989in}}{\pgfqpoint{3.930984in}{2.781803in}}%
\pgfpathcurveto{\pgfqpoint{3.923171in}{2.789616in}}{\pgfqpoint{3.912572in}{2.794007in}}{\pgfqpoint{3.901522in}{2.794007in}}%
\pgfpathcurveto{\pgfqpoint{3.890472in}{2.794007in}}{\pgfqpoint{3.879873in}{2.789616in}}{\pgfqpoint{3.872059in}{2.781803in}}%
\pgfpathcurveto{\pgfqpoint{3.864245in}{2.773989in}}{\pgfqpoint{3.859855in}{2.763390in}}{\pgfqpoint{3.859855in}{2.752340in}}%
\pgfpathcurveto{\pgfqpoint{3.859855in}{2.741290in}}{\pgfqpoint{3.864245in}{2.730691in}}{\pgfqpoint{3.872059in}{2.722877in}}%
\pgfpathcurveto{\pgfqpoint{3.879873in}{2.715064in}}{\pgfqpoint{3.890472in}{2.710673in}}{\pgfqpoint{3.901522in}{2.710673in}}%
\pgfpathclose%
\pgfusepath{stroke,fill}%
\end{pgfscope}%
\begin{pgfscope}%
\pgfpathrectangle{\pgfqpoint{0.481978in}{0.331635in}}{\pgfqpoint{9.300000in}{7.700000in}}%
\pgfusepath{clip}%
\pgfsetbuttcap%
\pgfsetroundjoin%
\definecolor{currentfill}{rgb}{0.870588,0.733333,0.607843}%
\pgfsetfillcolor{currentfill}%
\pgfsetlinewidth{0.481800pt}%
\definecolor{currentstroke}{rgb}{1.000000,1.000000,1.000000}%
\pgfsetstrokecolor{currentstroke}%
\pgfsetdash{}{0pt}%
\pgfpathmoveto{\pgfqpoint{2.577647in}{1.919824in}}%
\pgfpathcurveto{\pgfqpoint{2.588697in}{1.919824in}}{\pgfqpoint{2.599296in}{1.924214in}}{\pgfqpoint{2.607110in}{1.932027in}}%
\pgfpathcurveto{\pgfqpoint{2.614923in}{1.939841in}}{\pgfqpoint{2.619314in}{1.950440in}}{\pgfqpoint{2.619314in}{1.961490in}}%
\pgfpathcurveto{\pgfqpoint{2.619314in}{1.972540in}}{\pgfqpoint{2.614923in}{1.983139in}}{\pgfqpoint{2.607110in}{1.990953in}}%
\pgfpathcurveto{\pgfqpoint{2.599296in}{1.998767in}}{\pgfqpoint{2.588697in}{2.003157in}}{\pgfqpoint{2.577647in}{2.003157in}}%
\pgfpathcurveto{\pgfqpoint{2.566597in}{2.003157in}}{\pgfqpoint{2.555998in}{1.998767in}}{\pgfqpoint{2.548184in}{1.990953in}}%
\pgfpathcurveto{\pgfqpoint{2.540371in}{1.983139in}}{\pgfqpoint{2.535980in}{1.972540in}}{\pgfqpoint{2.535980in}{1.961490in}}%
\pgfpathcurveto{\pgfqpoint{2.535980in}{1.950440in}}{\pgfqpoint{2.540371in}{1.939841in}}{\pgfqpoint{2.548184in}{1.932027in}}%
\pgfpathcurveto{\pgfqpoint{2.555998in}{1.924214in}}{\pgfqpoint{2.566597in}{1.919824in}}{\pgfqpoint{2.577647in}{1.919824in}}%
\pgfpathclose%
\pgfusepath{stroke,fill}%
\end{pgfscope}%
\begin{pgfscope}%
\pgfpathrectangle{\pgfqpoint{0.481978in}{0.331635in}}{\pgfqpoint{9.300000in}{7.700000in}}%
\pgfusepath{clip}%
\pgfsetbuttcap%
\pgfsetroundjoin%
\definecolor{currentfill}{rgb}{0.870588,0.733333,0.607843}%
\pgfsetfillcolor{currentfill}%
\pgfsetlinewidth{0.481800pt}%
\definecolor{currentstroke}{rgb}{1.000000,1.000000,1.000000}%
\pgfsetstrokecolor{currentstroke}%
\pgfsetdash{}{0pt}%
\pgfpathmoveto{\pgfqpoint{2.542882in}{4.387959in}}%
\pgfpathcurveto{\pgfqpoint{2.553932in}{4.387959in}}{\pgfqpoint{2.564531in}{4.392349in}}{\pgfqpoint{2.572344in}{4.400162in}}%
\pgfpathcurveto{\pgfqpoint{2.580158in}{4.407976in}}{\pgfqpoint{2.584548in}{4.418575in}}{\pgfqpoint{2.584548in}{4.429625in}}%
\pgfpathcurveto{\pgfqpoint{2.584548in}{4.440675in}}{\pgfqpoint{2.580158in}{4.451274in}}{\pgfqpoint{2.572344in}{4.459088in}}%
\pgfpathcurveto{\pgfqpoint{2.564531in}{4.466902in}}{\pgfqpoint{2.553932in}{4.471292in}}{\pgfqpoint{2.542882in}{4.471292in}}%
\pgfpathcurveto{\pgfqpoint{2.531831in}{4.471292in}}{\pgfqpoint{2.521232in}{4.466902in}}{\pgfqpoint{2.513419in}{4.459088in}}%
\pgfpathcurveto{\pgfqpoint{2.505605in}{4.451274in}}{\pgfqpoint{2.501215in}{4.440675in}}{\pgfqpoint{2.501215in}{4.429625in}}%
\pgfpathcurveto{\pgfqpoint{2.501215in}{4.418575in}}{\pgfqpoint{2.505605in}{4.407976in}}{\pgfqpoint{2.513419in}{4.400162in}}%
\pgfpathcurveto{\pgfqpoint{2.521232in}{4.392349in}}{\pgfqpoint{2.531831in}{4.387959in}}{\pgfqpoint{2.542882in}{4.387959in}}%
\pgfpathclose%
\pgfusepath{stroke,fill}%
\end{pgfscope}%
\begin{pgfscope}%
\pgfpathrectangle{\pgfqpoint{0.481978in}{0.331635in}}{\pgfqpoint{9.300000in}{7.700000in}}%
\pgfusepath{clip}%
\pgfsetbuttcap%
\pgfsetroundjoin%
\definecolor{currentfill}{rgb}{0.870588,0.733333,0.607843}%
\pgfsetfillcolor{currentfill}%
\pgfsetlinewidth{0.481800pt}%
\definecolor{currentstroke}{rgb}{1.000000,1.000000,1.000000}%
\pgfsetstrokecolor{currentstroke}%
\pgfsetdash{}{0pt}%
\pgfpathmoveto{\pgfqpoint{6.100140in}{5.036268in}}%
\pgfpathcurveto{\pgfqpoint{6.111190in}{5.036268in}}{\pgfqpoint{6.121789in}{5.040658in}}{\pgfqpoint{6.129603in}{5.048472in}}%
\pgfpathcurveto{\pgfqpoint{6.137416in}{5.056285in}}{\pgfqpoint{6.141806in}{5.066884in}}{\pgfqpoint{6.141806in}{5.077934in}}%
\pgfpathcurveto{\pgfqpoint{6.141806in}{5.088984in}}{\pgfqpoint{6.137416in}{5.099583in}}{\pgfqpoint{6.129603in}{5.107397in}}%
\pgfpathcurveto{\pgfqpoint{6.121789in}{5.115211in}}{\pgfqpoint{6.111190in}{5.119601in}}{\pgfqpoint{6.100140in}{5.119601in}}%
\pgfpathcurveto{\pgfqpoint{6.089090in}{5.119601in}}{\pgfqpoint{6.078491in}{5.115211in}}{\pgfqpoint{6.070677in}{5.107397in}}%
\pgfpathcurveto{\pgfqpoint{6.062863in}{5.099583in}}{\pgfqpoint{6.058473in}{5.088984in}}{\pgfqpoint{6.058473in}{5.077934in}}%
\pgfpathcurveto{\pgfqpoint{6.058473in}{5.066884in}}{\pgfqpoint{6.062863in}{5.056285in}}{\pgfqpoint{6.070677in}{5.048472in}}%
\pgfpathcurveto{\pgfqpoint{6.078491in}{5.040658in}}{\pgfqpoint{6.089090in}{5.036268in}}{\pgfqpoint{6.100140in}{5.036268in}}%
\pgfpathclose%
\pgfusepath{stroke,fill}%
\end{pgfscope}%
\begin{pgfscope}%
\pgfpathrectangle{\pgfqpoint{0.481978in}{0.331635in}}{\pgfqpoint{9.300000in}{7.700000in}}%
\pgfusepath{clip}%
\pgfsetbuttcap%
\pgfsetroundjoin%
\definecolor{currentfill}{rgb}{0.870588,0.733333,0.607843}%
\pgfsetfillcolor{currentfill}%
\pgfsetlinewidth{0.481800pt}%
\definecolor{currentstroke}{rgb}{1.000000,1.000000,1.000000}%
\pgfsetstrokecolor{currentstroke}%
\pgfsetdash{}{0pt}%
\pgfpathmoveto{\pgfqpoint{4.111074in}{4.530151in}}%
\pgfpathcurveto{\pgfqpoint{4.122124in}{4.530151in}}{\pgfqpoint{4.132723in}{4.534541in}}{\pgfqpoint{4.140537in}{4.542355in}}%
\pgfpathcurveto{\pgfqpoint{4.148351in}{4.550168in}}{\pgfqpoint{4.152741in}{4.560767in}}{\pgfqpoint{4.152741in}{4.571818in}}%
\pgfpathcurveto{\pgfqpoint{4.152741in}{4.582868in}}{\pgfqpoint{4.148351in}{4.593467in}}{\pgfqpoint{4.140537in}{4.601280in}}%
\pgfpathcurveto{\pgfqpoint{4.132723in}{4.609094in}}{\pgfqpoint{4.122124in}{4.613484in}}{\pgfqpoint{4.111074in}{4.613484in}}%
\pgfpathcurveto{\pgfqpoint{4.100024in}{4.613484in}}{\pgfqpoint{4.089425in}{4.609094in}}{\pgfqpoint{4.081611in}{4.601280in}}%
\pgfpathcurveto{\pgfqpoint{4.073798in}{4.593467in}}{\pgfqpoint{4.069408in}{4.582868in}}{\pgfqpoint{4.069408in}{4.571818in}}%
\pgfpathcurveto{\pgfqpoint{4.069408in}{4.560767in}}{\pgfqpoint{4.073798in}{4.550168in}}{\pgfqpoint{4.081611in}{4.542355in}}%
\pgfpathcurveto{\pgfqpoint{4.089425in}{4.534541in}}{\pgfqpoint{4.100024in}{4.530151in}}{\pgfqpoint{4.111074in}{4.530151in}}%
\pgfpathclose%
\pgfusepath{stroke,fill}%
\end{pgfscope}%
\begin{pgfscope}%
\pgfpathrectangle{\pgfqpoint{0.481978in}{0.331635in}}{\pgfqpoint{9.300000in}{7.700000in}}%
\pgfusepath{clip}%
\pgfsetbuttcap%
\pgfsetroundjoin%
\definecolor{currentfill}{rgb}{0.870588,0.733333,0.607843}%
\pgfsetfillcolor{currentfill}%
\pgfsetlinewidth{0.481800pt}%
\definecolor{currentstroke}{rgb}{1.000000,1.000000,1.000000}%
\pgfsetstrokecolor{currentstroke}%
\pgfsetdash{}{0pt}%
\pgfpathmoveto{\pgfqpoint{2.567413in}{4.041877in}}%
\pgfpathcurveto{\pgfqpoint{2.578463in}{4.041877in}}{\pgfqpoint{2.589062in}{4.046267in}}{\pgfqpoint{2.596875in}{4.054081in}}%
\pgfpathcurveto{\pgfqpoint{2.604689in}{4.061894in}}{\pgfqpoint{2.609079in}{4.072493in}}{\pgfqpoint{2.609079in}{4.083544in}}%
\pgfpathcurveto{\pgfqpoint{2.609079in}{4.094594in}}{\pgfqpoint{2.604689in}{4.105193in}}{\pgfqpoint{2.596875in}{4.113006in}}%
\pgfpathcurveto{\pgfqpoint{2.589062in}{4.120820in}}{\pgfqpoint{2.578463in}{4.125210in}}{\pgfqpoint{2.567413in}{4.125210in}}%
\pgfpathcurveto{\pgfqpoint{2.556363in}{4.125210in}}{\pgfqpoint{2.545763in}{4.120820in}}{\pgfqpoint{2.537950in}{4.113006in}}%
\pgfpathcurveto{\pgfqpoint{2.530136in}{4.105193in}}{\pgfqpoint{2.525746in}{4.094594in}}{\pgfqpoint{2.525746in}{4.083544in}}%
\pgfpathcurveto{\pgfqpoint{2.525746in}{4.072493in}}{\pgfqpoint{2.530136in}{4.061894in}}{\pgfqpoint{2.537950in}{4.054081in}}%
\pgfpathcurveto{\pgfqpoint{2.545763in}{4.046267in}}{\pgfqpoint{2.556363in}{4.041877in}}{\pgfqpoint{2.567413in}{4.041877in}}%
\pgfpathclose%
\pgfusepath{stroke,fill}%
\end{pgfscope}%
\begin{pgfscope}%
\pgfpathrectangle{\pgfqpoint{0.481978in}{0.331635in}}{\pgfqpoint{9.300000in}{7.700000in}}%
\pgfusepath{clip}%
\pgfsetbuttcap%
\pgfsetroundjoin%
\definecolor{currentfill}{rgb}{0.870588,0.733333,0.607843}%
\pgfsetfillcolor{currentfill}%
\pgfsetlinewidth{0.481800pt}%
\definecolor{currentstroke}{rgb}{1.000000,1.000000,1.000000}%
\pgfsetstrokecolor{currentstroke}%
\pgfsetdash{}{0pt}%
\pgfpathmoveto{\pgfqpoint{2.475746in}{3.962700in}}%
\pgfpathcurveto{\pgfqpoint{2.486796in}{3.962700in}}{\pgfqpoint{2.497395in}{3.967090in}}{\pgfqpoint{2.505209in}{3.974904in}}%
\pgfpathcurveto{\pgfqpoint{2.513022in}{3.982717in}}{\pgfqpoint{2.517413in}{3.993316in}}{\pgfqpoint{2.517413in}{4.004366in}}%
\pgfpathcurveto{\pgfqpoint{2.517413in}{4.015417in}}{\pgfqpoint{2.513022in}{4.026016in}}{\pgfqpoint{2.505209in}{4.033829in}}%
\pgfpathcurveto{\pgfqpoint{2.497395in}{4.041643in}}{\pgfqpoint{2.486796in}{4.046033in}}{\pgfqpoint{2.475746in}{4.046033in}}%
\pgfpathcurveto{\pgfqpoint{2.464696in}{4.046033in}}{\pgfqpoint{2.454097in}{4.041643in}}{\pgfqpoint{2.446283in}{4.033829in}}%
\pgfpathcurveto{\pgfqpoint{2.438470in}{4.026016in}}{\pgfqpoint{2.434079in}{4.015417in}}{\pgfqpoint{2.434079in}{4.004366in}}%
\pgfpathcurveto{\pgfqpoint{2.434079in}{3.993316in}}{\pgfqpoint{2.438470in}{3.982717in}}{\pgfqpoint{2.446283in}{3.974904in}}%
\pgfpathcurveto{\pgfqpoint{2.454097in}{3.967090in}}{\pgfqpoint{2.464696in}{3.962700in}}{\pgfqpoint{2.475746in}{3.962700in}}%
\pgfpathclose%
\pgfusepath{stroke,fill}%
\end{pgfscope}%
\begin{pgfscope}%
\pgfpathrectangle{\pgfqpoint{0.481978in}{0.331635in}}{\pgfqpoint{9.300000in}{7.700000in}}%
\pgfusepath{clip}%
\pgfsetbuttcap%
\pgfsetroundjoin%
\definecolor{currentfill}{rgb}{0.870588,0.733333,0.607843}%
\pgfsetfillcolor{currentfill}%
\pgfsetlinewidth{0.481800pt}%
\definecolor{currentstroke}{rgb}{1.000000,1.000000,1.000000}%
\pgfsetstrokecolor{currentstroke}%
\pgfsetdash{}{0pt}%
\pgfpathmoveto{\pgfqpoint{7.321265in}{5.989130in}}%
\pgfpathcurveto{\pgfqpoint{7.332315in}{5.989130in}}{\pgfqpoint{7.342914in}{5.993521in}}{\pgfqpoint{7.350728in}{6.001334in}}%
\pgfpathcurveto{\pgfqpoint{7.358541in}{6.009148in}}{\pgfqpoint{7.362932in}{6.019747in}}{\pgfqpoint{7.362932in}{6.030797in}}%
\pgfpathcurveto{\pgfqpoint{7.362932in}{6.041847in}}{\pgfqpoint{7.358541in}{6.052446in}}{\pgfqpoint{7.350728in}{6.060260in}}%
\pgfpathcurveto{\pgfqpoint{7.342914in}{6.068073in}}{\pgfqpoint{7.332315in}{6.072464in}}{\pgfqpoint{7.321265in}{6.072464in}}%
\pgfpathcurveto{\pgfqpoint{7.310215in}{6.072464in}}{\pgfqpoint{7.299616in}{6.068073in}}{\pgfqpoint{7.291802in}{6.060260in}}%
\pgfpathcurveto{\pgfqpoint{7.283988in}{6.052446in}}{\pgfqpoint{7.279598in}{6.041847in}}{\pgfqpoint{7.279598in}{6.030797in}}%
\pgfpathcurveto{\pgfqpoint{7.279598in}{6.019747in}}{\pgfqpoint{7.283988in}{6.009148in}}{\pgfqpoint{7.291802in}{6.001334in}}%
\pgfpathcurveto{\pgfqpoint{7.299616in}{5.993521in}}{\pgfqpoint{7.310215in}{5.989130in}}{\pgfqpoint{7.321265in}{5.989130in}}%
\pgfpathclose%
\pgfusepath{stroke,fill}%
\end{pgfscope}%
\begin{pgfscope}%
\pgfpathrectangle{\pgfqpoint{0.481978in}{0.331635in}}{\pgfqpoint{9.300000in}{7.700000in}}%
\pgfusepath{clip}%
\pgfsetbuttcap%
\pgfsetroundjoin%
\definecolor{currentfill}{rgb}{0.870588,0.733333,0.607843}%
\pgfsetfillcolor{currentfill}%
\pgfsetlinewidth{0.481800pt}%
\definecolor{currentstroke}{rgb}{1.000000,1.000000,1.000000}%
\pgfsetstrokecolor{currentstroke}%
\pgfsetdash{}{0pt}%
\pgfpathmoveto{\pgfqpoint{1.982524in}{5.164596in}}%
\pgfpathcurveto{\pgfqpoint{1.993574in}{5.164596in}}{\pgfqpoint{2.004173in}{5.168986in}}{\pgfqpoint{2.011987in}{5.176800in}}%
\pgfpathcurveto{\pgfqpoint{2.019800in}{5.184613in}}{\pgfqpoint{2.024191in}{5.195212in}}{\pgfqpoint{2.024191in}{5.206262in}}%
\pgfpathcurveto{\pgfqpoint{2.024191in}{5.217313in}}{\pgfqpoint{2.019800in}{5.227912in}}{\pgfqpoint{2.011987in}{5.235725in}}%
\pgfpathcurveto{\pgfqpoint{2.004173in}{5.243539in}}{\pgfqpoint{1.993574in}{5.247929in}}{\pgfqpoint{1.982524in}{5.247929in}}%
\pgfpathcurveto{\pgfqpoint{1.971474in}{5.247929in}}{\pgfqpoint{1.960875in}{5.243539in}}{\pgfqpoint{1.953061in}{5.235725in}}%
\pgfpathcurveto{\pgfqpoint{1.945248in}{5.227912in}}{\pgfqpoint{1.940857in}{5.217313in}}{\pgfqpoint{1.940857in}{5.206262in}}%
\pgfpathcurveto{\pgfqpoint{1.940857in}{5.195212in}}{\pgfqpoint{1.945248in}{5.184613in}}{\pgfqpoint{1.953061in}{5.176800in}}%
\pgfpathcurveto{\pgfqpoint{1.960875in}{5.168986in}}{\pgfqpoint{1.971474in}{5.164596in}}{\pgfqpoint{1.982524in}{5.164596in}}%
\pgfpathclose%
\pgfusepath{stroke,fill}%
\end{pgfscope}%
\begin{pgfscope}%
\pgfpathrectangle{\pgfqpoint{0.481978in}{0.331635in}}{\pgfqpoint{9.300000in}{7.700000in}}%
\pgfusepath{clip}%
\pgfsetbuttcap%
\pgfsetroundjoin%
\definecolor{currentfill}{rgb}{0.870588,0.733333,0.607843}%
\pgfsetfillcolor{currentfill}%
\pgfsetlinewidth{0.481800pt}%
\definecolor{currentstroke}{rgb}{1.000000,1.000000,1.000000}%
\pgfsetstrokecolor{currentstroke}%
\pgfsetdash{}{0pt}%
\pgfpathmoveto{\pgfqpoint{2.801903in}{4.359870in}}%
\pgfpathcurveto{\pgfqpoint{2.812953in}{4.359870in}}{\pgfqpoint{2.823552in}{4.364261in}}{\pgfqpoint{2.831365in}{4.372074in}}%
\pgfpathcurveto{\pgfqpoint{2.839179in}{4.379888in}}{\pgfqpoint{2.843569in}{4.390487in}}{\pgfqpoint{2.843569in}{4.401537in}}%
\pgfpathcurveto{\pgfqpoint{2.843569in}{4.412587in}}{\pgfqpoint{2.839179in}{4.423186in}}{\pgfqpoint{2.831365in}{4.431000in}}%
\pgfpathcurveto{\pgfqpoint{2.823552in}{4.438813in}}{\pgfqpoint{2.812953in}{4.443204in}}{\pgfqpoint{2.801903in}{4.443204in}}%
\pgfpathcurveto{\pgfqpoint{2.790852in}{4.443204in}}{\pgfqpoint{2.780253in}{4.438813in}}{\pgfqpoint{2.772440in}{4.431000in}}%
\pgfpathcurveto{\pgfqpoint{2.764626in}{4.423186in}}{\pgfqpoint{2.760236in}{4.412587in}}{\pgfqpoint{2.760236in}{4.401537in}}%
\pgfpathcurveto{\pgfqpoint{2.760236in}{4.390487in}}{\pgfqpoint{2.764626in}{4.379888in}}{\pgfqpoint{2.772440in}{4.372074in}}%
\pgfpathcurveto{\pgfqpoint{2.780253in}{4.364261in}}{\pgfqpoint{2.790852in}{4.359870in}}{\pgfqpoint{2.801903in}{4.359870in}}%
\pgfpathclose%
\pgfusepath{stroke,fill}%
\end{pgfscope}%
\begin{pgfscope}%
\pgfpathrectangle{\pgfqpoint{0.481978in}{0.331635in}}{\pgfqpoint{9.300000in}{7.700000in}}%
\pgfusepath{clip}%
\pgfsetbuttcap%
\pgfsetroundjoin%
\definecolor{currentfill}{rgb}{0.870588,0.733333,0.607843}%
\pgfsetfillcolor{currentfill}%
\pgfsetlinewidth{0.481800pt}%
\definecolor{currentstroke}{rgb}{1.000000,1.000000,1.000000}%
\pgfsetstrokecolor{currentstroke}%
\pgfsetdash{}{0pt}%
\pgfpathmoveto{\pgfqpoint{3.503709in}{4.271868in}}%
\pgfpathcurveto{\pgfqpoint{3.514759in}{4.271868in}}{\pgfqpoint{3.525358in}{4.276258in}}{\pgfqpoint{3.533171in}{4.284072in}}%
\pgfpathcurveto{\pgfqpoint{3.540985in}{4.291886in}}{\pgfqpoint{3.545375in}{4.302485in}}{\pgfqpoint{3.545375in}{4.313535in}}%
\pgfpathcurveto{\pgfqpoint{3.545375in}{4.324585in}}{\pgfqpoint{3.540985in}{4.335184in}}{\pgfqpoint{3.533171in}{4.342998in}}%
\pgfpathcurveto{\pgfqpoint{3.525358in}{4.350811in}}{\pgfqpoint{3.514759in}{4.355201in}}{\pgfqpoint{3.503709in}{4.355201in}}%
\pgfpathcurveto{\pgfqpoint{3.492659in}{4.355201in}}{\pgfqpoint{3.482059in}{4.350811in}}{\pgfqpoint{3.474246in}{4.342998in}}%
\pgfpathcurveto{\pgfqpoint{3.466432in}{4.335184in}}{\pgfqpoint{3.462042in}{4.324585in}}{\pgfqpoint{3.462042in}{4.313535in}}%
\pgfpathcurveto{\pgfqpoint{3.462042in}{4.302485in}}{\pgfqpoint{3.466432in}{4.291886in}}{\pgfqpoint{3.474246in}{4.284072in}}%
\pgfpathcurveto{\pgfqpoint{3.482059in}{4.276258in}}{\pgfqpoint{3.492659in}{4.271868in}}{\pgfqpoint{3.503709in}{4.271868in}}%
\pgfpathclose%
\pgfusepath{stroke,fill}%
\end{pgfscope}%
\begin{pgfscope}%
\pgfpathrectangle{\pgfqpoint{0.481978in}{0.331635in}}{\pgfqpoint{9.300000in}{7.700000in}}%
\pgfusepath{clip}%
\pgfsetbuttcap%
\pgfsetroundjoin%
\definecolor{currentfill}{rgb}{0.870588,0.733333,0.607843}%
\pgfsetfillcolor{currentfill}%
\pgfsetlinewidth{0.481800pt}%
\definecolor{currentstroke}{rgb}{1.000000,1.000000,1.000000}%
\pgfsetstrokecolor{currentstroke}%
\pgfsetdash{}{0pt}%
\pgfpathmoveto{\pgfqpoint{7.049429in}{6.014836in}}%
\pgfpathcurveto{\pgfqpoint{7.060479in}{6.014836in}}{\pgfqpoint{7.071078in}{6.019226in}}{\pgfqpoint{7.078892in}{6.027039in}}%
\pgfpathcurveto{\pgfqpoint{7.086705in}{6.034853in}}{\pgfqpoint{7.091096in}{6.045452in}}{\pgfqpoint{7.091096in}{6.056502in}}%
\pgfpathcurveto{\pgfqpoint{7.091096in}{6.067552in}}{\pgfqpoint{7.086705in}{6.078151in}}{\pgfqpoint{7.078892in}{6.085965in}}%
\pgfpathcurveto{\pgfqpoint{7.071078in}{6.093779in}}{\pgfqpoint{7.060479in}{6.098169in}}{\pgfqpoint{7.049429in}{6.098169in}}%
\pgfpathcurveto{\pgfqpoint{7.038379in}{6.098169in}}{\pgfqpoint{7.027780in}{6.093779in}}{\pgfqpoint{7.019966in}{6.085965in}}%
\pgfpathcurveto{\pgfqpoint{7.012153in}{6.078151in}}{\pgfqpoint{7.007762in}{6.067552in}}{\pgfqpoint{7.007762in}{6.056502in}}%
\pgfpathcurveto{\pgfqpoint{7.007762in}{6.045452in}}{\pgfqpoint{7.012153in}{6.034853in}}{\pgfqpoint{7.019966in}{6.027039in}}%
\pgfpathcurveto{\pgfqpoint{7.027780in}{6.019226in}}{\pgfqpoint{7.038379in}{6.014836in}}{\pgfqpoint{7.049429in}{6.014836in}}%
\pgfpathclose%
\pgfusepath{stroke,fill}%
\end{pgfscope}%
\begin{pgfscope}%
\pgfpathrectangle{\pgfqpoint{0.481978in}{0.331635in}}{\pgfqpoint{9.300000in}{7.700000in}}%
\pgfusepath{clip}%
\pgfsetbuttcap%
\pgfsetroundjoin%
\definecolor{currentfill}{rgb}{0.870588,0.733333,0.607843}%
\pgfsetfillcolor{currentfill}%
\pgfsetlinewidth{0.481800pt}%
\definecolor{currentstroke}{rgb}{1.000000,1.000000,1.000000}%
\pgfsetstrokecolor{currentstroke}%
\pgfsetdash{}{0pt}%
\pgfpathmoveto{\pgfqpoint{6.348280in}{4.805197in}}%
\pgfpathcurveto{\pgfqpoint{6.359330in}{4.805197in}}{\pgfqpoint{6.369929in}{4.809587in}}{\pgfqpoint{6.377743in}{4.817401in}}%
\pgfpathcurveto{\pgfqpoint{6.385557in}{4.825215in}}{\pgfqpoint{6.389947in}{4.835814in}}{\pgfqpoint{6.389947in}{4.846864in}}%
\pgfpathcurveto{\pgfqpoint{6.389947in}{4.857914in}}{\pgfqpoint{6.385557in}{4.868513in}}{\pgfqpoint{6.377743in}{4.876327in}}%
\pgfpathcurveto{\pgfqpoint{6.369929in}{4.884140in}}{\pgfqpoint{6.359330in}{4.888530in}}{\pgfqpoint{6.348280in}{4.888530in}}%
\pgfpathcurveto{\pgfqpoint{6.337230in}{4.888530in}}{\pgfqpoint{6.326631in}{4.884140in}}{\pgfqpoint{6.318818in}{4.876327in}}%
\pgfpathcurveto{\pgfqpoint{6.311004in}{4.868513in}}{\pgfqpoint{6.306614in}{4.857914in}}{\pgfqpoint{6.306614in}{4.846864in}}%
\pgfpathcurveto{\pgfqpoint{6.306614in}{4.835814in}}{\pgfqpoint{6.311004in}{4.825215in}}{\pgfqpoint{6.318818in}{4.817401in}}%
\pgfpathcurveto{\pgfqpoint{6.326631in}{4.809587in}}{\pgfqpoint{6.337230in}{4.805197in}}{\pgfqpoint{6.348280in}{4.805197in}}%
\pgfpathclose%
\pgfusepath{stroke,fill}%
\end{pgfscope}%
\begin{pgfscope}%
\pgfpathrectangle{\pgfqpoint{0.481978in}{0.331635in}}{\pgfqpoint{9.300000in}{7.700000in}}%
\pgfusepath{clip}%
\pgfsetbuttcap%
\pgfsetroundjoin%
\definecolor{currentfill}{rgb}{0.870588,0.733333,0.607843}%
\pgfsetfillcolor{currentfill}%
\pgfsetlinewidth{0.481800pt}%
\definecolor{currentstroke}{rgb}{1.000000,1.000000,1.000000}%
\pgfsetstrokecolor{currentstroke}%
\pgfsetdash{}{0pt}%
\pgfpathmoveto{\pgfqpoint{5.923605in}{3.640188in}}%
\pgfpathcurveto{\pgfqpoint{5.934656in}{3.640188in}}{\pgfqpoint{5.945255in}{3.644578in}}{\pgfqpoint{5.953068in}{3.652392in}}%
\pgfpathcurveto{\pgfqpoint{5.960882in}{3.660205in}}{\pgfqpoint{5.965272in}{3.670804in}}{\pgfqpoint{5.965272in}{3.681854in}}%
\pgfpathcurveto{\pgfqpoint{5.965272in}{3.692905in}}{\pgfqpoint{5.960882in}{3.703504in}}{\pgfqpoint{5.953068in}{3.711317in}}%
\pgfpathcurveto{\pgfqpoint{5.945255in}{3.719131in}}{\pgfqpoint{5.934656in}{3.723521in}}{\pgfqpoint{5.923605in}{3.723521in}}%
\pgfpathcurveto{\pgfqpoint{5.912555in}{3.723521in}}{\pgfqpoint{5.901956in}{3.719131in}}{\pgfqpoint{5.894143in}{3.711317in}}%
\pgfpathcurveto{\pgfqpoint{5.886329in}{3.703504in}}{\pgfqpoint{5.881939in}{3.692905in}}{\pgfqpoint{5.881939in}{3.681854in}}%
\pgfpathcurveto{\pgfqpoint{5.881939in}{3.670804in}}{\pgfqpoint{5.886329in}{3.660205in}}{\pgfqpoint{5.894143in}{3.652392in}}%
\pgfpathcurveto{\pgfqpoint{5.901956in}{3.644578in}}{\pgfqpoint{5.912555in}{3.640188in}}{\pgfqpoint{5.923605in}{3.640188in}}%
\pgfpathclose%
\pgfusepath{stroke,fill}%
\end{pgfscope}%
\begin{pgfscope}%
\pgfpathrectangle{\pgfqpoint{0.481978in}{0.331635in}}{\pgfqpoint{9.300000in}{7.700000in}}%
\pgfusepath{clip}%
\pgfsetbuttcap%
\pgfsetroundjoin%
\definecolor{currentfill}{rgb}{0.870588,0.733333,0.607843}%
\pgfsetfillcolor{currentfill}%
\pgfsetlinewidth{0.481800pt}%
\definecolor{currentstroke}{rgb}{1.000000,1.000000,1.000000}%
\pgfsetstrokecolor{currentstroke}%
\pgfsetdash{}{0pt}%
\pgfpathmoveto{\pgfqpoint{5.344914in}{3.994410in}}%
\pgfpathcurveto{\pgfqpoint{5.355965in}{3.994410in}}{\pgfqpoint{5.366564in}{3.998800in}}{\pgfqpoint{5.374377in}{4.006614in}}%
\pgfpathcurveto{\pgfqpoint{5.382191in}{4.014427in}}{\pgfqpoint{5.386581in}{4.025027in}}{\pgfqpoint{5.386581in}{4.036077in}}%
\pgfpathcurveto{\pgfqpoint{5.386581in}{4.047127in}}{\pgfqpoint{5.382191in}{4.057726in}}{\pgfqpoint{5.374377in}{4.065539in}}%
\pgfpathcurveto{\pgfqpoint{5.366564in}{4.073353in}}{\pgfqpoint{5.355965in}{4.077743in}}{\pgfqpoint{5.344914in}{4.077743in}}%
\pgfpathcurveto{\pgfqpoint{5.333864in}{4.077743in}}{\pgfqpoint{5.323265in}{4.073353in}}{\pgfqpoint{5.315452in}{4.065539in}}%
\pgfpathcurveto{\pgfqpoint{5.307638in}{4.057726in}}{\pgfqpoint{5.303248in}{4.047127in}}{\pgfqpoint{5.303248in}{4.036077in}}%
\pgfpathcurveto{\pgfqpoint{5.303248in}{4.025027in}}{\pgfqpoint{5.307638in}{4.014427in}}{\pgfqpoint{5.315452in}{4.006614in}}%
\pgfpathcurveto{\pgfqpoint{5.323265in}{3.998800in}}{\pgfqpoint{5.333864in}{3.994410in}}{\pgfqpoint{5.344914in}{3.994410in}}%
\pgfpathclose%
\pgfusepath{stroke,fill}%
\end{pgfscope}%
\begin{pgfscope}%
\pgfpathrectangle{\pgfqpoint{0.481978in}{0.331635in}}{\pgfqpoint{9.300000in}{7.700000in}}%
\pgfusepath{clip}%
\pgfsetbuttcap%
\pgfsetroundjoin%
\definecolor{currentfill}{rgb}{0.870588,0.733333,0.607843}%
\pgfsetfillcolor{currentfill}%
\pgfsetlinewidth{0.481800pt}%
\definecolor{currentstroke}{rgb}{1.000000,1.000000,1.000000}%
\pgfsetstrokecolor{currentstroke}%
\pgfsetdash{}{0pt}%
\pgfpathmoveto{\pgfqpoint{4.527667in}{3.702609in}}%
\pgfpathcurveto{\pgfqpoint{4.538717in}{3.702609in}}{\pgfqpoint{4.549316in}{3.706999in}}{\pgfqpoint{4.557129in}{3.714813in}}%
\pgfpathcurveto{\pgfqpoint{4.564943in}{3.722627in}}{\pgfqpoint{4.569333in}{3.733226in}}{\pgfqpoint{4.569333in}{3.744276in}}%
\pgfpathcurveto{\pgfqpoint{4.569333in}{3.755326in}}{\pgfqpoint{4.564943in}{3.765925in}}{\pgfqpoint{4.557129in}{3.773739in}}%
\pgfpathcurveto{\pgfqpoint{4.549316in}{3.781552in}}{\pgfqpoint{4.538717in}{3.785943in}}{\pgfqpoint{4.527667in}{3.785943in}}%
\pgfpathcurveto{\pgfqpoint{4.516616in}{3.785943in}}{\pgfqpoint{4.506017in}{3.781552in}}{\pgfqpoint{4.498204in}{3.773739in}}%
\pgfpathcurveto{\pgfqpoint{4.490390in}{3.765925in}}{\pgfqpoint{4.486000in}{3.755326in}}{\pgfqpoint{4.486000in}{3.744276in}}%
\pgfpathcurveto{\pgfqpoint{4.486000in}{3.733226in}}{\pgfqpoint{4.490390in}{3.722627in}}{\pgfqpoint{4.498204in}{3.714813in}}%
\pgfpathcurveto{\pgfqpoint{4.506017in}{3.706999in}}{\pgfqpoint{4.516616in}{3.702609in}}{\pgfqpoint{4.527667in}{3.702609in}}%
\pgfpathclose%
\pgfusepath{stroke,fill}%
\end{pgfscope}%
\begin{pgfscope}%
\pgfpathrectangle{\pgfqpoint{0.481978in}{0.331635in}}{\pgfqpoint{9.300000in}{7.700000in}}%
\pgfusepath{clip}%
\pgfsetbuttcap%
\pgfsetroundjoin%
\definecolor{currentfill}{rgb}{0.870588,0.733333,0.607843}%
\pgfsetfillcolor{currentfill}%
\pgfsetlinewidth{0.481800pt}%
\definecolor{currentstroke}{rgb}{1.000000,1.000000,1.000000}%
\pgfsetstrokecolor{currentstroke}%
\pgfsetdash{}{0pt}%
\pgfpathmoveto{\pgfqpoint{2.059850in}{3.994794in}}%
\pgfpathcurveto{\pgfqpoint{2.070900in}{3.994794in}}{\pgfqpoint{2.081499in}{3.999184in}}{\pgfqpoint{2.089313in}{4.006997in}}%
\pgfpathcurveto{\pgfqpoint{2.097126in}{4.014811in}}{\pgfqpoint{2.101516in}{4.025410in}}{\pgfqpoint{2.101516in}{4.036460in}}%
\pgfpathcurveto{\pgfqpoint{2.101516in}{4.047510in}}{\pgfqpoint{2.097126in}{4.058109in}}{\pgfqpoint{2.089313in}{4.065923in}}%
\pgfpathcurveto{\pgfqpoint{2.081499in}{4.073737in}}{\pgfqpoint{2.070900in}{4.078127in}}{\pgfqpoint{2.059850in}{4.078127in}}%
\pgfpathcurveto{\pgfqpoint{2.048800in}{4.078127in}}{\pgfqpoint{2.038201in}{4.073737in}}{\pgfqpoint{2.030387in}{4.065923in}}%
\pgfpathcurveto{\pgfqpoint{2.022573in}{4.058109in}}{\pgfqpoint{2.018183in}{4.047510in}}{\pgfqpoint{2.018183in}{4.036460in}}%
\pgfpathcurveto{\pgfqpoint{2.018183in}{4.025410in}}{\pgfqpoint{2.022573in}{4.014811in}}{\pgfqpoint{2.030387in}{4.006997in}}%
\pgfpathcurveto{\pgfqpoint{2.038201in}{3.999184in}}{\pgfqpoint{2.048800in}{3.994794in}}{\pgfqpoint{2.059850in}{3.994794in}}%
\pgfpathclose%
\pgfusepath{stroke,fill}%
\end{pgfscope}%
\begin{pgfscope}%
\pgfpathrectangle{\pgfqpoint{0.481978in}{0.331635in}}{\pgfqpoint{9.300000in}{7.700000in}}%
\pgfusepath{clip}%
\pgfsetbuttcap%
\pgfsetroundjoin%
\definecolor{currentfill}{rgb}{0.870588,0.733333,0.607843}%
\pgfsetfillcolor{currentfill}%
\pgfsetlinewidth{0.481800pt}%
\definecolor{currentstroke}{rgb}{1.000000,1.000000,1.000000}%
\pgfsetstrokecolor{currentstroke}%
\pgfsetdash{}{0pt}%
\pgfpathmoveto{\pgfqpoint{5.974496in}{4.908445in}}%
\pgfpathcurveto{\pgfqpoint{5.985546in}{4.908445in}}{\pgfqpoint{5.996145in}{4.912835in}}{\pgfqpoint{6.003958in}{4.920649in}}%
\pgfpathcurveto{\pgfqpoint{6.011772in}{4.928462in}}{\pgfqpoint{6.016162in}{4.939061in}}{\pgfqpoint{6.016162in}{4.950112in}}%
\pgfpathcurveto{\pgfqpoint{6.016162in}{4.961162in}}{\pgfqpoint{6.011772in}{4.971761in}}{\pgfqpoint{6.003958in}{4.979574in}}%
\pgfpathcurveto{\pgfqpoint{5.996145in}{4.987388in}}{\pgfqpoint{5.985546in}{4.991778in}}{\pgfqpoint{5.974496in}{4.991778in}}%
\pgfpathcurveto{\pgfqpoint{5.963445in}{4.991778in}}{\pgfqpoint{5.952846in}{4.987388in}}{\pgfqpoint{5.945033in}{4.979574in}}%
\pgfpathcurveto{\pgfqpoint{5.937219in}{4.971761in}}{\pgfqpoint{5.932829in}{4.961162in}}{\pgfqpoint{5.932829in}{4.950112in}}%
\pgfpathcurveto{\pgfqpoint{5.932829in}{4.939061in}}{\pgfqpoint{5.937219in}{4.928462in}}{\pgfqpoint{5.945033in}{4.920649in}}%
\pgfpathcurveto{\pgfqpoint{5.952846in}{4.912835in}}{\pgfqpoint{5.963445in}{4.908445in}}{\pgfqpoint{5.974496in}{4.908445in}}%
\pgfpathclose%
\pgfusepath{stroke,fill}%
\end{pgfscope}%
\begin{pgfscope}%
\pgfpathrectangle{\pgfqpoint{0.481978in}{0.331635in}}{\pgfqpoint{9.300000in}{7.700000in}}%
\pgfusepath{clip}%
\pgfsetbuttcap%
\pgfsetroundjoin%
\definecolor{currentfill}{rgb}{0.870588,0.733333,0.607843}%
\pgfsetfillcolor{currentfill}%
\pgfsetlinewidth{0.481800pt}%
\definecolor{currentstroke}{rgb}{1.000000,1.000000,1.000000}%
\pgfsetstrokecolor{currentstroke}%
\pgfsetdash{}{0pt}%
\pgfpathmoveto{\pgfqpoint{2.319348in}{4.753136in}}%
\pgfpathcurveto{\pgfqpoint{2.330398in}{4.753136in}}{\pgfqpoint{2.340997in}{4.757526in}}{\pgfqpoint{2.348811in}{4.765339in}}%
\pgfpathcurveto{\pgfqpoint{2.356624in}{4.773153in}}{\pgfqpoint{2.361015in}{4.783752in}}{\pgfqpoint{2.361015in}{4.794802in}}%
\pgfpathcurveto{\pgfqpoint{2.361015in}{4.805852in}}{\pgfqpoint{2.356624in}{4.816451in}}{\pgfqpoint{2.348811in}{4.824265in}}%
\pgfpathcurveto{\pgfqpoint{2.340997in}{4.832079in}}{\pgfqpoint{2.330398in}{4.836469in}}{\pgfqpoint{2.319348in}{4.836469in}}%
\pgfpathcurveto{\pgfqpoint{2.308298in}{4.836469in}}{\pgfqpoint{2.297699in}{4.832079in}}{\pgfqpoint{2.289885in}{4.824265in}}%
\pgfpathcurveto{\pgfqpoint{2.282071in}{4.816451in}}{\pgfqpoint{2.277681in}{4.805852in}}{\pgfqpoint{2.277681in}{4.794802in}}%
\pgfpathcurveto{\pgfqpoint{2.277681in}{4.783752in}}{\pgfqpoint{2.282071in}{4.773153in}}{\pgfqpoint{2.289885in}{4.765339in}}%
\pgfpathcurveto{\pgfqpoint{2.297699in}{4.757526in}}{\pgfqpoint{2.308298in}{4.753136in}}{\pgfqpoint{2.319348in}{4.753136in}}%
\pgfpathclose%
\pgfusepath{stroke,fill}%
\end{pgfscope}%
\begin{pgfscope}%
\pgfpathrectangle{\pgfqpoint{0.481978in}{0.331635in}}{\pgfqpoint{9.300000in}{7.700000in}}%
\pgfusepath{clip}%
\pgfsetbuttcap%
\pgfsetroundjoin%
\definecolor{currentfill}{rgb}{0.870588,0.733333,0.607843}%
\pgfsetfillcolor{currentfill}%
\pgfsetlinewidth{0.481800pt}%
\definecolor{currentstroke}{rgb}{1.000000,1.000000,1.000000}%
\pgfsetstrokecolor{currentstroke}%
\pgfsetdash{}{0pt}%
\pgfpathmoveto{\pgfqpoint{6.238443in}{6.200764in}}%
\pgfpathcurveto{\pgfqpoint{6.249493in}{6.200764in}}{\pgfqpoint{6.260092in}{6.205154in}}{\pgfqpoint{6.267905in}{6.212968in}}%
\pgfpathcurveto{\pgfqpoint{6.275719in}{6.220782in}}{\pgfqpoint{6.280109in}{6.231381in}}{\pgfqpoint{6.280109in}{6.242431in}}%
\pgfpathcurveto{\pgfqpoint{6.280109in}{6.253481in}}{\pgfqpoint{6.275719in}{6.264080in}}{\pgfqpoint{6.267905in}{6.271894in}}%
\pgfpathcurveto{\pgfqpoint{6.260092in}{6.279707in}}{\pgfqpoint{6.249493in}{6.284097in}}{\pgfqpoint{6.238443in}{6.284097in}}%
\pgfpathcurveto{\pgfqpoint{6.227392in}{6.284097in}}{\pgfqpoint{6.216793in}{6.279707in}}{\pgfqpoint{6.208980in}{6.271894in}}%
\pgfpathcurveto{\pgfqpoint{6.201166in}{6.264080in}}{\pgfqpoint{6.196776in}{6.253481in}}{\pgfqpoint{6.196776in}{6.242431in}}%
\pgfpathcurveto{\pgfqpoint{6.196776in}{6.231381in}}{\pgfqpoint{6.201166in}{6.220782in}}{\pgfqpoint{6.208980in}{6.212968in}}%
\pgfpathcurveto{\pgfqpoint{6.216793in}{6.205154in}}{\pgfqpoint{6.227392in}{6.200764in}}{\pgfqpoint{6.238443in}{6.200764in}}%
\pgfpathclose%
\pgfusepath{stroke,fill}%
\end{pgfscope}%
\begin{pgfscope}%
\pgfpathrectangle{\pgfqpoint{0.481978in}{0.331635in}}{\pgfqpoint{9.300000in}{7.700000in}}%
\pgfusepath{clip}%
\pgfsetbuttcap%
\pgfsetroundjoin%
\definecolor{currentfill}{rgb}{0.870588,0.733333,0.607843}%
\pgfsetfillcolor{currentfill}%
\pgfsetlinewidth{0.481800pt}%
\definecolor{currentstroke}{rgb}{1.000000,1.000000,1.000000}%
\pgfsetstrokecolor{currentstroke}%
\pgfsetdash{}{0pt}%
\pgfpathmoveto{\pgfqpoint{4.167128in}{3.855525in}}%
\pgfpathcurveto{\pgfqpoint{4.178178in}{3.855525in}}{\pgfqpoint{4.188778in}{3.859915in}}{\pgfqpoint{4.196591in}{3.867729in}}%
\pgfpathcurveto{\pgfqpoint{4.204405in}{3.875542in}}{\pgfqpoint{4.208795in}{3.886141in}}{\pgfqpoint{4.208795in}{3.897191in}}%
\pgfpathcurveto{\pgfqpoint{4.208795in}{3.908241in}}{\pgfqpoint{4.204405in}{3.918841in}}{\pgfqpoint{4.196591in}{3.926654in}}%
\pgfpathcurveto{\pgfqpoint{4.188778in}{3.934468in}}{\pgfqpoint{4.178178in}{3.938858in}}{\pgfqpoint{4.167128in}{3.938858in}}%
\pgfpathcurveto{\pgfqpoint{4.156078in}{3.938858in}}{\pgfqpoint{4.145479in}{3.934468in}}{\pgfqpoint{4.137666in}{3.926654in}}%
\pgfpathcurveto{\pgfqpoint{4.129852in}{3.918841in}}{\pgfqpoint{4.125462in}{3.908241in}}{\pgfqpoint{4.125462in}{3.897191in}}%
\pgfpathcurveto{\pgfqpoint{4.125462in}{3.886141in}}{\pgfqpoint{4.129852in}{3.875542in}}{\pgfqpoint{4.137666in}{3.867729in}}%
\pgfpathcurveto{\pgfqpoint{4.145479in}{3.859915in}}{\pgfqpoint{4.156078in}{3.855525in}}{\pgfqpoint{4.167128in}{3.855525in}}%
\pgfpathclose%
\pgfusepath{stroke,fill}%
\end{pgfscope}%
\begin{pgfscope}%
\pgfpathrectangle{\pgfqpoint{0.481978in}{0.331635in}}{\pgfqpoint{9.300000in}{7.700000in}}%
\pgfusepath{clip}%
\pgfsetbuttcap%
\pgfsetroundjoin%
\definecolor{currentfill}{rgb}{0.870588,0.733333,0.607843}%
\pgfsetfillcolor{currentfill}%
\pgfsetlinewidth{0.481800pt}%
\definecolor{currentstroke}{rgb}{1.000000,1.000000,1.000000}%
\pgfsetstrokecolor{currentstroke}%
\pgfsetdash{}{0pt}%
\pgfpathmoveto{\pgfqpoint{4.383560in}{3.556162in}}%
\pgfpathcurveto{\pgfqpoint{4.394610in}{3.556162in}}{\pgfqpoint{4.405209in}{3.560553in}}{\pgfqpoint{4.413023in}{3.568366in}}%
\pgfpathcurveto{\pgfqpoint{4.420836in}{3.576180in}}{\pgfqpoint{4.425227in}{3.586779in}}{\pgfqpoint{4.425227in}{3.597829in}}%
\pgfpathcurveto{\pgfqpoint{4.425227in}{3.608879in}}{\pgfqpoint{4.420836in}{3.619478in}}{\pgfqpoint{4.413023in}{3.627292in}}%
\pgfpathcurveto{\pgfqpoint{4.405209in}{3.635105in}}{\pgfqpoint{4.394610in}{3.639496in}}{\pgfqpoint{4.383560in}{3.639496in}}%
\pgfpathcurveto{\pgfqpoint{4.372510in}{3.639496in}}{\pgfqpoint{4.361911in}{3.635105in}}{\pgfqpoint{4.354097in}{3.627292in}}%
\pgfpathcurveto{\pgfqpoint{4.346284in}{3.619478in}}{\pgfqpoint{4.341893in}{3.608879in}}{\pgfqpoint{4.341893in}{3.597829in}}%
\pgfpathcurveto{\pgfqpoint{4.341893in}{3.586779in}}{\pgfqpoint{4.346284in}{3.576180in}}{\pgfqpoint{4.354097in}{3.568366in}}%
\pgfpathcurveto{\pgfqpoint{4.361911in}{3.560553in}}{\pgfqpoint{4.372510in}{3.556162in}}{\pgfqpoint{4.383560in}{3.556162in}}%
\pgfpathclose%
\pgfusepath{stroke,fill}%
\end{pgfscope}%
\begin{pgfscope}%
\pgfpathrectangle{\pgfqpoint{0.481978in}{0.331635in}}{\pgfqpoint{9.300000in}{7.700000in}}%
\pgfusepath{clip}%
\pgfsetbuttcap%
\pgfsetroundjoin%
\definecolor{currentfill}{rgb}{0.870588,0.733333,0.607843}%
\pgfsetfillcolor{currentfill}%
\pgfsetlinewidth{0.481800pt}%
\definecolor{currentstroke}{rgb}{1.000000,1.000000,1.000000}%
\pgfsetstrokecolor{currentstroke}%
\pgfsetdash{}{0pt}%
\pgfpathmoveto{\pgfqpoint{5.769577in}{3.960324in}}%
\pgfpathcurveto{\pgfqpoint{5.780628in}{3.960324in}}{\pgfqpoint{5.791227in}{3.964714in}}{\pgfqpoint{5.799040in}{3.972527in}}%
\pgfpathcurveto{\pgfqpoint{5.806854in}{3.980341in}}{\pgfqpoint{5.811244in}{3.990940in}}{\pgfqpoint{5.811244in}{4.001990in}}%
\pgfpathcurveto{\pgfqpoint{5.811244in}{4.013040in}}{\pgfqpoint{5.806854in}{4.023639in}}{\pgfqpoint{5.799040in}{4.031453in}}%
\pgfpathcurveto{\pgfqpoint{5.791227in}{4.039267in}}{\pgfqpoint{5.780628in}{4.043657in}}{\pgfqpoint{5.769577in}{4.043657in}}%
\pgfpathcurveto{\pgfqpoint{5.758527in}{4.043657in}}{\pgfqpoint{5.747928in}{4.039267in}}{\pgfqpoint{5.740115in}{4.031453in}}%
\pgfpathcurveto{\pgfqpoint{5.732301in}{4.023639in}}{\pgfqpoint{5.727911in}{4.013040in}}{\pgfqpoint{5.727911in}{4.001990in}}%
\pgfpathcurveto{\pgfqpoint{5.727911in}{3.990940in}}{\pgfqpoint{5.732301in}{3.980341in}}{\pgfqpoint{5.740115in}{3.972527in}}%
\pgfpathcurveto{\pgfqpoint{5.747928in}{3.964714in}}{\pgfqpoint{5.758527in}{3.960324in}}{\pgfqpoint{5.769577in}{3.960324in}}%
\pgfpathclose%
\pgfusepath{stroke,fill}%
\end{pgfscope}%
\begin{pgfscope}%
\pgfpathrectangle{\pgfqpoint{0.481978in}{0.331635in}}{\pgfqpoint{9.300000in}{7.700000in}}%
\pgfusepath{clip}%
\pgfsetbuttcap%
\pgfsetroundjoin%
\definecolor{currentfill}{rgb}{0.870588,0.733333,0.607843}%
\pgfsetfillcolor{currentfill}%
\pgfsetlinewidth{0.481800pt}%
\definecolor{currentstroke}{rgb}{1.000000,1.000000,1.000000}%
\pgfsetstrokecolor{currentstroke}%
\pgfsetdash{}{0pt}%
\pgfpathmoveto{\pgfqpoint{9.359251in}{4.528409in}}%
\pgfpathcurveto{\pgfqpoint{9.370301in}{4.528409in}}{\pgfqpoint{9.380900in}{4.532799in}}{\pgfqpoint{9.388713in}{4.540613in}}%
\pgfpathcurveto{\pgfqpoint{9.396527in}{4.548427in}}{\pgfqpoint{9.400917in}{4.559026in}}{\pgfqpoint{9.400917in}{4.570076in}}%
\pgfpathcurveto{\pgfqpoint{9.400917in}{4.581126in}}{\pgfqpoint{9.396527in}{4.591725in}}{\pgfqpoint{9.388713in}{4.599539in}}%
\pgfpathcurveto{\pgfqpoint{9.380900in}{4.607352in}}{\pgfqpoint{9.370301in}{4.611743in}}{\pgfqpoint{9.359251in}{4.611743in}}%
\pgfpathcurveto{\pgfqpoint{9.348201in}{4.611743in}}{\pgfqpoint{9.337601in}{4.607352in}}{\pgfqpoint{9.329788in}{4.599539in}}%
\pgfpathcurveto{\pgfqpoint{9.321974in}{4.591725in}}{\pgfqpoint{9.317584in}{4.581126in}}{\pgfqpoint{9.317584in}{4.570076in}}%
\pgfpathcurveto{\pgfqpoint{9.317584in}{4.559026in}}{\pgfqpoint{9.321974in}{4.548427in}}{\pgfqpoint{9.329788in}{4.540613in}}%
\pgfpathcurveto{\pgfqpoint{9.337601in}{4.532799in}}{\pgfqpoint{9.348201in}{4.528409in}}{\pgfqpoint{9.359251in}{4.528409in}}%
\pgfpathclose%
\pgfusepath{stroke,fill}%
\end{pgfscope}%
\begin{pgfscope}%
\pgfpathrectangle{\pgfqpoint{0.481978in}{0.331635in}}{\pgfqpoint{9.300000in}{7.700000in}}%
\pgfusepath{clip}%
\pgfsetbuttcap%
\pgfsetroundjoin%
\definecolor{currentfill}{rgb}{0.870588,0.733333,0.607843}%
\pgfsetfillcolor{currentfill}%
\pgfsetlinewidth{0.481800pt}%
\definecolor{currentstroke}{rgb}{1.000000,1.000000,1.000000}%
\pgfsetstrokecolor{currentstroke}%
\pgfsetdash{}{0pt}%
\pgfpathmoveto{\pgfqpoint{5.675309in}{2.630964in}}%
\pgfpathcurveto{\pgfqpoint{5.686359in}{2.630964in}}{\pgfqpoint{5.696958in}{2.635354in}}{\pgfqpoint{5.704772in}{2.643168in}}%
\pgfpathcurveto{\pgfqpoint{5.712586in}{2.650981in}}{\pgfqpoint{5.716976in}{2.661581in}}{\pgfqpoint{5.716976in}{2.672631in}}%
\pgfpathcurveto{\pgfqpoint{5.716976in}{2.683681in}}{\pgfqpoint{5.712586in}{2.694280in}}{\pgfqpoint{5.704772in}{2.702093in}}%
\pgfpathcurveto{\pgfqpoint{5.696958in}{2.709907in}}{\pgfqpoint{5.686359in}{2.714297in}}{\pgfqpoint{5.675309in}{2.714297in}}%
\pgfpathcurveto{\pgfqpoint{5.664259in}{2.714297in}}{\pgfqpoint{5.653660in}{2.709907in}}{\pgfqpoint{5.645846in}{2.702093in}}%
\pgfpathcurveto{\pgfqpoint{5.638033in}{2.694280in}}{\pgfqpoint{5.633643in}{2.683681in}}{\pgfqpoint{5.633643in}{2.672631in}}%
\pgfpathcurveto{\pgfqpoint{5.633643in}{2.661581in}}{\pgfqpoint{5.638033in}{2.650981in}}{\pgfqpoint{5.645846in}{2.643168in}}%
\pgfpathcurveto{\pgfqpoint{5.653660in}{2.635354in}}{\pgfqpoint{5.664259in}{2.630964in}}{\pgfqpoint{5.675309in}{2.630964in}}%
\pgfpathclose%
\pgfusepath{stroke,fill}%
\end{pgfscope}%
\begin{pgfscope}%
\pgfpathrectangle{\pgfqpoint{0.481978in}{0.331635in}}{\pgfqpoint{9.300000in}{7.700000in}}%
\pgfusepath{clip}%
\pgfsetbuttcap%
\pgfsetroundjoin%
\definecolor{currentfill}{rgb}{0.870588,0.733333,0.607843}%
\pgfsetfillcolor{currentfill}%
\pgfsetlinewidth{0.481800pt}%
\definecolor{currentstroke}{rgb}{1.000000,1.000000,1.000000}%
\pgfsetstrokecolor{currentstroke}%
\pgfsetdash{}{0pt}%
\pgfpathmoveto{\pgfqpoint{3.026449in}{4.306297in}}%
\pgfpathcurveto{\pgfqpoint{3.037499in}{4.306297in}}{\pgfqpoint{3.048098in}{4.310688in}}{\pgfqpoint{3.055912in}{4.318501in}}%
\pgfpathcurveto{\pgfqpoint{3.063726in}{4.326315in}}{\pgfqpoint{3.068116in}{4.336914in}}{\pgfqpoint{3.068116in}{4.347964in}}%
\pgfpathcurveto{\pgfqpoint{3.068116in}{4.359014in}}{\pgfqpoint{3.063726in}{4.369613in}}{\pgfqpoint{3.055912in}{4.377427in}}%
\pgfpathcurveto{\pgfqpoint{3.048098in}{4.385240in}}{\pgfqpoint{3.037499in}{4.389631in}}{\pgfqpoint{3.026449in}{4.389631in}}%
\pgfpathcurveto{\pgfqpoint{3.015399in}{4.389631in}}{\pgfqpoint{3.004800in}{4.385240in}}{\pgfqpoint{2.996986in}{4.377427in}}%
\pgfpathcurveto{\pgfqpoint{2.989173in}{4.369613in}}{\pgfqpoint{2.984783in}{4.359014in}}{\pgfqpoint{2.984783in}{4.347964in}}%
\pgfpathcurveto{\pgfqpoint{2.984783in}{4.336914in}}{\pgfqpoint{2.989173in}{4.326315in}}{\pgfqpoint{2.996986in}{4.318501in}}%
\pgfpathcurveto{\pgfqpoint{3.004800in}{4.310688in}}{\pgfqpoint{3.015399in}{4.306297in}}{\pgfqpoint{3.026449in}{4.306297in}}%
\pgfpathclose%
\pgfusepath{stroke,fill}%
\end{pgfscope}%
\begin{pgfscope}%
\pgfpathrectangle{\pgfqpoint{0.481978in}{0.331635in}}{\pgfqpoint{9.300000in}{7.700000in}}%
\pgfusepath{clip}%
\pgfsetbuttcap%
\pgfsetroundjoin%
\definecolor{currentfill}{rgb}{0.870588,0.733333,0.607843}%
\pgfsetfillcolor{currentfill}%
\pgfsetlinewidth{0.481800pt}%
\definecolor{currentstroke}{rgb}{1.000000,1.000000,1.000000}%
\pgfsetstrokecolor{currentstroke}%
\pgfsetdash{}{0pt}%
\pgfpathmoveto{\pgfqpoint{6.761280in}{5.824363in}}%
\pgfpathcurveto{\pgfqpoint{6.772330in}{5.824363in}}{\pgfqpoint{6.782929in}{5.828753in}}{\pgfqpoint{6.790742in}{5.836567in}}%
\pgfpathcurveto{\pgfqpoint{6.798556in}{5.844381in}}{\pgfqpoint{6.802946in}{5.854980in}}{\pgfqpoint{6.802946in}{5.866030in}}%
\pgfpathcurveto{\pgfqpoint{6.802946in}{5.877080in}}{\pgfqpoint{6.798556in}{5.887679in}}{\pgfqpoint{6.790742in}{5.895492in}}%
\pgfpathcurveto{\pgfqpoint{6.782929in}{5.903306in}}{\pgfqpoint{6.772330in}{5.907696in}}{\pgfqpoint{6.761280in}{5.907696in}}%
\pgfpathcurveto{\pgfqpoint{6.750229in}{5.907696in}}{\pgfqpoint{6.739630in}{5.903306in}}{\pgfqpoint{6.731817in}{5.895492in}}%
\pgfpathcurveto{\pgfqpoint{6.724003in}{5.887679in}}{\pgfqpoint{6.719613in}{5.877080in}}{\pgfqpoint{6.719613in}{5.866030in}}%
\pgfpathcurveto{\pgfqpoint{6.719613in}{5.854980in}}{\pgfqpoint{6.724003in}{5.844381in}}{\pgfqpoint{6.731817in}{5.836567in}}%
\pgfpathcurveto{\pgfqpoint{6.739630in}{5.828753in}}{\pgfqpoint{6.750229in}{5.824363in}}{\pgfqpoint{6.761280in}{5.824363in}}%
\pgfpathclose%
\pgfusepath{stroke,fill}%
\end{pgfscope}%
\begin{pgfscope}%
\pgfpathrectangle{\pgfqpoint{0.481978in}{0.331635in}}{\pgfqpoint{9.300000in}{7.700000in}}%
\pgfusepath{clip}%
\pgfsetbuttcap%
\pgfsetroundjoin%
\definecolor{currentfill}{rgb}{0.870588,0.733333,0.607843}%
\pgfsetfillcolor{currentfill}%
\pgfsetlinewidth{0.481800pt}%
\definecolor{currentstroke}{rgb}{1.000000,1.000000,1.000000}%
\pgfsetstrokecolor{currentstroke}%
\pgfsetdash{}{0pt}%
\pgfpathmoveto{\pgfqpoint{2.445109in}{3.067913in}}%
\pgfpathcurveto{\pgfqpoint{2.456159in}{3.067913in}}{\pgfqpoint{2.466758in}{3.072303in}}{\pgfqpoint{2.474571in}{3.080117in}}%
\pgfpathcurveto{\pgfqpoint{2.482385in}{3.087930in}}{\pgfqpoint{2.486775in}{3.098529in}}{\pgfqpoint{2.486775in}{3.109579in}}%
\pgfpathcurveto{\pgfqpoint{2.486775in}{3.120630in}}{\pgfqpoint{2.482385in}{3.131229in}}{\pgfqpoint{2.474571in}{3.139042in}}%
\pgfpathcurveto{\pgfqpoint{2.466758in}{3.146856in}}{\pgfqpoint{2.456159in}{3.151246in}}{\pgfqpoint{2.445109in}{3.151246in}}%
\pgfpathcurveto{\pgfqpoint{2.434059in}{3.151246in}}{\pgfqpoint{2.423459in}{3.146856in}}{\pgfqpoint{2.415646in}{3.139042in}}%
\pgfpathcurveto{\pgfqpoint{2.407832in}{3.131229in}}{\pgfqpoint{2.403442in}{3.120630in}}{\pgfqpoint{2.403442in}{3.109579in}}%
\pgfpathcurveto{\pgfqpoint{2.403442in}{3.098529in}}{\pgfqpoint{2.407832in}{3.087930in}}{\pgfqpoint{2.415646in}{3.080117in}}%
\pgfpathcurveto{\pgfqpoint{2.423459in}{3.072303in}}{\pgfqpoint{2.434059in}{3.067913in}}{\pgfqpoint{2.445109in}{3.067913in}}%
\pgfpathclose%
\pgfusepath{stroke,fill}%
\end{pgfscope}%
\begin{pgfscope}%
\pgfpathrectangle{\pgfqpoint{0.481978in}{0.331635in}}{\pgfqpoint{9.300000in}{7.700000in}}%
\pgfusepath{clip}%
\pgfsetbuttcap%
\pgfsetroundjoin%
\definecolor{currentfill}{rgb}{0.870588,0.733333,0.607843}%
\pgfsetfillcolor{currentfill}%
\pgfsetlinewidth{0.481800pt}%
\definecolor{currentstroke}{rgb}{1.000000,1.000000,1.000000}%
\pgfsetstrokecolor{currentstroke}%
\pgfsetdash{}{0pt}%
\pgfpathmoveto{\pgfqpoint{2.556572in}{4.445719in}}%
\pgfpathcurveto{\pgfqpoint{2.567622in}{4.445719in}}{\pgfqpoint{2.578221in}{4.450109in}}{\pgfqpoint{2.586035in}{4.457922in}}%
\pgfpathcurveto{\pgfqpoint{2.593848in}{4.465736in}}{\pgfqpoint{2.598239in}{4.476335in}}{\pgfqpoint{2.598239in}{4.487385in}}%
\pgfpathcurveto{\pgfqpoint{2.598239in}{4.498435in}}{\pgfqpoint{2.593848in}{4.509034in}}{\pgfqpoint{2.586035in}{4.516848in}}%
\pgfpathcurveto{\pgfqpoint{2.578221in}{4.524662in}}{\pgfqpoint{2.567622in}{4.529052in}}{\pgfqpoint{2.556572in}{4.529052in}}%
\pgfpathcurveto{\pgfqpoint{2.545522in}{4.529052in}}{\pgfqpoint{2.534923in}{4.524662in}}{\pgfqpoint{2.527109in}{4.516848in}}%
\pgfpathcurveto{\pgfqpoint{2.519296in}{4.509034in}}{\pgfqpoint{2.514905in}{4.498435in}}{\pgfqpoint{2.514905in}{4.487385in}}%
\pgfpathcurveto{\pgfqpoint{2.514905in}{4.476335in}}{\pgfqpoint{2.519296in}{4.465736in}}{\pgfqpoint{2.527109in}{4.457922in}}%
\pgfpathcurveto{\pgfqpoint{2.534923in}{4.450109in}}{\pgfqpoint{2.545522in}{4.445719in}}{\pgfqpoint{2.556572in}{4.445719in}}%
\pgfpathclose%
\pgfusepath{stroke,fill}%
\end{pgfscope}%
\begin{pgfscope}%
\pgfpathrectangle{\pgfqpoint{0.481978in}{0.331635in}}{\pgfqpoint{9.300000in}{7.700000in}}%
\pgfusepath{clip}%
\pgfsetbuttcap%
\pgfsetroundjoin%
\definecolor{currentfill}{rgb}{0.870588,0.733333,0.607843}%
\pgfsetfillcolor{currentfill}%
\pgfsetlinewidth{0.481800pt}%
\definecolor{currentstroke}{rgb}{1.000000,1.000000,1.000000}%
\pgfsetstrokecolor{currentstroke}%
\pgfsetdash{}{0pt}%
\pgfpathmoveto{\pgfqpoint{3.956527in}{3.265885in}}%
\pgfpathcurveto{\pgfqpoint{3.967578in}{3.265885in}}{\pgfqpoint{3.978177in}{3.270275in}}{\pgfqpoint{3.985990in}{3.278089in}}%
\pgfpathcurveto{\pgfqpoint{3.993804in}{3.285902in}}{\pgfqpoint{3.998194in}{3.296501in}}{\pgfqpoint{3.998194in}{3.307551in}}%
\pgfpathcurveto{\pgfqpoint{3.998194in}{3.318602in}}{\pgfqpoint{3.993804in}{3.329201in}}{\pgfqpoint{3.985990in}{3.337014in}}%
\pgfpathcurveto{\pgfqpoint{3.978177in}{3.344828in}}{\pgfqpoint{3.967578in}{3.349218in}}{\pgfqpoint{3.956527in}{3.349218in}}%
\pgfpathcurveto{\pgfqpoint{3.945477in}{3.349218in}}{\pgfqpoint{3.934878in}{3.344828in}}{\pgfqpoint{3.927065in}{3.337014in}}%
\pgfpathcurveto{\pgfqpoint{3.919251in}{3.329201in}}{\pgfqpoint{3.914861in}{3.318602in}}{\pgfqpoint{3.914861in}{3.307551in}}%
\pgfpathcurveto{\pgfqpoint{3.914861in}{3.296501in}}{\pgfqpoint{3.919251in}{3.285902in}}{\pgfqpoint{3.927065in}{3.278089in}}%
\pgfpathcurveto{\pgfqpoint{3.934878in}{3.270275in}}{\pgfqpoint{3.945477in}{3.265885in}}{\pgfqpoint{3.956527in}{3.265885in}}%
\pgfpathclose%
\pgfusepath{stroke,fill}%
\end{pgfscope}%
\begin{pgfscope}%
\pgfpathrectangle{\pgfqpoint{0.481978in}{0.331635in}}{\pgfqpoint{9.300000in}{7.700000in}}%
\pgfusepath{clip}%
\pgfsetbuttcap%
\pgfsetroundjoin%
\definecolor{currentfill}{rgb}{0.870588,0.733333,0.607843}%
\pgfsetfillcolor{currentfill}%
\pgfsetlinewidth{0.481800pt}%
\definecolor{currentstroke}{rgb}{1.000000,1.000000,1.000000}%
\pgfsetstrokecolor{currentstroke}%
\pgfsetdash{}{0pt}%
\pgfpathmoveto{\pgfqpoint{2.593480in}{4.701301in}}%
\pgfpathcurveto{\pgfqpoint{2.604530in}{4.701301in}}{\pgfqpoint{2.615129in}{4.705692in}}{\pgfqpoint{2.622942in}{4.713505in}}%
\pgfpathcurveto{\pgfqpoint{2.630756in}{4.721319in}}{\pgfqpoint{2.635146in}{4.731918in}}{\pgfqpoint{2.635146in}{4.742968in}}%
\pgfpathcurveto{\pgfqpoint{2.635146in}{4.754018in}}{\pgfqpoint{2.630756in}{4.764617in}}{\pgfqpoint{2.622942in}{4.772431in}}%
\pgfpathcurveto{\pgfqpoint{2.615129in}{4.780244in}}{\pgfqpoint{2.604530in}{4.784635in}}{\pgfqpoint{2.593480in}{4.784635in}}%
\pgfpathcurveto{\pgfqpoint{2.582429in}{4.784635in}}{\pgfqpoint{2.571830in}{4.780244in}}{\pgfqpoint{2.564017in}{4.772431in}}%
\pgfpathcurveto{\pgfqpoint{2.556203in}{4.764617in}}{\pgfqpoint{2.551813in}{4.754018in}}{\pgfqpoint{2.551813in}{4.742968in}}%
\pgfpathcurveto{\pgfqpoint{2.551813in}{4.731918in}}{\pgfqpoint{2.556203in}{4.721319in}}{\pgfqpoint{2.564017in}{4.713505in}}%
\pgfpathcurveto{\pgfqpoint{2.571830in}{4.705692in}}{\pgfqpoint{2.582429in}{4.701301in}}{\pgfqpoint{2.593480in}{4.701301in}}%
\pgfpathclose%
\pgfusepath{stroke,fill}%
\end{pgfscope}%
\begin{pgfscope}%
\pgfpathrectangle{\pgfqpoint{0.481978in}{0.331635in}}{\pgfqpoint{9.300000in}{7.700000in}}%
\pgfusepath{clip}%
\pgfsetbuttcap%
\pgfsetroundjoin%
\definecolor{currentfill}{rgb}{0.870588,0.733333,0.607843}%
\pgfsetfillcolor{currentfill}%
\pgfsetlinewidth{0.481800pt}%
\definecolor{currentstroke}{rgb}{1.000000,1.000000,1.000000}%
\pgfsetstrokecolor{currentstroke}%
\pgfsetdash{}{0pt}%
\pgfpathmoveto{\pgfqpoint{1.766805in}{3.461988in}}%
\pgfpathcurveto{\pgfqpoint{1.777855in}{3.461988in}}{\pgfqpoint{1.788454in}{3.466378in}}{\pgfqpoint{1.796268in}{3.474192in}}%
\pgfpathcurveto{\pgfqpoint{1.804082in}{3.482006in}}{\pgfqpoint{1.808472in}{3.492605in}}{\pgfqpoint{1.808472in}{3.503655in}}%
\pgfpathcurveto{\pgfqpoint{1.808472in}{3.514705in}}{\pgfqpoint{1.804082in}{3.525304in}}{\pgfqpoint{1.796268in}{3.533118in}}%
\pgfpathcurveto{\pgfqpoint{1.788454in}{3.540931in}}{\pgfqpoint{1.777855in}{3.545321in}}{\pgfqpoint{1.766805in}{3.545321in}}%
\pgfpathcurveto{\pgfqpoint{1.755755in}{3.545321in}}{\pgfqpoint{1.745156in}{3.540931in}}{\pgfqpoint{1.737343in}{3.533118in}}%
\pgfpathcurveto{\pgfqpoint{1.729529in}{3.525304in}}{\pgfqpoint{1.725139in}{3.514705in}}{\pgfqpoint{1.725139in}{3.503655in}}%
\pgfpathcurveto{\pgfqpoint{1.725139in}{3.492605in}}{\pgfqpoint{1.729529in}{3.482006in}}{\pgfqpoint{1.737343in}{3.474192in}}%
\pgfpathcurveto{\pgfqpoint{1.745156in}{3.466378in}}{\pgfqpoint{1.755755in}{3.461988in}}{\pgfqpoint{1.766805in}{3.461988in}}%
\pgfpathclose%
\pgfusepath{stroke,fill}%
\end{pgfscope}%
\begin{pgfscope}%
\pgfpathrectangle{\pgfqpoint{0.481978in}{0.331635in}}{\pgfqpoint{9.300000in}{7.700000in}}%
\pgfusepath{clip}%
\pgfsetbuttcap%
\pgfsetroundjoin%
\definecolor{currentfill}{rgb}{0.870588,0.733333,0.607843}%
\pgfsetfillcolor{currentfill}%
\pgfsetlinewidth{0.481800pt}%
\definecolor{currentstroke}{rgb}{1.000000,1.000000,1.000000}%
\pgfsetstrokecolor{currentstroke}%
\pgfsetdash{}{0pt}%
\pgfpathmoveto{\pgfqpoint{4.977562in}{4.240982in}}%
\pgfpathcurveto{\pgfqpoint{4.988613in}{4.240982in}}{\pgfqpoint{4.999212in}{4.245373in}}{\pgfqpoint{5.007025in}{4.253186in}}%
\pgfpathcurveto{\pgfqpoint{5.014839in}{4.261000in}}{\pgfqpoint{5.019229in}{4.271599in}}{\pgfqpoint{5.019229in}{4.282649in}}%
\pgfpathcurveto{\pgfqpoint{5.019229in}{4.293699in}}{\pgfqpoint{5.014839in}{4.304298in}}{\pgfqpoint{5.007025in}{4.312112in}}%
\pgfpathcurveto{\pgfqpoint{4.999212in}{4.319926in}}{\pgfqpoint{4.988613in}{4.324316in}}{\pgfqpoint{4.977562in}{4.324316in}}%
\pgfpathcurveto{\pgfqpoint{4.966512in}{4.324316in}}{\pgfqpoint{4.955913in}{4.319926in}}{\pgfqpoint{4.948100in}{4.312112in}}%
\pgfpathcurveto{\pgfqpoint{4.940286in}{4.304298in}}{\pgfqpoint{4.935896in}{4.293699in}}{\pgfqpoint{4.935896in}{4.282649in}}%
\pgfpathcurveto{\pgfqpoint{4.935896in}{4.271599in}}{\pgfqpoint{4.940286in}{4.261000in}}{\pgfqpoint{4.948100in}{4.253186in}}%
\pgfpathcurveto{\pgfqpoint{4.955913in}{4.245373in}}{\pgfqpoint{4.966512in}{4.240982in}}{\pgfqpoint{4.977562in}{4.240982in}}%
\pgfpathclose%
\pgfusepath{stroke,fill}%
\end{pgfscope}%
\begin{pgfscope}%
\pgfpathrectangle{\pgfqpoint{0.481978in}{0.331635in}}{\pgfqpoint{9.300000in}{7.700000in}}%
\pgfusepath{clip}%
\pgfsetbuttcap%
\pgfsetroundjoin%
\definecolor{currentfill}{rgb}{0.870588,0.733333,0.607843}%
\pgfsetfillcolor{currentfill}%
\pgfsetlinewidth{0.481800pt}%
\definecolor{currentstroke}{rgb}{1.000000,1.000000,1.000000}%
\pgfsetstrokecolor{currentstroke}%
\pgfsetdash{}{0pt}%
\pgfpathmoveto{\pgfqpoint{5.616567in}{3.948372in}}%
\pgfpathcurveto{\pgfqpoint{5.627617in}{3.948372in}}{\pgfqpoint{5.638216in}{3.952762in}}{\pgfqpoint{5.646030in}{3.960576in}}%
\pgfpathcurveto{\pgfqpoint{5.653844in}{3.968390in}}{\pgfqpoint{5.658234in}{3.978989in}}{\pgfqpoint{5.658234in}{3.990039in}}%
\pgfpathcurveto{\pgfqpoint{5.658234in}{4.001089in}}{\pgfqpoint{5.653844in}{4.011688in}}{\pgfqpoint{5.646030in}{4.019502in}}%
\pgfpathcurveto{\pgfqpoint{5.638216in}{4.027315in}}{\pgfqpoint{5.627617in}{4.031705in}}{\pgfqpoint{5.616567in}{4.031705in}}%
\pgfpathcurveto{\pgfqpoint{5.605517in}{4.031705in}}{\pgfqpoint{5.594918in}{4.027315in}}{\pgfqpoint{5.587105in}{4.019502in}}%
\pgfpathcurveto{\pgfqpoint{5.579291in}{4.011688in}}{\pgfqpoint{5.574901in}{4.001089in}}{\pgfqpoint{5.574901in}{3.990039in}}%
\pgfpathcurveto{\pgfqpoint{5.574901in}{3.978989in}}{\pgfqpoint{5.579291in}{3.968390in}}{\pgfqpoint{5.587105in}{3.960576in}}%
\pgfpathcurveto{\pgfqpoint{5.594918in}{3.952762in}}{\pgfqpoint{5.605517in}{3.948372in}}{\pgfqpoint{5.616567in}{3.948372in}}%
\pgfpathclose%
\pgfusepath{stroke,fill}%
\end{pgfscope}%
\begin{pgfscope}%
\pgfpathrectangle{\pgfqpoint{0.481978in}{0.331635in}}{\pgfqpoint{9.300000in}{7.700000in}}%
\pgfusepath{clip}%
\pgfsetbuttcap%
\pgfsetroundjoin%
\definecolor{currentfill}{rgb}{0.870588,0.733333,0.607843}%
\pgfsetfillcolor{currentfill}%
\pgfsetlinewidth{0.481800pt}%
\definecolor{currentstroke}{rgb}{1.000000,1.000000,1.000000}%
\pgfsetstrokecolor{currentstroke}%
\pgfsetdash{}{0pt}%
\pgfpathmoveto{\pgfqpoint{2.747420in}{2.538850in}}%
\pgfpathcurveto{\pgfqpoint{2.758470in}{2.538850in}}{\pgfqpoint{2.769069in}{2.543241in}}{\pgfqpoint{2.776883in}{2.551054in}}%
\pgfpathcurveto{\pgfqpoint{2.784696in}{2.558868in}}{\pgfqpoint{2.789087in}{2.569467in}}{\pgfqpoint{2.789087in}{2.580517in}}%
\pgfpathcurveto{\pgfqpoint{2.789087in}{2.591567in}}{\pgfqpoint{2.784696in}{2.602166in}}{\pgfqpoint{2.776883in}{2.609980in}}%
\pgfpathcurveto{\pgfqpoint{2.769069in}{2.617793in}}{\pgfqpoint{2.758470in}{2.622184in}}{\pgfqpoint{2.747420in}{2.622184in}}%
\pgfpathcurveto{\pgfqpoint{2.736370in}{2.622184in}}{\pgfqpoint{2.725771in}{2.617793in}}{\pgfqpoint{2.717957in}{2.609980in}}%
\pgfpathcurveto{\pgfqpoint{2.710143in}{2.602166in}}{\pgfqpoint{2.705753in}{2.591567in}}{\pgfqpoint{2.705753in}{2.580517in}}%
\pgfpathcurveto{\pgfqpoint{2.705753in}{2.569467in}}{\pgfqpoint{2.710143in}{2.558868in}}{\pgfqpoint{2.717957in}{2.551054in}}%
\pgfpathcurveto{\pgfqpoint{2.725771in}{2.543241in}}{\pgfqpoint{2.736370in}{2.538850in}}{\pgfqpoint{2.747420in}{2.538850in}}%
\pgfpathclose%
\pgfusepath{stroke,fill}%
\end{pgfscope}%
\begin{pgfscope}%
\pgfpathrectangle{\pgfqpoint{0.481978in}{0.331635in}}{\pgfqpoint{9.300000in}{7.700000in}}%
\pgfusepath{clip}%
\pgfsetbuttcap%
\pgfsetroundjoin%
\definecolor{currentfill}{rgb}{0.870588,0.733333,0.607843}%
\pgfsetfillcolor{currentfill}%
\pgfsetlinewidth{0.481800pt}%
\definecolor{currentstroke}{rgb}{1.000000,1.000000,1.000000}%
\pgfsetstrokecolor{currentstroke}%
\pgfsetdash{}{0pt}%
\pgfpathmoveto{\pgfqpoint{3.710302in}{3.401243in}}%
\pgfpathcurveto{\pgfqpoint{3.721352in}{3.401243in}}{\pgfqpoint{3.731951in}{3.405633in}}{\pgfqpoint{3.739765in}{3.413447in}}%
\pgfpathcurveto{\pgfqpoint{3.747578in}{3.421260in}}{\pgfqpoint{3.751969in}{3.431859in}}{\pgfqpoint{3.751969in}{3.442909in}}%
\pgfpathcurveto{\pgfqpoint{3.751969in}{3.453960in}}{\pgfqpoint{3.747578in}{3.464559in}}{\pgfqpoint{3.739765in}{3.472372in}}%
\pgfpathcurveto{\pgfqpoint{3.731951in}{3.480186in}}{\pgfqpoint{3.721352in}{3.484576in}}{\pgfqpoint{3.710302in}{3.484576in}}%
\pgfpathcurveto{\pgfqpoint{3.699252in}{3.484576in}}{\pgfqpoint{3.688653in}{3.480186in}}{\pgfqpoint{3.680839in}{3.472372in}}%
\pgfpathcurveto{\pgfqpoint{3.673026in}{3.464559in}}{\pgfqpoint{3.668635in}{3.453960in}}{\pgfqpoint{3.668635in}{3.442909in}}%
\pgfpathcurveto{\pgfqpoint{3.668635in}{3.431859in}}{\pgfqpoint{3.673026in}{3.421260in}}{\pgfqpoint{3.680839in}{3.413447in}}%
\pgfpathcurveto{\pgfqpoint{3.688653in}{3.405633in}}{\pgfqpoint{3.699252in}{3.401243in}}{\pgfqpoint{3.710302in}{3.401243in}}%
\pgfpathclose%
\pgfusepath{stroke,fill}%
\end{pgfscope}%
\begin{pgfscope}%
\pgfpathrectangle{\pgfqpoint{0.481978in}{0.331635in}}{\pgfqpoint{9.300000in}{7.700000in}}%
\pgfusepath{clip}%
\pgfsetbuttcap%
\pgfsetroundjoin%
\definecolor{currentfill}{rgb}{0.870588,0.733333,0.607843}%
\pgfsetfillcolor{currentfill}%
\pgfsetlinewidth{0.481800pt}%
\definecolor{currentstroke}{rgb}{1.000000,1.000000,1.000000}%
\pgfsetstrokecolor{currentstroke}%
\pgfsetdash{}{0pt}%
\pgfpathmoveto{\pgfqpoint{2.149626in}{4.841912in}}%
\pgfpathcurveto{\pgfqpoint{2.160677in}{4.841912in}}{\pgfqpoint{2.171276in}{4.846302in}}{\pgfqpoint{2.179089in}{4.854116in}}%
\pgfpathcurveto{\pgfqpoint{2.186903in}{4.861929in}}{\pgfqpoint{2.191293in}{4.872528in}}{\pgfqpoint{2.191293in}{4.883578in}}%
\pgfpathcurveto{\pgfqpoint{2.191293in}{4.894629in}}{\pgfqpoint{2.186903in}{4.905228in}}{\pgfqpoint{2.179089in}{4.913041in}}%
\pgfpathcurveto{\pgfqpoint{2.171276in}{4.920855in}}{\pgfqpoint{2.160677in}{4.925245in}}{\pgfqpoint{2.149626in}{4.925245in}}%
\pgfpathcurveto{\pgfqpoint{2.138576in}{4.925245in}}{\pgfqpoint{2.127977in}{4.920855in}}{\pgfqpoint{2.120164in}{4.913041in}}%
\pgfpathcurveto{\pgfqpoint{2.112350in}{4.905228in}}{\pgfqpoint{2.107960in}{4.894629in}}{\pgfqpoint{2.107960in}{4.883578in}}%
\pgfpathcurveto{\pgfqpoint{2.107960in}{4.872528in}}{\pgfqpoint{2.112350in}{4.861929in}}{\pgfqpoint{2.120164in}{4.854116in}}%
\pgfpathcurveto{\pgfqpoint{2.127977in}{4.846302in}}{\pgfqpoint{2.138576in}{4.841912in}}{\pgfqpoint{2.149626in}{4.841912in}}%
\pgfpathclose%
\pgfusepath{stroke,fill}%
\end{pgfscope}%
\begin{pgfscope}%
\pgfpathrectangle{\pgfqpoint{0.481978in}{0.331635in}}{\pgfqpoint{9.300000in}{7.700000in}}%
\pgfusepath{clip}%
\pgfsetbuttcap%
\pgfsetroundjoin%
\definecolor{currentfill}{rgb}{0.870588,0.733333,0.607843}%
\pgfsetfillcolor{currentfill}%
\pgfsetlinewidth{0.481800pt}%
\definecolor{currentstroke}{rgb}{1.000000,1.000000,1.000000}%
\pgfsetstrokecolor{currentstroke}%
\pgfsetdash{}{0pt}%
\pgfpathmoveto{\pgfqpoint{2.077561in}{3.061756in}}%
\pgfpathcurveto{\pgfqpoint{2.088611in}{3.061756in}}{\pgfqpoint{2.099210in}{3.066146in}}{\pgfqpoint{2.107024in}{3.073960in}}%
\pgfpathcurveto{\pgfqpoint{2.114838in}{3.081773in}}{\pgfqpoint{2.119228in}{3.092372in}}{\pgfqpoint{2.119228in}{3.103422in}}%
\pgfpathcurveto{\pgfqpoint{2.119228in}{3.114473in}}{\pgfqpoint{2.114838in}{3.125072in}}{\pgfqpoint{2.107024in}{3.132885in}}%
\pgfpathcurveto{\pgfqpoint{2.099210in}{3.140699in}}{\pgfqpoint{2.088611in}{3.145089in}}{\pgfqpoint{2.077561in}{3.145089in}}%
\pgfpathcurveto{\pgfqpoint{2.066511in}{3.145089in}}{\pgfqpoint{2.055912in}{3.140699in}}{\pgfqpoint{2.048099in}{3.132885in}}%
\pgfpathcurveto{\pgfqpoint{2.040285in}{3.125072in}}{\pgfqpoint{2.035895in}{3.114473in}}{\pgfqpoint{2.035895in}{3.103422in}}%
\pgfpathcurveto{\pgfqpoint{2.035895in}{3.092372in}}{\pgfqpoint{2.040285in}{3.081773in}}{\pgfqpoint{2.048099in}{3.073960in}}%
\pgfpathcurveto{\pgfqpoint{2.055912in}{3.066146in}}{\pgfqpoint{2.066511in}{3.061756in}}{\pgfqpoint{2.077561in}{3.061756in}}%
\pgfpathclose%
\pgfusepath{stroke,fill}%
\end{pgfscope}%
\begin{pgfscope}%
\pgfpathrectangle{\pgfqpoint{0.481978in}{0.331635in}}{\pgfqpoint{9.300000in}{7.700000in}}%
\pgfusepath{clip}%
\pgfsetbuttcap%
\pgfsetroundjoin%
\definecolor{currentfill}{rgb}{0.870588,0.733333,0.607843}%
\pgfsetfillcolor{currentfill}%
\pgfsetlinewidth{0.481800pt}%
\definecolor{currentstroke}{rgb}{1.000000,1.000000,1.000000}%
\pgfsetstrokecolor{currentstroke}%
\pgfsetdash{}{0pt}%
\pgfpathmoveto{\pgfqpoint{1.957320in}{5.188017in}}%
\pgfpathcurveto{\pgfqpoint{1.968370in}{5.188017in}}{\pgfqpoint{1.978969in}{5.192407in}}{\pgfqpoint{1.986783in}{5.200221in}}%
\pgfpathcurveto{\pgfqpoint{1.994596in}{5.208035in}}{\pgfqpoint{1.998987in}{5.218634in}}{\pgfqpoint{1.998987in}{5.229684in}}%
\pgfpathcurveto{\pgfqpoint{1.998987in}{5.240734in}}{\pgfqpoint{1.994596in}{5.251333in}}{\pgfqpoint{1.986783in}{5.259147in}}%
\pgfpathcurveto{\pgfqpoint{1.978969in}{5.266960in}}{\pgfqpoint{1.968370in}{5.271351in}}{\pgfqpoint{1.957320in}{5.271351in}}%
\pgfpathcurveto{\pgfqpoint{1.946270in}{5.271351in}}{\pgfqpoint{1.935671in}{5.266960in}}{\pgfqpoint{1.927857in}{5.259147in}}%
\pgfpathcurveto{\pgfqpoint{1.920044in}{5.251333in}}{\pgfqpoint{1.915653in}{5.240734in}}{\pgfqpoint{1.915653in}{5.229684in}}%
\pgfpathcurveto{\pgfqpoint{1.915653in}{5.218634in}}{\pgfqpoint{1.920044in}{5.208035in}}{\pgfqpoint{1.927857in}{5.200221in}}%
\pgfpathcurveto{\pgfqpoint{1.935671in}{5.192407in}}{\pgfqpoint{1.946270in}{5.188017in}}{\pgfqpoint{1.957320in}{5.188017in}}%
\pgfpathclose%
\pgfusepath{stroke,fill}%
\end{pgfscope}%
\begin{pgfscope}%
\pgfpathrectangle{\pgfqpoint{0.481978in}{0.331635in}}{\pgfqpoint{9.300000in}{7.700000in}}%
\pgfusepath{clip}%
\pgfsetbuttcap%
\pgfsetroundjoin%
\definecolor{currentfill}{rgb}{0.870588,0.733333,0.607843}%
\pgfsetfillcolor{currentfill}%
\pgfsetlinewidth{0.481800pt}%
\definecolor{currentstroke}{rgb}{1.000000,1.000000,1.000000}%
\pgfsetstrokecolor{currentstroke}%
\pgfsetdash{}{0pt}%
\pgfpathmoveto{\pgfqpoint{2.660918in}{3.508920in}}%
\pgfpathcurveto{\pgfqpoint{2.671968in}{3.508920in}}{\pgfqpoint{2.682567in}{3.513311in}}{\pgfqpoint{2.690380in}{3.521124in}}%
\pgfpathcurveto{\pgfqpoint{2.698194in}{3.528938in}}{\pgfqpoint{2.702584in}{3.539537in}}{\pgfqpoint{2.702584in}{3.550587in}}%
\pgfpathcurveto{\pgfqpoint{2.702584in}{3.561637in}}{\pgfqpoint{2.698194in}{3.572236in}}{\pgfqpoint{2.690380in}{3.580050in}}%
\pgfpathcurveto{\pgfqpoint{2.682567in}{3.587864in}}{\pgfqpoint{2.671968in}{3.592254in}}{\pgfqpoint{2.660918in}{3.592254in}}%
\pgfpathcurveto{\pgfqpoint{2.649868in}{3.592254in}}{\pgfqpoint{2.639269in}{3.587864in}}{\pgfqpoint{2.631455in}{3.580050in}}%
\pgfpathcurveto{\pgfqpoint{2.623641in}{3.572236in}}{\pgfqpoint{2.619251in}{3.561637in}}{\pgfqpoint{2.619251in}{3.550587in}}%
\pgfpathcurveto{\pgfqpoint{2.619251in}{3.539537in}}{\pgfqpoint{2.623641in}{3.528938in}}{\pgfqpoint{2.631455in}{3.521124in}}%
\pgfpathcurveto{\pgfqpoint{2.639269in}{3.513311in}}{\pgfqpoint{2.649868in}{3.508920in}}{\pgfqpoint{2.660918in}{3.508920in}}%
\pgfpathclose%
\pgfusepath{stroke,fill}%
\end{pgfscope}%
\begin{pgfscope}%
\pgfpathrectangle{\pgfqpoint{0.481978in}{0.331635in}}{\pgfqpoint{9.300000in}{7.700000in}}%
\pgfusepath{clip}%
\pgfsetbuttcap%
\pgfsetroundjoin%
\definecolor{currentfill}{rgb}{0.870588,0.733333,0.607843}%
\pgfsetfillcolor{currentfill}%
\pgfsetlinewidth{0.481800pt}%
\definecolor{currentstroke}{rgb}{1.000000,1.000000,1.000000}%
\pgfsetstrokecolor{currentstroke}%
\pgfsetdash{}{0pt}%
\pgfpathmoveto{\pgfqpoint{7.871019in}{5.508865in}}%
\pgfpathcurveto{\pgfqpoint{7.882069in}{5.508865in}}{\pgfqpoint{7.892668in}{5.513255in}}{\pgfqpoint{7.900481in}{5.521069in}}%
\pgfpathcurveto{\pgfqpoint{7.908295in}{5.528882in}}{\pgfqpoint{7.912685in}{5.539481in}}{\pgfqpoint{7.912685in}{5.550531in}}%
\pgfpathcurveto{\pgfqpoint{7.912685in}{5.561582in}}{\pgfqpoint{7.908295in}{5.572181in}}{\pgfqpoint{7.900481in}{5.579994in}}%
\pgfpathcurveto{\pgfqpoint{7.892668in}{5.587808in}}{\pgfqpoint{7.882069in}{5.592198in}}{\pgfqpoint{7.871019in}{5.592198in}}%
\pgfpathcurveto{\pgfqpoint{7.859968in}{5.592198in}}{\pgfqpoint{7.849369in}{5.587808in}}{\pgfqpoint{7.841556in}{5.579994in}}%
\pgfpathcurveto{\pgfqpoint{7.833742in}{5.572181in}}{\pgfqpoint{7.829352in}{5.561582in}}{\pgfqpoint{7.829352in}{5.550531in}}%
\pgfpathcurveto{\pgfqpoint{7.829352in}{5.539481in}}{\pgfqpoint{7.833742in}{5.528882in}}{\pgfqpoint{7.841556in}{5.521069in}}%
\pgfpathcurveto{\pgfqpoint{7.849369in}{5.513255in}}{\pgfqpoint{7.859968in}{5.508865in}}{\pgfqpoint{7.871019in}{5.508865in}}%
\pgfpathclose%
\pgfusepath{stroke,fill}%
\end{pgfscope}%
\begin{pgfscope}%
\pgfpathrectangle{\pgfqpoint{0.481978in}{0.331635in}}{\pgfqpoint{9.300000in}{7.700000in}}%
\pgfusepath{clip}%
\pgfsetbuttcap%
\pgfsetroundjoin%
\definecolor{currentfill}{rgb}{0.870588,0.733333,0.607843}%
\pgfsetfillcolor{currentfill}%
\pgfsetlinewidth{0.481800pt}%
\definecolor{currentstroke}{rgb}{1.000000,1.000000,1.000000}%
\pgfsetstrokecolor{currentstroke}%
\pgfsetdash{}{0pt}%
\pgfpathmoveto{\pgfqpoint{5.367711in}{4.135078in}}%
\pgfpathcurveto{\pgfqpoint{5.378761in}{4.135078in}}{\pgfqpoint{5.389360in}{4.139468in}}{\pgfqpoint{5.397173in}{4.147281in}}%
\pgfpathcurveto{\pgfqpoint{5.404987in}{4.155095in}}{\pgfqpoint{5.409377in}{4.165694in}}{\pgfqpoint{5.409377in}{4.176744in}}%
\pgfpathcurveto{\pgfqpoint{5.409377in}{4.187794in}}{\pgfqpoint{5.404987in}{4.198393in}}{\pgfqpoint{5.397173in}{4.206207in}}%
\pgfpathcurveto{\pgfqpoint{5.389360in}{4.214021in}}{\pgfqpoint{5.378761in}{4.218411in}}{\pgfqpoint{5.367711in}{4.218411in}}%
\pgfpathcurveto{\pgfqpoint{5.356660in}{4.218411in}}{\pgfqpoint{5.346061in}{4.214021in}}{\pgfqpoint{5.338248in}{4.206207in}}%
\pgfpathcurveto{\pgfqpoint{5.330434in}{4.198393in}}{\pgfqpoint{5.326044in}{4.187794in}}{\pgfqpoint{5.326044in}{4.176744in}}%
\pgfpathcurveto{\pgfqpoint{5.326044in}{4.165694in}}{\pgfqpoint{5.330434in}{4.155095in}}{\pgfqpoint{5.338248in}{4.147281in}}%
\pgfpathcurveto{\pgfqpoint{5.346061in}{4.139468in}}{\pgfqpoint{5.356660in}{4.135078in}}{\pgfqpoint{5.367711in}{4.135078in}}%
\pgfpathclose%
\pgfusepath{stroke,fill}%
\end{pgfscope}%
\begin{pgfscope}%
\pgfpathrectangle{\pgfqpoint{0.481978in}{0.331635in}}{\pgfqpoint{9.300000in}{7.700000in}}%
\pgfusepath{clip}%
\pgfsetbuttcap%
\pgfsetroundjoin%
\definecolor{currentfill}{rgb}{0.870588,0.733333,0.607843}%
\pgfsetfillcolor{currentfill}%
\pgfsetlinewidth{0.481800pt}%
\definecolor{currentstroke}{rgb}{1.000000,1.000000,1.000000}%
\pgfsetstrokecolor{currentstroke}%
\pgfsetdash{}{0pt}%
\pgfpathmoveto{\pgfqpoint{2.321412in}{4.202976in}}%
\pgfpathcurveto{\pgfqpoint{2.332462in}{4.202976in}}{\pgfqpoint{2.343061in}{4.207366in}}{\pgfqpoint{2.350875in}{4.215179in}}%
\pgfpathcurveto{\pgfqpoint{2.358689in}{4.222993in}}{\pgfqpoint{2.363079in}{4.233592in}}{\pgfqpoint{2.363079in}{4.244642in}}%
\pgfpathcurveto{\pgfqpoint{2.363079in}{4.255692in}}{\pgfqpoint{2.358689in}{4.266291in}}{\pgfqpoint{2.350875in}{4.274105in}}%
\pgfpathcurveto{\pgfqpoint{2.343061in}{4.281919in}}{\pgfqpoint{2.332462in}{4.286309in}}{\pgfqpoint{2.321412in}{4.286309in}}%
\pgfpathcurveto{\pgfqpoint{2.310362in}{4.286309in}}{\pgfqpoint{2.299763in}{4.281919in}}{\pgfqpoint{2.291949in}{4.274105in}}%
\pgfpathcurveto{\pgfqpoint{2.284136in}{4.266291in}}{\pgfqpoint{2.279745in}{4.255692in}}{\pgfqpoint{2.279745in}{4.244642in}}%
\pgfpathcurveto{\pgfqpoint{2.279745in}{4.233592in}}{\pgfqpoint{2.284136in}{4.222993in}}{\pgfqpoint{2.291949in}{4.215179in}}%
\pgfpathcurveto{\pgfqpoint{2.299763in}{4.207366in}}{\pgfqpoint{2.310362in}{4.202976in}}{\pgfqpoint{2.321412in}{4.202976in}}%
\pgfpathclose%
\pgfusepath{stroke,fill}%
\end{pgfscope}%
\begin{pgfscope}%
\pgfpathrectangle{\pgfqpoint{0.481978in}{0.331635in}}{\pgfqpoint{9.300000in}{7.700000in}}%
\pgfusepath{clip}%
\pgfsetbuttcap%
\pgfsetroundjoin%
\definecolor{currentfill}{rgb}{0.870588,0.733333,0.607843}%
\pgfsetfillcolor{currentfill}%
\pgfsetlinewidth{0.481800pt}%
\definecolor{currentstroke}{rgb}{1.000000,1.000000,1.000000}%
\pgfsetstrokecolor{currentstroke}%
\pgfsetdash{}{0pt}%
\pgfpathmoveto{\pgfqpoint{7.119603in}{6.261982in}}%
\pgfpathcurveto{\pgfqpoint{7.130653in}{6.261982in}}{\pgfqpoint{7.141252in}{6.266372in}}{\pgfqpoint{7.149065in}{6.274186in}}%
\pgfpathcurveto{\pgfqpoint{7.156879in}{6.282000in}}{\pgfqpoint{7.161269in}{6.292599in}}{\pgfqpoint{7.161269in}{6.303649in}}%
\pgfpathcurveto{\pgfqpoint{7.161269in}{6.314699in}}{\pgfqpoint{7.156879in}{6.325298in}}{\pgfqpoint{7.149065in}{6.333111in}}%
\pgfpathcurveto{\pgfqpoint{7.141252in}{6.340925in}}{\pgfqpoint{7.130653in}{6.345315in}}{\pgfqpoint{7.119603in}{6.345315in}}%
\pgfpathcurveto{\pgfqpoint{7.108553in}{6.345315in}}{\pgfqpoint{7.097954in}{6.340925in}}{\pgfqpoint{7.090140in}{6.333111in}}%
\pgfpathcurveto{\pgfqpoint{7.082326in}{6.325298in}}{\pgfqpoint{7.077936in}{6.314699in}}{\pgfqpoint{7.077936in}{6.303649in}}%
\pgfpathcurveto{\pgfqpoint{7.077936in}{6.292599in}}{\pgfqpoint{7.082326in}{6.282000in}}{\pgfqpoint{7.090140in}{6.274186in}}%
\pgfpathcurveto{\pgfqpoint{7.097954in}{6.266372in}}{\pgfqpoint{7.108553in}{6.261982in}}{\pgfqpoint{7.119603in}{6.261982in}}%
\pgfpathclose%
\pgfusepath{stroke,fill}%
\end{pgfscope}%
\begin{pgfscope}%
\pgfpathrectangle{\pgfqpoint{0.481978in}{0.331635in}}{\pgfqpoint{9.300000in}{7.700000in}}%
\pgfusepath{clip}%
\pgfsetbuttcap%
\pgfsetroundjoin%
\definecolor{currentfill}{rgb}{0.870588,0.733333,0.607843}%
\pgfsetfillcolor{currentfill}%
\pgfsetlinewidth{0.481800pt}%
\definecolor{currentstroke}{rgb}{1.000000,1.000000,1.000000}%
\pgfsetstrokecolor{currentstroke}%
\pgfsetdash{}{0pt}%
\pgfpathmoveto{\pgfqpoint{6.487284in}{5.462464in}}%
\pgfpathcurveto{\pgfqpoint{6.498334in}{5.462464in}}{\pgfqpoint{6.508933in}{5.466854in}}{\pgfqpoint{6.516747in}{5.474668in}}%
\pgfpathcurveto{\pgfqpoint{6.524560in}{5.482482in}}{\pgfqpoint{6.528951in}{5.493081in}}{\pgfqpoint{6.528951in}{5.504131in}}%
\pgfpathcurveto{\pgfqpoint{6.528951in}{5.515181in}}{\pgfqpoint{6.524560in}{5.525780in}}{\pgfqpoint{6.516747in}{5.533594in}}%
\pgfpathcurveto{\pgfqpoint{6.508933in}{5.541407in}}{\pgfqpoint{6.498334in}{5.545798in}}{\pgfqpoint{6.487284in}{5.545798in}}%
\pgfpathcurveto{\pgfqpoint{6.476234in}{5.545798in}}{\pgfqpoint{6.465635in}{5.541407in}}{\pgfqpoint{6.457821in}{5.533594in}}%
\pgfpathcurveto{\pgfqpoint{6.450008in}{5.525780in}}{\pgfqpoint{6.445617in}{5.515181in}}{\pgfqpoint{6.445617in}{5.504131in}}%
\pgfpathcurveto{\pgfqpoint{6.445617in}{5.493081in}}{\pgfqpoint{6.450008in}{5.482482in}}{\pgfqpoint{6.457821in}{5.474668in}}%
\pgfpathcurveto{\pgfqpoint{6.465635in}{5.466854in}}{\pgfqpoint{6.476234in}{5.462464in}}{\pgfqpoint{6.487284in}{5.462464in}}%
\pgfpathclose%
\pgfusepath{stroke,fill}%
\end{pgfscope}%
\begin{pgfscope}%
\pgfpathrectangle{\pgfqpoint{0.481978in}{0.331635in}}{\pgfqpoint{9.300000in}{7.700000in}}%
\pgfusepath{clip}%
\pgfsetbuttcap%
\pgfsetroundjoin%
\definecolor{currentfill}{rgb}{0.870588,0.733333,0.607843}%
\pgfsetfillcolor{currentfill}%
\pgfsetlinewidth{0.481800pt}%
\definecolor{currentstroke}{rgb}{1.000000,1.000000,1.000000}%
\pgfsetstrokecolor{currentstroke}%
\pgfsetdash{}{0pt}%
\pgfpathmoveto{\pgfqpoint{8.920274in}{3.513533in}}%
\pgfpathcurveto{\pgfqpoint{8.931324in}{3.513533in}}{\pgfqpoint{8.941923in}{3.517923in}}{\pgfqpoint{8.949737in}{3.525737in}}%
\pgfpathcurveto{\pgfqpoint{8.957550in}{3.533551in}}{\pgfqpoint{8.961941in}{3.544150in}}{\pgfqpoint{8.961941in}{3.555200in}}%
\pgfpathcurveto{\pgfqpoint{8.961941in}{3.566250in}}{\pgfqpoint{8.957550in}{3.576849in}}{\pgfqpoint{8.949737in}{3.584663in}}%
\pgfpathcurveto{\pgfqpoint{8.941923in}{3.592476in}}{\pgfqpoint{8.931324in}{3.596867in}}{\pgfqpoint{8.920274in}{3.596867in}}%
\pgfpathcurveto{\pgfqpoint{8.909224in}{3.596867in}}{\pgfqpoint{8.898625in}{3.592476in}}{\pgfqpoint{8.890811in}{3.584663in}}%
\pgfpathcurveto{\pgfqpoint{8.882998in}{3.576849in}}{\pgfqpoint{8.878607in}{3.566250in}}{\pgfqpoint{8.878607in}{3.555200in}}%
\pgfpathcurveto{\pgfqpoint{8.878607in}{3.544150in}}{\pgfqpoint{8.882998in}{3.533551in}}{\pgfqpoint{8.890811in}{3.525737in}}%
\pgfpathcurveto{\pgfqpoint{8.898625in}{3.517923in}}{\pgfqpoint{8.909224in}{3.513533in}}{\pgfqpoint{8.920274in}{3.513533in}}%
\pgfpathclose%
\pgfusepath{stroke,fill}%
\end{pgfscope}%
\begin{pgfscope}%
\pgfpathrectangle{\pgfqpoint{0.481978in}{0.331635in}}{\pgfqpoint{9.300000in}{7.700000in}}%
\pgfusepath{clip}%
\pgfsetbuttcap%
\pgfsetroundjoin%
\definecolor{currentfill}{rgb}{0.870588,0.733333,0.607843}%
\pgfsetfillcolor{currentfill}%
\pgfsetlinewidth{0.481800pt}%
\definecolor{currentstroke}{rgb}{1.000000,1.000000,1.000000}%
\pgfsetstrokecolor{currentstroke}%
\pgfsetdash{}{0pt}%
\pgfpathmoveto{\pgfqpoint{7.303090in}{6.103993in}}%
\pgfpathcurveto{\pgfqpoint{7.314140in}{6.103993in}}{\pgfqpoint{7.324739in}{6.108384in}}{\pgfqpoint{7.332553in}{6.116197in}}%
\pgfpathcurveto{\pgfqpoint{7.340366in}{6.124011in}}{\pgfqpoint{7.344757in}{6.134610in}}{\pgfqpoint{7.344757in}{6.145660in}}%
\pgfpathcurveto{\pgfqpoint{7.344757in}{6.156710in}}{\pgfqpoint{7.340366in}{6.167309in}}{\pgfqpoint{7.332553in}{6.175123in}}%
\pgfpathcurveto{\pgfqpoint{7.324739in}{6.182936in}}{\pgfqpoint{7.314140in}{6.187327in}}{\pgfqpoint{7.303090in}{6.187327in}}%
\pgfpathcurveto{\pgfqpoint{7.292040in}{6.187327in}}{\pgfqpoint{7.281441in}{6.182936in}}{\pgfqpoint{7.273627in}{6.175123in}}%
\pgfpathcurveto{\pgfqpoint{7.265814in}{6.167309in}}{\pgfqpoint{7.261423in}{6.156710in}}{\pgfqpoint{7.261423in}{6.145660in}}%
\pgfpathcurveto{\pgfqpoint{7.261423in}{6.134610in}}{\pgfqpoint{7.265814in}{6.124011in}}{\pgfqpoint{7.273627in}{6.116197in}}%
\pgfpathcurveto{\pgfqpoint{7.281441in}{6.108384in}}{\pgfqpoint{7.292040in}{6.103993in}}{\pgfqpoint{7.303090in}{6.103993in}}%
\pgfpathclose%
\pgfusepath{stroke,fill}%
\end{pgfscope}%
\begin{pgfscope}%
\pgfpathrectangle{\pgfqpoint{0.481978in}{0.331635in}}{\pgfqpoint{9.300000in}{7.700000in}}%
\pgfusepath{clip}%
\pgfsetbuttcap%
\pgfsetroundjoin%
\definecolor{currentfill}{rgb}{0.870588,0.733333,0.607843}%
\pgfsetfillcolor{currentfill}%
\pgfsetlinewidth{0.481800pt}%
\definecolor{currentstroke}{rgb}{1.000000,1.000000,1.000000}%
\pgfsetstrokecolor{currentstroke}%
\pgfsetdash{}{0pt}%
\pgfpathmoveto{\pgfqpoint{9.189203in}{4.278681in}}%
\pgfpathcurveto{\pgfqpoint{9.200253in}{4.278681in}}{\pgfqpoint{9.210852in}{4.283071in}}{\pgfqpoint{9.218666in}{4.290885in}}%
\pgfpathcurveto{\pgfqpoint{9.226480in}{4.298699in}}{\pgfqpoint{9.230870in}{4.309298in}}{\pgfqpoint{9.230870in}{4.320348in}}%
\pgfpathcurveto{\pgfqpoint{9.230870in}{4.331398in}}{\pgfqpoint{9.226480in}{4.341997in}}{\pgfqpoint{9.218666in}{4.349811in}}%
\pgfpathcurveto{\pgfqpoint{9.210852in}{4.357624in}}{\pgfqpoint{9.200253in}{4.362014in}}{\pgfqpoint{9.189203in}{4.362014in}}%
\pgfpathcurveto{\pgfqpoint{9.178153in}{4.362014in}}{\pgfqpoint{9.167554in}{4.357624in}}{\pgfqpoint{9.159740in}{4.349811in}}%
\pgfpathcurveto{\pgfqpoint{9.151927in}{4.341997in}}{\pgfqpoint{9.147537in}{4.331398in}}{\pgfqpoint{9.147537in}{4.320348in}}%
\pgfpathcurveto{\pgfqpoint{9.147537in}{4.309298in}}{\pgfqpoint{9.151927in}{4.298699in}}{\pgfqpoint{9.159740in}{4.290885in}}%
\pgfpathcurveto{\pgfqpoint{9.167554in}{4.283071in}}{\pgfqpoint{9.178153in}{4.278681in}}{\pgfqpoint{9.189203in}{4.278681in}}%
\pgfpathclose%
\pgfusepath{stroke,fill}%
\end{pgfscope}%
\begin{pgfscope}%
\pgfpathrectangle{\pgfqpoint{0.481978in}{0.331635in}}{\pgfqpoint{9.300000in}{7.700000in}}%
\pgfusepath{clip}%
\pgfsetbuttcap%
\pgfsetroundjoin%
\definecolor{currentfill}{rgb}{0.631373,0.788235,0.956863}%
\pgfsetfillcolor{currentfill}%
\pgfsetlinewidth{1.003750pt}%
\definecolor{currentstroke}{rgb}{0.631373,0.788235,0.956863}%
\pgfsetstrokecolor{currentstroke}%
\pgfsetdash{}{0pt}%
\pgfsys@defobject{currentmarker}{\pgfqpoint{-0.041667in}{-0.041667in}}{\pgfqpoint{0.041667in}{0.041667in}}{%
\pgfpathmoveto{\pgfqpoint{0.000000in}{-0.041667in}}%
\pgfpathcurveto{\pgfqpoint{0.011050in}{-0.041667in}}{\pgfqpoint{0.021649in}{-0.037276in}}{\pgfqpoint{0.029463in}{-0.029463in}}%
\pgfpathcurveto{\pgfqpoint{0.037276in}{-0.021649in}}{\pgfqpoint{0.041667in}{-0.011050in}}{\pgfqpoint{0.041667in}{0.000000in}}%
\pgfpathcurveto{\pgfqpoint{0.041667in}{0.011050in}}{\pgfqpoint{0.037276in}{0.021649in}}{\pgfqpoint{0.029463in}{0.029463in}}%
\pgfpathcurveto{\pgfqpoint{0.021649in}{0.037276in}}{\pgfqpoint{0.011050in}{0.041667in}}{\pgfqpoint{0.000000in}{0.041667in}}%
\pgfpathcurveto{\pgfqpoint{-0.011050in}{0.041667in}}{\pgfqpoint{-0.021649in}{0.037276in}}{\pgfqpoint{-0.029463in}{0.029463in}}%
\pgfpathcurveto{\pgfqpoint{-0.037276in}{0.021649in}}{\pgfqpoint{-0.041667in}{0.011050in}}{\pgfqpoint{-0.041667in}{0.000000in}}%
\pgfpathcurveto{\pgfqpoint{-0.041667in}{-0.011050in}}{\pgfqpoint{-0.037276in}{-0.021649in}}{\pgfqpoint{-0.029463in}{-0.029463in}}%
\pgfpathcurveto{\pgfqpoint{-0.021649in}{-0.037276in}}{\pgfqpoint{-0.011050in}{-0.041667in}}{\pgfqpoint{0.000000in}{-0.041667in}}%
\pgfpathclose%
\pgfusepath{stroke,fill}%
}%
\end{pgfscope}%
\begin{pgfscope}%
\pgfpathrectangle{\pgfqpoint{0.481978in}{0.331635in}}{\pgfqpoint{9.300000in}{7.700000in}}%
\pgfusepath{clip}%
\pgfsetbuttcap%
\pgfsetroundjoin%
\definecolor{currentfill}{rgb}{1.000000,0.705882,0.509804}%
\pgfsetfillcolor{currentfill}%
\pgfsetlinewidth{1.003750pt}%
\definecolor{currentstroke}{rgb}{1.000000,0.705882,0.509804}%
\pgfsetstrokecolor{currentstroke}%
\pgfsetdash{}{0pt}%
\pgfsys@defobject{currentmarker}{\pgfqpoint{-0.041667in}{-0.041667in}}{\pgfqpoint{0.041667in}{0.041667in}}{%
\pgfpathmoveto{\pgfqpoint{0.000000in}{-0.041667in}}%
\pgfpathcurveto{\pgfqpoint{0.011050in}{-0.041667in}}{\pgfqpoint{0.021649in}{-0.037276in}}{\pgfqpoint{0.029463in}{-0.029463in}}%
\pgfpathcurveto{\pgfqpoint{0.037276in}{-0.021649in}}{\pgfqpoint{0.041667in}{-0.011050in}}{\pgfqpoint{0.041667in}{0.000000in}}%
\pgfpathcurveto{\pgfqpoint{0.041667in}{0.011050in}}{\pgfqpoint{0.037276in}{0.021649in}}{\pgfqpoint{0.029463in}{0.029463in}}%
\pgfpathcurveto{\pgfqpoint{0.021649in}{0.037276in}}{\pgfqpoint{0.011050in}{0.041667in}}{\pgfqpoint{0.000000in}{0.041667in}}%
\pgfpathcurveto{\pgfqpoint{-0.011050in}{0.041667in}}{\pgfqpoint{-0.021649in}{0.037276in}}{\pgfqpoint{-0.029463in}{0.029463in}}%
\pgfpathcurveto{\pgfqpoint{-0.037276in}{0.021649in}}{\pgfqpoint{-0.041667in}{0.011050in}}{\pgfqpoint{-0.041667in}{0.000000in}}%
\pgfpathcurveto{\pgfqpoint{-0.041667in}{-0.011050in}}{\pgfqpoint{-0.037276in}{-0.021649in}}{\pgfqpoint{-0.029463in}{-0.029463in}}%
\pgfpathcurveto{\pgfqpoint{-0.021649in}{-0.037276in}}{\pgfqpoint{-0.011050in}{-0.041667in}}{\pgfqpoint{0.000000in}{-0.041667in}}%
\pgfpathclose%
\pgfusepath{stroke,fill}%
}%
\end{pgfscope}%
\begin{pgfscope}%
\pgfpathrectangle{\pgfqpoint{0.481978in}{0.331635in}}{\pgfqpoint{9.300000in}{7.700000in}}%
\pgfusepath{clip}%
\pgfsetbuttcap%
\pgfsetroundjoin%
\definecolor{currentfill}{rgb}{0.552941,0.898039,0.631373}%
\pgfsetfillcolor{currentfill}%
\pgfsetlinewidth{1.003750pt}%
\definecolor{currentstroke}{rgb}{0.552941,0.898039,0.631373}%
\pgfsetstrokecolor{currentstroke}%
\pgfsetdash{}{0pt}%
\pgfsys@defobject{currentmarker}{\pgfqpoint{-0.041667in}{-0.041667in}}{\pgfqpoint{0.041667in}{0.041667in}}{%
\pgfpathmoveto{\pgfqpoint{0.000000in}{-0.041667in}}%
\pgfpathcurveto{\pgfqpoint{0.011050in}{-0.041667in}}{\pgfqpoint{0.021649in}{-0.037276in}}{\pgfqpoint{0.029463in}{-0.029463in}}%
\pgfpathcurveto{\pgfqpoint{0.037276in}{-0.021649in}}{\pgfqpoint{0.041667in}{-0.011050in}}{\pgfqpoint{0.041667in}{0.000000in}}%
\pgfpathcurveto{\pgfqpoint{0.041667in}{0.011050in}}{\pgfqpoint{0.037276in}{0.021649in}}{\pgfqpoint{0.029463in}{0.029463in}}%
\pgfpathcurveto{\pgfqpoint{0.021649in}{0.037276in}}{\pgfqpoint{0.011050in}{0.041667in}}{\pgfqpoint{0.000000in}{0.041667in}}%
\pgfpathcurveto{\pgfqpoint{-0.011050in}{0.041667in}}{\pgfqpoint{-0.021649in}{0.037276in}}{\pgfqpoint{-0.029463in}{0.029463in}}%
\pgfpathcurveto{\pgfqpoint{-0.037276in}{0.021649in}}{\pgfqpoint{-0.041667in}{0.011050in}}{\pgfqpoint{-0.041667in}{0.000000in}}%
\pgfpathcurveto{\pgfqpoint{-0.041667in}{-0.011050in}}{\pgfqpoint{-0.037276in}{-0.021649in}}{\pgfqpoint{-0.029463in}{-0.029463in}}%
\pgfpathcurveto{\pgfqpoint{-0.021649in}{-0.037276in}}{\pgfqpoint{-0.011050in}{-0.041667in}}{\pgfqpoint{0.000000in}{-0.041667in}}%
\pgfpathclose%
\pgfusepath{stroke,fill}%
}%
\end{pgfscope}%
\begin{pgfscope}%
\pgfpathrectangle{\pgfqpoint{0.481978in}{0.331635in}}{\pgfqpoint{9.300000in}{7.700000in}}%
\pgfusepath{clip}%
\pgfsetbuttcap%
\pgfsetroundjoin%
\definecolor{currentfill}{rgb}{1.000000,0.623529,0.607843}%
\pgfsetfillcolor{currentfill}%
\pgfsetlinewidth{1.003750pt}%
\definecolor{currentstroke}{rgb}{1.000000,0.623529,0.607843}%
\pgfsetstrokecolor{currentstroke}%
\pgfsetdash{}{0pt}%
\pgfsys@defobject{currentmarker}{\pgfqpoint{-0.041667in}{-0.041667in}}{\pgfqpoint{0.041667in}{0.041667in}}{%
\pgfpathmoveto{\pgfqpoint{0.000000in}{-0.041667in}}%
\pgfpathcurveto{\pgfqpoint{0.011050in}{-0.041667in}}{\pgfqpoint{0.021649in}{-0.037276in}}{\pgfqpoint{0.029463in}{-0.029463in}}%
\pgfpathcurveto{\pgfqpoint{0.037276in}{-0.021649in}}{\pgfqpoint{0.041667in}{-0.011050in}}{\pgfqpoint{0.041667in}{0.000000in}}%
\pgfpathcurveto{\pgfqpoint{0.041667in}{0.011050in}}{\pgfqpoint{0.037276in}{0.021649in}}{\pgfqpoint{0.029463in}{0.029463in}}%
\pgfpathcurveto{\pgfqpoint{0.021649in}{0.037276in}}{\pgfqpoint{0.011050in}{0.041667in}}{\pgfqpoint{0.000000in}{0.041667in}}%
\pgfpathcurveto{\pgfqpoint{-0.011050in}{0.041667in}}{\pgfqpoint{-0.021649in}{0.037276in}}{\pgfqpoint{-0.029463in}{0.029463in}}%
\pgfpathcurveto{\pgfqpoint{-0.037276in}{0.021649in}}{\pgfqpoint{-0.041667in}{0.011050in}}{\pgfqpoint{-0.041667in}{0.000000in}}%
\pgfpathcurveto{\pgfqpoint{-0.041667in}{-0.011050in}}{\pgfqpoint{-0.037276in}{-0.021649in}}{\pgfqpoint{-0.029463in}{-0.029463in}}%
\pgfpathcurveto{\pgfqpoint{-0.021649in}{-0.037276in}}{\pgfqpoint{-0.011050in}{-0.041667in}}{\pgfqpoint{0.000000in}{-0.041667in}}%
\pgfpathclose%
\pgfusepath{stroke,fill}%
}%
\end{pgfscope}%
\begin{pgfscope}%
\pgfpathrectangle{\pgfqpoint{0.481978in}{0.331635in}}{\pgfqpoint{9.300000in}{7.700000in}}%
\pgfusepath{clip}%
\pgfsetbuttcap%
\pgfsetroundjoin%
\definecolor{currentfill}{rgb}{0.815686,0.733333,1.000000}%
\pgfsetfillcolor{currentfill}%
\pgfsetlinewidth{1.003750pt}%
\definecolor{currentstroke}{rgb}{0.815686,0.733333,1.000000}%
\pgfsetstrokecolor{currentstroke}%
\pgfsetdash{}{0pt}%
\pgfsys@defobject{currentmarker}{\pgfqpoint{-0.041667in}{-0.041667in}}{\pgfqpoint{0.041667in}{0.041667in}}{%
\pgfpathmoveto{\pgfqpoint{0.000000in}{-0.041667in}}%
\pgfpathcurveto{\pgfqpoint{0.011050in}{-0.041667in}}{\pgfqpoint{0.021649in}{-0.037276in}}{\pgfqpoint{0.029463in}{-0.029463in}}%
\pgfpathcurveto{\pgfqpoint{0.037276in}{-0.021649in}}{\pgfqpoint{0.041667in}{-0.011050in}}{\pgfqpoint{0.041667in}{0.000000in}}%
\pgfpathcurveto{\pgfqpoint{0.041667in}{0.011050in}}{\pgfqpoint{0.037276in}{0.021649in}}{\pgfqpoint{0.029463in}{0.029463in}}%
\pgfpathcurveto{\pgfqpoint{0.021649in}{0.037276in}}{\pgfqpoint{0.011050in}{0.041667in}}{\pgfqpoint{0.000000in}{0.041667in}}%
\pgfpathcurveto{\pgfqpoint{-0.011050in}{0.041667in}}{\pgfqpoint{-0.021649in}{0.037276in}}{\pgfqpoint{-0.029463in}{0.029463in}}%
\pgfpathcurveto{\pgfqpoint{-0.037276in}{0.021649in}}{\pgfqpoint{-0.041667in}{0.011050in}}{\pgfqpoint{-0.041667in}{0.000000in}}%
\pgfpathcurveto{\pgfqpoint{-0.041667in}{-0.011050in}}{\pgfqpoint{-0.037276in}{-0.021649in}}{\pgfqpoint{-0.029463in}{-0.029463in}}%
\pgfpathcurveto{\pgfqpoint{-0.021649in}{-0.037276in}}{\pgfqpoint{-0.011050in}{-0.041667in}}{\pgfqpoint{0.000000in}{-0.041667in}}%
\pgfpathclose%
\pgfusepath{stroke,fill}%
}%
\end{pgfscope}%
\begin{pgfscope}%
\pgfpathrectangle{\pgfqpoint{0.481978in}{0.331635in}}{\pgfqpoint{9.300000in}{7.700000in}}%
\pgfusepath{clip}%
\pgfsetbuttcap%
\pgfsetroundjoin%
\definecolor{currentfill}{rgb}{0.870588,0.733333,0.607843}%
\pgfsetfillcolor{currentfill}%
\pgfsetlinewidth{1.003750pt}%
\definecolor{currentstroke}{rgb}{0.870588,0.733333,0.607843}%
\pgfsetstrokecolor{currentstroke}%
\pgfsetdash{}{0pt}%
\pgfsys@defobject{currentmarker}{\pgfqpoint{-0.041667in}{-0.041667in}}{\pgfqpoint{0.041667in}{0.041667in}}{%
\pgfpathmoveto{\pgfqpoint{0.000000in}{-0.041667in}}%
\pgfpathcurveto{\pgfqpoint{0.011050in}{-0.041667in}}{\pgfqpoint{0.021649in}{-0.037276in}}{\pgfqpoint{0.029463in}{-0.029463in}}%
\pgfpathcurveto{\pgfqpoint{0.037276in}{-0.021649in}}{\pgfqpoint{0.041667in}{-0.011050in}}{\pgfqpoint{0.041667in}{0.000000in}}%
\pgfpathcurveto{\pgfqpoint{0.041667in}{0.011050in}}{\pgfqpoint{0.037276in}{0.021649in}}{\pgfqpoint{0.029463in}{0.029463in}}%
\pgfpathcurveto{\pgfqpoint{0.021649in}{0.037276in}}{\pgfqpoint{0.011050in}{0.041667in}}{\pgfqpoint{0.000000in}{0.041667in}}%
\pgfpathcurveto{\pgfqpoint{-0.011050in}{0.041667in}}{\pgfqpoint{-0.021649in}{0.037276in}}{\pgfqpoint{-0.029463in}{0.029463in}}%
\pgfpathcurveto{\pgfqpoint{-0.037276in}{0.021649in}}{\pgfqpoint{-0.041667in}{0.011050in}}{\pgfqpoint{-0.041667in}{0.000000in}}%
\pgfpathcurveto{\pgfqpoint{-0.041667in}{-0.011050in}}{\pgfqpoint{-0.037276in}{-0.021649in}}{\pgfqpoint{-0.029463in}{-0.029463in}}%
\pgfpathcurveto{\pgfqpoint{-0.021649in}{-0.037276in}}{\pgfqpoint{-0.011050in}{-0.041667in}}{\pgfqpoint{0.000000in}{-0.041667in}}%
\pgfpathclose%
\pgfusepath{stroke,fill}%
}%
\end{pgfscope}%
\begin{pgfscope}%
\pgfsetbuttcap%
\pgfsetroundjoin%
\definecolor{currentfill}{rgb}{0.000000,0.000000,0.000000}%
\pgfsetfillcolor{currentfill}%
\pgfsetlinewidth{0.803000pt}%
\definecolor{currentstroke}{rgb}{0.000000,0.000000,0.000000}%
\pgfsetstrokecolor{currentstroke}%
\pgfsetdash{}{0pt}%
\pgfsys@defobject{currentmarker}{\pgfqpoint{0.000000in}{-0.048611in}}{\pgfqpoint{0.000000in}{0.000000in}}{%
\pgfpathmoveto{\pgfqpoint{0.000000in}{0.000000in}}%
\pgfpathlineto{\pgfqpoint{0.000000in}{-0.048611in}}%
\pgfusepath{stroke,fill}%
}%
\begin{pgfscope}%
\pgfsys@transformshift{0.980433in}{0.331635in}%
\pgfsys@useobject{currentmarker}{}%
\end{pgfscope}%
\end{pgfscope}%
\begin{pgfscope}%
\definecolor{textcolor}{rgb}{0.000000,0.000000,0.000000}%
\pgfsetstrokecolor{textcolor}%
\pgfsetfillcolor{textcolor}%
\pgftext[x=0.980433in,y=0.234413in,,top]{\color{textcolor}\sffamily\fontsize{10.000000}{12.000000}\selectfont \ensuremath{-}20}%
\end{pgfscope}%
\begin{pgfscope}%
\pgfsetbuttcap%
\pgfsetroundjoin%
\definecolor{currentfill}{rgb}{0.000000,0.000000,0.000000}%
\pgfsetfillcolor{currentfill}%
\pgfsetlinewidth{0.803000pt}%
\definecolor{currentstroke}{rgb}{0.000000,0.000000,0.000000}%
\pgfsetstrokecolor{currentstroke}%
\pgfsetdash{}{0pt}%
\pgfsys@defobject{currentmarker}{\pgfqpoint{0.000000in}{-0.048611in}}{\pgfqpoint{0.000000in}{0.000000in}}{%
\pgfpathmoveto{\pgfqpoint{0.000000in}{0.000000in}}%
\pgfpathlineto{\pgfqpoint{0.000000in}{-0.048611in}}%
\pgfusepath{stroke,fill}%
}%
\begin{pgfscope}%
\pgfsys@transformshift{2.182672in}{0.331635in}%
\pgfsys@useobject{currentmarker}{}%
\end{pgfscope}%
\end{pgfscope}%
\begin{pgfscope}%
\definecolor{textcolor}{rgb}{0.000000,0.000000,0.000000}%
\pgfsetstrokecolor{textcolor}%
\pgfsetfillcolor{textcolor}%
\pgftext[x=2.182672in,y=0.234413in,,top]{\color{textcolor}\sffamily\fontsize{10.000000}{12.000000}\selectfont \ensuremath{-}15}%
\end{pgfscope}%
\begin{pgfscope}%
\pgfsetbuttcap%
\pgfsetroundjoin%
\definecolor{currentfill}{rgb}{0.000000,0.000000,0.000000}%
\pgfsetfillcolor{currentfill}%
\pgfsetlinewidth{0.803000pt}%
\definecolor{currentstroke}{rgb}{0.000000,0.000000,0.000000}%
\pgfsetstrokecolor{currentstroke}%
\pgfsetdash{}{0pt}%
\pgfsys@defobject{currentmarker}{\pgfqpoint{0.000000in}{-0.048611in}}{\pgfqpoint{0.000000in}{0.000000in}}{%
\pgfpathmoveto{\pgfqpoint{0.000000in}{0.000000in}}%
\pgfpathlineto{\pgfqpoint{0.000000in}{-0.048611in}}%
\pgfusepath{stroke,fill}%
}%
\begin{pgfscope}%
\pgfsys@transformshift{3.384911in}{0.331635in}%
\pgfsys@useobject{currentmarker}{}%
\end{pgfscope}%
\end{pgfscope}%
\begin{pgfscope}%
\definecolor{textcolor}{rgb}{0.000000,0.000000,0.000000}%
\pgfsetstrokecolor{textcolor}%
\pgfsetfillcolor{textcolor}%
\pgftext[x=3.384911in,y=0.234413in,,top]{\color{textcolor}\sffamily\fontsize{10.000000}{12.000000}\selectfont \ensuremath{-}10}%
\end{pgfscope}%
\begin{pgfscope}%
\pgfsetbuttcap%
\pgfsetroundjoin%
\definecolor{currentfill}{rgb}{0.000000,0.000000,0.000000}%
\pgfsetfillcolor{currentfill}%
\pgfsetlinewidth{0.803000pt}%
\definecolor{currentstroke}{rgb}{0.000000,0.000000,0.000000}%
\pgfsetstrokecolor{currentstroke}%
\pgfsetdash{}{0pt}%
\pgfsys@defobject{currentmarker}{\pgfqpoint{0.000000in}{-0.048611in}}{\pgfqpoint{0.000000in}{0.000000in}}{%
\pgfpathmoveto{\pgfqpoint{0.000000in}{0.000000in}}%
\pgfpathlineto{\pgfqpoint{0.000000in}{-0.048611in}}%
\pgfusepath{stroke,fill}%
}%
\begin{pgfscope}%
\pgfsys@transformshift{4.587150in}{0.331635in}%
\pgfsys@useobject{currentmarker}{}%
\end{pgfscope}%
\end{pgfscope}%
\begin{pgfscope}%
\definecolor{textcolor}{rgb}{0.000000,0.000000,0.000000}%
\pgfsetstrokecolor{textcolor}%
\pgfsetfillcolor{textcolor}%
\pgftext[x=4.587150in,y=0.234413in,,top]{\color{textcolor}\sffamily\fontsize{10.000000}{12.000000}\selectfont \ensuremath{-}5}%
\end{pgfscope}%
\begin{pgfscope}%
\pgfsetbuttcap%
\pgfsetroundjoin%
\definecolor{currentfill}{rgb}{0.000000,0.000000,0.000000}%
\pgfsetfillcolor{currentfill}%
\pgfsetlinewidth{0.803000pt}%
\definecolor{currentstroke}{rgb}{0.000000,0.000000,0.000000}%
\pgfsetstrokecolor{currentstroke}%
\pgfsetdash{}{0pt}%
\pgfsys@defobject{currentmarker}{\pgfqpoint{0.000000in}{-0.048611in}}{\pgfqpoint{0.000000in}{0.000000in}}{%
\pgfpathmoveto{\pgfqpoint{0.000000in}{0.000000in}}%
\pgfpathlineto{\pgfqpoint{0.000000in}{-0.048611in}}%
\pgfusepath{stroke,fill}%
}%
\begin{pgfscope}%
\pgfsys@transformshift{5.789389in}{0.331635in}%
\pgfsys@useobject{currentmarker}{}%
\end{pgfscope}%
\end{pgfscope}%
\begin{pgfscope}%
\definecolor{textcolor}{rgb}{0.000000,0.000000,0.000000}%
\pgfsetstrokecolor{textcolor}%
\pgfsetfillcolor{textcolor}%
\pgftext[x=5.789389in,y=0.234413in,,top]{\color{textcolor}\sffamily\fontsize{10.000000}{12.000000}\selectfont 0}%
\end{pgfscope}%
\begin{pgfscope}%
\pgfsetbuttcap%
\pgfsetroundjoin%
\definecolor{currentfill}{rgb}{0.000000,0.000000,0.000000}%
\pgfsetfillcolor{currentfill}%
\pgfsetlinewidth{0.803000pt}%
\definecolor{currentstroke}{rgb}{0.000000,0.000000,0.000000}%
\pgfsetstrokecolor{currentstroke}%
\pgfsetdash{}{0pt}%
\pgfsys@defobject{currentmarker}{\pgfqpoint{0.000000in}{-0.048611in}}{\pgfqpoint{0.000000in}{0.000000in}}{%
\pgfpathmoveto{\pgfqpoint{0.000000in}{0.000000in}}%
\pgfpathlineto{\pgfqpoint{0.000000in}{-0.048611in}}%
\pgfusepath{stroke,fill}%
}%
\begin{pgfscope}%
\pgfsys@transformshift{6.991629in}{0.331635in}%
\pgfsys@useobject{currentmarker}{}%
\end{pgfscope}%
\end{pgfscope}%
\begin{pgfscope}%
\definecolor{textcolor}{rgb}{0.000000,0.000000,0.000000}%
\pgfsetstrokecolor{textcolor}%
\pgfsetfillcolor{textcolor}%
\pgftext[x=6.991629in,y=0.234413in,,top]{\color{textcolor}\sffamily\fontsize{10.000000}{12.000000}\selectfont 5}%
\end{pgfscope}%
\begin{pgfscope}%
\pgfsetbuttcap%
\pgfsetroundjoin%
\definecolor{currentfill}{rgb}{0.000000,0.000000,0.000000}%
\pgfsetfillcolor{currentfill}%
\pgfsetlinewidth{0.803000pt}%
\definecolor{currentstroke}{rgb}{0.000000,0.000000,0.000000}%
\pgfsetstrokecolor{currentstroke}%
\pgfsetdash{}{0pt}%
\pgfsys@defobject{currentmarker}{\pgfqpoint{0.000000in}{-0.048611in}}{\pgfqpoint{0.000000in}{0.000000in}}{%
\pgfpathmoveto{\pgfqpoint{0.000000in}{0.000000in}}%
\pgfpathlineto{\pgfqpoint{0.000000in}{-0.048611in}}%
\pgfusepath{stroke,fill}%
}%
\begin{pgfscope}%
\pgfsys@transformshift{8.193868in}{0.331635in}%
\pgfsys@useobject{currentmarker}{}%
\end{pgfscope}%
\end{pgfscope}%
\begin{pgfscope}%
\definecolor{textcolor}{rgb}{0.000000,0.000000,0.000000}%
\pgfsetstrokecolor{textcolor}%
\pgfsetfillcolor{textcolor}%
\pgftext[x=8.193868in,y=0.234413in,,top]{\color{textcolor}\sffamily\fontsize{10.000000}{12.000000}\selectfont 10}%
\end{pgfscope}%
\begin{pgfscope}%
\pgfsetbuttcap%
\pgfsetroundjoin%
\definecolor{currentfill}{rgb}{0.000000,0.000000,0.000000}%
\pgfsetfillcolor{currentfill}%
\pgfsetlinewidth{0.803000pt}%
\definecolor{currentstroke}{rgb}{0.000000,0.000000,0.000000}%
\pgfsetstrokecolor{currentstroke}%
\pgfsetdash{}{0pt}%
\pgfsys@defobject{currentmarker}{\pgfqpoint{0.000000in}{-0.048611in}}{\pgfqpoint{0.000000in}{0.000000in}}{%
\pgfpathmoveto{\pgfqpoint{0.000000in}{0.000000in}}%
\pgfpathlineto{\pgfqpoint{0.000000in}{-0.048611in}}%
\pgfusepath{stroke,fill}%
}%
\begin{pgfscope}%
\pgfsys@transformshift{9.396107in}{0.331635in}%
\pgfsys@useobject{currentmarker}{}%
\end{pgfscope}%
\end{pgfscope}%
\begin{pgfscope}%
\definecolor{textcolor}{rgb}{0.000000,0.000000,0.000000}%
\pgfsetstrokecolor{textcolor}%
\pgfsetfillcolor{textcolor}%
\pgftext[x=9.396107in,y=0.234413in,,top]{\color{textcolor}\sffamily\fontsize{10.000000}{12.000000}\selectfont 15}%
\end{pgfscope}%
\begin{pgfscope}%
\pgfsetbuttcap%
\pgfsetroundjoin%
\definecolor{currentfill}{rgb}{0.000000,0.000000,0.000000}%
\pgfsetfillcolor{currentfill}%
\pgfsetlinewidth{0.803000pt}%
\definecolor{currentstroke}{rgb}{0.000000,0.000000,0.000000}%
\pgfsetstrokecolor{currentstroke}%
\pgfsetdash{}{0pt}%
\pgfsys@defobject{currentmarker}{\pgfqpoint{-0.048611in}{0.000000in}}{\pgfqpoint{-0.000000in}{0.000000in}}{%
\pgfpathmoveto{\pgfqpoint{-0.000000in}{0.000000in}}%
\pgfpathlineto{\pgfqpoint{-0.048611in}{0.000000in}}%
\pgfusepath{stroke,fill}%
}%
\begin{pgfscope}%
\pgfsys@transformshift{0.481978in}{0.607744in}%
\pgfsys@useobject{currentmarker}{}%
\end{pgfscope}%
\end{pgfscope}%
\begin{pgfscope}%
\definecolor{textcolor}{rgb}{0.000000,0.000000,0.000000}%
\pgfsetstrokecolor{textcolor}%
\pgfsetfillcolor{textcolor}%
\pgftext[x=0.100000in, y=0.554982in, left, base]{\color{textcolor}\sffamily\fontsize{10.000000}{12.000000}\selectfont \ensuremath{-}15}%
\end{pgfscope}%
\begin{pgfscope}%
\pgfsetbuttcap%
\pgfsetroundjoin%
\definecolor{currentfill}{rgb}{0.000000,0.000000,0.000000}%
\pgfsetfillcolor{currentfill}%
\pgfsetlinewidth{0.803000pt}%
\definecolor{currentstroke}{rgb}{0.000000,0.000000,0.000000}%
\pgfsetstrokecolor{currentstroke}%
\pgfsetdash{}{0pt}%
\pgfsys@defobject{currentmarker}{\pgfqpoint{-0.048611in}{0.000000in}}{\pgfqpoint{-0.000000in}{0.000000in}}{%
\pgfpathmoveto{\pgfqpoint{-0.000000in}{0.000000in}}%
\pgfpathlineto{\pgfqpoint{-0.048611in}{0.000000in}}%
\pgfusepath{stroke,fill}%
}%
\begin{pgfscope}%
\pgfsys@transformshift{0.481978in}{1.846943in}%
\pgfsys@useobject{currentmarker}{}%
\end{pgfscope}%
\end{pgfscope}%
\begin{pgfscope}%
\definecolor{textcolor}{rgb}{0.000000,0.000000,0.000000}%
\pgfsetstrokecolor{textcolor}%
\pgfsetfillcolor{textcolor}%
\pgftext[x=0.100000in, y=1.794182in, left, base]{\color{textcolor}\sffamily\fontsize{10.000000}{12.000000}\selectfont \ensuremath{-}10}%
\end{pgfscope}%
\begin{pgfscope}%
\pgfsetbuttcap%
\pgfsetroundjoin%
\definecolor{currentfill}{rgb}{0.000000,0.000000,0.000000}%
\pgfsetfillcolor{currentfill}%
\pgfsetlinewidth{0.803000pt}%
\definecolor{currentstroke}{rgb}{0.000000,0.000000,0.000000}%
\pgfsetstrokecolor{currentstroke}%
\pgfsetdash{}{0pt}%
\pgfsys@defobject{currentmarker}{\pgfqpoint{-0.048611in}{0.000000in}}{\pgfqpoint{-0.000000in}{0.000000in}}{%
\pgfpathmoveto{\pgfqpoint{-0.000000in}{0.000000in}}%
\pgfpathlineto{\pgfqpoint{-0.048611in}{0.000000in}}%
\pgfusepath{stroke,fill}%
}%
\begin{pgfscope}%
\pgfsys@transformshift{0.481978in}{3.086143in}%
\pgfsys@useobject{currentmarker}{}%
\end{pgfscope}%
\end{pgfscope}%
\begin{pgfscope}%
\definecolor{textcolor}{rgb}{0.000000,0.000000,0.000000}%
\pgfsetstrokecolor{textcolor}%
\pgfsetfillcolor{textcolor}%
\pgftext[x=0.188365in, y=3.033381in, left, base]{\color{textcolor}\sffamily\fontsize{10.000000}{12.000000}\selectfont \ensuremath{-}5}%
\end{pgfscope}%
\begin{pgfscope}%
\pgfsetbuttcap%
\pgfsetroundjoin%
\definecolor{currentfill}{rgb}{0.000000,0.000000,0.000000}%
\pgfsetfillcolor{currentfill}%
\pgfsetlinewidth{0.803000pt}%
\definecolor{currentstroke}{rgb}{0.000000,0.000000,0.000000}%
\pgfsetstrokecolor{currentstroke}%
\pgfsetdash{}{0pt}%
\pgfsys@defobject{currentmarker}{\pgfqpoint{-0.048611in}{0.000000in}}{\pgfqpoint{-0.000000in}{0.000000in}}{%
\pgfpathmoveto{\pgfqpoint{-0.000000in}{0.000000in}}%
\pgfpathlineto{\pgfqpoint{-0.048611in}{0.000000in}}%
\pgfusepath{stroke,fill}%
}%
\begin{pgfscope}%
\pgfsys@transformshift{0.481978in}{4.325342in}%
\pgfsys@useobject{currentmarker}{}%
\end{pgfscope}%
\end{pgfscope}%
\begin{pgfscope}%
\definecolor{textcolor}{rgb}{0.000000,0.000000,0.000000}%
\pgfsetstrokecolor{textcolor}%
\pgfsetfillcolor{textcolor}%
\pgftext[x=0.296390in, y=4.272581in, left, base]{\color{textcolor}\sffamily\fontsize{10.000000}{12.000000}\selectfont 0}%
\end{pgfscope}%
\begin{pgfscope}%
\pgfsetbuttcap%
\pgfsetroundjoin%
\definecolor{currentfill}{rgb}{0.000000,0.000000,0.000000}%
\pgfsetfillcolor{currentfill}%
\pgfsetlinewidth{0.803000pt}%
\definecolor{currentstroke}{rgb}{0.000000,0.000000,0.000000}%
\pgfsetstrokecolor{currentstroke}%
\pgfsetdash{}{0pt}%
\pgfsys@defobject{currentmarker}{\pgfqpoint{-0.048611in}{0.000000in}}{\pgfqpoint{-0.000000in}{0.000000in}}{%
\pgfpathmoveto{\pgfqpoint{-0.000000in}{0.000000in}}%
\pgfpathlineto{\pgfqpoint{-0.048611in}{0.000000in}}%
\pgfusepath{stroke,fill}%
}%
\begin{pgfscope}%
\pgfsys@transformshift{0.481978in}{5.564542in}%
\pgfsys@useobject{currentmarker}{}%
\end{pgfscope}%
\end{pgfscope}%
\begin{pgfscope}%
\definecolor{textcolor}{rgb}{0.000000,0.000000,0.000000}%
\pgfsetstrokecolor{textcolor}%
\pgfsetfillcolor{textcolor}%
\pgftext[x=0.296390in, y=5.511780in, left, base]{\color{textcolor}\sffamily\fontsize{10.000000}{12.000000}\selectfont 5}%
\end{pgfscope}%
\begin{pgfscope}%
\pgfsetbuttcap%
\pgfsetroundjoin%
\definecolor{currentfill}{rgb}{0.000000,0.000000,0.000000}%
\pgfsetfillcolor{currentfill}%
\pgfsetlinewidth{0.803000pt}%
\definecolor{currentstroke}{rgb}{0.000000,0.000000,0.000000}%
\pgfsetstrokecolor{currentstroke}%
\pgfsetdash{}{0pt}%
\pgfsys@defobject{currentmarker}{\pgfqpoint{-0.048611in}{0.000000in}}{\pgfqpoint{-0.000000in}{0.000000in}}{%
\pgfpathmoveto{\pgfqpoint{-0.000000in}{0.000000in}}%
\pgfpathlineto{\pgfqpoint{-0.048611in}{0.000000in}}%
\pgfusepath{stroke,fill}%
}%
\begin{pgfscope}%
\pgfsys@transformshift{0.481978in}{6.803741in}%
\pgfsys@useobject{currentmarker}{}%
\end{pgfscope}%
\end{pgfscope}%
\begin{pgfscope}%
\definecolor{textcolor}{rgb}{0.000000,0.000000,0.000000}%
\pgfsetstrokecolor{textcolor}%
\pgfsetfillcolor{textcolor}%
\pgftext[x=0.208025in, y=6.750979in, left, base]{\color{textcolor}\sffamily\fontsize{10.000000}{12.000000}\selectfont 10}%
\end{pgfscope}%
\begin{pgfscope}%
\pgfpathrectangle{\pgfqpoint{0.481978in}{0.331635in}}{\pgfqpoint{9.300000in}{7.700000in}}%
\pgfusepath{clip}%
\pgfsetrectcap%
\pgfsetroundjoin%
\pgfsetlinewidth{1.505625pt}%
\definecolor{currentstroke}{rgb}{0.631373,0.788235,0.956863}%
\pgfsetstrokecolor{currentstroke}%
\pgfsetstrokeopacity{0.200000}%
\pgfsetdash{}{0pt}%
\pgfpathmoveto{\pgfqpoint{3.350250in}{0.837537in}}%
\pgfpathlineto{\pgfqpoint{3.792680in}{4.507363in}}%
\pgfusepath{stroke}%
\end{pgfscope}%
\begin{pgfscope}%
\pgfpathrectangle{\pgfqpoint{0.481978in}{0.331635in}}{\pgfqpoint{9.300000in}{7.700000in}}%
\pgfusepath{clip}%
\pgfsetrectcap%
\pgfsetroundjoin%
\pgfsetlinewidth{1.505625pt}%
\definecolor{currentstroke}{rgb}{0.631373,0.788235,0.956863}%
\pgfsetstrokecolor{currentstroke}%
\pgfsetstrokeopacity{0.200000}%
\pgfsetdash{}{0pt}%
\pgfpathmoveto{\pgfqpoint{3.059115in}{5.261945in}}%
\pgfpathlineto{\pgfqpoint{3.792680in}{4.507363in}}%
\pgfusepath{stroke}%
\end{pgfscope}%
\begin{pgfscope}%
\pgfpathrectangle{\pgfqpoint{0.481978in}{0.331635in}}{\pgfqpoint{9.300000in}{7.700000in}}%
\pgfusepath{clip}%
\pgfsetrectcap%
\pgfsetroundjoin%
\pgfsetlinewidth{1.505625pt}%
\definecolor{currentstroke}{rgb}{0.631373,0.788235,0.956863}%
\pgfsetstrokecolor{currentstroke}%
\pgfsetstrokeopacity{0.200000}%
\pgfsetdash{}{0pt}%
\pgfpathmoveto{\pgfqpoint{4.845549in}{3.770354in}}%
\pgfpathlineto{\pgfqpoint{3.792680in}{4.507363in}}%
\pgfusepath{stroke}%
\end{pgfscope}%
\begin{pgfscope}%
\pgfpathrectangle{\pgfqpoint{0.481978in}{0.331635in}}{\pgfqpoint{9.300000in}{7.700000in}}%
\pgfusepath{clip}%
\pgfsetrectcap%
\pgfsetroundjoin%
\pgfsetlinewidth{1.505625pt}%
\definecolor{currentstroke}{rgb}{0.631373,0.788235,0.956863}%
\pgfsetstrokecolor{currentstroke}%
\pgfsetstrokeopacity{0.200000}%
\pgfsetdash{}{0pt}%
\pgfpathmoveto{\pgfqpoint{1.304480in}{3.089887in}}%
\pgfpathlineto{\pgfqpoint{3.792680in}{4.507363in}}%
\pgfusepath{stroke}%
\end{pgfscope}%
\begin{pgfscope}%
\pgfpathrectangle{\pgfqpoint{0.481978in}{0.331635in}}{\pgfqpoint{9.300000in}{7.700000in}}%
\pgfusepath{clip}%
\pgfsetrectcap%
\pgfsetroundjoin%
\pgfsetlinewidth{1.505625pt}%
\definecolor{currentstroke}{rgb}{0.631373,0.788235,0.956863}%
\pgfsetstrokecolor{currentstroke}%
\pgfsetstrokeopacity{0.200000}%
\pgfsetdash{}{0pt}%
\pgfpathmoveto{\pgfqpoint{4.583472in}{2.434606in}}%
\pgfpathlineto{\pgfqpoint{3.792680in}{4.507363in}}%
\pgfusepath{stroke}%
\end{pgfscope}%
\begin{pgfscope}%
\pgfpathrectangle{\pgfqpoint{0.481978in}{0.331635in}}{\pgfqpoint{9.300000in}{7.700000in}}%
\pgfusepath{clip}%
\pgfsetrectcap%
\pgfsetroundjoin%
\pgfsetlinewidth{1.505625pt}%
\definecolor{currentstroke}{rgb}{0.631373,0.788235,0.956863}%
\pgfsetstrokecolor{currentstroke}%
\pgfsetstrokeopacity{0.200000}%
\pgfsetdash{}{0pt}%
\pgfpathmoveto{\pgfqpoint{5.509106in}{5.302557in}}%
\pgfpathlineto{\pgfqpoint{3.792680in}{4.507363in}}%
\pgfusepath{stroke}%
\end{pgfscope}%
\begin{pgfscope}%
\pgfpathrectangle{\pgfqpoint{0.481978in}{0.331635in}}{\pgfqpoint{9.300000in}{7.700000in}}%
\pgfusepath{clip}%
\pgfsetrectcap%
\pgfsetroundjoin%
\pgfsetlinewidth{1.505625pt}%
\definecolor{currentstroke}{rgb}{0.631373,0.788235,0.956863}%
\pgfsetstrokecolor{currentstroke}%
\pgfsetstrokeopacity{0.200000}%
\pgfsetdash{}{0pt}%
\pgfpathmoveto{\pgfqpoint{1.270538in}{3.063432in}}%
\pgfpathlineto{\pgfqpoint{3.792680in}{4.507363in}}%
\pgfusepath{stroke}%
\end{pgfscope}%
\begin{pgfscope}%
\pgfpathrectangle{\pgfqpoint{0.481978in}{0.331635in}}{\pgfqpoint{9.300000in}{7.700000in}}%
\pgfusepath{clip}%
\pgfsetrectcap%
\pgfsetroundjoin%
\pgfsetlinewidth{1.505625pt}%
\definecolor{currentstroke}{rgb}{0.631373,0.788235,0.956863}%
\pgfsetstrokecolor{currentstroke}%
\pgfsetstrokeopacity{0.200000}%
\pgfsetdash{}{0pt}%
\pgfpathmoveto{\pgfqpoint{4.233419in}{2.543996in}}%
\pgfpathlineto{\pgfqpoint{3.792680in}{4.507363in}}%
\pgfusepath{stroke}%
\end{pgfscope}%
\begin{pgfscope}%
\pgfpathrectangle{\pgfqpoint{0.481978in}{0.331635in}}{\pgfqpoint{9.300000in}{7.700000in}}%
\pgfusepath{clip}%
\pgfsetrectcap%
\pgfsetroundjoin%
\pgfsetlinewidth{1.505625pt}%
\definecolor{currentstroke}{rgb}{0.631373,0.788235,0.956863}%
\pgfsetstrokecolor{currentstroke}%
\pgfsetstrokeopacity{0.200000}%
\pgfsetdash{}{0pt}%
\pgfpathmoveto{\pgfqpoint{2.873447in}{4.588040in}}%
\pgfpathlineto{\pgfqpoint{3.792680in}{4.507363in}}%
\pgfusepath{stroke}%
\end{pgfscope}%
\begin{pgfscope}%
\pgfpathrectangle{\pgfqpoint{0.481978in}{0.331635in}}{\pgfqpoint{9.300000in}{7.700000in}}%
\pgfusepath{clip}%
\pgfsetrectcap%
\pgfsetroundjoin%
\pgfsetlinewidth{1.505625pt}%
\definecolor{currentstroke}{rgb}{0.631373,0.788235,0.956863}%
\pgfsetstrokecolor{currentstroke}%
\pgfsetstrokeopacity{0.200000}%
\pgfsetdash{}{0pt}%
\pgfpathmoveto{\pgfqpoint{6.038124in}{7.681635in}}%
\pgfpathlineto{\pgfqpoint{3.792680in}{4.507363in}}%
\pgfusepath{stroke}%
\end{pgfscope}%
\begin{pgfscope}%
\pgfpathrectangle{\pgfqpoint{0.481978in}{0.331635in}}{\pgfqpoint{9.300000in}{7.700000in}}%
\pgfusepath{clip}%
\pgfsetrectcap%
\pgfsetroundjoin%
\pgfsetlinewidth{1.505625pt}%
\definecolor{currentstroke}{rgb}{0.631373,0.788235,0.956863}%
\pgfsetstrokecolor{currentstroke}%
\pgfsetstrokeopacity{0.200000}%
\pgfsetdash{}{0pt}%
\pgfpathmoveto{\pgfqpoint{6.068535in}{7.049396in}}%
\pgfpathlineto{\pgfqpoint{3.792680in}{4.507363in}}%
\pgfusepath{stroke}%
\end{pgfscope}%
\begin{pgfscope}%
\pgfpathrectangle{\pgfqpoint{0.481978in}{0.331635in}}{\pgfqpoint{9.300000in}{7.700000in}}%
\pgfusepath{clip}%
\pgfsetrectcap%
\pgfsetroundjoin%
\pgfsetlinewidth{1.505625pt}%
\definecolor{currentstroke}{rgb}{0.631373,0.788235,0.956863}%
\pgfsetstrokecolor{currentstroke}%
\pgfsetstrokeopacity{0.200000}%
\pgfsetdash{}{0pt}%
\pgfpathmoveto{\pgfqpoint{0.904705in}{2.706623in}}%
\pgfpathlineto{\pgfqpoint{3.792680in}{4.507363in}}%
\pgfusepath{stroke}%
\end{pgfscope}%
\begin{pgfscope}%
\pgfpathrectangle{\pgfqpoint{0.481978in}{0.331635in}}{\pgfqpoint{9.300000in}{7.700000in}}%
\pgfusepath{clip}%
\pgfsetrectcap%
\pgfsetroundjoin%
\pgfsetlinewidth{1.505625pt}%
\definecolor{currentstroke}{rgb}{0.631373,0.788235,0.956863}%
\pgfsetstrokecolor{currentstroke}%
\pgfsetstrokeopacity{0.200000}%
\pgfsetdash{}{0pt}%
\pgfpathmoveto{\pgfqpoint{3.355697in}{0.924546in}}%
\pgfpathlineto{\pgfqpoint{3.792680in}{4.507363in}}%
\pgfusepath{stroke}%
\end{pgfscope}%
\begin{pgfscope}%
\pgfpathrectangle{\pgfqpoint{0.481978in}{0.331635in}}{\pgfqpoint{9.300000in}{7.700000in}}%
\pgfusepath{clip}%
\pgfsetrectcap%
\pgfsetroundjoin%
\pgfsetlinewidth{1.505625pt}%
\definecolor{currentstroke}{rgb}{0.631373,0.788235,0.956863}%
\pgfsetstrokecolor{currentstroke}%
\pgfsetstrokeopacity{0.200000}%
\pgfsetdash{}{0pt}%
\pgfpathmoveto{\pgfqpoint{1.822383in}{3.081016in}}%
\pgfpathlineto{\pgfqpoint{3.792680in}{4.507363in}}%
\pgfusepath{stroke}%
\end{pgfscope}%
\begin{pgfscope}%
\pgfpathrectangle{\pgfqpoint{0.481978in}{0.331635in}}{\pgfqpoint{9.300000in}{7.700000in}}%
\pgfusepath{clip}%
\pgfsetrectcap%
\pgfsetroundjoin%
\pgfsetlinewidth{1.505625pt}%
\definecolor{currentstroke}{rgb}{0.631373,0.788235,0.956863}%
\pgfsetstrokecolor{currentstroke}%
\pgfsetstrokeopacity{0.200000}%
\pgfsetdash{}{0pt}%
\pgfpathmoveto{\pgfqpoint{6.431702in}{7.311578in}}%
\pgfpathlineto{\pgfqpoint{3.792680in}{4.507363in}}%
\pgfusepath{stroke}%
\end{pgfscope}%
\begin{pgfscope}%
\pgfpathrectangle{\pgfqpoint{0.481978in}{0.331635in}}{\pgfqpoint{9.300000in}{7.700000in}}%
\pgfusepath{clip}%
\pgfsetrectcap%
\pgfsetroundjoin%
\pgfsetlinewidth{1.505625pt}%
\definecolor{currentstroke}{rgb}{0.631373,0.788235,0.956863}%
\pgfsetstrokecolor{currentstroke}%
\pgfsetstrokeopacity{0.200000}%
\pgfsetdash{}{0pt}%
\pgfpathmoveto{\pgfqpoint{1.601212in}{5.287720in}}%
\pgfpathlineto{\pgfqpoint{3.792680in}{4.507363in}}%
\pgfusepath{stroke}%
\end{pgfscope}%
\begin{pgfscope}%
\pgfpathrectangle{\pgfqpoint{0.481978in}{0.331635in}}{\pgfqpoint{9.300000in}{7.700000in}}%
\pgfusepath{clip}%
\pgfsetrectcap%
\pgfsetroundjoin%
\pgfsetlinewidth{1.505625pt}%
\definecolor{currentstroke}{rgb}{0.631373,0.788235,0.956863}%
\pgfsetstrokecolor{currentstroke}%
\pgfsetstrokeopacity{0.200000}%
\pgfsetdash{}{0pt}%
\pgfpathmoveto{\pgfqpoint{2.280847in}{6.346489in}}%
\pgfpathlineto{\pgfqpoint{3.792680in}{4.507363in}}%
\pgfusepath{stroke}%
\end{pgfscope}%
\begin{pgfscope}%
\pgfpathrectangle{\pgfqpoint{0.481978in}{0.331635in}}{\pgfqpoint{9.300000in}{7.700000in}}%
\pgfusepath{clip}%
\pgfsetrectcap%
\pgfsetroundjoin%
\pgfsetlinewidth{1.505625pt}%
\definecolor{currentstroke}{rgb}{0.631373,0.788235,0.956863}%
\pgfsetstrokecolor{currentstroke}%
\pgfsetstrokeopacity{0.200000}%
\pgfsetdash{}{0pt}%
\pgfpathmoveto{\pgfqpoint{1.914988in}{4.735660in}}%
\pgfpathlineto{\pgfqpoint{3.792680in}{4.507363in}}%
\pgfusepath{stroke}%
\end{pgfscope}%
\begin{pgfscope}%
\pgfpathrectangle{\pgfqpoint{0.481978in}{0.331635in}}{\pgfqpoint{9.300000in}{7.700000in}}%
\pgfusepath{clip}%
\pgfsetrectcap%
\pgfsetroundjoin%
\pgfsetlinewidth{1.505625pt}%
\definecolor{currentstroke}{rgb}{0.631373,0.788235,0.956863}%
\pgfsetstrokecolor{currentstroke}%
\pgfsetstrokeopacity{0.200000}%
\pgfsetdash{}{0pt}%
\pgfpathmoveto{\pgfqpoint{3.751753in}{1.453248in}}%
\pgfpathlineto{\pgfqpoint{3.792680in}{4.507363in}}%
\pgfusepath{stroke}%
\end{pgfscope}%
\begin{pgfscope}%
\pgfpathrectangle{\pgfqpoint{0.481978in}{0.331635in}}{\pgfqpoint{9.300000in}{7.700000in}}%
\pgfusepath{clip}%
\pgfsetrectcap%
\pgfsetroundjoin%
\pgfsetlinewidth{1.505625pt}%
\definecolor{currentstroke}{rgb}{0.631373,0.788235,0.956863}%
\pgfsetstrokecolor{currentstroke}%
\pgfsetstrokeopacity{0.200000}%
\pgfsetdash{}{0pt}%
\pgfpathmoveto{\pgfqpoint{5.753665in}{7.255589in}}%
\pgfpathlineto{\pgfqpoint{3.792680in}{4.507363in}}%
\pgfusepath{stroke}%
\end{pgfscope}%
\begin{pgfscope}%
\pgfpathrectangle{\pgfqpoint{0.481978in}{0.331635in}}{\pgfqpoint{9.300000in}{7.700000in}}%
\pgfusepath{clip}%
\pgfsetrectcap%
\pgfsetroundjoin%
\pgfsetlinewidth{1.505625pt}%
\definecolor{currentstroke}{rgb}{0.631373,0.788235,0.956863}%
\pgfsetstrokecolor{currentstroke}%
\pgfsetstrokeopacity{0.200000}%
\pgfsetdash{}{0pt}%
\pgfpathmoveto{\pgfqpoint{5.495955in}{5.233300in}}%
\pgfpathlineto{\pgfqpoint{3.792680in}{4.507363in}}%
\pgfusepath{stroke}%
\end{pgfscope}%
\begin{pgfscope}%
\pgfpathrectangle{\pgfqpoint{0.481978in}{0.331635in}}{\pgfqpoint{9.300000in}{7.700000in}}%
\pgfusepath{clip}%
\pgfsetrectcap%
\pgfsetroundjoin%
\pgfsetlinewidth{1.505625pt}%
\definecolor{currentstroke}{rgb}{0.631373,0.788235,0.956863}%
\pgfsetstrokecolor{currentstroke}%
\pgfsetstrokeopacity{0.200000}%
\pgfsetdash{}{0pt}%
\pgfpathmoveto{\pgfqpoint{4.238672in}{4.834740in}}%
\pgfpathlineto{\pgfqpoint{3.792680in}{4.507363in}}%
\pgfusepath{stroke}%
\end{pgfscope}%
\begin{pgfscope}%
\pgfpathrectangle{\pgfqpoint{0.481978in}{0.331635in}}{\pgfqpoint{9.300000in}{7.700000in}}%
\pgfusepath{clip}%
\pgfsetrectcap%
\pgfsetroundjoin%
\pgfsetlinewidth{1.505625pt}%
\definecolor{currentstroke}{rgb}{0.631373,0.788235,0.956863}%
\pgfsetstrokecolor{currentstroke}%
\pgfsetstrokeopacity{0.200000}%
\pgfsetdash{}{0pt}%
\pgfpathmoveto{\pgfqpoint{6.175837in}{7.515803in}}%
\pgfpathlineto{\pgfqpoint{3.792680in}{4.507363in}}%
\pgfusepath{stroke}%
\end{pgfscope}%
\begin{pgfscope}%
\pgfpathrectangle{\pgfqpoint{0.481978in}{0.331635in}}{\pgfqpoint{9.300000in}{7.700000in}}%
\pgfusepath{clip}%
\pgfsetrectcap%
\pgfsetroundjoin%
\pgfsetlinewidth{1.505625pt}%
\definecolor{currentstroke}{rgb}{0.631373,0.788235,0.956863}%
\pgfsetstrokecolor{currentstroke}%
\pgfsetstrokeopacity{0.200000}%
\pgfsetdash{}{0pt}%
\pgfpathmoveto{\pgfqpoint{4.277708in}{2.816465in}}%
\pgfpathlineto{\pgfqpoint{3.792680in}{4.507363in}}%
\pgfusepath{stroke}%
\end{pgfscope}%
\begin{pgfscope}%
\pgfpathrectangle{\pgfqpoint{0.481978in}{0.331635in}}{\pgfqpoint{9.300000in}{7.700000in}}%
\pgfusepath{clip}%
\pgfsetrectcap%
\pgfsetroundjoin%
\pgfsetlinewidth{1.505625pt}%
\definecolor{currentstroke}{rgb}{0.631373,0.788235,0.956863}%
\pgfsetstrokecolor{currentstroke}%
\pgfsetstrokeopacity{0.200000}%
\pgfsetdash{}{0pt}%
\pgfpathmoveto{\pgfqpoint{1.180315in}{4.870402in}}%
\pgfpathlineto{\pgfqpoint{3.792680in}{4.507363in}}%
\pgfusepath{stroke}%
\end{pgfscope}%
\begin{pgfscope}%
\pgfpathrectangle{\pgfqpoint{0.481978in}{0.331635in}}{\pgfqpoint{9.300000in}{7.700000in}}%
\pgfusepath{clip}%
\pgfsetrectcap%
\pgfsetroundjoin%
\pgfsetlinewidth{1.505625pt}%
\definecolor{currentstroke}{rgb}{0.631373,0.788235,0.956863}%
\pgfsetstrokecolor{currentstroke}%
\pgfsetstrokeopacity{0.200000}%
\pgfsetdash{}{0pt}%
\pgfpathmoveto{\pgfqpoint{4.575953in}{4.837612in}}%
\pgfpathlineto{\pgfqpoint{3.792680in}{4.507363in}}%
\pgfusepath{stroke}%
\end{pgfscope}%
\begin{pgfscope}%
\pgfpathrectangle{\pgfqpoint{0.481978in}{0.331635in}}{\pgfqpoint{9.300000in}{7.700000in}}%
\pgfusepath{clip}%
\pgfsetrectcap%
\pgfsetroundjoin%
\pgfsetlinewidth{1.505625pt}%
\definecolor{currentstroke}{rgb}{0.631373,0.788235,0.956863}%
\pgfsetstrokecolor{currentstroke}%
\pgfsetstrokeopacity{0.200000}%
\pgfsetdash{}{0pt}%
\pgfpathmoveto{\pgfqpoint{7.415126in}{4.572989in}}%
\pgfpathlineto{\pgfqpoint{3.792680in}{4.507363in}}%
\pgfusepath{stroke}%
\end{pgfscope}%
\begin{pgfscope}%
\pgfpathrectangle{\pgfqpoint{0.481978in}{0.331635in}}{\pgfqpoint{9.300000in}{7.700000in}}%
\pgfusepath{clip}%
\pgfsetrectcap%
\pgfsetroundjoin%
\pgfsetlinewidth{1.505625pt}%
\definecolor{currentstroke}{rgb}{0.631373,0.788235,0.956863}%
\pgfsetstrokecolor{currentstroke}%
\pgfsetstrokeopacity{0.200000}%
\pgfsetdash{}{0pt}%
\pgfpathmoveto{\pgfqpoint{5.743162in}{7.249576in}}%
\pgfpathlineto{\pgfqpoint{3.792680in}{4.507363in}}%
\pgfusepath{stroke}%
\end{pgfscope}%
\begin{pgfscope}%
\pgfpathrectangle{\pgfqpoint{0.481978in}{0.331635in}}{\pgfqpoint{9.300000in}{7.700000in}}%
\pgfusepath{clip}%
\pgfsetrectcap%
\pgfsetroundjoin%
\pgfsetlinewidth{1.505625pt}%
\definecolor{currentstroke}{rgb}{0.631373,0.788235,0.956863}%
\pgfsetstrokecolor{currentstroke}%
\pgfsetstrokeopacity{0.200000}%
\pgfsetdash{}{0pt}%
\pgfpathmoveto{\pgfqpoint{3.731632in}{6.363091in}}%
\pgfpathlineto{\pgfqpoint{3.792680in}{4.507363in}}%
\pgfusepath{stroke}%
\end{pgfscope}%
\begin{pgfscope}%
\pgfpathrectangle{\pgfqpoint{0.481978in}{0.331635in}}{\pgfqpoint{9.300000in}{7.700000in}}%
\pgfusepath{clip}%
\pgfsetrectcap%
\pgfsetroundjoin%
\pgfsetlinewidth{1.505625pt}%
\definecolor{currentstroke}{rgb}{0.631373,0.788235,0.956863}%
\pgfsetstrokecolor{currentstroke}%
\pgfsetstrokeopacity{0.200000}%
\pgfsetdash{}{0pt}%
\pgfpathmoveto{\pgfqpoint{3.354108in}{0.681635in}}%
\pgfpathlineto{\pgfqpoint{3.792680in}{4.507363in}}%
\pgfusepath{stroke}%
\end{pgfscope}%
\begin{pgfscope}%
\pgfpathrectangle{\pgfqpoint{0.481978in}{0.331635in}}{\pgfqpoint{9.300000in}{7.700000in}}%
\pgfusepath{clip}%
\pgfsetrectcap%
\pgfsetroundjoin%
\pgfsetlinewidth{1.505625pt}%
\definecolor{currentstroke}{rgb}{0.631373,0.788235,0.956863}%
\pgfsetstrokecolor{currentstroke}%
\pgfsetstrokeopacity{0.200000}%
\pgfsetdash{}{0pt}%
\pgfpathmoveto{\pgfqpoint{2.507727in}{5.244559in}}%
\pgfpathlineto{\pgfqpoint{3.792680in}{4.507363in}}%
\pgfusepath{stroke}%
\end{pgfscope}%
\begin{pgfscope}%
\pgfpathrectangle{\pgfqpoint{0.481978in}{0.331635in}}{\pgfqpoint{9.300000in}{7.700000in}}%
\pgfusepath{clip}%
\pgfsetrectcap%
\pgfsetroundjoin%
\pgfsetlinewidth{1.505625pt}%
\definecolor{currentstroke}{rgb}{0.631373,0.788235,0.956863}%
\pgfsetstrokecolor{currentstroke}%
\pgfsetstrokeopacity{0.200000}%
\pgfsetdash{}{0pt}%
\pgfpathmoveto{\pgfqpoint{3.345563in}{4.717554in}}%
\pgfpathlineto{\pgfqpoint{3.792680in}{4.507363in}}%
\pgfusepath{stroke}%
\end{pgfscope}%
\begin{pgfscope}%
\pgfpathrectangle{\pgfqpoint{0.481978in}{0.331635in}}{\pgfqpoint{9.300000in}{7.700000in}}%
\pgfusepath{clip}%
\pgfsetrectcap%
\pgfsetroundjoin%
\pgfsetlinewidth{1.505625pt}%
\definecolor{currentstroke}{rgb}{0.631373,0.788235,0.956863}%
\pgfsetstrokecolor{currentstroke}%
\pgfsetstrokeopacity{0.200000}%
\pgfsetdash{}{0pt}%
\pgfpathmoveto{\pgfqpoint{5.829195in}{5.254417in}}%
\pgfpathlineto{\pgfqpoint{3.792680in}{4.507363in}}%
\pgfusepath{stroke}%
\end{pgfscope}%
\begin{pgfscope}%
\pgfpathrectangle{\pgfqpoint{0.481978in}{0.331635in}}{\pgfqpoint{9.300000in}{7.700000in}}%
\pgfusepath{clip}%
\pgfsetrectcap%
\pgfsetroundjoin%
\pgfsetlinewidth{1.505625pt}%
\definecolor{currentstroke}{rgb}{0.631373,0.788235,0.956863}%
\pgfsetstrokecolor{currentstroke}%
\pgfsetstrokeopacity{0.200000}%
\pgfsetdash{}{0pt}%
\pgfpathmoveto{\pgfqpoint{1.266407in}{2.544584in}}%
\pgfpathlineto{\pgfqpoint{3.792680in}{4.507363in}}%
\pgfusepath{stroke}%
\end{pgfscope}%
\begin{pgfscope}%
\pgfpathrectangle{\pgfqpoint{0.481978in}{0.331635in}}{\pgfqpoint{9.300000in}{7.700000in}}%
\pgfusepath{clip}%
\pgfsetrectcap%
\pgfsetroundjoin%
\pgfsetlinewidth{1.505625pt}%
\definecolor{currentstroke}{rgb}{0.631373,0.788235,0.956863}%
\pgfsetstrokecolor{currentstroke}%
\pgfsetstrokeopacity{0.200000}%
\pgfsetdash{}{0pt}%
\pgfpathmoveto{\pgfqpoint{1.280257in}{3.743911in}}%
\pgfpathlineto{\pgfqpoint{3.792680in}{4.507363in}}%
\pgfusepath{stroke}%
\end{pgfscope}%
\begin{pgfscope}%
\pgfpathrectangle{\pgfqpoint{0.481978in}{0.331635in}}{\pgfqpoint{9.300000in}{7.700000in}}%
\pgfusepath{clip}%
\pgfsetrectcap%
\pgfsetroundjoin%
\pgfsetlinewidth{1.505625pt}%
\definecolor{currentstroke}{rgb}{0.631373,0.788235,0.956863}%
\pgfsetstrokecolor{currentstroke}%
\pgfsetstrokeopacity{0.200000}%
\pgfsetdash{}{0pt}%
\pgfpathmoveto{\pgfqpoint{1.366790in}{2.477899in}}%
\pgfpathlineto{\pgfqpoint{3.792680in}{4.507363in}}%
\pgfusepath{stroke}%
\end{pgfscope}%
\begin{pgfscope}%
\pgfpathrectangle{\pgfqpoint{0.481978in}{0.331635in}}{\pgfqpoint{9.300000in}{7.700000in}}%
\pgfusepath{clip}%
\pgfsetrectcap%
\pgfsetroundjoin%
\pgfsetlinewidth{1.505625pt}%
\definecolor{currentstroke}{rgb}{0.631373,0.788235,0.956863}%
\pgfsetstrokecolor{currentstroke}%
\pgfsetstrokeopacity{0.200000}%
\pgfsetdash{}{0pt}%
\pgfpathmoveto{\pgfqpoint{5.131403in}{3.817222in}}%
\pgfpathlineto{\pgfqpoint{3.792680in}{4.507363in}}%
\pgfusepath{stroke}%
\end{pgfscope}%
\begin{pgfscope}%
\pgfpathrectangle{\pgfqpoint{0.481978in}{0.331635in}}{\pgfqpoint{9.300000in}{7.700000in}}%
\pgfusepath{clip}%
\pgfsetrectcap%
\pgfsetroundjoin%
\pgfsetlinewidth{1.505625pt}%
\definecolor{currentstroke}{rgb}{0.631373,0.788235,0.956863}%
\pgfsetstrokecolor{currentstroke}%
\pgfsetstrokeopacity{0.200000}%
\pgfsetdash{}{0pt}%
\pgfpathmoveto{\pgfqpoint{6.077197in}{7.016351in}}%
\pgfpathlineto{\pgfqpoint{3.792680in}{4.507363in}}%
\pgfusepath{stroke}%
\end{pgfscope}%
\begin{pgfscope}%
\pgfpathrectangle{\pgfqpoint{0.481978in}{0.331635in}}{\pgfqpoint{9.300000in}{7.700000in}}%
\pgfusepath{clip}%
\pgfsetrectcap%
\pgfsetroundjoin%
\pgfsetlinewidth{1.505625pt}%
\definecolor{currentstroke}{rgb}{0.631373,0.788235,0.956863}%
\pgfsetstrokecolor{currentstroke}%
\pgfsetstrokeopacity{0.200000}%
\pgfsetdash{}{0pt}%
\pgfpathmoveto{\pgfqpoint{5.267507in}{3.479137in}}%
\pgfpathlineto{\pgfqpoint{3.792680in}{4.507363in}}%
\pgfusepath{stroke}%
\end{pgfscope}%
\begin{pgfscope}%
\pgfpathrectangle{\pgfqpoint{0.481978in}{0.331635in}}{\pgfqpoint{9.300000in}{7.700000in}}%
\pgfusepath{clip}%
\pgfsetrectcap%
\pgfsetroundjoin%
\pgfsetlinewidth{1.505625pt}%
\definecolor{currentstroke}{rgb}{0.631373,0.788235,0.956863}%
\pgfsetstrokecolor{currentstroke}%
\pgfsetstrokeopacity{0.200000}%
\pgfsetdash{}{0pt}%
\pgfpathmoveto{\pgfqpoint{5.422432in}{4.413168in}}%
\pgfpathlineto{\pgfqpoint{3.792680in}{4.507363in}}%
\pgfusepath{stroke}%
\end{pgfscope}%
\begin{pgfscope}%
\pgfpathrectangle{\pgfqpoint{0.481978in}{0.331635in}}{\pgfqpoint{9.300000in}{7.700000in}}%
\pgfusepath{clip}%
\pgfsetrectcap%
\pgfsetroundjoin%
\pgfsetlinewidth{1.505625pt}%
\definecolor{currentstroke}{rgb}{0.631373,0.788235,0.956863}%
\pgfsetstrokecolor{currentstroke}%
\pgfsetstrokeopacity{0.200000}%
\pgfsetdash{}{0pt}%
\pgfpathmoveto{\pgfqpoint{4.576144in}{3.894072in}}%
\pgfpathlineto{\pgfqpoint{3.792680in}{4.507363in}}%
\pgfusepath{stroke}%
\end{pgfscope}%
\begin{pgfscope}%
\pgfpathrectangle{\pgfqpoint{0.481978in}{0.331635in}}{\pgfqpoint{9.300000in}{7.700000in}}%
\pgfusepath{clip}%
\pgfsetrectcap%
\pgfsetroundjoin%
\pgfsetlinewidth{1.505625pt}%
\definecolor{currentstroke}{rgb}{0.631373,0.788235,0.956863}%
\pgfsetstrokecolor{currentstroke}%
\pgfsetstrokeopacity{0.200000}%
\pgfsetdash{}{0pt}%
\pgfpathmoveto{\pgfqpoint{2.109185in}{3.384062in}}%
\pgfpathlineto{\pgfqpoint{3.792680in}{4.507363in}}%
\pgfusepath{stroke}%
\end{pgfscope}%
\begin{pgfscope}%
\pgfpathrectangle{\pgfqpoint{0.481978in}{0.331635in}}{\pgfqpoint{9.300000in}{7.700000in}}%
\pgfusepath{clip}%
\pgfsetrectcap%
\pgfsetroundjoin%
\pgfsetlinewidth{1.505625pt}%
\definecolor{currentstroke}{rgb}{0.631373,0.788235,0.956863}%
\pgfsetstrokecolor{currentstroke}%
\pgfsetstrokeopacity{0.200000}%
\pgfsetdash{}{0pt}%
\pgfpathmoveto{\pgfqpoint{3.045397in}{4.082787in}}%
\pgfpathlineto{\pgfqpoint{3.792680in}{4.507363in}}%
\pgfusepath{stroke}%
\end{pgfscope}%
\begin{pgfscope}%
\pgfpathrectangle{\pgfqpoint{0.481978in}{0.331635in}}{\pgfqpoint{9.300000in}{7.700000in}}%
\pgfusepath{clip}%
\pgfsetrectcap%
\pgfsetroundjoin%
\pgfsetlinewidth{1.505625pt}%
\definecolor{currentstroke}{rgb}{0.631373,0.788235,0.956863}%
\pgfsetstrokecolor{currentstroke}%
\pgfsetstrokeopacity{0.200000}%
\pgfsetdash{}{0pt}%
\pgfpathmoveto{\pgfqpoint{2.937522in}{5.111573in}}%
\pgfpathlineto{\pgfqpoint{3.792680in}{4.507363in}}%
\pgfusepath{stroke}%
\end{pgfscope}%
\begin{pgfscope}%
\pgfpathrectangle{\pgfqpoint{0.481978in}{0.331635in}}{\pgfqpoint{9.300000in}{7.700000in}}%
\pgfusepath{clip}%
\pgfsetrectcap%
\pgfsetroundjoin%
\pgfsetlinewidth{1.505625pt}%
\definecolor{currentstroke}{rgb}{0.631373,0.788235,0.956863}%
\pgfsetstrokecolor{currentstroke}%
\pgfsetstrokeopacity{0.200000}%
\pgfsetdash{}{0pt}%
\pgfpathmoveto{\pgfqpoint{2.204208in}{6.251751in}}%
\pgfpathlineto{\pgfqpoint{3.792680in}{4.507363in}}%
\pgfusepath{stroke}%
\end{pgfscope}%
\begin{pgfscope}%
\pgfpathrectangle{\pgfqpoint{0.481978in}{0.331635in}}{\pgfqpoint{9.300000in}{7.700000in}}%
\pgfusepath{clip}%
\pgfsetrectcap%
\pgfsetroundjoin%
\pgfsetlinewidth{1.505625pt}%
\definecolor{currentstroke}{rgb}{0.631373,0.788235,0.956863}%
\pgfsetstrokecolor{currentstroke}%
\pgfsetstrokeopacity{0.200000}%
\pgfsetdash{}{0pt}%
\pgfpathmoveto{\pgfqpoint{3.780219in}{6.387783in}}%
\pgfpathlineto{\pgfqpoint{3.792680in}{4.507363in}}%
\pgfusepath{stroke}%
\end{pgfscope}%
\begin{pgfscope}%
\pgfpathrectangle{\pgfqpoint{0.481978in}{0.331635in}}{\pgfqpoint{9.300000in}{7.700000in}}%
\pgfusepath{clip}%
\pgfsetrectcap%
\pgfsetroundjoin%
\pgfsetlinewidth{1.505625pt}%
\definecolor{currentstroke}{rgb}{0.631373,0.788235,0.956863}%
\pgfsetstrokecolor{currentstroke}%
\pgfsetstrokeopacity{0.200000}%
\pgfsetdash{}{0pt}%
\pgfpathmoveto{\pgfqpoint{4.814839in}{2.794696in}}%
\pgfpathlineto{\pgfqpoint{3.792680in}{4.507363in}}%
\pgfusepath{stroke}%
\end{pgfscope}%
\begin{pgfscope}%
\pgfpathrectangle{\pgfqpoint{0.481978in}{0.331635in}}{\pgfqpoint{9.300000in}{7.700000in}}%
\pgfusepath{clip}%
\pgfsetrectcap%
\pgfsetroundjoin%
\pgfsetlinewidth{1.505625pt}%
\definecolor{currentstroke}{rgb}{0.631373,0.788235,0.956863}%
\pgfsetstrokecolor{currentstroke}%
\pgfsetstrokeopacity{0.200000}%
\pgfsetdash{}{0pt}%
\pgfpathmoveto{\pgfqpoint{2.753007in}{3.869593in}}%
\pgfpathlineto{\pgfqpoint{3.792680in}{4.507363in}}%
\pgfusepath{stroke}%
\end{pgfscope}%
\begin{pgfscope}%
\pgfpathrectangle{\pgfqpoint{0.481978in}{0.331635in}}{\pgfqpoint{9.300000in}{7.700000in}}%
\pgfusepath{clip}%
\pgfsetrectcap%
\pgfsetroundjoin%
\pgfsetlinewidth{1.505625pt}%
\definecolor{currentstroke}{rgb}{0.631373,0.788235,0.956863}%
\pgfsetstrokecolor{currentstroke}%
\pgfsetstrokeopacity{0.200000}%
\pgfsetdash{}{0pt}%
\pgfpathmoveto{\pgfqpoint{2.663463in}{5.701280in}}%
\pgfpathlineto{\pgfqpoint{3.792680in}{4.507363in}}%
\pgfusepath{stroke}%
\end{pgfscope}%
\begin{pgfscope}%
\pgfpathrectangle{\pgfqpoint{0.481978in}{0.331635in}}{\pgfqpoint{9.300000in}{7.700000in}}%
\pgfusepath{clip}%
\pgfsetrectcap%
\pgfsetroundjoin%
\pgfsetlinewidth{1.505625pt}%
\definecolor{currentstroke}{rgb}{0.631373,0.788235,0.956863}%
\pgfsetstrokecolor{currentstroke}%
\pgfsetstrokeopacity{0.200000}%
\pgfsetdash{}{0pt}%
\pgfpathmoveto{\pgfqpoint{6.114065in}{7.480299in}}%
\pgfpathlineto{\pgfqpoint{3.792680in}{4.507363in}}%
\pgfusepath{stroke}%
\end{pgfscope}%
\begin{pgfscope}%
\pgfpathrectangle{\pgfqpoint{0.481978in}{0.331635in}}{\pgfqpoint{9.300000in}{7.700000in}}%
\pgfusepath{clip}%
\pgfsetrectcap%
\pgfsetroundjoin%
\pgfsetlinewidth{1.505625pt}%
\definecolor{currentstroke}{rgb}{1.000000,0.705882,0.509804}%
\pgfsetstrokecolor{currentstroke}%
\pgfsetstrokeopacity{0.200000}%
\pgfsetdash{}{0pt}%
\pgfpathmoveto{\pgfqpoint{3.154468in}{2.488328in}}%
\pgfpathlineto{\pgfqpoint{5.493778in}{3.750666in}}%
\pgfusepath{stroke}%
\end{pgfscope}%
\begin{pgfscope}%
\pgfpathrectangle{\pgfqpoint{0.481978in}{0.331635in}}{\pgfqpoint{9.300000in}{7.700000in}}%
\pgfusepath{clip}%
\pgfsetrectcap%
\pgfsetroundjoin%
\pgfsetlinewidth{1.505625pt}%
\definecolor{currentstroke}{rgb}{1.000000,0.705882,0.509804}%
\pgfsetstrokecolor{currentstroke}%
\pgfsetstrokeopacity{0.200000}%
\pgfsetdash{}{0pt}%
\pgfpathmoveto{\pgfqpoint{1.545207in}{2.100004in}}%
\pgfpathlineto{\pgfqpoint{5.493778in}{3.750666in}}%
\pgfusepath{stroke}%
\end{pgfscope}%
\begin{pgfscope}%
\pgfpathrectangle{\pgfqpoint{0.481978in}{0.331635in}}{\pgfqpoint{9.300000in}{7.700000in}}%
\pgfusepath{clip}%
\pgfsetrectcap%
\pgfsetroundjoin%
\pgfsetlinewidth{1.505625pt}%
\definecolor{currentstroke}{rgb}{1.000000,0.705882,0.509804}%
\pgfsetstrokecolor{currentstroke}%
\pgfsetstrokeopacity{0.200000}%
\pgfsetdash{}{0pt}%
\pgfpathmoveto{\pgfqpoint{3.307338in}{3.407488in}}%
\pgfpathlineto{\pgfqpoint{5.493778in}{3.750666in}}%
\pgfusepath{stroke}%
\end{pgfscope}%
\begin{pgfscope}%
\pgfpathrectangle{\pgfqpoint{0.481978in}{0.331635in}}{\pgfqpoint{9.300000in}{7.700000in}}%
\pgfusepath{clip}%
\pgfsetrectcap%
\pgfsetroundjoin%
\pgfsetlinewidth{1.505625pt}%
\definecolor{currentstroke}{rgb}{1.000000,0.705882,0.509804}%
\pgfsetstrokecolor{currentstroke}%
\pgfsetstrokeopacity{0.200000}%
\pgfsetdash{}{0pt}%
\pgfpathmoveto{\pgfqpoint{5.143128in}{7.188196in}}%
\pgfpathlineto{\pgfqpoint{5.493778in}{3.750666in}}%
\pgfusepath{stroke}%
\end{pgfscope}%
\begin{pgfscope}%
\pgfpathrectangle{\pgfqpoint{0.481978in}{0.331635in}}{\pgfqpoint{9.300000in}{7.700000in}}%
\pgfusepath{clip}%
\pgfsetrectcap%
\pgfsetroundjoin%
\pgfsetlinewidth{1.505625pt}%
\definecolor{currentstroke}{rgb}{1.000000,0.705882,0.509804}%
\pgfsetstrokecolor{currentstroke}%
\pgfsetstrokeopacity{0.200000}%
\pgfsetdash{}{0pt}%
\pgfpathmoveto{\pgfqpoint{3.238886in}{1.944951in}}%
\pgfpathlineto{\pgfqpoint{5.493778in}{3.750666in}}%
\pgfusepath{stroke}%
\end{pgfscope}%
\begin{pgfscope}%
\pgfpathrectangle{\pgfqpoint{0.481978in}{0.331635in}}{\pgfqpoint{9.300000in}{7.700000in}}%
\pgfusepath{clip}%
\pgfsetrectcap%
\pgfsetroundjoin%
\pgfsetlinewidth{1.505625pt}%
\definecolor{currentstroke}{rgb}{1.000000,0.705882,0.509804}%
\pgfsetstrokecolor{currentstroke}%
\pgfsetstrokeopacity{0.200000}%
\pgfsetdash{}{0pt}%
\pgfpathmoveto{\pgfqpoint{6.096137in}{1.275008in}}%
\pgfpathlineto{\pgfqpoint{5.493778in}{3.750666in}}%
\pgfusepath{stroke}%
\end{pgfscope}%
\begin{pgfscope}%
\pgfpathrectangle{\pgfqpoint{0.481978in}{0.331635in}}{\pgfqpoint{9.300000in}{7.700000in}}%
\pgfusepath{clip}%
\pgfsetrectcap%
\pgfsetroundjoin%
\pgfsetlinewidth{1.505625pt}%
\definecolor{currentstroke}{rgb}{1.000000,0.705882,0.509804}%
\pgfsetstrokecolor{currentstroke}%
\pgfsetstrokeopacity{0.200000}%
\pgfsetdash{}{0pt}%
\pgfpathmoveto{\pgfqpoint{8.892538in}{6.041691in}}%
\pgfpathlineto{\pgfqpoint{5.493778in}{3.750666in}}%
\pgfusepath{stroke}%
\end{pgfscope}%
\begin{pgfscope}%
\pgfpathrectangle{\pgfqpoint{0.481978in}{0.331635in}}{\pgfqpoint{9.300000in}{7.700000in}}%
\pgfusepath{clip}%
\pgfsetrectcap%
\pgfsetroundjoin%
\pgfsetlinewidth{1.505625pt}%
\definecolor{currentstroke}{rgb}{1.000000,0.705882,0.509804}%
\pgfsetstrokecolor{currentstroke}%
\pgfsetstrokeopacity{0.200000}%
\pgfsetdash{}{0pt}%
\pgfpathmoveto{\pgfqpoint{8.792161in}{5.474544in}}%
\pgfpathlineto{\pgfqpoint{5.493778in}{3.750666in}}%
\pgfusepath{stroke}%
\end{pgfscope}%
\begin{pgfscope}%
\pgfpathrectangle{\pgfqpoint{0.481978in}{0.331635in}}{\pgfqpoint{9.300000in}{7.700000in}}%
\pgfusepath{clip}%
\pgfsetrectcap%
\pgfsetroundjoin%
\pgfsetlinewidth{1.505625pt}%
\definecolor{currentstroke}{rgb}{1.000000,0.705882,0.509804}%
\pgfsetstrokecolor{currentstroke}%
\pgfsetstrokeopacity{0.200000}%
\pgfsetdash{}{0pt}%
\pgfpathmoveto{\pgfqpoint{4.884502in}{4.809519in}}%
\pgfpathlineto{\pgfqpoint{5.493778in}{3.750666in}}%
\pgfusepath{stroke}%
\end{pgfscope}%
\begin{pgfscope}%
\pgfpathrectangle{\pgfqpoint{0.481978in}{0.331635in}}{\pgfqpoint{9.300000in}{7.700000in}}%
\pgfusepath{clip}%
\pgfsetrectcap%
\pgfsetroundjoin%
\pgfsetlinewidth{1.505625pt}%
\definecolor{currentstroke}{rgb}{1.000000,0.705882,0.509804}%
\pgfsetstrokecolor{currentstroke}%
\pgfsetstrokeopacity{0.200000}%
\pgfsetdash{}{0pt}%
\pgfpathmoveto{\pgfqpoint{8.839374in}{4.841088in}}%
\pgfpathlineto{\pgfqpoint{5.493778in}{3.750666in}}%
\pgfusepath{stroke}%
\end{pgfscope}%
\begin{pgfscope}%
\pgfpathrectangle{\pgfqpoint{0.481978in}{0.331635in}}{\pgfqpoint{9.300000in}{7.700000in}}%
\pgfusepath{clip}%
\pgfsetrectcap%
\pgfsetroundjoin%
\pgfsetlinewidth{1.505625pt}%
\definecolor{currentstroke}{rgb}{1.000000,0.705882,0.509804}%
\pgfsetstrokecolor{currentstroke}%
\pgfsetstrokeopacity{0.200000}%
\pgfsetdash{}{0pt}%
\pgfpathmoveto{\pgfqpoint{8.645625in}{6.005204in}}%
\pgfpathlineto{\pgfqpoint{5.493778in}{3.750666in}}%
\pgfusepath{stroke}%
\end{pgfscope}%
\begin{pgfscope}%
\pgfpathrectangle{\pgfqpoint{0.481978in}{0.331635in}}{\pgfqpoint{9.300000in}{7.700000in}}%
\pgfusepath{clip}%
\pgfsetrectcap%
\pgfsetroundjoin%
\pgfsetlinewidth{1.505625pt}%
\definecolor{currentstroke}{rgb}{1.000000,0.705882,0.509804}%
\pgfsetstrokecolor{currentstroke}%
\pgfsetstrokeopacity{0.200000}%
\pgfsetdash{}{0pt}%
\pgfpathmoveto{\pgfqpoint{1.541644in}{2.071167in}}%
\pgfpathlineto{\pgfqpoint{5.493778in}{3.750666in}}%
\pgfusepath{stroke}%
\end{pgfscope}%
\begin{pgfscope}%
\pgfpathrectangle{\pgfqpoint{0.481978in}{0.331635in}}{\pgfqpoint{9.300000in}{7.700000in}}%
\pgfusepath{clip}%
\pgfsetrectcap%
\pgfsetroundjoin%
\pgfsetlinewidth{1.505625pt}%
\definecolor{currentstroke}{rgb}{1.000000,0.705882,0.509804}%
\pgfsetstrokecolor{currentstroke}%
\pgfsetstrokeopacity{0.200000}%
\pgfsetdash{}{0pt}%
\pgfpathmoveto{\pgfqpoint{8.938916in}{5.410430in}}%
\pgfpathlineto{\pgfqpoint{5.493778in}{3.750666in}}%
\pgfusepath{stroke}%
\end{pgfscope}%
\begin{pgfscope}%
\pgfpathrectangle{\pgfqpoint{0.481978in}{0.331635in}}{\pgfqpoint{9.300000in}{7.700000in}}%
\pgfusepath{clip}%
\pgfsetrectcap%
\pgfsetroundjoin%
\pgfsetlinewidth{1.505625pt}%
\definecolor{currentstroke}{rgb}{1.000000,0.705882,0.509804}%
\pgfsetstrokecolor{currentstroke}%
\pgfsetstrokeopacity{0.200000}%
\pgfsetdash{}{0pt}%
\pgfpathmoveto{\pgfqpoint{3.682482in}{4.997977in}}%
\pgfpathlineto{\pgfqpoint{5.493778in}{3.750666in}}%
\pgfusepath{stroke}%
\end{pgfscope}%
\begin{pgfscope}%
\pgfpathrectangle{\pgfqpoint{0.481978in}{0.331635in}}{\pgfqpoint{9.300000in}{7.700000in}}%
\pgfusepath{clip}%
\pgfsetrectcap%
\pgfsetroundjoin%
\pgfsetlinewidth{1.505625pt}%
\definecolor{currentstroke}{rgb}{1.000000,0.705882,0.509804}%
\pgfsetstrokecolor{currentstroke}%
\pgfsetstrokeopacity{0.200000}%
\pgfsetdash{}{0pt}%
\pgfpathmoveto{\pgfqpoint{7.449113in}{2.488977in}}%
\pgfpathlineto{\pgfqpoint{5.493778in}{3.750666in}}%
\pgfusepath{stroke}%
\end{pgfscope}%
\begin{pgfscope}%
\pgfpathrectangle{\pgfqpoint{0.481978in}{0.331635in}}{\pgfqpoint{9.300000in}{7.700000in}}%
\pgfusepath{clip}%
\pgfsetrectcap%
\pgfsetroundjoin%
\pgfsetlinewidth{1.505625pt}%
\definecolor{currentstroke}{rgb}{1.000000,0.705882,0.509804}%
\pgfsetstrokecolor{currentstroke}%
\pgfsetstrokeopacity{0.200000}%
\pgfsetdash{}{0pt}%
\pgfpathmoveto{\pgfqpoint{8.325442in}{5.192998in}}%
\pgfpathlineto{\pgfqpoint{5.493778in}{3.750666in}}%
\pgfusepath{stroke}%
\end{pgfscope}%
\begin{pgfscope}%
\pgfpathrectangle{\pgfqpoint{0.481978in}{0.331635in}}{\pgfqpoint{9.300000in}{7.700000in}}%
\pgfusepath{clip}%
\pgfsetrectcap%
\pgfsetroundjoin%
\pgfsetlinewidth{1.505625pt}%
\definecolor{currentstroke}{rgb}{1.000000,0.705882,0.509804}%
\pgfsetstrokecolor{currentstroke}%
\pgfsetstrokeopacity{0.200000}%
\pgfsetdash{}{0pt}%
\pgfpathmoveto{\pgfqpoint{7.578449in}{4.392360in}}%
\pgfpathlineto{\pgfqpoint{5.493778in}{3.750666in}}%
\pgfusepath{stroke}%
\end{pgfscope}%
\begin{pgfscope}%
\pgfpathrectangle{\pgfqpoint{0.481978in}{0.331635in}}{\pgfqpoint{9.300000in}{7.700000in}}%
\pgfusepath{clip}%
\pgfsetrectcap%
\pgfsetroundjoin%
\pgfsetlinewidth{1.505625pt}%
\definecolor{currentstroke}{rgb}{1.000000,0.705882,0.509804}%
\pgfsetstrokecolor{currentstroke}%
\pgfsetstrokeopacity{0.200000}%
\pgfsetdash{}{0pt}%
\pgfpathmoveto{\pgfqpoint{6.440915in}{3.400690in}}%
\pgfpathlineto{\pgfqpoint{5.493778in}{3.750666in}}%
\pgfusepath{stroke}%
\end{pgfscope}%
\begin{pgfscope}%
\pgfpathrectangle{\pgfqpoint{0.481978in}{0.331635in}}{\pgfqpoint{9.300000in}{7.700000in}}%
\pgfusepath{clip}%
\pgfsetrectcap%
\pgfsetroundjoin%
\pgfsetlinewidth{1.505625pt}%
\definecolor{currentstroke}{rgb}{1.000000,0.705882,0.509804}%
\pgfsetstrokecolor{currentstroke}%
\pgfsetstrokeopacity{0.200000}%
\pgfsetdash{}{0pt}%
\pgfpathmoveto{\pgfqpoint{3.469279in}{3.869225in}}%
\pgfpathlineto{\pgfqpoint{5.493778in}{3.750666in}}%
\pgfusepath{stroke}%
\end{pgfscope}%
\begin{pgfscope}%
\pgfpathrectangle{\pgfqpoint{0.481978in}{0.331635in}}{\pgfqpoint{9.300000in}{7.700000in}}%
\pgfusepath{clip}%
\pgfsetrectcap%
\pgfsetroundjoin%
\pgfsetlinewidth{1.505625pt}%
\definecolor{currentstroke}{rgb}{1.000000,0.705882,0.509804}%
\pgfsetstrokecolor{currentstroke}%
\pgfsetstrokeopacity{0.200000}%
\pgfsetdash{}{0pt}%
\pgfpathmoveto{\pgfqpoint{6.295661in}{1.608501in}}%
\pgfpathlineto{\pgfqpoint{5.493778in}{3.750666in}}%
\pgfusepath{stroke}%
\end{pgfscope}%
\begin{pgfscope}%
\pgfpathrectangle{\pgfqpoint{0.481978in}{0.331635in}}{\pgfqpoint{9.300000in}{7.700000in}}%
\pgfusepath{clip}%
\pgfsetrectcap%
\pgfsetroundjoin%
\pgfsetlinewidth{1.505625pt}%
\definecolor{currentstroke}{rgb}{1.000000,0.705882,0.509804}%
\pgfsetstrokecolor{currentstroke}%
\pgfsetstrokeopacity{0.200000}%
\pgfsetdash{}{0pt}%
\pgfpathmoveto{\pgfqpoint{3.458025in}{3.980640in}}%
\pgfpathlineto{\pgfqpoint{5.493778in}{3.750666in}}%
\pgfusepath{stroke}%
\end{pgfscope}%
\begin{pgfscope}%
\pgfpathrectangle{\pgfqpoint{0.481978in}{0.331635in}}{\pgfqpoint{9.300000in}{7.700000in}}%
\pgfusepath{clip}%
\pgfsetrectcap%
\pgfsetroundjoin%
\pgfsetlinewidth{1.505625pt}%
\definecolor{currentstroke}{rgb}{1.000000,0.705882,0.509804}%
\pgfsetstrokecolor{currentstroke}%
\pgfsetstrokeopacity{0.200000}%
\pgfsetdash{}{0pt}%
\pgfpathmoveto{\pgfqpoint{4.270877in}{5.423784in}}%
\pgfpathlineto{\pgfqpoint{5.493778in}{3.750666in}}%
\pgfusepath{stroke}%
\end{pgfscope}%
\begin{pgfscope}%
\pgfpathrectangle{\pgfqpoint{0.481978in}{0.331635in}}{\pgfqpoint{9.300000in}{7.700000in}}%
\pgfusepath{clip}%
\pgfsetrectcap%
\pgfsetroundjoin%
\pgfsetlinewidth{1.505625pt}%
\definecolor{currentstroke}{rgb}{1.000000,0.705882,0.509804}%
\pgfsetstrokecolor{currentstroke}%
\pgfsetstrokeopacity{0.200000}%
\pgfsetdash{}{0pt}%
\pgfpathmoveto{\pgfqpoint{5.749725in}{1.958324in}}%
\pgfpathlineto{\pgfqpoint{5.493778in}{3.750666in}}%
\pgfusepath{stroke}%
\end{pgfscope}%
\begin{pgfscope}%
\pgfpathrectangle{\pgfqpoint{0.481978in}{0.331635in}}{\pgfqpoint{9.300000in}{7.700000in}}%
\pgfusepath{clip}%
\pgfsetrectcap%
\pgfsetroundjoin%
\pgfsetlinewidth{1.505625pt}%
\definecolor{currentstroke}{rgb}{1.000000,0.705882,0.509804}%
\pgfsetstrokecolor{currentstroke}%
\pgfsetstrokeopacity{0.200000}%
\pgfsetdash{}{0pt}%
\pgfpathmoveto{\pgfqpoint{2.458775in}{1.729372in}}%
\pgfpathlineto{\pgfqpoint{5.493778in}{3.750666in}}%
\pgfusepath{stroke}%
\end{pgfscope}%
\begin{pgfscope}%
\pgfpathrectangle{\pgfqpoint{0.481978in}{0.331635in}}{\pgfqpoint{9.300000in}{7.700000in}}%
\pgfusepath{clip}%
\pgfsetrectcap%
\pgfsetroundjoin%
\pgfsetlinewidth{1.505625pt}%
\definecolor{currentstroke}{rgb}{1.000000,0.705882,0.509804}%
\pgfsetstrokecolor{currentstroke}%
\pgfsetstrokeopacity{0.200000}%
\pgfsetdash{}{0pt}%
\pgfpathmoveto{\pgfqpoint{4.692250in}{5.305518in}}%
\pgfpathlineto{\pgfqpoint{5.493778in}{3.750666in}}%
\pgfusepath{stroke}%
\end{pgfscope}%
\begin{pgfscope}%
\pgfpathrectangle{\pgfqpoint{0.481978in}{0.331635in}}{\pgfqpoint{9.300000in}{7.700000in}}%
\pgfusepath{clip}%
\pgfsetrectcap%
\pgfsetroundjoin%
\pgfsetlinewidth{1.505625pt}%
\definecolor{currentstroke}{rgb}{1.000000,0.705882,0.509804}%
\pgfsetstrokecolor{currentstroke}%
\pgfsetstrokeopacity{0.200000}%
\pgfsetdash{}{0pt}%
\pgfpathmoveto{\pgfqpoint{7.916388in}{4.222966in}}%
\pgfpathlineto{\pgfqpoint{5.493778in}{3.750666in}}%
\pgfusepath{stroke}%
\end{pgfscope}%
\begin{pgfscope}%
\pgfpathrectangle{\pgfqpoint{0.481978in}{0.331635in}}{\pgfqpoint{9.300000in}{7.700000in}}%
\pgfusepath{clip}%
\pgfsetrectcap%
\pgfsetroundjoin%
\pgfsetlinewidth{1.505625pt}%
\definecolor{currentstroke}{rgb}{1.000000,0.705882,0.509804}%
\pgfsetstrokecolor{currentstroke}%
\pgfsetstrokeopacity{0.200000}%
\pgfsetdash{}{0pt}%
\pgfpathmoveto{\pgfqpoint{4.752028in}{3.460097in}}%
\pgfpathlineto{\pgfqpoint{5.493778in}{3.750666in}}%
\pgfusepath{stroke}%
\end{pgfscope}%
\begin{pgfscope}%
\pgfpathrectangle{\pgfqpoint{0.481978in}{0.331635in}}{\pgfqpoint{9.300000in}{7.700000in}}%
\pgfusepath{clip}%
\pgfsetrectcap%
\pgfsetroundjoin%
\pgfsetlinewidth{1.505625pt}%
\definecolor{currentstroke}{rgb}{1.000000,0.705882,0.509804}%
\pgfsetstrokecolor{currentstroke}%
\pgfsetstrokeopacity{0.200000}%
\pgfsetdash{}{0pt}%
\pgfpathmoveto{\pgfqpoint{8.514292in}{5.266180in}}%
\pgfpathlineto{\pgfqpoint{5.493778in}{3.750666in}}%
\pgfusepath{stroke}%
\end{pgfscope}%
\begin{pgfscope}%
\pgfpathrectangle{\pgfqpoint{0.481978in}{0.331635in}}{\pgfqpoint{9.300000in}{7.700000in}}%
\pgfusepath{clip}%
\pgfsetrectcap%
\pgfsetroundjoin%
\pgfsetlinewidth{1.505625pt}%
\definecolor{currentstroke}{rgb}{1.000000,0.705882,0.509804}%
\pgfsetstrokecolor{currentstroke}%
\pgfsetstrokeopacity{0.200000}%
\pgfsetdash{}{0pt}%
\pgfpathmoveto{\pgfqpoint{3.988115in}{1.945915in}}%
\pgfpathlineto{\pgfqpoint{5.493778in}{3.750666in}}%
\pgfusepath{stroke}%
\end{pgfscope}%
\begin{pgfscope}%
\pgfpathrectangle{\pgfqpoint{0.481978in}{0.331635in}}{\pgfqpoint{9.300000in}{7.700000in}}%
\pgfusepath{clip}%
\pgfsetrectcap%
\pgfsetroundjoin%
\pgfsetlinewidth{1.505625pt}%
\definecolor{currentstroke}{rgb}{1.000000,0.705882,0.509804}%
\pgfsetstrokecolor{currentstroke}%
\pgfsetstrokeopacity{0.200000}%
\pgfsetdash{}{0pt}%
\pgfpathmoveto{\pgfqpoint{3.637460in}{4.480745in}}%
\pgfpathlineto{\pgfqpoint{5.493778in}{3.750666in}}%
\pgfusepath{stroke}%
\end{pgfscope}%
\begin{pgfscope}%
\pgfpathrectangle{\pgfqpoint{0.481978in}{0.331635in}}{\pgfqpoint{9.300000in}{7.700000in}}%
\pgfusepath{clip}%
\pgfsetrectcap%
\pgfsetroundjoin%
\pgfsetlinewidth{1.505625pt}%
\definecolor{currentstroke}{rgb}{1.000000,0.705882,0.509804}%
\pgfsetstrokecolor{currentstroke}%
\pgfsetstrokeopacity{0.200000}%
\pgfsetdash{}{0pt}%
\pgfpathmoveto{\pgfqpoint{3.666577in}{4.495055in}}%
\pgfpathlineto{\pgfqpoint{5.493778in}{3.750666in}}%
\pgfusepath{stroke}%
\end{pgfscope}%
\begin{pgfscope}%
\pgfpathrectangle{\pgfqpoint{0.481978in}{0.331635in}}{\pgfqpoint{9.300000in}{7.700000in}}%
\pgfusepath{clip}%
\pgfsetrectcap%
\pgfsetroundjoin%
\pgfsetlinewidth{1.505625pt}%
\definecolor{currentstroke}{rgb}{1.000000,0.705882,0.509804}%
\pgfsetstrokecolor{currentstroke}%
\pgfsetstrokeopacity{0.200000}%
\pgfsetdash{}{0pt}%
\pgfpathmoveto{\pgfqpoint{3.344754in}{3.254443in}}%
\pgfpathlineto{\pgfqpoint{5.493778in}{3.750666in}}%
\pgfusepath{stroke}%
\end{pgfscope}%
\begin{pgfscope}%
\pgfpathrectangle{\pgfqpoint{0.481978in}{0.331635in}}{\pgfqpoint{9.300000in}{7.700000in}}%
\pgfusepath{clip}%
\pgfsetrectcap%
\pgfsetroundjoin%
\pgfsetlinewidth{1.505625pt}%
\definecolor{currentstroke}{rgb}{1.000000,0.705882,0.509804}%
\pgfsetstrokecolor{currentstroke}%
\pgfsetstrokeopacity{0.200000}%
\pgfsetdash{}{0pt}%
\pgfpathmoveto{\pgfqpoint{7.394228in}{2.358529in}}%
\pgfpathlineto{\pgfqpoint{5.493778in}{3.750666in}}%
\pgfusepath{stroke}%
\end{pgfscope}%
\begin{pgfscope}%
\pgfpathrectangle{\pgfqpoint{0.481978in}{0.331635in}}{\pgfqpoint{9.300000in}{7.700000in}}%
\pgfusepath{clip}%
\pgfsetrectcap%
\pgfsetroundjoin%
\pgfsetlinewidth{1.505625pt}%
\definecolor{currentstroke}{rgb}{1.000000,0.705882,0.509804}%
\pgfsetstrokecolor{currentstroke}%
\pgfsetstrokeopacity{0.200000}%
\pgfsetdash{}{0pt}%
\pgfpathmoveto{\pgfqpoint{2.694400in}{2.135292in}}%
\pgfpathlineto{\pgfqpoint{5.493778in}{3.750666in}}%
\pgfusepath{stroke}%
\end{pgfscope}%
\begin{pgfscope}%
\pgfpathrectangle{\pgfqpoint{0.481978in}{0.331635in}}{\pgfqpoint{9.300000in}{7.700000in}}%
\pgfusepath{clip}%
\pgfsetrectcap%
\pgfsetroundjoin%
\pgfsetlinewidth{1.505625pt}%
\definecolor{currentstroke}{rgb}{1.000000,0.705882,0.509804}%
\pgfsetstrokecolor{currentstroke}%
\pgfsetstrokeopacity{0.200000}%
\pgfsetdash{}{0pt}%
\pgfpathmoveto{\pgfqpoint{5.916550in}{1.543110in}}%
\pgfpathlineto{\pgfqpoint{5.493778in}{3.750666in}}%
\pgfusepath{stroke}%
\end{pgfscope}%
\begin{pgfscope}%
\pgfpathrectangle{\pgfqpoint{0.481978in}{0.331635in}}{\pgfqpoint{9.300000in}{7.700000in}}%
\pgfusepath{clip}%
\pgfsetrectcap%
\pgfsetroundjoin%
\pgfsetlinewidth{1.505625pt}%
\definecolor{currentstroke}{rgb}{1.000000,0.705882,0.509804}%
\pgfsetstrokecolor{currentstroke}%
\pgfsetstrokeopacity{0.200000}%
\pgfsetdash{}{0pt}%
\pgfpathmoveto{\pgfqpoint{6.083860in}{2.968841in}}%
\pgfpathlineto{\pgfqpoint{5.493778in}{3.750666in}}%
\pgfusepath{stroke}%
\end{pgfscope}%
\begin{pgfscope}%
\pgfpathrectangle{\pgfqpoint{0.481978in}{0.331635in}}{\pgfqpoint{9.300000in}{7.700000in}}%
\pgfusepath{clip}%
\pgfsetrectcap%
\pgfsetroundjoin%
\pgfsetlinewidth{1.505625pt}%
\definecolor{currentstroke}{rgb}{1.000000,0.705882,0.509804}%
\pgfsetstrokecolor{currentstroke}%
\pgfsetstrokeopacity{0.200000}%
\pgfsetdash{}{0pt}%
\pgfpathmoveto{\pgfqpoint{4.639314in}{3.292339in}}%
\pgfpathlineto{\pgfqpoint{5.493778in}{3.750666in}}%
\pgfusepath{stroke}%
\end{pgfscope}%
\begin{pgfscope}%
\pgfpathrectangle{\pgfqpoint{0.481978in}{0.331635in}}{\pgfqpoint{9.300000in}{7.700000in}}%
\pgfusepath{clip}%
\pgfsetrectcap%
\pgfsetroundjoin%
\pgfsetlinewidth{1.505625pt}%
\definecolor{currentstroke}{rgb}{1.000000,0.705882,0.509804}%
\pgfsetstrokecolor{currentstroke}%
\pgfsetstrokeopacity{0.200000}%
\pgfsetdash{}{0pt}%
\pgfpathmoveto{\pgfqpoint{8.886711in}{5.746544in}}%
\pgfpathlineto{\pgfqpoint{5.493778in}{3.750666in}}%
\pgfusepath{stroke}%
\end{pgfscope}%
\begin{pgfscope}%
\pgfpathrectangle{\pgfqpoint{0.481978in}{0.331635in}}{\pgfqpoint{9.300000in}{7.700000in}}%
\pgfusepath{clip}%
\pgfsetrectcap%
\pgfsetroundjoin%
\pgfsetlinewidth{1.505625pt}%
\definecolor{currentstroke}{rgb}{1.000000,0.705882,0.509804}%
\pgfsetstrokecolor{currentstroke}%
\pgfsetstrokeopacity{0.200000}%
\pgfsetdash{}{0pt}%
\pgfpathmoveto{\pgfqpoint{3.960687in}{2.310033in}}%
\pgfpathlineto{\pgfqpoint{5.493778in}{3.750666in}}%
\pgfusepath{stroke}%
\end{pgfscope}%
\begin{pgfscope}%
\pgfpathrectangle{\pgfqpoint{0.481978in}{0.331635in}}{\pgfqpoint{9.300000in}{7.700000in}}%
\pgfusepath{clip}%
\pgfsetrectcap%
\pgfsetroundjoin%
\pgfsetlinewidth{1.505625pt}%
\definecolor{currentstroke}{rgb}{1.000000,0.705882,0.509804}%
\pgfsetstrokecolor{currentstroke}%
\pgfsetstrokeopacity{0.200000}%
\pgfsetdash{}{0pt}%
\pgfpathmoveto{\pgfqpoint{3.907332in}{1.760958in}}%
\pgfpathlineto{\pgfqpoint{5.493778in}{3.750666in}}%
\pgfusepath{stroke}%
\end{pgfscope}%
\begin{pgfscope}%
\pgfpathrectangle{\pgfqpoint{0.481978in}{0.331635in}}{\pgfqpoint{9.300000in}{7.700000in}}%
\pgfusepath{clip}%
\pgfsetrectcap%
\pgfsetroundjoin%
\pgfsetlinewidth{1.505625pt}%
\definecolor{currentstroke}{rgb}{1.000000,0.705882,0.509804}%
\pgfsetstrokecolor{currentstroke}%
\pgfsetstrokeopacity{0.200000}%
\pgfsetdash{}{0pt}%
\pgfpathmoveto{\pgfqpoint{4.745564in}{5.377197in}}%
\pgfpathlineto{\pgfqpoint{5.493778in}{3.750666in}}%
\pgfusepath{stroke}%
\end{pgfscope}%
\begin{pgfscope}%
\pgfpathrectangle{\pgfqpoint{0.481978in}{0.331635in}}{\pgfqpoint{9.300000in}{7.700000in}}%
\pgfusepath{clip}%
\pgfsetrectcap%
\pgfsetroundjoin%
\pgfsetlinewidth{1.505625pt}%
\definecolor{currentstroke}{rgb}{1.000000,0.705882,0.509804}%
\pgfsetstrokecolor{currentstroke}%
\pgfsetstrokeopacity{0.200000}%
\pgfsetdash{}{0pt}%
\pgfpathmoveto{\pgfqpoint{4.104700in}{5.460602in}}%
\pgfpathlineto{\pgfqpoint{5.493778in}{3.750666in}}%
\pgfusepath{stroke}%
\end{pgfscope}%
\begin{pgfscope}%
\pgfpathrectangle{\pgfqpoint{0.481978in}{0.331635in}}{\pgfqpoint{9.300000in}{7.700000in}}%
\pgfusepath{clip}%
\pgfsetrectcap%
\pgfsetroundjoin%
\pgfsetlinewidth{1.505625pt}%
\definecolor{currentstroke}{rgb}{1.000000,0.705882,0.509804}%
\pgfsetstrokecolor{currentstroke}%
\pgfsetstrokeopacity{0.200000}%
\pgfsetdash{}{0pt}%
\pgfpathmoveto{\pgfqpoint{8.529282in}{4.574529in}}%
\pgfpathlineto{\pgfqpoint{5.493778in}{3.750666in}}%
\pgfusepath{stroke}%
\end{pgfscope}%
\begin{pgfscope}%
\pgfpathrectangle{\pgfqpoint{0.481978in}{0.331635in}}{\pgfqpoint{9.300000in}{7.700000in}}%
\pgfusepath{clip}%
\pgfsetrectcap%
\pgfsetroundjoin%
\pgfsetlinewidth{1.505625pt}%
\definecolor{currentstroke}{rgb}{1.000000,0.705882,0.509804}%
\pgfsetstrokecolor{currentstroke}%
\pgfsetstrokeopacity{0.200000}%
\pgfsetdash{}{0pt}%
\pgfpathmoveto{\pgfqpoint{7.907970in}{4.014679in}}%
\pgfpathlineto{\pgfqpoint{5.493778in}{3.750666in}}%
\pgfusepath{stroke}%
\end{pgfscope}%
\begin{pgfscope}%
\pgfpathrectangle{\pgfqpoint{0.481978in}{0.331635in}}{\pgfqpoint{9.300000in}{7.700000in}}%
\pgfusepath{clip}%
\pgfsetrectcap%
\pgfsetroundjoin%
\pgfsetlinewidth{1.505625pt}%
\definecolor{currentstroke}{rgb}{1.000000,0.705882,0.509804}%
\pgfsetstrokecolor{currentstroke}%
\pgfsetstrokeopacity{0.200000}%
\pgfsetdash{}{0pt}%
\pgfpathmoveto{\pgfqpoint{8.433076in}{4.560190in}}%
\pgfpathlineto{\pgfqpoint{5.493778in}{3.750666in}}%
\pgfusepath{stroke}%
\end{pgfscope}%
\begin{pgfscope}%
\pgfpathrectangle{\pgfqpoint{0.481978in}{0.331635in}}{\pgfqpoint{9.300000in}{7.700000in}}%
\pgfusepath{clip}%
\pgfsetrectcap%
\pgfsetroundjoin%
\pgfsetlinewidth{1.505625pt}%
\definecolor{currentstroke}{rgb}{1.000000,0.705882,0.509804}%
\pgfsetstrokecolor{currentstroke}%
\pgfsetstrokeopacity{0.200000}%
\pgfsetdash{}{0pt}%
\pgfpathmoveto{\pgfqpoint{3.128738in}{2.095357in}}%
\pgfpathlineto{\pgfqpoint{5.493778in}{3.750666in}}%
\pgfusepath{stroke}%
\end{pgfscope}%
\begin{pgfscope}%
\pgfpathrectangle{\pgfqpoint{0.481978in}{0.331635in}}{\pgfqpoint{9.300000in}{7.700000in}}%
\pgfusepath{clip}%
\pgfsetrectcap%
\pgfsetroundjoin%
\pgfsetlinewidth{1.505625pt}%
\definecolor{currentstroke}{rgb}{1.000000,0.705882,0.509804}%
\pgfsetstrokecolor{currentstroke}%
\pgfsetstrokeopacity{0.200000}%
\pgfsetdash{}{0pt}%
\pgfpathmoveto{\pgfqpoint{6.750608in}{3.072454in}}%
\pgfpathlineto{\pgfqpoint{5.493778in}{3.750666in}}%
\pgfusepath{stroke}%
\end{pgfscope}%
\begin{pgfscope}%
\pgfpathrectangle{\pgfqpoint{0.481978in}{0.331635in}}{\pgfqpoint{9.300000in}{7.700000in}}%
\pgfusepath{clip}%
\pgfsetrectcap%
\pgfsetroundjoin%
\pgfsetlinewidth{1.505625pt}%
\definecolor{currentstroke}{rgb}{1.000000,0.705882,0.509804}%
\pgfsetstrokecolor{currentstroke}%
\pgfsetstrokeopacity{0.200000}%
\pgfsetdash{}{0pt}%
\pgfpathmoveto{\pgfqpoint{3.434837in}{4.071894in}}%
\pgfpathlineto{\pgfqpoint{5.493778in}{3.750666in}}%
\pgfusepath{stroke}%
\end{pgfscope}%
\begin{pgfscope}%
\pgfpathrectangle{\pgfqpoint{0.481978in}{0.331635in}}{\pgfqpoint{9.300000in}{7.700000in}}%
\pgfusepath{clip}%
\pgfsetrectcap%
\pgfsetroundjoin%
\pgfsetlinewidth{1.505625pt}%
\definecolor{currentstroke}{rgb}{1.000000,0.705882,0.509804}%
\pgfsetstrokecolor{currentstroke}%
\pgfsetstrokeopacity{0.200000}%
\pgfsetdash{}{0pt}%
\pgfpathmoveto{\pgfqpoint{2.745959in}{2.321911in}}%
\pgfpathlineto{\pgfqpoint{5.493778in}{3.750666in}}%
\pgfusepath{stroke}%
\end{pgfscope}%
\begin{pgfscope}%
\pgfpathrectangle{\pgfqpoint{0.481978in}{0.331635in}}{\pgfqpoint{9.300000in}{7.700000in}}%
\pgfusepath{clip}%
\pgfsetrectcap%
\pgfsetroundjoin%
\pgfsetlinewidth{1.505625pt}%
\definecolor{currentstroke}{rgb}{1.000000,0.705882,0.509804}%
\pgfsetstrokecolor{currentstroke}%
\pgfsetstrokeopacity{0.200000}%
\pgfsetdash{}{0pt}%
\pgfpathmoveto{\pgfqpoint{8.714562in}{5.337458in}}%
\pgfpathlineto{\pgfqpoint{5.493778in}{3.750666in}}%
\pgfusepath{stroke}%
\end{pgfscope}%
\begin{pgfscope}%
\pgfpathrectangle{\pgfqpoint{0.481978in}{0.331635in}}{\pgfqpoint{9.300000in}{7.700000in}}%
\pgfusepath{clip}%
\pgfsetrectcap%
\pgfsetroundjoin%
\pgfsetlinewidth{1.505625pt}%
\definecolor{currentstroke}{rgb}{0.552941,0.898039,0.631373}%
\pgfsetstrokecolor{currentstroke}%
\pgfsetstrokeopacity{0.200000}%
\pgfsetdash{}{0pt}%
\pgfpathmoveto{\pgfqpoint{5.163299in}{7.343168in}}%
\pgfpathlineto{\pgfqpoint{5.849815in}{4.522824in}}%
\pgfusepath{stroke}%
\end{pgfscope}%
\begin{pgfscope}%
\pgfpathrectangle{\pgfqpoint{0.481978in}{0.331635in}}{\pgfqpoint{9.300000in}{7.700000in}}%
\pgfusepath{clip}%
\pgfsetrectcap%
\pgfsetroundjoin%
\pgfsetlinewidth{1.505625pt}%
\definecolor{currentstroke}{rgb}{0.552941,0.898039,0.631373}%
\pgfsetstrokecolor{currentstroke}%
\pgfsetstrokeopacity{0.200000}%
\pgfsetdash{}{0pt}%
\pgfpathmoveto{\pgfqpoint{3.761360in}{5.764646in}}%
\pgfpathlineto{\pgfqpoint{5.849815in}{4.522824in}}%
\pgfusepath{stroke}%
\end{pgfscope}%
\begin{pgfscope}%
\pgfpathrectangle{\pgfqpoint{0.481978in}{0.331635in}}{\pgfqpoint{9.300000in}{7.700000in}}%
\pgfusepath{clip}%
\pgfsetrectcap%
\pgfsetroundjoin%
\pgfsetlinewidth{1.505625pt}%
\definecolor{currentstroke}{rgb}{0.552941,0.898039,0.631373}%
\pgfsetstrokecolor{currentstroke}%
\pgfsetstrokeopacity{0.200000}%
\pgfsetdash{}{0pt}%
\pgfpathmoveto{\pgfqpoint{3.580350in}{3.088303in}}%
\pgfpathlineto{\pgfqpoint{5.849815in}{4.522824in}}%
\pgfusepath{stroke}%
\end{pgfscope}%
\begin{pgfscope}%
\pgfpathrectangle{\pgfqpoint{0.481978in}{0.331635in}}{\pgfqpoint{9.300000in}{7.700000in}}%
\pgfusepath{clip}%
\pgfsetrectcap%
\pgfsetroundjoin%
\pgfsetlinewidth{1.505625pt}%
\definecolor{currentstroke}{rgb}{0.552941,0.898039,0.631373}%
\pgfsetstrokecolor{currentstroke}%
\pgfsetstrokeopacity{0.200000}%
\pgfsetdash{}{0pt}%
\pgfpathmoveto{\pgfqpoint{8.242673in}{4.221424in}}%
\pgfpathlineto{\pgfqpoint{5.849815in}{4.522824in}}%
\pgfusepath{stroke}%
\end{pgfscope}%
\begin{pgfscope}%
\pgfpathrectangle{\pgfqpoint{0.481978in}{0.331635in}}{\pgfqpoint{9.300000in}{7.700000in}}%
\pgfusepath{clip}%
\pgfsetrectcap%
\pgfsetroundjoin%
\pgfsetlinewidth{1.505625pt}%
\definecolor{currentstroke}{rgb}{0.552941,0.898039,0.631373}%
\pgfsetstrokecolor{currentstroke}%
\pgfsetstrokeopacity{0.200000}%
\pgfsetdash{}{0pt}%
\pgfpathmoveto{\pgfqpoint{4.433234in}{5.070981in}}%
\pgfpathlineto{\pgfqpoint{5.849815in}{4.522824in}}%
\pgfusepath{stroke}%
\end{pgfscope}%
\begin{pgfscope}%
\pgfpathrectangle{\pgfqpoint{0.481978in}{0.331635in}}{\pgfqpoint{9.300000in}{7.700000in}}%
\pgfusepath{clip}%
\pgfsetrectcap%
\pgfsetroundjoin%
\pgfsetlinewidth{1.505625pt}%
\definecolor{currentstroke}{rgb}{0.552941,0.898039,0.631373}%
\pgfsetstrokecolor{currentstroke}%
\pgfsetstrokeopacity{0.200000}%
\pgfsetdash{}{0pt}%
\pgfpathmoveto{\pgfqpoint{9.194655in}{5.663935in}}%
\pgfpathlineto{\pgfqpoint{5.849815in}{4.522824in}}%
\pgfusepath{stroke}%
\end{pgfscope}%
\begin{pgfscope}%
\pgfpathrectangle{\pgfqpoint{0.481978in}{0.331635in}}{\pgfqpoint{9.300000in}{7.700000in}}%
\pgfusepath{clip}%
\pgfsetrectcap%
\pgfsetroundjoin%
\pgfsetlinewidth{1.505625pt}%
\definecolor{currentstroke}{rgb}{0.552941,0.898039,0.631373}%
\pgfsetstrokecolor{currentstroke}%
\pgfsetstrokeopacity{0.200000}%
\pgfsetdash{}{0pt}%
\pgfpathmoveto{\pgfqpoint{9.032589in}{5.080119in}}%
\pgfpathlineto{\pgfqpoint{5.849815in}{4.522824in}}%
\pgfusepath{stroke}%
\end{pgfscope}%
\begin{pgfscope}%
\pgfpathrectangle{\pgfqpoint{0.481978in}{0.331635in}}{\pgfqpoint{9.300000in}{7.700000in}}%
\pgfusepath{clip}%
\pgfsetrectcap%
\pgfsetroundjoin%
\pgfsetlinewidth{1.505625pt}%
\definecolor{currentstroke}{rgb}{0.552941,0.898039,0.631373}%
\pgfsetstrokecolor{currentstroke}%
\pgfsetstrokeopacity{0.200000}%
\pgfsetdash{}{0pt}%
\pgfpathmoveto{\pgfqpoint{3.942708in}{5.673543in}}%
\pgfpathlineto{\pgfqpoint{5.849815in}{4.522824in}}%
\pgfusepath{stroke}%
\end{pgfscope}%
\begin{pgfscope}%
\pgfpathrectangle{\pgfqpoint{0.481978in}{0.331635in}}{\pgfqpoint{9.300000in}{7.700000in}}%
\pgfusepath{clip}%
\pgfsetrectcap%
\pgfsetroundjoin%
\pgfsetlinewidth{1.505625pt}%
\definecolor{currentstroke}{rgb}{0.552941,0.898039,0.631373}%
\pgfsetstrokecolor{currentstroke}%
\pgfsetstrokeopacity{0.200000}%
\pgfsetdash{}{0pt}%
\pgfpathmoveto{\pgfqpoint{5.874679in}{2.722779in}}%
\pgfpathlineto{\pgfqpoint{5.849815in}{4.522824in}}%
\pgfusepath{stroke}%
\end{pgfscope}%
\begin{pgfscope}%
\pgfpathrectangle{\pgfqpoint{0.481978in}{0.331635in}}{\pgfqpoint{9.300000in}{7.700000in}}%
\pgfusepath{clip}%
\pgfsetrectcap%
\pgfsetroundjoin%
\pgfsetlinewidth{1.505625pt}%
\definecolor{currentstroke}{rgb}{0.552941,0.898039,0.631373}%
\pgfsetstrokecolor{currentstroke}%
\pgfsetstrokeopacity{0.200000}%
\pgfsetdash{}{0pt}%
\pgfpathmoveto{\pgfqpoint{6.449863in}{3.180896in}}%
\pgfpathlineto{\pgfqpoint{5.849815in}{4.522824in}}%
\pgfusepath{stroke}%
\end{pgfscope}%
\begin{pgfscope}%
\pgfpathrectangle{\pgfqpoint{0.481978in}{0.331635in}}{\pgfqpoint{9.300000in}{7.700000in}}%
\pgfusepath{clip}%
\pgfsetrectcap%
\pgfsetroundjoin%
\pgfsetlinewidth{1.505625pt}%
\definecolor{currentstroke}{rgb}{0.552941,0.898039,0.631373}%
\pgfsetstrokecolor{currentstroke}%
\pgfsetstrokeopacity{0.200000}%
\pgfsetdash{}{0pt}%
\pgfpathmoveto{\pgfqpoint{2.941597in}{6.366988in}}%
\pgfpathlineto{\pgfqpoint{5.849815in}{4.522824in}}%
\pgfusepath{stroke}%
\end{pgfscope}%
\begin{pgfscope}%
\pgfpathrectangle{\pgfqpoint{0.481978in}{0.331635in}}{\pgfqpoint{9.300000in}{7.700000in}}%
\pgfusepath{clip}%
\pgfsetrectcap%
\pgfsetroundjoin%
\pgfsetlinewidth{1.505625pt}%
\definecolor{currentstroke}{rgb}{0.552941,0.898039,0.631373}%
\pgfsetstrokecolor{currentstroke}%
\pgfsetstrokeopacity{0.200000}%
\pgfsetdash{}{0pt}%
\pgfpathmoveto{\pgfqpoint{4.671581in}{6.114771in}}%
\pgfpathlineto{\pgfqpoint{5.849815in}{4.522824in}}%
\pgfusepath{stroke}%
\end{pgfscope}%
\begin{pgfscope}%
\pgfpathrectangle{\pgfqpoint{0.481978in}{0.331635in}}{\pgfqpoint{9.300000in}{7.700000in}}%
\pgfusepath{clip}%
\pgfsetrectcap%
\pgfsetroundjoin%
\pgfsetlinewidth{1.505625pt}%
\definecolor{currentstroke}{rgb}{0.552941,0.898039,0.631373}%
\pgfsetstrokecolor{currentstroke}%
\pgfsetstrokeopacity{0.200000}%
\pgfsetdash{}{0pt}%
\pgfpathmoveto{\pgfqpoint{3.259508in}{2.958122in}}%
\pgfpathlineto{\pgfqpoint{5.849815in}{4.522824in}}%
\pgfusepath{stroke}%
\end{pgfscope}%
\begin{pgfscope}%
\pgfpathrectangle{\pgfqpoint{0.481978in}{0.331635in}}{\pgfqpoint{9.300000in}{7.700000in}}%
\pgfusepath{clip}%
\pgfsetrectcap%
\pgfsetroundjoin%
\pgfsetlinewidth{1.505625pt}%
\definecolor{currentstroke}{rgb}{0.552941,0.898039,0.631373}%
\pgfsetstrokecolor{currentstroke}%
\pgfsetstrokeopacity{0.200000}%
\pgfsetdash{}{0pt}%
\pgfpathmoveto{\pgfqpoint{7.805062in}{3.444564in}}%
\pgfpathlineto{\pgfqpoint{5.849815in}{4.522824in}}%
\pgfusepath{stroke}%
\end{pgfscope}%
\begin{pgfscope}%
\pgfpathrectangle{\pgfqpoint{0.481978in}{0.331635in}}{\pgfqpoint{9.300000in}{7.700000in}}%
\pgfusepath{clip}%
\pgfsetrectcap%
\pgfsetroundjoin%
\pgfsetlinewidth{1.505625pt}%
\definecolor{currentstroke}{rgb}{0.552941,0.898039,0.631373}%
\pgfsetstrokecolor{currentstroke}%
\pgfsetstrokeopacity{0.200000}%
\pgfsetdash{}{0pt}%
\pgfpathmoveto{\pgfqpoint{4.750388in}{7.194691in}}%
\pgfpathlineto{\pgfqpoint{5.849815in}{4.522824in}}%
\pgfusepath{stroke}%
\end{pgfscope}%
\begin{pgfscope}%
\pgfpathrectangle{\pgfqpoint{0.481978in}{0.331635in}}{\pgfqpoint{9.300000in}{7.700000in}}%
\pgfusepath{clip}%
\pgfsetrectcap%
\pgfsetroundjoin%
\pgfsetlinewidth{1.505625pt}%
\definecolor{currentstroke}{rgb}{0.552941,0.898039,0.631373}%
\pgfsetstrokecolor{currentstroke}%
\pgfsetstrokeopacity{0.200000}%
\pgfsetdash{}{0pt}%
\pgfpathmoveto{\pgfqpoint{4.386424in}{5.571510in}}%
\pgfpathlineto{\pgfqpoint{5.849815in}{4.522824in}}%
\pgfusepath{stroke}%
\end{pgfscope}%
\begin{pgfscope}%
\pgfpathrectangle{\pgfqpoint{0.481978in}{0.331635in}}{\pgfqpoint{9.300000in}{7.700000in}}%
\pgfusepath{clip}%
\pgfsetrectcap%
\pgfsetroundjoin%
\pgfsetlinewidth{1.505625pt}%
\definecolor{currentstroke}{rgb}{0.552941,0.898039,0.631373}%
\pgfsetstrokecolor{currentstroke}%
\pgfsetstrokeopacity{0.200000}%
\pgfsetdash{}{0pt}%
\pgfpathmoveto{\pgfqpoint{4.878045in}{6.103326in}}%
\pgfpathlineto{\pgfqpoint{5.849815in}{4.522824in}}%
\pgfusepath{stroke}%
\end{pgfscope}%
\begin{pgfscope}%
\pgfpathrectangle{\pgfqpoint{0.481978in}{0.331635in}}{\pgfqpoint{9.300000in}{7.700000in}}%
\pgfusepath{clip}%
\pgfsetrectcap%
\pgfsetroundjoin%
\pgfsetlinewidth{1.505625pt}%
\definecolor{currentstroke}{rgb}{0.552941,0.898039,0.631373}%
\pgfsetstrokecolor{currentstroke}%
\pgfsetstrokeopacity{0.200000}%
\pgfsetdash{}{0pt}%
\pgfpathmoveto{\pgfqpoint{9.259474in}{5.199290in}}%
\pgfpathlineto{\pgfqpoint{5.849815in}{4.522824in}}%
\pgfusepath{stroke}%
\end{pgfscope}%
\begin{pgfscope}%
\pgfpathrectangle{\pgfqpoint{0.481978in}{0.331635in}}{\pgfqpoint{9.300000in}{7.700000in}}%
\pgfusepath{clip}%
\pgfsetrectcap%
\pgfsetroundjoin%
\pgfsetlinewidth{1.505625pt}%
\definecolor{currentstroke}{rgb}{0.552941,0.898039,0.631373}%
\pgfsetstrokecolor{currentstroke}%
\pgfsetstrokeopacity{0.200000}%
\pgfsetdash{}{0pt}%
\pgfpathmoveto{\pgfqpoint{8.569592in}{4.792319in}}%
\pgfpathlineto{\pgfqpoint{5.849815in}{4.522824in}}%
\pgfusepath{stroke}%
\end{pgfscope}%
\begin{pgfscope}%
\pgfpathrectangle{\pgfqpoint{0.481978in}{0.331635in}}{\pgfqpoint{9.300000in}{7.700000in}}%
\pgfusepath{clip}%
\pgfsetrectcap%
\pgfsetroundjoin%
\pgfsetlinewidth{1.505625pt}%
\definecolor{currentstroke}{rgb}{0.552941,0.898039,0.631373}%
\pgfsetstrokecolor{currentstroke}%
\pgfsetstrokeopacity{0.200000}%
\pgfsetdash{}{0pt}%
\pgfpathmoveto{\pgfqpoint{8.786772in}{4.571593in}}%
\pgfpathlineto{\pgfqpoint{5.849815in}{4.522824in}}%
\pgfusepath{stroke}%
\end{pgfscope}%
\begin{pgfscope}%
\pgfpathrectangle{\pgfqpoint{0.481978in}{0.331635in}}{\pgfqpoint{9.300000in}{7.700000in}}%
\pgfusepath{clip}%
\pgfsetrectcap%
\pgfsetroundjoin%
\pgfsetlinewidth{1.505625pt}%
\definecolor{currentstroke}{rgb}{0.552941,0.898039,0.631373}%
\pgfsetstrokecolor{currentstroke}%
\pgfsetstrokeopacity{0.200000}%
\pgfsetdash{}{0pt}%
\pgfpathmoveto{\pgfqpoint{7.733235in}{4.906244in}}%
\pgfpathlineto{\pgfqpoint{5.849815in}{4.522824in}}%
\pgfusepath{stroke}%
\end{pgfscope}%
\begin{pgfscope}%
\pgfpathrectangle{\pgfqpoint{0.481978in}{0.331635in}}{\pgfqpoint{9.300000in}{7.700000in}}%
\pgfusepath{clip}%
\pgfsetrectcap%
\pgfsetroundjoin%
\pgfsetlinewidth{1.505625pt}%
\definecolor{currentstroke}{rgb}{0.552941,0.898039,0.631373}%
\pgfsetstrokecolor{currentstroke}%
\pgfsetstrokeopacity{0.200000}%
\pgfsetdash{}{0pt}%
\pgfpathmoveto{\pgfqpoint{5.023248in}{5.832186in}}%
\pgfpathlineto{\pgfqpoint{5.849815in}{4.522824in}}%
\pgfusepath{stroke}%
\end{pgfscope}%
\begin{pgfscope}%
\pgfpathrectangle{\pgfqpoint{0.481978in}{0.331635in}}{\pgfqpoint{9.300000in}{7.700000in}}%
\pgfusepath{clip}%
\pgfsetrectcap%
\pgfsetroundjoin%
\pgfsetlinewidth{1.505625pt}%
\definecolor{currentstroke}{rgb}{0.552941,0.898039,0.631373}%
\pgfsetstrokecolor{currentstroke}%
\pgfsetstrokeopacity{0.200000}%
\pgfsetdash{}{0pt}%
\pgfpathmoveto{\pgfqpoint{4.955252in}{6.273953in}}%
\pgfpathlineto{\pgfqpoint{5.849815in}{4.522824in}}%
\pgfusepath{stroke}%
\end{pgfscope}%
\begin{pgfscope}%
\pgfpathrectangle{\pgfqpoint{0.481978in}{0.331635in}}{\pgfqpoint{9.300000in}{7.700000in}}%
\pgfusepath{clip}%
\pgfsetrectcap%
\pgfsetroundjoin%
\pgfsetlinewidth{1.505625pt}%
\definecolor{currentstroke}{rgb}{0.552941,0.898039,0.631373}%
\pgfsetstrokecolor{currentstroke}%
\pgfsetstrokeopacity{0.200000}%
\pgfsetdash{}{0pt}%
\pgfpathmoveto{\pgfqpoint{7.433098in}{2.265608in}}%
\pgfpathlineto{\pgfqpoint{5.849815in}{4.522824in}}%
\pgfusepath{stroke}%
\end{pgfscope}%
\begin{pgfscope}%
\pgfpathrectangle{\pgfqpoint{0.481978in}{0.331635in}}{\pgfqpoint{9.300000in}{7.700000in}}%
\pgfusepath{clip}%
\pgfsetrectcap%
\pgfsetroundjoin%
\pgfsetlinewidth{1.505625pt}%
\definecolor{currentstroke}{rgb}{0.552941,0.898039,0.631373}%
\pgfsetstrokecolor{currentstroke}%
\pgfsetstrokeopacity{0.200000}%
\pgfsetdash{}{0pt}%
\pgfpathmoveto{\pgfqpoint{8.232070in}{5.815087in}}%
\pgfpathlineto{\pgfqpoint{5.849815in}{4.522824in}}%
\pgfusepath{stroke}%
\end{pgfscope}%
\begin{pgfscope}%
\pgfpathrectangle{\pgfqpoint{0.481978in}{0.331635in}}{\pgfqpoint{9.300000in}{7.700000in}}%
\pgfusepath{clip}%
\pgfsetrectcap%
\pgfsetroundjoin%
\pgfsetlinewidth{1.505625pt}%
\definecolor{currentstroke}{rgb}{0.552941,0.898039,0.631373}%
\pgfsetstrokecolor{currentstroke}%
\pgfsetstrokeopacity{0.200000}%
\pgfsetdash{}{0pt}%
\pgfpathmoveto{\pgfqpoint{3.011222in}{1.685197in}}%
\pgfpathlineto{\pgfqpoint{5.849815in}{4.522824in}}%
\pgfusepath{stroke}%
\end{pgfscope}%
\begin{pgfscope}%
\pgfpathrectangle{\pgfqpoint{0.481978in}{0.331635in}}{\pgfqpoint{9.300000in}{7.700000in}}%
\pgfusepath{clip}%
\pgfsetrectcap%
\pgfsetroundjoin%
\pgfsetlinewidth{1.505625pt}%
\definecolor{currentstroke}{rgb}{0.552941,0.898039,0.631373}%
\pgfsetstrokecolor{currentstroke}%
\pgfsetstrokeopacity{0.200000}%
\pgfsetdash{}{0pt}%
\pgfpathmoveto{\pgfqpoint{4.782892in}{6.148818in}}%
\pgfpathlineto{\pgfqpoint{5.849815in}{4.522824in}}%
\pgfusepath{stroke}%
\end{pgfscope}%
\begin{pgfscope}%
\pgfpathrectangle{\pgfqpoint{0.481978in}{0.331635in}}{\pgfqpoint{9.300000in}{7.700000in}}%
\pgfusepath{clip}%
\pgfsetrectcap%
\pgfsetroundjoin%
\pgfsetlinewidth{1.505625pt}%
\definecolor{currentstroke}{rgb}{0.552941,0.898039,0.631373}%
\pgfsetstrokecolor{currentstroke}%
\pgfsetstrokeopacity{0.200000}%
\pgfsetdash{}{0pt}%
\pgfpathmoveto{\pgfqpoint{3.628307in}{5.297078in}}%
\pgfpathlineto{\pgfqpoint{5.849815in}{4.522824in}}%
\pgfusepath{stroke}%
\end{pgfscope}%
\begin{pgfscope}%
\pgfpathrectangle{\pgfqpoint{0.481978in}{0.331635in}}{\pgfqpoint{9.300000in}{7.700000in}}%
\pgfusepath{clip}%
\pgfsetrectcap%
\pgfsetroundjoin%
\pgfsetlinewidth{1.505625pt}%
\definecolor{currentstroke}{rgb}{0.552941,0.898039,0.631373}%
\pgfsetstrokecolor{currentstroke}%
\pgfsetstrokeopacity{0.200000}%
\pgfsetdash{}{0pt}%
\pgfpathmoveto{\pgfqpoint{3.687739in}{5.241357in}}%
\pgfpathlineto{\pgfqpoint{5.849815in}{4.522824in}}%
\pgfusepath{stroke}%
\end{pgfscope}%
\begin{pgfscope}%
\pgfpathrectangle{\pgfqpoint{0.481978in}{0.331635in}}{\pgfqpoint{9.300000in}{7.700000in}}%
\pgfusepath{clip}%
\pgfsetrectcap%
\pgfsetroundjoin%
\pgfsetlinewidth{1.505625pt}%
\definecolor{currentstroke}{rgb}{0.552941,0.898039,0.631373}%
\pgfsetstrokecolor{currentstroke}%
\pgfsetstrokeopacity{0.200000}%
\pgfsetdash{}{0pt}%
\pgfpathmoveto{\pgfqpoint{7.445589in}{2.826774in}}%
\pgfpathlineto{\pgfqpoint{5.849815in}{4.522824in}}%
\pgfusepath{stroke}%
\end{pgfscope}%
\begin{pgfscope}%
\pgfpathrectangle{\pgfqpoint{0.481978in}{0.331635in}}{\pgfqpoint{9.300000in}{7.700000in}}%
\pgfusepath{clip}%
\pgfsetrectcap%
\pgfsetroundjoin%
\pgfsetlinewidth{1.505625pt}%
\definecolor{currentstroke}{rgb}{0.552941,0.898039,0.631373}%
\pgfsetstrokecolor{currentstroke}%
\pgfsetstrokeopacity{0.200000}%
\pgfsetdash{}{0pt}%
\pgfpathmoveto{\pgfqpoint{8.934791in}{5.143139in}}%
\pgfpathlineto{\pgfqpoint{5.849815in}{4.522824in}}%
\pgfusepath{stroke}%
\end{pgfscope}%
\begin{pgfscope}%
\pgfpathrectangle{\pgfqpoint{0.481978in}{0.331635in}}{\pgfqpoint{9.300000in}{7.700000in}}%
\pgfusepath{clip}%
\pgfsetrectcap%
\pgfsetroundjoin%
\pgfsetlinewidth{1.505625pt}%
\definecolor{currentstroke}{rgb}{0.552941,0.898039,0.631373}%
\pgfsetstrokecolor{currentstroke}%
\pgfsetstrokeopacity{0.200000}%
\pgfsetdash{}{0pt}%
\pgfpathmoveto{\pgfqpoint{3.508572in}{3.721693in}}%
\pgfpathlineto{\pgfqpoint{5.849815in}{4.522824in}}%
\pgfusepath{stroke}%
\end{pgfscope}%
\begin{pgfscope}%
\pgfpathrectangle{\pgfqpoint{0.481978in}{0.331635in}}{\pgfqpoint{9.300000in}{7.700000in}}%
\pgfusepath{clip}%
\pgfsetrectcap%
\pgfsetroundjoin%
\pgfsetlinewidth{1.505625pt}%
\definecolor{currentstroke}{rgb}{0.552941,0.898039,0.631373}%
\pgfsetstrokecolor{currentstroke}%
\pgfsetstrokeopacity{0.200000}%
\pgfsetdash{}{0pt}%
\pgfpathmoveto{\pgfqpoint{4.192874in}{1.559524in}}%
\pgfpathlineto{\pgfqpoint{5.849815in}{4.522824in}}%
\pgfusepath{stroke}%
\end{pgfscope}%
\begin{pgfscope}%
\pgfpathrectangle{\pgfqpoint{0.481978in}{0.331635in}}{\pgfqpoint{9.300000in}{7.700000in}}%
\pgfusepath{clip}%
\pgfsetrectcap%
\pgfsetroundjoin%
\pgfsetlinewidth{1.505625pt}%
\definecolor{currentstroke}{rgb}{0.552941,0.898039,0.631373}%
\pgfsetstrokecolor{currentstroke}%
\pgfsetstrokeopacity{0.200000}%
\pgfsetdash{}{0pt}%
\pgfpathmoveto{\pgfqpoint{4.001930in}{5.032636in}}%
\pgfpathlineto{\pgfqpoint{5.849815in}{4.522824in}}%
\pgfusepath{stroke}%
\end{pgfscope}%
\begin{pgfscope}%
\pgfpathrectangle{\pgfqpoint{0.481978in}{0.331635in}}{\pgfqpoint{9.300000in}{7.700000in}}%
\pgfusepath{clip}%
\pgfsetrectcap%
\pgfsetroundjoin%
\pgfsetlinewidth{1.505625pt}%
\definecolor{currentstroke}{rgb}{0.552941,0.898039,0.631373}%
\pgfsetstrokecolor{currentstroke}%
\pgfsetstrokeopacity{0.200000}%
\pgfsetdash{}{0pt}%
\pgfpathmoveto{\pgfqpoint{6.060999in}{1.792912in}}%
\pgfpathlineto{\pgfqpoint{5.849815in}{4.522824in}}%
\pgfusepath{stroke}%
\end{pgfscope}%
\begin{pgfscope}%
\pgfpathrectangle{\pgfqpoint{0.481978in}{0.331635in}}{\pgfqpoint{9.300000in}{7.700000in}}%
\pgfusepath{clip}%
\pgfsetrectcap%
\pgfsetroundjoin%
\pgfsetlinewidth{1.505625pt}%
\definecolor{currentstroke}{rgb}{0.552941,0.898039,0.631373}%
\pgfsetstrokecolor{currentstroke}%
\pgfsetstrokeopacity{0.200000}%
\pgfsetdash{}{0pt}%
\pgfpathmoveto{\pgfqpoint{2.449894in}{1.698466in}}%
\pgfpathlineto{\pgfqpoint{5.849815in}{4.522824in}}%
\pgfusepath{stroke}%
\end{pgfscope}%
\begin{pgfscope}%
\pgfpathrectangle{\pgfqpoint{0.481978in}{0.331635in}}{\pgfqpoint{9.300000in}{7.700000in}}%
\pgfusepath{clip}%
\pgfsetrectcap%
\pgfsetroundjoin%
\pgfsetlinewidth{1.505625pt}%
\definecolor{currentstroke}{rgb}{0.552941,0.898039,0.631373}%
\pgfsetstrokecolor{currentstroke}%
\pgfsetstrokeopacity{0.200000}%
\pgfsetdash{}{0pt}%
\pgfpathmoveto{\pgfqpoint{7.028940in}{4.751090in}}%
\pgfpathlineto{\pgfqpoint{5.849815in}{4.522824in}}%
\pgfusepath{stroke}%
\end{pgfscope}%
\begin{pgfscope}%
\pgfpathrectangle{\pgfqpoint{0.481978in}{0.331635in}}{\pgfqpoint{9.300000in}{7.700000in}}%
\pgfusepath{clip}%
\pgfsetrectcap%
\pgfsetroundjoin%
\pgfsetlinewidth{1.505625pt}%
\definecolor{currentstroke}{rgb}{0.552941,0.898039,0.631373}%
\pgfsetstrokecolor{currentstroke}%
\pgfsetstrokeopacity{0.200000}%
\pgfsetdash{}{0pt}%
\pgfpathmoveto{\pgfqpoint{5.259017in}{7.595751in}}%
\pgfpathlineto{\pgfqpoint{5.849815in}{4.522824in}}%
\pgfusepath{stroke}%
\end{pgfscope}%
\begin{pgfscope}%
\pgfpathrectangle{\pgfqpoint{0.481978in}{0.331635in}}{\pgfqpoint{9.300000in}{7.700000in}}%
\pgfusepath{clip}%
\pgfsetrectcap%
\pgfsetroundjoin%
\pgfsetlinewidth{1.505625pt}%
\definecolor{currentstroke}{rgb}{0.552941,0.898039,0.631373}%
\pgfsetstrokecolor{currentstroke}%
\pgfsetstrokeopacity{0.200000}%
\pgfsetdash{}{0pt}%
\pgfpathmoveto{\pgfqpoint{9.184137in}{5.106329in}}%
\pgfpathlineto{\pgfqpoint{5.849815in}{4.522824in}}%
\pgfusepath{stroke}%
\end{pgfscope}%
\begin{pgfscope}%
\pgfpathrectangle{\pgfqpoint{0.481978in}{0.331635in}}{\pgfqpoint{9.300000in}{7.700000in}}%
\pgfusepath{clip}%
\pgfsetrectcap%
\pgfsetroundjoin%
\pgfsetlinewidth{1.505625pt}%
\definecolor{currentstroke}{rgb}{0.552941,0.898039,0.631373}%
\pgfsetstrokecolor{currentstroke}%
\pgfsetstrokeopacity{0.200000}%
\pgfsetdash{}{0pt}%
\pgfpathmoveto{\pgfqpoint{8.944873in}{6.293748in}}%
\pgfpathlineto{\pgfqpoint{5.849815in}{4.522824in}}%
\pgfusepath{stroke}%
\end{pgfscope}%
\begin{pgfscope}%
\pgfpathrectangle{\pgfqpoint{0.481978in}{0.331635in}}{\pgfqpoint{9.300000in}{7.700000in}}%
\pgfusepath{clip}%
\pgfsetrectcap%
\pgfsetroundjoin%
\pgfsetlinewidth{1.505625pt}%
\definecolor{currentstroke}{rgb}{0.552941,0.898039,0.631373}%
\pgfsetstrokecolor{currentstroke}%
\pgfsetstrokeopacity{0.200000}%
\pgfsetdash{}{0pt}%
\pgfpathmoveto{\pgfqpoint{7.733013in}{3.491196in}}%
\pgfpathlineto{\pgfqpoint{5.849815in}{4.522824in}}%
\pgfusepath{stroke}%
\end{pgfscope}%
\begin{pgfscope}%
\pgfpathrectangle{\pgfqpoint{0.481978in}{0.331635in}}{\pgfqpoint{9.300000in}{7.700000in}}%
\pgfusepath{clip}%
\pgfsetrectcap%
\pgfsetroundjoin%
\pgfsetlinewidth{1.505625pt}%
\definecolor{currentstroke}{rgb}{0.552941,0.898039,0.631373}%
\pgfsetstrokecolor{currentstroke}%
\pgfsetstrokeopacity{0.200000}%
\pgfsetdash{}{0pt}%
\pgfpathmoveto{\pgfqpoint{5.907061in}{2.100838in}}%
\pgfpathlineto{\pgfqpoint{5.849815in}{4.522824in}}%
\pgfusepath{stroke}%
\end{pgfscope}%
\begin{pgfscope}%
\pgfpathrectangle{\pgfqpoint{0.481978in}{0.331635in}}{\pgfqpoint{9.300000in}{7.700000in}}%
\pgfusepath{clip}%
\pgfsetrectcap%
\pgfsetroundjoin%
\pgfsetlinewidth{1.505625pt}%
\definecolor{currentstroke}{rgb}{0.552941,0.898039,0.631373}%
\pgfsetstrokecolor{currentstroke}%
\pgfsetstrokeopacity{0.200000}%
\pgfsetdash{}{0pt}%
\pgfpathmoveto{\pgfqpoint{4.839240in}{5.911338in}}%
\pgfpathlineto{\pgfqpoint{5.849815in}{4.522824in}}%
\pgfusepath{stroke}%
\end{pgfscope}%
\begin{pgfscope}%
\pgfpathrectangle{\pgfqpoint{0.481978in}{0.331635in}}{\pgfqpoint{9.300000in}{7.700000in}}%
\pgfusepath{clip}%
\pgfsetrectcap%
\pgfsetroundjoin%
\pgfsetlinewidth{1.505625pt}%
\definecolor{currentstroke}{rgb}{0.552941,0.898039,0.631373}%
\pgfsetstrokecolor{currentstroke}%
\pgfsetstrokeopacity{0.200000}%
\pgfsetdash{}{0pt}%
\pgfpathmoveto{\pgfqpoint{2.381245in}{2.304607in}}%
\pgfpathlineto{\pgfqpoint{5.849815in}{4.522824in}}%
\pgfusepath{stroke}%
\end{pgfscope}%
\begin{pgfscope}%
\pgfpathrectangle{\pgfqpoint{0.481978in}{0.331635in}}{\pgfqpoint{9.300000in}{7.700000in}}%
\pgfusepath{clip}%
\pgfsetrectcap%
\pgfsetroundjoin%
\pgfsetlinewidth{1.505625pt}%
\definecolor{currentstroke}{rgb}{0.552941,0.898039,0.631373}%
\pgfsetstrokecolor{currentstroke}%
\pgfsetstrokeopacity{0.200000}%
\pgfsetdash{}{0pt}%
\pgfpathmoveto{\pgfqpoint{7.266400in}{2.402434in}}%
\pgfpathlineto{\pgfqpoint{5.849815in}{4.522824in}}%
\pgfusepath{stroke}%
\end{pgfscope}%
\begin{pgfscope}%
\pgfpathrectangle{\pgfqpoint{0.481978in}{0.331635in}}{\pgfqpoint{9.300000in}{7.700000in}}%
\pgfusepath{clip}%
\pgfsetrectcap%
\pgfsetroundjoin%
\pgfsetlinewidth{1.505625pt}%
\definecolor{currentstroke}{rgb}{0.552941,0.898039,0.631373}%
\pgfsetstrokecolor{currentstroke}%
\pgfsetstrokeopacity{0.200000}%
\pgfsetdash{}{0pt}%
\pgfpathmoveto{\pgfqpoint{3.410609in}{3.381276in}}%
\pgfpathlineto{\pgfqpoint{5.849815in}{4.522824in}}%
\pgfusepath{stroke}%
\end{pgfscope}%
\begin{pgfscope}%
\pgfpathrectangle{\pgfqpoint{0.481978in}{0.331635in}}{\pgfqpoint{9.300000in}{7.700000in}}%
\pgfusepath{clip}%
\pgfsetrectcap%
\pgfsetroundjoin%
\pgfsetlinewidth{1.505625pt}%
\definecolor{currentstroke}{rgb}{0.552941,0.898039,0.631373}%
\pgfsetstrokecolor{currentstroke}%
\pgfsetstrokeopacity{0.200000}%
\pgfsetdash{}{0pt}%
\pgfpathmoveto{\pgfqpoint{9.044900in}{5.310730in}}%
\pgfpathlineto{\pgfqpoint{5.849815in}{4.522824in}}%
\pgfusepath{stroke}%
\end{pgfscope}%
\begin{pgfscope}%
\pgfpathrectangle{\pgfqpoint{0.481978in}{0.331635in}}{\pgfqpoint{9.300000in}{7.700000in}}%
\pgfusepath{clip}%
\pgfsetrectcap%
\pgfsetroundjoin%
\pgfsetlinewidth{1.505625pt}%
\definecolor{currentstroke}{rgb}{0.552941,0.898039,0.631373}%
\pgfsetstrokecolor{currentstroke}%
\pgfsetstrokeopacity{0.200000}%
\pgfsetdash{}{0pt}%
\pgfpathmoveto{\pgfqpoint{8.024517in}{4.689990in}}%
\pgfpathlineto{\pgfqpoint{5.849815in}{4.522824in}}%
\pgfusepath{stroke}%
\end{pgfscope}%
\begin{pgfscope}%
\pgfpathrectangle{\pgfqpoint{0.481978in}{0.331635in}}{\pgfqpoint{9.300000in}{7.700000in}}%
\pgfusepath{clip}%
\pgfsetrectcap%
\pgfsetroundjoin%
\pgfsetlinewidth{1.505625pt}%
\definecolor{currentstroke}{rgb}{0.552941,0.898039,0.631373}%
\pgfsetstrokecolor{currentstroke}%
\pgfsetstrokeopacity{0.200000}%
\pgfsetdash{}{0pt}%
\pgfpathmoveto{\pgfqpoint{3.710429in}{4.755530in}}%
\pgfpathlineto{\pgfqpoint{5.849815in}{4.522824in}}%
\pgfusepath{stroke}%
\end{pgfscope}%
\begin{pgfscope}%
\pgfpathrectangle{\pgfqpoint{0.481978in}{0.331635in}}{\pgfqpoint{9.300000in}{7.700000in}}%
\pgfusepath{clip}%
\pgfsetrectcap%
\pgfsetroundjoin%
\pgfsetlinewidth{1.505625pt}%
\definecolor{currentstroke}{rgb}{0.552941,0.898039,0.631373}%
\pgfsetstrokecolor{currentstroke}%
\pgfsetstrokeopacity{0.200000}%
\pgfsetdash{}{0pt}%
\pgfpathmoveto{\pgfqpoint{5.690812in}{2.678714in}}%
\pgfpathlineto{\pgfqpoint{5.849815in}{4.522824in}}%
\pgfusepath{stroke}%
\end{pgfscope}%
\begin{pgfscope}%
\pgfpathrectangle{\pgfqpoint{0.481978in}{0.331635in}}{\pgfqpoint{9.300000in}{7.700000in}}%
\pgfusepath{clip}%
\pgfsetrectcap%
\pgfsetroundjoin%
\pgfsetlinewidth{1.505625pt}%
\definecolor{currentstroke}{rgb}{1.000000,0.623529,0.607843}%
\pgfsetstrokecolor{currentstroke}%
\pgfsetstrokeopacity{0.200000}%
\pgfsetdash{}{0pt}%
\pgfpathmoveto{\pgfqpoint{7.737423in}{5.882190in}}%
\pgfpathlineto{\pgfqpoint{5.342251in}{3.998981in}}%
\pgfusepath{stroke}%
\end{pgfscope}%
\begin{pgfscope}%
\pgfpathrectangle{\pgfqpoint{0.481978in}{0.331635in}}{\pgfqpoint{9.300000in}{7.700000in}}%
\pgfusepath{clip}%
\pgfsetrectcap%
\pgfsetroundjoin%
\pgfsetlinewidth{1.505625pt}%
\definecolor{currentstroke}{rgb}{1.000000,0.623529,0.607843}%
\pgfsetstrokecolor{currentstroke}%
\pgfsetstrokeopacity{0.200000}%
\pgfsetdash{}{0pt}%
\pgfpathmoveto{\pgfqpoint{8.478500in}{5.602262in}}%
\pgfpathlineto{\pgfqpoint{5.342251in}{3.998981in}}%
\pgfusepath{stroke}%
\end{pgfscope}%
\begin{pgfscope}%
\pgfpathrectangle{\pgfqpoint{0.481978in}{0.331635in}}{\pgfqpoint{9.300000in}{7.700000in}}%
\pgfusepath{clip}%
\pgfsetrectcap%
\pgfsetroundjoin%
\pgfsetlinewidth{1.505625pt}%
\definecolor{currentstroke}{rgb}{1.000000,0.623529,0.607843}%
\pgfsetstrokecolor{currentstroke}%
\pgfsetstrokeopacity{0.200000}%
\pgfsetdash{}{0pt}%
\pgfpathmoveto{\pgfqpoint{5.299357in}{2.456920in}}%
\pgfpathlineto{\pgfqpoint{5.342251in}{3.998981in}}%
\pgfusepath{stroke}%
\end{pgfscope}%
\begin{pgfscope}%
\pgfpathrectangle{\pgfqpoint{0.481978in}{0.331635in}}{\pgfqpoint{9.300000in}{7.700000in}}%
\pgfusepath{clip}%
\pgfsetrectcap%
\pgfsetroundjoin%
\pgfsetlinewidth{1.505625pt}%
\definecolor{currentstroke}{rgb}{1.000000,0.623529,0.607843}%
\pgfsetstrokecolor{currentstroke}%
\pgfsetstrokeopacity{0.200000}%
\pgfsetdash{}{0pt}%
\pgfpathmoveto{\pgfqpoint{5.050465in}{2.427924in}}%
\pgfpathlineto{\pgfqpoint{5.342251in}{3.998981in}}%
\pgfusepath{stroke}%
\end{pgfscope}%
\begin{pgfscope}%
\pgfpathrectangle{\pgfqpoint{0.481978in}{0.331635in}}{\pgfqpoint{9.300000in}{7.700000in}}%
\pgfusepath{clip}%
\pgfsetrectcap%
\pgfsetroundjoin%
\pgfsetlinewidth{1.505625pt}%
\definecolor{currentstroke}{rgb}{1.000000,0.623529,0.607843}%
\pgfsetstrokecolor{currentstroke}%
\pgfsetstrokeopacity{0.200000}%
\pgfsetdash{}{0pt}%
\pgfpathmoveto{\pgfqpoint{4.424855in}{3.491433in}}%
\pgfpathlineto{\pgfqpoint{5.342251in}{3.998981in}}%
\pgfusepath{stroke}%
\end{pgfscope}%
\begin{pgfscope}%
\pgfpathrectangle{\pgfqpoint{0.481978in}{0.331635in}}{\pgfqpoint{9.300000in}{7.700000in}}%
\pgfusepath{clip}%
\pgfsetrectcap%
\pgfsetroundjoin%
\pgfsetlinewidth{1.505625pt}%
\definecolor{currentstroke}{rgb}{1.000000,0.623529,0.607843}%
\pgfsetstrokecolor{currentstroke}%
\pgfsetstrokeopacity{0.200000}%
\pgfsetdash{}{0pt}%
\pgfpathmoveto{\pgfqpoint{4.703040in}{4.152158in}}%
\pgfpathlineto{\pgfqpoint{5.342251in}{3.998981in}}%
\pgfusepath{stroke}%
\end{pgfscope}%
\begin{pgfscope}%
\pgfpathrectangle{\pgfqpoint{0.481978in}{0.331635in}}{\pgfqpoint{9.300000in}{7.700000in}}%
\pgfusepath{clip}%
\pgfsetrectcap%
\pgfsetroundjoin%
\pgfsetlinewidth{1.505625pt}%
\definecolor{currentstroke}{rgb}{1.000000,0.623529,0.607843}%
\pgfsetstrokecolor{currentstroke}%
\pgfsetstrokeopacity{0.200000}%
\pgfsetdash{}{0pt}%
\pgfpathmoveto{\pgfqpoint{2.809853in}{3.881073in}}%
\pgfpathlineto{\pgfqpoint{5.342251in}{3.998981in}}%
\pgfusepath{stroke}%
\end{pgfscope}%
\begin{pgfscope}%
\pgfpathrectangle{\pgfqpoint{0.481978in}{0.331635in}}{\pgfqpoint{9.300000in}{7.700000in}}%
\pgfusepath{clip}%
\pgfsetrectcap%
\pgfsetroundjoin%
\pgfsetlinewidth{1.505625pt}%
\definecolor{currentstroke}{rgb}{1.000000,0.623529,0.607843}%
\pgfsetstrokecolor{currentstroke}%
\pgfsetstrokeopacity{0.200000}%
\pgfsetdash{}{0pt}%
\pgfpathmoveto{\pgfqpoint{3.971771in}{3.049785in}}%
\pgfpathlineto{\pgfqpoint{5.342251in}{3.998981in}}%
\pgfusepath{stroke}%
\end{pgfscope}%
\begin{pgfscope}%
\pgfpathrectangle{\pgfqpoint{0.481978in}{0.331635in}}{\pgfqpoint{9.300000in}{7.700000in}}%
\pgfusepath{clip}%
\pgfsetrectcap%
\pgfsetroundjoin%
\pgfsetlinewidth{1.505625pt}%
\definecolor{currentstroke}{rgb}{1.000000,0.623529,0.607843}%
\pgfsetstrokecolor{currentstroke}%
\pgfsetstrokeopacity{0.200000}%
\pgfsetdash{}{0pt}%
\pgfpathmoveto{\pgfqpoint{7.681317in}{5.102335in}}%
\pgfpathlineto{\pgfqpoint{5.342251in}{3.998981in}}%
\pgfusepath{stroke}%
\end{pgfscope}%
\begin{pgfscope}%
\pgfpathrectangle{\pgfqpoint{0.481978in}{0.331635in}}{\pgfqpoint{9.300000in}{7.700000in}}%
\pgfusepath{clip}%
\pgfsetrectcap%
\pgfsetroundjoin%
\pgfsetlinewidth{1.505625pt}%
\definecolor{currentstroke}{rgb}{1.000000,0.623529,0.607843}%
\pgfsetstrokecolor{currentstroke}%
\pgfsetstrokeopacity{0.200000}%
\pgfsetdash{}{0pt}%
\pgfpathmoveto{\pgfqpoint{4.922074in}{3.141067in}}%
\pgfpathlineto{\pgfqpoint{5.342251in}{3.998981in}}%
\pgfusepath{stroke}%
\end{pgfscope}%
\begin{pgfscope}%
\pgfpathrectangle{\pgfqpoint{0.481978in}{0.331635in}}{\pgfqpoint{9.300000in}{7.700000in}}%
\pgfusepath{clip}%
\pgfsetrectcap%
\pgfsetroundjoin%
\pgfsetlinewidth{1.505625pt}%
\definecolor{currentstroke}{rgb}{1.000000,0.623529,0.607843}%
\pgfsetstrokecolor{currentstroke}%
\pgfsetstrokeopacity{0.200000}%
\pgfsetdash{}{0pt}%
\pgfpathmoveto{\pgfqpoint{4.349051in}{3.146159in}}%
\pgfpathlineto{\pgfqpoint{5.342251in}{3.998981in}}%
\pgfusepath{stroke}%
\end{pgfscope}%
\begin{pgfscope}%
\pgfpathrectangle{\pgfqpoint{0.481978in}{0.331635in}}{\pgfqpoint{9.300000in}{7.700000in}}%
\pgfusepath{clip}%
\pgfsetrectcap%
\pgfsetroundjoin%
\pgfsetlinewidth{1.505625pt}%
\definecolor{currentstroke}{rgb}{1.000000,0.623529,0.607843}%
\pgfsetstrokecolor{currentstroke}%
\pgfsetstrokeopacity{0.200000}%
\pgfsetdash{}{0pt}%
\pgfpathmoveto{\pgfqpoint{4.285306in}{3.897257in}}%
\pgfpathlineto{\pgfqpoint{5.342251in}{3.998981in}}%
\pgfusepath{stroke}%
\end{pgfscope}%
\begin{pgfscope}%
\pgfpathrectangle{\pgfqpoint{0.481978in}{0.331635in}}{\pgfqpoint{9.300000in}{7.700000in}}%
\pgfusepath{clip}%
\pgfsetrectcap%
\pgfsetroundjoin%
\pgfsetlinewidth{1.505625pt}%
\definecolor{currentstroke}{rgb}{1.000000,0.623529,0.607843}%
\pgfsetstrokecolor{currentstroke}%
\pgfsetstrokeopacity{0.200000}%
\pgfsetdash{}{0pt}%
\pgfpathmoveto{\pgfqpoint{5.628759in}{1.359840in}}%
\pgfpathlineto{\pgfqpoint{5.342251in}{3.998981in}}%
\pgfusepath{stroke}%
\end{pgfscope}%
\begin{pgfscope}%
\pgfpathrectangle{\pgfqpoint{0.481978in}{0.331635in}}{\pgfqpoint{9.300000in}{7.700000in}}%
\pgfusepath{clip}%
\pgfsetrectcap%
\pgfsetroundjoin%
\pgfsetlinewidth{1.505625pt}%
\definecolor{currentstroke}{rgb}{1.000000,0.623529,0.607843}%
\pgfsetstrokecolor{currentstroke}%
\pgfsetstrokeopacity{0.200000}%
\pgfsetdash{}{0pt}%
\pgfpathmoveto{\pgfqpoint{2.208081in}{3.731947in}}%
\pgfpathlineto{\pgfqpoint{5.342251in}{3.998981in}}%
\pgfusepath{stroke}%
\end{pgfscope}%
\begin{pgfscope}%
\pgfpathrectangle{\pgfqpoint{0.481978in}{0.331635in}}{\pgfqpoint{9.300000in}{7.700000in}}%
\pgfusepath{clip}%
\pgfsetrectcap%
\pgfsetroundjoin%
\pgfsetlinewidth{1.505625pt}%
\definecolor{currentstroke}{rgb}{1.000000,0.623529,0.607843}%
\pgfsetstrokecolor{currentstroke}%
\pgfsetstrokeopacity{0.200000}%
\pgfsetdash{}{0pt}%
\pgfpathmoveto{\pgfqpoint{5.627987in}{1.357817in}}%
\pgfpathlineto{\pgfqpoint{5.342251in}{3.998981in}}%
\pgfusepath{stroke}%
\end{pgfscope}%
\begin{pgfscope}%
\pgfpathrectangle{\pgfqpoint{0.481978in}{0.331635in}}{\pgfqpoint{9.300000in}{7.700000in}}%
\pgfusepath{clip}%
\pgfsetrectcap%
\pgfsetroundjoin%
\pgfsetlinewidth{1.505625pt}%
\definecolor{currentstroke}{rgb}{1.000000,0.623529,0.607843}%
\pgfsetstrokecolor{currentstroke}%
\pgfsetstrokeopacity{0.200000}%
\pgfsetdash{}{0pt}%
\pgfpathmoveto{\pgfqpoint{7.678869in}{4.514349in}}%
\pgfpathlineto{\pgfqpoint{5.342251in}{3.998981in}}%
\pgfusepath{stroke}%
\end{pgfscope}%
\begin{pgfscope}%
\pgfpathrectangle{\pgfqpoint{0.481978in}{0.331635in}}{\pgfqpoint{9.300000in}{7.700000in}}%
\pgfusepath{clip}%
\pgfsetrectcap%
\pgfsetroundjoin%
\pgfsetlinewidth{1.505625pt}%
\definecolor{currentstroke}{rgb}{1.000000,0.623529,0.607843}%
\pgfsetstrokecolor{currentstroke}%
\pgfsetstrokeopacity{0.200000}%
\pgfsetdash{}{0pt}%
\pgfpathmoveto{\pgfqpoint{3.663088in}{2.022372in}}%
\pgfpathlineto{\pgfqpoint{5.342251in}{3.998981in}}%
\pgfusepath{stroke}%
\end{pgfscope}%
\begin{pgfscope}%
\pgfpathrectangle{\pgfqpoint{0.481978in}{0.331635in}}{\pgfqpoint{9.300000in}{7.700000in}}%
\pgfusepath{clip}%
\pgfsetrectcap%
\pgfsetroundjoin%
\pgfsetlinewidth{1.505625pt}%
\definecolor{currentstroke}{rgb}{1.000000,0.623529,0.607843}%
\pgfsetstrokecolor{currentstroke}%
\pgfsetstrokeopacity{0.200000}%
\pgfsetdash{}{0pt}%
\pgfpathmoveto{\pgfqpoint{2.974676in}{3.711091in}}%
\pgfpathlineto{\pgfqpoint{5.342251in}{3.998981in}}%
\pgfusepath{stroke}%
\end{pgfscope}%
\begin{pgfscope}%
\pgfpathrectangle{\pgfqpoint{0.481978in}{0.331635in}}{\pgfqpoint{9.300000in}{7.700000in}}%
\pgfusepath{clip}%
\pgfsetrectcap%
\pgfsetroundjoin%
\pgfsetlinewidth{1.505625pt}%
\definecolor{currentstroke}{rgb}{1.000000,0.623529,0.607843}%
\pgfsetstrokecolor{currentstroke}%
\pgfsetstrokeopacity{0.200000}%
\pgfsetdash{}{0pt}%
\pgfpathmoveto{\pgfqpoint{4.287456in}{3.591981in}}%
\pgfpathlineto{\pgfqpoint{5.342251in}{3.998981in}}%
\pgfusepath{stroke}%
\end{pgfscope}%
\begin{pgfscope}%
\pgfpathrectangle{\pgfqpoint{0.481978in}{0.331635in}}{\pgfqpoint{9.300000in}{7.700000in}}%
\pgfusepath{clip}%
\pgfsetrectcap%
\pgfsetroundjoin%
\pgfsetlinewidth{1.505625pt}%
\definecolor{currentstroke}{rgb}{1.000000,0.623529,0.607843}%
\pgfsetstrokecolor{currentstroke}%
\pgfsetstrokeopacity{0.200000}%
\pgfsetdash{}{0pt}%
\pgfpathmoveto{\pgfqpoint{1.967877in}{5.589834in}}%
\pgfpathlineto{\pgfqpoint{5.342251in}{3.998981in}}%
\pgfusepath{stroke}%
\end{pgfscope}%
\begin{pgfscope}%
\pgfpathrectangle{\pgfqpoint{0.481978in}{0.331635in}}{\pgfqpoint{9.300000in}{7.700000in}}%
\pgfusepath{clip}%
\pgfsetrectcap%
\pgfsetroundjoin%
\pgfsetlinewidth{1.505625pt}%
\definecolor{currentstroke}{rgb}{1.000000,0.623529,0.607843}%
\pgfsetstrokecolor{currentstroke}%
\pgfsetstrokeopacity{0.200000}%
\pgfsetdash{}{0pt}%
\pgfpathmoveto{\pgfqpoint{3.570734in}{2.934416in}}%
\pgfpathlineto{\pgfqpoint{5.342251in}{3.998981in}}%
\pgfusepath{stroke}%
\end{pgfscope}%
\begin{pgfscope}%
\pgfpathrectangle{\pgfqpoint{0.481978in}{0.331635in}}{\pgfqpoint{9.300000in}{7.700000in}}%
\pgfusepath{clip}%
\pgfsetrectcap%
\pgfsetroundjoin%
\pgfsetlinewidth{1.505625pt}%
\definecolor{currentstroke}{rgb}{1.000000,0.623529,0.607843}%
\pgfsetstrokecolor{currentstroke}%
\pgfsetstrokeopacity{0.200000}%
\pgfsetdash{}{0pt}%
\pgfpathmoveto{\pgfqpoint{5.661130in}{4.673478in}}%
\pgfpathlineto{\pgfqpoint{5.342251in}{3.998981in}}%
\pgfusepath{stroke}%
\end{pgfscope}%
\begin{pgfscope}%
\pgfpathrectangle{\pgfqpoint{0.481978in}{0.331635in}}{\pgfqpoint{9.300000in}{7.700000in}}%
\pgfusepath{clip}%
\pgfsetrectcap%
\pgfsetroundjoin%
\pgfsetlinewidth{1.505625pt}%
\definecolor{currentstroke}{rgb}{1.000000,0.623529,0.607843}%
\pgfsetstrokecolor{currentstroke}%
\pgfsetstrokeopacity{0.200000}%
\pgfsetdash{}{0pt}%
\pgfpathmoveto{\pgfqpoint{4.720327in}{4.363239in}}%
\pgfpathlineto{\pgfqpoint{5.342251in}{3.998981in}}%
\pgfusepath{stroke}%
\end{pgfscope}%
\begin{pgfscope}%
\pgfpathrectangle{\pgfqpoint{0.481978in}{0.331635in}}{\pgfqpoint{9.300000in}{7.700000in}}%
\pgfusepath{clip}%
\pgfsetrectcap%
\pgfsetroundjoin%
\pgfsetlinewidth{1.505625pt}%
\definecolor{currentstroke}{rgb}{1.000000,0.623529,0.607843}%
\pgfsetstrokecolor{currentstroke}%
\pgfsetstrokeopacity{0.200000}%
\pgfsetdash{}{0pt}%
\pgfpathmoveto{\pgfqpoint{2.339602in}{4.130396in}}%
\pgfpathlineto{\pgfqpoint{5.342251in}{3.998981in}}%
\pgfusepath{stroke}%
\end{pgfscope}%
\begin{pgfscope}%
\pgfpathrectangle{\pgfqpoint{0.481978in}{0.331635in}}{\pgfqpoint{9.300000in}{7.700000in}}%
\pgfusepath{clip}%
\pgfsetrectcap%
\pgfsetroundjoin%
\pgfsetlinewidth{1.505625pt}%
\definecolor{currentstroke}{rgb}{1.000000,0.623529,0.607843}%
\pgfsetstrokecolor{currentstroke}%
\pgfsetstrokeopacity{0.200000}%
\pgfsetdash{}{0pt}%
\pgfpathmoveto{\pgfqpoint{7.776459in}{5.380558in}}%
\pgfpathlineto{\pgfqpoint{5.342251in}{3.998981in}}%
\pgfusepath{stroke}%
\end{pgfscope}%
\begin{pgfscope}%
\pgfpathrectangle{\pgfqpoint{0.481978in}{0.331635in}}{\pgfqpoint{9.300000in}{7.700000in}}%
\pgfusepath{clip}%
\pgfsetrectcap%
\pgfsetroundjoin%
\pgfsetlinewidth{1.505625pt}%
\definecolor{currentstroke}{rgb}{1.000000,0.623529,0.607843}%
\pgfsetstrokecolor{currentstroke}%
\pgfsetstrokeopacity{0.200000}%
\pgfsetdash{}{0pt}%
\pgfpathmoveto{\pgfqpoint{5.832034in}{1.507144in}}%
\pgfpathlineto{\pgfqpoint{5.342251in}{3.998981in}}%
\pgfusepath{stroke}%
\end{pgfscope}%
\begin{pgfscope}%
\pgfpathrectangle{\pgfqpoint{0.481978in}{0.331635in}}{\pgfqpoint{9.300000in}{7.700000in}}%
\pgfusepath{clip}%
\pgfsetrectcap%
\pgfsetroundjoin%
\pgfsetlinewidth{1.505625pt}%
\definecolor{currentstroke}{rgb}{1.000000,0.623529,0.607843}%
\pgfsetstrokecolor{currentstroke}%
\pgfsetstrokeopacity{0.200000}%
\pgfsetdash{}{0pt}%
\pgfpathmoveto{\pgfqpoint{8.011004in}{6.276407in}}%
\pgfpathlineto{\pgfqpoint{5.342251in}{3.998981in}}%
\pgfusepath{stroke}%
\end{pgfscope}%
\begin{pgfscope}%
\pgfpathrectangle{\pgfqpoint{0.481978in}{0.331635in}}{\pgfqpoint{9.300000in}{7.700000in}}%
\pgfusepath{clip}%
\pgfsetrectcap%
\pgfsetroundjoin%
\pgfsetlinewidth{1.505625pt}%
\definecolor{currentstroke}{rgb}{1.000000,0.623529,0.607843}%
\pgfsetstrokecolor{currentstroke}%
\pgfsetstrokeopacity{0.200000}%
\pgfsetdash{}{0pt}%
\pgfpathmoveto{\pgfqpoint{2.908514in}{3.329410in}}%
\pgfpathlineto{\pgfqpoint{5.342251in}{3.998981in}}%
\pgfusepath{stroke}%
\end{pgfscope}%
\begin{pgfscope}%
\pgfpathrectangle{\pgfqpoint{0.481978in}{0.331635in}}{\pgfqpoint{9.300000in}{7.700000in}}%
\pgfusepath{clip}%
\pgfsetrectcap%
\pgfsetroundjoin%
\pgfsetlinewidth{1.505625pt}%
\definecolor{currentstroke}{rgb}{1.000000,0.623529,0.607843}%
\pgfsetstrokecolor{currentstroke}%
\pgfsetstrokeopacity{0.200000}%
\pgfsetdash{}{0pt}%
\pgfpathmoveto{\pgfqpoint{6.465338in}{2.366611in}}%
\pgfpathlineto{\pgfqpoint{5.342251in}{3.998981in}}%
\pgfusepath{stroke}%
\end{pgfscope}%
\begin{pgfscope}%
\pgfpathrectangle{\pgfqpoint{0.481978in}{0.331635in}}{\pgfqpoint{9.300000in}{7.700000in}}%
\pgfusepath{clip}%
\pgfsetrectcap%
\pgfsetroundjoin%
\pgfsetlinewidth{1.505625pt}%
\definecolor{currentstroke}{rgb}{1.000000,0.623529,0.607843}%
\pgfsetstrokecolor{currentstroke}%
\pgfsetstrokeopacity{0.200000}%
\pgfsetdash{}{0pt}%
\pgfpathmoveto{\pgfqpoint{8.106589in}{6.343346in}}%
\pgfpathlineto{\pgfqpoint{5.342251in}{3.998981in}}%
\pgfusepath{stroke}%
\end{pgfscope}%
\begin{pgfscope}%
\pgfpathrectangle{\pgfqpoint{0.481978in}{0.331635in}}{\pgfqpoint{9.300000in}{7.700000in}}%
\pgfusepath{clip}%
\pgfsetrectcap%
\pgfsetroundjoin%
\pgfsetlinewidth{1.505625pt}%
\definecolor{currentstroke}{rgb}{1.000000,0.623529,0.607843}%
\pgfsetstrokecolor{currentstroke}%
\pgfsetstrokeopacity{0.200000}%
\pgfsetdash{}{0pt}%
\pgfpathmoveto{\pgfqpoint{7.991875in}{5.264052in}}%
\pgfpathlineto{\pgfqpoint{5.342251in}{3.998981in}}%
\pgfusepath{stroke}%
\end{pgfscope}%
\begin{pgfscope}%
\pgfpathrectangle{\pgfqpoint{0.481978in}{0.331635in}}{\pgfqpoint{9.300000in}{7.700000in}}%
\pgfusepath{clip}%
\pgfsetrectcap%
\pgfsetroundjoin%
\pgfsetlinewidth{1.505625pt}%
\definecolor{currentstroke}{rgb}{1.000000,0.623529,0.607843}%
\pgfsetstrokecolor{currentstroke}%
\pgfsetstrokeopacity{0.200000}%
\pgfsetdash{}{0pt}%
\pgfpathmoveto{\pgfqpoint{7.423833in}{5.573837in}}%
\pgfpathlineto{\pgfqpoint{5.342251in}{3.998981in}}%
\pgfusepath{stroke}%
\end{pgfscope}%
\begin{pgfscope}%
\pgfpathrectangle{\pgfqpoint{0.481978in}{0.331635in}}{\pgfqpoint{9.300000in}{7.700000in}}%
\pgfusepath{clip}%
\pgfsetrectcap%
\pgfsetroundjoin%
\pgfsetlinewidth{1.505625pt}%
\definecolor{currentstroke}{rgb}{1.000000,0.623529,0.607843}%
\pgfsetstrokecolor{currentstroke}%
\pgfsetstrokeopacity{0.200000}%
\pgfsetdash{}{0pt}%
\pgfpathmoveto{\pgfqpoint{4.192970in}{5.950908in}}%
\pgfpathlineto{\pgfqpoint{5.342251in}{3.998981in}}%
\pgfusepath{stroke}%
\end{pgfscope}%
\begin{pgfscope}%
\pgfpathrectangle{\pgfqpoint{0.481978in}{0.331635in}}{\pgfqpoint{9.300000in}{7.700000in}}%
\pgfusepath{clip}%
\pgfsetrectcap%
\pgfsetroundjoin%
\pgfsetlinewidth{1.505625pt}%
\definecolor{currentstroke}{rgb}{1.000000,0.623529,0.607843}%
\pgfsetstrokecolor{currentstroke}%
\pgfsetstrokeopacity{0.200000}%
\pgfsetdash{}{0pt}%
\pgfpathmoveto{\pgfqpoint{6.770367in}{5.489298in}}%
\pgfpathlineto{\pgfqpoint{5.342251in}{3.998981in}}%
\pgfusepath{stroke}%
\end{pgfscope}%
\begin{pgfscope}%
\pgfpathrectangle{\pgfqpoint{0.481978in}{0.331635in}}{\pgfqpoint{9.300000in}{7.700000in}}%
\pgfusepath{clip}%
\pgfsetrectcap%
\pgfsetroundjoin%
\pgfsetlinewidth{1.505625pt}%
\definecolor{currentstroke}{rgb}{1.000000,0.623529,0.607843}%
\pgfsetstrokecolor{currentstroke}%
\pgfsetstrokeopacity{0.200000}%
\pgfsetdash{}{0pt}%
\pgfpathmoveto{\pgfqpoint{7.446866in}{5.698685in}}%
\pgfpathlineto{\pgfqpoint{5.342251in}{3.998981in}}%
\pgfusepath{stroke}%
\end{pgfscope}%
\begin{pgfscope}%
\pgfpathrectangle{\pgfqpoint{0.481978in}{0.331635in}}{\pgfqpoint{9.300000in}{7.700000in}}%
\pgfusepath{clip}%
\pgfsetrectcap%
\pgfsetroundjoin%
\pgfsetlinewidth{1.505625pt}%
\definecolor{currentstroke}{rgb}{1.000000,0.623529,0.607843}%
\pgfsetstrokecolor{currentstroke}%
\pgfsetstrokeopacity{0.200000}%
\pgfsetdash{}{0pt}%
\pgfpathmoveto{\pgfqpoint{4.422769in}{4.172874in}}%
\pgfpathlineto{\pgfqpoint{5.342251in}{3.998981in}}%
\pgfusepath{stroke}%
\end{pgfscope}%
\begin{pgfscope}%
\pgfpathrectangle{\pgfqpoint{0.481978in}{0.331635in}}{\pgfqpoint{9.300000in}{7.700000in}}%
\pgfusepath{clip}%
\pgfsetrectcap%
\pgfsetroundjoin%
\pgfsetlinewidth{1.505625pt}%
\definecolor{currentstroke}{rgb}{1.000000,0.623529,0.607843}%
\pgfsetstrokecolor{currentstroke}%
\pgfsetstrokeopacity{0.200000}%
\pgfsetdash{}{0pt}%
\pgfpathmoveto{\pgfqpoint{4.632997in}{1.495729in}}%
\pgfpathlineto{\pgfqpoint{5.342251in}{3.998981in}}%
\pgfusepath{stroke}%
\end{pgfscope}%
\begin{pgfscope}%
\pgfpathrectangle{\pgfqpoint{0.481978in}{0.331635in}}{\pgfqpoint{9.300000in}{7.700000in}}%
\pgfusepath{clip}%
\pgfsetrectcap%
\pgfsetroundjoin%
\pgfsetlinewidth{1.505625pt}%
\definecolor{currentstroke}{rgb}{1.000000,0.623529,0.607843}%
\pgfsetstrokecolor{currentstroke}%
\pgfsetstrokeopacity{0.200000}%
\pgfsetdash{}{0pt}%
\pgfpathmoveto{\pgfqpoint{3.319473in}{2.345162in}}%
\pgfpathlineto{\pgfqpoint{5.342251in}{3.998981in}}%
\pgfusepath{stroke}%
\end{pgfscope}%
\begin{pgfscope}%
\pgfpathrectangle{\pgfqpoint{0.481978in}{0.331635in}}{\pgfqpoint{9.300000in}{7.700000in}}%
\pgfusepath{clip}%
\pgfsetrectcap%
\pgfsetroundjoin%
\pgfsetlinewidth{1.505625pt}%
\definecolor{currentstroke}{rgb}{1.000000,0.623529,0.607843}%
\pgfsetstrokecolor{currentstroke}%
\pgfsetstrokeopacity{0.200000}%
\pgfsetdash{}{0pt}%
\pgfpathmoveto{\pgfqpoint{3.661029in}{2.526550in}}%
\pgfpathlineto{\pgfqpoint{5.342251in}{3.998981in}}%
\pgfusepath{stroke}%
\end{pgfscope}%
\begin{pgfscope}%
\pgfpathrectangle{\pgfqpoint{0.481978in}{0.331635in}}{\pgfqpoint{9.300000in}{7.700000in}}%
\pgfusepath{clip}%
\pgfsetrectcap%
\pgfsetroundjoin%
\pgfsetlinewidth{1.505625pt}%
\definecolor{currentstroke}{rgb}{1.000000,0.623529,0.607843}%
\pgfsetstrokecolor{currentstroke}%
\pgfsetstrokeopacity{0.200000}%
\pgfsetdash{}{0pt}%
\pgfpathmoveto{\pgfqpoint{3.845381in}{3.244228in}}%
\pgfpathlineto{\pgfqpoint{5.342251in}{3.998981in}}%
\pgfusepath{stroke}%
\end{pgfscope}%
\begin{pgfscope}%
\pgfpathrectangle{\pgfqpoint{0.481978in}{0.331635in}}{\pgfqpoint{9.300000in}{7.700000in}}%
\pgfusepath{clip}%
\pgfsetrectcap%
\pgfsetroundjoin%
\pgfsetlinewidth{1.505625pt}%
\definecolor{currentstroke}{rgb}{1.000000,0.623529,0.607843}%
\pgfsetstrokecolor{currentstroke}%
\pgfsetstrokeopacity{0.200000}%
\pgfsetdash{}{0pt}%
\pgfpathmoveto{\pgfqpoint{4.050059in}{3.536130in}}%
\pgfpathlineto{\pgfqpoint{5.342251in}{3.998981in}}%
\pgfusepath{stroke}%
\end{pgfscope}%
\begin{pgfscope}%
\pgfpathrectangle{\pgfqpoint{0.481978in}{0.331635in}}{\pgfqpoint{9.300000in}{7.700000in}}%
\pgfusepath{clip}%
\pgfsetrectcap%
\pgfsetroundjoin%
\pgfsetlinewidth{1.505625pt}%
\definecolor{currentstroke}{rgb}{1.000000,0.623529,0.607843}%
\pgfsetstrokecolor{currentstroke}%
\pgfsetstrokeopacity{0.200000}%
\pgfsetdash{}{0pt}%
\pgfpathmoveto{\pgfqpoint{4.925237in}{7.465473in}}%
\pgfpathlineto{\pgfqpoint{5.342251in}{3.998981in}}%
\pgfusepath{stroke}%
\end{pgfscope}%
\begin{pgfscope}%
\pgfpathrectangle{\pgfqpoint{0.481978in}{0.331635in}}{\pgfqpoint{9.300000in}{7.700000in}}%
\pgfusepath{clip}%
\pgfsetrectcap%
\pgfsetroundjoin%
\pgfsetlinewidth{1.505625pt}%
\definecolor{currentstroke}{rgb}{1.000000,0.623529,0.607843}%
\pgfsetstrokecolor{currentstroke}%
\pgfsetstrokeopacity{0.200000}%
\pgfsetdash{}{0pt}%
\pgfpathmoveto{\pgfqpoint{6.991308in}{3.587991in}}%
\pgfpathlineto{\pgfqpoint{5.342251in}{3.998981in}}%
\pgfusepath{stroke}%
\end{pgfscope}%
\begin{pgfscope}%
\pgfpathrectangle{\pgfqpoint{0.481978in}{0.331635in}}{\pgfqpoint{9.300000in}{7.700000in}}%
\pgfusepath{clip}%
\pgfsetrectcap%
\pgfsetroundjoin%
\pgfsetlinewidth{1.505625pt}%
\definecolor{currentstroke}{rgb}{1.000000,0.623529,0.607843}%
\pgfsetstrokecolor{currentstroke}%
\pgfsetstrokeopacity{0.200000}%
\pgfsetdash{}{0pt}%
\pgfpathmoveto{\pgfqpoint{8.618470in}{6.062569in}}%
\pgfpathlineto{\pgfqpoint{5.342251in}{3.998981in}}%
\pgfusepath{stroke}%
\end{pgfscope}%
\begin{pgfscope}%
\pgfpathrectangle{\pgfqpoint{0.481978in}{0.331635in}}{\pgfqpoint{9.300000in}{7.700000in}}%
\pgfusepath{clip}%
\pgfsetrectcap%
\pgfsetroundjoin%
\pgfsetlinewidth{1.505625pt}%
\definecolor{currentstroke}{rgb}{1.000000,0.623529,0.607843}%
\pgfsetstrokecolor{currentstroke}%
\pgfsetstrokeopacity{0.200000}%
\pgfsetdash{}{0pt}%
\pgfpathmoveto{\pgfqpoint{7.337079in}{5.258872in}}%
\pgfpathlineto{\pgfqpoint{5.342251in}{3.998981in}}%
\pgfusepath{stroke}%
\end{pgfscope}%
\begin{pgfscope}%
\pgfpathrectangle{\pgfqpoint{0.481978in}{0.331635in}}{\pgfqpoint{9.300000in}{7.700000in}}%
\pgfusepath{clip}%
\pgfsetrectcap%
\pgfsetroundjoin%
\pgfsetlinewidth{1.505625pt}%
\definecolor{currentstroke}{rgb}{1.000000,0.623529,0.607843}%
\pgfsetstrokecolor{currentstroke}%
\pgfsetstrokeopacity{0.200000}%
\pgfsetdash{}{0pt}%
\pgfpathmoveto{\pgfqpoint{4.553085in}{3.023307in}}%
\pgfpathlineto{\pgfqpoint{5.342251in}{3.998981in}}%
\pgfusepath{stroke}%
\end{pgfscope}%
\begin{pgfscope}%
\pgfpathrectangle{\pgfqpoint{0.481978in}{0.331635in}}{\pgfqpoint{9.300000in}{7.700000in}}%
\pgfusepath{clip}%
\pgfsetrectcap%
\pgfsetroundjoin%
\pgfsetlinewidth{1.505625pt}%
\definecolor{currentstroke}{rgb}{1.000000,0.623529,0.607843}%
\pgfsetstrokecolor{currentstroke}%
\pgfsetstrokeopacity{0.200000}%
\pgfsetdash{}{0pt}%
\pgfpathmoveto{\pgfqpoint{9.095716in}{5.890233in}}%
\pgfpathlineto{\pgfqpoint{5.342251in}{3.998981in}}%
\pgfusepath{stroke}%
\end{pgfscope}%
\begin{pgfscope}%
\pgfpathrectangle{\pgfqpoint{0.481978in}{0.331635in}}{\pgfqpoint{9.300000in}{7.700000in}}%
\pgfusepath{clip}%
\pgfsetrectcap%
\pgfsetroundjoin%
\pgfsetlinewidth{1.505625pt}%
\definecolor{currentstroke}{rgb}{1.000000,0.623529,0.607843}%
\pgfsetstrokecolor{currentstroke}%
\pgfsetstrokeopacity{0.200000}%
\pgfsetdash{}{0pt}%
\pgfpathmoveto{\pgfqpoint{2.403697in}{3.472029in}}%
\pgfpathlineto{\pgfqpoint{5.342251in}{3.998981in}}%
\pgfusepath{stroke}%
\end{pgfscope}%
\begin{pgfscope}%
\pgfpathrectangle{\pgfqpoint{0.481978in}{0.331635in}}{\pgfqpoint{9.300000in}{7.700000in}}%
\pgfusepath{clip}%
\pgfsetrectcap%
\pgfsetroundjoin%
\pgfsetlinewidth{1.505625pt}%
\definecolor{currentstroke}{rgb}{1.000000,0.623529,0.607843}%
\pgfsetstrokecolor{currentstroke}%
\pgfsetstrokeopacity{0.200000}%
\pgfsetdash{}{0pt}%
\pgfpathmoveto{\pgfqpoint{7.815154in}{4.452079in}}%
\pgfpathlineto{\pgfqpoint{5.342251in}{3.998981in}}%
\pgfusepath{stroke}%
\end{pgfscope}%
\begin{pgfscope}%
\pgfpathrectangle{\pgfqpoint{0.481978in}{0.331635in}}{\pgfqpoint{9.300000in}{7.700000in}}%
\pgfusepath{clip}%
\pgfsetrectcap%
\pgfsetroundjoin%
\pgfsetlinewidth{1.505625pt}%
\definecolor{currentstroke}{rgb}{1.000000,0.623529,0.607843}%
\pgfsetstrokecolor{currentstroke}%
\pgfsetstrokeopacity{0.200000}%
\pgfsetdash{}{0pt}%
\pgfpathmoveto{\pgfqpoint{4.473626in}{2.024232in}}%
\pgfpathlineto{\pgfqpoint{5.342251in}{3.998981in}}%
\pgfusepath{stroke}%
\end{pgfscope}%
\begin{pgfscope}%
\pgfpathrectangle{\pgfqpoint{0.481978in}{0.331635in}}{\pgfqpoint{9.300000in}{7.700000in}}%
\pgfusepath{clip}%
\pgfsetrectcap%
\pgfsetroundjoin%
\pgfsetlinewidth{1.505625pt}%
\definecolor{currentstroke}{rgb}{0.815686,0.733333,1.000000}%
\pgfsetstrokecolor{currentstroke}%
\pgfsetstrokeopacity{0.800000}%
\pgfsetdash{}{0pt}%
\pgfpathmoveto{\pgfqpoint{7.152848in}{5.392946in}}%
\pgfpathlineto{\pgfqpoint{5.293604in}{4.165376in}}%
\pgfusepath{stroke}%
\end{pgfscope}%
\begin{pgfscope}%
\pgfpathrectangle{\pgfqpoint{0.481978in}{0.331635in}}{\pgfqpoint{9.300000in}{7.700000in}}%
\pgfusepath{clip}%
\pgfsetrectcap%
\pgfsetroundjoin%
\pgfsetlinewidth{1.505625pt}%
\definecolor{currentstroke}{rgb}{0.815686,0.733333,1.000000}%
\pgfsetstrokecolor{currentstroke}%
\pgfsetstrokeopacity{0.800000}%
\pgfsetdash{}{0pt}%
\pgfpathmoveto{\pgfqpoint{7.702166in}{5.107203in}}%
\pgfpathlineto{\pgfqpoint{5.293604in}{4.165376in}}%
\pgfusepath{stroke}%
\end{pgfscope}%
\begin{pgfscope}%
\pgfpathrectangle{\pgfqpoint{0.481978in}{0.331635in}}{\pgfqpoint{9.300000in}{7.700000in}}%
\pgfusepath{clip}%
\pgfsetrectcap%
\pgfsetroundjoin%
\pgfsetlinewidth{1.505625pt}%
\definecolor{currentstroke}{rgb}{0.815686,0.733333,1.000000}%
\pgfsetstrokecolor{currentstroke}%
\pgfsetstrokeopacity{0.800000}%
\pgfsetdash{}{0pt}%
\pgfpathmoveto{\pgfqpoint{2.872669in}{3.217019in}}%
\pgfpathlineto{\pgfqpoint{5.293604in}{4.165376in}}%
\pgfusepath{stroke}%
\end{pgfscope}%
\begin{pgfscope}%
\pgfpathrectangle{\pgfqpoint{0.481978in}{0.331635in}}{\pgfqpoint{9.300000in}{7.700000in}}%
\pgfusepath{clip}%
\pgfsetrectcap%
\pgfsetroundjoin%
\pgfsetlinewidth{1.505625pt}%
\definecolor{currentstroke}{rgb}{0.815686,0.733333,1.000000}%
\pgfsetstrokecolor{currentstroke}%
\pgfsetstrokeopacity{0.800000}%
\pgfsetdash{}{0pt}%
\pgfpathmoveto{\pgfqpoint{3.619377in}{2.362886in}}%
\pgfpathlineto{\pgfqpoint{5.293604in}{4.165376in}}%
\pgfusepath{stroke}%
\end{pgfscope}%
\begin{pgfscope}%
\pgfpathrectangle{\pgfqpoint{0.481978in}{0.331635in}}{\pgfqpoint{9.300000in}{7.700000in}}%
\pgfusepath{clip}%
\pgfsetrectcap%
\pgfsetroundjoin%
\pgfsetlinewidth{1.505625pt}%
\definecolor{currentstroke}{rgb}{0.815686,0.733333,1.000000}%
\pgfsetstrokecolor{currentstroke}%
\pgfsetstrokeopacity{0.800000}%
\pgfsetdash{}{0pt}%
\pgfpathmoveto{\pgfqpoint{3.448106in}{2.512230in}}%
\pgfpathlineto{\pgfqpoint{5.293604in}{4.165376in}}%
\pgfusepath{stroke}%
\end{pgfscope}%
\begin{pgfscope}%
\pgfpathrectangle{\pgfqpoint{0.481978in}{0.331635in}}{\pgfqpoint{9.300000in}{7.700000in}}%
\pgfusepath{clip}%
\pgfsetrectcap%
\pgfsetroundjoin%
\pgfsetlinewidth{1.505625pt}%
\definecolor{currentstroke}{rgb}{0.815686,0.733333,1.000000}%
\pgfsetstrokecolor{currentstroke}%
\pgfsetstrokeopacity{0.800000}%
\pgfsetdash{}{0pt}%
\pgfpathmoveto{\pgfqpoint{6.968468in}{3.346064in}}%
\pgfpathlineto{\pgfqpoint{5.293604in}{4.165376in}}%
\pgfusepath{stroke}%
\end{pgfscope}%
\begin{pgfscope}%
\pgfpathrectangle{\pgfqpoint{0.481978in}{0.331635in}}{\pgfqpoint{9.300000in}{7.700000in}}%
\pgfusepath{clip}%
\pgfsetrectcap%
\pgfsetroundjoin%
\pgfsetlinewidth{1.505625pt}%
\definecolor{currentstroke}{rgb}{0.815686,0.733333,1.000000}%
\pgfsetstrokecolor{currentstroke}%
\pgfsetstrokeopacity{0.800000}%
\pgfsetdash{}{0pt}%
\pgfpathmoveto{\pgfqpoint{3.260742in}{7.031269in}}%
\pgfpathlineto{\pgfqpoint{5.293604in}{4.165376in}}%
\pgfusepath{stroke}%
\end{pgfscope}%
\begin{pgfscope}%
\pgfpathrectangle{\pgfqpoint{0.481978in}{0.331635in}}{\pgfqpoint{9.300000in}{7.700000in}}%
\pgfusepath{clip}%
\pgfsetrectcap%
\pgfsetroundjoin%
\pgfsetlinewidth{1.505625pt}%
\definecolor{currentstroke}{rgb}{0.815686,0.733333,1.000000}%
\pgfsetstrokecolor{currentstroke}%
\pgfsetstrokeopacity{0.800000}%
\pgfsetdash{}{0pt}%
\pgfpathmoveto{\pgfqpoint{3.172404in}{5.775014in}}%
\pgfpathlineto{\pgfqpoint{5.293604in}{4.165376in}}%
\pgfusepath{stroke}%
\end{pgfscope}%
\begin{pgfscope}%
\pgfpathrectangle{\pgfqpoint{0.481978in}{0.331635in}}{\pgfqpoint{9.300000in}{7.700000in}}%
\pgfusepath{clip}%
\pgfsetrectcap%
\pgfsetroundjoin%
\pgfsetlinewidth{1.505625pt}%
\definecolor{currentstroke}{rgb}{0.815686,0.733333,1.000000}%
\pgfsetstrokecolor{currentstroke}%
\pgfsetstrokeopacity{0.800000}%
\pgfsetdash{}{0pt}%
\pgfpathmoveto{\pgfqpoint{2.953247in}{2.700897in}}%
\pgfpathlineto{\pgfqpoint{5.293604in}{4.165376in}}%
\pgfusepath{stroke}%
\end{pgfscope}%
\begin{pgfscope}%
\pgfpathrectangle{\pgfqpoint{0.481978in}{0.331635in}}{\pgfqpoint{9.300000in}{7.700000in}}%
\pgfusepath{clip}%
\pgfsetrectcap%
\pgfsetroundjoin%
\pgfsetlinewidth{1.505625pt}%
\definecolor{currentstroke}{rgb}{0.815686,0.733333,1.000000}%
\pgfsetstrokecolor{currentstroke}%
\pgfsetstrokeopacity{0.800000}%
\pgfsetdash{}{0pt}%
\pgfpathmoveto{\pgfqpoint{8.116420in}{5.616284in}}%
\pgfpathlineto{\pgfqpoint{5.293604in}{4.165376in}}%
\pgfusepath{stroke}%
\end{pgfscope}%
\begin{pgfscope}%
\pgfpathrectangle{\pgfqpoint{0.481978in}{0.331635in}}{\pgfqpoint{9.300000in}{7.700000in}}%
\pgfusepath{clip}%
\pgfsetrectcap%
\pgfsetroundjoin%
\pgfsetlinewidth{1.505625pt}%
\definecolor{currentstroke}{rgb}{0.815686,0.733333,1.000000}%
\pgfsetstrokecolor{currentstroke}%
\pgfsetstrokeopacity{0.800000}%
\pgfsetdash{}{0pt}%
\pgfpathmoveto{\pgfqpoint{6.906230in}{4.730496in}}%
\pgfpathlineto{\pgfqpoint{5.293604in}{4.165376in}}%
\pgfusepath{stroke}%
\end{pgfscope}%
\begin{pgfscope}%
\pgfpathrectangle{\pgfqpoint{0.481978in}{0.331635in}}{\pgfqpoint{9.300000in}{7.700000in}}%
\pgfusepath{clip}%
\pgfsetrectcap%
\pgfsetroundjoin%
\pgfsetlinewidth{1.505625pt}%
\definecolor{currentstroke}{rgb}{0.815686,0.733333,1.000000}%
\pgfsetstrokecolor{currentstroke}%
\pgfsetstrokeopacity{0.800000}%
\pgfsetdash{}{0pt}%
\pgfpathmoveto{\pgfqpoint{3.815826in}{4.087628in}}%
\pgfpathlineto{\pgfqpoint{5.293604in}{4.165376in}}%
\pgfusepath{stroke}%
\end{pgfscope}%
\begin{pgfscope}%
\pgfpathrectangle{\pgfqpoint{0.481978in}{0.331635in}}{\pgfqpoint{9.300000in}{7.700000in}}%
\pgfusepath{clip}%
\pgfsetrectcap%
\pgfsetroundjoin%
\pgfsetlinewidth{1.505625pt}%
\definecolor{currentstroke}{rgb}{0.815686,0.733333,1.000000}%
\pgfsetstrokecolor{currentstroke}%
\pgfsetstrokeopacity{0.800000}%
\pgfsetdash{}{0pt}%
\pgfpathmoveto{\pgfqpoint{6.962413in}{5.802045in}}%
\pgfpathlineto{\pgfqpoint{5.293604in}{4.165376in}}%
\pgfusepath{stroke}%
\end{pgfscope}%
\begin{pgfscope}%
\pgfpathrectangle{\pgfqpoint{0.481978in}{0.331635in}}{\pgfqpoint{9.300000in}{7.700000in}}%
\pgfusepath{clip}%
\pgfsetrectcap%
\pgfsetroundjoin%
\pgfsetlinewidth{1.505625pt}%
\definecolor{currentstroke}{rgb}{0.815686,0.733333,1.000000}%
\pgfsetstrokecolor{currentstroke}%
\pgfsetstrokeopacity{0.800000}%
\pgfsetdash{}{0pt}%
\pgfpathmoveto{\pgfqpoint{7.236127in}{5.393044in}}%
\pgfpathlineto{\pgfqpoint{5.293604in}{4.165376in}}%
\pgfusepath{stroke}%
\end{pgfscope}%
\begin{pgfscope}%
\pgfpathrectangle{\pgfqpoint{0.481978in}{0.331635in}}{\pgfqpoint{9.300000in}{7.700000in}}%
\pgfusepath{clip}%
\pgfsetrectcap%
\pgfsetroundjoin%
\pgfsetlinewidth{1.505625pt}%
\definecolor{currentstroke}{rgb}{0.815686,0.733333,1.000000}%
\pgfsetstrokecolor{currentstroke}%
\pgfsetstrokeopacity{0.800000}%
\pgfsetdash{}{0pt}%
\pgfpathmoveto{\pgfqpoint{8.116846in}{6.524386in}}%
\pgfpathlineto{\pgfqpoint{5.293604in}{4.165376in}}%
\pgfusepath{stroke}%
\end{pgfscope}%
\begin{pgfscope}%
\pgfpathrectangle{\pgfqpoint{0.481978in}{0.331635in}}{\pgfqpoint{9.300000in}{7.700000in}}%
\pgfusepath{clip}%
\pgfsetrectcap%
\pgfsetroundjoin%
\pgfsetlinewidth{1.505625pt}%
\definecolor{currentstroke}{rgb}{0.815686,0.733333,1.000000}%
\pgfsetstrokecolor{currentstroke}%
\pgfsetstrokeopacity{0.800000}%
\pgfsetdash{}{0pt}%
\pgfpathmoveto{\pgfqpoint{3.449504in}{2.766701in}}%
\pgfpathlineto{\pgfqpoint{5.293604in}{4.165376in}}%
\pgfusepath{stroke}%
\end{pgfscope}%
\begin{pgfscope}%
\pgfpathrectangle{\pgfqpoint{0.481978in}{0.331635in}}{\pgfqpoint{9.300000in}{7.700000in}}%
\pgfusepath{clip}%
\pgfsetrectcap%
\pgfsetroundjoin%
\pgfsetlinewidth{1.505625pt}%
\definecolor{currentstroke}{rgb}{0.815686,0.733333,1.000000}%
\pgfsetstrokecolor{currentstroke}%
\pgfsetstrokeopacity{0.800000}%
\pgfsetdash{}{0pt}%
\pgfpathmoveto{\pgfqpoint{4.947086in}{5.421637in}}%
\pgfpathlineto{\pgfqpoint{5.293604in}{4.165376in}}%
\pgfusepath{stroke}%
\end{pgfscope}%
\begin{pgfscope}%
\pgfpathrectangle{\pgfqpoint{0.481978in}{0.331635in}}{\pgfqpoint{9.300000in}{7.700000in}}%
\pgfusepath{clip}%
\pgfsetrectcap%
\pgfsetroundjoin%
\pgfsetlinewidth{1.505625pt}%
\definecolor{currentstroke}{rgb}{0.815686,0.733333,1.000000}%
\pgfsetstrokecolor{currentstroke}%
\pgfsetstrokeopacity{0.800000}%
\pgfsetdash{}{0pt}%
\pgfpathmoveto{\pgfqpoint{6.925838in}{2.915473in}}%
\pgfpathlineto{\pgfqpoint{5.293604in}{4.165376in}}%
\pgfusepath{stroke}%
\end{pgfscope}%
\begin{pgfscope}%
\pgfpathrectangle{\pgfqpoint{0.481978in}{0.331635in}}{\pgfqpoint{9.300000in}{7.700000in}}%
\pgfusepath{clip}%
\pgfsetrectcap%
\pgfsetroundjoin%
\pgfsetlinewidth{1.505625pt}%
\definecolor{currentstroke}{rgb}{0.815686,0.733333,1.000000}%
\pgfsetstrokecolor{currentstroke}%
\pgfsetstrokeopacity{0.800000}%
\pgfsetdash{}{0pt}%
\pgfpathmoveto{\pgfqpoint{7.338785in}{3.335035in}}%
\pgfpathlineto{\pgfqpoint{5.293604in}{4.165376in}}%
\pgfusepath{stroke}%
\end{pgfscope}%
\begin{pgfscope}%
\pgfpathrectangle{\pgfqpoint{0.481978in}{0.331635in}}{\pgfqpoint{9.300000in}{7.700000in}}%
\pgfusepath{clip}%
\pgfsetrectcap%
\pgfsetroundjoin%
\pgfsetlinewidth{1.505625pt}%
\definecolor{currentstroke}{rgb}{0.815686,0.733333,1.000000}%
\pgfsetstrokecolor{currentstroke}%
\pgfsetstrokeopacity{0.800000}%
\pgfsetdash{}{0pt}%
\pgfpathmoveto{\pgfqpoint{3.955717in}{4.134945in}}%
\pgfpathlineto{\pgfqpoint{5.293604in}{4.165376in}}%
\pgfusepath{stroke}%
\end{pgfscope}%
\begin{pgfscope}%
\pgfpathrectangle{\pgfqpoint{0.481978in}{0.331635in}}{\pgfqpoint{9.300000in}{7.700000in}}%
\pgfusepath{clip}%
\pgfsetrectcap%
\pgfsetroundjoin%
\pgfsetlinewidth{1.505625pt}%
\definecolor{currentstroke}{rgb}{0.815686,0.733333,1.000000}%
\pgfsetstrokecolor{currentstroke}%
\pgfsetstrokeopacity{0.800000}%
\pgfsetdash{}{0pt}%
\pgfpathmoveto{\pgfqpoint{2.160067in}{1.924338in}}%
\pgfpathlineto{\pgfqpoint{5.293604in}{4.165376in}}%
\pgfusepath{stroke}%
\end{pgfscope}%
\begin{pgfscope}%
\pgfpathrectangle{\pgfqpoint{0.481978in}{0.331635in}}{\pgfqpoint{9.300000in}{7.700000in}}%
\pgfusepath{clip}%
\pgfsetrectcap%
\pgfsetroundjoin%
\pgfsetlinewidth{1.505625pt}%
\definecolor{currentstroke}{rgb}{0.815686,0.733333,1.000000}%
\pgfsetstrokecolor{currentstroke}%
\pgfsetstrokeopacity{0.800000}%
\pgfsetdash{}{0pt}%
\pgfpathmoveto{\pgfqpoint{4.198073in}{4.262822in}}%
\pgfpathlineto{\pgfqpoint{5.293604in}{4.165376in}}%
\pgfusepath{stroke}%
\end{pgfscope}%
\begin{pgfscope}%
\pgfpathrectangle{\pgfqpoint{0.481978in}{0.331635in}}{\pgfqpoint{9.300000in}{7.700000in}}%
\pgfusepath{clip}%
\pgfsetrectcap%
\pgfsetroundjoin%
\pgfsetlinewidth{1.505625pt}%
\definecolor{currentstroke}{rgb}{0.815686,0.733333,1.000000}%
\pgfsetstrokecolor{currentstroke}%
\pgfsetstrokeopacity{0.800000}%
\pgfsetdash{}{0pt}%
\pgfpathmoveto{\pgfqpoint{3.029876in}{4.931933in}}%
\pgfpathlineto{\pgfqpoint{5.293604in}{4.165376in}}%
\pgfusepath{stroke}%
\end{pgfscope}%
\begin{pgfscope}%
\pgfpathrectangle{\pgfqpoint{0.481978in}{0.331635in}}{\pgfqpoint{9.300000in}{7.700000in}}%
\pgfusepath{clip}%
\pgfsetrectcap%
\pgfsetroundjoin%
\pgfsetlinewidth{1.505625pt}%
\definecolor{currentstroke}{rgb}{0.815686,0.733333,1.000000}%
\pgfsetstrokecolor{currentstroke}%
\pgfsetstrokeopacity{0.800000}%
\pgfsetdash{}{0pt}%
\pgfpathmoveto{\pgfqpoint{1.310658in}{4.258773in}}%
\pgfpathlineto{\pgfqpoint{5.293604in}{4.165376in}}%
\pgfusepath{stroke}%
\end{pgfscope}%
\begin{pgfscope}%
\pgfpathrectangle{\pgfqpoint{0.481978in}{0.331635in}}{\pgfqpoint{9.300000in}{7.700000in}}%
\pgfusepath{clip}%
\pgfsetrectcap%
\pgfsetroundjoin%
\pgfsetlinewidth{1.505625pt}%
\definecolor{currentstroke}{rgb}{0.815686,0.733333,1.000000}%
\pgfsetstrokecolor{currentstroke}%
\pgfsetstrokeopacity{0.800000}%
\pgfsetdash{}{0pt}%
\pgfpathmoveto{\pgfqpoint{5.061821in}{4.710122in}}%
\pgfpathlineto{\pgfqpoint{5.293604in}{4.165376in}}%
\pgfusepath{stroke}%
\end{pgfscope}%
\begin{pgfscope}%
\pgfpathrectangle{\pgfqpoint{0.481978in}{0.331635in}}{\pgfqpoint{9.300000in}{7.700000in}}%
\pgfusepath{clip}%
\pgfsetrectcap%
\pgfsetroundjoin%
\pgfsetlinewidth{1.505625pt}%
\definecolor{currentstroke}{rgb}{0.815686,0.733333,1.000000}%
\pgfsetstrokecolor{currentstroke}%
\pgfsetstrokeopacity{0.800000}%
\pgfsetdash{}{0pt}%
\pgfpathmoveto{\pgfqpoint{5.046020in}{5.054040in}}%
\pgfpathlineto{\pgfqpoint{5.293604in}{4.165376in}}%
\pgfusepath{stroke}%
\end{pgfscope}%
\begin{pgfscope}%
\pgfpathrectangle{\pgfqpoint{0.481978in}{0.331635in}}{\pgfqpoint{9.300000in}{7.700000in}}%
\pgfusepath{clip}%
\pgfsetrectcap%
\pgfsetroundjoin%
\pgfsetlinewidth{1.505625pt}%
\definecolor{currentstroke}{rgb}{0.815686,0.733333,1.000000}%
\pgfsetstrokecolor{currentstroke}%
\pgfsetstrokeopacity{0.800000}%
\pgfsetdash{}{0pt}%
\pgfpathmoveto{\pgfqpoint{7.569943in}{3.679389in}}%
\pgfpathlineto{\pgfqpoint{5.293604in}{4.165376in}}%
\pgfusepath{stroke}%
\end{pgfscope}%
\begin{pgfscope}%
\pgfpathrectangle{\pgfqpoint{0.481978in}{0.331635in}}{\pgfqpoint{9.300000in}{7.700000in}}%
\pgfusepath{clip}%
\pgfsetrectcap%
\pgfsetroundjoin%
\pgfsetlinewidth{1.505625pt}%
\definecolor{currentstroke}{rgb}{0.815686,0.733333,1.000000}%
\pgfsetstrokecolor{currentstroke}%
\pgfsetstrokeopacity{0.800000}%
\pgfsetdash{}{0pt}%
\pgfpathmoveto{\pgfqpoint{7.674879in}{3.075539in}}%
\pgfpathlineto{\pgfqpoint{5.293604in}{4.165376in}}%
\pgfusepath{stroke}%
\end{pgfscope}%
\begin{pgfscope}%
\pgfpathrectangle{\pgfqpoint{0.481978in}{0.331635in}}{\pgfqpoint{9.300000in}{7.700000in}}%
\pgfusepath{clip}%
\pgfsetrectcap%
\pgfsetroundjoin%
\pgfsetlinewidth{1.505625pt}%
\definecolor{currentstroke}{rgb}{0.815686,0.733333,1.000000}%
\pgfsetstrokecolor{currentstroke}%
\pgfsetstrokeopacity{0.800000}%
\pgfsetdash{}{0pt}%
\pgfpathmoveto{\pgfqpoint{5.428071in}{3.114634in}}%
\pgfpathlineto{\pgfqpoint{5.293604in}{4.165376in}}%
\pgfusepath{stroke}%
\end{pgfscope}%
\begin{pgfscope}%
\pgfpathrectangle{\pgfqpoint{0.481978in}{0.331635in}}{\pgfqpoint{9.300000in}{7.700000in}}%
\pgfusepath{clip}%
\pgfsetrectcap%
\pgfsetroundjoin%
\pgfsetlinewidth{1.505625pt}%
\definecolor{currentstroke}{rgb}{0.815686,0.733333,1.000000}%
\pgfsetstrokecolor{currentstroke}%
\pgfsetstrokeopacity{0.800000}%
\pgfsetdash{}{0pt}%
\pgfpathmoveto{\pgfqpoint{2.403350in}{3.643277in}}%
\pgfpathlineto{\pgfqpoint{5.293604in}{4.165376in}}%
\pgfusepath{stroke}%
\end{pgfscope}%
\begin{pgfscope}%
\pgfpathrectangle{\pgfqpoint{0.481978in}{0.331635in}}{\pgfqpoint{9.300000in}{7.700000in}}%
\pgfusepath{clip}%
\pgfsetrectcap%
\pgfsetroundjoin%
\pgfsetlinewidth{1.505625pt}%
\definecolor{currentstroke}{rgb}{0.815686,0.733333,1.000000}%
\pgfsetstrokecolor{currentstroke}%
\pgfsetstrokeopacity{0.800000}%
\pgfsetdash{}{0pt}%
\pgfpathmoveto{\pgfqpoint{6.041612in}{5.634142in}}%
\pgfpathlineto{\pgfqpoint{5.293604in}{4.165376in}}%
\pgfusepath{stroke}%
\end{pgfscope}%
\begin{pgfscope}%
\pgfpathrectangle{\pgfqpoint{0.481978in}{0.331635in}}{\pgfqpoint{9.300000in}{7.700000in}}%
\pgfusepath{clip}%
\pgfsetrectcap%
\pgfsetroundjoin%
\pgfsetlinewidth{1.505625pt}%
\definecolor{currentstroke}{rgb}{0.815686,0.733333,1.000000}%
\pgfsetstrokecolor{currentstroke}%
\pgfsetstrokeopacity{0.800000}%
\pgfsetdash{}{0pt}%
\pgfpathmoveto{\pgfqpoint{8.383007in}{6.203756in}}%
\pgfpathlineto{\pgfqpoint{5.293604in}{4.165376in}}%
\pgfusepath{stroke}%
\end{pgfscope}%
\begin{pgfscope}%
\pgfpathrectangle{\pgfqpoint{0.481978in}{0.331635in}}{\pgfqpoint{9.300000in}{7.700000in}}%
\pgfusepath{clip}%
\pgfsetrectcap%
\pgfsetroundjoin%
\pgfsetlinewidth{1.505625pt}%
\definecolor{currentstroke}{rgb}{0.815686,0.733333,1.000000}%
\pgfsetstrokecolor{currentstroke}%
\pgfsetstrokeopacity{0.800000}%
\pgfsetdash{}{0pt}%
\pgfpathmoveto{\pgfqpoint{4.398769in}{4.567491in}}%
\pgfpathlineto{\pgfqpoint{5.293604in}{4.165376in}}%
\pgfusepath{stroke}%
\end{pgfscope}%
\begin{pgfscope}%
\pgfpathrectangle{\pgfqpoint{0.481978in}{0.331635in}}{\pgfqpoint{9.300000in}{7.700000in}}%
\pgfusepath{clip}%
\pgfsetrectcap%
\pgfsetroundjoin%
\pgfsetlinewidth{1.505625pt}%
\definecolor{currentstroke}{rgb}{0.815686,0.733333,1.000000}%
\pgfsetstrokecolor{currentstroke}%
\pgfsetstrokeopacity{0.800000}%
\pgfsetdash{}{0pt}%
\pgfpathmoveto{\pgfqpoint{8.682490in}{5.819369in}}%
\pgfpathlineto{\pgfqpoint{5.293604in}{4.165376in}}%
\pgfusepath{stroke}%
\end{pgfscope}%
\begin{pgfscope}%
\pgfpathrectangle{\pgfqpoint{0.481978in}{0.331635in}}{\pgfqpoint{9.300000in}{7.700000in}}%
\pgfusepath{clip}%
\pgfsetrectcap%
\pgfsetroundjoin%
\pgfsetlinewidth{1.505625pt}%
\definecolor{currentstroke}{rgb}{0.815686,0.733333,1.000000}%
\pgfsetstrokecolor{currentstroke}%
\pgfsetstrokeopacity{0.800000}%
\pgfsetdash{}{0pt}%
\pgfpathmoveto{\pgfqpoint{5.664598in}{1.155031in}}%
\pgfpathlineto{\pgfqpoint{5.293604in}{4.165376in}}%
\pgfusepath{stroke}%
\end{pgfscope}%
\begin{pgfscope}%
\pgfpathrectangle{\pgfqpoint{0.481978in}{0.331635in}}{\pgfqpoint{9.300000in}{7.700000in}}%
\pgfusepath{clip}%
\pgfsetrectcap%
\pgfsetroundjoin%
\pgfsetlinewidth{1.505625pt}%
\definecolor{currentstroke}{rgb}{0.815686,0.733333,1.000000}%
\pgfsetstrokecolor{currentstroke}%
\pgfsetstrokeopacity{0.800000}%
\pgfsetdash{}{0pt}%
\pgfpathmoveto{\pgfqpoint{1.939227in}{4.353523in}}%
\pgfpathlineto{\pgfqpoint{5.293604in}{4.165376in}}%
\pgfusepath{stroke}%
\end{pgfscope}%
\begin{pgfscope}%
\pgfpathrectangle{\pgfqpoint{0.481978in}{0.331635in}}{\pgfqpoint{9.300000in}{7.700000in}}%
\pgfusepath{clip}%
\pgfsetrectcap%
\pgfsetroundjoin%
\pgfsetlinewidth{1.505625pt}%
\definecolor{currentstroke}{rgb}{0.815686,0.733333,1.000000}%
\pgfsetstrokecolor{currentstroke}%
\pgfsetstrokeopacity{0.800000}%
\pgfsetdash{}{0pt}%
\pgfpathmoveto{\pgfqpoint{6.777031in}{4.449477in}}%
\pgfpathlineto{\pgfqpoint{5.293604in}{4.165376in}}%
\pgfusepath{stroke}%
\end{pgfscope}%
\begin{pgfscope}%
\pgfpathrectangle{\pgfqpoint{0.481978in}{0.331635in}}{\pgfqpoint{9.300000in}{7.700000in}}%
\pgfusepath{clip}%
\pgfsetrectcap%
\pgfsetroundjoin%
\pgfsetlinewidth{1.505625pt}%
\definecolor{currentstroke}{rgb}{0.815686,0.733333,1.000000}%
\pgfsetstrokecolor{currentstroke}%
\pgfsetstrokeopacity{0.800000}%
\pgfsetdash{}{0pt}%
\pgfpathmoveto{\pgfqpoint{3.798724in}{2.165621in}}%
\pgfpathlineto{\pgfqpoint{5.293604in}{4.165376in}}%
\pgfusepath{stroke}%
\end{pgfscope}%
\begin{pgfscope}%
\pgfpathrectangle{\pgfqpoint{0.481978in}{0.331635in}}{\pgfqpoint{9.300000in}{7.700000in}}%
\pgfusepath{clip}%
\pgfsetrectcap%
\pgfsetroundjoin%
\pgfsetlinewidth{1.505625pt}%
\definecolor{currentstroke}{rgb}{0.815686,0.733333,1.000000}%
\pgfsetstrokecolor{currentstroke}%
\pgfsetstrokeopacity{0.800000}%
\pgfsetdash{}{0pt}%
\pgfpathmoveto{\pgfqpoint{1.738451in}{3.903724in}}%
\pgfpathlineto{\pgfqpoint{5.293604in}{4.165376in}}%
\pgfusepath{stroke}%
\end{pgfscope}%
\begin{pgfscope}%
\pgfpathrectangle{\pgfqpoint{0.481978in}{0.331635in}}{\pgfqpoint{9.300000in}{7.700000in}}%
\pgfusepath{clip}%
\pgfsetrectcap%
\pgfsetroundjoin%
\pgfsetlinewidth{1.505625pt}%
\definecolor{currentstroke}{rgb}{0.815686,0.733333,1.000000}%
\pgfsetstrokecolor{currentstroke}%
\pgfsetstrokeopacity{0.800000}%
\pgfsetdash{}{0pt}%
\pgfpathmoveto{\pgfqpoint{8.508088in}{4.005881in}}%
\pgfpathlineto{\pgfqpoint{5.293604in}{4.165376in}}%
\pgfusepath{stroke}%
\end{pgfscope}%
\begin{pgfscope}%
\pgfpathrectangle{\pgfqpoint{0.481978in}{0.331635in}}{\pgfqpoint{9.300000in}{7.700000in}}%
\pgfusepath{clip}%
\pgfsetrectcap%
\pgfsetroundjoin%
\pgfsetlinewidth{1.505625pt}%
\definecolor{currentstroke}{rgb}{0.815686,0.733333,1.000000}%
\pgfsetstrokecolor{currentstroke}%
\pgfsetstrokeopacity{0.800000}%
\pgfsetdash{}{0pt}%
\pgfpathmoveto{\pgfqpoint{6.837191in}{4.890982in}}%
\pgfpathlineto{\pgfqpoint{5.293604in}{4.165376in}}%
\pgfusepath{stroke}%
\end{pgfscope}%
\begin{pgfscope}%
\pgfpathrectangle{\pgfqpoint{0.481978in}{0.331635in}}{\pgfqpoint{9.300000in}{7.700000in}}%
\pgfusepath{clip}%
\pgfsetrectcap%
\pgfsetroundjoin%
\pgfsetlinewidth{1.505625pt}%
\definecolor{currentstroke}{rgb}{0.815686,0.733333,1.000000}%
\pgfsetstrokecolor{currentstroke}%
\pgfsetstrokeopacity{0.800000}%
\pgfsetdash{}{0pt}%
\pgfpathmoveto{\pgfqpoint{4.101155in}{3.759624in}}%
\pgfpathlineto{\pgfqpoint{5.293604in}{4.165376in}}%
\pgfusepath{stroke}%
\end{pgfscope}%
\begin{pgfscope}%
\pgfpathrectangle{\pgfqpoint{0.481978in}{0.331635in}}{\pgfqpoint{9.300000in}{7.700000in}}%
\pgfusepath{clip}%
\pgfsetrectcap%
\pgfsetroundjoin%
\pgfsetlinewidth{1.505625pt}%
\definecolor{currentstroke}{rgb}{0.815686,0.733333,1.000000}%
\pgfsetstrokecolor{currentstroke}%
\pgfsetstrokeopacity{0.800000}%
\pgfsetdash{}{0pt}%
\pgfpathmoveto{\pgfqpoint{7.340850in}{3.294690in}}%
\pgfpathlineto{\pgfqpoint{5.293604in}{4.165376in}}%
\pgfusepath{stroke}%
\end{pgfscope}%
\begin{pgfscope}%
\pgfpathrectangle{\pgfqpoint{0.481978in}{0.331635in}}{\pgfqpoint{9.300000in}{7.700000in}}%
\pgfusepath{clip}%
\pgfsetrectcap%
\pgfsetroundjoin%
\pgfsetlinewidth{1.505625pt}%
\definecolor{currentstroke}{rgb}{0.815686,0.733333,1.000000}%
\pgfsetstrokecolor{currentstroke}%
\pgfsetstrokeopacity{0.800000}%
\pgfsetdash{}{0pt}%
\pgfpathmoveto{\pgfqpoint{4.878178in}{5.104000in}}%
\pgfpathlineto{\pgfqpoint{5.293604in}{4.165376in}}%
\pgfusepath{stroke}%
\end{pgfscope}%
\begin{pgfscope}%
\pgfpathrectangle{\pgfqpoint{0.481978in}{0.331635in}}{\pgfqpoint{9.300000in}{7.700000in}}%
\pgfusepath{clip}%
\pgfsetrectcap%
\pgfsetroundjoin%
\pgfsetlinewidth{1.505625pt}%
\definecolor{currentstroke}{rgb}{0.815686,0.733333,1.000000}%
\pgfsetstrokecolor{currentstroke}%
\pgfsetstrokeopacity{0.800000}%
\pgfsetdash{}{0pt}%
\pgfpathmoveto{\pgfqpoint{3.093874in}{4.888839in}}%
\pgfpathlineto{\pgfqpoint{5.293604in}{4.165376in}}%
\pgfusepath{stroke}%
\end{pgfscope}%
\begin{pgfscope}%
\pgfpathrectangle{\pgfqpoint{0.481978in}{0.331635in}}{\pgfqpoint{9.300000in}{7.700000in}}%
\pgfusepath{clip}%
\pgfsetrectcap%
\pgfsetroundjoin%
\pgfsetlinewidth{1.505625pt}%
\definecolor{currentstroke}{rgb}{0.815686,0.733333,1.000000}%
\pgfsetstrokecolor{currentstroke}%
\pgfsetstrokeopacity{0.800000}%
\pgfsetdash{}{0pt}%
\pgfpathmoveto{\pgfqpoint{5.273114in}{1.907651in}}%
\pgfpathlineto{\pgfqpoint{5.293604in}{4.165376in}}%
\pgfusepath{stroke}%
\end{pgfscope}%
\begin{pgfscope}%
\pgfpathrectangle{\pgfqpoint{0.481978in}{0.331635in}}{\pgfqpoint{9.300000in}{7.700000in}}%
\pgfusepath{clip}%
\pgfsetrectcap%
\pgfsetroundjoin%
\pgfsetlinewidth{1.505625pt}%
\definecolor{currentstroke}{rgb}{0.815686,0.733333,1.000000}%
\pgfsetstrokecolor{currentstroke}%
\pgfsetstrokeopacity{0.800000}%
\pgfsetdash{}{0pt}%
\pgfpathmoveto{\pgfqpoint{8.330701in}{5.648832in}}%
\pgfpathlineto{\pgfqpoint{5.293604in}{4.165376in}}%
\pgfusepath{stroke}%
\end{pgfscope}%
\begin{pgfscope}%
\pgfpathrectangle{\pgfqpoint{0.481978in}{0.331635in}}{\pgfqpoint{9.300000in}{7.700000in}}%
\pgfusepath{clip}%
\pgfsetrectcap%
\pgfsetroundjoin%
\pgfsetlinewidth{1.505625pt}%
\definecolor{currentstroke}{rgb}{0.815686,0.733333,1.000000}%
\pgfsetstrokecolor{currentstroke}%
\pgfsetstrokeopacity{0.800000}%
\pgfsetdash{}{0pt}%
\pgfpathmoveto{\pgfqpoint{7.364685in}{2.940268in}}%
\pgfpathlineto{\pgfqpoint{5.293604in}{4.165376in}}%
\pgfusepath{stroke}%
\end{pgfscope}%
\begin{pgfscope}%
\pgfpathrectangle{\pgfqpoint{0.481978in}{0.331635in}}{\pgfqpoint{9.300000in}{7.700000in}}%
\pgfusepath{clip}%
\pgfsetrectcap%
\pgfsetroundjoin%
\pgfsetlinewidth{1.505625pt}%
\definecolor{currentstroke}{rgb}{0.815686,0.733333,1.000000}%
\pgfsetstrokecolor{currentstroke}%
\pgfsetstrokeopacity{0.800000}%
\pgfsetdash{}{0pt}%
\pgfpathmoveto{\pgfqpoint{2.332471in}{2.516497in}}%
\pgfpathlineto{\pgfqpoint{5.293604in}{4.165376in}}%
\pgfusepath{stroke}%
\end{pgfscope}%
\begin{pgfscope}%
\pgfpathrectangle{\pgfqpoint{0.481978in}{0.331635in}}{\pgfqpoint{9.300000in}{7.700000in}}%
\pgfusepath{clip}%
\pgfsetrectcap%
\pgfsetroundjoin%
\pgfsetlinewidth{1.505625pt}%
\definecolor{currentstroke}{rgb}{0.815686,0.733333,1.000000}%
\pgfsetstrokecolor{currentstroke}%
\pgfsetstrokeopacity{0.800000}%
\pgfsetdash{}{0pt}%
\pgfpathmoveto{\pgfqpoint{6.392422in}{4.231341in}}%
\pgfpathlineto{\pgfqpoint{5.293604in}{4.165376in}}%
\pgfusepath{stroke}%
\end{pgfscope}%
\begin{pgfscope}%
\pgfpathrectangle{\pgfqpoint{0.481978in}{0.331635in}}{\pgfqpoint{9.300000in}{7.700000in}}%
\pgfusepath{clip}%
\pgfsetrectcap%
\pgfsetroundjoin%
\pgfsetlinewidth{1.505625pt}%
\definecolor{currentstroke}{rgb}{0.870588,0.733333,0.607843}%
\pgfsetstrokecolor{currentstroke}%
\pgfsetstrokeopacity{0.800000}%
\pgfsetdash{}{0pt}%
\pgfpathmoveto{\pgfqpoint{3.278906in}{4.376840in}}%
\pgfpathlineto{\pgfqpoint{4.473902in}{4.321654in}}%
\pgfusepath{stroke}%
\end{pgfscope}%
\begin{pgfscope}%
\pgfpathrectangle{\pgfqpoint{0.481978in}{0.331635in}}{\pgfqpoint{9.300000in}{7.700000in}}%
\pgfusepath{clip}%
\pgfsetrectcap%
\pgfsetroundjoin%
\pgfsetlinewidth{1.505625pt}%
\definecolor{currentstroke}{rgb}{0.870588,0.733333,0.607843}%
\pgfsetstrokecolor{currentstroke}%
\pgfsetstrokeopacity{0.800000}%
\pgfsetdash{}{0pt}%
\pgfpathmoveto{\pgfqpoint{1.806406in}{3.514068in}}%
\pgfpathlineto{\pgfqpoint{4.473902in}{4.321654in}}%
\pgfusepath{stroke}%
\end{pgfscope}%
\begin{pgfscope}%
\pgfpathrectangle{\pgfqpoint{0.481978in}{0.331635in}}{\pgfqpoint{9.300000in}{7.700000in}}%
\pgfusepath{clip}%
\pgfsetrectcap%
\pgfsetroundjoin%
\pgfsetlinewidth{1.505625pt}%
\definecolor{currentstroke}{rgb}{0.870588,0.733333,0.607843}%
\pgfsetstrokecolor{currentstroke}%
\pgfsetstrokeopacity{0.800000}%
\pgfsetdash{}{0pt}%
\pgfpathmoveto{\pgfqpoint{3.901522in}{2.752340in}}%
\pgfpathlineto{\pgfqpoint{4.473902in}{4.321654in}}%
\pgfusepath{stroke}%
\end{pgfscope}%
\begin{pgfscope}%
\pgfpathrectangle{\pgfqpoint{0.481978in}{0.331635in}}{\pgfqpoint{9.300000in}{7.700000in}}%
\pgfusepath{clip}%
\pgfsetrectcap%
\pgfsetroundjoin%
\pgfsetlinewidth{1.505625pt}%
\definecolor{currentstroke}{rgb}{0.870588,0.733333,0.607843}%
\pgfsetstrokecolor{currentstroke}%
\pgfsetstrokeopacity{0.800000}%
\pgfsetdash{}{0pt}%
\pgfpathmoveto{\pgfqpoint{2.577647in}{1.961490in}}%
\pgfpathlineto{\pgfqpoint{4.473902in}{4.321654in}}%
\pgfusepath{stroke}%
\end{pgfscope}%
\begin{pgfscope}%
\pgfpathrectangle{\pgfqpoint{0.481978in}{0.331635in}}{\pgfqpoint{9.300000in}{7.700000in}}%
\pgfusepath{clip}%
\pgfsetrectcap%
\pgfsetroundjoin%
\pgfsetlinewidth{1.505625pt}%
\definecolor{currentstroke}{rgb}{0.870588,0.733333,0.607843}%
\pgfsetstrokecolor{currentstroke}%
\pgfsetstrokeopacity{0.800000}%
\pgfsetdash{}{0pt}%
\pgfpathmoveto{\pgfqpoint{2.542882in}{4.429625in}}%
\pgfpathlineto{\pgfqpoint{4.473902in}{4.321654in}}%
\pgfusepath{stroke}%
\end{pgfscope}%
\begin{pgfscope}%
\pgfpathrectangle{\pgfqpoint{0.481978in}{0.331635in}}{\pgfqpoint{9.300000in}{7.700000in}}%
\pgfusepath{clip}%
\pgfsetrectcap%
\pgfsetroundjoin%
\pgfsetlinewidth{1.505625pt}%
\definecolor{currentstroke}{rgb}{0.870588,0.733333,0.607843}%
\pgfsetstrokecolor{currentstroke}%
\pgfsetstrokeopacity{0.800000}%
\pgfsetdash{}{0pt}%
\pgfpathmoveto{\pgfqpoint{6.100140in}{5.077934in}}%
\pgfpathlineto{\pgfqpoint{4.473902in}{4.321654in}}%
\pgfusepath{stroke}%
\end{pgfscope}%
\begin{pgfscope}%
\pgfpathrectangle{\pgfqpoint{0.481978in}{0.331635in}}{\pgfqpoint{9.300000in}{7.700000in}}%
\pgfusepath{clip}%
\pgfsetrectcap%
\pgfsetroundjoin%
\pgfsetlinewidth{1.505625pt}%
\definecolor{currentstroke}{rgb}{0.870588,0.733333,0.607843}%
\pgfsetstrokecolor{currentstroke}%
\pgfsetstrokeopacity{0.800000}%
\pgfsetdash{}{0pt}%
\pgfpathmoveto{\pgfqpoint{4.111074in}{4.571818in}}%
\pgfpathlineto{\pgfqpoint{4.473902in}{4.321654in}}%
\pgfusepath{stroke}%
\end{pgfscope}%
\begin{pgfscope}%
\pgfpathrectangle{\pgfqpoint{0.481978in}{0.331635in}}{\pgfqpoint{9.300000in}{7.700000in}}%
\pgfusepath{clip}%
\pgfsetrectcap%
\pgfsetroundjoin%
\pgfsetlinewidth{1.505625pt}%
\definecolor{currentstroke}{rgb}{0.870588,0.733333,0.607843}%
\pgfsetstrokecolor{currentstroke}%
\pgfsetstrokeopacity{0.800000}%
\pgfsetdash{}{0pt}%
\pgfpathmoveto{\pgfqpoint{2.567413in}{4.083544in}}%
\pgfpathlineto{\pgfqpoint{4.473902in}{4.321654in}}%
\pgfusepath{stroke}%
\end{pgfscope}%
\begin{pgfscope}%
\pgfpathrectangle{\pgfqpoint{0.481978in}{0.331635in}}{\pgfqpoint{9.300000in}{7.700000in}}%
\pgfusepath{clip}%
\pgfsetrectcap%
\pgfsetroundjoin%
\pgfsetlinewidth{1.505625pt}%
\definecolor{currentstroke}{rgb}{0.870588,0.733333,0.607843}%
\pgfsetstrokecolor{currentstroke}%
\pgfsetstrokeopacity{0.800000}%
\pgfsetdash{}{0pt}%
\pgfpathmoveto{\pgfqpoint{2.475746in}{4.004366in}}%
\pgfpathlineto{\pgfqpoint{4.473902in}{4.321654in}}%
\pgfusepath{stroke}%
\end{pgfscope}%
\begin{pgfscope}%
\pgfpathrectangle{\pgfqpoint{0.481978in}{0.331635in}}{\pgfqpoint{9.300000in}{7.700000in}}%
\pgfusepath{clip}%
\pgfsetrectcap%
\pgfsetroundjoin%
\pgfsetlinewidth{1.505625pt}%
\definecolor{currentstroke}{rgb}{0.870588,0.733333,0.607843}%
\pgfsetstrokecolor{currentstroke}%
\pgfsetstrokeopacity{0.800000}%
\pgfsetdash{}{0pt}%
\pgfpathmoveto{\pgfqpoint{7.321265in}{6.030797in}}%
\pgfpathlineto{\pgfqpoint{4.473902in}{4.321654in}}%
\pgfusepath{stroke}%
\end{pgfscope}%
\begin{pgfscope}%
\pgfpathrectangle{\pgfqpoint{0.481978in}{0.331635in}}{\pgfqpoint{9.300000in}{7.700000in}}%
\pgfusepath{clip}%
\pgfsetrectcap%
\pgfsetroundjoin%
\pgfsetlinewidth{1.505625pt}%
\definecolor{currentstroke}{rgb}{0.870588,0.733333,0.607843}%
\pgfsetstrokecolor{currentstroke}%
\pgfsetstrokeopacity{0.800000}%
\pgfsetdash{}{0pt}%
\pgfpathmoveto{\pgfqpoint{1.982524in}{5.206262in}}%
\pgfpathlineto{\pgfqpoint{4.473902in}{4.321654in}}%
\pgfusepath{stroke}%
\end{pgfscope}%
\begin{pgfscope}%
\pgfpathrectangle{\pgfqpoint{0.481978in}{0.331635in}}{\pgfqpoint{9.300000in}{7.700000in}}%
\pgfusepath{clip}%
\pgfsetrectcap%
\pgfsetroundjoin%
\pgfsetlinewidth{1.505625pt}%
\definecolor{currentstroke}{rgb}{0.870588,0.733333,0.607843}%
\pgfsetstrokecolor{currentstroke}%
\pgfsetstrokeopacity{0.800000}%
\pgfsetdash{}{0pt}%
\pgfpathmoveto{\pgfqpoint{2.801903in}{4.401537in}}%
\pgfpathlineto{\pgfqpoint{4.473902in}{4.321654in}}%
\pgfusepath{stroke}%
\end{pgfscope}%
\begin{pgfscope}%
\pgfpathrectangle{\pgfqpoint{0.481978in}{0.331635in}}{\pgfqpoint{9.300000in}{7.700000in}}%
\pgfusepath{clip}%
\pgfsetrectcap%
\pgfsetroundjoin%
\pgfsetlinewidth{1.505625pt}%
\definecolor{currentstroke}{rgb}{0.870588,0.733333,0.607843}%
\pgfsetstrokecolor{currentstroke}%
\pgfsetstrokeopacity{0.800000}%
\pgfsetdash{}{0pt}%
\pgfpathmoveto{\pgfqpoint{3.503709in}{4.313535in}}%
\pgfpathlineto{\pgfqpoint{4.473902in}{4.321654in}}%
\pgfusepath{stroke}%
\end{pgfscope}%
\begin{pgfscope}%
\pgfpathrectangle{\pgfqpoint{0.481978in}{0.331635in}}{\pgfqpoint{9.300000in}{7.700000in}}%
\pgfusepath{clip}%
\pgfsetrectcap%
\pgfsetroundjoin%
\pgfsetlinewidth{1.505625pt}%
\definecolor{currentstroke}{rgb}{0.870588,0.733333,0.607843}%
\pgfsetstrokecolor{currentstroke}%
\pgfsetstrokeopacity{0.800000}%
\pgfsetdash{}{0pt}%
\pgfpathmoveto{\pgfqpoint{7.049429in}{6.056502in}}%
\pgfpathlineto{\pgfqpoint{4.473902in}{4.321654in}}%
\pgfusepath{stroke}%
\end{pgfscope}%
\begin{pgfscope}%
\pgfpathrectangle{\pgfqpoint{0.481978in}{0.331635in}}{\pgfqpoint{9.300000in}{7.700000in}}%
\pgfusepath{clip}%
\pgfsetrectcap%
\pgfsetroundjoin%
\pgfsetlinewidth{1.505625pt}%
\definecolor{currentstroke}{rgb}{0.870588,0.733333,0.607843}%
\pgfsetstrokecolor{currentstroke}%
\pgfsetstrokeopacity{0.800000}%
\pgfsetdash{}{0pt}%
\pgfpathmoveto{\pgfqpoint{6.348280in}{4.846864in}}%
\pgfpathlineto{\pgfqpoint{4.473902in}{4.321654in}}%
\pgfusepath{stroke}%
\end{pgfscope}%
\begin{pgfscope}%
\pgfpathrectangle{\pgfqpoint{0.481978in}{0.331635in}}{\pgfqpoint{9.300000in}{7.700000in}}%
\pgfusepath{clip}%
\pgfsetrectcap%
\pgfsetroundjoin%
\pgfsetlinewidth{1.505625pt}%
\definecolor{currentstroke}{rgb}{0.870588,0.733333,0.607843}%
\pgfsetstrokecolor{currentstroke}%
\pgfsetstrokeopacity{0.800000}%
\pgfsetdash{}{0pt}%
\pgfpathmoveto{\pgfqpoint{5.923605in}{3.681854in}}%
\pgfpathlineto{\pgfqpoint{4.473902in}{4.321654in}}%
\pgfusepath{stroke}%
\end{pgfscope}%
\begin{pgfscope}%
\pgfpathrectangle{\pgfqpoint{0.481978in}{0.331635in}}{\pgfqpoint{9.300000in}{7.700000in}}%
\pgfusepath{clip}%
\pgfsetrectcap%
\pgfsetroundjoin%
\pgfsetlinewidth{1.505625pt}%
\definecolor{currentstroke}{rgb}{0.870588,0.733333,0.607843}%
\pgfsetstrokecolor{currentstroke}%
\pgfsetstrokeopacity{0.800000}%
\pgfsetdash{}{0pt}%
\pgfpathmoveto{\pgfqpoint{5.344914in}{4.036077in}}%
\pgfpathlineto{\pgfqpoint{4.473902in}{4.321654in}}%
\pgfusepath{stroke}%
\end{pgfscope}%
\begin{pgfscope}%
\pgfpathrectangle{\pgfqpoint{0.481978in}{0.331635in}}{\pgfqpoint{9.300000in}{7.700000in}}%
\pgfusepath{clip}%
\pgfsetrectcap%
\pgfsetroundjoin%
\pgfsetlinewidth{1.505625pt}%
\definecolor{currentstroke}{rgb}{0.870588,0.733333,0.607843}%
\pgfsetstrokecolor{currentstroke}%
\pgfsetstrokeopacity{0.800000}%
\pgfsetdash{}{0pt}%
\pgfpathmoveto{\pgfqpoint{4.527667in}{3.744276in}}%
\pgfpathlineto{\pgfqpoint{4.473902in}{4.321654in}}%
\pgfusepath{stroke}%
\end{pgfscope}%
\begin{pgfscope}%
\pgfpathrectangle{\pgfqpoint{0.481978in}{0.331635in}}{\pgfqpoint{9.300000in}{7.700000in}}%
\pgfusepath{clip}%
\pgfsetrectcap%
\pgfsetroundjoin%
\pgfsetlinewidth{1.505625pt}%
\definecolor{currentstroke}{rgb}{0.870588,0.733333,0.607843}%
\pgfsetstrokecolor{currentstroke}%
\pgfsetstrokeopacity{0.800000}%
\pgfsetdash{}{0pt}%
\pgfpathmoveto{\pgfqpoint{2.059850in}{4.036460in}}%
\pgfpathlineto{\pgfqpoint{4.473902in}{4.321654in}}%
\pgfusepath{stroke}%
\end{pgfscope}%
\begin{pgfscope}%
\pgfpathrectangle{\pgfqpoint{0.481978in}{0.331635in}}{\pgfqpoint{9.300000in}{7.700000in}}%
\pgfusepath{clip}%
\pgfsetrectcap%
\pgfsetroundjoin%
\pgfsetlinewidth{1.505625pt}%
\definecolor{currentstroke}{rgb}{0.870588,0.733333,0.607843}%
\pgfsetstrokecolor{currentstroke}%
\pgfsetstrokeopacity{0.800000}%
\pgfsetdash{}{0pt}%
\pgfpathmoveto{\pgfqpoint{5.974496in}{4.950112in}}%
\pgfpathlineto{\pgfqpoint{4.473902in}{4.321654in}}%
\pgfusepath{stroke}%
\end{pgfscope}%
\begin{pgfscope}%
\pgfpathrectangle{\pgfqpoint{0.481978in}{0.331635in}}{\pgfqpoint{9.300000in}{7.700000in}}%
\pgfusepath{clip}%
\pgfsetrectcap%
\pgfsetroundjoin%
\pgfsetlinewidth{1.505625pt}%
\definecolor{currentstroke}{rgb}{0.870588,0.733333,0.607843}%
\pgfsetstrokecolor{currentstroke}%
\pgfsetstrokeopacity{0.800000}%
\pgfsetdash{}{0pt}%
\pgfpathmoveto{\pgfqpoint{2.319348in}{4.794802in}}%
\pgfpathlineto{\pgfqpoint{4.473902in}{4.321654in}}%
\pgfusepath{stroke}%
\end{pgfscope}%
\begin{pgfscope}%
\pgfpathrectangle{\pgfqpoint{0.481978in}{0.331635in}}{\pgfqpoint{9.300000in}{7.700000in}}%
\pgfusepath{clip}%
\pgfsetrectcap%
\pgfsetroundjoin%
\pgfsetlinewidth{1.505625pt}%
\definecolor{currentstroke}{rgb}{0.870588,0.733333,0.607843}%
\pgfsetstrokecolor{currentstroke}%
\pgfsetstrokeopacity{0.800000}%
\pgfsetdash{}{0pt}%
\pgfpathmoveto{\pgfqpoint{6.238443in}{6.242431in}}%
\pgfpathlineto{\pgfqpoint{4.473902in}{4.321654in}}%
\pgfusepath{stroke}%
\end{pgfscope}%
\begin{pgfscope}%
\pgfpathrectangle{\pgfqpoint{0.481978in}{0.331635in}}{\pgfqpoint{9.300000in}{7.700000in}}%
\pgfusepath{clip}%
\pgfsetrectcap%
\pgfsetroundjoin%
\pgfsetlinewidth{1.505625pt}%
\definecolor{currentstroke}{rgb}{0.870588,0.733333,0.607843}%
\pgfsetstrokecolor{currentstroke}%
\pgfsetstrokeopacity{0.800000}%
\pgfsetdash{}{0pt}%
\pgfpathmoveto{\pgfqpoint{4.167128in}{3.897191in}}%
\pgfpathlineto{\pgfqpoint{4.473902in}{4.321654in}}%
\pgfusepath{stroke}%
\end{pgfscope}%
\begin{pgfscope}%
\pgfpathrectangle{\pgfqpoint{0.481978in}{0.331635in}}{\pgfqpoint{9.300000in}{7.700000in}}%
\pgfusepath{clip}%
\pgfsetrectcap%
\pgfsetroundjoin%
\pgfsetlinewidth{1.505625pt}%
\definecolor{currentstroke}{rgb}{0.870588,0.733333,0.607843}%
\pgfsetstrokecolor{currentstroke}%
\pgfsetstrokeopacity{0.800000}%
\pgfsetdash{}{0pt}%
\pgfpathmoveto{\pgfqpoint{4.383560in}{3.597829in}}%
\pgfpathlineto{\pgfqpoint{4.473902in}{4.321654in}}%
\pgfusepath{stroke}%
\end{pgfscope}%
\begin{pgfscope}%
\pgfpathrectangle{\pgfqpoint{0.481978in}{0.331635in}}{\pgfqpoint{9.300000in}{7.700000in}}%
\pgfusepath{clip}%
\pgfsetrectcap%
\pgfsetroundjoin%
\pgfsetlinewidth{1.505625pt}%
\definecolor{currentstroke}{rgb}{0.870588,0.733333,0.607843}%
\pgfsetstrokecolor{currentstroke}%
\pgfsetstrokeopacity{0.800000}%
\pgfsetdash{}{0pt}%
\pgfpathmoveto{\pgfqpoint{5.769577in}{4.001990in}}%
\pgfpathlineto{\pgfqpoint{4.473902in}{4.321654in}}%
\pgfusepath{stroke}%
\end{pgfscope}%
\begin{pgfscope}%
\pgfpathrectangle{\pgfqpoint{0.481978in}{0.331635in}}{\pgfqpoint{9.300000in}{7.700000in}}%
\pgfusepath{clip}%
\pgfsetrectcap%
\pgfsetroundjoin%
\pgfsetlinewidth{1.505625pt}%
\definecolor{currentstroke}{rgb}{0.870588,0.733333,0.607843}%
\pgfsetstrokecolor{currentstroke}%
\pgfsetstrokeopacity{0.800000}%
\pgfsetdash{}{0pt}%
\pgfpathmoveto{\pgfqpoint{9.359251in}{4.570076in}}%
\pgfpathlineto{\pgfqpoint{4.473902in}{4.321654in}}%
\pgfusepath{stroke}%
\end{pgfscope}%
\begin{pgfscope}%
\pgfpathrectangle{\pgfqpoint{0.481978in}{0.331635in}}{\pgfqpoint{9.300000in}{7.700000in}}%
\pgfusepath{clip}%
\pgfsetrectcap%
\pgfsetroundjoin%
\pgfsetlinewidth{1.505625pt}%
\definecolor{currentstroke}{rgb}{0.870588,0.733333,0.607843}%
\pgfsetstrokecolor{currentstroke}%
\pgfsetstrokeopacity{0.800000}%
\pgfsetdash{}{0pt}%
\pgfpathmoveto{\pgfqpoint{5.675309in}{2.672631in}}%
\pgfpathlineto{\pgfqpoint{4.473902in}{4.321654in}}%
\pgfusepath{stroke}%
\end{pgfscope}%
\begin{pgfscope}%
\pgfpathrectangle{\pgfqpoint{0.481978in}{0.331635in}}{\pgfqpoint{9.300000in}{7.700000in}}%
\pgfusepath{clip}%
\pgfsetrectcap%
\pgfsetroundjoin%
\pgfsetlinewidth{1.505625pt}%
\definecolor{currentstroke}{rgb}{0.870588,0.733333,0.607843}%
\pgfsetstrokecolor{currentstroke}%
\pgfsetstrokeopacity{0.800000}%
\pgfsetdash{}{0pt}%
\pgfpathmoveto{\pgfqpoint{3.026449in}{4.347964in}}%
\pgfpathlineto{\pgfqpoint{4.473902in}{4.321654in}}%
\pgfusepath{stroke}%
\end{pgfscope}%
\begin{pgfscope}%
\pgfpathrectangle{\pgfqpoint{0.481978in}{0.331635in}}{\pgfqpoint{9.300000in}{7.700000in}}%
\pgfusepath{clip}%
\pgfsetrectcap%
\pgfsetroundjoin%
\pgfsetlinewidth{1.505625pt}%
\definecolor{currentstroke}{rgb}{0.870588,0.733333,0.607843}%
\pgfsetstrokecolor{currentstroke}%
\pgfsetstrokeopacity{0.800000}%
\pgfsetdash{}{0pt}%
\pgfpathmoveto{\pgfqpoint{6.761280in}{5.866030in}}%
\pgfpathlineto{\pgfqpoint{4.473902in}{4.321654in}}%
\pgfusepath{stroke}%
\end{pgfscope}%
\begin{pgfscope}%
\pgfpathrectangle{\pgfqpoint{0.481978in}{0.331635in}}{\pgfqpoint{9.300000in}{7.700000in}}%
\pgfusepath{clip}%
\pgfsetrectcap%
\pgfsetroundjoin%
\pgfsetlinewidth{1.505625pt}%
\definecolor{currentstroke}{rgb}{0.870588,0.733333,0.607843}%
\pgfsetstrokecolor{currentstroke}%
\pgfsetstrokeopacity{0.800000}%
\pgfsetdash{}{0pt}%
\pgfpathmoveto{\pgfqpoint{2.445109in}{3.109579in}}%
\pgfpathlineto{\pgfqpoint{4.473902in}{4.321654in}}%
\pgfusepath{stroke}%
\end{pgfscope}%
\begin{pgfscope}%
\pgfpathrectangle{\pgfqpoint{0.481978in}{0.331635in}}{\pgfqpoint{9.300000in}{7.700000in}}%
\pgfusepath{clip}%
\pgfsetrectcap%
\pgfsetroundjoin%
\pgfsetlinewidth{1.505625pt}%
\definecolor{currentstroke}{rgb}{0.870588,0.733333,0.607843}%
\pgfsetstrokecolor{currentstroke}%
\pgfsetstrokeopacity{0.800000}%
\pgfsetdash{}{0pt}%
\pgfpathmoveto{\pgfqpoint{2.556572in}{4.487385in}}%
\pgfpathlineto{\pgfqpoint{4.473902in}{4.321654in}}%
\pgfusepath{stroke}%
\end{pgfscope}%
\begin{pgfscope}%
\pgfpathrectangle{\pgfqpoint{0.481978in}{0.331635in}}{\pgfqpoint{9.300000in}{7.700000in}}%
\pgfusepath{clip}%
\pgfsetrectcap%
\pgfsetroundjoin%
\pgfsetlinewidth{1.505625pt}%
\definecolor{currentstroke}{rgb}{0.870588,0.733333,0.607843}%
\pgfsetstrokecolor{currentstroke}%
\pgfsetstrokeopacity{0.800000}%
\pgfsetdash{}{0pt}%
\pgfpathmoveto{\pgfqpoint{3.956527in}{3.307551in}}%
\pgfpathlineto{\pgfqpoint{4.473902in}{4.321654in}}%
\pgfusepath{stroke}%
\end{pgfscope}%
\begin{pgfscope}%
\pgfpathrectangle{\pgfqpoint{0.481978in}{0.331635in}}{\pgfqpoint{9.300000in}{7.700000in}}%
\pgfusepath{clip}%
\pgfsetrectcap%
\pgfsetroundjoin%
\pgfsetlinewidth{1.505625pt}%
\definecolor{currentstroke}{rgb}{0.870588,0.733333,0.607843}%
\pgfsetstrokecolor{currentstroke}%
\pgfsetstrokeopacity{0.800000}%
\pgfsetdash{}{0pt}%
\pgfpathmoveto{\pgfqpoint{2.593480in}{4.742968in}}%
\pgfpathlineto{\pgfqpoint{4.473902in}{4.321654in}}%
\pgfusepath{stroke}%
\end{pgfscope}%
\begin{pgfscope}%
\pgfpathrectangle{\pgfqpoint{0.481978in}{0.331635in}}{\pgfqpoint{9.300000in}{7.700000in}}%
\pgfusepath{clip}%
\pgfsetrectcap%
\pgfsetroundjoin%
\pgfsetlinewidth{1.505625pt}%
\definecolor{currentstroke}{rgb}{0.870588,0.733333,0.607843}%
\pgfsetstrokecolor{currentstroke}%
\pgfsetstrokeopacity{0.800000}%
\pgfsetdash{}{0pt}%
\pgfpathmoveto{\pgfqpoint{1.766805in}{3.503655in}}%
\pgfpathlineto{\pgfqpoint{4.473902in}{4.321654in}}%
\pgfusepath{stroke}%
\end{pgfscope}%
\begin{pgfscope}%
\pgfpathrectangle{\pgfqpoint{0.481978in}{0.331635in}}{\pgfqpoint{9.300000in}{7.700000in}}%
\pgfusepath{clip}%
\pgfsetrectcap%
\pgfsetroundjoin%
\pgfsetlinewidth{1.505625pt}%
\definecolor{currentstroke}{rgb}{0.870588,0.733333,0.607843}%
\pgfsetstrokecolor{currentstroke}%
\pgfsetstrokeopacity{0.800000}%
\pgfsetdash{}{0pt}%
\pgfpathmoveto{\pgfqpoint{4.977562in}{4.282649in}}%
\pgfpathlineto{\pgfqpoint{4.473902in}{4.321654in}}%
\pgfusepath{stroke}%
\end{pgfscope}%
\begin{pgfscope}%
\pgfpathrectangle{\pgfqpoint{0.481978in}{0.331635in}}{\pgfqpoint{9.300000in}{7.700000in}}%
\pgfusepath{clip}%
\pgfsetrectcap%
\pgfsetroundjoin%
\pgfsetlinewidth{1.505625pt}%
\definecolor{currentstroke}{rgb}{0.870588,0.733333,0.607843}%
\pgfsetstrokecolor{currentstroke}%
\pgfsetstrokeopacity{0.800000}%
\pgfsetdash{}{0pt}%
\pgfpathmoveto{\pgfqpoint{5.616567in}{3.990039in}}%
\pgfpathlineto{\pgfqpoint{4.473902in}{4.321654in}}%
\pgfusepath{stroke}%
\end{pgfscope}%
\begin{pgfscope}%
\pgfpathrectangle{\pgfqpoint{0.481978in}{0.331635in}}{\pgfqpoint{9.300000in}{7.700000in}}%
\pgfusepath{clip}%
\pgfsetrectcap%
\pgfsetroundjoin%
\pgfsetlinewidth{1.505625pt}%
\definecolor{currentstroke}{rgb}{0.870588,0.733333,0.607843}%
\pgfsetstrokecolor{currentstroke}%
\pgfsetstrokeopacity{0.800000}%
\pgfsetdash{}{0pt}%
\pgfpathmoveto{\pgfqpoint{2.747420in}{2.580517in}}%
\pgfpathlineto{\pgfqpoint{4.473902in}{4.321654in}}%
\pgfusepath{stroke}%
\end{pgfscope}%
\begin{pgfscope}%
\pgfpathrectangle{\pgfqpoint{0.481978in}{0.331635in}}{\pgfqpoint{9.300000in}{7.700000in}}%
\pgfusepath{clip}%
\pgfsetrectcap%
\pgfsetroundjoin%
\pgfsetlinewidth{1.505625pt}%
\definecolor{currentstroke}{rgb}{0.870588,0.733333,0.607843}%
\pgfsetstrokecolor{currentstroke}%
\pgfsetstrokeopacity{0.800000}%
\pgfsetdash{}{0pt}%
\pgfpathmoveto{\pgfqpoint{3.710302in}{3.442909in}}%
\pgfpathlineto{\pgfqpoint{4.473902in}{4.321654in}}%
\pgfusepath{stroke}%
\end{pgfscope}%
\begin{pgfscope}%
\pgfpathrectangle{\pgfqpoint{0.481978in}{0.331635in}}{\pgfqpoint{9.300000in}{7.700000in}}%
\pgfusepath{clip}%
\pgfsetrectcap%
\pgfsetroundjoin%
\pgfsetlinewidth{1.505625pt}%
\definecolor{currentstroke}{rgb}{0.870588,0.733333,0.607843}%
\pgfsetstrokecolor{currentstroke}%
\pgfsetstrokeopacity{0.800000}%
\pgfsetdash{}{0pt}%
\pgfpathmoveto{\pgfqpoint{2.149626in}{4.883578in}}%
\pgfpathlineto{\pgfqpoint{4.473902in}{4.321654in}}%
\pgfusepath{stroke}%
\end{pgfscope}%
\begin{pgfscope}%
\pgfpathrectangle{\pgfqpoint{0.481978in}{0.331635in}}{\pgfqpoint{9.300000in}{7.700000in}}%
\pgfusepath{clip}%
\pgfsetrectcap%
\pgfsetroundjoin%
\pgfsetlinewidth{1.505625pt}%
\definecolor{currentstroke}{rgb}{0.870588,0.733333,0.607843}%
\pgfsetstrokecolor{currentstroke}%
\pgfsetstrokeopacity{0.800000}%
\pgfsetdash{}{0pt}%
\pgfpathmoveto{\pgfqpoint{2.077561in}{3.103422in}}%
\pgfpathlineto{\pgfqpoint{4.473902in}{4.321654in}}%
\pgfusepath{stroke}%
\end{pgfscope}%
\begin{pgfscope}%
\pgfpathrectangle{\pgfqpoint{0.481978in}{0.331635in}}{\pgfqpoint{9.300000in}{7.700000in}}%
\pgfusepath{clip}%
\pgfsetrectcap%
\pgfsetroundjoin%
\pgfsetlinewidth{1.505625pt}%
\definecolor{currentstroke}{rgb}{0.870588,0.733333,0.607843}%
\pgfsetstrokecolor{currentstroke}%
\pgfsetstrokeopacity{0.800000}%
\pgfsetdash{}{0pt}%
\pgfpathmoveto{\pgfqpoint{1.957320in}{5.229684in}}%
\pgfpathlineto{\pgfqpoint{4.473902in}{4.321654in}}%
\pgfusepath{stroke}%
\end{pgfscope}%
\begin{pgfscope}%
\pgfpathrectangle{\pgfqpoint{0.481978in}{0.331635in}}{\pgfqpoint{9.300000in}{7.700000in}}%
\pgfusepath{clip}%
\pgfsetrectcap%
\pgfsetroundjoin%
\pgfsetlinewidth{1.505625pt}%
\definecolor{currentstroke}{rgb}{0.870588,0.733333,0.607843}%
\pgfsetstrokecolor{currentstroke}%
\pgfsetstrokeopacity{0.800000}%
\pgfsetdash{}{0pt}%
\pgfpathmoveto{\pgfqpoint{2.660918in}{3.550587in}}%
\pgfpathlineto{\pgfqpoint{4.473902in}{4.321654in}}%
\pgfusepath{stroke}%
\end{pgfscope}%
\begin{pgfscope}%
\pgfpathrectangle{\pgfqpoint{0.481978in}{0.331635in}}{\pgfqpoint{9.300000in}{7.700000in}}%
\pgfusepath{clip}%
\pgfsetrectcap%
\pgfsetroundjoin%
\pgfsetlinewidth{1.505625pt}%
\definecolor{currentstroke}{rgb}{0.870588,0.733333,0.607843}%
\pgfsetstrokecolor{currentstroke}%
\pgfsetstrokeopacity{0.800000}%
\pgfsetdash{}{0pt}%
\pgfpathmoveto{\pgfqpoint{7.871019in}{5.550531in}}%
\pgfpathlineto{\pgfqpoint{4.473902in}{4.321654in}}%
\pgfusepath{stroke}%
\end{pgfscope}%
\begin{pgfscope}%
\pgfpathrectangle{\pgfqpoint{0.481978in}{0.331635in}}{\pgfqpoint{9.300000in}{7.700000in}}%
\pgfusepath{clip}%
\pgfsetrectcap%
\pgfsetroundjoin%
\pgfsetlinewidth{1.505625pt}%
\definecolor{currentstroke}{rgb}{0.870588,0.733333,0.607843}%
\pgfsetstrokecolor{currentstroke}%
\pgfsetstrokeopacity{0.800000}%
\pgfsetdash{}{0pt}%
\pgfpathmoveto{\pgfqpoint{5.367711in}{4.176744in}}%
\pgfpathlineto{\pgfqpoint{4.473902in}{4.321654in}}%
\pgfusepath{stroke}%
\end{pgfscope}%
\begin{pgfscope}%
\pgfpathrectangle{\pgfqpoint{0.481978in}{0.331635in}}{\pgfqpoint{9.300000in}{7.700000in}}%
\pgfusepath{clip}%
\pgfsetrectcap%
\pgfsetroundjoin%
\pgfsetlinewidth{1.505625pt}%
\definecolor{currentstroke}{rgb}{0.870588,0.733333,0.607843}%
\pgfsetstrokecolor{currentstroke}%
\pgfsetstrokeopacity{0.800000}%
\pgfsetdash{}{0pt}%
\pgfpathmoveto{\pgfqpoint{2.321412in}{4.244642in}}%
\pgfpathlineto{\pgfqpoint{4.473902in}{4.321654in}}%
\pgfusepath{stroke}%
\end{pgfscope}%
\begin{pgfscope}%
\pgfpathrectangle{\pgfqpoint{0.481978in}{0.331635in}}{\pgfqpoint{9.300000in}{7.700000in}}%
\pgfusepath{clip}%
\pgfsetrectcap%
\pgfsetroundjoin%
\pgfsetlinewidth{1.505625pt}%
\definecolor{currentstroke}{rgb}{0.870588,0.733333,0.607843}%
\pgfsetstrokecolor{currentstroke}%
\pgfsetstrokeopacity{0.800000}%
\pgfsetdash{}{0pt}%
\pgfpathmoveto{\pgfqpoint{7.119603in}{6.303649in}}%
\pgfpathlineto{\pgfqpoint{4.473902in}{4.321654in}}%
\pgfusepath{stroke}%
\end{pgfscope}%
\begin{pgfscope}%
\pgfpathrectangle{\pgfqpoint{0.481978in}{0.331635in}}{\pgfqpoint{9.300000in}{7.700000in}}%
\pgfusepath{clip}%
\pgfsetrectcap%
\pgfsetroundjoin%
\pgfsetlinewidth{1.505625pt}%
\definecolor{currentstroke}{rgb}{0.870588,0.733333,0.607843}%
\pgfsetstrokecolor{currentstroke}%
\pgfsetstrokeopacity{0.800000}%
\pgfsetdash{}{0pt}%
\pgfpathmoveto{\pgfqpoint{6.487284in}{5.504131in}}%
\pgfpathlineto{\pgfqpoint{4.473902in}{4.321654in}}%
\pgfusepath{stroke}%
\end{pgfscope}%
\begin{pgfscope}%
\pgfpathrectangle{\pgfqpoint{0.481978in}{0.331635in}}{\pgfqpoint{9.300000in}{7.700000in}}%
\pgfusepath{clip}%
\pgfsetrectcap%
\pgfsetroundjoin%
\pgfsetlinewidth{1.505625pt}%
\definecolor{currentstroke}{rgb}{0.870588,0.733333,0.607843}%
\pgfsetstrokecolor{currentstroke}%
\pgfsetstrokeopacity{0.800000}%
\pgfsetdash{}{0pt}%
\pgfpathmoveto{\pgfqpoint{8.920274in}{3.555200in}}%
\pgfpathlineto{\pgfqpoint{4.473902in}{4.321654in}}%
\pgfusepath{stroke}%
\end{pgfscope}%
\begin{pgfscope}%
\pgfpathrectangle{\pgfqpoint{0.481978in}{0.331635in}}{\pgfqpoint{9.300000in}{7.700000in}}%
\pgfusepath{clip}%
\pgfsetrectcap%
\pgfsetroundjoin%
\pgfsetlinewidth{1.505625pt}%
\definecolor{currentstroke}{rgb}{0.870588,0.733333,0.607843}%
\pgfsetstrokecolor{currentstroke}%
\pgfsetstrokeopacity{0.800000}%
\pgfsetdash{}{0pt}%
\pgfpathmoveto{\pgfqpoint{7.303090in}{6.145660in}}%
\pgfpathlineto{\pgfqpoint{4.473902in}{4.321654in}}%
\pgfusepath{stroke}%
\end{pgfscope}%
\begin{pgfscope}%
\pgfpathrectangle{\pgfqpoint{0.481978in}{0.331635in}}{\pgfqpoint{9.300000in}{7.700000in}}%
\pgfusepath{clip}%
\pgfsetrectcap%
\pgfsetroundjoin%
\pgfsetlinewidth{1.505625pt}%
\definecolor{currentstroke}{rgb}{0.870588,0.733333,0.607843}%
\pgfsetstrokecolor{currentstroke}%
\pgfsetstrokeopacity{0.800000}%
\pgfsetdash{}{0pt}%
\pgfpathmoveto{\pgfqpoint{9.189203in}{4.320348in}}%
\pgfpathlineto{\pgfqpoint{4.473902in}{4.321654in}}%
\pgfusepath{stroke}%
\end{pgfscope}%
\begin{pgfscope}%
\pgfsetrectcap%
\pgfsetmiterjoin%
\pgfsetlinewidth{0.803000pt}%
\definecolor{currentstroke}{rgb}{0.000000,0.000000,0.000000}%
\pgfsetstrokecolor{currentstroke}%
\pgfsetdash{}{0pt}%
\pgfpathmoveto{\pgfqpoint{0.481978in}{0.331635in}}%
\pgfpathlineto{\pgfqpoint{0.481978in}{8.031635in}}%
\pgfusepath{stroke}%
\end{pgfscope}%
\begin{pgfscope}%
\pgfsetrectcap%
\pgfsetmiterjoin%
\pgfsetlinewidth{0.803000pt}%
\definecolor{currentstroke}{rgb}{0.000000,0.000000,0.000000}%
\pgfsetstrokecolor{currentstroke}%
\pgfsetdash{}{0pt}%
\pgfpathmoveto{\pgfqpoint{9.781978in}{0.331635in}}%
\pgfpathlineto{\pgfqpoint{9.781978in}{8.031635in}}%
\pgfusepath{stroke}%
\end{pgfscope}%
\begin{pgfscope}%
\pgfsetrectcap%
\pgfsetmiterjoin%
\pgfsetlinewidth{0.803000pt}%
\definecolor{currentstroke}{rgb}{0.000000,0.000000,0.000000}%
\pgfsetstrokecolor{currentstroke}%
\pgfsetdash{}{0pt}%
\pgfpathmoveto{\pgfqpoint{0.481978in}{0.331635in}}%
\pgfpathlineto{\pgfqpoint{9.781978in}{0.331635in}}%
\pgfusepath{stroke}%
\end{pgfscope}%
\begin{pgfscope}%
\pgfsetrectcap%
\pgfsetmiterjoin%
\pgfsetlinewidth{0.803000pt}%
\definecolor{currentstroke}{rgb}{0.000000,0.000000,0.000000}%
\pgfsetstrokecolor{currentstroke}%
\pgfsetdash{}{0pt}%
\pgfpathmoveto{\pgfqpoint{0.481978in}{8.031635in}}%
\pgfpathlineto{\pgfqpoint{9.781978in}{8.031635in}}%
\pgfusepath{stroke}%
\end{pgfscope}%
\begin{pgfscope}%
\definecolor{textcolor}{rgb}{0.000000,0.000000,0.000000}%
\pgfsetstrokecolor{textcolor}%
\pgfsetfillcolor{textcolor}%
\pgftext[x=5.131978in,y=8.114968in,,base]{\color{textcolor}\sffamily\fontsize{12.000000}{14.400000}\selectfont T-SNE for chair images with domain randomisation}%
\end{pgfscope}%
\begin{pgfscope}%
\pgfsetbuttcap%
\pgfsetmiterjoin%
\definecolor{currentfill}{rgb}{1.000000,1.000000,1.000000}%
\pgfsetfillcolor{currentfill}%
\pgfsetfillopacity{0.800000}%
\pgfsetlinewidth{1.003750pt}%
\definecolor{currentstroke}{rgb}{0.800000,0.800000,0.800000}%
\pgfsetstrokecolor{currentstroke}%
\pgfsetstrokeopacity{0.800000}%
\pgfsetdash{}{0pt}%
\pgfpathmoveto{\pgfqpoint{9.879200in}{3.539566in}}%
\pgfpathlineto{\pgfqpoint{12.348384in}{3.539566in}}%
\pgfpathquadraticcurveto{\pgfqpoint{12.376162in}{3.539566in}}{\pgfqpoint{12.376162in}{3.567344in}}%
\pgfpathlineto{\pgfqpoint{12.376162in}{4.795926in}}%
\pgfpathquadraticcurveto{\pgfqpoint{12.376162in}{4.823704in}}{\pgfqpoint{12.348384in}{4.823704in}}%
\pgfpathlineto{\pgfqpoint{9.879200in}{4.823704in}}%
\pgfpathquadraticcurveto{\pgfqpoint{9.851422in}{4.823704in}}{\pgfqpoint{9.851422in}{4.795926in}}%
\pgfpathlineto{\pgfqpoint{9.851422in}{3.567344in}}%
\pgfpathquadraticcurveto{\pgfqpoint{9.851422in}{3.539566in}}{\pgfqpoint{9.879200in}{3.539566in}}%
\pgfpathclose%
\pgfusepath{stroke,fill}%
\end{pgfscope}%
\begin{pgfscope}%
\pgfsetbuttcap%
\pgfsetroundjoin%
\definecolor{currentfill}{rgb}{0.631373,0.788235,0.956863}%
\pgfsetfillcolor{currentfill}%
\pgfsetlinewidth{1.003750pt}%
\definecolor{currentstroke}{rgb}{0.631373,0.788235,0.956863}%
\pgfsetstrokecolor{currentstroke}%
\pgfsetdash{}{0pt}%
\pgfsys@defobject{currentmarker}{\pgfqpoint{-0.041667in}{-0.041667in}}{\pgfqpoint{0.041667in}{0.041667in}}{%
\pgfpathmoveto{\pgfqpoint{0.000000in}{-0.041667in}}%
\pgfpathcurveto{\pgfqpoint{0.011050in}{-0.041667in}}{\pgfqpoint{0.021649in}{-0.037276in}}{\pgfqpoint{0.029463in}{-0.029463in}}%
\pgfpathcurveto{\pgfqpoint{0.037276in}{-0.021649in}}{\pgfqpoint{0.041667in}{-0.011050in}}{\pgfqpoint{0.041667in}{0.000000in}}%
\pgfpathcurveto{\pgfqpoint{0.041667in}{0.011050in}}{\pgfqpoint{0.037276in}{0.021649in}}{\pgfqpoint{0.029463in}{0.029463in}}%
\pgfpathcurveto{\pgfqpoint{0.021649in}{0.037276in}}{\pgfqpoint{0.011050in}{0.041667in}}{\pgfqpoint{0.000000in}{0.041667in}}%
\pgfpathcurveto{\pgfqpoint{-0.011050in}{0.041667in}}{\pgfqpoint{-0.021649in}{0.037276in}}{\pgfqpoint{-0.029463in}{0.029463in}}%
\pgfpathcurveto{\pgfqpoint{-0.037276in}{0.021649in}}{\pgfqpoint{-0.041667in}{0.011050in}}{\pgfqpoint{-0.041667in}{0.000000in}}%
\pgfpathcurveto{\pgfqpoint{-0.041667in}{-0.011050in}}{\pgfqpoint{-0.037276in}{-0.021649in}}{\pgfqpoint{-0.029463in}{-0.029463in}}%
\pgfpathcurveto{\pgfqpoint{-0.021649in}{-0.037276in}}{\pgfqpoint{-0.011050in}{-0.041667in}}{\pgfqpoint{0.000000in}{-0.041667in}}%
\pgfpathclose%
\pgfusepath{stroke,fill}%
}%
\begin{pgfscope}%
\pgfsys@transformshift{10.045867in}{4.699084in}%
\pgfsys@useobject{currentmarker}{}%
\end{pgfscope}%
\end{pgfscope}%
\begin{pgfscope}%
\definecolor{textcolor}{rgb}{0.000000,0.000000,0.000000}%
\pgfsetstrokecolor{textcolor}%
\pgfsetfillcolor{textcolor}%
\pgftext[x=10.295867in,y=4.662625in,left,base]{\color{textcolor}\sffamily\fontsize{10.000000}{12.000000}\selectfont Pix3D}%
\end{pgfscope}%
\begin{pgfscope}%
\pgfsetbuttcap%
\pgfsetroundjoin%
\definecolor{currentfill}{rgb}{1.000000,0.705882,0.509804}%
\pgfsetfillcolor{currentfill}%
\pgfsetlinewidth{1.003750pt}%
\definecolor{currentstroke}{rgb}{1.000000,0.705882,0.509804}%
\pgfsetstrokecolor{currentstroke}%
\pgfsetdash{}{0pt}%
\pgfsys@defobject{currentmarker}{\pgfqpoint{-0.041667in}{-0.041667in}}{\pgfqpoint{0.041667in}{0.041667in}}{%
\pgfpathmoveto{\pgfqpoint{0.000000in}{-0.041667in}}%
\pgfpathcurveto{\pgfqpoint{0.011050in}{-0.041667in}}{\pgfqpoint{0.021649in}{-0.037276in}}{\pgfqpoint{0.029463in}{-0.029463in}}%
\pgfpathcurveto{\pgfqpoint{0.037276in}{-0.021649in}}{\pgfqpoint{0.041667in}{-0.011050in}}{\pgfqpoint{0.041667in}{0.000000in}}%
\pgfpathcurveto{\pgfqpoint{0.041667in}{0.011050in}}{\pgfqpoint{0.037276in}{0.021649in}}{\pgfqpoint{0.029463in}{0.029463in}}%
\pgfpathcurveto{\pgfqpoint{0.021649in}{0.037276in}}{\pgfqpoint{0.011050in}{0.041667in}}{\pgfqpoint{0.000000in}{0.041667in}}%
\pgfpathcurveto{\pgfqpoint{-0.011050in}{0.041667in}}{\pgfqpoint{-0.021649in}{0.037276in}}{\pgfqpoint{-0.029463in}{0.029463in}}%
\pgfpathcurveto{\pgfqpoint{-0.037276in}{0.021649in}}{\pgfqpoint{-0.041667in}{0.011050in}}{\pgfqpoint{-0.041667in}{0.000000in}}%
\pgfpathcurveto{\pgfqpoint{-0.041667in}{-0.011050in}}{\pgfqpoint{-0.037276in}{-0.021649in}}{\pgfqpoint{-0.029463in}{-0.029463in}}%
\pgfpathcurveto{\pgfqpoint{-0.021649in}{-0.037276in}}{\pgfqpoint{-0.011050in}{-0.041667in}}{\pgfqpoint{0.000000in}{-0.041667in}}%
\pgfpathclose%
\pgfusepath{stroke,fill}%
}%
\begin{pgfscope}%
\pgfsys@transformshift{10.045867in}{4.495226in}%
\pgfsys@useobject{currentmarker}{}%
\end{pgfscope}%
\end{pgfscope}%
\begin{pgfscope}%
\definecolor{textcolor}{rgb}{0.000000,0.000000,0.000000}%
\pgfsetstrokecolor{textcolor}%
\pgfsetfillcolor{textcolor}%
\pgftext[x=10.295867in,y=4.458768in,left,base]{\color{textcolor}\sffamily\fontsize{10.000000}{12.000000}\selectfont s2r3dfree\_textureless}%
\end{pgfscope}%
\begin{pgfscope}%
\pgfsetbuttcap%
\pgfsetroundjoin%
\definecolor{currentfill}{rgb}{0.552941,0.898039,0.631373}%
\pgfsetfillcolor{currentfill}%
\pgfsetlinewidth{1.003750pt}%
\definecolor{currentstroke}{rgb}{0.552941,0.898039,0.631373}%
\pgfsetstrokecolor{currentstroke}%
\pgfsetdash{}{0pt}%
\pgfsys@defobject{currentmarker}{\pgfqpoint{-0.041667in}{-0.041667in}}{\pgfqpoint{0.041667in}{0.041667in}}{%
\pgfpathmoveto{\pgfqpoint{0.000000in}{-0.041667in}}%
\pgfpathcurveto{\pgfqpoint{0.011050in}{-0.041667in}}{\pgfqpoint{0.021649in}{-0.037276in}}{\pgfqpoint{0.029463in}{-0.029463in}}%
\pgfpathcurveto{\pgfqpoint{0.037276in}{-0.021649in}}{\pgfqpoint{0.041667in}{-0.011050in}}{\pgfqpoint{0.041667in}{0.000000in}}%
\pgfpathcurveto{\pgfqpoint{0.041667in}{0.011050in}}{\pgfqpoint{0.037276in}{0.021649in}}{\pgfqpoint{0.029463in}{0.029463in}}%
\pgfpathcurveto{\pgfqpoint{0.021649in}{0.037276in}}{\pgfqpoint{0.011050in}{0.041667in}}{\pgfqpoint{0.000000in}{0.041667in}}%
\pgfpathcurveto{\pgfqpoint{-0.011050in}{0.041667in}}{\pgfqpoint{-0.021649in}{0.037276in}}{\pgfqpoint{-0.029463in}{0.029463in}}%
\pgfpathcurveto{\pgfqpoint{-0.037276in}{0.021649in}}{\pgfqpoint{-0.041667in}{0.011050in}}{\pgfqpoint{-0.041667in}{0.000000in}}%
\pgfpathcurveto{\pgfqpoint{-0.041667in}{-0.011050in}}{\pgfqpoint{-0.037276in}{-0.021649in}}{\pgfqpoint{-0.029463in}{-0.029463in}}%
\pgfpathcurveto{\pgfqpoint{-0.021649in}{-0.037276in}}{\pgfqpoint{-0.011050in}{-0.041667in}}{\pgfqpoint{0.000000in}{-0.041667in}}%
\pgfpathclose%
\pgfusepath{stroke,fill}%
}%
\begin{pgfscope}%
\pgfsys@transformshift{10.045867in}{4.287504in}%
\pgfsys@useobject{currentmarker}{}%
\end{pgfscope}%
\end{pgfscope}%
\begin{pgfscope}%
\definecolor{textcolor}{rgb}{0.000000,0.000000,0.000000}%
\pgfsetstrokecolor{textcolor}%
\pgfsetfillcolor{textcolor}%
\pgftext[x=10.295867in,y=4.251045in,left,base]{\color{textcolor}\sffamily\fontsize{10.000000}{12.000000}\selectfont s2r3dfree\_textureless\_light}%
\end{pgfscope}%
\begin{pgfscope}%
\pgfsetbuttcap%
\pgfsetroundjoin%
\definecolor{currentfill}{rgb}{1.000000,0.623529,0.607843}%
\pgfsetfillcolor{currentfill}%
\pgfsetlinewidth{1.003750pt}%
\definecolor{currentstroke}{rgb}{1.000000,0.623529,0.607843}%
\pgfsetstrokecolor{currentstroke}%
\pgfsetdash{}{0pt}%
\pgfsys@defobject{currentmarker}{\pgfqpoint{-0.041667in}{-0.041667in}}{\pgfqpoint{0.041667in}{0.041667in}}{%
\pgfpathmoveto{\pgfqpoint{0.000000in}{-0.041667in}}%
\pgfpathcurveto{\pgfqpoint{0.011050in}{-0.041667in}}{\pgfqpoint{0.021649in}{-0.037276in}}{\pgfqpoint{0.029463in}{-0.029463in}}%
\pgfpathcurveto{\pgfqpoint{0.037276in}{-0.021649in}}{\pgfqpoint{0.041667in}{-0.011050in}}{\pgfqpoint{0.041667in}{0.000000in}}%
\pgfpathcurveto{\pgfqpoint{0.041667in}{0.011050in}}{\pgfqpoint{0.037276in}{0.021649in}}{\pgfqpoint{0.029463in}{0.029463in}}%
\pgfpathcurveto{\pgfqpoint{0.021649in}{0.037276in}}{\pgfqpoint{0.011050in}{0.041667in}}{\pgfqpoint{0.000000in}{0.041667in}}%
\pgfpathcurveto{\pgfqpoint{-0.011050in}{0.041667in}}{\pgfqpoint{-0.021649in}{0.037276in}}{\pgfqpoint{-0.029463in}{0.029463in}}%
\pgfpathcurveto{\pgfqpoint{-0.037276in}{0.021649in}}{\pgfqpoint{-0.041667in}{0.011050in}}{\pgfqpoint{-0.041667in}{0.000000in}}%
\pgfpathcurveto{\pgfqpoint{-0.041667in}{-0.011050in}}{\pgfqpoint{-0.037276in}{-0.021649in}}{\pgfqpoint{-0.029463in}{-0.029463in}}%
\pgfpathcurveto{\pgfqpoint{-0.021649in}{-0.037276in}}{\pgfqpoint{-0.011050in}{-0.041667in}}{\pgfqpoint{0.000000in}{-0.041667in}}%
\pgfpathclose%
\pgfusepath{stroke,fill}%
}%
\begin{pgfscope}%
\pgfsys@transformshift{10.045867in}{4.079781in}%
\pgfsys@useobject{currentmarker}{}%
\end{pgfscope}%
\end{pgfscope}%
\begin{pgfscope}%
\definecolor{textcolor}{rgb}{0.000000,0.000000,0.000000}%
\pgfsetstrokecolor{textcolor}%
\pgfsetfillcolor{textcolor}%
\pgftext[x=10.295867in,y=4.043322in,left,base]{\color{textcolor}\sffamily\fontsize{10.000000}{12.000000}\selectfont s2r3dfree\_background}%
\end{pgfscope}%
\begin{pgfscope}%
\pgfsetbuttcap%
\pgfsetroundjoin%
\definecolor{currentfill}{rgb}{0.815686,0.733333,1.000000}%
\pgfsetfillcolor{currentfill}%
\pgfsetlinewidth{1.003750pt}%
\definecolor{currentstroke}{rgb}{0.815686,0.733333,1.000000}%
\pgfsetstrokecolor{currentstroke}%
\pgfsetdash{}{0pt}%
\pgfsys@defobject{currentmarker}{\pgfqpoint{-0.041667in}{-0.041667in}}{\pgfqpoint{0.041667in}{0.041667in}}{%
\pgfpathmoveto{\pgfqpoint{0.000000in}{-0.041667in}}%
\pgfpathcurveto{\pgfqpoint{0.011050in}{-0.041667in}}{\pgfqpoint{0.021649in}{-0.037276in}}{\pgfqpoint{0.029463in}{-0.029463in}}%
\pgfpathcurveto{\pgfqpoint{0.037276in}{-0.021649in}}{\pgfqpoint{0.041667in}{-0.011050in}}{\pgfqpoint{0.041667in}{0.000000in}}%
\pgfpathcurveto{\pgfqpoint{0.041667in}{0.011050in}}{\pgfqpoint{0.037276in}{0.021649in}}{\pgfqpoint{0.029463in}{0.029463in}}%
\pgfpathcurveto{\pgfqpoint{0.021649in}{0.037276in}}{\pgfqpoint{0.011050in}{0.041667in}}{\pgfqpoint{0.000000in}{0.041667in}}%
\pgfpathcurveto{\pgfqpoint{-0.011050in}{0.041667in}}{\pgfqpoint{-0.021649in}{0.037276in}}{\pgfqpoint{-0.029463in}{0.029463in}}%
\pgfpathcurveto{\pgfqpoint{-0.037276in}{0.021649in}}{\pgfqpoint{-0.041667in}{0.011050in}}{\pgfqpoint{-0.041667in}{0.000000in}}%
\pgfpathcurveto{\pgfqpoint{-0.041667in}{-0.011050in}}{\pgfqpoint{-0.037276in}{-0.021649in}}{\pgfqpoint{-0.029463in}{-0.029463in}}%
\pgfpathcurveto{\pgfqpoint{-0.021649in}{-0.037276in}}{\pgfqpoint{-0.011050in}{-0.041667in}}{\pgfqpoint{0.000000in}{-0.041667in}}%
\pgfpathclose%
\pgfusepath{stroke,fill}%
}%
\begin{pgfscope}%
\pgfsys@transformshift{10.045867in}{3.872058in}%
\pgfsys@useobject{currentmarker}{}%
\end{pgfscope}%
\end{pgfscope}%
\begin{pgfscope}%
\definecolor{textcolor}{rgb}{0.000000,0.000000,0.000000}%
\pgfsetstrokecolor{textcolor}%
\pgfsetfillcolor{textcolor}%
\pgftext[x=10.295867in,y=3.835600in,left,base]{\color{textcolor}\sffamily\fontsize{10.000000}{12.000000}\selectfont s2r3dfree\_background\_light2}%
\end{pgfscope}%
\begin{pgfscope}%
\pgfsetbuttcap%
\pgfsetroundjoin%
\definecolor{currentfill}{rgb}{0.870588,0.733333,0.607843}%
\pgfsetfillcolor{currentfill}%
\pgfsetlinewidth{1.003750pt}%
\definecolor{currentstroke}{rgb}{0.870588,0.733333,0.607843}%
\pgfsetstrokecolor{currentstroke}%
\pgfsetdash{}{0pt}%
\pgfsys@defobject{currentmarker}{\pgfqpoint{-0.041667in}{-0.041667in}}{\pgfqpoint{0.041667in}{0.041667in}}{%
\pgfpathmoveto{\pgfqpoint{0.000000in}{-0.041667in}}%
\pgfpathcurveto{\pgfqpoint{0.011050in}{-0.041667in}}{\pgfqpoint{0.021649in}{-0.037276in}}{\pgfqpoint{0.029463in}{-0.029463in}}%
\pgfpathcurveto{\pgfqpoint{0.037276in}{-0.021649in}}{\pgfqpoint{0.041667in}{-0.011050in}}{\pgfqpoint{0.041667in}{0.000000in}}%
\pgfpathcurveto{\pgfqpoint{0.041667in}{0.011050in}}{\pgfqpoint{0.037276in}{0.021649in}}{\pgfqpoint{0.029463in}{0.029463in}}%
\pgfpathcurveto{\pgfqpoint{0.021649in}{0.037276in}}{\pgfqpoint{0.011050in}{0.041667in}}{\pgfqpoint{0.000000in}{0.041667in}}%
\pgfpathcurveto{\pgfqpoint{-0.011050in}{0.041667in}}{\pgfqpoint{-0.021649in}{0.037276in}}{\pgfqpoint{-0.029463in}{0.029463in}}%
\pgfpathcurveto{\pgfqpoint{-0.037276in}{0.021649in}}{\pgfqpoint{-0.041667in}{0.011050in}}{\pgfqpoint{-0.041667in}{0.000000in}}%
\pgfpathcurveto{\pgfqpoint{-0.041667in}{-0.011050in}}{\pgfqpoint{-0.037276in}{-0.021649in}}{\pgfqpoint{-0.029463in}{-0.029463in}}%
\pgfpathcurveto{\pgfqpoint{-0.021649in}{-0.037276in}}{\pgfqpoint{-0.011050in}{-0.041667in}}{\pgfqpoint{0.000000in}{-0.041667in}}%
\pgfpathclose%
\pgfusepath{stroke,fill}%
}%
\begin{pgfscope}%
\pgfsys@transformshift{10.045867in}{3.664335in}%
\pgfsys@useobject{currentmarker}{}%
\end{pgfscope}%
\end{pgfscope}%
\begin{pgfscope}%
\definecolor{textcolor}{rgb}{0.000000,0.000000,0.000000}%
\pgfsetstrokecolor{textcolor}%
\pgfsetfillcolor{textcolor}%
\pgftext[x=10.295867in,y=3.627877in,left,base]{\color{textcolor}\sffamily\fontsize{10.000000}{12.000000}\selectfont s2r3dfree\_chair}%
\end{pgfscope}%
\end{pgfpicture}%
\makeatother%
\endgroup%
}\\
    \caption[\gls{tsne} for Ablation Datasets]{\gls{tsne} visualization for images from individual synthetic chair dataset with different domain randomization parameter compared with Pix3D chair latent space.
        (Left to right, top to bottom) Textureless, Textureless with light, Textured, Textured with light, Multi-Object and Combined.}
    \label{fig:tsne per chair dataset}
\end{figure}

\autoref{fig:tsne per chair dataset} represents latent space for each variation of domain randomization for chairs compared to chairs from the real dataset(Pix3D).
Two hundred images were randomly sampled from each dataset and embedded using pre-trained \gls{vgg} as in \autoref{subsec:qualitative}.
The last \gls{tsne} representation in \autoref{fig:tsne per chair dataset} shows a combined latent space of all the randomized datasets for chairs.
We see that combined space is close to the real dataset space and spreads across the latent space, while individual parameters have limited spread.

In \autoref{tab:quantitative-dataset-comparison-chair-dataset}, we see the corresponding quantitative measure with \gls{fid} for 100 randomly chosen images from each dataset compared to Pix3D chair dataset.
We see that the textureless dataset has an \gls{fid} of 125.91.
When light is added to this dataset, the \gls{fid} increases to 136.78.
Similarly, for textured dataset \gls{fid} is 146.48 and with light added it increases to 155.76.
We establish that light plays a major role for the datasets to be similar, but since it is showing a negative impact, we need a better control over its parameters.
Multi-Object dataset has the best similarity with least \gls{fid} of 123.45, which reflects in its performance in \autoref{sec:ablation-study-on-chairs}.

\begin{table}[ht]
    \centering
    \begin{tabular}{|c |c |}
        \hline
        Dataset & \gls{fid} \\ [0.5ex]
        \hline\hline
        Textureless & 125.91 \\
        \hline
        Textureless with light & 136.78 \\
        \hline
        Textured  & 146.48 \\
        \hline
        Textured with light  & 155.76 \\
        \hline
        Multi-Object & 123.45 \\[1ex]
        \hline
    \end{tabular}
    \caption[\gls{fid} Comparison for Synthetic Chair Datasets with Real Dataset.]{Table represents quantitative - \gls{fid} measure to compare chair synthetic dataset distribution with the real dataset(Pix3D).
    Multi-Object dataset has the least \gls{fid} and hence is most similar to real dataset.}
    \label{tab:quantitative-dataset-comparison-chair-dataset}
\end{table}

\section{Baseline}\label{sec:baseline}

As mentioned in \autoref{subsec:pix2vox-and-pix2vox++}, Pix2Vox and Pix2Vox++ are the models which will act as the baselines for all the experiments.
For the dataset from \autoref{sec:datasets}, Pix3D is the real dataset and will be acting as the base dataset.
The models are also compared with and without 2D augmentation.
The 2D augmentations include Random Flip, Random Crop, Color Jitter, RandomPermuteRGB\@.
A performance test with no augmentation will explain the importance of 2D augmentation in the performance of the 3D reconstruction tasks.
\autoref{fig:baseline1} represents the performance of baseline models on different datasets, both with and without data augmentation.
For the validation step, we save two checkpoints for the best epoch.
One is validated with the real dataset, and the other is validated with the respective synthetic dataset.

The models are trained on 70\% of the data for the synthetic dataset, and 30\% is used for validation.
For testing, we use the same 30\% of real data from the Pix3D dataset.
For the real dataset, the models are trained on 70\% of data and validated/tested on the same 30\% used in testing models trained on synthetic data.

%\begin{figure}[ht]
%    \centering
%    \resizebox{0.49\linewidth}{0.5\linewidth}{%% Creator: Matplotlib, PGF backend
%%
%% To include the figure in your LaTeX document, write
%%   \input{<filename>.pgf}
%%
%% Make sure the required packages are loaded in your preamble
%%   \usepackage{pgf}
%%
%% Figures using additional raster images can only be included by \input if
%% they are in the same directory as the main LaTeX file. For loading figures
%% from other directories you can use the `import` package
%%   \usepackage{import}
%%
%% and then include the figures with
%%   \import{<path to file>}{<filename>.pgf}
%%
%% Matplotlib used the following preamble
%%   \usepackage{fontspec}
%%   \setmainfont{DejaVuSerif.ttf}[Path=\detokenize{/Users/apple/opt/anaconda3/envs/kaolin/lib/python3.7/site-packages/matplotlib/mpl-data/fonts/ttf/}]
%%   \setsansfont{DejaVuSans.ttf}[Path=\detokenize{/Users/apple/opt/anaconda3/envs/kaolin/lib/python3.7/site-packages/matplotlib/mpl-data/fonts/ttf/}]
%%   \setmonofont{DejaVuSansMono.ttf}[Path=\detokenize{/Users/apple/opt/anaconda3/envs/kaolin/lib/python3.7/site-packages/matplotlib/mpl-data/fonts/ttf/}]
%%
\begingroup%
\makeatletter%
\begin{pgfpicture}%
\pgfpathrectangle{\pgfpointorigin}{\pgfqpoint{6.293658in}{4.697602in}}%
\pgfusepath{use as bounding box, clip}%
\begin{pgfscope}%
\pgfsetbuttcap%
\pgfsetmiterjoin%
\definecolor{currentfill}{rgb}{1.000000,1.000000,1.000000}%
\pgfsetfillcolor{currentfill}%
\pgfsetlinewidth{0.000000pt}%
\definecolor{currentstroke}{rgb}{1.000000,1.000000,1.000000}%
\pgfsetstrokecolor{currentstroke}%
\pgfsetdash{}{0pt}%
\pgfpathmoveto{\pgfqpoint{-0.000000in}{0.000000in}}%
\pgfpathlineto{\pgfqpoint{6.293658in}{0.000000in}}%
\pgfpathlineto{\pgfqpoint{6.293658in}{4.697602in}}%
\pgfpathlineto{\pgfqpoint{-0.000000in}{4.697602in}}%
\pgfpathclose%
\pgfusepath{fill}%
\end{pgfscope}%
\begin{pgfscope}%
\pgfsetbuttcap%
\pgfsetmiterjoin%
\definecolor{currentfill}{rgb}{1.000000,1.000000,1.000000}%
\pgfsetfillcolor{currentfill}%
\pgfsetlinewidth{0.000000pt}%
\definecolor{currentstroke}{rgb}{0.000000,0.000000,0.000000}%
\pgfsetstrokecolor{currentstroke}%
\pgfsetstrokeopacity{0.000000}%
\pgfsetdash{}{0pt}%
\pgfpathmoveto{\pgfqpoint{0.696435in}{1.223552in}}%
\pgfpathlineto{\pgfqpoint{6.193658in}{1.223552in}}%
\pgfpathlineto{\pgfqpoint{6.193658in}{4.387641in}}%
\pgfpathlineto{\pgfqpoint{0.696435in}{4.387641in}}%
\pgfpathclose%
\pgfusepath{fill}%
\end{pgfscope}%
\begin{pgfscope}%
\pgfpathrectangle{\pgfqpoint{0.696435in}{1.223552in}}{\pgfqpoint{5.497222in}{3.164089in}}%
\pgfusepath{clip}%
\pgfsetbuttcap%
\pgfsetmiterjoin%
\definecolor{currentfill}{rgb}{0.121569,0.466667,0.705882}%
\pgfsetfillcolor{currentfill}%
\pgfsetlinewidth{0.000000pt}%
\definecolor{currentstroke}{rgb}{0.000000,0.000000,0.000000}%
\pgfsetstrokecolor{currentstroke}%
\pgfsetstrokeopacity{0.000000}%
\pgfsetdash{}{0pt}%
\pgfpathmoveto{\pgfqpoint{0.946309in}{1.223552in}}%
\pgfpathlineto{\pgfqpoint{1.405261in}{1.223552in}}%
\pgfpathlineto{\pgfqpoint{1.405261in}{3.377505in}}%
\pgfpathlineto{\pgfqpoint{0.946309in}{3.377505in}}%
\pgfpathclose%
\pgfusepath{fill}%
\end{pgfscope}%
\begin{pgfscope}%
\pgfpathrectangle{\pgfqpoint{0.696435in}{1.223552in}}{\pgfqpoint{5.497222in}{3.164089in}}%
\pgfusepath{clip}%
\pgfsetbuttcap%
\pgfsetmiterjoin%
\definecolor{currentfill}{rgb}{0.121569,0.466667,0.705882}%
\pgfsetfillcolor{currentfill}%
\pgfsetlinewidth{0.000000pt}%
\definecolor{currentstroke}{rgb}{0.000000,0.000000,0.000000}%
\pgfsetstrokecolor{currentstroke}%
\pgfsetstrokeopacity{0.000000}%
\pgfsetdash{}{0pt}%
\pgfpathmoveto{\pgfqpoint{1.966202in}{1.223552in}}%
\pgfpathlineto{\pgfqpoint{2.425154in}{1.223552in}}%
\pgfpathlineto{\pgfqpoint{2.425154in}{4.015069in}}%
\pgfpathlineto{\pgfqpoint{1.966202in}{4.015069in}}%
\pgfpathclose%
\pgfusepath{fill}%
\end{pgfscope}%
\begin{pgfscope}%
\pgfpathrectangle{\pgfqpoint{0.696435in}{1.223552in}}{\pgfqpoint{5.497222in}{3.164089in}}%
\pgfusepath{clip}%
\pgfsetbuttcap%
\pgfsetmiterjoin%
\definecolor{currentfill}{rgb}{0.121569,0.466667,0.705882}%
\pgfsetfillcolor{currentfill}%
\pgfsetlinewidth{0.000000pt}%
\definecolor{currentstroke}{rgb}{0.000000,0.000000,0.000000}%
\pgfsetstrokecolor{currentstroke}%
\pgfsetstrokeopacity{0.000000}%
\pgfsetdash{}{0pt}%
\pgfpathmoveto{\pgfqpoint{2.986095in}{1.223552in}}%
\pgfpathlineto{\pgfqpoint{3.445047in}{1.223552in}}%
\pgfpathlineto{\pgfqpoint{3.445047in}{2.599139in}}%
\pgfpathlineto{\pgfqpoint{2.986095in}{2.599139in}}%
\pgfpathclose%
\pgfusepath{fill}%
\end{pgfscope}%
\begin{pgfscope}%
\pgfpathrectangle{\pgfqpoint{0.696435in}{1.223552in}}{\pgfqpoint{5.497222in}{3.164089in}}%
\pgfusepath{clip}%
\pgfsetbuttcap%
\pgfsetmiterjoin%
\definecolor{currentfill}{rgb}{0.121569,0.466667,0.705882}%
\pgfsetfillcolor{currentfill}%
\pgfsetlinewidth{0.000000pt}%
\definecolor{currentstroke}{rgb}{0.000000,0.000000,0.000000}%
\pgfsetstrokecolor{currentstroke}%
\pgfsetstrokeopacity{0.000000}%
\pgfsetdash{}{0pt}%
\pgfpathmoveto{\pgfqpoint{4.005988in}{1.223552in}}%
\pgfpathlineto{\pgfqpoint{4.464939in}{1.223552in}}%
\pgfpathlineto{\pgfqpoint{4.464939in}{2.715420in}}%
\pgfpathlineto{\pgfqpoint{4.005988in}{2.715420in}}%
\pgfpathclose%
\pgfusepath{fill}%
\end{pgfscope}%
\begin{pgfscope}%
\pgfpathrectangle{\pgfqpoint{0.696435in}{1.223552in}}{\pgfqpoint{5.497222in}{3.164089in}}%
\pgfusepath{clip}%
\pgfsetbuttcap%
\pgfsetmiterjoin%
\definecolor{currentfill}{rgb}{0.121569,0.466667,0.705882}%
\pgfsetfillcolor{currentfill}%
\pgfsetlinewidth{0.000000pt}%
\definecolor{currentstroke}{rgb}{0.000000,0.000000,0.000000}%
\pgfsetstrokecolor{currentstroke}%
\pgfsetstrokeopacity{0.000000}%
\pgfsetdash{}{0pt}%
\pgfpathmoveto{\pgfqpoint{5.025880in}{1.223552in}}%
\pgfpathlineto{\pgfqpoint{5.484832in}{1.223552in}}%
\pgfpathlineto{\pgfqpoint{5.484832in}{2.723330in}}%
\pgfpathlineto{\pgfqpoint{5.025880in}{2.723330in}}%
\pgfpathclose%
\pgfusepath{fill}%
\end{pgfscope}%
\begin{pgfscope}%
\pgfpathrectangle{\pgfqpoint{0.696435in}{1.223552in}}{\pgfqpoint{5.497222in}{3.164089in}}%
\pgfusepath{clip}%
\pgfsetbuttcap%
\pgfsetmiterjoin%
\definecolor{currentfill}{rgb}{1.000000,0.498039,0.054902}%
\pgfsetfillcolor{currentfill}%
\pgfsetlinewidth{0.000000pt}%
\definecolor{currentstroke}{rgb}{0.000000,0.000000,0.000000}%
\pgfsetstrokecolor{currentstroke}%
\pgfsetstrokeopacity{0.000000}%
\pgfsetdash{}{0pt}%
\pgfpathmoveto{\pgfqpoint{1.405261in}{1.223552in}}%
\pgfpathlineto{\pgfqpoint{1.864213in}{1.223552in}}%
\pgfpathlineto{\pgfqpoint{1.864213in}{3.682049in}}%
\pgfpathlineto{\pgfqpoint{1.405261in}{3.682049in}}%
\pgfpathclose%
\pgfusepath{fill}%
\end{pgfscope}%
\begin{pgfscope}%
\pgfpathrectangle{\pgfqpoint{0.696435in}{1.223552in}}{\pgfqpoint{5.497222in}{3.164089in}}%
\pgfusepath{clip}%
\pgfsetbuttcap%
\pgfsetmiterjoin%
\definecolor{currentfill}{rgb}{1.000000,0.498039,0.054902}%
\pgfsetfillcolor{currentfill}%
\pgfsetlinewidth{0.000000pt}%
\definecolor{currentstroke}{rgb}{0.000000,0.000000,0.000000}%
\pgfsetstrokecolor{currentstroke}%
\pgfsetstrokeopacity{0.000000}%
\pgfsetdash{}{0pt}%
\pgfpathmoveto{\pgfqpoint{2.425154in}{1.223552in}}%
\pgfpathlineto{\pgfqpoint{2.884105in}{1.223552in}}%
\pgfpathlineto{\pgfqpoint{2.884105in}{3.778554in}}%
\pgfpathlineto{\pgfqpoint{2.425154in}{3.778554in}}%
\pgfpathclose%
\pgfusepath{fill}%
\end{pgfscope}%
\begin{pgfscope}%
\pgfpathrectangle{\pgfqpoint{0.696435in}{1.223552in}}{\pgfqpoint{5.497222in}{3.164089in}}%
\pgfusepath{clip}%
\pgfsetbuttcap%
\pgfsetmiterjoin%
\definecolor{currentfill}{rgb}{1.000000,0.498039,0.054902}%
\pgfsetfillcolor{currentfill}%
\pgfsetlinewidth{0.000000pt}%
\definecolor{currentstroke}{rgb}{0.000000,0.000000,0.000000}%
\pgfsetstrokecolor{currentstroke}%
\pgfsetstrokeopacity{0.000000}%
\pgfsetdash{}{0pt}%
\pgfpathmoveto{\pgfqpoint{3.445047in}{1.223552in}}%
\pgfpathlineto{\pgfqpoint{3.903998in}{1.223552in}}%
\pgfpathlineto{\pgfqpoint{3.903998in}{2.243179in}}%
\pgfpathlineto{\pgfqpoint{3.445047in}{2.243179in}}%
\pgfpathclose%
\pgfusepath{fill}%
\end{pgfscope}%
\begin{pgfscope}%
\pgfpathrectangle{\pgfqpoint{0.696435in}{1.223552in}}{\pgfqpoint{5.497222in}{3.164089in}}%
\pgfusepath{clip}%
\pgfsetbuttcap%
\pgfsetmiterjoin%
\definecolor{currentfill}{rgb}{1.000000,0.498039,0.054902}%
\pgfsetfillcolor{currentfill}%
\pgfsetlinewidth{0.000000pt}%
\definecolor{currentstroke}{rgb}{0.000000,0.000000,0.000000}%
\pgfsetstrokecolor{currentstroke}%
\pgfsetstrokeopacity{0.000000}%
\pgfsetdash{}{0pt}%
\pgfpathmoveto{\pgfqpoint{4.464939in}{1.223552in}}%
\pgfpathlineto{\pgfqpoint{4.923891in}{1.223552in}}%
\pgfpathlineto{\pgfqpoint{4.923891in}{2.707509in}}%
\pgfpathlineto{\pgfqpoint{4.464939in}{2.707509in}}%
\pgfpathclose%
\pgfusepath{fill}%
\end{pgfscope}%
\begin{pgfscope}%
\pgfpathrectangle{\pgfqpoint{0.696435in}{1.223552in}}{\pgfqpoint{5.497222in}{3.164089in}}%
\pgfusepath{clip}%
\pgfsetbuttcap%
\pgfsetmiterjoin%
\definecolor{currentfill}{rgb}{1.000000,0.498039,0.054902}%
\pgfsetfillcolor{currentfill}%
\pgfsetlinewidth{0.000000pt}%
\definecolor{currentstroke}{rgb}{0.000000,0.000000,0.000000}%
\pgfsetstrokecolor{currentstroke}%
\pgfsetstrokeopacity{0.000000}%
\pgfsetdash{}{0pt}%
\pgfpathmoveto{\pgfqpoint{5.484832in}{1.223552in}}%
\pgfpathlineto{\pgfqpoint{5.943784in}{1.223552in}}%
\pgfpathlineto{\pgfqpoint{5.943784in}{2.691689in}}%
\pgfpathlineto{\pgfqpoint{5.484832in}{2.691689in}}%
\pgfpathclose%
\pgfusepath{fill}%
\end{pgfscope}%
\begin{pgfscope}%
\pgfsetbuttcap%
\pgfsetroundjoin%
\definecolor{currentfill}{rgb}{0.000000,0.000000,0.000000}%
\pgfsetfillcolor{currentfill}%
\pgfsetlinewidth{0.803000pt}%
\definecolor{currentstroke}{rgb}{0.000000,0.000000,0.000000}%
\pgfsetstrokecolor{currentstroke}%
\pgfsetdash{}{0pt}%
\pgfsys@defobject{currentmarker}{\pgfqpoint{0.000000in}{-0.048611in}}{\pgfqpoint{0.000000in}{0.000000in}}{%
\pgfpathmoveto{\pgfqpoint{0.000000in}{0.000000in}}%
\pgfpathlineto{\pgfqpoint{0.000000in}{-0.048611in}}%
\pgfusepath{stroke,fill}%
}%
\begin{pgfscope}%
\pgfsys@transformshift{1.405261in}{1.223552in}%
\pgfsys@useobject{currentmarker}{}%
\end{pgfscope}%
\end{pgfscope}%
\begin{pgfscope}%
\definecolor{textcolor}{rgb}{0.000000,0.000000,0.000000}%
\pgfsetstrokecolor{textcolor}%
\pgfsetfillcolor{textcolor}%
\pgftext[x=1.091236in, y=0.369475in, left, base,rotate=45.000000]{\color{textcolor}\sffamily\fontsize{10.000000}{12.000000}\selectfont Pix3d(no aug)}%
\end{pgfscope}%
\begin{pgfscope}%
\pgfsetbuttcap%
\pgfsetroundjoin%
\definecolor{currentfill}{rgb}{0.000000,0.000000,0.000000}%
\pgfsetfillcolor{currentfill}%
\pgfsetlinewidth{0.803000pt}%
\definecolor{currentstroke}{rgb}{0.000000,0.000000,0.000000}%
\pgfsetstrokecolor{currentstroke}%
\pgfsetdash{}{0pt}%
\pgfsys@defobject{currentmarker}{\pgfqpoint{0.000000in}{-0.048611in}}{\pgfqpoint{0.000000in}{0.000000in}}{%
\pgfpathmoveto{\pgfqpoint{0.000000in}{0.000000in}}%
\pgfpathlineto{\pgfqpoint{0.000000in}{-0.048611in}}%
\pgfusepath{stroke,fill}%
}%
\begin{pgfscope}%
\pgfsys@transformshift{2.425154in}{1.223552in}%
\pgfsys@useobject{currentmarker}{}%
\end{pgfscope}%
\end{pgfscope}%
\begin{pgfscope}%
\definecolor{textcolor}{rgb}{0.000000,0.000000,0.000000}%
\pgfsetstrokecolor{textcolor}%
\pgfsetfillcolor{textcolor}%
\pgftext[x=2.318601in, y=0.784419in, left, base,rotate=45.000000]{\color{textcolor}\sffamily\fontsize{10.000000}{12.000000}\selectfont Pix3d}%
\end{pgfscope}%
\begin{pgfscope}%
\pgfsetbuttcap%
\pgfsetroundjoin%
\definecolor{currentfill}{rgb}{0.000000,0.000000,0.000000}%
\pgfsetfillcolor{currentfill}%
\pgfsetlinewidth{0.803000pt}%
\definecolor{currentstroke}{rgb}{0.000000,0.000000,0.000000}%
\pgfsetstrokecolor{currentstroke}%
\pgfsetdash{}{0pt}%
\pgfsys@defobject{currentmarker}{\pgfqpoint{0.000000in}{-0.048611in}}{\pgfqpoint{0.000000in}{0.000000in}}{%
\pgfpathmoveto{\pgfqpoint{0.000000in}{0.000000in}}%
\pgfpathlineto{\pgfqpoint{0.000000in}{-0.048611in}}%
\pgfusepath{stroke,fill}%
}%
\begin{pgfscope}%
\pgfsys@transformshift{3.445047in}{1.223552in}%
\pgfsys@useobject{currentmarker}{}%
\end{pgfscope}%
\end{pgfscope}%
\begin{pgfscope}%
\definecolor{textcolor}{rgb}{0.000000,0.000000,0.000000}%
\pgfsetstrokecolor{textcolor}%
\pgfsetfillcolor{textcolor}%
\pgftext[x=3.101386in, y=0.313130in, left, base,rotate=45.000000]{\color{textcolor}\sffamily\fontsize{10.000000}{12.000000}\selectfont s2r\_v1(no aug)}%
\end{pgfscope}%
\begin{pgfscope}%
\pgfsetbuttcap%
\pgfsetroundjoin%
\definecolor{currentfill}{rgb}{0.000000,0.000000,0.000000}%
\pgfsetfillcolor{currentfill}%
\pgfsetlinewidth{0.803000pt}%
\definecolor{currentstroke}{rgb}{0.000000,0.000000,0.000000}%
\pgfsetstrokecolor{currentstroke}%
\pgfsetdash{}{0pt}%
\pgfsys@defobject{currentmarker}{\pgfqpoint{0.000000in}{-0.048611in}}{\pgfqpoint{0.000000in}{0.000000in}}{%
\pgfpathmoveto{\pgfqpoint{0.000000in}{0.000000in}}%
\pgfpathlineto{\pgfqpoint{0.000000in}{-0.048611in}}%
\pgfusepath{stroke,fill}%
}%
\begin{pgfscope}%
\pgfsys@transformshift{4.464939in}{1.223552in}%
\pgfsys@useobject{currentmarker}{}%
\end{pgfscope}%
\end{pgfscope}%
\begin{pgfscope}%
\definecolor{textcolor}{rgb}{0.000000,0.000000,0.000000}%
\pgfsetstrokecolor{textcolor}%
\pgfsetfillcolor{textcolor}%
\pgftext[x=4.327936in, y=0.729704in, left, base,rotate=45.000000]{\color{textcolor}\sffamily\fontsize{10.000000}{12.000000}\selectfont s2r\_v1}%
\end{pgfscope}%
\begin{pgfscope}%
\pgfsetbuttcap%
\pgfsetroundjoin%
\definecolor{currentfill}{rgb}{0.000000,0.000000,0.000000}%
\pgfsetfillcolor{currentfill}%
\pgfsetlinewidth{0.803000pt}%
\definecolor{currentstroke}{rgb}{0.000000,0.000000,0.000000}%
\pgfsetstrokecolor{currentstroke}%
\pgfsetdash{}{0pt}%
\pgfsys@defobject{currentmarker}{\pgfqpoint{0.000000in}{-0.048611in}}{\pgfqpoint{0.000000in}{0.000000in}}{%
\pgfpathmoveto{\pgfqpoint{0.000000in}{0.000000in}}%
\pgfpathlineto{\pgfqpoint{0.000000in}{-0.048611in}}%
\pgfusepath{stroke,fill}%
}%
\begin{pgfscope}%
\pgfsys@transformshift{5.484832in}{1.223552in}%
\pgfsys@useobject{currentmarker}{}%
\end{pgfscope}%
\end{pgfscope}%
\begin{pgfscope}%
\definecolor{textcolor}{rgb}{0.000000,0.000000,0.000000}%
\pgfsetstrokecolor{textcolor}%
\pgfsetfillcolor{textcolor}%
\pgftext[x=5.347828in, y=0.729704in, left, base,rotate=45.000000]{\color{textcolor}\sffamily\fontsize{10.000000}{12.000000}\selectfont s2r\_v2}%
\end{pgfscope}%
\begin{pgfscope}%
\definecolor{textcolor}{rgb}{0.000000,0.000000,0.000000}%
\pgfsetstrokecolor{textcolor}%
\pgfsetfillcolor{textcolor}%
\pgftext[x=3.445047in,y=0.234413in,,top]{\color{textcolor}\sffamily\fontsize{10.000000}{12.000000}\selectfont Dataset}%
\end{pgfscope}%
\begin{pgfscope}%
\pgfsetbuttcap%
\pgfsetroundjoin%
\definecolor{currentfill}{rgb}{0.000000,0.000000,0.000000}%
\pgfsetfillcolor{currentfill}%
\pgfsetlinewidth{0.803000pt}%
\definecolor{currentstroke}{rgb}{0.000000,0.000000,0.000000}%
\pgfsetstrokecolor{currentstroke}%
\pgfsetdash{}{0pt}%
\pgfsys@defobject{currentmarker}{\pgfqpoint{-0.048611in}{0.000000in}}{\pgfqpoint{-0.000000in}{0.000000in}}{%
\pgfpathmoveto{\pgfqpoint{-0.000000in}{0.000000in}}%
\pgfpathlineto{\pgfqpoint{-0.048611in}{0.000000in}}%
\pgfusepath{stroke,fill}%
}%
\begin{pgfscope}%
\pgfsys@transformshift{0.696435in}{1.223552in}%
\pgfsys@useobject{currentmarker}{}%
\end{pgfscope}%
\end{pgfscope}%
\begin{pgfscope}%
\definecolor{textcolor}{rgb}{0.000000,0.000000,0.000000}%
\pgfsetstrokecolor{textcolor}%
\pgfsetfillcolor{textcolor}%
\pgftext[x=0.289968in, y=1.170790in, left, base]{\color{textcolor}\sffamily\fontsize{10.000000}{12.000000}\selectfont 0.00}%
\end{pgfscope}%
\begin{pgfscope}%
\pgfsetbuttcap%
\pgfsetroundjoin%
\definecolor{currentfill}{rgb}{0.000000,0.000000,0.000000}%
\pgfsetfillcolor{currentfill}%
\pgfsetlinewidth{0.803000pt}%
\definecolor{currentstroke}{rgb}{0.000000,0.000000,0.000000}%
\pgfsetstrokecolor{currentstroke}%
\pgfsetdash{}{0pt}%
\pgfsys@defobject{currentmarker}{\pgfqpoint{-0.048611in}{0.000000in}}{\pgfqpoint{-0.000000in}{0.000000in}}{%
\pgfpathmoveto{\pgfqpoint{-0.000000in}{0.000000in}}%
\pgfpathlineto{\pgfqpoint{-0.048611in}{0.000000in}}%
\pgfusepath{stroke,fill}%
}%
\begin{pgfscope}%
\pgfsys@transformshift{0.696435in}{1.619063in}%
\pgfsys@useobject{currentmarker}{}%
\end{pgfscope}%
\end{pgfscope}%
\begin{pgfscope}%
\definecolor{textcolor}{rgb}{0.000000,0.000000,0.000000}%
\pgfsetstrokecolor{textcolor}%
\pgfsetfillcolor{textcolor}%
\pgftext[x=0.289968in, y=1.566301in, left, base]{\color{textcolor}\sffamily\fontsize{10.000000}{12.000000}\selectfont 0.05}%
\end{pgfscope}%
\begin{pgfscope}%
\pgfsetbuttcap%
\pgfsetroundjoin%
\definecolor{currentfill}{rgb}{0.000000,0.000000,0.000000}%
\pgfsetfillcolor{currentfill}%
\pgfsetlinewidth{0.803000pt}%
\definecolor{currentstroke}{rgb}{0.000000,0.000000,0.000000}%
\pgfsetstrokecolor{currentstroke}%
\pgfsetdash{}{0pt}%
\pgfsys@defobject{currentmarker}{\pgfqpoint{-0.048611in}{0.000000in}}{\pgfqpoint{-0.000000in}{0.000000in}}{%
\pgfpathmoveto{\pgfqpoint{-0.000000in}{0.000000in}}%
\pgfpathlineto{\pgfqpoint{-0.048611in}{0.000000in}}%
\pgfusepath{stroke,fill}%
}%
\begin{pgfscope}%
\pgfsys@transformshift{0.696435in}{2.014574in}%
\pgfsys@useobject{currentmarker}{}%
\end{pgfscope}%
\end{pgfscope}%
\begin{pgfscope}%
\definecolor{textcolor}{rgb}{0.000000,0.000000,0.000000}%
\pgfsetstrokecolor{textcolor}%
\pgfsetfillcolor{textcolor}%
\pgftext[x=0.289968in, y=1.961812in, left, base]{\color{textcolor}\sffamily\fontsize{10.000000}{12.000000}\selectfont 0.10}%
\end{pgfscope}%
\begin{pgfscope}%
\pgfsetbuttcap%
\pgfsetroundjoin%
\definecolor{currentfill}{rgb}{0.000000,0.000000,0.000000}%
\pgfsetfillcolor{currentfill}%
\pgfsetlinewidth{0.803000pt}%
\definecolor{currentstroke}{rgb}{0.000000,0.000000,0.000000}%
\pgfsetstrokecolor{currentstroke}%
\pgfsetdash{}{0pt}%
\pgfsys@defobject{currentmarker}{\pgfqpoint{-0.048611in}{0.000000in}}{\pgfqpoint{-0.000000in}{0.000000in}}{%
\pgfpathmoveto{\pgfqpoint{-0.000000in}{0.000000in}}%
\pgfpathlineto{\pgfqpoint{-0.048611in}{0.000000in}}%
\pgfusepath{stroke,fill}%
}%
\begin{pgfscope}%
\pgfsys@transformshift{0.696435in}{2.410085in}%
\pgfsys@useobject{currentmarker}{}%
\end{pgfscope}%
\end{pgfscope}%
\begin{pgfscope}%
\definecolor{textcolor}{rgb}{0.000000,0.000000,0.000000}%
\pgfsetstrokecolor{textcolor}%
\pgfsetfillcolor{textcolor}%
\pgftext[x=0.289968in, y=2.357324in, left, base]{\color{textcolor}\sffamily\fontsize{10.000000}{12.000000}\selectfont 0.15}%
\end{pgfscope}%
\begin{pgfscope}%
\pgfsetbuttcap%
\pgfsetroundjoin%
\definecolor{currentfill}{rgb}{0.000000,0.000000,0.000000}%
\pgfsetfillcolor{currentfill}%
\pgfsetlinewidth{0.803000pt}%
\definecolor{currentstroke}{rgb}{0.000000,0.000000,0.000000}%
\pgfsetstrokecolor{currentstroke}%
\pgfsetdash{}{0pt}%
\pgfsys@defobject{currentmarker}{\pgfqpoint{-0.048611in}{0.000000in}}{\pgfqpoint{-0.000000in}{0.000000in}}{%
\pgfpathmoveto{\pgfqpoint{-0.000000in}{0.000000in}}%
\pgfpathlineto{\pgfqpoint{-0.048611in}{0.000000in}}%
\pgfusepath{stroke,fill}%
}%
\begin{pgfscope}%
\pgfsys@transformshift{0.696435in}{2.805596in}%
\pgfsys@useobject{currentmarker}{}%
\end{pgfscope}%
\end{pgfscope}%
\begin{pgfscope}%
\definecolor{textcolor}{rgb}{0.000000,0.000000,0.000000}%
\pgfsetstrokecolor{textcolor}%
\pgfsetfillcolor{textcolor}%
\pgftext[x=0.289968in, y=2.752835in, left, base]{\color{textcolor}\sffamily\fontsize{10.000000}{12.000000}\selectfont 0.20}%
\end{pgfscope}%
\begin{pgfscope}%
\pgfsetbuttcap%
\pgfsetroundjoin%
\definecolor{currentfill}{rgb}{0.000000,0.000000,0.000000}%
\pgfsetfillcolor{currentfill}%
\pgfsetlinewidth{0.803000pt}%
\definecolor{currentstroke}{rgb}{0.000000,0.000000,0.000000}%
\pgfsetstrokecolor{currentstroke}%
\pgfsetdash{}{0pt}%
\pgfsys@defobject{currentmarker}{\pgfqpoint{-0.048611in}{0.000000in}}{\pgfqpoint{-0.000000in}{0.000000in}}{%
\pgfpathmoveto{\pgfqpoint{-0.000000in}{0.000000in}}%
\pgfpathlineto{\pgfqpoint{-0.048611in}{0.000000in}}%
\pgfusepath{stroke,fill}%
}%
\begin{pgfscope}%
\pgfsys@transformshift{0.696435in}{3.201107in}%
\pgfsys@useobject{currentmarker}{}%
\end{pgfscope}%
\end{pgfscope}%
\begin{pgfscope}%
\definecolor{textcolor}{rgb}{0.000000,0.000000,0.000000}%
\pgfsetstrokecolor{textcolor}%
\pgfsetfillcolor{textcolor}%
\pgftext[x=0.289968in, y=3.148346in, left, base]{\color{textcolor}\sffamily\fontsize{10.000000}{12.000000}\selectfont 0.25}%
\end{pgfscope}%
\begin{pgfscope}%
\pgfsetbuttcap%
\pgfsetroundjoin%
\definecolor{currentfill}{rgb}{0.000000,0.000000,0.000000}%
\pgfsetfillcolor{currentfill}%
\pgfsetlinewidth{0.803000pt}%
\definecolor{currentstroke}{rgb}{0.000000,0.000000,0.000000}%
\pgfsetstrokecolor{currentstroke}%
\pgfsetdash{}{0pt}%
\pgfsys@defobject{currentmarker}{\pgfqpoint{-0.048611in}{0.000000in}}{\pgfqpoint{-0.000000in}{0.000000in}}{%
\pgfpathmoveto{\pgfqpoint{-0.000000in}{0.000000in}}%
\pgfpathlineto{\pgfqpoint{-0.048611in}{0.000000in}}%
\pgfusepath{stroke,fill}%
}%
\begin{pgfscope}%
\pgfsys@transformshift{0.696435in}{3.596618in}%
\pgfsys@useobject{currentmarker}{}%
\end{pgfscope}%
\end{pgfscope}%
\begin{pgfscope}%
\definecolor{textcolor}{rgb}{0.000000,0.000000,0.000000}%
\pgfsetstrokecolor{textcolor}%
\pgfsetfillcolor{textcolor}%
\pgftext[x=0.289968in, y=3.543857in, left, base]{\color{textcolor}\sffamily\fontsize{10.000000}{12.000000}\selectfont 0.30}%
\end{pgfscope}%
\begin{pgfscope}%
\pgfsetbuttcap%
\pgfsetroundjoin%
\definecolor{currentfill}{rgb}{0.000000,0.000000,0.000000}%
\pgfsetfillcolor{currentfill}%
\pgfsetlinewidth{0.803000pt}%
\definecolor{currentstroke}{rgb}{0.000000,0.000000,0.000000}%
\pgfsetstrokecolor{currentstroke}%
\pgfsetdash{}{0pt}%
\pgfsys@defobject{currentmarker}{\pgfqpoint{-0.048611in}{0.000000in}}{\pgfqpoint{-0.000000in}{0.000000in}}{%
\pgfpathmoveto{\pgfqpoint{-0.000000in}{0.000000in}}%
\pgfpathlineto{\pgfqpoint{-0.048611in}{0.000000in}}%
\pgfusepath{stroke,fill}%
}%
\begin{pgfscope}%
\pgfsys@transformshift{0.696435in}{3.992130in}%
\pgfsys@useobject{currentmarker}{}%
\end{pgfscope}%
\end{pgfscope}%
\begin{pgfscope}%
\definecolor{textcolor}{rgb}{0.000000,0.000000,0.000000}%
\pgfsetstrokecolor{textcolor}%
\pgfsetfillcolor{textcolor}%
\pgftext[x=0.289968in, y=3.939368in, left, base]{\color{textcolor}\sffamily\fontsize{10.000000}{12.000000}\selectfont 0.35}%
\end{pgfscope}%
\begin{pgfscope}%
\pgfsetbuttcap%
\pgfsetroundjoin%
\definecolor{currentfill}{rgb}{0.000000,0.000000,0.000000}%
\pgfsetfillcolor{currentfill}%
\pgfsetlinewidth{0.803000pt}%
\definecolor{currentstroke}{rgb}{0.000000,0.000000,0.000000}%
\pgfsetstrokecolor{currentstroke}%
\pgfsetdash{}{0pt}%
\pgfsys@defobject{currentmarker}{\pgfqpoint{-0.048611in}{0.000000in}}{\pgfqpoint{-0.000000in}{0.000000in}}{%
\pgfpathmoveto{\pgfqpoint{-0.000000in}{0.000000in}}%
\pgfpathlineto{\pgfqpoint{-0.048611in}{0.000000in}}%
\pgfusepath{stroke,fill}%
}%
\begin{pgfscope}%
\pgfsys@transformshift{0.696435in}{4.387641in}%
\pgfsys@useobject{currentmarker}{}%
\end{pgfscope}%
\end{pgfscope}%
\begin{pgfscope}%
\definecolor{textcolor}{rgb}{0.000000,0.000000,0.000000}%
\pgfsetstrokecolor{textcolor}%
\pgfsetfillcolor{textcolor}%
\pgftext[x=0.289968in, y=4.334879in, left, base]{\color{textcolor}\sffamily\fontsize{10.000000}{12.000000}\selectfont 0.40}%
\end{pgfscope}%
\begin{pgfscope}%
\definecolor{textcolor}{rgb}{0.000000,0.000000,0.000000}%
\pgfsetstrokecolor{textcolor}%
\pgfsetfillcolor{textcolor}%
\pgftext[x=0.234413in,y=2.805596in,,bottom,rotate=90.000000]{\color{textcolor}\sffamily\fontsize{10.000000}{12.000000}\selectfont IoU}%
\end{pgfscope}%
\begin{pgfscope}%
\pgfsetrectcap%
\pgfsetmiterjoin%
\pgfsetlinewidth{0.803000pt}%
\definecolor{currentstroke}{rgb}{0.000000,0.000000,0.000000}%
\pgfsetstrokecolor{currentstroke}%
\pgfsetdash{}{0pt}%
\pgfpathmoveto{\pgfqpoint{0.696435in}{1.223552in}}%
\pgfpathlineto{\pgfqpoint{0.696435in}{4.387641in}}%
\pgfusepath{stroke}%
\end{pgfscope}%
\begin{pgfscope}%
\pgfsetrectcap%
\pgfsetmiterjoin%
\pgfsetlinewidth{0.803000pt}%
\definecolor{currentstroke}{rgb}{0.000000,0.000000,0.000000}%
\pgfsetstrokecolor{currentstroke}%
\pgfsetdash{}{0pt}%
\pgfpathmoveto{\pgfqpoint{6.193658in}{1.223552in}}%
\pgfpathlineto{\pgfqpoint{6.193658in}{4.387641in}}%
\pgfusepath{stroke}%
\end{pgfscope}%
\begin{pgfscope}%
\pgfsetrectcap%
\pgfsetmiterjoin%
\pgfsetlinewidth{0.803000pt}%
\definecolor{currentstroke}{rgb}{0.000000,0.000000,0.000000}%
\pgfsetstrokecolor{currentstroke}%
\pgfsetdash{}{0pt}%
\pgfpathmoveto{\pgfqpoint{0.696435in}{1.223552in}}%
\pgfpathlineto{\pgfqpoint{6.193658in}{1.223552in}}%
\pgfusepath{stroke}%
\end{pgfscope}%
\begin{pgfscope}%
\pgfsetrectcap%
\pgfsetmiterjoin%
\pgfsetlinewidth{0.803000pt}%
\definecolor{currentstroke}{rgb}{0.000000,0.000000,0.000000}%
\pgfsetstrokecolor{currentstroke}%
\pgfsetdash{}{0pt}%
\pgfpathmoveto{\pgfqpoint{0.696435in}{4.387641in}}%
\pgfpathlineto{\pgfqpoint{6.193658in}{4.387641in}}%
\pgfusepath{stroke}%
\end{pgfscope}%
\begin{pgfscope}%
\definecolor{textcolor}{rgb}{0.000000,0.000000,0.000000}%
\pgfsetstrokecolor{textcolor}%
\pgfsetfillcolor{textcolor}%
\pgftext[x=1.175785in,y=3.419172in,,bottom]{\color{textcolor}\sffamily\fontsize{9.000000}{10.800000}\selectfont 0.2723}%
\end{pgfscope}%
\begin{pgfscope}%
\definecolor{textcolor}{rgb}{0.000000,0.000000,0.000000}%
\pgfsetstrokecolor{textcolor}%
\pgfsetfillcolor{textcolor}%
\pgftext[x=2.195678in,y=4.056736in,,bottom]{\color{textcolor}\sffamily\fontsize{9.000000}{10.800000}\selectfont 0.3529}%
\end{pgfscope}%
\begin{pgfscope}%
\definecolor{textcolor}{rgb}{0.000000,0.000000,0.000000}%
\pgfsetstrokecolor{textcolor}%
\pgfsetfillcolor{textcolor}%
\pgftext[x=3.215571in,y=2.640806in,,bottom]{\color{textcolor}\sffamily\fontsize{9.000000}{10.800000}\selectfont 0.1739}%
\end{pgfscope}%
\begin{pgfscope}%
\definecolor{textcolor}{rgb}{0.000000,0.000000,0.000000}%
\pgfsetstrokecolor{textcolor}%
\pgfsetfillcolor{textcolor}%
\pgftext[x=4.235463in,y=2.757086in,,bottom]{\color{textcolor}\sffamily\fontsize{9.000000}{10.800000}\selectfont 0.1886}%
\end{pgfscope}%
\begin{pgfscope}%
\definecolor{textcolor}{rgb}{0.000000,0.000000,0.000000}%
\pgfsetstrokecolor{textcolor}%
\pgfsetfillcolor{textcolor}%
\pgftext[x=5.255356in,y=2.764997in,,bottom]{\color{textcolor}\sffamily\fontsize{9.000000}{10.800000}\selectfont 0.1896}%
\end{pgfscope}%
\begin{pgfscope}%
\definecolor{textcolor}{rgb}{0.000000,0.000000,0.000000}%
\pgfsetstrokecolor{textcolor}%
\pgfsetfillcolor{textcolor}%
\pgftext[x=1.634737in,y=3.723716in,,bottom]{\color{textcolor}\sffamily\fontsize{9.000000}{10.800000}\selectfont 0.3108}%
\end{pgfscope}%
\begin{pgfscope}%
\definecolor{textcolor}{rgb}{0.000000,0.000000,0.000000}%
\pgfsetstrokecolor{textcolor}%
\pgfsetfillcolor{textcolor}%
\pgftext[x=2.654630in,y=3.820220in,,bottom]{\color{textcolor}\sffamily\fontsize{9.000000}{10.800000}\selectfont 0.323}%
\end{pgfscope}%
\begin{pgfscope}%
\definecolor{textcolor}{rgb}{0.000000,0.000000,0.000000}%
\pgfsetstrokecolor{textcolor}%
\pgfsetfillcolor{textcolor}%
\pgftext[x=3.674522in,y=2.284846in,,bottom]{\color{textcolor}\sffamily\fontsize{9.000000}{10.800000}\selectfont 0.1289}%
\end{pgfscope}%
\begin{pgfscope}%
\definecolor{textcolor}{rgb}{0.000000,0.000000,0.000000}%
\pgfsetstrokecolor{textcolor}%
\pgfsetfillcolor{textcolor}%
\pgftext[x=4.694415in,y=2.749176in,,bottom]{\color{textcolor}\sffamily\fontsize{9.000000}{10.800000}\selectfont 0.1876}%
\end{pgfscope}%
\begin{pgfscope}%
\definecolor{textcolor}{rgb}{0.000000,0.000000,0.000000}%
\pgfsetstrokecolor{textcolor}%
\pgfsetfillcolor{textcolor}%
\pgftext[x=5.714308in,y=2.733356in,,bottom]{\color{textcolor}\sffamily\fontsize{9.000000}{10.800000}\selectfont 0.1856}%
\end{pgfscope}%
\begin{pgfscope}%
\definecolor{textcolor}{rgb}{0.000000,0.000000,0.000000}%
\pgfsetstrokecolor{textcolor}%
\pgfsetfillcolor{textcolor}%
\pgftext[x=3.445047in,y=4.470974in,,base]{\color{textcolor}\sffamily\fontsize{12.000000}{14.400000}\selectfont Baselines trained on Pix3D and S2R:3DFREE}%
\end{pgfscope}%
\begin{pgfscope}%
\pgfsetbuttcap%
\pgfsetmiterjoin%
\definecolor{currentfill}{rgb}{1.000000,1.000000,1.000000}%
\pgfsetfillcolor{currentfill}%
\pgfsetfillopacity{0.800000}%
\pgfsetlinewidth{1.003750pt}%
\definecolor{currentstroke}{rgb}{0.800000,0.800000,0.800000}%
\pgfsetstrokecolor{currentstroke}%
\pgfsetstrokeopacity{0.800000}%
\pgfsetdash{}{0pt}%
\pgfpathmoveto{\pgfqpoint{4.882270in}{3.868815in}}%
\pgfpathlineto{\pgfqpoint{6.096435in}{3.868815in}}%
\pgfpathquadraticcurveto{\pgfqpoint{6.124213in}{3.868815in}}{\pgfqpoint{6.124213in}{3.896593in}}%
\pgfpathlineto{\pgfqpoint{6.124213in}{4.290419in}}%
\pgfpathquadraticcurveto{\pgfqpoint{6.124213in}{4.318196in}}{\pgfqpoint{6.096435in}{4.318196in}}%
\pgfpathlineto{\pgfqpoint{4.882270in}{4.318196in}}%
\pgfpathquadraticcurveto{\pgfqpoint{4.854492in}{4.318196in}}{\pgfqpoint{4.854492in}{4.290419in}}%
\pgfpathlineto{\pgfqpoint{4.854492in}{3.896593in}}%
\pgfpathquadraticcurveto{\pgfqpoint{4.854492in}{3.868815in}}{\pgfqpoint{4.882270in}{3.868815in}}%
\pgfpathclose%
\pgfusepath{stroke,fill}%
\end{pgfscope}%
\begin{pgfscope}%
\pgfsetbuttcap%
\pgfsetmiterjoin%
\definecolor{currentfill}{rgb}{0.121569,0.466667,0.705882}%
\pgfsetfillcolor{currentfill}%
\pgfsetlinewidth{0.000000pt}%
\definecolor{currentstroke}{rgb}{0.000000,0.000000,0.000000}%
\pgfsetstrokecolor{currentstroke}%
\pgfsetstrokeopacity{0.000000}%
\pgfsetdash{}{0pt}%
\pgfpathmoveto{\pgfqpoint{4.910048in}{4.157118in}}%
\pgfpathlineto{\pgfqpoint{5.187825in}{4.157118in}}%
\pgfpathlineto{\pgfqpoint{5.187825in}{4.254340in}}%
\pgfpathlineto{\pgfqpoint{4.910048in}{4.254340in}}%
\pgfpathclose%
\pgfusepath{fill}%
\end{pgfscope}%
\begin{pgfscope}%
\definecolor{textcolor}{rgb}{0.000000,0.000000,0.000000}%
\pgfsetstrokecolor{textcolor}%
\pgfsetfillcolor{textcolor}%
\pgftext[x=5.298937in,y=4.157118in,left,base]{\color{textcolor}\sffamily\fontsize{10.000000}{12.000000}\selectfont Pix2Vox++}%
\end{pgfscope}%
\begin{pgfscope}%
\pgfsetbuttcap%
\pgfsetmiterjoin%
\definecolor{currentfill}{rgb}{1.000000,0.498039,0.054902}%
\pgfsetfillcolor{currentfill}%
\pgfsetlinewidth{0.000000pt}%
\definecolor{currentstroke}{rgb}{0.000000,0.000000,0.000000}%
\pgfsetstrokecolor{currentstroke}%
\pgfsetstrokeopacity{0.000000}%
\pgfsetdash{}{0pt}%
\pgfpathmoveto{\pgfqpoint{4.910048in}{3.953260in}}%
\pgfpathlineto{\pgfqpoint{5.187825in}{3.953260in}}%
\pgfpathlineto{\pgfqpoint{5.187825in}{4.050483in}}%
\pgfpathlineto{\pgfqpoint{4.910048in}{4.050483in}}%
\pgfpathclose%
\pgfusepath{fill}%
\end{pgfscope}%
\begin{pgfscope}%
\definecolor{textcolor}{rgb}{0.000000,0.000000,0.000000}%
\pgfsetstrokecolor{textcolor}%
\pgfsetfillcolor{textcolor}%
\pgftext[x=5.298937in,y=3.953260in,left,base]{\color{textcolor}\sffamily\fontsize{10.000000}{12.000000}\selectfont Pix2Vox}%
\end{pgfscope}%
\end{pgfpicture}%
\makeatother%
\endgroup%
}
%    \resizebox{0.49\linewidth}{0.5\linewidth}{%% Creator: Matplotlib, PGF backend
%%
%% To include the figure in your LaTeX document, write
%%   \input{<filename>.pgf}
%%
%% Make sure the required packages are loaded in your preamble
%%   \usepackage{pgf}
%%
%% Figures using additional raster images can only be included by \input if
%% they are in the same directory as the main LaTeX file. For loading figures
%% from other directories you can use the `import` package
%%   \usepackage{import}
%%
%% and then include the figures with
%%   \import{<path to file>}{<filename>.pgf}
%%
%% Matplotlib used the following preamble
%%   \usepackage{fontspec}
%%   \setmainfont{DejaVuSerif.ttf}[Path=\detokenize{/Users/apple/opt/anaconda3/envs/kaolin/lib/python3.7/site-packages/matplotlib/mpl-data/fonts/ttf/}]
%%   \setsansfont{DejaVuSans.ttf}[Path=\detokenize{/Users/apple/opt/anaconda3/envs/kaolin/lib/python3.7/site-packages/matplotlib/mpl-data/fonts/ttf/}]
%%   \setmonofont{DejaVuSansMono.ttf}[Path=\detokenize{/Users/apple/opt/anaconda3/envs/kaolin/lib/python3.7/site-packages/matplotlib/mpl-data/fonts/ttf/}]
%%
\begingroup%
\makeatletter%
\begin{pgfpicture}%
\pgfpathrectangle{\pgfpointorigin}{\pgfqpoint{6.293658in}{4.697602in}}%
\pgfusepath{use as bounding box, clip}%
\begin{pgfscope}%
\pgfsetbuttcap%
\pgfsetmiterjoin%
\definecolor{currentfill}{rgb}{1.000000,1.000000,1.000000}%
\pgfsetfillcolor{currentfill}%
\pgfsetlinewidth{0.000000pt}%
\definecolor{currentstroke}{rgb}{1.000000,1.000000,1.000000}%
\pgfsetstrokecolor{currentstroke}%
\pgfsetdash{}{0pt}%
\pgfpathmoveto{\pgfqpoint{-0.000000in}{0.000000in}}%
\pgfpathlineto{\pgfqpoint{6.293658in}{0.000000in}}%
\pgfpathlineto{\pgfqpoint{6.293658in}{4.697602in}}%
\pgfpathlineto{\pgfqpoint{-0.000000in}{4.697602in}}%
\pgfpathclose%
\pgfusepath{fill}%
\end{pgfscope}%
\begin{pgfscope}%
\pgfsetbuttcap%
\pgfsetmiterjoin%
\definecolor{currentfill}{rgb}{1.000000,1.000000,1.000000}%
\pgfsetfillcolor{currentfill}%
\pgfsetlinewidth{0.000000pt}%
\definecolor{currentstroke}{rgb}{0.000000,0.000000,0.000000}%
\pgfsetstrokecolor{currentstroke}%
\pgfsetstrokeopacity{0.000000}%
\pgfsetdash{}{0pt}%
\pgfpathmoveto{\pgfqpoint{0.696435in}{1.223552in}}%
\pgfpathlineto{\pgfqpoint{6.193658in}{1.223552in}}%
\pgfpathlineto{\pgfqpoint{6.193658in}{4.387641in}}%
\pgfpathlineto{\pgfqpoint{0.696435in}{4.387641in}}%
\pgfpathclose%
\pgfusepath{fill}%
\end{pgfscope}%
\begin{pgfscope}%
\pgfpathrectangle{\pgfqpoint{0.696435in}{1.223552in}}{\pgfqpoint{5.497222in}{3.164089in}}%
\pgfusepath{clip}%
\pgfsetbuttcap%
\pgfsetmiterjoin%
\definecolor{currentfill}{rgb}{0.121569,0.466667,0.705882}%
\pgfsetfillcolor{currentfill}%
\pgfsetlinewidth{0.000000pt}%
\definecolor{currentstroke}{rgb}{0.000000,0.000000,0.000000}%
\pgfsetstrokecolor{currentstroke}%
\pgfsetstrokeopacity{0.000000}%
\pgfsetdash{}{0pt}%
\pgfpathmoveto{\pgfqpoint{0.946309in}{1.223552in}}%
\pgfpathlineto{\pgfqpoint{1.405261in}{1.223552in}}%
\pgfpathlineto{\pgfqpoint{1.405261in}{3.420221in}}%
\pgfpathlineto{\pgfqpoint{0.946309in}{3.420221in}}%
\pgfpathclose%
\pgfusepath{fill}%
\end{pgfscope}%
\begin{pgfscope}%
\pgfpathrectangle{\pgfqpoint{0.696435in}{1.223552in}}{\pgfqpoint{5.497222in}{3.164089in}}%
\pgfusepath{clip}%
\pgfsetbuttcap%
\pgfsetmiterjoin%
\definecolor{currentfill}{rgb}{0.121569,0.466667,0.705882}%
\pgfsetfillcolor{currentfill}%
\pgfsetlinewidth{0.000000pt}%
\definecolor{currentstroke}{rgb}{0.000000,0.000000,0.000000}%
\pgfsetstrokecolor{currentstroke}%
\pgfsetstrokeopacity{0.000000}%
\pgfsetdash{}{0pt}%
\pgfpathmoveto{\pgfqpoint{1.966202in}{1.223552in}}%
\pgfpathlineto{\pgfqpoint{2.425154in}{1.223552in}}%
\pgfpathlineto{\pgfqpoint{2.425154in}{3.947041in}}%
\pgfpathlineto{\pgfqpoint{1.966202in}{3.947041in}}%
\pgfpathclose%
\pgfusepath{fill}%
\end{pgfscope}%
\begin{pgfscope}%
\pgfpathrectangle{\pgfqpoint{0.696435in}{1.223552in}}{\pgfqpoint{5.497222in}{3.164089in}}%
\pgfusepath{clip}%
\pgfsetbuttcap%
\pgfsetmiterjoin%
\definecolor{currentfill}{rgb}{0.121569,0.466667,0.705882}%
\pgfsetfillcolor{currentfill}%
\pgfsetlinewidth{0.000000pt}%
\definecolor{currentstroke}{rgb}{0.000000,0.000000,0.000000}%
\pgfsetstrokecolor{currentstroke}%
\pgfsetstrokeopacity{0.000000}%
\pgfsetdash{}{0pt}%
\pgfpathmoveto{\pgfqpoint{2.986095in}{1.223552in}}%
\pgfpathlineto{\pgfqpoint{3.445047in}{1.223552in}}%
\pgfpathlineto{\pgfqpoint{3.445047in}{1.679971in}}%
\pgfpathlineto{\pgfqpoint{2.986095in}{1.679971in}}%
\pgfpathclose%
\pgfusepath{fill}%
\end{pgfscope}%
\begin{pgfscope}%
\pgfpathrectangle{\pgfqpoint{0.696435in}{1.223552in}}{\pgfqpoint{5.497222in}{3.164089in}}%
\pgfusepath{clip}%
\pgfsetbuttcap%
\pgfsetmiterjoin%
\definecolor{currentfill}{rgb}{0.121569,0.466667,0.705882}%
\pgfsetfillcolor{currentfill}%
\pgfsetlinewidth{0.000000pt}%
\definecolor{currentstroke}{rgb}{0.000000,0.000000,0.000000}%
\pgfsetstrokecolor{currentstroke}%
\pgfsetstrokeopacity{0.000000}%
\pgfsetdash{}{0pt}%
\pgfpathmoveto{\pgfqpoint{4.005988in}{1.223552in}}%
\pgfpathlineto{\pgfqpoint{4.464939in}{1.223552in}}%
\pgfpathlineto{\pgfqpoint{4.464939in}{1.893548in}}%
\pgfpathlineto{\pgfqpoint{4.005988in}{1.893548in}}%
\pgfpathclose%
\pgfusepath{fill}%
\end{pgfscope}%
\begin{pgfscope}%
\pgfpathrectangle{\pgfqpoint{0.696435in}{1.223552in}}{\pgfqpoint{5.497222in}{3.164089in}}%
\pgfusepath{clip}%
\pgfsetbuttcap%
\pgfsetmiterjoin%
\definecolor{currentfill}{rgb}{0.121569,0.466667,0.705882}%
\pgfsetfillcolor{currentfill}%
\pgfsetlinewidth{0.000000pt}%
\definecolor{currentstroke}{rgb}{0.000000,0.000000,0.000000}%
\pgfsetstrokecolor{currentstroke}%
\pgfsetstrokeopacity{0.000000}%
\pgfsetdash{}{0pt}%
\pgfpathmoveto{\pgfqpoint{5.025880in}{1.223552in}}%
\pgfpathlineto{\pgfqpoint{5.484832in}{1.223552in}}%
\pgfpathlineto{\pgfqpoint{5.484832in}{2.249508in}}%
\pgfpathlineto{\pgfqpoint{5.025880in}{2.249508in}}%
\pgfpathclose%
\pgfusepath{fill}%
\end{pgfscope}%
\begin{pgfscope}%
\pgfpathrectangle{\pgfqpoint{0.696435in}{1.223552in}}{\pgfqpoint{5.497222in}{3.164089in}}%
\pgfusepath{clip}%
\pgfsetbuttcap%
\pgfsetmiterjoin%
\definecolor{currentfill}{rgb}{1.000000,0.498039,0.054902}%
\pgfsetfillcolor{currentfill}%
\pgfsetlinewidth{0.000000pt}%
\definecolor{currentstroke}{rgb}{0.000000,0.000000,0.000000}%
\pgfsetstrokecolor{currentstroke}%
\pgfsetstrokeopacity{0.000000}%
\pgfsetdash{}{0pt}%
\pgfpathmoveto{\pgfqpoint{1.405261in}{1.223552in}}%
\pgfpathlineto{\pgfqpoint{1.864213in}{1.223552in}}%
\pgfpathlineto{\pgfqpoint{1.864213in}{3.682049in}}%
\pgfpathlineto{\pgfqpoint{1.405261in}{3.682049in}}%
\pgfpathclose%
\pgfusepath{fill}%
\end{pgfscope}%
\begin{pgfscope}%
\pgfpathrectangle{\pgfqpoint{0.696435in}{1.223552in}}{\pgfqpoint{5.497222in}{3.164089in}}%
\pgfusepath{clip}%
\pgfsetbuttcap%
\pgfsetmiterjoin%
\definecolor{currentfill}{rgb}{1.000000,0.498039,0.054902}%
\pgfsetfillcolor{currentfill}%
\pgfsetlinewidth{0.000000pt}%
\definecolor{currentstroke}{rgb}{0.000000,0.000000,0.000000}%
\pgfsetstrokecolor{currentstroke}%
\pgfsetstrokeopacity{0.000000}%
\pgfsetdash{}{0pt}%
\pgfpathmoveto{\pgfqpoint{2.425154in}{1.223552in}}%
\pgfpathlineto{\pgfqpoint{2.884105in}{1.223552in}}%
\pgfpathlineto{\pgfqpoint{2.884105in}{3.794374in}}%
\pgfpathlineto{\pgfqpoint{2.425154in}{3.794374in}}%
\pgfpathclose%
\pgfusepath{fill}%
\end{pgfscope}%
\begin{pgfscope}%
\pgfpathrectangle{\pgfqpoint{0.696435in}{1.223552in}}{\pgfqpoint{5.497222in}{3.164089in}}%
\pgfusepath{clip}%
\pgfsetbuttcap%
\pgfsetmiterjoin%
\definecolor{currentfill}{rgb}{1.000000,0.498039,0.054902}%
\pgfsetfillcolor{currentfill}%
\pgfsetlinewidth{0.000000pt}%
\definecolor{currentstroke}{rgb}{0.000000,0.000000,0.000000}%
\pgfsetstrokecolor{currentstroke}%
\pgfsetstrokeopacity{0.000000}%
\pgfsetdash{}{0pt}%
\pgfpathmoveto{\pgfqpoint{3.445047in}{1.223552in}}%
\pgfpathlineto{\pgfqpoint{3.903998in}{1.223552in}}%
\pgfpathlineto{\pgfqpoint{3.903998in}{1.760656in}}%
\pgfpathlineto{\pgfqpoint{3.445047in}{1.760656in}}%
\pgfpathclose%
\pgfusepath{fill}%
\end{pgfscope}%
\begin{pgfscope}%
\pgfpathrectangle{\pgfqpoint{0.696435in}{1.223552in}}{\pgfqpoint{5.497222in}{3.164089in}}%
\pgfusepath{clip}%
\pgfsetbuttcap%
\pgfsetmiterjoin%
\definecolor{currentfill}{rgb}{1.000000,0.498039,0.054902}%
\pgfsetfillcolor{currentfill}%
\pgfsetlinewidth{0.000000pt}%
\definecolor{currentstroke}{rgb}{0.000000,0.000000,0.000000}%
\pgfsetstrokecolor{currentstroke}%
\pgfsetstrokeopacity{0.000000}%
\pgfsetdash{}{0pt}%
\pgfpathmoveto{\pgfqpoint{4.464939in}{1.223552in}}%
\pgfpathlineto{\pgfqpoint{4.923891in}{1.223552in}}%
\pgfpathlineto{\pgfqpoint{4.923891in}{1.675225in}}%
\pgfpathlineto{\pgfqpoint{4.464939in}{1.675225in}}%
\pgfpathclose%
\pgfusepath{fill}%
\end{pgfscope}%
\begin{pgfscope}%
\pgfpathrectangle{\pgfqpoint{0.696435in}{1.223552in}}{\pgfqpoint{5.497222in}{3.164089in}}%
\pgfusepath{clip}%
\pgfsetbuttcap%
\pgfsetmiterjoin%
\definecolor{currentfill}{rgb}{1.000000,0.498039,0.054902}%
\pgfsetfillcolor{currentfill}%
\pgfsetlinewidth{0.000000pt}%
\definecolor{currentstroke}{rgb}{0.000000,0.000000,0.000000}%
\pgfsetstrokecolor{currentstroke}%
\pgfsetstrokeopacity{0.000000}%
\pgfsetdash{}{0pt}%
\pgfpathmoveto{\pgfqpoint{5.484832in}{1.223552in}}%
\pgfpathlineto{\pgfqpoint{5.943784in}{1.223552in}}%
\pgfpathlineto{\pgfqpoint{5.943784in}{2.523201in}}%
\pgfpathlineto{\pgfqpoint{5.484832in}{2.523201in}}%
\pgfpathclose%
\pgfusepath{fill}%
\end{pgfscope}%
\begin{pgfscope}%
\pgfsetbuttcap%
\pgfsetroundjoin%
\definecolor{currentfill}{rgb}{0.000000,0.000000,0.000000}%
\pgfsetfillcolor{currentfill}%
\pgfsetlinewidth{0.803000pt}%
\definecolor{currentstroke}{rgb}{0.000000,0.000000,0.000000}%
\pgfsetstrokecolor{currentstroke}%
\pgfsetdash{}{0pt}%
\pgfsys@defobject{currentmarker}{\pgfqpoint{0.000000in}{-0.048611in}}{\pgfqpoint{0.000000in}{0.000000in}}{%
\pgfpathmoveto{\pgfqpoint{0.000000in}{0.000000in}}%
\pgfpathlineto{\pgfqpoint{0.000000in}{-0.048611in}}%
\pgfusepath{stroke,fill}%
}%
\begin{pgfscope}%
\pgfsys@transformshift{1.405261in}{1.223552in}%
\pgfsys@useobject{currentmarker}{}%
\end{pgfscope}%
\end{pgfscope}%
\begin{pgfscope}%
\definecolor{textcolor}{rgb}{0.000000,0.000000,0.000000}%
\pgfsetstrokecolor{textcolor}%
\pgfsetfillcolor{textcolor}%
\pgftext[x=1.091236in, y=0.369475in, left, base,rotate=45.000000]{\color{textcolor}\sffamily\fontsize{10.000000}{12.000000}\selectfont Pix3d(no aug)}%
\end{pgfscope}%
\begin{pgfscope}%
\pgfsetbuttcap%
\pgfsetroundjoin%
\definecolor{currentfill}{rgb}{0.000000,0.000000,0.000000}%
\pgfsetfillcolor{currentfill}%
\pgfsetlinewidth{0.803000pt}%
\definecolor{currentstroke}{rgb}{0.000000,0.000000,0.000000}%
\pgfsetstrokecolor{currentstroke}%
\pgfsetdash{}{0pt}%
\pgfsys@defobject{currentmarker}{\pgfqpoint{0.000000in}{-0.048611in}}{\pgfqpoint{0.000000in}{0.000000in}}{%
\pgfpathmoveto{\pgfqpoint{0.000000in}{0.000000in}}%
\pgfpathlineto{\pgfqpoint{0.000000in}{-0.048611in}}%
\pgfusepath{stroke,fill}%
}%
\begin{pgfscope}%
\pgfsys@transformshift{2.425154in}{1.223552in}%
\pgfsys@useobject{currentmarker}{}%
\end{pgfscope}%
\end{pgfscope}%
\begin{pgfscope}%
\definecolor{textcolor}{rgb}{0.000000,0.000000,0.000000}%
\pgfsetstrokecolor{textcolor}%
\pgfsetfillcolor{textcolor}%
\pgftext[x=2.318601in, y=0.784419in, left, base,rotate=45.000000]{\color{textcolor}\sffamily\fontsize{10.000000}{12.000000}\selectfont Pix3d}%
\end{pgfscope}%
\begin{pgfscope}%
\pgfsetbuttcap%
\pgfsetroundjoin%
\definecolor{currentfill}{rgb}{0.000000,0.000000,0.000000}%
\pgfsetfillcolor{currentfill}%
\pgfsetlinewidth{0.803000pt}%
\definecolor{currentstroke}{rgb}{0.000000,0.000000,0.000000}%
\pgfsetstrokecolor{currentstroke}%
\pgfsetdash{}{0pt}%
\pgfsys@defobject{currentmarker}{\pgfqpoint{0.000000in}{-0.048611in}}{\pgfqpoint{0.000000in}{0.000000in}}{%
\pgfpathmoveto{\pgfqpoint{0.000000in}{0.000000in}}%
\pgfpathlineto{\pgfqpoint{0.000000in}{-0.048611in}}%
\pgfusepath{stroke,fill}%
}%
\begin{pgfscope}%
\pgfsys@transformshift{3.445047in}{1.223552in}%
\pgfsys@useobject{currentmarker}{}%
\end{pgfscope}%
\end{pgfscope}%
\begin{pgfscope}%
\definecolor{textcolor}{rgb}{0.000000,0.000000,0.000000}%
\pgfsetstrokecolor{textcolor}%
\pgfsetfillcolor{textcolor}%
\pgftext[x=3.101386in, y=0.313130in, left, base,rotate=45.000000]{\color{textcolor}\sffamily\fontsize{10.000000}{12.000000}\selectfont s2r\_v1(no aug)}%
\end{pgfscope}%
\begin{pgfscope}%
\pgfsetbuttcap%
\pgfsetroundjoin%
\definecolor{currentfill}{rgb}{0.000000,0.000000,0.000000}%
\pgfsetfillcolor{currentfill}%
\pgfsetlinewidth{0.803000pt}%
\definecolor{currentstroke}{rgb}{0.000000,0.000000,0.000000}%
\pgfsetstrokecolor{currentstroke}%
\pgfsetdash{}{0pt}%
\pgfsys@defobject{currentmarker}{\pgfqpoint{0.000000in}{-0.048611in}}{\pgfqpoint{0.000000in}{0.000000in}}{%
\pgfpathmoveto{\pgfqpoint{0.000000in}{0.000000in}}%
\pgfpathlineto{\pgfqpoint{0.000000in}{-0.048611in}}%
\pgfusepath{stroke,fill}%
}%
\begin{pgfscope}%
\pgfsys@transformshift{4.464939in}{1.223552in}%
\pgfsys@useobject{currentmarker}{}%
\end{pgfscope}%
\end{pgfscope}%
\begin{pgfscope}%
\definecolor{textcolor}{rgb}{0.000000,0.000000,0.000000}%
\pgfsetstrokecolor{textcolor}%
\pgfsetfillcolor{textcolor}%
\pgftext[x=4.327936in, y=0.729704in, left, base,rotate=45.000000]{\color{textcolor}\sffamily\fontsize{10.000000}{12.000000}\selectfont s2r\_v1}%
\end{pgfscope}%
\begin{pgfscope}%
\pgfsetbuttcap%
\pgfsetroundjoin%
\definecolor{currentfill}{rgb}{0.000000,0.000000,0.000000}%
\pgfsetfillcolor{currentfill}%
\pgfsetlinewidth{0.803000pt}%
\definecolor{currentstroke}{rgb}{0.000000,0.000000,0.000000}%
\pgfsetstrokecolor{currentstroke}%
\pgfsetdash{}{0pt}%
\pgfsys@defobject{currentmarker}{\pgfqpoint{0.000000in}{-0.048611in}}{\pgfqpoint{0.000000in}{0.000000in}}{%
\pgfpathmoveto{\pgfqpoint{0.000000in}{0.000000in}}%
\pgfpathlineto{\pgfqpoint{0.000000in}{-0.048611in}}%
\pgfusepath{stroke,fill}%
}%
\begin{pgfscope}%
\pgfsys@transformshift{5.484832in}{1.223552in}%
\pgfsys@useobject{currentmarker}{}%
\end{pgfscope}%
\end{pgfscope}%
\begin{pgfscope}%
\definecolor{textcolor}{rgb}{0.000000,0.000000,0.000000}%
\pgfsetstrokecolor{textcolor}%
\pgfsetfillcolor{textcolor}%
\pgftext[x=5.347828in, y=0.729704in, left, base,rotate=45.000000]{\color{textcolor}\sffamily\fontsize{10.000000}{12.000000}\selectfont s2r\_v2}%
\end{pgfscope}%
\begin{pgfscope}%
\definecolor{textcolor}{rgb}{0.000000,0.000000,0.000000}%
\pgfsetstrokecolor{textcolor}%
\pgfsetfillcolor{textcolor}%
\pgftext[x=3.445047in,y=0.234413in,,top]{\color{textcolor}\sffamily\fontsize{10.000000}{12.000000}\selectfont Dataset}%
\end{pgfscope}%
\begin{pgfscope}%
\pgfsetbuttcap%
\pgfsetroundjoin%
\definecolor{currentfill}{rgb}{0.000000,0.000000,0.000000}%
\pgfsetfillcolor{currentfill}%
\pgfsetlinewidth{0.803000pt}%
\definecolor{currentstroke}{rgb}{0.000000,0.000000,0.000000}%
\pgfsetstrokecolor{currentstroke}%
\pgfsetdash{}{0pt}%
\pgfsys@defobject{currentmarker}{\pgfqpoint{-0.048611in}{0.000000in}}{\pgfqpoint{-0.000000in}{0.000000in}}{%
\pgfpathmoveto{\pgfqpoint{-0.000000in}{0.000000in}}%
\pgfpathlineto{\pgfqpoint{-0.048611in}{0.000000in}}%
\pgfusepath{stroke,fill}%
}%
\begin{pgfscope}%
\pgfsys@transformshift{0.696435in}{1.223552in}%
\pgfsys@useobject{currentmarker}{}%
\end{pgfscope}%
\end{pgfscope}%
\begin{pgfscope}%
\definecolor{textcolor}{rgb}{0.000000,0.000000,0.000000}%
\pgfsetstrokecolor{textcolor}%
\pgfsetfillcolor{textcolor}%
\pgftext[x=0.289968in, y=1.170790in, left, base]{\color{textcolor}\sffamily\fontsize{10.000000}{12.000000}\selectfont 0.00}%
\end{pgfscope}%
\begin{pgfscope}%
\pgfsetbuttcap%
\pgfsetroundjoin%
\definecolor{currentfill}{rgb}{0.000000,0.000000,0.000000}%
\pgfsetfillcolor{currentfill}%
\pgfsetlinewidth{0.803000pt}%
\definecolor{currentstroke}{rgb}{0.000000,0.000000,0.000000}%
\pgfsetstrokecolor{currentstroke}%
\pgfsetdash{}{0pt}%
\pgfsys@defobject{currentmarker}{\pgfqpoint{-0.048611in}{0.000000in}}{\pgfqpoint{-0.000000in}{0.000000in}}{%
\pgfpathmoveto{\pgfqpoint{-0.000000in}{0.000000in}}%
\pgfpathlineto{\pgfqpoint{-0.048611in}{0.000000in}}%
\pgfusepath{stroke,fill}%
}%
\begin{pgfscope}%
\pgfsys@transformshift{0.696435in}{1.619063in}%
\pgfsys@useobject{currentmarker}{}%
\end{pgfscope}%
\end{pgfscope}%
\begin{pgfscope}%
\definecolor{textcolor}{rgb}{0.000000,0.000000,0.000000}%
\pgfsetstrokecolor{textcolor}%
\pgfsetfillcolor{textcolor}%
\pgftext[x=0.289968in, y=1.566301in, left, base]{\color{textcolor}\sffamily\fontsize{10.000000}{12.000000}\selectfont 0.05}%
\end{pgfscope}%
\begin{pgfscope}%
\pgfsetbuttcap%
\pgfsetroundjoin%
\definecolor{currentfill}{rgb}{0.000000,0.000000,0.000000}%
\pgfsetfillcolor{currentfill}%
\pgfsetlinewidth{0.803000pt}%
\definecolor{currentstroke}{rgb}{0.000000,0.000000,0.000000}%
\pgfsetstrokecolor{currentstroke}%
\pgfsetdash{}{0pt}%
\pgfsys@defobject{currentmarker}{\pgfqpoint{-0.048611in}{0.000000in}}{\pgfqpoint{-0.000000in}{0.000000in}}{%
\pgfpathmoveto{\pgfqpoint{-0.000000in}{0.000000in}}%
\pgfpathlineto{\pgfqpoint{-0.048611in}{0.000000in}}%
\pgfusepath{stroke,fill}%
}%
\begin{pgfscope}%
\pgfsys@transformshift{0.696435in}{2.014574in}%
\pgfsys@useobject{currentmarker}{}%
\end{pgfscope}%
\end{pgfscope}%
\begin{pgfscope}%
\definecolor{textcolor}{rgb}{0.000000,0.000000,0.000000}%
\pgfsetstrokecolor{textcolor}%
\pgfsetfillcolor{textcolor}%
\pgftext[x=0.289968in, y=1.961812in, left, base]{\color{textcolor}\sffamily\fontsize{10.000000}{12.000000}\selectfont 0.10}%
\end{pgfscope}%
\begin{pgfscope}%
\pgfsetbuttcap%
\pgfsetroundjoin%
\definecolor{currentfill}{rgb}{0.000000,0.000000,0.000000}%
\pgfsetfillcolor{currentfill}%
\pgfsetlinewidth{0.803000pt}%
\definecolor{currentstroke}{rgb}{0.000000,0.000000,0.000000}%
\pgfsetstrokecolor{currentstroke}%
\pgfsetdash{}{0pt}%
\pgfsys@defobject{currentmarker}{\pgfqpoint{-0.048611in}{0.000000in}}{\pgfqpoint{-0.000000in}{0.000000in}}{%
\pgfpathmoveto{\pgfqpoint{-0.000000in}{0.000000in}}%
\pgfpathlineto{\pgfqpoint{-0.048611in}{0.000000in}}%
\pgfusepath{stroke,fill}%
}%
\begin{pgfscope}%
\pgfsys@transformshift{0.696435in}{2.410085in}%
\pgfsys@useobject{currentmarker}{}%
\end{pgfscope}%
\end{pgfscope}%
\begin{pgfscope}%
\definecolor{textcolor}{rgb}{0.000000,0.000000,0.000000}%
\pgfsetstrokecolor{textcolor}%
\pgfsetfillcolor{textcolor}%
\pgftext[x=0.289968in, y=2.357324in, left, base]{\color{textcolor}\sffamily\fontsize{10.000000}{12.000000}\selectfont 0.15}%
\end{pgfscope}%
\begin{pgfscope}%
\pgfsetbuttcap%
\pgfsetroundjoin%
\definecolor{currentfill}{rgb}{0.000000,0.000000,0.000000}%
\pgfsetfillcolor{currentfill}%
\pgfsetlinewidth{0.803000pt}%
\definecolor{currentstroke}{rgb}{0.000000,0.000000,0.000000}%
\pgfsetstrokecolor{currentstroke}%
\pgfsetdash{}{0pt}%
\pgfsys@defobject{currentmarker}{\pgfqpoint{-0.048611in}{0.000000in}}{\pgfqpoint{-0.000000in}{0.000000in}}{%
\pgfpathmoveto{\pgfqpoint{-0.000000in}{0.000000in}}%
\pgfpathlineto{\pgfqpoint{-0.048611in}{0.000000in}}%
\pgfusepath{stroke,fill}%
}%
\begin{pgfscope}%
\pgfsys@transformshift{0.696435in}{2.805596in}%
\pgfsys@useobject{currentmarker}{}%
\end{pgfscope}%
\end{pgfscope}%
\begin{pgfscope}%
\definecolor{textcolor}{rgb}{0.000000,0.000000,0.000000}%
\pgfsetstrokecolor{textcolor}%
\pgfsetfillcolor{textcolor}%
\pgftext[x=0.289968in, y=2.752835in, left, base]{\color{textcolor}\sffamily\fontsize{10.000000}{12.000000}\selectfont 0.20}%
\end{pgfscope}%
\begin{pgfscope}%
\pgfsetbuttcap%
\pgfsetroundjoin%
\definecolor{currentfill}{rgb}{0.000000,0.000000,0.000000}%
\pgfsetfillcolor{currentfill}%
\pgfsetlinewidth{0.803000pt}%
\definecolor{currentstroke}{rgb}{0.000000,0.000000,0.000000}%
\pgfsetstrokecolor{currentstroke}%
\pgfsetdash{}{0pt}%
\pgfsys@defobject{currentmarker}{\pgfqpoint{-0.048611in}{0.000000in}}{\pgfqpoint{-0.000000in}{0.000000in}}{%
\pgfpathmoveto{\pgfqpoint{-0.000000in}{0.000000in}}%
\pgfpathlineto{\pgfqpoint{-0.048611in}{0.000000in}}%
\pgfusepath{stroke,fill}%
}%
\begin{pgfscope}%
\pgfsys@transformshift{0.696435in}{3.201107in}%
\pgfsys@useobject{currentmarker}{}%
\end{pgfscope}%
\end{pgfscope}%
\begin{pgfscope}%
\definecolor{textcolor}{rgb}{0.000000,0.000000,0.000000}%
\pgfsetstrokecolor{textcolor}%
\pgfsetfillcolor{textcolor}%
\pgftext[x=0.289968in, y=3.148346in, left, base]{\color{textcolor}\sffamily\fontsize{10.000000}{12.000000}\selectfont 0.25}%
\end{pgfscope}%
\begin{pgfscope}%
\pgfsetbuttcap%
\pgfsetroundjoin%
\definecolor{currentfill}{rgb}{0.000000,0.000000,0.000000}%
\pgfsetfillcolor{currentfill}%
\pgfsetlinewidth{0.803000pt}%
\definecolor{currentstroke}{rgb}{0.000000,0.000000,0.000000}%
\pgfsetstrokecolor{currentstroke}%
\pgfsetdash{}{0pt}%
\pgfsys@defobject{currentmarker}{\pgfqpoint{-0.048611in}{0.000000in}}{\pgfqpoint{-0.000000in}{0.000000in}}{%
\pgfpathmoveto{\pgfqpoint{-0.000000in}{0.000000in}}%
\pgfpathlineto{\pgfqpoint{-0.048611in}{0.000000in}}%
\pgfusepath{stroke,fill}%
}%
\begin{pgfscope}%
\pgfsys@transformshift{0.696435in}{3.596618in}%
\pgfsys@useobject{currentmarker}{}%
\end{pgfscope}%
\end{pgfscope}%
\begin{pgfscope}%
\definecolor{textcolor}{rgb}{0.000000,0.000000,0.000000}%
\pgfsetstrokecolor{textcolor}%
\pgfsetfillcolor{textcolor}%
\pgftext[x=0.289968in, y=3.543857in, left, base]{\color{textcolor}\sffamily\fontsize{10.000000}{12.000000}\selectfont 0.30}%
\end{pgfscope}%
\begin{pgfscope}%
\pgfsetbuttcap%
\pgfsetroundjoin%
\definecolor{currentfill}{rgb}{0.000000,0.000000,0.000000}%
\pgfsetfillcolor{currentfill}%
\pgfsetlinewidth{0.803000pt}%
\definecolor{currentstroke}{rgb}{0.000000,0.000000,0.000000}%
\pgfsetstrokecolor{currentstroke}%
\pgfsetdash{}{0pt}%
\pgfsys@defobject{currentmarker}{\pgfqpoint{-0.048611in}{0.000000in}}{\pgfqpoint{-0.000000in}{0.000000in}}{%
\pgfpathmoveto{\pgfqpoint{-0.000000in}{0.000000in}}%
\pgfpathlineto{\pgfqpoint{-0.048611in}{0.000000in}}%
\pgfusepath{stroke,fill}%
}%
\begin{pgfscope}%
\pgfsys@transformshift{0.696435in}{3.992130in}%
\pgfsys@useobject{currentmarker}{}%
\end{pgfscope}%
\end{pgfscope}%
\begin{pgfscope}%
\definecolor{textcolor}{rgb}{0.000000,0.000000,0.000000}%
\pgfsetstrokecolor{textcolor}%
\pgfsetfillcolor{textcolor}%
\pgftext[x=0.289968in, y=3.939368in, left, base]{\color{textcolor}\sffamily\fontsize{10.000000}{12.000000}\selectfont 0.35}%
\end{pgfscope}%
\begin{pgfscope}%
\pgfsetbuttcap%
\pgfsetroundjoin%
\definecolor{currentfill}{rgb}{0.000000,0.000000,0.000000}%
\pgfsetfillcolor{currentfill}%
\pgfsetlinewidth{0.803000pt}%
\definecolor{currentstroke}{rgb}{0.000000,0.000000,0.000000}%
\pgfsetstrokecolor{currentstroke}%
\pgfsetdash{}{0pt}%
\pgfsys@defobject{currentmarker}{\pgfqpoint{-0.048611in}{0.000000in}}{\pgfqpoint{-0.000000in}{0.000000in}}{%
\pgfpathmoveto{\pgfqpoint{-0.000000in}{0.000000in}}%
\pgfpathlineto{\pgfqpoint{-0.048611in}{0.000000in}}%
\pgfusepath{stroke,fill}%
}%
\begin{pgfscope}%
\pgfsys@transformshift{0.696435in}{4.387641in}%
\pgfsys@useobject{currentmarker}{}%
\end{pgfscope}%
\end{pgfscope}%
\begin{pgfscope}%
\definecolor{textcolor}{rgb}{0.000000,0.000000,0.000000}%
\pgfsetstrokecolor{textcolor}%
\pgfsetfillcolor{textcolor}%
\pgftext[x=0.289968in, y=4.334879in, left, base]{\color{textcolor}\sffamily\fontsize{10.000000}{12.000000}\selectfont 0.40}%
\end{pgfscope}%
\begin{pgfscope}%
\definecolor{textcolor}{rgb}{0.000000,0.000000,0.000000}%
\pgfsetstrokecolor{textcolor}%
\pgfsetfillcolor{textcolor}%
\pgftext[x=0.234413in,y=2.805596in,,bottom,rotate=90.000000]{\color{textcolor}\sffamily\fontsize{10.000000}{12.000000}\selectfont IoU}%
\end{pgfscope}%
\begin{pgfscope}%
\pgfsetrectcap%
\pgfsetmiterjoin%
\pgfsetlinewidth{0.803000pt}%
\definecolor{currentstroke}{rgb}{0.000000,0.000000,0.000000}%
\pgfsetstrokecolor{currentstroke}%
\pgfsetdash{}{0pt}%
\pgfpathmoveto{\pgfqpoint{0.696435in}{1.223552in}}%
\pgfpathlineto{\pgfqpoint{0.696435in}{4.387641in}}%
\pgfusepath{stroke}%
\end{pgfscope}%
\begin{pgfscope}%
\pgfsetrectcap%
\pgfsetmiterjoin%
\pgfsetlinewidth{0.803000pt}%
\definecolor{currentstroke}{rgb}{0.000000,0.000000,0.000000}%
\pgfsetstrokecolor{currentstroke}%
\pgfsetdash{}{0pt}%
\pgfpathmoveto{\pgfqpoint{6.193658in}{1.223552in}}%
\pgfpathlineto{\pgfqpoint{6.193658in}{4.387641in}}%
\pgfusepath{stroke}%
\end{pgfscope}%
\begin{pgfscope}%
\pgfsetrectcap%
\pgfsetmiterjoin%
\pgfsetlinewidth{0.803000pt}%
\definecolor{currentstroke}{rgb}{0.000000,0.000000,0.000000}%
\pgfsetstrokecolor{currentstroke}%
\pgfsetdash{}{0pt}%
\pgfpathmoveto{\pgfqpoint{0.696435in}{1.223552in}}%
\pgfpathlineto{\pgfqpoint{6.193658in}{1.223552in}}%
\pgfusepath{stroke}%
\end{pgfscope}%
\begin{pgfscope}%
\pgfsetrectcap%
\pgfsetmiterjoin%
\pgfsetlinewidth{0.803000pt}%
\definecolor{currentstroke}{rgb}{0.000000,0.000000,0.000000}%
\pgfsetstrokecolor{currentstroke}%
\pgfsetdash{}{0pt}%
\pgfpathmoveto{\pgfqpoint{0.696435in}{4.387641in}}%
\pgfpathlineto{\pgfqpoint{6.193658in}{4.387641in}}%
\pgfusepath{stroke}%
\end{pgfscope}%
\begin{pgfscope}%
\definecolor{textcolor}{rgb}{0.000000,0.000000,0.000000}%
\pgfsetstrokecolor{textcolor}%
\pgfsetfillcolor{textcolor}%
\pgftext[x=1.175785in,y=3.461887in,,bottom]{\color{textcolor}\sffamily\fontsize{9.000000}{10.800000}\selectfont 0.2777}%
\end{pgfscope}%
\begin{pgfscope}%
\definecolor{textcolor}{rgb}{0.000000,0.000000,0.000000}%
\pgfsetstrokecolor{textcolor}%
\pgfsetfillcolor{textcolor}%
\pgftext[x=2.195678in,y=3.988708in,,bottom]{\color{textcolor}\sffamily\fontsize{9.000000}{10.800000}\selectfont 0.3443}%
\end{pgfscope}%
\begin{pgfscope}%
\definecolor{textcolor}{rgb}{0.000000,0.000000,0.000000}%
\pgfsetstrokecolor{textcolor}%
\pgfsetfillcolor{textcolor}%
\pgftext[x=3.215571in,y=1.721638in,,bottom]{\color{textcolor}\sffamily\fontsize{9.000000}{10.800000}\selectfont 0.0577}%
\end{pgfscope}%
\begin{pgfscope}%
\definecolor{textcolor}{rgb}{0.000000,0.000000,0.000000}%
\pgfsetstrokecolor{textcolor}%
\pgfsetfillcolor{textcolor}%
\pgftext[x=4.235463in,y=1.935214in,,bottom]{\color{textcolor}\sffamily\fontsize{9.000000}{10.800000}\selectfont 0.0847}%
\end{pgfscope}%
\begin{pgfscope}%
\definecolor{textcolor}{rgb}{0.000000,0.000000,0.000000}%
\pgfsetstrokecolor{textcolor}%
\pgfsetfillcolor{textcolor}%
\pgftext[x=5.255356in,y=2.291174in,,bottom]{\color{textcolor}\sffamily\fontsize{9.000000}{10.800000}\selectfont 0.1297}%
\end{pgfscope}%
\begin{pgfscope}%
\definecolor{textcolor}{rgb}{0.000000,0.000000,0.000000}%
\pgfsetstrokecolor{textcolor}%
\pgfsetfillcolor{textcolor}%
\pgftext[x=1.634737in,y=3.723716in,,bottom]{\color{textcolor}\sffamily\fontsize{9.000000}{10.800000}\selectfont 0.3108}%
\end{pgfscope}%
\begin{pgfscope}%
\definecolor{textcolor}{rgb}{0.000000,0.000000,0.000000}%
\pgfsetstrokecolor{textcolor}%
\pgfsetfillcolor{textcolor}%
\pgftext[x=2.654630in,y=3.836041in,,bottom]{\color{textcolor}\sffamily\fontsize{9.000000}{10.800000}\selectfont 0.325}%
\end{pgfscope}%
\begin{pgfscope}%
\definecolor{textcolor}{rgb}{0.000000,0.000000,0.000000}%
\pgfsetstrokecolor{textcolor}%
\pgfsetfillcolor{textcolor}%
\pgftext[x=3.674522in,y=1.802322in,,bottom]{\color{textcolor}\sffamily\fontsize{9.000000}{10.800000}\selectfont 0.0679}%
\end{pgfscope}%
\begin{pgfscope}%
\definecolor{textcolor}{rgb}{0.000000,0.000000,0.000000}%
\pgfsetstrokecolor{textcolor}%
\pgfsetfillcolor{textcolor}%
\pgftext[x=4.694415in,y=1.716892in,,bottom]{\color{textcolor}\sffamily\fontsize{9.000000}{10.800000}\selectfont 0.0571}%
\end{pgfscope}%
\begin{pgfscope}%
\definecolor{textcolor}{rgb}{0.000000,0.000000,0.000000}%
\pgfsetstrokecolor{textcolor}%
\pgfsetfillcolor{textcolor}%
\pgftext[x=5.714308in,y=2.564868in,,bottom]{\color{textcolor}\sffamily\fontsize{9.000000}{10.800000}\selectfont 0.1643}%
\end{pgfscope}%
\begin{pgfscope}%
\definecolor{textcolor}{rgb}{0.000000,0.000000,0.000000}%
\pgfsetstrokecolor{textcolor}%
\pgfsetfillcolor{textcolor}%
\pgftext[x=3.445047in,y=4.470974in,,base]{\color{textcolor}\sffamily\fontsize{12.000000}{14.400000}\selectfont Baselines trained on Pix3D and S2R:3DFREE}%
\end{pgfscope}%
\begin{pgfscope}%
\pgfsetbuttcap%
\pgfsetmiterjoin%
\definecolor{currentfill}{rgb}{1.000000,1.000000,1.000000}%
\pgfsetfillcolor{currentfill}%
\pgfsetfillopacity{0.800000}%
\pgfsetlinewidth{1.003750pt}%
\definecolor{currentstroke}{rgb}{0.800000,0.800000,0.800000}%
\pgfsetstrokecolor{currentstroke}%
\pgfsetstrokeopacity{0.800000}%
\pgfsetdash{}{0pt}%
\pgfpathmoveto{\pgfqpoint{4.882270in}{3.868815in}}%
\pgfpathlineto{\pgfqpoint{6.096435in}{3.868815in}}%
\pgfpathquadraticcurveto{\pgfqpoint{6.124213in}{3.868815in}}{\pgfqpoint{6.124213in}{3.896593in}}%
\pgfpathlineto{\pgfqpoint{6.124213in}{4.290419in}}%
\pgfpathquadraticcurveto{\pgfqpoint{6.124213in}{4.318196in}}{\pgfqpoint{6.096435in}{4.318196in}}%
\pgfpathlineto{\pgfqpoint{4.882270in}{4.318196in}}%
\pgfpathquadraticcurveto{\pgfqpoint{4.854492in}{4.318196in}}{\pgfqpoint{4.854492in}{4.290419in}}%
\pgfpathlineto{\pgfqpoint{4.854492in}{3.896593in}}%
\pgfpathquadraticcurveto{\pgfqpoint{4.854492in}{3.868815in}}{\pgfqpoint{4.882270in}{3.868815in}}%
\pgfpathclose%
\pgfusepath{stroke,fill}%
\end{pgfscope}%
\begin{pgfscope}%
\pgfsetbuttcap%
\pgfsetmiterjoin%
\definecolor{currentfill}{rgb}{0.121569,0.466667,0.705882}%
\pgfsetfillcolor{currentfill}%
\pgfsetlinewidth{0.000000pt}%
\definecolor{currentstroke}{rgb}{0.000000,0.000000,0.000000}%
\pgfsetstrokecolor{currentstroke}%
\pgfsetstrokeopacity{0.000000}%
\pgfsetdash{}{0pt}%
\pgfpathmoveto{\pgfqpoint{4.910048in}{4.157118in}}%
\pgfpathlineto{\pgfqpoint{5.187825in}{4.157118in}}%
\pgfpathlineto{\pgfqpoint{5.187825in}{4.254340in}}%
\pgfpathlineto{\pgfqpoint{4.910048in}{4.254340in}}%
\pgfpathclose%
\pgfusepath{fill}%
\end{pgfscope}%
\begin{pgfscope}%
\definecolor{textcolor}{rgb}{0.000000,0.000000,0.000000}%
\pgfsetstrokecolor{textcolor}%
\pgfsetfillcolor{textcolor}%
\pgftext[x=5.298937in,y=4.157118in,left,base]{\color{textcolor}\sffamily\fontsize{10.000000}{12.000000}\selectfont Pix2Vox++}%
\end{pgfscope}%
\begin{pgfscope}%
\pgfsetbuttcap%
\pgfsetmiterjoin%
\definecolor{currentfill}{rgb}{1.000000,0.498039,0.054902}%
\pgfsetfillcolor{currentfill}%
\pgfsetlinewidth{0.000000pt}%
\definecolor{currentstroke}{rgb}{0.000000,0.000000,0.000000}%
\pgfsetstrokecolor{currentstroke}%
\pgfsetstrokeopacity{0.000000}%
\pgfsetdash{}{0pt}%
\pgfpathmoveto{\pgfqpoint{4.910048in}{3.953260in}}%
\pgfpathlineto{\pgfqpoint{5.187825in}{3.953260in}}%
\pgfpathlineto{\pgfqpoint{5.187825in}{4.050483in}}%
\pgfpathlineto{\pgfqpoint{4.910048in}{4.050483in}}%
\pgfpathclose%
\pgfusepath{fill}%
\end{pgfscope}%
\begin{pgfscope}%
\definecolor{textcolor}{rgb}{0.000000,0.000000,0.000000}%
\pgfsetstrokecolor{textcolor}%
\pgfsetfillcolor{textcolor}%
\pgftext[x=5.298937in,y=3.953260in,left,base]{\color{textcolor}\sffamily\fontsize{10.000000}{12.000000}\selectfont Pix2Vox}%
\end{pgfscope}%
\end{pgfpicture}%
\makeatother%
\endgroup%
}
%    \caption{Bar plot for the \gls{iou}  for \textbf{baselines} trained on real and synthetic datasets, with and without 2D augmentation.
%        (left)The checkpoint was saved using real dataset for validation and test, (right) the checkpoint was saved using corresponding synthetic data for validation step and tested with real data.
%        In both the cases we see that ~\gls{free} does not perform adequately on its own. \gls{s2rv2} contributes better than \gls{s2rv1}.}
%    \label{fig:baseline1}
%\end{figure}


\begin{figure}[!ht]
    \centering
    \subfloat[][]{\resizebox{0.75\linewidth}{!}{%% Creator: Matplotlib, PGF backend
%%
%% To include the figure in your LaTeX document, write
%%   \input{<filename>.pgf}
%%
%% Make sure the required packages are loaded in your preamble
%%   \usepackage{pgf}
%%
%% Figures using additional raster images can only be included by \input if
%% they are in the same directory as the main LaTeX file. For loading figures
%% from other directories you can use the `import` package
%%   \usepackage{import}
%%
%% and then include the figures with
%%   \import{<path to file>}{<filename>.pgf}
%%
%% Matplotlib used the following preamble
%%   \usepackage{fontspec}
%%   \setmainfont{DejaVuSerif.ttf}[Path=\detokenize{/Users/apple/opt/anaconda3/envs/kaolin/lib/python3.7/site-packages/matplotlib/mpl-data/fonts/ttf/}]
%%   \setsansfont{DejaVuSans.ttf}[Path=\detokenize{/Users/apple/opt/anaconda3/envs/kaolin/lib/python3.7/site-packages/matplotlib/mpl-data/fonts/ttf/}]
%%   \setmonofont{DejaVuSansMono.ttf}[Path=\detokenize{/Users/apple/opt/anaconda3/envs/kaolin/lib/python3.7/site-packages/matplotlib/mpl-data/fonts/ttf/}]
%%
\begingroup%
\makeatletter%
\begin{pgfpicture}%
\pgfpathrectangle{\pgfpointorigin}{\pgfqpoint{6.293658in}{4.697602in}}%
\pgfusepath{use as bounding box, clip}%
\begin{pgfscope}%
\pgfsetbuttcap%
\pgfsetmiterjoin%
\definecolor{currentfill}{rgb}{1.000000,1.000000,1.000000}%
\pgfsetfillcolor{currentfill}%
\pgfsetlinewidth{0.000000pt}%
\definecolor{currentstroke}{rgb}{1.000000,1.000000,1.000000}%
\pgfsetstrokecolor{currentstroke}%
\pgfsetdash{}{0pt}%
\pgfpathmoveto{\pgfqpoint{-0.000000in}{0.000000in}}%
\pgfpathlineto{\pgfqpoint{6.293658in}{0.000000in}}%
\pgfpathlineto{\pgfqpoint{6.293658in}{4.697602in}}%
\pgfpathlineto{\pgfqpoint{-0.000000in}{4.697602in}}%
\pgfpathclose%
\pgfusepath{fill}%
\end{pgfscope}%
\begin{pgfscope}%
\pgfsetbuttcap%
\pgfsetmiterjoin%
\definecolor{currentfill}{rgb}{1.000000,1.000000,1.000000}%
\pgfsetfillcolor{currentfill}%
\pgfsetlinewidth{0.000000pt}%
\definecolor{currentstroke}{rgb}{0.000000,0.000000,0.000000}%
\pgfsetstrokecolor{currentstroke}%
\pgfsetstrokeopacity{0.000000}%
\pgfsetdash{}{0pt}%
\pgfpathmoveto{\pgfqpoint{0.696435in}{1.223552in}}%
\pgfpathlineto{\pgfqpoint{6.193658in}{1.223552in}}%
\pgfpathlineto{\pgfqpoint{6.193658in}{4.387641in}}%
\pgfpathlineto{\pgfqpoint{0.696435in}{4.387641in}}%
\pgfpathclose%
\pgfusepath{fill}%
\end{pgfscope}%
\begin{pgfscope}%
\pgfpathrectangle{\pgfqpoint{0.696435in}{1.223552in}}{\pgfqpoint{5.497222in}{3.164089in}}%
\pgfusepath{clip}%
\pgfsetbuttcap%
\pgfsetmiterjoin%
\definecolor{currentfill}{rgb}{0.121569,0.466667,0.705882}%
\pgfsetfillcolor{currentfill}%
\pgfsetlinewidth{0.000000pt}%
\definecolor{currentstroke}{rgb}{0.000000,0.000000,0.000000}%
\pgfsetstrokecolor{currentstroke}%
\pgfsetstrokeopacity{0.000000}%
\pgfsetdash{}{0pt}%
\pgfpathmoveto{\pgfqpoint{0.946309in}{1.223552in}}%
\pgfpathlineto{\pgfqpoint{1.405261in}{1.223552in}}%
\pgfpathlineto{\pgfqpoint{1.405261in}{3.377505in}}%
\pgfpathlineto{\pgfqpoint{0.946309in}{3.377505in}}%
\pgfpathclose%
\pgfusepath{fill}%
\end{pgfscope}%
\begin{pgfscope}%
\pgfpathrectangle{\pgfqpoint{0.696435in}{1.223552in}}{\pgfqpoint{5.497222in}{3.164089in}}%
\pgfusepath{clip}%
\pgfsetbuttcap%
\pgfsetmiterjoin%
\definecolor{currentfill}{rgb}{0.121569,0.466667,0.705882}%
\pgfsetfillcolor{currentfill}%
\pgfsetlinewidth{0.000000pt}%
\definecolor{currentstroke}{rgb}{0.000000,0.000000,0.000000}%
\pgfsetstrokecolor{currentstroke}%
\pgfsetstrokeopacity{0.000000}%
\pgfsetdash{}{0pt}%
\pgfpathmoveto{\pgfqpoint{1.966202in}{1.223552in}}%
\pgfpathlineto{\pgfqpoint{2.425154in}{1.223552in}}%
\pgfpathlineto{\pgfqpoint{2.425154in}{4.015069in}}%
\pgfpathlineto{\pgfqpoint{1.966202in}{4.015069in}}%
\pgfpathclose%
\pgfusepath{fill}%
\end{pgfscope}%
\begin{pgfscope}%
\pgfpathrectangle{\pgfqpoint{0.696435in}{1.223552in}}{\pgfqpoint{5.497222in}{3.164089in}}%
\pgfusepath{clip}%
\pgfsetbuttcap%
\pgfsetmiterjoin%
\definecolor{currentfill}{rgb}{0.121569,0.466667,0.705882}%
\pgfsetfillcolor{currentfill}%
\pgfsetlinewidth{0.000000pt}%
\definecolor{currentstroke}{rgb}{0.000000,0.000000,0.000000}%
\pgfsetstrokecolor{currentstroke}%
\pgfsetstrokeopacity{0.000000}%
\pgfsetdash{}{0pt}%
\pgfpathmoveto{\pgfqpoint{2.986095in}{1.223552in}}%
\pgfpathlineto{\pgfqpoint{3.445047in}{1.223552in}}%
\pgfpathlineto{\pgfqpoint{3.445047in}{2.599139in}}%
\pgfpathlineto{\pgfqpoint{2.986095in}{2.599139in}}%
\pgfpathclose%
\pgfusepath{fill}%
\end{pgfscope}%
\begin{pgfscope}%
\pgfpathrectangle{\pgfqpoint{0.696435in}{1.223552in}}{\pgfqpoint{5.497222in}{3.164089in}}%
\pgfusepath{clip}%
\pgfsetbuttcap%
\pgfsetmiterjoin%
\definecolor{currentfill}{rgb}{0.121569,0.466667,0.705882}%
\pgfsetfillcolor{currentfill}%
\pgfsetlinewidth{0.000000pt}%
\definecolor{currentstroke}{rgb}{0.000000,0.000000,0.000000}%
\pgfsetstrokecolor{currentstroke}%
\pgfsetstrokeopacity{0.000000}%
\pgfsetdash{}{0pt}%
\pgfpathmoveto{\pgfqpoint{4.005988in}{1.223552in}}%
\pgfpathlineto{\pgfqpoint{4.464939in}{1.223552in}}%
\pgfpathlineto{\pgfqpoint{4.464939in}{2.715420in}}%
\pgfpathlineto{\pgfqpoint{4.005988in}{2.715420in}}%
\pgfpathclose%
\pgfusepath{fill}%
\end{pgfscope}%
\begin{pgfscope}%
\pgfpathrectangle{\pgfqpoint{0.696435in}{1.223552in}}{\pgfqpoint{5.497222in}{3.164089in}}%
\pgfusepath{clip}%
\pgfsetbuttcap%
\pgfsetmiterjoin%
\definecolor{currentfill}{rgb}{0.121569,0.466667,0.705882}%
\pgfsetfillcolor{currentfill}%
\pgfsetlinewidth{0.000000pt}%
\definecolor{currentstroke}{rgb}{0.000000,0.000000,0.000000}%
\pgfsetstrokecolor{currentstroke}%
\pgfsetstrokeopacity{0.000000}%
\pgfsetdash{}{0pt}%
\pgfpathmoveto{\pgfqpoint{5.025880in}{1.223552in}}%
\pgfpathlineto{\pgfqpoint{5.484832in}{1.223552in}}%
\pgfpathlineto{\pgfqpoint{5.484832in}{2.723330in}}%
\pgfpathlineto{\pgfqpoint{5.025880in}{2.723330in}}%
\pgfpathclose%
\pgfusepath{fill}%
\end{pgfscope}%
\begin{pgfscope}%
\pgfpathrectangle{\pgfqpoint{0.696435in}{1.223552in}}{\pgfqpoint{5.497222in}{3.164089in}}%
\pgfusepath{clip}%
\pgfsetbuttcap%
\pgfsetmiterjoin%
\definecolor{currentfill}{rgb}{1.000000,0.498039,0.054902}%
\pgfsetfillcolor{currentfill}%
\pgfsetlinewidth{0.000000pt}%
\definecolor{currentstroke}{rgb}{0.000000,0.000000,0.000000}%
\pgfsetstrokecolor{currentstroke}%
\pgfsetstrokeopacity{0.000000}%
\pgfsetdash{}{0pt}%
\pgfpathmoveto{\pgfqpoint{1.405261in}{1.223552in}}%
\pgfpathlineto{\pgfqpoint{1.864213in}{1.223552in}}%
\pgfpathlineto{\pgfqpoint{1.864213in}{3.682049in}}%
\pgfpathlineto{\pgfqpoint{1.405261in}{3.682049in}}%
\pgfpathclose%
\pgfusepath{fill}%
\end{pgfscope}%
\begin{pgfscope}%
\pgfpathrectangle{\pgfqpoint{0.696435in}{1.223552in}}{\pgfqpoint{5.497222in}{3.164089in}}%
\pgfusepath{clip}%
\pgfsetbuttcap%
\pgfsetmiterjoin%
\definecolor{currentfill}{rgb}{1.000000,0.498039,0.054902}%
\pgfsetfillcolor{currentfill}%
\pgfsetlinewidth{0.000000pt}%
\definecolor{currentstroke}{rgb}{0.000000,0.000000,0.000000}%
\pgfsetstrokecolor{currentstroke}%
\pgfsetstrokeopacity{0.000000}%
\pgfsetdash{}{0pt}%
\pgfpathmoveto{\pgfqpoint{2.425154in}{1.223552in}}%
\pgfpathlineto{\pgfqpoint{2.884105in}{1.223552in}}%
\pgfpathlineto{\pgfqpoint{2.884105in}{3.778554in}}%
\pgfpathlineto{\pgfqpoint{2.425154in}{3.778554in}}%
\pgfpathclose%
\pgfusepath{fill}%
\end{pgfscope}%
\begin{pgfscope}%
\pgfpathrectangle{\pgfqpoint{0.696435in}{1.223552in}}{\pgfqpoint{5.497222in}{3.164089in}}%
\pgfusepath{clip}%
\pgfsetbuttcap%
\pgfsetmiterjoin%
\definecolor{currentfill}{rgb}{1.000000,0.498039,0.054902}%
\pgfsetfillcolor{currentfill}%
\pgfsetlinewidth{0.000000pt}%
\definecolor{currentstroke}{rgb}{0.000000,0.000000,0.000000}%
\pgfsetstrokecolor{currentstroke}%
\pgfsetstrokeopacity{0.000000}%
\pgfsetdash{}{0pt}%
\pgfpathmoveto{\pgfqpoint{3.445047in}{1.223552in}}%
\pgfpathlineto{\pgfqpoint{3.903998in}{1.223552in}}%
\pgfpathlineto{\pgfqpoint{3.903998in}{2.243179in}}%
\pgfpathlineto{\pgfqpoint{3.445047in}{2.243179in}}%
\pgfpathclose%
\pgfusepath{fill}%
\end{pgfscope}%
\begin{pgfscope}%
\pgfpathrectangle{\pgfqpoint{0.696435in}{1.223552in}}{\pgfqpoint{5.497222in}{3.164089in}}%
\pgfusepath{clip}%
\pgfsetbuttcap%
\pgfsetmiterjoin%
\definecolor{currentfill}{rgb}{1.000000,0.498039,0.054902}%
\pgfsetfillcolor{currentfill}%
\pgfsetlinewidth{0.000000pt}%
\definecolor{currentstroke}{rgb}{0.000000,0.000000,0.000000}%
\pgfsetstrokecolor{currentstroke}%
\pgfsetstrokeopacity{0.000000}%
\pgfsetdash{}{0pt}%
\pgfpathmoveto{\pgfqpoint{4.464939in}{1.223552in}}%
\pgfpathlineto{\pgfqpoint{4.923891in}{1.223552in}}%
\pgfpathlineto{\pgfqpoint{4.923891in}{2.707509in}}%
\pgfpathlineto{\pgfqpoint{4.464939in}{2.707509in}}%
\pgfpathclose%
\pgfusepath{fill}%
\end{pgfscope}%
\begin{pgfscope}%
\pgfpathrectangle{\pgfqpoint{0.696435in}{1.223552in}}{\pgfqpoint{5.497222in}{3.164089in}}%
\pgfusepath{clip}%
\pgfsetbuttcap%
\pgfsetmiterjoin%
\definecolor{currentfill}{rgb}{1.000000,0.498039,0.054902}%
\pgfsetfillcolor{currentfill}%
\pgfsetlinewidth{0.000000pt}%
\definecolor{currentstroke}{rgb}{0.000000,0.000000,0.000000}%
\pgfsetstrokecolor{currentstroke}%
\pgfsetstrokeopacity{0.000000}%
\pgfsetdash{}{0pt}%
\pgfpathmoveto{\pgfqpoint{5.484832in}{1.223552in}}%
\pgfpathlineto{\pgfqpoint{5.943784in}{1.223552in}}%
\pgfpathlineto{\pgfqpoint{5.943784in}{2.691689in}}%
\pgfpathlineto{\pgfqpoint{5.484832in}{2.691689in}}%
\pgfpathclose%
\pgfusepath{fill}%
\end{pgfscope}%
\begin{pgfscope}%
\pgfsetbuttcap%
\pgfsetroundjoin%
\definecolor{currentfill}{rgb}{0.000000,0.000000,0.000000}%
\pgfsetfillcolor{currentfill}%
\pgfsetlinewidth{0.803000pt}%
\definecolor{currentstroke}{rgb}{0.000000,0.000000,0.000000}%
\pgfsetstrokecolor{currentstroke}%
\pgfsetdash{}{0pt}%
\pgfsys@defobject{currentmarker}{\pgfqpoint{0.000000in}{-0.048611in}}{\pgfqpoint{0.000000in}{0.000000in}}{%
\pgfpathmoveto{\pgfqpoint{0.000000in}{0.000000in}}%
\pgfpathlineto{\pgfqpoint{0.000000in}{-0.048611in}}%
\pgfusepath{stroke,fill}%
}%
\begin{pgfscope}%
\pgfsys@transformshift{1.405261in}{1.223552in}%
\pgfsys@useobject{currentmarker}{}%
\end{pgfscope}%
\end{pgfscope}%
\begin{pgfscope}%
\definecolor{textcolor}{rgb}{0.000000,0.000000,0.000000}%
\pgfsetstrokecolor{textcolor}%
\pgfsetfillcolor{textcolor}%
\pgftext[x=1.091236in, y=0.369475in, left, base,rotate=45.000000]{\color{textcolor}\sffamily\fontsize{10.000000}{12.000000}\selectfont Pix3d(no aug)}%
\end{pgfscope}%
\begin{pgfscope}%
\pgfsetbuttcap%
\pgfsetroundjoin%
\definecolor{currentfill}{rgb}{0.000000,0.000000,0.000000}%
\pgfsetfillcolor{currentfill}%
\pgfsetlinewidth{0.803000pt}%
\definecolor{currentstroke}{rgb}{0.000000,0.000000,0.000000}%
\pgfsetstrokecolor{currentstroke}%
\pgfsetdash{}{0pt}%
\pgfsys@defobject{currentmarker}{\pgfqpoint{0.000000in}{-0.048611in}}{\pgfqpoint{0.000000in}{0.000000in}}{%
\pgfpathmoveto{\pgfqpoint{0.000000in}{0.000000in}}%
\pgfpathlineto{\pgfqpoint{0.000000in}{-0.048611in}}%
\pgfusepath{stroke,fill}%
}%
\begin{pgfscope}%
\pgfsys@transformshift{2.425154in}{1.223552in}%
\pgfsys@useobject{currentmarker}{}%
\end{pgfscope}%
\end{pgfscope}%
\begin{pgfscope}%
\definecolor{textcolor}{rgb}{0.000000,0.000000,0.000000}%
\pgfsetstrokecolor{textcolor}%
\pgfsetfillcolor{textcolor}%
\pgftext[x=2.318601in, y=0.784419in, left, base,rotate=45.000000]{\color{textcolor}\sffamily\fontsize{10.000000}{12.000000}\selectfont Pix3d}%
\end{pgfscope}%
\begin{pgfscope}%
\pgfsetbuttcap%
\pgfsetroundjoin%
\definecolor{currentfill}{rgb}{0.000000,0.000000,0.000000}%
\pgfsetfillcolor{currentfill}%
\pgfsetlinewidth{0.803000pt}%
\definecolor{currentstroke}{rgb}{0.000000,0.000000,0.000000}%
\pgfsetstrokecolor{currentstroke}%
\pgfsetdash{}{0pt}%
\pgfsys@defobject{currentmarker}{\pgfqpoint{0.000000in}{-0.048611in}}{\pgfqpoint{0.000000in}{0.000000in}}{%
\pgfpathmoveto{\pgfqpoint{0.000000in}{0.000000in}}%
\pgfpathlineto{\pgfqpoint{0.000000in}{-0.048611in}}%
\pgfusepath{stroke,fill}%
}%
\begin{pgfscope}%
\pgfsys@transformshift{3.445047in}{1.223552in}%
\pgfsys@useobject{currentmarker}{}%
\end{pgfscope}%
\end{pgfscope}%
\begin{pgfscope}%
\definecolor{textcolor}{rgb}{0.000000,0.000000,0.000000}%
\pgfsetstrokecolor{textcolor}%
\pgfsetfillcolor{textcolor}%
\pgftext[x=3.101386in, y=0.313130in, left, base,rotate=45.000000]{\color{textcolor}\sffamily\fontsize{10.000000}{12.000000}\selectfont s2r\_v1(no aug)}%
\end{pgfscope}%
\begin{pgfscope}%
\pgfsetbuttcap%
\pgfsetroundjoin%
\definecolor{currentfill}{rgb}{0.000000,0.000000,0.000000}%
\pgfsetfillcolor{currentfill}%
\pgfsetlinewidth{0.803000pt}%
\definecolor{currentstroke}{rgb}{0.000000,0.000000,0.000000}%
\pgfsetstrokecolor{currentstroke}%
\pgfsetdash{}{0pt}%
\pgfsys@defobject{currentmarker}{\pgfqpoint{0.000000in}{-0.048611in}}{\pgfqpoint{0.000000in}{0.000000in}}{%
\pgfpathmoveto{\pgfqpoint{0.000000in}{0.000000in}}%
\pgfpathlineto{\pgfqpoint{0.000000in}{-0.048611in}}%
\pgfusepath{stroke,fill}%
}%
\begin{pgfscope}%
\pgfsys@transformshift{4.464939in}{1.223552in}%
\pgfsys@useobject{currentmarker}{}%
\end{pgfscope}%
\end{pgfscope}%
\begin{pgfscope}%
\definecolor{textcolor}{rgb}{0.000000,0.000000,0.000000}%
\pgfsetstrokecolor{textcolor}%
\pgfsetfillcolor{textcolor}%
\pgftext[x=4.327936in, y=0.729704in, left, base,rotate=45.000000]{\color{textcolor}\sffamily\fontsize{10.000000}{12.000000}\selectfont s2r\_v1}%
\end{pgfscope}%
\begin{pgfscope}%
\pgfsetbuttcap%
\pgfsetroundjoin%
\definecolor{currentfill}{rgb}{0.000000,0.000000,0.000000}%
\pgfsetfillcolor{currentfill}%
\pgfsetlinewidth{0.803000pt}%
\definecolor{currentstroke}{rgb}{0.000000,0.000000,0.000000}%
\pgfsetstrokecolor{currentstroke}%
\pgfsetdash{}{0pt}%
\pgfsys@defobject{currentmarker}{\pgfqpoint{0.000000in}{-0.048611in}}{\pgfqpoint{0.000000in}{0.000000in}}{%
\pgfpathmoveto{\pgfqpoint{0.000000in}{0.000000in}}%
\pgfpathlineto{\pgfqpoint{0.000000in}{-0.048611in}}%
\pgfusepath{stroke,fill}%
}%
\begin{pgfscope}%
\pgfsys@transformshift{5.484832in}{1.223552in}%
\pgfsys@useobject{currentmarker}{}%
\end{pgfscope}%
\end{pgfscope}%
\begin{pgfscope}%
\definecolor{textcolor}{rgb}{0.000000,0.000000,0.000000}%
\pgfsetstrokecolor{textcolor}%
\pgfsetfillcolor{textcolor}%
\pgftext[x=5.347828in, y=0.729704in, left, base,rotate=45.000000]{\color{textcolor}\sffamily\fontsize{10.000000}{12.000000}\selectfont s2r\_v2}%
\end{pgfscope}%
\begin{pgfscope}%
\definecolor{textcolor}{rgb}{0.000000,0.000000,0.000000}%
\pgfsetstrokecolor{textcolor}%
\pgfsetfillcolor{textcolor}%
\pgftext[x=3.445047in,y=0.234413in,,top]{\color{textcolor}\sffamily\fontsize{10.000000}{12.000000}\selectfont Dataset}%
\end{pgfscope}%
\begin{pgfscope}%
\pgfsetbuttcap%
\pgfsetroundjoin%
\definecolor{currentfill}{rgb}{0.000000,0.000000,0.000000}%
\pgfsetfillcolor{currentfill}%
\pgfsetlinewidth{0.803000pt}%
\definecolor{currentstroke}{rgb}{0.000000,0.000000,0.000000}%
\pgfsetstrokecolor{currentstroke}%
\pgfsetdash{}{0pt}%
\pgfsys@defobject{currentmarker}{\pgfqpoint{-0.048611in}{0.000000in}}{\pgfqpoint{-0.000000in}{0.000000in}}{%
\pgfpathmoveto{\pgfqpoint{-0.000000in}{0.000000in}}%
\pgfpathlineto{\pgfqpoint{-0.048611in}{0.000000in}}%
\pgfusepath{stroke,fill}%
}%
\begin{pgfscope}%
\pgfsys@transformshift{0.696435in}{1.223552in}%
\pgfsys@useobject{currentmarker}{}%
\end{pgfscope}%
\end{pgfscope}%
\begin{pgfscope}%
\definecolor{textcolor}{rgb}{0.000000,0.000000,0.000000}%
\pgfsetstrokecolor{textcolor}%
\pgfsetfillcolor{textcolor}%
\pgftext[x=0.289968in, y=1.170790in, left, base]{\color{textcolor}\sffamily\fontsize{10.000000}{12.000000}\selectfont 0.00}%
\end{pgfscope}%
\begin{pgfscope}%
\pgfsetbuttcap%
\pgfsetroundjoin%
\definecolor{currentfill}{rgb}{0.000000,0.000000,0.000000}%
\pgfsetfillcolor{currentfill}%
\pgfsetlinewidth{0.803000pt}%
\definecolor{currentstroke}{rgb}{0.000000,0.000000,0.000000}%
\pgfsetstrokecolor{currentstroke}%
\pgfsetdash{}{0pt}%
\pgfsys@defobject{currentmarker}{\pgfqpoint{-0.048611in}{0.000000in}}{\pgfqpoint{-0.000000in}{0.000000in}}{%
\pgfpathmoveto{\pgfqpoint{-0.000000in}{0.000000in}}%
\pgfpathlineto{\pgfqpoint{-0.048611in}{0.000000in}}%
\pgfusepath{stroke,fill}%
}%
\begin{pgfscope}%
\pgfsys@transformshift{0.696435in}{1.619063in}%
\pgfsys@useobject{currentmarker}{}%
\end{pgfscope}%
\end{pgfscope}%
\begin{pgfscope}%
\definecolor{textcolor}{rgb}{0.000000,0.000000,0.000000}%
\pgfsetstrokecolor{textcolor}%
\pgfsetfillcolor{textcolor}%
\pgftext[x=0.289968in, y=1.566301in, left, base]{\color{textcolor}\sffamily\fontsize{10.000000}{12.000000}\selectfont 0.05}%
\end{pgfscope}%
\begin{pgfscope}%
\pgfsetbuttcap%
\pgfsetroundjoin%
\definecolor{currentfill}{rgb}{0.000000,0.000000,0.000000}%
\pgfsetfillcolor{currentfill}%
\pgfsetlinewidth{0.803000pt}%
\definecolor{currentstroke}{rgb}{0.000000,0.000000,0.000000}%
\pgfsetstrokecolor{currentstroke}%
\pgfsetdash{}{0pt}%
\pgfsys@defobject{currentmarker}{\pgfqpoint{-0.048611in}{0.000000in}}{\pgfqpoint{-0.000000in}{0.000000in}}{%
\pgfpathmoveto{\pgfqpoint{-0.000000in}{0.000000in}}%
\pgfpathlineto{\pgfqpoint{-0.048611in}{0.000000in}}%
\pgfusepath{stroke,fill}%
}%
\begin{pgfscope}%
\pgfsys@transformshift{0.696435in}{2.014574in}%
\pgfsys@useobject{currentmarker}{}%
\end{pgfscope}%
\end{pgfscope}%
\begin{pgfscope}%
\definecolor{textcolor}{rgb}{0.000000,0.000000,0.000000}%
\pgfsetstrokecolor{textcolor}%
\pgfsetfillcolor{textcolor}%
\pgftext[x=0.289968in, y=1.961812in, left, base]{\color{textcolor}\sffamily\fontsize{10.000000}{12.000000}\selectfont 0.10}%
\end{pgfscope}%
\begin{pgfscope}%
\pgfsetbuttcap%
\pgfsetroundjoin%
\definecolor{currentfill}{rgb}{0.000000,0.000000,0.000000}%
\pgfsetfillcolor{currentfill}%
\pgfsetlinewidth{0.803000pt}%
\definecolor{currentstroke}{rgb}{0.000000,0.000000,0.000000}%
\pgfsetstrokecolor{currentstroke}%
\pgfsetdash{}{0pt}%
\pgfsys@defobject{currentmarker}{\pgfqpoint{-0.048611in}{0.000000in}}{\pgfqpoint{-0.000000in}{0.000000in}}{%
\pgfpathmoveto{\pgfqpoint{-0.000000in}{0.000000in}}%
\pgfpathlineto{\pgfqpoint{-0.048611in}{0.000000in}}%
\pgfusepath{stroke,fill}%
}%
\begin{pgfscope}%
\pgfsys@transformshift{0.696435in}{2.410085in}%
\pgfsys@useobject{currentmarker}{}%
\end{pgfscope}%
\end{pgfscope}%
\begin{pgfscope}%
\definecolor{textcolor}{rgb}{0.000000,0.000000,0.000000}%
\pgfsetstrokecolor{textcolor}%
\pgfsetfillcolor{textcolor}%
\pgftext[x=0.289968in, y=2.357324in, left, base]{\color{textcolor}\sffamily\fontsize{10.000000}{12.000000}\selectfont 0.15}%
\end{pgfscope}%
\begin{pgfscope}%
\pgfsetbuttcap%
\pgfsetroundjoin%
\definecolor{currentfill}{rgb}{0.000000,0.000000,0.000000}%
\pgfsetfillcolor{currentfill}%
\pgfsetlinewidth{0.803000pt}%
\definecolor{currentstroke}{rgb}{0.000000,0.000000,0.000000}%
\pgfsetstrokecolor{currentstroke}%
\pgfsetdash{}{0pt}%
\pgfsys@defobject{currentmarker}{\pgfqpoint{-0.048611in}{0.000000in}}{\pgfqpoint{-0.000000in}{0.000000in}}{%
\pgfpathmoveto{\pgfqpoint{-0.000000in}{0.000000in}}%
\pgfpathlineto{\pgfqpoint{-0.048611in}{0.000000in}}%
\pgfusepath{stroke,fill}%
}%
\begin{pgfscope}%
\pgfsys@transformshift{0.696435in}{2.805596in}%
\pgfsys@useobject{currentmarker}{}%
\end{pgfscope}%
\end{pgfscope}%
\begin{pgfscope}%
\definecolor{textcolor}{rgb}{0.000000,0.000000,0.000000}%
\pgfsetstrokecolor{textcolor}%
\pgfsetfillcolor{textcolor}%
\pgftext[x=0.289968in, y=2.752835in, left, base]{\color{textcolor}\sffamily\fontsize{10.000000}{12.000000}\selectfont 0.20}%
\end{pgfscope}%
\begin{pgfscope}%
\pgfsetbuttcap%
\pgfsetroundjoin%
\definecolor{currentfill}{rgb}{0.000000,0.000000,0.000000}%
\pgfsetfillcolor{currentfill}%
\pgfsetlinewidth{0.803000pt}%
\definecolor{currentstroke}{rgb}{0.000000,0.000000,0.000000}%
\pgfsetstrokecolor{currentstroke}%
\pgfsetdash{}{0pt}%
\pgfsys@defobject{currentmarker}{\pgfqpoint{-0.048611in}{0.000000in}}{\pgfqpoint{-0.000000in}{0.000000in}}{%
\pgfpathmoveto{\pgfqpoint{-0.000000in}{0.000000in}}%
\pgfpathlineto{\pgfqpoint{-0.048611in}{0.000000in}}%
\pgfusepath{stroke,fill}%
}%
\begin{pgfscope}%
\pgfsys@transformshift{0.696435in}{3.201107in}%
\pgfsys@useobject{currentmarker}{}%
\end{pgfscope}%
\end{pgfscope}%
\begin{pgfscope}%
\definecolor{textcolor}{rgb}{0.000000,0.000000,0.000000}%
\pgfsetstrokecolor{textcolor}%
\pgfsetfillcolor{textcolor}%
\pgftext[x=0.289968in, y=3.148346in, left, base]{\color{textcolor}\sffamily\fontsize{10.000000}{12.000000}\selectfont 0.25}%
\end{pgfscope}%
\begin{pgfscope}%
\pgfsetbuttcap%
\pgfsetroundjoin%
\definecolor{currentfill}{rgb}{0.000000,0.000000,0.000000}%
\pgfsetfillcolor{currentfill}%
\pgfsetlinewidth{0.803000pt}%
\definecolor{currentstroke}{rgb}{0.000000,0.000000,0.000000}%
\pgfsetstrokecolor{currentstroke}%
\pgfsetdash{}{0pt}%
\pgfsys@defobject{currentmarker}{\pgfqpoint{-0.048611in}{0.000000in}}{\pgfqpoint{-0.000000in}{0.000000in}}{%
\pgfpathmoveto{\pgfqpoint{-0.000000in}{0.000000in}}%
\pgfpathlineto{\pgfqpoint{-0.048611in}{0.000000in}}%
\pgfusepath{stroke,fill}%
}%
\begin{pgfscope}%
\pgfsys@transformshift{0.696435in}{3.596618in}%
\pgfsys@useobject{currentmarker}{}%
\end{pgfscope}%
\end{pgfscope}%
\begin{pgfscope}%
\definecolor{textcolor}{rgb}{0.000000,0.000000,0.000000}%
\pgfsetstrokecolor{textcolor}%
\pgfsetfillcolor{textcolor}%
\pgftext[x=0.289968in, y=3.543857in, left, base]{\color{textcolor}\sffamily\fontsize{10.000000}{12.000000}\selectfont 0.30}%
\end{pgfscope}%
\begin{pgfscope}%
\pgfsetbuttcap%
\pgfsetroundjoin%
\definecolor{currentfill}{rgb}{0.000000,0.000000,0.000000}%
\pgfsetfillcolor{currentfill}%
\pgfsetlinewidth{0.803000pt}%
\definecolor{currentstroke}{rgb}{0.000000,0.000000,0.000000}%
\pgfsetstrokecolor{currentstroke}%
\pgfsetdash{}{0pt}%
\pgfsys@defobject{currentmarker}{\pgfqpoint{-0.048611in}{0.000000in}}{\pgfqpoint{-0.000000in}{0.000000in}}{%
\pgfpathmoveto{\pgfqpoint{-0.000000in}{0.000000in}}%
\pgfpathlineto{\pgfqpoint{-0.048611in}{0.000000in}}%
\pgfusepath{stroke,fill}%
}%
\begin{pgfscope}%
\pgfsys@transformshift{0.696435in}{3.992130in}%
\pgfsys@useobject{currentmarker}{}%
\end{pgfscope}%
\end{pgfscope}%
\begin{pgfscope}%
\definecolor{textcolor}{rgb}{0.000000,0.000000,0.000000}%
\pgfsetstrokecolor{textcolor}%
\pgfsetfillcolor{textcolor}%
\pgftext[x=0.289968in, y=3.939368in, left, base]{\color{textcolor}\sffamily\fontsize{10.000000}{12.000000}\selectfont 0.35}%
\end{pgfscope}%
\begin{pgfscope}%
\pgfsetbuttcap%
\pgfsetroundjoin%
\definecolor{currentfill}{rgb}{0.000000,0.000000,0.000000}%
\pgfsetfillcolor{currentfill}%
\pgfsetlinewidth{0.803000pt}%
\definecolor{currentstroke}{rgb}{0.000000,0.000000,0.000000}%
\pgfsetstrokecolor{currentstroke}%
\pgfsetdash{}{0pt}%
\pgfsys@defobject{currentmarker}{\pgfqpoint{-0.048611in}{0.000000in}}{\pgfqpoint{-0.000000in}{0.000000in}}{%
\pgfpathmoveto{\pgfqpoint{-0.000000in}{0.000000in}}%
\pgfpathlineto{\pgfqpoint{-0.048611in}{0.000000in}}%
\pgfusepath{stroke,fill}%
}%
\begin{pgfscope}%
\pgfsys@transformshift{0.696435in}{4.387641in}%
\pgfsys@useobject{currentmarker}{}%
\end{pgfscope}%
\end{pgfscope}%
\begin{pgfscope}%
\definecolor{textcolor}{rgb}{0.000000,0.000000,0.000000}%
\pgfsetstrokecolor{textcolor}%
\pgfsetfillcolor{textcolor}%
\pgftext[x=0.289968in, y=4.334879in, left, base]{\color{textcolor}\sffamily\fontsize{10.000000}{12.000000}\selectfont 0.40}%
\end{pgfscope}%
\begin{pgfscope}%
\definecolor{textcolor}{rgb}{0.000000,0.000000,0.000000}%
\pgfsetstrokecolor{textcolor}%
\pgfsetfillcolor{textcolor}%
\pgftext[x=0.234413in,y=2.805596in,,bottom,rotate=90.000000]{\color{textcolor}\sffamily\fontsize{10.000000}{12.000000}\selectfont IoU}%
\end{pgfscope}%
\begin{pgfscope}%
\pgfsetrectcap%
\pgfsetmiterjoin%
\pgfsetlinewidth{0.803000pt}%
\definecolor{currentstroke}{rgb}{0.000000,0.000000,0.000000}%
\pgfsetstrokecolor{currentstroke}%
\pgfsetdash{}{0pt}%
\pgfpathmoveto{\pgfqpoint{0.696435in}{1.223552in}}%
\pgfpathlineto{\pgfqpoint{0.696435in}{4.387641in}}%
\pgfusepath{stroke}%
\end{pgfscope}%
\begin{pgfscope}%
\pgfsetrectcap%
\pgfsetmiterjoin%
\pgfsetlinewidth{0.803000pt}%
\definecolor{currentstroke}{rgb}{0.000000,0.000000,0.000000}%
\pgfsetstrokecolor{currentstroke}%
\pgfsetdash{}{0pt}%
\pgfpathmoveto{\pgfqpoint{6.193658in}{1.223552in}}%
\pgfpathlineto{\pgfqpoint{6.193658in}{4.387641in}}%
\pgfusepath{stroke}%
\end{pgfscope}%
\begin{pgfscope}%
\pgfsetrectcap%
\pgfsetmiterjoin%
\pgfsetlinewidth{0.803000pt}%
\definecolor{currentstroke}{rgb}{0.000000,0.000000,0.000000}%
\pgfsetstrokecolor{currentstroke}%
\pgfsetdash{}{0pt}%
\pgfpathmoveto{\pgfqpoint{0.696435in}{1.223552in}}%
\pgfpathlineto{\pgfqpoint{6.193658in}{1.223552in}}%
\pgfusepath{stroke}%
\end{pgfscope}%
\begin{pgfscope}%
\pgfsetrectcap%
\pgfsetmiterjoin%
\pgfsetlinewidth{0.803000pt}%
\definecolor{currentstroke}{rgb}{0.000000,0.000000,0.000000}%
\pgfsetstrokecolor{currentstroke}%
\pgfsetdash{}{0pt}%
\pgfpathmoveto{\pgfqpoint{0.696435in}{4.387641in}}%
\pgfpathlineto{\pgfqpoint{6.193658in}{4.387641in}}%
\pgfusepath{stroke}%
\end{pgfscope}%
\begin{pgfscope}%
\definecolor{textcolor}{rgb}{0.000000,0.000000,0.000000}%
\pgfsetstrokecolor{textcolor}%
\pgfsetfillcolor{textcolor}%
\pgftext[x=1.175785in,y=3.419172in,,bottom]{\color{textcolor}\sffamily\fontsize{9.000000}{10.800000}\selectfont 0.2723}%
\end{pgfscope}%
\begin{pgfscope}%
\definecolor{textcolor}{rgb}{0.000000,0.000000,0.000000}%
\pgfsetstrokecolor{textcolor}%
\pgfsetfillcolor{textcolor}%
\pgftext[x=2.195678in,y=4.056736in,,bottom]{\color{textcolor}\sffamily\fontsize{9.000000}{10.800000}\selectfont 0.3529}%
\end{pgfscope}%
\begin{pgfscope}%
\definecolor{textcolor}{rgb}{0.000000,0.000000,0.000000}%
\pgfsetstrokecolor{textcolor}%
\pgfsetfillcolor{textcolor}%
\pgftext[x=3.215571in,y=2.640806in,,bottom]{\color{textcolor}\sffamily\fontsize{9.000000}{10.800000}\selectfont 0.1739}%
\end{pgfscope}%
\begin{pgfscope}%
\definecolor{textcolor}{rgb}{0.000000,0.000000,0.000000}%
\pgfsetstrokecolor{textcolor}%
\pgfsetfillcolor{textcolor}%
\pgftext[x=4.235463in,y=2.757086in,,bottom]{\color{textcolor}\sffamily\fontsize{9.000000}{10.800000}\selectfont 0.1886}%
\end{pgfscope}%
\begin{pgfscope}%
\definecolor{textcolor}{rgb}{0.000000,0.000000,0.000000}%
\pgfsetstrokecolor{textcolor}%
\pgfsetfillcolor{textcolor}%
\pgftext[x=5.255356in,y=2.764997in,,bottom]{\color{textcolor}\sffamily\fontsize{9.000000}{10.800000}\selectfont 0.1896}%
\end{pgfscope}%
\begin{pgfscope}%
\definecolor{textcolor}{rgb}{0.000000,0.000000,0.000000}%
\pgfsetstrokecolor{textcolor}%
\pgfsetfillcolor{textcolor}%
\pgftext[x=1.634737in,y=3.723716in,,bottom]{\color{textcolor}\sffamily\fontsize{9.000000}{10.800000}\selectfont 0.3108}%
\end{pgfscope}%
\begin{pgfscope}%
\definecolor{textcolor}{rgb}{0.000000,0.000000,0.000000}%
\pgfsetstrokecolor{textcolor}%
\pgfsetfillcolor{textcolor}%
\pgftext[x=2.654630in,y=3.820220in,,bottom]{\color{textcolor}\sffamily\fontsize{9.000000}{10.800000}\selectfont 0.323}%
\end{pgfscope}%
\begin{pgfscope}%
\definecolor{textcolor}{rgb}{0.000000,0.000000,0.000000}%
\pgfsetstrokecolor{textcolor}%
\pgfsetfillcolor{textcolor}%
\pgftext[x=3.674522in,y=2.284846in,,bottom]{\color{textcolor}\sffamily\fontsize{9.000000}{10.800000}\selectfont 0.1289}%
\end{pgfscope}%
\begin{pgfscope}%
\definecolor{textcolor}{rgb}{0.000000,0.000000,0.000000}%
\pgfsetstrokecolor{textcolor}%
\pgfsetfillcolor{textcolor}%
\pgftext[x=4.694415in,y=2.749176in,,bottom]{\color{textcolor}\sffamily\fontsize{9.000000}{10.800000}\selectfont 0.1876}%
\end{pgfscope}%
\begin{pgfscope}%
\definecolor{textcolor}{rgb}{0.000000,0.000000,0.000000}%
\pgfsetstrokecolor{textcolor}%
\pgfsetfillcolor{textcolor}%
\pgftext[x=5.714308in,y=2.733356in,,bottom]{\color{textcolor}\sffamily\fontsize{9.000000}{10.800000}\selectfont 0.1856}%
\end{pgfscope}%
\begin{pgfscope}%
\definecolor{textcolor}{rgb}{0.000000,0.000000,0.000000}%
\pgfsetstrokecolor{textcolor}%
\pgfsetfillcolor{textcolor}%
\pgftext[x=3.445047in,y=4.470974in,,base]{\color{textcolor}\sffamily\fontsize{12.000000}{14.400000}\selectfont Baselines trained on Pix3D and S2R:3DFREE}%
\end{pgfscope}%
\begin{pgfscope}%
\pgfsetbuttcap%
\pgfsetmiterjoin%
\definecolor{currentfill}{rgb}{1.000000,1.000000,1.000000}%
\pgfsetfillcolor{currentfill}%
\pgfsetfillopacity{0.800000}%
\pgfsetlinewidth{1.003750pt}%
\definecolor{currentstroke}{rgb}{0.800000,0.800000,0.800000}%
\pgfsetstrokecolor{currentstroke}%
\pgfsetstrokeopacity{0.800000}%
\pgfsetdash{}{0pt}%
\pgfpathmoveto{\pgfqpoint{4.882270in}{3.868815in}}%
\pgfpathlineto{\pgfqpoint{6.096435in}{3.868815in}}%
\pgfpathquadraticcurveto{\pgfqpoint{6.124213in}{3.868815in}}{\pgfqpoint{6.124213in}{3.896593in}}%
\pgfpathlineto{\pgfqpoint{6.124213in}{4.290419in}}%
\pgfpathquadraticcurveto{\pgfqpoint{6.124213in}{4.318196in}}{\pgfqpoint{6.096435in}{4.318196in}}%
\pgfpathlineto{\pgfqpoint{4.882270in}{4.318196in}}%
\pgfpathquadraticcurveto{\pgfqpoint{4.854492in}{4.318196in}}{\pgfqpoint{4.854492in}{4.290419in}}%
\pgfpathlineto{\pgfqpoint{4.854492in}{3.896593in}}%
\pgfpathquadraticcurveto{\pgfqpoint{4.854492in}{3.868815in}}{\pgfqpoint{4.882270in}{3.868815in}}%
\pgfpathclose%
\pgfusepath{stroke,fill}%
\end{pgfscope}%
\begin{pgfscope}%
\pgfsetbuttcap%
\pgfsetmiterjoin%
\definecolor{currentfill}{rgb}{0.121569,0.466667,0.705882}%
\pgfsetfillcolor{currentfill}%
\pgfsetlinewidth{0.000000pt}%
\definecolor{currentstroke}{rgb}{0.000000,0.000000,0.000000}%
\pgfsetstrokecolor{currentstroke}%
\pgfsetstrokeopacity{0.000000}%
\pgfsetdash{}{0pt}%
\pgfpathmoveto{\pgfqpoint{4.910048in}{4.157118in}}%
\pgfpathlineto{\pgfqpoint{5.187825in}{4.157118in}}%
\pgfpathlineto{\pgfqpoint{5.187825in}{4.254340in}}%
\pgfpathlineto{\pgfqpoint{4.910048in}{4.254340in}}%
\pgfpathclose%
\pgfusepath{fill}%
\end{pgfscope}%
\begin{pgfscope}%
\definecolor{textcolor}{rgb}{0.000000,0.000000,0.000000}%
\pgfsetstrokecolor{textcolor}%
\pgfsetfillcolor{textcolor}%
\pgftext[x=5.298937in,y=4.157118in,left,base]{\color{textcolor}\sffamily\fontsize{10.000000}{12.000000}\selectfont Pix2Vox++}%
\end{pgfscope}%
\begin{pgfscope}%
\pgfsetbuttcap%
\pgfsetmiterjoin%
\definecolor{currentfill}{rgb}{1.000000,0.498039,0.054902}%
\pgfsetfillcolor{currentfill}%
\pgfsetlinewidth{0.000000pt}%
\definecolor{currentstroke}{rgb}{0.000000,0.000000,0.000000}%
\pgfsetstrokecolor{currentstroke}%
\pgfsetstrokeopacity{0.000000}%
\pgfsetdash{}{0pt}%
\pgfpathmoveto{\pgfqpoint{4.910048in}{3.953260in}}%
\pgfpathlineto{\pgfqpoint{5.187825in}{3.953260in}}%
\pgfpathlineto{\pgfqpoint{5.187825in}{4.050483in}}%
\pgfpathlineto{\pgfqpoint{4.910048in}{4.050483in}}%
\pgfpathclose%
\pgfusepath{fill}%
\end{pgfscope}%
\begin{pgfscope}%
\definecolor{textcolor}{rgb}{0.000000,0.000000,0.000000}%
\pgfsetstrokecolor{textcolor}%
\pgfsetfillcolor{textcolor}%
\pgftext[x=5.298937in,y=3.953260in,left,base]{\color{textcolor}\sffamily\fontsize{10.000000}{12.000000}\selectfont Pix2Vox}%
\end{pgfscope}%
\end{pgfpicture}%
\makeatother%
\endgroup%
}}\\
    \subfloat[][]{\resizebox{0.75\linewidth}{!}{%% Creator: Matplotlib, PGF backend
%%
%% To include the figure in your LaTeX document, write
%%   \input{<filename>.pgf}
%%
%% Make sure the required packages are loaded in your preamble
%%   \usepackage{pgf}
%%
%% Figures using additional raster images can only be included by \input if
%% they are in the same directory as the main LaTeX file. For loading figures
%% from other directories you can use the `import` package
%%   \usepackage{import}
%%
%% and then include the figures with
%%   \import{<path to file>}{<filename>.pgf}
%%
%% Matplotlib used the following preamble
%%   \usepackage{fontspec}
%%   \setmainfont{DejaVuSerif.ttf}[Path=\detokenize{/Users/apple/opt/anaconda3/envs/kaolin/lib/python3.7/site-packages/matplotlib/mpl-data/fonts/ttf/}]
%%   \setsansfont{DejaVuSans.ttf}[Path=\detokenize{/Users/apple/opt/anaconda3/envs/kaolin/lib/python3.7/site-packages/matplotlib/mpl-data/fonts/ttf/}]
%%   \setmonofont{DejaVuSansMono.ttf}[Path=\detokenize{/Users/apple/opt/anaconda3/envs/kaolin/lib/python3.7/site-packages/matplotlib/mpl-data/fonts/ttf/}]
%%
\begingroup%
\makeatletter%
\begin{pgfpicture}%
\pgfpathrectangle{\pgfpointorigin}{\pgfqpoint{6.293658in}{4.697602in}}%
\pgfusepath{use as bounding box, clip}%
\begin{pgfscope}%
\pgfsetbuttcap%
\pgfsetmiterjoin%
\definecolor{currentfill}{rgb}{1.000000,1.000000,1.000000}%
\pgfsetfillcolor{currentfill}%
\pgfsetlinewidth{0.000000pt}%
\definecolor{currentstroke}{rgb}{1.000000,1.000000,1.000000}%
\pgfsetstrokecolor{currentstroke}%
\pgfsetdash{}{0pt}%
\pgfpathmoveto{\pgfqpoint{-0.000000in}{0.000000in}}%
\pgfpathlineto{\pgfqpoint{6.293658in}{0.000000in}}%
\pgfpathlineto{\pgfqpoint{6.293658in}{4.697602in}}%
\pgfpathlineto{\pgfqpoint{-0.000000in}{4.697602in}}%
\pgfpathclose%
\pgfusepath{fill}%
\end{pgfscope}%
\begin{pgfscope}%
\pgfsetbuttcap%
\pgfsetmiterjoin%
\definecolor{currentfill}{rgb}{1.000000,1.000000,1.000000}%
\pgfsetfillcolor{currentfill}%
\pgfsetlinewidth{0.000000pt}%
\definecolor{currentstroke}{rgb}{0.000000,0.000000,0.000000}%
\pgfsetstrokecolor{currentstroke}%
\pgfsetstrokeopacity{0.000000}%
\pgfsetdash{}{0pt}%
\pgfpathmoveto{\pgfqpoint{0.696435in}{1.223552in}}%
\pgfpathlineto{\pgfqpoint{6.193658in}{1.223552in}}%
\pgfpathlineto{\pgfqpoint{6.193658in}{4.387641in}}%
\pgfpathlineto{\pgfqpoint{0.696435in}{4.387641in}}%
\pgfpathclose%
\pgfusepath{fill}%
\end{pgfscope}%
\begin{pgfscope}%
\pgfpathrectangle{\pgfqpoint{0.696435in}{1.223552in}}{\pgfqpoint{5.497222in}{3.164089in}}%
\pgfusepath{clip}%
\pgfsetbuttcap%
\pgfsetmiterjoin%
\definecolor{currentfill}{rgb}{0.121569,0.466667,0.705882}%
\pgfsetfillcolor{currentfill}%
\pgfsetlinewidth{0.000000pt}%
\definecolor{currentstroke}{rgb}{0.000000,0.000000,0.000000}%
\pgfsetstrokecolor{currentstroke}%
\pgfsetstrokeopacity{0.000000}%
\pgfsetdash{}{0pt}%
\pgfpathmoveto{\pgfqpoint{0.946309in}{1.223552in}}%
\pgfpathlineto{\pgfqpoint{1.405261in}{1.223552in}}%
\pgfpathlineto{\pgfqpoint{1.405261in}{3.420221in}}%
\pgfpathlineto{\pgfqpoint{0.946309in}{3.420221in}}%
\pgfpathclose%
\pgfusepath{fill}%
\end{pgfscope}%
\begin{pgfscope}%
\pgfpathrectangle{\pgfqpoint{0.696435in}{1.223552in}}{\pgfqpoint{5.497222in}{3.164089in}}%
\pgfusepath{clip}%
\pgfsetbuttcap%
\pgfsetmiterjoin%
\definecolor{currentfill}{rgb}{0.121569,0.466667,0.705882}%
\pgfsetfillcolor{currentfill}%
\pgfsetlinewidth{0.000000pt}%
\definecolor{currentstroke}{rgb}{0.000000,0.000000,0.000000}%
\pgfsetstrokecolor{currentstroke}%
\pgfsetstrokeopacity{0.000000}%
\pgfsetdash{}{0pt}%
\pgfpathmoveto{\pgfqpoint{1.966202in}{1.223552in}}%
\pgfpathlineto{\pgfqpoint{2.425154in}{1.223552in}}%
\pgfpathlineto{\pgfqpoint{2.425154in}{3.947041in}}%
\pgfpathlineto{\pgfqpoint{1.966202in}{3.947041in}}%
\pgfpathclose%
\pgfusepath{fill}%
\end{pgfscope}%
\begin{pgfscope}%
\pgfpathrectangle{\pgfqpoint{0.696435in}{1.223552in}}{\pgfqpoint{5.497222in}{3.164089in}}%
\pgfusepath{clip}%
\pgfsetbuttcap%
\pgfsetmiterjoin%
\definecolor{currentfill}{rgb}{0.121569,0.466667,0.705882}%
\pgfsetfillcolor{currentfill}%
\pgfsetlinewidth{0.000000pt}%
\definecolor{currentstroke}{rgb}{0.000000,0.000000,0.000000}%
\pgfsetstrokecolor{currentstroke}%
\pgfsetstrokeopacity{0.000000}%
\pgfsetdash{}{0pt}%
\pgfpathmoveto{\pgfqpoint{2.986095in}{1.223552in}}%
\pgfpathlineto{\pgfqpoint{3.445047in}{1.223552in}}%
\pgfpathlineto{\pgfqpoint{3.445047in}{1.679971in}}%
\pgfpathlineto{\pgfqpoint{2.986095in}{1.679971in}}%
\pgfpathclose%
\pgfusepath{fill}%
\end{pgfscope}%
\begin{pgfscope}%
\pgfpathrectangle{\pgfqpoint{0.696435in}{1.223552in}}{\pgfqpoint{5.497222in}{3.164089in}}%
\pgfusepath{clip}%
\pgfsetbuttcap%
\pgfsetmiterjoin%
\definecolor{currentfill}{rgb}{0.121569,0.466667,0.705882}%
\pgfsetfillcolor{currentfill}%
\pgfsetlinewidth{0.000000pt}%
\definecolor{currentstroke}{rgb}{0.000000,0.000000,0.000000}%
\pgfsetstrokecolor{currentstroke}%
\pgfsetstrokeopacity{0.000000}%
\pgfsetdash{}{0pt}%
\pgfpathmoveto{\pgfqpoint{4.005988in}{1.223552in}}%
\pgfpathlineto{\pgfqpoint{4.464939in}{1.223552in}}%
\pgfpathlineto{\pgfqpoint{4.464939in}{1.893548in}}%
\pgfpathlineto{\pgfqpoint{4.005988in}{1.893548in}}%
\pgfpathclose%
\pgfusepath{fill}%
\end{pgfscope}%
\begin{pgfscope}%
\pgfpathrectangle{\pgfqpoint{0.696435in}{1.223552in}}{\pgfqpoint{5.497222in}{3.164089in}}%
\pgfusepath{clip}%
\pgfsetbuttcap%
\pgfsetmiterjoin%
\definecolor{currentfill}{rgb}{0.121569,0.466667,0.705882}%
\pgfsetfillcolor{currentfill}%
\pgfsetlinewidth{0.000000pt}%
\definecolor{currentstroke}{rgb}{0.000000,0.000000,0.000000}%
\pgfsetstrokecolor{currentstroke}%
\pgfsetstrokeopacity{0.000000}%
\pgfsetdash{}{0pt}%
\pgfpathmoveto{\pgfqpoint{5.025880in}{1.223552in}}%
\pgfpathlineto{\pgfqpoint{5.484832in}{1.223552in}}%
\pgfpathlineto{\pgfqpoint{5.484832in}{2.249508in}}%
\pgfpathlineto{\pgfqpoint{5.025880in}{2.249508in}}%
\pgfpathclose%
\pgfusepath{fill}%
\end{pgfscope}%
\begin{pgfscope}%
\pgfpathrectangle{\pgfqpoint{0.696435in}{1.223552in}}{\pgfqpoint{5.497222in}{3.164089in}}%
\pgfusepath{clip}%
\pgfsetbuttcap%
\pgfsetmiterjoin%
\definecolor{currentfill}{rgb}{1.000000,0.498039,0.054902}%
\pgfsetfillcolor{currentfill}%
\pgfsetlinewidth{0.000000pt}%
\definecolor{currentstroke}{rgb}{0.000000,0.000000,0.000000}%
\pgfsetstrokecolor{currentstroke}%
\pgfsetstrokeopacity{0.000000}%
\pgfsetdash{}{0pt}%
\pgfpathmoveto{\pgfqpoint{1.405261in}{1.223552in}}%
\pgfpathlineto{\pgfqpoint{1.864213in}{1.223552in}}%
\pgfpathlineto{\pgfqpoint{1.864213in}{3.682049in}}%
\pgfpathlineto{\pgfqpoint{1.405261in}{3.682049in}}%
\pgfpathclose%
\pgfusepath{fill}%
\end{pgfscope}%
\begin{pgfscope}%
\pgfpathrectangle{\pgfqpoint{0.696435in}{1.223552in}}{\pgfqpoint{5.497222in}{3.164089in}}%
\pgfusepath{clip}%
\pgfsetbuttcap%
\pgfsetmiterjoin%
\definecolor{currentfill}{rgb}{1.000000,0.498039,0.054902}%
\pgfsetfillcolor{currentfill}%
\pgfsetlinewidth{0.000000pt}%
\definecolor{currentstroke}{rgb}{0.000000,0.000000,0.000000}%
\pgfsetstrokecolor{currentstroke}%
\pgfsetstrokeopacity{0.000000}%
\pgfsetdash{}{0pt}%
\pgfpathmoveto{\pgfqpoint{2.425154in}{1.223552in}}%
\pgfpathlineto{\pgfqpoint{2.884105in}{1.223552in}}%
\pgfpathlineto{\pgfqpoint{2.884105in}{3.794374in}}%
\pgfpathlineto{\pgfqpoint{2.425154in}{3.794374in}}%
\pgfpathclose%
\pgfusepath{fill}%
\end{pgfscope}%
\begin{pgfscope}%
\pgfpathrectangle{\pgfqpoint{0.696435in}{1.223552in}}{\pgfqpoint{5.497222in}{3.164089in}}%
\pgfusepath{clip}%
\pgfsetbuttcap%
\pgfsetmiterjoin%
\definecolor{currentfill}{rgb}{1.000000,0.498039,0.054902}%
\pgfsetfillcolor{currentfill}%
\pgfsetlinewidth{0.000000pt}%
\definecolor{currentstroke}{rgb}{0.000000,0.000000,0.000000}%
\pgfsetstrokecolor{currentstroke}%
\pgfsetstrokeopacity{0.000000}%
\pgfsetdash{}{0pt}%
\pgfpathmoveto{\pgfqpoint{3.445047in}{1.223552in}}%
\pgfpathlineto{\pgfqpoint{3.903998in}{1.223552in}}%
\pgfpathlineto{\pgfqpoint{3.903998in}{1.760656in}}%
\pgfpathlineto{\pgfqpoint{3.445047in}{1.760656in}}%
\pgfpathclose%
\pgfusepath{fill}%
\end{pgfscope}%
\begin{pgfscope}%
\pgfpathrectangle{\pgfqpoint{0.696435in}{1.223552in}}{\pgfqpoint{5.497222in}{3.164089in}}%
\pgfusepath{clip}%
\pgfsetbuttcap%
\pgfsetmiterjoin%
\definecolor{currentfill}{rgb}{1.000000,0.498039,0.054902}%
\pgfsetfillcolor{currentfill}%
\pgfsetlinewidth{0.000000pt}%
\definecolor{currentstroke}{rgb}{0.000000,0.000000,0.000000}%
\pgfsetstrokecolor{currentstroke}%
\pgfsetstrokeopacity{0.000000}%
\pgfsetdash{}{0pt}%
\pgfpathmoveto{\pgfqpoint{4.464939in}{1.223552in}}%
\pgfpathlineto{\pgfqpoint{4.923891in}{1.223552in}}%
\pgfpathlineto{\pgfqpoint{4.923891in}{1.675225in}}%
\pgfpathlineto{\pgfqpoint{4.464939in}{1.675225in}}%
\pgfpathclose%
\pgfusepath{fill}%
\end{pgfscope}%
\begin{pgfscope}%
\pgfpathrectangle{\pgfqpoint{0.696435in}{1.223552in}}{\pgfqpoint{5.497222in}{3.164089in}}%
\pgfusepath{clip}%
\pgfsetbuttcap%
\pgfsetmiterjoin%
\definecolor{currentfill}{rgb}{1.000000,0.498039,0.054902}%
\pgfsetfillcolor{currentfill}%
\pgfsetlinewidth{0.000000pt}%
\definecolor{currentstroke}{rgb}{0.000000,0.000000,0.000000}%
\pgfsetstrokecolor{currentstroke}%
\pgfsetstrokeopacity{0.000000}%
\pgfsetdash{}{0pt}%
\pgfpathmoveto{\pgfqpoint{5.484832in}{1.223552in}}%
\pgfpathlineto{\pgfqpoint{5.943784in}{1.223552in}}%
\pgfpathlineto{\pgfqpoint{5.943784in}{2.523201in}}%
\pgfpathlineto{\pgfqpoint{5.484832in}{2.523201in}}%
\pgfpathclose%
\pgfusepath{fill}%
\end{pgfscope}%
\begin{pgfscope}%
\pgfsetbuttcap%
\pgfsetroundjoin%
\definecolor{currentfill}{rgb}{0.000000,0.000000,0.000000}%
\pgfsetfillcolor{currentfill}%
\pgfsetlinewidth{0.803000pt}%
\definecolor{currentstroke}{rgb}{0.000000,0.000000,0.000000}%
\pgfsetstrokecolor{currentstroke}%
\pgfsetdash{}{0pt}%
\pgfsys@defobject{currentmarker}{\pgfqpoint{0.000000in}{-0.048611in}}{\pgfqpoint{0.000000in}{0.000000in}}{%
\pgfpathmoveto{\pgfqpoint{0.000000in}{0.000000in}}%
\pgfpathlineto{\pgfqpoint{0.000000in}{-0.048611in}}%
\pgfusepath{stroke,fill}%
}%
\begin{pgfscope}%
\pgfsys@transformshift{1.405261in}{1.223552in}%
\pgfsys@useobject{currentmarker}{}%
\end{pgfscope}%
\end{pgfscope}%
\begin{pgfscope}%
\definecolor{textcolor}{rgb}{0.000000,0.000000,0.000000}%
\pgfsetstrokecolor{textcolor}%
\pgfsetfillcolor{textcolor}%
\pgftext[x=1.091236in, y=0.369475in, left, base,rotate=45.000000]{\color{textcolor}\sffamily\fontsize{10.000000}{12.000000}\selectfont Pix3d(no aug)}%
\end{pgfscope}%
\begin{pgfscope}%
\pgfsetbuttcap%
\pgfsetroundjoin%
\definecolor{currentfill}{rgb}{0.000000,0.000000,0.000000}%
\pgfsetfillcolor{currentfill}%
\pgfsetlinewidth{0.803000pt}%
\definecolor{currentstroke}{rgb}{0.000000,0.000000,0.000000}%
\pgfsetstrokecolor{currentstroke}%
\pgfsetdash{}{0pt}%
\pgfsys@defobject{currentmarker}{\pgfqpoint{0.000000in}{-0.048611in}}{\pgfqpoint{0.000000in}{0.000000in}}{%
\pgfpathmoveto{\pgfqpoint{0.000000in}{0.000000in}}%
\pgfpathlineto{\pgfqpoint{0.000000in}{-0.048611in}}%
\pgfusepath{stroke,fill}%
}%
\begin{pgfscope}%
\pgfsys@transformshift{2.425154in}{1.223552in}%
\pgfsys@useobject{currentmarker}{}%
\end{pgfscope}%
\end{pgfscope}%
\begin{pgfscope}%
\definecolor{textcolor}{rgb}{0.000000,0.000000,0.000000}%
\pgfsetstrokecolor{textcolor}%
\pgfsetfillcolor{textcolor}%
\pgftext[x=2.318601in, y=0.784419in, left, base,rotate=45.000000]{\color{textcolor}\sffamily\fontsize{10.000000}{12.000000}\selectfont Pix3d}%
\end{pgfscope}%
\begin{pgfscope}%
\pgfsetbuttcap%
\pgfsetroundjoin%
\definecolor{currentfill}{rgb}{0.000000,0.000000,0.000000}%
\pgfsetfillcolor{currentfill}%
\pgfsetlinewidth{0.803000pt}%
\definecolor{currentstroke}{rgb}{0.000000,0.000000,0.000000}%
\pgfsetstrokecolor{currentstroke}%
\pgfsetdash{}{0pt}%
\pgfsys@defobject{currentmarker}{\pgfqpoint{0.000000in}{-0.048611in}}{\pgfqpoint{0.000000in}{0.000000in}}{%
\pgfpathmoveto{\pgfqpoint{0.000000in}{0.000000in}}%
\pgfpathlineto{\pgfqpoint{0.000000in}{-0.048611in}}%
\pgfusepath{stroke,fill}%
}%
\begin{pgfscope}%
\pgfsys@transformshift{3.445047in}{1.223552in}%
\pgfsys@useobject{currentmarker}{}%
\end{pgfscope}%
\end{pgfscope}%
\begin{pgfscope}%
\definecolor{textcolor}{rgb}{0.000000,0.000000,0.000000}%
\pgfsetstrokecolor{textcolor}%
\pgfsetfillcolor{textcolor}%
\pgftext[x=3.101386in, y=0.313130in, left, base,rotate=45.000000]{\color{textcolor}\sffamily\fontsize{10.000000}{12.000000}\selectfont s2r\_v1(no aug)}%
\end{pgfscope}%
\begin{pgfscope}%
\pgfsetbuttcap%
\pgfsetroundjoin%
\definecolor{currentfill}{rgb}{0.000000,0.000000,0.000000}%
\pgfsetfillcolor{currentfill}%
\pgfsetlinewidth{0.803000pt}%
\definecolor{currentstroke}{rgb}{0.000000,0.000000,0.000000}%
\pgfsetstrokecolor{currentstroke}%
\pgfsetdash{}{0pt}%
\pgfsys@defobject{currentmarker}{\pgfqpoint{0.000000in}{-0.048611in}}{\pgfqpoint{0.000000in}{0.000000in}}{%
\pgfpathmoveto{\pgfqpoint{0.000000in}{0.000000in}}%
\pgfpathlineto{\pgfqpoint{0.000000in}{-0.048611in}}%
\pgfusepath{stroke,fill}%
}%
\begin{pgfscope}%
\pgfsys@transformshift{4.464939in}{1.223552in}%
\pgfsys@useobject{currentmarker}{}%
\end{pgfscope}%
\end{pgfscope}%
\begin{pgfscope}%
\definecolor{textcolor}{rgb}{0.000000,0.000000,0.000000}%
\pgfsetstrokecolor{textcolor}%
\pgfsetfillcolor{textcolor}%
\pgftext[x=4.327936in, y=0.729704in, left, base,rotate=45.000000]{\color{textcolor}\sffamily\fontsize{10.000000}{12.000000}\selectfont s2r\_v1}%
\end{pgfscope}%
\begin{pgfscope}%
\pgfsetbuttcap%
\pgfsetroundjoin%
\definecolor{currentfill}{rgb}{0.000000,0.000000,0.000000}%
\pgfsetfillcolor{currentfill}%
\pgfsetlinewidth{0.803000pt}%
\definecolor{currentstroke}{rgb}{0.000000,0.000000,0.000000}%
\pgfsetstrokecolor{currentstroke}%
\pgfsetdash{}{0pt}%
\pgfsys@defobject{currentmarker}{\pgfqpoint{0.000000in}{-0.048611in}}{\pgfqpoint{0.000000in}{0.000000in}}{%
\pgfpathmoveto{\pgfqpoint{0.000000in}{0.000000in}}%
\pgfpathlineto{\pgfqpoint{0.000000in}{-0.048611in}}%
\pgfusepath{stroke,fill}%
}%
\begin{pgfscope}%
\pgfsys@transformshift{5.484832in}{1.223552in}%
\pgfsys@useobject{currentmarker}{}%
\end{pgfscope}%
\end{pgfscope}%
\begin{pgfscope}%
\definecolor{textcolor}{rgb}{0.000000,0.000000,0.000000}%
\pgfsetstrokecolor{textcolor}%
\pgfsetfillcolor{textcolor}%
\pgftext[x=5.347828in, y=0.729704in, left, base,rotate=45.000000]{\color{textcolor}\sffamily\fontsize{10.000000}{12.000000}\selectfont s2r\_v2}%
\end{pgfscope}%
\begin{pgfscope}%
\definecolor{textcolor}{rgb}{0.000000,0.000000,0.000000}%
\pgfsetstrokecolor{textcolor}%
\pgfsetfillcolor{textcolor}%
\pgftext[x=3.445047in,y=0.234413in,,top]{\color{textcolor}\sffamily\fontsize{10.000000}{12.000000}\selectfont Dataset}%
\end{pgfscope}%
\begin{pgfscope}%
\pgfsetbuttcap%
\pgfsetroundjoin%
\definecolor{currentfill}{rgb}{0.000000,0.000000,0.000000}%
\pgfsetfillcolor{currentfill}%
\pgfsetlinewidth{0.803000pt}%
\definecolor{currentstroke}{rgb}{0.000000,0.000000,0.000000}%
\pgfsetstrokecolor{currentstroke}%
\pgfsetdash{}{0pt}%
\pgfsys@defobject{currentmarker}{\pgfqpoint{-0.048611in}{0.000000in}}{\pgfqpoint{-0.000000in}{0.000000in}}{%
\pgfpathmoveto{\pgfqpoint{-0.000000in}{0.000000in}}%
\pgfpathlineto{\pgfqpoint{-0.048611in}{0.000000in}}%
\pgfusepath{stroke,fill}%
}%
\begin{pgfscope}%
\pgfsys@transformshift{0.696435in}{1.223552in}%
\pgfsys@useobject{currentmarker}{}%
\end{pgfscope}%
\end{pgfscope}%
\begin{pgfscope}%
\definecolor{textcolor}{rgb}{0.000000,0.000000,0.000000}%
\pgfsetstrokecolor{textcolor}%
\pgfsetfillcolor{textcolor}%
\pgftext[x=0.289968in, y=1.170790in, left, base]{\color{textcolor}\sffamily\fontsize{10.000000}{12.000000}\selectfont 0.00}%
\end{pgfscope}%
\begin{pgfscope}%
\pgfsetbuttcap%
\pgfsetroundjoin%
\definecolor{currentfill}{rgb}{0.000000,0.000000,0.000000}%
\pgfsetfillcolor{currentfill}%
\pgfsetlinewidth{0.803000pt}%
\definecolor{currentstroke}{rgb}{0.000000,0.000000,0.000000}%
\pgfsetstrokecolor{currentstroke}%
\pgfsetdash{}{0pt}%
\pgfsys@defobject{currentmarker}{\pgfqpoint{-0.048611in}{0.000000in}}{\pgfqpoint{-0.000000in}{0.000000in}}{%
\pgfpathmoveto{\pgfqpoint{-0.000000in}{0.000000in}}%
\pgfpathlineto{\pgfqpoint{-0.048611in}{0.000000in}}%
\pgfusepath{stroke,fill}%
}%
\begin{pgfscope}%
\pgfsys@transformshift{0.696435in}{1.619063in}%
\pgfsys@useobject{currentmarker}{}%
\end{pgfscope}%
\end{pgfscope}%
\begin{pgfscope}%
\definecolor{textcolor}{rgb}{0.000000,0.000000,0.000000}%
\pgfsetstrokecolor{textcolor}%
\pgfsetfillcolor{textcolor}%
\pgftext[x=0.289968in, y=1.566301in, left, base]{\color{textcolor}\sffamily\fontsize{10.000000}{12.000000}\selectfont 0.05}%
\end{pgfscope}%
\begin{pgfscope}%
\pgfsetbuttcap%
\pgfsetroundjoin%
\definecolor{currentfill}{rgb}{0.000000,0.000000,0.000000}%
\pgfsetfillcolor{currentfill}%
\pgfsetlinewidth{0.803000pt}%
\definecolor{currentstroke}{rgb}{0.000000,0.000000,0.000000}%
\pgfsetstrokecolor{currentstroke}%
\pgfsetdash{}{0pt}%
\pgfsys@defobject{currentmarker}{\pgfqpoint{-0.048611in}{0.000000in}}{\pgfqpoint{-0.000000in}{0.000000in}}{%
\pgfpathmoveto{\pgfqpoint{-0.000000in}{0.000000in}}%
\pgfpathlineto{\pgfqpoint{-0.048611in}{0.000000in}}%
\pgfusepath{stroke,fill}%
}%
\begin{pgfscope}%
\pgfsys@transformshift{0.696435in}{2.014574in}%
\pgfsys@useobject{currentmarker}{}%
\end{pgfscope}%
\end{pgfscope}%
\begin{pgfscope}%
\definecolor{textcolor}{rgb}{0.000000,0.000000,0.000000}%
\pgfsetstrokecolor{textcolor}%
\pgfsetfillcolor{textcolor}%
\pgftext[x=0.289968in, y=1.961812in, left, base]{\color{textcolor}\sffamily\fontsize{10.000000}{12.000000}\selectfont 0.10}%
\end{pgfscope}%
\begin{pgfscope}%
\pgfsetbuttcap%
\pgfsetroundjoin%
\definecolor{currentfill}{rgb}{0.000000,0.000000,0.000000}%
\pgfsetfillcolor{currentfill}%
\pgfsetlinewidth{0.803000pt}%
\definecolor{currentstroke}{rgb}{0.000000,0.000000,0.000000}%
\pgfsetstrokecolor{currentstroke}%
\pgfsetdash{}{0pt}%
\pgfsys@defobject{currentmarker}{\pgfqpoint{-0.048611in}{0.000000in}}{\pgfqpoint{-0.000000in}{0.000000in}}{%
\pgfpathmoveto{\pgfqpoint{-0.000000in}{0.000000in}}%
\pgfpathlineto{\pgfqpoint{-0.048611in}{0.000000in}}%
\pgfusepath{stroke,fill}%
}%
\begin{pgfscope}%
\pgfsys@transformshift{0.696435in}{2.410085in}%
\pgfsys@useobject{currentmarker}{}%
\end{pgfscope}%
\end{pgfscope}%
\begin{pgfscope}%
\definecolor{textcolor}{rgb}{0.000000,0.000000,0.000000}%
\pgfsetstrokecolor{textcolor}%
\pgfsetfillcolor{textcolor}%
\pgftext[x=0.289968in, y=2.357324in, left, base]{\color{textcolor}\sffamily\fontsize{10.000000}{12.000000}\selectfont 0.15}%
\end{pgfscope}%
\begin{pgfscope}%
\pgfsetbuttcap%
\pgfsetroundjoin%
\definecolor{currentfill}{rgb}{0.000000,0.000000,0.000000}%
\pgfsetfillcolor{currentfill}%
\pgfsetlinewidth{0.803000pt}%
\definecolor{currentstroke}{rgb}{0.000000,0.000000,0.000000}%
\pgfsetstrokecolor{currentstroke}%
\pgfsetdash{}{0pt}%
\pgfsys@defobject{currentmarker}{\pgfqpoint{-0.048611in}{0.000000in}}{\pgfqpoint{-0.000000in}{0.000000in}}{%
\pgfpathmoveto{\pgfqpoint{-0.000000in}{0.000000in}}%
\pgfpathlineto{\pgfqpoint{-0.048611in}{0.000000in}}%
\pgfusepath{stroke,fill}%
}%
\begin{pgfscope}%
\pgfsys@transformshift{0.696435in}{2.805596in}%
\pgfsys@useobject{currentmarker}{}%
\end{pgfscope}%
\end{pgfscope}%
\begin{pgfscope}%
\definecolor{textcolor}{rgb}{0.000000,0.000000,0.000000}%
\pgfsetstrokecolor{textcolor}%
\pgfsetfillcolor{textcolor}%
\pgftext[x=0.289968in, y=2.752835in, left, base]{\color{textcolor}\sffamily\fontsize{10.000000}{12.000000}\selectfont 0.20}%
\end{pgfscope}%
\begin{pgfscope}%
\pgfsetbuttcap%
\pgfsetroundjoin%
\definecolor{currentfill}{rgb}{0.000000,0.000000,0.000000}%
\pgfsetfillcolor{currentfill}%
\pgfsetlinewidth{0.803000pt}%
\definecolor{currentstroke}{rgb}{0.000000,0.000000,0.000000}%
\pgfsetstrokecolor{currentstroke}%
\pgfsetdash{}{0pt}%
\pgfsys@defobject{currentmarker}{\pgfqpoint{-0.048611in}{0.000000in}}{\pgfqpoint{-0.000000in}{0.000000in}}{%
\pgfpathmoveto{\pgfqpoint{-0.000000in}{0.000000in}}%
\pgfpathlineto{\pgfqpoint{-0.048611in}{0.000000in}}%
\pgfusepath{stroke,fill}%
}%
\begin{pgfscope}%
\pgfsys@transformshift{0.696435in}{3.201107in}%
\pgfsys@useobject{currentmarker}{}%
\end{pgfscope}%
\end{pgfscope}%
\begin{pgfscope}%
\definecolor{textcolor}{rgb}{0.000000,0.000000,0.000000}%
\pgfsetstrokecolor{textcolor}%
\pgfsetfillcolor{textcolor}%
\pgftext[x=0.289968in, y=3.148346in, left, base]{\color{textcolor}\sffamily\fontsize{10.000000}{12.000000}\selectfont 0.25}%
\end{pgfscope}%
\begin{pgfscope}%
\pgfsetbuttcap%
\pgfsetroundjoin%
\definecolor{currentfill}{rgb}{0.000000,0.000000,0.000000}%
\pgfsetfillcolor{currentfill}%
\pgfsetlinewidth{0.803000pt}%
\definecolor{currentstroke}{rgb}{0.000000,0.000000,0.000000}%
\pgfsetstrokecolor{currentstroke}%
\pgfsetdash{}{0pt}%
\pgfsys@defobject{currentmarker}{\pgfqpoint{-0.048611in}{0.000000in}}{\pgfqpoint{-0.000000in}{0.000000in}}{%
\pgfpathmoveto{\pgfqpoint{-0.000000in}{0.000000in}}%
\pgfpathlineto{\pgfqpoint{-0.048611in}{0.000000in}}%
\pgfusepath{stroke,fill}%
}%
\begin{pgfscope}%
\pgfsys@transformshift{0.696435in}{3.596618in}%
\pgfsys@useobject{currentmarker}{}%
\end{pgfscope}%
\end{pgfscope}%
\begin{pgfscope}%
\definecolor{textcolor}{rgb}{0.000000,0.000000,0.000000}%
\pgfsetstrokecolor{textcolor}%
\pgfsetfillcolor{textcolor}%
\pgftext[x=0.289968in, y=3.543857in, left, base]{\color{textcolor}\sffamily\fontsize{10.000000}{12.000000}\selectfont 0.30}%
\end{pgfscope}%
\begin{pgfscope}%
\pgfsetbuttcap%
\pgfsetroundjoin%
\definecolor{currentfill}{rgb}{0.000000,0.000000,0.000000}%
\pgfsetfillcolor{currentfill}%
\pgfsetlinewidth{0.803000pt}%
\definecolor{currentstroke}{rgb}{0.000000,0.000000,0.000000}%
\pgfsetstrokecolor{currentstroke}%
\pgfsetdash{}{0pt}%
\pgfsys@defobject{currentmarker}{\pgfqpoint{-0.048611in}{0.000000in}}{\pgfqpoint{-0.000000in}{0.000000in}}{%
\pgfpathmoveto{\pgfqpoint{-0.000000in}{0.000000in}}%
\pgfpathlineto{\pgfqpoint{-0.048611in}{0.000000in}}%
\pgfusepath{stroke,fill}%
}%
\begin{pgfscope}%
\pgfsys@transformshift{0.696435in}{3.992130in}%
\pgfsys@useobject{currentmarker}{}%
\end{pgfscope}%
\end{pgfscope}%
\begin{pgfscope}%
\definecolor{textcolor}{rgb}{0.000000,0.000000,0.000000}%
\pgfsetstrokecolor{textcolor}%
\pgfsetfillcolor{textcolor}%
\pgftext[x=0.289968in, y=3.939368in, left, base]{\color{textcolor}\sffamily\fontsize{10.000000}{12.000000}\selectfont 0.35}%
\end{pgfscope}%
\begin{pgfscope}%
\pgfsetbuttcap%
\pgfsetroundjoin%
\definecolor{currentfill}{rgb}{0.000000,0.000000,0.000000}%
\pgfsetfillcolor{currentfill}%
\pgfsetlinewidth{0.803000pt}%
\definecolor{currentstroke}{rgb}{0.000000,0.000000,0.000000}%
\pgfsetstrokecolor{currentstroke}%
\pgfsetdash{}{0pt}%
\pgfsys@defobject{currentmarker}{\pgfqpoint{-0.048611in}{0.000000in}}{\pgfqpoint{-0.000000in}{0.000000in}}{%
\pgfpathmoveto{\pgfqpoint{-0.000000in}{0.000000in}}%
\pgfpathlineto{\pgfqpoint{-0.048611in}{0.000000in}}%
\pgfusepath{stroke,fill}%
}%
\begin{pgfscope}%
\pgfsys@transformshift{0.696435in}{4.387641in}%
\pgfsys@useobject{currentmarker}{}%
\end{pgfscope}%
\end{pgfscope}%
\begin{pgfscope}%
\definecolor{textcolor}{rgb}{0.000000,0.000000,0.000000}%
\pgfsetstrokecolor{textcolor}%
\pgfsetfillcolor{textcolor}%
\pgftext[x=0.289968in, y=4.334879in, left, base]{\color{textcolor}\sffamily\fontsize{10.000000}{12.000000}\selectfont 0.40}%
\end{pgfscope}%
\begin{pgfscope}%
\definecolor{textcolor}{rgb}{0.000000,0.000000,0.000000}%
\pgfsetstrokecolor{textcolor}%
\pgfsetfillcolor{textcolor}%
\pgftext[x=0.234413in,y=2.805596in,,bottom,rotate=90.000000]{\color{textcolor}\sffamily\fontsize{10.000000}{12.000000}\selectfont IoU}%
\end{pgfscope}%
\begin{pgfscope}%
\pgfsetrectcap%
\pgfsetmiterjoin%
\pgfsetlinewidth{0.803000pt}%
\definecolor{currentstroke}{rgb}{0.000000,0.000000,0.000000}%
\pgfsetstrokecolor{currentstroke}%
\pgfsetdash{}{0pt}%
\pgfpathmoveto{\pgfqpoint{0.696435in}{1.223552in}}%
\pgfpathlineto{\pgfqpoint{0.696435in}{4.387641in}}%
\pgfusepath{stroke}%
\end{pgfscope}%
\begin{pgfscope}%
\pgfsetrectcap%
\pgfsetmiterjoin%
\pgfsetlinewidth{0.803000pt}%
\definecolor{currentstroke}{rgb}{0.000000,0.000000,0.000000}%
\pgfsetstrokecolor{currentstroke}%
\pgfsetdash{}{0pt}%
\pgfpathmoveto{\pgfqpoint{6.193658in}{1.223552in}}%
\pgfpathlineto{\pgfqpoint{6.193658in}{4.387641in}}%
\pgfusepath{stroke}%
\end{pgfscope}%
\begin{pgfscope}%
\pgfsetrectcap%
\pgfsetmiterjoin%
\pgfsetlinewidth{0.803000pt}%
\definecolor{currentstroke}{rgb}{0.000000,0.000000,0.000000}%
\pgfsetstrokecolor{currentstroke}%
\pgfsetdash{}{0pt}%
\pgfpathmoveto{\pgfqpoint{0.696435in}{1.223552in}}%
\pgfpathlineto{\pgfqpoint{6.193658in}{1.223552in}}%
\pgfusepath{stroke}%
\end{pgfscope}%
\begin{pgfscope}%
\pgfsetrectcap%
\pgfsetmiterjoin%
\pgfsetlinewidth{0.803000pt}%
\definecolor{currentstroke}{rgb}{0.000000,0.000000,0.000000}%
\pgfsetstrokecolor{currentstroke}%
\pgfsetdash{}{0pt}%
\pgfpathmoveto{\pgfqpoint{0.696435in}{4.387641in}}%
\pgfpathlineto{\pgfqpoint{6.193658in}{4.387641in}}%
\pgfusepath{stroke}%
\end{pgfscope}%
\begin{pgfscope}%
\definecolor{textcolor}{rgb}{0.000000,0.000000,0.000000}%
\pgfsetstrokecolor{textcolor}%
\pgfsetfillcolor{textcolor}%
\pgftext[x=1.175785in,y=3.461887in,,bottom]{\color{textcolor}\sffamily\fontsize{9.000000}{10.800000}\selectfont 0.2777}%
\end{pgfscope}%
\begin{pgfscope}%
\definecolor{textcolor}{rgb}{0.000000,0.000000,0.000000}%
\pgfsetstrokecolor{textcolor}%
\pgfsetfillcolor{textcolor}%
\pgftext[x=2.195678in,y=3.988708in,,bottom]{\color{textcolor}\sffamily\fontsize{9.000000}{10.800000}\selectfont 0.3443}%
\end{pgfscope}%
\begin{pgfscope}%
\definecolor{textcolor}{rgb}{0.000000,0.000000,0.000000}%
\pgfsetstrokecolor{textcolor}%
\pgfsetfillcolor{textcolor}%
\pgftext[x=3.215571in,y=1.721638in,,bottom]{\color{textcolor}\sffamily\fontsize{9.000000}{10.800000}\selectfont 0.0577}%
\end{pgfscope}%
\begin{pgfscope}%
\definecolor{textcolor}{rgb}{0.000000,0.000000,0.000000}%
\pgfsetstrokecolor{textcolor}%
\pgfsetfillcolor{textcolor}%
\pgftext[x=4.235463in,y=1.935214in,,bottom]{\color{textcolor}\sffamily\fontsize{9.000000}{10.800000}\selectfont 0.0847}%
\end{pgfscope}%
\begin{pgfscope}%
\definecolor{textcolor}{rgb}{0.000000,0.000000,0.000000}%
\pgfsetstrokecolor{textcolor}%
\pgfsetfillcolor{textcolor}%
\pgftext[x=5.255356in,y=2.291174in,,bottom]{\color{textcolor}\sffamily\fontsize{9.000000}{10.800000}\selectfont 0.1297}%
\end{pgfscope}%
\begin{pgfscope}%
\definecolor{textcolor}{rgb}{0.000000,0.000000,0.000000}%
\pgfsetstrokecolor{textcolor}%
\pgfsetfillcolor{textcolor}%
\pgftext[x=1.634737in,y=3.723716in,,bottom]{\color{textcolor}\sffamily\fontsize{9.000000}{10.800000}\selectfont 0.3108}%
\end{pgfscope}%
\begin{pgfscope}%
\definecolor{textcolor}{rgb}{0.000000,0.000000,0.000000}%
\pgfsetstrokecolor{textcolor}%
\pgfsetfillcolor{textcolor}%
\pgftext[x=2.654630in,y=3.836041in,,bottom]{\color{textcolor}\sffamily\fontsize{9.000000}{10.800000}\selectfont 0.325}%
\end{pgfscope}%
\begin{pgfscope}%
\definecolor{textcolor}{rgb}{0.000000,0.000000,0.000000}%
\pgfsetstrokecolor{textcolor}%
\pgfsetfillcolor{textcolor}%
\pgftext[x=3.674522in,y=1.802322in,,bottom]{\color{textcolor}\sffamily\fontsize{9.000000}{10.800000}\selectfont 0.0679}%
\end{pgfscope}%
\begin{pgfscope}%
\definecolor{textcolor}{rgb}{0.000000,0.000000,0.000000}%
\pgfsetstrokecolor{textcolor}%
\pgfsetfillcolor{textcolor}%
\pgftext[x=4.694415in,y=1.716892in,,bottom]{\color{textcolor}\sffamily\fontsize{9.000000}{10.800000}\selectfont 0.0571}%
\end{pgfscope}%
\begin{pgfscope}%
\definecolor{textcolor}{rgb}{0.000000,0.000000,0.000000}%
\pgfsetstrokecolor{textcolor}%
\pgfsetfillcolor{textcolor}%
\pgftext[x=5.714308in,y=2.564868in,,bottom]{\color{textcolor}\sffamily\fontsize{9.000000}{10.800000}\selectfont 0.1643}%
\end{pgfscope}%
\begin{pgfscope}%
\definecolor{textcolor}{rgb}{0.000000,0.000000,0.000000}%
\pgfsetstrokecolor{textcolor}%
\pgfsetfillcolor{textcolor}%
\pgftext[x=3.445047in,y=4.470974in,,base]{\color{textcolor}\sffamily\fontsize{12.000000}{14.400000}\selectfont Baselines trained on Pix3D and S2R:3DFREE}%
\end{pgfscope}%
\begin{pgfscope}%
\pgfsetbuttcap%
\pgfsetmiterjoin%
\definecolor{currentfill}{rgb}{1.000000,1.000000,1.000000}%
\pgfsetfillcolor{currentfill}%
\pgfsetfillopacity{0.800000}%
\pgfsetlinewidth{1.003750pt}%
\definecolor{currentstroke}{rgb}{0.800000,0.800000,0.800000}%
\pgfsetstrokecolor{currentstroke}%
\pgfsetstrokeopacity{0.800000}%
\pgfsetdash{}{0pt}%
\pgfpathmoveto{\pgfqpoint{4.882270in}{3.868815in}}%
\pgfpathlineto{\pgfqpoint{6.096435in}{3.868815in}}%
\pgfpathquadraticcurveto{\pgfqpoint{6.124213in}{3.868815in}}{\pgfqpoint{6.124213in}{3.896593in}}%
\pgfpathlineto{\pgfqpoint{6.124213in}{4.290419in}}%
\pgfpathquadraticcurveto{\pgfqpoint{6.124213in}{4.318196in}}{\pgfqpoint{6.096435in}{4.318196in}}%
\pgfpathlineto{\pgfqpoint{4.882270in}{4.318196in}}%
\pgfpathquadraticcurveto{\pgfqpoint{4.854492in}{4.318196in}}{\pgfqpoint{4.854492in}{4.290419in}}%
\pgfpathlineto{\pgfqpoint{4.854492in}{3.896593in}}%
\pgfpathquadraticcurveto{\pgfqpoint{4.854492in}{3.868815in}}{\pgfqpoint{4.882270in}{3.868815in}}%
\pgfpathclose%
\pgfusepath{stroke,fill}%
\end{pgfscope}%
\begin{pgfscope}%
\pgfsetbuttcap%
\pgfsetmiterjoin%
\definecolor{currentfill}{rgb}{0.121569,0.466667,0.705882}%
\pgfsetfillcolor{currentfill}%
\pgfsetlinewidth{0.000000pt}%
\definecolor{currentstroke}{rgb}{0.000000,0.000000,0.000000}%
\pgfsetstrokecolor{currentstroke}%
\pgfsetstrokeopacity{0.000000}%
\pgfsetdash{}{0pt}%
\pgfpathmoveto{\pgfqpoint{4.910048in}{4.157118in}}%
\pgfpathlineto{\pgfqpoint{5.187825in}{4.157118in}}%
\pgfpathlineto{\pgfqpoint{5.187825in}{4.254340in}}%
\pgfpathlineto{\pgfqpoint{4.910048in}{4.254340in}}%
\pgfpathclose%
\pgfusepath{fill}%
\end{pgfscope}%
\begin{pgfscope}%
\definecolor{textcolor}{rgb}{0.000000,0.000000,0.000000}%
\pgfsetstrokecolor{textcolor}%
\pgfsetfillcolor{textcolor}%
\pgftext[x=5.298937in,y=4.157118in,left,base]{\color{textcolor}\sffamily\fontsize{10.000000}{12.000000}\selectfont Pix2Vox++}%
\end{pgfscope}%
\begin{pgfscope}%
\pgfsetbuttcap%
\pgfsetmiterjoin%
\definecolor{currentfill}{rgb}{1.000000,0.498039,0.054902}%
\pgfsetfillcolor{currentfill}%
\pgfsetlinewidth{0.000000pt}%
\definecolor{currentstroke}{rgb}{0.000000,0.000000,0.000000}%
\pgfsetstrokecolor{currentstroke}%
\pgfsetstrokeopacity{0.000000}%
\pgfsetdash{}{0pt}%
\pgfpathmoveto{\pgfqpoint{4.910048in}{3.953260in}}%
\pgfpathlineto{\pgfqpoint{5.187825in}{3.953260in}}%
\pgfpathlineto{\pgfqpoint{5.187825in}{4.050483in}}%
\pgfpathlineto{\pgfqpoint{4.910048in}{4.050483in}}%
\pgfpathclose%
\pgfusepath{fill}%
\end{pgfscope}%
\begin{pgfscope}%
\definecolor{textcolor}{rgb}{0.000000,0.000000,0.000000}%
\pgfsetstrokecolor{textcolor}%
\pgfsetfillcolor{textcolor}%
\pgftext[x=5.298937in,y=3.953260in,left,base]{\color{textcolor}\sffamily\fontsize{10.000000}{12.000000}\selectfont Pix2Vox}%
\end{pgfscope}%
\end{pgfpicture}%
\makeatother%
\endgroup%
}}\\
    \caption[\gls{iou} Comparison for Baselines.]{Bar plot for the \gls{iou}  for \textbf{baselines} trained on real and synthetic datasets, with and without 2D augmentation.
        (a)The checkpoint was saved using real dataset for validation and test, (b) The checkpoint was saved using corresponding synthetic data for validation step and tested with real data.
        In both the cases we see that ~\gls{free} does not perform adequately on its own. \gls{s2rv2} contributes better than \gls{s2rv1}.}
    \label{fig:baseline1}
\end{figure}

\autoref{fig:baseline1}(a) is a comparison of models trained with 2D augmentation, synthetic dataset, and real dataset.
For models validated with the real dataset, it is seen that 2D augmentation improves \gls{iou}  by 8.06\% for Pix3D on Pix2Vox++ and 1.22\% on Pix2Vox.
In the case of the synthetic dataset, 2D augmentation increases the \gls{iou}  by 1.42\% and 6.66\% for Pix2Vox++ and Pix2Vox, respectively.
When it comes to whether a synthetic dataset gives an equivalent performance as a real dataset, we can see a dip in the performance, demonstrating that there is a domain gap between real and synthetic data.
Out of the two synthetic datasets, \gls{s2rv2} gives better results than \gls{s2rv1} when tested with real data.
We hypothesize that since Pix3D has multi-objects in the scenes, same as \gls{s2rv2}, it performs better than single object images from \gls{s2rv1}.
A similar observation is seen when a checkpoint is saved with the synthetic dataset itself being the validation set as in \autoref{fig:baseline1}(b).

A study with \gls{f1} for the baselines is discussed in \autoref{subsec:baseline_dice}.
We observe similar behaviour even with \gls{f1}.

\begin{figure}[!ht]
    \begin{tabular}{llll}
        Pix3D images & \includegraphics[width=.2\linewidth]{/Users/apple/OVGU/Thesis/code/3dReconstruction/report/images/evaluation/reconstruction/baseline/bed1} &
        \includegraphics[width=.2\linewidth]{/Users/apple/OVGU/Thesis/code/3dReconstruction/report/images/evaluation/reconstruction/baseline/sofa1} &
        \includegraphics[width=.2\linewidth]{/Users/apple/OVGU/Thesis/code/3dReconstruction/report/images/evaluation/reconstruction/baseline/table2}\\

        Ground Truth & \includegraphics[trim={0 0 {.1\width} 0},clip,width=.2\linewidth]{/Users/apple/OVGU/Thesis/code/3dReconstruction/report/images/evaluation/reconstruction/baseline/bed1_original} &
        \includegraphics[trim={0 0 {.1\width} 0},clip,width=.2\linewidth]{/Users/apple/OVGU/Thesis/code/3dReconstruction/report/images/evaluation/reconstruction/baseline/sofa1_original} &
        \includegraphics[trim={0 0 {.1\width} 0},clip,width=.2\linewidth]{/Users/apple/OVGU/Thesis/code/3dReconstruction/report/images/evaluation/reconstruction/baseline/table2_original}\\

        Output1 & \includegraphics[width=.2\linewidth]{/Users/apple/OVGU/Thesis/code/3dReconstruction/report/images/evaluation/reconstruction/baseline/pix3d_p2vpp_bed1_output} &
        \includegraphics[width=.2\linewidth]{/Users/apple/OVGU/Thesis/code/3dReconstruction/report/images/evaluation/reconstruction/baseline/pix3d_p2vpp_sofa1_output} &
        \includegraphics[width=.2\linewidth]{/Users/apple/OVGU/Thesis/code/3dReconstruction/report/images/evaluation/reconstruction/baseline/pix3d_p2vpp_table2}\\

        Output2 & \includegraphics[width=.2\linewidth]{/Users/apple/OVGU/Thesis/code/3dReconstruction/report/images/evaluation/reconstruction/baseline/pix3d_p2v_bed1} &
        \includegraphics[width=.2\linewidth]{/Users/apple/OVGU/Thesis/code/3dReconstruction/report/images/evaluation/reconstruction/baseline/pix3d_p2v_sofa1} &
        \includegraphics[width=.2\linewidth]{/Users/apple/OVGU/Thesis/code/3dReconstruction/report/images/evaluation/reconstruction/baseline/pix3d_p2v_table2}\\

        Output3 & \includegraphics[width=.2\linewidth]{/Users/apple/OVGU/Thesis/code/3dReconstruction/report/images/evaluation/reconstruction/baseline/s2rv3_p2vpp_bed1} &
        \includegraphics[width=.2\linewidth]{/Users/apple/OVGU/Thesis/code/3dReconstruction/report/images/evaluation/reconstruction/baseline/s2rv3_p2vpp_sofa1} &
        \includegraphics[width=.2\linewidth]{/Users/apple/OVGU/Thesis/code/3dReconstruction/report/images/evaluation/reconstruction/baseline/s2rv3_p2vpp_table2}\\

        Output4 & \includegraphics[width=.2\linewidth]{/Users/apple/OVGU/Thesis/code/3dReconstruction/report/images/evaluation/reconstruction/baseline/s2rv3_p2v_bed1} &
        \includegraphics[width=.2\linewidth]{/Users/apple/OVGU/Thesis/code/3dReconstruction/report/images/evaluation/reconstruction/baseline/s2rv3_p2v_sofa1} &
        \includegraphics[width=.2\linewidth]{/Users/apple/OVGU/Thesis/code/3dReconstruction/report/images/evaluation/reconstruction/baseline/s2rv3_p2v_table2}\\

    \end{tabular}
    \caption[3D Reconstruction Outputs for Baselines.]{3D reconstruction outputs for models trained on \textbf{only real dataset}, and \textbf{only synthetic dataset}. Output1-2: Pix2Vox++ and Pix2Vox trained on Pix3D(real dataset).
    Output3-4: Pix2Vox++ and Pix2Vox trained with only \gls{s2rv2} synthetic dataset. This corresponds to the bad \gls{iou} when trained on only synthetic dataset.}
    \label{fig:baseline_images1}
\end{figure}

In \autoref{fig:baseline_images1}, we see the 3D reconstruction output for models trained on only real and only synthetic datasets.
The outputs were collected for images from the real dataset with the threshold, which gave the best \gls{iou}.
The output of models trained on only synthetic holds to the \gls{iou} values seen in \autoref{fig:baseline1}.
The reconstructions show that the model has a domain gap, and hence the output does not match the ground truth.

%\begin{figure}
%    \centering
%    \resizebox{0.49\linewidth}{!}{\input{/Users/apple/OVGU/Thesis/code/3dReconstruction/report/images/evaluation/performance/baseline_linegraph1.pgf}}
%    \resizebox{0.49\linewidth}{!}{%% Creator: Matplotlib, PGF backend
%%
%% To include the figure in your LaTeX document, write
%%   \input{<filename>.pgf}
%%
%% Make sure the required packages are loaded in your preamble
%%   \usepackage{pgf}
%%
%% Figures using additional raster images can only be included by \input if
%% they are in the same directory as the main LaTeX file. For loading figures
%% from other directories you can use the `import` package
%%   \usepackage{import}
%%
%% and then include the figures with
%%   \import{<path to file>}{<filename>.pgf}
%%
%% Matplotlib used the following preamble
%%   \usepackage{fontspec}
%%   \setmainfont{DejaVuSerif.ttf}[Path=\detokenize{/Users/apple/opt/anaconda3/envs/kaolin/lib/python3.7/site-packages/matplotlib/mpl-data/fonts/ttf/}]
%%   \setsansfont{DejaVuSans.ttf}[Path=\detokenize{/Users/apple/opt/anaconda3/envs/kaolin/lib/python3.7/site-packages/matplotlib/mpl-data/fonts/ttf/}]
%%   \setmonofont{DejaVuSansMono.ttf}[Path=\detokenize{/Users/apple/opt/anaconda3/envs/kaolin/lib/python3.7/site-packages/matplotlib/mpl-data/fonts/ttf/}]
%%
\begingroup%
\makeatletter%
\begin{pgfpicture}%
\pgfpathrectangle{\pgfpointorigin}{\pgfqpoint{5.668070in}{5.072313in}}%
\pgfusepath{use as bounding box, clip}%
\begin{pgfscope}%
\pgfsetbuttcap%
\pgfsetmiterjoin%
\definecolor{currentfill}{rgb}{1.000000,1.000000,1.000000}%
\pgfsetfillcolor{currentfill}%
\pgfsetlinewidth{0.000000pt}%
\definecolor{currentstroke}{rgb}{1.000000,1.000000,1.000000}%
\pgfsetstrokecolor{currentstroke}%
\pgfsetdash{}{0pt}%
\pgfpathmoveto{\pgfqpoint{0.000000in}{0.000000in}}%
\pgfpathlineto{\pgfqpoint{5.668070in}{0.000000in}}%
\pgfpathlineto{\pgfqpoint{5.668070in}{5.072313in}}%
\pgfpathlineto{\pgfqpoint{0.000000in}{5.072313in}}%
\pgfpathclose%
\pgfusepath{fill}%
\end{pgfscope}%
\begin{pgfscope}%
\pgfsetbuttcap%
\pgfsetmiterjoin%
\definecolor{currentfill}{rgb}{1.000000,1.000000,1.000000}%
\pgfsetfillcolor{currentfill}%
\pgfsetlinewidth{0.000000pt}%
\definecolor{currentstroke}{rgb}{0.000000,0.000000,0.000000}%
\pgfsetstrokecolor{currentstroke}%
\pgfsetstrokeopacity{0.000000}%
\pgfsetdash{}{0pt}%
\pgfpathmoveto{\pgfqpoint{0.608070in}{1.223552in}}%
\pgfpathlineto{\pgfqpoint{5.568070in}{1.223552in}}%
\pgfpathlineto{\pgfqpoint{5.568070in}{4.919552in}}%
\pgfpathlineto{\pgfqpoint{0.608070in}{4.919552in}}%
\pgfpathclose%
\pgfusepath{fill}%
\end{pgfscope}%
\begin{pgfscope}%
\pgfsetbuttcap%
\pgfsetroundjoin%
\definecolor{currentfill}{rgb}{0.000000,0.000000,0.000000}%
\pgfsetfillcolor{currentfill}%
\pgfsetlinewidth{0.803000pt}%
\definecolor{currentstroke}{rgb}{0.000000,0.000000,0.000000}%
\pgfsetstrokecolor{currentstroke}%
\pgfsetdash{}{0pt}%
\pgfsys@defobject{currentmarker}{\pgfqpoint{0.000000in}{-0.048611in}}{\pgfqpoint{0.000000in}{0.000000in}}{%
\pgfpathmoveto{\pgfqpoint{0.000000in}{0.000000in}}%
\pgfpathlineto{\pgfqpoint{0.000000in}{-0.048611in}}%
\pgfusepath{stroke,fill}%
}%
\begin{pgfscope}%
\pgfsys@transformshift{0.833525in}{1.223552in}%
\pgfsys@useobject{currentmarker}{}%
\end{pgfscope}%
\end{pgfscope}%
\begin{pgfscope}%
\definecolor{textcolor}{rgb}{0.000000,0.000000,0.000000}%
\pgfsetstrokecolor{textcolor}%
\pgfsetfillcolor{textcolor}%
\pgftext[x=0.519500in, y=0.369475in, left, base,rotate=45.000000]{\color{textcolor}\sffamily\fontsize{10.000000}{12.000000}\selectfont Pix3d(no aug)}%
\end{pgfscope}%
\begin{pgfscope}%
\pgfsetbuttcap%
\pgfsetroundjoin%
\definecolor{currentfill}{rgb}{0.000000,0.000000,0.000000}%
\pgfsetfillcolor{currentfill}%
\pgfsetlinewidth{0.803000pt}%
\definecolor{currentstroke}{rgb}{0.000000,0.000000,0.000000}%
\pgfsetstrokecolor{currentstroke}%
\pgfsetdash{}{0pt}%
\pgfsys@defobject{currentmarker}{\pgfqpoint{0.000000in}{-0.048611in}}{\pgfqpoint{0.000000in}{0.000000in}}{%
\pgfpathmoveto{\pgfqpoint{0.000000in}{0.000000in}}%
\pgfpathlineto{\pgfqpoint{0.000000in}{-0.048611in}}%
\pgfusepath{stroke,fill}%
}%
\begin{pgfscope}%
\pgfsys@transformshift{1.960797in}{1.223552in}%
\pgfsys@useobject{currentmarker}{}%
\end{pgfscope}%
\end{pgfscope}%
\begin{pgfscope}%
\definecolor{textcolor}{rgb}{0.000000,0.000000,0.000000}%
\pgfsetstrokecolor{textcolor}%
\pgfsetfillcolor{textcolor}%
\pgftext[x=1.854244in, y=0.784419in, left, base,rotate=45.000000]{\color{textcolor}\sffamily\fontsize{10.000000}{12.000000}\selectfont Pix3d}%
\end{pgfscope}%
\begin{pgfscope}%
\pgfsetbuttcap%
\pgfsetroundjoin%
\definecolor{currentfill}{rgb}{0.000000,0.000000,0.000000}%
\pgfsetfillcolor{currentfill}%
\pgfsetlinewidth{0.803000pt}%
\definecolor{currentstroke}{rgb}{0.000000,0.000000,0.000000}%
\pgfsetstrokecolor{currentstroke}%
\pgfsetdash{}{0pt}%
\pgfsys@defobject{currentmarker}{\pgfqpoint{0.000000in}{-0.048611in}}{\pgfqpoint{0.000000in}{0.000000in}}{%
\pgfpathmoveto{\pgfqpoint{0.000000in}{0.000000in}}%
\pgfpathlineto{\pgfqpoint{0.000000in}{-0.048611in}}%
\pgfusepath{stroke,fill}%
}%
\begin{pgfscope}%
\pgfsys@transformshift{3.088070in}{1.223552in}%
\pgfsys@useobject{currentmarker}{}%
\end{pgfscope}%
\end{pgfscope}%
\begin{pgfscope}%
\definecolor{textcolor}{rgb}{0.000000,0.000000,0.000000}%
\pgfsetstrokecolor{textcolor}%
\pgfsetfillcolor{textcolor}%
\pgftext[x=2.744410in, y=0.313130in, left, base,rotate=45.000000]{\color{textcolor}\sffamily\fontsize{10.000000}{12.000000}\selectfont s2r\_v1(no aug)}%
\end{pgfscope}%
\begin{pgfscope}%
\pgfsetbuttcap%
\pgfsetroundjoin%
\definecolor{currentfill}{rgb}{0.000000,0.000000,0.000000}%
\pgfsetfillcolor{currentfill}%
\pgfsetlinewidth{0.803000pt}%
\definecolor{currentstroke}{rgb}{0.000000,0.000000,0.000000}%
\pgfsetstrokecolor{currentstroke}%
\pgfsetdash{}{0pt}%
\pgfsys@defobject{currentmarker}{\pgfqpoint{0.000000in}{-0.048611in}}{\pgfqpoint{0.000000in}{0.000000in}}{%
\pgfpathmoveto{\pgfqpoint{0.000000in}{0.000000in}}%
\pgfpathlineto{\pgfqpoint{0.000000in}{-0.048611in}}%
\pgfusepath{stroke,fill}%
}%
\begin{pgfscope}%
\pgfsys@transformshift{4.215343in}{1.223552in}%
\pgfsys@useobject{currentmarker}{}%
\end{pgfscope}%
\end{pgfscope}%
\begin{pgfscope}%
\definecolor{textcolor}{rgb}{0.000000,0.000000,0.000000}%
\pgfsetstrokecolor{textcolor}%
\pgfsetfillcolor{textcolor}%
\pgftext[x=4.078339in, y=0.729704in, left, base,rotate=45.000000]{\color{textcolor}\sffamily\fontsize{10.000000}{12.000000}\selectfont s2r\_v1}%
\end{pgfscope}%
\begin{pgfscope}%
\pgfsetbuttcap%
\pgfsetroundjoin%
\definecolor{currentfill}{rgb}{0.000000,0.000000,0.000000}%
\pgfsetfillcolor{currentfill}%
\pgfsetlinewidth{0.803000pt}%
\definecolor{currentstroke}{rgb}{0.000000,0.000000,0.000000}%
\pgfsetstrokecolor{currentstroke}%
\pgfsetdash{}{0pt}%
\pgfsys@defobject{currentmarker}{\pgfqpoint{0.000000in}{-0.048611in}}{\pgfqpoint{0.000000in}{0.000000in}}{%
\pgfpathmoveto{\pgfqpoint{0.000000in}{0.000000in}}%
\pgfpathlineto{\pgfqpoint{0.000000in}{-0.048611in}}%
\pgfusepath{stroke,fill}%
}%
\begin{pgfscope}%
\pgfsys@transformshift{5.342615in}{1.223552in}%
\pgfsys@useobject{currentmarker}{}%
\end{pgfscope}%
\end{pgfscope}%
\begin{pgfscope}%
\definecolor{textcolor}{rgb}{0.000000,0.000000,0.000000}%
\pgfsetstrokecolor{textcolor}%
\pgfsetfillcolor{textcolor}%
\pgftext[x=5.205612in, y=0.729704in, left, base,rotate=45.000000]{\color{textcolor}\sffamily\fontsize{10.000000}{12.000000}\selectfont s2r\_v2}%
\end{pgfscope}%
\begin{pgfscope}%
\definecolor{textcolor}{rgb}{0.000000,0.000000,0.000000}%
\pgfsetstrokecolor{textcolor}%
\pgfsetfillcolor{textcolor}%
\pgftext[x=3.088070in,y=0.234413in,,top]{\color{textcolor}\sffamily\fontsize{10.000000}{12.000000}\selectfont Datasets}%
\end{pgfscope}%
\begin{pgfscope}%
\pgfsetbuttcap%
\pgfsetroundjoin%
\definecolor{currentfill}{rgb}{0.000000,0.000000,0.000000}%
\pgfsetfillcolor{currentfill}%
\pgfsetlinewidth{0.803000pt}%
\definecolor{currentstroke}{rgb}{0.000000,0.000000,0.000000}%
\pgfsetstrokecolor{currentstroke}%
\pgfsetdash{}{0pt}%
\pgfsys@defobject{currentmarker}{\pgfqpoint{-0.048611in}{0.000000in}}{\pgfqpoint{-0.000000in}{0.000000in}}{%
\pgfpathmoveto{\pgfqpoint{-0.000000in}{0.000000in}}%
\pgfpathlineto{\pgfqpoint{-0.048611in}{0.000000in}}%
\pgfusepath{stroke,fill}%
}%
\begin{pgfscope}%
\pgfsys@transformshift{0.608070in}{1.223552in}%
\pgfsys@useobject{currentmarker}{}%
\end{pgfscope}%
\end{pgfscope}%
\begin{pgfscope}%
\definecolor{textcolor}{rgb}{0.000000,0.000000,0.000000}%
\pgfsetstrokecolor{textcolor}%
\pgfsetfillcolor{textcolor}%
\pgftext[x=0.289968in, y=1.170790in, left, base]{\color{textcolor}\sffamily\fontsize{10.000000}{12.000000}\selectfont 0.0}%
\end{pgfscope}%
\begin{pgfscope}%
\pgfsetbuttcap%
\pgfsetroundjoin%
\definecolor{currentfill}{rgb}{0.000000,0.000000,0.000000}%
\pgfsetfillcolor{currentfill}%
\pgfsetlinewidth{0.803000pt}%
\definecolor{currentstroke}{rgb}{0.000000,0.000000,0.000000}%
\pgfsetstrokecolor{currentstroke}%
\pgfsetdash{}{0pt}%
\pgfsys@defobject{currentmarker}{\pgfqpoint{-0.048611in}{0.000000in}}{\pgfqpoint{-0.000000in}{0.000000in}}{%
\pgfpathmoveto{\pgfqpoint{-0.000000in}{0.000000in}}%
\pgfpathlineto{\pgfqpoint{-0.048611in}{0.000000in}}%
\pgfusepath{stroke,fill}%
}%
\begin{pgfscope}%
\pgfsys@transformshift{0.608070in}{1.962752in}%
\pgfsys@useobject{currentmarker}{}%
\end{pgfscope}%
\end{pgfscope}%
\begin{pgfscope}%
\definecolor{textcolor}{rgb}{0.000000,0.000000,0.000000}%
\pgfsetstrokecolor{textcolor}%
\pgfsetfillcolor{textcolor}%
\pgftext[x=0.289968in, y=1.909990in, left, base]{\color{textcolor}\sffamily\fontsize{10.000000}{12.000000}\selectfont 0.1}%
\end{pgfscope}%
\begin{pgfscope}%
\pgfsetbuttcap%
\pgfsetroundjoin%
\definecolor{currentfill}{rgb}{0.000000,0.000000,0.000000}%
\pgfsetfillcolor{currentfill}%
\pgfsetlinewidth{0.803000pt}%
\definecolor{currentstroke}{rgb}{0.000000,0.000000,0.000000}%
\pgfsetstrokecolor{currentstroke}%
\pgfsetdash{}{0pt}%
\pgfsys@defobject{currentmarker}{\pgfqpoint{-0.048611in}{0.000000in}}{\pgfqpoint{-0.000000in}{0.000000in}}{%
\pgfpathmoveto{\pgfqpoint{-0.000000in}{0.000000in}}%
\pgfpathlineto{\pgfqpoint{-0.048611in}{0.000000in}}%
\pgfusepath{stroke,fill}%
}%
\begin{pgfscope}%
\pgfsys@transformshift{0.608070in}{2.701952in}%
\pgfsys@useobject{currentmarker}{}%
\end{pgfscope}%
\end{pgfscope}%
\begin{pgfscope}%
\definecolor{textcolor}{rgb}{0.000000,0.000000,0.000000}%
\pgfsetstrokecolor{textcolor}%
\pgfsetfillcolor{textcolor}%
\pgftext[x=0.289968in, y=2.649190in, left, base]{\color{textcolor}\sffamily\fontsize{10.000000}{12.000000}\selectfont 0.2}%
\end{pgfscope}%
\begin{pgfscope}%
\pgfsetbuttcap%
\pgfsetroundjoin%
\definecolor{currentfill}{rgb}{0.000000,0.000000,0.000000}%
\pgfsetfillcolor{currentfill}%
\pgfsetlinewidth{0.803000pt}%
\definecolor{currentstroke}{rgb}{0.000000,0.000000,0.000000}%
\pgfsetstrokecolor{currentstroke}%
\pgfsetdash{}{0pt}%
\pgfsys@defobject{currentmarker}{\pgfqpoint{-0.048611in}{0.000000in}}{\pgfqpoint{-0.000000in}{0.000000in}}{%
\pgfpathmoveto{\pgfqpoint{-0.000000in}{0.000000in}}%
\pgfpathlineto{\pgfqpoint{-0.048611in}{0.000000in}}%
\pgfusepath{stroke,fill}%
}%
\begin{pgfscope}%
\pgfsys@transformshift{0.608070in}{3.441152in}%
\pgfsys@useobject{currentmarker}{}%
\end{pgfscope}%
\end{pgfscope}%
\begin{pgfscope}%
\definecolor{textcolor}{rgb}{0.000000,0.000000,0.000000}%
\pgfsetstrokecolor{textcolor}%
\pgfsetfillcolor{textcolor}%
\pgftext[x=0.289968in, y=3.388390in, left, base]{\color{textcolor}\sffamily\fontsize{10.000000}{12.000000}\selectfont 0.3}%
\end{pgfscope}%
\begin{pgfscope}%
\pgfsetbuttcap%
\pgfsetroundjoin%
\definecolor{currentfill}{rgb}{0.000000,0.000000,0.000000}%
\pgfsetfillcolor{currentfill}%
\pgfsetlinewidth{0.803000pt}%
\definecolor{currentstroke}{rgb}{0.000000,0.000000,0.000000}%
\pgfsetstrokecolor{currentstroke}%
\pgfsetdash{}{0pt}%
\pgfsys@defobject{currentmarker}{\pgfqpoint{-0.048611in}{0.000000in}}{\pgfqpoint{-0.000000in}{0.000000in}}{%
\pgfpathmoveto{\pgfqpoint{-0.000000in}{0.000000in}}%
\pgfpathlineto{\pgfqpoint{-0.048611in}{0.000000in}}%
\pgfusepath{stroke,fill}%
}%
\begin{pgfscope}%
\pgfsys@transformshift{0.608070in}{4.180352in}%
\pgfsys@useobject{currentmarker}{}%
\end{pgfscope}%
\end{pgfscope}%
\begin{pgfscope}%
\definecolor{textcolor}{rgb}{0.000000,0.000000,0.000000}%
\pgfsetstrokecolor{textcolor}%
\pgfsetfillcolor{textcolor}%
\pgftext[x=0.289968in, y=4.127590in, left, base]{\color{textcolor}\sffamily\fontsize{10.000000}{12.000000}\selectfont 0.4}%
\end{pgfscope}%
\begin{pgfscope}%
\pgfsetbuttcap%
\pgfsetroundjoin%
\definecolor{currentfill}{rgb}{0.000000,0.000000,0.000000}%
\pgfsetfillcolor{currentfill}%
\pgfsetlinewidth{0.803000pt}%
\definecolor{currentstroke}{rgb}{0.000000,0.000000,0.000000}%
\pgfsetstrokecolor{currentstroke}%
\pgfsetdash{}{0pt}%
\pgfsys@defobject{currentmarker}{\pgfqpoint{-0.048611in}{0.000000in}}{\pgfqpoint{-0.000000in}{0.000000in}}{%
\pgfpathmoveto{\pgfqpoint{-0.000000in}{0.000000in}}%
\pgfpathlineto{\pgfqpoint{-0.048611in}{0.000000in}}%
\pgfusepath{stroke,fill}%
}%
\begin{pgfscope}%
\pgfsys@transformshift{0.608070in}{4.919552in}%
\pgfsys@useobject{currentmarker}{}%
\end{pgfscope}%
\end{pgfscope}%
\begin{pgfscope}%
\definecolor{textcolor}{rgb}{0.000000,0.000000,0.000000}%
\pgfsetstrokecolor{textcolor}%
\pgfsetfillcolor{textcolor}%
\pgftext[x=0.289968in, y=4.866790in, left, base]{\color{textcolor}\sffamily\fontsize{10.000000}{12.000000}\selectfont 0.5}%
\end{pgfscope}%
\begin{pgfscope}%
\definecolor{textcolor}{rgb}{0.000000,0.000000,0.000000}%
\pgfsetstrokecolor{textcolor}%
\pgfsetfillcolor{textcolor}%
\pgftext[x=0.234413in,y=3.071552in,,bottom,rotate=90.000000]{\color{textcolor}\sffamily\fontsize{10.000000}{12.000000}\selectfont IoU}%
\end{pgfscope}%
\begin{pgfscope}%
\pgfpathrectangle{\pgfqpoint{0.608070in}{1.223552in}}{\pgfqpoint{4.960000in}{3.696000in}}%
\pgfusepath{clip}%
\pgfsetrectcap%
\pgfsetroundjoin%
\pgfsetlinewidth{1.505625pt}%
\definecolor{currentstroke}{rgb}{0.121569,0.466667,0.705882}%
\pgfsetstrokecolor{currentstroke}%
\pgfsetdash{}{0pt}%
\pgfpathmoveto{\pgfqpoint{0.833525in}{3.276310in}}%
\pgfpathlineto{\pgfqpoint{1.960797in}{3.768617in}}%
\pgfpathlineto{\pgfqpoint{3.088070in}{1.650070in}}%
\pgfpathlineto{\pgfqpoint{4.215343in}{1.849654in}}%
\pgfpathlineto{\pgfqpoint{5.342615in}{2.182294in}}%
\pgfusepath{stroke}%
\end{pgfscope}%
\begin{pgfscope}%
\pgfpathrectangle{\pgfqpoint{0.608070in}{1.223552in}}{\pgfqpoint{4.960000in}{3.696000in}}%
\pgfusepath{clip}%
\pgfsetbuttcap%
\pgfsetroundjoin%
\definecolor{currentfill}{rgb}{0.121569,0.466667,0.705882}%
\pgfsetfillcolor{currentfill}%
\pgfsetlinewidth{1.003750pt}%
\definecolor{currentstroke}{rgb}{0.121569,0.466667,0.705882}%
\pgfsetstrokecolor{currentstroke}%
\pgfsetdash{}{0pt}%
\pgfsys@defobject{currentmarker}{\pgfqpoint{-0.041667in}{-0.041667in}}{\pgfqpoint{0.041667in}{0.041667in}}{%
\pgfpathmoveto{\pgfqpoint{0.000000in}{-0.041667in}}%
\pgfpathcurveto{\pgfqpoint{0.011050in}{-0.041667in}}{\pgfqpoint{0.021649in}{-0.037276in}}{\pgfqpoint{0.029463in}{-0.029463in}}%
\pgfpathcurveto{\pgfqpoint{0.037276in}{-0.021649in}}{\pgfqpoint{0.041667in}{-0.011050in}}{\pgfqpoint{0.041667in}{0.000000in}}%
\pgfpathcurveto{\pgfqpoint{0.041667in}{0.011050in}}{\pgfqpoint{0.037276in}{0.021649in}}{\pgfqpoint{0.029463in}{0.029463in}}%
\pgfpathcurveto{\pgfqpoint{0.021649in}{0.037276in}}{\pgfqpoint{0.011050in}{0.041667in}}{\pgfqpoint{0.000000in}{0.041667in}}%
\pgfpathcurveto{\pgfqpoint{-0.011050in}{0.041667in}}{\pgfqpoint{-0.021649in}{0.037276in}}{\pgfqpoint{-0.029463in}{0.029463in}}%
\pgfpathcurveto{\pgfqpoint{-0.037276in}{0.021649in}}{\pgfqpoint{-0.041667in}{0.011050in}}{\pgfqpoint{-0.041667in}{0.000000in}}%
\pgfpathcurveto{\pgfqpoint{-0.041667in}{-0.011050in}}{\pgfqpoint{-0.037276in}{-0.021649in}}{\pgfqpoint{-0.029463in}{-0.029463in}}%
\pgfpathcurveto{\pgfqpoint{-0.021649in}{-0.037276in}}{\pgfqpoint{-0.011050in}{-0.041667in}}{\pgfqpoint{0.000000in}{-0.041667in}}%
\pgfpathclose%
\pgfusepath{stroke,fill}%
}%
\begin{pgfscope}%
\pgfsys@transformshift{0.833525in}{3.276310in}%
\pgfsys@useobject{currentmarker}{}%
\end{pgfscope}%
\begin{pgfscope}%
\pgfsys@transformshift{1.960797in}{3.768617in}%
\pgfsys@useobject{currentmarker}{}%
\end{pgfscope}%
\begin{pgfscope}%
\pgfsys@transformshift{3.088070in}{1.650070in}%
\pgfsys@useobject{currentmarker}{}%
\end{pgfscope}%
\begin{pgfscope}%
\pgfsys@transformshift{4.215343in}{1.849654in}%
\pgfsys@useobject{currentmarker}{}%
\end{pgfscope}%
\begin{pgfscope}%
\pgfsys@transformshift{5.342615in}{2.182294in}%
\pgfsys@useobject{currentmarker}{}%
\end{pgfscope}%
\end{pgfscope}%
\begin{pgfscope}%
\pgfpathrectangle{\pgfqpoint{0.608070in}{1.223552in}}{\pgfqpoint{4.960000in}{3.696000in}}%
\pgfusepath{clip}%
\pgfsetrectcap%
\pgfsetroundjoin%
\pgfsetlinewidth{1.505625pt}%
\definecolor{currentstroke}{rgb}{1.000000,0.498039,0.054902}%
\pgfsetstrokecolor{currentstroke}%
\pgfsetdash{}{0pt}%
\pgfpathmoveto{\pgfqpoint{0.833525in}{3.520985in}}%
\pgfpathlineto{\pgfqpoint{1.960797in}{3.625952in}}%
\pgfpathlineto{\pgfqpoint{3.088070in}{1.725468in}}%
\pgfpathlineto{\pgfqpoint{4.215343in}{1.645635in}}%
\pgfpathlineto{\pgfqpoint{5.342615in}{2.438057in}}%
\pgfusepath{stroke}%
\end{pgfscope}%
\begin{pgfscope}%
\pgfpathrectangle{\pgfqpoint{0.608070in}{1.223552in}}{\pgfqpoint{4.960000in}{3.696000in}}%
\pgfusepath{clip}%
\pgfsetbuttcap%
\pgfsetmiterjoin%
\definecolor{currentfill}{rgb}{1.000000,0.498039,0.054902}%
\pgfsetfillcolor{currentfill}%
\pgfsetlinewidth{1.003750pt}%
\definecolor{currentstroke}{rgb}{1.000000,0.498039,0.054902}%
\pgfsetstrokecolor{currentstroke}%
\pgfsetdash{}{0pt}%
\pgfsys@defobject{currentmarker}{\pgfqpoint{-0.041667in}{-0.041667in}}{\pgfqpoint{0.041667in}{0.041667in}}{%
\pgfpathmoveto{\pgfqpoint{-0.000000in}{-0.041667in}}%
\pgfpathlineto{\pgfqpoint{0.041667in}{0.041667in}}%
\pgfpathlineto{\pgfqpoint{-0.041667in}{0.041667in}}%
\pgfpathclose%
\pgfusepath{stroke,fill}%
}%
\begin{pgfscope}%
\pgfsys@transformshift{0.833525in}{3.520985in}%
\pgfsys@useobject{currentmarker}{}%
\end{pgfscope}%
\begin{pgfscope}%
\pgfsys@transformshift{1.960797in}{3.625952in}%
\pgfsys@useobject{currentmarker}{}%
\end{pgfscope}%
\begin{pgfscope}%
\pgfsys@transformshift{3.088070in}{1.725468in}%
\pgfsys@useobject{currentmarker}{}%
\end{pgfscope}%
\begin{pgfscope}%
\pgfsys@transformshift{4.215343in}{1.645635in}%
\pgfsys@useobject{currentmarker}{}%
\end{pgfscope}%
\begin{pgfscope}%
\pgfsys@transformshift{5.342615in}{2.438057in}%
\pgfsys@useobject{currentmarker}{}%
\end{pgfscope}%
\end{pgfscope}%
\begin{pgfscope}%
\pgfsetrectcap%
\pgfsetmiterjoin%
\pgfsetlinewidth{0.803000pt}%
\definecolor{currentstroke}{rgb}{0.000000,0.000000,0.000000}%
\pgfsetstrokecolor{currentstroke}%
\pgfsetdash{}{0pt}%
\pgfpathmoveto{\pgfqpoint{0.608070in}{1.223552in}}%
\pgfpathlineto{\pgfqpoint{0.608070in}{4.919552in}}%
\pgfusepath{stroke}%
\end{pgfscope}%
\begin{pgfscope}%
\pgfsetrectcap%
\pgfsetmiterjoin%
\pgfsetlinewidth{0.803000pt}%
\definecolor{currentstroke}{rgb}{0.000000,0.000000,0.000000}%
\pgfsetstrokecolor{currentstroke}%
\pgfsetdash{}{0pt}%
\pgfpathmoveto{\pgfqpoint{5.568070in}{1.223552in}}%
\pgfpathlineto{\pgfqpoint{5.568070in}{4.919552in}}%
\pgfusepath{stroke}%
\end{pgfscope}%
\begin{pgfscope}%
\pgfsetrectcap%
\pgfsetmiterjoin%
\pgfsetlinewidth{0.803000pt}%
\definecolor{currentstroke}{rgb}{0.000000,0.000000,0.000000}%
\pgfsetstrokecolor{currentstroke}%
\pgfsetdash{}{0pt}%
\pgfpathmoveto{\pgfqpoint{0.608070in}{1.223552in}}%
\pgfpathlineto{\pgfqpoint{5.568070in}{1.223552in}}%
\pgfusepath{stroke}%
\end{pgfscope}%
\begin{pgfscope}%
\pgfsetrectcap%
\pgfsetmiterjoin%
\pgfsetlinewidth{0.803000pt}%
\definecolor{currentstroke}{rgb}{0.000000,0.000000,0.000000}%
\pgfsetstrokecolor{currentstroke}%
\pgfsetdash{}{0pt}%
\pgfpathmoveto{\pgfqpoint{0.608070in}{4.919552in}}%
\pgfpathlineto{\pgfqpoint{5.568070in}{4.919552in}}%
\pgfusepath{stroke}%
\end{pgfscope}%
\begin{pgfscope}%
\pgfsetbuttcap%
\pgfsetmiterjoin%
\definecolor{currentfill}{rgb}{1.000000,1.000000,1.000000}%
\pgfsetfillcolor{currentfill}%
\pgfsetfillopacity{0.800000}%
\pgfsetlinewidth{1.003750pt}%
\definecolor{currentstroke}{rgb}{0.800000,0.800000,0.800000}%
\pgfsetstrokecolor{currentstroke}%
\pgfsetstrokeopacity{0.800000}%
\pgfsetdash{}{0pt}%
\pgfpathmoveto{\pgfqpoint{4.256682in}{4.400726in}}%
\pgfpathlineto{\pgfqpoint{5.470848in}{4.400726in}}%
\pgfpathquadraticcurveto{\pgfqpoint{5.498626in}{4.400726in}}{\pgfqpoint{5.498626in}{4.428504in}}%
\pgfpathlineto{\pgfqpoint{5.498626in}{4.822329in}}%
\pgfpathquadraticcurveto{\pgfqpoint{5.498626in}{4.850107in}}{\pgfqpoint{5.470848in}{4.850107in}}%
\pgfpathlineto{\pgfqpoint{4.256682in}{4.850107in}}%
\pgfpathquadraticcurveto{\pgfqpoint{4.228904in}{4.850107in}}{\pgfqpoint{4.228904in}{4.822329in}}%
\pgfpathlineto{\pgfqpoint{4.228904in}{4.428504in}}%
\pgfpathquadraticcurveto{\pgfqpoint{4.228904in}{4.400726in}}{\pgfqpoint{4.256682in}{4.400726in}}%
\pgfpathclose%
\pgfusepath{stroke,fill}%
\end{pgfscope}%
\begin{pgfscope}%
\pgfsetrectcap%
\pgfsetroundjoin%
\pgfsetlinewidth{1.505625pt}%
\definecolor{currentstroke}{rgb}{0.121569,0.466667,0.705882}%
\pgfsetstrokecolor{currentstroke}%
\pgfsetdash{}{0pt}%
\pgfpathmoveto{\pgfqpoint{4.284460in}{4.737640in}}%
\pgfpathlineto{\pgfqpoint{4.562238in}{4.737640in}}%
\pgfusepath{stroke}%
\end{pgfscope}%
\begin{pgfscope}%
\pgfsetbuttcap%
\pgfsetroundjoin%
\definecolor{currentfill}{rgb}{0.121569,0.466667,0.705882}%
\pgfsetfillcolor{currentfill}%
\pgfsetlinewidth{1.003750pt}%
\definecolor{currentstroke}{rgb}{0.121569,0.466667,0.705882}%
\pgfsetstrokecolor{currentstroke}%
\pgfsetdash{}{0pt}%
\pgfsys@defobject{currentmarker}{\pgfqpoint{-0.041667in}{-0.041667in}}{\pgfqpoint{0.041667in}{0.041667in}}{%
\pgfpathmoveto{\pgfqpoint{0.000000in}{-0.041667in}}%
\pgfpathcurveto{\pgfqpoint{0.011050in}{-0.041667in}}{\pgfqpoint{0.021649in}{-0.037276in}}{\pgfqpoint{0.029463in}{-0.029463in}}%
\pgfpathcurveto{\pgfqpoint{0.037276in}{-0.021649in}}{\pgfqpoint{0.041667in}{-0.011050in}}{\pgfqpoint{0.041667in}{0.000000in}}%
\pgfpathcurveto{\pgfqpoint{0.041667in}{0.011050in}}{\pgfqpoint{0.037276in}{0.021649in}}{\pgfqpoint{0.029463in}{0.029463in}}%
\pgfpathcurveto{\pgfqpoint{0.021649in}{0.037276in}}{\pgfqpoint{0.011050in}{0.041667in}}{\pgfqpoint{0.000000in}{0.041667in}}%
\pgfpathcurveto{\pgfqpoint{-0.011050in}{0.041667in}}{\pgfqpoint{-0.021649in}{0.037276in}}{\pgfqpoint{-0.029463in}{0.029463in}}%
\pgfpathcurveto{\pgfqpoint{-0.037276in}{0.021649in}}{\pgfqpoint{-0.041667in}{0.011050in}}{\pgfqpoint{-0.041667in}{0.000000in}}%
\pgfpathcurveto{\pgfqpoint{-0.041667in}{-0.011050in}}{\pgfqpoint{-0.037276in}{-0.021649in}}{\pgfqpoint{-0.029463in}{-0.029463in}}%
\pgfpathcurveto{\pgfqpoint{-0.021649in}{-0.037276in}}{\pgfqpoint{-0.011050in}{-0.041667in}}{\pgfqpoint{0.000000in}{-0.041667in}}%
\pgfpathclose%
\pgfusepath{stroke,fill}%
}%
\begin{pgfscope}%
\pgfsys@transformshift{4.423349in}{4.737640in}%
\pgfsys@useobject{currentmarker}{}%
\end{pgfscope}%
\end{pgfscope}%
\begin{pgfscope}%
\definecolor{textcolor}{rgb}{0.000000,0.000000,0.000000}%
\pgfsetstrokecolor{textcolor}%
\pgfsetfillcolor{textcolor}%
\pgftext[x=4.673349in,y=4.689029in,left,base]{\color{textcolor}\sffamily\fontsize{10.000000}{12.000000}\selectfont Pix2Vox++}%
\end{pgfscope}%
\begin{pgfscope}%
\pgfsetrectcap%
\pgfsetroundjoin%
\pgfsetlinewidth{1.505625pt}%
\definecolor{currentstroke}{rgb}{1.000000,0.498039,0.054902}%
\pgfsetstrokecolor{currentstroke}%
\pgfsetdash{}{0pt}%
\pgfpathmoveto{\pgfqpoint{4.284460in}{4.533782in}}%
\pgfpathlineto{\pgfqpoint{4.562238in}{4.533782in}}%
\pgfusepath{stroke}%
\end{pgfscope}%
\begin{pgfscope}%
\pgfsetbuttcap%
\pgfsetmiterjoin%
\definecolor{currentfill}{rgb}{1.000000,0.498039,0.054902}%
\pgfsetfillcolor{currentfill}%
\pgfsetlinewidth{1.003750pt}%
\definecolor{currentstroke}{rgb}{1.000000,0.498039,0.054902}%
\pgfsetstrokecolor{currentstroke}%
\pgfsetdash{}{0pt}%
\pgfsys@defobject{currentmarker}{\pgfqpoint{-0.041667in}{-0.041667in}}{\pgfqpoint{0.041667in}{0.041667in}}{%
\pgfpathmoveto{\pgfqpoint{-0.000000in}{-0.041667in}}%
\pgfpathlineto{\pgfqpoint{0.041667in}{0.041667in}}%
\pgfpathlineto{\pgfqpoint{-0.041667in}{0.041667in}}%
\pgfpathclose%
\pgfusepath{stroke,fill}%
}%
\begin{pgfscope}%
\pgfsys@transformshift{4.423349in}{4.533782in}%
\pgfsys@useobject{currentmarker}{}%
\end{pgfscope}%
\end{pgfscope}%
\begin{pgfscope}%
\definecolor{textcolor}{rgb}{0.000000,0.000000,0.000000}%
\pgfsetstrokecolor{textcolor}%
\pgfsetfillcolor{textcolor}%
\pgftext[x=4.673349in,y=4.485171in,left,base]{\color{textcolor}\sffamily\fontsize{10.000000}{12.000000}\selectfont Pix2Vox}%
\end{pgfscope}%
\end{pgfpicture}%
\makeatother%
\endgroup%
}
%    \caption{Line plot for the \gls{iou}  for baselines trained on real and synthetic datasets, with and without 2D augmentation.
%        (Left)The checkpoint was saved using real dataset for validation and test, (right) the checkpoint was saved using corresponding synthetic data for validation step and tested with real data.
%        In both the cases we see that ~\gls{free} does not perform adequately on its own. \gls{s2rv2} contributes better than \gls{s2rv1}.}
%    \label{fig:baseline1}
%\end{figure}


\section{Fine Tuning}\label{sec:fine-tuning}
Fine-tuning or Transfer Learning is a common way of domain adaptation.
For this experiment, the model is first trained on a synthetic dataset and then used as a pre-trained model to be fine-tuned using a real dataset.

In \autoref{fig:finetuning1}, compares \gls{iou} of a pure real and a pure synthetic dataset, followed by fine-tuning the models with the real dataset.
The core comparison is between real data and fine-tuned model.
Models are pre-trained with two versions of \gls{free}(\gls{s2rv1} and \gls{s2rv2}) as mentioned in \autoref{sec:datasets}.
It is noticed that for \gls{s2rv1}, there is a decrement of 3.04\% and 1.25\% on Pix2Vox++ and Pix2Vox, respectively.
For \gls{s2rv2}, a decrement of 3.54\% on Pix2Vox++, but an increment of 0.8\% on the Pix2Vox model.

\begin{figure}[ht]
    \centering
    \resizebox{0.75\textwidth}{!}{%% Creator: Matplotlib, PGF backend
%%
%% To include the figure in your LaTeX document, write
%%   \input{<filename>.pgf}
%%
%% Make sure the required packages are loaded in your preamble
%%   \usepackage{pgf}
%%
%% Figures using additional raster images can only be included by \input if
%% they are in the same directory as the main LaTeX file. For loading figures
%% from other directories you can use the `import` package
%%   \usepackage{import}
%%
%% and then include the figures with
%%   \import{<path to file>}{<filename>.pgf}
%%
%% Matplotlib used the following preamble
%%   \usepackage{fontspec}
%%   \setmainfont{DejaVuSerif.ttf}[Path=\detokenize{/Users/apple/opt/anaconda3/envs/kaolin/lib/python3.7/site-packages/matplotlib/mpl-data/fonts/ttf/}]
%%   \setsansfont{DejaVuSans.ttf}[Path=\detokenize{/Users/apple/opt/anaconda3/envs/kaolin/lib/python3.7/site-packages/matplotlib/mpl-data/fonts/ttf/}]
%%   \setmonofont{DejaVuSansMono.ttf}[Path=\detokenize{/Users/apple/opt/anaconda3/envs/kaolin/lib/python3.7/site-packages/matplotlib/mpl-data/fonts/ttf/}]
%%
\begingroup%
\makeatletter%
\begin{pgfpicture}%
\pgfpathrectangle{\pgfpointorigin}{\pgfqpoint{6.471732in}{4.697158in}}%
\pgfusepath{use as bounding box, clip}%
\begin{pgfscope}%
\pgfsetbuttcap%
\pgfsetmiterjoin%
\definecolor{currentfill}{rgb}{1.000000,1.000000,1.000000}%
\pgfsetfillcolor{currentfill}%
\pgfsetlinewidth{0.000000pt}%
\definecolor{currentstroke}{rgb}{1.000000,1.000000,1.000000}%
\pgfsetstrokecolor{currentstroke}%
\pgfsetdash{}{0pt}%
\pgfpathmoveto{\pgfqpoint{0.000000in}{0.000000in}}%
\pgfpathlineto{\pgfqpoint{6.471732in}{0.000000in}}%
\pgfpathlineto{\pgfqpoint{6.471732in}{4.697158in}}%
\pgfpathlineto{\pgfqpoint{0.000000in}{4.697158in}}%
\pgfpathclose%
\pgfusepath{fill}%
\end{pgfscope}%
\begin{pgfscope}%
\pgfsetbuttcap%
\pgfsetmiterjoin%
\definecolor{currentfill}{rgb}{1.000000,1.000000,1.000000}%
\pgfsetfillcolor{currentfill}%
\pgfsetlinewidth{0.000000pt}%
\definecolor{currentstroke}{rgb}{0.000000,0.000000,0.000000}%
\pgfsetstrokecolor{currentstroke}%
\pgfsetstrokeopacity{0.000000}%
\pgfsetdash{}{0pt}%
\pgfpathmoveto{\pgfqpoint{0.706011in}{1.372601in}}%
\pgfpathlineto{\pgfqpoint{6.198233in}{1.372601in}}%
\pgfpathlineto{\pgfqpoint{6.198233in}{4.366092in}}%
\pgfpathlineto{\pgfqpoint{0.706011in}{4.366092in}}%
\pgfpathclose%
\pgfusepath{fill}%
\end{pgfscope}%
\begin{pgfscope}%
\pgfpathrectangle{\pgfqpoint{0.706011in}{1.372601in}}{\pgfqpoint{5.492222in}{2.993491in}}%
\pgfusepath{clip}%
\pgfsetbuttcap%
\pgfsetmiterjoin%
\definecolor{currentfill}{rgb}{0.121569,0.466667,0.705882}%
\pgfsetfillcolor{currentfill}%
\pgfsetlinewidth{0.000000pt}%
\definecolor{currentstroke}{rgb}{0.000000,0.000000,0.000000}%
\pgfsetstrokecolor{currentstroke}%
\pgfsetstrokeopacity{0.000000}%
\pgfsetdash{}{0pt}%
\pgfpathmoveto{\pgfqpoint{0.955658in}{1.372601in}}%
\pgfpathlineto{\pgfqpoint{1.414192in}{1.372601in}}%
\pgfpathlineto{\pgfqpoint{1.414192in}{3.433919in}}%
\pgfpathlineto{\pgfqpoint{0.955658in}{3.433919in}}%
\pgfpathclose%
\pgfusepath{fill}%
\end{pgfscope}%
\begin{pgfscope}%
\pgfpathrectangle{\pgfqpoint{0.706011in}{1.372601in}}{\pgfqpoint{5.492222in}{2.993491in}}%
\pgfusepath{clip}%
\pgfsetbuttcap%
\pgfsetmiterjoin%
\definecolor{currentfill}{rgb}{0.121569,0.466667,0.705882}%
\pgfsetfillcolor{currentfill}%
\pgfsetlinewidth{0.000000pt}%
\definecolor{currentstroke}{rgb}{0.000000,0.000000,0.000000}%
\pgfsetstrokecolor{currentstroke}%
\pgfsetstrokeopacity{0.000000}%
\pgfsetdash{}{0pt}%
\pgfpathmoveto{\pgfqpoint{1.974623in}{1.372601in}}%
\pgfpathlineto{\pgfqpoint{2.433157in}{1.372601in}}%
\pgfpathlineto{\pgfqpoint{2.433157in}{1.879699in}}%
\pgfpathlineto{\pgfqpoint{1.974623in}{1.879699in}}%
\pgfpathclose%
\pgfusepath{fill}%
\end{pgfscope}%
\begin{pgfscope}%
\pgfpathrectangle{\pgfqpoint{0.706011in}{1.372601in}}{\pgfqpoint{5.492222in}{2.993491in}}%
\pgfusepath{clip}%
\pgfsetbuttcap%
\pgfsetmiterjoin%
\definecolor{currentfill}{rgb}{0.121569,0.466667,0.705882}%
\pgfsetfillcolor{currentfill}%
\pgfsetlinewidth{0.000000pt}%
\definecolor{currentstroke}{rgb}{0.000000,0.000000,0.000000}%
\pgfsetstrokecolor{currentstroke}%
\pgfsetstrokeopacity{0.000000}%
\pgfsetdash{}{0pt}%
\pgfpathmoveto{\pgfqpoint{2.993588in}{1.372601in}}%
\pgfpathlineto{\pgfqpoint{3.452122in}{1.372601in}}%
\pgfpathlineto{\pgfqpoint{3.452122in}{2.149113in}}%
\pgfpathlineto{\pgfqpoint{2.993588in}{2.149113in}}%
\pgfpathclose%
\pgfusepath{fill}%
\end{pgfscope}%
\begin{pgfscope}%
\pgfpathrectangle{\pgfqpoint{0.706011in}{1.372601in}}{\pgfqpoint{5.492222in}{2.993491in}}%
\pgfusepath{clip}%
\pgfsetbuttcap%
\pgfsetmiterjoin%
\definecolor{currentfill}{rgb}{0.121569,0.466667,0.705882}%
\pgfsetfillcolor{currentfill}%
\pgfsetlinewidth{0.000000pt}%
\definecolor{currentstroke}{rgb}{0.000000,0.000000,0.000000}%
\pgfsetstrokecolor{currentstroke}%
\pgfsetstrokeopacity{0.000000}%
\pgfsetdash{}{0pt}%
\pgfpathmoveto{\pgfqpoint{4.012553in}{1.372601in}}%
\pgfpathlineto{\pgfqpoint{4.471087in}{1.372601in}}%
\pgfpathlineto{\pgfqpoint{4.471087in}{3.251915in}}%
\pgfpathlineto{\pgfqpoint{4.012553in}{3.251915in}}%
\pgfpathclose%
\pgfusepath{fill}%
\end{pgfscope}%
\begin{pgfscope}%
\pgfpathrectangle{\pgfqpoint{0.706011in}{1.372601in}}{\pgfqpoint{5.492222in}{2.993491in}}%
\pgfusepath{clip}%
\pgfsetbuttcap%
\pgfsetmiterjoin%
\definecolor{currentfill}{rgb}{0.121569,0.466667,0.705882}%
\pgfsetfillcolor{currentfill}%
\pgfsetlinewidth{0.000000pt}%
\definecolor{currentstroke}{rgb}{0.000000,0.000000,0.000000}%
\pgfsetstrokecolor{currentstroke}%
\pgfsetstrokeopacity{0.000000}%
\pgfsetdash{}{0pt}%
\pgfpathmoveto{\pgfqpoint{5.031518in}{1.372601in}}%
\pgfpathlineto{\pgfqpoint{5.490053in}{1.372601in}}%
\pgfpathlineto{\pgfqpoint{5.490053in}{3.221980in}}%
\pgfpathlineto{\pgfqpoint{5.031518in}{3.221980in}}%
\pgfpathclose%
\pgfusepath{fill}%
\end{pgfscope}%
\begin{pgfscope}%
\pgfpathrectangle{\pgfqpoint{0.706011in}{1.372601in}}{\pgfqpoint{5.492222in}{2.993491in}}%
\pgfusepath{clip}%
\pgfsetbuttcap%
\pgfsetmiterjoin%
\definecolor{currentfill}{rgb}{1.000000,0.498039,0.054902}%
\pgfsetfillcolor{currentfill}%
\pgfsetlinewidth{0.000000pt}%
\definecolor{currentstroke}{rgb}{0.000000,0.000000,0.000000}%
\pgfsetstrokecolor{currentstroke}%
\pgfsetstrokeopacity{0.000000}%
\pgfsetdash{}{0pt}%
\pgfpathmoveto{\pgfqpoint{1.414192in}{1.372601in}}%
\pgfpathlineto{\pgfqpoint{1.872726in}{1.372601in}}%
\pgfpathlineto{\pgfqpoint{1.872726in}{3.318370in}}%
\pgfpathlineto{\pgfqpoint{1.414192in}{3.318370in}}%
\pgfpathclose%
\pgfusepath{fill}%
\end{pgfscope}%
\begin{pgfscope}%
\pgfpathrectangle{\pgfqpoint{0.706011in}{1.372601in}}{\pgfqpoint{5.492222in}{2.993491in}}%
\pgfusepath{clip}%
\pgfsetbuttcap%
\pgfsetmiterjoin%
\definecolor{currentfill}{rgb}{1.000000,0.498039,0.054902}%
\pgfsetfillcolor{currentfill}%
\pgfsetlinewidth{0.000000pt}%
\definecolor{currentstroke}{rgb}{0.000000,0.000000,0.000000}%
\pgfsetstrokecolor{currentstroke}%
\pgfsetstrokeopacity{0.000000}%
\pgfsetdash{}{0pt}%
\pgfpathmoveto{\pgfqpoint{2.433157in}{1.372601in}}%
\pgfpathlineto{\pgfqpoint{2.891691in}{1.372601in}}%
\pgfpathlineto{\pgfqpoint{2.891691in}{1.714458in}}%
\pgfpathlineto{\pgfqpoint{2.433157in}{1.714458in}}%
\pgfpathclose%
\pgfusepath{fill}%
\end{pgfscope}%
\begin{pgfscope}%
\pgfpathrectangle{\pgfqpoint{0.706011in}{1.372601in}}{\pgfqpoint{5.492222in}{2.993491in}}%
\pgfusepath{clip}%
\pgfsetbuttcap%
\pgfsetmiterjoin%
\definecolor{currentfill}{rgb}{1.000000,0.498039,0.054902}%
\pgfsetfillcolor{currentfill}%
\pgfsetlinewidth{0.000000pt}%
\definecolor{currentstroke}{rgb}{0.000000,0.000000,0.000000}%
\pgfsetstrokecolor{currentstroke}%
\pgfsetstrokeopacity{0.000000}%
\pgfsetdash{}{0pt}%
\pgfpathmoveto{\pgfqpoint{3.452122in}{1.372601in}}%
\pgfpathlineto{\pgfqpoint{3.910657in}{1.372601in}}%
\pgfpathlineto{\pgfqpoint{3.910657in}{2.356262in}}%
\pgfpathlineto{\pgfqpoint{3.452122in}{2.356262in}}%
\pgfpathclose%
\pgfusepath{fill}%
\end{pgfscope}%
\begin{pgfscope}%
\pgfpathrectangle{\pgfqpoint{0.706011in}{1.372601in}}{\pgfqpoint{5.492222in}{2.993491in}}%
\pgfusepath{clip}%
\pgfsetbuttcap%
\pgfsetmiterjoin%
\definecolor{currentfill}{rgb}{1.000000,0.498039,0.054902}%
\pgfsetfillcolor{currentfill}%
\pgfsetlinewidth{0.000000pt}%
\definecolor{currentstroke}{rgb}{0.000000,0.000000,0.000000}%
\pgfsetstrokecolor{currentstroke}%
\pgfsetstrokeopacity{0.000000}%
\pgfsetdash{}{0pt}%
\pgfpathmoveto{\pgfqpoint{4.471087in}{1.372601in}}%
\pgfpathlineto{\pgfqpoint{4.929622in}{1.372601in}}%
\pgfpathlineto{\pgfqpoint{4.929622in}{3.243533in}}%
\pgfpathlineto{\pgfqpoint{4.471087in}{3.243533in}}%
\pgfpathclose%
\pgfusepath{fill}%
\end{pgfscope}%
\begin{pgfscope}%
\pgfpathrectangle{\pgfqpoint{0.706011in}{1.372601in}}{\pgfqpoint{5.492222in}{2.993491in}}%
\pgfusepath{clip}%
\pgfsetbuttcap%
\pgfsetmiterjoin%
\definecolor{currentfill}{rgb}{1.000000,0.498039,0.054902}%
\pgfsetfillcolor{currentfill}%
\pgfsetlinewidth{0.000000pt}%
\definecolor{currentstroke}{rgb}{0.000000,0.000000,0.000000}%
\pgfsetstrokecolor{currentstroke}%
\pgfsetstrokeopacity{0.000000}%
\pgfsetdash{}{0pt}%
\pgfpathmoveto{\pgfqpoint{5.490053in}{1.372601in}}%
\pgfpathlineto{\pgfqpoint{5.948587in}{1.372601in}}%
\pgfpathlineto{\pgfqpoint{5.948587in}{3.371056in}}%
\pgfpathlineto{\pgfqpoint{5.490053in}{3.371056in}}%
\pgfpathclose%
\pgfusepath{fill}%
\end{pgfscope}%
\begin{pgfscope}%
\pgfsetbuttcap%
\pgfsetroundjoin%
\definecolor{currentfill}{rgb}{0.000000,0.000000,0.000000}%
\pgfsetfillcolor{currentfill}%
\pgfsetlinewidth{0.803000pt}%
\definecolor{currentstroke}{rgb}{0.000000,0.000000,0.000000}%
\pgfsetstrokecolor{currentstroke}%
\pgfsetdash{}{0pt}%
\pgfsys@defobject{currentmarker}{\pgfqpoint{0.000000in}{-0.048611in}}{\pgfqpoint{0.000000in}{0.000000in}}{%
\pgfpathmoveto{\pgfqpoint{0.000000in}{0.000000in}}%
\pgfpathlineto{\pgfqpoint{0.000000in}{-0.048611in}}%
\pgfusepath{stroke,fill}%
}%
\begin{pgfscope}%
\pgfsys@transformshift{1.414192in}{1.372601in}%
\pgfsys@useobject{currentmarker}{}%
\end{pgfscope}%
\end{pgfscope}%
\begin{pgfscope}%
\definecolor{textcolor}{rgb}{0.000000,0.000000,0.000000}%
\pgfsetstrokecolor{textcolor}%
\pgfsetfillcolor{textcolor}%
\pgftext[x=1.278358in, y=0.849147in, left, base,rotate=45.000000]{\color{textcolor}\sffamily\fontsize{12.000000}{14.400000}\selectfont Pix3D}%
\end{pgfscope}%
\begin{pgfscope}%
\pgfsetbuttcap%
\pgfsetroundjoin%
\definecolor{currentfill}{rgb}{0.000000,0.000000,0.000000}%
\pgfsetfillcolor{currentfill}%
\pgfsetlinewidth{0.803000pt}%
\definecolor{currentstroke}{rgb}{0.000000,0.000000,0.000000}%
\pgfsetstrokecolor{currentstroke}%
\pgfsetdash{}{0pt}%
\pgfsys@defobject{currentmarker}{\pgfqpoint{0.000000in}{-0.048611in}}{\pgfqpoint{0.000000in}{0.000000in}}{%
\pgfpathmoveto{\pgfqpoint{0.000000in}{0.000000in}}%
\pgfpathlineto{\pgfqpoint{0.000000in}{-0.048611in}}%
\pgfusepath{stroke,fill}%
}%
\begin{pgfscope}%
\pgfsys@transformshift{2.433157in}{1.372601in}%
\pgfsys@useobject{currentmarker}{}%
\end{pgfscope}%
\end{pgfscope}%
\begin{pgfscope}%
\definecolor{textcolor}{rgb}{0.000000,0.000000,0.000000}%
\pgfsetstrokecolor{textcolor}%
\pgfsetfillcolor{textcolor}%
\pgftext[x=2.268753in, y=0.799428in, left, base,rotate=45.000000]{\color{textcolor}\sffamily\fontsize{12.000000}{14.400000}\selectfont s2r\_v1}%
\end{pgfscope}%
\begin{pgfscope}%
\pgfsetbuttcap%
\pgfsetroundjoin%
\definecolor{currentfill}{rgb}{0.000000,0.000000,0.000000}%
\pgfsetfillcolor{currentfill}%
\pgfsetlinewidth{0.803000pt}%
\definecolor{currentstroke}{rgb}{0.000000,0.000000,0.000000}%
\pgfsetstrokecolor{currentstroke}%
\pgfsetdash{}{0pt}%
\pgfsys@defobject{currentmarker}{\pgfqpoint{0.000000in}{-0.048611in}}{\pgfqpoint{0.000000in}{0.000000in}}{%
\pgfpathmoveto{\pgfqpoint{0.000000in}{0.000000in}}%
\pgfpathlineto{\pgfqpoint{0.000000in}{-0.048611in}}%
\pgfusepath{stroke,fill}%
}%
\begin{pgfscope}%
\pgfsys@transformshift{3.452122in}{1.372601in}%
\pgfsys@useobject{currentmarker}{}%
\end{pgfscope}%
\end{pgfscope}%
\begin{pgfscope}%
\definecolor{textcolor}{rgb}{0.000000,0.000000,0.000000}%
\pgfsetstrokecolor{textcolor}%
\pgfsetfillcolor{textcolor}%
\pgftext[x=3.287718in, y=0.799428in, left, base,rotate=45.000000]{\color{textcolor}\sffamily\fontsize{12.000000}{14.400000}\selectfont s2r\_v2}%
\end{pgfscope}%
\begin{pgfscope}%
\pgfsetbuttcap%
\pgfsetroundjoin%
\definecolor{currentfill}{rgb}{0.000000,0.000000,0.000000}%
\pgfsetfillcolor{currentfill}%
\pgfsetlinewidth{0.803000pt}%
\definecolor{currentstroke}{rgb}{0.000000,0.000000,0.000000}%
\pgfsetstrokecolor{currentstroke}%
\pgfsetdash{}{0pt}%
\pgfsys@defobject{currentmarker}{\pgfqpoint{0.000000in}{-0.048611in}}{\pgfqpoint{0.000000in}{0.000000in}}{%
\pgfpathmoveto{\pgfqpoint{0.000000in}{0.000000in}}%
\pgfpathlineto{\pgfqpoint{0.000000in}{-0.048611in}}%
\pgfusepath{stroke,fill}%
}%
\begin{pgfscope}%
\pgfsys@transformshift{4.471087in}{1.372601in}%
\pgfsys@useobject{currentmarker}{}%
\end{pgfscope}%
\end{pgfscope}%
\begin{pgfscope}%
\definecolor{textcolor}{rgb}{0.000000,0.000000,0.000000}%
\pgfsetstrokecolor{textcolor}%
\pgfsetfillcolor{textcolor}%
\pgftext[x=4.094804in, y=0.371527in, left, base,rotate=45.000000]{\color{textcolor}\sffamily\fontsize{12.000000}{14.400000}\selectfont s2r\_v1+pix3d}%
\end{pgfscope}%
\begin{pgfscope}%
\pgfsetbuttcap%
\pgfsetroundjoin%
\definecolor{currentfill}{rgb}{0.000000,0.000000,0.000000}%
\pgfsetfillcolor{currentfill}%
\pgfsetlinewidth{0.803000pt}%
\definecolor{currentstroke}{rgb}{0.000000,0.000000,0.000000}%
\pgfsetstrokecolor{currentstroke}%
\pgfsetdash{}{0pt}%
\pgfsys@defobject{currentmarker}{\pgfqpoint{0.000000in}{-0.048611in}}{\pgfqpoint{0.000000in}{0.000000in}}{%
\pgfpathmoveto{\pgfqpoint{0.000000in}{0.000000in}}%
\pgfpathlineto{\pgfqpoint{0.000000in}{-0.048611in}}%
\pgfusepath{stroke,fill}%
}%
\begin{pgfscope}%
\pgfsys@transformshift{5.490053in}{1.372601in}%
\pgfsys@useobject{currentmarker}{}%
\end{pgfscope}%
\end{pgfscope}%
\begin{pgfscope}%
\definecolor{textcolor}{rgb}{0.000000,0.000000,0.000000}%
\pgfsetstrokecolor{textcolor}%
\pgfsetfillcolor{textcolor}%
\pgftext[x=5.113769in, y=0.371527in, left, base,rotate=45.000000]{\color{textcolor}\sffamily\fontsize{12.000000}{14.400000}\selectfont s2r\_v2+pix3d}%
\end{pgfscope}%
\begin{pgfscope}%
\definecolor{textcolor}{rgb}{0.000000,0.000000,0.000000}%
\pgfsetstrokecolor{textcolor}%
\pgfsetfillcolor{textcolor}%
\pgftext[x=3.452122in,y=0.288178in,,top]{\color{textcolor}\sffamily\fontsize{14.000000}{16.800000}\bfseries\selectfont Dataset}%
\end{pgfscope}%
\begin{pgfscope}%
\pgfsetbuttcap%
\pgfsetroundjoin%
\definecolor{currentfill}{rgb}{0.000000,0.000000,0.000000}%
\pgfsetfillcolor{currentfill}%
\pgfsetlinewidth{0.803000pt}%
\definecolor{currentstroke}{rgb}{0.000000,0.000000,0.000000}%
\pgfsetstrokecolor{currentstroke}%
\pgfsetdash{}{0pt}%
\pgfsys@defobject{currentmarker}{\pgfqpoint{-0.048611in}{0.000000in}}{\pgfqpoint{-0.000000in}{0.000000in}}{%
\pgfpathmoveto{\pgfqpoint{-0.000000in}{0.000000in}}%
\pgfpathlineto{\pgfqpoint{-0.048611in}{0.000000in}}%
\pgfusepath{stroke,fill}%
}%
\begin{pgfscope}%
\pgfsys@transformshift{0.706011in}{1.372601in}%
\pgfsys@useobject{currentmarker}{}%
\end{pgfscope}%
\end{pgfscope}%
\begin{pgfscope}%
\definecolor{textcolor}{rgb}{0.000000,0.000000,0.000000}%
\pgfsetstrokecolor{textcolor}%
\pgfsetfillcolor{textcolor}%
\pgftext[x=0.343734in, y=1.309287in, left, base]{\color{textcolor}\sffamily\fontsize{12.000000}{14.400000}\selectfont 0.0}%
\end{pgfscope}%
\begin{pgfscope}%
\pgfsetbuttcap%
\pgfsetroundjoin%
\definecolor{currentfill}{rgb}{0.000000,0.000000,0.000000}%
\pgfsetfillcolor{currentfill}%
\pgfsetlinewidth{0.803000pt}%
\definecolor{currentstroke}{rgb}{0.000000,0.000000,0.000000}%
\pgfsetstrokecolor{currentstroke}%
\pgfsetdash{}{0pt}%
\pgfsys@defobject{currentmarker}{\pgfqpoint{-0.048611in}{0.000000in}}{\pgfqpoint{-0.000000in}{0.000000in}}{%
\pgfpathmoveto{\pgfqpoint{-0.000000in}{0.000000in}}%
\pgfpathlineto{\pgfqpoint{-0.048611in}{0.000000in}}%
\pgfusepath{stroke,fill}%
}%
\begin{pgfscope}%
\pgfsys@transformshift{0.706011in}{1.971299in}%
\pgfsys@useobject{currentmarker}{}%
\end{pgfscope}%
\end{pgfscope}%
\begin{pgfscope}%
\definecolor{textcolor}{rgb}{0.000000,0.000000,0.000000}%
\pgfsetstrokecolor{textcolor}%
\pgfsetfillcolor{textcolor}%
\pgftext[x=0.343734in, y=1.907986in, left, base]{\color{textcolor}\sffamily\fontsize{12.000000}{14.400000}\selectfont 0.1}%
\end{pgfscope}%
\begin{pgfscope}%
\pgfsetbuttcap%
\pgfsetroundjoin%
\definecolor{currentfill}{rgb}{0.000000,0.000000,0.000000}%
\pgfsetfillcolor{currentfill}%
\pgfsetlinewidth{0.803000pt}%
\definecolor{currentstroke}{rgb}{0.000000,0.000000,0.000000}%
\pgfsetstrokecolor{currentstroke}%
\pgfsetdash{}{0pt}%
\pgfsys@defobject{currentmarker}{\pgfqpoint{-0.048611in}{0.000000in}}{\pgfqpoint{-0.000000in}{0.000000in}}{%
\pgfpathmoveto{\pgfqpoint{-0.000000in}{0.000000in}}%
\pgfpathlineto{\pgfqpoint{-0.048611in}{0.000000in}}%
\pgfusepath{stroke,fill}%
}%
\begin{pgfscope}%
\pgfsys@transformshift{0.706011in}{2.569998in}%
\pgfsys@useobject{currentmarker}{}%
\end{pgfscope}%
\end{pgfscope}%
\begin{pgfscope}%
\definecolor{textcolor}{rgb}{0.000000,0.000000,0.000000}%
\pgfsetstrokecolor{textcolor}%
\pgfsetfillcolor{textcolor}%
\pgftext[x=0.343734in, y=2.506684in, left, base]{\color{textcolor}\sffamily\fontsize{12.000000}{14.400000}\selectfont 0.2}%
\end{pgfscope}%
\begin{pgfscope}%
\pgfsetbuttcap%
\pgfsetroundjoin%
\definecolor{currentfill}{rgb}{0.000000,0.000000,0.000000}%
\pgfsetfillcolor{currentfill}%
\pgfsetlinewidth{0.803000pt}%
\definecolor{currentstroke}{rgb}{0.000000,0.000000,0.000000}%
\pgfsetstrokecolor{currentstroke}%
\pgfsetdash{}{0pt}%
\pgfsys@defobject{currentmarker}{\pgfqpoint{-0.048611in}{0.000000in}}{\pgfqpoint{-0.000000in}{0.000000in}}{%
\pgfpathmoveto{\pgfqpoint{-0.000000in}{0.000000in}}%
\pgfpathlineto{\pgfqpoint{-0.048611in}{0.000000in}}%
\pgfusepath{stroke,fill}%
}%
\begin{pgfscope}%
\pgfsys@transformshift{0.706011in}{3.168696in}%
\pgfsys@useobject{currentmarker}{}%
\end{pgfscope}%
\end{pgfscope}%
\begin{pgfscope}%
\definecolor{textcolor}{rgb}{0.000000,0.000000,0.000000}%
\pgfsetstrokecolor{textcolor}%
\pgfsetfillcolor{textcolor}%
\pgftext[x=0.343734in, y=3.105382in, left, base]{\color{textcolor}\sffamily\fontsize{12.000000}{14.400000}\selectfont 0.3}%
\end{pgfscope}%
\begin{pgfscope}%
\pgfsetbuttcap%
\pgfsetroundjoin%
\definecolor{currentfill}{rgb}{0.000000,0.000000,0.000000}%
\pgfsetfillcolor{currentfill}%
\pgfsetlinewidth{0.803000pt}%
\definecolor{currentstroke}{rgb}{0.000000,0.000000,0.000000}%
\pgfsetstrokecolor{currentstroke}%
\pgfsetdash{}{0pt}%
\pgfsys@defobject{currentmarker}{\pgfqpoint{-0.048611in}{0.000000in}}{\pgfqpoint{-0.000000in}{0.000000in}}{%
\pgfpathmoveto{\pgfqpoint{-0.000000in}{0.000000in}}%
\pgfpathlineto{\pgfqpoint{-0.048611in}{0.000000in}}%
\pgfusepath{stroke,fill}%
}%
\begin{pgfscope}%
\pgfsys@transformshift{0.706011in}{3.767394in}%
\pgfsys@useobject{currentmarker}{}%
\end{pgfscope}%
\end{pgfscope}%
\begin{pgfscope}%
\definecolor{textcolor}{rgb}{0.000000,0.000000,0.000000}%
\pgfsetstrokecolor{textcolor}%
\pgfsetfillcolor{textcolor}%
\pgftext[x=0.343734in, y=3.704080in, left, base]{\color{textcolor}\sffamily\fontsize{12.000000}{14.400000}\selectfont 0.4}%
\end{pgfscope}%
\begin{pgfscope}%
\pgfsetbuttcap%
\pgfsetroundjoin%
\definecolor{currentfill}{rgb}{0.000000,0.000000,0.000000}%
\pgfsetfillcolor{currentfill}%
\pgfsetlinewidth{0.803000pt}%
\definecolor{currentstroke}{rgb}{0.000000,0.000000,0.000000}%
\pgfsetstrokecolor{currentstroke}%
\pgfsetdash{}{0pt}%
\pgfsys@defobject{currentmarker}{\pgfqpoint{-0.048611in}{0.000000in}}{\pgfqpoint{-0.000000in}{0.000000in}}{%
\pgfpathmoveto{\pgfqpoint{-0.000000in}{0.000000in}}%
\pgfpathlineto{\pgfqpoint{-0.048611in}{0.000000in}}%
\pgfusepath{stroke,fill}%
}%
\begin{pgfscope}%
\pgfsys@transformshift{0.706011in}{4.366092in}%
\pgfsys@useobject{currentmarker}{}%
\end{pgfscope}%
\end{pgfscope}%
\begin{pgfscope}%
\definecolor{textcolor}{rgb}{0.000000,0.000000,0.000000}%
\pgfsetstrokecolor{textcolor}%
\pgfsetfillcolor{textcolor}%
\pgftext[x=0.343734in, y=4.302778in, left, base]{\color{textcolor}\sffamily\fontsize{12.000000}{14.400000}\selectfont 0.5}%
\end{pgfscope}%
\begin{pgfscope}%
\definecolor{textcolor}{rgb}{0.000000,0.000000,0.000000}%
\pgfsetstrokecolor{textcolor}%
\pgfsetfillcolor{textcolor}%
\pgftext[x=0.288178in,y=2.869347in,,bottom,rotate=90.000000]{\color{textcolor}\sffamily\fontsize{14.000000}{16.800000}\bfseries\selectfont IoU}%
\end{pgfscope}%
\begin{pgfscope}%
\pgfsetrectcap%
\pgfsetmiterjoin%
\pgfsetlinewidth{0.803000pt}%
\definecolor{currentstroke}{rgb}{0.000000,0.000000,0.000000}%
\pgfsetstrokecolor{currentstroke}%
\pgfsetdash{}{0pt}%
\pgfpathmoveto{\pgfqpoint{0.706011in}{1.372601in}}%
\pgfpathlineto{\pgfqpoint{0.706011in}{4.366092in}}%
\pgfusepath{stroke}%
\end{pgfscope}%
\begin{pgfscope}%
\pgfsetrectcap%
\pgfsetmiterjoin%
\pgfsetlinewidth{0.803000pt}%
\definecolor{currentstroke}{rgb}{0.000000,0.000000,0.000000}%
\pgfsetstrokecolor{currentstroke}%
\pgfsetdash{}{0pt}%
\pgfpathmoveto{\pgfqpoint{6.198233in}{1.372601in}}%
\pgfpathlineto{\pgfqpoint{6.198233in}{4.366092in}}%
\pgfusepath{stroke}%
\end{pgfscope}%
\begin{pgfscope}%
\pgfsetrectcap%
\pgfsetmiterjoin%
\pgfsetlinewidth{0.803000pt}%
\definecolor{currentstroke}{rgb}{0.000000,0.000000,0.000000}%
\pgfsetstrokecolor{currentstroke}%
\pgfsetdash{}{0pt}%
\pgfpathmoveto{\pgfqpoint{0.706011in}{1.372601in}}%
\pgfpathlineto{\pgfqpoint{6.198233in}{1.372601in}}%
\pgfusepath{stroke}%
\end{pgfscope}%
\begin{pgfscope}%
\pgfsetrectcap%
\pgfsetmiterjoin%
\pgfsetlinewidth{0.803000pt}%
\definecolor{currentstroke}{rgb}{0.000000,0.000000,0.000000}%
\pgfsetstrokecolor{currentstroke}%
\pgfsetdash{}{0pt}%
\pgfpathmoveto{\pgfqpoint{0.706011in}{4.366092in}}%
\pgfpathlineto{\pgfqpoint{6.198233in}{4.366092in}}%
\pgfusepath{stroke}%
\end{pgfscope}%
\begin{pgfscope}%
\definecolor{textcolor}{rgb}{0.000000,0.000000,0.000000}%
\pgfsetstrokecolor{textcolor}%
\pgfsetfillcolor{textcolor}%
\pgftext[x=1.184925in,y=3.475586in,,bottom]{\color{textcolor}\sffamily\fontsize{9.000000}{10.800000}\selectfont 0.3443}%
\end{pgfscope}%
\begin{pgfscope}%
\definecolor{textcolor}{rgb}{0.000000,0.000000,0.000000}%
\pgfsetstrokecolor{textcolor}%
\pgfsetfillcolor{textcolor}%
\pgftext[x=2.203890in,y=1.921365in,,bottom]{\color{textcolor}\sffamily\fontsize{9.000000}{10.800000}\selectfont 0.0847}%
\end{pgfscope}%
\begin{pgfscope}%
\definecolor{textcolor}{rgb}{0.000000,0.000000,0.000000}%
\pgfsetstrokecolor{textcolor}%
\pgfsetfillcolor{textcolor}%
\pgftext[x=3.222855in,y=2.190779in,,bottom]{\color{textcolor}\sffamily\fontsize{9.000000}{10.800000}\selectfont 0.1297}%
\end{pgfscope}%
\begin{pgfscope}%
\definecolor{textcolor}{rgb}{0.000000,0.000000,0.000000}%
\pgfsetstrokecolor{textcolor}%
\pgfsetfillcolor{textcolor}%
\pgftext[x=4.241820in,y=3.293581in,,bottom]{\color{textcolor}\sffamily\fontsize{9.000000}{10.800000}\selectfont 0.3139}%
\end{pgfscope}%
\begin{pgfscope}%
\definecolor{textcolor}{rgb}{0.000000,0.000000,0.000000}%
\pgfsetstrokecolor{textcolor}%
\pgfsetfillcolor{textcolor}%
\pgftext[x=5.260785in,y=3.263647in,,bottom]{\color{textcolor}\sffamily\fontsize{9.000000}{10.800000}\selectfont 0.3089}%
\end{pgfscope}%
\begin{pgfscope}%
\definecolor{textcolor}{rgb}{0.000000,0.000000,0.000000}%
\pgfsetstrokecolor{textcolor}%
\pgfsetfillcolor{textcolor}%
\pgftext[x=1.643459in,y=3.360037in,,bottom]{\color{textcolor}\sffamily\fontsize{9.000000}{10.800000}\selectfont 0.325}%
\end{pgfscope}%
\begin{pgfscope}%
\definecolor{textcolor}{rgb}{0.000000,0.000000,0.000000}%
\pgfsetstrokecolor{textcolor}%
\pgfsetfillcolor{textcolor}%
\pgftext[x=2.662424in,y=1.756124in,,bottom]{\color{textcolor}\sffamily\fontsize{9.000000}{10.800000}\selectfont 0.0571}%
\end{pgfscope}%
\begin{pgfscope}%
\definecolor{textcolor}{rgb}{0.000000,0.000000,0.000000}%
\pgfsetstrokecolor{textcolor}%
\pgfsetfillcolor{textcolor}%
\pgftext[x=3.681389in,y=2.397929in,,bottom]{\color{textcolor}\sffamily\fontsize{9.000000}{10.800000}\selectfont 0.1643}%
\end{pgfscope}%
\begin{pgfscope}%
\definecolor{textcolor}{rgb}{0.000000,0.000000,0.000000}%
\pgfsetstrokecolor{textcolor}%
\pgfsetfillcolor{textcolor}%
\pgftext[x=4.700355in,y=3.285200in,,bottom]{\color{textcolor}\sffamily\fontsize{9.000000}{10.800000}\selectfont 0.3125}%
\end{pgfscope}%
\begin{pgfscope}%
\definecolor{textcolor}{rgb}{0.000000,0.000000,0.000000}%
\pgfsetstrokecolor{textcolor}%
\pgfsetfillcolor{textcolor}%
\pgftext[x=5.719320in,y=3.412722in,,bottom]{\color{textcolor}\sffamily\fontsize{9.000000}{10.800000}\selectfont 0.3338}%
\end{pgfscope}%
\begin{pgfscope}%
\definecolor{textcolor}{rgb}{0.000000,0.000000,0.000000}%
\pgfsetstrokecolor{textcolor}%
\pgfsetfillcolor{textcolor}%
\pgftext[x=3.452122in,y=4.449426in,,base]{\color{textcolor}\sffamily\fontsize{14.000000}{16.800000}\selectfont Baselines trained on S2R:3DFREE and fine-tuned with Pix3D}%
\end{pgfscope}%
\begin{pgfscope}%
\pgfsetbuttcap%
\pgfsetmiterjoin%
\definecolor{currentfill}{rgb}{1.000000,1.000000,1.000000}%
\pgfsetfillcolor{currentfill}%
\pgfsetfillopacity{0.800000}%
\pgfsetlinewidth{1.003750pt}%
\definecolor{currentstroke}{rgb}{0.800000,0.800000,0.800000}%
\pgfsetstrokecolor{currentstroke}%
\pgfsetstrokeopacity{0.800000}%
\pgfsetdash{}{0pt}%
\pgfpathmoveto{\pgfqpoint{4.886846in}{3.847267in}}%
\pgfpathlineto{\pgfqpoint{6.101011in}{3.847267in}}%
\pgfpathquadraticcurveto{\pgfqpoint{6.128789in}{3.847267in}}{\pgfqpoint{6.128789in}{3.875044in}}%
\pgfpathlineto{\pgfqpoint{6.128789in}{4.268870in}}%
\pgfpathquadraticcurveto{\pgfqpoint{6.128789in}{4.296648in}}{\pgfqpoint{6.101011in}{4.296648in}}%
\pgfpathlineto{\pgfqpoint{4.886846in}{4.296648in}}%
\pgfpathquadraticcurveto{\pgfqpoint{4.859068in}{4.296648in}}{\pgfqpoint{4.859068in}{4.268870in}}%
\pgfpathlineto{\pgfqpoint{4.859068in}{3.875044in}}%
\pgfpathquadraticcurveto{\pgfqpoint{4.859068in}{3.847267in}}{\pgfqpoint{4.886846in}{3.847267in}}%
\pgfpathclose%
\pgfusepath{stroke,fill}%
\end{pgfscope}%
\begin{pgfscope}%
\pgfsetbuttcap%
\pgfsetmiterjoin%
\definecolor{currentfill}{rgb}{0.121569,0.466667,0.705882}%
\pgfsetfillcolor{currentfill}%
\pgfsetlinewidth{0.000000pt}%
\definecolor{currentstroke}{rgb}{0.000000,0.000000,0.000000}%
\pgfsetstrokecolor{currentstroke}%
\pgfsetstrokeopacity{0.000000}%
\pgfsetdash{}{0pt}%
\pgfpathmoveto{\pgfqpoint{4.914623in}{4.135569in}}%
\pgfpathlineto{\pgfqpoint{5.192401in}{4.135569in}}%
\pgfpathlineto{\pgfqpoint{5.192401in}{4.232791in}}%
\pgfpathlineto{\pgfqpoint{4.914623in}{4.232791in}}%
\pgfpathclose%
\pgfusepath{fill}%
\end{pgfscope}%
\begin{pgfscope}%
\definecolor{textcolor}{rgb}{0.000000,0.000000,0.000000}%
\pgfsetstrokecolor{textcolor}%
\pgfsetfillcolor{textcolor}%
\pgftext[x=5.303512in,y=4.135569in,left,base]{\color{textcolor}\sffamily\fontsize{10.000000}{12.000000}\selectfont Pix2Vox++}%
\end{pgfscope}%
\begin{pgfscope}%
\pgfsetbuttcap%
\pgfsetmiterjoin%
\definecolor{currentfill}{rgb}{1.000000,0.498039,0.054902}%
\pgfsetfillcolor{currentfill}%
\pgfsetlinewidth{0.000000pt}%
\definecolor{currentstroke}{rgb}{0.000000,0.000000,0.000000}%
\pgfsetstrokecolor{currentstroke}%
\pgfsetstrokeopacity{0.000000}%
\pgfsetdash{}{0pt}%
\pgfpathmoveto{\pgfqpoint{4.914623in}{3.931712in}}%
\pgfpathlineto{\pgfqpoint{5.192401in}{3.931712in}}%
\pgfpathlineto{\pgfqpoint{5.192401in}{4.028934in}}%
\pgfpathlineto{\pgfqpoint{4.914623in}{4.028934in}}%
\pgfpathclose%
\pgfusepath{fill}%
\end{pgfscope}%
\begin{pgfscope}%
\definecolor{textcolor}{rgb}{0.000000,0.000000,0.000000}%
\pgfsetstrokecolor{textcolor}%
\pgfsetfillcolor{textcolor}%
\pgftext[x=5.303512in,y=3.931712in,left,base]{\color{textcolor}\sffamily\fontsize{10.000000}{12.000000}\selectfont Pix2Vox}%
\end{pgfscope}%
\end{pgfpicture}%
\makeatother%
\endgroup%
}
    \caption[\gls{iou} Comparison for Fine-Tuned Baselines.]{Bar plot for the \gls{iou} for baseline models(Pix2Vox++ and Pix2Vox) trained on synthetic and \textbf{fine-tuned} with real dataset.
    We see that even after fine-tuning both the models do not perform as good as models trained on only real dataset.}
    \label{fig:finetuning1}
\end{figure}

In \autoref{fig:finetuning2} and \autoref{fig:finetuning3}, we observe the \gls{iou} for each category, for Pix2Vox++ and Pix2Vox, respectively.
We can see that model trained on only Pix3D has more \gls{iou} for majority of the categories on average.
For Pix2Vox, \gls{s2rv2} has a slight gain over Pix3D in some categories.
Overall, fine-tuning is not helping in increasing the performance of the baseline models.

A study with \gls{f1} for the fine-tuned baselines is discussed in \autoref{subsec:fine-tuning-dice}.
We observe similar behaviour even with \gls{f1}.

\begin{figure}[!ht]
    \centering
    \resizebox{0.65\textwidth}{!}{\input{/Users/apple/OVGU/Thesis/code/3dReconstruction/report/images/evaluation/performance/finetune_barplot_pix2voxpp.pgf}}
    \caption[\gls{iou} Comparison for Each Category from Fine-Tuned Pix2Vox++.]{Bar plot for the \gls{iou} for baseline \texbf{Pix2Vox++} trained on (\gls{s2rv1}, \gls{s2rv2}) and \textbf{fine-tuned} with Pix3D.
    The categories are listed along with the number of images.
    The performance of \texbf{Pix2Vox++} mixed with both the synthetic dataset is less than model trained on only Pix3D, for majority of the categories.}
    \label{fig:finetuning2}
\end{figure}

\begin{figure}[!ht]
    \centering
    \resizebox{0.65\textwidth}{!}{\input{/Users/apple/OVGU/Thesis/code/3dReconstruction/report/images/evaluation/performance/finetune_barplot_pix2vox.pgf}}
    \caption[\gls{iou} Comparison for Each Category from Fine-Tuned Pix2Vox.]{Bar plot for the \gls{iou} for baseline \texbf{Pix2Vox} trained on (\gls{s2rv1}, \gls{s2rv2}) and \texbf{fine-tuned} with Pix3D.
    The categories are listed along with the number of images.
    The performance of \textbf{Pix2Vox} mixed with both the synthetic dataset is less than model trained on only Pix3D, for majority the categories.}
    \label{fig:finetuning3}
\end{figure}

\begin{figure}[!ht]
    \begin{tabular}{llll}
        Pix3D images & \includegraphics[width=.2\linewidth]{/Users/apple/OVGU/Thesis/code/3dReconstruction/report/images/evaluation/reconstruction/baseline/bed1} &
        \includegraphics[width=.2\linewidth]{/Users/apple/OVGU/Thesis/code/3dReconstruction/report/images/evaluation/reconstruction/baseline/sofa1} &
        \includegraphics[width=.2\linewidth]{/Users/apple/OVGU/Thesis/code/3dReconstruction/report/images/evaluation/reconstruction/baseline/table2}\\

        Ground Truth & \includegraphics[trim={0 0 {.1\width} 0},clip,width=.2\linewidth]{/Users/apple/OVGU/Thesis/code/3dReconstruction/report/images/evaluation/reconstruction/baseline/bed1_original} &
        \includegraphics[trim={0 0 {.1\width} 0},clip,width=.2\linewidth]{/Users/apple/OVGU/Thesis/code/3dReconstruction/report/images/evaluation/reconstruction/baseline/sofa1_original} &
        \includegraphics[trim={0 0 {.1\width} 0},clip,width=.2\linewidth]{/Users/apple/OVGU/Thesis/code/3dReconstruction/report/images/evaluation/reconstruction/baseline/table2_original}\\

        Output1 & \includegraphics[width=.2\linewidth]{/Users/apple/OVGU/Thesis/code/3dReconstruction/report/images/evaluation/reconstruction/baseline/pix3d_p2vpp_bed1_output} &
        \includegraphics[width=.2\linewidth]{/Users/apple/OVGU/Thesis/code/3dReconstruction/report/images/evaluation/reconstruction/baseline/pix3d_p2vpp_sofa1_output} &
        \includegraphics[width=.2\linewidth]{/Users/apple/OVGU/Thesis/code/3dReconstruction/report/images/evaluation/reconstruction/baseline/pix3d_p2vpp_table2}\\

        Output2 & \includegraphics[width=.2\linewidth]{/Users/apple/OVGU/Thesis/code/3dReconstruction/report/images/evaluation/reconstruction/baseline/pix3d_p2v_bed1} &
        \includegraphics[width=.2\linewidth]{/Users/apple/OVGU/Thesis/code/3dReconstruction/report/images/evaluation/reconstruction/baseline/pix3d_p2v_sofa1} &
        \includegraphics[width=.2\linewidth]{/Users/apple/OVGU/Thesis/code/3dReconstruction/report/images/evaluation/reconstruction/baseline/pix3d_p2v_table2}\\

        Output3 & \includegraphics[width=.2\linewidth]{/Users/apple/OVGU/Thesis/code/3dReconstruction/report/images/evaluation/reconstruction/finetuning/s2rv3_p2vpp_bed1} &
        \includegraphics[width=.2\linewidth]{/Users/apple/OVGU/Thesis/code/3dReconstruction/report/images/evaluation/reconstruction/finetuning/s2rv3_p2vpp_sofa1} &
        \includegraphics[width=.2\linewidth]{/Users/apple/OVGU/Thesis/code/3dReconstruction/report/images/evaluation/reconstruction/finetuning/s2rv3_p2vpp_table2}\\

        Output4 & \includegraphics[width=.2\linewidth]{/Users/apple/OVGU/Thesis/code/3dReconstruction/report/images/evaluation/reconstruction/finetuning/s2rv3_p2v_bed1} &
        \includegraphics[width=.2\linewidth]{/Users/apple/OVGU/Thesis/code/3dReconstruction/report/images/evaluation/reconstruction/finetuning/s2rv3_p2v_sofa1} &
        \includegraphics[width=.2\linewidth]{/Users/apple/OVGU/Thesis/code/3dReconstruction/report/images/evaluation/reconstruction/finetuning/s2rv3_p2v_table2}\\

    \end{tabular}
    \caption[3D Reconstruction Outputs for Fine-Tuned Baselines.]{3D reconstruction outputs for models trained on real dataset and synthetic datasets with \textbf{fine-tuning}. Output1-2: Pix2Vox++ and Pix2Vox trained on Pix3D(real dataset).
    Output3-4: Pix2Vox++ and Pix2Vox pre-trained with only \gls{s2rv2} synthetic dataset and then fine-tuned with Pix3D. The reconstruction is better than models trained on only synthetic dataset.}
    \label{fig:finetuning_images1}
\end{figure}

In \autoref{fig:finetuning_images1}, we see the 3D reconstruction output for models trained on synthetic and fine-tuned with real.
The outputs were collected for images from the real dataset with the threshold, which gave the best \gls{iou}.
The output of models improved over models trained on only synthetic dataset as in \autoref{fig:baseline_images1}.
When compared to models trained on only real dataset as in \autoref{fig:finetuning_images1}, the output is noisier with less detailed output.


%\begin{figure}
%    \centering
%    \resizebox{0.75\textwidth}{!}{%% Creator: Matplotlib, PGF backend
%%
%% To include the figure in your LaTeX document, write
%%   \input{<filename>.pgf}
%%
%% Make sure the required packages are loaded in your preamble
%%   \usepackage{pgf}
%%
%% Figures using additional raster images can only be included by \input if
%% they are in the same directory as the main LaTeX file. For loading figures
%% from other directories you can use the `import` package
%%   \usepackage{import}
%%
%% and then include the figures with
%%   \import{<path to file>}{<filename>.pgf}
%%
%% Matplotlib used the following preamble
%%   \usepackage{fontspec}
%%   \setmainfont{DejaVuSerif.ttf}[Path=\detokenize{/Users/apple/opt/anaconda3/envs/kaolin/lib/python3.7/site-packages/matplotlib/mpl-data/fonts/ttf/}]
%%   \setsansfont{DejaVuSans.ttf}[Path=\detokenize{/Users/apple/opt/anaconda3/envs/kaolin/lib/python3.7/site-packages/matplotlib/mpl-data/fonts/ttf/}]
%%   \setmonofont{DejaVuSansMono.ttf}[Path=\detokenize{/Users/apple/opt/anaconda3/envs/kaolin/lib/python3.7/site-packages/matplotlib/mpl-data/fonts/ttf/}]
%%
\begingroup%
\makeatletter%
\begin{pgfpicture}%
\pgfpathrectangle{\pgfpointorigin}{\pgfqpoint{5.830801in}{5.012323in}}%
\pgfusepath{use as bounding box, clip}%
\begin{pgfscope}%
\pgfsetbuttcap%
\pgfsetmiterjoin%
\definecolor{currentfill}{rgb}{1.000000,1.000000,1.000000}%
\pgfsetfillcolor{currentfill}%
\pgfsetlinewidth{0.000000pt}%
\definecolor{currentstroke}{rgb}{1.000000,1.000000,1.000000}%
\pgfsetstrokecolor{currentstroke}%
\pgfsetdash{}{0pt}%
\pgfpathmoveto{\pgfqpoint{0.000000in}{0.000000in}}%
\pgfpathlineto{\pgfqpoint{5.830801in}{0.000000in}}%
\pgfpathlineto{\pgfqpoint{5.830801in}{5.012323in}}%
\pgfpathlineto{\pgfqpoint{0.000000in}{5.012323in}}%
\pgfpathclose%
\pgfusepath{fill}%
\end{pgfscope}%
\begin{pgfscope}%
\pgfsetbuttcap%
\pgfsetmiterjoin%
\definecolor{currentfill}{rgb}{1.000000,1.000000,1.000000}%
\pgfsetfillcolor{currentfill}%
\pgfsetlinewidth{0.000000pt}%
\definecolor{currentstroke}{rgb}{0.000000,0.000000,0.000000}%
\pgfsetstrokecolor{currentstroke}%
\pgfsetstrokeopacity{0.000000}%
\pgfsetdash{}{0pt}%
\pgfpathmoveto{\pgfqpoint{0.608070in}{1.163562in}}%
\pgfpathlineto{\pgfqpoint{5.568070in}{1.163562in}}%
\pgfpathlineto{\pgfqpoint{5.568070in}{4.859562in}}%
\pgfpathlineto{\pgfqpoint{0.608070in}{4.859562in}}%
\pgfpathclose%
\pgfusepath{fill}%
\end{pgfscope}%
\begin{pgfscope}%
\pgfsetbuttcap%
\pgfsetroundjoin%
\definecolor{currentfill}{rgb}{0.000000,0.000000,0.000000}%
\pgfsetfillcolor{currentfill}%
\pgfsetlinewidth{0.803000pt}%
\definecolor{currentstroke}{rgb}{0.000000,0.000000,0.000000}%
\pgfsetstrokecolor{currentstroke}%
\pgfsetdash{}{0pt}%
\pgfsys@defobject{currentmarker}{\pgfqpoint{0.000000in}{-0.048611in}}{\pgfqpoint{0.000000in}{0.000000in}}{%
\pgfpathmoveto{\pgfqpoint{0.000000in}{0.000000in}}%
\pgfpathlineto{\pgfqpoint{0.000000in}{-0.048611in}}%
\pgfusepath{stroke,fill}%
}%
\begin{pgfscope}%
\pgfsys@transformshift{0.833525in}{1.163562in}%
\pgfsys@useobject{currentmarker}{}%
\end{pgfscope}%
\end{pgfscope}%
\begin{pgfscope}%
\definecolor{textcolor}{rgb}{0.000000,0.000000,0.000000}%
\pgfsetstrokecolor{textcolor}%
\pgfsetfillcolor{textcolor}%
\pgftext[x=0.720330in, y=0.711146in, left, base,rotate=45.000000]{\color{textcolor}\sffamily\fontsize{10.000000}{12.000000}\selectfont Pix3D}%
\end{pgfscope}%
\begin{pgfscope}%
\pgfsetbuttcap%
\pgfsetroundjoin%
\definecolor{currentfill}{rgb}{0.000000,0.000000,0.000000}%
\pgfsetfillcolor{currentfill}%
\pgfsetlinewidth{0.803000pt}%
\definecolor{currentstroke}{rgb}{0.000000,0.000000,0.000000}%
\pgfsetstrokecolor{currentstroke}%
\pgfsetdash{}{0pt}%
\pgfsys@defobject{currentmarker}{\pgfqpoint{0.000000in}{-0.048611in}}{\pgfqpoint{0.000000in}{0.000000in}}{%
\pgfpathmoveto{\pgfqpoint{0.000000in}{0.000000in}}%
\pgfpathlineto{\pgfqpoint{0.000000in}{-0.048611in}}%
\pgfusepath{stroke,fill}%
}%
\begin{pgfscope}%
\pgfsys@transformshift{1.960797in}{1.163562in}%
\pgfsys@useobject{currentmarker}{}%
\end{pgfscope}%
\end{pgfscope}%
\begin{pgfscope}%
\definecolor{textcolor}{rgb}{0.000000,0.000000,0.000000}%
\pgfsetstrokecolor{textcolor}%
\pgfsetfillcolor{textcolor}%
\pgftext[x=1.823794in, y=0.669714in, left, base,rotate=45.000000]{\color{textcolor}\sffamily\fontsize{10.000000}{12.000000}\selectfont s2r\_v1}%
\end{pgfscope}%
\begin{pgfscope}%
\pgfsetbuttcap%
\pgfsetroundjoin%
\definecolor{currentfill}{rgb}{0.000000,0.000000,0.000000}%
\pgfsetfillcolor{currentfill}%
\pgfsetlinewidth{0.803000pt}%
\definecolor{currentstroke}{rgb}{0.000000,0.000000,0.000000}%
\pgfsetstrokecolor{currentstroke}%
\pgfsetdash{}{0pt}%
\pgfsys@defobject{currentmarker}{\pgfqpoint{0.000000in}{-0.048611in}}{\pgfqpoint{0.000000in}{0.000000in}}{%
\pgfpathmoveto{\pgfqpoint{0.000000in}{0.000000in}}%
\pgfpathlineto{\pgfqpoint{0.000000in}{-0.048611in}}%
\pgfusepath{stroke,fill}%
}%
\begin{pgfscope}%
\pgfsys@transformshift{3.088070in}{1.163562in}%
\pgfsys@useobject{currentmarker}{}%
\end{pgfscope}%
\end{pgfscope}%
\begin{pgfscope}%
\definecolor{textcolor}{rgb}{0.000000,0.000000,0.000000}%
\pgfsetstrokecolor{textcolor}%
\pgfsetfillcolor{textcolor}%
\pgftext[x=2.951066in, y=0.669714in, left, base,rotate=45.000000]{\color{textcolor}\sffamily\fontsize{10.000000}{12.000000}\selectfont s2r\_v2}%
\end{pgfscope}%
\begin{pgfscope}%
\pgfsetbuttcap%
\pgfsetroundjoin%
\definecolor{currentfill}{rgb}{0.000000,0.000000,0.000000}%
\pgfsetfillcolor{currentfill}%
\pgfsetlinewidth{0.803000pt}%
\definecolor{currentstroke}{rgb}{0.000000,0.000000,0.000000}%
\pgfsetstrokecolor{currentstroke}%
\pgfsetdash{}{0pt}%
\pgfsys@defobject{currentmarker}{\pgfqpoint{0.000000in}{-0.048611in}}{\pgfqpoint{0.000000in}{0.000000in}}{%
\pgfpathmoveto{\pgfqpoint{0.000000in}{0.000000in}}%
\pgfpathlineto{\pgfqpoint{0.000000in}{-0.048611in}}%
\pgfusepath{stroke,fill}%
}%
\begin{pgfscope}%
\pgfsys@transformshift{4.215343in}{1.163562in}%
\pgfsys@useobject{currentmarker}{}%
\end{pgfscope}%
\end{pgfscope}%
\begin{pgfscope}%
\definecolor{textcolor}{rgb}{0.000000,0.000000,0.000000}%
\pgfsetstrokecolor{textcolor}%
\pgfsetfillcolor{textcolor}%
\pgftext[x=3.901773in, y=0.313130in, left, base,rotate=45.000000]{\color{textcolor}\sffamily\fontsize{10.000000}{12.000000}\selectfont s2r\_v1+pix3d}%
\end{pgfscope}%
\begin{pgfscope}%
\pgfsetbuttcap%
\pgfsetroundjoin%
\definecolor{currentfill}{rgb}{0.000000,0.000000,0.000000}%
\pgfsetfillcolor{currentfill}%
\pgfsetlinewidth{0.803000pt}%
\definecolor{currentstroke}{rgb}{0.000000,0.000000,0.000000}%
\pgfsetstrokecolor{currentstroke}%
\pgfsetdash{}{0pt}%
\pgfsys@defobject{currentmarker}{\pgfqpoint{0.000000in}{-0.048611in}}{\pgfqpoint{0.000000in}{0.000000in}}{%
\pgfpathmoveto{\pgfqpoint{0.000000in}{0.000000in}}%
\pgfpathlineto{\pgfqpoint{0.000000in}{-0.048611in}}%
\pgfusepath{stroke,fill}%
}%
\begin{pgfscope}%
\pgfsys@transformshift{5.342615in}{1.163562in}%
\pgfsys@useobject{currentmarker}{}%
\end{pgfscope}%
\end{pgfscope}%
\begin{pgfscope}%
\definecolor{textcolor}{rgb}{0.000000,0.000000,0.000000}%
\pgfsetstrokecolor{textcolor}%
\pgfsetfillcolor{textcolor}%
\pgftext[x=5.029046in, y=0.313130in, left, base,rotate=45.000000]{\color{textcolor}\sffamily\fontsize{10.000000}{12.000000}\selectfont s2r\_v2+pix3d}%
\end{pgfscope}%
\begin{pgfscope}%
\definecolor{textcolor}{rgb}{0.000000,0.000000,0.000000}%
\pgfsetstrokecolor{textcolor}%
\pgfsetfillcolor{textcolor}%
\pgftext[x=3.088070in,y=0.234413in,,top]{\color{textcolor}\sffamily\fontsize{10.000000}{12.000000}\selectfont Datasets}%
\end{pgfscope}%
\begin{pgfscope}%
\pgfsetbuttcap%
\pgfsetroundjoin%
\definecolor{currentfill}{rgb}{0.000000,0.000000,0.000000}%
\pgfsetfillcolor{currentfill}%
\pgfsetlinewidth{0.803000pt}%
\definecolor{currentstroke}{rgb}{0.000000,0.000000,0.000000}%
\pgfsetstrokecolor{currentstroke}%
\pgfsetdash{}{0pt}%
\pgfsys@defobject{currentmarker}{\pgfqpoint{-0.048611in}{0.000000in}}{\pgfqpoint{-0.000000in}{0.000000in}}{%
\pgfpathmoveto{\pgfqpoint{-0.000000in}{0.000000in}}%
\pgfpathlineto{\pgfqpoint{-0.048611in}{0.000000in}}%
\pgfusepath{stroke,fill}%
}%
\begin{pgfscope}%
\pgfsys@transformshift{0.608070in}{1.163562in}%
\pgfsys@useobject{currentmarker}{}%
\end{pgfscope}%
\end{pgfscope}%
\begin{pgfscope}%
\definecolor{textcolor}{rgb}{0.000000,0.000000,0.000000}%
\pgfsetstrokecolor{textcolor}%
\pgfsetfillcolor{textcolor}%
\pgftext[x=0.289968in, y=1.110800in, left, base]{\color{textcolor}\sffamily\fontsize{10.000000}{12.000000}\selectfont 0.0}%
\end{pgfscope}%
\begin{pgfscope}%
\pgfsetbuttcap%
\pgfsetroundjoin%
\definecolor{currentfill}{rgb}{0.000000,0.000000,0.000000}%
\pgfsetfillcolor{currentfill}%
\pgfsetlinewidth{0.803000pt}%
\definecolor{currentstroke}{rgb}{0.000000,0.000000,0.000000}%
\pgfsetstrokecolor{currentstroke}%
\pgfsetdash{}{0pt}%
\pgfsys@defobject{currentmarker}{\pgfqpoint{-0.048611in}{0.000000in}}{\pgfqpoint{-0.000000in}{0.000000in}}{%
\pgfpathmoveto{\pgfqpoint{-0.000000in}{0.000000in}}%
\pgfpathlineto{\pgfqpoint{-0.048611in}{0.000000in}}%
\pgfusepath{stroke,fill}%
}%
\begin{pgfscope}%
\pgfsys@transformshift{0.608070in}{1.902762in}%
\pgfsys@useobject{currentmarker}{}%
\end{pgfscope}%
\end{pgfscope}%
\begin{pgfscope}%
\definecolor{textcolor}{rgb}{0.000000,0.000000,0.000000}%
\pgfsetstrokecolor{textcolor}%
\pgfsetfillcolor{textcolor}%
\pgftext[x=0.289968in, y=1.850000in, left, base]{\color{textcolor}\sffamily\fontsize{10.000000}{12.000000}\selectfont 0.1}%
\end{pgfscope}%
\begin{pgfscope}%
\pgfsetbuttcap%
\pgfsetroundjoin%
\definecolor{currentfill}{rgb}{0.000000,0.000000,0.000000}%
\pgfsetfillcolor{currentfill}%
\pgfsetlinewidth{0.803000pt}%
\definecolor{currentstroke}{rgb}{0.000000,0.000000,0.000000}%
\pgfsetstrokecolor{currentstroke}%
\pgfsetdash{}{0pt}%
\pgfsys@defobject{currentmarker}{\pgfqpoint{-0.048611in}{0.000000in}}{\pgfqpoint{-0.000000in}{0.000000in}}{%
\pgfpathmoveto{\pgfqpoint{-0.000000in}{0.000000in}}%
\pgfpathlineto{\pgfqpoint{-0.048611in}{0.000000in}}%
\pgfusepath{stroke,fill}%
}%
\begin{pgfscope}%
\pgfsys@transformshift{0.608070in}{2.641962in}%
\pgfsys@useobject{currentmarker}{}%
\end{pgfscope}%
\end{pgfscope}%
\begin{pgfscope}%
\definecolor{textcolor}{rgb}{0.000000,0.000000,0.000000}%
\pgfsetstrokecolor{textcolor}%
\pgfsetfillcolor{textcolor}%
\pgftext[x=0.289968in, y=2.589200in, left, base]{\color{textcolor}\sffamily\fontsize{10.000000}{12.000000}\selectfont 0.2}%
\end{pgfscope}%
\begin{pgfscope}%
\pgfsetbuttcap%
\pgfsetroundjoin%
\definecolor{currentfill}{rgb}{0.000000,0.000000,0.000000}%
\pgfsetfillcolor{currentfill}%
\pgfsetlinewidth{0.803000pt}%
\definecolor{currentstroke}{rgb}{0.000000,0.000000,0.000000}%
\pgfsetstrokecolor{currentstroke}%
\pgfsetdash{}{0pt}%
\pgfsys@defobject{currentmarker}{\pgfqpoint{-0.048611in}{0.000000in}}{\pgfqpoint{-0.000000in}{0.000000in}}{%
\pgfpathmoveto{\pgfqpoint{-0.000000in}{0.000000in}}%
\pgfpathlineto{\pgfqpoint{-0.048611in}{0.000000in}}%
\pgfusepath{stroke,fill}%
}%
\begin{pgfscope}%
\pgfsys@transformshift{0.608070in}{3.381162in}%
\pgfsys@useobject{currentmarker}{}%
\end{pgfscope}%
\end{pgfscope}%
\begin{pgfscope}%
\definecolor{textcolor}{rgb}{0.000000,0.000000,0.000000}%
\pgfsetstrokecolor{textcolor}%
\pgfsetfillcolor{textcolor}%
\pgftext[x=0.289968in, y=3.328400in, left, base]{\color{textcolor}\sffamily\fontsize{10.000000}{12.000000}\selectfont 0.3}%
\end{pgfscope}%
\begin{pgfscope}%
\pgfsetbuttcap%
\pgfsetroundjoin%
\definecolor{currentfill}{rgb}{0.000000,0.000000,0.000000}%
\pgfsetfillcolor{currentfill}%
\pgfsetlinewidth{0.803000pt}%
\definecolor{currentstroke}{rgb}{0.000000,0.000000,0.000000}%
\pgfsetstrokecolor{currentstroke}%
\pgfsetdash{}{0pt}%
\pgfsys@defobject{currentmarker}{\pgfqpoint{-0.048611in}{0.000000in}}{\pgfqpoint{-0.000000in}{0.000000in}}{%
\pgfpathmoveto{\pgfqpoint{-0.000000in}{0.000000in}}%
\pgfpathlineto{\pgfqpoint{-0.048611in}{0.000000in}}%
\pgfusepath{stroke,fill}%
}%
\begin{pgfscope}%
\pgfsys@transformshift{0.608070in}{4.120362in}%
\pgfsys@useobject{currentmarker}{}%
\end{pgfscope}%
\end{pgfscope}%
\begin{pgfscope}%
\definecolor{textcolor}{rgb}{0.000000,0.000000,0.000000}%
\pgfsetstrokecolor{textcolor}%
\pgfsetfillcolor{textcolor}%
\pgftext[x=0.289968in, y=4.067600in, left, base]{\color{textcolor}\sffamily\fontsize{10.000000}{12.000000}\selectfont 0.4}%
\end{pgfscope}%
\begin{pgfscope}%
\pgfsetbuttcap%
\pgfsetroundjoin%
\definecolor{currentfill}{rgb}{0.000000,0.000000,0.000000}%
\pgfsetfillcolor{currentfill}%
\pgfsetlinewidth{0.803000pt}%
\definecolor{currentstroke}{rgb}{0.000000,0.000000,0.000000}%
\pgfsetstrokecolor{currentstroke}%
\pgfsetdash{}{0pt}%
\pgfsys@defobject{currentmarker}{\pgfqpoint{-0.048611in}{0.000000in}}{\pgfqpoint{-0.000000in}{0.000000in}}{%
\pgfpathmoveto{\pgfqpoint{-0.000000in}{0.000000in}}%
\pgfpathlineto{\pgfqpoint{-0.048611in}{0.000000in}}%
\pgfusepath{stroke,fill}%
}%
\begin{pgfscope}%
\pgfsys@transformshift{0.608070in}{4.859562in}%
\pgfsys@useobject{currentmarker}{}%
\end{pgfscope}%
\end{pgfscope}%
\begin{pgfscope}%
\definecolor{textcolor}{rgb}{0.000000,0.000000,0.000000}%
\pgfsetstrokecolor{textcolor}%
\pgfsetfillcolor{textcolor}%
\pgftext[x=0.289968in, y=4.806800in, left, base]{\color{textcolor}\sffamily\fontsize{10.000000}{12.000000}\selectfont 0.5}%
\end{pgfscope}%
\begin{pgfscope}%
\definecolor{textcolor}{rgb}{0.000000,0.000000,0.000000}%
\pgfsetstrokecolor{textcolor}%
\pgfsetfillcolor{textcolor}%
\pgftext[x=0.234413in,y=3.011562in,,bottom,rotate=90.000000]{\color{textcolor}\sffamily\fontsize{10.000000}{12.000000}\selectfont IoU}%
\end{pgfscope}%
\begin{pgfscope}%
\pgfpathrectangle{\pgfqpoint{0.608070in}{1.163562in}}{\pgfqpoint{4.960000in}{3.696000in}}%
\pgfusepath{clip}%
\pgfsetrectcap%
\pgfsetroundjoin%
\pgfsetlinewidth{1.505625pt}%
\definecolor{currentstroke}{rgb}{0.121569,0.466667,0.705882}%
\pgfsetstrokecolor{currentstroke}%
\pgfsetdash{}{0pt}%
\pgfpathmoveto{\pgfqpoint{0.833525in}{3.708627in}}%
\pgfpathlineto{\pgfqpoint{1.960797in}{1.789664in}}%
\pgfpathlineto{\pgfqpoint{3.088070in}{2.122304in}}%
\pgfpathlineto{\pgfqpoint{4.215343in}{3.483910in}}%
\pgfpathlineto{\pgfqpoint{5.342615in}{3.446950in}}%
\pgfusepath{stroke}%
\end{pgfscope}%
\begin{pgfscope}%
\pgfpathrectangle{\pgfqpoint{0.608070in}{1.163562in}}{\pgfqpoint{4.960000in}{3.696000in}}%
\pgfusepath{clip}%
\pgfsetbuttcap%
\pgfsetroundjoin%
\definecolor{currentfill}{rgb}{0.121569,0.466667,0.705882}%
\pgfsetfillcolor{currentfill}%
\pgfsetlinewidth{1.003750pt}%
\definecolor{currentstroke}{rgb}{0.121569,0.466667,0.705882}%
\pgfsetstrokecolor{currentstroke}%
\pgfsetdash{}{0pt}%
\pgfsys@defobject{currentmarker}{\pgfqpoint{-0.041667in}{-0.041667in}}{\pgfqpoint{0.041667in}{0.041667in}}{%
\pgfpathmoveto{\pgfqpoint{0.000000in}{-0.041667in}}%
\pgfpathcurveto{\pgfqpoint{0.011050in}{-0.041667in}}{\pgfqpoint{0.021649in}{-0.037276in}}{\pgfqpoint{0.029463in}{-0.029463in}}%
\pgfpathcurveto{\pgfqpoint{0.037276in}{-0.021649in}}{\pgfqpoint{0.041667in}{-0.011050in}}{\pgfqpoint{0.041667in}{0.000000in}}%
\pgfpathcurveto{\pgfqpoint{0.041667in}{0.011050in}}{\pgfqpoint{0.037276in}{0.021649in}}{\pgfqpoint{0.029463in}{0.029463in}}%
\pgfpathcurveto{\pgfqpoint{0.021649in}{0.037276in}}{\pgfqpoint{0.011050in}{0.041667in}}{\pgfqpoint{0.000000in}{0.041667in}}%
\pgfpathcurveto{\pgfqpoint{-0.011050in}{0.041667in}}{\pgfqpoint{-0.021649in}{0.037276in}}{\pgfqpoint{-0.029463in}{0.029463in}}%
\pgfpathcurveto{\pgfqpoint{-0.037276in}{0.021649in}}{\pgfqpoint{-0.041667in}{0.011050in}}{\pgfqpoint{-0.041667in}{0.000000in}}%
\pgfpathcurveto{\pgfqpoint{-0.041667in}{-0.011050in}}{\pgfqpoint{-0.037276in}{-0.021649in}}{\pgfqpoint{-0.029463in}{-0.029463in}}%
\pgfpathcurveto{\pgfqpoint{-0.021649in}{-0.037276in}}{\pgfqpoint{-0.011050in}{-0.041667in}}{\pgfqpoint{0.000000in}{-0.041667in}}%
\pgfpathclose%
\pgfusepath{stroke,fill}%
}%
\begin{pgfscope}%
\pgfsys@transformshift{0.833525in}{3.708627in}%
\pgfsys@useobject{currentmarker}{}%
\end{pgfscope}%
\begin{pgfscope}%
\pgfsys@transformshift{1.960797in}{1.789664in}%
\pgfsys@useobject{currentmarker}{}%
\end{pgfscope}%
\begin{pgfscope}%
\pgfsys@transformshift{3.088070in}{2.122304in}%
\pgfsys@useobject{currentmarker}{}%
\end{pgfscope}%
\begin{pgfscope}%
\pgfsys@transformshift{4.215343in}{3.483910in}%
\pgfsys@useobject{currentmarker}{}%
\end{pgfscope}%
\begin{pgfscope}%
\pgfsys@transformshift{5.342615in}{3.446950in}%
\pgfsys@useobject{currentmarker}{}%
\end{pgfscope}%
\end{pgfscope}%
\begin{pgfscope}%
\pgfpathrectangle{\pgfqpoint{0.608070in}{1.163562in}}{\pgfqpoint{4.960000in}{3.696000in}}%
\pgfusepath{clip}%
\pgfsetrectcap%
\pgfsetroundjoin%
\pgfsetlinewidth{1.505625pt}%
\definecolor{currentstroke}{rgb}{1.000000,0.498039,0.054902}%
\pgfsetstrokecolor{currentstroke}%
\pgfsetdash{}{0pt}%
\pgfpathmoveto{\pgfqpoint{0.833525in}{3.565962in}}%
\pgfpathlineto{\pgfqpoint{1.960797in}{1.585645in}}%
\pgfpathlineto{\pgfqpoint{3.088070in}{2.378067in}}%
\pgfpathlineto{\pgfqpoint{4.215343in}{3.473562in}}%
\pgfpathlineto{\pgfqpoint{5.342615in}{3.631011in}}%
\pgfusepath{stroke}%
\end{pgfscope}%
\begin{pgfscope}%
\pgfpathrectangle{\pgfqpoint{0.608070in}{1.163562in}}{\pgfqpoint{4.960000in}{3.696000in}}%
\pgfusepath{clip}%
\pgfsetbuttcap%
\pgfsetmiterjoin%
\definecolor{currentfill}{rgb}{1.000000,0.498039,0.054902}%
\pgfsetfillcolor{currentfill}%
\pgfsetlinewidth{1.003750pt}%
\definecolor{currentstroke}{rgb}{1.000000,0.498039,0.054902}%
\pgfsetstrokecolor{currentstroke}%
\pgfsetdash{}{0pt}%
\pgfsys@defobject{currentmarker}{\pgfqpoint{-0.041667in}{-0.041667in}}{\pgfqpoint{0.041667in}{0.041667in}}{%
\pgfpathmoveto{\pgfqpoint{-0.000000in}{-0.041667in}}%
\pgfpathlineto{\pgfqpoint{0.041667in}{0.041667in}}%
\pgfpathlineto{\pgfqpoint{-0.041667in}{0.041667in}}%
\pgfpathclose%
\pgfusepath{stroke,fill}%
}%
\begin{pgfscope}%
\pgfsys@transformshift{0.833525in}{3.565962in}%
\pgfsys@useobject{currentmarker}{}%
\end{pgfscope}%
\begin{pgfscope}%
\pgfsys@transformshift{1.960797in}{1.585645in}%
\pgfsys@useobject{currentmarker}{}%
\end{pgfscope}%
\begin{pgfscope}%
\pgfsys@transformshift{3.088070in}{2.378067in}%
\pgfsys@useobject{currentmarker}{}%
\end{pgfscope}%
\begin{pgfscope}%
\pgfsys@transformshift{4.215343in}{3.473562in}%
\pgfsys@useobject{currentmarker}{}%
\end{pgfscope}%
\begin{pgfscope}%
\pgfsys@transformshift{5.342615in}{3.631011in}%
\pgfsys@useobject{currentmarker}{}%
\end{pgfscope}%
\end{pgfscope}%
\begin{pgfscope}%
\pgfsetrectcap%
\pgfsetmiterjoin%
\pgfsetlinewidth{0.803000pt}%
\definecolor{currentstroke}{rgb}{0.000000,0.000000,0.000000}%
\pgfsetstrokecolor{currentstroke}%
\pgfsetdash{}{0pt}%
\pgfpathmoveto{\pgfqpoint{0.608070in}{1.163562in}}%
\pgfpathlineto{\pgfqpoint{0.608070in}{4.859562in}}%
\pgfusepath{stroke}%
\end{pgfscope}%
\begin{pgfscope}%
\pgfsetrectcap%
\pgfsetmiterjoin%
\pgfsetlinewidth{0.803000pt}%
\definecolor{currentstroke}{rgb}{0.000000,0.000000,0.000000}%
\pgfsetstrokecolor{currentstroke}%
\pgfsetdash{}{0pt}%
\pgfpathmoveto{\pgfqpoint{5.568070in}{1.163562in}}%
\pgfpathlineto{\pgfqpoint{5.568070in}{4.859562in}}%
\pgfusepath{stroke}%
\end{pgfscope}%
\begin{pgfscope}%
\pgfsetrectcap%
\pgfsetmiterjoin%
\pgfsetlinewidth{0.803000pt}%
\definecolor{currentstroke}{rgb}{0.000000,0.000000,0.000000}%
\pgfsetstrokecolor{currentstroke}%
\pgfsetdash{}{0pt}%
\pgfpathmoveto{\pgfqpoint{0.608070in}{1.163562in}}%
\pgfpathlineto{\pgfqpoint{5.568070in}{1.163562in}}%
\pgfusepath{stroke}%
\end{pgfscope}%
\begin{pgfscope}%
\pgfsetrectcap%
\pgfsetmiterjoin%
\pgfsetlinewidth{0.803000pt}%
\definecolor{currentstroke}{rgb}{0.000000,0.000000,0.000000}%
\pgfsetstrokecolor{currentstroke}%
\pgfsetdash{}{0pt}%
\pgfpathmoveto{\pgfqpoint{0.608070in}{4.859562in}}%
\pgfpathlineto{\pgfqpoint{5.568070in}{4.859562in}}%
\pgfusepath{stroke}%
\end{pgfscope}%
\begin{pgfscope}%
\pgfsetbuttcap%
\pgfsetmiterjoin%
\definecolor{currentfill}{rgb}{1.000000,1.000000,1.000000}%
\pgfsetfillcolor{currentfill}%
\pgfsetfillopacity{0.800000}%
\pgfsetlinewidth{1.003750pt}%
\definecolor{currentstroke}{rgb}{0.800000,0.800000,0.800000}%
\pgfsetstrokecolor{currentstroke}%
\pgfsetstrokeopacity{0.800000}%
\pgfsetdash{}{0pt}%
\pgfpathmoveto{\pgfqpoint{4.256682in}{4.340736in}}%
\pgfpathlineto{\pgfqpoint{5.470848in}{4.340736in}}%
\pgfpathquadraticcurveto{\pgfqpoint{5.498626in}{4.340736in}}{\pgfqpoint{5.498626in}{4.368514in}}%
\pgfpathlineto{\pgfqpoint{5.498626in}{4.762339in}}%
\pgfpathquadraticcurveto{\pgfqpoint{5.498626in}{4.790117in}}{\pgfqpoint{5.470848in}{4.790117in}}%
\pgfpathlineto{\pgfqpoint{4.256682in}{4.790117in}}%
\pgfpathquadraticcurveto{\pgfqpoint{4.228904in}{4.790117in}}{\pgfqpoint{4.228904in}{4.762339in}}%
\pgfpathlineto{\pgfqpoint{4.228904in}{4.368514in}}%
\pgfpathquadraticcurveto{\pgfqpoint{4.228904in}{4.340736in}}{\pgfqpoint{4.256682in}{4.340736in}}%
\pgfpathclose%
\pgfusepath{stroke,fill}%
\end{pgfscope}%
\begin{pgfscope}%
\pgfsetrectcap%
\pgfsetroundjoin%
\pgfsetlinewidth{1.505625pt}%
\definecolor{currentstroke}{rgb}{0.121569,0.466667,0.705882}%
\pgfsetstrokecolor{currentstroke}%
\pgfsetdash{}{0pt}%
\pgfpathmoveto{\pgfqpoint{4.284460in}{4.677650in}}%
\pgfpathlineto{\pgfqpoint{4.562238in}{4.677650in}}%
\pgfusepath{stroke}%
\end{pgfscope}%
\begin{pgfscope}%
\pgfsetbuttcap%
\pgfsetroundjoin%
\definecolor{currentfill}{rgb}{0.121569,0.466667,0.705882}%
\pgfsetfillcolor{currentfill}%
\pgfsetlinewidth{1.003750pt}%
\definecolor{currentstroke}{rgb}{0.121569,0.466667,0.705882}%
\pgfsetstrokecolor{currentstroke}%
\pgfsetdash{}{0pt}%
\pgfsys@defobject{currentmarker}{\pgfqpoint{-0.041667in}{-0.041667in}}{\pgfqpoint{0.041667in}{0.041667in}}{%
\pgfpathmoveto{\pgfqpoint{0.000000in}{-0.041667in}}%
\pgfpathcurveto{\pgfqpoint{0.011050in}{-0.041667in}}{\pgfqpoint{0.021649in}{-0.037276in}}{\pgfqpoint{0.029463in}{-0.029463in}}%
\pgfpathcurveto{\pgfqpoint{0.037276in}{-0.021649in}}{\pgfqpoint{0.041667in}{-0.011050in}}{\pgfqpoint{0.041667in}{0.000000in}}%
\pgfpathcurveto{\pgfqpoint{0.041667in}{0.011050in}}{\pgfqpoint{0.037276in}{0.021649in}}{\pgfqpoint{0.029463in}{0.029463in}}%
\pgfpathcurveto{\pgfqpoint{0.021649in}{0.037276in}}{\pgfqpoint{0.011050in}{0.041667in}}{\pgfqpoint{0.000000in}{0.041667in}}%
\pgfpathcurveto{\pgfqpoint{-0.011050in}{0.041667in}}{\pgfqpoint{-0.021649in}{0.037276in}}{\pgfqpoint{-0.029463in}{0.029463in}}%
\pgfpathcurveto{\pgfqpoint{-0.037276in}{0.021649in}}{\pgfqpoint{-0.041667in}{0.011050in}}{\pgfqpoint{-0.041667in}{0.000000in}}%
\pgfpathcurveto{\pgfqpoint{-0.041667in}{-0.011050in}}{\pgfqpoint{-0.037276in}{-0.021649in}}{\pgfqpoint{-0.029463in}{-0.029463in}}%
\pgfpathcurveto{\pgfqpoint{-0.021649in}{-0.037276in}}{\pgfqpoint{-0.011050in}{-0.041667in}}{\pgfqpoint{0.000000in}{-0.041667in}}%
\pgfpathclose%
\pgfusepath{stroke,fill}%
}%
\begin{pgfscope}%
\pgfsys@transformshift{4.423349in}{4.677650in}%
\pgfsys@useobject{currentmarker}{}%
\end{pgfscope}%
\end{pgfscope}%
\begin{pgfscope}%
\definecolor{textcolor}{rgb}{0.000000,0.000000,0.000000}%
\pgfsetstrokecolor{textcolor}%
\pgfsetfillcolor{textcolor}%
\pgftext[x=4.673349in,y=4.629039in,left,base]{\color{textcolor}\sffamily\fontsize{10.000000}{12.000000}\selectfont Pix2Vox++}%
\end{pgfscope}%
\begin{pgfscope}%
\pgfsetrectcap%
\pgfsetroundjoin%
\pgfsetlinewidth{1.505625pt}%
\definecolor{currentstroke}{rgb}{1.000000,0.498039,0.054902}%
\pgfsetstrokecolor{currentstroke}%
\pgfsetdash{}{0pt}%
\pgfpathmoveto{\pgfqpoint{4.284460in}{4.473792in}}%
\pgfpathlineto{\pgfqpoint{4.562238in}{4.473792in}}%
\pgfusepath{stroke}%
\end{pgfscope}%
\begin{pgfscope}%
\pgfsetbuttcap%
\pgfsetmiterjoin%
\definecolor{currentfill}{rgb}{1.000000,0.498039,0.054902}%
\pgfsetfillcolor{currentfill}%
\pgfsetlinewidth{1.003750pt}%
\definecolor{currentstroke}{rgb}{1.000000,0.498039,0.054902}%
\pgfsetstrokecolor{currentstroke}%
\pgfsetdash{}{0pt}%
\pgfsys@defobject{currentmarker}{\pgfqpoint{-0.041667in}{-0.041667in}}{\pgfqpoint{0.041667in}{0.041667in}}{%
\pgfpathmoveto{\pgfqpoint{-0.000000in}{-0.041667in}}%
\pgfpathlineto{\pgfqpoint{0.041667in}{0.041667in}}%
\pgfpathlineto{\pgfqpoint{-0.041667in}{0.041667in}}%
\pgfpathclose%
\pgfusepath{stroke,fill}%
}%
\begin{pgfscope}%
\pgfsys@transformshift{4.423349in}{4.473792in}%
\pgfsys@useobject{currentmarker}{}%
\end{pgfscope}%
\end{pgfscope}%
\begin{pgfscope}%
\definecolor{textcolor}{rgb}{0.000000,0.000000,0.000000}%
\pgfsetstrokecolor{textcolor}%
\pgfsetfillcolor{textcolor}%
\pgftext[x=4.673349in,y=4.425181in,left,base]{\color{textcolor}\sffamily\fontsize{10.000000}{12.000000}\selectfont Pix2Vox}%
\end{pgfscope}%
\end{pgfpicture}%
\makeatother%
\endgroup%
}
%    \caption{Line plot for the \gls{iou} for baseline models(Pix2Vox++ and Pix2Vox) trained on synthetic and fine-tuned with real dataset.
%    We see that even after fine-tuning both the models do not perform as good as models trained on only real dataset.}
%    \label{fig:finetuning1}
%\end{figure}


\section{Mixed Training}\label{sec:mixed-training}

\begin{figure}[!ht]
    \centering
    \subfloat[][]{\resizebox{0.75\linewidth}{!}{%% Creator: Matplotlib, PGF backend
%%
%% To include the figure in your LaTeX document, write
%%   \input{<filename>.pgf}
%%
%% Make sure the required packages are loaded in your preamble
%%   \usepackage{pgf}
%%
%% Figures using additional raster images can only be included by \input if
%% they are in the same directory as the main LaTeX file. For loading figures
%% from other directories you can use the `import` package
%%   \usepackage{import}
%%
%% and then include the figures with
%%   \import{<path to file>}{<filename>.pgf}
%%
%% Matplotlib used the following preamble
%%   \usepackage{fontspec}
%%   \setmainfont{DejaVuSerif.ttf}[Path=\detokenize{/Users/apple/opt/anaconda3/envs/kaolin/lib/python3.7/site-packages/matplotlib/mpl-data/fonts/ttf/}]
%%   \setsansfont{DejaVuSans.ttf}[Path=\detokenize{/Users/apple/opt/anaconda3/envs/kaolin/lib/python3.7/site-packages/matplotlib/mpl-data/fonts/ttf/}]
%%   \setmonofont{DejaVuSansMono.ttf}[Path=\detokenize{/Users/apple/opt/anaconda3/envs/kaolin/lib/python3.7/site-packages/matplotlib/mpl-data/fonts/ttf/}]
%%
\begingroup%
\makeatletter%
\begin{pgfpicture}%
\pgfpathrectangle{\pgfpointorigin}{\pgfqpoint{6.293658in}{4.697056in}}%
\pgfusepath{use as bounding box, clip}%
\begin{pgfscope}%
\pgfsetbuttcap%
\pgfsetmiterjoin%
\definecolor{currentfill}{rgb}{1.000000,1.000000,1.000000}%
\pgfsetfillcolor{currentfill}%
\pgfsetlinewidth{0.000000pt}%
\definecolor{currentstroke}{rgb}{1.000000,1.000000,1.000000}%
\pgfsetstrokecolor{currentstroke}%
\pgfsetdash{}{0pt}%
\pgfpathmoveto{\pgfqpoint{-0.000000in}{0.000000in}}%
\pgfpathlineto{\pgfqpoint{6.293658in}{0.000000in}}%
\pgfpathlineto{\pgfqpoint{6.293658in}{4.697056in}}%
\pgfpathlineto{\pgfqpoint{-0.000000in}{4.697056in}}%
\pgfpathclose%
\pgfusepath{fill}%
\end{pgfscope}%
\begin{pgfscope}%
\pgfsetbuttcap%
\pgfsetmiterjoin%
\definecolor{currentfill}{rgb}{1.000000,1.000000,1.000000}%
\pgfsetfillcolor{currentfill}%
\pgfsetlinewidth{0.000000pt}%
\definecolor{currentstroke}{rgb}{0.000000,0.000000,0.000000}%
\pgfsetstrokecolor{currentstroke}%
\pgfsetstrokeopacity{0.000000}%
\pgfsetdash{}{0pt}%
\pgfpathmoveto{\pgfqpoint{0.696435in}{0.700520in}}%
\pgfpathlineto{\pgfqpoint{6.193658in}{0.700520in}}%
\pgfpathlineto{\pgfqpoint{6.193658in}{4.387095in}}%
\pgfpathlineto{\pgfqpoint{0.696435in}{4.387095in}}%
\pgfpathclose%
\pgfusepath{fill}%
\end{pgfscope}%
\begin{pgfscope}%
\pgfpathrectangle{\pgfqpoint{0.696435in}{0.700520in}}{\pgfqpoint{5.497222in}{3.686575in}}%
\pgfusepath{clip}%
\pgfsetbuttcap%
\pgfsetmiterjoin%
\definecolor{currentfill}{rgb}{0.121569,0.466667,0.705882}%
\pgfsetfillcolor{currentfill}%
\pgfsetlinewidth{0.000000pt}%
\definecolor{currentstroke}{rgb}{0.000000,0.000000,0.000000}%
\pgfsetstrokecolor{currentstroke}%
\pgfsetstrokeopacity{0.000000}%
\pgfsetdash{}{0pt}%
\pgfpathmoveto{\pgfqpoint{0.946309in}{-5.443771in}}%
\pgfpathlineto{\pgfqpoint{1.405261in}{-5.443771in}}%
\pgfpathlineto{\pgfqpoint{1.405261in}{3.177898in}}%
\pgfpathlineto{\pgfqpoint{0.946309in}{3.177898in}}%
\pgfpathclose%
\pgfusepath{fill}%
\end{pgfscope}%
\begin{pgfscope}%
\pgfpathrectangle{\pgfqpoint{0.696435in}{0.700520in}}{\pgfqpoint{5.497222in}{3.686575in}}%
\pgfusepath{clip}%
\pgfsetbuttcap%
\pgfsetmiterjoin%
\definecolor{currentfill}{rgb}{0.121569,0.466667,0.705882}%
\pgfsetfillcolor{currentfill}%
\pgfsetlinewidth{0.000000pt}%
\definecolor{currentstroke}{rgb}{0.000000,0.000000,0.000000}%
\pgfsetstrokecolor{currentstroke}%
\pgfsetstrokeopacity{0.000000}%
\pgfsetdash{}{0pt}%
\pgfpathmoveto{\pgfqpoint{1.966202in}{-5.443771in}}%
\pgfpathlineto{\pgfqpoint{2.425154in}{-5.443771in}}%
\pgfpathlineto{\pgfqpoint{2.425154in}{3.421212in}}%
\pgfpathlineto{\pgfqpoint{1.966202in}{3.421212in}}%
\pgfpathclose%
\pgfusepath{fill}%
\end{pgfscope}%
\begin{pgfscope}%
\pgfpathrectangle{\pgfqpoint{0.696435in}{0.700520in}}{\pgfqpoint{5.497222in}{3.686575in}}%
\pgfusepath{clip}%
\pgfsetbuttcap%
\pgfsetmiterjoin%
\definecolor{currentfill}{rgb}{0.121569,0.466667,0.705882}%
\pgfsetfillcolor{currentfill}%
\pgfsetlinewidth{0.000000pt}%
\definecolor{currentstroke}{rgb}{0.000000,0.000000,0.000000}%
\pgfsetstrokecolor{currentstroke}%
\pgfsetstrokeopacity{0.000000}%
\pgfsetdash{}{0pt}%
\pgfpathmoveto{\pgfqpoint{2.986095in}{-5.443771in}}%
\pgfpathlineto{\pgfqpoint{3.445047in}{-5.443771in}}%
\pgfpathlineto{\pgfqpoint{3.445047in}{3.492486in}}%
\pgfpathlineto{\pgfqpoint{2.986095in}{3.492486in}}%
\pgfpathclose%
\pgfusepath{fill}%
\end{pgfscope}%
\begin{pgfscope}%
\pgfpathrectangle{\pgfqpoint{0.696435in}{0.700520in}}{\pgfqpoint{5.497222in}{3.686575in}}%
\pgfusepath{clip}%
\pgfsetbuttcap%
\pgfsetmiterjoin%
\definecolor{currentfill}{rgb}{0.121569,0.466667,0.705882}%
\pgfsetfillcolor{currentfill}%
\pgfsetlinewidth{0.000000pt}%
\definecolor{currentstroke}{rgb}{0.000000,0.000000,0.000000}%
\pgfsetstrokecolor{currentstroke}%
\pgfsetstrokeopacity{0.000000}%
\pgfsetdash{}{0pt}%
\pgfpathmoveto{\pgfqpoint{4.005988in}{-5.443771in}}%
\pgfpathlineto{\pgfqpoint{4.464939in}{-5.443771in}}%
\pgfpathlineto{\pgfqpoint{4.464939in}{3.308157in}}%
\pgfpathlineto{\pgfqpoint{4.005988in}{3.308157in}}%
\pgfpathclose%
\pgfusepath{fill}%
\end{pgfscope}%
\begin{pgfscope}%
\pgfpathrectangle{\pgfqpoint{0.696435in}{0.700520in}}{\pgfqpoint{5.497222in}{3.686575in}}%
\pgfusepath{clip}%
\pgfsetbuttcap%
\pgfsetmiterjoin%
\definecolor{currentfill}{rgb}{0.121569,0.466667,0.705882}%
\pgfsetfillcolor{currentfill}%
\pgfsetlinewidth{0.000000pt}%
\definecolor{currentstroke}{rgb}{0.000000,0.000000,0.000000}%
\pgfsetstrokecolor{currentstroke}%
\pgfsetstrokeopacity{0.000000}%
\pgfsetdash{}{0pt}%
\pgfpathmoveto{\pgfqpoint{5.025880in}{-5.443771in}}%
\pgfpathlineto{\pgfqpoint{5.484832in}{-5.443771in}}%
\pgfpathlineto{\pgfqpoint{5.484832in}{3.057470in}}%
\pgfpathlineto{\pgfqpoint{5.025880in}{3.057470in}}%
\pgfpathclose%
\pgfusepath{fill}%
\end{pgfscope}%
\begin{pgfscope}%
\pgfpathrectangle{\pgfqpoint{0.696435in}{0.700520in}}{\pgfqpoint{5.497222in}{3.686575in}}%
\pgfusepath{clip}%
\pgfsetbuttcap%
\pgfsetmiterjoin%
\definecolor{currentfill}{rgb}{1.000000,0.498039,0.054902}%
\pgfsetfillcolor{currentfill}%
\pgfsetlinewidth{0.000000pt}%
\definecolor{currentstroke}{rgb}{0.000000,0.000000,0.000000}%
\pgfsetstrokecolor{currentstroke}%
\pgfsetstrokeopacity{0.000000}%
\pgfsetdash{}{0pt}%
\pgfpathmoveto{\pgfqpoint{1.405261in}{-5.443771in}}%
\pgfpathlineto{\pgfqpoint{1.864213in}{-5.443771in}}%
\pgfpathlineto{\pgfqpoint{1.864213in}{2.258712in}}%
\pgfpathlineto{\pgfqpoint{1.405261in}{2.258712in}}%
\pgfpathclose%
\pgfusepath{fill}%
\end{pgfscope}%
\begin{pgfscope}%
\pgfpathrectangle{\pgfqpoint{0.696435in}{0.700520in}}{\pgfqpoint{5.497222in}{3.686575in}}%
\pgfusepath{clip}%
\pgfsetbuttcap%
\pgfsetmiterjoin%
\definecolor{currentfill}{rgb}{1.000000,0.498039,0.054902}%
\pgfsetfillcolor{currentfill}%
\pgfsetlinewidth{0.000000pt}%
\definecolor{currentstroke}{rgb}{0.000000,0.000000,0.000000}%
\pgfsetstrokecolor{currentstroke}%
\pgfsetstrokeopacity{0.000000}%
\pgfsetdash{}{0pt}%
\pgfpathmoveto{\pgfqpoint{2.425154in}{-5.443771in}}%
\pgfpathlineto{\pgfqpoint{2.884105in}{-5.443771in}}%
\pgfpathlineto{\pgfqpoint{2.884105in}{3.295869in}}%
\pgfpathlineto{\pgfqpoint{2.425154in}{3.295869in}}%
\pgfpathclose%
\pgfusepath{fill}%
\end{pgfscope}%
\begin{pgfscope}%
\pgfpathrectangle{\pgfqpoint{0.696435in}{0.700520in}}{\pgfqpoint{5.497222in}{3.686575in}}%
\pgfusepath{clip}%
\pgfsetbuttcap%
\pgfsetmiterjoin%
\definecolor{currentfill}{rgb}{1.000000,0.498039,0.054902}%
\pgfsetfillcolor{currentfill}%
\pgfsetlinewidth{0.000000pt}%
\definecolor{currentstroke}{rgb}{0.000000,0.000000,0.000000}%
\pgfsetstrokecolor{currentstroke}%
\pgfsetstrokeopacity{0.000000}%
\pgfsetdash{}{0pt}%
\pgfpathmoveto{\pgfqpoint{3.445047in}{-5.443771in}}%
\pgfpathlineto{\pgfqpoint{3.903998in}{-5.443771in}}%
\pgfpathlineto{\pgfqpoint{3.903998in}{3.372058in}}%
\pgfpathlineto{\pgfqpoint{3.445047in}{3.372058in}}%
\pgfpathclose%
\pgfusepath{fill}%
\end{pgfscope}%
\begin{pgfscope}%
\pgfpathrectangle{\pgfqpoint{0.696435in}{0.700520in}}{\pgfqpoint{5.497222in}{3.686575in}}%
\pgfusepath{clip}%
\pgfsetbuttcap%
\pgfsetmiterjoin%
\definecolor{currentfill}{rgb}{1.000000,0.498039,0.054902}%
\pgfsetfillcolor{currentfill}%
\pgfsetlinewidth{0.000000pt}%
\definecolor{currentstroke}{rgb}{0.000000,0.000000,0.000000}%
\pgfsetstrokecolor{currentstroke}%
\pgfsetstrokeopacity{0.000000}%
\pgfsetdash{}{0pt}%
\pgfpathmoveto{\pgfqpoint{4.464939in}{-5.443771in}}%
\pgfpathlineto{\pgfqpoint{4.923891in}{-5.443771in}}%
\pgfpathlineto{\pgfqpoint{4.923891in}{3.143490in}}%
\pgfpathlineto{\pgfqpoint{4.464939in}{3.143490in}}%
\pgfpathclose%
\pgfusepath{fill}%
\end{pgfscope}%
\begin{pgfscope}%
\pgfpathrectangle{\pgfqpoint{0.696435in}{0.700520in}}{\pgfqpoint{5.497222in}{3.686575in}}%
\pgfusepath{clip}%
\pgfsetbuttcap%
\pgfsetmiterjoin%
\definecolor{currentfill}{rgb}{1.000000,0.498039,0.054902}%
\pgfsetfillcolor{currentfill}%
\pgfsetlinewidth{0.000000pt}%
\definecolor{currentstroke}{rgb}{0.000000,0.000000,0.000000}%
\pgfsetstrokecolor{currentstroke}%
\pgfsetstrokeopacity{0.000000}%
\pgfsetdash{}{0pt}%
\pgfpathmoveto{\pgfqpoint{5.484832in}{-5.443771in}}%
\pgfpathlineto{\pgfqpoint{5.943784in}{-5.443771in}}%
\pgfpathlineto{\pgfqpoint{5.943784in}{3.113998in}}%
\pgfpathlineto{\pgfqpoint{5.484832in}{3.113998in}}%
\pgfpathclose%
\pgfusepath{fill}%
\end{pgfscope}%
\begin{pgfscope}%
\pgfsetbuttcap%
\pgfsetroundjoin%
\definecolor{currentfill}{rgb}{0.000000,0.000000,0.000000}%
\pgfsetfillcolor{currentfill}%
\pgfsetlinewidth{0.803000pt}%
\definecolor{currentstroke}{rgb}{0.000000,0.000000,0.000000}%
\pgfsetstrokecolor{currentstroke}%
\pgfsetdash{}{0pt}%
\pgfsys@defobject{currentmarker}{\pgfqpoint{0.000000in}{-0.048611in}}{\pgfqpoint{0.000000in}{0.000000in}}{%
\pgfpathmoveto{\pgfqpoint{0.000000in}{0.000000in}}%
\pgfpathlineto{\pgfqpoint{0.000000in}{-0.048611in}}%
\pgfusepath{stroke,fill}%
}%
\begin{pgfscope}%
\pgfsys@transformshift{1.405261in}{0.700520in}%
\pgfsys@useobject{currentmarker}{}%
\end{pgfscope}%
\end{pgfscope}%
\begin{pgfscope}%
\definecolor{textcolor}{rgb}{0.000000,0.000000,0.000000}%
\pgfsetstrokecolor{textcolor}%
\pgfsetfillcolor{textcolor}%
\pgftext[x=1.323212in, y=0.310396in, left, base,rotate=45.000000]{\color{textcolor}\sffamily\fontsize{10.000000}{12.000000}\selectfont 15\%}%
\end{pgfscope}%
\begin{pgfscope}%
\pgfsetbuttcap%
\pgfsetroundjoin%
\definecolor{currentfill}{rgb}{0.000000,0.000000,0.000000}%
\pgfsetfillcolor{currentfill}%
\pgfsetlinewidth{0.803000pt}%
\definecolor{currentstroke}{rgb}{0.000000,0.000000,0.000000}%
\pgfsetstrokecolor{currentstroke}%
\pgfsetdash{}{0pt}%
\pgfsys@defobject{currentmarker}{\pgfqpoint{0.000000in}{-0.048611in}}{\pgfqpoint{0.000000in}{0.000000in}}{%
\pgfpathmoveto{\pgfqpoint{0.000000in}{0.000000in}}%
\pgfpathlineto{\pgfqpoint{0.000000in}{-0.048611in}}%
\pgfusepath{stroke,fill}%
}%
\begin{pgfscope}%
\pgfsys@transformshift{2.425154in}{0.700520in}%
\pgfsys@useobject{currentmarker}{}%
\end{pgfscope}%
\end{pgfscope}%
\begin{pgfscope}%
\definecolor{textcolor}{rgb}{0.000000,0.000000,0.000000}%
\pgfsetstrokecolor{textcolor}%
\pgfsetfillcolor{textcolor}%
\pgftext[x=2.343105in, y=0.310396in, left, base,rotate=45.000000]{\color{textcolor}\sffamily\fontsize{10.000000}{12.000000}\selectfont 25\%}%
\end{pgfscope}%
\begin{pgfscope}%
\pgfsetbuttcap%
\pgfsetroundjoin%
\definecolor{currentfill}{rgb}{0.000000,0.000000,0.000000}%
\pgfsetfillcolor{currentfill}%
\pgfsetlinewidth{0.803000pt}%
\definecolor{currentstroke}{rgb}{0.000000,0.000000,0.000000}%
\pgfsetstrokecolor{currentstroke}%
\pgfsetdash{}{0pt}%
\pgfsys@defobject{currentmarker}{\pgfqpoint{0.000000in}{-0.048611in}}{\pgfqpoint{0.000000in}{0.000000in}}{%
\pgfpathmoveto{\pgfqpoint{0.000000in}{0.000000in}}%
\pgfpathlineto{\pgfqpoint{0.000000in}{-0.048611in}}%
\pgfusepath{stroke,fill}%
}%
\begin{pgfscope}%
\pgfsys@transformshift{3.445047in}{0.700520in}%
\pgfsys@useobject{currentmarker}{}%
\end{pgfscope}%
\end{pgfscope}%
\begin{pgfscope}%
\definecolor{textcolor}{rgb}{0.000000,0.000000,0.000000}%
\pgfsetstrokecolor{textcolor}%
\pgfsetfillcolor{textcolor}%
\pgftext[x=3.362998in, y=0.310396in, left, base,rotate=45.000000]{\color{textcolor}\sffamily\fontsize{10.000000}{12.000000}\selectfont 50\%}%
\end{pgfscope}%
\begin{pgfscope}%
\pgfsetbuttcap%
\pgfsetroundjoin%
\definecolor{currentfill}{rgb}{0.000000,0.000000,0.000000}%
\pgfsetfillcolor{currentfill}%
\pgfsetlinewidth{0.803000pt}%
\definecolor{currentstroke}{rgb}{0.000000,0.000000,0.000000}%
\pgfsetstrokecolor{currentstroke}%
\pgfsetdash{}{0pt}%
\pgfsys@defobject{currentmarker}{\pgfqpoint{0.000000in}{-0.048611in}}{\pgfqpoint{0.000000in}{0.000000in}}{%
\pgfpathmoveto{\pgfqpoint{0.000000in}{0.000000in}}%
\pgfpathlineto{\pgfqpoint{0.000000in}{-0.048611in}}%
\pgfusepath{stroke,fill}%
}%
\begin{pgfscope}%
\pgfsys@transformshift{4.464939in}{0.700520in}%
\pgfsys@useobject{currentmarker}{}%
\end{pgfscope}%
\end{pgfscope}%
\begin{pgfscope}%
\definecolor{textcolor}{rgb}{0.000000,0.000000,0.000000}%
\pgfsetstrokecolor{textcolor}%
\pgfsetfillcolor{textcolor}%
\pgftext[x=4.382891in, y=0.310396in, left, base,rotate=45.000000]{\color{textcolor}\sffamily\fontsize{10.000000}{12.000000}\selectfont 75\%}%
\end{pgfscope}%
\begin{pgfscope}%
\pgfsetbuttcap%
\pgfsetroundjoin%
\definecolor{currentfill}{rgb}{0.000000,0.000000,0.000000}%
\pgfsetfillcolor{currentfill}%
\pgfsetlinewidth{0.803000pt}%
\definecolor{currentstroke}{rgb}{0.000000,0.000000,0.000000}%
\pgfsetstrokecolor{currentstroke}%
\pgfsetdash{}{0pt}%
\pgfsys@defobject{currentmarker}{\pgfqpoint{0.000000in}{-0.048611in}}{\pgfqpoint{0.000000in}{0.000000in}}{%
\pgfpathmoveto{\pgfqpoint{0.000000in}{0.000000in}}%
\pgfpathlineto{\pgfqpoint{0.000000in}{-0.048611in}}%
\pgfusepath{stroke,fill}%
}%
\begin{pgfscope}%
\pgfsys@transformshift{5.484832in}{0.700520in}%
\pgfsys@useobject{currentmarker}{}%
\end{pgfscope}%
\end{pgfscope}%
\begin{pgfscope}%
\definecolor{textcolor}{rgb}{0.000000,0.000000,0.000000}%
\pgfsetstrokecolor{textcolor}%
\pgfsetfillcolor{textcolor}%
\pgftext[x=5.402783in, y=0.310396in, left, base,rotate=45.000000]{\color{textcolor}\sffamily\fontsize{10.000000}{12.000000}\selectfont 90\%}%
\end{pgfscope}%
\begin{pgfscope}%
\definecolor{textcolor}{rgb}{0.000000,0.000000,0.000000}%
\pgfsetstrokecolor{textcolor}%
\pgfsetfillcolor{textcolor}%
\pgftext[x=3.445047in,y=0.234413in,,top]{\color{textcolor}\sffamily\fontsize{10.000000}{12.000000}\selectfont Dataset}%
\end{pgfscope}%
\begin{pgfscope}%
\pgfsetbuttcap%
\pgfsetroundjoin%
\definecolor{currentfill}{rgb}{0.000000,0.000000,0.000000}%
\pgfsetfillcolor{currentfill}%
\pgfsetlinewidth{0.803000pt}%
\definecolor{currentstroke}{rgb}{0.000000,0.000000,0.000000}%
\pgfsetstrokecolor{currentstroke}%
\pgfsetdash{}{0pt}%
\pgfsys@defobject{currentmarker}{\pgfqpoint{-0.048611in}{0.000000in}}{\pgfqpoint{-0.000000in}{0.000000in}}{%
\pgfpathmoveto{\pgfqpoint{-0.000000in}{0.000000in}}%
\pgfpathlineto{\pgfqpoint{-0.048611in}{0.000000in}}%
\pgfusepath{stroke,fill}%
}%
\begin{pgfscope}%
\pgfsys@transformshift{0.696435in}{0.946292in}%
\pgfsys@useobject{currentmarker}{}%
\end{pgfscope}%
\end{pgfscope}%
\begin{pgfscope}%
\definecolor{textcolor}{rgb}{0.000000,0.000000,0.000000}%
\pgfsetstrokecolor{textcolor}%
\pgfsetfillcolor{textcolor}%
\pgftext[x=0.289968in, y=0.893530in, left, base]{\color{textcolor}\sffamily\fontsize{10.000000}{12.000000}\selectfont 0.26}%
\end{pgfscope}%
\begin{pgfscope}%
\pgfsetbuttcap%
\pgfsetroundjoin%
\definecolor{currentfill}{rgb}{0.000000,0.000000,0.000000}%
\pgfsetfillcolor{currentfill}%
\pgfsetlinewidth{0.803000pt}%
\definecolor{currentstroke}{rgb}{0.000000,0.000000,0.000000}%
\pgfsetstrokecolor{currentstroke}%
\pgfsetdash{}{0pt}%
\pgfsys@defobject{currentmarker}{\pgfqpoint{-0.048611in}{0.000000in}}{\pgfqpoint{-0.000000in}{0.000000in}}{%
\pgfpathmoveto{\pgfqpoint{-0.000000in}{0.000000in}}%
\pgfpathlineto{\pgfqpoint{-0.048611in}{0.000000in}}%
\pgfusepath{stroke,fill}%
}%
\begin{pgfscope}%
\pgfsys@transformshift{0.696435in}{1.437835in}%
\pgfsys@useobject{currentmarker}{}%
\end{pgfscope}%
\end{pgfscope}%
\begin{pgfscope}%
\definecolor{textcolor}{rgb}{0.000000,0.000000,0.000000}%
\pgfsetstrokecolor{textcolor}%
\pgfsetfillcolor{textcolor}%
\pgftext[x=0.289968in, y=1.385074in, left, base]{\color{textcolor}\sffamily\fontsize{10.000000}{12.000000}\selectfont 0.28}%
\end{pgfscope}%
\begin{pgfscope}%
\pgfsetbuttcap%
\pgfsetroundjoin%
\definecolor{currentfill}{rgb}{0.000000,0.000000,0.000000}%
\pgfsetfillcolor{currentfill}%
\pgfsetlinewidth{0.803000pt}%
\definecolor{currentstroke}{rgb}{0.000000,0.000000,0.000000}%
\pgfsetstrokecolor{currentstroke}%
\pgfsetdash{}{0pt}%
\pgfsys@defobject{currentmarker}{\pgfqpoint{-0.048611in}{0.000000in}}{\pgfqpoint{-0.000000in}{0.000000in}}{%
\pgfpathmoveto{\pgfqpoint{-0.000000in}{0.000000in}}%
\pgfpathlineto{\pgfqpoint{-0.048611in}{0.000000in}}%
\pgfusepath{stroke,fill}%
}%
\begin{pgfscope}%
\pgfsys@transformshift{0.696435in}{1.929378in}%
\pgfsys@useobject{currentmarker}{}%
\end{pgfscope}%
\end{pgfscope}%
\begin{pgfscope}%
\definecolor{textcolor}{rgb}{0.000000,0.000000,0.000000}%
\pgfsetstrokecolor{textcolor}%
\pgfsetfillcolor{textcolor}%
\pgftext[x=0.289968in, y=1.876617in, left, base]{\color{textcolor}\sffamily\fontsize{10.000000}{12.000000}\selectfont 0.30}%
\end{pgfscope}%
\begin{pgfscope}%
\pgfsetbuttcap%
\pgfsetroundjoin%
\definecolor{currentfill}{rgb}{0.000000,0.000000,0.000000}%
\pgfsetfillcolor{currentfill}%
\pgfsetlinewidth{0.803000pt}%
\definecolor{currentstroke}{rgb}{0.000000,0.000000,0.000000}%
\pgfsetstrokecolor{currentstroke}%
\pgfsetdash{}{0pt}%
\pgfsys@defobject{currentmarker}{\pgfqpoint{-0.048611in}{0.000000in}}{\pgfqpoint{-0.000000in}{0.000000in}}{%
\pgfpathmoveto{\pgfqpoint{-0.000000in}{0.000000in}}%
\pgfpathlineto{\pgfqpoint{-0.048611in}{0.000000in}}%
\pgfusepath{stroke,fill}%
}%
\begin{pgfscope}%
\pgfsys@transformshift{0.696435in}{2.420922in}%
\pgfsys@useobject{currentmarker}{}%
\end{pgfscope}%
\end{pgfscope}%
\begin{pgfscope}%
\definecolor{textcolor}{rgb}{0.000000,0.000000,0.000000}%
\pgfsetstrokecolor{textcolor}%
\pgfsetfillcolor{textcolor}%
\pgftext[x=0.289968in, y=2.368160in, left, base]{\color{textcolor}\sffamily\fontsize{10.000000}{12.000000}\selectfont 0.32}%
\end{pgfscope}%
\begin{pgfscope}%
\pgfsetbuttcap%
\pgfsetroundjoin%
\definecolor{currentfill}{rgb}{0.000000,0.000000,0.000000}%
\pgfsetfillcolor{currentfill}%
\pgfsetlinewidth{0.803000pt}%
\definecolor{currentstroke}{rgb}{0.000000,0.000000,0.000000}%
\pgfsetstrokecolor{currentstroke}%
\pgfsetdash{}{0pt}%
\pgfsys@defobject{currentmarker}{\pgfqpoint{-0.048611in}{0.000000in}}{\pgfqpoint{-0.000000in}{0.000000in}}{%
\pgfpathmoveto{\pgfqpoint{-0.000000in}{0.000000in}}%
\pgfpathlineto{\pgfqpoint{-0.048611in}{0.000000in}}%
\pgfusepath{stroke,fill}%
}%
\begin{pgfscope}%
\pgfsys@transformshift{0.696435in}{2.912465in}%
\pgfsys@useobject{currentmarker}{}%
\end{pgfscope}%
\end{pgfscope}%
\begin{pgfscope}%
\definecolor{textcolor}{rgb}{0.000000,0.000000,0.000000}%
\pgfsetstrokecolor{textcolor}%
\pgfsetfillcolor{textcolor}%
\pgftext[x=0.289968in, y=2.859703in, left, base]{\color{textcolor}\sffamily\fontsize{10.000000}{12.000000}\selectfont 0.34}%
\end{pgfscope}%
\begin{pgfscope}%
\pgfsetbuttcap%
\pgfsetroundjoin%
\definecolor{currentfill}{rgb}{0.000000,0.000000,0.000000}%
\pgfsetfillcolor{currentfill}%
\pgfsetlinewidth{0.803000pt}%
\definecolor{currentstroke}{rgb}{0.000000,0.000000,0.000000}%
\pgfsetstrokecolor{currentstroke}%
\pgfsetdash{}{0pt}%
\pgfsys@defobject{currentmarker}{\pgfqpoint{-0.048611in}{0.000000in}}{\pgfqpoint{-0.000000in}{0.000000in}}{%
\pgfpathmoveto{\pgfqpoint{-0.000000in}{0.000000in}}%
\pgfpathlineto{\pgfqpoint{-0.048611in}{0.000000in}}%
\pgfusepath{stroke,fill}%
}%
\begin{pgfscope}%
\pgfsys@transformshift{0.696435in}{3.404008in}%
\pgfsys@useobject{currentmarker}{}%
\end{pgfscope}%
\end{pgfscope}%
\begin{pgfscope}%
\definecolor{textcolor}{rgb}{0.000000,0.000000,0.000000}%
\pgfsetstrokecolor{textcolor}%
\pgfsetfillcolor{textcolor}%
\pgftext[x=0.289968in, y=3.351247in, left, base]{\color{textcolor}\sffamily\fontsize{10.000000}{12.000000}\selectfont 0.36}%
\end{pgfscope}%
\begin{pgfscope}%
\pgfsetbuttcap%
\pgfsetroundjoin%
\definecolor{currentfill}{rgb}{0.000000,0.000000,0.000000}%
\pgfsetfillcolor{currentfill}%
\pgfsetlinewidth{0.803000pt}%
\definecolor{currentstroke}{rgb}{0.000000,0.000000,0.000000}%
\pgfsetstrokecolor{currentstroke}%
\pgfsetdash{}{0pt}%
\pgfsys@defobject{currentmarker}{\pgfqpoint{-0.048611in}{0.000000in}}{\pgfqpoint{-0.000000in}{0.000000in}}{%
\pgfpathmoveto{\pgfqpoint{-0.000000in}{0.000000in}}%
\pgfpathlineto{\pgfqpoint{-0.048611in}{0.000000in}}%
\pgfusepath{stroke,fill}%
}%
\begin{pgfscope}%
\pgfsys@transformshift{0.696435in}{3.895552in}%
\pgfsys@useobject{currentmarker}{}%
\end{pgfscope}%
\end{pgfscope}%
\begin{pgfscope}%
\definecolor{textcolor}{rgb}{0.000000,0.000000,0.000000}%
\pgfsetstrokecolor{textcolor}%
\pgfsetfillcolor{textcolor}%
\pgftext[x=0.289968in, y=3.842790in, left, base]{\color{textcolor}\sffamily\fontsize{10.000000}{12.000000}\selectfont 0.38}%
\end{pgfscope}%
\begin{pgfscope}%
\pgfsetbuttcap%
\pgfsetroundjoin%
\definecolor{currentfill}{rgb}{0.000000,0.000000,0.000000}%
\pgfsetfillcolor{currentfill}%
\pgfsetlinewidth{0.803000pt}%
\definecolor{currentstroke}{rgb}{0.000000,0.000000,0.000000}%
\pgfsetstrokecolor{currentstroke}%
\pgfsetdash{}{0pt}%
\pgfsys@defobject{currentmarker}{\pgfqpoint{-0.048611in}{0.000000in}}{\pgfqpoint{-0.000000in}{0.000000in}}{%
\pgfpathmoveto{\pgfqpoint{-0.000000in}{0.000000in}}%
\pgfpathlineto{\pgfqpoint{-0.048611in}{0.000000in}}%
\pgfusepath{stroke,fill}%
}%
\begin{pgfscope}%
\pgfsys@transformshift{0.696435in}{4.387095in}%
\pgfsys@useobject{currentmarker}{}%
\end{pgfscope}%
\end{pgfscope}%
\begin{pgfscope}%
\definecolor{textcolor}{rgb}{0.000000,0.000000,0.000000}%
\pgfsetstrokecolor{textcolor}%
\pgfsetfillcolor{textcolor}%
\pgftext[x=0.289968in, y=4.334333in, left, base]{\color{textcolor}\sffamily\fontsize{10.000000}{12.000000}\selectfont 0.40}%
\end{pgfscope}%
\begin{pgfscope}%
\definecolor{textcolor}{rgb}{0.000000,0.000000,0.000000}%
\pgfsetstrokecolor{textcolor}%
\pgfsetfillcolor{textcolor}%
\pgftext[x=0.234413in,y=2.543808in,,bottom,rotate=90.000000]{\color{textcolor}\sffamily\fontsize{10.000000}{12.000000}\selectfont IoU}%
\end{pgfscope}%
\begin{pgfscope}%
\pgfsetrectcap%
\pgfsetmiterjoin%
\pgfsetlinewidth{0.803000pt}%
\definecolor{currentstroke}{rgb}{0.000000,0.000000,0.000000}%
\pgfsetstrokecolor{currentstroke}%
\pgfsetdash{}{0pt}%
\pgfpathmoveto{\pgfqpoint{0.696435in}{0.700520in}}%
\pgfpathlineto{\pgfqpoint{0.696435in}{4.387095in}}%
\pgfusepath{stroke}%
\end{pgfscope}%
\begin{pgfscope}%
\pgfsetrectcap%
\pgfsetmiterjoin%
\pgfsetlinewidth{0.803000pt}%
\definecolor{currentstroke}{rgb}{0.000000,0.000000,0.000000}%
\pgfsetstrokecolor{currentstroke}%
\pgfsetdash{}{0pt}%
\pgfpathmoveto{\pgfqpoint{6.193658in}{0.700520in}}%
\pgfpathlineto{\pgfqpoint{6.193658in}{4.387095in}}%
\pgfusepath{stroke}%
\end{pgfscope}%
\begin{pgfscope}%
\pgfsetrectcap%
\pgfsetmiterjoin%
\pgfsetlinewidth{0.803000pt}%
\definecolor{currentstroke}{rgb}{0.000000,0.000000,0.000000}%
\pgfsetstrokecolor{currentstroke}%
\pgfsetdash{}{0pt}%
\pgfpathmoveto{\pgfqpoint{0.696435in}{0.700520in}}%
\pgfpathlineto{\pgfqpoint{6.193658in}{0.700520in}}%
\pgfusepath{stroke}%
\end{pgfscope}%
\begin{pgfscope}%
\pgfsetrectcap%
\pgfsetmiterjoin%
\pgfsetlinewidth{0.803000pt}%
\definecolor{currentstroke}{rgb}{0.000000,0.000000,0.000000}%
\pgfsetstrokecolor{currentstroke}%
\pgfsetdash{}{0pt}%
\pgfpathmoveto{\pgfqpoint{0.696435in}{4.387095in}}%
\pgfpathlineto{\pgfqpoint{6.193658in}{4.387095in}}%
\pgfusepath{stroke}%
\end{pgfscope}%
\begin{pgfscope}%
\definecolor{textcolor}{rgb}{0.000000,0.000000,0.000000}%
\pgfsetstrokecolor{textcolor}%
\pgfsetfillcolor{textcolor}%
\pgftext[x=1.175785in,y=3.219565in,,bottom]{\color{textcolor}\sffamily\fontsize{9.000000}{10.800000}\selectfont 0.3508}%
\end{pgfscope}%
\begin{pgfscope}%
\definecolor{textcolor}{rgb}{0.000000,0.000000,0.000000}%
\pgfsetstrokecolor{textcolor}%
\pgfsetfillcolor{textcolor}%
\pgftext[x=2.195678in,y=3.462879in,,bottom]{\color{textcolor}\sffamily\fontsize{9.000000}{10.800000}\selectfont 0.3607}%
\end{pgfscope}%
\begin{pgfscope}%
\definecolor{textcolor}{rgb}{0.000000,0.000000,0.000000}%
\pgfsetstrokecolor{textcolor}%
\pgfsetfillcolor{textcolor}%
\pgftext[x=3.215571in,y=3.534153in,,bottom]{\color{textcolor}\sffamily\fontsize{9.000000}{10.800000}\selectfont 0.3636}%
\end{pgfscope}%
\begin{pgfscope}%
\definecolor{textcolor}{rgb}{0.000000,0.000000,0.000000}%
\pgfsetstrokecolor{textcolor}%
\pgfsetfillcolor{textcolor}%
\pgftext[x=4.235463in,y=3.349824in,,bottom]{\color{textcolor}\sffamily\fontsize{9.000000}{10.800000}\selectfont 0.3561}%
\end{pgfscope}%
\begin{pgfscope}%
\definecolor{textcolor}{rgb}{0.000000,0.000000,0.000000}%
\pgfsetstrokecolor{textcolor}%
\pgfsetfillcolor{textcolor}%
\pgftext[x=5.255356in,y=3.099137in,,bottom]{\color{textcolor}\sffamily\fontsize{9.000000}{10.800000}\selectfont 0.3459}%
\end{pgfscope}%
\begin{pgfscope}%
\definecolor{textcolor}{rgb}{0.000000,0.000000,0.000000}%
\pgfsetstrokecolor{textcolor}%
\pgfsetfillcolor{textcolor}%
\pgftext[x=1.634737in,y=2.300379in,,bottom]{\color{textcolor}\sffamily\fontsize{9.000000}{10.800000}\selectfont 0.3134}%
\end{pgfscope}%
\begin{pgfscope}%
\definecolor{textcolor}{rgb}{0.000000,0.000000,0.000000}%
\pgfsetstrokecolor{textcolor}%
\pgfsetfillcolor{textcolor}%
\pgftext[x=2.654630in,y=3.337535in,,bottom]{\color{textcolor}\sffamily\fontsize{9.000000}{10.800000}\selectfont 0.3556}%
\end{pgfscope}%
\begin{pgfscope}%
\definecolor{textcolor}{rgb}{0.000000,0.000000,0.000000}%
\pgfsetstrokecolor{textcolor}%
\pgfsetfillcolor{textcolor}%
\pgftext[x=3.674522in,y=3.413725in,,bottom]{\color{textcolor}\sffamily\fontsize{9.000000}{10.800000}\selectfont 0.3587}%
\end{pgfscope}%
\begin{pgfscope}%
\definecolor{textcolor}{rgb}{0.000000,0.000000,0.000000}%
\pgfsetstrokecolor{textcolor}%
\pgfsetfillcolor{textcolor}%
\pgftext[x=4.694415in,y=3.185157in,,bottom]{\color{textcolor}\sffamily\fontsize{9.000000}{10.800000}\selectfont 0.3494}%
\end{pgfscope}%
\begin{pgfscope}%
\definecolor{textcolor}{rgb}{0.000000,0.000000,0.000000}%
\pgfsetstrokecolor{textcolor}%
\pgfsetfillcolor{textcolor}%
\pgftext[x=5.714308in,y=3.155664in,,bottom]{\color{textcolor}\sffamily\fontsize{9.000000}{10.800000}\selectfont 0.3482}%
\end{pgfscope}%
\begin{pgfscope}%
\definecolor{textcolor}{rgb}{0.000000,0.000000,0.000000}%
\pgfsetstrokecolor{textcolor}%
\pgfsetfillcolor{textcolor}%
\pgftext[x=3.445047in,y=4.470428in,,base]{\color{textcolor}\sffamily\fontsize{12.000000}{14.400000}\selectfont Mixed training on Pix2Vox++ using 2 versions of S2R:3DFREE}%
\end{pgfscope}%
\begin{pgfscope}%
\pgfsetbuttcap%
\pgfsetmiterjoin%
\definecolor{currentfill}{rgb}{1.000000,1.000000,1.000000}%
\pgfsetfillcolor{currentfill}%
\pgfsetfillopacity{0.800000}%
\pgfsetlinewidth{1.003750pt}%
\definecolor{currentstroke}{rgb}{0.800000,0.800000,0.800000}%
\pgfsetstrokecolor{currentstroke}%
\pgfsetstrokeopacity{0.800000}%
\pgfsetdash{}{0pt}%
\pgfpathmoveto{\pgfqpoint{4.437595in}{3.868269in}}%
\pgfpathlineto{\pgfqpoint{6.096435in}{3.868269in}}%
\pgfpathquadraticcurveto{\pgfqpoint{6.124213in}{3.868269in}}{\pgfqpoint{6.124213in}{3.896047in}}%
\pgfpathlineto{\pgfqpoint{6.124213in}{4.289873in}}%
\pgfpathquadraticcurveto{\pgfqpoint{6.124213in}{4.317650in}}{\pgfqpoint{6.096435in}{4.317650in}}%
\pgfpathlineto{\pgfqpoint{4.437595in}{4.317650in}}%
\pgfpathquadraticcurveto{\pgfqpoint{4.409817in}{4.317650in}}{\pgfqpoint{4.409817in}{4.289873in}}%
\pgfpathlineto{\pgfqpoint{4.409817in}{3.896047in}}%
\pgfpathquadraticcurveto{\pgfqpoint{4.409817in}{3.868269in}}{\pgfqpoint{4.437595in}{3.868269in}}%
\pgfpathclose%
\pgfusepath{stroke,fill}%
\end{pgfscope}%
\begin{pgfscope}%
\pgfsetbuttcap%
\pgfsetmiterjoin%
\definecolor{currentfill}{rgb}{0.121569,0.466667,0.705882}%
\pgfsetfillcolor{currentfill}%
\pgfsetlinewidth{0.000000pt}%
\definecolor{currentstroke}{rgb}{0.000000,0.000000,0.000000}%
\pgfsetstrokecolor{currentstroke}%
\pgfsetstrokeopacity{0.000000}%
\pgfsetdash{}{0pt}%
\pgfpathmoveto{\pgfqpoint{4.465373in}{4.156572in}}%
\pgfpathlineto{\pgfqpoint{4.743150in}{4.156572in}}%
\pgfpathlineto{\pgfqpoint{4.743150in}{4.253794in}}%
\pgfpathlineto{\pgfqpoint{4.465373in}{4.253794in}}%
\pgfpathclose%
\pgfusepath{fill}%
\end{pgfscope}%
\begin{pgfscope}%
\definecolor{textcolor}{rgb}{0.000000,0.000000,0.000000}%
\pgfsetstrokecolor{textcolor}%
\pgfsetfillcolor{textcolor}%
\pgftext[x=4.854262in,y=4.156572in,left,base]{\color{textcolor}\sffamily\fontsize{10.000000}{12.000000}\selectfont V1 on Pix2Vox++}%
\end{pgfscope}%
\begin{pgfscope}%
\pgfsetbuttcap%
\pgfsetmiterjoin%
\definecolor{currentfill}{rgb}{1.000000,0.498039,0.054902}%
\pgfsetfillcolor{currentfill}%
\pgfsetlinewidth{0.000000pt}%
\definecolor{currentstroke}{rgb}{0.000000,0.000000,0.000000}%
\pgfsetstrokecolor{currentstroke}%
\pgfsetstrokeopacity{0.000000}%
\pgfsetdash{}{0pt}%
\pgfpathmoveto{\pgfqpoint{4.465373in}{3.952715in}}%
\pgfpathlineto{\pgfqpoint{4.743150in}{3.952715in}}%
\pgfpathlineto{\pgfqpoint{4.743150in}{4.049937in}}%
\pgfpathlineto{\pgfqpoint{4.465373in}{4.049937in}}%
\pgfpathclose%
\pgfusepath{fill}%
\end{pgfscope}%
\begin{pgfscope}%
\definecolor{textcolor}{rgb}{0.000000,0.000000,0.000000}%
\pgfsetstrokecolor{textcolor}%
\pgfsetfillcolor{textcolor}%
\pgftext[x=4.854262in,y=3.952715in,left,base]{\color{textcolor}\sffamily\fontsize{10.000000}{12.000000}\selectfont V2 on Pix2Vox++}%
\end{pgfscope}%
\end{pgfpicture}%
\makeatother%
\endgroup%
}}\\
    \subfloat[][]{\resizebox{0.75\linewidth}{!}{%% Creator: Matplotlib, PGF backend
%%
%% To include the figure in your LaTeX document, write
%%   \input{<filename>.pgf}
%%
%% Make sure the required packages are loaded in your preamble
%%   \usepackage{pgf}
%%
%% Figures using additional raster images can only be included by \input if
%% they are in the same directory as the main LaTeX file. For loading figures
%% from other directories you can use the `import` package
%%   \usepackage{import}
%%
%% and then include the figures with
%%   \import{<path to file>}{<filename>.pgf}
%%
%% Matplotlib used the following preamble
%%   \usepackage{fontspec}
%%   \setmainfont{DejaVuSerif.ttf}[Path=\detokenize{/Users/apple/opt/anaconda3/envs/kaolin/lib/python3.7/site-packages/matplotlib/mpl-data/fonts/ttf/}]
%%   \setsansfont{DejaVuSans.ttf}[Path=\detokenize{/Users/apple/opt/anaconda3/envs/kaolin/lib/python3.7/site-packages/matplotlib/mpl-data/fonts/ttf/}]
%%   \setmonofont{DejaVuSansMono.ttf}[Path=\detokenize{/Users/apple/opt/anaconda3/envs/kaolin/lib/python3.7/site-packages/matplotlib/mpl-data/fonts/ttf/}]
%%
\begingroup%
\makeatletter%
\begin{pgfpicture}%
\pgfpathrectangle{\pgfpointorigin}{\pgfqpoint{6.441712in}{4.704210in}}%
\pgfusepath{use as bounding box, clip}%
\begin{pgfscope}%
\pgfsetbuttcap%
\pgfsetmiterjoin%
\definecolor{currentfill}{rgb}{1.000000,1.000000,1.000000}%
\pgfsetfillcolor{currentfill}%
\pgfsetlinewidth{0.000000pt}%
\definecolor{currentstroke}{rgb}{1.000000,1.000000,1.000000}%
\pgfsetstrokecolor{currentstroke}%
\pgfsetdash{}{0pt}%
\pgfpathmoveto{\pgfqpoint{0.000000in}{0.000000in}}%
\pgfpathlineto{\pgfqpoint{6.441712in}{0.000000in}}%
\pgfpathlineto{\pgfqpoint{6.441712in}{4.704210in}}%
\pgfpathlineto{\pgfqpoint{0.000000in}{4.704210in}}%
\pgfpathclose%
\pgfusepath{fill}%
\end{pgfscope}%
\begin{pgfscope}%
\pgfsetbuttcap%
\pgfsetmiterjoin%
\definecolor{currentfill}{rgb}{1.000000,1.000000,1.000000}%
\pgfsetfillcolor{currentfill}%
\pgfsetlinewidth{0.000000pt}%
\definecolor{currentstroke}{rgb}{0.000000,0.000000,0.000000}%
\pgfsetstrokecolor{currentstroke}%
\pgfsetstrokeopacity{0.000000}%
\pgfsetdash{}{0pt}%
\pgfpathmoveto{\pgfqpoint{0.812050in}{0.816952in}}%
\pgfpathlineto{\pgfqpoint{6.198022in}{0.816952in}}%
\pgfpathlineto{\pgfqpoint{6.198022in}{4.373145in}}%
\pgfpathlineto{\pgfqpoint{0.812050in}{4.373145in}}%
\pgfpathclose%
\pgfusepath{fill}%
\end{pgfscope}%
\begin{pgfscope}%
\pgfpathrectangle{\pgfqpoint{0.812050in}{0.816952in}}{\pgfqpoint{5.385972in}{3.556193in}}%
\pgfusepath{clip}%
\pgfsetbuttcap%
\pgfsetmiterjoin%
\definecolor{currentfill}{rgb}{0.121569,0.466667,0.705882}%
\pgfsetfillcolor{currentfill}%
\pgfsetlinewidth{0.000000pt}%
\definecolor{currentstroke}{rgb}{0.000000,0.000000,0.000000}%
\pgfsetstrokecolor{currentstroke}%
\pgfsetstrokeopacity{0.000000}%
\pgfsetdash{}{0pt}%
\pgfpathmoveto{\pgfqpoint{1.056867in}{-5.110037in}}%
\pgfpathlineto{\pgfqpoint{1.506530in}{-5.110037in}}%
\pgfpathlineto{\pgfqpoint{1.506530in}{3.147444in}}%
\pgfpathlineto{\pgfqpoint{1.056867in}{3.147444in}}%
\pgfpathclose%
\pgfusepath{fill}%
\end{pgfscope}%
\begin{pgfscope}%
\pgfpathrectangle{\pgfqpoint{0.812050in}{0.816952in}}{\pgfqpoint{5.385972in}{3.556193in}}%
\pgfusepath{clip}%
\pgfsetbuttcap%
\pgfsetmiterjoin%
\definecolor{currentfill}{rgb}{0.121569,0.466667,0.705882}%
\pgfsetfillcolor{currentfill}%
\pgfsetlinewidth{0.000000pt}%
\definecolor{currentstroke}{rgb}{0.000000,0.000000,0.000000}%
\pgfsetstrokecolor{currentstroke}%
\pgfsetstrokeopacity{0.000000}%
\pgfsetdash{}{0pt}%
\pgfpathmoveto{\pgfqpoint{2.056119in}{-5.110037in}}%
\pgfpathlineto{\pgfqpoint{2.505783in}{-5.110037in}}%
\pgfpathlineto{\pgfqpoint{2.505783in}{3.026533in}}%
\pgfpathlineto{\pgfqpoint{2.056119in}{3.026533in}}%
\pgfpathclose%
\pgfusepath{fill}%
\end{pgfscope}%
\begin{pgfscope}%
\pgfpathrectangle{\pgfqpoint{0.812050in}{0.816952in}}{\pgfqpoint{5.385972in}{3.556193in}}%
\pgfusepath{clip}%
\pgfsetbuttcap%
\pgfsetmiterjoin%
\definecolor{currentfill}{rgb}{0.121569,0.466667,0.705882}%
\pgfsetfillcolor{currentfill}%
\pgfsetlinewidth{0.000000pt}%
\definecolor{currentstroke}{rgb}{0.000000,0.000000,0.000000}%
\pgfsetstrokecolor{currentstroke}%
\pgfsetstrokeopacity{0.000000}%
\pgfsetdash{}{0pt}%
\pgfpathmoveto{\pgfqpoint{3.055372in}{-5.110037in}}%
\pgfpathlineto{\pgfqpoint{3.505036in}{-5.110037in}}%
\pgfpathlineto{\pgfqpoint{3.505036in}{3.220938in}}%
\pgfpathlineto{\pgfqpoint{3.055372in}{3.220938in}}%
\pgfpathclose%
\pgfusepath{fill}%
\end{pgfscope}%
\begin{pgfscope}%
\pgfpathrectangle{\pgfqpoint{0.812050in}{0.816952in}}{\pgfqpoint{5.385972in}{3.556193in}}%
\pgfusepath{clip}%
\pgfsetbuttcap%
\pgfsetmiterjoin%
\definecolor{currentfill}{rgb}{0.121569,0.466667,0.705882}%
\pgfsetfillcolor{currentfill}%
\pgfsetlinewidth{0.000000pt}%
\definecolor{currentstroke}{rgb}{0.000000,0.000000,0.000000}%
\pgfsetstrokecolor{currentstroke}%
\pgfsetstrokeopacity{0.000000}%
\pgfsetdash{}{0pt}%
\pgfpathmoveto{\pgfqpoint{4.054625in}{-5.110037in}}%
\pgfpathlineto{\pgfqpoint{4.504288in}{-5.110037in}}%
\pgfpathlineto{\pgfqpoint{4.504288in}{3.021791in}}%
\pgfpathlineto{\pgfqpoint{4.054625in}{3.021791in}}%
\pgfpathclose%
\pgfusepath{fill}%
\end{pgfscope}%
\begin{pgfscope}%
\pgfpathrectangle{\pgfqpoint{0.812050in}{0.816952in}}{\pgfqpoint{5.385972in}{3.556193in}}%
\pgfusepath{clip}%
\pgfsetbuttcap%
\pgfsetmiterjoin%
\definecolor{currentfill}{rgb}{0.121569,0.466667,0.705882}%
\pgfsetfillcolor{currentfill}%
\pgfsetlinewidth{0.000000pt}%
\definecolor{currentstroke}{rgb}{0.000000,0.000000,0.000000}%
\pgfsetstrokecolor{currentstroke}%
\pgfsetstrokeopacity{0.000000}%
\pgfsetdash{}{0pt}%
\pgfpathmoveto{\pgfqpoint{5.053877in}{-5.110037in}}%
\pgfpathlineto{\pgfqpoint{5.503541in}{-5.110037in}}%
\pgfpathlineto{\pgfqpoint{5.503541in}{2.924589in}}%
\pgfpathlineto{\pgfqpoint{5.053877in}{2.924589in}}%
\pgfpathclose%
\pgfusepath{fill}%
\end{pgfscope}%
\begin{pgfscope}%
\pgfpathrectangle{\pgfqpoint{0.812050in}{0.816952in}}{\pgfqpoint{5.385972in}{3.556193in}}%
\pgfusepath{clip}%
\pgfsetbuttcap%
\pgfsetmiterjoin%
\definecolor{currentfill}{rgb}{1.000000,0.498039,0.054902}%
\pgfsetfillcolor{currentfill}%
\pgfsetlinewidth{0.000000pt}%
\definecolor{currentstroke}{rgb}{0.000000,0.000000,0.000000}%
\pgfsetstrokecolor{currentstroke}%
\pgfsetstrokeopacity{0.000000}%
\pgfsetdash{}{0pt}%
\pgfpathmoveto{\pgfqpoint{1.506530in}{-5.110037in}}%
\pgfpathlineto{\pgfqpoint{1.956194in}{-5.110037in}}%
\pgfpathlineto{\pgfqpoint{1.956194in}{2.066361in}}%
\pgfpathlineto{\pgfqpoint{1.506530in}{2.066361in}}%
\pgfpathclose%
\pgfusepath{fill}%
\end{pgfscope}%
\begin{pgfscope}%
\pgfpathrectangle{\pgfqpoint{0.812050in}{0.816952in}}{\pgfqpoint{5.385972in}{3.556193in}}%
\pgfusepath{clip}%
\pgfsetbuttcap%
\pgfsetmiterjoin%
\definecolor{currentfill}{rgb}{1.000000,0.498039,0.054902}%
\pgfsetfillcolor{currentfill}%
\pgfsetlinewidth{0.000000pt}%
\definecolor{currentstroke}{rgb}{0.000000,0.000000,0.000000}%
\pgfsetstrokecolor{currentstroke}%
\pgfsetstrokeopacity{0.000000}%
\pgfsetdash{}{0pt}%
\pgfpathmoveto{\pgfqpoint{2.505783in}{-5.110037in}}%
\pgfpathlineto{\pgfqpoint{2.955447in}{-5.110037in}}%
\pgfpathlineto{\pgfqpoint{2.955447in}{3.218567in}}%
\pgfpathlineto{\pgfqpoint{2.505783in}{3.218567in}}%
\pgfpathclose%
\pgfusepath{fill}%
\end{pgfscope}%
\begin{pgfscope}%
\pgfpathrectangle{\pgfqpoint{0.812050in}{0.816952in}}{\pgfqpoint{5.385972in}{3.556193in}}%
\pgfusepath{clip}%
\pgfsetbuttcap%
\pgfsetmiterjoin%
\definecolor{currentfill}{rgb}{1.000000,0.498039,0.054902}%
\pgfsetfillcolor{currentfill}%
\pgfsetlinewidth{0.000000pt}%
\definecolor{currentstroke}{rgb}{0.000000,0.000000,0.000000}%
\pgfsetstrokecolor{currentstroke}%
\pgfsetstrokeopacity{0.000000}%
\pgfsetdash{}{0pt}%
\pgfpathmoveto{\pgfqpoint{3.505036in}{-5.110037in}}%
\pgfpathlineto{\pgfqpoint{3.954699in}{-5.110037in}}%
\pgfpathlineto{\pgfqpoint{3.954699in}{2.988600in}}%
\pgfpathlineto{\pgfqpoint{3.505036in}{2.988600in}}%
\pgfpathclose%
\pgfusepath{fill}%
\end{pgfscope}%
\begin{pgfscope}%
\pgfpathrectangle{\pgfqpoint{0.812050in}{0.816952in}}{\pgfqpoint{5.385972in}{3.556193in}}%
\pgfusepath{clip}%
\pgfsetbuttcap%
\pgfsetmiterjoin%
\definecolor{currentfill}{rgb}{1.000000,0.498039,0.054902}%
\pgfsetfillcolor{currentfill}%
\pgfsetlinewidth{0.000000pt}%
\definecolor{currentstroke}{rgb}{0.000000,0.000000,0.000000}%
\pgfsetstrokecolor{currentstroke}%
\pgfsetstrokeopacity{0.000000}%
\pgfsetdash{}{0pt}%
\pgfpathmoveto{\pgfqpoint{4.504288in}{-5.110037in}}%
\pgfpathlineto{\pgfqpoint{4.953952in}{-5.110037in}}%
\pgfpathlineto{\pgfqpoint{4.953952in}{2.981488in}}%
\pgfpathlineto{\pgfqpoint{4.504288in}{2.981488in}}%
\pgfpathclose%
\pgfusepath{fill}%
\end{pgfscope}%
\begin{pgfscope}%
\pgfpathrectangle{\pgfqpoint{0.812050in}{0.816952in}}{\pgfqpoint{5.385972in}{3.556193in}}%
\pgfusepath{clip}%
\pgfsetbuttcap%
\pgfsetmiterjoin%
\definecolor{currentfill}{rgb}{1.000000,0.498039,0.054902}%
\pgfsetfillcolor{currentfill}%
\pgfsetlinewidth{0.000000pt}%
\definecolor{currentstroke}{rgb}{0.000000,0.000000,0.000000}%
\pgfsetstrokecolor{currentstroke}%
\pgfsetstrokeopacity{0.000000}%
\pgfsetdash{}{0pt}%
\pgfpathmoveto{\pgfqpoint{5.503541in}{-5.110037in}}%
\pgfpathlineto{\pgfqpoint{5.953205in}{-5.110037in}}%
\pgfpathlineto{\pgfqpoint{5.953205in}{3.031275in}}%
\pgfpathlineto{\pgfqpoint{5.503541in}{3.031275in}}%
\pgfpathclose%
\pgfusepath{fill}%
\end{pgfscope}%
\begin{pgfscope}%
\pgfsetbuttcap%
\pgfsetroundjoin%
\definecolor{currentfill}{rgb}{0.000000,0.000000,0.000000}%
\pgfsetfillcolor{currentfill}%
\pgfsetlinewidth{0.803000pt}%
\definecolor{currentstroke}{rgb}{0.000000,0.000000,0.000000}%
\pgfsetstrokecolor{currentstroke}%
\pgfsetdash{}{0pt}%
\pgfsys@defobject{currentmarker}{\pgfqpoint{0.000000in}{-0.048611in}}{\pgfqpoint{0.000000in}{0.000000in}}{%
\pgfpathmoveto{\pgfqpoint{0.000000in}{0.000000in}}%
\pgfpathlineto{\pgfqpoint{0.000000in}{-0.048611in}}%
\pgfusepath{stroke,fill}%
}%
\begin{pgfscope}%
\pgfsys@transformshift{1.506530in}{0.816952in}%
\pgfsys@useobject{currentmarker}{}%
\end{pgfscope}%
\end{pgfscope}%
\begin{pgfscope}%
\definecolor{textcolor}{rgb}{0.000000,0.000000,0.000000}%
\pgfsetstrokecolor{textcolor}%
\pgfsetfillcolor{textcolor}%
\pgftext[x=1.408072in, y=0.368247in, left, base,rotate=45.000000]{\color{textcolor}\sffamily\fontsize{12.000000}{14.400000}\selectfont 15\%}%
\end{pgfscope}%
\begin{pgfscope}%
\pgfsetbuttcap%
\pgfsetroundjoin%
\definecolor{currentfill}{rgb}{0.000000,0.000000,0.000000}%
\pgfsetfillcolor{currentfill}%
\pgfsetlinewidth{0.803000pt}%
\definecolor{currentstroke}{rgb}{0.000000,0.000000,0.000000}%
\pgfsetstrokecolor{currentstroke}%
\pgfsetdash{}{0pt}%
\pgfsys@defobject{currentmarker}{\pgfqpoint{0.000000in}{-0.048611in}}{\pgfqpoint{0.000000in}{0.000000in}}{%
\pgfpathmoveto{\pgfqpoint{0.000000in}{0.000000in}}%
\pgfpathlineto{\pgfqpoint{0.000000in}{-0.048611in}}%
\pgfusepath{stroke,fill}%
}%
\begin{pgfscope}%
\pgfsys@transformshift{2.505783in}{0.816952in}%
\pgfsys@useobject{currentmarker}{}%
\end{pgfscope}%
\end{pgfscope}%
\begin{pgfscope}%
\definecolor{textcolor}{rgb}{0.000000,0.000000,0.000000}%
\pgfsetstrokecolor{textcolor}%
\pgfsetfillcolor{textcolor}%
\pgftext[x=2.407324in, y=0.368247in, left, base,rotate=45.000000]{\color{textcolor}\sffamily\fontsize{12.000000}{14.400000}\selectfont 25\%}%
\end{pgfscope}%
\begin{pgfscope}%
\pgfsetbuttcap%
\pgfsetroundjoin%
\definecolor{currentfill}{rgb}{0.000000,0.000000,0.000000}%
\pgfsetfillcolor{currentfill}%
\pgfsetlinewidth{0.803000pt}%
\definecolor{currentstroke}{rgb}{0.000000,0.000000,0.000000}%
\pgfsetstrokecolor{currentstroke}%
\pgfsetdash{}{0pt}%
\pgfsys@defobject{currentmarker}{\pgfqpoint{0.000000in}{-0.048611in}}{\pgfqpoint{0.000000in}{0.000000in}}{%
\pgfpathmoveto{\pgfqpoint{0.000000in}{0.000000in}}%
\pgfpathlineto{\pgfqpoint{0.000000in}{-0.048611in}}%
\pgfusepath{stroke,fill}%
}%
\begin{pgfscope}%
\pgfsys@transformshift{3.505036in}{0.816952in}%
\pgfsys@useobject{currentmarker}{}%
\end{pgfscope}%
\end{pgfscope}%
\begin{pgfscope}%
\definecolor{textcolor}{rgb}{0.000000,0.000000,0.000000}%
\pgfsetstrokecolor{textcolor}%
\pgfsetfillcolor{textcolor}%
\pgftext[x=3.406577in, y=0.368247in, left, base,rotate=45.000000]{\color{textcolor}\sffamily\fontsize{12.000000}{14.400000}\selectfont 50\%}%
\end{pgfscope}%
\begin{pgfscope}%
\pgfsetbuttcap%
\pgfsetroundjoin%
\definecolor{currentfill}{rgb}{0.000000,0.000000,0.000000}%
\pgfsetfillcolor{currentfill}%
\pgfsetlinewidth{0.803000pt}%
\definecolor{currentstroke}{rgb}{0.000000,0.000000,0.000000}%
\pgfsetstrokecolor{currentstroke}%
\pgfsetdash{}{0pt}%
\pgfsys@defobject{currentmarker}{\pgfqpoint{0.000000in}{-0.048611in}}{\pgfqpoint{0.000000in}{0.000000in}}{%
\pgfpathmoveto{\pgfqpoint{0.000000in}{0.000000in}}%
\pgfpathlineto{\pgfqpoint{0.000000in}{-0.048611in}}%
\pgfusepath{stroke,fill}%
}%
\begin{pgfscope}%
\pgfsys@transformshift{4.504288in}{0.816952in}%
\pgfsys@useobject{currentmarker}{}%
\end{pgfscope}%
\end{pgfscope}%
\begin{pgfscope}%
\definecolor{textcolor}{rgb}{0.000000,0.000000,0.000000}%
\pgfsetstrokecolor{textcolor}%
\pgfsetfillcolor{textcolor}%
\pgftext[x=4.405830in, y=0.368247in, left, base,rotate=45.000000]{\color{textcolor}\sffamily\fontsize{12.000000}{14.400000}\selectfont 75\%}%
\end{pgfscope}%
\begin{pgfscope}%
\pgfsetbuttcap%
\pgfsetroundjoin%
\definecolor{currentfill}{rgb}{0.000000,0.000000,0.000000}%
\pgfsetfillcolor{currentfill}%
\pgfsetlinewidth{0.803000pt}%
\definecolor{currentstroke}{rgb}{0.000000,0.000000,0.000000}%
\pgfsetstrokecolor{currentstroke}%
\pgfsetdash{}{0pt}%
\pgfsys@defobject{currentmarker}{\pgfqpoint{0.000000in}{-0.048611in}}{\pgfqpoint{0.000000in}{0.000000in}}{%
\pgfpathmoveto{\pgfqpoint{0.000000in}{0.000000in}}%
\pgfpathlineto{\pgfqpoint{0.000000in}{-0.048611in}}%
\pgfusepath{stroke,fill}%
}%
\begin{pgfscope}%
\pgfsys@transformshift{5.503541in}{0.816952in}%
\pgfsys@useobject{currentmarker}{}%
\end{pgfscope}%
\end{pgfscope}%
\begin{pgfscope}%
\definecolor{textcolor}{rgb}{0.000000,0.000000,0.000000}%
\pgfsetstrokecolor{textcolor}%
\pgfsetfillcolor{textcolor}%
\pgftext[x=5.405083in, y=0.368247in, left, base,rotate=45.000000]{\color{textcolor}\sffamily\fontsize{12.000000}{14.400000}\selectfont 90\%}%
\end{pgfscope}%
\begin{pgfscope}%
\definecolor{textcolor}{rgb}{0.000000,0.000000,0.000000}%
\pgfsetstrokecolor{textcolor}%
\pgfsetfillcolor{textcolor}%
\pgftext[x=3.505036in,y=0.288178in,,top]{\color{textcolor}\sffamily\fontsize{14.000000}{16.800000}\bfseries\selectfont Dataset}%
\end{pgfscope}%
\begin{pgfscope}%
\pgfsetbuttcap%
\pgfsetroundjoin%
\definecolor{currentfill}{rgb}{0.000000,0.000000,0.000000}%
\pgfsetfillcolor{currentfill}%
\pgfsetlinewidth{0.803000pt}%
\definecolor{currentstroke}{rgb}{0.000000,0.000000,0.000000}%
\pgfsetstrokecolor{currentstroke}%
\pgfsetdash{}{0pt}%
\pgfsys@defobject{currentmarker}{\pgfqpoint{-0.048611in}{0.000000in}}{\pgfqpoint{-0.000000in}{0.000000in}}{%
\pgfpathmoveto{\pgfqpoint{-0.000000in}{0.000000in}}%
\pgfpathlineto{\pgfqpoint{-0.048611in}{0.000000in}}%
\pgfusepath{stroke,fill}%
}%
\begin{pgfscope}%
\pgfsys@transformshift{0.812050in}{1.054031in}%
\pgfsys@useobject{currentmarker}{}%
\end{pgfscope}%
\end{pgfscope}%
\begin{pgfscope}%
\definecolor{textcolor}{rgb}{0.000000,0.000000,0.000000}%
\pgfsetstrokecolor{textcolor}%
\pgfsetfillcolor{textcolor}%
\pgftext[x=0.343734in, y=0.990717in, left, base]{\color{textcolor}\sffamily\fontsize{12.000000}{14.400000}\selectfont 0.26}%
\end{pgfscope}%
\begin{pgfscope}%
\pgfsetbuttcap%
\pgfsetroundjoin%
\definecolor{currentfill}{rgb}{0.000000,0.000000,0.000000}%
\pgfsetfillcolor{currentfill}%
\pgfsetlinewidth{0.803000pt}%
\definecolor{currentstroke}{rgb}{0.000000,0.000000,0.000000}%
\pgfsetstrokecolor{currentstroke}%
\pgfsetdash{}{0pt}%
\pgfsys@defobject{currentmarker}{\pgfqpoint{-0.048611in}{0.000000in}}{\pgfqpoint{-0.000000in}{0.000000in}}{%
\pgfpathmoveto{\pgfqpoint{-0.000000in}{0.000000in}}%
\pgfpathlineto{\pgfqpoint{-0.048611in}{0.000000in}}%
\pgfusepath{stroke,fill}%
}%
\begin{pgfscope}%
\pgfsys@transformshift{0.812050in}{1.528190in}%
\pgfsys@useobject{currentmarker}{}%
\end{pgfscope}%
\end{pgfscope}%
\begin{pgfscope}%
\definecolor{textcolor}{rgb}{0.000000,0.000000,0.000000}%
\pgfsetstrokecolor{textcolor}%
\pgfsetfillcolor{textcolor}%
\pgftext[x=0.343734in, y=1.464876in, left, base]{\color{textcolor}\sffamily\fontsize{12.000000}{14.400000}\selectfont 0.28}%
\end{pgfscope}%
\begin{pgfscope}%
\pgfsetbuttcap%
\pgfsetroundjoin%
\definecolor{currentfill}{rgb}{0.000000,0.000000,0.000000}%
\pgfsetfillcolor{currentfill}%
\pgfsetlinewidth{0.803000pt}%
\definecolor{currentstroke}{rgb}{0.000000,0.000000,0.000000}%
\pgfsetstrokecolor{currentstroke}%
\pgfsetdash{}{0pt}%
\pgfsys@defobject{currentmarker}{\pgfqpoint{-0.048611in}{0.000000in}}{\pgfqpoint{-0.000000in}{0.000000in}}{%
\pgfpathmoveto{\pgfqpoint{-0.000000in}{0.000000in}}%
\pgfpathlineto{\pgfqpoint{-0.048611in}{0.000000in}}%
\pgfusepath{stroke,fill}%
}%
\begin{pgfscope}%
\pgfsys@transformshift{0.812050in}{2.002349in}%
\pgfsys@useobject{currentmarker}{}%
\end{pgfscope}%
\end{pgfscope}%
\begin{pgfscope}%
\definecolor{textcolor}{rgb}{0.000000,0.000000,0.000000}%
\pgfsetstrokecolor{textcolor}%
\pgfsetfillcolor{textcolor}%
\pgftext[x=0.343734in, y=1.939035in, left, base]{\color{textcolor}\sffamily\fontsize{12.000000}{14.400000}\selectfont 0.30}%
\end{pgfscope}%
\begin{pgfscope}%
\pgfsetbuttcap%
\pgfsetroundjoin%
\definecolor{currentfill}{rgb}{0.000000,0.000000,0.000000}%
\pgfsetfillcolor{currentfill}%
\pgfsetlinewidth{0.803000pt}%
\definecolor{currentstroke}{rgb}{0.000000,0.000000,0.000000}%
\pgfsetstrokecolor{currentstroke}%
\pgfsetdash{}{0pt}%
\pgfsys@defobject{currentmarker}{\pgfqpoint{-0.048611in}{0.000000in}}{\pgfqpoint{-0.000000in}{0.000000in}}{%
\pgfpathmoveto{\pgfqpoint{-0.000000in}{0.000000in}}%
\pgfpathlineto{\pgfqpoint{-0.048611in}{0.000000in}}%
\pgfusepath{stroke,fill}%
}%
\begin{pgfscope}%
\pgfsys@transformshift{0.812050in}{2.476508in}%
\pgfsys@useobject{currentmarker}{}%
\end{pgfscope}%
\end{pgfscope}%
\begin{pgfscope}%
\definecolor{textcolor}{rgb}{0.000000,0.000000,0.000000}%
\pgfsetstrokecolor{textcolor}%
\pgfsetfillcolor{textcolor}%
\pgftext[x=0.343734in, y=2.413195in, left, base]{\color{textcolor}\sffamily\fontsize{12.000000}{14.400000}\selectfont 0.32}%
\end{pgfscope}%
\begin{pgfscope}%
\pgfsetbuttcap%
\pgfsetroundjoin%
\definecolor{currentfill}{rgb}{0.000000,0.000000,0.000000}%
\pgfsetfillcolor{currentfill}%
\pgfsetlinewidth{0.803000pt}%
\definecolor{currentstroke}{rgb}{0.000000,0.000000,0.000000}%
\pgfsetstrokecolor{currentstroke}%
\pgfsetdash{}{0pt}%
\pgfsys@defobject{currentmarker}{\pgfqpoint{-0.048611in}{0.000000in}}{\pgfqpoint{-0.000000in}{0.000000in}}{%
\pgfpathmoveto{\pgfqpoint{-0.000000in}{0.000000in}}%
\pgfpathlineto{\pgfqpoint{-0.048611in}{0.000000in}}%
\pgfusepath{stroke,fill}%
}%
\begin{pgfscope}%
\pgfsys@transformshift{0.812050in}{2.950668in}%
\pgfsys@useobject{currentmarker}{}%
\end{pgfscope}%
\end{pgfscope}%
\begin{pgfscope}%
\definecolor{textcolor}{rgb}{0.000000,0.000000,0.000000}%
\pgfsetstrokecolor{textcolor}%
\pgfsetfillcolor{textcolor}%
\pgftext[x=0.343734in, y=2.887354in, left, base]{\color{textcolor}\sffamily\fontsize{12.000000}{14.400000}\selectfont 0.34}%
\end{pgfscope}%
\begin{pgfscope}%
\pgfsetbuttcap%
\pgfsetroundjoin%
\definecolor{currentfill}{rgb}{0.000000,0.000000,0.000000}%
\pgfsetfillcolor{currentfill}%
\pgfsetlinewidth{0.803000pt}%
\definecolor{currentstroke}{rgb}{0.000000,0.000000,0.000000}%
\pgfsetstrokecolor{currentstroke}%
\pgfsetdash{}{0pt}%
\pgfsys@defobject{currentmarker}{\pgfqpoint{-0.048611in}{0.000000in}}{\pgfqpoint{-0.000000in}{0.000000in}}{%
\pgfpathmoveto{\pgfqpoint{-0.000000in}{0.000000in}}%
\pgfpathlineto{\pgfqpoint{-0.048611in}{0.000000in}}%
\pgfusepath{stroke,fill}%
}%
\begin{pgfscope}%
\pgfsys@transformshift{0.812050in}{3.424827in}%
\pgfsys@useobject{currentmarker}{}%
\end{pgfscope}%
\end{pgfscope}%
\begin{pgfscope}%
\definecolor{textcolor}{rgb}{0.000000,0.000000,0.000000}%
\pgfsetstrokecolor{textcolor}%
\pgfsetfillcolor{textcolor}%
\pgftext[x=0.343734in, y=3.361513in, left, base]{\color{textcolor}\sffamily\fontsize{12.000000}{14.400000}\selectfont 0.36}%
\end{pgfscope}%
\begin{pgfscope}%
\pgfsetbuttcap%
\pgfsetroundjoin%
\definecolor{currentfill}{rgb}{0.000000,0.000000,0.000000}%
\pgfsetfillcolor{currentfill}%
\pgfsetlinewidth{0.803000pt}%
\definecolor{currentstroke}{rgb}{0.000000,0.000000,0.000000}%
\pgfsetstrokecolor{currentstroke}%
\pgfsetdash{}{0pt}%
\pgfsys@defobject{currentmarker}{\pgfqpoint{-0.048611in}{0.000000in}}{\pgfqpoint{-0.000000in}{0.000000in}}{%
\pgfpathmoveto{\pgfqpoint{-0.000000in}{0.000000in}}%
\pgfpathlineto{\pgfqpoint{-0.048611in}{0.000000in}}%
\pgfusepath{stroke,fill}%
}%
\begin{pgfscope}%
\pgfsys@transformshift{0.812050in}{3.898986in}%
\pgfsys@useobject{currentmarker}{}%
\end{pgfscope}%
\end{pgfscope}%
\begin{pgfscope}%
\definecolor{textcolor}{rgb}{0.000000,0.000000,0.000000}%
\pgfsetstrokecolor{textcolor}%
\pgfsetfillcolor{textcolor}%
\pgftext[x=0.343734in, y=3.835672in, left, base]{\color{textcolor}\sffamily\fontsize{12.000000}{14.400000}\selectfont 0.38}%
\end{pgfscope}%
\begin{pgfscope}%
\pgfsetbuttcap%
\pgfsetroundjoin%
\definecolor{currentfill}{rgb}{0.000000,0.000000,0.000000}%
\pgfsetfillcolor{currentfill}%
\pgfsetlinewidth{0.803000pt}%
\definecolor{currentstroke}{rgb}{0.000000,0.000000,0.000000}%
\pgfsetstrokecolor{currentstroke}%
\pgfsetdash{}{0pt}%
\pgfsys@defobject{currentmarker}{\pgfqpoint{-0.048611in}{0.000000in}}{\pgfqpoint{-0.000000in}{0.000000in}}{%
\pgfpathmoveto{\pgfqpoint{-0.000000in}{0.000000in}}%
\pgfpathlineto{\pgfqpoint{-0.048611in}{0.000000in}}%
\pgfusepath{stroke,fill}%
}%
\begin{pgfscope}%
\pgfsys@transformshift{0.812050in}{4.373145in}%
\pgfsys@useobject{currentmarker}{}%
\end{pgfscope}%
\end{pgfscope}%
\begin{pgfscope}%
\definecolor{textcolor}{rgb}{0.000000,0.000000,0.000000}%
\pgfsetstrokecolor{textcolor}%
\pgfsetfillcolor{textcolor}%
\pgftext[x=0.343734in, y=4.309831in, left, base]{\color{textcolor}\sffamily\fontsize{12.000000}{14.400000}\selectfont 0.40}%
\end{pgfscope}%
\begin{pgfscope}%
\definecolor{textcolor}{rgb}{0.000000,0.000000,0.000000}%
\pgfsetstrokecolor{textcolor}%
\pgfsetfillcolor{textcolor}%
\pgftext[x=0.288178in,y=2.595048in,,bottom,rotate=90.000000]{\color{textcolor}\sffamily\fontsize{14.000000}{16.800000}\bfseries\selectfont IoU}%
\end{pgfscope}%
\begin{pgfscope}%
\pgfsetrectcap%
\pgfsetmiterjoin%
\pgfsetlinewidth{0.803000pt}%
\definecolor{currentstroke}{rgb}{0.000000,0.000000,0.000000}%
\pgfsetstrokecolor{currentstroke}%
\pgfsetdash{}{0pt}%
\pgfpathmoveto{\pgfqpoint{0.812050in}{0.816952in}}%
\pgfpathlineto{\pgfqpoint{0.812050in}{4.373145in}}%
\pgfusepath{stroke}%
\end{pgfscope}%
\begin{pgfscope}%
\pgfsetrectcap%
\pgfsetmiterjoin%
\pgfsetlinewidth{0.803000pt}%
\definecolor{currentstroke}{rgb}{0.000000,0.000000,0.000000}%
\pgfsetstrokecolor{currentstroke}%
\pgfsetdash{}{0pt}%
\pgfpathmoveto{\pgfqpoint{6.198022in}{0.816952in}}%
\pgfpathlineto{\pgfqpoint{6.198022in}{4.373145in}}%
\pgfusepath{stroke}%
\end{pgfscope}%
\begin{pgfscope}%
\pgfsetrectcap%
\pgfsetmiterjoin%
\pgfsetlinewidth{0.803000pt}%
\definecolor{currentstroke}{rgb}{0.000000,0.000000,0.000000}%
\pgfsetstrokecolor{currentstroke}%
\pgfsetdash{}{0pt}%
\pgfpathmoveto{\pgfqpoint{0.812050in}{0.816952in}}%
\pgfpathlineto{\pgfqpoint{6.198022in}{0.816952in}}%
\pgfusepath{stroke}%
\end{pgfscope}%
\begin{pgfscope}%
\pgfsetrectcap%
\pgfsetmiterjoin%
\pgfsetlinewidth{0.803000pt}%
\definecolor{currentstroke}{rgb}{0.000000,0.000000,0.000000}%
\pgfsetstrokecolor{currentstroke}%
\pgfsetdash{}{0pt}%
\pgfpathmoveto{\pgfqpoint{0.812050in}{4.373145in}}%
\pgfpathlineto{\pgfqpoint{6.198022in}{4.373145in}}%
\pgfusepath{stroke}%
\end{pgfscope}%
\begin{pgfscope}%
\definecolor{textcolor}{rgb}{0.000000,0.000000,0.000000}%
\pgfsetstrokecolor{textcolor}%
\pgfsetfillcolor{textcolor}%
\pgftext[x=1.281698in,y=3.189110in,,bottom]{\color{textcolor}\sffamily\fontsize{9.000000}{10.800000}\selectfont 0.3483}%
\end{pgfscope}%
\begin{pgfscope}%
\definecolor{textcolor}{rgb}{0.000000,0.000000,0.000000}%
\pgfsetstrokecolor{textcolor}%
\pgfsetfillcolor{textcolor}%
\pgftext[x=2.280951in,y=3.068200in,,bottom]{\color{textcolor}\sffamily\fontsize{9.000000}{10.800000}\selectfont 0.3432}%
\end{pgfscope}%
\begin{pgfscope}%
\definecolor{textcolor}{rgb}{0.000000,0.000000,0.000000}%
\pgfsetstrokecolor{textcolor}%
\pgfsetfillcolor{textcolor}%
\pgftext[x=3.280204in,y=3.262605in,,bottom]{\color{textcolor}\sffamily\fontsize{9.000000}{10.800000}\selectfont 0.3514}%
\end{pgfscope}%
\begin{pgfscope}%
\definecolor{textcolor}{rgb}{0.000000,0.000000,0.000000}%
\pgfsetstrokecolor{textcolor}%
\pgfsetfillcolor{textcolor}%
\pgftext[x=4.279457in,y=3.063458in,,bottom]{\color{textcolor}\sffamily\fontsize{9.000000}{10.800000}\selectfont 0.343}%
\end{pgfscope}%
\begin{pgfscope}%
\definecolor{textcolor}{rgb}{0.000000,0.000000,0.000000}%
\pgfsetstrokecolor{textcolor}%
\pgfsetfillcolor{textcolor}%
\pgftext[x=5.278709in,y=2.966255in,,bottom]{\color{textcolor}\sffamily\fontsize{9.000000}{10.800000}\selectfont 0.3389}%
\end{pgfscope}%
\begin{pgfscope}%
\definecolor{textcolor}{rgb}{0.000000,0.000000,0.000000}%
\pgfsetstrokecolor{textcolor}%
\pgfsetfillcolor{textcolor}%
\pgftext[x=1.731362in,y=2.108027in,,bottom]{\color{textcolor}\sffamily\fontsize{9.000000}{10.800000}\selectfont 0.3027}%
\end{pgfscope}%
\begin{pgfscope}%
\definecolor{textcolor}{rgb}{0.000000,0.000000,0.000000}%
\pgfsetstrokecolor{textcolor}%
\pgfsetfillcolor{textcolor}%
\pgftext[x=2.730615in,y=3.260234in,,bottom]{\color{textcolor}\sffamily\fontsize{9.000000}{10.800000}\selectfont 0.3513}%
\end{pgfscope}%
\begin{pgfscope}%
\definecolor{textcolor}{rgb}{0.000000,0.000000,0.000000}%
\pgfsetstrokecolor{textcolor}%
\pgfsetfillcolor{textcolor}%
\pgftext[x=3.729868in,y=3.030267in,,bottom]{\color{textcolor}\sffamily\fontsize{9.000000}{10.800000}\selectfont 0.3416}%
\end{pgfscope}%
\begin{pgfscope}%
\definecolor{textcolor}{rgb}{0.000000,0.000000,0.000000}%
\pgfsetstrokecolor{textcolor}%
\pgfsetfillcolor{textcolor}%
\pgftext[x=4.729120in,y=3.023155in,,bottom]{\color{textcolor}\sffamily\fontsize{9.000000}{10.800000}\selectfont 0.3413}%
\end{pgfscope}%
\begin{pgfscope}%
\definecolor{textcolor}{rgb}{0.000000,0.000000,0.000000}%
\pgfsetstrokecolor{textcolor}%
\pgfsetfillcolor{textcolor}%
\pgftext[x=5.728373in,y=3.072941in,,bottom]{\color{textcolor}\sffamily\fontsize{9.000000}{10.800000}\selectfont 0.3434}%
\end{pgfscope}%
\begin{pgfscope}%
\definecolor{textcolor}{rgb}{0.000000,0.000000,0.000000}%
\pgfsetstrokecolor{textcolor}%
\pgfsetfillcolor{textcolor}%
\pgftext[x=3.505036in,y=4.456478in,,base]{\color{textcolor}\sffamily\fontsize{14.000000}{16.800000}\selectfont Mixed training on Pix2Vox using 2 versions of S2R:3DFREE}%
\end{pgfscope}%
\begin{pgfscope}%
\pgfsetbuttcap%
\pgfsetmiterjoin%
\definecolor{currentfill}{rgb}{1.000000,1.000000,1.000000}%
\pgfsetfillcolor{currentfill}%
\pgfsetfillopacity{0.800000}%
\pgfsetlinewidth{1.003750pt}%
\definecolor{currentstroke}{rgb}{0.800000,0.800000,0.800000}%
\pgfsetstrokecolor{currentstroke}%
\pgfsetstrokeopacity{0.800000}%
\pgfsetdash{}{0pt}%
\pgfpathmoveto{\pgfqpoint{4.674706in}{3.854319in}}%
\pgfpathlineto{\pgfqpoint{6.100800in}{3.854319in}}%
\pgfpathquadraticcurveto{\pgfqpoint{6.128577in}{3.854319in}}{\pgfqpoint{6.128577in}{3.882097in}}%
\pgfpathlineto{\pgfqpoint{6.128577in}{4.275923in}}%
\pgfpathquadraticcurveto{\pgfqpoint{6.128577in}{4.303700in}}{\pgfqpoint{6.100800in}{4.303700in}}%
\pgfpathlineto{\pgfqpoint{4.674706in}{4.303700in}}%
\pgfpathquadraticcurveto{\pgfqpoint{4.646929in}{4.303700in}}{\pgfqpoint{4.646929in}{4.275923in}}%
\pgfpathlineto{\pgfqpoint{4.646929in}{3.882097in}}%
\pgfpathquadraticcurveto{\pgfqpoint{4.646929in}{3.854319in}}{\pgfqpoint{4.674706in}{3.854319in}}%
\pgfpathclose%
\pgfusepath{stroke,fill}%
\end{pgfscope}%
\begin{pgfscope}%
\pgfsetbuttcap%
\pgfsetmiterjoin%
\definecolor{currentfill}{rgb}{0.121569,0.466667,0.705882}%
\pgfsetfillcolor{currentfill}%
\pgfsetlinewidth{0.000000pt}%
\definecolor{currentstroke}{rgb}{0.000000,0.000000,0.000000}%
\pgfsetstrokecolor{currentstroke}%
\pgfsetstrokeopacity{0.000000}%
\pgfsetdash{}{0pt}%
\pgfpathmoveto{\pgfqpoint{4.702484in}{4.142622in}}%
\pgfpathlineto{\pgfqpoint{4.980262in}{4.142622in}}%
\pgfpathlineto{\pgfqpoint{4.980262in}{4.239844in}}%
\pgfpathlineto{\pgfqpoint{4.702484in}{4.239844in}}%
\pgfpathclose%
\pgfusepath{fill}%
\end{pgfscope}%
\begin{pgfscope}%
\definecolor{textcolor}{rgb}{0.000000,0.000000,0.000000}%
\pgfsetstrokecolor{textcolor}%
\pgfsetfillcolor{textcolor}%
\pgftext[x=5.091373in,y=4.142622in,left,base]{\color{textcolor}\sffamily\fontsize{10.000000}{12.000000}\selectfont V1 on Pix2Vox}%
\end{pgfscope}%
\begin{pgfscope}%
\pgfsetbuttcap%
\pgfsetmiterjoin%
\definecolor{currentfill}{rgb}{1.000000,0.498039,0.054902}%
\pgfsetfillcolor{currentfill}%
\pgfsetlinewidth{0.000000pt}%
\definecolor{currentstroke}{rgb}{0.000000,0.000000,0.000000}%
\pgfsetstrokecolor{currentstroke}%
\pgfsetstrokeopacity{0.000000}%
\pgfsetdash{}{0pt}%
\pgfpathmoveto{\pgfqpoint{4.702484in}{3.938765in}}%
\pgfpathlineto{\pgfqpoint{4.980262in}{3.938765in}}%
\pgfpathlineto{\pgfqpoint{4.980262in}{4.035987in}}%
\pgfpathlineto{\pgfqpoint{4.702484in}{4.035987in}}%
\pgfpathclose%
\pgfusepath{fill}%
\end{pgfscope}%
\begin{pgfscope}%
\definecolor{textcolor}{rgb}{0.000000,0.000000,0.000000}%
\pgfsetstrokecolor{textcolor}%
\pgfsetfillcolor{textcolor}%
\pgftext[x=5.091373in,y=3.938765in,left,base]{\color{textcolor}\sffamily\fontsize{10.000000}{12.000000}\selectfont V2 on Pix2Vox}%
\end{pgfscope}%
\end{pgfpicture}%
\makeatother%
\endgroup%
}}\\
    \caption[\gls{iou} Comparison for Mix-Trained Baselines.]{Bar plot for the \gls{iou} for baselines trained on different ratios of synthetic and real dataset per mini-batch.(a)\textbf{Mixed training on Pix2Vox++}, (b)\textbf{Mixed training on Pix2Vox}.
    In both cases we see a slight increase in \gls{iou} with addition of real data, and a gradual decrease till it reaches 100\% real data}
    \label{fig:mixed1}
\end{figure}

For mixed training, we mix synthetic and real datasets with a fixed ratio in each mini-batch.
The ratios used were \emph{0.15, 0.25, 0.5, 0.75 and 0.9}.
The higher the ratio, the closer the mixed dataset becomes a real dataset.
Both the real and synthetic datasets were drawn according to these ratios for each mini-batch.
The synthetic dataset is much more abundant than the real, so the real dataset will be oversampled to achieve the mentioned ratios.


\begin{figure}[!ht]
    \centering
    \resizebox{0.65\textwidth}{!}{\input{/Users/apple/OVGU/Thesis/code/3dReconstruction/report/images/evaluation/performance/mixed_barplot_pix2voxpp.pgf}}
    \caption[\gls{iou} Comparison for Each Category from Mix-Trained Pix2Vox++.]{Bar plot for the \gls{iou} for baseline \texbf{Pix2Vox++} trained on 50\% of \textbf{mixed dataset}(\gls{s2rv1}, \gls{s2rv2}) and with Pix3D.
    The categories are listed along with the number of images.
    The performance of Pix2Vox++ mixed with both the synthetic dataset is more than model trained on only Pix3D, for majority of the categories.
    }
    \label{fig:mixed2}
\end{figure}

\begin{figure}[!ht]
    \centering
    \resizebox{0.65\textwidth}{!}{\input{/Users/apple/OVGU/Thesis/code/3dReconstruction/report/images/evaluation/performance/mixed_barplot_pix2vox.pgf}}
    \caption[\gls{iou} Comparison for Each Category from Mix-Trained Pix2Vox.]{Bar plot for the \gls{iou} for baseline \texbf{Pix2Vox} trained on 50\% of \textbf{mixed dataset}(\gls{s2rv1}, \gls{s2rv2}) and with Pix3D.
    The categories are listed along with the number of images.
    The performance of Pix2Vox mixed with both the synthetic dataset is more than model trained on only Pix3D, for all the categories.}
    \label{fig:mixed3}
\end{figure}



\begin{figure}[!ht]
    \begin{tabular}{llll}
        Pix3D images & \includegraphics[width=.2\linewidth]{/Users/apple/OVGU/Thesis/code/3dReconstruction/report/images/evaluation/reconstruction/baseline/bed1} &
        \includegraphics[width=.2\linewidth]{/Users/apple/OVGU/Thesis/code/3dReconstruction/report/images/evaluation/reconstruction/baseline/sofa1} &
        \includegraphics[width=.2\linewidth]{/Users/apple/OVGU/Thesis/code/3dReconstruction/report/images/evaluation/reconstruction/baseline/table2}\\

        Ground Truth & \includegraphics[trim={0 0 {.1\width} 0},clip,width=.2\linewidth]{/Users/apple/OVGU/Thesis/code/3dReconstruction/report/images/evaluation/reconstruction/baseline/bed1_original} &
        \includegraphics[trim={0 0 {.1\width} 0},clip,width=.2\linewidth]{/Users/apple/OVGU/Thesis/code/3dReconstruction/report/images/evaluation/reconstruction/baseline/sofa1_original} &
        \includegraphics[trim={0 0 {.1\width} 0},clip,width=.2\linewidth]{/Users/apple/OVGU/Thesis/code/3dReconstruction/report/images/evaluation/reconstruction/baseline/table2_original}\\

        Output1 & \includegraphics[width=.2\linewidth]{/Users/apple/OVGU/Thesis/code/3dReconstruction/report/images/evaluation/reconstruction/mixed/mixed1_p2vpp_bed1} &
        \includegraphics[width=.2\linewidth]{/Users/apple/OVGU/Thesis/code/3dReconstruction/report/images/evaluation/reconstruction/mixed/mixed1_p2vpp_sofa1} &
        \includegraphics[width=.2\linewidth]{/Users/apple/OVGU/Thesis/code/3dReconstruction/report/images/evaluation/reconstruction/mixed/mixed1_p2vpp_table2}\\

        Output2 & \includegraphics[width=.2\linewidth]{/Users/apple/OVGU/Thesis/code/3dReconstruction/report/images/evaluation/reconstruction/mixed/mixed1_p2v_bed1} &
        \includegraphics[width=.2\linewidth]{/Users/apple/OVGU/Thesis/code/3dReconstruction/report/images/evaluation/reconstruction/mixed/mixed1_p2v_sofa1} &
        \includegraphics[width=.2\linewidth]{/Users/apple/OVGU/Thesis/code/3dReconstruction/report/images/evaluation/reconstruction/mixed/mixed1_p2v_table2}\\

        Output3 & \includegraphics[width=.2\linewidth]{/Users/apple/OVGU/Thesis/code/3dReconstruction/report/images/evaluation/reconstruction/mixed/mixed2_p2vpp_bed1} &
        \includegraphics[width=.2\linewidth]{/Users/apple/OVGU/Thesis/code/3dReconstruction/report/images/evaluation/reconstruction/mixed/mixed2_p2vpp_sofa1} &
        \includegraphics[width=.2\linewidth]{/Users/apple/OVGU/Thesis/code/3dReconstruction/report/images/evaluation/reconstruction/mixed/mixed2_p2vpp_table2}\\

        Output4 & \includegraphics[width=.2\linewidth]{/Users/apple/OVGU/Thesis/code/3dReconstruction/report/images/evaluation/reconstruction/mixed/mixed2_p2v_bed1} &
        \includegraphics[width=.2\linewidth]{/Users/apple/OVGU/Thesis/code/3dReconstruction/report/images/evaluation/reconstruction/mixed/mixed2_p2v_sofa1} &
        \includegraphics[width=.2\linewidth]{/Users/apple/OVGU/Thesis/code/3dReconstruction/report/images/evaluation/reconstruction/mixed/mixed2_p2v_table2}\\

    \end{tabular}
    \caption[3D Reconstruction Output for Mix-Trained Baselines.]{3D reconstruction outputs for best \textbf{mixed training}(50\% per mini-batch) models. Output1-2: Pix2Vox++ and Pix2Vox mixed trained with \gls{s2rv1}.
    Output3-4:Pix2Vox++ and Pix2Vox mixed trained with \gls{s2rv2}}
    \label{fig:mixed_images1}
\end{figure}

In \autoref{fig:mixed1}, models trained with each of the ratios have better performance than the baseline of 0.3443 for Pix2Vox++,
except for a ratio of 15\%, which in some cases has a slightly lower \gls{iou}  value.
The most significant difference is seen at 50\% using the \gls{s2rv2} dataset with an increase of 3.04\%.
We saw an increase in \gls{iou} for Pix2Vox rather than Pix2Vox++.
An increase of 2.62\% is noted for the 25\% mix with \gls{s2rv2} on Pix2Vox.


%\begin{figure}[ht]
%    \centering
%    \resizebox{0.49\linewidth}{0.45\linewidth}{%% Creator: Matplotlib, PGF backend
%%
%% To include the figure in your LaTeX document, write
%%   \input{<filename>.pgf}
%%
%% Make sure the required packages are loaded in your preamble
%%   \usepackage{pgf}
%%
%% Figures using additional raster images can only be included by \input if
%% they are in the same directory as the main LaTeX file. For loading figures
%% from other directories you can use the `import` package
%%   \usepackage{import}
%%
%% and then include the figures with
%%   \import{<path to file>}{<filename>.pgf}
%%
%% Matplotlib used the following preamble
%%   \usepackage{fontspec}
%%   \setmainfont{DejaVuSerif.ttf}[Path=\detokenize{/Users/apple/opt/anaconda3/envs/kaolin/lib/python3.7/site-packages/matplotlib/mpl-data/fonts/ttf/}]
%%   \setsansfont{DejaVuSans.ttf}[Path=\detokenize{/Users/apple/opt/anaconda3/envs/kaolin/lib/python3.7/site-packages/matplotlib/mpl-data/fonts/ttf/}]
%%   \setmonofont{DejaVuSansMono.ttf}[Path=\detokenize{/Users/apple/opt/anaconda3/envs/kaolin/lib/python3.7/site-packages/matplotlib/mpl-data/fonts/ttf/}]
%%
\begingroup%
\makeatletter%
\begin{pgfpicture}%
\pgfpathrectangle{\pgfpointorigin}{\pgfqpoint{6.293658in}{4.697056in}}%
\pgfusepath{use as bounding box, clip}%
\begin{pgfscope}%
\pgfsetbuttcap%
\pgfsetmiterjoin%
\definecolor{currentfill}{rgb}{1.000000,1.000000,1.000000}%
\pgfsetfillcolor{currentfill}%
\pgfsetlinewidth{0.000000pt}%
\definecolor{currentstroke}{rgb}{1.000000,1.000000,1.000000}%
\pgfsetstrokecolor{currentstroke}%
\pgfsetdash{}{0pt}%
\pgfpathmoveto{\pgfqpoint{-0.000000in}{0.000000in}}%
\pgfpathlineto{\pgfqpoint{6.293658in}{0.000000in}}%
\pgfpathlineto{\pgfqpoint{6.293658in}{4.697056in}}%
\pgfpathlineto{\pgfqpoint{-0.000000in}{4.697056in}}%
\pgfpathclose%
\pgfusepath{fill}%
\end{pgfscope}%
\begin{pgfscope}%
\pgfsetbuttcap%
\pgfsetmiterjoin%
\definecolor{currentfill}{rgb}{1.000000,1.000000,1.000000}%
\pgfsetfillcolor{currentfill}%
\pgfsetlinewidth{0.000000pt}%
\definecolor{currentstroke}{rgb}{0.000000,0.000000,0.000000}%
\pgfsetstrokecolor{currentstroke}%
\pgfsetstrokeopacity{0.000000}%
\pgfsetdash{}{0pt}%
\pgfpathmoveto{\pgfqpoint{0.696435in}{0.700520in}}%
\pgfpathlineto{\pgfqpoint{6.193658in}{0.700520in}}%
\pgfpathlineto{\pgfqpoint{6.193658in}{4.387095in}}%
\pgfpathlineto{\pgfqpoint{0.696435in}{4.387095in}}%
\pgfpathclose%
\pgfusepath{fill}%
\end{pgfscope}%
\begin{pgfscope}%
\pgfpathrectangle{\pgfqpoint{0.696435in}{0.700520in}}{\pgfqpoint{5.497222in}{3.686575in}}%
\pgfusepath{clip}%
\pgfsetbuttcap%
\pgfsetmiterjoin%
\definecolor{currentfill}{rgb}{0.121569,0.466667,0.705882}%
\pgfsetfillcolor{currentfill}%
\pgfsetlinewidth{0.000000pt}%
\definecolor{currentstroke}{rgb}{0.000000,0.000000,0.000000}%
\pgfsetstrokecolor{currentstroke}%
\pgfsetstrokeopacity{0.000000}%
\pgfsetdash{}{0pt}%
\pgfpathmoveto{\pgfqpoint{0.946309in}{-5.443771in}}%
\pgfpathlineto{\pgfqpoint{1.405261in}{-5.443771in}}%
\pgfpathlineto{\pgfqpoint{1.405261in}{3.177898in}}%
\pgfpathlineto{\pgfqpoint{0.946309in}{3.177898in}}%
\pgfpathclose%
\pgfusepath{fill}%
\end{pgfscope}%
\begin{pgfscope}%
\pgfpathrectangle{\pgfqpoint{0.696435in}{0.700520in}}{\pgfqpoint{5.497222in}{3.686575in}}%
\pgfusepath{clip}%
\pgfsetbuttcap%
\pgfsetmiterjoin%
\definecolor{currentfill}{rgb}{0.121569,0.466667,0.705882}%
\pgfsetfillcolor{currentfill}%
\pgfsetlinewidth{0.000000pt}%
\definecolor{currentstroke}{rgb}{0.000000,0.000000,0.000000}%
\pgfsetstrokecolor{currentstroke}%
\pgfsetstrokeopacity{0.000000}%
\pgfsetdash{}{0pt}%
\pgfpathmoveto{\pgfqpoint{1.966202in}{-5.443771in}}%
\pgfpathlineto{\pgfqpoint{2.425154in}{-5.443771in}}%
\pgfpathlineto{\pgfqpoint{2.425154in}{3.421212in}}%
\pgfpathlineto{\pgfqpoint{1.966202in}{3.421212in}}%
\pgfpathclose%
\pgfusepath{fill}%
\end{pgfscope}%
\begin{pgfscope}%
\pgfpathrectangle{\pgfqpoint{0.696435in}{0.700520in}}{\pgfqpoint{5.497222in}{3.686575in}}%
\pgfusepath{clip}%
\pgfsetbuttcap%
\pgfsetmiterjoin%
\definecolor{currentfill}{rgb}{0.121569,0.466667,0.705882}%
\pgfsetfillcolor{currentfill}%
\pgfsetlinewidth{0.000000pt}%
\definecolor{currentstroke}{rgb}{0.000000,0.000000,0.000000}%
\pgfsetstrokecolor{currentstroke}%
\pgfsetstrokeopacity{0.000000}%
\pgfsetdash{}{0pt}%
\pgfpathmoveto{\pgfqpoint{2.986095in}{-5.443771in}}%
\pgfpathlineto{\pgfqpoint{3.445047in}{-5.443771in}}%
\pgfpathlineto{\pgfqpoint{3.445047in}{3.492486in}}%
\pgfpathlineto{\pgfqpoint{2.986095in}{3.492486in}}%
\pgfpathclose%
\pgfusepath{fill}%
\end{pgfscope}%
\begin{pgfscope}%
\pgfpathrectangle{\pgfqpoint{0.696435in}{0.700520in}}{\pgfqpoint{5.497222in}{3.686575in}}%
\pgfusepath{clip}%
\pgfsetbuttcap%
\pgfsetmiterjoin%
\definecolor{currentfill}{rgb}{0.121569,0.466667,0.705882}%
\pgfsetfillcolor{currentfill}%
\pgfsetlinewidth{0.000000pt}%
\definecolor{currentstroke}{rgb}{0.000000,0.000000,0.000000}%
\pgfsetstrokecolor{currentstroke}%
\pgfsetstrokeopacity{0.000000}%
\pgfsetdash{}{0pt}%
\pgfpathmoveto{\pgfqpoint{4.005988in}{-5.443771in}}%
\pgfpathlineto{\pgfqpoint{4.464939in}{-5.443771in}}%
\pgfpathlineto{\pgfqpoint{4.464939in}{3.308157in}}%
\pgfpathlineto{\pgfqpoint{4.005988in}{3.308157in}}%
\pgfpathclose%
\pgfusepath{fill}%
\end{pgfscope}%
\begin{pgfscope}%
\pgfpathrectangle{\pgfqpoint{0.696435in}{0.700520in}}{\pgfqpoint{5.497222in}{3.686575in}}%
\pgfusepath{clip}%
\pgfsetbuttcap%
\pgfsetmiterjoin%
\definecolor{currentfill}{rgb}{0.121569,0.466667,0.705882}%
\pgfsetfillcolor{currentfill}%
\pgfsetlinewidth{0.000000pt}%
\definecolor{currentstroke}{rgb}{0.000000,0.000000,0.000000}%
\pgfsetstrokecolor{currentstroke}%
\pgfsetstrokeopacity{0.000000}%
\pgfsetdash{}{0pt}%
\pgfpathmoveto{\pgfqpoint{5.025880in}{-5.443771in}}%
\pgfpathlineto{\pgfqpoint{5.484832in}{-5.443771in}}%
\pgfpathlineto{\pgfqpoint{5.484832in}{3.057470in}}%
\pgfpathlineto{\pgfqpoint{5.025880in}{3.057470in}}%
\pgfpathclose%
\pgfusepath{fill}%
\end{pgfscope}%
\begin{pgfscope}%
\pgfpathrectangle{\pgfqpoint{0.696435in}{0.700520in}}{\pgfqpoint{5.497222in}{3.686575in}}%
\pgfusepath{clip}%
\pgfsetbuttcap%
\pgfsetmiterjoin%
\definecolor{currentfill}{rgb}{1.000000,0.498039,0.054902}%
\pgfsetfillcolor{currentfill}%
\pgfsetlinewidth{0.000000pt}%
\definecolor{currentstroke}{rgb}{0.000000,0.000000,0.000000}%
\pgfsetstrokecolor{currentstroke}%
\pgfsetstrokeopacity{0.000000}%
\pgfsetdash{}{0pt}%
\pgfpathmoveto{\pgfqpoint{1.405261in}{-5.443771in}}%
\pgfpathlineto{\pgfqpoint{1.864213in}{-5.443771in}}%
\pgfpathlineto{\pgfqpoint{1.864213in}{2.258712in}}%
\pgfpathlineto{\pgfqpoint{1.405261in}{2.258712in}}%
\pgfpathclose%
\pgfusepath{fill}%
\end{pgfscope}%
\begin{pgfscope}%
\pgfpathrectangle{\pgfqpoint{0.696435in}{0.700520in}}{\pgfqpoint{5.497222in}{3.686575in}}%
\pgfusepath{clip}%
\pgfsetbuttcap%
\pgfsetmiterjoin%
\definecolor{currentfill}{rgb}{1.000000,0.498039,0.054902}%
\pgfsetfillcolor{currentfill}%
\pgfsetlinewidth{0.000000pt}%
\definecolor{currentstroke}{rgb}{0.000000,0.000000,0.000000}%
\pgfsetstrokecolor{currentstroke}%
\pgfsetstrokeopacity{0.000000}%
\pgfsetdash{}{0pt}%
\pgfpathmoveto{\pgfqpoint{2.425154in}{-5.443771in}}%
\pgfpathlineto{\pgfqpoint{2.884105in}{-5.443771in}}%
\pgfpathlineto{\pgfqpoint{2.884105in}{3.295869in}}%
\pgfpathlineto{\pgfqpoint{2.425154in}{3.295869in}}%
\pgfpathclose%
\pgfusepath{fill}%
\end{pgfscope}%
\begin{pgfscope}%
\pgfpathrectangle{\pgfqpoint{0.696435in}{0.700520in}}{\pgfqpoint{5.497222in}{3.686575in}}%
\pgfusepath{clip}%
\pgfsetbuttcap%
\pgfsetmiterjoin%
\definecolor{currentfill}{rgb}{1.000000,0.498039,0.054902}%
\pgfsetfillcolor{currentfill}%
\pgfsetlinewidth{0.000000pt}%
\definecolor{currentstroke}{rgb}{0.000000,0.000000,0.000000}%
\pgfsetstrokecolor{currentstroke}%
\pgfsetstrokeopacity{0.000000}%
\pgfsetdash{}{0pt}%
\pgfpathmoveto{\pgfqpoint{3.445047in}{-5.443771in}}%
\pgfpathlineto{\pgfqpoint{3.903998in}{-5.443771in}}%
\pgfpathlineto{\pgfqpoint{3.903998in}{3.372058in}}%
\pgfpathlineto{\pgfqpoint{3.445047in}{3.372058in}}%
\pgfpathclose%
\pgfusepath{fill}%
\end{pgfscope}%
\begin{pgfscope}%
\pgfpathrectangle{\pgfqpoint{0.696435in}{0.700520in}}{\pgfqpoint{5.497222in}{3.686575in}}%
\pgfusepath{clip}%
\pgfsetbuttcap%
\pgfsetmiterjoin%
\definecolor{currentfill}{rgb}{1.000000,0.498039,0.054902}%
\pgfsetfillcolor{currentfill}%
\pgfsetlinewidth{0.000000pt}%
\definecolor{currentstroke}{rgb}{0.000000,0.000000,0.000000}%
\pgfsetstrokecolor{currentstroke}%
\pgfsetstrokeopacity{0.000000}%
\pgfsetdash{}{0pt}%
\pgfpathmoveto{\pgfqpoint{4.464939in}{-5.443771in}}%
\pgfpathlineto{\pgfqpoint{4.923891in}{-5.443771in}}%
\pgfpathlineto{\pgfqpoint{4.923891in}{3.143490in}}%
\pgfpathlineto{\pgfqpoint{4.464939in}{3.143490in}}%
\pgfpathclose%
\pgfusepath{fill}%
\end{pgfscope}%
\begin{pgfscope}%
\pgfpathrectangle{\pgfqpoint{0.696435in}{0.700520in}}{\pgfqpoint{5.497222in}{3.686575in}}%
\pgfusepath{clip}%
\pgfsetbuttcap%
\pgfsetmiterjoin%
\definecolor{currentfill}{rgb}{1.000000,0.498039,0.054902}%
\pgfsetfillcolor{currentfill}%
\pgfsetlinewidth{0.000000pt}%
\definecolor{currentstroke}{rgb}{0.000000,0.000000,0.000000}%
\pgfsetstrokecolor{currentstroke}%
\pgfsetstrokeopacity{0.000000}%
\pgfsetdash{}{0pt}%
\pgfpathmoveto{\pgfqpoint{5.484832in}{-5.443771in}}%
\pgfpathlineto{\pgfqpoint{5.943784in}{-5.443771in}}%
\pgfpathlineto{\pgfqpoint{5.943784in}{3.113998in}}%
\pgfpathlineto{\pgfqpoint{5.484832in}{3.113998in}}%
\pgfpathclose%
\pgfusepath{fill}%
\end{pgfscope}%
\begin{pgfscope}%
\pgfsetbuttcap%
\pgfsetroundjoin%
\definecolor{currentfill}{rgb}{0.000000,0.000000,0.000000}%
\pgfsetfillcolor{currentfill}%
\pgfsetlinewidth{0.803000pt}%
\definecolor{currentstroke}{rgb}{0.000000,0.000000,0.000000}%
\pgfsetstrokecolor{currentstroke}%
\pgfsetdash{}{0pt}%
\pgfsys@defobject{currentmarker}{\pgfqpoint{0.000000in}{-0.048611in}}{\pgfqpoint{0.000000in}{0.000000in}}{%
\pgfpathmoveto{\pgfqpoint{0.000000in}{0.000000in}}%
\pgfpathlineto{\pgfqpoint{0.000000in}{-0.048611in}}%
\pgfusepath{stroke,fill}%
}%
\begin{pgfscope}%
\pgfsys@transformshift{1.405261in}{0.700520in}%
\pgfsys@useobject{currentmarker}{}%
\end{pgfscope}%
\end{pgfscope}%
\begin{pgfscope}%
\definecolor{textcolor}{rgb}{0.000000,0.000000,0.000000}%
\pgfsetstrokecolor{textcolor}%
\pgfsetfillcolor{textcolor}%
\pgftext[x=1.323212in, y=0.310396in, left, base,rotate=45.000000]{\color{textcolor}\sffamily\fontsize{10.000000}{12.000000}\selectfont 15\%}%
\end{pgfscope}%
\begin{pgfscope}%
\pgfsetbuttcap%
\pgfsetroundjoin%
\definecolor{currentfill}{rgb}{0.000000,0.000000,0.000000}%
\pgfsetfillcolor{currentfill}%
\pgfsetlinewidth{0.803000pt}%
\definecolor{currentstroke}{rgb}{0.000000,0.000000,0.000000}%
\pgfsetstrokecolor{currentstroke}%
\pgfsetdash{}{0pt}%
\pgfsys@defobject{currentmarker}{\pgfqpoint{0.000000in}{-0.048611in}}{\pgfqpoint{0.000000in}{0.000000in}}{%
\pgfpathmoveto{\pgfqpoint{0.000000in}{0.000000in}}%
\pgfpathlineto{\pgfqpoint{0.000000in}{-0.048611in}}%
\pgfusepath{stroke,fill}%
}%
\begin{pgfscope}%
\pgfsys@transformshift{2.425154in}{0.700520in}%
\pgfsys@useobject{currentmarker}{}%
\end{pgfscope}%
\end{pgfscope}%
\begin{pgfscope}%
\definecolor{textcolor}{rgb}{0.000000,0.000000,0.000000}%
\pgfsetstrokecolor{textcolor}%
\pgfsetfillcolor{textcolor}%
\pgftext[x=2.343105in, y=0.310396in, left, base,rotate=45.000000]{\color{textcolor}\sffamily\fontsize{10.000000}{12.000000}\selectfont 25\%}%
\end{pgfscope}%
\begin{pgfscope}%
\pgfsetbuttcap%
\pgfsetroundjoin%
\definecolor{currentfill}{rgb}{0.000000,0.000000,0.000000}%
\pgfsetfillcolor{currentfill}%
\pgfsetlinewidth{0.803000pt}%
\definecolor{currentstroke}{rgb}{0.000000,0.000000,0.000000}%
\pgfsetstrokecolor{currentstroke}%
\pgfsetdash{}{0pt}%
\pgfsys@defobject{currentmarker}{\pgfqpoint{0.000000in}{-0.048611in}}{\pgfqpoint{0.000000in}{0.000000in}}{%
\pgfpathmoveto{\pgfqpoint{0.000000in}{0.000000in}}%
\pgfpathlineto{\pgfqpoint{0.000000in}{-0.048611in}}%
\pgfusepath{stroke,fill}%
}%
\begin{pgfscope}%
\pgfsys@transformshift{3.445047in}{0.700520in}%
\pgfsys@useobject{currentmarker}{}%
\end{pgfscope}%
\end{pgfscope}%
\begin{pgfscope}%
\definecolor{textcolor}{rgb}{0.000000,0.000000,0.000000}%
\pgfsetstrokecolor{textcolor}%
\pgfsetfillcolor{textcolor}%
\pgftext[x=3.362998in, y=0.310396in, left, base,rotate=45.000000]{\color{textcolor}\sffamily\fontsize{10.000000}{12.000000}\selectfont 50\%}%
\end{pgfscope}%
\begin{pgfscope}%
\pgfsetbuttcap%
\pgfsetroundjoin%
\definecolor{currentfill}{rgb}{0.000000,0.000000,0.000000}%
\pgfsetfillcolor{currentfill}%
\pgfsetlinewidth{0.803000pt}%
\definecolor{currentstroke}{rgb}{0.000000,0.000000,0.000000}%
\pgfsetstrokecolor{currentstroke}%
\pgfsetdash{}{0pt}%
\pgfsys@defobject{currentmarker}{\pgfqpoint{0.000000in}{-0.048611in}}{\pgfqpoint{0.000000in}{0.000000in}}{%
\pgfpathmoveto{\pgfqpoint{0.000000in}{0.000000in}}%
\pgfpathlineto{\pgfqpoint{0.000000in}{-0.048611in}}%
\pgfusepath{stroke,fill}%
}%
\begin{pgfscope}%
\pgfsys@transformshift{4.464939in}{0.700520in}%
\pgfsys@useobject{currentmarker}{}%
\end{pgfscope}%
\end{pgfscope}%
\begin{pgfscope}%
\definecolor{textcolor}{rgb}{0.000000,0.000000,0.000000}%
\pgfsetstrokecolor{textcolor}%
\pgfsetfillcolor{textcolor}%
\pgftext[x=4.382891in, y=0.310396in, left, base,rotate=45.000000]{\color{textcolor}\sffamily\fontsize{10.000000}{12.000000}\selectfont 75\%}%
\end{pgfscope}%
\begin{pgfscope}%
\pgfsetbuttcap%
\pgfsetroundjoin%
\definecolor{currentfill}{rgb}{0.000000,0.000000,0.000000}%
\pgfsetfillcolor{currentfill}%
\pgfsetlinewidth{0.803000pt}%
\definecolor{currentstroke}{rgb}{0.000000,0.000000,0.000000}%
\pgfsetstrokecolor{currentstroke}%
\pgfsetdash{}{0pt}%
\pgfsys@defobject{currentmarker}{\pgfqpoint{0.000000in}{-0.048611in}}{\pgfqpoint{0.000000in}{0.000000in}}{%
\pgfpathmoveto{\pgfqpoint{0.000000in}{0.000000in}}%
\pgfpathlineto{\pgfqpoint{0.000000in}{-0.048611in}}%
\pgfusepath{stroke,fill}%
}%
\begin{pgfscope}%
\pgfsys@transformshift{5.484832in}{0.700520in}%
\pgfsys@useobject{currentmarker}{}%
\end{pgfscope}%
\end{pgfscope}%
\begin{pgfscope}%
\definecolor{textcolor}{rgb}{0.000000,0.000000,0.000000}%
\pgfsetstrokecolor{textcolor}%
\pgfsetfillcolor{textcolor}%
\pgftext[x=5.402783in, y=0.310396in, left, base,rotate=45.000000]{\color{textcolor}\sffamily\fontsize{10.000000}{12.000000}\selectfont 90\%}%
\end{pgfscope}%
\begin{pgfscope}%
\definecolor{textcolor}{rgb}{0.000000,0.000000,0.000000}%
\pgfsetstrokecolor{textcolor}%
\pgfsetfillcolor{textcolor}%
\pgftext[x=3.445047in,y=0.234413in,,top]{\color{textcolor}\sffamily\fontsize{10.000000}{12.000000}\selectfont Dataset}%
\end{pgfscope}%
\begin{pgfscope}%
\pgfsetbuttcap%
\pgfsetroundjoin%
\definecolor{currentfill}{rgb}{0.000000,0.000000,0.000000}%
\pgfsetfillcolor{currentfill}%
\pgfsetlinewidth{0.803000pt}%
\definecolor{currentstroke}{rgb}{0.000000,0.000000,0.000000}%
\pgfsetstrokecolor{currentstroke}%
\pgfsetdash{}{0pt}%
\pgfsys@defobject{currentmarker}{\pgfqpoint{-0.048611in}{0.000000in}}{\pgfqpoint{-0.000000in}{0.000000in}}{%
\pgfpathmoveto{\pgfqpoint{-0.000000in}{0.000000in}}%
\pgfpathlineto{\pgfqpoint{-0.048611in}{0.000000in}}%
\pgfusepath{stroke,fill}%
}%
\begin{pgfscope}%
\pgfsys@transformshift{0.696435in}{0.946292in}%
\pgfsys@useobject{currentmarker}{}%
\end{pgfscope}%
\end{pgfscope}%
\begin{pgfscope}%
\definecolor{textcolor}{rgb}{0.000000,0.000000,0.000000}%
\pgfsetstrokecolor{textcolor}%
\pgfsetfillcolor{textcolor}%
\pgftext[x=0.289968in, y=0.893530in, left, base]{\color{textcolor}\sffamily\fontsize{10.000000}{12.000000}\selectfont 0.26}%
\end{pgfscope}%
\begin{pgfscope}%
\pgfsetbuttcap%
\pgfsetroundjoin%
\definecolor{currentfill}{rgb}{0.000000,0.000000,0.000000}%
\pgfsetfillcolor{currentfill}%
\pgfsetlinewidth{0.803000pt}%
\definecolor{currentstroke}{rgb}{0.000000,0.000000,0.000000}%
\pgfsetstrokecolor{currentstroke}%
\pgfsetdash{}{0pt}%
\pgfsys@defobject{currentmarker}{\pgfqpoint{-0.048611in}{0.000000in}}{\pgfqpoint{-0.000000in}{0.000000in}}{%
\pgfpathmoveto{\pgfqpoint{-0.000000in}{0.000000in}}%
\pgfpathlineto{\pgfqpoint{-0.048611in}{0.000000in}}%
\pgfusepath{stroke,fill}%
}%
\begin{pgfscope}%
\pgfsys@transformshift{0.696435in}{1.437835in}%
\pgfsys@useobject{currentmarker}{}%
\end{pgfscope}%
\end{pgfscope}%
\begin{pgfscope}%
\definecolor{textcolor}{rgb}{0.000000,0.000000,0.000000}%
\pgfsetstrokecolor{textcolor}%
\pgfsetfillcolor{textcolor}%
\pgftext[x=0.289968in, y=1.385074in, left, base]{\color{textcolor}\sffamily\fontsize{10.000000}{12.000000}\selectfont 0.28}%
\end{pgfscope}%
\begin{pgfscope}%
\pgfsetbuttcap%
\pgfsetroundjoin%
\definecolor{currentfill}{rgb}{0.000000,0.000000,0.000000}%
\pgfsetfillcolor{currentfill}%
\pgfsetlinewidth{0.803000pt}%
\definecolor{currentstroke}{rgb}{0.000000,0.000000,0.000000}%
\pgfsetstrokecolor{currentstroke}%
\pgfsetdash{}{0pt}%
\pgfsys@defobject{currentmarker}{\pgfqpoint{-0.048611in}{0.000000in}}{\pgfqpoint{-0.000000in}{0.000000in}}{%
\pgfpathmoveto{\pgfqpoint{-0.000000in}{0.000000in}}%
\pgfpathlineto{\pgfqpoint{-0.048611in}{0.000000in}}%
\pgfusepath{stroke,fill}%
}%
\begin{pgfscope}%
\pgfsys@transformshift{0.696435in}{1.929378in}%
\pgfsys@useobject{currentmarker}{}%
\end{pgfscope}%
\end{pgfscope}%
\begin{pgfscope}%
\definecolor{textcolor}{rgb}{0.000000,0.000000,0.000000}%
\pgfsetstrokecolor{textcolor}%
\pgfsetfillcolor{textcolor}%
\pgftext[x=0.289968in, y=1.876617in, left, base]{\color{textcolor}\sffamily\fontsize{10.000000}{12.000000}\selectfont 0.30}%
\end{pgfscope}%
\begin{pgfscope}%
\pgfsetbuttcap%
\pgfsetroundjoin%
\definecolor{currentfill}{rgb}{0.000000,0.000000,0.000000}%
\pgfsetfillcolor{currentfill}%
\pgfsetlinewidth{0.803000pt}%
\definecolor{currentstroke}{rgb}{0.000000,0.000000,0.000000}%
\pgfsetstrokecolor{currentstroke}%
\pgfsetdash{}{0pt}%
\pgfsys@defobject{currentmarker}{\pgfqpoint{-0.048611in}{0.000000in}}{\pgfqpoint{-0.000000in}{0.000000in}}{%
\pgfpathmoveto{\pgfqpoint{-0.000000in}{0.000000in}}%
\pgfpathlineto{\pgfqpoint{-0.048611in}{0.000000in}}%
\pgfusepath{stroke,fill}%
}%
\begin{pgfscope}%
\pgfsys@transformshift{0.696435in}{2.420922in}%
\pgfsys@useobject{currentmarker}{}%
\end{pgfscope}%
\end{pgfscope}%
\begin{pgfscope}%
\definecolor{textcolor}{rgb}{0.000000,0.000000,0.000000}%
\pgfsetstrokecolor{textcolor}%
\pgfsetfillcolor{textcolor}%
\pgftext[x=0.289968in, y=2.368160in, left, base]{\color{textcolor}\sffamily\fontsize{10.000000}{12.000000}\selectfont 0.32}%
\end{pgfscope}%
\begin{pgfscope}%
\pgfsetbuttcap%
\pgfsetroundjoin%
\definecolor{currentfill}{rgb}{0.000000,0.000000,0.000000}%
\pgfsetfillcolor{currentfill}%
\pgfsetlinewidth{0.803000pt}%
\definecolor{currentstroke}{rgb}{0.000000,0.000000,0.000000}%
\pgfsetstrokecolor{currentstroke}%
\pgfsetdash{}{0pt}%
\pgfsys@defobject{currentmarker}{\pgfqpoint{-0.048611in}{0.000000in}}{\pgfqpoint{-0.000000in}{0.000000in}}{%
\pgfpathmoveto{\pgfqpoint{-0.000000in}{0.000000in}}%
\pgfpathlineto{\pgfqpoint{-0.048611in}{0.000000in}}%
\pgfusepath{stroke,fill}%
}%
\begin{pgfscope}%
\pgfsys@transformshift{0.696435in}{2.912465in}%
\pgfsys@useobject{currentmarker}{}%
\end{pgfscope}%
\end{pgfscope}%
\begin{pgfscope}%
\definecolor{textcolor}{rgb}{0.000000,0.000000,0.000000}%
\pgfsetstrokecolor{textcolor}%
\pgfsetfillcolor{textcolor}%
\pgftext[x=0.289968in, y=2.859703in, left, base]{\color{textcolor}\sffamily\fontsize{10.000000}{12.000000}\selectfont 0.34}%
\end{pgfscope}%
\begin{pgfscope}%
\pgfsetbuttcap%
\pgfsetroundjoin%
\definecolor{currentfill}{rgb}{0.000000,0.000000,0.000000}%
\pgfsetfillcolor{currentfill}%
\pgfsetlinewidth{0.803000pt}%
\definecolor{currentstroke}{rgb}{0.000000,0.000000,0.000000}%
\pgfsetstrokecolor{currentstroke}%
\pgfsetdash{}{0pt}%
\pgfsys@defobject{currentmarker}{\pgfqpoint{-0.048611in}{0.000000in}}{\pgfqpoint{-0.000000in}{0.000000in}}{%
\pgfpathmoveto{\pgfqpoint{-0.000000in}{0.000000in}}%
\pgfpathlineto{\pgfqpoint{-0.048611in}{0.000000in}}%
\pgfusepath{stroke,fill}%
}%
\begin{pgfscope}%
\pgfsys@transformshift{0.696435in}{3.404008in}%
\pgfsys@useobject{currentmarker}{}%
\end{pgfscope}%
\end{pgfscope}%
\begin{pgfscope}%
\definecolor{textcolor}{rgb}{0.000000,0.000000,0.000000}%
\pgfsetstrokecolor{textcolor}%
\pgfsetfillcolor{textcolor}%
\pgftext[x=0.289968in, y=3.351247in, left, base]{\color{textcolor}\sffamily\fontsize{10.000000}{12.000000}\selectfont 0.36}%
\end{pgfscope}%
\begin{pgfscope}%
\pgfsetbuttcap%
\pgfsetroundjoin%
\definecolor{currentfill}{rgb}{0.000000,0.000000,0.000000}%
\pgfsetfillcolor{currentfill}%
\pgfsetlinewidth{0.803000pt}%
\definecolor{currentstroke}{rgb}{0.000000,0.000000,0.000000}%
\pgfsetstrokecolor{currentstroke}%
\pgfsetdash{}{0pt}%
\pgfsys@defobject{currentmarker}{\pgfqpoint{-0.048611in}{0.000000in}}{\pgfqpoint{-0.000000in}{0.000000in}}{%
\pgfpathmoveto{\pgfqpoint{-0.000000in}{0.000000in}}%
\pgfpathlineto{\pgfqpoint{-0.048611in}{0.000000in}}%
\pgfusepath{stroke,fill}%
}%
\begin{pgfscope}%
\pgfsys@transformshift{0.696435in}{3.895552in}%
\pgfsys@useobject{currentmarker}{}%
\end{pgfscope}%
\end{pgfscope}%
\begin{pgfscope}%
\definecolor{textcolor}{rgb}{0.000000,0.000000,0.000000}%
\pgfsetstrokecolor{textcolor}%
\pgfsetfillcolor{textcolor}%
\pgftext[x=0.289968in, y=3.842790in, left, base]{\color{textcolor}\sffamily\fontsize{10.000000}{12.000000}\selectfont 0.38}%
\end{pgfscope}%
\begin{pgfscope}%
\pgfsetbuttcap%
\pgfsetroundjoin%
\definecolor{currentfill}{rgb}{0.000000,0.000000,0.000000}%
\pgfsetfillcolor{currentfill}%
\pgfsetlinewidth{0.803000pt}%
\definecolor{currentstroke}{rgb}{0.000000,0.000000,0.000000}%
\pgfsetstrokecolor{currentstroke}%
\pgfsetdash{}{0pt}%
\pgfsys@defobject{currentmarker}{\pgfqpoint{-0.048611in}{0.000000in}}{\pgfqpoint{-0.000000in}{0.000000in}}{%
\pgfpathmoveto{\pgfqpoint{-0.000000in}{0.000000in}}%
\pgfpathlineto{\pgfqpoint{-0.048611in}{0.000000in}}%
\pgfusepath{stroke,fill}%
}%
\begin{pgfscope}%
\pgfsys@transformshift{0.696435in}{4.387095in}%
\pgfsys@useobject{currentmarker}{}%
\end{pgfscope}%
\end{pgfscope}%
\begin{pgfscope}%
\definecolor{textcolor}{rgb}{0.000000,0.000000,0.000000}%
\pgfsetstrokecolor{textcolor}%
\pgfsetfillcolor{textcolor}%
\pgftext[x=0.289968in, y=4.334333in, left, base]{\color{textcolor}\sffamily\fontsize{10.000000}{12.000000}\selectfont 0.40}%
\end{pgfscope}%
\begin{pgfscope}%
\definecolor{textcolor}{rgb}{0.000000,0.000000,0.000000}%
\pgfsetstrokecolor{textcolor}%
\pgfsetfillcolor{textcolor}%
\pgftext[x=0.234413in,y=2.543808in,,bottom,rotate=90.000000]{\color{textcolor}\sffamily\fontsize{10.000000}{12.000000}\selectfont IoU}%
\end{pgfscope}%
\begin{pgfscope}%
\pgfsetrectcap%
\pgfsetmiterjoin%
\pgfsetlinewidth{0.803000pt}%
\definecolor{currentstroke}{rgb}{0.000000,0.000000,0.000000}%
\pgfsetstrokecolor{currentstroke}%
\pgfsetdash{}{0pt}%
\pgfpathmoveto{\pgfqpoint{0.696435in}{0.700520in}}%
\pgfpathlineto{\pgfqpoint{0.696435in}{4.387095in}}%
\pgfusepath{stroke}%
\end{pgfscope}%
\begin{pgfscope}%
\pgfsetrectcap%
\pgfsetmiterjoin%
\pgfsetlinewidth{0.803000pt}%
\definecolor{currentstroke}{rgb}{0.000000,0.000000,0.000000}%
\pgfsetstrokecolor{currentstroke}%
\pgfsetdash{}{0pt}%
\pgfpathmoveto{\pgfqpoint{6.193658in}{0.700520in}}%
\pgfpathlineto{\pgfqpoint{6.193658in}{4.387095in}}%
\pgfusepath{stroke}%
\end{pgfscope}%
\begin{pgfscope}%
\pgfsetrectcap%
\pgfsetmiterjoin%
\pgfsetlinewidth{0.803000pt}%
\definecolor{currentstroke}{rgb}{0.000000,0.000000,0.000000}%
\pgfsetstrokecolor{currentstroke}%
\pgfsetdash{}{0pt}%
\pgfpathmoveto{\pgfqpoint{0.696435in}{0.700520in}}%
\pgfpathlineto{\pgfqpoint{6.193658in}{0.700520in}}%
\pgfusepath{stroke}%
\end{pgfscope}%
\begin{pgfscope}%
\pgfsetrectcap%
\pgfsetmiterjoin%
\pgfsetlinewidth{0.803000pt}%
\definecolor{currentstroke}{rgb}{0.000000,0.000000,0.000000}%
\pgfsetstrokecolor{currentstroke}%
\pgfsetdash{}{0pt}%
\pgfpathmoveto{\pgfqpoint{0.696435in}{4.387095in}}%
\pgfpathlineto{\pgfqpoint{6.193658in}{4.387095in}}%
\pgfusepath{stroke}%
\end{pgfscope}%
\begin{pgfscope}%
\definecolor{textcolor}{rgb}{0.000000,0.000000,0.000000}%
\pgfsetstrokecolor{textcolor}%
\pgfsetfillcolor{textcolor}%
\pgftext[x=1.175785in,y=3.219565in,,bottom]{\color{textcolor}\sffamily\fontsize{9.000000}{10.800000}\selectfont 0.3508}%
\end{pgfscope}%
\begin{pgfscope}%
\definecolor{textcolor}{rgb}{0.000000,0.000000,0.000000}%
\pgfsetstrokecolor{textcolor}%
\pgfsetfillcolor{textcolor}%
\pgftext[x=2.195678in,y=3.462879in,,bottom]{\color{textcolor}\sffamily\fontsize{9.000000}{10.800000}\selectfont 0.3607}%
\end{pgfscope}%
\begin{pgfscope}%
\definecolor{textcolor}{rgb}{0.000000,0.000000,0.000000}%
\pgfsetstrokecolor{textcolor}%
\pgfsetfillcolor{textcolor}%
\pgftext[x=3.215571in,y=3.534153in,,bottom]{\color{textcolor}\sffamily\fontsize{9.000000}{10.800000}\selectfont 0.3636}%
\end{pgfscope}%
\begin{pgfscope}%
\definecolor{textcolor}{rgb}{0.000000,0.000000,0.000000}%
\pgfsetstrokecolor{textcolor}%
\pgfsetfillcolor{textcolor}%
\pgftext[x=4.235463in,y=3.349824in,,bottom]{\color{textcolor}\sffamily\fontsize{9.000000}{10.800000}\selectfont 0.3561}%
\end{pgfscope}%
\begin{pgfscope}%
\definecolor{textcolor}{rgb}{0.000000,0.000000,0.000000}%
\pgfsetstrokecolor{textcolor}%
\pgfsetfillcolor{textcolor}%
\pgftext[x=5.255356in,y=3.099137in,,bottom]{\color{textcolor}\sffamily\fontsize{9.000000}{10.800000}\selectfont 0.3459}%
\end{pgfscope}%
\begin{pgfscope}%
\definecolor{textcolor}{rgb}{0.000000,0.000000,0.000000}%
\pgfsetstrokecolor{textcolor}%
\pgfsetfillcolor{textcolor}%
\pgftext[x=1.634737in,y=2.300379in,,bottom]{\color{textcolor}\sffamily\fontsize{9.000000}{10.800000}\selectfont 0.3134}%
\end{pgfscope}%
\begin{pgfscope}%
\definecolor{textcolor}{rgb}{0.000000,0.000000,0.000000}%
\pgfsetstrokecolor{textcolor}%
\pgfsetfillcolor{textcolor}%
\pgftext[x=2.654630in,y=3.337535in,,bottom]{\color{textcolor}\sffamily\fontsize{9.000000}{10.800000}\selectfont 0.3556}%
\end{pgfscope}%
\begin{pgfscope}%
\definecolor{textcolor}{rgb}{0.000000,0.000000,0.000000}%
\pgfsetstrokecolor{textcolor}%
\pgfsetfillcolor{textcolor}%
\pgftext[x=3.674522in,y=3.413725in,,bottom]{\color{textcolor}\sffamily\fontsize{9.000000}{10.800000}\selectfont 0.3587}%
\end{pgfscope}%
\begin{pgfscope}%
\definecolor{textcolor}{rgb}{0.000000,0.000000,0.000000}%
\pgfsetstrokecolor{textcolor}%
\pgfsetfillcolor{textcolor}%
\pgftext[x=4.694415in,y=3.185157in,,bottom]{\color{textcolor}\sffamily\fontsize{9.000000}{10.800000}\selectfont 0.3494}%
\end{pgfscope}%
\begin{pgfscope}%
\definecolor{textcolor}{rgb}{0.000000,0.000000,0.000000}%
\pgfsetstrokecolor{textcolor}%
\pgfsetfillcolor{textcolor}%
\pgftext[x=5.714308in,y=3.155664in,,bottom]{\color{textcolor}\sffamily\fontsize{9.000000}{10.800000}\selectfont 0.3482}%
\end{pgfscope}%
\begin{pgfscope}%
\definecolor{textcolor}{rgb}{0.000000,0.000000,0.000000}%
\pgfsetstrokecolor{textcolor}%
\pgfsetfillcolor{textcolor}%
\pgftext[x=3.445047in,y=4.470428in,,base]{\color{textcolor}\sffamily\fontsize{12.000000}{14.400000}\selectfont Mixed training on Pix2Vox++ using 2 versions of S2R:3DFREE}%
\end{pgfscope}%
\begin{pgfscope}%
\pgfsetbuttcap%
\pgfsetmiterjoin%
\definecolor{currentfill}{rgb}{1.000000,1.000000,1.000000}%
\pgfsetfillcolor{currentfill}%
\pgfsetfillopacity{0.800000}%
\pgfsetlinewidth{1.003750pt}%
\definecolor{currentstroke}{rgb}{0.800000,0.800000,0.800000}%
\pgfsetstrokecolor{currentstroke}%
\pgfsetstrokeopacity{0.800000}%
\pgfsetdash{}{0pt}%
\pgfpathmoveto{\pgfqpoint{4.437595in}{3.868269in}}%
\pgfpathlineto{\pgfqpoint{6.096435in}{3.868269in}}%
\pgfpathquadraticcurveto{\pgfqpoint{6.124213in}{3.868269in}}{\pgfqpoint{6.124213in}{3.896047in}}%
\pgfpathlineto{\pgfqpoint{6.124213in}{4.289873in}}%
\pgfpathquadraticcurveto{\pgfqpoint{6.124213in}{4.317650in}}{\pgfqpoint{6.096435in}{4.317650in}}%
\pgfpathlineto{\pgfqpoint{4.437595in}{4.317650in}}%
\pgfpathquadraticcurveto{\pgfqpoint{4.409817in}{4.317650in}}{\pgfqpoint{4.409817in}{4.289873in}}%
\pgfpathlineto{\pgfqpoint{4.409817in}{3.896047in}}%
\pgfpathquadraticcurveto{\pgfqpoint{4.409817in}{3.868269in}}{\pgfqpoint{4.437595in}{3.868269in}}%
\pgfpathclose%
\pgfusepath{stroke,fill}%
\end{pgfscope}%
\begin{pgfscope}%
\pgfsetbuttcap%
\pgfsetmiterjoin%
\definecolor{currentfill}{rgb}{0.121569,0.466667,0.705882}%
\pgfsetfillcolor{currentfill}%
\pgfsetlinewidth{0.000000pt}%
\definecolor{currentstroke}{rgb}{0.000000,0.000000,0.000000}%
\pgfsetstrokecolor{currentstroke}%
\pgfsetstrokeopacity{0.000000}%
\pgfsetdash{}{0pt}%
\pgfpathmoveto{\pgfqpoint{4.465373in}{4.156572in}}%
\pgfpathlineto{\pgfqpoint{4.743150in}{4.156572in}}%
\pgfpathlineto{\pgfqpoint{4.743150in}{4.253794in}}%
\pgfpathlineto{\pgfqpoint{4.465373in}{4.253794in}}%
\pgfpathclose%
\pgfusepath{fill}%
\end{pgfscope}%
\begin{pgfscope}%
\definecolor{textcolor}{rgb}{0.000000,0.000000,0.000000}%
\pgfsetstrokecolor{textcolor}%
\pgfsetfillcolor{textcolor}%
\pgftext[x=4.854262in,y=4.156572in,left,base]{\color{textcolor}\sffamily\fontsize{10.000000}{12.000000}\selectfont V1 on Pix2Vox++}%
\end{pgfscope}%
\begin{pgfscope}%
\pgfsetbuttcap%
\pgfsetmiterjoin%
\definecolor{currentfill}{rgb}{1.000000,0.498039,0.054902}%
\pgfsetfillcolor{currentfill}%
\pgfsetlinewidth{0.000000pt}%
\definecolor{currentstroke}{rgb}{0.000000,0.000000,0.000000}%
\pgfsetstrokecolor{currentstroke}%
\pgfsetstrokeopacity{0.000000}%
\pgfsetdash{}{0pt}%
\pgfpathmoveto{\pgfqpoint{4.465373in}{3.952715in}}%
\pgfpathlineto{\pgfqpoint{4.743150in}{3.952715in}}%
\pgfpathlineto{\pgfqpoint{4.743150in}{4.049937in}}%
\pgfpathlineto{\pgfqpoint{4.465373in}{4.049937in}}%
\pgfpathclose%
\pgfusepath{fill}%
\end{pgfscope}%
\begin{pgfscope}%
\definecolor{textcolor}{rgb}{0.000000,0.000000,0.000000}%
\pgfsetstrokecolor{textcolor}%
\pgfsetfillcolor{textcolor}%
\pgftext[x=4.854262in,y=3.952715in,left,base]{\color{textcolor}\sffamily\fontsize{10.000000}{12.000000}\selectfont V2 on Pix2Vox++}%
\end{pgfscope}%
\end{pgfpicture}%
\makeatother%
\endgroup%
}
%    \resizebox{0.49\linewidth}{0.45\linewidth}{%% Creator: Matplotlib, PGF backend
%%
%% To include the figure in your LaTeX document, write
%%   \input{<filename>.pgf}
%%
%% Make sure the required packages are loaded in your preamble
%%   \usepackage{pgf}
%%
%% Figures using additional raster images can only be included by \input if
%% they are in the same directory as the main LaTeX file. For loading figures
%% from other directories you can use the `import` package
%%   \usepackage{import}
%%
%% and then include the figures with
%%   \import{<path to file>}{<filename>.pgf}
%%
%% Matplotlib used the following preamble
%%   \usepackage{fontspec}
%%   \setmainfont{DejaVuSerif.ttf}[Path=\detokenize{/Users/apple/opt/anaconda3/envs/kaolin/lib/python3.7/site-packages/matplotlib/mpl-data/fonts/ttf/}]
%%   \setsansfont{DejaVuSans.ttf}[Path=\detokenize{/Users/apple/opt/anaconda3/envs/kaolin/lib/python3.7/site-packages/matplotlib/mpl-data/fonts/ttf/}]
%%   \setmonofont{DejaVuSansMono.ttf}[Path=\detokenize{/Users/apple/opt/anaconda3/envs/kaolin/lib/python3.7/site-packages/matplotlib/mpl-data/fonts/ttf/}]
%%
\begingroup%
\makeatletter%
\begin{pgfpicture}%
\pgfpathrectangle{\pgfpointorigin}{\pgfqpoint{6.441712in}{4.704210in}}%
\pgfusepath{use as bounding box, clip}%
\begin{pgfscope}%
\pgfsetbuttcap%
\pgfsetmiterjoin%
\definecolor{currentfill}{rgb}{1.000000,1.000000,1.000000}%
\pgfsetfillcolor{currentfill}%
\pgfsetlinewidth{0.000000pt}%
\definecolor{currentstroke}{rgb}{1.000000,1.000000,1.000000}%
\pgfsetstrokecolor{currentstroke}%
\pgfsetdash{}{0pt}%
\pgfpathmoveto{\pgfqpoint{0.000000in}{0.000000in}}%
\pgfpathlineto{\pgfqpoint{6.441712in}{0.000000in}}%
\pgfpathlineto{\pgfqpoint{6.441712in}{4.704210in}}%
\pgfpathlineto{\pgfqpoint{0.000000in}{4.704210in}}%
\pgfpathclose%
\pgfusepath{fill}%
\end{pgfscope}%
\begin{pgfscope}%
\pgfsetbuttcap%
\pgfsetmiterjoin%
\definecolor{currentfill}{rgb}{1.000000,1.000000,1.000000}%
\pgfsetfillcolor{currentfill}%
\pgfsetlinewidth{0.000000pt}%
\definecolor{currentstroke}{rgb}{0.000000,0.000000,0.000000}%
\pgfsetstrokecolor{currentstroke}%
\pgfsetstrokeopacity{0.000000}%
\pgfsetdash{}{0pt}%
\pgfpathmoveto{\pgfqpoint{0.812050in}{0.816952in}}%
\pgfpathlineto{\pgfqpoint{6.198022in}{0.816952in}}%
\pgfpathlineto{\pgfqpoint{6.198022in}{4.373145in}}%
\pgfpathlineto{\pgfqpoint{0.812050in}{4.373145in}}%
\pgfpathclose%
\pgfusepath{fill}%
\end{pgfscope}%
\begin{pgfscope}%
\pgfpathrectangle{\pgfqpoint{0.812050in}{0.816952in}}{\pgfqpoint{5.385972in}{3.556193in}}%
\pgfusepath{clip}%
\pgfsetbuttcap%
\pgfsetmiterjoin%
\definecolor{currentfill}{rgb}{0.121569,0.466667,0.705882}%
\pgfsetfillcolor{currentfill}%
\pgfsetlinewidth{0.000000pt}%
\definecolor{currentstroke}{rgb}{0.000000,0.000000,0.000000}%
\pgfsetstrokecolor{currentstroke}%
\pgfsetstrokeopacity{0.000000}%
\pgfsetdash{}{0pt}%
\pgfpathmoveto{\pgfqpoint{1.056867in}{-5.110037in}}%
\pgfpathlineto{\pgfqpoint{1.506530in}{-5.110037in}}%
\pgfpathlineto{\pgfqpoint{1.506530in}{3.147444in}}%
\pgfpathlineto{\pgfqpoint{1.056867in}{3.147444in}}%
\pgfpathclose%
\pgfusepath{fill}%
\end{pgfscope}%
\begin{pgfscope}%
\pgfpathrectangle{\pgfqpoint{0.812050in}{0.816952in}}{\pgfqpoint{5.385972in}{3.556193in}}%
\pgfusepath{clip}%
\pgfsetbuttcap%
\pgfsetmiterjoin%
\definecolor{currentfill}{rgb}{0.121569,0.466667,0.705882}%
\pgfsetfillcolor{currentfill}%
\pgfsetlinewidth{0.000000pt}%
\definecolor{currentstroke}{rgb}{0.000000,0.000000,0.000000}%
\pgfsetstrokecolor{currentstroke}%
\pgfsetstrokeopacity{0.000000}%
\pgfsetdash{}{0pt}%
\pgfpathmoveto{\pgfqpoint{2.056119in}{-5.110037in}}%
\pgfpathlineto{\pgfqpoint{2.505783in}{-5.110037in}}%
\pgfpathlineto{\pgfqpoint{2.505783in}{3.026533in}}%
\pgfpathlineto{\pgfqpoint{2.056119in}{3.026533in}}%
\pgfpathclose%
\pgfusepath{fill}%
\end{pgfscope}%
\begin{pgfscope}%
\pgfpathrectangle{\pgfqpoint{0.812050in}{0.816952in}}{\pgfqpoint{5.385972in}{3.556193in}}%
\pgfusepath{clip}%
\pgfsetbuttcap%
\pgfsetmiterjoin%
\definecolor{currentfill}{rgb}{0.121569,0.466667,0.705882}%
\pgfsetfillcolor{currentfill}%
\pgfsetlinewidth{0.000000pt}%
\definecolor{currentstroke}{rgb}{0.000000,0.000000,0.000000}%
\pgfsetstrokecolor{currentstroke}%
\pgfsetstrokeopacity{0.000000}%
\pgfsetdash{}{0pt}%
\pgfpathmoveto{\pgfqpoint{3.055372in}{-5.110037in}}%
\pgfpathlineto{\pgfqpoint{3.505036in}{-5.110037in}}%
\pgfpathlineto{\pgfqpoint{3.505036in}{3.220938in}}%
\pgfpathlineto{\pgfqpoint{3.055372in}{3.220938in}}%
\pgfpathclose%
\pgfusepath{fill}%
\end{pgfscope}%
\begin{pgfscope}%
\pgfpathrectangle{\pgfqpoint{0.812050in}{0.816952in}}{\pgfqpoint{5.385972in}{3.556193in}}%
\pgfusepath{clip}%
\pgfsetbuttcap%
\pgfsetmiterjoin%
\definecolor{currentfill}{rgb}{0.121569,0.466667,0.705882}%
\pgfsetfillcolor{currentfill}%
\pgfsetlinewidth{0.000000pt}%
\definecolor{currentstroke}{rgb}{0.000000,0.000000,0.000000}%
\pgfsetstrokecolor{currentstroke}%
\pgfsetstrokeopacity{0.000000}%
\pgfsetdash{}{0pt}%
\pgfpathmoveto{\pgfqpoint{4.054625in}{-5.110037in}}%
\pgfpathlineto{\pgfqpoint{4.504288in}{-5.110037in}}%
\pgfpathlineto{\pgfqpoint{4.504288in}{3.021791in}}%
\pgfpathlineto{\pgfqpoint{4.054625in}{3.021791in}}%
\pgfpathclose%
\pgfusepath{fill}%
\end{pgfscope}%
\begin{pgfscope}%
\pgfpathrectangle{\pgfqpoint{0.812050in}{0.816952in}}{\pgfqpoint{5.385972in}{3.556193in}}%
\pgfusepath{clip}%
\pgfsetbuttcap%
\pgfsetmiterjoin%
\definecolor{currentfill}{rgb}{0.121569,0.466667,0.705882}%
\pgfsetfillcolor{currentfill}%
\pgfsetlinewidth{0.000000pt}%
\definecolor{currentstroke}{rgb}{0.000000,0.000000,0.000000}%
\pgfsetstrokecolor{currentstroke}%
\pgfsetstrokeopacity{0.000000}%
\pgfsetdash{}{0pt}%
\pgfpathmoveto{\pgfqpoint{5.053877in}{-5.110037in}}%
\pgfpathlineto{\pgfqpoint{5.503541in}{-5.110037in}}%
\pgfpathlineto{\pgfqpoint{5.503541in}{2.924589in}}%
\pgfpathlineto{\pgfqpoint{5.053877in}{2.924589in}}%
\pgfpathclose%
\pgfusepath{fill}%
\end{pgfscope}%
\begin{pgfscope}%
\pgfpathrectangle{\pgfqpoint{0.812050in}{0.816952in}}{\pgfqpoint{5.385972in}{3.556193in}}%
\pgfusepath{clip}%
\pgfsetbuttcap%
\pgfsetmiterjoin%
\definecolor{currentfill}{rgb}{1.000000,0.498039,0.054902}%
\pgfsetfillcolor{currentfill}%
\pgfsetlinewidth{0.000000pt}%
\definecolor{currentstroke}{rgb}{0.000000,0.000000,0.000000}%
\pgfsetstrokecolor{currentstroke}%
\pgfsetstrokeopacity{0.000000}%
\pgfsetdash{}{0pt}%
\pgfpathmoveto{\pgfqpoint{1.506530in}{-5.110037in}}%
\pgfpathlineto{\pgfqpoint{1.956194in}{-5.110037in}}%
\pgfpathlineto{\pgfqpoint{1.956194in}{2.066361in}}%
\pgfpathlineto{\pgfqpoint{1.506530in}{2.066361in}}%
\pgfpathclose%
\pgfusepath{fill}%
\end{pgfscope}%
\begin{pgfscope}%
\pgfpathrectangle{\pgfqpoint{0.812050in}{0.816952in}}{\pgfqpoint{5.385972in}{3.556193in}}%
\pgfusepath{clip}%
\pgfsetbuttcap%
\pgfsetmiterjoin%
\definecolor{currentfill}{rgb}{1.000000,0.498039,0.054902}%
\pgfsetfillcolor{currentfill}%
\pgfsetlinewidth{0.000000pt}%
\definecolor{currentstroke}{rgb}{0.000000,0.000000,0.000000}%
\pgfsetstrokecolor{currentstroke}%
\pgfsetstrokeopacity{0.000000}%
\pgfsetdash{}{0pt}%
\pgfpathmoveto{\pgfqpoint{2.505783in}{-5.110037in}}%
\pgfpathlineto{\pgfqpoint{2.955447in}{-5.110037in}}%
\pgfpathlineto{\pgfqpoint{2.955447in}{3.218567in}}%
\pgfpathlineto{\pgfqpoint{2.505783in}{3.218567in}}%
\pgfpathclose%
\pgfusepath{fill}%
\end{pgfscope}%
\begin{pgfscope}%
\pgfpathrectangle{\pgfqpoint{0.812050in}{0.816952in}}{\pgfqpoint{5.385972in}{3.556193in}}%
\pgfusepath{clip}%
\pgfsetbuttcap%
\pgfsetmiterjoin%
\definecolor{currentfill}{rgb}{1.000000,0.498039,0.054902}%
\pgfsetfillcolor{currentfill}%
\pgfsetlinewidth{0.000000pt}%
\definecolor{currentstroke}{rgb}{0.000000,0.000000,0.000000}%
\pgfsetstrokecolor{currentstroke}%
\pgfsetstrokeopacity{0.000000}%
\pgfsetdash{}{0pt}%
\pgfpathmoveto{\pgfqpoint{3.505036in}{-5.110037in}}%
\pgfpathlineto{\pgfqpoint{3.954699in}{-5.110037in}}%
\pgfpathlineto{\pgfqpoint{3.954699in}{2.988600in}}%
\pgfpathlineto{\pgfqpoint{3.505036in}{2.988600in}}%
\pgfpathclose%
\pgfusepath{fill}%
\end{pgfscope}%
\begin{pgfscope}%
\pgfpathrectangle{\pgfqpoint{0.812050in}{0.816952in}}{\pgfqpoint{5.385972in}{3.556193in}}%
\pgfusepath{clip}%
\pgfsetbuttcap%
\pgfsetmiterjoin%
\definecolor{currentfill}{rgb}{1.000000,0.498039,0.054902}%
\pgfsetfillcolor{currentfill}%
\pgfsetlinewidth{0.000000pt}%
\definecolor{currentstroke}{rgb}{0.000000,0.000000,0.000000}%
\pgfsetstrokecolor{currentstroke}%
\pgfsetstrokeopacity{0.000000}%
\pgfsetdash{}{0pt}%
\pgfpathmoveto{\pgfqpoint{4.504288in}{-5.110037in}}%
\pgfpathlineto{\pgfqpoint{4.953952in}{-5.110037in}}%
\pgfpathlineto{\pgfqpoint{4.953952in}{2.981488in}}%
\pgfpathlineto{\pgfqpoint{4.504288in}{2.981488in}}%
\pgfpathclose%
\pgfusepath{fill}%
\end{pgfscope}%
\begin{pgfscope}%
\pgfpathrectangle{\pgfqpoint{0.812050in}{0.816952in}}{\pgfqpoint{5.385972in}{3.556193in}}%
\pgfusepath{clip}%
\pgfsetbuttcap%
\pgfsetmiterjoin%
\definecolor{currentfill}{rgb}{1.000000,0.498039,0.054902}%
\pgfsetfillcolor{currentfill}%
\pgfsetlinewidth{0.000000pt}%
\definecolor{currentstroke}{rgb}{0.000000,0.000000,0.000000}%
\pgfsetstrokecolor{currentstroke}%
\pgfsetstrokeopacity{0.000000}%
\pgfsetdash{}{0pt}%
\pgfpathmoveto{\pgfqpoint{5.503541in}{-5.110037in}}%
\pgfpathlineto{\pgfqpoint{5.953205in}{-5.110037in}}%
\pgfpathlineto{\pgfqpoint{5.953205in}{3.031275in}}%
\pgfpathlineto{\pgfqpoint{5.503541in}{3.031275in}}%
\pgfpathclose%
\pgfusepath{fill}%
\end{pgfscope}%
\begin{pgfscope}%
\pgfsetbuttcap%
\pgfsetroundjoin%
\definecolor{currentfill}{rgb}{0.000000,0.000000,0.000000}%
\pgfsetfillcolor{currentfill}%
\pgfsetlinewidth{0.803000pt}%
\definecolor{currentstroke}{rgb}{0.000000,0.000000,0.000000}%
\pgfsetstrokecolor{currentstroke}%
\pgfsetdash{}{0pt}%
\pgfsys@defobject{currentmarker}{\pgfqpoint{0.000000in}{-0.048611in}}{\pgfqpoint{0.000000in}{0.000000in}}{%
\pgfpathmoveto{\pgfqpoint{0.000000in}{0.000000in}}%
\pgfpathlineto{\pgfqpoint{0.000000in}{-0.048611in}}%
\pgfusepath{stroke,fill}%
}%
\begin{pgfscope}%
\pgfsys@transformshift{1.506530in}{0.816952in}%
\pgfsys@useobject{currentmarker}{}%
\end{pgfscope}%
\end{pgfscope}%
\begin{pgfscope}%
\definecolor{textcolor}{rgb}{0.000000,0.000000,0.000000}%
\pgfsetstrokecolor{textcolor}%
\pgfsetfillcolor{textcolor}%
\pgftext[x=1.408072in, y=0.368247in, left, base,rotate=45.000000]{\color{textcolor}\sffamily\fontsize{12.000000}{14.400000}\selectfont 15\%}%
\end{pgfscope}%
\begin{pgfscope}%
\pgfsetbuttcap%
\pgfsetroundjoin%
\definecolor{currentfill}{rgb}{0.000000,0.000000,0.000000}%
\pgfsetfillcolor{currentfill}%
\pgfsetlinewidth{0.803000pt}%
\definecolor{currentstroke}{rgb}{0.000000,0.000000,0.000000}%
\pgfsetstrokecolor{currentstroke}%
\pgfsetdash{}{0pt}%
\pgfsys@defobject{currentmarker}{\pgfqpoint{0.000000in}{-0.048611in}}{\pgfqpoint{0.000000in}{0.000000in}}{%
\pgfpathmoveto{\pgfqpoint{0.000000in}{0.000000in}}%
\pgfpathlineto{\pgfqpoint{0.000000in}{-0.048611in}}%
\pgfusepath{stroke,fill}%
}%
\begin{pgfscope}%
\pgfsys@transformshift{2.505783in}{0.816952in}%
\pgfsys@useobject{currentmarker}{}%
\end{pgfscope}%
\end{pgfscope}%
\begin{pgfscope}%
\definecolor{textcolor}{rgb}{0.000000,0.000000,0.000000}%
\pgfsetstrokecolor{textcolor}%
\pgfsetfillcolor{textcolor}%
\pgftext[x=2.407324in, y=0.368247in, left, base,rotate=45.000000]{\color{textcolor}\sffamily\fontsize{12.000000}{14.400000}\selectfont 25\%}%
\end{pgfscope}%
\begin{pgfscope}%
\pgfsetbuttcap%
\pgfsetroundjoin%
\definecolor{currentfill}{rgb}{0.000000,0.000000,0.000000}%
\pgfsetfillcolor{currentfill}%
\pgfsetlinewidth{0.803000pt}%
\definecolor{currentstroke}{rgb}{0.000000,0.000000,0.000000}%
\pgfsetstrokecolor{currentstroke}%
\pgfsetdash{}{0pt}%
\pgfsys@defobject{currentmarker}{\pgfqpoint{0.000000in}{-0.048611in}}{\pgfqpoint{0.000000in}{0.000000in}}{%
\pgfpathmoveto{\pgfqpoint{0.000000in}{0.000000in}}%
\pgfpathlineto{\pgfqpoint{0.000000in}{-0.048611in}}%
\pgfusepath{stroke,fill}%
}%
\begin{pgfscope}%
\pgfsys@transformshift{3.505036in}{0.816952in}%
\pgfsys@useobject{currentmarker}{}%
\end{pgfscope}%
\end{pgfscope}%
\begin{pgfscope}%
\definecolor{textcolor}{rgb}{0.000000,0.000000,0.000000}%
\pgfsetstrokecolor{textcolor}%
\pgfsetfillcolor{textcolor}%
\pgftext[x=3.406577in, y=0.368247in, left, base,rotate=45.000000]{\color{textcolor}\sffamily\fontsize{12.000000}{14.400000}\selectfont 50\%}%
\end{pgfscope}%
\begin{pgfscope}%
\pgfsetbuttcap%
\pgfsetroundjoin%
\definecolor{currentfill}{rgb}{0.000000,0.000000,0.000000}%
\pgfsetfillcolor{currentfill}%
\pgfsetlinewidth{0.803000pt}%
\definecolor{currentstroke}{rgb}{0.000000,0.000000,0.000000}%
\pgfsetstrokecolor{currentstroke}%
\pgfsetdash{}{0pt}%
\pgfsys@defobject{currentmarker}{\pgfqpoint{0.000000in}{-0.048611in}}{\pgfqpoint{0.000000in}{0.000000in}}{%
\pgfpathmoveto{\pgfqpoint{0.000000in}{0.000000in}}%
\pgfpathlineto{\pgfqpoint{0.000000in}{-0.048611in}}%
\pgfusepath{stroke,fill}%
}%
\begin{pgfscope}%
\pgfsys@transformshift{4.504288in}{0.816952in}%
\pgfsys@useobject{currentmarker}{}%
\end{pgfscope}%
\end{pgfscope}%
\begin{pgfscope}%
\definecolor{textcolor}{rgb}{0.000000,0.000000,0.000000}%
\pgfsetstrokecolor{textcolor}%
\pgfsetfillcolor{textcolor}%
\pgftext[x=4.405830in, y=0.368247in, left, base,rotate=45.000000]{\color{textcolor}\sffamily\fontsize{12.000000}{14.400000}\selectfont 75\%}%
\end{pgfscope}%
\begin{pgfscope}%
\pgfsetbuttcap%
\pgfsetroundjoin%
\definecolor{currentfill}{rgb}{0.000000,0.000000,0.000000}%
\pgfsetfillcolor{currentfill}%
\pgfsetlinewidth{0.803000pt}%
\definecolor{currentstroke}{rgb}{0.000000,0.000000,0.000000}%
\pgfsetstrokecolor{currentstroke}%
\pgfsetdash{}{0pt}%
\pgfsys@defobject{currentmarker}{\pgfqpoint{0.000000in}{-0.048611in}}{\pgfqpoint{0.000000in}{0.000000in}}{%
\pgfpathmoveto{\pgfqpoint{0.000000in}{0.000000in}}%
\pgfpathlineto{\pgfqpoint{0.000000in}{-0.048611in}}%
\pgfusepath{stroke,fill}%
}%
\begin{pgfscope}%
\pgfsys@transformshift{5.503541in}{0.816952in}%
\pgfsys@useobject{currentmarker}{}%
\end{pgfscope}%
\end{pgfscope}%
\begin{pgfscope}%
\definecolor{textcolor}{rgb}{0.000000,0.000000,0.000000}%
\pgfsetstrokecolor{textcolor}%
\pgfsetfillcolor{textcolor}%
\pgftext[x=5.405083in, y=0.368247in, left, base,rotate=45.000000]{\color{textcolor}\sffamily\fontsize{12.000000}{14.400000}\selectfont 90\%}%
\end{pgfscope}%
\begin{pgfscope}%
\definecolor{textcolor}{rgb}{0.000000,0.000000,0.000000}%
\pgfsetstrokecolor{textcolor}%
\pgfsetfillcolor{textcolor}%
\pgftext[x=3.505036in,y=0.288178in,,top]{\color{textcolor}\sffamily\fontsize{14.000000}{16.800000}\bfseries\selectfont Dataset}%
\end{pgfscope}%
\begin{pgfscope}%
\pgfsetbuttcap%
\pgfsetroundjoin%
\definecolor{currentfill}{rgb}{0.000000,0.000000,0.000000}%
\pgfsetfillcolor{currentfill}%
\pgfsetlinewidth{0.803000pt}%
\definecolor{currentstroke}{rgb}{0.000000,0.000000,0.000000}%
\pgfsetstrokecolor{currentstroke}%
\pgfsetdash{}{0pt}%
\pgfsys@defobject{currentmarker}{\pgfqpoint{-0.048611in}{0.000000in}}{\pgfqpoint{-0.000000in}{0.000000in}}{%
\pgfpathmoveto{\pgfqpoint{-0.000000in}{0.000000in}}%
\pgfpathlineto{\pgfqpoint{-0.048611in}{0.000000in}}%
\pgfusepath{stroke,fill}%
}%
\begin{pgfscope}%
\pgfsys@transformshift{0.812050in}{1.054031in}%
\pgfsys@useobject{currentmarker}{}%
\end{pgfscope}%
\end{pgfscope}%
\begin{pgfscope}%
\definecolor{textcolor}{rgb}{0.000000,0.000000,0.000000}%
\pgfsetstrokecolor{textcolor}%
\pgfsetfillcolor{textcolor}%
\pgftext[x=0.343734in, y=0.990717in, left, base]{\color{textcolor}\sffamily\fontsize{12.000000}{14.400000}\selectfont 0.26}%
\end{pgfscope}%
\begin{pgfscope}%
\pgfsetbuttcap%
\pgfsetroundjoin%
\definecolor{currentfill}{rgb}{0.000000,0.000000,0.000000}%
\pgfsetfillcolor{currentfill}%
\pgfsetlinewidth{0.803000pt}%
\definecolor{currentstroke}{rgb}{0.000000,0.000000,0.000000}%
\pgfsetstrokecolor{currentstroke}%
\pgfsetdash{}{0pt}%
\pgfsys@defobject{currentmarker}{\pgfqpoint{-0.048611in}{0.000000in}}{\pgfqpoint{-0.000000in}{0.000000in}}{%
\pgfpathmoveto{\pgfqpoint{-0.000000in}{0.000000in}}%
\pgfpathlineto{\pgfqpoint{-0.048611in}{0.000000in}}%
\pgfusepath{stroke,fill}%
}%
\begin{pgfscope}%
\pgfsys@transformshift{0.812050in}{1.528190in}%
\pgfsys@useobject{currentmarker}{}%
\end{pgfscope}%
\end{pgfscope}%
\begin{pgfscope}%
\definecolor{textcolor}{rgb}{0.000000,0.000000,0.000000}%
\pgfsetstrokecolor{textcolor}%
\pgfsetfillcolor{textcolor}%
\pgftext[x=0.343734in, y=1.464876in, left, base]{\color{textcolor}\sffamily\fontsize{12.000000}{14.400000}\selectfont 0.28}%
\end{pgfscope}%
\begin{pgfscope}%
\pgfsetbuttcap%
\pgfsetroundjoin%
\definecolor{currentfill}{rgb}{0.000000,0.000000,0.000000}%
\pgfsetfillcolor{currentfill}%
\pgfsetlinewidth{0.803000pt}%
\definecolor{currentstroke}{rgb}{0.000000,0.000000,0.000000}%
\pgfsetstrokecolor{currentstroke}%
\pgfsetdash{}{0pt}%
\pgfsys@defobject{currentmarker}{\pgfqpoint{-0.048611in}{0.000000in}}{\pgfqpoint{-0.000000in}{0.000000in}}{%
\pgfpathmoveto{\pgfqpoint{-0.000000in}{0.000000in}}%
\pgfpathlineto{\pgfqpoint{-0.048611in}{0.000000in}}%
\pgfusepath{stroke,fill}%
}%
\begin{pgfscope}%
\pgfsys@transformshift{0.812050in}{2.002349in}%
\pgfsys@useobject{currentmarker}{}%
\end{pgfscope}%
\end{pgfscope}%
\begin{pgfscope}%
\definecolor{textcolor}{rgb}{0.000000,0.000000,0.000000}%
\pgfsetstrokecolor{textcolor}%
\pgfsetfillcolor{textcolor}%
\pgftext[x=0.343734in, y=1.939035in, left, base]{\color{textcolor}\sffamily\fontsize{12.000000}{14.400000}\selectfont 0.30}%
\end{pgfscope}%
\begin{pgfscope}%
\pgfsetbuttcap%
\pgfsetroundjoin%
\definecolor{currentfill}{rgb}{0.000000,0.000000,0.000000}%
\pgfsetfillcolor{currentfill}%
\pgfsetlinewidth{0.803000pt}%
\definecolor{currentstroke}{rgb}{0.000000,0.000000,0.000000}%
\pgfsetstrokecolor{currentstroke}%
\pgfsetdash{}{0pt}%
\pgfsys@defobject{currentmarker}{\pgfqpoint{-0.048611in}{0.000000in}}{\pgfqpoint{-0.000000in}{0.000000in}}{%
\pgfpathmoveto{\pgfqpoint{-0.000000in}{0.000000in}}%
\pgfpathlineto{\pgfqpoint{-0.048611in}{0.000000in}}%
\pgfusepath{stroke,fill}%
}%
\begin{pgfscope}%
\pgfsys@transformshift{0.812050in}{2.476508in}%
\pgfsys@useobject{currentmarker}{}%
\end{pgfscope}%
\end{pgfscope}%
\begin{pgfscope}%
\definecolor{textcolor}{rgb}{0.000000,0.000000,0.000000}%
\pgfsetstrokecolor{textcolor}%
\pgfsetfillcolor{textcolor}%
\pgftext[x=0.343734in, y=2.413195in, left, base]{\color{textcolor}\sffamily\fontsize{12.000000}{14.400000}\selectfont 0.32}%
\end{pgfscope}%
\begin{pgfscope}%
\pgfsetbuttcap%
\pgfsetroundjoin%
\definecolor{currentfill}{rgb}{0.000000,0.000000,0.000000}%
\pgfsetfillcolor{currentfill}%
\pgfsetlinewidth{0.803000pt}%
\definecolor{currentstroke}{rgb}{0.000000,0.000000,0.000000}%
\pgfsetstrokecolor{currentstroke}%
\pgfsetdash{}{0pt}%
\pgfsys@defobject{currentmarker}{\pgfqpoint{-0.048611in}{0.000000in}}{\pgfqpoint{-0.000000in}{0.000000in}}{%
\pgfpathmoveto{\pgfqpoint{-0.000000in}{0.000000in}}%
\pgfpathlineto{\pgfqpoint{-0.048611in}{0.000000in}}%
\pgfusepath{stroke,fill}%
}%
\begin{pgfscope}%
\pgfsys@transformshift{0.812050in}{2.950668in}%
\pgfsys@useobject{currentmarker}{}%
\end{pgfscope}%
\end{pgfscope}%
\begin{pgfscope}%
\definecolor{textcolor}{rgb}{0.000000,0.000000,0.000000}%
\pgfsetstrokecolor{textcolor}%
\pgfsetfillcolor{textcolor}%
\pgftext[x=0.343734in, y=2.887354in, left, base]{\color{textcolor}\sffamily\fontsize{12.000000}{14.400000}\selectfont 0.34}%
\end{pgfscope}%
\begin{pgfscope}%
\pgfsetbuttcap%
\pgfsetroundjoin%
\definecolor{currentfill}{rgb}{0.000000,0.000000,0.000000}%
\pgfsetfillcolor{currentfill}%
\pgfsetlinewidth{0.803000pt}%
\definecolor{currentstroke}{rgb}{0.000000,0.000000,0.000000}%
\pgfsetstrokecolor{currentstroke}%
\pgfsetdash{}{0pt}%
\pgfsys@defobject{currentmarker}{\pgfqpoint{-0.048611in}{0.000000in}}{\pgfqpoint{-0.000000in}{0.000000in}}{%
\pgfpathmoveto{\pgfqpoint{-0.000000in}{0.000000in}}%
\pgfpathlineto{\pgfqpoint{-0.048611in}{0.000000in}}%
\pgfusepath{stroke,fill}%
}%
\begin{pgfscope}%
\pgfsys@transformshift{0.812050in}{3.424827in}%
\pgfsys@useobject{currentmarker}{}%
\end{pgfscope}%
\end{pgfscope}%
\begin{pgfscope}%
\definecolor{textcolor}{rgb}{0.000000,0.000000,0.000000}%
\pgfsetstrokecolor{textcolor}%
\pgfsetfillcolor{textcolor}%
\pgftext[x=0.343734in, y=3.361513in, left, base]{\color{textcolor}\sffamily\fontsize{12.000000}{14.400000}\selectfont 0.36}%
\end{pgfscope}%
\begin{pgfscope}%
\pgfsetbuttcap%
\pgfsetroundjoin%
\definecolor{currentfill}{rgb}{0.000000,0.000000,0.000000}%
\pgfsetfillcolor{currentfill}%
\pgfsetlinewidth{0.803000pt}%
\definecolor{currentstroke}{rgb}{0.000000,0.000000,0.000000}%
\pgfsetstrokecolor{currentstroke}%
\pgfsetdash{}{0pt}%
\pgfsys@defobject{currentmarker}{\pgfqpoint{-0.048611in}{0.000000in}}{\pgfqpoint{-0.000000in}{0.000000in}}{%
\pgfpathmoveto{\pgfqpoint{-0.000000in}{0.000000in}}%
\pgfpathlineto{\pgfqpoint{-0.048611in}{0.000000in}}%
\pgfusepath{stroke,fill}%
}%
\begin{pgfscope}%
\pgfsys@transformshift{0.812050in}{3.898986in}%
\pgfsys@useobject{currentmarker}{}%
\end{pgfscope}%
\end{pgfscope}%
\begin{pgfscope}%
\definecolor{textcolor}{rgb}{0.000000,0.000000,0.000000}%
\pgfsetstrokecolor{textcolor}%
\pgfsetfillcolor{textcolor}%
\pgftext[x=0.343734in, y=3.835672in, left, base]{\color{textcolor}\sffamily\fontsize{12.000000}{14.400000}\selectfont 0.38}%
\end{pgfscope}%
\begin{pgfscope}%
\pgfsetbuttcap%
\pgfsetroundjoin%
\definecolor{currentfill}{rgb}{0.000000,0.000000,0.000000}%
\pgfsetfillcolor{currentfill}%
\pgfsetlinewidth{0.803000pt}%
\definecolor{currentstroke}{rgb}{0.000000,0.000000,0.000000}%
\pgfsetstrokecolor{currentstroke}%
\pgfsetdash{}{0pt}%
\pgfsys@defobject{currentmarker}{\pgfqpoint{-0.048611in}{0.000000in}}{\pgfqpoint{-0.000000in}{0.000000in}}{%
\pgfpathmoveto{\pgfqpoint{-0.000000in}{0.000000in}}%
\pgfpathlineto{\pgfqpoint{-0.048611in}{0.000000in}}%
\pgfusepath{stroke,fill}%
}%
\begin{pgfscope}%
\pgfsys@transformshift{0.812050in}{4.373145in}%
\pgfsys@useobject{currentmarker}{}%
\end{pgfscope}%
\end{pgfscope}%
\begin{pgfscope}%
\definecolor{textcolor}{rgb}{0.000000,0.000000,0.000000}%
\pgfsetstrokecolor{textcolor}%
\pgfsetfillcolor{textcolor}%
\pgftext[x=0.343734in, y=4.309831in, left, base]{\color{textcolor}\sffamily\fontsize{12.000000}{14.400000}\selectfont 0.40}%
\end{pgfscope}%
\begin{pgfscope}%
\definecolor{textcolor}{rgb}{0.000000,0.000000,0.000000}%
\pgfsetstrokecolor{textcolor}%
\pgfsetfillcolor{textcolor}%
\pgftext[x=0.288178in,y=2.595048in,,bottom,rotate=90.000000]{\color{textcolor}\sffamily\fontsize{14.000000}{16.800000}\bfseries\selectfont IoU}%
\end{pgfscope}%
\begin{pgfscope}%
\pgfsetrectcap%
\pgfsetmiterjoin%
\pgfsetlinewidth{0.803000pt}%
\definecolor{currentstroke}{rgb}{0.000000,0.000000,0.000000}%
\pgfsetstrokecolor{currentstroke}%
\pgfsetdash{}{0pt}%
\pgfpathmoveto{\pgfqpoint{0.812050in}{0.816952in}}%
\pgfpathlineto{\pgfqpoint{0.812050in}{4.373145in}}%
\pgfusepath{stroke}%
\end{pgfscope}%
\begin{pgfscope}%
\pgfsetrectcap%
\pgfsetmiterjoin%
\pgfsetlinewidth{0.803000pt}%
\definecolor{currentstroke}{rgb}{0.000000,0.000000,0.000000}%
\pgfsetstrokecolor{currentstroke}%
\pgfsetdash{}{0pt}%
\pgfpathmoveto{\pgfqpoint{6.198022in}{0.816952in}}%
\pgfpathlineto{\pgfqpoint{6.198022in}{4.373145in}}%
\pgfusepath{stroke}%
\end{pgfscope}%
\begin{pgfscope}%
\pgfsetrectcap%
\pgfsetmiterjoin%
\pgfsetlinewidth{0.803000pt}%
\definecolor{currentstroke}{rgb}{0.000000,0.000000,0.000000}%
\pgfsetstrokecolor{currentstroke}%
\pgfsetdash{}{0pt}%
\pgfpathmoveto{\pgfqpoint{0.812050in}{0.816952in}}%
\pgfpathlineto{\pgfqpoint{6.198022in}{0.816952in}}%
\pgfusepath{stroke}%
\end{pgfscope}%
\begin{pgfscope}%
\pgfsetrectcap%
\pgfsetmiterjoin%
\pgfsetlinewidth{0.803000pt}%
\definecolor{currentstroke}{rgb}{0.000000,0.000000,0.000000}%
\pgfsetstrokecolor{currentstroke}%
\pgfsetdash{}{0pt}%
\pgfpathmoveto{\pgfqpoint{0.812050in}{4.373145in}}%
\pgfpathlineto{\pgfqpoint{6.198022in}{4.373145in}}%
\pgfusepath{stroke}%
\end{pgfscope}%
\begin{pgfscope}%
\definecolor{textcolor}{rgb}{0.000000,0.000000,0.000000}%
\pgfsetstrokecolor{textcolor}%
\pgfsetfillcolor{textcolor}%
\pgftext[x=1.281698in,y=3.189110in,,bottom]{\color{textcolor}\sffamily\fontsize{9.000000}{10.800000}\selectfont 0.3483}%
\end{pgfscope}%
\begin{pgfscope}%
\definecolor{textcolor}{rgb}{0.000000,0.000000,0.000000}%
\pgfsetstrokecolor{textcolor}%
\pgfsetfillcolor{textcolor}%
\pgftext[x=2.280951in,y=3.068200in,,bottom]{\color{textcolor}\sffamily\fontsize{9.000000}{10.800000}\selectfont 0.3432}%
\end{pgfscope}%
\begin{pgfscope}%
\definecolor{textcolor}{rgb}{0.000000,0.000000,0.000000}%
\pgfsetstrokecolor{textcolor}%
\pgfsetfillcolor{textcolor}%
\pgftext[x=3.280204in,y=3.262605in,,bottom]{\color{textcolor}\sffamily\fontsize{9.000000}{10.800000}\selectfont 0.3514}%
\end{pgfscope}%
\begin{pgfscope}%
\definecolor{textcolor}{rgb}{0.000000,0.000000,0.000000}%
\pgfsetstrokecolor{textcolor}%
\pgfsetfillcolor{textcolor}%
\pgftext[x=4.279457in,y=3.063458in,,bottom]{\color{textcolor}\sffamily\fontsize{9.000000}{10.800000}\selectfont 0.343}%
\end{pgfscope}%
\begin{pgfscope}%
\definecolor{textcolor}{rgb}{0.000000,0.000000,0.000000}%
\pgfsetstrokecolor{textcolor}%
\pgfsetfillcolor{textcolor}%
\pgftext[x=5.278709in,y=2.966255in,,bottom]{\color{textcolor}\sffamily\fontsize{9.000000}{10.800000}\selectfont 0.3389}%
\end{pgfscope}%
\begin{pgfscope}%
\definecolor{textcolor}{rgb}{0.000000,0.000000,0.000000}%
\pgfsetstrokecolor{textcolor}%
\pgfsetfillcolor{textcolor}%
\pgftext[x=1.731362in,y=2.108027in,,bottom]{\color{textcolor}\sffamily\fontsize{9.000000}{10.800000}\selectfont 0.3027}%
\end{pgfscope}%
\begin{pgfscope}%
\definecolor{textcolor}{rgb}{0.000000,0.000000,0.000000}%
\pgfsetstrokecolor{textcolor}%
\pgfsetfillcolor{textcolor}%
\pgftext[x=2.730615in,y=3.260234in,,bottom]{\color{textcolor}\sffamily\fontsize{9.000000}{10.800000}\selectfont 0.3513}%
\end{pgfscope}%
\begin{pgfscope}%
\definecolor{textcolor}{rgb}{0.000000,0.000000,0.000000}%
\pgfsetstrokecolor{textcolor}%
\pgfsetfillcolor{textcolor}%
\pgftext[x=3.729868in,y=3.030267in,,bottom]{\color{textcolor}\sffamily\fontsize{9.000000}{10.800000}\selectfont 0.3416}%
\end{pgfscope}%
\begin{pgfscope}%
\definecolor{textcolor}{rgb}{0.000000,0.000000,0.000000}%
\pgfsetstrokecolor{textcolor}%
\pgfsetfillcolor{textcolor}%
\pgftext[x=4.729120in,y=3.023155in,,bottom]{\color{textcolor}\sffamily\fontsize{9.000000}{10.800000}\selectfont 0.3413}%
\end{pgfscope}%
\begin{pgfscope}%
\definecolor{textcolor}{rgb}{0.000000,0.000000,0.000000}%
\pgfsetstrokecolor{textcolor}%
\pgfsetfillcolor{textcolor}%
\pgftext[x=5.728373in,y=3.072941in,,bottom]{\color{textcolor}\sffamily\fontsize{9.000000}{10.800000}\selectfont 0.3434}%
\end{pgfscope}%
\begin{pgfscope}%
\definecolor{textcolor}{rgb}{0.000000,0.000000,0.000000}%
\pgfsetstrokecolor{textcolor}%
\pgfsetfillcolor{textcolor}%
\pgftext[x=3.505036in,y=4.456478in,,base]{\color{textcolor}\sffamily\fontsize{14.000000}{16.800000}\selectfont Mixed training on Pix2Vox using 2 versions of S2R:3DFREE}%
\end{pgfscope}%
\begin{pgfscope}%
\pgfsetbuttcap%
\pgfsetmiterjoin%
\definecolor{currentfill}{rgb}{1.000000,1.000000,1.000000}%
\pgfsetfillcolor{currentfill}%
\pgfsetfillopacity{0.800000}%
\pgfsetlinewidth{1.003750pt}%
\definecolor{currentstroke}{rgb}{0.800000,0.800000,0.800000}%
\pgfsetstrokecolor{currentstroke}%
\pgfsetstrokeopacity{0.800000}%
\pgfsetdash{}{0pt}%
\pgfpathmoveto{\pgfqpoint{4.674706in}{3.854319in}}%
\pgfpathlineto{\pgfqpoint{6.100800in}{3.854319in}}%
\pgfpathquadraticcurveto{\pgfqpoint{6.128577in}{3.854319in}}{\pgfqpoint{6.128577in}{3.882097in}}%
\pgfpathlineto{\pgfqpoint{6.128577in}{4.275923in}}%
\pgfpathquadraticcurveto{\pgfqpoint{6.128577in}{4.303700in}}{\pgfqpoint{6.100800in}{4.303700in}}%
\pgfpathlineto{\pgfqpoint{4.674706in}{4.303700in}}%
\pgfpathquadraticcurveto{\pgfqpoint{4.646929in}{4.303700in}}{\pgfqpoint{4.646929in}{4.275923in}}%
\pgfpathlineto{\pgfqpoint{4.646929in}{3.882097in}}%
\pgfpathquadraticcurveto{\pgfqpoint{4.646929in}{3.854319in}}{\pgfqpoint{4.674706in}{3.854319in}}%
\pgfpathclose%
\pgfusepath{stroke,fill}%
\end{pgfscope}%
\begin{pgfscope}%
\pgfsetbuttcap%
\pgfsetmiterjoin%
\definecolor{currentfill}{rgb}{0.121569,0.466667,0.705882}%
\pgfsetfillcolor{currentfill}%
\pgfsetlinewidth{0.000000pt}%
\definecolor{currentstroke}{rgb}{0.000000,0.000000,0.000000}%
\pgfsetstrokecolor{currentstroke}%
\pgfsetstrokeopacity{0.000000}%
\pgfsetdash{}{0pt}%
\pgfpathmoveto{\pgfqpoint{4.702484in}{4.142622in}}%
\pgfpathlineto{\pgfqpoint{4.980262in}{4.142622in}}%
\pgfpathlineto{\pgfqpoint{4.980262in}{4.239844in}}%
\pgfpathlineto{\pgfqpoint{4.702484in}{4.239844in}}%
\pgfpathclose%
\pgfusepath{fill}%
\end{pgfscope}%
\begin{pgfscope}%
\definecolor{textcolor}{rgb}{0.000000,0.000000,0.000000}%
\pgfsetstrokecolor{textcolor}%
\pgfsetfillcolor{textcolor}%
\pgftext[x=5.091373in,y=4.142622in,left,base]{\color{textcolor}\sffamily\fontsize{10.000000}{12.000000}\selectfont V1 on Pix2Vox}%
\end{pgfscope}%
\begin{pgfscope}%
\pgfsetbuttcap%
\pgfsetmiterjoin%
\definecolor{currentfill}{rgb}{1.000000,0.498039,0.054902}%
\pgfsetfillcolor{currentfill}%
\pgfsetlinewidth{0.000000pt}%
\definecolor{currentstroke}{rgb}{0.000000,0.000000,0.000000}%
\pgfsetstrokecolor{currentstroke}%
\pgfsetstrokeopacity{0.000000}%
\pgfsetdash{}{0pt}%
\pgfpathmoveto{\pgfqpoint{4.702484in}{3.938765in}}%
\pgfpathlineto{\pgfqpoint{4.980262in}{3.938765in}}%
\pgfpathlineto{\pgfqpoint{4.980262in}{4.035987in}}%
\pgfpathlineto{\pgfqpoint{4.702484in}{4.035987in}}%
\pgfpathclose%
\pgfusepath{fill}%
\end{pgfscope}%
\begin{pgfscope}%
\definecolor{textcolor}{rgb}{0.000000,0.000000,0.000000}%
\pgfsetstrokecolor{textcolor}%
\pgfsetfillcolor{textcolor}%
\pgftext[x=5.091373in,y=3.938765in,left,base]{\color{textcolor}\sffamily\fontsize{10.000000}{12.000000}\selectfont V2 on Pix2Vox}%
\end{pgfscope}%
\end{pgfpicture}%
\makeatother%
\endgroup%
}
%    \caption{Bar plot for the \gls{iou} for baselines trained on different ratios of synthetic and real dataset per mini-batch.(left)\textbf{Mixed training on Pix2Vox++}, (right)\textbf{Mixed training on Pix2Vox}.
%    In both cases we see a slight increase in \gls{iou} with addition of real data, and a gradual decrease till it reaches 100\% real data}
%    \label{fig:mixed1}
%\end{figure}

\autoref{fig:mixed2} and \autoref{fig:mixed3}, represent \gls{iou} per category present in the Pix3D and \gls{free}.
The number of images per category is also mentioned on the x-label.
We see that table and desk are most difficult to reconstruct as the \gls{iou} is least, while the wardrobe is the easiest category to reconstruct and gets the most points.
In the majority of the cases, we see that \fls{iou} of models trained on mixed data achieves higher performance than models trained on only real dataset.

In \autoref{fig:mixed_images1}, we see the 3D reconstruction output for models mixed trained with 50\% synthetic and real data per mini-batch.
The outputs were collected for images from the real dataset with the threshold which gave the best \gls{iou}.
The output of models is less noisy than models fine-tuned, as in \autoref{fig:finetuning_images1}, for most reconstructions.
The models seem to have much better details as well.

%\begin{figure}
%    \centering
%    \resizebox{0.49\linewidth}{!}{%% Creator: Matplotlib, PGF backend
%%
%% To include the figure in your LaTeX document, write
%%   \input{<filename>.pgf}
%%
%% Make sure the required packages are loaded in your preamble
%%   \usepackage{pgf}
%%
%% Figures using additional raster images can only be included by \input if
%% they are in the same directory as the main LaTeX file. For loading figures
%% from other directories you can use the `import` package
%%   \usepackage{import}
%%
%% and then include the figures with
%%   \import{<path to file>}{<filename>.pgf}
%%
%% Matplotlib used the following preamble
%%   \usepackage{fontspec}
%%   \setmainfont{DejaVuSerif.ttf}[Path=\detokenize{/Users/apple/opt/anaconda3/envs/kaolin/lib/python3.7/site-packages/matplotlib/mpl-data/fonts/ttf/}]
%%   \setsansfont{DejaVuSans.ttf}[Path=\detokenize{/Users/apple/opt/anaconda3/envs/kaolin/lib/python3.7/site-packages/matplotlib/mpl-data/fonts/ttf/}]
%%   \setmonofont{DejaVuSansMono.ttf}[Path=\detokenize{/Users/apple/opt/anaconda3/envs/kaolin/lib/python3.7/site-packages/matplotlib/mpl-data/fonts/ttf/}]
%%
\begingroup%
\makeatletter%
\begin{pgfpicture}%
\pgfpathrectangle{\pgfpointorigin}{\pgfqpoint{5.756435in}{4.549282in}}%
\pgfusepath{use as bounding box, clip}%
\begin{pgfscope}%
\pgfsetbuttcap%
\pgfsetmiterjoin%
\definecolor{currentfill}{rgb}{1.000000,1.000000,1.000000}%
\pgfsetfillcolor{currentfill}%
\pgfsetlinewidth{0.000000pt}%
\definecolor{currentstroke}{rgb}{1.000000,1.000000,1.000000}%
\pgfsetstrokecolor{currentstroke}%
\pgfsetdash{}{0pt}%
\pgfpathmoveto{\pgfqpoint{0.000000in}{0.000000in}}%
\pgfpathlineto{\pgfqpoint{5.756435in}{0.000000in}}%
\pgfpathlineto{\pgfqpoint{5.756435in}{4.549282in}}%
\pgfpathlineto{\pgfqpoint{0.000000in}{4.549282in}}%
\pgfpathclose%
\pgfusepath{fill}%
\end{pgfscope}%
\begin{pgfscope}%
\pgfsetbuttcap%
\pgfsetmiterjoin%
\definecolor{currentfill}{rgb}{1.000000,1.000000,1.000000}%
\pgfsetfillcolor{currentfill}%
\pgfsetlinewidth{0.000000pt}%
\definecolor{currentstroke}{rgb}{0.000000,0.000000,0.000000}%
\pgfsetstrokecolor{currentstroke}%
\pgfsetstrokeopacity{0.000000}%
\pgfsetdash{}{0pt}%
\pgfpathmoveto{\pgfqpoint{0.696435in}{0.700520in}}%
\pgfpathlineto{\pgfqpoint{5.656435in}{0.700520in}}%
\pgfpathlineto{\pgfqpoint{5.656435in}{4.396520in}}%
\pgfpathlineto{\pgfqpoint{0.696435in}{4.396520in}}%
\pgfpathclose%
\pgfusepath{fill}%
\end{pgfscope}%
\begin{pgfscope}%
\pgfsetbuttcap%
\pgfsetroundjoin%
\definecolor{currentfill}{rgb}{0.000000,0.000000,0.000000}%
\pgfsetfillcolor{currentfill}%
\pgfsetlinewidth{0.803000pt}%
\definecolor{currentstroke}{rgb}{0.000000,0.000000,0.000000}%
\pgfsetstrokecolor{currentstroke}%
\pgfsetdash{}{0pt}%
\pgfsys@defobject{currentmarker}{\pgfqpoint{0.000000in}{-0.048611in}}{\pgfqpoint{0.000000in}{0.000000in}}{%
\pgfpathmoveto{\pgfqpoint{0.000000in}{0.000000in}}%
\pgfpathlineto{\pgfqpoint{0.000000in}{-0.048611in}}%
\pgfusepath{stroke,fill}%
}%
\begin{pgfscope}%
\pgfsys@transformshift{0.921890in}{0.700520in}%
\pgfsys@useobject{currentmarker}{}%
\end{pgfscope}%
\end{pgfscope}%
\begin{pgfscope}%
\definecolor{textcolor}{rgb}{0.000000,0.000000,0.000000}%
\pgfsetstrokecolor{textcolor}%
\pgfsetfillcolor{textcolor}%
\pgftext[x=0.839841in, y=0.310396in, left, base,rotate=45.000000]{\color{textcolor}\sffamily\fontsize{10.000000}{12.000000}\selectfont 15\%}%
\end{pgfscope}%
\begin{pgfscope}%
\pgfsetbuttcap%
\pgfsetroundjoin%
\definecolor{currentfill}{rgb}{0.000000,0.000000,0.000000}%
\pgfsetfillcolor{currentfill}%
\pgfsetlinewidth{0.803000pt}%
\definecolor{currentstroke}{rgb}{0.000000,0.000000,0.000000}%
\pgfsetstrokecolor{currentstroke}%
\pgfsetdash{}{0pt}%
\pgfsys@defobject{currentmarker}{\pgfqpoint{0.000000in}{-0.048611in}}{\pgfqpoint{0.000000in}{0.000000in}}{%
\pgfpathmoveto{\pgfqpoint{0.000000in}{0.000000in}}%
\pgfpathlineto{\pgfqpoint{0.000000in}{-0.048611in}}%
\pgfusepath{stroke,fill}%
}%
\begin{pgfscope}%
\pgfsys@transformshift{2.049163in}{0.700520in}%
\pgfsys@useobject{currentmarker}{}%
\end{pgfscope}%
\end{pgfscope}%
\begin{pgfscope}%
\definecolor{textcolor}{rgb}{0.000000,0.000000,0.000000}%
\pgfsetstrokecolor{textcolor}%
\pgfsetfillcolor{textcolor}%
\pgftext[x=1.967114in, y=0.310396in, left, base,rotate=45.000000]{\color{textcolor}\sffamily\fontsize{10.000000}{12.000000}\selectfont 25\%}%
\end{pgfscope}%
\begin{pgfscope}%
\pgfsetbuttcap%
\pgfsetroundjoin%
\definecolor{currentfill}{rgb}{0.000000,0.000000,0.000000}%
\pgfsetfillcolor{currentfill}%
\pgfsetlinewidth{0.803000pt}%
\definecolor{currentstroke}{rgb}{0.000000,0.000000,0.000000}%
\pgfsetstrokecolor{currentstroke}%
\pgfsetdash{}{0pt}%
\pgfsys@defobject{currentmarker}{\pgfqpoint{0.000000in}{-0.048611in}}{\pgfqpoint{0.000000in}{0.000000in}}{%
\pgfpathmoveto{\pgfqpoint{0.000000in}{0.000000in}}%
\pgfpathlineto{\pgfqpoint{0.000000in}{-0.048611in}}%
\pgfusepath{stroke,fill}%
}%
\begin{pgfscope}%
\pgfsys@transformshift{3.176435in}{0.700520in}%
\pgfsys@useobject{currentmarker}{}%
\end{pgfscope}%
\end{pgfscope}%
\begin{pgfscope}%
\definecolor{textcolor}{rgb}{0.000000,0.000000,0.000000}%
\pgfsetstrokecolor{textcolor}%
\pgfsetfillcolor{textcolor}%
\pgftext[x=3.094387in, y=0.310396in, left, base,rotate=45.000000]{\color{textcolor}\sffamily\fontsize{10.000000}{12.000000}\selectfont 50\%}%
\end{pgfscope}%
\begin{pgfscope}%
\pgfsetbuttcap%
\pgfsetroundjoin%
\definecolor{currentfill}{rgb}{0.000000,0.000000,0.000000}%
\pgfsetfillcolor{currentfill}%
\pgfsetlinewidth{0.803000pt}%
\definecolor{currentstroke}{rgb}{0.000000,0.000000,0.000000}%
\pgfsetstrokecolor{currentstroke}%
\pgfsetdash{}{0pt}%
\pgfsys@defobject{currentmarker}{\pgfqpoint{0.000000in}{-0.048611in}}{\pgfqpoint{0.000000in}{0.000000in}}{%
\pgfpathmoveto{\pgfqpoint{0.000000in}{0.000000in}}%
\pgfpathlineto{\pgfqpoint{0.000000in}{-0.048611in}}%
\pgfusepath{stroke,fill}%
}%
\begin{pgfscope}%
\pgfsys@transformshift{4.303708in}{0.700520in}%
\pgfsys@useobject{currentmarker}{}%
\end{pgfscope}%
\end{pgfscope}%
\begin{pgfscope}%
\definecolor{textcolor}{rgb}{0.000000,0.000000,0.000000}%
\pgfsetstrokecolor{textcolor}%
\pgfsetfillcolor{textcolor}%
\pgftext[x=4.221659in, y=0.310396in, left, base,rotate=45.000000]{\color{textcolor}\sffamily\fontsize{10.000000}{12.000000}\selectfont 75\%}%
\end{pgfscope}%
\begin{pgfscope}%
\pgfsetbuttcap%
\pgfsetroundjoin%
\definecolor{currentfill}{rgb}{0.000000,0.000000,0.000000}%
\pgfsetfillcolor{currentfill}%
\pgfsetlinewidth{0.803000pt}%
\definecolor{currentstroke}{rgb}{0.000000,0.000000,0.000000}%
\pgfsetstrokecolor{currentstroke}%
\pgfsetdash{}{0pt}%
\pgfsys@defobject{currentmarker}{\pgfqpoint{0.000000in}{-0.048611in}}{\pgfqpoint{0.000000in}{0.000000in}}{%
\pgfpathmoveto{\pgfqpoint{0.000000in}{0.000000in}}%
\pgfpathlineto{\pgfqpoint{0.000000in}{-0.048611in}}%
\pgfusepath{stroke,fill}%
}%
\begin{pgfscope}%
\pgfsys@transformshift{5.430981in}{0.700520in}%
\pgfsys@useobject{currentmarker}{}%
\end{pgfscope}%
\end{pgfscope}%
\begin{pgfscope}%
\definecolor{textcolor}{rgb}{0.000000,0.000000,0.000000}%
\pgfsetstrokecolor{textcolor}%
\pgfsetfillcolor{textcolor}%
\pgftext[x=5.348932in, y=0.310396in, left, base,rotate=45.000000]{\color{textcolor}\sffamily\fontsize{10.000000}{12.000000}\selectfont 90\%}%
\end{pgfscope}%
\begin{pgfscope}%
\definecolor{textcolor}{rgb}{0.000000,0.000000,0.000000}%
\pgfsetstrokecolor{textcolor}%
\pgfsetfillcolor{textcolor}%
\pgftext[x=3.176435in,y=0.234413in,,top]{\color{textcolor}\sffamily\fontsize{10.000000}{12.000000}\selectfont Percentage of real data(Pix3D) per mini-batch}%
\end{pgfscope}%
\begin{pgfscope}%
\pgfsetbuttcap%
\pgfsetroundjoin%
\definecolor{currentfill}{rgb}{0.000000,0.000000,0.000000}%
\pgfsetfillcolor{currentfill}%
\pgfsetlinewidth{0.803000pt}%
\definecolor{currentstroke}{rgb}{0.000000,0.000000,0.000000}%
\pgfsetstrokecolor{currentstroke}%
\pgfsetdash{}{0pt}%
\pgfsys@defobject{currentmarker}{\pgfqpoint{-0.048611in}{0.000000in}}{\pgfqpoint{-0.000000in}{0.000000in}}{%
\pgfpathmoveto{\pgfqpoint{-0.000000in}{0.000000in}}%
\pgfpathlineto{\pgfqpoint{-0.048611in}{0.000000in}}%
\pgfusepath{stroke,fill}%
}%
\begin{pgfscope}%
\pgfsys@transformshift{0.696435in}{0.700520in}%
\pgfsys@useobject{currentmarker}{}%
\end{pgfscope}%
\end{pgfscope}%
\begin{pgfscope}%
\definecolor{textcolor}{rgb}{0.000000,0.000000,0.000000}%
\pgfsetstrokecolor{textcolor}%
\pgfsetfillcolor{textcolor}%
\pgftext[x=0.289968in, y=0.647759in, left, base]{\color{textcolor}\sffamily\fontsize{10.000000}{12.000000}\selectfont 0.28}%
\end{pgfscope}%
\begin{pgfscope}%
\pgfsetbuttcap%
\pgfsetroundjoin%
\definecolor{currentfill}{rgb}{0.000000,0.000000,0.000000}%
\pgfsetfillcolor{currentfill}%
\pgfsetlinewidth{0.803000pt}%
\definecolor{currentstroke}{rgb}{0.000000,0.000000,0.000000}%
\pgfsetstrokecolor{currentstroke}%
\pgfsetdash{}{0pt}%
\pgfsys@defobject{currentmarker}{\pgfqpoint{-0.048611in}{0.000000in}}{\pgfqpoint{-0.000000in}{0.000000in}}{%
\pgfpathmoveto{\pgfqpoint{-0.000000in}{0.000000in}}%
\pgfpathlineto{\pgfqpoint{-0.048611in}{0.000000in}}%
\pgfusepath{stroke,fill}%
}%
\begin{pgfscope}%
\pgfsys@transformshift{0.696435in}{1.439720in}%
\pgfsys@useobject{currentmarker}{}%
\end{pgfscope}%
\end{pgfscope}%
\begin{pgfscope}%
\definecolor{textcolor}{rgb}{0.000000,0.000000,0.000000}%
\pgfsetstrokecolor{textcolor}%
\pgfsetfillcolor{textcolor}%
\pgftext[x=0.289968in, y=1.386959in, left, base]{\color{textcolor}\sffamily\fontsize{10.000000}{12.000000}\selectfont 0.30}%
\end{pgfscope}%
\begin{pgfscope}%
\pgfsetbuttcap%
\pgfsetroundjoin%
\definecolor{currentfill}{rgb}{0.000000,0.000000,0.000000}%
\pgfsetfillcolor{currentfill}%
\pgfsetlinewidth{0.803000pt}%
\definecolor{currentstroke}{rgb}{0.000000,0.000000,0.000000}%
\pgfsetstrokecolor{currentstroke}%
\pgfsetdash{}{0pt}%
\pgfsys@defobject{currentmarker}{\pgfqpoint{-0.048611in}{0.000000in}}{\pgfqpoint{-0.000000in}{0.000000in}}{%
\pgfpathmoveto{\pgfqpoint{-0.000000in}{0.000000in}}%
\pgfpathlineto{\pgfqpoint{-0.048611in}{0.000000in}}%
\pgfusepath{stroke,fill}%
}%
\begin{pgfscope}%
\pgfsys@transformshift{0.696435in}{2.178920in}%
\pgfsys@useobject{currentmarker}{}%
\end{pgfscope}%
\end{pgfscope}%
\begin{pgfscope}%
\definecolor{textcolor}{rgb}{0.000000,0.000000,0.000000}%
\pgfsetstrokecolor{textcolor}%
\pgfsetfillcolor{textcolor}%
\pgftext[x=0.289968in, y=2.126159in, left, base]{\color{textcolor}\sffamily\fontsize{10.000000}{12.000000}\selectfont 0.32}%
\end{pgfscope}%
\begin{pgfscope}%
\pgfsetbuttcap%
\pgfsetroundjoin%
\definecolor{currentfill}{rgb}{0.000000,0.000000,0.000000}%
\pgfsetfillcolor{currentfill}%
\pgfsetlinewidth{0.803000pt}%
\definecolor{currentstroke}{rgb}{0.000000,0.000000,0.000000}%
\pgfsetstrokecolor{currentstroke}%
\pgfsetdash{}{0pt}%
\pgfsys@defobject{currentmarker}{\pgfqpoint{-0.048611in}{0.000000in}}{\pgfqpoint{-0.000000in}{0.000000in}}{%
\pgfpathmoveto{\pgfqpoint{-0.000000in}{0.000000in}}%
\pgfpathlineto{\pgfqpoint{-0.048611in}{0.000000in}}%
\pgfusepath{stroke,fill}%
}%
\begin{pgfscope}%
\pgfsys@transformshift{0.696435in}{2.918120in}%
\pgfsys@useobject{currentmarker}{}%
\end{pgfscope}%
\end{pgfscope}%
\begin{pgfscope}%
\definecolor{textcolor}{rgb}{0.000000,0.000000,0.000000}%
\pgfsetstrokecolor{textcolor}%
\pgfsetfillcolor{textcolor}%
\pgftext[x=0.289968in, y=2.865359in, left, base]{\color{textcolor}\sffamily\fontsize{10.000000}{12.000000}\selectfont 0.34}%
\end{pgfscope}%
\begin{pgfscope}%
\pgfsetbuttcap%
\pgfsetroundjoin%
\definecolor{currentfill}{rgb}{0.000000,0.000000,0.000000}%
\pgfsetfillcolor{currentfill}%
\pgfsetlinewidth{0.803000pt}%
\definecolor{currentstroke}{rgb}{0.000000,0.000000,0.000000}%
\pgfsetstrokecolor{currentstroke}%
\pgfsetdash{}{0pt}%
\pgfsys@defobject{currentmarker}{\pgfqpoint{-0.048611in}{0.000000in}}{\pgfqpoint{-0.000000in}{0.000000in}}{%
\pgfpathmoveto{\pgfqpoint{-0.000000in}{0.000000in}}%
\pgfpathlineto{\pgfqpoint{-0.048611in}{0.000000in}}%
\pgfusepath{stroke,fill}%
}%
\begin{pgfscope}%
\pgfsys@transformshift{0.696435in}{3.657320in}%
\pgfsys@useobject{currentmarker}{}%
\end{pgfscope}%
\end{pgfscope}%
\begin{pgfscope}%
\definecolor{textcolor}{rgb}{0.000000,0.000000,0.000000}%
\pgfsetstrokecolor{textcolor}%
\pgfsetfillcolor{textcolor}%
\pgftext[x=0.289968in, y=3.604559in, left, base]{\color{textcolor}\sffamily\fontsize{10.000000}{12.000000}\selectfont 0.36}%
\end{pgfscope}%
\begin{pgfscope}%
\pgfsetbuttcap%
\pgfsetroundjoin%
\definecolor{currentfill}{rgb}{0.000000,0.000000,0.000000}%
\pgfsetfillcolor{currentfill}%
\pgfsetlinewidth{0.803000pt}%
\definecolor{currentstroke}{rgb}{0.000000,0.000000,0.000000}%
\pgfsetstrokecolor{currentstroke}%
\pgfsetdash{}{0pt}%
\pgfsys@defobject{currentmarker}{\pgfqpoint{-0.048611in}{0.000000in}}{\pgfqpoint{-0.000000in}{0.000000in}}{%
\pgfpathmoveto{\pgfqpoint{-0.000000in}{0.000000in}}%
\pgfpathlineto{\pgfqpoint{-0.048611in}{0.000000in}}%
\pgfusepath{stroke,fill}%
}%
\begin{pgfscope}%
\pgfsys@transformshift{0.696435in}{4.396520in}%
\pgfsys@useobject{currentmarker}{}%
\end{pgfscope}%
\end{pgfscope}%
\begin{pgfscope}%
\definecolor{textcolor}{rgb}{0.000000,0.000000,0.000000}%
\pgfsetstrokecolor{textcolor}%
\pgfsetfillcolor{textcolor}%
\pgftext[x=0.289968in, y=4.343759in, left, base]{\color{textcolor}\sffamily\fontsize{10.000000}{12.000000}\selectfont 0.38}%
\end{pgfscope}%
\begin{pgfscope}%
\definecolor{textcolor}{rgb}{0.000000,0.000000,0.000000}%
\pgfsetstrokecolor{textcolor}%
\pgfsetfillcolor{textcolor}%
\pgftext[x=0.234413in,y=2.548520in,,bottom,rotate=90.000000]{\color{textcolor}\sffamily\fontsize{10.000000}{12.000000}\selectfont IoU}%
\end{pgfscope}%
\begin{pgfscope}%
\pgfpathrectangle{\pgfqpoint{0.696435in}{0.700520in}}{\pgfqpoint{4.960000in}{3.696000in}}%
\pgfusepath{clip}%
\pgfsetrectcap%
\pgfsetroundjoin%
\pgfsetlinewidth{1.505625pt}%
\definecolor{currentstroke}{rgb}{0.121569,0.466667,0.705882}%
\pgfsetstrokecolor{currentstroke}%
\pgfsetdash{}{0pt}%
\pgfpathmoveto{\pgfqpoint{0.921890in}{3.317288in}}%
\pgfpathlineto{\pgfqpoint{2.049163in}{3.683192in}}%
\pgfpathlineto{\pgfqpoint{3.176435in}{3.790376in}}%
\pgfpathlineto{\pgfqpoint{4.303708in}{3.513176in}}%
\pgfpathlineto{\pgfqpoint{5.430981in}{3.136184in}}%
\pgfusepath{stroke}%
\end{pgfscope}%
\begin{pgfscope}%
\pgfpathrectangle{\pgfqpoint{0.696435in}{0.700520in}}{\pgfqpoint{4.960000in}{3.696000in}}%
\pgfusepath{clip}%
\pgfsetbuttcap%
\pgfsetroundjoin%
\definecolor{currentfill}{rgb}{0.121569,0.466667,0.705882}%
\pgfsetfillcolor{currentfill}%
\pgfsetlinewidth{1.003750pt}%
\definecolor{currentstroke}{rgb}{0.121569,0.466667,0.705882}%
\pgfsetstrokecolor{currentstroke}%
\pgfsetdash{}{0pt}%
\pgfsys@defobject{currentmarker}{\pgfqpoint{-0.041667in}{-0.041667in}}{\pgfqpoint{0.041667in}{0.041667in}}{%
\pgfpathmoveto{\pgfqpoint{0.000000in}{-0.041667in}}%
\pgfpathcurveto{\pgfqpoint{0.011050in}{-0.041667in}}{\pgfqpoint{0.021649in}{-0.037276in}}{\pgfqpoint{0.029463in}{-0.029463in}}%
\pgfpathcurveto{\pgfqpoint{0.037276in}{-0.021649in}}{\pgfqpoint{0.041667in}{-0.011050in}}{\pgfqpoint{0.041667in}{0.000000in}}%
\pgfpathcurveto{\pgfqpoint{0.041667in}{0.011050in}}{\pgfqpoint{0.037276in}{0.021649in}}{\pgfqpoint{0.029463in}{0.029463in}}%
\pgfpathcurveto{\pgfqpoint{0.021649in}{0.037276in}}{\pgfqpoint{0.011050in}{0.041667in}}{\pgfqpoint{0.000000in}{0.041667in}}%
\pgfpathcurveto{\pgfqpoint{-0.011050in}{0.041667in}}{\pgfqpoint{-0.021649in}{0.037276in}}{\pgfqpoint{-0.029463in}{0.029463in}}%
\pgfpathcurveto{\pgfqpoint{-0.037276in}{0.021649in}}{\pgfqpoint{-0.041667in}{0.011050in}}{\pgfqpoint{-0.041667in}{0.000000in}}%
\pgfpathcurveto{\pgfqpoint{-0.041667in}{-0.011050in}}{\pgfqpoint{-0.037276in}{-0.021649in}}{\pgfqpoint{-0.029463in}{-0.029463in}}%
\pgfpathcurveto{\pgfqpoint{-0.021649in}{-0.037276in}}{\pgfqpoint{-0.011050in}{-0.041667in}}{\pgfqpoint{0.000000in}{-0.041667in}}%
\pgfpathclose%
\pgfusepath{stroke,fill}%
}%
\begin{pgfscope}%
\pgfsys@transformshift{0.921890in}{3.317288in}%
\pgfsys@useobject{currentmarker}{}%
\end{pgfscope}%
\begin{pgfscope}%
\pgfsys@transformshift{2.049163in}{3.683192in}%
\pgfsys@useobject{currentmarker}{}%
\end{pgfscope}%
\begin{pgfscope}%
\pgfsys@transformshift{3.176435in}{3.790376in}%
\pgfsys@useobject{currentmarker}{}%
\end{pgfscope}%
\begin{pgfscope}%
\pgfsys@transformshift{4.303708in}{3.513176in}%
\pgfsys@useobject{currentmarker}{}%
\end{pgfscope}%
\begin{pgfscope}%
\pgfsys@transformshift{5.430981in}{3.136184in}%
\pgfsys@useobject{currentmarker}{}%
\end{pgfscope}%
\end{pgfscope}%
\begin{pgfscope}%
\pgfpathrectangle{\pgfqpoint{0.696435in}{0.700520in}}{\pgfqpoint{4.960000in}{3.696000in}}%
\pgfusepath{clip}%
\pgfsetrectcap%
\pgfsetroundjoin%
\pgfsetlinewidth{1.505625pt}%
\definecolor{currentstroke}{rgb}{1.000000,0.498039,0.054902}%
\pgfsetstrokecolor{currentstroke}%
\pgfsetdash{}{0pt}%
\pgfpathmoveto{\pgfqpoint{0.921890in}{1.934984in}}%
\pgfpathlineto{\pgfqpoint{2.049163in}{3.494696in}}%
\pgfpathlineto{\pgfqpoint{3.176435in}{3.609272in}}%
\pgfpathlineto{\pgfqpoint{4.303708in}{3.265544in}}%
\pgfpathlineto{\pgfqpoint{5.430981in}{3.221192in}}%
\pgfusepath{stroke}%
\end{pgfscope}%
\begin{pgfscope}%
\pgfpathrectangle{\pgfqpoint{0.696435in}{0.700520in}}{\pgfqpoint{4.960000in}{3.696000in}}%
\pgfusepath{clip}%
\pgfsetbuttcap%
\pgfsetmiterjoin%
\definecolor{currentfill}{rgb}{1.000000,0.498039,0.054902}%
\pgfsetfillcolor{currentfill}%
\pgfsetlinewidth{1.003750pt}%
\definecolor{currentstroke}{rgb}{1.000000,0.498039,0.054902}%
\pgfsetstrokecolor{currentstroke}%
\pgfsetdash{}{0pt}%
\pgfsys@defobject{currentmarker}{\pgfqpoint{-0.041667in}{-0.041667in}}{\pgfqpoint{0.041667in}{0.041667in}}{%
\pgfpathmoveto{\pgfqpoint{-0.000000in}{-0.041667in}}%
\pgfpathlineto{\pgfqpoint{0.041667in}{0.041667in}}%
\pgfpathlineto{\pgfqpoint{-0.041667in}{0.041667in}}%
\pgfpathclose%
\pgfusepath{stroke,fill}%
}%
\begin{pgfscope}%
\pgfsys@transformshift{0.921890in}{1.934984in}%
\pgfsys@useobject{currentmarker}{}%
\end{pgfscope}%
\begin{pgfscope}%
\pgfsys@transformshift{2.049163in}{3.494696in}%
\pgfsys@useobject{currentmarker}{}%
\end{pgfscope}%
\begin{pgfscope}%
\pgfsys@transformshift{3.176435in}{3.609272in}%
\pgfsys@useobject{currentmarker}{}%
\end{pgfscope}%
\begin{pgfscope}%
\pgfsys@transformshift{4.303708in}{3.265544in}%
\pgfsys@useobject{currentmarker}{}%
\end{pgfscope}%
\begin{pgfscope}%
\pgfsys@transformshift{5.430981in}{3.221192in}%
\pgfsys@useobject{currentmarker}{}%
\end{pgfscope}%
\end{pgfscope}%
\begin{pgfscope}%
\pgfsetrectcap%
\pgfsetmiterjoin%
\pgfsetlinewidth{0.803000pt}%
\definecolor{currentstroke}{rgb}{0.000000,0.000000,0.000000}%
\pgfsetstrokecolor{currentstroke}%
\pgfsetdash{}{0pt}%
\pgfpathmoveto{\pgfqpoint{0.696435in}{0.700520in}}%
\pgfpathlineto{\pgfqpoint{0.696435in}{4.396520in}}%
\pgfusepath{stroke}%
\end{pgfscope}%
\begin{pgfscope}%
\pgfsetrectcap%
\pgfsetmiterjoin%
\pgfsetlinewidth{0.803000pt}%
\definecolor{currentstroke}{rgb}{0.000000,0.000000,0.000000}%
\pgfsetstrokecolor{currentstroke}%
\pgfsetdash{}{0pt}%
\pgfpathmoveto{\pgfqpoint{5.656435in}{0.700520in}}%
\pgfpathlineto{\pgfqpoint{5.656435in}{4.396520in}}%
\pgfusepath{stroke}%
\end{pgfscope}%
\begin{pgfscope}%
\pgfsetrectcap%
\pgfsetmiterjoin%
\pgfsetlinewidth{0.803000pt}%
\definecolor{currentstroke}{rgb}{0.000000,0.000000,0.000000}%
\pgfsetstrokecolor{currentstroke}%
\pgfsetdash{}{0pt}%
\pgfpathmoveto{\pgfqpoint{0.696435in}{0.700520in}}%
\pgfpathlineto{\pgfqpoint{5.656435in}{0.700520in}}%
\pgfusepath{stroke}%
\end{pgfscope}%
\begin{pgfscope}%
\pgfsetrectcap%
\pgfsetmiterjoin%
\pgfsetlinewidth{0.803000pt}%
\definecolor{currentstroke}{rgb}{0.000000,0.000000,0.000000}%
\pgfsetstrokecolor{currentstroke}%
\pgfsetdash{}{0pt}%
\pgfpathmoveto{\pgfqpoint{0.696435in}{4.396520in}}%
\pgfpathlineto{\pgfqpoint{5.656435in}{4.396520in}}%
\pgfusepath{stroke}%
\end{pgfscope}%
\begin{pgfscope}%
\pgfsetbuttcap%
\pgfsetmiterjoin%
\definecolor{currentfill}{rgb}{1.000000,1.000000,1.000000}%
\pgfsetfillcolor{currentfill}%
\pgfsetfillopacity{0.800000}%
\pgfsetlinewidth{1.003750pt}%
\definecolor{currentstroke}{rgb}{0.800000,0.800000,0.800000}%
\pgfsetstrokecolor{currentstroke}%
\pgfsetstrokeopacity{0.800000}%
\pgfsetdash{}{0pt}%
\pgfpathmoveto{\pgfqpoint{3.900373in}{3.877695in}}%
\pgfpathlineto{\pgfqpoint{5.559213in}{3.877695in}}%
\pgfpathquadraticcurveto{\pgfqpoint{5.586991in}{3.877695in}}{\pgfqpoint{5.586991in}{3.905472in}}%
\pgfpathlineto{\pgfqpoint{5.586991in}{4.299298in}}%
\pgfpathquadraticcurveto{\pgfqpoint{5.586991in}{4.327076in}}{\pgfqpoint{5.559213in}{4.327076in}}%
\pgfpathlineto{\pgfqpoint{3.900373in}{4.327076in}}%
\pgfpathquadraticcurveto{\pgfqpoint{3.872595in}{4.327076in}}{\pgfqpoint{3.872595in}{4.299298in}}%
\pgfpathlineto{\pgfqpoint{3.872595in}{3.905472in}}%
\pgfpathquadraticcurveto{\pgfqpoint{3.872595in}{3.877695in}}{\pgfqpoint{3.900373in}{3.877695in}}%
\pgfpathclose%
\pgfusepath{stroke,fill}%
\end{pgfscope}%
\begin{pgfscope}%
\pgfsetrectcap%
\pgfsetroundjoin%
\pgfsetlinewidth{1.505625pt}%
\definecolor{currentstroke}{rgb}{0.121569,0.466667,0.705882}%
\pgfsetstrokecolor{currentstroke}%
\pgfsetdash{}{0pt}%
\pgfpathmoveto{\pgfqpoint{3.928150in}{4.214608in}}%
\pgfpathlineto{\pgfqpoint{4.205928in}{4.214608in}}%
\pgfusepath{stroke}%
\end{pgfscope}%
\begin{pgfscope}%
\pgfsetbuttcap%
\pgfsetroundjoin%
\definecolor{currentfill}{rgb}{0.121569,0.466667,0.705882}%
\pgfsetfillcolor{currentfill}%
\pgfsetlinewidth{1.003750pt}%
\definecolor{currentstroke}{rgb}{0.121569,0.466667,0.705882}%
\pgfsetstrokecolor{currentstroke}%
\pgfsetdash{}{0pt}%
\pgfsys@defobject{currentmarker}{\pgfqpoint{-0.041667in}{-0.041667in}}{\pgfqpoint{0.041667in}{0.041667in}}{%
\pgfpathmoveto{\pgfqpoint{0.000000in}{-0.041667in}}%
\pgfpathcurveto{\pgfqpoint{0.011050in}{-0.041667in}}{\pgfqpoint{0.021649in}{-0.037276in}}{\pgfqpoint{0.029463in}{-0.029463in}}%
\pgfpathcurveto{\pgfqpoint{0.037276in}{-0.021649in}}{\pgfqpoint{0.041667in}{-0.011050in}}{\pgfqpoint{0.041667in}{0.000000in}}%
\pgfpathcurveto{\pgfqpoint{0.041667in}{0.011050in}}{\pgfqpoint{0.037276in}{0.021649in}}{\pgfqpoint{0.029463in}{0.029463in}}%
\pgfpathcurveto{\pgfqpoint{0.021649in}{0.037276in}}{\pgfqpoint{0.011050in}{0.041667in}}{\pgfqpoint{0.000000in}{0.041667in}}%
\pgfpathcurveto{\pgfqpoint{-0.011050in}{0.041667in}}{\pgfqpoint{-0.021649in}{0.037276in}}{\pgfqpoint{-0.029463in}{0.029463in}}%
\pgfpathcurveto{\pgfqpoint{-0.037276in}{0.021649in}}{\pgfqpoint{-0.041667in}{0.011050in}}{\pgfqpoint{-0.041667in}{0.000000in}}%
\pgfpathcurveto{\pgfqpoint{-0.041667in}{-0.011050in}}{\pgfqpoint{-0.037276in}{-0.021649in}}{\pgfqpoint{-0.029463in}{-0.029463in}}%
\pgfpathcurveto{\pgfqpoint{-0.021649in}{-0.037276in}}{\pgfqpoint{-0.011050in}{-0.041667in}}{\pgfqpoint{0.000000in}{-0.041667in}}%
\pgfpathclose%
\pgfusepath{stroke,fill}%
}%
\begin{pgfscope}%
\pgfsys@transformshift{4.067039in}{4.214608in}%
\pgfsys@useobject{currentmarker}{}%
\end{pgfscope}%
\end{pgfscope}%
\begin{pgfscope}%
\definecolor{textcolor}{rgb}{0.000000,0.000000,0.000000}%
\pgfsetstrokecolor{textcolor}%
\pgfsetfillcolor{textcolor}%
\pgftext[x=4.317039in,y=4.165997in,left,base]{\color{textcolor}\sffamily\fontsize{10.000000}{12.000000}\selectfont V1 on Pix2Vox++}%
\end{pgfscope}%
\begin{pgfscope}%
\pgfsetrectcap%
\pgfsetroundjoin%
\pgfsetlinewidth{1.505625pt}%
\definecolor{currentstroke}{rgb}{1.000000,0.498039,0.054902}%
\pgfsetstrokecolor{currentstroke}%
\pgfsetdash{}{0pt}%
\pgfpathmoveto{\pgfqpoint{3.928150in}{4.010751in}}%
\pgfpathlineto{\pgfqpoint{4.205928in}{4.010751in}}%
\pgfusepath{stroke}%
\end{pgfscope}%
\begin{pgfscope}%
\pgfsetbuttcap%
\pgfsetmiterjoin%
\definecolor{currentfill}{rgb}{1.000000,0.498039,0.054902}%
\pgfsetfillcolor{currentfill}%
\pgfsetlinewidth{1.003750pt}%
\definecolor{currentstroke}{rgb}{1.000000,0.498039,0.054902}%
\pgfsetstrokecolor{currentstroke}%
\pgfsetdash{}{0pt}%
\pgfsys@defobject{currentmarker}{\pgfqpoint{-0.041667in}{-0.041667in}}{\pgfqpoint{0.041667in}{0.041667in}}{%
\pgfpathmoveto{\pgfqpoint{-0.000000in}{-0.041667in}}%
\pgfpathlineto{\pgfqpoint{0.041667in}{0.041667in}}%
\pgfpathlineto{\pgfqpoint{-0.041667in}{0.041667in}}%
\pgfpathclose%
\pgfusepath{stroke,fill}%
}%
\begin{pgfscope}%
\pgfsys@transformshift{4.067039in}{4.010751in}%
\pgfsys@useobject{currentmarker}{}%
\end{pgfscope}%
\end{pgfscope}%
\begin{pgfscope}%
\definecolor{textcolor}{rgb}{0.000000,0.000000,0.000000}%
\pgfsetstrokecolor{textcolor}%
\pgfsetfillcolor{textcolor}%
\pgftext[x=4.317039in,y=3.962140in,left,base]{\color{textcolor}\sffamily\fontsize{10.000000}{12.000000}\selectfont V2 on Pix2Vox++}%
\end{pgfscope}%
\end{pgfpicture}%
\makeatother%
\endgroup%
}
%    \resizebox{0.49\linewidth}{!}{\input{/Users/apple/OVGU/Thesis/code/3dReconstruction/report/images/evaluation/performance/mixed_linegraph2.pgf}}
%    \caption{Line plot for the \gls{iou}  for baselines trained on different ratios of synthetic and real dataset.
%        (Left)Mixed training on Pix2Vox++, (right)Mixed training on Pix2Vox. In both cases we see a slight increase in \gls{iou} with addition of real data, and a gradual decrease till it reaches 100\% real data}
%    \label{fig:mixed1}
%\end{figure}

\section{Ablation Study on Chairs}\label{sec:ablation-study-on-chairs}
In this section, we conduct ablation study on chairs by changing domain randomization property.
The dataset used for the study is explained in \autoref{subsec:s2r:3dfree-ablation}, samples of which can be found in \autoref{fig:domain_randomisation_for_ablation_study}.

\subsection{Domain Randomization on Chair Dataset}\label{subsec:domain-randomisation-on-chair-dataset}
For comparison with the real dataset, we extract only the chair models from Pix3D and compare the results  with and without 2D augmentation on the synthetic dataset.
In \autoref{fig:ablation1}, we see that the textureless chair dataset outperforms the rest of the dataset.
The hypothesis is that the more domain randomization, the better the performance.

\begin{figure}[ht]
    \centering
    \resizebox{0.9\textwidth}{!}{%% Creator: Matplotlib, PGF backend
%%
%% To include the figure in your LaTeX document, write
%%   \input{<filename>.pgf}
%%
%% Make sure the required packages are loaded in your preamble
%%   \usepackage{pgf}
%%
%% Figures using additional raster images can only be included by \input if
%% they are in the same directory as the main LaTeX file. For loading figures
%% from other directories you can use the `import` package
%%   \usepackage{import}
%%
%% and then include the figures with
%%   \import{<path to file>}{<filename>.pgf}
%%
%% Matplotlib used the following preamble
%%   \usepackage{fontspec}
%%   \setmainfont{DejaVuSerif.ttf}[Path=\detokenize{/Users/apple/opt/anaconda3/envs/kaolin/lib/python3.7/site-packages/matplotlib/mpl-data/fonts/ttf/}]
%%   \setsansfont{DejaVuSans.ttf}[Path=\detokenize{/Users/apple/opt/anaconda3/envs/kaolin/lib/python3.7/site-packages/matplotlib/mpl-data/fonts/ttf/}]
%%   \setmonofont{DejaVuSansMono.ttf}[Path=\detokenize{/Users/apple/opt/anaconda3/envs/kaolin/lib/python3.7/site-packages/matplotlib/mpl-data/fonts/ttf/}]
%%
\begingroup%
\makeatletter%
\begin{pgfpicture}%
\pgfpathrectangle{\pgfpointorigin}{\pgfqpoint{6.298022in}{4.704781in}}%
\pgfusepath{use as bounding box, clip}%
\begin{pgfscope}%
\pgfsetbuttcap%
\pgfsetmiterjoin%
\definecolor{currentfill}{rgb}{1.000000,1.000000,1.000000}%
\pgfsetfillcolor{currentfill}%
\pgfsetlinewidth{0.000000pt}%
\definecolor{currentstroke}{rgb}{1.000000,1.000000,1.000000}%
\pgfsetstrokecolor{currentstroke}%
\pgfsetdash{}{0pt}%
\pgfpathmoveto{\pgfqpoint{0.000000in}{0.000000in}}%
\pgfpathlineto{\pgfqpoint{6.298022in}{0.000000in}}%
\pgfpathlineto{\pgfqpoint{6.298022in}{4.704781in}}%
\pgfpathlineto{\pgfqpoint{0.000000in}{4.704781in}}%
\pgfpathclose%
\pgfusepath{fill}%
\end{pgfscope}%
\begin{pgfscope}%
\pgfsetbuttcap%
\pgfsetmiterjoin%
\definecolor{currentfill}{rgb}{1.000000,1.000000,1.000000}%
\pgfsetfillcolor{currentfill}%
\pgfsetlinewidth{0.000000pt}%
\definecolor{currentstroke}{rgb}{0.000000,0.000000,0.000000}%
\pgfsetstrokecolor{currentstroke}%
\pgfsetstrokeopacity{0.000000}%
\pgfsetdash{}{0pt}%
\pgfpathmoveto{\pgfqpoint{0.812050in}{1.704057in}}%
\pgfpathlineto{\pgfqpoint{6.198022in}{1.704057in}}%
\pgfpathlineto{\pgfqpoint{6.198022in}{4.373716in}}%
\pgfpathlineto{\pgfqpoint{0.812050in}{4.373716in}}%
\pgfpathclose%
\pgfusepath{fill}%
\end{pgfscope}%
\begin{pgfscope}%
\pgfpathrectangle{\pgfqpoint{0.812050in}{1.704057in}}{\pgfqpoint{5.385972in}{2.669658in}}%
\pgfusepath{clip}%
\pgfsetbuttcap%
\pgfsetmiterjoin%
\definecolor{currentfill}{rgb}{0.121569,0.466667,0.705882}%
\pgfsetfillcolor{currentfill}%
\pgfsetlinewidth{0.000000pt}%
\definecolor{currentstroke}{rgb}{0.000000,0.000000,0.000000}%
\pgfsetstrokecolor{currentstroke}%
\pgfsetstrokeopacity{0.000000}%
\pgfsetdash{}{0pt}%
\pgfpathmoveto{\pgfqpoint{1.056867in}{1.704057in}}%
\pgfpathlineto{\pgfqpoint{1.335772in}{1.704057in}}%
\pgfpathlineto{\pgfqpoint{1.335772in}{3.570816in}}%
\pgfpathlineto{\pgfqpoint{1.056867in}{3.570816in}}%
\pgfpathclose%
\pgfusepath{fill}%
\end{pgfscope}%
\begin{pgfscope}%
\pgfpathrectangle{\pgfqpoint{0.812050in}{1.704057in}}{\pgfqpoint{5.385972in}{2.669658in}}%
\pgfusepath{clip}%
\pgfsetbuttcap%
\pgfsetmiterjoin%
\definecolor{currentfill}{rgb}{0.121569,0.466667,0.705882}%
\pgfsetfillcolor{currentfill}%
\pgfsetlinewidth{0.000000pt}%
\definecolor{currentstroke}{rgb}{0.000000,0.000000,0.000000}%
\pgfsetstrokecolor{currentstroke}%
\pgfsetstrokeopacity{0.000000}%
\pgfsetdash{}{0pt}%
\pgfpathmoveto{\pgfqpoint{1.676656in}{1.704057in}}%
\pgfpathlineto{\pgfqpoint{1.955562in}{1.704057in}}%
\pgfpathlineto{\pgfqpoint{1.955562in}{3.909863in}}%
\pgfpathlineto{\pgfqpoint{1.676656in}{3.909863in}}%
\pgfpathclose%
\pgfusepath{fill}%
\end{pgfscope}%
\begin{pgfscope}%
\pgfpathrectangle{\pgfqpoint{0.812050in}{1.704057in}}{\pgfqpoint{5.385972in}{2.669658in}}%
\pgfusepath{clip}%
\pgfsetbuttcap%
\pgfsetmiterjoin%
\definecolor{currentfill}{rgb}{0.121569,0.466667,0.705882}%
\pgfsetfillcolor{currentfill}%
\pgfsetlinewidth{0.000000pt}%
\definecolor{currentstroke}{rgb}{0.000000,0.000000,0.000000}%
\pgfsetstrokecolor{currentstroke}%
\pgfsetstrokeopacity{0.000000}%
\pgfsetdash{}{0pt}%
\pgfpathmoveto{\pgfqpoint{2.296446in}{1.704057in}}%
\pgfpathlineto{\pgfqpoint{2.575351in}{1.704057in}}%
\pgfpathlineto{\pgfqpoint{2.575351in}{2.904069in}}%
\pgfpathlineto{\pgfqpoint{2.296446in}{2.904069in}}%
\pgfpathclose%
\pgfusepath{fill}%
\end{pgfscope}%
\begin{pgfscope}%
\pgfpathrectangle{\pgfqpoint{0.812050in}{1.704057in}}{\pgfqpoint{5.385972in}{2.669658in}}%
\pgfusepath{clip}%
\pgfsetbuttcap%
\pgfsetmiterjoin%
\definecolor{currentfill}{rgb}{0.121569,0.466667,0.705882}%
\pgfsetfillcolor{currentfill}%
\pgfsetlinewidth{0.000000pt}%
\definecolor{currentstroke}{rgb}{0.000000,0.000000,0.000000}%
\pgfsetstrokecolor{currentstroke}%
\pgfsetstrokeopacity{0.000000}%
\pgfsetdash{}{0pt}%
\pgfpathmoveto{\pgfqpoint{2.916236in}{1.704057in}}%
\pgfpathlineto{\pgfqpoint{3.195141in}{1.704057in}}%
\pgfpathlineto{\pgfqpoint{3.195141in}{2.658460in}}%
\pgfpathlineto{\pgfqpoint{2.916236in}{2.658460in}}%
\pgfpathclose%
\pgfusepath{fill}%
\end{pgfscope}%
\begin{pgfscope}%
\pgfpathrectangle{\pgfqpoint{0.812050in}{1.704057in}}{\pgfqpoint{5.385972in}{2.669658in}}%
\pgfusepath{clip}%
\pgfsetbuttcap%
\pgfsetmiterjoin%
\definecolor{currentfill}{rgb}{0.121569,0.466667,0.705882}%
\pgfsetfillcolor{currentfill}%
\pgfsetlinewidth{0.000000pt}%
\definecolor{currentstroke}{rgb}{0.000000,0.000000,0.000000}%
\pgfsetstrokecolor{currentstroke}%
\pgfsetstrokeopacity{0.000000}%
\pgfsetdash{}{0pt}%
\pgfpathmoveto{\pgfqpoint{3.536025in}{1.704057in}}%
\pgfpathlineto{\pgfqpoint{3.814931in}{1.704057in}}%
\pgfpathlineto{\pgfqpoint{3.814931in}{3.183716in}}%
\pgfpathlineto{\pgfqpoint{3.536025in}{3.183716in}}%
\pgfpathclose%
\pgfusepath{fill}%
\end{pgfscope}%
\begin{pgfscope}%
\pgfpathrectangle{\pgfqpoint{0.812050in}{1.704057in}}{\pgfqpoint{5.385972in}{2.669658in}}%
\pgfusepath{clip}%
\pgfsetbuttcap%
\pgfsetmiterjoin%
\definecolor{currentfill}{rgb}{0.121569,0.466667,0.705882}%
\pgfsetfillcolor{currentfill}%
\pgfsetlinewidth{0.000000pt}%
\definecolor{currentstroke}{rgb}{0.000000,0.000000,0.000000}%
\pgfsetstrokecolor{currentstroke}%
\pgfsetstrokeopacity{0.000000}%
\pgfsetdash{}{0pt}%
\pgfpathmoveto{\pgfqpoint{4.155815in}{1.704057in}}%
\pgfpathlineto{\pgfqpoint{4.434720in}{1.704057in}}%
\pgfpathlineto{\pgfqpoint{4.434720in}{3.186385in}}%
\pgfpathlineto{\pgfqpoint{4.155815in}{3.186385in}}%
\pgfpathclose%
\pgfusepath{fill}%
\end{pgfscope}%
\begin{pgfscope}%
\pgfpathrectangle{\pgfqpoint{0.812050in}{1.704057in}}{\pgfqpoint{5.385972in}{2.669658in}}%
\pgfusepath{clip}%
\pgfsetbuttcap%
\pgfsetmiterjoin%
\definecolor{currentfill}{rgb}{0.121569,0.466667,0.705882}%
\pgfsetfillcolor{currentfill}%
\pgfsetlinewidth{0.000000pt}%
\definecolor{currentstroke}{rgb}{0.000000,0.000000,0.000000}%
\pgfsetstrokecolor{currentstroke}%
\pgfsetstrokeopacity{0.000000}%
\pgfsetdash{}{0pt}%
\pgfpathmoveto{\pgfqpoint{4.775605in}{1.704057in}}%
\pgfpathlineto{\pgfqpoint{5.054510in}{1.704057in}}%
\pgfpathlineto{\pgfqpoint{5.054510in}{3.287165in}}%
\pgfpathlineto{\pgfqpoint{4.775605in}{3.287165in}}%
\pgfpathclose%
\pgfusepath{fill}%
\end{pgfscope}%
\begin{pgfscope}%
\pgfpathrectangle{\pgfqpoint{0.812050in}{1.704057in}}{\pgfqpoint{5.385972in}{2.669658in}}%
\pgfusepath{clip}%
\pgfsetbuttcap%
\pgfsetmiterjoin%
\definecolor{currentfill}{rgb}{0.121569,0.466667,0.705882}%
\pgfsetfillcolor{currentfill}%
\pgfsetlinewidth{0.000000pt}%
\definecolor{currentstroke}{rgb}{0.000000,0.000000,0.000000}%
\pgfsetstrokecolor{currentstroke}%
\pgfsetstrokeopacity{0.000000}%
\pgfsetdash{}{0pt}%
\pgfpathmoveto{\pgfqpoint{5.395394in}{1.704057in}}%
\pgfpathlineto{\pgfqpoint{5.674300in}{1.704057in}}%
\pgfpathlineto{\pgfqpoint{5.674300in}{3.370592in}}%
\pgfpathlineto{\pgfqpoint{5.395394in}{3.370592in}}%
\pgfpathclose%
\pgfusepath{fill}%
\end{pgfscope}%
\begin{pgfscope}%
\pgfpathrectangle{\pgfqpoint{0.812050in}{1.704057in}}{\pgfqpoint{5.385972in}{2.669658in}}%
\pgfusepath{clip}%
\pgfsetbuttcap%
\pgfsetmiterjoin%
\definecolor{currentfill}{rgb}{1.000000,0.498039,0.054902}%
\pgfsetfillcolor{currentfill}%
\pgfsetlinewidth{0.000000pt}%
\definecolor{currentstroke}{rgb}{0.000000,0.000000,0.000000}%
\pgfsetstrokecolor{currentstroke}%
\pgfsetstrokeopacity{0.000000}%
\pgfsetdash{}{0pt}%
\pgfpathmoveto{\pgfqpoint{1.335772in}{1.704057in}}%
\pgfpathlineto{\pgfqpoint{1.614677in}{1.704057in}}%
\pgfpathlineto{\pgfqpoint{1.614677in}{3.502072in}}%
\pgfpathlineto{\pgfqpoint{1.335772in}{3.502072in}}%
\pgfpathclose%
\pgfusepath{fill}%
\end{pgfscope}%
\begin{pgfscope}%
\pgfpathrectangle{\pgfqpoint{0.812050in}{1.704057in}}{\pgfqpoint{5.385972in}{2.669658in}}%
\pgfusepath{clip}%
\pgfsetbuttcap%
\pgfsetmiterjoin%
\definecolor{currentfill}{rgb}{1.000000,0.498039,0.054902}%
\pgfsetfillcolor{currentfill}%
\pgfsetlinewidth{0.000000pt}%
\definecolor{currentstroke}{rgb}{0.000000,0.000000,0.000000}%
\pgfsetstrokecolor{currentstroke}%
\pgfsetstrokeopacity{0.000000}%
\pgfsetdash{}{0pt}%
\pgfpathmoveto{\pgfqpoint{1.955562in}{1.704057in}}%
\pgfpathlineto{\pgfqpoint{2.234467in}{1.704057in}}%
\pgfpathlineto{\pgfqpoint{2.234467in}{3.650238in}}%
\pgfpathlineto{\pgfqpoint{1.955562in}{3.650238in}}%
\pgfpathclose%
\pgfusepath{fill}%
\end{pgfscope}%
\begin{pgfscope}%
\pgfpathrectangle{\pgfqpoint{0.812050in}{1.704057in}}{\pgfqpoint{5.385972in}{2.669658in}}%
\pgfusepath{clip}%
\pgfsetbuttcap%
\pgfsetmiterjoin%
\definecolor{currentfill}{rgb}{1.000000,0.498039,0.054902}%
\pgfsetfillcolor{currentfill}%
\pgfsetlinewidth{0.000000pt}%
\definecolor{currentstroke}{rgb}{0.000000,0.000000,0.000000}%
\pgfsetstrokecolor{currentstroke}%
\pgfsetstrokeopacity{0.000000}%
\pgfsetdash{}{0pt}%
\pgfpathmoveto{\pgfqpoint{2.575351in}{1.704057in}}%
\pgfpathlineto{\pgfqpoint{2.854257in}{1.704057in}}%
\pgfpathlineto{\pgfqpoint{2.854257in}{2.203284in}}%
\pgfpathlineto{\pgfqpoint{2.575351in}{2.203284in}}%
\pgfpathclose%
\pgfusepath{fill}%
\end{pgfscope}%
\begin{pgfscope}%
\pgfpathrectangle{\pgfqpoint{0.812050in}{1.704057in}}{\pgfqpoint{5.385972in}{2.669658in}}%
\pgfusepath{clip}%
\pgfsetbuttcap%
\pgfsetmiterjoin%
\definecolor{currentfill}{rgb}{1.000000,0.498039,0.054902}%
\pgfsetfillcolor{currentfill}%
\pgfsetlinewidth{0.000000pt}%
\definecolor{currentstroke}{rgb}{0.000000,0.000000,0.000000}%
\pgfsetstrokecolor{currentstroke}%
\pgfsetstrokeopacity{0.000000}%
\pgfsetdash{}{0pt}%
\pgfpathmoveto{\pgfqpoint{3.195141in}{1.704057in}}%
\pgfpathlineto{\pgfqpoint{3.474046in}{1.704057in}}%
\pgfpathlineto{\pgfqpoint{3.474046in}{2.388825in}}%
\pgfpathlineto{\pgfqpoint{3.195141in}{2.388825in}}%
\pgfpathclose%
\pgfusepath{fill}%
\end{pgfscope}%
\begin{pgfscope}%
\pgfpathrectangle{\pgfqpoint{0.812050in}{1.704057in}}{\pgfqpoint{5.385972in}{2.669658in}}%
\pgfusepath{clip}%
\pgfsetbuttcap%
\pgfsetmiterjoin%
\definecolor{currentfill}{rgb}{1.000000,0.498039,0.054902}%
\pgfsetfillcolor{currentfill}%
\pgfsetlinewidth{0.000000pt}%
\definecolor{currentstroke}{rgb}{0.000000,0.000000,0.000000}%
\pgfsetstrokecolor{currentstroke}%
\pgfsetstrokeopacity{0.000000}%
\pgfsetdash{}{0pt}%
\pgfpathmoveto{\pgfqpoint{3.814931in}{1.704057in}}%
\pgfpathlineto{\pgfqpoint{4.093836in}{1.704057in}}%
\pgfpathlineto{\pgfqpoint{4.093836in}{2.451562in}}%
\pgfpathlineto{\pgfqpoint{3.814931in}{2.451562in}}%
\pgfpathclose%
\pgfusepath{fill}%
\end{pgfscope}%
\begin{pgfscope}%
\pgfpathrectangle{\pgfqpoint{0.812050in}{1.704057in}}{\pgfqpoint{5.385972in}{2.669658in}}%
\pgfusepath{clip}%
\pgfsetbuttcap%
\pgfsetmiterjoin%
\definecolor{currentfill}{rgb}{1.000000,0.498039,0.054902}%
\pgfsetfillcolor{currentfill}%
\pgfsetlinewidth{0.000000pt}%
\definecolor{currentstroke}{rgb}{0.000000,0.000000,0.000000}%
\pgfsetstrokecolor{currentstroke}%
\pgfsetstrokeopacity{0.000000}%
\pgfsetdash{}{0pt}%
\pgfpathmoveto{\pgfqpoint{4.434720in}{1.704057in}}%
\pgfpathlineto{\pgfqpoint{4.713626in}{1.704057in}}%
\pgfpathlineto{\pgfqpoint{4.713626in}{2.365465in}}%
\pgfpathlineto{\pgfqpoint{4.434720in}{2.365465in}}%
\pgfpathclose%
\pgfusepath{fill}%
\end{pgfscope}%
\begin{pgfscope}%
\pgfpathrectangle{\pgfqpoint{0.812050in}{1.704057in}}{\pgfqpoint{5.385972in}{2.669658in}}%
\pgfusepath{clip}%
\pgfsetbuttcap%
\pgfsetmiterjoin%
\definecolor{currentfill}{rgb}{1.000000,0.498039,0.054902}%
\pgfsetfillcolor{currentfill}%
\pgfsetlinewidth{0.000000pt}%
\definecolor{currentstroke}{rgb}{0.000000,0.000000,0.000000}%
\pgfsetstrokecolor{currentstroke}%
\pgfsetstrokeopacity{0.000000}%
\pgfsetdash{}{0pt}%
\pgfpathmoveto{\pgfqpoint{5.054510in}{1.704057in}}%
\pgfpathlineto{\pgfqpoint{5.333415in}{1.704057in}}%
\pgfpathlineto{\pgfqpoint{5.333415in}{2.333429in}}%
\pgfpathlineto{\pgfqpoint{5.054510in}{2.333429in}}%
\pgfpathclose%
\pgfusepath{fill}%
\end{pgfscope}%
\begin{pgfscope}%
\pgfpathrectangle{\pgfqpoint{0.812050in}{1.704057in}}{\pgfqpoint{5.385972in}{2.669658in}}%
\pgfusepath{clip}%
\pgfsetbuttcap%
\pgfsetmiterjoin%
\definecolor{currentfill}{rgb}{1.000000,0.498039,0.054902}%
\pgfsetfillcolor{currentfill}%
\pgfsetlinewidth{0.000000pt}%
\definecolor{currentstroke}{rgb}{0.000000,0.000000,0.000000}%
\pgfsetstrokecolor{currentstroke}%
\pgfsetstrokeopacity{0.000000}%
\pgfsetdash{}{0pt}%
\pgfpathmoveto{\pgfqpoint{5.674300in}{1.704057in}}%
\pgfpathlineto{\pgfqpoint{5.953205in}{1.704057in}}%
\pgfpathlineto{\pgfqpoint{5.953205in}{2.591719in}}%
\pgfpathlineto{\pgfqpoint{5.674300in}{2.591719in}}%
\pgfpathclose%
\pgfusepath{fill}%
\end{pgfscope}%
\begin{pgfscope}%
\pgfsetbuttcap%
\pgfsetroundjoin%
\definecolor{currentfill}{rgb}{0.000000,0.000000,0.000000}%
\pgfsetfillcolor{currentfill}%
\pgfsetlinewidth{0.803000pt}%
\definecolor{currentstroke}{rgb}{0.000000,0.000000,0.000000}%
\pgfsetstrokecolor{currentstroke}%
\pgfsetdash{}{0pt}%
\pgfsys@defobject{currentmarker}{\pgfqpoint{0.000000in}{-0.048611in}}{\pgfqpoint{0.000000in}{0.000000in}}{%
\pgfpathmoveto{\pgfqpoint{0.000000in}{0.000000in}}%
\pgfpathlineto{\pgfqpoint{0.000000in}{-0.048611in}}%
\pgfusepath{stroke,fill}%
}%
\begin{pgfscope}%
\pgfsys@transformshift{1.335772in}{1.704057in}%
\pgfsys@useobject{currentmarker}{}%
\end{pgfscope}%
\end{pgfscope}%
\begin{pgfscope}%
\definecolor{textcolor}{rgb}{0.000000,0.000000,0.000000}%
\pgfsetstrokecolor{textcolor}%
\pgfsetfillcolor{textcolor}%
\pgftext[x=0.793760in, y=0.368247in, left, base,rotate=45.000000]{\color{textcolor}\sffamily\fontsize{12.000000}{14.400000}\selectfont Pix3d(chair,no aug)}%
\end{pgfscope}%
\begin{pgfscope}%
\pgfsetbuttcap%
\pgfsetroundjoin%
\definecolor{currentfill}{rgb}{0.000000,0.000000,0.000000}%
\pgfsetfillcolor{currentfill}%
\pgfsetlinewidth{0.803000pt}%
\definecolor{currentstroke}{rgb}{0.000000,0.000000,0.000000}%
\pgfsetstrokecolor{currentstroke}%
\pgfsetdash{}{0pt}%
\pgfsys@defobject{currentmarker}{\pgfqpoint{0.000000in}{-0.048611in}}{\pgfqpoint{0.000000in}{0.000000in}}{%
\pgfpathmoveto{\pgfqpoint{0.000000in}{0.000000in}}%
\pgfpathlineto{\pgfqpoint{0.000000in}{-0.048611in}}%
\pgfusepath{stroke,fill}%
}%
\begin{pgfscope}%
\pgfsys@transformshift{1.955562in}{1.704057in}%
\pgfsys@useobject{currentmarker}{}%
\end{pgfscope}%
\end{pgfscope}%
\begin{pgfscope}%
\definecolor{textcolor}{rgb}{0.000000,0.000000,0.000000}%
\pgfsetstrokecolor{textcolor}%
\pgfsetfillcolor{textcolor}%
\pgftext[x=1.635269in, y=0.811685in, left, base,rotate=45.000000]{\color{textcolor}\sffamily\fontsize{12.000000}{14.400000}\selectfont Pix3d(chair)}%
\end{pgfscope}%
\begin{pgfscope}%
\pgfsetbuttcap%
\pgfsetroundjoin%
\definecolor{currentfill}{rgb}{0.000000,0.000000,0.000000}%
\pgfsetfillcolor{currentfill}%
\pgfsetlinewidth{0.803000pt}%
\definecolor{currentstroke}{rgb}{0.000000,0.000000,0.000000}%
\pgfsetstrokecolor{currentstroke}%
\pgfsetdash{}{0pt}%
\pgfsys@defobject{currentmarker}{\pgfqpoint{0.000000in}{-0.048611in}}{\pgfqpoint{0.000000in}{0.000000in}}{%
\pgfpathmoveto{\pgfqpoint{0.000000in}{0.000000in}}%
\pgfpathlineto{\pgfqpoint{0.000000in}{-0.048611in}}%
\pgfusepath{stroke,fill}%
}%
\begin{pgfscope}%
\pgfsys@transformshift{2.575351in}{1.704057in}%
\pgfsys@useobject{currentmarker}{}%
\end{pgfscope}%
\end{pgfscope}%
\begin{pgfscope}%
\definecolor{textcolor}{rgb}{0.000000,0.000000,0.000000}%
\pgfsetstrokecolor{textcolor}%
\pgfsetfillcolor{textcolor}%
\pgftext[x=2.278134in, y=0.857836in, left, base,rotate=45.000000]{\color{textcolor}\sffamily\fontsize{12.000000}{14.400000}\selectfont Textureless}%
\end{pgfscope}%
\begin{pgfscope}%
\pgfsetbuttcap%
\pgfsetroundjoin%
\definecolor{currentfill}{rgb}{0.000000,0.000000,0.000000}%
\pgfsetfillcolor{currentfill}%
\pgfsetlinewidth{0.803000pt}%
\definecolor{currentstroke}{rgb}{0.000000,0.000000,0.000000}%
\pgfsetstrokecolor{currentstroke}%
\pgfsetdash{}{0pt}%
\pgfsys@defobject{currentmarker}{\pgfqpoint{0.000000in}{-0.048611in}}{\pgfqpoint{0.000000in}{0.000000in}}{%
\pgfpathmoveto{\pgfqpoint{0.000000in}{0.000000in}}%
\pgfpathlineto{\pgfqpoint{0.000000in}{-0.048611in}}%
\pgfusepath{stroke,fill}%
}%
\begin{pgfscope}%
\pgfsys@transformshift{3.195141in}{1.704057in}%
\pgfsys@useobject{currentmarker}{}%
\end{pgfscope}%
\end{pgfscope}%
\begin{pgfscope}%
\definecolor{textcolor}{rgb}{0.000000,0.000000,0.000000}%
\pgfsetstrokecolor{textcolor}%
\pgfsetfillcolor{textcolor}%
\pgftext[x=2.701495in, y=0.464980in, left, base,rotate=45.000000]{\color{textcolor}\sffamily\fontsize{12.000000}{14.400000}\selectfont Textureless+Light}%
\end{pgfscope}%
\begin{pgfscope}%
\pgfsetbuttcap%
\pgfsetroundjoin%
\definecolor{currentfill}{rgb}{0.000000,0.000000,0.000000}%
\pgfsetfillcolor{currentfill}%
\pgfsetlinewidth{0.803000pt}%
\definecolor{currentstroke}{rgb}{0.000000,0.000000,0.000000}%
\pgfsetstrokecolor{currentstroke}%
\pgfsetdash{}{0pt}%
\pgfsys@defobject{currentmarker}{\pgfqpoint{0.000000in}{-0.048611in}}{\pgfqpoint{0.000000in}{0.000000in}}{%
\pgfpathmoveto{\pgfqpoint{0.000000in}{0.000000in}}%
\pgfpathlineto{\pgfqpoint{0.000000in}{-0.048611in}}%
\pgfusepath{stroke,fill}%
}%
\begin{pgfscope}%
\pgfsys@transformshift{3.814931in}{1.704057in}%
\pgfsys@useobject{currentmarker}{}%
\end{pgfscope}%
\end{pgfscope}%
\begin{pgfscope}%
\definecolor{textcolor}{rgb}{0.000000,0.000000,0.000000}%
\pgfsetstrokecolor{textcolor}%
\pgfsetfillcolor{textcolor}%
\pgftext[x=3.594334in, y=1.011077in, left, base,rotate=45.000000]{\color{textcolor}\sffamily\fontsize{12.000000}{14.400000}\selectfont Textured}%
\end{pgfscope}%
\begin{pgfscope}%
\pgfsetbuttcap%
\pgfsetroundjoin%
\definecolor{currentfill}{rgb}{0.000000,0.000000,0.000000}%
\pgfsetfillcolor{currentfill}%
\pgfsetlinewidth{0.803000pt}%
\definecolor{currentstroke}{rgb}{0.000000,0.000000,0.000000}%
\pgfsetstrokecolor{currentstroke}%
\pgfsetdash{}{0pt}%
\pgfsys@defobject{currentmarker}{\pgfqpoint{0.000000in}{-0.048611in}}{\pgfqpoint{0.000000in}{0.000000in}}{%
\pgfpathmoveto{\pgfqpoint{0.000000in}{0.000000in}}%
\pgfpathlineto{\pgfqpoint{0.000000in}{-0.048611in}}%
\pgfusepath{stroke,fill}%
}%
\begin{pgfscope}%
\pgfsys@transformshift{4.434720in}{1.704057in}%
\pgfsys@useobject{currentmarker}{}%
\end{pgfscope}%
\end{pgfscope}%
\begin{pgfscope}%
\definecolor{textcolor}{rgb}{0.000000,0.000000,0.000000}%
\pgfsetstrokecolor{textcolor}%
\pgfsetfillcolor{textcolor}%
\pgftext[x=4.017695in, y=0.618221in, left, base,rotate=45.000000]{\color{textcolor}\sffamily\fontsize{12.000000}{14.400000}\selectfont Textured+Light}%
\end{pgfscope}%
\begin{pgfscope}%
\pgfsetbuttcap%
\pgfsetroundjoin%
\definecolor{currentfill}{rgb}{0.000000,0.000000,0.000000}%
\pgfsetfillcolor{currentfill}%
\pgfsetlinewidth{0.803000pt}%
\definecolor{currentstroke}{rgb}{0.000000,0.000000,0.000000}%
\pgfsetstrokecolor{currentstroke}%
\pgfsetdash{}{0pt}%
\pgfsys@defobject{currentmarker}{\pgfqpoint{0.000000in}{-0.048611in}}{\pgfqpoint{0.000000in}{0.000000in}}{%
\pgfpathmoveto{\pgfqpoint{0.000000in}{0.000000in}}%
\pgfpathlineto{\pgfqpoint{0.000000in}{-0.048611in}}%
\pgfusepath{stroke,fill}%
}%
\begin{pgfscope}%
\pgfsys@transformshift{5.054510in}{1.704057in}%
\pgfsys@useobject{currentmarker}{}%
\end{pgfscope}%
\end{pgfscope}%
\begin{pgfscope}%
\definecolor{textcolor}{rgb}{0.000000,0.000000,0.000000}%
\pgfsetstrokecolor{textcolor}%
\pgfsetfillcolor{textcolor}%
\pgftext[x=4.728175in, y=0.799601in, left, base,rotate=45.000000]{\color{textcolor}\sffamily\fontsize{12.000000}{14.400000}\selectfont Multi-Object}%
\end{pgfscope}%
\begin{pgfscope}%
\pgfsetbuttcap%
\pgfsetroundjoin%
\definecolor{currentfill}{rgb}{0.000000,0.000000,0.000000}%
\pgfsetfillcolor{currentfill}%
\pgfsetlinewidth{0.803000pt}%
\definecolor{currentstroke}{rgb}{0.000000,0.000000,0.000000}%
\pgfsetstrokecolor{currentstroke}%
\pgfsetdash{}{0pt}%
\pgfsys@defobject{currentmarker}{\pgfqpoint{0.000000in}{-0.048611in}}{\pgfqpoint{0.000000in}{0.000000in}}{%
\pgfpathmoveto{\pgfqpoint{0.000000in}{0.000000in}}%
\pgfpathlineto{\pgfqpoint{0.000000in}{-0.048611in}}%
\pgfusepath{stroke,fill}%
}%
\begin{pgfscope}%
\pgfsys@transformshift{5.674300in}{1.704057in}%
\pgfsys@useobject{currentmarker}{}%
\end{pgfscope}%
\end{pgfscope}%
\begin{pgfscope}%
\definecolor{textcolor}{rgb}{0.000000,0.000000,0.000000}%
\pgfsetstrokecolor{textcolor}%
\pgfsetfillcolor{textcolor}%
\pgftext[x=5.407437in, y=0.918545in, left, base,rotate=45.000000]{\color{textcolor}\sffamily\fontsize{12.000000}{14.400000}\selectfont Combined}%
\end{pgfscope}%
\begin{pgfscope}%
\definecolor{textcolor}{rgb}{0.000000,0.000000,0.000000}%
\pgfsetstrokecolor{textcolor}%
\pgfsetfillcolor{textcolor}%
\pgftext[x=3.505036in,y=0.288178in,,top]{\color{textcolor}\sffamily\fontsize{14.000000}{16.800000}\bfseries\selectfont Dataset}%
\end{pgfscope}%
\begin{pgfscope}%
\pgfsetbuttcap%
\pgfsetroundjoin%
\definecolor{currentfill}{rgb}{0.000000,0.000000,0.000000}%
\pgfsetfillcolor{currentfill}%
\pgfsetlinewidth{0.803000pt}%
\definecolor{currentstroke}{rgb}{0.000000,0.000000,0.000000}%
\pgfsetstrokecolor{currentstroke}%
\pgfsetdash{}{0pt}%
\pgfsys@defobject{currentmarker}{\pgfqpoint{-0.048611in}{0.000000in}}{\pgfqpoint{-0.000000in}{0.000000in}}{%
\pgfpathmoveto{\pgfqpoint{-0.000000in}{0.000000in}}%
\pgfpathlineto{\pgfqpoint{-0.048611in}{0.000000in}}%
\pgfusepath{stroke,fill}%
}%
\begin{pgfscope}%
\pgfsys@transformshift{0.812050in}{1.704057in}%
\pgfsys@useobject{currentmarker}{}%
\end{pgfscope}%
\end{pgfscope}%
\begin{pgfscope}%
\definecolor{textcolor}{rgb}{0.000000,0.000000,0.000000}%
\pgfsetstrokecolor{textcolor}%
\pgfsetfillcolor{textcolor}%
\pgftext[x=0.343734in, y=1.640744in, left, base]{\color{textcolor}\sffamily\fontsize{12.000000}{14.400000}\selectfont 0.00}%
\end{pgfscope}%
\begin{pgfscope}%
\pgfsetbuttcap%
\pgfsetroundjoin%
\definecolor{currentfill}{rgb}{0.000000,0.000000,0.000000}%
\pgfsetfillcolor{currentfill}%
\pgfsetlinewidth{0.803000pt}%
\definecolor{currentstroke}{rgb}{0.000000,0.000000,0.000000}%
\pgfsetstrokecolor{currentstroke}%
\pgfsetdash{}{0pt}%
\pgfsys@defobject{currentmarker}{\pgfqpoint{-0.048611in}{0.000000in}}{\pgfqpoint{-0.000000in}{0.000000in}}{%
\pgfpathmoveto{\pgfqpoint{-0.000000in}{0.000000in}}%
\pgfpathlineto{\pgfqpoint{-0.048611in}{0.000000in}}%
\pgfusepath{stroke,fill}%
}%
\begin{pgfscope}%
\pgfsys@transformshift{0.812050in}{2.037765in}%
\pgfsys@useobject{currentmarker}{}%
\end{pgfscope}%
\end{pgfscope}%
\begin{pgfscope}%
\definecolor{textcolor}{rgb}{0.000000,0.000000,0.000000}%
\pgfsetstrokecolor{textcolor}%
\pgfsetfillcolor{textcolor}%
\pgftext[x=0.343734in, y=1.974451in, left, base]{\color{textcolor}\sffamily\fontsize{12.000000}{14.400000}\selectfont 0.05}%
\end{pgfscope}%
\begin{pgfscope}%
\pgfsetbuttcap%
\pgfsetroundjoin%
\definecolor{currentfill}{rgb}{0.000000,0.000000,0.000000}%
\pgfsetfillcolor{currentfill}%
\pgfsetlinewidth{0.803000pt}%
\definecolor{currentstroke}{rgb}{0.000000,0.000000,0.000000}%
\pgfsetstrokecolor{currentstroke}%
\pgfsetdash{}{0pt}%
\pgfsys@defobject{currentmarker}{\pgfqpoint{-0.048611in}{0.000000in}}{\pgfqpoint{-0.000000in}{0.000000in}}{%
\pgfpathmoveto{\pgfqpoint{-0.000000in}{0.000000in}}%
\pgfpathlineto{\pgfqpoint{-0.048611in}{0.000000in}}%
\pgfusepath{stroke,fill}%
}%
\begin{pgfscope}%
\pgfsys@transformshift{0.812050in}{2.371472in}%
\pgfsys@useobject{currentmarker}{}%
\end{pgfscope}%
\end{pgfscope}%
\begin{pgfscope}%
\definecolor{textcolor}{rgb}{0.000000,0.000000,0.000000}%
\pgfsetstrokecolor{textcolor}%
\pgfsetfillcolor{textcolor}%
\pgftext[x=0.343734in, y=2.308158in, left, base]{\color{textcolor}\sffamily\fontsize{12.000000}{14.400000}\selectfont 0.10}%
\end{pgfscope}%
\begin{pgfscope}%
\pgfsetbuttcap%
\pgfsetroundjoin%
\definecolor{currentfill}{rgb}{0.000000,0.000000,0.000000}%
\pgfsetfillcolor{currentfill}%
\pgfsetlinewidth{0.803000pt}%
\definecolor{currentstroke}{rgb}{0.000000,0.000000,0.000000}%
\pgfsetstrokecolor{currentstroke}%
\pgfsetdash{}{0pt}%
\pgfsys@defobject{currentmarker}{\pgfqpoint{-0.048611in}{0.000000in}}{\pgfqpoint{-0.000000in}{0.000000in}}{%
\pgfpathmoveto{\pgfqpoint{-0.000000in}{0.000000in}}%
\pgfpathlineto{\pgfqpoint{-0.048611in}{0.000000in}}%
\pgfusepath{stroke,fill}%
}%
\begin{pgfscope}%
\pgfsys@transformshift{0.812050in}{2.705179in}%
\pgfsys@useobject{currentmarker}{}%
\end{pgfscope}%
\end{pgfscope}%
\begin{pgfscope}%
\definecolor{textcolor}{rgb}{0.000000,0.000000,0.000000}%
\pgfsetstrokecolor{textcolor}%
\pgfsetfillcolor{textcolor}%
\pgftext[x=0.343734in, y=2.641865in, left, base]{\color{textcolor}\sffamily\fontsize{12.000000}{14.400000}\selectfont 0.15}%
\end{pgfscope}%
\begin{pgfscope}%
\pgfsetbuttcap%
\pgfsetroundjoin%
\definecolor{currentfill}{rgb}{0.000000,0.000000,0.000000}%
\pgfsetfillcolor{currentfill}%
\pgfsetlinewidth{0.803000pt}%
\definecolor{currentstroke}{rgb}{0.000000,0.000000,0.000000}%
\pgfsetstrokecolor{currentstroke}%
\pgfsetdash{}{0pt}%
\pgfsys@defobject{currentmarker}{\pgfqpoint{-0.048611in}{0.000000in}}{\pgfqpoint{-0.000000in}{0.000000in}}{%
\pgfpathmoveto{\pgfqpoint{-0.000000in}{0.000000in}}%
\pgfpathlineto{\pgfqpoint{-0.048611in}{0.000000in}}%
\pgfusepath{stroke,fill}%
}%
\begin{pgfscope}%
\pgfsys@transformshift{0.812050in}{3.038887in}%
\pgfsys@useobject{currentmarker}{}%
\end{pgfscope}%
\end{pgfscope}%
\begin{pgfscope}%
\definecolor{textcolor}{rgb}{0.000000,0.000000,0.000000}%
\pgfsetstrokecolor{textcolor}%
\pgfsetfillcolor{textcolor}%
\pgftext[x=0.343734in, y=2.975573in, left, base]{\color{textcolor}\sffamily\fontsize{12.000000}{14.400000}\selectfont 0.20}%
\end{pgfscope}%
\begin{pgfscope}%
\pgfsetbuttcap%
\pgfsetroundjoin%
\definecolor{currentfill}{rgb}{0.000000,0.000000,0.000000}%
\pgfsetfillcolor{currentfill}%
\pgfsetlinewidth{0.803000pt}%
\definecolor{currentstroke}{rgb}{0.000000,0.000000,0.000000}%
\pgfsetstrokecolor{currentstroke}%
\pgfsetdash{}{0pt}%
\pgfsys@defobject{currentmarker}{\pgfqpoint{-0.048611in}{0.000000in}}{\pgfqpoint{-0.000000in}{0.000000in}}{%
\pgfpathmoveto{\pgfqpoint{-0.000000in}{0.000000in}}%
\pgfpathlineto{\pgfqpoint{-0.048611in}{0.000000in}}%
\pgfusepath{stroke,fill}%
}%
\begin{pgfscope}%
\pgfsys@transformshift{0.812050in}{3.372594in}%
\pgfsys@useobject{currentmarker}{}%
\end{pgfscope}%
\end{pgfscope}%
\begin{pgfscope}%
\definecolor{textcolor}{rgb}{0.000000,0.000000,0.000000}%
\pgfsetstrokecolor{textcolor}%
\pgfsetfillcolor{textcolor}%
\pgftext[x=0.343734in, y=3.309280in, left, base]{\color{textcolor}\sffamily\fontsize{12.000000}{14.400000}\selectfont 0.25}%
\end{pgfscope}%
\begin{pgfscope}%
\pgfsetbuttcap%
\pgfsetroundjoin%
\definecolor{currentfill}{rgb}{0.000000,0.000000,0.000000}%
\pgfsetfillcolor{currentfill}%
\pgfsetlinewidth{0.803000pt}%
\definecolor{currentstroke}{rgb}{0.000000,0.000000,0.000000}%
\pgfsetstrokecolor{currentstroke}%
\pgfsetdash{}{0pt}%
\pgfsys@defobject{currentmarker}{\pgfqpoint{-0.048611in}{0.000000in}}{\pgfqpoint{-0.000000in}{0.000000in}}{%
\pgfpathmoveto{\pgfqpoint{-0.000000in}{0.000000in}}%
\pgfpathlineto{\pgfqpoint{-0.048611in}{0.000000in}}%
\pgfusepath{stroke,fill}%
}%
\begin{pgfscope}%
\pgfsys@transformshift{0.812050in}{3.706301in}%
\pgfsys@useobject{currentmarker}{}%
\end{pgfscope}%
\end{pgfscope}%
\begin{pgfscope}%
\definecolor{textcolor}{rgb}{0.000000,0.000000,0.000000}%
\pgfsetstrokecolor{textcolor}%
\pgfsetfillcolor{textcolor}%
\pgftext[x=0.343734in, y=3.642987in, left, base]{\color{textcolor}\sffamily\fontsize{12.000000}{14.400000}\selectfont 0.30}%
\end{pgfscope}%
\begin{pgfscope}%
\pgfsetbuttcap%
\pgfsetroundjoin%
\definecolor{currentfill}{rgb}{0.000000,0.000000,0.000000}%
\pgfsetfillcolor{currentfill}%
\pgfsetlinewidth{0.803000pt}%
\definecolor{currentstroke}{rgb}{0.000000,0.000000,0.000000}%
\pgfsetstrokecolor{currentstroke}%
\pgfsetdash{}{0pt}%
\pgfsys@defobject{currentmarker}{\pgfqpoint{-0.048611in}{0.000000in}}{\pgfqpoint{-0.000000in}{0.000000in}}{%
\pgfpathmoveto{\pgfqpoint{-0.000000in}{0.000000in}}%
\pgfpathlineto{\pgfqpoint{-0.048611in}{0.000000in}}%
\pgfusepath{stroke,fill}%
}%
\begin{pgfscope}%
\pgfsys@transformshift{0.812050in}{4.040008in}%
\pgfsys@useobject{currentmarker}{}%
\end{pgfscope}%
\end{pgfscope}%
\begin{pgfscope}%
\definecolor{textcolor}{rgb}{0.000000,0.000000,0.000000}%
\pgfsetstrokecolor{textcolor}%
\pgfsetfillcolor{textcolor}%
\pgftext[x=0.343734in, y=3.976695in, left, base]{\color{textcolor}\sffamily\fontsize{12.000000}{14.400000}\selectfont 0.35}%
\end{pgfscope}%
\begin{pgfscope}%
\pgfsetbuttcap%
\pgfsetroundjoin%
\definecolor{currentfill}{rgb}{0.000000,0.000000,0.000000}%
\pgfsetfillcolor{currentfill}%
\pgfsetlinewidth{0.803000pt}%
\definecolor{currentstroke}{rgb}{0.000000,0.000000,0.000000}%
\pgfsetstrokecolor{currentstroke}%
\pgfsetdash{}{0pt}%
\pgfsys@defobject{currentmarker}{\pgfqpoint{-0.048611in}{0.000000in}}{\pgfqpoint{-0.000000in}{0.000000in}}{%
\pgfpathmoveto{\pgfqpoint{-0.000000in}{0.000000in}}%
\pgfpathlineto{\pgfqpoint{-0.048611in}{0.000000in}}%
\pgfusepath{stroke,fill}%
}%
\begin{pgfscope}%
\pgfsys@transformshift{0.812050in}{4.373716in}%
\pgfsys@useobject{currentmarker}{}%
\end{pgfscope}%
\end{pgfscope}%
\begin{pgfscope}%
\definecolor{textcolor}{rgb}{0.000000,0.000000,0.000000}%
\pgfsetstrokecolor{textcolor}%
\pgfsetfillcolor{textcolor}%
\pgftext[x=0.343734in, y=4.310402in, left, base]{\color{textcolor}\sffamily\fontsize{12.000000}{14.400000}\selectfont 0.40}%
\end{pgfscope}%
\begin{pgfscope}%
\definecolor{textcolor}{rgb}{0.000000,0.000000,0.000000}%
\pgfsetstrokecolor{textcolor}%
\pgfsetfillcolor{textcolor}%
\pgftext[x=0.288178in,y=3.038887in,,bottom,rotate=90.000000]{\color{textcolor}\sffamily\fontsize{14.000000}{16.800000}\bfseries\selectfont IoU}%
\end{pgfscope}%
\begin{pgfscope}%
\pgfsetrectcap%
\pgfsetmiterjoin%
\pgfsetlinewidth{0.803000pt}%
\definecolor{currentstroke}{rgb}{0.000000,0.000000,0.000000}%
\pgfsetstrokecolor{currentstroke}%
\pgfsetdash{}{0pt}%
\pgfpathmoveto{\pgfqpoint{0.812050in}{1.704057in}}%
\pgfpathlineto{\pgfqpoint{0.812050in}{4.373716in}}%
\pgfusepath{stroke}%
\end{pgfscope}%
\begin{pgfscope}%
\pgfsetrectcap%
\pgfsetmiterjoin%
\pgfsetlinewidth{0.803000pt}%
\definecolor{currentstroke}{rgb}{0.000000,0.000000,0.000000}%
\pgfsetstrokecolor{currentstroke}%
\pgfsetdash{}{0pt}%
\pgfpathmoveto{\pgfqpoint{6.198022in}{1.704057in}}%
\pgfpathlineto{\pgfqpoint{6.198022in}{4.373716in}}%
\pgfusepath{stroke}%
\end{pgfscope}%
\begin{pgfscope}%
\pgfsetrectcap%
\pgfsetmiterjoin%
\pgfsetlinewidth{0.803000pt}%
\definecolor{currentstroke}{rgb}{0.000000,0.000000,0.000000}%
\pgfsetstrokecolor{currentstroke}%
\pgfsetdash{}{0pt}%
\pgfpathmoveto{\pgfqpoint{0.812050in}{1.704057in}}%
\pgfpathlineto{\pgfqpoint{6.198022in}{1.704057in}}%
\pgfusepath{stroke}%
\end{pgfscope}%
\begin{pgfscope}%
\pgfsetrectcap%
\pgfsetmiterjoin%
\pgfsetlinewidth{0.803000pt}%
\definecolor{currentstroke}{rgb}{0.000000,0.000000,0.000000}%
\pgfsetstrokecolor{currentstroke}%
\pgfsetdash{}{0pt}%
\pgfpathmoveto{\pgfqpoint{0.812050in}{4.373716in}}%
\pgfpathlineto{\pgfqpoint{6.198022in}{4.373716in}}%
\pgfusepath{stroke}%
\end{pgfscope}%
\begin{pgfscope}%
\definecolor{textcolor}{rgb}{0.000000,0.000000,0.000000}%
\pgfsetstrokecolor{textcolor}%
\pgfsetfillcolor{textcolor}%
\pgftext[x=1.196319in,y=3.612483in,,bottom]{\color{textcolor}\sffamily\fontsize{7.000000}{8.400000}\selectfont 0.2797}%
\end{pgfscope}%
\begin{pgfscope}%
\definecolor{textcolor}{rgb}{0.000000,0.000000,0.000000}%
\pgfsetstrokecolor{textcolor}%
\pgfsetfillcolor{textcolor}%
\pgftext[x=1.816109in,y=3.951529in,,bottom]{\color{textcolor}\sffamily\fontsize{7.000000}{8.400000}\selectfont 0.3305}%
\end{pgfscope}%
\begin{pgfscope}%
\definecolor{textcolor}{rgb}{0.000000,0.000000,0.000000}%
\pgfsetstrokecolor{textcolor}%
\pgfsetfillcolor{textcolor}%
\pgftext[x=2.435899in,y=2.945735in,,bottom]{\color{textcolor}\sffamily\fontsize{7.000000}{8.400000}\selectfont 0.1798}%
\end{pgfscope}%
\begin{pgfscope}%
\definecolor{textcolor}{rgb}{0.000000,0.000000,0.000000}%
\pgfsetstrokecolor{textcolor}%
\pgfsetfillcolor{textcolor}%
\pgftext[x=3.055688in,y=2.700127in,,bottom]{\color{textcolor}\sffamily\fontsize{7.000000}{8.400000}\selectfont 0.143}%
\end{pgfscope}%
\begin{pgfscope}%
\definecolor{textcolor}{rgb}{0.000000,0.000000,0.000000}%
\pgfsetstrokecolor{textcolor}%
\pgfsetfillcolor{textcolor}%
\pgftext[x=3.675478in,y=3.225382in,,bottom]{\color{textcolor}\sffamily\fontsize{7.000000}{8.400000}\selectfont 0.2217}%
\end{pgfscope}%
\begin{pgfscope}%
\definecolor{textcolor}{rgb}{0.000000,0.000000,0.000000}%
\pgfsetstrokecolor{textcolor}%
\pgfsetfillcolor{textcolor}%
\pgftext[x=4.295268in,y=3.228052in,,bottom]{\color{textcolor}\sffamily\fontsize{7.000000}{8.400000}\selectfont 0.2221}%
\end{pgfscope}%
\begin{pgfscope}%
\definecolor{textcolor}{rgb}{0.000000,0.000000,0.000000}%
\pgfsetstrokecolor{textcolor}%
\pgfsetfillcolor{textcolor}%
\pgftext[x=4.915057in,y=3.328831in,,bottom]{\color{textcolor}\sffamily\fontsize{7.000000}{8.400000}\selectfont 0.2372}%
\end{pgfscope}%
\begin{pgfscope}%
\definecolor{textcolor}{rgb}{0.000000,0.000000,0.000000}%
\pgfsetstrokecolor{textcolor}%
\pgfsetfillcolor{textcolor}%
\pgftext[x=5.534847in,y=3.412258in,,bottom]{\color{textcolor}\sffamily\fontsize{7.000000}{8.400000}\selectfont 0.2497}%
\end{pgfscope}%
\begin{pgfscope}%
\definecolor{textcolor}{rgb}{0.000000,0.000000,0.000000}%
\pgfsetstrokecolor{textcolor}%
\pgfsetfillcolor{textcolor}%
\pgftext[x=1.475225in,y=3.543739in,,bottom]{\color{textcolor}\sffamily\fontsize{7.000000}{8.400000}\selectfont 0.2694}%
\end{pgfscope}%
\begin{pgfscope}%
\definecolor{textcolor}{rgb}{0.000000,0.000000,0.000000}%
\pgfsetstrokecolor{textcolor}%
\pgfsetfillcolor{textcolor}%
\pgftext[x=2.095014in,y=3.691905in,,bottom]{\color{textcolor}\sffamily\fontsize{7.000000}{8.400000}\selectfont 0.2916}%
\end{pgfscope}%
\begin{pgfscope}%
\definecolor{textcolor}{rgb}{0.000000,0.000000,0.000000}%
\pgfsetstrokecolor{textcolor}%
\pgfsetfillcolor{textcolor}%
\pgftext[x=2.714804in,y=2.244950in,,bottom]{\color{textcolor}\sffamily\fontsize{7.000000}{8.400000}\selectfont 0.0748}%
\end{pgfscope}%
\begin{pgfscope}%
\definecolor{textcolor}{rgb}{0.000000,0.000000,0.000000}%
\pgfsetstrokecolor{textcolor}%
\pgfsetfillcolor{textcolor}%
\pgftext[x=3.334594in,y=2.430491in,,bottom]{\color{textcolor}\sffamily\fontsize{7.000000}{8.400000}\selectfont 0.1026}%
\end{pgfscope}%
\begin{pgfscope}%
\definecolor{textcolor}{rgb}{0.000000,0.000000,0.000000}%
\pgfsetstrokecolor{textcolor}%
\pgfsetfillcolor{textcolor}%
\pgftext[x=3.954383in,y=2.493228in,,bottom]{\color{textcolor}\sffamily\fontsize{7.000000}{8.400000}\selectfont 0.112}%
\end{pgfscope}%
\begin{pgfscope}%
\definecolor{textcolor}{rgb}{0.000000,0.000000,0.000000}%
\pgfsetstrokecolor{textcolor}%
\pgfsetfillcolor{textcolor}%
\pgftext[x=4.574173in,y=2.407132in,,bottom]{\color{textcolor}\sffamily\fontsize{7.000000}{8.400000}\selectfont 0.0991}%
\end{pgfscope}%
\begin{pgfscope}%
\definecolor{textcolor}{rgb}{0.000000,0.000000,0.000000}%
\pgfsetstrokecolor{textcolor}%
\pgfsetfillcolor{textcolor}%
\pgftext[x=5.193963in,y=2.375096in,,bottom]{\color{textcolor}\sffamily\fontsize{7.000000}{8.400000}\selectfont 0.0943}%
\end{pgfscope}%
\begin{pgfscope}%
\definecolor{textcolor}{rgb}{0.000000,0.000000,0.000000}%
\pgfsetstrokecolor{textcolor}%
\pgfsetfillcolor{textcolor}%
\pgftext[x=5.813752in,y=2.633385in,,bottom]{\color{textcolor}\sffamily\fontsize{7.000000}{8.400000}\selectfont 0.133}%
\end{pgfscope}%
\begin{pgfscope}%
\definecolor{textcolor}{rgb}{0.000000,0.000000,0.000000}%
\pgfsetstrokecolor{textcolor}%
\pgfsetfillcolor{textcolor}%
\pgftext[x=3.505036in,y=4.457049in,,base]{\color{textcolor}\sffamily\fontsize{14.000000}{16.800000}\selectfont Abalation study on chairs}%
\end{pgfscope}%
\begin{pgfscope}%
\pgfsetbuttcap%
\pgfsetmiterjoin%
\definecolor{currentfill}{rgb}{1.000000,1.000000,1.000000}%
\pgfsetfillcolor{currentfill}%
\pgfsetfillopacity{0.800000}%
\pgfsetlinewidth{1.003750pt}%
\definecolor{currentstroke}{rgb}{0.800000,0.800000,0.800000}%
\pgfsetstrokecolor{currentstroke}%
\pgfsetstrokeopacity{0.800000}%
\pgfsetdash{}{0pt}%
\pgfpathmoveto{\pgfqpoint{4.886634in}{3.854890in}}%
\pgfpathlineto{\pgfqpoint{6.100800in}{3.854890in}}%
\pgfpathquadraticcurveto{\pgfqpoint{6.128577in}{3.854890in}}{\pgfqpoint{6.128577in}{3.882668in}}%
\pgfpathlineto{\pgfqpoint{6.128577in}{4.276493in}}%
\pgfpathquadraticcurveto{\pgfqpoint{6.128577in}{4.304271in}}{\pgfqpoint{6.100800in}{4.304271in}}%
\pgfpathlineto{\pgfqpoint{4.886634in}{4.304271in}}%
\pgfpathquadraticcurveto{\pgfqpoint{4.858856in}{4.304271in}}{\pgfqpoint{4.858856in}{4.276493in}}%
\pgfpathlineto{\pgfqpoint{4.858856in}{3.882668in}}%
\pgfpathquadraticcurveto{\pgfqpoint{4.858856in}{3.854890in}}{\pgfqpoint{4.886634in}{3.854890in}}%
\pgfpathclose%
\pgfusepath{stroke,fill}%
\end{pgfscope}%
\begin{pgfscope}%
\pgfsetbuttcap%
\pgfsetmiterjoin%
\definecolor{currentfill}{rgb}{0.121569,0.466667,0.705882}%
\pgfsetfillcolor{currentfill}%
\pgfsetlinewidth{0.000000pt}%
\definecolor{currentstroke}{rgb}{0.000000,0.000000,0.000000}%
\pgfsetstrokecolor{currentstroke}%
\pgfsetstrokeopacity{0.000000}%
\pgfsetdash{}{0pt}%
\pgfpathmoveto{\pgfqpoint{4.914412in}{4.143193in}}%
\pgfpathlineto{\pgfqpoint{5.192190in}{4.143193in}}%
\pgfpathlineto{\pgfqpoint{5.192190in}{4.240415in}}%
\pgfpathlineto{\pgfqpoint{4.914412in}{4.240415in}}%
\pgfpathclose%
\pgfusepath{fill}%
\end{pgfscope}%
\begin{pgfscope}%
\definecolor{textcolor}{rgb}{0.000000,0.000000,0.000000}%
\pgfsetstrokecolor{textcolor}%
\pgfsetfillcolor{textcolor}%
\pgftext[x=5.303301in,y=4.143193in,left,base]{\color{textcolor}\sffamily\fontsize{10.000000}{12.000000}\selectfont Pix2Vox++}%
\end{pgfscope}%
\begin{pgfscope}%
\pgfsetbuttcap%
\pgfsetmiterjoin%
\definecolor{currentfill}{rgb}{1.000000,0.498039,0.054902}%
\pgfsetfillcolor{currentfill}%
\pgfsetlinewidth{0.000000pt}%
\definecolor{currentstroke}{rgb}{0.000000,0.000000,0.000000}%
\pgfsetstrokecolor{currentstroke}%
\pgfsetstrokeopacity{0.000000}%
\pgfsetdash{}{0pt}%
\pgfpathmoveto{\pgfqpoint{4.914412in}{3.939335in}}%
\pgfpathlineto{\pgfqpoint{5.192190in}{3.939335in}}%
\pgfpathlineto{\pgfqpoint{5.192190in}{4.036558in}}%
\pgfpathlineto{\pgfqpoint{4.914412in}{4.036558in}}%
\pgfpathclose%
\pgfusepath{fill}%
\end{pgfscope}%
\begin{pgfscope}%
\definecolor{textcolor}{rgb}{0.000000,0.000000,0.000000}%
\pgfsetstrokecolor{textcolor}%
\pgfsetfillcolor{textcolor}%
\pgftext[x=5.303301in,y=3.939335in,left,base]{\color{textcolor}\sffamily\fontsize{10.000000}{12.000000}\selectfont Pix2Vox}%
\end{pgfscope}%
\end{pgfpicture}%
\makeatother%
\endgroup%
}
    \caption[\gls{iou} Comparison for Ablation Datasets.]{Bar plot for the \gls{iou} for \textbf{baseline} trained on chair dataset with different domain randomization parameters and tested on real dataset.
    We see a dip in performance near textureless dataset, but it gradually increases with addition of domain randomization parameter.}
    \label{fig:ablation1}
\end{figure}


\begin{figure}[!ht]
    \begin{tabular}{llll}
        Pix3D images & \includegraphics[width=.2\linewidth]{/Users/apple/OVGU/Thesis/code/3dReconstruction/report/images/evaluation/reconstruction/ablation/chair1} &
        \includegraphics[width=.2\linewidth]{/Users/apple/OVGU/Thesis/code/3dReconstruction/report/images/evaluation/reconstruction/ablation/chair2} &
        \includegraphics[width=.2\linewidth]{/Users/apple/OVGU/Thesis/code/3dReconstruction/report/images/evaluation/reconstruction/ablation/chair3}\\

        Ground Truth & \includegraphics[trim={0 0 {.1\width} 0},clip,width=.2\linewidth]{/Users/apple/OVGU/Thesis/code/3dReconstruction/report/images/evaluation/reconstruction/ablation/chair1_original} &
        \includegraphics[trim={0 0 {.1\width} 0},clip,width=.2\linewidth]{/Users/apple/OVGU/Thesis/code/3dReconstruction/report/images/evaluation/reconstruction/ablation/chair2_original} &
        \includegraphics[trim={0 0 {.1\width} 0},clip,width=.2\linewidth]{/Users/apple/OVGU/Thesis/code/3dReconstruction/report/images/evaluation/reconstruction/ablation/chair3_original}\\

        Output1 & \includegraphics[width=.2\linewidth]{/Users/apple/OVGU/Thesis/code/3dReconstruction/report/images/evaluation/reconstruction/ablation/pix3d_p2vpp_chair1} &
        \includegraphics[width=.2\linewidth]{/Users/apple/OVGU/Thesis/code/3dReconstruction/report/images/evaluation/reconstruction/ablation/pix3d_p2vpp_chair2} &
        \includegraphics[width=.2\linewidth]{/Users/apple/OVGU/Thesis/code/3dReconstruction/report/images/evaluation/reconstruction/ablation/pix3d_p2vpp_chair3}\\

        Output2 & \includegraphics[width=.2\linewidth]{/Users/apple/OVGU/Thesis/code/3dReconstruction/report/images/evaluation/reconstruction/ablation/pix3d_p2v_chair1} &
        \includegraphics[width=.2\linewidth]{/Users/apple/OVGU/Thesis/code/3dReconstruction/report/images/evaluation/reconstruction/ablation/pix3d_p2v_chair2} &
        \includegraphics[width=.2\linewidth]{/Users/apple/OVGU/Thesis/code/3dReconstruction/report/images/evaluation/reconstruction/ablation/pix3d_p2v_chair3}\\

        Output3 & \includegraphics[width=.2\linewidth]{/Users/apple/OVGU/Thesis/code/3dReconstruction/report/images/evaluation/reconstruction/ablation/ablataion_p2vpp_chair1} &
        \includegraphics[width=.2\linewidth]{/Users/apple/OVGU/Thesis/code/3dReconstruction/report/images/evaluation/reconstruction/ablation/ablataion_p2vpp_chair2} &
        \includegraphics[width=.2\linewidth]{/Users/apple/OVGU/Thesis/code/3dReconstruction/report/images/evaluation/reconstruction/ablation/ablataion_p2vpp_chair3}\\

        Output4 & \includegraphics[width=.2\linewidth]{/Users/apple/OVGU/Thesis/code/3dReconstruction/report/images/evaluation/reconstruction/ablation/ablation_p2v_chair1} &
        \includegraphics[width=.2\linewidth]{/Users/apple/OVGU/Thesis/code/3dReconstruction/report/images/evaluation/reconstruction/ablation/ablation_p2v_chair2} &
        \includegraphics[width=.2\linewidth]{/Users/apple/OVGU/Thesis/code/3dReconstruction/report/images/evaluation/reconstruction/ablation/ablation_p2v_chair3}\\

    \end{tabular}
    \caption[3D Reconstruction Outputs for Ablation Datasets.]{3D reconstruction outputs for best \textbf{ablation} models and models trained on only synthetic dataset. Output1-2: Pix2Vox++ and Pix2Vox trained on Pix3D.
    Output3-4: Pix2Vox++ and Pix2Vox trained on Multi-object chair synthetic dataset, reconstructs a generic chair with less detail.}
    \label{fig:ablation_images1}
\end{figure}

The baseline on real chair dataset from Pix3D is 0.2797 and 0.2916, without and with 2D augmentations on Pix2Vox++.
For Pix2Vox we observe 0.2694 and 0.3305 for the same 2 set up.
The textureless chair dataset from \gls{free} gives an \gls{iou} of 0.1798 and 0.143 with light for Pix2Vox++, 0.0748 and 0.1026 for Pix2Vox.
From then on, we see a slight increase in \gls{iou} for Pix2Vox++, but not for Pix2Vox.
One possible explanation for this behavior can be seen in \autoref{fig:tsne per chair dataset}.
The textureless dataset seems to have more overlap in latent space compared to textureless with light.
In \autoref{fig:ablation1}, we see a slight increase in performance in the 3D reconstruction task, which also holds with \gls{tsne} visualization,
where we see the latent space occupying common regions.
The combined dataset of all other domain randomization showed the best performance of the set for both the baseline models.

In \autoref{fig:ablation_images1}, we see the 3D reconstruction output for models trained on only real and only synthetic chair dataset.
The outputs were collected for images from the real dataset with the threshold which gave the best \gls{iou}.
The outputs of models trained on synthetic dataset predict a shape similar to a chair but do not provide fine details as in ground truth.

%\begin{figure}
%    \centering
%    \resizebox{0.7\textwidth}{!}{%% Creator: Matplotlib, PGF backend
%%
%% To include the figure in your LaTeX document, write
%%   \input{<filename>.pgf}
%%
%% Make sure the required packages are loaded in your preamble
%%   \usepackage{pgf}
%%
%% Figures using additional raster images can only be included by \input if
%% they are in the same directory as the main LaTeX file. For loading figures
%% from other directories you can use the `import` package
%%   \usepackage{import}
%%
%% and then include the figures with
%%   \import{<path to file>}{<filename>.pgf}
%%
%% Matplotlib used the following preamble
%%   \usepackage{fontspec}
%%   \setmainfont{DejaVuSerif.ttf}[Path=\detokenize{/Users/apple/opt/anaconda3/envs/kaolin/lib/python3.7/site-packages/matplotlib/mpl-data/fonts/ttf/}]
%%   \setsansfont{DejaVuSans.ttf}[Path=\detokenize{/Users/apple/opt/anaconda3/envs/kaolin/lib/python3.7/site-packages/matplotlib/mpl-data/fonts/ttf/}]
%%   \setmonofont{DejaVuSansMono.ttf}[Path=\detokenize{/Users/apple/opt/anaconda3/envs/kaolin/lib/python3.7/site-packages/matplotlib/mpl-data/fonts/ttf/}]
%%
\begingroup%
\makeatletter%
\begin{pgfpicture}%
\pgfpathrectangle{\pgfpointorigin}{\pgfqpoint{5.789177in}{5.288537in}}%
\pgfusepath{use as bounding box, clip}%
\begin{pgfscope}%
\pgfsetbuttcap%
\pgfsetmiterjoin%
\definecolor{currentfill}{rgb}{1.000000,1.000000,1.000000}%
\pgfsetfillcolor{currentfill}%
\pgfsetlinewidth{0.000000pt}%
\definecolor{currentstroke}{rgb}{1.000000,1.000000,1.000000}%
\pgfsetstrokecolor{currentstroke}%
\pgfsetdash{}{0pt}%
\pgfpathmoveto{\pgfqpoint{0.000000in}{0.000000in}}%
\pgfpathlineto{\pgfqpoint{5.789177in}{0.000000in}}%
\pgfpathlineto{\pgfqpoint{5.789177in}{5.288537in}}%
\pgfpathlineto{\pgfqpoint{0.000000in}{5.288537in}}%
\pgfpathclose%
\pgfusepath{fill}%
\end{pgfscope}%
\begin{pgfscope}%
\pgfsetbuttcap%
\pgfsetmiterjoin%
\definecolor{currentfill}{rgb}{1.000000,1.000000,1.000000}%
\pgfsetfillcolor{currentfill}%
\pgfsetlinewidth{0.000000pt}%
\definecolor{currentstroke}{rgb}{0.000000,0.000000,0.000000}%
\pgfsetstrokecolor{currentstroke}%
\pgfsetstrokeopacity{0.000000}%
\pgfsetdash{}{0pt}%
\pgfpathmoveto{\pgfqpoint{0.608070in}{1.439775in}}%
\pgfpathlineto{\pgfqpoint{5.568070in}{1.439775in}}%
\pgfpathlineto{\pgfqpoint{5.568070in}{5.135775in}}%
\pgfpathlineto{\pgfqpoint{0.608070in}{5.135775in}}%
\pgfpathclose%
\pgfusepath{fill}%
\end{pgfscope}%
\begin{pgfscope}%
\pgfsetbuttcap%
\pgfsetroundjoin%
\definecolor{currentfill}{rgb}{0.000000,0.000000,0.000000}%
\pgfsetfillcolor{currentfill}%
\pgfsetlinewidth{0.803000pt}%
\definecolor{currentstroke}{rgb}{0.000000,0.000000,0.000000}%
\pgfsetstrokecolor{currentstroke}%
\pgfsetdash{}{0pt}%
\pgfsys@defobject{currentmarker}{\pgfqpoint{0.000000in}{-0.048611in}}{\pgfqpoint{0.000000in}{0.000000in}}{%
\pgfpathmoveto{\pgfqpoint{0.000000in}{0.000000in}}%
\pgfpathlineto{\pgfqpoint{0.000000in}{-0.048611in}}%
\pgfusepath{stroke,fill}%
}%
\begin{pgfscope}%
\pgfsys@transformshift{0.833525in}{1.439775in}%
\pgfsys@useobject{currentmarker}{}%
\end{pgfscope}%
\end{pgfscope}%
\begin{pgfscope}%
\definecolor{textcolor}{rgb}{0.000000,0.000000,0.000000}%
\pgfsetstrokecolor{textcolor}%
\pgfsetfillcolor{textcolor}%
\pgftext[x=0.381848in, y=0.310396in, left, base,rotate=45.000000]{\color{textcolor}\sffamily\fontsize{10.000000}{12.000000}\selectfont Pix3d(chair,no aug)}%
\end{pgfscope}%
\begin{pgfscope}%
\pgfsetbuttcap%
\pgfsetroundjoin%
\definecolor{currentfill}{rgb}{0.000000,0.000000,0.000000}%
\pgfsetfillcolor{currentfill}%
\pgfsetlinewidth{0.803000pt}%
\definecolor{currentstroke}{rgb}{0.000000,0.000000,0.000000}%
\pgfsetstrokecolor{currentstroke}%
\pgfsetdash{}{0pt}%
\pgfsys@defobject{currentmarker}{\pgfqpoint{0.000000in}{-0.048611in}}{\pgfqpoint{0.000000in}{0.000000in}}{%
\pgfpathmoveto{\pgfqpoint{0.000000in}{0.000000in}}%
\pgfpathlineto{\pgfqpoint{0.000000in}{-0.048611in}}%
\pgfusepath{stroke,fill}%
}%
\begin{pgfscope}%
\pgfsys@transformshift{1.585040in}{1.439775in}%
\pgfsys@useobject{currentmarker}{}%
\end{pgfscope}%
\end{pgfscope}%
\begin{pgfscope}%
\definecolor{textcolor}{rgb}{0.000000,0.000000,0.000000}%
\pgfsetstrokecolor{textcolor}%
\pgfsetfillcolor{textcolor}%
\pgftext[x=1.318129in, y=0.679928in, left, base,rotate=45.000000]{\color{textcolor}\sffamily\fontsize{10.000000}{12.000000}\selectfont Pix3d(chair)}%
\end{pgfscope}%
\begin{pgfscope}%
\pgfsetbuttcap%
\pgfsetroundjoin%
\definecolor{currentfill}{rgb}{0.000000,0.000000,0.000000}%
\pgfsetfillcolor{currentfill}%
\pgfsetlinewidth{0.803000pt}%
\definecolor{currentstroke}{rgb}{0.000000,0.000000,0.000000}%
\pgfsetstrokecolor{currentstroke}%
\pgfsetdash{}{0pt}%
\pgfsys@defobject{currentmarker}{\pgfqpoint{0.000000in}{-0.048611in}}{\pgfqpoint{0.000000in}{0.000000in}}{%
\pgfpathmoveto{\pgfqpoint{0.000000in}{0.000000in}}%
\pgfpathlineto{\pgfqpoint{0.000000in}{-0.048611in}}%
\pgfusepath{stroke,fill}%
}%
\begin{pgfscope}%
\pgfsys@transformshift{2.336555in}{1.439775in}%
\pgfsys@useobject{currentmarker}{}%
\end{pgfscope}%
\end{pgfscope}%
\begin{pgfscope}%
\definecolor{textcolor}{rgb}{0.000000,0.000000,0.000000}%
\pgfsetstrokecolor{textcolor}%
\pgfsetfillcolor{textcolor}%
\pgftext[x=2.088874in, y=0.718387in, left, base,rotate=45.000000]{\color{textcolor}\sffamily\fontsize{10.000000}{12.000000}\selectfont Textureless}%
\end{pgfscope}%
\begin{pgfscope}%
\pgfsetbuttcap%
\pgfsetroundjoin%
\definecolor{currentfill}{rgb}{0.000000,0.000000,0.000000}%
\pgfsetfillcolor{currentfill}%
\pgfsetlinewidth{0.803000pt}%
\definecolor{currentstroke}{rgb}{0.000000,0.000000,0.000000}%
\pgfsetstrokecolor{currentstroke}%
\pgfsetdash{}{0pt}%
\pgfsys@defobject{currentmarker}{\pgfqpoint{0.000000in}{-0.048611in}}{\pgfqpoint{0.000000in}{0.000000in}}{%
\pgfpathmoveto{\pgfqpoint{0.000000in}{0.000000in}}%
\pgfpathlineto{\pgfqpoint{0.000000in}{-0.048611in}}%
\pgfusepath{stroke,fill}%
}%
\begin{pgfscope}%
\pgfsys@transformshift{3.088070in}{1.439775in}%
\pgfsys@useobject{currentmarker}{}%
\end{pgfscope}%
\end{pgfscope}%
\begin{pgfscope}%
\definecolor{textcolor}{rgb}{0.000000,0.000000,0.000000}%
\pgfsetstrokecolor{textcolor}%
\pgfsetfillcolor{textcolor}%
\pgftext[x=2.676699in, y=0.391007in, left, base,rotate=45.000000]{\color{textcolor}\sffamily\fontsize{10.000000}{12.000000}\selectfont Textureless+Light}%
\end{pgfscope}%
\begin{pgfscope}%
\pgfsetbuttcap%
\pgfsetroundjoin%
\definecolor{currentfill}{rgb}{0.000000,0.000000,0.000000}%
\pgfsetfillcolor{currentfill}%
\pgfsetlinewidth{0.803000pt}%
\definecolor{currentstroke}{rgb}{0.000000,0.000000,0.000000}%
\pgfsetstrokecolor{currentstroke}%
\pgfsetdash{}{0pt}%
\pgfsys@defobject{currentmarker}{\pgfqpoint{0.000000in}{-0.048611in}}{\pgfqpoint{0.000000in}{0.000000in}}{%
\pgfpathmoveto{\pgfqpoint{0.000000in}{0.000000in}}%
\pgfpathlineto{\pgfqpoint{0.000000in}{-0.048611in}}%
\pgfusepath{stroke,fill}%
}%
\begin{pgfscope}%
\pgfsys@transformshift{3.839585in}{1.439775in}%
\pgfsys@useobject{currentmarker}{}%
\end{pgfscope}%
\end{pgfscope}%
\begin{pgfscope}%
\definecolor{textcolor}{rgb}{0.000000,0.000000,0.000000}%
\pgfsetstrokecolor{textcolor}%
\pgfsetfillcolor{textcolor}%
\pgftext[x=3.655755in, y=0.846088in, left, base,rotate=45.000000]{\color{textcolor}\sffamily\fontsize{10.000000}{12.000000}\selectfont Textured}%
\end{pgfscope}%
\begin{pgfscope}%
\pgfsetbuttcap%
\pgfsetroundjoin%
\definecolor{currentfill}{rgb}{0.000000,0.000000,0.000000}%
\pgfsetfillcolor{currentfill}%
\pgfsetlinewidth{0.803000pt}%
\definecolor{currentstroke}{rgb}{0.000000,0.000000,0.000000}%
\pgfsetstrokecolor{currentstroke}%
\pgfsetdash{}{0pt}%
\pgfsys@defobject{currentmarker}{\pgfqpoint{0.000000in}{-0.048611in}}{\pgfqpoint{0.000000in}{0.000000in}}{%
\pgfpathmoveto{\pgfqpoint{0.000000in}{0.000000in}}%
\pgfpathlineto{\pgfqpoint{0.000000in}{-0.048611in}}%
\pgfusepath{stroke,fill}%
}%
\begin{pgfscope}%
\pgfsys@transformshift{4.591100in}{1.439775in}%
\pgfsys@useobject{currentmarker}{}%
\end{pgfscope}%
\end{pgfscope}%
\begin{pgfscope}%
\definecolor{textcolor}{rgb}{0.000000,0.000000,0.000000}%
\pgfsetstrokecolor{textcolor}%
\pgfsetfillcolor{textcolor}%
\pgftext[x=4.243580in, y=0.518708in, left, base,rotate=45.000000]{\color{textcolor}\sffamily\fontsize{10.000000}{12.000000}\selectfont Textured+Light}%
\end{pgfscope}%
\begin{pgfscope}%
\pgfsetbuttcap%
\pgfsetroundjoin%
\definecolor{currentfill}{rgb}{0.000000,0.000000,0.000000}%
\pgfsetfillcolor{currentfill}%
\pgfsetlinewidth{0.803000pt}%
\definecolor{currentstroke}{rgb}{0.000000,0.000000,0.000000}%
\pgfsetstrokecolor{currentstroke}%
\pgfsetdash{}{0pt}%
\pgfsys@defobject{currentmarker}{\pgfqpoint{0.000000in}{-0.048611in}}{\pgfqpoint{0.000000in}{0.000000in}}{%
\pgfpathmoveto{\pgfqpoint{0.000000in}{0.000000in}}%
\pgfpathlineto{\pgfqpoint{0.000000in}{-0.048611in}}%
\pgfusepath{stroke,fill}%
}%
\begin{pgfscope}%
\pgfsys@transformshift{5.342615in}{1.439775in}%
\pgfsys@useobject{currentmarker}{}%
\end{pgfscope}%
\end{pgfscope}%
\begin{pgfscope}%
\definecolor{textcolor}{rgb}{0.000000,0.000000,0.000000}%
\pgfsetstrokecolor{textcolor}%
\pgfsetfillcolor{textcolor}%
\pgftext[x=5.070670in, y=0.669858in, left, base,rotate=45.000000]{\color{textcolor}\sffamily\fontsize{10.000000}{12.000000}\selectfont Multi-Object}%
\end{pgfscope}%
\begin{pgfscope}%
\definecolor{textcolor}{rgb}{0.000000,0.000000,0.000000}%
\pgfsetstrokecolor{textcolor}%
\pgfsetfillcolor{textcolor}%
\pgftext[x=3.088070in,y=0.234413in,,top]{\color{textcolor}\sffamily\fontsize{10.000000}{12.000000}\selectfont Datasets}%
\end{pgfscope}%
\begin{pgfscope}%
\pgfsetbuttcap%
\pgfsetroundjoin%
\definecolor{currentfill}{rgb}{0.000000,0.000000,0.000000}%
\pgfsetfillcolor{currentfill}%
\pgfsetlinewidth{0.803000pt}%
\definecolor{currentstroke}{rgb}{0.000000,0.000000,0.000000}%
\pgfsetstrokecolor{currentstroke}%
\pgfsetdash{}{0pt}%
\pgfsys@defobject{currentmarker}{\pgfqpoint{-0.048611in}{0.000000in}}{\pgfqpoint{-0.000000in}{0.000000in}}{%
\pgfpathmoveto{\pgfqpoint{-0.000000in}{0.000000in}}%
\pgfpathlineto{\pgfqpoint{-0.048611in}{0.000000in}}%
\pgfusepath{stroke,fill}%
}%
\begin{pgfscope}%
\pgfsys@transformshift{0.608070in}{1.439775in}%
\pgfsys@useobject{currentmarker}{}%
\end{pgfscope}%
\end{pgfscope}%
\begin{pgfscope}%
\definecolor{textcolor}{rgb}{0.000000,0.000000,0.000000}%
\pgfsetstrokecolor{textcolor}%
\pgfsetfillcolor{textcolor}%
\pgftext[x=0.289968in, y=1.387014in, left, base]{\color{textcolor}\sffamily\fontsize{10.000000}{12.000000}\selectfont 0.0}%
\end{pgfscope}%
\begin{pgfscope}%
\pgfsetbuttcap%
\pgfsetroundjoin%
\definecolor{currentfill}{rgb}{0.000000,0.000000,0.000000}%
\pgfsetfillcolor{currentfill}%
\pgfsetlinewidth{0.803000pt}%
\definecolor{currentstroke}{rgb}{0.000000,0.000000,0.000000}%
\pgfsetstrokecolor{currentstroke}%
\pgfsetdash{}{0pt}%
\pgfsys@defobject{currentmarker}{\pgfqpoint{-0.048611in}{0.000000in}}{\pgfqpoint{-0.000000in}{0.000000in}}{%
\pgfpathmoveto{\pgfqpoint{-0.000000in}{0.000000in}}%
\pgfpathlineto{\pgfqpoint{-0.048611in}{0.000000in}}%
\pgfusepath{stroke,fill}%
}%
\begin{pgfscope}%
\pgfsys@transformshift{0.608070in}{2.178975in}%
\pgfsys@useobject{currentmarker}{}%
\end{pgfscope}%
\end{pgfscope}%
\begin{pgfscope}%
\definecolor{textcolor}{rgb}{0.000000,0.000000,0.000000}%
\pgfsetstrokecolor{textcolor}%
\pgfsetfillcolor{textcolor}%
\pgftext[x=0.289968in, y=2.126214in, left, base]{\color{textcolor}\sffamily\fontsize{10.000000}{12.000000}\selectfont 0.1}%
\end{pgfscope}%
\begin{pgfscope}%
\pgfsetbuttcap%
\pgfsetroundjoin%
\definecolor{currentfill}{rgb}{0.000000,0.000000,0.000000}%
\pgfsetfillcolor{currentfill}%
\pgfsetlinewidth{0.803000pt}%
\definecolor{currentstroke}{rgb}{0.000000,0.000000,0.000000}%
\pgfsetstrokecolor{currentstroke}%
\pgfsetdash{}{0pt}%
\pgfsys@defobject{currentmarker}{\pgfqpoint{-0.048611in}{0.000000in}}{\pgfqpoint{-0.000000in}{0.000000in}}{%
\pgfpathmoveto{\pgfqpoint{-0.000000in}{0.000000in}}%
\pgfpathlineto{\pgfqpoint{-0.048611in}{0.000000in}}%
\pgfusepath{stroke,fill}%
}%
\begin{pgfscope}%
\pgfsys@transformshift{0.608070in}{2.918175in}%
\pgfsys@useobject{currentmarker}{}%
\end{pgfscope}%
\end{pgfscope}%
\begin{pgfscope}%
\definecolor{textcolor}{rgb}{0.000000,0.000000,0.000000}%
\pgfsetstrokecolor{textcolor}%
\pgfsetfillcolor{textcolor}%
\pgftext[x=0.289968in, y=2.865414in, left, base]{\color{textcolor}\sffamily\fontsize{10.000000}{12.000000}\selectfont 0.2}%
\end{pgfscope}%
\begin{pgfscope}%
\pgfsetbuttcap%
\pgfsetroundjoin%
\definecolor{currentfill}{rgb}{0.000000,0.000000,0.000000}%
\pgfsetfillcolor{currentfill}%
\pgfsetlinewidth{0.803000pt}%
\definecolor{currentstroke}{rgb}{0.000000,0.000000,0.000000}%
\pgfsetstrokecolor{currentstroke}%
\pgfsetdash{}{0pt}%
\pgfsys@defobject{currentmarker}{\pgfqpoint{-0.048611in}{0.000000in}}{\pgfqpoint{-0.000000in}{0.000000in}}{%
\pgfpathmoveto{\pgfqpoint{-0.000000in}{0.000000in}}%
\pgfpathlineto{\pgfqpoint{-0.048611in}{0.000000in}}%
\pgfusepath{stroke,fill}%
}%
\begin{pgfscope}%
\pgfsys@transformshift{0.608070in}{3.657375in}%
\pgfsys@useobject{currentmarker}{}%
\end{pgfscope}%
\end{pgfscope}%
\begin{pgfscope}%
\definecolor{textcolor}{rgb}{0.000000,0.000000,0.000000}%
\pgfsetstrokecolor{textcolor}%
\pgfsetfillcolor{textcolor}%
\pgftext[x=0.289968in, y=3.604614in, left, base]{\color{textcolor}\sffamily\fontsize{10.000000}{12.000000}\selectfont 0.3}%
\end{pgfscope}%
\begin{pgfscope}%
\pgfsetbuttcap%
\pgfsetroundjoin%
\definecolor{currentfill}{rgb}{0.000000,0.000000,0.000000}%
\pgfsetfillcolor{currentfill}%
\pgfsetlinewidth{0.803000pt}%
\definecolor{currentstroke}{rgb}{0.000000,0.000000,0.000000}%
\pgfsetstrokecolor{currentstroke}%
\pgfsetdash{}{0pt}%
\pgfsys@defobject{currentmarker}{\pgfqpoint{-0.048611in}{0.000000in}}{\pgfqpoint{-0.000000in}{0.000000in}}{%
\pgfpathmoveto{\pgfqpoint{-0.000000in}{0.000000in}}%
\pgfpathlineto{\pgfqpoint{-0.048611in}{0.000000in}}%
\pgfusepath{stroke,fill}%
}%
\begin{pgfscope}%
\pgfsys@transformshift{0.608070in}{4.396575in}%
\pgfsys@useobject{currentmarker}{}%
\end{pgfscope}%
\end{pgfscope}%
\begin{pgfscope}%
\definecolor{textcolor}{rgb}{0.000000,0.000000,0.000000}%
\pgfsetstrokecolor{textcolor}%
\pgfsetfillcolor{textcolor}%
\pgftext[x=0.289968in, y=4.343814in, left, base]{\color{textcolor}\sffamily\fontsize{10.000000}{12.000000}\selectfont 0.4}%
\end{pgfscope}%
\begin{pgfscope}%
\pgfsetbuttcap%
\pgfsetroundjoin%
\definecolor{currentfill}{rgb}{0.000000,0.000000,0.000000}%
\pgfsetfillcolor{currentfill}%
\pgfsetlinewidth{0.803000pt}%
\definecolor{currentstroke}{rgb}{0.000000,0.000000,0.000000}%
\pgfsetstrokecolor{currentstroke}%
\pgfsetdash{}{0pt}%
\pgfsys@defobject{currentmarker}{\pgfqpoint{-0.048611in}{0.000000in}}{\pgfqpoint{-0.000000in}{0.000000in}}{%
\pgfpathmoveto{\pgfqpoint{-0.000000in}{0.000000in}}%
\pgfpathlineto{\pgfqpoint{-0.048611in}{0.000000in}}%
\pgfusepath{stroke,fill}%
}%
\begin{pgfscope}%
\pgfsys@transformshift{0.608070in}{5.135775in}%
\pgfsys@useobject{currentmarker}{}%
\end{pgfscope}%
\end{pgfscope}%
\begin{pgfscope}%
\definecolor{textcolor}{rgb}{0.000000,0.000000,0.000000}%
\pgfsetstrokecolor{textcolor}%
\pgfsetfillcolor{textcolor}%
\pgftext[x=0.289968in, y=5.083014in, left, base]{\color{textcolor}\sffamily\fontsize{10.000000}{12.000000}\selectfont 0.5}%
\end{pgfscope}%
\begin{pgfscope}%
\definecolor{textcolor}{rgb}{0.000000,0.000000,0.000000}%
\pgfsetstrokecolor{textcolor}%
\pgfsetfillcolor{textcolor}%
\pgftext[x=0.234413in,y=3.287775in,,bottom,rotate=90.000000]{\color{textcolor}\sffamily\fontsize{10.000000}{12.000000}\selectfont IoU}%
\end{pgfscope}%
\begin{pgfscope}%
\pgfpathrectangle{\pgfqpoint{0.608070in}{1.439775in}}{\pgfqpoint{4.960000in}{3.696000in}}%
\pgfusepath{clip}%
\pgfsetrectcap%
\pgfsetroundjoin%
\pgfsetlinewidth{1.505625pt}%
\definecolor{currentstroke}{rgb}{0.121569,0.466667,0.705882}%
\pgfsetstrokecolor{currentstroke}%
\pgfsetdash{}{0pt}%
\pgfpathmoveto{\pgfqpoint{0.833525in}{3.507318in}}%
\pgfpathlineto{\pgfqpoint{1.585040in}{3.882831in}}%
\pgfpathlineto{\pgfqpoint{2.336555in}{2.768857in}}%
\pgfpathlineto{\pgfqpoint{3.088070in}{2.496831in}}%
\pgfpathlineto{\pgfqpoint{3.839585in}{3.078582in}}%
\pgfpathlineto{\pgfqpoint{4.591100in}{3.081538in}}%
\pgfpathlineto{\pgfqpoint{5.342615in}{3.193158in}}%
\pgfusepath{stroke}%
\end{pgfscope}%
\begin{pgfscope}%
\pgfpathrectangle{\pgfqpoint{0.608070in}{1.439775in}}{\pgfqpoint{4.960000in}{3.696000in}}%
\pgfusepath{clip}%
\pgfsetbuttcap%
\pgfsetroundjoin%
\definecolor{currentfill}{rgb}{0.121569,0.466667,0.705882}%
\pgfsetfillcolor{currentfill}%
\pgfsetlinewidth{1.003750pt}%
\definecolor{currentstroke}{rgb}{0.121569,0.466667,0.705882}%
\pgfsetstrokecolor{currentstroke}%
\pgfsetdash{}{0pt}%
\pgfsys@defobject{currentmarker}{\pgfqpoint{-0.041667in}{-0.041667in}}{\pgfqpoint{0.041667in}{0.041667in}}{%
\pgfpathmoveto{\pgfqpoint{0.000000in}{-0.041667in}}%
\pgfpathcurveto{\pgfqpoint{0.011050in}{-0.041667in}}{\pgfqpoint{0.021649in}{-0.037276in}}{\pgfqpoint{0.029463in}{-0.029463in}}%
\pgfpathcurveto{\pgfqpoint{0.037276in}{-0.021649in}}{\pgfqpoint{0.041667in}{-0.011050in}}{\pgfqpoint{0.041667in}{0.000000in}}%
\pgfpathcurveto{\pgfqpoint{0.041667in}{0.011050in}}{\pgfqpoint{0.037276in}{0.021649in}}{\pgfqpoint{0.029463in}{0.029463in}}%
\pgfpathcurveto{\pgfqpoint{0.021649in}{0.037276in}}{\pgfqpoint{0.011050in}{0.041667in}}{\pgfqpoint{0.000000in}{0.041667in}}%
\pgfpathcurveto{\pgfqpoint{-0.011050in}{0.041667in}}{\pgfqpoint{-0.021649in}{0.037276in}}{\pgfqpoint{-0.029463in}{0.029463in}}%
\pgfpathcurveto{\pgfqpoint{-0.037276in}{0.021649in}}{\pgfqpoint{-0.041667in}{0.011050in}}{\pgfqpoint{-0.041667in}{0.000000in}}%
\pgfpathcurveto{\pgfqpoint{-0.041667in}{-0.011050in}}{\pgfqpoint{-0.037276in}{-0.021649in}}{\pgfqpoint{-0.029463in}{-0.029463in}}%
\pgfpathcurveto{\pgfqpoint{-0.021649in}{-0.037276in}}{\pgfqpoint{-0.011050in}{-0.041667in}}{\pgfqpoint{0.000000in}{-0.041667in}}%
\pgfpathclose%
\pgfusepath{stroke,fill}%
}%
\begin{pgfscope}%
\pgfsys@transformshift{0.833525in}{3.507318in}%
\pgfsys@useobject{currentmarker}{}%
\end{pgfscope}%
\begin{pgfscope}%
\pgfsys@transformshift{1.585040in}{3.882831in}%
\pgfsys@useobject{currentmarker}{}%
\end{pgfscope}%
\begin{pgfscope}%
\pgfsys@transformshift{2.336555in}{2.768857in}%
\pgfsys@useobject{currentmarker}{}%
\end{pgfscope}%
\begin{pgfscope}%
\pgfsys@transformshift{3.088070in}{2.496831in}%
\pgfsys@useobject{currentmarker}{}%
\end{pgfscope}%
\begin{pgfscope}%
\pgfsys@transformshift{3.839585in}{3.078582in}%
\pgfsys@useobject{currentmarker}{}%
\end{pgfscope}%
\begin{pgfscope}%
\pgfsys@transformshift{4.591100in}{3.081538in}%
\pgfsys@useobject{currentmarker}{}%
\end{pgfscope}%
\begin{pgfscope}%
\pgfsys@transformshift{5.342615in}{3.193158in}%
\pgfsys@useobject{currentmarker}{}%
\end{pgfscope}%
\end{pgfscope}%
\begin{pgfscope}%
\pgfpathrectangle{\pgfqpoint{0.608070in}{1.439775in}}{\pgfqpoint{4.960000in}{3.696000in}}%
\pgfusepath{clip}%
\pgfsetrectcap%
\pgfsetroundjoin%
\pgfsetlinewidth{1.505625pt}%
\definecolor{currentstroke}{rgb}{1.000000,0.498039,0.054902}%
\pgfsetstrokecolor{currentstroke}%
\pgfsetdash{}{0pt}%
\pgfpathmoveto{\pgfqpoint{0.833525in}{3.431180in}}%
\pgfpathlineto{\pgfqpoint{1.585040in}{3.595282in}}%
\pgfpathlineto{\pgfqpoint{2.336555in}{1.992697in}}%
\pgfpathlineto{\pgfqpoint{3.088070in}{2.198194in}}%
\pgfpathlineto{\pgfqpoint{3.839585in}{2.267679in}}%
\pgfpathlineto{\pgfqpoint{4.591100in}{2.172322in}}%
\pgfpathlineto{\pgfqpoint{5.342615in}{2.136841in}}%
\pgfusepath{stroke}%
\end{pgfscope}%
\begin{pgfscope}%
\pgfpathrectangle{\pgfqpoint{0.608070in}{1.439775in}}{\pgfqpoint{4.960000in}{3.696000in}}%
\pgfusepath{clip}%
\pgfsetbuttcap%
\pgfsetmiterjoin%
\definecolor{currentfill}{rgb}{1.000000,0.498039,0.054902}%
\pgfsetfillcolor{currentfill}%
\pgfsetlinewidth{1.003750pt}%
\definecolor{currentstroke}{rgb}{1.000000,0.498039,0.054902}%
\pgfsetstrokecolor{currentstroke}%
\pgfsetdash{}{0pt}%
\pgfsys@defobject{currentmarker}{\pgfqpoint{-0.041667in}{-0.041667in}}{\pgfqpoint{0.041667in}{0.041667in}}{%
\pgfpathmoveto{\pgfqpoint{-0.000000in}{-0.041667in}}%
\pgfpathlineto{\pgfqpoint{0.041667in}{0.041667in}}%
\pgfpathlineto{\pgfqpoint{-0.041667in}{0.041667in}}%
\pgfpathclose%
\pgfusepath{stroke,fill}%
}%
\begin{pgfscope}%
\pgfsys@transformshift{0.833525in}{3.431180in}%
\pgfsys@useobject{currentmarker}{}%
\end{pgfscope}%
\begin{pgfscope}%
\pgfsys@transformshift{1.585040in}{3.595282in}%
\pgfsys@useobject{currentmarker}{}%
\end{pgfscope}%
\begin{pgfscope}%
\pgfsys@transformshift{2.336555in}{1.992697in}%
\pgfsys@useobject{currentmarker}{}%
\end{pgfscope}%
\begin{pgfscope}%
\pgfsys@transformshift{3.088070in}{2.198194in}%
\pgfsys@useobject{currentmarker}{}%
\end{pgfscope}%
\begin{pgfscope}%
\pgfsys@transformshift{3.839585in}{2.267679in}%
\pgfsys@useobject{currentmarker}{}%
\end{pgfscope}%
\begin{pgfscope}%
\pgfsys@transformshift{4.591100in}{2.172322in}%
\pgfsys@useobject{currentmarker}{}%
\end{pgfscope}%
\begin{pgfscope}%
\pgfsys@transformshift{5.342615in}{2.136841in}%
\pgfsys@useobject{currentmarker}{}%
\end{pgfscope}%
\end{pgfscope}%
\begin{pgfscope}%
\pgfsetrectcap%
\pgfsetmiterjoin%
\pgfsetlinewidth{0.803000pt}%
\definecolor{currentstroke}{rgb}{0.000000,0.000000,0.000000}%
\pgfsetstrokecolor{currentstroke}%
\pgfsetdash{}{0pt}%
\pgfpathmoveto{\pgfqpoint{0.608070in}{1.439775in}}%
\pgfpathlineto{\pgfqpoint{0.608070in}{5.135775in}}%
\pgfusepath{stroke}%
\end{pgfscope}%
\begin{pgfscope}%
\pgfsetrectcap%
\pgfsetmiterjoin%
\pgfsetlinewidth{0.803000pt}%
\definecolor{currentstroke}{rgb}{0.000000,0.000000,0.000000}%
\pgfsetstrokecolor{currentstroke}%
\pgfsetdash{}{0pt}%
\pgfpathmoveto{\pgfqpoint{5.568070in}{1.439775in}}%
\pgfpathlineto{\pgfqpoint{5.568070in}{5.135775in}}%
\pgfusepath{stroke}%
\end{pgfscope}%
\begin{pgfscope}%
\pgfsetrectcap%
\pgfsetmiterjoin%
\pgfsetlinewidth{0.803000pt}%
\definecolor{currentstroke}{rgb}{0.000000,0.000000,0.000000}%
\pgfsetstrokecolor{currentstroke}%
\pgfsetdash{}{0pt}%
\pgfpathmoveto{\pgfqpoint{0.608070in}{1.439775in}}%
\pgfpathlineto{\pgfqpoint{5.568070in}{1.439775in}}%
\pgfusepath{stroke}%
\end{pgfscope}%
\begin{pgfscope}%
\pgfsetrectcap%
\pgfsetmiterjoin%
\pgfsetlinewidth{0.803000pt}%
\definecolor{currentstroke}{rgb}{0.000000,0.000000,0.000000}%
\pgfsetstrokecolor{currentstroke}%
\pgfsetdash{}{0pt}%
\pgfpathmoveto{\pgfqpoint{0.608070in}{5.135775in}}%
\pgfpathlineto{\pgfqpoint{5.568070in}{5.135775in}}%
\pgfusepath{stroke}%
\end{pgfscope}%
\begin{pgfscope}%
\pgfsetbuttcap%
\pgfsetmiterjoin%
\definecolor{currentfill}{rgb}{1.000000,1.000000,1.000000}%
\pgfsetfillcolor{currentfill}%
\pgfsetfillopacity{0.800000}%
\pgfsetlinewidth{1.003750pt}%
\definecolor{currentstroke}{rgb}{0.800000,0.800000,0.800000}%
\pgfsetstrokecolor{currentstroke}%
\pgfsetstrokeopacity{0.800000}%
\pgfsetdash{}{0pt}%
\pgfpathmoveto{\pgfqpoint{4.256682in}{4.616950in}}%
\pgfpathlineto{\pgfqpoint{5.470848in}{4.616950in}}%
\pgfpathquadraticcurveto{\pgfqpoint{5.498626in}{4.616950in}}{\pgfqpoint{5.498626in}{4.644727in}}%
\pgfpathlineto{\pgfqpoint{5.498626in}{5.038553in}}%
\pgfpathquadraticcurveto{\pgfqpoint{5.498626in}{5.066331in}}{\pgfqpoint{5.470848in}{5.066331in}}%
\pgfpathlineto{\pgfqpoint{4.256682in}{5.066331in}}%
\pgfpathquadraticcurveto{\pgfqpoint{4.228904in}{5.066331in}}{\pgfqpoint{4.228904in}{5.038553in}}%
\pgfpathlineto{\pgfqpoint{4.228904in}{4.644727in}}%
\pgfpathquadraticcurveto{\pgfqpoint{4.228904in}{4.616950in}}{\pgfqpoint{4.256682in}{4.616950in}}%
\pgfpathclose%
\pgfusepath{stroke,fill}%
\end{pgfscope}%
\begin{pgfscope}%
\pgfsetrectcap%
\pgfsetroundjoin%
\pgfsetlinewidth{1.505625pt}%
\definecolor{currentstroke}{rgb}{0.121569,0.466667,0.705882}%
\pgfsetstrokecolor{currentstroke}%
\pgfsetdash{}{0pt}%
\pgfpathmoveto{\pgfqpoint{4.284460in}{4.953863in}}%
\pgfpathlineto{\pgfqpoint{4.562238in}{4.953863in}}%
\pgfusepath{stroke}%
\end{pgfscope}%
\begin{pgfscope}%
\pgfsetbuttcap%
\pgfsetroundjoin%
\definecolor{currentfill}{rgb}{0.121569,0.466667,0.705882}%
\pgfsetfillcolor{currentfill}%
\pgfsetlinewidth{1.003750pt}%
\definecolor{currentstroke}{rgb}{0.121569,0.466667,0.705882}%
\pgfsetstrokecolor{currentstroke}%
\pgfsetdash{}{0pt}%
\pgfsys@defobject{currentmarker}{\pgfqpoint{-0.041667in}{-0.041667in}}{\pgfqpoint{0.041667in}{0.041667in}}{%
\pgfpathmoveto{\pgfqpoint{0.000000in}{-0.041667in}}%
\pgfpathcurveto{\pgfqpoint{0.011050in}{-0.041667in}}{\pgfqpoint{0.021649in}{-0.037276in}}{\pgfqpoint{0.029463in}{-0.029463in}}%
\pgfpathcurveto{\pgfqpoint{0.037276in}{-0.021649in}}{\pgfqpoint{0.041667in}{-0.011050in}}{\pgfqpoint{0.041667in}{0.000000in}}%
\pgfpathcurveto{\pgfqpoint{0.041667in}{0.011050in}}{\pgfqpoint{0.037276in}{0.021649in}}{\pgfqpoint{0.029463in}{0.029463in}}%
\pgfpathcurveto{\pgfqpoint{0.021649in}{0.037276in}}{\pgfqpoint{0.011050in}{0.041667in}}{\pgfqpoint{0.000000in}{0.041667in}}%
\pgfpathcurveto{\pgfqpoint{-0.011050in}{0.041667in}}{\pgfqpoint{-0.021649in}{0.037276in}}{\pgfqpoint{-0.029463in}{0.029463in}}%
\pgfpathcurveto{\pgfqpoint{-0.037276in}{0.021649in}}{\pgfqpoint{-0.041667in}{0.011050in}}{\pgfqpoint{-0.041667in}{0.000000in}}%
\pgfpathcurveto{\pgfqpoint{-0.041667in}{-0.011050in}}{\pgfqpoint{-0.037276in}{-0.021649in}}{\pgfqpoint{-0.029463in}{-0.029463in}}%
\pgfpathcurveto{\pgfqpoint{-0.021649in}{-0.037276in}}{\pgfqpoint{-0.011050in}{-0.041667in}}{\pgfqpoint{0.000000in}{-0.041667in}}%
\pgfpathclose%
\pgfusepath{stroke,fill}%
}%
\begin{pgfscope}%
\pgfsys@transformshift{4.423349in}{4.953863in}%
\pgfsys@useobject{currentmarker}{}%
\end{pgfscope}%
\end{pgfscope}%
\begin{pgfscope}%
\definecolor{textcolor}{rgb}{0.000000,0.000000,0.000000}%
\pgfsetstrokecolor{textcolor}%
\pgfsetfillcolor{textcolor}%
\pgftext[x=4.673349in,y=4.905252in,left,base]{\color{textcolor}\sffamily\fontsize{10.000000}{12.000000}\selectfont Pix2Vox++}%
\end{pgfscope}%
\begin{pgfscope}%
\pgfsetrectcap%
\pgfsetroundjoin%
\pgfsetlinewidth{1.505625pt}%
\definecolor{currentstroke}{rgb}{1.000000,0.498039,0.054902}%
\pgfsetstrokecolor{currentstroke}%
\pgfsetdash{}{0pt}%
\pgfpathmoveto{\pgfqpoint{4.284460in}{4.750006in}}%
\pgfpathlineto{\pgfqpoint{4.562238in}{4.750006in}}%
\pgfusepath{stroke}%
\end{pgfscope}%
\begin{pgfscope}%
\pgfsetbuttcap%
\pgfsetmiterjoin%
\definecolor{currentfill}{rgb}{1.000000,0.498039,0.054902}%
\pgfsetfillcolor{currentfill}%
\pgfsetlinewidth{1.003750pt}%
\definecolor{currentstroke}{rgb}{1.000000,0.498039,0.054902}%
\pgfsetstrokecolor{currentstroke}%
\pgfsetdash{}{0pt}%
\pgfsys@defobject{currentmarker}{\pgfqpoint{-0.041667in}{-0.041667in}}{\pgfqpoint{0.041667in}{0.041667in}}{%
\pgfpathmoveto{\pgfqpoint{-0.000000in}{-0.041667in}}%
\pgfpathlineto{\pgfqpoint{0.041667in}{0.041667in}}%
\pgfpathlineto{\pgfqpoint{-0.041667in}{0.041667in}}%
\pgfpathclose%
\pgfusepath{stroke,fill}%
}%
\begin{pgfscope}%
\pgfsys@transformshift{4.423349in}{4.750006in}%
\pgfsys@useobject{currentmarker}{}%
\end{pgfscope}%
\end{pgfscope}%
\begin{pgfscope}%
\definecolor{textcolor}{rgb}{0.000000,0.000000,0.000000}%
\pgfsetstrokecolor{textcolor}%
\pgfsetfillcolor{textcolor}%
\pgftext[x=4.673349in,y=4.701395in,left,base]{\color{textcolor}\sffamily\fontsize{10.000000}{12.000000}\selectfont Pix2Vox}%
\end{pgfscope}%
\end{pgfpicture}%
\makeatother%
\endgroup%
}
%    \caption{Line plot for the \gls{iou}  for baseline trained on chair dataset with different domain randomization parameters and tested on real dataset.
%    We see a dip in performance near textureless dataset, but it gradually increases with addition of domain randomization parameter.}
%    \label{fig:ablation1}
%\end{figure}


\subsection{Domain Randomization with Mixed Training}\label{subsec:domain-randomisation-with-mixed-training}
To reiterate,the performance of Pix2Vox++ on real chair dataset(Pix3D) is 0.2797 and 0.3305, without and with 2D augmentations.
For Pix2Vox, it was 0.2694 and 0.2916, with and without 2D augmentation, respectively.

\begin{figure}[ht]
    \centering
    \resizebox{0.9\textwidth}{!}{%% Creator: Matplotlib, PGF backend
%%
%% To include the figure in your LaTeX document, write
%%   \input{<filename>.pgf}
%%
%% Make sure the required packages are loaded in your preamble
%%   \usepackage{pgf}
%%
%% Figures using additional raster images can only be included by \input if
%% they are in the same directory as the main LaTeX file. For loading figures
%% from other directories you can use the `import` package
%%   \usepackage{import}
%%
%% and then include the figures with
%%   \import{<path to file>}{<filename>.pgf}
%%
%% Matplotlib used the following preamble
%%   \usepackage{fontspec}
%%   \setmainfont{DejaVuSerif.ttf}[Path=\detokenize{/Users/apple/opt/anaconda3/envs/kaolin/lib/python3.7/site-packages/matplotlib/mpl-data/fonts/ttf/}]
%%   \setsansfont{DejaVuSans.ttf}[Path=\detokenize{/Users/apple/opt/anaconda3/envs/kaolin/lib/python3.7/site-packages/matplotlib/mpl-data/fonts/ttf/}]
%%   \setmonofont{DejaVuSansMono.ttf}[Path=\detokenize{/Users/apple/opt/anaconda3/envs/kaolin/lib/python3.7/site-packages/matplotlib/mpl-data/fonts/ttf/}]
%%
\begingroup%
\makeatletter%
\begin{pgfpicture}%
\pgfpathrectangle{\pgfpointorigin}{\pgfqpoint{6.294042in}{4.696324in}}%
\pgfusepath{use as bounding box, clip}%
\begin{pgfscope}%
\pgfsetbuttcap%
\pgfsetmiterjoin%
\definecolor{currentfill}{rgb}{1.000000,1.000000,1.000000}%
\pgfsetfillcolor{currentfill}%
\pgfsetlinewidth{0.000000pt}%
\definecolor{currentstroke}{rgb}{1.000000,1.000000,1.000000}%
\pgfsetstrokecolor{currentstroke}%
\pgfsetdash{}{0pt}%
\pgfpathmoveto{\pgfqpoint{0.000000in}{0.000000in}}%
\pgfpathlineto{\pgfqpoint{6.294042in}{0.000000in}}%
\pgfpathlineto{\pgfqpoint{6.294042in}{4.696324in}}%
\pgfpathlineto{\pgfqpoint{0.000000in}{4.696324in}}%
\pgfpathclose%
\pgfusepath{fill}%
\end{pgfscope}%
\begin{pgfscope}%
\pgfsetbuttcap%
\pgfsetmiterjoin%
\definecolor{currentfill}{rgb}{1.000000,1.000000,1.000000}%
\pgfsetfillcolor{currentfill}%
\pgfsetlinewidth{0.000000pt}%
\definecolor{currentstroke}{rgb}{0.000000,0.000000,0.000000}%
\pgfsetstrokecolor{currentstroke}%
\pgfsetstrokeopacity{0.000000}%
\pgfsetdash{}{0pt}%
\pgfpathmoveto{\pgfqpoint{0.608070in}{1.359165in}}%
\pgfpathlineto{\pgfqpoint{6.194042in}{1.359165in}}%
\pgfpathlineto{\pgfqpoint{6.194042in}{4.386363in}}%
\pgfpathlineto{\pgfqpoint{0.608070in}{4.386363in}}%
\pgfpathclose%
\pgfusepath{fill}%
\end{pgfscope}%
\begin{pgfscope}%
\pgfpathrectangle{\pgfqpoint{0.608070in}{1.359165in}}{\pgfqpoint{5.585972in}{3.027198in}}%
\pgfusepath{clip}%
\pgfsetbuttcap%
\pgfsetmiterjoin%
\definecolor{currentfill}{rgb}{0.121569,0.466667,0.705882}%
\pgfsetfillcolor{currentfill}%
\pgfsetlinewidth{0.000000pt}%
\definecolor{currentstroke}{rgb}{0.000000,0.000000,0.000000}%
\pgfsetstrokecolor{currentstroke}%
\pgfsetstrokeopacity{0.000000}%
\pgfsetdash{}{0pt}%
\pgfpathmoveto{\pgfqpoint{0.861978in}{1.359165in}}%
\pgfpathlineto{\pgfqpoint{1.249295in}{1.359165in}}%
\pgfpathlineto{\pgfqpoint{1.249295in}{3.567808in}}%
\pgfpathlineto{\pgfqpoint{0.861978in}{3.567808in}}%
\pgfpathclose%
\pgfusepath{fill}%
\end{pgfscope}%
\begin{pgfscope}%
\pgfpathrectangle{\pgfqpoint{0.608070in}{1.359165in}}{\pgfqpoint{5.585972in}{3.027198in}}%
\pgfusepath{clip}%
\pgfsetbuttcap%
\pgfsetmiterjoin%
\definecolor{currentfill}{rgb}{0.121569,0.466667,0.705882}%
\pgfsetfillcolor{currentfill}%
\pgfsetlinewidth{0.000000pt}%
\definecolor{currentstroke}{rgb}{0.000000,0.000000,0.000000}%
\pgfsetstrokecolor{currentstroke}%
\pgfsetstrokeopacity{0.000000}%
\pgfsetdash{}{0pt}%
\pgfpathmoveto{\pgfqpoint{1.722682in}{1.359165in}}%
\pgfpathlineto{\pgfqpoint{2.109999in}{1.359165in}}%
\pgfpathlineto{\pgfqpoint{2.109999in}{3.548434in}}%
\pgfpathlineto{\pgfqpoint{1.722682in}{3.548434in}}%
\pgfpathclose%
\pgfusepath{fill}%
\end{pgfscope}%
\begin{pgfscope}%
\pgfpathrectangle{\pgfqpoint{0.608070in}{1.359165in}}{\pgfqpoint{5.585972in}{3.027198in}}%
\pgfusepath{clip}%
\pgfsetbuttcap%
\pgfsetmiterjoin%
\definecolor{currentfill}{rgb}{0.121569,0.466667,0.705882}%
\pgfsetfillcolor{currentfill}%
\pgfsetlinewidth{0.000000pt}%
\definecolor{currentstroke}{rgb}{0.000000,0.000000,0.000000}%
\pgfsetstrokecolor{currentstroke}%
\pgfsetstrokeopacity{0.000000}%
\pgfsetdash{}{0pt}%
\pgfpathmoveto{\pgfqpoint{2.583387in}{1.359165in}}%
\pgfpathlineto{\pgfqpoint{2.970704in}{1.359165in}}%
\pgfpathlineto{\pgfqpoint{2.970704in}{3.466094in}}%
\pgfpathlineto{\pgfqpoint{2.583387in}{3.466094in}}%
\pgfpathclose%
\pgfusepath{fill}%
\end{pgfscope}%
\begin{pgfscope}%
\pgfpathrectangle{\pgfqpoint{0.608070in}{1.359165in}}{\pgfqpoint{5.585972in}{3.027198in}}%
\pgfusepath{clip}%
\pgfsetbuttcap%
\pgfsetmiterjoin%
\definecolor{currentfill}{rgb}{0.121569,0.466667,0.705882}%
\pgfsetfillcolor{currentfill}%
\pgfsetlinewidth{0.000000pt}%
\definecolor{currentstroke}{rgb}{0.000000,0.000000,0.000000}%
\pgfsetstrokecolor{currentstroke}%
\pgfsetstrokeopacity{0.000000}%
\pgfsetdash{}{0pt}%
\pgfpathmoveto{\pgfqpoint{3.444091in}{1.359165in}}%
\pgfpathlineto{\pgfqpoint{3.831408in}{1.359165in}}%
\pgfpathlineto{\pgfqpoint{3.831408in}{3.432795in}}%
\pgfpathlineto{\pgfqpoint{3.444091in}{3.432795in}}%
\pgfpathclose%
\pgfusepath{fill}%
\end{pgfscope}%
\begin{pgfscope}%
\pgfpathrectangle{\pgfqpoint{0.608070in}{1.359165in}}{\pgfqpoint{5.585972in}{3.027198in}}%
\pgfusepath{clip}%
\pgfsetbuttcap%
\pgfsetmiterjoin%
\definecolor{currentfill}{rgb}{0.121569,0.466667,0.705882}%
\pgfsetfillcolor{currentfill}%
\pgfsetlinewidth{0.000000pt}%
\definecolor{currentstroke}{rgb}{0.000000,0.000000,0.000000}%
\pgfsetstrokecolor{currentstroke}%
\pgfsetstrokeopacity{0.000000}%
\pgfsetdash{}{0pt}%
\pgfpathmoveto{\pgfqpoint{4.304796in}{1.359165in}}%
\pgfpathlineto{\pgfqpoint{4.692113in}{1.359165in}}%
\pgfpathlineto{\pgfqpoint{4.692113in}{3.480020in}}%
\pgfpathlineto{\pgfqpoint{4.304796in}{3.480020in}}%
\pgfpathclose%
\pgfusepath{fill}%
\end{pgfscope}%
\begin{pgfscope}%
\pgfpathrectangle{\pgfqpoint{0.608070in}{1.359165in}}{\pgfqpoint{5.585972in}{3.027198in}}%
\pgfusepath{clip}%
\pgfsetbuttcap%
\pgfsetmiterjoin%
\definecolor{currentfill}{rgb}{0.121569,0.466667,0.705882}%
\pgfsetfillcolor{currentfill}%
\pgfsetlinewidth{0.000000pt}%
\definecolor{currentstroke}{rgb}{0.000000,0.000000,0.000000}%
\pgfsetstrokecolor{currentstroke}%
\pgfsetstrokeopacity{0.000000}%
\pgfsetdash{}{0pt}%
\pgfpathmoveto{\pgfqpoint{5.165500in}{1.359165in}}%
\pgfpathlineto{\pgfqpoint{5.552817in}{1.359165in}}%
\pgfpathlineto{\pgfqpoint{5.552817in}{3.644094in}}%
\pgfpathlineto{\pgfqpoint{5.165500in}{3.644094in}}%
\pgfpathclose%
\pgfusepath{fill}%
\end{pgfscope}%
\begin{pgfscope}%
\pgfpathrectangle{\pgfqpoint{0.608070in}{1.359165in}}{\pgfqpoint{5.585972in}{3.027198in}}%
\pgfusepath{clip}%
\pgfsetbuttcap%
\pgfsetmiterjoin%
\definecolor{currentfill}{rgb}{1.000000,0.498039,0.054902}%
\pgfsetfillcolor{currentfill}%
\pgfsetlinewidth{0.000000pt}%
\definecolor{currentstroke}{rgb}{0.000000,0.000000,0.000000}%
\pgfsetstrokecolor{currentstroke}%
\pgfsetstrokeopacity{0.000000}%
\pgfsetdash{}{0pt}%
\pgfpathmoveto{\pgfqpoint{1.249295in}{1.359165in}}%
\pgfpathlineto{\pgfqpoint{1.636612in}{1.359165in}}%
\pgfpathlineto{\pgfqpoint{1.636612in}{3.423714in}}%
\pgfpathlineto{\pgfqpoint{1.249295in}{3.423714in}}%
\pgfpathclose%
\pgfusepath{fill}%
\end{pgfscope}%
\begin{pgfscope}%
\pgfpathrectangle{\pgfqpoint{0.608070in}{1.359165in}}{\pgfqpoint{5.585972in}{3.027198in}}%
\pgfusepath{clip}%
\pgfsetbuttcap%
\pgfsetmiterjoin%
\definecolor{currentfill}{rgb}{1.000000,0.498039,0.054902}%
\pgfsetfillcolor{currentfill}%
\pgfsetlinewidth{0.000000pt}%
\definecolor{currentstroke}{rgb}{0.000000,0.000000,0.000000}%
\pgfsetstrokecolor{currentstroke}%
\pgfsetstrokeopacity{0.000000}%
\pgfsetdash{}{0pt}%
\pgfpathmoveto{\pgfqpoint{2.109999in}{1.359165in}}%
\pgfpathlineto{\pgfqpoint{2.497316in}{1.359165in}}%
\pgfpathlineto{\pgfqpoint{2.497316in}{3.411605in}}%
\pgfpathlineto{\pgfqpoint{2.109999in}{3.411605in}}%
\pgfpathclose%
\pgfusepath{fill}%
\end{pgfscope}%
\begin{pgfscope}%
\pgfpathrectangle{\pgfqpoint{0.608070in}{1.359165in}}{\pgfqpoint{5.585972in}{3.027198in}}%
\pgfusepath{clip}%
\pgfsetbuttcap%
\pgfsetmiterjoin%
\definecolor{currentfill}{rgb}{1.000000,0.498039,0.054902}%
\pgfsetfillcolor{currentfill}%
\pgfsetlinewidth{0.000000pt}%
\definecolor{currentstroke}{rgb}{0.000000,0.000000,0.000000}%
\pgfsetstrokecolor{currentstroke}%
\pgfsetstrokeopacity{0.000000}%
\pgfsetdash{}{0pt}%
\pgfpathmoveto{\pgfqpoint{2.970704in}{1.359165in}}%
\pgfpathlineto{\pgfqpoint{3.358021in}{1.359165in}}%
\pgfpathlineto{\pgfqpoint{3.358021in}{3.420081in}}%
\pgfpathlineto{\pgfqpoint{2.970704in}{3.420081in}}%
\pgfpathclose%
\pgfusepath{fill}%
\end{pgfscope}%
\begin{pgfscope}%
\pgfpathrectangle{\pgfqpoint{0.608070in}{1.359165in}}{\pgfqpoint{5.585972in}{3.027198in}}%
\pgfusepath{clip}%
\pgfsetbuttcap%
\pgfsetmiterjoin%
\definecolor{currentfill}{rgb}{1.000000,0.498039,0.054902}%
\pgfsetfillcolor{currentfill}%
\pgfsetlinewidth{0.000000pt}%
\definecolor{currentstroke}{rgb}{0.000000,0.000000,0.000000}%
\pgfsetstrokecolor{currentstroke}%
\pgfsetstrokeopacity{0.000000}%
\pgfsetdash{}{0pt}%
\pgfpathmoveto{\pgfqpoint{3.831408in}{1.359165in}}%
\pgfpathlineto{\pgfqpoint{4.218725in}{1.359165in}}%
\pgfpathlineto{\pgfqpoint{4.218725in}{3.357721in}}%
\pgfpathlineto{\pgfqpoint{3.831408in}{3.357721in}}%
\pgfpathclose%
\pgfusepath{fill}%
\end{pgfscope}%
\begin{pgfscope}%
\pgfpathrectangle{\pgfqpoint{0.608070in}{1.359165in}}{\pgfqpoint{5.585972in}{3.027198in}}%
\pgfusepath{clip}%
\pgfsetbuttcap%
\pgfsetmiterjoin%
\definecolor{currentfill}{rgb}{1.000000,0.498039,0.054902}%
\pgfsetfillcolor{currentfill}%
\pgfsetlinewidth{0.000000pt}%
\definecolor{currentstroke}{rgb}{0.000000,0.000000,0.000000}%
\pgfsetstrokecolor{currentstroke}%
\pgfsetstrokeopacity{0.000000}%
\pgfsetdash{}{0pt}%
\pgfpathmoveto{\pgfqpoint{4.692113in}{1.359165in}}%
\pgfpathlineto{\pgfqpoint{5.079430in}{1.359165in}}%
\pgfpathlineto{\pgfqpoint{5.079430in}{3.436428in}}%
\pgfpathlineto{\pgfqpoint{4.692113in}{3.436428in}}%
\pgfpathclose%
\pgfusepath{fill}%
\end{pgfscope}%
\begin{pgfscope}%
\pgfpathrectangle{\pgfqpoint{0.608070in}{1.359165in}}{\pgfqpoint{5.585972in}{3.027198in}}%
\pgfusepath{clip}%
\pgfsetbuttcap%
\pgfsetmiterjoin%
\definecolor{currentfill}{rgb}{1.000000,0.498039,0.054902}%
\pgfsetfillcolor{currentfill}%
\pgfsetlinewidth{0.000000pt}%
\definecolor{currentstroke}{rgb}{0.000000,0.000000,0.000000}%
\pgfsetstrokecolor{currentstroke}%
\pgfsetstrokeopacity{0.000000}%
\pgfsetdash{}{0pt}%
\pgfpathmoveto{\pgfqpoint{5.552817in}{1.359165in}}%
\pgfpathlineto{\pgfqpoint{5.940134in}{1.359165in}}%
\pgfpathlineto{\pgfqpoint{5.940134in}{3.543591in}}%
\pgfpathlineto{\pgfqpoint{5.552817in}{3.543591in}}%
\pgfpathclose%
\pgfusepath{fill}%
\end{pgfscope}%
\begin{pgfscope}%
\pgfsetbuttcap%
\pgfsetroundjoin%
\definecolor{currentfill}{rgb}{0.000000,0.000000,0.000000}%
\pgfsetfillcolor{currentfill}%
\pgfsetlinewidth{0.803000pt}%
\definecolor{currentstroke}{rgb}{0.000000,0.000000,0.000000}%
\pgfsetstrokecolor{currentstroke}%
\pgfsetdash{}{0pt}%
\pgfsys@defobject{currentmarker}{\pgfqpoint{0.000000in}{-0.048611in}}{\pgfqpoint{0.000000in}{0.000000in}}{%
\pgfpathmoveto{\pgfqpoint{0.000000in}{0.000000in}}%
\pgfpathlineto{\pgfqpoint{0.000000in}{-0.048611in}}%
\pgfusepath{stroke,fill}%
}%
\begin{pgfscope}%
\pgfsys@transformshift{1.249295in}{1.359165in}%
\pgfsys@useobject{currentmarker}{}%
\end{pgfscope}%
\end{pgfscope}%
\begin{pgfscope}%
\definecolor{textcolor}{rgb}{0.000000,0.000000,0.000000}%
\pgfsetstrokecolor{textcolor}%
\pgfsetfillcolor{textcolor}%
\pgftext[x=1.001614in, y=0.637777in, left, base,rotate=45.000000]{\color{textcolor}\sffamily\fontsize{10.000000}{12.000000}\selectfont Textureless}%
\end{pgfscope}%
\begin{pgfscope}%
\pgfsetbuttcap%
\pgfsetroundjoin%
\definecolor{currentfill}{rgb}{0.000000,0.000000,0.000000}%
\pgfsetfillcolor{currentfill}%
\pgfsetlinewidth{0.803000pt}%
\definecolor{currentstroke}{rgb}{0.000000,0.000000,0.000000}%
\pgfsetstrokecolor{currentstroke}%
\pgfsetdash{}{0pt}%
\pgfsys@defobject{currentmarker}{\pgfqpoint{0.000000in}{-0.048611in}}{\pgfqpoint{0.000000in}{0.000000in}}{%
\pgfpathmoveto{\pgfqpoint{0.000000in}{0.000000in}}%
\pgfpathlineto{\pgfqpoint{0.000000in}{-0.048611in}}%
\pgfusepath{stroke,fill}%
}%
\begin{pgfscope}%
\pgfsys@transformshift{2.109999in}{1.359165in}%
\pgfsys@useobject{currentmarker}{}%
\end{pgfscope}%
\end{pgfscope}%
\begin{pgfscope}%
\definecolor{textcolor}{rgb}{0.000000,0.000000,0.000000}%
\pgfsetstrokecolor{textcolor}%
\pgfsetfillcolor{textcolor}%
\pgftext[x=1.698628in, y=0.310396in, left, base,rotate=45.000000]{\color{textcolor}\sffamily\fontsize{10.000000}{12.000000}\selectfont Textureless+Light}%
\end{pgfscope}%
\begin{pgfscope}%
\pgfsetbuttcap%
\pgfsetroundjoin%
\definecolor{currentfill}{rgb}{0.000000,0.000000,0.000000}%
\pgfsetfillcolor{currentfill}%
\pgfsetlinewidth{0.803000pt}%
\definecolor{currentstroke}{rgb}{0.000000,0.000000,0.000000}%
\pgfsetstrokecolor{currentstroke}%
\pgfsetdash{}{0pt}%
\pgfsys@defobject{currentmarker}{\pgfqpoint{0.000000in}{-0.048611in}}{\pgfqpoint{0.000000in}{0.000000in}}{%
\pgfpathmoveto{\pgfqpoint{0.000000in}{0.000000in}}%
\pgfpathlineto{\pgfqpoint{0.000000in}{-0.048611in}}%
\pgfusepath{stroke,fill}%
}%
\begin{pgfscope}%
\pgfsys@transformshift{2.970704in}{1.359165in}%
\pgfsys@useobject{currentmarker}{}%
\end{pgfscope}%
\end{pgfscope}%
\begin{pgfscope}%
\definecolor{textcolor}{rgb}{0.000000,0.000000,0.000000}%
\pgfsetstrokecolor{textcolor}%
\pgfsetfillcolor{textcolor}%
\pgftext[x=2.786873in, y=0.765478in, left, base,rotate=45.000000]{\color{textcolor}\sffamily\fontsize{10.000000}{12.000000}\selectfont Textured}%
\end{pgfscope}%
\begin{pgfscope}%
\pgfsetbuttcap%
\pgfsetroundjoin%
\definecolor{currentfill}{rgb}{0.000000,0.000000,0.000000}%
\pgfsetfillcolor{currentfill}%
\pgfsetlinewidth{0.803000pt}%
\definecolor{currentstroke}{rgb}{0.000000,0.000000,0.000000}%
\pgfsetstrokecolor{currentstroke}%
\pgfsetdash{}{0pt}%
\pgfsys@defobject{currentmarker}{\pgfqpoint{0.000000in}{-0.048611in}}{\pgfqpoint{0.000000in}{0.000000in}}{%
\pgfpathmoveto{\pgfqpoint{0.000000in}{0.000000in}}%
\pgfpathlineto{\pgfqpoint{0.000000in}{-0.048611in}}%
\pgfusepath{stroke,fill}%
}%
\begin{pgfscope}%
\pgfsys@transformshift{3.831408in}{1.359165in}%
\pgfsys@useobject{currentmarker}{}%
\end{pgfscope}%
\end{pgfscope}%
\begin{pgfscope}%
\definecolor{textcolor}{rgb}{0.000000,0.000000,0.000000}%
\pgfsetstrokecolor{textcolor}%
\pgfsetfillcolor{textcolor}%
\pgftext[x=3.483888in, y=0.438097in, left, base,rotate=45.000000]{\color{textcolor}\sffamily\fontsize{10.000000}{12.000000}\selectfont Textured+Light}%
\end{pgfscope}%
\begin{pgfscope}%
\pgfsetbuttcap%
\pgfsetroundjoin%
\definecolor{currentfill}{rgb}{0.000000,0.000000,0.000000}%
\pgfsetfillcolor{currentfill}%
\pgfsetlinewidth{0.803000pt}%
\definecolor{currentstroke}{rgb}{0.000000,0.000000,0.000000}%
\pgfsetstrokecolor{currentstroke}%
\pgfsetdash{}{0pt}%
\pgfsys@defobject{currentmarker}{\pgfqpoint{0.000000in}{-0.048611in}}{\pgfqpoint{0.000000in}{0.000000in}}{%
\pgfpathmoveto{\pgfqpoint{0.000000in}{0.000000in}}%
\pgfpathlineto{\pgfqpoint{0.000000in}{-0.048611in}}%
\pgfusepath{stroke,fill}%
}%
\begin{pgfscope}%
\pgfsys@transformshift{4.692113in}{1.359165in}%
\pgfsys@useobject{currentmarker}{}%
\end{pgfscope}%
\end{pgfscope}%
\begin{pgfscope}%
\definecolor{textcolor}{rgb}{0.000000,0.000000,0.000000}%
\pgfsetstrokecolor{textcolor}%
\pgfsetfillcolor{textcolor}%
\pgftext[x=4.420167in, y=0.589247in, left, base,rotate=45.000000]{\color{textcolor}\sffamily\fontsize{10.000000}{12.000000}\selectfont Multi-Object}%
\end{pgfscope}%
\begin{pgfscope}%
\pgfsetbuttcap%
\pgfsetroundjoin%
\definecolor{currentfill}{rgb}{0.000000,0.000000,0.000000}%
\pgfsetfillcolor{currentfill}%
\pgfsetlinewidth{0.803000pt}%
\definecolor{currentstroke}{rgb}{0.000000,0.000000,0.000000}%
\pgfsetstrokecolor{currentstroke}%
\pgfsetdash{}{0pt}%
\pgfsys@defobject{currentmarker}{\pgfqpoint{0.000000in}{-0.048611in}}{\pgfqpoint{0.000000in}{0.000000in}}{%
\pgfpathmoveto{\pgfqpoint{0.000000in}{0.000000in}}%
\pgfpathlineto{\pgfqpoint{0.000000in}{-0.048611in}}%
\pgfusepath{stroke,fill}%
}%
\begin{pgfscope}%
\pgfsys@transformshift{5.552817in}{1.359165in}%
\pgfsys@useobject{currentmarker}{}%
\end{pgfscope}%
\end{pgfscope}%
\begin{pgfscope}%
\definecolor{textcolor}{rgb}{0.000000,0.000000,0.000000}%
\pgfsetstrokecolor{textcolor}%
\pgfsetfillcolor{textcolor}%
\pgftext[x=5.330432in, y=0.688368in, left, base,rotate=45.000000]{\color{textcolor}\sffamily\fontsize{10.000000}{12.000000}\selectfont Combined}%
\end{pgfscope}%
\begin{pgfscope}%
\definecolor{textcolor}{rgb}{0.000000,0.000000,0.000000}%
\pgfsetstrokecolor{textcolor}%
\pgfsetfillcolor{textcolor}%
\pgftext[x=3.401056in,y=0.234413in,,top]{\color{textcolor}\sffamily\fontsize{10.000000}{12.000000}\selectfont Dataset}%
\end{pgfscope}%
\begin{pgfscope}%
\pgfsetbuttcap%
\pgfsetroundjoin%
\definecolor{currentfill}{rgb}{0.000000,0.000000,0.000000}%
\pgfsetfillcolor{currentfill}%
\pgfsetlinewidth{0.803000pt}%
\definecolor{currentstroke}{rgb}{0.000000,0.000000,0.000000}%
\pgfsetstrokecolor{currentstroke}%
\pgfsetdash{}{0pt}%
\pgfsys@defobject{currentmarker}{\pgfqpoint{-0.048611in}{0.000000in}}{\pgfqpoint{-0.000000in}{0.000000in}}{%
\pgfpathmoveto{\pgfqpoint{-0.000000in}{0.000000in}}%
\pgfpathlineto{\pgfqpoint{-0.048611in}{0.000000in}}%
\pgfusepath{stroke,fill}%
}%
\begin{pgfscope}%
\pgfsys@transformshift{0.608070in}{1.359165in}%
\pgfsys@useobject{currentmarker}{}%
\end{pgfscope}%
\end{pgfscope}%
\begin{pgfscope}%
\definecolor{textcolor}{rgb}{0.000000,0.000000,0.000000}%
\pgfsetstrokecolor{textcolor}%
\pgfsetfillcolor{textcolor}%
\pgftext[x=0.289968in, y=1.306403in, left, base]{\color{textcolor}\sffamily\fontsize{10.000000}{12.000000}\selectfont 0.0}%
\end{pgfscope}%
\begin{pgfscope}%
\pgfsetbuttcap%
\pgfsetroundjoin%
\definecolor{currentfill}{rgb}{0.000000,0.000000,0.000000}%
\pgfsetfillcolor{currentfill}%
\pgfsetlinewidth{0.803000pt}%
\definecolor{currentstroke}{rgb}{0.000000,0.000000,0.000000}%
\pgfsetstrokecolor{currentstroke}%
\pgfsetdash{}{0pt}%
\pgfsys@defobject{currentmarker}{\pgfqpoint{-0.048611in}{0.000000in}}{\pgfqpoint{-0.000000in}{0.000000in}}{%
\pgfpathmoveto{\pgfqpoint{-0.000000in}{0.000000in}}%
\pgfpathlineto{\pgfqpoint{-0.048611in}{0.000000in}}%
\pgfusepath{stroke,fill}%
}%
\begin{pgfscope}%
\pgfsys@transformshift{0.608070in}{1.964604in}%
\pgfsys@useobject{currentmarker}{}%
\end{pgfscope}%
\end{pgfscope}%
\begin{pgfscope}%
\definecolor{textcolor}{rgb}{0.000000,0.000000,0.000000}%
\pgfsetstrokecolor{textcolor}%
\pgfsetfillcolor{textcolor}%
\pgftext[x=0.289968in, y=1.911843in, left, base]{\color{textcolor}\sffamily\fontsize{10.000000}{12.000000}\selectfont 0.1}%
\end{pgfscope}%
\begin{pgfscope}%
\pgfsetbuttcap%
\pgfsetroundjoin%
\definecolor{currentfill}{rgb}{0.000000,0.000000,0.000000}%
\pgfsetfillcolor{currentfill}%
\pgfsetlinewidth{0.803000pt}%
\definecolor{currentstroke}{rgb}{0.000000,0.000000,0.000000}%
\pgfsetstrokecolor{currentstroke}%
\pgfsetdash{}{0pt}%
\pgfsys@defobject{currentmarker}{\pgfqpoint{-0.048611in}{0.000000in}}{\pgfqpoint{-0.000000in}{0.000000in}}{%
\pgfpathmoveto{\pgfqpoint{-0.000000in}{0.000000in}}%
\pgfpathlineto{\pgfqpoint{-0.048611in}{0.000000in}}%
\pgfusepath{stroke,fill}%
}%
\begin{pgfscope}%
\pgfsys@transformshift{0.608070in}{2.570044in}%
\pgfsys@useobject{currentmarker}{}%
\end{pgfscope}%
\end{pgfscope}%
\begin{pgfscope}%
\definecolor{textcolor}{rgb}{0.000000,0.000000,0.000000}%
\pgfsetstrokecolor{textcolor}%
\pgfsetfillcolor{textcolor}%
\pgftext[x=0.289968in, y=2.517282in, left, base]{\color{textcolor}\sffamily\fontsize{10.000000}{12.000000}\selectfont 0.2}%
\end{pgfscope}%
\begin{pgfscope}%
\pgfsetbuttcap%
\pgfsetroundjoin%
\definecolor{currentfill}{rgb}{0.000000,0.000000,0.000000}%
\pgfsetfillcolor{currentfill}%
\pgfsetlinewidth{0.803000pt}%
\definecolor{currentstroke}{rgb}{0.000000,0.000000,0.000000}%
\pgfsetstrokecolor{currentstroke}%
\pgfsetdash{}{0pt}%
\pgfsys@defobject{currentmarker}{\pgfqpoint{-0.048611in}{0.000000in}}{\pgfqpoint{-0.000000in}{0.000000in}}{%
\pgfpathmoveto{\pgfqpoint{-0.000000in}{0.000000in}}%
\pgfpathlineto{\pgfqpoint{-0.048611in}{0.000000in}}%
\pgfusepath{stroke,fill}%
}%
\begin{pgfscope}%
\pgfsys@transformshift{0.608070in}{3.175483in}%
\pgfsys@useobject{currentmarker}{}%
\end{pgfscope}%
\end{pgfscope}%
\begin{pgfscope}%
\definecolor{textcolor}{rgb}{0.000000,0.000000,0.000000}%
\pgfsetstrokecolor{textcolor}%
\pgfsetfillcolor{textcolor}%
\pgftext[x=0.289968in, y=3.122722in, left, base]{\color{textcolor}\sffamily\fontsize{10.000000}{12.000000}\selectfont 0.3}%
\end{pgfscope}%
\begin{pgfscope}%
\pgfsetbuttcap%
\pgfsetroundjoin%
\definecolor{currentfill}{rgb}{0.000000,0.000000,0.000000}%
\pgfsetfillcolor{currentfill}%
\pgfsetlinewidth{0.803000pt}%
\definecolor{currentstroke}{rgb}{0.000000,0.000000,0.000000}%
\pgfsetstrokecolor{currentstroke}%
\pgfsetdash{}{0pt}%
\pgfsys@defobject{currentmarker}{\pgfqpoint{-0.048611in}{0.000000in}}{\pgfqpoint{-0.000000in}{0.000000in}}{%
\pgfpathmoveto{\pgfqpoint{-0.000000in}{0.000000in}}%
\pgfpathlineto{\pgfqpoint{-0.048611in}{0.000000in}}%
\pgfusepath{stroke,fill}%
}%
\begin{pgfscope}%
\pgfsys@transformshift{0.608070in}{3.780923in}%
\pgfsys@useobject{currentmarker}{}%
\end{pgfscope}%
\end{pgfscope}%
\begin{pgfscope}%
\definecolor{textcolor}{rgb}{0.000000,0.000000,0.000000}%
\pgfsetstrokecolor{textcolor}%
\pgfsetfillcolor{textcolor}%
\pgftext[x=0.289968in, y=3.728161in, left, base]{\color{textcolor}\sffamily\fontsize{10.000000}{12.000000}\selectfont 0.4}%
\end{pgfscope}%
\begin{pgfscope}%
\pgfsetbuttcap%
\pgfsetroundjoin%
\definecolor{currentfill}{rgb}{0.000000,0.000000,0.000000}%
\pgfsetfillcolor{currentfill}%
\pgfsetlinewidth{0.803000pt}%
\definecolor{currentstroke}{rgb}{0.000000,0.000000,0.000000}%
\pgfsetstrokecolor{currentstroke}%
\pgfsetdash{}{0pt}%
\pgfsys@defobject{currentmarker}{\pgfqpoint{-0.048611in}{0.000000in}}{\pgfqpoint{-0.000000in}{0.000000in}}{%
\pgfpathmoveto{\pgfqpoint{-0.000000in}{0.000000in}}%
\pgfpathlineto{\pgfqpoint{-0.048611in}{0.000000in}}%
\pgfusepath{stroke,fill}%
}%
\begin{pgfscope}%
\pgfsys@transformshift{0.608070in}{4.386363in}%
\pgfsys@useobject{currentmarker}{}%
\end{pgfscope}%
\end{pgfscope}%
\begin{pgfscope}%
\definecolor{textcolor}{rgb}{0.000000,0.000000,0.000000}%
\pgfsetstrokecolor{textcolor}%
\pgfsetfillcolor{textcolor}%
\pgftext[x=0.289968in, y=4.333601in, left, base]{\color{textcolor}\sffamily\fontsize{10.000000}{12.000000}\selectfont 0.5}%
\end{pgfscope}%
\begin{pgfscope}%
\definecolor{textcolor}{rgb}{0.000000,0.000000,0.000000}%
\pgfsetstrokecolor{textcolor}%
\pgfsetfillcolor{textcolor}%
\pgftext[x=0.234413in,y=2.872764in,,bottom,rotate=90.000000]{\color{textcolor}\sffamily\fontsize{10.000000}{12.000000}\selectfont IoU}%
\end{pgfscope}%
\begin{pgfscope}%
\pgfsetrectcap%
\pgfsetmiterjoin%
\pgfsetlinewidth{0.803000pt}%
\definecolor{currentstroke}{rgb}{0.000000,0.000000,0.000000}%
\pgfsetstrokecolor{currentstroke}%
\pgfsetdash{}{0pt}%
\pgfpathmoveto{\pgfqpoint{0.608070in}{1.359165in}}%
\pgfpathlineto{\pgfqpoint{0.608070in}{4.386363in}}%
\pgfusepath{stroke}%
\end{pgfscope}%
\begin{pgfscope}%
\pgfsetrectcap%
\pgfsetmiterjoin%
\pgfsetlinewidth{0.803000pt}%
\definecolor{currentstroke}{rgb}{0.000000,0.000000,0.000000}%
\pgfsetstrokecolor{currentstroke}%
\pgfsetdash{}{0pt}%
\pgfpathmoveto{\pgfqpoint{6.194042in}{1.359165in}}%
\pgfpathlineto{\pgfqpoint{6.194042in}{4.386363in}}%
\pgfusepath{stroke}%
\end{pgfscope}%
\begin{pgfscope}%
\pgfsetrectcap%
\pgfsetmiterjoin%
\pgfsetlinewidth{0.803000pt}%
\definecolor{currentstroke}{rgb}{0.000000,0.000000,0.000000}%
\pgfsetstrokecolor{currentstroke}%
\pgfsetdash{}{0pt}%
\pgfpathmoveto{\pgfqpoint{0.608070in}{1.359165in}}%
\pgfpathlineto{\pgfqpoint{6.194042in}{1.359165in}}%
\pgfusepath{stroke}%
\end{pgfscope}%
\begin{pgfscope}%
\pgfsetrectcap%
\pgfsetmiterjoin%
\pgfsetlinewidth{0.803000pt}%
\definecolor{currentstroke}{rgb}{0.000000,0.000000,0.000000}%
\pgfsetstrokecolor{currentstroke}%
\pgfsetdash{}{0pt}%
\pgfpathmoveto{\pgfqpoint{0.608070in}{4.386363in}}%
\pgfpathlineto{\pgfqpoint{6.194042in}{4.386363in}}%
\pgfusepath{stroke}%
\end{pgfscope}%
\begin{pgfscope}%
\definecolor{textcolor}{rgb}{0.000000,0.000000,0.000000}%
\pgfsetstrokecolor{textcolor}%
\pgfsetfillcolor{textcolor}%
\pgftext[x=1.055636in,y=3.609475in,,bottom]{\color{textcolor}\sffamily\fontsize{9.000000}{10.800000}\selectfont 0.3648}%
\end{pgfscope}%
\begin{pgfscope}%
\definecolor{textcolor}{rgb}{0.000000,0.000000,0.000000}%
\pgfsetstrokecolor{textcolor}%
\pgfsetfillcolor{textcolor}%
\pgftext[x=1.916341in,y=3.590101in,,bottom]{\color{textcolor}\sffamily\fontsize{9.000000}{10.800000}\selectfont 0.3616}%
\end{pgfscope}%
\begin{pgfscope}%
\definecolor{textcolor}{rgb}{0.000000,0.000000,0.000000}%
\pgfsetstrokecolor{textcolor}%
\pgfsetfillcolor{textcolor}%
\pgftext[x=2.777045in,y=3.507761in,,bottom]{\color{textcolor}\sffamily\fontsize{9.000000}{10.800000}\selectfont 0.348}%
\end{pgfscope}%
\begin{pgfscope}%
\definecolor{textcolor}{rgb}{0.000000,0.000000,0.000000}%
\pgfsetstrokecolor{textcolor}%
\pgfsetfillcolor{textcolor}%
\pgftext[x=3.637750in,y=3.474462in,,bottom]{\color{textcolor}\sffamily\fontsize{9.000000}{10.800000}\selectfont 0.3425}%
\end{pgfscope}%
\begin{pgfscope}%
\definecolor{textcolor}{rgb}{0.000000,0.000000,0.000000}%
\pgfsetstrokecolor{textcolor}%
\pgfsetfillcolor{textcolor}%
\pgftext[x=4.498454in,y=3.521686in,,bottom]{\color{textcolor}\sffamily\fontsize{9.000000}{10.800000}\selectfont 0.3503}%
\end{pgfscope}%
\begin{pgfscope}%
\definecolor{textcolor}{rgb}{0.000000,0.000000,0.000000}%
\pgfsetstrokecolor{textcolor}%
\pgfsetfillcolor{textcolor}%
\pgftext[x=5.359159in,y=3.685760in,,bottom]{\color{textcolor}\sffamily\fontsize{9.000000}{10.800000}\selectfont 0.3774}%
\end{pgfscope}%
\begin{pgfscope}%
\definecolor{textcolor}{rgb}{0.000000,0.000000,0.000000}%
\pgfsetstrokecolor{textcolor}%
\pgfsetfillcolor{textcolor}%
\pgftext[x=1.442953in,y=3.465380in,,bottom]{\color{textcolor}\sffamily\fontsize{9.000000}{10.800000}\selectfont 0.341}%
\end{pgfscope}%
\begin{pgfscope}%
\definecolor{textcolor}{rgb}{0.000000,0.000000,0.000000}%
\pgfsetstrokecolor{textcolor}%
\pgfsetfillcolor{textcolor}%
\pgftext[x=2.303658in,y=3.453272in,,bottom]{\color{textcolor}\sffamily\fontsize{9.000000}{10.800000}\selectfont 0.339}%
\end{pgfscope}%
\begin{pgfscope}%
\definecolor{textcolor}{rgb}{0.000000,0.000000,0.000000}%
\pgfsetstrokecolor{textcolor}%
\pgfsetfillcolor{textcolor}%
\pgftext[x=3.164362in,y=3.461748in,,bottom]{\color{textcolor}\sffamily\fontsize{9.000000}{10.800000}\selectfont 0.3404}%
\end{pgfscope}%
\begin{pgfscope}%
\definecolor{textcolor}{rgb}{0.000000,0.000000,0.000000}%
\pgfsetstrokecolor{textcolor}%
\pgfsetfillcolor{textcolor}%
\pgftext[x=4.025067in,y=3.399387in,,bottom]{\color{textcolor}\sffamily\fontsize{9.000000}{10.800000}\selectfont 0.3301}%
\end{pgfscope}%
\begin{pgfscope}%
\definecolor{textcolor}{rgb}{0.000000,0.000000,0.000000}%
\pgfsetstrokecolor{textcolor}%
\pgfsetfillcolor{textcolor}%
\pgftext[x=4.885771in,y=3.478095in,,bottom]{\color{textcolor}\sffamily\fontsize{9.000000}{10.800000}\selectfont 0.3431}%
\end{pgfscope}%
\begin{pgfscope}%
\definecolor{textcolor}{rgb}{0.000000,0.000000,0.000000}%
\pgfsetstrokecolor{textcolor}%
\pgfsetfillcolor{textcolor}%
\pgftext[x=5.746476in,y=3.585257in,,bottom]{\color{textcolor}\sffamily\fontsize{9.000000}{10.800000}\selectfont 0.3608}%
\end{pgfscope}%
\begin{pgfscope}%
\definecolor{textcolor}{rgb}{0.000000,0.000000,0.000000}%
\pgfsetstrokecolor{textcolor}%
\pgfsetfillcolor{textcolor}%
\pgftext[x=3.401056in,y=4.469696in,,base]{\color{textcolor}\sffamily\fontsize{12.000000}{14.400000}\selectfont Abalation study on chairs with mixed training}%
\end{pgfscope}%
\begin{pgfscope}%
\pgfsetbuttcap%
\pgfsetmiterjoin%
\definecolor{currentfill}{rgb}{1.000000,1.000000,1.000000}%
\pgfsetfillcolor{currentfill}%
\pgfsetfillopacity{0.800000}%
\pgfsetlinewidth{1.003750pt}%
\definecolor{currentstroke}{rgb}{0.800000,0.800000,0.800000}%
\pgfsetstrokecolor{currentstroke}%
\pgfsetstrokeopacity{0.800000}%
\pgfsetdash{}{0pt}%
\pgfpathmoveto{\pgfqpoint{4.882654in}{3.867537in}}%
\pgfpathlineto{\pgfqpoint{6.096820in}{3.867537in}}%
\pgfpathquadraticcurveto{\pgfqpoint{6.124598in}{3.867537in}}{\pgfqpoint{6.124598in}{3.895315in}}%
\pgfpathlineto{\pgfqpoint{6.124598in}{4.289140in}}%
\pgfpathquadraticcurveto{\pgfqpoint{6.124598in}{4.316918in}}{\pgfqpoint{6.096820in}{4.316918in}}%
\pgfpathlineto{\pgfqpoint{4.882654in}{4.316918in}}%
\pgfpathquadraticcurveto{\pgfqpoint{4.854877in}{4.316918in}}{\pgfqpoint{4.854877in}{4.289140in}}%
\pgfpathlineto{\pgfqpoint{4.854877in}{3.895315in}}%
\pgfpathquadraticcurveto{\pgfqpoint{4.854877in}{3.867537in}}{\pgfqpoint{4.882654in}{3.867537in}}%
\pgfpathclose%
\pgfusepath{stroke,fill}%
\end{pgfscope}%
\begin{pgfscope}%
\pgfsetbuttcap%
\pgfsetmiterjoin%
\definecolor{currentfill}{rgb}{0.121569,0.466667,0.705882}%
\pgfsetfillcolor{currentfill}%
\pgfsetlinewidth{0.000000pt}%
\definecolor{currentstroke}{rgb}{0.000000,0.000000,0.000000}%
\pgfsetstrokecolor{currentstroke}%
\pgfsetstrokeopacity{0.000000}%
\pgfsetdash{}{0pt}%
\pgfpathmoveto{\pgfqpoint{4.910432in}{4.155839in}}%
\pgfpathlineto{\pgfqpoint{5.188210in}{4.155839in}}%
\pgfpathlineto{\pgfqpoint{5.188210in}{4.253062in}}%
\pgfpathlineto{\pgfqpoint{4.910432in}{4.253062in}}%
\pgfpathclose%
\pgfusepath{fill}%
\end{pgfscope}%
\begin{pgfscope}%
\definecolor{textcolor}{rgb}{0.000000,0.000000,0.000000}%
\pgfsetstrokecolor{textcolor}%
\pgfsetfillcolor{textcolor}%
\pgftext[x=5.299321in,y=4.155839in,left,base]{\color{textcolor}\sffamily\fontsize{10.000000}{12.000000}\selectfont Pix2Vox++}%
\end{pgfscope}%
\begin{pgfscope}%
\pgfsetbuttcap%
\pgfsetmiterjoin%
\definecolor{currentfill}{rgb}{1.000000,0.498039,0.054902}%
\pgfsetfillcolor{currentfill}%
\pgfsetlinewidth{0.000000pt}%
\definecolor{currentstroke}{rgb}{0.000000,0.000000,0.000000}%
\pgfsetstrokecolor{currentstroke}%
\pgfsetstrokeopacity{0.000000}%
\pgfsetdash{}{0pt}%
\pgfpathmoveto{\pgfqpoint{4.910432in}{3.951982in}}%
\pgfpathlineto{\pgfqpoint{5.188210in}{3.951982in}}%
\pgfpathlineto{\pgfqpoint{5.188210in}{4.049204in}}%
\pgfpathlineto{\pgfqpoint{4.910432in}{4.049204in}}%
\pgfpathclose%
\pgfusepath{fill}%
\end{pgfscope}%
\begin{pgfscope}%
\definecolor{textcolor}{rgb}{0.000000,0.000000,0.000000}%
\pgfsetstrokecolor{textcolor}%
\pgfsetfillcolor{textcolor}%
\pgftext[x=5.299321in,y=3.951982in,left,base]{\color{textcolor}\sffamily\fontsize{10.000000}{12.000000}\selectfont Pix2Vox}%
\end{pgfscope}%
\end{pgfpicture}%
\makeatother%
\endgroup%
}
    \caption[\gls{iou} Comparison for Ablation Datasets with Mixed Training.]{Bar plot for the \gls{iou}  for baseline trained by \textbf{mixing} chair dataset from real and synthetic dataset with ratio of 50\%.
    Observe that the \fls{IoU} is consistent for all types of randomization proving that mixed training negates loss from randomization.}
    \label{fig:ablation2}
\end{figure}


\begin{figure}[!ht]
    \begin{tabular}{llll}
        Pix3D images & \includegraphics[width=.2\linewidth]{/Users/apple/OVGU/Thesis/code/3dReconstruction/report/images/evaluation/reconstruction/ablation/chair1} &
        \includegraphics[width=.2\linewidth]{/Users/apple/OVGU/Thesis/code/3dReconstruction/report/images/evaluation/reconstruction/ablation/chair2} &
        \includegraphics[width=.2\linewidth]{/Users/apple/OVGU/Thesis/code/3dReconstruction/report/images/evaluation/reconstruction/ablation/chair3}\\

        Ground Truth & \includegraphics[trim={0 0 {.1\width} 0},clip,width=.2\linewidth]{/Users/apple/OVGU/Thesis/code/3dReconstruction/report/images/evaluation/reconstruction/ablation/chair1_original} &
        \includegraphics[trim={0 0 {.1\width} 0},clip,width=.2\linewidth]{/Users/apple/OVGU/Thesis/code/3dReconstruction/report/images/evaluation/reconstruction/ablation/chair2_original} &
        \includegraphics[trim={0 0 {.1\width} 0},clip,width=.2\linewidth]{/Users/apple/OVGU/Thesis/code/3dReconstruction/report/images/evaluation/reconstruction/ablation/chair3_original}\\

        Output1 & \includegraphics[width=.2\linewidth]{/Users/apple/OVGU/Thesis/code/3dReconstruction/report/images/evaluation/reconstruction/ablation/pix3d_p2vpp_chair1} &
        \includegraphics[width=.2\linewidth]{/Users/apple/OVGU/Thesis/code/3dReconstruction/report/images/evaluation/reconstruction/ablation/pix3d_p2vpp_chair2} &
        \includegraphics[width=.2\linewidth]{/Users/apple/OVGU/Thesis/code/3dReconstruction/report/images/evaluation/reconstruction/ablation/pix3d_p2vpp_chair3}\\

        Output2 & \includegraphics[width=.2\linewidth]{/Users/apple/OVGU/Thesis/code/3dReconstruction/report/images/evaluation/reconstruction/ablation/pix3d_p2v_chair1} &
        \includegraphics[width=.2\linewidth]{/Users/apple/OVGU/Thesis/code/3dReconstruction/report/images/evaluation/reconstruction/ablation/pix3d_p2v_chair2} &
        \includegraphics[width=.2\linewidth]{/Users/apple/OVGU/Thesis/code/3dReconstruction/report/images/evaluation/reconstruction/ablation/pix3d_p2v_chair3}\\

        Output3 & \includegraphics[width=.2\linewidth]{/Users/apple/OVGU/Thesis/code/3dReconstruction/report/images/evaluation/reconstruction/ablation/mixed_p2vpp_chair1} &
        \includegraphics[width=.2\linewidth]{/Users/apple/OVGU/Thesis/code/3dReconstruction/report/images/evaluation/reconstruction/ablation/mixed_p2vpp_chair2} &
        \includegraphics[width=.2\linewidth]{/Users/apple/OVGU/Thesis/code/3dReconstruction/report/images/evaluation/reconstruction/ablation/mixed_p2vpp_chair3}\\

        Output4 & \includegraphics[width=.2\linewidth]{/Users/apple/OVGU/Thesis/code/3dReconstruction/report/images/evaluation/reconstruction/ablation/mixed_p2v_chair1} &
        \includegraphics[width=.2\linewidth]{/Users/apple/OVGU/Thesis/code/3dReconstruction/report/images/evaluation/reconstruction/ablation/mixed_p2v_chair2} &
        \includegraphics[width=.2\linewidth]{/Users/apple/OVGU/Thesis/code/3dReconstruction/report/images/evaluation/reconstruction/ablation/mixed_p2v_chair3}\\

    \end{tabular}
    \caption[3D Reconstruction Outputs for Ablation Datasets with Mixed Training.]{3D reconstruction outputs for best ablation models and models with \textbf{mixed} training. Output1-2: Pix2Vox++ and Pix2Vox trained on Pix3D.
    Output3-4:Pix2Vox++ and Pix2Vox trained with mixed ratio of 50\% with multi-object chair synthetic dataset, reconstructs chair with more details.}
    \label{fig:mixed_ablation_images2}
\end{figure}

In mixed training with a ratio of 50\%, we see a maximum increase of 3.43\% increment in Pix2Vox++ and 5.15\% in Pix2Vox.
The behavior of models for the different randomization parameters is similar to what we observed in \autoref{subsec:domain-randomisation-on-chair-dataset}.
For Pix2Vox++, the textureless chair dataset gives the best performance, with a gradual decrease with the addition of each parameter and a slight increase for the multi-object dataset.
Similarly, for Pix2Vox, a gradual decrease is observed, but multi-object gives better performance than the textureless dataset, as in \autoref{fig:ablation2}.
The combined dataset of all other domain randomization showed the best performance of the set for both the baseline models.

We initially expected to see the performance same as in \autoref{fig:ablation1}, with each component of domain randomization.
However, Mixed training eliminated the inconsistency, and thus irrespective of the type domain randomization in synthetic data, the validation achieves good consistent performance as in \autoref{fig:ablation2}.

In \autoref{fig:mixed_ablation_images2}, we see the 3D reconstruction output for models trained on mixed dataset of 50\% of synthetic and real per mini-batch.
The outputs were collected for images from the real dataset with the threshold which gave the best \gls{iou}.
The output is better than models trained on only synthetic dataset as in \autoref{fig:ablation_images1}.
The detailing in the reconstructed chair even seems to be better than the models trained on only the real dataset in the same image.

A study with \gls{f1} for the chair dataset is discussed in \autoref{subsec:ablation-study-on-chairs}.
We observe similar behaviour even with \gls{f1}.

%\begin{figure}
%    \centering
%    \resizebox{0.7\textwidth}{!}{%% Creator: Matplotlib, PGF backend
%%
%% To include the figure in your LaTeX document, write
%%   \input{<filename>.pgf}
%%
%% Make sure the required packages are loaded in your preamble
%%   \usepackage{pgf}
%%
%% Figures using additional raster images can only be included by \input if
%% they are in the same directory as the main LaTeX file. For loading figures
%% from other directories you can use the `import` package
%%   \usepackage{import}
%%
%% and then include the figures with
%%   \import{<path to file>}{<filename>.pgf}
%%
%% Matplotlib used the following preamble
%%   \usepackage{fontspec}
%%   \setmainfont{DejaVuSerif.ttf}[Path=\detokenize{/Users/apple/opt/anaconda3/envs/kaolin/lib/python3.7/site-packages/matplotlib/mpl-data/fonts/ttf/}]
%%   \setsansfont{DejaVuSans.ttf}[Path=\detokenize{/Users/apple/opt/anaconda3/envs/kaolin/lib/python3.7/site-packages/matplotlib/mpl-data/fonts/ttf/}]
%%   \setmonofont{DejaVuSansMono.ttf}[Path=\detokenize{/Users/apple/opt/anaconda3/envs/kaolin/lib/python3.7/site-packages/matplotlib/mpl-data/fonts/ttf/}]
%%
\begingroup%
\makeatletter%
\begin{pgfpicture}%
\pgfpathrectangle{\pgfpointorigin}{\pgfqpoint{5.739617in}{5.207926in}}%
\pgfusepath{use as bounding box, clip}%
\begin{pgfscope}%
\pgfsetbuttcap%
\pgfsetmiterjoin%
\definecolor{currentfill}{rgb}{1.000000,1.000000,1.000000}%
\pgfsetfillcolor{currentfill}%
\pgfsetlinewidth{0.000000pt}%
\definecolor{currentstroke}{rgb}{1.000000,1.000000,1.000000}%
\pgfsetstrokecolor{currentstroke}%
\pgfsetdash{}{0pt}%
\pgfpathmoveto{\pgfqpoint{0.000000in}{0.000000in}}%
\pgfpathlineto{\pgfqpoint{5.739617in}{0.000000in}}%
\pgfpathlineto{\pgfqpoint{5.739617in}{5.207926in}}%
\pgfpathlineto{\pgfqpoint{0.000000in}{5.207926in}}%
\pgfpathclose%
\pgfusepath{fill}%
\end{pgfscope}%
\begin{pgfscope}%
\pgfsetbuttcap%
\pgfsetmiterjoin%
\definecolor{currentfill}{rgb}{1.000000,1.000000,1.000000}%
\pgfsetfillcolor{currentfill}%
\pgfsetlinewidth{0.000000pt}%
\definecolor{currentstroke}{rgb}{0.000000,0.000000,0.000000}%
\pgfsetstrokecolor{currentstroke}%
\pgfsetstrokeopacity{0.000000}%
\pgfsetdash{}{0pt}%
\pgfpathmoveto{\pgfqpoint{0.608070in}{1.359165in}}%
\pgfpathlineto{\pgfqpoint{5.568070in}{1.359165in}}%
\pgfpathlineto{\pgfqpoint{5.568070in}{5.055165in}}%
\pgfpathlineto{\pgfqpoint{0.608070in}{5.055165in}}%
\pgfpathclose%
\pgfusepath{fill}%
\end{pgfscope}%
\begin{pgfscope}%
\pgfsetbuttcap%
\pgfsetroundjoin%
\definecolor{currentfill}{rgb}{0.000000,0.000000,0.000000}%
\pgfsetfillcolor{currentfill}%
\pgfsetlinewidth{0.803000pt}%
\definecolor{currentstroke}{rgb}{0.000000,0.000000,0.000000}%
\pgfsetstrokecolor{currentstroke}%
\pgfsetdash{}{0pt}%
\pgfsys@defobject{currentmarker}{\pgfqpoint{0.000000in}{-0.048611in}}{\pgfqpoint{0.000000in}{0.000000in}}{%
\pgfpathmoveto{\pgfqpoint{0.000000in}{0.000000in}}%
\pgfpathlineto{\pgfqpoint{0.000000in}{-0.048611in}}%
\pgfusepath{stroke,fill}%
}%
\begin{pgfscope}%
\pgfsys@transformshift{0.833525in}{1.359165in}%
\pgfsys@useobject{currentmarker}{}%
\end{pgfscope}%
\end{pgfscope}%
\begin{pgfscope}%
\definecolor{textcolor}{rgb}{0.000000,0.000000,0.000000}%
\pgfsetstrokecolor{textcolor}%
\pgfsetfillcolor{textcolor}%
\pgftext[x=0.585844in, y=0.637777in, left, base,rotate=45.000000]{\color{textcolor}\sffamily\fontsize{10.000000}{12.000000}\selectfont Textureless}%
\end{pgfscope}%
\begin{pgfscope}%
\pgfsetbuttcap%
\pgfsetroundjoin%
\definecolor{currentfill}{rgb}{0.000000,0.000000,0.000000}%
\pgfsetfillcolor{currentfill}%
\pgfsetlinewidth{0.803000pt}%
\definecolor{currentstroke}{rgb}{0.000000,0.000000,0.000000}%
\pgfsetstrokecolor{currentstroke}%
\pgfsetdash{}{0pt}%
\pgfsys@defobject{currentmarker}{\pgfqpoint{0.000000in}{-0.048611in}}{\pgfqpoint{0.000000in}{0.000000in}}{%
\pgfpathmoveto{\pgfqpoint{0.000000in}{0.000000in}}%
\pgfpathlineto{\pgfqpoint{0.000000in}{-0.048611in}}%
\pgfusepath{stroke,fill}%
}%
\begin{pgfscope}%
\pgfsys@transformshift{1.735343in}{1.359165in}%
\pgfsys@useobject{currentmarker}{}%
\end{pgfscope}%
\end{pgfscope}%
\begin{pgfscope}%
\definecolor{textcolor}{rgb}{0.000000,0.000000,0.000000}%
\pgfsetstrokecolor{textcolor}%
\pgfsetfillcolor{textcolor}%
\pgftext[x=1.323972in, y=0.310396in, left, base,rotate=45.000000]{\color{textcolor}\sffamily\fontsize{10.000000}{12.000000}\selectfont Textureless+Light}%
\end{pgfscope}%
\begin{pgfscope}%
\pgfsetbuttcap%
\pgfsetroundjoin%
\definecolor{currentfill}{rgb}{0.000000,0.000000,0.000000}%
\pgfsetfillcolor{currentfill}%
\pgfsetlinewidth{0.803000pt}%
\definecolor{currentstroke}{rgb}{0.000000,0.000000,0.000000}%
\pgfsetstrokecolor{currentstroke}%
\pgfsetdash{}{0pt}%
\pgfsys@defobject{currentmarker}{\pgfqpoint{0.000000in}{-0.048611in}}{\pgfqpoint{0.000000in}{0.000000in}}{%
\pgfpathmoveto{\pgfqpoint{0.000000in}{0.000000in}}%
\pgfpathlineto{\pgfqpoint{0.000000in}{-0.048611in}}%
\pgfusepath{stroke,fill}%
}%
\begin{pgfscope}%
\pgfsys@transformshift{2.637161in}{1.359165in}%
\pgfsys@useobject{currentmarker}{}%
\end{pgfscope}%
\end{pgfscope}%
\begin{pgfscope}%
\definecolor{textcolor}{rgb}{0.000000,0.000000,0.000000}%
\pgfsetstrokecolor{textcolor}%
\pgfsetfillcolor{textcolor}%
\pgftext[x=2.453330in, y=0.765478in, left, base,rotate=45.000000]{\color{textcolor}\sffamily\fontsize{10.000000}{12.000000}\selectfont Textured}%
\end{pgfscope}%
\begin{pgfscope}%
\pgfsetbuttcap%
\pgfsetroundjoin%
\definecolor{currentfill}{rgb}{0.000000,0.000000,0.000000}%
\pgfsetfillcolor{currentfill}%
\pgfsetlinewidth{0.803000pt}%
\definecolor{currentstroke}{rgb}{0.000000,0.000000,0.000000}%
\pgfsetstrokecolor{currentstroke}%
\pgfsetdash{}{0pt}%
\pgfsys@defobject{currentmarker}{\pgfqpoint{0.000000in}{-0.048611in}}{\pgfqpoint{0.000000in}{0.000000in}}{%
\pgfpathmoveto{\pgfqpoint{0.000000in}{0.000000in}}%
\pgfpathlineto{\pgfqpoint{0.000000in}{-0.048611in}}%
\pgfusepath{stroke,fill}%
}%
\begin{pgfscope}%
\pgfsys@transformshift{3.538979in}{1.359165in}%
\pgfsys@useobject{currentmarker}{}%
\end{pgfscope}%
\end{pgfscope}%
\begin{pgfscope}%
\definecolor{textcolor}{rgb}{0.000000,0.000000,0.000000}%
\pgfsetstrokecolor{textcolor}%
\pgfsetfillcolor{textcolor}%
\pgftext[x=3.191458in, y=0.438097in, left, base,rotate=45.000000]{\color{textcolor}\sffamily\fontsize{10.000000}{12.000000}\selectfont Textured+Light}%
\end{pgfscope}%
\begin{pgfscope}%
\pgfsetbuttcap%
\pgfsetroundjoin%
\definecolor{currentfill}{rgb}{0.000000,0.000000,0.000000}%
\pgfsetfillcolor{currentfill}%
\pgfsetlinewidth{0.803000pt}%
\definecolor{currentstroke}{rgb}{0.000000,0.000000,0.000000}%
\pgfsetstrokecolor{currentstroke}%
\pgfsetdash{}{0pt}%
\pgfsys@defobject{currentmarker}{\pgfqpoint{0.000000in}{-0.048611in}}{\pgfqpoint{0.000000in}{0.000000in}}{%
\pgfpathmoveto{\pgfqpoint{0.000000in}{0.000000in}}%
\pgfpathlineto{\pgfqpoint{0.000000in}{-0.048611in}}%
\pgfusepath{stroke,fill}%
}%
\begin{pgfscope}%
\pgfsys@transformshift{4.440797in}{1.359165in}%
\pgfsys@useobject{currentmarker}{}%
\end{pgfscope}%
\end{pgfscope}%
\begin{pgfscope}%
\definecolor{textcolor}{rgb}{0.000000,0.000000,0.000000}%
\pgfsetstrokecolor{textcolor}%
\pgfsetfillcolor{textcolor}%
\pgftext[x=4.168852in, y=0.589247in, left, base,rotate=45.000000]{\color{textcolor}\sffamily\fontsize{10.000000}{12.000000}\selectfont Multi-Object}%
\end{pgfscope}%
\begin{pgfscope}%
\pgfsetbuttcap%
\pgfsetroundjoin%
\definecolor{currentfill}{rgb}{0.000000,0.000000,0.000000}%
\pgfsetfillcolor{currentfill}%
\pgfsetlinewidth{0.803000pt}%
\definecolor{currentstroke}{rgb}{0.000000,0.000000,0.000000}%
\pgfsetstrokecolor{currentstroke}%
\pgfsetdash{}{0pt}%
\pgfsys@defobject{currentmarker}{\pgfqpoint{0.000000in}{-0.048611in}}{\pgfqpoint{0.000000in}{0.000000in}}{%
\pgfpathmoveto{\pgfqpoint{0.000000in}{0.000000in}}%
\pgfpathlineto{\pgfqpoint{0.000000in}{-0.048611in}}%
\pgfusepath{stroke,fill}%
}%
\begin{pgfscope}%
\pgfsys@transformshift{5.342615in}{1.359165in}%
\pgfsys@useobject{currentmarker}{}%
\end{pgfscope}%
\end{pgfscope}%
\begin{pgfscope}%
\definecolor{textcolor}{rgb}{0.000000,0.000000,0.000000}%
\pgfsetstrokecolor{textcolor}%
\pgfsetfillcolor{textcolor}%
\pgftext[x=5.120230in, y=0.688368in, left, base,rotate=45.000000]{\color{textcolor}\sffamily\fontsize{10.000000}{12.000000}\selectfont Combined}%
\end{pgfscope}%
\begin{pgfscope}%
\definecolor{textcolor}{rgb}{0.000000,0.000000,0.000000}%
\pgfsetstrokecolor{textcolor}%
\pgfsetfillcolor{textcolor}%
\pgftext[x=3.088070in,y=0.234413in,,top]{\color{textcolor}\sffamily\fontsize{10.000000}{12.000000}\selectfont Datasets}%
\end{pgfscope}%
\begin{pgfscope}%
\pgfsetbuttcap%
\pgfsetroundjoin%
\definecolor{currentfill}{rgb}{0.000000,0.000000,0.000000}%
\pgfsetfillcolor{currentfill}%
\pgfsetlinewidth{0.803000pt}%
\definecolor{currentstroke}{rgb}{0.000000,0.000000,0.000000}%
\pgfsetstrokecolor{currentstroke}%
\pgfsetdash{}{0pt}%
\pgfsys@defobject{currentmarker}{\pgfqpoint{-0.048611in}{0.000000in}}{\pgfqpoint{-0.000000in}{0.000000in}}{%
\pgfpathmoveto{\pgfqpoint{-0.000000in}{0.000000in}}%
\pgfpathlineto{\pgfqpoint{-0.048611in}{0.000000in}}%
\pgfusepath{stroke,fill}%
}%
\begin{pgfscope}%
\pgfsys@transformshift{0.608070in}{1.359165in}%
\pgfsys@useobject{currentmarker}{}%
\end{pgfscope}%
\end{pgfscope}%
\begin{pgfscope}%
\definecolor{textcolor}{rgb}{0.000000,0.000000,0.000000}%
\pgfsetstrokecolor{textcolor}%
\pgfsetfillcolor{textcolor}%
\pgftext[x=0.289968in, y=1.306403in, left, base]{\color{textcolor}\sffamily\fontsize{10.000000}{12.000000}\selectfont 0.0}%
\end{pgfscope}%
\begin{pgfscope}%
\pgfsetbuttcap%
\pgfsetroundjoin%
\definecolor{currentfill}{rgb}{0.000000,0.000000,0.000000}%
\pgfsetfillcolor{currentfill}%
\pgfsetlinewidth{0.803000pt}%
\definecolor{currentstroke}{rgb}{0.000000,0.000000,0.000000}%
\pgfsetstrokecolor{currentstroke}%
\pgfsetdash{}{0pt}%
\pgfsys@defobject{currentmarker}{\pgfqpoint{-0.048611in}{0.000000in}}{\pgfqpoint{-0.000000in}{0.000000in}}{%
\pgfpathmoveto{\pgfqpoint{-0.000000in}{0.000000in}}%
\pgfpathlineto{\pgfqpoint{-0.048611in}{0.000000in}}%
\pgfusepath{stroke,fill}%
}%
\begin{pgfscope}%
\pgfsys@transformshift{0.608070in}{2.098365in}%
\pgfsys@useobject{currentmarker}{}%
\end{pgfscope}%
\end{pgfscope}%
\begin{pgfscope}%
\definecolor{textcolor}{rgb}{0.000000,0.000000,0.000000}%
\pgfsetstrokecolor{textcolor}%
\pgfsetfillcolor{textcolor}%
\pgftext[x=0.289968in, y=2.045603in, left, base]{\color{textcolor}\sffamily\fontsize{10.000000}{12.000000}\selectfont 0.1}%
\end{pgfscope}%
\begin{pgfscope}%
\pgfsetbuttcap%
\pgfsetroundjoin%
\definecolor{currentfill}{rgb}{0.000000,0.000000,0.000000}%
\pgfsetfillcolor{currentfill}%
\pgfsetlinewidth{0.803000pt}%
\definecolor{currentstroke}{rgb}{0.000000,0.000000,0.000000}%
\pgfsetstrokecolor{currentstroke}%
\pgfsetdash{}{0pt}%
\pgfsys@defobject{currentmarker}{\pgfqpoint{-0.048611in}{0.000000in}}{\pgfqpoint{-0.000000in}{0.000000in}}{%
\pgfpathmoveto{\pgfqpoint{-0.000000in}{0.000000in}}%
\pgfpathlineto{\pgfqpoint{-0.048611in}{0.000000in}}%
\pgfusepath{stroke,fill}%
}%
\begin{pgfscope}%
\pgfsys@transformshift{0.608070in}{2.837565in}%
\pgfsys@useobject{currentmarker}{}%
\end{pgfscope}%
\end{pgfscope}%
\begin{pgfscope}%
\definecolor{textcolor}{rgb}{0.000000,0.000000,0.000000}%
\pgfsetstrokecolor{textcolor}%
\pgfsetfillcolor{textcolor}%
\pgftext[x=0.289968in, y=2.784803in, left, base]{\color{textcolor}\sffamily\fontsize{10.000000}{12.000000}\selectfont 0.2}%
\end{pgfscope}%
\begin{pgfscope}%
\pgfsetbuttcap%
\pgfsetroundjoin%
\definecolor{currentfill}{rgb}{0.000000,0.000000,0.000000}%
\pgfsetfillcolor{currentfill}%
\pgfsetlinewidth{0.803000pt}%
\definecolor{currentstroke}{rgb}{0.000000,0.000000,0.000000}%
\pgfsetstrokecolor{currentstroke}%
\pgfsetdash{}{0pt}%
\pgfsys@defobject{currentmarker}{\pgfqpoint{-0.048611in}{0.000000in}}{\pgfqpoint{-0.000000in}{0.000000in}}{%
\pgfpathmoveto{\pgfqpoint{-0.000000in}{0.000000in}}%
\pgfpathlineto{\pgfqpoint{-0.048611in}{0.000000in}}%
\pgfusepath{stroke,fill}%
}%
\begin{pgfscope}%
\pgfsys@transformshift{0.608070in}{3.576765in}%
\pgfsys@useobject{currentmarker}{}%
\end{pgfscope}%
\end{pgfscope}%
\begin{pgfscope}%
\definecolor{textcolor}{rgb}{0.000000,0.000000,0.000000}%
\pgfsetstrokecolor{textcolor}%
\pgfsetfillcolor{textcolor}%
\pgftext[x=0.289968in, y=3.524003in, left, base]{\color{textcolor}\sffamily\fontsize{10.000000}{12.000000}\selectfont 0.3}%
\end{pgfscope}%
\begin{pgfscope}%
\pgfsetbuttcap%
\pgfsetroundjoin%
\definecolor{currentfill}{rgb}{0.000000,0.000000,0.000000}%
\pgfsetfillcolor{currentfill}%
\pgfsetlinewidth{0.803000pt}%
\definecolor{currentstroke}{rgb}{0.000000,0.000000,0.000000}%
\pgfsetstrokecolor{currentstroke}%
\pgfsetdash{}{0pt}%
\pgfsys@defobject{currentmarker}{\pgfqpoint{-0.048611in}{0.000000in}}{\pgfqpoint{-0.000000in}{0.000000in}}{%
\pgfpathmoveto{\pgfqpoint{-0.000000in}{0.000000in}}%
\pgfpathlineto{\pgfqpoint{-0.048611in}{0.000000in}}%
\pgfusepath{stroke,fill}%
}%
\begin{pgfscope}%
\pgfsys@transformshift{0.608070in}{4.315965in}%
\pgfsys@useobject{currentmarker}{}%
\end{pgfscope}%
\end{pgfscope}%
\begin{pgfscope}%
\definecolor{textcolor}{rgb}{0.000000,0.000000,0.000000}%
\pgfsetstrokecolor{textcolor}%
\pgfsetfillcolor{textcolor}%
\pgftext[x=0.289968in, y=4.263203in, left, base]{\color{textcolor}\sffamily\fontsize{10.000000}{12.000000}\selectfont 0.4}%
\end{pgfscope}%
\begin{pgfscope}%
\pgfsetbuttcap%
\pgfsetroundjoin%
\definecolor{currentfill}{rgb}{0.000000,0.000000,0.000000}%
\pgfsetfillcolor{currentfill}%
\pgfsetlinewidth{0.803000pt}%
\definecolor{currentstroke}{rgb}{0.000000,0.000000,0.000000}%
\pgfsetstrokecolor{currentstroke}%
\pgfsetdash{}{0pt}%
\pgfsys@defobject{currentmarker}{\pgfqpoint{-0.048611in}{0.000000in}}{\pgfqpoint{-0.000000in}{0.000000in}}{%
\pgfpathmoveto{\pgfqpoint{-0.000000in}{0.000000in}}%
\pgfpathlineto{\pgfqpoint{-0.048611in}{0.000000in}}%
\pgfusepath{stroke,fill}%
}%
\begin{pgfscope}%
\pgfsys@transformshift{0.608070in}{5.055165in}%
\pgfsys@useobject{currentmarker}{}%
\end{pgfscope}%
\end{pgfscope}%
\begin{pgfscope}%
\definecolor{textcolor}{rgb}{0.000000,0.000000,0.000000}%
\pgfsetstrokecolor{textcolor}%
\pgfsetfillcolor{textcolor}%
\pgftext[x=0.289968in, y=5.002403in, left, base]{\color{textcolor}\sffamily\fontsize{10.000000}{12.000000}\selectfont 0.5}%
\end{pgfscope}%
\begin{pgfscope}%
\definecolor{textcolor}{rgb}{0.000000,0.000000,0.000000}%
\pgfsetstrokecolor{textcolor}%
\pgfsetfillcolor{textcolor}%
\pgftext[x=0.234413in,y=3.207165in,,bottom,rotate=90.000000]{\color{textcolor}\sffamily\fontsize{10.000000}{12.000000}\selectfont IoU}%
\end{pgfscope}%
\begin{pgfscope}%
\pgfpathrectangle{\pgfqpoint{0.608070in}{1.359165in}}{\pgfqpoint{4.960000in}{3.696000in}}%
\pgfusepath{clip}%
\pgfsetrectcap%
\pgfsetroundjoin%
\pgfsetlinewidth{1.505625pt}%
\definecolor{currentstroke}{rgb}{0.121569,0.466667,0.705882}%
\pgfsetstrokecolor{currentstroke}%
\pgfsetdash{}{0pt}%
\pgfpathmoveto{\pgfqpoint{0.833525in}{4.055767in}}%
\pgfpathlineto{\pgfqpoint{1.735343in}{4.032112in}}%
\pgfpathlineto{\pgfqpoint{2.637161in}{3.931581in}}%
\pgfpathlineto{\pgfqpoint{3.538979in}{3.890925in}}%
\pgfpathlineto{\pgfqpoint{4.440797in}{3.948583in}}%
\pgfpathlineto{\pgfqpoint{5.342615in}{4.148906in}}%
\pgfusepath{stroke}%
\end{pgfscope}%
\begin{pgfscope}%
\pgfpathrectangle{\pgfqpoint{0.608070in}{1.359165in}}{\pgfqpoint{4.960000in}{3.696000in}}%
\pgfusepath{clip}%
\pgfsetbuttcap%
\pgfsetroundjoin%
\definecolor{currentfill}{rgb}{0.121569,0.466667,0.705882}%
\pgfsetfillcolor{currentfill}%
\pgfsetlinewidth{1.003750pt}%
\definecolor{currentstroke}{rgb}{0.121569,0.466667,0.705882}%
\pgfsetstrokecolor{currentstroke}%
\pgfsetdash{}{0pt}%
\pgfsys@defobject{currentmarker}{\pgfqpoint{-0.041667in}{-0.041667in}}{\pgfqpoint{0.041667in}{0.041667in}}{%
\pgfpathmoveto{\pgfqpoint{0.000000in}{-0.041667in}}%
\pgfpathcurveto{\pgfqpoint{0.011050in}{-0.041667in}}{\pgfqpoint{0.021649in}{-0.037276in}}{\pgfqpoint{0.029463in}{-0.029463in}}%
\pgfpathcurveto{\pgfqpoint{0.037276in}{-0.021649in}}{\pgfqpoint{0.041667in}{-0.011050in}}{\pgfqpoint{0.041667in}{0.000000in}}%
\pgfpathcurveto{\pgfqpoint{0.041667in}{0.011050in}}{\pgfqpoint{0.037276in}{0.021649in}}{\pgfqpoint{0.029463in}{0.029463in}}%
\pgfpathcurveto{\pgfqpoint{0.021649in}{0.037276in}}{\pgfqpoint{0.011050in}{0.041667in}}{\pgfqpoint{0.000000in}{0.041667in}}%
\pgfpathcurveto{\pgfqpoint{-0.011050in}{0.041667in}}{\pgfqpoint{-0.021649in}{0.037276in}}{\pgfqpoint{-0.029463in}{0.029463in}}%
\pgfpathcurveto{\pgfqpoint{-0.037276in}{0.021649in}}{\pgfqpoint{-0.041667in}{0.011050in}}{\pgfqpoint{-0.041667in}{0.000000in}}%
\pgfpathcurveto{\pgfqpoint{-0.041667in}{-0.011050in}}{\pgfqpoint{-0.037276in}{-0.021649in}}{\pgfqpoint{-0.029463in}{-0.029463in}}%
\pgfpathcurveto{\pgfqpoint{-0.021649in}{-0.037276in}}{\pgfqpoint{-0.011050in}{-0.041667in}}{\pgfqpoint{0.000000in}{-0.041667in}}%
\pgfpathclose%
\pgfusepath{stroke,fill}%
}%
\begin{pgfscope}%
\pgfsys@transformshift{0.833525in}{4.055767in}%
\pgfsys@useobject{currentmarker}{}%
\end{pgfscope}%
\begin{pgfscope}%
\pgfsys@transformshift{1.735343in}{4.032112in}%
\pgfsys@useobject{currentmarker}{}%
\end{pgfscope}%
\begin{pgfscope}%
\pgfsys@transformshift{2.637161in}{3.931581in}%
\pgfsys@useobject{currentmarker}{}%
\end{pgfscope}%
\begin{pgfscope}%
\pgfsys@transformshift{3.538979in}{3.890925in}%
\pgfsys@useobject{currentmarker}{}%
\end{pgfscope}%
\begin{pgfscope}%
\pgfsys@transformshift{4.440797in}{3.948583in}%
\pgfsys@useobject{currentmarker}{}%
\end{pgfscope}%
\begin{pgfscope}%
\pgfsys@transformshift{5.342615in}{4.148906in}%
\pgfsys@useobject{currentmarker}{}%
\end{pgfscope}%
\end{pgfscope}%
\begin{pgfscope}%
\pgfpathrectangle{\pgfqpoint{0.608070in}{1.359165in}}{\pgfqpoint{4.960000in}{3.696000in}}%
\pgfusepath{clip}%
\pgfsetrectcap%
\pgfsetroundjoin%
\pgfsetlinewidth{1.505625pt}%
\definecolor{currentstroke}{rgb}{1.000000,0.498039,0.054902}%
\pgfsetstrokecolor{currentstroke}%
\pgfsetdash{}{0pt}%
\pgfpathmoveto{\pgfqpoint{0.833525in}{3.879837in}}%
\pgfpathlineto{\pgfqpoint{1.735343in}{3.865053in}}%
\pgfpathlineto{\pgfqpoint{2.637161in}{3.875402in}}%
\pgfpathlineto{\pgfqpoint{3.538979in}{3.799264in}}%
\pgfpathlineto{\pgfqpoint{4.440797in}{3.895360in}}%
\pgfpathlineto{\pgfqpoint{5.342615in}{4.026199in}}%
\pgfusepath{stroke}%
\end{pgfscope}%
\begin{pgfscope}%
\pgfpathrectangle{\pgfqpoint{0.608070in}{1.359165in}}{\pgfqpoint{4.960000in}{3.696000in}}%
\pgfusepath{clip}%
\pgfsetbuttcap%
\pgfsetmiterjoin%
\definecolor{currentfill}{rgb}{1.000000,0.498039,0.054902}%
\pgfsetfillcolor{currentfill}%
\pgfsetlinewidth{1.003750pt}%
\definecolor{currentstroke}{rgb}{1.000000,0.498039,0.054902}%
\pgfsetstrokecolor{currentstroke}%
\pgfsetdash{}{0pt}%
\pgfsys@defobject{currentmarker}{\pgfqpoint{-0.041667in}{-0.041667in}}{\pgfqpoint{0.041667in}{0.041667in}}{%
\pgfpathmoveto{\pgfqpoint{-0.000000in}{-0.041667in}}%
\pgfpathlineto{\pgfqpoint{0.041667in}{0.041667in}}%
\pgfpathlineto{\pgfqpoint{-0.041667in}{0.041667in}}%
\pgfpathclose%
\pgfusepath{stroke,fill}%
}%
\begin{pgfscope}%
\pgfsys@transformshift{0.833525in}{3.879837in}%
\pgfsys@useobject{currentmarker}{}%
\end{pgfscope}%
\begin{pgfscope}%
\pgfsys@transformshift{1.735343in}{3.865053in}%
\pgfsys@useobject{currentmarker}{}%
\end{pgfscope}%
\begin{pgfscope}%
\pgfsys@transformshift{2.637161in}{3.875402in}%
\pgfsys@useobject{currentmarker}{}%
\end{pgfscope}%
\begin{pgfscope}%
\pgfsys@transformshift{3.538979in}{3.799264in}%
\pgfsys@useobject{currentmarker}{}%
\end{pgfscope}%
\begin{pgfscope}%
\pgfsys@transformshift{4.440797in}{3.895360in}%
\pgfsys@useobject{currentmarker}{}%
\end{pgfscope}%
\begin{pgfscope}%
\pgfsys@transformshift{5.342615in}{4.026199in}%
\pgfsys@useobject{currentmarker}{}%
\end{pgfscope}%
\end{pgfscope}%
\begin{pgfscope}%
\pgfsetrectcap%
\pgfsetmiterjoin%
\pgfsetlinewidth{0.803000pt}%
\definecolor{currentstroke}{rgb}{0.000000,0.000000,0.000000}%
\pgfsetstrokecolor{currentstroke}%
\pgfsetdash{}{0pt}%
\pgfpathmoveto{\pgfqpoint{0.608070in}{1.359165in}}%
\pgfpathlineto{\pgfqpoint{0.608070in}{5.055165in}}%
\pgfusepath{stroke}%
\end{pgfscope}%
\begin{pgfscope}%
\pgfsetrectcap%
\pgfsetmiterjoin%
\pgfsetlinewidth{0.803000pt}%
\definecolor{currentstroke}{rgb}{0.000000,0.000000,0.000000}%
\pgfsetstrokecolor{currentstroke}%
\pgfsetdash{}{0pt}%
\pgfpathmoveto{\pgfqpoint{5.568070in}{1.359165in}}%
\pgfpathlineto{\pgfqpoint{5.568070in}{5.055165in}}%
\pgfusepath{stroke}%
\end{pgfscope}%
\begin{pgfscope}%
\pgfsetrectcap%
\pgfsetmiterjoin%
\pgfsetlinewidth{0.803000pt}%
\definecolor{currentstroke}{rgb}{0.000000,0.000000,0.000000}%
\pgfsetstrokecolor{currentstroke}%
\pgfsetdash{}{0pt}%
\pgfpathmoveto{\pgfqpoint{0.608070in}{1.359165in}}%
\pgfpathlineto{\pgfqpoint{5.568070in}{1.359165in}}%
\pgfusepath{stroke}%
\end{pgfscope}%
\begin{pgfscope}%
\pgfsetrectcap%
\pgfsetmiterjoin%
\pgfsetlinewidth{0.803000pt}%
\definecolor{currentstroke}{rgb}{0.000000,0.000000,0.000000}%
\pgfsetstrokecolor{currentstroke}%
\pgfsetdash{}{0pt}%
\pgfpathmoveto{\pgfqpoint{0.608070in}{5.055165in}}%
\pgfpathlineto{\pgfqpoint{5.568070in}{5.055165in}}%
\pgfusepath{stroke}%
\end{pgfscope}%
\begin{pgfscope}%
\pgfsetbuttcap%
\pgfsetmiterjoin%
\definecolor{currentfill}{rgb}{1.000000,1.000000,1.000000}%
\pgfsetfillcolor{currentfill}%
\pgfsetfillopacity{0.800000}%
\pgfsetlinewidth{1.003750pt}%
\definecolor{currentstroke}{rgb}{0.800000,0.800000,0.800000}%
\pgfsetstrokecolor{currentstroke}%
\pgfsetstrokeopacity{0.800000}%
\pgfsetdash{}{0pt}%
\pgfpathmoveto{\pgfqpoint{4.256682in}{4.536339in}}%
\pgfpathlineto{\pgfqpoint{5.470848in}{4.536339in}}%
\pgfpathquadraticcurveto{\pgfqpoint{5.498626in}{4.536339in}}{\pgfqpoint{5.498626in}{4.564117in}}%
\pgfpathlineto{\pgfqpoint{5.498626in}{4.957943in}}%
\pgfpathquadraticcurveto{\pgfqpoint{5.498626in}{4.985720in}}{\pgfqpoint{5.470848in}{4.985720in}}%
\pgfpathlineto{\pgfqpoint{4.256682in}{4.985720in}}%
\pgfpathquadraticcurveto{\pgfqpoint{4.228904in}{4.985720in}}{\pgfqpoint{4.228904in}{4.957943in}}%
\pgfpathlineto{\pgfqpoint{4.228904in}{4.564117in}}%
\pgfpathquadraticcurveto{\pgfqpoint{4.228904in}{4.536339in}}{\pgfqpoint{4.256682in}{4.536339in}}%
\pgfpathclose%
\pgfusepath{stroke,fill}%
\end{pgfscope}%
\begin{pgfscope}%
\pgfsetrectcap%
\pgfsetroundjoin%
\pgfsetlinewidth{1.505625pt}%
\definecolor{currentstroke}{rgb}{0.121569,0.466667,0.705882}%
\pgfsetstrokecolor{currentstroke}%
\pgfsetdash{}{0pt}%
\pgfpathmoveto{\pgfqpoint{4.284460in}{4.873253in}}%
\pgfpathlineto{\pgfqpoint{4.562238in}{4.873253in}}%
\pgfusepath{stroke}%
\end{pgfscope}%
\begin{pgfscope}%
\pgfsetbuttcap%
\pgfsetroundjoin%
\definecolor{currentfill}{rgb}{0.121569,0.466667,0.705882}%
\pgfsetfillcolor{currentfill}%
\pgfsetlinewidth{1.003750pt}%
\definecolor{currentstroke}{rgb}{0.121569,0.466667,0.705882}%
\pgfsetstrokecolor{currentstroke}%
\pgfsetdash{}{0pt}%
\pgfsys@defobject{currentmarker}{\pgfqpoint{-0.041667in}{-0.041667in}}{\pgfqpoint{0.041667in}{0.041667in}}{%
\pgfpathmoveto{\pgfqpoint{0.000000in}{-0.041667in}}%
\pgfpathcurveto{\pgfqpoint{0.011050in}{-0.041667in}}{\pgfqpoint{0.021649in}{-0.037276in}}{\pgfqpoint{0.029463in}{-0.029463in}}%
\pgfpathcurveto{\pgfqpoint{0.037276in}{-0.021649in}}{\pgfqpoint{0.041667in}{-0.011050in}}{\pgfqpoint{0.041667in}{0.000000in}}%
\pgfpathcurveto{\pgfqpoint{0.041667in}{0.011050in}}{\pgfqpoint{0.037276in}{0.021649in}}{\pgfqpoint{0.029463in}{0.029463in}}%
\pgfpathcurveto{\pgfqpoint{0.021649in}{0.037276in}}{\pgfqpoint{0.011050in}{0.041667in}}{\pgfqpoint{0.000000in}{0.041667in}}%
\pgfpathcurveto{\pgfqpoint{-0.011050in}{0.041667in}}{\pgfqpoint{-0.021649in}{0.037276in}}{\pgfqpoint{-0.029463in}{0.029463in}}%
\pgfpathcurveto{\pgfqpoint{-0.037276in}{0.021649in}}{\pgfqpoint{-0.041667in}{0.011050in}}{\pgfqpoint{-0.041667in}{0.000000in}}%
\pgfpathcurveto{\pgfqpoint{-0.041667in}{-0.011050in}}{\pgfqpoint{-0.037276in}{-0.021649in}}{\pgfqpoint{-0.029463in}{-0.029463in}}%
\pgfpathcurveto{\pgfqpoint{-0.021649in}{-0.037276in}}{\pgfqpoint{-0.011050in}{-0.041667in}}{\pgfqpoint{0.000000in}{-0.041667in}}%
\pgfpathclose%
\pgfusepath{stroke,fill}%
}%
\begin{pgfscope}%
\pgfsys@transformshift{4.423349in}{4.873253in}%
\pgfsys@useobject{currentmarker}{}%
\end{pgfscope}%
\end{pgfscope}%
\begin{pgfscope}%
\definecolor{textcolor}{rgb}{0.000000,0.000000,0.000000}%
\pgfsetstrokecolor{textcolor}%
\pgfsetfillcolor{textcolor}%
\pgftext[x=4.673349in,y=4.824642in,left,base]{\color{textcolor}\sffamily\fontsize{10.000000}{12.000000}\selectfont Pix2Vox++}%
\end{pgfscope}%
\begin{pgfscope}%
\pgfsetrectcap%
\pgfsetroundjoin%
\pgfsetlinewidth{1.505625pt}%
\definecolor{currentstroke}{rgb}{1.000000,0.498039,0.054902}%
\pgfsetstrokecolor{currentstroke}%
\pgfsetdash{}{0pt}%
\pgfpathmoveto{\pgfqpoint{4.284460in}{4.669396in}}%
\pgfpathlineto{\pgfqpoint{4.562238in}{4.669396in}}%
\pgfusepath{stroke}%
\end{pgfscope}%
\begin{pgfscope}%
\pgfsetbuttcap%
\pgfsetmiterjoin%
\definecolor{currentfill}{rgb}{1.000000,0.498039,0.054902}%
\pgfsetfillcolor{currentfill}%
\pgfsetlinewidth{1.003750pt}%
\definecolor{currentstroke}{rgb}{1.000000,0.498039,0.054902}%
\pgfsetstrokecolor{currentstroke}%
\pgfsetdash{}{0pt}%
\pgfsys@defobject{currentmarker}{\pgfqpoint{-0.041667in}{-0.041667in}}{\pgfqpoint{0.041667in}{0.041667in}}{%
\pgfpathmoveto{\pgfqpoint{-0.000000in}{-0.041667in}}%
\pgfpathlineto{\pgfqpoint{0.041667in}{0.041667in}}%
\pgfpathlineto{\pgfqpoint{-0.041667in}{0.041667in}}%
\pgfpathclose%
\pgfusepath{stroke,fill}%
}%
\begin{pgfscope}%
\pgfsys@transformshift{4.423349in}{4.669396in}%
\pgfsys@useobject{currentmarker}{}%
\end{pgfscope}%
\end{pgfscope}%
\begin{pgfscope}%
\definecolor{textcolor}{rgb}{0.000000,0.000000,0.000000}%
\pgfsetstrokecolor{textcolor}%
\pgfsetfillcolor{textcolor}%
\pgftext[x=4.673349in,y=4.620785in,left,base]{\color{textcolor}\sffamily\fontsize{10.000000}{12.000000}\selectfont Pix2Vox}%
\end{pgfscope}%
\end{pgfpicture}%
\makeatother%
\endgroup%
}
%    \caption{Line plot for the \gls{iou}  for baseline trained by mixing chair dataset from real and synthetic dataset with ratio of 50\%.
%    Observe that the \fls{IoU} is consistent for all types of randomization proving that mixed training negates loss from randomization.]}
%    \label{fig:ablation2}
%\end{figure}


\section{Discussion}\label{subsec:discussion}

Pix3D is a dataset that contains images from the real world as discussed in \autoref{sec:pix3d}.
However, these images are not just from the real environment.
Some images are taken from advertisements with text on the image, while others are without any background textures.
The furniture under observation may not even be in the focus of the image.

We train the models trained on synthetic \gls{free} datasets and evaluate them on real datasets to randomize most generic environments.
The ground truth has only 1 furniture model as output per image.
However, we also see a model predicting more than one piece of furniture as output when more than one types of furniture are in focus as in \autoref{fig:interesting1}.
The model seems to be learning the context of furniture placement without us feeding any additional information.

Another scenario where the model predicts something other than ground truth is when the furniture under focus is different.
In \autoref{fig:interesting2},, the furniture under observation is the desk. However, the model chose to focus on the chair in front of it to reconstruct.
These are some interesting findings inference for which can be the future scope of this thesis.

\begin{figure}[ht]
    \begin{tabular}{lll}
        \includegraphics[width=.3\linewidth]{/Users/apple/OVGU/Thesis/code/3dReconstruction/report/images/evaluation/reconstruction/interesting/p2ppp_table_input} &
        \includegraphics[width=.3\linewidth]{/Users/apple/OVGU/Thesis/code/3dReconstruction/report/images/evaluation/reconstruction/interesting/p2ppp_table1_original}&
        \includegraphics[width=.3\linewidth]{/Users/apple/OVGU/Thesis/code/3dReconstruction/report/images/evaluation/reconstruction/interesting/p2ppp_table1_output}\\
    \end{tabular}
    \caption[Sample Output with Multiple Predictions.]{A sample output where more than one furniture seem to be predicted.
            (left) Input image, (center) Ground truth, (right) Predicted output.
    The model seems to predict both the chair and the table with the context it learnt during training.}
    \label{fig:interesting1}
\end{figure}


\begin{figure}[ht]
    \begin{tabular}{lll}
        \includegraphics[width=.3\linewidth]{/Users/apple/OVGU/Thesis/code/3dReconstruction/report/images/evaluation/reconstruction/interesting/p2ppp_table2_input} &
        \includegraphics[width=.3\linewidth]{/Users/apple/OVGU/Thesis/code/3dReconstruction/report/images/evaluation/reconstruction/interesting/p2ppp_table1_original}&
        \includegraphics[width=.3\linewidth]{/Users/apple/OVGU/Thesis/code/3dReconstruction/report/images/evaluation/reconstruction/interesting/p2ppp_table2_output}\\
    \end{tabular}
    \caption[Sample Output with Wrong Prediction.]{A sample output where a more focused furniture is getting predicted instead of ground truth.
        (left) Input image, (center) Ground truth, (right) Predicted output. The model seems to predict the chair instead of the table.
    }
    \label{fig:interesting2}
\end{figure}
