\chapter{\iftoggle{german}{Konzept}{Concept}}\label{ch:concept}

In this chapter, we discuss the critical datasets used to create a new synthetic dataset and the reason for their selection in \autoref{sec:3d-furniture-models-from-pix3d}.
Furthermore, how we combine multiple datasets to our advantage of creating new datasets in the Ersatz environment.
In \autoref{sec:s2r:3d-free-a-pix3d-based-synthetic-dataset}, we will introduce the proposed \gls{free} dataset used for the 3D-reconstruction task.

The rationale for choosing Unity as a framework to build the pipeline will be discussed in \autoref{sec:unity-based-pipeline}.
In \autoref{sec:3d-scene-framework}, we introduce a new tool to create a synthetic dataset, and discuss in detail the design of the application and the rationale behind each module.

In the Deep Learning domain(\autoref{subsec:why-pix3d?}), we explain the rationale behind selecting baseline models for the 3D-reconstruction task.

In \autoref{sec:datasets}, we describe various synthetic datasets created using the 3D-Scene tool for 3D-Reconstruction tasks.
The collection will also contain chair datasets that will be used to study the effect of domain randomization.

\section{A New Dataset for 3D Reconstruction Task}\label{sec:3d-furniture-models-from-pix3d}
As discussed in \autoref{subsec:indoor-synthetic-datasets}, we have several synthetic datasets.
However, not many support 3D reconstruction tasks.
The dataset, which supports 3D reconstruction, does not have real-world images to compare the results.
This section discusses the datasets we considered for creating a new synthetic dataset and the reason for the selection.

\subsection{Disadvantages of Pix3D}\label{subsec:disadvantages-of-pix3d}
The distribution of models in Pix3D is as shown in \autoref{fig:pix3d_histogram}.
As we can see, the dataset distribution is uneven across classes, and more than 50\% of classes have less than 1000 images.

Though Pix3D set a benchmark for 2D-3D alignment, here are few disadvantages of using this real dataset.
\begin{enumerate}
    \item For Deep Learning approaches, we need large-scale data, and 14,600 images for Nine categories might not be sufficient.
    \item The orientation of an object is not randomized.
    \item The dataset does not provide 2.5D information (i.e., depth maps and normals), which can be crucial for 3D learning.
\end{enumerate}

\begin{figure}[!ht]
    \resizebox{0.49\textwidth}{6cm}{%% Creator: Matplotlib, PGF backend
%%
%% To include the figure in your LaTeX document, write
%%   \input{<filename>.pgf}
%%
%% Make sure the required packages are loaded in your preamble
%%   \usepackage{pgf}
%%
%% Figures using additional raster images can only be included by \input if
%% they are in the same directory as the main LaTeX file. For loading figures
%% from other directories you can use the `import` package
%%   \usepackage{import}
%%
%% and then include the figures with
%%   \import{<path to file>}{<filename>.pgf}
%%
%% Matplotlib used the following preamble
%%   \usepackage{fontspec}
%%   \setmainfont{DejaVuSerif.ttf}[Path=\detokenize{/Users/apple/opt/anaconda3/envs/kaolin/lib/python3.7/site-packages/matplotlib/mpl-data/fonts/ttf/}]
%%   \setsansfont{DejaVuSans.ttf}[Path=\detokenize{/Users/apple/opt/anaconda3/envs/kaolin/lib/python3.7/site-packages/matplotlib/mpl-data/fonts/ttf/}]
%%   \setmonofont{DejaVuSansMono.ttf}[Path=\detokenize{/Users/apple/opt/anaconda3/envs/kaolin/lib/python3.7/site-packages/matplotlib/mpl-data/fonts/ttf/}]
%%
\begingroup%
\makeatletter%
\begin{pgfpicture}%
\pgfpathrectangle{\pgfpointorigin}{\pgfqpoint{5.752068in}{4.226888in}}%
\pgfusepath{use as bounding box, clip}%
\begin{pgfscope}%
\pgfsetbuttcap%
\pgfsetmiterjoin%
\definecolor{currentfill}{rgb}{1.000000,1.000000,1.000000}%
\pgfsetfillcolor{currentfill}%
\pgfsetlinewidth{0.000000pt}%
\definecolor{currentstroke}{rgb}{1.000000,1.000000,1.000000}%
\pgfsetstrokecolor{currentstroke}%
\pgfsetdash{}{0pt}%
\pgfpathmoveto{\pgfqpoint{0.000000in}{0.000000in}}%
\pgfpathlineto{\pgfqpoint{5.752068in}{0.000000in}}%
\pgfpathlineto{\pgfqpoint{5.752068in}{4.226888in}}%
\pgfpathlineto{\pgfqpoint{0.000000in}{4.226888in}}%
\pgfpathclose%
\pgfusepath{fill}%
\end{pgfscope}%
\begin{pgfscope}%
\pgfsetbuttcap%
\pgfsetmiterjoin%
\definecolor{currentfill}{rgb}{1.000000,1.000000,1.000000}%
\pgfsetfillcolor{currentfill}%
\pgfsetlinewidth{0.000000pt}%
\definecolor{currentstroke}{rgb}{0.000000,0.000000,0.000000}%
\pgfsetstrokecolor{currentstroke}%
\pgfsetstrokeopacity{0.000000}%
\pgfsetdash{}{0pt}%
\pgfpathmoveto{\pgfqpoint{0.692068in}{0.385400in}}%
\pgfpathlineto{\pgfqpoint{5.652068in}{0.385400in}}%
\pgfpathlineto{\pgfqpoint{5.652068in}{4.081400in}}%
\pgfpathlineto{\pgfqpoint{0.692068in}{4.081400in}}%
\pgfpathclose%
\pgfusepath{fill}%
\end{pgfscope}%
\begin{pgfscope}%
\pgfpathrectangle{\pgfqpoint{0.692068in}{0.385400in}}{\pgfqpoint{4.960000in}{3.696000in}}%
\pgfusepath{clip}%
\pgfsetbuttcap%
\pgfsetmiterjoin%
\definecolor{currentfill}{rgb}{0.121569,0.466667,0.705882}%
\pgfsetfillcolor{currentfill}%
\pgfsetlinewidth{0.000000pt}%
\definecolor{currentstroke}{rgb}{0.000000,0.000000,0.000000}%
\pgfsetstrokecolor{currentstroke}%
\pgfsetstrokeopacity{0.000000}%
\pgfsetdash{}{0pt}%
\pgfpathmoveto{\pgfqpoint{0.917523in}{0.385400in}}%
\pgfpathlineto{\pgfqpoint{1.448004in}{0.385400in}}%
\pgfpathlineto{\pgfqpoint{1.448004in}{0.608208in}}%
\pgfpathlineto{\pgfqpoint{0.917523in}{0.608208in}}%
\pgfpathclose%
\pgfusepath{fill}%
\end{pgfscope}%
\begin{pgfscope}%
\pgfpathrectangle{\pgfqpoint{0.692068in}{0.385400in}}{\pgfqpoint{4.960000in}{3.696000in}}%
\pgfusepath{clip}%
\pgfsetbuttcap%
\pgfsetmiterjoin%
\definecolor{currentfill}{rgb}{0.121569,0.466667,0.705882}%
\pgfsetfillcolor{currentfill}%
\pgfsetlinewidth{0.000000pt}%
\definecolor{currentstroke}{rgb}{0.000000,0.000000,0.000000}%
\pgfsetstrokecolor{currentstroke}%
\pgfsetstrokeopacity{0.000000}%
\pgfsetdash{}{0pt}%
\pgfpathmoveto{\pgfqpoint{1.580624in}{0.385400in}}%
\pgfpathlineto{\pgfqpoint{2.111105in}{0.385400in}}%
\pgfpathlineto{\pgfqpoint{2.111105in}{1.296804in}}%
\pgfpathlineto{\pgfqpoint{1.580624in}{1.296804in}}%
\pgfpathclose%
\pgfusepath{fill}%
\end{pgfscope}%
\begin{pgfscope}%
\pgfpathrectangle{\pgfqpoint{0.692068in}{0.385400in}}{\pgfqpoint{4.960000in}{3.696000in}}%
\pgfusepath{clip}%
\pgfsetbuttcap%
\pgfsetmiterjoin%
\definecolor{currentfill}{rgb}{0.121569,0.466667,0.705882}%
\pgfsetfillcolor{currentfill}%
\pgfsetlinewidth{0.000000pt}%
\definecolor{currentstroke}{rgb}{0.000000,0.000000,0.000000}%
\pgfsetstrokecolor{currentstroke}%
\pgfsetstrokeopacity{0.000000}%
\pgfsetdash{}{0pt}%
\pgfpathmoveto{\pgfqpoint{2.243726in}{0.385400in}}%
\pgfpathlineto{\pgfqpoint{2.774207in}{0.385400in}}%
\pgfpathlineto{\pgfqpoint{2.774207in}{1.028151in}}%
\pgfpathlineto{\pgfqpoint{2.243726in}{1.028151in}}%
\pgfpathclose%
\pgfusepath{fill}%
\end{pgfscope}%
\begin{pgfscope}%
\pgfpathrectangle{\pgfqpoint{0.692068in}{0.385400in}}{\pgfqpoint{4.960000in}{3.696000in}}%
\pgfusepath{clip}%
\pgfsetbuttcap%
\pgfsetmiterjoin%
\definecolor{currentfill}{rgb}{0.121569,0.466667,0.705882}%
\pgfsetfillcolor{currentfill}%
\pgfsetlinewidth{0.000000pt}%
\definecolor{currentstroke}{rgb}{0.000000,0.000000,0.000000}%
\pgfsetstrokecolor{currentstroke}%
\pgfsetstrokeopacity{0.000000}%
\pgfsetdash{}{0pt}%
\pgfpathmoveto{\pgfqpoint{2.906827in}{0.385400in}}%
\pgfpathlineto{\pgfqpoint{3.437309in}{0.385400in}}%
\pgfpathlineto{\pgfqpoint{3.437309in}{3.905400in}}%
\pgfpathlineto{\pgfqpoint{2.906827in}{3.905400in}}%
\pgfpathclose%
\pgfusepath{fill}%
\end{pgfscope}%
\begin{pgfscope}%
\pgfpathrectangle{\pgfqpoint{0.692068in}{0.385400in}}{\pgfqpoint{4.960000in}{3.696000in}}%
\pgfusepath{clip}%
\pgfsetbuttcap%
\pgfsetmiterjoin%
\definecolor{currentfill}{rgb}{0.121569,0.466667,0.705882}%
\pgfsetfillcolor{currentfill}%
\pgfsetlinewidth{0.000000pt}%
\definecolor{currentstroke}{rgb}{0.000000,0.000000,0.000000}%
\pgfsetstrokecolor{currentstroke}%
\pgfsetstrokeopacity{0.000000}%
\pgfsetdash{}{0pt}%
\pgfpathmoveto{\pgfqpoint{3.569929in}{0.385400in}}%
\pgfpathlineto{\pgfqpoint{4.100410in}{0.385400in}}%
\pgfpathlineto{\pgfqpoint{4.100410in}{0.716403in}}%
\pgfpathlineto{\pgfqpoint{3.569929in}{0.716403in}}%
\pgfpathclose%
\pgfusepath{fill}%
\end{pgfscope}%
\begin{pgfscope}%
\pgfpathrectangle{\pgfqpoint{0.692068in}{0.385400in}}{\pgfqpoint{4.960000in}{3.696000in}}%
\pgfusepath{clip}%
\pgfsetbuttcap%
\pgfsetmiterjoin%
\definecolor{currentfill}{rgb}{0.121569,0.466667,0.705882}%
\pgfsetfillcolor{currentfill}%
\pgfsetlinewidth{0.000000pt}%
\definecolor{currentstroke}{rgb}{0.000000,0.000000,0.000000}%
\pgfsetstrokecolor{currentstroke}%
\pgfsetstrokeopacity{0.000000}%
\pgfsetdash{}{0pt}%
\pgfpathmoveto{\pgfqpoint{4.233031in}{0.385400in}}%
\pgfpathlineto{\pgfqpoint{4.763512in}{0.385400in}}%
\pgfpathlineto{\pgfqpoint{4.763512in}{2.171532in}}%
\pgfpathlineto{\pgfqpoint{4.233031in}{2.171532in}}%
\pgfpathclose%
\pgfusepath{fill}%
\end{pgfscope}%
\begin{pgfscope}%
\pgfpathrectangle{\pgfqpoint{0.692068in}{0.385400in}}{\pgfqpoint{4.960000in}{3.696000in}}%
\pgfusepath{clip}%
\pgfsetbuttcap%
\pgfsetmiterjoin%
\definecolor{currentfill}{rgb}{0.121569,0.466667,0.705882}%
\pgfsetfillcolor{currentfill}%
\pgfsetlinewidth{0.000000pt}%
\definecolor{currentstroke}{rgb}{0.000000,0.000000,0.000000}%
\pgfsetstrokecolor{currentstroke}%
\pgfsetstrokeopacity{0.000000}%
\pgfsetdash{}{0pt}%
\pgfpathmoveto{\pgfqpoint{4.896132in}{0.385400in}}%
\pgfpathlineto{\pgfqpoint{5.426614in}{0.385400in}}%
\pgfpathlineto{\pgfqpoint{5.426614in}{2.100013in}}%
\pgfpathlineto{\pgfqpoint{4.896132in}{2.100013in}}%
\pgfpathclose%
\pgfusepath{fill}%
\end{pgfscope}%
\begin{pgfscope}%
\pgfsetbuttcap%
\pgfsetroundjoin%
\definecolor{currentfill}{rgb}{0.000000,0.000000,0.000000}%
\pgfsetfillcolor{currentfill}%
\pgfsetlinewidth{0.803000pt}%
\definecolor{currentstroke}{rgb}{0.000000,0.000000,0.000000}%
\pgfsetstrokecolor{currentstroke}%
\pgfsetdash{}{0pt}%
\pgfsys@defobject{currentmarker}{\pgfqpoint{0.000000in}{-0.048611in}}{\pgfqpoint{0.000000in}{0.000000in}}{%
\pgfpathmoveto{\pgfqpoint{0.000000in}{0.000000in}}%
\pgfpathlineto{\pgfqpoint{0.000000in}{-0.048611in}}%
\pgfusepath{stroke,fill}%
}%
\begin{pgfscope}%
\pgfsys@transformshift{1.182763in}{0.385400in}%
\pgfsys@useobject{currentmarker}{}%
\end{pgfscope}%
\end{pgfscope}%
\begin{pgfscope}%
\definecolor{textcolor}{rgb}{0.000000,0.000000,0.000000}%
\pgfsetstrokecolor{textcolor}%
\pgfsetfillcolor{textcolor}%
\pgftext[x=1.182763in,y=0.288178in,,top]{\color{textcolor}\sffamily\fontsize{14.000000}{16.800000}\selectfont wardrobe}%
\end{pgfscope}%
\begin{pgfscope}%
\pgfsetbuttcap%
\pgfsetroundjoin%
\definecolor{currentfill}{rgb}{0.000000,0.000000,0.000000}%
\pgfsetfillcolor{currentfill}%
\pgfsetlinewidth{0.803000pt}%
\definecolor{currentstroke}{rgb}{0.000000,0.000000,0.000000}%
\pgfsetstrokecolor{currentstroke}%
\pgfsetdash{}{0pt}%
\pgfsys@defobject{currentmarker}{\pgfqpoint{0.000000in}{-0.048611in}}{\pgfqpoint{0.000000in}{0.000000in}}{%
\pgfpathmoveto{\pgfqpoint{0.000000in}{0.000000in}}%
\pgfpathlineto{\pgfqpoint{0.000000in}{-0.048611in}}%
\pgfusepath{stroke,fill}%
}%
\begin{pgfscope}%
\pgfsys@transformshift{1.845865in}{0.385400in}%
\pgfsys@useobject{currentmarker}{}%
\end{pgfscope}%
\end{pgfscope}%
\begin{pgfscope}%
\definecolor{textcolor}{rgb}{0.000000,0.000000,0.000000}%
\pgfsetstrokecolor{textcolor}%
\pgfsetfillcolor{textcolor}%
\pgftext[x=1.845865in,y=0.288178in,,top]{\color{textcolor}\sffamily\fontsize{14.000000}{16.800000}\selectfont bed}%
\end{pgfscope}%
\begin{pgfscope}%
\pgfsetbuttcap%
\pgfsetroundjoin%
\definecolor{currentfill}{rgb}{0.000000,0.000000,0.000000}%
\pgfsetfillcolor{currentfill}%
\pgfsetlinewidth{0.803000pt}%
\definecolor{currentstroke}{rgb}{0.000000,0.000000,0.000000}%
\pgfsetstrokecolor{currentstroke}%
\pgfsetdash{}{0pt}%
\pgfsys@defobject{currentmarker}{\pgfqpoint{0.000000in}{-0.048611in}}{\pgfqpoint{0.000000in}{0.000000in}}{%
\pgfpathmoveto{\pgfqpoint{0.000000in}{0.000000in}}%
\pgfpathlineto{\pgfqpoint{0.000000in}{-0.048611in}}%
\pgfusepath{stroke,fill}%
}%
\begin{pgfscope}%
\pgfsys@transformshift{2.508966in}{0.385400in}%
\pgfsys@useobject{currentmarker}{}%
\end{pgfscope}%
\end{pgfscope}%
\begin{pgfscope}%
\definecolor{textcolor}{rgb}{0.000000,0.000000,0.000000}%
\pgfsetstrokecolor{textcolor}%
\pgfsetfillcolor{textcolor}%
\pgftext[x=2.508966in,y=0.288178in,,top]{\color{textcolor}\sffamily\fontsize{14.000000}{16.800000}\selectfont desk}%
\end{pgfscope}%
\begin{pgfscope}%
\pgfsetbuttcap%
\pgfsetroundjoin%
\definecolor{currentfill}{rgb}{0.000000,0.000000,0.000000}%
\pgfsetfillcolor{currentfill}%
\pgfsetlinewidth{0.803000pt}%
\definecolor{currentstroke}{rgb}{0.000000,0.000000,0.000000}%
\pgfsetstrokecolor{currentstroke}%
\pgfsetdash{}{0pt}%
\pgfsys@defobject{currentmarker}{\pgfqpoint{0.000000in}{-0.048611in}}{\pgfqpoint{0.000000in}{0.000000in}}{%
\pgfpathmoveto{\pgfqpoint{0.000000in}{0.000000in}}%
\pgfpathlineto{\pgfqpoint{0.000000in}{-0.048611in}}%
\pgfusepath{stroke,fill}%
}%
\begin{pgfscope}%
\pgfsys@transformshift{3.172068in}{0.385400in}%
\pgfsys@useobject{currentmarker}{}%
\end{pgfscope}%
\end{pgfscope}%
\begin{pgfscope}%
\definecolor{textcolor}{rgb}{0.000000,0.000000,0.000000}%
\pgfsetstrokecolor{textcolor}%
\pgfsetfillcolor{textcolor}%
\pgftext[x=3.172068in,y=0.288178in,,top]{\color{textcolor}\sffamily\fontsize{14.000000}{16.800000}\selectfont chair}%
\end{pgfscope}%
\begin{pgfscope}%
\pgfsetbuttcap%
\pgfsetroundjoin%
\definecolor{currentfill}{rgb}{0.000000,0.000000,0.000000}%
\pgfsetfillcolor{currentfill}%
\pgfsetlinewidth{0.803000pt}%
\definecolor{currentstroke}{rgb}{0.000000,0.000000,0.000000}%
\pgfsetstrokecolor{currentstroke}%
\pgfsetdash{}{0pt}%
\pgfsys@defobject{currentmarker}{\pgfqpoint{0.000000in}{-0.048611in}}{\pgfqpoint{0.000000in}{0.000000in}}{%
\pgfpathmoveto{\pgfqpoint{0.000000in}{0.000000in}}%
\pgfpathlineto{\pgfqpoint{0.000000in}{-0.048611in}}%
\pgfusepath{stroke,fill}%
}%
\begin{pgfscope}%
\pgfsys@transformshift{3.835170in}{0.385400in}%
\pgfsys@useobject{currentmarker}{}%
\end{pgfscope}%
\end{pgfscope}%
\begin{pgfscope}%
\definecolor{textcolor}{rgb}{0.000000,0.000000,0.000000}%
\pgfsetstrokecolor{textcolor}%
\pgfsetfillcolor{textcolor}%
\pgftext[x=3.835170in,y=0.288178in,,top]{\color{textcolor}\sffamily\fontsize{14.000000}{16.800000}\selectfont bookcase}%
\end{pgfscope}%
\begin{pgfscope}%
\pgfsetbuttcap%
\pgfsetroundjoin%
\definecolor{currentfill}{rgb}{0.000000,0.000000,0.000000}%
\pgfsetfillcolor{currentfill}%
\pgfsetlinewidth{0.803000pt}%
\definecolor{currentstroke}{rgb}{0.000000,0.000000,0.000000}%
\pgfsetstrokecolor{currentstroke}%
\pgfsetdash{}{0pt}%
\pgfsys@defobject{currentmarker}{\pgfqpoint{0.000000in}{-0.048611in}}{\pgfqpoint{0.000000in}{0.000000in}}{%
\pgfpathmoveto{\pgfqpoint{0.000000in}{0.000000in}}%
\pgfpathlineto{\pgfqpoint{0.000000in}{-0.048611in}}%
\pgfusepath{stroke,fill}%
}%
\begin{pgfscope}%
\pgfsys@transformshift{4.498271in}{0.385400in}%
\pgfsys@useobject{currentmarker}{}%
\end{pgfscope}%
\end{pgfscope}%
\begin{pgfscope}%
\definecolor{textcolor}{rgb}{0.000000,0.000000,0.000000}%
\pgfsetstrokecolor{textcolor}%
\pgfsetfillcolor{textcolor}%
\pgftext[x=4.498271in,y=0.288178in,,top]{\color{textcolor}\sffamily\fontsize{14.000000}{16.800000}\selectfont sofa}%
\end{pgfscope}%
\begin{pgfscope}%
\pgfsetbuttcap%
\pgfsetroundjoin%
\definecolor{currentfill}{rgb}{0.000000,0.000000,0.000000}%
\pgfsetfillcolor{currentfill}%
\pgfsetlinewidth{0.803000pt}%
\definecolor{currentstroke}{rgb}{0.000000,0.000000,0.000000}%
\pgfsetstrokecolor{currentstroke}%
\pgfsetdash{}{0pt}%
\pgfsys@defobject{currentmarker}{\pgfqpoint{0.000000in}{-0.048611in}}{\pgfqpoint{0.000000in}{0.000000in}}{%
\pgfpathmoveto{\pgfqpoint{0.000000in}{0.000000in}}%
\pgfpathlineto{\pgfqpoint{0.000000in}{-0.048611in}}%
\pgfusepath{stroke,fill}%
}%
\begin{pgfscope}%
\pgfsys@transformshift{5.161373in}{0.385400in}%
\pgfsys@useobject{currentmarker}{}%
\end{pgfscope}%
\end{pgfscope}%
\begin{pgfscope}%
\definecolor{textcolor}{rgb}{0.000000,0.000000,0.000000}%
\pgfsetstrokecolor{textcolor}%
\pgfsetfillcolor{textcolor}%
\pgftext[x=5.161373in,y=0.288178in,,top]{\color{textcolor}\sffamily\fontsize{14.000000}{16.800000}\selectfont table}%
\end{pgfscope}%
\begin{pgfscope}%
\pgfsetbuttcap%
\pgfsetroundjoin%
\definecolor{currentfill}{rgb}{0.000000,0.000000,0.000000}%
\pgfsetfillcolor{currentfill}%
\pgfsetlinewidth{0.803000pt}%
\definecolor{currentstroke}{rgb}{0.000000,0.000000,0.000000}%
\pgfsetstrokecolor{currentstroke}%
\pgfsetdash{}{0pt}%
\pgfsys@defobject{currentmarker}{\pgfqpoint{-0.048611in}{0.000000in}}{\pgfqpoint{-0.000000in}{0.000000in}}{%
\pgfpathmoveto{\pgfqpoint{-0.000000in}{0.000000in}}%
\pgfpathlineto{\pgfqpoint{-0.048611in}{0.000000in}}%
\pgfusepath{stroke,fill}%
}%
\begin{pgfscope}%
\pgfsys@transformshift{0.692068in}{0.385400in}%
\pgfsys@useobject{currentmarker}{}%
\end{pgfscope}%
\end{pgfscope}%
\begin{pgfscope}%
\definecolor{textcolor}{rgb}{0.000000,0.000000,0.000000}%
\pgfsetstrokecolor{textcolor}%
\pgfsetfillcolor{textcolor}%
\pgftext[x=0.471134in, y=0.311534in, left, base]{\color{textcolor}\sffamily\fontsize{14.000000}{16.800000}\selectfont 0}%
\end{pgfscope}%
\begin{pgfscope}%
\pgfsetbuttcap%
\pgfsetroundjoin%
\definecolor{currentfill}{rgb}{0.000000,0.000000,0.000000}%
\pgfsetfillcolor{currentfill}%
\pgfsetlinewidth{0.803000pt}%
\definecolor{currentstroke}{rgb}{0.000000,0.000000,0.000000}%
\pgfsetstrokecolor{currentstroke}%
\pgfsetdash{}{0pt}%
\pgfsys@defobject{currentmarker}{\pgfqpoint{-0.048611in}{0.000000in}}{\pgfqpoint{-0.000000in}{0.000000in}}{%
\pgfpathmoveto{\pgfqpoint{-0.000000in}{0.000000in}}%
\pgfpathlineto{\pgfqpoint{-0.048611in}{0.000000in}}%
\pgfusepath{stroke,fill}%
}%
\begin{pgfscope}%
\pgfsys@transformshift{0.692068in}{0.843853in}%
\pgfsys@useobject{currentmarker}{}%
\end{pgfscope}%
\end{pgfscope}%
\begin{pgfscope}%
\definecolor{textcolor}{rgb}{0.000000,0.000000,0.000000}%
\pgfsetstrokecolor{textcolor}%
\pgfsetfillcolor{textcolor}%
\pgftext[x=0.223711in, y=0.769987in, left, base]{\color{textcolor}\sffamily\fontsize{14.000000}{16.800000}\selectfont 500}%
\end{pgfscope}%
\begin{pgfscope}%
\pgfsetbuttcap%
\pgfsetroundjoin%
\definecolor{currentfill}{rgb}{0.000000,0.000000,0.000000}%
\pgfsetfillcolor{currentfill}%
\pgfsetlinewidth{0.803000pt}%
\definecolor{currentstroke}{rgb}{0.000000,0.000000,0.000000}%
\pgfsetstrokecolor{currentstroke}%
\pgfsetdash{}{0pt}%
\pgfsys@defobject{currentmarker}{\pgfqpoint{-0.048611in}{0.000000in}}{\pgfqpoint{-0.000000in}{0.000000in}}{%
\pgfpathmoveto{\pgfqpoint{-0.000000in}{0.000000in}}%
\pgfpathlineto{\pgfqpoint{-0.048611in}{0.000000in}}%
\pgfusepath{stroke,fill}%
}%
\begin{pgfscope}%
\pgfsys@transformshift{0.692068in}{1.302306in}%
\pgfsys@useobject{currentmarker}{}%
\end{pgfscope}%
\end{pgfscope}%
\begin{pgfscope}%
\definecolor{textcolor}{rgb}{0.000000,0.000000,0.000000}%
\pgfsetstrokecolor{textcolor}%
\pgfsetfillcolor{textcolor}%
\pgftext[x=0.100000in, y=1.228440in, left, base]{\color{textcolor}\sffamily\fontsize{14.000000}{16.800000}\selectfont 1000}%
\end{pgfscope}%
\begin{pgfscope}%
\pgfsetbuttcap%
\pgfsetroundjoin%
\definecolor{currentfill}{rgb}{0.000000,0.000000,0.000000}%
\pgfsetfillcolor{currentfill}%
\pgfsetlinewidth{0.803000pt}%
\definecolor{currentstroke}{rgb}{0.000000,0.000000,0.000000}%
\pgfsetstrokecolor{currentstroke}%
\pgfsetdash{}{0pt}%
\pgfsys@defobject{currentmarker}{\pgfqpoint{-0.048611in}{0.000000in}}{\pgfqpoint{-0.000000in}{0.000000in}}{%
\pgfpathmoveto{\pgfqpoint{-0.000000in}{0.000000in}}%
\pgfpathlineto{\pgfqpoint{-0.048611in}{0.000000in}}%
\pgfusepath{stroke,fill}%
}%
\begin{pgfscope}%
\pgfsys@transformshift{0.692068in}{1.760758in}%
\pgfsys@useobject{currentmarker}{}%
\end{pgfscope}%
\end{pgfscope}%
\begin{pgfscope}%
\definecolor{textcolor}{rgb}{0.000000,0.000000,0.000000}%
\pgfsetstrokecolor{textcolor}%
\pgfsetfillcolor{textcolor}%
\pgftext[x=0.100000in, y=1.686892in, left, base]{\color{textcolor}\sffamily\fontsize{14.000000}{16.800000}\selectfont 1500}%
\end{pgfscope}%
\begin{pgfscope}%
\pgfsetbuttcap%
\pgfsetroundjoin%
\definecolor{currentfill}{rgb}{0.000000,0.000000,0.000000}%
\pgfsetfillcolor{currentfill}%
\pgfsetlinewidth{0.803000pt}%
\definecolor{currentstroke}{rgb}{0.000000,0.000000,0.000000}%
\pgfsetstrokecolor{currentstroke}%
\pgfsetdash{}{0pt}%
\pgfsys@defobject{currentmarker}{\pgfqpoint{-0.048611in}{0.000000in}}{\pgfqpoint{-0.000000in}{0.000000in}}{%
\pgfpathmoveto{\pgfqpoint{-0.000000in}{0.000000in}}%
\pgfpathlineto{\pgfqpoint{-0.048611in}{0.000000in}}%
\pgfusepath{stroke,fill}%
}%
\begin{pgfscope}%
\pgfsys@transformshift{0.692068in}{2.219211in}%
\pgfsys@useobject{currentmarker}{}%
\end{pgfscope}%
\end{pgfscope}%
\begin{pgfscope}%
\definecolor{textcolor}{rgb}{0.000000,0.000000,0.000000}%
\pgfsetstrokecolor{textcolor}%
\pgfsetfillcolor{textcolor}%
\pgftext[x=0.100000in, y=2.145345in, left, base]{\color{textcolor}\sffamily\fontsize{14.000000}{16.800000}\selectfont 2000}%
\end{pgfscope}%
\begin{pgfscope}%
\pgfsetbuttcap%
\pgfsetroundjoin%
\definecolor{currentfill}{rgb}{0.000000,0.000000,0.000000}%
\pgfsetfillcolor{currentfill}%
\pgfsetlinewidth{0.803000pt}%
\definecolor{currentstroke}{rgb}{0.000000,0.000000,0.000000}%
\pgfsetstrokecolor{currentstroke}%
\pgfsetdash{}{0pt}%
\pgfsys@defobject{currentmarker}{\pgfqpoint{-0.048611in}{0.000000in}}{\pgfqpoint{-0.000000in}{0.000000in}}{%
\pgfpathmoveto{\pgfqpoint{-0.000000in}{0.000000in}}%
\pgfpathlineto{\pgfqpoint{-0.048611in}{0.000000in}}%
\pgfusepath{stroke,fill}%
}%
\begin{pgfscope}%
\pgfsys@transformshift{0.692068in}{2.677664in}%
\pgfsys@useobject{currentmarker}{}%
\end{pgfscope}%
\end{pgfscope}%
\begin{pgfscope}%
\definecolor{textcolor}{rgb}{0.000000,0.000000,0.000000}%
\pgfsetstrokecolor{textcolor}%
\pgfsetfillcolor{textcolor}%
\pgftext[x=0.100000in, y=2.603798in, left, base]{\color{textcolor}\sffamily\fontsize{14.000000}{16.800000}\selectfont 2500}%
\end{pgfscope}%
\begin{pgfscope}%
\pgfsetbuttcap%
\pgfsetroundjoin%
\definecolor{currentfill}{rgb}{0.000000,0.000000,0.000000}%
\pgfsetfillcolor{currentfill}%
\pgfsetlinewidth{0.803000pt}%
\definecolor{currentstroke}{rgb}{0.000000,0.000000,0.000000}%
\pgfsetstrokecolor{currentstroke}%
\pgfsetdash{}{0pt}%
\pgfsys@defobject{currentmarker}{\pgfqpoint{-0.048611in}{0.000000in}}{\pgfqpoint{-0.000000in}{0.000000in}}{%
\pgfpathmoveto{\pgfqpoint{-0.000000in}{0.000000in}}%
\pgfpathlineto{\pgfqpoint{-0.048611in}{0.000000in}}%
\pgfusepath{stroke,fill}%
}%
\begin{pgfscope}%
\pgfsys@transformshift{0.692068in}{3.136117in}%
\pgfsys@useobject{currentmarker}{}%
\end{pgfscope}%
\end{pgfscope}%
\begin{pgfscope}%
\definecolor{textcolor}{rgb}{0.000000,0.000000,0.000000}%
\pgfsetstrokecolor{textcolor}%
\pgfsetfillcolor{textcolor}%
\pgftext[x=0.100000in, y=3.062250in, left, base]{\color{textcolor}\sffamily\fontsize{14.000000}{16.800000}\selectfont 3000}%
\end{pgfscope}%
\begin{pgfscope}%
\pgfsetbuttcap%
\pgfsetroundjoin%
\definecolor{currentfill}{rgb}{0.000000,0.000000,0.000000}%
\pgfsetfillcolor{currentfill}%
\pgfsetlinewidth{0.803000pt}%
\definecolor{currentstroke}{rgb}{0.000000,0.000000,0.000000}%
\pgfsetstrokecolor{currentstroke}%
\pgfsetdash{}{0pt}%
\pgfsys@defobject{currentmarker}{\pgfqpoint{-0.048611in}{0.000000in}}{\pgfqpoint{-0.000000in}{0.000000in}}{%
\pgfpathmoveto{\pgfqpoint{-0.000000in}{0.000000in}}%
\pgfpathlineto{\pgfqpoint{-0.048611in}{0.000000in}}%
\pgfusepath{stroke,fill}%
}%
\begin{pgfscope}%
\pgfsys@transformshift{0.692068in}{3.594569in}%
\pgfsys@useobject{currentmarker}{}%
\end{pgfscope}%
\end{pgfscope}%
\begin{pgfscope}%
\definecolor{textcolor}{rgb}{0.000000,0.000000,0.000000}%
\pgfsetstrokecolor{textcolor}%
\pgfsetfillcolor{textcolor}%
\pgftext[x=0.100000in, y=3.520703in, left, base]{\color{textcolor}\sffamily\fontsize{14.000000}{16.800000}\selectfont 3500}%
\end{pgfscope}%
\begin{pgfscope}%
\pgfsetbuttcap%
\pgfsetroundjoin%
\definecolor{currentfill}{rgb}{0.000000,0.000000,0.000000}%
\pgfsetfillcolor{currentfill}%
\pgfsetlinewidth{0.803000pt}%
\definecolor{currentstroke}{rgb}{0.000000,0.000000,0.000000}%
\pgfsetstrokecolor{currentstroke}%
\pgfsetdash{}{0pt}%
\pgfsys@defobject{currentmarker}{\pgfqpoint{-0.048611in}{0.000000in}}{\pgfqpoint{-0.000000in}{0.000000in}}{%
\pgfpathmoveto{\pgfqpoint{-0.000000in}{0.000000in}}%
\pgfpathlineto{\pgfqpoint{-0.048611in}{0.000000in}}%
\pgfusepath{stroke,fill}%
}%
\begin{pgfscope}%
\pgfsys@transformshift{0.692068in}{4.053022in}%
\pgfsys@useobject{currentmarker}{}%
\end{pgfscope}%
\end{pgfscope}%
\begin{pgfscope}%
\definecolor{textcolor}{rgb}{0.000000,0.000000,0.000000}%
\pgfsetstrokecolor{textcolor}%
\pgfsetfillcolor{textcolor}%
\pgftext[x=0.100000in, y=3.979156in, left, base]{\color{textcolor}\sffamily\fontsize{14.000000}{16.800000}\selectfont 4000}%
\end{pgfscope}%
\begin{pgfscope}%
\pgfsetrectcap%
\pgfsetmiterjoin%
\pgfsetlinewidth{0.803000pt}%
\definecolor{currentstroke}{rgb}{0.000000,0.000000,0.000000}%
\pgfsetstrokecolor{currentstroke}%
\pgfsetdash{}{0pt}%
\pgfpathmoveto{\pgfqpoint{0.692068in}{0.385400in}}%
\pgfpathlineto{\pgfqpoint{0.692068in}{4.081400in}}%
\pgfusepath{stroke}%
\end{pgfscope}%
\begin{pgfscope}%
\pgfsetrectcap%
\pgfsetmiterjoin%
\pgfsetlinewidth{0.803000pt}%
\definecolor{currentstroke}{rgb}{0.000000,0.000000,0.000000}%
\pgfsetstrokecolor{currentstroke}%
\pgfsetdash{}{0pt}%
\pgfpathmoveto{\pgfqpoint{5.652068in}{0.385400in}}%
\pgfpathlineto{\pgfqpoint{5.652068in}{4.081400in}}%
\pgfusepath{stroke}%
\end{pgfscope}%
\begin{pgfscope}%
\pgfsetrectcap%
\pgfsetmiterjoin%
\pgfsetlinewidth{0.803000pt}%
\definecolor{currentstroke}{rgb}{0.000000,0.000000,0.000000}%
\pgfsetstrokecolor{currentstroke}%
\pgfsetdash{}{0pt}%
\pgfpathmoveto{\pgfqpoint{0.692068in}{0.385400in}}%
\pgfpathlineto{\pgfqpoint{5.652068in}{0.385400in}}%
\pgfusepath{stroke}%
\end{pgfscope}%
\begin{pgfscope}%
\pgfsetrectcap%
\pgfsetmiterjoin%
\pgfsetlinewidth{0.803000pt}%
\definecolor{currentstroke}{rgb}{0.000000,0.000000,0.000000}%
\pgfsetstrokecolor{currentstroke}%
\pgfsetdash{}{0pt}%
\pgfpathmoveto{\pgfqpoint{0.692068in}{4.081400in}}%
\pgfpathlineto{\pgfqpoint{5.652068in}{4.081400in}}%
\pgfusepath{stroke}%
\end{pgfscope}%
\end{pgfpicture}%
\makeatother%
\endgroup%
}
    \resizebox{0.49\textwidth}{6cm}{%% Creator: Matplotlib, PGF backend
%%
%% To include the figure in your LaTeX document, write
%%   \input{<filename>.pgf}
%%
%% Make sure the required packages are loaded in your preamble
%%   \usepackage{pgf}
%%
%% Figures using additional raster images can only be included by \input if
%% they are in the same directory as the main LaTeX file. For loading figures
%% from other directories you can use the `import` package
%%   \usepackage{import}
%%
%% and then include the figures with
%%   \import{<path to file>}{<filename>.pgf}
%%
%% Matplotlib used the following preamble
%%   \usepackage{fontspec}
%%   \setmainfont{DejaVuSerif.ttf}[Path=\detokenize{/Users/apple/opt/anaconda3/envs/kaolin/lib/python3.7/site-packages/matplotlib/mpl-data/fonts/ttf/}]
%%   \setsansfont{DejaVuSans.ttf}[Path=\detokenize{/Users/apple/opt/anaconda3/envs/kaolin/lib/python3.7/site-packages/matplotlib/mpl-data/fonts/ttf/}]
%%   \setmonofont{DejaVuSansMono.ttf}[Path=\detokenize{/Users/apple/opt/anaconda3/envs/kaolin/lib/python3.7/site-packages/matplotlib/mpl-data/fonts/ttf/}]
%%
\begingroup%
\makeatletter%
\begin{pgfpicture}%
\pgfpathrectangle{\pgfpointorigin}{\pgfqpoint{5.628357in}{4.181400in}}%
\pgfusepath{use as bounding box, clip}%
\begin{pgfscope}%
\pgfsetbuttcap%
\pgfsetmiterjoin%
\definecolor{currentfill}{rgb}{1.000000,1.000000,1.000000}%
\pgfsetfillcolor{currentfill}%
\pgfsetlinewidth{0.000000pt}%
\definecolor{currentstroke}{rgb}{1.000000,1.000000,1.000000}%
\pgfsetstrokecolor{currentstroke}%
\pgfsetdash{}{0pt}%
\pgfpathmoveto{\pgfqpoint{-0.000000in}{0.000000in}}%
\pgfpathlineto{\pgfqpoint{5.628357in}{0.000000in}}%
\pgfpathlineto{\pgfqpoint{5.628357in}{4.181400in}}%
\pgfpathlineto{\pgfqpoint{-0.000000in}{4.181400in}}%
\pgfpathclose%
\pgfusepath{fill}%
\end{pgfscope}%
\begin{pgfscope}%
\pgfsetbuttcap%
\pgfsetmiterjoin%
\definecolor{currentfill}{rgb}{1.000000,1.000000,1.000000}%
\pgfsetfillcolor{currentfill}%
\pgfsetlinewidth{0.000000pt}%
\definecolor{currentstroke}{rgb}{0.000000,0.000000,0.000000}%
\pgfsetstrokecolor{currentstroke}%
\pgfsetstrokeopacity{0.000000}%
\pgfsetdash{}{0pt}%
\pgfpathmoveto{\pgfqpoint{0.568357in}{0.385400in}}%
\pgfpathlineto{\pgfqpoint{5.528357in}{0.385400in}}%
\pgfpathlineto{\pgfqpoint{5.528357in}{4.081400in}}%
\pgfpathlineto{\pgfqpoint{0.568357in}{4.081400in}}%
\pgfpathclose%
\pgfusepath{fill}%
\end{pgfscope}%
\begin{pgfscope}%
\pgfpathrectangle{\pgfqpoint{0.568357in}{0.385400in}}{\pgfqpoint{4.960000in}{3.696000in}}%
\pgfusepath{clip}%
\pgfsetbuttcap%
\pgfsetmiterjoin%
\definecolor{currentfill}{rgb}{0.121569,0.466667,0.705882}%
\pgfsetfillcolor{currentfill}%
\pgfsetlinewidth{0.000000pt}%
\definecolor{currentstroke}{rgb}{0.000000,0.000000,0.000000}%
\pgfsetstrokecolor{currentstroke}%
\pgfsetstrokeopacity{0.000000}%
\pgfsetdash{}{0pt}%
\pgfpathmoveto{\pgfqpoint{0.793811in}{0.385400in}}%
\pgfpathlineto{\pgfqpoint{1.324292in}{0.385400in}}%
\pgfpathlineto{\pgfqpoint{1.324292in}{0.559815in}}%
\pgfpathlineto{\pgfqpoint{0.793811in}{0.559815in}}%
\pgfpathclose%
\pgfusepath{fill}%
\end{pgfscope}%
\begin{pgfscope}%
\pgfpathrectangle{\pgfqpoint{0.568357in}{0.385400in}}{\pgfqpoint{4.960000in}{3.696000in}}%
\pgfusepath{clip}%
\pgfsetbuttcap%
\pgfsetmiterjoin%
\definecolor{currentfill}{rgb}{0.121569,0.466667,0.705882}%
\pgfsetfillcolor{currentfill}%
\pgfsetlinewidth{0.000000pt}%
\definecolor{currentstroke}{rgb}{0.000000,0.000000,0.000000}%
\pgfsetstrokecolor{currentstroke}%
\pgfsetstrokeopacity{0.000000}%
\pgfsetdash{}{0pt}%
\pgfpathmoveto{\pgfqpoint{1.456913in}{0.385400in}}%
\pgfpathlineto{\pgfqpoint{1.987394in}{0.385400in}}%
\pgfpathlineto{\pgfqpoint{1.987394in}{0.718373in}}%
\pgfpathlineto{\pgfqpoint{1.456913in}{0.718373in}}%
\pgfpathclose%
\pgfusepath{fill}%
\end{pgfscope}%
\begin{pgfscope}%
\pgfpathrectangle{\pgfqpoint{0.568357in}{0.385400in}}{\pgfqpoint{4.960000in}{3.696000in}}%
\pgfusepath{clip}%
\pgfsetbuttcap%
\pgfsetmiterjoin%
\definecolor{currentfill}{rgb}{0.121569,0.466667,0.705882}%
\pgfsetfillcolor{currentfill}%
\pgfsetlinewidth{0.000000pt}%
\definecolor{currentstroke}{rgb}{0.000000,0.000000,0.000000}%
\pgfsetstrokecolor{currentstroke}%
\pgfsetstrokeopacity{0.000000}%
\pgfsetdash{}{0pt}%
\pgfpathmoveto{\pgfqpoint{2.120014in}{0.385400in}}%
\pgfpathlineto{\pgfqpoint{2.650496in}{0.385400in}}%
\pgfpathlineto{\pgfqpoint{2.650496in}{0.765941in}}%
\pgfpathlineto{\pgfqpoint{2.120014in}{0.765941in}}%
\pgfpathclose%
\pgfusepath{fill}%
\end{pgfscope}%
\begin{pgfscope}%
\pgfpathrectangle{\pgfqpoint{0.568357in}{0.385400in}}{\pgfqpoint{4.960000in}{3.696000in}}%
\pgfusepath{clip}%
\pgfsetbuttcap%
\pgfsetmiterjoin%
\definecolor{currentfill}{rgb}{0.121569,0.466667,0.705882}%
\pgfsetfillcolor{currentfill}%
\pgfsetlinewidth{0.000000pt}%
\definecolor{currentstroke}{rgb}{0.000000,0.000000,0.000000}%
\pgfsetstrokecolor{currentstroke}%
\pgfsetstrokeopacity{0.000000}%
\pgfsetdash{}{0pt}%
\pgfpathmoveto{\pgfqpoint{2.783116in}{0.385400in}}%
\pgfpathlineto{\pgfqpoint{3.313597in}{0.385400in}}%
\pgfpathlineto{\pgfqpoint{3.313597in}{3.905400in}}%
\pgfpathlineto{\pgfqpoint{2.783116in}{3.905400in}}%
\pgfpathclose%
\pgfusepath{fill}%
\end{pgfscope}%
\begin{pgfscope}%
\pgfpathrectangle{\pgfqpoint{0.568357in}{0.385400in}}{\pgfqpoint{4.960000in}{3.696000in}}%
\pgfusepath{clip}%
\pgfsetbuttcap%
\pgfsetmiterjoin%
\definecolor{currentfill}{rgb}{0.121569,0.466667,0.705882}%
\pgfsetfillcolor{currentfill}%
\pgfsetlinewidth{0.000000pt}%
\definecolor{currentstroke}{rgb}{0.000000,0.000000,0.000000}%
\pgfsetstrokecolor{currentstroke}%
\pgfsetstrokeopacity{0.000000}%
\pgfsetdash{}{0pt}%
\pgfpathmoveto{\pgfqpoint{3.446218in}{0.385400in}}%
\pgfpathlineto{\pgfqpoint{3.976699in}{0.385400in}}%
\pgfpathlineto{\pgfqpoint{3.976699in}{0.670806in}}%
\pgfpathlineto{\pgfqpoint{3.446218in}{0.670806in}}%
\pgfpathclose%
\pgfusepath{fill}%
\end{pgfscope}%
\begin{pgfscope}%
\pgfpathrectangle{\pgfqpoint{0.568357in}{0.385400in}}{\pgfqpoint{4.960000in}{3.696000in}}%
\pgfusepath{clip}%
\pgfsetbuttcap%
\pgfsetmiterjoin%
\definecolor{currentfill}{rgb}{0.121569,0.466667,0.705882}%
\pgfsetfillcolor{currentfill}%
\pgfsetlinewidth{0.000000pt}%
\definecolor{currentstroke}{rgb}{0.000000,0.000000,0.000000}%
\pgfsetstrokecolor{currentstroke}%
\pgfsetstrokeopacity{0.000000}%
\pgfsetdash{}{0pt}%
\pgfpathmoveto{\pgfqpoint{4.109319in}{0.385400in}}%
\pgfpathlineto{\pgfqpoint{4.639801in}{0.385400in}}%
\pgfpathlineto{\pgfqpoint{4.639801in}{0.718373in}}%
\pgfpathlineto{\pgfqpoint{4.109319in}{0.718373in}}%
\pgfpathclose%
\pgfusepath{fill}%
\end{pgfscope}%
\begin{pgfscope}%
\pgfpathrectangle{\pgfqpoint{0.568357in}{0.385400in}}{\pgfqpoint{4.960000in}{3.696000in}}%
\pgfusepath{clip}%
\pgfsetbuttcap%
\pgfsetmiterjoin%
\definecolor{currentfill}{rgb}{0.121569,0.466667,0.705882}%
\pgfsetfillcolor{currentfill}%
\pgfsetlinewidth{0.000000pt}%
\definecolor{currentstroke}{rgb}{0.000000,0.000000,0.000000}%
\pgfsetstrokecolor{currentstroke}%
\pgfsetstrokeopacity{0.000000}%
\pgfsetdash{}{0pt}%
\pgfpathmoveto{\pgfqpoint{4.772421in}{0.385400in}}%
\pgfpathlineto{\pgfqpoint{5.302902in}{0.385400in}}%
\pgfpathlineto{\pgfqpoint{5.302902in}{1.400175in}}%
\pgfpathlineto{\pgfqpoint{4.772421in}{1.400175in}}%
\pgfpathclose%
\pgfusepath{fill}%
\end{pgfscope}%
\begin{pgfscope}%
\pgfsetbuttcap%
\pgfsetroundjoin%
\definecolor{currentfill}{rgb}{0.000000,0.000000,0.000000}%
\pgfsetfillcolor{currentfill}%
\pgfsetlinewidth{0.803000pt}%
\definecolor{currentstroke}{rgb}{0.000000,0.000000,0.000000}%
\pgfsetstrokecolor{currentstroke}%
\pgfsetdash{}{0pt}%
\pgfsys@defobject{currentmarker}{\pgfqpoint{0.000000in}{-0.048611in}}{\pgfqpoint{0.000000in}{0.000000in}}{%
\pgfpathmoveto{\pgfqpoint{0.000000in}{0.000000in}}%
\pgfpathlineto{\pgfqpoint{0.000000in}{-0.048611in}}%
\pgfusepath{stroke,fill}%
}%
\begin{pgfscope}%
\pgfsys@transformshift{1.059052in}{0.385400in}%
\pgfsys@useobject{currentmarker}{}%
\end{pgfscope}%
\end{pgfscope}%
\begin{pgfscope}%
\definecolor{textcolor}{rgb}{0.000000,0.000000,0.000000}%
\pgfsetstrokecolor{textcolor}%
\pgfsetfillcolor{textcolor}%
\pgftext[x=1.059052in,y=0.288178in,,top]{\color{textcolor}\sffamily\fontsize{14.000000}{16.800000}\selectfont wardrobe}%
\end{pgfscope}%
\begin{pgfscope}%
\pgfsetbuttcap%
\pgfsetroundjoin%
\definecolor{currentfill}{rgb}{0.000000,0.000000,0.000000}%
\pgfsetfillcolor{currentfill}%
\pgfsetlinewidth{0.803000pt}%
\definecolor{currentstroke}{rgb}{0.000000,0.000000,0.000000}%
\pgfsetstrokecolor{currentstroke}%
\pgfsetdash{}{0pt}%
\pgfsys@defobject{currentmarker}{\pgfqpoint{0.000000in}{-0.048611in}}{\pgfqpoint{0.000000in}{0.000000in}}{%
\pgfpathmoveto{\pgfqpoint{0.000000in}{0.000000in}}%
\pgfpathlineto{\pgfqpoint{0.000000in}{-0.048611in}}%
\pgfusepath{stroke,fill}%
}%
\begin{pgfscope}%
\pgfsys@transformshift{1.722153in}{0.385400in}%
\pgfsys@useobject{currentmarker}{}%
\end{pgfscope}%
\end{pgfscope}%
\begin{pgfscope}%
\definecolor{textcolor}{rgb}{0.000000,0.000000,0.000000}%
\pgfsetstrokecolor{textcolor}%
\pgfsetfillcolor{textcolor}%
\pgftext[x=1.722153in,y=0.288178in,,top]{\color{textcolor}\sffamily\fontsize{14.000000}{16.800000}\selectfont bed}%
\end{pgfscope}%
\begin{pgfscope}%
\pgfsetbuttcap%
\pgfsetroundjoin%
\definecolor{currentfill}{rgb}{0.000000,0.000000,0.000000}%
\pgfsetfillcolor{currentfill}%
\pgfsetlinewidth{0.803000pt}%
\definecolor{currentstroke}{rgb}{0.000000,0.000000,0.000000}%
\pgfsetstrokecolor{currentstroke}%
\pgfsetdash{}{0pt}%
\pgfsys@defobject{currentmarker}{\pgfqpoint{0.000000in}{-0.048611in}}{\pgfqpoint{0.000000in}{0.000000in}}{%
\pgfpathmoveto{\pgfqpoint{0.000000in}{0.000000in}}%
\pgfpathlineto{\pgfqpoint{0.000000in}{-0.048611in}}%
\pgfusepath{stroke,fill}%
}%
\begin{pgfscope}%
\pgfsys@transformshift{2.385255in}{0.385400in}%
\pgfsys@useobject{currentmarker}{}%
\end{pgfscope}%
\end{pgfscope}%
\begin{pgfscope}%
\definecolor{textcolor}{rgb}{0.000000,0.000000,0.000000}%
\pgfsetstrokecolor{textcolor}%
\pgfsetfillcolor{textcolor}%
\pgftext[x=2.385255in,y=0.288178in,,top]{\color{textcolor}\sffamily\fontsize{14.000000}{16.800000}\selectfont desk}%
\end{pgfscope}%
\begin{pgfscope}%
\pgfsetbuttcap%
\pgfsetroundjoin%
\definecolor{currentfill}{rgb}{0.000000,0.000000,0.000000}%
\pgfsetfillcolor{currentfill}%
\pgfsetlinewidth{0.803000pt}%
\definecolor{currentstroke}{rgb}{0.000000,0.000000,0.000000}%
\pgfsetstrokecolor{currentstroke}%
\pgfsetdash{}{0pt}%
\pgfsys@defobject{currentmarker}{\pgfqpoint{0.000000in}{-0.048611in}}{\pgfqpoint{0.000000in}{0.000000in}}{%
\pgfpathmoveto{\pgfqpoint{0.000000in}{0.000000in}}%
\pgfpathlineto{\pgfqpoint{0.000000in}{-0.048611in}}%
\pgfusepath{stroke,fill}%
}%
\begin{pgfscope}%
\pgfsys@transformshift{3.048357in}{0.385400in}%
\pgfsys@useobject{currentmarker}{}%
\end{pgfscope}%
\end{pgfscope}%
\begin{pgfscope}%
\definecolor{textcolor}{rgb}{0.000000,0.000000,0.000000}%
\pgfsetstrokecolor{textcolor}%
\pgfsetfillcolor{textcolor}%
\pgftext[x=3.048357in,y=0.288178in,,top]{\color{textcolor}\sffamily\fontsize{14.000000}{16.800000}\selectfont chair}%
\end{pgfscope}%
\begin{pgfscope}%
\pgfsetbuttcap%
\pgfsetroundjoin%
\definecolor{currentfill}{rgb}{0.000000,0.000000,0.000000}%
\pgfsetfillcolor{currentfill}%
\pgfsetlinewidth{0.803000pt}%
\definecolor{currentstroke}{rgb}{0.000000,0.000000,0.000000}%
\pgfsetstrokecolor{currentstroke}%
\pgfsetdash{}{0pt}%
\pgfsys@defobject{currentmarker}{\pgfqpoint{0.000000in}{-0.048611in}}{\pgfqpoint{0.000000in}{0.000000in}}{%
\pgfpathmoveto{\pgfqpoint{0.000000in}{0.000000in}}%
\pgfpathlineto{\pgfqpoint{0.000000in}{-0.048611in}}%
\pgfusepath{stroke,fill}%
}%
\begin{pgfscope}%
\pgfsys@transformshift{3.711458in}{0.385400in}%
\pgfsys@useobject{currentmarker}{}%
\end{pgfscope}%
\end{pgfscope}%
\begin{pgfscope}%
\definecolor{textcolor}{rgb}{0.000000,0.000000,0.000000}%
\pgfsetstrokecolor{textcolor}%
\pgfsetfillcolor{textcolor}%
\pgftext[x=3.711458in,y=0.288178in,,top]{\color{textcolor}\sffamily\fontsize{14.000000}{16.800000}\selectfont bookcase}%
\end{pgfscope}%
\begin{pgfscope}%
\pgfsetbuttcap%
\pgfsetroundjoin%
\definecolor{currentfill}{rgb}{0.000000,0.000000,0.000000}%
\pgfsetfillcolor{currentfill}%
\pgfsetlinewidth{0.803000pt}%
\definecolor{currentstroke}{rgb}{0.000000,0.000000,0.000000}%
\pgfsetstrokecolor{currentstroke}%
\pgfsetdash{}{0pt}%
\pgfsys@defobject{currentmarker}{\pgfqpoint{0.000000in}{-0.048611in}}{\pgfqpoint{0.000000in}{0.000000in}}{%
\pgfpathmoveto{\pgfqpoint{0.000000in}{0.000000in}}%
\pgfpathlineto{\pgfqpoint{0.000000in}{-0.048611in}}%
\pgfusepath{stroke,fill}%
}%
\begin{pgfscope}%
\pgfsys@transformshift{4.374560in}{0.385400in}%
\pgfsys@useobject{currentmarker}{}%
\end{pgfscope}%
\end{pgfscope}%
\begin{pgfscope}%
\definecolor{textcolor}{rgb}{0.000000,0.000000,0.000000}%
\pgfsetstrokecolor{textcolor}%
\pgfsetfillcolor{textcolor}%
\pgftext[x=4.374560in,y=0.288178in,,top]{\color{textcolor}\sffamily\fontsize{14.000000}{16.800000}\selectfont sofa}%
\end{pgfscope}%
\begin{pgfscope}%
\pgfsetbuttcap%
\pgfsetroundjoin%
\definecolor{currentfill}{rgb}{0.000000,0.000000,0.000000}%
\pgfsetfillcolor{currentfill}%
\pgfsetlinewidth{0.803000pt}%
\definecolor{currentstroke}{rgb}{0.000000,0.000000,0.000000}%
\pgfsetstrokecolor{currentstroke}%
\pgfsetdash{}{0pt}%
\pgfsys@defobject{currentmarker}{\pgfqpoint{0.000000in}{-0.048611in}}{\pgfqpoint{0.000000in}{0.000000in}}{%
\pgfpathmoveto{\pgfqpoint{0.000000in}{0.000000in}}%
\pgfpathlineto{\pgfqpoint{0.000000in}{-0.048611in}}%
\pgfusepath{stroke,fill}%
}%
\begin{pgfscope}%
\pgfsys@transformshift{5.037661in}{0.385400in}%
\pgfsys@useobject{currentmarker}{}%
\end{pgfscope}%
\end{pgfscope}%
\begin{pgfscope}%
\definecolor{textcolor}{rgb}{0.000000,0.000000,0.000000}%
\pgfsetstrokecolor{textcolor}%
\pgfsetfillcolor{textcolor}%
\pgftext[x=5.037661in,y=0.288178in,,top]{\color{textcolor}\sffamily\fontsize{14.000000}{16.800000}\selectfont table}%
\end{pgfscope}%
\begin{pgfscope}%
\pgfsetbuttcap%
\pgfsetroundjoin%
\definecolor{currentfill}{rgb}{0.000000,0.000000,0.000000}%
\pgfsetfillcolor{currentfill}%
\pgfsetlinewidth{0.803000pt}%
\definecolor{currentstroke}{rgb}{0.000000,0.000000,0.000000}%
\pgfsetstrokecolor{currentstroke}%
\pgfsetdash{}{0pt}%
\pgfsys@defobject{currentmarker}{\pgfqpoint{-0.048611in}{0.000000in}}{\pgfqpoint{-0.000000in}{0.000000in}}{%
\pgfpathmoveto{\pgfqpoint{-0.000000in}{0.000000in}}%
\pgfpathlineto{\pgfqpoint{-0.048611in}{0.000000in}}%
\pgfusepath{stroke,fill}%
}%
\begin{pgfscope}%
\pgfsys@transformshift{0.568357in}{0.385400in}%
\pgfsys@useobject{currentmarker}{}%
\end{pgfscope}%
\end{pgfscope}%
\begin{pgfscope}%
\definecolor{textcolor}{rgb}{0.000000,0.000000,0.000000}%
\pgfsetstrokecolor{textcolor}%
\pgfsetfillcolor{textcolor}%
\pgftext[x=0.347423in, y=0.311534in, left, base]{\color{textcolor}\sffamily\fontsize{14.000000}{16.800000}\selectfont 0}%
\end{pgfscope}%
\begin{pgfscope}%
\pgfsetbuttcap%
\pgfsetroundjoin%
\definecolor{currentfill}{rgb}{0.000000,0.000000,0.000000}%
\pgfsetfillcolor{currentfill}%
\pgfsetlinewidth{0.803000pt}%
\definecolor{currentstroke}{rgb}{0.000000,0.000000,0.000000}%
\pgfsetstrokecolor{currentstroke}%
\pgfsetdash{}{0pt}%
\pgfsys@defobject{currentmarker}{\pgfqpoint{-0.048611in}{0.000000in}}{\pgfqpoint{-0.000000in}{0.000000in}}{%
\pgfpathmoveto{\pgfqpoint{-0.000000in}{0.000000in}}%
\pgfpathlineto{\pgfqpoint{-0.048611in}{0.000000in}}%
\pgfusepath{stroke,fill}%
}%
\begin{pgfscope}%
\pgfsys@transformshift{0.568357in}{1.178193in}%
\pgfsys@useobject{currentmarker}{}%
\end{pgfscope}%
\end{pgfscope}%
\begin{pgfscope}%
\definecolor{textcolor}{rgb}{0.000000,0.000000,0.000000}%
\pgfsetstrokecolor{textcolor}%
\pgfsetfillcolor{textcolor}%
\pgftext[x=0.223712in, y=1.104327in, left, base]{\color{textcolor}\sffamily\fontsize{14.000000}{16.800000}\selectfont 50}%
\end{pgfscope}%
\begin{pgfscope}%
\pgfsetbuttcap%
\pgfsetroundjoin%
\definecolor{currentfill}{rgb}{0.000000,0.000000,0.000000}%
\pgfsetfillcolor{currentfill}%
\pgfsetlinewidth{0.803000pt}%
\definecolor{currentstroke}{rgb}{0.000000,0.000000,0.000000}%
\pgfsetstrokecolor{currentstroke}%
\pgfsetdash{}{0pt}%
\pgfsys@defobject{currentmarker}{\pgfqpoint{-0.048611in}{0.000000in}}{\pgfqpoint{-0.000000in}{0.000000in}}{%
\pgfpathmoveto{\pgfqpoint{-0.000000in}{0.000000in}}%
\pgfpathlineto{\pgfqpoint{-0.048611in}{0.000000in}}%
\pgfusepath{stroke,fill}%
}%
\begin{pgfscope}%
\pgfsys@transformshift{0.568357in}{1.970986in}%
\pgfsys@useobject{currentmarker}{}%
\end{pgfscope}%
\end{pgfscope}%
\begin{pgfscope}%
\definecolor{textcolor}{rgb}{0.000000,0.000000,0.000000}%
\pgfsetstrokecolor{textcolor}%
\pgfsetfillcolor{textcolor}%
\pgftext[x=0.100000in, y=1.897120in, left, base]{\color{textcolor}\sffamily\fontsize{14.000000}{16.800000}\selectfont 100}%
\end{pgfscope}%
\begin{pgfscope}%
\pgfsetbuttcap%
\pgfsetroundjoin%
\definecolor{currentfill}{rgb}{0.000000,0.000000,0.000000}%
\pgfsetfillcolor{currentfill}%
\pgfsetlinewidth{0.803000pt}%
\definecolor{currentstroke}{rgb}{0.000000,0.000000,0.000000}%
\pgfsetstrokecolor{currentstroke}%
\pgfsetdash{}{0pt}%
\pgfsys@defobject{currentmarker}{\pgfqpoint{-0.048611in}{0.000000in}}{\pgfqpoint{-0.000000in}{0.000000in}}{%
\pgfpathmoveto{\pgfqpoint{-0.000000in}{0.000000in}}%
\pgfpathlineto{\pgfqpoint{-0.048611in}{0.000000in}}%
\pgfusepath{stroke,fill}%
}%
\begin{pgfscope}%
\pgfsys@transformshift{0.568357in}{2.763779in}%
\pgfsys@useobject{currentmarker}{}%
\end{pgfscope}%
\end{pgfscope}%
\begin{pgfscope}%
\definecolor{textcolor}{rgb}{0.000000,0.000000,0.000000}%
\pgfsetstrokecolor{textcolor}%
\pgfsetfillcolor{textcolor}%
\pgftext[x=0.100000in, y=2.689913in, left, base]{\color{textcolor}\sffamily\fontsize{14.000000}{16.800000}\selectfont 150}%
\end{pgfscope}%
\begin{pgfscope}%
\pgfsetbuttcap%
\pgfsetroundjoin%
\definecolor{currentfill}{rgb}{0.000000,0.000000,0.000000}%
\pgfsetfillcolor{currentfill}%
\pgfsetlinewidth{0.803000pt}%
\definecolor{currentstroke}{rgb}{0.000000,0.000000,0.000000}%
\pgfsetstrokecolor{currentstroke}%
\pgfsetdash{}{0pt}%
\pgfsys@defobject{currentmarker}{\pgfqpoint{-0.048611in}{0.000000in}}{\pgfqpoint{-0.000000in}{0.000000in}}{%
\pgfpathmoveto{\pgfqpoint{-0.000000in}{0.000000in}}%
\pgfpathlineto{\pgfqpoint{-0.048611in}{0.000000in}}%
\pgfusepath{stroke,fill}%
}%
\begin{pgfscope}%
\pgfsys@transformshift{0.568357in}{3.556571in}%
\pgfsys@useobject{currentmarker}{}%
\end{pgfscope}%
\end{pgfscope}%
\begin{pgfscope}%
\definecolor{textcolor}{rgb}{0.000000,0.000000,0.000000}%
\pgfsetstrokecolor{textcolor}%
\pgfsetfillcolor{textcolor}%
\pgftext[x=0.100000in, y=3.482705in, left, base]{\color{textcolor}\sffamily\fontsize{14.000000}{16.800000}\selectfont 200}%
\end{pgfscope}%
\begin{pgfscope}%
\pgfsetrectcap%
\pgfsetmiterjoin%
\pgfsetlinewidth{0.803000pt}%
\definecolor{currentstroke}{rgb}{0.000000,0.000000,0.000000}%
\pgfsetstrokecolor{currentstroke}%
\pgfsetdash{}{0pt}%
\pgfpathmoveto{\pgfqpoint{0.568357in}{0.385400in}}%
\pgfpathlineto{\pgfqpoint{0.568357in}{4.081400in}}%
\pgfusepath{stroke}%
\end{pgfscope}%
\begin{pgfscope}%
\pgfsetrectcap%
\pgfsetmiterjoin%
\pgfsetlinewidth{0.803000pt}%
\definecolor{currentstroke}{rgb}{0.000000,0.000000,0.000000}%
\pgfsetstrokecolor{currentstroke}%
\pgfsetdash{}{0pt}%
\pgfpathmoveto{\pgfqpoint{5.528357in}{0.385400in}}%
\pgfpathlineto{\pgfqpoint{5.528357in}{4.081400in}}%
\pgfusepath{stroke}%
\end{pgfscope}%
\begin{pgfscope}%
\pgfsetrectcap%
\pgfsetmiterjoin%
\pgfsetlinewidth{0.803000pt}%
\definecolor{currentstroke}{rgb}{0.000000,0.000000,0.000000}%
\pgfsetstrokecolor{currentstroke}%
\pgfsetdash{}{0pt}%
\pgfpathmoveto{\pgfqpoint{0.568357in}{0.385400in}}%
\pgfpathlineto{\pgfqpoint{5.528357in}{0.385400in}}%
\pgfusepath{stroke}%
\end{pgfscope}%
\begin{pgfscope}%
\pgfsetrectcap%
\pgfsetmiterjoin%
\pgfsetlinewidth{0.803000pt}%
\definecolor{currentstroke}{rgb}{0.000000,0.000000,0.000000}%
\pgfsetstrokecolor{currentstroke}%
\pgfsetdash{}{0pt}%
\pgfpathmoveto{\pgfqpoint{0.568357in}{4.081400in}}%
\pgfpathlineto{\pgfqpoint{5.528357in}{4.081400in}}%
\pgfusepath{stroke}%
\end{pgfscope}%
\end{pgfpicture}%
\makeatother%
\endgroup%
}
    \caption[Distribution of Pix3D.]{Distribution of Pix3D~\cite{Sun2018} images(left), unique models(right)}
    \label{fig:pix3d_histogram}
\end{figure}

\subsection{Why Pix3D?}\label{subsec:why-pix3d?}
We discuss the drawbacks of Pix3D in \autoref{subsec:disadvantages-of-pix3d}.
Apart from those reasons, Pix3D is a collection of indoor scenes with a complex background, varying light conditions with shadows, reflective surfaces, and even varying occlusion levels.
Each image comprises a collection of objects in the scene, but only one object from the category is annotated.
It is a perfect example of having limited real-world data.
Since 3D models are available for each piece of furniture, synthetic data can be generated in abundance from those models.
It would be interesting to see if a Game Engine can replicate such complex features of the real world or create randomization to an extent where such complex images appear to be a part of randomization.
A study on Pix3D and synthetic datasetsThe placement is done such that the contex should demonstrate the usefulness of Game Engines.
For our synthetic to real dataset experiment, we chose to select only furniture classes from Pix3D, leaving out ``misc and tools" classes, which were significantly less in comparison.

\subsection{Role of SceneNet}\label{sec:role-of-scenenet}
Now that we decided on the 3D furniture models, we need an environment with complex background.
SceneNet~\cite{McCormac2017} is an extensive collection of photorealistic indoor scene trajectories.
In our approach, we utilize the scene provided by SceneNet as a layout for our indoor scenes.
Each scene includes initial Shapenet models with furniture placement.
The placement is done such that the context between the the pieces of furniture is maintained.
For example, chairs are placed near the table,
Sofa and coffee table are facing towards a television, etc.
Using these position, we will be able to derive sensible images for our synthetic dataset.

\subsection{\gls{free}, a Pix3D Based Synthetic Dataset}\label{sec:s2r:3d-free-a-pix3d-based-synthetic-dataset}

\gls{free} dataset, which stands for Synth2Real: 3-Dimensional Furniture Reconstruction from Ersatz Environment, combines SceneNet and Pix3D dataset.
We utilize the availability of 3D models of rooms and pieces of furniture from these two datasets to create an ersatz environment using Unity as our framework.
We randomize the indoor scenes from SceneNet~\cite{McCormac2017} and textures provided by them with some additional complex textures.
The ShapeNet and Pix3D datasets have common categories.
We replace 3D furniture in the scene with a model from Pix3D of the same type.
This process is randomized and will be discussed in \autoref{sec:3d-scene-framework}.
The other option would have been to use the ShapeNet or \gls{front} as the target model, but we would not have a real dataset to compare to this case.
Ideally, a model trained on ShapeNet should also perform well with a real dataset like Pix3D, but we decided to have a synthetic dataset based on Pix3D itself for better comparison.

\section{Unity-Based Pipeline}\label{sec:unity-based-pipeline}
To create the synthetic dataset, we use Unity Game Engine.
The reasons for selecting Unity as our platform for the application are:
\begin{enumerate}
    \item Cross-platform game engine and hence usable on any Operating system.
    \item The basic version is available for free.
    \item Provides all the necessary tools to create an ersatz environment with well-maintained documentation.
    \item An active developer community.
\end{enumerate}

There is no official comparison between Unreal Engine and Unity engine to have a deciding factor.
The selection of the Unity engine was a purely individual choice and ease of use.
However, both these game engines can create realistic-looking scenes that are usable for synthetic dataset generation.
With the Unity game engine, the available scenes from the Scenenet and 3D models from Pix3D can be imported to form an ersatz environment.
Further domain randomization can be applied to create a dataset of photorealistic images and 2.5D data like normals, depth maps, masks.


\section{3D-Scene Framework}\label{sec:3d-scene-framework}
3D-Scene is a Unity-based application created as a proof of concept for generating photorealistic images using a Game engine.
The core pipeline and the workflow is as shown in \autoref{fig:pipeline process}.
For an automation-based pipeline, the process loops through each step, while in the manual pipeline, the user decides which block to execute.
In this section, each of the blocks is explained briefly.
The implementation of the blocks will be discussed in \autoref{ch:implementation}.

\begin{figure}[!ht]
    \centering
    \includegraphics[width=\textwidth]{/Users/apple/OVGU/Thesis/code/3dReconstruction/report/images/concept/process}
    \caption[Overview of 3D-Scene Tool]{An overview of pipeline for image generation with different blocks and external libraries.
    Each block is discussed in sections mentioned in the figure.}
    \label{fig:pipeline process}
\end{figure}

\subsection{Furniture Models from Pix3D}\label{subsec:furniture-models-from-pix3d}
As discussed in \autoref{subsec:why-pix3d?}, we utilize 3D furniture models from Pix3D.
Pix3D provides 381 models for pieces of furniture with seven categories(Bed, Bookcase, Chair, Desk, Sofa, Table, Wardrobe).
Most of the models have default textures, while some do not.
These textures will be changed randomly within the pipeline.

\subsection{Scenes from SceneNet}\label{subsec:scenes-from-scenenet}
We utilize scenes provided by SceneNet~\cite{McCormac2017} for the rooms in our ersatz environments.
All the categories from Pix3D are also present in ShapeNet, which are used to fill the space in SceneNet.
The distribution of furniture matching the category of Pix3D in the scenes is as shown in \autoref{fig:distribution of scenes}.
The application modifies each scene for every snap we take for the dataset we create at random.

\begin{figure}[!ht]
    \resizebox{0.49\textwidth}{6cm}{%% Creator: Matplotlib, PGF backend
%%
%% To include the figure in your LaTeX document, write
%%   \input{<filename>.pgf}
%%
%% Make sure the required packages are loaded in your preamble
%%   \usepackage{pgf}
%%
%% Figures using additional raster images can only be included by \input if
%% they are in the same directory as the main LaTeX file. For loading figures
%% from other directories you can use the `import` package
%%   \usepackage{import}
%%
%% and then include the figures with
%%   \import{<path to file>}{<filename>.pgf}
%%
%% Matplotlib used the following preamble
%%   \usepackage{fontspec}
%%   \setmainfont{DejaVuSerif.ttf}[Path=\detokenize{/Users/apple/opt/anaconda3/envs/kaolin/lib/python3.7/site-packages/matplotlib/mpl-data/fonts/ttf/}]
%%   \setsansfont{DejaVuSans.ttf}[Path=\detokenize{/Users/apple/opt/anaconda3/envs/kaolin/lib/python3.7/site-packages/matplotlib/mpl-data/fonts/ttf/}]
%%   \setmonofont{DejaVuSansMono.ttf}[Path=\detokenize{/Users/apple/opt/anaconda3/envs/kaolin/lib/python3.7/site-packages/matplotlib/mpl-data/fonts/ttf/}]
%%
\begingroup%
\makeatletter%
\begin{pgfpicture}%
\pgfpathrectangle{\pgfpointorigin}{\pgfqpoint{5.504645in}{4.391361in}}%
\pgfusepath{use as bounding box, clip}%
\begin{pgfscope}%
\pgfsetbuttcap%
\pgfsetmiterjoin%
\definecolor{currentfill}{rgb}{1.000000,1.000000,1.000000}%
\pgfsetfillcolor{currentfill}%
\pgfsetlinewidth{0.000000pt}%
\definecolor{currentstroke}{rgb}{1.000000,1.000000,1.000000}%
\pgfsetstrokecolor{currentstroke}%
\pgfsetdash{}{0pt}%
\pgfpathmoveto{\pgfqpoint{0.000000in}{0.000000in}}%
\pgfpathlineto{\pgfqpoint{5.504645in}{0.000000in}}%
\pgfpathlineto{\pgfqpoint{5.504645in}{4.391361in}}%
\pgfpathlineto{\pgfqpoint{0.000000in}{4.391361in}}%
\pgfpathclose%
\pgfusepath{fill}%
\end{pgfscope}%
\begin{pgfscope}%
\pgfsetbuttcap%
\pgfsetmiterjoin%
\definecolor{currentfill}{rgb}{1.000000,1.000000,1.000000}%
\pgfsetfillcolor{currentfill}%
\pgfsetlinewidth{0.000000pt}%
\definecolor{currentstroke}{rgb}{0.000000,0.000000,0.000000}%
\pgfsetstrokecolor{currentstroke}%
\pgfsetstrokeopacity{0.000000}%
\pgfsetdash{}{0pt}%
\pgfpathmoveto{\pgfqpoint{0.444645in}{0.385400in}}%
\pgfpathlineto{\pgfqpoint{5.404645in}{0.385400in}}%
\pgfpathlineto{\pgfqpoint{5.404645in}{4.081400in}}%
\pgfpathlineto{\pgfqpoint{0.444645in}{4.081400in}}%
\pgfpathclose%
\pgfusepath{fill}%
\end{pgfscope}%
\begin{pgfscope}%
\pgfpathrectangle{\pgfqpoint{0.444645in}{0.385400in}}{\pgfqpoint{4.960000in}{3.696000in}}%
\pgfusepath{clip}%
\pgfsetbuttcap%
\pgfsetmiterjoin%
\definecolor{currentfill}{rgb}{0.121569,0.466667,0.705882}%
\pgfsetfillcolor{currentfill}%
\pgfsetlinewidth{0.000000pt}%
\definecolor{currentstroke}{rgb}{0.000000,0.000000,0.000000}%
\pgfsetstrokecolor{currentstroke}%
\pgfsetstrokeopacity{0.000000}%
\pgfsetdash{}{0pt}%
\pgfpathmoveto{\pgfqpoint{0.670100in}{0.385400in}}%
\pgfpathlineto{\pgfqpoint{1.619382in}{0.385400in}}%
\pgfpathlineto{\pgfqpoint{1.619382in}{2.625400in}}%
\pgfpathlineto{\pgfqpoint{0.670100in}{2.625400in}}%
\pgfpathclose%
\pgfusepath{fill}%
\end{pgfscope}%
\begin{pgfscope}%
\pgfpathrectangle{\pgfqpoint{0.444645in}{0.385400in}}{\pgfqpoint{4.960000in}{3.696000in}}%
\pgfusepath{clip}%
\pgfsetbuttcap%
\pgfsetmiterjoin%
\definecolor{currentfill}{rgb}{0.121569,0.466667,0.705882}%
\pgfsetfillcolor{currentfill}%
\pgfsetlinewidth{0.000000pt}%
\definecolor{currentstroke}{rgb}{0.000000,0.000000,0.000000}%
\pgfsetstrokecolor{currentstroke}%
\pgfsetstrokeopacity{0.000000}%
\pgfsetdash{}{0pt}%
\pgfpathmoveto{\pgfqpoint{1.856703in}{0.385400in}}%
\pgfpathlineto{\pgfqpoint{2.805985in}{0.385400in}}%
\pgfpathlineto{\pgfqpoint{2.805985in}{0.705400in}}%
\pgfpathlineto{\pgfqpoint{1.856703in}{0.705400in}}%
\pgfpathclose%
\pgfusepath{fill}%
\end{pgfscope}%
\begin{pgfscope}%
\pgfpathrectangle{\pgfqpoint{0.444645in}{0.385400in}}{\pgfqpoint{4.960000in}{3.696000in}}%
\pgfusepath{clip}%
\pgfsetbuttcap%
\pgfsetmiterjoin%
\definecolor{currentfill}{rgb}{0.121569,0.466667,0.705882}%
\pgfsetfillcolor{currentfill}%
\pgfsetlinewidth{0.000000pt}%
\definecolor{currentstroke}{rgb}{0.000000,0.000000,0.000000}%
\pgfsetstrokecolor{currentstroke}%
\pgfsetstrokeopacity{0.000000}%
\pgfsetdash{}{0pt}%
\pgfpathmoveto{\pgfqpoint{3.043305in}{0.385400in}}%
\pgfpathlineto{\pgfqpoint{3.992588in}{0.385400in}}%
\pgfpathlineto{\pgfqpoint{3.992588in}{3.905400in}}%
\pgfpathlineto{\pgfqpoint{3.043305in}{3.905400in}}%
\pgfpathclose%
\pgfusepath{fill}%
\end{pgfscope}%
\begin{pgfscope}%
\pgfpathrectangle{\pgfqpoint{0.444645in}{0.385400in}}{\pgfqpoint{4.960000in}{3.696000in}}%
\pgfusepath{clip}%
\pgfsetbuttcap%
\pgfsetmiterjoin%
\definecolor{currentfill}{rgb}{0.121569,0.466667,0.705882}%
\pgfsetfillcolor{currentfill}%
\pgfsetlinewidth{0.000000pt}%
\definecolor{currentstroke}{rgb}{0.000000,0.000000,0.000000}%
\pgfsetstrokecolor{currentstroke}%
\pgfsetstrokeopacity{0.000000}%
\pgfsetdash{}{0pt}%
\pgfpathmoveto{\pgfqpoint{4.229908in}{0.385400in}}%
\pgfpathlineto{\pgfqpoint{5.179191in}{0.385400in}}%
\pgfpathlineto{\pgfqpoint{5.179191in}{2.305400in}}%
\pgfpathlineto{\pgfqpoint{4.229908in}{2.305400in}}%
\pgfpathclose%
\pgfusepath{fill}%
\end{pgfscope}%
\begin{pgfscope}%
\pgfsetbuttcap%
\pgfsetroundjoin%
\definecolor{currentfill}{rgb}{0.000000,0.000000,0.000000}%
\pgfsetfillcolor{currentfill}%
\pgfsetlinewidth{0.803000pt}%
\definecolor{currentstroke}{rgb}{0.000000,0.000000,0.000000}%
\pgfsetstrokecolor{currentstroke}%
\pgfsetdash{}{0pt}%
\pgfsys@defobject{currentmarker}{\pgfqpoint{0.000000in}{-0.048611in}}{\pgfqpoint{0.000000in}{0.000000in}}{%
\pgfpathmoveto{\pgfqpoint{0.000000in}{0.000000in}}%
\pgfpathlineto{\pgfqpoint{0.000000in}{-0.048611in}}%
\pgfusepath{stroke,fill}%
}%
\begin{pgfscope}%
\pgfsys@transformshift{1.144741in}{0.385400in}%
\pgfsys@useobject{currentmarker}{}%
\end{pgfscope}%
\end{pgfscope}%
\begin{pgfscope}%
\definecolor{textcolor}{rgb}{0.000000,0.000000,0.000000}%
\pgfsetstrokecolor{textcolor}%
\pgfsetfillcolor{textcolor}%
\pgftext[x=1.144741in,y=0.288178in,,top]{\color{textcolor}\sffamily\fontsize{14.000000}{16.800000}\selectfont LivingRoom}%
\end{pgfscope}%
\begin{pgfscope}%
\pgfsetbuttcap%
\pgfsetroundjoin%
\definecolor{currentfill}{rgb}{0.000000,0.000000,0.000000}%
\pgfsetfillcolor{currentfill}%
\pgfsetlinewidth{0.803000pt}%
\definecolor{currentstroke}{rgb}{0.000000,0.000000,0.000000}%
\pgfsetstrokecolor{currentstroke}%
\pgfsetdash{}{0pt}%
\pgfsys@defobject{currentmarker}{\pgfqpoint{0.000000in}{-0.048611in}}{\pgfqpoint{0.000000in}{0.000000in}}{%
\pgfpathmoveto{\pgfqpoint{0.000000in}{0.000000in}}%
\pgfpathlineto{\pgfqpoint{0.000000in}{-0.048611in}}%
\pgfusepath{stroke,fill}%
}%
\begin{pgfscope}%
\pgfsys@transformshift{2.331344in}{0.385400in}%
\pgfsys@useobject{currentmarker}{}%
\end{pgfscope}%
\end{pgfscope}%
\begin{pgfscope}%
\definecolor{textcolor}{rgb}{0.000000,0.000000,0.000000}%
\pgfsetstrokecolor{textcolor}%
\pgfsetfillcolor{textcolor}%
\pgftext[x=2.331344in,y=0.288178in,,top]{\color{textcolor}\sffamily\fontsize{14.000000}{16.800000}\selectfont Kitchen}%
\end{pgfscope}%
\begin{pgfscope}%
\pgfsetbuttcap%
\pgfsetroundjoin%
\definecolor{currentfill}{rgb}{0.000000,0.000000,0.000000}%
\pgfsetfillcolor{currentfill}%
\pgfsetlinewidth{0.803000pt}%
\definecolor{currentstroke}{rgb}{0.000000,0.000000,0.000000}%
\pgfsetstrokecolor{currentstroke}%
\pgfsetdash{}{0pt}%
\pgfsys@defobject{currentmarker}{\pgfqpoint{0.000000in}{-0.048611in}}{\pgfqpoint{0.000000in}{0.000000in}}{%
\pgfpathmoveto{\pgfqpoint{0.000000in}{0.000000in}}%
\pgfpathlineto{\pgfqpoint{0.000000in}{-0.048611in}}%
\pgfusepath{stroke,fill}%
}%
\begin{pgfscope}%
\pgfsys@transformshift{3.517947in}{0.385400in}%
\pgfsys@useobject{currentmarker}{}%
\end{pgfscope}%
\end{pgfscope}%
\begin{pgfscope}%
\definecolor{textcolor}{rgb}{0.000000,0.000000,0.000000}%
\pgfsetstrokecolor{textcolor}%
\pgfsetfillcolor{textcolor}%
\pgftext[x=3.517947in,y=0.288178in,,top]{\color{textcolor}\sffamily\fontsize{14.000000}{16.800000}\selectfont Bedroom}%
\end{pgfscope}%
\begin{pgfscope}%
\pgfsetbuttcap%
\pgfsetroundjoin%
\definecolor{currentfill}{rgb}{0.000000,0.000000,0.000000}%
\pgfsetfillcolor{currentfill}%
\pgfsetlinewidth{0.803000pt}%
\definecolor{currentstroke}{rgb}{0.000000,0.000000,0.000000}%
\pgfsetstrokecolor{currentstroke}%
\pgfsetdash{}{0pt}%
\pgfsys@defobject{currentmarker}{\pgfqpoint{0.000000in}{-0.048611in}}{\pgfqpoint{0.000000in}{0.000000in}}{%
\pgfpathmoveto{\pgfqpoint{0.000000in}{0.000000in}}%
\pgfpathlineto{\pgfqpoint{0.000000in}{-0.048611in}}%
\pgfusepath{stroke,fill}%
}%
\begin{pgfscope}%
\pgfsys@transformshift{4.704549in}{0.385400in}%
\pgfsys@useobject{currentmarker}{}%
\end{pgfscope}%
\end{pgfscope}%
\begin{pgfscope}%
\definecolor{textcolor}{rgb}{0.000000,0.000000,0.000000}%
\pgfsetstrokecolor{textcolor}%
\pgfsetfillcolor{textcolor}%
\pgftext[x=4.704549in,y=0.288178in,,top]{\color{textcolor}\sffamily\fontsize{14.000000}{16.800000}\selectfont Office}%
\end{pgfscope}%
\begin{pgfscope}%
\pgfsetbuttcap%
\pgfsetroundjoin%
\definecolor{currentfill}{rgb}{0.000000,0.000000,0.000000}%
\pgfsetfillcolor{currentfill}%
\pgfsetlinewidth{0.803000pt}%
\definecolor{currentstroke}{rgb}{0.000000,0.000000,0.000000}%
\pgfsetstrokecolor{currentstroke}%
\pgfsetdash{}{0pt}%
\pgfsys@defobject{currentmarker}{\pgfqpoint{-0.048611in}{0.000000in}}{\pgfqpoint{-0.000000in}{0.000000in}}{%
\pgfpathmoveto{\pgfqpoint{-0.000000in}{0.000000in}}%
\pgfpathlineto{\pgfqpoint{-0.048611in}{0.000000in}}%
\pgfusepath{stroke,fill}%
}%
\begin{pgfscope}%
\pgfsys@transformshift{0.444645in}{0.385400in}%
\pgfsys@useobject{currentmarker}{}%
\end{pgfscope}%
\end{pgfscope}%
\begin{pgfscope}%
\definecolor{textcolor}{rgb}{0.000000,0.000000,0.000000}%
\pgfsetstrokecolor{textcolor}%
\pgfsetfillcolor{textcolor}%
\pgftext[x=0.223711in, y=0.311534in, left, base]{\color{textcolor}\sffamily\fontsize{14.000000}{16.800000}\selectfont 0}%
\end{pgfscope}%
\begin{pgfscope}%
\pgfsetbuttcap%
\pgfsetroundjoin%
\definecolor{currentfill}{rgb}{0.000000,0.000000,0.000000}%
\pgfsetfillcolor{currentfill}%
\pgfsetlinewidth{0.803000pt}%
\definecolor{currentstroke}{rgb}{0.000000,0.000000,0.000000}%
\pgfsetstrokecolor{currentstroke}%
\pgfsetdash{}{0pt}%
\pgfsys@defobject{currentmarker}{\pgfqpoint{-0.048611in}{0.000000in}}{\pgfqpoint{-0.000000in}{0.000000in}}{%
\pgfpathmoveto{\pgfqpoint{-0.000000in}{0.000000in}}%
\pgfpathlineto{\pgfqpoint{-0.048611in}{0.000000in}}%
\pgfusepath{stroke,fill}%
}%
\begin{pgfscope}%
\pgfsys@transformshift{0.444645in}{1.025400in}%
\pgfsys@useobject{currentmarker}{}%
\end{pgfscope}%
\end{pgfscope}%
\begin{pgfscope}%
\definecolor{textcolor}{rgb}{0.000000,0.000000,0.000000}%
\pgfsetstrokecolor{textcolor}%
\pgfsetfillcolor{textcolor}%
\pgftext[x=0.223711in, y=0.951534in, left, base]{\color{textcolor}\sffamily\fontsize{14.000000}{16.800000}\selectfont 2}%
\end{pgfscope}%
\begin{pgfscope}%
\pgfsetbuttcap%
\pgfsetroundjoin%
\definecolor{currentfill}{rgb}{0.000000,0.000000,0.000000}%
\pgfsetfillcolor{currentfill}%
\pgfsetlinewidth{0.803000pt}%
\definecolor{currentstroke}{rgb}{0.000000,0.000000,0.000000}%
\pgfsetstrokecolor{currentstroke}%
\pgfsetdash{}{0pt}%
\pgfsys@defobject{currentmarker}{\pgfqpoint{-0.048611in}{0.000000in}}{\pgfqpoint{-0.000000in}{0.000000in}}{%
\pgfpathmoveto{\pgfqpoint{-0.000000in}{0.000000in}}%
\pgfpathlineto{\pgfqpoint{-0.048611in}{0.000000in}}%
\pgfusepath{stroke,fill}%
}%
\begin{pgfscope}%
\pgfsys@transformshift{0.444645in}{1.665400in}%
\pgfsys@useobject{currentmarker}{}%
\end{pgfscope}%
\end{pgfscope}%
\begin{pgfscope}%
\definecolor{textcolor}{rgb}{0.000000,0.000000,0.000000}%
\pgfsetstrokecolor{textcolor}%
\pgfsetfillcolor{textcolor}%
\pgftext[x=0.223711in, y=1.591534in, left, base]{\color{textcolor}\sffamily\fontsize{14.000000}{16.800000}\selectfont 4}%
\end{pgfscope}%
\begin{pgfscope}%
\pgfsetbuttcap%
\pgfsetroundjoin%
\definecolor{currentfill}{rgb}{0.000000,0.000000,0.000000}%
\pgfsetfillcolor{currentfill}%
\pgfsetlinewidth{0.803000pt}%
\definecolor{currentstroke}{rgb}{0.000000,0.000000,0.000000}%
\pgfsetstrokecolor{currentstroke}%
\pgfsetdash{}{0pt}%
\pgfsys@defobject{currentmarker}{\pgfqpoint{-0.048611in}{0.000000in}}{\pgfqpoint{-0.000000in}{0.000000in}}{%
\pgfpathmoveto{\pgfqpoint{-0.000000in}{0.000000in}}%
\pgfpathlineto{\pgfqpoint{-0.048611in}{0.000000in}}%
\pgfusepath{stroke,fill}%
}%
\begin{pgfscope}%
\pgfsys@transformshift{0.444645in}{2.305400in}%
\pgfsys@useobject{currentmarker}{}%
\end{pgfscope}%
\end{pgfscope}%
\begin{pgfscope}%
\definecolor{textcolor}{rgb}{0.000000,0.000000,0.000000}%
\pgfsetstrokecolor{textcolor}%
\pgfsetfillcolor{textcolor}%
\pgftext[x=0.223711in, y=2.231534in, left, base]{\color{textcolor}\sffamily\fontsize{14.000000}{16.800000}\selectfont 6}%
\end{pgfscope}%
\begin{pgfscope}%
\pgfsetbuttcap%
\pgfsetroundjoin%
\definecolor{currentfill}{rgb}{0.000000,0.000000,0.000000}%
\pgfsetfillcolor{currentfill}%
\pgfsetlinewidth{0.803000pt}%
\definecolor{currentstroke}{rgb}{0.000000,0.000000,0.000000}%
\pgfsetstrokecolor{currentstroke}%
\pgfsetdash{}{0pt}%
\pgfsys@defobject{currentmarker}{\pgfqpoint{-0.048611in}{0.000000in}}{\pgfqpoint{-0.000000in}{0.000000in}}{%
\pgfpathmoveto{\pgfqpoint{-0.000000in}{0.000000in}}%
\pgfpathlineto{\pgfqpoint{-0.048611in}{0.000000in}}%
\pgfusepath{stroke,fill}%
}%
\begin{pgfscope}%
\pgfsys@transformshift{0.444645in}{2.945400in}%
\pgfsys@useobject{currentmarker}{}%
\end{pgfscope}%
\end{pgfscope}%
\begin{pgfscope}%
\definecolor{textcolor}{rgb}{0.000000,0.000000,0.000000}%
\pgfsetstrokecolor{textcolor}%
\pgfsetfillcolor{textcolor}%
\pgftext[x=0.223711in, y=2.871534in, left, base]{\color{textcolor}\sffamily\fontsize{14.000000}{16.800000}\selectfont 8}%
\end{pgfscope}%
\begin{pgfscope}%
\pgfsetbuttcap%
\pgfsetroundjoin%
\definecolor{currentfill}{rgb}{0.000000,0.000000,0.000000}%
\pgfsetfillcolor{currentfill}%
\pgfsetlinewidth{0.803000pt}%
\definecolor{currentstroke}{rgb}{0.000000,0.000000,0.000000}%
\pgfsetstrokecolor{currentstroke}%
\pgfsetdash{}{0pt}%
\pgfsys@defobject{currentmarker}{\pgfqpoint{-0.048611in}{0.000000in}}{\pgfqpoint{-0.000000in}{0.000000in}}{%
\pgfpathmoveto{\pgfqpoint{-0.000000in}{0.000000in}}%
\pgfpathlineto{\pgfqpoint{-0.048611in}{0.000000in}}%
\pgfusepath{stroke,fill}%
}%
\begin{pgfscope}%
\pgfsys@transformshift{0.444645in}{3.585400in}%
\pgfsys@useobject{currentmarker}{}%
\end{pgfscope}%
\end{pgfscope}%
\begin{pgfscope}%
\definecolor{textcolor}{rgb}{0.000000,0.000000,0.000000}%
\pgfsetstrokecolor{textcolor}%
\pgfsetfillcolor{textcolor}%
\pgftext[x=0.100000in, y=3.511534in, left, base]{\color{textcolor}\sffamily\fontsize{14.000000}{16.800000}\selectfont 10}%
\end{pgfscope}%
\begin{pgfscope}%
\pgfsetrectcap%
\pgfsetmiterjoin%
\pgfsetlinewidth{0.803000pt}%
\definecolor{currentstroke}{rgb}{0.000000,0.000000,0.000000}%
\pgfsetstrokecolor{currentstroke}%
\pgfsetdash{}{0pt}%
\pgfpathmoveto{\pgfqpoint{0.444645in}{0.385400in}}%
\pgfpathlineto{\pgfqpoint{0.444645in}{4.081400in}}%
\pgfusepath{stroke}%
\end{pgfscope}%
\begin{pgfscope}%
\pgfsetrectcap%
\pgfsetmiterjoin%
\pgfsetlinewidth{0.803000pt}%
\definecolor{currentstroke}{rgb}{0.000000,0.000000,0.000000}%
\pgfsetstrokecolor{currentstroke}%
\pgfsetdash{}{0pt}%
\pgfpathmoveto{\pgfqpoint{5.404645in}{0.385400in}}%
\pgfpathlineto{\pgfqpoint{5.404645in}{4.081400in}}%
\pgfusepath{stroke}%
\end{pgfscope}%
\begin{pgfscope}%
\pgfsetrectcap%
\pgfsetmiterjoin%
\pgfsetlinewidth{0.803000pt}%
\definecolor{currentstroke}{rgb}{0.000000,0.000000,0.000000}%
\pgfsetstrokecolor{currentstroke}%
\pgfsetdash{}{0pt}%
\pgfpathmoveto{\pgfqpoint{0.444645in}{0.385400in}}%
\pgfpathlineto{\pgfqpoint{5.404645in}{0.385400in}}%
\pgfusepath{stroke}%
\end{pgfscope}%
\begin{pgfscope}%
\pgfsetrectcap%
\pgfsetmiterjoin%
\pgfsetlinewidth{0.803000pt}%
\definecolor{currentstroke}{rgb}{0.000000,0.000000,0.000000}%
\pgfsetstrokecolor{currentstroke}%
\pgfsetdash{}{0pt}%
\pgfpathmoveto{\pgfqpoint{0.444645in}{4.081400in}}%
\pgfpathlineto{\pgfqpoint{5.404645in}{4.081400in}}%
\pgfusepath{stroke}%
\end{pgfscope}%
\begin{pgfscope}%
\definecolor{textcolor}{rgb}{0.000000,0.000000,0.000000}%
\pgfsetstrokecolor{textcolor}%
\pgfsetfillcolor{textcolor}%
\pgftext[x=2.924645in,y=4.164734in,,base]{\color{textcolor}\sffamily\fontsize{12.000000}{14.400000}\selectfont Type of scenes}%
\end{pgfscope}%
\end{pgfpicture}%
\makeatother%
\endgroup%
}
    \resizebox{0.49\textwidth}{6cm}{%% Creator: Matplotlib, PGF backend
%%
%% To include the figure in your LaTeX document, write
%%   \input{<filename>.pgf}
%%
%% Make sure the required packages are loaded in your preamble
%%   \usepackage{pgf}
%%
%% Figures using additional raster images can only be included by \input if
%% they are in the same directory as the main LaTeX file. For loading figures
%% from other directories you can use the `import` package
%%   \usepackage{import}
%%
%% and then include the figures with
%%   \import{<path to file>}{<filename>.pgf}
%%
%% Matplotlib used the following preamble
%%   \usepackage{fontspec}
%%   \setmainfont{DejaVuSerif.ttf}[Path=\detokenize{/Users/apple/opt/anaconda3/envs/kaolin/lib/python3.7/site-packages/matplotlib/mpl-data/fonts/ttf/}]
%%   \setsansfont{DejaVuSans.ttf}[Path=\detokenize{/Users/apple/opt/anaconda3/envs/kaolin/lib/python3.7/site-packages/matplotlib/mpl-data/fonts/ttf/}]
%%   \setmonofont{DejaVuSansMono.ttf}[Path=\detokenize{/Users/apple/opt/anaconda3/envs/kaolin/lib/python3.7/site-packages/matplotlib/mpl-data/fonts/ttf/}]
%%
\begingroup%
\makeatletter%
\begin{pgfpicture}%
\pgfpathrectangle{\pgfpointorigin}{\pgfqpoint{5.522318in}{4.337596in}}%
\pgfusepath{use as bounding box, clip}%
\begin{pgfscope}%
\pgfsetbuttcap%
\pgfsetmiterjoin%
\definecolor{currentfill}{rgb}{1.000000,1.000000,1.000000}%
\pgfsetfillcolor{currentfill}%
\pgfsetlinewidth{0.000000pt}%
\definecolor{currentstroke}{rgb}{1.000000,1.000000,1.000000}%
\pgfsetstrokecolor{currentstroke}%
\pgfsetdash{}{0pt}%
\pgfpathmoveto{\pgfqpoint{0.000000in}{0.000000in}}%
\pgfpathlineto{\pgfqpoint{5.522318in}{0.000000in}}%
\pgfpathlineto{\pgfqpoint{5.522318in}{4.337596in}}%
\pgfpathlineto{\pgfqpoint{0.000000in}{4.337596in}}%
\pgfpathclose%
\pgfusepath{fill}%
\end{pgfscope}%
\begin{pgfscope}%
\pgfsetbuttcap%
\pgfsetmiterjoin%
\definecolor{currentfill}{rgb}{1.000000,1.000000,1.000000}%
\pgfsetfillcolor{currentfill}%
\pgfsetlinewidth{0.000000pt}%
\definecolor{currentstroke}{rgb}{0.000000,0.000000,0.000000}%
\pgfsetstrokecolor{currentstroke}%
\pgfsetstrokeopacity{0.000000}%
\pgfsetdash{}{0pt}%
\pgfpathmoveto{\pgfqpoint{0.462318in}{0.331635in}}%
\pgfpathlineto{\pgfqpoint{5.422318in}{0.331635in}}%
\pgfpathlineto{\pgfqpoint{5.422318in}{4.027635in}}%
\pgfpathlineto{\pgfqpoint{0.462318in}{4.027635in}}%
\pgfpathclose%
\pgfusepath{fill}%
\end{pgfscope}%
\begin{pgfscope}%
\pgfpathrectangle{\pgfqpoint{0.462318in}{0.331635in}}{\pgfqpoint{4.960000in}{3.696000in}}%
\pgfusepath{clip}%
\pgfsetbuttcap%
\pgfsetmiterjoin%
\definecolor{currentfill}{rgb}{0.000000,0.000000,1.000000}%
\pgfsetfillcolor{currentfill}%
\pgfsetlinewidth{0.000000pt}%
\definecolor{currentstroke}{rgb}{0.000000,0.000000,0.000000}%
\pgfsetstrokecolor{currentstroke}%
\pgfsetstrokeopacity{0.000000}%
\pgfsetdash{}{0pt}%
\pgfpathmoveto{\pgfqpoint{0.687773in}{0.331635in}}%
\pgfpathlineto{\pgfqpoint{1.218254in}{0.331635in}}%
\pgfpathlineto{\pgfqpoint{1.218254in}{3.851635in}}%
\pgfpathlineto{\pgfqpoint{0.687773in}{3.851635in}}%
\pgfpathclose%
\pgfusepath{fill}%
\end{pgfscope}%
\begin{pgfscope}%
\pgfpathrectangle{\pgfqpoint{0.462318in}{0.331635in}}{\pgfqpoint{4.960000in}{3.696000in}}%
\pgfusepath{clip}%
\pgfsetbuttcap%
\pgfsetmiterjoin%
\definecolor{currentfill}{rgb}{0.000000,0.000000,1.000000}%
\pgfsetfillcolor{currentfill}%
\pgfsetlinewidth{0.000000pt}%
\definecolor{currentstroke}{rgb}{0.000000,0.000000,0.000000}%
\pgfsetstrokecolor{currentstroke}%
\pgfsetstrokeopacity{0.000000}%
\pgfsetdash{}{0pt}%
\pgfpathmoveto{\pgfqpoint{1.350874in}{0.331635in}}%
\pgfpathlineto{\pgfqpoint{1.881356in}{0.331635in}}%
\pgfpathlineto{\pgfqpoint{1.881356in}{1.079246in}}%
\pgfpathlineto{\pgfqpoint{1.350874in}{1.079246in}}%
\pgfpathclose%
\pgfusepath{fill}%
\end{pgfscope}%
\begin{pgfscope}%
\pgfpathrectangle{\pgfqpoint{0.462318in}{0.331635in}}{\pgfqpoint{4.960000in}{3.696000in}}%
\pgfusepath{clip}%
\pgfsetbuttcap%
\pgfsetmiterjoin%
\definecolor{currentfill}{rgb}{0.000000,0.000000,1.000000}%
\pgfsetfillcolor{currentfill}%
\pgfsetlinewidth{0.000000pt}%
\definecolor{currentstroke}{rgb}{0.000000,0.000000,0.000000}%
\pgfsetstrokecolor{currentstroke}%
\pgfsetstrokeopacity{0.000000}%
\pgfsetdash{}{0pt}%
\pgfpathmoveto{\pgfqpoint{2.013976in}{0.331635in}}%
\pgfpathlineto{\pgfqpoint{2.544457in}{0.331635in}}%
\pgfpathlineto{\pgfqpoint{2.544457in}{1.484201in}}%
\pgfpathlineto{\pgfqpoint{2.013976in}{1.484201in}}%
\pgfpathclose%
\pgfusepath{fill}%
\end{pgfscope}%
\begin{pgfscope}%
\pgfpathrectangle{\pgfqpoint{0.462318in}{0.331635in}}{\pgfqpoint{4.960000in}{3.696000in}}%
\pgfusepath{clip}%
\pgfsetbuttcap%
\pgfsetmiterjoin%
\definecolor{currentfill}{rgb}{0.000000,0.000000,1.000000}%
\pgfsetfillcolor{currentfill}%
\pgfsetlinewidth{0.000000pt}%
\definecolor{currentstroke}{rgb}{0.000000,0.000000,0.000000}%
\pgfsetstrokecolor{currentstroke}%
\pgfsetstrokeopacity{0.000000}%
\pgfsetdash{}{0pt}%
\pgfpathmoveto{\pgfqpoint{2.677078in}{0.331635in}}%
\pgfpathlineto{\pgfqpoint{3.207559in}{0.331635in}}%
\pgfpathlineto{\pgfqpoint{3.207559in}{0.954644in}}%
\pgfpathlineto{\pgfqpoint{2.677078in}{0.954644in}}%
\pgfpathclose%
\pgfusepath{fill}%
\end{pgfscope}%
\begin{pgfscope}%
\pgfpathrectangle{\pgfqpoint{0.462318in}{0.331635in}}{\pgfqpoint{4.960000in}{3.696000in}}%
\pgfusepath{clip}%
\pgfsetbuttcap%
\pgfsetmiterjoin%
\definecolor{currentfill}{rgb}{0.000000,0.000000,1.000000}%
\pgfsetfillcolor{currentfill}%
\pgfsetlinewidth{0.000000pt}%
\definecolor{currentstroke}{rgb}{0.000000,0.000000,0.000000}%
\pgfsetstrokecolor{currentstroke}%
\pgfsetstrokeopacity{0.000000}%
\pgfsetdash{}{0pt}%
\pgfpathmoveto{\pgfqpoint{3.340179in}{0.331635in}}%
\pgfpathlineto{\pgfqpoint{3.870660in}{0.331635in}}%
\pgfpathlineto{\pgfqpoint{3.870660in}{1.048095in}}%
\pgfpathlineto{\pgfqpoint{3.340179in}{1.048095in}}%
\pgfpathclose%
\pgfusepath{fill}%
\end{pgfscope}%
\begin{pgfscope}%
\pgfpathrectangle{\pgfqpoint{0.462318in}{0.331635in}}{\pgfqpoint{4.960000in}{3.696000in}}%
\pgfusepath{clip}%
\pgfsetbuttcap%
\pgfsetmiterjoin%
\definecolor{currentfill}{rgb}{0.000000,0.000000,1.000000}%
\pgfsetfillcolor{currentfill}%
\pgfsetlinewidth{0.000000pt}%
\definecolor{currentstroke}{rgb}{0.000000,0.000000,0.000000}%
\pgfsetstrokecolor{currentstroke}%
\pgfsetstrokeopacity{0.000000}%
\pgfsetdash{}{0pt}%
\pgfpathmoveto{\pgfqpoint{4.003281in}{0.331635in}}%
\pgfpathlineto{\pgfqpoint{4.533762in}{0.331635in}}%
\pgfpathlineto{\pgfqpoint{4.533762in}{1.390750in}}%
\pgfpathlineto{\pgfqpoint{4.003281in}{1.390750in}}%
\pgfpathclose%
\pgfusepath{fill}%
\end{pgfscope}%
\begin{pgfscope}%
\pgfpathrectangle{\pgfqpoint{0.462318in}{0.331635in}}{\pgfqpoint{4.960000in}{3.696000in}}%
\pgfusepath{clip}%
\pgfsetbuttcap%
\pgfsetmiterjoin%
\definecolor{currentfill}{rgb}{0.000000,0.000000,1.000000}%
\pgfsetfillcolor{currentfill}%
\pgfsetlinewidth{0.000000pt}%
\definecolor{currentstroke}{rgb}{0.000000,0.000000,0.000000}%
\pgfsetstrokecolor{currentstroke}%
\pgfsetstrokeopacity{0.000000}%
\pgfsetdash{}{0pt}%
\pgfpathmoveto{\pgfqpoint{4.666382in}{0.331635in}}%
\pgfpathlineto{\pgfqpoint{5.196864in}{0.331635in}}%
\pgfpathlineto{\pgfqpoint{5.196864in}{0.674290in}}%
\pgfpathlineto{\pgfqpoint{4.666382in}{0.674290in}}%
\pgfpathclose%
\pgfusepath{fill}%
\end{pgfscope}%
\begin{pgfscope}%
\pgfsetbuttcap%
\pgfsetroundjoin%
\definecolor{currentfill}{rgb}{0.000000,0.000000,0.000000}%
\pgfsetfillcolor{currentfill}%
\pgfsetlinewidth{0.803000pt}%
\definecolor{currentstroke}{rgb}{0.000000,0.000000,0.000000}%
\pgfsetstrokecolor{currentstroke}%
\pgfsetdash{}{0pt}%
\pgfsys@defobject{currentmarker}{\pgfqpoint{0.000000in}{-0.048611in}}{\pgfqpoint{0.000000in}{0.000000in}}{%
\pgfpathmoveto{\pgfqpoint{0.000000in}{0.000000in}}%
\pgfpathlineto{\pgfqpoint{0.000000in}{-0.048611in}}%
\pgfusepath{stroke,fill}%
}%
\begin{pgfscope}%
\pgfsys@transformshift{0.953013in}{0.331635in}%
\pgfsys@useobject{currentmarker}{}%
\end{pgfscope}%
\end{pgfscope}%
\begin{pgfscope}%
\definecolor{textcolor}{rgb}{0.000000,0.000000,0.000000}%
\pgfsetstrokecolor{textcolor}%
\pgfsetfillcolor{textcolor}%
\pgftext[x=0.953013in,y=0.234413in,,top]{\color{textcolor}\sffamily\fontsize{10.000000}{12.000000}\selectfont chair}%
\end{pgfscope}%
\begin{pgfscope}%
\pgfsetbuttcap%
\pgfsetroundjoin%
\definecolor{currentfill}{rgb}{0.000000,0.000000,0.000000}%
\pgfsetfillcolor{currentfill}%
\pgfsetlinewidth{0.803000pt}%
\definecolor{currentstroke}{rgb}{0.000000,0.000000,0.000000}%
\pgfsetstrokecolor{currentstroke}%
\pgfsetdash{}{0pt}%
\pgfsys@defobject{currentmarker}{\pgfqpoint{0.000000in}{-0.048611in}}{\pgfqpoint{0.000000in}{0.000000in}}{%
\pgfpathmoveto{\pgfqpoint{0.000000in}{0.000000in}}%
\pgfpathlineto{\pgfqpoint{0.000000in}{-0.048611in}}%
\pgfusepath{stroke,fill}%
}%
\begin{pgfscope}%
\pgfsys@transformshift{1.616115in}{0.331635in}%
\pgfsys@useobject{currentmarker}{}%
\end{pgfscope}%
\end{pgfscope}%
\begin{pgfscope}%
\definecolor{textcolor}{rgb}{0.000000,0.000000,0.000000}%
\pgfsetstrokecolor{textcolor}%
\pgfsetfillcolor{textcolor}%
\pgftext[x=1.616115in,y=0.234413in,,top]{\color{textcolor}\sffamily\fontsize{10.000000}{12.000000}\selectfont bed}%
\end{pgfscope}%
\begin{pgfscope}%
\pgfsetbuttcap%
\pgfsetroundjoin%
\definecolor{currentfill}{rgb}{0.000000,0.000000,0.000000}%
\pgfsetfillcolor{currentfill}%
\pgfsetlinewidth{0.803000pt}%
\definecolor{currentstroke}{rgb}{0.000000,0.000000,0.000000}%
\pgfsetstrokecolor{currentstroke}%
\pgfsetdash{}{0pt}%
\pgfsys@defobject{currentmarker}{\pgfqpoint{0.000000in}{-0.048611in}}{\pgfqpoint{0.000000in}{0.000000in}}{%
\pgfpathmoveto{\pgfqpoint{0.000000in}{0.000000in}}%
\pgfpathlineto{\pgfqpoint{0.000000in}{-0.048611in}}%
\pgfusepath{stroke,fill}%
}%
\begin{pgfscope}%
\pgfsys@transformshift{2.279217in}{0.331635in}%
\pgfsys@useobject{currentmarker}{}%
\end{pgfscope}%
\end{pgfscope}%
\begin{pgfscope}%
\definecolor{textcolor}{rgb}{0.000000,0.000000,0.000000}%
\pgfsetstrokecolor{textcolor}%
\pgfsetfillcolor{textcolor}%
\pgftext[x=2.279217in,y=0.234413in,,top]{\color{textcolor}\sffamily\fontsize{10.000000}{12.000000}\selectfont desk}%
\end{pgfscope}%
\begin{pgfscope}%
\pgfsetbuttcap%
\pgfsetroundjoin%
\definecolor{currentfill}{rgb}{0.000000,0.000000,0.000000}%
\pgfsetfillcolor{currentfill}%
\pgfsetlinewidth{0.803000pt}%
\definecolor{currentstroke}{rgb}{0.000000,0.000000,0.000000}%
\pgfsetstrokecolor{currentstroke}%
\pgfsetdash{}{0pt}%
\pgfsys@defobject{currentmarker}{\pgfqpoint{0.000000in}{-0.048611in}}{\pgfqpoint{0.000000in}{0.000000in}}{%
\pgfpathmoveto{\pgfqpoint{0.000000in}{0.000000in}}%
\pgfpathlineto{\pgfqpoint{0.000000in}{-0.048611in}}%
\pgfusepath{stroke,fill}%
}%
\begin{pgfscope}%
\pgfsys@transformshift{2.942318in}{0.331635in}%
\pgfsys@useobject{currentmarker}{}%
\end{pgfscope}%
\end{pgfscope}%
\begin{pgfscope}%
\definecolor{textcolor}{rgb}{0.000000,0.000000,0.000000}%
\pgfsetstrokecolor{textcolor}%
\pgfsetfillcolor{textcolor}%
\pgftext[x=2.942318in,y=0.234413in,,top]{\color{textcolor}\sffamily\fontsize{10.000000}{12.000000}\selectfont bookcase}%
\end{pgfscope}%
\begin{pgfscope}%
\pgfsetbuttcap%
\pgfsetroundjoin%
\definecolor{currentfill}{rgb}{0.000000,0.000000,0.000000}%
\pgfsetfillcolor{currentfill}%
\pgfsetlinewidth{0.803000pt}%
\definecolor{currentstroke}{rgb}{0.000000,0.000000,0.000000}%
\pgfsetstrokecolor{currentstroke}%
\pgfsetdash{}{0pt}%
\pgfsys@defobject{currentmarker}{\pgfqpoint{0.000000in}{-0.048611in}}{\pgfqpoint{0.000000in}{0.000000in}}{%
\pgfpathmoveto{\pgfqpoint{0.000000in}{0.000000in}}%
\pgfpathlineto{\pgfqpoint{0.000000in}{-0.048611in}}%
\pgfusepath{stroke,fill}%
}%
\begin{pgfscope}%
\pgfsys@transformshift{3.605420in}{0.331635in}%
\pgfsys@useobject{currentmarker}{}%
\end{pgfscope}%
\end{pgfscope}%
\begin{pgfscope}%
\definecolor{textcolor}{rgb}{0.000000,0.000000,0.000000}%
\pgfsetstrokecolor{textcolor}%
\pgfsetfillcolor{textcolor}%
\pgftext[x=3.605420in,y=0.234413in,,top]{\color{textcolor}\sffamily\fontsize{10.000000}{12.000000}\selectfont sofa}%
\end{pgfscope}%
\begin{pgfscope}%
\pgfsetbuttcap%
\pgfsetroundjoin%
\definecolor{currentfill}{rgb}{0.000000,0.000000,0.000000}%
\pgfsetfillcolor{currentfill}%
\pgfsetlinewidth{0.803000pt}%
\definecolor{currentstroke}{rgb}{0.000000,0.000000,0.000000}%
\pgfsetstrokecolor{currentstroke}%
\pgfsetdash{}{0pt}%
\pgfsys@defobject{currentmarker}{\pgfqpoint{0.000000in}{-0.048611in}}{\pgfqpoint{0.000000in}{0.000000in}}{%
\pgfpathmoveto{\pgfqpoint{0.000000in}{0.000000in}}%
\pgfpathlineto{\pgfqpoint{0.000000in}{-0.048611in}}%
\pgfusepath{stroke,fill}%
}%
\begin{pgfscope}%
\pgfsys@transformshift{4.268521in}{0.331635in}%
\pgfsys@useobject{currentmarker}{}%
\end{pgfscope}%
\end{pgfscope}%
\begin{pgfscope}%
\definecolor{textcolor}{rgb}{0.000000,0.000000,0.000000}%
\pgfsetstrokecolor{textcolor}%
\pgfsetfillcolor{textcolor}%
\pgftext[x=4.268521in,y=0.234413in,,top]{\color{textcolor}\sffamily\fontsize{10.000000}{12.000000}\selectfont table}%
\end{pgfscope}%
\begin{pgfscope}%
\pgfsetbuttcap%
\pgfsetroundjoin%
\definecolor{currentfill}{rgb}{0.000000,0.000000,0.000000}%
\pgfsetfillcolor{currentfill}%
\pgfsetlinewidth{0.803000pt}%
\definecolor{currentstroke}{rgb}{0.000000,0.000000,0.000000}%
\pgfsetstrokecolor{currentstroke}%
\pgfsetdash{}{0pt}%
\pgfsys@defobject{currentmarker}{\pgfqpoint{0.000000in}{-0.048611in}}{\pgfqpoint{0.000000in}{0.000000in}}{%
\pgfpathmoveto{\pgfqpoint{0.000000in}{0.000000in}}%
\pgfpathlineto{\pgfqpoint{0.000000in}{-0.048611in}}%
\pgfusepath{stroke,fill}%
}%
\begin{pgfscope}%
\pgfsys@transformshift{4.931623in}{0.331635in}%
\pgfsys@useobject{currentmarker}{}%
\end{pgfscope}%
\end{pgfscope}%
\begin{pgfscope}%
\definecolor{textcolor}{rgb}{0.000000,0.000000,0.000000}%
\pgfsetstrokecolor{textcolor}%
\pgfsetfillcolor{textcolor}%
\pgftext[x=4.931623in,y=0.234413in,,top]{\color{textcolor}\sffamily\fontsize{10.000000}{12.000000}\selectfont wardrobe}%
\end{pgfscope}%
\begin{pgfscope}%
\pgfsetbuttcap%
\pgfsetroundjoin%
\definecolor{currentfill}{rgb}{0.000000,0.000000,0.000000}%
\pgfsetfillcolor{currentfill}%
\pgfsetlinewidth{0.803000pt}%
\definecolor{currentstroke}{rgb}{0.000000,0.000000,0.000000}%
\pgfsetstrokecolor{currentstroke}%
\pgfsetdash{}{0pt}%
\pgfsys@defobject{currentmarker}{\pgfqpoint{-0.048611in}{0.000000in}}{\pgfqpoint{-0.000000in}{0.000000in}}{%
\pgfpathmoveto{\pgfqpoint{-0.000000in}{0.000000in}}%
\pgfpathlineto{\pgfqpoint{-0.048611in}{0.000000in}}%
\pgfusepath{stroke,fill}%
}%
\begin{pgfscope}%
\pgfsys@transformshift{0.462318in}{0.331635in}%
\pgfsys@useobject{currentmarker}{}%
\end{pgfscope}%
\end{pgfscope}%
\begin{pgfscope}%
\definecolor{textcolor}{rgb}{0.000000,0.000000,0.000000}%
\pgfsetstrokecolor{textcolor}%
\pgfsetfillcolor{textcolor}%
\pgftext[x=0.276731in, y=0.278873in, left, base]{\color{textcolor}\sffamily\fontsize{10.000000}{12.000000}\selectfont 0}%
\end{pgfscope}%
\begin{pgfscope}%
\pgfsetbuttcap%
\pgfsetroundjoin%
\definecolor{currentfill}{rgb}{0.000000,0.000000,0.000000}%
\pgfsetfillcolor{currentfill}%
\pgfsetlinewidth{0.803000pt}%
\definecolor{currentstroke}{rgb}{0.000000,0.000000,0.000000}%
\pgfsetstrokecolor{currentstroke}%
\pgfsetdash{}{0pt}%
\pgfsys@defobject{currentmarker}{\pgfqpoint{-0.048611in}{0.000000in}}{\pgfqpoint{-0.000000in}{0.000000in}}{%
\pgfpathmoveto{\pgfqpoint{-0.000000in}{0.000000in}}%
\pgfpathlineto{\pgfqpoint{-0.048611in}{0.000000in}}%
\pgfusepath{stroke,fill}%
}%
\begin{pgfscope}%
\pgfsys@transformshift{0.462318in}{0.954644in}%
\pgfsys@useobject{currentmarker}{}%
\end{pgfscope}%
\end{pgfscope}%
\begin{pgfscope}%
\definecolor{textcolor}{rgb}{0.000000,0.000000,0.000000}%
\pgfsetstrokecolor{textcolor}%
\pgfsetfillcolor{textcolor}%
\pgftext[x=0.188365in, y=0.901882in, left, base]{\color{textcolor}\sffamily\fontsize{10.000000}{12.000000}\selectfont 20}%
\end{pgfscope}%
\begin{pgfscope}%
\pgfsetbuttcap%
\pgfsetroundjoin%
\definecolor{currentfill}{rgb}{0.000000,0.000000,0.000000}%
\pgfsetfillcolor{currentfill}%
\pgfsetlinewidth{0.803000pt}%
\definecolor{currentstroke}{rgb}{0.000000,0.000000,0.000000}%
\pgfsetstrokecolor{currentstroke}%
\pgfsetdash{}{0pt}%
\pgfsys@defobject{currentmarker}{\pgfqpoint{-0.048611in}{0.000000in}}{\pgfqpoint{-0.000000in}{0.000000in}}{%
\pgfpathmoveto{\pgfqpoint{-0.000000in}{0.000000in}}%
\pgfpathlineto{\pgfqpoint{-0.048611in}{0.000000in}}%
\pgfusepath{stroke,fill}%
}%
\begin{pgfscope}%
\pgfsys@transformshift{0.462318in}{1.577653in}%
\pgfsys@useobject{currentmarker}{}%
\end{pgfscope}%
\end{pgfscope}%
\begin{pgfscope}%
\definecolor{textcolor}{rgb}{0.000000,0.000000,0.000000}%
\pgfsetstrokecolor{textcolor}%
\pgfsetfillcolor{textcolor}%
\pgftext[x=0.188365in, y=1.524891in, left, base]{\color{textcolor}\sffamily\fontsize{10.000000}{12.000000}\selectfont 40}%
\end{pgfscope}%
\begin{pgfscope}%
\pgfsetbuttcap%
\pgfsetroundjoin%
\definecolor{currentfill}{rgb}{0.000000,0.000000,0.000000}%
\pgfsetfillcolor{currentfill}%
\pgfsetlinewidth{0.803000pt}%
\definecolor{currentstroke}{rgb}{0.000000,0.000000,0.000000}%
\pgfsetstrokecolor{currentstroke}%
\pgfsetdash{}{0pt}%
\pgfsys@defobject{currentmarker}{\pgfqpoint{-0.048611in}{0.000000in}}{\pgfqpoint{-0.000000in}{0.000000in}}{%
\pgfpathmoveto{\pgfqpoint{-0.000000in}{0.000000in}}%
\pgfpathlineto{\pgfqpoint{-0.048611in}{0.000000in}}%
\pgfusepath{stroke,fill}%
}%
\begin{pgfscope}%
\pgfsys@transformshift{0.462318in}{2.200662in}%
\pgfsys@useobject{currentmarker}{}%
\end{pgfscope}%
\end{pgfscope}%
\begin{pgfscope}%
\definecolor{textcolor}{rgb}{0.000000,0.000000,0.000000}%
\pgfsetstrokecolor{textcolor}%
\pgfsetfillcolor{textcolor}%
\pgftext[x=0.188365in, y=2.147900in, left, base]{\color{textcolor}\sffamily\fontsize{10.000000}{12.000000}\selectfont 60}%
\end{pgfscope}%
\begin{pgfscope}%
\pgfsetbuttcap%
\pgfsetroundjoin%
\definecolor{currentfill}{rgb}{0.000000,0.000000,0.000000}%
\pgfsetfillcolor{currentfill}%
\pgfsetlinewidth{0.803000pt}%
\definecolor{currentstroke}{rgb}{0.000000,0.000000,0.000000}%
\pgfsetstrokecolor{currentstroke}%
\pgfsetdash{}{0pt}%
\pgfsys@defobject{currentmarker}{\pgfqpoint{-0.048611in}{0.000000in}}{\pgfqpoint{-0.000000in}{0.000000in}}{%
\pgfpathmoveto{\pgfqpoint{-0.000000in}{0.000000in}}%
\pgfpathlineto{\pgfqpoint{-0.048611in}{0.000000in}}%
\pgfusepath{stroke,fill}%
}%
\begin{pgfscope}%
\pgfsys@transformshift{0.462318in}{2.823670in}%
\pgfsys@useobject{currentmarker}{}%
\end{pgfscope}%
\end{pgfscope}%
\begin{pgfscope}%
\definecolor{textcolor}{rgb}{0.000000,0.000000,0.000000}%
\pgfsetstrokecolor{textcolor}%
\pgfsetfillcolor{textcolor}%
\pgftext[x=0.188365in, y=2.770909in, left, base]{\color{textcolor}\sffamily\fontsize{10.000000}{12.000000}\selectfont 80}%
\end{pgfscope}%
\begin{pgfscope}%
\pgfsetbuttcap%
\pgfsetroundjoin%
\definecolor{currentfill}{rgb}{0.000000,0.000000,0.000000}%
\pgfsetfillcolor{currentfill}%
\pgfsetlinewidth{0.803000pt}%
\definecolor{currentstroke}{rgb}{0.000000,0.000000,0.000000}%
\pgfsetstrokecolor{currentstroke}%
\pgfsetdash{}{0pt}%
\pgfsys@defobject{currentmarker}{\pgfqpoint{-0.048611in}{0.000000in}}{\pgfqpoint{-0.000000in}{0.000000in}}{%
\pgfpathmoveto{\pgfqpoint{-0.000000in}{0.000000in}}%
\pgfpathlineto{\pgfqpoint{-0.048611in}{0.000000in}}%
\pgfusepath{stroke,fill}%
}%
\begin{pgfscope}%
\pgfsys@transformshift{0.462318in}{3.446679in}%
\pgfsys@useobject{currentmarker}{}%
\end{pgfscope}%
\end{pgfscope}%
\begin{pgfscope}%
\definecolor{textcolor}{rgb}{0.000000,0.000000,0.000000}%
\pgfsetstrokecolor{textcolor}%
\pgfsetfillcolor{textcolor}%
\pgftext[x=0.100000in, y=3.393918in, left, base]{\color{textcolor}\sffamily\fontsize{10.000000}{12.000000}\selectfont 100}%
\end{pgfscope}%
\begin{pgfscope}%
\pgfsetrectcap%
\pgfsetmiterjoin%
\pgfsetlinewidth{0.803000pt}%
\definecolor{currentstroke}{rgb}{0.000000,0.000000,0.000000}%
\pgfsetstrokecolor{currentstroke}%
\pgfsetdash{}{0pt}%
\pgfpathmoveto{\pgfqpoint{0.462318in}{0.331635in}}%
\pgfpathlineto{\pgfqpoint{0.462318in}{4.027635in}}%
\pgfusepath{stroke}%
\end{pgfscope}%
\begin{pgfscope}%
\pgfsetrectcap%
\pgfsetmiterjoin%
\pgfsetlinewidth{0.803000pt}%
\definecolor{currentstroke}{rgb}{0.000000,0.000000,0.000000}%
\pgfsetstrokecolor{currentstroke}%
\pgfsetdash{}{0pt}%
\pgfpathmoveto{\pgfqpoint{5.422318in}{0.331635in}}%
\pgfpathlineto{\pgfqpoint{5.422318in}{4.027635in}}%
\pgfusepath{stroke}%
\end{pgfscope}%
\begin{pgfscope}%
\pgfsetrectcap%
\pgfsetmiterjoin%
\pgfsetlinewidth{0.803000pt}%
\definecolor{currentstroke}{rgb}{0.000000,0.000000,0.000000}%
\pgfsetstrokecolor{currentstroke}%
\pgfsetdash{}{0pt}%
\pgfpathmoveto{\pgfqpoint{0.462318in}{0.331635in}}%
\pgfpathlineto{\pgfqpoint{5.422318in}{0.331635in}}%
\pgfusepath{stroke}%
\end{pgfscope}%
\begin{pgfscope}%
\pgfsetrectcap%
\pgfsetmiterjoin%
\pgfsetlinewidth{0.803000pt}%
\definecolor{currentstroke}{rgb}{0.000000,0.000000,0.000000}%
\pgfsetstrokecolor{currentstroke}%
\pgfsetdash{}{0pt}%
\pgfpathmoveto{\pgfqpoint{0.462318in}{4.027635in}}%
\pgfpathlineto{\pgfqpoint{5.422318in}{4.027635in}}%
\pgfusepath{stroke}%
\end{pgfscope}%
\begin{pgfscope}%
\definecolor{textcolor}{rgb}{0.000000,0.000000,0.000000}%
\pgfsetstrokecolor{textcolor}%
\pgfsetfillcolor{textcolor}%
\pgftext[x=2.942318in,y=4.110968in,,base]{\color{textcolor}\sffamily\fontsize{12.000000}{14.400000}\selectfont Pix3D Categories in SceneNet}%
\end{pgfscope}%
\end{pgfpicture}%
\makeatother%
\endgroup%
}
    \caption[Distribution of SceneNet.]{(Left)Distribution of types of scenes, (Right) Distribution of objects matching the categories of Pix3D in scenes.
    The chair category has most number of possible positions, while wardrobe has the least.}
    \label{fig:distribution of scenes}
\end{figure}

SceneNet has made 25 rooms available to the public.
We only use scenes of type Bedroom(11), Kitchen(1), Living room(6), and office(7).
We did not consider the bathroom for our case, as the categories under observations are furniture which we rarely see in the bathroom
\autoref{fig:Scene Types} shows different types of scenes utilized to generate a synthetic dataset.
These scenes are further modified by adding a few more objects of categories present in Pix3D to get more variations in the dataset.


\begin{figure}[!ht]
    \centering
    \subfloat[][]{\includegraphics[width=.35\textwidth, height = .35\textwidth]{/Users/apple/OVGU/Thesis/code/3dReconstruction/report/images/implementation/scenenet_scenes/scene_bedroom}}\quad
    \subfloat[][]{\includegraphics[width=.35\textwidth, height = .35\textwidth]{/Users/apple/OVGU/Thesis/code/3dReconstruction/report/images/implementation/scenenet_scenes/scene_livingroom}}\\
    \subfloat[][]{\includegraphics[width=.35\textwidth, height = .35\textwidth]{/Users/apple/OVGU/Thesis/code/3dReconstruction/report/images/implementation/scenenet_scenes/scene_kitchen}}\quad
    \subfloat[][]{\includegraphics[width=.35\textwidth, height = .35\textwidth]{/Users/apple/OVGU/Thesis/code/3dReconstruction/report/images/implementation/scenenet_scenes/scene_office}}
    \caption[Top View for SceneNet Layouts]{The top view of sample scene layouts from SceneNet. Types: (a)Bedroom, (b)LivingRoom, (c)Kitchen and (d)Office.}
    \label{fig:Scene Types}
\end{figure}

\subsection{Randomized Texture}\label{subsec:randomised-texture}

SceneNet~\cite{McCormac2017} also provide textures for different categories in the scene.
We further increase the texture database by adding more textures from ambientCG.com, which provides free licenses.
A total of 982 texture images with 58 regular object categories, some of which are just JPG or PNG images, while there are few textures from cgtextures\footnote{https://www.textures.com/} with more details like normals, displacement, and roughness.
This makes the textures more realistic.
The distribution of the top 40 texture categories used for randomization of scenes is as shown in \autoref{fig:Distribution of textures}, with categories of Pix3D
having a higher number of images.
Each scene is randomized for every snap taken of the target object.
Samples of texture randomization of background in the scene with the target object in focus are shown in \autoref{fig:Texture Randomisation}.

The textures need to be grouped together into a folder with category names.
Since this library is external to Unity, the users can add or remove textures according to their preferences.
Even new categories can be added by just creating a new folder.

\begin{figure}[!ht]
    \centering
    \includegraphics[width=.4\textwidth, height = .3\textwidth,valign=m]{/Users/apple/OVGU/Thesis/code/3dReconstruction/report/images/implementation/randomisation/background_texture1}
    \includegraphics[width=.4\textwidth, height = .3\textwidth,valign=m]{/Users/apple/OVGU/Thesis/code/3dReconstruction/report/images/implementation/randomisation/background_texture2}\\
    \vspace{0.1cm}
    \includegraphics[width=.4\textwidth, height = .3\textwidth,valign=m]{/Users/apple/OVGU/Thesis/code/3dReconstruction/report/images/implementation/randomisation/background_texture3}
    \includegraphics[width=.4\textwidth, height = .3\textwidth,valign=m]{/Users/apple/OVGU/Thesis/code/3dReconstruction/report/images/implementation/randomisation/background_texture4}\\
    \caption[Samples for Different Textures.]{Sample images with different textures for same scene.}
    \label{fig:Texture Randomisation}
\end{figure}

\begin{figure}[!ht]
    \centering
    \resizebox{\textwidth}{11.5cm}{%% Creator: Matplotlib, PGF backend
%%
%% To include the figure in your LaTeX document, write
%%   \input{<filename>.pgf}
%%
%% Make sure the required packages are loaded in your preamble
%%   \usepackage{pgf}
%%
%% Figures using additional raster images can only be included by \input if
%% they are in the same directory as the main LaTeX file. For loading figures
%% from other directories you can use the `import` package
%%   \usepackage{import}
%%
%% and then include the figures with
%%   \import{<path to file>}{<filename>.pgf}
%%
%% Matplotlib used the following preamble
%%   \usepackage{fontspec}
%%   \setmainfont{DejaVuSerif.ttf}[Path=\detokenize{/Users/apple/opt/anaconda3/envs/kaolin/lib/python3.7/site-packages/matplotlib/mpl-data/fonts/ttf/}]
%%   \setsansfont{DejaVuSans.ttf}[Path=\detokenize{/Users/apple/opt/anaconda3/envs/kaolin/lib/python3.7/site-packages/matplotlib/mpl-data/fonts/ttf/}]
%%   \setmonofont{DejaVuSansMono.ttf}[Path=\detokenize{/Users/apple/opt/anaconda3/envs/kaolin/lib/python3.7/site-packages/matplotlib/mpl-data/fonts/ttf/}]
%%
\begingroup%
\makeatletter%
\begin{pgfpicture}%
\pgfpathrectangle{\pgfpointorigin}{\pgfqpoint{5.522318in}{4.872128in}}%
\pgfusepath{use as bounding box, clip}%
\begin{pgfscope}%
\pgfsetbuttcap%
\pgfsetmiterjoin%
\definecolor{currentfill}{rgb}{1.000000,1.000000,1.000000}%
\pgfsetfillcolor{currentfill}%
\pgfsetlinewidth{0.000000pt}%
\definecolor{currentstroke}{rgb}{1.000000,1.000000,1.000000}%
\pgfsetstrokecolor{currentstroke}%
\pgfsetdash{}{0pt}%
\pgfpathmoveto{\pgfqpoint{0.000000in}{0.000000in}}%
\pgfpathlineto{\pgfqpoint{5.522318in}{0.000000in}}%
\pgfpathlineto{\pgfqpoint{5.522318in}{4.872128in}}%
\pgfpathlineto{\pgfqpoint{0.000000in}{4.872128in}}%
\pgfpathclose%
\pgfusepath{fill}%
\end{pgfscope}%
\begin{pgfscope}%
\pgfsetbuttcap%
\pgfsetmiterjoin%
\definecolor{currentfill}{rgb}{1.000000,1.000000,1.000000}%
\pgfsetfillcolor{currentfill}%
\pgfsetlinewidth{0.000000pt}%
\definecolor{currentstroke}{rgb}{0.000000,0.000000,0.000000}%
\pgfsetstrokecolor{currentstroke}%
\pgfsetstrokeopacity{0.000000}%
\pgfsetdash{}{0pt}%
\pgfpathmoveto{\pgfqpoint{0.462318in}{0.866167in}}%
\pgfpathlineto{\pgfqpoint{5.422318in}{0.866167in}}%
\pgfpathlineto{\pgfqpoint{5.422318in}{4.562168in}}%
\pgfpathlineto{\pgfqpoint{0.462318in}{4.562168in}}%
\pgfpathclose%
\pgfusepath{fill}%
\end{pgfscope}%
\begin{pgfscope}%
\pgfpathrectangle{\pgfqpoint{0.462318in}{0.866167in}}{\pgfqpoint{4.960000in}{3.696000in}}%
\pgfusepath{clip}%
\pgfsetbuttcap%
\pgfsetmiterjoin%
\definecolor{currentfill}{rgb}{0.121569,0.466667,0.705882}%
\pgfsetfillcolor{currentfill}%
\pgfsetlinewidth{0.000000pt}%
\definecolor{currentstroke}{rgb}{0.000000,0.000000,0.000000}%
\pgfsetstrokecolor{currentstroke}%
\pgfsetstrokeopacity{0.000000}%
\pgfsetdash{}{0pt}%
\pgfpathmoveto{\pgfqpoint{0.687773in}{0.866167in}}%
\pgfpathlineto{\pgfqpoint{0.778408in}{0.866167in}}%
\pgfpathlineto{\pgfqpoint{0.778408in}{4.386167in}}%
\pgfpathlineto{\pgfqpoint{0.687773in}{4.386167in}}%
\pgfpathclose%
\pgfusepath{fill}%
\end{pgfscope}%
\begin{pgfscope}%
\pgfpathrectangle{\pgfqpoint{0.462318in}{0.866167in}}{\pgfqpoint{4.960000in}{3.696000in}}%
\pgfusepath{clip}%
\pgfsetbuttcap%
\pgfsetmiterjoin%
\definecolor{currentfill}{rgb}{0.121569,0.466667,0.705882}%
\pgfsetfillcolor{currentfill}%
\pgfsetlinewidth{0.000000pt}%
\definecolor{currentstroke}{rgb}{0.000000,0.000000,0.000000}%
\pgfsetstrokecolor{currentstroke}%
\pgfsetstrokeopacity{0.000000}%
\pgfsetdash{}{0pt}%
\pgfpathmoveto{\pgfqpoint{0.801066in}{0.866167in}}%
\pgfpathlineto{\pgfqpoint{0.891701in}{0.866167in}}%
\pgfpathlineto{\pgfqpoint{0.891701in}{4.023281in}}%
\pgfpathlineto{\pgfqpoint{0.801066in}{4.023281in}}%
\pgfpathclose%
\pgfusepath{fill}%
\end{pgfscope}%
\begin{pgfscope}%
\pgfpathrectangle{\pgfqpoint{0.462318in}{0.866167in}}{\pgfqpoint{4.960000in}{3.696000in}}%
\pgfusepath{clip}%
\pgfsetbuttcap%
\pgfsetmiterjoin%
\definecolor{currentfill}{rgb}{0.121569,0.466667,0.705882}%
\pgfsetfillcolor{currentfill}%
\pgfsetlinewidth{0.000000pt}%
\definecolor{currentstroke}{rgb}{0.000000,0.000000,0.000000}%
\pgfsetstrokecolor{currentstroke}%
\pgfsetstrokeopacity{0.000000}%
\pgfsetdash{}{0pt}%
\pgfpathmoveto{\pgfqpoint{0.914360in}{0.866167in}}%
\pgfpathlineto{\pgfqpoint{1.004995in}{0.866167in}}%
\pgfpathlineto{\pgfqpoint{1.004995in}{2.825755in}}%
\pgfpathlineto{\pgfqpoint{0.914360in}{2.825755in}}%
\pgfpathclose%
\pgfusepath{fill}%
\end{pgfscope}%
\begin{pgfscope}%
\pgfpathrectangle{\pgfqpoint{0.462318in}{0.866167in}}{\pgfqpoint{4.960000in}{3.696000in}}%
\pgfusepath{clip}%
\pgfsetbuttcap%
\pgfsetmiterjoin%
\definecolor{currentfill}{rgb}{0.121569,0.466667,0.705882}%
\pgfsetfillcolor{currentfill}%
\pgfsetlinewidth{0.000000pt}%
\definecolor{currentstroke}{rgb}{0.000000,0.000000,0.000000}%
\pgfsetstrokecolor{currentstroke}%
\pgfsetstrokeopacity{0.000000}%
\pgfsetdash{}{0pt}%
\pgfpathmoveto{\pgfqpoint{1.027654in}{0.866167in}}%
\pgfpathlineto{\pgfqpoint{1.118289in}{0.866167in}}%
\pgfpathlineto{\pgfqpoint{1.118289in}{2.716889in}}%
\pgfpathlineto{\pgfqpoint{1.027654in}{2.716889in}}%
\pgfpathclose%
\pgfusepath{fill}%
\end{pgfscope}%
\begin{pgfscope}%
\pgfpathrectangle{\pgfqpoint{0.462318in}{0.866167in}}{\pgfqpoint{4.960000in}{3.696000in}}%
\pgfusepath{clip}%
\pgfsetbuttcap%
\pgfsetmiterjoin%
\definecolor{currentfill}{rgb}{0.121569,0.466667,0.705882}%
\pgfsetfillcolor{currentfill}%
\pgfsetlinewidth{0.000000pt}%
\definecolor{currentstroke}{rgb}{0.000000,0.000000,0.000000}%
\pgfsetstrokecolor{currentstroke}%
\pgfsetstrokeopacity{0.000000}%
\pgfsetdash{}{0pt}%
\pgfpathmoveto{\pgfqpoint{1.140948in}{0.866167in}}%
\pgfpathlineto{\pgfqpoint{1.231583in}{0.866167in}}%
\pgfpathlineto{\pgfqpoint{1.231583in}{2.716889in}}%
\pgfpathlineto{\pgfqpoint{1.140948in}{2.716889in}}%
\pgfpathclose%
\pgfusepath{fill}%
\end{pgfscope}%
\begin{pgfscope}%
\pgfpathrectangle{\pgfqpoint{0.462318in}{0.866167in}}{\pgfqpoint{4.960000in}{3.696000in}}%
\pgfusepath{clip}%
\pgfsetbuttcap%
\pgfsetmiterjoin%
\definecolor{currentfill}{rgb}{0.121569,0.466667,0.705882}%
\pgfsetfillcolor{currentfill}%
\pgfsetlinewidth{0.000000pt}%
\definecolor{currentstroke}{rgb}{0.000000,0.000000,0.000000}%
\pgfsetstrokecolor{currentstroke}%
\pgfsetstrokeopacity{0.000000}%
\pgfsetdash{}{0pt}%
\pgfpathmoveto{\pgfqpoint{1.254241in}{0.866167in}}%
\pgfpathlineto{\pgfqpoint{1.344876in}{0.866167in}}%
\pgfpathlineto{\pgfqpoint{1.344876in}{2.716889in}}%
\pgfpathlineto{\pgfqpoint{1.254241in}{2.716889in}}%
\pgfpathclose%
\pgfusepath{fill}%
\end{pgfscope}%
\begin{pgfscope}%
\pgfpathrectangle{\pgfqpoint{0.462318in}{0.866167in}}{\pgfqpoint{4.960000in}{3.696000in}}%
\pgfusepath{clip}%
\pgfsetbuttcap%
\pgfsetmiterjoin%
\definecolor{currentfill}{rgb}{0.121569,0.466667,0.705882}%
\pgfsetfillcolor{currentfill}%
\pgfsetlinewidth{0.000000pt}%
\definecolor{currentstroke}{rgb}{0.000000,0.000000,0.000000}%
\pgfsetstrokecolor{currentstroke}%
\pgfsetstrokeopacity{0.000000}%
\pgfsetdash{}{0pt}%
\pgfpathmoveto{\pgfqpoint{1.367535in}{0.866167in}}%
\pgfpathlineto{\pgfqpoint{1.458170in}{0.866167in}}%
\pgfpathlineto{\pgfqpoint{1.458170in}{2.571735in}}%
\pgfpathlineto{\pgfqpoint{1.367535in}{2.571735in}}%
\pgfpathclose%
\pgfusepath{fill}%
\end{pgfscope}%
\begin{pgfscope}%
\pgfpathrectangle{\pgfqpoint{0.462318in}{0.866167in}}{\pgfqpoint{4.960000in}{3.696000in}}%
\pgfusepath{clip}%
\pgfsetbuttcap%
\pgfsetmiterjoin%
\definecolor{currentfill}{rgb}{0.121569,0.466667,0.705882}%
\pgfsetfillcolor{currentfill}%
\pgfsetlinewidth{0.000000pt}%
\definecolor{currentstroke}{rgb}{0.000000,0.000000,0.000000}%
\pgfsetstrokecolor{currentstroke}%
\pgfsetstrokeopacity{0.000000}%
\pgfsetdash{}{0pt}%
\pgfpathmoveto{\pgfqpoint{1.480829in}{0.866167in}}%
\pgfpathlineto{\pgfqpoint{1.571464in}{0.866167in}}%
\pgfpathlineto{\pgfqpoint{1.571464in}{2.354003in}}%
\pgfpathlineto{\pgfqpoint{1.480829in}{2.354003in}}%
\pgfpathclose%
\pgfusepath{fill}%
\end{pgfscope}%
\begin{pgfscope}%
\pgfpathrectangle{\pgfqpoint{0.462318in}{0.866167in}}{\pgfqpoint{4.960000in}{3.696000in}}%
\pgfusepath{clip}%
\pgfsetbuttcap%
\pgfsetmiterjoin%
\definecolor{currentfill}{rgb}{0.121569,0.466667,0.705882}%
\pgfsetfillcolor{currentfill}%
\pgfsetlinewidth{0.000000pt}%
\definecolor{currentstroke}{rgb}{0.000000,0.000000,0.000000}%
\pgfsetstrokecolor{currentstroke}%
\pgfsetstrokeopacity{0.000000}%
\pgfsetdash{}{0pt}%
\pgfpathmoveto{\pgfqpoint{1.594123in}{0.866167in}}%
\pgfpathlineto{\pgfqpoint{1.684758in}{0.866167in}}%
\pgfpathlineto{\pgfqpoint{1.684758in}{2.208848in}}%
\pgfpathlineto{\pgfqpoint{1.594123in}{2.208848in}}%
\pgfpathclose%
\pgfusepath{fill}%
\end{pgfscope}%
\begin{pgfscope}%
\pgfpathrectangle{\pgfqpoint{0.462318in}{0.866167in}}{\pgfqpoint{4.960000in}{3.696000in}}%
\pgfusepath{clip}%
\pgfsetbuttcap%
\pgfsetmiterjoin%
\definecolor{currentfill}{rgb}{0.121569,0.466667,0.705882}%
\pgfsetfillcolor{currentfill}%
\pgfsetlinewidth{0.000000pt}%
\definecolor{currentstroke}{rgb}{0.000000,0.000000,0.000000}%
\pgfsetstrokecolor{currentstroke}%
\pgfsetstrokeopacity{0.000000}%
\pgfsetdash{}{0pt}%
\pgfpathmoveto{\pgfqpoint{1.707416in}{0.866167in}}%
\pgfpathlineto{\pgfqpoint{1.798051in}{0.866167in}}%
\pgfpathlineto{\pgfqpoint{1.798051in}{1.918539in}}%
\pgfpathlineto{\pgfqpoint{1.707416in}{1.918539in}}%
\pgfpathclose%
\pgfusepath{fill}%
\end{pgfscope}%
\begin{pgfscope}%
\pgfpathrectangle{\pgfqpoint{0.462318in}{0.866167in}}{\pgfqpoint{4.960000in}{3.696000in}}%
\pgfusepath{clip}%
\pgfsetbuttcap%
\pgfsetmiterjoin%
\definecolor{currentfill}{rgb}{0.121569,0.466667,0.705882}%
\pgfsetfillcolor{currentfill}%
\pgfsetlinewidth{0.000000pt}%
\definecolor{currentstroke}{rgb}{0.000000,0.000000,0.000000}%
\pgfsetstrokecolor{currentstroke}%
\pgfsetstrokeopacity{0.000000}%
\pgfsetdash{}{0pt}%
\pgfpathmoveto{\pgfqpoint{1.820710in}{0.866167in}}%
\pgfpathlineto{\pgfqpoint{1.911345in}{0.866167in}}%
\pgfpathlineto{\pgfqpoint{1.911345in}{1.882250in}}%
\pgfpathlineto{\pgfqpoint{1.820710in}{1.882250in}}%
\pgfpathclose%
\pgfusepath{fill}%
\end{pgfscope}%
\begin{pgfscope}%
\pgfpathrectangle{\pgfqpoint{0.462318in}{0.866167in}}{\pgfqpoint{4.960000in}{3.696000in}}%
\pgfusepath{clip}%
\pgfsetbuttcap%
\pgfsetmiterjoin%
\definecolor{currentfill}{rgb}{0.121569,0.466667,0.705882}%
\pgfsetfillcolor{currentfill}%
\pgfsetlinewidth{0.000000pt}%
\definecolor{currentstroke}{rgb}{0.000000,0.000000,0.000000}%
\pgfsetstrokecolor{currentstroke}%
\pgfsetstrokeopacity{0.000000}%
\pgfsetdash{}{0pt}%
\pgfpathmoveto{\pgfqpoint{1.934004in}{0.866167in}}%
\pgfpathlineto{\pgfqpoint{2.024639in}{0.866167in}}%
\pgfpathlineto{\pgfqpoint{2.024639in}{1.882250in}}%
\pgfpathlineto{\pgfqpoint{1.934004in}{1.882250in}}%
\pgfpathclose%
\pgfusepath{fill}%
\end{pgfscope}%
\begin{pgfscope}%
\pgfpathrectangle{\pgfqpoint{0.462318in}{0.866167in}}{\pgfqpoint{4.960000in}{3.696000in}}%
\pgfusepath{clip}%
\pgfsetbuttcap%
\pgfsetmiterjoin%
\definecolor{currentfill}{rgb}{0.121569,0.466667,0.705882}%
\pgfsetfillcolor{currentfill}%
\pgfsetlinewidth{0.000000pt}%
\definecolor{currentstroke}{rgb}{0.000000,0.000000,0.000000}%
\pgfsetstrokecolor{currentstroke}%
\pgfsetstrokeopacity{0.000000}%
\pgfsetdash{}{0pt}%
\pgfpathmoveto{\pgfqpoint{2.047298in}{0.866167in}}%
\pgfpathlineto{\pgfqpoint{2.137933in}{0.866167in}}%
\pgfpathlineto{\pgfqpoint{2.137933in}{1.845961in}}%
\pgfpathlineto{\pgfqpoint{2.047298in}{1.845961in}}%
\pgfpathclose%
\pgfusepath{fill}%
\end{pgfscope}%
\begin{pgfscope}%
\pgfpathrectangle{\pgfqpoint{0.462318in}{0.866167in}}{\pgfqpoint{4.960000in}{3.696000in}}%
\pgfusepath{clip}%
\pgfsetbuttcap%
\pgfsetmiterjoin%
\definecolor{currentfill}{rgb}{0.121569,0.466667,0.705882}%
\pgfsetfillcolor{currentfill}%
\pgfsetlinewidth{0.000000pt}%
\definecolor{currentstroke}{rgb}{0.000000,0.000000,0.000000}%
\pgfsetstrokecolor{currentstroke}%
\pgfsetstrokeopacity{0.000000}%
\pgfsetdash{}{0pt}%
\pgfpathmoveto{\pgfqpoint{2.160591in}{0.866167in}}%
\pgfpathlineto{\pgfqpoint{2.251226in}{0.866167in}}%
\pgfpathlineto{\pgfqpoint{2.251226in}{1.664518in}}%
\pgfpathlineto{\pgfqpoint{2.160591in}{1.664518in}}%
\pgfpathclose%
\pgfusepath{fill}%
\end{pgfscope}%
\begin{pgfscope}%
\pgfpathrectangle{\pgfqpoint{0.462318in}{0.866167in}}{\pgfqpoint{4.960000in}{3.696000in}}%
\pgfusepath{clip}%
\pgfsetbuttcap%
\pgfsetmiterjoin%
\definecolor{currentfill}{rgb}{0.121569,0.466667,0.705882}%
\pgfsetfillcolor{currentfill}%
\pgfsetlinewidth{0.000000pt}%
\definecolor{currentstroke}{rgb}{0.000000,0.000000,0.000000}%
\pgfsetstrokecolor{currentstroke}%
\pgfsetstrokeopacity{0.000000}%
\pgfsetdash{}{0pt}%
\pgfpathmoveto{\pgfqpoint{2.273885in}{0.866167in}}%
\pgfpathlineto{\pgfqpoint{2.364520in}{0.866167in}}%
\pgfpathlineto{\pgfqpoint{2.364520in}{1.664518in}}%
\pgfpathlineto{\pgfqpoint{2.273885in}{1.664518in}}%
\pgfpathclose%
\pgfusepath{fill}%
\end{pgfscope}%
\begin{pgfscope}%
\pgfpathrectangle{\pgfqpoint{0.462318in}{0.866167in}}{\pgfqpoint{4.960000in}{3.696000in}}%
\pgfusepath{clip}%
\pgfsetbuttcap%
\pgfsetmiterjoin%
\definecolor{currentfill}{rgb}{0.121569,0.466667,0.705882}%
\pgfsetfillcolor{currentfill}%
\pgfsetlinewidth{0.000000pt}%
\definecolor{currentstroke}{rgb}{0.000000,0.000000,0.000000}%
\pgfsetstrokecolor{currentstroke}%
\pgfsetstrokeopacity{0.000000}%
\pgfsetdash{}{0pt}%
\pgfpathmoveto{\pgfqpoint{2.387179in}{0.866167in}}%
\pgfpathlineto{\pgfqpoint{2.477814in}{0.866167in}}%
\pgfpathlineto{\pgfqpoint{2.477814in}{1.591941in}}%
\pgfpathlineto{\pgfqpoint{2.387179in}{1.591941in}}%
\pgfpathclose%
\pgfusepath{fill}%
\end{pgfscope}%
\begin{pgfscope}%
\pgfpathrectangle{\pgfqpoint{0.462318in}{0.866167in}}{\pgfqpoint{4.960000in}{3.696000in}}%
\pgfusepath{clip}%
\pgfsetbuttcap%
\pgfsetmiterjoin%
\definecolor{currentfill}{rgb}{0.121569,0.466667,0.705882}%
\pgfsetfillcolor{currentfill}%
\pgfsetlinewidth{0.000000pt}%
\definecolor{currentstroke}{rgb}{0.000000,0.000000,0.000000}%
\pgfsetstrokecolor{currentstroke}%
\pgfsetstrokeopacity{0.000000}%
\pgfsetdash{}{0pt}%
\pgfpathmoveto{\pgfqpoint{2.500473in}{0.866167in}}%
\pgfpathlineto{\pgfqpoint{2.591108in}{0.866167in}}%
\pgfpathlineto{\pgfqpoint{2.591108in}{1.446786in}}%
\pgfpathlineto{\pgfqpoint{2.500473in}{1.446786in}}%
\pgfpathclose%
\pgfusepath{fill}%
\end{pgfscope}%
\begin{pgfscope}%
\pgfpathrectangle{\pgfqpoint{0.462318in}{0.866167in}}{\pgfqpoint{4.960000in}{3.696000in}}%
\pgfusepath{clip}%
\pgfsetbuttcap%
\pgfsetmiterjoin%
\definecolor{currentfill}{rgb}{0.121569,0.466667,0.705882}%
\pgfsetfillcolor{currentfill}%
\pgfsetlinewidth{0.000000pt}%
\definecolor{currentstroke}{rgb}{0.000000,0.000000,0.000000}%
\pgfsetstrokecolor{currentstroke}%
\pgfsetstrokeopacity{0.000000}%
\pgfsetdash{}{0pt}%
\pgfpathmoveto{\pgfqpoint{2.613766in}{0.866167in}}%
\pgfpathlineto{\pgfqpoint{2.704401in}{0.866167in}}%
\pgfpathlineto{\pgfqpoint{2.704401in}{1.446786in}}%
\pgfpathlineto{\pgfqpoint{2.613766in}{1.446786in}}%
\pgfpathclose%
\pgfusepath{fill}%
\end{pgfscope}%
\begin{pgfscope}%
\pgfpathrectangle{\pgfqpoint{0.462318in}{0.866167in}}{\pgfqpoint{4.960000in}{3.696000in}}%
\pgfusepath{clip}%
\pgfsetbuttcap%
\pgfsetmiterjoin%
\definecolor{currentfill}{rgb}{0.121569,0.466667,0.705882}%
\pgfsetfillcolor{currentfill}%
\pgfsetlinewidth{0.000000pt}%
\definecolor{currentstroke}{rgb}{0.000000,0.000000,0.000000}%
\pgfsetstrokecolor{currentstroke}%
\pgfsetstrokeopacity{0.000000}%
\pgfsetdash{}{0pt}%
\pgfpathmoveto{\pgfqpoint{2.727060in}{0.866167in}}%
\pgfpathlineto{\pgfqpoint{2.817695in}{0.866167in}}%
\pgfpathlineto{\pgfqpoint{2.817695in}{1.446786in}}%
\pgfpathlineto{\pgfqpoint{2.727060in}{1.446786in}}%
\pgfpathclose%
\pgfusepath{fill}%
\end{pgfscope}%
\begin{pgfscope}%
\pgfpathrectangle{\pgfqpoint{0.462318in}{0.866167in}}{\pgfqpoint{4.960000in}{3.696000in}}%
\pgfusepath{clip}%
\pgfsetbuttcap%
\pgfsetmiterjoin%
\definecolor{currentfill}{rgb}{0.121569,0.466667,0.705882}%
\pgfsetfillcolor{currentfill}%
\pgfsetlinewidth{0.000000pt}%
\definecolor{currentstroke}{rgb}{0.000000,0.000000,0.000000}%
\pgfsetstrokecolor{currentstroke}%
\pgfsetstrokeopacity{0.000000}%
\pgfsetdash{}{0pt}%
\pgfpathmoveto{\pgfqpoint{2.840354in}{0.866167in}}%
\pgfpathlineto{\pgfqpoint{2.930989in}{0.866167in}}%
\pgfpathlineto{\pgfqpoint{2.930989in}{1.410497in}}%
\pgfpathlineto{\pgfqpoint{2.840354in}{1.410497in}}%
\pgfpathclose%
\pgfusepath{fill}%
\end{pgfscope}%
\begin{pgfscope}%
\pgfpathrectangle{\pgfqpoint{0.462318in}{0.866167in}}{\pgfqpoint{4.960000in}{3.696000in}}%
\pgfusepath{clip}%
\pgfsetbuttcap%
\pgfsetmiterjoin%
\definecolor{currentfill}{rgb}{0.121569,0.466667,0.705882}%
\pgfsetfillcolor{currentfill}%
\pgfsetlinewidth{0.000000pt}%
\definecolor{currentstroke}{rgb}{0.000000,0.000000,0.000000}%
\pgfsetstrokecolor{currentstroke}%
\pgfsetstrokeopacity{0.000000}%
\pgfsetdash{}{0pt}%
\pgfpathmoveto{\pgfqpoint{2.953648in}{0.866167in}}%
\pgfpathlineto{\pgfqpoint{3.044283in}{0.866167in}}%
\pgfpathlineto{\pgfqpoint{3.044283in}{1.374209in}}%
\pgfpathlineto{\pgfqpoint{2.953648in}{1.374209in}}%
\pgfpathclose%
\pgfusepath{fill}%
\end{pgfscope}%
\begin{pgfscope}%
\pgfpathrectangle{\pgfqpoint{0.462318in}{0.866167in}}{\pgfqpoint{4.960000in}{3.696000in}}%
\pgfusepath{clip}%
\pgfsetbuttcap%
\pgfsetmiterjoin%
\definecolor{currentfill}{rgb}{0.121569,0.466667,0.705882}%
\pgfsetfillcolor{currentfill}%
\pgfsetlinewidth{0.000000pt}%
\definecolor{currentstroke}{rgb}{0.000000,0.000000,0.000000}%
\pgfsetstrokecolor{currentstroke}%
\pgfsetstrokeopacity{0.000000}%
\pgfsetdash{}{0pt}%
\pgfpathmoveto{\pgfqpoint{3.066941in}{0.866167in}}%
\pgfpathlineto{\pgfqpoint{3.157576in}{0.866167in}}%
\pgfpathlineto{\pgfqpoint{3.157576in}{1.374209in}}%
\pgfpathlineto{\pgfqpoint{3.066941in}{1.374209in}}%
\pgfpathclose%
\pgfusepath{fill}%
\end{pgfscope}%
\begin{pgfscope}%
\pgfpathrectangle{\pgfqpoint{0.462318in}{0.866167in}}{\pgfqpoint{4.960000in}{3.696000in}}%
\pgfusepath{clip}%
\pgfsetbuttcap%
\pgfsetmiterjoin%
\definecolor{currentfill}{rgb}{0.121569,0.466667,0.705882}%
\pgfsetfillcolor{currentfill}%
\pgfsetlinewidth{0.000000pt}%
\definecolor{currentstroke}{rgb}{0.000000,0.000000,0.000000}%
\pgfsetstrokecolor{currentstroke}%
\pgfsetstrokeopacity{0.000000}%
\pgfsetdash{}{0pt}%
\pgfpathmoveto{\pgfqpoint{3.180235in}{0.866167in}}%
\pgfpathlineto{\pgfqpoint{3.270870in}{0.866167in}}%
\pgfpathlineto{\pgfqpoint{3.270870in}{1.374209in}}%
\pgfpathlineto{\pgfqpoint{3.180235in}{1.374209in}}%
\pgfpathclose%
\pgfusepath{fill}%
\end{pgfscope}%
\begin{pgfscope}%
\pgfpathrectangle{\pgfqpoint{0.462318in}{0.866167in}}{\pgfqpoint{4.960000in}{3.696000in}}%
\pgfusepath{clip}%
\pgfsetbuttcap%
\pgfsetmiterjoin%
\definecolor{currentfill}{rgb}{0.121569,0.466667,0.705882}%
\pgfsetfillcolor{currentfill}%
\pgfsetlinewidth{0.000000pt}%
\definecolor{currentstroke}{rgb}{0.000000,0.000000,0.000000}%
\pgfsetstrokecolor{currentstroke}%
\pgfsetstrokeopacity{0.000000}%
\pgfsetdash{}{0pt}%
\pgfpathmoveto{\pgfqpoint{3.293529in}{0.866167in}}%
\pgfpathlineto{\pgfqpoint{3.384164in}{0.866167in}}%
\pgfpathlineto{\pgfqpoint{3.384164in}{1.301631in}}%
\pgfpathlineto{\pgfqpoint{3.293529in}{1.301631in}}%
\pgfpathclose%
\pgfusepath{fill}%
\end{pgfscope}%
\begin{pgfscope}%
\pgfpathrectangle{\pgfqpoint{0.462318in}{0.866167in}}{\pgfqpoint{4.960000in}{3.696000in}}%
\pgfusepath{clip}%
\pgfsetbuttcap%
\pgfsetmiterjoin%
\definecolor{currentfill}{rgb}{0.121569,0.466667,0.705882}%
\pgfsetfillcolor{currentfill}%
\pgfsetlinewidth{0.000000pt}%
\definecolor{currentstroke}{rgb}{0.000000,0.000000,0.000000}%
\pgfsetstrokecolor{currentstroke}%
\pgfsetstrokeopacity{0.000000}%
\pgfsetdash{}{0pt}%
\pgfpathmoveto{\pgfqpoint{3.406823in}{0.866167in}}%
\pgfpathlineto{\pgfqpoint{3.497458in}{0.866167in}}%
\pgfpathlineto{\pgfqpoint{3.497458in}{1.301631in}}%
\pgfpathlineto{\pgfqpoint{3.406823in}{1.301631in}}%
\pgfpathclose%
\pgfusepath{fill}%
\end{pgfscope}%
\begin{pgfscope}%
\pgfpathrectangle{\pgfqpoint{0.462318in}{0.866167in}}{\pgfqpoint{4.960000in}{3.696000in}}%
\pgfusepath{clip}%
\pgfsetbuttcap%
\pgfsetmiterjoin%
\definecolor{currentfill}{rgb}{0.121569,0.466667,0.705882}%
\pgfsetfillcolor{currentfill}%
\pgfsetlinewidth{0.000000pt}%
\definecolor{currentstroke}{rgb}{0.000000,0.000000,0.000000}%
\pgfsetstrokecolor{currentstroke}%
\pgfsetstrokeopacity{0.000000}%
\pgfsetdash{}{0pt}%
\pgfpathmoveto{\pgfqpoint{3.520116in}{0.866167in}}%
\pgfpathlineto{\pgfqpoint{3.610751in}{0.866167in}}%
\pgfpathlineto{\pgfqpoint{3.610751in}{1.301631in}}%
\pgfpathlineto{\pgfqpoint{3.520116in}{1.301631in}}%
\pgfpathclose%
\pgfusepath{fill}%
\end{pgfscope}%
\begin{pgfscope}%
\pgfpathrectangle{\pgfqpoint{0.462318in}{0.866167in}}{\pgfqpoint{4.960000in}{3.696000in}}%
\pgfusepath{clip}%
\pgfsetbuttcap%
\pgfsetmiterjoin%
\definecolor{currentfill}{rgb}{0.121569,0.466667,0.705882}%
\pgfsetfillcolor{currentfill}%
\pgfsetlinewidth{0.000000pt}%
\definecolor{currentstroke}{rgb}{0.000000,0.000000,0.000000}%
\pgfsetstrokecolor{currentstroke}%
\pgfsetstrokeopacity{0.000000}%
\pgfsetdash{}{0pt}%
\pgfpathmoveto{\pgfqpoint{3.633410in}{0.866167in}}%
\pgfpathlineto{\pgfqpoint{3.724045in}{0.866167in}}%
\pgfpathlineto{\pgfqpoint{3.724045in}{1.156477in}}%
\pgfpathlineto{\pgfqpoint{3.633410in}{1.156477in}}%
\pgfpathclose%
\pgfusepath{fill}%
\end{pgfscope}%
\begin{pgfscope}%
\pgfpathrectangle{\pgfqpoint{0.462318in}{0.866167in}}{\pgfqpoint{4.960000in}{3.696000in}}%
\pgfusepath{clip}%
\pgfsetbuttcap%
\pgfsetmiterjoin%
\definecolor{currentfill}{rgb}{0.121569,0.466667,0.705882}%
\pgfsetfillcolor{currentfill}%
\pgfsetlinewidth{0.000000pt}%
\definecolor{currentstroke}{rgb}{0.000000,0.000000,0.000000}%
\pgfsetstrokecolor{currentstroke}%
\pgfsetstrokeopacity{0.000000}%
\pgfsetdash{}{0pt}%
\pgfpathmoveto{\pgfqpoint{3.746704in}{0.866167in}}%
\pgfpathlineto{\pgfqpoint{3.837339in}{0.866167in}}%
\pgfpathlineto{\pgfqpoint{3.837339in}{1.156477in}}%
\pgfpathlineto{\pgfqpoint{3.746704in}{1.156477in}}%
\pgfpathclose%
\pgfusepath{fill}%
\end{pgfscope}%
\begin{pgfscope}%
\pgfpathrectangle{\pgfqpoint{0.462318in}{0.866167in}}{\pgfqpoint{4.960000in}{3.696000in}}%
\pgfusepath{clip}%
\pgfsetbuttcap%
\pgfsetmiterjoin%
\definecolor{currentfill}{rgb}{0.121569,0.466667,0.705882}%
\pgfsetfillcolor{currentfill}%
\pgfsetlinewidth{0.000000pt}%
\definecolor{currentstroke}{rgb}{0.000000,0.000000,0.000000}%
\pgfsetstrokecolor{currentstroke}%
\pgfsetstrokeopacity{0.000000}%
\pgfsetdash{}{0pt}%
\pgfpathmoveto{\pgfqpoint{3.859998in}{0.866167in}}%
\pgfpathlineto{\pgfqpoint{3.950632in}{0.866167in}}%
\pgfpathlineto{\pgfqpoint{3.950632in}{1.120188in}}%
\pgfpathlineto{\pgfqpoint{3.859998in}{1.120188in}}%
\pgfpathclose%
\pgfusepath{fill}%
\end{pgfscope}%
\begin{pgfscope}%
\pgfpathrectangle{\pgfqpoint{0.462318in}{0.866167in}}{\pgfqpoint{4.960000in}{3.696000in}}%
\pgfusepath{clip}%
\pgfsetbuttcap%
\pgfsetmiterjoin%
\definecolor{currentfill}{rgb}{0.121569,0.466667,0.705882}%
\pgfsetfillcolor{currentfill}%
\pgfsetlinewidth{0.000000pt}%
\definecolor{currentstroke}{rgb}{0.000000,0.000000,0.000000}%
\pgfsetstrokecolor{currentstroke}%
\pgfsetstrokeopacity{0.000000}%
\pgfsetdash{}{0pt}%
\pgfpathmoveto{\pgfqpoint{3.973291in}{0.866167in}}%
\pgfpathlineto{\pgfqpoint{4.063926in}{0.866167in}}%
\pgfpathlineto{\pgfqpoint{4.063926in}{1.120188in}}%
\pgfpathlineto{\pgfqpoint{3.973291in}{1.120188in}}%
\pgfpathclose%
\pgfusepath{fill}%
\end{pgfscope}%
\begin{pgfscope}%
\pgfpathrectangle{\pgfqpoint{0.462318in}{0.866167in}}{\pgfqpoint{4.960000in}{3.696000in}}%
\pgfusepath{clip}%
\pgfsetbuttcap%
\pgfsetmiterjoin%
\definecolor{currentfill}{rgb}{0.121569,0.466667,0.705882}%
\pgfsetfillcolor{currentfill}%
\pgfsetlinewidth{0.000000pt}%
\definecolor{currentstroke}{rgb}{0.000000,0.000000,0.000000}%
\pgfsetstrokecolor{currentstroke}%
\pgfsetstrokeopacity{0.000000}%
\pgfsetdash{}{0pt}%
\pgfpathmoveto{\pgfqpoint{4.086585in}{0.866167in}}%
\pgfpathlineto{\pgfqpoint{4.177220in}{0.866167in}}%
\pgfpathlineto{\pgfqpoint{4.177220in}{1.120188in}}%
\pgfpathlineto{\pgfqpoint{4.086585in}{1.120188in}}%
\pgfpathclose%
\pgfusepath{fill}%
\end{pgfscope}%
\begin{pgfscope}%
\pgfpathrectangle{\pgfqpoint{0.462318in}{0.866167in}}{\pgfqpoint{4.960000in}{3.696000in}}%
\pgfusepath{clip}%
\pgfsetbuttcap%
\pgfsetmiterjoin%
\definecolor{currentfill}{rgb}{0.121569,0.466667,0.705882}%
\pgfsetfillcolor{currentfill}%
\pgfsetlinewidth{0.000000pt}%
\definecolor{currentstroke}{rgb}{0.000000,0.000000,0.000000}%
\pgfsetstrokecolor{currentstroke}%
\pgfsetstrokeopacity{0.000000}%
\pgfsetdash{}{0pt}%
\pgfpathmoveto{\pgfqpoint{4.199879in}{0.866167in}}%
\pgfpathlineto{\pgfqpoint{4.290514in}{0.866167in}}%
\pgfpathlineto{\pgfqpoint{4.290514in}{1.120188in}}%
\pgfpathlineto{\pgfqpoint{4.199879in}{1.120188in}}%
\pgfpathclose%
\pgfusepath{fill}%
\end{pgfscope}%
\begin{pgfscope}%
\pgfpathrectangle{\pgfqpoint{0.462318in}{0.866167in}}{\pgfqpoint{4.960000in}{3.696000in}}%
\pgfusepath{clip}%
\pgfsetbuttcap%
\pgfsetmiterjoin%
\definecolor{currentfill}{rgb}{0.121569,0.466667,0.705882}%
\pgfsetfillcolor{currentfill}%
\pgfsetlinewidth{0.000000pt}%
\definecolor{currentstroke}{rgb}{0.000000,0.000000,0.000000}%
\pgfsetstrokecolor{currentstroke}%
\pgfsetstrokeopacity{0.000000}%
\pgfsetdash{}{0pt}%
\pgfpathmoveto{\pgfqpoint{4.313172in}{0.866167in}}%
\pgfpathlineto{\pgfqpoint{4.403807in}{0.866167in}}%
\pgfpathlineto{\pgfqpoint{4.403807in}{1.083899in}}%
\pgfpathlineto{\pgfqpoint{4.313172in}{1.083899in}}%
\pgfpathclose%
\pgfusepath{fill}%
\end{pgfscope}%
\begin{pgfscope}%
\pgfpathrectangle{\pgfqpoint{0.462318in}{0.866167in}}{\pgfqpoint{4.960000in}{3.696000in}}%
\pgfusepath{clip}%
\pgfsetbuttcap%
\pgfsetmiterjoin%
\definecolor{currentfill}{rgb}{0.121569,0.466667,0.705882}%
\pgfsetfillcolor{currentfill}%
\pgfsetlinewidth{0.000000pt}%
\definecolor{currentstroke}{rgb}{0.000000,0.000000,0.000000}%
\pgfsetstrokecolor{currentstroke}%
\pgfsetstrokeopacity{0.000000}%
\pgfsetdash{}{0pt}%
\pgfpathmoveto{\pgfqpoint{4.426466in}{0.866167in}}%
\pgfpathlineto{\pgfqpoint{4.517101in}{0.866167in}}%
\pgfpathlineto{\pgfqpoint{4.517101in}{1.083899in}}%
\pgfpathlineto{\pgfqpoint{4.426466in}{1.083899in}}%
\pgfpathclose%
\pgfusepath{fill}%
\end{pgfscope}%
\begin{pgfscope}%
\pgfpathrectangle{\pgfqpoint{0.462318in}{0.866167in}}{\pgfqpoint{4.960000in}{3.696000in}}%
\pgfusepath{clip}%
\pgfsetbuttcap%
\pgfsetmiterjoin%
\definecolor{currentfill}{rgb}{0.121569,0.466667,0.705882}%
\pgfsetfillcolor{currentfill}%
\pgfsetlinewidth{0.000000pt}%
\definecolor{currentstroke}{rgb}{0.000000,0.000000,0.000000}%
\pgfsetstrokecolor{currentstroke}%
\pgfsetstrokeopacity{0.000000}%
\pgfsetdash{}{0pt}%
\pgfpathmoveto{\pgfqpoint{4.539760in}{0.866167in}}%
\pgfpathlineto{\pgfqpoint{4.630395in}{0.866167in}}%
\pgfpathlineto{\pgfqpoint{4.630395in}{1.083899in}}%
\pgfpathlineto{\pgfqpoint{4.539760in}{1.083899in}}%
\pgfpathclose%
\pgfusepath{fill}%
\end{pgfscope}%
\begin{pgfscope}%
\pgfpathrectangle{\pgfqpoint{0.462318in}{0.866167in}}{\pgfqpoint{4.960000in}{3.696000in}}%
\pgfusepath{clip}%
\pgfsetbuttcap%
\pgfsetmiterjoin%
\definecolor{currentfill}{rgb}{0.121569,0.466667,0.705882}%
\pgfsetfillcolor{currentfill}%
\pgfsetlinewidth{0.000000pt}%
\definecolor{currentstroke}{rgb}{0.000000,0.000000,0.000000}%
\pgfsetstrokecolor{currentstroke}%
\pgfsetstrokeopacity{0.000000}%
\pgfsetdash{}{0pt}%
\pgfpathmoveto{\pgfqpoint{4.653054in}{0.866167in}}%
\pgfpathlineto{\pgfqpoint{4.743689in}{0.866167in}}%
\pgfpathlineto{\pgfqpoint{4.743689in}{1.083899in}}%
\pgfpathlineto{\pgfqpoint{4.653054in}{1.083899in}}%
\pgfpathclose%
\pgfusepath{fill}%
\end{pgfscope}%
\begin{pgfscope}%
\pgfpathrectangle{\pgfqpoint{0.462318in}{0.866167in}}{\pgfqpoint{4.960000in}{3.696000in}}%
\pgfusepath{clip}%
\pgfsetbuttcap%
\pgfsetmiterjoin%
\definecolor{currentfill}{rgb}{0.121569,0.466667,0.705882}%
\pgfsetfillcolor{currentfill}%
\pgfsetlinewidth{0.000000pt}%
\definecolor{currentstroke}{rgb}{0.000000,0.000000,0.000000}%
\pgfsetstrokecolor{currentstroke}%
\pgfsetstrokeopacity{0.000000}%
\pgfsetdash{}{0pt}%
\pgfpathmoveto{\pgfqpoint{4.766347in}{0.866167in}}%
\pgfpathlineto{\pgfqpoint{4.856982in}{0.866167in}}%
\pgfpathlineto{\pgfqpoint{4.856982in}{1.083899in}}%
\pgfpathlineto{\pgfqpoint{4.766347in}{1.083899in}}%
\pgfpathclose%
\pgfusepath{fill}%
\end{pgfscope}%
\begin{pgfscope}%
\pgfpathrectangle{\pgfqpoint{0.462318in}{0.866167in}}{\pgfqpoint{4.960000in}{3.696000in}}%
\pgfusepath{clip}%
\pgfsetbuttcap%
\pgfsetmiterjoin%
\definecolor{currentfill}{rgb}{0.121569,0.466667,0.705882}%
\pgfsetfillcolor{currentfill}%
\pgfsetlinewidth{0.000000pt}%
\definecolor{currentstroke}{rgb}{0.000000,0.000000,0.000000}%
\pgfsetstrokecolor{currentstroke}%
\pgfsetstrokeopacity{0.000000}%
\pgfsetdash{}{0pt}%
\pgfpathmoveto{\pgfqpoint{4.879641in}{0.866167in}}%
\pgfpathlineto{\pgfqpoint{4.970276in}{0.866167in}}%
\pgfpathlineto{\pgfqpoint{4.970276in}{1.083899in}}%
\pgfpathlineto{\pgfqpoint{4.879641in}{1.083899in}}%
\pgfpathclose%
\pgfusepath{fill}%
\end{pgfscope}%
\begin{pgfscope}%
\pgfpathrectangle{\pgfqpoint{0.462318in}{0.866167in}}{\pgfqpoint{4.960000in}{3.696000in}}%
\pgfusepath{clip}%
\pgfsetbuttcap%
\pgfsetmiterjoin%
\definecolor{currentfill}{rgb}{0.121569,0.466667,0.705882}%
\pgfsetfillcolor{currentfill}%
\pgfsetlinewidth{0.000000pt}%
\definecolor{currentstroke}{rgb}{0.000000,0.000000,0.000000}%
\pgfsetstrokecolor{currentstroke}%
\pgfsetstrokeopacity{0.000000}%
\pgfsetdash{}{0pt}%
\pgfpathmoveto{\pgfqpoint{4.992935in}{0.866167in}}%
\pgfpathlineto{\pgfqpoint{5.083570in}{0.866167in}}%
\pgfpathlineto{\pgfqpoint{5.083570in}{1.083899in}}%
\pgfpathlineto{\pgfqpoint{4.992935in}{1.083899in}}%
\pgfpathclose%
\pgfusepath{fill}%
\end{pgfscope}%
\begin{pgfscope}%
\pgfpathrectangle{\pgfqpoint{0.462318in}{0.866167in}}{\pgfqpoint{4.960000in}{3.696000in}}%
\pgfusepath{clip}%
\pgfsetbuttcap%
\pgfsetmiterjoin%
\definecolor{currentfill}{rgb}{0.121569,0.466667,0.705882}%
\pgfsetfillcolor{currentfill}%
\pgfsetlinewidth{0.000000pt}%
\definecolor{currentstroke}{rgb}{0.000000,0.000000,0.000000}%
\pgfsetstrokecolor{currentstroke}%
\pgfsetstrokeopacity{0.000000}%
\pgfsetdash{}{0pt}%
\pgfpathmoveto{\pgfqpoint{5.106229in}{0.866167in}}%
\pgfpathlineto{\pgfqpoint{5.196864in}{0.866167in}}%
\pgfpathlineto{\pgfqpoint{5.196864in}{1.047611in}}%
\pgfpathlineto{\pgfqpoint{5.106229in}{1.047611in}}%
\pgfpathclose%
\pgfusepath{fill}%
\end{pgfscope}%
\begin{pgfscope}%
\pgfsetbuttcap%
\pgfsetroundjoin%
\definecolor{currentfill}{rgb}{0.000000,0.000000,0.000000}%
\pgfsetfillcolor{currentfill}%
\pgfsetlinewidth{0.803000pt}%
\definecolor{currentstroke}{rgb}{0.000000,0.000000,0.000000}%
\pgfsetstrokecolor{currentstroke}%
\pgfsetdash{}{0pt}%
\pgfsys@defobject{currentmarker}{\pgfqpoint{0.000000in}{-0.048611in}}{\pgfqpoint{0.000000in}{0.000000in}}{%
\pgfpathmoveto{\pgfqpoint{0.000000in}{0.000000in}}%
\pgfpathlineto{\pgfqpoint{0.000000in}{-0.048611in}}%
\pgfusepath{stroke,fill}%
}%
\begin{pgfscope}%
\pgfsys@transformshift{0.733090in}{0.866167in}%
\pgfsys@useobject{currentmarker}{}%
\end{pgfscope}%
\end{pgfscope}%
\begin{pgfscope}%
\definecolor{textcolor}{rgb}{0.000000,0.000000,0.000000}%
\pgfsetstrokecolor{textcolor}%
\pgfsetfillcolor{textcolor}%
\pgftext[x=0.771407in, y=0.493066in, left, base,rotate=90.000000]{\color{textcolor}\sffamily\fontsize{10.000000}{12.000000}\selectfont wall}%
\end{pgfscope}%
\begin{pgfscope}%
\pgfsetbuttcap%
\pgfsetroundjoin%
\definecolor{currentfill}{rgb}{0.000000,0.000000,0.000000}%
\pgfsetfillcolor{currentfill}%
\pgfsetlinewidth{0.803000pt}%
\definecolor{currentstroke}{rgb}{0.000000,0.000000,0.000000}%
\pgfsetstrokecolor{currentstroke}%
\pgfsetdash{}{0pt}%
\pgfsys@defobject{currentmarker}{\pgfqpoint{0.000000in}{-0.048611in}}{\pgfqpoint{0.000000in}{0.000000in}}{%
\pgfpathmoveto{\pgfqpoint{0.000000in}{0.000000in}}%
\pgfpathlineto{\pgfqpoint{0.000000in}{-0.048611in}}%
\pgfusepath{stroke,fill}%
}%
\begin{pgfscope}%
\pgfsys@transformshift{0.846384in}{0.866167in}%
\pgfsys@useobject{currentmarker}{}%
\end{pgfscope}%
\end{pgfscope}%
\begin{pgfscope}%
\definecolor{textcolor}{rgb}{0.000000,0.000000,0.000000}%
\pgfsetstrokecolor{textcolor}%
\pgfsetfillcolor{textcolor}%
\pgftext[x=0.884701in, y=0.454411in, left, base,rotate=90.000000]{\color{textcolor}\sffamily\fontsize{10.000000}{12.000000}\selectfont floor}%
\end{pgfscope}%
\begin{pgfscope}%
\pgfsetbuttcap%
\pgfsetroundjoin%
\definecolor{currentfill}{rgb}{0.000000,0.000000,0.000000}%
\pgfsetfillcolor{currentfill}%
\pgfsetlinewidth{0.803000pt}%
\definecolor{currentstroke}{rgb}{0.000000,0.000000,0.000000}%
\pgfsetstrokecolor{currentstroke}%
\pgfsetdash{}{0pt}%
\pgfsys@defobject{currentmarker}{\pgfqpoint{0.000000in}{-0.048611in}}{\pgfqpoint{0.000000in}{0.000000in}}{%
\pgfpathmoveto{\pgfqpoint{0.000000in}{0.000000in}}%
\pgfpathlineto{\pgfqpoint{0.000000in}{-0.048611in}}%
\pgfusepath{stroke,fill}%
}%
\begin{pgfscope}%
\pgfsys@transformshift{0.959678in}{0.866167in}%
\pgfsys@useobject{currentmarker}{}%
\end{pgfscope}%
\end{pgfscope}%
\begin{pgfscope}%
\definecolor{textcolor}{rgb}{0.000000,0.000000,0.000000}%
\pgfsetstrokecolor{textcolor}%
\pgfsetfillcolor{textcolor}%
\pgftext[x=0.997994in, y=0.417179in, left, base,rotate=90.000000]{\color{textcolor}\sffamily\fontsize{10.000000}{12.000000}\selectfont table}%
\end{pgfscope}%
\begin{pgfscope}%
\pgfsetbuttcap%
\pgfsetroundjoin%
\definecolor{currentfill}{rgb}{0.000000,0.000000,0.000000}%
\pgfsetfillcolor{currentfill}%
\pgfsetlinewidth{0.803000pt}%
\definecolor{currentstroke}{rgb}{0.000000,0.000000,0.000000}%
\pgfsetstrokecolor{currentstroke}%
\pgfsetdash{}{0pt}%
\pgfsys@defobject{currentmarker}{\pgfqpoint{0.000000in}{-0.048611in}}{\pgfqpoint{0.000000in}{0.000000in}}{%
\pgfpathmoveto{\pgfqpoint{0.000000in}{0.000000in}}%
\pgfpathlineto{\pgfqpoint{0.000000in}{-0.048611in}}%
\pgfusepath{stroke,fill}%
}%
\begin{pgfscope}%
\pgfsys@transformshift{1.072971in}{0.866167in}%
\pgfsys@useobject{currentmarker}{}%
\end{pgfscope}%
\end{pgfscope}%
\begin{pgfscope}%
\definecolor{textcolor}{rgb}{0.000000,0.000000,0.000000}%
\pgfsetstrokecolor{textcolor}%
\pgfsetfillcolor{textcolor}%
\pgftext[x=1.111288in, y=0.442543in, left, base,rotate=90.000000]{\color{textcolor}\sffamily\fontsize{10.000000}{12.000000}\selectfont desk}%
\end{pgfscope}%
\begin{pgfscope}%
\pgfsetbuttcap%
\pgfsetroundjoin%
\definecolor{currentfill}{rgb}{0.000000,0.000000,0.000000}%
\pgfsetfillcolor{currentfill}%
\pgfsetlinewidth{0.803000pt}%
\definecolor{currentstroke}{rgb}{0.000000,0.000000,0.000000}%
\pgfsetstrokecolor{currentstroke}%
\pgfsetdash{}{0pt}%
\pgfsys@defobject{currentmarker}{\pgfqpoint{0.000000in}{-0.048611in}}{\pgfqpoint{0.000000in}{0.000000in}}{%
\pgfpathmoveto{\pgfqpoint{0.000000in}{0.000000in}}%
\pgfpathlineto{\pgfqpoint{0.000000in}{-0.048611in}}%
\pgfusepath{stroke,fill}%
}%
\begin{pgfscope}%
\pgfsys@transformshift{1.186265in}{0.866167in}%
\pgfsys@useobject{currentmarker}{}%
\end{pgfscope}%
\end{pgfscope}%
\begin{pgfscope}%
\definecolor{textcolor}{rgb}{0.000000,0.000000,0.000000}%
\pgfsetstrokecolor{textcolor}%
\pgfsetfillcolor{textcolor}%
\pgftext[x=1.224582in, y=0.111122in, left, base,rotate=90.000000]{\color{textcolor}\sffamily\fontsize{10.000000}{12.000000}\selectfont bookcase}%
\end{pgfscope}%
\begin{pgfscope}%
\pgfsetbuttcap%
\pgfsetroundjoin%
\definecolor{currentfill}{rgb}{0.000000,0.000000,0.000000}%
\pgfsetfillcolor{currentfill}%
\pgfsetlinewidth{0.803000pt}%
\definecolor{currentstroke}{rgb}{0.000000,0.000000,0.000000}%
\pgfsetstrokecolor{currentstroke}%
\pgfsetdash{}{0pt}%
\pgfsys@defobject{currentmarker}{\pgfqpoint{0.000000in}{-0.048611in}}{\pgfqpoint{0.000000in}{0.000000in}}{%
\pgfpathmoveto{\pgfqpoint{0.000000in}{0.000000in}}%
\pgfpathlineto{\pgfqpoint{0.000000in}{-0.048611in}}%
\pgfusepath{stroke,fill}%
}%
\begin{pgfscope}%
\pgfsys@transformshift{1.299559in}{0.866167in}%
\pgfsys@useobject{currentmarker}{}%
\end{pgfscope}%
\end{pgfscope}%
\begin{pgfscope}%
\definecolor{textcolor}{rgb}{0.000000,0.000000,0.000000}%
\pgfsetstrokecolor{textcolor}%
\pgfsetfillcolor{textcolor}%
\pgftext[x=1.337876in, y=0.114784in, left, base,rotate=90.000000]{\color{textcolor}\sffamily\fontsize{10.000000}{12.000000}\selectfont wardrobe}%
\end{pgfscope}%
\begin{pgfscope}%
\pgfsetbuttcap%
\pgfsetroundjoin%
\definecolor{currentfill}{rgb}{0.000000,0.000000,0.000000}%
\pgfsetfillcolor{currentfill}%
\pgfsetlinewidth{0.803000pt}%
\definecolor{currentstroke}{rgb}{0.000000,0.000000,0.000000}%
\pgfsetstrokecolor{currentstroke}%
\pgfsetdash{}{0pt}%
\pgfsys@defobject{currentmarker}{\pgfqpoint{0.000000in}{-0.048611in}}{\pgfqpoint{0.000000in}{0.000000in}}{%
\pgfpathmoveto{\pgfqpoint{0.000000in}{0.000000in}}%
\pgfpathlineto{\pgfqpoint{0.000000in}{-0.048611in}}%
\pgfusepath{stroke,fill}%
}%
\begin{pgfscope}%
\pgfsys@transformshift{1.412853in}{0.866167in}%
\pgfsys@useobject{currentmarker}{}%
\end{pgfscope}%
\end{pgfscope}%
\begin{pgfscope}%
\definecolor{textcolor}{rgb}{0.000000,0.000000,0.000000}%
\pgfsetstrokecolor{textcolor}%
\pgfsetfillcolor{textcolor}%
\pgftext[x=1.451169in, y=0.199826in, left, base,rotate=90.000000]{\color{textcolor}\sffamily\fontsize{10.000000}{12.000000}\selectfont painting}%
\end{pgfscope}%
\begin{pgfscope}%
\pgfsetbuttcap%
\pgfsetroundjoin%
\definecolor{currentfill}{rgb}{0.000000,0.000000,0.000000}%
\pgfsetfillcolor{currentfill}%
\pgfsetlinewidth{0.803000pt}%
\definecolor{currentstroke}{rgb}{0.000000,0.000000,0.000000}%
\pgfsetstrokecolor{currentstroke}%
\pgfsetdash{}{0pt}%
\pgfsys@defobject{currentmarker}{\pgfqpoint{0.000000in}{-0.048611in}}{\pgfqpoint{0.000000in}{0.000000in}}{%
\pgfpathmoveto{\pgfqpoint{0.000000in}{0.000000in}}%
\pgfpathlineto{\pgfqpoint{0.000000in}{-0.048611in}}%
\pgfusepath{stroke,fill}%
}%
\begin{pgfscope}%
\pgfsys@transformshift{1.526146in}{0.866167in}%
\pgfsys@useobject{currentmarker}{}%
\end{pgfscope}%
\end{pgfscope}%
\begin{pgfscope}%
\definecolor{textcolor}{rgb}{0.000000,0.000000,0.000000}%
\pgfsetstrokecolor{textcolor}%
\pgfsetfillcolor{textcolor}%
\pgftext[x=1.564463in, y=0.322304in, left, base,rotate=90.000000]{\color{textcolor}\sffamily\fontsize{10.000000}{12.000000}\selectfont carpet}%
\end{pgfscope}%
\begin{pgfscope}%
\pgfsetbuttcap%
\pgfsetroundjoin%
\definecolor{currentfill}{rgb}{0.000000,0.000000,0.000000}%
\pgfsetfillcolor{currentfill}%
\pgfsetlinewidth{0.803000pt}%
\definecolor{currentstroke}{rgb}{0.000000,0.000000,0.000000}%
\pgfsetstrokecolor{currentstroke}%
\pgfsetdash{}{0pt}%
\pgfsys@defobject{currentmarker}{\pgfqpoint{0.000000in}{-0.048611in}}{\pgfqpoint{0.000000in}{0.000000in}}{%
\pgfpathmoveto{\pgfqpoint{0.000000in}{0.000000in}}%
\pgfpathlineto{\pgfqpoint{0.000000in}{-0.048611in}}%
\pgfusepath{stroke,fill}%
}%
\begin{pgfscope}%
\pgfsys@transformshift{1.639440in}{0.866167in}%
\pgfsys@useobject{currentmarker}{}%
\end{pgfscope}%
\end{pgfscope}%
\begin{pgfscope}%
\definecolor{textcolor}{rgb}{0.000000,0.000000,0.000000}%
\pgfsetstrokecolor{textcolor}%
\pgfsetfillcolor{textcolor}%
\pgftext[x=1.677757in, y=0.423757in, left, base,rotate=90.000000]{\color{textcolor}\sffamily\fontsize{10.000000}{12.000000}\selectfont chair}%
\end{pgfscope}%
\begin{pgfscope}%
\pgfsetbuttcap%
\pgfsetroundjoin%
\definecolor{currentfill}{rgb}{0.000000,0.000000,0.000000}%
\pgfsetfillcolor{currentfill}%
\pgfsetlinewidth{0.803000pt}%
\definecolor{currentstroke}{rgb}{0.000000,0.000000,0.000000}%
\pgfsetstrokecolor{currentstroke}%
\pgfsetdash{}{0pt}%
\pgfsys@defobject{currentmarker}{\pgfqpoint{0.000000in}{-0.048611in}}{\pgfqpoint{0.000000in}{0.000000in}}{%
\pgfpathmoveto{\pgfqpoint{0.000000in}{0.000000in}}%
\pgfpathlineto{\pgfqpoint{0.000000in}{-0.048611in}}%
\pgfusepath{stroke,fill}%
}%
\begin{pgfscope}%
\pgfsys@transformshift{1.752734in}{0.866167in}%
\pgfsys@useobject{currentmarker}{}%
\end{pgfscope}%
\end{pgfscope}%
\begin{pgfscope}%
\definecolor{textcolor}{rgb}{0.000000,0.000000,0.000000}%
\pgfsetstrokecolor{textcolor}%
\pgfsetfillcolor{textcolor}%
\pgftext[x=1.791051in, y=0.632294in, left, base,rotate=90.000000]{\color{textcolor}\sffamily\fontsize{10.000000}{12.000000}\selectfont tv}%
\end{pgfscope}%
\begin{pgfscope}%
\pgfsetbuttcap%
\pgfsetroundjoin%
\definecolor{currentfill}{rgb}{0.000000,0.000000,0.000000}%
\pgfsetfillcolor{currentfill}%
\pgfsetlinewidth{0.803000pt}%
\definecolor{currentstroke}{rgb}{0.000000,0.000000,0.000000}%
\pgfsetstrokecolor{currentstroke}%
\pgfsetdash{}{0pt}%
\pgfsys@defobject{currentmarker}{\pgfqpoint{0.000000in}{-0.048611in}}{\pgfqpoint{0.000000in}{0.000000in}}{%
\pgfpathmoveto{\pgfqpoint{0.000000in}{0.000000in}}%
\pgfpathlineto{\pgfqpoint{0.000000in}{-0.048611in}}%
\pgfusepath{stroke,fill}%
}%
\begin{pgfscope}%
\pgfsys@transformshift{1.866028in}{0.866167in}%
\pgfsys@useobject{currentmarker}{}%
\end{pgfscope}%
\end{pgfscope}%
\begin{pgfscope}%
\definecolor{textcolor}{rgb}{0.000000,0.000000,0.000000}%
\pgfsetstrokecolor{textcolor}%
\pgfsetfillcolor{textcolor}%
\pgftext[x=1.904344in, y=0.168766in, left, base,rotate=90.000000]{\color{textcolor}\sffamily\fontsize{10.000000}{12.000000}\selectfont furniture}%
\end{pgfscope}%
\begin{pgfscope}%
\pgfsetbuttcap%
\pgfsetroundjoin%
\definecolor{currentfill}{rgb}{0.000000,0.000000,0.000000}%
\pgfsetfillcolor{currentfill}%
\pgfsetlinewidth{0.803000pt}%
\definecolor{currentstroke}{rgb}{0.000000,0.000000,0.000000}%
\pgfsetstrokecolor{currentstroke}%
\pgfsetdash{}{0pt}%
\pgfsys@defobject{currentmarker}{\pgfqpoint{0.000000in}{-0.048611in}}{\pgfqpoint{0.000000in}{0.000000in}}{%
\pgfpathmoveto{\pgfqpoint{0.000000in}{0.000000in}}%
\pgfpathlineto{\pgfqpoint{0.000000in}{-0.048611in}}%
\pgfusepath{stroke,fill}%
}%
\begin{pgfscope}%
\pgfsys@transformshift{1.979321in}{0.866167in}%
\pgfsys@useobject{currentmarker}{}%
\end{pgfscope}%
\end{pgfscope}%
\begin{pgfscope}%
\definecolor{textcolor}{rgb}{0.000000,0.000000,0.000000}%
\pgfsetstrokecolor{textcolor}%
\pgfsetfillcolor{textcolor}%
\pgftext[x=2.017638in, y=0.100271in, left, base,rotate=90.000000]{\color{textcolor}\sffamily\fontsize{10.000000}{12.000000}\selectfont television}%
\end{pgfscope}%
\begin{pgfscope}%
\pgfsetbuttcap%
\pgfsetroundjoin%
\definecolor{currentfill}{rgb}{0.000000,0.000000,0.000000}%
\pgfsetfillcolor{currentfill}%
\pgfsetlinewidth{0.803000pt}%
\definecolor{currentstroke}{rgb}{0.000000,0.000000,0.000000}%
\pgfsetstrokecolor{currentstroke}%
\pgfsetdash{}{0pt}%
\pgfsys@defobject{currentmarker}{\pgfqpoint{0.000000in}{-0.048611in}}{\pgfqpoint{0.000000in}{0.000000in}}{%
\pgfpathmoveto{\pgfqpoint{0.000000in}{0.000000in}}%
\pgfpathlineto{\pgfqpoint{0.000000in}{-0.048611in}}%
\pgfusepath{stroke,fill}%
}%
\begin{pgfscope}%
\pgfsys@transformshift{2.092615in}{0.866167in}%
\pgfsys@useobject{currentmarker}{}%
\end{pgfscope}%
\end{pgfscope}%
\begin{pgfscope}%
\definecolor{textcolor}{rgb}{0.000000,0.000000,0.000000}%
\pgfsetstrokecolor{textcolor}%
\pgfsetfillcolor{textcolor}%
\pgftext[x=2.130932in, y=0.507172in, left, base,rotate=90.000000]{\color{textcolor}\sffamily\fontsize{10.000000}{12.000000}\selectfont bed}%
\end{pgfscope}%
\begin{pgfscope}%
\pgfsetbuttcap%
\pgfsetroundjoin%
\definecolor{currentfill}{rgb}{0.000000,0.000000,0.000000}%
\pgfsetfillcolor{currentfill}%
\pgfsetlinewidth{0.803000pt}%
\definecolor{currentstroke}{rgb}{0.000000,0.000000,0.000000}%
\pgfsetstrokecolor{currentstroke}%
\pgfsetdash{}{0pt}%
\pgfsys@defobject{currentmarker}{\pgfqpoint{0.000000in}{-0.048611in}}{\pgfqpoint{0.000000in}{0.000000in}}{%
\pgfpathmoveto{\pgfqpoint{0.000000in}{0.000000in}}%
\pgfpathlineto{\pgfqpoint{0.000000in}{-0.048611in}}%
\pgfusepath{stroke,fill}%
}%
\begin{pgfscope}%
\pgfsys@transformshift{2.205909in}{0.866167in}%
\pgfsys@useobject{currentmarker}{}%
\end{pgfscope}%
\end{pgfscope}%
\begin{pgfscope}%
\definecolor{textcolor}{rgb}{0.000000,0.000000,0.000000}%
\pgfsetstrokecolor{textcolor}%
\pgfsetfillcolor{textcolor}%
\pgftext[x=2.244226in, y=0.366452in, left, base,rotate=90.000000]{\color{textcolor}\sffamily\fontsize{10.000000}{12.000000}\selectfont pillow}%
\end{pgfscope}%
\begin{pgfscope}%
\pgfsetbuttcap%
\pgfsetroundjoin%
\definecolor{currentfill}{rgb}{0.000000,0.000000,0.000000}%
\pgfsetfillcolor{currentfill}%
\pgfsetlinewidth{0.803000pt}%
\definecolor{currentstroke}{rgb}{0.000000,0.000000,0.000000}%
\pgfsetstrokecolor{currentstroke}%
\pgfsetdash{}{0pt}%
\pgfsys@defobject{currentmarker}{\pgfqpoint{0.000000in}{-0.048611in}}{\pgfqpoint{0.000000in}{0.000000in}}{%
\pgfpathmoveto{\pgfqpoint{0.000000in}{0.000000in}}%
\pgfpathlineto{\pgfqpoint{0.000000in}{-0.048611in}}%
\pgfusepath{stroke,fill}%
}%
\begin{pgfscope}%
\pgfsys@transformshift{2.319203in}{0.866167in}%
\pgfsys@useobject{currentmarker}{}%
\end{pgfscope}%
\end{pgfscope}%
\begin{pgfscope}%
\definecolor{textcolor}{rgb}{0.000000,0.000000,0.000000}%
\pgfsetstrokecolor{textcolor}%
\pgfsetfillcolor{textcolor}%
\pgftext[x=2.357519in, y=0.433998in, left, base,rotate=90.000000]{\color{textcolor}\sffamily\fontsize{10.000000}{12.000000}\selectfont deco}%
\end{pgfscope}%
\begin{pgfscope}%
\pgfsetbuttcap%
\pgfsetroundjoin%
\definecolor{currentfill}{rgb}{0.000000,0.000000,0.000000}%
\pgfsetfillcolor{currentfill}%
\pgfsetlinewidth{0.803000pt}%
\definecolor{currentstroke}{rgb}{0.000000,0.000000,0.000000}%
\pgfsetstrokecolor{currentstroke}%
\pgfsetdash{}{0pt}%
\pgfsys@defobject{currentmarker}{\pgfqpoint{0.000000in}{-0.048611in}}{\pgfqpoint{0.000000in}{0.000000in}}{%
\pgfpathmoveto{\pgfqpoint{0.000000in}{0.000000in}}%
\pgfpathlineto{\pgfqpoint{0.000000in}{-0.048611in}}%
\pgfusepath{stroke,fill}%
}%
\begin{pgfscope}%
\pgfsys@transformshift{2.432496in}{0.866167in}%
\pgfsys@useobject{currentmarker}{}%
\end{pgfscope}%
\end{pgfscope}%
\begin{pgfscope}%
\definecolor{textcolor}{rgb}{0.000000,0.000000,0.000000}%
\pgfsetstrokecolor{textcolor}%
\pgfsetfillcolor{textcolor}%
\pgftext[x=2.470813in, y=0.137842in, left, base,rotate=90.000000]{\color{textcolor}\sffamily\fontsize{10.000000}{12.000000}\selectfont unknown}%
\end{pgfscope}%
\begin{pgfscope}%
\pgfsetbuttcap%
\pgfsetroundjoin%
\definecolor{currentfill}{rgb}{0.000000,0.000000,0.000000}%
\pgfsetfillcolor{currentfill}%
\pgfsetlinewidth{0.803000pt}%
\definecolor{currentstroke}{rgb}{0.000000,0.000000,0.000000}%
\pgfsetstrokecolor{currentstroke}%
\pgfsetdash{}{0pt}%
\pgfsys@defobject{currentmarker}{\pgfqpoint{0.000000in}{-0.048611in}}{\pgfqpoint{0.000000in}{0.000000in}}{%
\pgfpathmoveto{\pgfqpoint{0.000000in}{0.000000in}}%
\pgfpathlineto{\pgfqpoint{0.000000in}{-0.048611in}}%
\pgfusepath{stroke,fill}%
}%
\begin{pgfscope}%
\pgfsys@transformshift{2.545790in}{0.866167in}%
\pgfsys@useobject{currentmarker}{}%
\end{pgfscope}%
\end{pgfscope}%
\begin{pgfscope}%
\definecolor{textcolor}{rgb}{0.000000,0.000000,0.000000}%
\pgfsetstrokecolor{textcolor}%
\pgfsetfillcolor{textcolor}%
\pgftext[x=2.584107in, y=0.426538in, left, base,rotate=90.000000]{\color{textcolor}\sffamily\fontsize{10.000000}{12.000000}\selectfont cloth}%
\end{pgfscope}%
\begin{pgfscope}%
\pgfsetbuttcap%
\pgfsetroundjoin%
\definecolor{currentfill}{rgb}{0.000000,0.000000,0.000000}%
\pgfsetfillcolor{currentfill}%
\pgfsetlinewidth{0.803000pt}%
\definecolor{currentstroke}{rgb}{0.000000,0.000000,0.000000}%
\pgfsetstrokecolor{currentstroke}%
\pgfsetdash{}{0pt}%
\pgfsys@defobject{currentmarker}{\pgfqpoint{0.000000in}{-0.048611in}}{\pgfqpoint{0.000000in}{0.000000in}}{%
\pgfpathmoveto{\pgfqpoint{0.000000in}{0.000000in}}%
\pgfpathlineto{\pgfqpoint{0.000000in}{-0.048611in}}%
\pgfusepath{stroke,fill}%
}%
\begin{pgfscope}%
\pgfsys@transformshift{2.659084in}{0.866167in}%
\pgfsys@useobject{currentmarker}{}%
\end{pgfscope}%
\end{pgfscope}%
\begin{pgfscope}%
\definecolor{textcolor}{rgb}{0.000000,0.000000,0.000000}%
\pgfsetstrokecolor{textcolor}%
\pgfsetfillcolor{textcolor}%
\pgftext[x=2.697401in, y=0.477604in, left, base,rotate=90.000000]{\color{textcolor}\sffamily\fontsize{10.000000}{12.000000}\selectfont sofa}%
\end{pgfscope}%
\begin{pgfscope}%
\pgfsetbuttcap%
\pgfsetroundjoin%
\definecolor{currentfill}{rgb}{0.000000,0.000000,0.000000}%
\pgfsetfillcolor{currentfill}%
\pgfsetlinewidth{0.803000pt}%
\definecolor{currentstroke}{rgb}{0.000000,0.000000,0.000000}%
\pgfsetstrokecolor{currentstroke}%
\pgfsetdash{}{0pt}%
\pgfsys@defobject{currentmarker}{\pgfqpoint{0.000000in}{-0.048611in}}{\pgfqpoint{0.000000in}{0.000000in}}{%
\pgfpathmoveto{\pgfqpoint{0.000000in}{0.000000in}}%
\pgfpathlineto{\pgfqpoint{0.000000in}{-0.048611in}}%
\pgfusepath{stroke,fill}%
}%
\begin{pgfscope}%
\pgfsys@transformshift{2.772378in}{0.866167in}%
\pgfsys@useobject{currentmarker}{}%
\end{pgfscope}%
\end{pgfscope}%
\begin{pgfscope}%
\definecolor{textcolor}{rgb}{0.000000,0.000000,0.000000}%
\pgfsetstrokecolor{textcolor}%
\pgfsetfillcolor{textcolor}%
\pgftext[x=2.810694in, y=0.443831in, left, base,rotate=90.000000]{\color{textcolor}\sffamily\fontsize{10.000000}{12.000000}\selectfont vase}%
\end{pgfscope}%
\begin{pgfscope}%
\pgfsetbuttcap%
\pgfsetroundjoin%
\definecolor{currentfill}{rgb}{0.000000,0.000000,0.000000}%
\pgfsetfillcolor{currentfill}%
\pgfsetlinewidth{0.803000pt}%
\definecolor{currentstroke}{rgb}{0.000000,0.000000,0.000000}%
\pgfsetstrokecolor{currentstroke}%
\pgfsetdash{}{0pt}%
\pgfsys@defobject{currentmarker}{\pgfqpoint{0.000000in}{-0.048611in}}{\pgfqpoint{0.000000in}{0.000000in}}{%
\pgfpathmoveto{\pgfqpoint{0.000000in}{0.000000in}}%
\pgfpathlineto{\pgfqpoint{0.000000in}{-0.048611in}}%
\pgfusepath{stroke,fill}%
}%
\begin{pgfscope}%
\pgfsys@transformshift{2.885671in}{0.866167in}%
\pgfsys@useobject{currentmarker}{}%
\end{pgfscope}%
\end{pgfscope}%
\begin{pgfscope}%
\definecolor{textcolor}{rgb}{0.000000,0.000000,0.000000}%
\pgfsetstrokecolor{textcolor}%
\pgfsetfillcolor{textcolor}%
\pgftext[x=2.923988in, y=0.430404in, left, base,rotate=90.000000]{\color{textcolor}\sffamily\fontsize{10.000000}{12.000000}\selectfont book}%
\end{pgfscope}%
\begin{pgfscope}%
\pgfsetbuttcap%
\pgfsetroundjoin%
\definecolor{currentfill}{rgb}{0.000000,0.000000,0.000000}%
\pgfsetfillcolor{currentfill}%
\pgfsetlinewidth{0.803000pt}%
\definecolor{currentstroke}{rgb}{0.000000,0.000000,0.000000}%
\pgfsetstrokecolor{currentstroke}%
\pgfsetdash{}{0pt}%
\pgfsys@defobject{currentmarker}{\pgfqpoint{0.000000in}{-0.048611in}}{\pgfqpoint{0.000000in}{0.000000in}}{%
\pgfpathmoveto{\pgfqpoint{0.000000in}{0.000000in}}%
\pgfpathlineto{\pgfqpoint{0.000000in}{-0.048611in}}%
\pgfusepath{stroke,fill}%
}%
\begin{pgfscope}%
\pgfsys@transformshift{2.998965in}{0.866167in}%
\pgfsys@useobject{currentmarker}{}%
\end{pgfscope}%
\end{pgfscope}%
\begin{pgfscope}%
\definecolor{textcolor}{rgb}{0.000000,0.000000,0.000000}%
\pgfsetstrokecolor{textcolor}%
\pgfsetfillcolor{textcolor}%
\pgftext[x=3.037282in, y=0.281274in, left, base,rotate=90.000000]{\color{textcolor}\sffamily\fontsize{10.000000}{12.000000}\selectfont curtain}%
\end{pgfscope}%
\begin{pgfscope}%
\pgfsetbuttcap%
\pgfsetroundjoin%
\definecolor{currentfill}{rgb}{0.000000,0.000000,0.000000}%
\pgfsetfillcolor{currentfill}%
\pgfsetlinewidth{0.803000pt}%
\definecolor{currentstroke}{rgb}{0.000000,0.000000,0.000000}%
\pgfsetstrokecolor{currentstroke}%
\pgfsetdash{}{0pt}%
\pgfsys@defobject{currentmarker}{\pgfqpoint{0.000000in}{-0.048611in}}{\pgfqpoint{0.000000in}{0.000000in}}{%
\pgfpathmoveto{\pgfqpoint{0.000000in}{0.000000in}}%
\pgfpathlineto{\pgfqpoint{0.000000in}{-0.048611in}}%
\pgfusepath{stroke,fill}%
}%
\begin{pgfscope}%
\pgfsys@transformshift{3.112259in}{0.866167in}%
\pgfsys@useobject{currentmarker}{}%
\end{pgfscope}%
\end{pgfscope}%
\begin{pgfscope}%
\definecolor{textcolor}{rgb}{0.000000,0.000000,0.000000}%
\pgfsetstrokecolor{textcolor}%
\pgfsetfillcolor{textcolor}%
\pgftext[x=3.150575in, y=0.315318in, left, base,rotate=90.000000]{\color{textcolor}\sffamily\fontsize{10.000000}{12.000000}\selectfont plastic}%
\end{pgfscope}%
\begin{pgfscope}%
\pgfsetbuttcap%
\pgfsetroundjoin%
\definecolor{currentfill}{rgb}{0.000000,0.000000,0.000000}%
\pgfsetfillcolor{currentfill}%
\pgfsetlinewidth{0.803000pt}%
\definecolor{currentstroke}{rgb}{0.000000,0.000000,0.000000}%
\pgfsetstrokecolor{currentstroke}%
\pgfsetdash{}{0pt}%
\pgfsys@defobject{currentmarker}{\pgfqpoint{0.000000in}{-0.048611in}}{\pgfqpoint{0.000000in}{0.000000in}}{%
\pgfpathmoveto{\pgfqpoint{0.000000in}{0.000000in}}%
\pgfpathlineto{\pgfqpoint{0.000000in}{-0.048611in}}%
\pgfusepath{stroke,fill}%
}%
\begin{pgfscope}%
\pgfsys@transformshift{3.225553in}{0.866167in}%
\pgfsys@useobject{currentmarker}{}%
\end{pgfscope}%
\end{pgfscope}%
\begin{pgfscope}%
\definecolor{textcolor}{rgb}{0.000000,0.000000,0.000000}%
\pgfsetstrokecolor{textcolor}%
\pgfsetfillcolor{textcolor}%
\pgftext[x=3.263869in, y=0.370657in, left, base,rotate=90.000000]{\color{textcolor}\sffamily\fontsize{10.000000}{12.000000}\selectfont duvet}%
\end{pgfscope}%
\begin{pgfscope}%
\pgfsetbuttcap%
\pgfsetroundjoin%
\definecolor{currentfill}{rgb}{0.000000,0.000000,0.000000}%
\pgfsetfillcolor{currentfill}%
\pgfsetlinewidth{0.803000pt}%
\definecolor{currentstroke}{rgb}{0.000000,0.000000,0.000000}%
\pgfsetstrokecolor{currentstroke}%
\pgfsetdash{}{0pt}%
\pgfsys@defobject{currentmarker}{\pgfqpoint{0.000000in}{-0.048611in}}{\pgfqpoint{0.000000in}{0.000000in}}{%
\pgfpathmoveto{\pgfqpoint{0.000000in}{0.000000in}}%
\pgfpathlineto{\pgfqpoint{0.000000in}{-0.048611in}}%
\pgfusepath{stroke,fill}%
}%
\begin{pgfscope}%
\pgfsys@transformshift{3.338846in}{0.866167in}%
\pgfsys@useobject{currentmarker}{}%
\end{pgfscope}%
\end{pgfscope}%
\begin{pgfscope}%
\definecolor{textcolor}{rgb}{0.000000,0.000000,0.000000}%
\pgfsetstrokecolor{textcolor}%
\pgfsetfillcolor{textcolor}%
\pgftext[x=3.377163in, y=0.435626in, left, base,rotate=90.000000]{\color{textcolor}\sffamily\fontsize{10.000000}{12.000000}\selectfont shelf}%
\end{pgfscope}%
\begin{pgfscope}%
\pgfsetbuttcap%
\pgfsetroundjoin%
\definecolor{currentfill}{rgb}{0.000000,0.000000,0.000000}%
\pgfsetfillcolor{currentfill}%
\pgfsetlinewidth{0.803000pt}%
\definecolor{currentstroke}{rgb}{0.000000,0.000000,0.000000}%
\pgfsetstrokecolor{currentstroke}%
\pgfsetdash{}{0pt}%
\pgfsys@defobject{currentmarker}{\pgfqpoint{0.000000in}{-0.048611in}}{\pgfqpoint{0.000000in}{0.000000in}}{%
\pgfpathmoveto{\pgfqpoint{0.000000in}{0.000000in}}%
\pgfpathlineto{\pgfqpoint{0.000000in}{-0.048611in}}%
\pgfusepath{stroke,fill}%
}%
\begin{pgfscope}%
\pgfsys@transformshift{3.452140in}{0.866167in}%
\pgfsys@useobject{currentmarker}{}%
\end{pgfscope}%
\end{pgfscope}%
\begin{pgfscope}%
\definecolor{textcolor}{rgb}{0.000000,0.000000,0.000000}%
\pgfsetstrokecolor{textcolor}%
\pgfsetfillcolor{textcolor}%
\pgftext[x=3.490457in, y=0.214678in, left, base,rotate=90.000000]{\color{textcolor}\sffamily\fontsize{10.000000}{12.000000}\selectfont ceramic}%
\end{pgfscope}%
\begin{pgfscope}%
\pgfsetbuttcap%
\pgfsetroundjoin%
\definecolor{currentfill}{rgb}{0.000000,0.000000,0.000000}%
\pgfsetfillcolor{currentfill}%
\pgfsetlinewidth{0.803000pt}%
\definecolor{currentstroke}{rgb}{0.000000,0.000000,0.000000}%
\pgfsetstrokecolor{currentstroke}%
\pgfsetdash{}{0pt}%
\pgfsys@defobject{currentmarker}{\pgfqpoint{0.000000in}{-0.048611in}}{\pgfqpoint{0.000000in}{0.000000in}}{%
\pgfpathmoveto{\pgfqpoint{0.000000in}{0.000000in}}%
\pgfpathlineto{\pgfqpoint{0.000000in}{-0.048611in}}%
\pgfusepath{stroke,fill}%
}%
\begin{pgfscope}%
\pgfsys@transformshift{3.565434in}{0.866167in}%
\pgfsys@useobject{currentmarker}{}%
\end{pgfscope}%
\end{pgfscope}%
\begin{pgfscope}%
\definecolor{textcolor}{rgb}{0.000000,0.000000,0.000000}%
\pgfsetstrokecolor{textcolor}%
\pgfsetfillcolor{textcolor}%
\pgftext[x=3.603750in, y=0.421791in, left, base,rotate=90.000000]{\color{textcolor}\sffamily\fontsize{10.000000}{12.000000}\selectfont lamp}%
\end{pgfscope}%
\begin{pgfscope}%
\pgfsetbuttcap%
\pgfsetroundjoin%
\definecolor{currentfill}{rgb}{0.000000,0.000000,0.000000}%
\pgfsetfillcolor{currentfill}%
\pgfsetlinewidth{0.803000pt}%
\definecolor{currentstroke}{rgb}{0.000000,0.000000,0.000000}%
\pgfsetstrokecolor{currentstroke}%
\pgfsetdash{}{0pt}%
\pgfsys@defobject{currentmarker}{\pgfqpoint{0.000000in}{-0.048611in}}{\pgfqpoint{0.000000in}{0.000000in}}{%
\pgfpathmoveto{\pgfqpoint{0.000000in}{0.000000in}}%
\pgfpathlineto{\pgfqpoint{0.000000in}{-0.048611in}}%
\pgfusepath{stroke,fill}%
}%
\begin{pgfscope}%
\pgfsys@transformshift{3.678728in}{0.866167in}%
\pgfsys@useobject{currentmarker}{}%
\end{pgfscope}%
\end{pgfscope}%
\begin{pgfscope}%
\definecolor{textcolor}{rgb}{0.000000,0.000000,0.000000}%
\pgfsetstrokecolor{textcolor}%
\pgfsetfillcolor{textcolor}%
\pgftext[x=3.717044in, y=0.414602in, left, base,rotate=90.000000]{\color{textcolor}\sffamily\fontsize{10.000000}{12.000000}\selectfont plant}%
\end{pgfscope}%
\begin{pgfscope}%
\pgfsetbuttcap%
\pgfsetroundjoin%
\definecolor{currentfill}{rgb}{0.000000,0.000000,0.000000}%
\pgfsetfillcolor{currentfill}%
\pgfsetlinewidth{0.803000pt}%
\definecolor{currentstroke}{rgb}{0.000000,0.000000,0.000000}%
\pgfsetstrokecolor{currentstroke}%
\pgfsetdash{}{0pt}%
\pgfsys@defobject{currentmarker}{\pgfqpoint{0.000000in}{-0.048611in}}{\pgfqpoint{0.000000in}{0.000000in}}{%
\pgfpathmoveto{\pgfqpoint{0.000000in}{0.000000in}}%
\pgfpathlineto{\pgfqpoint{0.000000in}{-0.048611in}}%
\pgfusepath{stroke,fill}%
}%
\begin{pgfscope}%
\pgfsys@transformshift{3.792021in}{0.866167in}%
\pgfsys@useobject{currentmarker}{}%
\end{pgfscope}%
\end{pgfscope}%
\begin{pgfscope}%
\definecolor{textcolor}{rgb}{0.000000,0.000000,0.000000}%
\pgfsetstrokecolor{textcolor}%
\pgfsetfillcolor{textcolor}%
\pgftext[x=3.830338in, y=0.242008in, left, base,rotate=90.000000]{\color{textcolor}\sffamily\fontsize{10.000000}{12.000000}\selectfont window}%
\end{pgfscope}%
\begin{pgfscope}%
\pgfsetbuttcap%
\pgfsetroundjoin%
\definecolor{currentfill}{rgb}{0.000000,0.000000,0.000000}%
\pgfsetfillcolor{currentfill}%
\pgfsetlinewidth{0.803000pt}%
\definecolor{currentstroke}{rgb}{0.000000,0.000000,0.000000}%
\pgfsetstrokecolor{currentstroke}%
\pgfsetdash{}{0pt}%
\pgfsys@defobject{currentmarker}{\pgfqpoint{0.000000in}{-0.048611in}}{\pgfqpoint{0.000000in}{0.000000in}}{%
\pgfpathmoveto{\pgfqpoint{0.000000in}{0.000000in}}%
\pgfpathlineto{\pgfqpoint{0.000000in}{-0.048611in}}%
\pgfusepath{stroke,fill}%
}%
\begin{pgfscope}%
\pgfsys@transformshift{3.905315in}{0.866167in}%
\pgfsys@useobject{currentmarker}{}%
\end{pgfscope}%
\end{pgfscope}%
\begin{pgfscope}%
\definecolor{textcolor}{rgb}{0.000000,0.000000,0.000000}%
\pgfsetstrokecolor{textcolor}%
\pgfsetfillcolor{textcolor}%
\pgftext[x=3.943632in, y=0.124753in, left, base,rotate=90.000000]{\color{textcolor}\sffamily\fontsize{10.000000}{12.000000}\selectfont keyboard}%
\end{pgfscope}%
\begin{pgfscope}%
\pgfsetbuttcap%
\pgfsetroundjoin%
\definecolor{currentfill}{rgb}{0.000000,0.000000,0.000000}%
\pgfsetfillcolor{currentfill}%
\pgfsetlinewidth{0.803000pt}%
\definecolor{currentstroke}{rgb}{0.000000,0.000000,0.000000}%
\pgfsetstrokecolor{currentstroke}%
\pgfsetdash{}{0pt}%
\pgfsys@defobject{currentmarker}{\pgfqpoint{0.000000in}{-0.048611in}}{\pgfqpoint{0.000000in}{0.000000in}}{%
\pgfpathmoveto{\pgfqpoint{0.000000in}{0.000000in}}%
\pgfpathlineto{\pgfqpoint{0.000000in}{-0.048611in}}%
\pgfusepath{stroke,fill}%
}%
\begin{pgfscope}%
\pgfsys@transformshift{4.018609in}{0.866167in}%
\pgfsys@useobject{currentmarker}{}%
\end{pgfscope}%
\end{pgfscope}%
\begin{pgfscope}%
\definecolor{textcolor}{rgb}{0.000000,0.000000,0.000000}%
\pgfsetstrokecolor{textcolor}%
\pgfsetfillcolor{textcolor}%
\pgftext[x=4.056925in, y=0.282427in, left, base,rotate=90.000000]{\color{textcolor}\sffamily\fontsize{10.000000}{12.000000}\selectfont drawer}%
\end{pgfscope}%
\begin{pgfscope}%
\pgfsetbuttcap%
\pgfsetroundjoin%
\definecolor{currentfill}{rgb}{0.000000,0.000000,0.000000}%
\pgfsetfillcolor{currentfill}%
\pgfsetlinewidth{0.803000pt}%
\definecolor{currentstroke}{rgb}{0.000000,0.000000,0.000000}%
\pgfsetstrokecolor{currentstroke}%
\pgfsetdash{}{0pt}%
\pgfsys@defobject{currentmarker}{\pgfqpoint{0.000000in}{-0.048611in}}{\pgfqpoint{0.000000in}{0.000000in}}{%
\pgfpathmoveto{\pgfqpoint{0.000000in}{0.000000in}}%
\pgfpathlineto{\pgfqpoint{0.000000in}{-0.048611in}}%
\pgfusepath{stroke,fill}%
}%
\begin{pgfscope}%
\pgfsys@transformshift{4.131902in}{0.866167in}%
\pgfsys@useobject{currentmarker}{}%
\end{pgfscope}%
\end{pgfscope}%
\begin{pgfscope}%
\definecolor{textcolor}{rgb}{0.000000,0.000000,0.000000}%
\pgfsetstrokecolor{textcolor}%
\pgfsetfillcolor{textcolor}%
\pgftext[x=4.170219in, y=0.362587in, left, base,rotate=90.000000]{\color{textcolor}\sffamily\fontsize{10.000000}{12.000000}\selectfont fridge}%
\end{pgfscope}%
\begin{pgfscope}%
\pgfsetbuttcap%
\pgfsetroundjoin%
\definecolor{currentfill}{rgb}{0.000000,0.000000,0.000000}%
\pgfsetfillcolor{currentfill}%
\pgfsetlinewidth{0.803000pt}%
\definecolor{currentstroke}{rgb}{0.000000,0.000000,0.000000}%
\pgfsetstrokecolor{currentstroke}%
\pgfsetdash{}{0pt}%
\pgfsys@defobject{currentmarker}{\pgfqpoint{0.000000in}{-0.048611in}}{\pgfqpoint{0.000000in}{0.000000in}}{%
\pgfpathmoveto{\pgfqpoint{0.000000in}{0.000000in}}%
\pgfpathlineto{\pgfqpoint{0.000000in}{-0.048611in}}%
\pgfusepath{stroke,fill}%
}%
\begin{pgfscope}%
\pgfsys@transformshift{4.245196in}{0.866167in}%
\pgfsys@useobject{currentmarker}{}%
\end{pgfscope}%
\end{pgfscope}%
\begin{pgfscope}%
\definecolor{textcolor}{rgb}{0.000000,0.000000,0.000000}%
\pgfsetstrokecolor{textcolor}%
\pgfsetfillcolor{textcolor}%
\pgftext[x=4.283513in, y=0.453733in, left, base,rotate=90.000000]{\color{textcolor}\sffamily\fontsize{10.000000}{12.000000}\selectfont door}%
\end{pgfscope}%
\begin{pgfscope}%
\pgfsetbuttcap%
\pgfsetroundjoin%
\definecolor{currentfill}{rgb}{0.000000,0.000000,0.000000}%
\pgfsetfillcolor{currentfill}%
\pgfsetlinewidth{0.803000pt}%
\definecolor{currentstroke}{rgb}{0.000000,0.000000,0.000000}%
\pgfsetstrokecolor{currentstroke}%
\pgfsetdash{}{0pt}%
\pgfsys@defobject{currentmarker}{\pgfqpoint{0.000000in}{-0.048611in}}{\pgfqpoint{0.000000in}{0.000000in}}{%
\pgfpathmoveto{\pgfqpoint{0.000000in}{0.000000in}}%
\pgfpathlineto{\pgfqpoint{0.000000in}{-0.048611in}}%
\pgfusepath{stroke,fill}%
}%
\begin{pgfscope}%
\pgfsys@transformshift{4.358490in}{0.866167in}%
\pgfsys@useobject{currentmarker}{}%
\end{pgfscope}%
\end{pgfscope}%
\begin{pgfscope}%
\definecolor{textcolor}{rgb}{0.000000,0.000000,0.000000}%
\pgfsetstrokecolor{textcolor}%
\pgfsetfillcolor{textcolor}%
\pgftext[x=4.396807in, y=0.517887in, left, base,rotate=90.000000]{\color{textcolor}\sffamily\fontsize{10.000000}{12.000000}\selectfont box}%
\end{pgfscope}%
\begin{pgfscope}%
\pgfsetbuttcap%
\pgfsetroundjoin%
\definecolor{currentfill}{rgb}{0.000000,0.000000,0.000000}%
\pgfsetfillcolor{currentfill}%
\pgfsetlinewidth{0.803000pt}%
\definecolor{currentstroke}{rgb}{0.000000,0.000000,0.000000}%
\pgfsetstrokecolor{currentstroke}%
\pgfsetdash{}{0pt}%
\pgfsys@defobject{currentmarker}{\pgfqpoint{0.000000in}{-0.048611in}}{\pgfqpoint{0.000000in}{0.000000in}}{%
\pgfpathmoveto{\pgfqpoint{0.000000in}{0.000000in}}%
\pgfpathlineto{\pgfqpoint{0.000000in}{-0.048611in}}%
\pgfusepath{stroke,fill}%
}%
\begin{pgfscope}%
\pgfsys@transformshift{4.471784in}{0.866167in}%
\pgfsys@useobject{currentmarker}{}%
\end{pgfscope}%
\end{pgfscope}%
\begin{pgfscope}%
\definecolor{textcolor}{rgb}{0.000000,0.000000,0.000000}%
\pgfsetstrokecolor{textcolor}%
\pgfsetfillcolor{textcolor}%
\pgftext[x=4.510100in, y=0.307926in, left, base,rotate=90.000000]{\color{textcolor}\sffamily\fontsize{10.000000}{12.000000}\selectfont basket}%
\end{pgfscope}%
\begin{pgfscope}%
\pgfsetbuttcap%
\pgfsetroundjoin%
\definecolor{currentfill}{rgb}{0.000000,0.000000,0.000000}%
\pgfsetfillcolor{currentfill}%
\pgfsetlinewidth{0.803000pt}%
\definecolor{currentstroke}{rgb}{0.000000,0.000000,0.000000}%
\pgfsetstrokecolor{currentstroke}%
\pgfsetdash{}{0pt}%
\pgfsys@defobject{currentmarker}{\pgfqpoint{0.000000in}{-0.048611in}}{\pgfqpoint{0.000000in}{0.000000in}}{%
\pgfpathmoveto{\pgfqpoint{0.000000in}{0.000000in}}%
\pgfpathlineto{\pgfqpoint{0.000000in}{-0.048611in}}%
\pgfusepath{stroke,fill}%
}%
\begin{pgfscope}%
\pgfsys@transformshift{4.585077in}{0.866167in}%
\pgfsys@useobject{currentmarker}{}%
\end{pgfscope}%
\end{pgfscope}%
\begin{pgfscope}%
\definecolor{textcolor}{rgb}{0.000000,0.000000,0.000000}%
\pgfsetstrokecolor{textcolor}%
\pgfsetfillcolor{textcolor}%
\pgftext[x=4.623394in, y=0.232582in, left, base,rotate=90.000000]{\color{textcolor}\sffamily\fontsize{10.000000}{12.000000}\selectfont cushion}%
\end{pgfscope}%
\begin{pgfscope}%
\pgfsetbuttcap%
\pgfsetroundjoin%
\definecolor{currentfill}{rgb}{0.000000,0.000000,0.000000}%
\pgfsetfillcolor{currentfill}%
\pgfsetlinewidth{0.803000pt}%
\definecolor{currentstroke}{rgb}{0.000000,0.000000,0.000000}%
\pgfsetstrokecolor{currentstroke}%
\pgfsetdash{}{0pt}%
\pgfsys@defobject{currentmarker}{\pgfqpoint{0.000000in}{-0.048611in}}{\pgfqpoint{0.000000in}{0.000000in}}{%
\pgfpathmoveto{\pgfqpoint{0.000000in}{0.000000in}}%
\pgfpathlineto{\pgfqpoint{0.000000in}{-0.048611in}}%
\pgfusepath{stroke,fill}%
}%
\begin{pgfscope}%
\pgfsys@transformshift{4.698371in}{0.866167in}%
\pgfsys@useobject{currentmarker}{}%
\end{pgfscope}%
\end{pgfscope}%
\begin{pgfscope}%
\definecolor{textcolor}{rgb}{0.000000,0.000000,0.000000}%
\pgfsetstrokecolor{textcolor}%
\pgfsetfillcolor{textcolor}%
\pgftext[x=4.736688in, y=0.417179in, left, base,rotate=90.000000]{\color{textcolor}\sffamily\fontsize{10.000000}{12.000000}\selectfont plate}%
\end{pgfscope}%
\begin{pgfscope}%
\pgfsetbuttcap%
\pgfsetroundjoin%
\definecolor{currentfill}{rgb}{0.000000,0.000000,0.000000}%
\pgfsetfillcolor{currentfill}%
\pgfsetlinewidth{0.803000pt}%
\definecolor{currentstroke}{rgb}{0.000000,0.000000,0.000000}%
\pgfsetstrokecolor{currentstroke}%
\pgfsetdash{}{0pt}%
\pgfsys@defobject{currentmarker}{\pgfqpoint{0.000000in}{-0.048611in}}{\pgfqpoint{0.000000in}{0.000000in}}{%
\pgfpathmoveto{\pgfqpoint{0.000000in}{0.000000in}}%
\pgfpathlineto{\pgfqpoint{0.000000in}{-0.048611in}}%
\pgfusepath{stroke,fill}%
}%
\begin{pgfscope}%
\pgfsys@transformshift{4.811665in}{0.866167in}%
\pgfsys@useobject{currentmarker}{}%
\end{pgfscope}%
\end{pgfscope}%
\begin{pgfscope}%
\definecolor{textcolor}{rgb}{0.000000,0.000000,0.000000}%
\pgfsetstrokecolor{textcolor}%
\pgfsetfillcolor{textcolor}%
\pgftext[x=4.849982in, y=0.364960in, left, base,rotate=90.000000]{\color{textcolor}\sffamily\fontsize{10.000000}{12.000000}\selectfont paper}%
\end{pgfscope}%
\begin{pgfscope}%
\pgfsetbuttcap%
\pgfsetroundjoin%
\definecolor{currentfill}{rgb}{0.000000,0.000000,0.000000}%
\pgfsetfillcolor{currentfill}%
\pgfsetlinewidth{0.803000pt}%
\definecolor{currentstroke}{rgb}{0.000000,0.000000,0.000000}%
\pgfsetstrokecolor{currentstroke}%
\pgfsetdash{}{0pt}%
\pgfsys@defobject{currentmarker}{\pgfqpoint{0.000000in}{-0.048611in}}{\pgfqpoint{0.000000in}{0.000000in}}{%
\pgfpathmoveto{\pgfqpoint{0.000000in}{0.000000in}}%
\pgfpathlineto{\pgfqpoint{0.000000in}{-0.048611in}}%
\pgfusepath{stroke,fill}%
}%
\begin{pgfscope}%
\pgfsys@transformshift{4.924959in}{0.866167in}%
\pgfsys@useobject{currentmarker}{}%
\end{pgfscope}%
\end{pgfscope}%
\begin{pgfscope}%
\definecolor{textcolor}{rgb}{0.000000,0.000000,0.000000}%
\pgfsetstrokecolor{textcolor}%
\pgfsetfillcolor{textcolor}%
\pgftext[x=4.963275in, y=0.307248in, left, base,rotate=90.000000]{\color{textcolor}\sffamily\fontsize{10.000000}{12.000000}\selectfont candle}%
\end{pgfscope}%
\begin{pgfscope}%
\pgfsetbuttcap%
\pgfsetroundjoin%
\definecolor{currentfill}{rgb}{0.000000,0.000000,0.000000}%
\pgfsetfillcolor{currentfill}%
\pgfsetlinewidth{0.803000pt}%
\definecolor{currentstroke}{rgb}{0.000000,0.000000,0.000000}%
\pgfsetstrokecolor{currentstroke}%
\pgfsetdash{}{0pt}%
\pgfsys@defobject{currentmarker}{\pgfqpoint{0.000000in}{-0.048611in}}{\pgfqpoint{0.000000in}{0.000000in}}{%
\pgfpathmoveto{\pgfqpoint{0.000000in}{0.000000in}}%
\pgfpathlineto{\pgfqpoint{0.000000in}{-0.048611in}}%
\pgfusepath{stroke,fill}%
}%
\begin{pgfscope}%
\pgfsys@transformshift{5.038252in}{0.866167in}%
\pgfsys@useobject{currentmarker}{}%
\end{pgfscope}%
\end{pgfscope}%
\begin{pgfscope}%
\definecolor{textcolor}{rgb}{0.000000,0.000000,0.000000}%
\pgfsetstrokecolor{textcolor}%
\pgfsetfillcolor{textcolor}%
\pgftext[x=5.076569in, y=0.100000in, left, base,rotate=90.000000]{\color{textcolor}\sffamily\fontsize{10.000000}{12.000000}\selectfont ventilator}%
\end{pgfscope}%
\begin{pgfscope}%
\pgfsetbuttcap%
\pgfsetroundjoin%
\definecolor{currentfill}{rgb}{0.000000,0.000000,0.000000}%
\pgfsetfillcolor{currentfill}%
\pgfsetlinewidth{0.803000pt}%
\definecolor{currentstroke}{rgb}{0.000000,0.000000,0.000000}%
\pgfsetstrokecolor{currentstroke}%
\pgfsetdash{}{0pt}%
\pgfsys@defobject{currentmarker}{\pgfqpoint{0.000000in}{-0.048611in}}{\pgfqpoint{0.000000in}{0.000000in}}{%
\pgfpathmoveto{\pgfqpoint{0.000000in}{0.000000in}}%
\pgfpathlineto{\pgfqpoint{0.000000in}{-0.048611in}}%
\pgfusepath{stroke,fill}%
}%
\begin{pgfscope}%
\pgfsys@transformshift{5.151546in}{0.866167in}%
\pgfsys@useobject{currentmarker}{}%
\end{pgfscope}%
\end{pgfscope}%
\begin{pgfscope}%
\definecolor{textcolor}{rgb}{0.000000,0.000000,0.000000}%
\pgfsetstrokecolor{textcolor}%
\pgfsetfillcolor{textcolor}%
\pgftext[x=5.189863in, y=0.433591in, left, base,rotate=90.000000]{\color{textcolor}\sffamily\fontsize{10.000000}{12.000000}\selectfont stool}%
\end{pgfscope}%
\begin{pgfscope}%
\pgfsetbuttcap%
\pgfsetroundjoin%
\definecolor{currentfill}{rgb}{0.000000,0.000000,0.000000}%
\pgfsetfillcolor{currentfill}%
\pgfsetlinewidth{0.803000pt}%
\definecolor{currentstroke}{rgb}{0.000000,0.000000,0.000000}%
\pgfsetstrokecolor{currentstroke}%
\pgfsetdash{}{0pt}%
\pgfsys@defobject{currentmarker}{\pgfqpoint{-0.048611in}{0.000000in}}{\pgfqpoint{-0.000000in}{0.000000in}}{%
\pgfpathmoveto{\pgfqpoint{-0.000000in}{0.000000in}}%
\pgfpathlineto{\pgfqpoint{-0.048611in}{0.000000in}}%
\pgfusepath{stroke,fill}%
}%
\begin{pgfscope}%
\pgfsys@transformshift{0.462318in}{0.866167in}%
\pgfsys@useobject{currentmarker}{}%
\end{pgfscope}%
\end{pgfscope}%
\begin{pgfscope}%
\definecolor{textcolor}{rgb}{0.000000,0.000000,0.000000}%
\pgfsetstrokecolor{textcolor}%
\pgfsetfillcolor{textcolor}%
\pgftext[x=0.276731in, y=0.813406in, left, base]{\color{textcolor}\sffamily\fontsize{10.000000}{12.000000}\selectfont 0}%
\end{pgfscope}%
\begin{pgfscope}%
\pgfsetbuttcap%
\pgfsetroundjoin%
\definecolor{currentfill}{rgb}{0.000000,0.000000,0.000000}%
\pgfsetfillcolor{currentfill}%
\pgfsetlinewidth{0.803000pt}%
\definecolor{currentstroke}{rgb}{0.000000,0.000000,0.000000}%
\pgfsetstrokecolor{currentstroke}%
\pgfsetdash{}{0pt}%
\pgfsys@defobject{currentmarker}{\pgfqpoint{-0.048611in}{0.000000in}}{\pgfqpoint{-0.000000in}{0.000000in}}{%
\pgfpathmoveto{\pgfqpoint{-0.000000in}{0.000000in}}%
\pgfpathlineto{\pgfqpoint{-0.048611in}{0.000000in}}%
\pgfusepath{stroke,fill}%
}%
\begin{pgfscope}%
\pgfsys@transformshift{0.462318in}{1.591941in}%
\pgfsys@useobject{currentmarker}{}%
\end{pgfscope}%
\end{pgfscope}%
\begin{pgfscope}%
\definecolor{textcolor}{rgb}{0.000000,0.000000,0.000000}%
\pgfsetstrokecolor{textcolor}%
\pgfsetfillcolor{textcolor}%
\pgftext[x=0.188365in, y=1.539179in, left, base]{\color{textcolor}\sffamily\fontsize{10.000000}{12.000000}\selectfont 20}%
\end{pgfscope}%
\begin{pgfscope}%
\pgfsetbuttcap%
\pgfsetroundjoin%
\definecolor{currentfill}{rgb}{0.000000,0.000000,0.000000}%
\pgfsetfillcolor{currentfill}%
\pgfsetlinewidth{0.803000pt}%
\definecolor{currentstroke}{rgb}{0.000000,0.000000,0.000000}%
\pgfsetstrokecolor{currentstroke}%
\pgfsetdash{}{0pt}%
\pgfsys@defobject{currentmarker}{\pgfqpoint{-0.048611in}{0.000000in}}{\pgfqpoint{-0.000000in}{0.000000in}}{%
\pgfpathmoveto{\pgfqpoint{-0.000000in}{0.000000in}}%
\pgfpathlineto{\pgfqpoint{-0.048611in}{0.000000in}}%
\pgfusepath{stroke,fill}%
}%
\begin{pgfscope}%
\pgfsys@transformshift{0.462318in}{2.317714in}%
\pgfsys@useobject{currentmarker}{}%
\end{pgfscope}%
\end{pgfscope}%
\begin{pgfscope}%
\definecolor{textcolor}{rgb}{0.000000,0.000000,0.000000}%
\pgfsetstrokecolor{textcolor}%
\pgfsetfillcolor{textcolor}%
\pgftext[x=0.188365in, y=2.264952in, left, base]{\color{textcolor}\sffamily\fontsize{10.000000}{12.000000}\selectfont 40}%
\end{pgfscope}%
\begin{pgfscope}%
\pgfsetbuttcap%
\pgfsetroundjoin%
\definecolor{currentfill}{rgb}{0.000000,0.000000,0.000000}%
\pgfsetfillcolor{currentfill}%
\pgfsetlinewidth{0.803000pt}%
\definecolor{currentstroke}{rgb}{0.000000,0.000000,0.000000}%
\pgfsetstrokecolor{currentstroke}%
\pgfsetdash{}{0pt}%
\pgfsys@defobject{currentmarker}{\pgfqpoint{-0.048611in}{0.000000in}}{\pgfqpoint{-0.000000in}{0.000000in}}{%
\pgfpathmoveto{\pgfqpoint{-0.000000in}{0.000000in}}%
\pgfpathlineto{\pgfqpoint{-0.048611in}{0.000000in}}%
\pgfusepath{stroke,fill}%
}%
\begin{pgfscope}%
\pgfsys@transformshift{0.462318in}{3.043487in}%
\pgfsys@useobject{currentmarker}{}%
\end{pgfscope}%
\end{pgfscope}%
\begin{pgfscope}%
\definecolor{textcolor}{rgb}{0.000000,0.000000,0.000000}%
\pgfsetstrokecolor{textcolor}%
\pgfsetfillcolor{textcolor}%
\pgftext[x=0.188365in, y=2.990726in, left, base]{\color{textcolor}\sffamily\fontsize{10.000000}{12.000000}\selectfont 60}%
\end{pgfscope}%
\begin{pgfscope}%
\pgfsetbuttcap%
\pgfsetroundjoin%
\definecolor{currentfill}{rgb}{0.000000,0.000000,0.000000}%
\pgfsetfillcolor{currentfill}%
\pgfsetlinewidth{0.803000pt}%
\definecolor{currentstroke}{rgb}{0.000000,0.000000,0.000000}%
\pgfsetstrokecolor{currentstroke}%
\pgfsetdash{}{0pt}%
\pgfsys@defobject{currentmarker}{\pgfqpoint{-0.048611in}{0.000000in}}{\pgfqpoint{-0.000000in}{0.000000in}}{%
\pgfpathmoveto{\pgfqpoint{-0.000000in}{0.000000in}}%
\pgfpathlineto{\pgfqpoint{-0.048611in}{0.000000in}}%
\pgfusepath{stroke,fill}%
}%
\begin{pgfscope}%
\pgfsys@transformshift{0.462318in}{3.769260in}%
\pgfsys@useobject{currentmarker}{}%
\end{pgfscope}%
\end{pgfscope}%
\begin{pgfscope}%
\definecolor{textcolor}{rgb}{0.000000,0.000000,0.000000}%
\pgfsetstrokecolor{textcolor}%
\pgfsetfillcolor{textcolor}%
\pgftext[x=0.188365in, y=3.716499in, left, base]{\color{textcolor}\sffamily\fontsize{10.000000}{12.000000}\selectfont 80}%
\end{pgfscope}%
\begin{pgfscope}%
\pgfsetbuttcap%
\pgfsetroundjoin%
\definecolor{currentfill}{rgb}{0.000000,0.000000,0.000000}%
\pgfsetfillcolor{currentfill}%
\pgfsetlinewidth{0.803000pt}%
\definecolor{currentstroke}{rgb}{0.000000,0.000000,0.000000}%
\pgfsetstrokecolor{currentstroke}%
\pgfsetdash{}{0pt}%
\pgfsys@defobject{currentmarker}{\pgfqpoint{-0.048611in}{0.000000in}}{\pgfqpoint{-0.000000in}{0.000000in}}{%
\pgfpathmoveto{\pgfqpoint{-0.000000in}{0.000000in}}%
\pgfpathlineto{\pgfqpoint{-0.048611in}{0.000000in}}%
\pgfusepath{stroke,fill}%
}%
\begin{pgfscope}%
\pgfsys@transformshift{0.462318in}{4.495033in}%
\pgfsys@useobject{currentmarker}{}%
\end{pgfscope}%
\end{pgfscope}%
\begin{pgfscope}%
\definecolor{textcolor}{rgb}{0.000000,0.000000,0.000000}%
\pgfsetstrokecolor{textcolor}%
\pgfsetfillcolor{textcolor}%
\pgftext[x=0.100000in, y=4.442272in, left, base]{\color{textcolor}\sffamily\fontsize{10.000000}{12.000000}\selectfont 100}%
\end{pgfscope}%
\begin{pgfscope}%
\pgfsetrectcap%
\pgfsetmiterjoin%
\pgfsetlinewidth{0.803000pt}%
\definecolor{currentstroke}{rgb}{0.000000,0.000000,0.000000}%
\pgfsetstrokecolor{currentstroke}%
\pgfsetdash{}{0pt}%
\pgfpathmoveto{\pgfqpoint{0.462318in}{0.866167in}}%
\pgfpathlineto{\pgfqpoint{0.462318in}{4.562167in}}%
\pgfusepath{stroke}%
\end{pgfscope}%
\begin{pgfscope}%
\pgfsetrectcap%
\pgfsetmiterjoin%
\pgfsetlinewidth{0.803000pt}%
\definecolor{currentstroke}{rgb}{0.000000,0.000000,0.000000}%
\pgfsetstrokecolor{currentstroke}%
\pgfsetdash{}{0pt}%
\pgfpathmoveto{\pgfqpoint{5.422318in}{0.866167in}}%
\pgfpathlineto{\pgfqpoint{5.422318in}{4.562167in}}%
\pgfusepath{stroke}%
\end{pgfscope}%
\begin{pgfscope}%
\pgfsetrectcap%
\pgfsetmiterjoin%
\pgfsetlinewidth{0.803000pt}%
\definecolor{currentstroke}{rgb}{0.000000,0.000000,0.000000}%
\pgfsetstrokecolor{currentstroke}%
\pgfsetdash{}{0pt}%
\pgfpathmoveto{\pgfqpoint{0.462318in}{0.866167in}}%
\pgfpathlineto{\pgfqpoint{5.422318in}{0.866167in}}%
\pgfusepath{stroke}%
\end{pgfscope}%
\begin{pgfscope}%
\pgfsetrectcap%
\pgfsetmiterjoin%
\pgfsetlinewidth{0.803000pt}%
\definecolor{currentstroke}{rgb}{0.000000,0.000000,0.000000}%
\pgfsetstrokecolor{currentstroke}%
\pgfsetdash{}{0pt}%
\pgfpathmoveto{\pgfqpoint{0.462318in}{4.562168in}}%
\pgfpathlineto{\pgfqpoint{5.422318in}{4.562168in}}%
\pgfusepath{stroke}%
\end{pgfscope}%
\begin{pgfscope}%
\definecolor{textcolor}{rgb}{0.000000,0.000000,0.000000}%
\pgfsetstrokecolor{textcolor}%
\pgfsetfillcolor{textcolor}%
\pgftext[x=2.942318in,y=4.645501in,,base]{\color{textcolor}\sffamily\fontsize{12.000000}{14.400000}\selectfont Distribution of textures}%
\end{pgfscope}%
\end{pgfpicture}%
\makeatother%
\endgroup%
}
    \caption[Distribution of Textures.]{Distribution of textures used on scenes. The categories of Pix3D(target furniture) have higher number of images.}
    \label{fig:Distribution of textures}
\end{figure}

Randomizing the outdoor environment is important in cases where we have open doors and windows.
Unity provides a wrapper for scenes called \emph{Skyboxes}.
This is spherical texture around the room under observation.
This gives us a varying outdoor environment for the scene with the scenery at the horizon.
We randomize ten open license skyboxes\footnote{https://polyhaven.com/hdris/outdoor} to achieve this.
Samples of the skybox are shown in \autoref{fig:skybox samples}.


\begin{figure}
    \centering
    \includegraphics[width=.4\textwidth, height = .3\textwidth,valign=m]{/Users/apple/OVGU/Thesis/code/3dReconstruction/report/images/implementation/randomisation/skybox_1}
    \includegraphics[width=.4\textwidth, height = .3\textwidth,valign=m]{/Users/apple/OVGU/Thesis/code/3dReconstruction/report/images/implementation/randomisation/skybox_2} \\
    \vspace{0.1cm}
    \includegraphics[width=.4\textwidth, height = .3\textwidth,valign=m]{/Users/apple/OVGU/Thesis/code/3dReconstruction/report/images/implementation/randomisation/skybox_3}
    \includegraphics[width=.4\textwidth, height = .3\textwidth,valign=m]{/Users/apple/OVGU/Thesis/code/3dReconstruction/report/images/implementation/randomisation/skybox_4}\\
    \caption[Samples for Skyboxes.]{Samples for different skyboxes which change the outdoor environment for the scenes. In the figure we see an open window with changing skybox.}
    \label{fig:skybox samples}
\end{figure}



\subsection{Replacing Target Objects}\label{subsec:replacing-target-objects}

To further randomize the scene, the category of target objects is replaced by the object under observation.
When more than one object of the same category is present, we randomize the object to be replaced, further randomizing the captured data.
The target object is scaled such that the least dimension of the target object matches the least dimension of the category object in the scene.
For example, if the length of the category object in the scene is the most petite amount length, width, and height, then the target object is scaled to match this length.
The rescaling makes the target object blend in with the scene.
Samples for replacing a target object from an original scene from Scenenet are shown in \autoref{fig:replace model}.

\begin{figure}
    \centering
    \includegraphics[width=.4\textwidth, height = .3\textwidth,valign=m]{/Users/apple/OVGU/Thesis/code/3dReconstruction/report/images/implementation/randomisation/replace_1-1}
    \includegraphics[width=.4\textwidth, height = .3\textwidth,valign=m]{/Users/apple/OVGU/Thesis/code/3dReconstruction/report/images/implementation/randomisation/replace_1-2} \\
    \vspace{0.1cm}
    \includegraphics[width=.4\textwidth, height = .3\textwidth,valign=m]{/Users/apple/OVGU/Thesis/code/3dReconstruction/report/images/implementation/randomisation/replace_2-1}
    \includegraphics[width=.4\textwidth, height = .3\textwidth,valign=m]{/Users/apple/OVGU/Thesis/code/3dReconstruction/report/images/implementation/randomisation/replace_2-2}\\
    \caption[Samples for Object Replacement]{Samples for object replacement. The Left column shows a scene from SceneNet, while the right column shows an object being replaced in original scene.}
    \label{fig:replace model}
\end{figure}


\subsection{Camera ViewPoints}\label{subsec:camera-viewpoints}

We randomize camera position with some constraints such that we get different orientations of the target object with different backgrounds.
The constraints will include the height of the camera, minimum and maximum distance to the target object.
A random point is selected within this bound for the camera position.
We make sure that the target object is within the camera frame and is visible.
\autoref{fig:Camera viewpoints} shows samples of different camera viewpoints on a constant object and scene.

\begin{figure}
    \centering
    \includegraphics[width=.4\textwidth, height = .3\textwidth,valign=m]{/Users/apple/OVGU/Thesis/code/3dReconstruction/report/images/implementation/randomisation/camera1}
    \includegraphics[width=.4\textwidth, height = .3\textwidth,valign=m]{/Users/apple/OVGU/Thesis/code/3dReconstruction/report/images/implementation/randomisation/camera2}\\
    \vspace{0.1cm}
    \includegraphics[width=.4\textwidth, height = .3\textwidth,valign=m]{/Users/apple/OVGU/Thesis/code/3dReconstruction/report/images/implementation/randomisation/camera3}
    \includegraphics[width=.4\textwidth, height = .3\textwidth,valign=m]{/Users/apple/OVGU/Thesis/code/3dReconstruction/report/images/implementation/randomisation/camera4}\\
    \caption[Samples for Camera ViewPoints.]{Sample images with different camera viewpoints of same object with a constant scene.}
    \label{fig:Camera viewpoints}
\end{figure}

\subsection{Lightings and Shadows}\label{subsec:lightings-and-shadows}

Lighting plays a vital role in photorealism.
The shadows formed with different lighting conditions enhance the photorealism of the images.
Unity offers a wide variety of illumination like global light, which acts like sunlight, and various indoor lighting systems.
Ideally, we should make the luminous objects like lamps, chandeliers, bulbs, etc.,
the source of light for indoor scenes, but we observed that the room does not light up uniformly, making it less photorealistic.
Hence we use some pre-determined lighting settings, discussed further in implementation(Section 4.1).

%\begin{figure}
%    \centering
%    \includegraphics[width=.3\textwidth, height = .3\textwidth,valign=m]{/Users/apple/OVGU/Thesis/code/3dReconstruction/report/images/implementation/randomisation/lighting1}
%    \includegraphics[width=.3\textwidth, height = .3\textwidth,valign=m]{/Users/apple/OVGU/Thesis/code/3dReconstruction/report/images/implementation/randomisation/lighting2}\\
%    \vspace{0.1cm}
%    \includegraphics[width=.3\textwidth, height = .3\textwidth,valign=m]{/Users/apple/OVGU/Thesis/code/3dReconstruction/report/images/implementation/randomisation/lighting3}
%    \includegraphics[width=.3\textwidth, height = .3\textwidth,valign=m]{/Users/apple/OVGU/Thesis/code/3dReconstruction/report/images/implementation/randomisation/lighting4}\\
%    \caption{Sample images with different lighting and shadows conditions}
%    \label{fig:Lighting and shadows}
%\end{figure}

\begin{figure}[ht]
    \centering
    \includegraphics[width=.24\linewidth,valign=m]{/Users/apple/OVGU/Thesis/code/3dReconstruction/report/images/implementation/randomisation/lighting1}
    \includegraphics[width=.24\linewidth,valign=m]{/Users/apple/OVGU/Thesis/code/3dReconstruction/report/images/implementation/randomisation/lighting2}
    \includegraphics[width=.24\linewidth,valign=m]{/Users/apple/OVGU/Thesis/code/3dReconstruction/report/images/implementation/randomisation/lighting3}
    \includegraphics[width=.24\linewidth,valign=m]{/Users/apple/OVGU/Thesis/code/3dReconstruction/report/images/implementation/randomisation/lighting4}\\
    \vspace{0.1cm}
    \includegraphics[width=.24\linewidth,valign=m]{/Users/apple/OVGU/Thesis/code/3dReconstruction/report/images/implementation/randomisation/lighting5}
    \includegraphics[width=.24\linewidth,valign=m]{/Users/apple/OVGU/Thesis/code/3dReconstruction/report/images/implementation/randomisation/lighting6}
    \includegraphics[width=.24\linewidth,valign=m]{/Users/apple/OVGU/Thesis/code/3dReconstruction/report/images/implementation/randomisation/lighting7}
    \includegraphics[width=.24\linewidth,valign=m]{/Users/apple/OVGU/Thesis/code/3dReconstruction/report/images/implementation/randomisation/lighting8}\\
    \caption[Samples for Lightings and Shadows.]{Sample images with different lighting and shadows conditions.First row is samples for light with different intensity and direction. Second row is differnt color for light.}
    \label{fig:Lighting and shadows}
\end{figure}

\section{3D Reconstruction Pipeline: Why Pix2Vox?}\label{sec:3D reconstruction pipeline}
We create a Deep Learning pipeline for processing the 3D reconstruction task.
The backbone of the pipeline is the base model and the dataset being used to train.
In this section, we discuss the model used as a base and the rationale behind its selection.

Pix2Vox has been used as a baseline by most of the research-oriented to 3D reconstruction.
This network is one of the few networks to be tested on the Pix3D dataset.
According to the survey conducted by~\cite{Han2021ImageBased3O}, the performance of Pix2Vox~\cite{Xie_2019}
is significantly higher compared to previous work(\cite{Tulsiani2017,tatarchenko2016multiview,Roth2018,Gwak2018,Johnston2017}), as shown in \autoref{fig:survey on 3d reconstruction}.
At the same time, this comparison was made on the 3D reconstruction of the ShapeNet dataset since Pix3D was not available when previous work was published.
From our survey, only CoReNet~\cite{popov2020corenet} had a slight gain in performance compared to Pix2Vox.
When trained on ShapeNet and tested with Pix3D, CoReNet gave a result of 29.7\% \gls{iou} while Pix2Vox gave a result of 28.8\% \gls{iou}  and Pix2Vox++ a result of 29.2\% \gls{iou}\@.
Since the difference in the performance was not significant, we decided to stick with the baseline model.
Another reason for selecting the Pix2Vox model is that the backbone of the architecture is pre-trained with ImageNet.
Hence, the embeddings generated from this encoder can help visualize the domain space of both Pix3D (real images)  and \gls{free}(synthetic images).
As mentioned above, for Pix2Vox++, the \gls{resnet} is the backbone encoder with 25\% lesser parameters and 5\% lesser inference time than \gls{vgg}\@.
In addition, the author even demonstrated that Pix2vox++ performs 1.5\% better than Pix2Vox.
The architecture of Pix2Vox++ is as shown in \autoref{fig:architectures}(b).
Furthermore, the focus of this thesis is not to check which is the best model to reconstruct the furniture but to check if game engines can produce photorealistic images usable for 3D reconstruction.
Hence the selection of the model was not of utmost importance.
However, since the two architectures are relatable, it would be interesting to compare the results for the 3D Reconstruction task.

%\begin{figure}
%    \centering
%    \includegraphics[width=\textwidth]{/Users/apple/OVGU/Thesis/code/3dReconstruction/report/images/concept/pix2vox}
%    \caption{Network architecture for pix2vox~\cite{Xie_2019}}
%    \label{fig:pix2vox architecture}
%\end{figure}
%
%\begin{figure}
%    \centering
%    \includegraphics[width=\textwidth]{/Users/apple/OVGU/Thesis/code/3dReconstruction/report/images/concept/pix2voxpp}
%    \caption{Network architecture for pix2vox++~\cite{Xie_2020}}
%    \label{fig:pix2voxpp architecture}
%\end{figure}

\begin{figure}[ht]
    \centering
    \includegraphics[width=\textwidth]{/Users/apple/OVGU/Thesis/code/3dReconstruction/report/images/concept/survey}
    \caption[Survey Results for 3D-Reconstruction.]{A survey conducted by~\cite{Han2021ImageBased3O}, demonstrates that Pix2Vox is considerably a good 3D reconstruction model.
    The values are from 3D reconstruction of ShapeNet~\cite{shapenet2015} since Pix3D was not published by then.}
    \label{fig:survey on 3d reconstruction}
\end{figure}


\section{Datasets}\label{sec:datasets}
In this section, we describe the datasets used for the evaluations.
Datasets intend to have variations in domain randomization to check their performance on the 3D reconstruction tasks.

\subsection{Pix3D}\label{subsec:pix3d}
As mentioned in \autoref{subsec:why-pix3d?}, we use a real dataset from~\cite{Sun2018}, a collection of indoor scenes.
The two classes ’misc’ and ’tools’ are eliminated to focus only on furniture.
The total images after the reduction are 9954 with 354 unique models.
The dataset is divided into 70:30, giving us 6814 images from training and 3140 images for validation/test.
We do not have a test set only for this dataset since it is already limited, and the validation set is used as a test set while testing with synthetic data.
Samples are as shown in \autoref{fig:samples for synthetic and real comparison}.

\subsection{\gls{s2rv1}}\label{subsec:gls{free}-version-1}
Version 1 of \gls{free} was created by keeping the models in the center of a default 3D room.
The camera distance was randomized between 1 and 2.5 meters from the model.
The camera viewpoints and textures were randomized.
A total of 70000 images were synthetically generated using the \gls{free} `Single Room pipeline' with 10000 images per category.
Samples are as shown in \autoref{fig:samples for synthetic and real comparison}.

\subsection{\gls{s2rv2}}\label{subsec:gls{free}-version-2}
Version 1 of \gls{free} was created by keeping the models in the center of a default 3D room.
The distance of the camera was randomly chosen between 1 and 2.5 meters from the model.
The camera viewpoints and textures were randomized.
A total of 21000 images were synthetically generated using the \gls{free} `Multi-Object pipeline' with 3000 images per category.
Samples are as shown in \autoref{fig:samples for synthetic and real comparison}.

\begin{figure}[ht]
    \centering
    \begin{tabular}{llll}
        Pix3D & \includegraphics[width=.2\textwidth, height =.19\textwidth]{/Users/apple/OVGU/Thesis/code/3dReconstruction/report/images/evaluation/datasets/pix3d_1} &
        \includegraphics[width=.2\textwidth, height =.19\textwidth]{/Users/apple/OVGU/Thesis/code/3dReconstruction/report/images/evaluation/datasets/pix3d_2} &
        \includegraphics[width=.2\textwidth, height =.19\textwidth]{/Users/apple/OVGU/Thesis/code/3dReconstruction/report/images/evaluation/datasets/pix3d_3}\\

        \gls{s2rv1} & \includegraphics[width=.2\textwidth, height =.19\textwidth]{/Users/apple/OVGU/Thesis/code/3dReconstruction/report/images/evaluation/datasets/s2r_v1_1} &
        \includegraphics[width=.2\textwidth, height=.19\textwidth]{/Users/apple/OVGU/Thesis/code/3dReconstruction/report/images/evaluation/datasets/s2r_v1_2} &
        \includegraphics[width=.2\textwidth, height=.19\textwidth]{/Users/apple/OVGU/Thesis/code/3dReconstruction/report/images/evaluation/datasets/s2r_v1_3}\\

        \gls{s2rv2} & \includegraphics[width=.19\textwidth, height =.2\textwidth]{/Users/apple/OVGU/Thesis/code/3dReconstruction/report/images/evaluation/datasets/s2r_v3_1} &
        \includegraphics[width=.2\textwidth, height=.19\textwidth]{/Users/apple/OVGU/Thesis/code/3dReconstruction/report/images/evaluation/datasets/s2r_v3_2} &
        \includegraphics[width=.2\textwidth, height=.19\textwidth]{/Users/apple/OVGU/Thesis/code/3dReconstruction/report/images/evaluation/datasets/s2r_v3_3}\\

    \end{tabular}
    \caption[Samples for Real and Synthetic Datasets]{Samples of images from real and synthetic datasets.}
    \label{fig:samples for synthetic and real comparison}
\end{figure}

\subsection{\gls{free} Ablation}\label{subsec:s2r:3dfree-ablation}
Pix3D is composed of 3839 chairs which is the maximum among all categories.
To study the effects of domain randomization, we create multiple chair datasets by omitting randomizing factors one at a time and study the model behavior.
We divide the datasets into five categories with different randomization parameters.
The sample images with different randomization are as shown in \autoref{fig:domain_randomisation_for_ablation_study}.

\begin{figure}[!ht]
    \begin{tabular}{llll}
        Textureless & \includegraphics[width=.2\linewidth]{/Users/apple/OVGU/Thesis/code/3dReconstruction/report/images/evaluation/datasets/s2r_textureless_1} &
        \includegraphics[width=.2\linewidth]{/Users/apple/OVGU/Thesis/code/3dReconstruction/report/images/evaluation/datasets/s2r_textureless_2} &
        \includegraphics[width=.2\linewidth]{/Users/apple/OVGU/Thesis/code/3dReconstruction/report/images/evaluation/datasets/s2r_textureless_3}\\

        Textureless with light & \includegraphics[width=.2\linewidth]{/Users/apple/OVGU/Thesis/code/3dReconstruction/report/images/evaluation/datasets/s2r_textureless_light_1} &
        \includegraphics[width=.2\linewidth]{/Users/apple/OVGU/Thesis/code/3dReconstruction/report/images/evaluation/datasets/s2r_textureless_light_2} &
        \includegraphics[width=.2\linewidth]{/Users/apple/OVGU/Thesis/code/3dReconstruction/report/images/evaluation/datasets/s2r_textureless_light_3}\\

        Textured & \includegraphics[width=.2\linewidth]{/Users/apple/OVGU/Thesis/code/3dReconstruction/report/images/evaluation/datasets/s2r_textured_1} &
        \includegraphics[width=.2\linewidth]{/Users/apple/OVGU/Thesis/code/3dReconstruction/report/images/evaluation/datasets/s2r_textured_2} &
        \includegraphics[width=.2\linewidth]{/Users/apple/OVGU/Thesis/code/3dReconstruction/report/images/evaluation/datasets/s2r_textured_3}\\

        Textured with light & \includegraphics[width=.2\linewidth]{/Users/apple/OVGU/Thesis/code/3dReconstruction/report/images/evaluation/datasets/s2r_textured_light_1} &
        \includegraphics[width=.2\linewidth]{/Users/apple/OVGU/Thesis/code/3dReconstruction/report/images/evaluation/datasets/s2r_textured_light_2} &
        \includegraphics[width=.2\linewidth]{/Users/apple/OVGU/Thesis/code/3dReconstruction/report/images/evaluation/datasets/s2r_textured_light_4}\\

        Multi-Object & \includegraphics[width=.2\linewidth]{/Users/apple/OVGU/Thesis/code/3dReconstruction/report/images/evaluation/datasets/s2r_chair_1} &
        \includegraphics[width=.2\linewidth]{/Users/apple/OVGU/Thesis/code/3dReconstruction/report/images/evaluation/datasets/s2r_chair_2} &
        \includegraphics[width=.2\linewidth]{/Users/apple/OVGU/Thesis/code/3dReconstruction/report/images/evaluation/datasets/s2r_chair_3}\\

    \end{tabular}
    \caption[Samples for Datasets Used for Ablastion Study.]{Samples of images used for ablation study on chairs with different parameters of domain randomization.}
    \label{fig:domain_randomisation_for_ablation_study}
\end{figure}

%\begin{figure}
%    \centering
%    \resizebox{\textwidth}{!}{%% Creator: Matplotlib, PGF backend
%%
%% To include the figure in your LaTeX document, write
%%   \input{<filename>.pgf}
%%
%% Make sure the required packages are loaded in your preamble
%%   \usepackage{pgf}
%%
%% Figures using additional raster images can only be included by \input if
%% they are in the same directory as the main LaTeX file. For loading figures
%% from other directories you can use the `import` package
%%   \usepackage{import}
%%
%% and then include the figures with
%%   \import{<path to file>}{<filename>.pgf}
%%
%% Matplotlib used the following preamble
%%   \usepackage{fontspec}
%%   \setmainfont{DejaVuSerif.ttf}[Path=\detokenize{/Users/apple/opt/anaconda3/envs/kaolin/lib/python3.7/site-packages/matplotlib/mpl-data/fonts/ttf/}]
%%   \setsansfont{DejaVuSans.ttf}[Path=\detokenize{/Users/apple/opt/anaconda3/envs/kaolin/lib/python3.7/site-packages/matplotlib/mpl-data/fonts/ttf/}]
%%   \setmonofont{DejaVuSansMono.ttf}[Path=\detokenize{/Users/apple/opt/anaconda3/envs/kaolin/lib/python3.7/site-packages/matplotlib/mpl-data/fonts/ttf/}]
%%
\begingroup%
\makeatletter%
\begin{pgfpicture}%
\pgfpathrectangle{\pgfpointorigin}{\pgfqpoint{12.476162in}{8.341596in}}%
\pgfusepath{use as bounding box, clip}%
\begin{pgfscope}%
\pgfsetbuttcap%
\pgfsetmiterjoin%
\definecolor{currentfill}{rgb}{1.000000,1.000000,1.000000}%
\pgfsetfillcolor{currentfill}%
\pgfsetlinewidth{0.000000pt}%
\definecolor{currentstroke}{rgb}{1.000000,1.000000,1.000000}%
\pgfsetstrokecolor{currentstroke}%
\pgfsetdash{}{0pt}%
\pgfpathmoveto{\pgfqpoint{0.000000in}{0.000000in}}%
\pgfpathlineto{\pgfqpoint{12.476162in}{0.000000in}}%
\pgfpathlineto{\pgfqpoint{12.476162in}{8.341596in}}%
\pgfpathlineto{\pgfqpoint{0.000000in}{8.341596in}}%
\pgfpathclose%
\pgfusepath{fill}%
\end{pgfscope}%
\begin{pgfscope}%
\pgfsetbuttcap%
\pgfsetmiterjoin%
\definecolor{currentfill}{rgb}{1.000000,1.000000,1.000000}%
\pgfsetfillcolor{currentfill}%
\pgfsetlinewidth{0.000000pt}%
\definecolor{currentstroke}{rgb}{0.000000,0.000000,0.000000}%
\pgfsetstrokecolor{currentstroke}%
\pgfsetstrokeopacity{0.000000}%
\pgfsetdash{}{0pt}%
\pgfpathmoveto{\pgfqpoint{0.481978in}{0.331635in}}%
\pgfpathlineto{\pgfqpoint{9.781978in}{0.331635in}}%
\pgfpathlineto{\pgfqpoint{9.781978in}{8.031635in}}%
\pgfpathlineto{\pgfqpoint{0.481978in}{8.031635in}}%
\pgfpathclose%
\pgfusepath{fill}%
\end{pgfscope}%
\begin{pgfscope}%
\pgfpathrectangle{\pgfqpoint{0.481978in}{0.331635in}}{\pgfqpoint{9.300000in}{7.700000in}}%
\pgfusepath{clip}%
\pgfsetbuttcap%
\pgfsetroundjoin%
\definecolor{currentfill}{rgb}{0.631373,0.788235,0.956863}%
\pgfsetfillcolor{currentfill}%
\pgfsetlinewidth{0.481800pt}%
\definecolor{currentstroke}{rgb}{1.000000,1.000000,1.000000}%
\pgfsetstrokecolor{currentstroke}%
\pgfsetdash{}{0pt}%
\pgfpathmoveto{\pgfqpoint{3.350250in}{0.795870in}}%
\pgfpathcurveto{\pgfqpoint{3.361300in}{0.795870in}}{\pgfqpoint{3.371899in}{0.800261in}}{\pgfqpoint{3.379713in}{0.808074in}}%
\pgfpathcurveto{\pgfqpoint{3.387527in}{0.815888in}}{\pgfqpoint{3.391917in}{0.826487in}}{\pgfqpoint{3.391917in}{0.837537in}}%
\pgfpathcurveto{\pgfqpoint{3.391917in}{0.848587in}}{\pgfqpoint{3.387527in}{0.859186in}}{\pgfqpoint{3.379713in}{0.867000in}}%
\pgfpathcurveto{\pgfqpoint{3.371899in}{0.874813in}}{\pgfqpoint{3.361300in}{0.879204in}}{\pgfqpoint{3.350250in}{0.879204in}}%
\pgfpathcurveto{\pgfqpoint{3.339200in}{0.879204in}}{\pgfqpoint{3.328601in}{0.874813in}}{\pgfqpoint{3.320788in}{0.867000in}}%
\pgfpathcurveto{\pgfqpoint{3.312974in}{0.859186in}}{\pgfqpoint{3.308584in}{0.848587in}}{\pgfqpoint{3.308584in}{0.837537in}}%
\pgfpathcurveto{\pgfqpoint{3.308584in}{0.826487in}}{\pgfqpoint{3.312974in}{0.815888in}}{\pgfqpoint{3.320788in}{0.808074in}}%
\pgfpathcurveto{\pgfqpoint{3.328601in}{0.800261in}}{\pgfqpoint{3.339200in}{0.795870in}}{\pgfqpoint{3.350250in}{0.795870in}}%
\pgfpathclose%
\pgfusepath{stroke,fill}%
\end{pgfscope}%
\begin{pgfscope}%
\pgfpathrectangle{\pgfqpoint{0.481978in}{0.331635in}}{\pgfqpoint{9.300000in}{7.700000in}}%
\pgfusepath{clip}%
\pgfsetbuttcap%
\pgfsetroundjoin%
\definecolor{currentfill}{rgb}{0.631373,0.788235,0.956863}%
\pgfsetfillcolor{currentfill}%
\pgfsetlinewidth{0.481800pt}%
\definecolor{currentstroke}{rgb}{1.000000,1.000000,1.000000}%
\pgfsetstrokecolor{currentstroke}%
\pgfsetdash{}{0pt}%
\pgfpathmoveto{\pgfqpoint{3.059115in}{5.220279in}}%
\pgfpathcurveto{\pgfqpoint{3.070165in}{5.220279in}}{\pgfqpoint{3.080764in}{5.224669in}}{\pgfqpoint{3.088578in}{5.232482in}}%
\pgfpathcurveto{\pgfqpoint{3.096392in}{5.240296in}}{\pgfqpoint{3.100782in}{5.250895in}}{\pgfqpoint{3.100782in}{5.261945in}}%
\pgfpathcurveto{\pgfqpoint{3.100782in}{5.272995in}}{\pgfqpoint{3.096392in}{5.283594in}}{\pgfqpoint{3.088578in}{5.291408in}}%
\pgfpathcurveto{\pgfqpoint{3.080764in}{5.299222in}}{\pgfqpoint{3.070165in}{5.303612in}}{\pgfqpoint{3.059115in}{5.303612in}}%
\pgfpathcurveto{\pgfqpoint{3.048065in}{5.303612in}}{\pgfqpoint{3.037466in}{5.299222in}}{\pgfqpoint{3.029652in}{5.291408in}}%
\pgfpathcurveto{\pgfqpoint{3.021839in}{5.283594in}}{\pgfqpoint{3.017448in}{5.272995in}}{\pgfqpoint{3.017448in}{5.261945in}}%
\pgfpathcurveto{\pgfqpoint{3.017448in}{5.250895in}}{\pgfqpoint{3.021839in}{5.240296in}}{\pgfqpoint{3.029652in}{5.232482in}}%
\pgfpathcurveto{\pgfqpoint{3.037466in}{5.224669in}}{\pgfqpoint{3.048065in}{5.220279in}}{\pgfqpoint{3.059115in}{5.220279in}}%
\pgfpathclose%
\pgfusepath{stroke,fill}%
\end{pgfscope}%
\begin{pgfscope}%
\pgfpathrectangle{\pgfqpoint{0.481978in}{0.331635in}}{\pgfqpoint{9.300000in}{7.700000in}}%
\pgfusepath{clip}%
\pgfsetbuttcap%
\pgfsetroundjoin%
\definecolor{currentfill}{rgb}{0.631373,0.788235,0.956863}%
\pgfsetfillcolor{currentfill}%
\pgfsetlinewidth{0.481800pt}%
\definecolor{currentstroke}{rgb}{1.000000,1.000000,1.000000}%
\pgfsetstrokecolor{currentstroke}%
\pgfsetdash{}{0pt}%
\pgfpathmoveto{\pgfqpoint{4.845549in}{3.728687in}}%
\pgfpathcurveto{\pgfqpoint{4.856599in}{3.728687in}}{\pgfqpoint{4.867198in}{3.733077in}}{\pgfqpoint{4.875011in}{3.740891in}}%
\pgfpathcurveto{\pgfqpoint{4.882825in}{3.748705in}}{\pgfqpoint{4.887215in}{3.759304in}}{\pgfqpoint{4.887215in}{3.770354in}}%
\pgfpathcurveto{\pgfqpoint{4.887215in}{3.781404in}}{\pgfqpoint{4.882825in}{3.792003in}}{\pgfqpoint{4.875011in}{3.799817in}}%
\pgfpathcurveto{\pgfqpoint{4.867198in}{3.807630in}}{\pgfqpoint{4.856599in}{3.812021in}}{\pgfqpoint{4.845549in}{3.812021in}}%
\pgfpathcurveto{\pgfqpoint{4.834499in}{3.812021in}}{\pgfqpoint{4.823900in}{3.807630in}}{\pgfqpoint{4.816086in}{3.799817in}}%
\pgfpathcurveto{\pgfqpoint{4.808272in}{3.792003in}}{\pgfqpoint{4.803882in}{3.781404in}}{\pgfqpoint{4.803882in}{3.770354in}}%
\pgfpathcurveto{\pgfqpoint{4.803882in}{3.759304in}}{\pgfqpoint{4.808272in}{3.748705in}}{\pgfqpoint{4.816086in}{3.740891in}}%
\pgfpathcurveto{\pgfqpoint{4.823900in}{3.733077in}}{\pgfqpoint{4.834499in}{3.728687in}}{\pgfqpoint{4.845549in}{3.728687in}}%
\pgfpathclose%
\pgfusepath{stroke,fill}%
\end{pgfscope}%
\begin{pgfscope}%
\pgfpathrectangle{\pgfqpoint{0.481978in}{0.331635in}}{\pgfqpoint{9.300000in}{7.700000in}}%
\pgfusepath{clip}%
\pgfsetbuttcap%
\pgfsetroundjoin%
\definecolor{currentfill}{rgb}{0.631373,0.788235,0.956863}%
\pgfsetfillcolor{currentfill}%
\pgfsetlinewidth{0.481800pt}%
\definecolor{currentstroke}{rgb}{1.000000,1.000000,1.000000}%
\pgfsetstrokecolor{currentstroke}%
\pgfsetdash{}{0pt}%
\pgfpathmoveto{\pgfqpoint{1.304480in}{3.048221in}}%
\pgfpathcurveto{\pgfqpoint{1.315530in}{3.048221in}}{\pgfqpoint{1.326129in}{3.052611in}}{\pgfqpoint{1.333942in}{3.060424in}}%
\pgfpathcurveto{\pgfqpoint{1.341756in}{3.068238in}}{\pgfqpoint{1.346146in}{3.078837in}}{\pgfqpoint{1.346146in}{3.089887in}}%
\pgfpathcurveto{\pgfqpoint{1.346146in}{3.100937in}}{\pgfqpoint{1.341756in}{3.111536in}}{\pgfqpoint{1.333942in}{3.119350in}}%
\pgfpathcurveto{\pgfqpoint{1.326129in}{3.127164in}}{\pgfqpoint{1.315530in}{3.131554in}}{\pgfqpoint{1.304480in}{3.131554in}}%
\pgfpathcurveto{\pgfqpoint{1.293429in}{3.131554in}}{\pgfqpoint{1.282830in}{3.127164in}}{\pgfqpoint{1.275017in}{3.119350in}}%
\pgfpathcurveto{\pgfqpoint{1.267203in}{3.111536in}}{\pgfqpoint{1.262813in}{3.100937in}}{\pgfqpoint{1.262813in}{3.089887in}}%
\pgfpathcurveto{\pgfqpoint{1.262813in}{3.078837in}}{\pgfqpoint{1.267203in}{3.068238in}}{\pgfqpoint{1.275017in}{3.060424in}}%
\pgfpathcurveto{\pgfqpoint{1.282830in}{3.052611in}}{\pgfqpoint{1.293429in}{3.048221in}}{\pgfqpoint{1.304480in}{3.048221in}}%
\pgfpathclose%
\pgfusepath{stroke,fill}%
\end{pgfscope}%
\begin{pgfscope}%
\pgfpathrectangle{\pgfqpoint{0.481978in}{0.331635in}}{\pgfqpoint{9.300000in}{7.700000in}}%
\pgfusepath{clip}%
\pgfsetbuttcap%
\pgfsetroundjoin%
\definecolor{currentfill}{rgb}{0.631373,0.788235,0.956863}%
\pgfsetfillcolor{currentfill}%
\pgfsetlinewidth{0.481800pt}%
\definecolor{currentstroke}{rgb}{1.000000,1.000000,1.000000}%
\pgfsetstrokecolor{currentstroke}%
\pgfsetdash{}{0pt}%
\pgfpathmoveto{\pgfqpoint{4.583472in}{2.392940in}}%
\pgfpathcurveto{\pgfqpoint{4.594522in}{2.392940in}}{\pgfqpoint{4.605121in}{2.397330in}}{\pgfqpoint{4.612935in}{2.405144in}}%
\pgfpathcurveto{\pgfqpoint{4.620748in}{2.412957in}}{\pgfqpoint{4.625139in}{2.423556in}}{\pgfqpoint{4.625139in}{2.434606in}}%
\pgfpathcurveto{\pgfqpoint{4.625139in}{2.445657in}}{\pgfqpoint{4.620748in}{2.456256in}}{\pgfqpoint{4.612935in}{2.464069in}}%
\pgfpathcurveto{\pgfqpoint{4.605121in}{2.471883in}}{\pgfqpoint{4.594522in}{2.476273in}}{\pgfqpoint{4.583472in}{2.476273in}}%
\pgfpathcurveto{\pgfqpoint{4.572422in}{2.476273in}}{\pgfqpoint{4.561823in}{2.471883in}}{\pgfqpoint{4.554009in}{2.464069in}}%
\pgfpathcurveto{\pgfqpoint{4.546196in}{2.456256in}}{\pgfqpoint{4.541805in}{2.445657in}}{\pgfqpoint{4.541805in}{2.434606in}}%
\pgfpathcurveto{\pgfqpoint{4.541805in}{2.423556in}}{\pgfqpoint{4.546196in}{2.412957in}}{\pgfqpoint{4.554009in}{2.405144in}}%
\pgfpathcurveto{\pgfqpoint{4.561823in}{2.397330in}}{\pgfqpoint{4.572422in}{2.392940in}}{\pgfqpoint{4.583472in}{2.392940in}}%
\pgfpathclose%
\pgfusepath{stroke,fill}%
\end{pgfscope}%
\begin{pgfscope}%
\pgfpathrectangle{\pgfqpoint{0.481978in}{0.331635in}}{\pgfqpoint{9.300000in}{7.700000in}}%
\pgfusepath{clip}%
\pgfsetbuttcap%
\pgfsetroundjoin%
\definecolor{currentfill}{rgb}{0.631373,0.788235,0.956863}%
\pgfsetfillcolor{currentfill}%
\pgfsetlinewidth{0.481800pt}%
\definecolor{currentstroke}{rgb}{1.000000,1.000000,1.000000}%
\pgfsetstrokecolor{currentstroke}%
\pgfsetdash{}{0pt}%
\pgfpathmoveto{\pgfqpoint{5.509106in}{5.260891in}}%
\pgfpathcurveto{\pgfqpoint{5.520156in}{5.260891in}}{\pgfqpoint{5.530755in}{5.265281in}}{\pgfqpoint{5.538568in}{5.273094in}}%
\pgfpathcurveto{\pgfqpoint{5.546382in}{5.280908in}}{\pgfqpoint{5.550772in}{5.291507in}}{\pgfqpoint{5.550772in}{5.302557in}}%
\pgfpathcurveto{\pgfqpoint{5.550772in}{5.313607in}}{\pgfqpoint{5.546382in}{5.324206in}}{\pgfqpoint{5.538568in}{5.332020in}}%
\pgfpathcurveto{\pgfqpoint{5.530755in}{5.339834in}}{\pgfqpoint{5.520156in}{5.344224in}}{\pgfqpoint{5.509106in}{5.344224in}}%
\pgfpathcurveto{\pgfqpoint{5.498055in}{5.344224in}}{\pgfqpoint{5.487456in}{5.339834in}}{\pgfqpoint{5.479643in}{5.332020in}}%
\pgfpathcurveto{\pgfqpoint{5.471829in}{5.324206in}}{\pgfqpoint{5.467439in}{5.313607in}}{\pgfqpoint{5.467439in}{5.302557in}}%
\pgfpathcurveto{\pgfqpoint{5.467439in}{5.291507in}}{\pgfqpoint{5.471829in}{5.280908in}}{\pgfqpoint{5.479643in}{5.273094in}}%
\pgfpathcurveto{\pgfqpoint{5.487456in}{5.265281in}}{\pgfqpoint{5.498055in}{5.260891in}}{\pgfqpoint{5.509106in}{5.260891in}}%
\pgfpathclose%
\pgfusepath{stroke,fill}%
\end{pgfscope}%
\begin{pgfscope}%
\pgfpathrectangle{\pgfqpoint{0.481978in}{0.331635in}}{\pgfqpoint{9.300000in}{7.700000in}}%
\pgfusepath{clip}%
\pgfsetbuttcap%
\pgfsetroundjoin%
\definecolor{currentfill}{rgb}{0.631373,0.788235,0.956863}%
\pgfsetfillcolor{currentfill}%
\pgfsetlinewidth{0.481800pt}%
\definecolor{currentstroke}{rgb}{1.000000,1.000000,1.000000}%
\pgfsetstrokecolor{currentstroke}%
\pgfsetdash{}{0pt}%
\pgfpathmoveto{\pgfqpoint{1.270538in}{3.021766in}}%
\pgfpathcurveto{\pgfqpoint{1.281588in}{3.021766in}}{\pgfqpoint{1.292187in}{3.026156in}}{\pgfqpoint{1.300001in}{3.033970in}}%
\pgfpathcurveto{\pgfqpoint{1.307815in}{3.041783in}}{\pgfqpoint{1.312205in}{3.052382in}}{\pgfqpoint{1.312205in}{3.063432in}}%
\pgfpathcurveto{\pgfqpoint{1.312205in}{3.074483in}}{\pgfqpoint{1.307815in}{3.085082in}}{\pgfqpoint{1.300001in}{3.092895in}}%
\pgfpathcurveto{\pgfqpoint{1.292187in}{3.100709in}}{\pgfqpoint{1.281588in}{3.105099in}}{\pgfqpoint{1.270538in}{3.105099in}}%
\pgfpathcurveto{\pgfqpoint{1.259488in}{3.105099in}}{\pgfqpoint{1.248889in}{3.100709in}}{\pgfqpoint{1.241075in}{3.092895in}}%
\pgfpathcurveto{\pgfqpoint{1.233262in}{3.085082in}}{\pgfqpoint{1.228872in}{3.074483in}}{\pgfqpoint{1.228872in}{3.063432in}}%
\pgfpathcurveto{\pgfqpoint{1.228872in}{3.052382in}}{\pgfqpoint{1.233262in}{3.041783in}}{\pgfqpoint{1.241075in}{3.033970in}}%
\pgfpathcurveto{\pgfqpoint{1.248889in}{3.026156in}}{\pgfqpoint{1.259488in}{3.021766in}}{\pgfqpoint{1.270538in}{3.021766in}}%
\pgfpathclose%
\pgfusepath{stroke,fill}%
\end{pgfscope}%
\begin{pgfscope}%
\pgfpathrectangle{\pgfqpoint{0.481978in}{0.331635in}}{\pgfqpoint{9.300000in}{7.700000in}}%
\pgfusepath{clip}%
\pgfsetbuttcap%
\pgfsetroundjoin%
\definecolor{currentfill}{rgb}{0.631373,0.788235,0.956863}%
\pgfsetfillcolor{currentfill}%
\pgfsetlinewidth{0.481800pt}%
\definecolor{currentstroke}{rgb}{1.000000,1.000000,1.000000}%
\pgfsetstrokecolor{currentstroke}%
\pgfsetdash{}{0pt}%
\pgfpathmoveto{\pgfqpoint{4.233419in}{2.502329in}}%
\pgfpathcurveto{\pgfqpoint{4.244470in}{2.502329in}}{\pgfqpoint{4.255069in}{2.506720in}}{\pgfqpoint{4.262882in}{2.514533in}}%
\pgfpathcurveto{\pgfqpoint{4.270696in}{2.522347in}}{\pgfqpoint{4.275086in}{2.532946in}}{\pgfqpoint{4.275086in}{2.543996in}}%
\pgfpathcurveto{\pgfqpoint{4.275086in}{2.555046in}}{\pgfqpoint{4.270696in}{2.565645in}}{\pgfqpoint{4.262882in}{2.573459in}}%
\pgfpathcurveto{\pgfqpoint{4.255069in}{2.581272in}}{\pgfqpoint{4.244470in}{2.585663in}}{\pgfqpoint{4.233419in}{2.585663in}}%
\pgfpathcurveto{\pgfqpoint{4.222369in}{2.585663in}}{\pgfqpoint{4.211770in}{2.581272in}}{\pgfqpoint{4.203957in}{2.573459in}}%
\pgfpathcurveto{\pgfqpoint{4.196143in}{2.565645in}}{\pgfqpoint{4.191753in}{2.555046in}}{\pgfqpoint{4.191753in}{2.543996in}}%
\pgfpathcurveto{\pgfqpoint{4.191753in}{2.532946in}}{\pgfqpoint{4.196143in}{2.522347in}}{\pgfqpoint{4.203957in}{2.514533in}}%
\pgfpathcurveto{\pgfqpoint{4.211770in}{2.506720in}}{\pgfqpoint{4.222369in}{2.502329in}}{\pgfqpoint{4.233419in}{2.502329in}}%
\pgfpathclose%
\pgfusepath{stroke,fill}%
\end{pgfscope}%
\begin{pgfscope}%
\pgfpathrectangle{\pgfqpoint{0.481978in}{0.331635in}}{\pgfqpoint{9.300000in}{7.700000in}}%
\pgfusepath{clip}%
\pgfsetbuttcap%
\pgfsetroundjoin%
\definecolor{currentfill}{rgb}{0.631373,0.788235,0.956863}%
\pgfsetfillcolor{currentfill}%
\pgfsetlinewidth{0.481800pt}%
\definecolor{currentstroke}{rgb}{1.000000,1.000000,1.000000}%
\pgfsetstrokecolor{currentstroke}%
\pgfsetdash{}{0pt}%
\pgfpathmoveto{\pgfqpoint{2.873447in}{4.546373in}}%
\pgfpathcurveto{\pgfqpoint{2.884497in}{4.546373in}}{\pgfqpoint{2.895096in}{4.550764in}}{\pgfqpoint{2.902910in}{4.558577in}}%
\pgfpathcurveto{\pgfqpoint{2.910723in}{4.566391in}}{\pgfqpoint{2.915114in}{4.576990in}}{\pgfqpoint{2.915114in}{4.588040in}}%
\pgfpathcurveto{\pgfqpoint{2.915114in}{4.599090in}}{\pgfqpoint{2.910723in}{4.609689in}}{\pgfqpoint{2.902910in}{4.617503in}}%
\pgfpathcurveto{\pgfqpoint{2.895096in}{4.625317in}}{\pgfqpoint{2.884497in}{4.629707in}}{\pgfqpoint{2.873447in}{4.629707in}}%
\pgfpathcurveto{\pgfqpoint{2.862397in}{4.629707in}}{\pgfqpoint{2.851798in}{4.625317in}}{\pgfqpoint{2.843984in}{4.617503in}}%
\pgfpathcurveto{\pgfqpoint{2.836171in}{4.609689in}}{\pgfqpoint{2.831780in}{4.599090in}}{\pgfqpoint{2.831780in}{4.588040in}}%
\pgfpathcurveto{\pgfqpoint{2.831780in}{4.576990in}}{\pgfqpoint{2.836171in}{4.566391in}}{\pgfqpoint{2.843984in}{4.558577in}}%
\pgfpathcurveto{\pgfqpoint{2.851798in}{4.550764in}}{\pgfqpoint{2.862397in}{4.546373in}}{\pgfqpoint{2.873447in}{4.546373in}}%
\pgfpathclose%
\pgfusepath{stroke,fill}%
\end{pgfscope}%
\begin{pgfscope}%
\pgfpathrectangle{\pgfqpoint{0.481978in}{0.331635in}}{\pgfqpoint{9.300000in}{7.700000in}}%
\pgfusepath{clip}%
\pgfsetbuttcap%
\pgfsetroundjoin%
\definecolor{currentfill}{rgb}{0.631373,0.788235,0.956863}%
\pgfsetfillcolor{currentfill}%
\pgfsetlinewidth{0.481800pt}%
\definecolor{currentstroke}{rgb}{1.000000,1.000000,1.000000}%
\pgfsetstrokecolor{currentstroke}%
\pgfsetdash{}{0pt}%
\pgfpathmoveto{\pgfqpoint{6.038124in}{7.639968in}}%
\pgfpathcurveto{\pgfqpoint{6.049174in}{7.639968in}}{\pgfqpoint{6.059773in}{7.644359in}}{\pgfqpoint{6.067587in}{7.652172in}}%
\pgfpathcurveto{\pgfqpoint{6.075401in}{7.659986in}}{\pgfqpoint{6.079791in}{7.670585in}}{\pgfqpoint{6.079791in}{7.681635in}}%
\pgfpathcurveto{\pgfqpoint{6.079791in}{7.692685in}}{\pgfqpoint{6.075401in}{7.703284in}}{\pgfqpoint{6.067587in}{7.711098in}}%
\pgfpathcurveto{\pgfqpoint{6.059773in}{7.718911in}}{\pgfqpoint{6.049174in}{7.723302in}}{\pgfqpoint{6.038124in}{7.723302in}}%
\pgfpathcurveto{\pgfqpoint{6.027074in}{7.723302in}}{\pgfqpoint{6.016475in}{7.718911in}}{\pgfqpoint{6.008661in}{7.711098in}}%
\pgfpathcurveto{\pgfqpoint{6.000848in}{7.703284in}}{\pgfqpoint{5.996458in}{7.692685in}}{\pgfqpoint{5.996458in}{7.681635in}}%
\pgfpathcurveto{\pgfqpoint{5.996458in}{7.670585in}}{\pgfqpoint{6.000848in}{7.659986in}}{\pgfqpoint{6.008661in}{7.652172in}}%
\pgfpathcurveto{\pgfqpoint{6.016475in}{7.644359in}}{\pgfqpoint{6.027074in}{7.639968in}}{\pgfqpoint{6.038124in}{7.639968in}}%
\pgfpathclose%
\pgfusepath{stroke,fill}%
\end{pgfscope}%
\begin{pgfscope}%
\pgfpathrectangle{\pgfqpoint{0.481978in}{0.331635in}}{\pgfqpoint{9.300000in}{7.700000in}}%
\pgfusepath{clip}%
\pgfsetbuttcap%
\pgfsetroundjoin%
\definecolor{currentfill}{rgb}{0.631373,0.788235,0.956863}%
\pgfsetfillcolor{currentfill}%
\pgfsetlinewidth{0.481800pt}%
\definecolor{currentstroke}{rgb}{1.000000,1.000000,1.000000}%
\pgfsetstrokecolor{currentstroke}%
\pgfsetdash{}{0pt}%
\pgfpathmoveto{\pgfqpoint{6.068535in}{7.007729in}}%
\pgfpathcurveto{\pgfqpoint{6.079585in}{7.007729in}}{\pgfqpoint{6.090184in}{7.012120in}}{\pgfqpoint{6.097998in}{7.019933in}}%
\pgfpathcurveto{\pgfqpoint{6.105812in}{7.027747in}}{\pgfqpoint{6.110202in}{7.038346in}}{\pgfqpoint{6.110202in}{7.049396in}}%
\pgfpathcurveto{\pgfqpoint{6.110202in}{7.060446in}}{\pgfqpoint{6.105812in}{7.071045in}}{\pgfqpoint{6.097998in}{7.078859in}}%
\pgfpathcurveto{\pgfqpoint{6.090184in}{7.086672in}}{\pgfqpoint{6.079585in}{7.091063in}}{\pgfqpoint{6.068535in}{7.091063in}}%
\pgfpathcurveto{\pgfqpoint{6.057485in}{7.091063in}}{\pgfqpoint{6.046886in}{7.086672in}}{\pgfqpoint{6.039072in}{7.078859in}}%
\pgfpathcurveto{\pgfqpoint{6.031259in}{7.071045in}}{\pgfqpoint{6.026869in}{7.060446in}}{\pgfqpoint{6.026869in}{7.049396in}}%
\pgfpathcurveto{\pgfqpoint{6.026869in}{7.038346in}}{\pgfqpoint{6.031259in}{7.027747in}}{\pgfqpoint{6.039072in}{7.019933in}}%
\pgfpathcurveto{\pgfqpoint{6.046886in}{7.012120in}}{\pgfqpoint{6.057485in}{7.007729in}}{\pgfqpoint{6.068535in}{7.007729in}}%
\pgfpathclose%
\pgfusepath{stroke,fill}%
\end{pgfscope}%
\begin{pgfscope}%
\pgfpathrectangle{\pgfqpoint{0.481978in}{0.331635in}}{\pgfqpoint{9.300000in}{7.700000in}}%
\pgfusepath{clip}%
\pgfsetbuttcap%
\pgfsetroundjoin%
\definecolor{currentfill}{rgb}{0.631373,0.788235,0.956863}%
\pgfsetfillcolor{currentfill}%
\pgfsetlinewidth{0.481800pt}%
\definecolor{currentstroke}{rgb}{1.000000,1.000000,1.000000}%
\pgfsetstrokecolor{currentstroke}%
\pgfsetdash{}{0pt}%
\pgfpathmoveto{\pgfqpoint{0.904705in}{2.664956in}}%
\pgfpathcurveto{\pgfqpoint{0.915755in}{2.664956in}}{\pgfqpoint{0.926354in}{2.669346in}}{\pgfqpoint{0.934168in}{2.677160in}}%
\pgfpathcurveto{\pgfqpoint{0.941982in}{2.684973in}}{\pgfqpoint{0.946372in}{2.695572in}}{\pgfqpoint{0.946372in}{2.706623in}}%
\pgfpathcurveto{\pgfqpoint{0.946372in}{2.717673in}}{\pgfqpoint{0.941982in}{2.728272in}}{\pgfqpoint{0.934168in}{2.736085in}}%
\pgfpathcurveto{\pgfqpoint{0.926354in}{2.743899in}}{\pgfqpoint{0.915755in}{2.748289in}}{\pgfqpoint{0.904705in}{2.748289in}}%
\pgfpathcurveto{\pgfqpoint{0.893655in}{2.748289in}}{\pgfqpoint{0.883056in}{2.743899in}}{\pgfqpoint{0.875242in}{2.736085in}}%
\pgfpathcurveto{\pgfqpoint{0.867429in}{2.728272in}}{\pgfqpoint{0.863039in}{2.717673in}}{\pgfqpoint{0.863039in}{2.706623in}}%
\pgfpathcurveto{\pgfqpoint{0.863039in}{2.695572in}}{\pgfqpoint{0.867429in}{2.684973in}}{\pgfqpoint{0.875242in}{2.677160in}}%
\pgfpathcurveto{\pgfqpoint{0.883056in}{2.669346in}}{\pgfqpoint{0.893655in}{2.664956in}}{\pgfqpoint{0.904705in}{2.664956in}}%
\pgfpathclose%
\pgfusepath{stroke,fill}%
\end{pgfscope}%
\begin{pgfscope}%
\pgfpathrectangle{\pgfqpoint{0.481978in}{0.331635in}}{\pgfqpoint{9.300000in}{7.700000in}}%
\pgfusepath{clip}%
\pgfsetbuttcap%
\pgfsetroundjoin%
\definecolor{currentfill}{rgb}{0.631373,0.788235,0.956863}%
\pgfsetfillcolor{currentfill}%
\pgfsetlinewidth{0.481800pt}%
\definecolor{currentstroke}{rgb}{1.000000,1.000000,1.000000}%
\pgfsetstrokecolor{currentstroke}%
\pgfsetdash{}{0pt}%
\pgfpathmoveto{\pgfqpoint{3.355697in}{0.882879in}}%
\pgfpathcurveto{\pgfqpoint{3.366747in}{0.882879in}}{\pgfqpoint{3.377346in}{0.887270in}}{\pgfqpoint{3.385160in}{0.895083in}}%
\pgfpathcurveto{\pgfqpoint{3.392973in}{0.902897in}}{\pgfqpoint{3.397364in}{0.913496in}}{\pgfqpoint{3.397364in}{0.924546in}}%
\pgfpathcurveto{\pgfqpoint{3.397364in}{0.935596in}}{\pgfqpoint{3.392973in}{0.946195in}}{\pgfqpoint{3.385160in}{0.954009in}}%
\pgfpathcurveto{\pgfqpoint{3.377346in}{0.961823in}}{\pgfqpoint{3.366747in}{0.966213in}}{\pgfqpoint{3.355697in}{0.966213in}}%
\pgfpathcurveto{\pgfqpoint{3.344647in}{0.966213in}}{\pgfqpoint{3.334048in}{0.961823in}}{\pgfqpoint{3.326234in}{0.954009in}}%
\pgfpathcurveto{\pgfqpoint{3.318420in}{0.946195in}}{\pgfqpoint{3.314030in}{0.935596in}}{\pgfqpoint{3.314030in}{0.924546in}}%
\pgfpathcurveto{\pgfqpoint{3.314030in}{0.913496in}}{\pgfqpoint{3.318420in}{0.902897in}}{\pgfqpoint{3.326234in}{0.895083in}}%
\pgfpathcurveto{\pgfqpoint{3.334048in}{0.887270in}}{\pgfqpoint{3.344647in}{0.882879in}}{\pgfqpoint{3.355697in}{0.882879in}}%
\pgfpathclose%
\pgfusepath{stroke,fill}%
\end{pgfscope}%
\begin{pgfscope}%
\pgfpathrectangle{\pgfqpoint{0.481978in}{0.331635in}}{\pgfqpoint{9.300000in}{7.700000in}}%
\pgfusepath{clip}%
\pgfsetbuttcap%
\pgfsetroundjoin%
\definecolor{currentfill}{rgb}{0.631373,0.788235,0.956863}%
\pgfsetfillcolor{currentfill}%
\pgfsetlinewidth{0.481800pt}%
\definecolor{currentstroke}{rgb}{1.000000,1.000000,1.000000}%
\pgfsetstrokecolor{currentstroke}%
\pgfsetdash{}{0pt}%
\pgfpathmoveto{\pgfqpoint{1.822383in}{3.039349in}}%
\pgfpathcurveto{\pgfqpoint{1.833433in}{3.039349in}}{\pgfqpoint{1.844032in}{3.043739in}}{\pgfqpoint{1.851846in}{3.051553in}}%
\pgfpathcurveto{\pgfqpoint{1.859659in}{3.059367in}}{\pgfqpoint{1.864050in}{3.069966in}}{\pgfqpoint{1.864050in}{3.081016in}}%
\pgfpathcurveto{\pgfqpoint{1.864050in}{3.092066in}}{\pgfqpoint{1.859659in}{3.102665in}}{\pgfqpoint{1.851846in}{3.110479in}}%
\pgfpathcurveto{\pgfqpoint{1.844032in}{3.118292in}}{\pgfqpoint{1.833433in}{3.122682in}}{\pgfqpoint{1.822383in}{3.122682in}}%
\pgfpathcurveto{\pgfqpoint{1.811333in}{3.122682in}}{\pgfqpoint{1.800734in}{3.118292in}}{\pgfqpoint{1.792920in}{3.110479in}}%
\pgfpathcurveto{\pgfqpoint{1.785107in}{3.102665in}}{\pgfqpoint{1.780716in}{3.092066in}}{\pgfqpoint{1.780716in}{3.081016in}}%
\pgfpathcurveto{\pgfqpoint{1.780716in}{3.069966in}}{\pgfqpoint{1.785107in}{3.059367in}}{\pgfqpoint{1.792920in}{3.051553in}}%
\pgfpathcurveto{\pgfqpoint{1.800734in}{3.043739in}}{\pgfqpoint{1.811333in}{3.039349in}}{\pgfqpoint{1.822383in}{3.039349in}}%
\pgfpathclose%
\pgfusepath{stroke,fill}%
\end{pgfscope}%
\begin{pgfscope}%
\pgfpathrectangle{\pgfqpoint{0.481978in}{0.331635in}}{\pgfqpoint{9.300000in}{7.700000in}}%
\pgfusepath{clip}%
\pgfsetbuttcap%
\pgfsetroundjoin%
\definecolor{currentfill}{rgb}{0.631373,0.788235,0.956863}%
\pgfsetfillcolor{currentfill}%
\pgfsetlinewidth{0.481800pt}%
\definecolor{currentstroke}{rgb}{1.000000,1.000000,1.000000}%
\pgfsetstrokecolor{currentstroke}%
\pgfsetdash{}{0pt}%
\pgfpathmoveto{\pgfqpoint{6.431702in}{7.269912in}}%
\pgfpathcurveto{\pgfqpoint{6.442753in}{7.269912in}}{\pgfqpoint{6.453352in}{7.274302in}}{\pgfqpoint{6.461165in}{7.282116in}}%
\pgfpathcurveto{\pgfqpoint{6.468979in}{7.289929in}}{\pgfqpoint{6.473369in}{7.300528in}}{\pgfqpoint{6.473369in}{7.311578in}}%
\pgfpathcurveto{\pgfqpoint{6.473369in}{7.322629in}}{\pgfqpoint{6.468979in}{7.333228in}}{\pgfqpoint{6.461165in}{7.341041in}}%
\pgfpathcurveto{\pgfqpoint{6.453352in}{7.348855in}}{\pgfqpoint{6.442753in}{7.353245in}}{\pgfqpoint{6.431702in}{7.353245in}}%
\pgfpathcurveto{\pgfqpoint{6.420652in}{7.353245in}}{\pgfqpoint{6.410053in}{7.348855in}}{\pgfqpoint{6.402240in}{7.341041in}}%
\pgfpathcurveto{\pgfqpoint{6.394426in}{7.333228in}}{\pgfqpoint{6.390036in}{7.322629in}}{\pgfqpoint{6.390036in}{7.311578in}}%
\pgfpathcurveto{\pgfqpoint{6.390036in}{7.300528in}}{\pgfqpoint{6.394426in}{7.289929in}}{\pgfqpoint{6.402240in}{7.282116in}}%
\pgfpathcurveto{\pgfqpoint{6.410053in}{7.274302in}}{\pgfqpoint{6.420652in}{7.269912in}}{\pgfqpoint{6.431702in}{7.269912in}}%
\pgfpathclose%
\pgfusepath{stroke,fill}%
\end{pgfscope}%
\begin{pgfscope}%
\pgfpathrectangle{\pgfqpoint{0.481978in}{0.331635in}}{\pgfqpoint{9.300000in}{7.700000in}}%
\pgfusepath{clip}%
\pgfsetbuttcap%
\pgfsetroundjoin%
\definecolor{currentfill}{rgb}{0.631373,0.788235,0.956863}%
\pgfsetfillcolor{currentfill}%
\pgfsetlinewidth{0.481800pt}%
\definecolor{currentstroke}{rgb}{1.000000,1.000000,1.000000}%
\pgfsetstrokecolor{currentstroke}%
\pgfsetdash{}{0pt}%
\pgfpathmoveto{\pgfqpoint{1.601212in}{5.246053in}}%
\pgfpathcurveto{\pgfqpoint{1.612262in}{5.246053in}}{\pgfqpoint{1.622861in}{5.250443in}}{\pgfqpoint{1.630675in}{5.258257in}}%
\pgfpathcurveto{\pgfqpoint{1.638488in}{5.266070in}}{\pgfqpoint{1.642878in}{5.276669in}}{\pgfqpoint{1.642878in}{5.287720in}}%
\pgfpathcurveto{\pgfqpoint{1.642878in}{5.298770in}}{\pgfqpoint{1.638488in}{5.309369in}}{\pgfqpoint{1.630675in}{5.317182in}}%
\pgfpathcurveto{\pgfqpoint{1.622861in}{5.324996in}}{\pgfqpoint{1.612262in}{5.329386in}}{\pgfqpoint{1.601212in}{5.329386in}}%
\pgfpathcurveto{\pgfqpoint{1.590162in}{5.329386in}}{\pgfqpoint{1.579563in}{5.324996in}}{\pgfqpoint{1.571749in}{5.317182in}}%
\pgfpathcurveto{\pgfqpoint{1.563935in}{5.309369in}}{\pgfqpoint{1.559545in}{5.298770in}}{\pgfqpoint{1.559545in}{5.287720in}}%
\pgfpathcurveto{\pgfqpoint{1.559545in}{5.276669in}}{\pgfqpoint{1.563935in}{5.266070in}}{\pgfqpoint{1.571749in}{5.258257in}}%
\pgfpathcurveto{\pgfqpoint{1.579563in}{5.250443in}}{\pgfqpoint{1.590162in}{5.246053in}}{\pgfqpoint{1.601212in}{5.246053in}}%
\pgfpathclose%
\pgfusepath{stroke,fill}%
\end{pgfscope}%
\begin{pgfscope}%
\pgfpathrectangle{\pgfqpoint{0.481978in}{0.331635in}}{\pgfqpoint{9.300000in}{7.700000in}}%
\pgfusepath{clip}%
\pgfsetbuttcap%
\pgfsetroundjoin%
\definecolor{currentfill}{rgb}{0.631373,0.788235,0.956863}%
\pgfsetfillcolor{currentfill}%
\pgfsetlinewidth{0.481800pt}%
\definecolor{currentstroke}{rgb}{1.000000,1.000000,1.000000}%
\pgfsetstrokecolor{currentstroke}%
\pgfsetdash{}{0pt}%
\pgfpathmoveto{\pgfqpoint{2.280847in}{6.304822in}}%
\pgfpathcurveto{\pgfqpoint{2.291897in}{6.304822in}}{\pgfqpoint{2.302496in}{6.309212in}}{\pgfqpoint{2.310310in}{6.317026in}}%
\pgfpathcurveto{\pgfqpoint{2.318123in}{6.324840in}}{\pgfqpoint{2.322514in}{6.335439in}}{\pgfqpoint{2.322514in}{6.346489in}}%
\pgfpathcurveto{\pgfqpoint{2.322514in}{6.357539in}}{\pgfqpoint{2.318123in}{6.368138in}}{\pgfqpoint{2.310310in}{6.375951in}}%
\pgfpathcurveto{\pgfqpoint{2.302496in}{6.383765in}}{\pgfqpoint{2.291897in}{6.388155in}}{\pgfqpoint{2.280847in}{6.388155in}}%
\pgfpathcurveto{\pgfqpoint{2.269797in}{6.388155in}}{\pgfqpoint{2.259198in}{6.383765in}}{\pgfqpoint{2.251384in}{6.375951in}}%
\pgfpathcurveto{\pgfqpoint{2.243570in}{6.368138in}}{\pgfqpoint{2.239180in}{6.357539in}}{\pgfqpoint{2.239180in}{6.346489in}}%
\pgfpathcurveto{\pgfqpoint{2.239180in}{6.335439in}}{\pgfqpoint{2.243570in}{6.324840in}}{\pgfqpoint{2.251384in}{6.317026in}}%
\pgfpathcurveto{\pgfqpoint{2.259198in}{6.309212in}}{\pgfqpoint{2.269797in}{6.304822in}}{\pgfqpoint{2.280847in}{6.304822in}}%
\pgfpathclose%
\pgfusepath{stroke,fill}%
\end{pgfscope}%
\begin{pgfscope}%
\pgfpathrectangle{\pgfqpoint{0.481978in}{0.331635in}}{\pgfqpoint{9.300000in}{7.700000in}}%
\pgfusepath{clip}%
\pgfsetbuttcap%
\pgfsetroundjoin%
\definecolor{currentfill}{rgb}{0.631373,0.788235,0.956863}%
\pgfsetfillcolor{currentfill}%
\pgfsetlinewidth{0.481800pt}%
\definecolor{currentstroke}{rgb}{1.000000,1.000000,1.000000}%
\pgfsetstrokecolor{currentstroke}%
\pgfsetdash{}{0pt}%
\pgfpathmoveto{\pgfqpoint{1.914988in}{4.693993in}}%
\pgfpathcurveto{\pgfqpoint{1.926039in}{4.693993in}}{\pgfqpoint{1.936638in}{4.698383in}}{\pgfqpoint{1.944451in}{4.706197in}}%
\pgfpathcurveto{\pgfqpoint{1.952265in}{4.714010in}}{\pgfqpoint{1.956655in}{4.724609in}}{\pgfqpoint{1.956655in}{4.735660in}}%
\pgfpathcurveto{\pgfqpoint{1.956655in}{4.746710in}}{\pgfqpoint{1.952265in}{4.757309in}}{\pgfqpoint{1.944451in}{4.765122in}}%
\pgfpathcurveto{\pgfqpoint{1.936638in}{4.772936in}}{\pgfqpoint{1.926039in}{4.777326in}}{\pgfqpoint{1.914988in}{4.777326in}}%
\pgfpathcurveto{\pgfqpoint{1.903938in}{4.777326in}}{\pgfqpoint{1.893339in}{4.772936in}}{\pgfqpoint{1.885526in}{4.765122in}}%
\pgfpathcurveto{\pgfqpoint{1.877712in}{4.757309in}}{\pgfqpoint{1.873322in}{4.746710in}}{\pgfqpoint{1.873322in}{4.735660in}}%
\pgfpathcurveto{\pgfqpoint{1.873322in}{4.724609in}}{\pgfqpoint{1.877712in}{4.714010in}}{\pgfqpoint{1.885526in}{4.706197in}}%
\pgfpathcurveto{\pgfqpoint{1.893339in}{4.698383in}}{\pgfqpoint{1.903938in}{4.693993in}}{\pgfqpoint{1.914988in}{4.693993in}}%
\pgfpathclose%
\pgfusepath{stroke,fill}%
\end{pgfscope}%
\begin{pgfscope}%
\pgfpathrectangle{\pgfqpoint{0.481978in}{0.331635in}}{\pgfqpoint{9.300000in}{7.700000in}}%
\pgfusepath{clip}%
\pgfsetbuttcap%
\pgfsetroundjoin%
\definecolor{currentfill}{rgb}{0.631373,0.788235,0.956863}%
\pgfsetfillcolor{currentfill}%
\pgfsetlinewidth{0.481800pt}%
\definecolor{currentstroke}{rgb}{1.000000,1.000000,1.000000}%
\pgfsetstrokecolor{currentstroke}%
\pgfsetdash{}{0pt}%
\pgfpathmoveto{\pgfqpoint{3.751753in}{1.411582in}}%
\pgfpathcurveto{\pgfqpoint{3.762803in}{1.411582in}}{\pgfqpoint{3.773402in}{1.415972in}}{\pgfqpoint{3.781215in}{1.423785in}}%
\pgfpathcurveto{\pgfqpoint{3.789029in}{1.431599in}}{\pgfqpoint{3.793419in}{1.442198in}}{\pgfqpoint{3.793419in}{1.453248in}}%
\pgfpathcurveto{\pgfqpoint{3.793419in}{1.464298in}}{\pgfqpoint{3.789029in}{1.474897in}}{\pgfqpoint{3.781215in}{1.482711in}}%
\pgfpathcurveto{\pgfqpoint{3.773402in}{1.490525in}}{\pgfqpoint{3.762803in}{1.494915in}}{\pgfqpoint{3.751753in}{1.494915in}}%
\pgfpathcurveto{\pgfqpoint{3.740702in}{1.494915in}}{\pgfqpoint{3.730103in}{1.490525in}}{\pgfqpoint{3.722290in}{1.482711in}}%
\pgfpathcurveto{\pgfqpoint{3.714476in}{1.474897in}}{\pgfqpoint{3.710086in}{1.464298in}}{\pgfqpoint{3.710086in}{1.453248in}}%
\pgfpathcurveto{\pgfqpoint{3.710086in}{1.442198in}}{\pgfqpoint{3.714476in}{1.431599in}}{\pgfqpoint{3.722290in}{1.423785in}}%
\pgfpathcurveto{\pgfqpoint{3.730103in}{1.415972in}}{\pgfqpoint{3.740702in}{1.411582in}}{\pgfqpoint{3.751753in}{1.411582in}}%
\pgfpathclose%
\pgfusepath{stroke,fill}%
\end{pgfscope}%
\begin{pgfscope}%
\pgfpathrectangle{\pgfqpoint{0.481978in}{0.331635in}}{\pgfqpoint{9.300000in}{7.700000in}}%
\pgfusepath{clip}%
\pgfsetbuttcap%
\pgfsetroundjoin%
\definecolor{currentfill}{rgb}{0.631373,0.788235,0.956863}%
\pgfsetfillcolor{currentfill}%
\pgfsetlinewidth{0.481800pt}%
\definecolor{currentstroke}{rgb}{1.000000,1.000000,1.000000}%
\pgfsetstrokecolor{currentstroke}%
\pgfsetdash{}{0pt}%
\pgfpathmoveto{\pgfqpoint{5.753665in}{7.213922in}}%
\pgfpathcurveto{\pgfqpoint{5.764715in}{7.213922in}}{\pgfqpoint{5.775314in}{7.218313in}}{\pgfqpoint{5.783128in}{7.226126in}}%
\pgfpathcurveto{\pgfqpoint{5.790942in}{7.233940in}}{\pgfqpoint{5.795332in}{7.244539in}}{\pgfqpoint{5.795332in}{7.255589in}}%
\pgfpathcurveto{\pgfqpoint{5.795332in}{7.266639in}}{\pgfqpoint{5.790942in}{7.277238in}}{\pgfqpoint{5.783128in}{7.285052in}}%
\pgfpathcurveto{\pgfqpoint{5.775314in}{7.292865in}}{\pgfqpoint{5.764715in}{7.297256in}}{\pgfqpoint{5.753665in}{7.297256in}}%
\pgfpathcurveto{\pgfqpoint{5.742615in}{7.297256in}}{\pgfqpoint{5.732016in}{7.292865in}}{\pgfqpoint{5.724203in}{7.285052in}}%
\pgfpathcurveto{\pgfqpoint{5.716389in}{7.277238in}}{\pgfqpoint{5.711999in}{7.266639in}}{\pgfqpoint{5.711999in}{7.255589in}}%
\pgfpathcurveto{\pgfqpoint{5.711999in}{7.244539in}}{\pgfqpoint{5.716389in}{7.233940in}}{\pgfqpoint{5.724203in}{7.226126in}}%
\pgfpathcurveto{\pgfqpoint{5.732016in}{7.218313in}}{\pgfqpoint{5.742615in}{7.213922in}}{\pgfqpoint{5.753665in}{7.213922in}}%
\pgfpathclose%
\pgfusepath{stroke,fill}%
\end{pgfscope}%
\begin{pgfscope}%
\pgfpathrectangle{\pgfqpoint{0.481978in}{0.331635in}}{\pgfqpoint{9.300000in}{7.700000in}}%
\pgfusepath{clip}%
\pgfsetbuttcap%
\pgfsetroundjoin%
\definecolor{currentfill}{rgb}{0.631373,0.788235,0.956863}%
\pgfsetfillcolor{currentfill}%
\pgfsetlinewidth{0.481800pt}%
\definecolor{currentstroke}{rgb}{1.000000,1.000000,1.000000}%
\pgfsetstrokecolor{currentstroke}%
\pgfsetdash{}{0pt}%
\pgfpathmoveto{\pgfqpoint{5.495955in}{5.191634in}}%
\pgfpathcurveto{\pgfqpoint{5.507005in}{5.191634in}}{\pgfqpoint{5.517604in}{5.196024in}}{\pgfqpoint{5.525417in}{5.203837in}}%
\pgfpathcurveto{\pgfqpoint{5.533231in}{5.211651in}}{\pgfqpoint{5.537621in}{5.222250in}}{\pgfqpoint{5.537621in}{5.233300in}}%
\pgfpathcurveto{\pgfqpoint{5.537621in}{5.244350in}}{\pgfqpoint{5.533231in}{5.254949in}}{\pgfqpoint{5.525417in}{5.262763in}}%
\pgfpathcurveto{\pgfqpoint{5.517604in}{5.270577in}}{\pgfqpoint{5.507005in}{5.274967in}}{\pgfqpoint{5.495955in}{5.274967in}}%
\pgfpathcurveto{\pgfqpoint{5.484904in}{5.274967in}}{\pgfqpoint{5.474305in}{5.270577in}}{\pgfqpoint{5.466492in}{5.262763in}}%
\pgfpathcurveto{\pgfqpoint{5.458678in}{5.254949in}}{\pgfqpoint{5.454288in}{5.244350in}}{\pgfqpoint{5.454288in}{5.233300in}}%
\pgfpathcurveto{\pgfqpoint{5.454288in}{5.222250in}}{\pgfqpoint{5.458678in}{5.211651in}}{\pgfqpoint{5.466492in}{5.203837in}}%
\pgfpathcurveto{\pgfqpoint{5.474305in}{5.196024in}}{\pgfqpoint{5.484904in}{5.191634in}}{\pgfqpoint{5.495955in}{5.191634in}}%
\pgfpathclose%
\pgfusepath{stroke,fill}%
\end{pgfscope}%
\begin{pgfscope}%
\pgfpathrectangle{\pgfqpoint{0.481978in}{0.331635in}}{\pgfqpoint{9.300000in}{7.700000in}}%
\pgfusepath{clip}%
\pgfsetbuttcap%
\pgfsetroundjoin%
\definecolor{currentfill}{rgb}{0.631373,0.788235,0.956863}%
\pgfsetfillcolor{currentfill}%
\pgfsetlinewidth{0.481800pt}%
\definecolor{currentstroke}{rgb}{1.000000,1.000000,1.000000}%
\pgfsetstrokecolor{currentstroke}%
\pgfsetdash{}{0pt}%
\pgfpathmoveto{\pgfqpoint{4.238672in}{4.793073in}}%
\pgfpathcurveto{\pgfqpoint{4.249723in}{4.793073in}}{\pgfqpoint{4.260322in}{4.797463in}}{\pgfqpoint{4.268135in}{4.805277in}}%
\pgfpathcurveto{\pgfqpoint{4.275949in}{4.813090in}}{\pgfqpoint{4.280339in}{4.823689in}}{\pgfqpoint{4.280339in}{4.834740in}}%
\pgfpathcurveto{\pgfqpoint{4.280339in}{4.845790in}}{\pgfqpoint{4.275949in}{4.856389in}}{\pgfqpoint{4.268135in}{4.864202in}}%
\pgfpathcurveto{\pgfqpoint{4.260322in}{4.872016in}}{\pgfqpoint{4.249723in}{4.876406in}}{\pgfqpoint{4.238672in}{4.876406in}}%
\pgfpathcurveto{\pgfqpoint{4.227622in}{4.876406in}}{\pgfqpoint{4.217023in}{4.872016in}}{\pgfqpoint{4.209210in}{4.864202in}}%
\pgfpathcurveto{\pgfqpoint{4.201396in}{4.856389in}}{\pgfqpoint{4.197006in}{4.845790in}}{\pgfqpoint{4.197006in}{4.834740in}}%
\pgfpathcurveto{\pgfqpoint{4.197006in}{4.823689in}}{\pgfqpoint{4.201396in}{4.813090in}}{\pgfqpoint{4.209210in}{4.805277in}}%
\pgfpathcurveto{\pgfqpoint{4.217023in}{4.797463in}}{\pgfqpoint{4.227622in}{4.793073in}}{\pgfqpoint{4.238672in}{4.793073in}}%
\pgfpathclose%
\pgfusepath{stroke,fill}%
\end{pgfscope}%
\begin{pgfscope}%
\pgfpathrectangle{\pgfqpoint{0.481978in}{0.331635in}}{\pgfqpoint{9.300000in}{7.700000in}}%
\pgfusepath{clip}%
\pgfsetbuttcap%
\pgfsetroundjoin%
\definecolor{currentfill}{rgb}{0.631373,0.788235,0.956863}%
\pgfsetfillcolor{currentfill}%
\pgfsetlinewidth{0.481800pt}%
\definecolor{currentstroke}{rgb}{1.000000,1.000000,1.000000}%
\pgfsetstrokecolor{currentstroke}%
\pgfsetdash{}{0pt}%
\pgfpathmoveto{\pgfqpoint{6.175837in}{7.474137in}}%
\pgfpathcurveto{\pgfqpoint{6.186887in}{7.474137in}}{\pgfqpoint{6.197486in}{7.478527in}}{\pgfqpoint{6.205300in}{7.486341in}}%
\pgfpathcurveto{\pgfqpoint{6.213114in}{7.494154in}}{\pgfqpoint{6.217504in}{7.504753in}}{\pgfqpoint{6.217504in}{7.515803in}}%
\pgfpathcurveto{\pgfqpoint{6.217504in}{7.526854in}}{\pgfqpoint{6.213114in}{7.537453in}}{\pgfqpoint{6.205300in}{7.545266in}}%
\pgfpathcurveto{\pgfqpoint{6.197486in}{7.553080in}}{\pgfqpoint{6.186887in}{7.557470in}}{\pgfqpoint{6.175837in}{7.557470in}}%
\pgfpathcurveto{\pgfqpoint{6.164787in}{7.557470in}}{\pgfqpoint{6.154188in}{7.553080in}}{\pgfqpoint{6.146374in}{7.545266in}}%
\pgfpathcurveto{\pgfqpoint{6.138561in}{7.537453in}}{\pgfqpoint{6.134170in}{7.526854in}}{\pgfqpoint{6.134170in}{7.515803in}}%
\pgfpathcurveto{\pgfqpoint{6.134170in}{7.504753in}}{\pgfqpoint{6.138561in}{7.494154in}}{\pgfqpoint{6.146374in}{7.486341in}}%
\pgfpathcurveto{\pgfqpoint{6.154188in}{7.478527in}}{\pgfqpoint{6.164787in}{7.474137in}}{\pgfqpoint{6.175837in}{7.474137in}}%
\pgfpathclose%
\pgfusepath{stroke,fill}%
\end{pgfscope}%
\begin{pgfscope}%
\pgfpathrectangle{\pgfqpoint{0.481978in}{0.331635in}}{\pgfqpoint{9.300000in}{7.700000in}}%
\pgfusepath{clip}%
\pgfsetbuttcap%
\pgfsetroundjoin%
\definecolor{currentfill}{rgb}{0.631373,0.788235,0.956863}%
\pgfsetfillcolor{currentfill}%
\pgfsetlinewidth{0.481800pt}%
\definecolor{currentstroke}{rgb}{1.000000,1.000000,1.000000}%
\pgfsetstrokecolor{currentstroke}%
\pgfsetdash{}{0pt}%
\pgfpathmoveto{\pgfqpoint{4.277708in}{2.774798in}}%
\pgfpathcurveto{\pgfqpoint{4.288759in}{2.774798in}}{\pgfqpoint{4.299358in}{2.779188in}}{\pgfqpoint{4.307171in}{2.787002in}}%
\pgfpathcurveto{\pgfqpoint{4.314985in}{2.794816in}}{\pgfqpoint{4.319375in}{2.805415in}}{\pgfqpoint{4.319375in}{2.816465in}}%
\pgfpathcurveto{\pgfqpoint{4.319375in}{2.827515in}}{\pgfqpoint{4.314985in}{2.838114in}}{\pgfqpoint{4.307171in}{2.845928in}}%
\pgfpathcurveto{\pgfqpoint{4.299358in}{2.853741in}}{\pgfqpoint{4.288759in}{2.858131in}}{\pgfqpoint{4.277708in}{2.858131in}}%
\pgfpathcurveto{\pgfqpoint{4.266658in}{2.858131in}}{\pgfqpoint{4.256059in}{2.853741in}}{\pgfqpoint{4.248246in}{2.845928in}}%
\pgfpathcurveto{\pgfqpoint{4.240432in}{2.838114in}}{\pgfqpoint{4.236042in}{2.827515in}}{\pgfqpoint{4.236042in}{2.816465in}}%
\pgfpathcurveto{\pgfqpoint{4.236042in}{2.805415in}}{\pgfqpoint{4.240432in}{2.794816in}}{\pgfqpoint{4.248246in}{2.787002in}}%
\pgfpathcurveto{\pgfqpoint{4.256059in}{2.779188in}}{\pgfqpoint{4.266658in}{2.774798in}}{\pgfqpoint{4.277708in}{2.774798in}}%
\pgfpathclose%
\pgfusepath{stroke,fill}%
\end{pgfscope}%
\begin{pgfscope}%
\pgfpathrectangle{\pgfqpoint{0.481978in}{0.331635in}}{\pgfqpoint{9.300000in}{7.700000in}}%
\pgfusepath{clip}%
\pgfsetbuttcap%
\pgfsetroundjoin%
\definecolor{currentfill}{rgb}{0.631373,0.788235,0.956863}%
\pgfsetfillcolor{currentfill}%
\pgfsetlinewidth{0.481800pt}%
\definecolor{currentstroke}{rgb}{1.000000,1.000000,1.000000}%
\pgfsetstrokecolor{currentstroke}%
\pgfsetdash{}{0pt}%
\pgfpathmoveto{\pgfqpoint{1.180315in}{4.828736in}}%
\pgfpathcurveto{\pgfqpoint{1.191365in}{4.828736in}}{\pgfqpoint{1.201964in}{4.833126in}}{\pgfqpoint{1.209778in}{4.840939in}}%
\pgfpathcurveto{\pgfqpoint{1.217592in}{4.848753in}}{\pgfqpoint{1.221982in}{4.859352in}}{\pgfqpoint{1.221982in}{4.870402in}}%
\pgfpathcurveto{\pgfqpoint{1.221982in}{4.881452in}}{\pgfqpoint{1.217592in}{4.892051in}}{\pgfqpoint{1.209778in}{4.899865in}}%
\pgfpathcurveto{\pgfqpoint{1.201964in}{4.907679in}}{\pgfqpoint{1.191365in}{4.912069in}}{\pgfqpoint{1.180315in}{4.912069in}}%
\pgfpathcurveto{\pgfqpoint{1.169265in}{4.912069in}}{\pgfqpoint{1.158666in}{4.907679in}}{\pgfqpoint{1.150852in}{4.899865in}}%
\pgfpathcurveto{\pgfqpoint{1.143039in}{4.892051in}}{\pgfqpoint{1.138649in}{4.881452in}}{\pgfqpoint{1.138649in}{4.870402in}}%
\pgfpathcurveto{\pgfqpoint{1.138649in}{4.859352in}}{\pgfqpoint{1.143039in}{4.848753in}}{\pgfqpoint{1.150852in}{4.840939in}}%
\pgfpathcurveto{\pgfqpoint{1.158666in}{4.833126in}}{\pgfqpoint{1.169265in}{4.828736in}}{\pgfqpoint{1.180315in}{4.828736in}}%
\pgfpathclose%
\pgfusepath{stroke,fill}%
\end{pgfscope}%
\begin{pgfscope}%
\pgfpathrectangle{\pgfqpoint{0.481978in}{0.331635in}}{\pgfqpoint{9.300000in}{7.700000in}}%
\pgfusepath{clip}%
\pgfsetbuttcap%
\pgfsetroundjoin%
\definecolor{currentfill}{rgb}{0.631373,0.788235,0.956863}%
\pgfsetfillcolor{currentfill}%
\pgfsetlinewidth{0.481800pt}%
\definecolor{currentstroke}{rgb}{1.000000,1.000000,1.000000}%
\pgfsetstrokecolor{currentstroke}%
\pgfsetdash{}{0pt}%
\pgfpathmoveto{\pgfqpoint{4.575953in}{4.795945in}}%
\pgfpathcurveto{\pgfqpoint{4.587003in}{4.795945in}}{\pgfqpoint{4.597602in}{4.800336in}}{\pgfqpoint{4.605416in}{4.808149in}}%
\pgfpathcurveto{\pgfqpoint{4.613229in}{4.815963in}}{\pgfqpoint{4.617619in}{4.826562in}}{\pgfqpoint{4.617619in}{4.837612in}}%
\pgfpathcurveto{\pgfqpoint{4.617619in}{4.848662in}}{\pgfqpoint{4.613229in}{4.859261in}}{\pgfqpoint{4.605416in}{4.867075in}}%
\pgfpathcurveto{\pgfqpoint{4.597602in}{4.874888in}}{\pgfqpoint{4.587003in}{4.879279in}}{\pgfqpoint{4.575953in}{4.879279in}}%
\pgfpathcurveto{\pgfqpoint{4.564903in}{4.879279in}}{\pgfqpoint{4.554304in}{4.874888in}}{\pgfqpoint{4.546490in}{4.867075in}}%
\pgfpathcurveto{\pgfqpoint{4.538676in}{4.859261in}}{\pgfqpoint{4.534286in}{4.848662in}}{\pgfqpoint{4.534286in}{4.837612in}}%
\pgfpathcurveto{\pgfqpoint{4.534286in}{4.826562in}}{\pgfqpoint{4.538676in}{4.815963in}}{\pgfqpoint{4.546490in}{4.808149in}}%
\pgfpathcurveto{\pgfqpoint{4.554304in}{4.800336in}}{\pgfqpoint{4.564903in}{4.795945in}}{\pgfqpoint{4.575953in}{4.795945in}}%
\pgfpathclose%
\pgfusepath{stroke,fill}%
\end{pgfscope}%
\begin{pgfscope}%
\pgfpathrectangle{\pgfqpoint{0.481978in}{0.331635in}}{\pgfqpoint{9.300000in}{7.700000in}}%
\pgfusepath{clip}%
\pgfsetbuttcap%
\pgfsetroundjoin%
\definecolor{currentfill}{rgb}{0.631373,0.788235,0.956863}%
\pgfsetfillcolor{currentfill}%
\pgfsetlinewidth{0.481800pt}%
\definecolor{currentstroke}{rgb}{1.000000,1.000000,1.000000}%
\pgfsetstrokecolor{currentstroke}%
\pgfsetdash{}{0pt}%
\pgfpathmoveto{\pgfqpoint{7.415126in}{4.531322in}}%
\pgfpathcurveto{\pgfqpoint{7.426176in}{4.531322in}}{\pgfqpoint{7.436775in}{4.535712in}}{\pgfqpoint{7.444589in}{4.543526in}}%
\pgfpathcurveto{\pgfqpoint{7.452402in}{4.551340in}}{\pgfqpoint{7.456793in}{4.561939in}}{\pgfqpoint{7.456793in}{4.572989in}}%
\pgfpathcurveto{\pgfqpoint{7.456793in}{4.584039in}}{\pgfqpoint{7.452402in}{4.594638in}}{\pgfqpoint{7.444589in}{4.602452in}}%
\pgfpathcurveto{\pgfqpoint{7.436775in}{4.610265in}}{\pgfqpoint{7.426176in}{4.614655in}}{\pgfqpoint{7.415126in}{4.614655in}}%
\pgfpathcurveto{\pgfqpoint{7.404076in}{4.614655in}}{\pgfqpoint{7.393477in}{4.610265in}}{\pgfqpoint{7.385663in}{4.602452in}}%
\pgfpathcurveto{\pgfqpoint{7.377850in}{4.594638in}}{\pgfqpoint{7.373459in}{4.584039in}}{\pgfqpoint{7.373459in}{4.572989in}}%
\pgfpathcurveto{\pgfqpoint{7.373459in}{4.561939in}}{\pgfqpoint{7.377850in}{4.551340in}}{\pgfqpoint{7.385663in}{4.543526in}}%
\pgfpathcurveto{\pgfqpoint{7.393477in}{4.535712in}}{\pgfqpoint{7.404076in}{4.531322in}}{\pgfqpoint{7.415126in}{4.531322in}}%
\pgfpathclose%
\pgfusepath{stroke,fill}%
\end{pgfscope}%
\begin{pgfscope}%
\pgfpathrectangle{\pgfqpoint{0.481978in}{0.331635in}}{\pgfqpoint{9.300000in}{7.700000in}}%
\pgfusepath{clip}%
\pgfsetbuttcap%
\pgfsetroundjoin%
\definecolor{currentfill}{rgb}{0.631373,0.788235,0.956863}%
\pgfsetfillcolor{currentfill}%
\pgfsetlinewidth{0.481800pt}%
\definecolor{currentstroke}{rgb}{1.000000,1.000000,1.000000}%
\pgfsetstrokecolor{currentstroke}%
\pgfsetdash{}{0pt}%
\pgfpathmoveto{\pgfqpoint{5.743162in}{7.207909in}}%
\pgfpathcurveto{\pgfqpoint{5.754212in}{7.207909in}}{\pgfqpoint{5.764811in}{7.212299in}}{\pgfqpoint{5.772625in}{7.220113in}}%
\pgfpathcurveto{\pgfqpoint{5.780439in}{7.227926in}}{\pgfqpoint{5.784829in}{7.238525in}}{\pgfqpoint{5.784829in}{7.249576in}}%
\pgfpathcurveto{\pgfqpoint{5.784829in}{7.260626in}}{\pgfqpoint{5.780439in}{7.271225in}}{\pgfqpoint{5.772625in}{7.279038in}}%
\pgfpathcurveto{\pgfqpoint{5.764811in}{7.286852in}}{\pgfqpoint{5.754212in}{7.291242in}}{\pgfqpoint{5.743162in}{7.291242in}}%
\pgfpathcurveto{\pgfqpoint{5.732112in}{7.291242in}}{\pgfqpoint{5.721513in}{7.286852in}}{\pgfqpoint{5.713699in}{7.279038in}}%
\pgfpathcurveto{\pgfqpoint{5.705886in}{7.271225in}}{\pgfqpoint{5.701495in}{7.260626in}}{\pgfqpoint{5.701495in}{7.249576in}}%
\pgfpathcurveto{\pgfqpoint{5.701495in}{7.238525in}}{\pgfqpoint{5.705886in}{7.227926in}}{\pgfqpoint{5.713699in}{7.220113in}}%
\pgfpathcurveto{\pgfqpoint{5.721513in}{7.212299in}}{\pgfqpoint{5.732112in}{7.207909in}}{\pgfqpoint{5.743162in}{7.207909in}}%
\pgfpathclose%
\pgfusepath{stroke,fill}%
\end{pgfscope}%
\begin{pgfscope}%
\pgfpathrectangle{\pgfqpoint{0.481978in}{0.331635in}}{\pgfqpoint{9.300000in}{7.700000in}}%
\pgfusepath{clip}%
\pgfsetbuttcap%
\pgfsetroundjoin%
\definecolor{currentfill}{rgb}{0.631373,0.788235,0.956863}%
\pgfsetfillcolor{currentfill}%
\pgfsetlinewidth{0.481800pt}%
\definecolor{currentstroke}{rgb}{1.000000,1.000000,1.000000}%
\pgfsetstrokecolor{currentstroke}%
\pgfsetdash{}{0pt}%
\pgfpathmoveto{\pgfqpoint{3.731632in}{6.321424in}}%
\pgfpathcurveto{\pgfqpoint{3.742682in}{6.321424in}}{\pgfqpoint{3.753281in}{6.325815in}}{\pgfqpoint{3.761095in}{6.333628in}}%
\pgfpathcurveto{\pgfqpoint{3.768908in}{6.341442in}}{\pgfqpoint{3.773299in}{6.352041in}}{\pgfqpoint{3.773299in}{6.363091in}}%
\pgfpathcurveto{\pgfqpoint{3.773299in}{6.374141in}}{\pgfqpoint{3.768908in}{6.384740in}}{\pgfqpoint{3.761095in}{6.392554in}}%
\pgfpathcurveto{\pgfqpoint{3.753281in}{6.400367in}}{\pgfqpoint{3.742682in}{6.404758in}}{\pgfqpoint{3.731632in}{6.404758in}}%
\pgfpathcurveto{\pgfqpoint{3.720582in}{6.404758in}}{\pgfqpoint{3.709983in}{6.400367in}}{\pgfqpoint{3.702169in}{6.392554in}}%
\pgfpathcurveto{\pgfqpoint{3.694356in}{6.384740in}}{\pgfqpoint{3.689965in}{6.374141in}}{\pgfqpoint{3.689965in}{6.363091in}}%
\pgfpathcurveto{\pgfqpoint{3.689965in}{6.352041in}}{\pgfqpoint{3.694356in}{6.341442in}}{\pgfqpoint{3.702169in}{6.333628in}}%
\pgfpathcurveto{\pgfqpoint{3.709983in}{6.325815in}}{\pgfqpoint{3.720582in}{6.321424in}}{\pgfqpoint{3.731632in}{6.321424in}}%
\pgfpathclose%
\pgfusepath{stroke,fill}%
\end{pgfscope}%
\begin{pgfscope}%
\pgfpathrectangle{\pgfqpoint{0.481978in}{0.331635in}}{\pgfqpoint{9.300000in}{7.700000in}}%
\pgfusepath{clip}%
\pgfsetbuttcap%
\pgfsetroundjoin%
\definecolor{currentfill}{rgb}{0.631373,0.788235,0.956863}%
\pgfsetfillcolor{currentfill}%
\pgfsetlinewidth{0.481800pt}%
\definecolor{currentstroke}{rgb}{1.000000,1.000000,1.000000}%
\pgfsetstrokecolor{currentstroke}%
\pgfsetdash{}{0pt}%
\pgfpathmoveto{\pgfqpoint{3.354108in}{0.639968in}}%
\pgfpathcurveto{\pgfqpoint{3.365158in}{0.639968in}}{\pgfqpoint{3.375757in}{0.644359in}}{\pgfqpoint{3.383571in}{0.652172in}}%
\pgfpathcurveto{\pgfqpoint{3.391384in}{0.659986in}}{\pgfqpoint{3.395775in}{0.670585in}}{\pgfqpoint{3.395775in}{0.681635in}}%
\pgfpathcurveto{\pgfqpoint{3.395775in}{0.692685in}}{\pgfqpoint{3.391384in}{0.703284in}}{\pgfqpoint{3.383571in}{0.711098in}}%
\pgfpathcurveto{\pgfqpoint{3.375757in}{0.718911in}}{\pgfqpoint{3.365158in}{0.723302in}}{\pgfqpoint{3.354108in}{0.723302in}}%
\pgfpathcurveto{\pgfqpoint{3.343058in}{0.723302in}}{\pgfqpoint{3.332459in}{0.718911in}}{\pgfqpoint{3.324645in}{0.711098in}}%
\pgfpathcurveto{\pgfqpoint{3.316832in}{0.703284in}}{\pgfqpoint{3.312441in}{0.692685in}}{\pgfqpoint{3.312441in}{0.681635in}}%
\pgfpathcurveto{\pgfqpoint{3.312441in}{0.670585in}}{\pgfqpoint{3.316832in}{0.659986in}}{\pgfqpoint{3.324645in}{0.652172in}}%
\pgfpathcurveto{\pgfqpoint{3.332459in}{0.644359in}}{\pgfqpoint{3.343058in}{0.639968in}}{\pgfqpoint{3.354108in}{0.639968in}}%
\pgfpathclose%
\pgfusepath{stroke,fill}%
\end{pgfscope}%
\begin{pgfscope}%
\pgfpathrectangle{\pgfqpoint{0.481978in}{0.331635in}}{\pgfqpoint{9.300000in}{7.700000in}}%
\pgfusepath{clip}%
\pgfsetbuttcap%
\pgfsetroundjoin%
\definecolor{currentfill}{rgb}{0.631373,0.788235,0.956863}%
\pgfsetfillcolor{currentfill}%
\pgfsetlinewidth{0.481800pt}%
\definecolor{currentstroke}{rgb}{1.000000,1.000000,1.000000}%
\pgfsetstrokecolor{currentstroke}%
\pgfsetdash{}{0pt}%
\pgfpathmoveto{\pgfqpoint{2.507727in}{5.202892in}}%
\pgfpathcurveto{\pgfqpoint{2.518777in}{5.202892in}}{\pgfqpoint{2.529376in}{5.207282in}}{\pgfqpoint{2.537190in}{5.215096in}}%
\pgfpathcurveto{\pgfqpoint{2.545003in}{5.222910in}}{\pgfqpoint{2.549394in}{5.233509in}}{\pgfqpoint{2.549394in}{5.244559in}}%
\pgfpathcurveto{\pgfqpoint{2.549394in}{5.255609in}}{\pgfqpoint{2.545003in}{5.266208in}}{\pgfqpoint{2.537190in}{5.274022in}}%
\pgfpathcurveto{\pgfqpoint{2.529376in}{5.281835in}}{\pgfqpoint{2.518777in}{5.286225in}}{\pgfqpoint{2.507727in}{5.286225in}}%
\pgfpathcurveto{\pgfqpoint{2.496677in}{5.286225in}}{\pgfqpoint{2.486078in}{5.281835in}}{\pgfqpoint{2.478264in}{5.274022in}}%
\pgfpathcurveto{\pgfqpoint{2.470451in}{5.266208in}}{\pgfqpoint{2.466060in}{5.255609in}}{\pgfqpoint{2.466060in}{5.244559in}}%
\pgfpathcurveto{\pgfqpoint{2.466060in}{5.233509in}}{\pgfqpoint{2.470451in}{5.222910in}}{\pgfqpoint{2.478264in}{5.215096in}}%
\pgfpathcurveto{\pgfqpoint{2.486078in}{5.207282in}}{\pgfqpoint{2.496677in}{5.202892in}}{\pgfqpoint{2.507727in}{5.202892in}}%
\pgfpathclose%
\pgfusepath{stroke,fill}%
\end{pgfscope}%
\begin{pgfscope}%
\pgfpathrectangle{\pgfqpoint{0.481978in}{0.331635in}}{\pgfqpoint{9.300000in}{7.700000in}}%
\pgfusepath{clip}%
\pgfsetbuttcap%
\pgfsetroundjoin%
\definecolor{currentfill}{rgb}{0.631373,0.788235,0.956863}%
\pgfsetfillcolor{currentfill}%
\pgfsetlinewidth{0.481800pt}%
\definecolor{currentstroke}{rgb}{1.000000,1.000000,1.000000}%
\pgfsetstrokecolor{currentstroke}%
\pgfsetdash{}{0pt}%
\pgfpathmoveto{\pgfqpoint{3.345563in}{4.675887in}}%
\pgfpathcurveto{\pgfqpoint{3.356613in}{4.675887in}}{\pgfqpoint{3.367212in}{4.680278in}}{\pgfqpoint{3.375026in}{4.688091in}}%
\pgfpathcurveto{\pgfqpoint{3.382839in}{4.695905in}}{\pgfqpoint{3.387230in}{4.706504in}}{\pgfqpoint{3.387230in}{4.717554in}}%
\pgfpathcurveto{\pgfqpoint{3.387230in}{4.728604in}}{\pgfqpoint{3.382839in}{4.739203in}}{\pgfqpoint{3.375026in}{4.747017in}}%
\pgfpathcurveto{\pgfqpoint{3.367212in}{4.754830in}}{\pgfqpoint{3.356613in}{4.759221in}}{\pgfqpoint{3.345563in}{4.759221in}}%
\pgfpathcurveto{\pgfqpoint{3.334513in}{4.759221in}}{\pgfqpoint{3.323914in}{4.754830in}}{\pgfqpoint{3.316100in}{4.747017in}}%
\pgfpathcurveto{\pgfqpoint{3.308287in}{4.739203in}}{\pgfqpoint{3.303896in}{4.728604in}}{\pgfqpoint{3.303896in}{4.717554in}}%
\pgfpathcurveto{\pgfqpoint{3.303896in}{4.706504in}}{\pgfqpoint{3.308287in}{4.695905in}}{\pgfqpoint{3.316100in}{4.688091in}}%
\pgfpathcurveto{\pgfqpoint{3.323914in}{4.680278in}}{\pgfqpoint{3.334513in}{4.675887in}}{\pgfqpoint{3.345563in}{4.675887in}}%
\pgfpathclose%
\pgfusepath{stroke,fill}%
\end{pgfscope}%
\begin{pgfscope}%
\pgfpathrectangle{\pgfqpoint{0.481978in}{0.331635in}}{\pgfqpoint{9.300000in}{7.700000in}}%
\pgfusepath{clip}%
\pgfsetbuttcap%
\pgfsetroundjoin%
\definecolor{currentfill}{rgb}{0.631373,0.788235,0.956863}%
\pgfsetfillcolor{currentfill}%
\pgfsetlinewidth{0.481800pt}%
\definecolor{currentstroke}{rgb}{1.000000,1.000000,1.000000}%
\pgfsetstrokecolor{currentstroke}%
\pgfsetdash{}{0pt}%
\pgfpathmoveto{\pgfqpoint{5.829195in}{5.212750in}}%
\pgfpathcurveto{\pgfqpoint{5.840246in}{5.212750in}}{\pgfqpoint{5.850845in}{5.217140in}}{\pgfqpoint{5.858658in}{5.224954in}}%
\pgfpathcurveto{\pgfqpoint{5.866472in}{5.232767in}}{\pgfqpoint{5.870862in}{5.243367in}}{\pgfqpoint{5.870862in}{5.254417in}}%
\pgfpathcurveto{\pgfqpoint{5.870862in}{5.265467in}}{\pgfqpoint{5.866472in}{5.276066in}}{\pgfqpoint{5.858658in}{5.283879in}}%
\pgfpathcurveto{\pgfqpoint{5.850845in}{5.291693in}}{\pgfqpoint{5.840246in}{5.296083in}}{\pgfqpoint{5.829195in}{5.296083in}}%
\pgfpathcurveto{\pgfqpoint{5.818145in}{5.296083in}}{\pgfqpoint{5.807546in}{5.291693in}}{\pgfqpoint{5.799733in}{5.283879in}}%
\pgfpathcurveto{\pgfqpoint{5.791919in}{5.276066in}}{\pgfqpoint{5.787529in}{5.265467in}}{\pgfqpoint{5.787529in}{5.254417in}}%
\pgfpathcurveto{\pgfqpoint{5.787529in}{5.243367in}}{\pgfqpoint{5.791919in}{5.232767in}}{\pgfqpoint{5.799733in}{5.224954in}}%
\pgfpathcurveto{\pgfqpoint{5.807546in}{5.217140in}}{\pgfqpoint{5.818145in}{5.212750in}}{\pgfqpoint{5.829195in}{5.212750in}}%
\pgfpathclose%
\pgfusepath{stroke,fill}%
\end{pgfscope}%
\begin{pgfscope}%
\pgfpathrectangle{\pgfqpoint{0.481978in}{0.331635in}}{\pgfqpoint{9.300000in}{7.700000in}}%
\pgfusepath{clip}%
\pgfsetbuttcap%
\pgfsetroundjoin%
\definecolor{currentfill}{rgb}{0.631373,0.788235,0.956863}%
\pgfsetfillcolor{currentfill}%
\pgfsetlinewidth{0.481800pt}%
\definecolor{currentstroke}{rgb}{1.000000,1.000000,1.000000}%
\pgfsetstrokecolor{currentstroke}%
\pgfsetdash{}{0pt}%
\pgfpathmoveto{\pgfqpoint{1.266407in}{2.502917in}}%
\pgfpathcurveto{\pgfqpoint{1.277457in}{2.502917in}}{\pgfqpoint{1.288056in}{2.507308in}}{\pgfqpoint{1.295870in}{2.515121in}}%
\pgfpathcurveto{\pgfqpoint{1.303683in}{2.522935in}}{\pgfqpoint{1.308074in}{2.533534in}}{\pgfqpoint{1.308074in}{2.544584in}}%
\pgfpathcurveto{\pgfqpoint{1.308074in}{2.555634in}}{\pgfqpoint{1.303683in}{2.566233in}}{\pgfqpoint{1.295870in}{2.574047in}}%
\pgfpathcurveto{\pgfqpoint{1.288056in}{2.581861in}}{\pgfqpoint{1.277457in}{2.586251in}}{\pgfqpoint{1.266407in}{2.586251in}}%
\pgfpathcurveto{\pgfqpoint{1.255357in}{2.586251in}}{\pgfqpoint{1.244758in}{2.581861in}}{\pgfqpoint{1.236944in}{2.574047in}}%
\pgfpathcurveto{\pgfqpoint{1.229131in}{2.566233in}}{\pgfqpoint{1.224740in}{2.555634in}}{\pgfqpoint{1.224740in}{2.544584in}}%
\pgfpathcurveto{\pgfqpoint{1.224740in}{2.533534in}}{\pgfqpoint{1.229131in}{2.522935in}}{\pgfqpoint{1.236944in}{2.515121in}}%
\pgfpathcurveto{\pgfqpoint{1.244758in}{2.507308in}}{\pgfqpoint{1.255357in}{2.502917in}}{\pgfqpoint{1.266407in}{2.502917in}}%
\pgfpathclose%
\pgfusepath{stroke,fill}%
\end{pgfscope}%
\begin{pgfscope}%
\pgfpathrectangle{\pgfqpoint{0.481978in}{0.331635in}}{\pgfqpoint{9.300000in}{7.700000in}}%
\pgfusepath{clip}%
\pgfsetbuttcap%
\pgfsetroundjoin%
\definecolor{currentfill}{rgb}{0.631373,0.788235,0.956863}%
\pgfsetfillcolor{currentfill}%
\pgfsetlinewidth{0.481800pt}%
\definecolor{currentstroke}{rgb}{1.000000,1.000000,1.000000}%
\pgfsetstrokecolor{currentstroke}%
\pgfsetdash{}{0pt}%
\pgfpathmoveto{\pgfqpoint{1.280257in}{3.702244in}}%
\pgfpathcurveto{\pgfqpoint{1.291307in}{3.702244in}}{\pgfqpoint{1.301906in}{3.706634in}}{\pgfqpoint{1.309720in}{3.714448in}}%
\pgfpathcurveto{\pgfqpoint{1.317533in}{3.722261in}}{\pgfqpoint{1.321923in}{3.732860in}}{\pgfqpoint{1.321923in}{3.743911in}}%
\pgfpathcurveto{\pgfqpoint{1.321923in}{3.754961in}}{\pgfqpoint{1.317533in}{3.765560in}}{\pgfqpoint{1.309720in}{3.773373in}}%
\pgfpathcurveto{\pgfqpoint{1.301906in}{3.781187in}}{\pgfqpoint{1.291307in}{3.785577in}}{\pgfqpoint{1.280257in}{3.785577in}}%
\pgfpathcurveto{\pgfqpoint{1.269207in}{3.785577in}}{\pgfqpoint{1.258608in}{3.781187in}}{\pgfqpoint{1.250794in}{3.773373in}}%
\pgfpathcurveto{\pgfqpoint{1.242980in}{3.765560in}}{\pgfqpoint{1.238590in}{3.754961in}}{\pgfqpoint{1.238590in}{3.743911in}}%
\pgfpathcurveto{\pgfqpoint{1.238590in}{3.732860in}}{\pgfqpoint{1.242980in}{3.722261in}}{\pgfqpoint{1.250794in}{3.714448in}}%
\pgfpathcurveto{\pgfqpoint{1.258608in}{3.706634in}}{\pgfqpoint{1.269207in}{3.702244in}}{\pgfqpoint{1.280257in}{3.702244in}}%
\pgfpathclose%
\pgfusepath{stroke,fill}%
\end{pgfscope}%
\begin{pgfscope}%
\pgfpathrectangle{\pgfqpoint{0.481978in}{0.331635in}}{\pgfqpoint{9.300000in}{7.700000in}}%
\pgfusepath{clip}%
\pgfsetbuttcap%
\pgfsetroundjoin%
\definecolor{currentfill}{rgb}{0.631373,0.788235,0.956863}%
\pgfsetfillcolor{currentfill}%
\pgfsetlinewidth{0.481800pt}%
\definecolor{currentstroke}{rgb}{1.000000,1.000000,1.000000}%
\pgfsetstrokecolor{currentstroke}%
\pgfsetdash{}{0pt}%
\pgfpathmoveto{\pgfqpoint{1.366790in}{2.436232in}}%
\pgfpathcurveto{\pgfqpoint{1.377840in}{2.436232in}}{\pgfqpoint{1.388439in}{2.440622in}}{\pgfqpoint{1.396252in}{2.448436in}}%
\pgfpathcurveto{\pgfqpoint{1.404066in}{2.456249in}}{\pgfqpoint{1.408456in}{2.466848in}}{\pgfqpoint{1.408456in}{2.477899in}}%
\pgfpathcurveto{\pgfqpoint{1.408456in}{2.488949in}}{\pgfqpoint{1.404066in}{2.499548in}}{\pgfqpoint{1.396252in}{2.507361in}}%
\pgfpathcurveto{\pgfqpoint{1.388439in}{2.515175in}}{\pgfqpoint{1.377840in}{2.519565in}}{\pgfqpoint{1.366790in}{2.519565in}}%
\pgfpathcurveto{\pgfqpoint{1.355740in}{2.519565in}}{\pgfqpoint{1.345141in}{2.515175in}}{\pgfqpoint{1.337327in}{2.507361in}}%
\pgfpathcurveto{\pgfqpoint{1.329513in}{2.499548in}}{\pgfqpoint{1.325123in}{2.488949in}}{\pgfqpoint{1.325123in}{2.477899in}}%
\pgfpathcurveto{\pgfqpoint{1.325123in}{2.466848in}}{\pgfqpoint{1.329513in}{2.456249in}}{\pgfqpoint{1.337327in}{2.448436in}}%
\pgfpathcurveto{\pgfqpoint{1.345141in}{2.440622in}}{\pgfqpoint{1.355740in}{2.436232in}}{\pgfqpoint{1.366790in}{2.436232in}}%
\pgfpathclose%
\pgfusepath{stroke,fill}%
\end{pgfscope}%
\begin{pgfscope}%
\pgfpathrectangle{\pgfqpoint{0.481978in}{0.331635in}}{\pgfqpoint{9.300000in}{7.700000in}}%
\pgfusepath{clip}%
\pgfsetbuttcap%
\pgfsetroundjoin%
\definecolor{currentfill}{rgb}{0.631373,0.788235,0.956863}%
\pgfsetfillcolor{currentfill}%
\pgfsetlinewidth{0.481800pt}%
\definecolor{currentstroke}{rgb}{1.000000,1.000000,1.000000}%
\pgfsetstrokecolor{currentstroke}%
\pgfsetdash{}{0pt}%
\pgfpathmoveto{\pgfqpoint{5.131403in}{3.775556in}}%
\pgfpathcurveto{\pgfqpoint{5.142453in}{3.775556in}}{\pgfqpoint{5.153052in}{3.779946in}}{\pgfqpoint{5.160866in}{3.787760in}}%
\pgfpathcurveto{\pgfqpoint{5.168679in}{3.795573in}}{\pgfqpoint{5.173070in}{3.806172in}}{\pgfqpoint{5.173070in}{3.817222in}}%
\pgfpathcurveto{\pgfqpoint{5.173070in}{3.828273in}}{\pgfqpoint{5.168679in}{3.838872in}}{\pgfqpoint{5.160866in}{3.846685in}}%
\pgfpathcurveto{\pgfqpoint{5.153052in}{3.854499in}}{\pgfqpoint{5.142453in}{3.858889in}}{\pgfqpoint{5.131403in}{3.858889in}}%
\pgfpathcurveto{\pgfqpoint{5.120353in}{3.858889in}}{\pgfqpoint{5.109754in}{3.854499in}}{\pgfqpoint{5.101940in}{3.846685in}}%
\pgfpathcurveto{\pgfqpoint{5.094127in}{3.838872in}}{\pgfqpoint{5.089736in}{3.828273in}}{\pgfqpoint{5.089736in}{3.817222in}}%
\pgfpathcurveto{\pgfqpoint{5.089736in}{3.806172in}}{\pgfqpoint{5.094127in}{3.795573in}}{\pgfqpoint{5.101940in}{3.787760in}}%
\pgfpathcurveto{\pgfqpoint{5.109754in}{3.779946in}}{\pgfqpoint{5.120353in}{3.775556in}}{\pgfqpoint{5.131403in}{3.775556in}}%
\pgfpathclose%
\pgfusepath{stroke,fill}%
\end{pgfscope}%
\begin{pgfscope}%
\pgfpathrectangle{\pgfqpoint{0.481978in}{0.331635in}}{\pgfqpoint{9.300000in}{7.700000in}}%
\pgfusepath{clip}%
\pgfsetbuttcap%
\pgfsetroundjoin%
\definecolor{currentfill}{rgb}{0.631373,0.788235,0.956863}%
\pgfsetfillcolor{currentfill}%
\pgfsetlinewidth{0.481800pt}%
\definecolor{currentstroke}{rgb}{1.000000,1.000000,1.000000}%
\pgfsetstrokecolor{currentstroke}%
\pgfsetdash{}{0pt}%
\pgfpathmoveto{\pgfqpoint{6.077197in}{6.974685in}}%
\pgfpathcurveto{\pgfqpoint{6.088248in}{6.974685in}}{\pgfqpoint{6.098847in}{6.979075in}}{\pgfqpoint{6.106660in}{6.986889in}}%
\pgfpathcurveto{\pgfqpoint{6.114474in}{6.994702in}}{\pgfqpoint{6.118864in}{7.005301in}}{\pgfqpoint{6.118864in}{7.016351in}}%
\pgfpathcurveto{\pgfqpoint{6.118864in}{7.027402in}}{\pgfqpoint{6.114474in}{7.038001in}}{\pgfqpoint{6.106660in}{7.045814in}}%
\pgfpathcurveto{\pgfqpoint{6.098847in}{7.053628in}}{\pgfqpoint{6.088248in}{7.058018in}}{\pgfqpoint{6.077197in}{7.058018in}}%
\pgfpathcurveto{\pgfqpoint{6.066147in}{7.058018in}}{\pgfqpoint{6.055548in}{7.053628in}}{\pgfqpoint{6.047735in}{7.045814in}}%
\pgfpathcurveto{\pgfqpoint{6.039921in}{7.038001in}}{\pgfqpoint{6.035531in}{7.027402in}}{\pgfqpoint{6.035531in}{7.016351in}}%
\pgfpathcurveto{\pgfqpoint{6.035531in}{7.005301in}}{\pgfqpoint{6.039921in}{6.994702in}}{\pgfqpoint{6.047735in}{6.986889in}}%
\pgfpathcurveto{\pgfqpoint{6.055548in}{6.979075in}}{\pgfqpoint{6.066147in}{6.974685in}}{\pgfqpoint{6.077197in}{6.974685in}}%
\pgfpathclose%
\pgfusepath{stroke,fill}%
\end{pgfscope}%
\begin{pgfscope}%
\pgfpathrectangle{\pgfqpoint{0.481978in}{0.331635in}}{\pgfqpoint{9.300000in}{7.700000in}}%
\pgfusepath{clip}%
\pgfsetbuttcap%
\pgfsetroundjoin%
\definecolor{currentfill}{rgb}{0.631373,0.788235,0.956863}%
\pgfsetfillcolor{currentfill}%
\pgfsetlinewidth{0.481800pt}%
\definecolor{currentstroke}{rgb}{1.000000,1.000000,1.000000}%
\pgfsetstrokecolor{currentstroke}%
\pgfsetdash{}{0pt}%
\pgfpathmoveto{\pgfqpoint{5.267507in}{3.437471in}}%
\pgfpathcurveto{\pgfqpoint{5.278557in}{3.437471in}}{\pgfqpoint{5.289156in}{3.441861in}}{\pgfqpoint{5.296970in}{3.449674in}}%
\pgfpathcurveto{\pgfqpoint{5.304783in}{3.457488in}}{\pgfqpoint{5.309173in}{3.468087in}}{\pgfqpoint{5.309173in}{3.479137in}}%
\pgfpathcurveto{\pgfqpoint{5.309173in}{3.490187in}}{\pgfqpoint{5.304783in}{3.500786in}}{\pgfqpoint{5.296970in}{3.508600in}}%
\pgfpathcurveto{\pgfqpoint{5.289156in}{3.516414in}}{\pgfqpoint{5.278557in}{3.520804in}}{\pgfqpoint{5.267507in}{3.520804in}}%
\pgfpathcurveto{\pgfqpoint{5.256457in}{3.520804in}}{\pgfqpoint{5.245858in}{3.516414in}}{\pgfqpoint{5.238044in}{3.508600in}}%
\pgfpathcurveto{\pgfqpoint{5.230230in}{3.500786in}}{\pgfqpoint{5.225840in}{3.490187in}}{\pgfqpoint{5.225840in}{3.479137in}}%
\pgfpathcurveto{\pgfqpoint{5.225840in}{3.468087in}}{\pgfqpoint{5.230230in}{3.457488in}}{\pgfqpoint{5.238044in}{3.449674in}}%
\pgfpathcurveto{\pgfqpoint{5.245858in}{3.441861in}}{\pgfqpoint{5.256457in}{3.437471in}}{\pgfqpoint{5.267507in}{3.437471in}}%
\pgfpathclose%
\pgfusepath{stroke,fill}%
\end{pgfscope}%
\begin{pgfscope}%
\pgfpathrectangle{\pgfqpoint{0.481978in}{0.331635in}}{\pgfqpoint{9.300000in}{7.700000in}}%
\pgfusepath{clip}%
\pgfsetbuttcap%
\pgfsetroundjoin%
\definecolor{currentfill}{rgb}{0.631373,0.788235,0.956863}%
\pgfsetfillcolor{currentfill}%
\pgfsetlinewidth{0.481800pt}%
\definecolor{currentstroke}{rgb}{1.000000,1.000000,1.000000}%
\pgfsetstrokecolor{currentstroke}%
\pgfsetdash{}{0pt}%
\pgfpathmoveto{\pgfqpoint{5.422432in}{4.371502in}}%
\pgfpathcurveto{\pgfqpoint{5.433483in}{4.371502in}}{\pgfqpoint{5.444082in}{4.375892in}}{\pgfqpoint{5.451895in}{4.383705in}}%
\pgfpathcurveto{\pgfqpoint{5.459709in}{4.391519in}}{\pgfqpoint{5.464099in}{4.402118in}}{\pgfqpoint{5.464099in}{4.413168in}}%
\pgfpathcurveto{\pgfqpoint{5.464099in}{4.424218in}}{\pgfqpoint{5.459709in}{4.434817in}}{\pgfqpoint{5.451895in}{4.442631in}}%
\pgfpathcurveto{\pgfqpoint{5.444082in}{4.450445in}}{\pgfqpoint{5.433483in}{4.454835in}}{\pgfqpoint{5.422432in}{4.454835in}}%
\pgfpathcurveto{\pgfqpoint{5.411382in}{4.454835in}}{\pgfqpoint{5.400783in}{4.450445in}}{\pgfqpoint{5.392970in}{4.442631in}}%
\pgfpathcurveto{\pgfqpoint{5.385156in}{4.434817in}}{\pgfqpoint{5.380766in}{4.424218in}}{\pgfqpoint{5.380766in}{4.413168in}}%
\pgfpathcurveto{\pgfqpoint{5.380766in}{4.402118in}}{\pgfqpoint{5.385156in}{4.391519in}}{\pgfqpoint{5.392970in}{4.383705in}}%
\pgfpathcurveto{\pgfqpoint{5.400783in}{4.375892in}}{\pgfqpoint{5.411382in}{4.371502in}}{\pgfqpoint{5.422432in}{4.371502in}}%
\pgfpathclose%
\pgfusepath{stroke,fill}%
\end{pgfscope}%
\begin{pgfscope}%
\pgfpathrectangle{\pgfqpoint{0.481978in}{0.331635in}}{\pgfqpoint{9.300000in}{7.700000in}}%
\pgfusepath{clip}%
\pgfsetbuttcap%
\pgfsetroundjoin%
\definecolor{currentfill}{rgb}{0.631373,0.788235,0.956863}%
\pgfsetfillcolor{currentfill}%
\pgfsetlinewidth{0.481800pt}%
\definecolor{currentstroke}{rgb}{1.000000,1.000000,1.000000}%
\pgfsetstrokecolor{currentstroke}%
\pgfsetdash{}{0pt}%
\pgfpathmoveto{\pgfqpoint{4.576144in}{3.852406in}}%
\pgfpathcurveto{\pgfqpoint{4.587194in}{3.852406in}}{\pgfqpoint{4.597793in}{3.856796in}}{\pgfqpoint{4.605606in}{3.864609in}}%
\pgfpathcurveto{\pgfqpoint{4.613420in}{3.872423in}}{\pgfqpoint{4.617810in}{3.883022in}}{\pgfqpoint{4.617810in}{3.894072in}}%
\pgfpathcurveto{\pgfqpoint{4.617810in}{3.905122in}}{\pgfqpoint{4.613420in}{3.915721in}}{\pgfqpoint{4.605606in}{3.923535in}}%
\pgfpathcurveto{\pgfqpoint{4.597793in}{3.931349in}}{\pgfqpoint{4.587194in}{3.935739in}}{\pgfqpoint{4.576144in}{3.935739in}}%
\pgfpathcurveto{\pgfqpoint{4.565093in}{3.935739in}}{\pgfqpoint{4.554494in}{3.931349in}}{\pgfqpoint{4.546681in}{3.923535in}}%
\pgfpathcurveto{\pgfqpoint{4.538867in}{3.915721in}}{\pgfqpoint{4.534477in}{3.905122in}}{\pgfqpoint{4.534477in}{3.894072in}}%
\pgfpathcurveto{\pgfqpoint{4.534477in}{3.883022in}}{\pgfqpoint{4.538867in}{3.872423in}}{\pgfqpoint{4.546681in}{3.864609in}}%
\pgfpathcurveto{\pgfqpoint{4.554494in}{3.856796in}}{\pgfqpoint{4.565093in}{3.852406in}}{\pgfqpoint{4.576144in}{3.852406in}}%
\pgfpathclose%
\pgfusepath{stroke,fill}%
\end{pgfscope}%
\begin{pgfscope}%
\pgfpathrectangle{\pgfqpoint{0.481978in}{0.331635in}}{\pgfqpoint{9.300000in}{7.700000in}}%
\pgfusepath{clip}%
\pgfsetbuttcap%
\pgfsetroundjoin%
\definecolor{currentfill}{rgb}{0.631373,0.788235,0.956863}%
\pgfsetfillcolor{currentfill}%
\pgfsetlinewidth{0.481800pt}%
\definecolor{currentstroke}{rgb}{1.000000,1.000000,1.000000}%
\pgfsetstrokecolor{currentstroke}%
\pgfsetdash{}{0pt}%
\pgfpathmoveto{\pgfqpoint{2.109185in}{3.342395in}}%
\pgfpathcurveto{\pgfqpoint{2.120235in}{3.342395in}}{\pgfqpoint{2.130834in}{3.346786in}}{\pgfqpoint{2.138648in}{3.354599in}}%
\pgfpathcurveto{\pgfqpoint{2.146461in}{3.362413in}}{\pgfqpoint{2.150852in}{3.373012in}}{\pgfqpoint{2.150852in}{3.384062in}}%
\pgfpathcurveto{\pgfqpoint{2.150852in}{3.395112in}}{\pgfqpoint{2.146461in}{3.405711in}}{\pgfqpoint{2.138648in}{3.413525in}}%
\pgfpathcurveto{\pgfqpoint{2.130834in}{3.421338in}}{\pgfqpoint{2.120235in}{3.425729in}}{\pgfqpoint{2.109185in}{3.425729in}}%
\pgfpathcurveto{\pgfqpoint{2.098135in}{3.425729in}}{\pgfqpoint{2.087536in}{3.421338in}}{\pgfqpoint{2.079722in}{3.413525in}}%
\pgfpathcurveto{\pgfqpoint{2.071908in}{3.405711in}}{\pgfqpoint{2.067518in}{3.395112in}}{\pgfqpoint{2.067518in}{3.384062in}}%
\pgfpathcurveto{\pgfqpoint{2.067518in}{3.373012in}}{\pgfqpoint{2.071908in}{3.362413in}}{\pgfqpoint{2.079722in}{3.354599in}}%
\pgfpathcurveto{\pgfqpoint{2.087536in}{3.346786in}}{\pgfqpoint{2.098135in}{3.342395in}}{\pgfqpoint{2.109185in}{3.342395in}}%
\pgfpathclose%
\pgfusepath{stroke,fill}%
\end{pgfscope}%
\begin{pgfscope}%
\pgfpathrectangle{\pgfqpoint{0.481978in}{0.331635in}}{\pgfqpoint{9.300000in}{7.700000in}}%
\pgfusepath{clip}%
\pgfsetbuttcap%
\pgfsetroundjoin%
\definecolor{currentfill}{rgb}{0.631373,0.788235,0.956863}%
\pgfsetfillcolor{currentfill}%
\pgfsetlinewidth{0.481800pt}%
\definecolor{currentstroke}{rgb}{1.000000,1.000000,1.000000}%
\pgfsetstrokecolor{currentstroke}%
\pgfsetdash{}{0pt}%
\pgfpathmoveto{\pgfqpoint{3.045397in}{4.041120in}}%
\pgfpathcurveto{\pgfqpoint{3.056447in}{4.041120in}}{\pgfqpoint{3.067046in}{4.045510in}}{\pgfqpoint{3.074860in}{4.053324in}}%
\pgfpathcurveto{\pgfqpoint{3.082673in}{4.061137in}}{\pgfqpoint{3.087063in}{4.071736in}}{\pgfqpoint{3.087063in}{4.082787in}}%
\pgfpathcurveto{\pgfqpoint{3.087063in}{4.093837in}}{\pgfqpoint{3.082673in}{4.104436in}}{\pgfqpoint{3.074860in}{4.112249in}}%
\pgfpathcurveto{\pgfqpoint{3.067046in}{4.120063in}}{\pgfqpoint{3.056447in}{4.124453in}}{\pgfqpoint{3.045397in}{4.124453in}}%
\pgfpathcurveto{\pgfqpoint{3.034347in}{4.124453in}}{\pgfqpoint{3.023748in}{4.120063in}}{\pgfqpoint{3.015934in}{4.112249in}}%
\pgfpathcurveto{\pgfqpoint{3.008120in}{4.104436in}}{\pgfqpoint{3.003730in}{4.093837in}}{\pgfqpoint{3.003730in}{4.082787in}}%
\pgfpathcurveto{\pgfqpoint{3.003730in}{4.071736in}}{\pgfqpoint{3.008120in}{4.061137in}}{\pgfqpoint{3.015934in}{4.053324in}}%
\pgfpathcurveto{\pgfqpoint{3.023748in}{4.045510in}}{\pgfqpoint{3.034347in}{4.041120in}}{\pgfqpoint{3.045397in}{4.041120in}}%
\pgfpathclose%
\pgfusepath{stroke,fill}%
\end{pgfscope}%
\begin{pgfscope}%
\pgfpathrectangle{\pgfqpoint{0.481978in}{0.331635in}}{\pgfqpoint{9.300000in}{7.700000in}}%
\pgfusepath{clip}%
\pgfsetbuttcap%
\pgfsetroundjoin%
\definecolor{currentfill}{rgb}{0.631373,0.788235,0.956863}%
\pgfsetfillcolor{currentfill}%
\pgfsetlinewidth{0.481800pt}%
\definecolor{currentstroke}{rgb}{1.000000,1.000000,1.000000}%
\pgfsetstrokecolor{currentstroke}%
\pgfsetdash{}{0pt}%
\pgfpathmoveto{\pgfqpoint{2.937522in}{5.069906in}}%
\pgfpathcurveto{\pgfqpoint{2.948572in}{5.069906in}}{\pgfqpoint{2.959171in}{5.074297in}}{\pgfqpoint{2.966985in}{5.082110in}}%
\pgfpathcurveto{\pgfqpoint{2.974799in}{5.089924in}}{\pgfqpoint{2.979189in}{5.100523in}}{\pgfqpoint{2.979189in}{5.111573in}}%
\pgfpathcurveto{\pgfqpoint{2.979189in}{5.122623in}}{\pgfqpoint{2.974799in}{5.133222in}}{\pgfqpoint{2.966985in}{5.141036in}}%
\pgfpathcurveto{\pgfqpoint{2.959171in}{5.148849in}}{\pgfqpoint{2.948572in}{5.153240in}}{\pgfqpoint{2.937522in}{5.153240in}}%
\pgfpathcurveto{\pgfqpoint{2.926472in}{5.153240in}}{\pgfqpoint{2.915873in}{5.148849in}}{\pgfqpoint{2.908059in}{5.141036in}}%
\pgfpathcurveto{\pgfqpoint{2.900246in}{5.133222in}}{\pgfqpoint{2.895856in}{5.122623in}}{\pgfqpoint{2.895856in}{5.111573in}}%
\pgfpathcurveto{\pgfqpoint{2.895856in}{5.100523in}}{\pgfqpoint{2.900246in}{5.089924in}}{\pgfqpoint{2.908059in}{5.082110in}}%
\pgfpathcurveto{\pgfqpoint{2.915873in}{5.074297in}}{\pgfqpoint{2.926472in}{5.069906in}}{\pgfqpoint{2.937522in}{5.069906in}}%
\pgfpathclose%
\pgfusepath{stroke,fill}%
\end{pgfscope}%
\begin{pgfscope}%
\pgfpathrectangle{\pgfqpoint{0.481978in}{0.331635in}}{\pgfqpoint{9.300000in}{7.700000in}}%
\pgfusepath{clip}%
\pgfsetbuttcap%
\pgfsetroundjoin%
\definecolor{currentfill}{rgb}{0.631373,0.788235,0.956863}%
\pgfsetfillcolor{currentfill}%
\pgfsetlinewidth{0.481800pt}%
\definecolor{currentstroke}{rgb}{1.000000,1.000000,1.000000}%
\pgfsetstrokecolor{currentstroke}%
\pgfsetdash{}{0pt}%
\pgfpathmoveto{\pgfqpoint{2.204208in}{6.210084in}}%
\pgfpathcurveto{\pgfqpoint{2.215259in}{6.210084in}}{\pgfqpoint{2.225858in}{6.214475in}}{\pgfqpoint{2.233671in}{6.222288in}}%
\pgfpathcurveto{\pgfqpoint{2.241485in}{6.230102in}}{\pgfqpoint{2.245875in}{6.240701in}}{\pgfqpoint{2.245875in}{6.251751in}}%
\pgfpathcurveto{\pgfqpoint{2.245875in}{6.262801in}}{\pgfqpoint{2.241485in}{6.273400in}}{\pgfqpoint{2.233671in}{6.281214in}}%
\pgfpathcurveto{\pgfqpoint{2.225858in}{6.289027in}}{\pgfqpoint{2.215259in}{6.293418in}}{\pgfqpoint{2.204208in}{6.293418in}}%
\pgfpathcurveto{\pgfqpoint{2.193158in}{6.293418in}}{\pgfqpoint{2.182559in}{6.289027in}}{\pgfqpoint{2.174746in}{6.281214in}}%
\pgfpathcurveto{\pgfqpoint{2.166932in}{6.273400in}}{\pgfqpoint{2.162542in}{6.262801in}}{\pgfqpoint{2.162542in}{6.251751in}}%
\pgfpathcurveto{\pgfqpoint{2.162542in}{6.240701in}}{\pgfqpoint{2.166932in}{6.230102in}}{\pgfqpoint{2.174746in}{6.222288in}}%
\pgfpathcurveto{\pgfqpoint{2.182559in}{6.214475in}}{\pgfqpoint{2.193158in}{6.210084in}}{\pgfqpoint{2.204208in}{6.210084in}}%
\pgfpathclose%
\pgfusepath{stroke,fill}%
\end{pgfscope}%
\begin{pgfscope}%
\pgfpathrectangle{\pgfqpoint{0.481978in}{0.331635in}}{\pgfqpoint{9.300000in}{7.700000in}}%
\pgfusepath{clip}%
\pgfsetbuttcap%
\pgfsetroundjoin%
\definecolor{currentfill}{rgb}{0.631373,0.788235,0.956863}%
\pgfsetfillcolor{currentfill}%
\pgfsetlinewidth{0.481800pt}%
\definecolor{currentstroke}{rgb}{1.000000,1.000000,1.000000}%
\pgfsetstrokecolor{currentstroke}%
\pgfsetdash{}{0pt}%
\pgfpathmoveto{\pgfqpoint{3.780219in}{6.346116in}}%
\pgfpathcurveto{\pgfqpoint{3.791269in}{6.346116in}}{\pgfqpoint{3.801868in}{6.350506in}}{\pgfqpoint{3.809681in}{6.358320in}}%
\pgfpathcurveto{\pgfqpoint{3.817495in}{6.366133in}}{\pgfqpoint{3.821885in}{6.376732in}}{\pgfqpoint{3.821885in}{6.387783in}}%
\pgfpathcurveto{\pgfqpoint{3.821885in}{6.398833in}}{\pgfqpoint{3.817495in}{6.409432in}}{\pgfqpoint{3.809681in}{6.417245in}}%
\pgfpathcurveto{\pgfqpoint{3.801868in}{6.425059in}}{\pgfqpoint{3.791269in}{6.429449in}}{\pgfqpoint{3.780219in}{6.429449in}}%
\pgfpathcurveto{\pgfqpoint{3.769168in}{6.429449in}}{\pgfqpoint{3.758569in}{6.425059in}}{\pgfqpoint{3.750756in}{6.417245in}}%
\pgfpathcurveto{\pgfqpoint{3.742942in}{6.409432in}}{\pgfqpoint{3.738552in}{6.398833in}}{\pgfqpoint{3.738552in}{6.387783in}}%
\pgfpathcurveto{\pgfqpoint{3.738552in}{6.376732in}}{\pgfqpoint{3.742942in}{6.366133in}}{\pgfqpoint{3.750756in}{6.358320in}}%
\pgfpathcurveto{\pgfqpoint{3.758569in}{6.350506in}}{\pgfqpoint{3.769168in}{6.346116in}}{\pgfqpoint{3.780219in}{6.346116in}}%
\pgfpathclose%
\pgfusepath{stroke,fill}%
\end{pgfscope}%
\begin{pgfscope}%
\pgfpathrectangle{\pgfqpoint{0.481978in}{0.331635in}}{\pgfqpoint{9.300000in}{7.700000in}}%
\pgfusepath{clip}%
\pgfsetbuttcap%
\pgfsetroundjoin%
\definecolor{currentfill}{rgb}{0.631373,0.788235,0.956863}%
\pgfsetfillcolor{currentfill}%
\pgfsetlinewidth{0.481800pt}%
\definecolor{currentstroke}{rgb}{1.000000,1.000000,1.000000}%
\pgfsetstrokecolor{currentstroke}%
\pgfsetdash{}{0pt}%
\pgfpathmoveto{\pgfqpoint{4.814839in}{2.753029in}}%
\pgfpathcurveto{\pgfqpoint{4.825890in}{2.753029in}}{\pgfqpoint{4.836489in}{2.757419in}}{\pgfqpoint{4.844302in}{2.765233in}}%
\pgfpathcurveto{\pgfqpoint{4.852116in}{2.773046in}}{\pgfqpoint{4.856506in}{2.783646in}}{\pgfqpoint{4.856506in}{2.794696in}}%
\pgfpathcurveto{\pgfqpoint{4.856506in}{2.805746in}}{\pgfqpoint{4.852116in}{2.816345in}}{\pgfqpoint{4.844302in}{2.824158in}}%
\pgfpathcurveto{\pgfqpoint{4.836489in}{2.831972in}}{\pgfqpoint{4.825890in}{2.836362in}}{\pgfqpoint{4.814839in}{2.836362in}}%
\pgfpathcurveto{\pgfqpoint{4.803789in}{2.836362in}}{\pgfqpoint{4.793190in}{2.831972in}}{\pgfqpoint{4.785377in}{2.824158in}}%
\pgfpathcurveto{\pgfqpoint{4.777563in}{2.816345in}}{\pgfqpoint{4.773173in}{2.805746in}}{\pgfqpoint{4.773173in}{2.794696in}}%
\pgfpathcurveto{\pgfqpoint{4.773173in}{2.783646in}}{\pgfqpoint{4.777563in}{2.773046in}}{\pgfqpoint{4.785377in}{2.765233in}}%
\pgfpathcurveto{\pgfqpoint{4.793190in}{2.757419in}}{\pgfqpoint{4.803789in}{2.753029in}}{\pgfqpoint{4.814839in}{2.753029in}}%
\pgfpathclose%
\pgfusepath{stroke,fill}%
\end{pgfscope}%
\begin{pgfscope}%
\pgfpathrectangle{\pgfqpoint{0.481978in}{0.331635in}}{\pgfqpoint{9.300000in}{7.700000in}}%
\pgfusepath{clip}%
\pgfsetbuttcap%
\pgfsetroundjoin%
\definecolor{currentfill}{rgb}{0.631373,0.788235,0.956863}%
\pgfsetfillcolor{currentfill}%
\pgfsetlinewidth{0.481800pt}%
\definecolor{currentstroke}{rgb}{1.000000,1.000000,1.000000}%
\pgfsetstrokecolor{currentstroke}%
\pgfsetdash{}{0pt}%
\pgfpathmoveto{\pgfqpoint{2.753007in}{3.827926in}}%
\pgfpathcurveto{\pgfqpoint{2.764057in}{3.827926in}}{\pgfqpoint{2.774656in}{3.832316in}}{\pgfqpoint{2.782470in}{3.840130in}}%
\pgfpathcurveto{\pgfqpoint{2.790284in}{3.847944in}}{\pgfqpoint{2.794674in}{3.858543in}}{\pgfqpoint{2.794674in}{3.869593in}}%
\pgfpathcurveto{\pgfqpoint{2.794674in}{3.880643in}}{\pgfqpoint{2.790284in}{3.891242in}}{\pgfqpoint{2.782470in}{3.899056in}}%
\pgfpathcurveto{\pgfqpoint{2.774656in}{3.906869in}}{\pgfqpoint{2.764057in}{3.911259in}}{\pgfqpoint{2.753007in}{3.911259in}}%
\pgfpathcurveto{\pgfqpoint{2.741957in}{3.911259in}}{\pgfqpoint{2.731358in}{3.906869in}}{\pgfqpoint{2.723544in}{3.899056in}}%
\pgfpathcurveto{\pgfqpoint{2.715731in}{3.891242in}}{\pgfqpoint{2.711341in}{3.880643in}}{\pgfqpoint{2.711341in}{3.869593in}}%
\pgfpathcurveto{\pgfqpoint{2.711341in}{3.858543in}}{\pgfqpoint{2.715731in}{3.847944in}}{\pgfqpoint{2.723544in}{3.840130in}}%
\pgfpathcurveto{\pgfqpoint{2.731358in}{3.832316in}}{\pgfqpoint{2.741957in}{3.827926in}}{\pgfqpoint{2.753007in}{3.827926in}}%
\pgfpathclose%
\pgfusepath{stroke,fill}%
\end{pgfscope}%
\begin{pgfscope}%
\pgfpathrectangle{\pgfqpoint{0.481978in}{0.331635in}}{\pgfqpoint{9.300000in}{7.700000in}}%
\pgfusepath{clip}%
\pgfsetbuttcap%
\pgfsetroundjoin%
\definecolor{currentfill}{rgb}{0.631373,0.788235,0.956863}%
\pgfsetfillcolor{currentfill}%
\pgfsetlinewidth{0.481800pt}%
\definecolor{currentstroke}{rgb}{1.000000,1.000000,1.000000}%
\pgfsetstrokecolor{currentstroke}%
\pgfsetdash{}{0pt}%
\pgfpathmoveto{\pgfqpoint{2.663463in}{5.659614in}}%
\pgfpathcurveto{\pgfqpoint{2.674513in}{5.659614in}}{\pgfqpoint{2.685112in}{5.664004in}}{\pgfqpoint{2.692925in}{5.671818in}}%
\pgfpathcurveto{\pgfqpoint{2.700739in}{5.679631in}}{\pgfqpoint{2.705129in}{5.690230in}}{\pgfqpoint{2.705129in}{5.701280in}}%
\pgfpathcurveto{\pgfqpoint{2.705129in}{5.712330in}}{\pgfqpoint{2.700739in}{5.722929in}}{\pgfqpoint{2.692925in}{5.730743in}}%
\pgfpathcurveto{\pgfqpoint{2.685112in}{5.738557in}}{\pgfqpoint{2.674513in}{5.742947in}}{\pgfqpoint{2.663463in}{5.742947in}}%
\pgfpathcurveto{\pgfqpoint{2.652412in}{5.742947in}}{\pgfqpoint{2.641813in}{5.738557in}}{\pgfqpoint{2.634000in}{5.730743in}}%
\pgfpathcurveto{\pgfqpoint{2.626186in}{5.722929in}}{\pgfqpoint{2.621796in}{5.712330in}}{\pgfqpoint{2.621796in}{5.701280in}}%
\pgfpathcurveto{\pgfqpoint{2.621796in}{5.690230in}}{\pgfqpoint{2.626186in}{5.679631in}}{\pgfqpoint{2.634000in}{5.671818in}}%
\pgfpathcurveto{\pgfqpoint{2.641813in}{5.664004in}}{\pgfqpoint{2.652412in}{5.659614in}}{\pgfqpoint{2.663463in}{5.659614in}}%
\pgfpathclose%
\pgfusepath{stroke,fill}%
\end{pgfscope}%
\begin{pgfscope}%
\pgfpathrectangle{\pgfqpoint{0.481978in}{0.331635in}}{\pgfqpoint{9.300000in}{7.700000in}}%
\pgfusepath{clip}%
\pgfsetbuttcap%
\pgfsetroundjoin%
\definecolor{currentfill}{rgb}{0.631373,0.788235,0.956863}%
\pgfsetfillcolor{currentfill}%
\pgfsetlinewidth{0.481800pt}%
\definecolor{currentstroke}{rgb}{1.000000,1.000000,1.000000}%
\pgfsetstrokecolor{currentstroke}%
\pgfsetdash{}{0pt}%
\pgfpathmoveto{\pgfqpoint{6.114065in}{7.438633in}}%
\pgfpathcurveto{\pgfqpoint{6.125115in}{7.438633in}}{\pgfqpoint{6.135714in}{7.443023in}}{\pgfqpoint{6.143528in}{7.450837in}}%
\pgfpathcurveto{\pgfqpoint{6.151342in}{7.458650in}}{\pgfqpoint{6.155732in}{7.469249in}}{\pgfqpoint{6.155732in}{7.480299in}}%
\pgfpathcurveto{\pgfqpoint{6.155732in}{7.491349in}}{\pgfqpoint{6.151342in}{7.501949in}}{\pgfqpoint{6.143528in}{7.509762in}}%
\pgfpathcurveto{\pgfqpoint{6.135714in}{7.517576in}}{\pgfqpoint{6.125115in}{7.521966in}}{\pgfqpoint{6.114065in}{7.521966in}}%
\pgfpathcurveto{\pgfqpoint{6.103015in}{7.521966in}}{\pgfqpoint{6.092416in}{7.517576in}}{\pgfqpoint{6.084602in}{7.509762in}}%
\pgfpathcurveto{\pgfqpoint{6.076789in}{7.501949in}}{\pgfqpoint{6.072398in}{7.491349in}}{\pgfqpoint{6.072398in}{7.480299in}}%
\pgfpathcurveto{\pgfqpoint{6.072398in}{7.469249in}}{\pgfqpoint{6.076789in}{7.458650in}}{\pgfqpoint{6.084602in}{7.450837in}}%
\pgfpathcurveto{\pgfqpoint{6.092416in}{7.443023in}}{\pgfqpoint{6.103015in}{7.438633in}}{\pgfqpoint{6.114065in}{7.438633in}}%
\pgfpathclose%
\pgfusepath{stroke,fill}%
\end{pgfscope}%
\begin{pgfscope}%
\pgfpathrectangle{\pgfqpoint{0.481978in}{0.331635in}}{\pgfqpoint{9.300000in}{7.700000in}}%
\pgfusepath{clip}%
\pgfsetbuttcap%
\pgfsetroundjoin%
\definecolor{currentfill}{rgb}{1.000000,0.705882,0.509804}%
\pgfsetfillcolor{currentfill}%
\pgfsetlinewidth{0.481800pt}%
\definecolor{currentstroke}{rgb}{1.000000,1.000000,1.000000}%
\pgfsetstrokecolor{currentstroke}%
\pgfsetdash{}{0pt}%
\pgfpathmoveto{\pgfqpoint{3.154468in}{2.446661in}}%
\pgfpathcurveto{\pgfqpoint{3.165519in}{2.446661in}}{\pgfqpoint{3.176118in}{2.451052in}}{\pgfqpoint{3.183931in}{2.458865in}}%
\pgfpathcurveto{\pgfqpoint{3.191745in}{2.466679in}}{\pgfqpoint{3.196135in}{2.477278in}}{\pgfqpoint{3.196135in}{2.488328in}}%
\pgfpathcurveto{\pgfqpoint{3.196135in}{2.499378in}}{\pgfqpoint{3.191745in}{2.509977in}}{\pgfqpoint{3.183931in}{2.517791in}}%
\pgfpathcurveto{\pgfqpoint{3.176118in}{2.525604in}}{\pgfqpoint{3.165519in}{2.529995in}}{\pgfqpoint{3.154468in}{2.529995in}}%
\pgfpathcurveto{\pgfqpoint{3.143418in}{2.529995in}}{\pgfqpoint{3.132819in}{2.525604in}}{\pgfqpoint{3.125006in}{2.517791in}}%
\pgfpathcurveto{\pgfqpoint{3.117192in}{2.509977in}}{\pgfqpoint{3.112802in}{2.499378in}}{\pgfqpoint{3.112802in}{2.488328in}}%
\pgfpathcurveto{\pgfqpoint{3.112802in}{2.477278in}}{\pgfqpoint{3.117192in}{2.466679in}}{\pgfqpoint{3.125006in}{2.458865in}}%
\pgfpathcurveto{\pgfqpoint{3.132819in}{2.451052in}}{\pgfqpoint{3.143418in}{2.446661in}}{\pgfqpoint{3.154468in}{2.446661in}}%
\pgfpathclose%
\pgfusepath{stroke,fill}%
\end{pgfscope}%
\begin{pgfscope}%
\pgfpathrectangle{\pgfqpoint{0.481978in}{0.331635in}}{\pgfqpoint{9.300000in}{7.700000in}}%
\pgfusepath{clip}%
\pgfsetbuttcap%
\pgfsetroundjoin%
\definecolor{currentfill}{rgb}{1.000000,0.705882,0.509804}%
\pgfsetfillcolor{currentfill}%
\pgfsetlinewidth{0.481800pt}%
\definecolor{currentstroke}{rgb}{1.000000,1.000000,1.000000}%
\pgfsetstrokecolor{currentstroke}%
\pgfsetdash{}{0pt}%
\pgfpathmoveto{\pgfqpoint{1.545207in}{2.058337in}}%
\pgfpathcurveto{\pgfqpoint{1.556257in}{2.058337in}}{\pgfqpoint{1.566856in}{2.062727in}}{\pgfqpoint{1.574670in}{2.070541in}}%
\pgfpathcurveto{\pgfqpoint{1.582484in}{2.078355in}}{\pgfqpoint{1.586874in}{2.088954in}}{\pgfqpoint{1.586874in}{2.100004in}}%
\pgfpathcurveto{\pgfqpoint{1.586874in}{2.111054in}}{\pgfqpoint{1.582484in}{2.121653in}}{\pgfqpoint{1.574670in}{2.129467in}}%
\pgfpathcurveto{\pgfqpoint{1.566856in}{2.137280in}}{\pgfqpoint{1.556257in}{2.141671in}}{\pgfqpoint{1.545207in}{2.141671in}}%
\pgfpathcurveto{\pgfqpoint{1.534157in}{2.141671in}}{\pgfqpoint{1.523558in}{2.137280in}}{\pgfqpoint{1.515744in}{2.129467in}}%
\pgfpathcurveto{\pgfqpoint{1.507931in}{2.121653in}}{\pgfqpoint{1.503541in}{2.111054in}}{\pgfqpoint{1.503541in}{2.100004in}}%
\pgfpathcurveto{\pgfqpoint{1.503541in}{2.088954in}}{\pgfqpoint{1.507931in}{2.078355in}}{\pgfqpoint{1.515744in}{2.070541in}}%
\pgfpathcurveto{\pgfqpoint{1.523558in}{2.062727in}}{\pgfqpoint{1.534157in}{2.058337in}}{\pgfqpoint{1.545207in}{2.058337in}}%
\pgfpathclose%
\pgfusepath{stroke,fill}%
\end{pgfscope}%
\begin{pgfscope}%
\pgfpathrectangle{\pgfqpoint{0.481978in}{0.331635in}}{\pgfqpoint{9.300000in}{7.700000in}}%
\pgfusepath{clip}%
\pgfsetbuttcap%
\pgfsetroundjoin%
\definecolor{currentfill}{rgb}{1.000000,0.705882,0.509804}%
\pgfsetfillcolor{currentfill}%
\pgfsetlinewidth{0.481800pt}%
\definecolor{currentstroke}{rgb}{1.000000,1.000000,1.000000}%
\pgfsetstrokecolor{currentstroke}%
\pgfsetdash{}{0pt}%
\pgfpathmoveto{\pgfqpoint{3.307338in}{3.365821in}}%
\pgfpathcurveto{\pgfqpoint{3.318388in}{3.365821in}}{\pgfqpoint{3.328987in}{3.370211in}}{\pgfqpoint{3.336801in}{3.378025in}}%
\pgfpathcurveto{\pgfqpoint{3.344614in}{3.385838in}}{\pgfqpoint{3.349005in}{3.396437in}}{\pgfqpoint{3.349005in}{3.407488in}}%
\pgfpathcurveto{\pgfqpoint{3.349005in}{3.418538in}}{\pgfqpoint{3.344614in}{3.429137in}}{\pgfqpoint{3.336801in}{3.436950in}}%
\pgfpathcurveto{\pgfqpoint{3.328987in}{3.444764in}}{\pgfqpoint{3.318388in}{3.449154in}}{\pgfqpoint{3.307338in}{3.449154in}}%
\pgfpathcurveto{\pgfqpoint{3.296288in}{3.449154in}}{\pgfqpoint{3.285689in}{3.444764in}}{\pgfqpoint{3.277875in}{3.436950in}}%
\pgfpathcurveto{\pgfqpoint{3.270062in}{3.429137in}}{\pgfqpoint{3.265671in}{3.418538in}}{\pgfqpoint{3.265671in}{3.407488in}}%
\pgfpathcurveto{\pgfqpoint{3.265671in}{3.396437in}}{\pgfqpoint{3.270062in}{3.385838in}}{\pgfqpoint{3.277875in}{3.378025in}}%
\pgfpathcurveto{\pgfqpoint{3.285689in}{3.370211in}}{\pgfqpoint{3.296288in}{3.365821in}}{\pgfqpoint{3.307338in}{3.365821in}}%
\pgfpathclose%
\pgfusepath{stroke,fill}%
\end{pgfscope}%
\begin{pgfscope}%
\pgfpathrectangle{\pgfqpoint{0.481978in}{0.331635in}}{\pgfqpoint{9.300000in}{7.700000in}}%
\pgfusepath{clip}%
\pgfsetbuttcap%
\pgfsetroundjoin%
\definecolor{currentfill}{rgb}{1.000000,0.705882,0.509804}%
\pgfsetfillcolor{currentfill}%
\pgfsetlinewidth{0.481800pt}%
\definecolor{currentstroke}{rgb}{1.000000,1.000000,1.000000}%
\pgfsetstrokecolor{currentstroke}%
\pgfsetdash{}{0pt}%
\pgfpathmoveto{\pgfqpoint{5.143128in}{7.146530in}}%
\pgfpathcurveto{\pgfqpoint{5.154178in}{7.146530in}}{\pgfqpoint{5.164777in}{7.150920in}}{\pgfqpoint{5.172591in}{7.158734in}}%
\pgfpathcurveto{\pgfqpoint{5.180405in}{7.166547in}}{\pgfqpoint{5.184795in}{7.177146in}}{\pgfqpoint{5.184795in}{7.188196in}}%
\pgfpathcurveto{\pgfqpoint{5.184795in}{7.199247in}}{\pgfqpoint{5.180405in}{7.209846in}}{\pgfqpoint{5.172591in}{7.217659in}}%
\pgfpathcurveto{\pgfqpoint{5.164777in}{7.225473in}}{\pgfqpoint{5.154178in}{7.229863in}}{\pgfqpoint{5.143128in}{7.229863in}}%
\pgfpathcurveto{\pgfqpoint{5.132078in}{7.229863in}}{\pgfqpoint{5.121479in}{7.225473in}}{\pgfqpoint{5.113665in}{7.217659in}}%
\pgfpathcurveto{\pgfqpoint{5.105852in}{7.209846in}}{\pgfqpoint{5.101461in}{7.199247in}}{\pgfqpoint{5.101461in}{7.188196in}}%
\pgfpathcurveto{\pgfqpoint{5.101461in}{7.177146in}}{\pgfqpoint{5.105852in}{7.166547in}}{\pgfqpoint{5.113665in}{7.158734in}}%
\pgfpathcurveto{\pgfqpoint{5.121479in}{7.150920in}}{\pgfqpoint{5.132078in}{7.146530in}}{\pgfqpoint{5.143128in}{7.146530in}}%
\pgfpathclose%
\pgfusepath{stroke,fill}%
\end{pgfscope}%
\begin{pgfscope}%
\pgfpathrectangle{\pgfqpoint{0.481978in}{0.331635in}}{\pgfqpoint{9.300000in}{7.700000in}}%
\pgfusepath{clip}%
\pgfsetbuttcap%
\pgfsetroundjoin%
\definecolor{currentfill}{rgb}{1.000000,0.705882,0.509804}%
\pgfsetfillcolor{currentfill}%
\pgfsetlinewidth{0.481800pt}%
\definecolor{currentstroke}{rgb}{1.000000,1.000000,1.000000}%
\pgfsetstrokecolor{currentstroke}%
\pgfsetdash{}{0pt}%
\pgfpathmoveto{\pgfqpoint{3.238886in}{1.903284in}}%
\pgfpathcurveto{\pgfqpoint{3.249936in}{1.903284in}}{\pgfqpoint{3.260535in}{1.907675in}}{\pgfqpoint{3.268349in}{1.915488in}}%
\pgfpathcurveto{\pgfqpoint{3.276162in}{1.923302in}}{\pgfqpoint{3.280553in}{1.933901in}}{\pgfqpoint{3.280553in}{1.944951in}}%
\pgfpathcurveto{\pgfqpoint{3.280553in}{1.956001in}}{\pgfqpoint{3.276162in}{1.966600in}}{\pgfqpoint{3.268349in}{1.974414in}}%
\pgfpathcurveto{\pgfqpoint{3.260535in}{1.982227in}}{\pgfqpoint{3.249936in}{1.986618in}}{\pgfqpoint{3.238886in}{1.986618in}}%
\pgfpathcurveto{\pgfqpoint{3.227836in}{1.986618in}}{\pgfqpoint{3.217237in}{1.982227in}}{\pgfqpoint{3.209423in}{1.974414in}}%
\pgfpathcurveto{\pgfqpoint{3.201610in}{1.966600in}}{\pgfqpoint{3.197219in}{1.956001in}}{\pgfqpoint{3.197219in}{1.944951in}}%
\pgfpathcurveto{\pgfqpoint{3.197219in}{1.933901in}}{\pgfqpoint{3.201610in}{1.923302in}}{\pgfqpoint{3.209423in}{1.915488in}}%
\pgfpathcurveto{\pgfqpoint{3.217237in}{1.907675in}}{\pgfqpoint{3.227836in}{1.903284in}}{\pgfqpoint{3.238886in}{1.903284in}}%
\pgfpathclose%
\pgfusepath{stroke,fill}%
\end{pgfscope}%
\begin{pgfscope}%
\pgfpathrectangle{\pgfqpoint{0.481978in}{0.331635in}}{\pgfqpoint{9.300000in}{7.700000in}}%
\pgfusepath{clip}%
\pgfsetbuttcap%
\pgfsetroundjoin%
\definecolor{currentfill}{rgb}{1.000000,0.705882,0.509804}%
\pgfsetfillcolor{currentfill}%
\pgfsetlinewidth{0.481800pt}%
\definecolor{currentstroke}{rgb}{1.000000,1.000000,1.000000}%
\pgfsetstrokecolor{currentstroke}%
\pgfsetdash{}{0pt}%
\pgfpathmoveto{\pgfqpoint{6.096137in}{1.233341in}}%
\pgfpathcurveto{\pgfqpoint{6.107187in}{1.233341in}}{\pgfqpoint{6.117786in}{1.237731in}}{\pgfqpoint{6.125600in}{1.245545in}}%
\pgfpathcurveto{\pgfqpoint{6.133413in}{1.253359in}}{\pgfqpoint{6.137804in}{1.263958in}}{\pgfqpoint{6.137804in}{1.275008in}}%
\pgfpathcurveto{\pgfqpoint{6.137804in}{1.286058in}}{\pgfqpoint{6.133413in}{1.296657in}}{\pgfqpoint{6.125600in}{1.304471in}}%
\pgfpathcurveto{\pgfqpoint{6.117786in}{1.312284in}}{\pgfqpoint{6.107187in}{1.316675in}}{\pgfqpoint{6.096137in}{1.316675in}}%
\pgfpathcurveto{\pgfqpoint{6.085087in}{1.316675in}}{\pgfqpoint{6.074488in}{1.312284in}}{\pgfqpoint{6.066674in}{1.304471in}}%
\pgfpathcurveto{\pgfqpoint{6.058861in}{1.296657in}}{\pgfqpoint{6.054470in}{1.286058in}}{\pgfqpoint{6.054470in}{1.275008in}}%
\pgfpathcurveto{\pgfqpoint{6.054470in}{1.263958in}}{\pgfqpoint{6.058861in}{1.253359in}}{\pgfqpoint{6.066674in}{1.245545in}}%
\pgfpathcurveto{\pgfqpoint{6.074488in}{1.237731in}}{\pgfqpoint{6.085087in}{1.233341in}}{\pgfqpoint{6.096137in}{1.233341in}}%
\pgfpathclose%
\pgfusepath{stroke,fill}%
\end{pgfscope}%
\begin{pgfscope}%
\pgfpathrectangle{\pgfqpoint{0.481978in}{0.331635in}}{\pgfqpoint{9.300000in}{7.700000in}}%
\pgfusepath{clip}%
\pgfsetbuttcap%
\pgfsetroundjoin%
\definecolor{currentfill}{rgb}{1.000000,0.705882,0.509804}%
\pgfsetfillcolor{currentfill}%
\pgfsetlinewidth{0.481800pt}%
\definecolor{currentstroke}{rgb}{1.000000,1.000000,1.000000}%
\pgfsetstrokecolor{currentstroke}%
\pgfsetdash{}{0pt}%
\pgfpathmoveto{\pgfqpoint{8.892538in}{6.000024in}}%
\pgfpathcurveto{\pgfqpoint{8.903589in}{6.000024in}}{\pgfqpoint{8.914188in}{6.004414in}}{\pgfqpoint{8.922001in}{6.012228in}}%
\pgfpathcurveto{\pgfqpoint{8.929815in}{6.020041in}}{\pgfqpoint{8.934205in}{6.030640in}}{\pgfqpoint{8.934205in}{6.041691in}}%
\pgfpathcurveto{\pgfqpoint{8.934205in}{6.052741in}}{\pgfqpoint{8.929815in}{6.063340in}}{\pgfqpoint{8.922001in}{6.071153in}}%
\pgfpathcurveto{\pgfqpoint{8.914188in}{6.078967in}}{\pgfqpoint{8.903589in}{6.083357in}}{\pgfqpoint{8.892538in}{6.083357in}}%
\pgfpathcurveto{\pgfqpoint{8.881488in}{6.083357in}}{\pgfqpoint{8.870889in}{6.078967in}}{\pgfqpoint{8.863076in}{6.071153in}}%
\pgfpathcurveto{\pgfqpoint{8.855262in}{6.063340in}}{\pgfqpoint{8.850872in}{6.052741in}}{\pgfqpoint{8.850872in}{6.041691in}}%
\pgfpathcurveto{\pgfqpoint{8.850872in}{6.030640in}}{\pgfqpoint{8.855262in}{6.020041in}}{\pgfqpoint{8.863076in}{6.012228in}}%
\pgfpathcurveto{\pgfqpoint{8.870889in}{6.004414in}}{\pgfqpoint{8.881488in}{6.000024in}}{\pgfqpoint{8.892538in}{6.000024in}}%
\pgfpathclose%
\pgfusepath{stroke,fill}%
\end{pgfscope}%
\begin{pgfscope}%
\pgfpathrectangle{\pgfqpoint{0.481978in}{0.331635in}}{\pgfqpoint{9.300000in}{7.700000in}}%
\pgfusepath{clip}%
\pgfsetbuttcap%
\pgfsetroundjoin%
\definecolor{currentfill}{rgb}{1.000000,0.705882,0.509804}%
\pgfsetfillcolor{currentfill}%
\pgfsetlinewidth{0.481800pt}%
\definecolor{currentstroke}{rgb}{1.000000,1.000000,1.000000}%
\pgfsetstrokecolor{currentstroke}%
\pgfsetdash{}{0pt}%
\pgfpathmoveto{\pgfqpoint{8.792161in}{5.432877in}}%
\pgfpathcurveto{\pgfqpoint{8.803212in}{5.432877in}}{\pgfqpoint{8.813811in}{5.437267in}}{\pgfqpoint{8.821624in}{5.445081in}}%
\pgfpathcurveto{\pgfqpoint{8.829438in}{5.452895in}}{\pgfqpoint{8.833828in}{5.463494in}}{\pgfqpoint{8.833828in}{5.474544in}}%
\pgfpathcurveto{\pgfqpoint{8.833828in}{5.485594in}}{\pgfqpoint{8.829438in}{5.496193in}}{\pgfqpoint{8.821624in}{5.504006in}}%
\pgfpathcurveto{\pgfqpoint{8.813811in}{5.511820in}}{\pgfqpoint{8.803212in}{5.516210in}}{\pgfqpoint{8.792161in}{5.516210in}}%
\pgfpathcurveto{\pgfqpoint{8.781111in}{5.516210in}}{\pgfqpoint{8.770512in}{5.511820in}}{\pgfqpoint{8.762699in}{5.504006in}}%
\pgfpathcurveto{\pgfqpoint{8.754885in}{5.496193in}}{\pgfqpoint{8.750495in}{5.485594in}}{\pgfqpoint{8.750495in}{5.474544in}}%
\pgfpathcurveto{\pgfqpoint{8.750495in}{5.463494in}}{\pgfqpoint{8.754885in}{5.452895in}}{\pgfqpoint{8.762699in}{5.445081in}}%
\pgfpathcurveto{\pgfqpoint{8.770512in}{5.437267in}}{\pgfqpoint{8.781111in}{5.432877in}}{\pgfqpoint{8.792161in}{5.432877in}}%
\pgfpathclose%
\pgfusepath{stroke,fill}%
\end{pgfscope}%
\begin{pgfscope}%
\pgfpathrectangle{\pgfqpoint{0.481978in}{0.331635in}}{\pgfqpoint{9.300000in}{7.700000in}}%
\pgfusepath{clip}%
\pgfsetbuttcap%
\pgfsetroundjoin%
\definecolor{currentfill}{rgb}{1.000000,0.705882,0.509804}%
\pgfsetfillcolor{currentfill}%
\pgfsetlinewidth{0.481800pt}%
\definecolor{currentstroke}{rgb}{1.000000,1.000000,1.000000}%
\pgfsetstrokecolor{currentstroke}%
\pgfsetdash{}{0pt}%
\pgfpathmoveto{\pgfqpoint{4.884502in}{4.767853in}}%
\pgfpathcurveto{\pgfqpoint{4.895552in}{4.767853in}}{\pgfqpoint{4.906151in}{4.772243in}}{\pgfqpoint{4.913965in}{4.780057in}}%
\pgfpathcurveto{\pgfqpoint{4.921778in}{4.787870in}}{\pgfqpoint{4.926169in}{4.798469in}}{\pgfqpoint{4.926169in}{4.809519in}}%
\pgfpathcurveto{\pgfqpoint{4.926169in}{4.820570in}}{\pgfqpoint{4.921778in}{4.831169in}}{\pgfqpoint{4.913965in}{4.838982in}}%
\pgfpathcurveto{\pgfqpoint{4.906151in}{4.846796in}}{\pgfqpoint{4.895552in}{4.851186in}}{\pgfqpoint{4.884502in}{4.851186in}}%
\pgfpathcurveto{\pgfqpoint{4.873452in}{4.851186in}}{\pgfqpoint{4.862853in}{4.846796in}}{\pgfqpoint{4.855039in}{4.838982in}}%
\pgfpathcurveto{\pgfqpoint{4.847226in}{4.831169in}}{\pgfqpoint{4.842835in}{4.820570in}}{\pgfqpoint{4.842835in}{4.809519in}}%
\pgfpathcurveto{\pgfqpoint{4.842835in}{4.798469in}}{\pgfqpoint{4.847226in}{4.787870in}}{\pgfqpoint{4.855039in}{4.780057in}}%
\pgfpathcurveto{\pgfqpoint{4.862853in}{4.772243in}}{\pgfqpoint{4.873452in}{4.767853in}}{\pgfqpoint{4.884502in}{4.767853in}}%
\pgfpathclose%
\pgfusepath{stroke,fill}%
\end{pgfscope}%
\begin{pgfscope}%
\pgfpathrectangle{\pgfqpoint{0.481978in}{0.331635in}}{\pgfqpoint{9.300000in}{7.700000in}}%
\pgfusepath{clip}%
\pgfsetbuttcap%
\pgfsetroundjoin%
\definecolor{currentfill}{rgb}{1.000000,0.705882,0.509804}%
\pgfsetfillcolor{currentfill}%
\pgfsetlinewidth{0.481800pt}%
\definecolor{currentstroke}{rgb}{1.000000,1.000000,1.000000}%
\pgfsetstrokecolor{currentstroke}%
\pgfsetdash{}{0pt}%
\pgfpathmoveto{\pgfqpoint{8.839374in}{4.799422in}}%
\pgfpathcurveto{\pgfqpoint{8.850424in}{4.799422in}}{\pgfqpoint{8.861023in}{4.803812in}}{\pgfqpoint{8.868836in}{4.811626in}}%
\pgfpathcurveto{\pgfqpoint{8.876650in}{4.819439in}}{\pgfqpoint{8.881040in}{4.830038in}}{\pgfqpoint{8.881040in}{4.841088in}}%
\pgfpathcurveto{\pgfqpoint{8.881040in}{4.852138in}}{\pgfqpoint{8.876650in}{4.862737in}}{\pgfqpoint{8.868836in}{4.870551in}}%
\pgfpathcurveto{\pgfqpoint{8.861023in}{4.878365in}}{\pgfqpoint{8.850424in}{4.882755in}}{\pgfqpoint{8.839374in}{4.882755in}}%
\pgfpathcurveto{\pgfqpoint{8.828324in}{4.882755in}}{\pgfqpoint{8.817724in}{4.878365in}}{\pgfqpoint{8.809911in}{4.870551in}}%
\pgfpathcurveto{\pgfqpoint{8.802097in}{4.862737in}}{\pgfqpoint{8.797707in}{4.852138in}}{\pgfqpoint{8.797707in}{4.841088in}}%
\pgfpathcurveto{\pgfqpoint{8.797707in}{4.830038in}}{\pgfqpoint{8.802097in}{4.819439in}}{\pgfqpoint{8.809911in}{4.811626in}}%
\pgfpathcurveto{\pgfqpoint{8.817724in}{4.803812in}}{\pgfqpoint{8.828324in}{4.799422in}}{\pgfqpoint{8.839374in}{4.799422in}}%
\pgfpathclose%
\pgfusepath{stroke,fill}%
\end{pgfscope}%
\begin{pgfscope}%
\pgfpathrectangle{\pgfqpoint{0.481978in}{0.331635in}}{\pgfqpoint{9.300000in}{7.700000in}}%
\pgfusepath{clip}%
\pgfsetbuttcap%
\pgfsetroundjoin%
\definecolor{currentfill}{rgb}{1.000000,0.705882,0.509804}%
\pgfsetfillcolor{currentfill}%
\pgfsetlinewidth{0.481800pt}%
\definecolor{currentstroke}{rgb}{1.000000,1.000000,1.000000}%
\pgfsetstrokecolor{currentstroke}%
\pgfsetdash{}{0pt}%
\pgfpathmoveto{\pgfqpoint{8.645625in}{5.963537in}}%
\pgfpathcurveto{\pgfqpoint{8.656675in}{5.963537in}}{\pgfqpoint{8.667274in}{5.967928in}}{\pgfqpoint{8.675088in}{5.975741in}}%
\pgfpathcurveto{\pgfqpoint{8.682902in}{5.983555in}}{\pgfqpoint{8.687292in}{5.994154in}}{\pgfqpoint{8.687292in}{6.005204in}}%
\pgfpathcurveto{\pgfqpoint{8.687292in}{6.016254in}}{\pgfqpoint{8.682902in}{6.026853in}}{\pgfqpoint{8.675088in}{6.034667in}}%
\pgfpathcurveto{\pgfqpoint{8.667274in}{6.042480in}}{\pgfqpoint{8.656675in}{6.046871in}}{\pgfqpoint{8.645625in}{6.046871in}}%
\pgfpathcurveto{\pgfqpoint{8.634575in}{6.046871in}}{\pgfqpoint{8.623976in}{6.042480in}}{\pgfqpoint{8.616162in}{6.034667in}}%
\pgfpathcurveto{\pgfqpoint{8.608349in}{6.026853in}}{\pgfqpoint{8.603959in}{6.016254in}}{\pgfqpoint{8.603959in}{6.005204in}}%
\pgfpathcurveto{\pgfqpoint{8.603959in}{5.994154in}}{\pgfqpoint{8.608349in}{5.983555in}}{\pgfqpoint{8.616162in}{5.975741in}}%
\pgfpathcurveto{\pgfqpoint{8.623976in}{5.967928in}}{\pgfqpoint{8.634575in}{5.963537in}}{\pgfqpoint{8.645625in}{5.963537in}}%
\pgfpathclose%
\pgfusepath{stroke,fill}%
\end{pgfscope}%
\begin{pgfscope}%
\pgfpathrectangle{\pgfqpoint{0.481978in}{0.331635in}}{\pgfqpoint{9.300000in}{7.700000in}}%
\pgfusepath{clip}%
\pgfsetbuttcap%
\pgfsetroundjoin%
\definecolor{currentfill}{rgb}{1.000000,0.705882,0.509804}%
\pgfsetfillcolor{currentfill}%
\pgfsetlinewidth{0.481800pt}%
\definecolor{currentstroke}{rgb}{1.000000,1.000000,1.000000}%
\pgfsetstrokecolor{currentstroke}%
\pgfsetdash{}{0pt}%
\pgfpathmoveto{\pgfqpoint{1.541644in}{2.029501in}}%
\pgfpathcurveto{\pgfqpoint{1.552694in}{2.029501in}}{\pgfqpoint{1.563293in}{2.033891in}}{\pgfqpoint{1.571107in}{2.041705in}}%
\pgfpathcurveto{\pgfqpoint{1.578921in}{2.049518in}}{\pgfqpoint{1.583311in}{2.060117in}}{\pgfqpoint{1.583311in}{2.071167in}}%
\pgfpathcurveto{\pgfqpoint{1.583311in}{2.082218in}}{\pgfqpoint{1.578921in}{2.092817in}}{\pgfqpoint{1.571107in}{2.100630in}}%
\pgfpathcurveto{\pgfqpoint{1.563293in}{2.108444in}}{\pgfqpoint{1.552694in}{2.112834in}}{\pgfqpoint{1.541644in}{2.112834in}}%
\pgfpathcurveto{\pgfqpoint{1.530594in}{2.112834in}}{\pgfqpoint{1.519995in}{2.108444in}}{\pgfqpoint{1.512181in}{2.100630in}}%
\pgfpathcurveto{\pgfqpoint{1.504368in}{2.092817in}}{\pgfqpoint{1.499978in}{2.082218in}}{\pgfqpoint{1.499978in}{2.071167in}}%
\pgfpathcurveto{\pgfqpoint{1.499978in}{2.060117in}}{\pgfqpoint{1.504368in}{2.049518in}}{\pgfqpoint{1.512181in}{2.041705in}}%
\pgfpathcurveto{\pgfqpoint{1.519995in}{2.033891in}}{\pgfqpoint{1.530594in}{2.029501in}}{\pgfqpoint{1.541644in}{2.029501in}}%
\pgfpathclose%
\pgfusepath{stroke,fill}%
\end{pgfscope}%
\begin{pgfscope}%
\pgfpathrectangle{\pgfqpoint{0.481978in}{0.331635in}}{\pgfqpoint{9.300000in}{7.700000in}}%
\pgfusepath{clip}%
\pgfsetbuttcap%
\pgfsetroundjoin%
\definecolor{currentfill}{rgb}{1.000000,0.705882,0.509804}%
\pgfsetfillcolor{currentfill}%
\pgfsetlinewidth{0.481800pt}%
\definecolor{currentstroke}{rgb}{1.000000,1.000000,1.000000}%
\pgfsetstrokecolor{currentstroke}%
\pgfsetdash{}{0pt}%
\pgfpathmoveto{\pgfqpoint{8.938916in}{5.368764in}}%
\pgfpathcurveto{\pgfqpoint{8.949966in}{5.368764in}}{\pgfqpoint{8.960565in}{5.373154in}}{\pgfqpoint{8.968379in}{5.380967in}}%
\pgfpathcurveto{\pgfqpoint{8.976193in}{5.388781in}}{\pgfqpoint{8.980583in}{5.399380in}}{\pgfqpoint{8.980583in}{5.410430in}}%
\pgfpathcurveto{\pgfqpoint{8.980583in}{5.421480in}}{\pgfqpoint{8.976193in}{5.432079in}}{\pgfqpoint{8.968379in}{5.439893in}}%
\pgfpathcurveto{\pgfqpoint{8.960565in}{5.447707in}}{\pgfqpoint{8.949966in}{5.452097in}}{\pgfqpoint{8.938916in}{5.452097in}}%
\pgfpathcurveto{\pgfqpoint{8.927866in}{5.452097in}}{\pgfqpoint{8.917267in}{5.447707in}}{\pgfqpoint{8.909453in}{5.439893in}}%
\pgfpathcurveto{\pgfqpoint{8.901640in}{5.432079in}}{\pgfqpoint{8.897250in}{5.421480in}}{\pgfqpoint{8.897250in}{5.410430in}}%
\pgfpathcurveto{\pgfqpoint{8.897250in}{5.399380in}}{\pgfqpoint{8.901640in}{5.388781in}}{\pgfqpoint{8.909453in}{5.380967in}}%
\pgfpathcurveto{\pgfqpoint{8.917267in}{5.373154in}}{\pgfqpoint{8.927866in}{5.368764in}}{\pgfqpoint{8.938916in}{5.368764in}}%
\pgfpathclose%
\pgfusepath{stroke,fill}%
\end{pgfscope}%
\begin{pgfscope}%
\pgfpathrectangle{\pgfqpoint{0.481978in}{0.331635in}}{\pgfqpoint{9.300000in}{7.700000in}}%
\pgfusepath{clip}%
\pgfsetbuttcap%
\pgfsetroundjoin%
\definecolor{currentfill}{rgb}{1.000000,0.705882,0.509804}%
\pgfsetfillcolor{currentfill}%
\pgfsetlinewidth{0.481800pt}%
\definecolor{currentstroke}{rgb}{1.000000,1.000000,1.000000}%
\pgfsetstrokecolor{currentstroke}%
\pgfsetdash{}{0pt}%
\pgfpathmoveto{\pgfqpoint{3.682482in}{4.956310in}}%
\pgfpathcurveto{\pgfqpoint{3.693532in}{4.956310in}}{\pgfqpoint{3.704131in}{4.960700in}}{\pgfqpoint{3.711945in}{4.968514in}}%
\pgfpathcurveto{\pgfqpoint{3.719759in}{4.976328in}}{\pgfqpoint{3.724149in}{4.986927in}}{\pgfqpoint{3.724149in}{4.997977in}}%
\pgfpathcurveto{\pgfqpoint{3.724149in}{5.009027in}}{\pgfqpoint{3.719759in}{5.019626in}}{\pgfqpoint{3.711945in}{5.027440in}}%
\pgfpathcurveto{\pgfqpoint{3.704131in}{5.035253in}}{\pgfqpoint{3.693532in}{5.039644in}}{\pgfqpoint{3.682482in}{5.039644in}}%
\pgfpathcurveto{\pgfqpoint{3.671432in}{5.039644in}}{\pgfqpoint{3.660833in}{5.035253in}}{\pgfqpoint{3.653019in}{5.027440in}}%
\pgfpathcurveto{\pgfqpoint{3.645206in}{5.019626in}}{\pgfqpoint{3.640816in}{5.009027in}}{\pgfqpoint{3.640816in}{4.997977in}}%
\pgfpathcurveto{\pgfqpoint{3.640816in}{4.986927in}}{\pgfqpoint{3.645206in}{4.976328in}}{\pgfqpoint{3.653019in}{4.968514in}}%
\pgfpathcurveto{\pgfqpoint{3.660833in}{4.960700in}}{\pgfqpoint{3.671432in}{4.956310in}}{\pgfqpoint{3.682482in}{4.956310in}}%
\pgfpathclose%
\pgfusepath{stroke,fill}%
\end{pgfscope}%
\begin{pgfscope}%
\pgfpathrectangle{\pgfqpoint{0.481978in}{0.331635in}}{\pgfqpoint{9.300000in}{7.700000in}}%
\pgfusepath{clip}%
\pgfsetbuttcap%
\pgfsetroundjoin%
\definecolor{currentfill}{rgb}{1.000000,0.705882,0.509804}%
\pgfsetfillcolor{currentfill}%
\pgfsetlinewidth{0.481800pt}%
\definecolor{currentstroke}{rgb}{1.000000,1.000000,1.000000}%
\pgfsetstrokecolor{currentstroke}%
\pgfsetdash{}{0pt}%
\pgfpathmoveto{\pgfqpoint{7.449113in}{2.447310in}}%
\pgfpathcurveto{\pgfqpoint{7.460163in}{2.447310in}}{\pgfqpoint{7.470762in}{2.451700in}}{\pgfqpoint{7.478576in}{2.459514in}}%
\pgfpathcurveto{\pgfqpoint{7.486390in}{2.467327in}}{\pgfqpoint{7.490780in}{2.477927in}}{\pgfqpoint{7.490780in}{2.488977in}}%
\pgfpathcurveto{\pgfqpoint{7.490780in}{2.500027in}}{\pgfqpoint{7.486390in}{2.510626in}}{\pgfqpoint{7.478576in}{2.518439in}}%
\pgfpathcurveto{\pgfqpoint{7.470762in}{2.526253in}}{\pgfqpoint{7.460163in}{2.530643in}}{\pgfqpoint{7.449113in}{2.530643in}}%
\pgfpathcurveto{\pgfqpoint{7.438063in}{2.530643in}}{\pgfqpoint{7.427464in}{2.526253in}}{\pgfqpoint{7.419650in}{2.518439in}}%
\pgfpathcurveto{\pgfqpoint{7.411837in}{2.510626in}}{\pgfqpoint{7.407447in}{2.500027in}}{\pgfqpoint{7.407447in}{2.488977in}}%
\pgfpathcurveto{\pgfqpoint{7.407447in}{2.477927in}}{\pgfqpoint{7.411837in}{2.467327in}}{\pgfqpoint{7.419650in}{2.459514in}}%
\pgfpathcurveto{\pgfqpoint{7.427464in}{2.451700in}}{\pgfqpoint{7.438063in}{2.447310in}}{\pgfqpoint{7.449113in}{2.447310in}}%
\pgfpathclose%
\pgfusepath{stroke,fill}%
\end{pgfscope}%
\begin{pgfscope}%
\pgfpathrectangle{\pgfqpoint{0.481978in}{0.331635in}}{\pgfqpoint{9.300000in}{7.700000in}}%
\pgfusepath{clip}%
\pgfsetbuttcap%
\pgfsetroundjoin%
\definecolor{currentfill}{rgb}{1.000000,0.705882,0.509804}%
\pgfsetfillcolor{currentfill}%
\pgfsetlinewidth{0.481800pt}%
\definecolor{currentstroke}{rgb}{1.000000,1.000000,1.000000}%
\pgfsetstrokecolor{currentstroke}%
\pgfsetdash{}{0pt}%
\pgfpathmoveto{\pgfqpoint{8.325442in}{5.151331in}}%
\pgfpathcurveto{\pgfqpoint{8.336492in}{5.151331in}}{\pgfqpoint{8.347091in}{5.155721in}}{\pgfqpoint{8.354904in}{5.163535in}}%
\pgfpathcurveto{\pgfqpoint{8.362718in}{5.171348in}}{\pgfqpoint{8.367108in}{5.181947in}}{\pgfqpoint{8.367108in}{5.192998in}}%
\pgfpathcurveto{\pgfqpoint{8.367108in}{5.204048in}}{\pgfqpoint{8.362718in}{5.214647in}}{\pgfqpoint{8.354904in}{5.222460in}}%
\pgfpathcurveto{\pgfqpoint{8.347091in}{5.230274in}}{\pgfqpoint{8.336492in}{5.234664in}}{\pgfqpoint{8.325442in}{5.234664in}}%
\pgfpathcurveto{\pgfqpoint{8.314392in}{5.234664in}}{\pgfqpoint{8.303793in}{5.230274in}}{\pgfqpoint{8.295979in}{5.222460in}}%
\pgfpathcurveto{\pgfqpoint{8.288165in}{5.214647in}}{\pgfqpoint{8.283775in}{5.204048in}}{\pgfqpoint{8.283775in}{5.192998in}}%
\pgfpathcurveto{\pgfqpoint{8.283775in}{5.181947in}}{\pgfqpoint{8.288165in}{5.171348in}}{\pgfqpoint{8.295979in}{5.163535in}}%
\pgfpathcurveto{\pgfqpoint{8.303793in}{5.155721in}}{\pgfqpoint{8.314392in}{5.151331in}}{\pgfqpoint{8.325442in}{5.151331in}}%
\pgfpathclose%
\pgfusepath{stroke,fill}%
\end{pgfscope}%
\begin{pgfscope}%
\pgfpathrectangle{\pgfqpoint{0.481978in}{0.331635in}}{\pgfqpoint{9.300000in}{7.700000in}}%
\pgfusepath{clip}%
\pgfsetbuttcap%
\pgfsetroundjoin%
\definecolor{currentfill}{rgb}{1.000000,0.705882,0.509804}%
\pgfsetfillcolor{currentfill}%
\pgfsetlinewidth{0.481800pt}%
\definecolor{currentstroke}{rgb}{1.000000,1.000000,1.000000}%
\pgfsetstrokecolor{currentstroke}%
\pgfsetdash{}{0pt}%
\pgfpathmoveto{\pgfqpoint{7.578449in}{4.350693in}}%
\pgfpathcurveto{\pgfqpoint{7.589499in}{4.350693in}}{\pgfqpoint{7.600098in}{4.355084in}}{\pgfqpoint{7.607912in}{4.362897in}}%
\pgfpathcurveto{\pgfqpoint{7.615726in}{4.370711in}}{\pgfqpoint{7.620116in}{4.381310in}}{\pgfqpoint{7.620116in}{4.392360in}}%
\pgfpathcurveto{\pgfqpoint{7.620116in}{4.403410in}}{\pgfqpoint{7.615726in}{4.414009in}}{\pgfqpoint{7.607912in}{4.421823in}}%
\pgfpathcurveto{\pgfqpoint{7.600098in}{4.429636in}}{\pgfqpoint{7.589499in}{4.434027in}}{\pgfqpoint{7.578449in}{4.434027in}}%
\pgfpathcurveto{\pgfqpoint{7.567399in}{4.434027in}}{\pgfqpoint{7.556800in}{4.429636in}}{\pgfqpoint{7.548986in}{4.421823in}}%
\pgfpathcurveto{\pgfqpoint{7.541173in}{4.414009in}}{\pgfqpoint{7.536783in}{4.403410in}}{\pgfqpoint{7.536783in}{4.392360in}}%
\pgfpathcurveto{\pgfqpoint{7.536783in}{4.381310in}}{\pgfqpoint{7.541173in}{4.370711in}}{\pgfqpoint{7.548986in}{4.362897in}}%
\pgfpathcurveto{\pgfqpoint{7.556800in}{4.355084in}}{\pgfqpoint{7.567399in}{4.350693in}}{\pgfqpoint{7.578449in}{4.350693in}}%
\pgfpathclose%
\pgfusepath{stroke,fill}%
\end{pgfscope}%
\begin{pgfscope}%
\pgfpathrectangle{\pgfqpoint{0.481978in}{0.331635in}}{\pgfqpoint{9.300000in}{7.700000in}}%
\pgfusepath{clip}%
\pgfsetbuttcap%
\pgfsetroundjoin%
\definecolor{currentfill}{rgb}{1.000000,0.705882,0.509804}%
\pgfsetfillcolor{currentfill}%
\pgfsetlinewidth{0.481800pt}%
\definecolor{currentstroke}{rgb}{1.000000,1.000000,1.000000}%
\pgfsetstrokecolor{currentstroke}%
\pgfsetdash{}{0pt}%
\pgfpathmoveto{\pgfqpoint{6.440915in}{3.359023in}}%
\pgfpathcurveto{\pgfqpoint{6.451966in}{3.359023in}}{\pgfqpoint{6.462565in}{3.363413in}}{\pgfqpoint{6.470378in}{3.371227in}}%
\pgfpathcurveto{\pgfqpoint{6.478192in}{3.379041in}}{\pgfqpoint{6.482582in}{3.389640in}}{\pgfqpoint{6.482582in}{3.400690in}}%
\pgfpathcurveto{\pgfqpoint{6.482582in}{3.411740in}}{\pgfqpoint{6.478192in}{3.422339in}}{\pgfqpoint{6.470378in}{3.430152in}}%
\pgfpathcurveto{\pgfqpoint{6.462565in}{3.437966in}}{\pgfqpoint{6.451966in}{3.442356in}}{\pgfqpoint{6.440915in}{3.442356in}}%
\pgfpathcurveto{\pgfqpoint{6.429865in}{3.442356in}}{\pgfqpoint{6.419266in}{3.437966in}}{\pgfqpoint{6.411453in}{3.430152in}}%
\pgfpathcurveto{\pgfqpoint{6.403639in}{3.422339in}}{\pgfqpoint{6.399249in}{3.411740in}}{\pgfqpoint{6.399249in}{3.400690in}}%
\pgfpathcurveto{\pgfqpoint{6.399249in}{3.389640in}}{\pgfqpoint{6.403639in}{3.379041in}}{\pgfqpoint{6.411453in}{3.371227in}}%
\pgfpathcurveto{\pgfqpoint{6.419266in}{3.363413in}}{\pgfqpoint{6.429865in}{3.359023in}}{\pgfqpoint{6.440915in}{3.359023in}}%
\pgfpathclose%
\pgfusepath{stroke,fill}%
\end{pgfscope}%
\begin{pgfscope}%
\pgfpathrectangle{\pgfqpoint{0.481978in}{0.331635in}}{\pgfqpoint{9.300000in}{7.700000in}}%
\pgfusepath{clip}%
\pgfsetbuttcap%
\pgfsetroundjoin%
\definecolor{currentfill}{rgb}{1.000000,0.705882,0.509804}%
\pgfsetfillcolor{currentfill}%
\pgfsetlinewidth{0.481800pt}%
\definecolor{currentstroke}{rgb}{1.000000,1.000000,1.000000}%
\pgfsetstrokecolor{currentstroke}%
\pgfsetdash{}{0pt}%
\pgfpathmoveto{\pgfqpoint{3.469279in}{3.827559in}}%
\pgfpathcurveto{\pgfqpoint{3.480329in}{3.827559in}}{\pgfqpoint{3.490928in}{3.831949in}}{\pgfqpoint{3.498742in}{3.839763in}}%
\pgfpathcurveto{\pgfqpoint{3.506555in}{3.847576in}}{\pgfqpoint{3.510946in}{3.858175in}}{\pgfqpoint{3.510946in}{3.869225in}}%
\pgfpathcurveto{\pgfqpoint{3.510946in}{3.880275in}}{\pgfqpoint{3.506555in}{3.890874in}}{\pgfqpoint{3.498742in}{3.898688in}}%
\pgfpathcurveto{\pgfqpoint{3.490928in}{3.906502in}}{\pgfqpoint{3.480329in}{3.910892in}}{\pgfqpoint{3.469279in}{3.910892in}}%
\pgfpathcurveto{\pgfqpoint{3.458229in}{3.910892in}}{\pgfqpoint{3.447630in}{3.906502in}}{\pgfqpoint{3.439816in}{3.898688in}}%
\pgfpathcurveto{\pgfqpoint{3.432003in}{3.890874in}}{\pgfqpoint{3.427612in}{3.880275in}}{\pgfqpoint{3.427612in}{3.869225in}}%
\pgfpathcurveto{\pgfqpoint{3.427612in}{3.858175in}}{\pgfqpoint{3.432003in}{3.847576in}}{\pgfqpoint{3.439816in}{3.839763in}}%
\pgfpathcurveto{\pgfqpoint{3.447630in}{3.831949in}}{\pgfqpoint{3.458229in}{3.827559in}}{\pgfqpoint{3.469279in}{3.827559in}}%
\pgfpathclose%
\pgfusepath{stroke,fill}%
\end{pgfscope}%
\begin{pgfscope}%
\pgfpathrectangle{\pgfqpoint{0.481978in}{0.331635in}}{\pgfqpoint{9.300000in}{7.700000in}}%
\pgfusepath{clip}%
\pgfsetbuttcap%
\pgfsetroundjoin%
\definecolor{currentfill}{rgb}{1.000000,0.705882,0.509804}%
\pgfsetfillcolor{currentfill}%
\pgfsetlinewidth{0.481800pt}%
\definecolor{currentstroke}{rgb}{1.000000,1.000000,1.000000}%
\pgfsetstrokecolor{currentstroke}%
\pgfsetdash{}{0pt}%
\pgfpathmoveto{\pgfqpoint{6.295661in}{1.566834in}}%
\pgfpathcurveto{\pgfqpoint{6.306711in}{1.566834in}}{\pgfqpoint{6.317310in}{1.571225in}}{\pgfqpoint{6.325124in}{1.579038in}}%
\pgfpathcurveto{\pgfqpoint{6.332938in}{1.586852in}}{\pgfqpoint{6.337328in}{1.597451in}}{\pgfqpoint{6.337328in}{1.608501in}}%
\pgfpathcurveto{\pgfqpoint{6.337328in}{1.619551in}}{\pgfqpoint{6.332938in}{1.630150in}}{\pgfqpoint{6.325124in}{1.637964in}}%
\pgfpathcurveto{\pgfqpoint{6.317310in}{1.645777in}}{\pgfqpoint{6.306711in}{1.650168in}}{\pgfqpoint{6.295661in}{1.650168in}}%
\pgfpathcurveto{\pgfqpoint{6.284611in}{1.650168in}}{\pgfqpoint{6.274012in}{1.645777in}}{\pgfqpoint{6.266198in}{1.637964in}}%
\pgfpathcurveto{\pgfqpoint{6.258385in}{1.630150in}}{\pgfqpoint{6.253994in}{1.619551in}}{\pgfqpoint{6.253994in}{1.608501in}}%
\pgfpathcurveto{\pgfqpoint{6.253994in}{1.597451in}}{\pgfqpoint{6.258385in}{1.586852in}}{\pgfqpoint{6.266198in}{1.579038in}}%
\pgfpathcurveto{\pgfqpoint{6.274012in}{1.571225in}}{\pgfqpoint{6.284611in}{1.566834in}}{\pgfqpoint{6.295661in}{1.566834in}}%
\pgfpathclose%
\pgfusepath{stroke,fill}%
\end{pgfscope}%
\begin{pgfscope}%
\pgfpathrectangle{\pgfqpoint{0.481978in}{0.331635in}}{\pgfqpoint{9.300000in}{7.700000in}}%
\pgfusepath{clip}%
\pgfsetbuttcap%
\pgfsetroundjoin%
\definecolor{currentfill}{rgb}{1.000000,0.705882,0.509804}%
\pgfsetfillcolor{currentfill}%
\pgfsetlinewidth{0.481800pt}%
\definecolor{currentstroke}{rgb}{1.000000,1.000000,1.000000}%
\pgfsetstrokecolor{currentstroke}%
\pgfsetdash{}{0pt}%
\pgfpathmoveto{\pgfqpoint{3.458025in}{3.938973in}}%
\pgfpathcurveto{\pgfqpoint{3.469075in}{3.938973in}}{\pgfqpoint{3.479674in}{3.943364in}}{\pgfqpoint{3.487487in}{3.951177in}}%
\pgfpathcurveto{\pgfqpoint{3.495301in}{3.958991in}}{\pgfqpoint{3.499691in}{3.969590in}}{\pgfqpoint{3.499691in}{3.980640in}}%
\pgfpathcurveto{\pgfqpoint{3.499691in}{3.991690in}}{\pgfqpoint{3.495301in}{4.002289in}}{\pgfqpoint{3.487487in}{4.010103in}}%
\pgfpathcurveto{\pgfqpoint{3.479674in}{4.017916in}}{\pgfqpoint{3.469075in}{4.022307in}}{\pgfqpoint{3.458025in}{4.022307in}}%
\pgfpathcurveto{\pgfqpoint{3.446974in}{4.022307in}}{\pgfqpoint{3.436375in}{4.017916in}}{\pgfqpoint{3.428562in}{4.010103in}}%
\pgfpathcurveto{\pgfqpoint{3.420748in}{4.002289in}}{\pgfqpoint{3.416358in}{3.991690in}}{\pgfqpoint{3.416358in}{3.980640in}}%
\pgfpathcurveto{\pgfqpoint{3.416358in}{3.969590in}}{\pgfqpoint{3.420748in}{3.958991in}}{\pgfqpoint{3.428562in}{3.951177in}}%
\pgfpathcurveto{\pgfqpoint{3.436375in}{3.943364in}}{\pgfqpoint{3.446974in}{3.938973in}}{\pgfqpoint{3.458025in}{3.938973in}}%
\pgfpathclose%
\pgfusepath{stroke,fill}%
\end{pgfscope}%
\begin{pgfscope}%
\pgfpathrectangle{\pgfqpoint{0.481978in}{0.331635in}}{\pgfqpoint{9.300000in}{7.700000in}}%
\pgfusepath{clip}%
\pgfsetbuttcap%
\pgfsetroundjoin%
\definecolor{currentfill}{rgb}{1.000000,0.705882,0.509804}%
\pgfsetfillcolor{currentfill}%
\pgfsetlinewidth{0.481800pt}%
\definecolor{currentstroke}{rgb}{1.000000,1.000000,1.000000}%
\pgfsetstrokecolor{currentstroke}%
\pgfsetdash{}{0pt}%
\pgfpathmoveto{\pgfqpoint{4.270877in}{5.382117in}}%
\pgfpathcurveto{\pgfqpoint{4.281927in}{5.382117in}}{\pgfqpoint{4.292526in}{5.386508in}}{\pgfqpoint{4.300340in}{5.394321in}}%
\pgfpathcurveto{\pgfqpoint{4.308153in}{5.402135in}}{\pgfqpoint{4.312543in}{5.412734in}}{\pgfqpoint{4.312543in}{5.423784in}}%
\pgfpathcurveto{\pgfqpoint{4.312543in}{5.434834in}}{\pgfqpoint{4.308153in}{5.445433in}}{\pgfqpoint{4.300340in}{5.453247in}}%
\pgfpathcurveto{\pgfqpoint{4.292526in}{5.461060in}}{\pgfqpoint{4.281927in}{5.465451in}}{\pgfqpoint{4.270877in}{5.465451in}}%
\pgfpathcurveto{\pgfqpoint{4.259827in}{5.465451in}}{\pgfqpoint{4.249228in}{5.461060in}}{\pgfqpoint{4.241414in}{5.453247in}}%
\pgfpathcurveto{\pgfqpoint{4.233600in}{5.445433in}}{\pgfqpoint{4.229210in}{5.434834in}}{\pgfqpoint{4.229210in}{5.423784in}}%
\pgfpathcurveto{\pgfqpoint{4.229210in}{5.412734in}}{\pgfqpoint{4.233600in}{5.402135in}}{\pgfqpoint{4.241414in}{5.394321in}}%
\pgfpathcurveto{\pgfqpoint{4.249228in}{5.386508in}}{\pgfqpoint{4.259827in}{5.382117in}}{\pgfqpoint{4.270877in}{5.382117in}}%
\pgfpathclose%
\pgfusepath{stroke,fill}%
\end{pgfscope}%
\begin{pgfscope}%
\pgfpathrectangle{\pgfqpoint{0.481978in}{0.331635in}}{\pgfqpoint{9.300000in}{7.700000in}}%
\pgfusepath{clip}%
\pgfsetbuttcap%
\pgfsetroundjoin%
\definecolor{currentfill}{rgb}{1.000000,0.705882,0.509804}%
\pgfsetfillcolor{currentfill}%
\pgfsetlinewidth{0.481800pt}%
\definecolor{currentstroke}{rgb}{1.000000,1.000000,1.000000}%
\pgfsetstrokecolor{currentstroke}%
\pgfsetdash{}{0pt}%
\pgfpathmoveto{\pgfqpoint{5.749725in}{1.916657in}}%
\pgfpathcurveto{\pgfqpoint{5.760775in}{1.916657in}}{\pgfqpoint{5.771374in}{1.921047in}}{\pgfqpoint{5.779188in}{1.928861in}}%
\pgfpathcurveto{\pgfqpoint{5.787002in}{1.936675in}}{\pgfqpoint{5.791392in}{1.947274in}}{\pgfqpoint{5.791392in}{1.958324in}}%
\pgfpathcurveto{\pgfqpoint{5.791392in}{1.969374in}}{\pgfqpoint{5.787002in}{1.979973in}}{\pgfqpoint{5.779188in}{1.987787in}}%
\pgfpathcurveto{\pgfqpoint{5.771374in}{1.995600in}}{\pgfqpoint{5.760775in}{1.999990in}}{\pgfqpoint{5.749725in}{1.999990in}}%
\pgfpathcurveto{\pgfqpoint{5.738675in}{1.999990in}}{\pgfqpoint{5.728076in}{1.995600in}}{\pgfqpoint{5.720262in}{1.987787in}}%
\pgfpathcurveto{\pgfqpoint{5.712449in}{1.979973in}}{\pgfqpoint{5.708059in}{1.969374in}}{\pgfqpoint{5.708059in}{1.958324in}}%
\pgfpathcurveto{\pgfqpoint{5.708059in}{1.947274in}}{\pgfqpoint{5.712449in}{1.936675in}}{\pgfqpoint{5.720262in}{1.928861in}}%
\pgfpathcurveto{\pgfqpoint{5.728076in}{1.921047in}}{\pgfqpoint{5.738675in}{1.916657in}}{\pgfqpoint{5.749725in}{1.916657in}}%
\pgfpathclose%
\pgfusepath{stroke,fill}%
\end{pgfscope}%
\begin{pgfscope}%
\pgfpathrectangle{\pgfqpoint{0.481978in}{0.331635in}}{\pgfqpoint{9.300000in}{7.700000in}}%
\pgfusepath{clip}%
\pgfsetbuttcap%
\pgfsetroundjoin%
\definecolor{currentfill}{rgb}{1.000000,0.705882,0.509804}%
\pgfsetfillcolor{currentfill}%
\pgfsetlinewidth{0.481800pt}%
\definecolor{currentstroke}{rgb}{1.000000,1.000000,1.000000}%
\pgfsetstrokecolor{currentstroke}%
\pgfsetdash{}{0pt}%
\pgfpathmoveto{\pgfqpoint{2.458775in}{1.687705in}}%
\pgfpathcurveto{\pgfqpoint{2.469825in}{1.687705in}}{\pgfqpoint{2.480424in}{1.692095in}}{\pgfqpoint{2.488238in}{1.699909in}}%
\pgfpathcurveto{\pgfqpoint{2.496051in}{1.707723in}}{\pgfqpoint{2.500442in}{1.718322in}}{\pgfqpoint{2.500442in}{1.729372in}}%
\pgfpathcurveto{\pgfqpoint{2.500442in}{1.740422in}}{\pgfqpoint{2.496051in}{1.751021in}}{\pgfqpoint{2.488238in}{1.758835in}}%
\pgfpathcurveto{\pgfqpoint{2.480424in}{1.766648in}}{\pgfqpoint{2.469825in}{1.771038in}}{\pgfqpoint{2.458775in}{1.771038in}}%
\pgfpathcurveto{\pgfqpoint{2.447725in}{1.771038in}}{\pgfqpoint{2.437126in}{1.766648in}}{\pgfqpoint{2.429312in}{1.758835in}}%
\pgfpathcurveto{\pgfqpoint{2.421499in}{1.751021in}}{\pgfqpoint{2.417108in}{1.740422in}}{\pgfqpoint{2.417108in}{1.729372in}}%
\pgfpathcurveto{\pgfqpoint{2.417108in}{1.718322in}}{\pgfqpoint{2.421499in}{1.707723in}}{\pgfqpoint{2.429312in}{1.699909in}}%
\pgfpathcurveto{\pgfqpoint{2.437126in}{1.692095in}}{\pgfqpoint{2.447725in}{1.687705in}}{\pgfqpoint{2.458775in}{1.687705in}}%
\pgfpathclose%
\pgfusepath{stroke,fill}%
\end{pgfscope}%
\begin{pgfscope}%
\pgfpathrectangle{\pgfqpoint{0.481978in}{0.331635in}}{\pgfqpoint{9.300000in}{7.700000in}}%
\pgfusepath{clip}%
\pgfsetbuttcap%
\pgfsetroundjoin%
\definecolor{currentfill}{rgb}{1.000000,0.705882,0.509804}%
\pgfsetfillcolor{currentfill}%
\pgfsetlinewidth{0.481800pt}%
\definecolor{currentstroke}{rgb}{1.000000,1.000000,1.000000}%
\pgfsetstrokecolor{currentstroke}%
\pgfsetdash{}{0pt}%
\pgfpathmoveto{\pgfqpoint{4.692250in}{5.263851in}}%
\pgfpathcurveto{\pgfqpoint{4.703300in}{5.263851in}}{\pgfqpoint{4.713899in}{5.268242in}}{\pgfqpoint{4.721713in}{5.276055in}}%
\pgfpathcurveto{\pgfqpoint{4.729527in}{5.283869in}}{\pgfqpoint{4.733917in}{5.294468in}}{\pgfqpoint{4.733917in}{5.305518in}}%
\pgfpathcurveto{\pgfqpoint{4.733917in}{5.316568in}}{\pgfqpoint{4.729527in}{5.327167in}}{\pgfqpoint{4.721713in}{5.334981in}}%
\pgfpathcurveto{\pgfqpoint{4.713899in}{5.342795in}}{\pgfqpoint{4.703300in}{5.347185in}}{\pgfqpoint{4.692250in}{5.347185in}}%
\pgfpathcurveto{\pgfqpoint{4.681200in}{5.347185in}}{\pgfqpoint{4.670601in}{5.342795in}}{\pgfqpoint{4.662787in}{5.334981in}}%
\pgfpathcurveto{\pgfqpoint{4.654974in}{5.327167in}}{\pgfqpoint{4.650584in}{5.316568in}}{\pgfqpoint{4.650584in}{5.305518in}}%
\pgfpathcurveto{\pgfqpoint{4.650584in}{5.294468in}}{\pgfqpoint{4.654974in}{5.283869in}}{\pgfqpoint{4.662787in}{5.276055in}}%
\pgfpathcurveto{\pgfqpoint{4.670601in}{5.268242in}}{\pgfqpoint{4.681200in}{5.263851in}}{\pgfqpoint{4.692250in}{5.263851in}}%
\pgfpathclose%
\pgfusepath{stroke,fill}%
\end{pgfscope}%
\begin{pgfscope}%
\pgfpathrectangle{\pgfqpoint{0.481978in}{0.331635in}}{\pgfqpoint{9.300000in}{7.700000in}}%
\pgfusepath{clip}%
\pgfsetbuttcap%
\pgfsetroundjoin%
\definecolor{currentfill}{rgb}{1.000000,0.705882,0.509804}%
\pgfsetfillcolor{currentfill}%
\pgfsetlinewidth{0.481800pt}%
\definecolor{currentstroke}{rgb}{1.000000,1.000000,1.000000}%
\pgfsetstrokecolor{currentstroke}%
\pgfsetdash{}{0pt}%
\pgfpathmoveto{\pgfqpoint{7.916388in}{4.181299in}}%
\pgfpathcurveto{\pgfqpoint{7.927438in}{4.181299in}}{\pgfqpoint{7.938037in}{4.185689in}}{\pgfqpoint{7.945851in}{4.193503in}}%
\pgfpathcurveto{\pgfqpoint{7.953665in}{4.201317in}}{\pgfqpoint{7.958055in}{4.211916in}}{\pgfqpoint{7.958055in}{4.222966in}}%
\pgfpathcurveto{\pgfqpoint{7.958055in}{4.234016in}}{\pgfqpoint{7.953665in}{4.244615in}}{\pgfqpoint{7.945851in}{4.252429in}}%
\pgfpathcurveto{\pgfqpoint{7.938037in}{4.260242in}}{\pgfqpoint{7.927438in}{4.264632in}}{\pgfqpoint{7.916388in}{4.264632in}}%
\pgfpathcurveto{\pgfqpoint{7.905338in}{4.264632in}}{\pgfqpoint{7.894739in}{4.260242in}}{\pgfqpoint{7.886926in}{4.252429in}}%
\pgfpathcurveto{\pgfqpoint{7.879112in}{4.244615in}}{\pgfqpoint{7.874722in}{4.234016in}}{\pgfqpoint{7.874722in}{4.222966in}}%
\pgfpathcurveto{\pgfqpoint{7.874722in}{4.211916in}}{\pgfqpoint{7.879112in}{4.201317in}}{\pgfqpoint{7.886926in}{4.193503in}}%
\pgfpathcurveto{\pgfqpoint{7.894739in}{4.185689in}}{\pgfqpoint{7.905338in}{4.181299in}}{\pgfqpoint{7.916388in}{4.181299in}}%
\pgfpathclose%
\pgfusepath{stroke,fill}%
\end{pgfscope}%
\begin{pgfscope}%
\pgfpathrectangle{\pgfqpoint{0.481978in}{0.331635in}}{\pgfqpoint{9.300000in}{7.700000in}}%
\pgfusepath{clip}%
\pgfsetbuttcap%
\pgfsetroundjoin%
\definecolor{currentfill}{rgb}{1.000000,0.705882,0.509804}%
\pgfsetfillcolor{currentfill}%
\pgfsetlinewidth{0.481800pt}%
\definecolor{currentstroke}{rgb}{1.000000,1.000000,1.000000}%
\pgfsetstrokecolor{currentstroke}%
\pgfsetdash{}{0pt}%
\pgfpathmoveto{\pgfqpoint{4.752028in}{3.418430in}}%
\pgfpathcurveto{\pgfqpoint{4.763078in}{3.418430in}}{\pgfqpoint{4.773677in}{3.422821in}}{\pgfqpoint{4.781491in}{3.430634in}}%
\pgfpathcurveto{\pgfqpoint{4.789304in}{3.438448in}}{\pgfqpoint{4.793694in}{3.449047in}}{\pgfqpoint{4.793694in}{3.460097in}}%
\pgfpathcurveto{\pgfqpoint{4.793694in}{3.471147in}}{\pgfqpoint{4.789304in}{3.481746in}}{\pgfqpoint{4.781491in}{3.489560in}}%
\pgfpathcurveto{\pgfqpoint{4.773677in}{3.497374in}}{\pgfqpoint{4.763078in}{3.501764in}}{\pgfqpoint{4.752028in}{3.501764in}}%
\pgfpathcurveto{\pgfqpoint{4.740978in}{3.501764in}}{\pgfqpoint{4.730379in}{3.497374in}}{\pgfqpoint{4.722565in}{3.489560in}}%
\pgfpathcurveto{\pgfqpoint{4.714751in}{3.481746in}}{\pgfqpoint{4.710361in}{3.471147in}}{\pgfqpoint{4.710361in}{3.460097in}}%
\pgfpathcurveto{\pgfqpoint{4.710361in}{3.449047in}}{\pgfqpoint{4.714751in}{3.438448in}}{\pgfqpoint{4.722565in}{3.430634in}}%
\pgfpathcurveto{\pgfqpoint{4.730379in}{3.422821in}}{\pgfqpoint{4.740978in}{3.418430in}}{\pgfqpoint{4.752028in}{3.418430in}}%
\pgfpathclose%
\pgfusepath{stroke,fill}%
\end{pgfscope}%
\begin{pgfscope}%
\pgfpathrectangle{\pgfqpoint{0.481978in}{0.331635in}}{\pgfqpoint{9.300000in}{7.700000in}}%
\pgfusepath{clip}%
\pgfsetbuttcap%
\pgfsetroundjoin%
\definecolor{currentfill}{rgb}{1.000000,0.705882,0.509804}%
\pgfsetfillcolor{currentfill}%
\pgfsetlinewidth{0.481800pt}%
\definecolor{currentstroke}{rgb}{1.000000,1.000000,1.000000}%
\pgfsetstrokecolor{currentstroke}%
\pgfsetdash{}{0pt}%
\pgfpathmoveto{\pgfqpoint{8.514292in}{5.224514in}}%
\pgfpathcurveto{\pgfqpoint{8.525342in}{5.224514in}}{\pgfqpoint{8.535941in}{5.228904in}}{\pgfqpoint{8.543755in}{5.236717in}}%
\pgfpathcurveto{\pgfqpoint{8.551569in}{5.244531in}}{\pgfqpoint{8.555959in}{5.255130in}}{\pgfqpoint{8.555959in}{5.266180in}}%
\pgfpathcurveto{\pgfqpoint{8.555959in}{5.277230in}}{\pgfqpoint{8.551569in}{5.287829in}}{\pgfqpoint{8.543755in}{5.295643in}}%
\pgfpathcurveto{\pgfqpoint{8.535941in}{5.303457in}}{\pgfqpoint{8.525342in}{5.307847in}}{\pgfqpoint{8.514292in}{5.307847in}}%
\pgfpathcurveto{\pgfqpoint{8.503242in}{5.307847in}}{\pgfqpoint{8.492643in}{5.303457in}}{\pgfqpoint{8.484829in}{5.295643in}}%
\pgfpathcurveto{\pgfqpoint{8.477016in}{5.287829in}}{\pgfqpoint{8.472626in}{5.277230in}}{\pgfqpoint{8.472626in}{5.266180in}}%
\pgfpathcurveto{\pgfqpoint{8.472626in}{5.255130in}}{\pgfqpoint{8.477016in}{5.244531in}}{\pgfqpoint{8.484829in}{5.236717in}}%
\pgfpathcurveto{\pgfqpoint{8.492643in}{5.228904in}}{\pgfqpoint{8.503242in}{5.224514in}}{\pgfqpoint{8.514292in}{5.224514in}}%
\pgfpathclose%
\pgfusepath{stroke,fill}%
\end{pgfscope}%
\begin{pgfscope}%
\pgfpathrectangle{\pgfqpoint{0.481978in}{0.331635in}}{\pgfqpoint{9.300000in}{7.700000in}}%
\pgfusepath{clip}%
\pgfsetbuttcap%
\pgfsetroundjoin%
\definecolor{currentfill}{rgb}{1.000000,0.705882,0.509804}%
\pgfsetfillcolor{currentfill}%
\pgfsetlinewidth{0.481800pt}%
\definecolor{currentstroke}{rgb}{1.000000,1.000000,1.000000}%
\pgfsetstrokecolor{currentstroke}%
\pgfsetdash{}{0pt}%
\pgfpathmoveto{\pgfqpoint{3.988115in}{1.904248in}}%
\pgfpathcurveto{\pgfqpoint{3.999165in}{1.904248in}}{\pgfqpoint{4.009764in}{1.908639in}}{\pgfqpoint{4.017578in}{1.916452in}}%
\pgfpathcurveto{\pgfqpoint{4.025392in}{1.924266in}}{\pgfqpoint{4.029782in}{1.934865in}}{\pgfqpoint{4.029782in}{1.945915in}}%
\pgfpathcurveto{\pgfqpoint{4.029782in}{1.956965in}}{\pgfqpoint{4.025392in}{1.967564in}}{\pgfqpoint{4.017578in}{1.975378in}}%
\pgfpathcurveto{\pgfqpoint{4.009764in}{1.983191in}}{\pgfqpoint{3.999165in}{1.987582in}}{\pgfqpoint{3.988115in}{1.987582in}}%
\pgfpathcurveto{\pgfqpoint{3.977065in}{1.987582in}}{\pgfqpoint{3.966466in}{1.983191in}}{\pgfqpoint{3.958652in}{1.975378in}}%
\pgfpathcurveto{\pgfqpoint{3.950839in}{1.967564in}}{\pgfqpoint{3.946448in}{1.956965in}}{\pgfqpoint{3.946448in}{1.945915in}}%
\pgfpathcurveto{\pgfqpoint{3.946448in}{1.934865in}}{\pgfqpoint{3.950839in}{1.924266in}}{\pgfqpoint{3.958652in}{1.916452in}}%
\pgfpathcurveto{\pgfqpoint{3.966466in}{1.908639in}}{\pgfqpoint{3.977065in}{1.904248in}}{\pgfqpoint{3.988115in}{1.904248in}}%
\pgfpathclose%
\pgfusepath{stroke,fill}%
\end{pgfscope}%
\begin{pgfscope}%
\pgfpathrectangle{\pgfqpoint{0.481978in}{0.331635in}}{\pgfqpoint{9.300000in}{7.700000in}}%
\pgfusepath{clip}%
\pgfsetbuttcap%
\pgfsetroundjoin%
\definecolor{currentfill}{rgb}{1.000000,0.705882,0.509804}%
\pgfsetfillcolor{currentfill}%
\pgfsetlinewidth{0.481800pt}%
\definecolor{currentstroke}{rgb}{1.000000,1.000000,1.000000}%
\pgfsetstrokecolor{currentstroke}%
\pgfsetdash{}{0pt}%
\pgfpathmoveto{\pgfqpoint{3.637460in}{4.439078in}}%
\pgfpathcurveto{\pgfqpoint{3.648510in}{4.439078in}}{\pgfqpoint{3.659109in}{4.443469in}}{\pgfqpoint{3.666923in}{4.451282in}}%
\pgfpathcurveto{\pgfqpoint{3.674737in}{4.459096in}}{\pgfqpoint{3.679127in}{4.469695in}}{\pgfqpoint{3.679127in}{4.480745in}}%
\pgfpathcurveto{\pgfqpoint{3.679127in}{4.491795in}}{\pgfqpoint{3.674737in}{4.502394in}}{\pgfqpoint{3.666923in}{4.510208in}}%
\pgfpathcurveto{\pgfqpoint{3.659109in}{4.518022in}}{\pgfqpoint{3.648510in}{4.522412in}}{\pgfqpoint{3.637460in}{4.522412in}}%
\pgfpathcurveto{\pgfqpoint{3.626410in}{4.522412in}}{\pgfqpoint{3.615811in}{4.518022in}}{\pgfqpoint{3.607997in}{4.510208in}}%
\pgfpathcurveto{\pgfqpoint{3.600184in}{4.502394in}}{\pgfqpoint{3.595794in}{4.491795in}}{\pgfqpoint{3.595794in}{4.480745in}}%
\pgfpathcurveto{\pgfqpoint{3.595794in}{4.469695in}}{\pgfqpoint{3.600184in}{4.459096in}}{\pgfqpoint{3.607997in}{4.451282in}}%
\pgfpathcurveto{\pgfqpoint{3.615811in}{4.443469in}}{\pgfqpoint{3.626410in}{4.439078in}}{\pgfqpoint{3.637460in}{4.439078in}}%
\pgfpathclose%
\pgfusepath{stroke,fill}%
\end{pgfscope}%
\begin{pgfscope}%
\pgfpathrectangle{\pgfqpoint{0.481978in}{0.331635in}}{\pgfqpoint{9.300000in}{7.700000in}}%
\pgfusepath{clip}%
\pgfsetbuttcap%
\pgfsetroundjoin%
\definecolor{currentfill}{rgb}{1.000000,0.705882,0.509804}%
\pgfsetfillcolor{currentfill}%
\pgfsetlinewidth{0.481800pt}%
\definecolor{currentstroke}{rgb}{1.000000,1.000000,1.000000}%
\pgfsetstrokecolor{currentstroke}%
\pgfsetdash{}{0pt}%
\pgfpathmoveto{\pgfqpoint{3.666577in}{4.453388in}}%
\pgfpathcurveto{\pgfqpoint{3.677627in}{4.453388in}}{\pgfqpoint{3.688226in}{4.457779in}}{\pgfqpoint{3.696040in}{4.465592in}}%
\pgfpathcurveto{\pgfqpoint{3.703854in}{4.473406in}}{\pgfqpoint{3.708244in}{4.484005in}}{\pgfqpoint{3.708244in}{4.495055in}}%
\pgfpathcurveto{\pgfqpoint{3.708244in}{4.506105in}}{\pgfqpoint{3.703854in}{4.516704in}}{\pgfqpoint{3.696040in}{4.524518in}}%
\pgfpathcurveto{\pgfqpoint{3.688226in}{4.532331in}}{\pgfqpoint{3.677627in}{4.536722in}}{\pgfqpoint{3.666577in}{4.536722in}}%
\pgfpathcurveto{\pgfqpoint{3.655527in}{4.536722in}}{\pgfqpoint{3.644928in}{4.532331in}}{\pgfqpoint{3.637114in}{4.524518in}}%
\pgfpathcurveto{\pgfqpoint{3.629301in}{4.516704in}}{\pgfqpoint{3.624910in}{4.506105in}}{\pgfqpoint{3.624910in}{4.495055in}}%
\pgfpathcurveto{\pgfqpoint{3.624910in}{4.484005in}}{\pgfqpoint{3.629301in}{4.473406in}}{\pgfqpoint{3.637114in}{4.465592in}}%
\pgfpathcurveto{\pgfqpoint{3.644928in}{4.457779in}}{\pgfqpoint{3.655527in}{4.453388in}}{\pgfqpoint{3.666577in}{4.453388in}}%
\pgfpathclose%
\pgfusepath{stroke,fill}%
\end{pgfscope}%
\begin{pgfscope}%
\pgfpathrectangle{\pgfqpoint{0.481978in}{0.331635in}}{\pgfqpoint{9.300000in}{7.700000in}}%
\pgfusepath{clip}%
\pgfsetbuttcap%
\pgfsetroundjoin%
\definecolor{currentfill}{rgb}{1.000000,0.705882,0.509804}%
\pgfsetfillcolor{currentfill}%
\pgfsetlinewidth{0.481800pt}%
\definecolor{currentstroke}{rgb}{1.000000,1.000000,1.000000}%
\pgfsetstrokecolor{currentstroke}%
\pgfsetdash{}{0pt}%
\pgfpathmoveto{\pgfqpoint{3.344754in}{3.212776in}}%
\pgfpathcurveto{\pgfqpoint{3.355805in}{3.212776in}}{\pgfqpoint{3.366404in}{3.217166in}}{\pgfqpoint{3.374217in}{3.224980in}}%
\pgfpathcurveto{\pgfqpoint{3.382031in}{3.232794in}}{\pgfqpoint{3.386421in}{3.243393in}}{\pgfqpoint{3.386421in}{3.254443in}}%
\pgfpathcurveto{\pgfqpoint{3.386421in}{3.265493in}}{\pgfqpoint{3.382031in}{3.276092in}}{\pgfqpoint{3.374217in}{3.283905in}}%
\pgfpathcurveto{\pgfqpoint{3.366404in}{3.291719in}}{\pgfqpoint{3.355805in}{3.296109in}}{\pgfqpoint{3.344754in}{3.296109in}}%
\pgfpathcurveto{\pgfqpoint{3.333704in}{3.296109in}}{\pgfqpoint{3.323105in}{3.291719in}}{\pgfqpoint{3.315292in}{3.283905in}}%
\pgfpathcurveto{\pgfqpoint{3.307478in}{3.276092in}}{\pgfqpoint{3.303088in}{3.265493in}}{\pgfqpoint{3.303088in}{3.254443in}}%
\pgfpathcurveto{\pgfqpoint{3.303088in}{3.243393in}}{\pgfqpoint{3.307478in}{3.232794in}}{\pgfqpoint{3.315292in}{3.224980in}}%
\pgfpathcurveto{\pgfqpoint{3.323105in}{3.217166in}}{\pgfqpoint{3.333704in}{3.212776in}}{\pgfqpoint{3.344754in}{3.212776in}}%
\pgfpathclose%
\pgfusepath{stroke,fill}%
\end{pgfscope}%
\begin{pgfscope}%
\pgfpathrectangle{\pgfqpoint{0.481978in}{0.331635in}}{\pgfqpoint{9.300000in}{7.700000in}}%
\pgfusepath{clip}%
\pgfsetbuttcap%
\pgfsetroundjoin%
\definecolor{currentfill}{rgb}{1.000000,0.705882,0.509804}%
\pgfsetfillcolor{currentfill}%
\pgfsetlinewidth{0.481800pt}%
\definecolor{currentstroke}{rgb}{1.000000,1.000000,1.000000}%
\pgfsetstrokecolor{currentstroke}%
\pgfsetdash{}{0pt}%
\pgfpathmoveto{\pgfqpoint{7.394228in}{2.316863in}}%
\pgfpathcurveto{\pgfqpoint{7.405278in}{2.316863in}}{\pgfqpoint{7.415877in}{2.321253in}}{\pgfqpoint{7.423691in}{2.329066in}}%
\pgfpathcurveto{\pgfqpoint{7.431504in}{2.336880in}}{\pgfqpoint{7.435894in}{2.347479in}}{\pgfqpoint{7.435894in}{2.358529in}}%
\pgfpathcurveto{\pgfqpoint{7.435894in}{2.369579in}}{\pgfqpoint{7.431504in}{2.380178in}}{\pgfqpoint{7.423691in}{2.387992in}}%
\pgfpathcurveto{\pgfqpoint{7.415877in}{2.395806in}}{\pgfqpoint{7.405278in}{2.400196in}}{\pgfqpoint{7.394228in}{2.400196in}}%
\pgfpathcurveto{\pgfqpoint{7.383178in}{2.400196in}}{\pgfqpoint{7.372579in}{2.395806in}}{\pgfqpoint{7.364765in}{2.387992in}}%
\pgfpathcurveto{\pgfqpoint{7.356951in}{2.380178in}}{\pgfqpoint{7.352561in}{2.369579in}}{\pgfqpoint{7.352561in}{2.358529in}}%
\pgfpathcurveto{\pgfqpoint{7.352561in}{2.347479in}}{\pgfqpoint{7.356951in}{2.336880in}}{\pgfqpoint{7.364765in}{2.329066in}}%
\pgfpathcurveto{\pgfqpoint{7.372579in}{2.321253in}}{\pgfqpoint{7.383178in}{2.316863in}}{\pgfqpoint{7.394228in}{2.316863in}}%
\pgfpathclose%
\pgfusepath{stroke,fill}%
\end{pgfscope}%
\begin{pgfscope}%
\pgfpathrectangle{\pgfqpoint{0.481978in}{0.331635in}}{\pgfqpoint{9.300000in}{7.700000in}}%
\pgfusepath{clip}%
\pgfsetbuttcap%
\pgfsetroundjoin%
\definecolor{currentfill}{rgb}{1.000000,0.705882,0.509804}%
\pgfsetfillcolor{currentfill}%
\pgfsetlinewidth{0.481800pt}%
\definecolor{currentstroke}{rgb}{1.000000,1.000000,1.000000}%
\pgfsetstrokecolor{currentstroke}%
\pgfsetdash{}{0pt}%
\pgfpathmoveto{\pgfqpoint{2.694400in}{2.093625in}}%
\pgfpathcurveto{\pgfqpoint{2.705451in}{2.093625in}}{\pgfqpoint{2.716050in}{2.098015in}}{\pgfqpoint{2.723863in}{2.105829in}}%
\pgfpathcurveto{\pgfqpoint{2.731677in}{2.113642in}}{\pgfqpoint{2.736067in}{2.124241in}}{\pgfqpoint{2.736067in}{2.135292in}}%
\pgfpathcurveto{\pgfqpoint{2.736067in}{2.146342in}}{\pgfqpoint{2.731677in}{2.156941in}}{\pgfqpoint{2.723863in}{2.164754in}}%
\pgfpathcurveto{\pgfqpoint{2.716050in}{2.172568in}}{\pgfqpoint{2.705451in}{2.176958in}}{\pgfqpoint{2.694400in}{2.176958in}}%
\pgfpathcurveto{\pgfqpoint{2.683350in}{2.176958in}}{\pgfqpoint{2.672751in}{2.172568in}}{\pgfqpoint{2.664938in}{2.164754in}}%
\pgfpathcurveto{\pgfqpoint{2.657124in}{2.156941in}}{\pgfqpoint{2.652734in}{2.146342in}}{\pgfqpoint{2.652734in}{2.135292in}}%
\pgfpathcurveto{\pgfqpoint{2.652734in}{2.124241in}}{\pgfqpoint{2.657124in}{2.113642in}}{\pgfqpoint{2.664938in}{2.105829in}}%
\pgfpathcurveto{\pgfqpoint{2.672751in}{2.098015in}}{\pgfqpoint{2.683350in}{2.093625in}}{\pgfqpoint{2.694400in}{2.093625in}}%
\pgfpathclose%
\pgfusepath{stroke,fill}%
\end{pgfscope}%
\begin{pgfscope}%
\pgfpathrectangle{\pgfqpoint{0.481978in}{0.331635in}}{\pgfqpoint{9.300000in}{7.700000in}}%
\pgfusepath{clip}%
\pgfsetbuttcap%
\pgfsetroundjoin%
\definecolor{currentfill}{rgb}{1.000000,0.705882,0.509804}%
\pgfsetfillcolor{currentfill}%
\pgfsetlinewidth{0.481800pt}%
\definecolor{currentstroke}{rgb}{1.000000,1.000000,1.000000}%
\pgfsetstrokecolor{currentstroke}%
\pgfsetdash{}{0pt}%
\pgfpathmoveto{\pgfqpoint{5.916550in}{1.501443in}}%
\pgfpathcurveto{\pgfqpoint{5.927600in}{1.501443in}}{\pgfqpoint{5.938199in}{1.505833in}}{\pgfqpoint{5.946013in}{1.513647in}}%
\pgfpathcurveto{\pgfqpoint{5.953826in}{1.521460in}}{\pgfqpoint{5.958217in}{1.532059in}}{\pgfqpoint{5.958217in}{1.543110in}}%
\pgfpathcurveto{\pgfqpoint{5.958217in}{1.554160in}}{\pgfqpoint{5.953826in}{1.564759in}}{\pgfqpoint{5.946013in}{1.572572in}}%
\pgfpathcurveto{\pgfqpoint{5.938199in}{1.580386in}}{\pgfqpoint{5.927600in}{1.584776in}}{\pgfqpoint{5.916550in}{1.584776in}}%
\pgfpathcurveto{\pgfqpoint{5.905500in}{1.584776in}}{\pgfqpoint{5.894901in}{1.580386in}}{\pgfqpoint{5.887087in}{1.572572in}}%
\pgfpathcurveto{\pgfqpoint{5.879274in}{1.564759in}}{\pgfqpoint{5.874883in}{1.554160in}}{\pgfqpoint{5.874883in}{1.543110in}}%
\pgfpathcurveto{\pgfqpoint{5.874883in}{1.532059in}}{\pgfqpoint{5.879274in}{1.521460in}}{\pgfqpoint{5.887087in}{1.513647in}}%
\pgfpathcurveto{\pgfqpoint{5.894901in}{1.505833in}}{\pgfqpoint{5.905500in}{1.501443in}}{\pgfqpoint{5.916550in}{1.501443in}}%
\pgfpathclose%
\pgfusepath{stroke,fill}%
\end{pgfscope}%
\begin{pgfscope}%
\pgfpathrectangle{\pgfqpoint{0.481978in}{0.331635in}}{\pgfqpoint{9.300000in}{7.700000in}}%
\pgfusepath{clip}%
\pgfsetbuttcap%
\pgfsetroundjoin%
\definecolor{currentfill}{rgb}{1.000000,0.705882,0.509804}%
\pgfsetfillcolor{currentfill}%
\pgfsetlinewidth{0.481800pt}%
\definecolor{currentstroke}{rgb}{1.000000,1.000000,1.000000}%
\pgfsetstrokecolor{currentstroke}%
\pgfsetdash{}{0pt}%
\pgfpathmoveto{\pgfqpoint{6.083860in}{2.927174in}}%
\pgfpathcurveto{\pgfqpoint{6.094910in}{2.927174in}}{\pgfqpoint{6.105509in}{2.931564in}}{\pgfqpoint{6.113322in}{2.939378in}}%
\pgfpathcurveto{\pgfqpoint{6.121136in}{2.947192in}}{\pgfqpoint{6.125526in}{2.957791in}}{\pgfqpoint{6.125526in}{2.968841in}}%
\pgfpathcurveto{\pgfqpoint{6.125526in}{2.979891in}}{\pgfqpoint{6.121136in}{2.990490in}}{\pgfqpoint{6.113322in}{2.998304in}}%
\pgfpathcurveto{\pgfqpoint{6.105509in}{3.006117in}}{\pgfqpoint{6.094910in}{3.010507in}}{\pgfqpoint{6.083860in}{3.010507in}}%
\pgfpathcurveto{\pgfqpoint{6.072810in}{3.010507in}}{\pgfqpoint{6.062211in}{3.006117in}}{\pgfqpoint{6.054397in}{2.998304in}}%
\pgfpathcurveto{\pgfqpoint{6.046583in}{2.990490in}}{\pgfqpoint{6.042193in}{2.979891in}}{\pgfqpoint{6.042193in}{2.968841in}}%
\pgfpathcurveto{\pgfqpoint{6.042193in}{2.957791in}}{\pgfqpoint{6.046583in}{2.947192in}}{\pgfqpoint{6.054397in}{2.939378in}}%
\pgfpathcurveto{\pgfqpoint{6.062211in}{2.931564in}}{\pgfqpoint{6.072810in}{2.927174in}}{\pgfqpoint{6.083860in}{2.927174in}}%
\pgfpathclose%
\pgfusepath{stroke,fill}%
\end{pgfscope}%
\begin{pgfscope}%
\pgfpathrectangle{\pgfqpoint{0.481978in}{0.331635in}}{\pgfqpoint{9.300000in}{7.700000in}}%
\pgfusepath{clip}%
\pgfsetbuttcap%
\pgfsetroundjoin%
\definecolor{currentfill}{rgb}{1.000000,0.705882,0.509804}%
\pgfsetfillcolor{currentfill}%
\pgfsetlinewidth{0.481800pt}%
\definecolor{currentstroke}{rgb}{1.000000,1.000000,1.000000}%
\pgfsetstrokecolor{currentstroke}%
\pgfsetdash{}{0pt}%
\pgfpathmoveto{\pgfqpoint{4.639314in}{3.250673in}}%
\pgfpathcurveto{\pgfqpoint{4.650364in}{3.250673in}}{\pgfqpoint{4.660963in}{3.255063in}}{\pgfqpoint{4.668776in}{3.262877in}}%
\pgfpathcurveto{\pgfqpoint{4.676590in}{3.270690in}}{\pgfqpoint{4.680980in}{3.281289in}}{\pgfqpoint{4.680980in}{3.292339in}}%
\pgfpathcurveto{\pgfqpoint{4.680980in}{3.303390in}}{\pgfqpoint{4.676590in}{3.313989in}}{\pgfqpoint{4.668776in}{3.321802in}}%
\pgfpathcurveto{\pgfqpoint{4.660963in}{3.329616in}}{\pgfqpoint{4.650364in}{3.334006in}}{\pgfqpoint{4.639314in}{3.334006in}}%
\pgfpathcurveto{\pgfqpoint{4.628264in}{3.334006in}}{\pgfqpoint{4.617665in}{3.329616in}}{\pgfqpoint{4.609851in}{3.321802in}}%
\pgfpathcurveto{\pgfqpoint{4.602037in}{3.313989in}}{\pgfqpoint{4.597647in}{3.303390in}}{\pgfqpoint{4.597647in}{3.292339in}}%
\pgfpathcurveto{\pgfqpoint{4.597647in}{3.281289in}}{\pgfqpoint{4.602037in}{3.270690in}}{\pgfqpoint{4.609851in}{3.262877in}}%
\pgfpathcurveto{\pgfqpoint{4.617665in}{3.255063in}}{\pgfqpoint{4.628264in}{3.250673in}}{\pgfqpoint{4.639314in}{3.250673in}}%
\pgfpathclose%
\pgfusepath{stroke,fill}%
\end{pgfscope}%
\begin{pgfscope}%
\pgfpathrectangle{\pgfqpoint{0.481978in}{0.331635in}}{\pgfqpoint{9.300000in}{7.700000in}}%
\pgfusepath{clip}%
\pgfsetbuttcap%
\pgfsetroundjoin%
\definecolor{currentfill}{rgb}{1.000000,0.705882,0.509804}%
\pgfsetfillcolor{currentfill}%
\pgfsetlinewidth{0.481800pt}%
\definecolor{currentstroke}{rgb}{1.000000,1.000000,1.000000}%
\pgfsetstrokecolor{currentstroke}%
\pgfsetdash{}{0pt}%
\pgfpathmoveto{\pgfqpoint{8.886711in}{5.704878in}}%
\pgfpathcurveto{\pgfqpoint{8.897762in}{5.704878in}}{\pgfqpoint{8.908361in}{5.709268in}}{\pgfqpoint{8.916174in}{5.717082in}}%
\pgfpathcurveto{\pgfqpoint{8.923988in}{5.724895in}}{\pgfqpoint{8.928378in}{5.735494in}}{\pgfqpoint{8.928378in}{5.746544in}}%
\pgfpathcurveto{\pgfqpoint{8.928378in}{5.757595in}}{\pgfqpoint{8.923988in}{5.768194in}}{\pgfqpoint{8.916174in}{5.776007in}}%
\pgfpathcurveto{\pgfqpoint{8.908361in}{5.783821in}}{\pgfqpoint{8.897762in}{5.788211in}}{\pgfqpoint{8.886711in}{5.788211in}}%
\pgfpathcurveto{\pgfqpoint{8.875661in}{5.788211in}}{\pgfqpoint{8.865062in}{5.783821in}}{\pgfqpoint{8.857249in}{5.776007in}}%
\pgfpathcurveto{\pgfqpoint{8.849435in}{5.768194in}}{\pgfqpoint{8.845045in}{5.757595in}}{\pgfqpoint{8.845045in}{5.746544in}}%
\pgfpathcurveto{\pgfqpoint{8.845045in}{5.735494in}}{\pgfqpoint{8.849435in}{5.724895in}}{\pgfqpoint{8.857249in}{5.717082in}}%
\pgfpathcurveto{\pgfqpoint{8.865062in}{5.709268in}}{\pgfqpoint{8.875661in}{5.704878in}}{\pgfqpoint{8.886711in}{5.704878in}}%
\pgfpathclose%
\pgfusepath{stroke,fill}%
\end{pgfscope}%
\begin{pgfscope}%
\pgfpathrectangle{\pgfqpoint{0.481978in}{0.331635in}}{\pgfqpoint{9.300000in}{7.700000in}}%
\pgfusepath{clip}%
\pgfsetbuttcap%
\pgfsetroundjoin%
\definecolor{currentfill}{rgb}{1.000000,0.705882,0.509804}%
\pgfsetfillcolor{currentfill}%
\pgfsetlinewidth{0.481800pt}%
\definecolor{currentstroke}{rgb}{1.000000,1.000000,1.000000}%
\pgfsetstrokecolor{currentstroke}%
\pgfsetdash{}{0pt}%
\pgfpathmoveto{\pgfqpoint{3.960687in}{2.268367in}}%
\pgfpathcurveto{\pgfqpoint{3.971737in}{2.268367in}}{\pgfqpoint{3.982336in}{2.272757in}}{\pgfqpoint{3.990150in}{2.280571in}}%
\pgfpathcurveto{\pgfqpoint{3.997963in}{2.288384in}}{\pgfqpoint{4.002353in}{2.298983in}}{\pgfqpoint{4.002353in}{2.310033in}}%
\pgfpathcurveto{\pgfqpoint{4.002353in}{2.321084in}}{\pgfqpoint{3.997963in}{2.331683in}}{\pgfqpoint{3.990150in}{2.339496in}}%
\pgfpathcurveto{\pgfqpoint{3.982336in}{2.347310in}}{\pgfqpoint{3.971737in}{2.351700in}}{\pgfqpoint{3.960687in}{2.351700in}}%
\pgfpathcurveto{\pgfqpoint{3.949637in}{2.351700in}}{\pgfqpoint{3.939038in}{2.347310in}}{\pgfqpoint{3.931224in}{2.339496in}}%
\pgfpathcurveto{\pgfqpoint{3.923410in}{2.331683in}}{\pgfqpoint{3.919020in}{2.321084in}}{\pgfqpoint{3.919020in}{2.310033in}}%
\pgfpathcurveto{\pgfqpoint{3.919020in}{2.298983in}}{\pgfqpoint{3.923410in}{2.288384in}}{\pgfqpoint{3.931224in}{2.280571in}}%
\pgfpathcurveto{\pgfqpoint{3.939038in}{2.272757in}}{\pgfqpoint{3.949637in}{2.268367in}}{\pgfqpoint{3.960687in}{2.268367in}}%
\pgfpathclose%
\pgfusepath{stroke,fill}%
\end{pgfscope}%
\begin{pgfscope}%
\pgfpathrectangle{\pgfqpoint{0.481978in}{0.331635in}}{\pgfqpoint{9.300000in}{7.700000in}}%
\pgfusepath{clip}%
\pgfsetbuttcap%
\pgfsetroundjoin%
\definecolor{currentfill}{rgb}{1.000000,0.705882,0.509804}%
\pgfsetfillcolor{currentfill}%
\pgfsetlinewidth{0.481800pt}%
\definecolor{currentstroke}{rgb}{1.000000,1.000000,1.000000}%
\pgfsetstrokecolor{currentstroke}%
\pgfsetdash{}{0pt}%
\pgfpathmoveto{\pgfqpoint{3.907332in}{1.719292in}}%
\pgfpathcurveto{\pgfqpoint{3.918382in}{1.719292in}}{\pgfqpoint{3.928981in}{1.723682in}}{\pgfqpoint{3.936795in}{1.731496in}}%
\pgfpathcurveto{\pgfqpoint{3.944608in}{1.739309in}}{\pgfqpoint{3.948999in}{1.749908in}}{\pgfqpoint{3.948999in}{1.760958in}}%
\pgfpathcurveto{\pgfqpoint{3.948999in}{1.772009in}}{\pgfqpoint{3.944608in}{1.782608in}}{\pgfqpoint{3.936795in}{1.790421in}}%
\pgfpathcurveto{\pgfqpoint{3.928981in}{1.798235in}}{\pgfqpoint{3.918382in}{1.802625in}}{\pgfqpoint{3.907332in}{1.802625in}}%
\pgfpathcurveto{\pgfqpoint{3.896282in}{1.802625in}}{\pgfqpoint{3.885683in}{1.798235in}}{\pgfqpoint{3.877869in}{1.790421in}}%
\pgfpathcurveto{\pgfqpoint{3.870055in}{1.782608in}}{\pgfqpoint{3.865665in}{1.772009in}}{\pgfqpoint{3.865665in}{1.760958in}}%
\pgfpathcurveto{\pgfqpoint{3.865665in}{1.749908in}}{\pgfqpoint{3.870055in}{1.739309in}}{\pgfqpoint{3.877869in}{1.731496in}}%
\pgfpathcurveto{\pgfqpoint{3.885683in}{1.723682in}}{\pgfqpoint{3.896282in}{1.719292in}}{\pgfqpoint{3.907332in}{1.719292in}}%
\pgfpathclose%
\pgfusepath{stroke,fill}%
\end{pgfscope}%
\begin{pgfscope}%
\pgfpathrectangle{\pgfqpoint{0.481978in}{0.331635in}}{\pgfqpoint{9.300000in}{7.700000in}}%
\pgfusepath{clip}%
\pgfsetbuttcap%
\pgfsetroundjoin%
\definecolor{currentfill}{rgb}{1.000000,0.705882,0.509804}%
\pgfsetfillcolor{currentfill}%
\pgfsetlinewidth{0.481800pt}%
\definecolor{currentstroke}{rgb}{1.000000,1.000000,1.000000}%
\pgfsetstrokecolor{currentstroke}%
\pgfsetdash{}{0pt}%
\pgfpathmoveto{\pgfqpoint{4.745564in}{5.335530in}}%
\pgfpathcurveto{\pgfqpoint{4.756614in}{5.335530in}}{\pgfqpoint{4.767213in}{5.339920in}}{\pgfqpoint{4.775026in}{5.347734in}}%
\pgfpathcurveto{\pgfqpoint{4.782840in}{5.355548in}}{\pgfqpoint{4.787230in}{5.366147in}}{\pgfqpoint{4.787230in}{5.377197in}}%
\pgfpathcurveto{\pgfqpoint{4.787230in}{5.388247in}}{\pgfqpoint{4.782840in}{5.398846in}}{\pgfqpoint{4.775026in}{5.406660in}}%
\pgfpathcurveto{\pgfqpoint{4.767213in}{5.414473in}}{\pgfqpoint{4.756614in}{5.418864in}}{\pgfqpoint{4.745564in}{5.418864in}}%
\pgfpathcurveto{\pgfqpoint{4.734513in}{5.418864in}}{\pgfqpoint{4.723914in}{5.414473in}}{\pgfqpoint{4.716101in}{5.406660in}}%
\pgfpathcurveto{\pgfqpoint{4.708287in}{5.398846in}}{\pgfqpoint{4.703897in}{5.388247in}}{\pgfqpoint{4.703897in}{5.377197in}}%
\pgfpathcurveto{\pgfqpoint{4.703897in}{5.366147in}}{\pgfqpoint{4.708287in}{5.355548in}}{\pgfqpoint{4.716101in}{5.347734in}}%
\pgfpathcurveto{\pgfqpoint{4.723914in}{5.339920in}}{\pgfqpoint{4.734513in}{5.335530in}}{\pgfqpoint{4.745564in}{5.335530in}}%
\pgfpathclose%
\pgfusepath{stroke,fill}%
\end{pgfscope}%
\begin{pgfscope}%
\pgfpathrectangle{\pgfqpoint{0.481978in}{0.331635in}}{\pgfqpoint{9.300000in}{7.700000in}}%
\pgfusepath{clip}%
\pgfsetbuttcap%
\pgfsetroundjoin%
\definecolor{currentfill}{rgb}{1.000000,0.705882,0.509804}%
\pgfsetfillcolor{currentfill}%
\pgfsetlinewidth{0.481800pt}%
\definecolor{currentstroke}{rgb}{1.000000,1.000000,1.000000}%
\pgfsetstrokecolor{currentstroke}%
\pgfsetdash{}{0pt}%
\pgfpathmoveto{\pgfqpoint{4.104700in}{5.418936in}}%
\pgfpathcurveto{\pgfqpoint{4.115750in}{5.418936in}}{\pgfqpoint{4.126350in}{5.423326in}}{\pgfqpoint{4.134163in}{5.431139in}}%
\pgfpathcurveto{\pgfqpoint{4.141977in}{5.438953in}}{\pgfqpoint{4.146367in}{5.449552in}}{\pgfqpoint{4.146367in}{5.460602in}}%
\pgfpathcurveto{\pgfqpoint{4.146367in}{5.471652in}}{\pgfqpoint{4.141977in}{5.482251in}}{\pgfqpoint{4.134163in}{5.490065in}}%
\pgfpathcurveto{\pgfqpoint{4.126350in}{5.497879in}}{\pgfqpoint{4.115750in}{5.502269in}}{\pgfqpoint{4.104700in}{5.502269in}}%
\pgfpathcurveto{\pgfqpoint{4.093650in}{5.502269in}}{\pgfqpoint{4.083051in}{5.497879in}}{\pgfqpoint{4.075238in}{5.490065in}}%
\pgfpathcurveto{\pgfqpoint{4.067424in}{5.482251in}}{\pgfqpoint{4.063034in}{5.471652in}}{\pgfqpoint{4.063034in}{5.460602in}}%
\pgfpathcurveto{\pgfqpoint{4.063034in}{5.449552in}}{\pgfqpoint{4.067424in}{5.438953in}}{\pgfqpoint{4.075238in}{5.431139in}}%
\pgfpathcurveto{\pgfqpoint{4.083051in}{5.423326in}}{\pgfqpoint{4.093650in}{5.418936in}}{\pgfqpoint{4.104700in}{5.418936in}}%
\pgfpathclose%
\pgfusepath{stroke,fill}%
\end{pgfscope}%
\begin{pgfscope}%
\pgfpathrectangle{\pgfqpoint{0.481978in}{0.331635in}}{\pgfqpoint{9.300000in}{7.700000in}}%
\pgfusepath{clip}%
\pgfsetbuttcap%
\pgfsetroundjoin%
\definecolor{currentfill}{rgb}{1.000000,0.705882,0.509804}%
\pgfsetfillcolor{currentfill}%
\pgfsetlinewidth{0.481800pt}%
\definecolor{currentstroke}{rgb}{1.000000,1.000000,1.000000}%
\pgfsetstrokecolor{currentstroke}%
\pgfsetdash{}{0pt}%
\pgfpathmoveto{\pgfqpoint{8.529282in}{4.532862in}}%
\pgfpathcurveto{\pgfqpoint{8.540332in}{4.532862in}}{\pgfqpoint{8.550931in}{4.537253in}}{\pgfqpoint{8.558745in}{4.545066in}}%
\pgfpathcurveto{\pgfqpoint{8.566558in}{4.552880in}}{\pgfqpoint{8.570948in}{4.563479in}}{\pgfqpoint{8.570948in}{4.574529in}}%
\pgfpathcurveto{\pgfqpoint{8.570948in}{4.585579in}}{\pgfqpoint{8.566558in}{4.596178in}}{\pgfqpoint{8.558745in}{4.603992in}}%
\pgfpathcurveto{\pgfqpoint{8.550931in}{4.611806in}}{\pgfqpoint{8.540332in}{4.616196in}}{\pgfqpoint{8.529282in}{4.616196in}}%
\pgfpathcurveto{\pgfqpoint{8.518232in}{4.616196in}}{\pgfqpoint{8.507633in}{4.611806in}}{\pgfqpoint{8.499819in}{4.603992in}}%
\pgfpathcurveto{\pgfqpoint{8.492005in}{4.596178in}}{\pgfqpoint{8.487615in}{4.585579in}}{\pgfqpoint{8.487615in}{4.574529in}}%
\pgfpathcurveto{\pgfqpoint{8.487615in}{4.563479in}}{\pgfqpoint{8.492005in}{4.552880in}}{\pgfqpoint{8.499819in}{4.545066in}}%
\pgfpathcurveto{\pgfqpoint{8.507633in}{4.537253in}}{\pgfqpoint{8.518232in}{4.532862in}}{\pgfqpoint{8.529282in}{4.532862in}}%
\pgfpathclose%
\pgfusepath{stroke,fill}%
\end{pgfscope}%
\begin{pgfscope}%
\pgfpathrectangle{\pgfqpoint{0.481978in}{0.331635in}}{\pgfqpoint{9.300000in}{7.700000in}}%
\pgfusepath{clip}%
\pgfsetbuttcap%
\pgfsetroundjoin%
\definecolor{currentfill}{rgb}{1.000000,0.705882,0.509804}%
\pgfsetfillcolor{currentfill}%
\pgfsetlinewidth{0.481800pt}%
\definecolor{currentstroke}{rgb}{1.000000,1.000000,1.000000}%
\pgfsetstrokecolor{currentstroke}%
\pgfsetdash{}{0pt}%
\pgfpathmoveto{\pgfqpoint{7.907970in}{3.973013in}}%
\pgfpathcurveto{\pgfqpoint{7.919021in}{3.973013in}}{\pgfqpoint{7.929620in}{3.977403in}}{\pgfqpoint{7.937433in}{3.985217in}}%
\pgfpathcurveto{\pgfqpoint{7.945247in}{3.993030in}}{\pgfqpoint{7.949637in}{4.003629in}}{\pgfqpoint{7.949637in}{4.014679in}}%
\pgfpathcurveto{\pgfqpoint{7.949637in}{4.025730in}}{\pgfqpoint{7.945247in}{4.036329in}}{\pgfqpoint{7.937433in}{4.044142in}}%
\pgfpathcurveto{\pgfqpoint{7.929620in}{4.051956in}}{\pgfqpoint{7.919021in}{4.056346in}}{\pgfqpoint{7.907970in}{4.056346in}}%
\pgfpathcurveto{\pgfqpoint{7.896920in}{4.056346in}}{\pgfqpoint{7.886321in}{4.051956in}}{\pgfqpoint{7.878508in}{4.044142in}}%
\pgfpathcurveto{\pgfqpoint{7.870694in}{4.036329in}}{\pgfqpoint{7.866304in}{4.025730in}}{\pgfqpoint{7.866304in}{4.014679in}}%
\pgfpathcurveto{\pgfqpoint{7.866304in}{4.003629in}}{\pgfqpoint{7.870694in}{3.993030in}}{\pgfqpoint{7.878508in}{3.985217in}}%
\pgfpathcurveto{\pgfqpoint{7.886321in}{3.977403in}}{\pgfqpoint{7.896920in}{3.973013in}}{\pgfqpoint{7.907970in}{3.973013in}}%
\pgfpathclose%
\pgfusepath{stroke,fill}%
\end{pgfscope}%
\begin{pgfscope}%
\pgfpathrectangle{\pgfqpoint{0.481978in}{0.331635in}}{\pgfqpoint{9.300000in}{7.700000in}}%
\pgfusepath{clip}%
\pgfsetbuttcap%
\pgfsetroundjoin%
\definecolor{currentfill}{rgb}{1.000000,0.705882,0.509804}%
\pgfsetfillcolor{currentfill}%
\pgfsetlinewidth{0.481800pt}%
\definecolor{currentstroke}{rgb}{1.000000,1.000000,1.000000}%
\pgfsetstrokecolor{currentstroke}%
\pgfsetdash{}{0pt}%
\pgfpathmoveto{\pgfqpoint{8.433076in}{4.518523in}}%
\pgfpathcurveto{\pgfqpoint{8.444126in}{4.518523in}}{\pgfqpoint{8.454725in}{4.522914in}}{\pgfqpoint{8.462538in}{4.530727in}}%
\pgfpathcurveto{\pgfqpoint{8.470352in}{4.538541in}}{\pgfqpoint{8.474742in}{4.549140in}}{\pgfqpoint{8.474742in}{4.560190in}}%
\pgfpathcurveto{\pgfqpoint{8.474742in}{4.571240in}}{\pgfqpoint{8.470352in}{4.581839in}}{\pgfqpoint{8.462538in}{4.589653in}}%
\pgfpathcurveto{\pgfqpoint{8.454725in}{4.597467in}}{\pgfqpoint{8.444126in}{4.601857in}}{\pgfqpoint{8.433076in}{4.601857in}}%
\pgfpathcurveto{\pgfqpoint{8.422025in}{4.601857in}}{\pgfqpoint{8.411426in}{4.597467in}}{\pgfqpoint{8.403613in}{4.589653in}}%
\pgfpathcurveto{\pgfqpoint{8.395799in}{4.581839in}}{\pgfqpoint{8.391409in}{4.571240in}}{\pgfqpoint{8.391409in}{4.560190in}}%
\pgfpathcurveto{\pgfqpoint{8.391409in}{4.549140in}}{\pgfqpoint{8.395799in}{4.538541in}}{\pgfqpoint{8.403613in}{4.530727in}}%
\pgfpathcurveto{\pgfqpoint{8.411426in}{4.522914in}}{\pgfqpoint{8.422025in}{4.518523in}}{\pgfqpoint{8.433076in}{4.518523in}}%
\pgfpathclose%
\pgfusepath{stroke,fill}%
\end{pgfscope}%
\begin{pgfscope}%
\pgfpathrectangle{\pgfqpoint{0.481978in}{0.331635in}}{\pgfqpoint{9.300000in}{7.700000in}}%
\pgfusepath{clip}%
\pgfsetbuttcap%
\pgfsetroundjoin%
\definecolor{currentfill}{rgb}{1.000000,0.705882,0.509804}%
\pgfsetfillcolor{currentfill}%
\pgfsetlinewidth{0.481800pt}%
\definecolor{currentstroke}{rgb}{1.000000,1.000000,1.000000}%
\pgfsetstrokecolor{currentstroke}%
\pgfsetdash{}{0pt}%
\pgfpathmoveto{\pgfqpoint{3.128738in}{2.053690in}}%
\pgfpathcurveto{\pgfqpoint{3.139789in}{2.053690in}}{\pgfqpoint{3.150388in}{2.058080in}}{\pgfqpoint{3.158201in}{2.065894in}}%
\pgfpathcurveto{\pgfqpoint{3.166015in}{2.073708in}}{\pgfqpoint{3.170405in}{2.084307in}}{\pgfqpoint{3.170405in}{2.095357in}}%
\pgfpathcurveto{\pgfqpoint{3.170405in}{2.106407in}}{\pgfqpoint{3.166015in}{2.117006in}}{\pgfqpoint{3.158201in}{2.124820in}}%
\pgfpathcurveto{\pgfqpoint{3.150388in}{2.132633in}}{\pgfqpoint{3.139789in}{2.137024in}}{\pgfqpoint{3.128738in}{2.137024in}}%
\pgfpathcurveto{\pgfqpoint{3.117688in}{2.137024in}}{\pgfqpoint{3.107089in}{2.132633in}}{\pgfqpoint{3.099276in}{2.124820in}}%
\pgfpathcurveto{\pgfqpoint{3.091462in}{2.117006in}}{\pgfqpoint{3.087072in}{2.106407in}}{\pgfqpoint{3.087072in}{2.095357in}}%
\pgfpathcurveto{\pgfqpoint{3.087072in}{2.084307in}}{\pgfqpoint{3.091462in}{2.073708in}}{\pgfqpoint{3.099276in}{2.065894in}}%
\pgfpathcurveto{\pgfqpoint{3.107089in}{2.058080in}}{\pgfqpoint{3.117688in}{2.053690in}}{\pgfqpoint{3.128738in}{2.053690in}}%
\pgfpathclose%
\pgfusepath{stroke,fill}%
\end{pgfscope}%
\begin{pgfscope}%
\pgfpathrectangle{\pgfqpoint{0.481978in}{0.331635in}}{\pgfqpoint{9.300000in}{7.700000in}}%
\pgfusepath{clip}%
\pgfsetbuttcap%
\pgfsetroundjoin%
\definecolor{currentfill}{rgb}{1.000000,0.705882,0.509804}%
\pgfsetfillcolor{currentfill}%
\pgfsetlinewidth{0.481800pt}%
\definecolor{currentstroke}{rgb}{1.000000,1.000000,1.000000}%
\pgfsetstrokecolor{currentstroke}%
\pgfsetdash{}{0pt}%
\pgfpathmoveto{\pgfqpoint{6.750608in}{3.030787in}}%
\pgfpathcurveto{\pgfqpoint{6.761658in}{3.030787in}}{\pgfqpoint{6.772257in}{3.035178in}}{\pgfqpoint{6.780070in}{3.042991in}}%
\pgfpathcurveto{\pgfqpoint{6.787884in}{3.050805in}}{\pgfqpoint{6.792274in}{3.061404in}}{\pgfqpoint{6.792274in}{3.072454in}}%
\pgfpathcurveto{\pgfqpoint{6.792274in}{3.083504in}}{\pgfqpoint{6.787884in}{3.094103in}}{\pgfqpoint{6.780070in}{3.101917in}}%
\pgfpathcurveto{\pgfqpoint{6.772257in}{3.109730in}}{\pgfqpoint{6.761658in}{3.114121in}}{\pgfqpoint{6.750608in}{3.114121in}}%
\pgfpathcurveto{\pgfqpoint{6.739557in}{3.114121in}}{\pgfqpoint{6.728958in}{3.109730in}}{\pgfqpoint{6.721145in}{3.101917in}}%
\pgfpathcurveto{\pgfqpoint{6.713331in}{3.094103in}}{\pgfqpoint{6.708941in}{3.083504in}}{\pgfqpoint{6.708941in}{3.072454in}}%
\pgfpathcurveto{\pgfqpoint{6.708941in}{3.061404in}}{\pgfqpoint{6.713331in}{3.050805in}}{\pgfqpoint{6.721145in}{3.042991in}}%
\pgfpathcurveto{\pgfqpoint{6.728958in}{3.035178in}}{\pgfqpoint{6.739557in}{3.030787in}}{\pgfqpoint{6.750608in}{3.030787in}}%
\pgfpathclose%
\pgfusepath{stroke,fill}%
\end{pgfscope}%
\begin{pgfscope}%
\pgfpathrectangle{\pgfqpoint{0.481978in}{0.331635in}}{\pgfqpoint{9.300000in}{7.700000in}}%
\pgfusepath{clip}%
\pgfsetbuttcap%
\pgfsetroundjoin%
\definecolor{currentfill}{rgb}{1.000000,0.705882,0.509804}%
\pgfsetfillcolor{currentfill}%
\pgfsetlinewidth{0.481800pt}%
\definecolor{currentstroke}{rgb}{1.000000,1.000000,1.000000}%
\pgfsetstrokecolor{currentstroke}%
\pgfsetdash{}{0pt}%
\pgfpathmoveto{\pgfqpoint{3.434837in}{4.030228in}}%
\pgfpathcurveto{\pgfqpoint{3.445887in}{4.030228in}}{\pgfqpoint{3.456486in}{4.034618in}}{\pgfqpoint{3.464300in}{4.042432in}}%
\pgfpathcurveto{\pgfqpoint{3.472114in}{4.050245in}}{\pgfqpoint{3.476504in}{4.060844in}}{\pgfqpoint{3.476504in}{4.071894in}}%
\pgfpathcurveto{\pgfqpoint{3.476504in}{4.082945in}}{\pgfqpoint{3.472114in}{4.093544in}}{\pgfqpoint{3.464300in}{4.101357in}}%
\pgfpathcurveto{\pgfqpoint{3.456486in}{4.109171in}}{\pgfqpoint{3.445887in}{4.113561in}}{\pgfqpoint{3.434837in}{4.113561in}}%
\pgfpathcurveto{\pgfqpoint{3.423787in}{4.113561in}}{\pgfqpoint{3.413188in}{4.109171in}}{\pgfqpoint{3.405375in}{4.101357in}}%
\pgfpathcurveto{\pgfqpoint{3.397561in}{4.093544in}}{\pgfqpoint{3.393171in}{4.082945in}}{\pgfqpoint{3.393171in}{4.071894in}}%
\pgfpathcurveto{\pgfqpoint{3.393171in}{4.060844in}}{\pgfqpoint{3.397561in}{4.050245in}}{\pgfqpoint{3.405375in}{4.042432in}}%
\pgfpathcurveto{\pgfqpoint{3.413188in}{4.034618in}}{\pgfqpoint{3.423787in}{4.030228in}}{\pgfqpoint{3.434837in}{4.030228in}}%
\pgfpathclose%
\pgfusepath{stroke,fill}%
\end{pgfscope}%
\begin{pgfscope}%
\pgfpathrectangle{\pgfqpoint{0.481978in}{0.331635in}}{\pgfqpoint{9.300000in}{7.700000in}}%
\pgfusepath{clip}%
\pgfsetbuttcap%
\pgfsetroundjoin%
\definecolor{currentfill}{rgb}{1.000000,0.705882,0.509804}%
\pgfsetfillcolor{currentfill}%
\pgfsetlinewidth{0.481800pt}%
\definecolor{currentstroke}{rgb}{1.000000,1.000000,1.000000}%
\pgfsetstrokecolor{currentstroke}%
\pgfsetdash{}{0pt}%
\pgfpathmoveto{\pgfqpoint{2.745959in}{2.280244in}}%
\pgfpathcurveto{\pgfqpoint{2.757010in}{2.280244in}}{\pgfqpoint{2.767609in}{2.284634in}}{\pgfqpoint{2.775422in}{2.292448in}}%
\pgfpathcurveto{\pgfqpoint{2.783236in}{2.300262in}}{\pgfqpoint{2.787626in}{2.310861in}}{\pgfqpoint{2.787626in}{2.321911in}}%
\pgfpathcurveto{\pgfqpoint{2.787626in}{2.332961in}}{\pgfqpoint{2.783236in}{2.343560in}}{\pgfqpoint{2.775422in}{2.351374in}}%
\pgfpathcurveto{\pgfqpoint{2.767609in}{2.359187in}}{\pgfqpoint{2.757010in}{2.363577in}}{\pgfqpoint{2.745959in}{2.363577in}}%
\pgfpathcurveto{\pgfqpoint{2.734909in}{2.363577in}}{\pgfqpoint{2.724310in}{2.359187in}}{\pgfqpoint{2.716497in}{2.351374in}}%
\pgfpathcurveto{\pgfqpoint{2.708683in}{2.343560in}}{\pgfqpoint{2.704293in}{2.332961in}}{\pgfqpoint{2.704293in}{2.321911in}}%
\pgfpathcurveto{\pgfqpoint{2.704293in}{2.310861in}}{\pgfqpoint{2.708683in}{2.300262in}}{\pgfqpoint{2.716497in}{2.292448in}}%
\pgfpathcurveto{\pgfqpoint{2.724310in}{2.284634in}}{\pgfqpoint{2.734909in}{2.280244in}}{\pgfqpoint{2.745959in}{2.280244in}}%
\pgfpathclose%
\pgfusepath{stroke,fill}%
\end{pgfscope}%
\begin{pgfscope}%
\pgfpathrectangle{\pgfqpoint{0.481978in}{0.331635in}}{\pgfqpoint{9.300000in}{7.700000in}}%
\pgfusepath{clip}%
\pgfsetbuttcap%
\pgfsetroundjoin%
\definecolor{currentfill}{rgb}{1.000000,0.705882,0.509804}%
\pgfsetfillcolor{currentfill}%
\pgfsetlinewidth{0.481800pt}%
\definecolor{currentstroke}{rgb}{1.000000,1.000000,1.000000}%
\pgfsetstrokecolor{currentstroke}%
\pgfsetdash{}{0pt}%
\pgfpathmoveto{\pgfqpoint{8.714562in}{5.295792in}}%
\pgfpathcurveto{\pgfqpoint{8.725612in}{5.295792in}}{\pgfqpoint{8.736211in}{5.300182in}}{\pgfqpoint{8.744024in}{5.307996in}}%
\pgfpathcurveto{\pgfqpoint{8.751838in}{5.315809in}}{\pgfqpoint{8.756228in}{5.326408in}}{\pgfqpoint{8.756228in}{5.337458in}}%
\pgfpathcurveto{\pgfqpoint{8.756228in}{5.348509in}}{\pgfqpoint{8.751838in}{5.359108in}}{\pgfqpoint{8.744024in}{5.366921in}}%
\pgfpathcurveto{\pgfqpoint{8.736211in}{5.374735in}}{\pgfqpoint{8.725612in}{5.379125in}}{\pgfqpoint{8.714562in}{5.379125in}}%
\pgfpathcurveto{\pgfqpoint{8.703512in}{5.379125in}}{\pgfqpoint{8.692913in}{5.374735in}}{\pgfqpoint{8.685099in}{5.366921in}}%
\pgfpathcurveto{\pgfqpoint{8.677285in}{5.359108in}}{\pgfqpoint{8.672895in}{5.348509in}}{\pgfqpoint{8.672895in}{5.337458in}}%
\pgfpathcurveto{\pgfqpoint{8.672895in}{5.326408in}}{\pgfqpoint{8.677285in}{5.315809in}}{\pgfqpoint{8.685099in}{5.307996in}}%
\pgfpathcurveto{\pgfqpoint{8.692913in}{5.300182in}}{\pgfqpoint{8.703512in}{5.295792in}}{\pgfqpoint{8.714562in}{5.295792in}}%
\pgfpathclose%
\pgfusepath{stroke,fill}%
\end{pgfscope}%
\begin{pgfscope}%
\pgfpathrectangle{\pgfqpoint{0.481978in}{0.331635in}}{\pgfqpoint{9.300000in}{7.700000in}}%
\pgfusepath{clip}%
\pgfsetbuttcap%
\pgfsetroundjoin%
\definecolor{currentfill}{rgb}{0.552941,0.898039,0.631373}%
\pgfsetfillcolor{currentfill}%
\pgfsetlinewidth{0.481800pt}%
\definecolor{currentstroke}{rgb}{1.000000,1.000000,1.000000}%
\pgfsetstrokecolor{currentstroke}%
\pgfsetdash{}{0pt}%
\pgfpathmoveto{\pgfqpoint{5.163299in}{7.301501in}}%
\pgfpathcurveto{\pgfqpoint{5.174349in}{7.301501in}}{\pgfqpoint{5.184948in}{7.305892in}}{\pgfqpoint{5.192762in}{7.313705in}}%
\pgfpathcurveto{\pgfqpoint{5.200575in}{7.321519in}}{\pgfqpoint{5.204966in}{7.332118in}}{\pgfqpoint{5.204966in}{7.343168in}}%
\pgfpathcurveto{\pgfqpoint{5.204966in}{7.354218in}}{\pgfqpoint{5.200575in}{7.364817in}}{\pgfqpoint{5.192762in}{7.372631in}}%
\pgfpathcurveto{\pgfqpoint{5.184948in}{7.380445in}}{\pgfqpoint{5.174349in}{7.384835in}}{\pgfqpoint{5.163299in}{7.384835in}}%
\pgfpathcurveto{\pgfqpoint{5.152249in}{7.384835in}}{\pgfqpoint{5.141650in}{7.380445in}}{\pgfqpoint{5.133836in}{7.372631in}}%
\pgfpathcurveto{\pgfqpoint{5.126023in}{7.364817in}}{\pgfqpoint{5.121632in}{7.354218in}}{\pgfqpoint{5.121632in}{7.343168in}}%
\pgfpathcurveto{\pgfqpoint{5.121632in}{7.332118in}}{\pgfqpoint{5.126023in}{7.321519in}}{\pgfqpoint{5.133836in}{7.313705in}}%
\pgfpathcurveto{\pgfqpoint{5.141650in}{7.305892in}}{\pgfqpoint{5.152249in}{7.301501in}}{\pgfqpoint{5.163299in}{7.301501in}}%
\pgfpathclose%
\pgfusepath{stroke,fill}%
\end{pgfscope}%
\begin{pgfscope}%
\pgfpathrectangle{\pgfqpoint{0.481978in}{0.331635in}}{\pgfqpoint{9.300000in}{7.700000in}}%
\pgfusepath{clip}%
\pgfsetbuttcap%
\pgfsetroundjoin%
\definecolor{currentfill}{rgb}{0.552941,0.898039,0.631373}%
\pgfsetfillcolor{currentfill}%
\pgfsetlinewidth{0.481800pt}%
\definecolor{currentstroke}{rgb}{1.000000,1.000000,1.000000}%
\pgfsetstrokecolor{currentstroke}%
\pgfsetdash{}{0pt}%
\pgfpathmoveto{\pgfqpoint{3.761360in}{5.722980in}}%
\pgfpathcurveto{\pgfqpoint{3.772410in}{5.722980in}}{\pgfqpoint{3.783009in}{5.727370in}}{\pgfqpoint{3.790823in}{5.735183in}}%
\pgfpathcurveto{\pgfqpoint{3.798637in}{5.742997in}}{\pgfqpoint{3.803027in}{5.753596in}}{\pgfqpoint{3.803027in}{5.764646in}}%
\pgfpathcurveto{\pgfqpoint{3.803027in}{5.775696in}}{\pgfqpoint{3.798637in}{5.786295in}}{\pgfqpoint{3.790823in}{5.794109in}}%
\pgfpathcurveto{\pgfqpoint{3.783009in}{5.801923in}}{\pgfqpoint{3.772410in}{5.806313in}}{\pgfqpoint{3.761360in}{5.806313in}}%
\pgfpathcurveto{\pgfqpoint{3.750310in}{5.806313in}}{\pgfqpoint{3.739711in}{5.801923in}}{\pgfqpoint{3.731897in}{5.794109in}}%
\pgfpathcurveto{\pgfqpoint{3.724084in}{5.786295in}}{\pgfqpoint{3.719693in}{5.775696in}}{\pgfqpoint{3.719693in}{5.764646in}}%
\pgfpathcurveto{\pgfqpoint{3.719693in}{5.753596in}}{\pgfqpoint{3.724084in}{5.742997in}}{\pgfqpoint{3.731897in}{5.735183in}}%
\pgfpathcurveto{\pgfqpoint{3.739711in}{5.727370in}}{\pgfqpoint{3.750310in}{5.722980in}}{\pgfqpoint{3.761360in}{5.722980in}}%
\pgfpathclose%
\pgfusepath{stroke,fill}%
\end{pgfscope}%
\begin{pgfscope}%
\pgfpathrectangle{\pgfqpoint{0.481978in}{0.331635in}}{\pgfqpoint{9.300000in}{7.700000in}}%
\pgfusepath{clip}%
\pgfsetbuttcap%
\pgfsetroundjoin%
\definecolor{currentfill}{rgb}{0.552941,0.898039,0.631373}%
\pgfsetfillcolor{currentfill}%
\pgfsetlinewidth{0.481800pt}%
\definecolor{currentstroke}{rgb}{1.000000,1.000000,1.000000}%
\pgfsetstrokecolor{currentstroke}%
\pgfsetdash{}{0pt}%
\pgfpathmoveto{\pgfqpoint{3.580350in}{3.046636in}}%
\pgfpathcurveto{\pgfqpoint{3.591400in}{3.046636in}}{\pgfqpoint{3.601999in}{3.051026in}}{\pgfqpoint{3.609813in}{3.058840in}}%
\pgfpathcurveto{\pgfqpoint{3.617626in}{3.066654in}}{\pgfqpoint{3.622017in}{3.077253in}}{\pgfqpoint{3.622017in}{3.088303in}}%
\pgfpathcurveto{\pgfqpoint{3.622017in}{3.099353in}}{\pgfqpoint{3.617626in}{3.109952in}}{\pgfqpoint{3.609813in}{3.117765in}}%
\pgfpathcurveto{\pgfqpoint{3.601999in}{3.125579in}}{\pgfqpoint{3.591400in}{3.129969in}}{\pgfqpoint{3.580350in}{3.129969in}}%
\pgfpathcurveto{\pgfqpoint{3.569300in}{3.129969in}}{\pgfqpoint{3.558701in}{3.125579in}}{\pgfqpoint{3.550887in}{3.117765in}}%
\pgfpathcurveto{\pgfqpoint{3.543073in}{3.109952in}}{\pgfqpoint{3.538683in}{3.099353in}}{\pgfqpoint{3.538683in}{3.088303in}}%
\pgfpathcurveto{\pgfqpoint{3.538683in}{3.077253in}}{\pgfqpoint{3.543073in}{3.066654in}}{\pgfqpoint{3.550887in}{3.058840in}}%
\pgfpathcurveto{\pgfqpoint{3.558701in}{3.051026in}}{\pgfqpoint{3.569300in}{3.046636in}}{\pgfqpoint{3.580350in}{3.046636in}}%
\pgfpathclose%
\pgfusepath{stroke,fill}%
\end{pgfscope}%
\begin{pgfscope}%
\pgfpathrectangle{\pgfqpoint{0.481978in}{0.331635in}}{\pgfqpoint{9.300000in}{7.700000in}}%
\pgfusepath{clip}%
\pgfsetbuttcap%
\pgfsetroundjoin%
\definecolor{currentfill}{rgb}{0.552941,0.898039,0.631373}%
\pgfsetfillcolor{currentfill}%
\pgfsetlinewidth{0.481800pt}%
\definecolor{currentstroke}{rgb}{1.000000,1.000000,1.000000}%
\pgfsetstrokecolor{currentstroke}%
\pgfsetdash{}{0pt}%
\pgfpathmoveto{\pgfqpoint{8.242673in}{4.179758in}}%
\pgfpathcurveto{\pgfqpoint{8.253723in}{4.179758in}}{\pgfqpoint{8.264322in}{4.184148in}}{\pgfqpoint{8.272136in}{4.191962in}}%
\pgfpathcurveto{\pgfqpoint{8.279949in}{4.199775in}}{\pgfqpoint{8.284340in}{4.210374in}}{\pgfqpoint{8.284340in}{4.221424in}}%
\pgfpathcurveto{\pgfqpoint{8.284340in}{4.232475in}}{\pgfqpoint{8.279949in}{4.243074in}}{\pgfqpoint{8.272136in}{4.250887in}}%
\pgfpathcurveto{\pgfqpoint{8.264322in}{4.258701in}}{\pgfqpoint{8.253723in}{4.263091in}}{\pgfqpoint{8.242673in}{4.263091in}}%
\pgfpathcurveto{\pgfqpoint{8.231623in}{4.263091in}}{\pgfqpoint{8.221024in}{4.258701in}}{\pgfqpoint{8.213210in}{4.250887in}}%
\pgfpathcurveto{\pgfqpoint{8.205396in}{4.243074in}}{\pgfqpoint{8.201006in}{4.232475in}}{\pgfqpoint{8.201006in}{4.221424in}}%
\pgfpathcurveto{\pgfqpoint{8.201006in}{4.210374in}}{\pgfqpoint{8.205396in}{4.199775in}}{\pgfqpoint{8.213210in}{4.191962in}}%
\pgfpathcurveto{\pgfqpoint{8.221024in}{4.184148in}}{\pgfqpoint{8.231623in}{4.179758in}}{\pgfqpoint{8.242673in}{4.179758in}}%
\pgfpathclose%
\pgfusepath{stroke,fill}%
\end{pgfscope}%
\begin{pgfscope}%
\pgfpathrectangle{\pgfqpoint{0.481978in}{0.331635in}}{\pgfqpoint{9.300000in}{7.700000in}}%
\pgfusepath{clip}%
\pgfsetbuttcap%
\pgfsetroundjoin%
\definecolor{currentfill}{rgb}{0.552941,0.898039,0.631373}%
\pgfsetfillcolor{currentfill}%
\pgfsetlinewidth{0.481800pt}%
\definecolor{currentstroke}{rgb}{1.000000,1.000000,1.000000}%
\pgfsetstrokecolor{currentstroke}%
\pgfsetdash{}{0pt}%
\pgfpathmoveto{\pgfqpoint{4.433234in}{5.029314in}}%
\pgfpathcurveto{\pgfqpoint{4.444284in}{5.029314in}}{\pgfqpoint{4.454883in}{5.033704in}}{\pgfqpoint{4.462696in}{5.041518in}}%
\pgfpathcurveto{\pgfqpoint{4.470510in}{5.049332in}}{\pgfqpoint{4.474900in}{5.059931in}}{\pgfqpoint{4.474900in}{5.070981in}}%
\pgfpathcurveto{\pgfqpoint{4.474900in}{5.082031in}}{\pgfqpoint{4.470510in}{5.092630in}}{\pgfqpoint{4.462696in}{5.100444in}}%
\pgfpathcurveto{\pgfqpoint{4.454883in}{5.108257in}}{\pgfqpoint{4.444284in}{5.112648in}}{\pgfqpoint{4.433234in}{5.112648in}}%
\pgfpathcurveto{\pgfqpoint{4.422183in}{5.112648in}}{\pgfqpoint{4.411584in}{5.108257in}}{\pgfqpoint{4.403771in}{5.100444in}}%
\pgfpathcurveto{\pgfqpoint{4.395957in}{5.092630in}}{\pgfqpoint{4.391567in}{5.082031in}}{\pgfqpoint{4.391567in}{5.070981in}}%
\pgfpathcurveto{\pgfqpoint{4.391567in}{5.059931in}}{\pgfqpoint{4.395957in}{5.049332in}}{\pgfqpoint{4.403771in}{5.041518in}}%
\pgfpathcurveto{\pgfqpoint{4.411584in}{5.033704in}}{\pgfqpoint{4.422183in}{5.029314in}}{\pgfqpoint{4.433234in}{5.029314in}}%
\pgfpathclose%
\pgfusepath{stroke,fill}%
\end{pgfscope}%
\begin{pgfscope}%
\pgfpathrectangle{\pgfqpoint{0.481978in}{0.331635in}}{\pgfqpoint{9.300000in}{7.700000in}}%
\pgfusepath{clip}%
\pgfsetbuttcap%
\pgfsetroundjoin%
\definecolor{currentfill}{rgb}{0.552941,0.898039,0.631373}%
\pgfsetfillcolor{currentfill}%
\pgfsetlinewidth{0.481800pt}%
\definecolor{currentstroke}{rgb}{1.000000,1.000000,1.000000}%
\pgfsetstrokecolor{currentstroke}%
\pgfsetdash{}{0pt}%
\pgfpathmoveto{\pgfqpoint{9.194655in}{5.622269in}}%
\pgfpathcurveto{\pgfqpoint{9.205705in}{5.622269in}}{\pgfqpoint{9.216304in}{5.626659in}}{\pgfqpoint{9.224117in}{5.634473in}}%
\pgfpathcurveto{\pgfqpoint{9.231931in}{5.642286in}}{\pgfqpoint{9.236321in}{5.652885in}}{\pgfqpoint{9.236321in}{5.663935in}}%
\pgfpathcurveto{\pgfqpoint{9.236321in}{5.674985in}}{\pgfqpoint{9.231931in}{5.685584in}}{\pgfqpoint{9.224117in}{5.693398in}}%
\pgfpathcurveto{\pgfqpoint{9.216304in}{5.701212in}}{\pgfqpoint{9.205705in}{5.705602in}}{\pgfqpoint{9.194655in}{5.705602in}}%
\pgfpathcurveto{\pgfqpoint{9.183604in}{5.705602in}}{\pgfqpoint{9.173005in}{5.701212in}}{\pgfqpoint{9.165192in}{5.693398in}}%
\pgfpathcurveto{\pgfqpoint{9.157378in}{5.685584in}}{\pgfqpoint{9.152988in}{5.674985in}}{\pgfqpoint{9.152988in}{5.663935in}}%
\pgfpathcurveto{\pgfqpoint{9.152988in}{5.652885in}}{\pgfqpoint{9.157378in}{5.642286in}}{\pgfqpoint{9.165192in}{5.634473in}}%
\pgfpathcurveto{\pgfqpoint{9.173005in}{5.626659in}}{\pgfqpoint{9.183604in}{5.622269in}}{\pgfqpoint{9.194655in}{5.622269in}}%
\pgfpathclose%
\pgfusepath{stroke,fill}%
\end{pgfscope}%
\begin{pgfscope}%
\pgfpathrectangle{\pgfqpoint{0.481978in}{0.331635in}}{\pgfqpoint{9.300000in}{7.700000in}}%
\pgfusepath{clip}%
\pgfsetbuttcap%
\pgfsetroundjoin%
\definecolor{currentfill}{rgb}{0.552941,0.898039,0.631373}%
\pgfsetfillcolor{currentfill}%
\pgfsetlinewidth{0.481800pt}%
\definecolor{currentstroke}{rgb}{1.000000,1.000000,1.000000}%
\pgfsetstrokecolor{currentstroke}%
\pgfsetdash{}{0pt}%
\pgfpathmoveto{\pgfqpoint{9.032589in}{5.038453in}}%
\pgfpathcurveto{\pgfqpoint{9.043639in}{5.038453in}}{\pgfqpoint{9.054238in}{5.042843in}}{\pgfqpoint{9.062052in}{5.050657in}}%
\pgfpathcurveto{\pgfqpoint{9.069866in}{5.058470in}}{\pgfqpoint{9.074256in}{5.069069in}}{\pgfqpoint{9.074256in}{5.080119in}}%
\pgfpathcurveto{\pgfqpoint{9.074256in}{5.091169in}}{\pgfqpoint{9.069866in}{5.101769in}}{\pgfqpoint{9.062052in}{5.109582in}}%
\pgfpathcurveto{\pgfqpoint{9.054238in}{5.117396in}}{\pgfqpoint{9.043639in}{5.121786in}}{\pgfqpoint{9.032589in}{5.121786in}}%
\pgfpathcurveto{\pgfqpoint{9.021539in}{5.121786in}}{\pgfqpoint{9.010940in}{5.117396in}}{\pgfqpoint{9.003127in}{5.109582in}}%
\pgfpathcurveto{\pgfqpoint{8.995313in}{5.101769in}}{\pgfqpoint{8.990923in}{5.091169in}}{\pgfqpoint{8.990923in}{5.080119in}}%
\pgfpathcurveto{\pgfqpoint{8.990923in}{5.069069in}}{\pgfqpoint{8.995313in}{5.058470in}}{\pgfqpoint{9.003127in}{5.050657in}}%
\pgfpathcurveto{\pgfqpoint{9.010940in}{5.042843in}}{\pgfqpoint{9.021539in}{5.038453in}}{\pgfqpoint{9.032589in}{5.038453in}}%
\pgfpathclose%
\pgfusepath{stroke,fill}%
\end{pgfscope}%
\begin{pgfscope}%
\pgfpathrectangle{\pgfqpoint{0.481978in}{0.331635in}}{\pgfqpoint{9.300000in}{7.700000in}}%
\pgfusepath{clip}%
\pgfsetbuttcap%
\pgfsetroundjoin%
\definecolor{currentfill}{rgb}{0.552941,0.898039,0.631373}%
\pgfsetfillcolor{currentfill}%
\pgfsetlinewidth{0.481800pt}%
\definecolor{currentstroke}{rgb}{1.000000,1.000000,1.000000}%
\pgfsetstrokecolor{currentstroke}%
\pgfsetdash{}{0pt}%
\pgfpathmoveto{\pgfqpoint{3.942708in}{5.631876in}}%
\pgfpathcurveto{\pgfqpoint{3.953758in}{5.631876in}}{\pgfqpoint{3.964357in}{5.636267in}}{\pgfqpoint{3.972171in}{5.644080in}}%
\pgfpathcurveto{\pgfqpoint{3.979985in}{5.651894in}}{\pgfqpoint{3.984375in}{5.662493in}}{\pgfqpoint{3.984375in}{5.673543in}}%
\pgfpathcurveto{\pgfqpoint{3.984375in}{5.684593in}}{\pgfqpoint{3.979985in}{5.695192in}}{\pgfqpoint{3.972171in}{5.703006in}}%
\pgfpathcurveto{\pgfqpoint{3.964357in}{5.710819in}}{\pgfqpoint{3.953758in}{5.715210in}}{\pgfqpoint{3.942708in}{5.715210in}}%
\pgfpathcurveto{\pgfqpoint{3.931658in}{5.715210in}}{\pgfqpoint{3.921059in}{5.710819in}}{\pgfqpoint{3.913245in}{5.703006in}}%
\pgfpathcurveto{\pgfqpoint{3.905432in}{5.695192in}}{\pgfqpoint{3.901042in}{5.684593in}}{\pgfqpoint{3.901042in}{5.673543in}}%
\pgfpathcurveto{\pgfqpoint{3.901042in}{5.662493in}}{\pgfqpoint{3.905432in}{5.651894in}}{\pgfqpoint{3.913245in}{5.644080in}}%
\pgfpathcurveto{\pgfqpoint{3.921059in}{5.636267in}}{\pgfqpoint{3.931658in}{5.631876in}}{\pgfqpoint{3.942708in}{5.631876in}}%
\pgfpathclose%
\pgfusepath{stroke,fill}%
\end{pgfscope}%
\begin{pgfscope}%
\pgfpathrectangle{\pgfqpoint{0.481978in}{0.331635in}}{\pgfqpoint{9.300000in}{7.700000in}}%
\pgfusepath{clip}%
\pgfsetbuttcap%
\pgfsetroundjoin%
\definecolor{currentfill}{rgb}{0.552941,0.898039,0.631373}%
\pgfsetfillcolor{currentfill}%
\pgfsetlinewidth{0.481800pt}%
\definecolor{currentstroke}{rgb}{1.000000,1.000000,1.000000}%
\pgfsetstrokecolor{currentstroke}%
\pgfsetdash{}{0pt}%
\pgfpathmoveto{\pgfqpoint{5.874679in}{2.681112in}}%
\pgfpathcurveto{\pgfqpoint{5.885729in}{2.681112in}}{\pgfqpoint{5.896328in}{2.685502in}}{\pgfqpoint{5.904141in}{2.693316in}}%
\pgfpathcurveto{\pgfqpoint{5.911955in}{2.701129in}}{\pgfqpoint{5.916345in}{2.711728in}}{\pgfqpoint{5.916345in}{2.722779in}}%
\pgfpathcurveto{\pgfqpoint{5.916345in}{2.733829in}}{\pgfqpoint{5.911955in}{2.744428in}}{\pgfqpoint{5.904141in}{2.752241in}}%
\pgfpathcurveto{\pgfqpoint{5.896328in}{2.760055in}}{\pgfqpoint{5.885729in}{2.764445in}}{\pgfqpoint{5.874679in}{2.764445in}}%
\pgfpathcurveto{\pgfqpoint{5.863628in}{2.764445in}}{\pgfqpoint{5.853029in}{2.760055in}}{\pgfqpoint{5.845216in}{2.752241in}}%
\pgfpathcurveto{\pgfqpoint{5.837402in}{2.744428in}}{\pgfqpoint{5.833012in}{2.733829in}}{\pgfqpoint{5.833012in}{2.722779in}}%
\pgfpathcurveto{\pgfqpoint{5.833012in}{2.711728in}}{\pgfqpoint{5.837402in}{2.701129in}}{\pgfqpoint{5.845216in}{2.693316in}}%
\pgfpathcurveto{\pgfqpoint{5.853029in}{2.685502in}}{\pgfqpoint{5.863628in}{2.681112in}}{\pgfqpoint{5.874679in}{2.681112in}}%
\pgfpathclose%
\pgfusepath{stroke,fill}%
\end{pgfscope}%
\begin{pgfscope}%
\pgfpathrectangle{\pgfqpoint{0.481978in}{0.331635in}}{\pgfqpoint{9.300000in}{7.700000in}}%
\pgfusepath{clip}%
\pgfsetbuttcap%
\pgfsetroundjoin%
\definecolor{currentfill}{rgb}{0.552941,0.898039,0.631373}%
\pgfsetfillcolor{currentfill}%
\pgfsetlinewidth{0.481800pt}%
\definecolor{currentstroke}{rgb}{1.000000,1.000000,1.000000}%
\pgfsetstrokecolor{currentstroke}%
\pgfsetdash{}{0pt}%
\pgfpathmoveto{\pgfqpoint{6.449863in}{3.139229in}}%
\pgfpathcurveto{\pgfqpoint{6.460914in}{3.139229in}}{\pgfqpoint{6.471513in}{3.143620in}}{\pgfqpoint{6.479326in}{3.151433in}}%
\pgfpathcurveto{\pgfqpoint{6.487140in}{3.159247in}}{\pgfqpoint{6.491530in}{3.169846in}}{\pgfqpoint{6.491530in}{3.180896in}}%
\pgfpathcurveto{\pgfqpoint{6.491530in}{3.191946in}}{\pgfqpoint{6.487140in}{3.202545in}}{\pgfqpoint{6.479326in}{3.210359in}}%
\pgfpathcurveto{\pgfqpoint{6.471513in}{3.218173in}}{\pgfqpoint{6.460914in}{3.222563in}}{\pgfqpoint{6.449863in}{3.222563in}}%
\pgfpathcurveto{\pgfqpoint{6.438813in}{3.222563in}}{\pgfqpoint{6.428214in}{3.218173in}}{\pgfqpoint{6.420401in}{3.210359in}}%
\pgfpathcurveto{\pgfqpoint{6.412587in}{3.202545in}}{\pgfqpoint{6.408197in}{3.191946in}}{\pgfqpoint{6.408197in}{3.180896in}}%
\pgfpathcurveto{\pgfqpoint{6.408197in}{3.169846in}}{\pgfqpoint{6.412587in}{3.159247in}}{\pgfqpoint{6.420401in}{3.151433in}}%
\pgfpathcurveto{\pgfqpoint{6.428214in}{3.143620in}}{\pgfqpoint{6.438813in}{3.139229in}}{\pgfqpoint{6.449863in}{3.139229in}}%
\pgfpathclose%
\pgfusepath{stroke,fill}%
\end{pgfscope}%
\begin{pgfscope}%
\pgfpathrectangle{\pgfqpoint{0.481978in}{0.331635in}}{\pgfqpoint{9.300000in}{7.700000in}}%
\pgfusepath{clip}%
\pgfsetbuttcap%
\pgfsetroundjoin%
\definecolor{currentfill}{rgb}{0.552941,0.898039,0.631373}%
\pgfsetfillcolor{currentfill}%
\pgfsetlinewidth{0.481800pt}%
\definecolor{currentstroke}{rgb}{1.000000,1.000000,1.000000}%
\pgfsetstrokecolor{currentstroke}%
\pgfsetdash{}{0pt}%
\pgfpathmoveto{\pgfqpoint{2.941597in}{6.325321in}}%
\pgfpathcurveto{\pgfqpoint{2.952647in}{6.325321in}}{\pgfqpoint{2.963246in}{6.329712in}}{\pgfqpoint{2.971060in}{6.337525in}}%
\pgfpathcurveto{\pgfqpoint{2.978874in}{6.345339in}}{\pgfqpoint{2.983264in}{6.355938in}}{\pgfqpoint{2.983264in}{6.366988in}}%
\pgfpathcurveto{\pgfqpoint{2.983264in}{6.378038in}}{\pgfqpoint{2.978874in}{6.388637in}}{\pgfqpoint{2.971060in}{6.396451in}}%
\pgfpathcurveto{\pgfqpoint{2.963246in}{6.404264in}}{\pgfqpoint{2.952647in}{6.408655in}}{\pgfqpoint{2.941597in}{6.408655in}}%
\pgfpathcurveto{\pgfqpoint{2.930547in}{6.408655in}}{\pgfqpoint{2.919948in}{6.404264in}}{\pgfqpoint{2.912135in}{6.396451in}}%
\pgfpathcurveto{\pgfqpoint{2.904321in}{6.388637in}}{\pgfqpoint{2.899931in}{6.378038in}}{\pgfqpoint{2.899931in}{6.366988in}}%
\pgfpathcurveto{\pgfqpoint{2.899931in}{6.355938in}}{\pgfqpoint{2.904321in}{6.345339in}}{\pgfqpoint{2.912135in}{6.337525in}}%
\pgfpathcurveto{\pgfqpoint{2.919948in}{6.329712in}}{\pgfqpoint{2.930547in}{6.325321in}}{\pgfqpoint{2.941597in}{6.325321in}}%
\pgfpathclose%
\pgfusepath{stroke,fill}%
\end{pgfscope}%
\begin{pgfscope}%
\pgfpathrectangle{\pgfqpoint{0.481978in}{0.331635in}}{\pgfqpoint{9.300000in}{7.700000in}}%
\pgfusepath{clip}%
\pgfsetbuttcap%
\pgfsetroundjoin%
\definecolor{currentfill}{rgb}{0.552941,0.898039,0.631373}%
\pgfsetfillcolor{currentfill}%
\pgfsetlinewidth{0.481800pt}%
\definecolor{currentstroke}{rgb}{1.000000,1.000000,1.000000}%
\pgfsetstrokecolor{currentstroke}%
\pgfsetdash{}{0pt}%
\pgfpathmoveto{\pgfqpoint{4.671581in}{6.073104in}}%
\pgfpathcurveto{\pgfqpoint{4.682631in}{6.073104in}}{\pgfqpoint{4.693230in}{6.077495in}}{\pgfqpoint{4.701044in}{6.085308in}}%
\pgfpathcurveto{\pgfqpoint{4.708857in}{6.093122in}}{\pgfqpoint{4.713248in}{6.103721in}}{\pgfqpoint{4.713248in}{6.114771in}}%
\pgfpathcurveto{\pgfqpoint{4.713248in}{6.125821in}}{\pgfqpoint{4.708857in}{6.136420in}}{\pgfqpoint{4.701044in}{6.144234in}}%
\pgfpathcurveto{\pgfqpoint{4.693230in}{6.152047in}}{\pgfqpoint{4.682631in}{6.156438in}}{\pgfqpoint{4.671581in}{6.156438in}}%
\pgfpathcurveto{\pgfqpoint{4.660531in}{6.156438in}}{\pgfqpoint{4.649932in}{6.152047in}}{\pgfqpoint{4.642118in}{6.144234in}}%
\pgfpathcurveto{\pgfqpoint{4.634304in}{6.136420in}}{\pgfqpoint{4.629914in}{6.125821in}}{\pgfqpoint{4.629914in}{6.114771in}}%
\pgfpathcurveto{\pgfqpoint{4.629914in}{6.103721in}}{\pgfqpoint{4.634304in}{6.093122in}}{\pgfqpoint{4.642118in}{6.085308in}}%
\pgfpathcurveto{\pgfqpoint{4.649932in}{6.077495in}}{\pgfqpoint{4.660531in}{6.073104in}}{\pgfqpoint{4.671581in}{6.073104in}}%
\pgfpathclose%
\pgfusepath{stroke,fill}%
\end{pgfscope}%
\begin{pgfscope}%
\pgfpathrectangle{\pgfqpoint{0.481978in}{0.331635in}}{\pgfqpoint{9.300000in}{7.700000in}}%
\pgfusepath{clip}%
\pgfsetbuttcap%
\pgfsetroundjoin%
\definecolor{currentfill}{rgb}{0.552941,0.898039,0.631373}%
\pgfsetfillcolor{currentfill}%
\pgfsetlinewidth{0.481800pt}%
\definecolor{currentstroke}{rgb}{1.000000,1.000000,1.000000}%
\pgfsetstrokecolor{currentstroke}%
\pgfsetdash{}{0pt}%
\pgfpathmoveto{\pgfqpoint{3.259508in}{2.916456in}}%
\pgfpathcurveto{\pgfqpoint{3.270559in}{2.916456in}}{\pgfqpoint{3.281158in}{2.920846in}}{\pgfqpoint{3.288971in}{2.928660in}}%
\pgfpathcurveto{\pgfqpoint{3.296785in}{2.936473in}}{\pgfqpoint{3.301175in}{2.947072in}}{\pgfqpoint{3.301175in}{2.958122in}}%
\pgfpathcurveto{\pgfqpoint{3.301175in}{2.969172in}}{\pgfqpoint{3.296785in}{2.979771in}}{\pgfqpoint{3.288971in}{2.987585in}}%
\pgfpathcurveto{\pgfqpoint{3.281158in}{2.995399in}}{\pgfqpoint{3.270559in}{2.999789in}}{\pgfqpoint{3.259508in}{2.999789in}}%
\pgfpathcurveto{\pgfqpoint{3.248458in}{2.999789in}}{\pgfqpoint{3.237859in}{2.995399in}}{\pgfqpoint{3.230046in}{2.987585in}}%
\pgfpathcurveto{\pgfqpoint{3.222232in}{2.979771in}}{\pgfqpoint{3.217842in}{2.969172in}}{\pgfqpoint{3.217842in}{2.958122in}}%
\pgfpathcurveto{\pgfqpoint{3.217842in}{2.947072in}}{\pgfqpoint{3.222232in}{2.936473in}}{\pgfqpoint{3.230046in}{2.928660in}}%
\pgfpathcurveto{\pgfqpoint{3.237859in}{2.920846in}}{\pgfqpoint{3.248458in}{2.916456in}}{\pgfqpoint{3.259508in}{2.916456in}}%
\pgfpathclose%
\pgfusepath{stroke,fill}%
\end{pgfscope}%
\begin{pgfscope}%
\pgfpathrectangle{\pgfqpoint{0.481978in}{0.331635in}}{\pgfqpoint{9.300000in}{7.700000in}}%
\pgfusepath{clip}%
\pgfsetbuttcap%
\pgfsetroundjoin%
\definecolor{currentfill}{rgb}{0.552941,0.898039,0.631373}%
\pgfsetfillcolor{currentfill}%
\pgfsetlinewidth{0.481800pt}%
\definecolor{currentstroke}{rgb}{1.000000,1.000000,1.000000}%
\pgfsetstrokecolor{currentstroke}%
\pgfsetdash{}{0pt}%
\pgfpathmoveto{\pgfqpoint{7.805062in}{3.402898in}}%
\pgfpathcurveto{\pgfqpoint{7.816112in}{3.402898in}}{\pgfqpoint{7.826711in}{3.407288in}}{\pgfqpoint{7.834525in}{3.415101in}}%
\pgfpathcurveto{\pgfqpoint{7.842338in}{3.422915in}}{\pgfqpoint{7.846728in}{3.433514in}}{\pgfqpoint{7.846728in}{3.444564in}}%
\pgfpathcurveto{\pgfqpoint{7.846728in}{3.455614in}}{\pgfqpoint{7.842338in}{3.466213in}}{\pgfqpoint{7.834525in}{3.474027in}}%
\pgfpathcurveto{\pgfqpoint{7.826711in}{3.481841in}}{\pgfqpoint{7.816112in}{3.486231in}}{\pgfqpoint{7.805062in}{3.486231in}}%
\pgfpathcurveto{\pgfqpoint{7.794012in}{3.486231in}}{\pgfqpoint{7.783413in}{3.481841in}}{\pgfqpoint{7.775599in}{3.474027in}}%
\pgfpathcurveto{\pgfqpoint{7.767785in}{3.466213in}}{\pgfqpoint{7.763395in}{3.455614in}}{\pgfqpoint{7.763395in}{3.444564in}}%
\pgfpathcurveto{\pgfqpoint{7.763395in}{3.433514in}}{\pgfqpoint{7.767785in}{3.422915in}}{\pgfqpoint{7.775599in}{3.415101in}}%
\pgfpathcurveto{\pgfqpoint{7.783413in}{3.407288in}}{\pgfqpoint{7.794012in}{3.402898in}}{\pgfqpoint{7.805062in}{3.402898in}}%
\pgfpathclose%
\pgfusepath{stroke,fill}%
\end{pgfscope}%
\begin{pgfscope}%
\pgfpathrectangle{\pgfqpoint{0.481978in}{0.331635in}}{\pgfqpoint{9.300000in}{7.700000in}}%
\pgfusepath{clip}%
\pgfsetbuttcap%
\pgfsetroundjoin%
\definecolor{currentfill}{rgb}{0.552941,0.898039,0.631373}%
\pgfsetfillcolor{currentfill}%
\pgfsetlinewidth{0.481800pt}%
\definecolor{currentstroke}{rgb}{1.000000,1.000000,1.000000}%
\pgfsetstrokecolor{currentstroke}%
\pgfsetdash{}{0pt}%
\pgfpathmoveto{\pgfqpoint{4.750388in}{7.153024in}}%
\pgfpathcurveto{\pgfqpoint{4.761438in}{7.153024in}}{\pgfqpoint{4.772037in}{7.157414in}}{\pgfqpoint{4.779850in}{7.165228in}}%
\pgfpathcurveto{\pgfqpoint{4.787664in}{7.173042in}}{\pgfqpoint{4.792054in}{7.183641in}}{\pgfqpoint{4.792054in}{7.194691in}}%
\pgfpathcurveto{\pgfqpoint{4.792054in}{7.205741in}}{\pgfqpoint{4.787664in}{7.216340in}}{\pgfqpoint{4.779850in}{7.224154in}}%
\pgfpathcurveto{\pgfqpoint{4.772037in}{7.231967in}}{\pgfqpoint{4.761438in}{7.236357in}}{\pgfqpoint{4.750388in}{7.236357in}}%
\pgfpathcurveto{\pgfqpoint{4.739338in}{7.236357in}}{\pgfqpoint{4.728739in}{7.231967in}}{\pgfqpoint{4.720925in}{7.224154in}}%
\pgfpathcurveto{\pgfqpoint{4.713111in}{7.216340in}}{\pgfqpoint{4.708721in}{7.205741in}}{\pgfqpoint{4.708721in}{7.194691in}}%
\pgfpathcurveto{\pgfqpoint{4.708721in}{7.183641in}}{\pgfqpoint{4.713111in}{7.173042in}}{\pgfqpoint{4.720925in}{7.165228in}}%
\pgfpathcurveto{\pgfqpoint{4.728739in}{7.157414in}}{\pgfqpoint{4.739338in}{7.153024in}}{\pgfqpoint{4.750388in}{7.153024in}}%
\pgfpathclose%
\pgfusepath{stroke,fill}%
\end{pgfscope}%
\begin{pgfscope}%
\pgfpathrectangle{\pgfqpoint{0.481978in}{0.331635in}}{\pgfqpoint{9.300000in}{7.700000in}}%
\pgfusepath{clip}%
\pgfsetbuttcap%
\pgfsetroundjoin%
\definecolor{currentfill}{rgb}{0.552941,0.898039,0.631373}%
\pgfsetfillcolor{currentfill}%
\pgfsetlinewidth{0.481800pt}%
\definecolor{currentstroke}{rgb}{1.000000,1.000000,1.000000}%
\pgfsetstrokecolor{currentstroke}%
\pgfsetdash{}{0pt}%
\pgfpathmoveto{\pgfqpoint{4.386424in}{5.529843in}}%
\pgfpathcurveto{\pgfqpoint{4.397474in}{5.529843in}}{\pgfqpoint{4.408073in}{5.534233in}}{\pgfqpoint{4.415887in}{5.542047in}}%
\pgfpathcurveto{\pgfqpoint{4.423700in}{5.549861in}}{\pgfqpoint{4.428091in}{5.560460in}}{\pgfqpoint{4.428091in}{5.571510in}}%
\pgfpathcurveto{\pgfqpoint{4.428091in}{5.582560in}}{\pgfqpoint{4.423700in}{5.593159in}}{\pgfqpoint{4.415887in}{5.600972in}}%
\pgfpathcurveto{\pgfqpoint{4.408073in}{5.608786in}}{\pgfqpoint{4.397474in}{5.613176in}}{\pgfqpoint{4.386424in}{5.613176in}}%
\pgfpathcurveto{\pgfqpoint{4.375374in}{5.613176in}}{\pgfqpoint{4.364775in}{5.608786in}}{\pgfqpoint{4.356961in}{5.600972in}}%
\pgfpathcurveto{\pgfqpoint{4.349148in}{5.593159in}}{\pgfqpoint{4.344757in}{5.582560in}}{\pgfqpoint{4.344757in}{5.571510in}}%
\pgfpathcurveto{\pgfqpoint{4.344757in}{5.560460in}}{\pgfqpoint{4.349148in}{5.549861in}}{\pgfqpoint{4.356961in}{5.542047in}}%
\pgfpathcurveto{\pgfqpoint{4.364775in}{5.534233in}}{\pgfqpoint{4.375374in}{5.529843in}}{\pgfqpoint{4.386424in}{5.529843in}}%
\pgfpathclose%
\pgfusepath{stroke,fill}%
\end{pgfscope}%
\begin{pgfscope}%
\pgfpathrectangle{\pgfqpoint{0.481978in}{0.331635in}}{\pgfqpoint{9.300000in}{7.700000in}}%
\pgfusepath{clip}%
\pgfsetbuttcap%
\pgfsetroundjoin%
\definecolor{currentfill}{rgb}{0.552941,0.898039,0.631373}%
\pgfsetfillcolor{currentfill}%
\pgfsetlinewidth{0.481800pt}%
\definecolor{currentstroke}{rgb}{1.000000,1.000000,1.000000}%
\pgfsetstrokecolor{currentstroke}%
\pgfsetdash{}{0pt}%
\pgfpathmoveto{\pgfqpoint{4.878045in}{6.061659in}}%
\pgfpathcurveto{\pgfqpoint{4.889095in}{6.061659in}}{\pgfqpoint{4.899694in}{6.066049in}}{\pgfqpoint{4.907508in}{6.073863in}}%
\pgfpathcurveto{\pgfqpoint{4.915321in}{6.081677in}}{\pgfqpoint{4.919712in}{6.092276in}}{\pgfqpoint{4.919712in}{6.103326in}}%
\pgfpathcurveto{\pgfqpoint{4.919712in}{6.114376in}}{\pgfqpoint{4.915321in}{6.124975in}}{\pgfqpoint{4.907508in}{6.132789in}}%
\pgfpathcurveto{\pgfqpoint{4.899694in}{6.140602in}}{\pgfqpoint{4.889095in}{6.144993in}}{\pgfqpoint{4.878045in}{6.144993in}}%
\pgfpathcurveto{\pgfqpoint{4.866995in}{6.144993in}}{\pgfqpoint{4.856396in}{6.140602in}}{\pgfqpoint{4.848582in}{6.132789in}}%
\pgfpathcurveto{\pgfqpoint{4.840769in}{6.124975in}}{\pgfqpoint{4.836378in}{6.114376in}}{\pgfqpoint{4.836378in}{6.103326in}}%
\pgfpathcurveto{\pgfqpoint{4.836378in}{6.092276in}}{\pgfqpoint{4.840769in}{6.081677in}}{\pgfqpoint{4.848582in}{6.073863in}}%
\pgfpathcurveto{\pgfqpoint{4.856396in}{6.066049in}}{\pgfqpoint{4.866995in}{6.061659in}}{\pgfqpoint{4.878045in}{6.061659in}}%
\pgfpathclose%
\pgfusepath{stroke,fill}%
\end{pgfscope}%
\begin{pgfscope}%
\pgfpathrectangle{\pgfqpoint{0.481978in}{0.331635in}}{\pgfqpoint{9.300000in}{7.700000in}}%
\pgfusepath{clip}%
\pgfsetbuttcap%
\pgfsetroundjoin%
\definecolor{currentfill}{rgb}{0.552941,0.898039,0.631373}%
\pgfsetfillcolor{currentfill}%
\pgfsetlinewidth{0.481800pt}%
\definecolor{currentstroke}{rgb}{1.000000,1.000000,1.000000}%
\pgfsetstrokecolor{currentstroke}%
\pgfsetdash{}{0pt}%
\pgfpathmoveto{\pgfqpoint{9.259474in}{5.157624in}}%
\pgfpathcurveto{\pgfqpoint{9.270524in}{5.157624in}}{\pgfqpoint{9.281123in}{5.162014in}}{\pgfqpoint{9.288937in}{5.169827in}}%
\pgfpathcurveto{\pgfqpoint{9.296750in}{5.177641in}}{\pgfqpoint{9.301140in}{5.188240in}}{\pgfqpoint{9.301140in}{5.199290in}}%
\pgfpathcurveto{\pgfqpoint{9.301140in}{5.210340in}}{\pgfqpoint{9.296750in}{5.220939in}}{\pgfqpoint{9.288937in}{5.228753in}}%
\pgfpathcurveto{\pgfqpoint{9.281123in}{5.236567in}}{\pgfqpoint{9.270524in}{5.240957in}}{\pgfqpoint{9.259474in}{5.240957in}}%
\pgfpathcurveto{\pgfqpoint{9.248424in}{5.240957in}}{\pgfqpoint{9.237825in}{5.236567in}}{\pgfqpoint{9.230011in}{5.228753in}}%
\pgfpathcurveto{\pgfqpoint{9.222197in}{5.220939in}}{\pgfqpoint{9.217807in}{5.210340in}}{\pgfqpoint{9.217807in}{5.199290in}}%
\pgfpathcurveto{\pgfqpoint{9.217807in}{5.188240in}}{\pgfqpoint{9.222197in}{5.177641in}}{\pgfqpoint{9.230011in}{5.169827in}}%
\pgfpathcurveto{\pgfqpoint{9.237825in}{5.162014in}}{\pgfqpoint{9.248424in}{5.157624in}}{\pgfqpoint{9.259474in}{5.157624in}}%
\pgfpathclose%
\pgfusepath{stroke,fill}%
\end{pgfscope}%
\begin{pgfscope}%
\pgfpathrectangle{\pgfqpoint{0.481978in}{0.331635in}}{\pgfqpoint{9.300000in}{7.700000in}}%
\pgfusepath{clip}%
\pgfsetbuttcap%
\pgfsetroundjoin%
\definecolor{currentfill}{rgb}{0.552941,0.898039,0.631373}%
\pgfsetfillcolor{currentfill}%
\pgfsetlinewidth{0.481800pt}%
\definecolor{currentstroke}{rgb}{1.000000,1.000000,1.000000}%
\pgfsetstrokecolor{currentstroke}%
\pgfsetdash{}{0pt}%
\pgfpathmoveto{\pgfqpoint{8.569592in}{4.750652in}}%
\pgfpathcurveto{\pgfqpoint{8.580643in}{4.750652in}}{\pgfqpoint{8.591242in}{4.755042in}}{\pgfqpoint{8.599055in}{4.762856in}}%
\pgfpathcurveto{\pgfqpoint{8.606869in}{4.770670in}}{\pgfqpoint{8.611259in}{4.781269in}}{\pgfqpoint{8.611259in}{4.792319in}}%
\pgfpathcurveto{\pgfqpoint{8.611259in}{4.803369in}}{\pgfqpoint{8.606869in}{4.813968in}}{\pgfqpoint{8.599055in}{4.821782in}}%
\pgfpathcurveto{\pgfqpoint{8.591242in}{4.829595in}}{\pgfqpoint{8.580643in}{4.833985in}}{\pgfqpoint{8.569592in}{4.833985in}}%
\pgfpathcurveto{\pgfqpoint{8.558542in}{4.833985in}}{\pgfqpoint{8.547943in}{4.829595in}}{\pgfqpoint{8.540130in}{4.821782in}}%
\pgfpathcurveto{\pgfqpoint{8.532316in}{4.813968in}}{\pgfqpoint{8.527926in}{4.803369in}}{\pgfqpoint{8.527926in}{4.792319in}}%
\pgfpathcurveto{\pgfqpoint{8.527926in}{4.781269in}}{\pgfqpoint{8.532316in}{4.770670in}}{\pgfqpoint{8.540130in}{4.762856in}}%
\pgfpathcurveto{\pgfqpoint{8.547943in}{4.755042in}}{\pgfqpoint{8.558542in}{4.750652in}}{\pgfqpoint{8.569592in}{4.750652in}}%
\pgfpathclose%
\pgfusepath{stroke,fill}%
\end{pgfscope}%
\begin{pgfscope}%
\pgfpathrectangle{\pgfqpoint{0.481978in}{0.331635in}}{\pgfqpoint{9.300000in}{7.700000in}}%
\pgfusepath{clip}%
\pgfsetbuttcap%
\pgfsetroundjoin%
\definecolor{currentfill}{rgb}{0.552941,0.898039,0.631373}%
\pgfsetfillcolor{currentfill}%
\pgfsetlinewidth{0.481800pt}%
\definecolor{currentstroke}{rgb}{1.000000,1.000000,1.000000}%
\pgfsetstrokecolor{currentstroke}%
\pgfsetdash{}{0pt}%
\pgfpathmoveto{\pgfqpoint{8.786772in}{4.529927in}}%
\pgfpathcurveto{\pgfqpoint{8.797823in}{4.529927in}}{\pgfqpoint{8.808422in}{4.534317in}}{\pgfqpoint{8.816235in}{4.542131in}}%
\pgfpathcurveto{\pgfqpoint{8.824049in}{4.549944in}}{\pgfqpoint{8.828439in}{4.560543in}}{\pgfqpoint{8.828439in}{4.571593in}}%
\pgfpathcurveto{\pgfqpoint{8.828439in}{4.582644in}}{\pgfqpoint{8.824049in}{4.593243in}}{\pgfqpoint{8.816235in}{4.601056in}}%
\pgfpathcurveto{\pgfqpoint{8.808422in}{4.608870in}}{\pgfqpoint{8.797823in}{4.613260in}}{\pgfqpoint{8.786772in}{4.613260in}}%
\pgfpathcurveto{\pgfqpoint{8.775722in}{4.613260in}}{\pgfqpoint{8.765123in}{4.608870in}}{\pgfqpoint{8.757310in}{4.601056in}}%
\pgfpathcurveto{\pgfqpoint{8.749496in}{4.593243in}}{\pgfqpoint{8.745106in}{4.582644in}}{\pgfqpoint{8.745106in}{4.571593in}}%
\pgfpathcurveto{\pgfqpoint{8.745106in}{4.560543in}}{\pgfqpoint{8.749496in}{4.549944in}}{\pgfqpoint{8.757310in}{4.542131in}}%
\pgfpathcurveto{\pgfqpoint{8.765123in}{4.534317in}}{\pgfqpoint{8.775722in}{4.529927in}}{\pgfqpoint{8.786772in}{4.529927in}}%
\pgfpathclose%
\pgfusepath{stroke,fill}%
\end{pgfscope}%
\begin{pgfscope}%
\pgfpathrectangle{\pgfqpoint{0.481978in}{0.331635in}}{\pgfqpoint{9.300000in}{7.700000in}}%
\pgfusepath{clip}%
\pgfsetbuttcap%
\pgfsetroundjoin%
\definecolor{currentfill}{rgb}{0.552941,0.898039,0.631373}%
\pgfsetfillcolor{currentfill}%
\pgfsetlinewidth{0.481800pt}%
\definecolor{currentstroke}{rgb}{1.000000,1.000000,1.000000}%
\pgfsetstrokecolor{currentstroke}%
\pgfsetdash{}{0pt}%
\pgfpathmoveto{\pgfqpoint{7.733235in}{4.864578in}}%
\pgfpathcurveto{\pgfqpoint{7.744285in}{4.864578in}}{\pgfqpoint{7.754884in}{4.868968in}}{\pgfqpoint{7.762698in}{4.876782in}}%
\pgfpathcurveto{\pgfqpoint{7.770512in}{4.884595in}}{\pgfqpoint{7.774902in}{4.895194in}}{\pgfqpoint{7.774902in}{4.906244in}}%
\pgfpathcurveto{\pgfqpoint{7.774902in}{4.917294in}}{\pgfqpoint{7.770512in}{4.927893in}}{\pgfqpoint{7.762698in}{4.935707in}}%
\pgfpathcurveto{\pgfqpoint{7.754884in}{4.943521in}}{\pgfqpoint{7.744285in}{4.947911in}}{\pgfqpoint{7.733235in}{4.947911in}}%
\pgfpathcurveto{\pgfqpoint{7.722185in}{4.947911in}}{\pgfqpoint{7.711586in}{4.943521in}}{\pgfqpoint{7.703772in}{4.935707in}}%
\pgfpathcurveto{\pgfqpoint{7.695959in}{4.927893in}}{\pgfqpoint{7.691568in}{4.917294in}}{\pgfqpoint{7.691568in}{4.906244in}}%
\pgfpathcurveto{\pgfqpoint{7.691568in}{4.895194in}}{\pgfqpoint{7.695959in}{4.884595in}}{\pgfqpoint{7.703772in}{4.876782in}}%
\pgfpathcurveto{\pgfqpoint{7.711586in}{4.868968in}}{\pgfqpoint{7.722185in}{4.864578in}}{\pgfqpoint{7.733235in}{4.864578in}}%
\pgfpathclose%
\pgfusepath{stroke,fill}%
\end{pgfscope}%
\begin{pgfscope}%
\pgfpathrectangle{\pgfqpoint{0.481978in}{0.331635in}}{\pgfqpoint{9.300000in}{7.700000in}}%
\pgfusepath{clip}%
\pgfsetbuttcap%
\pgfsetroundjoin%
\definecolor{currentfill}{rgb}{0.552941,0.898039,0.631373}%
\pgfsetfillcolor{currentfill}%
\pgfsetlinewidth{0.481800pt}%
\definecolor{currentstroke}{rgb}{1.000000,1.000000,1.000000}%
\pgfsetstrokecolor{currentstroke}%
\pgfsetdash{}{0pt}%
\pgfpathmoveto{\pgfqpoint{5.023248in}{5.790519in}}%
\pgfpathcurveto{\pgfqpoint{5.034298in}{5.790519in}}{\pgfqpoint{5.044897in}{5.794910in}}{\pgfqpoint{5.052710in}{5.802723in}}%
\pgfpathcurveto{\pgfqpoint{5.060524in}{5.810537in}}{\pgfqpoint{5.064914in}{5.821136in}}{\pgfqpoint{5.064914in}{5.832186in}}%
\pgfpathcurveto{\pgfqpoint{5.064914in}{5.843236in}}{\pgfqpoint{5.060524in}{5.853835in}}{\pgfqpoint{5.052710in}{5.861649in}}%
\pgfpathcurveto{\pgfqpoint{5.044897in}{5.869462in}}{\pgfqpoint{5.034298in}{5.873853in}}{\pgfqpoint{5.023248in}{5.873853in}}%
\pgfpathcurveto{\pgfqpoint{5.012198in}{5.873853in}}{\pgfqpoint{5.001599in}{5.869462in}}{\pgfqpoint{4.993785in}{5.861649in}}%
\pgfpathcurveto{\pgfqpoint{4.985971in}{5.853835in}}{\pgfqpoint{4.981581in}{5.843236in}}{\pgfqpoint{4.981581in}{5.832186in}}%
\pgfpathcurveto{\pgfqpoint{4.981581in}{5.821136in}}{\pgfqpoint{4.985971in}{5.810537in}}{\pgfqpoint{4.993785in}{5.802723in}}%
\pgfpathcurveto{\pgfqpoint{5.001599in}{5.794910in}}{\pgfqpoint{5.012198in}{5.790519in}}{\pgfqpoint{5.023248in}{5.790519in}}%
\pgfpathclose%
\pgfusepath{stroke,fill}%
\end{pgfscope}%
\begin{pgfscope}%
\pgfpathrectangle{\pgfqpoint{0.481978in}{0.331635in}}{\pgfqpoint{9.300000in}{7.700000in}}%
\pgfusepath{clip}%
\pgfsetbuttcap%
\pgfsetroundjoin%
\definecolor{currentfill}{rgb}{0.552941,0.898039,0.631373}%
\pgfsetfillcolor{currentfill}%
\pgfsetlinewidth{0.481800pt}%
\definecolor{currentstroke}{rgb}{1.000000,1.000000,1.000000}%
\pgfsetstrokecolor{currentstroke}%
\pgfsetdash{}{0pt}%
\pgfpathmoveto{\pgfqpoint{4.955252in}{6.232287in}}%
\pgfpathcurveto{\pgfqpoint{4.966302in}{6.232287in}}{\pgfqpoint{4.976901in}{6.236677in}}{\pgfqpoint{4.984715in}{6.244490in}}%
\pgfpathcurveto{\pgfqpoint{4.992528in}{6.252304in}}{\pgfqpoint{4.996918in}{6.262903in}}{\pgfqpoint{4.996918in}{6.273953in}}%
\pgfpathcurveto{\pgfqpoint{4.996918in}{6.285003in}}{\pgfqpoint{4.992528in}{6.295602in}}{\pgfqpoint{4.984715in}{6.303416in}}%
\pgfpathcurveto{\pgfqpoint{4.976901in}{6.311230in}}{\pgfqpoint{4.966302in}{6.315620in}}{\pgfqpoint{4.955252in}{6.315620in}}%
\pgfpathcurveto{\pgfqpoint{4.944202in}{6.315620in}}{\pgfqpoint{4.933603in}{6.311230in}}{\pgfqpoint{4.925789in}{6.303416in}}%
\pgfpathcurveto{\pgfqpoint{4.917975in}{6.295602in}}{\pgfqpoint{4.913585in}{6.285003in}}{\pgfqpoint{4.913585in}{6.273953in}}%
\pgfpathcurveto{\pgfqpoint{4.913585in}{6.262903in}}{\pgfqpoint{4.917975in}{6.252304in}}{\pgfqpoint{4.925789in}{6.244490in}}%
\pgfpathcurveto{\pgfqpoint{4.933603in}{6.236677in}}{\pgfqpoint{4.944202in}{6.232287in}}{\pgfqpoint{4.955252in}{6.232287in}}%
\pgfpathclose%
\pgfusepath{stroke,fill}%
\end{pgfscope}%
\begin{pgfscope}%
\pgfpathrectangle{\pgfqpoint{0.481978in}{0.331635in}}{\pgfqpoint{9.300000in}{7.700000in}}%
\pgfusepath{clip}%
\pgfsetbuttcap%
\pgfsetroundjoin%
\definecolor{currentfill}{rgb}{0.552941,0.898039,0.631373}%
\pgfsetfillcolor{currentfill}%
\pgfsetlinewidth{0.481800pt}%
\definecolor{currentstroke}{rgb}{1.000000,1.000000,1.000000}%
\pgfsetstrokecolor{currentstroke}%
\pgfsetdash{}{0pt}%
\pgfpathmoveto{\pgfqpoint{7.433098in}{2.223941in}}%
\pgfpathcurveto{\pgfqpoint{7.444148in}{2.223941in}}{\pgfqpoint{7.454747in}{2.228332in}}{\pgfqpoint{7.462561in}{2.236145in}}%
\pgfpathcurveto{\pgfqpoint{7.470375in}{2.243959in}}{\pgfqpoint{7.474765in}{2.254558in}}{\pgfqpoint{7.474765in}{2.265608in}}%
\pgfpathcurveto{\pgfqpoint{7.474765in}{2.276658in}}{\pgfqpoint{7.470375in}{2.287257in}}{\pgfqpoint{7.462561in}{2.295071in}}%
\pgfpathcurveto{\pgfqpoint{7.454747in}{2.302884in}}{\pgfqpoint{7.444148in}{2.307275in}}{\pgfqpoint{7.433098in}{2.307275in}}%
\pgfpathcurveto{\pgfqpoint{7.422048in}{2.307275in}}{\pgfqpoint{7.411449in}{2.302884in}}{\pgfqpoint{7.403635in}{2.295071in}}%
\pgfpathcurveto{\pgfqpoint{7.395822in}{2.287257in}}{\pgfqpoint{7.391431in}{2.276658in}}{\pgfqpoint{7.391431in}{2.265608in}}%
\pgfpathcurveto{\pgfqpoint{7.391431in}{2.254558in}}{\pgfqpoint{7.395822in}{2.243959in}}{\pgfqpoint{7.403635in}{2.236145in}}%
\pgfpathcurveto{\pgfqpoint{7.411449in}{2.228332in}}{\pgfqpoint{7.422048in}{2.223941in}}{\pgfqpoint{7.433098in}{2.223941in}}%
\pgfpathclose%
\pgfusepath{stroke,fill}%
\end{pgfscope}%
\begin{pgfscope}%
\pgfpathrectangle{\pgfqpoint{0.481978in}{0.331635in}}{\pgfqpoint{9.300000in}{7.700000in}}%
\pgfusepath{clip}%
\pgfsetbuttcap%
\pgfsetroundjoin%
\definecolor{currentfill}{rgb}{0.552941,0.898039,0.631373}%
\pgfsetfillcolor{currentfill}%
\pgfsetlinewidth{0.481800pt}%
\definecolor{currentstroke}{rgb}{1.000000,1.000000,1.000000}%
\pgfsetstrokecolor{currentstroke}%
\pgfsetdash{}{0pt}%
\pgfpathmoveto{\pgfqpoint{8.232070in}{5.773420in}}%
\pgfpathcurveto{\pgfqpoint{8.243120in}{5.773420in}}{\pgfqpoint{8.253719in}{5.777810in}}{\pgfqpoint{8.261532in}{5.785624in}}%
\pgfpathcurveto{\pgfqpoint{8.269346in}{5.793438in}}{\pgfqpoint{8.273736in}{5.804037in}}{\pgfqpoint{8.273736in}{5.815087in}}%
\pgfpathcurveto{\pgfqpoint{8.273736in}{5.826137in}}{\pgfqpoint{8.269346in}{5.836736in}}{\pgfqpoint{8.261532in}{5.844550in}}%
\pgfpathcurveto{\pgfqpoint{8.253719in}{5.852363in}}{\pgfqpoint{8.243120in}{5.856753in}}{\pgfqpoint{8.232070in}{5.856753in}}%
\pgfpathcurveto{\pgfqpoint{8.221019in}{5.856753in}}{\pgfqpoint{8.210420in}{5.852363in}}{\pgfqpoint{8.202607in}{5.844550in}}%
\pgfpathcurveto{\pgfqpoint{8.194793in}{5.836736in}}{\pgfqpoint{8.190403in}{5.826137in}}{\pgfqpoint{8.190403in}{5.815087in}}%
\pgfpathcurveto{\pgfqpoint{8.190403in}{5.804037in}}{\pgfqpoint{8.194793in}{5.793438in}}{\pgfqpoint{8.202607in}{5.785624in}}%
\pgfpathcurveto{\pgfqpoint{8.210420in}{5.777810in}}{\pgfqpoint{8.221019in}{5.773420in}}{\pgfqpoint{8.232070in}{5.773420in}}%
\pgfpathclose%
\pgfusepath{stroke,fill}%
\end{pgfscope}%
\begin{pgfscope}%
\pgfpathrectangle{\pgfqpoint{0.481978in}{0.331635in}}{\pgfqpoint{9.300000in}{7.700000in}}%
\pgfusepath{clip}%
\pgfsetbuttcap%
\pgfsetroundjoin%
\definecolor{currentfill}{rgb}{0.552941,0.898039,0.631373}%
\pgfsetfillcolor{currentfill}%
\pgfsetlinewidth{0.481800pt}%
\definecolor{currentstroke}{rgb}{1.000000,1.000000,1.000000}%
\pgfsetstrokecolor{currentstroke}%
\pgfsetdash{}{0pt}%
\pgfpathmoveto{\pgfqpoint{3.011222in}{1.643530in}}%
\pgfpathcurveto{\pgfqpoint{3.022272in}{1.643530in}}{\pgfqpoint{3.032871in}{1.647920in}}{\pgfqpoint{3.040685in}{1.655734in}}%
\pgfpathcurveto{\pgfqpoint{3.048498in}{1.663547in}}{\pgfqpoint{3.052889in}{1.674146in}}{\pgfqpoint{3.052889in}{1.685197in}}%
\pgfpathcurveto{\pgfqpoint{3.052889in}{1.696247in}}{\pgfqpoint{3.048498in}{1.706846in}}{\pgfqpoint{3.040685in}{1.714659in}}%
\pgfpathcurveto{\pgfqpoint{3.032871in}{1.722473in}}{\pgfqpoint{3.022272in}{1.726863in}}{\pgfqpoint{3.011222in}{1.726863in}}%
\pgfpathcurveto{\pgfqpoint{3.000172in}{1.726863in}}{\pgfqpoint{2.989573in}{1.722473in}}{\pgfqpoint{2.981759in}{1.714659in}}%
\pgfpathcurveto{\pgfqpoint{2.973946in}{1.706846in}}{\pgfqpoint{2.969555in}{1.696247in}}{\pgfqpoint{2.969555in}{1.685197in}}%
\pgfpathcurveto{\pgfqpoint{2.969555in}{1.674146in}}{\pgfqpoint{2.973946in}{1.663547in}}{\pgfqpoint{2.981759in}{1.655734in}}%
\pgfpathcurveto{\pgfqpoint{2.989573in}{1.647920in}}{\pgfqpoint{3.000172in}{1.643530in}}{\pgfqpoint{3.011222in}{1.643530in}}%
\pgfpathclose%
\pgfusepath{stroke,fill}%
\end{pgfscope}%
\begin{pgfscope}%
\pgfpathrectangle{\pgfqpoint{0.481978in}{0.331635in}}{\pgfqpoint{9.300000in}{7.700000in}}%
\pgfusepath{clip}%
\pgfsetbuttcap%
\pgfsetroundjoin%
\definecolor{currentfill}{rgb}{0.552941,0.898039,0.631373}%
\pgfsetfillcolor{currentfill}%
\pgfsetlinewidth{0.481800pt}%
\definecolor{currentstroke}{rgb}{1.000000,1.000000,1.000000}%
\pgfsetstrokecolor{currentstroke}%
\pgfsetdash{}{0pt}%
\pgfpathmoveto{\pgfqpoint{4.782892in}{6.107152in}}%
\pgfpathcurveto{\pgfqpoint{4.793942in}{6.107152in}}{\pgfqpoint{4.804541in}{6.111542in}}{\pgfqpoint{4.812355in}{6.119355in}}%
\pgfpathcurveto{\pgfqpoint{4.820169in}{6.127169in}}{\pgfqpoint{4.824559in}{6.137768in}}{\pgfqpoint{4.824559in}{6.148818in}}%
\pgfpathcurveto{\pgfqpoint{4.824559in}{6.159868in}}{\pgfqpoint{4.820169in}{6.170467in}}{\pgfqpoint{4.812355in}{6.178281in}}%
\pgfpathcurveto{\pgfqpoint{4.804541in}{6.186095in}}{\pgfqpoint{4.793942in}{6.190485in}}{\pgfqpoint{4.782892in}{6.190485in}}%
\pgfpathcurveto{\pgfqpoint{4.771842in}{6.190485in}}{\pgfqpoint{4.761243in}{6.186095in}}{\pgfqpoint{4.753429in}{6.178281in}}%
\pgfpathcurveto{\pgfqpoint{4.745616in}{6.170467in}}{\pgfqpoint{4.741225in}{6.159868in}}{\pgfqpoint{4.741225in}{6.148818in}}%
\pgfpathcurveto{\pgfqpoint{4.741225in}{6.137768in}}{\pgfqpoint{4.745616in}{6.127169in}}{\pgfqpoint{4.753429in}{6.119355in}}%
\pgfpathcurveto{\pgfqpoint{4.761243in}{6.111542in}}{\pgfqpoint{4.771842in}{6.107152in}}{\pgfqpoint{4.782892in}{6.107152in}}%
\pgfpathclose%
\pgfusepath{stroke,fill}%
\end{pgfscope}%
\begin{pgfscope}%
\pgfpathrectangle{\pgfqpoint{0.481978in}{0.331635in}}{\pgfqpoint{9.300000in}{7.700000in}}%
\pgfusepath{clip}%
\pgfsetbuttcap%
\pgfsetroundjoin%
\definecolor{currentfill}{rgb}{0.552941,0.898039,0.631373}%
\pgfsetfillcolor{currentfill}%
\pgfsetlinewidth{0.481800pt}%
\definecolor{currentstroke}{rgb}{1.000000,1.000000,1.000000}%
\pgfsetstrokecolor{currentstroke}%
\pgfsetdash{}{0pt}%
\pgfpathmoveto{\pgfqpoint{3.628307in}{5.255411in}}%
\pgfpathcurveto{\pgfqpoint{3.639357in}{5.255411in}}{\pgfqpoint{3.649956in}{5.259801in}}{\pgfqpoint{3.657770in}{5.267615in}}%
\pgfpathcurveto{\pgfqpoint{3.665584in}{5.275428in}}{\pgfqpoint{3.669974in}{5.286028in}}{\pgfqpoint{3.669974in}{5.297078in}}%
\pgfpathcurveto{\pgfqpoint{3.669974in}{5.308128in}}{\pgfqpoint{3.665584in}{5.318727in}}{\pgfqpoint{3.657770in}{5.326540in}}%
\pgfpathcurveto{\pgfqpoint{3.649956in}{5.334354in}}{\pgfqpoint{3.639357in}{5.338744in}}{\pgfqpoint{3.628307in}{5.338744in}}%
\pgfpathcurveto{\pgfqpoint{3.617257in}{5.338744in}}{\pgfqpoint{3.606658in}{5.334354in}}{\pgfqpoint{3.598845in}{5.326540in}}%
\pgfpathcurveto{\pgfqpoint{3.591031in}{5.318727in}}{\pgfqpoint{3.586641in}{5.308128in}}{\pgfqpoint{3.586641in}{5.297078in}}%
\pgfpathcurveto{\pgfqpoint{3.586641in}{5.286028in}}{\pgfqpoint{3.591031in}{5.275428in}}{\pgfqpoint{3.598845in}{5.267615in}}%
\pgfpathcurveto{\pgfqpoint{3.606658in}{5.259801in}}{\pgfqpoint{3.617257in}{5.255411in}}{\pgfqpoint{3.628307in}{5.255411in}}%
\pgfpathclose%
\pgfusepath{stroke,fill}%
\end{pgfscope}%
\begin{pgfscope}%
\pgfpathrectangle{\pgfqpoint{0.481978in}{0.331635in}}{\pgfqpoint{9.300000in}{7.700000in}}%
\pgfusepath{clip}%
\pgfsetbuttcap%
\pgfsetroundjoin%
\definecolor{currentfill}{rgb}{0.552941,0.898039,0.631373}%
\pgfsetfillcolor{currentfill}%
\pgfsetlinewidth{0.481800pt}%
\definecolor{currentstroke}{rgb}{1.000000,1.000000,1.000000}%
\pgfsetstrokecolor{currentstroke}%
\pgfsetdash{}{0pt}%
\pgfpathmoveto{\pgfqpoint{3.687739in}{5.199691in}}%
\pgfpathcurveto{\pgfqpoint{3.698789in}{5.199691in}}{\pgfqpoint{3.709388in}{5.204081in}}{\pgfqpoint{3.717201in}{5.211895in}}%
\pgfpathcurveto{\pgfqpoint{3.725015in}{5.219708in}}{\pgfqpoint{3.729405in}{5.230307in}}{\pgfqpoint{3.729405in}{5.241357in}}%
\pgfpathcurveto{\pgfqpoint{3.729405in}{5.252408in}}{\pgfqpoint{3.725015in}{5.263007in}}{\pgfqpoint{3.717201in}{5.270820in}}%
\pgfpathcurveto{\pgfqpoint{3.709388in}{5.278634in}}{\pgfqpoint{3.698789in}{5.283024in}}{\pgfqpoint{3.687739in}{5.283024in}}%
\pgfpathcurveto{\pgfqpoint{3.676689in}{5.283024in}}{\pgfqpoint{3.666090in}{5.278634in}}{\pgfqpoint{3.658276in}{5.270820in}}%
\pgfpathcurveto{\pgfqpoint{3.650462in}{5.263007in}}{\pgfqpoint{3.646072in}{5.252408in}}{\pgfqpoint{3.646072in}{5.241357in}}%
\pgfpathcurveto{\pgfqpoint{3.646072in}{5.230307in}}{\pgfqpoint{3.650462in}{5.219708in}}{\pgfqpoint{3.658276in}{5.211895in}}%
\pgfpathcurveto{\pgfqpoint{3.666090in}{5.204081in}}{\pgfqpoint{3.676689in}{5.199691in}}{\pgfqpoint{3.687739in}{5.199691in}}%
\pgfpathclose%
\pgfusepath{stroke,fill}%
\end{pgfscope}%
\begin{pgfscope}%
\pgfpathrectangle{\pgfqpoint{0.481978in}{0.331635in}}{\pgfqpoint{9.300000in}{7.700000in}}%
\pgfusepath{clip}%
\pgfsetbuttcap%
\pgfsetroundjoin%
\definecolor{currentfill}{rgb}{0.552941,0.898039,0.631373}%
\pgfsetfillcolor{currentfill}%
\pgfsetlinewidth{0.481800pt}%
\definecolor{currentstroke}{rgb}{1.000000,1.000000,1.000000}%
\pgfsetstrokecolor{currentstroke}%
\pgfsetdash{}{0pt}%
\pgfpathmoveto{\pgfqpoint{7.445589in}{2.785107in}}%
\pgfpathcurveto{\pgfqpoint{7.456639in}{2.785107in}}{\pgfqpoint{7.467238in}{2.789497in}}{\pgfqpoint{7.475051in}{2.797311in}}%
\pgfpathcurveto{\pgfqpoint{7.482865in}{2.805124in}}{\pgfqpoint{7.487255in}{2.815723in}}{\pgfqpoint{7.487255in}{2.826774in}}%
\pgfpathcurveto{\pgfqpoint{7.487255in}{2.837824in}}{\pgfqpoint{7.482865in}{2.848423in}}{\pgfqpoint{7.475051in}{2.856236in}}%
\pgfpathcurveto{\pgfqpoint{7.467238in}{2.864050in}}{\pgfqpoint{7.456639in}{2.868440in}}{\pgfqpoint{7.445589in}{2.868440in}}%
\pgfpathcurveto{\pgfqpoint{7.434539in}{2.868440in}}{\pgfqpoint{7.423940in}{2.864050in}}{\pgfqpoint{7.416126in}{2.856236in}}%
\pgfpathcurveto{\pgfqpoint{7.408312in}{2.848423in}}{\pgfqpoint{7.403922in}{2.837824in}}{\pgfqpoint{7.403922in}{2.826774in}}%
\pgfpathcurveto{\pgfqpoint{7.403922in}{2.815723in}}{\pgfqpoint{7.408312in}{2.805124in}}{\pgfqpoint{7.416126in}{2.797311in}}%
\pgfpathcurveto{\pgfqpoint{7.423940in}{2.789497in}}{\pgfqpoint{7.434539in}{2.785107in}}{\pgfqpoint{7.445589in}{2.785107in}}%
\pgfpathclose%
\pgfusepath{stroke,fill}%
\end{pgfscope}%
\begin{pgfscope}%
\pgfpathrectangle{\pgfqpoint{0.481978in}{0.331635in}}{\pgfqpoint{9.300000in}{7.700000in}}%
\pgfusepath{clip}%
\pgfsetbuttcap%
\pgfsetroundjoin%
\definecolor{currentfill}{rgb}{0.552941,0.898039,0.631373}%
\pgfsetfillcolor{currentfill}%
\pgfsetlinewidth{0.481800pt}%
\definecolor{currentstroke}{rgb}{1.000000,1.000000,1.000000}%
\pgfsetstrokecolor{currentstroke}%
\pgfsetdash{}{0pt}%
\pgfpathmoveto{\pgfqpoint{8.934791in}{5.101472in}}%
\pgfpathcurveto{\pgfqpoint{8.945841in}{5.101472in}}{\pgfqpoint{8.956440in}{5.105862in}}{\pgfqpoint{8.964253in}{5.113676in}}%
\pgfpathcurveto{\pgfqpoint{8.972067in}{5.121489in}}{\pgfqpoint{8.976457in}{5.132088in}}{\pgfqpoint{8.976457in}{5.143139in}}%
\pgfpathcurveto{\pgfqpoint{8.976457in}{5.154189in}}{\pgfqpoint{8.972067in}{5.164788in}}{\pgfqpoint{8.964253in}{5.172601in}}%
\pgfpathcurveto{\pgfqpoint{8.956440in}{5.180415in}}{\pgfqpoint{8.945841in}{5.184805in}}{\pgfqpoint{8.934791in}{5.184805in}}%
\pgfpathcurveto{\pgfqpoint{8.923741in}{5.184805in}}{\pgfqpoint{8.913142in}{5.180415in}}{\pgfqpoint{8.905328in}{5.172601in}}%
\pgfpathcurveto{\pgfqpoint{8.897514in}{5.164788in}}{\pgfqpoint{8.893124in}{5.154189in}}{\pgfqpoint{8.893124in}{5.143139in}}%
\pgfpathcurveto{\pgfqpoint{8.893124in}{5.132088in}}{\pgfqpoint{8.897514in}{5.121489in}}{\pgfqpoint{8.905328in}{5.113676in}}%
\pgfpathcurveto{\pgfqpoint{8.913142in}{5.105862in}}{\pgfqpoint{8.923741in}{5.101472in}}{\pgfqpoint{8.934791in}{5.101472in}}%
\pgfpathclose%
\pgfusepath{stroke,fill}%
\end{pgfscope}%
\begin{pgfscope}%
\pgfpathrectangle{\pgfqpoint{0.481978in}{0.331635in}}{\pgfqpoint{9.300000in}{7.700000in}}%
\pgfusepath{clip}%
\pgfsetbuttcap%
\pgfsetroundjoin%
\definecolor{currentfill}{rgb}{0.552941,0.898039,0.631373}%
\pgfsetfillcolor{currentfill}%
\pgfsetlinewidth{0.481800pt}%
\definecolor{currentstroke}{rgb}{1.000000,1.000000,1.000000}%
\pgfsetstrokecolor{currentstroke}%
\pgfsetdash{}{0pt}%
\pgfpathmoveto{\pgfqpoint{3.508572in}{3.680026in}}%
\pgfpathcurveto{\pgfqpoint{3.519622in}{3.680026in}}{\pgfqpoint{3.530221in}{3.684417in}}{\pgfqpoint{3.538035in}{3.692230in}}%
\pgfpathcurveto{\pgfqpoint{3.545848in}{3.700044in}}{\pgfqpoint{3.550239in}{3.710643in}}{\pgfqpoint{3.550239in}{3.721693in}}%
\pgfpathcurveto{\pgfqpoint{3.550239in}{3.732743in}}{\pgfqpoint{3.545848in}{3.743342in}}{\pgfqpoint{3.538035in}{3.751156in}}%
\pgfpathcurveto{\pgfqpoint{3.530221in}{3.758969in}}{\pgfqpoint{3.519622in}{3.763360in}}{\pgfqpoint{3.508572in}{3.763360in}}%
\pgfpathcurveto{\pgfqpoint{3.497522in}{3.763360in}}{\pgfqpoint{3.486923in}{3.758969in}}{\pgfqpoint{3.479109in}{3.751156in}}%
\pgfpathcurveto{\pgfqpoint{3.471295in}{3.743342in}}{\pgfqpoint{3.466905in}{3.732743in}}{\pgfqpoint{3.466905in}{3.721693in}}%
\pgfpathcurveto{\pgfqpoint{3.466905in}{3.710643in}}{\pgfqpoint{3.471295in}{3.700044in}}{\pgfqpoint{3.479109in}{3.692230in}}%
\pgfpathcurveto{\pgfqpoint{3.486923in}{3.684417in}}{\pgfqpoint{3.497522in}{3.680026in}}{\pgfqpoint{3.508572in}{3.680026in}}%
\pgfpathclose%
\pgfusepath{stroke,fill}%
\end{pgfscope}%
\begin{pgfscope}%
\pgfpathrectangle{\pgfqpoint{0.481978in}{0.331635in}}{\pgfqpoint{9.300000in}{7.700000in}}%
\pgfusepath{clip}%
\pgfsetbuttcap%
\pgfsetroundjoin%
\definecolor{currentfill}{rgb}{0.552941,0.898039,0.631373}%
\pgfsetfillcolor{currentfill}%
\pgfsetlinewidth{0.481800pt}%
\definecolor{currentstroke}{rgb}{1.000000,1.000000,1.000000}%
\pgfsetstrokecolor{currentstroke}%
\pgfsetdash{}{0pt}%
\pgfpathmoveto{\pgfqpoint{4.192874in}{1.517858in}}%
\pgfpathcurveto{\pgfqpoint{4.203924in}{1.517858in}}{\pgfqpoint{4.214523in}{1.522248in}}{\pgfqpoint{4.222337in}{1.530062in}}%
\pgfpathcurveto{\pgfqpoint{4.230151in}{1.537875in}}{\pgfqpoint{4.234541in}{1.548474in}}{\pgfqpoint{4.234541in}{1.559524in}}%
\pgfpathcurveto{\pgfqpoint{4.234541in}{1.570575in}}{\pgfqpoint{4.230151in}{1.581174in}}{\pgfqpoint{4.222337in}{1.588987in}}%
\pgfpathcurveto{\pgfqpoint{4.214523in}{1.596801in}}{\pgfqpoint{4.203924in}{1.601191in}}{\pgfqpoint{4.192874in}{1.601191in}}%
\pgfpathcurveto{\pgfqpoint{4.181824in}{1.601191in}}{\pgfqpoint{4.171225in}{1.596801in}}{\pgfqpoint{4.163411in}{1.588987in}}%
\pgfpathcurveto{\pgfqpoint{4.155598in}{1.581174in}}{\pgfqpoint{4.151207in}{1.570575in}}{\pgfqpoint{4.151207in}{1.559524in}}%
\pgfpathcurveto{\pgfqpoint{4.151207in}{1.548474in}}{\pgfqpoint{4.155598in}{1.537875in}}{\pgfqpoint{4.163411in}{1.530062in}}%
\pgfpathcurveto{\pgfqpoint{4.171225in}{1.522248in}}{\pgfqpoint{4.181824in}{1.517858in}}{\pgfqpoint{4.192874in}{1.517858in}}%
\pgfpathclose%
\pgfusepath{stroke,fill}%
\end{pgfscope}%
\begin{pgfscope}%
\pgfpathrectangle{\pgfqpoint{0.481978in}{0.331635in}}{\pgfqpoint{9.300000in}{7.700000in}}%
\pgfusepath{clip}%
\pgfsetbuttcap%
\pgfsetroundjoin%
\definecolor{currentfill}{rgb}{0.552941,0.898039,0.631373}%
\pgfsetfillcolor{currentfill}%
\pgfsetlinewidth{0.481800pt}%
\definecolor{currentstroke}{rgb}{1.000000,1.000000,1.000000}%
\pgfsetstrokecolor{currentstroke}%
\pgfsetdash{}{0pt}%
\pgfpathmoveto{\pgfqpoint{4.001930in}{4.990969in}}%
\pgfpathcurveto{\pgfqpoint{4.012980in}{4.990969in}}{\pgfqpoint{4.023579in}{4.995360in}}{\pgfqpoint{4.031393in}{5.003173in}}%
\pgfpathcurveto{\pgfqpoint{4.039206in}{5.010987in}}{\pgfqpoint{4.043597in}{5.021586in}}{\pgfqpoint{4.043597in}{5.032636in}}%
\pgfpathcurveto{\pgfqpoint{4.043597in}{5.043686in}}{\pgfqpoint{4.039206in}{5.054285in}}{\pgfqpoint{4.031393in}{5.062099in}}%
\pgfpathcurveto{\pgfqpoint{4.023579in}{5.069912in}}{\pgfqpoint{4.012980in}{5.074303in}}{\pgfqpoint{4.001930in}{5.074303in}}%
\pgfpathcurveto{\pgfqpoint{3.990880in}{5.074303in}}{\pgfqpoint{3.980281in}{5.069912in}}{\pgfqpoint{3.972467in}{5.062099in}}%
\pgfpathcurveto{\pgfqpoint{3.964654in}{5.054285in}}{\pgfqpoint{3.960263in}{5.043686in}}{\pgfqpoint{3.960263in}{5.032636in}}%
\pgfpathcurveto{\pgfqpoint{3.960263in}{5.021586in}}{\pgfqpoint{3.964654in}{5.010987in}}{\pgfqpoint{3.972467in}{5.003173in}}%
\pgfpathcurveto{\pgfqpoint{3.980281in}{4.995360in}}{\pgfqpoint{3.990880in}{4.990969in}}{\pgfqpoint{4.001930in}{4.990969in}}%
\pgfpathclose%
\pgfusepath{stroke,fill}%
\end{pgfscope}%
\begin{pgfscope}%
\pgfpathrectangle{\pgfqpoint{0.481978in}{0.331635in}}{\pgfqpoint{9.300000in}{7.700000in}}%
\pgfusepath{clip}%
\pgfsetbuttcap%
\pgfsetroundjoin%
\definecolor{currentfill}{rgb}{0.552941,0.898039,0.631373}%
\pgfsetfillcolor{currentfill}%
\pgfsetlinewidth{0.481800pt}%
\definecolor{currentstroke}{rgb}{1.000000,1.000000,1.000000}%
\pgfsetstrokecolor{currentstroke}%
\pgfsetdash{}{0pt}%
\pgfpathmoveto{\pgfqpoint{6.060999in}{1.751246in}}%
\pgfpathcurveto{\pgfqpoint{6.072049in}{1.751246in}}{\pgfqpoint{6.082648in}{1.755636in}}{\pgfqpoint{6.090462in}{1.763449in}}%
\pgfpathcurveto{\pgfqpoint{6.098275in}{1.771263in}}{\pgfqpoint{6.102666in}{1.781862in}}{\pgfqpoint{6.102666in}{1.792912in}}%
\pgfpathcurveto{\pgfqpoint{6.102666in}{1.803962in}}{\pgfqpoint{6.098275in}{1.814561in}}{\pgfqpoint{6.090462in}{1.822375in}}%
\pgfpathcurveto{\pgfqpoint{6.082648in}{1.830189in}}{\pgfqpoint{6.072049in}{1.834579in}}{\pgfqpoint{6.060999in}{1.834579in}}%
\pgfpathcurveto{\pgfqpoint{6.049949in}{1.834579in}}{\pgfqpoint{6.039350in}{1.830189in}}{\pgfqpoint{6.031536in}{1.822375in}}%
\pgfpathcurveto{\pgfqpoint{6.023723in}{1.814561in}}{\pgfqpoint{6.019332in}{1.803962in}}{\pgfqpoint{6.019332in}{1.792912in}}%
\pgfpathcurveto{\pgfqpoint{6.019332in}{1.781862in}}{\pgfqpoint{6.023723in}{1.771263in}}{\pgfqpoint{6.031536in}{1.763449in}}%
\pgfpathcurveto{\pgfqpoint{6.039350in}{1.755636in}}{\pgfqpoint{6.049949in}{1.751246in}}{\pgfqpoint{6.060999in}{1.751246in}}%
\pgfpathclose%
\pgfusepath{stroke,fill}%
\end{pgfscope}%
\begin{pgfscope}%
\pgfpathrectangle{\pgfqpoint{0.481978in}{0.331635in}}{\pgfqpoint{9.300000in}{7.700000in}}%
\pgfusepath{clip}%
\pgfsetbuttcap%
\pgfsetroundjoin%
\definecolor{currentfill}{rgb}{0.552941,0.898039,0.631373}%
\pgfsetfillcolor{currentfill}%
\pgfsetlinewidth{0.481800pt}%
\definecolor{currentstroke}{rgb}{1.000000,1.000000,1.000000}%
\pgfsetstrokecolor{currentstroke}%
\pgfsetdash{}{0pt}%
\pgfpathmoveto{\pgfqpoint{2.449894in}{1.656799in}}%
\pgfpathcurveto{\pgfqpoint{2.460944in}{1.656799in}}{\pgfqpoint{2.471543in}{1.661190in}}{\pgfqpoint{2.479357in}{1.669003in}}%
\pgfpathcurveto{\pgfqpoint{2.487170in}{1.676817in}}{\pgfqpoint{2.491561in}{1.687416in}}{\pgfqpoint{2.491561in}{1.698466in}}%
\pgfpathcurveto{\pgfqpoint{2.491561in}{1.709516in}}{\pgfqpoint{2.487170in}{1.720115in}}{\pgfqpoint{2.479357in}{1.727929in}}%
\pgfpathcurveto{\pgfqpoint{2.471543in}{1.735742in}}{\pgfqpoint{2.460944in}{1.740133in}}{\pgfqpoint{2.449894in}{1.740133in}}%
\pgfpathcurveto{\pgfqpoint{2.438844in}{1.740133in}}{\pgfqpoint{2.428245in}{1.735742in}}{\pgfqpoint{2.420431in}{1.727929in}}%
\pgfpathcurveto{\pgfqpoint{2.412617in}{1.720115in}}{\pgfqpoint{2.408227in}{1.709516in}}{\pgfqpoint{2.408227in}{1.698466in}}%
\pgfpathcurveto{\pgfqpoint{2.408227in}{1.687416in}}{\pgfqpoint{2.412617in}{1.676817in}}{\pgfqpoint{2.420431in}{1.669003in}}%
\pgfpathcurveto{\pgfqpoint{2.428245in}{1.661190in}}{\pgfqpoint{2.438844in}{1.656799in}}{\pgfqpoint{2.449894in}{1.656799in}}%
\pgfpathclose%
\pgfusepath{stroke,fill}%
\end{pgfscope}%
\begin{pgfscope}%
\pgfpathrectangle{\pgfqpoint{0.481978in}{0.331635in}}{\pgfqpoint{9.300000in}{7.700000in}}%
\pgfusepath{clip}%
\pgfsetbuttcap%
\pgfsetroundjoin%
\definecolor{currentfill}{rgb}{0.552941,0.898039,0.631373}%
\pgfsetfillcolor{currentfill}%
\pgfsetlinewidth{0.481800pt}%
\definecolor{currentstroke}{rgb}{1.000000,1.000000,1.000000}%
\pgfsetstrokecolor{currentstroke}%
\pgfsetdash{}{0pt}%
\pgfpathmoveto{\pgfqpoint{7.028940in}{4.709424in}}%
\pgfpathcurveto{\pgfqpoint{7.039990in}{4.709424in}}{\pgfqpoint{7.050589in}{4.713814in}}{\pgfqpoint{7.058402in}{4.721628in}}%
\pgfpathcurveto{\pgfqpoint{7.066216in}{4.729441in}}{\pgfqpoint{7.070606in}{4.740040in}}{\pgfqpoint{7.070606in}{4.751090in}}%
\pgfpathcurveto{\pgfqpoint{7.070606in}{4.762141in}}{\pgfqpoint{7.066216in}{4.772740in}}{\pgfqpoint{7.058402in}{4.780553in}}%
\pgfpathcurveto{\pgfqpoint{7.050589in}{4.788367in}}{\pgfqpoint{7.039990in}{4.792757in}}{\pgfqpoint{7.028940in}{4.792757in}}%
\pgfpathcurveto{\pgfqpoint{7.017890in}{4.792757in}}{\pgfqpoint{7.007290in}{4.788367in}}{\pgfqpoint{6.999477in}{4.780553in}}%
\pgfpathcurveto{\pgfqpoint{6.991663in}{4.772740in}}{\pgfqpoint{6.987273in}{4.762141in}}{\pgfqpoint{6.987273in}{4.751090in}}%
\pgfpathcurveto{\pgfqpoint{6.987273in}{4.740040in}}{\pgfqpoint{6.991663in}{4.729441in}}{\pgfqpoint{6.999477in}{4.721628in}}%
\pgfpathcurveto{\pgfqpoint{7.007290in}{4.713814in}}{\pgfqpoint{7.017890in}{4.709424in}}{\pgfqpoint{7.028940in}{4.709424in}}%
\pgfpathclose%
\pgfusepath{stroke,fill}%
\end{pgfscope}%
\begin{pgfscope}%
\pgfpathrectangle{\pgfqpoint{0.481978in}{0.331635in}}{\pgfqpoint{9.300000in}{7.700000in}}%
\pgfusepath{clip}%
\pgfsetbuttcap%
\pgfsetroundjoin%
\definecolor{currentfill}{rgb}{0.552941,0.898039,0.631373}%
\pgfsetfillcolor{currentfill}%
\pgfsetlinewidth{0.481800pt}%
\definecolor{currentstroke}{rgb}{1.000000,1.000000,1.000000}%
\pgfsetstrokecolor{currentstroke}%
\pgfsetdash{}{0pt}%
\pgfpathmoveto{\pgfqpoint{5.259017in}{7.554084in}}%
\pgfpathcurveto{\pgfqpoint{5.270067in}{7.554084in}}{\pgfqpoint{5.280666in}{7.558475in}}{\pgfqpoint{5.288480in}{7.566288in}}%
\pgfpathcurveto{\pgfqpoint{5.296294in}{7.574102in}}{\pgfqpoint{5.300684in}{7.584701in}}{\pgfqpoint{5.300684in}{7.595751in}}%
\pgfpathcurveto{\pgfqpoint{5.300684in}{7.606801in}}{\pgfqpoint{5.296294in}{7.617400in}}{\pgfqpoint{5.288480in}{7.625214in}}%
\pgfpathcurveto{\pgfqpoint{5.280666in}{7.633027in}}{\pgfqpoint{5.270067in}{7.637418in}}{\pgfqpoint{5.259017in}{7.637418in}}%
\pgfpathcurveto{\pgfqpoint{5.247967in}{7.637418in}}{\pgfqpoint{5.237368in}{7.633027in}}{\pgfqpoint{5.229554in}{7.625214in}}%
\pgfpathcurveto{\pgfqpoint{5.221741in}{7.617400in}}{\pgfqpoint{5.217350in}{7.606801in}}{\pgfqpoint{5.217350in}{7.595751in}}%
\pgfpathcurveto{\pgfqpoint{5.217350in}{7.584701in}}{\pgfqpoint{5.221741in}{7.574102in}}{\pgfqpoint{5.229554in}{7.566288in}}%
\pgfpathcurveto{\pgfqpoint{5.237368in}{7.558475in}}{\pgfqpoint{5.247967in}{7.554084in}}{\pgfqpoint{5.259017in}{7.554084in}}%
\pgfpathclose%
\pgfusepath{stroke,fill}%
\end{pgfscope}%
\begin{pgfscope}%
\pgfpathrectangle{\pgfqpoint{0.481978in}{0.331635in}}{\pgfqpoint{9.300000in}{7.700000in}}%
\pgfusepath{clip}%
\pgfsetbuttcap%
\pgfsetroundjoin%
\definecolor{currentfill}{rgb}{0.552941,0.898039,0.631373}%
\pgfsetfillcolor{currentfill}%
\pgfsetlinewidth{0.481800pt}%
\definecolor{currentstroke}{rgb}{1.000000,1.000000,1.000000}%
\pgfsetstrokecolor{currentstroke}%
\pgfsetdash{}{0pt}%
\pgfpathmoveto{\pgfqpoint{9.184137in}{5.064662in}}%
\pgfpathcurveto{\pgfqpoint{9.195187in}{5.064662in}}{\pgfqpoint{9.205786in}{5.069052in}}{\pgfqpoint{9.213600in}{5.076866in}}%
\pgfpathcurveto{\pgfqpoint{9.221414in}{5.084680in}}{\pgfqpoint{9.225804in}{5.095279in}}{\pgfqpoint{9.225804in}{5.106329in}}%
\pgfpathcurveto{\pgfqpoint{9.225804in}{5.117379in}}{\pgfqpoint{9.221414in}{5.127978in}}{\pgfqpoint{9.213600in}{5.135792in}}%
\pgfpathcurveto{\pgfqpoint{9.205786in}{5.143605in}}{\pgfqpoint{9.195187in}{5.147995in}}{\pgfqpoint{9.184137in}{5.147995in}}%
\pgfpathcurveto{\pgfqpoint{9.173087in}{5.147995in}}{\pgfqpoint{9.162488in}{5.143605in}}{\pgfqpoint{9.154675in}{5.135792in}}%
\pgfpathcurveto{\pgfqpoint{9.146861in}{5.127978in}}{\pgfqpoint{9.142471in}{5.117379in}}{\pgfqpoint{9.142471in}{5.106329in}}%
\pgfpathcurveto{\pgfqpoint{9.142471in}{5.095279in}}{\pgfqpoint{9.146861in}{5.084680in}}{\pgfqpoint{9.154675in}{5.076866in}}%
\pgfpathcurveto{\pgfqpoint{9.162488in}{5.069052in}}{\pgfqpoint{9.173087in}{5.064662in}}{\pgfqpoint{9.184137in}{5.064662in}}%
\pgfpathclose%
\pgfusepath{stroke,fill}%
\end{pgfscope}%
\begin{pgfscope}%
\pgfpathrectangle{\pgfqpoint{0.481978in}{0.331635in}}{\pgfqpoint{9.300000in}{7.700000in}}%
\pgfusepath{clip}%
\pgfsetbuttcap%
\pgfsetroundjoin%
\definecolor{currentfill}{rgb}{0.552941,0.898039,0.631373}%
\pgfsetfillcolor{currentfill}%
\pgfsetlinewidth{0.481800pt}%
\definecolor{currentstroke}{rgb}{1.000000,1.000000,1.000000}%
\pgfsetstrokecolor{currentstroke}%
\pgfsetdash{}{0pt}%
\pgfpathmoveto{\pgfqpoint{8.944873in}{6.252081in}}%
\pgfpathcurveto{\pgfqpoint{8.955923in}{6.252081in}}{\pgfqpoint{8.966522in}{6.256471in}}{\pgfqpoint{8.974336in}{6.264285in}}%
\pgfpathcurveto{\pgfqpoint{8.982149in}{6.272098in}}{\pgfqpoint{8.986540in}{6.282697in}}{\pgfqpoint{8.986540in}{6.293748in}}%
\pgfpathcurveto{\pgfqpoint{8.986540in}{6.304798in}}{\pgfqpoint{8.982149in}{6.315397in}}{\pgfqpoint{8.974336in}{6.323210in}}%
\pgfpathcurveto{\pgfqpoint{8.966522in}{6.331024in}}{\pgfqpoint{8.955923in}{6.335414in}}{\pgfqpoint{8.944873in}{6.335414in}}%
\pgfpathcurveto{\pgfqpoint{8.933823in}{6.335414in}}{\pgfqpoint{8.923224in}{6.331024in}}{\pgfqpoint{8.915410in}{6.323210in}}%
\pgfpathcurveto{\pgfqpoint{8.907597in}{6.315397in}}{\pgfqpoint{8.903206in}{6.304798in}}{\pgfqpoint{8.903206in}{6.293748in}}%
\pgfpathcurveto{\pgfqpoint{8.903206in}{6.282697in}}{\pgfqpoint{8.907597in}{6.272098in}}{\pgfqpoint{8.915410in}{6.264285in}}%
\pgfpathcurveto{\pgfqpoint{8.923224in}{6.256471in}}{\pgfqpoint{8.933823in}{6.252081in}}{\pgfqpoint{8.944873in}{6.252081in}}%
\pgfpathclose%
\pgfusepath{stroke,fill}%
\end{pgfscope}%
\begin{pgfscope}%
\pgfpathrectangle{\pgfqpoint{0.481978in}{0.331635in}}{\pgfqpoint{9.300000in}{7.700000in}}%
\pgfusepath{clip}%
\pgfsetbuttcap%
\pgfsetroundjoin%
\definecolor{currentfill}{rgb}{0.552941,0.898039,0.631373}%
\pgfsetfillcolor{currentfill}%
\pgfsetlinewidth{0.481800pt}%
\definecolor{currentstroke}{rgb}{1.000000,1.000000,1.000000}%
\pgfsetstrokecolor{currentstroke}%
\pgfsetdash{}{0pt}%
\pgfpathmoveto{\pgfqpoint{7.733013in}{3.449529in}}%
\pgfpathcurveto{\pgfqpoint{7.744063in}{3.449529in}}{\pgfqpoint{7.754662in}{3.453919in}}{\pgfqpoint{7.762476in}{3.461733in}}%
\pgfpathcurveto{\pgfqpoint{7.770289in}{3.469547in}}{\pgfqpoint{7.774679in}{3.480146in}}{\pgfqpoint{7.774679in}{3.491196in}}%
\pgfpathcurveto{\pgfqpoint{7.774679in}{3.502246in}}{\pgfqpoint{7.770289in}{3.512845in}}{\pgfqpoint{7.762476in}{3.520659in}}%
\pgfpathcurveto{\pgfqpoint{7.754662in}{3.528472in}}{\pgfqpoint{7.744063in}{3.532862in}}{\pgfqpoint{7.733013in}{3.532862in}}%
\pgfpathcurveto{\pgfqpoint{7.721963in}{3.532862in}}{\pgfqpoint{7.711364in}{3.528472in}}{\pgfqpoint{7.703550in}{3.520659in}}%
\pgfpathcurveto{\pgfqpoint{7.695736in}{3.512845in}}{\pgfqpoint{7.691346in}{3.502246in}}{\pgfqpoint{7.691346in}{3.491196in}}%
\pgfpathcurveto{\pgfqpoint{7.691346in}{3.480146in}}{\pgfqpoint{7.695736in}{3.469547in}}{\pgfqpoint{7.703550in}{3.461733in}}%
\pgfpathcurveto{\pgfqpoint{7.711364in}{3.453919in}}{\pgfqpoint{7.721963in}{3.449529in}}{\pgfqpoint{7.733013in}{3.449529in}}%
\pgfpathclose%
\pgfusepath{stroke,fill}%
\end{pgfscope}%
\begin{pgfscope}%
\pgfpathrectangle{\pgfqpoint{0.481978in}{0.331635in}}{\pgfqpoint{9.300000in}{7.700000in}}%
\pgfusepath{clip}%
\pgfsetbuttcap%
\pgfsetroundjoin%
\definecolor{currentfill}{rgb}{0.552941,0.898039,0.631373}%
\pgfsetfillcolor{currentfill}%
\pgfsetlinewidth{0.481800pt}%
\definecolor{currentstroke}{rgb}{1.000000,1.000000,1.000000}%
\pgfsetstrokecolor{currentstroke}%
\pgfsetdash{}{0pt}%
\pgfpathmoveto{\pgfqpoint{5.907061in}{2.059171in}}%
\pgfpathcurveto{\pgfqpoint{5.918111in}{2.059171in}}{\pgfqpoint{5.928710in}{2.063562in}}{\pgfqpoint{5.936524in}{2.071375in}}%
\pgfpathcurveto{\pgfqpoint{5.944338in}{2.079189in}}{\pgfqpoint{5.948728in}{2.089788in}}{\pgfqpoint{5.948728in}{2.100838in}}%
\pgfpathcurveto{\pgfqpoint{5.948728in}{2.111888in}}{\pgfqpoint{5.944338in}{2.122487in}}{\pgfqpoint{5.936524in}{2.130301in}}%
\pgfpathcurveto{\pgfqpoint{5.928710in}{2.138114in}}{\pgfqpoint{5.918111in}{2.142505in}}{\pgfqpoint{5.907061in}{2.142505in}}%
\pgfpathcurveto{\pgfqpoint{5.896011in}{2.142505in}}{\pgfqpoint{5.885412in}{2.138114in}}{\pgfqpoint{5.877598in}{2.130301in}}%
\pgfpathcurveto{\pgfqpoint{5.869785in}{2.122487in}}{\pgfqpoint{5.865395in}{2.111888in}}{\pgfqpoint{5.865395in}{2.100838in}}%
\pgfpathcurveto{\pgfqpoint{5.865395in}{2.089788in}}{\pgfqpoint{5.869785in}{2.079189in}}{\pgfqpoint{5.877598in}{2.071375in}}%
\pgfpathcurveto{\pgfqpoint{5.885412in}{2.063562in}}{\pgfqpoint{5.896011in}{2.059171in}}{\pgfqpoint{5.907061in}{2.059171in}}%
\pgfpathclose%
\pgfusepath{stroke,fill}%
\end{pgfscope}%
\begin{pgfscope}%
\pgfpathrectangle{\pgfqpoint{0.481978in}{0.331635in}}{\pgfqpoint{9.300000in}{7.700000in}}%
\pgfusepath{clip}%
\pgfsetbuttcap%
\pgfsetroundjoin%
\definecolor{currentfill}{rgb}{0.552941,0.898039,0.631373}%
\pgfsetfillcolor{currentfill}%
\pgfsetlinewidth{0.481800pt}%
\definecolor{currentstroke}{rgb}{1.000000,1.000000,1.000000}%
\pgfsetstrokecolor{currentstroke}%
\pgfsetdash{}{0pt}%
\pgfpathmoveto{\pgfqpoint{4.839240in}{5.869672in}}%
\pgfpathcurveto{\pgfqpoint{4.850291in}{5.869672in}}{\pgfqpoint{4.860890in}{5.874062in}}{\pgfqpoint{4.868703in}{5.881876in}}%
\pgfpathcurveto{\pgfqpoint{4.876517in}{5.889689in}}{\pgfqpoint{4.880907in}{5.900288in}}{\pgfqpoint{4.880907in}{5.911338in}}%
\pgfpathcurveto{\pgfqpoint{4.880907in}{5.922389in}}{\pgfqpoint{4.876517in}{5.932988in}}{\pgfqpoint{4.868703in}{5.940801in}}%
\pgfpathcurveto{\pgfqpoint{4.860890in}{5.948615in}}{\pgfqpoint{4.850291in}{5.953005in}}{\pgfqpoint{4.839240in}{5.953005in}}%
\pgfpathcurveto{\pgfqpoint{4.828190in}{5.953005in}}{\pgfqpoint{4.817591in}{5.948615in}}{\pgfqpoint{4.809778in}{5.940801in}}%
\pgfpathcurveto{\pgfqpoint{4.801964in}{5.932988in}}{\pgfqpoint{4.797574in}{5.922389in}}{\pgfqpoint{4.797574in}{5.911338in}}%
\pgfpathcurveto{\pgfqpoint{4.797574in}{5.900288in}}{\pgfqpoint{4.801964in}{5.889689in}}{\pgfqpoint{4.809778in}{5.881876in}}%
\pgfpathcurveto{\pgfqpoint{4.817591in}{5.874062in}}{\pgfqpoint{4.828190in}{5.869672in}}{\pgfqpoint{4.839240in}{5.869672in}}%
\pgfpathclose%
\pgfusepath{stroke,fill}%
\end{pgfscope}%
\begin{pgfscope}%
\pgfpathrectangle{\pgfqpoint{0.481978in}{0.331635in}}{\pgfqpoint{9.300000in}{7.700000in}}%
\pgfusepath{clip}%
\pgfsetbuttcap%
\pgfsetroundjoin%
\definecolor{currentfill}{rgb}{0.552941,0.898039,0.631373}%
\pgfsetfillcolor{currentfill}%
\pgfsetlinewidth{0.481800pt}%
\definecolor{currentstroke}{rgb}{1.000000,1.000000,1.000000}%
\pgfsetstrokecolor{currentstroke}%
\pgfsetdash{}{0pt}%
\pgfpathmoveto{\pgfqpoint{2.381245in}{2.262941in}}%
\pgfpathcurveto{\pgfqpoint{2.392295in}{2.262941in}}{\pgfqpoint{2.402894in}{2.267331in}}{\pgfqpoint{2.410708in}{2.275144in}}%
\pgfpathcurveto{\pgfqpoint{2.418522in}{2.282958in}}{\pgfqpoint{2.422912in}{2.293557in}}{\pgfqpoint{2.422912in}{2.304607in}}%
\pgfpathcurveto{\pgfqpoint{2.422912in}{2.315657in}}{\pgfqpoint{2.418522in}{2.326256in}}{\pgfqpoint{2.410708in}{2.334070in}}%
\pgfpathcurveto{\pgfqpoint{2.402894in}{2.341884in}}{\pgfqpoint{2.392295in}{2.346274in}}{\pgfqpoint{2.381245in}{2.346274in}}%
\pgfpathcurveto{\pgfqpoint{2.370195in}{2.346274in}}{\pgfqpoint{2.359596in}{2.341884in}}{\pgfqpoint{2.351782in}{2.334070in}}%
\pgfpathcurveto{\pgfqpoint{2.343969in}{2.326256in}}{\pgfqpoint{2.339579in}{2.315657in}}{\pgfqpoint{2.339579in}{2.304607in}}%
\pgfpathcurveto{\pgfqpoint{2.339579in}{2.293557in}}{\pgfqpoint{2.343969in}{2.282958in}}{\pgfqpoint{2.351782in}{2.275144in}}%
\pgfpathcurveto{\pgfqpoint{2.359596in}{2.267331in}}{\pgfqpoint{2.370195in}{2.262941in}}{\pgfqpoint{2.381245in}{2.262941in}}%
\pgfpathclose%
\pgfusepath{stroke,fill}%
\end{pgfscope}%
\begin{pgfscope}%
\pgfpathrectangle{\pgfqpoint{0.481978in}{0.331635in}}{\pgfqpoint{9.300000in}{7.700000in}}%
\pgfusepath{clip}%
\pgfsetbuttcap%
\pgfsetroundjoin%
\definecolor{currentfill}{rgb}{0.552941,0.898039,0.631373}%
\pgfsetfillcolor{currentfill}%
\pgfsetlinewidth{0.481800pt}%
\definecolor{currentstroke}{rgb}{1.000000,1.000000,1.000000}%
\pgfsetstrokecolor{currentstroke}%
\pgfsetdash{}{0pt}%
\pgfpathmoveto{\pgfqpoint{7.266400in}{2.360768in}}%
\pgfpathcurveto{\pgfqpoint{7.277450in}{2.360768in}}{\pgfqpoint{7.288049in}{2.365158in}}{\pgfqpoint{7.295863in}{2.372972in}}%
\pgfpathcurveto{\pgfqpoint{7.303677in}{2.380785in}}{\pgfqpoint{7.308067in}{2.391384in}}{\pgfqpoint{7.308067in}{2.402434in}}%
\pgfpathcurveto{\pgfqpoint{7.308067in}{2.413484in}}{\pgfqpoint{7.303677in}{2.424083in}}{\pgfqpoint{7.295863in}{2.431897in}}%
\pgfpathcurveto{\pgfqpoint{7.288049in}{2.439711in}}{\pgfqpoint{7.277450in}{2.444101in}}{\pgfqpoint{7.266400in}{2.444101in}}%
\pgfpathcurveto{\pgfqpoint{7.255350in}{2.444101in}}{\pgfqpoint{7.244751in}{2.439711in}}{\pgfqpoint{7.236937in}{2.431897in}}%
\pgfpathcurveto{\pgfqpoint{7.229124in}{2.424083in}}{\pgfqpoint{7.224734in}{2.413484in}}{\pgfqpoint{7.224734in}{2.402434in}}%
\pgfpathcurveto{\pgfqpoint{7.224734in}{2.391384in}}{\pgfqpoint{7.229124in}{2.380785in}}{\pgfqpoint{7.236937in}{2.372972in}}%
\pgfpathcurveto{\pgfqpoint{7.244751in}{2.365158in}}{\pgfqpoint{7.255350in}{2.360768in}}{\pgfqpoint{7.266400in}{2.360768in}}%
\pgfpathclose%
\pgfusepath{stroke,fill}%
\end{pgfscope}%
\begin{pgfscope}%
\pgfpathrectangle{\pgfqpoint{0.481978in}{0.331635in}}{\pgfqpoint{9.300000in}{7.700000in}}%
\pgfusepath{clip}%
\pgfsetbuttcap%
\pgfsetroundjoin%
\definecolor{currentfill}{rgb}{0.552941,0.898039,0.631373}%
\pgfsetfillcolor{currentfill}%
\pgfsetlinewidth{0.481800pt}%
\definecolor{currentstroke}{rgb}{1.000000,1.000000,1.000000}%
\pgfsetstrokecolor{currentstroke}%
\pgfsetdash{}{0pt}%
\pgfpathmoveto{\pgfqpoint{3.410609in}{3.339609in}}%
\pgfpathcurveto{\pgfqpoint{3.421659in}{3.339609in}}{\pgfqpoint{3.432258in}{3.344000in}}{\pgfqpoint{3.440072in}{3.351813in}}%
\pgfpathcurveto{\pgfqpoint{3.447885in}{3.359627in}}{\pgfqpoint{3.452276in}{3.370226in}}{\pgfqpoint{3.452276in}{3.381276in}}%
\pgfpathcurveto{\pgfqpoint{3.452276in}{3.392326in}}{\pgfqpoint{3.447885in}{3.402925in}}{\pgfqpoint{3.440072in}{3.410739in}}%
\pgfpathcurveto{\pgfqpoint{3.432258in}{3.418553in}}{\pgfqpoint{3.421659in}{3.422943in}}{\pgfqpoint{3.410609in}{3.422943in}}%
\pgfpathcurveto{\pgfqpoint{3.399559in}{3.422943in}}{\pgfqpoint{3.388960in}{3.418553in}}{\pgfqpoint{3.381146in}{3.410739in}}%
\pgfpathcurveto{\pgfqpoint{3.373333in}{3.402925in}}{\pgfqpoint{3.368942in}{3.392326in}}{\pgfqpoint{3.368942in}{3.381276in}}%
\pgfpathcurveto{\pgfqpoint{3.368942in}{3.370226in}}{\pgfqpoint{3.373333in}{3.359627in}}{\pgfqpoint{3.381146in}{3.351813in}}%
\pgfpathcurveto{\pgfqpoint{3.388960in}{3.344000in}}{\pgfqpoint{3.399559in}{3.339609in}}{\pgfqpoint{3.410609in}{3.339609in}}%
\pgfpathclose%
\pgfusepath{stroke,fill}%
\end{pgfscope}%
\begin{pgfscope}%
\pgfpathrectangle{\pgfqpoint{0.481978in}{0.331635in}}{\pgfqpoint{9.300000in}{7.700000in}}%
\pgfusepath{clip}%
\pgfsetbuttcap%
\pgfsetroundjoin%
\definecolor{currentfill}{rgb}{0.552941,0.898039,0.631373}%
\pgfsetfillcolor{currentfill}%
\pgfsetlinewidth{0.481800pt}%
\definecolor{currentstroke}{rgb}{1.000000,1.000000,1.000000}%
\pgfsetstrokecolor{currentstroke}%
\pgfsetdash{}{0pt}%
\pgfpathmoveto{\pgfqpoint{9.044900in}{5.269063in}}%
\pgfpathcurveto{\pgfqpoint{9.055950in}{5.269063in}}{\pgfqpoint{9.066549in}{5.273453in}}{\pgfqpoint{9.074363in}{5.281267in}}%
\pgfpathcurveto{\pgfqpoint{9.082176in}{5.289081in}}{\pgfqpoint{9.086566in}{5.299680in}}{\pgfqpoint{9.086566in}{5.310730in}}%
\pgfpathcurveto{\pgfqpoint{9.086566in}{5.321780in}}{\pgfqpoint{9.082176in}{5.332379in}}{\pgfqpoint{9.074363in}{5.340192in}}%
\pgfpathcurveto{\pgfqpoint{9.066549in}{5.348006in}}{\pgfqpoint{9.055950in}{5.352396in}}{\pgfqpoint{9.044900in}{5.352396in}}%
\pgfpathcurveto{\pgfqpoint{9.033850in}{5.352396in}}{\pgfqpoint{9.023251in}{5.348006in}}{\pgfqpoint{9.015437in}{5.340192in}}%
\pgfpathcurveto{\pgfqpoint{9.007623in}{5.332379in}}{\pgfqpoint{9.003233in}{5.321780in}}{\pgfqpoint{9.003233in}{5.310730in}}%
\pgfpathcurveto{\pgfqpoint{9.003233in}{5.299680in}}{\pgfqpoint{9.007623in}{5.289081in}}{\pgfqpoint{9.015437in}{5.281267in}}%
\pgfpathcurveto{\pgfqpoint{9.023251in}{5.273453in}}{\pgfqpoint{9.033850in}{5.269063in}}{\pgfqpoint{9.044900in}{5.269063in}}%
\pgfpathclose%
\pgfusepath{stroke,fill}%
\end{pgfscope}%
\begin{pgfscope}%
\pgfpathrectangle{\pgfqpoint{0.481978in}{0.331635in}}{\pgfqpoint{9.300000in}{7.700000in}}%
\pgfusepath{clip}%
\pgfsetbuttcap%
\pgfsetroundjoin%
\definecolor{currentfill}{rgb}{0.552941,0.898039,0.631373}%
\pgfsetfillcolor{currentfill}%
\pgfsetlinewidth{0.481800pt}%
\definecolor{currentstroke}{rgb}{1.000000,1.000000,1.000000}%
\pgfsetstrokecolor{currentstroke}%
\pgfsetdash{}{0pt}%
\pgfpathmoveto{\pgfqpoint{8.024517in}{4.648323in}}%
\pgfpathcurveto{\pgfqpoint{8.035567in}{4.648323in}}{\pgfqpoint{8.046166in}{4.652713in}}{\pgfqpoint{8.053980in}{4.660527in}}%
\pgfpathcurveto{\pgfqpoint{8.061793in}{4.668341in}}{\pgfqpoint{8.066184in}{4.678940in}}{\pgfqpoint{8.066184in}{4.689990in}}%
\pgfpathcurveto{\pgfqpoint{8.066184in}{4.701040in}}{\pgfqpoint{8.061793in}{4.711639in}}{\pgfqpoint{8.053980in}{4.719453in}}%
\pgfpathcurveto{\pgfqpoint{8.046166in}{4.727266in}}{\pgfqpoint{8.035567in}{4.731656in}}{\pgfqpoint{8.024517in}{4.731656in}}%
\pgfpathcurveto{\pgfqpoint{8.013467in}{4.731656in}}{\pgfqpoint{8.002868in}{4.727266in}}{\pgfqpoint{7.995054in}{4.719453in}}%
\pgfpathcurveto{\pgfqpoint{7.987240in}{4.711639in}}{\pgfqpoint{7.982850in}{4.701040in}}{\pgfqpoint{7.982850in}{4.689990in}}%
\pgfpathcurveto{\pgfqpoint{7.982850in}{4.678940in}}{\pgfqpoint{7.987240in}{4.668341in}}{\pgfqpoint{7.995054in}{4.660527in}}%
\pgfpathcurveto{\pgfqpoint{8.002868in}{4.652713in}}{\pgfqpoint{8.013467in}{4.648323in}}{\pgfqpoint{8.024517in}{4.648323in}}%
\pgfpathclose%
\pgfusepath{stroke,fill}%
\end{pgfscope}%
\begin{pgfscope}%
\pgfpathrectangle{\pgfqpoint{0.481978in}{0.331635in}}{\pgfqpoint{9.300000in}{7.700000in}}%
\pgfusepath{clip}%
\pgfsetbuttcap%
\pgfsetroundjoin%
\definecolor{currentfill}{rgb}{0.552941,0.898039,0.631373}%
\pgfsetfillcolor{currentfill}%
\pgfsetlinewidth{0.481800pt}%
\definecolor{currentstroke}{rgb}{1.000000,1.000000,1.000000}%
\pgfsetstrokecolor{currentstroke}%
\pgfsetdash{}{0pt}%
\pgfpathmoveto{\pgfqpoint{3.710429in}{4.713864in}}%
\pgfpathcurveto{\pgfqpoint{3.721480in}{4.713864in}}{\pgfqpoint{3.732079in}{4.718254in}}{\pgfqpoint{3.739892in}{4.726068in}}%
\pgfpathcurveto{\pgfqpoint{3.747706in}{4.733881in}}{\pgfqpoint{3.752096in}{4.744480in}}{\pgfqpoint{3.752096in}{4.755530in}}%
\pgfpathcurveto{\pgfqpoint{3.752096in}{4.766580in}}{\pgfqpoint{3.747706in}{4.777179in}}{\pgfqpoint{3.739892in}{4.784993in}}%
\pgfpathcurveto{\pgfqpoint{3.732079in}{4.792807in}}{\pgfqpoint{3.721480in}{4.797197in}}{\pgfqpoint{3.710429in}{4.797197in}}%
\pgfpathcurveto{\pgfqpoint{3.699379in}{4.797197in}}{\pgfqpoint{3.688780in}{4.792807in}}{\pgfqpoint{3.680967in}{4.784993in}}%
\pgfpathcurveto{\pgfqpoint{3.673153in}{4.777179in}}{\pgfqpoint{3.668763in}{4.766580in}}{\pgfqpoint{3.668763in}{4.755530in}}%
\pgfpathcurveto{\pgfqpoint{3.668763in}{4.744480in}}{\pgfqpoint{3.673153in}{4.733881in}}{\pgfqpoint{3.680967in}{4.726068in}}%
\pgfpathcurveto{\pgfqpoint{3.688780in}{4.718254in}}{\pgfqpoint{3.699379in}{4.713864in}}{\pgfqpoint{3.710429in}{4.713864in}}%
\pgfpathclose%
\pgfusepath{stroke,fill}%
\end{pgfscope}%
\begin{pgfscope}%
\pgfpathrectangle{\pgfqpoint{0.481978in}{0.331635in}}{\pgfqpoint{9.300000in}{7.700000in}}%
\pgfusepath{clip}%
\pgfsetbuttcap%
\pgfsetroundjoin%
\definecolor{currentfill}{rgb}{0.552941,0.898039,0.631373}%
\pgfsetfillcolor{currentfill}%
\pgfsetlinewidth{0.481800pt}%
\definecolor{currentstroke}{rgb}{1.000000,1.000000,1.000000}%
\pgfsetstrokecolor{currentstroke}%
\pgfsetdash{}{0pt}%
\pgfpathmoveto{\pgfqpoint{5.690812in}{2.637048in}}%
\pgfpathcurveto{\pgfqpoint{5.701862in}{2.637048in}}{\pgfqpoint{5.712461in}{2.641438in}}{\pgfqpoint{5.720275in}{2.649252in}}%
\pgfpathcurveto{\pgfqpoint{5.728088in}{2.657065in}}{\pgfqpoint{5.732479in}{2.667664in}}{\pgfqpoint{5.732479in}{2.678714in}}%
\pgfpathcurveto{\pgfqpoint{5.732479in}{2.689765in}}{\pgfqpoint{5.728088in}{2.700364in}}{\pgfqpoint{5.720275in}{2.708177in}}%
\pgfpathcurveto{\pgfqpoint{5.712461in}{2.715991in}}{\pgfqpoint{5.701862in}{2.720381in}}{\pgfqpoint{5.690812in}{2.720381in}}%
\pgfpathcurveto{\pgfqpoint{5.679762in}{2.720381in}}{\pgfqpoint{5.669163in}{2.715991in}}{\pgfqpoint{5.661349in}{2.708177in}}%
\pgfpathcurveto{\pgfqpoint{5.653536in}{2.700364in}}{\pgfqpoint{5.649145in}{2.689765in}}{\pgfqpoint{5.649145in}{2.678714in}}%
\pgfpathcurveto{\pgfqpoint{5.649145in}{2.667664in}}{\pgfqpoint{5.653536in}{2.657065in}}{\pgfqpoint{5.661349in}{2.649252in}}%
\pgfpathcurveto{\pgfqpoint{5.669163in}{2.641438in}}{\pgfqpoint{5.679762in}{2.637048in}}{\pgfqpoint{5.690812in}{2.637048in}}%
\pgfpathclose%
\pgfusepath{stroke,fill}%
\end{pgfscope}%
\begin{pgfscope}%
\pgfpathrectangle{\pgfqpoint{0.481978in}{0.331635in}}{\pgfqpoint{9.300000in}{7.700000in}}%
\pgfusepath{clip}%
\pgfsetbuttcap%
\pgfsetroundjoin%
\definecolor{currentfill}{rgb}{1.000000,0.623529,0.607843}%
\pgfsetfillcolor{currentfill}%
\pgfsetlinewidth{0.481800pt}%
\definecolor{currentstroke}{rgb}{1.000000,1.000000,1.000000}%
\pgfsetstrokecolor{currentstroke}%
\pgfsetdash{}{0pt}%
\pgfpathmoveto{\pgfqpoint{7.737423in}{5.840523in}}%
\pgfpathcurveto{\pgfqpoint{7.748473in}{5.840523in}}{\pgfqpoint{7.759072in}{5.844913in}}{\pgfqpoint{7.766885in}{5.852727in}}%
\pgfpathcurveto{\pgfqpoint{7.774699in}{5.860541in}}{\pgfqpoint{7.779089in}{5.871140in}}{\pgfqpoint{7.779089in}{5.882190in}}%
\pgfpathcurveto{\pgfqpoint{7.779089in}{5.893240in}}{\pgfqpoint{7.774699in}{5.903839in}}{\pgfqpoint{7.766885in}{5.911653in}}%
\pgfpathcurveto{\pgfqpoint{7.759072in}{5.919466in}}{\pgfqpoint{7.748473in}{5.923856in}}{\pgfqpoint{7.737423in}{5.923856in}}%
\pgfpathcurveto{\pgfqpoint{7.726372in}{5.923856in}}{\pgfqpoint{7.715773in}{5.919466in}}{\pgfqpoint{7.707960in}{5.911653in}}%
\pgfpathcurveto{\pgfqpoint{7.700146in}{5.903839in}}{\pgfqpoint{7.695756in}{5.893240in}}{\pgfqpoint{7.695756in}{5.882190in}}%
\pgfpathcurveto{\pgfqpoint{7.695756in}{5.871140in}}{\pgfqpoint{7.700146in}{5.860541in}}{\pgfqpoint{7.707960in}{5.852727in}}%
\pgfpathcurveto{\pgfqpoint{7.715773in}{5.844913in}}{\pgfqpoint{7.726372in}{5.840523in}}{\pgfqpoint{7.737423in}{5.840523in}}%
\pgfpathclose%
\pgfusepath{stroke,fill}%
\end{pgfscope}%
\begin{pgfscope}%
\pgfpathrectangle{\pgfqpoint{0.481978in}{0.331635in}}{\pgfqpoint{9.300000in}{7.700000in}}%
\pgfusepath{clip}%
\pgfsetbuttcap%
\pgfsetroundjoin%
\definecolor{currentfill}{rgb}{1.000000,0.623529,0.607843}%
\pgfsetfillcolor{currentfill}%
\pgfsetlinewidth{0.481800pt}%
\definecolor{currentstroke}{rgb}{1.000000,1.000000,1.000000}%
\pgfsetstrokecolor{currentstroke}%
\pgfsetdash{}{0pt}%
\pgfpathmoveto{\pgfqpoint{8.478500in}{5.560595in}}%
\pgfpathcurveto{\pgfqpoint{8.489550in}{5.560595in}}{\pgfqpoint{8.500149in}{5.564985in}}{\pgfqpoint{8.507962in}{5.572799in}}%
\pgfpathcurveto{\pgfqpoint{8.515776in}{5.580613in}}{\pgfqpoint{8.520166in}{5.591212in}}{\pgfqpoint{8.520166in}{5.602262in}}%
\pgfpathcurveto{\pgfqpoint{8.520166in}{5.613312in}}{\pgfqpoint{8.515776in}{5.623911in}}{\pgfqpoint{8.507962in}{5.631724in}}%
\pgfpathcurveto{\pgfqpoint{8.500149in}{5.639538in}}{\pgfqpoint{8.489550in}{5.643928in}}{\pgfqpoint{8.478500in}{5.643928in}}%
\pgfpathcurveto{\pgfqpoint{8.467450in}{5.643928in}}{\pgfqpoint{8.456850in}{5.639538in}}{\pgfqpoint{8.449037in}{5.631724in}}%
\pgfpathcurveto{\pgfqpoint{8.441223in}{5.623911in}}{\pgfqpoint{8.436833in}{5.613312in}}{\pgfqpoint{8.436833in}{5.602262in}}%
\pgfpathcurveto{\pgfqpoint{8.436833in}{5.591212in}}{\pgfqpoint{8.441223in}{5.580613in}}{\pgfqpoint{8.449037in}{5.572799in}}%
\pgfpathcurveto{\pgfqpoint{8.456850in}{5.564985in}}{\pgfqpoint{8.467450in}{5.560595in}}{\pgfqpoint{8.478500in}{5.560595in}}%
\pgfpathclose%
\pgfusepath{stroke,fill}%
\end{pgfscope}%
\begin{pgfscope}%
\pgfpathrectangle{\pgfqpoint{0.481978in}{0.331635in}}{\pgfqpoint{9.300000in}{7.700000in}}%
\pgfusepath{clip}%
\pgfsetbuttcap%
\pgfsetroundjoin%
\definecolor{currentfill}{rgb}{1.000000,0.623529,0.607843}%
\pgfsetfillcolor{currentfill}%
\pgfsetlinewidth{0.481800pt}%
\definecolor{currentstroke}{rgb}{1.000000,1.000000,1.000000}%
\pgfsetstrokecolor{currentstroke}%
\pgfsetdash{}{0pt}%
\pgfpathmoveto{\pgfqpoint{5.299357in}{2.415253in}}%
\pgfpathcurveto{\pgfqpoint{5.310407in}{2.415253in}}{\pgfqpoint{5.321006in}{2.419644in}}{\pgfqpoint{5.328820in}{2.427457in}}%
\pgfpathcurveto{\pgfqpoint{5.336633in}{2.435271in}}{\pgfqpoint{5.341024in}{2.445870in}}{\pgfqpoint{5.341024in}{2.456920in}}%
\pgfpathcurveto{\pgfqpoint{5.341024in}{2.467970in}}{\pgfqpoint{5.336633in}{2.478569in}}{\pgfqpoint{5.328820in}{2.486383in}}%
\pgfpathcurveto{\pgfqpoint{5.321006in}{2.494196in}}{\pgfqpoint{5.310407in}{2.498587in}}{\pgfqpoint{5.299357in}{2.498587in}}%
\pgfpathcurveto{\pgfqpoint{5.288307in}{2.498587in}}{\pgfqpoint{5.277708in}{2.494196in}}{\pgfqpoint{5.269894in}{2.486383in}}%
\pgfpathcurveto{\pgfqpoint{5.262081in}{2.478569in}}{\pgfqpoint{5.257690in}{2.467970in}}{\pgfqpoint{5.257690in}{2.456920in}}%
\pgfpathcurveto{\pgfqpoint{5.257690in}{2.445870in}}{\pgfqpoint{5.262081in}{2.435271in}}{\pgfqpoint{5.269894in}{2.427457in}}%
\pgfpathcurveto{\pgfqpoint{5.277708in}{2.419644in}}{\pgfqpoint{5.288307in}{2.415253in}}{\pgfqpoint{5.299357in}{2.415253in}}%
\pgfpathclose%
\pgfusepath{stroke,fill}%
\end{pgfscope}%
\begin{pgfscope}%
\pgfpathrectangle{\pgfqpoint{0.481978in}{0.331635in}}{\pgfqpoint{9.300000in}{7.700000in}}%
\pgfusepath{clip}%
\pgfsetbuttcap%
\pgfsetroundjoin%
\definecolor{currentfill}{rgb}{1.000000,0.623529,0.607843}%
\pgfsetfillcolor{currentfill}%
\pgfsetlinewidth{0.481800pt}%
\definecolor{currentstroke}{rgb}{1.000000,1.000000,1.000000}%
\pgfsetstrokecolor{currentstroke}%
\pgfsetdash{}{0pt}%
\pgfpathmoveto{\pgfqpoint{5.050465in}{2.386257in}}%
\pgfpathcurveto{\pgfqpoint{5.061515in}{2.386257in}}{\pgfqpoint{5.072114in}{2.390648in}}{\pgfqpoint{5.079928in}{2.398461in}}%
\pgfpathcurveto{\pgfqpoint{5.087741in}{2.406275in}}{\pgfqpoint{5.092132in}{2.416874in}}{\pgfqpoint{5.092132in}{2.427924in}}%
\pgfpathcurveto{\pgfqpoint{5.092132in}{2.438974in}}{\pgfqpoint{5.087741in}{2.449573in}}{\pgfqpoint{5.079928in}{2.457387in}}%
\pgfpathcurveto{\pgfqpoint{5.072114in}{2.465200in}}{\pgfqpoint{5.061515in}{2.469591in}}{\pgfqpoint{5.050465in}{2.469591in}}%
\pgfpathcurveto{\pgfqpoint{5.039415in}{2.469591in}}{\pgfqpoint{5.028816in}{2.465200in}}{\pgfqpoint{5.021002in}{2.457387in}}%
\pgfpathcurveto{\pgfqpoint{5.013189in}{2.449573in}}{\pgfqpoint{5.008798in}{2.438974in}}{\pgfqpoint{5.008798in}{2.427924in}}%
\pgfpathcurveto{\pgfqpoint{5.008798in}{2.416874in}}{\pgfqpoint{5.013189in}{2.406275in}}{\pgfqpoint{5.021002in}{2.398461in}}%
\pgfpathcurveto{\pgfqpoint{5.028816in}{2.390648in}}{\pgfqpoint{5.039415in}{2.386257in}}{\pgfqpoint{5.050465in}{2.386257in}}%
\pgfpathclose%
\pgfusepath{stroke,fill}%
\end{pgfscope}%
\begin{pgfscope}%
\pgfpathrectangle{\pgfqpoint{0.481978in}{0.331635in}}{\pgfqpoint{9.300000in}{7.700000in}}%
\pgfusepath{clip}%
\pgfsetbuttcap%
\pgfsetroundjoin%
\definecolor{currentfill}{rgb}{1.000000,0.623529,0.607843}%
\pgfsetfillcolor{currentfill}%
\pgfsetlinewidth{0.481800pt}%
\definecolor{currentstroke}{rgb}{1.000000,1.000000,1.000000}%
\pgfsetstrokecolor{currentstroke}%
\pgfsetdash{}{0pt}%
\pgfpathmoveto{\pgfqpoint{4.424855in}{3.449766in}}%
\pgfpathcurveto{\pgfqpoint{4.435905in}{3.449766in}}{\pgfqpoint{4.446504in}{3.454156in}}{\pgfqpoint{4.454318in}{3.461970in}}%
\pgfpathcurveto{\pgfqpoint{4.462131in}{3.469783in}}{\pgfqpoint{4.466522in}{3.480382in}}{\pgfqpoint{4.466522in}{3.491433in}}%
\pgfpathcurveto{\pgfqpoint{4.466522in}{3.502483in}}{\pgfqpoint{4.462131in}{3.513082in}}{\pgfqpoint{4.454318in}{3.520895in}}%
\pgfpathcurveto{\pgfqpoint{4.446504in}{3.528709in}}{\pgfqpoint{4.435905in}{3.533099in}}{\pgfqpoint{4.424855in}{3.533099in}}%
\pgfpathcurveto{\pgfqpoint{4.413805in}{3.533099in}}{\pgfqpoint{4.403206in}{3.528709in}}{\pgfqpoint{4.395392in}{3.520895in}}%
\pgfpathcurveto{\pgfqpoint{4.387579in}{3.513082in}}{\pgfqpoint{4.383188in}{3.502483in}}{\pgfqpoint{4.383188in}{3.491433in}}%
\pgfpathcurveto{\pgfqpoint{4.383188in}{3.480382in}}{\pgfqpoint{4.387579in}{3.469783in}}{\pgfqpoint{4.395392in}{3.461970in}}%
\pgfpathcurveto{\pgfqpoint{4.403206in}{3.454156in}}{\pgfqpoint{4.413805in}{3.449766in}}{\pgfqpoint{4.424855in}{3.449766in}}%
\pgfpathclose%
\pgfusepath{stroke,fill}%
\end{pgfscope}%
\begin{pgfscope}%
\pgfpathrectangle{\pgfqpoint{0.481978in}{0.331635in}}{\pgfqpoint{9.300000in}{7.700000in}}%
\pgfusepath{clip}%
\pgfsetbuttcap%
\pgfsetroundjoin%
\definecolor{currentfill}{rgb}{1.000000,0.623529,0.607843}%
\pgfsetfillcolor{currentfill}%
\pgfsetlinewidth{0.481800pt}%
\definecolor{currentstroke}{rgb}{1.000000,1.000000,1.000000}%
\pgfsetstrokecolor{currentstroke}%
\pgfsetdash{}{0pt}%
\pgfpathmoveto{\pgfqpoint{4.703040in}{4.110492in}}%
\pgfpathcurveto{\pgfqpoint{4.714090in}{4.110492in}}{\pgfqpoint{4.724689in}{4.114882in}}{\pgfqpoint{4.732503in}{4.122696in}}%
\pgfpathcurveto{\pgfqpoint{4.740316in}{4.130509in}}{\pgfqpoint{4.744706in}{4.141108in}}{\pgfqpoint{4.744706in}{4.152158in}}%
\pgfpathcurveto{\pgfqpoint{4.744706in}{4.163209in}}{\pgfqpoint{4.740316in}{4.173808in}}{\pgfqpoint{4.732503in}{4.181621in}}%
\pgfpathcurveto{\pgfqpoint{4.724689in}{4.189435in}}{\pgfqpoint{4.714090in}{4.193825in}}{\pgfqpoint{4.703040in}{4.193825in}}%
\pgfpathcurveto{\pgfqpoint{4.691990in}{4.193825in}}{\pgfqpoint{4.681391in}{4.189435in}}{\pgfqpoint{4.673577in}{4.181621in}}%
\pgfpathcurveto{\pgfqpoint{4.665763in}{4.173808in}}{\pgfqpoint{4.661373in}{4.163209in}}{\pgfqpoint{4.661373in}{4.152158in}}%
\pgfpathcurveto{\pgfqpoint{4.661373in}{4.141108in}}{\pgfqpoint{4.665763in}{4.130509in}}{\pgfqpoint{4.673577in}{4.122696in}}%
\pgfpathcurveto{\pgfqpoint{4.681391in}{4.114882in}}{\pgfqpoint{4.691990in}{4.110492in}}{\pgfqpoint{4.703040in}{4.110492in}}%
\pgfpathclose%
\pgfusepath{stroke,fill}%
\end{pgfscope}%
\begin{pgfscope}%
\pgfpathrectangle{\pgfqpoint{0.481978in}{0.331635in}}{\pgfqpoint{9.300000in}{7.700000in}}%
\pgfusepath{clip}%
\pgfsetbuttcap%
\pgfsetroundjoin%
\definecolor{currentfill}{rgb}{1.000000,0.623529,0.607843}%
\pgfsetfillcolor{currentfill}%
\pgfsetlinewidth{0.481800pt}%
\definecolor{currentstroke}{rgb}{1.000000,1.000000,1.000000}%
\pgfsetstrokecolor{currentstroke}%
\pgfsetdash{}{0pt}%
\pgfpathmoveto{\pgfqpoint{2.809853in}{3.839407in}}%
\pgfpathcurveto{\pgfqpoint{2.820903in}{3.839407in}}{\pgfqpoint{2.831502in}{3.843797in}}{\pgfqpoint{2.839316in}{3.851610in}}%
\pgfpathcurveto{\pgfqpoint{2.847129in}{3.859424in}}{\pgfqpoint{2.851520in}{3.870023in}}{\pgfqpoint{2.851520in}{3.881073in}}%
\pgfpathcurveto{\pgfqpoint{2.851520in}{3.892123in}}{\pgfqpoint{2.847129in}{3.902722in}}{\pgfqpoint{2.839316in}{3.910536in}}%
\pgfpathcurveto{\pgfqpoint{2.831502in}{3.918350in}}{\pgfqpoint{2.820903in}{3.922740in}}{\pgfqpoint{2.809853in}{3.922740in}}%
\pgfpathcurveto{\pgfqpoint{2.798803in}{3.922740in}}{\pgfqpoint{2.788204in}{3.918350in}}{\pgfqpoint{2.780390in}{3.910536in}}%
\pgfpathcurveto{\pgfqpoint{2.772577in}{3.902722in}}{\pgfqpoint{2.768186in}{3.892123in}}{\pgfqpoint{2.768186in}{3.881073in}}%
\pgfpathcurveto{\pgfqpoint{2.768186in}{3.870023in}}{\pgfqpoint{2.772577in}{3.859424in}}{\pgfqpoint{2.780390in}{3.851610in}}%
\pgfpathcurveto{\pgfqpoint{2.788204in}{3.843797in}}{\pgfqpoint{2.798803in}{3.839407in}}{\pgfqpoint{2.809853in}{3.839407in}}%
\pgfpathclose%
\pgfusepath{stroke,fill}%
\end{pgfscope}%
\begin{pgfscope}%
\pgfpathrectangle{\pgfqpoint{0.481978in}{0.331635in}}{\pgfqpoint{9.300000in}{7.700000in}}%
\pgfusepath{clip}%
\pgfsetbuttcap%
\pgfsetroundjoin%
\definecolor{currentfill}{rgb}{1.000000,0.623529,0.607843}%
\pgfsetfillcolor{currentfill}%
\pgfsetlinewidth{0.481800pt}%
\definecolor{currentstroke}{rgb}{1.000000,1.000000,1.000000}%
\pgfsetstrokecolor{currentstroke}%
\pgfsetdash{}{0pt}%
\pgfpathmoveto{\pgfqpoint{3.971771in}{3.008119in}}%
\pgfpathcurveto{\pgfqpoint{3.982821in}{3.008119in}}{\pgfqpoint{3.993421in}{3.012509in}}{\pgfqpoint{4.001234in}{3.020323in}}%
\pgfpathcurveto{\pgfqpoint{4.009048in}{3.028136in}}{\pgfqpoint{4.013438in}{3.038735in}}{\pgfqpoint{4.013438in}{3.049785in}}%
\pgfpathcurveto{\pgfqpoint{4.013438in}{3.060835in}}{\pgfqpoint{4.009048in}{3.071435in}}{\pgfqpoint{4.001234in}{3.079248in}}%
\pgfpathcurveto{\pgfqpoint{3.993421in}{3.087062in}}{\pgfqpoint{3.982821in}{3.091452in}}{\pgfqpoint{3.971771in}{3.091452in}}%
\pgfpathcurveto{\pgfqpoint{3.960721in}{3.091452in}}{\pgfqpoint{3.950122in}{3.087062in}}{\pgfqpoint{3.942309in}{3.079248in}}%
\pgfpathcurveto{\pgfqpoint{3.934495in}{3.071435in}}{\pgfqpoint{3.930105in}{3.060835in}}{\pgfqpoint{3.930105in}{3.049785in}}%
\pgfpathcurveto{\pgfqpoint{3.930105in}{3.038735in}}{\pgfqpoint{3.934495in}{3.028136in}}{\pgfqpoint{3.942309in}{3.020323in}}%
\pgfpathcurveto{\pgfqpoint{3.950122in}{3.012509in}}{\pgfqpoint{3.960721in}{3.008119in}}{\pgfqpoint{3.971771in}{3.008119in}}%
\pgfpathclose%
\pgfusepath{stroke,fill}%
\end{pgfscope}%
\begin{pgfscope}%
\pgfpathrectangle{\pgfqpoint{0.481978in}{0.331635in}}{\pgfqpoint{9.300000in}{7.700000in}}%
\pgfusepath{clip}%
\pgfsetbuttcap%
\pgfsetroundjoin%
\definecolor{currentfill}{rgb}{1.000000,0.623529,0.607843}%
\pgfsetfillcolor{currentfill}%
\pgfsetlinewidth{0.481800pt}%
\definecolor{currentstroke}{rgb}{1.000000,1.000000,1.000000}%
\pgfsetstrokecolor{currentstroke}%
\pgfsetdash{}{0pt}%
\pgfpathmoveto{\pgfqpoint{7.681317in}{5.060668in}}%
\pgfpathcurveto{\pgfqpoint{7.692368in}{5.060668in}}{\pgfqpoint{7.702967in}{5.065058in}}{\pgfqpoint{7.710780in}{5.072872in}}%
\pgfpathcurveto{\pgfqpoint{7.718594in}{5.080685in}}{\pgfqpoint{7.722984in}{5.091284in}}{\pgfqpoint{7.722984in}{5.102335in}}%
\pgfpathcurveto{\pgfqpoint{7.722984in}{5.113385in}}{\pgfqpoint{7.718594in}{5.123984in}}{\pgfqpoint{7.710780in}{5.131797in}}%
\pgfpathcurveto{\pgfqpoint{7.702967in}{5.139611in}}{\pgfqpoint{7.692368in}{5.144001in}}{\pgfqpoint{7.681317in}{5.144001in}}%
\pgfpathcurveto{\pgfqpoint{7.670267in}{5.144001in}}{\pgfqpoint{7.659668in}{5.139611in}}{\pgfqpoint{7.651855in}{5.131797in}}%
\pgfpathcurveto{\pgfqpoint{7.644041in}{5.123984in}}{\pgfqpoint{7.639651in}{5.113385in}}{\pgfqpoint{7.639651in}{5.102335in}}%
\pgfpathcurveto{\pgfqpoint{7.639651in}{5.091284in}}{\pgfqpoint{7.644041in}{5.080685in}}{\pgfqpoint{7.651855in}{5.072872in}}%
\pgfpathcurveto{\pgfqpoint{7.659668in}{5.065058in}}{\pgfqpoint{7.670267in}{5.060668in}}{\pgfqpoint{7.681317in}{5.060668in}}%
\pgfpathclose%
\pgfusepath{stroke,fill}%
\end{pgfscope}%
\begin{pgfscope}%
\pgfpathrectangle{\pgfqpoint{0.481978in}{0.331635in}}{\pgfqpoint{9.300000in}{7.700000in}}%
\pgfusepath{clip}%
\pgfsetbuttcap%
\pgfsetroundjoin%
\definecolor{currentfill}{rgb}{1.000000,0.623529,0.607843}%
\pgfsetfillcolor{currentfill}%
\pgfsetlinewidth{0.481800pt}%
\definecolor{currentstroke}{rgb}{1.000000,1.000000,1.000000}%
\pgfsetstrokecolor{currentstroke}%
\pgfsetdash{}{0pt}%
\pgfpathmoveto{\pgfqpoint{4.922074in}{3.099400in}}%
\pgfpathcurveto{\pgfqpoint{4.933124in}{3.099400in}}{\pgfqpoint{4.943723in}{3.103790in}}{\pgfqpoint{4.951537in}{3.111604in}}%
\pgfpathcurveto{\pgfqpoint{4.959350in}{3.119417in}}{\pgfqpoint{4.963741in}{3.130016in}}{\pgfqpoint{4.963741in}{3.141067in}}%
\pgfpathcurveto{\pgfqpoint{4.963741in}{3.152117in}}{\pgfqpoint{4.959350in}{3.162716in}}{\pgfqpoint{4.951537in}{3.170529in}}%
\pgfpathcurveto{\pgfqpoint{4.943723in}{3.178343in}}{\pgfqpoint{4.933124in}{3.182733in}}{\pgfqpoint{4.922074in}{3.182733in}}%
\pgfpathcurveto{\pgfqpoint{4.911024in}{3.182733in}}{\pgfqpoint{4.900425in}{3.178343in}}{\pgfqpoint{4.892611in}{3.170529in}}%
\pgfpathcurveto{\pgfqpoint{4.884798in}{3.162716in}}{\pgfqpoint{4.880407in}{3.152117in}}{\pgfqpoint{4.880407in}{3.141067in}}%
\pgfpathcurveto{\pgfqpoint{4.880407in}{3.130016in}}{\pgfqpoint{4.884798in}{3.119417in}}{\pgfqpoint{4.892611in}{3.111604in}}%
\pgfpathcurveto{\pgfqpoint{4.900425in}{3.103790in}}{\pgfqpoint{4.911024in}{3.099400in}}{\pgfqpoint{4.922074in}{3.099400in}}%
\pgfpathclose%
\pgfusepath{stroke,fill}%
\end{pgfscope}%
\begin{pgfscope}%
\pgfpathrectangle{\pgfqpoint{0.481978in}{0.331635in}}{\pgfqpoint{9.300000in}{7.700000in}}%
\pgfusepath{clip}%
\pgfsetbuttcap%
\pgfsetroundjoin%
\definecolor{currentfill}{rgb}{1.000000,0.623529,0.607843}%
\pgfsetfillcolor{currentfill}%
\pgfsetlinewidth{0.481800pt}%
\definecolor{currentstroke}{rgb}{1.000000,1.000000,1.000000}%
\pgfsetstrokecolor{currentstroke}%
\pgfsetdash{}{0pt}%
\pgfpathmoveto{\pgfqpoint{4.349051in}{3.104492in}}%
\pgfpathcurveto{\pgfqpoint{4.360101in}{3.104492in}}{\pgfqpoint{4.370700in}{3.108882in}}{\pgfqpoint{4.378514in}{3.116696in}}%
\pgfpathcurveto{\pgfqpoint{4.386327in}{3.124510in}}{\pgfqpoint{4.390718in}{3.135109in}}{\pgfqpoint{4.390718in}{3.146159in}}%
\pgfpathcurveto{\pgfqpoint{4.390718in}{3.157209in}}{\pgfqpoint{4.386327in}{3.167808in}}{\pgfqpoint{4.378514in}{3.175621in}}%
\pgfpathcurveto{\pgfqpoint{4.370700in}{3.183435in}}{\pgfqpoint{4.360101in}{3.187825in}}{\pgfqpoint{4.349051in}{3.187825in}}%
\pgfpathcurveto{\pgfqpoint{4.338001in}{3.187825in}}{\pgfqpoint{4.327402in}{3.183435in}}{\pgfqpoint{4.319588in}{3.175621in}}%
\pgfpathcurveto{\pgfqpoint{4.311775in}{3.167808in}}{\pgfqpoint{4.307384in}{3.157209in}}{\pgfqpoint{4.307384in}{3.146159in}}%
\pgfpathcurveto{\pgfqpoint{4.307384in}{3.135109in}}{\pgfqpoint{4.311775in}{3.124510in}}{\pgfqpoint{4.319588in}{3.116696in}}%
\pgfpathcurveto{\pgfqpoint{4.327402in}{3.108882in}}{\pgfqpoint{4.338001in}{3.104492in}}{\pgfqpoint{4.349051in}{3.104492in}}%
\pgfpathclose%
\pgfusepath{stroke,fill}%
\end{pgfscope}%
\begin{pgfscope}%
\pgfpathrectangle{\pgfqpoint{0.481978in}{0.331635in}}{\pgfqpoint{9.300000in}{7.700000in}}%
\pgfusepath{clip}%
\pgfsetbuttcap%
\pgfsetroundjoin%
\definecolor{currentfill}{rgb}{1.000000,0.623529,0.607843}%
\pgfsetfillcolor{currentfill}%
\pgfsetlinewidth{0.481800pt}%
\definecolor{currentstroke}{rgb}{1.000000,1.000000,1.000000}%
\pgfsetstrokecolor{currentstroke}%
\pgfsetdash{}{0pt}%
\pgfpathmoveto{\pgfqpoint{4.285306in}{3.855591in}}%
\pgfpathcurveto{\pgfqpoint{4.296356in}{3.855591in}}{\pgfqpoint{4.306955in}{3.859981in}}{\pgfqpoint{4.314769in}{3.867794in}}%
\pgfpathcurveto{\pgfqpoint{4.322583in}{3.875608in}}{\pgfqpoint{4.326973in}{3.886207in}}{\pgfqpoint{4.326973in}{3.897257in}}%
\pgfpathcurveto{\pgfqpoint{4.326973in}{3.908307in}}{\pgfqpoint{4.322583in}{3.918906in}}{\pgfqpoint{4.314769in}{3.926720in}}%
\pgfpathcurveto{\pgfqpoint{4.306955in}{3.934534in}}{\pgfqpoint{4.296356in}{3.938924in}}{\pgfqpoint{4.285306in}{3.938924in}}%
\pgfpathcurveto{\pgfqpoint{4.274256in}{3.938924in}}{\pgfqpoint{4.263657in}{3.934534in}}{\pgfqpoint{4.255843in}{3.926720in}}%
\pgfpathcurveto{\pgfqpoint{4.248030in}{3.918906in}}{\pgfqpoint{4.243639in}{3.908307in}}{\pgfqpoint{4.243639in}{3.897257in}}%
\pgfpathcurveto{\pgfqpoint{4.243639in}{3.886207in}}{\pgfqpoint{4.248030in}{3.875608in}}{\pgfqpoint{4.255843in}{3.867794in}}%
\pgfpathcurveto{\pgfqpoint{4.263657in}{3.859981in}}{\pgfqpoint{4.274256in}{3.855591in}}{\pgfqpoint{4.285306in}{3.855591in}}%
\pgfpathclose%
\pgfusepath{stroke,fill}%
\end{pgfscope}%
\begin{pgfscope}%
\pgfpathrectangle{\pgfqpoint{0.481978in}{0.331635in}}{\pgfqpoint{9.300000in}{7.700000in}}%
\pgfusepath{clip}%
\pgfsetbuttcap%
\pgfsetroundjoin%
\definecolor{currentfill}{rgb}{1.000000,0.623529,0.607843}%
\pgfsetfillcolor{currentfill}%
\pgfsetlinewidth{0.481800pt}%
\definecolor{currentstroke}{rgb}{1.000000,1.000000,1.000000}%
\pgfsetstrokecolor{currentstroke}%
\pgfsetdash{}{0pt}%
\pgfpathmoveto{\pgfqpoint{5.628759in}{1.318174in}}%
\pgfpathcurveto{\pgfqpoint{5.639809in}{1.318174in}}{\pgfqpoint{5.650408in}{1.322564in}}{\pgfqpoint{5.658221in}{1.330378in}}%
\pgfpathcurveto{\pgfqpoint{5.666035in}{1.338191in}}{\pgfqpoint{5.670425in}{1.348790in}}{\pgfqpoint{5.670425in}{1.359840in}}%
\pgfpathcurveto{\pgfqpoint{5.670425in}{1.370890in}}{\pgfqpoint{5.666035in}{1.381489in}}{\pgfqpoint{5.658221in}{1.389303in}}%
\pgfpathcurveto{\pgfqpoint{5.650408in}{1.397117in}}{\pgfqpoint{5.639809in}{1.401507in}}{\pgfqpoint{5.628759in}{1.401507in}}%
\pgfpathcurveto{\pgfqpoint{5.617708in}{1.401507in}}{\pgfqpoint{5.607109in}{1.397117in}}{\pgfqpoint{5.599296in}{1.389303in}}%
\pgfpathcurveto{\pgfqpoint{5.591482in}{1.381489in}}{\pgfqpoint{5.587092in}{1.370890in}}{\pgfqpoint{5.587092in}{1.359840in}}%
\pgfpathcurveto{\pgfqpoint{5.587092in}{1.348790in}}{\pgfqpoint{5.591482in}{1.338191in}}{\pgfqpoint{5.599296in}{1.330378in}}%
\pgfpathcurveto{\pgfqpoint{5.607109in}{1.322564in}}{\pgfqpoint{5.617708in}{1.318174in}}{\pgfqpoint{5.628759in}{1.318174in}}%
\pgfpathclose%
\pgfusepath{stroke,fill}%
\end{pgfscope}%
\begin{pgfscope}%
\pgfpathrectangle{\pgfqpoint{0.481978in}{0.331635in}}{\pgfqpoint{9.300000in}{7.700000in}}%
\pgfusepath{clip}%
\pgfsetbuttcap%
\pgfsetroundjoin%
\definecolor{currentfill}{rgb}{1.000000,0.623529,0.607843}%
\pgfsetfillcolor{currentfill}%
\pgfsetlinewidth{0.481800pt}%
\definecolor{currentstroke}{rgb}{1.000000,1.000000,1.000000}%
\pgfsetstrokecolor{currentstroke}%
\pgfsetdash{}{0pt}%
\pgfpathmoveto{\pgfqpoint{2.208081in}{3.690280in}}%
\pgfpathcurveto{\pgfqpoint{2.219131in}{3.690280in}}{\pgfqpoint{2.229730in}{3.694670in}}{\pgfqpoint{2.237544in}{3.702484in}}%
\pgfpathcurveto{\pgfqpoint{2.245357in}{3.710298in}}{\pgfqpoint{2.249748in}{3.720897in}}{\pgfqpoint{2.249748in}{3.731947in}}%
\pgfpathcurveto{\pgfqpoint{2.249748in}{3.742997in}}{\pgfqpoint{2.245357in}{3.753596in}}{\pgfqpoint{2.237544in}{3.761410in}}%
\pgfpathcurveto{\pgfqpoint{2.229730in}{3.769223in}}{\pgfqpoint{2.219131in}{3.773614in}}{\pgfqpoint{2.208081in}{3.773614in}}%
\pgfpathcurveto{\pgfqpoint{2.197031in}{3.773614in}}{\pgfqpoint{2.186432in}{3.769223in}}{\pgfqpoint{2.178618in}{3.761410in}}%
\pgfpathcurveto{\pgfqpoint{2.170805in}{3.753596in}}{\pgfqpoint{2.166414in}{3.742997in}}{\pgfqpoint{2.166414in}{3.731947in}}%
\pgfpathcurveto{\pgfqpoint{2.166414in}{3.720897in}}{\pgfqpoint{2.170805in}{3.710298in}}{\pgfqpoint{2.178618in}{3.702484in}}%
\pgfpathcurveto{\pgfqpoint{2.186432in}{3.694670in}}{\pgfqpoint{2.197031in}{3.690280in}}{\pgfqpoint{2.208081in}{3.690280in}}%
\pgfpathclose%
\pgfusepath{stroke,fill}%
\end{pgfscope}%
\begin{pgfscope}%
\pgfpathrectangle{\pgfqpoint{0.481978in}{0.331635in}}{\pgfqpoint{9.300000in}{7.700000in}}%
\pgfusepath{clip}%
\pgfsetbuttcap%
\pgfsetroundjoin%
\definecolor{currentfill}{rgb}{1.000000,0.623529,0.607843}%
\pgfsetfillcolor{currentfill}%
\pgfsetlinewidth{0.481800pt}%
\definecolor{currentstroke}{rgb}{1.000000,1.000000,1.000000}%
\pgfsetstrokecolor{currentstroke}%
\pgfsetdash{}{0pt}%
\pgfpathmoveto{\pgfqpoint{5.627987in}{1.316150in}}%
\pgfpathcurveto{\pgfqpoint{5.639037in}{1.316150in}}{\pgfqpoint{5.649636in}{1.320540in}}{\pgfqpoint{5.657450in}{1.328354in}}%
\pgfpathcurveto{\pgfqpoint{5.665263in}{1.336167in}}{\pgfqpoint{5.669654in}{1.346766in}}{\pgfqpoint{5.669654in}{1.357817in}}%
\pgfpathcurveto{\pgfqpoint{5.669654in}{1.368867in}}{\pgfqpoint{5.665263in}{1.379466in}}{\pgfqpoint{5.657450in}{1.387279in}}%
\pgfpathcurveto{\pgfqpoint{5.649636in}{1.395093in}}{\pgfqpoint{5.639037in}{1.399483in}}{\pgfqpoint{5.627987in}{1.399483in}}%
\pgfpathcurveto{\pgfqpoint{5.616937in}{1.399483in}}{\pgfqpoint{5.606338in}{1.395093in}}{\pgfqpoint{5.598524in}{1.387279in}}%
\pgfpathcurveto{\pgfqpoint{5.590711in}{1.379466in}}{\pgfqpoint{5.586320in}{1.368867in}}{\pgfqpoint{5.586320in}{1.357817in}}%
\pgfpathcurveto{\pgfqpoint{5.586320in}{1.346766in}}{\pgfqpoint{5.590711in}{1.336167in}}{\pgfqpoint{5.598524in}{1.328354in}}%
\pgfpathcurveto{\pgfqpoint{5.606338in}{1.320540in}}{\pgfqpoint{5.616937in}{1.316150in}}{\pgfqpoint{5.627987in}{1.316150in}}%
\pgfpathclose%
\pgfusepath{stroke,fill}%
\end{pgfscope}%
\begin{pgfscope}%
\pgfpathrectangle{\pgfqpoint{0.481978in}{0.331635in}}{\pgfqpoint{9.300000in}{7.700000in}}%
\pgfusepath{clip}%
\pgfsetbuttcap%
\pgfsetroundjoin%
\definecolor{currentfill}{rgb}{1.000000,0.623529,0.607843}%
\pgfsetfillcolor{currentfill}%
\pgfsetlinewidth{0.481800pt}%
\definecolor{currentstroke}{rgb}{1.000000,1.000000,1.000000}%
\pgfsetstrokecolor{currentstroke}%
\pgfsetdash{}{0pt}%
\pgfpathmoveto{\pgfqpoint{7.678869in}{4.472682in}}%
\pgfpathcurveto{\pgfqpoint{7.689919in}{4.472682in}}{\pgfqpoint{7.700518in}{4.477072in}}{\pgfqpoint{7.708332in}{4.484886in}}%
\pgfpathcurveto{\pgfqpoint{7.716145in}{4.492700in}}{\pgfqpoint{7.720536in}{4.503299in}}{\pgfqpoint{7.720536in}{4.514349in}}%
\pgfpathcurveto{\pgfqpoint{7.720536in}{4.525399in}}{\pgfqpoint{7.716145in}{4.535998in}}{\pgfqpoint{7.708332in}{4.543812in}}%
\pgfpathcurveto{\pgfqpoint{7.700518in}{4.551625in}}{\pgfqpoint{7.689919in}{4.556015in}}{\pgfqpoint{7.678869in}{4.556015in}}%
\pgfpathcurveto{\pgfqpoint{7.667819in}{4.556015in}}{\pgfqpoint{7.657220in}{4.551625in}}{\pgfqpoint{7.649406in}{4.543812in}}%
\pgfpathcurveto{\pgfqpoint{7.641593in}{4.535998in}}{\pgfqpoint{7.637202in}{4.525399in}}{\pgfqpoint{7.637202in}{4.514349in}}%
\pgfpathcurveto{\pgfqpoint{7.637202in}{4.503299in}}{\pgfqpoint{7.641593in}{4.492700in}}{\pgfqpoint{7.649406in}{4.484886in}}%
\pgfpathcurveto{\pgfqpoint{7.657220in}{4.477072in}}{\pgfqpoint{7.667819in}{4.472682in}}{\pgfqpoint{7.678869in}{4.472682in}}%
\pgfpathclose%
\pgfusepath{stroke,fill}%
\end{pgfscope}%
\begin{pgfscope}%
\pgfpathrectangle{\pgfqpoint{0.481978in}{0.331635in}}{\pgfqpoint{9.300000in}{7.700000in}}%
\pgfusepath{clip}%
\pgfsetbuttcap%
\pgfsetroundjoin%
\definecolor{currentfill}{rgb}{1.000000,0.623529,0.607843}%
\pgfsetfillcolor{currentfill}%
\pgfsetlinewidth{0.481800pt}%
\definecolor{currentstroke}{rgb}{1.000000,1.000000,1.000000}%
\pgfsetstrokecolor{currentstroke}%
\pgfsetdash{}{0pt}%
\pgfpathmoveto{\pgfqpoint{3.663088in}{1.980706in}}%
\pgfpathcurveto{\pgfqpoint{3.674139in}{1.980706in}}{\pgfqpoint{3.684738in}{1.985096in}}{\pgfqpoint{3.692551in}{1.992910in}}%
\pgfpathcurveto{\pgfqpoint{3.700365in}{2.000723in}}{\pgfqpoint{3.704755in}{2.011322in}}{\pgfqpoint{3.704755in}{2.022372in}}%
\pgfpathcurveto{\pgfqpoint{3.704755in}{2.033423in}}{\pgfqpoint{3.700365in}{2.044022in}}{\pgfqpoint{3.692551in}{2.051835in}}%
\pgfpathcurveto{\pgfqpoint{3.684738in}{2.059649in}}{\pgfqpoint{3.674139in}{2.064039in}}{\pgfqpoint{3.663088in}{2.064039in}}%
\pgfpathcurveto{\pgfqpoint{3.652038in}{2.064039in}}{\pgfqpoint{3.641439in}{2.059649in}}{\pgfqpoint{3.633626in}{2.051835in}}%
\pgfpathcurveto{\pgfqpoint{3.625812in}{2.044022in}}{\pgfqpoint{3.621422in}{2.033423in}}{\pgfqpoint{3.621422in}{2.022372in}}%
\pgfpathcurveto{\pgfqpoint{3.621422in}{2.011322in}}{\pgfqpoint{3.625812in}{2.000723in}}{\pgfqpoint{3.633626in}{1.992910in}}%
\pgfpathcurveto{\pgfqpoint{3.641439in}{1.985096in}}{\pgfqpoint{3.652038in}{1.980706in}}{\pgfqpoint{3.663088in}{1.980706in}}%
\pgfpathclose%
\pgfusepath{stroke,fill}%
\end{pgfscope}%
\begin{pgfscope}%
\pgfpathrectangle{\pgfqpoint{0.481978in}{0.331635in}}{\pgfqpoint{9.300000in}{7.700000in}}%
\pgfusepath{clip}%
\pgfsetbuttcap%
\pgfsetroundjoin%
\definecolor{currentfill}{rgb}{1.000000,0.623529,0.607843}%
\pgfsetfillcolor{currentfill}%
\pgfsetlinewidth{0.481800pt}%
\definecolor{currentstroke}{rgb}{1.000000,1.000000,1.000000}%
\pgfsetstrokecolor{currentstroke}%
\pgfsetdash{}{0pt}%
\pgfpathmoveto{\pgfqpoint{2.974676in}{3.669425in}}%
\pgfpathcurveto{\pgfqpoint{2.985726in}{3.669425in}}{\pgfqpoint{2.996325in}{3.673815in}}{\pgfqpoint{3.004138in}{3.681629in}}%
\pgfpathcurveto{\pgfqpoint{3.011952in}{3.689442in}}{\pgfqpoint{3.016342in}{3.700041in}}{\pgfqpoint{3.016342in}{3.711091in}}%
\pgfpathcurveto{\pgfqpoint{3.016342in}{3.722141in}}{\pgfqpoint{3.011952in}{3.732741in}}{\pgfqpoint{3.004138in}{3.740554in}}%
\pgfpathcurveto{\pgfqpoint{2.996325in}{3.748368in}}{\pgfqpoint{2.985726in}{3.752758in}}{\pgfqpoint{2.974676in}{3.752758in}}%
\pgfpathcurveto{\pgfqpoint{2.963625in}{3.752758in}}{\pgfqpoint{2.953026in}{3.748368in}}{\pgfqpoint{2.945213in}{3.740554in}}%
\pgfpathcurveto{\pgfqpoint{2.937399in}{3.732741in}}{\pgfqpoint{2.933009in}{3.722141in}}{\pgfqpoint{2.933009in}{3.711091in}}%
\pgfpathcurveto{\pgfqpoint{2.933009in}{3.700041in}}{\pgfqpoint{2.937399in}{3.689442in}}{\pgfqpoint{2.945213in}{3.681629in}}%
\pgfpathcurveto{\pgfqpoint{2.953026in}{3.673815in}}{\pgfqpoint{2.963625in}{3.669425in}}{\pgfqpoint{2.974676in}{3.669425in}}%
\pgfpathclose%
\pgfusepath{stroke,fill}%
\end{pgfscope}%
\begin{pgfscope}%
\pgfpathrectangle{\pgfqpoint{0.481978in}{0.331635in}}{\pgfqpoint{9.300000in}{7.700000in}}%
\pgfusepath{clip}%
\pgfsetbuttcap%
\pgfsetroundjoin%
\definecolor{currentfill}{rgb}{1.000000,0.623529,0.607843}%
\pgfsetfillcolor{currentfill}%
\pgfsetlinewidth{0.481800pt}%
\definecolor{currentstroke}{rgb}{1.000000,1.000000,1.000000}%
\pgfsetstrokecolor{currentstroke}%
\pgfsetdash{}{0pt}%
\pgfpathmoveto{\pgfqpoint{4.287456in}{3.550314in}}%
\pgfpathcurveto{\pgfqpoint{4.298506in}{3.550314in}}{\pgfqpoint{4.309105in}{3.554704in}}{\pgfqpoint{4.316919in}{3.562518in}}%
\pgfpathcurveto{\pgfqpoint{4.324732in}{3.570331in}}{\pgfqpoint{4.329122in}{3.580930in}}{\pgfqpoint{4.329122in}{3.591981in}}%
\pgfpathcurveto{\pgfqpoint{4.329122in}{3.603031in}}{\pgfqpoint{4.324732in}{3.613630in}}{\pgfqpoint{4.316919in}{3.621443in}}%
\pgfpathcurveto{\pgfqpoint{4.309105in}{3.629257in}}{\pgfqpoint{4.298506in}{3.633647in}}{\pgfqpoint{4.287456in}{3.633647in}}%
\pgfpathcurveto{\pgfqpoint{4.276406in}{3.633647in}}{\pgfqpoint{4.265807in}{3.629257in}}{\pgfqpoint{4.257993in}{3.621443in}}%
\pgfpathcurveto{\pgfqpoint{4.250179in}{3.613630in}}{\pgfqpoint{4.245789in}{3.603031in}}{\pgfqpoint{4.245789in}{3.591981in}}%
\pgfpathcurveto{\pgfqpoint{4.245789in}{3.580930in}}{\pgfqpoint{4.250179in}{3.570331in}}{\pgfqpoint{4.257993in}{3.562518in}}%
\pgfpathcurveto{\pgfqpoint{4.265807in}{3.554704in}}{\pgfqpoint{4.276406in}{3.550314in}}{\pgfqpoint{4.287456in}{3.550314in}}%
\pgfpathclose%
\pgfusepath{stroke,fill}%
\end{pgfscope}%
\begin{pgfscope}%
\pgfpathrectangle{\pgfqpoint{0.481978in}{0.331635in}}{\pgfqpoint{9.300000in}{7.700000in}}%
\pgfusepath{clip}%
\pgfsetbuttcap%
\pgfsetroundjoin%
\definecolor{currentfill}{rgb}{1.000000,0.623529,0.607843}%
\pgfsetfillcolor{currentfill}%
\pgfsetlinewidth{0.481800pt}%
\definecolor{currentstroke}{rgb}{1.000000,1.000000,1.000000}%
\pgfsetstrokecolor{currentstroke}%
\pgfsetdash{}{0pt}%
\pgfpathmoveto{\pgfqpoint{1.967877in}{5.548168in}}%
\pgfpathcurveto{\pgfqpoint{1.978927in}{5.548168in}}{\pgfqpoint{1.989526in}{5.552558in}}{\pgfqpoint{1.997339in}{5.560372in}}%
\pgfpathcurveto{\pgfqpoint{2.005153in}{5.568185in}}{\pgfqpoint{2.009543in}{5.578784in}}{\pgfqpoint{2.009543in}{5.589834in}}%
\pgfpathcurveto{\pgfqpoint{2.009543in}{5.600885in}}{\pgfqpoint{2.005153in}{5.611484in}}{\pgfqpoint{1.997339in}{5.619297in}}%
\pgfpathcurveto{\pgfqpoint{1.989526in}{5.627111in}}{\pgfqpoint{1.978927in}{5.631501in}}{\pgfqpoint{1.967877in}{5.631501in}}%
\pgfpathcurveto{\pgfqpoint{1.956827in}{5.631501in}}{\pgfqpoint{1.946228in}{5.627111in}}{\pgfqpoint{1.938414in}{5.619297in}}%
\pgfpathcurveto{\pgfqpoint{1.930600in}{5.611484in}}{\pgfqpoint{1.926210in}{5.600885in}}{\pgfqpoint{1.926210in}{5.589834in}}%
\pgfpathcurveto{\pgfqpoint{1.926210in}{5.578784in}}{\pgfqpoint{1.930600in}{5.568185in}}{\pgfqpoint{1.938414in}{5.560372in}}%
\pgfpathcurveto{\pgfqpoint{1.946228in}{5.552558in}}{\pgfqpoint{1.956827in}{5.548168in}}{\pgfqpoint{1.967877in}{5.548168in}}%
\pgfpathclose%
\pgfusepath{stroke,fill}%
\end{pgfscope}%
\begin{pgfscope}%
\pgfpathrectangle{\pgfqpoint{0.481978in}{0.331635in}}{\pgfqpoint{9.300000in}{7.700000in}}%
\pgfusepath{clip}%
\pgfsetbuttcap%
\pgfsetroundjoin%
\definecolor{currentfill}{rgb}{1.000000,0.623529,0.607843}%
\pgfsetfillcolor{currentfill}%
\pgfsetlinewidth{0.481800pt}%
\definecolor{currentstroke}{rgb}{1.000000,1.000000,1.000000}%
\pgfsetstrokecolor{currentstroke}%
\pgfsetdash{}{0pt}%
\pgfpathmoveto{\pgfqpoint{3.570734in}{2.892749in}}%
\pgfpathcurveto{\pgfqpoint{3.581784in}{2.892749in}}{\pgfqpoint{3.592383in}{2.897139in}}{\pgfqpoint{3.600197in}{2.904953in}}%
\pgfpathcurveto{\pgfqpoint{3.608010in}{2.912766in}}{\pgfqpoint{3.612401in}{2.923365in}}{\pgfqpoint{3.612401in}{2.934416in}}%
\pgfpathcurveto{\pgfqpoint{3.612401in}{2.945466in}}{\pgfqpoint{3.608010in}{2.956065in}}{\pgfqpoint{3.600197in}{2.963878in}}%
\pgfpathcurveto{\pgfqpoint{3.592383in}{2.971692in}}{\pgfqpoint{3.581784in}{2.976082in}}{\pgfqpoint{3.570734in}{2.976082in}}%
\pgfpathcurveto{\pgfqpoint{3.559684in}{2.976082in}}{\pgfqpoint{3.549085in}{2.971692in}}{\pgfqpoint{3.541271in}{2.963878in}}%
\pgfpathcurveto{\pgfqpoint{3.533458in}{2.956065in}}{\pgfqpoint{3.529067in}{2.945466in}}{\pgfqpoint{3.529067in}{2.934416in}}%
\pgfpathcurveto{\pgfqpoint{3.529067in}{2.923365in}}{\pgfqpoint{3.533458in}{2.912766in}}{\pgfqpoint{3.541271in}{2.904953in}}%
\pgfpathcurveto{\pgfqpoint{3.549085in}{2.897139in}}{\pgfqpoint{3.559684in}{2.892749in}}{\pgfqpoint{3.570734in}{2.892749in}}%
\pgfpathclose%
\pgfusepath{stroke,fill}%
\end{pgfscope}%
\begin{pgfscope}%
\pgfpathrectangle{\pgfqpoint{0.481978in}{0.331635in}}{\pgfqpoint{9.300000in}{7.700000in}}%
\pgfusepath{clip}%
\pgfsetbuttcap%
\pgfsetroundjoin%
\definecolor{currentfill}{rgb}{1.000000,0.623529,0.607843}%
\pgfsetfillcolor{currentfill}%
\pgfsetlinewidth{0.481800pt}%
\definecolor{currentstroke}{rgb}{1.000000,1.000000,1.000000}%
\pgfsetstrokecolor{currentstroke}%
\pgfsetdash{}{0pt}%
\pgfpathmoveto{\pgfqpoint{5.661130in}{4.631812in}}%
\pgfpathcurveto{\pgfqpoint{5.672180in}{4.631812in}}{\pgfqpoint{5.682779in}{4.636202in}}{\pgfqpoint{5.690593in}{4.644015in}}%
\pgfpathcurveto{\pgfqpoint{5.698407in}{4.651829in}}{\pgfqpoint{5.702797in}{4.662428in}}{\pgfqpoint{5.702797in}{4.673478in}}%
\pgfpathcurveto{\pgfqpoint{5.702797in}{4.684528in}}{\pgfqpoint{5.698407in}{4.695127in}}{\pgfqpoint{5.690593in}{4.702941in}}%
\pgfpathcurveto{\pgfqpoint{5.682779in}{4.710755in}}{\pgfqpoint{5.672180in}{4.715145in}}{\pgfqpoint{5.661130in}{4.715145in}}%
\pgfpathcurveto{\pgfqpoint{5.650080in}{4.715145in}}{\pgfqpoint{5.639481in}{4.710755in}}{\pgfqpoint{5.631667in}{4.702941in}}%
\pgfpathcurveto{\pgfqpoint{5.623854in}{4.695127in}}{\pgfqpoint{5.619463in}{4.684528in}}{\pgfqpoint{5.619463in}{4.673478in}}%
\pgfpathcurveto{\pgfqpoint{5.619463in}{4.662428in}}{\pgfqpoint{5.623854in}{4.651829in}}{\pgfqpoint{5.631667in}{4.644015in}}%
\pgfpathcurveto{\pgfqpoint{5.639481in}{4.636202in}}{\pgfqpoint{5.650080in}{4.631812in}}{\pgfqpoint{5.661130in}{4.631812in}}%
\pgfpathclose%
\pgfusepath{stroke,fill}%
\end{pgfscope}%
\begin{pgfscope}%
\pgfpathrectangle{\pgfqpoint{0.481978in}{0.331635in}}{\pgfqpoint{9.300000in}{7.700000in}}%
\pgfusepath{clip}%
\pgfsetbuttcap%
\pgfsetroundjoin%
\definecolor{currentfill}{rgb}{1.000000,0.623529,0.607843}%
\pgfsetfillcolor{currentfill}%
\pgfsetlinewidth{0.481800pt}%
\definecolor{currentstroke}{rgb}{1.000000,1.000000,1.000000}%
\pgfsetstrokecolor{currentstroke}%
\pgfsetdash{}{0pt}%
\pgfpathmoveto{\pgfqpoint{4.720327in}{4.321572in}}%
\pgfpathcurveto{\pgfqpoint{4.731377in}{4.321572in}}{\pgfqpoint{4.741976in}{4.325963in}}{\pgfqpoint{4.749790in}{4.333776in}}%
\pgfpathcurveto{\pgfqpoint{4.757604in}{4.341590in}}{\pgfqpoint{4.761994in}{4.352189in}}{\pgfqpoint{4.761994in}{4.363239in}}%
\pgfpathcurveto{\pgfqpoint{4.761994in}{4.374289in}}{\pgfqpoint{4.757604in}{4.384888in}}{\pgfqpoint{4.749790in}{4.392702in}}%
\pgfpathcurveto{\pgfqpoint{4.741976in}{4.400515in}}{\pgfqpoint{4.731377in}{4.404906in}}{\pgfqpoint{4.720327in}{4.404906in}}%
\pgfpathcurveto{\pgfqpoint{4.709277in}{4.404906in}}{\pgfqpoint{4.698678in}{4.400515in}}{\pgfqpoint{4.690864in}{4.392702in}}%
\pgfpathcurveto{\pgfqpoint{4.683051in}{4.384888in}}{\pgfqpoint{4.678660in}{4.374289in}}{\pgfqpoint{4.678660in}{4.363239in}}%
\pgfpathcurveto{\pgfqpoint{4.678660in}{4.352189in}}{\pgfqpoint{4.683051in}{4.341590in}}{\pgfqpoint{4.690864in}{4.333776in}}%
\pgfpathcurveto{\pgfqpoint{4.698678in}{4.325963in}}{\pgfqpoint{4.709277in}{4.321572in}}{\pgfqpoint{4.720327in}{4.321572in}}%
\pgfpathclose%
\pgfusepath{stroke,fill}%
\end{pgfscope}%
\begin{pgfscope}%
\pgfpathrectangle{\pgfqpoint{0.481978in}{0.331635in}}{\pgfqpoint{9.300000in}{7.700000in}}%
\pgfusepath{clip}%
\pgfsetbuttcap%
\pgfsetroundjoin%
\definecolor{currentfill}{rgb}{1.000000,0.623529,0.607843}%
\pgfsetfillcolor{currentfill}%
\pgfsetlinewidth{0.481800pt}%
\definecolor{currentstroke}{rgb}{1.000000,1.000000,1.000000}%
\pgfsetstrokecolor{currentstroke}%
\pgfsetdash{}{0pt}%
\pgfpathmoveto{\pgfqpoint{2.339602in}{4.088730in}}%
\pgfpathcurveto{\pgfqpoint{2.350652in}{4.088730in}}{\pgfqpoint{2.361251in}{4.093120in}}{\pgfqpoint{2.369065in}{4.100934in}}%
\pgfpathcurveto{\pgfqpoint{2.376879in}{4.108747in}}{\pgfqpoint{2.381269in}{4.119346in}}{\pgfqpoint{2.381269in}{4.130396in}}%
\pgfpathcurveto{\pgfqpoint{2.381269in}{4.141446in}}{\pgfqpoint{2.376879in}{4.152045in}}{\pgfqpoint{2.369065in}{4.159859in}}%
\pgfpathcurveto{\pgfqpoint{2.361251in}{4.167673in}}{\pgfqpoint{2.350652in}{4.172063in}}{\pgfqpoint{2.339602in}{4.172063in}}%
\pgfpathcurveto{\pgfqpoint{2.328552in}{4.172063in}}{\pgfqpoint{2.317953in}{4.167673in}}{\pgfqpoint{2.310139in}{4.159859in}}%
\pgfpathcurveto{\pgfqpoint{2.302326in}{4.152045in}}{\pgfqpoint{2.297936in}{4.141446in}}{\pgfqpoint{2.297936in}{4.130396in}}%
\pgfpathcurveto{\pgfqpoint{2.297936in}{4.119346in}}{\pgfqpoint{2.302326in}{4.108747in}}{\pgfqpoint{2.310139in}{4.100934in}}%
\pgfpathcurveto{\pgfqpoint{2.317953in}{4.093120in}}{\pgfqpoint{2.328552in}{4.088730in}}{\pgfqpoint{2.339602in}{4.088730in}}%
\pgfpathclose%
\pgfusepath{stroke,fill}%
\end{pgfscope}%
\begin{pgfscope}%
\pgfpathrectangle{\pgfqpoint{0.481978in}{0.331635in}}{\pgfqpoint{9.300000in}{7.700000in}}%
\pgfusepath{clip}%
\pgfsetbuttcap%
\pgfsetroundjoin%
\definecolor{currentfill}{rgb}{1.000000,0.623529,0.607843}%
\pgfsetfillcolor{currentfill}%
\pgfsetlinewidth{0.481800pt}%
\definecolor{currentstroke}{rgb}{1.000000,1.000000,1.000000}%
\pgfsetstrokecolor{currentstroke}%
\pgfsetdash{}{0pt}%
\pgfpathmoveto{\pgfqpoint{7.776459in}{5.338891in}}%
\pgfpathcurveto{\pgfqpoint{7.787510in}{5.338891in}}{\pgfqpoint{7.798109in}{5.343282in}}{\pgfqpoint{7.805922in}{5.351095in}}%
\pgfpathcurveto{\pgfqpoint{7.813736in}{5.358909in}}{\pgfqpoint{7.818126in}{5.369508in}}{\pgfqpoint{7.818126in}{5.380558in}}%
\pgfpathcurveto{\pgfqpoint{7.818126in}{5.391608in}}{\pgfqpoint{7.813736in}{5.402207in}}{\pgfqpoint{7.805922in}{5.410021in}}%
\pgfpathcurveto{\pgfqpoint{7.798109in}{5.417834in}}{\pgfqpoint{7.787510in}{5.422225in}}{\pgfqpoint{7.776459in}{5.422225in}}%
\pgfpathcurveto{\pgfqpoint{7.765409in}{5.422225in}}{\pgfqpoint{7.754810in}{5.417834in}}{\pgfqpoint{7.746997in}{5.410021in}}%
\pgfpathcurveto{\pgfqpoint{7.739183in}{5.402207in}}{\pgfqpoint{7.734793in}{5.391608in}}{\pgfqpoint{7.734793in}{5.380558in}}%
\pgfpathcurveto{\pgfqpoint{7.734793in}{5.369508in}}{\pgfqpoint{7.739183in}{5.358909in}}{\pgfqpoint{7.746997in}{5.351095in}}%
\pgfpathcurveto{\pgfqpoint{7.754810in}{5.343282in}}{\pgfqpoint{7.765409in}{5.338891in}}{\pgfqpoint{7.776459in}{5.338891in}}%
\pgfpathclose%
\pgfusepath{stroke,fill}%
\end{pgfscope}%
\begin{pgfscope}%
\pgfpathrectangle{\pgfqpoint{0.481978in}{0.331635in}}{\pgfqpoint{9.300000in}{7.700000in}}%
\pgfusepath{clip}%
\pgfsetbuttcap%
\pgfsetroundjoin%
\definecolor{currentfill}{rgb}{1.000000,0.623529,0.607843}%
\pgfsetfillcolor{currentfill}%
\pgfsetlinewidth{0.481800pt}%
\definecolor{currentstroke}{rgb}{1.000000,1.000000,1.000000}%
\pgfsetstrokecolor{currentstroke}%
\pgfsetdash{}{0pt}%
\pgfpathmoveto{\pgfqpoint{5.832034in}{1.465477in}}%
\pgfpathcurveto{\pgfqpoint{5.843084in}{1.465477in}}{\pgfqpoint{5.853683in}{1.469868in}}{\pgfqpoint{5.861497in}{1.477681in}}%
\pgfpathcurveto{\pgfqpoint{5.869310in}{1.485495in}}{\pgfqpoint{5.873701in}{1.496094in}}{\pgfqpoint{5.873701in}{1.507144in}}%
\pgfpathcurveto{\pgfqpoint{5.873701in}{1.518194in}}{\pgfqpoint{5.869310in}{1.528793in}}{\pgfqpoint{5.861497in}{1.536607in}}%
\pgfpathcurveto{\pgfqpoint{5.853683in}{1.544421in}}{\pgfqpoint{5.843084in}{1.548811in}}{\pgfqpoint{5.832034in}{1.548811in}}%
\pgfpathcurveto{\pgfqpoint{5.820984in}{1.548811in}}{\pgfqpoint{5.810385in}{1.544421in}}{\pgfqpoint{5.802571in}{1.536607in}}%
\pgfpathcurveto{\pgfqpoint{5.794757in}{1.528793in}}{\pgfqpoint{5.790367in}{1.518194in}}{\pgfqpoint{5.790367in}{1.507144in}}%
\pgfpathcurveto{\pgfqpoint{5.790367in}{1.496094in}}{\pgfqpoint{5.794757in}{1.485495in}}{\pgfqpoint{5.802571in}{1.477681in}}%
\pgfpathcurveto{\pgfqpoint{5.810385in}{1.469868in}}{\pgfqpoint{5.820984in}{1.465477in}}{\pgfqpoint{5.832034in}{1.465477in}}%
\pgfpathclose%
\pgfusepath{stroke,fill}%
\end{pgfscope}%
\begin{pgfscope}%
\pgfpathrectangle{\pgfqpoint{0.481978in}{0.331635in}}{\pgfqpoint{9.300000in}{7.700000in}}%
\pgfusepath{clip}%
\pgfsetbuttcap%
\pgfsetroundjoin%
\definecolor{currentfill}{rgb}{1.000000,0.623529,0.607843}%
\pgfsetfillcolor{currentfill}%
\pgfsetlinewidth{0.481800pt}%
\definecolor{currentstroke}{rgb}{1.000000,1.000000,1.000000}%
\pgfsetstrokecolor{currentstroke}%
\pgfsetdash{}{0pt}%
\pgfpathmoveto{\pgfqpoint{8.011004in}{6.234740in}}%
\pgfpathcurveto{\pgfqpoint{8.022054in}{6.234740in}}{\pgfqpoint{8.032653in}{6.239130in}}{\pgfqpoint{8.040467in}{6.246944in}}%
\pgfpathcurveto{\pgfqpoint{8.048281in}{6.254757in}}{\pgfqpoint{8.052671in}{6.265356in}}{\pgfqpoint{8.052671in}{6.276407in}}%
\pgfpathcurveto{\pgfqpoint{8.052671in}{6.287457in}}{\pgfqpoint{8.048281in}{6.298056in}}{\pgfqpoint{8.040467in}{6.305869in}}%
\pgfpathcurveto{\pgfqpoint{8.032653in}{6.313683in}}{\pgfqpoint{8.022054in}{6.318073in}}{\pgfqpoint{8.011004in}{6.318073in}}%
\pgfpathcurveto{\pgfqpoint{7.999954in}{6.318073in}}{\pgfqpoint{7.989355in}{6.313683in}}{\pgfqpoint{7.981541in}{6.305869in}}%
\pgfpathcurveto{\pgfqpoint{7.973728in}{6.298056in}}{\pgfqpoint{7.969337in}{6.287457in}}{\pgfqpoint{7.969337in}{6.276407in}}%
\pgfpathcurveto{\pgfqpoint{7.969337in}{6.265356in}}{\pgfqpoint{7.973728in}{6.254757in}}{\pgfqpoint{7.981541in}{6.246944in}}%
\pgfpathcurveto{\pgfqpoint{7.989355in}{6.239130in}}{\pgfqpoint{7.999954in}{6.234740in}}{\pgfqpoint{8.011004in}{6.234740in}}%
\pgfpathclose%
\pgfusepath{stroke,fill}%
\end{pgfscope}%
\begin{pgfscope}%
\pgfpathrectangle{\pgfqpoint{0.481978in}{0.331635in}}{\pgfqpoint{9.300000in}{7.700000in}}%
\pgfusepath{clip}%
\pgfsetbuttcap%
\pgfsetroundjoin%
\definecolor{currentfill}{rgb}{1.000000,0.623529,0.607843}%
\pgfsetfillcolor{currentfill}%
\pgfsetlinewidth{0.481800pt}%
\definecolor{currentstroke}{rgb}{1.000000,1.000000,1.000000}%
\pgfsetstrokecolor{currentstroke}%
\pgfsetdash{}{0pt}%
\pgfpathmoveto{\pgfqpoint{2.908514in}{3.287743in}}%
\pgfpathcurveto{\pgfqpoint{2.919564in}{3.287743in}}{\pgfqpoint{2.930163in}{3.292133in}}{\pgfqpoint{2.937976in}{3.299947in}}%
\pgfpathcurveto{\pgfqpoint{2.945790in}{3.307760in}}{\pgfqpoint{2.950180in}{3.318359in}}{\pgfqpoint{2.950180in}{3.329410in}}%
\pgfpathcurveto{\pgfqpoint{2.950180in}{3.340460in}}{\pgfqpoint{2.945790in}{3.351059in}}{\pgfqpoint{2.937976in}{3.358872in}}%
\pgfpathcurveto{\pgfqpoint{2.930163in}{3.366686in}}{\pgfqpoint{2.919564in}{3.371076in}}{\pgfqpoint{2.908514in}{3.371076in}}%
\pgfpathcurveto{\pgfqpoint{2.897463in}{3.371076in}}{\pgfqpoint{2.886864in}{3.366686in}}{\pgfqpoint{2.879051in}{3.358872in}}%
\pgfpathcurveto{\pgfqpoint{2.871237in}{3.351059in}}{\pgfqpoint{2.866847in}{3.340460in}}{\pgfqpoint{2.866847in}{3.329410in}}%
\pgfpathcurveto{\pgfqpoint{2.866847in}{3.318359in}}{\pgfqpoint{2.871237in}{3.307760in}}{\pgfqpoint{2.879051in}{3.299947in}}%
\pgfpathcurveto{\pgfqpoint{2.886864in}{3.292133in}}{\pgfqpoint{2.897463in}{3.287743in}}{\pgfqpoint{2.908514in}{3.287743in}}%
\pgfpathclose%
\pgfusepath{stroke,fill}%
\end{pgfscope}%
\begin{pgfscope}%
\pgfpathrectangle{\pgfqpoint{0.481978in}{0.331635in}}{\pgfqpoint{9.300000in}{7.700000in}}%
\pgfusepath{clip}%
\pgfsetbuttcap%
\pgfsetroundjoin%
\definecolor{currentfill}{rgb}{1.000000,0.623529,0.607843}%
\pgfsetfillcolor{currentfill}%
\pgfsetlinewidth{0.481800pt}%
\definecolor{currentstroke}{rgb}{1.000000,1.000000,1.000000}%
\pgfsetstrokecolor{currentstroke}%
\pgfsetdash{}{0pt}%
\pgfpathmoveto{\pgfqpoint{6.465338in}{2.324944in}}%
\pgfpathcurveto{\pgfqpoint{6.476388in}{2.324944in}}{\pgfqpoint{6.486987in}{2.329335in}}{\pgfqpoint{6.494801in}{2.337148in}}%
\pgfpathcurveto{\pgfqpoint{6.502614in}{2.344962in}}{\pgfqpoint{6.507004in}{2.355561in}}{\pgfqpoint{6.507004in}{2.366611in}}%
\pgfpathcurveto{\pgfqpoint{6.507004in}{2.377661in}}{\pgfqpoint{6.502614in}{2.388260in}}{\pgfqpoint{6.494801in}{2.396074in}}%
\pgfpathcurveto{\pgfqpoint{6.486987in}{2.403887in}}{\pgfqpoint{6.476388in}{2.408278in}}{\pgfqpoint{6.465338in}{2.408278in}}%
\pgfpathcurveto{\pgfqpoint{6.454288in}{2.408278in}}{\pgfqpoint{6.443689in}{2.403887in}}{\pgfqpoint{6.435875in}{2.396074in}}%
\pgfpathcurveto{\pgfqpoint{6.428061in}{2.388260in}}{\pgfqpoint{6.423671in}{2.377661in}}{\pgfqpoint{6.423671in}{2.366611in}}%
\pgfpathcurveto{\pgfqpoint{6.423671in}{2.355561in}}{\pgfqpoint{6.428061in}{2.344962in}}{\pgfqpoint{6.435875in}{2.337148in}}%
\pgfpathcurveto{\pgfqpoint{6.443689in}{2.329335in}}{\pgfqpoint{6.454288in}{2.324944in}}{\pgfqpoint{6.465338in}{2.324944in}}%
\pgfpathclose%
\pgfusepath{stroke,fill}%
\end{pgfscope}%
\begin{pgfscope}%
\pgfpathrectangle{\pgfqpoint{0.481978in}{0.331635in}}{\pgfqpoint{9.300000in}{7.700000in}}%
\pgfusepath{clip}%
\pgfsetbuttcap%
\pgfsetroundjoin%
\definecolor{currentfill}{rgb}{1.000000,0.623529,0.607843}%
\pgfsetfillcolor{currentfill}%
\pgfsetlinewidth{0.481800pt}%
\definecolor{currentstroke}{rgb}{1.000000,1.000000,1.000000}%
\pgfsetstrokecolor{currentstroke}%
\pgfsetdash{}{0pt}%
\pgfpathmoveto{\pgfqpoint{8.106589in}{6.301680in}}%
\pgfpathcurveto{\pgfqpoint{8.117639in}{6.301680in}}{\pgfqpoint{8.128238in}{6.306070in}}{\pgfqpoint{8.136051in}{6.313883in}}%
\pgfpathcurveto{\pgfqpoint{8.143865in}{6.321697in}}{\pgfqpoint{8.148255in}{6.332296in}}{\pgfqpoint{8.148255in}{6.343346in}}%
\pgfpathcurveto{\pgfqpoint{8.148255in}{6.354396in}}{\pgfqpoint{8.143865in}{6.364995in}}{\pgfqpoint{8.136051in}{6.372809in}}%
\pgfpathcurveto{\pgfqpoint{8.128238in}{6.380623in}}{\pgfqpoint{8.117639in}{6.385013in}}{\pgfqpoint{8.106589in}{6.385013in}}%
\pgfpathcurveto{\pgfqpoint{8.095538in}{6.385013in}}{\pgfqpoint{8.084939in}{6.380623in}}{\pgfqpoint{8.077126in}{6.372809in}}%
\pgfpathcurveto{\pgfqpoint{8.069312in}{6.364995in}}{\pgfqpoint{8.064922in}{6.354396in}}{\pgfqpoint{8.064922in}{6.343346in}}%
\pgfpathcurveto{\pgfqpoint{8.064922in}{6.332296in}}{\pgfqpoint{8.069312in}{6.321697in}}{\pgfqpoint{8.077126in}{6.313883in}}%
\pgfpathcurveto{\pgfqpoint{8.084939in}{6.306070in}}{\pgfqpoint{8.095538in}{6.301680in}}{\pgfqpoint{8.106589in}{6.301680in}}%
\pgfpathclose%
\pgfusepath{stroke,fill}%
\end{pgfscope}%
\begin{pgfscope}%
\pgfpathrectangle{\pgfqpoint{0.481978in}{0.331635in}}{\pgfqpoint{9.300000in}{7.700000in}}%
\pgfusepath{clip}%
\pgfsetbuttcap%
\pgfsetroundjoin%
\definecolor{currentfill}{rgb}{1.000000,0.623529,0.607843}%
\pgfsetfillcolor{currentfill}%
\pgfsetlinewidth{0.481800pt}%
\definecolor{currentstroke}{rgb}{1.000000,1.000000,1.000000}%
\pgfsetstrokecolor{currentstroke}%
\pgfsetdash{}{0pt}%
\pgfpathmoveto{\pgfqpoint{7.991875in}{5.222385in}}%
\pgfpathcurveto{\pgfqpoint{8.002925in}{5.222385in}}{\pgfqpoint{8.013524in}{5.226776in}}{\pgfqpoint{8.021338in}{5.234589in}}%
\pgfpathcurveto{\pgfqpoint{8.029151in}{5.242403in}}{\pgfqpoint{8.033541in}{5.253002in}}{\pgfqpoint{8.033541in}{5.264052in}}%
\pgfpathcurveto{\pgfqpoint{8.033541in}{5.275102in}}{\pgfqpoint{8.029151in}{5.285701in}}{\pgfqpoint{8.021338in}{5.293515in}}%
\pgfpathcurveto{\pgfqpoint{8.013524in}{5.301328in}}{\pgfqpoint{8.002925in}{5.305719in}}{\pgfqpoint{7.991875in}{5.305719in}}%
\pgfpathcurveto{\pgfqpoint{7.980825in}{5.305719in}}{\pgfqpoint{7.970226in}{5.301328in}}{\pgfqpoint{7.962412in}{5.293515in}}%
\pgfpathcurveto{\pgfqpoint{7.954598in}{5.285701in}}{\pgfqpoint{7.950208in}{5.275102in}}{\pgfqpoint{7.950208in}{5.264052in}}%
\pgfpathcurveto{\pgfqpoint{7.950208in}{5.253002in}}{\pgfqpoint{7.954598in}{5.242403in}}{\pgfqpoint{7.962412in}{5.234589in}}%
\pgfpathcurveto{\pgfqpoint{7.970226in}{5.226776in}}{\pgfqpoint{7.980825in}{5.222385in}}{\pgfqpoint{7.991875in}{5.222385in}}%
\pgfpathclose%
\pgfusepath{stroke,fill}%
\end{pgfscope}%
\begin{pgfscope}%
\pgfpathrectangle{\pgfqpoint{0.481978in}{0.331635in}}{\pgfqpoint{9.300000in}{7.700000in}}%
\pgfusepath{clip}%
\pgfsetbuttcap%
\pgfsetroundjoin%
\definecolor{currentfill}{rgb}{1.000000,0.623529,0.607843}%
\pgfsetfillcolor{currentfill}%
\pgfsetlinewidth{0.481800pt}%
\definecolor{currentstroke}{rgb}{1.000000,1.000000,1.000000}%
\pgfsetstrokecolor{currentstroke}%
\pgfsetdash{}{0pt}%
\pgfpathmoveto{\pgfqpoint{7.423833in}{5.532170in}}%
\pgfpathcurveto{\pgfqpoint{7.434883in}{5.532170in}}{\pgfqpoint{7.445482in}{5.536561in}}{\pgfqpoint{7.453296in}{5.544374in}}%
\pgfpathcurveto{\pgfqpoint{7.461110in}{5.552188in}}{\pgfqpoint{7.465500in}{5.562787in}}{\pgfqpoint{7.465500in}{5.573837in}}%
\pgfpathcurveto{\pgfqpoint{7.465500in}{5.584887in}}{\pgfqpoint{7.461110in}{5.595486in}}{\pgfqpoint{7.453296in}{5.603300in}}%
\pgfpathcurveto{\pgfqpoint{7.445482in}{5.611113in}}{\pgfqpoint{7.434883in}{5.615504in}}{\pgfqpoint{7.423833in}{5.615504in}}%
\pgfpathcurveto{\pgfqpoint{7.412783in}{5.615504in}}{\pgfqpoint{7.402184in}{5.611113in}}{\pgfqpoint{7.394371in}{5.603300in}}%
\pgfpathcurveto{\pgfqpoint{7.386557in}{5.595486in}}{\pgfqpoint{7.382167in}{5.584887in}}{\pgfqpoint{7.382167in}{5.573837in}}%
\pgfpathcurveto{\pgfqpoint{7.382167in}{5.562787in}}{\pgfqpoint{7.386557in}{5.552188in}}{\pgfqpoint{7.394371in}{5.544374in}}%
\pgfpathcurveto{\pgfqpoint{7.402184in}{5.536561in}}{\pgfqpoint{7.412783in}{5.532170in}}{\pgfqpoint{7.423833in}{5.532170in}}%
\pgfpathclose%
\pgfusepath{stroke,fill}%
\end{pgfscope}%
\begin{pgfscope}%
\pgfpathrectangle{\pgfqpoint{0.481978in}{0.331635in}}{\pgfqpoint{9.300000in}{7.700000in}}%
\pgfusepath{clip}%
\pgfsetbuttcap%
\pgfsetroundjoin%
\definecolor{currentfill}{rgb}{1.000000,0.623529,0.607843}%
\pgfsetfillcolor{currentfill}%
\pgfsetlinewidth{0.481800pt}%
\definecolor{currentstroke}{rgb}{1.000000,1.000000,1.000000}%
\pgfsetstrokecolor{currentstroke}%
\pgfsetdash{}{0pt}%
\pgfpathmoveto{\pgfqpoint{4.192970in}{5.909242in}}%
\pgfpathcurveto{\pgfqpoint{4.204020in}{5.909242in}}{\pgfqpoint{4.214619in}{5.913632in}}{\pgfqpoint{4.222433in}{5.921445in}}%
\pgfpathcurveto{\pgfqpoint{4.230247in}{5.929259in}}{\pgfqpoint{4.234637in}{5.939858in}}{\pgfqpoint{4.234637in}{5.950908in}}%
\pgfpathcurveto{\pgfqpoint{4.234637in}{5.961958in}}{\pgfqpoint{4.230247in}{5.972557in}}{\pgfqpoint{4.222433in}{5.980371in}}%
\pgfpathcurveto{\pgfqpoint{4.214619in}{5.988185in}}{\pgfqpoint{4.204020in}{5.992575in}}{\pgfqpoint{4.192970in}{5.992575in}}%
\pgfpathcurveto{\pgfqpoint{4.181920in}{5.992575in}}{\pgfqpoint{4.171321in}{5.988185in}}{\pgfqpoint{4.163508in}{5.980371in}}%
\pgfpathcurveto{\pgfqpoint{4.155694in}{5.972557in}}{\pgfqpoint{4.151304in}{5.961958in}}{\pgfqpoint{4.151304in}{5.950908in}}%
\pgfpathcurveto{\pgfqpoint{4.151304in}{5.939858in}}{\pgfqpoint{4.155694in}{5.929259in}}{\pgfqpoint{4.163508in}{5.921445in}}%
\pgfpathcurveto{\pgfqpoint{4.171321in}{5.913632in}}{\pgfqpoint{4.181920in}{5.909242in}}{\pgfqpoint{4.192970in}{5.909242in}}%
\pgfpathclose%
\pgfusepath{stroke,fill}%
\end{pgfscope}%
\begin{pgfscope}%
\pgfpathrectangle{\pgfqpoint{0.481978in}{0.331635in}}{\pgfqpoint{9.300000in}{7.700000in}}%
\pgfusepath{clip}%
\pgfsetbuttcap%
\pgfsetroundjoin%
\definecolor{currentfill}{rgb}{1.000000,0.623529,0.607843}%
\pgfsetfillcolor{currentfill}%
\pgfsetlinewidth{0.481800pt}%
\definecolor{currentstroke}{rgb}{1.000000,1.000000,1.000000}%
\pgfsetstrokecolor{currentstroke}%
\pgfsetdash{}{0pt}%
\pgfpathmoveto{\pgfqpoint{6.770367in}{5.447632in}}%
\pgfpathcurveto{\pgfqpoint{6.781417in}{5.447632in}}{\pgfqpoint{6.792016in}{5.452022in}}{\pgfqpoint{6.799829in}{5.459836in}}%
\pgfpathcurveto{\pgfqpoint{6.807643in}{5.467649in}}{\pgfqpoint{6.812033in}{5.478248in}}{\pgfqpoint{6.812033in}{5.489298in}}%
\pgfpathcurveto{\pgfqpoint{6.812033in}{5.500349in}}{\pgfqpoint{6.807643in}{5.510948in}}{\pgfqpoint{6.799829in}{5.518761in}}%
\pgfpathcurveto{\pgfqpoint{6.792016in}{5.526575in}}{\pgfqpoint{6.781417in}{5.530965in}}{\pgfqpoint{6.770367in}{5.530965in}}%
\pgfpathcurveto{\pgfqpoint{6.759317in}{5.530965in}}{\pgfqpoint{6.748718in}{5.526575in}}{\pgfqpoint{6.740904in}{5.518761in}}%
\pgfpathcurveto{\pgfqpoint{6.733090in}{5.510948in}}{\pgfqpoint{6.728700in}{5.500349in}}{\pgfqpoint{6.728700in}{5.489298in}}%
\pgfpathcurveto{\pgfqpoint{6.728700in}{5.478248in}}{\pgfqpoint{6.733090in}{5.467649in}}{\pgfqpoint{6.740904in}{5.459836in}}%
\pgfpathcurveto{\pgfqpoint{6.748718in}{5.452022in}}{\pgfqpoint{6.759317in}{5.447632in}}{\pgfqpoint{6.770367in}{5.447632in}}%
\pgfpathclose%
\pgfusepath{stroke,fill}%
\end{pgfscope}%
\begin{pgfscope}%
\pgfpathrectangle{\pgfqpoint{0.481978in}{0.331635in}}{\pgfqpoint{9.300000in}{7.700000in}}%
\pgfusepath{clip}%
\pgfsetbuttcap%
\pgfsetroundjoin%
\definecolor{currentfill}{rgb}{1.000000,0.623529,0.607843}%
\pgfsetfillcolor{currentfill}%
\pgfsetlinewidth{0.481800pt}%
\definecolor{currentstroke}{rgb}{1.000000,1.000000,1.000000}%
\pgfsetstrokecolor{currentstroke}%
\pgfsetdash{}{0pt}%
\pgfpathmoveto{\pgfqpoint{7.446866in}{5.657018in}}%
\pgfpathcurveto{\pgfqpoint{7.457917in}{5.657018in}}{\pgfqpoint{7.468516in}{5.661409in}}{\pgfqpoint{7.476329in}{5.669222in}}%
\pgfpathcurveto{\pgfqpoint{7.484143in}{5.677036in}}{\pgfqpoint{7.488533in}{5.687635in}}{\pgfqpoint{7.488533in}{5.698685in}}%
\pgfpathcurveto{\pgfqpoint{7.488533in}{5.709735in}}{\pgfqpoint{7.484143in}{5.720334in}}{\pgfqpoint{7.476329in}{5.728148in}}%
\pgfpathcurveto{\pgfqpoint{7.468516in}{5.735961in}}{\pgfqpoint{7.457917in}{5.740352in}}{\pgfqpoint{7.446866in}{5.740352in}}%
\pgfpathcurveto{\pgfqpoint{7.435816in}{5.740352in}}{\pgfqpoint{7.425217in}{5.735961in}}{\pgfqpoint{7.417404in}{5.728148in}}%
\pgfpathcurveto{\pgfqpoint{7.409590in}{5.720334in}}{\pgfqpoint{7.405200in}{5.709735in}}{\pgfqpoint{7.405200in}{5.698685in}}%
\pgfpathcurveto{\pgfqpoint{7.405200in}{5.687635in}}{\pgfqpoint{7.409590in}{5.677036in}}{\pgfqpoint{7.417404in}{5.669222in}}%
\pgfpathcurveto{\pgfqpoint{7.425217in}{5.661409in}}{\pgfqpoint{7.435816in}{5.657018in}}{\pgfqpoint{7.446866in}{5.657018in}}%
\pgfpathclose%
\pgfusepath{stroke,fill}%
\end{pgfscope}%
\begin{pgfscope}%
\pgfpathrectangle{\pgfqpoint{0.481978in}{0.331635in}}{\pgfqpoint{9.300000in}{7.700000in}}%
\pgfusepath{clip}%
\pgfsetbuttcap%
\pgfsetroundjoin%
\definecolor{currentfill}{rgb}{1.000000,0.623529,0.607843}%
\pgfsetfillcolor{currentfill}%
\pgfsetlinewidth{0.481800pt}%
\definecolor{currentstroke}{rgb}{1.000000,1.000000,1.000000}%
\pgfsetstrokecolor{currentstroke}%
\pgfsetdash{}{0pt}%
\pgfpathmoveto{\pgfqpoint{4.422769in}{4.131207in}}%
\pgfpathcurveto{\pgfqpoint{4.433819in}{4.131207in}}{\pgfqpoint{4.444418in}{4.135598in}}{\pgfqpoint{4.452231in}{4.143411in}}%
\pgfpathcurveto{\pgfqpoint{4.460045in}{4.151225in}}{\pgfqpoint{4.464435in}{4.161824in}}{\pgfqpoint{4.464435in}{4.172874in}}%
\pgfpathcurveto{\pgfqpoint{4.464435in}{4.183924in}}{\pgfqpoint{4.460045in}{4.194523in}}{\pgfqpoint{4.452231in}{4.202337in}}%
\pgfpathcurveto{\pgfqpoint{4.444418in}{4.210150in}}{\pgfqpoint{4.433819in}{4.214541in}}{\pgfqpoint{4.422769in}{4.214541in}}%
\pgfpathcurveto{\pgfqpoint{4.411719in}{4.214541in}}{\pgfqpoint{4.401120in}{4.210150in}}{\pgfqpoint{4.393306in}{4.202337in}}%
\pgfpathcurveto{\pgfqpoint{4.385492in}{4.194523in}}{\pgfqpoint{4.381102in}{4.183924in}}{\pgfqpoint{4.381102in}{4.172874in}}%
\pgfpathcurveto{\pgfqpoint{4.381102in}{4.161824in}}{\pgfqpoint{4.385492in}{4.151225in}}{\pgfqpoint{4.393306in}{4.143411in}}%
\pgfpathcurveto{\pgfqpoint{4.401120in}{4.135598in}}{\pgfqpoint{4.411719in}{4.131207in}}{\pgfqpoint{4.422769in}{4.131207in}}%
\pgfpathclose%
\pgfusepath{stroke,fill}%
\end{pgfscope}%
\begin{pgfscope}%
\pgfpathrectangle{\pgfqpoint{0.481978in}{0.331635in}}{\pgfqpoint{9.300000in}{7.700000in}}%
\pgfusepath{clip}%
\pgfsetbuttcap%
\pgfsetroundjoin%
\definecolor{currentfill}{rgb}{1.000000,0.623529,0.607843}%
\pgfsetfillcolor{currentfill}%
\pgfsetlinewidth{0.481800pt}%
\definecolor{currentstroke}{rgb}{1.000000,1.000000,1.000000}%
\pgfsetstrokecolor{currentstroke}%
\pgfsetdash{}{0pt}%
\pgfpathmoveto{\pgfqpoint{4.632997in}{1.454063in}}%
\pgfpathcurveto{\pgfqpoint{4.644047in}{1.454063in}}{\pgfqpoint{4.654646in}{1.458453in}}{\pgfqpoint{4.662460in}{1.466266in}}%
\pgfpathcurveto{\pgfqpoint{4.670273in}{1.474080in}}{\pgfqpoint{4.674663in}{1.484679in}}{\pgfqpoint{4.674663in}{1.495729in}}%
\pgfpathcurveto{\pgfqpoint{4.674663in}{1.506779in}}{\pgfqpoint{4.670273in}{1.517378in}}{\pgfqpoint{4.662460in}{1.525192in}}%
\pgfpathcurveto{\pgfqpoint{4.654646in}{1.533006in}}{\pgfqpoint{4.644047in}{1.537396in}}{\pgfqpoint{4.632997in}{1.537396in}}%
\pgfpathcurveto{\pgfqpoint{4.621947in}{1.537396in}}{\pgfqpoint{4.611348in}{1.533006in}}{\pgfqpoint{4.603534in}{1.525192in}}%
\pgfpathcurveto{\pgfqpoint{4.595720in}{1.517378in}}{\pgfqpoint{4.591330in}{1.506779in}}{\pgfqpoint{4.591330in}{1.495729in}}%
\pgfpathcurveto{\pgfqpoint{4.591330in}{1.484679in}}{\pgfqpoint{4.595720in}{1.474080in}}{\pgfqpoint{4.603534in}{1.466266in}}%
\pgfpathcurveto{\pgfqpoint{4.611348in}{1.458453in}}{\pgfqpoint{4.621947in}{1.454063in}}{\pgfqpoint{4.632997in}{1.454063in}}%
\pgfpathclose%
\pgfusepath{stroke,fill}%
\end{pgfscope}%
\begin{pgfscope}%
\pgfpathrectangle{\pgfqpoint{0.481978in}{0.331635in}}{\pgfqpoint{9.300000in}{7.700000in}}%
\pgfusepath{clip}%
\pgfsetbuttcap%
\pgfsetroundjoin%
\definecolor{currentfill}{rgb}{1.000000,0.623529,0.607843}%
\pgfsetfillcolor{currentfill}%
\pgfsetlinewidth{0.481800pt}%
\definecolor{currentstroke}{rgb}{1.000000,1.000000,1.000000}%
\pgfsetstrokecolor{currentstroke}%
\pgfsetdash{}{0pt}%
\pgfpathmoveto{\pgfqpoint{3.319473in}{2.303496in}}%
\pgfpathcurveto{\pgfqpoint{3.330523in}{2.303496in}}{\pgfqpoint{3.341122in}{2.307886in}}{\pgfqpoint{3.348936in}{2.315699in}}%
\pgfpathcurveto{\pgfqpoint{3.356750in}{2.323513in}}{\pgfqpoint{3.361140in}{2.334112in}}{\pgfqpoint{3.361140in}{2.345162in}}%
\pgfpathcurveto{\pgfqpoint{3.361140in}{2.356212in}}{\pgfqpoint{3.356750in}{2.366811in}}{\pgfqpoint{3.348936in}{2.374625in}}%
\pgfpathcurveto{\pgfqpoint{3.341122in}{2.382439in}}{\pgfqpoint{3.330523in}{2.386829in}}{\pgfqpoint{3.319473in}{2.386829in}}%
\pgfpathcurveto{\pgfqpoint{3.308423in}{2.386829in}}{\pgfqpoint{3.297824in}{2.382439in}}{\pgfqpoint{3.290010in}{2.374625in}}%
\pgfpathcurveto{\pgfqpoint{3.282197in}{2.366811in}}{\pgfqpoint{3.277807in}{2.356212in}}{\pgfqpoint{3.277807in}{2.345162in}}%
\pgfpathcurveto{\pgfqpoint{3.277807in}{2.334112in}}{\pgfqpoint{3.282197in}{2.323513in}}{\pgfqpoint{3.290010in}{2.315699in}}%
\pgfpathcurveto{\pgfqpoint{3.297824in}{2.307886in}}{\pgfqpoint{3.308423in}{2.303496in}}{\pgfqpoint{3.319473in}{2.303496in}}%
\pgfpathclose%
\pgfusepath{stroke,fill}%
\end{pgfscope}%
\begin{pgfscope}%
\pgfpathrectangle{\pgfqpoint{0.481978in}{0.331635in}}{\pgfqpoint{9.300000in}{7.700000in}}%
\pgfusepath{clip}%
\pgfsetbuttcap%
\pgfsetroundjoin%
\definecolor{currentfill}{rgb}{1.000000,0.623529,0.607843}%
\pgfsetfillcolor{currentfill}%
\pgfsetlinewidth{0.481800pt}%
\definecolor{currentstroke}{rgb}{1.000000,1.000000,1.000000}%
\pgfsetstrokecolor{currentstroke}%
\pgfsetdash{}{0pt}%
\pgfpathmoveto{\pgfqpoint{3.661029in}{2.484883in}}%
\pgfpathcurveto{\pgfqpoint{3.672079in}{2.484883in}}{\pgfqpoint{3.682678in}{2.489274in}}{\pgfqpoint{3.690491in}{2.497087in}}%
\pgfpathcurveto{\pgfqpoint{3.698305in}{2.504901in}}{\pgfqpoint{3.702695in}{2.515500in}}{\pgfqpoint{3.702695in}{2.526550in}}%
\pgfpathcurveto{\pgfqpoint{3.702695in}{2.537600in}}{\pgfqpoint{3.698305in}{2.548199in}}{\pgfqpoint{3.690491in}{2.556013in}}%
\pgfpathcurveto{\pgfqpoint{3.682678in}{2.563826in}}{\pgfqpoint{3.672079in}{2.568217in}}{\pgfqpoint{3.661029in}{2.568217in}}%
\pgfpathcurveto{\pgfqpoint{3.649978in}{2.568217in}}{\pgfqpoint{3.639379in}{2.563826in}}{\pgfqpoint{3.631566in}{2.556013in}}%
\pgfpathcurveto{\pgfqpoint{3.623752in}{2.548199in}}{\pgfqpoint{3.619362in}{2.537600in}}{\pgfqpoint{3.619362in}{2.526550in}}%
\pgfpathcurveto{\pgfqpoint{3.619362in}{2.515500in}}{\pgfqpoint{3.623752in}{2.504901in}}{\pgfqpoint{3.631566in}{2.497087in}}%
\pgfpathcurveto{\pgfqpoint{3.639379in}{2.489274in}}{\pgfqpoint{3.649978in}{2.484883in}}{\pgfqpoint{3.661029in}{2.484883in}}%
\pgfpathclose%
\pgfusepath{stroke,fill}%
\end{pgfscope}%
\begin{pgfscope}%
\pgfpathrectangle{\pgfqpoint{0.481978in}{0.331635in}}{\pgfqpoint{9.300000in}{7.700000in}}%
\pgfusepath{clip}%
\pgfsetbuttcap%
\pgfsetroundjoin%
\definecolor{currentfill}{rgb}{1.000000,0.623529,0.607843}%
\pgfsetfillcolor{currentfill}%
\pgfsetlinewidth{0.481800pt}%
\definecolor{currentstroke}{rgb}{1.000000,1.000000,1.000000}%
\pgfsetstrokecolor{currentstroke}%
\pgfsetdash{}{0pt}%
\pgfpathmoveto{\pgfqpoint{3.845381in}{3.202561in}}%
\pgfpathcurveto{\pgfqpoint{3.856431in}{3.202561in}}{\pgfqpoint{3.867030in}{3.206951in}}{\pgfqpoint{3.874844in}{3.214765in}}%
\pgfpathcurveto{\pgfqpoint{3.882657in}{3.222578in}}{\pgfqpoint{3.887048in}{3.233177in}}{\pgfqpoint{3.887048in}{3.244228in}}%
\pgfpathcurveto{\pgfqpoint{3.887048in}{3.255278in}}{\pgfqpoint{3.882657in}{3.265877in}}{\pgfqpoint{3.874844in}{3.273690in}}%
\pgfpathcurveto{\pgfqpoint{3.867030in}{3.281504in}}{\pgfqpoint{3.856431in}{3.285894in}}{\pgfqpoint{3.845381in}{3.285894in}}%
\pgfpathcurveto{\pgfqpoint{3.834331in}{3.285894in}}{\pgfqpoint{3.823732in}{3.281504in}}{\pgfqpoint{3.815918in}{3.273690in}}%
\pgfpathcurveto{\pgfqpoint{3.808105in}{3.265877in}}{\pgfqpoint{3.803714in}{3.255278in}}{\pgfqpoint{3.803714in}{3.244228in}}%
\pgfpathcurveto{\pgfqpoint{3.803714in}{3.233177in}}{\pgfqpoint{3.808105in}{3.222578in}}{\pgfqpoint{3.815918in}{3.214765in}}%
\pgfpathcurveto{\pgfqpoint{3.823732in}{3.206951in}}{\pgfqpoint{3.834331in}{3.202561in}}{\pgfqpoint{3.845381in}{3.202561in}}%
\pgfpathclose%
\pgfusepath{stroke,fill}%
\end{pgfscope}%
\begin{pgfscope}%
\pgfpathrectangle{\pgfqpoint{0.481978in}{0.331635in}}{\pgfqpoint{9.300000in}{7.700000in}}%
\pgfusepath{clip}%
\pgfsetbuttcap%
\pgfsetroundjoin%
\definecolor{currentfill}{rgb}{1.000000,0.623529,0.607843}%
\pgfsetfillcolor{currentfill}%
\pgfsetlinewidth{0.481800pt}%
\definecolor{currentstroke}{rgb}{1.000000,1.000000,1.000000}%
\pgfsetstrokecolor{currentstroke}%
\pgfsetdash{}{0pt}%
\pgfpathmoveto{\pgfqpoint{4.050059in}{3.494463in}}%
\pgfpathcurveto{\pgfqpoint{4.061109in}{3.494463in}}{\pgfqpoint{4.071708in}{3.498853in}}{\pgfqpoint{4.079522in}{3.506667in}}%
\pgfpathcurveto{\pgfqpoint{4.087336in}{3.514480in}}{\pgfqpoint{4.091726in}{3.525079in}}{\pgfqpoint{4.091726in}{3.536130in}}%
\pgfpathcurveto{\pgfqpoint{4.091726in}{3.547180in}}{\pgfqpoint{4.087336in}{3.557779in}}{\pgfqpoint{4.079522in}{3.565592in}}%
\pgfpathcurveto{\pgfqpoint{4.071708in}{3.573406in}}{\pgfqpoint{4.061109in}{3.577796in}}{\pgfqpoint{4.050059in}{3.577796in}}%
\pgfpathcurveto{\pgfqpoint{4.039009in}{3.577796in}}{\pgfqpoint{4.028410in}{3.573406in}}{\pgfqpoint{4.020596in}{3.565592in}}%
\pgfpathcurveto{\pgfqpoint{4.012783in}{3.557779in}}{\pgfqpoint{4.008393in}{3.547180in}}{\pgfqpoint{4.008393in}{3.536130in}}%
\pgfpathcurveto{\pgfqpoint{4.008393in}{3.525079in}}{\pgfqpoint{4.012783in}{3.514480in}}{\pgfqpoint{4.020596in}{3.506667in}}%
\pgfpathcurveto{\pgfqpoint{4.028410in}{3.498853in}}{\pgfqpoint{4.039009in}{3.494463in}}{\pgfqpoint{4.050059in}{3.494463in}}%
\pgfpathclose%
\pgfusepath{stroke,fill}%
\end{pgfscope}%
\begin{pgfscope}%
\pgfpathrectangle{\pgfqpoint{0.481978in}{0.331635in}}{\pgfqpoint{9.300000in}{7.700000in}}%
\pgfusepath{clip}%
\pgfsetbuttcap%
\pgfsetroundjoin%
\definecolor{currentfill}{rgb}{1.000000,0.623529,0.607843}%
\pgfsetfillcolor{currentfill}%
\pgfsetlinewidth{0.481800pt}%
\definecolor{currentstroke}{rgb}{1.000000,1.000000,1.000000}%
\pgfsetstrokecolor{currentstroke}%
\pgfsetdash{}{0pt}%
\pgfpathmoveto{\pgfqpoint{4.925237in}{7.423806in}}%
\pgfpathcurveto{\pgfqpoint{4.936287in}{7.423806in}}{\pgfqpoint{4.946887in}{7.428197in}}{\pgfqpoint{4.954700in}{7.436010in}}%
\pgfpathcurveto{\pgfqpoint{4.962514in}{7.443824in}}{\pgfqpoint{4.966904in}{7.454423in}}{\pgfqpoint{4.966904in}{7.465473in}}%
\pgfpathcurveto{\pgfqpoint{4.966904in}{7.476523in}}{\pgfqpoint{4.962514in}{7.487122in}}{\pgfqpoint{4.954700in}{7.494936in}}%
\pgfpathcurveto{\pgfqpoint{4.946887in}{7.502749in}}{\pgfqpoint{4.936287in}{7.507140in}}{\pgfqpoint{4.925237in}{7.507140in}}%
\pgfpathcurveto{\pgfqpoint{4.914187in}{7.507140in}}{\pgfqpoint{4.903588in}{7.502749in}}{\pgfqpoint{4.895775in}{7.494936in}}%
\pgfpathcurveto{\pgfqpoint{4.887961in}{7.487122in}}{\pgfqpoint{4.883571in}{7.476523in}}{\pgfqpoint{4.883571in}{7.465473in}}%
\pgfpathcurveto{\pgfqpoint{4.883571in}{7.454423in}}{\pgfqpoint{4.887961in}{7.443824in}}{\pgfqpoint{4.895775in}{7.436010in}}%
\pgfpathcurveto{\pgfqpoint{4.903588in}{7.428197in}}{\pgfqpoint{4.914187in}{7.423806in}}{\pgfqpoint{4.925237in}{7.423806in}}%
\pgfpathclose%
\pgfusepath{stroke,fill}%
\end{pgfscope}%
\begin{pgfscope}%
\pgfpathrectangle{\pgfqpoint{0.481978in}{0.331635in}}{\pgfqpoint{9.300000in}{7.700000in}}%
\pgfusepath{clip}%
\pgfsetbuttcap%
\pgfsetroundjoin%
\definecolor{currentfill}{rgb}{1.000000,0.623529,0.607843}%
\pgfsetfillcolor{currentfill}%
\pgfsetlinewidth{0.481800pt}%
\definecolor{currentstroke}{rgb}{1.000000,1.000000,1.000000}%
\pgfsetstrokecolor{currentstroke}%
\pgfsetdash{}{0pt}%
\pgfpathmoveto{\pgfqpoint{6.991308in}{3.546324in}}%
\pgfpathcurveto{\pgfqpoint{7.002358in}{3.546324in}}{\pgfqpoint{7.012957in}{3.550714in}}{\pgfqpoint{7.020771in}{3.558528in}}%
\pgfpathcurveto{\pgfqpoint{7.028584in}{3.566341in}}{\pgfqpoint{7.032975in}{3.576940in}}{\pgfqpoint{7.032975in}{3.587991in}}%
\pgfpathcurveto{\pgfqpoint{7.032975in}{3.599041in}}{\pgfqpoint{7.028584in}{3.609640in}}{\pgfqpoint{7.020771in}{3.617453in}}%
\pgfpathcurveto{\pgfqpoint{7.012957in}{3.625267in}}{\pgfqpoint{7.002358in}{3.629657in}}{\pgfqpoint{6.991308in}{3.629657in}}%
\pgfpathcurveto{\pgfqpoint{6.980258in}{3.629657in}}{\pgfqpoint{6.969659in}{3.625267in}}{\pgfqpoint{6.961845in}{3.617453in}}%
\pgfpathcurveto{\pgfqpoint{6.954032in}{3.609640in}}{\pgfqpoint{6.949641in}{3.599041in}}{\pgfqpoint{6.949641in}{3.587991in}}%
\pgfpathcurveto{\pgfqpoint{6.949641in}{3.576940in}}{\pgfqpoint{6.954032in}{3.566341in}}{\pgfqpoint{6.961845in}{3.558528in}}%
\pgfpathcurveto{\pgfqpoint{6.969659in}{3.550714in}}{\pgfqpoint{6.980258in}{3.546324in}}{\pgfqpoint{6.991308in}{3.546324in}}%
\pgfpathclose%
\pgfusepath{stroke,fill}%
\end{pgfscope}%
\begin{pgfscope}%
\pgfpathrectangle{\pgfqpoint{0.481978in}{0.331635in}}{\pgfqpoint{9.300000in}{7.700000in}}%
\pgfusepath{clip}%
\pgfsetbuttcap%
\pgfsetroundjoin%
\definecolor{currentfill}{rgb}{1.000000,0.623529,0.607843}%
\pgfsetfillcolor{currentfill}%
\pgfsetlinewidth{0.481800pt}%
\definecolor{currentstroke}{rgb}{1.000000,1.000000,1.000000}%
\pgfsetstrokecolor{currentstroke}%
\pgfsetdash{}{0pt}%
\pgfpathmoveto{\pgfqpoint{8.618470in}{6.020902in}}%
\pgfpathcurveto{\pgfqpoint{8.629520in}{6.020902in}}{\pgfqpoint{8.640119in}{6.025293in}}{\pgfqpoint{8.647933in}{6.033106in}}%
\pgfpathcurveto{\pgfqpoint{8.655747in}{6.040920in}}{\pgfqpoint{8.660137in}{6.051519in}}{\pgfqpoint{8.660137in}{6.062569in}}%
\pgfpathcurveto{\pgfqpoint{8.660137in}{6.073619in}}{\pgfqpoint{8.655747in}{6.084218in}}{\pgfqpoint{8.647933in}{6.092032in}}%
\pgfpathcurveto{\pgfqpoint{8.640119in}{6.099845in}}{\pgfqpoint{8.629520in}{6.104236in}}{\pgfqpoint{8.618470in}{6.104236in}}%
\pgfpathcurveto{\pgfqpoint{8.607420in}{6.104236in}}{\pgfqpoint{8.596821in}{6.099845in}}{\pgfqpoint{8.589007in}{6.092032in}}%
\pgfpathcurveto{\pgfqpoint{8.581194in}{6.084218in}}{\pgfqpoint{8.576804in}{6.073619in}}{\pgfqpoint{8.576804in}{6.062569in}}%
\pgfpathcurveto{\pgfqpoint{8.576804in}{6.051519in}}{\pgfqpoint{8.581194in}{6.040920in}}{\pgfqpoint{8.589007in}{6.033106in}}%
\pgfpathcurveto{\pgfqpoint{8.596821in}{6.025293in}}{\pgfqpoint{8.607420in}{6.020902in}}{\pgfqpoint{8.618470in}{6.020902in}}%
\pgfpathclose%
\pgfusepath{stroke,fill}%
\end{pgfscope}%
\begin{pgfscope}%
\pgfpathrectangle{\pgfqpoint{0.481978in}{0.331635in}}{\pgfqpoint{9.300000in}{7.700000in}}%
\pgfusepath{clip}%
\pgfsetbuttcap%
\pgfsetroundjoin%
\definecolor{currentfill}{rgb}{1.000000,0.623529,0.607843}%
\pgfsetfillcolor{currentfill}%
\pgfsetlinewidth{0.481800pt}%
\definecolor{currentstroke}{rgb}{1.000000,1.000000,1.000000}%
\pgfsetstrokecolor{currentstroke}%
\pgfsetdash{}{0pt}%
\pgfpathmoveto{\pgfqpoint{7.337079in}{5.217206in}}%
\pgfpathcurveto{\pgfqpoint{7.348130in}{5.217206in}}{\pgfqpoint{7.358729in}{5.221596in}}{\pgfqpoint{7.366542in}{5.229410in}}%
\pgfpathcurveto{\pgfqpoint{7.374356in}{5.237223in}}{\pgfqpoint{7.378746in}{5.247822in}}{\pgfqpoint{7.378746in}{5.258872in}}%
\pgfpathcurveto{\pgfqpoint{7.378746in}{5.269922in}}{\pgfqpoint{7.374356in}{5.280521in}}{\pgfqpoint{7.366542in}{5.288335in}}%
\pgfpathcurveto{\pgfqpoint{7.358729in}{5.296149in}}{\pgfqpoint{7.348130in}{5.300539in}}{\pgfqpoint{7.337079in}{5.300539in}}%
\pgfpathcurveto{\pgfqpoint{7.326029in}{5.300539in}}{\pgfqpoint{7.315430in}{5.296149in}}{\pgfqpoint{7.307617in}{5.288335in}}%
\pgfpathcurveto{\pgfqpoint{7.299803in}{5.280521in}}{\pgfqpoint{7.295413in}{5.269922in}}{\pgfqpoint{7.295413in}{5.258872in}}%
\pgfpathcurveto{\pgfqpoint{7.295413in}{5.247822in}}{\pgfqpoint{7.299803in}{5.237223in}}{\pgfqpoint{7.307617in}{5.229410in}}%
\pgfpathcurveto{\pgfqpoint{7.315430in}{5.221596in}}{\pgfqpoint{7.326029in}{5.217206in}}{\pgfqpoint{7.337079in}{5.217206in}}%
\pgfpathclose%
\pgfusepath{stroke,fill}%
\end{pgfscope}%
\begin{pgfscope}%
\pgfpathrectangle{\pgfqpoint{0.481978in}{0.331635in}}{\pgfqpoint{9.300000in}{7.700000in}}%
\pgfusepath{clip}%
\pgfsetbuttcap%
\pgfsetroundjoin%
\definecolor{currentfill}{rgb}{1.000000,0.623529,0.607843}%
\pgfsetfillcolor{currentfill}%
\pgfsetlinewidth{0.481800pt}%
\definecolor{currentstroke}{rgb}{1.000000,1.000000,1.000000}%
\pgfsetstrokecolor{currentstroke}%
\pgfsetdash{}{0pt}%
\pgfpathmoveto{\pgfqpoint{4.553085in}{2.981640in}}%
\pgfpathcurveto{\pgfqpoint{4.564135in}{2.981640in}}{\pgfqpoint{4.574734in}{2.986030in}}{\pgfqpoint{4.582547in}{2.993844in}}%
\pgfpathcurveto{\pgfqpoint{4.590361in}{3.001658in}}{\pgfqpoint{4.594751in}{3.012257in}}{\pgfqpoint{4.594751in}{3.023307in}}%
\pgfpathcurveto{\pgfqpoint{4.594751in}{3.034357in}}{\pgfqpoint{4.590361in}{3.044956in}}{\pgfqpoint{4.582547in}{3.052769in}}%
\pgfpathcurveto{\pgfqpoint{4.574734in}{3.060583in}}{\pgfqpoint{4.564135in}{3.064973in}}{\pgfqpoint{4.553085in}{3.064973in}}%
\pgfpathcurveto{\pgfqpoint{4.542035in}{3.064973in}}{\pgfqpoint{4.531436in}{3.060583in}}{\pgfqpoint{4.523622in}{3.052769in}}%
\pgfpathcurveto{\pgfqpoint{4.515808in}{3.044956in}}{\pgfqpoint{4.511418in}{3.034357in}}{\pgfqpoint{4.511418in}{3.023307in}}%
\pgfpathcurveto{\pgfqpoint{4.511418in}{3.012257in}}{\pgfqpoint{4.515808in}{3.001658in}}{\pgfqpoint{4.523622in}{2.993844in}}%
\pgfpathcurveto{\pgfqpoint{4.531436in}{2.986030in}}{\pgfqpoint{4.542035in}{2.981640in}}{\pgfqpoint{4.553085in}{2.981640in}}%
\pgfpathclose%
\pgfusepath{stroke,fill}%
\end{pgfscope}%
\begin{pgfscope}%
\pgfpathrectangle{\pgfqpoint{0.481978in}{0.331635in}}{\pgfqpoint{9.300000in}{7.700000in}}%
\pgfusepath{clip}%
\pgfsetbuttcap%
\pgfsetroundjoin%
\definecolor{currentfill}{rgb}{1.000000,0.623529,0.607843}%
\pgfsetfillcolor{currentfill}%
\pgfsetlinewidth{0.481800pt}%
\definecolor{currentstroke}{rgb}{1.000000,1.000000,1.000000}%
\pgfsetstrokecolor{currentstroke}%
\pgfsetdash{}{0pt}%
\pgfpathmoveto{\pgfqpoint{9.095716in}{5.848566in}}%
\pgfpathcurveto{\pgfqpoint{9.106766in}{5.848566in}}{\pgfqpoint{9.117365in}{5.852956in}}{\pgfqpoint{9.125179in}{5.860770in}}%
\pgfpathcurveto{\pgfqpoint{9.132993in}{5.868584in}}{\pgfqpoint{9.137383in}{5.879183in}}{\pgfqpoint{9.137383in}{5.890233in}}%
\pgfpathcurveto{\pgfqpoint{9.137383in}{5.901283in}}{\pgfqpoint{9.132993in}{5.911882in}}{\pgfqpoint{9.125179in}{5.919696in}}%
\pgfpathcurveto{\pgfqpoint{9.117365in}{5.927509in}}{\pgfqpoint{9.106766in}{5.931899in}}{\pgfqpoint{9.095716in}{5.931899in}}%
\pgfpathcurveto{\pgfqpoint{9.084666in}{5.931899in}}{\pgfqpoint{9.074067in}{5.927509in}}{\pgfqpoint{9.066253in}{5.919696in}}%
\pgfpathcurveto{\pgfqpoint{9.058440in}{5.911882in}}{\pgfqpoint{9.054050in}{5.901283in}}{\pgfqpoint{9.054050in}{5.890233in}}%
\pgfpathcurveto{\pgfqpoint{9.054050in}{5.879183in}}{\pgfqpoint{9.058440in}{5.868584in}}{\pgfqpoint{9.066253in}{5.860770in}}%
\pgfpathcurveto{\pgfqpoint{9.074067in}{5.852956in}}{\pgfqpoint{9.084666in}{5.848566in}}{\pgfqpoint{9.095716in}{5.848566in}}%
\pgfpathclose%
\pgfusepath{stroke,fill}%
\end{pgfscope}%
\begin{pgfscope}%
\pgfpathrectangle{\pgfqpoint{0.481978in}{0.331635in}}{\pgfqpoint{9.300000in}{7.700000in}}%
\pgfusepath{clip}%
\pgfsetbuttcap%
\pgfsetroundjoin%
\definecolor{currentfill}{rgb}{1.000000,0.623529,0.607843}%
\pgfsetfillcolor{currentfill}%
\pgfsetlinewidth{0.481800pt}%
\definecolor{currentstroke}{rgb}{1.000000,1.000000,1.000000}%
\pgfsetstrokecolor{currentstroke}%
\pgfsetdash{}{0pt}%
\pgfpathmoveto{\pgfqpoint{2.403697in}{3.430362in}}%
\pgfpathcurveto{\pgfqpoint{2.414747in}{3.430362in}}{\pgfqpoint{2.425346in}{3.434752in}}{\pgfqpoint{2.433160in}{3.442566in}}%
\pgfpathcurveto{\pgfqpoint{2.440973in}{3.450380in}}{\pgfqpoint{2.445364in}{3.460979in}}{\pgfqpoint{2.445364in}{3.472029in}}%
\pgfpathcurveto{\pgfqpoint{2.445364in}{3.483079in}}{\pgfqpoint{2.440973in}{3.493678in}}{\pgfqpoint{2.433160in}{3.501491in}}%
\pgfpathcurveto{\pgfqpoint{2.425346in}{3.509305in}}{\pgfqpoint{2.414747in}{3.513695in}}{\pgfqpoint{2.403697in}{3.513695in}}%
\pgfpathcurveto{\pgfqpoint{2.392647in}{3.513695in}}{\pgfqpoint{2.382048in}{3.509305in}}{\pgfqpoint{2.374234in}{3.501491in}}%
\pgfpathcurveto{\pgfqpoint{2.366420in}{3.493678in}}{\pgfqpoint{2.362030in}{3.483079in}}{\pgfqpoint{2.362030in}{3.472029in}}%
\pgfpathcurveto{\pgfqpoint{2.362030in}{3.460979in}}{\pgfqpoint{2.366420in}{3.450380in}}{\pgfqpoint{2.374234in}{3.442566in}}%
\pgfpathcurveto{\pgfqpoint{2.382048in}{3.434752in}}{\pgfqpoint{2.392647in}{3.430362in}}{\pgfqpoint{2.403697in}{3.430362in}}%
\pgfpathclose%
\pgfusepath{stroke,fill}%
\end{pgfscope}%
\begin{pgfscope}%
\pgfpathrectangle{\pgfqpoint{0.481978in}{0.331635in}}{\pgfqpoint{9.300000in}{7.700000in}}%
\pgfusepath{clip}%
\pgfsetbuttcap%
\pgfsetroundjoin%
\definecolor{currentfill}{rgb}{1.000000,0.623529,0.607843}%
\pgfsetfillcolor{currentfill}%
\pgfsetlinewidth{0.481800pt}%
\definecolor{currentstroke}{rgb}{1.000000,1.000000,1.000000}%
\pgfsetstrokecolor{currentstroke}%
\pgfsetdash{}{0pt}%
\pgfpathmoveto{\pgfqpoint{7.815154in}{4.410412in}}%
\pgfpathcurveto{\pgfqpoint{7.826205in}{4.410412in}}{\pgfqpoint{7.836804in}{4.414802in}}{\pgfqpoint{7.844617in}{4.422616in}}%
\pgfpathcurveto{\pgfqpoint{7.852431in}{4.430429in}}{\pgfqpoint{7.856821in}{4.441028in}}{\pgfqpoint{7.856821in}{4.452079in}}%
\pgfpathcurveto{\pgfqpoint{7.856821in}{4.463129in}}{\pgfqpoint{7.852431in}{4.473728in}}{\pgfqpoint{7.844617in}{4.481541in}}%
\pgfpathcurveto{\pgfqpoint{7.836804in}{4.489355in}}{\pgfqpoint{7.826205in}{4.493745in}}{\pgfqpoint{7.815154in}{4.493745in}}%
\pgfpathcurveto{\pgfqpoint{7.804104in}{4.493745in}}{\pgfqpoint{7.793505in}{4.489355in}}{\pgfqpoint{7.785692in}{4.481541in}}%
\pgfpathcurveto{\pgfqpoint{7.777878in}{4.473728in}}{\pgfqpoint{7.773488in}{4.463129in}}{\pgfqpoint{7.773488in}{4.452079in}}%
\pgfpathcurveto{\pgfqpoint{7.773488in}{4.441028in}}{\pgfqpoint{7.777878in}{4.430429in}}{\pgfqpoint{7.785692in}{4.422616in}}%
\pgfpathcurveto{\pgfqpoint{7.793505in}{4.414802in}}{\pgfqpoint{7.804104in}{4.410412in}}{\pgfqpoint{7.815154in}{4.410412in}}%
\pgfpathclose%
\pgfusepath{stroke,fill}%
\end{pgfscope}%
\begin{pgfscope}%
\pgfpathrectangle{\pgfqpoint{0.481978in}{0.331635in}}{\pgfqpoint{9.300000in}{7.700000in}}%
\pgfusepath{clip}%
\pgfsetbuttcap%
\pgfsetroundjoin%
\definecolor{currentfill}{rgb}{1.000000,0.623529,0.607843}%
\pgfsetfillcolor{currentfill}%
\pgfsetlinewidth{0.481800pt}%
\definecolor{currentstroke}{rgb}{1.000000,1.000000,1.000000}%
\pgfsetstrokecolor{currentstroke}%
\pgfsetdash{}{0pt}%
\pgfpathmoveto{\pgfqpoint{4.473626in}{1.982566in}}%
\pgfpathcurveto{\pgfqpoint{4.484676in}{1.982566in}}{\pgfqpoint{4.495275in}{1.986956in}}{\pgfqpoint{4.503089in}{1.994770in}}%
\pgfpathcurveto{\pgfqpoint{4.510902in}{2.002583in}}{\pgfqpoint{4.515293in}{2.013182in}}{\pgfqpoint{4.515293in}{2.024232in}}%
\pgfpathcurveto{\pgfqpoint{4.515293in}{2.035282in}}{\pgfqpoint{4.510902in}{2.045881in}}{\pgfqpoint{4.503089in}{2.053695in}}%
\pgfpathcurveto{\pgfqpoint{4.495275in}{2.061509in}}{\pgfqpoint{4.484676in}{2.065899in}}{\pgfqpoint{4.473626in}{2.065899in}}%
\pgfpathcurveto{\pgfqpoint{4.462576in}{2.065899in}}{\pgfqpoint{4.451977in}{2.061509in}}{\pgfqpoint{4.444163in}{2.053695in}}%
\pgfpathcurveto{\pgfqpoint{4.436350in}{2.045881in}}{\pgfqpoint{4.431959in}{2.035282in}}{\pgfqpoint{4.431959in}{2.024232in}}%
\pgfpathcurveto{\pgfqpoint{4.431959in}{2.013182in}}{\pgfqpoint{4.436350in}{2.002583in}}{\pgfqpoint{4.444163in}{1.994770in}}%
\pgfpathcurveto{\pgfqpoint{4.451977in}{1.986956in}}{\pgfqpoint{4.462576in}{1.982566in}}{\pgfqpoint{4.473626in}{1.982566in}}%
\pgfpathclose%
\pgfusepath{stroke,fill}%
\end{pgfscope}%
\begin{pgfscope}%
\pgfpathrectangle{\pgfqpoint{0.481978in}{0.331635in}}{\pgfqpoint{9.300000in}{7.700000in}}%
\pgfusepath{clip}%
\pgfsetbuttcap%
\pgfsetroundjoin%
\definecolor{currentfill}{rgb}{0.815686,0.733333,1.000000}%
\pgfsetfillcolor{currentfill}%
\pgfsetlinewidth{0.481800pt}%
\definecolor{currentstroke}{rgb}{1.000000,1.000000,1.000000}%
\pgfsetstrokecolor{currentstroke}%
\pgfsetdash{}{0pt}%
\pgfpathmoveto{\pgfqpoint{7.152848in}{5.351280in}}%
\pgfpathcurveto{\pgfqpoint{7.163898in}{5.351280in}}{\pgfqpoint{7.174497in}{5.355670in}}{\pgfqpoint{7.182311in}{5.363484in}}%
\pgfpathcurveto{\pgfqpoint{7.190125in}{5.371297in}}{\pgfqpoint{7.194515in}{5.381896in}}{\pgfqpoint{7.194515in}{5.392946in}}%
\pgfpathcurveto{\pgfqpoint{7.194515in}{5.403996in}}{\pgfqpoint{7.190125in}{5.414596in}}{\pgfqpoint{7.182311in}{5.422409in}}%
\pgfpathcurveto{\pgfqpoint{7.174497in}{5.430223in}}{\pgfqpoint{7.163898in}{5.434613in}}{\pgfqpoint{7.152848in}{5.434613in}}%
\pgfpathcurveto{\pgfqpoint{7.141798in}{5.434613in}}{\pgfqpoint{7.131199in}{5.430223in}}{\pgfqpoint{7.123385in}{5.422409in}}%
\pgfpathcurveto{\pgfqpoint{7.115572in}{5.414596in}}{\pgfqpoint{7.111181in}{5.403996in}}{\pgfqpoint{7.111181in}{5.392946in}}%
\pgfpathcurveto{\pgfqpoint{7.111181in}{5.381896in}}{\pgfqpoint{7.115572in}{5.371297in}}{\pgfqpoint{7.123385in}{5.363484in}}%
\pgfpathcurveto{\pgfqpoint{7.131199in}{5.355670in}}{\pgfqpoint{7.141798in}{5.351280in}}{\pgfqpoint{7.152848in}{5.351280in}}%
\pgfpathclose%
\pgfusepath{stroke,fill}%
\end{pgfscope}%
\begin{pgfscope}%
\pgfpathrectangle{\pgfqpoint{0.481978in}{0.331635in}}{\pgfqpoint{9.300000in}{7.700000in}}%
\pgfusepath{clip}%
\pgfsetbuttcap%
\pgfsetroundjoin%
\definecolor{currentfill}{rgb}{0.815686,0.733333,1.000000}%
\pgfsetfillcolor{currentfill}%
\pgfsetlinewidth{0.481800pt}%
\definecolor{currentstroke}{rgb}{1.000000,1.000000,1.000000}%
\pgfsetstrokecolor{currentstroke}%
\pgfsetdash{}{0pt}%
\pgfpathmoveto{\pgfqpoint{7.702166in}{5.065536in}}%
\pgfpathcurveto{\pgfqpoint{7.713216in}{5.065536in}}{\pgfqpoint{7.723815in}{5.069927in}}{\pgfqpoint{7.731629in}{5.077740in}}%
\pgfpathcurveto{\pgfqpoint{7.739442in}{5.085554in}}{\pgfqpoint{7.743832in}{5.096153in}}{\pgfqpoint{7.743832in}{5.107203in}}%
\pgfpathcurveto{\pgfqpoint{7.743832in}{5.118253in}}{\pgfqpoint{7.739442in}{5.128852in}}{\pgfqpoint{7.731629in}{5.136666in}}%
\pgfpathcurveto{\pgfqpoint{7.723815in}{5.144480in}}{\pgfqpoint{7.713216in}{5.148870in}}{\pgfqpoint{7.702166in}{5.148870in}}%
\pgfpathcurveto{\pgfqpoint{7.691116in}{5.148870in}}{\pgfqpoint{7.680517in}{5.144480in}}{\pgfqpoint{7.672703in}{5.136666in}}%
\pgfpathcurveto{\pgfqpoint{7.664889in}{5.128852in}}{\pgfqpoint{7.660499in}{5.118253in}}{\pgfqpoint{7.660499in}{5.107203in}}%
\pgfpathcurveto{\pgfqpoint{7.660499in}{5.096153in}}{\pgfqpoint{7.664889in}{5.085554in}}{\pgfqpoint{7.672703in}{5.077740in}}%
\pgfpathcurveto{\pgfqpoint{7.680517in}{5.069927in}}{\pgfqpoint{7.691116in}{5.065536in}}{\pgfqpoint{7.702166in}{5.065536in}}%
\pgfpathclose%
\pgfusepath{stroke,fill}%
\end{pgfscope}%
\begin{pgfscope}%
\pgfpathrectangle{\pgfqpoint{0.481978in}{0.331635in}}{\pgfqpoint{9.300000in}{7.700000in}}%
\pgfusepath{clip}%
\pgfsetbuttcap%
\pgfsetroundjoin%
\definecolor{currentfill}{rgb}{0.815686,0.733333,1.000000}%
\pgfsetfillcolor{currentfill}%
\pgfsetlinewidth{0.481800pt}%
\definecolor{currentstroke}{rgb}{1.000000,1.000000,1.000000}%
\pgfsetstrokecolor{currentstroke}%
\pgfsetdash{}{0pt}%
\pgfpathmoveto{\pgfqpoint{2.872669in}{3.175352in}}%
\pgfpathcurveto{\pgfqpoint{2.883719in}{3.175352in}}{\pgfqpoint{2.894318in}{3.179742in}}{\pgfqpoint{2.902132in}{3.187556in}}%
\pgfpathcurveto{\pgfqpoint{2.909945in}{3.195370in}}{\pgfqpoint{2.914336in}{3.205969in}}{\pgfqpoint{2.914336in}{3.217019in}}%
\pgfpathcurveto{\pgfqpoint{2.914336in}{3.228069in}}{\pgfqpoint{2.909945in}{3.238668in}}{\pgfqpoint{2.902132in}{3.246482in}}%
\pgfpathcurveto{\pgfqpoint{2.894318in}{3.254295in}}{\pgfqpoint{2.883719in}{3.258686in}}{\pgfqpoint{2.872669in}{3.258686in}}%
\pgfpathcurveto{\pgfqpoint{2.861619in}{3.258686in}}{\pgfqpoint{2.851020in}{3.254295in}}{\pgfqpoint{2.843206in}{3.246482in}}%
\pgfpathcurveto{\pgfqpoint{2.835392in}{3.238668in}}{\pgfqpoint{2.831002in}{3.228069in}}{\pgfqpoint{2.831002in}{3.217019in}}%
\pgfpathcurveto{\pgfqpoint{2.831002in}{3.205969in}}{\pgfqpoint{2.835392in}{3.195370in}}{\pgfqpoint{2.843206in}{3.187556in}}%
\pgfpathcurveto{\pgfqpoint{2.851020in}{3.179742in}}{\pgfqpoint{2.861619in}{3.175352in}}{\pgfqpoint{2.872669in}{3.175352in}}%
\pgfpathclose%
\pgfusepath{stroke,fill}%
\end{pgfscope}%
\begin{pgfscope}%
\pgfpathrectangle{\pgfqpoint{0.481978in}{0.331635in}}{\pgfqpoint{9.300000in}{7.700000in}}%
\pgfusepath{clip}%
\pgfsetbuttcap%
\pgfsetroundjoin%
\definecolor{currentfill}{rgb}{0.815686,0.733333,1.000000}%
\pgfsetfillcolor{currentfill}%
\pgfsetlinewidth{0.481800pt}%
\definecolor{currentstroke}{rgb}{1.000000,1.000000,1.000000}%
\pgfsetstrokecolor{currentstroke}%
\pgfsetdash{}{0pt}%
\pgfpathmoveto{\pgfqpoint{3.619377in}{2.321220in}}%
\pgfpathcurveto{\pgfqpoint{3.630427in}{2.321220in}}{\pgfqpoint{3.641026in}{2.325610in}}{\pgfqpoint{3.648840in}{2.333423in}}%
\pgfpathcurveto{\pgfqpoint{3.656654in}{2.341237in}}{\pgfqpoint{3.661044in}{2.351836in}}{\pgfqpoint{3.661044in}{2.362886in}}%
\pgfpathcurveto{\pgfqpoint{3.661044in}{2.373936in}}{\pgfqpoint{3.656654in}{2.384535in}}{\pgfqpoint{3.648840in}{2.392349in}}%
\pgfpathcurveto{\pgfqpoint{3.641026in}{2.400163in}}{\pgfqpoint{3.630427in}{2.404553in}}{\pgfqpoint{3.619377in}{2.404553in}}%
\pgfpathcurveto{\pgfqpoint{3.608327in}{2.404553in}}{\pgfqpoint{3.597728in}{2.400163in}}{\pgfqpoint{3.589914in}{2.392349in}}%
\pgfpathcurveto{\pgfqpoint{3.582101in}{2.384535in}}{\pgfqpoint{3.577710in}{2.373936in}}{\pgfqpoint{3.577710in}{2.362886in}}%
\pgfpathcurveto{\pgfqpoint{3.577710in}{2.351836in}}{\pgfqpoint{3.582101in}{2.341237in}}{\pgfqpoint{3.589914in}{2.333423in}}%
\pgfpathcurveto{\pgfqpoint{3.597728in}{2.325610in}}{\pgfqpoint{3.608327in}{2.321220in}}{\pgfqpoint{3.619377in}{2.321220in}}%
\pgfpathclose%
\pgfusepath{stroke,fill}%
\end{pgfscope}%
\begin{pgfscope}%
\pgfpathrectangle{\pgfqpoint{0.481978in}{0.331635in}}{\pgfqpoint{9.300000in}{7.700000in}}%
\pgfusepath{clip}%
\pgfsetbuttcap%
\pgfsetroundjoin%
\definecolor{currentfill}{rgb}{0.815686,0.733333,1.000000}%
\pgfsetfillcolor{currentfill}%
\pgfsetlinewidth{0.481800pt}%
\definecolor{currentstroke}{rgb}{1.000000,1.000000,1.000000}%
\pgfsetstrokecolor{currentstroke}%
\pgfsetdash{}{0pt}%
\pgfpathmoveto{\pgfqpoint{3.448106in}{2.470564in}}%
\pgfpathcurveto{\pgfqpoint{3.459156in}{2.470564in}}{\pgfqpoint{3.469755in}{2.474954in}}{\pgfqpoint{3.477568in}{2.482767in}}%
\pgfpathcurveto{\pgfqpoint{3.485382in}{2.490581in}}{\pgfqpoint{3.489772in}{2.501180in}}{\pgfqpoint{3.489772in}{2.512230in}}%
\pgfpathcurveto{\pgfqpoint{3.489772in}{2.523280in}}{\pgfqpoint{3.485382in}{2.533879in}}{\pgfqpoint{3.477568in}{2.541693in}}%
\pgfpathcurveto{\pgfqpoint{3.469755in}{2.549507in}}{\pgfqpoint{3.459156in}{2.553897in}}{\pgfqpoint{3.448106in}{2.553897in}}%
\pgfpathcurveto{\pgfqpoint{3.437055in}{2.553897in}}{\pgfqpoint{3.426456in}{2.549507in}}{\pgfqpoint{3.418643in}{2.541693in}}%
\pgfpathcurveto{\pgfqpoint{3.410829in}{2.533879in}}{\pgfqpoint{3.406439in}{2.523280in}}{\pgfqpoint{3.406439in}{2.512230in}}%
\pgfpathcurveto{\pgfqpoint{3.406439in}{2.501180in}}{\pgfqpoint{3.410829in}{2.490581in}}{\pgfqpoint{3.418643in}{2.482767in}}%
\pgfpathcurveto{\pgfqpoint{3.426456in}{2.474954in}}{\pgfqpoint{3.437055in}{2.470564in}}{\pgfqpoint{3.448106in}{2.470564in}}%
\pgfpathclose%
\pgfusepath{stroke,fill}%
\end{pgfscope}%
\begin{pgfscope}%
\pgfpathrectangle{\pgfqpoint{0.481978in}{0.331635in}}{\pgfqpoint{9.300000in}{7.700000in}}%
\pgfusepath{clip}%
\pgfsetbuttcap%
\pgfsetroundjoin%
\definecolor{currentfill}{rgb}{0.815686,0.733333,1.000000}%
\pgfsetfillcolor{currentfill}%
\pgfsetlinewidth{0.481800pt}%
\definecolor{currentstroke}{rgb}{1.000000,1.000000,1.000000}%
\pgfsetstrokecolor{currentstroke}%
\pgfsetdash{}{0pt}%
\pgfpathmoveto{\pgfqpoint{6.968468in}{3.304397in}}%
\pgfpathcurveto{\pgfqpoint{6.979518in}{3.304397in}}{\pgfqpoint{6.990117in}{3.308788in}}{\pgfqpoint{6.997931in}{3.316601in}}%
\pgfpathcurveto{\pgfqpoint{7.005744in}{3.324415in}}{\pgfqpoint{7.010135in}{3.335014in}}{\pgfqpoint{7.010135in}{3.346064in}}%
\pgfpathcurveto{\pgfqpoint{7.010135in}{3.357114in}}{\pgfqpoint{7.005744in}{3.367713in}}{\pgfqpoint{6.997931in}{3.375527in}}%
\pgfpathcurveto{\pgfqpoint{6.990117in}{3.383340in}}{\pgfqpoint{6.979518in}{3.387731in}}{\pgfqpoint{6.968468in}{3.387731in}}%
\pgfpathcurveto{\pgfqpoint{6.957418in}{3.387731in}}{\pgfqpoint{6.946819in}{3.383340in}}{\pgfqpoint{6.939005in}{3.375527in}}%
\pgfpathcurveto{\pgfqpoint{6.931192in}{3.367713in}}{\pgfqpoint{6.926801in}{3.357114in}}{\pgfqpoint{6.926801in}{3.346064in}}%
\pgfpathcurveto{\pgfqpoint{6.926801in}{3.335014in}}{\pgfqpoint{6.931192in}{3.324415in}}{\pgfqpoint{6.939005in}{3.316601in}}%
\pgfpathcurveto{\pgfqpoint{6.946819in}{3.308788in}}{\pgfqpoint{6.957418in}{3.304397in}}{\pgfqpoint{6.968468in}{3.304397in}}%
\pgfpathclose%
\pgfusepath{stroke,fill}%
\end{pgfscope}%
\begin{pgfscope}%
\pgfpathrectangle{\pgfqpoint{0.481978in}{0.331635in}}{\pgfqpoint{9.300000in}{7.700000in}}%
\pgfusepath{clip}%
\pgfsetbuttcap%
\pgfsetroundjoin%
\definecolor{currentfill}{rgb}{0.815686,0.733333,1.000000}%
\pgfsetfillcolor{currentfill}%
\pgfsetlinewidth{0.481800pt}%
\definecolor{currentstroke}{rgb}{1.000000,1.000000,1.000000}%
\pgfsetstrokecolor{currentstroke}%
\pgfsetdash{}{0pt}%
\pgfpathmoveto{\pgfqpoint{3.260742in}{6.989602in}}%
\pgfpathcurveto{\pgfqpoint{3.271792in}{6.989602in}}{\pgfqpoint{3.282391in}{6.993992in}}{\pgfqpoint{3.290204in}{7.001806in}}%
\pgfpathcurveto{\pgfqpoint{3.298018in}{7.009620in}}{\pgfqpoint{3.302408in}{7.020219in}}{\pgfqpoint{3.302408in}{7.031269in}}%
\pgfpathcurveto{\pgfqpoint{3.302408in}{7.042319in}}{\pgfqpoint{3.298018in}{7.052918in}}{\pgfqpoint{3.290204in}{7.060732in}}%
\pgfpathcurveto{\pgfqpoint{3.282391in}{7.068545in}}{\pgfqpoint{3.271792in}{7.072935in}}{\pgfqpoint{3.260742in}{7.072935in}}%
\pgfpathcurveto{\pgfqpoint{3.249692in}{7.072935in}}{\pgfqpoint{3.239092in}{7.068545in}}{\pgfqpoint{3.231279in}{7.060732in}}%
\pgfpathcurveto{\pgfqpoint{3.223465in}{7.052918in}}{\pgfqpoint{3.219075in}{7.042319in}}{\pgfqpoint{3.219075in}{7.031269in}}%
\pgfpathcurveto{\pgfqpoint{3.219075in}{7.020219in}}{\pgfqpoint{3.223465in}{7.009620in}}{\pgfqpoint{3.231279in}{7.001806in}}%
\pgfpathcurveto{\pgfqpoint{3.239092in}{6.993992in}}{\pgfqpoint{3.249692in}{6.989602in}}{\pgfqpoint{3.260742in}{6.989602in}}%
\pgfpathclose%
\pgfusepath{stroke,fill}%
\end{pgfscope}%
\begin{pgfscope}%
\pgfpathrectangle{\pgfqpoint{0.481978in}{0.331635in}}{\pgfqpoint{9.300000in}{7.700000in}}%
\pgfusepath{clip}%
\pgfsetbuttcap%
\pgfsetroundjoin%
\definecolor{currentfill}{rgb}{0.815686,0.733333,1.000000}%
\pgfsetfillcolor{currentfill}%
\pgfsetlinewidth{0.481800pt}%
\definecolor{currentstroke}{rgb}{1.000000,1.000000,1.000000}%
\pgfsetstrokecolor{currentstroke}%
\pgfsetdash{}{0pt}%
\pgfpathmoveto{\pgfqpoint{3.172404in}{5.733347in}}%
\pgfpathcurveto{\pgfqpoint{3.183454in}{5.733347in}}{\pgfqpoint{3.194053in}{5.737738in}}{\pgfqpoint{3.201867in}{5.745551in}}%
\pgfpathcurveto{\pgfqpoint{3.209680in}{5.753365in}}{\pgfqpoint{3.214071in}{5.763964in}}{\pgfqpoint{3.214071in}{5.775014in}}%
\pgfpathcurveto{\pgfqpoint{3.214071in}{5.786064in}}{\pgfqpoint{3.209680in}{5.796663in}}{\pgfqpoint{3.201867in}{5.804477in}}%
\pgfpathcurveto{\pgfqpoint{3.194053in}{5.812290in}}{\pgfqpoint{3.183454in}{5.816681in}}{\pgfqpoint{3.172404in}{5.816681in}}%
\pgfpathcurveto{\pgfqpoint{3.161354in}{5.816681in}}{\pgfqpoint{3.150755in}{5.812290in}}{\pgfqpoint{3.142941in}{5.804477in}}%
\pgfpathcurveto{\pgfqpoint{3.135127in}{5.796663in}}{\pgfqpoint{3.130737in}{5.786064in}}{\pgfqpoint{3.130737in}{5.775014in}}%
\pgfpathcurveto{\pgfqpoint{3.130737in}{5.763964in}}{\pgfqpoint{3.135127in}{5.753365in}}{\pgfqpoint{3.142941in}{5.745551in}}%
\pgfpathcurveto{\pgfqpoint{3.150755in}{5.737738in}}{\pgfqpoint{3.161354in}{5.733347in}}{\pgfqpoint{3.172404in}{5.733347in}}%
\pgfpathclose%
\pgfusepath{stroke,fill}%
\end{pgfscope}%
\begin{pgfscope}%
\pgfpathrectangle{\pgfqpoint{0.481978in}{0.331635in}}{\pgfqpoint{9.300000in}{7.700000in}}%
\pgfusepath{clip}%
\pgfsetbuttcap%
\pgfsetroundjoin%
\definecolor{currentfill}{rgb}{0.815686,0.733333,1.000000}%
\pgfsetfillcolor{currentfill}%
\pgfsetlinewidth{0.481800pt}%
\definecolor{currentstroke}{rgb}{1.000000,1.000000,1.000000}%
\pgfsetstrokecolor{currentstroke}%
\pgfsetdash{}{0pt}%
\pgfpathmoveto{\pgfqpoint{2.953247in}{2.659230in}}%
\pgfpathcurveto{\pgfqpoint{2.964297in}{2.659230in}}{\pgfqpoint{2.974896in}{2.663621in}}{\pgfqpoint{2.982710in}{2.671434in}}%
\pgfpathcurveto{\pgfqpoint{2.990523in}{2.679248in}}{\pgfqpoint{2.994914in}{2.689847in}}{\pgfqpoint{2.994914in}{2.700897in}}%
\pgfpathcurveto{\pgfqpoint{2.994914in}{2.711947in}}{\pgfqpoint{2.990523in}{2.722546in}}{\pgfqpoint{2.982710in}{2.730360in}}%
\pgfpathcurveto{\pgfqpoint{2.974896in}{2.738173in}}{\pgfqpoint{2.964297in}{2.742564in}}{\pgfqpoint{2.953247in}{2.742564in}}%
\pgfpathcurveto{\pgfqpoint{2.942197in}{2.742564in}}{\pgfqpoint{2.931598in}{2.738173in}}{\pgfqpoint{2.923784in}{2.730360in}}%
\pgfpathcurveto{\pgfqpoint{2.915970in}{2.722546in}}{\pgfqpoint{2.911580in}{2.711947in}}{\pgfqpoint{2.911580in}{2.700897in}}%
\pgfpathcurveto{\pgfqpoint{2.911580in}{2.689847in}}{\pgfqpoint{2.915970in}{2.679248in}}{\pgfqpoint{2.923784in}{2.671434in}}%
\pgfpathcurveto{\pgfqpoint{2.931598in}{2.663621in}}{\pgfqpoint{2.942197in}{2.659230in}}{\pgfqpoint{2.953247in}{2.659230in}}%
\pgfpathclose%
\pgfusepath{stroke,fill}%
\end{pgfscope}%
\begin{pgfscope}%
\pgfpathrectangle{\pgfqpoint{0.481978in}{0.331635in}}{\pgfqpoint{9.300000in}{7.700000in}}%
\pgfusepath{clip}%
\pgfsetbuttcap%
\pgfsetroundjoin%
\definecolor{currentfill}{rgb}{0.815686,0.733333,1.000000}%
\pgfsetfillcolor{currentfill}%
\pgfsetlinewidth{0.481800pt}%
\definecolor{currentstroke}{rgb}{1.000000,1.000000,1.000000}%
\pgfsetstrokecolor{currentstroke}%
\pgfsetdash{}{0pt}%
\pgfpathmoveto{\pgfqpoint{8.116420in}{5.574617in}}%
\pgfpathcurveto{\pgfqpoint{8.127470in}{5.574617in}}{\pgfqpoint{8.138069in}{5.579007in}}{\pgfqpoint{8.145883in}{5.586821in}}%
\pgfpathcurveto{\pgfqpoint{8.153697in}{5.594635in}}{\pgfqpoint{8.158087in}{5.605234in}}{\pgfqpoint{8.158087in}{5.616284in}}%
\pgfpathcurveto{\pgfqpoint{8.158087in}{5.627334in}}{\pgfqpoint{8.153697in}{5.637933in}}{\pgfqpoint{8.145883in}{5.645747in}}%
\pgfpathcurveto{\pgfqpoint{8.138069in}{5.653560in}}{\pgfqpoint{8.127470in}{5.657950in}}{\pgfqpoint{8.116420in}{5.657950in}}%
\pgfpathcurveto{\pgfqpoint{8.105370in}{5.657950in}}{\pgfqpoint{8.094771in}{5.653560in}}{\pgfqpoint{8.086957in}{5.645747in}}%
\pgfpathcurveto{\pgfqpoint{8.079144in}{5.637933in}}{\pgfqpoint{8.074754in}{5.627334in}}{\pgfqpoint{8.074754in}{5.616284in}}%
\pgfpathcurveto{\pgfqpoint{8.074754in}{5.605234in}}{\pgfqpoint{8.079144in}{5.594635in}}{\pgfqpoint{8.086957in}{5.586821in}}%
\pgfpathcurveto{\pgfqpoint{8.094771in}{5.579007in}}{\pgfqpoint{8.105370in}{5.574617in}}{\pgfqpoint{8.116420in}{5.574617in}}%
\pgfpathclose%
\pgfusepath{stroke,fill}%
\end{pgfscope}%
\begin{pgfscope}%
\pgfpathrectangle{\pgfqpoint{0.481978in}{0.331635in}}{\pgfqpoint{9.300000in}{7.700000in}}%
\pgfusepath{clip}%
\pgfsetbuttcap%
\pgfsetroundjoin%
\definecolor{currentfill}{rgb}{0.815686,0.733333,1.000000}%
\pgfsetfillcolor{currentfill}%
\pgfsetlinewidth{0.481800pt}%
\definecolor{currentstroke}{rgb}{1.000000,1.000000,1.000000}%
\pgfsetstrokecolor{currentstroke}%
\pgfsetdash{}{0pt}%
\pgfpathmoveto{\pgfqpoint{6.906230in}{4.688829in}}%
\pgfpathcurveto{\pgfqpoint{6.917280in}{4.688829in}}{\pgfqpoint{6.927879in}{4.693220in}}{\pgfqpoint{6.935692in}{4.701033in}}%
\pgfpathcurveto{\pgfqpoint{6.943506in}{4.708847in}}{\pgfqpoint{6.947896in}{4.719446in}}{\pgfqpoint{6.947896in}{4.730496in}}%
\pgfpathcurveto{\pgfqpoint{6.947896in}{4.741546in}}{\pgfqpoint{6.943506in}{4.752145in}}{\pgfqpoint{6.935692in}{4.759959in}}%
\pgfpathcurveto{\pgfqpoint{6.927879in}{4.767772in}}{\pgfqpoint{6.917280in}{4.772163in}}{\pgfqpoint{6.906230in}{4.772163in}}%
\pgfpathcurveto{\pgfqpoint{6.895179in}{4.772163in}}{\pgfqpoint{6.884580in}{4.767772in}}{\pgfqpoint{6.876767in}{4.759959in}}%
\pgfpathcurveto{\pgfqpoint{6.868953in}{4.752145in}}{\pgfqpoint{6.864563in}{4.741546in}}{\pgfqpoint{6.864563in}{4.730496in}}%
\pgfpathcurveto{\pgfqpoint{6.864563in}{4.719446in}}{\pgfqpoint{6.868953in}{4.708847in}}{\pgfqpoint{6.876767in}{4.701033in}}%
\pgfpathcurveto{\pgfqpoint{6.884580in}{4.693220in}}{\pgfqpoint{6.895179in}{4.688829in}}{\pgfqpoint{6.906230in}{4.688829in}}%
\pgfpathclose%
\pgfusepath{stroke,fill}%
\end{pgfscope}%
\begin{pgfscope}%
\pgfpathrectangle{\pgfqpoint{0.481978in}{0.331635in}}{\pgfqpoint{9.300000in}{7.700000in}}%
\pgfusepath{clip}%
\pgfsetbuttcap%
\pgfsetroundjoin%
\definecolor{currentfill}{rgb}{0.815686,0.733333,1.000000}%
\pgfsetfillcolor{currentfill}%
\pgfsetlinewidth{0.481800pt}%
\definecolor{currentstroke}{rgb}{1.000000,1.000000,1.000000}%
\pgfsetstrokecolor{currentstroke}%
\pgfsetdash{}{0pt}%
\pgfpathmoveto{\pgfqpoint{3.815826in}{4.045962in}}%
\pgfpathcurveto{\pgfqpoint{3.826876in}{4.045962in}}{\pgfqpoint{3.837475in}{4.050352in}}{\pgfqpoint{3.845289in}{4.058166in}}%
\pgfpathcurveto{\pgfqpoint{3.853103in}{4.065979in}}{\pgfqpoint{3.857493in}{4.076578in}}{\pgfqpoint{3.857493in}{4.087628in}}%
\pgfpathcurveto{\pgfqpoint{3.857493in}{4.098678in}}{\pgfqpoint{3.853103in}{4.109278in}}{\pgfqpoint{3.845289in}{4.117091in}}%
\pgfpathcurveto{\pgfqpoint{3.837475in}{4.124905in}}{\pgfqpoint{3.826876in}{4.129295in}}{\pgfqpoint{3.815826in}{4.129295in}}%
\pgfpathcurveto{\pgfqpoint{3.804776in}{4.129295in}}{\pgfqpoint{3.794177in}{4.124905in}}{\pgfqpoint{3.786364in}{4.117091in}}%
\pgfpathcurveto{\pgfqpoint{3.778550in}{4.109278in}}{\pgfqpoint{3.774160in}{4.098678in}}{\pgfqpoint{3.774160in}{4.087628in}}%
\pgfpathcurveto{\pgfqpoint{3.774160in}{4.076578in}}{\pgfqpoint{3.778550in}{4.065979in}}{\pgfqpoint{3.786364in}{4.058166in}}%
\pgfpathcurveto{\pgfqpoint{3.794177in}{4.050352in}}{\pgfqpoint{3.804776in}{4.045962in}}{\pgfqpoint{3.815826in}{4.045962in}}%
\pgfpathclose%
\pgfusepath{stroke,fill}%
\end{pgfscope}%
\begin{pgfscope}%
\pgfpathrectangle{\pgfqpoint{0.481978in}{0.331635in}}{\pgfqpoint{9.300000in}{7.700000in}}%
\pgfusepath{clip}%
\pgfsetbuttcap%
\pgfsetroundjoin%
\definecolor{currentfill}{rgb}{0.815686,0.733333,1.000000}%
\pgfsetfillcolor{currentfill}%
\pgfsetlinewidth{0.481800pt}%
\definecolor{currentstroke}{rgb}{1.000000,1.000000,1.000000}%
\pgfsetstrokecolor{currentstroke}%
\pgfsetdash{}{0pt}%
\pgfpathmoveto{\pgfqpoint{6.962413in}{5.760379in}}%
\pgfpathcurveto{\pgfqpoint{6.973463in}{5.760379in}}{\pgfqpoint{6.984062in}{5.764769in}}{\pgfqpoint{6.991875in}{5.772582in}}%
\pgfpathcurveto{\pgfqpoint{6.999689in}{5.780396in}}{\pgfqpoint{7.004079in}{5.790995in}}{\pgfqpoint{7.004079in}{5.802045in}}%
\pgfpathcurveto{\pgfqpoint{7.004079in}{5.813095in}}{\pgfqpoint{6.999689in}{5.823694in}}{\pgfqpoint{6.991875in}{5.831508in}}%
\pgfpathcurveto{\pgfqpoint{6.984062in}{5.839322in}}{\pgfqpoint{6.973463in}{5.843712in}}{\pgfqpoint{6.962413in}{5.843712in}}%
\pgfpathcurveto{\pgfqpoint{6.951362in}{5.843712in}}{\pgfqpoint{6.940763in}{5.839322in}}{\pgfqpoint{6.932950in}{5.831508in}}%
\pgfpathcurveto{\pgfqpoint{6.925136in}{5.823694in}}{\pgfqpoint{6.920746in}{5.813095in}}{\pgfqpoint{6.920746in}{5.802045in}}%
\pgfpathcurveto{\pgfqpoint{6.920746in}{5.790995in}}{\pgfqpoint{6.925136in}{5.780396in}}{\pgfqpoint{6.932950in}{5.772582in}}%
\pgfpathcurveto{\pgfqpoint{6.940763in}{5.764769in}}{\pgfqpoint{6.951362in}{5.760379in}}{\pgfqpoint{6.962413in}{5.760379in}}%
\pgfpathclose%
\pgfusepath{stroke,fill}%
\end{pgfscope}%
\begin{pgfscope}%
\pgfpathrectangle{\pgfqpoint{0.481978in}{0.331635in}}{\pgfqpoint{9.300000in}{7.700000in}}%
\pgfusepath{clip}%
\pgfsetbuttcap%
\pgfsetroundjoin%
\definecolor{currentfill}{rgb}{0.815686,0.733333,1.000000}%
\pgfsetfillcolor{currentfill}%
\pgfsetlinewidth{0.481800pt}%
\definecolor{currentstroke}{rgb}{1.000000,1.000000,1.000000}%
\pgfsetstrokecolor{currentstroke}%
\pgfsetdash{}{0pt}%
\pgfpathmoveto{\pgfqpoint{7.236127in}{5.351377in}}%
\pgfpathcurveto{\pgfqpoint{7.247177in}{5.351377in}}{\pgfqpoint{7.257776in}{5.355768in}}{\pgfqpoint{7.265589in}{5.363581in}}%
\pgfpathcurveto{\pgfqpoint{7.273403in}{5.371395in}}{\pgfqpoint{7.277793in}{5.381994in}}{\pgfqpoint{7.277793in}{5.393044in}}%
\pgfpathcurveto{\pgfqpoint{7.277793in}{5.404094in}}{\pgfqpoint{7.273403in}{5.414693in}}{\pgfqpoint{7.265589in}{5.422507in}}%
\pgfpathcurveto{\pgfqpoint{7.257776in}{5.430321in}}{\pgfqpoint{7.247177in}{5.434711in}}{\pgfqpoint{7.236127in}{5.434711in}}%
\pgfpathcurveto{\pgfqpoint{7.225076in}{5.434711in}}{\pgfqpoint{7.214477in}{5.430321in}}{\pgfqpoint{7.206664in}{5.422507in}}%
\pgfpathcurveto{\pgfqpoint{7.198850in}{5.414693in}}{\pgfqpoint{7.194460in}{5.404094in}}{\pgfqpoint{7.194460in}{5.393044in}}%
\pgfpathcurveto{\pgfqpoint{7.194460in}{5.381994in}}{\pgfqpoint{7.198850in}{5.371395in}}{\pgfqpoint{7.206664in}{5.363581in}}%
\pgfpathcurveto{\pgfqpoint{7.214477in}{5.355768in}}{\pgfqpoint{7.225076in}{5.351377in}}{\pgfqpoint{7.236127in}{5.351377in}}%
\pgfpathclose%
\pgfusepath{stroke,fill}%
\end{pgfscope}%
\begin{pgfscope}%
\pgfpathrectangle{\pgfqpoint{0.481978in}{0.331635in}}{\pgfqpoint{9.300000in}{7.700000in}}%
\pgfusepath{clip}%
\pgfsetbuttcap%
\pgfsetroundjoin%
\definecolor{currentfill}{rgb}{0.815686,0.733333,1.000000}%
\pgfsetfillcolor{currentfill}%
\pgfsetlinewidth{0.481800pt}%
\definecolor{currentstroke}{rgb}{1.000000,1.000000,1.000000}%
\pgfsetstrokecolor{currentstroke}%
\pgfsetdash{}{0pt}%
\pgfpathmoveto{\pgfqpoint{8.116846in}{6.482719in}}%
\pgfpathcurveto{\pgfqpoint{8.127896in}{6.482719in}}{\pgfqpoint{8.138495in}{6.487109in}}{\pgfqpoint{8.146309in}{6.494923in}}%
\pgfpathcurveto{\pgfqpoint{8.154122in}{6.502737in}}{\pgfqpoint{8.158512in}{6.513336in}}{\pgfqpoint{8.158512in}{6.524386in}}%
\pgfpathcurveto{\pgfqpoint{8.158512in}{6.535436in}}{\pgfqpoint{8.154122in}{6.546035in}}{\pgfqpoint{8.146309in}{6.553849in}}%
\pgfpathcurveto{\pgfqpoint{8.138495in}{6.561662in}}{\pgfqpoint{8.127896in}{6.566052in}}{\pgfqpoint{8.116846in}{6.566052in}}%
\pgfpathcurveto{\pgfqpoint{8.105796in}{6.566052in}}{\pgfqpoint{8.095197in}{6.561662in}}{\pgfqpoint{8.087383in}{6.553849in}}%
\pgfpathcurveto{\pgfqpoint{8.079569in}{6.546035in}}{\pgfqpoint{8.075179in}{6.535436in}}{\pgfqpoint{8.075179in}{6.524386in}}%
\pgfpathcurveto{\pgfqpoint{8.075179in}{6.513336in}}{\pgfqpoint{8.079569in}{6.502737in}}{\pgfqpoint{8.087383in}{6.494923in}}%
\pgfpathcurveto{\pgfqpoint{8.095197in}{6.487109in}}{\pgfqpoint{8.105796in}{6.482719in}}{\pgfqpoint{8.116846in}{6.482719in}}%
\pgfpathclose%
\pgfusepath{stroke,fill}%
\end{pgfscope}%
\begin{pgfscope}%
\pgfpathrectangle{\pgfqpoint{0.481978in}{0.331635in}}{\pgfqpoint{9.300000in}{7.700000in}}%
\pgfusepath{clip}%
\pgfsetbuttcap%
\pgfsetroundjoin%
\definecolor{currentfill}{rgb}{0.815686,0.733333,1.000000}%
\pgfsetfillcolor{currentfill}%
\pgfsetlinewidth{0.481800pt}%
\definecolor{currentstroke}{rgb}{1.000000,1.000000,1.000000}%
\pgfsetstrokecolor{currentstroke}%
\pgfsetdash{}{0pt}%
\pgfpathmoveto{\pgfqpoint{3.449504in}{2.725034in}}%
\pgfpathcurveto{\pgfqpoint{3.460554in}{2.725034in}}{\pgfqpoint{3.471153in}{2.729424in}}{\pgfqpoint{3.478967in}{2.737238in}}%
\pgfpathcurveto{\pgfqpoint{3.486780in}{2.745051in}}{\pgfqpoint{3.491171in}{2.755651in}}{\pgfqpoint{3.491171in}{2.766701in}}%
\pgfpathcurveto{\pgfqpoint{3.491171in}{2.777751in}}{\pgfqpoint{3.486780in}{2.788350in}}{\pgfqpoint{3.478967in}{2.796163in}}%
\pgfpathcurveto{\pgfqpoint{3.471153in}{2.803977in}}{\pgfqpoint{3.460554in}{2.808367in}}{\pgfqpoint{3.449504in}{2.808367in}}%
\pgfpathcurveto{\pgfqpoint{3.438454in}{2.808367in}}{\pgfqpoint{3.427855in}{2.803977in}}{\pgfqpoint{3.420041in}{2.796163in}}%
\pgfpathcurveto{\pgfqpoint{3.412228in}{2.788350in}}{\pgfqpoint{3.407837in}{2.777751in}}{\pgfqpoint{3.407837in}{2.766701in}}%
\pgfpathcurveto{\pgfqpoint{3.407837in}{2.755651in}}{\pgfqpoint{3.412228in}{2.745051in}}{\pgfqpoint{3.420041in}{2.737238in}}%
\pgfpathcurveto{\pgfqpoint{3.427855in}{2.729424in}}{\pgfqpoint{3.438454in}{2.725034in}}{\pgfqpoint{3.449504in}{2.725034in}}%
\pgfpathclose%
\pgfusepath{stroke,fill}%
\end{pgfscope}%
\begin{pgfscope}%
\pgfpathrectangle{\pgfqpoint{0.481978in}{0.331635in}}{\pgfqpoint{9.300000in}{7.700000in}}%
\pgfusepath{clip}%
\pgfsetbuttcap%
\pgfsetroundjoin%
\definecolor{currentfill}{rgb}{0.815686,0.733333,1.000000}%
\pgfsetfillcolor{currentfill}%
\pgfsetlinewidth{0.481800pt}%
\definecolor{currentstroke}{rgb}{1.000000,1.000000,1.000000}%
\pgfsetstrokecolor{currentstroke}%
\pgfsetdash{}{0pt}%
\pgfpathmoveto{\pgfqpoint{4.947086in}{5.379971in}}%
\pgfpathcurveto{\pgfqpoint{4.958136in}{5.379971in}}{\pgfqpoint{4.968735in}{5.384361in}}{\pgfqpoint{4.976549in}{5.392175in}}%
\pgfpathcurveto{\pgfqpoint{4.984363in}{5.399988in}}{\pgfqpoint{4.988753in}{5.410587in}}{\pgfqpoint{4.988753in}{5.421637in}}%
\pgfpathcurveto{\pgfqpoint{4.988753in}{5.432688in}}{\pgfqpoint{4.984363in}{5.443287in}}{\pgfqpoint{4.976549in}{5.451100in}}%
\pgfpathcurveto{\pgfqpoint{4.968735in}{5.458914in}}{\pgfqpoint{4.958136in}{5.463304in}}{\pgfqpoint{4.947086in}{5.463304in}}%
\pgfpathcurveto{\pgfqpoint{4.936036in}{5.463304in}}{\pgfqpoint{4.925437in}{5.458914in}}{\pgfqpoint{4.917623in}{5.451100in}}%
\pgfpathcurveto{\pgfqpoint{4.909810in}{5.443287in}}{\pgfqpoint{4.905420in}{5.432688in}}{\pgfqpoint{4.905420in}{5.421637in}}%
\pgfpathcurveto{\pgfqpoint{4.905420in}{5.410587in}}{\pgfqpoint{4.909810in}{5.399988in}}{\pgfqpoint{4.917623in}{5.392175in}}%
\pgfpathcurveto{\pgfqpoint{4.925437in}{5.384361in}}{\pgfqpoint{4.936036in}{5.379971in}}{\pgfqpoint{4.947086in}{5.379971in}}%
\pgfpathclose%
\pgfusepath{stroke,fill}%
\end{pgfscope}%
\begin{pgfscope}%
\pgfpathrectangle{\pgfqpoint{0.481978in}{0.331635in}}{\pgfqpoint{9.300000in}{7.700000in}}%
\pgfusepath{clip}%
\pgfsetbuttcap%
\pgfsetroundjoin%
\definecolor{currentfill}{rgb}{0.815686,0.733333,1.000000}%
\pgfsetfillcolor{currentfill}%
\pgfsetlinewidth{0.481800pt}%
\definecolor{currentstroke}{rgb}{1.000000,1.000000,1.000000}%
\pgfsetstrokecolor{currentstroke}%
\pgfsetdash{}{0pt}%
\pgfpathmoveto{\pgfqpoint{6.925838in}{2.873806in}}%
\pgfpathcurveto{\pgfqpoint{6.936889in}{2.873806in}}{\pgfqpoint{6.947488in}{2.878197in}}{\pgfqpoint{6.955301in}{2.886010in}}%
\pgfpathcurveto{\pgfqpoint{6.963115in}{2.893824in}}{\pgfqpoint{6.967505in}{2.904423in}}{\pgfqpoint{6.967505in}{2.915473in}}%
\pgfpathcurveto{\pgfqpoint{6.967505in}{2.926523in}}{\pgfqpoint{6.963115in}{2.937122in}}{\pgfqpoint{6.955301in}{2.944936in}}%
\pgfpathcurveto{\pgfqpoint{6.947488in}{2.952749in}}{\pgfqpoint{6.936889in}{2.957140in}}{\pgfqpoint{6.925838in}{2.957140in}}%
\pgfpathcurveto{\pgfqpoint{6.914788in}{2.957140in}}{\pgfqpoint{6.904189in}{2.952749in}}{\pgfqpoint{6.896376in}{2.944936in}}%
\pgfpathcurveto{\pgfqpoint{6.888562in}{2.937122in}}{\pgfqpoint{6.884172in}{2.926523in}}{\pgfqpoint{6.884172in}{2.915473in}}%
\pgfpathcurveto{\pgfqpoint{6.884172in}{2.904423in}}{\pgfqpoint{6.888562in}{2.893824in}}{\pgfqpoint{6.896376in}{2.886010in}}%
\pgfpathcurveto{\pgfqpoint{6.904189in}{2.878197in}}{\pgfqpoint{6.914788in}{2.873806in}}{\pgfqpoint{6.925838in}{2.873806in}}%
\pgfpathclose%
\pgfusepath{stroke,fill}%
\end{pgfscope}%
\begin{pgfscope}%
\pgfpathrectangle{\pgfqpoint{0.481978in}{0.331635in}}{\pgfqpoint{9.300000in}{7.700000in}}%
\pgfusepath{clip}%
\pgfsetbuttcap%
\pgfsetroundjoin%
\definecolor{currentfill}{rgb}{0.815686,0.733333,1.000000}%
\pgfsetfillcolor{currentfill}%
\pgfsetlinewidth{0.481800pt}%
\definecolor{currentstroke}{rgb}{1.000000,1.000000,1.000000}%
\pgfsetstrokecolor{currentstroke}%
\pgfsetdash{}{0pt}%
\pgfpathmoveto{\pgfqpoint{7.338785in}{3.293368in}}%
\pgfpathcurveto{\pgfqpoint{7.349835in}{3.293368in}}{\pgfqpoint{7.360434in}{3.297758in}}{\pgfqpoint{7.368247in}{3.305572in}}%
\pgfpathcurveto{\pgfqpoint{7.376061in}{3.313385in}}{\pgfqpoint{7.380451in}{3.323984in}}{\pgfqpoint{7.380451in}{3.335035in}}%
\pgfpathcurveto{\pgfqpoint{7.380451in}{3.346085in}}{\pgfqpoint{7.376061in}{3.356684in}}{\pgfqpoint{7.368247in}{3.364497in}}%
\pgfpathcurveto{\pgfqpoint{7.360434in}{3.372311in}}{\pgfqpoint{7.349835in}{3.376701in}}{\pgfqpoint{7.338785in}{3.376701in}}%
\pgfpathcurveto{\pgfqpoint{7.327734in}{3.376701in}}{\pgfqpoint{7.317135in}{3.372311in}}{\pgfqpoint{7.309322in}{3.364497in}}%
\pgfpathcurveto{\pgfqpoint{7.301508in}{3.356684in}}{\pgfqpoint{7.297118in}{3.346085in}}{\pgfqpoint{7.297118in}{3.335035in}}%
\pgfpathcurveto{\pgfqpoint{7.297118in}{3.323984in}}{\pgfqpoint{7.301508in}{3.313385in}}{\pgfqpoint{7.309322in}{3.305572in}}%
\pgfpathcurveto{\pgfqpoint{7.317135in}{3.297758in}}{\pgfqpoint{7.327734in}{3.293368in}}{\pgfqpoint{7.338785in}{3.293368in}}%
\pgfpathclose%
\pgfusepath{stroke,fill}%
\end{pgfscope}%
\begin{pgfscope}%
\pgfpathrectangle{\pgfqpoint{0.481978in}{0.331635in}}{\pgfqpoint{9.300000in}{7.700000in}}%
\pgfusepath{clip}%
\pgfsetbuttcap%
\pgfsetroundjoin%
\definecolor{currentfill}{rgb}{0.815686,0.733333,1.000000}%
\pgfsetfillcolor{currentfill}%
\pgfsetlinewidth{0.481800pt}%
\definecolor{currentstroke}{rgb}{1.000000,1.000000,1.000000}%
\pgfsetstrokecolor{currentstroke}%
\pgfsetdash{}{0pt}%
\pgfpathmoveto{\pgfqpoint{3.955717in}{4.093278in}}%
\pgfpathcurveto{\pgfqpoint{3.966767in}{4.093278in}}{\pgfqpoint{3.977366in}{4.097669in}}{\pgfqpoint{3.985180in}{4.105482in}}%
\pgfpathcurveto{\pgfqpoint{3.992994in}{4.113296in}}{\pgfqpoint{3.997384in}{4.123895in}}{\pgfqpoint{3.997384in}{4.134945in}}%
\pgfpathcurveto{\pgfqpoint{3.997384in}{4.145995in}}{\pgfqpoint{3.992994in}{4.156594in}}{\pgfqpoint{3.985180in}{4.164408in}}%
\pgfpathcurveto{\pgfqpoint{3.977366in}{4.172221in}}{\pgfqpoint{3.966767in}{4.176612in}}{\pgfqpoint{3.955717in}{4.176612in}}%
\pgfpathcurveto{\pgfqpoint{3.944667in}{4.176612in}}{\pgfqpoint{3.934068in}{4.172221in}}{\pgfqpoint{3.926254in}{4.164408in}}%
\pgfpathcurveto{\pgfqpoint{3.918441in}{4.156594in}}{\pgfqpoint{3.914050in}{4.145995in}}{\pgfqpoint{3.914050in}{4.134945in}}%
\pgfpathcurveto{\pgfqpoint{3.914050in}{4.123895in}}{\pgfqpoint{3.918441in}{4.113296in}}{\pgfqpoint{3.926254in}{4.105482in}}%
\pgfpathcurveto{\pgfqpoint{3.934068in}{4.097669in}}{\pgfqpoint{3.944667in}{4.093278in}}{\pgfqpoint{3.955717in}{4.093278in}}%
\pgfpathclose%
\pgfusepath{stroke,fill}%
\end{pgfscope}%
\begin{pgfscope}%
\pgfpathrectangle{\pgfqpoint{0.481978in}{0.331635in}}{\pgfqpoint{9.300000in}{7.700000in}}%
\pgfusepath{clip}%
\pgfsetbuttcap%
\pgfsetroundjoin%
\definecolor{currentfill}{rgb}{0.815686,0.733333,1.000000}%
\pgfsetfillcolor{currentfill}%
\pgfsetlinewidth{0.481800pt}%
\definecolor{currentstroke}{rgb}{1.000000,1.000000,1.000000}%
\pgfsetstrokecolor{currentstroke}%
\pgfsetdash{}{0pt}%
\pgfpathmoveto{\pgfqpoint{2.160067in}{1.882672in}}%
\pgfpathcurveto{\pgfqpoint{2.171117in}{1.882672in}}{\pgfqpoint{2.181717in}{1.887062in}}{\pgfqpoint{2.189530in}{1.894876in}}%
\pgfpathcurveto{\pgfqpoint{2.197344in}{1.902689in}}{\pgfqpoint{2.201734in}{1.913288in}}{\pgfqpoint{2.201734in}{1.924338in}}%
\pgfpathcurveto{\pgfqpoint{2.201734in}{1.935389in}}{\pgfqpoint{2.197344in}{1.945988in}}{\pgfqpoint{2.189530in}{1.953801in}}%
\pgfpathcurveto{\pgfqpoint{2.181717in}{1.961615in}}{\pgfqpoint{2.171117in}{1.966005in}}{\pgfqpoint{2.160067in}{1.966005in}}%
\pgfpathcurveto{\pgfqpoint{2.149017in}{1.966005in}}{\pgfqpoint{2.138418in}{1.961615in}}{\pgfqpoint{2.130605in}{1.953801in}}%
\pgfpathcurveto{\pgfqpoint{2.122791in}{1.945988in}}{\pgfqpoint{2.118401in}{1.935389in}}{\pgfqpoint{2.118401in}{1.924338in}}%
\pgfpathcurveto{\pgfqpoint{2.118401in}{1.913288in}}{\pgfqpoint{2.122791in}{1.902689in}}{\pgfqpoint{2.130605in}{1.894876in}}%
\pgfpathcurveto{\pgfqpoint{2.138418in}{1.887062in}}{\pgfqpoint{2.149017in}{1.882672in}}{\pgfqpoint{2.160067in}{1.882672in}}%
\pgfpathclose%
\pgfusepath{stroke,fill}%
\end{pgfscope}%
\begin{pgfscope}%
\pgfpathrectangle{\pgfqpoint{0.481978in}{0.331635in}}{\pgfqpoint{9.300000in}{7.700000in}}%
\pgfusepath{clip}%
\pgfsetbuttcap%
\pgfsetroundjoin%
\definecolor{currentfill}{rgb}{0.815686,0.733333,1.000000}%
\pgfsetfillcolor{currentfill}%
\pgfsetlinewidth{0.481800pt}%
\definecolor{currentstroke}{rgb}{1.000000,1.000000,1.000000}%
\pgfsetstrokecolor{currentstroke}%
\pgfsetdash{}{0pt}%
\pgfpathmoveto{\pgfqpoint{4.198073in}{4.221155in}}%
\pgfpathcurveto{\pgfqpoint{4.209123in}{4.221155in}}{\pgfqpoint{4.219722in}{4.225546in}}{\pgfqpoint{4.227535in}{4.233359in}}%
\pgfpathcurveto{\pgfqpoint{4.235349in}{4.241173in}}{\pgfqpoint{4.239739in}{4.251772in}}{\pgfqpoint{4.239739in}{4.262822in}}%
\pgfpathcurveto{\pgfqpoint{4.239739in}{4.273872in}}{\pgfqpoint{4.235349in}{4.284471in}}{\pgfqpoint{4.227535in}{4.292285in}}%
\pgfpathcurveto{\pgfqpoint{4.219722in}{4.300098in}}{\pgfqpoint{4.209123in}{4.304489in}}{\pgfqpoint{4.198073in}{4.304489in}}%
\pgfpathcurveto{\pgfqpoint{4.187023in}{4.304489in}}{\pgfqpoint{4.176424in}{4.300098in}}{\pgfqpoint{4.168610in}{4.292285in}}%
\pgfpathcurveto{\pgfqpoint{4.160796in}{4.284471in}}{\pgfqpoint{4.156406in}{4.273872in}}{\pgfqpoint{4.156406in}{4.262822in}}%
\pgfpathcurveto{\pgfqpoint{4.156406in}{4.251772in}}{\pgfqpoint{4.160796in}{4.241173in}}{\pgfqpoint{4.168610in}{4.233359in}}%
\pgfpathcurveto{\pgfqpoint{4.176424in}{4.225546in}}{\pgfqpoint{4.187023in}{4.221155in}}{\pgfqpoint{4.198073in}{4.221155in}}%
\pgfpathclose%
\pgfusepath{stroke,fill}%
\end{pgfscope}%
\begin{pgfscope}%
\pgfpathrectangle{\pgfqpoint{0.481978in}{0.331635in}}{\pgfqpoint{9.300000in}{7.700000in}}%
\pgfusepath{clip}%
\pgfsetbuttcap%
\pgfsetroundjoin%
\definecolor{currentfill}{rgb}{0.815686,0.733333,1.000000}%
\pgfsetfillcolor{currentfill}%
\pgfsetlinewidth{0.481800pt}%
\definecolor{currentstroke}{rgb}{1.000000,1.000000,1.000000}%
\pgfsetstrokecolor{currentstroke}%
\pgfsetdash{}{0pt}%
\pgfpathmoveto{\pgfqpoint{3.029876in}{4.890266in}}%
\pgfpathcurveto{\pgfqpoint{3.040926in}{4.890266in}}{\pgfqpoint{3.051525in}{4.894657in}}{\pgfqpoint{3.059339in}{4.902470in}}%
\pgfpathcurveto{\pgfqpoint{3.067152in}{4.910284in}}{\pgfqpoint{3.071542in}{4.920883in}}{\pgfqpoint{3.071542in}{4.931933in}}%
\pgfpathcurveto{\pgfqpoint{3.071542in}{4.942983in}}{\pgfqpoint{3.067152in}{4.953582in}}{\pgfqpoint{3.059339in}{4.961396in}}%
\pgfpathcurveto{\pgfqpoint{3.051525in}{4.969209in}}{\pgfqpoint{3.040926in}{4.973600in}}{\pgfqpoint{3.029876in}{4.973600in}}%
\pgfpathcurveto{\pgfqpoint{3.018826in}{4.973600in}}{\pgfqpoint{3.008227in}{4.969209in}}{\pgfqpoint{3.000413in}{4.961396in}}%
\pgfpathcurveto{\pgfqpoint{2.992599in}{4.953582in}}{\pgfqpoint{2.988209in}{4.942983in}}{\pgfqpoint{2.988209in}{4.931933in}}%
\pgfpathcurveto{\pgfqpoint{2.988209in}{4.920883in}}{\pgfqpoint{2.992599in}{4.910284in}}{\pgfqpoint{3.000413in}{4.902470in}}%
\pgfpathcurveto{\pgfqpoint{3.008227in}{4.894657in}}{\pgfqpoint{3.018826in}{4.890266in}}{\pgfqpoint{3.029876in}{4.890266in}}%
\pgfpathclose%
\pgfusepath{stroke,fill}%
\end{pgfscope}%
\begin{pgfscope}%
\pgfpathrectangle{\pgfqpoint{0.481978in}{0.331635in}}{\pgfqpoint{9.300000in}{7.700000in}}%
\pgfusepath{clip}%
\pgfsetbuttcap%
\pgfsetroundjoin%
\definecolor{currentfill}{rgb}{0.815686,0.733333,1.000000}%
\pgfsetfillcolor{currentfill}%
\pgfsetlinewidth{0.481800pt}%
\definecolor{currentstroke}{rgb}{1.000000,1.000000,1.000000}%
\pgfsetstrokecolor{currentstroke}%
\pgfsetdash{}{0pt}%
\pgfpathmoveto{\pgfqpoint{1.310658in}{4.217106in}}%
\pgfpathcurveto{\pgfqpoint{1.321708in}{4.217106in}}{\pgfqpoint{1.332307in}{4.221497in}}{\pgfqpoint{1.340121in}{4.229310in}}%
\pgfpathcurveto{\pgfqpoint{1.347934in}{4.237124in}}{\pgfqpoint{1.352325in}{4.247723in}}{\pgfqpoint{1.352325in}{4.258773in}}%
\pgfpathcurveto{\pgfqpoint{1.352325in}{4.269823in}}{\pgfqpoint{1.347934in}{4.280422in}}{\pgfqpoint{1.340121in}{4.288236in}}%
\pgfpathcurveto{\pgfqpoint{1.332307in}{4.296050in}}{\pgfqpoint{1.321708in}{4.300440in}}{\pgfqpoint{1.310658in}{4.300440in}}%
\pgfpathcurveto{\pgfqpoint{1.299608in}{4.300440in}}{\pgfqpoint{1.289009in}{4.296050in}}{\pgfqpoint{1.281195in}{4.288236in}}%
\pgfpathcurveto{\pgfqpoint{1.273382in}{4.280422in}}{\pgfqpoint{1.268991in}{4.269823in}}{\pgfqpoint{1.268991in}{4.258773in}}%
\pgfpathcurveto{\pgfqpoint{1.268991in}{4.247723in}}{\pgfqpoint{1.273382in}{4.237124in}}{\pgfqpoint{1.281195in}{4.229310in}}%
\pgfpathcurveto{\pgfqpoint{1.289009in}{4.221497in}}{\pgfqpoint{1.299608in}{4.217106in}}{\pgfqpoint{1.310658in}{4.217106in}}%
\pgfpathclose%
\pgfusepath{stroke,fill}%
\end{pgfscope}%
\begin{pgfscope}%
\pgfpathrectangle{\pgfqpoint{0.481978in}{0.331635in}}{\pgfqpoint{9.300000in}{7.700000in}}%
\pgfusepath{clip}%
\pgfsetbuttcap%
\pgfsetroundjoin%
\definecolor{currentfill}{rgb}{0.815686,0.733333,1.000000}%
\pgfsetfillcolor{currentfill}%
\pgfsetlinewidth{0.481800pt}%
\definecolor{currentstroke}{rgb}{1.000000,1.000000,1.000000}%
\pgfsetstrokecolor{currentstroke}%
\pgfsetdash{}{0pt}%
\pgfpathmoveto{\pgfqpoint{5.061821in}{4.668455in}}%
\pgfpathcurveto{\pgfqpoint{5.072872in}{4.668455in}}{\pgfqpoint{5.083471in}{4.672845in}}{\pgfqpoint{5.091284in}{4.680659in}}%
\pgfpathcurveto{\pgfqpoint{5.099098in}{4.688473in}}{\pgfqpoint{5.103488in}{4.699072in}}{\pgfqpoint{5.103488in}{4.710122in}}%
\pgfpathcurveto{\pgfqpoint{5.103488in}{4.721172in}}{\pgfqpoint{5.099098in}{4.731771in}}{\pgfqpoint{5.091284in}{4.739585in}}%
\pgfpathcurveto{\pgfqpoint{5.083471in}{4.747398in}}{\pgfqpoint{5.072872in}{4.751788in}}{\pgfqpoint{5.061821in}{4.751788in}}%
\pgfpathcurveto{\pgfqpoint{5.050771in}{4.751788in}}{\pgfqpoint{5.040172in}{4.747398in}}{\pgfqpoint{5.032359in}{4.739585in}}%
\pgfpathcurveto{\pgfqpoint{5.024545in}{4.731771in}}{\pgfqpoint{5.020155in}{4.721172in}}{\pgfqpoint{5.020155in}{4.710122in}}%
\pgfpathcurveto{\pgfqpoint{5.020155in}{4.699072in}}{\pgfqpoint{5.024545in}{4.688473in}}{\pgfqpoint{5.032359in}{4.680659in}}%
\pgfpathcurveto{\pgfqpoint{5.040172in}{4.672845in}}{\pgfqpoint{5.050771in}{4.668455in}}{\pgfqpoint{5.061821in}{4.668455in}}%
\pgfpathclose%
\pgfusepath{stroke,fill}%
\end{pgfscope}%
\begin{pgfscope}%
\pgfpathrectangle{\pgfqpoint{0.481978in}{0.331635in}}{\pgfqpoint{9.300000in}{7.700000in}}%
\pgfusepath{clip}%
\pgfsetbuttcap%
\pgfsetroundjoin%
\definecolor{currentfill}{rgb}{0.815686,0.733333,1.000000}%
\pgfsetfillcolor{currentfill}%
\pgfsetlinewidth{0.481800pt}%
\definecolor{currentstroke}{rgb}{1.000000,1.000000,1.000000}%
\pgfsetstrokecolor{currentstroke}%
\pgfsetdash{}{0pt}%
\pgfpathmoveto{\pgfqpoint{5.046020in}{5.012374in}}%
\pgfpathcurveto{\pgfqpoint{5.057070in}{5.012374in}}{\pgfqpoint{5.067669in}{5.016764in}}{\pgfqpoint{5.075482in}{5.024578in}}%
\pgfpathcurveto{\pgfqpoint{5.083296in}{5.032391in}}{\pgfqpoint{5.087686in}{5.042990in}}{\pgfqpoint{5.087686in}{5.054040in}}%
\pgfpathcurveto{\pgfqpoint{5.087686in}{5.065091in}}{\pgfqpoint{5.083296in}{5.075690in}}{\pgfqpoint{5.075482in}{5.083503in}}%
\pgfpathcurveto{\pgfqpoint{5.067669in}{5.091317in}}{\pgfqpoint{5.057070in}{5.095707in}}{\pgfqpoint{5.046020in}{5.095707in}}%
\pgfpathcurveto{\pgfqpoint{5.034970in}{5.095707in}}{\pgfqpoint{5.024371in}{5.091317in}}{\pgfqpoint{5.016557in}{5.083503in}}%
\pgfpathcurveto{\pgfqpoint{5.008743in}{5.075690in}}{\pgfqpoint{5.004353in}{5.065091in}}{\pgfqpoint{5.004353in}{5.054040in}}%
\pgfpathcurveto{\pgfqpoint{5.004353in}{5.042990in}}{\pgfqpoint{5.008743in}{5.032391in}}{\pgfqpoint{5.016557in}{5.024578in}}%
\pgfpathcurveto{\pgfqpoint{5.024371in}{5.016764in}}{\pgfqpoint{5.034970in}{5.012374in}}{\pgfqpoint{5.046020in}{5.012374in}}%
\pgfpathclose%
\pgfusepath{stroke,fill}%
\end{pgfscope}%
\begin{pgfscope}%
\pgfpathrectangle{\pgfqpoint{0.481978in}{0.331635in}}{\pgfqpoint{9.300000in}{7.700000in}}%
\pgfusepath{clip}%
\pgfsetbuttcap%
\pgfsetroundjoin%
\definecolor{currentfill}{rgb}{0.815686,0.733333,1.000000}%
\pgfsetfillcolor{currentfill}%
\pgfsetlinewidth{0.481800pt}%
\definecolor{currentstroke}{rgb}{1.000000,1.000000,1.000000}%
\pgfsetstrokecolor{currentstroke}%
\pgfsetdash{}{0pt}%
\pgfpathmoveto{\pgfqpoint{7.569943in}{3.637722in}}%
\pgfpathcurveto{\pgfqpoint{7.580993in}{3.637722in}}{\pgfqpoint{7.591592in}{3.642112in}}{\pgfqpoint{7.599406in}{3.649926in}}%
\pgfpathcurveto{\pgfqpoint{7.607219in}{3.657740in}}{\pgfqpoint{7.611610in}{3.668339in}}{\pgfqpoint{7.611610in}{3.679389in}}%
\pgfpathcurveto{\pgfqpoint{7.611610in}{3.690439in}}{\pgfqpoint{7.607219in}{3.701038in}}{\pgfqpoint{7.599406in}{3.708851in}}%
\pgfpathcurveto{\pgfqpoint{7.591592in}{3.716665in}}{\pgfqpoint{7.580993in}{3.721055in}}{\pgfqpoint{7.569943in}{3.721055in}}%
\pgfpathcurveto{\pgfqpoint{7.558893in}{3.721055in}}{\pgfqpoint{7.548294in}{3.716665in}}{\pgfqpoint{7.540480in}{3.708851in}}%
\pgfpathcurveto{\pgfqpoint{7.532667in}{3.701038in}}{\pgfqpoint{7.528276in}{3.690439in}}{\pgfqpoint{7.528276in}{3.679389in}}%
\pgfpathcurveto{\pgfqpoint{7.528276in}{3.668339in}}{\pgfqpoint{7.532667in}{3.657740in}}{\pgfqpoint{7.540480in}{3.649926in}}%
\pgfpathcurveto{\pgfqpoint{7.548294in}{3.642112in}}{\pgfqpoint{7.558893in}{3.637722in}}{\pgfqpoint{7.569943in}{3.637722in}}%
\pgfpathclose%
\pgfusepath{stroke,fill}%
\end{pgfscope}%
\begin{pgfscope}%
\pgfpathrectangle{\pgfqpoint{0.481978in}{0.331635in}}{\pgfqpoint{9.300000in}{7.700000in}}%
\pgfusepath{clip}%
\pgfsetbuttcap%
\pgfsetroundjoin%
\definecolor{currentfill}{rgb}{0.815686,0.733333,1.000000}%
\pgfsetfillcolor{currentfill}%
\pgfsetlinewidth{0.481800pt}%
\definecolor{currentstroke}{rgb}{1.000000,1.000000,1.000000}%
\pgfsetstrokecolor{currentstroke}%
\pgfsetdash{}{0pt}%
\pgfpathmoveto{\pgfqpoint{7.674879in}{3.033872in}}%
\pgfpathcurveto{\pgfqpoint{7.685929in}{3.033872in}}{\pgfqpoint{7.696528in}{3.038262in}}{\pgfqpoint{7.704342in}{3.046076in}}%
\pgfpathcurveto{\pgfqpoint{7.712155in}{3.053889in}}{\pgfqpoint{7.716546in}{3.064489in}}{\pgfqpoint{7.716546in}{3.075539in}}%
\pgfpathcurveto{\pgfqpoint{7.716546in}{3.086589in}}{\pgfqpoint{7.712155in}{3.097188in}}{\pgfqpoint{7.704342in}{3.105001in}}%
\pgfpathcurveto{\pgfqpoint{7.696528in}{3.112815in}}{\pgfqpoint{7.685929in}{3.117205in}}{\pgfqpoint{7.674879in}{3.117205in}}%
\pgfpathcurveto{\pgfqpoint{7.663829in}{3.117205in}}{\pgfqpoint{7.653230in}{3.112815in}}{\pgfqpoint{7.645416in}{3.105001in}}%
\pgfpathcurveto{\pgfqpoint{7.637603in}{3.097188in}}{\pgfqpoint{7.633212in}{3.086589in}}{\pgfqpoint{7.633212in}{3.075539in}}%
\pgfpathcurveto{\pgfqpoint{7.633212in}{3.064489in}}{\pgfqpoint{7.637603in}{3.053889in}}{\pgfqpoint{7.645416in}{3.046076in}}%
\pgfpathcurveto{\pgfqpoint{7.653230in}{3.038262in}}{\pgfqpoint{7.663829in}{3.033872in}}{\pgfqpoint{7.674879in}{3.033872in}}%
\pgfpathclose%
\pgfusepath{stroke,fill}%
\end{pgfscope}%
\begin{pgfscope}%
\pgfpathrectangle{\pgfqpoint{0.481978in}{0.331635in}}{\pgfqpoint{9.300000in}{7.700000in}}%
\pgfusepath{clip}%
\pgfsetbuttcap%
\pgfsetroundjoin%
\definecolor{currentfill}{rgb}{0.815686,0.733333,1.000000}%
\pgfsetfillcolor{currentfill}%
\pgfsetlinewidth{0.481800pt}%
\definecolor{currentstroke}{rgb}{1.000000,1.000000,1.000000}%
\pgfsetstrokecolor{currentstroke}%
\pgfsetdash{}{0pt}%
\pgfpathmoveto{\pgfqpoint{5.428071in}{3.072967in}}%
\pgfpathcurveto{\pgfqpoint{5.439121in}{3.072967in}}{\pgfqpoint{5.449720in}{3.077357in}}{\pgfqpoint{5.457534in}{3.085171in}}%
\pgfpathcurveto{\pgfqpoint{5.465347in}{3.092985in}}{\pgfqpoint{5.469737in}{3.103584in}}{\pgfqpoint{5.469737in}{3.114634in}}%
\pgfpathcurveto{\pgfqpoint{5.469737in}{3.125684in}}{\pgfqpoint{5.465347in}{3.136283in}}{\pgfqpoint{5.457534in}{3.144097in}}%
\pgfpathcurveto{\pgfqpoint{5.449720in}{3.151910in}}{\pgfqpoint{5.439121in}{3.156300in}}{\pgfqpoint{5.428071in}{3.156300in}}%
\pgfpathcurveto{\pgfqpoint{5.417021in}{3.156300in}}{\pgfqpoint{5.406422in}{3.151910in}}{\pgfqpoint{5.398608in}{3.144097in}}%
\pgfpathcurveto{\pgfqpoint{5.390794in}{3.136283in}}{\pgfqpoint{5.386404in}{3.125684in}}{\pgfqpoint{5.386404in}{3.114634in}}%
\pgfpathcurveto{\pgfqpoint{5.386404in}{3.103584in}}{\pgfqpoint{5.390794in}{3.092985in}}{\pgfqpoint{5.398608in}{3.085171in}}%
\pgfpathcurveto{\pgfqpoint{5.406422in}{3.077357in}}{\pgfqpoint{5.417021in}{3.072967in}}{\pgfqpoint{5.428071in}{3.072967in}}%
\pgfpathclose%
\pgfusepath{stroke,fill}%
\end{pgfscope}%
\begin{pgfscope}%
\pgfpathrectangle{\pgfqpoint{0.481978in}{0.331635in}}{\pgfqpoint{9.300000in}{7.700000in}}%
\pgfusepath{clip}%
\pgfsetbuttcap%
\pgfsetroundjoin%
\definecolor{currentfill}{rgb}{0.815686,0.733333,1.000000}%
\pgfsetfillcolor{currentfill}%
\pgfsetlinewidth{0.481800pt}%
\definecolor{currentstroke}{rgb}{1.000000,1.000000,1.000000}%
\pgfsetstrokecolor{currentstroke}%
\pgfsetdash{}{0pt}%
\pgfpathmoveto{\pgfqpoint{2.403350in}{3.601610in}}%
\pgfpathcurveto{\pgfqpoint{2.414400in}{3.601610in}}{\pgfqpoint{2.425000in}{3.606000in}}{\pgfqpoint{2.432813in}{3.613814in}}%
\pgfpathcurveto{\pgfqpoint{2.440627in}{3.621627in}}{\pgfqpoint{2.445017in}{3.632226in}}{\pgfqpoint{2.445017in}{3.643277in}}%
\pgfpathcurveto{\pgfqpoint{2.445017in}{3.654327in}}{\pgfqpoint{2.440627in}{3.664926in}}{\pgfqpoint{2.432813in}{3.672739in}}%
\pgfpathcurveto{\pgfqpoint{2.425000in}{3.680553in}}{\pgfqpoint{2.414400in}{3.684943in}}{\pgfqpoint{2.403350in}{3.684943in}}%
\pgfpathcurveto{\pgfqpoint{2.392300in}{3.684943in}}{\pgfqpoint{2.381701in}{3.680553in}}{\pgfqpoint{2.373888in}{3.672739in}}%
\pgfpathcurveto{\pgfqpoint{2.366074in}{3.664926in}}{\pgfqpoint{2.361684in}{3.654327in}}{\pgfqpoint{2.361684in}{3.643277in}}%
\pgfpathcurveto{\pgfqpoint{2.361684in}{3.632226in}}{\pgfqpoint{2.366074in}{3.621627in}}{\pgfqpoint{2.373888in}{3.613814in}}%
\pgfpathcurveto{\pgfqpoint{2.381701in}{3.606000in}}{\pgfqpoint{2.392300in}{3.601610in}}{\pgfqpoint{2.403350in}{3.601610in}}%
\pgfpathclose%
\pgfusepath{stroke,fill}%
\end{pgfscope}%
\begin{pgfscope}%
\pgfpathrectangle{\pgfqpoint{0.481978in}{0.331635in}}{\pgfqpoint{9.300000in}{7.700000in}}%
\pgfusepath{clip}%
\pgfsetbuttcap%
\pgfsetroundjoin%
\definecolor{currentfill}{rgb}{0.815686,0.733333,1.000000}%
\pgfsetfillcolor{currentfill}%
\pgfsetlinewidth{0.481800pt}%
\definecolor{currentstroke}{rgb}{1.000000,1.000000,1.000000}%
\pgfsetstrokecolor{currentstroke}%
\pgfsetdash{}{0pt}%
\pgfpathmoveto{\pgfqpoint{6.041612in}{5.592475in}}%
\pgfpathcurveto{\pgfqpoint{6.052662in}{5.592475in}}{\pgfqpoint{6.063261in}{5.596865in}}{\pgfqpoint{6.071075in}{5.604679in}}%
\pgfpathcurveto{\pgfqpoint{6.078889in}{5.612492in}}{\pgfqpoint{6.083279in}{5.623091in}}{\pgfqpoint{6.083279in}{5.634142in}}%
\pgfpathcurveto{\pgfqpoint{6.083279in}{5.645192in}}{\pgfqpoint{6.078889in}{5.655791in}}{\pgfqpoint{6.071075in}{5.663604in}}%
\pgfpathcurveto{\pgfqpoint{6.063261in}{5.671418in}}{\pgfqpoint{6.052662in}{5.675808in}}{\pgfqpoint{6.041612in}{5.675808in}}%
\pgfpathcurveto{\pgfqpoint{6.030562in}{5.675808in}}{\pgfqpoint{6.019963in}{5.671418in}}{\pgfqpoint{6.012150in}{5.663604in}}%
\pgfpathcurveto{\pgfqpoint{6.004336in}{5.655791in}}{\pgfqpoint{5.999946in}{5.645192in}}{\pgfqpoint{5.999946in}{5.634142in}}%
\pgfpathcurveto{\pgfqpoint{5.999946in}{5.623091in}}{\pgfqpoint{6.004336in}{5.612492in}}{\pgfqpoint{6.012150in}{5.604679in}}%
\pgfpathcurveto{\pgfqpoint{6.019963in}{5.596865in}}{\pgfqpoint{6.030562in}{5.592475in}}{\pgfqpoint{6.041612in}{5.592475in}}%
\pgfpathclose%
\pgfusepath{stroke,fill}%
\end{pgfscope}%
\begin{pgfscope}%
\pgfpathrectangle{\pgfqpoint{0.481978in}{0.331635in}}{\pgfqpoint{9.300000in}{7.700000in}}%
\pgfusepath{clip}%
\pgfsetbuttcap%
\pgfsetroundjoin%
\definecolor{currentfill}{rgb}{0.815686,0.733333,1.000000}%
\pgfsetfillcolor{currentfill}%
\pgfsetlinewidth{0.481800pt}%
\definecolor{currentstroke}{rgb}{1.000000,1.000000,1.000000}%
\pgfsetstrokecolor{currentstroke}%
\pgfsetdash{}{0pt}%
\pgfpathmoveto{\pgfqpoint{8.383007in}{6.162090in}}%
\pgfpathcurveto{\pgfqpoint{8.394058in}{6.162090in}}{\pgfqpoint{8.404657in}{6.166480in}}{\pgfqpoint{8.412470in}{6.174294in}}%
\pgfpathcurveto{\pgfqpoint{8.420284in}{6.182107in}}{\pgfqpoint{8.424674in}{6.192706in}}{\pgfqpoint{8.424674in}{6.203756in}}%
\pgfpathcurveto{\pgfqpoint{8.424674in}{6.214807in}}{\pgfqpoint{8.420284in}{6.225406in}}{\pgfqpoint{8.412470in}{6.233219in}}%
\pgfpathcurveto{\pgfqpoint{8.404657in}{6.241033in}}{\pgfqpoint{8.394058in}{6.245423in}}{\pgfqpoint{8.383007in}{6.245423in}}%
\pgfpathcurveto{\pgfqpoint{8.371957in}{6.245423in}}{\pgfqpoint{8.361358in}{6.241033in}}{\pgfqpoint{8.353545in}{6.233219in}}%
\pgfpathcurveto{\pgfqpoint{8.345731in}{6.225406in}}{\pgfqpoint{8.341341in}{6.214807in}}{\pgfqpoint{8.341341in}{6.203756in}}%
\pgfpathcurveto{\pgfqpoint{8.341341in}{6.192706in}}{\pgfqpoint{8.345731in}{6.182107in}}{\pgfqpoint{8.353545in}{6.174294in}}%
\pgfpathcurveto{\pgfqpoint{8.361358in}{6.166480in}}{\pgfqpoint{8.371957in}{6.162090in}}{\pgfqpoint{8.383007in}{6.162090in}}%
\pgfpathclose%
\pgfusepath{stroke,fill}%
\end{pgfscope}%
\begin{pgfscope}%
\pgfpathrectangle{\pgfqpoint{0.481978in}{0.331635in}}{\pgfqpoint{9.300000in}{7.700000in}}%
\pgfusepath{clip}%
\pgfsetbuttcap%
\pgfsetroundjoin%
\definecolor{currentfill}{rgb}{0.815686,0.733333,1.000000}%
\pgfsetfillcolor{currentfill}%
\pgfsetlinewidth{0.481800pt}%
\definecolor{currentstroke}{rgb}{1.000000,1.000000,1.000000}%
\pgfsetstrokecolor{currentstroke}%
\pgfsetdash{}{0pt}%
\pgfpathmoveto{\pgfqpoint{4.398769in}{4.525825in}}%
\pgfpathcurveto{\pgfqpoint{4.409819in}{4.525825in}}{\pgfqpoint{4.420418in}{4.530215in}}{\pgfqpoint{4.428231in}{4.538029in}}%
\pgfpathcurveto{\pgfqpoint{4.436045in}{4.545842in}}{\pgfqpoint{4.440435in}{4.556441in}}{\pgfqpoint{4.440435in}{4.567491in}}%
\pgfpathcurveto{\pgfqpoint{4.440435in}{4.578542in}}{\pgfqpoint{4.436045in}{4.589141in}}{\pgfqpoint{4.428231in}{4.596954in}}%
\pgfpathcurveto{\pgfqpoint{4.420418in}{4.604768in}}{\pgfqpoint{4.409819in}{4.609158in}}{\pgfqpoint{4.398769in}{4.609158in}}%
\pgfpathcurveto{\pgfqpoint{4.387718in}{4.609158in}}{\pgfqpoint{4.377119in}{4.604768in}}{\pgfqpoint{4.369306in}{4.596954in}}%
\pgfpathcurveto{\pgfqpoint{4.361492in}{4.589141in}}{\pgfqpoint{4.357102in}{4.578542in}}{\pgfqpoint{4.357102in}{4.567491in}}%
\pgfpathcurveto{\pgfqpoint{4.357102in}{4.556441in}}{\pgfqpoint{4.361492in}{4.545842in}}{\pgfqpoint{4.369306in}{4.538029in}}%
\pgfpathcurveto{\pgfqpoint{4.377119in}{4.530215in}}{\pgfqpoint{4.387718in}{4.525825in}}{\pgfqpoint{4.398769in}{4.525825in}}%
\pgfpathclose%
\pgfusepath{stroke,fill}%
\end{pgfscope}%
\begin{pgfscope}%
\pgfpathrectangle{\pgfqpoint{0.481978in}{0.331635in}}{\pgfqpoint{9.300000in}{7.700000in}}%
\pgfusepath{clip}%
\pgfsetbuttcap%
\pgfsetroundjoin%
\definecolor{currentfill}{rgb}{0.815686,0.733333,1.000000}%
\pgfsetfillcolor{currentfill}%
\pgfsetlinewidth{0.481800pt}%
\definecolor{currentstroke}{rgb}{1.000000,1.000000,1.000000}%
\pgfsetstrokecolor{currentstroke}%
\pgfsetdash{}{0pt}%
\pgfpathmoveto{\pgfqpoint{8.682490in}{5.777703in}}%
\pgfpathcurveto{\pgfqpoint{8.693540in}{5.777703in}}{\pgfqpoint{8.704139in}{5.782093in}}{\pgfqpoint{8.711953in}{5.789906in}}%
\pgfpathcurveto{\pgfqpoint{8.719766in}{5.797720in}}{\pgfqpoint{8.724157in}{5.808319in}}{\pgfqpoint{8.724157in}{5.819369in}}%
\pgfpathcurveto{\pgfqpoint{8.724157in}{5.830419in}}{\pgfqpoint{8.719766in}{5.841018in}}{\pgfqpoint{8.711953in}{5.848832in}}%
\pgfpathcurveto{\pgfqpoint{8.704139in}{5.856646in}}{\pgfqpoint{8.693540in}{5.861036in}}{\pgfqpoint{8.682490in}{5.861036in}}%
\pgfpathcurveto{\pgfqpoint{8.671440in}{5.861036in}}{\pgfqpoint{8.660841in}{5.856646in}}{\pgfqpoint{8.653027in}{5.848832in}}%
\pgfpathcurveto{\pgfqpoint{8.645213in}{5.841018in}}{\pgfqpoint{8.640823in}{5.830419in}}{\pgfqpoint{8.640823in}{5.819369in}}%
\pgfpathcurveto{\pgfqpoint{8.640823in}{5.808319in}}{\pgfqpoint{8.645213in}{5.797720in}}{\pgfqpoint{8.653027in}{5.789906in}}%
\pgfpathcurveto{\pgfqpoint{8.660841in}{5.782093in}}{\pgfqpoint{8.671440in}{5.777703in}}{\pgfqpoint{8.682490in}{5.777703in}}%
\pgfpathclose%
\pgfusepath{stroke,fill}%
\end{pgfscope}%
\begin{pgfscope}%
\pgfpathrectangle{\pgfqpoint{0.481978in}{0.331635in}}{\pgfqpoint{9.300000in}{7.700000in}}%
\pgfusepath{clip}%
\pgfsetbuttcap%
\pgfsetroundjoin%
\definecolor{currentfill}{rgb}{0.815686,0.733333,1.000000}%
\pgfsetfillcolor{currentfill}%
\pgfsetlinewidth{0.481800pt}%
\definecolor{currentstroke}{rgb}{1.000000,1.000000,1.000000}%
\pgfsetstrokecolor{currentstroke}%
\pgfsetdash{}{0pt}%
\pgfpathmoveto{\pgfqpoint{5.664598in}{1.113364in}}%
\pgfpathcurveto{\pgfqpoint{5.675648in}{1.113364in}}{\pgfqpoint{5.686248in}{1.117754in}}{\pgfqpoint{5.694061in}{1.125568in}}%
\pgfpathcurveto{\pgfqpoint{5.701875in}{1.133382in}}{\pgfqpoint{5.706265in}{1.143981in}}{\pgfqpoint{5.706265in}{1.155031in}}%
\pgfpathcurveto{\pgfqpoint{5.706265in}{1.166081in}}{\pgfqpoint{5.701875in}{1.176680in}}{\pgfqpoint{5.694061in}{1.184494in}}%
\pgfpathcurveto{\pgfqpoint{5.686248in}{1.192307in}}{\pgfqpoint{5.675648in}{1.196698in}}{\pgfqpoint{5.664598in}{1.196698in}}%
\pgfpathcurveto{\pgfqpoint{5.653548in}{1.196698in}}{\pgfqpoint{5.642949in}{1.192307in}}{\pgfqpoint{5.635136in}{1.184494in}}%
\pgfpathcurveto{\pgfqpoint{5.627322in}{1.176680in}}{\pgfqpoint{5.622932in}{1.166081in}}{\pgfqpoint{5.622932in}{1.155031in}}%
\pgfpathcurveto{\pgfqpoint{5.622932in}{1.143981in}}{\pgfqpoint{5.627322in}{1.133382in}}{\pgfqpoint{5.635136in}{1.125568in}}%
\pgfpathcurveto{\pgfqpoint{5.642949in}{1.117754in}}{\pgfqpoint{5.653548in}{1.113364in}}{\pgfqpoint{5.664598in}{1.113364in}}%
\pgfpathclose%
\pgfusepath{stroke,fill}%
\end{pgfscope}%
\begin{pgfscope}%
\pgfpathrectangle{\pgfqpoint{0.481978in}{0.331635in}}{\pgfqpoint{9.300000in}{7.700000in}}%
\pgfusepath{clip}%
\pgfsetbuttcap%
\pgfsetroundjoin%
\definecolor{currentfill}{rgb}{0.815686,0.733333,1.000000}%
\pgfsetfillcolor{currentfill}%
\pgfsetlinewidth{0.481800pt}%
\definecolor{currentstroke}{rgb}{1.000000,1.000000,1.000000}%
\pgfsetstrokecolor{currentstroke}%
\pgfsetdash{}{0pt}%
\pgfpathmoveto{\pgfqpoint{1.939227in}{4.311857in}}%
\pgfpathcurveto{\pgfqpoint{1.950277in}{4.311857in}}{\pgfqpoint{1.960876in}{4.316247in}}{\pgfqpoint{1.968690in}{4.324061in}}%
\pgfpathcurveto{\pgfqpoint{1.976504in}{4.331874in}}{\pgfqpoint{1.980894in}{4.342473in}}{\pgfqpoint{1.980894in}{4.353523in}}%
\pgfpathcurveto{\pgfqpoint{1.980894in}{4.364574in}}{\pgfqpoint{1.976504in}{4.375173in}}{\pgfqpoint{1.968690in}{4.382986in}}%
\pgfpathcurveto{\pgfqpoint{1.960876in}{4.390800in}}{\pgfqpoint{1.950277in}{4.395190in}}{\pgfqpoint{1.939227in}{4.395190in}}%
\pgfpathcurveto{\pgfqpoint{1.928177in}{4.395190in}}{\pgfqpoint{1.917578in}{4.390800in}}{\pgfqpoint{1.909765in}{4.382986in}}%
\pgfpathcurveto{\pgfqpoint{1.901951in}{4.375173in}}{\pgfqpoint{1.897561in}{4.364574in}}{\pgfqpoint{1.897561in}{4.353523in}}%
\pgfpathcurveto{\pgfqpoint{1.897561in}{4.342473in}}{\pgfqpoint{1.901951in}{4.331874in}}{\pgfqpoint{1.909765in}{4.324061in}}%
\pgfpathcurveto{\pgfqpoint{1.917578in}{4.316247in}}{\pgfqpoint{1.928177in}{4.311857in}}{\pgfqpoint{1.939227in}{4.311857in}}%
\pgfpathclose%
\pgfusepath{stroke,fill}%
\end{pgfscope}%
\begin{pgfscope}%
\pgfpathrectangle{\pgfqpoint{0.481978in}{0.331635in}}{\pgfqpoint{9.300000in}{7.700000in}}%
\pgfusepath{clip}%
\pgfsetbuttcap%
\pgfsetroundjoin%
\definecolor{currentfill}{rgb}{0.815686,0.733333,1.000000}%
\pgfsetfillcolor{currentfill}%
\pgfsetlinewidth{0.481800pt}%
\definecolor{currentstroke}{rgb}{1.000000,1.000000,1.000000}%
\pgfsetstrokecolor{currentstroke}%
\pgfsetdash{}{0pt}%
\pgfpathmoveto{\pgfqpoint{6.777031in}{4.407810in}}%
\pgfpathcurveto{\pgfqpoint{6.788081in}{4.407810in}}{\pgfqpoint{6.798680in}{4.412200in}}{\pgfqpoint{6.806494in}{4.420014in}}%
\pgfpathcurveto{\pgfqpoint{6.814308in}{4.427828in}}{\pgfqpoint{6.818698in}{4.438427in}}{\pgfqpoint{6.818698in}{4.449477in}}%
\pgfpathcurveto{\pgfqpoint{6.818698in}{4.460527in}}{\pgfqpoint{6.814308in}{4.471126in}}{\pgfqpoint{6.806494in}{4.478939in}}%
\pgfpathcurveto{\pgfqpoint{6.798680in}{4.486753in}}{\pgfqpoint{6.788081in}{4.491143in}}{\pgfqpoint{6.777031in}{4.491143in}}%
\pgfpathcurveto{\pgfqpoint{6.765981in}{4.491143in}}{\pgfqpoint{6.755382in}{4.486753in}}{\pgfqpoint{6.747568in}{4.478939in}}%
\pgfpathcurveto{\pgfqpoint{6.739755in}{4.471126in}}{\pgfqpoint{6.735365in}{4.460527in}}{\pgfqpoint{6.735365in}{4.449477in}}%
\pgfpathcurveto{\pgfqpoint{6.735365in}{4.438427in}}{\pgfqpoint{6.739755in}{4.427828in}}{\pgfqpoint{6.747568in}{4.420014in}}%
\pgfpathcurveto{\pgfqpoint{6.755382in}{4.412200in}}{\pgfqpoint{6.765981in}{4.407810in}}{\pgfqpoint{6.777031in}{4.407810in}}%
\pgfpathclose%
\pgfusepath{stroke,fill}%
\end{pgfscope}%
\begin{pgfscope}%
\pgfpathrectangle{\pgfqpoint{0.481978in}{0.331635in}}{\pgfqpoint{9.300000in}{7.700000in}}%
\pgfusepath{clip}%
\pgfsetbuttcap%
\pgfsetroundjoin%
\definecolor{currentfill}{rgb}{0.815686,0.733333,1.000000}%
\pgfsetfillcolor{currentfill}%
\pgfsetlinewidth{0.481800pt}%
\definecolor{currentstroke}{rgb}{1.000000,1.000000,1.000000}%
\pgfsetstrokecolor{currentstroke}%
\pgfsetdash{}{0pt}%
\pgfpathmoveto{\pgfqpoint{3.798724in}{2.123955in}}%
\pgfpathcurveto{\pgfqpoint{3.809775in}{2.123955in}}{\pgfqpoint{3.820374in}{2.128345in}}{\pgfqpoint{3.828187in}{2.136158in}}%
\pgfpathcurveto{\pgfqpoint{3.836001in}{2.143972in}}{\pgfqpoint{3.840391in}{2.154571in}}{\pgfqpoint{3.840391in}{2.165621in}}%
\pgfpathcurveto{\pgfqpoint{3.840391in}{2.176671in}}{\pgfqpoint{3.836001in}{2.187270in}}{\pgfqpoint{3.828187in}{2.195084in}}%
\pgfpathcurveto{\pgfqpoint{3.820374in}{2.202898in}}{\pgfqpoint{3.809775in}{2.207288in}}{\pgfqpoint{3.798724in}{2.207288in}}%
\pgfpathcurveto{\pgfqpoint{3.787674in}{2.207288in}}{\pgfqpoint{3.777075in}{2.202898in}}{\pgfqpoint{3.769262in}{2.195084in}}%
\pgfpathcurveto{\pgfqpoint{3.761448in}{2.187270in}}{\pgfqpoint{3.757058in}{2.176671in}}{\pgfqpoint{3.757058in}{2.165621in}}%
\pgfpathcurveto{\pgfqpoint{3.757058in}{2.154571in}}{\pgfqpoint{3.761448in}{2.143972in}}{\pgfqpoint{3.769262in}{2.136158in}}%
\pgfpathcurveto{\pgfqpoint{3.777075in}{2.128345in}}{\pgfqpoint{3.787674in}{2.123955in}}{\pgfqpoint{3.798724in}{2.123955in}}%
\pgfpathclose%
\pgfusepath{stroke,fill}%
\end{pgfscope}%
\begin{pgfscope}%
\pgfpathrectangle{\pgfqpoint{0.481978in}{0.331635in}}{\pgfqpoint{9.300000in}{7.700000in}}%
\pgfusepath{clip}%
\pgfsetbuttcap%
\pgfsetroundjoin%
\definecolor{currentfill}{rgb}{0.815686,0.733333,1.000000}%
\pgfsetfillcolor{currentfill}%
\pgfsetlinewidth{0.481800pt}%
\definecolor{currentstroke}{rgb}{1.000000,1.000000,1.000000}%
\pgfsetstrokecolor{currentstroke}%
\pgfsetdash{}{0pt}%
\pgfpathmoveto{\pgfqpoint{1.738451in}{3.862058in}}%
\pgfpathcurveto{\pgfqpoint{1.749501in}{3.862058in}}{\pgfqpoint{1.760100in}{3.866448in}}{\pgfqpoint{1.767914in}{3.874262in}}%
\pgfpathcurveto{\pgfqpoint{1.775728in}{3.882075in}}{\pgfqpoint{1.780118in}{3.892674in}}{\pgfqpoint{1.780118in}{3.903724in}}%
\pgfpathcurveto{\pgfqpoint{1.780118in}{3.914774in}}{\pgfqpoint{1.775728in}{3.925373in}}{\pgfqpoint{1.767914in}{3.933187in}}%
\pgfpathcurveto{\pgfqpoint{1.760100in}{3.941001in}}{\pgfqpoint{1.749501in}{3.945391in}}{\pgfqpoint{1.738451in}{3.945391in}}%
\pgfpathcurveto{\pgfqpoint{1.727401in}{3.945391in}}{\pgfqpoint{1.716802in}{3.941001in}}{\pgfqpoint{1.708989in}{3.933187in}}%
\pgfpathcurveto{\pgfqpoint{1.701175in}{3.925373in}}{\pgfqpoint{1.696785in}{3.914774in}}{\pgfqpoint{1.696785in}{3.903724in}}%
\pgfpathcurveto{\pgfqpoint{1.696785in}{3.892674in}}{\pgfqpoint{1.701175in}{3.882075in}}{\pgfqpoint{1.708989in}{3.874262in}}%
\pgfpathcurveto{\pgfqpoint{1.716802in}{3.866448in}}{\pgfqpoint{1.727401in}{3.862058in}}{\pgfqpoint{1.738451in}{3.862058in}}%
\pgfpathclose%
\pgfusepath{stroke,fill}%
\end{pgfscope}%
\begin{pgfscope}%
\pgfpathrectangle{\pgfqpoint{0.481978in}{0.331635in}}{\pgfqpoint{9.300000in}{7.700000in}}%
\pgfusepath{clip}%
\pgfsetbuttcap%
\pgfsetroundjoin%
\definecolor{currentfill}{rgb}{0.815686,0.733333,1.000000}%
\pgfsetfillcolor{currentfill}%
\pgfsetlinewidth{0.481800pt}%
\definecolor{currentstroke}{rgb}{1.000000,1.000000,1.000000}%
\pgfsetstrokecolor{currentstroke}%
\pgfsetdash{}{0pt}%
\pgfpathmoveto{\pgfqpoint{8.508088in}{3.964215in}}%
\pgfpathcurveto{\pgfqpoint{8.519138in}{3.964215in}}{\pgfqpoint{8.529737in}{3.968605in}}{\pgfqpoint{8.537550in}{3.976419in}}%
\pgfpathcurveto{\pgfqpoint{8.545364in}{3.984232in}}{\pgfqpoint{8.549754in}{3.994831in}}{\pgfqpoint{8.549754in}{4.005881in}}%
\pgfpathcurveto{\pgfqpoint{8.549754in}{4.016932in}}{\pgfqpoint{8.545364in}{4.027531in}}{\pgfqpoint{8.537550in}{4.035344in}}%
\pgfpathcurveto{\pgfqpoint{8.529737in}{4.043158in}}{\pgfqpoint{8.519138in}{4.047548in}}{\pgfqpoint{8.508088in}{4.047548in}}%
\pgfpathcurveto{\pgfqpoint{8.497037in}{4.047548in}}{\pgfqpoint{8.486438in}{4.043158in}}{\pgfqpoint{8.478625in}{4.035344in}}%
\pgfpathcurveto{\pgfqpoint{8.470811in}{4.027531in}}{\pgfqpoint{8.466421in}{4.016932in}}{\pgfqpoint{8.466421in}{4.005881in}}%
\pgfpathcurveto{\pgfqpoint{8.466421in}{3.994831in}}{\pgfqpoint{8.470811in}{3.984232in}}{\pgfqpoint{8.478625in}{3.976419in}}%
\pgfpathcurveto{\pgfqpoint{8.486438in}{3.968605in}}{\pgfqpoint{8.497037in}{3.964215in}}{\pgfqpoint{8.508088in}{3.964215in}}%
\pgfpathclose%
\pgfusepath{stroke,fill}%
\end{pgfscope}%
\begin{pgfscope}%
\pgfpathrectangle{\pgfqpoint{0.481978in}{0.331635in}}{\pgfqpoint{9.300000in}{7.700000in}}%
\pgfusepath{clip}%
\pgfsetbuttcap%
\pgfsetroundjoin%
\definecolor{currentfill}{rgb}{0.815686,0.733333,1.000000}%
\pgfsetfillcolor{currentfill}%
\pgfsetlinewidth{0.481800pt}%
\definecolor{currentstroke}{rgb}{1.000000,1.000000,1.000000}%
\pgfsetstrokecolor{currentstroke}%
\pgfsetdash{}{0pt}%
\pgfpathmoveto{\pgfqpoint{6.837191in}{4.849315in}}%
\pgfpathcurveto{\pgfqpoint{6.848241in}{4.849315in}}{\pgfqpoint{6.858840in}{4.853705in}}{\pgfqpoint{6.866654in}{4.861519in}}%
\pgfpathcurveto{\pgfqpoint{6.874468in}{4.869333in}}{\pgfqpoint{6.878858in}{4.879932in}}{\pgfqpoint{6.878858in}{4.890982in}}%
\pgfpathcurveto{\pgfqpoint{6.878858in}{4.902032in}}{\pgfqpoint{6.874468in}{4.912631in}}{\pgfqpoint{6.866654in}{4.920444in}}%
\pgfpathcurveto{\pgfqpoint{6.858840in}{4.928258in}}{\pgfqpoint{6.848241in}{4.932648in}}{\pgfqpoint{6.837191in}{4.932648in}}%
\pgfpathcurveto{\pgfqpoint{6.826141in}{4.932648in}}{\pgfqpoint{6.815542in}{4.928258in}}{\pgfqpoint{6.807729in}{4.920444in}}%
\pgfpathcurveto{\pgfqpoint{6.799915in}{4.912631in}}{\pgfqpoint{6.795525in}{4.902032in}}{\pgfqpoint{6.795525in}{4.890982in}}%
\pgfpathcurveto{\pgfqpoint{6.795525in}{4.879932in}}{\pgfqpoint{6.799915in}{4.869333in}}{\pgfqpoint{6.807729in}{4.861519in}}%
\pgfpathcurveto{\pgfqpoint{6.815542in}{4.853705in}}{\pgfqpoint{6.826141in}{4.849315in}}{\pgfqpoint{6.837191in}{4.849315in}}%
\pgfpathclose%
\pgfusepath{stroke,fill}%
\end{pgfscope}%
\begin{pgfscope}%
\pgfpathrectangle{\pgfqpoint{0.481978in}{0.331635in}}{\pgfqpoint{9.300000in}{7.700000in}}%
\pgfusepath{clip}%
\pgfsetbuttcap%
\pgfsetroundjoin%
\definecolor{currentfill}{rgb}{0.815686,0.733333,1.000000}%
\pgfsetfillcolor{currentfill}%
\pgfsetlinewidth{0.481800pt}%
\definecolor{currentstroke}{rgb}{1.000000,1.000000,1.000000}%
\pgfsetstrokecolor{currentstroke}%
\pgfsetdash{}{0pt}%
\pgfpathmoveto{\pgfqpoint{4.101155in}{3.717957in}}%
\pgfpathcurveto{\pgfqpoint{4.112206in}{3.717957in}}{\pgfqpoint{4.122805in}{3.722347in}}{\pgfqpoint{4.130618in}{3.730161in}}%
\pgfpathcurveto{\pgfqpoint{4.138432in}{3.737975in}}{\pgfqpoint{4.142822in}{3.748574in}}{\pgfqpoint{4.142822in}{3.759624in}}%
\pgfpathcurveto{\pgfqpoint{4.142822in}{3.770674in}}{\pgfqpoint{4.138432in}{3.781273in}}{\pgfqpoint{4.130618in}{3.789086in}}%
\pgfpathcurveto{\pgfqpoint{4.122805in}{3.796900in}}{\pgfqpoint{4.112206in}{3.801290in}}{\pgfqpoint{4.101155in}{3.801290in}}%
\pgfpathcurveto{\pgfqpoint{4.090105in}{3.801290in}}{\pgfqpoint{4.079506in}{3.796900in}}{\pgfqpoint{4.071693in}{3.789086in}}%
\pgfpathcurveto{\pgfqpoint{4.063879in}{3.781273in}}{\pgfqpoint{4.059489in}{3.770674in}}{\pgfqpoint{4.059489in}{3.759624in}}%
\pgfpathcurveto{\pgfqpoint{4.059489in}{3.748574in}}{\pgfqpoint{4.063879in}{3.737975in}}{\pgfqpoint{4.071693in}{3.730161in}}%
\pgfpathcurveto{\pgfqpoint{4.079506in}{3.722347in}}{\pgfqpoint{4.090105in}{3.717957in}}{\pgfqpoint{4.101155in}{3.717957in}}%
\pgfpathclose%
\pgfusepath{stroke,fill}%
\end{pgfscope}%
\begin{pgfscope}%
\pgfpathrectangle{\pgfqpoint{0.481978in}{0.331635in}}{\pgfqpoint{9.300000in}{7.700000in}}%
\pgfusepath{clip}%
\pgfsetbuttcap%
\pgfsetroundjoin%
\definecolor{currentfill}{rgb}{0.815686,0.733333,1.000000}%
\pgfsetfillcolor{currentfill}%
\pgfsetlinewidth{0.481800pt}%
\definecolor{currentstroke}{rgb}{1.000000,1.000000,1.000000}%
\pgfsetstrokecolor{currentstroke}%
\pgfsetdash{}{0pt}%
\pgfpathmoveto{\pgfqpoint{7.340850in}{3.253024in}}%
\pgfpathcurveto{\pgfqpoint{7.351900in}{3.253024in}}{\pgfqpoint{7.362499in}{3.257414in}}{\pgfqpoint{7.370313in}{3.265227in}}%
\pgfpathcurveto{\pgfqpoint{7.378127in}{3.273041in}}{\pgfqpoint{7.382517in}{3.283640in}}{\pgfqpoint{7.382517in}{3.294690in}}%
\pgfpathcurveto{\pgfqpoint{7.382517in}{3.305740in}}{\pgfqpoint{7.378127in}{3.316339in}}{\pgfqpoint{7.370313in}{3.324153in}}%
\pgfpathcurveto{\pgfqpoint{7.362499in}{3.331967in}}{\pgfqpoint{7.351900in}{3.336357in}}{\pgfqpoint{7.340850in}{3.336357in}}%
\pgfpathcurveto{\pgfqpoint{7.329800in}{3.336357in}}{\pgfqpoint{7.319201in}{3.331967in}}{\pgfqpoint{7.311387in}{3.324153in}}%
\pgfpathcurveto{\pgfqpoint{7.303574in}{3.316339in}}{\pgfqpoint{7.299184in}{3.305740in}}{\pgfqpoint{7.299184in}{3.294690in}}%
\pgfpathcurveto{\pgfqpoint{7.299184in}{3.283640in}}{\pgfqpoint{7.303574in}{3.273041in}}{\pgfqpoint{7.311387in}{3.265227in}}%
\pgfpathcurveto{\pgfqpoint{7.319201in}{3.257414in}}{\pgfqpoint{7.329800in}{3.253024in}}{\pgfqpoint{7.340850in}{3.253024in}}%
\pgfpathclose%
\pgfusepath{stroke,fill}%
\end{pgfscope}%
\begin{pgfscope}%
\pgfpathrectangle{\pgfqpoint{0.481978in}{0.331635in}}{\pgfqpoint{9.300000in}{7.700000in}}%
\pgfusepath{clip}%
\pgfsetbuttcap%
\pgfsetroundjoin%
\definecolor{currentfill}{rgb}{0.815686,0.733333,1.000000}%
\pgfsetfillcolor{currentfill}%
\pgfsetlinewidth{0.481800pt}%
\definecolor{currentstroke}{rgb}{1.000000,1.000000,1.000000}%
\pgfsetstrokecolor{currentstroke}%
\pgfsetdash{}{0pt}%
\pgfpathmoveto{\pgfqpoint{4.878178in}{5.062333in}}%
\pgfpathcurveto{\pgfqpoint{4.889228in}{5.062333in}}{\pgfqpoint{4.899827in}{5.066724in}}{\pgfqpoint{4.907641in}{5.074537in}}%
\pgfpathcurveto{\pgfqpoint{4.915454in}{5.082351in}}{\pgfqpoint{4.919845in}{5.092950in}}{\pgfqpoint{4.919845in}{5.104000in}}%
\pgfpathcurveto{\pgfqpoint{4.919845in}{5.115050in}}{\pgfqpoint{4.915454in}{5.125649in}}{\pgfqpoint{4.907641in}{5.133463in}}%
\pgfpathcurveto{\pgfqpoint{4.899827in}{5.141277in}}{\pgfqpoint{4.889228in}{5.145667in}}{\pgfqpoint{4.878178in}{5.145667in}}%
\pgfpathcurveto{\pgfqpoint{4.867128in}{5.145667in}}{\pgfqpoint{4.856529in}{5.141277in}}{\pgfqpoint{4.848715in}{5.133463in}}%
\pgfpathcurveto{\pgfqpoint{4.840902in}{5.125649in}}{\pgfqpoint{4.836511in}{5.115050in}}{\pgfqpoint{4.836511in}{5.104000in}}%
\pgfpathcurveto{\pgfqpoint{4.836511in}{5.092950in}}{\pgfqpoint{4.840902in}{5.082351in}}{\pgfqpoint{4.848715in}{5.074537in}}%
\pgfpathcurveto{\pgfqpoint{4.856529in}{5.066724in}}{\pgfqpoint{4.867128in}{5.062333in}}{\pgfqpoint{4.878178in}{5.062333in}}%
\pgfpathclose%
\pgfusepath{stroke,fill}%
\end{pgfscope}%
\begin{pgfscope}%
\pgfpathrectangle{\pgfqpoint{0.481978in}{0.331635in}}{\pgfqpoint{9.300000in}{7.700000in}}%
\pgfusepath{clip}%
\pgfsetbuttcap%
\pgfsetroundjoin%
\definecolor{currentfill}{rgb}{0.815686,0.733333,1.000000}%
\pgfsetfillcolor{currentfill}%
\pgfsetlinewidth{0.481800pt}%
\definecolor{currentstroke}{rgb}{1.000000,1.000000,1.000000}%
\pgfsetstrokecolor{currentstroke}%
\pgfsetdash{}{0pt}%
\pgfpathmoveto{\pgfqpoint{3.093874in}{4.847172in}}%
\pgfpathcurveto{\pgfqpoint{3.104924in}{4.847172in}}{\pgfqpoint{3.115523in}{4.851563in}}{\pgfqpoint{3.123337in}{4.859376in}}%
\pgfpathcurveto{\pgfqpoint{3.131150in}{4.867190in}}{\pgfqpoint{3.135541in}{4.877789in}}{\pgfqpoint{3.135541in}{4.888839in}}%
\pgfpathcurveto{\pgfqpoint{3.135541in}{4.899889in}}{\pgfqpoint{3.131150in}{4.910488in}}{\pgfqpoint{3.123337in}{4.918302in}}%
\pgfpathcurveto{\pgfqpoint{3.115523in}{4.926115in}}{\pgfqpoint{3.104924in}{4.930506in}}{\pgfqpoint{3.093874in}{4.930506in}}%
\pgfpathcurveto{\pgfqpoint{3.082824in}{4.930506in}}{\pgfqpoint{3.072225in}{4.926115in}}{\pgfqpoint{3.064411in}{4.918302in}}%
\pgfpathcurveto{\pgfqpoint{3.056598in}{4.910488in}}{\pgfqpoint{3.052207in}{4.899889in}}{\pgfqpoint{3.052207in}{4.888839in}}%
\pgfpathcurveto{\pgfqpoint{3.052207in}{4.877789in}}{\pgfqpoint{3.056598in}{4.867190in}}{\pgfqpoint{3.064411in}{4.859376in}}%
\pgfpathcurveto{\pgfqpoint{3.072225in}{4.851563in}}{\pgfqpoint{3.082824in}{4.847172in}}{\pgfqpoint{3.093874in}{4.847172in}}%
\pgfpathclose%
\pgfusepath{stroke,fill}%
\end{pgfscope}%
\begin{pgfscope}%
\pgfpathrectangle{\pgfqpoint{0.481978in}{0.331635in}}{\pgfqpoint{9.300000in}{7.700000in}}%
\pgfusepath{clip}%
\pgfsetbuttcap%
\pgfsetroundjoin%
\definecolor{currentfill}{rgb}{0.815686,0.733333,1.000000}%
\pgfsetfillcolor{currentfill}%
\pgfsetlinewidth{0.481800pt}%
\definecolor{currentstroke}{rgb}{1.000000,1.000000,1.000000}%
\pgfsetstrokecolor{currentstroke}%
\pgfsetdash{}{0pt}%
\pgfpathmoveto{\pgfqpoint{5.273114in}{1.865984in}}%
\pgfpathcurveto{\pgfqpoint{5.284164in}{1.865984in}}{\pgfqpoint{5.294763in}{1.870375in}}{\pgfqpoint{5.302577in}{1.878188in}}%
\pgfpathcurveto{\pgfqpoint{5.310390in}{1.886002in}}{\pgfqpoint{5.314781in}{1.896601in}}{\pgfqpoint{5.314781in}{1.907651in}}%
\pgfpathcurveto{\pgfqpoint{5.314781in}{1.918701in}}{\pgfqpoint{5.310390in}{1.929300in}}{\pgfqpoint{5.302577in}{1.937114in}}%
\pgfpathcurveto{\pgfqpoint{5.294763in}{1.944927in}}{\pgfqpoint{5.284164in}{1.949318in}}{\pgfqpoint{5.273114in}{1.949318in}}%
\pgfpathcurveto{\pgfqpoint{5.262064in}{1.949318in}}{\pgfqpoint{5.251465in}{1.944927in}}{\pgfqpoint{5.243651in}{1.937114in}}%
\pgfpathcurveto{\pgfqpoint{5.235838in}{1.929300in}}{\pgfqpoint{5.231447in}{1.918701in}}{\pgfqpoint{5.231447in}{1.907651in}}%
\pgfpathcurveto{\pgfqpoint{5.231447in}{1.896601in}}{\pgfqpoint{5.235838in}{1.886002in}}{\pgfqpoint{5.243651in}{1.878188in}}%
\pgfpathcurveto{\pgfqpoint{5.251465in}{1.870375in}}{\pgfqpoint{5.262064in}{1.865984in}}{\pgfqpoint{5.273114in}{1.865984in}}%
\pgfpathclose%
\pgfusepath{stroke,fill}%
\end{pgfscope}%
\begin{pgfscope}%
\pgfpathrectangle{\pgfqpoint{0.481978in}{0.331635in}}{\pgfqpoint{9.300000in}{7.700000in}}%
\pgfusepath{clip}%
\pgfsetbuttcap%
\pgfsetroundjoin%
\definecolor{currentfill}{rgb}{0.815686,0.733333,1.000000}%
\pgfsetfillcolor{currentfill}%
\pgfsetlinewidth{0.481800pt}%
\definecolor{currentstroke}{rgb}{1.000000,1.000000,1.000000}%
\pgfsetstrokecolor{currentstroke}%
\pgfsetdash{}{0pt}%
\pgfpathmoveto{\pgfqpoint{8.330701in}{5.607165in}}%
\pgfpathcurveto{\pgfqpoint{8.341751in}{5.607165in}}{\pgfqpoint{8.352350in}{5.611556in}}{\pgfqpoint{8.360163in}{5.619369in}}%
\pgfpathcurveto{\pgfqpoint{8.367977in}{5.627183in}}{\pgfqpoint{8.372367in}{5.637782in}}{\pgfqpoint{8.372367in}{5.648832in}}%
\pgfpathcurveto{\pgfqpoint{8.372367in}{5.659882in}}{\pgfqpoint{8.367977in}{5.670481in}}{\pgfqpoint{8.360163in}{5.678295in}}%
\pgfpathcurveto{\pgfqpoint{8.352350in}{5.686109in}}{\pgfqpoint{8.341751in}{5.690499in}}{\pgfqpoint{8.330701in}{5.690499in}}%
\pgfpathcurveto{\pgfqpoint{8.319651in}{5.690499in}}{\pgfqpoint{8.309052in}{5.686109in}}{\pgfqpoint{8.301238in}{5.678295in}}%
\pgfpathcurveto{\pgfqpoint{8.293424in}{5.670481in}}{\pgfqpoint{8.289034in}{5.659882in}}{\pgfqpoint{8.289034in}{5.648832in}}%
\pgfpathcurveto{\pgfqpoint{8.289034in}{5.637782in}}{\pgfqpoint{8.293424in}{5.627183in}}{\pgfqpoint{8.301238in}{5.619369in}}%
\pgfpathcurveto{\pgfqpoint{8.309052in}{5.611556in}}{\pgfqpoint{8.319651in}{5.607165in}}{\pgfqpoint{8.330701in}{5.607165in}}%
\pgfpathclose%
\pgfusepath{stroke,fill}%
\end{pgfscope}%
\begin{pgfscope}%
\pgfpathrectangle{\pgfqpoint{0.481978in}{0.331635in}}{\pgfqpoint{9.300000in}{7.700000in}}%
\pgfusepath{clip}%
\pgfsetbuttcap%
\pgfsetroundjoin%
\definecolor{currentfill}{rgb}{0.815686,0.733333,1.000000}%
\pgfsetfillcolor{currentfill}%
\pgfsetlinewidth{0.481800pt}%
\definecolor{currentstroke}{rgb}{1.000000,1.000000,1.000000}%
\pgfsetstrokecolor{currentstroke}%
\pgfsetdash{}{0pt}%
\pgfpathmoveto{\pgfqpoint{7.364685in}{2.898601in}}%
\pgfpathcurveto{\pgfqpoint{7.375735in}{2.898601in}}{\pgfqpoint{7.386334in}{2.902991in}}{\pgfqpoint{7.394148in}{2.910805in}}%
\pgfpathcurveto{\pgfqpoint{7.401961in}{2.918618in}}{\pgfqpoint{7.406351in}{2.929217in}}{\pgfqpoint{7.406351in}{2.940268in}}%
\pgfpathcurveto{\pgfqpoint{7.406351in}{2.951318in}}{\pgfqpoint{7.401961in}{2.961917in}}{\pgfqpoint{7.394148in}{2.969730in}}%
\pgfpathcurveto{\pgfqpoint{7.386334in}{2.977544in}}{\pgfqpoint{7.375735in}{2.981934in}}{\pgfqpoint{7.364685in}{2.981934in}}%
\pgfpathcurveto{\pgfqpoint{7.353635in}{2.981934in}}{\pgfqpoint{7.343036in}{2.977544in}}{\pgfqpoint{7.335222in}{2.969730in}}%
\pgfpathcurveto{\pgfqpoint{7.327408in}{2.961917in}}{\pgfqpoint{7.323018in}{2.951318in}}{\pgfqpoint{7.323018in}{2.940268in}}%
\pgfpathcurveto{\pgfqpoint{7.323018in}{2.929217in}}{\pgfqpoint{7.327408in}{2.918618in}}{\pgfqpoint{7.335222in}{2.910805in}}%
\pgfpathcurveto{\pgfqpoint{7.343036in}{2.902991in}}{\pgfqpoint{7.353635in}{2.898601in}}{\pgfqpoint{7.364685in}{2.898601in}}%
\pgfpathclose%
\pgfusepath{stroke,fill}%
\end{pgfscope}%
\begin{pgfscope}%
\pgfpathrectangle{\pgfqpoint{0.481978in}{0.331635in}}{\pgfqpoint{9.300000in}{7.700000in}}%
\pgfusepath{clip}%
\pgfsetbuttcap%
\pgfsetroundjoin%
\definecolor{currentfill}{rgb}{0.815686,0.733333,1.000000}%
\pgfsetfillcolor{currentfill}%
\pgfsetlinewidth{0.481800pt}%
\definecolor{currentstroke}{rgb}{1.000000,1.000000,1.000000}%
\pgfsetstrokecolor{currentstroke}%
\pgfsetdash{}{0pt}%
\pgfpathmoveto{\pgfqpoint{2.332471in}{2.474830in}}%
\pgfpathcurveto{\pgfqpoint{2.343521in}{2.474830in}}{\pgfqpoint{2.354120in}{2.479221in}}{\pgfqpoint{2.361934in}{2.487034in}}%
\pgfpathcurveto{\pgfqpoint{2.369747in}{2.494848in}}{\pgfqpoint{2.374138in}{2.505447in}}{\pgfqpoint{2.374138in}{2.516497in}}%
\pgfpathcurveto{\pgfqpoint{2.374138in}{2.527547in}}{\pgfqpoint{2.369747in}{2.538146in}}{\pgfqpoint{2.361934in}{2.545960in}}%
\pgfpathcurveto{\pgfqpoint{2.354120in}{2.553773in}}{\pgfqpoint{2.343521in}{2.558164in}}{\pgfqpoint{2.332471in}{2.558164in}}%
\pgfpathcurveto{\pgfqpoint{2.321421in}{2.558164in}}{\pgfqpoint{2.310822in}{2.553773in}}{\pgfqpoint{2.303008in}{2.545960in}}%
\pgfpathcurveto{\pgfqpoint{2.295195in}{2.538146in}}{\pgfqpoint{2.290804in}{2.527547in}}{\pgfqpoint{2.290804in}{2.516497in}}%
\pgfpathcurveto{\pgfqpoint{2.290804in}{2.505447in}}{\pgfqpoint{2.295195in}{2.494848in}}{\pgfqpoint{2.303008in}{2.487034in}}%
\pgfpathcurveto{\pgfqpoint{2.310822in}{2.479221in}}{\pgfqpoint{2.321421in}{2.474830in}}{\pgfqpoint{2.332471in}{2.474830in}}%
\pgfpathclose%
\pgfusepath{stroke,fill}%
\end{pgfscope}%
\begin{pgfscope}%
\pgfpathrectangle{\pgfqpoint{0.481978in}{0.331635in}}{\pgfqpoint{9.300000in}{7.700000in}}%
\pgfusepath{clip}%
\pgfsetbuttcap%
\pgfsetroundjoin%
\definecolor{currentfill}{rgb}{0.815686,0.733333,1.000000}%
\pgfsetfillcolor{currentfill}%
\pgfsetlinewidth{0.481800pt}%
\definecolor{currentstroke}{rgb}{1.000000,1.000000,1.000000}%
\pgfsetstrokecolor{currentstroke}%
\pgfsetdash{}{0pt}%
\pgfpathmoveto{\pgfqpoint{6.392422in}{4.189674in}}%
\pgfpathcurveto{\pgfqpoint{6.403472in}{4.189674in}}{\pgfqpoint{6.414071in}{4.194065in}}{\pgfqpoint{6.421884in}{4.201878in}}%
\pgfpathcurveto{\pgfqpoint{6.429698in}{4.209692in}}{\pgfqpoint{6.434088in}{4.220291in}}{\pgfqpoint{6.434088in}{4.231341in}}%
\pgfpathcurveto{\pgfqpoint{6.434088in}{4.242391in}}{\pgfqpoint{6.429698in}{4.252990in}}{\pgfqpoint{6.421884in}{4.260804in}}%
\pgfpathcurveto{\pgfqpoint{6.414071in}{4.268617in}}{\pgfqpoint{6.403472in}{4.273008in}}{\pgfqpoint{6.392422in}{4.273008in}}%
\pgfpathcurveto{\pgfqpoint{6.381371in}{4.273008in}}{\pgfqpoint{6.370772in}{4.268617in}}{\pgfqpoint{6.362959in}{4.260804in}}%
\pgfpathcurveto{\pgfqpoint{6.355145in}{4.252990in}}{\pgfqpoint{6.350755in}{4.242391in}}{\pgfqpoint{6.350755in}{4.231341in}}%
\pgfpathcurveto{\pgfqpoint{6.350755in}{4.220291in}}{\pgfqpoint{6.355145in}{4.209692in}}{\pgfqpoint{6.362959in}{4.201878in}}%
\pgfpathcurveto{\pgfqpoint{6.370772in}{4.194065in}}{\pgfqpoint{6.381371in}{4.189674in}}{\pgfqpoint{6.392422in}{4.189674in}}%
\pgfpathclose%
\pgfusepath{stroke,fill}%
\end{pgfscope}%
\begin{pgfscope}%
\pgfpathrectangle{\pgfqpoint{0.481978in}{0.331635in}}{\pgfqpoint{9.300000in}{7.700000in}}%
\pgfusepath{clip}%
\pgfsetbuttcap%
\pgfsetroundjoin%
\definecolor{currentfill}{rgb}{0.870588,0.733333,0.607843}%
\pgfsetfillcolor{currentfill}%
\pgfsetlinewidth{0.481800pt}%
\definecolor{currentstroke}{rgb}{1.000000,1.000000,1.000000}%
\pgfsetstrokecolor{currentstroke}%
\pgfsetdash{}{0pt}%
\pgfpathmoveto{\pgfqpoint{3.278906in}{4.335174in}}%
\pgfpathcurveto{\pgfqpoint{3.289956in}{4.335174in}}{\pgfqpoint{3.300555in}{4.339564in}}{\pgfqpoint{3.308368in}{4.347377in}}%
\pgfpathcurveto{\pgfqpoint{3.316182in}{4.355191in}}{\pgfqpoint{3.320572in}{4.365790in}}{\pgfqpoint{3.320572in}{4.376840in}}%
\pgfpathcurveto{\pgfqpoint{3.320572in}{4.387890in}}{\pgfqpoint{3.316182in}{4.398489in}}{\pgfqpoint{3.308368in}{4.406303in}}%
\pgfpathcurveto{\pgfqpoint{3.300555in}{4.414117in}}{\pgfqpoint{3.289956in}{4.418507in}}{\pgfqpoint{3.278906in}{4.418507in}}%
\pgfpathcurveto{\pgfqpoint{3.267856in}{4.418507in}}{\pgfqpoint{3.257257in}{4.414117in}}{\pgfqpoint{3.249443in}{4.406303in}}%
\pgfpathcurveto{\pgfqpoint{3.241629in}{4.398489in}}{\pgfqpoint{3.237239in}{4.387890in}}{\pgfqpoint{3.237239in}{4.376840in}}%
\pgfpathcurveto{\pgfqpoint{3.237239in}{4.365790in}}{\pgfqpoint{3.241629in}{4.355191in}}{\pgfqpoint{3.249443in}{4.347377in}}%
\pgfpathcurveto{\pgfqpoint{3.257257in}{4.339564in}}{\pgfqpoint{3.267856in}{4.335174in}}{\pgfqpoint{3.278906in}{4.335174in}}%
\pgfpathclose%
\pgfusepath{stroke,fill}%
\end{pgfscope}%
\begin{pgfscope}%
\pgfpathrectangle{\pgfqpoint{0.481978in}{0.331635in}}{\pgfqpoint{9.300000in}{7.700000in}}%
\pgfusepath{clip}%
\pgfsetbuttcap%
\pgfsetroundjoin%
\definecolor{currentfill}{rgb}{0.870588,0.733333,0.607843}%
\pgfsetfillcolor{currentfill}%
\pgfsetlinewidth{0.481800pt}%
\definecolor{currentstroke}{rgb}{1.000000,1.000000,1.000000}%
\pgfsetstrokecolor{currentstroke}%
\pgfsetdash{}{0pt}%
\pgfpathmoveto{\pgfqpoint{1.806406in}{3.472401in}}%
\pgfpathcurveto{\pgfqpoint{1.817456in}{3.472401in}}{\pgfqpoint{1.828055in}{3.476791in}}{\pgfqpoint{1.835868in}{3.484605in}}%
\pgfpathcurveto{\pgfqpoint{1.843682in}{3.492418in}}{\pgfqpoint{1.848072in}{3.503017in}}{\pgfqpoint{1.848072in}{3.514068in}}%
\pgfpathcurveto{\pgfqpoint{1.848072in}{3.525118in}}{\pgfqpoint{1.843682in}{3.535717in}}{\pgfqpoint{1.835868in}{3.543530in}}%
\pgfpathcurveto{\pgfqpoint{1.828055in}{3.551344in}}{\pgfqpoint{1.817456in}{3.555734in}}{\pgfqpoint{1.806406in}{3.555734in}}%
\pgfpathcurveto{\pgfqpoint{1.795355in}{3.555734in}}{\pgfqpoint{1.784756in}{3.551344in}}{\pgfqpoint{1.776943in}{3.543530in}}%
\pgfpathcurveto{\pgfqpoint{1.769129in}{3.535717in}}{\pgfqpoint{1.764739in}{3.525118in}}{\pgfqpoint{1.764739in}{3.514068in}}%
\pgfpathcurveto{\pgfqpoint{1.764739in}{3.503017in}}{\pgfqpoint{1.769129in}{3.492418in}}{\pgfqpoint{1.776943in}{3.484605in}}%
\pgfpathcurveto{\pgfqpoint{1.784756in}{3.476791in}}{\pgfqpoint{1.795355in}{3.472401in}}{\pgfqpoint{1.806406in}{3.472401in}}%
\pgfpathclose%
\pgfusepath{stroke,fill}%
\end{pgfscope}%
\begin{pgfscope}%
\pgfpathrectangle{\pgfqpoint{0.481978in}{0.331635in}}{\pgfqpoint{9.300000in}{7.700000in}}%
\pgfusepath{clip}%
\pgfsetbuttcap%
\pgfsetroundjoin%
\definecolor{currentfill}{rgb}{0.870588,0.733333,0.607843}%
\pgfsetfillcolor{currentfill}%
\pgfsetlinewidth{0.481800pt}%
\definecolor{currentstroke}{rgb}{1.000000,1.000000,1.000000}%
\pgfsetstrokecolor{currentstroke}%
\pgfsetdash{}{0pt}%
\pgfpathmoveto{\pgfqpoint{3.901522in}{2.710673in}}%
\pgfpathcurveto{\pgfqpoint{3.912572in}{2.710673in}}{\pgfqpoint{3.923171in}{2.715064in}}{\pgfqpoint{3.930984in}{2.722877in}}%
\pgfpathcurveto{\pgfqpoint{3.938798in}{2.730691in}}{\pgfqpoint{3.943188in}{2.741290in}}{\pgfqpoint{3.943188in}{2.752340in}}%
\pgfpathcurveto{\pgfqpoint{3.943188in}{2.763390in}}{\pgfqpoint{3.938798in}{2.773989in}}{\pgfqpoint{3.930984in}{2.781803in}}%
\pgfpathcurveto{\pgfqpoint{3.923171in}{2.789616in}}{\pgfqpoint{3.912572in}{2.794007in}}{\pgfqpoint{3.901522in}{2.794007in}}%
\pgfpathcurveto{\pgfqpoint{3.890472in}{2.794007in}}{\pgfqpoint{3.879873in}{2.789616in}}{\pgfqpoint{3.872059in}{2.781803in}}%
\pgfpathcurveto{\pgfqpoint{3.864245in}{2.773989in}}{\pgfqpoint{3.859855in}{2.763390in}}{\pgfqpoint{3.859855in}{2.752340in}}%
\pgfpathcurveto{\pgfqpoint{3.859855in}{2.741290in}}{\pgfqpoint{3.864245in}{2.730691in}}{\pgfqpoint{3.872059in}{2.722877in}}%
\pgfpathcurveto{\pgfqpoint{3.879873in}{2.715064in}}{\pgfqpoint{3.890472in}{2.710673in}}{\pgfqpoint{3.901522in}{2.710673in}}%
\pgfpathclose%
\pgfusepath{stroke,fill}%
\end{pgfscope}%
\begin{pgfscope}%
\pgfpathrectangle{\pgfqpoint{0.481978in}{0.331635in}}{\pgfqpoint{9.300000in}{7.700000in}}%
\pgfusepath{clip}%
\pgfsetbuttcap%
\pgfsetroundjoin%
\definecolor{currentfill}{rgb}{0.870588,0.733333,0.607843}%
\pgfsetfillcolor{currentfill}%
\pgfsetlinewidth{0.481800pt}%
\definecolor{currentstroke}{rgb}{1.000000,1.000000,1.000000}%
\pgfsetstrokecolor{currentstroke}%
\pgfsetdash{}{0pt}%
\pgfpathmoveto{\pgfqpoint{2.577647in}{1.919824in}}%
\pgfpathcurveto{\pgfqpoint{2.588697in}{1.919824in}}{\pgfqpoint{2.599296in}{1.924214in}}{\pgfqpoint{2.607110in}{1.932027in}}%
\pgfpathcurveto{\pgfqpoint{2.614923in}{1.939841in}}{\pgfqpoint{2.619314in}{1.950440in}}{\pgfqpoint{2.619314in}{1.961490in}}%
\pgfpathcurveto{\pgfqpoint{2.619314in}{1.972540in}}{\pgfqpoint{2.614923in}{1.983139in}}{\pgfqpoint{2.607110in}{1.990953in}}%
\pgfpathcurveto{\pgfqpoint{2.599296in}{1.998767in}}{\pgfqpoint{2.588697in}{2.003157in}}{\pgfqpoint{2.577647in}{2.003157in}}%
\pgfpathcurveto{\pgfqpoint{2.566597in}{2.003157in}}{\pgfqpoint{2.555998in}{1.998767in}}{\pgfqpoint{2.548184in}{1.990953in}}%
\pgfpathcurveto{\pgfqpoint{2.540371in}{1.983139in}}{\pgfqpoint{2.535980in}{1.972540in}}{\pgfqpoint{2.535980in}{1.961490in}}%
\pgfpathcurveto{\pgfqpoint{2.535980in}{1.950440in}}{\pgfqpoint{2.540371in}{1.939841in}}{\pgfqpoint{2.548184in}{1.932027in}}%
\pgfpathcurveto{\pgfqpoint{2.555998in}{1.924214in}}{\pgfqpoint{2.566597in}{1.919824in}}{\pgfqpoint{2.577647in}{1.919824in}}%
\pgfpathclose%
\pgfusepath{stroke,fill}%
\end{pgfscope}%
\begin{pgfscope}%
\pgfpathrectangle{\pgfqpoint{0.481978in}{0.331635in}}{\pgfqpoint{9.300000in}{7.700000in}}%
\pgfusepath{clip}%
\pgfsetbuttcap%
\pgfsetroundjoin%
\definecolor{currentfill}{rgb}{0.870588,0.733333,0.607843}%
\pgfsetfillcolor{currentfill}%
\pgfsetlinewidth{0.481800pt}%
\definecolor{currentstroke}{rgb}{1.000000,1.000000,1.000000}%
\pgfsetstrokecolor{currentstroke}%
\pgfsetdash{}{0pt}%
\pgfpathmoveto{\pgfqpoint{2.542882in}{4.387959in}}%
\pgfpathcurveto{\pgfqpoint{2.553932in}{4.387959in}}{\pgfqpoint{2.564531in}{4.392349in}}{\pgfqpoint{2.572344in}{4.400162in}}%
\pgfpathcurveto{\pgfqpoint{2.580158in}{4.407976in}}{\pgfqpoint{2.584548in}{4.418575in}}{\pgfqpoint{2.584548in}{4.429625in}}%
\pgfpathcurveto{\pgfqpoint{2.584548in}{4.440675in}}{\pgfqpoint{2.580158in}{4.451274in}}{\pgfqpoint{2.572344in}{4.459088in}}%
\pgfpathcurveto{\pgfqpoint{2.564531in}{4.466902in}}{\pgfqpoint{2.553932in}{4.471292in}}{\pgfqpoint{2.542882in}{4.471292in}}%
\pgfpathcurveto{\pgfqpoint{2.531831in}{4.471292in}}{\pgfqpoint{2.521232in}{4.466902in}}{\pgfqpoint{2.513419in}{4.459088in}}%
\pgfpathcurveto{\pgfqpoint{2.505605in}{4.451274in}}{\pgfqpoint{2.501215in}{4.440675in}}{\pgfqpoint{2.501215in}{4.429625in}}%
\pgfpathcurveto{\pgfqpoint{2.501215in}{4.418575in}}{\pgfqpoint{2.505605in}{4.407976in}}{\pgfqpoint{2.513419in}{4.400162in}}%
\pgfpathcurveto{\pgfqpoint{2.521232in}{4.392349in}}{\pgfqpoint{2.531831in}{4.387959in}}{\pgfqpoint{2.542882in}{4.387959in}}%
\pgfpathclose%
\pgfusepath{stroke,fill}%
\end{pgfscope}%
\begin{pgfscope}%
\pgfpathrectangle{\pgfqpoint{0.481978in}{0.331635in}}{\pgfqpoint{9.300000in}{7.700000in}}%
\pgfusepath{clip}%
\pgfsetbuttcap%
\pgfsetroundjoin%
\definecolor{currentfill}{rgb}{0.870588,0.733333,0.607843}%
\pgfsetfillcolor{currentfill}%
\pgfsetlinewidth{0.481800pt}%
\definecolor{currentstroke}{rgb}{1.000000,1.000000,1.000000}%
\pgfsetstrokecolor{currentstroke}%
\pgfsetdash{}{0pt}%
\pgfpathmoveto{\pgfqpoint{6.100140in}{5.036268in}}%
\pgfpathcurveto{\pgfqpoint{6.111190in}{5.036268in}}{\pgfqpoint{6.121789in}{5.040658in}}{\pgfqpoint{6.129603in}{5.048472in}}%
\pgfpathcurveto{\pgfqpoint{6.137416in}{5.056285in}}{\pgfqpoint{6.141806in}{5.066884in}}{\pgfqpoint{6.141806in}{5.077934in}}%
\pgfpathcurveto{\pgfqpoint{6.141806in}{5.088984in}}{\pgfqpoint{6.137416in}{5.099583in}}{\pgfqpoint{6.129603in}{5.107397in}}%
\pgfpathcurveto{\pgfqpoint{6.121789in}{5.115211in}}{\pgfqpoint{6.111190in}{5.119601in}}{\pgfqpoint{6.100140in}{5.119601in}}%
\pgfpathcurveto{\pgfqpoint{6.089090in}{5.119601in}}{\pgfqpoint{6.078491in}{5.115211in}}{\pgfqpoint{6.070677in}{5.107397in}}%
\pgfpathcurveto{\pgfqpoint{6.062863in}{5.099583in}}{\pgfqpoint{6.058473in}{5.088984in}}{\pgfqpoint{6.058473in}{5.077934in}}%
\pgfpathcurveto{\pgfqpoint{6.058473in}{5.066884in}}{\pgfqpoint{6.062863in}{5.056285in}}{\pgfqpoint{6.070677in}{5.048472in}}%
\pgfpathcurveto{\pgfqpoint{6.078491in}{5.040658in}}{\pgfqpoint{6.089090in}{5.036268in}}{\pgfqpoint{6.100140in}{5.036268in}}%
\pgfpathclose%
\pgfusepath{stroke,fill}%
\end{pgfscope}%
\begin{pgfscope}%
\pgfpathrectangle{\pgfqpoint{0.481978in}{0.331635in}}{\pgfqpoint{9.300000in}{7.700000in}}%
\pgfusepath{clip}%
\pgfsetbuttcap%
\pgfsetroundjoin%
\definecolor{currentfill}{rgb}{0.870588,0.733333,0.607843}%
\pgfsetfillcolor{currentfill}%
\pgfsetlinewidth{0.481800pt}%
\definecolor{currentstroke}{rgb}{1.000000,1.000000,1.000000}%
\pgfsetstrokecolor{currentstroke}%
\pgfsetdash{}{0pt}%
\pgfpathmoveto{\pgfqpoint{4.111074in}{4.530151in}}%
\pgfpathcurveto{\pgfqpoint{4.122124in}{4.530151in}}{\pgfqpoint{4.132723in}{4.534541in}}{\pgfqpoint{4.140537in}{4.542355in}}%
\pgfpathcurveto{\pgfqpoint{4.148351in}{4.550168in}}{\pgfqpoint{4.152741in}{4.560767in}}{\pgfqpoint{4.152741in}{4.571818in}}%
\pgfpathcurveto{\pgfqpoint{4.152741in}{4.582868in}}{\pgfqpoint{4.148351in}{4.593467in}}{\pgfqpoint{4.140537in}{4.601280in}}%
\pgfpathcurveto{\pgfqpoint{4.132723in}{4.609094in}}{\pgfqpoint{4.122124in}{4.613484in}}{\pgfqpoint{4.111074in}{4.613484in}}%
\pgfpathcurveto{\pgfqpoint{4.100024in}{4.613484in}}{\pgfqpoint{4.089425in}{4.609094in}}{\pgfqpoint{4.081611in}{4.601280in}}%
\pgfpathcurveto{\pgfqpoint{4.073798in}{4.593467in}}{\pgfqpoint{4.069408in}{4.582868in}}{\pgfqpoint{4.069408in}{4.571818in}}%
\pgfpathcurveto{\pgfqpoint{4.069408in}{4.560767in}}{\pgfqpoint{4.073798in}{4.550168in}}{\pgfqpoint{4.081611in}{4.542355in}}%
\pgfpathcurveto{\pgfqpoint{4.089425in}{4.534541in}}{\pgfqpoint{4.100024in}{4.530151in}}{\pgfqpoint{4.111074in}{4.530151in}}%
\pgfpathclose%
\pgfusepath{stroke,fill}%
\end{pgfscope}%
\begin{pgfscope}%
\pgfpathrectangle{\pgfqpoint{0.481978in}{0.331635in}}{\pgfqpoint{9.300000in}{7.700000in}}%
\pgfusepath{clip}%
\pgfsetbuttcap%
\pgfsetroundjoin%
\definecolor{currentfill}{rgb}{0.870588,0.733333,0.607843}%
\pgfsetfillcolor{currentfill}%
\pgfsetlinewidth{0.481800pt}%
\definecolor{currentstroke}{rgb}{1.000000,1.000000,1.000000}%
\pgfsetstrokecolor{currentstroke}%
\pgfsetdash{}{0pt}%
\pgfpathmoveto{\pgfqpoint{2.567413in}{4.041877in}}%
\pgfpathcurveto{\pgfqpoint{2.578463in}{4.041877in}}{\pgfqpoint{2.589062in}{4.046267in}}{\pgfqpoint{2.596875in}{4.054081in}}%
\pgfpathcurveto{\pgfqpoint{2.604689in}{4.061894in}}{\pgfqpoint{2.609079in}{4.072493in}}{\pgfqpoint{2.609079in}{4.083544in}}%
\pgfpathcurveto{\pgfqpoint{2.609079in}{4.094594in}}{\pgfqpoint{2.604689in}{4.105193in}}{\pgfqpoint{2.596875in}{4.113006in}}%
\pgfpathcurveto{\pgfqpoint{2.589062in}{4.120820in}}{\pgfqpoint{2.578463in}{4.125210in}}{\pgfqpoint{2.567413in}{4.125210in}}%
\pgfpathcurveto{\pgfqpoint{2.556363in}{4.125210in}}{\pgfqpoint{2.545763in}{4.120820in}}{\pgfqpoint{2.537950in}{4.113006in}}%
\pgfpathcurveto{\pgfqpoint{2.530136in}{4.105193in}}{\pgfqpoint{2.525746in}{4.094594in}}{\pgfqpoint{2.525746in}{4.083544in}}%
\pgfpathcurveto{\pgfqpoint{2.525746in}{4.072493in}}{\pgfqpoint{2.530136in}{4.061894in}}{\pgfqpoint{2.537950in}{4.054081in}}%
\pgfpathcurveto{\pgfqpoint{2.545763in}{4.046267in}}{\pgfqpoint{2.556363in}{4.041877in}}{\pgfqpoint{2.567413in}{4.041877in}}%
\pgfpathclose%
\pgfusepath{stroke,fill}%
\end{pgfscope}%
\begin{pgfscope}%
\pgfpathrectangle{\pgfqpoint{0.481978in}{0.331635in}}{\pgfqpoint{9.300000in}{7.700000in}}%
\pgfusepath{clip}%
\pgfsetbuttcap%
\pgfsetroundjoin%
\definecolor{currentfill}{rgb}{0.870588,0.733333,0.607843}%
\pgfsetfillcolor{currentfill}%
\pgfsetlinewidth{0.481800pt}%
\definecolor{currentstroke}{rgb}{1.000000,1.000000,1.000000}%
\pgfsetstrokecolor{currentstroke}%
\pgfsetdash{}{0pt}%
\pgfpathmoveto{\pgfqpoint{2.475746in}{3.962700in}}%
\pgfpathcurveto{\pgfqpoint{2.486796in}{3.962700in}}{\pgfqpoint{2.497395in}{3.967090in}}{\pgfqpoint{2.505209in}{3.974904in}}%
\pgfpathcurveto{\pgfqpoint{2.513022in}{3.982717in}}{\pgfqpoint{2.517413in}{3.993316in}}{\pgfqpoint{2.517413in}{4.004366in}}%
\pgfpathcurveto{\pgfqpoint{2.517413in}{4.015417in}}{\pgfqpoint{2.513022in}{4.026016in}}{\pgfqpoint{2.505209in}{4.033829in}}%
\pgfpathcurveto{\pgfqpoint{2.497395in}{4.041643in}}{\pgfqpoint{2.486796in}{4.046033in}}{\pgfqpoint{2.475746in}{4.046033in}}%
\pgfpathcurveto{\pgfqpoint{2.464696in}{4.046033in}}{\pgfqpoint{2.454097in}{4.041643in}}{\pgfqpoint{2.446283in}{4.033829in}}%
\pgfpathcurveto{\pgfqpoint{2.438470in}{4.026016in}}{\pgfqpoint{2.434079in}{4.015417in}}{\pgfqpoint{2.434079in}{4.004366in}}%
\pgfpathcurveto{\pgfqpoint{2.434079in}{3.993316in}}{\pgfqpoint{2.438470in}{3.982717in}}{\pgfqpoint{2.446283in}{3.974904in}}%
\pgfpathcurveto{\pgfqpoint{2.454097in}{3.967090in}}{\pgfqpoint{2.464696in}{3.962700in}}{\pgfqpoint{2.475746in}{3.962700in}}%
\pgfpathclose%
\pgfusepath{stroke,fill}%
\end{pgfscope}%
\begin{pgfscope}%
\pgfpathrectangle{\pgfqpoint{0.481978in}{0.331635in}}{\pgfqpoint{9.300000in}{7.700000in}}%
\pgfusepath{clip}%
\pgfsetbuttcap%
\pgfsetroundjoin%
\definecolor{currentfill}{rgb}{0.870588,0.733333,0.607843}%
\pgfsetfillcolor{currentfill}%
\pgfsetlinewidth{0.481800pt}%
\definecolor{currentstroke}{rgb}{1.000000,1.000000,1.000000}%
\pgfsetstrokecolor{currentstroke}%
\pgfsetdash{}{0pt}%
\pgfpathmoveto{\pgfqpoint{7.321265in}{5.989130in}}%
\pgfpathcurveto{\pgfqpoint{7.332315in}{5.989130in}}{\pgfqpoint{7.342914in}{5.993521in}}{\pgfqpoint{7.350728in}{6.001334in}}%
\pgfpathcurveto{\pgfqpoint{7.358541in}{6.009148in}}{\pgfqpoint{7.362932in}{6.019747in}}{\pgfqpoint{7.362932in}{6.030797in}}%
\pgfpathcurveto{\pgfqpoint{7.362932in}{6.041847in}}{\pgfqpoint{7.358541in}{6.052446in}}{\pgfqpoint{7.350728in}{6.060260in}}%
\pgfpathcurveto{\pgfqpoint{7.342914in}{6.068073in}}{\pgfqpoint{7.332315in}{6.072464in}}{\pgfqpoint{7.321265in}{6.072464in}}%
\pgfpathcurveto{\pgfqpoint{7.310215in}{6.072464in}}{\pgfqpoint{7.299616in}{6.068073in}}{\pgfqpoint{7.291802in}{6.060260in}}%
\pgfpathcurveto{\pgfqpoint{7.283988in}{6.052446in}}{\pgfqpoint{7.279598in}{6.041847in}}{\pgfqpoint{7.279598in}{6.030797in}}%
\pgfpathcurveto{\pgfqpoint{7.279598in}{6.019747in}}{\pgfqpoint{7.283988in}{6.009148in}}{\pgfqpoint{7.291802in}{6.001334in}}%
\pgfpathcurveto{\pgfqpoint{7.299616in}{5.993521in}}{\pgfqpoint{7.310215in}{5.989130in}}{\pgfqpoint{7.321265in}{5.989130in}}%
\pgfpathclose%
\pgfusepath{stroke,fill}%
\end{pgfscope}%
\begin{pgfscope}%
\pgfpathrectangle{\pgfqpoint{0.481978in}{0.331635in}}{\pgfqpoint{9.300000in}{7.700000in}}%
\pgfusepath{clip}%
\pgfsetbuttcap%
\pgfsetroundjoin%
\definecolor{currentfill}{rgb}{0.870588,0.733333,0.607843}%
\pgfsetfillcolor{currentfill}%
\pgfsetlinewidth{0.481800pt}%
\definecolor{currentstroke}{rgb}{1.000000,1.000000,1.000000}%
\pgfsetstrokecolor{currentstroke}%
\pgfsetdash{}{0pt}%
\pgfpathmoveto{\pgfqpoint{1.982524in}{5.164596in}}%
\pgfpathcurveto{\pgfqpoint{1.993574in}{5.164596in}}{\pgfqpoint{2.004173in}{5.168986in}}{\pgfqpoint{2.011987in}{5.176800in}}%
\pgfpathcurveto{\pgfqpoint{2.019800in}{5.184613in}}{\pgfqpoint{2.024191in}{5.195212in}}{\pgfqpoint{2.024191in}{5.206262in}}%
\pgfpathcurveto{\pgfqpoint{2.024191in}{5.217313in}}{\pgfqpoint{2.019800in}{5.227912in}}{\pgfqpoint{2.011987in}{5.235725in}}%
\pgfpathcurveto{\pgfqpoint{2.004173in}{5.243539in}}{\pgfqpoint{1.993574in}{5.247929in}}{\pgfqpoint{1.982524in}{5.247929in}}%
\pgfpathcurveto{\pgfqpoint{1.971474in}{5.247929in}}{\pgfqpoint{1.960875in}{5.243539in}}{\pgfqpoint{1.953061in}{5.235725in}}%
\pgfpathcurveto{\pgfqpoint{1.945248in}{5.227912in}}{\pgfqpoint{1.940857in}{5.217313in}}{\pgfqpoint{1.940857in}{5.206262in}}%
\pgfpathcurveto{\pgfqpoint{1.940857in}{5.195212in}}{\pgfqpoint{1.945248in}{5.184613in}}{\pgfqpoint{1.953061in}{5.176800in}}%
\pgfpathcurveto{\pgfqpoint{1.960875in}{5.168986in}}{\pgfqpoint{1.971474in}{5.164596in}}{\pgfqpoint{1.982524in}{5.164596in}}%
\pgfpathclose%
\pgfusepath{stroke,fill}%
\end{pgfscope}%
\begin{pgfscope}%
\pgfpathrectangle{\pgfqpoint{0.481978in}{0.331635in}}{\pgfqpoint{9.300000in}{7.700000in}}%
\pgfusepath{clip}%
\pgfsetbuttcap%
\pgfsetroundjoin%
\definecolor{currentfill}{rgb}{0.870588,0.733333,0.607843}%
\pgfsetfillcolor{currentfill}%
\pgfsetlinewidth{0.481800pt}%
\definecolor{currentstroke}{rgb}{1.000000,1.000000,1.000000}%
\pgfsetstrokecolor{currentstroke}%
\pgfsetdash{}{0pt}%
\pgfpathmoveto{\pgfqpoint{2.801903in}{4.359870in}}%
\pgfpathcurveto{\pgfqpoint{2.812953in}{4.359870in}}{\pgfqpoint{2.823552in}{4.364261in}}{\pgfqpoint{2.831365in}{4.372074in}}%
\pgfpathcurveto{\pgfqpoint{2.839179in}{4.379888in}}{\pgfqpoint{2.843569in}{4.390487in}}{\pgfqpoint{2.843569in}{4.401537in}}%
\pgfpathcurveto{\pgfqpoint{2.843569in}{4.412587in}}{\pgfqpoint{2.839179in}{4.423186in}}{\pgfqpoint{2.831365in}{4.431000in}}%
\pgfpathcurveto{\pgfqpoint{2.823552in}{4.438813in}}{\pgfqpoint{2.812953in}{4.443204in}}{\pgfqpoint{2.801903in}{4.443204in}}%
\pgfpathcurveto{\pgfqpoint{2.790852in}{4.443204in}}{\pgfqpoint{2.780253in}{4.438813in}}{\pgfqpoint{2.772440in}{4.431000in}}%
\pgfpathcurveto{\pgfqpoint{2.764626in}{4.423186in}}{\pgfqpoint{2.760236in}{4.412587in}}{\pgfqpoint{2.760236in}{4.401537in}}%
\pgfpathcurveto{\pgfqpoint{2.760236in}{4.390487in}}{\pgfqpoint{2.764626in}{4.379888in}}{\pgfqpoint{2.772440in}{4.372074in}}%
\pgfpathcurveto{\pgfqpoint{2.780253in}{4.364261in}}{\pgfqpoint{2.790852in}{4.359870in}}{\pgfqpoint{2.801903in}{4.359870in}}%
\pgfpathclose%
\pgfusepath{stroke,fill}%
\end{pgfscope}%
\begin{pgfscope}%
\pgfpathrectangle{\pgfqpoint{0.481978in}{0.331635in}}{\pgfqpoint{9.300000in}{7.700000in}}%
\pgfusepath{clip}%
\pgfsetbuttcap%
\pgfsetroundjoin%
\definecolor{currentfill}{rgb}{0.870588,0.733333,0.607843}%
\pgfsetfillcolor{currentfill}%
\pgfsetlinewidth{0.481800pt}%
\definecolor{currentstroke}{rgb}{1.000000,1.000000,1.000000}%
\pgfsetstrokecolor{currentstroke}%
\pgfsetdash{}{0pt}%
\pgfpathmoveto{\pgfqpoint{3.503709in}{4.271868in}}%
\pgfpathcurveto{\pgfqpoint{3.514759in}{4.271868in}}{\pgfqpoint{3.525358in}{4.276258in}}{\pgfqpoint{3.533171in}{4.284072in}}%
\pgfpathcurveto{\pgfqpoint{3.540985in}{4.291886in}}{\pgfqpoint{3.545375in}{4.302485in}}{\pgfqpoint{3.545375in}{4.313535in}}%
\pgfpathcurveto{\pgfqpoint{3.545375in}{4.324585in}}{\pgfqpoint{3.540985in}{4.335184in}}{\pgfqpoint{3.533171in}{4.342998in}}%
\pgfpathcurveto{\pgfqpoint{3.525358in}{4.350811in}}{\pgfqpoint{3.514759in}{4.355201in}}{\pgfqpoint{3.503709in}{4.355201in}}%
\pgfpathcurveto{\pgfqpoint{3.492659in}{4.355201in}}{\pgfqpoint{3.482059in}{4.350811in}}{\pgfqpoint{3.474246in}{4.342998in}}%
\pgfpathcurveto{\pgfqpoint{3.466432in}{4.335184in}}{\pgfqpoint{3.462042in}{4.324585in}}{\pgfqpoint{3.462042in}{4.313535in}}%
\pgfpathcurveto{\pgfqpoint{3.462042in}{4.302485in}}{\pgfqpoint{3.466432in}{4.291886in}}{\pgfqpoint{3.474246in}{4.284072in}}%
\pgfpathcurveto{\pgfqpoint{3.482059in}{4.276258in}}{\pgfqpoint{3.492659in}{4.271868in}}{\pgfqpoint{3.503709in}{4.271868in}}%
\pgfpathclose%
\pgfusepath{stroke,fill}%
\end{pgfscope}%
\begin{pgfscope}%
\pgfpathrectangle{\pgfqpoint{0.481978in}{0.331635in}}{\pgfqpoint{9.300000in}{7.700000in}}%
\pgfusepath{clip}%
\pgfsetbuttcap%
\pgfsetroundjoin%
\definecolor{currentfill}{rgb}{0.870588,0.733333,0.607843}%
\pgfsetfillcolor{currentfill}%
\pgfsetlinewidth{0.481800pt}%
\definecolor{currentstroke}{rgb}{1.000000,1.000000,1.000000}%
\pgfsetstrokecolor{currentstroke}%
\pgfsetdash{}{0pt}%
\pgfpathmoveto{\pgfqpoint{7.049429in}{6.014836in}}%
\pgfpathcurveto{\pgfqpoint{7.060479in}{6.014836in}}{\pgfqpoint{7.071078in}{6.019226in}}{\pgfqpoint{7.078892in}{6.027039in}}%
\pgfpathcurveto{\pgfqpoint{7.086705in}{6.034853in}}{\pgfqpoint{7.091096in}{6.045452in}}{\pgfqpoint{7.091096in}{6.056502in}}%
\pgfpathcurveto{\pgfqpoint{7.091096in}{6.067552in}}{\pgfqpoint{7.086705in}{6.078151in}}{\pgfqpoint{7.078892in}{6.085965in}}%
\pgfpathcurveto{\pgfqpoint{7.071078in}{6.093779in}}{\pgfqpoint{7.060479in}{6.098169in}}{\pgfqpoint{7.049429in}{6.098169in}}%
\pgfpathcurveto{\pgfqpoint{7.038379in}{6.098169in}}{\pgfqpoint{7.027780in}{6.093779in}}{\pgfqpoint{7.019966in}{6.085965in}}%
\pgfpathcurveto{\pgfqpoint{7.012153in}{6.078151in}}{\pgfqpoint{7.007762in}{6.067552in}}{\pgfqpoint{7.007762in}{6.056502in}}%
\pgfpathcurveto{\pgfqpoint{7.007762in}{6.045452in}}{\pgfqpoint{7.012153in}{6.034853in}}{\pgfqpoint{7.019966in}{6.027039in}}%
\pgfpathcurveto{\pgfqpoint{7.027780in}{6.019226in}}{\pgfqpoint{7.038379in}{6.014836in}}{\pgfqpoint{7.049429in}{6.014836in}}%
\pgfpathclose%
\pgfusepath{stroke,fill}%
\end{pgfscope}%
\begin{pgfscope}%
\pgfpathrectangle{\pgfqpoint{0.481978in}{0.331635in}}{\pgfqpoint{9.300000in}{7.700000in}}%
\pgfusepath{clip}%
\pgfsetbuttcap%
\pgfsetroundjoin%
\definecolor{currentfill}{rgb}{0.870588,0.733333,0.607843}%
\pgfsetfillcolor{currentfill}%
\pgfsetlinewidth{0.481800pt}%
\definecolor{currentstroke}{rgb}{1.000000,1.000000,1.000000}%
\pgfsetstrokecolor{currentstroke}%
\pgfsetdash{}{0pt}%
\pgfpathmoveto{\pgfqpoint{6.348280in}{4.805197in}}%
\pgfpathcurveto{\pgfqpoint{6.359330in}{4.805197in}}{\pgfqpoint{6.369929in}{4.809587in}}{\pgfqpoint{6.377743in}{4.817401in}}%
\pgfpathcurveto{\pgfqpoint{6.385557in}{4.825215in}}{\pgfqpoint{6.389947in}{4.835814in}}{\pgfqpoint{6.389947in}{4.846864in}}%
\pgfpathcurveto{\pgfqpoint{6.389947in}{4.857914in}}{\pgfqpoint{6.385557in}{4.868513in}}{\pgfqpoint{6.377743in}{4.876327in}}%
\pgfpathcurveto{\pgfqpoint{6.369929in}{4.884140in}}{\pgfqpoint{6.359330in}{4.888530in}}{\pgfqpoint{6.348280in}{4.888530in}}%
\pgfpathcurveto{\pgfqpoint{6.337230in}{4.888530in}}{\pgfqpoint{6.326631in}{4.884140in}}{\pgfqpoint{6.318818in}{4.876327in}}%
\pgfpathcurveto{\pgfqpoint{6.311004in}{4.868513in}}{\pgfqpoint{6.306614in}{4.857914in}}{\pgfqpoint{6.306614in}{4.846864in}}%
\pgfpathcurveto{\pgfqpoint{6.306614in}{4.835814in}}{\pgfqpoint{6.311004in}{4.825215in}}{\pgfqpoint{6.318818in}{4.817401in}}%
\pgfpathcurveto{\pgfqpoint{6.326631in}{4.809587in}}{\pgfqpoint{6.337230in}{4.805197in}}{\pgfqpoint{6.348280in}{4.805197in}}%
\pgfpathclose%
\pgfusepath{stroke,fill}%
\end{pgfscope}%
\begin{pgfscope}%
\pgfpathrectangle{\pgfqpoint{0.481978in}{0.331635in}}{\pgfqpoint{9.300000in}{7.700000in}}%
\pgfusepath{clip}%
\pgfsetbuttcap%
\pgfsetroundjoin%
\definecolor{currentfill}{rgb}{0.870588,0.733333,0.607843}%
\pgfsetfillcolor{currentfill}%
\pgfsetlinewidth{0.481800pt}%
\definecolor{currentstroke}{rgb}{1.000000,1.000000,1.000000}%
\pgfsetstrokecolor{currentstroke}%
\pgfsetdash{}{0pt}%
\pgfpathmoveto{\pgfqpoint{5.923605in}{3.640188in}}%
\pgfpathcurveto{\pgfqpoint{5.934656in}{3.640188in}}{\pgfqpoint{5.945255in}{3.644578in}}{\pgfqpoint{5.953068in}{3.652392in}}%
\pgfpathcurveto{\pgfqpoint{5.960882in}{3.660205in}}{\pgfqpoint{5.965272in}{3.670804in}}{\pgfqpoint{5.965272in}{3.681854in}}%
\pgfpathcurveto{\pgfqpoint{5.965272in}{3.692905in}}{\pgfqpoint{5.960882in}{3.703504in}}{\pgfqpoint{5.953068in}{3.711317in}}%
\pgfpathcurveto{\pgfqpoint{5.945255in}{3.719131in}}{\pgfqpoint{5.934656in}{3.723521in}}{\pgfqpoint{5.923605in}{3.723521in}}%
\pgfpathcurveto{\pgfqpoint{5.912555in}{3.723521in}}{\pgfqpoint{5.901956in}{3.719131in}}{\pgfqpoint{5.894143in}{3.711317in}}%
\pgfpathcurveto{\pgfqpoint{5.886329in}{3.703504in}}{\pgfqpoint{5.881939in}{3.692905in}}{\pgfqpoint{5.881939in}{3.681854in}}%
\pgfpathcurveto{\pgfqpoint{5.881939in}{3.670804in}}{\pgfqpoint{5.886329in}{3.660205in}}{\pgfqpoint{5.894143in}{3.652392in}}%
\pgfpathcurveto{\pgfqpoint{5.901956in}{3.644578in}}{\pgfqpoint{5.912555in}{3.640188in}}{\pgfqpoint{5.923605in}{3.640188in}}%
\pgfpathclose%
\pgfusepath{stroke,fill}%
\end{pgfscope}%
\begin{pgfscope}%
\pgfpathrectangle{\pgfqpoint{0.481978in}{0.331635in}}{\pgfqpoint{9.300000in}{7.700000in}}%
\pgfusepath{clip}%
\pgfsetbuttcap%
\pgfsetroundjoin%
\definecolor{currentfill}{rgb}{0.870588,0.733333,0.607843}%
\pgfsetfillcolor{currentfill}%
\pgfsetlinewidth{0.481800pt}%
\definecolor{currentstroke}{rgb}{1.000000,1.000000,1.000000}%
\pgfsetstrokecolor{currentstroke}%
\pgfsetdash{}{0pt}%
\pgfpathmoveto{\pgfqpoint{5.344914in}{3.994410in}}%
\pgfpathcurveto{\pgfqpoint{5.355965in}{3.994410in}}{\pgfqpoint{5.366564in}{3.998800in}}{\pgfqpoint{5.374377in}{4.006614in}}%
\pgfpathcurveto{\pgfqpoint{5.382191in}{4.014427in}}{\pgfqpoint{5.386581in}{4.025027in}}{\pgfqpoint{5.386581in}{4.036077in}}%
\pgfpathcurveto{\pgfqpoint{5.386581in}{4.047127in}}{\pgfqpoint{5.382191in}{4.057726in}}{\pgfqpoint{5.374377in}{4.065539in}}%
\pgfpathcurveto{\pgfqpoint{5.366564in}{4.073353in}}{\pgfqpoint{5.355965in}{4.077743in}}{\pgfqpoint{5.344914in}{4.077743in}}%
\pgfpathcurveto{\pgfqpoint{5.333864in}{4.077743in}}{\pgfqpoint{5.323265in}{4.073353in}}{\pgfqpoint{5.315452in}{4.065539in}}%
\pgfpathcurveto{\pgfqpoint{5.307638in}{4.057726in}}{\pgfqpoint{5.303248in}{4.047127in}}{\pgfqpoint{5.303248in}{4.036077in}}%
\pgfpathcurveto{\pgfqpoint{5.303248in}{4.025027in}}{\pgfqpoint{5.307638in}{4.014427in}}{\pgfqpoint{5.315452in}{4.006614in}}%
\pgfpathcurveto{\pgfqpoint{5.323265in}{3.998800in}}{\pgfqpoint{5.333864in}{3.994410in}}{\pgfqpoint{5.344914in}{3.994410in}}%
\pgfpathclose%
\pgfusepath{stroke,fill}%
\end{pgfscope}%
\begin{pgfscope}%
\pgfpathrectangle{\pgfqpoint{0.481978in}{0.331635in}}{\pgfqpoint{9.300000in}{7.700000in}}%
\pgfusepath{clip}%
\pgfsetbuttcap%
\pgfsetroundjoin%
\definecolor{currentfill}{rgb}{0.870588,0.733333,0.607843}%
\pgfsetfillcolor{currentfill}%
\pgfsetlinewidth{0.481800pt}%
\definecolor{currentstroke}{rgb}{1.000000,1.000000,1.000000}%
\pgfsetstrokecolor{currentstroke}%
\pgfsetdash{}{0pt}%
\pgfpathmoveto{\pgfqpoint{4.527667in}{3.702609in}}%
\pgfpathcurveto{\pgfqpoint{4.538717in}{3.702609in}}{\pgfqpoint{4.549316in}{3.706999in}}{\pgfqpoint{4.557129in}{3.714813in}}%
\pgfpathcurveto{\pgfqpoint{4.564943in}{3.722627in}}{\pgfqpoint{4.569333in}{3.733226in}}{\pgfqpoint{4.569333in}{3.744276in}}%
\pgfpathcurveto{\pgfqpoint{4.569333in}{3.755326in}}{\pgfqpoint{4.564943in}{3.765925in}}{\pgfqpoint{4.557129in}{3.773739in}}%
\pgfpathcurveto{\pgfqpoint{4.549316in}{3.781552in}}{\pgfqpoint{4.538717in}{3.785943in}}{\pgfqpoint{4.527667in}{3.785943in}}%
\pgfpathcurveto{\pgfqpoint{4.516616in}{3.785943in}}{\pgfqpoint{4.506017in}{3.781552in}}{\pgfqpoint{4.498204in}{3.773739in}}%
\pgfpathcurveto{\pgfqpoint{4.490390in}{3.765925in}}{\pgfqpoint{4.486000in}{3.755326in}}{\pgfqpoint{4.486000in}{3.744276in}}%
\pgfpathcurveto{\pgfqpoint{4.486000in}{3.733226in}}{\pgfqpoint{4.490390in}{3.722627in}}{\pgfqpoint{4.498204in}{3.714813in}}%
\pgfpathcurveto{\pgfqpoint{4.506017in}{3.706999in}}{\pgfqpoint{4.516616in}{3.702609in}}{\pgfqpoint{4.527667in}{3.702609in}}%
\pgfpathclose%
\pgfusepath{stroke,fill}%
\end{pgfscope}%
\begin{pgfscope}%
\pgfpathrectangle{\pgfqpoint{0.481978in}{0.331635in}}{\pgfqpoint{9.300000in}{7.700000in}}%
\pgfusepath{clip}%
\pgfsetbuttcap%
\pgfsetroundjoin%
\definecolor{currentfill}{rgb}{0.870588,0.733333,0.607843}%
\pgfsetfillcolor{currentfill}%
\pgfsetlinewidth{0.481800pt}%
\definecolor{currentstroke}{rgb}{1.000000,1.000000,1.000000}%
\pgfsetstrokecolor{currentstroke}%
\pgfsetdash{}{0pt}%
\pgfpathmoveto{\pgfqpoint{2.059850in}{3.994794in}}%
\pgfpathcurveto{\pgfqpoint{2.070900in}{3.994794in}}{\pgfqpoint{2.081499in}{3.999184in}}{\pgfqpoint{2.089313in}{4.006997in}}%
\pgfpathcurveto{\pgfqpoint{2.097126in}{4.014811in}}{\pgfqpoint{2.101516in}{4.025410in}}{\pgfqpoint{2.101516in}{4.036460in}}%
\pgfpathcurveto{\pgfqpoint{2.101516in}{4.047510in}}{\pgfqpoint{2.097126in}{4.058109in}}{\pgfqpoint{2.089313in}{4.065923in}}%
\pgfpathcurveto{\pgfqpoint{2.081499in}{4.073737in}}{\pgfqpoint{2.070900in}{4.078127in}}{\pgfqpoint{2.059850in}{4.078127in}}%
\pgfpathcurveto{\pgfqpoint{2.048800in}{4.078127in}}{\pgfqpoint{2.038201in}{4.073737in}}{\pgfqpoint{2.030387in}{4.065923in}}%
\pgfpathcurveto{\pgfqpoint{2.022573in}{4.058109in}}{\pgfqpoint{2.018183in}{4.047510in}}{\pgfqpoint{2.018183in}{4.036460in}}%
\pgfpathcurveto{\pgfqpoint{2.018183in}{4.025410in}}{\pgfqpoint{2.022573in}{4.014811in}}{\pgfqpoint{2.030387in}{4.006997in}}%
\pgfpathcurveto{\pgfqpoint{2.038201in}{3.999184in}}{\pgfqpoint{2.048800in}{3.994794in}}{\pgfqpoint{2.059850in}{3.994794in}}%
\pgfpathclose%
\pgfusepath{stroke,fill}%
\end{pgfscope}%
\begin{pgfscope}%
\pgfpathrectangle{\pgfqpoint{0.481978in}{0.331635in}}{\pgfqpoint{9.300000in}{7.700000in}}%
\pgfusepath{clip}%
\pgfsetbuttcap%
\pgfsetroundjoin%
\definecolor{currentfill}{rgb}{0.870588,0.733333,0.607843}%
\pgfsetfillcolor{currentfill}%
\pgfsetlinewidth{0.481800pt}%
\definecolor{currentstroke}{rgb}{1.000000,1.000000,1.000000}%
\pgfsetstrokecolor{currentstroke}%
\pgfsetdash{}{0pt}%
\pgfpathmoveto{\pgfqpoint{5.974496in}{4.908445in}}%
\pgfpathcurveto{\pgfqpoint{5.985546in}{4.908445in}}{\pgfqpoint{5.996145in}{4.912835in}}{\pgfqpoint{6.003958in}{4.920649in}}%
\pgfpathcurveto{\pgfqpoint{6.011772in}{4.928462in}}{\pgfqpoint{6.016162in}{4.939061in}}{\pgfqpoint{6.016162in}{4.950112in}}%
\pgfpathcurveto{\pgfqpoint{6.016162in}{4.961162in}}{\pgfqpoint{6.011772in}{4.971761in}}{\pgfqpoint{6.003958in}{4.979574in}}%
\pgfpathcurveto{\pgfqpoint{5.996145in}{4.987388in}}{\pgfqpoint{5.985546in}{4.991778in}}{\pgfqpoint{5.974496in}{4.991778in}}%
\pgfpathcurveto{\pgfqpoint{5.963445in}{4.991778in}}{\pgfqpoint{5.952846in}{4.987388in}}{\pgfqpoint{5.945033in}{4.979574in}}%
\pgfpathcurveto{\pgfqpoint{5.937219in}{4.971761in}}{\pgfqpoint{5.932829in}{4.961162in}}{\pgfqpoint{5.932829in}{4.950112in}}%
\pgfpathcurveto{\pgfqpoint{5.932829in}{4.939061in}}{\pgfqpoint{5.937219in}{4.928462in}}{\pgfqpoint{5.945033in}{4.920649in}}%
\pgfpathcurveto{\pgfqpoint{5.952846in}{4.912835in}}{\pgfqpoint{5.963445in}{4.908445in}}{\pgfqpoint{5.974496in}{4.908445in}}%
\pgfpathclose%
\pgfusepath{stroke,fill}%
\end{pgfscope}%
\begin{pgfscope}%
\pgfpathrectangle{\pgfqpoint{0.481978in}{0.331635in}}{\pgfqpoint{9.300000in}{7.700000in}}%
\pgfusepath{clip}%
\pgfsetbuttcap%
\pgfsetroundjoin%
\definecolor{currentfill}{rgb}{0.870588,0.733333,0.607843}%
\pgfsetfillcolor{currentfill}%
\pgfsetlinewidth{0.481800pt}%
\definecolor{currentstroke}{rgb}{1.000000,1.000000,1.000000}%
\pgfsetstrokecolor{currentstroke}%
\pgfsetdash{}{0pt}%
\pgfpathmoveto{\pgfqpoint{2.319348in}{4.753136in}}%
\pgfpathcurveto{\pgfqpoint{2.330398in}{4.753136in}}{\pgfqpoint{2.340997in}{4.757526in}}{\pgfqpoint{2.348811in}{4.765339in}}%
\pgfpathcurveto{\pgfqpoint{2.356624in}{4.773153in}}{\pgfqpoint{2.361015in}{4.783752in}}{\pgfqpoint{2.361015in}{4.794802in}}%
\pgfpathcurveto{\pgfqpoint{2.361015in}{4.805852in}}{\pgfqpoint{2.356624in}{4.816451in}}{\pgfqpoint{2.348811in}{4.824265in}}%
\pgfpathcurveto{\pgfqpoint{2.340997in}{4.832079in}}{\pgfqpoint{2.330398in}{4.836469in}}{\pgfqpoint{2.319348in}{4.836469in}}%
\pgfpathcurveto{\pgfqpoint{2.308298in}{4.836469in}}{\pgfqpoint{2.297699in}{4.832079in}}{\pgfqpoint{2.289885in}{4.824265in}}%
\pgfpathcurveto{\pgfqpoint{2.282071in}{4.816451in}}{\pgfqpoint{2.277681in}{4.805852in}}{\pgfqpoint{2.277681in}{4.794802in}}%
\pgfpathcurveto{\pgfqpoint{2.277681in}{4.783752in}}{\pgfqpoint{2.282071in}{4.773153in}}{\pgfqpoint{2.289885in}{4.765339in}}%
\pgfpathcurveto{\pgfqpoint{2.297699in}{4.757526in}}{\pgfqpoint{2.308298in}{4.753136in}}{\pgfqpoint{2.319348in}{4.753136in}}%
\pgfpathclose%
\pgfusepath{stroke,fill}%
\end{pgfscope}%
\begin{pgfscope}%
\pgfpathrectangle{\pgfqpoint{0.481978in}{0.331635in}}{\pgfqpoint{9.300000in}{7.700000in}}%
\pgfusepath{clip}%
\pgfsetbuttcap%
\pgfsetroundjoin%
\definecolor{currentfill}{rgb}{0.870588,0.733333,0.607843}%
\pgfsetfillcolor{currentfill}%
\pgfsetlinewidth{0.481800pt}%
\definecolor{currentstroke}{rgb}{1.000000,1.000000,1.000000}%
\pgfsetstrokecolor{currentstroke}%
\pgfsetdash{}{0pt}%
\pgfpathmoveto{\pgfqpoint{6.238443in}{6.200764in}}%
\pgfpathcurveto{\pgfqpoint{6.249493in}{6.200764in}}{\pgfqpoint{6.260092in}{6.205154in}}{\pgfqpoint{6.267905in}{6.212968in}}%
\pgfpathcurveto{\pgfqpoint{6.275719in}{6.220782in}}{\pgfqpoint{6.280109in}{6.231381in}}{\pgfqpoint{6.280109in}{6.242431in}}%
\pgfpathcurveto{\pgfqpoint{6.280109in}{6.253481in}}{\pgfqpoint{6.275719in}{6.264080in}}{\pgfqpoint{6.267905in}{6.271894in}}%
\pgfpathcurveto{\pgfqpoint{6.260092in}{6.279707in}}{\pgfqpoint{6.249493in}{6.284097in}}{\pgfqpoint{6.238443in}{6.284097in}}%
\pgfpathcurveto{\pgfqpoint{6.227392in}{6.284097in}}{\pgfqpoint{6.216793in}{6.279707in}}{\pgfqpoint{6.208980in}{6.271894in}}%
\pgfpathcurveto{\pgfqpoint{6.201166in}{6.264080in}}{\pgfqpoint{6.196776in}{6.253481in}}{\pgfqpoint{6.196776in}{6.242431in}}%
\pgfpathcurveto{\pgfqpoint{6.196776in}{6.231381in}}{\pgfqpoint{6.201166in}{6.220782in}}{\pgfqpoint{6.208980in}{6.212968in}}%
\pgfpathcurveto{\pgfqpoint{6.216793in}{6.205154in}}{\pgfqpoint{6.227392in}{6.200764in}}{\pgfqpoint{6.238443in}{6.200764in}}%
\pgfpathclose%
\pgfusepath{stroke,fill}%
\end{pgfscope}%
\begin{pgfscope}%
\pgfpathrectangle{\pgfqpoint{0.481978in}{0.331635in}}{\pgfqpoint{9.300000in}{7.700000in}}%
\pgfusepath{clip}%
\pgfsetbuttcap%
\pgfsetroundjoin%
\definecolor{currentfill}{rgb}{0.870588,0.733333,0.607843}%
\pgfsetfillcolor{currentfill}%
\pgfsetlinewidth{0.481800pt}%
\definecolor{currentstroke}{rgb}{1.000000,1.000000,1.000000}%
\pgfsetstrokecolor{currentstroke}%
\pgfsetdash{}{0pt}%
\pgfpathmoveto{\pgfqpoint{4.167128in}{3.855525in}}%
\pgfpathcurveto{\pgfqpoint{4.178178in}{3.855525in}}{\pgfqpoint{4.188778in}{3.859915in}}{\pgfqpoint{4.196591in}{3.867729in}}%
\pgfpathcurveto{\pgfqpoint{4.204405in}{3.875542in}}{\pgfqpoint{4.208795in}{3.886141in}}{\pgfqpoint{4.208795in}{3.897191in}}%
\pgfpathcurveto{\pgfqpoint{4.208795in}{3.908241in}}{\pgfqpoint{4.204405in}{3.918841in}}{\pgfqpoint{4.196591in}{3.926654in}}%
\pgfpathcurveto{\pgfqpoint{4.188778in}{3.934468in}}{\pgfqpoint{4.178178in}{3.938858in}}{\pgfqpoint{4.167128in}{3.938858in}}%
\pgfpathcurveto{\pgfqpoint{4.156078in}{3.938858in}}{\pgfqpoint{4.145479in}{3.934468in}}{\pgfqpoint{4.137666in}{3.926654in}}%
\pgfpathcurveto{\pgfqpoint{4.129852in}{3.918841in}}{\pgfqpoint{4.125462in}{3.908241in}}{\pgfqpoint{4.125462in}{3.897191in}}%
\pgfpathcurveto{\pgfqpoint{4.125462in}{3.886141in}}{\pgfqpoint{4.129852in}{3.875542in}}{\pgfqpoint{4.137666in}{3.867729in}}%
\pgfpathcurveto{\pgfqpoint{4.145479in}{3.859915in}}{\pgfqpoint{4.156078in}{3.855525in}}{\pgfqpoint{4.167128in}{3.855525in}}%
\pgfpathclose%
\pgfusepath{stroke,fill}%
\end{pgfscope}%
\begin{pgfscope}%
\pgfpathrectangle{\pgfqpoint{0.481978in}{0.331635in}}{\pgfqpoint{9.300000in}{7.700000in}}%
\pgfusepath{clip}%
\pgfsetbuttcap%
\pgfsetroundjoin%
\definecolor{currentfill}{rgb}{0.870588,0.733333,0.607843}%
\pgfsetfillcolor{currentfill}%
\pgfsetlinewidth{0.481800pt}%
\definecolor{currentstroke}{rgb}{1.000000,1.000000,1.000000}%
\pgfsetstrokecolor{currentstroke}%
\pgfsetdash{}{0pt}%
\pgfpathmoveto{\pgfqpoint{4.383560in}{3.556162in}}%
\pgfpathcurveto{\pgfqpoint{4.394610in}{3.556162in}}{\pgfqpoint{4.405209in}{3.560553in}}{\pgfqpoint{4.413023in}{3.568366in}}%
\pgfpathcurveto{\pgfqpoint{4.420836in}{3.576180in}}{\pgfqpoint{4.425227in}{3.586779in}}{\pgfqpoint{4.425227in}{3.597829in}}%
\pgfpathcurveto{\pgfqpoint{4.425227in}{3.608879in}}{\pgfqpoint{4.420836in}{3.619478in}}{\pgfqpoint{4.413023in}{3.627292in}}%
\pgfpathcurveto{\pgfqpoint{4.405209in}{3.635105in}}{\pgfqpoint{4.394610in}{3.639496in}}{\pgfqpoint{4.383560in}{3.639496in}}%
\pgfpathcurveto{\pgfqpoint{4.372510in}{3.639496in}}{\pgfqpoint{4.361911in}{3.635105in}}{\pgfqpoint{4.354097in}{3.627292in}}%
\pgfpathcurveto{\pgfqpoint{4.346284in}{3.619478in}}{\pgfqpoint{4.341893in}{3.608879in}}{\pgfqpoint{4.341893in}{3.597829in}}%
\pgfpathcurveto{\pgfqpoint{4.341893in}{3.586779in}}{\pgfqpoint{4.346284in}{3.576180in}}{\pgfqpoint{4.354097in}{3.568366in}}%
\pgfpathcurveto{\pgfqpoint{4.361911in}{3.560553in}}{\pgfqpoint{4.372510in}{3.556162in}}{\pgfqpoint{4.383560in}{3.556162in}}%
\pgfpathclose%
\pgfusepath{stroke,fill}%
\end{pgfscope}%
\begin{pgfscope}%
\pgfpathrectangle{\pgfqpoint{0.481978in}{0.331635in}}{\pgfqpoint{9.300000in}{7.700000in}}%
\pgfusepath{clip}%
\pgfsetbuttcap%
\pgfsetroundjoin%
\definecolor{currentfill}{rgb}{0.870588,0.733333,0.607843}%
\pgfsetfillcolor{currentfill}%
\pgfsetlinewidth{0.481800pt}%
\definecolor{currentstroke}{rgb}{1.000000,1.000000,1.000000}%
\pgfsetstrokecolor{currentstroke}%
\pgfsetdash{}{0pt}%
\pgfpathmoveto{\pgfqpoint{5.769577in}{3.960324in}}%
\pgfpathcurveto{\pgfqpoint{5.780628in}{3.960324in}}{\pgfqpoint{5.791227in}{3.964714in}}{\pgfqpoint{5.799040in}{3.972527in}}%
\pgfpathcurveto{\pgfqpoint{5.806854in}{3.980341in}}{\pgfqpoint{5.811244in}{3.990940in}}{\pgfqpoint{5.811244in}{4.001990in}}%
\pgfpathcurveto{\pgfqpoint{5.811244in}{4.013040in}}{\pgfqpoint{5.806854in}{4.023639in}}{\pgfqpoint{5.799040in}{4.031453in}}%
\pgfpathcurveto{\pgfqpoint{5.791227in}{4.039267in}}{\pgfqpoint{5.780628in}{4.043657in}}{\pgfqpoint{5.769577in}{4.043657in}}%
\pgfpathcurveto{\pgfqpoint{5.758527in}{4.043657in}}{\pgfqpoint{5.747928in}{4.039267in}}{\pgfqpoint{5.740115in}{4.031453in}}%
\pgfpathcurveto{\pgfqpoint{5.732301in}{4.023639in}}{\pgfqpoint{5.727911in}{4.013040in}}{\pgfqpoint{5.727911in}{4.001990in}}%
\pgfpathcurveto{\pgfqpoint{5.727911in}{3.990940in}}{\pgfqpoint{5.732301in}{3.980341in}}{\pgfqpoint{5.740115in}{3.972527in}}%
\pgfpathcurveto{\pgfqpoint{5.747928in}{3.964714in}}{\pgfqpoint{5.758527in}{3.960324in}}{\pgfqpoint{5.769577in}{3.960324in}}%
\pgfpathclose%
\pgfusepath{stroke,fill}%
\end{pgfscope}%
\begin{pgfscope}%
\pgfpathrectangle{\pgfqpoint{0.481978in}{0.331635in}}{\pgfqpoint{9.300000in}{7.700000in}}%
\pgfusepath{clip}%
\pgfsetbuttcap%
\pgfsetroundjoin%
\definecolor{currentfill}{rgb}{0.870588,0.733333,0.607843}%
\pgfsetfillcolor{currentfill}%
\pgfsetlinewidth{0.481800pt}%
\definecolor{currentstroke}{rgb}{1.000000,1.000000,1.000000}%
\pgfsetstrokecolor{currentstroke}%
\pgfsetdash{}{0pt}%
\pgfpathmoveto{\pgfqpoint{9.359251in}{4.528409in}}%
\pgfpathcurveto{\pgfqpoint{9.370301in}{4.528409in}}{\pgfqpoint{9.380900in}{4.532799in}}{\pgfqpoint{9.388713in}{4.540613in}}%
\pgfpathcurveto{\pgfqpoint{9.396527in}{4.548427in}}{\pgfqpoint{9.400917in}{4.559026in}}{\pgfqpoint{9.400917in}{4.570076in}}%
\pgfpathcurveto{\pgfqpoint{9.400917in}{4.581126in}}{\pgfqpoint{9.396527in}{4.591725in}}{\pgfqpoint{9.388713in}{4.599539in}}%
\pgfpathcurveto{\pgfqpoint{9.380900in}{4.607352in}}{\pgfqpoint{9.370301in}{4.611743in}}{\pgfqpoint{9.359251in}{4.611743in}}%
\pgfpathcurveto{\pgfqpoint{9.348201in}{4.611743in}}{\pgfqpoint{9.337601in}{4.607352in}}{\pgfqpoint{9.329788in}{4.599539in}}%
\pgfpathcurveto{\pgfqpoint{9.321974in}{4.591725in}}{\pgfqpoint{9.317584in}{4.581126in}}{\pgfqpoint{9.317584in}{4.570076in}}%
\pgfpathcurveto{\pgfqpoint{9.317584in}{4.559026in}}{\pgfqpoint{9.321974in}{4.548427in}}{\pgfqpoint{9.329788in}{4.540613in}}%
\pgfpathcurveto{\pgfqpoint{9.337601in}{4.532799in}}{\pgfqpoint{9.348201in}{4.528409in}}{\pgfqpoint{9.359251in}{4.528409in}}%
\pgfpathclose%
\pgfusepath{stroke,fill}%
\end{pgfscope}%
\begin{pgfscope}%
\pgfpathrectangle{\pgfqpoint{0.481978in}{0.331635in}}{\pgfqpoint{9.300000in}{7.700000in}}%
\pgfusepath{clip}%
\pgfsetbuttcap%
\pgfsetroundjoin%
\definecolor{currentfill}{rgb}{0.870588,0.733333,0.607843}%
\pgfsetfillcolor{currentfill}%
\pgfsetlinewidth{0.481800pt}%
\definecolor{currentstroke}{rgb}{1.000000,1.000000,1.000000}%
\pgfsetstrokecolor{currentstroke}%
\pgfsetdash{}{0pt}%
\pgfpathmoveto{\pgfqpoint{5.675309in}{2.630964in}}%
\pgfpathcurveto{\pgfqpoint{5.686359in}{2.630964in}}{\pgfqpoint{5.696958in}{2.635354in}}{\pgfqpoint{5.704772in}{2.643168in}}%
\pgfpathcurveto{\pgfqpoint{5.712586in}{2.650981in}}{\pgfqpoint{5.716976in}{2.661581in}}{\pgfqpoint{5.716976in}{2.672631in}}%
\pgfpathcurveto{\pgfqpoint{5.716976in}{2.683681in}}{\pgfqpoint{5.712586in}{2.694280in}}{\pgfqpoint{5.704772in}{2.702093in}}%
\pgfpathcurveto{\pgfqpoint{5.696958in}{2.709907in}}{\pgfqpoint{5.686359in}{2.714297in}}{\pgfqpoint{5.675309in}{2.714297in}}%
\pgfpathcurveto{\pgfqpoint{5.664259in}{2.714297in}}{\pgfqpoint{5.653660in}{2.709907in}}{\pgfqpoint{5.645846in}{2.702093in}}%
\pgfpathcurveto{\pgfqpoint{5.638033in}{2.694280in}}{\pgfqpoint{5.633643in}{2.683681in}}{\pgfqpoint{5.633643in}{2.672631in}}%
\pgfpathcurveto{\pgfqpoint{5.633643in}{2.661581in}}{\pgfqpoint{5.638033in}{2.650981in}}{\pgfqpoint{5.645846in}{2.643168in}}%
\pgfpathcurveto{\pgfqpoint{5.653660in}{2.635354in}}{\pgfqpoint{5.664259in}{2.630964in}}{\pgfqpoint{5.675309in}{2.630964in}}%
\pgfpathclose%
\pgfusepath{stroke,fill}%
\end{pgfscope}%
\begin{pgfscope}%
\pgfpathrectangle{\pgfqpoint{0.481978in}{0.331635in}}{\pgfqpoint{9.300000in}{7.700000in}}%
\pgfusepath{clip}%
\pgfsetbuttcap%
\pgfsetroundjoin%
\definecolor{currentfill}{rgb}{0.870588,0.733333,0.607843}%
\pgfsetfillcolor{currentfill}%
\pgfsetlinewidth{0.481800pt}%
\definecolor{currentstroke}{rgb}{1.000000,1.000000,1.000000}%
\pgfsetstrokecolor{currentstroke}%
\pgfsetdash{}{0pt}%
\pgfpathmoveto{\pgfqpoint{3.026449in}{4.306297in}}%
\pgfpathcurveto{\pgfqpoint{3.037499in}{4.306297in}}{\pgfqpoint{3.048098in}{4.310688in}}{\pgfqpoint{3.055912in}{4.318501in}}%
\pgfpathcurveto{\pgfqpoint{3.063726in}{4.326315in}}{\pgfqpoint{3.068116in}{4.336914in}}{\pgfqpoint{3.068116in}{4.347964in}}%
\pgfpathcurveto{\pgfqpoint{3.068116in}{4.359014in}}{\pgfqpoint{3.063726in}{4.369613in}}{\pgfqpoint{3.055912in}{4.377427in}}%
\pgfpathcurveto{\pgfqpoint{3.048098in}{4.385240in}}{\pgfqpoint{3.037499in}{4.389631in}}{\pgfqpoint{3.026449in}{4.389631in}}%
\pgfpathcurveto{\pgfqpoint{3.015399in}{4.389631in}}{\pgfqpoint{3.004800in}{4.385240in}}{\pgfqpoint{2.996986in}{4.377427in}}%
\pgfpathcurveto{\pgfqpoint{2.989173in}{4.369613in}}{\pgfqpoint{2.984783in}{4.359014in}}{\pgfqpoint{2.984783in}{4.347964in}}%
\pgfpathcurveto{\pgfqpoint{2.984783in}{4.336914in}}{\pgfqpoint{2.989173in}{4.326315in}}{\pgfqpoint{2.996986in}{4.318501in}}%
\pgfpathcurveto{\pgfqpoint{3.004800in}{4.310688in}}{\pgfqpoint{3.015399in}{4.306297in}}{\pgfqpoint{3.026449in}{4.306297in}}%
\pgfpathclose%
\pgfusepath{stroke,fill}%
\end{pgfscope}%
\begin{pgfscope}%
\pgfpathrectangle{\pgfqpoint{0.481978in}{0.331635in}}{\pgfqpoint{9.300000in}{7.700000in}}%
\pgfusepath{clip}%
\pgfsetbuttcap%
\pgfsetroundjoin%
\definecolor{currentfill}{rgb}{0.870588,0.733333,0.607843}%
\pgfsetfillcolor{currentfill}%
\pgfsetlinewidth{0.481800pt}%
\definecolor{currentstroke}{rgb}{1.000000,1.000000,1.000000}%
\pgfsetstrokecolor{currentstroke}%
\pgfsetdash{}{0pt}%
\pgfpathmoveto{\pgfqpoint{6.761280in}{5.824363in}}%
\pgfpathcurveto{\pgfqpoint{6.772330in}{5.824363in}}{\pgfqpoint{6.782929in}{5.828753in}}{\pgfqpoint{6.790742in}{5.836567in}}%
\pgfpathcurveto{\pgfqpoint{6.798556in}{5.844381in}}{\pgfqpoint{6.802946in}{5.854980in}}{\pgfqpoint{6.802946in}{5.866030in}}%
\pgfpathcurveto{\pgfqpoint{6.802946in}{5.877080in}}{\pgfqpoint{6.798556in}{5.887679in}}{\pgfqpoint{6.790742in}{5.895492in}}%
\pgfpathcurveto{\pgfqpoint{6.782929in}{5.903306in}}{\pgfqpoint{6.772330in}{5.907696in}}{\pgfqpoint{6.761280in}{5.907696in}}%
\pgfpathcurveto{\pgfqpoint{6.750229in}{5.907696in}}{\pgfqpoint{6.739630in}{5.903306in}}{\pgfqpoint{6.731817in}{5.895492in}}%
\pgfpathcurveto{\pgfqpoint{6.724003in}{5.887679in}}{\pgfqpoint{6.719613in}{5.877080in}}{\pgfqpoint{6.719613in}{5.866030in}}%
\pgfpathcurveto{\pgfqpoint{6.719613in}{5.854980in}}{\pgfqpoint{6.724003in}{5.844381in}}{\pgfqpoint{6.731817in}{5.836567in}}%
\pgfpathcurveto{\pgfqpoint{6.739630in}{5.828753in}}{\pgfqpoint{6.750229in}{5.824363in}}{\pgfqpoint{6.761280in}{5.824363in}}%
\pgfpathclose%
\pgfusepath{stroke,fill}%
\end{pgfscope}%
\begin{pgfscope}%
\pgfpathrectangle{\pgfqpoint{0.481978in}{0.331635in}}{\pgfqpoint{9.300000in}{7.700000in}}%
\pgfusepath{clip}%
\pgfsetbuttcap%
\pgfsetroundjoin%
\definecolor{currentfill}{rgb}{0.870588,0.733333,0.607843}%
\pgfsetfillcolor{currentfill}%
\pgfsetlinewidth{0.481800pt}%
\definecolor{currentstroke}{rgb}{1.000000,1.000000,1.000000}%
\pgfsetstrokecolor{currentstroke}%
\pgfsetdash{}{0pt}%
\pgfpathmoveto{\pgfqpoint{2.445109in}{3.067913in}}%
\pgfpathcurveto{\pgfqpoint{2.456159in}{3.067913in}}{\pgfqpoint{2.466758in}{3.072303in}}{\pgfqpoint{2.474571in}{3.080117in}}%
\pgfpathcurveto{\pgfqpoint{2.482385in}{3.087930in}}{\pgfqpoint{2.486775in}{3.098529in}}{\pgfqpoint{2.486775in}{3.109579in}}%
\pgfpathcurveto{\pgfqpoint{2.486775in}{3.120630in}}{\pgfqpoint{2.482385in}{3.131229in}}{\pgfqpoint{2.474571in}{3.139042in}}%
\pgfpathcurveto{\pgfqpoint{2.466758in}{3.146856in}}{\pgfqpoint{2.456159in}{3.151246in}}{\pgfqpoint{2.445109in}{3.151246in}}%
\pgfpathcurveto{\pgfqpoint{2.434059in}{3.151246in}}{\pgfqpoint{2.423459in}{3.146856in}}{\pgfqpoint{2.415646in}{3.139042in}}%
\pgfpathcurveto{\pgfqpoint{2.407832in}{3.131229in}}{\pgfqpoint{2.403442in}{3.120630in}}{\pgfqpoint{2.403442in}{3.109579in}}%
\pgfpathcurveto{\pgfqpoint{2.403442in}{3.098529in}}{\pgfqpoint{2.407832in}{3.087930in}}{\pgfqpoint{2.415646in}{3.080117in}}%
\pgfpathcurveto{\pgfqpoint{2.423459in}{3.072303in}}{\pgfqpoint{2.434059in}{3.067913in}}{\pgfqpoint{2.445109in}{3.067913in}}%
\pgfpathclose%
\pgfusepath{stroke,fill}%
\end{pgfscope}%
\begin{pgfscope}%
\pgfpathrectangle{\pgfqpoint{0.481978in}{0.331635in}}{\pgfqpoint{9.300000in}{7.700000in}}%
\pgfusepath{clip}%
\pgfsetbuttcap%
\pgfsetroundjoin%
\definecolor{currentfill}{rgb}{0.870588,0.733333,0.607843}%
\pgfsetfillcolor{currentfill}%
\pgfsetlinewidth{0.481800pt}%
\definecolor{currentstroke}{rgb}{1.000000,1.000000,1.000000}%
\pgfsetstrokecolor{currentstroke}%
\pgfsetdash{}{0pt}%
\pgfpathmoveto{\pgfqpoint{2.556572in}{4.445719in}}%
\pgfpathcurveto{\pgfqpoint{2.567622in}{4.445719in}}{\pgfqpoint{2.578221in}{4.450109in}}{\pgfqpoint{2.586035in}{4.457922in}}%
\pgfpathcurveto{\pgfqpoint{2.593848in}{4.465736in}}{\pgfqpoint{2.598239in}{4.476335in}}{\pgfqpoint{2.598239in}{4.487385in}}%
\pgfpathcurveto{\pgfqpoint{2.598239in}{4.498435in}}{\pgfqpoint{2.593848in}{4.509034in}}{\pgfqpoint{2.586035in}{4.516848in}}%
\pgfpathcurveto{\pgfqpoint{2.578221in}{4.524662in}}{\pgfqpoint{2.567622in}{4.529052in}}{\pgfqpoint{2.556572in}{4.529052in}}%
\pgfpathcurveto{\pgfqpoint{2.545522in}{4.529052in}}{\pgfqpoint{2.534923in}{4.524662in}}{\pgfqpoint{2.527109in}{4.516848in}}%
\pgfpathcurveto{\pgfqpoint{2.519296in}{4.509034in}}{\pgfqpoint{2.514905in}{4.498435in}}{\pgfqpoint{2.514905in}{4.487385in}}%
\pgfpathcurveto{\pgfqpoint{2.514905in}{4.476335in}}{\pgfqpoint{2.519296in}{4.465736in}}{\pgfqpoint{2.527109in}{4.457922in}}%
\pgfpathcurveto{\pgfqpoint{2.534923in}{4.450109in}}{\pgfqpoint{2.545522in}{4.445719in}}{\pgfqpoint{2.556572in}{4.445719in}}%
\pgfpathclose%
\pgfusepath{stroke,fill}%
\end{pgfscope}%
\begin{pgfscope}%
\pgfpathrectangle{\pgfqpoint{0.481978in}{0.331635in}}{\pgfqpoint{9.300000in}{7.700000in}}%
\pgfusepath{clip}%
\pgfsetbuttcap%
\pgfsetroundjoin%
\definecolor{currentfill}{rgb}{0.870588,0.733333,0.607843}%
\pgfsetfillcolor{currentfill}%
\pgfsetlinewidth{0.481800pt}%
\definecolor{currentstroke}{rgb}{1.000000,1.000000,1.000000}%
\pgfsetstrokecolor{currentstroke}%
\pgfsetdash{}{0pt}%
\pgfpathmoveto{\pgfqpoint{3.956527in}{3.265885in}}%
\pgfpathcurveto{\pgfqpoint{3.967578in}{3.265885in}}{\pgfqpoint{3.978177in}{3.270275in}}{\pgfqpoint{3.985990in}{3.278089in}}%
\pgfpathcurveto{\pgfqpoint{3.993804in}{3.285902in}}{\pgfqpoint{3.998194in}{3.296501in}}{\pgfqpoint{3.998194in}{3.307551in}}%
\pgfpathcurveto{\pgfqpoint{3.998194in}{3.318602in}}{\pgfqpoint{3.993804in}{3.329201in}}{\pgfqpoint{3.985990in}{3.337014in}}%
\pgfpathcurveto{\pgfqpoint{3.978177in}{3.344828in}}{\pgfqpoint{3.967578in}{3.349218in}}{\pgfqpoint{3.956527in}{3.349218in}}%
\pgfpathcurveto{\pgfqpoint{3.945477in}{3.349218in}}{\pgfqpoint{3.934878in}{3.344828in}}{\pgfqpoint{3.927065in}{3.337014in}}%
\pgfpathcurveto{\pgfqpoint{3.919251in}{3.329201in}}{\pgfqpoint{3.914861in}{3.318602in}}{\pgfqpoint{3.914861in}{3.307551in}}%
\pgfpathcurveto{\pgfqpoint{3.914861in}{3.296501in}}{\pgfqpoint{3.919251in}{3.285902in}}{\pgfqpoint{3.927065in}{3.278089in}}%
\pgfpathcurveto{\pgfqpoint{3.934878in}{3.270275in}}{\pgfqpoint{3.945477in}{3.265885in}}{\pgfqpoint{3.956527in}{3.265885in}}%
\pgfpathclose%
\pgfusepath{stroke,fill}%
\end{pgfscope}%
\begin{pgfscope}%
\pgfpathrectangle{\pgfqpoint{0.481978in}{0.331635in}}{\pgfqpoint{9.300000in}{7.700000in}}%
\pgfusepath{clip}%
\pgfsetbuttcap%
\pgfsetroundjoin%
\definecolor{currentfill}{rgb}{0.870588,0.733333,0.607843}%
\pgfsetfillcolor{currentfill}%
\pgfsetlinewidth{0.481800pt}%
\definecolor{currentstroke}{rgb}{1.000000,1.000000,1.000000}%
\pgfsetstrokecolor{currentstroke}%
\pgfsetdash{}{0pt}%
\pgfpathmoveto{\pgfqpoint{2.593480in}{4.701301in}}%
\pgfpathcurveto{\pgfqpoint{2.604530in}{4.701301in}}{\pgfqpoint{2.615129in}{4.705692in}}{\pgfqpoint{2.622942in}{4.713505in}}%
\pgfpathcurveto{\pgfqpoint{2.630756in}{4.721319in}}{\pgfqpoint{2.635146in}{4.731918in}}{\pgfqpoint{2.635146in}{4.742968in}}%
\pgfpathcurveto{\pgfqpoint{2.635146in}{4.754018in}}{\pgfqpoint{2.630756in}{4.764617in}}{\pgfqpoint{2.622942in}{4.772431in}}%
\pgfpathcurveto{\pgfqpoint{2.615129in}{4.780244in}}{\pgfqpoint{2.604530in}{4.784635in}}{\pgfqpoint{2.593480in}{4.784635in}}%
\pgfpathcurveto{\pgfqpoint{2.582429in}{4.784635in}}{\pgfqpoint{2.571830in}{4.780244in}}{\pgfqpoint{2.564017in}{4.772431in}}%
\pgfpathcurveto{\pgfqpoint{2.556203in}{4.764617in}}{\pgfqpoint{2.551813in}{4.754018in}}{\pgfqpoint{2.551813in}{4.742968in}}%
\pgfpathcurveto{\pgfqpoint{2.551813in}{4.731918in}}{\pgfqpoint{2.556203in}{4.721319in}}{\pgfqpoint{2.564017in}{4.713505in}}%
\pgfpathcurveto{\pgfqpoint{2.571830in}{4.705692in}}{\pgfqpoint{2.582429in}{4.701301in}}{\pgfqpoint{2.593480in}{4.701301in}}%
\pgfpathclose%
\pgfusepath{stroke,fill}%
\end{pgfscope}%
\begin{pgfscope}%
\pgfpathrectangle{\pgfqpoint{0.481978in}{0.331635in}}{\pgfqpoint{9.300000in}{7.700000in}}%
\pgfusepath{clip}%
\pgfsetbuttcap%
\pgfsetroundjoin%
\definecolor{currentfill}{rgb}{0.870588,0.733333,0.607843}%
\pgfsetfillcolor{currentfill}%
\pgfsetlinewidth{0.481800pt}%
\definecolor{currentstroke}{rgb}{1.000000,1.000000,1.000000}%
\pgfsetstrokecolor{currentstroke}%
\pgfsetdash{}{0pt}%
\pgfpathmoveto{\pgfqpoint{1.766805in}{3.461988in}}%
\pgfpathcurveto{\pgfqpoint{1.777855in}{3.461988in}}{\pgfqpoint{1.788454in}{3.466378in}}{\pgfqpoint{1.796268in}{3.474192in}}%
\pgfpathcurveto{\pgfqpoint{1.804082in}{3.482006in}}{\pgfqpoint{1.808472in}{3.492605in}}{\pgfqpoint{1.808472in}{3.503655in}}%
\pgfpathcurveto{\pgfqpoint{1.808472in}{3.514705in}}{\pgfqpoint{1.804082in}{3.525304in}}{\pgfqpoint{1.796268in}{3.533118in}}%
\pgfpathcurveto{\pgfqpoint{1.788454in}{3.540931in}}{\pgfqpoint{1.777855in}{3.545321in}}{\pgfqpoint{1.766805in}{3.545321in}}%
\pgfpathcurveto{\pgfqpoint{1.755755in}{3.545321in}}{\pgfqpoint{1.745156in}{3.540931in}}{\pgfqpoint{1.737343in}{3.533118in}}%
\pgfpathcurveto{\pgfqpoint{1.729529in}{3.525304in}}{\pgfqpoint{1.725139in}{3.514705in}}{\pgfqpoint{1.725139in}{3.503655in}}%
\pgfpathcurveto{\pgfqpoint{1.725139in}{3.492605in}}{\pgfqpoint{1.729529in}{3.482006in}}{\pgfqpoint{1.737343in}{3.474192in}}%
\pgfpathcurveto{\pgfqpoint{1.745156in}{3.466378in}}{\pgfqpoint{1.755755in}{3.461988in}}{\pgfqpoint{1.766805in}{3.461988in}}%
\pgfpathclose%
\pgfusepath{stroke,fill}%
\end{pgfscope}%
\begin{pgfscope}%
\pgfpathrectangle{\pgfqpoint{0.481978in}{0.331635in}}{\pgfqpoint{9.300000in}{7.700000in}}%
\pgfusepath{clip}%
\pgfsetbuttcap%
\pgfsetroundjoin%
\definecolor{currentfill}{rgb}{0.870588,0.733333,0.607843}%
\pgfsetfillcolor{currentfill}%
\pgfsetlinewidth{0.481800pt}%
\definecolor{currentstroke}{rgb}{1.000000,1.000000,1.000000}%
\pgfsetstrokecolor{currentstroke}%
\pgfsetdash{}{0pt}%
\pgfpathmoveto{\pgfqpoint{4.977562in}{4.240982in}}%
\pgfpathcurveto{\pgfqpoint{4.988613in}{4.240982in}}{\pgfqpoint{4.999212in}{4.245373in}}{\pgfqpoint{5.007025in}{4.253186in}}%
\pgfpathcurveto{\pgfqpoint{5.014839in}{4.261000in}}{\pgfqpoint{5.019229in}{4.271599in}}{\pgfqpoint{5.019229in}{4.282649in}}%
\pgfpathcurveto{\pgfqpoint{5.019229in}{4.293699in}}{\pgfqpoint{5.014839in}{4.304298in}}{\pgfqpoint{5.007025in}{4.312112in}}%
\pgfpathcurveto{\pgfqpoint{4.999212in}{4.319926in}}{\pgfqpoint{4.988613in}{4.324316in}}{\pgfqpoint{4.977562in}{4.324316in}}%
\pgfpathcurveto{\pgfqpoint{4.966512in}{4.324316in}}{\pgfqpoint{4.955913in}{4.319926in}}{\pgfqpoint{4.948100in}{4.312112in}}%
\pgfpathcurveto{\pgfqpoint{4.940286in}{4.304298in}}{\pgfqpoint{4.935896in}{4.293699in}}{\pgfqpoint{4.935896in}{4.282649in}}%
\pgfpathcurveto{\pgfqpoint{4.935896in}{4.271599in}}{\pgfqpoint{4.940286in}{4.261000in}}{\pgfqpoint{4.948100in}{4.253186in}}%
\pgfpathcurveto{\pgfqpoint{4.955913in}{4.245373in}}{\pgfqpoint{4.966512in}{4.240982in}}{\pgfqpoint{4.977562in}{4.240982in}}%
\pgfpathclose%
\pgfusepath{stroke,fill}%
\end{pgfscope}%
\begin{pgfscope}%
\pgfpathrectangle{\pgfqpoint{0.481978in}{0.331635in}}{\pgfqpoint{9.300000in}{7.700000in}}%
\pgfusepath{clip}%
\pgfsetbuttcap%
\pgfsetroundjoin%
\definecolor{currentfill}{rgb}{0.870588,0.733333,0.607843}%
\pgfsetfillcolor{currentfill}%
\pgfsetlinewidth{0.481800pt}%
\definecolor{currentstroke}{rgb}{1.000000,1.000000,1.000000}%
\pgfsetstrokecolor{currentstroke}%
\pgfsetdash{}{0pt}%
\pgfpathmoveto{\pgfqpoint{5.616567in}{3.948372in}}%
\pgfpathcurveto{\pgfqpoint{5.627617in}{3.948372in}}{\pgfqpoint{5.638216in}{3.952762in}}{\pgfqpoint{5.646030in}{3.960576in}}%
\pgfpathcurveto{\pgfqpoint{5.653844in}{3.968390in}}{\pgfqpoint{5.658234in}{3.978989in}}{\pgfqpoint{5.658234in}{3.990039in}}%
\pgfpathcurveto{\pgfqpoint{5.658234in}{4.001089in}}{\pgfqpoint{5.653844in}{4.011688in}}{\pgfqpoint{5.646030in}{4.019502in}}%
\pgfpathcurveto{\pgfqpoint{5.638216in}{4.027315in}}{\pgfqpoint{5.627617in}{4.031705in}}{\pgfqpoint{5.616567in}{4.031705in}}%
\pgfpathcurveto{\pgfqpoint{5.605517in}{4.031705in}}{\pgfqpoint{5.594918in}{4.027315in}}{\pgfqpoint{5.587105in}{4.019502in}}%
\pgfpathcurveto{\pgfqpoint{5.579291in}{4.011688in}}{\pgfqpoint{5.574901in}{4.001089in}}{\pgfqpoint{5.574901in}{3.990039in}}%
\pgfpathcurveto{\pgfqpoint{5.574901in}{3.978989in}}{\pgfqpoint{5.579291in}{3.968390in}}{\pgfqpoint{5.587105in}{3.960576in}}%
\pgfpathcurveto{\pgfqpoint{5.594918in}{3.952762in}}{\pgfqpoint{5.605517in}{3.948372in}}{\pgfqpoint{5.616567in}{3.948372in}}%
\pgfpathclose%
\pgfusepath{stroke,fill}%
\end{pgfscope}%
\begin{pgfscope}%
\pgfpathrectangle{\pgfqpoint{0.481978in}{0.331635in}}{\pgfqpoint{9.300000in}{7.700000in}}%
\pgfusepath{clip}%
\pgfsetbuttcap%
\pgfsetroundjoin%
\definecolor{currentfill}{rgb}{0.870588,0.733333,0.607843}%
\pgfsetfillcolor{currentfill}%
\pgfsetlinewidth{0.481800pt}%
\definecolor{currentstroke}{rgb}{1.000000,1.000000,1.000000}%
\pgfsetstrokecolor{currentstroke}%
\pgfsetdash{}{0pt}%
\pgfpathmoveto{\pgfqpoint{2.747420in}{2.538850in}}%
\pgfpathcurveto{\pgfqpoint{2.758470in}{2.538850in}}{\pgfqpoint{2.769069in}{2.543241in}}{\pgfqpoint{2.776883in}{2.551054in}}%
\pgfpathcurveto{\pgfqpoint{2.784696in}{2.558868in}}{\pgfqpoint{2.789087in}{2.569467in}}{\pgfqpoint{2.789087in}{2.580517in}}%
\pgfpathcurveto{\pgfqpoint{2.789087in}{2.591567in}}{\pgfqpoint{2.784696in}{2.602166in}}{\pgfqpoint{2.776883in}{2.609980in}}%
\pgfpathcurveto{\pgfqpoint{2.769069in}{2.617793in}}{\pgfqpoint{2.758470in}{2.622184in}}{\pgfqpoint{2.747420in}{2.622184in}}%
\pgfpathcurveto{\pgfqpoint{2.736370in}{2.622184in}}{\pgfqpoint{2.725771in}{2.617793in}}{\pgfqpoint{2.717957in}{2.609980in}}%
\pgfpathcurveto{\pgfqpoint{2.710143in}{2.602166in}}{\pgfqpoint{2.705753in}{2.591567in}}{\pgfqpoint{2.705753in}{2.580517in}}%
\pgfpathcurveto{\pgfqpoint{2.705753in}{2.569467in}}{\pgfqpoint{2.710143in}{2.558868in}}{\pgfqpoint{2.717957in}{2.551054in}}%
\pgfpathcurveto{\pgfqpoint{2.725771in}{2.543241in}}{\pgfqpoint{2.736370in}{2.538850in}}{\pgfqpoint{2.747420in}{2.538850in}}%
\pgfpathclose%
\pgfusepath{stroke,fill}%
\end{pgfscope}%
\begin{pgfscope}%
\pgfpathrectangle{\pgfqpoint{0.481978in}{0.331635in}}{\pgfqpoint{9.300000in}{7.700000in}}%
\pgfusepath{clip}%
\pgfsetbuttcap%
\pgfsetroundjoin%
\definecolor{currentfill}{rgb}{0.870588,0.733333,0.607843}%
\pgfsetfillcolor{currentfill}%
\pgfsetlinewidth{0.481800pt}%
\definecolor{currentstroke}{rgb}{1.000000,1.000000,1.000000}%
\pgfsetstrokecolor{currentstroke}%
\pgfsetdash{}{0pt}%
\pgfpathmoveto{\pgfqpoint{3.710302in}{3.401243in}}%
\pgfpathcurveto{\pgfqpoint{3.721352in}{3.401243in}}{\pgfqpoint{3.731951in}{3.405633in}}{\pgfqpoint{3.739765in}{3.413447in}}%
\pgfpathcurveto{\pgfqpoint{3.747578in}{3.421260in}}{\pgfqpoint{3.751969in}{3.431859in}}{\pgfqpoint{3.751969in}{3.442909in}}%
\pgfpathcurveto{\pgfqpoint{3.751969in}{3.453960in}}{\pgfqpoint{3.747578in}{3.464559in}}{\pgfqpoint{3.739765in}{3.472372in}}%
\pgfpathcurveto{\pgfqpoint{3.731951in}{3.480186in}}{\pgfqpoint{3.721352in}{3.484576in}}{\pgfqpoint{3.710302in}{3.484576in}}%
\pgfpathcurveto{\pgfqpoint{3.699252in}{3.484576in}}{\pgfqpoint{3.688653in}{3.480186in}}{\pgfqpoint{3.680839in}{3.472372in}}%
\pgfpathcurveto{\pgfqpoint{3.673026in}{3.464559in}}{\pgfqpoint{3.668635in}{3.453960in}}{\pgfqpoint{3.668635in}{3.442909in}}%
\pgfpathcurveto{\pgfqpoint{3.668635in}{3.431859in}}{\pgfqpoint{3.673026in}{3.421260in}}{\pgfqpoint{3.680839in}{3.413447in}}%
\pgfpathcurveto{\pgfqpoint{3.688653in}{3.405633in}}{\pgfqpoint{3.699252in}{3.401243in}}{\pgfqpoint{3.710302in}{3.401243in}}%
\pgfpathclose%
\pgfusepath{stroke,fill}%
\end{pgfscope}%
\begin{pgfscope}%
\pgfpathrectangle{\pgfqpoint{0.481978in}{0.331635in}}{\pgfqpoint{9.300000in}{7.700000in}}%
\pgfusepath{clip}%
\pgfsetbuttcap%
\pgfsetroundjoin%
\definecolor{currentfill}{rgb}{0.870588,0.733333,0.607843}%
\pgfsetfillcolor{currentfill}%
\pgfsetlinewidth{0.481800pt}%
\definecolor{currentstroke}{rgb}{1.000000,1.000000,1.000000}%
\pgfsetstrokecolor{currentstroke}%
\pgfsetdash{}{0pt}%
\pgfpathmoveto{\pgfqpoint{2.149626in}{4.841912in}}%
\pgfpathcurveto{\pgfqpoint{2.160677in}{4.841912in}}{\pgfqpoint{2.171276in}{4.846302in}}{\pgfqpoint{2.179089in}{4.854116in}}%
\pgfpathcurveto{\pgfqpoint{2.186903in}{4.861929in}}{\pgfqpoint{2.191293in}{4.872528in}}{\pgfqpoint{2.191293in}{4.883578in}}%
\pgfpathcurveto{\pgfqpoint{2.191293in}{4.894629in}}{\pgfqpoint{2.186903in}{4.905228in}}{\pgfqpoint{2.179089in}{4.913041in}}%
\pgfpathcurveto{\pgfqpoint{2.171276in}{4.920855in}}{\pgfqpoint{2.160677in}{4.925245in}}{\pgfqpoint{2.149626in}{4.925245in}}%
\pgfpathcurveto{\pgfqpoint{2.138576in}{4.925245in}}{\pgfqpoint{2.127977in}{4.920855in}}{\pgfqpoint{2.120164in}{4.913041in}}%
\pgfpathcurveto{\pgfqpoint{2.112350in}{4.905228in}}{\pgfqpoint{2.107960in}{4.894629in}}{\pgfqpoint{2.107960in}{4.883578in}}%
\pgfpathcurveto{\pgfqpoint{2.107960in}{4.872528in}}{\pgfqpoint{2.112350in}{4.861929in}}{\pgfqpoint{2.120164in}{4.854116in}}%
\pgfpathcurveto{\pgfqpoint{2.127977in}{4.846302in}}{\pgfqpoint{2.138576in}{4.841912in}}{\pgfqpoint{2.149626in}{4.841912in}}%
\pgfpathclose%
\pgfusepath{stroke,fill}%
\end{pgfscope}%
\begin{pgfscope}%
\pgfpathrectangle{\pgfqpoint{0.481978in}{0.331635in}}{\pgfqpoint{9.300000in}{7.700000in}}%
\pgfusepath{clip}%
\pgfsetbuttcap%
\pgfsetroundjoin%
\definecolor{currentfill}{rgb}{0.870588,0.733333,0.607843}%
\pgfsetfillcolor{currentfill}%
\pgfsetlinewidth{0.481800pt}%
\definecolor{currentstroke}{rgb}{1.000000,1.000000,1.000000}%
\pgfsetstrokecolor{currentstroke}%
\pgfsetdash{}{0pt}%
\pgfpathmoveto{\pgfqpoint{2.077561in}{3.061756in}}%
\pgfpathcurveto{\pgfqpoint{2.088611in}{3.061756in}}{\pgfqpoint{2.099210in}{3.066146in}}{\pgfqpoint{2.107024in}{3.073960in}}%
\pgfpathcurveto{\pgfqpoint{2.114838in}{3.081773in}}{\pgfqpoint{2.119228in}{3.092372in}}{\pgfqpoint{2.119228in}{3.103422in}}%
\pgfpathcurveto{\pgfqpoint{2.119228in}{3.114473in}}{\pgfqpoint{2.114838in}{3.125072in}}{\pgfqpoint{2.107024in}{3.132885in}}%
\pgfpathcurveto{\pgfqpoint{2.099210in}{3.140699in}}{\pgfqpoint{2.088611in}{3.145089in}}{\pgfqpoint{2.077561in}{3.145089in}}%
\pgfpathcurveto{\pgfqpoint{2.066511in}{3.145089in}}{\pgfqpoint{2.055912in}{3.140699in}}{\pgfqpoint{2.048099in}{3.132885in}}%
\pgfpathcurveto{\pgfqpoint{2.040285in}{3.125072in}}{\pgfqpoint{2.035895in}{3.114473in}}{\pgfqpoint{2.035895in}{3.103422in}}%
\pgfpathcurveto{\pgfqpoint{2.035895in}{3.092372in}}{\pgfqpoint{2.040285in}{3.081773in}}{\pgfqpoint{2.048099in}{3.073960in}}%
\pgfpathcurveto{\pgfqpoint{2.055912in}{3.066146in}}{\pgfqpoint{2.066511in}{3.061756in}}{\pgfqpoint{2.077561in}{3.061756in}}%
\pgfpathclose%
\pgfusepath{stroke,fill}%
\end{pgfscope}%
\begin{pgfscope}%
\pgfpathrectangle{\pgfqpoint{0.481978in}{0.331635in}}{\pgfqpoint{9.300000in}{7.700000in}}%
\pgfusepath{clip}%
\pgfsetbuttcap%
\pgfsetroundjoin%
\definecolor{currentfill}{rgb}{0.870588,0.733333,0.607843}%
\pgfsetfillcolor{currentfill}%
\pgfsetlinewidth{0.481800pt}%
\definecolor{currentstroke}{rgb}{1.000000,1.000000,1.000000}%
\pgfsetstrokecolor{currentstroke}%
\pgfsetdash{}{0pt}%
\pgfpathmoveto{\pgfqpoint{1.957320in}{5.188017in}}%
\pgfpathcurveto{\pgfqpoint{1.968370in}{5.188017in}}{\pgfqpoint{1.978969in}{5.192407in}}{\pgfqpoint{1.986783in}{5.200221in}}%
\pgfpathcurveto{\pgfqpoint{1.994596in}{5.208035in}}{\pgfqpoint{1.998987in}{5.218634in}}{\pgfqpoint{1.998987in}{5.229684in}}%
\pgfpathcurveto{\pgfqpoint{1.998987in}{5.240734in}}{\pgfqpoint{1.994596in}{5.251333in}}{\pgfqpoint{1.986783in}{5.259147in}}%
\pgfpathcurveto{\pgfqpoint{1.978969in}{5.266960in}}{\pgfqpoint{1.968370in}{5.271351in}}{\pgfqpoint{1.957320in}{5.271351in}}%
\pgfpathcurveto{\pgfqpoint{1.946270in}{5.271351in}}{\pgfqpoint{1.935671in}{5.266960in}}{\pgfqpoint{1.927857in}{5.259147in}}%
\pgfpathcurveto{\pgfqpoint{1.920044in}{5.251333in}}{\pgfqpoint{1.915653in}{5.240734in}}{\pgfqpoint{1.915653in}{5.229684in}}%
\pgfpathcurveto{\pgfqpoint{1.915653in}{5.218634in}}{\pgfqpoint{1.920044in}{5.208035in}}{\pgfqpoint{1.927857in}{5.200221in}}%
\pgfpathcurveto{\pgfqpoint{1.935671in}{5.192407in}}{\pgfqpoint{1.946270in}{5.188017in}}{\pgfqpoint{1.957320in}{5.188017in}}%
\pgfpathclose%
\pgfusepath{stroke,fill}%
\end{pgfscope}%
\begin{pgfscope}%
\pgfpathrectangle{\pgfqpoint{0.481978in}{0.331635in}}{\pgfqpoint{9.300000in}{7.700000in}}%
\pgfusepath{clip}%
\pgfsetbuttcap%
\pgfsetroundjoin%
\definecolor{currentfill}{rgb}{0.870588,0.733333,0.607843}%
\pgfsetfillcolor{currentfill}%
\pgfsetlinewidth{0.481800pt}%
\definecolor{currentstroke}{rgb}{1.000000,1.000000,1.000000}%
\pgfsetstrokecolor{currentstroke}%
\pgfsetdash{}{0pt}%
\pgfpathmoveto{\pgfqpoint{2.660918in}{3.508920in}}%
\pgfpathcurveto{\pgfqpoint{2.671968in}{3.508920in}}{\pgfqpoint{2.682567in}{3.513311in}}{\pgfqpoint{2.690380in}{3.521124in}}%
\pgfpathcurveto{\pgfqpoint{2.698194in}{3.528938in}}{\pgfqpoint{2.702584in}{3.539537in}}{\pgfqpoint{2.702584in}{3.550587in}}%
\pgfpathcurveto{\pgfqpoint{2.702584in}{3.561637in}}{\pgfqpoint{2.698194in}{3.572236in}}{\pgfqpoint{2.690380in}{3.580050in}}%
\pgfpathcurveto{\pgfqpoint{2.682567in}{3.587864in}}{\pgfqpoint{2.671968in}{3.592254in}}{\pgfqpoint{2.660918in}{3.592254in}}%
\pgfpathcurveto{\pgfqpoint{2.649868in}{3.592254in}}{\pgfqpoint{2.639269in}{3.587864in}}{\pgfqpoint{2.631455in}{3.580050in}}%
\pgfpathcurveto{\pgfqpoint{2.623641in}{3.572236in}}{\pgfqpoint{2.619251in}{3.561637in}}{\pgfqpoint{2.619251in}{3.550587in}}%
\pgfpathcurveto{\pgfqpoint{2.619251in}{3.539537in}}{\pgfqpoint{2.623641in}{3.528938in}}{\pgfqpoint{2.631455in}{3.521124in}}%
\pgfpathcurveto{\pgfqpoint{2.639269in}{3.513311in}}{\pgfqpoint{2.649868in}{3.508920in}}{\pgfqpoint{2.660918in}{3.508920in}}%
\pgfpathclose%
\pgfusepath{stroke,fill}%
\end{pgfscope}%
\begin{pgfscope}%
\pgfpathrectangle{\pgfqpoint{0.481978in}{0.331635in}}{\pgfqpoint{9.300000in}{7.700000in}}%
\pgfusepath{clip}%
\pgfsetbuttcap%
\pgfsetroundjoin%
\definecolor{currentfill}{rgb}{0.870588,0.733333,0.607843}%
\pgfsetfillcolor{currentfill}%
\pgfsetlinewidth{0.481800pt}%
\definecolor{currentstroke}{rgb}{1.000000,1.000000,1.000000}%
\pgfsetstrokecolor{currentstroke}%
\pgfsetdash{}{0pt}%
\pgfpathmoveto{\pgfqpoint{7.871019in}{5.508865in}}%
\pgfpathcurveto{\pgfqpoint{7.882069in}{5.508865in}}{\pgfqpoint{7.892668in}{5.513255in}}{\pgfqpoint{7.900481in}{5.521069in}}%
\pgfpathcurveto{\pgfqpoint{7.908295in}{5.528882in}}{\pgfqpoint{7.912685in}{5.539481in}}{\pgfqpoint{7.912685in}{5.550531in}}%
\pgfpathcurveto{\pgfqpoint{7.912685in}{5.561582in}}{\pgfqpoint{7.908295in}{5.572181in}}{\pgfqpoint{7.900481in}{5.579994in}}%
\pgfpathcurveto{\pgfqpoint{7.892668in}{5.587808in}}{\pgfqpoint{7.882069in}{5.592198in}}{\pgfqpoint{7.871019in}{5.592198in}}%
\pgfpathcurveto{\pgfqpoint{7.859968in}{5.592198in}}{\pgfqpoint{7.849369in}{5.587808in}}{\pgfqpoint{7.841556in}{5.579994in}}%
\pgfpathcurveto{\pgfqpoint{7.833742in}{5.572181in}}{\pgfqpoint{7.829352in}{5.561582in}}{\pgfqpoint{7.829352in}{5.550531in}}%
\pgfpathcurveto{\pgfqpoint{7.829352in}{5.539481in}}{\pgfqpoint{7.833742in}{5.528882in}}{\pgfqpoint{7.841556in}{5.521069in}}%
\pgfpathcurveto{\pgfqpoint{7.849369in}{5.513255in}}{\pgfqpoint{7.859968in}{5.508865in}}{\pgfqpoint{7.871019in}{5.508865in}}%
\pgfpathclose%
\pgfusepath{stroke,fill}%
\end{pgfscope}%
\begin{pgfscope}%
\pgfpathrectangle{\pgfqpoint{0.481978in}{0.331635in}}{\pgfqpoint{9.300000in}{7.700000in}}%
\pgfusepath{clip}%
\pgfsetbuttcap%
\pgfsetroundjoin%
\definecolor{currentfill}{rgb}{0.870588,0.733333,0.607843}%
\pgfsetfillcolor{currentfill}%
\pgfsetlinewidth{0.481800pt}%
\definecolor{currentstroke}{rgb}{1.000000,1.000000,1.000000}%
\pgfsetstrokecolor{currentstroke}%
\pgfsetdash{}{0pt}%
\pgfpathmoveto{\pgfqpoint{5.367711in}{4.135078in}}%
\pgfpathcurveto{\pgfqpoint{5.378761in}{4.135078in}}{\pgfqpoint{5.389360in}{4.139468in}}{\pgfqpoint{5.397173in}{4.147281in}}%
\pgfpathcurveto{\pgfqpoint{5.404987in}{4.155095in}}{\pgfqpoint{5.409377in}{4.165694in}}{\pgfqpoint{5.409377in}{4.176744in}}%
\pgfpathcurveto{\pgfqpoint{5.409377in}{4.187794in}}{\pgfqpoint{5.404987in}{4.198393in}}{\pgfqpoint{5.397173in}{4.206207in}}%
\pgfpathcurveto{\pgfqpoint{5.389360in}{4.214021in}}{\pgfqpoint{5.378761in}{4.218411in}}{\pgfqpoint{5.367711in}{4.218411in}}%
\pgfpathcurveto{\pgfqpoint{5.356660in}{4.218411in}}{\pgfqpoint{5.346061in}{4.214021in}}{\pgfqpoint{5.338248in}{4.206207in}}%
\pgfpathcurveto{\pgfqpoint{5.330434in}{4.198393in}}{\pgfqpoint{5.326044in}{4.187794in}}{\pgfqpoint{5.326044in}{4.176744in}}%
\pgfpathcurveto{\pgfqpoint{5.326044in}{4.165694in}}{\pgfqpoint{5.330434in}{4.155095in}}{\pgfqpoint{5.338248in}{4.147281in}}%
\pgfpathcurveto{\pgfqpoint{5.346061in}{4.139468in}}{\pgfqpoint{5.356660in}{4.135078in}}{\pgfqpoint{5.367711in}{4.135078in}}%
\pgfpathclose%
\pgfusepath{stroke,fill}%
\end{pgfscope}%
\begin{pgfscope}%
\pgfpathrectangle{\pgfqpoint{0.481978in}{0.331635in}}{\pgfqpoint{9.300000in}{7.700000in}}%
\pgfusepath{clip}%
\pgfsetbuttcap%
\pgfsetroundjoin%
\definecolor{currentfill}{rgb}{0.870588,0.733333,0.607843}%
\pgfsetfillcolor{currentfill}%
\pgfsetlinewidth{0.481800pt}%
\definecolor{currentstroke}{rgb}{1.000000,1.000000,1.000000}%
\pgfsetstrokecolor{currentstroke}%
\pgfsetdash{}{0pt}%
\pgfpathmoveto{\pgfqpoint{2.321412in}{4.202976in}}%
\pgfpathcurveto{\pgfqpoint{2.332462in}{4.202976in}}{\pgfqpoint{2.343061in}{4.207366in}}{\pgfqpoint{2.350875in}{4.215179in}}%
\pgfpathcurveto{\pgfqpoint{2.358689in}{4.222993in}}{\pgfqpoint{2.363079in}{4.233592in}}{\pgfqpoint{2.363079in}{4.244642in}}%
\pgfpathcurveto{\pgfqpoint{2.363079in}{4.255692in}}{\pgfqpoint{2.358689in}{4.266291in}}{\pgfqpoint{2.350875in}{4.274105in}}%
\pgfpathcurveto{\pgfqpoint{2.343061in}{4.281919in}}{\pgfqpoint{2.332462in}{4.286309in}}{\pgfqpoint{2.321412in}{4.286309in}}%
\pgfpathcurveto{\pgfqpoint{2.310362in}{4.286309in}}{\pgfqpoint{2.299763in}{4.281919in}}{\pgfqpoint{2.291949in}{4.274105in}}%
\pgfpathcurveto{\pgfqpoint{2.284136in}{4.266291in}}{\pgfqpoint{2.279745in}{4.255692in}}{\pgfqpoint{2.279745in}{4.244642in}}%
\pgfpathcurveto{\pgfqpoint{2.279745in}{4.233592in}}{\pgfqpoint{2.284136in}{4.222993in}}{\pgfqpoint{2.291949in}{4.215179in}}%
\pgfpathcurveto{\pgfqpoint{2.299763in}{4.207366in}}{\pgfqpoint{2.310362in}{4.202976in}}{\pgfqpoint{2.321412in}{4.202976in}}%
\pgfpathclose%
\pgfusepath{stroke,fill}%
\end{pgfscope}%
\begin{pgfscope}%
\pgfpathrectangle{\pgfqpoint{0.481978in}{0.331635in}}{\pgfqpoint{9.300000in}{7.700000in}}%
\pgfusepath{clip}%
\pgfsetbuttcap%
\pgfsetroundjoin%
\definecolor{currentfill}{rgb}{0.870588,0.733333,0.607843}%
\pgfsetfillcolor{currentfill}%
\pgfsetlinewidth{0.481800pt}%
\definecolor{currentstroke}{rgb}{1.000000,1.000000,1.000000}%
\pgfsetstrokecolor{currentstroke}%
\pgfsetdash{}{0pt}%
\pgfpathmoveto{\pgfqpoint{7.119603in}{6.261982in}}%
\pgfpathcurveto{\pgfqpoint{7.130653in}{6.261982in}}{\pgfqpoint{7.141252in}{6.266372in}}{\pgfqpoint{7.149065in}{6.274186in}}%
\pgfpathcurveto{\pgfqpoint{7.156879in}{6.282000in}}{\pgfqpoint{7.161269in}{6.292599in}}{\pgfqpoint{7.161269in}{6.303649in}}%
\pgfpathcurveto{\pgfqpoint{7.161269in}{6.314699in}}{\pgfqpoint{7.156879in}{6.325298in}}{\pgfqpoint{7.149065in}{6.333111in}}%
\pgfpathcurveto{\pgfqpoint{7.141252in}{6.340925in}}{\pgfqpoint{7.130653in}{6.345315in}}{\pgfqpoint{7.119603in}{6.345315in}}%
\pgfpathcurveto{\pgfqpoint{7.108553in}{6.345315in}}{\pgfqpoint{7.097954in}{6.340925in}}{\pgfqpoint{7.090140in}{6.333111in}}%
\pgfpathcurveto{\pgfqpoint{7.082326in}{6.325298in}}{\pgfqpoint{7.077936in}{6.314699in}}{\pgfqpoint{7.077936in}{6.303649in}}%
\pgfpathcurveto{\pgfqpoint{7.077936in}{6.292599in}}{\pgfqpoint{7.082326in}{6.282000in}}{\pgfqpoint{7.090140in}{6.274186in}}%
\pgfpathcurveto{\pgfqpoint{7.097954in}{6.266372in}}{\pgfqpoint{7.108553in}{6.261982in}}{\pgfqpoint{7.119603in}{6.261982in}}%
\pgfpathclose%
\pgfusepath{stroke,fill}%
\end{pgfscope}%
\begin{pgfscope}%
\pgfpathrectangle{\pgfqpoint{0.481978in}{0.331635in}}{\pgfqpoint{9.300000in}{7.700000in}}%
\pgfusepath{clip}%
\pgfsetbuttcap%
\pgfsetroundjoin%
\definecolor{currentfill}{rgb}{0.870588,0.733333,0.607843}%
\pgfsetfillcolor{currentfill}%
\pgfsetlinewidth{0.481800pt}%
\definecolor{currentstroke}{rgb}{1.000000,1.000000,1.000000}%
\pgfsetstrokecolor{currentstroke}%
\pgfsetdash{}{0pt}%
\pgfpathmoveto{\pgfqpoint{6.487284in}{5.462464in}}%
\pgfpathcurveto{\pgfqpoint{6.498334in}{5.462464in}}{\pgfqpoint{6.508933in}{5.466854in}}{\pgfqpoint{6.516747in}{5.474668in}}%
\pgfpathcurveto{\pgfqpoint{6.524560in}{5.482482in}}{\pgfqpoint{6.528951in}{5.493081in}}{\pgfqpoint{6.528951in}{5.504131in}}%
\pgfpathcurveto{\pgfqpoint{6.528951in}{5.515181in}}{\pgfqpoint{6.524560in}{5.525780in}}{\pgfqpoint{6.516747in}{5.533594in}}%
\pgfpathcurveto{\pgfqpoint{6.508933in}{5.541407in}}{\pgfqpoint{6.498334in}{5.545798in}}{\pgfqpoint{6.487284in}{5.545798in}}%
\pgfpathcurveto{\pgfqpoint{6.476234in}{5.545798in}}{\pgfqpoint{6.465635in}{5.541407in}}{\pgfqpoint{6.457821in}{5.533594in}}%
\pgfpathcurveto{\pgfqpoint{6.450008in}{5.525780in}}{\pgfqpoint{6.445617in}{5.515181in}}{\pgfqpoint{6.445617in}{5.504131in}}%
\pgfpathcurveto{\pgfqpoint{6.445617in}{5.493081in}}{\pgfqpoint{6.450008in}{5.482482in}}{\pgfqpoint{6.457821in}{5.474668in}}%
\pgfpathcurveto{\pgfqpoint{6.465635in}{5.466854in}}{\pgfqpoint{6.476234in}{5.462464in}}{\pgfqpoint{6.487284in}{5.462464in}}%
\pgfpathclose%
\pgfusepath{stroke,fill}%
\end{pgfscope}%
\begin{pgfscope}%
\pgfpathrectangle{\pgfqpoint{0.481978in}{0.331635in}}{\pgfqpoint{9.300000in}{7.700000in}}%
\pgfusepath{clip}%
\pgfsetbuttcap%
\pgfsetroundjoin%
\definecolor{currentfill}{rgb}{0.870588,0.733333,0.607843}%
\pgfsetfillcolor{currentfill}%
\pgfsetlinewidth{0.481800pt}%
\definecolor{currentstroke}{rgb}{1.000000,1.000000,1.000000}%
\pgfsetstrokecolor{currentstroke}%
\pgfsetdash{}{0pt}%
\pgfpathmoveto{\pgfqpoint{8.920274in}{3.513533in}}%
\pgfpathcurveto{\pgfqpoint{8.931324in}{3.513533in}}{\pgfqpoint{8.941923in}{3.517923in}}{\pgfqpoint{8.949737in}{3.525737in}}%
\pgfpathcurveto{\pgfqpoint{8.957550in}{3.533551in}}{\pgfqpoint{8.961941in}{3.544150in}}{\pgfqpoint{8.961941in}{3.555200in}}%
\pgfpathcurveto{\pgfqpoint{8.961941in}{3.566250in}}{\pgfqpoint{8.957550in}{3.576849in}}{\pgfqpoint{8.949737in}{3.584663in}}%
\pgfpathcurveto{\pgfqpoint{8.941923in}{3.592476in}}{\pgfqpoint{8.931324in}{3.596867in}}{\pgfqpoint{8.920274in}{3.596867in}}%
\pgfpathcurveto{\pgfqpoint{8.909224in}{3.596867in}}{\pgfqpoint{8.898625in}{3.592476in}}{\pgfqpoint{8.890811in}{3.584663in}}%
\pgfpathcurveto{\pgfqpoint{8.882998in}{3.576849in}}{\pgfqpoint{8.878607in}{3.566250in}}{\pgfqpoint{8.878607in}{3.555200in}}%
\pgfpathcurveto{\pgfqpoint{8.878607in}{3.544150in}}{\pgfqpoint{8.882998in}{3.533551in}}{\pgfqpoint{8.890811in}{3.525737in}}%
\pgfpathcurveto{\pgfqpoint{8.898625in}{3.517923in}}{\pgfqpoint{8.909224in}{3.513533in}}{\pgfqpoint{8.920274in}{3.513533in}}%
\pgfpathclose%
\pgfusepath{stroke,fill}%
\end{pgfscope}%
\begin{pgfscope}%
\pgfpathrectangle{\pgfqpoint{0.481978in}{0.331635in}}{\pgfqpoint{9.300000in}{7.700000in}}%
\pgfusepath{clip}%
\pgfsetbuttcap%
\pgfsetroundjoin%
\definecolor{currentfill}{rgb}{0.870588,0.733333,0.607843}%
\pgfsetfillcolor{currentfill}%
\pgfsetlinewidth{0.481800pt}%
\definecolor{currentstroke}{rgb}{1.000000,1.000000,1.000000}%
\pgfsetstrokecolor{currentstroke}%
\pgfsetdash{}{0pt}%
\pgfpathmoveto{\pgfqpoint{7.303090in}{6.103993in}}%
\pgfpathcurveto{\pgfqpoint{7.314140in}{6.103993in}}{\pgfqpoint{7.324739in}{6.108384in}}{\pgfqpoint{7.332553in}{6.116197in}}%
\pgfpathcurveto{\pgfqpoint{7.340366in}{6.124011in}}{\pgfqpoint{7.344757in}{6.134610in}}{\pgfqpoint{7.344757in}{6.145660in}}%
\pgfpathcurveto{\pgfqpoint{7.344757in}{6.156710in}}{\pgfqpoint{7.340366in}{6.167309in}}{\pgfqpoint{7.332553in}{6.175123in}}%
\pgfpathcurveto{\pgfqpoint{7.324739in}{6.182936in}}{\pgfqpoint{7.314140in}{6.187327in}}{\pgfqpoint{7.303090in}{6.187327in}}%
\pgfpathcurveto{\pgfqpoint{7.292040in}{6.187327in}}{\pgfqpoint{7.281441in}{6.182936in}}{\pgfqpoint{7.273627in}{6.175123in}}%
\pgfpathcurveto{\pgfqpoint{7.265814in}{6.167309in}}{\pgfqpoint{7.261423in}{6.156710in}}{\pgfqpoint{7.261423in}{6.145660in}}%
\pgfpathcurveto{\pgfqpoint{7.261423in}{6.134610in}}{\pgfqpoint{7.265814in}{6.124011in}}{\pgfqpoint{7.273627in}{6.116197in}}%
\pgfpathcurveto{\pgfqpoint{7.281441in}{6.108384in}}{\pgfqpoint{7.292040in}{6.103993in}}{\pgfqpoint{7.303090in}{6.103993in}}%
\pgfpathclose%
\pgfusepath{stroke,fill}%
\end{pgfscope}%
\begin{pgfscope}%
\pgfpathrectangle{\pgfqpoint{0.481978in}{0.331635in}}{\pgfqpoint{9.300000in}{7.700000in}}%
\pgfusepath{clip}%
\pgfsetbuttcap%
\pgfsetroundjoin%
\definecolor{currentfill}{rgb}{0.870588,0.733333,0.607843}%
\pgfsetfillcolor{currentfill}%
\pgfsetlinewidth{0.481800pt}%
\definecolor{currentstroke}{rgb}{1.000000,1.000000,1.000000}%
\pgfsetstrokecolor{currentstroke}%
\pgfsetdash{}{0pt}%
\pgfpathmoveto{\pgfqpoint{9.189203in}{4.278681in}}%
\pgfpathcurveto{\pgfqpoint{9.200253in}{4.278681in}}{\pgfqpoint{9.210852in}{4.283071in}}{\pgfqpoint{9.218666in}{4.290885in}}%
\pgfpathcurveto{\pgfqpoint{9.226480in}{4.298699in}}{\pgfqpoint{9.230870in}{4.309298in}}{\pgfqpoint{9.230870in}{4.320348in}}%
\pgfpathcurveto{\pgfqpoint{9.230870in}{4.331398in}}{\pgfqpoint{9.226480in}{4.341997in}}{\pgfqpoint{9.218666in}{4.349811in}}%
\pgfpathcurveto{\pgfqpoint{9.210852in}{4.357624in}}{\pgfqpoint{9.200253in}{4.362014in}}{\pgfqpoint{9.189203in}{4.362014in}}%
\pgfpathcurveto{\pgfqpoint{9.178153in}{4.362014in}}{\pgfqpoint{9.167554in}{4.357624in}}{\pgfqpoint{9.159740in}{4.349811in}}%
\pgfpathcurveto{\pgfqpoint{9.151927in}{4.341997in}}{\pgfqpoint{9.147537in}{4.331398in}}{\pgfqpoint{9.147537in}{4.320348in}}%
\pgfpathcurveto{\pgfqpoint{9.147537in}{4.309298in}}{\pgfqpoint{9.151927in}{4.298699in}}{\pgfqpoint{9.159740in}{4.290885in}}%
\pgfpathcurveto{\pgfqpoint{9.167554in}{4.283071in}}{\pgfqpoint{9.178153in}{4.278681in}}{\pgfqpoint{9.189203in}{4.278681in}}%
\pgfpathclose%
\pgfusepath{stroke,fill}%
\end{pgfscope}%
\begin{pgfscope}%
\pgfpathrectangle{\pgfqpoint{0.481978in}{0.331635in}}{\pgfqpoint{9.300000in}{7.700000in}}%
\pgfusepath{clip}%
\pgfsetbuttcap%
\pgfsetroundjoin%
\definecolor{currentfill}{rgb}{0.631373,0.788235,0.956863}%
\pgfsetfillcolor{currentfill}%
\pgfsetlinewidth{1.003750pt}%
\definecolor{currentstroke}{rgb}{0.631373,0.788235,0.956863}%
\pgfsetstrokecolor{currentstroke}%
\pgfsetdash{}{0pt}%
\pgfsys@defobject{currentmarker}{\pgfqpoint{-0.041667in}{-0.041667in}}{\pgfqpoint{0.041667in}{0.041667in}}{%
\pgfpathmoveto{\pgfqpoint{0.000000in}{-0.041667in}}%
\pgfpathcurveto{\pgfqpoint{0.011050in}{-0.041667in}}{\pgfqpoint{0.021649in}{-0.037276in}}{\pgfqpoint{0.029463in}{-0.029463in}}%
\pgfpathcurveto{\pgfqpoint{0.037276in}{-0.021649in}}{\pgfqpoint{0.041667in}{-0.011050in}}{\pgfqpoint{0.041667in}{0.000000in}}%
\pgfpathcurveto{\pgfqpoint{0.041667in}{0.011050in}}{\pgfqpoint{0.037276in}{0.021649in}}{\pgfqpoint{0.029463in}{0.029463in}}%
\pgfpathcurveto{\pgfqpoint{0.021649in}{0.037276in}}{\pgfqpoint{0.011050in}{0.041667in}}{\pgfqpoint{0.000000in}{0.041667in}}%
\pgfpathcurveto{\pgfqpoint{-0.011050in}{0.041667in}}{\pgfqpoint{-0.021649in}{0.037276in}}{\pgfqpoint{-0.029463in}{0.029463in}}%
\pgfpathcurveto{\pgfqpoint{-0.037276in}{0.021649in}}{\pgfqpoint{-0.041667in}{0.011050in}}{\pgfqpoint{-0.041667in}{0.000000in}}%
\pgfpathcurveto{\pgfqpoint{-0.041667in}{-0.011050in}}{\pgfqpoint{-0.037276in}{-0.021649in}}{\pgfqpoint{-0.029463in}{-0.029463in}}%
\pgfpathcurveto{\pgfqpoint{-0.021649in}{-0.037276in}}{\pgfqpoint{-0.011050in}{-0.041667in}}{\pgfqpoint{0.000000in}{-0.041667in}}%
\pgfpathclose%
\pgfusepath{stroke,fill}%
}%
\end{pgfscope}%
\begin{pgfscope}%
\pgfpathrectangle{\pgfqpoint{0.481978in}{0.331635in}}{\pgfqpoint{9.300000in}{7.700000in}}%
\pgfusepath{clip}%
\pgfsetbuttcap%
\pgfsetroundjoin%
\definecolor{currentfill}{rgb}{1.000000,0.705882,0.509804}%
\pgfsetfillcolor{currentfill}%
\pgfsetlinewidth{1.003750pt}%
\definecolor{currentstroke}{rgb}{1.000000,0.705882,0.509804}%
\pgfsetstrokecolor{currentstroke}%
\pgfsetdash{}{0pt}%
\pgfsys@defobject{currentmarker}{\pgfqpoint{-0.041667in}{-0.041667in}}{\pgfqpoint{0.041667in}{0.041667in}}{%
\pgfpathmoveto{\pgfqpoint{0.000000in}{-0.041667in}}%
\pgfpathcurveto{\pgfqpoint{0.011050in}{-0.041667in}}{\pgfqpoint{0.021649in}{-0.037276in}}{\pgfqpoint{0.029463in}{-0.029463in}}%
\pgfpathcurveto{\pgfqpoint{0.037276in}{-0.021649in}}{\pgfqpoint{0.041667in}{-0.011050in}}{\pgfqpoint{0.041667in}{0.000000in}}%
\pgfpathcurveto{\pgfqpoint{0.041667in}{0.011050in}}{\pgfqpoint{0.037276in}{0.021649in}}{\pgfqpoint{0.029463in}{0.029463in}}%
\pgfpathcurveto{\pgfqpoint{0.021649in}{0.037276in}}{\pgfqpoint{0.011050in}{0.041667in}}{\pgfqpoint{0.000000in}{0.041667in}}%
\pgfpathcurveto{\pgfqpoint{-0.011050in}{0.041667in}}{\pgfqpoint{-0.021649in}{0.037276in}}{\pgfqpoint{-0.029463in}{0.029463in}}%
\pgfpathcurveto{\pgfqpoint{-0.037276in}{0.021649in}}{\pgfqpoint{-0.041667in}{0.011050in}}{\pgfqpoint{-0.041667in}{0.000000in}}%
\pgfpathcurveto{\pgfqpoint{-0.041667in}{-0.011050in}}{\pgfqpoint{-0.037276in}{-0.021649in}}{\pgfqpoint{-0.029463in}{-0.029463in}}%
\pgfpathcurveto{\pgfqpoint{-0.021649in}{-0.037276in}}{\pgfqpoint{-0.011050in}{-0.041667in}}{\pgfqpoint{0.000000in}{-0.041667in}}%
\pgfpathclose%
\pgfusepath{stroke,fill}%
}%
\end{pgfscope}%
\begin{pgfscope}%
\pgfpathrectangle{\pgfqpoint{0.481978in}{0.331635in}}{\pgfqpoint{9.300000in}{7.700000in}}%
\pgfusepath{clip}%
\pgfsetbuttcap%
\pgfsetroundjoin%
\definecolor{currentfill}{rgb}{0.552941,0.898039,0.631373}%
\pgfsetfillcolor{currentfill}%
\pgfsetlinewidth{1.003750pt}%
\definecolor{currentstroke}{rgb}{0.552941,0.898039,0.631373}%
\pgfsetstrokecolor{currentstroke}%
\pgfsetdash{}{0pt}%
\pgfsys@defobject{currentmarker}{\pgfqpoint{-0.041667in}{-0.041667in}}{\pgfqpoint{0.041667in}{0.041667in}}{%
\pgfpathmoveto{\pgfqpoint{0.000000in}{-0.041667in}}%
\pgfpathcurveto{\pgfqpoint{0.011050in}{-0.041667in}}{\pgfqpoint{0.021649in}{-0.037276in}}{\pgfqpoint{0.029463in}{-0.029463in}}%
\pgfpathcurveto{\pgfqpoint{0.037276in}{-0.021649in}}{\pgfqpoint{0.041667in}{-0.011050in}}{\pgfqpoint{0.041667in}{0.000000in}}%
\pgfpathcurveto{\pgfqpoint{0.041667in}{0.011050in}}{\pgfqpoint{0.037276in}{0.021649in}}{\pgfqpoint{0.029463in}{0.029463in}}%
\pgfpathcurveto{\pgfqpoint{0.021649in}{0.037276in}}{\pgfqpoint{0.011050in}{0.041667in}}{\pgfqpoint{0.000000in}{0.041667in}}%
\pgfpathcurveto{\pgfqpoint{-0.011050in}{0.041667in}}{\pgfqpoint{-0.021649in}{0.037276in}}{\pgfqpoint{-0.029463in}{0.029463in}}%
\pgfpathcurveto{\pgfqpoint{-0.037276in}{0.021649in}}{\pgfqpoint{-0.041667in}{0.011050in}}{\pgfqpoint{-0.041667in}{0.000000in}}%
\pgfpathcurveto{\pgfqpoint{-0.041667in}{-0.011050in}}{\pgfqpoint{-0.037276in}{-0.021649in}}{\pgfqpoint{-0.029463in}{-0.029463in}}%
\pgfpathcurveto{\pgfqpoint{-0.021649in}{-0.037276in}}{\pgfqpoint{-0.011050in}{-0.041667in}}{\pgfqpoint{0.000000in}{-0.041667in}}%
\pgfpathclose%
\pgfusepath{stroke,fill}%
}%
\end{pgfscope}%
\begin{pgfscope}%
\pgfpathrectangle{\pgfqpoint{0.481978in}{0.331635in}}{\pgfqpoint{9.300000in}{7.700000in}}%
\pgfusepath{clip}%
\pgfsetbuttcap%
\pgfsetroundjoin%
\definecolor{currentfill}{rgb}{1.000000,0.623529,0.607843}%
\pgfsetfillcolor{currentfill}%
\pgfsetlinewidth{1.003750pt}%
\definecolor{currentstroke}{rgb}{1.000000,0.623529,0.607843}%
\pgfsetstrokecolor{currentstroke}%
\pgfsetdash{}{0pt}%
\pgfsys@defobject{currentmarker}{\pgfqpoint{-0.041667in}{-0.041667in}}{\pgfqpoint{0.041667in}{0.041667in}}{%
\pgfpathmoveto{\pgfqpoint{0.000000in}{-0.041667in}}%
\pgfpathcurveto{\pgfqpoint{0.011050in}{-0.041667in}}{\pgfqpoint{0.021649in}{-0.037276in}}{\pgfqpoint{0.029463in}{-0.029463in}}%
\pgfpathcurveto{\pgfqpoint{0.037276in}{-0.021649in}}{\pgfqpoint{0.041667in}{-0.011050in}}{\pgfqpoint{0.041667in}{0.000000in}}%
\pgfpathcurveto{\pgfqpoint{0.041667in}{0.011050in}}{\pgfqpoint{0.037276in}{0.021649in}}{\pgfqpoint{0.029463in}{0.029463in}}%
\pgfpathcurveto{\pgfqpoint{0.021649in}{0.037276in}}{\pgfqpoint{0.011050in}{0.041667in}}{\pgfqpoint{0.000000in}{0.041667in}}%
\pgfpathcurveto{\pgfqpoint{-0.011050in}{0.041667in}}{\pgfqpoint{-0.021649in}{0.037276in}}{\pgfqpoint{-0.029463in}{0.029463in}}%
\pgfpathcurveto{\pgfqpoint{-0.037276in}{0.021649in}}{\pgfqpoint{-0.041667in}{0.011050in}}{\pgfqpoint{-0.041667in}{0.000000in}}%
\pgfpathcurveto{\pgfqpoint{-0.041667in}{-0.011050in}}{\pgfqpoint{-0.037276in}{-0.021649in}}{\pgfqpoint{-0.029463in}{-0.029463in}}%
\pgfpathcurveto{\pgfqpoint{-0.021649in}{-0.037276in}}{\pgfqpoint{-0.011050in}{-0.041667in}}{\pgfqpoint{0.000000in}{-0.041667in}}%
\pgfpathclose%
\pgfusepath{stroke,fill}%
}%
\end{pgfscope}%
\begin{pgfscope}%
\pgfpathrectangle{\pgfqpoint{0.481978in}{0.331635in}}{\pgfqpoint{9.300000in}{7.700000in}}%
\pgfusepath{clip}%
\pgfsetbuttcap%
\pgfsetroundjoin%
\definecolor{currentfill}{rgb}{0.815686,0.733333,1.000000}%
\pgfsetfillcolor{currentfill}%
\pgfsetlinewidth{1.003750pt}%
\definecolor{currentstroke}{rgb}{0.815686,0.733333,1.000000}%
\pgfsetstrokecolor{currentstroke}%
\pgfsetdash{}{0pt}%
\pgfsys@defobject{currentmarker}{\pgfqpoint{-0.041667in}{-0.041667in}}{\pgfqpoint{0.041667in}{0.041667in}}{%
\pgfpathmoveto{\pgfqpoint{0.000000in}{-0.041667in}}%
\pgfpathcurveto{\pgfqpoint{0.011050in}{-0.041667in}}{\pgfqpoint{0.021649in}{-0.037276in}}{\pgfqpoint{0.029463in}{-0.029463in}}%
\pgfpathcurveto{\pgfqpoint{0.037276in}{-0.021649in}}{\pgfqpoint{0.041667in}{-0.011050in}}{\pgfqpoint{0.041667in}{0.000000in}}%
\pgfpathcurveto{\pgfqpoint{0.041667in}{0.011050in}}{\pgfqpoint{0.037276in}{0.021649in}}{\pgfqpoint{0.029463in}{0.029463in}}%
\pgfpathcurveto{\pgfqpoint{0.021649in}{0.037276in}}{\pgfqpoint{0.011050in}{0.041667in}}{\pgfqpoint{0.000000in}{0.041667in}}%
\pgfpathcurveto{\pgfqpoint{-0.011050in}{0.041667in}}{\pgfqpoint{-0.021649in}{0.037276in}}{\pgfqpoint{-0.029463in}{0.029463in}}%
\pgfpathcurveto{\pgfqpoint{-0.037276in}{0.021649in}}{\pgfqpoint{-0.041667in}{0.011050in}}{\pgfqpoint{-0.041667in}{0.000000in}}%
\pgfpathcurveto{\pgfqpoint{-0.041667in}{-0.011050in}}{\pgfqpoint{-0.037276in}{-0.021649in}}{\pgfqpoint{-0.029463in}{-0.029463in}}%
\pgfpathcurveto{\pgfqpoint{-0.021649in}{-0.037276in}}{\pgfqpoint{-0.011050in}{-0.041667in}}{\pgfqpoint{0.000000in}{-0.041667in}}%
\pgfpathclose%
\pgfusepath{stroke,fill}%
}%
\end{pgfscope}%
\begin{pgfscope}%
\pgfpathrectangle{\pgfqpoint{0.481978in}{0.331635in}}{\pgfqpoint{9.300000in}{7.700000in}}%
\pgfusepath{clip}%
\pgfsetbuttcap%
\pgfsetroundjoin%
\definecolor{currentfill}{rgb}{0.870588,0.733333,0.607843}%
\pgfsetfillcolor{currentfill}%
\pgfsetlinewidth{1.003750pt}%
\definecolor{currentstroke}{rgb}{0.870588,0.733333,0.607843}%
\pgfsetstrokecolor{currentstroke}%
\pgfsetdash{}{0pt}%
\pgfsys@defobject{currentmarker}{\pgfqpoint{-0.041667in}{-0.041667in}}{\pgfqpoint{0.041667in}{0.041667in}}{%
\pgfpathmoveto{\pgfqpoint{0.000000in}{-0.041667in}}%
\pgfpathcurveto{\pgfqpoint{0.011050in}{-0.041667in}}{\pgfqpoint{0.021649in}{-0.037276in}}{\pgfqpoint{0.029463in}{-0.029463in}}%
\pgfpathcurveto{\pgfqpoint{0.037276in}{-0.021649in}}{\pgfqpoint{0.041667in}{-0.011050in}}{\pgfqpoint{0.041667in}{0.000000in}}%
\pgfpathcurveto{\pgfqpoint{0.041667in}{0.011050in}}{\pgfqpoint{0.037276in}{0.021649in}}{\pgfqpoint{0.029463in}{0.029463in}}%
\pgfpathcurveto{\pgfqpoint{0.021649in}{0.037276in}}{\pgfqpoint{0.011050in}{0.041667in}}{\pgfqpoint{0.000000in}{0.041667in}}%
\pgfpathcurveto{\pgfqpoint{-0.011050in}{0.041667in}}{\pgfqpoint{-0.021649in}{0.037276in}}{\pgfqpoint{-0.029463in}{0.029463in}}%
\pgfpathcurveto{\pgfqpoint{-0.037276in}{0.021649in}}{\pgfqpoint{-0.041667in}{0.011050in}}{\pgfqpoint{-0.041667in}{0.000000in}}%
\pgfpathcurveto{\pgfqpoint{-0.041667in}{-0.011050in}}{\pgfqpoint{-0.037276in}{-0.021649in}}{\pgfqpoint{-0.029463in}{-0.029463in}}%
\pgfpathcurveto{\pgfqpoint{-0.021649in}{-0.037276in}}{\pgfqpoint{-0.011050in}{-0.041667in}}{\pgfqpoint{0.000000in}{-0.041667in}}%
\pgfpathclose%
\pgfusepath{stroke,fill}%
}%
\end{pgfscope}%
\begin{pgfscope}%
\pgfsetbuttcap%
\pgfsetroundjoin%
\definecolor{currentfill}{rgb}{0.000000,0.000000,0.000000}%
\pgfsetfillcolor{currentfill}%
\pgfsetlinewidth{0.803000pt}%
\definecolor{currentstroke}{rgb}{0.000000,0.000000,0.000000}%
\pgfsetstrokecolor{currentstroke}%
\pgfsetdash{}{0pt}%
\pgfsys@defobject{currentmarker}{\pgfqpoint{0.000000in}{-0.048611in}}{\pgfqpoint{0.000000in}{0.000000in}}{%
\pgfpathmoveto{\pgfqpoint{0.000000in}{0.000000in}}%
\pgfpathlineto{\pgfqpoint{0.000000in}{-0.048611in}}%
\pgfusepath{stroke,fill}%
}%
\begin{pgfscope}%
\pgfsys@transformshift{0.980433in}{0.331635in}%
\pgfsys@useobject{currentmarker}{}%
\end{pgfscope}%
\end{pgfscope}%
\begin{pgfscope}%
\definecolor{textcolor}{rgb}{0.000000,0.000000,0.000000}%
\pgfsetstrokecolor{textcolor}%
\pgfsetfillcolor{textcolor}%
\pgftext[x=0.980433in,y=0.234413in,,top]{\color{textcolor}\sffamily\fontsize{10.000000}{12.000000}\selectfont \ensuremath{-}20}%
\end{pgfscope}%
\begin{pgfscope}%
\pgfsetbuttcap%
\pgfsetroundjoin%
\definecolor{currentfill}{rgb}{0.000000,0.000000,0.000000}%
\pgfsetfillcolor{currentfill}%
\pgfsetlinewidth{0.803000pt}%
\definecolor{currentstroke}{rgb}{0.000000,0.000000,0.000000}%
\pgfsetstrokecolor{currentstroke}%
\pgfsetdash{}{0pt}%
\pgfsys@defobject{currentmarker}{\pgfqpoint{0.000000in}{-0.048611in}}{\pgfqpoint{0.000000in}{0.000000in}}{%
\pgfpathmoveto{\pgfqpoint{0.000000in}{0.000000in}}%
\pgfpathlineto{\pgfqpoint{0.000000in}{-0.048611in}}%
\pgfusepath{stroke,fill}%
}%
\begin{pgfscope}%
\pgfsys@transformshift{2.182672in}{0.331635in}%
\pgfsys@useobject{currentmarker}{}%
\end{pgfscope}%
\end{pgfscope}%
\begin{pgfscope}%
\definecolor{textcolor}{rgb}{0.000000,0.000000,0.000000}%
\pgfsetstrokecolor{textcolor}%
\pgfsetfillcolor{textcolor}%
\pgftext[x=2.182672in,y=0.234413in,,top]{\color{textcolor}\sffamily\fontsize{10.000000}{12.000000}\selectfont \ensuremath{-}15}%
\end{pgfscope}%
\begin{pgfscope}%
\pgfsetbuttcap%
\pgfsetroundjoin%
\definecolor{currentfill}{rgb}{0.000000,0.000000,0.000000}%
\pgfsetfillcolor{currentfill}%
\pgfsetlinewidth{0.803000pt}%
\definecolor{currentstroke}{rgb}{0.000000,0.000000,0.000000}%
\pgfsetstrokecolor{currentstroke}%
\pgfsetdash{}{0pt}%
\pgfsys@defobject{currentmarker}{\pgfqpoint{0.000000in}{-0.048611in}}{\pgfqpoint{0.000000in}{0.000000in}}{%
\pgfpathmoveto{\pgfqpoint{0.000000in}{0.000000in}}%
\pgfpathlineto{\pgfqpoint{0.000000in}{-0.048611in}}%
\pgfusepath{stroke,fill}%
}%
\begin{pgfscope}%
\pgfsys@transformshift{3.384911in}{0.331635in}%
\pgfsys@useobject{currentmarker}{}%
\end{pgfscope}%
\end{pgfscope}%
\begin{pgfscope}%
\definecolor{textcolor}{rgb}{0.000000,0.000000,0.000000}%
\pgfsetstrokecolor{textcolor}%
\pgfsetfillcolor{textcolor}%
\pgftext[x=3.384911in,y=0.234413in,,top]{\color{textcolor}\sffamily\fontsize{10.000000}{12.000000}\selectfont \ensuremath{-}10}%
\end{pgfscope}%
\begin{pgfscope}%
\pgfsetbuttcap%
\pgfsetroundjoin%
\definecolor{currentfill}{rgb}{0.000000,0.000000,0.000000}%
\pgfsetfillcolor{currentfill}%
\pgfsetlinewidth{0.803000pt}%
\definecolor{currentstroke}{rgb}{0.000000,0.000000,0.000000}%
\pgfsetstrokecolor{currentstroke}%
\pgfsetdash{}{0pt}%
\pgfsys@defobject{currentmarker}{\pgfqpoint{0.000000in}{-0.048611in}}{\pgfqpoint{0.000000in}{0.000000in}}{%
\pgfpathmoveto{\pgfqpoint{0.000000in}{0.000000in}}%
\pgfpathlineto{\pgfqpoint{0.000000in}{-0.048611in}}%
\pgfusepath{stroke,fill}%
}%
\begin{pgfscope}%
\pgfsys@transformshift{4.587150in}{0.331635in}%
\pgfsys@useobject{currentmarker}{}%
\end{pgfscope}%
\end{pgfscope}%
\begin{pgfscope}%
\definecolor{textcolor}{rgb}{0.000000,0.000000,0.000000}%
\pgfsetstrokecolor{textcolor}%
\pgfsetfillcolor{textcolor}%
\pgftext[x=4.587150in,y=0.234413in,,top]{\color{textcolor}\sffamily\fontsize{10.000000}{12.000000}\selectfont \ensuremath{-}5}%
\end{pgfscope}%
\begin{pgfscope}%
\pgfsetbuttcap%
\pgfsetroundjoin%
\definecolor{currentfill}{rgb}{0.000000,0.000000,0.000000}%
\pgfsetfillcolor{currentfill}%
\pgfsetlinewidth{0.803000pt}%
\definecolor{currentstroke}{rgb}{0.000000,0.000000,0.000000}%
\pgfsetstrokecolor{currentstroke}%
\pgfsetdash{}{0pt}%
\pgfsys@defobject{currentmarker}{\pgfqpoint{0.000000in}{-0.048611in}}{\pgfqpoint{0.000000in}{0.000000in}}{%
\pgfpathmoveto{\pgfqpoint{0.000000in}{0.000000in}}%
\pgfpathlineto{\pgfqpoint{0.000000in}{-0.048611in}}%
\pgfusepath{stroke,fill}%
}%
\begin{pgfscope}%
\pgfsys@transformshift{5.789389in}{0.331635in}%
\pgfsys@useobject{currentmarker}{}%
\end{pgfscope}%
\end{pgfscope}%
\begin{pgfscope}%
\definecolor{textcolor}{rgb}{0.000000,0.000000,0.000000}%
\pgfsetstrokecolor{textcolor}%
\pgfsetfillcolor{textcolor}%
\pgftext[x=5.789389in,y=0.234413in,,top]{\color{textcolor}\sffamily\fontsize{10.000000}{12.000000}\selectfont 0}%
\end{pgfscope}%
\begin{pgfscope}%
\pgfsetbuttcap%
\pgfsetroundjoin%
\definecolor{currentfill}{rgb}{0.000000,0.000000,0.000000}%
\pgfsetfillcolor{currentfill}%
\pgfsetlinewidth{0.803000pt}%
\definecolor{currentstroke}{rgb}{0.000000,0.000000,0.000000}%
\pgfsetstrokecolor{currentstroke}%
\pgfsetdash{}{0pt}%
\pgfsys@defobject{currentmarker}{\pgfqpoint{0.000000in}{-0.048611in}}{\pgfqpoint{0.000000in}{0.000000in}}{%
\pgfpathmoveto{\pgfqpoint{0.000000in}{0.000000in}}%
\pgfpathlineto{\pgfqpoint{0.000000in}{-0.048611in}}%
\pgfusepath{stroke,fill}%
}%
\begin{pgfscope}%
\pgfsys@transformshift{6.991629in}{0.331635in}%
\pgfsys@useobject{currentmarker}{}%
\end{pgfscope}%
\end{pgfscope}%
\begin{pgfscope}%
\definecolor{textcolor}{rgb}{0.000000,0.000000,0.000000}%
\pgfsetstrokecolor{textcolor}%
\pgfsetfillcolor{textcolor}%
\pgftext[x=6.991629in,y=0.234413in,,top]{\color{textcolor}\sffamily\fontsize{10.000000}{12.000000}\selectfont 5}%
\end{pgfscope}%
\begin{pgfscope}%
\pgfsetbuttcap%
\pgfsetroundjoin%
\definecolor{currentfill}{rgb}{0.000000,0.000000,0.000000}%
\pgfsetfillcolor{currentfill}%
\pgfsetlinewidth{0.803000pt}%
\definecolor{currentstroke}{rgb}{0.000000,0.000000,0.000000}%
\pgfsetstrokecolor{currentstroke}%
\pgfsetdash{}{0pt}%
\pgfsys@defobject{currentmarker}{\pgfqpoint{0.000000in}{-0.048611in}}{\pgfqpoint{0.000000in}{0.000000in}}{%
\pgfpathmoveto{\pgfqpoint{0.000000in}{0.000000in}}%
\pgfpathlineto{\pgfqpoint{0.000000in}{-0.048611in}}%
\pgfusepath{stroke,fill}%
}%
\begin{pgfscope}%
\pgfsys@transformshift{8.193868in}{0.331635in}%
\pgfsys@useobject{currentmarker}{}%
\end{pgfscope}%
\end{pgfscope}%
\begin{pgfscope}%
\definecolor{textcolor}{rgb}{0.000000,0.000000,0.000000}%
\pgfsetstrokecolor{textcolor}%
\pgfsetfillcolor{textcolor}%
\pgftext[x=8.193868in,y=0.234413in,,top]{\color{textcolor}\sffamily\fontsize{10.000000}{12.000000}\selectfont 10}%
\end{pgfscope}%
\begin{pgfscope}%
\pgfsetbuttcap%
\pgfsetroundjoin%
\definecolor{currentfill}{rgb}{0.000000,0.000000,0.000000}%
\pgfsetfillcolor{currentfill}%
\pgfsetlinewidth{0.803000pt}%
\definecolor{currentstroke}{rgb}{0.000000,0.000000,0.000000}%
\pgfsetstrokecolor{currentstroke}%
\pgfsetdash{}{0pt}%
\pgfsys@defobject{currentmarker}{\pgfqpoint{0.000000in}{-0.048611in}}{\pgfqpoint{0.000000in}{0.000000in}}{%
\pgfpathmoveto{\pgfqpoint{0.000000in}{0.000000in}}%
\pgfpathlineto{\pgfqpoint{0.000000in}{-0.048611in}}%
\pgfusepath{stroke,fill}%
}%
\begin{pgfscope}%
\pgfsys@transformshift{9.396107in}{0.331635in}%
\pgfsys@useobject{currentmarker}{}%
\end{pgfscope}%
\end{pgfscope}%
\begin{pgfscope}%
\definecolor{textcolor}{rgb}{0.000000,0.000000,0.000000}%
\pgfsetstrokecolor{textcolor}%
\pgfsetfillcolor{textcolor}%
\pgftext[x=9.396107in,y=0.234413in,,top]{\color{textcolor}\sffamily\fontsize{10.000000}{12.000000}\selectfont 15}%
\end{pgfscope}%
\begin{pgfscope}%
\pgfsetbuttcap%
\pgfsetroundjoin%
\definecolor{currentfill}{rgb}{0.000000,0.000000,0.000000}%
\pgfsetfillcolor{currentfill}%
\pgfsetlinewidth{0.803000pt}%
\definecolor{currentstroke}{rgb}{0.000000,0.000000,0.000000}%
\pgfsetstrokecolor{currentstroke}%
\pgfsetdash{}{0pt}%
\pgfsys@defobject{currentmarker}{\pgfqpoint{-0.048611in}{0.000000in}}{\pgfqpoint{-0.000000in}{0.000000in}}{%
\pgfpathmoveto{\pgfqpoint{-0.000000in}{0.000000in}}%
\pgfpathlineto{\pgfqpoint{-0.048611in}{0.000000in}}%
\pgfusepath{stroke,fill}%
}%
\begin{pgfscope}%
\pgfsys@transformshift{0.481978in}{0.607744in}%
\pgfsys@useobject{currentmarker}{}%
\end{pgfscope}%
\end{pgfscope}%
\begin{pgfscope}%
\definecolor{textcolor}{rgb}{0.000000,0.000000,0.000000}%
\pgfsetstrokecolor{textcolor}%
\pgfsetfillcolor{textcolor}%
\pgftext[x=0.100000in, y=0.554982in, left, base]{\color{textcolor}\sffamily\fontsize{10.000000}{12.000000}\selectfont \ensuremath{-}15}%
\end{pgfscope}%
\begin{pgfscope}%
\pgfsetbuttcap%
\pgfsetroundjoin%
\definecolor{currentfill}{rgb}{0.000000,0.000000,0.000000}%
\pgfsetfillcolor{currentfill}%
\pgfsetlinewidth{0.803000pt}%
\definecolor{currentstroke}{rgb}{0.000000,0.000000,0.000000}%
\pgfsetstrokecolor{currentstroke}%
\pgfsetdash{}{0pt}%
\pgfsys@defobject{currentmarker}{\pgfqpoint{-0.048611in}{0.000000in}}{\pgfqpoint{-0.000000in}{0.000000in}}{%
\pgfpathmoveto{\pgfqpoint{-0.000000in}{0.000000in}}%
\pgfpathlineto{\pgfqpoint{-0.048611in}{0.000000in}}%
\pgfusepath{stroke,fill}%
}%
\begin{pgfscope}%
\pgfsys@transformshift{0.481978in}{1.846943in}%
\pgfsys@useobject{currentmarker}{}%
\end{pgfscope}%
\end{pgfscope}%
\begin{pgfscope}%
\definecolor{textcolor}{rgb}{0.000000,0.000000,0.000000}%
\pgfsetstrokecolor{textcolor}%
\pgfsetfillcolor{textcolor}%
\pgftext[x=0.100000in, y=1.794182in, left, base]{\color{textcolor}\sffamily\fontsize{10.000000}{12.000000}\selectfont \ensuremath{-}10}%
\end{pgfscope}%
\begin{pgfscope}%
\pgfsetbuttcap%
\pgfsetroundjoin%
\definecolor{currentfill}{rgb}{0.000000,0.000000,0.000000}%
\pgfsetfillcolor{currentfill}%
\pgfsetlinewidth{0.803000pt}%
\definecolor{currentstroke}{rgb}{0.000000,0.000000,0.000000}%
\pgfsetstrokecolor{currentstroke}%
\pgfsetdash{}{0pt}%
\pgfsys@defobject{currentmarker}{\pgfqpoint{-0.048611in}{0.000000in}}{\pgfqpoint{-0.000000in}{0.000000in}}{%
\pgfpathmoveto{\pgfqpoint{-0.000000in}{0.000000in}}%
\pgfpathlineto{\pgfqpoint{-0.048611in}{0.000000in}}%
\pgfusepath{stroke,fill}%
}%
\begin{pgfscope}%
\pgfsys@transformshift{0.481978in}{3.086143in}%
\pgfsys@useobject{currentmarker}{}%
\end{pgfscope}%
\end{pgfscope}%
\begin{pgfscope}%
\definecolor{textcolor}{rgb}{0.000000,0.000000,0.000000}%
\pgfsetstrokecolor{textcolor}%
\pgfsetfillcolor{textcolor}%
\pgftext[x=0.188365in, y=3.033381in, left, base]{\color{textcolor}\sffamily\fontsize{10.000000}{12.000000}\selectfont \ensuremath{-}5}%
\end{pgfscope}%
\begin{pgfscope}%
\pgfsetbuttcap%
\pgfsetroundjoin%
\definecolor{currentfill}{rgb}{0.000000,0.000000,0.000000}%
\pgfsetfillcolor{currentfill}%
\pgfsetlinewidth{0.803000pt}%
\definecolor{currentstroke}{rgb}{0.000000,0.000000,0.000000}%
\pgfsetstrokecolor{currentstroke}%
\pgfsetdash{}{0pt}%
\pgfsys@defobject{currentmarker}{\pgfqpoint{-0.048611in}{0.000000in}}{\pgfqpoint{-0.000000in}{0.000000in}}{%
\pgfpathmoveto{\pgfqpoint{-0.000000in}{0.000000in}}%
\pgfpathlineto{\pgfqpoint{-0.048611in}{0.000000in}}%
\pgfusepath{stroke,fill}%
}%
\begin{pgfscope}%
\pgfsys@transformshift{0.481978in}{4.325342in}%
\pgfsys@useobject{currentmarker}{}%
\end{pgfscope}%
\end{pgfscope}%
\begin{pgfscope}%
\definecolor{textcolor}{rgb}{0.000000,0.000000,0.000000}%
\pgfsetstrokecolor{textcolor}%
\pgfsetfillcolor{textcolor}%
\pgftext[x=0.296390in, y=4.272581in, left, base]{\color{textcolor}\sffamily\fontsize{10.000000}{12.000000}\selectfont 0}%
\end{pgfscope}%
\begin{pgfscope}%
\pgfsetbuttcap%
\pgfsetroundjoin%
\definecolor{currentfill}{rgb}{0.000000,0.000000,0.000000}%
\pgfsetfillcolor{currentfill}%
\pgfsetlinewidth{0.803000pt}%
\definecolor{currentstroke}{rgb}{0.000000,0.000000,0.000000}%
\pgfsetstrokecolor{currentstroke}%
\pgfsetdash{}{0pt}%
\pgfsys@defobject{currentmarker}{\pgfqpoint{-0.048611in}{0.000000in}}{\pgfqpoint{-0.000000in}{0.000000in}}{%
\pgfpathmoveto{\pgfqpoint{-0.000000in}{0.000000in}}%
\pgfpathlineto{\pgfqpoint{-0.048611in}{0.000000in}}%
\pgfusepath{stroke,fill}%
}%
\begin{pgfscope}%
\pgfsys@transformshift{0.481978in}{5.564542in}%
\pgfsys@useobject{currentmarker}{}%
\end{pgfscope}%
\end{pgfscope}%
\begin{pgfscope}%
\definecolor{textcolor}{rgb}{0.000000,0.000000,0.000000}%
\pgfsetstrokecolor{textcolor}%
\pgfsetfillcolor{textcolor}%
\pgftext[x=0.296390in, y=5.511780in, left, base]{\color{textcolor}\sffamily\fontsize{10.000000}{12.000000}\selectfont 5}%
\end{pgfscope}%
\begin{pgfscope}%
\pgfsetbuttcap%
\pgfsetroundjoin%
\definecolor{currentfill}{rgb}{0.000000,0.000000,0.000000}%
\pgfsetfillcolor{currentfill}%
\pgfsetlinewidth{0.803000pt}%
\definecolor{currentstroke}{rgb}{0.000000,0.000000,0.000000}%
\pgfsetstrokecolor{currentstroke}%
\pgfsetdash{}{0pt}%
\pgfsys@defobject{currentmarker}{\pgfqpoint{-0.048611in}{0.000000in}}{\pgfqpoint{-0.000000in}{0.000000in}}{%
\pgfpathmoveto{\pgfqpoint{-0.000000in}{0.000000in}}%
\pgfpathlineto{\pgfqpoint{-0.048611in}{0.000000in}}%
\pgfusepath{stroke,fill}%
}%
\begin{pgfscope}%
\pgfsys@transformshift{0.481978in}{6.803741in}%
\pgfsys@useobject{currentmarker}{}%
\end{pgfscope}%
\end{pgfscope}%
\begin{pgfscope}%
\definecolor{textcolor}{rgb}{0.000000,0.000000,0.000000}%
\pgfsetstrokecolor{textcolor}%
\pgfsetfillcolor{textcolor}%
\pgftext[x=0.208025in, y=6.750979in, left, base]{\color{textcolor}\sffamily\fontsize{10.000000}{12.000000}\selectfont 10}%
\end{pgfscope}%
\begin{pgfscope}%
\pgfpathrectangle{\pgfqpoint{0.481978in}{0.331635in}}{\pgfqpoint{9.300000in}{7.700000in}}%
\pgfusepath{clip}%
\pgfsetrectcap%
\pgfsetroundjoin%
\pgfsetlinewidth{1.505625pt}%
\definecolor{currentstroke}{rgb}{0.631373,0.788235,0.956863}%
\pgfsetstrokecolor{currentstroke}%
\pgfsetstrokeopacity{0.200000}%
\pgfsetdash{}{0pt}%
\pgfpathmoveto{\pgfqpoint{3.350250in}{0.837537in}}%
\pgfpathlineto{\pgfqpoint{3.792680in}{4.507363in}}%
\pgfusepath{stroke}%
\end{pgfscope}%
\begin{pgfscope}%
\pgfpathrectangle{\pgfqpoint{0.481978in}{0.331635in}}{\pgfqpoint{9.300000in}{7.700000in}}%
\pgfusepath{clip}%
\pgfsetrectcap%
\pgfsetroundjoin%
\pgfsetlinewidth{1.505625pt}%
\definecolor{currentstroke}{rgb}{0.631373,0.788235,0.956863}%
\pgfsetstrokecolor{currentstroke}%
\pgfsetstrokeopacity{0.200000}%
\pgfsetdash{}{0pt}%
\pgfpathmoveto{\pgfqpoint{3.059115in}{5.261945in}}%
\pgfpathlineto{\pgfqpoint{3.792680in}{4.507363in}}%
\pgfusepath{stroke}%
\end{pgfscope}%
\begin{pgfscope}%
\pgfpathrectangle{\pgfqpoint{0.481978in}{0.331635in}}{\pgfqpoint{9.300000in}{7.700000in}}%
\pgfusepath{clip}%
\pgfsetrectcap%
\pgfsetroundjoin%
\pgfsetlinewidth{1.505625pt}%
\definecolor{currentstroke}{rgb}{0.631373,0.788235,0.956863}%
\pgfsetstrokecolor{currentstroke}%
\pgfsetstrokeopacity{0.200000}%
\pgfsetdash{}{0pt}%
\pgfpathmoveto{\pgfqpoint{4.845549in}{3.770354in}}%
\pgfpathlineto{\pgfqpoint{3.792680in}{4.507363in}}%
\pgfusepath{stroke}%
\end{pgfscope}%
\begin{pgfscope}%
\pgfpathrectangle{\pgfqpoint{0.481978in}{0.331635in}}{\pgfqpoint{9.300000in}{7.700000in}}%
\pgfusepath{clip}%
\pgfsetrectcap%
\pgfsetroundjoin%
\pgfsetlinewidth{1.505625pt}%
\definecolor{currentstroke}{rgb}{0.631373,0.788235,0.956863}%
\pgfsetstrokecolor{currentstroke}%
\pgfsetstrokeopacity{0.200000}%
\pgfsetdash{}{0pt}%
\pgfpathmoveto{\pgfqpoint{1.304480in}{3.089887in}}%
\pgfpathlineto{\pgfqpoint{3.792680in}{4.507363in}}%
\pgfusepath{stroke}%
\end{pgfscope}%
\begin{pgfscope}%
\pgfpathrectangle{\pgfqpoint{0.481978in}{0.331635in}}{\pgfqpoint{9.300000in}{7.700000in}}%
\pgfusepath{clip}%
\pgfsetrectcap%
\pgfsetroundjoin%
\pgfsetlinewidth{1.505625pt}%
\definecolor{currentstroke}{rgb}{0.631373,0.788235,0.956863}%
\pgfsetstrokecolor{currentstroke}%
\pgfsetstrokeopacity{0.200000}%
\pgfsetdash{}{0pt}%
\pgfpathmoveto{\pgfqpoint{4.583472in}{2.434606in}}%
\pgfpathlineto{\pgfqpoint{3.792680in}{4.507363in}}%
\pgfusepath{stroke}%
\end{pgfscope}%
\begin{pgfscope}%
\pgfpathrectangle{\pgfqpoint{0.481978in}{0.331635in}}{\pgfqpoint{9.300000in}{7.700000in}}%
\pgfusepath{clip}%
\pgfsetrectcap%
\pgfsetroundjoin%
\pgfsetlinewidth{1.505625pt}%
\definecolor{currentstroke}{rgb}{0.631373,0.788235,0.956863}%
\pgfsetstrokecolor{currentstroke}%
\pgfsetstrokeopacity{0.200000}%
\pgfsetdash{}{0pt}%
\pgfpathmoveto{\pgfqpoint{5.509106in}{5.302557in}}%
\pgfpathlineto{\pgfqpoint{3.792680in}{4.507363in}}%
\pgfusepath{stroke}%
\end{pgfscope}%
\begin{pgfscope}%
\pgfpathrectangle{\pgfqpoint{0.481978in}{0.331635in}}{\pgfqpoint{9.300000in}{7.700000in}}%
\pgfusepath{clip}%
\pgfsetrectcap%
\pgfsetroundjoin%
\pgfsetlinewidth{1.505625pt}%
\definecolor{currentstroke}{rgb}{0.631373,0.788235,0.956863}%
\pgfsetstrokecolor{currentstroke}%
\pgfsetstrokeopacity{0.200000}%
\pgfsetdash{}{0pt}%
\pgfpathmoveto{\pgfqpoint{1.270538in}{3.063432in}}%
\pgfpathlineto{\pgfqpoint{3.792680in}{4.507363in}}%
\pgfusepath{stroke}%
\end{pgfscope}%
\begin{pgfscope}%
\pgfpathrectangle{\pgfqpoint{0.481978in}{0.331635in}}{\pgfqpoint{9.300000in}{7.700000in}}%
\pgfusepath{clip}%
\pgfsetrectcap%
\pgfsetroundjoin%
\pgfsetlinewidth{1.505625pt}%
\definecolor{currentstroke}{rgb}{0.631373,0.788235,0.956863}%
\pgfsetstrokecolor{currentstroke}%
\pgfsetstrokeopacity{0.200000}%
\pgfsetdash{}{0pt}%
\pgfpathmoveto{\pgfqpoint{4.233419in}{2.543996in}}%
\pgfpathlineto{\pgfqpoint{3.792680in}{4.507363in}}%
\pgfusepath{stroke}%
\end{pgfscope}%
\begin{pgfscope}%
\pgfpathrectangle{\pgfqpoint{0.481978in}{0.331635in}}{\pgfqpoint{9.300000in}{7.700000in}}%
\pgfusepath{clip}%
\pgfsetrectcap%
\pgfsetroundjoin%
\pgfsetlinewidth{1.505625pt}%
\definecolor{currentstroke}{rgb}{0.631373,0.788235,0.956863}%
\pgfsetstrokecolor{currentstroke}%
\pgfsetstrokeopacity{0.200000}%
\pgfsetdash{}{0pt}%
\pgfpathmoveto{\pgfqpoint{2.873447in}{4.588040in}}%
\pgfpathlineto{\pgfqpoint{3.792680in}{4.507363in}}%
\pgfusepath{stroke}%
\end{pgfscope}%
\begin{pgfscope}%
\pgfpathrectangle{\pgfqpoint{0.481978in}{0.331635in}}{\pgfqpoint{9.300000in}{7.700000in}}%
\pgfusepath{clip}%
\pgfsetrectcap%
\pgfsetroundjoin%
\pgfsetlinewidth{1.505625pt}%
\definecolor{currentstroke}{rgb}{0.631373,0.788235,0.956863}%
\pgfsetstrokecolor{currentstroke}%
\pgfsetstrokeopacity{0.200000}%
\pgfsetdash{}{0pt}%
\pgfpathmoveto{\pgfqpoint{6.038124in}{7.681635in}}%
\pgfpathlineto{\pgfqpoint{3.792680in}{4.507363in}}%
\pgfusepath{stroke}%
\end{pgfscope}%
\begin{pgfscope}%
\pgfpathrectangle{\pgfqpoint{0.481978in}{0.331635in}}{\pgfqpoint{9.300000in}{7.700000in}}%
\pgfusepath{clip}%
\pgfsetrectcap%
\pgfsetroundjoin%
\pgfsetlinewidth{1.505625pt}%
\definecolor{currentstroke}{rgb}{0.631373,0.788235,0.956863}%
\pgfsetstrokecolor{currentstroke}%
\pgfsetstrokeopacity{0.200000}%
\pgfsetdash{}{0pt}%
\pgfpathmoveto{\pgfqpoint{6.068535in}{7.049396in}}%
\pgfpathlineto{\pgfqpoint{3.792680in}{4.507363in}}%
\pgfusepath{stroke}%
\end{pgfscope}%
\begin{pgfscope}%
\pgfpathrectangle{\pgfqpoint{0.481978in}{0.331635in}}{\pgfqpoint{9.300000in}{7.700000in}}%
\pgfusepath{clip}%
\pgfsetrectcap%
\pgfsetroundjoin%
\pgfsetlinewidth{1.505625pt}%
\definecolor{currentstroke}{rgb}{0.631373,0.788235,0.956863}%
\pgfsetstrokecolor{currentstroke}%
\pgfsetstrokeopacity{0.200000}%
\pgfsetdash{}{0pt}%
\pgfpathmoveto{\pgfqpoint{0.904705in}{2.706623in}}%
\pgfpathlineto{\pgfqpoint{3.792680in}{4.507363in}}%
\pgfusepath{stroke}%
\end{pgfscope}%
\begin{pgfscope}%
\pgfpathrectangle{\pgfqpoint{0.481978in}{0.331635in}}{\pgfqpoint{9.300000in}{7.700000in}}%
\pgfusepath{clip}%
\pgfsetrectcap%
\pgfsetroundjoin%
\pgfsetlinewidth{1.505625pt}%
\definecolor{currentstroke}{rgb}{0.631373,0.788235,0.956863}%
\pgfsetstrokecolor{currentstroke}%
\pgfsetstrokeopacity{0.200000}%
\pgfsetdash{}{0pt}%
\pgfpathmoveto{\pgfqpoint{3.355697in}{0.924546in}}%
\pgfpathlineto{\pgfqpoint{3.792680in}{4.507363in}}%
\pgfusepath{stroke}%
\end{pgfscope}%
\begin{pgfscope}%
\pgfpathrectangle{\pgfqpoint{0.481978in}{0.331635in}}{\pgfqpoint{9.300000in}{7.700000in}}%
\pgfusepath{clip}%
\pgfsetrectcap%
\pgfsetroundjoin%
\pgfsetlinewidth{1.505625pt}%
\definecolor{currentstroke}{rgb}{0.631373,0.788235,0.956863}%
\pgfsetstrokecolor{currentstroke}%
\pgfsetstrokeopacity{0.200000}%
\pgfsetdash{}{0pt}%
\pgfpathmoveto{\pgfqpoint{1.822383in}{3.081016in}}%
\pgfpathlineto{\pgfqpoint{3.792680in}{4.507363in}}%
\pgfusepath{stroke}%
\end{pgfscope}%
\begin{pgfscope}%
\pgfpathrectangle{\pgfqpoint{0.481978in}{0.331635in}}{\pgfqpoint{9.300000in}{7.700000in}}%
\pgfusepath{clip}%
\pgfsetrectcap%
\pgfsetroundjoin%
\pgfsetlinewidth{1.505625pt}%
\definecolor{currentstroke}{rgb}{0.631373,0.788235,0.956863}%
\pgfsetstrokecolor{currentstroke}%
\pgfsetstrokeopacity{0.200000}%
\pgfsetdash{}{0pt}%
\pgfpathmoveto{\pgfqpoint{6.431702in}{7.311578in}}%
\pgfpathlineto{\pgfqpoint{3.792680in}{4.507363in}}%
\pgfusepath{stroke}%
\end{pgfscope}%
\begin{pgfscope}%
\pgfpathrectangle{\pgfqpoint{0.481978in}{0.331635in}}{\pgfqpoint{9.300000in}{7.700000in}}%
\pgfusepath{clip}%
\pgfsetrectcap%
\pgfsetroundjoin%
\pgfsetlinewidth{1.505625pt}%
\definecolor{currentstroke}{rgb}{0.631373,0.788235,0.956863}%
\pgfsetstrokecolor{currentstroke}%
\pgfsetstrokeopacity{0.200000}%
\pgfsetdash{}{0pt}%
\pgfpathmoveto{\pgfqpoint{1.601212in}{5.287720in}}%
\pgfpathlineto{\pgfqpoint{3.792680in}{4.507363in}}%
\pgfusepath{stroke}%
\end{pgfscope}%
\begin{pgfscope}%
\pgfpathrectangle{\pgfqpoint{0.481978in}{0.331635in}}{\pgfqpoint{9.300000in}{7.700000in}}%
\pgfusepath{clip}%
\pgfsetrectcap%
\pgfsetroundjoin%
\pgfsetlinewidth{1.505625pt}%
\definecolor{currentstroke}{rgb}{0.631373,0.788235,0.956863}%
\pgfsetstrokecolor{currentstroke}%
\pgfsetstrokeopacity{0.200000}%
\pgfsetdash{}{0pt}%
\pgfpathmoveto{\pgfqpoint{2.280847in}{6.346489in}}%
\pgfpathlineto{\pgfqpoint{3.792680in}{4.507363in}}%
\pgfusepath{stroke}%
\end{pgfscope}%
\begin{pgfscope}%
\pgfpathrectangle{\pgfqpoint{0.481978in}{0.331635in}}{\pgfqpoint{9.300000in}{7.700000in}}%
\pgfusepath{clip}%
\pgfsetrectcap%
\pgfsetroundjoin%
\pgfsetlinewidth{1.505625pt}%
\definecolor{currentstroke}{rgb}{0.631373,0.788235,0.956863}%
\pgfsetstrokecolor{currentstroke}%
\pgfsetstrokeopacity{0.200000}%
\pgfsetdash{}{0pt}%
\pgfpathmoveto{\pgfqpoint{1.914988in}{4.735660in}}%
\pgfpathlineto{\pgfqpoint{3.792680in}{4.507363in}}%
\pgfusepath{stroke}%
\end{pgfscope}%
\begin{pgfscope}%
\pgfpathrectangle{\pgfqpoint{0.481978in}{0.331635in}}{\pgfqpoint{9.300000in}{7.700000in}}%
\pgfusepath{clip}%
\pgfsetrectcap%
\pgfsetroundjoin%
\pgfsetlinewidth{1.505625pt}%
\definecolor{currentstroke}{rgb}{0.631373,0.788235,0.956863}%
\pgfsetstrokecolor{currentstroke}%
\pgfsetstrokeopacity{0.200000}%
\pgfsetdash{}{0pt}%
\pgfpathmoveto{\pgfqpoint{3.751753in}{1.453248in}}%
\pgfpathlineto{\pgfqpoint{3.792680in}{4.507363in}}%
\pgfusepath{stroke}%
\end{pgfscope}%
\begin{pgfscope}%
\pgfpathrectangle{\pgfqpoint{0.481978in}{0.331635in}}{\pgfqpoint{9.300000in}{7.700000in}}%
\pgfusepath{clip}%
\pgfsetrectcap%
\pgfsetroundjoin%
\pgfsetlinewidth{1.505625pt}%
\definecolor{currentstroke}{rgb}{0.631373,0.788235,0.956863}%
\pgfsetstrokecolor{currentstroke}%
\pgfsetstrokeopacity{0.200000}%
\pgfsetdash{}{0pt}%
\pgfpathmoveto{\pgfqpoint{5.753665in}{7.255589in}}%
\pgfpathlineto{\pgfqpoint{3.792680in}{4.507363in}}%
\pgfusepath{stroke}%
\end{pgfscope}%
\begin{pgfscope}%
\pgfpathrectangle{\pgfqpoint{0.481978in}{0.331635in}}{\pgfqpoint{9.300000in}{7.700000in}}%
\pgfusepath{clip}%
\pgfsetrectcap%
\pgfsetroundjoin%
\pgfsetlinewidth{1.505625pt}%
\definecolor{currentstroke}{rgb}{0.631373,0.788235,0.956863}%
\pgfsetstrokecolor{currentstroke}%
\pgfsetstrokeopacity{0.200000}%
\pgfsetdash{}{0pt}%
\pgfpathmoveto{\pgfqpoint{5.495955in}{5.233300in}}%
\pgfpathlineto{\pgfqpoint{3.792680in}{4.507363in}}%
\pgfusepath{stroke}%
\end{pgfscope}%
\begin{pgfscope}%
\pgfpathrectangle{\pgfqpoint{0.481978in}{0.331635in}}{\pgfqpoint{9.300000in}{7.700000in}}%
\pgfusepath{clip}%
\pgfsetrectcap%
\pgfsetroundjoin%
\pgfsetlinewidth{1.505625pt}%
\definecolor{currentstroke}{rgb}{0.631373,0.788235,0.956863}%
\pgfsetstrokecolor{currentstroke}%
\pgfsetstrokeopacity{0.200000}%
\pgfsetdash{}{0pt}%
\pgfpathmoveto{\pgfqpoint{4.238672in}{4.834740in}}%
\pgfpathlineto{\pgfqpoint{3.792680in}{4.507363in}}%
\pgfusepath{stroke}%
\end{pgfscope}%
\begin{pgfscope}%
\pgfpathrectangle{\pgfqpoint{0.481978in}{0.331635in}}{\pgfqpoint{9.300000in}{7.700000in}}%
\pgfusepath{clip}%
\pgfsetrectcap%
\pgfsetroundjoin%
\pgfsetlinewidth{1.505625pt}%
\definecolor{currentstroke}{rgb}{0.631373,0.788235,0.956863}%
\pgfsetstrokecolor{currentstroke}%
\pgfsetstrokeopacity{0.200000}%
\pgfsetdash{}{0pt}%
\pgfpathmoveto{\pgfqpoint{6.175837in}{7.515803in}}%
\pgfpathlineto{\pgfqpoint{3.792680in}{4.507363in}}%
\pgfusepath{stroke}%
\end{pgfscope}%
\begin{pgfscope}%
\pgfpathrectangle{\pgfqpoint{0.481978in}{0.331635in}}{\pgfqpoint{9.300000in}{7.700000in}}%
\pgfusepath{clip}%
\pgfsetrectcap%
\pgfsetroundjoin%
\pgfsetlinewidth{1.505625pt}%
\definecolor{currentstroke}{rgb}{0.631373,0.788235,0.956863}%
\pgfsetstrokecolor{currentstroke}%
\pgfsetstrokeopacity{0.200000}%
\pgfsetdash{}{0pt}%
\pgfpathmoveto{\pgfqpoint{4.277708in}{2.816465in}}%
\pgfpathlineto{\pgfqpoint{3.792680in}{4.507363in}}%
\pgfusepath{stroke}%
\end{pgfscope}%
\begin{pgfscope}%
\pgfpathrectangle{\pgfqpoint{0.481978in}{0.331635in}}{\pgfqpoint{9.300000in}{7.700000in}}%
\pgfusepath{clip}%
\pgfsetrectcap%
\pgfsetroundjoin%
\pgfsetlinewidth{1.505625pt}%
\definecolor{currentstroke}{rgb}{0.631373,0.788235,0.956863}%
\pgfsetstrokecolor{currentstroke}%
\pgfsetstrokeopacity{0.200000}%
\pgfsetdash{}{0pt}%
\pgfpathmoveto{\pgfqpoint{1.180315in}{4.870402in}}%
\pgfpathlineto{\pgfqpoint{3.792680in}{4.507363in}}%
\pgfusepath{stroke}%
\end{pgfscope}%
\begin{pgfscope}%
\pgfpathrectangle{\pgfqpoint{0.481978in}{0.331635in}}{\pgfqpoint{9.300000in}{7.700000in}}%
\pgfusepath{clip}%
\pgfsetrectcap%
\pgfsetroundjoin%
\pgfsetlinewidth{1.505625pt}%
\definecolor{currentstroke}{rgb}{0.631373,0.788235,0.956863}%
\pgfsetstrokecolor{currentstroke}%
\pgfsetstrokeopacity{0.200000}%
\pgfsetdash{}{0pt}%
\pgfpathmoveto{\pgfqpoint{4.575953in}{4.837612in}}%
\pgfpathlineto{\pgfqpoint{3.792680in}{4.507363in}}%
\pgfusepath{stroke}%
\end{pgfscope}%
\begin{pgfscope}%
\pgfpathrectangle{\pgfqpoint{0.481978in}{0.331635in}}{\pgfqpoint{9.300000in}{7.700000in}}%
\pgfusepath{clip}%
\pgfsetrectcap%
\pgfsetroundjoin%
\pgfsetlinewidth{1.505625pt}%
\definecolor{currentstroke}{rgb}{0.631373,0.788235,0.956863}%
\pgfsetstrokecolor{currentstroke}%
\pgfsetstrokeopacity{0.200000}%
\pgfsetdash{}{0pt}%
\pgfpathmoveto{\pgfqpoint{7.415126in}{4.572989in}}%
\pgfpathlineto{\pgfqpoint{3.792680in}{4.507363in}}%
\pgfusepath{stroke}%
\end{pgfscope}%
\begin{pgfscope}%
\pgfpathrectangle{\pgfqpoint{0.481978in}{0.331635in}}{\pgfqpoint{9.300000in}{7.700000in}}%
\pgfusepath{clip}%
\pgfsetrectcap%
\pgfsetroundjoin%
\pgfsetlinewidth{1.505625pt}%
\definecolor{currentstroke}{rgb}{0.631373,0.788235,0.956863}%
\pgfsetstrokecolor{currentstroke}%
\pgfsetstrokeopacity{0.200000}%
\pgfsetdash{}{0pt}%
\pgfpathmoveto{\pgfqpoint{5.743162in}{7.249576in}}%
\pgfpathlineto{\pgfqpoint{3.792680in}{4.507363in}}%
\pgfusepath{stroke}%
\end{pgfscope}%
\begin{pgfscope}%
\pgfpathrectangle{\pgfqpoint{0.481978in}{0.331635in}}{\pgfqpoint{9.300000in}{7.700000in}}%
\pgfusepath{clip}%
\pgfsetrectcap%
\pgfsetroundjoin%
\pgfsetlinewidth{1.505625pt}%
\definecolor{currentstroke}{rgb}{0.631373,0.788235,0.956863}%
\pgfsetstrokecolor{currentstroke}%
\pgfsetstrokeopacity{0.200000}%
\pgfsetdash{}{0pt}%
\pgfpathmoveto{\pgfqpoint{3.731632in}{6.363091in}}%
\pgfpathlineto{\pgfqpoint{3.792680in}{4.507363in}}%
\pgfusepath{stroke}%
\end{pgfscope}%
\begin{pgfscope}%
\pgfpathrectangle{\pgfqpoint{0.481978in}{0.331635in}}{\pgfqpoint{9.300000in}{7.700000in}}%
\pgfusepath{clip}%
\pgfsetrectcap%
\pgfsetroundjoin%
\pgfsetlinewidth{1.505625pt}%
\definecolor{currentstroke}{rgb}{0.631373,0.788235,0.956863}%
\pgfsetstrokecolor{currentstroke}%
\pgfsetstrokeopacity{0.200000}%
\pgfsetdash{}{0pt}%
\pgfpathmoveto{\pgfqpoint{3.354108in}{0.681635in}}%
\pgfpathlineto{\pgfqpoint{3.792680in}{4.507363in}}%
\pgfusepath{stroke}%
\end{pgfscope}%
\begin{pgfscope}%
\pgfpathrectangle{\pgfqpoint{0.481978in}{0.331635in}}{\pgfqpoint{9.300000in}{7.700000in}}%
\pgfusepath{clip}%
\pgfsetrectcap%
\pgfsetroundjoin%
\pgfsetlinewidth{1.505625pt}%
\definecolor{currentstroke}{rgb}{0.631373,0.788235,0.956863}%
\pgfsetstrokecolor{currentstroke}%
\pgfsetstrokeopacity{0.200000}%
\pgfsetdash{}{0pt}%
\pgfpathmoveto{\pgfqpoint{2.507727in}{5.244559in}}%
\pgfpathlineto{\pgfqpoint{3.792680in}{4.507363in}}%
\pgfusepath{stroke}%
\end{pgfscope}%
\begin{pgfscope}%
\pgfpathrectangle{\pgfqpoint{0.481978in}{0.331635in}}{\pgfqpoint{9.300000in}{7.700000in}}%
\pgfusepath{clip}%
\pgfsetrectcap%
\pgfsetroundjoin%
\pgfsetlinewidth{1.505625pt}%
\definecolor{currentstroke}{rgb}{0.631373,0.788235,0.956863}%
\pgfsetstrokecolor{currentstroke}%
\pgfsetstrokeopacity{0.200000}%
\pgfsetdash{}{0pt}%
\pgfpathmoveto{\pgfqpoint{3.345563in}{4.717554in}}%
\pgfpathlineto{\pgfqpoint{3.792680in}{4.507363in}}%
\pgfusepath{stroke}%
\end{pgfscope}%
\begin{pgfscope}%
\pgfpathrectangle{\pgfqpoint{0.481978in}{0.331635in}}{\pgfqpoint{9.300000in}{7.700000in}}%
\pgfusepath{clip}%
\pgfsetrectcap%
\pgfsetroundjoin%
\pgfsetlinewidth{1.505625pt}%
\definecolor{currentstroke}{rgb}{0.631373,0.788235,0.956863}%
\pgfsetstrokecolor{currentstroke}%
\pgfsetstrokeopacity{0.200000}%
\pgfsetdash{}{0pt}%
\pgfpathmoveto{\pgfqpoint{5.829195in}{5.254417in}}%
\pgfpathlineto{\pgfqpoint{3.792680in}{4.507363in}}%
\pgfusepath{stroke}%
\end{pgfscope}%
\begin{pgfscope}%
\pgfpathrectangle{\pgfqpoint{0.481978in}{0.331635in}}{\pgfqpoint{9.300000in}{7.700000in}}%
\pgfusepath{clip}%
\pgfsetrectcap%
\pgfsetroundjoin%
\pgfsetlinewidth{1.505625pt}%
\definecolor{currentstroke}{rgb}{0.631373,0.788235,0.956863}%
\pgfsetstrokecolor{currentstroke}%
\pgfsetstrokeopacity{0.200000}%
\pgfsetdash{}{0pt}%
\pgfpathmoveto{\pgfqpoint{1.266407in}{2.544584in}}%
\pgfpathlineto{\pgfqpoint{3.792680in}{4.507363in}}%
\pgfusepath{stroke}%
\end{pgfscope}%
\begin{pgfscope}%
\pgfpathrectangle{\pgfqpoint{0.481978in}{0.331635in}}{\pgfqpoint{9.300000in}{7.700000in}}%
\pgfusepath{clip}%
\pgfsetrectcap%
\pgfsetroundjoin%
\pgfsetlinewidth{1.505625pt}%
\definecolor{currentstroke}{rgb}{0.631373,0.788235,0.956863}%
\pgfsetstrokecolor{currentstroke}%
\pgfsetstrokeopacity{0.200000}%
\pgfsetdash{}{0pt}%
\pgfpathmoveto{\pgfqpoint{1.280257in}{3.743911in}}%
\pgfpathlineto{\pgfqpoint{3.792680in}{4.507363in}}%
\pgfusepath{stroke}%
\end{pgfscope}%
\begin{pgfscope}%
\pgfpathrectangle{\pgfqpoint{0.481978in}{0.331635in}}{\pgfqpoint{9.300000in}{7.700000in}}%
\pgfusepath{clip}%
\pgfsetrectcap%
\pgfsetroundjoin%
\pgfsetlinewidth{1.505625pt}%
\definecolor{currentstroke}{rgb}{0.631373,0.788235,0.956863}%
\pgfsetstrokecolor{currentstroke}%
\pgfsetstrokeopacity{0.200000}%
\pgfsetdash{}{0pt}%
\pgfpathmoveto{\pgfqpoint{1.366790in}{2.477899in}}%
\pgfpathlineto{\pgfqpoint{3.792680in}{4.507363in}}%
\pgfusepath{stroke}%
\end{pgfscope}%
\begin{pgfscope}%
\pgfpathrectangle{\pgfqpoint{0.481978in}{0.331635in}}{\pgfqpoint{9.300000in}{7.700000in}}%
\pgfusepath{clip}%
\pgfsetrectcap%
\pgfsetroundjoin%
\pgfsetlinewidth{1.505625pt}%
\definecolor{currentstroke}{rgb}{0.631373,0.788235,0.956863}%
\pgfsetstrokecolor{currentstroke}%
\pgfsetstrokeopacity{0.200000}%
\pgfsetdash{}{0pt}%
\pgfpathmoveto{\pgfqpoint{5.131403in}{3.817222in}}%
\pgfpathlineto{\pgfqpoint{3.792680in}{4.507363in}}%
\pgfusepath{stroke}%
\end{pgfscope}%
\begin{pgfscope}%
\pgfpathrectangle{\pgfqpoint{0.481978in}{0.331635in}}{\pgfqpoint{9.300000in}{7.700000in}}%
\pgfusepath{clip}%
\pgfsetrectcap%
\pgfsetroundjoin%
\pgfsetlinewidth{1.505625pt}%
\definecolor{currentstroke}{rgb}{0.631373,0.788235,0.956863}%
\pgfsetstrokecolor{currentstroke}%
\pgfsetstrokeopacity{0.200000}%
\pgfsetdash{}{0pt}%
\pgfpathmoveto{\pgfqpoint{6.077197in}{7.016351in}}%
\pgfpathlineto{\pgfqpoint{3.792680in}{4.507363in}}%
\pgfusepath{stroke}%
\end{pgfscope}%
\begin{pgfscope}%
\pgfpathrectangle{\pgfqpoint{0.481978in}{0.331635in}}{\pgfqpoint{9.300000in}{7.700000in}}%
\pgfusepath{clip}%
\pgfsetrectcap%
\pgfsetroundjoin%
\pgfsetlinewidth{1.505625pt}%
\definecolor{currentstroke}{rgb}{0.631373,0.788235,0.956863}%
\pgfsetstrokecolor{currentstroke}%
\pgfsetstrokeopacity{0.200000}%
\pgfsetdash{}{0pt}%
\pgfpathmoveto{\pgfqpoint{5.267507in}{3.479137in}}%
\pgfpathlineto{\pgfqpoint{3.792680in}{4.507363in}}%
\pgfusepath{stroke}%
\end{pgfscope}%
\begin{pgfscope}%
\pgfpathrectangle{\pgfqpoint{0.481978in}{0.331635in}}{\pgfqpoint{9.300000in}{7.700000in}}%
\pgfusepath{clip}%
\pgfsetrectcap%
\pgfsetroundjoin%
\pgfsetlinewidth{1.505625pt}%
\definecolor{currentstroke}{rgb}{0.631373,0.788235,0.956863}%
\pgfsetstrokecolor{currentstroke}%
\pgfsetstrokeopacity{0.200000}%
\pgfsetdash{}{0pt}%
\pgfpathmoveto{\pgfqpoint{5.422432in}{4.413168in}}%
\pgfpathlineto{\pgfqpoint{3.792680in}{4.507363in}}%
\pgfusepath{stroke}%
\end{pgfscope}%
\begin{pgfscope}%
\pgfpathrectangle{\pgfqpoint{0.481978in}{0.331635in}}{\pgfqpoint{9.300000in}{7.700000in}}%
\pgfusepath{clip}%
\pgfsetrectcap%
\pgfsetroundjoin%
\pgfsetlinewidth{1.505625pt}%
\definecolor{currentstroke}{rgb}{0.631373,0.788235,0.956863}%
\pgfsetstrokecolor{currentstroke}%
\pgfsetstrokeopacity{0.200000}%
\pgfsetdash{}{0pt}%
\pgfpathmoveto{\pgfqpoint{4.576144in}{3.894072in}}%
\pgfpathlineto{\pgfqpoint{3.792680in}{4.507363in}}%
\pgfusepath{stroke}%
\end{pgfscope}%
\begin{pgfscope}%
\pgfpathrectangle{\pgfqpoint{0.481978in}{0.331635in}}{\pgfqpoint{9.300000in}{7.700000in}}%
\pgfusepath{clip}%
\pgfsetrectcap%
\pgfsetroundjoin%
\pgfsetlinewidth{1.505625pt}%
\definecolor{currentstroke}{rgb}{0.631373,0.788235,0.956863}%
\pgfsetstrokecolor{currentstroke}%
\pgfsetstrokeopacity{0.200000}%
\pgfsetdash{}{0pt}%
\pgfpathmoveto{\pgfqpoint{2.109185in}{3.384062in}}%
\pgfpathlineto{\pgfqpoint{3.792680in}{4.507363in}}%
\pgfusepath{stroke}%
\end{pgfscope}%
\begin{pgfscope}%
\pgfpathrectangle{\pgfqpoint{0.481978in}{0.331635in}}{\pgfqpoint{9.300000in}{7.700000in}}%
\pgfusepath{clip}%
\pgfsetrectcap%
\pgfsetroundjoin%
\pgfsetlinewidth{1.505625pt}%
\definecolor{currentstroke}{rgb}{0.631373,0.788235,0.956863}%
\pgfsetstrokecolor{currentstroke}%
\pgfsetstrokeopacity{0.200000}%
\pgfsetdash{}{0pt}%
\pgfpathmoveto{\pgfqpoint{3.045397in}{4.082787in}}%
\pgfpathlineto{\pgfqpoint{3.792680in}{4.507363in}}%
\pgfusepath{stroke}%
\end{pgfscope}%
\begin{pgfscope}%
\pgfpathrectangle{\pgfqpoint{0.481978in}{0.331635in}}{\pgfqpoint{9.300000in}{7.700000in}}%
\pgfusepath{clip}%
\pgfsetrectcap%
\pgfsetroundjoin%
\pgfsetlinewidth{1.505625pt}%
\definecolor{currentstroke}{rgb}{0.631373,0.788235,0.956863}%
\pgfsetstrokecolor{currentstroke}%
\pgfsetstrokeopacity{0.200000}%
\pgfsetdash{}{0pt}%
\pgfpathmoveto{\pgfqpoint{2.937522in}{5.111573in}}%
\pgfpathlineto{\pgfqpoint{3.792680in}{4.507363in}}%
\pgfusepath{stroke}%
\end{pgfscope}%
\begin{pgfscope}%
\pgfpathrectangle{\pgfqpoint{0.481978in}{0.331635in}}{\pgfqpoint{9.300000in}{7.700000in}}%
\pgfusepath{clip}%
\pgfsetrectcap%
\pgfsetroundjoin%
\pgfsetlinewidth{1.505625pt}%
\definecolor{currentstroke}{rgb}{0.631373,0.788235,0.956863}%
\pgfsetstrokecolor{currentstroke}%
\pgfsetstrokeopacity{0.200000}%
\pgfsetdash{}{0pt}%
\pgfpathmoveto{\pgfqpoint{2.204208in}{6.251751in}}%
\pgfpathlineto{\pgfqpoint{3.792680in}{4.507363in}}%
\pgfusepath{stroke}%
\end{pgfscope}%
\begin{pgfscope}%
\pgfpathrectangle{\pgfqpoint{0.481978in}{0.331635in}}{\pgfqpoint{9.300000in}{7.700000in}}%
\pgfusepath{clip}%
\pgfsetrectcap%
\pgfsetroundjoin%
\pgfsetlinewidth{1.505625pt}%
\definecolor{currentstroke}{rgb}{0.631373,0.788235,0.956863}%
\pgfsetstrokecolor{currentstroke}%
\pgfsetstrokeopacity{0.200000}%
\pgfsetdash{}{0pt}%
\pgfpathmoveto{\pgfqpoint{3.780219in}{6.387783in}}%
\pgfpathlineto{\pgfqpoint{3.792680in}{4.507363in}}%
\pgfusepath{stroke}%
\end{pgfscope}%
\begin{pgfscope}%
\pgfpathrectangle{\pgfqpoint{0.481978in}{0.331635in}}{\pgfqpoint{9.300000in}{7.700000in}}%
\pgfusepath{clip}%
\pgfsetrectcap%
\pgfsetroundjoin%
\pgfsetlinewidth{1.505625pt}%
\definecolor{currentstroke}{rgb}{0.631373,0.788235,0.956863}%
\pgfsetstrokecolor{currentstroke}%
\pgfsetstrokeopacity{0.200000}%
\pgfsetdash{}{0pt}%
\pgfpathmoveto{\pgfqpoint{4.814839in}{2.794696in}}%
\pgfpathlineto{\pgfqpoint{3.792680in}{4.507363in}}%
\pgfusepath{stroke}%
\end{pgfscope}%
\begin{pgfscope}%
\pgfpathrectangle{\pgfqpoint{0.481978in}{0.331635in}}{\pgfqpoint{9.300000in}{7.700000in}}%
\pgfusepath{clip}%
\pgfsetrectcap%
\pgfsetroundjoin%
\pgfsetlinewidth{1.505625pt}%
\definecolor{currentstroke}{rgb}{0.631373,0.788235,0.956863}%
\pgfsetstrokecolor{currentstroke}%
\pgfsetstrokeopacity{0.200000}%
\pgfsetdash{}{0pt}%
\pgfpathmoveto{\pgfqpoint{2.753007in}{3.869593in}}%
\pgfpathlineto{\pgfqpoint{3.792680in}{4.507363in}}%
\pgfusepath{stroke}%
\end{pgfscope}%
\begin{pgfscope}%
\pgfpathrectangle{\pgfqpoint{0.481978in}{0.331635in}}{\pgfqpoint{9.300000in}{7.700000in}}%
\pgfusepath{clip}%
\pgfsetrectcap%
\pgfsetroundjoin%
\pgfsetlinewidth{1.505625pt}%
\definecolor{currentstroke}{rgb}{0.631373,0.788235,0.956863}%
\pgfsetstrokecolor{currentstroke}%
\pgfsetstrokeopacity{0.200000}%
\pgfsetdash{}{0pt}%
\pgfpathmoveto{\pgfqpoint{2.663463in}{5.701280in}}%
\pgfpathlineto{\pgfqpoint{3.792680in}{4.507363in}}%
\pgfusepath{stroke}%
\end{pgfscope}%
\begin{pgfscope}%
\pgfpathrectangle{\pgfqpoint{0.481978in}{0.331635in}}{\pgfqpoint{9.300000in}{7.700000in}}%
\pgfusepath{clip}%
\pgfsetrectcap%
\pgfsetroundjoin%
\pgfsetlinewidth{1.505625pt}%
\definecolor{currentstroke}{rgb}{0.631373,0.788235,0.956863}%
\pgfsetstrokecolor{currentstroke}%
\pgfsetstrokeopacity{0.200000}%
\pgfsetdash{}{0pt}%
\pgfpathmoveto{\pgfqpoint{6.114065in}{7.480299in}}%
\pgfpathlineto{\pgfqpoint{3.792680in}{4.507363in}}%
\pgfusepath{stroke}%
\end{pgfscope}%
\begin{pgfscope}%
\pgfpathrectangle{\pgfqpoint{0.481978in}{0.331635in}}{\pgfqpoint{9.300000in}{7.700000in}}%
\pgfusepath{clip}%
\pgfsetrectcap%
\pgfsetroundjoin%
\pgfsetlinewidth{1.505625pt}%
\definecolor{currentstroke}{rgb}{1.000000,0.705882,0.509804}%
\pgfsetstrokecolor{currentstroke}%
\pgfsetstrokeopacity{0.200000}%
\pgfsetdash{}{0pt}%
\pgfpathmoveto{\pgfqpoint{3.154468in}{2.488328in}}%
\pgfpathlineto{\pgfqpoint{5.493778in}{3.750666in}}%
\pgfusepath{stroke}%
\end{pgfscope}%
\begin{pgfscope}%
\pgfpathrectangle{\pgfqpoint{0.481978in}{0.331635in}}{\pgfqpoint{9.300000in}{7.700000in}}%
\pgfusepath{clip}%
\pgfsetrectcap%
\pgfsetroundjoin%
\pgfsetlinewidth{1.505625pt}%
\definecolor{currentstroke}{rgb}{1.000000,0.705882,0.509804}%
\pgfsetstrokecolor{currentstroke}%
\pgfsetstrokeopacity{0.200000}%
\pgfsetdash{}{0pt}%
\pgfpathmoveto{\pgfqpoint{1.545207in}{2.100004in}}%
\pgfpathlineto{\pgfqpoint{5.493778in}{3.750666in}}%
\pgfusepath{stroke}%
\end{pgfscope}%
\begin{pgfscope}%
\pgfpathrectangle{\pgfqpoint{0.481978in}{0.331635in}}{\pgfqpoint{9.300000in}{7.700000in}}%
\pgfusepath{clip}%
\pgfsetrectcap%
\pgfsetroundjoin%
\pgfsetlinewidth{1.505625pt}%
\definecolor{currentstroke}{rgb}{1.000000,0.705882,0.509804}%
\pgfsetstrokecolor{currentstroke}%
\pgfsetstrokeopacity{0.200000}%
\pgfsetdash{}{0pt}%
\pgfpathmoveto{\pgfqpoint{3.307338in}{3.407488in}}%
\pgfpathlineto{\pgfqpoint{5.493778in}{3.750666in}}%
\pgfusepath{stroke}%
\end{pgfscope}%
\begin{pgfscope}%
\pgfpathrectangle{\pgfqpoint{0.481978in}{0.331635in}}{\pgfqpoint{9.300000in}{7.700000in}}%
\pgfusepath{clip}%
\pgfsetrectcap%
\pgfsetroundjoin%
\pgfsetlinewidth{1.505625pt}%
\definecolor{currentstroke}{rgb}{1.000000,0.705882,0.509804}%
\pgfsetstrokecolor{currentstroke}%
\pgfsetstrokeopacity{0.200000}%
\pgfsetdash{}{0pt}%
\pgfpathmoveto{\pgfqpoint{5.143128in}{7.188196in}}%
\pgfpathlineto{\pgfqpoint{5.493778in}{3.750666in}}%
\pgfusepath{stroke}%
\end{pgfscope}%
\begin{pgfscope}%
\pgfpathrectangle{\pgfqpoint{0.481978in}{0.331635in}}{\pgfqpoint{9.300000in}{7.700000in}}%
\pgfusepath{clip}%
\pgfsetrectcap%
\pgfsetroundjoin%
\pgfsetlinewidth{1.505625pt}%
\definecolor{currentstroke}{rgb}{1.000000,0.705882,0.509804}%
\pgfsetstrokecolor{currentstroke}%
\pgfsetstrokeopacity{0.200000}%
\pgfsetdash{}{0pt}%
\pgfpathmoveto{\pgfqpoint{3.238886in}{1.944951in}}%
\pgfpathlineto{\pgfqpoint{5.493778in}{3.750666in}}%
\pgfusepath{stroke}%
\end{pgfscope}%
\begin{pgfscope}%
\pgfpathrectangle{\pgfqpoint{0.481978in}{0.331635in}}{\pgfqpoint{9.300000in}{7.700000in}}%
\pgfusepath{clip}%
\pgfsetrectcap%
\pgfsetroundjoin%
\pgfsetlinewidth{1.505625pt}%
\definecolor{currentstroke}{rgb}{1.000000,0.705882,0.509804}%
\pgfsetstrokecolor{currentstroke}%
\pgfsetstrokeopacity{0.200000}%
\pgfsetdash{}{0pt}%
\pgfpathmoveto{\pgfqpoint{6.096137in}{1.275008in}}%
\pgfpathlineto{\pgfqpoint{5.493778in}{3.750666in}}%
\pgfusepath{stroke}%
\end{pgfscope}%
\begin{pgfscope}%
\pgfpathrectangle{\pgfqpoint{0.481978in}{0.331635in}}{\pgfqpoint{9.300000in}{7.700000in}}%
\pgfusepath{clip}%
\pgfsetrectcap%
\pgfsetroundjoin%
\pgfsetlinewidth{1.505625pt}%
\definecolor{currentstroke}{rgb}{1.000000,0.705882,0.509804}%
\pgfsetstrokecolor{currentstroke}%
\pgfsetstrokeopacity{0.200000}%
\pgfsetdash{}{0pt}%
\pgfpathmoveto{\pgfqpoint{8.892538in}{6.041691in}}%
\pgfpathlineto{\pgfqpoint{5.493778in}{3.750666in}}%
\pgfusepath{stroke}%
\end{pgfscope}%
\begin{pgfscope}%
\pgfpathrectangle{\pgfqpoint{0.481978in}{0.331635in}}{\pgfqpoint{9.300000in}{7.700000in}}%
\pgfusepath{clip}%
\pgfsetrectcap%
\pgfsetroundjoin%
\pgfsetlinewidth{1.505625pt}%
\definecolor{currentstroke}{rgb}{1.000000,0.705882,0.509804}%
\pgfsetstrokecolor{currentstroke}%
\pgfsetstrokeopacity{0.200000}%
\pgfsetdash{}{0pt}%
\pgfpathmoveto{\pgfqpoint{8.792161in}{5.474544in}}%
\pgfpathlineto{\pgfqpoint{5.493778in}{3.750666in}}%
\pgfusepath{stroke}%
\end{pgfscope}%
\begin{pgfscope}%
\pgfpathrectangle{\pgfqpoint{0.481978in}{0.331635in}}{\pgfqpoint{9.300000in}{7.700000in}}%
\pgfusepath{clip}%
\pgfsetrectcap%
\pgfsetroundjoin%
\pgfsetlinewidth{1.505625pt}%
\definecolor{currentstroke}{rgb}{1.000000,0.705882,0.509804}%
\pgfsetstrokecolor{currentstroke}%
\pgfsetstrokeopacity{0.200000}%
\pgfsetdash{}{0pt}%
\pgfpathmoveto{\pgfqpoint{4.884502in}{4.809519in}}%
\pgfpathlineto{\pgfqpoint{5.493778in}{3.750666in}}%
\pgfusepath{stroke}%
\end{pgfscope}%
\begin{pgfscope}%
\pgfpathrectangle{\pgfqpoint{0.481978in}{0.331635in}}{\pgfqpoint{9.300000in}{7.700000in}}%
\pgfusepath{clip}%
\pgfsetrectcap%
\pgfsetroundjoin%
\pgfsetlinewidth{1.505625pt}%
\definecolor{currentstroke}{rgb}{1.000000,0.705882,0.509804}%
\pgfsetstrokecolor{currentstroke}%
\pgfsetstrokeopacity{0.200000}%
\pgfsetdash{}{0pt}%
\pgfpathmoveto{\pgfqpoint{8.839374in}{4.841088in}}%
\pgfpathlineto{\pgfqpoint{5.493778in}{3.750666in}}%
\pgfusepath{stroke}%
\end{pgfscope}%
\begin{pgfscope}%
\pgfpathrectangle{\pgfqpoint{0.481978in}{0.331635in}}{\pgfqpoint{9.300000in}{7.700000in}}%
\pgfusepath{clip}%
\pgfsetrectcap%
\pgfsetroundjoin%
\pgfsetlinewidth{1.505625pt}%
\definecolor{currentstroke}{rgb}{1.000000,0.705882,0.509804}%
\pgfsetstrokecolor{currentstroke}%
\pgfsetstrokeopacity{0.200000}%
\pgfsetdash{}{0pt}%
\pgfpathmoveto{\pgfqpoint{8.645625in}{6.005204in}}%
\pgfpathlineto{\pgfqpoint{5.493778in}{3.750666in}}%
\pgfusepath{stroke}%
\end{pgfscope}%
\begin{pgfscope}%
\pgfpathrectangle{\pgfqpoint{0.481978in}{0.331635in}}{\pgfqpoint{9.300000in}{7.700000in}}%
\pgfusepath{clip}%
\pgfsetrectcap%
\pgfsetroundjoin%
\pgfsetlinewidth{1.505625pt}%
\definecolor{currentstroke}{rgb}{1.000000,0.705882,0.509804}%
\pgfsetstrokecolor{currentstroke}%
\pgfsetstrokeopacity{0.200000}%
\pgfsetdash{}{0pt}%
\pgfpathmoveto{\pgfqpoint{1.541644in}{2.071167in}}%
\pgfpathlineto{\pgfqpoint{5.493778in}{3.750666in}}%
\pgfusepath{stroke}%
\end{pgfscope}%
\begin{pgfscope}%
\pgfpathrectangle{\pgfqpoint{0.481978in}{0.331635in}}{\pgfqpoint{9.300000in}{7.700000in}}%
\pgfusepath{clip}%
\pgfsetrectcap%
\pgfsetroundjoin%
\pgfsetlinewidth{1.505625pt}%
\definecolor{currentstroke}{rgb}{1.000000,0.705882,0.509804}%
\pgfsetstrokecolor{currentstroke}%
\pgfsetstrokeopacity{0.200000}%
\pgfsetdash{}{0pt}%
\pgfpathmoveto{\pgfqpoint{8.938916in}{5.410430in}}%
\pgfpathlineto{\pgfqpoint{5.493778in}{3.750666in}}%
\pgfusepath{stroke}%
\end{pgfscope}%
\begin{pgfscope}%
\pgfpathrectangle{\pgfqpoint{0.481978in}{0.331635in}}{\pgfqpoint{9.300000in}{7.700000in}}%
\pgfusepath{clip}%
\pgfsetrectcap%
\pgfsetroundjoin%
\pgfsetlinewidth{1.505625pt}%
\definecolor{currentstroke}{rgb}{1.000000,0.705882,0.509804}%
\pgfsetstrokecolor{currentstroke}%
\pgfsetstrokeopacity{0.200000}%
\pgfsetdash{}{0pt}%
\pgfpathmoveto{\pgfqpoint{3.682482in}{4.997977in}}%
\pgfpathlineto{\pgfqpoint{5.493778in}{3.750666in}}%
\pgfusepath{stroke}%
\end{pgfscope}%
\begin{pgfscope}%
\pgfpathrectangle{\pgfqpoint{0.481978in}{0.331635in}}{\pgfqpoint{9.300000in}{7.700000in}}%
\pgfusepath{clip}%
\pgfsetrectcap%
\pgfsetroundjoin%
\pgfsetlinewidth{1.505625pt}%
\definecolor{currentstroke}{rgb}{1.000000,0.705882,0.509804}%
\pgfsetstrokecolor{currentstroke}%
\pgfsetstrokeopacity{0.200000}%
\pgfsetdash{}{0pt}%
\pgfpathmoveto{\pgfqpoint{7.449113in}{2.488977in}}%
\pgfpathlineto{\pgfqpoint{5.493778in}{3.750666in}}%
\pgfusepath{stroke}%
\end{pgfscope}%
\begin{pgfscope}%
\pgfpathrectangle{\pgfqpoint{0.481978in}{0.331635in}}{\pgfqpoint{9.300000in}{7.700000in}}%
\pgfusepath{clip}%
\pgfsetrectcap%
\pgfsetroundjoin%
\pgfsetlinewidth{1.505625pt}%
\definecolor{currentstroke}{rgb}{1.000000,0.705882,0.509804}%
\pgfsetstrokecolor{currentstroke}%
\pgfsetstrokeopacity{0.200000}%
\pgfsetdash{}{0pt}%
\pgfpathmoveto{\pgfqpoint{8.325442in}{5.192998in}}%
\pgfpathlineto{\pgfqpoint{5.493778in}{3.750666in}}%
\pgfusepath{stroke}%
\end{pgfscope}%
\begin{pgfscope}%
\pgfpathrectangle{\pgfqpoint{0.481978in}{0.331635in}}{\pgfqpoint{9.300000in}{7.700000in}}%
\pgfusepath{clip}%
\pgfsetrectcap%
\pgfsetroundjoin%
\pgfsetlinewidth{1.505625pt}%
\definecolor{currentstroke}{rgb}{1.000000,0.705882,0.509804}%
\pgfsetstrokecolor{currentstroke}%
\pgfsetstrokeopacity{0.200000}%
\pgfsetdash{}{0pt}%
\pgfpathmoveto{\pgfqpoint{7.578449in}{4.392360in}}%
\pgfpathlineto{\pgfqpoint{5.493778in}{3.750666in}}%
\pgfusepath{stroke}%
\end{pgfscope}%
\begin{pgfscope}%
\pgfpathrectangle{\pgfqpoint{0.481978in}{0.331635in}}{\pgfqpoint{9.300000in}{7.700000in}}%
\pgfusepath{clip}%
\pgfsetrectcap%
\pgfsetroundjoin%
\pgfsetlinewidth{1.505625pt}%
\definecolor{currentstroke}{rgb}{1.000000,0.705882,0.509804}%
\pgfsetstrokecolor{currentstroke}%
\pgfsetstrokeopacity{0.200000}%
\pgfsetdash{}{0pt}%
\pgfpathmoveto{\pgfqpoint{6.440915in}{3.400690in}}%
\pgfpathlineto{\pgfqpoint{5.493778in}{3.750666in}}%
\pgfusepath{stroke}%
\end{pgfscope}%
\begin{pgfscope}%
\pgfpathrectangle{\pgfqpoint{0.481978in}{0.331635in}}{\pgfqpoint{9.300000in}{7.700000in}}%
\pgfusepath{clip}%
\pgfsetrectcap%
\pgfsetroundjoin%
\pgfsetlinewidth{1.505625pt}%
\definecolor{currentstroke}{rgb}{1.000000,0.705882,0.509804}%
\pgfsetstrokecolor{currentstroke}%
\pgfsetstrokeopacity{0.200000}%
\pgfsetdash{}{0pt}%
\pgfpathmoveto{\pgfqpoint{3.469279in}{3.869225in}}%
\pgfpathlineto{\pgfqpoint{5.493778in}{3.750666in}}%
\pgfusepath{stroke}%
\end{pgfscope}%
\begin{pgfscope}%
\pgfpathrectangle{\pgfqpoint{0.481978in}{0.331635in}}{\pgfqpoint{9.300000in}{7.700000in}}%
\pgfusepath{clip}%
\pgfsetrectcap%
\pgfsetroundjoin%
\pgfsetlinewidth{1.505625pt}%
\definecolor{currentstroke}{rgb}{1.000000,0.705882,0.509804}%
\pgfsetstrokecolor{currentstroke}%
\pgfsetstrokeopacity{0.200000}%
\pgfsetdash{}{0pt}%
\pgfpathmoveto{\pgfqpoint{6.295661in}{1.608501in}}%
\pgfpathlineto{\pgfqpoint{5.493778in}{3.750666in}}%
\pgfusepath{stroke}%
\end{pgfscope}%
\begin{pgfscope}%
\pgfpathrectangle{\pgfqpoint{0.481978in}{0.331635in}}{\pgfqpoint{9.300000in}{7.700000in}}%
\pgfusepath{clip}%
\pgfsetrectcap%
\pgfsetroundjoin%
\pgfsetlinewidth{1.505625pt}%
\definecolor{currentstroke}{rgb}{1.000000,0.705882,0.509804}%
\pgfsetstrokecolor{currentstroke}%
\pgfsetstrokeopacity{0.200000}%
\pgfsetdash{}{0pt}%
\pgfpathmoveto{\pgfqpoint{3.458025in}{3.980640in}}%
\pgfpathlineto{\pgfqpoint{5.493778in}{3.750666in}}%
\pgfusepath{stroke}%
\end{pgfscope}%
\begin{pgfscope}%
\pgfpathrectangle{\pgfqpoint{0.481978in}{0.331635in}}{\pgfqpoint{9.300000in}{7.700000in}}%
\pgfusepath{clip}%
\pgfsetrectcap%
\pgfsetroundjoin%
\pgfsetlinewidth{1.505625pt}%
\definecolor{currentstroke}{rgb}{1.000000,0.705882,0.509804}%
\pgfsetstrokecolor{currentstroke}%
\pgfsetstrokeopacity{0.200000}%
\pgfsetdash{}{0pt}%
\pgfpathmoveto{\pgfqpoint{4.270877in}{5.423784in}}%
\pgfpathlineto{\pgfqpoint{5.493778in}{3.750666in}}%
\pgfusepath{stroke}%
\end{pgfscope}%
\begin{pgfscope}%
\pgfpathrectangle{\pgfqpoint{0.481978in}{0.331635in}}{\pgfqpoint{9.300000in}{7.700000in}}%
\pgfusepath{clip}%
\pgfsetrectcap%
\pgfsetroundjoin%
\pgfsetlinewidth{1.505625pt}%
\definecolor{currentstroke}{rgb}{1.000000,0.705882,0.509804}%
\pgfsetstrokecolor{currentstroke}%
\pgfsetstrokeopacity{0.200000}%
\pgfsetdash{}{0pt}%
\pgfpathmoveto{\pgfqpoint{5.749725in}{1.958324in}}%
\pgfpathlineto{\pgfqpoint{5.493778in}{3.750666in}}%
\pgfusepath{stroke}%
\end{pgfscope}%
\begin{pgfscope}%
\pgfpathrectangle{\pgfqpoint{0.481978in}{0.331635in}}{\pgfqpoint{9.300000in}{7.700000in}}%
\pgfusepath{clip}%
\pgfsetrectcap%
\pgfsetroundjoin%
\pgfsetlinewidth{1.505625pt}%
\definecolor{currentstroke}{rgb}{1.000000,0.705882,0.509804}%
\pgfsetstrokecolor{currentstroke}%
\pgfsetstrokeopacity{0.200000}%
\pgfsetdash{}{0pt}%
\pgfpathmoveto{\pgfqpoint{2.458775in}{1.729372in}}%
\pgfpathlineto{\pgfqpoint{5.493778in}{3.750666in}}%
\pgfusepath{stroke}%
\end{pgfscope}%
\begin{pgfscope}%
\pgfpathrectangle{\pgfqpoint{0.481978in}{0.331635in}}{\pgfqpoint{9.300000in}{7.700000in}}%
\pgfusepath{clip}%
\pgfsetrectcap%
\pgfsetroundjoin%
\pgfsetlinewidth{1.505625pt}%
\definecolor{currentstroke}{rgb}{1.000000,0.705882,0.509804}%
\pgfsetstrokecolor{currentstroke}%
\pgfsetstrokeopacity{0.200000}%
\pgfsetdash{}{0pt}%
\pgfpathmoveto{\pgfqpoint{4.692250in}{5.305518in}}%
\pgfpathlineto{\pgfqpoint{5.493778in}{3.750666in}}%
\pgfusepath{stroke}%
\end{pgfscope}%
\begin{pgfscope}%
\pgfpathrectangle{\pgfqpoint{0.481978in}{0.331635in}}{\pgfqpoint{9.300000in}{7.700000in}}%
\pgfusepath{clip}%
\pgfsetrectcap%
\pgfsetroundjoin%
\pgfsetlinewidth{1.505625pt}%
\definecolor{currentstroke}{rgb}{1.000000,0.705882,0.509804}%
\pgfsetstrokecolor{currentstroke}%
\pgfsetstrokeopacity{0.200000}%
\pgfsetdash{}{0pt}%
\pgfpathmoveto{\pgfqpoint{7.916388in}{4.222966in}}%
\pgfpathlineto{\pgfqpoint{5.493778in}{3.750666in}}%
\pgfusepath{stroke}%
\end{pgfscope}%
\begin{pgfscope}%
\pgfpathrectangle{\pgfqpoint{0.481978in}{0.331635in}}{\pgfqpoint{9.300000in}{7.700000in}}%
\pgfusepath{clip}%
\pgfsetrectcap%
\pgfsetroundjoin%
\pgfsetlinewidth{1.505625pt}%
\definecolor{currentstroke}{rgb}{1.000000,0.705882,0.509804}%
\pgfsetstrokecolor{currentstroke}%
\pgfsetstrokeopacity{0.200000}%
\pgfsetdash{}{0pt}%
\pgfpathmoveto{\pgfqpoint{4.752028in}{3.460097in}}%
\pgfpathlineto{\pgfqpoint{5.493778in}{3.750666in}}%
\pgfusepath{stroke}%
\end{pgfscope}%
\begin{pgfscope}%
\pgfpathrectangle{\pgfqpoint{0.481978in}{0.331635in}}{\pgfqpoint{9.300000in}{7.700000in}}%
\pgfusepath{clip}%
\pgfsetrectcap%
\pgfsetroundjoin%
\pgfsetlinewidth{1.505625pt}%
\definecolor{currentstroke}{rgb}{1.000000,0.705882,0.509804}%
\pgfsetstrokecolor{currentstroke}%
\pgfsetstrokeopacity{0.200000}%
\pgfsetdash{}{0pt}%
\pgfpathmoveto{\pgfqpoint{8.514292in}{5.266180in}}%
\pgfpathlineto{\pgfqpoint{5.493778in}{3.750666in}}%
\pgfusepath{stroke}%
\end{pgfscope}%
\begin{pgfscope}%
\pgfpathrectangle{\pgfqpoint{0.481978in}{0.331635in}}{\pgfqpoint{9.300000in}{7.700000in}}%
\pgfusepath{clip}%
\pgfsetrectcap%
\pgfsetroundjoin%
\pgfsetlinewidth{1.505625pt}%
\definecolor{currentstroke}{rgb}{1.000000,0.705882,0.509804}%
\pgfsetstrokecolor{currentstroke}%
\pgfsetstrokeopacity{0.200000}%
\pgfsetdash{}{0pt}%
\pgfpathmoveto{\pgfqpoint{3.988115in}{1.945915in}}%
\pgfpathlineto{\pgfqpoint{5.493778in}{3.750666in}}%
\pgfusepath{stroke}%
\end{pgfscope}%
\begin{pgfscope}%
\pgfpathrectangle{\pgfqpoint{0.481978in}{0.331635in}}{\pgfqpoint{9.300000in}{7.700000in}}%
\pgfusepath{clip}%
\pgfsetrectcap%
\pgfsetroundjoin%
\pgfsetlinewidth{1.505625pt}%
\definecolor{currentstroke}{rgb}{1.000000,0.705882,0.509804}%
\pgfsetstrokecolor{currentstroke}%
\pgfsetstrokeopacity{0.200000}%
\pgfsetdash{}{0pt}%
\pgfpathmoveto{\pgfqpoint{3.637460in}{4.480745in}}%
\pgfpathlineto{\pgfqpoint{5.493778in}{3.750666in}}%
\pgfusepath{stroke}%
\end{pgfscope}%
\begin{pgfscope}%
\pgfpathrectangle{\pgfqpoint{0.481978in}{0.331635in}}{\pgfqpoint{9.300000in}{7.700000in}}%
\pgfusepath{clip}%
\pgfsetrectcap%
\pgfsetroundjoin%
\pgfsetlinewidth{1.505625pt}%
\definecolor{currentstroke}{rgb}{1.000000,0.705882,0.509804}%
\pgfsetstrokecolor{currentstroke}%
\pgfsetstrokeopacity{0.200000}%
\pgfsetdash{}{0pt}%
\pgfpathmoveto{\pgfqpoint{3.666577in}{4.495055in}}%
\pgfpathlineto{\pgfqpoint{5.493778in}{3.750666in}}%
\pgfusepath{stroke}%
\end{pgfscope}%
\begin{pgfscope}%
\pgfpathrectangle{\pgfqpoint{0.481978in}{0.331635in}}{\pgfqpoint{9.300000in}{7.700000in}}%
\pgfusepath{clip}%
\pgfsetrectcap%
\pgfsetroundjoin%
\pgfsetlinewidth{1.505625pt}%
\definecolor{currentstroke}{rgb}{1.000000,0.705882,0.509804}%
\pgfsetstrokecolor{currentstroke}%
\pgfsetstrokeopacity{0.200000}%
\pgfsetdash{}{0pt}%
\pgfpathmoveto{\pgfqpoint{3.344754in}{3.254443in}}%
\pgfpathlineto{\pgfqpoint{5.493778in}{3.750666in}}%
\pgfusepath{stroke}%
\end{pgfscope}%
\begin{pgfscope}%
\pgfpathrectangle{\pgfqpoint{0.481978in}{0.331635in}}{\pgfqpoint{9.300000in}{7.700000in}}%
\pgfusepath{clip}%
\pgfsetrectcap%
\pgfsetroundjoin%
\pgfsetlinewidth{1.505625pt}%
\definecolor{currentstroke}{rgb}{1.000000,0.705882,0.509804}%
\pgfsetstrokecolor{currentstroke}%
\pgfsetstrokeopacity{0.200000}%
\pgfsetdash{}{0pt}%
\pgfpathmoveto{\pgfqpoint{7.394228in}{2.358529in}}%
\pgfpathlineto{\pgfqpoint{5.493778in}{3.750666in}}%
\pgfusepath{stroke}%
\end{pgfscope}%
\begin{pgfscope}%
\pgfpathrectangle{\pgfqpoint{0.481978in}{0.331635in}}{\pgfqpoint{9.300000in}{7.700000in}}%
\pgfusepath{clip}%
\pgfsetrectcap%
\pgfsetroundjoin%
\pgfsetlinewidth{1.505625pt}%
\definecolor{currentstroke}{rgb}{1.000000,0.705882,0.509804}%
\pgfsetstrokecolor{currentstroke}%
\pgfsetstrokeopacity{0.200000}%
\pgfsetdash{}{0pt}%
\pgfpathmoveto{\pgfqpoint{2.694400in}{2.135292in}}%
\pgfpathlineto{\pgfqpoint{5.493778in}{3.750666in}}%
\pgfusepath{stroke}%
\end{pgfscope}%
\begin{pgfscope}%
\pgfpathrectangle{\pgfqpoint{0.481978in}{0.331635in}}{\pgfqpoint{9.300000in}{7.700000in}}%
\pgfusepath{clip}%
\pgfsetrectcap%
\pgfsetroundjoin%
\pgfsetlinewidth{1.505625pt}%
\definecolor{currentstroke}{rgb}{1.000000,0.705882,0.509804}%
\pgfsetstrokecolor{currentstroke}%
\pgfsetstrokeopacity{0.200000}%
\pgfsetdash{}{0pt}%
\pgfpathmoveto{\pgfqpoint{5.916550in}{1.543110in}}%
\pgfpathlineto{\pgfqpoint{5.493778in}{3.750666in}}%
\pgfusepath{stroke}%
\end{pgfscope}%
\begin{pgfscope}%
\pgfpathrectangle{\pgfqpoint{0.481978in}{0.331635in}}{\pgfqpoint{9.300000in}{7.700000in}}%
\pgfusepath{clip}%
\pgfsetrectcap%
\pgfsetroundjoin%
\pgfsetlinewidth{1.505625pt}%
\definecolor{currentstroke}{rgb}{1.000000,0.705882,0.509804}%
\pgfsetstrokecolor{currentstroke}%
\pgfsetstrokeopacity{0.200000}%
\pgfsetdash{}{0pt}%
\pgfpathmoveto{\pgfqpoint{6.083860in}{2.968841in}}%
\pgfpathlineto{\pgfqpoint{5.493778in}{3.750666in}}%
\pgfusepath{stroke}%
\end{pgfscope}%
\begin{pgfscope}%
\pgfpathrectangle{\pgfqpoint{0.481978in}{0.331635in}}{\pgfqpoint{9.300000in}{7.700000in}}%
\pgfusepath{clip}%
\pgfsetrectcap%
\pgfsetroundjoin%
\pgfsetlinewidth{1.505625pt}%
\definecolor{currentstroke}{rgb}{1.000000,0.705882,0.509804}%
\pgfsetstrokecolor{currentstroke}%
\pgfsetstrokeopacity{0.200000}%
\pgfsetdash{}{0pt}%
\pgfpathmoveto{\pgfqpoint{4.639314in}{3.292339in}}%
\pgfpathlineto{\pgfqpoint{5.493778in}{3.750666in}}%
\pgfusepath{stroke}%
\end{pgfscope}%
\begin{pgfscope}%
\pgfpathrectangle{\pgfqpoint{0.481978in}{0.331635in}}{\pgfqpoint{9.300000in}{7.700000in}}%
\pgfusepath{clip}%
\pgfsetrectcap%
\pgfsetroundjoin%
\pgfsetlinewidth{1.505625pt}%
\definecolor{currentstroke}{rgb}{1.000000,0.705882,0.509804}%
\pgfsetstrokecolor{currentstroke}%
\pgfsetstrokeopacity{0.200000}%
\pgfsetdash{}{0pt}%
\pgfpathmoveto{\pgfqpoint{8.886711in}{5.746544in}}%
\pgfpathlineto{\pgfqpoint{5.493778in}{3.750666in}}%
\pgfusepath{stroke}%
\end{pgfscope}%
\begin{pgfscope}%
\pgfpathrectangle{\pgfqpoint{0.481978in}{0.331635in}}{\pgfqpoint{9.300000in}{7.700000in}}%
\pgfusepath{clip}%
\pgfsetrectcap%
\pgfsetroundjoin%
\pgfsetlinewidth{1.505625pt}%
\definecolor{currentstroke}{rgb}{1.000000,0.705882,0.509804}%
\pgfsetstrokecolor{currentstroke}%
\pgfsetstrokeopacity{0.200000}%
\pgfsetdash{}{0pt}%
\pgfpathmoveto{\pgfqpoint{3.960687in}{2.310033in}}%
\pgfpathlineto{\pgfqpoint{5.493778in}{3.750666in}}%
\pgfusepath{stroke}%
\end{pgfscope}%
\begin{pgfscope}%
\pgfpathrectangle{\pgfqpoint{0.481978in}{0.331635in}}{\pgfqpoint{9.300000in}{7.700000in}}%
\pgfusepath{clip}%
\pgfsetrectcap%
\pgfsetroundjoin%
\pgfsetlinewidth{1.505625pt}%
\definecolor{currentstroke}{rgb}{1.000000,0.705882,0.509804}%
\pgfsetstrokecolor{currentstroke}%
\pgfsetstrokeopacity{0.200000}%
\pgfsetdash{}{0pt}%
\pgfpathmoveto{\pgfqpoint{3.907332in}{1.760958in}}%
\pgfpathlineto{\pgfqpoint{5.493778in}{3.750666in}}%
\pgfusepath{stroke}%
\end{pgfscope}%
\begin{pgfscope}%
\pgfpathrectangle{\pgfqpoint{0.481978in}{0.331635in}}{\pgfqpoint{9.300000in}{7.700000in}}%
\pgfusepath{clip}%
\pgfsetrectcap%
\pgfsetroundjoin%
\pgfsetlinewidth{1.505625pt}%
\definecolor{currentstroke}{rgb}{1.000000,0.705882,0.509804}%
\pgfsetstrokecolor{currentstroke}%
\pgfsetstrokeopacity{0.200000}%
\pgfsetdash{}{0pt}%
\pgfpathmoveto{\pgfqpoint{4.745564in}{5.377197in}}%
\pgfpathlineto{\pgfqpoint{5.493778in}{3.750666in}}%
\pgfusepath{stroke}%
\end{pgfscope}%
\begin{pgfscope}%
\pgfpathrectangle{\pgfqpoint{0.481978in}{0.331635in}}{\pgfqpoint{9.300000in}{7.700000in}}%
\pgfusepath{clip}%
\pgfsetrectcap%
\pgfsetroundjoin%
\pgfsetlinewidth{1.505625pt}%
\definecolor{currentstroke}{rgb}{1.000000,0.705882,0.509804}%
\pgfsetstrokecolor{currentstroke}%
\pgfsetstrokeopacity{0.200000}%
\pgfsetdash{}{0pt}%
\pgfpathmoveto{\pgfqpoint{4.104700in}{5.460602in}}%
\pgfpathlineto{\pgfqpoint{5.493778in}{3.750666in}}%
\pgfusepath{stroke}%
\end{pgfscope}%
\begin{pgfscope}%
\pgfpathrectangle{\pgfqpoint{0.481978in}{0.331635in}}{\pgfqpoint{9.300000in}{7.700000in}}%
\pgfusepath{clip}%
\pgfsetrectcap%
\pgfsetroundjoin%
\pgfsetlinewidth{1.505625pt}%
\definecolor{currentstroke}{rgb}{1.000000,0.705882,0.509804}%
\pgfsetstrokecolor{currentstroke}%
\pgfsetstrokeopacity{0.200000}%
\pgfsetdash{}{0pt}%
\pgfpathmoveto{\pgfqpoint{8.529282in}{4.574529in}}%
\pgfpathlineto{\pgfqpoint{5.493778in}{3.750666in}}%
\pgfusepath{stroke}%
\end{pgfscope}%
\begin{pgfscope}%
\pgfpathrectangle{\pgfqpoint{0.481978in}{0.331635in}}{\pgfqpoint{9.300000in}{7.700000in}}%
\pgfusepath{clip}%
\pgfsetrectcap%
\pgfsetroundjoin%
\pgfsetlinewidth{1.505625pt}%
\definecolor{currentstroke}{rgb}{1.000000,0.705882,0.509804}%
\pgfsetstrokecolor{currentstroke}%
\pgfsetstrokeopacity{0.200000}%
\pgfsetdash{}{0pt}%
\pgfpathmoveto{\pgfqpoint{7.907970in}{4.014679in}}%
\pgfpathlineto{\pgfqpoint{5.493778in}{3.750666in}}%
\pgfusepath{stroke}%
\end{pgfscope}%
\begin{pgfscope}%
\pgfpathrectangle{\pgfqpoint{0.481978in}{0.331635in}}{\pgfqpoint{9.300000in}{7.700000in}}%
\pgfusepath{clip}%
\pgfsetrectcap%
\pgfsetroundjoin%
\pgfsetlinewidth{1.505625pt}%
\definecolor{currentstroke}{rgb}{1.000000,0.705882,0.509804}%
\pgfsetstrokecolor{currentstroke}%
\pgfsetstrokeopacity{0.200000}%
\pgfsetdash{}{0pt}%
\pgfpathmoveto{\pgfqpoint{8.433076in}{4.560190in}}%
\pgfpathlineto{\pgfqpoint{5.493778in}{3.750666in}}%
\pgfusepath{stroke}%
\end{pgfscope}%
\begin{pgfscope}%
\pgfpathrectangle{\pgfqpoint{0.481978in}{0.331635in}}{\pgfqpoint{9.300000in}{7.700000in}}%
\pgfusepath{clip}%
\pgfsetrectcap%
\pgfsetroundjoin%
\pgfsetlinewidth{1.505625pt}%
\definecolor{currentstroke}{rgb}{1.000000,0.705882,0.509804}%
\pgfsetstrokecolor{currentstroke}%
\pgfsetstrokeopacity{0.200000}%
\pgfsetdash{}{0pt}%
\pgfpathmoveto{\pgfqpoint{3.128738in}{2.095357in}}%
\pgfpathlineto{\pgfqpoint{5.493778in}{3.750666in}}%
\pgfusepath{stroke}%
\end{pgfscope}%
\begin{pgfscope}%
\pgfpathrectangle{\pgfqpoint{0.481978in}{0.331635in}}{\pgfqpoint{9.300000in}{7.700000in}}%
\pgfusepath{clip}%
\pgfsetrectcap%
\pgfsetroundjoin%
\pgfsetlinewidth{1.505625pt}%
\definecolor{currentstroke}{rgb}{1.000000,0.705882,0.509804}%
\pgfsetstrokecolor{currentstroke}%
\pgfsetstrokeopacity{0.200000}%
\pgfsetdash{}{0pt}%
\pgfpathmoveto{\pgfqpoint{6.750608in}{3.072454in}}%
\pgfpathlineto{\pgfqpoint{5.493778in}{3.750666in}}%
\pgfusepath{stroke}%
\end{pgfscope}%
\begin{pgfscope}%
\pgfpathrectangle{\pgfqpoint{0.481978in}{0.331635in}}{\pgfqpoint{9.300000in}{7.700000in}}%
\pgfusepath{clip}%
\pgfsetrectcap%
\pgfsetroundjoin%
\pgfsetlinewidth{1.505625pt}%
\definecolor{currentstroke}{rgb}{1.000000,0.705882,0.509804}%
\pgfsetstrokecolor{currentstroke}%
\pgfsetstrokeopacity{0.200000}%
\pgfsetdash{}{0pt}%
\pgfpathmoveto{\pgfqpoint{3.434837in}{4.071894in}}%
\pgfpathlineto{\pgfqpoint{5.493778in}{3.750666in}}%
\pgfusepath{stroke}%
\end{pgfscope}%
\begin{pgfscope}%
\pgfpathrectangle{\pgfqpoint{0.481978in}{0.331635in}}{\pgfqpoint{9.300000in}{7.700000in}}%
\pgfusepath{clip}%
\pgfsetrectcap%
\pgfsetroundjoin%
\pgfsetlinewidth{1.505625pt}%
\definecolor{currentstroke}{rgb}{1.000000,0.705882,0.509804}%
\pgfsetstrokecolor{currentstroke}%
\pgfsetstrokeopacity{0.200000}%
\pgfsetdash{}{0pt}%
\pgfpathmoveto{\pgfqpoint{2.745959in}{2.321911in}}%
\pgfpathlineto{\pgfqpoint{5.493778in}{3.750666in}}%
\pgfusepath{stroke}%
\end{pgfscope}%
\begin{pgfscope}%
\pgfpathrectangle{\pgfqpoint{0.481978in}{0.331635in}}{\pgfqpoint{9.300000in}{7.700000in}}%
\pgfusepath{clip}%
\pgfsetrectcap%
\pgfsetroundjoin%
\pgfsetlinewidth{1.505625pt}%
\definecolor{currentstroke}{rgb}{1.000000,0.705882,0.509804}%
\pgfsetstrokecolor{currentstroke}%
\pgfsetstrokeopacity{0.200000}%
\pgfsetdash{}{0pt}%
\pgfpathmoveto{\pgfqpoint{8.714562in}{5.337458in}}%
\pgfpathlineto{\pgfqpoint{5.493778in}{3.750666in}}%
\pgfusepath{stroke}%
\end{pgfscope}%
\begin{pgfscope}%
\pgfpathrectangle{\pgfqpoint{0.481978in}{0.331635in}}{\pgfqpoint{9.300000in}{7.700000in}}%
\pgfusepath{clip}%
\pgfsetrectcap%
\pgfsetroundjoin%
\pgfsetlinewidth{1.505625pt}%
\definecolor{currentstroke}{rgb}{0.552941,0.898039,0.631373}%
\pgfsetstrokecolor{currentstroke}%
\pgfsetstrokeopacity{0.200000}%
\pgfsetdash{}{0pt}%
\pgfpathmoveto{\pgfqpoint{5.163299in}{7.343168in}}%
\pgfpathlineto{\pgfqpoint{5.849815in}{4.522824in}}%
\pgfusepath{stroke}%
\end{pgfscope}%
\begin{pgfscope}%
\pgfpathrectangle{\pgfqpoint{0.481978in}{0.331635in}}{\pgfqpoint{9.300000in}{7.700000in}}%
\pgfusepath{clip}%
\pgfsetrectcap%
\pgfsetroundjoin%
\pgfsetlinewidth{1.505625pt}%
\definecolor{currentstroke}{rgb}{0.552941,0.898039,0.631373}%
\pgfsetstrokecolor{currentstroke}%
\pgfsetstrokeopacity{0.200000}%
\pgfsetdash{}{0pt}%
\pgfpathmoveto{\pgfqpoint{3.761360in}{5.764646in}}%
\pgfpathlineto{\pgfqpoint{5.849815in}{4.522824in}}%
\pgfusepath{stroke}%
\end{pgfscope}%
\begin{pgfscope}%
\pgfpathrectangle{\pgfqpoint{0.481978in}{0.331635in}}{\pgfqpoint{9.300000in}{7.700000in}}%
\pgfusepath{clip}%
\pgfsetrectcap%
\pgfsetroundjoin%
\pgfsetlinewidth{1.505625pt}%
\definecolor{currentstroke}{rgb}{0.552941,0.898039,0.631373}%
\pgfsetstrokecolor{currentstroke}%
\pgfsetstrokeopacity{0.200000}%
\pgfsetdash{}{0pt}%
\pgfpathmoveto{\pgfqpoint{3.580350in}{3.088303in}}%
\pgfpathlineto{\pgfqpoint{5.849815in}{4.522824in}}%
\pgfusepath{stroke}%
\end{pgfscope}%
\begin{pgfscope}%
\pgfpathrectangle{\pgfqpoint{0.481978in}{0.331635in}}{\pgfqpoint{9.300000in}{7.700000in}}%
\pgfusepath{clip}%
\pgfsetrectcap%
\pgfsetroundjoin%
\pgfsetlinewidth{1.505625pt}%
\definecolor{currentstroke}{rgb}{0.552941,0.898039,0.631373}%
\pgfsetstrokecolor{currentstroke}%
\pgfsetstrokeopacity{0.200000}%
\pgfsetdash{}{0pt}%
\pgfpathmoveto{\pgfqpoint{8.242673in}{4.221424in}}%
\pgfpathlineto{\pgfqpoint{5.849815in}{4.522824in}}%
\pgfusepath{stroke}%
\end{pgfscope}%
\begin{pgfscope}%
\pgfpathrectangle{\pgfqpoint{0.481978in}{0.331635in}}{\pgfqpoint{9.300000in}{7.700000in}}%
\pgfusepath{clip}%
\pgfsetrectcap%
\pgfsetroundjoin%
\pgfsetlinewidth{1.505625pt}%
\definecolor{currentstroke}{rgb}{0.552941,0.898039,0.631373}%
\pgfsetstrokecolor{currentstroke}%
\pgfsetstrokeopacity{0.200000}%
\pgfsetdash{}{0pt}%
\pgfpathmoveto{\pgfqpoint{4.433234in}{5.070981in}}%
\pgfpathlineto{\pgfqpoint{5.849815in}{4.522824in}}%
\pgfusepath{stroke}%
\end{pgfscope}%
\begin{pgfscope}%
\pgfpathrectangle{\pgfqpoint{0.481978in}{0.331635in}}{\pgfqpoint{9.300000in}{7.700000in}}%
\pgfusepath{clip}%
\pgfsetrectcap%
\pgfsetroundjoin%
\pgfsetlinewidth{1.505625pt}%
\definecolor{currentstroke}{rgb}{0.552941,0.898039,0.631373}%
\pgfsetstrokecolor{currentstroke}%
\pgfsetstrokeopacity{0.200000}%
\pgfsetdash{}{0pt}%
\pgfpathmoveto{\pgfqpoint{9.194655in}{5.663935in}}%
\pgfpathlineto{\pgfqpoint{5.849815in}{4.522824in}}%
\pgfusepath{stroke}%
\end{pgfscope}%
\begin{pgfscope}%
\pgfpathrectangle{\pgfqpoint{0.481978in}{0.331635in}}{\pgfqpoint{9.300000in}{7.700000in}}%
\pgfusepath{clip}%
\pgfsetrectcap%
\pgfsetroundjoin%
\pgfsetlinewidth{1.505625pt}%
\definecolor{currentstroke}{rgb}{0.552941,0.898039,0.631373}%
\pgfsetstrokecolor{currentstroke}%
\pgfsetstrokeopacity{0.200000}%
\pgfsetdash{}{0pt}%
\pgfpathmoveto{\pgfqpoint{9.032589in}{5.080119in}}%
\pgfpathlineto{\pgfqpoint{5.849815in}{4.522824in}}%
\pgfusepath{stroke}%
\end{pgfscope}%
\begin{pgfscope}%
\pgfpathrectangle{\pgfqpoint{0.481978in}{0.331635in}}{\pgfqpoint{9.300000in}{7.700000in}}%
\pgfusepath{clip}%
\pgfsetrectcap%
\pgfsetroundjoin%
\pgfsetlinewidth{1.505625pt}%
\definecolor{currentstroke}{rgb}{0.552941,0.898039,0.631373}%
\pgfsetstrokecolor{currentstroke}%
\pgfsetstrokeopacity{0.200000}%
\pgfsetdash{}{0pt}%
\pgfpathmoveto{\pgfqpoint{3.942708in}{5.673543in}}%
\pgfpathlineto{\pgfqpoint{5.849815in}{4.522824in}}%
\pgfusepath{stroke}%
\end{pgfscope}%
\begin{pgfscope}%
\pgfpathrectangle{\pgfqpoint{0.481978in}{0.331635in}}{\pgfqpoint{9.300000in}{7.700000in}}%
\pgfusepath{clip}%
\pgfsetrectcap%
\pgfsetroundjoin%
\pgfsetlinewidth{1.505625pt}%
\definecolor{currentstroke}{rgb}{0.552941,0.898039,0.631373}%
\pgfsetstrokecolor{currentstroke}%
\pgfsetstrokeopacity{0.200000}%
\pgfsetdash{}{0pt}%
\pgfpathmoveto{\pgfqpoint{5.874679in}{2.722779in}}%
\pgfpathlineto{\pgfqpoint{5.849815in}{4.522824in}}%
\pgfusepath{stroke}%
\end{pgfscope}%
\begin{pgfscope}%
\pgfpathrectangle{\pgfqpoint{0.481978in}{0.331635in}}{\pgfqpoint{9.300000in}{7.700000in}}%
\pgfusepath{clip}%
\pgfsetrectcap%
\pgfsetroundjoin%
\pgfsetlinewidth{1.505625pt}%
\definecolor{currentstroke}{rgb}{0.552941,0.898039,0.631373}%
\pgfsetstrokecolor{currentstroke}%
\pgfsetstrokeopacity{0.200000}%
\pgfsetdash{}{0pt}%
\pgfpathmoveto{\pgfqpoint{6.449863in}{3.180896in}}%
\pgfpathlineto{\pgfqpoint{5.849815in}{4.522824in}}%
\pgfusepath{stroke}%
\end{pgfscope}%
\begin{pgfscope}%
\pgfpathrectangle{\pgfqpoint{0.481978in}{0.331635in}}{\pgfqpoint{9.300000in}{7.700000in}}%
\pgfusepath{clip}%
\pgfsetrectcap%
\pgfsetroundjoin%
\pgfsetlinewidth{1.505625pt}%
\definecolor{currentstroke}{rgb}{0.552941,0.898039,0.631373}%
\pgfsetstrokecolor{currentstroke}%
\pgfsetstrokeopacity{0.200000}%
\pgfsetdash{}{0pt}%
\pgfpathmoveto{\pgfqpoint{2.941597in}{6.366988in}}%
\pgfpathlineto{\pgfqpoint{5.849815in}{4.522824in}}%
\pgfusepath{stroke}%
\end{pgfscope}%
\begin{pgfscope}%
\pgfpathrectangle{\pgfqpoint{0.481978in}{0.331635in}}{\pgfqpoint{9.300000in}{7.700000in}}%
\pgfusepath{clip}%
\pgfsetrectcap%
\pgfsetroundjoin%
\pgfsetlinewidth{1.505625pt}%
\definecolor{currentstroke}{rgb}{0.552941,0.898039,0.631373}%
\pgfsetstrokecolor{currentstroke}%
\pgfsetstrokeopacity{0.200000}%
\pgfsetdash{}{0pt}%
\pgfpathmoveto{\pgfqpoint{4.671581in}{6.114771in}}%
\pgfpathlineto{\pgfqpoint{5.849815in}{4.522824in}}%
\pgfusepath{stroke}%
\end{pgfscope}%
\begin{pgfscope}%
\pgfpathrectangle{\pgfqpoint{0.481978in}{0.331635in}}{\pgfqpoint{9.300000in}{7.700000in}}%
\pgfusepath{clip}%
\pgfsetrectcap%
\pgfsetroundjoin%
\pgfsetlinewidth{1.505625pt}%
\definecolor{currentstroke}{rgb}{0.552941,0.898039,0.631373}%
\pgfsetstrokecolor{currentstroke}%
\pgfsetstrokeopacity{0.200000}%
\pgfsetdash{}{0pt}%
\pgfpathmoveto{\pgfqpoint{3.259508in}{2.958122in}}%
\pgfpathlineto{\pgfqpoint{5.849815in}{4.522824in}}%
\pgfusepath{stroke}%
\end{pgfscope}%
\begin{pgfscope}%
\pgfpathrectangle{\pgfqpoint{0.481978in}{0.331635in}}{\pgfqpoint{9.300000in}{7.700000in}}%
\pgfusepath{clip}%
\pgfsetrectcap%
\pgfsetroundjoin%
\pgfsetlinewidth{1.505625pt}%
\definecolor{currentstroke}{rgb}{0.552941,0.898039,0.631373}%
\pgfsetstrokecolor{currentstroke}%
\pgfsetstrokeopacity{0.200000}%
\pgfsetdash{}{0pt}%
\pgfpathmoveto{\pgfqpoint{7.805062in}{3.444564in}}%
\pgfpathlineto{\pgfqpoint{5.849815in}{4.522824in}}%
\pgfusepath{stroke}%
\end{pgfscope}%
\begin{pgfscope}%
\pgfpathrectangle{\pgfqpoint{0.481978in}{0.331635in}}{\pgfqpoint{9.300000in}{7.700000in}}%
\pgfusepath{clip}%
\pgfsetrectcap%
\pgfsetroundjoin%
\pgfsetlinewidth{1.505625pt}%
\definecolor{currentstroke}{rgb}{0.552941,0.898039,0.631373}%
\pgfsetstrokecolor{currentstroke}%
\pgfsetstrokeopacity{0.200000}%
\pgfsetdash{}{0pt}%
\pgfpathmoveto{\pgfqpoint{4.750388in}{7.194691in}}%
\pgfpathlineto{\pgfqpoint{5.849815in}{4.522824in}}%
\pgfusepath{stroke}%
\end{pgfscope}%
\begin{pgfscope}%
\pgfpathrectangle{\pgfqpoint{0.481978in}{0.331635in}}{\pgfqpoint{9.300000in}{7.700000in}}%
\pgfusepath{clip}%
\pgfsetrectcap%
\pgfsetroundjoin%
\pgfsetlinewidth{1.505625pt}%
\definecolor{currentstroke}{rgb}{0.552941,0.898039,0.631373}%
\pgfsetstrokecolor{currentstroke}%
\pgfsetstrokeopacity{0.200000}%
\pgfsetdash{}{0pt}%
\pgfpathmoveto{\pgfqpoint{4.386424in}{5.571510in}}%
\pgfpathlineto{\pgfqpoint{5.849815in}{4.522824in}}%
\pgfusepath{stroke}%
\end{pgfscope}%
\begin{pgfscope}%
\pgfpathrectangle{\pgfqpoint{0.481978in}{0.331635in}}{\pgfqpoint{9.300000in}{7.700000in}}%
\pgfusepath{clip}%
\pgfsetrectcap%
\pgfsetroundjoin%
\pgfsetlinewidth{1.505625pt}%
\definecolor{currentstroke}{rgb}{0.552941,0.898039,0.631373}%
\pgfsetstrokecolor{currentstroke}%
\pgfsetstrokeopacity{0.200000}%
\pgfsetdash{}{0pt}%
\pgfpathmoveto{\pgfqpoint{4.878045in}{6.103326in}}%
\pgfpathlineto{\pgfqpoint{5.849815in}{4.522824in}}%
\pgfusepath{stroke}%
\end{pgfscope}%
\begin{pgfscope}%
\pgfpathrectangle{\pgfqpoint{0.481978in}{0.331635in}}{\pgfqpoint{9.300000in}{7.700000in}}%
\pgfusepath{clip}%
\pgfsetrectcap%
\pgfsetroundjoin%
\pgfsetlinewidth{1.505625pt}%
\definecolor{currentstroke}{rgb}{0.552941,0.898039,0.631373}%
\pgfsetstrokecolor{currentstroke}%
\pgfsetstrokeopacity{0.200000}%
\pgfsetdash{}{0pt}%
\pgfpathmoveto{\pgfqpoint{9.259474in}{5.199290in}}%
\pgfpathlineto{\pgfqpoint{5.849815in}{4.522824in}}%
\pgfusepath{stroke}%
\end{pgfscope}%
\begin{pgfscope}%
\pgfpathrectangle{\pgfqpoint{0.481978in}{0.331635in}}{\pgfqpoint{9.300000in}{7.700000in}}%
\pgfusepath{clip}%
\pgfsetrectcap%
\pgfsetroundjoin%
\pgfsetlinewidth{1.505625pt}%
\definecolor{currentstroke}{rgb}{0.552941,0.898039,0.631373}%
\pgfsetstrokecolor{currentstroke}%
\pgfsetstrokeopacity{0.200000}%
\pgfsetdash{}{0pt}%
\pgfpathmoveto{\pgfqpoint{8.569592in}{4.792319in}}%
\pgfpathlineto{\pgfqpoint{5.849815in}{4.522824in}}%
\pgfusepath{stroke}%
\end{pgfscope}%
\begin{pgfscope}%
\pgfpathrectangle{\pgfqpoint{0.481978in}{0.331635in}}{\pgfqpoint{9.300000in}{7.700000in}}%
\pgfusepath{clip}%
\pgfsetrectcap%
\pgfsetroundjoin%
\pgfsetlinewidth{1.505625pt}%
\definecolor{currentstroke}{rgb}{0.552941,0.898039,0.631373}%
\pgfsetstrokecolor{currentstroke}%
\pgfsetstrokeopacity{0.200000}%
\pgfsetdash{}{0pt}%
\pgfpathmoveto{\pgfqpoint{8.786772in}{4.571593in}}%
\pgfpathlineto{\pgfqpoint{5.849815in}{4.522824in}}%
\pgfusepath{stroke}%
\end{pgfscope}%
\begin{pgfscope}%
\pgfpathrectangle{\pgfqpoint{0.481978in}{0.331635in}}{\pgfqpoint{9.300000in}{7.700000in}}%
\pgfusepath{clip}%
\pgfsetrectcap%
\pgfsetroundjoin%
\pgfsetlinewidth{1.505625pt}%
\definecolor{currentstroke}{rgb}{0.552941,0.898039,0.631373}%
\pgfsetstrokecolor{currentstroke}%
\pgfsetstrokeopacity{0.200000}%
\pgfsetdash{}{0pt}%
\pgfpathmoveto{\pgfqpoint{7.733235in}{4.906244in}}%
\pgfpathlineto{\pgfqpoint{5.849815in}{4.522824in}}%
\pgfusepath{stroke}%
\end{pgfscope}%
\begin{pgfscope}%
\pgfpathrectangle{\pgfqpoint{0.481978in}{0.331635in}}{\pgfqpoint{9.300000in}{7.700000in}}%
\pgfusepath{clip}%
\pgfsetrectcap%
\pgfsetroundjoin%
\pgfsetlinewidth{1.505625pt}%
\definecolor{currentstroke}{rgb}{0.552941,0.898039,0.631373}%
\pgfsetstrokecolor{currentstroke}%
\pgfsetstrokeopacity{0.200000}%
\pgfsetdash{}{0pt}%
\pgfpathmoveto{\pgfqpoint{5.023248in}{5.832186in}}%
\pgfpathlineto{\pgfqpoint{5.849815in}{4.522824in}}%
\pgfusepath{stroke}%
\end{pgfscope}%
\begin{pgfscope}%
\pgfpathrectangle{\pgfqpoint{0.481978in}{0.331635in}}{\pgfqpoint{9.300000in}{7.700000in}}%
\pgfusepath{clip}%
\pgfsetrectcap%
\pgfsetroundjoin%
\pgfsetlinewidth{1.505625pt}%
\definecolor{currentstroke}{rgb}{0.552941,0.898039,0.631373}%
\pgfsetstrokecolor{currentstroke}%
\pgfsetstrokeopacity{0.200000}%
\pgfsetdash{}{0pt}%
\pgfpathmoveto{\pgfqpoint{4.955252in}{6.273953in}}%
\pgfpathlineto{\pgfqpoint{5.849815in}{4.522824in}}%
\pgfusepath{stroke}%
\end{pgfscope}%
\begin{pgfscope}%
\pgfpathrectangle{\pgfqpoint{0.481978in}{0.331635in}}{\pgfqpoint{9.300000in}{7.700000in}}%
\pgfusepath{clip}%
\pgfsetrectcap%
\pgfsetroundjoin%
\pgfsetlinewidth{1.505625pt}%
\definecolor{currentstroke}{rgb}{0.552941,0.898039,0.631373}%
\pgfsetstrokecolor{currentstroke}%
\pgfsetstrokeopacity{0.200000}%
\pgfsetdash{}{0pt}%
\pgfpathmoveto{\pgfqpoint{7.433098in}{2.265608in}}%
\pgfpathlineto{\pgfqpoint{5.849815in}{4.522824in}}%
\pgfusepath{stroke}%
\end{pgfscope}%
\begin{pgfscope}%
\pgfpathrectangle{\pgfqpoint{0.481978in}{0.331635in}}{\pgfqpoint{9.300000in}{7.700000in}}%
\pgfusepath{clip}%
\pgfsetrectcap%
\pgfsetroundjoin%
\pgfsetlinewidth{1.505625pt}%
\definecolor{currentstroke}{rgb}{0.552941,0.898039,0.631373}%
\pgfsetstrokecolor{currentstroke}%
\pgfsetstrokeopacity{0.200000}%
\pgfsetdash{}{0pt}%
\pgfpathmoveto{\pgfqpoint{8.232070in}{5.815087in}}%
\pgfpathlineto{\pgfqpoint{5.849815in}{4.522824in}}%
\pgfusepath{stroke}%
\end{pgfscope}%
\begin{pgfscope}%
\pgfpathrectangle{\pgfqpoint{0.481978in}{0.331635in}}{\pgfqpoint{9.300000in}{7.700000in}}%
\pgfusepath{clip}%
\pgfsetrectcap%
\pgfsetroundjoin%
\pgfsetlinewidth{1.505625pt}%
\definecolor{currentstroke}{rgb}{0.552941,0.898039,0.631373}%
\pgfsetstrokecolor{currentstroke}%
\pgfsetstrokeopacity{0.200000}%
\pgfsetdash{}{0pt}%
\pgfpathmoveto{\pgfqpoint{3.011222in}{1.685197in}}%
\pgfpathlineto{\pgfqpoint{5.849815in}{4.522824in}}%
\pgfusepath{stroke}%
\end{pgfscope}%
\begin{pgfscope}%
\pgfpathrectangle{\pgfqpoint{0.481978in}{0.331635in}}{\pgfqpoint{9.300000in}{7.700000in}}%
\pgfusepath{clip}%
\pgfsetrectcap%
\pgfsetroundjoin%
\pgfsetlinewidth{1.505625pt}%
\definecolor{currentstroke}{rgb}{0.552941,0.898039,0.631373}%
\pgfsetstrokecolor{currentstroke}%
\pgfsetstrokeopacity{0.200000}%
\pgfsetdash{}{0pt}%
\pgfpathmoveto{\pgfqpoint{4.782892in}{6.148818in}}%
\pgfpathlineto{\pgfqpoint{5.849815in}{4.522824in}}%
\pgfusepath{stroke}%
\end{pgfscope}%
\begin{pgfscope}%
\pgfpathrectangle{\pgfqpoint{0.481978in}{0.331635in}}{\pgfqpoint{9.300000in}{7.700000in}}%
\pgfusepath{clip}%
\pgfsetrectcap%
\pgfsetroundjoin%
\pgfsetlinewidth{1.505625pt}%
\definecolor{currentstroke}{rgb}{0.552941,0.898039,0.631373}%
\pgfsetstrokecolor{currentstroke}%
\pgfsetstrokeopacity{0.200000}%
\pgfsetdash{}{0pt}%
\pgfpathmoveto{\pgfqpoint{3.628307in}{5.297078in}}%
\pgfpathlineto{\pgfqpoint{5.849815in}{4.522824in}}%
\pgfusepath{stroke}%
\end{pgfscope}%
\begin{pgfscope}%
\pgfpathrectangle{\pgfqpoint{0.481978in}{0.331635in}}{\pgfqpoint{9.300000in}{7.700000in}}%
\pgfusepath{clip}%
\pgfsetrectcap%
\pgfsetroundjoin%
\pgfsetlinewidth{1.505625pt}%
\definecolor{currentstroke}{rgb}{0.552941,0.898039,0.631373}%
\pgfsetstrokecolor{currentstroke}%
\pgfsetstrokeopacity{0.200000}%
\pgfsetdash{}{0pt}%
\pgfpathmoveto{\pgfqpoint{3.687739in}{5.241357in}}%
\pgfpathlineto{\pgfqpoint{5.849815in}{4.522824in}}%
\pgfusepath{stroke}%
\end{pgfscope}%
\begin{pgfscope}%
\pgfpathrectangle{\pgfqpoint{0.481978in}{0.331635in}}{\pgfqpoint{9.300000in}{7.700000in}}%
\pgfusepath{clip}%
\pgfsetrectcap%
\pgfsetroundjoin%
\pgfsetlinewidth{1.505625pt}%
\definecolor{currentstroke}{rgb}{0.552941,0.898039,0.631373}%
\pgfsetstrokecolor{currentstroke}%
\pgfsetstrokeopacity{0.200000}%
\pgfsetdash{}{0pt}%
\pgfpathmoveto{\pgfqpoint{7.445589in}{2.826774in}}%
\pgfpathlineto{\pgfqpoint{5.849815in}{4.522824in}}%
\pgfusepath{stroke}%
\end{pgfscope}%
\begin{pgfscope}%
\pgfpathrectangle{\pgfqpoint{0.481978in}{0.331635in}}{\pgfqpoint{9.300000in}{7.700000in}}%
\pgfusepath{clip}%
\pgfsetrectcap%
\pgfsetroundjoin%
\pgfsetlinewidth{1.505625pt}%
\definecolor{currentstroke}{rgb}{0.552941,0.898039,0.631373}%
\pgfsetstrokecolor{currentstroke}%
\pgfsetstrokeopacity{0.200000}%
\pgfsetdash{}{0pt}%
\pgfpathmoveto{\pgfqpoint{8.934791in}{5.143139in}}%
\pgfpathlineto{\pgfqpoint{5.849815in}{4.522824in}}%
\pgfusepath{stroke}%
\end{pgfscope}%
\begin{pgfscope}%
\pgfpathrectangle{\pgfqpoint{0.481978in}{0.331635in}}{\pgfqpoint{9.300000in}{7.700000in}}%
\pgfusepath{clip}%
\pgfsetrectcap%
\pgfsetroundjoin%
\pgfsetlinewidth{1.505625pt}%
\definecolor{currentstroke}{rgb}{0.552941,0.898039,0.631373}%
\pgfsetstrokecolor{currentstroke}%
\pgfsetstrokeopacity{0.200000}%
\pgfsetdash{}{0pt}%
\pgfpathmoveto{\pgfqpoint{3.508572in}{3.721693in}}%
\pgfpathlineto{\pgfqpoint{5.849815in}{4.522824in}}%
\pgfusepath{stroke}%
\end{pgfscope}%
\begin{pgfscope}%
\pgfpathrectangle{\pgfqpoint{0.481978in}{0.331635in}}{\pgfqpoint{9.300000in}{7.700000in}}%
\pgfusepath{clip}%
\pgfsetrectcap%
\pgfsetroundjoin%
\pgfsetlinewidth{1.505625pt}%
\definecolor{currentstroke}{rgb}{0.552941,0.898039,0.631373}%
\pgfsetstrokecolor{currentstroke}%
\pgfsetstrokeopacity{0.200000}%
\pgfsetdash{}{0pt}%
\pgfpathmoveto{\pgfqpoint{4.192874in}{1.559524in}}%
\pgfpathlineto{\pgfqpoint{5.849815in}{4.522824in}}%
\pgfusepath{stroke}%
\end{pgfscope}%
\begin{pgfscope}%
\pgfpathrectangle{\pgfqpoint{0.481978in}{0.331635in}}{\pgfqpoint{9.300000in}{7.700000in}}%
\pgfusepath{clip}%
\pgfsetrectcap%
\pgfsetroundjoin%
\pgfsetlinewidth{1.505625pt}%
\definecolor{currentstroke}{rgb}{0.552941,0.898039,0.631373}%
\pgfsetstrokecolor{currentstroke}%
\pgfsetstrokeopacity{0.200000}%
\pgfsetdash{}{0pt}%
\pgfpathmoveto{\pgfqpoint{4.001930in}{5.032636in}}%
\pgfpathlineto{\pgfqpoint{5.849815in}{4.522824in}}%
\pgfusepath{stroke}%
\end{pgfscope}%
\begin{pgfscope}%
\pgfpathrectangle{\pgfqpoint{0.481978in}{0.331635in}}{\pgfqpoint{9.300000in}{7.700000in}}%
\pgfusepath{clip}%
\pgfsetrectcap%
\pgfsetroundjoin%
\pgfsetlinewidth{1.505625pt}%
\definecolor{currentstroke}{rgb}{0.552941,0.898039,0.631373}%
\pgfsetstrokecolor{currentstroke}%
\pgfsetstrokeopacity{0.200000}%
\pgfsetdash{}{0pt}%
\pgfpathmoveto{\pgfqpoint{6.060999in}{1.792912in}}%
\pgfpathlineto{\pgfqpoint{5.849815in}{4.522824in}}%
\pgfusepath{stroke}%
\end{pgfscope}%
\begin{pgfscope}%
\pgfpathrectangle{\pgfqpoint{0.481978in}{0.331635in}}{\pgfqpoint{9.300000in}{7.700000in}}%
\pgfusepath{clip}%
\pgfsetrectcap%
\pgfsetroundjoin%
\pgfsetlinewidth{1.505625pt}%
\definecolor{currentstroke}{rgb}{0.552941,0.898039,0.631373}%
\pgfsetstrokecolor{currentstroke}%
\pgfsetstrokeopacity{0.200000}%
\pgfsetdash{}{0pt}%
\pgfpathmoveto{\pgfqpoint{2.449894in}{1.698466in}}%
\pgfpathlineto{\pgfqpoint{5.849815in}{4.522824in}}%
\pgfusepath{stroke}%
\end{pgfscope}%
\begin{pgfscope}%
\pgfpathrectangle{\pgfqpoint{0.481978in}{0.331635in}}{\pgfqpoint{9.300000in}{7.700000in}}%
\pgfusepath{clip}%
\pgfsetrectcap%
\pgfsetroundjoin%
\pgfsetlinewidth{1.505625pt}%
\definecolor{currentstroke}{rgb}{0.552941,0.898039,0.631373}%
\pgfsetstrokecolor{currentstroke}%
\pgfsetstrokeopacity{0.200000}%
\pgfsetdash{}{0pt}%
\pgfpathmoveto{\pgfqpoint{7.028940in}{4.751090in}}%
\pgfpathlineto{\pgfqpoint{5.849815in}{4.522824in}}%
\pgfusepath{stroke}%
\end{pgfscope}%
\begin{pgfscope}%
\pgfpathrectangle{\pgfqpoint{0.481978in}{0.331635in}}{\pgfqpoint{9.300000in}{7.700000in}}%
\pgfusepath{clip}%
\pgfsetrectcap%
\pgfsetroundjoin%
\pgfsetlinewidth{1.505625pt}%
\definecolor{currentstroke}{rgb}{0.552941,0.898039,0.631373}%
\pgfsetstrokecolor{currentstroke}%
\pgfsetstrokeopacity{0.200000}%
\pgfsetdash{}{0pt}%
\pgfpathmoveto{\pgfqpoint{5.259017in}{7.595751in}}%
\pgfpathlineto{\pgfqpoint{5.849815in}{4.522824in}}%
\pgfusepath{stroke}%
\end{pgfscope}%
\begin{pgfscope}%
\pgfpathrectangle{\pgfqpoint{0.481978in}{0.331635in}}{\pgfqpoint{9.300000in}{7.700000in}}%
\pgfusepath{clip}%
\pgfsetrectcap%
\pgfsetroundjoin%
\pgfsetlinewidth{1.505625pt}%
\definecolor{currentstroke}{rgb}{0.552941,0.898039,0.631373}%
\pgfsetstrokecolor{currentstroke}%
\pgfsetstrokeopacity{0.200000}%
\pgfsetdash{}{0pt}%
\pgfpathmoveto{\pgfqpoint{9.184137in}{5.106329in}}%
\pgfpathlineto{\pgfqpoint{5.849815in}{4.522824in}}%
\pgfusepath{stroke}%
\end{pgfscope}%
\begin{pgfscope}%
\pgfpathrectangle{\pgfqpoint{0.481978in}{0.331635in}}{\pgfqpoint{9.300000in}{7.700000in}}%
\pgfusepath{clip}%
\pgfsetrectcap%
\pgfsetroundjoin%
\pgfsetlinewidth{1.505625pt}%
\definecolor{currentstroke}{rgb}{0.552941,0.898039,0.631373}%
\pgfsetstrokecolor{currentstroke}%
\pgfsetstrokeopacity{0.200000}%
\pgfsetdash{}{0pt}%
\pgfpathmoveto{\pgfqpoint{8.944873in}{6.293748in}}%
\pgfpathlineto{\pgfqpoint{5.849815in}{4.522824in}}%
\pgfusepath{stroke}%
\end{pgfscope}%
\begin{pgfscope}%
\pgfpathrectangle{\pgfqpoint{0.481978in}{0.331635in}}{\pgfqpoint{9.300000in}{7.700000in}}%
\pgfusepath{clip}%
\pgfsetrectcap%
\pgfsetroundjoin%
\pgfsetlinewidth{1.505625pt}%
\definecolor{currentstroke}{rgb}{0.552941,0.898039,0.631373}%
\pgfsetstrokecolor{currentstroke}%
\pgfsetstrokeopacity{0.200000}%
\pgfsetdash{}{0pt}%
\pgfpathmoveto{\pgfqpoint{7.733013in}{3.491196in}}%
\pgfpathlineto{\pgfqpoint{5.849815in}{4.522824in}}%
\pgfusepath{stroke}%
\end{pgfscope}%
\begin{pgfscope}%
\pgfpathrectangle{\pgfqpoint{0.481978in}{0.331635in}}{\pgfqpoint{9.300000in}{7.700000in}}%
\pgfusepath{clip}%
\pgfsetrectcap%
\pgfsetroundjoin%
\pgfsetlinewidth{1.505625pt}%
\definecolor{currentstroke}{rgb}{0.552941,0.898039,0.631373}%
\pgfsetstrokecolor{currentstroke}%
\pgfsetstrokeopacity{0.200000}%
\pgfsetdash{}{0pt}%
\pgfpathmoveto{\pgfqpoint{5.907061in}{2.100838in}}%
\pgfpathlineto{\pgfqpoint{5.849815in}{4.522824in}}%
\pgfusepath{stroke}%
\end{pgfscope}%
\begin{pgfscope}%
\pgfpathrectangle{\pgfqpoint{0.481978in}{0.331635in}}{\pgfqpoint{9.300000in}{7.700000in}}%
\pgfusepath{clip}%
\pgfsetrectcap%
\pgfsetroundjoin%
\pgfsetlinewidth{1.505625pt}%
\definecolor{currentstroke}{rgb}{0.552941,0.898039,0.631373}%
\pgfsetstrokecolor{currentstroke}%
\pgfsetstrokeopacity{0.200000}%
\pgfsetdash{}{0pt}%
\pgfpathmoveto{\pgfqpoint{4.839240in}{5.911338in}}%
\pgfpathlineto{\pgfqpoint{5.849815in}{4.522824in}}%
\pgfusepath{stroke}%
\end{pgfscope}%
\begin{pgfscope}%
\pgfpathrectangle{\pgfqpoint{0.481978in}{0.331635in}}{\pgfqpoint{9.300000in}{7.700000in}}%
\pgfusepath{clip}%
\pgfsetrectcap%
\pgfsetroundjoin%
\pgfsetlinewidth{1.505625pt}%
\definecolor{currentstroke}{rgb}{0.552941,0.898039,0.631373}%
\pgfsetstrokecolor{currentstroke}%
\pgfsetstrokeopacity{0.200000}%
\pgfsetdash{}{0pt}%
\pgfpathmoveto{\pgfqpoint{2.381245in}{2.304607in}}%
\pgfpathlineto{\pgfqpoint{5.849815in}{4.522824in}}%
\pgfusepath{stroke}%
\end{pgfscope}%
\begin{pgfscope}%
\pgfpathrectangle{\pgfqpoint{0.481978in}{0.331635in}}{\pgfqpoint{9.300000in}{7.700000in}}%
\pgfusepath{clip}%
\pgfsetrectcap%
\pgfsetroundjoin%
\pgfsetlinewidth{1.505625pt}%
\definecolor{currentstroke}{rgb}{0.552941,0.898039,0.631373}%
\pgfsetstrokecolor{currentstroke}%
\pgfsetstrokeopacity{0.200000}%
\pgfsetdash{}{0pt}%
\pgfpathmoveto{\pgfqpoint{7.266400in}{2.402434in}}%
\pgfpathlineto{\pgfqpoint{5.849815in}{4.522824in}}%
\pgfusepath{stroke}%
\end{pgfscope}%
\begin{pgfscope}%
\pgfpathrectangle{\pgfqpoint{0.481978in}{0.331635in}}{\pgfqpoint{9.300000in}{7.700000in}}%
\pgfusepath{clip}%
\pgfsetrectcap%
\pgfsetroundjoin%
\pgfsetlinewidth{1.505625pt}%
\definecolor{currentstroke}{rgb}{0.552941,0.898039,0.631373}%
\pgfsetstrokecolor{currentstroke}%
\pgfsetstrokeopacity{0.200000}%
\pgfsetdash{}{0pt}%
\pgfpathmoveto{\pgfqpoint{3.410609in}{3.381276in}}%
\pgfpathlineto{\pgfqpoint{5.849815in}{4.522824in}}%
\pgfusepath{stroke}%
\end{pgfscope}%
\begin{pgfscope}%
\pgfpathrectangle{\pgfqpoint{0.481978in}{0.331635in}}{\pgfqpoint{9.300000in}{7.700000in}}%
\pgfusepath{clip}%
\pgfsetrectcap%
\pgfsetroundjoin%
\pgfsetlinewidth{1.505625pt}%
\definecolor{currentstroke}{rgb}{0.552941,0.898039,0.631373}%
\pgfsetstrokecolor{currentstroke}%
\pgfsetstrokeopacity{0.200000}%
\pgfsetdash{}{0pt}%
\pgfpathmoveto{\pgfqpoint{9.044900in}{5.310730in}}%
\pgfpathlineto{\pgfqpoint{5.849815in}{4.522824in}}%
\pgfusepath{stroke}%
\end{pgfscope}%
\begin{pgfscope}%
\pgfpathrectangle{\pgfqpoint{0.481978in}{0.331635in}}{\pgfqpoint{9.300000in}{7.700000in}}%
\pgfusepath{clip}%
\pgfsetrectcap%
\pgfsetroundjoin%
\pgfsetlinewidth{1.505625pt}%
\definecolor{currentstroke}{rgb}{0.552941,0.898039,0.631373}%
\pgfsetstrokecolor{currentstroke}%
\pgfsetstrokeopacity{0.200000}%
\pgfsetdash{}{0pt}%
\pgfpathmoveto{\pgfqpoint{8.024517in}{4.689990in}}%
\pgfpathlineto{\pgfqpoint{5.849815in}{4.522824in}}%
\pgfusepath{stroke}%
\end{pgfscope}%
\begin{pgfscope}%
\pgfpathrectangle{\pgfqpoint{0.481978in}{0.331635in}}{\pgfqpoint{9.300000in}{7.700000in}}%
\pgfusepath{clip}%
\pgfsetrectcap%
\pgfsetroundjoin%
\pgfsetlinewidth{1.505625pt}%
\definecolor{currentstroke}{rgb}{0.552941,0.898039,0.631373}%
\pgfsetstrokecolor{currentstroke}%
\pgfsetstrokeopacity{0.200000}%
\pgfsetdash{}{0pt}%
\pgfpathmoveto{\pgfqpoint{3.710429in}{4.755530in}}%
\pgfpathlineto{\pgfqpoint{5.849815in}{4.522824in}}%
\pgfusepath{stroke}%
\end{pgfscope}%
\begin{pgfscope}%
\pgfpathrectangle{\pgfqpoint{0.481978in}{0.331635in}}{\pgfqpoint{9.300000in}{7.700000in}}%
\pgfusepath{clip}%
\pgfsetrectcap%
\pgfsetroundjoin%
\pgfsetlinewidth{1.505625pt}%
\definecolor{currentstroke}{rgb}{0.552941,0.898039,0.631373}%
\pgfsetstrokecolor{currentstroke}%
\pgfsetstrokeopacity{0.200000}%
\pgfsetdash{}{0pt}%
\pgfpathmoveto{\pgfqpoint{5.690812in}{2.678714in}}%
\pgfpathlineto{\pgfqpoint{5.849815in}{4.522824in}}%
\pgfusepath{stroke}%
\end{pgfscope}%
\begin{pgfscope}%
\pgfpathrectangle{\pgfqpoint{0.481978in}{0.331635in}}{\pgfqpoint{9.300000in}{7.700000in}}%
\pgfusepath{clip}%
\pgfsetrectcap%
\pgfsetroundjoin%
\pgfsetlinewidth{1.505625pt}%
\definecolor{currentstroke}{rgb}{1.000000,0.623529,0.607843}%
\pgfsetstrokecolor{currentstroke}%
\pgfsetstrokeopacity{0.200000}%
\pgfsetdash{}{0pt}%
\pgfpathmoveto{\pgfqpoint{7.737423in}{5.882190in}}%
\pgfpathlineto{\pgfqpoint{5.342251in}{3.998981in}}%
\pgfusepath{stroke}%
\end{pgfscope}%
\begin{pgfscope}%
\pgfpathrectangle{\pgfqpoint{0.481978in}{0.331635in}}{\pgfqpoint{9.300000in}{7.700000in}}%
\pgfusepath{clip}%
\pgfsetrectcap%
\pgfsetroundjoin%
\pgfsetlinewidth{1.505625pt}%
\definecolor{currentstroke}{rgb}{1.000000,0.623529,0.607843}%
\pgfsetstrokecolor{currentstroke}%
\pgfsetstrokeopacity{0.200000}%
\pgfsetdash{}{0pt}%
\pgfpathmoveto{\pgfqpoint{8.478500in}{5.602262in}}%
\pgfpathlineto{\pgfqpoint{5.342251in}{3.998981in}}%
\pgfusepath{stroke}%
\end{pgfscope}%
\begin{pgfscope}%
\pgfpathrectangle{\pgfqpoint{0.481978in}{0.331635in}}{\pgfqpoint{9.300000in}{7.700000in}}%
\pgfusepath{clip}%
\pgfsetrectcap%
\pgfsetroundjoin%
\pgfsetlinewidth{1.505625pt}%
\definecolor{currentstroke}{rgb}{1.000000,0.623529,0.607843}%
\pgfsetstrokecolor{currentstroke}%
\pgfsetstrokeopacity{0.200000}%
\pgfsetdash{}{0pt}%
\pgfpathmoveto{\pgfqpoint{5.299357in}{2.456920in}}%
\pgfpathlineto{\pgfqpoint{5.342251in}{3.998981in}}%
\pgfusepath{stroke}%
\end{pgfscope}%
\begin{pgfscope}%
\pgfpathrectangle{\pgfqpoint{0.481978in}{0.331635in}}{\pgfqpoint{9.300000in}{7.700000in}}%
\pgfusepath{clip}%
\pgfsetrectcap%
\pgfsetroundjoin%
\pgfsetlinewidth{1.505625pt}%
\definecolor{currentstroke}{rgb}{1.000000,0.623529,0.607843}%
\pgfsetstrokecolor{currentstroke}%
\pgfsetstrokeopacity{0.200000}%
\pgfsetdash{}{0pt}%
\pgfpathmoveto{\pgfqpoint{5.050465in}{2.427924in}}%
\pgfpathlineto{\pgfqpoint{5.342251in}{3.998981in}}%
\pgfusepath{stroke}%
\end{pgfscope}%
\begin{pgfscope}%
\pgfpathrectangle{\pgfqpoint{0.481978in}{0.331635in}}{\pgfqpoint{9.300000in}{7.700000in}}%
\pgfusepath{clip}%
\pgfsetrectcap%
\pgfsetroundjoin%
\pgfsetlinewidth{1.505625pt}%
\definecolor{currentstroke}{rgb}{1.000000,0.623529,0.607843}%
\pgfsetstrokecolor{currentstroke}%
\pgfsetstrokeopacity{0.200000}%
\pgfsetdash{}{0pt}%
\pgfpathmoveto{\pgfqpoint{4.424855in}{3.491433in}}%
\pgfpathlineto{\pgfqpoint{5.342251in}{3.998981in}}%
\pgfusepath{stroke}%
\end{pgfscope}%
\begin{pgfscope}%
\pgfpathrectangle{\pgfqpoint{0.481978in}{0.331635in}}{\pgfqpoint{9.300000in}{7.700000in}}%
\pgfusepath{clip}%
\pgfsetrectcap%
\pgfsetroundjoin%
\pgfsetlinewidth{1.505625pt}%
\definecolor{currentstroke}{rgb}{1.000000,0.623529,0.607843}%
\pgfsetstrokecolor{currentstroke}%
\pgfsetstrokeopacity{0.200000}%
\pgfsetdash{}{0pt}%
\pgfpathmoveto{\pgfqpoint{4.703040in}{4.152158in}}%
\pgfpathlineto{\pgfqpoint{5.342251in}{3.998981in}}%
\pgfusepath{stroke}%
\end{pgfscope}%
\begin{pgfscope}%
\pgfpathrectangle{\pgfqpoint{0.481978in}{0.331635in}}{\pgfqpoint{9.300000in}{7.700000in}}%
\pgfusepath{clip}%
\pgfsetrectcap%
\pgfsetroundjoin%
\pgfsetlinewidth{1.505625pt}%
\definecolor{currentstroke}{rgb}{1.000000,0.623529,0.607843}%
\pgfsetstrokecolor{currentstroke}%
\pgfsetstrokeopacity{0.200000}%
\pgfsetdash{}{0pt}%
\pgfpathmoveto{\pgfqpoint{2.809853in}{3.881073in}}%
\pgfpathlineto{\pgfqpoint{5.342251in}{3.998981in}}%
\pgfusepath{stroke}%
\end{pgfscope}%
\begin{pgfscope}%
\pgfpathrectangle{\pgfqpoint{0.481978in}{0.331635in}}{\pgfqpoint{9.300000in}{7.700000in}}%
\pgfusepath{clip}%
\pgfsetrectcap%
\pgfsetroundjoin%
\pgfsetlinewidth{1.505625pt}%
\definecolor{currentstroke}{rgb}{1.000000,0.623529,0.607843}%
\pgfsetstrokecolor{currentstroke}%
\pgfsetstrokeopacity{0.200000}%
\pgfsetdash{}{0pt}%
\pgfpathmoveto{\pgfqpoint{3.971771in}{3.049785in}}%
\pgfpathlineto{\pgfqpoint{5.342251in}{3.998981in}}%
\pgfusepath{stroke}%
\end{pgfscope}%
\begin{pgfscope}%
\pgfpathrectangle{\pgfqpoint{0.481978in}{0.331635in}}{\pgfqpoint{9.300000in}{7.700000in}}%
\pgfusepath{clip}%
\pgfsetrectcap%
\pgfsetroundjoin%
\pgfsetlinewidth{1.505625pt}%
\definecolor{currentstroke}{rgb}{1.000000,0.623529,0.607843}%
\pgfsetstrokecolor{currentstroke}%
\pgfsetstrokeopacity{0.200000}%
\pgfsetdash{}{0pt}%
\pgfpathmoveto{\pgfqpoint{7.681317in}{5.102335in}}%
\pgfpathlineto{\pgfqpoint{5.342251in}{3.998981in}}%
\pgfusepath{stroke}%
\end{pgfscope}%
\begin{pgfscope}%
\pgfpathrectangle{\pgfqpoint{0.481978in}{0.331635in}}{\pgfqpoint{9.300000in}{7.700000in}}%
\pgfusepath{clip}%
\pgfsetrectcap%
\pgfsetroundjoin%
\pgfsetlinewidth{1.505625pt}%
\definecolor{currentstroke}{rgb}{1.000000,0.623529,0.607843}%
\pgfsetstrokecolor{currentstroke}%
\pgfsetstrokeopacity{0.200000}%
\pgfsetdash{}{0pt}%
\pgfpathmoveto{\pgfqpoint{4.922074in}{3.141067in}}%
\pgfpathlineto{\pgfqpoint{5.342251in}{3.998981in}}%
\pgfusepath{stroke}%
\end{pgfscope}%
\begin{pgfscope}%
\pgfpathrectangle{\pgfqpoint{0.481978in}{0.331635in}}{\pgfqpoint{9.300000in}{7.700000in}}%
\pgfusepath{clip}%
\pgfsetrectcap%
\pgfsetroundjoin%
\pgfsetlinewidth{1.505625pt}%
\definecolor{currentstroke}{rgb}{1.000000,0.623529,0.607843}%
\pgfsetstrokecolor{currentstroke}%
\pgfsetstrokeopacity{0.200000}%
\pgfsetdash{}{0pt}%
\pgfpathmoveto{\pgfqpoint{4.349051in}{3.146159in}}%
\pgfpathlineto{\pgfqpoint{5.342251in}{3.998981in}}%
\pgfusepath{stroke}%
\end{pgfscope}%
\begin{pgfscope}%
\pgfpathrectangle{\pgfqpoint{0.481978in}{0.331635in}}{\pgfqpoint{9.300000in}{7.700000in}}%
\pgfusepath{clip}%
\pgfsetrectcap%
\pgfsetroundjoin%
\pgfsetlinewidth{1.505625pt}%
\definecolor{currentstroke}{rgb}{1.000000,0.623529,0.607843}%
\pgfsetstrokecolor{currentstroke}%
\pgfsetstrokeopacity{0.200000}%
\pgfsetdash{}{0pt}%
\pgfpathmoveto{\pgfqpoint{4.285306in}{3.897257in}}%
\pgfpathlineto{\pgfqpoint{5.342251in}{3.998981in}}%
\pgfusepath{stroke}%
\end{pgfscope}%
\begin{pgfscope}%
\pgfpathrectangle{\pgfqpoint{0.481978in}{0.331635in}}{\pgfqpoint{9.300000in}{7.700000in}}%
\pgfusepath{clip}%
\pgfsetrectcap%
\pgfsetroundjoin%
\pgfsetlinewidth{1.505625pt}%
\definecolor{currentstroke}{rgb}{1.000000,0.623529,0.607843}%
\pgfsetstrokecolor{currentstroke}%
\pgfsetstrokeopacity{0.200000}%
\pgfsetdash{}{0pt}%
\pgfpathmoveto{\pgfqpoint{5.628759in}{1.359840in}}%
\pgfpathlineto{\pgfqpoint{5.342251in}{3.998981in}}%
\pgfusepath{stroke}%
\end{pgfscope}%
\begin{pgfscope}%
\pgfpathrectangle{\pgfqpoint{0.481978in}{0.331635in}}{\pgfqpoint{9.300000in}{7.700000in}}%
\pgfusepath{clip}%
\pgfsetrectcap%
\pgfsetroundjoin%
\pgfsetlinewidth{1.505625pt}%
\definecolor{currentstroke}{rgb}{1.000000,0.623529,0.607843}%
\pgfsetstrokecolor{currentstroke}%
\pgfsetstrokeopacity{0.200000}%
\pgfsetdash{}{0pt}%
\pgfpathmoveto{\pgfqpoint{2.208081in}{3.731947in}}%
\pgfpathlineto{\pgfqpoint{5.342251in}{3.998981in}}%
\pgfusepath{stroke}%
\end{pgfscope}%
\begin{pgfscope}%
\pgfpathrectangle{\pgfqpoint{0.481978in}{0.331635in}}{\pgfqpoint{9.300000in}{7.700000in}}%
\pgfusepath{clip}%
\pgfsetrectcap%
\pgfsetroundjoin%
\pgfsetlinewidth{1.505625pt}%
\definecolor{currentstroke}{rgb}{1.000000,0.623529,0.607843}%
\pgfsetstrokecolor{currentstroke}%
\pgfsetstrokeopacity{0.200000}%
\pgfsetdash{}{0pt}%
\pgfpathmoveto{\pgfqpoint{5.627987in}{1.357817in}}%
\pgfpathlineto{\pgfqpoint{5.342251in}{3.998981in}}%
\pgfusepath{stroke}%
\end{pgfscope}%
\begin{pgfscope}%
\pgfpathrectangle{\pgfqpoint{0.481978in}{0.331635in}}{\pgfqpoint{9.300000in}{7.700000in}}%
\pgfusepath{clip}%
\pgfsetrectcap%
\pgfsetroundjoin%
\pgfsetlinewidth{1.505625pt}%
\definecolor{currentstroke}{rgb}{1.000000,0.623529,0.607843}%
\pgfsetstrokecolor{currentstroke}%
\pgfsetstrokeopacity{0.200000}%
\pgfsetdash{}{0pt}%
\pgfpathmoveto{\pgfqpoint{7.678869in}{4.514349in}}%
\pgfpathlineto{\pgfqpoint{5.342251in}{3.998981in}}%
\pgfusepath{stroke}%
\end{pgfscope}%
\begin{pgfscope}%
\pgfpathrectangle{\pgfqpoint{0.481978in}{0.331635in}}{\pgfqpoint{9.300000in}{7.700000in}}%
\pgfusepath{clip}%
\pgfsetrectcap%
\pgfsetroundjoin%
\pgfsetlinewidth{1.505625pt}%
\definecolor{currentstroke}{rgb}{1.000000,0.623529,0.607843}%
\pgfsetstrokecolor{currentstroke}%
\pgfsetstrokeopacity{0.200000}%
\pgfsetdash{}{0pt}%
\pgfpathmoveto{\pgfqpoint{3.663088in}{2.022372in}}%
\pgfpathlineto{\pgfqpoint{5.342251in}{3.998981in}}%
\pgfusepath{stroke}%
\end{pgfscope}%
\begin{pgfscope}%
\pgfpathrectangle{\pgfqpoint{0.481978in}{0.331635in}}{\pgfqpoint{9.300000in}{7.700000in}}%
\pgfusepath{clip}%
\pgfsetrectcap%
\pgfsetroundjoin%
\pgfsetlinewidth{1.505625pt}%
\definecolor{currentstroke}{rgb}{1.000000,0.623529,0.607843}%
\pgfsetstrokecolor{currentstroke}%
\pgfsetstrokeopacity{0.200000}%
\pgfsetdash{}{0pt}%
\pgfpathmoveto{\pgfqpoint{2.974676in}{3.711091in}}%
\pgfpathlineto{\pgfqpoint{5.342251in}{3.998981in}}%
\pgfusepath{stroke}%
\end{pgfscope}%
\begin{pgfscope}%
\pgfpathrectangle{\pgfqpoint{0.481978in}{0.331635in}}{\pgfqpoint{9.300000in}{7.700000in}}%
\pgfusepath{clip}%
\pgfsetrectcap%
\pgfsetroundjoin%
\pgfsetlinewidth{1.505625pt}%
\definecolor{currentstroke}{rgb}{1.000000,0.623529,0.607843}%
\pgfsetstrokecolor{currentstroke}%
\pgfsetstrokeopacity{0.200000}%
\pgfsetdash{}{0pt}%
\pgfpathmoveto{\pgfqpoint{4.287456in}{3.591981in}}%
\pgfpathlineto{\pgfqpoint{5.342251in}{3.998981in}}%
\pgfusepath{stroke}%
\end{pgfscope}%
\begin{pgfscope}%
\pgfpathrectangle{\pgfqpoint{0.481978in}{0.331635in}}{\pgfqpoint{9.300000in}{7.700000in}}%
\pgfusepath{clip}%
\pgfsetrectcap%
\pgfsetroundjoin%
\pgfsetlinewidth{1.505625pt}%
\definecolor{currentstroke}{rgb}{1.000000,0.623529,0.607843}%
\pgfsetstrokecolor{currentstroke}%
\pgfsetstrokeopacity{0.200000}%
\pgfsetdash{}{0pt}%
\pgfpathmoveto{\pgfqpoint{1.967877in}{5.589834in}}%
\pgfpathlineto{\pgfqpoint{5.342251in}{3.998981in}}%
\pgfusepath{stroke}%
\end{pgfscope}%
\begin{pgfscope}%
\pgfpathrectangle{\pgfqpoint{0.481978in}{0.331635in}}{\pgfqpoint{9.300000in}{7.700000in}}%
\pgfusepath{clip}%
\pgfsetrectcap%
\pgfsetroundjoin%
\pgfsetlinewidth{1.505625pt}%
\definecolor{currentstroke}{rgb}{1.000000,0.623529,0.607843}%
\pgfsetstrokecolor{currentstroke}%
\pgfsetstrokeopacity{0.200000}%
\pgfsetdash{}{0pt}%
\pgfpathmoveto{\pgfqpoint{3.570734in}{2.934416in}}%
\pgfpathlineto{\pgfqpoint{5.342251in}{3.998981in}}%
\pgfusepath{stroke}%
\end{pgfscope}%
\begin{pgfscope}%
\pgfpathrectangle{\pgfqpoint{0.481978in}{0.331635in}}{\pgfqpoint{9.300000in}{7.700000in}}%
\pgfusepath{clip}%
\pgfsetrectcap%
\pgfsetroundjoin%
\pgfsetlinewidth{1.505625pt}%
\definecolor{currentstroke}{rgb}{1.000000,0.623529,0.607843}%
\pgfsetstrokecolor{currentstroke}%
\pgfsetstrokeopacity{0.200000}%
\pgfsetdash{}{0pt}%
\pgfpathmoveto{\pgfqpoint{5.661130in}{4.673478in}}%
\pgfpathlineto{\pgfqpoint{5.342251in}{3.998981in}}%
\pgfusepath{stroke}%
\end{pgfscope}%
\begin{pgfscope}%
\pgfpathrectangle{\pgfqpoint{0.481978in}{0.331635in}}{\pgfqpoint{9.300000in}{7.700000in}}%
\pgfusepath{clip}%
\pgfsetrectcap%
\pgfsetroundjoin%
\pgfsetlinewidth{1.505625pt}%
\definecolor{currentstroke}{rgb}{1.000000,0.623529,0.607843}%
\pgfsetstrokecolor{currentstroke}%
\pgfsetstrokeopacity{0.200000}%
\pgfsetdash{}{0pt}%
\pgfpathmoveto{\pgfqpoint{4.720327in}{4.363239in}}%
\pgfpathlineto{\pgfqpoint{5.342251in}{3.998981in}}%
\pgfusepath{stroke}%
\end{pgfscope}%
\begin{pgfscope}%
\pgfpathrectangle{\pgfqpoint{0.481978in}{0.331635in}}{\pgfqpoint{9.300000in}{7.700000in}}%
\pgfusepath{clip}%
\pgfsetrectcap%
\pgfsetroundjoin%
\pgfsetlinewidth{1.505625pt}%
\definecolor{currentstroke}{rgb}{1.000000,0.623529,0.607843}%
\pgfsetstrokecolor{currentstroke}%
\pgfsetstrokeopacity{0.200000}%
\pgfsetdash{}{0pt}%
\pgfpathmoveto{\pgfqpoint{2.339602in}{4.130396in}}%
\pgfpathlineto{\pgfqpoint{5.342251in}{3.998981in}}%
\pgfusepath{stroke}%
\end{pgfscope}%
\begin{pgfscope}%
\pgfpathrectangle{\pgfqpoint{0.481978in}{0.331635in}}{\pgfqpoint{9.300000in}{7.700000in}}%
\pgfusepath{clip}%
\pgfsetrectcap%
\pgfsetroundjoin%
\pgfsetlinewidth{1.505625pt}%
\definecolor{currentstroke}{rgb}{1.000000,0.623529,0.607843}%
\pgfsetstrokecolor{currentstroke}%
\pgfsetstrokeopacity{0.200000}%
\pgfsetdash{}{0pt}%
\pgfpathmoveto{\pgfqpoint{7.776459in}{5.380558in}}%
\pgfpathlineto{\pgfqpoint{5.342251in}{3.998981in}}%
\pgfusepath{stroke}%
\end{pgfscope}%
\begin{pgfscope}%
\pgfpathrectangle{\pgfqpoint{0.481978in}{0.331635in}}{\pgfqpoint{9.300000in}{7.700000in}}%
\pgfusepath{clip}%
\pgfsetrectcap%
\pgfsetroundjoin%
\pgfsetlinewidth{1.505625pt}%
\definecolor{currentstroke}{rgb}{1.000000,0.623529,0.607843}%
\pgfsetstrokecolor{currentstroke}%
\pgfsetstrokeopacity{0.200000}%
\pgfsetdash{}{0pt}%
\pgfpathmoveto{\pgfqpoint{5.832034in}{1.507144in}}%
\pgfpathlineto{\pgfqpoint{5.342251in}{3.998981in}}%
\pgfusepath{stroke}%
\end{pgfscope}%
\begin{pgfscope}%
\pgfpathrectangle{\pgfqpoint{0.481978in}{0.331635in}}{\pgfqpoint{9.300000in}{7.700000in}}%
\pgfusepath{clip}%
\pgfsetrectcap%
\pgfsetroundjoin%
\pgfsetlinewidth{1.505625pt}%
\definecolor{currentstroke}{rgb}{1.000000,0.623529,0.607843}%
\pgfsetstrokecolor{currentstroke}%
\pgfsetstrokeopacity{0.200000}%
\pgfsetdash{}{0pt}%
\pgfpathmoveto{\pgfqpoint{8.011004in}{6.276407in}}%
\pgfpathlineto{\pgfqpoint{5.342251in}{3.998981in}}%
\pgfusepath{stroke}%
\end{pgfscope}%
\begin{pgfscope}%
\pgfpathrectangle{\pgfqpoint{0.481978in}{0.331635in}}{\pgfqpoint{9.300000in}{7.700000in}}%
\pgfusepath{clip}%
\pgfsetrectcap%
\pgfsetroundjoin%
\pgfsetlinewidth{1.505625pt}%
\definecolor{currentstroke}{rgb}{1.000000,0.623529,0.607843}%
\pgfsetstrokecolor{currentstroke}%
\pgfsetstrokeopacity{0.200000}%
\pgfsetdash{}{0pt}%
\pgfpathmoveto{\pgfqpoint{2.908514in}{3.329410in}}%
\pgfpathlineto{\pgfqpoint{5.342251in}{3.998981in}}%
\pgfusepath{stroke}%
\end{pgfscope}%
\begin{pgfscope}%
\pgfpathrectangle{\pgfqpoint{0.481978in}{0.331635in}}{\pgfqpoint{9.300000in}{7.700000in}}%
\pgfusepath{clip}%
\pgfsetrectcap%
\pgfsetroundjoin%
\pgfsetlinewidth{1.505625pt}%
\definecolor{currentstroke}{rgb}{1.000000,0.623529,0.607843}%
\pgfsetstrokecolor{currentstroke}%
\pgfsetstrokeopacity{0.200000}%
\pgfsetdash{}{0pt}%
\pgfpathmoveto{\pgfqpoint{6.465338in}{2.366611in}}%
\pgfpathlineto{\pgfqpoint{5.342251in}{3.998981in}}%
\pgfusepath{stroke}%
\end{pgfscope}%
\begin{pgfscope}%
\pgfpathrectangle{\pgfqpoint{0.481978in}{0.331635in}}{\pgfqpoint{9.300000in}{7.700000in}}%
\pgfusepath{clip}%
\pgfsetrectcap%
\pgfsetroundjoin%
\pgfsetlinewidth{1.505625pt}%
\definecolor{currentstroke}{rgb}{1.000000,0.623529,0.607843}%
\pgfsetstrokecolor{currentstroke}%
\pgfsetstrokeopacity{0.200000}%
\pgfsetdash{}{0pt}%
\pgfpathmoveto{\pgfqpoint{8.106589in}{6.343346in}}%
\pgfpathlineto{\pgfqpoint{5.342251in}{3.998981in}}%
\pgfusepath{stroke}%
\end{pgfscope}%
\begin{pgfscope}%
\pgfpathrectangle{\pgfqpoint{0.481978in}{0.331635in}}{\pgfqpoint{9.300000in}{7.700000in}}%
\pgfusepath{clip}%
\pgfsetrectcap%
\pgfsetroundjoin%
\pgfsetlinewidth{1.505625pt}%
\definecolor{currentstroke}{rgb}{1.000000,0.623529,0.607843}%
\pgfsetstrokecolor{currentstroke}%
\pgfsetstrokeopacity{0.200000}%
\pgfsetdash{}{0pt}%
\pgfpathmoveto{\pgfqpoint{7.991875in}{5.264052in}}%
\pgfpathlineto{\pgfqpoint{5.342251in}{3.998981in}}%
\pgfusepath{stroke}%
\end{pgfscope}%
\begin{pgfscope}%
\pgfpathrectangle{\pgfqpoint{0.481978in}{0.331635in}}{\pgfqpoint{9.300000in}{7.700000in}}%
\pgfusepath{clip}%
\pgfsetrectcap%
\pgfsetroundjoin%
\pgfsetlinewidth{1.505625pt}%
\definecolor{currentstroke}{rgb}{1.000000,0.623529,0.607843}%
\pgfsetstrokecolor{currentstroke}%
\pgfsetstrokeopacity{0.200000}%
\pgfsetdash{}{0pt}%
\pgfpathmoveto{\pgfqpoint{7.423833in}{5.573837in}}%
\pgfpathlineto{\pgfqpoint{5.342251in}{3.998981in}}%
\pgfusepath{stroke}%
\end{pgfscope}%
\begin{pgfscope}%
\pgfpathrectangle{\pgfqpoint{0.481978in}{0.331635in}}{\pgfqpoint{9.300000in}{7.700000in}}%
\pgfusepath{clip}%
\pgfsetrectcap%
\pgfsetroundjoin%
\pgfsetlinewidth{1.505625pt}%
\definecolor{currentstroke}{rgb}{1.000000,0.623529,0.607843}%
\pgfsetstrokecolor{currentstroke}%
\pgfsetstrokeopacity{0.200000}%
\pgfsetdash{}{0pt}%
\pgfpathmoveto{\pgfqpoint{4.192970in}{5.950908in}}%
\pgfpathlineto{\pgfqpoint{5.342251in}{3.998981in}}%
\pgfusepath{stroke}%
\end{pgfscope}%
\begin{pgfscope}%
\pgfpathrectangle{\pgfqpoint{0.481978in}{0.331635in}}{\pgfqpoint{9.300000in}{7.700000in}}%
\pgfusepath{clip}%
\pgfsetrectcap%
\pgfsetroundjoin%
\pgfsetlinewidth{1.505625pt}%
\definecolor{currentstroke}{rgb}{1.000000,0.623529,0.607843}%
\pgfsetstrokecolor{currentstroke}%
\pgfsetstrokeopacity{0.200000}%
\pgfsetdash{}{0pt}%
\pgfpathmoveto{\pgfqpoint{6.770367in}{5.489298in}}%
\pgfpathlineto{\pgfqpoint{5.342251in}{3.998981in}}%
\pgfusepath{stroke}%
\end{pgfscope}%
\begin{pgfscope}%
\pgfpathrectangle{\pgfqpoint{0.481978in}{0.331635in}}{\pgfqpoint{9.300000in}{7.700000in}}%
\pgfusepath{clip}%
\pgfsetrectcap%
\pgfsetroundjoin%
\pgfsetlinewidth{1.505625pt}%
\definecolor{currentstroke}{rgb}{1.000000,0.623529,0.607843}%
\pgfsetstrokecolor{currentstroke}%
\pgfsetstrokeopacity{0.200000}%
\pgfsetdash{}{0pt}%
\pgfpathmoveto{\pgfqpoint{7.446866in}{5.698685in}}%
\pgfpathlineto{\pgfqpoint{5.342251in}{3.998981in}}%
\pgfusepath{stroke}%
\end{pgfscope}%
\begin{pgfscope}%
\pgfpathrectangle{\pgfqpoint{0.481978in}{0.331635in}}{\pgfqpoint{9.300000in}{7.700000in}}%
\pgfusepath{clip}%
\pgfsetrectcap%
\pgfsetroundjoin%
\pgfsetlinewidth{1.505625pt}%
\definecolor{currentstroke}{rgb}{1.000000,0.623529,0.607843}%
\pgfsetstrokecolor{currentstroke}%
\pgfsetstrokeopacity{0.200000}%
\pgfsetdash{}{0pt}%
\pgfpathmoveto{\pgfqpoint{4.422769in}{4.172874in}}%
\pgfpathlineto{\pgfqpoint{5.342251in}{3.998981in}}%
\pgfusepath{stroke}%
\end{pgfscope}%
\begin{pgfscope}%
\pgfpathrectangle{\pgfqpoint{0.481978in}{0.331635in}}{\pgfqpoint{9.300000in}{7.700000in}}%
\pgfusepath{clip}%
\pgfsetrectcap%
\pgfsetroundjoin%
\pgfsetlinewidth{1.505625pt}%
\definecolor{currentstroke}{rgb}{1.000000,0.623529,0.607843}%
\pgfsetstrokecolor{currentstroke}%
\pgfsetstrokeopacity{0.200000}%
\pgfsetdash{}{0pt}%
\pgfpathmoveto{\pgfqpoint{4.632997in}{1.495729in}}%
\pgfpathlineto{\pgfqpoint{5.342251in}{3.998981in}}%
\pgfusepath{stroke}%
\end{pgfscope}%
\begin{pgfscope}%
\pgfpathrectangle{\pgfqpoint{0.481978in}{0.331635in}}{\pgfqpoint{9.300000in}{7.700000in}}%
\pgfusepath{clip}%
\pgfsetrectcap%
\pgfsetroundjoin%
\pgfsetlinewidth{1.505625pt}%
\definecolor{currentstroke}{rgb}{1.000000,0.623529,0.607843}%
\pgfsetstrokecolor{currentstroke}%
\pgfsetstrokeopacity{0.200000}%
\pgfsetdash{}{0pt}%
\pgfpathmoveto{\pgfqpoint{3.319473in}{2.345162in}}%
\pgfpathlineto{\pgfqpoint{5.342251in}{3.998981in}}%
\pgfusepath{stroke}%
\end{pgfscope}%
\begin{pgfscope}%
\pgfpathrectangle{\pgfqpoint{0.481978in}{0.331635in}}{\pgfqpoint{9.300000in}{7.700000in}}%
\pgfusepath{clip}%
\pgfsetrectcap%
\pgfsetroundjoin%
\pgfsetlinewidth{1.505625pt}%
\definecolor{currentstroke}{rgb}{1.000000,0.623529,0.607843}%
\pgfsetstrokecolor{currentstroke}%
\pgfsetstrokeopacity{0.200000}%
\pgfsetdash{}{0pt}%
\pgfpathmoveto{\pgfqpoint{3.661029in}{2.526550in}}%
\pgfpathlineto{\pgfqpoint{5.342251in}{3.998981in}}%
\pgfusepath{stroke}%
\end{pgfscope}%
\begin{pgfscope}%
\pgfpathrectangle{\pgfqpoint{0.481978in}{0.331635in}}{\pgfqpoint{9.300000in}{7.700000in}}%
\pgfusepath{clip}%
\pgfsetrectcap%
\pgfsetroundjoin%
\pgfsetlinewidth{1.505625pt}%
\definecolor{currentstroke}{rgb}{1.000000,0.623529,0.607843}%
\pgfsetstrokecolor{currentstroke}%
\pgfsetstrokeopacity{0.200000}%
\pgfsetdash{}{0pt}%
\pgfpathmoveto{\pgfqpoint{3.845381in}{3.244228in}}%
\pgfpathlineto{\pgfqpoint{5.342251in}{3.998981in}}%
\pgfusepath{stroke}%
\end{pgfscope}%
\begin{pgfscope}%
\pgfpathrectangle{\pgfqpoint{0.481978in}{0.331635in}}{\pgfqpoint{9.300000in}{7.700000in}}%
\pgfusepath{clip}%
\pgfsetrectcap%
\pgfsetroundjoin%
\pgfsetlinewidth{1.505625pt}%
\definecolor{currentstroke}{rgb}{1.000000,0.623529,0.607843}%
\pgfsetstrokecolor{currentstroke}%
\pgfsetstrokeopacity{0.200000}%
\pgfsetdash{}{0pt}%
\pgfpathmoveto{\pgfqpoint{4.050059in}{3.536130in}}%
\pgfpathlineto{\pgfqpoint{5.342251in}{3.998981in}}%
\pgfusepath{stroke}%
\end{pgfscope}%
\begin{pgfscope}%
\pgfpathrectangle{\pgfqpoint{0.481978in}{0.331635in}}{\pgfqpoint{9.300000in}{7.700000in}}%
\pgfusepath{clip}%
\pgfsetrectcap%
\pgfsetroundjoin%
\pgfsetlinewidth{1.505625pt}%
\definecolor{currentstroke}{rgb}{1.000000,0.623529,0.607843}%
\pgfsetstrokecolor{currentstroke}%
\pgfsetstrokeopacity{0.200000}%
\pgfsetdash{}{0pt}%
\pgfpathmoveto{\pgfqpoint{4.925237in}{7.465473in}}%
\pgfpathlineto{\pgfqpoint{5.342251in}{3.998981in}}%
\pgfusepath{stroke}%
\end{pgfscope}%
\begin{pgfscope}%
\pgfpathrectangle{\pgfqpoint{0.481978in}{0.331635in}}{\pgfqpoint{9.300000in}{7.700000in}}%
\pgfusepath{clip}%
\pgfsetrectcap%
\pgfsetroundjoin%
\pgfsetlinewidth{1.505625pt}%
\definecolor{currentstroke}{rgb}{1.000000,0.623529,0.607843}%
\pgfsetstrokecolor{currentstroke}%
\pgfsetstrokeopacity{0.200000}%
\pgfsetdash{}{0pt}%
\pgfpathmoveto{\pgfqpoint{6.991308in}{3.587991in}}%
\pgfpathlineto{\pgfqpoint{5.342251in}{3.998981in}}%
\pgfusepath{stroke}%
\end{pgfscope}%
\begin{pgfscope}%
\pgfpathrectangle{\pgfqpoint{0.481978in}{0.331635in}}{\pgfqpoint{9.300000in}{7.700000in}}%
\pgfusepath{clip}%
\pgfsetrectcap%
\pgfsetroundjoin%
\pgfsetlinewidth{1.505625pt}%
\definecolor{currentstroke}{rgb}{1.000000,0.623529,0.607843}%
\pgfsetstrokecolor{currentstroke}%
\pgfsetstrokeopacity{0.200000}%
\pgfsetdash{}{0pt}%
\pgfpathmoveto{\pgfqpoint{8.618470in}{6.062569in}}%
\pgfpathlineto{\pgfqpoint{5.342251in}{3.998981in}}%
\pgfusepath{stroke}%
\end{pgfscope}%
\begin{pgfscope}%
\pgfpathrectangle{\pgfqpoint{0.481978in}{0.331635in}}{\pgfqpoint{9.300000in}{7.700000in}}%
\pgfusepath{clip}%
\pgfsetrectcap%
\pgfsetroundjoin%
\pgfsetlinewidth{1.505625pt}%
\definecolor{currentstroke}{rgb}{1.000000,0.623529,0.607843}%
\pgfsetstrokecolor{currentstroke}%
\pgfsetstrokeopacity{0.200000}%
\pgfsetdash{}{0pt}%
\pgfpathmoveto{\pgfqpoint{7.337079in}{5.258872in}}%
\pgfpathlineto{\pgfqpoint{5.342251in}{3.998981in}}%
\pgfusepath{stroke}%
\end{pgfscope}%
\begin{pgfscope}%
\pgfpathrectangle{\pgfqpoint{0.481978in}{0.331635in}}{\pgfqpoint{9.300000in}{7.700000in}}%
\pgfusepath{clip}%
\pgfsetrectcap%
\pgfsetroundjoin%
\pgfsetlinewidth{1.505625pt}%
\definecolor{currentstroke}{rgb}{1.000000,0.623529,0.607843}%
\pgfsetstrokecolor{currentstroke}%
\pgfsetstrokeopacity{0.200000}%
\pgfsetdash{}{0pt}%
\pgfpathmoveto{\pgfqpoint{4.553085in}{3.023307in}}%
\pgfpathlineto{\pgfqpoint{5.342251in}{3.998981in}}%
\pgfusepath{stroke}%
\end{pgfscope}%
\begin{pgfscope}%
\pgfpathrectangle{\pgfqpoint{0.481978in}{0.331635in}}{\pgfqpoint{9.300000in}{7.700000in}}%
\pgfusepath{clip}%
\pgfsetrectcap%
\pgfsetroundjoin%
\pgfsetlinewidth{1.505625pt}%
\definecolor{currentstroke}{rgb}{1.000000,0.623529,0.607843}%
\pgfsetstrokecolor{currentstroke}%
\pgfsetstrokeopacity{0.200000}%
\pgfsetdash{}{0pt}%
\pgfpathmoveto{\pgfqpoint{9.095716in}{5.890233in}}%
\pgfpathlineto{\pgfqpoint{5.342251in}{3.998981in}}%
\pgfusepath{stroke}%
\end{pgfscope}%
\begin{pgfscope}%
\pgfpathrectangle{\pgfqpoint{0.481978in}{0.331635in}}{\pgfqpoint{9.300000in}{7.700000in}}%
\pgfusepath{clip}%
\pgfsetrectcap%
\pgfsetroundjoin%
\pgfsetlinewidth{1.505625pt}%
\definecolor{currentstroke}{rgb}{1.000000,0.623529,0.607843}%
\pgfsetstrokecolor{currentstroke}%
\pgfsetstrokeopacity{0.200000}%
\pgfsetdash{}{0pt}%
\pgfpathmoveto{\pgfqpoint{2.403697in}{3.472029in}}%
\pgfpathlineto{\pgfqpoint{5.342251in}{3.998981in}}%
\pgfusepath{stroke}%
\end{pgfscope}%
\begin{pgfscope}%
\pgfpathrectangle{\pgfqpoint{0.481978in}{0.331635in}}{\pgfqpoint{9.300000in}{7.700000in}}%
\pgfusepath{clip}%
\pgfsetrectcap%
\pgfsetroundjoin%
\pgfsetlinewidth{1.505625pt}%
\definecolor{currentstroke}{rgb}{1.000000,0.623529,0.607843}%
\pgfsetstrokecolor{currentstroke}%
\pgfsetstrokeopacity{0.200000}%
\pgfsetdash{}{0pt}%
\pgfpathmoveto{\pgfqpoint{7.815154in}{4.452079in}}%
\pgfpathlineto{\pgfqpoint{5.342251in}{3.998981in}}%
\pgfusepath{stroke}%
\end{pgfscope}%
\begin{pgfscope}%
\pgfpathrectangle{\pgfqpoint{0.481978in}{0.331635in}}{\pgfqpoint{9.300000in}{7.700000in}}%
\pgfusepath{clip}%
\pgfsetrectcap%
\pgfsetroundjoin%
\pgfsetlinewidth{1.505625pt}%
\definecolor{currentstroke}{rgb}{1.000000,0.623529,0.607843}%
\pgfsetstrokecolor{currentstroke}%
\pgfsetstrokeopacity{0.200000}%
\pgfsetdash{}{0pt}%
\pgfpathmoveto{\pgfqpoint{4.473626in}{2.024232in}}%
\pgfpathlineto{\pgfqpoint{5.342251in}{3.998981in}}%
\pgfusepath{stroke}%
\end{pgfscope}%
\begin{pgfscope}%
\pgfpathrectangle{\pgfqpoint{0.481978in}{0.331635in}}{\pgfqpoint{9.300000in}{7.700000in}}%
\pgfusepath{clip}%
\pgfsetrectcap%
\pgfsetroundjoin%
\pgfsetlinewidth{1.505625pt}%
\definecolor{currentstroke}{rgb}{0.815686,0.733333,1.000000}%
\pgfsetstrokecolor{currentstroke}%
\pgfsetstrokeopacity{0.800000}%
\pgfsetdash{}{0pt}%
\pgfpathmoveto{\pgfqpoint{7.152848in}{5.392946in}}%
\pgfpathlineto{\pgfqpoint{5.293604in}{4.165376in}}%
\pgfusepath{stroke}%
\end{pgfscope}%
\begin{pgfscope}%
\pgfpathrectangle{\pgfqpoint{0.481978in}{0.331635in}}{\pgfqpoint{9.300000in}{7.700000in}}%
\pgfusepath{clip}%
\pgfsetrectcap%
\pgfsetroundjoin%
\pgfsetlinewidth{1.505625pt}%
\definecolor{currentstroke}{rgb}{0.815686,0.733333,1.000000}%
\pgfsetstrokecolor{currentstroke}%
\pgfsetstrokeopacity{0.800000}%
\pgfsetdash{}{0pt}%
\pgfpathmoveto{\pgfqpoint{7.702166in}{5.107203in}}%
\pgfpathlineto{\pgfqpoint{5.293604in}{4.165376in}}%
\pgfusepath{stroke}%
\end{pgfscope}%
\begin{pgfscope}%
\pgfpathrectangle{\pgfqpoint{0.481978in}{0.331635in}}{\pgfqpoint{9.300000in}{7.700000in}}%
\pgfusepath{clip}%
\pgfsetrectcap%
\pgfsetroundjoin%
\pgfsetlinewidth{1.505625pt}%
\definecolor{currentstroke}{rgb}{0.815686,0.733333,1.000000}%
\pgfsetstrokecolor{currentstroke}%
\pgfsetstrokeopacity{0.800000}%
\pgfsetdash{}{0pt}%
\pgfpathmoveto{\pgfqpoint{2.872669in}{3.217019in}}%
\pgfpathlineto{\pgfqpoint{5.293604in}{4.165376in}}%
\pgfusepath{stroke}%
\end{pgfscope}%
\begin{pgfscope}%
\pgfpathrectangle{\pgfqpoint{0.481978in}{0.331635in}}{\pgfqpoint{9.300000in}{7.700000in}}%
\pgfusepath{clip}%
\pgfsetrectcap%
\pgfsetroundjoin%
\pgfsetlinewidth{1.505625pt}%
\definecolor{currentstroke}{rgb}{0.815686,0.733333,1.000000}%
\pgfsetstrokecolor{currentstroke}%
\pgfsetstrokeopacity{0.800000}%
\pgfsetdash{}{0pt}%
\pgfpathmoveto{\pgfqpoint{3.619377in}{2.362886in}}%
\pgfpathlineto{\pgfqpoint{5.293604in}{4.165376in}}%
\pgfusepath{stroke}%
\end{pgfscope}%
\begin{pgfscope}%
\pgfpathrectangle{\pgfqpoint{0.481978in}{0.331635in}}{\pgfqpoint{9.300000in}{7.700000in}}%
\pgfusepath{clip}%
\pgfsetrectcap%
\pgfsetroundjoin%
\pgfsetlinewidth{1.505625pt}%
\definecolor{currentstroke}{rgb}{0.815686,0.733333,1.000000}%
\pgfsetstrokecolor{currentstroke}%
\pgfsetstrokeopacity{0.800000}%
\pgfsetdash{}{0pt}%
\pgfpathmoveto{\pgfqpoint{3.448106in}{2.512230in}}%
\pgfpathlineto{\pgfqpoint{5.293604in}{4.165376in}}%
\pgfusepath{stroke}%
\end{pgfscope}%
\begin{pgfscope}%
\pgfpathrectangle{\pgfqpoint{0.481978in}{0.331635in}}{\pgfqpoint{9.300000in}{7.700000in}}%
\pgfusepath{clip}%
\pgfsetrectcap%
\pgfsetroundjoin%
\pgfsetlinewidth{1.505625pt}%
\definecolor{currentstroke}{rgb}{0.815686,0.733333,1.000000}%
\pgfsetstrokecolor{currentstroke}%
\pgfsetstrokeopacity{0.800000}%
\pgfsetdash{}{0pt}%
\pgfpathmoveto{\pgfqpoint{6.968468in}{3.346064in}}%
\pgfpathlineto{\pgfqpoint{5.293604in}{4.165376in}}%
\pgfusepath{stroke}%
\end{pgfscope}%
\begin{pgfscope}%
\pgfpathrectangle{\pgfqpoint{0.481978in}{0.331635in}}{\pgfqpoint{9.300000in}{7.700000in}}%
\pgfusepath{clip}%
\pgfsetrectcap%
\pgfsetroundjoin%
\pgfsetlinewidth{1.505625pt}%
\definecolor{currentstroke}{rgb}{0.815686,0.733333,1.000000}%
\pgfsetstrokecolor{currentstroke}%
\pgfsetstrokeopacity{0.800000}%
\pgfsetdash{}{0pt}%
\pgfpathmoveto{\pgfqpoint{3.260742in}{7.031269in}}%
\pgfpathlineto{\pgfqpoint{5.293604in}{4.165376in}}%
\pgfusepath{stroke}%
\end{pgfscope}%
\begin{pgfscope}%
\pgfpathrectangle{\pgfqpoint{0.481978in}{0.331635in}}{\pgfqpoint{9.300000in}{7.700000in}}%
\pgfusepath{clip}%
\pgfsetrectcap%
\pgfsetroundjoin%
\pgfsetlinewidth{1.505625pt}%
\definecolor{currentstroke}{rgb}{0.815686,0.733333,1.000000}%
\pgfsetstrokecolor{currentstroke}%
\pgfsetstrokeopacity{0.800000}%
\pgfsetdash{}{0pt}%
\pgfpathmoveto{\pgfqpoint{3.172404in}{5.775014in}}%
\pgfpathlineto{\pgfqpoint{5.293604in}{4.165376in}}%
\pgfusepath{stroke}%
\end{pgfscope}%
\begin{pgfscope}%
\pgfpathrectangle{\pgfqpoint{0.481978in}{0.331635in}}{\pgfqpoint{9.300000in}{7.700000in}}%
\pgfusepath{clip}%
\pgfsetrectcap%
\pgfsetroundjoin%
\pgfsetlinewidth{1.505625pt}%
\definecolor{currentstroke}{rgb}{0.815686,0.733333,1.000000}%
\pgfsetstrokecolor{currentstroke}%
\pgfsetstrokeopacity{0.800000}%
\pgfsetdash{}{0pt}%
\pgfpathmoveto{\pgfqpoint{2.953247in}{2.700897in}}%
\pgfpathlineto{\pgfqpoint{5.293604in}{4.165376in}}%
\pgfusepath{stroke}%
\end{pgfscope}%
\begin{pgfscope}%
\pgfpathrectangle{\pgfqpoint{0.481978in}{0.331635in}}{\pgfqpoint{9.300000in}{7.700000in}}%
\pgfusepath{clip}%
\pgfsetrectcap%
\pgfsetroundjoin%
\pgfsetlinewidth{1.505625pt}%
\definecolor{currentstroke}{rgb}{0.815686,0.733333,1.000000}%
\pgfsetstrokecolor{currentstroke}%
\pgfsetstrokeopacity{0.800000}%
\pgfsetdash{}{0pt}%
\pgfpathmoveto{\pgfqpoint{8.116420in}{5.616284in}}%
\pgfpathlineto{\pgfqpoint{5.293604in}{4.165376in}}%
\pgfusepath{stroke}%
\end{pgfscope}%
\begin{pgfscope}%
\pgfpathrectangle{\pgfqpoint{0.481978in}{0.331635in}}{\pgfqpoint{9.300000in}{7.700000in}}%
\pgfusepath{clip}%
\pgfsetrectcap%
\pgfsetroundjoin%
\pgfsetlinewidth{1.505625pt}%
\definecolor{currentstroke}{rgb}{0.815686,0.733333,1.000000}%
\pgfsetstrokecolor{currentstroke}%
\pgfsetstrokeopacity{0.800000}%
\pgfsetdash{}{0pt}%
\pgfpathmoveto{\pgfqpoint{6.906230in}{4.730496in}}%
\pgfpathlineto{\pgfqpoint{5.293604in}{4.165376in}}%
\pgfusepath{stroke}%
\end{pgfscope}%
\begin{pgfscope}%
\pgfpathrectangle{\pgfqpoint{0.481978in}{0.331635in}}{\pgfqpoint{9.300000in}{7.700000in}}%
\pgfusepath{clip}%
\pgfsetrectcap%
\pgfsetroundjoin%
\pgfsetlinewidth{1.505625pt}%
\definecolor{currentstroke}{rgb}{0.815686,0.733333,1.000000}%
\pgfsetstrokecolor{currentstroke}%
\pgfsetstrokeopacity{0.800000}%
\pgfsetdash{}{0pt}%
\pgfpathmoveto{\pgfqpoint{3.815826in}{4.087628in}}%
\pgfpathlineto{\pgfqpoint{5.293604in}{4.165376in}}%
\pgfusepath{stroke}%
\end{pgfscope}%
\begin{pgfscope}%
\pgfpathrectangle{\pgfqpoint{0.481978in}{0.331635in}}{\pgfqpoint{9.300000in}{7.700000in}}%
\pgfusepath{clip}%
\pgfsetrectcap%
\pgfsetroundjoin%
\pgfsetlinewidth{1.505625pt}%
\definecolor{currentstroke}{rgb}{0.815686,0.733333,1.000000}%
\pgfsetstrokecolor{currentstroke}%
\pgfsetstrokeopacity{0.800000}%
\pgfsetdash{}{0pt}%
\pgfpathmoveto{\pgfqpoint{6.962413in}{5.802045in}}%
\pgfpathlineto{\pgfqpoint{5.293604in}{4.165376in}}%
\pgfusepath{stroke}%
\end{pgfscope}%
\begin{pgfscope}%
\pgfpathrectangle{\pgfqpoint{0.481978in}{0.331635in}}{\pgfqpoint{9.300000in}{7.700000in}}%
\pgfusepath{clip}%
\pgfsetrectcap%
\pgfsetroundjoin%
\pgfsetlinewidth{1.505625pt}%
\definecolor{currentstroke}{rgb}{0.815686,0.733333,1.000000}%
\pgfsetstrokecolor{currentstroke}%
\pgfsetstrokeopacity{0.800000}%
\pgfsetdash{}{0pt}%
\pgfpathmoveto{\pgfqpoint{7.236127in}{5.393044in}}%
\pgfpathlineto{\pgfqpoint{5.293604in}{4.165376in}}%
\pgfusepath{stroke}%
\end{pgfscope}%
\begin{pgfscope}%
\pgfpathrectangle{\pgfqpoint{0.481978in}{0.331635in}}{\pgfqpoint{9.300000in}{7.700000in}}%
\pgfusepath{clip}%
\pgfsetrectcap%
\pgfsetroundjoin%
\pgfsetlinewidth{1.505625pt}%
\definecolor{currentstroke}{rgb}{0.815686,0.733333,1.000000}%
\pgfsetstrokecolor{currentstroke}%
\pgfsetstrokeopacity{0.800000}%
\pgfsetdash{}{0pt}%
\pgfpathmoveto{\pgfqpoint{8.116846in}{6.524386in}}%
\pgfpathlineto{\pgfqpoint{5.293604in}{4.165376in}}%
\pgfusepath{stroke}%
\end{pgfscope}%
\begin{pgfscope}%
\pgfpathrectangle{\pgfqpoint{0.481978in}{0.331635in}}{\pgfqpoint{9.300000in}{7.700000in}}%
\pgfusepath{clip}%
\pgfsetrectcap%
\pgfsetroundjoin%
\pgfsetlinewidth{1.505625pt}%
\definecolor{currentstroke}{rgb}{0.815686,0.733333,1.000000}%
\pgfsetstrokecolor{currentstroke}%
\pgfsetstrokeopacity{0.800000}%
\pgfsetdash{}{0pt}%
\pgfpathmoveto{\pgfqpoint{3.449504in}{2.766701in}}%
\pgfpathlineto{\pgfqpoint{5.293604in}{4.165376in}}%
\pgfusepath{stroke}%
\end{pgfscope}%
\begin{pgfscope}%
\pgfpathrectangle{\pgfqpoint{0.481978in}{0.331635in}}{\pgfqpoint{9.300000in}{7.700000in}}%
\pgfusepath{clip}%
\pgfsetrectcap%
\pgfsetroundjoin%
\pgfsetlinewidth{1.505625pt}%
\definecolor{currentstroke}{rgb}{0.815686,0.733333,1.000000}%
\pgfsetstrokecolor{currentstroke}%
\pgfsetstrokeopacity{0.800000}%
\pgfsetdash{}{0pt}%
\pgfpathmoveto{\pgfqpoint{4.947086in}{5.421637in}}%
\pgfpathlineto{\pgfqpoint{5.293604in}{4.165376in}}%
\pgfusepath{stroke}%
\end{pgfscope}%
\begin{pgfscope}%
\pgfpathrectangle{\pgfqpoint{0.481978in}{0.331635in}}{\pgfqpoint{9.300000in}{7.700000in}}%
\pgfusepath{clip}%
\pgfsetrectcap%
\pgfsetroundjoin%
\pgfsetlinewidth{1.505625pt}%
\definecolor{currentstroke}{rgb}{0.815686,0.733333,1.000000}%
\pgfsetstrokecolor{currentstroke}%
\pgfsetstrokeopacity{0.800000}%
\pgfsetdash{}{0pt}%
\pgfpathmoveto{\pgfqpoint{6.925838in}{2.915473in}}%
\pgfpathlineto{\pgfqpoint{5.293604in}{4.165376in}}%
\pgfusepath{stroke}%
\end{pgfscope}%
\begin{pgfscope}%
\pgfpathrectangle{\pgfqpoint{0.481978in}{0.331635in}}{\pgfqpoint{9.300000in}{7.700000in}}%
\pgfusepath{clip}%
\pgfsetrectcap%
\pgfsetroundjoin%
\pgfsetlinewidth{1.505625pt}%
\definecolor{currentstroke}{rgb}{0.815686,0.733333,1.000000}%
\pgfsetstrokecolor{currentstroke}%
\pgfsetstrokeopacity{0.800000}%
\pgfsetdash{}{0pt}%
\pgfpathmoveto{\pgfqpoint{7.338785in}{3.335035in}}%
\pgfpathlineto{\pgfqpoint{5.293604in}{4.165376in}}%
\pgfusepath{stroke}%
\end{pgfscope}%
\begin{pgfscope}%
\pgfpathrectangle{\pgfqpoint{0.481978in}{0.331635in}}{\pgfqpoint{9.300000in}{7.700000in}}%
\pgfusepath{clip}%
\pgfsetrectcap%
\pgfsetroundjoin%
\pgfsetlinewidth{1.505625pt}%
\definecolor{currentstroke}{rgb}{0.815686,0.733333,1.000000}%
\pgfsetstrokecolor{currentstroke}%
\pgfsetstrokeopacity{0.800000}%
\pgfsetdash{}{0pt}%
\pgfpathmoveto{\pgfqpoint{3.955717in}{4.134945in}}%
\pgfpathlineto{\pgfqpoint{5.293604in}{4.165376in}}%
\pgfusepath{stroke}%
\end{pgfscope}%
\begin{pgfscope}%
\pgfpathrectangle{\pgfqpoint{0.481978in}{0.331635in}}{\pgfqpoint{9.300000in}{7.700000in}}%
\pgfusepath{clip}%
\pgfsetrectcap%
\pgfsetroundjoin%
\pgfsetlinewidth{1.505625pt}%
\definecolor{currentstroke}{rgb}{0.815686,0.733333,1.000000}%
\pgfsetstrokecolor{currentstroke}%
\pgfsetstrokeopacity{0.800000}%
\pgfsetdash{}{0pt}%
\pgfpathmoveto{\pgfqpoint{2.160067in}{1.924338in}}%
\pgfpathlineto{\pgfqpoint{5.293604in}{4.165376in}}%
\pgfusepath{stroke}%
\end{pgfscope}%
\begin{pgfscope}%
\pgfpathrectangle{\pgfqpoint{0.481978in}{0.331635in}}{\pgfqpoint{9.300000in}{7.700000in}}%
\pgfusepath{clip}%
\pgfsetrectcap%
\pgfsetroundjoin%
\pgfsetlinewidth{1.505625pt}%
\definecolor{currentstroke}{rgb}{0.815686,0.733333,1.000000}%
\pgfsetstrokecolor{currentstroke}%
\pgfsetstrokeopacity{0.800000}%
\pgfsetdash{}{0pt}%
\pgfpathmoveto{\pgfqpoint{4.198073in}{4.262822in}}%
\pgfpathlineto{\pgfqpoint{5.293604in}{4.165376in}}%
\pgfusepath{stroke}%
\end{pgfscope}%
\begin{pgfscope}%
\pgfpathrectangle{\pgfqpoint{0.481978in}{0.331635in}}{\pgfqpoint{9.300000in}{7.700000in}}%
\pgfusepath{clip}%
\pgfsetrectcap%
\pgfsetroundjoin%
\pgfsetlinewidth{1.505625pt}%
\definecolor{currentstroke}{rgb}{0.815686,0.733333,1.000000}%
\pgfsetstrokecolor{currentstroke}%
\pgfsetstrokeopacity{0.800000}%
\pgfsetdash{}{0pt}%
\pgfpathmoveto{\pgfqpoint{3.029876in}{4.931933in}}%
\pgfpathlineto{\pgfqpoint{5.293604in}{4.165376in}}%
\pgfusepath{stroke}%
\end{pgfscope}%
\begin{pgfscope}%
\pgfpathrectangle{\pgfqpoint{0.481978in}{0.331635in}}{\pgfqpoint{9.300000in}{7.700000in}}%
\pgfusepath{clip}%
\pgfsetrectcap%
\pgfsetroundjoin%
\pgfsetlinewidth{1.505625pt}%
\definecolor{currentstroke}{rgb}{0.815686,0.733333,1.000000}%
\pgfsetstrokecolor{currentstroke}%
\pgfsetstrokeopacity{0.800000}%
\pgfsetdash{}{0pt}%
\pgfpathmoveto{\pgfqpoint{1.310658in}{4.258773in}}%
\pgfpathlineto{\pgfqpoint{5.293604in}{4.165376in}}%
\pgfusepath{stroke}%
\end{pgfscope}%
\begin{pgfscope}%
\pgfpathrectangle{\pgfqpoint{0.481978in}{0.331635in}}{\pgfqpoint{9.300000in}{7.700000in}}%
\pgfusepath{clip}%
\pgfsetrectcap%
\pgfsetroundjoin%
\pgfsetlinewidth{1.505625pt}%
\definecolor{currentstroke}{rgb}{0.815686,0.733333,1.000000}%
\pgfsetstrokecolor{currentstroke}%
\pgfsetstrokeopacity{0.800000}%
\pgfsetdash{}{0pt}%
\pgfpathmoveto{\pgfqpoint{5.061821in}{4.710122in}}%
\pgfpathlineto{\pgfqpoint{5.293604in}{4.165376in}}%
\pgfusepath{stroke}%
\end{pgfscope}%
\begin{pgfscope}%
\pgfpathrectangle{\pgfqpoint{0.481978in}{0.331635in}}{\pgfqpoint{9.300000in}{7.700000in}}%
\pgfusepath{clip}%
\pgfsetrectcap%
\pgfsetroundjoin%
\pgfsetlinewidth{1.505625pt}%
\definecolor{currentstroke}{rgb}{0.815686,0.733333,1.000000}%
\pgfsetstrokecolor{currentstroke}%
\pgfsetstrokeopacity{0.800000}%
\pgfsetdash{}{0pt}%
\pgfpathmoveto{\pgfqpoint{5.046020in}{5.054040in}}%
\pgfpathlineto{\pgfqpoint{5.293604in}{4.165376in}}%
\pgfusepath{stroke}%
\end{pgfscope}%
\begin{pgfscope}%
\pgfpathrectangle{\pgfqpoint{0.481978in}{0.331635in}}{\pgfqpoint{9.300000in}{7.700000in}}%
\pgfusepath{clip}%
\pgfsetrectcap%
\pgfsetroundjoin%
\pgfsetlinewidth{1.505625pt}%
\definecolor{currentstroke}{rgb}{0.815686,0.733333,1.000000}%
\pgfsetstrokecolor{currentstroke}%
\pgfsetstrokeopacity{0.800000}%
\pgfsetdash{}{0pt}%
\pgfpathmoveto{\pgfqpoint{7.569943in}{3.679389in}}%
\pgfpathlineto{\pgfqpoint{5.293604in}{4.165376in}}%
\pgfusepath{stroke}%
\end{pgfscope}%
\begin{pgfscope}%
\pgfpathrectangle{\pgfqpoint{0.481978in}{0.331635in}}{\pgfqpoint{9.300000in}{7.700000in}}%
\pgfusepath{clip}%
\pgfsetrectcap%
\pgfsetroundjoin%
\pgfsetlinewidth{1.505625pt}%
\definecolor{currentstroke}{rgb}{0.815686,0.733333,1.000000}%
\pgfsetstrokecolor{currentstroke}%
\pgfsetstrokeopacity{0.800000}%
\pgfsetdash{}{0pt}%
\pgfpathmoveto{\pgfqpoint{7.674879in}{3.075539in}}%
\pgfpathlineto{\pgfqpoint{5.293604in}{4.165376in}}%
\pgfusepath{stroke}%
\end{pgfscope}%
\begin{pgfscope}%
\pgfpathrectangle{\pgfqpoint{0.481978in}{0.331635in}}{\pgfqpoint{9.300000in}{7.700000in}}%
\pgfusepath{clip}%
\pgfsetrectcap%
\pgfsetroundjoin%
\pgfsetlinewidth{1.505625pt}%
\definecolor{currentstroke}{rgb}{0.815686,0.733333,1.000000}%
\pgfsetstrokecolor{currentstroke}%
\pgfsetstrokeopacity{0.800000}%
\pgfsetdash{}{0pt}%
\pgfpathmoveto{\pgfqpoint{5.428071in}{3.114634in}}%
\pgfpathlineto{\pgfqpoint{5.293604in}{4.165376in}}%
\pgfusepath{stroke}%
\end{pgfscope}%
\begin{pgfscope}%
\pgfpathrectangle{\pgfqpoint{0.481978in}{0.331635in}}{\pgfqpoint{9.300000in}{7.700000in}}%
\pgfusepath{clip}%
\pgfsetrectcap%
\pgfsetroundjoin%
\pgfsetlinewidth{1.505625pt}%
\definecolor{currentstroke}{rgb}{0.815686,0.733333,1.000000}%
\pgfsetstrokecolor{currentstroke}%
\pgfsetstrokeopacity{0.800000}%
\pgfsetdash{}{0pt}%
\pgfpathmoveto{\pgfqpoint{2.403350in}{3.643277in}}%
\pgfpathlineto{\pgfqpoint{5.293604in}{4.165376in}}%
\pgfusepath{stroke}%
\end{pgfscope}%
\begin{pgfscope}%
\pgfpathrectangle{\pgfqpoint{0.481978in}{0.331635in}}{\pgfqpoint{9.300000in}{7.700000in}}%
\pgfusepath{clip}%
\pgfsetrectcap%
\pgfsetroundjoin%
\pgfsetlinewidth{1.505625pt}%
\definecolor{currentstroke}{rgb}{0.815686,0.733333,1.000000}%
\pgfsetstrokecolor{currentstroke}%
\pgfsetstrokeopacity{0.800000}%
\pgfsetdash{}{0pt}%
\pgfpathmoveto{\pgfqpoint{6.041612in}{5.634142in}}%
\pgfpathlineto{\pgfqpoint{5.293604in}{4.165376in}}%
\pgfusepath{stroke}%
\end{pgfscope}%
\begin{pgfscope}%
\pgfpathrectangle{\pgfqpoint{0.481978in}{0.331635in}}{\pgfqpoint{9.300000in}{7.700000in}}%
\pgfusepath{clip}%
\pgfsetrectcap%
\pgfsetroundjoin%
\pgfsetlinewidth{1.505625pt}%
\definecolor{currentstroke}{rgb}{0.815686,0.733333,1.000000}%
\pgfsetstrokecolor{currentstroke}%
\pgfsetstrokeopacity{0.800000}%
\pgfsetdash{}{0pt}%
\pgfpathmoveto{\pgfqpoint{8.383007in}{6.203756in}}%
\pgfpathlineto{\pgfqpoint{5.293604in}{4.165376in}}%
\pgfusepath{stroke}%
\end{pgfscope}%
\begin{pgfscope}%
\pgfpathrectangle{\pgfqpoint{0.481978in}{0.331635in}}{\pgfqpoint{9.300000in}{7.700000in}}%
\pgfusepath{clip}%
\pgfsetrectcap%
\pgfsetroundjoin%
\pgfsetlinewidth{1.505625pt}%
\definecolor{currentstroke}{rgb}{0.815686,0.733333,1.000000}%
\pgfsetstrokecolor{currentstroke}%
\pgfsetstrokeopacity{0.800000}%
\pgfsetdash{}{0pt}%
\pgfpathmoveto{\pgfqpoint{4.398769in}{4.567491in}}%
\pgfpathlineto{\pgfqpoint{5.293604in}{4.165376in}}%
\pgfusepath{stroke}%
\end{pgfscope}%
\begin{pgfscope}%
\pgfpathrectangle{\pgfqpoint{0.481978in}{0.331635in}}{\pgfqpoint{9.300000in}{7.700000in}}%
\pgfusepath{clip}%
\pgfsetrectcap%
\pgfsetroundjoin%
\pgfsetlinewidth{1.505625pt}%
\definecolor{currentstroke}{rgb}{0.815686,0.733333,1.000000}%
\pgfsetstrokecolor{currentstroke}%
\pgfsetstrokeopacity{0.800000}%
\pgfsetdash{}{0pt}%
\pgfpathmoveto{\pgfqpoint{8.682490in}{5.819369in}}%
\pgfpathlineto{\pgfqpoint{5.293604in}{4.165376in}}%
\pgfusepath{stroke}%
\end{pgfscope}%
\begin{pgfscope}%
\pgfpathrectangle{\pgfqpoint{0.481978in}{0.331635in}}{\pgfqpoint{9.300000in}{7.700000in}}%
\pgfusepath{clip}%
\pgfsetrectcap%
\pgfsetroundjoin%
\pgfsetlinewidth{1.505625pt}%
\definecolor{currentstroke}{rgb}{0.815686,0.733333,1.000000}%
\pgfsetstrokecolor{currentstroke}%
\pgfsetstrokeopacity{0.800000}%
\pgfsetdash{}{0pt}%
\pgfpathmoveto{\pgfqpoint{5.664598in}{1.155031in}}%
\pgfpathlineto{\pgfqpoint{5.293604in}{4.165376in}}%
\pgfusepath{stroke}%
\end{pgfscope}%
\begin{pgfscope}%
\pgfpathrectangle{\pgfqpoint{0.481978in}{0.331635in}}{\pgfqpoint{9.300000in}{7.700000in}}%
\pgfusepath{clip}%
\pgfsetrectcap%
\pgfsetroundjoin%
\pgfsetlinewidth{1.505625pt}%
\definecolor{currentstroke}{rgb}{0.815686,0.733333,1.000000}%
\pgfsetstrokecolor{currentstroke}%
\pgfsetstrokeopacity{0.800000}%
\pgfsetdash{}{0pt}%
\pgfpathmoveto{\pgfqpoint{1.939227in}{4.353523in}}%
\pgfpathlineto{\pgfqpoint{5.293604in}{4.165376in}}%
\pgfusepath{stroke}%
\end{pgfscope}%
\begin{pgfscope}%
\pgfpathrectangle{\pgfqpoint{0.481978in}{0.331635in}}{\pgfqpoint{9.300000in}{7.700000in}}%
\pgfusepath{clip}%
\pgfsetrectcap%
\pgfsetroundjoin%
\pgfsetlinewidth{1.505625pt}%
\definecolor{currentstroke}{rgb}{0.815686,0.733333,1.000000}%
\pgfsetstrokecolor{currentstroke}%
\pgfsetstrokeopacity{0.800000}%
\pgfsetdash{}{0pt}%
\pgfpathmoveto{\pgfqpoint{6.777031in}{4.449477in}}%
\pgfpathlineto{\pgfqpoint{5.293604in}{4.165376in}}%
\pgfusepath{stroke}%
\end{pgfscope}%
\begin{pgfscope}%
\pgfpathrectangle{\pgfqpoint{0.481978in}{0.331635in}}{\pgfqpoint{9.300000in}{7.700000in}}%
\pgfusepath{clip}%
\pgfsetrectcap%
\pgfsetroundjoin%
\pgfsetlinewidth{1.505625pt}%
\definecolor{currentstroke}{rgb}{0.815686,0.733333,1.000000}%
\pgfsetstrokecolor{currentstroke}%
\pgfsetstrokeopacity{0.800000}%
\pgfsetdash{}{0pt}%
\pgfpathmoveto{\pgfqpoint{3.798724in}{2.165621in}}%
\pgfpathlineto{\pgfqpoint{5.293604in}{4.165376in}}%
\pgfusepath{stroke}%
\end{pgfscope}%
\begin{pgfscope}%
\pgfpathrectangle{\pgfqpoint{0.481978in}{0.331635in}}{\pgfqpoint{9.300000in}{7.700000in}}%
\pgfusepath{clip}%
\pgfsetrectcap%
\pgfsetroundjoin%
\pgfsetlinewidth{1.505625pt}%
\definecolor{currentstroke}{rgb}{0.815686,0.733333,1.000000}%
\pgfsetstrokecolor{currentstroke}%
\pgfsetstrokeopacity{0.800000}%
\pgfsetdash{}{0pt}%
\pgfpathmoveto{\pgfqpoint{1.738451in}{3.903724in}}%
\pgfpathlineto{\pgfqpoint{5.293604in}{4.165376in}}%
\pgfusepath{stroke}%
\end{pgfscope}%
\begin{pgfscope}%
\pgfpathrectangle{\pgfqpoint{0.481978in}{0.331635in}}{\pgfqpoint{9.300000in}{7.700000in}}%
\pgfusepath{clip}%
\pgfsetrectcap%
\pgfsetroundjoin%
\pgfsetlinewidth{1.505625pt}%
\definecolor{currentstroke}{rgb}{0.815686,0.733333,1.000000}%
\pgfsetstrokecolor{currentstroke}%
\pgfsetstrokeopacity{0.800000}%
\pgfsetdash{}{0pt}%
\pgfpathmoveto{\pgfqpoint{8.508088in}{4.005881in}}%
\pgfpathlineto{\pgfqpoint{5.293604in}{4.165376in}}%
\pgfusepath{stroke}%
\end{pgfscope}%
\begin{pgfscope}%
\pgfpathrectangle{\pgfqpoint{0.481978in}{0.331635in}}{\pgfqpoint{9.300000in}{7.700000in}}%
\pgfusepath{clip}%
\pgfsetrectcap%
\pgfsetroundjoin%
\pgfsetlinewidth{1.505625pt}%
\definecolor{currentstroke}{rgb}{0.815686,0.733333,1.000000}%
\pgfsetstrokecolor{currentstroke}%
\pgfsetstrokeopacity{0.800000}%
\pgfsetdash{}{0pt}%
\pgfpathmoveto{\pgfqpoint{6.837191in}{4.890982in}}%
\pgfpathlineto{\pgfqpoint{5.293604in}{4.165376in}}%
\pgfusepath{stroke}%
\end{pgfscope}%
\begin{pgfscope}%
\pgfpathrectangle{\pgfqpoint{0.481978in}{0.331635in}}{\pgfqpoint{9.300000in}{7.700000in}}%
\pgfusepath{clip}%
\pgfsetrectcap%
\pgfsetroundjoin%
\pgfsetlinewidth{1.505625pt}%
\definecolor{currentstroke}{rgb}{0.815686,0.733333,1.000000}%
\pgfsetstrokecolor{currentstroke}%
\pgfsetstrokeopacity{0.800000}%
\pgfsetdash{}{0pt}%
\pgfpathmoveto{\pgfqpoint{4.101155in}{3.759624in}}%
\pgfpathlineto{\pgfqpoint{5.293604in}{4.165376in}}%
\pgfusepath{stroke}%
\end{pgfscope}%
\begin{pgfscope}%
\pgfpathrectangle{\pgfqpoint{0.481978in}{0.331635in}}{\pgfqpoint{9.300000in}{7.700000in}}%
\pgfusepath{clip}%
\pgfsetrectcap%
\pgfsetroundjoin%
\pgfsetlinewidth{1.505625pt}%
\definecolor{currentstroke}{rgb}{0.815686,0.733333,1.000000}%
\pgfsetstrokecolor{currentstroke}%
\pgfsetstrokeopacity{0.800000}%
\pgfsetdash{}{0pt}%
\pgfpathmoveto{\pgfqpoint{7.340850in}{3.294690in}}%
\pgfpathlineto{\pgfqpoint{5.293604in}{4.165376in}}%
\pgfusepath{stroke}%
\end{pgfscope}%
\begin{pgfscope}%
\pgfpathrectangle{\pgfqpoint{0.481978in}{0.331635in}}{\pgfqpoint{9.300000in}{7.700000in}}%
\pgfusepath{clip}%
\pgfsetrectcap%
\pgfsetroundjoin%
\pgfsetlinewidth{1.505625pt}%
\definecolor{currentstroke}{rgb}{0.815686,0.733333,1.000000}%
\pgfsetstrokecolor{currentstroke}%
\pgfsetstrokeopacity{0.800000}%
\pgfsetdash{}{0pt}%
\pgfpathmoveto{\pgfqpoint{4.878178in}{5.104000in}}%
\pgfpathlineto{\pgfqpoint{5.293604in}{4.165376in}}%
\pgfusepath{stroke}%
\end{pgfscope}%
\begin{pgfscope}%
\pgfpathrectangle{\pgfqpoint{0.481978in}{0.331635in}}{\pgfqpoint{9.300000in}{7.700000in}}%
\pgfusepath{clip}%
\pgfsetrectcap%
\pgfsetroundjoin%
\pgfsetlinewidth{1.505625pt}%
\definecolor{currentstroke}{rgb}{0.815686,0.733333,1.000000}%
\pgfsetstrokecolor{currentstroke}%
\pgfsetstrokeopacity{0.800000}%
\pgfsetdash{}{0pt}%
\pgfpathmoveto{\pgfqpoint{3.093874in}{4.888839in}}%
\pgfpathlineto{\pgfqpoint{5.293604in}{4.165376in}}%
\pgfusepath{stroke}%
\end{pgfscope}%
\begin{pgfscope}%
\pgfpathrectangle{\pgfqpoint{0.481978in}{0.331635in}}{\pgfqpoint{9.300000in}{7.700000in}}%
\pgfusepath{clip}%
\pgfsetrectcap%
\pgfsetroundjoin%
\pgfsetlinewidth{1.505625pt}%
\definecolor{currentstroke}{rgb}{0.815686,0.733333,1.000000}%
\pgfsetstrokecolor{currentstroke}%
\pgfsetstrokeopacity{0.800000}%
\pgfsetdash{}{0pt}%
\pgfpathmoveto{\pgfqpoint{5.273114in}{1.907651in}}%
\pgfpathlineto{\pgfqpoint{5.293604in}{4.165376in}}%
\pgfusepath{stroke}%
\end{pgfscope}%
\begin{pgfscope}%
\pgfpathrectangle{\pgfqpoint{0.481978in}{0.331635in}}{\pgfqpoint{9.300000in}{7.700000in}}%
\pgfusepath{clip}%
\pgfsetrectcap%
\pgfsetroundjoin%
\pgfsetlinewidth{1.505625pt}%
\definecolor{currentstroke}{rgb}{0.815686,0.733333,1.000000}%
\pgfsetstrokecolor{currentstroke}%
\pgfsetstrokeopacity{0.800000}%
\pgfsetdash{}{0pt}%
\pgfpathmoveto{\pgfqpoint{8.330701in}{5.648832in}}%
\pgfpathlineto{\pgfqpoint{5.293604in}{4.165376in}}%
\pgfusepath{stroke}%
\end{pgfscope}%
\begin{pgfscope}%
\pgfpathrectangle{\pgfqpoint{0.481978in}{0.331635in}}{\pgfqpoint{9.300000in}{7.700000in}}%
\pgfusepath{clip}%
\pgfsetrectcap%
\pgfsetroundjoin%
\pgfsetlinewidth{1.505625pt}%
\definecolor{currentstroke}{rgb}{0.815686,0.733333,1.000000}%
\pgfsetstrokecolor{currentstroke}%
\pgfsetstrokeopacity{0.800000}%
\pgfsetdash{}{0pt}%
\pgfpathmoveto{\pgfqpoint{7.364685in}{2.940268in}}%
\pgfpathlineto{\pgfqpoint{5.293604in}{4.165376in}}%
\pgfusepath{stroke}%
\end{pgfscope}%
\begin{pgfscope}%
\pgfpathrectangle{\pgfqpoint{0.481978in}{0.331635in}}{\pgfqpoint{9.300000in}{7.700000in}}%
\pgfusepath{clip}%
\pgfsetrectcap%
\pgfsetroundjoin%
\pgfsetlinewidth{1.505625pt}%
\definecolor{currentstroke}{rgb}{0.815686,0.733333,1.000000}%
\pgfsetstrokecolor{currentstroke}%
\pgfsetstrokeopacity{0.800000}%
\pgfsetdash{}{0pt}%
\pgfpathmoveto{\pgfqpoint{2.332471in}{2.516497in}}%
\pgfpathlineto{\pgfqpoint{5.293604in}{4.165376in}}%
\pgfusepath{stroke}%
\end{pgfscope}%
\begin{pgfscope}%
\pgfpathrectangle{\pgfqpoint{0.481978in}{0.331635in}}{\pgfqpoint{9.300000in}{7.700000in}}%
\pgfusepath{clip}%
\pgfsetrectcap%
\pgfsetroundjoin%
\pgfsetlinewidth{1.505625pt}%
\definecolor{currentstroke}{rgb}{0.815686,0.733333,1.000000}%
\pgfsetstrokecolor{currentstroke}%
\pgfsetstrokeopacity{0.800000}%
\pgfsetdash{}{0pt}%
\pgfpathmoveto{\pgfqpoint{6.392422in}{4.231341in}}%
\pgfpathlineto{\pgfqpoint{5.293604in}{4.165376in}}%
\pgfusepath{stroke}%
\end{pgfscope}%
\begin{pgfscope}%
\pgfpathrectangle{\pgfqpoint{0.481978in}{0.331635in}}{\pgfqpoint{9.300000in}{7.700000in}}%
\pgfusepath{clip}%
\pgfsetrectcap%
\pgfsetroundjoin%
\pgfsetlinewidth{1.505625pt}%
\definecolor{currentstroke}{rgb}{0.870588,0.733333,0.607843}%
\pgfsetstrokecolor{currentstroke}%
\pgfsetstrokeopacity{0.800000}%
\pgfsetdash{}{0pt}%
\pgfpathmoveto{\pgfqpoint{3.278906in}{4.376840in}}%
\pgfpathlineto{\pgfqpoint{4.473902in}{4.321654in}}%
\pgfusepath{stroke}%
\end{pgfscope}%
\begin{pgfscope}%
\pgfpathrectangle{\pgfqpoint{0.481978in}{0.331635in}}{\pgfqpoint{9.300000in}{7.700000in}}%
\pgfusepath{clip}%
\pgfsetrectcap%
\pgfsetroundjoin%
\pgfsetlinewidth{1.505625pt}%
\definecolor{currentstroke}{rgb}{0.870588,0.733333,0.607843}%
\pgfsetstrokecolor{currentstroke}%
\pgfsetstrokeopacity{0.800000}%
\pgfsetdash{}{0pt}%
\pgfpathmoveto{\pgfqpoint{1.806406in}{3.514068in}}%
\pgfpathlineto{\pgfqpoint{4.473902in}{4.321654in}}%
\pgfusepath{stroke}%
\end{pgfscope}%
\begin{pgfscope}%
\pgfpathrectangle{\pgfqpoint{0.481978in}{0.331635in}}{\pgfqpoint{9.300000in}{7.700000in}}%
\pgfusepath{clip}%
\pgfsetrectcap%
\pgfsetroundjoin%
\pgfsetlinewidth{1.505625pt}%
\definecolor{currentstroke}{rgb}{0.870588,0.733333,0.607843}%
\pgfsetstrokecolor{currentstroke}%
\pgfsetstrokeopacity{0.800000}%
\pgfsetdash{}{0pt}%
\pgfpathmoveto{\pgfqpoint{3.901522in}{2.752340in}}%
\pgfpathlineto{\pgfqpoint{4.473902in}{4.321654in}}%
\pgfusepath{stroke}%
\end{pgfscope}%
\begin{pgfscope}%
\pgfpathrectangle{\pgfqpoint{0.481978in}{0.331635in}}{\pgfqpoint{9.300000in}{7.700000in}}%
\pgfusepath{clip}%
\pgfsetrectcap%
\pgfsetroundjoin%
\pgfsetlinewidth{1.505625pt}%
\definecolor{currentstroke}{rgb}{0.870588,0.733333,0.607843}%
\pgfsetstrokecolor{currentstroke}%
\pgfsetstrokeopacity{0.800000}%
\pgfsetdash{}{0pt}%
\pgfpathmoveto{\pgfqpoint{2.577647in}{1.961490in}}%
\pgfpathlineto{\pgfqpoint{4.473902in}{4.321654in}}%
\pgfusepath{stroke}%
\end{pgfscope}%
\begin{pgfscope}%
\pgfpathrectangle{\pgfqpoint{0.481978in}{0.331635in}}{\pgfqpoint{9.300000in}{7.700000in}}%
\pgfusepath{clip}%
\pgfsetrectcap%
\pgfsetroundjoin%
\pgfsetlinewidth{1.505625pt}%
\definecolor{currentstroke}{rgb}{0.870588,0.733333,0.607843}%
\pgfsetstrokecolor{currentstroke}%
\pgfsetstrokeopacity{0.800000}%
\pgfsetdash{}{0pt}%
\pgfpathmoveto{\pgfqpoint{2.542882in}{4.429625in}}%
\pgfpathlineto{\pgfqpoint{4.473902in}{4.321654in}}%
\pgfusepath{stroke}%
\end{pgfscope}%
\begin{pgfscope}%
\pgfpathrectangle{\pgfqpoint{0.481978in}{0.331635in}}{\pgfqpoint{9.300000in}{7.700000in}}%
\pgfusepath{clip}%
\pgfsetrectcap%
\pgfsetroundjoin%
\pgfsetlinewidth{1.505625pt}%
\definecolor{currentstroke}{rgb}{0.870588,0.733333,0.607843}%
\pgfsetstrokecolor{currentstroke}%
\pgfsetstrokeopacity{0.800000}%
\pgfsetdash{}{0pt}%
\pgfpathmoveto{\pgfqpoint{6.100140in}{5.077934in}}%
\pgfpathlineto{\pgfqpoint{4.473902in}{4.321654in}}%
\pgfusepath{stroke}%
\end{pgfscope}%
\begin{pgfscope}%
\pgfpathrectangle{\pgfqpoint{0.481978in}{0.331635in}}{\pgfqpoint{9.300000in}{7.700000in}}%
\pgfusepath{clip}%
\pgfsetrectcap%
\pgfsetroundjoin%
\pgfsetlinewidth{1.505625pt}%
\definecolor{currentstroke}{rgb}{0.870588,0.733333,0.607843}%
\pgfsetstrokecolor{currentstroke}%
\pgfsetstrokeopacity{0.800000}%
\pgfsetdash{}{0pt}%
\pgfpathmoveto{\pgfqpoint{4.111074in}{4.571818in}}%
\pgfpathlineto{\pgfqpoint{4.473902in}{4.321654in}}%
\pgfusepath{stroke}%
\end{pgfscope}%
\begin{pgfscope}%
\pgfpathrectangle{\pgfqpoint{0.481978in}{0.331635in}}{\pgfqpoint{9.300000in}{7.700000in}}%
\pgfusepath{clip}%
\pgfsetrectcap%
\pgfsetroundjoin%
\pgfsetlinewidth{1.505625pt}%
\definecolor{currentstroke}{rgb}{0.870588,0.733333,0.607843}%
\pgfsetstrokecolor{currentstroke}%
\pgfsetstrokeopacity{0.800000}%
\pgfsetdash{}{0pt}%
\pgfpathmoveto{\pgfqpoint{2.567413in}{4.083544in}}%
\pgfpathlineto{\pgfqpoint{4.473902in}{4.321654in}}%
\pgfusepath{stroke}%
\end{pgfscope}%
\begin{pgfscope}%
\pgfpathrectangle{\pgfqpoint{0.481978in}{0.331635in}}{\pgfqpoint{9.300000in}{7.700000in}}%
\pgfusepath{clip}%
\pgfsetrectcap%
\pgfsetroundjoin%
\pgfsetlinewidth{1.505625pt}%
\definecolor{currentstroke}{rgb}{0.870588,0.733333,0.607843}%
\pgfsetstrokecolor{currentstroke}%
\pgfsetstrokeopacity{0.800000}%
\pgfsetdash{}{0pt}%
\pgfpathmoveto{\pgfqpoint{2.475746in}{4.004366in}}%
\pgfpathlineto{\pgfqpoint{4.473902in}{4.321654in}}%
\pgfusepath{stroke}%
\end{pgfscope}%
\begin{pgfscope}%
\pgfpathrectangle{\pgfqpoint{0.481978in}{0.331635in}}{\pgfqpoint{9.300000in}{7.700000in}}%
\pgfusepath{clip}%
\pgfsetrectcap%
\pgfsetroundjoin%
\pgfsetlinewidth{1.505625pt}%
\definecolor{currentstroke}{rgb}{0.870588,0.733333,0.607843}%
\pgfsetstrokecolor{currentstroke}%
\pgfsetstrokeopacity{0.800000}%
\pgfsetdash{}{0pt}%
\pgfpathmoveto{\pgfqpoint{7.321265in}{6.030797in}}%
\pgfpathlineto{\pgfqpoint{4.473902in}{4.321654in}}%
\pgfusepath{stroke}%
\end{pgfscope}%
\begin{pgfscope}%
\pgfpathrectangle{\pgfqpoint{0.481978in}{0.331635in}}{\pgfqpoint{9.300000in}{7.700000in}}%
\pgfusepath{clip}%
\pgfsetrectcap%
\pgfsetroundjoin%
\pgfsetlinewidth{1.505625pt}%
\definecolor{currentstroke}{rgb}{0.870588,0.733333,0.607843}%
\pgfsetstrokecolor{currentstroke}%
\pgfsetstrokeopacity{0.800000}%
\pgfsetdash{}{0pt}%
\pgfpathmoveto{\pgfqpoint{1.982524in}{5.206262in}}%
\pgfpathlineto{\pgfqpoint{4.473902in}{4.321654in}}%
\pgfusepath{stroke}%
\end{pgfscope}%
\begin{pgfscope}%
\pgfpathrectangle{\pgfqpoint{0.481978in}{0.331635in}}{\pgfqpoint{9.300000in}{7.700000in}}%
\pgfusepath{clip}%
\pgfsetrectcap%
\pgfsetroundjoin%
\pgfsetlinewidth{1.505625pt}%
\definecolor{currentstroke}{rgb}{0.870588,0.733333,0.607843}%
\pgfsetstrokecolor{currentstroke}%
\pgfsetstrokeopacity{0.800000}%
\pgfsetdash{}{0pt}%
\pgfpathmoveto{\pgfqpoint{2.801903in}{4.401537in}}%
\pgfpathlineto{\pgfqpoint{4.473902in}{4.321654in}}%
\pgfusepath{stroke}%
\end{pgfscope}%
\begin{pgfscope}%
\pgfpathrectangle{\pgfqpoint{0.481978in}{0.331635in}}{\pgfqpoint{9.300000in}{7.700000in}}%
\pgfusepath{clip}%
\pgfsetrectcap%
\pgfsetroundjoin%
\pgfsetlinewidth{1.505625pt}%
\definecolor{currentstroke}{rgb}{0.870588,0.733333,0.607843}%
\pgfsetstrokecolor{currentstroke}%
\pgfsetstrokeopacity{0.800000}%
\pgfsetdash{}{0pt}%
\pgfpathmoveto{\pgfqpoint{3.503709in}{4.313535in}}%
\pgfpathlineto{\pgfqpoint{4.473902in}{4.321654in}}%
\pgfusepath{stroke}%
\end{pgfscope}%
\begin{pgfscope}%
\pgfpathrectangle{\pgfqpoint{0.481978in}{0.331635in}}{\pgfqpoint{9.300000in}{7.700000in}}%
\pgfusepath{clip}%
\pgfsetrectcap%
\pgfsetroundjoin%
\pgfsetlinewidth{1.505625pt}%
\definecolor{currentstroke}{rgb}{0.870588,0.733333,0.607843}%
\pgfsetstrokecolor{currentstroke}%
\pgfsetstrokeopacity{0.800000}%
\pgfsetdash{}{0pt}%
\pgfpathmoveto{\pgfqpoint{7.049429in}{6.056502in}}%
\pgfpathlineto{\pgfqpoint{4.473902in}{4.321654in}}%
\pgfusepath{stroke}%
\end{pgfscope}%
\begin{pgfscope}%
\pgfpathrectangle{\pgfqpoint{0.481978in}{0.331635in}}{\pgfqpoint{9.300000in}{7.700000in}}%
\pgfusepath{clip}%
\pgfsetrectcap%
\pgfsetroundjoin%
\pgfsetlinewidth{1.505625pt}%
\definecolor{currentstroke}{rgb}{0.870588,0.733333,0.607843}%
\pgfsetstrokecolor{currentstroke}%
\pgfsetstrokeopacity{0.800000}%
\pgfsetdash{}{0pt}%
\pgfpathmoveto{\pgfqpoint{6.348280in}{4.846864in}}%
\pgfpathlineto{\pgfqpoint{4.473902in}{4.321654in}}%
\pgfusepath{stroke}%
\end{pgfscope}%
\begin{pgfscope}%
\pgfpathrectangle{\pgfqpoint{0.481978in}{0.331635in}}{\pgfqpoint{9.300000in}{7.700000in}}%
\pgfusepath{clip}%
\pgfsetrectcap%
\pgfsetroundjoin%
\pgfsetlinewidth{1.505625pt}%
\definecolor{currentstroke}{rgb}{0.870588,0.733333,0.607843}%
\pgfsetstrokecolor{currentstroke}%
\pgfsetstrokeopacity{0.800000}%
\pgfsetdash{}{0pt}%
\pgfpathmoveto{\pgfqpoint{5.923605in}{3.681854in}}%
\pgfpathlineto{\pgfqpoint{4.473902in}{4.321654in}}%
\pgfusepath{stroke}%
\end{pgfscope}%
\begin{pgfscope}%
\pgfpathrectangle{\pgfqpoint{0.481978in}{0.331635in}}{\pgfqpoint{9.300000in}{7.700000in}}%
\pgfusepath{clip}%
\pgfsetrectcap%
\pgfsetroundjoin%
\pgfsetlinewidth{1.505625pt}%
\definecolor{currentstroke}{rgb}{0.870588,0.733333,0.607843}%
\pgfsetstrokecolor{currentstroke}%
\pgfsetstrokeopacity{0.800000}%
\pgfsetdash{}{0pt}%
\pgfpathmoveto{\pgfqpoint{5.344914in}{4.036077in}}%
\pgfpathlineto{\pgfqpoint{4.473902in}{4.321654in}}%
\pgfusepath{stroke}%
\end{pgfscope}%
\begin{pgfscope}%
\pgfpathrectangle{\pgfqpoint{0.481978in}{0.331635in}}{\pgfqpoint{9.300000in}{7.700000in}}%
\pgfusepath{clip}%
\pgfsetrectcap%
\pgfsetroundjoin%
\pgfsetlinewidth{1.505625pt}%
\definecolor{currentstroke}{rgb}{0.870588,0.733333,0.607843}%
\pgfsetstrokecolor{currentstroke}%
\pgfsetstrokeopacity{0.800000}%
\pgfsetdash{}{0pt}%
\pgfpathmoveto{\pgfqpoint{4.527667in}{3.744276in}}%
\pgfpathlineto{\pgfqpoint{4.473902in}{4.321654in}}%
\pgfusepath{stroke}%
\end{pgfscope}%
\begin{pgfscope}%
\pgfpathrectangle{\pgfqpoint{0.481978in}{0.331635in}}{\pgfqpoint{9.300000in}{7.700000in}}%
\pgfusepath{clip}%
\pgfsetrectcap%
\pgfsetroundjoin%
\pgfsetlinewidth{1.505625pt}%
\definecolor{currentstroke}{rgb}{0.870588,0.733333,0.607843}%
\pgfsetstrokecolor{currentstroke}%
\pgfsetstrokeopacity{0.800000}%
\pgfsetdash{}{0pt}%
\pgfpathmoveto{\pgfqpoint{2.059850in}{4.036460in}}%
\pgfpathlineto{\pgfqpoint{4.473902in}{4.321654in}}%
\pgfusepath{stroke}%
\end{pgfscope}%
\begin{pgfscope}%
\pgfpathrectangle{\pgfqpoint{0.481978in}{0.331635in}}{\pgfqpoint{9.300000in}{7.700000in}}%
\pgfusepath{clip}%
\pgfsetrectcap%
\pgfsetroundjoin%
\pgfsetlinewidth{1.505625pt}%
\definecolor{currentstroke}{rgb}{0.870588,0.733333,0.607843}%
\pgfsetstrokecolor{currentstroke}%
\pgfsetstrokeopacity{0.800000}%
\pgfsetdash{}{0pt}%
\pgfpathmoveto{\pgfqpoint{5.974496in}{4.950112in}}%
\pgfpathlineto{\pgfqpoint{4.473902in}{4.321654in}}%
\pgfusepath{stroke}%
\end{pgfscope}%
\begin{pgfscope}%
\pgfpathrectangle{\pgfqpoint{0.481978in}{0.331635in}}{\pgfqpoint{9.300000in}{7.700000in}}%
\pgfusepath{clip}%
\pgfsetrectcap%
\pgfsetroundjoin%
\pgfsetlinewidth{1.505625pt}%
\definecolor{currentstroke}{rgb}{0.870588,0.733333,0.607843}%
\pgfsetstrokecolor{currentstroke}%
\pgfsetstrokeopacity{0.800000}%
\pgfsetdash{}{0pt}%
\pgfpathmoveto{\pgfqpoint{2.319348in}{4.794802in}}%
\pgfpathlineto{\pgfqpoint{4.473902in}{4.321654in}}%
\pgfusepath{stroke}%
\end{pgfscope}%
\begin{pgfscope}%
\pgfpathrectangle{\pgfqpoint{0.481978in}{0.331635in}}{\pgfqpoint{9.300000in}{7.700000in}}%
\pgfusepath{clip}%
\pgfsetrectcap%
\pgfsetroundjoin%
\pgfsetlinewidth{1.505625pt}%
\definecolor{currentstroke}{rgb}{0.870588,0.733333,0.607843}%
\pgfsetstrokecolor{currentstroke}%
\pgfsetstrokeopacity{0.800000}%
\pgfsetdash{}{0pt}%
\pgfpathmoveto{\pgfqpoint{6.238443in}{6.242431in}}%
\pgfpathlineto{\pgfqpoint{4.473902in}{4.321654in}}%
\pgfusepath{stroke}%
\end{pgfscope}%
\begin{pgfscope}%
\pgfpathrectangle{\pgfqpoint{0.481978in}{0.331635in}}{\pgfqpoint{9.300000in}{7.700000in}}%
\pgfusepath{clip}%
\pgfsetrectcap%
\pgfsetroundjoin%
\pgfsetlinewidth{1.505625pt}%
\definecolor{currentstroke}{rgb}{0.870588,0.733333,0.607843}%
\pgfsetstrokecolor{currentstroke}%
\pgfsetstrokeopacity{0.800000}%
\pgfsetdash{}{0pt}%
\pgfpathmoveto{\pgfqpoint{4.167128in}{3.897191in}}%
\pgfpathlineto{\pgfqpoint{4.473902in}{4.321654in}}%
\pgfusepath{stroke}%
\end{pgfscope}%
\begin{pgfscope}%
\pgfpathrectangle{\pgfqpoint{0.481978in}{0.331635in}}{\pgfqpoint{9.300000in}{7.700000in}}%
\pgfusepath{clip}%
\pgfsetrectcap%
\pgfsetroundjoin%
\pgfsetlinewidth{1.505625pt}%
\definecolor{currentstroke}{rgb}{0.870588,0.733333,0.607843}%
\pgfsetstrokecolor{currentstroke}%
\pgfsetstrokeopacity{0.800000}%
\pgfsetdash{}{0pt}%
\pgfpathmoveto{\pgfqpoint{4.383560in}{3.597829in}}%
\pgfpathlineto{\pgfqpoint{4.473902in}{4.321654in}}%
\pgfusepath{stroke}%
\end{pgfscope}%
\begin{pgfscope}%
\pgfpathrectangle{\pgfqpoint{0.481978in}{0.331635in}}{\pgfqpoint{9.300000in}{7.700000in}}%
\pgfusepath{clip}%
\pgfsetrectcap%
\pgfsetroundjoin%
\pgfsetlinewidth{1.505625pt}%
\definecolor{currentstroke}{rgb}{0.870588,0.733333,0.607843}%
\pgfsetstrokecolor{currentstroke}%
\pgfsetstrokeopacity{0.800000}%
\pgfsetdash{}{0pt}%
\pgfpathmoveto{\pgfqpoint{5.769577in}{4.001990in}}%
\pgfpathlineto{\pgfqpoint{4.473902in}{4.321654in}}%
\pgfusepath{stroke}%
\end{pgfscope}%
\begin{pgfscope}%
\pgfpathrectangle{\pgfqpoint{0.481978in}{0.331635in}}{\pgfqpoint{9.300000in}{7.700000in}}%
\pgfusepath{clip}%
\pgfsetrectcap%
\pgfsetroundjoin%
\pgfsetlinewidth{1.505625pt}%
\definecolor{currentstroke}{rgb}{0.870588,0.733333,0.607843}%
\pgfsetstrokecolor{currentstroke}%
\pgfsetstrokeopacity{0.800000}%
\pgfsetdash{}{0pt}%
\pgfpathmoveto{\pgfqpoint{9.359251in}{4.570076in}}%
\pgfpathlineto{\pgfqpoint{4.473902in}{4.321654in}}%
\pgfusepath{stroke}%
\end{pgfscope}%
\begin{pgfscope}%
\pgfpathrectangle{\pgfqpoint{0.481978in}{0.331635in}}{\pgfqpoint{9.300000in}{7.700000in}}%
\pgfusepath{clip}%
\pgfsetrectcap%
\pgfsetroundjoin%
\pgfsetlinewidth{1.505625pt}%
\definecolor{currentstroke}{rgb}{0.870588,0.733333,0.607843}%
\pgfsetstrokecolor{currentstroke}%
\pgfsetstrokeopacity{0.800000}%
\pgfsetdash{}{0pt}%
\pgfpathmoveto{\pgfqpoint{5.675309in}{2.672631in}}%
\pgfpathlineto{\pgfqpoint{4.473902in}{4.321654in}}%
\pgfusepath{stroke}%
\end{pgfscope}%
\begin{pgfscope}%
\pgfpathrectangle{\pgfqpoint{0.481978in}{0.331635in}}{\pgfqpoint{9.300000in}{7.700000in}}%
\pgfusepath{clip}%
\pgfsetrectcap%
\pgfsetroundjoin%
\pgfsetlinewidth{1.505625pt}%
\definecolor{currentstroke}{rgb}{0.870588,0.733333,0.607843}%
\pgfsetstrokecolor{currentstroke}%
\pgfsetstrokeopacity{0.800000}%
\pgfsetdash{}{0pt}%
\pgfpathmoveto{\pgfqpoint{3.026449in}{4.347964in}}%
\pgfpathlineto{\pgfqpoint{4.473902in}{4.321654in}}%
\pgfusepath{stroke}%
\end{pgfscope}%
\begin{pgfscope}%
\pgfpathrectangle{\pgfqpoint{0.481978in}{0.331635in}}{\pgfqpoint{9.300000in}{7.700000in}}%
\pgfusepath{clip}%
\pgfsetrectcap%
\pgfsetroundjoin%
\pgfsetlinewidth{1.505625pt}%
\definecolor{currentstroke}{rgb}{0.870588,0.733333,0.607843}%
\pgfsetstrokecolor{currentstroke}%
\pgfsetstrokeopacity{0.800000}%
\pgfsetdash{}{0pt}%
\pgfpathmoveto{\pgfqpoint{6.761280in}{5.866030in}}%
\pgfpathlineto{\pgfqpoint{4.473902in}{4.321654in}}%
\pgfusepath{stroke}%
\end{pgfscope}%
\begin{pgfscope}%
\pgfpathrectangle{\pgfqpoint{0.481978in}{0.331635in}}{\pgfqpoint{9.300000in}{7.700000in}}%
\pgfusepath{clip}%
\pgfsetrectcap%
\pgfsetroundjoin%
\pgfsetlinewidth{1.505625pt}%
\definecolor{currentstroke}{rgb}{0.870588,0.733333,0.607843}%
\pgfsetstrokecolor{currentstroke}%
\pgfsetstrokeopacity{0.800000}%
\pgfsetdash{}{0pt}%
\pgfpathmoveto{\pgfqpoint{2.445109in}{3.109579in}}%
\pgfpathlineto{\pgfqpoint{4.473902in}{4.321654in}}%
\pgfusepath{stroke}%
\end{pgfscope}%
\begin{pgfscope}%
\pgfpathrectangle{\pgfqpoint{0.481978in}{0.331635in}}{\pgfqpoint{9.300000in}{7.700000in}}%
\pgfusepath{clip}%
\pgfsetrectcap%
\pgfsetroundjoin%
\pgfsetlinewidth{1.505625pt}%
\definecolor{currentstroke}{rgb}{0.870588,0.733333,0.607843}%
\pgfsetstrokecolor{currentstroke}%
\pgfsetstrokeopacity{0.800000}%
\pgfsetdash{}{0pt}%
\pgfpathmoveto{\pgfqpoint{2.556572in}{4.487385in}}%
\pgfpathlineto{\pgfqpoint{4.473902in}{4.321654in}}%
\pgfusepath{stroke}%
\end{pgfscope}%
\begin{pgfscope}%
\pgfpathrectangle{\pgfqpoint{0.481978in}{0.331635in}}{\pgfqpoint{9.300000in}{7.700000in}}%
\pgfusepath{clip}%
\pgfsetrectcap%
\pgfsetroundjoin%
\pgfsetlinewidth{1.505625pt}%
\definecolor{currentstroke}{rgb}{0.870588,0.733333,0.607843}%
\pgfsetstrokecolor{currentstroke}%
\pgfsetstrokeopacity{0.800000}%
\pgfsetdash{}{0pt}%
\pgfpathmoveto{\pgfqpoint{3.956527in}{3.307551in}}%
\pgfpathlineto{\pgfqpoint{4.473902in}{4.321654in}}%
\pgfusepath{stroke}%
\end{pgfscope}%
\begin{pgfscope}%
\pgfpathrectangle{\pgfqpoint{0.481978in}{0.331635in}}{\pgfqpoint{9.300000in}{7.700000in}}%
\pgfusepath{clip}%
\pgfsetrectcap%
\pgfsetroundjoin%
\pgfsetlinewidth{1.505625pt}%
\definecolor{currentstroke}{rgb}{0.870588,0.733333,0.607843}%
\pgfsetstrokecolor{currentstroke}%
\pgfsetstrokeopacity{0.800000}%
\pgfsetdash{}{0pt}%
\pgfpathmoveto{\pgfqpoint{2.593480in}{4.742968in}}%
\pgfpathlineto{\pgfqpoint{4.473902in}{4.321654in}}%
\pgfusepath{stroke}%
\end{pgfscope}%
\begin{pgfscope}%
\pgfpathrectangle{\pgfqpoint{0.481978in}{0.331635in}}{\pgfqpoint{9.300000in}{7.700000in}}%
\pgfusepath{clip}%
\pgfsetrectcap%
\pgfsetroundjoin%
\pgfsetlinewidth{1.505625pt}%
\definecolor{currentstroke}{rgb}{0.870588,0.733333,0.607843}%
\pgfsetstrokecolor{currentstroke}%
\pgfsetstrokeopacity{0.800000}%
\pgfsetdash{}{0pt}%
\pgfpathmoveto{\pgfqpoint{1.766805in}{3.503655in}}%
\pgfpathlineto{\pgfqpoint{4.473902in}{4.321654in}}%
\pgfusepath{stroke}%
\end{pgfscope}%
\begin{pgfscope}%
\pgfpathrectangle{\pgfqpoint{0.481978in}{0.331635in}}{\pgfqpoint{9.300000in}{7.700000in}}%
\pgfusepath{clip}%
\pgfsetrectcap%
\pgfsetroundjoin%
\pgfsetlinewidth{1.505625pt}%
\definecolor{currentstroke}{rgb}{0.870588,0.733333,0.607843}%
\pgfsetstrokecolor{currentstroke}%
\pgfsetstrokeopacity{0.800000}%
\pgfsetdash{}{0pt}%
\pgfpathmoveto{\pgfqpoint{4.977562in}{4.282649in}}%
\pgfpathlineto{\pgfqpoint{4.473902in}{4.321654in}}%
\pgfusepath{stroke}%
\end{pgfscope}%
\begin{pgfscope}%
\pgfpathrectangle{\pgfqpoint{0.481978in}{0.331635in}}{\pgfqpoint{9.300000in}{7.700000in}}%
\pgfusepath{clip}%
\pgfsetrectcap%
\pgfsetroundjoin%
\pgfsetlinewidth{1.505625pt}%
\definecolor{currentstroke}{rgb}{0.870588,0.733333,0.607843}%
\pgfsetstrokecolor{currentstroke}%
\pgfsetstrokeopacity{0.800000}%
\pgfsetdash{}{0pt}%
\pgfpathmoveto{\pgfqpoint{5.616567in}{3.990039in}}%
\pgfpathlineto{\pgfqpoint{4.473902in}{4.321654in}}%
\pgfusepath{stroke}%
\end{pgfscope}%
\begin{pgfscope}%
\pgfpathrectangle{\pgfqpoint{0.481978in}{0.331635in}}{\pgfqpoint{9.300000in}{7.700000in}}%
\pgfusepath{clip}%
\pgfsetrectcap%
\pgfsetroundjoin%
\pgfsetlinewidth{1.505625pt}%
\definecolor{currentstroke}{rgb}{0.870588,0.733333,0.607843}%
\pgfsetstrokecolor{currentstroke}%
\pgfsetstrokeopacity{0.800000}%
\pgfsetdash{}{0pt}%
\pgfpathmoveto{\pgfqpoint{2.747420in}{2.580517in}}%
\pgfpathlineto{\pgfqpoint{4.473902in}{4.321654in}}%
\pgfusepath{stroke}%
\end{pgfscope}%
\begin{pgfscope}%
\pgfpathrectangle{\pgfqpoint{0.481978in}{0.331635in}}{\pgfqpoint{9.300000in}{7.700000in}}%
\pgfusepath{clip}%
\pgfsetrectcap%
\pgfsetroundjoin%
\pgfsetlinewidth{1.505625pt}%
\definecolor{currentstroke}{rgb}{0.870588,0.733333,0.607843}%
\pgfsetstrokecolor{currentstroke}%
\pgfsetstrokeopacity{0.800000}%
\pgfsetdash{}{0pt}%
\pgfpathmoveto{\pgfqpoint{3.710302in}{3.442909in}}%
\pgfpathlineto{\pgfqpoint{4.473902in}{4.321654in}}%
\pgfusepath{stroke}%
\end{pgfscope}%
\begin{pgfscope}%
\pgfpathrectangle{\pgfqpoint{0.481978in}{0.331635in}}{\pgfqpoint{9.300000in}{7.700000in}}%
\pgfusepath{clip}%
\pgfsetrectcap%
\pgfsetroundjoin%
\pgfsetlinewidth{1.505625pt}%
\definecolor{currentstroke}{rgb}{0.870588,0.733333,0.607843}%
\pgfsetstrokecolor{currentstroke}%
\pgfsetstrokeopacity{0.800000}%
\pgfsetdash{}{0pt}%
\pgfpathmoveto{\pgfqpoint{2.149626in}{4.883578in}}%
\pgfpathlineto{\pgfqpoint{4.473902in}{4.321654in}}%
\pgfusepath{stroke}%
\end{pgfscope}%
\begin{pgfscope}%
\pgfpathrectangle{\pgfqpoint{0.481978in}{0.331635in}}{\pgfqpoint{9.300000in}{7.700000in}}%
\pgfusepath{clip}%
\pgfsetrectcap%
\pgfsetroundjoin%
\pgfsetlinewidth{1.505625pt}%
\definecolor{currentstroke}{rgb}{0.870588,0.733333,0.607843}%
\pgfsetstrokecolor{currentstroke}%
\pgfsetstrokeopacity{0.800000}%
\pgfsetdash{}{0pt}%
\pgfpathmoveto{\pgfqpoint{2.077561in}{3.103422in}}%
\pgfpathlineto{\pgfqpoint{4.473902in}{4.321654in}}%
\pgfusepath{stroke}%
\end{pgfscope}%
\begin{pgfscope}%
\pgfpathrectangle{\pgfqpoint{0.481978in}{0.331635in}}{\pgfqpoint{9.300000in}{7.700000in}}%
\pgfusepath{clip}%
\pgfsetrectcap%
\pgfsetroundjoin%
\pgfsetlinewidth{1.505625pt}%
\definecolor{currentstroke}{rgb}{0.870588,0.733333,0.607843}%
\pgfsetstrokecolor{currentstroke}%
\pgfsetstrokeopacity{0.800000}%
\pgfsetdash{}{0pt}%
\pgfpathmoveto{\pgfqpoint{1.957320in}{5.229684in}}%
\pgfpathlineto{\pgfqpoint{4.473902in}{4.321654in}}%
\pgfusepath{stroke}%
\end{pgfscope}%
\begin{pgfscope}%
\pgfpathrectangle{\pgfqpoint{0.481978in}{0.331635in}}{\pgfqpoint{9.300000in}{7.700000in}}%
\pgfusepath{clip}%
\pgfsetrectcap%
\pgfsetroundjoin%
\pgfsetlinewidth{1.505625pt}%
\definecolor{currentstroke}{rgb}{0.870588,0.733333,0.607843}%
\pgfsetstrokecolor{currentstroke}%
\pgfsetstrokeopacity{0.800000}%
\pgfsetdash{}{0pt}%
\pgfpathmoveto{\pgfqpoint{2.660918in}{3.550587in}}%
\pgfpathlineto{\pgfqpoint{4.473902in}{4.321654in}}%
\pgfusepath{stroke}%
\end{pgfscope}%
\begin{pgfscope}%
\pgfpathrectangle{\pgfqpoint{0.481978in}{0.331635in}}{\pgfqpoint{9.300000in}{7.700000in}}%
\pgfusepath{clip}%
\pgfsetrectcap%
\pgfsetroundjoin%
\pgfsetlinewidth{1.505625pt}%
\definecolor{currentstroke}{rgb}{0.870588,0.733333,0.607843}%
\pgfsetstrokecolor{currentstroke}%
\pgfsetstrokeopacity{0.800000}%
\pgfsetdash{}{0pt}%
\pgfpathmoveto{\pgfqpoint{7.871019in}{5.550531in}}%
\pgfpathlineto{\pgfqpoint{4.473902in}{4.321654in}}%
\pgfusepath{stroke}%
\end{pgfscope}%
\begin{pgfscope}%
\pgfpathrectangle{\pgfqpoint{0.481978in}{0.331635in}}{\pgfqpoint{9.300000in}{7.700000in}}%
\pgfusepath{clip}%
\pgfsetrectcap%
\pgfsetroundjoin%
\pgfsetlinewidth{1.505625pt}%
\definecolor{currentstroke}{rgb}{0.870588,0.733333,0.607843}%
\pgfsetstrokecolor{currentstroke}%
\pgfsetstrokeopacity{0.800000}%
\pgfsetdash{}{0pt}%
\pgfpathmoveto{\pgfqpoint{5.367711in}{4.176744in}}%
\pgfpathlineto{\pgfqpoint{4.473902in}{4.321654in}}%
\pgfusepath{stroke}%
\end{pgfscope}%
\begin{pgfscope}%
\pgfpathrectangle{\pgfqpoint{0.481978in}{0.331635in}}{\pgfqpoint{9.300000in}{7.700000in}}%
\pgfusepath{clip}%
\pgfsetrectcap%
\pgfsetroundjoin%
\pgfsetlinewidth{1.505625pt}%
\definecolor{currentstroke}{rgb}{0.870588,0.733333,0.607843}%
\pgfsetstrokecolor{currentstroke}%
\pgfsetstrokeopacity{0.800000}%
\pgfsetdash{}{0pt}%
\pgfpathmoveto{\pgfqpoint{2.321412in}{4.244642in}}%
\pgfpathlineto{\pgfqpoint{4.473902in}{4.321654in}}%
\pgfusepath{stroke}%
\end{pgfscope}%
\begin{pgfscope}%
\pgfpathrectangle{\pgfqpoint{0.481978in}{0.331635in}}{\pgfqpoint{9.300000in}{7.700000in}}%
\pgfusepath{clip}%
\pgfsetrectcap%
\pgfsetroundjoin%
\pgfsetlinewidth{1.505625pt}%
\definecolor{currentstroke}{rgb}{0.870588,0.733333,0.607843}%
\pgfsetstrokecolor{currentstroke}%
\pgfsetstrokeopacity{0.800000}%
\pgfsetdash{}{0pt}%
\pgfpathmoveto{\pgfqpoint{7.119603in}{6.303649in}}%
\pgfpathlineto{\pgfqpoint{4.473902in}{4.321654in}}%
\pgfusepath{stroke}%
\end{pgfscope}%
\begin{pgfscope}%
\pgfpathrectangle{\pgfqpoint{0.481978in}{0.331635in}}{\pgfqpoint{9.300000in}{7.700000in}}%
\pgfusepath{clip}%
\pgfsetrectcap%
\pgfsetroundjoin%
\pgfsetlinewidth{1.505625pt}%
\definecolor{currentstroke}{rgb}{0.870588,0.733333,0.607843}%
\pgfsetstrokecolor{currentstroke}%
\pgfsetstrokeopacity{0.800000}%
\pgfsetdash{}{0pt}%
\pgfpathmoveto{\pgfqpoint{6.487284in}{5.504131in}}%
\pgfpathlineto{\pgfqpoint{4.473902in}{4.321654in}}%
\pgfusepath{stroke}%
\end{pgfscope}%
\begin{pgfscope}%
\pgfpathrectangle{\pgfqpoint{0.481978in}{0.331635in}}{\pgfqpoint{9.300000in}{7.700000in}}%
\pgfusepath{clip}%
\pgfsetrectcap%
\pgfsetroundjoin%
\pgfsetlinewidth{1.505625pt}%
\definecolor{currentstroke}{rgb}{0.870588,0.733333,0.607843}%
\pgfsetstrokecolor{currentstroke}%
\pgfsetstrokeopacity{0.800000}%
\pgfsetdash{}{0pt}%
\pgfpathmoveto{\pgfqpoint{8.920274in}{3.555200in}}%
\pgfpathlineto{\pgfqpoint{4.473902in}{4.321654in}}%
\pgfusepath{stroke}%
\end{pgfscope}%
\begin{pgfscope}%
\pgfpathrectangle{\pgfqpoint{0.481978in}{0.331635in}}{\pgfqpoint{9.300000in}{7.700000in}}%
\pgfusepath{clip}%
\pgfsetrectcap%
\pgfsetroundjoin%
\pgfsetlinewidth{1.505625pt}%
\definecolor{currentstroke}{rgb}{0.870588,0.733333,0.607843}%
\pgfsetstrokecolor{currentstroke}%
\pgfsetstrokeopacity{0.800000}%
\pgfsetdash{}{0pt}%
\pgfpathmoveto{\pgfqpoint{7.303090in}{6.145660in}}%
\pgfpathlineto{\pgfqpoint{4.473902in}{4.321654in}}%
\pgfusepath{stroke}%
\end{pgfscope}%
\begin{pgfscope}%
\pgfpathrectangle{\pgfqpoint{0.481978in}{0.331635in}}{\pgfqpoint{9.300000in}{7.700000in}}%
\pgfusepath{clip}%
\pgfsetrectcap%
\pgfsetroundjoin%
\pgfsetlinewidth{1.505625pt}%
\definecolor{currentstroke}{rgb}{0.870588,0.733333,0.607843}%
\pgfsetstrokecolor{currentstroke}%
\pgfsetstrokeopacity{0.800000}%
\pgfsetdash{}{0pt}%
\pgfpathmoveto{\pgfqpoint{9.189203in}{4.320348in}}%
\pgfpathlineto{\pgfqpoint{4.473902in}{4.321654in}}%
\pgfusepath{stroke}%
\end{pgfscope}%
\begin{pgfscope}%
\pgfsetrectcap%
\pgfsetmiterjoin%
\pgfsetlinewidth{0.803000pt}%
\definecolor{currentstroke}{rgb}{0.000000,0.000000,0.000000}%
\pgfsetstrokecolor{currentstroke}%
\pgfsetdash{}{0pt}%
\pgfpathmoveto{\pgfqpoint{0.481978in}{0.331635in}}%
\pgfpathlineto{\pgfqpoint{0.481978in}{8.031635in}}%
\pgfusepath{stroke}%
\end{pgfscope}%
\begin{pgfscope}%
\pgfsetrectcap%
\pgfsetmiterjoin%
\pgfsetlinewidth{0.803000pt}%
\definecolor{currentstroke}{rgb}{0.000000,0.000000,0.000000}%
\pgfsetstrokecolor{currentstroke}%
\pgfsetdash{}{0pt}%
\pgfpathmoveto{\pgfqpoint{9.781978in}{0.331635in}}%
\pgfpathlineto{\pgfqpoint{9.781978in}{8.031635in}}%
\pgfusepath{stroke}%
\end{pgfscope}%
\begin{pgfscope}%
\pgfsetrectcap%
\pgfsetmiterjoin%
\pgfsetlinewidth{0.803000pt}%
\definecolor{currentstroke}{rgb}{0.000000,0.000000,0.000000}%
\pgfsetstrokecolor{currentstroke}%
\pgfsetdash{}{0pt}%
\pgfpathmoveto{\pgfqpoint{0.481978in}{0.331635in}}%
\pgfpathlineto{\pgfqpoint{9.781978in}{0.331635in}}%
\pgfusepath{stroke}%
\end{pgfscope}%
\begin{pgfscope}%
\pgfsetrectcap%
\pgfsetmiterjoin%
\pgfsetlinewidth{0.803000pt}%
\definecolor{currentstroke}{rgb}{0.000000,0.000000,0.000000}%
\pgfsetstrokecolor{currentstroke}%
\pgfsetdash{}{0pt}%
\pgfpathmoveto{\pgfqpoint{0.481978in}{8.031635in}}%
\pgfpathlineto{\pgfqpoint{9.781978in}{8.031635in}}%
\pgfusepath{stroke}%
\end{pgfscope}%
\begin{pgfscope}%
\definecolor{textcolor}{rgb}{0.000000,0.000000,0.000000}%
\pgfsetstrokecolor{textcolor}%
\pgfsetfillcolor{textcolor}%
\pgftext[x=5.131978in,y=8.114968in,,base]{\color{textcolor}\sffamily\fontsize{12.000000}{14.400000}\selectfont T-SNE for chair images with domain randomisation}%
\end{pgfscope}%
\begin{pgfscope}%
\pgfsetbuttcap%
\pgfsetmiterjoin%
\definecolor{currentfill}{rgb}{1.000000,1.000000,1.000000}%
\pgfsetfillcolor{currentfill}%
\pgfsetfillopacity{0.800000}%
\pgfsetlinewidth{1.003750pt}%
\definecolor{currentstroke}{rgb}{0.800000,0.800000,0.800000}%
\pgfsetstrokecolor{currentstroke}%
\pgfsetstrokeopacity{0.800000}%
\pgfsetdash{}{0pt}%
\pgfpathmoveto{\pgfqpoint{9.879200in}{3.539566in}}%
\pgfpathlineto{\pgfqpoint{12.348384in}{3.539566in}}%
\pgfpathquadraticcurveto{\pgfqpoint{12.376162in}{3.539566in}}{\pgfqpoint{12.376162in}{3.567344in}}%
\pgfpathlineto{\pgfqpoint{12.376162in}{4.795926in}}%
\pgfpathquadraticcurveto{\pgfqpoint{12.376162in}{4.823704in}}{\pgfqpoint{12.348384in}{4.823704in}}%
\pgfpathlineto{\pgfqpoint{9.879200in}{4.823704in}}%
\pgfpathquadraticcurveto{\pgfqpoint{9.851422in}{4.823704in}}{\pgfqpoint{9.851422in}{4.795926in}}%
\pgfpathlineto{\pgfqpoint{9.851422in}{3.567344in}}%
\pgfpathquadraticcurveto{\pgfqpoint{9.851422in}{3.539566in}}{\pgfqpoint{9.879200in}{3.539566in}}%
\pgfpathclose%
\pgfusepath{stroke,fill}%
\end{pgfscope}%
\begin{pgfscope}%
\pgfsetbuttcap%
\pgfsetroundjoin%
\definecolor{currentfill}{rgb}{0.631373,0.788235,0.956863}%
\pgfsetfillcolor{currentfill}%
\pgfsetlinewidth{1.003750pt}%
\definecolor{currentstroke}{rgb}{0.631373,0.788235,0.956863}%
\pgfsetstrokecolor{currentstroke}%
\pgfsetdash{}{0pt}%
\pgfsys@defobject{currentmarker}{\pgfqpoint{-0.041667in}{-0.041667in}}{\pgfqpoint{0.041667in}{0.041667in}}{%
\pgfpathmoveto{\pgfqpoint{0.000000in}{-0.041667in}}%
\pgfpathcurveto{\pgfqpoint{0.011050in}{-0.041667in}}{\pgfqpoint{0.021649in}{-0.037276in}}{\pgfqpoint{0.029463in}{-0.029463in}}%
\pgfpathcurveto{\pgfqpoint{0.037276in}{-0.021649in}}{\pgfqpoint{0.041667in}{-0.011050in}}{\pgfqpoint{0.041667in}{0.000000in}}%
\pgfpathcurveto{\pgfqpoint{0.041667in}{0.011050in}}{\pgfqpoint{0.037276in}{0.021649in}}{\pgfqpoint{0.029463in}{0.029463in}}%
\pgfpathcurveto{\pgfqpoint{0.021649in}{0.037276in}}{\pgfqpoint{0.011050in}{0.041667in}}{\pgfqpoint{0.000000in}{0.041667in}}%
\pgfpathcurveto{\pgfqpoint{-0.011050in}{0.041667in}}{\pgfqpoint{-0.021649in}{0.037276in}}{\pgfqpoint{-0.029463in}{0.029463in}}%
\pgfpathcurveto{\pgfqpoint{-0.037276in}{0.021649in}}{\pgfqpoint{-0.041667in}{0.011050in}}{\pgfqpoint{-0.041667in}{0.000000in}}%
\pgfpathcurveto{\pgfqpoint{-0.041667in}{-0.011050in}}{\pgfqpoint{-0.037276in}{-0.021649in}}{\pgfqpoint{-0.029463in}{-0.029463in}}%
\pgfpathcurveto{\pgfqpoint{-0.021649in}{-0.037276in}}{\pgfqpoint{-0.011050in}{-0.041667in}}{\pgfqpoint{0.000000in}{-0.041667in}}%
\pgfpathclose%
\pgfusepath{stroke,fill}%
}%
\begin{pgfscope}%
\pgfsys@transformshift{10.045867in}{4.699084in}%
\pgfsys@useobject{currentmarker}{}%
\end{pgfscope}%
\end{pgfscope}%
\begin{pgfscope}%
\definecolor{textcolor}{rgb}{0.000000,0.000000,0.000000}%
\pgfsetstrokecolor{textcolor}%
\pgfsetfillcolor{textcolor}%
\pgftext[x=10.295867in,y=4.662625in,left,base]{\color{textcolor}\sffamily\fontsize{10.000000}{12.000000}\selectfont Pix3D}%
\end{pgfscope}%
\begin{pgfscope}%
\pgfsetbuttcap%
\pgfsetroundjoin%
\definecolor{currentfill}{rgb}{1.000000,0.705882,0.509804}%
\pgfsetfillcolor{currentfill}%
\pgfsetlinewidth{1.003750pt}%
\definecolor{currentstroke}{rgb}{1.000000,0.705882,0.509804}%
\pgfsetstrokecolor{currentstroke}%
\pgfsetdash{}{0pt}%
\pgfsys@defobject{currentmarker}{\pgfqpoint{-0.041667in}{-0.041667in}}{\pgfqpoint{0.041667in}{0.041667in}}{%
\pgfpathmoveto{\pgfqpoint{0.000000in}{-0.041667in}}%
\pgfpathcurveto{\pgfqpoint{0.011050in}{-0.041667in}}{\pgfqpoint{0.021649in}{-0.037276in}}{\pgfqpoint{0.029463in}{-0.029463in}}%
\pgfpathcurveto{\pgfqpoint{0.037276in}{-0.021649in}}{\pgfqpoint{0.041667in}{-0.011050in}}{\pgfqpoint{0.041667in}{0.000000in}}%
\pgfpathcurveto{\pgfqpoint{0.041667in}{0.011050in}}{\pgfqpoint{0.037276in}{0.021649in}}{\pgfqpoint{0.029463in}{0.029463in}}%
\pgfpathcurveto{\pgfqpoint{0.021649in}{0.037276in}}{\pgfqpoint{0.011050in}{0.041667in}}{\pgfqpoint{0.000000in}{0.041667in}}%
\pgfpathcurveto{\pgfqpoint{-0.011050in}{0.041667in}}{\pgfqpoint{-0.021649in}{0.037276in}}{\pgfqpoint{-0.029463in}{0.029463in}}%
\pgfpathcurveto{\pgfqpoint{-0.037276in}{0.021649in}}{\pgfqpoint{-0.041667in}{0.011050in}}{\pgfqpoint{-0.041667in}{0.000000in}}%
\pgfpathcurveto{\pgfqpoint{-0.041667in}{-0.011050in}}{\pgfqpoint{-0.037276in}{-0.021649in}}{\pgfqpoint{-0.029463in}{-0.029463in}}%
\pgfpathcurveto{\pgfqpoint{-0.021649in}{-0.037276in}}{\pgfqpoint{-0.011050in}{-0.041667in}}{\pgfqpoint{0.000000in}{-0.041667in}}%
\pgfpathclose%
\pgfusepath{stroke,fill}%
}%
\begin{pgfscope}%
\pgfsys@transformshift{10.045867in}{4.495226in}%
\pgfsys@useobject{currentmarker}{}%
\end{pgfscope}%
\end{pgfscope}%
\begin{pgfscope}%
\definecolor{textcolor}{rgb}{0.000000,0.000000,0.000000}%
\pgfsetstrokecolor{textcolor}%
\pgfsetfillcolor{textcolor}%
\pgftext[x=10.295867in,y=4.458768in,left,base]{\color{textcolor}\sffamily\fontsize{10.000000}{12.000000}\selectfont s2r3dfree\_textureless}%
\end{pgfscope}%
\begin{pgfscope}%
\pgfsetbuttcap%
\pgfsetroundjoin%
\definecolor{currentfill}{rgb}{0.552941,0.898039,0.631373}%
\pgfsetfillcolor{currentfill}%
\pgfsetlinewidth{1.003750pt}%
\definecolor{currentstroke}{rgb}{0.552941,0.898039,0.631373}%
\pgfsetstrokecolor{currentstroke}%
\pgfsetdash{}{0pt}%
\pgfsys@defobject{currentmarker}{\pgfqpoint{-0.041667in}{-0.041667in}}{\pgfqpoint{0.041667in}{0.041667in}}{%
\pgfpathmoveto{\pgfqpoint{0.000000in}{-0.041667in}}%
\pgfpathcurveto{\pgfqpoint{0.011050in}{-0.041667in}}{\pgfqpoint{0.021649in}{-0.037276in}}{\pgfqpoint{0.029463in}{-0.029463in}}%
\pgfpathcurveto{\pgfqpoint{0.037276in}{-0.021649in}}{\pgfqpoint{0.041667in}{-0.011050in}}{\pgfqpoint{0.041667in}{0.000000in}}%
\pgfpathcurveto{\pgfqpoint{0.041667in}{0.011050in}}{\pgfqpoint{0.037276in}{0.021649in}}{\pgfqpoint{0.029463in}{0.029463in}}%
\pgfpathcurveto{\pgfqpoint{0.021649in}{0.037276in}}{\pgfqpoint{0.011050in}{0.041667in}}{\pgfqpoint{0.000000in}{0.041667in}}%
\pgfpathcurveto{\pgfqpoint{-0.011050in}{0.041667in}}{\pgfqpoint{-0.021649in}{0.037276in}}{\pgfqpoint{-0.029463in}{0.029463in}}%
\pgfpathcurveto{\pgfqpoint{-0.037276in}{0.021649in}}{\pgfqpoint{-0.041667in}{0.011050in}}{\pgfqpoint{-0.041667in}{0.000000in}}%
\pgfpathcurveto{\pgfqpoint{-0.041667in}{-0.011050in}}{\pgfqpoint{-0.037276in}{-0.021649in}}{\pgfqpoint{-0.029463in}{-0.029463in}}%
\pgfpathcurveto{\pgfqpoint{-0.021649in}{-0.037276in}}{\pgfqpoint{-0.011050in}{-0.041667in}}{\pgfqpoint{0.000000in}{-0.041667in}}%
\pgfpathclose%
\pgfusepath{stroke,fill}%
}%
\begin{pgfscope}%
\pgfsys@transformshift{10.045867in}{4.287504in}%
\pgfsys@useobject{currentmarker}{}%
\end{pgfscope}%
\end{pgfscope}%
\begin{pgfscope}%
\definecolor{textcolor}{rgb}{0.000000,0.000000,0.000000}%
\pgfsetstrokecolor{textcolor}%
\pgfsetfillcolor{textcolor}%
\pgftext[x=10.295867in,y=4.251045in,left,base]{\color{textcolor}\sffamily\fontsize{10.000000}{12.000000}\selectfont s2r3dfree\_textureless\_light}%
\end{pgfscope}%
\begin{pgfscope}%
\pgfsetbuttcap%
\pgfsetroundjoin%
\definecolor{currentfill}{rgb}{1.000000,0.623529,0.607843}%
\pgfsetfillcolor{currentfill}%
\pgfsetlinewidth{1.003750pt}%
\definecolor{currentstroke}{rgb}{1.000000,0.623529,0.607843}%
\pgfsetstrokecolor{currentstroke}%
\pgfsetdash{}{0pt}%
\pgfsys@defobject{currentmarker}{\pgfqpoint{-0.041667in}{-0.041667in}}{\pgfqpoint{0.041667in}{0.041667in}}{%
\pgfpathmoveto{\pgfqpoint{0.000000in}{-0.041667in}}%
\pgfpathcurveto{\pgfqpoint{0.011050in}{-0.041667in}}{\pgfqpoint{0.021649in}{-0.037276in}}{\pgfqpoint{0.029463in}{-0.029463in}}%
\pgfpathcurveto{\pgfqpoint{0.037276in}{-0.021649in}}{\pgfqpoint{0.041667in}{-0.011050in}}{\pgfqpoint{0.041667in}{0.000000in}}%
\pgfpathcurveto{\pgfqpoint{0.041667in}{0.011050in}}{\pgfqpoint{0.037276in}{0.021649in}}{\pgfqpoint{0.029463in}{0.029463in}}%
\pgfpathcurveto{\pgfqpoint{0.021649in}{0.037276in}}{\pgfqpoint{0.011050in}{0.041667in}}{\pgfqpoint{0.000000in}{0.041667in}}%
\pgfpathcurveto{\pgfqpoint{-0.011050in}{0.041667in}}{\pgfqpoint{-0.021649in}{0.037276in}}{\pgfqpoint{-0.029463in}{0.029463in}}%
\pgfpathcurveto{\pgfqpoint{-0.037276in}{0.021649in}}{\pgfqpoint{-0.041667in}{0.011050in}}{\pgfqpoint{-0.041667in}{0.000000in}}%
\pgfpathcurveto{\pgfqpoint{-0.041667in}{-0.011050in}}{\pgfqpoint{-0.037276in}{-0.021649in}}{\pgfqpoint{-0.029463in}{-0.029463in}}%
\pgfpathcurveto{\pgfqpoint{-0.021649in}{-0.037276in}}{\pgfqpoint{-0.011050in}{-0.041667in}}{\pgfqpoint{0.000000in}{-0.041667in}}%
\pgfpathclose%
\pgfusepath{stroke,fill}%
}%
\begin{pgfscope}%
\pgfsys@transformshift{10.045867in}{4.079781in}%
\pgfsys@useobject{currentmarker}{}%
\end{pgfscope}%
\end{pgfscope}%
\begin{pgfscope}%
\definecolor{textcolor}{rgb}{0.000000,0.000000,0.000000}%
\pgfsetstrokecolor{textcolor}%
\pgfsetfillcolor{textcolor}%
\pgftext[x=10.295867in,y=4.043322in,left,base]{\color{textcolor}\sffamily\fontsize{10.000000}{12.000000}\selectfont s2r3dfree\_background}%
\end{pgfscope}%
\begin{pgfscope}%
\pgfsetbuttcap%
\pgfsetroundjoin%
\definecolor{currentfill}{rgb}{0.815686,0.733333,1.000000}%
\pgfsetfillcolor{currentfill}%
\pgfsetlinewidth{1.003750pt}%
\definecolor{currentstroke}{rgb}{0.815686,0.733333,1.000000}%
\pgfsetstrokecolor{currentstroke}%
\pgfsetdash{}{0pt}%
\pgfsys@defobject{currentmarker}{\pgfqpoint{-0.041667in}{-0.041667in}}{\pgfqpoint{0.041667in}{0.041667in}}{%
\pgfpathmoveto{\pgfqpoint{0.000000in}{-0.041667in}}%
\pgfpathcurveto{\pgfqpoint{0.011050in}{-0.041667in}}{\pgfqpoint{0.021649in}{-0.037276in}}{\pgfqpoint{0.029463in}{-0.029463in}}%
\pgfpathcurveto{\pgfqpoint{0.037276in}{-0.021649in}}{\pgfqpoint{0.041667in}{-0.011050in}}{\pgfqpoint{0.041667in}{0.000000in}}%
\pgfpathcurveto{\pgfqpoint{0.041667in}{0.011050in}}{\pgfqpoint{0.037276in}{0.021649in}}{\pgfqpoint{0.029463in}{0.029463in}}%
\pgfpathcurveto{\pgfqpoint{0.021649in}{0.037276in}}{\pgfqpoint{0.011050in}{0.041667in}}{\pgfqpoint{0.000000in}{0.041667in}}%
\pgfpathcurveto{\pgfqpoint{-0.011050in}{0.041667in}}{\pgfqpoint{-0.021649in}{0.037276in}}{\pgfqpoint{-0.029463in}{0.029463in}}%
\pgfpathcurveto{\pgfqpoint{-0.037276in}{0.021649in}}{\pgfqpoint{-0.041667in}{0.011050in}}{\pgfqpoint{-0.041667in}{0.000000in}}%
\pgfpathcurveto{\pgfqpoint{-0.041667in}{-0.011050in}}{\pgfqpoint{-0.037276in}{-0.021649in}}{\pgfqpoint{-0.029463in}{-0.029463in}}%
\pgfpathcurveto{\pgfqpoint{-0.021649in}{-0.037276in}}{\pgfqpoint{-0.011050in}{-0.041667in}}{\pgfqpoint{0.000000in}{-0.041667in}}%
\pgfpathclose%
\pgfusepath{stroke,fill}%
}%
\begin{pgfscope}%
\pgfsys@transformshift{10.045867in}{3.872058in}%
\pgfsys@useobject{currentmarker}{}%
\end{pgfscope}%
\end{pgfscope}%
\begin{pgfscope}%
\definecolor{textcolor}{rgb}{0.000000,0.000000,0.000000}%
\pgfsetstrokecolor{textcolor}%
\pgfsetfillcolor{textcolor}%
\pgftext[x=10.295867in,y=3.835600in,left,base]{\color{textcolor}\sffamily\fontsize{10.000000}{12.000000}\selectfont s2r3dfree\_background\_light2}%
\end{pgfscope}%
\begin{pgfscope}%
\pgfsetbuttcap%
\pgfsetroundjoin%
\definecolor{currentfill}{rgb}{0.870588,0.733333,0.607843}%
\pgfsetfillcolor{currentfill}%
\pgfsetlinewidth{1.003750pt}%
\definecolor{currentstroke}{rgb}{0.870588,0.733333,0.607843}%
\pgfsetstrokecolor{currentstroke}%
\pgfsetdash{}{0pt}%
\pgfsys@defobject{currentmarker}{\pgfqpoint{-0.041667in}{-0.041667in}}{\pgfqpoint{0.041667in}{0.041667in}}{%
\pgfpathmoveto{\pgfqpoint{0.000000in}{-0.041667in}}%
\pgfpathcurveto{\pgfqpoint{0.011050in}{-0.041667in}}{\pgfqpoint{0.021649in}{-0.037276in}}{\pgfqpoint{0.029463in}{-0.029463in}}%
\pgfpathcurveto{\pgfqpoint{0.037276in}{-0.021649in}}{\pgfqpoint{0.041667in}{-0.011050in}}{\pgfqpoint{0.041667in}{0.000000in}}%
\pgfpathcurveto{\pgfqpoint{0.041667in}{0.011050in}}{\pgfqpoint{0.037276in}{0.021649in}}{\pgfqpoint{0.029463in}{0.029463in}}%
\pgfpathcurveto{\pgfqpoint{0.021649in}{0.037276in}}{\pgfqpoint{0.011050in}{0.041667in}}{\pgfqpoint{0.000000in}{0.041667in}}%
\pgfpathcurveto{\pgfqpoint{-0.011050in}{0.041667in}}{\pgfqpoint{-0.021649in}{0.037276in}}{\pgfqpoint{-0.029463in}{0.029463in}}%
\pgfpathcurveto{\pgfqpoint{-0.037276in}{0.021649in}}{\pgfqpoint{-0.041667in}{0.011050in}}{\pgfqpoint{-0.041667in}{0.000000in}}%
\pgfpathcurveto{\pgfqpoint{-0.041667in}{-0.011050in}}{\pgfqpoint{-0.037276in}{-0.021649in}}{\pgfqpoint{-0.029463in}{-0.029463in}}%
\pgfpathcurveto{\pgfqpoint{-0.021649in}{-0.037276in}}{\pgfqpoint{-0.011050in}{-0.041667in}}{\pgfqpoint{0.000000in}{-0.041667in}}%
\pgfpathclose%
\pgfusepath{stroke,fill}%
}%
\begin{pgfscope}%
\pgfsys@transformshift{10.045867in}{3.664335in}%
\pgfsys@useobject{currentmarker}{}%
\end{pgfscope}%
\end{pgfscope}%
\begin{pgfscope}%
\definecolor{textcolor}{rgb}{0.000000,0.000000,0.000000}%
\pgfsetstrokecolor{textcolor}%
\pgfsetfillcolor{textcolor}%
\pgftext[x=10.295867in,y=3.627877in,left,base]{\color{textcolor}\sffamily\fontsize{10.000000}{12.000000}\selectfont s2r3dfree\_chair}%
\end{pgfscope}%
\end{pgfpicture}%
\makeatother%
\endgroup%
}
%    \caption{A combined \gls{tsne} visualization for chair images from Pix3d and \gls{free} dataset with different parameters for randomization.}
%    \label{fig:pix3dchair_s2r3dfreechair}
%\end{figure}

\subsubsection{\gls{free}\_Textureless}

The chair models were kept at the center of an un-textured room with a constant light source for the textureless dataset.
A total of 10000 images were generated from different camera viewpoints.

\subsubsection{\gls{free}\_Textureless\_Light}

Like the textureless dataset, the chair models were kept at the center of an un-textured room for this dataset.
However, the light source was randomized.
The light variation was implemented as in \autoref{subsec:lightings-and-shadows}.
Along with randomized light sources, the camera viewpoints were randomized with a distance in the range of 0.75 to 1.5 meters from the model under observation.

\subsubsection{\gls{free}\_Textured}

As the name suggests, both the model and the single default room were textured randomly for each snapshot, as explained in \autoref{subsec:randomised-texture}.
Ten thousand snapshots of chair models were taken using different camera viewpoints.

\subsubsection{\gls{free}\_Textured\_Light}

This dataset is an extension of the above mention \gls{free}\_Textured dataset, with randomized light sources.
The lights are randomized as implemented in \autoref{subsec:lightings-and-shadows}.

\subsubsection{\gls{free}\_Multi-Object}

\gls{free}\_Multi-Object dataset was created using 'Multi-Object pipeline' with chair replacing a similar category from the scene as implemented in \autoref{subsec:replacing-target-objects}.
Both the light and camera viewpoints were randomized, ensuring that the model under observation is not completely occluded.

\subsubsection{\gls{free}\_Combined}
\gls{free}\_Combined dataset is a combination of all the five types of datasets mentioned above.
Adding 10,000 images from each dataset, we get a total of 50,000 chair images in this bundle.
This dataset is an aggregate of all types of domain randomization.





