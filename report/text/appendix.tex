\chapter*{\iftoggle{german}{Anhang}{Appendix}}
\markboth{}{}
\setcounter{chapter}{1}
\addcontentsline{toc}{chapter}{\iftoggle{german}{Anhang}{Appendix}}

\section{Study with F-Score}\label{sec:study-with-f-score}

In this section, we conduct a contemporary study on the performance of Deep Learning models using the \gls{f1} as an evaluation measure instead of \gls{iou}.
\gls{iou} tends to penalize a single wrong classification more than Dice score, and hence we can say that the \gls{f1} gives a measure that is closer to the average,
whereas \gls{iou} gives a score for the worst possible scenario.
Even though both \gls{iou} and \gls{f1} are positively correlated, a study with \gls{f1} might support the dataset with better average.

\subsection{Baseline}\label{subsec:baseline_dice}
The setup for this experiment is explained in \autoref{sec:baseline}, where the evaluation was done using \gls{iou}.
In \autoref{fig:baseline_dice1}, we see the comparison of models trained on a real and a synthetic dataset.
The performance is similar to what we see in \autoref{fig:baseline1}.

\begin{figure}[ht]
    \centering
    \resizebox{0.75\linewidth}{!}{%% Creator: Matplotlib, PGF backend
%%
%% To include the figure in your LaTeX document, write
%%   \input{<filename>.pgf}
%%
%% Make sure the required packages are loaded in your preamble
%%   \usepackage{pgf}
%%
%% Figures using additional raster images can only be included by \input if
%% they are in the same directory as the main LaTeX file. For loading figures
%% from other directories you can use the `import` package
%%   \usepackage{import}
%%
%% and then include the figures with
%%   \import{<path to file>}{<filename>.pgf}
%%
%% Matplotlib used the following preamble
%%   \usepackage{fontspec}
%%   \setmainfont{DejaVuSerif.ttf}[Path=\detokenize{/Users/apple/opt/anaconda3/envs/kaolin/lib/python3.7/site-packages/matplotlib/mpl-data/fonts/ttf/}]
%%   \setsansfont{DejaVuSans.ttf}[Path=\detokenize{/Users/apple/opt/anaconda3/envs/kaolin/lib/python3.7/site-packages/matplotlib/mpl-data/fonts/ttf/}]
%%   \setmonofont{DejaVuSansMono.ttf}[Path=\detokenize{/Users/apple/opt/anaconda3/envs/kaolin/lib/python3.7/site-packages/matplotlib/mpl-data/fonts/ttf/}]
%%
\begingroup%
\makeatletter%
\begin{pgfpicture}%
\pgfpathrectangle{\pgfpointorigin}{\pgfqpoint{6.294042in}{4.697602in}}%
\pgfusepath{use as bounding box, clip}%
\begin{pgfscope}%
\pgfsetbuttcap%
\pgfsetmiterjoin%
\definecolor{currentfill}{rgb}{1.000000,1.000000,1.000000}%
\pgfsetfillcolor{currentfill}%
\pgfsetlinewidth{0.000000pt}%
\definecolor{currentstroke}{rgb}{1.000000,1.000000,1.000000}%
\pgfsetstrokecolor{currentstroke}%
\pgfsetdash{}{0pt}%
\pgfpathmoveto{\pgfqpoint{0.000000in}{0.000000in}}%
\pgfpathlineto{\pgfqpoint{6.294042in}{0.000000in}}%
\pgfpathlineto{\pgfqpoint{6.294042in}{4.697602in}}%
\pgfpathlineto{\pgfqpoint{0.000000in}{4.697602in}}%
\pgfpathclose%
\pgfusepath{fill}%
\end{pgfscope}%
\begin{pgfscope}%
\pgfsetbuttcap%
\pgfsetmiterjoin%
\definecolor{currentfill}{rgb}{1.000000,1.000000,1.000000}%
\pgfsetfillcolor{currentfill}%
\pgfsetlinewidth{0.000000pt}%
\definecolor{currentstroke}{rgb}{0.000000,0.000000,0.000000}%
\pgfsetstrokecolor{currentstroke}%
\pgfsetstrokeopacity{0.000000}%
\pgfsetdash{}{0pt}%
\pgfpathmoveto{\pgfqpoint{0.608070in}{1.223552in}}%
\pgfpathlineto{\pgfqpoint{6.194042in}{1.223552in}}%
\pgfpathlineto{\pgfqpoint{6.194042in}{4.387641in}}%
\pgfpathlineto{\pgfqpoint{0.608070in}{4.387641in}}%
\pgfpathclose%
\pgfusepath{fill}%
\end{pgfscope}%
\begin{pgfscope}%
\pgfpathrectangle{\pgfqpoint{0.608070in}{1.223552in}}{\pgfqpoint{5.585972in}{3.164089in}}%
\pgfusepath{clip}%
\pgfsetbuttcap%
\pgfsetmiterjoin%
\definecolor{currentfill}{rgb}{0.121569,0.466667,0.705882}%
\pgfsetfillcolor{currentfill}%
\pgfsetlinewidth{0.000000pt}%
\definecolor{currentstroke}{rgb}{0.000000,0.000000,0.000000}%
\pgfsetstrokecolor{currentstroke}%
\pgfsetstrokeopacity{0.000000}%
\pgfsetdash{}{0pt}%
\pgfpathmoveto{\pgfqpoint{0.861978in}{1.223552in}}%
\pgfpathlineto{\pgfqpoint{1.328339in}{1.223552in}}%
\pgfpathlineto{\pgfqpoint{1.328339in}{3.616869in}}%
\pgfpathlineto{\pgfqpoint{0.861978in}{3.616869in}}%
\pgfpathclose%
\pgfusepath{fill}%
\end{pgfscope}%
\begin{pgfscope}%
\pgfpathrectangle{\pgfqpoint{0.608070in}{1.223552in}}{\pgfqpoint{5.585972in}{3.164089in}}%
\pgfusepath{clip}%
\pgfsetbuttcap%
\pgfsetmiterjoin%
\definecolor{currentfill}{rgb}{0.121569,0.466667,0.705882}%
\pgfsetfillcolor{currentfill}%
\pgfsetlinewidth{0.000000pt}%
\definecolor{currentstroke}{rgb}{0.000000,0.000000,0.000000}%
\pgfsetstrokecolor{currentstroke}%
\pgfsetstrokeopacity{0.000000}%
\pgfsetdash{}{0pt}%
\pgfpathmoveto{\pgfqpoint{1.898336in}{1.223552in}}%
\pgfpathlineto{\pgfqpoint{2.364698in}{1.223552in}}%
\pgfpathlineto{\pgfqpoint{2.364698in}{4.021239in}}%
\pgfpathlineto{\pgfqpoint{1.898336in}{4.021239in}}%
\pgfpathclose%
\pgfusepath{fill}%
\end{pgfscope}%
\begin{pgfscope}%
\pgfpathrectangle{\pgfqpoint{0.608070in}{1.223552in}}{\pgfqpoint{5.585972in}{3.164089in}}%
\pgfusepath{clip}%
\pgfsetbuttcap%
\pgfsetmiterjoin%
\definecolor{currentfill}{rgb}{0.121569,0.466667,0.705882}%
\pgfsetfillcolor{currentfill}%
\pgfsetlinewidth{0.000000pt}%
\definecolor{currentstroke}{rgb}{0.000000,0.000000,0.000000}%
\pgfsetstrokecolor{currentstroke}%
\pgfsetstrokeopacity{0.000000}%
\pgfsetdash{}{0pt}%
\pgfpathmoveto{\pgfqpoint{2.934695in}{1.223552in}}%
\pgfpathlineto{\pgfqpoint{3.401056in}{1.223552in}}%
\pgfpathlineto{\pgfqpoint{3.401056in}{1.890542in}}%
\pgfpathlineto{\pgfqpoint{2.934695in}{1.890542in}}%
\pgfpathclose%
\pgfusepath{fill}%
\end{pgfscope}%
\begin{pgfscope}%
\pgfpathrectangle{\pgfqpoint{0.608070in}{1.223552in}}{\pgfqpoint{5.585972in}{3.164089in}}%
\pgfusepath{clip}%
\pgfsetbuttcap%
\pgfsetmiterjoin%
\definecolor{currentfill}{rgb}{0.121569,0.466667,0.705882}%
\pgfsetfillcolor{currentfill}%
\pgfsetlinewidth{0.000000pt}%
\definecolor{currentstroke}{rgb}{0.000000,0.000000,0.000000}%
\pgfsetstrokecolor{currentstroke}%
\pgfsetstrokeopacity{0.000000}%
\pgfsetdash{}{0pt}%
\pgfpathmoveto{\pgfqpoint{3.971053in}{1.223552in}}%
\pgfpathlineto{\pgfqpoint{4.437415in}{1.223552in}}%
\pgfpathlineto{\pgfqpoint{4.437415in}{2.158224in}}%
\pgfpathlineto{\pgfqpoint{3.971053in}{2.158224in}}%
\pgfpathclose%
\pgfusepath{fill}%
\end{pgfscope}%
\begin{pgfscope}%
\pgfpathrectangle{\pgfqpoint{0.608070in}{1.223552in}}{\pgfqpoint{5.585972in}{3.164089in}}%
\pgfusepath{clip}%
\pgfsetbuttcap%
\pgfsetmiterjoin%
\definecolor{currentfill}{rgb}{0.121569,0.466667,0.705882}%
\pgfsetfillcolor{currentfill}%
\pgfsetlinewidth{0.000000pt}%
\definecolor{currentstroke}{rgb}{0.000000,0.000000,0.000000}%
\pgfsetstrokecolor{currentstroke}%
\pgfsetstrokeopacity{0.000000}%
\pgfsetdash{}{0pt}%
\pgfpathmoveto{\pgfqpoint{5.007412in}{1.223552in}}%
\pgfpathlineto{\pgfqpoint{5.473773in}{1.223552in}}%
\pgfpathlineto{\pgfqpoint{5.473773in}{2.594235in}}%
\pgfpathlineto{\pgfqpoint{5.007412in}{2.594235in}}%
\pgfpathclose%
\pgfusepath{fill}%
\end{pgfscope}%
\begin{pgfscope}%
\pgfpathrectangle{\pgfqpoint{0.608070in}{1.223552in}}{\pgfqpoint{5.585972in}{3.164089in}}%
\pgfusepath{clip}%
\pgfsetbuttcap%
\pgfsetmiterjoin%
\definecolor{currentfill}{rgb}{1.000000,0.498039,0.054902}%
\pgfsetfillcolor{currentfill}%
\pgfsetlinewidth{0.000000pt}%
\definecolor{currentstroke}{rgb}{0.000000,0.000000,0.000000}%
\pgfsetstrokecolor{currentstroke}%
\pgfsetstrokeopacity{0.000000}%
\pgfsetdash{}{0pt}%
\pgfpathmoveto{\pgfqpoint{1.328339in}{1.223552in}}%
\pgfpathlineto{\pgfqpoint{1.794700in}{1.223552in}}%
\pgfpathlineto{\pgfqpoint{1.794700in}{3.809878in}}%
\pgfpathlineto{\pgfqpoint{1.328339in}{3.809878in}}%
\pgfpathclose%
\pgfusepath{fill}%
\end{pgfscope}%
\begin{pgfscope}%
\pgfpathrectangle{\pgfqpoint{0.608070in}{1.223552in}}{\pgfqpoint{5.585972in}{3.164089in}}%
\pgfusepath{clip}%
\pgfsetbuttcap%
\pgfsetmiterjoin%
\definecolor{currentfill}{rgb}{1.000000,0.498039,0.054902}%
\pgfsetfillcolor{currentfill}%
\pgfsetlinewidth{0.000000pt}%
\definecolor{currentstroke}{rgb}{0.000000,0.000000,0.000000}%
\pgfsetstrokecolor{currentstroke}%
\pgfsetstrokeopacity{0.000000}%
\pgfsetdash{}{0pt}%
\pgfpathmoveto{\pgfqpoint{2.364698in}{1.223552in}}%
\pgfpathlineto{\pgfqpoint{2.831059in}{1.223552in}}%
\pgfpathlineto{\pgfqpoint{2.831059in}{3.932645in}}%
\pgfpathlineto{\pgfqpoint{2.364698in}{3.932645in}}%
\pgfpathclose%
\pgfusepath{fill}%
\end{pgfscope}%
\begin{pgfscope}%
\pgfpathrectangle{\pgfqpoint{0.608070in}{1.223552in}}{\pgfqpoint{5.585972in}{3.164089in}}%
\pgfusepath{clip}%
\pgfsetbuttcap%
\pgfsetmiterjoin%
\definecolor{currentfill}{rgb}{1.000000,0.498039,0.054902}%
\pgfsetfillcolor{currentfill}%
\pgfsetlinewidth{0.000000pt}%
\definecolor{currentstroke}{rgb}{0.000000,0.000000,0.000000}%
\pgfsetstrokecolor{currentstroke}%
\pgfsetstrokeopacity{0.000000}%
\pgfsetdash{}{0pt}%
\pgfpathmoveto{\pgfqpoint{3.401056in}{1.223552in}}%
\pgfpathlineto{\pgfqpoint{3.867417in}{1.223552in}}%
\pgfpathlineto{\pgfqpoint{3.867417in}{1.989894in}}%
\pgfpathlineto{\pgfqpoint{3.401056in}{1.989894in}}%
\pgfpathclose%
\pgfusepath{fill}%
\end{pgfscope}%
\begin{pgfscope}%
\pgfpathrectangle{\pgfqpoint{0.608070in}{1.223552in}}{\pgfqpoint{5.585972in}{3.164089in}}%
\pgfusepath{clip}%
\pgfsetbuttcap%
\pgfsetmiterjoin%
\definecolor{currentfill}{rgb}{1.000000,0.498039,0.054902}%
\pgfsetfillcolor{currentfill}%
\pgfsetlinewidth{0.000000pt}%
\definecolor{currentstroke}{rgb}{0.000000,0.000000,0.000000}%
\pgfsetstrokecolor{currentstroke}%
\pgfsetstrokeopacity{0.000000}%
\pgfsetdash{}{0pt}%
\pgfpathmoveto{\pgfqpoint{4.437415in}{1.223552in}}%
\pgfpathlineto{\pgfqpoint{4.903776in}{1.223552in}}%
\pgfpathlineto{\pgfqpoint{4.903776in}{1.881682in}}%
\pgfpathlineto{\pgfqpoint{4.437415in}{1.881682in}}%
\pgfpathclose%
\pgfusepath{fill}%
\end{pgfscope}%
\begin{pgfscope}%
\pgfpathrectangle{\pgfqpoint{0.608070in}{1.223552in}}{\pgfqpoint{5.585972in}{3.164089in}}%
\pgfusepath{clip}%
\pgfsetbuttcap%
\pgfsetmiterjoin%
\definecolor{currentfill}{rgb}{1.000000,0.498039,0.054902}%
\pgfsetfillcolor{currentfill}%
\pgfsetlinewidth{0.000000pt}%
\definecolor{currentstroke}{rgb}{0.000000,0.000000,0.000000}%
\pgfsetstrokecolor{currentstroke}%
\pgfsetstrokeopacity{0.000000}%
\pgfsetdash{}{0pt}%
\pgfpathmoveto{\pgfqpoint{5.473773in}{1.223552in}}%
\pgfpathlineto{\pgfqpoint{5.940134in}{1.223552in}}%
\pgfpathlineto{\pgfqpoint{5.940134in}{2.913175in}}%
\pgfpathlineto{\pgfqpoint{5.473773in}{2.913175in}}%
\pgfpathclose%
\pgfusepath{fill}%
\end{pgfscope}%
\begin{pgfscope}%
\pgfsetbuttcap%
\pgfsetroundjoin%
\definecolor{currentfill}{rgb}{0.000000,0.000000,0.000000}%
\pgfsetfillcolor{currentfill}%
\pgfsetlinewidth{0.803000pt}%
\definecolor{currentstroke}{rgb}{0.000000,0.000000,0.000000}%
\pgfsetstrokecolor{currentstroke}%
\pgfsetdash{}{0pt}%
\pgfsys@defobject{currentmarker}{\pgfqpoint{0.000000in}{-0.048611in}}{\pgfqpoint{0.000000in}{0.000000in}}{%
\pgfpathmoveto{\pgfqpoint{0.000000in}{0.000000in}}%
\pgfpathlineto{\pgfqpoint{0.000000in}{-0.048611in}}%
\pgfusepath{stroke,fill}%
}%
\begin{pgfscope}%
\pgfsys@transformshift{1.328339in}{1.223552in}%
\pgfsys@useobject{currentmarker}{}%
\end{pgfscope}%
\end{pgfscope}%
\begin{pgfscope}%
\definecolor{textcolor}{rgb}{0.000000,0.000000,0.000000}%
\pgfsetstrokecolor{textcolor}%
\pgfsetfillcolor{textcolor}%
\pgftext[x=1.014314in, y=0.369475in, left, base,rotate=45.000000]{\color{textcolor}\sffamily\fontsize{10.000000}{12.000000}\selectfont Pix3d(no aug)}%
\end{pgfscope}%
\begin{pgfscope}%
\pgfsetbuttcap%
\pgfsetroundjoin%
\definecolor{currentfill}{rgb}{0.000000,0.000000,0.000000}%
\pgfsetfillcolor{currentfill}%
\pgfsetlinewidth{0.803000pt}%
\definecolor{currentstroke}{rgb}{0.000000,0.000000,0.000000}%
\pgfsetstrokecolor{currentstroke}%
\pgfsetdash{}{0pt}%
\pgfsys@defobject{currentmarker}{\pgfqpoint{0.000000in}{-0.048611in}}{\pgfqpoint{0.000000in}{0.000000in}}{%
\pgfpathmoveto{\pgfqpoint{0.000000in}{0.000000in}}%
\pgfpathlineto{\pgfqpoint{0.000000in}{-0.048611in}}%
\pgfusepath{stroke,fill}%
}%
\begin{pgfscope}%
\pgfsys@transformshift{2.364698in}{1.223552in}%
\pgfsys@useobject{currentmarker}{}%
\end{pgfscope}%
\end{pgfscope}%
\begin{pgfscope}%
\definecolor{textcolor}{rgb}{0.000000,0.000000,0.000000}%
\pgfsetstrokecolor{textcolor}%
\pgfsetfillcolor{textcolor}%
\pgftext[x=2.258144in, y=0.784419in, left, base,rotate=45.000000]{\color{textcolor}\sffamily\fontsize{10.000000}{12.000000}\selectfont Pix3d}%
\end{pgfscope}%
\begin{pgfscope}%
\pgfsetbuttcap%
\pgfsetroundjoin%
\definecolor{currentfill}{rgb}{0.000000,0.000000,0.000000}%
\pgfsetfillcolor{currentfill}%
\pgfsetlinewidth{0.803000pt}%
\definecolor{currentstroke}{rgb}{0.000000,0.000000,0.000000}%
\pgfsetstrokecolor{currentstroke}%
\pgfsetdash{}{0pt}%
\pgfsys@defobject{currentmarker}{\pgfqpoint{0.000000in}{-0.048611in}}{\pgfqpoint{0.000000in}{0.000000in}}{%
\pgfpathmoveto{\pgfqpoint{0.000000in}{0.000000in}}%
\pgfpathlineto{\pgfqpoint{0.000000in}{-0.048611in}}%
\pgfusepath{stroke,fill}%
}%
\begin{pgfscope}%
\pgfsys@transformshift{3.401056in}{1.223552in}%
\pgfsys@useobject{currentmarker}{}%
\end{pgfscope}%
\end{pgfscope}%
\begin{pgfscope}%
\definecolor{textcolor}{rgb}{0.000000,0.000000,0.000000}%
\pgfsetstrokecolor{textcolor}%
\pgfsetfillcolor{textcolor}%
\pgftext[x=3.057396in, y=0.313130in, left, base,rotate=45.000000]{\color{textcolor}\sffamily\fontsize{10.000000}{12.000000}\selectfont s2r\_v1(no aug)}%
\end{pgfscope}%
\begin{pgfscope}%
\pgfsetbuttcap%
\pgfsetroundjoin%
\definecolor{currentfill}{rgb}{0.000000,0.000000,0.000000}%
\pgfsetfillcolor{currentfill}%
\pgfsetlinewidth{0.803000pt}%
\definecolor{currentstroke}{rgb}{0.000000,0.000000,0.000000}%
\pgfsetstrokecolor{currentstroke}%
\pgfsetdash{}{0pt}%
\pgfsys@defobject{currentmarker}{\pgfqpoint{0.000000in}{-0.048611in}}{\pgfqpoint{0.000000in}{0.000000in}}{%
\pgfpathmoveto{\pgfqpoint{0.000000in}{0.000000in}}%
\pgfpathlineto{\pgfqpoint{0.000000in}{-0.048611in}}%
\pgfusepath{stroke,fill}%
}%
\begin{pgfscope}%
\pgfsys@transformshift{4.437415in}{1.223552in}%
\pgfsys@useobject{currentmarker}{}%
\end{pgfscope}%
\end{pgfscope}%
\begin{pgfscope}%
\definecolor{textcolor}{rgb}{0.000000,0.000000,0.000000}%
\pgfsetstrokecolor{textcolor}%
\pgfsetfillcolor{textcolor}%
\pgftext[x=4.300411in, y=0.729704in, left, base,rotate=45.000000]{\color{textcolor}\sffamily\fontsize{10.000000}{12.000000}\selectfont s2r\_v1}%
\end{pgfscope}%
\begin{pgfscope}%
\pgfsetbuttcap%
\pgfsetroundjoin%
\definecolor{currentfill}{rgb}{0.000000,0.000000,0.000000}%
\pgfsetfillcolor{currentfill}%
\pgfsetlinewidth{0.803000pt}%
\definecolor{currentstroke}{rgb}{0.000000,0.000000,0.000000}%
\pgfsetstrokecolor{currentstroke}%
\pgfsetdash{}{0pt}%
\pgfsys@defobject{currentmarker}{\pgfqpoint{0.000000in}{-0.048611in}}{\pgfqpoint{0.000000in}{0.000000in}}{%
\pgfpathmoveto{\pgfqpoint{0.000000in}{0.000000in}}%
\pgfpathlineto{\pgfqpoint{0.000000in}{-0.048611in}}%
\pgfusepath{stroke,fill}%
}%
\begin{pgfscope}%
\pgfsys@transformshift{5.473773in}{1.223552in}%
\pgfsys@useobject{currentmarker}{}%
\end{pgfscope}%
\end{pgfscope}%
\begin{pgfscope}%
\definecolor{textcolor}{rgb}{0.000000,0.000000,0.000000}%
\pgfsetstrokecolor{textcolor}%
\pgfsetfillcolor{textcolor}%
\pgftext[x=5.336769in, y=0.729704in, left, base,rotate=45.000000]{\color{textcolor}\sffamily\fontsize{10.000000}{12.000000}\selectfont s2r\_v2}%
\end{pgfscope}%
\begin{pgfscope}%
\definecolor{textcolor}{rgb}{0.000000,0.000000,0.000000}%
\pgfsetstrokecolor{textcolor}%
\pgfsetfillcolor{textcolor}%
\pgftext[x=3.401056in,y=0.234413in,,top]{\color{textcolor}\sffamily\fontsize{10.000000}{12.000000}\selectfont Dataset}%
\end{pgfscope}%
\begin{pgfscope}%
\pgfsetbuttcap%
\pgfsetroundjoin%
\definecolor{currentfill}{rgb}{0.000000,0.000000,0.000000}%
\pgfsetfillcolor{currentfill}%
\pgfsetlinewidth{0.803000pt}%
\definecolor{currentstroke}{rgb}{0.000000,0.000000,0.000000}%
\pgfsetstrokecolor{currentstroke}%
\pgfsetdash{}{0pt}%
\pgfsys@defobject{currentmarker}{\pgfqpoint{-0.048611in}{0.000000in}}{\pgfqpoint{-0.000000in}{0.000000in}}{%
\pgfpathmoveto{\pgfqpoint{-0.000000in}{0.000000in}}%
\pgfpathlineto{\pgfqpoint{-0.048611in}{0.000000in}}%
\pgfusepath{stroke,fill}%
}%
\begin{pgfscope}%
\pgfsys@transformshift{0.608070in}{1.223552in}%
\pgfsys@useobject{currentmarker}{}%
\end{pgfscope}%
\end{pgfscope}%
\begin{pgfscope}%
\definecolor{textcolor}{rgb}{0.000000,0.000000,0.000000}%
\pgfsetstrokecolor{textcolor}%
\pgfsetfillcolor{textcolor}%
\pgftext[x=0.289968in, y=1.170790in, left, base]{\color{textcolor}\sffamily\fontsize{10.000000}{12.000000}\selectfont 0.0}%
\end{pgfscope}%
\begin{pgfscope}%
\pgfsetbuttcap%
\pgfsetroundjoin%
\definecolor{currentfill}{rgb}{0.000000,0.000000,0.000000}%
\pgfsetfillcolor{currentfill}%
\pgfsetlinewidth{0.803000pt}%
\definecolor{currentstroke}{rgb}{0.000000,0.000000,0.000000}%
\pgfsetstrokecolor{currentstroke}%
\pgfsetdash{}{0pt}%
\pgfsys@defobject{currentmarker}{\pgfqpoint{-0.048611in}{0.000000in}}{\pgfqpoint{-0.000000in}{0.000000in}}{%
\pgfpathmoveto{\pgfqpoint{-0.000000in}{0.000000in}}%
\pgfpathlineto{\pgfqpoint{-0.048611in}{0.000000in}}%
\pgfusepath{stroke,fill}%
}%
\begin{pgfscope}%
\pgfsys@transformshift{0.608070in}{1.856369in}%
\pgfsys@useobject{currentmarker}{}%
\end{pgfscope}%
\end{pgfscope}%
\begin{pgfscope}%
\definecolor{textcolor}{rgb}{0.000000,0.000000,0.000000}%
\pgfsetstrokecolor{textcolor}%
\pgfsetfillcolor{textcolor}%
\pgftext[x=0.289968in, y=1.803608in, left, base]{\color{textcolor}\sffamily\fontsize{10.000000}{12.000000}\selectfont 0.1}%
\end{pgfscope}%
\begin{pgfscope}%
\pgfsetbuttcap%
\pgfsetroundjoin%
\definecolor{currentfill}{rgb}{0.000000,0.000000,0.000000}%
\pgfsetfillcolor{currentfill}%
\pgfsetlinewidth{0.803000pt}%
\definecolor{currentstroke}{rgb}{0.000000,0.000000,0.000000}%
\pgfsetstrokecolor{currentstroke}%
\pgfsetdash{}{0pt}%
\pgfsys@defobject{currentmarker}{\pgfqpoint{-0.048611in}{0.000000in}}{\pgfqpoint{-0.000000in}{0.000000in}}{%
\pgfpathmoveto{\pgfqpoint{-0.000000in}{0.000000in}}%
\pgfpathlineto{\pgfqpoint{-0.048611in}{0.000000in}}%
\pgfusepath{stroke,fill}%
}%
\begin{pgfscope}%
\pgfsys@transformshift{0.608070in}{2.489187in}%
\pgfsys@useobject{currentmarker}{}%
\end{pgfscope}%
\end{pgfscope}%
\begin{pgfscope}%
\definecolor{textcolor}{rgb}{0.000000,0.000000,0.000000}%
\pgfsetstrokecolor{textcolor}%
\pgfsetfillcolor{textcolor}%
\pgftext[x=0.289968in, y=2.436426in, left, base]{\color{textcolor}\sffamily\fontsize{10.000000}{12.000000}\selectfont 0.2}%
\end{pgfscope}%
\begin{pgfscope}%
\pgfsetbuttcap%
\pgfsetroundjoin%
\definecolor{currentfill}{rgb}{0.000000,0.000000,0.000000}%
\pgfsetfillcolor{currentfill}%
\pgfsetlinewidth{0.803000pt}%
\definecolor{currentstroke}{rgb}{0.000000,0.000000,0.000000}%
\pgfsetstrokecolor{currentstroke}%
\pgfsetdash{}{0pt}%
\pgfsys@defobject{currentmarker}{\pgfqpoint{-0.048611in}{0.000000in}}{\pgfqpoint{-0.000000in}{0.000000in}}{%
\pgfpathmoveto{\pgfqpoint{-0.000000in}{0.000000in}}%
\pgfpathlineto{\pgfqpoint{-0.048611in}{0.000000in}}%
\pgfusepath{stroke,fill}%
}%
\begin{pgfscope}%
\pgfsys@transformshift{0.608070in}{3.122005in}%
\pgfsys@useobject{currentmarker}{}%
\end{pgfscope}%
\end{pgfscope}%
\begin{pgfscope}%
\definecolor{textcolor}{rgb}{0.000000,0.000000,0.000000}%
\pgfsetstrokecolor{textcolor}%
\pgfsetfillcolor{textcolor}%
\pgftext[x=0.289968in, y=3.069244in, left, base]{\color{textcolor}\sffamily\fontsize{10.000000}{12.000000}\selectfont 0.3}%
\end{pgfscope}%
\begin{pgfscope}%
\pgfsetbuttcap%
\pgfsetroundjoin%
\definecolor{currentfill}{rgb}{0.000000,0.000000,0.000000}%
\pgfsetfillcolor{currentfill}%
\pgfsetlinewidth{0.803000pt}%
\definecolor{currentstroke}{rgb}{0.000000,0.000000,0.000000}%
\pgfsetstrokecolor{currentstroke}%
\pgfsetdash{}{0pt}%
\pgfsys@defobject{currentmarker}{\pgfqpoint{-0.048611in}{0.000000in}}{\pgfqpoint{-0.000000in}{0.000000in}}{%
\pgfpathmoveto{\pgfqpoint{-0.000000in}{0.000000in}}%
\pgfpathlineto{\pgfqpoint{-0.048611in}{0.000000in}}%
\pgfusepath{stroke,fill}%
}%
\begin{pgfscope}%
\pgfsys@transformshift{0.608070in}{3.754823in}%
\pgfsys@useobject{currentmarker}{}%
\end{pgfscope}%
\end{pgfscope}%
\begin{pgfscope}%
\definecolor{textcolor}{rgb}{0.000000,0.000000,0.000000}%
\pgfsetstrokecolor{textcolor}%
\pgfsetfillcolor{textcolor}%
\pgftext[x=0.289968in, y=3.702061in, left, base]{\color{textcolor}\sffamily\fontsize{10.000000}{12.000000}\selectfont 0.4}%
\end{pgfscope}%
\begin{pgfscope}%
\pgfsetbuttcap%
\pgfsetroundjoin%
\definecolor{currentfill}{rgb}{0.000000,0.000000,0.000000}%
\pgfsetfillcolor{currentfill}%
\pgfsetlinewidth{0.803000pt}%
\definecolor{currentstroke}{rgb}{0.000000,0.000000,0.000000}%
\pgfsetstrokecolor{currentstroke}%
\pgfsetdash{}{0pt}%
\pgfsys@defobject{currentmarker}{\pgfqpoint{-0.048611in}{0.000000in}}{\pgfqpoint{-0.000000in}{0.000000in}}{%
\pgfpathmoveto{\pgfqpoint{-0.000000in}{0.000000in}}%
\pgfpathlineto{\pgfqpoint{-0.048611in}{0.000000in}}%
\pgfusepath{stroke,fill}%
}%
\begin{pgfscope}%
\pgfsys@transformshift{0.608070in}{4.387641in}%
\pgfsys@useobject{currentmarker}{}%
\end{pgfscope}%
\end{pgfscope}%
\begin{pgfscope}%
\definecolor{textcolor}{rgb}{0.000000,0.000000,0.000000}%
\pgfsetstrokecolor{textcolor}%
\pgfsetfillcolor{textcolor}%
\pgftext[x=0.289968in, y=4.334879in, left, base]{\color{textcolor}\sffamily\fontsize{10.000000}{12.000000}\selectfont 0.5}%
\end{pgfscope}%
\begin{pgfscope}%
\definecolor{textcolor}{rgb}{0.000000,0.000000,0.000000}%
\pgfsetstrokecolor{textcolor}%
\pgfsetfillcolor{textcolor}%
\pgftext[x=0.234413in,y=2.805596in,,bottom,rotate=90.000000]{\color{textcolor}\sffamily\fontsize{10.000000}{12.000000}\selectfont F1 Score}%
\end{pgfscope}%
\begin{pgfscope}%
\pgfsetrectcap%
\pgfsetmiterjoin%
\pgfsetlinewidth{0.803000pt}%
\definecolor{currentstroke}{rgb}{0.000000,0.000000,0.000000}%
\pgfsetstrokecolor{currentstroke}%
\pgfsetdash{}{0pt}%
\pgfpathmoveto{\pgfqpoint{0.608070in}{1.223552in}}%
\pgfpathlineto{\pgfqpoint{0.608070in}{4.387641in}}%
\pgfusepath{stroke}%
\end{pgfscope}%
\begin{pgfscope}%
\pgfsetrectcap%
\pgfsetmiterjoin%
\pgfsetlinewidth{0.803000pt}%
\definecolor{currentstroke}{rgb}{0.000000,0.000000,0.000000}%
\pgfsetstrokecolor{currentstroke}%
\pgfsetdash{}{0pt}%
\pgfpathmoveto{\pgfqpoint{6.194042in}{1.223552in}}%
\pgfpathlineto{\pgfqpoint{6.194042in}{4.387641in}}%
\pgfusepath{stroke}%
\end{pgfscope}%
\begin{pgfscope}%
\pgfsetrectcap%
\pgfsetmiterjoin%
\pgfsetlinewidth{0.803000pt}%
\definecolor{currentstroke}{rgb}{0.000000,0.000000,0.000000}%
\pgfsetstrokecolor{currentstroke}%
\pgfsetdash{}{0pt}%
\pgfpathmoveto{\pgfqpoint{0.608070in}{1.223552in}}%
\pgfpathlineto{\pgfqpoint{6.194042in}{1.223552in}}%
\pgfusepath{stroke}%
\end{pgfscope}%
\begin{pgfscope}%
\pgfsetrectcap%
\pgfsetmiterjoin%
\pgfsetlinewidth{0.803000pt}%
\definecolor{currentstroke}{rgb}{0.000000,0.000000,0.000000}%
\pgfsetstrokecolor{currentstroke}%
\pgfsetdash{}{0pt}%
\pgfpathmoveto{\pgfqpoint{0.608070in}{4.387641in}}%
\pgfpathlineto{\pgfqpoint{6.194042in}{4.387641in}}%
\pgfusepath{stroke}%
\end{pgfscope}%
\begin{pgfscope}%
\definecolor{textcolor}{rgb}{0.000000,0.000000,0.000000}%
\pgfsetstrokecolor{textcolor}%
\pgfsetfillcolor{textcolor}%
\pgftext[x=1.095158in,y=3.658535in,,bottom]{\color{textcolor}\sffamily\fontsize{9.000000}{10.800000}\selectfont 0.3782}%
\end{pgfscope}%
\begin{pgfscope}%
\definecolor{textcolor}{rgb}{0.000000,0.000000,0.000000}%
\pgfsetstrokecolor{textcolor}%
\pgfsetfillcolor{textcolor}%
\pgftext[x=2.131517in,y=4.062906in,,bottom]{\color{textcolor}\sffamily\fontsize{9.000000}{10.800000}\selectfont 0.4421}%
\end{pgfscope}%
\begin{pgfscope}%
\definecolor{textcolor}{rgb}{0.000000,0.000000,0.000000}%
\pgfsetstrokecolor{textcolor}%
\pgfsetfillcolor{textcolor}%
\pgftext[x=3.167875in,y=1.932208in,,bottom]{\color{textcolor}\sffamily\fontsize{9.000000}{10.800000}\selectfont 0.1054}%
\end{pgfscope}%
\begin{pgfscope}%
\definecolor{textcolor}{rgb}{0.000000,0.000000,0.000000}%
\pgfsetstrokecolor{textcolor}%
\pgfsetfillcolor{textcolor}%
\pgftext[x=4.204234in,y=2.199890in,,bottom]{\color{textcolor}\sffamily\fontsize{9.000000}{10.800000}\selectfont 0.1477}%
\end{pgfscope}%
\begin{pgfscope}%
\definecolor{textcolor}{rgb}{0.000000,0.000000,0.000000}%
\pgfsetstrokecolor{textcolor}%
\pgfsetfillcolor{textcolor}%
\pgftext[x=5.240592in,y=2.635902in,,bottom]{\color{textcolor}\sffamily\fontsize{9.000000}{10.800000}\selectfont 0.2166}%
\end{pgfscope}%
\begin{pgfscope}%
\definecolor{textcolor}{rgb}{0.000000,0.000000,0.000000}%
\pgfsetstrokecolor{textcolor}%
\pgfsetfillcolor{textcolor}%
\pgftext[x=1.561520in,y=3.851545in,,bottom]{\color{textcolor}\sffamily\fontsize{9.000000}{10.800000}\selectfont 0.4087}%
\end{pgfscope}%
\begin{pgfscope}%
\definecolor{textcolor}{rgb}{0.000000,0.000000,0.000000}%
\pgfsetstrokecolor{textcolor}%
\pgfsetfillcolor{textcolor}%
\pgftext[x=2.597878in,y=3.974311in,,bottom]{\color{textcolor}\sffamily\fontsize{9.000000}{10.800000}\selectfont 0.4281}%
\end{pgfscope}%
\begin{pgfscope}%
\definecolor{textcolor}{rgb}{0.000000,0.000000,0.000000}%
\pgfsetstrokecolor{textcolor}%
\pgfsetfillcolor{textcolor}%
\pgftext[x=3.634237in,y=2.031561in,,bottom]{\color{textcolor}\sffamily\fontsize{9.000000}{10.800000}\selectfont 0.1211}%
\end{pgfscope}%
\begin{pgfscope}%
\definecolor{textcolor}{rgb}{0.000000,0.000000,0.000000}%
\pgfsetstrokecolor{textcolor}%
\pgfsetfillcolor{textcolor}%
\pgftext[x=4.670595in,y=1.923349in,,bottom]{\color{textcolor}\sffamily\fontsize{9.000000}{10.800000}\selectfont 0.104}%
\end{pgfscope}%
\begin{pgfscope}%
\definecolor{textcolor}{rgb}{0.000000,0.000000,0.000000}%
\pgfsetstrokecolor{textcolor}%
\pgfsetfillcolor{textcolor}%
\pgftext[x=5.706954in,y=2.954842in,,bottom]{\color{textcolor}\sffamily\fontsize{9.000000}{10.800000}\selectfont 0.267}%
\end{pgfscope}%
\begin{pgfscope}%
\definecolor{textcolor}{rgb}{0.000000,0.000000,0.000000}%
\pgfsetstrokecolor{textcolor}%
\pgfsetfillcolor{textcolor}%
\pgftext[x=3.401056in,y=4.470974in,,base]{\color{textcolor}\sffamily\fontsize{12.000000}{14.400000}\selectfont Baselines trained on Pix3D and S2R:3DFREE}%
\end{pgfscope}%
\begin{pgfscope}%
\pgfsetbuttcap%
\pgfsetmiterjoin%
\definecolor{currentfill}{rgb}{1.000000,1.000000,1.000000}%
\pgfsetfillcolor{currentfill}%
\pgfsetfillopacity{0.800000}%
\pgfsetlinewidth{1.003750pt}%
\definecolor{currentstroke}{rgb}{0.800000,0.800000,0.800000}%
\pgfsetstrokecolor{currentstroke}%
\pgfsetstrokeopacity{0.800000}%
\pgfsetdash{}{0pt}%
\pgfpathmoveto{\pgfqpoint{4.882654in}{3.868815in}}%
\pgfpathlineto{\pgfqpoint{6.096820in}{3.868815in}}%
\pgfpathquadraticcurveto{\pgfqpoint{6.124598in}{3.868815in}}{\pgfqpoint{6.124598in}{3.896593in}}%
\pgfpathlineto{\pgfqpoint{6.124598in}{4.290419in}}%
\pgfpathquadraticcurveto{\pgfqpoint{6.124598in}{4.318196in}}{\pgfqpoint{6.096820in}{4.318196in}}%
\pgfpathlineto{\pgfqpoint{4.882654in}{4.318196in}}%
\pgfpathquadraticcurveto{\pgfqpoint{4.854877in}{4.318196in}}{\pgfqpoint{4.854877in}{4.290419in}}%
\pgfpathlineto{\pgfqpoint{4.854877in}{3.896593in}}%
\pgfpathquadraticcurveto{\pgfqpoint{4.854877in}{3.868815in}}{\pgfqpoint{4.882654in}{3.868815in}}%
\pgfpathclose%
\pgfusepath{stroke,fill}%
\end{pgfscope}%
\begin{pgfscope}%
\pgfsetbuttcap%
\pgfsetmiterjoin%
\definecolor{currentfill}{rgb}{0.121569,0.466667,0.705882}%
\pgfsetfillcolor{currentfill}%
\pgfsetlinewidth{0.000000pt}%
\definecolor{currentstroke}{rgb}{0.000000,0.000000,0.000000}%
\pgfsetstrokecolor{currentstroke}%
\pgfsetstrokeopacity{0.000000}%
\pgfsetdash{}{0pt}%
\pgfpathmoveto{\pgfqpoint{4.910432in}{4.157118in}}%
\pgfpathlineto{\pgfqpoint{5.188210in}{4.157118in}}%
\pgfpathlineto{\pgfqpoint{5.188210in}{4.254340in}}%
\pgfpathlineto{\pgfqpoint{4.910432in}{4.254340in}}%
\pgfpathclose%
\pgfusepath{fill}%
\end{pgfscope}%
\begin{pgfscope}%
\definecolor{textcolor}{rgb}{0.000000,0.000000,0.000000}%
\pgfsetstrokecolor{textcolor}%
\pgfsetfillcolor{textcolor}%
\pgftext[x=5.299321in,y=4.157118in,left,base]{\color{textcolor}\sffamily\fontsize{10.000000}{12.000000}\selectfont Pix2Vox++}%
\end{pgfscope}%
\begin{pgfscope}%
\pgfsetbuttcap%
\pgfsetmiterjoin%
\definecolor{currentfill}{rgb}{1.000000,0.498039,0.054902}%
\pgfsetfillcolor{currentfill}%
\pgfsetlinewidth{0.000000pt}%
\definecolor{currentstroke}{rgb}{0.000000,0.000000,0.000000}%
\pgfsetstrokecolor{currentstroke}%
\pgfsetstrokeopacity{0.000000}%
\pgfsetdash{}{0pt}%
\pgfpathmoveto{\pgfqpoint{4.910432in}{3.953260in}}%
\pgfpathlineto{\pgfqpoint{5.188210in}{3.953260in}}%
\pgfpathlineto{\pgfqpoint{5.188210in}{4.050483in}}%
\pgfpathlineto{\pgfqpoint{4.910432in}{4.050483in}}%
\pgfpathclose%
\pgfusepath{fill}%
\end{pgfscope}%
\begin{pgfscope}%
\definecolor{textcolor}{rgb}{0.000000,0.000000,0.000000}%
\pgfsetstrokecolor{textcolor}%
\pgfsetfillcolor{textcolor}%
\pgftext[x=5.299321in,y=3.953260in,left,base]{\color{textcolor}\sffamily\fontsize{10.000000}{12.000000}\selectfont Pix2Vox}%
\end{pgfscope}%
\end{pgfpicture}%
\makeatother%
\endgroup%
}
    \caption{Bar plot for the \gls{f1}  for baselines trained on real and synthetic datasets, with and without 2D augmentation.
    We see that ~\gls{free} does not perform adequately on its own. \gls{s2rv2} contributes slightly better than \gls{s2rv1}.}
    \label{fig:baseline_dice1}
\end{figure}

There is a difference of 6.39\% and 1.94\% for Pix2vox++ and Pix2vox when trained with and without 2D augmentation, which shows its importance.
There is a massive dip in performance when we train the models only on the proposed synthetic dataset (\gls(s2rv1) and \gls(s2rv2)).
Among the two versions, \gls(s2rv2) 6.8\% and 1.6\% more than \gls{s2rv1}, for pix2vox++ and pix2vox, respectively, proving that multi-object is the better dataset.
%
%\begin{figure}
%    \centering
%    \resizebox{0.6\linewidth}{!}{\input{/Users/apple/OVGU/Thesis/code/3dReconstruction/report/images/evaluation/performance/baseline_dice_linegraph1.pgf}}
%    \caption{Line plot for the \gls{f1}  for baselines trained on real and synthetic datasets, with and without 2D augmentation.
%        We see that ~\gls{free} does not perform adequately on its own. \gls{s2rv2} contributes slightly better than \gls{s2rv1}.}
%    \label{fig:baseline_dice1}
%\end{figure}

\subsection{Fine-Tuning}\label{subsec:fine-tuning-dice}

This section is contemporary of what we experimented in \autoref{sec:fine-tuning}.
\autoref{fig:finetuning_dice1} compares models trained with and without fine-tuning with real datasets and models trained on only a real dataset.
We see an increment of only 0.3\% for pix2vox trained on \gls{s2rv2} and fine-tuned with Pix3d.
In all other cases, the performance is decreased instead of improving.

We see the same results in  \autoref{fig:finetuning_dice2} and \autoref{fig:finetuning_dice3},
where the \gls{f1} for each category with fine-tuned models is less than the models trained only on the real dataset.
%
%\begin{figure}
%    \centering
%    \resizebox{0.6\textwidth}{!}{%% Creator: Matplotlib, PGF backend
%%
%% To include the figure in your LaTeX document, write
%%   \input{<filename>.pgf}
%%
%% Make sure the required packages are loaded in your preamble
%%   \usepackage{pgf}
%%
%% Figures using additional raster images can only be included by \input if
%% they are in the same directory as the main LaTeX file. For loading figures
%% from other directories you can use the `import` package
%%   \usepackage{import}
%%
%% and then include the figures with
%%   \import{<path to file>}{<filename>.pgf}
%%
%% Matplotlib used the following preamble
%%   \usepackage{fontspec}
%%   \setmainfont{DejaVuSerif.ttf}[Path=\detokenize{/Users/apple/opt/anaconda3/envs/kaolin/lib/python3.7/site-packages/matplotlib/mpl-data/fonts/ttf/}]
%%   \setsansfont{DejaVuSans.ttf}[Path=\detokenize{/Users/apple/opt/anaconda3/envs/kaolin/lib/python3.7/site-packages/matplotlib/mpl-data/fonts/ttf/}]
%%   \setmonofont{DejaVuSansMono.ttf}[Path=\detokenize{/Users/apple/opt/anaconda3/envs/kaolin/lib/python3.7/site-packages/matplotlib/mpl-data/fonts/ttf/}]
%%
\begingroup%
\makeatletter%
\begin{pgfpicture}%
\pgfpathrectangle{\pgfpointorigin}{\pgfqpoint{5.830801in}{5.012323in}}%
\pgfusepath{use as bounding box, clip}%
\begin{pgfscope}%
\pgfsetbuttcap%
\pgfsetmiterjoin%
\definecolor{currentfill}{rgb}{1.000000,1.000000,1.000000}%
\pgfsetfillcolor{currentfill}%
\pgfsetlinewidth{0.000000pt}%
\definecolor{currentstroke}{rgb}{1.000000,1.000000,1.000000}%
\pgfsetstrokecolor{currentstroke}%
\pgfsetdash{}{0pt}%
\pgfpathmoveto{\pgfqpoint{0.000000in}{0.000000in}}%
\pgfpathlineto{\pgfqpoint{5.830801in}{0.000000in}}%
\pgfpathlineto{\pgfqpoint{5.830801in}{5.012323in}}%
\pgfpathlineto{\pgfqpoint{0.000000in}{5.012323in}}%
\pgfpathclose%
\pgfusepath{fill}%
\end{pgfscope}%
\begin{pgfscope}%
\pgfsetbuttcap%
\pgfsetmiterjoin%
\definecolor{currentfill}{rgb}{1.000000,1.000000,1.000000}%
\pgfsetfillcolor{currentfill}%
\pgfsetlinewidth{0.000000pt}%
\definecolor{currentstroke}{rgb}{0.000000,0.000000,0.000000}%
\pgfsetstrokecolor{currentstroke}%
\pgfsetstrokeopacity{0.000000}%
\pgfsetdash{}{0pt}%
\pgfpathmoveto{\pgfqpoint{0.608070in}{1.163562in}}%
\pgfpathlineto{\pgfqpoint{5.568070in}{1.163562in}}%
\pgfpathlineto{\pgfqpoint{5.568070in}{4.859562in}}%
\pgfpathlineto{\pgfqpoint{0.608070in}{4.859562in}}%
\pgfpathclose%
\pgfusepath{fill}%
\end{pgfscope}%
\begin{pgfscope}%
\pgfsetbuttcap%
\pgfsetroundjoin%
\definecolor{currentfill}{rgb}{0.000000,0.000000,0.000000}%
\pgfsetfillcolor{currentfill}%
\pgfsetlinewidth{0.803000pt}%
\definecolor{currentstroke}{rgb}{0.000000,0.000000,0.000000}%
\pgfsetstrokecolor{currentstroke}%
\pgfsetdash{}{0pt}%
\pgfsys@defobject{currentmarker}{\pgfqpoint{0.000000in}{-0.048611in}}{\pgfqpoint{0.000000in}{0.000000in}}{%
\pgfpathmoveto{\pgfqpoint{0.000000in}{0.000000in}}%
\pgfpathlineto{\pgfqpoint{0.000000in}{-0.048611in}}%
\pgfusepath{stroke,fill}%
}%
\begin{pgfscope}%
\pgfsys@transformshift{0.833525in}{1.163562in}%
\pgfsys@useobject{currentmarker}{}%
\end{pgfscope}%
\end{pgfscope}%
\begin{pgfscope}%
\definecolor{textcolor}{rgb}{0.000000,0.000000,0.000000}%
\pgfsetstrokecolor{textcolor}%
\pgfsetfillcolor{textcolor}%
\pgftext[x=0.720330in, y=0.711146in, left, base,rotate=45.000000]{\color{textcolor}\sffamily\fontsize{10.000000}{12.000000}\selectfont Pix3D}%
\end{pgfscope}%
\begin{pgfscope}%
\pgfsetbuttcap%
\pgfsetroundjoin%
\definecolor{currentfill}{rgb}{0.000000,0.000000,0.000000}%
\pgfsetfillcolor{currentfill}%
\pgfsetlinewidth{0.803000pt}%
\definecolor{currentstroke}{rgb}{0.000000,0.000000,0.000000}%
\pgfsetstrokecolor{currentstroke}%
\pgfsetdash{}{0pt}%
\pgfsys@defobject{currentmarker}{\pgfqpoint{0.000000in}{-0.048611in}}{\pgfqpoint{0.000000in}{0.000000in}}{%
\pgfpathmoveto{\pgfqpoint{0.000000in}{0.000000in}}%
\pgfpathlineto{\pgfqpoint{0.000000in}{-0.048611in}}%
\pgfusepath{stroke,fill}%
}%
\begin{pgfscope}%
\pgfsys@transformshift{1.960797in}{1.163562in}%
\pgfsys@useobject{currentmarker}{}%
\end{pgfscope}%
\end{pgfscope}%
\begin{pgfscope}%
\definecolor{textcolor}{rgb}{0.000000,0.000000,0.000000}%
\pgfsetstrokecolor{textcolor}%
\pgfsetfillcolor{textcolor}%
\pgftext[x=1.823794in, y=0.669714in, left, base,rotate=45.000000]{\color{textcolor}\sffamily\fontsize{10.000000}{12.000000}\selectfont s2r\_v1}%
\end{pgfscope}%
\begin{pgfscope}%
\pgfsetbuttcap%
\pgfsetroundjoin%
\definecolor{currentfill}{rgb}{0.000000,0.000000,0.000000}%
\pgfsetfillcolor{currentfill}%
\pgfsetlinewidth{0.803000pt}%
\definecolor{currentstroke}{rgb}{0.000000,0.000000,0.000000}%
\pgfsetstrokecolor{currentstroke}%
\pgfsetdash{}{0pt}%
\pgfsys@defobject{currentmarker}{\pgfqpoint{0.000000in}{-0.048611in}}{\pgfqpoint{0.000000in}{0.000000in}}{%
\pgfpathmoveto{\pgfqpoint{0.000000in}{0.000000in}}%
\pgfpathlineto{\pgfqpoint{0.000000in}{-0.048611in}}%
\pgfusepath{stroke,fill}%
}%
\begin{pgfscope}%
\pgfsys@transformshift{3.088070in}{1.163562in}%
\pgfsys@useobject{currentmarker}{}%
\end{pgfscope}%
\end{pgfscope}%
\begin{pgfscope}%
\definecolor{textcolor}{rgb}{0.000000,0.000000,0.000000}%
\pgfsetstrokecolor{textcolor}%
\pgfsetfillcolor{textcolor}%
\pgftext[x=2.951066in, y=0.669714in, left, base,rotate=45.000000]{\color{textcolor}\sffamily\fontsize{10.000000}{12.000000}\selectfont s2r\_v2}%
\end{pgfscope}%
\begin{pgfscope}%
\pgfsetbuttcap%
\pgfsetroundjoin%
\definecolor{currentfill}{rgb}{0.000000,0.000000,0.000000}%
\pgfsetfillcolor{currentfill}%
\pgfsetlinewidth{0.803000pt}%
\definecolor{currentstroke}{rgb}{0.000000,0.000000,0.000000}%
\pgfsetstrokecolor{currentstroke}%
\pgfsetdash{}{0pt}%
\pgfsys@defobject{currentmarker}{\pgfqpoint{0.000000in}{-0.048611in}}{\pgfqpoint{0.000000in}{0.000000in}}{%
\pgfpathmoveto{\pgfqpoint{0.000000in}{0.000000in}}%
\pgfpathlineto{\pgfqpoint{0.000000in}{-0.048611in}}%
\pgfusepath{stroke,fill}%
}%
\begin{pgfscope}%
\pgfsys@transformshift{4.215343in}{1.163562in}%
\pgfsys@useobject{currentmarker}{}%
\end{pgfscope}%
\end{pgfscope}%
\begin{pgfscope}%
\definecolor{textcolor}{rgb}{0.000000,0.000000,0.000000}%
\pgfsetstrokecolor{textcolor}%
\pgfsetfillcolor{textcolor}%
\pgftext[x=3.901773in, y=0.313130in, left, base,rotate=45.000000]{\color{textcolor}\sffamily\fontsize{10.000000}{12.000000}\selectfont s2r\_v1+pix3d}%
\end{pgfscope}%
\begin{pgfscope}%
\pgfsetbuttcap%
\pgfsetroundjoin%
\definecolor{currentfill}{rgb}{0.000000,0.000000,0.000000}%
\pgfsetfillcolor{currentfill}%
\pgfsetlinewidth{0.803000pt}%
\definecolor{currentstroke}{rgb}{0.000000,0.000000,0.000000}%
\pgfsetstrokecolor{currentstroke}%
\pgfsetdash{}{0pt}%
\pgfsys@defobject{currentmarker}{\pgfqpoint{0.000000in}{-0.048611in}}{\pgfqpoint{0.000000in}{0.000000in}}{%
\pgfpathmoveto{\pgfqpoint{0.000000in}{0.000000in}}%
\pgfpathlineto{\pgfqpoint{0.000000in}{-0.048611in}}%
\pgfusepath{stroke,fill}%
}%
\begin{pgfscope}%
\pgfsys@transformshift{5.342615in}{1.163562in}%
\pgfsys@useobject{currentmarker}{}%
\end{pgfscope}%
\end{pgfscope}%
\begin{pgfscope}%
\definecolor{textcolor}{rgb}{0.000000,0.000000,0.000000}%
\pgfsetstrokecolor{textcolor}%
\pgfsetfillcolor{textcolor}%
\pgftext[x=5.029046in, y=0.313130in, left, base,rotate=45.000000]{\color{textcolor}\sffamily\fontsize{10.000000}{12.000000}\selectfont s2r\_v2+pix3d}%
\end{pgfscope}%
\begin{pgfscope}%
\definecolor{textcolor}{rgb}{0.000000,0.000000,0.000000}%
\pgfsetstrokecolor{textcolor}%
\pgfsetfillcolor{textcolor}%
\pgftext[x=3.088070in,y=0.234413in,,top]{\color{textcolor}\sffamily\fontsize{10.000000}{12.000000}\selectfont Datasets}%
\end{pgfscope}%
\begin{pgfscope}%
\pgfsetbuttcap%
\pgfsetroundjoin%
\definecolor{currentfill}{rgb}{0.000000,0.000000,0.000000}%
\pgfsetfillcolor{currentfill}%
\pgfsetlinewidth{0.803000pt}%
\definecolor{currentstroke}{rgb}{0.000000,0.000000,0.000000}%
\pgfsetstrokecolor{currentstroke}%
\pgfsetdash{}{0pt}%
\pgfsys@defobject{currentmarker}{\pgfqpoint{-0.048611in}{0.000000in}}{\pgfqpoint{-0.000000in}{0.000000in}}{%
\pgfpathmoveto{\pgfqpoint{-0.000000in}{0.000000in}}%
\pgfpathlineto{\pgfqpoint{-0.048611in}{0.000000in}}%
\pgfusepath{stroke,fill}%
}%
\begin{pgfscope}%
\pgfsys@transformshift{0.608070in}{1.163562in}%
\pgfsys@useobject{currentmarker}{}%
\end{pgfscope}%
\end{pgfscope}%
\begin{pgfscope}%
\definecolor{textcolor}{rgb}{0.000000,0.000000,0.000000}%
\pgfsetstrokecolor{textcolor}%
\pgfsetfillcolor{textcolor}%
\pgftext[x=0.289968in, y=1.110800in, left, base]{\color{textcolor}\sffamily\fontsize{10.000000}{12.000000}\selectfont 0.0}%
\end{pgfscope}%
\begin{pgfscope}%
\pgfsetbuttcap%
\pgfsetroundjoin%
\definecolor{currentfill}{rgb}{0.000000,0.000000,0.000000}%
\pgfsetfillcolor{currentfill}%
\pgfsetlinewidth{0.803000pt}%
\definecolor{currentstroke}{rgb}{0.000000,0.000000,0.000000}%
\pgfsetstrokecolor{currentstroke}%
\pgfsetdash{}{0pt}%
\pgfsys@defobject{currentmarker}{\pgfqpoint{-0.048611in}{0.000000in}}{\pgfqpoint{-0.000000in}{0.000000in}}{%
\pgfpathmoveto{\pgfqpoint{-0.000000in}{0.000000in}}%
\pgfpathlineto{\pgfqpoint{-0.048611in}{0.000000in}}%
\pgfusepath{stroke,fill}%
}%
\begin{pgfscope}%
\pgfsys@transformshift{0.608070in}{1.902762in}%
\pgfsys@useobject{currentmarker}{}%
\end{pgfscope}%
\end{pgfscope}%
\begin{pgfscope}%
\definecolor{textcolor}{rgb}{0.000000,0.000000,0.000000}%
\pgfsetstrokecolor{textcolor}%
\pgfsetfillcolor{textcolor}%
\pgftext[x=0.289968in, y=1.850000in, left, base]{\color{textcolor}\sffamily\fontsize{10.000000}{12.000000}\selectfont 0.1}%
\end{pgfscope}%
\begin{pgfscope}%
\pgfsetbuttcap%
\pgfsetroundjoin%
\definecolor{currentfill}{rgb}{0.000000,0.000000,0.000000}%
\pgfsetfillcolor{currentfill}%
\pgfsetlinewidth{0.803000pt}%
\definecolor{currentstroke}{rgb}{0.000000,0.000000,0.000000}%
\pgfsetstrokecolor{currentstroke}%
\pgfsetdash{}{0pt}%
\pgfsys@defobject{currentmarker}{\pgfqpoint{-0.048611in}{0.000000in}}{\pgfqpoint{-0.000000in}{0.000000in}}{%
\pgfpathmoveto{\pgfqpoint{-0.000000in}{0.000000in}}%
\pgfpathlineto{\pgfqpoint{-0.048611in}{0.000000in}}%
\pgfusepath{stroke,fill}%
}%
\begin{pgfscope}%
\pgfsys@transformshift{0.608070in}{2.641962in}%
\pgfsys@useobject{currentmarker}{}%
\end{pgfscope}%
\end{pgfscope}%
\begin{pgfscope}%
\definecolor{textcolor}{rgb}{0.000000,0.000000,0.000000}%
\pgfsetstrokecolor{textcolor}%
\pgfsetfillcolor{textcolor}%
\pgftext[x=0.289968in, y=2.589200in, left, base]{\color{textcolor}\sffamily\fontsize{10.000000}{12.000000}\selectfont 0.2}%
\end{pgfscope}%
\begin{pgfscope}%
\pgfsetbuttcap%
\pgfsetroundjoin%
\definecolor{currentfill}{rgb}{0.000000,0.000000,0.000000}%
\pgfsetfillcolor{currentfill}%
\pgfsetlinewidth{0.803000pt}%
\definecolor{currentstroke}{rgb}{0.000000,0.000000,0.000000}%
\pgfsetstrokecolor{currentstroke}%
\pgfsetdash{}{0pt}%
\pgfsys@defobject{currentmarker}{\pgfqpoint{-0.048611in}{0.000000in}}{\pgfqpoint{-0.000000in}{0.000000in}}{%
\pgfpathmoveto{\pgfqpoint{-0.000000in}{0.000000in}}%
\pgfpathlineto{\pgfqpoint{-0.048611in}{0.000000in}}%
\pgfusepath{stroke,fill}%
}%
\begin{pgfscope}%
\pgfsys@transformshift{0.608070in}{3.381162in}%
\pgfsys@useobject{currentmarker}{}%
\end{pgfscope}%
\end{pgfscope}%
\begin{pgfscope}%
\definecolor{textcolor}{rgb}{0.000000,0.000000,0.000000}%
\pgfsetstrokecolor{textcolor}%
\pgfsetfillcolor{textcolor}%
\pgftext[x=0.289968in, y=3.328400in, left, base]{\color{textcolor}\sffamily\fontsize{10.000000}{12.000000}\selectfont 0.3}%
\end{pgfscope}%
\begin{pgfscope}%
\pgfsetbuttcap%
\pgfsetroundjoin%
\definecolor{currentfill}{rgb}{0.000000,0.000000,0.000000}%
\pgfsetfillcolor{currentfill}%
\pgfsetlinewidth{0.803000pt}%
\definecolor{currentstroke}{rgb}{0.000000,0.000000,0.000000}%
\pgfsetstrokecolor{currentstroke}%
\pgfsetdash{}{0pt}%
\pgfsys@defobject{currentmarker}{\pgfqpoint{-0.048611in}{0.000000in}}{\pgfqpoint{-0.000000in}{0.000000in}}{%
\pgfpathmoveto{\pgfqpoint{-0.000000in}{0.000000in}}%
\pgfpathlineto{\pgfqpoint{-0.048611in}{0.000000in}}%
\pgfusepath{stroke,fill}%
}%
\begin{pgfscope}%
\pgfsys@transformshift{0.608070in}{4.120362in}%
\pgfsys@useobject{currentmarker}{}%
\end{pgfscope}%
\end{pgfscope}%
\begin{pgfscope}%
\definecolor{textcolor}{rgb}{0.000000,0.000000,0.000000}%
\pgfsetstrokecolor{textcolor}%
\pgfsetfillcolor{textcolor}%
\pgftext[x=0.289968in, y=4.067600in, left, base]{\color{textcolor}\sffamily\fontsize{10.000000}{12.000000}\selectfont 0.4}%
\end{pgfscope}%
\begin{pgfscope}%
\pgfsetbuttcap%
\pgfsetroundjoin%
\definecolor{currentfill}{rgb}{0.000000,0.000000,0.000000}%
\pgfsetfillcolor{currentfill}%
\pgfsetlinewidth{0.803000pt}%
\definecolor{currentstroke}{rgb}{0.000000,0.000000,0.000000}%
\pgfsetstrokecolor{currentstroke}%
\pgfsetdash{}{0pt}%
\pgfsys@defobject{currentmarker}{\pgfqpoint{-0.048611in}{0.000000in}}{\pgfqpoint{-0.000000in}{0.000000in}}{%
\pgfpathmoveto{\pgfqpoint{-0.000000in}{0.000000in}}%
\pgfpathlineto{\pgfqpoint{-0.048611in}{0.000000in}}%
\pgfusepath{stroke,fill}%
}%
\begin{pgfscope}%
\pgfsys@transformshift{0.608070in}{4.859562in}%
\pgfsys@useobject{currentmarker}{}%
\end{pgfscope}%
\end{pgfscope}%
\begin{pgfscope}%
\definecolor{textcolor}{rgb}{0.000000,0.000000,0.000000}%
\pgfsetstrokecolor{textcolor}%
\pgfsetfillcolor{textcolor}%
\pgftext[x=0.289968in, y=4.806800in, left, base]{\color{textcolor}\sffamily\fontsize{10.000000}{12.000000}\selectfont 0.5}%
\end{pgfscope}%
\begin{pgfscope}%
\definecolor{textcolor}{rgb}{0.000000,0.000000,0.000000}%
\pgfsetstrokecolor{textcolor}%
\pgfsetfillcolor{textcolor}%
\pgftext[x=0.234413in,y=3.011562in,,bottom,rotate=90.000000]{\color{textcolor}\sffamily\fontsize{10.000000}{12.000000}\selectfont F1 Score}%
\end{pgfscope}%
\begin{pgfscope}%
\pgfpathrectangle{\pgfqpoint{0.608070in}{1.163562in}}{\pgfqpoint{4.960000in}{3.696000in}}%
\pgfusepath{clip}%
\pgfsetrectcap%
\pgfsetroundjoin%
\pgfsetlinewidth{1.505625pt}%
\definecolor{currentstroke}{rgb}{0.121569,0.466667,0.705882}%
\pgfsetstrokecolor{currentstroke}%
\pgfsetdash{}{0pt}%
\pgfpathmoveto{\pgfqpoint{0.833525in}{4.431565in}}%
\pgfpathlineto{\pgfqpoint{1.960797in}{2.255360in}}%
\pgfpathlineto{\pgfqpoint{3.088070in}{2.764595in}}%
\pgfpathlineto{\pgfqpoint{4.215343in}{4.251200in}}%
\pgfpathlineto{\pgfqpoint{5.342615in}{4.168410in}}%
\pgfusepath{stroke}%
\end{pgfscope}%
\begin{pgfscope}%
\pgfpathrectangle{\pgfqpoint{0.608070in}{1.163562in}}{\pgfqpoint{4.960000in}{3.696000in}}%
\pgfusepath{clip}%
\pgfsetbuttcap%
\pgfsetroundjoin%
\definecolor{currentfill}{rgb}{0.121569,0.466667,0.705882}%
\pgfsetfillcolor{currentfill}%
\pgfsetlinewidth{1.003750pt}%
\definecolor{currentstroke}{rgb}{0.121569,0.466667,0.705882}%
\pgfsetstrokecolor{currentstroke}%
\pgfsetdash{}{0pt}%
\pgfsys@defobject{currentmarker}{\pgfqpoint{-0.041667in}{-0.041667in}}{\pgfqpoint{0.041667in}{0.041667in}}{%
\pgfpathmoveto{\pgfqpoint{0.000000in}{-0.041667in}}%
\pgfpathcurveto{\pgfqpoint{0.011050in}{-0.041667in}}{\pgfqpoint{0.021649in}{-0.037276in}}{\pgfqpoint{0.029463in}{-0.029463in}}%
\pgfpathcurveto{\pgfqpoint{0.037276in}{-0.021649in}}{\pgfqpoint{0.041667in}{-0.011050in}}{\pgfqpoint{0.041667in}{0.000000in}}%
\pgfpathcurveto{\pgfqpoint{0.041667in}{0.011050in}}{\pgfqpoint{0.037276in}{0.021649in}}{\pgfqpoint{0.029463in}{0.029463in}}%
\pgfpathcurveto{\pgfqpoint{0.021649in}{0.037276in}}{\pgfqpoint{0.011050in}{0.041667in}}{\pgfqpoint{0.000000in}{0.041667in}}%
\pgfpathcurveto{\pgfqpoint{-0.011050in}{0.041667in}}{\pgfqpoint{-0.021649in}{0.037276in}}{\pgfqpoint{-0.029463in}{0.029463in}}%
\pgfpathcurveto{\pgfqpoint{-0.037276in}{0.021649in}}{\pgfqpoint{-0.041667in}{0.011050in}}{\pgfqpoint{-0.041667in}{0.000000in}}%
\pgfpathcurveto{\pgfqpoint{-0.041667in}{-0.011050in}}{\pgfqpoint{-0.037276in}{-0.021649in}}{\pgfqpoint{-0.029463in}{-0.029463in}}%
\pgfpathcurveto{\pgfqpoint{-0.021649in}{-0.037276in}}{\pgfqpoint{-0.011050in}{-0.041667in}}{\pgfqpoint{0.000000in}{-0.041667in}}%
\pgfpathclose%
\pgfusepath{stroke,fill}%
}%
\begin{pgfscope}%
\pgfsys@transformshift{0.833525in}{4.431565in}%
\pgfsys@useobject{currentmarker}{}%
\end{pgfscope}%
\begin{pgfscope}%
\pgfsys@transformshift{1.960797in}{2.255360in}%
\pgfsys@useobject{currentmarker}{}%
\end{pgfscope}%
\begin{pgfscope}%
\pgfsys@transformshift{3.088070in}{2.764595in}%
\pgfsys@useobject{currentmarker}{}%
\end{pgfscope}%
\begin{pgfscope}%
\pgfsys@transformshift{4.215343in}{4.251200in}%
\pgfsys@useobject{currentmarker}{}%
\end{pgfscope}%
\begin{pgfscope}%
\pgfsys@transformshift{5.342615in}{4.168410in}%
\pgfsys@useobject{currentmarker}{}%
\end{pgfscope}%
\end{pgfscope}%
\begin{pgfscope}%
\pgfpathrectangle{\pgfqpoint{0.608070in}{1.163562in}}{\pgfqpoint{4.960000in}{3.696000in}}%
\pgfusepath{clip}%
\pgfsetrectcap%
\pgfsetroundjoin%
\pgfsetlinewidth{1.505625pt}%
\definecolor{currentstroke}{rgb}{1.000000,0.498039,0.054902}%
\pgfsetstrokecolor{currentstroke}%
\pgfsetdash{}{0pt}%
\pgfpathmoveto{\pgfqpoint{0.833525in}{4.328077in}}%
\pgfpathlineto{\pgfqpoint{1.960797in}{1.932330in}}%
\pgfpathlineto{\pgfqpoint{3.088070in}{3.137226in}}%
\pgfpathlineto{\pgfqpoint{4.215343in}{4.234938in}}%
\pgfpathlineto{\pgfqpoint{5.342615in}{4.349514in}}%
\pgfusepath{stroke}%
\end{pgfscope}%
\begin{pgfscope}%
\pgfpathrectangle{\pgfqpoint{0.608070in}{1.163562in}}{\pgfqpoint{4.960000in}{3.696000in}}%
\pgfusepath{clip}%
\pgfsetbuttcap%
\pgfsetmiterjoin%
\definecolor{currentfill}{rgb}{1.000000,0.498039,0.054902}%
\pgfsetfillcolor{currentfill}%
\pgfsetlinewidth{1.003750pt}%
\definecolor{currentstroke}{rgb}{1.000000,0.498039,0.054902}%
\pgfsetstrokecolor{currentstroke}%
\pgfsetdash{}{0pt}%
\pgfsys@defobject{currentmarker}{\pgfqpoint{-0.041667in}{-0.041667in}}{\pgfqpoint{0.041667in}{0.041667in}}{%
\pgfpathmoveto{\pgfqpoint{-0.000000in}{-0.041667in}}%
\pgfpathlineto{\pgfqpoint{0.041667in}{0.041667in}}%
\pgfpathlineto{\pgfqpoint{-0.041667in}{0.041667in}}%
\pgfpathclose%
\pgfusepath{stroke,fill}%
}%
\begin{pgfscope}%
\pgfsys@transformshift{0.833525in}{4.328077in}%
\pgfsys@useobject{currentmarker}{}%
\end{pgfscope}%
\begin{pgfscope}%
\pgfsys@transformshift{1.960797in}{1.932330in}%
\pgfsys@useobject{currentmarker}{}%
\end{pgfscope}%
\begin{pgfscope}%
\pgfsys@transformshift{3.088070in}{3.137226in}%
\pgfsys@useobject{currentmarker}{}%
\end{pgfscope}%
\begin{pgfscope}%
\pgfsys@transformshift{4.215343in}{4.234938in}%
\pgfsys@useobject{currentmarker}{}%
\end{pgfscope}%
\begin{pgfscope}%
\pgfsys@transformshift{5.342615in}{4.349514in}%
\pgfsys@useobject{currentmarker}{}%
\end{pgfscope}%
\end{pgfscope}%
\begin{pgfscope}%
\pgfsetrectcap%
\pgfsetmiterjoin%
\pgfsetlinewidth{0.803000pt}%
\definecolor{currentstroke}{rgb}{0.000000,0.000000,0.000000}%
\pgfsetstrokecolor{currentstroke}%
\pgfsetdash{}{0pt}%
\pgfpathmoveto{\pgfqpoint{0.608070in}{1.163562in}}%
\pgfpathlineto{\pgfqpoint{0.608070in}{4.859562in}}%
\pgfusepath{stroke}%
\end{pgfscope}%
\begin{pgfscope}%
\pgfsetrectcap%
\pgfsetmiterjoin%
\pgfsetlinewidth{0.803000pt}%
\definecolor{currentstroke}{rgb}{0.000000,0.000000,0.000000}%
\pgfsetstrokecolor{currentstroke}%
\pgfsetdash{}{0pt}%
\pgfpathmoveto{\pgfqpoint{5.568070in}{1.163562in}}%
\pgfpathlineto{\pgfqpoint{5.568070in}{4.859562in}}%
\pgfusepath{stroke}%
\end{pgfscope}%
\begin{pgfscope}%
\pgfsetrectcap%
\pgfsetmiterjoin%
\pgfsetlinewidth{0.803000pt}%
\definecolor{currentstroke}{rgb}{0.000000,0.000000,0.000000}%
\pgfsetstrokecolor{currentstroke}%
\pgfsetdash{}{0pt}%
\pgfpathmoveto{\pgfqpoint{0.608070in}{1.163562in}}%
\pgfpathlineto{\pgfqpoint{5.568070in}{1.163562in}}%
\pgfusepath{stroke}%
\end{pgfscope}%
\begin{pgfscope}%
\pgfsetrectcap%
\pgfsetmiterjoin%
\pgfsetlinewidth{0.803000pt}%
\definecolor{currentstroke}{rgb}{0.000000,0.000000,0.000000}%
\pgfsetstrokecolor{currentstroke}%
\pgfsetdash{}{0pt}%
\pgfpathmoveto{\pgfqpoint{0.608070in}{4.859562in}}%
\pgfpathlineto{\pgfqpoint{5.568070in}{4.859562in}}%
\pgfusepath{stroke}%
\end{pgfscope}%
\begin{pgfscope}%
\pgfsetbuttcap%
\pgfsetmiterjoin%
\definecolor{currentfill}{rgb}{1.000000,1.000000,1.000000}%
\pgfsetfillcolor{currentfill}%
\pgfsetfillopacity{0.800000}%
\pgfsetlinewidth{1.003750pt}%
\definecolor{currentstroke}{rgb}{0.800000,0.800000,0.800000}%
\pgfsetstrokecolor{currentstroke}%
\pgfsetstrokeopacity{0.800000}%
\pgfsetdash{}{0pt}%
\pgfpathmoveto{\pgfqpoint{0.705292in}{1.233006in}}%
\pgfpathlineto{\pgfqpoint{1.919458in}{1.233006in}}%
\pgfpathquadraticcurveto{\pgfqpoint{1.947236in}{1.233006in}}{\pgfqpoint{1.947236in}{1.260784in}}%
\pgfpathlineto{\pgfqpoint{1.947236in}{1.654609in}}%
\pgfpathquadraticcurveto{\pgfqpoint{1.947236in}{1.682387in}}{\pgfqpoint{1.919458in}{1.682387in}}%
\pgfpathlineto{\pgfqpoint{0.705292in}{1.682387in}}%
\pgfpathquadraticcurveto{\pgfqpoint{0.677514in}{1.682387in}}{\pgfqpoint{0.677514in}{1.654609in}}%
\pgfpathlineto{\pgfqpoint{0.677514in}{1.260784in}}%
\pgfpathquadraticcurveto{\pgfqpoint{0.677514in}{1.233006in}}{\pgfqpoint{0.705292in}{1.233006in}}%
\pgfpathclose%
\pgfusepath{stroke,fill}%
\end{pgfscope}%
\begin{pgfscope}%
\pgfsetrectcap%
\pgfsetroundjoin%
\pgfsetlinewidth{1.505625pt}%
\definecolor{currentstroke}{rgb}{0.121569,0.466667,0.705882}%
\pgfsetstrokecolor{currentstroke}%
\pgfsetdash{}{0pt}%
\pgfpathmoveto{\pgfqpoint{0.733070in}{1.569920in}}%
\pgfpathlineto{\pgfqpoint{1.010848in}{1.569920in}}%
\pgfusepath{stroke}%
\end{pgfscope}%
\begin{pgfscope}%
\pgfsetbuttcap%
\pgfsetroundjoin%
\definecolor{currentfill}{rgb}{0.121569,0.466667,0.705882}%
\pgfsetfillcolor{currentfill}%
\pgfsetlinewidth{1.003750pt}%
\definecolor{currentstroke}{rgb}{0.121569,0.466667,0.705882}%
\pgfsetstrokecolor{currentstroke}%
\pgfsetdash{}{0pt}%
\pgfsys@defobject{currentmarker}{\pgfqpoint{-0.041667in}{-0.041667in}}{\pgfqpoint{0.041667in}{0.041667in}}{%
\pgfpathmoveto{\pgfqpoint{0.000000in}{-0.041667in}}%
\pgfpathcurveto{\pgfqpoint{0.011050in}{-0.041667in}}{\pgfqpoint{0.021649in}{-0.037276in}}{\pgfqpoint{0.029463in}{-0.029463in}}%
\pgfpathcurveto{\pgfqpoint{0.037276in}{-0.021649in}}{\pgfqpoint{0.041667in}{-0.011050in}}{\pgfqpoint{0.041667in}{0.000000in}}%
\pgfpathcurveto{\pgfqpoint{0.041667in}{0.011050in}}{\pgfqpoint{0.037276in}{0.021649in}}{\pgfqpoint{0.029463in}{0.029463in}}%
\pgfpathcurveto{\pgfqpoint{0.021649in}{0.037276in}}{\pgfqpoint{0.011050in}{0.041667in}}{\pgfqpoint{0.000000in}{0.041667in}}%
\pgfpathcurveto{\pgfqpoint{-0.011050in}{0.041667in}}{\pgfqpoint{-0.021649in}{0.037276in}}{\pgfqpoint{-0.029463in}{0.029463in}}%
\pgfpathcurveto{\pgfqpoint{-0.037276in}{0.021649in}}{\pgfqpoint{-0.041667in}{0.011050in}}{\pgfqpoint{-0.041667in}{0.000000in}}%
\pgfpathcurveto{\pgfqpoint{-0.041667in}{-0.011050in}}{\pgfqpoint{-0.037276in}{-0.021649in}}{\pgfqpoint{-0.029463in}{-0.029463in}}%
\pgfpathcurveto{\pgfqpoint{-0.021649in}{-0.037276in}}{\pgfqpoint{-0.011050in}{-0.041667in}}{\pgfqpoint{0.000000in}{-0.041667in}}%
\pgfpathclose%
\pgfusepath{stroke,fill}%
}%
\begin{pgfscope}%
\pgfsys@transformshift{0.871959in}{1.569920in}%
\pgfsys@useobject{currentmarker}{}%
\end{pgfscope}%
\end{pgfscope}%
\begin{pgfscope}%
\definecolor{textcolor}{rgb}{0.000000,0.000000,0.000000}%
\pgfsetstrokecolor{textcolor}%
\pgfsetfillcolor{textcolor}%
\pgftext[x=1.121959in,y=1.521309in,left,base]{\color{textcolor}\sffamily\fontsize{10.000000}{12.000000}\selectfont Pix2Vox++}%
\end{pgfscope}%
\begin{pgfscope}%
\pgfsetrectcap%
\pgfsetroundjoin%
\pgfsetlinewidth{1.505625pt}%
\definecolor{currentstroke}{rgb}{1.000000,0.498039,0.054902}%
\pgfsetstrokecolor{currentstroke}%
\pgfsetdash{}{0pt}%
\pgfpathmoveto{\pgfqpoint{0.733070in}{1.366062in}}%
\pgfpathlineto{\pgfqpoint{1.010848in}{1.366062in}}%
\pgfusepath{stroke}%
\end{pgfscope}%
\begin{pgfscope}%
\pgfsetbuttcap%
\pgfsetmiterjoin%
\definecolor{currentfill}{rgb}{1.000000,0.498039,0.054902}%
\pgfsetfillcolor{currentfill}%
\pgfsetlinewidth{1.003750pt}%
\definecolor{currentstroke}{rgb}{1.000000,0.498039,0.054902}%
\pgfsetstrokecolor{currentstroke}%
\pgfsetdash{}{0pt}%
\pgfsys@defobject{currentmarker}{\pgfqpoint{-0.041667in}{-0.041667in}}{\pgfqpoint{0.041667in}{0.041667in}}{%
\pgfpathmoveto{\pgfqpoint{-0.000000in}{-0.041667in}}%
\pgfpathlineto{\pgfqpoint{0.041667in}{0.041667in}}%
\pgfpathlineto{\pgfqpoint{-0.041667in}{0.041667in}}%
\pgfpathclose%
\pgfusepath{stroke,fill}%
}%
\begin{pgfscope}%
\pgfsys@transformshift{0.871959in}{1.366062in}%
\pgfsys@useobject{currentmarker}{}%
\end{pgfscope}%
\end{pgfscope}%
\begin{pgfscope}%
\definecolor{textcolor}{rgb}{0.000000,0.000000,0.000000}%
\pgfsetstrokecolor{textcolor}%
\pgfsetfillcolor{textcolor}%
\pgftext[x=1.121959in,y=1.317451in,left,base]{\color{textcolor}\sffamily\fontsize{10.000000}{12.000000}\selectfont Pix2Vox}%
\end{pgfscope}%
\end{pgfpicture}%
\makeatother%
\endgroup%
}
%    \caption{Line plot for the \gls{f1} for baseline models(Pix2Vox++ and Pix2Vox) trained on synthetic and fine-tuned with real dataset.
%    We see that even after fine-tuning both the models do not perform as good as models trained on only real dataset.}
%    \label{fig:finetuning_dice1}
%\end{figure}


\begin{figure}[ht]
    \centering
    \resizebox{0.75\textwidth}{!}{%% Creator: Matplotlib, PGF backend
%%
%% To include the figure in your LaTeX document, write
%%   \input{<filename>.pgf}
%%
%% Make sure the required packages are loaded in your preamble
%%   \usepackage{pgf}
%%
%% Figures using additional raster images can only be included by \input if
%% they are in the same directory as the main LaTeX file. For loading figures
%% from other directories you can use the `import` package
%%   \usepackage{import}
%%
%% and then include the figures with
%%   \import{<path to file>}{<filename>.pgf}
%%
%% Matplotlib used the following preamble
%%   \usepackage{fontspec}
%%   \setmainfont{DejaVuSerif.ttf}[Path=\detokenize{/Users/apple/opt/anaconda3/envs/kaolin/lib/python3.7/site-packages/matplotlib/mpl-data/fonts/ttf/}]
%%   \setsansfont{DejaVuSans.ttf}[Path=\detokenize{/Users/apple/opt/anaconda3/envs/kaolin/lib/python3.7/site-packages/matplotlib/mpl-data/fonts/ttf/}]
%%   \setmonofont{DejaVuSansMono.ttf}[Path=\detokenize{/Users/apple/opt/anaconda3/envs/kaolin/lib/python3.7/site-packages/matplotlib/mpl-data/fonts/ttf/}]
%%
\begingroup%
\makeatletter%
\begin{pgfpicture}%
\pgfpathrectangle{\pgfpointorigin}{\pgfqpoint{6.294042in}{4.695948in}}%
\pgfusepath{use as bounding box, clip}%
\begin{pgfscope}%
\pgfsetbuttcap%
\pgfsetmiterjoin%
\definecolor{currentfill}{rgb}{1.000000,1.000000,1.000000}%
\pgfsetfillcolor{currentfill}%
\pgfsetlinewidth{0.000000pt}%
\definecolor{currentstroke}{rgb}{1.000000,1.000000,1.000000}%
\pgfsetstrokecolor{currentstroke}%
\pgfsetdash{}{0pt}%
\pgfpathmoveto{\pgfqpoint{0.000000in}{0.000000in}}%
\pgfpathlineto{\pgfqpoint{6.294042in}{0.000000in}}%
\pgfpathlineto{\pgfqpoint{6.294042in}{4.695948in}}%
\pgfpathlineto{\pgfqpoint{0.000000in}{4.695948in}}%
\pgfpathclose%
\pgfusepath{fill}%
\end{pgfscope}%
\begin{pgfscope}%
\pgfsetbuttcap%
\pgfsetmiterjoin%
\definecolor{currentfill}{rgb}{1.000000,1.000000,1.000000}%
\pgfsetfillcolor{currentfill}%
\pgfsetlinewidth{0.000000pt}%
\definecolor{currentstroke}{rgb}{0.000000,0.000000,0.000000}%
\pgfsetstrokecolor{currentstroke}%
\pgfsetstrokeopacity{0.000000}%
\pgfsetdash{}{0pt}%
\pgfpathmoveto{\pgfqpoint{0.608070in}{1.163562in}}%
\pgfpathlineto{\pgfqpoint{6.194042in}{1.163562in}}%
\pgfpathlineto{\pgfqpoint{6.194042in}{4.385987in}}%
\pgfpathlineto{\pgfqpoint{0.608070in}{4.385987in}}%
\pgfpathclose%
\pgfusepath{fill}%
\end{pgfscope}%
\begin{pgfscope}%
\pgfpathrectangle{\pgfqpoint{0.608070in}{1.163562in}}{\pgfqpoint{5.585972in}{3.222425in}}%
\pgfusepath{clip}%
\pgfsetbuttcap%
\pgfsetmiterjoin%
\definecolor{currentfill}{rgb}{0.121569,0.466667,0.705882}%
\pgfsetfillcolor{currentfill}%
\pgfsetlinewidth{0.000000pt}%
\definecolor{currentstroke}{rgb}{0.000000,0.000000,0.000000}%
\pgfsetstrokecolor{currentstroke}%
\pgfsetstrokeopacity{0.000000}%
\pgfsetdash{}{0pt}%
\pgfpathmoveto{\pgfqpoint{0.861978in}{1.163562in}}%
\pgfpathlineto{\pgfqpoint{1.328339in}{1.163562in}}%
\pgfpathlineto{\pgfqpoint{1.328339in}{3.537952in}}%
\pgfpathlineto{\pgfqpoint{0.861978in}{3.537952in}}%
\pgfpathclose%
\pgfusepath{fill}%
\end{pgfscope}%
\begin{pgfscope}%
\pgfpathrectangle{\pgfqpoint{0.608070in}{1.163562in}}{\pgfqpoint{5.585972in}{3.222425in}}%
\pgfusepath{clip}%
\pgfsetbuttcap%
\pgfsetmiterjoin%
\definecolor{currentfill}{rgb}{0.121569,0.466667,0.705882}%
\pgfsetfillcolor{currentfill}%
\pgfsetlinewidth{0.000000pt}%
\definecolor{currentstroke}{rgb}{0.000000,0.000000,0.000000}%
\pgfsetstrokecolor{currentstroke}%
\pgfsetstrokeopacity{0.000000}%
\pgfsetdash{}{0pt}%
\pgfpathmoveto{\pgfqpoint{1.898336in}{1.163562in}}%
\pgfpathlineto{\pgfqpoint{2.364698in}{1.163562in}}%
\pgfpathlineto{\pgfqpoint{2.364698in}{1.956815in}}%
\pgfpathlineto{\pgfqpoint{1.898336in}{1.956815in}}%
\pgfpathclose%
\pgfusepath{fill}%
\end{pgfscope}%
\begin{pgfscope}%
\pgfpathrectangle{\pgfqpoint{0.608070in}{1.163562in}}{\pgfqpoint{5.585972in}{3.222425in}}%
\pgfusepath{clip}%
\pgfsetbuttcap%
\pgfsetmiterjoin%
\definecolor{currentfill}{rgb}{0.121569,0.466667,0.705882}%
\pgfsetfillcolor{currentfill}%
\pgfsetlinewidth{0.000000pt}%
\definecolor{currentstroke}{rgb}{0.000000,0.000000,0.000000}%
\pgfsetstrokecolor{currentstroke}%
\pgfsetstrokeopacity{0.000000}%
\pgfsetdash{}{0pt}%
\pgfpathmoveto{\pgfqpoint{2.934695in}{1.163562in}}%
\pgfpathlineto{\pgfqpoint{3.401056in}{1.163562in}}%
\pgfpathlineto{\pgfqpoint{3.401056in}{2.326803in}}%
\pgfpathlineto{\pgfqpoint{2.934695in}{2.326803in}}%
\pgfpathclose%
\pgfusepath{fill}%
\end{pgfscope}%
\begin{pgfscope}%
\pgfpathrectangle{\pgfqpoint{0.608070in}{1.163562in}}{\pgfqpoint{5.585972in}{3.222425in}}%
\pgfusepath{clip}%
\pgfsetbuttcap%
\pgfsetmiterjoin%
\definecolor{currentfill}{rgb}{0.121569,0.466667,0.705882}%
\pgfsetfillcolor{currentfill}%
\pgfsetlinewidth{0.000000pt}%
\definecolor{currentstroke}{rgb}{0.000000,0.000000,0.000000}%
\pgfsetstrokecolor{currentstroke}%
\pgfsetstrokeopacity{0.000000}%
\pgfsetdash{}{0pt}%
\pgfpathmoveto{\pgfqpoint{3.971053in}{1.163562in}}%
\pgfpathlineto{\pgfqpoint{4.437415in}{1.163562in}}%
\pgfpathlineto{\pgfqpoint{4.437415in}{3.406907in}}%
\pgfpathlineto{\pgfqpoint{3.971053in}{3.406907in}}%
\pgfpathclose%
\pgfusepath{fill}%
\end{pgfscope}%
\begin{pgfscope}%
\pgfpathrectangle{\pgfqpoint{0.608070in}{1.163562in}}{\pgfqpoint{5.585972in}{3.222425in}}%
\pgfusepath{clip}%
\pgfsetbuttcap%
\pgfsetmiterjoin%
\definecolor{currentfill}{rgb}{0.121569,0.466667,0.705882}%
\pgfsetfillcolor{currentfill}%
\pgfsetlinewidth{0.000000pt}%
\definecolor{currentstroke}{rgb}{0.000000,0.000000,0.000000}%
\pgfsetstrokecolor{currentstroke}%
\pgfsetstrokeopacity{0.000000}%
\pgfsetdash{}{0pt}%
\pgfpathmoveto{\pgfqpoint{5.007412in}{1.163562in}}%
\pgfpathlineto{\pgfqpoint{5.473773in}{1.163562in}}%
\pgfpathlineto{\pgfqpoint{5.473773in}{3.346755in}}%
\pgfpathlineto{\pgfqpoint{5.007412in}{3.346755in}}%
\pgfpathclose%
\pgfusepath{fill}%
\end{pgfscope}%
\begin{pgfscope}%
\pgfpathrectangle{\pgfqpoint{0.608070in}{1.163562in}}{\pgfqpoint{5.585972in}{3.222425in}}%
\pgfusepath{clip}%
\pgfsetbuttcap%
\pgfsetmiterjoin%
\definecolor{currentfill}{rgb}{1.000000,0.498039,0.054902}%
\pgfsetfillcolor{currentfill}%
\pgfsetlinewidth{0.000000pt}%
\definecolor{currentstroke}{rgb}{0.000000,0.000000,0.000000}%
\pgfsetstrokecolor{currentstroke}%
\pgfsetstrokeopacity{0.000000}%
\pgfsetdash{}{0pt}%
\pgfpathmoveto{\pgfqpoint{1.328339in}{1.163562in}}%
\pgfpathlineto{\pgfqpoint{1.794700in}{1.163562in}}%
\pgfpathlineto{\pgfqpoint{1.794700in}{3.462762in}}%
\pgfpathlineto{\pgfqpoint{1.328339in}{3.462762in}}%
\pgfpathclose%
\pgfusepath{fill}%
\end{pgfscope}%
\begin{pgfscope}%
\pgfpathrectangle{\pgfqpoint{0.608070in}{1.163562in}}{\pgfqpoint{5.585972in}{3.222425in}}%
\pgfusepath{clip}%
\pgfsetbuttcap%
\pgfsetmiterjoin%
\definecolor{currentfill}{rgb}{1.000000,0.498039,0.054902}%
\pgfsetfillcolor{currentfill}%
\pgfsetlinewidth{0.000000pt}%
\definecolor{currentstroke}{rgb}{0.000000,0.000000,0.000000}%
\pgfsetstrokecolor{currentstroke}%
\pgfsetstrokeopacity{0.000000}%
\pgfsetdash{}{0pt}%
\pgfpathmoveto{\pgfqpoint{2.364698in}{1.163562in}}%
\pgfpathlineto{\pgfqpoint{2.831059in}{1.163562in}}%
\pgfpathlineto{\pgfqpoint{2.831059in}{1.722115in}}%
\pgfpathlineto{\pgfqpoint{2.364698in}{1.722115in}}%
\pgfpathclose%
\pgfusepath{fill}%
\end{pgfscope}%
\begin{pgfscope}%
\pgfpathrectangle{\pgfqpoint{0.608070in}{1.163562in}}{\pgfqpoint{5.585972in}{3.222425in}}%
\pgfusepath{clip}%
\pgfsetbuttcap%
\pgfsetmiterjoin%
\definecolor{currentfill}{rgb}{1.000000,0.498039,0.054902}%
\pgfsetfillcolor{currentfill}%
\pgfsetlinewidth{0.000000pt}%
\definecolor{currentstroke}{rgb}{0.000000,0.000000,0.000000}%
\pgfsetstrokecolor{currentstroke}%
\pgfsetstrokeopacity{0.000000}%
\pgfsetdash{}{0pt}%
\pgfpathmoveto{\pgfqpoint{3.401056in}{1.163562in}}%
\pgfpathlineto{\pgfqpoint{3.867417in}{1.163562in}}%
\pgfpathlineto{\pgfqpoint{3.867417in}{2.597541in}}%
\pgfpathlineto{\pgfqpoint{3.401056in}{2.597541in}}%
\pgfpathclose%
\pgfusepath{fill}%
\end{pgfscope}%
\begin{pgfscope}%
\pgfpathrectangle{\pgfqpoint{0.608070in}{1.163562in}}{\pgfqpoint{5.585972in}{3.222425in}}%
\pgfusepath{clip}%
\pgfsetbuttcap%
\pgfsetmiterjoin%
\definecolor{currentfill}{rgb}{1.000000,0.498039,0.054902}%
\pgfsetfillcolor{currentfill}%
\pgfsetlinewidth{0.000000pt}%
\definecolor{currentstroke}{rgb}{0.000000,0.000000,0.000000}%
\pgfsetstrokecolor{currentstroke}%
\pgfsetstrokeopacity{0.000000}%
\pgfsetdash{}{0pt}%
\pgfpathmoveto{\pgfqpoint{4.437415in}{1.163562in}}%
\pgfpathlineto{\pgfqpoint{4.903776in}{1.163562in}}%
\pgfpathlineto{\pgfqpoint{4.903776in}{3.395091in}}%
\pgfpathlineto{\pgfqpoint{4.437415in}{3.395091in}}%
\pgfpathclose%
\pgfusepath{fill}%
\end{pgfscope}%
\begin{pgfscope}%
\pgfpathrectangle{\pgfqpoint{0.608070in}{1.163562in}}{\pgfqpoint{5.585972in}{3.222425in}}%
\pgfusepath{clip}%
\pgfsetbuttcap%
\pgfsetmiterjoin%
\definecolor{currentfill}{rgb}{1.000000,0.498039,0.054902}%
\pgfsetfillcolor{currentfill}%
\pgfsetlinewidth{0.000000pt}%
\definecolor{currentstroke}{rgb}{0.000000,0.000000,0.000000}%
\pgfsetstrokecolor{currentstroke}%
\pgfsetstrokeopacity{0.000000}%
\pgfsetdash{}{0pt}%
\pgfpathmoveto{\pgfqpoint{5.473773in}{1.163562in}}%
\pgfpathlineto{\pgfqpoint{5.940134in}{1.163562in}}%
\pgfpathlineto{\pgfqpoint{5.940134in}{3.478337in}}%
\pgfpathlineto{\pgfqpoint{5.473773in}{3.478337in}}%
\pgfpathclose%
\pgfusepath{fill}%
\end{pgfscope}%
\begin{pgfscope}%
\pgfsetbuttcap%
\pgfsetroundjoin%
\definecolor{currentfill}{rgb}{0.000000,0.000000,0.000000}%
\pgfsetfillcolor{currentfill}%
\pgfsetlinewidth{0.803000pt}%
\definecolor{currentstroke}{rgb}{0.000000,0.000000,0.000000}%
\pgfsetstrokecolor{currentstroke}%
\pgfsetdash{}{0pt}%
\pgfsys@defobject{currentmarker}{\pgfqpoint{0.000000in}{-0.048611in}}{\pgfqpoint{0.000000in}{0.000000in}}{%
\pgfpathmoveto{\pgfqpoint{0.000000in}{0.000000in}}%
\pgfpathlineto{\pgfqpoint{0.000000in}{-0.048611in}}%
\pgfusepath{stroke,fill}%
}%
\begin{pgfscope}%
\pgfsys@transformshift{1.328339in}{1.163562in}%
\pgfsys@useobject{currentmarker}{}%
\end{pgfscope}%
\end{pgfscope}%
\begin{pgfscope}%
\definecolor{textcolor}{rgb}{0.000000,0.000000,0.000000}%
\pgfsetstrokecolor{textcolor}%
\pgfsetfillcolor{textcolor}%
\pgftext[x=1.215144in, y=0.711146in, left, base,rotate=45.000000]{\color{textcolor}\sffamily\fontsize{10.000000}{12.000000}\selectfont Pix3D}%
\end{pgfscope}%
\begin{pgfscope}%
\pgfsetbuttcap%
\pgfsetroundjoin%
\definecolor{currentfill}{rgb}{0.000000,0.000000,0.000000}%
\pgfsetfillcolor{currentfill}%
\pgfsetlinewidth{0.803000pt}%
\definecolor{currentstroke}{rgb}{0.000000,0.000000,0.000000}%
\pgfsetstrokecolor{currentstroke}%
\pgfsetdash{}{0pt}%
\pgfsys@defobject{currentmarker}{\pgfqpoint{0.000000in}{-0.048611in}}{\pgfqpoint{0.000000in}{0.000000in}}{%
\pgfpathmoveto{\pgfqpoint{0.000000in}{0.000000in}}%
\pgfpathlineto{\pgfqpoint{0.000000in}{-0.048611in}}%
\pgfusepath{stroke,fill}%
}%
\begin{pgfscope}%
\pgfsys@transformshift{2.364698in}{1.163562in}%
\pgfsys@useobject{currentmarker}{}%
\end{pgfscope}%
\end{pgfscope}%
\begin{pgfscope}%
\definecolor{textcolor}{rgb}{0.000000,0.000000,0.000000}%
\pgfsetstrokecolor{textcolor}%
\pgfsetfillcolor{textcolor}%
\pgftext[x=2.227694in, y=0.669714in, left, base,rotate=45.000000]{\color{textcolor}\sffamily\fontsize{10.000000}{12.000000}\selectfont s2r\_v1}%
\end{pgfscope}%
\begin{pgfscope}%
\pgfsetbuttcap%
\pgfsetroundjoin%
\definecolor{currentfill}{rgb}{0.000000,0.000000,0.000000}%
\pgfsetfillcolor{currentfill}%
\pgfsetlinewidth{0.803000pt}%
\definecolor{currentstroke}{rgb}{0.000000,0.000000,0.000000}%
\pgfsetstrokecolor{currentstroke}%
\pgfsetdash{}{0pt}%
\pgfsys@defobject{currentmarker}{\pgfqpoint{0.000000in}{-0.048611in}}{\pgfqpoint{0.000000in}{0.000000in}}{%
\pgfpathmoveto{\pgfqpoint{0.000000in}{0.000000in}}%
\pgfpathlineto{\pgfqpoint{0.000000in}{-0.048611in}}%
\pgfusepath{stroke,fill}%
}%
\begin{pgfscope}%
\pgfsys@transformshift{3.401056in}{1.163562in}%
\pgfsys@useobject{currentmarker}{}%
\end{pgfscope}%
\end{pgfscope}%
\begin{pgfscope}%
\definecolor{textcolor}{rgb}{0.000000,0.000000,0.000000}%
\pgfsetstrokecolor{textcolor}%
\pgfsetfillcolor{textcolor}%
\pgftext[x=3.264052in, y=0.669714in, left, base,rotate=45.000000]{\color{textcolor}\sffamily\fontsize{10.000000}{12.000000}\selectfont s2r\_v2}%
\end{pgfscope}%
\begin{pgfscope}%
\pgfsetbuttcap%
\pgfsetroundjoin%
\definecolor{currentfill}{rgb}{0.000000,0.000000,0.000000}%
\pgfsetfillcolor{currentfill}%
\pgfsetlinewidth{0.803000pt}%
\definecolor{currentstroke}{rgb}{0.000000,0.000000,0.000000}%
\pgfsetstrokecolor{currentstroke}%
\pgfsetdash{}{0pt}%
\pgfsys@defobject{currentmarker}{\pgfqpoint{0.000000in}{-0.048611in}}{\pgfqpoint{0.000000in}{0.000000in}}{%
\pgfpathmoveto{\pgfqpoint{0.000000in}{0.000000in}}%
\pgfpathlineto{\pgfqpoint{0.000000in}{-0.048611in}}%
\pgfusepath{stroke,fill}%
}%
\begin{pgfscope}%
\pgfsys@transformshift{4.437415in}{1.163562in}%
\pgfsys@useobject{currentmarker}{}%
\end{pgfscope}%
\end{pgfscope}%
\begin{pgfscope}%
\definecolor{textcolor}{rgb}{0.000000,0.000000,0.000000}%
\pgfsetstrokecolor{textcolor}%
\pgfsetfillcolor{textcolor}%
\pgftext[x=4.123845in, y=0.313130in, left, base,rotate=45.000000]{\color{textcolor}\sffamily\fontsize{10.000000}{12.000000}\selectfont s2r\_v1+pix3d}%
\end{pgfscope}%
\begin{pgfscope}%
\pgfsetbuttcap%
\pgfsetroundjoin%
\definecolor{currentfill}{rgb}{0.000000,0.000000,0.000000}%
\pgfsetfillcolor{currentfill}%
\pgfsetlinewidth{0.803000pt}%
\definecolor{currentstroke}{rgb}{0.000000,0.000000,0.000000}%
\pgfsetstrokecolor{currentstroke}%
\pgfsetdash{}{0pt}%
\pgfsys@defobject{currentmarker}{\pgfqpoint{0.000000in}{-0.048611in}}{\pgfqpoint{0.000000in}{0.000000in}}{%
\pgfpathmoveto{\pgfqpoint{0.000000in}{0.000000in}}%
\pgfpathlineto{\pgfqpoint{0.000000in}{-0.048611in}}%
\pgfusepath{stroke,fill}%
}%
\begin{pgfscope}%
\pgfsys@transformshift{5.473773in}{1.163562in}%
\pgfsys@useobject{currentmarker}{}%
\end{pgfscope}%
\end{pgfscope}%
\begin{pgfscope}%
\definecolor{textcolor}{rgb}{0.000000,0.000000,0.000000}%
\pgfsetstrokecolor{textcolor}%
\pgfsetfillcolor{textcolor}%
\pgftext[x=5.160204in, y=0.313130in, left, base,rotate=45.000000]{\color{textcolor}\sffamily\fontsize{10.000000}{12.000000}\selectfont s2r\_v2+pix3d}%
\end{pgfscope}%
\begin{pgfscope}%
\definecolor{textcolor}{rgb}{0.000000,0.000000,0.000000}%
\pgfsetstrokecolor{textcolor}%
\pgfsetfillcolor{textcolor}%
\pgftext[x=3.401056in,y=0.234413in,,top]{\color{textcolor}\sffamily\fontsize{10.000000}{12.000000}\bfseries\selectfont Dataset}%
\end{pgfscope}%
\begin{pgfscope}%
\pgfsetbuttcap%
\pgfsetroundjoin%
\definecolor{currentfill}{rgb}{0.000000,0.000000,0.000000}%
\pgfsetfillcolor{currentfill}%
\pgfsetlinewidth{0.803000pt}%
\definecolor{currentstroke}{rgb}{0.000000,0.000000,0.000000}%
\pgfsetstrokecolor{currentstroke}%
\pgfsetdash{}{0pt}%
\pgfsys@defobject{currentmarker}{\pgfqpoint{-0.048611in}{0.000000in}}{\pgfqpoint{-0.000000in}{0.000000in}}{%
\pgfpathmoveto{\pgfqpoint{-0.000000in}{0.000000in}}%
\pgfpathlineto{\pgfqpoint{-0.048611in}{0.000000in}}%
\pgfusepath{stroke,fill}%
}%
\begin{pgfscope}%
\pgfsys@transformshift{0.608070in}{1.163562in}%
\pgfsys@useobject{currentmarker}{}%
\end{pgfscope}%
\end{pgfscope}%
\begin{pgfscope}%
\definecolor{textcolor}{rgb}{0.000000,0.000000,0.000000}%
\pgfsetstrokecolor{textcolor}%
\pgfsetfillcolor{textcolor}%
\pgftext[x=0.289968in, y=1.110800in, left, base]{\color{textcolor}\sffamily\fontsize{10.000000}{12.000000}\selectfont 0.0}%
\end{pgfscope}%
\begin{pgfscope}%
\pgfsetbuttcap%
\pgfsetroundjoin%
\definecolor{currentfill}{rgb}{0.000000,0.000000,0.000000}%
\pgfsetfillcolor{currentfill}%
\pgfsetlinewidth{0.803000pt}%
\definecolor{currentstroke}{rgb}{0.000000,0.000000,0.000000}%
\pgfsetstrokecolor{currentstroke}%
\pgfsetdash{}{0pt}%
\pgfsys@defobject{currentmarker}{\pgfqpoint{-0.048611in}{0.000000in}}{\pgfqpoint{-0.000000in}{0.000000in}}{%
\pgfpathmoveto{\pgfqpoint{-0.000000in}{0.000000in}}%
\pgfpathlineto{\pgfqpoint{-0.048611in}{0.000000in}}%
\pgfusepath{stroke,fill}%
}%
\begin{pgfscope}%
\pgfsys@transformshift{0.608070in}{1.700632in}%
\pgfsys@useobject{currentmarker}{}%
\end{pgfscope}%
\end{pgfscope}%
\begin{pgfscope}%
\definecolor{textcolor}{rgb}{0.000000,0.000000,0.000000}%
\pgfsetstrokecolor{textcolor}%
\pgfsetfillcolor{textcolor}%
\pgftext[x=0.289968in, y=1.647871in, left, base]{\color{textcolor}\sffamily\fontsize{10.000000}{12.000000}\selectfont 0.1}%
\end{pgfscope}%
\begin{pgfscope}%
\pgfsetbuttcap%
\pgfsetroundjoin%
\definecolor{currentfill}{rgb}{0.000000,0.000000,0.000000}%
\pgfsetfillcolor{currentfill}%
\pgfsetlinewidth{0.803000pt}%
\definecolor{currentstroke}{rgb}{0.000000,0.000000,0.000000}%
\pgfsetstrokecolor{currentstroke}%
\pgfsetdash{}{0pt}%
\pgfsys@defobject{currentmarker}{\pgfqpoint{-0.048611in}{0.000000in}}{\pgfqpoint{-0.000000in}{0.000000in}}{%
\pgfpathmoveto{\pgfqpoint{-0.000000in}{0.000000in}}%
\pgfpathlineto{\pgfqpoint{-0.048611in}{0.000000in}}%
\pgfusepath{stroke,fill}%
}%
\begin{pgfscope}%
\pgfsys@transformshift{0.608070in}{2.237703in}%
\pgfsys@useobject{currentmarker}{}%
\end{pgfscope}%
\end{pgfscope}%
\begin{pgfscope}%
\definecolor{textcolor}{rgb}{0.000000,0.000000,0.000000}%
\pgfsetstrokecolor{textcolor}%
\pgfsetfillcolor{textcolor}%
\pgftext[x=0.289968in, y=2.184942in, left, base]{\color{textcolor}\sffamily\fontsize{10.000000}{12.000000}\selectfont 0.2}%
\end{pgfscope}%
\begin{pgfscope}%
\pgfsetbuttcap%
\pgfsetroundjoin%
\definecolor{currentfill}{rgb}{0.000000,0.000000,0.000000}%
\pgfsetfillcolor{currentfill}%
\pgfsetlinewidth{0.803000pt}%
\definecolor{currentstroke}{rgb}{0.000000,0.000000,0.000000}%
\pgfsetstrokecolor{currentstroke}%
\pgfsetdash{}{0pt}%
\pgfsys@defobject{currentmarker}{\pgfqpoint{-0.048611in}{0.000000in}}{\pgfqpoint{-0.000000in}{0.000000in}}{%
\pgfpathmoveto{\pgfqpoint{-0.000000in}{0.000000in}}%
\pgfpathlineto{\pgfqpoint{-0.048611in}{0.000000in}}%
\pgfusepath{stroke,fill}%
}%
\begin{pgfscope}%
\pgfsys@transformshift{0.608070in}{2.774774in}%
\pgfsys@useobject{currentmarker}{}%
\end{pgfscope}%
\end{pgfscope}%
\begin{pgfscope}%
\definecolor{textcolor}{rgb}{0.000000,0.000000,0.000000}%
\pgfsetstrokecolor{textcolor}%
\pgfsetfillcolor{textcolor}%
\pgftext[x=0.289968in, y=2.722013in, left, base]{\color{textcolor}\sffamily\fontsize{10.000000}{12.000000}\selectfont 0.3}%
\end{pgfscope}%
\begin{pgfscope}%
\pgfsetbuttcap%
\pgfsetroundjoin%
\definecolor{currentfill}{rgb}{0.000000,0.000000,0.000000}%
\pgfsetfillcolor{currentfill}%
\pgfsetlinewidth{0.803000pt}%
\definecolor{currentstroke}{rgb}{0.000000,0.000000,0.000000}%
\pgfsetstrokecolor{currentstroke}%
\pgfsetdash{}{0pt}%
\pgfsys@defobject{currentmarker}{\pgfqpoint{-0.048611in}{0.000000in}}{\pgfqpoint{-0.000000in}{0.000000in}}{%
\pgfpathmoveto{\pgfqpoint{-0.000000in}{0.000000in}}%
\pgfpathlineto{\pgfqpoint{-0.048611in}{0.000000in}}%
\pgfusepath{stroke,fill}%
}%
\begin{pgfscope}%
\pgfsys@transformshift{0.608070in}{3.311845in}%
\pgfsys@useobject{currentmarker}{}%
\end{pgfscope}%
\end{pgfscope}%
\begin{pgfscope}%
\definecolor{textcolor}{rgb}{0.000000,0.000000,0.000000}%
\pgfsetstrokecolor{textcolor}%
\pgfsetfillcolor{textcolor}%
\pgftext[x=0.289968in, y=3.259084in, left, base]{\color{textcolor}\sffamily\fontsize{10.000000}{12.000000}\selectfont 0.4}%
\end{pgfscope}%
\begin{pgfscope}%
\pgfsetbuttcap%
\pgfsetroundjoin%
\definecolor{currentfill}{rgb}{0.000000,0.000000,0.000000}%
\pgfsetfillcolor{currentfill}%
\pgfsetlinewidth{0.803000pt}%
\definecolor{currentstroke}{rgb}{0.000000,0.000000,0.000000}%
\pgfsetstrokecolor{currentstroke}%
\pgfsetdash{}{0pt}%
\pgfsys@defobject{currentmarker}{\pgfqpoint{-0.048611in}{0.000000in}}{\pgfqpoint{-0.000000in}{0.000000in}}{%
\pgfpathmoveto{\pgfqpoint{-0.000000in}{0.000000in}}%
\pgfpathlineto{\pgfqpoint{-0.048611in}{0.000000in}}%
\pgfusepath{stroke,fill}%
}%
\begin{pgfscope}%
\pgfsys@transformshift{0.608070in}{3.848916in}%
\pgfsys@useobject{currentmarker}{}%
\end{pgfscope}%
\end{pgfscope}%
\begin{pgfscope}%
\definecolor{textcolor}{rgb}{0.000000,0.000000,0.000000}%
\pgfsetstrokecolor{textcolor}%
\pgfsetfillcolor{textcolor}%
\pgftext[x=0.289968in, y=3.796155in, left, base]{\color{textcolor}\sffamily\fontsize{10.000000}{12.000000}\selectfont 0.5}%
\end{pgfscope}%
\begin{pgfscope}%
\pgfsetbuttcap%
\pgfsetroundjoin%
\definecolor{currentfill}{rgb}{0.000000,0.000000,0.000000}%
\pgfsetfillcolor{currentfill}%
\pgfsetlinewidth{0.803000pt}%
\definecolor{currentstroke}{rgb}{0.000000,0.000000,0.000000}%
\pgfsetstrokecolor{currentstroke}%
\pgfsetdash{}{0pt}%
\pgfsys@defobject{currentmarker}{\pgfqpoint{-0.048611in}{0.000000in}}{\pgfqpoint{-0.000000in}{0.000000in}}{%
\pgfpathmoveto{\pgfqpoint{-0.000000in}{0.000000in}}%
\pgfpathlineto{\pgfqpoint{-0.048611in}{0.000000in}}%
\pgfusepath{stroke,fill}%
}%
\begin{pgfscope}%
\pgfsys@transformshift{0.608070in}{4.385987in}%
\pgfsys@useobject{currentmarker}{}%
\end{pgfscope}%
\end{pgfscope}%
\begin{pgfscope}%
\definecolor{textcolor}{rgb}{0.000000,0.000000,0.000000}%
\pgfsetstrokecolor{textcolor}%
\pgfsetfillcolor{textcolor}%
\pgftext[x=0.289968in, y=4.333225in, left, base]{\color{textcolor}\sffamily\fontsize{10.000000}{12.000000}\selectfont 0.6}%
\end{pgfscope}%
\begin{pgfscope}%
\definecolor{textcolor}{rgb}{0.000000,0.000000,0.000000}%
\pgfsetstrokecolor{textcolor}%
\pgfsetfillcolor{textcolor}%
\pgftext[x=0.234413in,y=2.774774in,,bottom,rotate=90.000000]{\color{textcolor}\sffamily\fontsize{10.000000}{12.000000}\bfseries\selectfont F1 Score}%
\end{pgfscope}%
\begin{pgfscope}%
\pgfsetrectcap%
\pgfsetmiterjoin%
\pgfsetlinewidth{0.803000pt}%
\definecolor{currentstroke}{rgb}{0.000000,0.000000,0.000000}%
\pgfsetstrokecolor{currentstroke}%
\pgfsetdash{}{0pt}%
\pgfpathmoveto{\pgfqpoint{0.608070in}{1.163562in}}%
\pgfpathlineto{\pgfqpoint{0.608070in}{4.385987in}}%
\pgfusepath{stroke}%
\end{pgfscope}%
\begin{pgfscope}%
\pgfsetrectcap%
\pgfsetmiterjoin%
\pgfsetlinewidth{0.803000pt}%
\definecolor{currentstroke}{rgb}{0.000000,0.000000,0.000000}%
\pgfsetstrokecolor{currentstroke}%
\pgfsetdash{}{0pt}%
\pgfpathmoveto{\pgfqpoint{6.194042in}{1.163562in}}%
\pgfpathlineto{\pgfqpoint{6.194042in}{4.385987in}}%
\pgfusepath{stroke}%
\end{pgfscope}%
\begin{pgfscope}%
\pgfsetrectcap%
\pgfsetmiterjoin%
\pgfsetlinewidth{0.803000pt}%
\definecolor{currentstroke}{rgb}{0.000000,0.000000,0.000000}%
\pgfsetstrokecolor{currentstroke}%
\pgfsetdash{}{0pt}%
\pgfpathmoveto{\pgfqpoint{0.608070in}{1.163562in}}%
\pgfpathlineto{\pgfqpoint{6.194042in}{1.163562in}}%
\pgfusepath{stroke}%
\end{pgfscope}%
\begin{pgfscope}%
\pgfsetrectcap%
\pgfsetmiterjoin%
\pgfsetlinewidth{0.803000pt}%
\definecolor{currentstroke}{rgb}{0.000000,0.000000,0.000000}%
\pgfsetstrokecolor{currentstroke}%
\pgfsetdash{}{0pt}%
\pgfpathmoveto{\pgfqpoint{0.608070in}{4.385987in}}%
\pgfpathlineto{\pgfqpoint{6.194042in}{4.385987in}}%
\pgfusepath{stroke}%
\end{pgfscope}%
\begin{pgfscope}%
\definecolor{textcolor}{rgb}{0.000000,0.000000,0.000000}%
\pgfsetstrokecolor{textcolor}%
\pgfsetfillcolor{textcolor}%
\pgftext[x=1.095158in,y=3.579619in,,bottom]{\color{textcolor}\sffamily\fontsize{9.000000}{10.800000}\selectfont 0.4421}%
\end{pgfscope}%
\begin{pgfscope}%
\definecolor{textcolor}{rgb}{0.000000,0.000000,0.000000}%
\pgfsetstrokecolor{textcolor}%
\pgfsetfillcolor{textcolor}%
\pgftext[x=2.131517in,y=1.998482in,,bottom]{\color{textcolor}\sffamily\fontsize{9.000000}{10.800000}\selectfont 0.1477}%
\end{pgfscope}%
\begin{pgfscope}%
\definecolor{textcolor}{rgb}{0.000000,0.000000,0.000000}%
\pgfsetstrokecolor{textcolor}%
\pgfsetfillcolor{textcolor}%
\pgftext[x=3.167875in,y=2.368470in,,bottom]{\color{textcolor}\sffamily\fontsize{9.000000}{10.800000}\selectfont 0.21659}%
\end{pgfscope}%
\begin{pgfscope}%
\definecolor{textcolor}{rgb}{0.000000,0.000000,0.000000}%
\pgfsetstrokecolor{textcolor}%
\pgfsetfillcolor{textcolor}%
\pgftext[x=4.204234in,y=3.448573in,,bottom]{\color{textcolor}\sffamily\fontsize{9.000000}{10.800000}\selectfont 0.4177}%
\end{pgfscope}%
\begin{pgfscope}%
\definecolor{textcolor}{rgb}{0.000000,0.000000,0.000000}%
\pgfsetstrokecolor{textcolor}%
\pgfsetfillcolor{textcolor}%
\pgftext[x=5.240592in,y=3.388421in,,bottom]{\color{textcolor}\sffamily\fontsize{9.000000}{10.800000}\selectfont 0.4065}%
\end{pgfscope}%
\begin{pgfscope}%
\definecolor{textcolor}{rgb}{0.000000,0.000000,0.000000}%
\pgfsetstrokecolor{textcolor}%
\pgfsetfillcolor{textcolor}%
\pgftext[x=1.561520in,y=3.504429in,,bottom]{\color{textcolor}\sffamily\fontsize{9.000000}{10.800000}\selectfont 0.4281}%
\end{pgfscope}%
\begin{pgfscope}%
\definecolor{textcolor}{rgb}{0.000000,0.000000,0.000000}%
\pgfsetstrokecolor{textcolor}%
\pgfsetfillcolor{textcolor}%
\pgftext[x=2.597878in,y=1.763782in,,bottom]{\color{textcolor}\sffamily\fontsize{9.000000}{10.800000}\selectfont 0.104}%
\end{pgfscope}%
\begin{pgfscope}%
\definecolor{textcolor}{rgb}{0.000000,0.000000,0.000000}%
\pgfsetstrokecolor{textcolor}%
\pgfsetfillcolor{textcolor}%
\pgftext[x=3.634237in,y=2.639208in,,bottom]{\color{textcolor}\sffamily\fontsize{9.000000}{10.800000}\selectfont 0.267}%
\end{pgfscope}%
\begin{pgfscope}%
\definecolor{textcolor}{rgb}{0.000000,0.000000,0.000000}%
\pgfsetstrokecolor{textcolor}%
\pgfsetfillcolor{textcolor}%
\pgftext[x=4.670595in,y=3.436758in,,bottom]{\color{textcolor}\sffamily\fontsize{9.000000}{10.800000}\selectfont 0.4155}%
\end{pgfscope}%
\begin{pgfscope}%
\definecolor{textcolor}{rgb}{0.000000,0.000000,0.000000}%
\pgfsetstrokecolor{textcolor}%
\pgfsetfillcolor{textcolor}%
\pgftext[x=5.706954in,y=3.520004in,,bottom]{\color{textcolor}\sffamily\fontsize{9.000000}{10.800000}\selectfont 0.431}%
\end{pgfscope}%
\begin{pgfscope}%
\definecolor{textcolor}{rgb}{0.000000,0.000000,0.000000}%
\pgfsetstrokecolor{textcolor}%
\pgfsetfillcolor{textcolor}%
\pgftext[x=3.401056in,y=4.469320in,,base]{\color{textcolor}\sffamily\fontsize{12.000000}{14.400000}\selectfont Baselines trained on S2R:3DFREE and fine-tuned with Pix3D}%
\end{pgfscope}%
\begin{pgfscope}%
\pgfsetbuttcap%
\pgfsetmiterjoin%
\definecolor{currentfill}{rgb}{1.000000,1.000000,1.000000}%
\pgfsetfillcolor{currentfill}%
\pgfsetfillopacity{0.800000}%
\pgfsetlinewidth{1.003750pt}%
\definecolor{currentstroke}{rgb}{0.800000,0.800000,0.800000}%
\pgfsetstrokecolor{currentstroke}%
\pgfsetstrokeopacity{0.800000}%
\pgfsetdash{}{0pt}%
\pgfpathmoveto{\pgfqpoint{4.882654in}{3.867161in}}%
\pgfpathlineto{\pgfqpoint{6.096820in}{3.867161in}}%
\pgfpathquadraticcurveto{\pgfqpoint{6.124598in}{3.867161in}}{\pgfqpoint{6.124598in}{3.894939in}}%
\pgfpathlineto{\pgfqpoint{6.124598in}{4.288765in}}%
\pgfpathquadraticcurveto{\pgfqpoint{6.124598in}{4.316543in}}{\pgfqpoint{6.096820in}{4.316543in}}%
\pgfpathlineto{\pgfqpoint{4.882654in}{4.316543in}}%
\pgfpathquadraticcurveto{\pgfqpoint{4.854877in}{4.316543in}}{\pgfqpoint{4.854877in}{4.288765in}}%
\pgfpathlineto{\pgfqpoint{4.854877in}{3.894939in}}%
\pgfpathquadraticcurveto{\pgfqpoint{4.854877in}{3.867161in}}{\pgfqpoint{4.882654in}{3.867161in}}%
\pgfpathclose%
\pgfusepath{stroke,fill}%
\end{pgfscope}%
\begin{pgfscope}%
\pgfsetbuttcap%
\pgfsetmiterjoin%
\definecolor{currentfill}{rgb}{0.121569,0.466667,0.705882}%
\pgfsetfillcolor{currentfill}%
\pgfsetlinewidth{0.000000pt}%
\definecolor{currentstroke}{rgb}{0.000000,0.000000,0.000000}%
\pgfsetstrokecolor{currentstroke}%
\pgfsetstrokeopacity{0.000000}%
\pgfsetdash{}{0pt}%
\pgfpathmoveto{\pgfqpoint{4.910432in}{4.155464in}}%
\pgfpathlineto{\pgfqpoint{5.188210in}{4.155464in}}%
\pgfpathlineto{\pgfqpoint{5.188210in}{4.252686in}}%
\pgfpathlineto{\pgfqpoint{4.910432in}{4.252686in}}%
\pgfpathclose%
\pgfusepath{fill}%
\end{pgfscope}%
\begin{pgfscope}%
\definecolor{textcolor}{rgb}{0.000000,0.000000,0.000000}%
\pgfsetstrokecolor{textcolor}%
\pgfsetfillcolor{textcolor}%
\pgftext[x=5.299321in,y=4.155464in,left,base]{\color{textcolor}\sffamily\fontsize{10.000000}{12.000000}\selectfont Pix2Vox++}%
\end{pgfscope}%
\begin{pgfscope}%
\pgfsetbuttcap%
\pgfsetmiterjoin%
\definecolor{currentfill}{rgb}{1.000000,0.498039,0.054902}%
\pgfsetfillcolor{currentfill}%
\pgfsetlinewidth{0.000000pt}%
\definecolor{currentstroke}{rgb}{0.000000,0.000000,0.000000}%
\pgfsetstrokecolor{currentstroke}%
\pgfsetstrokeopacity{0.000000}%
\pgfsetdash{}{0pt}%
\pgfpathmoveto{\pgfqpoint{4.910432in}{3.951607in}}%
\pgfpathlineto{\pgfqpoint{5.188210in}{3.951607in}}%
\pgfpathlineto{\pgfqpoint{5.188210in}{4.048829in}}%
\pgfpathlineto{\pgfqpoint{4.910432in}{4.048829in}}%
\pgfpathclose%
\pgfusepath{fill}%
\end{pgfscope}%
\begin{pgfscope}%
\definecolor{textcolor}{rgb}{0.000000,0.000000,0.000000}%
\pgfsetstrokecolor{textcolor}%
\pgfsetfillcolor{textcolor}%
\pgftext[x=5.299321in,y=3.951607in,left,base]{\color{textcolor}\sffamily\fontsize{10.000000}{12.000000}\selectfont Pix2Vox}%
\end{pgfscope}%
\end{pgfpicture}%
\makeatother%
\endgroup%
}
    \caption{Bar plot for the \gls{f1} for baseline models(Pix2Vox++ and Pix2Vox) trained on synthetic and fine-tuned with real dataset.
    We see that even after fine-tuning both the models do not perform as good as models trained on only real dataset.}
    \label{fig:finetuning_dice1}
\end{figure}

\begin{figure}
    \centering
    \resizebox{0.75\textwidth}{!}{\input{/Users/apple/OVGU/Thesis/code/3dReconstruction/report/images/evaluation/performance/finetune_dice_barplot_pix2voxpp.pgf}}
    \caption{Bar plot for the \gls{f1} for baseline pix2vox++ trained on (\gls{s2rv1}, \gls{s2rv2}) and fine-tuned with pix3d.
    The categories are listed along with the number of images.
    The performance of \texbf{pix2vox++} mixed with both the synthetic dataset is less than model trained on only pix3d, for majority of the categories.}
    \label{fig:finetuning_dice2}
\end{figure}

\begin{figure}
    \centering
    \resizebox{0.75\textwidth}{!}{\input{/Users/apple/OVGU/Thesis/code/3dReconstruction/report/images/evaluation/performance/finetune_dice_barplot_pix2vox.pgf}}
    \caption{Bar plot for the \gls{f1} for baseline \texbf{pix2vox} trained on (\gls{s2rv1}, \gls{s2rv2}) and fine-tuned with pix3d.
    The categories are listed along with the number of images.
    The performance of pix2vox mixed with both the synthetic dataset is less than model trained on only pix3d, for majority the categories.}
    \label{fig:finetuning_dice3}
\end{figure}

\subsection{Mixed training}\label{subsec:mixed-training-dice}

Similar to the experiments conducted in \autoref{sec:mixed-training}, we also note the \gls{f1} for training executed with different mixing ratios per mini-batch.
The results in \autoref{fig:mixed_dice1} are consistent with previous experiment(\autoref{fig:mixed1}).
The performance increases slightly with an increase in the ratio of real data in mini-batch and decreases performance gradually until it is 100\% real dataset.
\autoref{fig:mixed_dice2} and \autoref{fig:mixed_dice3} are \gls{f1} per category when trained with a ratio of 50\% of real data per mini-batch, on pix2vox++ and pix2vox, respectively.
%\begin{figure}
%    \centering
%    \resizebox{0.49\linewidth}{!}{\input{/Users/apple/OVGU/Thesis/code/3dReconstruction/report/images/evaluation/performance/mixed_dice_linegraph1.pgf}}
%    \resizebox{0.49\linewidth}{!}{\input{/Users/apple/OVGU/Thesis/code/3dReconstruction/report/images/evaluation/performance/mixed_dice_linegraph2.pgf}}
%    \caption{Line plot for the \gls{f1}  for baselines trained on different ratios of synthetic and real dataset.
%        (Left)Mixed training on Pix2Vox++, (right)Mixed training on Pix2Vox. In both cases we see a slight increase in \gls{f1} with addition of real data, and a gradual decrease till it reaches 100\% real data}
%    \label{fig:mixed_dice1}
%\end{figure}

\begin{figure}
    \centering
    \resizebox{0.49\linewidth}{0.45\linewidth}{\input{/Users/apple/OVGU/Thesis/code/3dReconstruction/report/images/evaluation/performance/mixed_dice_barplot1.pgf}}
    \resizebox{0.49\linewidth}{0.45\linewidth}{%% Creator: Matplotlib, PGF backend
%%
%% To include the figure in your LaTeX document, write
%%   \input{<filename>.pgf}
%%
%% Make sure the required packages are loaded in your preamble
%%   \usepackage{pgf}
%%
%% Figures using additional raster images can only be included by \input if
%% they are in the same directory as the main LaTeX file. For loading figures
%% from other directories you can use the `import` package
%%   \usepackage{import}
%%
%% and then include the figures with
%%   \import{<path to file>}{<filename>.pgf}
%%
%% Matplotlib used the following preamble
%%   \usepackage{fontspec}
%%   \setmainfont{DejaVuSerif.ttf}[Path=\detokenize{/Users/apple/opt/anaconda3/envs/kaolin/lib/python3.7/site-packages/matplotlib/mpl-data/fonts/ttf/}]
%%   \setsansfont{DejaVuSans.ttf}[Path=\detokenize{/Users/apple/opt/anaconda3/envs/kaolin/lib/python3.7/site-packages/matplotlib/mpl-data/fonts/ttf/}]
%%   \setmonofont{DejaVuSansMono.ttf}[Path=\detokenize{/Users/apple/opt/anaconda3/envs/kaolin/lib/python3.7/site-packages/matplotlib/mpl-data/fonts/ttf/}]
%%
\begingroup%
\makeatletter%
\begin{pgfpicture}%
\pgfpathrectangle{\pgfpointorigin}{\pgfqpoint{6.293658in}{4.697056in}}%
\pgfusepath{use as bounding box, clip}%
\begin{pgfscope}%
\pgfsetbuttcap%
\pgfsetmiterjoin%
\definecolor{currentfill}{rgb}{1.000000,1.000000,1.000000}%
\pgfsetfillcolor{currentfill}%
\pgfsetlinewidth{0.000000pt}%
\definecolor{currentstroke}{rgb}{1.000000,1.000000,1.000000}%
\pgfsetstrokecolor{currentstroke}%
\pgfsetdash{}{0pt}%
\pgfpathmoveto{\pgfqpoint{0.000000in}{0.000000in}}%
\pgfpathlineto{\pgfqpoint{6.293658in}{0.000000in}}%
\pgfpathlineto{\pgfqpoint{6.293658in}{4.697056in}}%
\pgfpathlineto{\pgfqpoint{0.000000in}{4.697056in}}%
\pgfpathclose%
\pgfusepath{fill}%
\end{pgfscope}%
\begin{pgfscope}%
\pgfsetbuttcap%
\pgfsetmiterjoin%
\definecolor{currentfill}{rgb}{1.000000,1.000000,1.000000}%
\pgfsetfillcolor{currentfill}%
\pgfsetlinewidth{0.000000pt}%
\definecolor{currentstroke}{rgb}{0.000000,0.000000,0.000000}%
\pgfsetstrokecolor{currentstroke}%
\pgfsetstrokeopacity{0.000000}%
\pgfsetdash{}{0pt}%
\pgfpathmoveto{\pgfqpoint{0.696435in}{0.700520in}}%
\pgfpathlineto{\pgfqpoint{6.193658in}{0.700520in}}%
\pgfpathlineto{\pgfqpoint{6.193658in}{4.387095in}}%
\pgfpathlineto{\pgfqpoint{0.696435in}{4.387095in}}%
\pgfpathclose%
\pgfusepath{fill}%
\end{pgfscope}%
\begin{pgfscope}%
\pgfpathrectangle{\pgfqpoint{0.696435in}{0.700520in}}{\pgfqpoint{5.497222in}{3.686575in}}%
\pgfusepath{clip}%
\pgfsetbuttcap%
\pgfsetmiterjoin%
\definecolor{currentfill}{rgb}{0.121569,0.466667,0.705882}%
\pgfsetfillcolor{currentfill}%
\pgfsetlinewidth{0.000000pt}%
\definecolor{currentstroke}{rgb}{0.000000,0.000000,0.000000}%
\pgfsetstrokecolor{currentstroke}%
\pgfsetstrokeopacity{0.000000}%
\pgfsetdash{}{0pt}%
\pgfpathmoveto{\pgfqpoint{0.946309in}{-7.901488in}}%
\pgfpathlineto{\pgfqpoint{1.405261in}{-7.901488in}}%
\pgfpathlineto{\pgfqpoint{1.405261in}{3.099251in}}%
\pgfpathlineto{\pgfqpoint{0.946309in}{3.099251in}}%
\pgfpathclose%
\pgfusepath{fill}%
\end{pgfscope}%
\begin{pgfscope}%
\pgfpathrectangle{\pgfqpoint{0.696435in}{0.700520in}}{\pgfqpoint{5.497222in}{3.686575in}}%
\pgfusepath{clip}%
\pgfsetbuttcap%
\pgfsetmiterjoin%
\definecolor{currentfill}{rgb}{0.121569,0.466667,0.705882}%
\pgfsetfillcolor{currentfill}%
\pgfsetlinewidth{0.000000pt}%
\definecolor{currentstroke}{rgb}{0.000000,0.000000,0.000000}%
\pgfsetstrokecolor{currentstroke}%
\pgfsetstrokeopacity{0.000000}%
\pgfsetdash{}{0pt}%
\pgfpathmoveto{\pgfqpoint{1.966202in}{-7.901488in}}%
\pgfpathlineto{\pgfqpoint{2.425154in}{-7.901488in}}%
\pgfpathlineto{\pgfqpoint{2.425154in}{2.981281in}}%
\pgfpathlineto{\pgfqpoint{1.966202in}{2.981281in}}%
\pgfpathclose%
\pgfusepath{fill}%
\end{pgfscope}%
\begin{pgfscope}%
\pgfpathrectangle{\pgfqpoint{0.696435in}{0.700520in}}{\pgfqpoint{5.497222in}{3.686575in}}%
\pgfusepath{clip}%
\pgfsetbuttcap%
\pgfsetmiterjoin%
\definecolor{currentfill}{rgb}{0.121569,0.466667,0.705882}%
\pgfsetfillcolor{currentfill}%
\pgfsetlinewidth{0.000000pt}%
\definecolor{currentstroke}{rgb}{0.000000,0.000000,0.000000}%
\pgfsetstrokecolor{currentstroke}%
\pgfsetstrokeopacity{0.000000}%
\pgfsetdash{}{0pt}%
\pgfpathmoveto{\pgfqpoint{2.986095in}{-7.901488in}}%
\pgfpathlineto{\pgfqpoint{3.445047in}{-7.901488in}}%
\pgfpathlineto{\pgfqpoint{3.445047in}{3.155779in}}%
\pgfpathlineto{\pgfqpoint{2.986095in}{3.155779in}}%
\pgfpathclose%
\pgfusepath{fill}%
\end{pgfscope}%
\begin{pgfscope}%
\pgfpathrectangle{\pgfqpoint{0.696435in}{0.700520in}}{\pgfqpoint{5.497222in}{3.686575in}}%
\pgfusepath{clip}%
\pgfsetbuttcap%
\pgfsetmiterjoin%
\definecolor{currentfill}{rgb}{0.121569,0.466667,0.705882}%
\pgfsetfillcolor{currentfill}%
\pgfsetlinewidth{0.000000pt}%
\definecolor{currentstroke}{rgb}{0.000000,0.000000,0.000000}%
\pgfsetstrokecolor{currentstroke}%
\pgfsetstrokeopacity{0.000000}%
\pgfsetdash{}{0pt}%
\pgfpathmoveto{\pgfqpoint{4.005988in}{-7.901488in}}%
\pgfpathlineto{\pgfqpoint{4.464939in}{-7.901488in}}%
\pgfpathlineto{\pgfqpoint{4.464939in}{2.890346in}}%
\pgfpathlineto{\pgfqpoint{4.005988in}{2.890346in}}%
\pgfpathclose%
\pgfusepath{fill}%
\end{pgfscope}%
\begin{pgfscope}%
\pgfpathrectangle{\pgfqpoint{0.696435in}{0.700520in}}{\pgfqpoint{5.497222in}{3.686575in}}%
\pgfusepath{clip}%
\pgfsetbuttcap%
\pgfsetmiterjoin%
\definecolor{currentfill}{rgb}{0.121569,0.466667,0.705882}%
\pgfsetfillcolor{currentfill}%
\pgfsetlinewidth{0.000000pt}%
\definecolor{currentstroke}{rgb}{0.000000,0.000000,0.000000}%
\pgfsetstrokecolor{currentstroke}%
\pgfsetstrokeopacity{0.000000}%
\pgfsetdash{}{0pt}%
\pgfpathmoveto{\pgfqpoint{5.025880in}{-7.901488in}}%
\pgfpathlineto{\pgfqpoint{5.484832in}{-7.901488in}}%
\pgfpathlineto{\pgfqpoint{5.484832in}{2.801868in}}%
\pgfpathlineto{\pgfqpoint{5.025880in}{2.801868in}}%
\pgfpathclose%
\pgfusepath{fill}%
\end{pgfscope}%
\begin{pgfscope}%
\pgfpathrectangle{\pgfqpoint{0.696435in}{0.700520in}}{\pgfqpoint{5.497222in}{3.686575in}}%
\pgfusepath{clip}%
\pgfsetbuttcap%
\pgfsetmiterjoin%
\definecolor{currentfill}{rgb}{1.000000,0.498039,0.054902}%
\pgfsetfillcolor{currentfill}%
\pgfsetlinewidth{0.000000pt}%
\definecolor{currentstroke}{rgb}{0.000000,0.000000,0.000000}%
\pgfsetstrokecolor{currentstroke}%
\pgfsetstrokeopacity{0.000000}%
\pgfsetdash{}{0pt}%
\pgfpathmoveto{\pgfqpoint{1.405261in}{-7.901488in}}%
\pgfpathlineto{\pgfqpoint{1.864213in}{-7.901488in}}%
\pgfpathlineto{\pgfqpoint{1.864213in}{2.076841in}}%
\pgfpathlineto{\pgfqpoint{1.405261in}{2.076841in}}%
\pgfpathclose%
\pgfusepath{fill}%
\end{pgfscope}%
\begin{pgfscope}%
\pgfpathrectangle{\pgfqpoint{0.696435in}{0.700520in}}{\pgfqpoint{5.497222in}{3.686575in}}%
\pgfusepath{clip}%
\pgfsetbuttcap%
\pgfsetmiterjoin%
\definecolor{currentfill}{rgb}{1.000000,0.498039,0.054902}%
\pgfsetfillcolor{currentfill}%
\pgfsetlinewidth{0.000000pt}%
\definecolor{currentstroke}{rgb}{0.000000,0.000000,0.000000}%
\pgfsetstrokecolor{currentstroke}%
\pgfsetstrokeopacity{0.000000}%
\pgfsetdash{}{0pt}%
\pgfpathmoveto{\pgfqpoint{2.425154in}{-7.901488in}}%
\pgfpathlineto{\pgfqpoint{2.884105in}{-7.901488in}}%
\pgfpathlineto{\pgfqpoint{2.884105in}{3.190187in}}%
\pgfpathlineto{\pgfqpoint{2.425154in}{3.190187in}}%
\pgfpathclose%
\pgfusepath{fill}%
\end{pgfscope}%
\begin{pgfscope}%
\pgfpathrectangle{\pgfqpoint{0.696435in}{0.700520in}}{\pgfqpoint{5.497222in}{3.686575in}}%
\pgfusepath{clip}%
\pgfsetbuttcap%
\pgfsetmiterjoin%
\definecolor{currentfill}{rgb}{1.000000,0.498039,0.054902}%
\pgfsetfillcolor{currentfill}%
\pgfsetlinewidth{0.000000pt}%
\definecolor{currentstroke}{rgb}{0.000000,0.000000,0.000000}%
\pgfsetstrokecolor{currentstroke}%
\pgfsetstrokeopacity{0.000000}%
\pgfsetdash{}{0pt}%
\pgfpathmoveto{\pgfqpoint{3.445047in}{-7.901488in}}%
\pgfpathlineto{\pgfqpoint{3.903998in}{-7.901488in}}%
\pgfpathlineto{\pgfqpoint{3.903998in}{2.794495in}}%
\pgfpathlineto{\pgfqpoint{3.445047in}{2.794495in}}%
\pgfpathclose%
\pgfusepath{fill}%
\end{pgfscope}%
\begin{pgfscope}%
\pgfpathrectangle{\pgfqpoint{0.696435in}{0.700520in}}{\pgfqpoint{5.497222in}{3.686575in}}%
\pgfusepath{clip}%
\pgfsetbuttcap%
\pgfsetmiterjoin%
\definecolor{currentfill}{rgb}{1.000000,0.498039,0.054902}%
\pgfsetfillcolor{currentfill}%
\pgfsetlinewidth{0.000000pt}%
\definecolor{currentstroke}{rgb}{0.000000,0.000000,0.000000}%
\pgfsetstrokecolor{currentstroke}%
\pgfsetstrokeopacity{0.000000}%
\pgfsetdash{}{0pt}%
\pgfpathmoveto{\pgfqpoint{4.464939in}{-7.901488in}}%
\pgfpathlineto{\pgfqpoint{4.923891in}{-7.901488in}}%
\pgfpathlineto{\pgfqpoint{4.923891in}{2.846107in}}%
\pgfpathlineto{\pgfqpoint{4.464939in}{2.846107in}}%
\pgfpathclose%
\pgfusepath{fill}%
\end{pgfscope}%
\begin{pgfscope}%
\pgfpathrectangle{\pgfqpoint{0.696435in}{0.700520in}}{\pgfqpoint{5.497222in}{3.686575in}}%
\pgfusepath{clip}%
\pgfsetbuttcap%
\pgfsetmiterjoin%
\definecolor{currentfill}{rgb}{1.000000,0.498039,0.054902}%
\pgfsetfillcolor{currentfill}%
\pgfsetlinewidth{0.000000pt}%
\definecolor{currentstroke}{rgb}{0.000000,0.000000,0.000000}%
\pgfsetstrokecolor{currentstroke}%
\pgfsetstrokeopacity{0.000000}%
\pgfsetdash{}{0pt}%
\pgfpathmoveto{\pgfqpoint{5.484832in}{-7.901488in}}%
\pgfpathlineto{\pgfqpoint{5.943784in}{-7.901488in}}%
\pgfpathlineto{\pgfqpoint{5.943784in}{2.887888in}}%
\pgfpathlineto{\pgfqpoint{5.484832in}{2.887888in}}%
\pgfpathclose%
\pgfusepath{fill}%
\end{pgfscope}%
\begin{pgfscope}%
\pgfsetbuttcap%
\pgfsetroundjoin%
\definecolor{currentfill}{rgb}{0.000000,0.000000,0.000000}%
\pgfsetfillcolor{currentfill}%
\pgfsetlinewidth{0.803000pt}%
\definecolor{currentstroke}{rgb}{0.000000,0.000000,0.000000}%
\pgfsetstrokecolor{currentstroke}%
\pgfsetdash{}{0pt}%
\pgfsys@defobject{currentmarker}{\pgfqpoint{0.000000in}{-0.048611in}}{\pgfqpoint{0.000000in}{0.000000in}}{%
\pgfpathmoveto{\pgfqpoint{0.000000in}{0.000000in}}%
\pgfpathlineto{\pgfqpoint{0.000000in}{-0.048611in}}%
\pgfusepath{stroke,fill}%
}%
\begin{pgfscope}%
\pgfsys@transformshift{1.405261in}{0.700520in}%
\pgfsys@useobject{currentmarker}{}%
\end{pgfscope}%
\end{pgfscope}%
\begin{pgfscope}%
\definecolor{textcolor}{rgb}{0.000000,0.000000,0.000000}%
\pgfsetstrokecolor{textcolor}%
\pgfsetfillcolor{textcolor}%
\pgftext[x=1.323212in, y=0.310396in, left, base,rotate=45.000000]{\color{textcolor}\sffamily\fontsize{10.000000}{12.000000}\selectfont 15\%}%
\end{pgfscope}%
\begin{pgfscope}%
\pgfsetbuttcap%
\pgfsetroundjoin%
\definecolor{currentfill}{rgb}{0.000000,0.000000,0.000000}%
\pgfsetfillcolor{currentfill}%
\pgfsetlinewidth{0.803000pt}%
\definecolor{currentstroke}{rgb}{0.000000,0.000000,0.000000}%
\pgfsetstrokecolor{currentstroke}%
\pgfsetdash{}{0pt}%
\pgfsys@defobject{currentmarker}{\pgfqpoint{0.000000in}{-0.048611in}}{\pgfqpoint{0.000000in}{0.000000in}}{%
\pgfpathmoveto{\pgfqpoint{0.000000in}{0.000000in}}%
\pgfpathlineto{\pgfqpoint{0.000000in}{-0.048611in}}%
\pgfusepath{stroke,fill}%
}%
\begin{pgfscope}%
\pgfsys@transformshift{2.425154in}{0.700520in}%
\pgfsys@useobject{currentmarker}{}%
\end{pgfscope}%
\end{pgfscope}%
\begin{pgfscope}%
\definecolor{textcolor}{rgb}{0.000000,0.000000,0.000000}%
\pgfsetstrokecolor{textcolor}%
\pgfsetfillcolor{textcolor}%
\pgftext[x=2.343105in, y=0.310396in, left, base,rotate=45.000000]{\color{textcolor}\sffamily\fontsize{10.000000}{12.000000}\selectfont 25\%}%
\end{pgfscope}%
\begin{pgfscope}%
\pgfsetbuttcap%
\pgfsetroundjoin%
\definecolor{currentfill}{rgb}{0.000000,0.000000,0.000000}%
\pgfsetfillcolor{currentfill}%
\pgfsetlinewidth{0.803000pt}%
\definecolor{currentstroke}{rgb}{0.000000,0.000000,0.000000}%
\pgfsetstrokecolor{currentstroke}%
\pgfsetdash{}{0pt}%
\pgfsys@defobject{currentmarker}{\pgfqpoint{0.000000in}{-0.048611in}}{\pgfqpoint{0.000000in}{0.000000in}}{%
\pgfpathmoveto{\pgfqpoint{0.000000in}{0.000000in}}%
\pgfpathlineto{\pgfqpoint{0.000000in}{-0.048611in}}%
\pgfusepath{stroke,fill}%
}%
\begin{pgfscope}%
\pgfsys@transformshift{3.445047in}{0.700520in}%
\pgfsys@useobject{currentmarker}{}%
\end{pgfscope}%
\end{pgfscope}%
\begin{pgfscope}%
\definecolor{textcolor}{rgb}{0.000000,0.000000,0.000000}%
\pgfsetstrokecolor{textcolor}%
\pgfsetfillcolor{textcolor}%
\pgftext[x=3.362998in, y=0.310396in, left, base,rotate=45.000000]{\color{textcolor}\sffamily\fontsize{10.000000}{12.000000}\selectfont 50\%}%
\end{pgfscope}%
\begin{pgfscope}%
\pgfsetbuttcap%
\pgfsetroundjoin%
\definecolor{currentfill}{rgb}{0.000000,0.000000,0.000000}%
\pgfsetfillcolor{currentfill}%
\pgfsetlinewidth{0.803000pt}%
\definecolor{currentstroke}{rgb}{0.000000,0.000000,0.000000}%
\pgfsetstrokecolor{currentstroke}%
\pgfsetdash{}{0pt}%
\pgfsys@defobject{currentmarker}{\pgfqpoint{0.000000in}{-0.048611in}}{\pgfqpoint{0.000000in}{0.000000in}}{%
\pgfpathmoveto{\pgfqpoint{0.000000in}{0.000000in}}%
\pgfpathlineto{\pgfqpoint{0.000000in}{-0.048611in}}%
\pgfusepath{stroke,fill}%
}%
\begin{pgfscope}%
\pgfsys@transformshift{4.464939in}{0.700520in}%
\pgfsys@useobject{currentmarker}{}%
\end{pgfscope}%
\end{pgfscope}%
\begin{pgfscope}%
\definecolor{textcolor}{rgb}{0.000000,0.000000,0.000000}%
\pgfsetstrokecolor{textcolor}%
\pgfsetfillcolor{textcolor}%
\pgftext[x=4.382891in, y=0.310396in, left, base,rotate=45.000000]{\color{textcolor}\sffamily\fontsize{10.000000}{12.000000}\selectfont 75\%}%
\end{pgfscope}%
\begin{pgfscope}%
\pgfsetbuttcap%
\pgfsetroundjoin%
\definecolor{currentfill}{rgb}{0.000000,0.000000,0.000000}%
\pgfsetfillcolor{currentfill}%
\pgfsetlinewidth{0.803000pt}%
\definecolor{currentstroke}{rgb}{0.000000,0.000000,0.000000}%
\pgfsetstrokecolor{currentstroke}%
\pgfsetdash{}{0pt}%
\pgfsys@defobject{currentmarker}{\pgfqpoint{0.000000in}{-0.048611in}}{\pgfqpoint{0.000000in}{0.000000in}}{%
\pgfpathmoveto{\pgfqpoint{0.000000in}{0.000000in}}%
\pgfpathlineto{\pgfqpoint{0.000000in}{-0.048611in}}%
\pgfusepath{stroke,fill}%
}%
\begin{pgfscope}%
\pgfsys@transformshift{5.484832in}{0.700520in}%
\pgfsys@useobject{currentmarker}{}%
\end{pgfscope}%
\end{pgfscope}%
\begin{pgfscope}%
\definecolor{textcolor}{rgb}{0.000000,0.000000,0.000000}%
\pgfsetstrokecolor{textcolor}%
\pgfsetfillcolor{textcolor}%
\pgftext[x=5.402783in, y=0.310396in, left, base,rotate=45.000000]{\color{textcolor}\sffamily\fontsize{10.000000}{12.000000}\selectfont 90\%}%
\end{pgfscope}%
\begin{pgfscope}%
\definecolor{textcolor}{rgb}{0.000000,0.000000,0.000000}%
\pgfsetstrokecolor{textcolor}%
\pgfsetfillcolor{textcolor}%
\pgftext[x=3.445047in,y=0.234413in,,top]{\color{textcolor}\sffamily\fontsize{10.000000}{12.000000}\bfseries\selectfont Dataset}%
\end{pgfscope}%
\begin{pgfscope}%
\pgfsetbuttcap%
\pgfsetroundjoin%
\definecolor{currentfill}{rgb}{0.000000,0.000000,0.000000}%
\pgfsetfillcolor{currentfill}%
\pgfsetlinewidth{0.803000pt}%
\definecolor{currentstroke}{rgb}{0.000000,0.000000,0.000000}%
\pgfsetstrokecolor{currentstroke}%
\pgfsetdash{}{0pt}%
\pgfsys@defobject{currentmarker}{\pgfqpoint{-0.048611in}{0.000000in}}{\pgfqpoint{-0.000000in}{0.000000in}}{%
\pgfpathmoveto{\pgfqpoint{-0.000000in}{0.000000in}}%
\pgfpathlineto{\pgfqpoint{-0.048611in}{0.000000in}}%
\pgfusepath{stroke,fill}%
}%
\begin{pgfscope}%
\pgfsys@transformshift{0.696435in}{0.946292in}%
\pgfsys@useobject{currentmarker}{}%
\end{pgfscope}%
\end{pgfscope}%
\begin{pgfscope}%
\definecolor{textcolor}{rgb}{0.000000,0.000000,0.000000}%
\pgfsetstrokecolor{textcolor}%
\pgfsetfillcolor{textcolor}%
\pgftext[x=0.289968in, y=0.893530in, left, base]{\color{textcolor}\sffamily\fontsize{10.000000}{12.000000}\selectfont 0.36}%
\end{pgfscope}%
\begin{pgfscope}%
\pgfsetbuttcap%
\pgfsetroundjoin%
\definecolor{currentfill}{rgb}{0.000000,0.000000,0.000000}%
\pgfsetfillcolor{currentfill}%
\pgfsetlinewidth{0.803000pt}%
\definecolor{currentstroke}{rgb}{0.000000,0.000000,0.000000}%
\pgfsetstrokecolor{currentstroke}%
\pgfsetdash{}{0pt}%
\pgfsys@defobject{currentmarker}{\pgfqpoint{-0.048611in}{0.000000in}}{\pgfqpoint{-0.000000in}{0.000000in}}{%
\pgfpathmoveto{\pgfqpoint{-0.000000in}{0.000000in}}%
\pgfpathlineto{\pgfqpoint{-0.048611in}{0.000000in}}%
\pgfusepath{stroke,fill}%
}%
\begin{pgfscope}%
\pgfsys@transformshift{0.696435in}{1.437835in}%
\pgfsys@useobject{currentmarker}{}%
\end{pgfscope}%
\end{pgfscope}%
\begin{pgfscope}%
\definecolor{textcolor}{rgb}{0.000000,0.000000,0.000000}%
\pgfsetstrokecolor{textcolor}%
\pgfsetfillcolor{textcolor}%
\pgftext[x=0.289968in, y=1.385074in, left, base]{\color{textcolor}\sffamily\fontsize{10.000000}{12.000000}\selectfont 0.38}%
\end{pgfscope}%
\begin{pgfscope}%
\pgfsetbuttcap%
\pgfsetroundjoin%
\definecolor{currentfill}{rgb}{0.000000,0.000000,0.000000}%
\pgfsetfillcolor{currentfill}%
\pgfsetlinewidth{0.803000pt}%
\definecolor{currentstroke}{rgb}{0.000000,0.000000,0.000000}%
\pgfsetstrokecolor{currentstroke}%
\pgfsetdash{}{0pt}%
\pgfsys@defobject{currentmarker}{\pgfqpoint{-0.048611in}{0.000000in}}{\pgfqpoint{-0.000000in}{0.000000in}}{%
\pgfpathmoveto{\pgfqpoint{-0.000000in}{0.000000in}}%
\pgfpathlineto{\pgfqpoint{-0.048611in}{0.000000in}}%
\pgfusepath{stroke,fill}%
}%
\begin{pgfscope}%
\pgfsys@transformshift{0.696435in}{1.929378in}%
\pgfsys@useobject{currentmarker}{}%
\end{pgfscope}%
\end{pgfscope}%
\begin{pgfscope}%
\definecolor{textcolor}{rgb}{0.000000,0.000000,0.000000}%
\pgfsetstrokecolor{textcolor}%
\pgfsetfillcolor{textcolor}%
\pgftext[x=0.289968in, y=1.876617in, left, base]{\color{textcolor}\sffamily\fontsize{10.000000}{12.000000}\selectfont 0.40}%
\end{pgfscope}%
\begin{pgfscope}%
\pgfsetbuttcap%
\pgfsetroundjoin%
\definecolor{currentfill}{rgb}{0.000000,0.000000,0.000000}%
\pgfsetfillcolor{currentfill}%
\pgfsetlinewidth{0.803000pt}%
\definecolor{currentstroke}{rgb}{0.000000,0.000000,0.000000}%
\pgfsetstrokecolor{currentstroke}%
\pgfsetdash{}{0pt}%
\pgfsys@defobject{currentmarker}{\pgfqpoint{-0.048611in}{0.000000in}}{\pgfqpoint{-0.000000in}{0.000000in}}{%
\pgfpathmoveto{\pgfqpoint{-0.000000in}{0.000000in}}%
\pgfpathlineto{\pgfqpoint{-0.048611in}{0.000000in}}%
\pgfusepath{stroke,fill}%
}%
\begin{pgfscope}%
\pgfsys@transformshift{0.696435in}{2.420922in}%
\pgfsys@useobject{currentmarker}{}%
\end{pgfscope}%
\end{pgfscope}%
\begin{pgfscope}%
\definecolor{textcolor}{rgb}{0.000000,0.000000,0.000000}%
\pgfsetstrokecolor{textcolor}%
\pgfsetfillcolor{textcolor}%
\pgftext[x=0.289968in, y=2.368160in, left, base]{\color{textcolor}\sffamily\fontsize{10.000000}{12.000000}\selectfont 0.42}%
\end{pgfscope}%
\begin{pgfscope}%
\pgfsetbuttcap%
\pgfsetroundjoin%
\definecolor{currentfill}{rgb}{0.000000,0.000000,0.000000}%
\pgfsetfillcolor{currentfill}%
\pgfsetlinewidth{0.803000pt}%
\definecolor{currentstroke}{rgb}{0.000000,0.000000,0.000000}%
\pgfsetstrokecolor{currentstroke}%
\pgfsetdash{}{0pt}%
\pgfsys@defobject{currentmarker}{\pgfqpoint{-0.048611in}{0.000000in}}{\pgfqpoint{-0.000000in}{0.000000in}}{%
\pgfpathmoveto{\pgfqpoint{-0.000000in}{0.000000in}}%
\pgfpathlineto{\pgfqpoint{-0.048611in}{0.000000in}}%
\pgfusepath{stroke,fill}%
}%
\begin{pgfscope}%
\pgfsys@transformshift{0.696435in}{2.912465in}%
\pgfsys@useobject{currentmarker}{}%
\end{pgfscope}%
\end{pgfscope}%
\begin{pgfscope}%
\definecolor{textcolor}{rgb}{0.000000,0.000000,0.000000}%
\pgfsetstrokecolor{textcolor}%
\pgfsetfillcolor{textcolor}%
\pgftext[x=0.289968in, y=2.859703in, left, base]{\color{textcolor}\sffamily\fontsize{10.000000}{12.000000}\selectfont 0.44}%
\end{pgfscope}%
\begin{pgfscope}%
\pgfsetbuttcap%
\pgfsetroundjoin%
\definecolor{currentfill}{rgb}{0.000000,0.000000,0.000000}%
\pgfsetfillcolor{currentfill}%
\pgfsetlinewidth{0.803000pt}%
\definecolor{currentstroke}{rgb}{0.000000,0.000000,0.000000}%
\pgfsetstrokecolor{currentstroke}%
\pgfsetdash{}{0pt}%
\pgfsys@defobject{currentmarker}{\pgfqpoint{-0.048611in}{0.000000in}}{\pgfqpoint{-0.000000in}{0.000000in}}{%
\pgfpathmoveto{\pgfqpoint{-0.000000in}{0.000000in}}%
\pgfpathlineto{\pgfqpoint{-0.048611in}{0.000000in}}%
\pgfusepath{stroke,fill}%
}%
\begin{pgfscope}%
\pgfsys@transformshift{0.696435in}{3.404008in}%
\pgfsys@useobject{currentmarker}{}%
\end{pgfscope}%
\end{pgfscope}%
\begin{pgfscope}%
\definecolor{textcolor}{rgb}{0.000000,0.000000,0.000000}%
\pgfsetstrokecolor{textcolor}%
\pgfsetfillcolor{textcolor}%
\pgftext[x=0.289968in, y=3.351247in, left, base]{\color{textcolor}\sffamily\fontsize{10.000000}{12.000000}\selectfont 0.46}%
\end{pgfscope}%
\begin{pgfscope}%
\pgfsetbuttcap%
\pgfsetroundjoin%
\definecolor{currentfill}{rgb}{0.000000,0.000000,0.000000}%
\pgfsetfillcolor{currentfill}%
\pgfsetlinewidth{0.803000pt}%
\definecolor{currentstroke}{rgb}{0.000000,0.000000,0.000000}%
\pgfsetstrokecolor{currentstroke}%
\pgfsetdash{}{0pt}%
\pgfsys@defobject{currentmarker}{\pgfqpoint{-0.048611in}{0.000000in}}{\pgfqpoint{-0.000000in}{0.000000in}}{%
\pgfpathmoveto{\pgfqpoint{-0.000000in}{0.000000in}}%
\pgfpathlineto{\pgfqpoint{-0.048611in}{0.000000in}}%
\pgfusepath{stroke,fill}%
}%
\begin{pgfscope}%
\pgfsys@transformshift{0.696435in}{3.895552in}%
\pgfsys@useobject{currentmarker}{}%
\end{pgfscope}%
\end{pgfscope}%
\begin{pgfscope}%
\definecolor{textcolor}{rgb}{0.000000,0.000000,0.000000}%
\pgfsetstrokecolor{textcolor}%
\pgfsetfillcolor{textcolor}%
\pgftext[x=0.289968in, y=3.842790in, left, base]{\color{textcolor}\sffamily\fontsize{10.000000}{12.000000}\selectfont 0.48}%
\end{pgfscope}%
\begin{pgfscope}%
\pgfsetbuttcap%
\pgfsetroundjoin%
\definecolor{currentfill}{rgb}{0.000000,0.000000,0.000000}%
\pgfsetfillcolor{currentfill}%
\pgfsetlinewidth{0.803000pt}%
\definecolor{currentstroke}{rgb}{0.000000,0.000000,0.000000}%
\pgfsetstrokecolor{currentstroke}%
\pgfsetdash{}{0pt}%
\pgfsys@defobject{currentmarker}{\pgfqpoint{-0.048611in}{0.000000in}}{\pgfqpoint{-0.000000in}{0.000000in}}{%
\pgfpathmoveto{\pgfqpoint{-0.000000in}{0.000000in}}%
\pgfpathlineto{\pgfqpoint{-0.048611in}{0.000000in}}%
\pgfusepath{stroke,fill}%
}%
\begin{pgfscope}%
\pgfsys@transformshift{0.696435in}{4.387095in}%
\pgfsys@useobject{currentmarker}{}%
\end{pgfscope}%
\end{pgfscope}%
\begin{pgfscope}%
\definecolor{textcolor}{rgb}{0.000000,0.000000,0.000000}%
\pgfsetstrokecolor{textcolor}%
\pgfsetfillcolor{textcolor}%
\pgftext[x=0.289968in, y=4.334333in, left, base]{\color{textcolor}\sffamily\fontsize{10.000000}{12.000000}\selectfont 0.50}%
\end{pgfscope}%
\begin{pgfscope}%
\definecolor{textcolor}{rgb}{0.000000,0.000000,0.000000}%
\pgfsetstrokecolor{textcolor}%
\pgfsetfillcolor{textcolor}%
\pgftext[x=0.234413in,y=2.543808in,,bottom,rotate=90.000000]{\color{textcolor}\sffamily\fontsize{10.000000}{12.000000}\bfseries\selectfont F1 Score}%
\end{pgfscope}%
\begin{pgfscope}%
\pgfsetrectcap%
\pgfsetmiterjoin%
\pgfsetlinewidth{0.803000pt}%
\definecolor{currentstroke}{rgb}{0.000000,0.000000,0.000000}%
\pgfsetstrokecolor{currentstroke}%
\pgfsetdash{}{0pt}%
\pgfpathmoveto{\pgfqpoint{0.696435in}{0.700520in}}%
\pgfpathlineto{\pgfqpoint{0.696435in}{4.387095in}}%
\pgfusepath{stroke}%
\end{pgfscope}%
\begin{pgfscope}%
\pgfsetrectcap%
\pgfsetmiterjoin%
\pgfsetlinewidth{0.803000pt}%
\definecolor{currentstroke}{rgb}{0.000000,0.000000,0.000000}%
\pgfsetstrokecolor{currentstroke}%
\pgfsetdash{}{0pt}%
\pgfpathmoveto{\pgfqpoint{6.193658in}{0.700520in}}%
\pgfpathlineto{\pgfqpoint{6.193658in}{4.387095in}}%
\pgfusepath{stroke}%
\end{pgfscope}%
\begin{pgfscope}%
\pgfsetrectcap%
\pgfsetmiterjoin%
\pgfsetlinewidth{0.803000pt}%
\definecolor{currentstroke}{rgb}{0.000000,0.000000,0.000000}%
\pgfsetstrokecolor{currentstroke}%
\pgfsetdash{}{0pt}%
\pgfpathmoveto{\pgfqpoint{0.696435in}{0.700520in}}%
\pgfpathlineto{\pgfqpoint{6.193658in}{0.700520in}}%
\pgfusepath{stroke}%
\end{pgfscope}%
\begin{pgfscope}%
\pgfsetrectcap%
\pgfsetmiterjoin%
\pgfsetlinewidth{0.803000pt}%
\definecolor{currentstroke}{rgb}{0.000000,0.000000,0.000000}%
\pgfsetstrokecolor{currentstroke}%
\pgfsetdash{}{0pt}%
\pgfpathmoveto{\pgfqpoint{0.696435in}{4.387095in}}%
\pgfpathlineto{\pgfqpoint{6.193658in}{4.387095in}}%
\pgfusepath{stroke}%
\end{pgfscope}%
\begin{pgfscope}%
\definecolor{textcolor}{rgb}{0.000000,0.000000,0.000000}%
\pgfsetstrokecolor{textcolor}%
\pgfsetfillcolor{textcolor}%
\pgftext[x=1.175785in,y=3.140918in,,bottom]{\color{textcolor}\sffamily\fontsize{9.000000}{10.800000}\selectfont 0.4476}%
\end{pgfscope}%
\begin{pgfscope}%
\definecolor{textcolor}{rgb}{0.000000,0.000000,0.000000}%
\pgfsetstrokecolor{textcolor}%
\pgfsetfillcolor{textcolor}%
\pgftext[x=2.195678in,y=3.022948in,,bottom]{\color{textcolor}\sffamily\fontsize{9.000000}{10.800000}\selectfont 0.4428}%
\end{pgfscope}%
\begin{pgfscope}%
\definecolor{textcolor}{rgb}{0.000000,0.000000,0.000000}%
\pgfsetstrokecolor{textcolor}%
\pgfsetfillcolor{textcolor}%
\pgftext[x=3.215571in,y=3.197446in,,bottom]{\color{textcolor}\sffamily\fontsize{9.000000}{10.800000}\selectfont 0.4499}%
\end{pgfscope}%
\begin{pgfscope}%
\definecolor{textcolor}{rgb}{0.000000,0.000000,0.000000}%
\pgfsetstrokecolor{textcolor}%
\pgfsetfillcolor{textcolor}%
\pgftext[x=4.235463in,y=2.932012in,,bottom]{\color{textcolor}\sffamily\fontsize{9.000000}{10.800000}\selectfont 0.4391}%
\end{pgfscope}%
\begin{pgfscope}%
\definecolor{textcolor}{rgb}{0.000000,0.000000,0.000000}%
\pgfsetstrokecolor{textcolor}%
\pgfsetfillcolor{textcolor}%
\pgftext[x=5.255356in,y=2.843534in,,bottom]{\color{textcolor}\sffamily\fontsize{9.000000}{10.800000}\selectfont 0.4355}%
\end{pgfscope}%
\begin{pgfscope}%
\definecolor{textcolor}{rgb}{0.000000,0.000000,0.000000}%
\pgfsetstrokecolor{textcolor}%
\pgfsetfillcolor{textcolor}%
\pgftext[x=1.634737in,y=2.118508in,,bottom]{\color{textcolor}\sffamily\fontsize{9.000000}{10.800000}\selectfont 0.406}%
\end{pgfscope}%
\begin{pgfscope}%
\definecolor{textcolor}{rgb}{0.000000,0.000000,0.000000}%
\pgfsetstrokecolor{textcolor}%
\pgfsetfillcolor{textcolor}%
\pgftext[x=2.654630in,y=3.231854in,,bottom]{\color{textcolor}\sffamily\fontsize{9.000000}{10.800000}\selectfont 0.4513}%
\end{pgfscope}%
\begin{pgfscope}%
\definecolor{textcolor}{rgb}{0.000000,0.000000,0.000000}%
\pgfsetstrokecolor{textcolor}%
\pgfsetfillcolor{textcolor}%
\pgftext[x=3.674522in,y=2.836161in,,bottom]{\color{textcolor}\sffamily\fontsize{9.000000}{10.800000}\selectfont 0.4352}%
\end{pgfscope}%
\begin{pgfscope}%
\definecolor{textcolor}{rgb}{0.000000,0.000000,0.000000}%
\pgfsetstrokecolor{textcolor}%
\pgfsetfillcolor{textcolor}%
\pgftext[x=4.694415in,y=2.887773in,,bottom]{\color{textcolor}\sffamily\fontsize{9.000000}{10.800000}\selectfont 0.4373}%
\end{pgfscope}%
\begin{pgfscope}%
\definecolor{textcolor}{rgb}{0.000000,0.000000,0.000000}%
\pgfsetstrokecolor{textcolor}%
\pgfsetfillcolor{textcolor}%
\pgftext[x=5.714308in,y=2.929555in,,bottom]{\color{textcolor}\sffamily\fontsize{9.000000}{10.800000}\selectfont 0.439}%
\end{pgfscope}%
\begin{pgfscope}%
\definecolor{textcolor}{rgb}{0.000000,0.000000,0.000000}%
\pgfsetstrokecolor{textcolor}%
\pgfsetfillcolor{textcolor}%
\pgftext[x=3.445047in,y=4.470428in,,base]{\color{textcolor}\sffamily\fontsize{12.000000}{14.400000}\selectfont Mixed training on Pix2Vox using 2 versions of S2R:3DFREE}%
\end{pgfscope}%
\begin{pgfscope}%
\pgfsetbuttcap%
\pgfsetmiterjoin%
\definecolor{currentfill}{rgb}{1.000000,1.000000,1.000000}%
\pgfsetfillcolor{currentfill}%
\pgfsetfillopacity{0.800000}%
\pgfsetlinewidth{1.003750pt}%
\definecolor{currentstroke}{rgb}{0.800000,0.800000,0.800000}%
\pgfsetstrokecolor{currentstroke}%
\pgfsetstrokeopacity{0.800000}%
\pgfsetdash{}{0pt}%
\pgfpathmoveto{\pgfqpoint{4.670342in}{3.868269in}}%
\pgfpathlineto{\pgfqpoint{6.096435in}{3.868269in}}%
\pgfpathquadraticcurveto{\pgfqpoint{6.124213in}{3.868269in}}{\pgfqpoint{6.124213in}{3.896047in}}%
\pgfpathlineto{\pgfqpoint{6.124213in}{4.289873in}}%
\pgfpathquadraticcurveto{\pgfqpoint{6.124213in}{4.317650in}}{\pgfqpoint{6.096435in}{4.317650in}}%
\pgfpathlineto{\pgfqpoint{4.670342in}{4.317650in}}%
\pgfpathquadraticcurveto{\pgfqpoint{4.642564in}{4.317650in}}{\pgfqpoint{4.642564in}{4.289873in}}%
\pgfpathlineto{\pgfqpoint{4.642564in}{3.896047in}}%
\pgfpathquadraticcurveto{\pgfqpoint{4.642564in}{3.868269in}}{\pgfqpoint{4.670342in}{3.868269in}}%
\pgfpathclose%
\pgfusepath{stroke,fill}%
\end{pgfscope}%
\begin{pgfscope}%
\pgfsetbuttcap%
\pgfsetmiterjoin%
\definecolor{currentfill}{rgb}{0.121569,0.466667,0.705882}%
\pgfsetfillcolor{currentfill}%
\pgfsetlinewidth{0.000000pt}%
\definecolor{currentstroke}{rgb}{0.000000,0.000000,0.000000}%
\pgfsetstrokecolor{currentstroke}%
\pgfsetstrokeopacity{0.000000}%
\pgfsetdash{}{0pt}%
\pgfpathmoveto{\pgfqpoint{4.698120in}{4.156572in}}%
\pgfpathlineto{\pgfqpoint{4.975898in}{4.156572in}}%
\pgfpathlineto{\pgfqpoint{4.975898in}{4.253794in}}%
\pgfpathlineto{\pgfqpoint{4.698120in}{4.253794in}}%
\pgfpathclose%
\pgfusepath{fill}%
\end{pgfscope}%
\begin{pgfscope}%
\definecolor{textcolor}{rgb}{0.000000,0.000000,0.000000}%
\pgfsetstrokecolor{textcolor}%
\pgfsetfillcolor{textcolor}%
\pgftext[x=5.087009in,y=4.156572in,left,base]{\color{textcolor}\sffamily\fontsize{10.000000}{12.000000}\selectfont V1 on Pix2Vox}%
\end{pgfscope}%
\begin{pgfscope}%
\pgfsetbuttcap%
\pgfsetmiterjoin%
\definecolor{currentfill}{rgb}{1.000000,0.498039,0.054902}%
\pgfsetfillcolor{currentfill}%
\pgfsetlinewidth{0.000000pt}%
\definecolor{currentstroke}{rgb}{0.000000,0.000000,0.000000}%
\pgfsetstrokecolor{currentstroke}%
\pgfsetstrokeopacity{0.000000}%
\pgfsetdash{}{0pt}%
\pgfpathmoveto{\pgfqpoint{4.698120in}{3.952715in}}%
\pgfpathlineto{\pgfqpoint{4.975898in}{3.952715in}}%
\pgfpathlineto{\pgfqpoint{4.975898in}{4.049937in}}%
\pgfpathlineto{\pgfqpoint{4.698120in}{4.049937in}}%
\pgfpathclose%
\pgfusepath{fill}%
\end{pgfscope}%
\begin{pgfscope}%
\definecolor{textcolor}{rgb}{0.000000,0.000000,0.000000}%
\pgfsetstrokecolor{textcolor}%
\pgfsetfillcolor{textcolor}%
\pgftext[x=5.087009in,y=3.952715in,left,base]{\color{textcolor}\sffamily\fontsize{10.000000}{12.000000}\selectfont V2 on Pix2Vox}%
\end{pgfscope}%
\end{pgfpicture}%
\makeatother%
\endgroup%
}
    \caption{Bar plot for the \gls{f1}  for baselines trained on different ratios of synthetic and real dataset.
        (left)Mixed training on Pix2Vox++, (right)Mixed training on Pix2Vox. In both cases we see a slight increase in \gls{f1} with addition of real data, and a gradual decrease till it reaches 100\% real data}
    \label{fig:mixed_dice1}
\end{figure}

\begin{figure}
    \centering
    \resizebox{0.7\textwidth}{!}{\input{/Users/apple/OVGU/Thesis/code/3dReconstruction/report/images/evaluation/performance/mixed_dice_barplot_pix2voxpp.pgf}}
    \caption{Bar plot for the \gls{f1} for baseline \texbf{pix2vox++} trained on 50\% of mixed dataset(\gls{s2rv1}, \gls{s2rv2}) and with pix3d.
    The categories are listed along with the number of images.
    The performance of pix2vox++ mixed with both the synthetic dataset is more than model trained on only pix3d, for most of the categories.
    }
    \label{fig:mixed_dice2}
\end{figure}

\begin{figure}
    \centering
    \resizebox{0.7\textwidth}{!}{\input{/Users/apple/OVGU/Thesis/code/3dReconstruction/report/images/evaluation/performance/mixed_dice_barplot_pix2vox.pgf}}
    \caption{Bar plot for the \gls{f1} for baseline \texbf{pix2vox} trained on 50\% of mixed dataset(\gls{s2rv1}, \gls{s2rv2}) and with pix3d.
    The categories are listed along with the number of images.
    The performance of pix2vox mixed with both the synthetic dataset is more than model trained on only pix3d, for all the categories.}
    \label{fig:mixed_dice3}
\end{figure}

\subsection{Ablation study on chairs}\label{subsec:ablation-study-on-chairs}

For the ablation study on chairs, we created different datasets with different domain randomization components as in \autoref{subsec:s2r:3dfree-ablation}.
\autoref{fig:ablation_dice1} shows the performance of baseline models with each of the randomization components.
Pix2Vox++ seems to give a \gls{f1} closer to the real dataset when trained on \gls{s2rv2} than Pix2vox.
Overall the performance still does not match the real dataset on its own.

\begin{figure}[ht]
    \centering
    \resizebox{0.7\textwidth}{!}{%% Creator: Matplotlib, PGF backend
%%
%% To include the figure in your LaTeX document, write
%%   \input{<filename>.pgf}
%%
%% Make sure the required packages are loaded in your preamble
%%   \usepackage{pgf}
%%
%% Figures using additional raster images can only be included by \input if
%% they are in the same directory as the main LaTeX file. For loading figures
%% from other directories you can use the `import` package
%%   \usepackage{import}
%%
%% and then include the figures with
%%   \import{<path to file>}{<filename>.pgf}
%%
%% Matplotlib used the following preamble
%%   \usepackage{fontspec}
%%   \setmainfont{DejaVuSerif.ttf}[Path=\detokenize{/Users/apple/opt/anaconda3/envs/kaolin/lib/python3.7/site-packages/matplotlib/mpl-data/fonts/ttf/}]
%%   \setsansfont{DejaVuSans.ttf}[Path=\detokenize{/Users/apple/opt/anaconda3/envs/kaolin/lib/python3.7/site-packages/matplotlib/mpl-data/fonts/ttf/}]
%%   \setmonofont{DejaVuSansMono.ttf}[Path=\detokenize{/Users/apple/opt/anaconda3/envs/kaolin/lib/python3.7/site-packages/matplotlib/mpl-data/fonts/ttf/}]
%%
\begingroup%
\makeatletter%
\begin{pgfpicture}%
\pgfpathrectangle{\pgfpointorigin}{\pgfqpoint{6.294042in}{4.693849in}}%
\pgfusepath{use as bounding box, clip}%
\begin{pgfscope}%
\pgfsetbuttcap%
\pgfsetmiterjoin%
\definecolor{currentfill}{rgb}{1.000000,1.000000,1.000000}%
\pgfsetfillcolor{currentfill}%
\pgfsetlinewidth{0.000000pt}%
\definecolor{currentstroke}{rgb}{1.000000,1.000000,1.000000}%
\pgfsetstrokecolor{currentstroke}%
\pgfsetdash{}{0pt}%
\pgfpathmoveto{\pgfqpoint{0.000000in}{0.000000in}}%
\pgfpathlineto{\pgfqpoint{6.294042in}{0.000000in}}%
\pgfpathlineto{\pgfqpoint{6.294042in}{4.693849in}}%
\pgfpathlineto{\pgfqpoint{0.000000in}{4.693849in}}%
\pgfpathclose%
\pgfusepath{fill}%
\end{pgfscope}%
\begin{pgfscope}%
\pgfsetbuttcap%
\pgfsetmiterjoin%
\definecolor{currentfill}{rgb}{1.000000,1.000000,1.000000}%
\pgfsetfillcolor{currentfill}%
\pgfsetlinewidth{0.000000pt}%
\definecolor{currentstroke}{rgb}{0.000000,0.000000,0.000000}%
\pgfsetstrokecolor{currentstroke}%
\pgfsetstrokeopacity{0.000000}%
\pgfsetdash{}{0pt}%
\pgfpathmoveto{\pgfqpoint{0.608070in}{1.439775in}}%
\pgfpathlineto{\pgfqpoint{6.194042in}{1.439775in}}%
\pgfpathlineto{\pgfqpoint{6.194042in}{4.383888in}}%
\pgfpathlineto{\pgfqpoint{0.608070in}{4.383888in}}%
\pgfpathclose%
\pgfusepath{fill}%
\end{pgfscope}%
\begin{pgfscope}%
\pgfpathrectangle{\pgfqpoint{0.608070in}{1.439775in}}{\pgfqpoint{5.585972in}{2.944113in}}%
\pgfusepath{clip}%
\pgfsetbuttcap%
\pgfsetmiterjoin%
\definecolor{currentfill}{rgb}{0.121569,0.466667,0.705882}%
\pgfsetfillcolor{currentfill}%
\pgfsetlinewidth{0.000000pt}%
\definecolor{currentstroke}{rgb}{0.000000,0.000000,0.000000}%
\pgfsetstrokecolor{currentstroke}%
\pgfsetstrokeopacity{0.000000}%
\pgfsetdash{}{0pt}%
\pgfpathmoveto{\pgfqpoint{0.861978in}{1.439775in}}%
\pgfpathlineto{\pgfqpoint{1.193162in}{1.439775in}}%
\pgfpathlineto{\pgfqpoint{1.193162in}{3.822740in}}%
\pgfpathlineto{\pgfqpoint{0.861978in}{3.822740in}}%
\pgfpathclose%
\pgfusepath{fill}%
\end{pgfscope}%
\begin{pgfscope}%
\pgfpathrectangle{\pgfqpoint{0.608070in}{1.439775in}}{\pgfqpoint{5.585972in}{2.944113in}}%
\pgfusepath{clip}%
\pgfsetbuttcap%
\pgfsetmiterjoin%
\definecolor{currentfill}{rgb}{0.121569,0.466667,0.705882}%
\pgfsetfillcolor{currentfill}%
\pgfsetlinewidth{0.000000pt}%
\definecolor{currentstroke}{rgb}{0.000000,0.000000,0.000000}%
\pgfsetstrokecolor{currentstroke}%
\pgfsetstrokeopacity{0.000000}%
\pgfsetdash{}{0pt}%
\pgfpathmoveto{\pgfqpoint{1.597943in}{1.439775in}}%
\pgfpathlineto{\pgfqpoint{1.929127in}{1.439775in}}%
\pgfpathlineto{\pgfqpoint{1.929127in}{4.117151in}}%
\pgfpathlineto{\pgfqpoint{1.597943in}{4.117151in}}%
\pgfpathclose%
\pgfusepath{fill}%
\end{pgfscope}%
\begin{pgfscope}%
\pgfpathrectangle{\pgfqpoint{0.608070in}{1.439775in}}{\pgfqpoint{5.585972in}{2.944113in}}%
\pgfusepath{clip}%
\pgfsetbuttcap%
\pgfsetmiterjoin%
\definecolor{currentfill}{rgb}{0.121569,0.466667,0.705882}%
\pgfsetfillcolor{currentfill}%
\pgfsetlinewidth{0.000000pt}%
\definecolor{currentstroke}{rgb}{0.000000,0.000000,0.000000}%
\pgfsetstrokecolor{currentstroke}%
\pgfsetstrokeopacity{0.000000}%
\pgfsetdash{}{0pt}%
\pgfpathmoveto{\pgfqpoint{2.333907in}{1.439775in}}%
\pgfpathlineto{\pgfqpoint{2.665091in}{1.439775in}}%
\pgfpathlineto{\pgfqpoint{2.665091in}{3.166792in}}%
\pgfpathlineto{\pgfqpoint{2.333907in}{3.166792in}}%
\pgfpathclose%
\pgfusepath{fill}%
\end{pgfscope}%
\begin{pgfscope}%
\pgfpathrectangle{\pgfqpoint{0.608070in}{1.439775in}}{\pgfqpoint{5.585972in}{2.944113in}}%
\pgfusepath{clip}%
\pgfsetbuttcap%
\pgfsetmiterjoin%
\definecolor{currentfill}{rgb}{0.121569,0.466667,0.705882}%
\pgfsetfillcolor{currentfill}%
\pgfsetlinewidth{0.000000pt}%
\definecolor{currentstroke}{rgb}{0.000000,0.000000,0.000000}%
\pgfsetstrokecolor{currentstroke}%
\pgfsetstrokeopacity{0.000000}%
\pgfsetdash{}{0pt}%
\pgfpathmoveto{\pgfqpoint{3.069872in}{1.439775in}}%
\pgfpathlineto{\pgfqpoint{3.401056in}{1.439775in}}%
\pgfpathlineto{\pgfqpoint{3.401056in}{2.864726in}}%
\pgfpathlineto{\pgfqpoint{3.069872in}{2.864726in}}%
\pgfpathclose%
\pgfusepath{fill}%
\end{pgfscope}%
\begin{pgfscope}%
\pgfpathrectangle{\pgfqpoint{0.608070in}{1.439775in}}{\pgfqpoint{5.585972in}{2.944113in}}%
\pgfusepath{clip}%
\pgfsetbuttcap%
\pgfsetmiterjoin%
\definecolor{currentfill}{rgb}{0.121569,0.466667,0.705882}%
\pgfsetfillcolor{currentfill}%
\pgfsetlinewidth{0.000000pt}%
\definecolor{currentstroke}{rgb}{0.000000,0.000000,0.000000}%
\pgfsetstrokecolor{currentstroke}%
\pgfsetstrokeopacity{0.000000}%
\pgfsetdash{}{0pt}%
\pgfpathmoveto{\pgfqpoint{3.805837in}{1.439775in}}%
\pgfpathlineto{\pgfqpoint{4.137021in}{1.439775in}}%
\pgfpathlineto{\pgfqpoint{4.137021in}{3.508309in}}%
\pgfpathlineto{\pgfqpoint{3.805837in}{3.508309in}}%
\pgfpathclose%
\pgfusepath{fill}%
\end{pgfscope}%
\begin{pgfscope}%
\pgfpathrectangle{\pgfqpoint{0.608070in}{1.439775in}}{\pgfqpoint{5.585972in}{2.944113in}}%
\pgfusepath{clip}%
\pgfsetbuttcap%
\pgfsetmiterjoin%
\definecolor{currentfill}{rgb}{0.121569,0.466667,0.705882}%
\pgfsetfillcolor{currentfill}%
\pgfsetlinewidth{0.000000pt}%
\definecolor{currentstroke}{rgb}{0.000000,0.000000,0.000000}%
\pgfsetstrokecolor{currentstroke}%
\pgfsetstrokeopacity{0.000000}%
\pgfsetdash{}{0pt}%
\pgfpathmoveto{\pgfqpoint{4.541801in}{1.439775in}}%
\pgfpathlineto{\pgfqpoint{4.872986in}{1.439775in}}%
\pgfpathlineto{\pgfqpoint{4.872986in}{3.511842in}}%
\pgfpathlineto{\pgfqpoint{4.541801in}{3.511842in}}%
\pgfpathclose%
\pgfusepath{fill}%
\end{pgfscope}%
\begin{pgfscope}%
\pgfpathrectangle{\pgfqpoint{0.608070in}{1.439775in}}{\pgfqpoint{5.585972in}{2.944113in}}%
\pgfusepath{clip}%
\pgfsetbuttcap%
\pgfsetmiterjoin%
\definecolor{currentfill}{rgb}{0.121569,0.466667,0.705882}%
\pgfsetfillcolor{currentfill}%
\pgfsetlinewidth{0.000000pt}%
\definecolor{currentstroke}{rgb}{0.000000,0.000000,0.000000}%
\pgfsetstrokecolor{currentstroke}%
\pgfsetstrokeopacity{0.000000}%
\pgfsetdash{}{0pt}%
\pgfpathmoveto{\pgfqpoint{5.277766in}{1.439775in}}%
\pgfpathlineto{\pgfqpoint{5.608950in}{1.439775in}}%
\pgfpathlineto{\pgfqpoint{5.608950in}{3.621363in}}%
\pgfpathlineto{\pgfqpoint{5.277766in}{3.621363in}}%
\pgfpathclose%
\pgfusepath{fill}%
\end{pgfscope}%
\begin{pgfscope}%
\pgfpathrectangle{\pgfqpoint{0.608070in}{1.439775in}}{\pgfqpoint{5.585972in}{2.944113in}}%
\pgfusepath{clip}%
\pgfsetbuttcap%
\pgfsetmiterjoin%
\definecolor{currentfill}{rgb}{1.000000,0.498039,0.054902}%
\pgfsetfillcolor{currentfill}%
\pgfsetlinewidth{0.000000pt}%
\definecolor{currentstroke}{rgb}{0.000000,0.000000,0.000000}%
\pgfsetstrokecolor{currentstroke}%
\pgfsetstrokeopacity{0.000000}%
\pgfsetdash{}{0pt}%
\pgfpathmoveto{\pgfqpoint{1.193162in}{1.439775in}}%
\pgfpathlineto{\pgfqpoint{1.524346in}{1.439775in}}%
\pgfpathlineto{\pgfqpoint{1.524346in}{3.762091in}}%
\pgfpathlineto{\pgfqpoint{1.193162in}{3.762091in}}%
\pgfpathclose%
\pgfusepath{fill}%
\end{pgfscope}%
\begin{pgfscope}%
\pgfpathrectangle{\pgfqpoint{0.608070in}{1.439775in}}{\pgfqpoint{5.585972in}{2.944113in}}%
\pgfusepath{clip}%
\pgfsetbuttcap%
\pgfsetmiterjoin%
\definecolor{currentfill}{rgb}{1.000000,0.498039,0.054902}%
\pgfsetfillcolor{currentfill}%
\pgfsetlinewidth{0.000000pt}%
\definecolor{currentstroke}{rgb}{0.000000,0.000000,0.000000}%
\pgfsetstrokecolor{currentstroke}%
\pgfsetstrokeopacity{0.000000}%
\pgfsetdash{}{0pt}%
\pgfpathmoveto{\pgfqpoint{1.929127in}{1.439775in}}%
\pgfpathlineto{\pgfqpoint{2.260311in}{1.439775in}}%
\pgfpathlineto{\pgfqpoint{2.260311in}{3.898109in}}%
\pgfpathlineto{\pgfqpoint{1.929127in}{3.898109in}}%
\pgfpathclose%
\pgfusepath{fill}%
\end{pgfscope}%
\begin{pgfscope}%
\pgfpathrectangle{\pgfqpoint{0.608070in}{1.439775in}}{\pgfqpoint{5.585972in}{2.944113in}}%
\pgfusepath{clip}%
\pgfsetbuttcap%
\pgfsetmiterjoin%
\definecolor{currentfill}{rgb}{1.000000,0.498039,0.054902}%
\pgfsetfillcolor{currentfill}%
\pgfsetlinewidth{0.000000pt}%
\definecolor{currentstroke}{rgb}{0.000000,0.000000,0.000000}%
\pgfsetstrokecolor{currentstroke}%
\pgfsetstrokeopacity{0.000000}%
\pgfsetdash{}{0pt}%
\pgfpathmoveto{\pgfqpoint{2.665091in}{1.439775in}}%
\pgfpathlineto{\pgfqpoint{2.996276in}{1.439775in}}%
\pgfpathlineto{\pgfqpoint{2.996276in}{2.243518in}}%
\pgfpathlineto{\pgfqpoint{2.665091in}{2.243518in}}%
\pgfpathclose%
\pgfusepath{fill}%
\end{pgfscope}%
\begin{pgfscope}%
\pgfpathrectangle{\pgfqpoint{0.608070in}{1.439775in}}{\pgfqpoint{5.585972in}{2.944113in}}%
\pgfusepath{clip}%
\pgfsetbuttcap%
\pgfsetmiterjoin%
\definecolor{currentfill}{rgb}{1.000000,0.498039,0.054902}%
\pgfsetfillcolor{currentfill}%
\pgfsetlinewidth{0.000000pt}%
\definecolor{currentstroke}{rgb}{0.000000,0.000000,0.000000}%
\pgfsetstrokecolor{currentstroke}%
\pgfsetstrokeopacity{0.000000}%
\pgfsetdash{}{0pt}%
\pgfpathmoveto{\pgfqpoint{3.401056in}{1.439775in}}%
\pgfpathlineto{\pgfqpoint{3.732240in}{1.439775in}}%
\pgfpathlineto{\pgfqpoint{3.732240in}{2.496712in}}%
\pgfpathlineto{\pgfqpoint{3.401056in}{2.496712in}}%
\pgfpathclose%
\pgfusepath{fill}%
\end{pgfscope}%
\begin{pgfscope}%
\pgfpathrectangle{\pgfqpoint{0.608070in}{1.439775in}}{\pgfqpoint{5.585972in}{2.944113in}}%
\pgfusepath{clip}%
\pgfsetbuttcap%
\pgfsetmiterjoin%
\definecolor{currentfill}{rgb}{1.000000,0.498039,0.054902}%
\pgfsetfillcolor{currentfill}%
\pgfsetlinewidth{0.000000pt}%
\definecolor{currentstroke}{rgb}{0.000000,0.000000,0.000000}%
\pgfsetstrokecolor{currentstroke}%
\pgfsetstrokeopacity{0.000000}%
\pgfsetdash{}{0pt}%
\pgfpathmoveto{\pgfqpoint{4.137021in}{1.439775in}}%
\pgfpathlineto{\pgfqpoint{4.468205in}{1.439775in}}%
\pgfpathlineto{\pgfqpoint{4.468205in}{2.593278in}}%
\pgfpathlineto{\pgfqpoint{4.137021in}{2.593278in}}%
\pgfpathclose%
\pgfusepath{fill}%
\end{pgfscope}%
\begin{pgfscope}%
\pgfpathrectangle{\pgfqpoint{0.608070in}{1.439775in}}{\pgfqpoint{5.585972in}{2.944113in}}%
\pgfusepath{clip}%
\pgfsetbuttcap%
\pgfsetmiterjoin%
\definecolor{currentfill}{rgb}{1.000000,0.498039,0.054902}%
\pgfsetfillcolor{currentfill}%
\pgfsetlinewidth{0.000000pt}%
\definecolor{currentstroke}{rgb}{0.000000,0.000000,0.000000}%
\pgfsetstrokecolor{currentstroke}%
\pgfsetstrokeopacity{0.000000}%
\pgfsetdash{}{0pt}%
\pgfpathmoveto{\pgfqpoint{4.872986in}{1.439775in}}%
\pgfpathlineto{\pgfqpoint{5.204170in}{1.439775in}}%
\pgfpathlineto{\pgfqpoint{5.204170in}{2.481402in}}%
\pgfpathlineto{\pgfqpoint{4.872986in}{2.481402in}}%
\pgfpathclose%
\pgfusepath{fill}%
\end{pgfscope}%
\begin{pgfscope}%
\pgfpathrectangle{\pgfqpoint{0.608070in}{1.439775in}}{\pgfqpoint{5.585972in}{2.944113in}}%
\pgfusepath{clip}%
\pgfsetbuttcap%
\pgfsetmiterjoin%
\definecolor{currentfill}{rgb}{1.000000,0.498039,0.054902}%
\pgfsetfillcolor{currentfill}%
\pgfsetlinewidth{0.000000pt}%
\definecolor{currentstroke}{rgb}{0.000000,0.000000,0.000000}%
\pgfsetstrokecolor{currentstroke}%
\pgfsetstrokeopacity{0.000000}%
\pgfsetdash{}{0pt}%
\pgfpathmoveto{\pgfqpoint{5.608950in}{1.439775in}}%
\pgfpathlineto{\pgfqpoint{5.940134in}{1.439775in}}%
\pgfpathlineto{\pgfqpoint{5.940134in}{2.428997in}}%
\pgfpathlineto{\pgfqpoint{5.608950in}{2.428997in}}%
\pgfpathclose%
\pgfusepath{fill}%
\end{pgfscope}%
\begin{pgfscope}%
\pgfsetbuttcap%
\pgfsetroundjoin%
\definecolor{currentfill}{rgb}{0.000000,0.000000,0.000000}%
\pgfsetfillcolor{currentfill}%
\pgfsetlinewidth{0.803000pt}%
\definecolor{currentstroke}{rgb}{0.000000,0.000000,0.000000}%
\pgfsetstrokecolor{currentstroke}%
\pgfsetdash{}{0pt}%
\pgfsys@defobject{currentmarker}{\pgfqpoint{0.000000in}{-0.048611in}}{\pgfqpoint{0.000000in}{0.000000in}}{%
\pgfpathmoveto{\pgfqpoint{0.000000in}{0.000000in}}%
\pgfpathlineto{\pgfqpoint{0.000000in}{-0.048611in}}%
\pgfusepath{stroke,fill}%
}%
\begin{pgfscope}%
\pgfsys@transformshift{1.193162in}{1.439775in}%
\pgfsys@useobject{currentmarker}{}%
\end{pgfscope}%
\end{pgfscope}%
\begin{pgfscope}%
\definecolor{textcolor}{rgb}{0.000000,0.000000,0.000000}%
\pgfsetstrokecolor{textcolor}%
\pgfsetfillcolor{textcolor}%
\pgftext[x=0.741486in, y=0.310396in, left, base,rotate=45.000000]{\color{textcolor}\sffamily\fontsize{10.000000}{12.000000}\selectfont Pix3d(chair,no aug)}%
\end{pgfscope}%
\begin{pgfscope}%
\pgfsetbuttcap%
\pgfsetroundjoin%
\definecolor{currentfill}{rgb}{0.000000,0.000000,0.000000}%
\pgfsetfillcolor{currentfill}%
\pgfsetlinewidth{0.803000pt}%
\definecolor{currentstroke}{rgb}{0.000000,0.000000,0.000000}%
\pgfsetstrokecolor{currentstroke}%
\pgfsetdash{}{0pt}%
\pgfsys@defobject{currentmarker}{\pgfqpoint{0.000000in}{-0.048611in}}{\pgfqpoint{0.000000in}{0.000000in}}{%
\pgfpathmoveto{\pgfqpoint{0.000000in}{0.000000in}}%
\pgfpathlineto{\pgfqpoint{0.000000in}{-0.048611in}}%
\pgfusepath{stroke,fill}%
}%
\begin{pgfscope}%
\pgfsys@transformshift{1.929127in}{1.439775in}%
\pgfsys@useobject{currentmarker}{}%
\end{pgfscope}%
\end{pgfscope}%
\begin{pgfscope}%
\definecolor{textcolor}{rgb}{0.000000,0.000000,0.000000}%
\pgfsetstrokecolor{textcolor}%
\pgfsetfillcolor{textcolor}%
\pgftext[x=1.662216in, y=0.679928in, left, base,rotate=45.000000]{\color{textcolor}\sffamily\fontsize{10.000000}{12.000000}\selectfont Pix3d(chair)}%
\end{pgfscope}%
\begin{pgfscope}%
\pgfsetbuttcap%
\pgfsetroundjoin%
\definecolor{currentfill}{rgb}{0.000000,0.000000,0.000000}%
\pgfsetfillcolor{currentfill}%
\pgfsetlinewidth{0.803000pt}%
\definecolor{currentstroke}{rgb}{0.000000,0.000000,0.000000}%
\pgfsetstrokecolor{currentstroke}%
\pgfsetdash{}{0pt}%
\pgfsys@defobject{currentmarker}{\pgfqpoint{0.000000in}{-0.048611in}}{\pgfqpoint{0.000000in}{0.000000in}}{%
\pgfpathmoveto{\pgfqpoint{0.000000in}{0.000000in}}%
\pgfpathlineto{\pgfqpoint{0.000000in}{-0.048611in}}%
\pgfusepath{stroke,fill}%
}%
\begin{pgfscope}%
\pgfsys@transformshift{2.665091in}{1.439775in}%
\pgfsys@useobject{currentmarker}{}%
\end{pgfscope}%
\end{pgfscope}%
\begin{pgfscope}%
\definecolor{textcolor}{rgb}{0.000000,0.000000,0.000000}%
\pgfsetstrokecolor{textcolor}%
\pgfsetfillcolor{textcolor}%
\pgftext[x=2.417410in, y=0.718387in, left, base,rotate=45.000000]{\color{textcolor}\sffamily\fontsize{10.000000}{12.000000}\selectfont Textureless}%
\end{pgfscope}%
\begin{pgfscope}%
\pgfsetbuttcap%
\pgfsetroundjoin%
\definecolor{currentfill}{rgb}{0.000000,0.000000,0.000000}%
\pgfsetfillcolor{currentfill}%
\pgfsetlinewidth{0.803000pt}%
\definecolor{currentstroke}{rgb}{0.000000,0.000000,0.000000}%
\pgfsetstrokecolor{currentstroke}%
\pgfsetdash{}{0pt}%
\pgfsys@defobject{currentmarker}{\pgfqpoint{0.000000in}{-0.048611in}}{\pgfqpoint{0.000000in}{0.000000in}}{%
\pgfpathmoveto{\pgfqpoint{0.000000in}{0.000000in}}%
\pgfpathlineto{\pgfqpoint{0.000000in}{-0.048611in}}%
\pgfusepath{stroke,fill}%
}%
\begin{pgfscope}%
\pgfsys@transformshift{3.401056in}{1.439775in}%
\pgfsys@useobject{currentmarker}{}%
\end{pgfscope}%
\end{pgfscope}%
\begin{pgfscope}%
\definecolor{textcolor}{rgb}{0.000000,0.000000,0.000000}%
\pgfsetstrokecolor{textcolor}%
\pgfsetfillcolor{textcolor}%
\pgftext[x=2.989685in, y=0.391007in, left, base,rotate=45.000000]{\color{textcolor}\sffamily\fontsize{10.000000}{12.000000}\selectfont Textureless+Light}%
\end{pgfscope}%
\begin{pgfscope}%
\pgfsetbuttcap%
\pgfsetroundjoin%
\definecolor{currentfill}{rgb}{0.000000,0.000000,0.000000}%
\pgfsetfillcolor{currentfill}%
\pgfsetlinewidth{0.803000pt}%
\definecolor{currentstroke}{rgb}{0.000000,0.000000,0.000000}%
\pgfsetstrokecolor{currentstroke}%
\pgfsetdash{}{0pt}%
\pgfsys@defobject{currentmarker}{\pgfqpoint{0.000000in}{-0.048611in}}{\pgfqpoint{0.000000in}{0.000000in}}{%
\pgfpathmoveto{\pgfqpoint{0.000000in}{0.000000in}}%
\pgfpathlineto{\pgfqpoint{0.000000in}{-0.048611in}}%
\pgfusepath{stroke,fill}%
}%
\begin{pgfscope}%
\pgfsys@transformshift{4.137021in}{1.439775in}%
\pgfsys@useobject{currentmarker}{}%
\end{pgfscope}%
\end{pgfscope}%
\begin{pgfscope}%
\definecolor{textcolor}{rgb}{0.000000,0.000000,0.000000}%
\pgfsetstrokecolor{textcolor}%
\pgfsetfillcolor{textcolor}%
\pgftext[x=3.953190in, y=0.846088in, left, base,rotate=45.000000]{\color{textcolor}\sffamily\fontsize{10.000000}{12.000000}\selectfont Textured}%
\end{pgfscope}%
\begin{pgfscope}%
\pgfsetbuttcap%
\pgfsetroundjoin%
\definecolor{currentfill}{rgb}{0.000000,0.000000,0.000000}%
\pgfsetfillcolor{currentfill}%
\pgfsetlinewidth{0.803000pt}%
\definecolor{currentstroke}{rgb}{0.000000,0.000000,0.000000}%
\pgfsetstrokecolor{currentstroke}%
\pgfsetdash{}{0pt}%
\pgfsys@defobject{currentmarker}{\pgfqpoint{0.000000in}{-0.048611in}}{\pgfqpoint{0.000000in}{0.000000in}}{%
\pgfpathmoveto{\pgfqpoint{0.000000in}{0.000000in}}%
\pgfpathlineto{\pgfqpoint{0.000000in}{-0.048611in}}%
\pgfusepath{stroke,fill}%
}%
\begin{pgfscope}%
\pgfsys@transformshift{4.872986in}{1.439775in}%
\pgfsys@useobject{currentmarker}{}%
\end{pgfscope}%
\end{pgfscope}%
\begin{pgfscope}%
\definecolor{textcolor}{rgb}{0.000000,0.000000,0.000000}%
\pgfsetstrokecolor{textcolor}%
\pgfsetfillcolor{textcolor}%
\pgftext[x=4.525465in, y=0.518708in, left, base,rotate=45.000000]{\color{textcolor}\sffamily\fontsize{10.000000}{12.000000}\selectfont Textured+Light}%
\end{pgfscope}%
\begin{pgfscope}%
\pgfsetbuttcap%
\pgfsetroundjoin%
\definecolor{currentfill}{rgb}{0.000000,0.000000,0.000000}%
\pgfsetfillcolor{currentfill}%
\pgfsetlinewidth{0.803000pt}%
\definecolor{currentstroke}{rgb}{0.000000,0.000000,0.000000}%
\pgfsetstrokecolor{currentstroke}%
\pgfsetdash{}{0pt}%
\pgfsys@defobject{currentmarker}{\pgfqpoint{0.000000in}{-0.048611in}}{\pgfqpoint{0.000000in}{0.000000in}}{%
\pgfpathmoveto{\pgfqpoint{0.000000in}{0.000000in}}%
\pgfpathlineto{\pgfqpoint{0.000000in}{-0.048611in}}%
\pgfusepath{stroke,fill}%
}%
\begin{pgfscope}%
\pgfsys@transformshift{5.608950in}{1.439775in}%
\pgfsys@useobject{currentmarker}{}%
\end{pgfscope}%
\end{pgfscope}%
\begin{pgfscope}%
\definecolor{textcolor}{rgb}{0.000000,0.000000,0.000000}%
\pgfsetstrokecolor{textcolor}%
\pgfsetfillcolor{textcolor}%
\pgftext[x=5.337005in, y=0.669858in, left, base,rotate=45.000000]{\color{textcolor}\sffamily\fontsize{10.000000}{12.000000}\selectfont Multi-Object}%
\end{pgfscope}%
\begin{pgfscope}%
\definecolor{textcolor}{rgb}{0.000000,0.000000,0.000000}%
\pgfsetstrokecolor{textcolor}%
\pgfsetfillcolor{textcolor}%
\pgftext[x=3.401056in,y=0.234413in,,top]{\color{textcolor}\sffamily\fontsize{10.000000}{12.000000}\selectfont Dataset}%
\end{pgfscope}%
\begin{pgfscope}%
\pgfsetbuttcap%
\pgfsetroundjoin%
\definecolor{currentfill}{rgb}{0.000000,0.000000,0.000000}%
\pgfsetfillcolor{currentfill}%
\pgfsetlinewidth{0.803000pt}%
\definecolor{currentstroke}{rgb}{0.000000,0.000000,0.000000}%
\pgfsetstrokecolor{currentstroke}%
\pgfsetdash{}{0pt}%
\pgfsys@defobject{currentmarker}{\pgfqpoint{-0.048611in}{0.000000in}}{\pgfqpoint{-0.000000in}{0.000000in}}{%
\pgfpathmoveto{\pgfqpoint{-0.000000in}{0.000000in}}%
\pgfpathlineto{\pgfqpoint{-0.048611in}{0.000000in}}%
\pgfusepath{stroke,fill}%
}%
\begin{pgfscope}%
\pgfsys@transformshift{0.608070in}{1.439775in}%
\pgfsys@useobject{currentmarker}{}%
\end{pgfscope}%
\end{pgfscope}%
\begin{pgfscope}%
\definecolor{textcolor}{rgb}{0.000000,0.000000,0.000000}%
\pgfsetstrokecolor{textcolor}%
\pgfsetfillcolor{textcolor}%
\pgftext[x=0.289968in, y=1.387014in, left, base]{\color{textcolor}\sffamily\fontsize{10.000000}{12.000000}\selectfont 0.0}%
\end{pgfscope}%
\begin{pgfscope}%
\pgfsetbuttcap%
\pgfsetroundjoin%
\definecolor{currentfill}{rgb}{0.000000,0.000000,0.000000}%
\pgfsetfillcolor{currentfill}%
\pgfsetlinewidth{0.803000pt}%
\definecolor{currentstroke}{rgb}{0.000000,0.000000,0.000000}%
\pgfsetstrokecolor{currentstroke}%
\pgfsetdash{}{0pt}%
\pgfsys@defobject{currentmarker}{\pgfqpoint{-0.048611in}{0.000000in}}{\pgfqpoint{-0.000000in}{0.000000in}}{%
\pgfpathmoveto{\pgfqpoint{-0.000000in}{0.000000in}}%
\pgfpathlineto{\pgfqpoint{-0.048611in}{0.000000in}}%
\pgfusepath{stroke,fill}%
}%
\begin{pgfscope}%
\pgfsys@transformshift{0.608070in}{2.028598in}%
\pgfsys@useobject{currentmarker}{}%
\end{pgfscope}%
\end{pgfscope}%
\begin{pgfscope}%
\definecolor{textcolor}{rgb}{0.000000,0.000000,0.000000}%
\pgfsetstrokecolor{textcolor}%
\pgfsetfillcolor{textcolor}%
\pgftext[x=0.289968in, y=1.975836in, left, base]{\color{textcolor}\sffamily\fontsize{10.000000}{12.000000}\selectfont 0.1}%
\end{pgfscope}%
\begin{pgfscope}%
\pgfsetbuttcap%
\pgfsetroundjoin%
\definecolor{currentfill}{rgb}{0.000000,0.000000,0.000000}%
\pgfsetfillcolor{currentfill}%
\pgfsetlinewidth{0.803000pt}%
\definecolor{currentstroke}{rgb}{0.000000,0.000000,0.000000}%
\pgfsetstrokecolor{currentstroke}%
\pgfsetdash{}{0pt}%
\pgfsys@defobject{currentmarker}{\pgfqpoint{-0.048611in}{0.000000in}}{\pgfqpoint{-0.000000in}{0.000000in}}{%
\pgfpathmoveto{\pgfqpoint{-0.000000in}{0.000000in}}%
\pgfpathlineto{\pgfqpoint{-0.048611in}{0.000000in}}%
\pgfusepath{stroke,fill}%
}%
\begin{pgfscope}%
\pgfsys@transformshift{0.608070in}{2.617420in}%
\pgfsys@useobject{currentmarker}{}%
\end{pgfscope}%
\end{pgfscope}%
\begin{pgfscope}%
\definecolor{textcolor}{rgb}{0.000000,0.000000,0.000000}%
\pgfsetstrokecolor{textcolor}%
\pgfsetfillcolor{textcolor}%
\pgftext[x=0.289968in, y=2.564659in, left, base]{\color{textcolor}\sffamily\fontsize{10.000000}{12.000000}\selectfont 0.2}%
\end{pgfscope}%
\begin{pgfscope}%
\pgfsetbuttcap%
\pgfsetroundjoin%
\definecolor{currentfill}{rgb}{0.000000,0.000000,0.000000}%
\pgfsetfillcolor{currentfill}%
\pgfsetlinewidth{0.803000pt}%
\definecolor{currentstroke}{rgb}{0.000000,0.000000,0.000000}%
\pgfsetstrokecolor{currentstroke}%
\pgfsetdash{}{0pt}%
\pgfsys@defobject{currentmarker}{\pgfqpoint{-0.048611in}{0.000000in}}{\pgfqpoint{-0.000000in}{0.000000in}}{%
\pgfpathmoveto{\pgfqpoint{-0.000000in}{0.000000in}}%
\pgfpathlineto{\pgfqpoint{-0.048611in}{0.000000in}}%
\pgfusepath{stroke,fill}%
}%
\begin{pgfscope}%
\pgfsys@transformshift{0.608070in}{3.206243in}%
\pgfsys@useobject{currentmarker}{}%
\end{pgfscope}%
\end{pgfscope}%
\begin{pgfscope}%
\definecolor{textcolor}{rgb}{0.000000,0.000000,0.000000}%
\pgfsetstrokecolor{textcolor}%
\pgfsetfillcolor{textcolor}%
\pgftext[x=0.289968in, y=3.153481in, left, base]{\color{textcolor}\sffamily\fontsize{10.000000}{12.000000}\selectfont 0.3}%
\end{pgfscope}%
\begin{pgfscope}%
\pgfsetbuttcap%
\pgfsetroundjoin%
\definecolor{currentfill}{rgb}{0.000000,0.000000,0.000000}%
\pgfsetfillcolor{currentfill}%
\pgfsetlinewidth{0.803000pt}%
\definecolor{currentstroke}{rgb}{0.000000,0.000000,0.000000}%
\pgfsetstrokecolor{currentstroke}%
\pgfsetdash{}{0pt}%
\pgfsys@defobject{currentmarker}{\pgfqpoint{-0.048611in}{0.000000in}}{\pgfqpoint{-0.000000in}{0.000000in}}{%
\pgfpathmoveto{\pgfqpoint{-0.000000in}{0.000000in}}%
\pgfpathlineto{\pgfqpoint{-0.048611in}{0.000000in}}%
\pgfusepath{stroke,fill}%
}%
\begin{pgfscope}%
\pgfsys@transformshift{0.608070in}{3.795065in}%
\pgfsys@useobject{currentmarker}{}%
\end{pgfscope}%
\end{pgfscope}%
\begin{pgfscope}%
\definecolor{textcolor}{rgb}{0.000000,0.000000,0.000000}%
\pgfsetstrokecolor{textcolor}%
\pgfsetfillcolor{textcolor}%
\pgftext[x=0.289968in, y=3.742304in, left, base]{\color{textcolor}\sffamily\fontsize{10.000000}{12.000000}\selectfont 0.4}%
\end{pgfscope}%
\begin{pgfscope}%
\pgfsetbuttcap%
\pgfsetroundjoin%
\definecolor{currentfill}{rgb}{0.000000,0.000000,0.000000}%
\pgfsetfillcolor{currentfill}%
\pgfsetlinewidth{0.803000pt}%
\definecolor{currentstroke}{rgb}{0.000000,0.000000,0.000000}%
\pgfsetstrokecolor{currentstroke}%
\pgfsetdash{}{0pt}%
\pgfsys@defobject{currentmarker}{\pgfqpoint{-0.048611in}{0.000000in}}{\pgfqpoint{-0.000000in}{0.000000in}}{%
\pgfpathmoveto{\pgfqpoint{-0.000000in}{0.000000in}}%
\pgfpathlineto{\pgfqpoint{-0.048611in}{0.000000in}}%
\pgfusepath{stroke,fill}%
}%
\begin{pgfscope}%
\pgfsys@transformshift{0.608070in}{4.383888in}%
\pgfsys@useobject{currentmarker}{}%
\end{pgfscope}%
\end{pgfscope}%
\begin{pgfscope}%
\definecolor{textcolor}{rgb}{0.000000,0.000000,0.000000}%
\pgfsetstrokecolor{textcolor}%
\pgfsetfillcolor{textcolor}%
\pgftext[x=0.289968in, y=4.331126in, left, base]{\color{textcolor}\sffamily\fontsize{10.000000}{12.000000}\selectfont 0.5}%
\end{pgfscope}%
\begin{pgfscope}%
\definecolor{textcolor}{rgb}{0.000000,0.000000,0.000000}%
\pgfsetstrokecolor{textcolor}%
\pgfsetfillcolor{textcolor}%
\pgftext[x=0.234413in,y=2.911831in,,bottom,rotate=90.000000]{\color{textcolor}\sffamily\fontsize{10.000000}{12.000000}\selectfont F1 Score}%
\end{pgfscope}%
\begin{pgfscope}%
\pgfsetrectcap%
\pgfsetmiterjoin%
\pgfsetlinewidth{0.803000pt}%
\definecolor{currentstroke}{rgb}{0.000000,0.000000,0.000000}%
\pgfsetstrokecolor{currentstroke}%
\pgfsetdash{}{0pt}%
\pgfpathmoveto{\pgfqpoint{0.608070in}{1.439775in}}%
\pgfpathlineto{\pgfqpoint{0.608070in}{4.383888in}}%
\pgfusepath{stroke}%
\end{pgfscope}%
\begin{pgfscope}%
\pgfsetrectcap%
\pgfsetmiterjoin%
\pgfsetlinewidth{0.803000pt}%
\definecolor{currentstroke}{rgb}{0.000000,0.000000,0.000000}%
\pgfsetstrokecolor{currentstroke}%
\pgfsetdash{}{0pt}%
\pgfpathmoveto{\pgfqpoint{6.194042in}{1.439775in}}%
\pgfpathlineto{\pgfqpoint{6.194042in}{4.383888in}}%
\pgfusepath{stroke}%
\end{pgfscope}%
\begin{pgfscope}%
\pgfsetrectcap%
\pgfsetmiterjoin%
\pgfsetlinewidth{0.803000pt}%
\definecolor{currentstroke}{rgb}{0.000000,0.000000,0.000000}%
\pgfsetstrokecolor{currentstroke}%
\pgfsetdash{}{0pt}%
\pgfpathmoveto{\pgfqpoint{0.608070in}{1.439775in}}%
\pgfpathlineto{\pgfqpoint{6.194042in}{1.439775in}}%
\pgfusepath{stroke}%
\end{pgfscope}%
\begin{pgfscope}%
\pgfsetrectcap%
\pgfsetmiterjoin%
\pgfsetlinewidth{0.803000pt}%
\definecolor{currentstroke}{rgb}{0.000000,0.000000,0.000000}%
\pgfsetstrokecolor{currentstroke}%
\pgfsetdash{}{0pt}%
\pgfpathmoveto{\pgfqpoint{0.608070in}{4.383888in}}%
\pgfpathlineto{\pgfqpoint{6.194042in}{4.383888in}}%
\pgfusepath{stroke}%
\end{pgfscope}%
\begin{pgfscope}%
\definecolor{textcolor}{rgb}{0.000000,0.000000,0.000000}%
\pgfsetstrokecolor{textcolor}%
\pgfsetfillcolor{textcolor}%
\pgftext[x=1.027570in,y=3.864407in,,bottom]{\color{textcolor}\sffamily\fontsize{9.000000}{10.800000}\selectfont 0.4047}%
\end{pgfscope}%
\begin{pgfscope}%
\definecolor{textcolor}{rgb}{0.000000,0.000000,0.000000}%
\pgfsetstrokecolor{textcolor}%
\pgfsetfillcolor{textcolor}%
\pgftext[x=1.763535in,y=4.158818in,,bottom]{\color{textcolor}\sffamily\fontsize{9.000000}{10.800000}\selectfont 0.4547}%
\end{pgfscope}%
\begin{pgfscope}%
\definecolor{textcolor}{rgb}{0.000000,0.000000,0.000000}%
\pgfsetstrokecolor{textcolor}%
\pgfsetfillcolor{textcolor}%
\pgftext[x=2.499499in,y=3.208458in,,bottom]{\color{textcolor}\sffamily\fontsize{9.000000}{10.800000}\selectfont 0.2933}%
\end{pgfscope}%
\begin{pgfscope}%
\definecolor{textcolor}{rgb}{0.000000,0.000000,0.000000}%
\pgfsetstrokecolor{textcolor}%
\pgfsetfillcolor{textcolor}%
\pgftext[x=3.235464in,y=2.906392in,,bottom]{\color{textcolor}\sffamily\fontsize{9.000000}{10.800000}\selectfont 0.242}%
\end{pgfscope}%
\begin{pgfscope}%
\definecolor{textcolor}{rgb}{0.000000,0.000000,0.000000}%
\pgfsetstrokecolor{textcolor}%
\pgfsetfillcolor{textcolor}%
\pgftext[x=3.971429in,y=3.549975in,,bottom]{\color{textcolor}\sffamily\fontsize{9.000000}{10.800000}\selectfont 0.3513}%
\end{pgfscope}%
\begin{pgfscope}%
\definecolor{textcolor}{rgb}{0.000000,0.000000,0.000000}%
\pgfsetstrokecolor{textcolor}%
\pgfsetfillcolor{textcolor}%
\pgftext[x=4.707393in,y=3.553508in,,bottom]{\color{textcolor}\sffamily\fontsize{9.000000}{10.800000}\selectfont 0.3519}%
\end{pgfscope}%
\begin{pgfscope}%
\definecolor{textcolor}{rgb}{0.000000,0.000000,0.000000}%
\pgfsetstrokecolor{textcolor}%
\pgfsetfillcolor{textcolor}%
\pgftext[x=5.443358in,y=3.663029in,,bottom]{\color{textcolor}\sffamily\fontsize{9.000000}{10.800000}\selectfont 0.3705}%
\end{pgfscope}%
\begin{pgfscope}%
\definecolor{textcolor}{rgb}{0.000000,0.000000,0.000000}%
\pgfsetstrokecolor{textcolor}%
\pgfsetfillcolor{textcolor}%
\pgftext[x=1.358754in,y=3.803758in,,bottom]{\color{textcolor}\sffamily\fontsize{9.000000}{10.800000}\selectfont 0.3944}%
\end{pgfscope}%
\begin{pgfscope}%
\definecolor{textcolor}{rgb}{0.000000,0.000000,0.000000}%
\pgfsetstrokecolor{textcolor}%
\pgfsetfillcolor{textcolor}%
\pgftext[x=2.094719in,y=3.939776in,,bottom]{\color{textcolor}\sffamily\fontsize{9.000000}{10.800000}\selectfont 0.4175}%
\end{pgfscope}%
\begin{pgfscope}%
\definecolor{textcolor}{rgb}{0.000000,0.000000,0.000000}%
\pgfsetstrokecolor{textcolor}%
\pgfsetfillcolor{textcolor}%
\pgftext[x=2.830683in,y=2.285185in,,bottom]{\color{textcolor}\sffamily\fontsize{9.000000}{10.800000}\selectfont 0.1365}%
\end{pgfscope}%
\begin{pgfscope}%
\definecolor{textcolor}{rgb}{0.000000,0.000000,0.000000}%
\pgfsetstrokecolor{textcolor}%
\pgfsetfillcolor{textcolor}%
\pgftext[x=3.566648in,y=2.538378in,,bottom]{\color{textcolor}\sffamily\fontsize{9.000000}{10.800000}\selectfont 0.1795}%
\end{pgfscope}%
\begin{pgfscope}%
\definecolor{textcolor}{rgb}{0.000000,0.000000,0.000000}%
\pgfsetstrokecolor{textcolor}%
\pgfsetfillcolor{textcolor}%
\pgftext[x=4.302613in,y=2.634945in,,bottom]{\color{textcolor}\sffamily\fontsize{9.000000}{10.800000}\selectfont 0.1959}%
\end{pgfscope}%
\begin{pgfscope}%
\definecolor{textcolor}{rgb}{0.000000,0.000000,0.000000}%
\pgfsetstrokecolor{textcolor}%
\pgfsetfillcolor{textcolor}%
\pgftext[x=5.038578in,y=2.523069in,,bottom]{\color{textcolor}\sffamily\fontsize{9.000000}{10.800000}\selectfont 0.1769}%
\end{pgfscope}%
\begin{pgfscope}%
\definecolor{textcolor}{rgb}{0.000000,0.000000,0.000000}%
\pgfsetstrokecolor{textcolor}%
\pgfsetfillcolor{textcolor}%
\pgftext[x=5.774542in,y=2.470664in,,bottom]{\color{textcolor}\sffamily\fontsize{9.000000}{10.800000}\selectfont 0.168}%
\end{pgfscope}%
\begin{pgfscope}%
\definecolor{textcolor}{rgb}{0.000000,0.000000,0.000000}%
\pgfsetstrokecolor{textcolor}%
\pgfsetfillcolor{textcolor}%
\pgftext[x=3.401056in,y=4.467221in,,base]{\color{textcolor}\sffamily\fontsize{12.000000}{14.400000}\selectfont Abalation study on chairs}%
\end{pgfscope}%
\begin{pgfscope}%
\pgfsetbuttcap%
\pgfsetmiterjoin%
\definecolor{currentfill}{rgb}{1.000000,1.000000,1.000000}%
\pgfsetfillcolor{currentfill}%
\pgfsetfillopacity{0.800000}%
\pgfsetlinewidth{1.003750pt}%
\definecolor{currentstroke}{rgb}{0.800000,0.800000,0.800000}%
\pgfsetstrokecolor{currentstroke}%
\pgfsetstrokeopacity{0.800000}%
\pgfsetdash{}{0pt}%
\pgfpathmoveto{\pgfqpoint{4.882654in}{3.865062in}}%
\pgfpathlineto{\pgfqpoint{6.096820in}{3.865062in}}%
\pgfpathquadraticcurveto{\pgfqpoint{6.124598in}{3.865062in}}{\pgfqpoint{6.124598in}{3.892840in}}%
\pgfpathlineto{\pgfqpoint{6.124598in}{4.286666in}}%
\pgfpathquadraticcurveto{\pgfqpoint{6.124598in}{4.314443in}}{\pgfqpoint{6.096820in}{4.314443in}}%
\pgfpathlineto{\pgfqpoint{4.882654in}{4.314443in}}%
\pgfpathquadraticcurveto{\pgfqpoint{4.854877in}{4.314443in}}{\pgfqpoint{4.854877in}{4.286666in}}%
\pgfpathlineto{\pgfqpoint{4.854877in}{3.892840in}}%
\pgfpathquadraticcurveto{\pgfqpoint{4.854877in}{3.865062in}}{\pgfqpoint{4.882654in}{3.865062in}}%
\pgfpathclose%
\pgfusepath{stroke,fill}%
\end{pgfscope}%
\begin{pgfscope}%
\pgfsetbuttcap%
\pgfsetmiterjoin%
\definecolor{currentfill}{rgb}{0.121569,0.466667,0.705882}%
\pgfsetfillcolor{currentfill}%
\pgfsetlinewidth{0.000000pt}%
\definecolor{currentstroke}{rgb}{0.000000,0.000000,0.000000}%
\pgfsetstrokecolor{currentstroke}%
\pgfsetstrokeopacity{0.000000}%
\pgfsetdash{}{0pt}%
\pgfpathmoveto{\pgfqpoint{4.910432in}{4.153365in}}%
\pgfpathlineto{\pgfqpoint{5.188210in}{4.153365in}}%
\pgfpathlineto{\pgfqpoint{5.188210in}{4.250587in}}%
\pgfpathlineto{\pgfqpoint{4.910432in}{4.250587in}}%
\pgfpathclose%
\pgfusepath{fill}%
\end{pgfscope}%
\begin{pgfscope}%
\definecolor{textcolor}{rgb}{0.000000,0.000000,0.000000}%
\pgfsetstrokecolor{textcolor}%
\pgfsetfillcolor{textcolor}%
\pgftext[x=5.299321in,y=4.153365in,left,base]{\color{textcolor}\sffamily\fontsize{10.000000}{12.000000}\selectfont Pix2Vox++}%
\end{pgfscope}%
\begin{pgfscope}%
\pgfsetbuttcap%
\pgfsetmiterjoin%
\definecolor{currentfill}{rgb}{1.000000,0.498039,0.054902}%
\pgfsetfillcolor{currentfill}%
\pgfsetlinewidth{0.000000pt}%
\definecolor{currentstroke}{rgb}{0.000000,0.000000,0.000000}%
\pgfsetstrokecolor{currentstroke}%
\pgfsetstrokeopacity{0.000000}%
\pgfsetdash{}{0pt}%
\pgfpathmoveto{\pgfqpoint{4.910432in}{3.949507in}}%
\pgfpathlineto{\pgfqpoint{5.188210in}{3.949507in}}%
\pgfpathlineto{\pgfqpoint{5.188210in}{4.046730in}}%
\pgfpathlineto{\pgfqpoint{4.910432in}{4.046730in}}%
\pgfpathclose%
\pgfusepath{fill}%
\end{pgfscope}%
\begin{pgfscope}%
\definecolor{textcolor}{rgb}{0.000000,0.000000,0.000000}%
\pgfsetstrokecolor{textcolor}%
\pgfsetfillcolor{textcolor}%
\pgftext[x=5.299321in,y=3.949507in,left,base]{\color{textcolor}\sffamily\fontsize{10.000000}{12.000000}\selectfont Pix2Vox}%
\end{pgfscope}%
\end{pgfpicture}%
\makeatother%
\endgroup%
}
    \caption{Bar plot for the \gls{f1}  for baseline trained on chair dataset with different domain randomization parameters and tested on real dataset.
    We see a dip in performance near textureless dataset, but it gradually increases with addition of domain randomization parameter.}
    \label{fig:ablation_dice1}
\end{figure}

To further improve the performance(as in \autoref{fig:ablation_dice2}), we do a mixed training with 50\% of real data per mini-batch to witness an increment of average \gls{f1} in the range of 3 to 7\%,
While the baseline Pix2vox++ give a \gls{f1} of 0.4547 on real chairs, with the textureless chair we get a \gks{f1} of 0.4879.
For Pix2Vox, the difference is slightly lesser, as it is 0.4175and 0.4659 on real and textureless chairs.
With the combined dataset we see that Pix2Vox++ has \gls{f1} of 0.4962 and Pix2Vox has 0.4812, which is 4.15\% and 6.37\% improvement over real dataset.

%\begin{figure}
%    \centering
%    \resizebox{0.7\textwidth}{!}{\input{/Users/apple/OVGU/Thesis/code/3dReconstruction/report/images/evaluation/performance/ablation_dice_linegraph1.pgf}}
%    \caption{Line plot for the \gls{f1}  for baseline trained on chair dataset with different domain randomization parameters and tested on real dataset.
%    We see a dip in performance near textureless dataset, but it gradually increases with addition of domain randomization parameter.}
%    \label{fig:ablation_dice1}
%\end{figure}


%\begin{figure}
%    \centering
%    \resizebox{0.7\textwidth}{!}{%% Creator: Matplotlib, PGF backend
%%
%% To include the figure in your LaTeX document, write
%%   \input{<filename>.pgf}
%%
%% Make sure the required packages are loaded in your preamble
%%   \usepackage{pgf}
%%
%% Figures using additional raster images can only be included by \input if
%% they are in the same directory as the main LaTeX file. For loading figures
%% from other directories you can use the `import` package
%%   \usepackage{import}
%%
%% and then include the figures with
%%   \import{<path to file>}{<filename>.pgf}
%%
%% Matplotlib used the following preamble
%%   \usepackage{fontspec}
%%   \setmainfont{DejaVuSerif.ttf}[Path=\detokenize{/Users/apple/opt/anaconda3/envs/kaolin/lib/python3.7/site-packages/matplotlib/mpl-data/fonts/ttf/}]
%%   \setsansfont{DejaVuSans.ttf}[Path=\detokenize{/Users/apple/opt/anaconda3/envs/kaolin/lib/python3.7/site-packages/matplotlib/mpl-data/fonts/ttf/}]
%%   \setmonofont{DejaVuSansMono.ttf}[Path=\detokenize{/Users/apple/opt/anaconda3/envs/kaolin/lib/python3.7/site-packages/matplotlib/mpl-data/fonts/ttf/}]
%%
\begingroup%
\makeatletter%
\begin{pgfpicture}%
\pgfpathrectangle{\pgfpointorigin}{\pgfqpoint{5.827982in}{5.207926in}}%
\pgfusepath{use as bounding box, clip}%
\begin{pgfscope}%
\pgfsetbuttcap%
\pgfsetmiterjoin%
\definecolor{currentfill}{rgb}{1.000000,1.000000,1.000000}%
\pgfsetfillcolor{currentfill}%
\pgfsetlinewidth{0.000000pt}%
\definecolor{currentstroke}{rgb}{1.000000,1.000000,1.000000}%
\pgfsetstrokecolor{currentstroke}%
\pgfsetdash{}{0pt}%
\pgfpathmoveto{\pgfqpoint{0.000000in}{0.000000in}}%
\pgfpathlineto{\pgfqpoint{5.827982in}{0.000000in}}%
\pgfpathlineto{\pgfqpoint{5.827982in}{5.207926in}}%
\pgfpathlineto{\pgfqpoint{0.000000in}{5.207926in}}%
\pgfpathclose%
\pgfusepath{fill}%
\end{pgfscope}%
\begin{pgfscope}%
\pgfsetbuttcap%
\pgfsetmiterjoin%
\definecolor{currentfill}{rgb}{1.000000,1.000000,1.000000}%
\pgfsetfillcolor{currentfill}%
\pgfsetlinewidth{0.000000pt}%
\definecolor{currentstroke}{rgb}{0.000000,0.000000,0.000000}%
\pgfsetstrokecolor{currentstroke}%
\pgfsetstrokeopacity{0.000000}%
\pgfsetdash{}{0pt}%
\pgfpathmoveto{\pgfqpoint{0.696435in}{1.359165in}}%
\pgfpathlineto{\pgfqpoint{5.656435in}{1.359165in}}%
\pgfpathlineto{\pgfqpoint{5.656435in}{5.055165in}}%
\pgfpathlineto{\pgfqpoint{0.696435in}{5.055165in}}%
\pgfpathclose%
\pgfusepath{fill}%
\end{pgfscope}%
\begin{pgfscope}%
\pgfsetbuttcap%
\pgfsetroundjoin%
\definecolor{currentfill}{rgb}{0.000000,0.000000,0.000000}%
\pgfsetfillcolor{currentfill}%
\pgfsetlinewidth{0.803000pt}%
\definecolor{currentstroke}{rgb}{0.000000,0.000000,0.000000}%
\pgfsetstrokecolor{currentstroke}%
\pgfsetdash{}{0pt}%
\pgfsys@defobject{currentmarker}{\pgfqpoint{0.000000in}{-0.048611in}}{\pgfqpoint{0.000000in}{0.000000in}}{%
\pgfpathmoveto{\pgfqpoint{0.000000in}{0.000000in}}%
\pgfpathlineto{\pgfqpoint{0.000000in}{-0.048611in}}%
\pgfusepath{stroke,fill}%
}%
\begin{pgfscope}%
\pgfsys@transformshift{0.921890in}{1.359165in}%
\pgfsys@useobject{currentmarker}{}%
\end{pgfscope}%
\end{pgfscope}%
\begin{pgfscope}%
\definecolor{textcolor}{rgb}{0.000000,0.000000,0.000000}%
\pgfsetstrokecolor{textcolor}%
\pgfsetfillcolor{textcolor}%
\pgftext[x=0.654979in, y=0.599318in, left, base,rotate=45.000000]{\color{textcolor}\sffamily\fontsize{10.000000}{12.000000}\selectfont Pix3d(chair)}%
\end{pgfscope}%
\begin{pgfscope}%
\pgfsetbuttcap%
\pgfsetroundjoin%
\definecolor{currentfill}{rgb}{0.000000,0.000000,0.000000}%
\pgfsetfillcolor{currentfill}%
\pgfsetlinewidth{0.803000pt}%
\definecolor{currentstroke}{rgb}{0.000000,0.000000,0.000000}%
\pgfsetstrokecolor{currentstroke}%
\pgfsetdash{}{0pt}%
\pgfsys@defobject{currentmarker}{\pgfqpoint{0.000000in}{-0.048611in}}{\pgfqpoint{0.000000in}{0.000000in}}{%
\pgfpathmoveto{\pgfqpoint{0.000000in}{0.000000in}}%
\pgfpathlineto{\pgfqpoint{0.000000in}{-0.048611in}}%
\pgfusepath{stroke,fill}%
}%
\begin{pgfscope}%
\pgfsys@transformshift{1.673405in}{1.359165in}%
\pgfsys@useobject{currentmarker}{}%
\end{pgfscope}%
\end{pgfscope}%
\begin{pgfscope}%
\definecolor{textcolor}{rgb}{0.000000,0.000000,0.000000}%
\pgfsetstrokecolor{textcolor}%
\pgfsetfillcolor{textcolor}%
\pgftext[x=1.425724in, y=0.637777in, left, base,rotate=45.000000]{\color{textcolor}\sffamily\fontsize{10.000000}{12.000000}\selectfont Textureless}%
\end{pgfscope}%
\begin{pgfscope}%
\pgfsetbuttcap%
\pgfsetroundjoin%
\definecolor{currentfill}{rgb}{0.000000,0.000000,0.000000}%
\pgfsetfillcolor{currentfill}%
\pgfsetlinewidth{0.803000pt}%
\definecolor{currentstroke}{rgb}{0.000000,0.000000,0.000000}%
\pgfsetstrokecolor{currentstroke}%
\pgfsetdash{}{0pt}%
\pgfsys@defobject{currentmarker}{\pgfqpoint{0.000000in}{-0.048611in}}{\pgfqpoint{0.000000in}{0.000000in}}{%
\pgfpathmoveto{\pgfqpoint{0.000000in}{0.000000in}}%
\pgfpathlineto{\pgfqpoint{0.000000in}{-0.048611in}}%
\pgfusepath{stroke,fill}%
}%
\begin{pgfscope}%
\pgfsys@transformshift{2.424920in}{1.359165in}%
\pgfsys@useobject{currentmarker}{}%
\end{pgfscope}%
\end{pgfscope}%
\begin{pgfscope}%
\definecolor{textcolor}{rgb}{0.000000,0.000000,0.000000}%
\pgfsetstrokecolor{textcolor}%
\pgfsetfillcolor{textcolor}%
\pgftext[x=2.013549in, y=0.310396in, left, base,rotate=45.000000]{\color{textcolor}\sffamily\fontsize{10.000000}{12.000000}\selectfont Textureless+Light}%
\end{pgfscope}%
\begin{pgfscope}%
\pgfsetbuttcap%
\pgfsetroundjoin%
\definecolor{currentfill}{rgb}{0.000000,0.000000,0.000000}%
\pgfsetfillcolor{currentfill}%
\pgfsetlinewidth{0.803000pt}%
\definecolor{currentstroke}{rgb}{0.000000,0.000000,0.000000}%
\pgfsetstrokecolor{currentstroke}%
\pgfsetdash{}{0pt}%
\pgfsys@defobject{currentmarker}{\pgfqpoint{0.000000in}{-0.048611in}}{\pgfqpoint{0.000000in}{0.000000in}}{%
\pgfpathmoveto{\pgfqpoint{0.000000in}{0.000000in}}%
\pgfpathlineto{\pgfqpoint{0.000000in}{-0.048611in}}%
\pgfusepath{stroke,fill}%
}%
\begin{pgfscope}%
\pgfsys@transformshift{3.176435in}{1.359165in}%
\pgfsys@useobject{currentmarker}{}%
\end{pgfscope}%
\end{pgfscope}%
\begin{pgfscope}%
\definecolor{textcolor}{rgb}{0.000000,0.000000,0.000000}%
\pgfsetstrokecolor{textcolor}%
\pgfsetfillcolor{textcolor}%
\pgftext[x=2.992605in, y=0.765478in, left, base,rotate=45.000000]{\color{textcolor}\sffamily\fontsize{10.000000}{12.000000}\selectfont Textured}%
\end{pgfscope}%
\begin{pgfscope}%
\pgfsetbuttcap%
\pgfsetroundjoin%
\definecolor{currentfill}{rgb}{0.000000,0.000000,0.000000}%
\pgfsetfillcolor{currentfill}%
\pgfsetlinewidth{0.803000pt}%
\definecolor{currentstroke}{rgb}{0.000000,0.000000,0.000000}%
\pgfsetstrokecolor{currentstroke}%
\pgfsetdash{}{0pt}%
\pgfsys@defobject{currentmarker}{\pgfqpoint{0.000000in}{-0.048611in}}{\pgfqpoint{0.000000in}{0.000000in}}{%
\pgfpathmoveto{\pgfqpoint{0.000000in}{0.000000in}}%
\pgfpathlineto{\pgfqpoint{0.000000in}{-0.048611in}}%
\pgfusepath{stroke,fill}%
}%
\begin{pgfscope}%
\pgfsys@transformshift{3.927951in}{1.359165in}%
\pgfsys@useobject{currentmarker}{}%
\end{pgfscope}%
\end{pgfscope}%
\begin{pgfscope}%
\definecolor{textcolor}{rgb}{0.000000,0.000000,0.000000}%
\pgfsetstrokecolor{textcolor}%
\pgfsetfillcolor{textcolor}%
\pgftext[x=3.580430in, y=0.438097in, left, base,rotate=45.000000]{\color{textcolor}\sffamily\fontsize{10.000000}{12.000000}\selectfont Textured+Light}%
\end{pgfscope}%
\begin{pgfscope}%
\pgfsetbuttcap%
\pgfsetroundjoin%
\definecolor{currentfill}{rgb}{0.000000,0.000000,0.000000}%
\pgfsetfillcolor{currentfill}%
\pgfsetlinewidth{0.803000pt}%
\definecolor{currentstroke}{rgb}{0.000000,0.000000,0.000000}%
\pgfsetstrokecolor{currentstroke}%
\pgfsetdash{}{0pt}%
\pgfsys@defobject{currentmarker}{\pgfqpoint{0.000000in}{-0.048611in}}{\pgfqpoint{0.000000in}{0.000000in}}{%
\pgfpathmoveto{\pgfqpoint{0.000000in}{0.000000in}}%
\pgfpathlineto{\pgfqpoint{0.000000in}{-0.048611in}}%
\pgfusepath{stroke,fill}%
}%
\begin{pgfscope}%
\pgfsys@transformshift{4.679466in}{1.359165in}%
\pgfsys@useobject{currentmarker}{}%
\end{pgfscope}%
\end{pgfscope}%
\begin{pgfscope}%
\definecolor{textcolor}{rgb}{0.000000,0.000000,0.000000}%
\pgfsetstrokecolor{textcolor}%
\pgfsetfillcolor{textcolor}%
\pgftext[x=4.407520in, y=0.589247in, left, base,rotate=45.000000]{\color{textcolor}\sffamily\fontsize{10.000000}{12.000000}\selectfont Multi-Object}%
\end{pgfscope}%
\begin{pgfscope}%
\pgfsetbuttcap%
\pgfsetroundjoin%
\definecolor{currentfill}{rgb}{0.000000,0.000000,0.000000}%
\pgfsetfillcolor{currentfill}%
\pgfsetlinewidth{0.803000pt}%
\definecolor{currentstroke}{rgb}{0.000000,0.000000,0.000000}%
\pgfsetstrokecolor{currentstroke}%
\pgfsetdash{}{0pt}%
\pgfsys@defobject{currentmarker}{\pgfqpoint{0.000000in}{-0.048611in}}{\pgfqpoint{0.000000in}{0.000000in}}{%
\pgfpathmoveto{\pgfqpoint{0.000000in}{0.000000in}}%
\pgfpathlineto{\pgfqpoint{0.000000in}{-0.048611in}}%
\pgfusepath{stroke,fill}%
}%
\begin{pgfscope}%
\pgfsys@transformshift{5.430981in}{1.359165in}%
\pgfsys@useobject{currentmarker}{}%
\end{pgfscope}%
\end{pgfscope}%
\begin{pgfscope}%
\definecolor{textcolor}{rgb}{0.000000,0.000000,0.000000}%
\pgfsetstrokecolor{textcolor}%
\pgfsetfillcolor{textcolor}%
\pgftext[x=5.208595in, y=0.688368in, left, base,rotate=45.000000]{\color{textcolor}\sffamily\fontsize{10.000000}{12.000000}\selectfont Combined}%
\end{pgfscope}%
\begin{pgfscope}%
\definecolor{textcolor}{rgb}{0.000000,0.000000,0.000000}%
\pgfsetstrokecolor{textcolor}%
\pgfsetfillcolor{textcolor}%
\pgftext[x=3.176435in,y=0.234413in,,top]{\color{textcolor}\sffamily\fontsize{10.000000}{12.000000}\selectfont Datasets}%
\end{pgfscope}%
\begin{pgfscope}%
\pgfsetbuttcap%
\pgfsetroundjoin%
\definecolor{currentfill}{rgb}{0.000000,0.000000,0.000000}%
\pgfsetfillcolor{currentfill}%
\pgfsetlinewidth{0.803000pt}%
\definecolor{currentstroke}{rgb}{0.000000,0.000000,0.000000}%
\pgfsetstrokecolor{currentstroke}%
\pgfsetdash{}{0pt}%
\pgfsys@defobject{currentmarker}{\pgfqpoint{-0.048611in}{0.000000in}}{\pgfqpoint{-0.000000in}{0.000000in}}{%
\pgfpathmoveto{\pgfqpoint{-0.000000in}{0.000000in}}%
\pgfpathlineto{\pgfqpoint{-0.048611in}{0.000000in}}%
\pgfusepath{stroke,fill}%
}%
\begin{pgfscope}%
\pgfsys@transformshift{0.696435in}{1.359165in}%
\pgfsys@useobject{currentmarker}{}%
\end{pgfscope}%
\end{pgfscope}%
\begin{pgfscope}%
\definecolor{textcolor}{rgb}{0.000000,0.000000,0.000000}%
\pgfsetstrokecolor{textcolor}%
\pgfsetfillcolor{textcolor}%
\pgftext[x=0.289968in, y=1.306403in, left, base]{\color{textcolor}\sffamily\fontsize{10.000000}{12.000000}\selectfont 0.30}%
\end{pgfscope}%
\begin{pgfscope}%
\pgfsetbuttcap%
\pgfsetroundjoin%
\definecolor{currentfill}{rgb}{0.000000,0.000000,0.000000}%
\pgfsetfillcolor{currentfill}%
\pgfsetlinewidth{0.803000pt}%
\definecolor{currentstroke}{rgb}{0.000000,0.000000,0.000000}%
\pgfsetstrokecolor{currentstroke}%
\pgfsetdash{}{0pt}%
\pgfsys@defobject{currentmarker}{\pgfqpoint{-0.048611in}{0.000000in}}{\pgfqpoint{-0.000000in}{0.000000in}}{%
\pgfpathmoveto{\pgfqpoint{-0.000000in}{0.000000in}}%
\pgfpathlineto{\pgfqpoint{-0.048611in}{0.000000in}}%
\pgfusepath{stroke,fill}%
}%
\begin{pgfscope}%
\pgfsys@transformshift{0.696435in}{1.975165in}%
\pgfsys@useobject{currentmarker}{}%
\end{pgfscope}%
\end{pgfscope}%
\begin{pgfscope}%
\definecolor{textcolor}{rgb}{0.000000,0.000000,0.000000}%
\pgfsetstrokecolor{textcolor}%
\pgfsetfillcolor{textcolor}%
\pgftext[x=0.289968in, y=1.922403in, left, base]{\color{textcolor}\sffamily\fontsize{10.000000}{12.000000}\selectfont 0.35}%
\end{pgfscope}%
\begin{pgfscope}%
\pgfsetbuttcap%
\pgfsetroundjoin%
\definecolor{currentfill}{rgb}{0.000000,0.000000,0.000000}%
\pgfsetfillcolor{currentfill}%
\pgfsetlinewidth{0.803000pt}%
\definecolor{currentstroke}{rgb}{0.000000,0.000000,0.000000}%
\pgfsetstrokecolor{currentstroke}%
\pgfsetdash{}{0pt}%
\pgfsys@defobject{currentmarker}{\pgfqpoint{-0.048611in}{0.000000in}}{\pgfqpoint{-0.000000in}{0.000000in}}{%
\pgfpathmoveto{\pgfqpoint{-0.000000in}{0.000000in}}%
\pgfpathlineto{\pgfqpoint{-0.048611in}{0.000000in}}%
\pgfusepath{stroke,fill}%
}%
\begin{pgfscope}%
\pgfsys@transformshift{0.696435in}{2.591165in}%
\pgfsys@useobject{currentmarker}{}%
\end{pgfscope}%
\end{pgfscope}%
\begin{pgfscope}%
\definecolor{textcolor}{rgb}{0.000000,0.000000,0.000000}%
\pgfsetstrokecolor{textcolor}%
\pgfsetfillcolor{textcolor}%
\pgftext[x=0.289968in, y=2.538403in, left, base]{\color{textcolor}\sffamily\fontsize{10.000000}{12.000000}\selectfont 0.40}%
\end{pgfscope}%
\begin{pgfscope}%
\pgfsetbuttcap%
\pgfsetroundjoin%
\definecolor{currentfill}{rgb}{0.000000,0.000000,0.000000}%
\pgfsetfillcolor{currentfill}%
\pgfsetlinewidth{0.803000pt}%
\definecolor{currentstroke}{rgb}{0.000000,0.000000,0.000000}%
\pgfsetstrokecolor{currentstroke}%
\pgfsetdash{}{0pt}%
\pgfsys@defobject{currentmarker}{\pgfqpoint{-0.048611in}{0.000000in}}{\pgfqpoint{-0.000000in}{0.000000in}}{%
\pgfpathmoveto{\pgfqpoint{-0.000000in}{0.000000in}}%
\pgfpathlineto{\pgfqpoint{-0.048611in}{0.000000in}}%
\pgfusepath{stroke,fill}%
}%
\begin{pgfscope}%
\pgfsys@transformshift{0.696435in}{3.207165in}%
\pgfsys@useobject{currentmarker}{}%
\end{pgfscope}%
\end{pgfscope}%
\begin{pgfscope}%
\definecolor{textcolor}{rgb}{0.000000,0.000000,0.000000}%
\pgfsetstrokecolor{textcolor}%
\pgfsetfillcolor{textcolor}%
\pgftext[x=0.289968in, y=3.154403in, left, base]{\color{textcolor}\sffamily\fontsize{10.000000}{12.000000}\selectfont 0.45}%
\end{pgfscope}%
\begin{pgfscope}%
\pgfsetbuttcap%
\pgfsetroundjoin%
\definecolor{currentfill}{rgb}{0.000000,0.000000,0.000000}%
\pgfsetfillcolor{currentfill}%
\pgfsetlinewidth{0.803000pt}%
\definecolor{currentstroke}{rgb}{0.000000,0.000000,0.000000}%
\pgfsetstrokecolor{currentstroke}%
\pgfsetdash{}{0pt}%
\pgfsys@defobject{currentmarker}{\pgfqpoint{-0.048611in}{0.000000in}}{\pgfqpoint{-0.000000in}{0.000000in}}{%
\pgfpathmoveto{\pgfqpoint{-0.000000in}{0.000000in}}%
\pgfpathlineto{\pgfqpoint{-0.048611in}{0.000000in}}%
\pgfusepath{stroke,fill}%
}%
\begin{pgfscope}%
\pgfsys@transformshift{0.696435in}{3.823165in}%
\pgfsys@useobject{currentmarker}{}%
\end{pgfscope}%
\end{pgfscope}%
\begin{pgfscope}%
\definecolor{textcolor}{rgb}{0.000000,0.000000,0.000000}%
\pgfsetstrokecolor{textcolor}%
\pgfsetfillcolor{textcolor}%
\pgftext[x=0.289968in, y=3.770403in, left, base]{\color{textcolor}\sffamily\fontsize{10.000000}{12.000000}\selectfont 0.50}%
\end{pgfscope}%
\begin{pgfscope}%
\pgfsetbuttcap%
\pgfsetroundjoin%
\definecolor{currentfill}{rgb}{0.000000,0.000000,0.000000}%
\pgfsetfillcolor{currentfill}%
\pgfsetlinewidth{0.803000pt}%
\definecolor{currentstroke}{rgb}{0.000000,0.000000,0.000000}%
\pgfsetstrokecolor{currentstroke}%
\pgfsetdash{}{0pt}%
\pgfsys@defobject{currentmarker}{\pgfqpoint{-0.048611in}{0.000000in}}{\pgfqpoint{-0.000000in}{0.000000in}}{%
\pgfpathmoveto{\pgfqpoint{-0.000000in}{0.000000in}}%
\pgfpathlineto{\pgfqpoint{-0.048611in}{0.000000in}}%
\pgfusepath{stroke,fill}%
}%
\begin{pgfscope}%
\pgfsys@transformshift{0.696435in}{4.439165in}%
\pgfsys@useobject{currentmarker}{}%
\end{pgfscope}%
\end{pgfscope}%
\begin{pgfscope}%
\definecolor{textcolor}{rgb}{0.000000,0.000000,0.000000}%
\pgfsetstrokecolor{textcolor}%
\pgfsetfillcolor{textcolor}%
\pgftext[x=0.289968in, y=4.386403in, left, base]{\color{textcolor}\sffamily\fontsize{10.000000}{12.000000}\selectfont 0.55}%
\end{pgfscope}%
\begin{pgfscope}%
\pgfsetbuttcap%
\pgfsetroundjoin%
\definecolor{currentfill}{rgb}{0.000000,0.000000,0.000000}%
\pgfsetfillcolor{currentfill}%
\pgfsetlinewidth{0.803000pt}%
\definecolor{currentstroke}{rgb}{0.000000,0.000000,0.000000}%
\pgfsetstrokecolor{currentstroke}%
\pgfsetdash{}{0pt}%
\pgfsys@defobject{currentmarker}{\pgfqpoint{-0.048611in}{0.000000in}}{\pgfqpoint{-0.000000in}{0.000000in}}{%
\pgfpathmoveto{\pgfqpoint{-0.000000in}{0.000000in}}%
\pgfpathlineto{\pgfqpoint{-0.048611in}{0.000000in}}%
\pgfusepath{stroke,fill}%
}%
\begin{pgfscope}%
\pgfsys@transformshift{0.696435in}{5.055165in}%
\pgfsys@useobject{currentmarker}{}%
\end{pgfscope}%
\end{pgfscope}%
\begin{pgfscope}%
\definecolor{textcolor}{rgb}{0.000000,0.000000,0.000000}%
\pgfsetstrokecolor{textcolor}%
\pgfsetfillcolor{textcolor}%
\pgftext[x=0.289968in, y=5.002403in, left, base]{\color{textcolor}\sffamily\fontsize{10.000000}{12.000000}\selectfont 0.60}%
\end{pgfscope}%
\begin{pgfscope}%
\definecolor{textcolor}{rgb}{0.000000,0.000000,0.000000}%
\pgfsetstrokecolor{textcolor}%
\pgfsetfillcolor{textcolor}%
\pgftext[x=0.234413in,y=3.207165in,,bottom,rotate=90.000000]{\color{textcolor}\sffamily\fontsize{10.000000}{12.000000}\selectfont F1 Score}%
\end{pgfscope}%
\begin{pgfscope}%
\pgfpathrectangle{\pgfqpoint{0.696435in}{1.359165in}}{\pgfqpoint{4.960000in}{3.696000in}}%
\pgfusepath{clip}%
\pgfsetrectcap%
\pgfsetroundjoin%
\pgfsetlinewidth{1.505625pt}%
\definecolor{currentstroke}{rgb}{0.121569,0.466667,0.705882}%
\pgfsetstrokecolor{currentstroke}%
\pgfsetdash{}{0pt}%
\pgfpathmoveto{\pgfqpoint{0.921890in}{3.265069in}}%
\pgfpathlineto{\pgfqpoint{1.673405in}{3.674093in}}%
\pgfpathlineto{\pgfqpoint{2.424920in}{3.602637in}}%
\pgfpathlineto{\pgfqpoint{3.176435in}{3.417837in}}%
\pgfpathlineto{\pgfqpoint{3.927951in}{3.045773in}}%
\pgfpathlineto{\pgfqpoint{4.679466in}{3.039613in}}%
\pgfpathlineto{\pgfqpoint{5.430981in}{3.776349in}}%
\pgfusepath{stroke}%
\end{pgfscope}%
\begin{pgfscope}%
\pgfpathrectangle{\pgfqpoint{0.696435in}{1.359165in}}{\pgfqpoint{4.960000in}{3.696000in}}%
\pgfusepath{clip}%
\pgfsetbuttcap%
\pgfsetroundjoin%
\definecolor{currentfill}{rgb}{0.121569,0.466667,0.705882}%
\pgfsetfillcolor{currentfill}%
\pgfsetlinewidth{1.003750pt}%
\definecolor{currentstroke}{rgb}{0.121569,0.466667,0.705882}%
\pgfsetstrokecolor{currentstroke}%
\pgfsetdash{}{0pt}%
\pgfsys@defobject{currentmarker}{\pgfqpoint{-0.041667in}{-0.041667in}}{\pgfqpoint{0.041667in}{0.041667in}}{%
\pgfpathmoveto{\pgfqpoint{0.000000in}{-0.041667in}}%
\pgfpathcurveto{\pgfqpoint{0.011050in}{-0.041667in}}{\pgfqpoint{0.021649in}{-0.037276in}}{\pgfqpoint{0.029463in}{-0.029463in}}%
\pgfpathcurveto{\pgfqpoint{0.037276in}{-0.021649in}}{\pgfqpoint{0.041667in}{-0.011050in}}{\pgfqpoint{0.041667in}{0.000000in}}%
\pgfpathcurveto{\pgfqpoint{0.041667in}{0.011050in}}{\pgfqpoint{0.037276in}{0.021649in}}{\pgfqpoint{0.029463in}{0.029463in}}%
\pgfpathcurveto{\pgfqpoint{0.021649in}{0.037276in}}{\pgfqpoint{0.011050in}{0.041667in}}{\pgfqpoint{0.000000in}{0.041667in}}%
\pgfpathcurveto{\pgfqpoint{-0.011050in}{0.041667in}}{\pgfqpoint{-0.021649in}{0.037276in}}{\pgfqpoint{-0.029463in}{0.029463in}}%
\pgfpathcurveto{\pgfqpoint{-0.037276in}{0.021649in}}{\pgfqpoint{-0.041667in}{0.011050in}}{\pgfqpoint{-0.041667in}{0.000000in}}%
\pgfpathcurveto{\pgfqpoint{-0.041667in}{-0.011050in}}{\pgfqpoint{-0.037276in}{-0.021649in}}{\pgfqpoint{-0.029463in}{-0.029463in}}%
\pgfpathcurveto{\pgfqpoint{-0.021649in}{-0.037276in}}{\pgfqpoint{-0.011050in}{-0.041667in}}{\pgfqpoint{0.000000in}{-0.041667in}}%
\pgfpathclose%
\pgfusepath{stroke,fill}%
}%
\begin{pgfscope}%
\pgfsys@transformshift{0.921890in}{3.265069in}%
\pgfsys@useobject{currentmarker}{}%
\end{pgfscope}%
\begin{pgfscope}%
\pgfsys@transformshift{1.673405in}{3.674093in}%
\pgfsys@useobject{currentmarker}{}%
\end{pgfscope}%
\begin{pgfscope}%
\pgfsys@transformshift{2.424920in}{3.602637in}%
\pgfsys@useobject{currentmarker}{}%
\end{pgfscope}%
\begin{pgfscope}%
\pgfsys@transformshift{3.176435in}{3.417837in}%
\pgfsys@useobject{currentmarker}{}%
\end{pgfscope}%
\begin{pgfscope}%
\pgfsys@transformshift{3.927951in}{3.045773in}%
\pgfsys@useobject{currentmarker}{}%
\end{pgfscope}%
\begin{pgfscope}%
\pgfsys@transformshift{4.679466in}{3.039613in}%
\pgfsys@useobject{currentmarker}{}%
\end{pgfscope}%
\begin{pgfscope}%
\pgfsys@transformshift{5.430981in}{3.776349in}%
\pgfsys@useobject{currentmarker}{}%
\end{pgfscope}%
\end{pgfscope}%
\begin{pgfscope}%
\pgfpathrectangle{\pgfqpoint{0.696435in}{1.359165in}}{\pgfqpoint{4.960000in}{3.696000in}}%
\pgfusepath{clip}%
\pgfsetrectcap%
\pgfsetroundjoin%
\pgfsetlinewidth{1.505625pt}%
\definecolor{currentstroke}{rgb}{1.000000,0.498039,0.054902}%
\pgfsetstrokecolor{currentstroke}%
\pgfsetdash{}{0pt}%
\pgfpathmoveto{\pgfqpoint{0.921890in}{2.806765in}}%
\pgfpathlineto{\pgfqpoint{1.673405in}{3.403053in}}%
\pgfpathlineto{\pgfqpoint{2.424920in}{3.380877in}}%
\pgfpathlineto{\pgfqpoint{3.176435in}{3.419069in}}%
\pgfpathlineto{\pgfqpoint{3.927951in}{3.266301in}}%
\pgfpathlineto{\pgfqpoint{4.679466in}{3.410445in}}%
\pgfpathlineto{\pgfqpoint{5.430981in}{3.591549in}}%
\pgfusepath{stroke}%
\end{pgfscope}%
\begin{pgfscope}%
\pgfpathrectangle{\pgfqpoint{0.696435in}{1.359165in}}{\pgfqpoint{4.960000in}{3.696000in}}%
\pgfusepath{clip}%
\pgfsetbuttcap%
\pgfsetmiterjoin%
\definecolor{currentfill}{rgb}{1.000000,0.498039,0.054902}%
\pgfsetfillcolor{currentfill}%
\pgfsetlinewidth{1.003750pt}%
\definecolor{currentstroke}{rgb}{1.000000,0.498039,0.054902}%
\pgfsetstrokecolor{currentstroke}%
\pgfsetdash{}{0pt}%
\pgfsys@defobject{currentmarker}{\pgfqpoint{-0.041667in}{-0.041667in}}{\pgfqpoint{0.041667in}{0.041667in}}{%
\pgfpathmoveto{\pgfqpoint{-0.000000in}{-0.041667in}}%
\pgfpathlineto{\pgfqpoint{0.041667in}{0.041667in}}%
\pgfpathlineto{\pgfqpoint{-0.041667in}{0.041667in}}%
\pgfpathclose%
\pgfusepath{stroke,fill}%
}%
\begin{pgfscope}%
\pgfsys@transformshift{0.921890in}{2.806765in}%
\pgfsys@useobject{currentmarker}{}%
\end{pgfscope}%
\begin{pgfscope}%
\pgfsys@transformshift{1.673405in}{3.403053in}%
\pgfsys@useobject{currentmarker}{}%
\end{pgfscope}%
\begin{pgfscope}%
\pgfsys@transformshift{2.424920in}{3.380877in}%
\pgfsys@useobject{currentmarker}{}%
\end{pgfscope}%
\begin{pgfscope}%
\pgfsys@transformshift{3.176435in}{3.419069in}%
\pgfsys@useobject{currentmarker}{}%
\end{pgfscope}%
\begin{pgfscope}%
\pgfsys@transformshift{3.927951in}{3.266301in}%
\pgfsys@useobject{currentmarker}{}%
\end{pgfscope}%
\begin{pgfscope}%
\pgfsys@transformshift{4.679466in}{3.410445in}%
\pgfsys@useobject{currentmarker}{}%
\end{pgfscope}%
\begin{pgfscope}%
\pgfsys@transformshift{5.430981in}{3.591549in}%
\pgfsys@useobject{currentmarker}{}%
\end{pgfscope}%
\end{pgfscope}%
\begin{pgfscope}%
\pgfsetrectcap%
\pgfsetmiterjoin%
\pgfsetlinewidth{0.803000pt}%
\definecolor{currentstroke}{rgb}{0.000000,0.000000,0.000000}%
\pgfsetstrokecolor{currentstroke}%
\pgfsetdash{}{0pt}%
\pgfpathmoveto{\pgfqpoint{0.696435in}{1.359165in}}%
\pgfpathlineto{\pgfqpoint{0.696435in}{5.055165in}}%
\pgfusepath{stroke}%
\end{pgfscope}%
\begin{pgfscope}%
\pgfsetrectcap%
\pgfsetmiterjoin%
\pgfsetlinewidth{0.803000pt}%
\definecolor{currentstroke}{rgb}{0.000000,0.000000,0.000000}%
\pgfsetstrokecolor{currentstroke}%
\pgfsetdash{}{0pt}%
\pgfpathmoveto{\pgfqpoint{5.656435in}{1.359165in}}%
\pgfpathlineto{\pgfqpoint{5.656435in}{5.055165in}}%
\pgfusepath{stroke}%
\end{pgfscope}%
\begin{pgfscope}%
\pgfsetrectcap%
\pgfsetmiterjoin%
\pgfsetlinewidth{0.803000pt}%
\definecolor{currentstroke}{rgb}{0.000000,0.000000,0.000000}%
\pgfsetstrokecolor{currentstroke}%
\pgfsetdash{}{0pt}%
\pgfpathmoveto{\pgfqpoint{0.696435in}{1.359165in}}%
\pgfpathlineto{\pgfqpoint{5.656435in}{1.359165in}}%
\pgfusepath{stroke}%
\end{pgfscope}%
\begin{pgfscope}%
\pgfsetrectcap%
\pgfsetmiterjoin%
\pgfsetlinewidth{0.803000pt}%
\definecolor{currentstroke}{rgb}{0.000000,0.000000,0.000000}%
\pgfsetstrokecolor{currentstroke}%
\pgfsetdash{}{0pt}%
\pgfpathmoveto{\pgfqpoint{0.696435in}{5.055165in}}%
\pgfpathlineto{\pgfqpoint{5.656435in}{5.055165in}}%
\pgfusepath{stroke}%
\end{pgfscope}%
\begin{pgfscope}%
\pgfsetbuttcap%
\pgfsetmiterjoin%
\definecolor{currentfill}{rgb}{1.000000,1.000000,1.000000}%
\pgfsetfillcolor{currentfill}%
\pgfsetfillopacity{0.800000}%
\pgfsetlinewidth{1.003750pt}%
\definecolor{currentstroke}{rgb}{0.800000,0.800000,0.800000}%
\pgfsetstrokecolor{currentstroke}%
\pgfsetstrokeopacity{0.800000}%
\pgfsetdash{}{0pt}%
\pgfpathmoveto{\pgfqpoint{4.345048in}{4.536339in}}%
\pgfpathlineto{\pgfqpoint{5.559213in}{4.536339in}}%
\pgfpathquadraticcurveto{\pgfqpoint{5.586991in}{4.536339in}}{\pgfqpoint{5.586991in}{4.564117in}}%
\pgfpathlineto{\pgfqpoint{5.586991in}{4.957943in}}%
\pgfpathquadraticcurveto{\pgfqpoint{5.586991in}{4.985720in}}{\pgfqpoint{5.559213in}{4.985720in}}%
\pgfpathlineto{\pgfqpoint{4.345048in}{4.985720in}}%
\pgfpathquadraticcurveto{\pgfqpoint{4.317270in}{4.985720in}}{\pgfqpoint{4.317270in}{4.957943in}}%
\pgfpathlineto{\pgfqpoint{4.317270in}{4.564117in}}%
\pgfpathquadraticcurveto{\pgfqpoint{4.317270in}{4.536339in}}{\pgfqpoint{4.345048in}{4.536339in}}%
\pgfpathclose%
\pgfusepath{stroke,fill}%
\end{pgfscope}%
\begin{pgfscope}%
\pgfsetrectcap%
\pgfsetroundjoin%
\pgfsetlinewidth{1.505625pt}%
\definecolor{currentstroke}{rgb}{0.121569,0.466667,0.705882}%
\pgfsetstrokecolor{currentstroke}%
\pgfsetdash{}{0pt}%
\pgfpathmoveto{\pgfqpoint{4.372825in}{4.873253in}}%
\pgfpathlineto{\pgfqpoint{4.650603in}{4.873253in}}%
\pgfusepath{stroke}%
\end{pgfscope}%
\begin{pgfscope}%
\pgfsetbuttcap%
\pgfsetroundjoin%
\definecolor{currentfill}{rgb}{0.121569,0.466667,0.705882}%
\pgfsetfillcolor{currentfill}%
\pgfsetlinewidth{1.003750pt}%
\definecolor{currentstroke}{rgb}{0.121569,0.466667,0.705882}%
\pgfsetstrokecolor{currentstroke}%
\pgfsetdash{}{0pt}%
\pgfsys@defobject{currentmarker}{\pgfqpoint{-0.041667in}{-0.041667in}}{\pgfqpoint{0.041667in}{0.041667in}}{%
\pgfpathmoveto{\pgfqpoint{0.000000in}{-0.041667in}}%
\pgfpathcurveto{\pgfqpoint{0.011050in}{-0.041667in}}{\pgfqpoint{0.021649in}{-0.037276in}}{\pgfqpoint{0.029463in}{-0.029463in}}%
\pgfpathcurveto{\pgfqpoint{0.037276in}{-0.021649in}}{\pgfqpoint{0.041667in}{-0.011050in}}{\pgfqpoint{0.041667in}{0.000000in}}%
\pgfpathcurveto{\pgfqpoint{0.041667in}{0.011050in}}{\pgfqpoint{0.037276in}{0.021649in}}{\pgfqpoint{0.029463in}{0.029463in}}%
\pgfpathcurveto{\pgfqpoint{0.021649in}{0.037276in}}{\pgfqpoint{0.011050in}{0.041667in}}{\pgfqpoint{0.000000in}{0.041667in}}%
\pgfpathcurveto{\pgfqpoint{-0.011050in}{0.041667in}}{\pgfqpoint{-0.021649in}{0.037276in}}{\pgfqpoint{-0.029463in}{0.029463in}}%
\pgfpathcurveto{\pgfqpoint{-0.037276in}{0.021649in}}{\pgfqpoint{-0.041667in}{0.011050in}}{\pgfqpoint{-0.041667in}{0.000000in}}%
\pgfpathcurveto{\pgfqpoint{-0.041667in}{-0.011050in}}{\pgfqpoint{-0.037276in}{-0.021649in}}{\pgfqpoint{-0.029463in}{-0.029463in}}%
\pgfpathcurveto{\pgfqpoint{-0.021649in}{-0.037276in}}{\pgfqpoint{-0.011050in}{-0.041667in}}{\pgfqpoint{0.000000in}{-0.041667in}}%
\pgfpathclose%
\pgfusepath{stroke,fill}%
}%
\begin{pgfscope}%
\pgfsys@transformshift{4.511714in}{4.873253in}%
\pgfsys@useobject{currentmarker}{}%
\end{pgfscope}%
\end{pgfscope}%
\begin{pgfscope}%
\definecolor{textcolor}{rgb}{0.000000,0.000000,0.000000}%
\pgfsetstrokecolor{textcolor}%
\pgfsetfillcolor{textcolor}%
\pgftext[x=4.761714in,y=4.824642in,left,base]{\color{textcolor}\sffamily\fontsize{10.000000}{12.000000}\selectfont Pix2Vox++}%
\end{pgfscope}%
\begin{pgfscope}%
\pgfsetrectcap%
\pgfsetroundjoin%
\pgfsetlinewidth{1.505625pt}%
\definecolor{currentstroke}{rgb}{1.000000,0.498039,0.054902}%
\pgfsetstrokecolor{currentstroke}%
\pgfsetdash{}{0pt}%
\pgfpathmoveto{\pgfqpoint{4.372825in}{4.669396in}}%
\pgfpathlineto{\pgfqpoint{4.650603in}{4.669396in}}%
\pgfusepath{stroke}%
\end{pgfscope}%
\begin{pgfscope}%
\pgfsetbuttcap%
\pgfsetmiterjoin%
\definecolor{currentfill}{rgb}{1.000000,0.498039,0.054902}%
\pgfsetfillcolor{currentfill}%
\pgfsetlinewidth{1.003750pt}%
\definecolor{currentstroke}{rgb}{1.000000,0.498039,0.054902}%
\pgfsetstrokecolor{currentstroke}%
\pgfsetdash{}{0pt}%
\pgfsys@defobject{currentmarker}{\pgfqpoint{-0.041667in}{-0.041667in}}{\pgfqpoint{0.041667in}{0.041667in}}{%
\pgfpathmoveto{\pgfqpoint{-0.000000in}{-0.041667in}}%
\pgfpathlineto{\pgfqpoint{0.041667in}{0.041667in}}%
\pgfpathlineto{\pgfqpoint{-0.041667in}{0.041667in}}%
\pgfpathclose%
\pgfusepath{stroke,fill}%
}%
\begin{pgfscope}%
\pgfsys@transformshift{4.511714in}{4.669396in}%
\pgfsys@useobject{currentmarker}{}%
\end{pgfscope}%
\end{pgfscope}%
\begin{pgfscope}%
\definecolor{textcolor}{rgb}{0.000000,0.000000,0.000000}%
\pgfsetstrokecolor{textcolor}%
\pgfsetfillcolor{textcolor}%
\pgftext[x=4.761714in,y=4.620785in,left,base]{\color{textcolor}\sffamily\fontsize{10.000000}{12.000000}\selectfont Pix2Vox}%
\end{pgfscope}%
\end{pgfpicture}%
\makeatother%
\endgroup%
}
%    \caption{Line plot for the \gls{f1}  for baseline trained by mixing chair dataset from real and synthetic dataset with ratio of 50\%.
%    Observe that the \fls{IoU} is consistent for all types of randomization proving that mixed training negates loss from randomization.}
%    \label{fig:ablation_dice2}
%\end{figure}
\begin{figure}[ht]
    \centering
    \resizebox{0.7\textwidth}{!}{%% Creator: Matplotlib, PGF backend
%%
%% To include the figure in your LaTeX document, write
%%   \input{<filename>.pgf}
%%
%% Make sure the required packages are loaded in your preamble
%%   \usepackage{pgf}
%%
%% Figures using additional raster images can only be included by \input if
%% they are in the same directory as the main LaTeX file. For loading figures
%% from other directories you can use the `import` package
%%   \usepackage{import}
%%
%% and then include the figures with
%%   \import{<path to file>}{<filename>.pgf}
%%
%% Matplotlib used the following preamble
%%   \usepackage{fontspec}
%%   \setmainfont{DejaVuSerif.ttf}[Path=\detokenize{/Users/apple/opt/anaconda3/envs/kaolin/lib/python3.7/site-packages/matplotlib/mpl-data/fonts/ttf/}]
%%   \setsansfont{DejaVuSans.ttf}[Path=\detokenize{/Users/apple/opt/anaconda3/envs/kaolin/lib/python3.7/site-packages/matplotlib/mpl-data/fonts/ttf/}]
%%   \setmonofont{DejaVuSansMono.ttf}[Path=\detokenize{/Users/apple/opt/anaconda3/envs/kaolin/lib/python3.7/site-packages/matplotlib/mpl-data/fonts/ttf/}]
%%
\begingroup%
\makeatletter%
\begin{pgfpicture}%
\pgfpathrectangle{\pgfpointorigin}{\pgfqpoint{6.293658in}{4.696324in}}%
\pgfusepath{use as bounding box, clip}%
\begin{pgfscope}%
\pgfsetbuttcap%
\pgfsetmiterjoin%
\definecolor{currentfill}{rgb}{1.000000,1.000000,1.000000}%
\pgfsetfillcolor{currentfill}%
\pgfsetlinewidth{0.000000pt}%
\definecolor{currentstroke}{rgb}{1.000000,1.000000,1.000000}%
\pgfsetstrokecolor{currentstroke}%
\pgfsetdash{}{0pt}%
\pgfpathmoveto{\pgfqpoint{0.000000in}{0.000000in}}%
\pgfpathlineto{\pgfqpoint{6.293658in}{0.000000in}}%
\pgfpathlineto{\pgfqpoint{6.293658in}{4.696324in}}%
\pgfpathlineto{\pgfqpoint{0.000000in}{4.696324in}}%
\pgfpathclose%
\pgfusepath{fill}%
\end{pgfscope}%
\begin{pgfscope}%
\pgfsetbuttcap%
\pgfsetmiterjoin%
\definecolor{currentfill}{rgb}{1.000000,1.000000,1.000000}%
\pgfsetfillcolor{currentfill}%
\pgfsetlinewidth{0.000000pt}%
\definecolor{currentstroke}{rgb}{0.000000,0.000000,0.000000}%
\pgfsetstrokecolor{currentstroke}%
\pgfsetstrokeopacity{0.000000}%
\pgfsetdash{}{0pt}%
\pgfpathmoveto{\pgfqpoint{0.696435in}{1.359165in}}%
\pgfpathlineto{\pgfqpoint{6.193658in}{1.359165in}}%
\pgfpathlineto{\pgfqpoint{6.193658in}{4.386363in}}%
\pgfpathlineto{\pgfqpoint{0.696435in}{4.386363in}}%
\pgfpathclose%
\pgfusepath{fill}%
\end{pgfscope}%
\begin{pgfscope}%
\pgfpathrectangle{\pgfqpoint{0.696435in}{1.359165in}}{\pgfqpoint{5.497222in}{3.027198in}}%
\pgfusepath{clip}%
\pgfsetbuttcap%
\pgfsetmiterjoin%
\definecolor{currentfill}{rgb}{0.121569,0.466667,0.705882}%
\pgfsetfillcolor{currentfill}%
\pgfsetlinewidth{0.000000pt}%
\definecolor{currentstroke}{rgb}{0.000000,0.000000,0.000000}%
\pgfsetstrokecolor{currentstroke}%
\pgfsetstrokeopacity{0.000000}%
\pgfsetdash{}{0pt}%
\pgfpathmoveto{\pgfqpoint{0.946309in}{-1.668033in}}%
\pgfpathlineto{\pgfqpoint{1.272231in}{-1.668033in}}%
\pgfpathlineto{\pgfqpoint{1.272231in}{2.920190in}}%
\pgfpathlineto{\pgfqpoint{0.946309in}{2.920190in}}%
\pgfpathclose%
\pgfusepath{fill}%
\end{pgfscope}%
\begin{pgfscope}%
\pgfpathrectangle{\pgfqpoint{0.696435in}{1.359165in}}{\pgfqpoint{5.497222in}{3.027198in}}%
\pgfusepath{clip}%
\pgfsetbuttcap%
\pgfsetmiterjoin%
\definecolor{currentfill}{rgb}{0.121569,0.466667,0.705882}%
\pgfsetfillcolor{currentfill}%
\pgfsetlinewidth{0.000000pt}%
\definecolor{currentstroke}{rgb}{0.000000,0.000000,0.000000}%
\pgfsetstrokecolor{currentstroke}%
\pgfsetstrokeopacity{0.000000}%
\pgfsetdash{}{0pt}%
\pgfpathmoveto{\pgfqpoint{1.670581in}{-1.668033in}}%
\pgfpathlineto{\pgfqpoint{1.996503in}{-1.668033in}}%
\pgfpathlineto{\pgfqpoint{1.996503in}{3.255200in}}%
\pgfpathlineto{\pgfqpoint{1.670581in}{3.255200in}}%
\pgfpathclose%
\pgfusepath{fill}%
\end{pgfscope}%
\begin{pgfscope}%
\pgfpathrectangle{\pgfqpoint{0.696435in}{1.359165in}}{\pgfqpoint{5.497222in}{3.027198in}}%
\pgfusepath{clip}%
\pgfsetbuttcap%
\pgfsetmiterjoin%
\definecolor{currentfill}{rgb}{0.121569,0.466667,0.705882}%
\pgfsetfillcolor{currentfill}%
\pgfsetlinewidth{0.000000pt}%
\definecolor{currentstroke}{rgb}{0.000000,0.000000,0.000000}%
\pgfsetstrokecolor{currentstroke}%
\pgfsetstrokeopacity{0.000000}%
\pgfsetdash{}{0pt}%
\pgfpathmoveto{\pgfqpoint{2.394853in}{-1.668033in}}%
\pgfpathlineto{\pgfqpoint{2.720775in}{-1.668033in}}%
\pgfpathlineto{\pgfqpoint{2.720775in}{3.196674in}}%
\pgfpathlineto{\pgfqpoint{2.394853in}{3.196674in}}%
\pgfpathclose%
\pgfusepath{fill}%
\end{pgfscope}%
\begin{pgfscope}%
\pgfpathrectangle{\pgfqpoint{0.696435in}{1.359165in}}{\pgfqpoint{5.497222in}{3.027198in}}%
\pgfusepath{clip}%
\pgfsetbuttcap%
\pgfsetmiterjoin%
\definecolor{currentfill}{rgb}{0.121569,0.466667,0.705882}%
\pgfsetfillcolor{currentfill}%
\pgfsetlinewidth{0.000000pt}%
\definecolor{currentstroke}{rgb}{0.000000,0.000000,0.000000}%
\pgfsetstrokecolor{currentstroke}%
\pgfsetstrokeopacity{0.000000}%
\pgfsetdash{}{0pt}%
\pgfpathmoveto{\pgfqpoint{3.119124in}{-1.668033in}}%
\pgfpathlineto{\pgfqpoint{3.445047in}{-1.668033in}}%
\pgfpathlineto{\pgfqpoint{3.445047in}{3.045314in}}%
\pgfpathlineto{\pgfqpoint{3.119124in}{3.045314in}}%
\pgfpathclose%
\pgfusepath{fill}%
\end{pgfscope}%
\begin{pgfscope}%
\pgfpathrectangle{\pgfqpoint{0.696435in}{1.359165in}}{\pgfqpoint{5.497222in}{3.027198in}}%
\pgfusepath{clip}%
\pgfsetbuttcap%
\pgfsetmiterjoin%
\definecolor{currentfill}{rgb}{0.121569,0.466667,0.705882}%
\pgfsetfillcolor{currentfill}%
\pgfsetlinewidth{0.000000pt}%
\definecolor{currentstroke}{rgb}{0.000000,0.000000,0.000000}%
\pgfsetstrokecolor{currentstroke}%
\pgfsetstrokeopacity{0.000000}%
\pgfsetdash{}{0pt}%
\pgfpathmoveto{\pgfqpoint{3.843396in}{-1.668033in}}%
\pgfpathlineto{\pgfqpoint{4.169318in}{-1.668033in}}%
\pgfpathlineto{\pgfqpoint{4.169318in}{2.740576in}}%
\pgfpathlineto{\pgfqpoint{3.843396in}{2.740576in}}%
\pgfpathclose%
\pgfusepath{fill}%
\end{pgfscope}%
\begin{pgfscope}%
\pgfpathrectangle{\pgfqpoint{0.696435in}{1.359165in}}{\pgfqpoint{5.497222in}{3.027198in}}%
\pgfusepath{clip}%
\pgfsetbuttcap%
\pgfsetmiterjoin%
\definecolor{currentfill}{rgb}{0.121569,0.466667,0.705882}%
\pgfsetfillcolor{currentfill}%
\pgfsetlinewidth{0.000000pt}%
\definecolor{currentstroke}{rgb}{0.000000,0.000000,0.000000}%
\pgfsetstrokecolor{currentstroke}%
\pgfsetstrokeopacity{0.000000}%
\pgfsetdash{}{0pt}%
\pgfpathmoveto{\pgfqpoint{4.567668in}{-1.668033in}}%
\pgfpathlineto{\pgfqpoint{4.893590in}{-1.668033in}}%
\pgfpathlineto{\pgfqpoint{4.893590in}{2.735531in}}%
\pgfpathlineto{\pgfqpoint{4.567668in}{2.735531in}}%
\pgfpathclose%
\pgfusepath{fill}%
\end{pgfscope}%
\begin{pgfscope}%
\pgfpathrectangle{\pgfqpoint{0.696435in}{1.359165in}}{\pgfqpoint{5.497222in}{3.027198in}}%
\pgfusepath{clip}%
\pgfsetbuttcap%
\pgfsetmiterjoin%
\definecolor{currentfill}{rgb}{0.121569,0.466667,0.705882}%
\pgfsetfillcolor{currentfill}%
\pgfsetlinewidth{0.000000pt}%
\definecolor{currentstroke}{rgb}{0.000000,0.000000,0.000000}%
\pgfsetstrokecolor{currentstroke}%
\pgfsetstrokeopacity{0.000000}%
\pgfsetdash{}{0pt}%
\pgfpathmoveto{\pgfqpoint{5.291939in}{-1.668033in}}%
\pgfpathlineto{\pgfqpoint{5.617862in}{-1.668033in}}%
\pgfpathlineto{\pgfqpoint{5.617862in}{3.338952in}}%
\pgfpathlineto{\pgfqpoint{5.291939in}{3.338952in}}%
\pgfpathclose%
\pgfusepath{fill}%
\end{pgfscope}%
\begin{pgfscope}%
\pgfpathrectangle{\pgfqpoint{0.696435in}{1.359165in}}{\pgfqpoint{5.497222in}{3.027198in}}%
\pgfusepath{clip}%
\pgfsetbuttcap%
\pgfsetmiterjoin%
\definecolor{currentfill}{rgb}{1.000000,0.498039,0.054902}%
\pgfsetfillcolor{currentfill}%
\pgfsetlinewidth{0.000000pt}%
\definecolor{currentstroke}{rgb}{0.000000,0.000000,0.000000}%
\pgfsetstrokecolor{currentstroke}%
\pgfsetstrokeopacity{0.000000}%
\pgfsetdash{}{0pt}%
\pgfpathmoveto{\pgfqpoint{1.272231in}{-1.668033in}}%
\pgfpathlineto{\pgfqpoint{1.598154in}{-1.668033in}}%
\pgfpathlineto{\pgfqpoint{1.598154in}{2.544817in}}%
\pgfpathlineto{\pgfqpoint{1.272231in}{2.544817in}}%
\pgfpathclose%
\pgfusepath{fill}%
\end{pgfscope}%
\begin{pgfscope}%
\pgfpathrectangle{\pgfqpoint{0.696435in}{1.359165in}}{\pgfqpoint{5.497222in}{3.027198in}}%
\pgfusepath{clip}%
\pgfsetbuttcap%
\pgfsetmiterjoin%
\definecolor{currentfill}{rgb}{1.000000,0.498039,0.054902}%
\pgfsetfillcolor{currentfill}%
\pgfsetlinewidth{0.000000pt}%
\definecolor{currentstroke}{rgb}{0.000000,0.000000,0.000000}%
\pgfsetstrokecolor{currentstroke}%
\pgfsetstrokeopacity{0.000000}%
\pgfsetdash{}{0pt}%
\pgfpathmoveto{\pgfqpoint{1.996503in}{-1.668033in}}%
\pgfpathlineto{\pgfqpoint{2.322425in}{-1.668033in}}%
\pgfpathlineto{\pgfqpoint{2.322425in}{3.033205in}}%
\pgfpathlineto{\pgfqpoint{1.996503in}{3.033205in}}%
\pgfpathclose%
\pgfusepath{fill}%
\end{pgfscope}%
\begin{pgfscope}%
\pgfpathrectangle{\pgfqpoint{0.696435in}{1.359165in}}{\pgfqpoint{5.497222in}{3.027198in}}%
\pgfusepath{clip}%
\pgfsetbuttcap%
\pgfsetmiterjoin%
\definecolor{currentfill}{rgb}{1.000000,0.498039,0.054902}%
\pgfsetfillcolor{currentfill}%
\pgfsetlinewidth{0.000000pt}%
\definecolor{currentstroke}{rgb}{0.000000,0.000000,0.000000}%
\pgfsetstrokecolor{currentstroke}%
\pgfsetstrokeopacity{0.000000}%
\pgfsetdash{}{0pt}%
\pgfpathmoveto{\pgfqpoint{2.720775in}{-1.668033in}}%
\pgfpathlineto{\pgfqpoint{3.046697in}{-1.668033in}}%
\pgfpathlineto{\pgfqpoint{3.046697in}{3.015042in}}%
\pgfpathlineto{\pgfqpoint{2.720775in}{3.015042in}}%
\pgfpathclose%
\pgfusepath{fill}%
\end{pgfscope}%
\begin{pgfscope}%
\pgfpathrectangle{\pgfqpoint{0.696435in}{1.359165in}}{\pgfqpoint{5.497222in}{3.027198in}}%
\pgfusepath{clip}%
\pgfsetbuttcap%
\pgfsetmiterjoin%
\definecolor{currentfill}{rgb}{1.000000,0.498039,0.054902}%
\pgfsetfillcolor{currentfill}%
\pgfsetlinewidth{0.000000pt}%
\definecolor{currentstroke}{rgb}{0.000000,0.000000,0.000000}%
\pgfsetstrokecolor{currentstroke}%
\pgfsetstrokeopacity{0.000000}%
\pgfsetdash{}{0pt}%
\pgfpathmoveto{\pgfqpoint{3.445047in}{-1.668033in}}%
\pgfpathlineto{\pgfqpoint{3.770969in}{-1.668033in}}%
\pgfpathlineto{\pgfqpoint{3.770969in}{3.046323in}}%
\pgfpathlineto{\pgfqpoint{3.445047in}{3.046323in}}%
\pgfpathclose%
\pgfusepath{fill}%
\end{pgfscope}%
\begin{pgfscope}%
\pgfpathrectangle{\pgfqpoint{0.696435in}{1.359165in}}{\pgfqpoint{5.497222in}{3.027198in}}%
\pgfusepath{clip}%
\pgfsetbuttcap%
\pgfsetmiterjoin%
\definecolor{currentfill}{rgb}{1.000000,0.498039,0.054902}%
\pgfsetfillcolor{currentfill}%
\pgfsetlinewidth{0.000000pt}%
\definecolor{currentstroke}{rgb}{0.000000,0.000000,0.000000}%
\pgfsetstrokecolor{currentstroke}%
\pgfsetstrokeopacity{0.000000}%
\pgfsetdash{}{0pt}%
\pgfpathmoveto{\pgfqpoint{4.169318in}{-1.668033in}}%
\pgfpathlineto{\pgfqpoint{4.495240in}{-1.668033in}}%
\pgfpathlineto{\pgfqpoint{4.495240in}{2.921199in}}%
\pgfpathlineto{\pgfqpoint{4.169318in}{2.921199in}}%
\pgfpathclose%
\pgfusepath{fill}%
\end{pgfscope}%
\begin{pgfscope}%
\pgfpathrectangle{\pgfqpoint{0.696435in}{1.359165in}}{\pgfqpoint{5.497222in}{3.027198in}}%
\pgfusepath{clip}%
\pgfsetbuttcap%
\pgfsetmiterjoin%
\definecolor{currentfill}{rgb}{1.000000,0.498039,0.054902}%
\pgfsetfillcolor{currentfill}%
\pgfsetlinewidth{0.000000pt}%
\definecolor{currentstroke}{rgb}{0.000000,0.000000,0.000000}%
\pgfsetstrokecolor{currentstroke}%
\pgfsetstrokeopacity{0.000000}%
\pgfsetdash{}{0pt}%
\pgfpathmoveto{\pgfqpoint{4.893590in}{-1.668033in}}%
\pgfpathlineto{\pgfqpoint{5.219512in}{-1.668033in}}%
\pgfpathlineto{\pgfqpoint{5.219512in}{3.039260in}}%
\pgfpathlineto{\pgfqpoint{4.893590in}{3.039260in}}%
\pgfpathclose%
\pgfusepath{fill}%
\end{pgfscope}%
\begin{pgfscope}%
\pgfpathrectangle{\pgfqpoint{0.696435in}{1.359165in}}{\pgfqpoint{5.497222in}{3.027198in}}%
\pgfusepath{clip}%
\pgfsetbuttcap%
\pgfsetmiterjoin%
\definecolor{currentfill}{rgb}{1.000000,0.498039,0.054902}%
\pgfsetfillcolor{currentfill}%
\pgfsetlinewidth{0.000000pt}%
\definecolor{currentstroke}{rgb}{0.000000,0.000000,0.000000}%
\pgfsetstrokecolor{currentstroke}%
\pgfsetstrokeopacity{0.000000}%
\pgfsetdash{}{0pt}%
\pgfpathmoveto{\pgfqpoint{5.617862in}{-1.668033in}}%
\pgfpathlineto{\pgfqpoint{5.943784in}{-1.668033in}}%
\pgfpathlineto{\pgfqpoint{5.943784in}{3.187592in}}%
\pgfpathlineto{\pgfqpoint{5.617862in}{3.187592in}}%
\pgfpathclose%
\pgfusepath{fill}%
\end{pgfscope}%
\begin{pgfscope}%
\pgfsetbuttcap%
\pgfsetroundjoin%
\definecolor{currentfill}{rgb}{0.000000,0.000000,0.000000}%
\pgfsetfillcolor{currentfill}%
\pgfsetlinewidth{0.803000pt}%
\definecolor{currentstroke}{rgb}{0.000000,0.000000,0.000000}%
\pgfsetstrokecolor{currentstroke}%
\pgfsetdash{}{0pt}%
\pgfsys@defobject{currentmarker}{\pgfqpoint{0.000000in}{-0.048611in}}{\pgfqpoint{0.000000in}{0.000000in}}{%
\pgfpathmoveto{\pgfqpoint{0.000000in}{0.000000in}}%
\pgfpathlineto{\pgfqpoint{0.000000in}{-0.048611in}}%
\pgfusepath{stroke,fill}%
}%
\begin{pgfscope}%
\pgfsys@transformshift{1.272231in}{1.359165in}%
\pgfsys@useobject{currentmarker}{}%
\end{pgfscope}%
\end{pgfscope}%
\begin{pgfscope}%
\definecolor{textcolor}{rgb}{0.000000,0.000000,0.000000}%
\pgfsetstrokecolor{textcolor}%
\pgfsetfillcolor{textcolor}%
\pgftext[x=1.005321in, y=0.599318in, left, base,rotate=45.000000]{\color{textcolor}\sffamily\fontsize{10.000000}{12.000000}\selectfont Pix3d(chair)}%
\end{pgfscope}%
\begin{pgfscope}%
\pgfsetbuttcap%
\pgfsetroundjoin%
\definecolor{currentfill}{rgb}{0.000000,0.000000,0.000000}%
\pgfsetfillcolor{currentfill}%
\pgfsetlinewidth{0.803000pt}%
\definecolor{currentstroke}{rgb}{0.000000,0.000000,0.000000}%
\pgfsetstrokecolor{currentstroke}%
\pgfsetdash{}{0pt}%
\pgfsys@defobject{currentmarker}{\pgfqpoint{0.000000in}{-0.048611in}}{\pgfqpoint{0.000000in}{0.000000in}}{%
\pgfpathmoveto{\pgfqpoint{0.000000in}{0.000000in}}%
\pgfpathlineto{\pgfqpoint{0.000000in}{-0.048611in}}%
\pgfusepath{stroke,fill}%
}%
\begin{pgfscope}%
\pgfsys@transformshift{1.996503in}{1.359165in}%
\pgfsys@useobject{currentmarker}{}%
\end{pgfscope}%
\end{pgfscope}%
\begin{pgfscope}%
\definecolor{textcolor}{rgb}{0.000000,0.000000,0.000000}%
\pgfsetstrokecolor{textcolor}%
\pgfsetfillcolor{textcolor}%
\pgftext[x=1.748822in, y=0.637777in, left, base,rotate=45.000000]{\color{textcolor}\sffamily\fontsize{10.000000}{12.000000}\selectfont Textureless}%
\end{pgfscope}%
\begin{pgfscope}%
\pgfsetbuttcap%
\pgfsetroundjoin%
\definecolor{currentfill}{rgb}{0.000000,0.000000,0.000000}%
\pgfsetfillcolor{currentfill}%
\pgfsetlinewidth{0.803000pt}%
\definecolor{currentstroke}{rgb}{0.000000,0.000000,0.000000}%
\pgfsetstrokecolor{currentstroke}%
\pgfsetdash{}{0pt}%
\pgfsys@defobject{currentmarker}{\pgfqpoint{0.000000in}{-0.048611in}}{\pgfqpoint{0.000000in}{0.000000in}}{%
\pgfpathmoveto{\pgfqpoint{0.000000in}{0.000000in}}%
\pgfpathlineto{\pgfqpoint{0.000000in}{-0.048611in}}%
\pgfusepath{stroke,fill}%
}%
\begin{pgfscope}%
\pgfsys@transformshift{2.720775in}{1.359165in}%
\pgfsys@useobject{currentmarker}{}%
\end{pgfscope}%
\end{pgfscope}%
\begin{pgfscope}%
\definecolor{textcolor}{rgb}{0.000000,0.000000,0.000000}%
\pgfsetstrokecolor{textcolor}%
\pgfsetfillcolor{textcolor}%
\pgftext[x=2.309404in, y=0.310396in, left, base,rotate=45.000000]{\color{textcolor}\sffamily\fontsize{10.000000}{12.000000}\selectfont Textureless+Light}%
\end{pgfscope}%
\begin{pgfscope}%
\pgfsetbuttcap%
\pgfsetroundjoin%
\definecolor{currentfill}{rgb}{0.000000,0.000000,0.000000}%
\pgfsetfillcolor{currentfill}%
\pgfsetlinewidth{0.803000pt}%
\definecolor{currentstroke}{rgb}{0.000000,0.000000,0.000000}%
\pgfsetstrokecolor{currentstroke}%
\pgfsetdash{}{0pt}%
\pgfsys@defobject{currentmarker}{\pgfqpoint{0.000000in}{-0.048611in}}{\pgfqpoint{0.000000in}{0.000000in}}{%
\pgfpathmoveto{\pgfqpoint{0.000000in}{0.000000in}}%
\pgfpathlineto{\pgfqpoint{0.000000in}{-0.048611in}}%
\pgfusepath{stroke,fill}%
}%
\begin{pgfscope}%
\pgfsys@transformshift{3.445047in}{1.359165in}%
\pgfsys@useobject{currentmarker}{}%
\end{pgfscope}%
\end{pgfscope}%
\begin{pgfscope}%
\definecolor{textcolor}{rgb}{0.000000,0.000000,0.000000}%
\pgfsetstrokecolor{textcolor}%
\pgfsetfillcolor{textcolor}%
\pgftext[x=3.261216in, y=0.765478in, left, base,rotate=45.000000]{\color{textcolor}\sffamily\fontsize{10.000000}{12.000000}\selectfont Textured}%
\end{pgfscope}%
\begin{pgfscope}%
\pgfsetbuttcap%
\pgfsetroundjoin%
\definecolor{currentfill}{rgb}{0.000000,0.000000,0.000000}%
\pgfsetfillcolor{currentfill}%
\pgfsetlinewidth{0.803000pt}%
\definecolor{currentstroke}{rgb}{0.000000,0.000000,0.000000}%
\pgfsetstrokecolor{currentstroke}%
\pgfsetdash{}{0pt}%
\pgfsys@defobject{currentmarker}{\pgfqpoint{0.000000in}{-0.048611in}}{\pgfqpoint{0.000000in}{0.000000in}}{%
\pgfpathmoveto{\pgfqpoint{0.000000in}{0.000000in}}%
\pgfpathlineto{\pgfqpoint{0.000000in}{-0.048611in}}%
\pgfusepath{stroke,fill}%
}%
\begin{pgfscope}%
\pgfsys@transformshift{4.169318in}{1.359165in}%
\pgfsys@useobject{currentmarker}{}%
\end{pgfscope}%
\end{pgfscope}%
\begin{pgfscope}%
\definecolor{textcolor}{rgb}{0.000000,0.000000,0.000000}%
\pgfsetstrokecolor{textcolor}%
\pgfsetfillcolor{textcolor}%
\pgftext[x=3.821798in, y=0.438097in, left, base,rotate=45.000000]{\color{textcolor}\sffamily\fontsize{10.000000}{12.000000}\selectfont Textured+Light}%
\end{pgfscope}%
\begin{pgfscope}%
\pgfsetbuttcap%
\pgfsetroundjoin%
\definecolor{currentfill}{rgb}{0.000000,0.000000,0.000000}%
\pgfsetfillcolor{currentfill}%
\pgfsetlinewidth{0.803000pt}%
\definecolor{currentstroke}{rgb}{0.000000,0.000000,0.000000}%
\pgfsetstrokecolor{currentstroke}%
\pgfsetdash{}{0pt}%
\pgfsys@defobject{currentmarker}{\pgfqpoint{0.000000in}{-0.048611in}}{\pgfqpoint{0.000000in}{0.000000in}}{%
\pgfpathmoveto{\pgfqpoint{0.000000in}{0.000000in}}%
\pgfpathlineto{\pgfqpoint{0.000000in}{-0.048611in}}%
\pgfusepath{stroke,fill}%
}%
\begin{pgfscope}%
\pgfsys@transformshift{4.893590in}{1.359165in}%
\pgfsys@useobject{currentmarker}{}%
\end{pgfscope}%
\end{pgfscope}%
\begin{pgfscope}%
\definecolor{textcolor}{rgb}{0.000000,0.000000,0.000000}%
\pgfsetstrokecolor{textcolor}%
\pgfsetfillcolor{textcolor}%
\pgftext[x=4.621644in, y=0.589247in, left, base,rotate=45.000000]{\color{textcolor}\sffamily\fontsize{10.000000}{12.000000}\selectfont Multi-Object}%
\end{pgfscope}%
\begin{pgfscope}%
\pgfsetbuttcap%
\pgfsetroundjoin%
\definecolor{currentfill}{rgb}{0.000000,0.000000,0.000000}%
\pgfsetfillcolor{currentfill}%
\pgfsetlinewidth{0.803000pt}%
\definecolor{currentstroke}{rgb}{0.000000,0.000000,0.000000}%
\pgfsetstrokecolor{currentstroke}%
\pgfsetdash{}{0pt}%
\pgfsys@defobject{currentmarker}{\pgfqpoint{0.000000in}{-0.048611in}}{\pgfqpoint{0.000000in}{0.000000in}}{%
\pgfpathmoveto{\pgfqpoint{0.000000in}{0.000000in}}%
\pgfpathlineto{\pgfqpoint{0.000000in}{-0.048611in}}%
\pgfusepath{stroke,fill}%
}%
\begin{pgfscope}%
\pgfsys@transformshift{5.617862in}{1.359165in}%
\pgfsys@useobject{currentmarker}{}%
\end{pgfscope}%
\end{pgfscope}%
\begin{pgfscope}%
\definecolor{textcolor}{rgb}{0.000000,0.000000,0.000000}%
\pgfsetstrokecolor{textcolor}%
\pgfsetfillcolor{textcolor}%
\pgftext[x=5.395476in, y=0.688368in, left, base,rotate=45.000000]{\color{textcolor}\sffamily\fontsize{10.000000}{12.000000}\selectfont Combined}%
\end{pgfscope}%
\begin{pgfscope}%
\definecolor{textcolor}{rgb}{0.000000,0.000000,0.000000}%
\pgfsetstrokecolor{textcolor}%
\pgfsetfillcolor{textcolor}%
\pgftext[x=3.445047in,y=0.234413in,,top]{\color{textcolor}\sffamily\fontsize{10.000000}{12.000000}\bfseries\selectfont Dataset}%
\end{pgfscope}%
\begin{pgfscope}%
\pgfsetbuttcap%
\pgfsetroundjoin%
\definecolor{currentfill}{rgb}{0.000000,0.000000,0.000000}%
\pgfsetfillcolor{currentfill}%
\pgfsetlinewidth{0.803000pt}%
\definecolor{currentstroke}{rgb}{0.000000,0.000000,0.000000}%
\pgfsetstrokecolor{currentstroke}%
\pgfsetdash{}{0pt}%
\pgfsys@defobject{currentmarker}{\pgfqpoint{-0.048611in}{0.000000in}}{\pgfqpoint{-0.000000in}{0.000000in}}{%
\pgfpathmoveto{\pgfqpoint{-0.000000in}{0.000000in}}%
\pgfpathlineto{\pgfqpoint{-0.048611in}{0.000000in}}%
\pgfusepath{stroke,fill}%
}%
\begin{pgfscope}%
\pgfsys@transformshift{0.696435in}{1.359165in}%
\pgfsys@useobject{currentmarker}{}%
\end{pgfscope}%
\end{pgfscope}%
\begin{pgfscope}%
\definecolor{textcolor}{rgb}{0.000000,0.000000,0.000000}%
\pgfsetstrokecolor{textcolor}%
\pgfsetfillcolor{textcolor}%
\pgftext[x=0.289968in, y=1.306403in, left, base]{\color{textcolor}\sffamily\fontsize{10.000000}{12.000000}\selectfont 0.30}%
\end{pgfscope}%
\begin{pgfscope}%
\pgfsetbuttcap%
\pgfsetroundjoin%
\definecolor{currentfill}{rgb}{0.000000,0.000000,0.000000}%
\pgfsetfillcolor{currentfill}%
\pgfsetlinewidth{0.803000pt}%
\definecolor{currentstroke}{rgb}{0.000000,0.000000,0.000000}%
\pgfsetstrokecolor{currentstroke}%
\pgfsetdash{}{0pt}%
\pgfsys@defobject{currentmarker}{\pgfqpoint{-0.048611in}{0.000000in}}{\pgfqpoint{-0.000000in}{0.000000in}}{%
\pgfpathmoveto{\pgfqpoint{-0.000000in}{0.000000in}}%
\pgfpathlineto{\pgfqpoint{-0.048611in}{0.000000in}}%
\pgfusepath{stroke,fill}%
}%
\begin{pgfscope}%
\pgfsys@transformshift{0.696435in}{1.863698in}%
\pgfsys@useobject{currentmarker}{}%
\end{pgfscope}%
\end{pgfscope}%
\begin{pgfscope}%
\definecolor{textcolor}{rgb}{0.000000,0.000000,0.000000}%
\pgfsetstrokecolor{textcolor}%
\pgfsetfillcolor{textcolor}%
\pgftext[x=0.289968in, y=1.810936in, left, base]{\color{textcolor}\sffamily\fontsize{10.000000}{12.000000}\selectfont 0.35}%
\end{pgfscope}%
\begin{pgfscope}%
\pgfsetbuttcap%
\pgfsetroundjoin%
\definecolor{currentfill}{rgb}{0.000000,0.000000,0.000000}%
\pgfsetfillcolor{currentfill}%
\pgfsetlinewidth{0.803000pt}%
\definecolor{currentstroke}{rgb}{0.000000,0.000000,0.000000}%
\pgfsetstrokecolor{currentstroke}%
\pgfsetdash{}{0pt}%
\pgfsys@defobject{currentmarker}{\pgfqpoint{-0.048611in}{0.000000in}}{\pgfqpoint{-0.000000in}{0.000000in}}{%
\pgfpathmoveto{\pgfqpoint{-0.000000in}{0.000000in}}%
\pgfpathlineto{\pgfqpoint{-0.048611in}{0.000000in}}%
\pgfusepath{stroke,fill}%
}%
\begin{pgfscope}%
\pgfsys@transformshift{0.696435in}{2.368231in}%
\pgfsys@useobject{currentmarker}{}%
\end{pgfscope}%
\end{pgfscope}%
\begin{pgfscope}%
\definecolor{textcolor}{rgb}{0.000000,0.000000,0.000000}%
\pgfsetstrokecolor{textcolor}%
\pgfsetfillcolor{textcolor}%
\pgftext[x=0.289968in, y=2.315469in, left, base]{\color{textcolor}\sffamily\fontsize{10.000000}{12.000000}\selectfont 0.40}%
\end{pgfscope}%
\begin{pgfscope}%
\pgfsetbuttcap%
\pgfsetroundjoin%
\definecolor{currentfill}{rgb}{0.000000,0.000000,0.000000}%
\pgfsetfillcolor{currentfill}%
\pgfsetlinewidth{0.803000pt}%
\definecolor{currentstroke}{rgb}{0.000000,0.000000,0.000000}%
\pgfsetstrokecolor{currentstroke}%
\pgfsetdash{}{0pt}%
\pgfsys@defobject{currentmarker}{\pgfqpoint{-0.048611in}{0.000000in}}{\pgfqpoint{-0.000000in}{0.000000in}}{%
\pgfpathmoveto{\pgfqpoint{-0.000000in}{0.000000in}}%
\pgfpathlineto{\pgfqpoint{-0.048611in}{0.000000in}}%
\pgfusepath{stroke,fill}%
}%
\begin{pgfscope}%
\pgfsys@transformshift{0.696435in}{2.872764in}%
\pgfsys@useobject{currentmarker}{}%
\end{pgfscope}%
\end{pgfscope}%
\begin{pgfscope}%
\definecolor{textcolor}{rgb}{0.000000,0.000000,0.000000}%
\pgfsetstrokecolor{textcolor}%
\pgfsetfillcolor{textcolor}%
\pgftext[x=0.289968in, y=2.820002in, left, base]{\color{textcolor}\sffamily\fontsize{10.000000}{12.000000}\selectfont 0.45}%
\end{pgfscope}%
\begin{pgfscope}%
\pgfsetbuttcap%
\pgfsetroundjoin%
\definecolor{currentfill}{rgb}{0.000000,0.000000,0.000000}%
\pgfsetfillcolor{currentfill}%
\pgfsetlinewidth{0.803000pt}%
\definecolor{currentstroke}{rgb}{0.000000,0.000000,0.000000}%
\pgfsetstrokecolor{currentstroke}%
\pgfsetdash{}{0pt}%
\pgfsys@defobject{currentmarker}{\pgfqpoint{-0.048611in}{0.000000in}}{\pgfqpoint{-0.000000in}{0.000000in}}{%
\pgfpathmoveto{\pgfqpoint{-0.000000in}{0.000000in}}%
\pgfpathlineto{\pgfqpoint{-0.048611in}{0.000000in}}%
\pgfusepath{stroke,fill}%
}%
\begin{pgfscope}%
\pgfsys@transformshift{0.696435in}{3.377297in}%
\pgfsys@useobject{currentmarker}{}%
\end{pgfscope}%
\end{pgfscope}%
\begin{pgfscope}%
\definecolor{textcolor}{rgb}{0.000000,0.000000,0.000000}%
\pgfsetstrokecolor{textcolor}%
\pgfsetfillcolor{textcolor}%
\pgftext[x=0.289968in, y=3.324535in, left, base]{\color{textcolor}\sffamily\fontsize{10.000000}{12.000000}\selectfont 0.50}%
\end{pgfscope}%
\begin{pgfscope}%
\pgfsetbuttcap%
\pgfsetroundjoin%
\definecolor{currentfill}{rgb}{0.000000,0.000000,0.000000}%
\pgfsetfillcolor{currentfill}%
\pgfsetlinewidth{0.803000pt}%
\definecolor{currentstroke}{rgb}{0.000000,0.000000,0.000000}%
\pgfsetstrokecolor{currentstroke}%
\pgfsetdash{}{0pt}%
\pgfsys@defobject{currentmarker}{\pgfqpoint{-0.048611in}{0.000000in}}{\pgfqpoint{-0.000000in}{0.000000in}}{%
\pgfpathmoveto{\pgfqpoint{-0.000000in}{0.000000in}}%
\pgfpathlineto{\pgfqpoint{-0.048611in}{0.000000in}}%
\pgfusepath{stroke,fill}%
}%
\begin{pgfscope}%
\pgfsys@transformshift{0.696435in}{3.881830in}%
\pgfsys@useobject{currentmarker}{}%
\end{pgfscope}%
\end{pgfscope}%
\begin{pgfscope}%
\definecolor{textcolor}{rgb}{0.000000,0.000000,0.000000}%
\pgfsetstrokecolor{textcolor}%
\pgfsetfillcolor{textcolor}%
\pgftext[x=0.289968in, y=3.829068in, left, base]{\color{textcolor}\sffamily\fontsize{10.000000}{12.000000}\selectfont 0.55}%
\end{pgfscope}%
\begin{pgfscope}%
\pgfsetbuttcap%
\pgfsetroundjoin%
\definecolor{currentfill}{rgb}{0.000000,0.000000,0.000000}%
\pgfsetfillcolor{currentfill}%
\pgfsetlinewidth{0.803000pt}%
\definecolor{currentstroke}{rgb}{0.000000,0.000000,0.000000}%
\pgfsetstrokecolor{currentstroke}%
\pgfsetdash{}{0pt}%
\pgfsys@defobject{currentmarker}{\pgfqpoint{-0.048611in}{0.000000in}}{\pgfqpoint{-0.000000in}{0.000000in}}{%
\pgfpathmoveto{\pgfqpoint{-0.000000in}{0.000000in}}%
\pgfpathlineto{\pgfqpoint{-0.048611in}{0.000000in}}%
\pgfusepath{stroke,fill}%
}%
\begin{pgfscope}%
\pgfsys@transformshift{0.696435in}{4.386363in}%
\pgfsys@useobject{currentmarker}{}%
\end{pgfscope}%
\end{pgfscope}%
\begin{pgfscope}%
\definecolor{textcolor}{rgb}{0.000000,0.000000,0.000000}%
\pgfsetstrokecolor{textcolor}%
\pgfsetfillcolor{textcolor}%
\pgftext[x=0.289968in, y=4.333601in, left, base]{\color{textcolor}\sffamily\fontsize{10.000000}{12.000000}\selectfont 0.60}%
\end{pgfscope}%
\begin{pgfscope}%
\definecolor{textcolor}{rgb}{0.000000,0.000000,0.000000}%
\pgfsetstrokecolor{textcolor}%
\pgfsetfillcolor{textcolor}%
\pgftext[x=0.234413in,y=2.872764in,,bottom,rotate=90.000000]{\color{textcolor}\sffamily\fontsize{10.000000}{12.000000}\bfseries\selectfont F1 Score}%
\end{pgfscope}%
\begin{pgfscope}%
\pgfsetrectcap%
\pgfsetmiterjoin%
\pgfsetlinewidth{0.803000pt}%
\definecolor{currentstroke}{rgb}{0.000000,0.000000,0.000000}%
\pgfsetstrokecolor{currentstroke}%
\pgfsetdash{}{0pt}%
\pgfpathmoveto{\pgfqpoint{0.696435in}{1.359165in}}%
\pgfpathlineto{\pgfqpoint{0.696435in}{4.386363in}}%
\pgfusepath{stroke}%
\end{pgfscope}%
\begin{pgfscope}%
\pgfsetrectcap%
\pgfsetmiterjoin%
\pgfsetlinewidth{0.803000pt}%
\definecolor{currentstroke}{rgb}{0.000000,0.000000,0.000000}%
\pgfsetstrokecolor{currentstroke}%
\pgfsetdash{}{0pt}%
\pgfpathmoveto{\pgfqpoint{6.193658in}{1.359165in}}%
\pgfpathlineto{\pgfqpoint{6.193658in}{4.386363in}}%
\pgfusepath{stroke}%
\end{pgfscope}%
\begin{pgfscope}%
\pgfsetrectcap%
\pgfsetmiterjoin%
\pgfsetlinewidth{0.803000pt}%
\definecolor{currentstroke}{rgb}{0.000000,0.000000,0.000000}%
\pgfsetstrokecolor{currentstroke}%
\pgfsetdash{}{0pt}%
\pgfpathmoveto{\pgfqpoint{0.696435in}{1.359165in}}%
\pgfpathlineto{\pgfqpoint{6.193658in}{1.359165in}}%
\pgfusepath{stroke}%
\end{pgfscope}%
\begin{pgfscope}%
\pgfsetrectcap%
\pgfsetmiterjoin%
\pgfsetlinewidth{0.803000pt}%
\definecolor{currentstroke}{rgb}{0.000000,0.000000,0.000000}%
\pgfsetstrokecolor{currentstroke}%
\pgfsetdash{}{0pt}%
\pgfpathmoveto{\pgfqpoint{0.696435in}{4.386363in}}%
\pgfpathlineto{\pgfqpoint{6.193658in}{4.386363in}}%
\pgfusepath{stroke}%
\end{pgfscope}%
\begin{pgfscope}%
\definecolor{textcolor}{rgb}{0.000000,0.000000,0.000000}%
\pgfsetstrokecolor{textcolor}%
\pgfsetfillcolor{textcolor}%
\pgftext[x=1.109270in,y=2.961856in,,bottom]{\color{textcolor}\sffamily\fontsize{9.000000}{10.800000}\selectfont 0.4547}%
\end{pgfscope}%
\begin{pgfscope}%
\definecolor{textcolor}{rgb}{0.000000,0.000000,0.000000}%
\pgfsetstrokecolor{textcolor}%
\pgfsetfillcolor{textcolor}%
\pgftext[x=1.833542in,y=3.296866in,,bottom]{\color{textcolor}\sffamily\fontsize{9.000000}{10.800000}\selectfont 0.4879}%
\end{pgfscope}%
\begin{pgfscope}%
\definecolor{textcolor}{rgb}{0.000000,0.000000,0.000000}%
\pgfsetstrokecolor{textcolor}%
\pgfsetfillcolor{textcolor}%
\pgftext[x=2.557814in,y=3.238341in,,bottom]{\color{textcolor}\sffamily\fontsize{9.000000}{10.800000}\selectfont 0.4821}%
\end{pgfscope}%
\begin{pgfscope}%
\definecolor{textcolor}{rgb}{0.000000,0.000000,0.000000}%
\pgfsetstrokecolor{textcolor}%
\pgfsetfillcolor{textcolor}%
\pgftext[x=3.282085in,y=3.086981in,,bottom]{\color{textcolor}\sffamily\fontsize{9.000000}{10.800000}\selectfont 0.4671}%
\end{pgfscope}%
\begin{pgfscope}%
\definecolor{textcolor}{rgb}{0.000000,0.000000,0.000000}%
\pgfsetstrokecolor{textcolor}%
\pgfsetfillcolor{textcolor}%
\pgftext[x=4.006357in,y=2.782243in,,bottom]{\color{textcolor}\sffamily\fontsize{9.000000}{10.800000}\selectfont 0.4369}%
\end{pgfscope}%
\begin{pgfscope}%
\definecolor{textcolor}{rgb}{0.000000,0.000000,0.000000}%
\pgfsetstrokecolor{textcolor}%
\pgfsetfillcolor{textcolor}%
\pgftext[x=4.730629in,y=2.777197in,,bottom]{\color{textcolor}\sffamily\fontsize{9.000000}{10.800000}\selectfont 0.4364}%
\end{pgfscope}%
\begin{pgfscope}%
\definecolor{textcolor}{rgb}{0.000000,0.000000,0.000000}%
\pgfsetstrokecolor{textcolor}%
\pgfsetfillcolor{textcolor}%
\pgftext[x=5.454901in,y=3.380619in,,bottom]{\color{textcolor}\sffamily\fontsize{9.000000}{10.800000}\selectfont 0.4962}%
\end{pgfscope}%
\begin{pgfscope}%
\definecolor{textcolor}{rgb}{0.000000,0.000000,0.000000}%
\pgfsetstrokecolor{textcolor}%
\pgfsetfillcolor{textcolor}%
\pgftext[x=1.435193in,y=2.586484in,,bottom]{\color{textcolor}\sffamily\fontsize{9.000000}{10.800000}\selectfont 0.4175}%
\end{pgfscope}%
\begin{pgfscope}%
\definecolor{textcolor}{rgb}{0.000000,0.000000,0.000000}%
\pgfsetstrokecolor{textcolor}%
\pgfsetfillcolor{textcolor}%
\pgftext[x=2.159464in,y=3.074872in,,bottom]{\color{textcolor}\sffamily\fontsize{9.000000}{10.800000}\selectfont 0.4659}%
\end{pgfscope}%
\begin{pgfscope}%
\definecolor{textcolor}{rgb}{0.000000,0.000000,0.000000}%
\pgfsetstrokecolor{textcolor}%
\pgfsetfillcolor{textcolor}%
\pgftext[x=2.883736in,y=3.056709in,,bottom]{\color{textcolor}\sffamily\fontsize{9.000000}{10.800000}\selectfont 0.4641}%
\end{pgfscope}%
\begin{pgfscope}%
\definecolor{textcolor}{rgb}{0.000000,0.000000,0.000000}%
\pgfsetstrokecolor{textcolor}%
\pgfsetfillcolor{textcolor}%
\pgftext[x=3.608008in,y=3.087990in,,bottom]{\color{textcolor}\sffamily\fontsize{9.000000}{10.800000}\selectfont 0.4672}%
\end{pgfscope}%
\begin{pgfscope}%
\definecolor{textcolor}{rgb}{0.000000,0.000000,0.000000}%
\pgfsetstrokecolor{textcolor}%
\pgfsetfillcolor{textcolor}%
\pgftext[x=4.332279in,y=2.962866in,,bottom]{\color{textcolor}\sffamily\fontsize{9.000000}{10.800000}\selectfont 0.4548}%
\end{pgfscope}%
\begin{pgfscope}%
\definecolor{textcolor}{rgb}{0.000000,0.000000,0.000000}%
\pgfsetstrokecolor{textcolor}%
\pgfsetfillcolor{textcolor}%
\pgftext[x=5.056551in,y=3.080926in,,bottom]{\color{textcolor}\sffamily\fontsize{9.000000}{10.800000}\selectfont 0.4665}%
\end{pgfscope}%
\begin{pgfscope}%
\definecolor{textcolor}{rgb}{0.000000,0.000000,0.000000}%
\pgfsetstrokecolor{textcolor}%
\pgfsetfillcolor{textcolor}%
\pgftext[x=5.780823in,y=3.229259in,,bottom]{\color{textcolor}\sffamily\fontsize{9.000000}{10.800000}\selectfont 0.4812}%
\end{pgfscope}%
\begin{pgfscope}%
\definecolor{textcolor}{rgb}{0.000000,0.000000,0.000000}%
\pgfsetstrokecolor{textcolor}%
\pgfsetfillcolor{textcolor}%
\pgftext[x=3.445047in,y=4.469696in,,base]{\color{textcolor}\sffamily\fontsize{12.000000}{14.400000}\selectfont Abalation study on chairs with mixed training}%
\end{pgfscope}%
\begin{pgfscope}%
\pgfsetbuttcap%
\pgfsetmiterjoin%
\definecolor{currentfill}{rgb}{1.000000,1.000000,1.000000}%
\pgfsetfillcolor{currentfill}%
\pgfsetfillopacity{0.800000}%
\pgfsetlinewidth{1.003750pt}%
\definecolor{currentstroke}{rgb}{0.800000,0.800000,0.800000}%
\pgfsetstrokecolor{currentstroke}%
\pgfsetstrokeopacity{0.800000}%
\pgfsetdash{}{0pt}%
\pgfpathmoveto{\pgfqpoint{4.882270in}{3.867537in}}%
\pgfpathlineto{\pgfqpoint{6.096435in}{3.867537in}}%
\pgfpathquadraticcurveto{\pgfqpoint{6.124213in}{3.867537in}}{\pgfqpoint{6.124213in}{3.895315in}}%
\pgfpathlineto{\pgfqpoint{6.124213in}{4.289140in}}%
\pgfpathquadraticcurveto{\pgfqpoint{6.124213in}{4.316918in}}{\pgfqpoint{6.096435in}{4.316918in}}%
\pgfpathlineto{\pgfqpoint{4.882270in}{4.316918in}}%
\pgfpathquadraticcurveto{\pgfqpoint{4.854492in}{4.316918in}}{\pgfqpoint{4.854492in}{4.289140in}}%
\pgfpathlineto{\pgfqpoint{4.854492in}{3.895315in}}%
\pgfpathquadraticcurveto{\pgfqpoint{4.854492in}{3.867537in}}{\pgfqpoint{4.882270in}{3.867537in}}%
\pgfpathclose%
\pgfusepath{stroke,fill}%
\end{pgfscope}%
\begin{pgfscope}%
\pgfsetbuttcap%
\pgfsetmiterjoin%
\definecolor{currentfill}{rgb}{0.121569,0.466667,0.705882}%
\pgfsetfillcolor{currentfill}%
\pgfsetlinewidth{0.000000pt}%
\definecolor{currentstroke}{rgb}{0.000000,0.000000,0.000000}%
\pgfsetstrokecolor{currentstroke}%
\pgfsetstrokeopacity{0.000000}%
\pgfsetdash{}{0pt}%
\pgfpathmoveto{\pgfqpoint{4.910048in}{4.155839in}}%
\pgfpathlineto{\pgfqpoint{5.187825in}{4.155839in}}%
\pgfpathlineto{\pgfqpoint{5.187825in}{4.253062in}}%
\pgfpathlineto{\pgfqpoint{4.910048in}{4.253062in}}%
\pgfpathclose%
\pgfusepath{fill}%
\end{pgfscope}%
\begin{pgfscope}%
\definecolor{textcolor}{rgb}{0.000000,0.000000,0.000000}%
\pgfsetstrokecolor{textcolor}%
\pgfsetfillcolor{textcolor}%
\pgftext[x=5.298937in,y=4.155839in,left,base]{\color{textcolor}\sffamily\fontsize{10.000000}{12.000000}\selectfont Pix2Vox++}%
\end{pgfscope}%
\begin{pgfscope}%
\pgfsetbuttcap%
\pgfsetmiterjoin%
\definecolor{currentfill}{rgb}{1.000000,0.498039,0.054902}%
\pgfsetfillcolor{currentfill}%
\pgfsetlinewidth{0.000000pt}%
\definecolor{currentstroke}{rgb}{0.000000,0.000000,0.000000}%
\pgfsetstrokecolor{currentstroke}%
\pgfsetstrokeopacity{0.000000}%
\pgfsetdash{}{0pt}%
\pgfpathmoveto{\pgfqpoint{4.910048in}{3.951982in}}%
\pgfpathlineto{\pgfqpoint{5.187825in}{3.951982in}}%
\pgfpathlineto{\pgfqpoint{5.187825in}{4.049204in}}%
\pgfpathlineto{\pgfqpoint{4.910048in}{4.049204in}}%
\pgfpathclose%
\pgfusepath{fill}%
\end{pgfscope}%
\begin{pgfscope}%
\definecolor{textcolor}{rgb}{0.000000,0.000000,0.000000}%
\pgfsetstrokecolor{textcolor}%
\pgfsetfillcolor{textcolor}%
\pgftext[x=5.298937in,y=3.951982in,left,base]{\color{textcolor}\sffamily\fontsize{10.000000}{12.000000}\selectfont Pix2Vox}%
\end{pgfscope}%
\end{pgfpicture}%
\makeatother%
\endgroup%
}
    \caption{Bar plot for the \gls{f1}  for baseline trained by mixing chair dataset from real and synthetic dataset with ratio of 50\%.
    Observe that the \fls{IoU} is consistent for all types of randomization proving that mixed training negates loss from randomization.}
    \label{fig:ablation_dice2}
\end{figure}

\section{Additional outputs}\label{sec:additional-outputs}

This section displays more outputs discussed in \autoref{sec:baseline}, \autoref{sec:fine-tuning}, and \autoref{sec:mixed-training}.
The models trained on only synthetic datasets fail to reconstruct 3D voxels as seen in \autoref{fig:baseline_more_images1}.
By fine-tuning the models with a real dataset, the 3D reconstruction improves, as seen \autoref{fig:finetuning_more_images1}.
MMixed training further improves the model's output, as seen in \autoref{fig:mixed_more_images1}.

\begin{figure}
    \begin{tabular}{llll}
        Pix3D images & \includegraphics[width=.2\linewidth]{/Users/apple/OVGU/Thesis/code/3dReconstruction/report/images/evaluation/reconstruction/baseline/bookcase2} &
        \includegraphics[width=.2\linewidth]{/Users/apple/OVGU/Thesis/code/3dReconstruction/report/images/evaluation/reconstruction/baseline/sofa2} &
        \includegraphics[width=.2\linewidth]{/Users/apple/OVGU/Thesis/code/3dReconstruction/report/images/evaluation/reconstruction/baseline/desk1}\\

        Ground Truth & \includegraphics[trim={0 0 {.1\width} 0},clip,width=.2\linewidth]{/Users/apple/OVGU/Thesis/code/3dReconstruction/report/images/evaluation/reconstruction/baseline/bookcase2_original} &
        \includegraphics[trim={0 0 {.1\width} 0},clip,width=.2\linewidth]{/Users/apple/OVGU/Thesis/code/3dReconstruction/report/images/evaluation/reconstruction/baseline/sofa2_original} &
        \includegraphics[trim={0 0 {.1\width} 0},clip,width=.2\linewidth]{/Users/apple/OVGU/Thesis/code/3dReconstruction/report/images/evaluation/reconstruction/baseline/desk1_original}\\

        Output1 & \includegraphics[width=.2\linewidth]{/Users/apple/OVGU/Thesis/code/3dReconstruction/report/images/evaluation/reconstruction/baseline/pix3d_p2vpp_bookcase2_output} &
        \includegraphics[width=.2\linewidth]{/Users/apple/OVGU/Thesis/code/3dReconstruction/report/images/evaluation/reconstruction/baseline/pix3d_p2vpp_sofa2_output} &
        \includegraphics[width=.2\linewidth]{/Users/apple/OVGU/Thesis/code/3dReconstruction/report/images/evaluation/reconstruction/baseline/pix3d_p2vpp_desk1_output}\\

        Output2 & \includegraphics[width=.2\linewidth]{/Users/apple/OVGU/Thesis/code/3dReconstruction/report/images/evaluation/reconstruction/baseline/pix3d_p2v_bookcase2} &
        \includegraphics[width=.2\linewidth]{/Users/apple/OVGU/Thesis/code/3dReconstruction/report/images/evaluation/reconstruction/baseline/pix3d_p2v_sofa2} &
        \includegraphics[width=.2\linewidth]{/Users/apple/OVGU/Thesis/code/3dReconstruction/report/images/evaluation/reconstruction/baseline/pix3d_p2v_desk1}\\

        Output3 & \includegraphics[width=.2\linewidth]{/Users/apple/OVGU/Thesis/code/3dReconstruction/report/images/evaluation/reconstruction/baseline/s2rv3_p2vpp_bookcase2} &
        \includegraphics[width=.2\linewidth]{/Users/apple/OVGU/Thesis/code/3dReconstruction/report/images/evaluation/reconstruction/baseline/s2rv3_p2vpp_sofa2} &
        \includegraphics[width=.2\linewidth]{/Users/apple/OVGU/Thesis/code/3dReconstruction/report/images/evaluation/reconstruction/baseline/s2rv3_p2vpp_desk1}\\

        Output4 & \includegraphics[width=.2\linewidth]{/Users/apple/OVGU/Thesis/code/3dReconstruction/report/images/evaluation/reconstruction/baseline/s2rv3_p2v_bookcase2} &
        \includegraphics[width=.2\linewidth]{/Users/apple/OVGU/Thesis/code/3dReconstruction/report/images/evaluation/reconstruction/baseline/s2rv3_p2v_sofa2} &
        \includegraphics[width=.2\linewidth]{/Users/apple/OVGU/Thesis/code/3dReconstruction/report/images/evaluation/reconstruction/baseline/s2rv3_p2v_desk1}\\

    \end{tabular}
    \caption{3D reconstruction outputs for models trained on \textbf{only real dataset}, and \textbf{only synthetic dataset}. Output1-2: Pix2Vox++ and Pix2Vox trained on Pix3D(real dataset).
    Output3-4: Pix2Vox++ and Pix2Vox trained with only \gls{s2rv2} synthetic dataset. This corresponds to the bad \gls{iou} when trained on only synthetic dataset.}
    \label{fig:baseline_more_images1}
\end{figure}


\begin{figure}
    \begin{tabular}{llll}
        Pix3D images & \includegraphics[width=.2\linewidth]{/Users/apple/OVGU/Thesis/code/3dReconstruction/report/images/evaluation/reconstruction/baseline/bookcase2} &
        \includegraphics[width=.2\linewidth]{/Users/apple/OVGU/Thesis/code/3dReconstruction/report/images/evaluation/reconstruction/baseline/sofa2} &
        \includegraphics[width=.2\linewidth]{/Users/apple/OVGU/Thesis/code/3dReconstruction/report/images/evaluation/reconstruction/baseline/desk1}\\

        Ground Truth & \includegraphics[trim={0 0 {.1\width} 0},clip,width=.2\linewidth]{/Users/apple/OVGU/Thesis/code/3dReconstruction/report/images/evaluation/reconstruction/baseline/bookcase2_original} &
        \includegraphics[trim={0 0 {.1\width} 0},clip,width=.2\linewidth]{/Users/apple/OVGU/Thesis/code/3dReconstruction/report/images/evaluation/reconstruction/baseline/sofa2_original} &
        \includegraphics[trim={0 0 {.1\width} 0},clip,width=.2\linewidth]{/Users/apple/OVGU/Thesis/code/3dReconstruction/report/images/evaluation/reconstruction/baseline/desk1_original}\\

        Output1 & \includegraphics[width=.2\linewidth]{/Users/apple/OVGU/Thesis/code/3dReconstruction/report/images/evaluation/reconstruction/baseline/pix3d_p2vpp_bookcase2_output} &
        \includegraphics[width=.2\linewidth]{/Users/apple/OVGU/Thesis/code/3dReconstruction/report/images/evaluation/reconstruction/baseline/pix3d_p2vpp_sofa2_output} &
        \includegraphics[width=.2\linewidth]{/Users/apple/OVGU/Thesis/code/3dReconstruction/report/images/evaluation/reconstruction/baseline/pix3d_p2vpp_desk1_output}\\

        Output2 & \includegraphics[width=.2\linewidth]{/Users/apple/OVGU/Thesis/code/3dReconstruction/report/images/evaluation/reconstruction/baseline/pix3d_p2v_bookcase2} &
        \includegraphics[width=.2\linewidth]{/Users/apple/OVGU/Thesis/code/3dReconstruction/report/images/evaluation/reconstruction/baseline/pix3d_p2v_sofa2} &
        \includegraphics[width=.2\linewidth]{/Users/apple/OVGU/Thesis/code/3dReconstruction/report/images/evaluation/reconstruction/baseline/pix3d_p2v_desk1}\\

        Output3 & \includegraphics[width=.2\linewidth]{/Users/apple/OVGU/Thesis/code/3dReconstruction/report/images/evaluation/reconstruction/finetuning/s2rv3_p2vpp_bookcase2} &
        \includegraphics[width=.2\linewidth]{/Users/apple/OVGU/Thesis/code/3dReconstruction/report/images/evaluation/reconstruction/finetuning/s2rv3_p2vpp_sofa2} &
        \includegraphics[width=.2\linewidth]{/Users/apple/OVGU/Thesis/code/3dReconstruction/report/images/evaluation/reconstruction/finetuning/s2rv3_p2vpp_desk1}\\

        Output4 & \includegraphics[width=.2\linewidth]{/Users/apple/OVGU/Thesis/code/3dReconstruction/report/images/evaluation/reconstruction/finetuning/s2rv3_p2v_bookcase2} &
        \includegraphics[width=.2\linewidth]{/Users/apple/OVGU/Thesis/code/3dReconstruction/report/images/evaluation/reconstruction/finetuning/s2rv3_p2v_sofa2} &
        \includegraphics[width=.2\linewidth]{/Users/apple/OVGU/Thesis/code/3dReconstruction/report/images/evaluation/reconstruction/finetuning/s2rv3_p2v_desk1}\\

    \end{tabular}
    \caption{3D reconstruction outputs for models trained on real dataset and synthetic datasets with \textbf{fine-tuning}. Output1-2: Pix2Vox++ and Pix2Vox trained on Pix3D(real dataset).
    Output3-4: Pix2Vox++ and Pix2Vox pre-trained with only \gls{s2rv2} synthetic dataset and then fine-tuned with Pix3d. The reconstruction is better than models trained on only synthetic dataset.}
    \label{fig:finetuning_more_images1}
\end{figure}


\begin{figure}
    \begin{tabular}{llll}
        Pix3D images & \includegraphics[width=.2\linewidth]{/Users/apple/OVGU/Thesis/code/3dReconstruction/report/images/evaluation/reconstruction/baseline/bookcase2} &
        \includegraphics[width=.2\linewidth]{/Users/apple/OVGU/Thesis/code/3dReconstruction/report/images/evaluation/reconstruction/baseline/sofa2} &
        \includegraphics[width=.2\linewidth]{/Users/apple/OVGU/Thesis/code/3dReconstruction/report/images/evaluation/reconstruction/baseline/desk1}\\

        Ground Truth & \includegraphics[trim={0 0 {.1\width} 0},clip,width=.2\linewidth]{/Users/apple/OVGU/Thesis/code/3dReconstruction/report/images/evaluation/reconstruction/mixed/bookcase2_original} &
        \includegraphics[trim={0 0 {.1\width} 0},clip,width=.2\linewidth]{/Users/apple/OVGU/Thesis/code/3dReconstruction/report/images/evaluation/reconstruction/mixed/sofa2_original} &
        \includegraphics[trim={0 0 {.1\width} 0},clip,width=.2\linewidth]{/Users/apple/OVGU/Thesis/code/3dReconstruction/report/images/evaluation/reconstruction/mixed/desk2_original}\\

        Output1 & \includegraphics[width=.2\linewidth]{/Users/apple/OVGU/Thesis/code/3dReconstruction/report/images/evaluation/reconstruction/mixed/mixed1_p2vpp_bookcase2} &
        \includegraphics[width=.2\linewidth]{/Users/apple/OVGU/Thesis/code/3dReconstruction/report/images/evaluation/reconstruction/mixed/mixed1_p2vpp_sofa2} &
        \includegraphics[width=.2\linewidth]{/Users/apple/OVGU/Thesis/code/3dReconstruction/report/images/evaluation/reconstruction/mixed/mixed1_p2vpp_desk1}\\

        Output2 & \includegraphics[width=.2\linewidth]{/Users/apple/OVGU/Thesis/code/3dReconstruction/report/images/evaluation/reconstruction/mixed/mixed1_p2v_bookcase2} &
        \includegraphics[width=.2\linewidth]{/Users/apple/OVGU/Thesis/code/3dReconstruction/report/images/evaluation/reconstruction/mixed/mixed1_p2v_sofa2} &
        \includegraphics[width=.2\linewidth]{/Users/apple/OVGU/Thesis/code/3dReconstruction/report/images/evaluation/reconstruction/mixed/mixed1_p2v_desk1}\\

        Output3 & \includegraphics[width=.2\linewidth]{/Users/apple/OVGU/Thesis/code/3dReconstruction/report/images/evaluation/reconstruction/mixed/mixed2_p2vpp_bookcase2} &
        \includegraphics[width=.2\linewidth]{/Users/apple/OVGU/Thesis/code/3dReconstruction/report/images/evaluation/reconstruction/mixed/mixed2_p2vpp_sofa1} &
        \includegraphics[width=.2\linewidth]{/Users/apple/OVGU/Thesis/code/3dReconstruction/report/images/evaluation/reconstruction/mixed/mixed2_p2vpp_desk1}\\

        Output4 & \includegraphics[width=.2\linewidth]{/Users/apple/OVGU/Thesis/code/3dReconstruction/report/images/evaluation/reconstruction/mixed/mixed2_p2v_bookcase2} &
        \includegraphics[width=.2\linewidth]{/Users/apple/OVGU/Thesis/code/3dReconstruction/report/images/evaluation/reconstruction/mixed/mixed2_p2v_sofa2} &
        \includegraphics[width=.2\linewidth]{/Users/apple/OVGU/Thesis/code/3dReconstruction/report/images/evaluation/reconstruction/mixed/mixed2_p2v_desk1}\\

    \end{tabular}
    \caption{3D reconstruction outputs for best \textbf{mixed training}(50\% per mini-batch) models. Output1-2: Pix2Vox++ and Pix2Vox mixed trained with \gls{s2rv1}.
    Output3-4:Pix2Vox++ and Pix2Vox mixed trained with \gls{s2rv2}}
    \label{fig:mixed_more_images1}
\end{figure}

\clearpage

\section{Additional attempts}\label{sec:some-failed-attempts}

This section briefly discusses other attempts made to solve the research question related to this thesis.

\subsection{High Definition Render Pipeline(HDRP)}\label{subsec:high-definition-render-pipeline(hdrp)}

Unity provides three types of rendering pipelines: Built-in Render Pipeline, Universal Render Pipeline and \gls{hdrp}.
\gls{hdrp} has high-fidelity and high graphics and hence needs compatible GPU and other hardware and software requirements.
Each pipeline is not compatible with the other, and hence the developer has to pre-decide which pipeline to use.
Even the assets and the materials are dependent on the pipeline, and hence assets of one pipeline can not be used in the other.
In \autoref{fig:hdrp} we can see the furniture not being able to render in \gls{hdrp}.
Using \gls{hdrp} to improve the rendering would be interesting future work.

\begin{figure}[ht]
    \centering
    \includegraphics[width=.4\textwidth, height = .3\textwidth]{/Users/apple/OVGU/Thesis/code/3dReconstruction/report/images/appendix/hdrp1}
    \includegraphics[width=.4\textwidth, height = .3\textwidth,valign=m]{/Users/apple/OVGU/Thesis/code/3dReconstruction/report/images/appendix/hdrp2}\\
    \vspace{0.1cm}
    \includegraphics[width=.4\textwidth, height = .3\textwidth]{/Users/apple/OVGU/Thesis/code/3dReconstruction/report/images/appendix/hdrp3}
    \includegraphics[width=.4\textwidth, height = .3\textwidth]{/Users/apple/OVGU/Thesis/code/3dReconstruction/report/images/appendix/hdrp4}\\
    \caption{Sample images for models rendered in High Definition Render Pipeline of Unity}
    \label{fig:hdrp}
\end{figure}


%\subsection{Different loss functions}

\subsection{Empty image test}\label{subsec:empty-image-test}
In all the experiments conducted, the 2D image always had a furniture model in the frame.
This section determines if the model predicts something when there is no furniture in the frame, with an empty room.
This is just an attempt to see if the model hallucinates any furniture in an empty room.

Though a piece of furniture is not reconstructed with complete detail, models predict voxels that resemble the furniture category.
We can imagine some of them to be bed, sofa, bookshelf, and table.

\begin{figure}
    \begin{tabular}{llll}
%        \includegraphics[width=.2\linewidth]{/Users/apple/OVGU/Thesis/code/3dReconstruction/report/images/evaluation/reconstruction/empty/empty_room_1} &
        \includegraphics[width=.2\linewidth]{/Users/apple/OVGU/Thesis/code/3dReconstruction/report/images/evaluation/reconstruction/empty/empty_room_2} &
        \includegraphics[width=.2\linewidth]{/Users/apple/OVGU/Thesis/code/3dReconstruction/report/images/evaluation/reconstruction/empty/empty_room_3} &
        \includegraphics[width=.2\linewidth]{/Users/apple/OVGU/Thesis/code/3dReconstruction/report/images/evaluation/reconstruction/empty/empty_room_4} &
        \includegraphics[width=.2\linewidth]{/Users/apple/OVGU/Thesis/code/3dReconstruction/report/images/evaluation/reconstruction/empty/empty_room_5}\\

%        \includegraphics[width=.2\linewidth]{/Users/apple/OVGU/Thesis/code/3dReconstruction/report/images/evaluation/reconstruction/empty/p2vpp_empty_output_1} &
        \includegraphics[width=.2\linewidth]{/Users/apple/OVGU/Thesis/code/3dReconstruction/report/images/evaluation/reconstruction/empty/p2vpp_empty_output_2} &
        \includegraphics[width=.2\linewidth]{/Users/apple/OVGU/Thesis/code/3dReconstruction/report/images/evaluation/reconstruction/empty/p2vpp_empty_output_3} &
        \includegraphics[width=.2\linewidth]{/Users/apple/OVGU/Thesis/code/3dReconstruction/report/images/evaluation/reconstruction/empty/p2vpp_empty_output_4} &
        \includegraphics[width=.2\linewidth]{/Users/apple/OVGU/Thesis/code/3dReconstruction/report/images/evaluation/reconstruction/empty/p2vpp_empty_output_5}\\

%        \includegraphics[width=.2\linewidth]{/Users/apple/OVGU/Thesis/code/3dReconstruction/report/images/evaluation/reconstruction/empty/p2v_empty_output_1} &
        \includegraphics[width=.2\linewidth]{/Users/apple/OVGU/Thesis/code/3dReconstruction/report/images/evaluation/reconstruction/empty/p2v_empty_output_2} &
        \includegraphics[width=.2\linewidth]{/Users/apple/OVGU/Thesis/code/3dReconstruction/report/images/evaluation/reconstruction/empty/p2v_empty_output_3} &
        \includegraphics[width=.2\linewidth]{/Users/apple/OVGU/Thesis/code/3dReconstruction/report/images/evaluation/reconstruction/empty/p2v_empty_output_4} &
        \includegraphics[width=.2\linewidth]{/Users/apple/OVGU/Thesis/code/3dReconstruction/report/images/evaluation/reconstruction/empty/p2v_empty_output_5}\\

    \end{tabular}
    \caption{3D reconstruction outputs for best \textbf{mixed training}(50\% per mini-batch) models for empty rooms.
    We do not see a detailed reconstruction, but some of them resemble furniture category.}
    \label{fig:empty_images1}
\end{figure}

\clearpage
\newpage
\section{Samples from Survey}\label{sec:samples-from-survey}
We surveyed to realize the photorealism of the proposed \gls{free} dataset with other proclaimed photorealistic datasets.
The survey is discussed in detail in \autoref{subsec:survey-results}.
In this section, we display a sample each for all three sections from the collection.
\autoref{fig:survey_sample_section1}, \autoref{fig:survey_sample_section2}, \autoref{fig:survey_sample_section3} are one examples from each section.

\begin{figure}[ht]
    \centering
    \includegraphics[width=1\textwidth]{/Users/apple/OVGU/Thesis/code/3dReconstruction/report/images/appendix/survey/survey_section1}
    \caption{A sample question used for survey section 1. For section 1, participants were asked to select 'Real' or 'Not real' for 27 images, 3 images per dataset.}
    \label{fig:survey_sample_section1}
\end{figure}

\begin{figure}[ht]
    \centering
    \includegraphics[width=1\textwidth]{/Users/apple/OVGU/Thesis/code/3dReconstruction/report/images/appendix/survey/survey_section2}
    \caption{A sample question used for survey section 2. For section 2, participants were asked rate the 27 images from 1 to 10. We had 3 images per dataset.}
    \label{fig:survey_sample_section2}
\end{figure}

\begin{figure}[ht]
    \centering
    \includegraphics[width=1\textwidth]{/Users/apple/OVGU/Thesis/code/3dReconstruction/report/images/appendix/survey/survey_section3}
    \caption{A sample question used for survey section 3. For section 3, the users were shown 9 images together and asked to rank them on descending order of photorealism. }
    \label{fig:survey_sample_section3}
\end{figure}


\clearpage

\section{Tables}\label{sec:tables}

\begin{table}[ht]
    \centering
    \resizebox{\columnwidth}{!}{%
        \begin{tabular}{|c |c |c |c |c |c |c |c |c |}
            \hline
            Dataset & Chair & Wardrobe & Table & Bed &	Desk & Sofa & Bookcase & Avg.\\ [0.5ex]
            \hline\hline
            Pix3D & \textbf{0.2887} &\textbf{0.6782}	&\textbf{0.1888} &\textbf{0.3739} &\textbf{0.211}2 &\textbf{0.618} &\textbf{0.291} &\textbf{0.3443} \\
            \hline
            Pix3D(no aug.) & 0.2306 &0.5901 &0.1494	&0.2744	&0.1734	&0.5182 &0.2099 &0.2777 \\
            \hline
            \gls{s2rv1}(no aug.) &0.0455 &0.05362 &0.064 &0.0308 &0.1277 &0.0695 &0.03995 &0.0577 \\
            \hline
            \gls{s2rv1} &0.0793 &0.0947 &0.0553 &0.0674 &0.1074 &0.1237 &0.0869 &0.0847 \\
            \hline
            \gls{s2rv2} &0.0916	&0.3604	&0.0822	&0.1263	&0.1079	&0.23022 &0.1686	&0.1297 \\
            \hline
        \end{tabular}%
    }
    \caption{Table represents \gls{iou} values for each category when trained on Pix2Vox++.
    The models trained on synthetic data is significantly bad when tested on real-data.}
    \label{tab:baseline_categories_all_iou_pix2voxpp}
\end{table}


\begin{table}[ht]
    \centering
    \resizebox{\columnwidth}{!}{%
        \begin{tabular}{|c |c |c |c |c |c |c |c |c |}
            \hline
            Dataset & Chair & Wardrobe & Table & Bed &	Desk & Sofa & Bookcase & Avg.\\ [0.5ex]
            \hline\hline
            Pix3D & \textbf{0.2715}	&\textbf{0.7287}	&\textbf{0.1831}	&\textbf{0.355}	&\textbf{0.2004}	&\textbf{0.5703}	&\textbf{0.2747}	&\textbf{0.3254} \\
            \hline
            Pix3D(no aug.) & 0.2493	&0.6615	&0.1774	&0.3216	&0.1838	&0.5857	&0.231 &0.3108 \\
            \hline
            \gls{s2rv1}(no aug.) &0.0763	&0.0597	&0.0187	&0.0541	&0.03774 &0.1202	&0.0549 &0.0679  \\
            \hline
            \gls{s2rv1} &0.04	&0.1431	&0.0485	&0.0404 &0.10411	&0.0723	&0.1067 &0.0571\\
            \hline
            \gls{s2rv2} &0.1323	&0.3747	&0.0959	&0.1505	&0.1197	&0.305	&0.1206 &0.1643\\
            \hline
        \end{tabular}%
    }
    \caption{Table represents \gls{iou} values for each category when trained on Pix2Vox.
    The models trained on synthetic data is significantly bad when tested on real-data.}
    \label{tab:baseline_categories_all_iou_pix2vox}
\end{table}


\begin{table}[ht]
    \centering
    \resizebox{\columnwidth}{!}{%
        \begin{tabular}{|c |c |c |c |c |c |c |c |c |}
            \hline
            Dataset & Chair & Wardrobe & Table & Bed &	Desk & Sofa & Bookcase & Avg.\\ [0.5ex]
            \hline\hline
            Pix3D & \textbf{0.2887} &\textbf{0.6782}	&\textbf{0.1888} &\textbf{0.3739} &\textbf{0.211}2 &\textbf{0.618} &\textbf{0.291} &\textbf{0.3443} \\
            \hline
            \gls{s2rv1}+Pix3D  &0.2589 &0.6454 &0.1715	&0.3382	&0.2096 &0.5897	&0.2006	&0.3161\\
            \hline
            \gls{s2rv2}+Pix3D &0.2605	&0.63 &0.1579 &0.3051 &0.185 &0.5813 &0.2388	&0.3089\\

            \hline
        \end{tabular}%
    }
    \caption{Table represents \gls{iou} values for each category when trained on Pix2Vox++.
    The models trained on synthetic data and then fine-tuned with real data. No improvement is seen with fine-tuning.}
    \label{tab:finetuning_categories_all_iou_pix2voxpp}
\end{table}

\begin{table}[ht]
    \centering
    \resizebox{\columnwidth}{!}{%
        \begin{tabular}{|c |c |c |c |c |c |c |c |c |}
            \hline
            Dataset & Chair & Wardrobe & Table & Bed &	Desk & Sofa & Bookcase & Avg.\\ [0.5ex]
            \hline\hline
            Pix3D & \textbf{0.2715}	&\textbf{0.7287} &\textbf{0.1831} &0.355	&0.2004	&0.5703	&0.2747	&0.3254 \\
            \hline
            \gls{s2rv1}+Pix3D  &0.25347	&0.6806	&0.1735	&0.3176	&\textbf{0.2006}	&0.5824	&0.23386	&0.3125\\
            \hline
            \gls{s2rv2}+Pix3D &0.2684	&0.6965	&0.1765	&\textbf{0.3722}	&0.1963	&\textbf{0.6202}	&\textbf{0.296}	&\textbf{0.3338}\\

            \hline
        \end{tabular}%
    }
    \caption{Table represents \gls{iou} values for each category when trained on Pix2Vox.
    The models trained on synthetic dataand fine-tuned with real-data. A slight improvement is seen in some of the categories and the average \gls{iou}}
    \label{tab:finetuning_categories_all_iou_pix2vox}
\end{table}

\begin{table}[ht]
    \centering
    \resizebox{\columnwidth}{!}{%
        \begin{tabular}{|c |c |c |c |c |c |c |c |c |}
            \hline
            Dataset & Chair & Wardrobe & Table & Bed &	Desk & Sofa & Bookcase & Avg.\\ [0.5ex]
            \hline\hline
            Pix3D & 0.2887 &0.6782	&0.1888 &0.3739 &0.2112 &0.618 &0.291 &0.3443 \\
            \hline
            \gls{s2rv1}+0.15\%  &0.2972	&0.7229 &\textbf{0.2008} &0.3591 &0.2394	&0.6011	&0.2789	&0.3479\\
            \hline
            \gls{s2rv1}+0.25\% &\textbf{0.3114}	&0.6989	&0.1956	&\textbf{0.4034}	&0.2338	&0.6304	&0.2599	&\textbf{0.3607}\\
            \hline
            \gls{s2rv1}+0.5\% &0.3063	&0.7219	&0.1964	&0.397	&0.2533	&\textbf{0.6401}	&0.3051	&0.3636\\
            \hline
            \gls{s2rv1}+0.75\% &0.3005	&0.6787	&0.1797	&0.3867	&0.2241	&0.6168	&0.3049 &0.3498\\
            \hline
            \gls{s2rv1}+0.90\% &0.2933	&0.6514	&0.1901	&0.3692	&0.2135	&0.6388	&0.3021	&0.3499\\
            \hline
            \gls{s2rv2}+0.15\%  &0.2559	&0.7651	&0.1708	&0.3042	&0.2297	&0.554	&0.3159	&0.3134 \\
            \hline
            \gls{s2rv2}+0.25\% &0.3017	&0.7221	&0.1897	&0.3649	&0.2606	&0.6138	&0.372	&0.3556 \\
            \hline
            \gls{s2rv2}+0.5\% &0.3077	&0.7582	&0.1859	&0.3493	&\textbf{0.2664}	&0.6194	&\textbf{0.3874}	&0.3587 \\
            \hline
            \gls{s2rv2}+0.75\% &0.2944	&\textbf{0.7697} &0.1872 &0.3712	&0.2527	&0.6013	&0.3291	&0.3494 \\
            \hline
            \gls{s2rv2}+0.90\% &0.2947	&0.6893	&0.1794	&0.4013	&0.1914	&0.6139	&0.3558	&0.3482\\
            \hline
        \end{tabular}%
    }
    \caption{Table represents \gls{iou} values for each category when trained on Pix2Vox++.
    The first column indicates the synthetic dataset and ratio of real data used in each mini-batch.}
    \label{tab:mixed_categories_all_iou_pix2voxpp}
\end{table}

\begin{table}[ht]
    \centering
    \resizebox{\columnwidth}{!}{%
        \begin{tabular}{|c |c |c |c |c |c |c |c |c |}
            \hline
            Dataset & Chair & Wardrobe & Table & Bed &	Desk & Sofa & Bookcase & Avg.\\ [0.5ex]
            \hline\hline
            Pix3D & 0.2715	&0.7287	&0.1831 &0.355	&0.2004	&0.5703	&0.2747	&0.3254 \\
            \hline
            \gls{s2rv1}+0.15\%  &\textbf{0.2961}	&0.7001	&0.1918	&0.3894	&0.209	&0.6099	&0.294	&0.3483 \\
            \hline
            \gls{s2rv1}+0.25\% &0.2911	&0.7323	&\textbf{0.2026}	&0.367	&0.2297	&0.5889	&0.2644	&0.3432 \\
            \hline
            \gls{s2rv1}+0.5\% &0.2756	&0.7087	&0.1877	&\textbf{0.4072} &0.2365 &\textbf{0.652}	&0.3007	&\textbf{0.3514} \\
            \hline
            \gls{s2rv1}+0.75\% &0.281	&0.6571	&0.1778	&0.3869	&0.2110	&0.6369	&0.2769	&0.343\\
            \hline
            \gls{s2rv1}+0.90\% &0.2756	&0.7258	&0.1769	&0.3611	&0.218	&0.6192	&0.3263	&0.3389 \\
            \hline
            \gls{s2rv2}+0.15\%  &0.2447	&0.7086	&0.1481	&0.3199	&0.1923	&0.5589	&0.3191	&0.3027 \\
            \hline
            \gls{s2rv2}+0.25\% &0.2889	&0.741	&0.1823	&0.375	&0.2226	&0.6268	&\textbf{0.3966}	&0.3513  \\
            \hline
            \gls{s2rv2}+0.5\% &0.268	&0.7479	&0.1790	&0.362	&\textbf{0.2404}	&0.6337	&0.3393	&0.3416 \\
            \hline
            \gls{s2rv2}+0.75\% &0.2837	&0.743	&0.1784	&0.359	&0.1958	&0.6076	&0.3908	&0.3413 \\
            \hline
            \gls{s2rv2}+0.90\% &0.2764	&\textbf{0.7497}	&0.1821	&0.3776	&0.2006	&0.6227	&0.3701	&0.3434\\
            \hline
        \end{tabular}%
    }
    \caption{Table represents \gls{iou} values for each category when trained on Pix2Vox.
    The first column indicates the synthetic dataset and ratio of real data used in each mini-batch.}
    \label{tab:mixed_categories_all_iou_pix2vox}
\end{table}


\begin{table}[ht]
    \centering
    \resizebox{\columnwidth}{!}{%

        \begin{tabular}{|c |c |c | c| c|}
            \hline
            Dataset & Pix2Vox++ & Pix2Vox & Pix2Vox++ with 50\% & Pix2Vox with 50\%\\ [0.5ex]
            \hline\hline
            Pix3D Chair(no aug)  &\textbf{0.2797} &0.2694 &x &x\\
            \hline
            Pix3D Chair &\textbf{0.3305} &0.2918 & x & x\\
            \hline
            S2R\textunderscore Textureless  &0.1798 &0.0748 &\textbf{0.366} &0.3385\\
            \hline
            S2R \textunderscore Textureless\textunderscore Light &0.143 &0.1026 &\textbf{0.362}  &0.338\\
            \hline
            S2R\textunderscore Textured  &0.2217 &0.112 &\textbf{0.3426} &0.3404\\
            \hline
            S2R\textunderscore Textured\textunderscore Light &0.2221 &0.0991 &0.312 &\textbf{0.3303}\\
            \hline
            S2R\textunderscore Multi\textunderscore Object &0.2372 &0.0943 &0.3104 &\textbf{0.3429}\\
            \hline
        \end{tabular}%
    }
    \caption{Table represents \gls{iou} values for each of the dataset created with different domain randomization component.
    The last 2 colums is mixed training with 50\% of mini-batches being Pix3D.
    We see that mixed training consistently increases the performance of the models.}
    \label{tab:ablation1}
\end{table}




%Dice tables
\begin{table}[ht]
    \centering
    \resizebox{\columnwidth}{!}{%
        \begin{tabular}{|c |c |c |c |c |c |c |c |c |}
            \hline
            Dataset & Chair & Wardrobe & Table & Bed &	Desk & Sofa & Bookcase & Avg.\\ [0.5ex]
            \hline\hline
            Pix3D & \textbf{0.4016} &\textbf{0.7452} &\textbf{0.251} &\textbf{0.4783} &\textbf{0.2973}2 &\textbf{0.7234} &\textbf{0.3881} &\textbf{0.4421} \\
            \hline
            Pix3D(no aug.) &0.3392	&0.6894	&0.2114	&0.3831	&0.2534	0.&6442	&0.3041	&0.3782 \\
            \hline
            \gls{s2rv1}(no aug.) &0.0849 &0.1012	&0.1152	&0.0586	&0.2201	&0.1279	&0.07491	&0.1054\\
            \hline
            \gls{s2rv1} &0.1406	&0.1696	&0.0996	&0.1182	&0.1847	&0.2073	&0.1541	&0.1477    \\
            \hline
            \gls{s2rv2} &0.1619	&0.5132	&0.1438	&0.2156	&0.1855	&0.3667	&0.2745	&0.21659 \\
            \hline
        \end{tabular}%
    }
    \caption{Table represents \gls{f1} values for each category when trained on Pix2Vox++.
    The models trained on synthetic data is significantly bad when tested on real-data.}
    \label{tab:baseline_categories_all_f1_pix2voxpp}
\end{table}

\begin{table}[ht]
    \centering
    \resizebox{\columnwidth}{!}{%
        \begin{tabular}{|c |c |c |c |c |c |c |c |c |}
            \hline
            Dataset & Chair & Wardrobe & Table & Bed &	Desk & Sofa & Bookcase & Avg.\\ [0.5ex]
            \hline\hline
            Pix3D & \textbf{0.3883}	&\textbf{0.8013}	&\textbf{0.2456}	&\textbf{0.4656}	&\textbf{0.2856}	&0.6882	&\textbf{0.3807}	&\textbf{0.4281} \\
            \hline
            Pix3D(no aug.) & 0.3585	& 0.7439	& 0.2375	& 0.4258	&0.2625	&\textbf{0.7033}	&0.3256	&0.4087 \\
            \hline
            \gls{s2rv1}(no aug.) &0.1359	&0.1119	&0.0353	&0.0984	&0.0695	&0.2094	&0.103	&0.1211  \\
            \hline
            \gls{s2rv1} &0.0744	&0.2421	&0.0903	&0.0764	&0.1837	&0.13071	&0.1883	&0.104\\
            \hline
            \gls{s2rv2}  &0.2255	&0.5402	&0.165	&0.2562	&0.2052	&0.4608	&0.2112	&0.267
            \\
            \hline
        \end{tabular}%
    }
    \caption{Table represents \gls{f1} values for each category when trained on Pix2Vox.
    The models trained on synthetic data is significantly bad when tested on real-data.}
    \label{tab:baseline_categories_all_f1_pix2vox}
\end{table}


\begin{table}[ht]
    \centering
    \resizebox{\columnwidth}{!}{%
        \begin{tabular}{|c |c |c |c |c |c |c |c |c |}
            \hline
            Dataset & Chair & Wardrobe & Table & Bed &	Desk & Sofa & Bookcase & Avg.\\ [0.5ex]
            \hline\hline
            Pix3D & \textbf{0.4016} &\textbf{0.7452} &\textbf{0.251} &\textbf{0.4783} &\textbf{0.2973} &\textbf{0.7234} &\textbf{0.3881} &\textbf{0.4421} \\
            \hline
            \gls{s2rv1}+Pix3D  &0.3766	&0.7284	&0.2326	&0.4488	&0.2952	&0.7048	&0.2826	&0.4177\\
            \hline
            \gls{s2rv2}+Pix3D &0.374	&0.7064	&0.212	&0.4083	&0.2656	&0.69468	&0.3334	&0.4065\\

            \hline
        \end{tabular}%
    }
    \caption{Table represents \gls{f1} values for each category when trained on Pix2Vox++.
    The models trained on synthetic data and then fine-tuned with real data. No improvement is seen with fine-tuning.}
    \label{tab:finetuning_categories_all_f1_pix2voxpp}
\end{table}



\begin{table}[ht]
    \centering
    \resizebox{\columnwidth}{!}{%
        \begin{tabular}{|c |c |c |c |c |c |c |c |c |}
            \hline
            Dataset & Chair & Wardrobe & Table & Bed &	Desk & Sofa & Bookcase & Avg.\\ [0.5ex]
            \hline\hline
            Pix3D & \textbf{0.3883}	&\textbf{0.8013}	&\textbf{0.2456}	&0.4656	&\textbf{0.2856}	&0.6882	&0.3807	&0.4281 \\
            \hline
            \gls{s2rv1}+Pix3D  &0.3703	&0.7694	&0.2371	&0.4309	&0.2883	&\textbf{0.6977}	&0.335	&0.4155\\
            \hline
            \gls{s2rv2}+Pix3D &0.3798	&0.6646	&0.2311	&\textbf{0.4822}	&0.2757	&0.7327	&\textbf{0.3941}	&\textbf{0.431}\\

            \hline
        \end{tabular}%
    }
    \caption{Table represents \gls{f1} values for each category when trained on Pix2Vox.
    The models trained on synthetic dataand fine-tuned with real-data. A slight improvement is seen in some of the categories and the average \gls{iou}}
    \label{tab:finetuning_categories_all_f1_pix2vox}
\end{table}

\begin{table}[ht]
    \centering
    \resizebox{\columnwidth}{!}{%
        \begin{tabular}{|c |c |c |c |c |c |c |c |c |}
            \hline
            Dataset & Chair & Wardrobe & Table & Bed &	Desk & Sofa & Bookcase & Avg.\\ [0.5ex]
            \hline\hline
            Pix3D & 0.4016 &0.7452 &0.251 &0.4783 &0.2973 &0.7234 &0.3881 &0.4421 \\
            \hline
            \gls{s2rv1}+0.15\%  &0.4178	&0.7787	&\textbf{0.2638}	&0.4618	&0.3278	&0.7116	&0.3711	&0.4493 \\
            \hline
            \gls{s2rv1}+0.25\% &\textbf{0.4326}	&0.7635	&0.2551	&\textbf{0.5076}	&0.3193	&0.7364	&0.3483	&0.4608\\
            \hline
            \gls{s2rv1}+0.5\% &0.4253	&0.789	&0.2564	&0.5056	&0.3429	&\textbf{0.7468}	&0.4066	&\textbf{0.4644}\\
            \hline
            \gls{s2rv1}+0.75\% &0.4057	&0.7329	&0.2278	&0.3303	&0.2946	&0.7145	&0.3931	&0.4377 \\
            \hline
            \gls{s2rv1}+0.90\% &0.4091	&0.7212	&0.2495	&0.4719	&0.2952	&0.7413	&0.3968	&0.4472\\
            \hline
            \gls{s2rv2}+0.15\%  &0.3712	&0.8269	&0.2378	&0.4067	&0.3212	&0.6659	&0.4105	&0.4142 \\
            \hline
            \gls{s2rv2}+0.25\% &0.415	&0.6844	&0.2440	&0.4689	&0.3459	&0.7157	&0.4692	&0.4511 \\
            \hline
            \gls{s2rv2}+0.5\% &0.4211	&0.8135	&0.2422	&0.4485	&\textbf{0.3535} &0.7222	&\textbf{0.485}	&0.4543\\
            \hline
            \gls{s2rv2}+0.75\% &0.4089	&\textbf{0.8313}	&0.249	&0.4747	&0.3384	&0.709	&0.4273	&0.4479 \\
            \hline
            \gls{s2rv2}+0.90\% &0.4022	&0.7502	&0.2279	&0.5018	&0.2614	&0.7148	&0.4497	&0.4386\\
            \hline
        \end{tabular}%
    }
    \caption{Table represents \gls{f1} values for each category when trained on Pix2Vox++.
    The first column indicates the synthetic dataset and ratio of real data used in each mini-batch.}
    \label{tab:mixed_categories_all_f1_pix2voxpp}
\end{table}

\begin{table}[ht]
    \centering
    \resizebox{\columnwidth}{!}{%
        \begin{tabular}{|c |c |c |c |c |c |c |c |c |}
            \hline
            Dataset & Chair & Wardrobe & Table & Bed &	Desk & Sofa & Bookcase & Avg.\\ [0.5ex]
            \hline\hline
            Pix3D & 0.3883	&0.8013	&0.2456	&0.4656	&0.2856	&0.6882	&0.3807	&0.4281 \\
            \hline
            \gls{s2rv1}+0.15\%  &\textbf{0.4137}	&0.7597	&0.2518	&0.4914	&0.29474 &0.7195	&0.3862	&0.4476 \\
            \hline
            \gls{s2rv1}+0.25\% &0.4074	&0.7888	&\textbf{0.2691}	&0.4681	&0.3173	&0.6971	&0.3528	&0.4428 \\
            \hline
            \gls{s2rv1}+0.5\% &0.3894	&0.7748	&0.2458	&\textbf{0.519}	&0.3228	&\textbf{0.7603}	&0.3977	&0.4499 \\
            \hline
            \gls{s2rv1}+0.75\% &0.3931	&0.7284	&0.2338	&0.4916	&0.2917	&0.6704	&0.3657	&0.4391\\
            \hline
            \gls{s2rv1}+0.90\% &0.3851	&0.794	&0.2346	&0.4628	&0.2993	&0.7312	&0.4344	&0.4355 \\
            \hline
            \gls{s2rv2}+0.15\%  &0.3614	&0.7796	&0.2128	&0.4274	&0.2809	&0.6787	&0.4074	&0.406 \\
            \hline
            \gls{s2rv2}+0.25\% &0.4066	&0.7992	&0.2477	&0.304	&0.3074	&0.7337	&0.4904	&\textbf{0.4513}  \\
            \hline
            \gls{s2rv2}+0.5\% &0.3752	&0.8061	&0.2347	&0.4684	&\textbf{0.3247}	&0.7382	&0.427	&0.4352 \\
            \hline
            \gls{s2rv2}+0.75\% &0.3958	&0.8015	&0.2329	&0.4653	&0.268	&0.7172	&\textbf{0.4927}	&0.4373 \\
            \hline
            \gls{s2rv2}+0.90\% &0.3869	&\textbf{0.8116}	&0.238	&0.4823	&0.2778	&0.7298	&0.4769	&0.439\\
            \hline
        \end{tabular}%
    }
    \caption{Table represents \gls{iou} values for each category when trained on Pix2Vox.
    The first column indicates the synthetic dataset and ratio of real data used in each mini-batch.}
    \label{tab:mixed_categories_all_f1_pix2vox}
\end{table}


\begin{table}[ht]
    \centering
    \resizebox{\columnwidth}{!}{%

        \begin{tabular}{|c |c |c | c| c|}
            \hline
            Dataset & Pix2Vox++ & Pix2Vox & Pix2Vox++ with 50\% & Pix2Vox with 50\%\\ [0.5ex]
            \hline\hline
            Pix3D Chair(no aug)  &0.4047 &0.3944 &x &x\\
            \hline
            Pix3D Chair &\textbf{0.4175} &\textbf{0.4547} & x & x\\
            \hline
            S2R\textunderscore Textureless  &0.2933 &0.1365 &0.4879 &0.4659\\
            \hline
            S2R \textunderscore Textureless\textunderscore Light &0.242 &0.1795 &0.4821  &0.4641\\
            \hline
            S2R\textunderscore Textured  &0.3513 &0.1959 &0.4671 &0.4672\\
            \hline
            S2R\textunderscore Textured\textunderscore Light &0.35197 &0.1769 &0.4369 &0.4548\\
            \hline
            S2R\textunderscore Multi\textunderscore Object &0.3705 &0.168 &0.4364 &0.4665\\
            \hline
            S2R\textunderscore Combined &0.3851 &0.2289 &\textbf{0.4962} &\textbf{0.4812}\\
            \hline
        \end{tabular}%
    }
    \caption{Table represents \gls{f1} values for each of the dataset created with different domain randomization component.
    The last 2 colums is mixed training with 50\% of mini-batches being Pix3D.
    We see that mixed training consistently increases the performance of the models.}
    \label{tab:ablation1_f1}
\end{table}
